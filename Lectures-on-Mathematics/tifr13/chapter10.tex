\chapter{Complementary Results}\label{chap10}

\section{Generalization of Grothendieck's theorem}\label{chap10:sec1}

Let\pageoriginale $\Omega$ be an open set in $C^n$, $\alpha $ a $(0,1)$ form in
$\Omega$ whose coefficients are $m \times m$ (differentiable)
matrices. We ask for a condition that there exist a differentiable
mapping $f$ of $\Omega$ in $GL (m,C)$ such that
$$
f^{-1} d'' f = \alpha. 
$$
If $\alpha = \sum a_k d \bar{z}_k$ and $f^{-1} d'' f = \alpha$, then
$\dfrac{\partial^2 f}{\partial \bar{z}_l \partial \bar{z}_k} =
\dfrac{\partial f}{\partial \bar{z}_l} a_k + f\dfrac{\partial
  a_k}{\partial \bar{z}_l}$ 
\begin{gather*}
= f a_l a_k + f \frac{\partial a_k}{\partial \bar{z}_l} =
\frac{\partial^2 f}{\partial \bar{z}_k \partial \bar{z}_l} = f a_k
a_l + f \frac{\partial a_l}{\partial \bar{z}_k}, \\
\text{i.e., } \qquad \frac{\partial a_l}{\partial \bar{z}_k} -
\frac{\partial a_k}{\partial \bar{z}_l} + [a_k, a_l] = 0, \qquad 
\end{gather*}
and if we write $[\alpha, \alpha] = \sum\limits_{k <l } [a_k, a_l] d
\bar{z}_k \wedge d \bar{z}_l$, then we can write these equations as 
$$
d'' \alpha + [\alpha, \alpha] = 0.
$$
The following generalization of Grothendieck's theorem provides a
converse in the case of a cube. 

\setcounter{thm}{0}
\begin{thm}\label{chap10:thm1}
Let $K$ be a cube in $C^n$, $\alpha$ $a (0,1)$ differentiable form in a
neighbourhood of $K$. Suppose that 
$$
d'' \alpha + [\alpha, \alpha] = 0.
$$
Then there exists a differentiable, regular matrix $f$ in a
neighbourhood $U$ of $K$ such that, in $U$, 
$$
f^{-1 }d'' f =\alpha. 
$$
\end{thm}

\begin{proof}
We use induction as in Grothendieck's theorem. Consider the statement:

For\pageoriginale every form $\alpha$ of type $(0,1)$ such that $d''
\alpha + [\alpha, \alpha] =0$ and for which the coefficients of
$d\bar{z}_{k+1}, \ldots, d \bar{z}_n$ are zero, there exists $f$ with
values in $GL(m,C)$ so that $f^{-1} d'' f = \alpha$. For $k=0$, the
statement is trivially true, since $\alpha =0$ and we may take $f =
I$. Suppose the statement true when $k$ is replaced by $k-1$. Let
$\alpha = \sum\limits^k_{j=1} a_j d \bar{z}_j$. Since $d'' \alpha +
[\alpha , \alpha] =0$ and $[\alpha, \alpha]$ does not contain
$d\bar{z}_l, l >k$, the $a_j$ are holomorphic in $z_{k+1}, \ldots,
z_n$. By \ref{chap9}, Theorem \ref{chap9:thm1} there is a function
$g$, holomorphic in 
$z_{k+1}, \ldots, z_n$ such that $\dfrac{\partial g}{\partial
  \bar{z}_k} = g \cdot a_k$. If we set $f = f' \cdot g$ then $f^{-1}
d'' f = g^{-1} f'^{-1} d'' f' g + g^{-1} d'' g $ and the problem
reduces to the finding of an $f'$ such that 
$$
f'^{-1} d'' f' = g (\alpha - g^{-1} d'' g) g^{-1} = \beta
$$
say. It is easily verified that $d'' \beta + [\beta, \beta] = 0$ and
clearlyl the coefficients of $d \bar{z}_l$, $l \geq k$, in $\beta$ are
zero, and by inductive hypothesis $f'$ exists.
\end{proof}

\section{Linear bundles}\label{chap10:sec2}

Let $V$ be a topological space, $\{\mathscr{O}_i\}_{i\in
  \mathfrak{J}}$ an open covering of $V$. Suppose that in every
$\mathscr{O}_i \cap \mathscr{O}_j$ is defined a continuous function
$c_{ij}$ with values in $GL(m,C)$ such that the set $\{c_{ij}\}$
satisfies 
\begin{itemize}
\item[(i)] $c_{ij} c_{ji} = I$ in  $\mathscr{O}_i \cap \mathscr{O}_j$, 

\item[(ii)] $c_{ij} c_{jk} c_{ki} = I$ in $\mathscr{O}_i \cap
  \mathscr{O}_j \cap \mathscr{O}_k$.
\end{itemize}
Consider the set $\bigcup\limits_{i \in \mathscr{I}} (\mathscr{O}_i
\times C^m)$. Let $(x, y) \in \mathscr{O}_i \times C^m$, $(x', y') \in
\mathscr{O}_j \times C^m$. We identify $(x,y)$ and $(x', y')$ if $x =
x'$ in $V$ and $y' = c_{ij} (x) y$. The quotient is denoted by $E$. 

\begin{defi*}
$E$ together with the given system $\{\mathscr{O}_i, c_{ij}\}$ is
  called a \textit{linear bundle over} $V$ (with fibre $C^m$). $V$ is
  said to be the base of $E$. 

$E$ is\pageoriginale a  topological space with the following
  properties:
\begin{itemize}
\item[(a)] There is a canonical mapping $p: E \to V$ which is
  continuous and onto $V$.

\item[(b)] $p^{-1} (a) \simeq C^m$ for every $a \in V$ (topologically
  and as a vector space over $C$). 

\item[(c)] Every point $a \in V$ has a neighbourhood $\mathscr{O}$
  such that $p^{-1}(\mathscr{O}) \simeq \mathscr{O} \times C^m$ the
  isomorphism being topological and compatible with (a) and (b) in an
  obvious sense.
\end{itemize}
\end{defi*}

Suppose $E'$ is another linear boundle over $V$, defined by the system
$\{\mathscr{O}'_\alpha, c'_{\alpha \beta}\}_A$. Suppose $p'$ is the
$p$ corresponding to $E'$ and that there is a homeomorphism of $E$
onto $E'$ compatible with (a), (b) and (c). Then it is easily shown
that $\{\mathscr{O}_i, c_{ij}\}_{\mathscr{I}}$,
$\{\mathscr{O}'_\alpha, c'_{\alpha \beta}\}_A$ are related by a finite
number of applications of the following two operations:
\begin{itemize}
\item[$(1^\circ)$] Passage to refinements or the
  converse. $\{\mathscr{O}'_\alpha, c'_{\alpha \beta}\}_A$ is a
  refinement of $\{\mathscr{O}_i, c_{ij}\}$ if there is a mapping
  $\phi: A \to \mathfrak{J}$ such that $\mathscr{O}'_\alpha \subset
  \mathscr{O}_{\phi (\alpha)}$ and $c'_{\alpha\beta} = c_{\phi
    (\alpha) \phi (\beta)}$. 

\item[$(2^\circ)$] The covering $\{\mathscr{O}_i\}_{i
  \in\mathfrak{J}}$ being the same, one passes to new functions
  $c'_{ij}$ by defining $c'_{ij} = c_i c_{ij} c^{-1}_j$ where $c_i$ is
  continuous in $\mathscr{O}_i$.
\end{itemize}

(Moreover, it can be shown easily that if, in defining $E$ and $E'$
the same covering is used, then only one application, namely of
operation $(2^\circ)$, is necessary). If $E$ and $E'$ are related by
such a homeomorphism, we say that they are in the \textit{same
  clases}.

The \textit{trivial class} is defined as the class containing the
bundle defined by taking for the covering, just $V$. With respect to a
covering $\{\mathscr{O}_i\}_{ i \in\mathfrak{J}}$ this class can be
defined by taking for the functions $c_{ij}$ the\pageoriginale unit
matrix $I$ or, generally, any functions $c_i c^{-1}_j$ where $c_i$ is
continuous in $\mathscr{O}_i$. A bundle is \textit{trivial} if it is
in the trivial class. 

If $V$ were a differentiable (complex analytic) manifold, then we
define a \textit{differentiable (analytic) linear bundle} over $V$ in
the same way, but now requiring the $c_{ij}$ to be differentiable
(analytic). 

The differentiable or analytic class of a differentiable or analytic
bundle can be defined in the obvious way and we speak of
differentiable or \textit{analytic equivalence} and
\textit{triviality}.

The following important theorem holds.

\begin{thm}\label{chap10:thm2}
Let $K$ be a cube in $C^n$, $E$ an analytic bundle over a
neighbourhood of $K$. Then $E$ is trivial over a neighbourhood of
$K$. 
\end{thm}

\begin{proof}
By dividing $K$ into smaller (closed) cubes $K_i$ with faces parallel
to the coordinate hyperplanes, we obtain holomorphic regular matrices
$c_{ij}$ in neighbourhoods of $K_i \cap K_j$ respectivelyl. Also, if
the bundle is defined by $\{\mathscr{O}_i, c_{ij}\}$ and is trivial
over $\mathscr{O}_1$ and $\mathscr{O}_2$, we may replace
$\mathscr{O}_1, \mathscr{O}_2$ by their union and modify the $c_{ij}$
to obtain an equivalent bundle (if $c_{12} = c_1 c^{-1}_2$ set
$\mathscr{O}'_1 = \mathscr{O}_1 \cup \mathscr{O}_2$, $\mathscr{O}'_i =
\mathscr{O}_i$ if $i \neq 1, 2$, $c'_i = c_{ij}$ if $i, j \neq 1$,
$c'_{1j} = c^{-1}_1 c_{1j} (= c^{-1}_2 c_{2j})$. Then
$\{\mathscr{O}'_i, c_{ij}\}$ defines an equivalent bundle). We have
thus only to prove the following result: given two adjacent cubes
$K_1$, $K_2$ and a holomorphic regular matrix $c$ in a neighbourhood
of their common face, we can write $c = c_1 c^{-1}_2$ in a
neighbourhood of $K_1 \cap K_2$, $c_1$, $c_2$ being holomorphic
regular matrices in neighbourhoods of $K_1, K_2$ respectively. But
this follows at once from Cartan's theorem on holomorphic regular
matrices (\ref{chap9}, Theorem \ref{chap9:thm2}). 
\end{proof}

\section{Application to the second Cousin problem}\label{chap10:sec3}

\medskip
\noindent{\textbf{Divisor:}} A\pageoriginale \textit{divisor} can be defined in two
ways similar to the two definitions of meromorphic functions and of
principal parts. 
\begin{itemize}
\item[(a)] Let $\mathfrak{m}^\ast$ be the sheaf of multiplicative
  groups of germs of meromorphic functions $\not\equiv 0$ on the
  complex manifold $V, \mathscr{H}$ the sheaf of multiplicative groups
  of germs of invertible holomorphic functions. A divisor is a section
  of $\mathfrak{m}^\ast/ \mathscr{H}$ over $V$. 

\item[(b)] Let $\{\mathscr{O}_i\}$ be an open covering of $V$ and let
  $g_i$ be a meromorphic function in $\mathscr{O}_i$. The system
  $\{\mathscr{O}_i,g_i\}$ defines a divisor if the function $g_i
  g^{-1}_j$ and its reciprocal are holomorphic in $\mathscr{O}_i \cap
  \mathscr{O}_j$. Two systems $\{\mathscr{O}_i, g_i\}$ and
  $\{\mathscr{O}'_j, g'_j\}$ define the same divisor if $g_i
  g'^{-1}_j$ is holomorphic and invertible in $\mathscr{O}_i \cap
  \mathscr{O}'_j$. 
\end{itemize}

The second Cousin problem is as follows:

Given a divisor $\{\mathscr{O}_i, g_i\}$ on $V$, does there exist a
meromorphic function $f$ on $V$ such that $f = \gamma_ig_i$  in
$\mathscr{O}_i$, where $\gamma_i$ and $\gamma^{-1}_i$ are holomorphic
in $\mathscr{O}_i$, i.e., does there exist one meromorphic function
$f$ on $V$ which defines the same divisor as $\{\mathscr{O}_i, g_i\}
?$ As in the case  of the first Cousin problem, this problem can be
generalized. 

Let $\{\mathscr{O}_i\}$ be an open covering of $V$ and let there be
given a holomorphic invertible function $\gamma_{ij}$ in
$\mathscr{O}_i \cap \mathscr{O}_j$ such that 
\begin{align*}
& \gamma_{ij} \gamma_{ji} =1 \text{ in } \mathscr{O}_i \cap
  \mathscr{O}_j, \\
& \gamma_{ij} \gamma_{jk} \gamma_{ki} = 1 \text{ in } \mathscr{O}_i
  \cap \mathscr{O}_j \cap \mathscr{O}_k. 
\end{align*}
Then, does there exist a holomorphic invertible function $\gamma_i$ in
$\mathscr{O}_i$ such that 
$$
\gamma_i \gamma^{-1}_j = \gamma_{ij} \text{ in } \mathscr{O}_i \cap
\mathscr{O}_j ?
$$

It is\pageoriginale easily seen that a solution of this problem leads
to a solution of the second Cousin problem: for if we define
$\gamma_{ij} = g_i g^{-1}_j$ in $\mathscr{O}_i \cap \mathscr{O}_j$
($\{\mathscr{O}_i, g_i\}$ is the second Cousin datum) and if
$\gamma_{ij} = \gamma_i \gamma^{-1}_j$  (also in $\mathscr{O}_i \cap
\mathscr{O}_j$), then $\gamma^{-1}_i g_i = \gamma^{-1}_j g_j$  in
$\mathscr{O}_i \cap \mathscr{O}_j$, and the meromorphic function $f$
defined by $f = \gamma^{-1}_i g_i$ in $\mathscr{O}_i$ solves the
second Cousin problem.

\begin{thm}\label{chap10:thm3}
The generalized second Cousin problem is always solvable for a
neighbourhood of a cube.
\end{thm}

\begin{proof}
Given the system $\{\mathscr{O}_i, \gamma_{ij}\}$ in a neighbourhood
of the cube $K$, the system defines an \textit{analytic line bundle}
over a neighbourhood of $K$ (linear boundle with fibre $C$). The
solubility of the second Cousin problem is precisely the triviality of
this line bundle over a neighbourhood of $K$, and this has been proved
in Theorem \ref{chap10:thm2}.
\end{proof}


\chapter*{Eexercise}

\begin{enumerate}
\item Let $\alpha (z)$\pageoriginale be a $C^\infty$-function with
  compact support in the plane. Let  
$$
f(z) = - \frac{1}{2 \pi i} \iint \alpha (\zeta) \frac{\bar{z} -
  \bar{z}}{\zeta - z} d \zeta \wedge d \bar{\zeta}. 
$$
Prove that $\dfrac{\partial^2 f}{\partial \bar{z}^2} = \alpha (z)$. If
$\alpha$ is only a distribution with compact support, prove that this
equation holds in the sense of distributions. Deduce that if $f$ is a
distribution such that $\dfrac{\partial^2 f}{\partial \bar{z}^2}$ is a
continuous functions, then so are $f$, $\dfrac{\partial f}{\partial
  z}$, $\dfrac{\partial f}{\partial \bar{z}}$. 

\item Let $V$ be a complex analytic manifold, and suppose that the
  generalized first  Cousin problem is always solvable on $V$. Prove
  that the generalized second Cousin problem is solvable, if it is
  ``differentiably solvable'' (in an obvious sense). 

Prove also that on the Riemann sphere, the first Cousin problem is
always solvable, while the second is not.

\item Let $K$ be a compact set in $C$. Prove that an analytical bundle
  differentiably trivial over a neighbourhood of $K$ is analytically
  trivial over a neighbourhood of $K$. 

\item Let $V$ be a complex analytic manifold and let
  $\{\mathscr{O}_i, c_{ij}\}$ define an analytic bundle over $V$ which
  is differentiably trivial. Let $\gamma_i$ be a $C^\infty$-function
  in $\mathscr{O}_i$ with $c_{ij} = \gamma_i \gamma^{-1}_j$ in
  $\mathscr{O}_i \cap \mathscr{O}_j$. Let $\alpha_i = \gamma^{-1}_i
  d'' \gamma_i$. Show that the $\alpha_i$ define a form $\alpha$ of
  type $(0,1)$. What relation does $\alpha$ satisfy?

Given a form $\alpha$ of type $(0,1)$ with $d'' \alpha + [\alpha,
  \alpha] =0$, show that it defines a class of analytic bundles over
$V$ which is differentiably trivial. When do\pageoriginale two such
forms $\alpha$ and $\beta$ define the same (analytic) class of
analytic bundles?

\item Generalize the results of exercise 4 to the case of a nontrivial
  class of differentiable bundles on $V$. 

As an application, prove that if $V$ is a Riemann surface and $E$ and
an arbitrary differentiable bundle on $V$, there always exists an
analytic bundle which is differentiably equivalent with $E$ (use a
device similar to that used in Step \ref{chap8:step1} in \ref{chap8},
Theorem \ref{chap8:thm2}.). 
\end{enumerate}

\begin{thebibliography}{99}
\bibitem{p2:key1} H. Cartan:\pageoriginale Sur les matrices holomorphes
  de $n$ variables complexes, \textit{Journal de Math\'ematiques}, 
  $9^e$-serie, 19 (1940), 1-26.

\bibitem{p2:key2} H. Cartan: \textit{S\'eminaire E.N.S.} 1951/52
  (especially lecture XVII by J. Frenkel). 

\bibitem{p2:key3} H. Cartan: \textit{S\'eminaire E.N.S.} 1953/54
  (especially lecture  XVIII by J-P.Serre).

\bibitem{p2:key4} J. L. Koszul and B. Malgrange: Sur certaines structures
  fibr\'es complexes (to appear in Archiv $f$-Mathematik). 

\bibitem{p2:key5} J. L. Lions: Lectures on Elliptic partial differential
  equations, Tata Institute of Fundamental Research, Bombay, 1957. 

\bibitem{p2:key6} G. de Rham: Vari\'et\'es diff\'erentiables, Hermann,
  Paris, 1955. 
\end{thebibliography}
