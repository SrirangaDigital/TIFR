
\chapter{Domains of Holomorphy: Convexity Theory}\label{chap6}

An\pageoriginale important problem is to find necessary and sufficient
conditions under which a manifold which can be spread in $C^n$ is a
domain of holomorphy. The results that are presented below are due to
CartanThullen \cite{p1:key2} (see also \cite{p1:key1}).

$S(z,r)$ is the open polydisc with centre $z$ and radius $r$ in $C^n$,
i.e., the set of points $z' \in C^n$ with 
$$
|z'_j - z_j| < r, \quad j = 1, \ldots, n.
$$

\setcounter{defn}{0}
\begin{defn}\label{chap6:def1}
Let $V$ be a complex manifold spread in $C^n$ by $\phi$. Let $z \in
V$. By the \textit{polydisc} $S(z, r) \subset V$ with centre $z$ and
radius $r$ is meant the open set $\mathscr{O}$ containing $z$ (if it
exists) such that the restriction of $\phi$ to $\mathscr{O}$ is an
analytic isomorphism of $\theta$ onto $S(\phi(z), r) \subset C^n$.

By the distance of $z \in V$ to the boundary of $V$, $d(z)$, is meant
the radius of the maximal polydisc $S(z, r) \subset V$; the distance
$d(K)$ of a compact set $K$ to the boundary of $V$ is
$\inf\limits_{z\in K} d(z)$. \{$d(z), d(K)$ depend of course, on
$\phi$\}. 
\end{defn}

\begin{defi*}
Let $K$ be a compact subset of the complex manifold $V$, and let
$\mathscr{C}$ be a family of holomorphic functions on $V$. The
$\mathscr{C}$-envelope of $K$, $\hat{K}_\mathscr{C}$ is the set of $z
\in V$ for which there exists a $c_z > 0$ such that for all $f\in
\mathscr{C}$, $|f(z)| \leq c_z ||f||_K $ ($||f||_K = \sup\limits_{x
  \in K} |f(x)|$, see p.2). 
\end{defi*}

\begin{examples*}
(a) Let $V = C$ be the complex plane, $\mathscr{C}$ the family of
  polynomials. If $K$ is a compact set $\subset C$ and $L$ is the
  union of the relatively compact components\pageoriginale of the
  complement of $K$ (relatively compact in $C$), then it can be shown
  that $\hat{K}_\mathscr{C} = K \cup L$. If, on the other hand, we
  consider $C^\ast = C - (0)$ and take $K$ to be an annulus enclosing
  $0$, $\mathscr{C}$ as the set of all holomorphic functions in
  $C^\ast$, then $\hat{K}_\mathscr{C} = K$. 

(b) In $V = C^2$ if we take $K$ to be the closure of the domain in the
  example on p.12, viz., $|z_1| \leq 1$, $|z_2| \leq \epsilon$; $1-\epsilon \leq
  |z_1| \leq 1$, $|z_2|\leq 1$, $\mathscr{C}$ to be the family of all
  polynomials (or all holomorphic functions) in $C^2$, then
  $\hat{K}_\mathscr{C}$ is the polydisc $|z_1|\leq 1$, $|z_2| \leq 1$.

(c) If on the manifold $V$, $\mathscr{C}$ is such that $f \in
  \mathscr{C}$ implies $f^p \in\mathscr{C}$ for every integer $p >0$,
  then 
$$
\hat{K}_\mathscr{C} = \left\{z \in V \big| |f(z) | \leq ||f||_K \text{
for every } f \in \mathscr{C} \right\}.
$$
It the set of points defined above is denoted $K_1$, clearly $K_1
\subset \hat{K}_\mathscr{C}$. If $z \in \hat{K}_\mathscr{C}$,
$|f(z)|^p \leq c_z ||f||^p_K$, $|f(z)| \leq c_z^{1/p} ||f||_K$ and
letting $p\to \infty$, it follows that $z \in K_1$.

Let $V$ be a complex manifold and let $\phi$ spread $V$ in $C^n$. Let
$f$ be a holomorphic function on $V$ and $z \in V$. Let $U$ be an open
neighbourhood of $z$ such that $\phi|U$ is an isomorphism. Then
$\dfrac{\partial f(z)}{\partial z_i}$ is defined to be
$\dfrac{\partial (f \circ \phi^{-1})(\phi(z))}{\partial z_i}$,
$\phi^{-1}$ being the inverse of $\phi|U$.
\end{examples*}

\setcounter{thm}{0}
\begin{thm}\label{chap6:thm1}
Let $V$ be a complex manifold and suppose that $\phi$ spreads $V$ in
$C^n$. Let $\mathscr{C}$ be a family of holomorphic functions on $V$,
stable for derivation, i.e., $f \in \mathscr{C}$ implies
$\dfrac{\partial f}{\partial z_i} \in \mathscr{C}$. Suppose also that
the canonial mapping of $V$ to the maximal continuation of $V$ is an
isomorphism. If $K$ is an arbitrary compact subset of $V$, then the
distances of $K$ and $\hat{K}_{\mathscr{C}}$ to the boundary of $V$
are the same. 
\end{thm}

\begin{proof}
Since\pageoriginale $V$ is itself maximal for $\mathscr{C}$, it is
clearly sufficient to prove the following:

If $z \in \hat{K}_{\mathscr{C}}$ and $0< \rho < d(K)$, then any $f
\in\mathscr{C}$ can be continued to $S (z, \rho)$, i.e., that the
functions $f \circ \phi^{-1}$ in a neighbourhood of $\phi(z)$, can be
continued to the polydisc $S(\phi(z), \rho) \subset C^n$. 

Let $L = \bigcup\limits_{x \in K} \overline{S(x,\rho)}$. Then $L$ is
the continuous image of the compact space $K \times
\overline{S(0,\rho) } \{S(0,\rho) \subset C^n\}$ and so $L$ is
compact. Let, for $f \in \mathscr{C}$, $M(f) = \sup\limits_{z \in L}
|f(z)|$. It follows from the Cauchy inequalities that $|D^J f(z)| \leq
\dfrac{J!M(f)}{\rho^{|J|}}$ (where $J = (j_1, \ldots, j_n)$, $D^J$ is the
operator $\dfrac{\partial^{|J|}}{\partial z^{j_1}_{1} \ldots \partial
  z^{j_n}_n}$, $|J| = j_1 + \cdots + j_n$ and $J! = j_1 !\ldots j_n
!$). From the definition of $\hat{K}_{\mathscr{C}}$ and the fact that
$\mathscr{C}$  is stable for derivation, it follows that if $z
\in\hat{K}_{\mathscr{C}}$, $|D^J f(z)| \leq \dfrac{C_z
  J!M(f)}{\rho^{|J|}}$. If, therefore, 
$$
g(z') = \sum\limits_{J \in N^n} \frac{D^J f(z)}{J!} (\phi (z') - \phi (z))^J
$$
for any $z \in \hat{K}_{\mathscr{C}}$, the series converges normally
in the polydisc $S (\phi (z),\break \rho) \subset C^n$ and it is clear that
it continues $f$ to $S (\phi(z), \rho)$ for every $f \in
\mathscr{C}$. The theorem follows.
\end{proof}

\begin{thm}\label{chap6:thm2}
Let $V$ be a complex analytic manifold, $\phi$ a spread of $V$ in
$C^n$. Let $\mathscr{C}$ be a family of holomorphic functions on $V$
having the following properties:
\begin{itemize}
\item[$(1^\circ)$] $\mathscr{C}$ is a closed subalgebra of $\mathscr{H}_V$,
$1 \in\mathscr{C}$.

\item[$(2^\circ)$] If $\phi(z) = \phi(z') (z \neq z')$, then there
  exists a function $f \in\mathscr{C}$ such that $f_z \neq f_{z'}$
  ($f_a$ is the germ $f_a = (f \circ \phi^{-1})_{\phi(a)}$; see IV and
  V).  

\item[$(3^\circ)$] If\pageoriginale $K$ is a compact subset of $V$ and
  $z \in \hat{K}_{\mathscr{C}}$, $S(z,r)$, the maximal (open)
  polydisc about $z$ in $V$ contains points not in
  $\hat{K}_{\mathscr{C}}$. 
\end{itemize}

Then, there exists a function $g \in \mathscr{C}$ such that $g$ cannot
be continued outside $(V, \phi)$. 
\end{thm}

\begin{proof}
The main step in the proof is to construct a function $g \in
\mathscr{C}$ such that
\begin{itemize}
\item[(a)] if $\phi(z) = \phi(z')$, then $g_z \neq g_{z'}$ and it is
  clearly enough that for a countable dense set $\{z_m\}$ on $V$,
  $g_{z_m} \neq g_{z'_m}$ if $z'_m$ is any point such that $\phi(z_m) =
  \phi(z'_m)$. (The existence of a countable dense set follows from
  the Poincar\'e-Volterra theorem);

\item[(b)] for a countable dense set $\{z_m\}$ of points on $V$, let
  $S (z_m, r_m)$ denote the maximal open polydisc about $z_m$. Then
  $g(z)$ has zeros of arbitrarily large multiplicity in every $S(z_m ,
  r_m)$. 
\end{itemize}
\end{proof}

The proof then divides into three steps. 

\begin{step}\label{chap6:step1}
 The existence of the function $g$ implies
that $(V, \phi)$ is the maximal domain for $g$. 
\end{step}

\begin{proof}
Let $(\tilde{V}, \tilde{\phi}, \tilde{\psi})$ be the maximal domain of
$g$. $V$ consists of pairs $(\phi(z), (g \circ \phi^{-1})_{\phi(z)}) =
(\phi(z), g_z)$. We show that $\tilde{\psi}$ is one-one. If
$(\phi(z), g_z) = (\phi(z'), g_{z'})$ then $\phi(z) = \phi(z')$, $g_z
= g_{z'}$ and since, if $z \neq z'$, $g_z \neq g_{z'}$ this implies
that $z = z'$. $\tilde{\psi}$ therefore identifies $V$ with an open
subset of $\tilde{V}$. In what follows, we assume this identification
made, and show that $V = \tilde{V}$. In the first place, $S (z_m,
r_m)$ is the maximal polydisc about $z_m$ in $\tilde{V}$ for if it
were not, $S(z_m, r_m)$ is relatively compact in $\tilde{V}$ and
$\tilde{g}$, the continuation of $g$ to $\tilde{V}$ cannot have zeros
of unbounded multiplicity in $S(z_m, r_m)$. Now suppose
that\pageoriginale $V \neq \tilde{V}$. Since $\tilde{V}$  is
connected, there is a point $b \in \tilde{V} $ so that $b \notin V$,
$b \in\bar{V}$. If $c$ is near enough to $b$ it is clear that there is
a polydisc $S(c, \rho)$ containing $b$. But if $c$ is a $z_m$, then
there is a polydisc $S (z_m, \rho) \subset \tilde{V}$, $\not\subset V$
which is not the case. This completes Step \ref{chap6:step1}. 
\end{proof}

\begin{step}\label{chap6:step2}
Construction of a funtion $f \in \mathscr{C}$ having property (b). 
Let $S_m = S(z_m, r_m)$ and consider the following sequence of
polydiscs:
$$
S_1, S_2, S_1, S_2, S_3, S_1, S_2, S_3, S_4, \ldots
$$
Denote its $p$-th term by $\sum_p$. Let $K_p$ be a sequence of compact
sets $\subset V$ such that $K_p \subset \overset{o}{K}_{p+1}$ and
$\bigcup\limits^\infty_{p =1} K_p = V$. Now, by property $3^\circ$ in
the hypotheses of Theorem \ref{chap6:thm2}, $\sum_p \not\subset
(\hat{K}_p)_{\mathscr{C}}$. Hence there is a function $h
\in\mathscr{C}$ and a point $z(p) \in \sum_p$ so that $|h (z^{(p)})| >
||h||_{K_p}$, by example $(c)$ on p.36, since $\mathscr{C}$ is an
algebra. If $f_p(z) = (h(z)/ h (z^{(p)}))^\nu$ and $\nu$ is large
enough, $f_p$ satisfies $f_p \in\mathscr{C}$, $|f_p(z)| \leq 2^{-p}$
on $K_p$, $f_p(z^{(p)}) =1$ with $z^{(p)} \in \sum_p$. Let $f=
\prod\limits^\infty_{p=1} (1-f_p)^p$. It is easily verified that this
product converges in $\mathscr{H}_V$ and that $f \not\equiv 0$. Since
$\mathscr{C}$ is a closed subalgebra of $\mathscr{H}_V$, $f
\in\mathscr{C}$. Also $f$ has a zero of order at least $p$ in $\sum_p$
and since each $S_m = S(z_m, r_m)$ occurs infinitely often in the
sequence $\{\sum_p\}$, this concludes Step \ref{chap6:step2}. 
\end{step}

\begin{step}\label{chap6:step3}
Modification of the function $f$, such that the resulting function has
properties (a) and (b). 
\end{step}

Let $\mathscr{C}_f$ be the closure of the set of all function $f h$,
$h\in \mathscr{C}$, where $f$ is the function constructed in Step
\ref{chap6:step2}. Since $\mathscr{C}$ is closed, $\mathscr{C}_f
\subset \mathscr{C}$ 
and trivially, each $g \in\mathscr{C}_f$ has property (b).

Let\pageoriginale $(X_m)$ be a countable dense set on $V$ and $(Y_m)$
the set of all points having $\phi(Y_m) = \phi(X_m)$ (this set is
countable by the Poincar\'e-Volterra theorem). Let
$\mathscr{O}(m,Y_m)$ be the set of functions $h \in \mathscr{C}_f$
such that $h_{X_m} \neq h_{Y_m}$ and $\phi(X_m) = \phi(Y_m)$.Clearly
$\mathscr{O} (m, Y_m)$ is open in $\mathscr{C}_f$. We prove below that
each $\mathscr{O}(m, Y_m)$ is dense in $\mathscr{C}_f$. It then
follows from Baire's theorem applied to $\mathscr{C}_f$
($\mathscr{C}_f$ is a complete metrizable space since $V$ is countable
at infinity) that $\mathscr{O} = \cap \mathscr{O} (m, Y_m)$ is dense
in $\mathscr{C}_f$ and if $g \in \mathscr{O}$, $g \not\equiv 0$, $g$
has properties (a) and (b). Thus to complete Step \ref{chap6:step3},
it remains only 
to prove that $\mathscr{O} (m, Y_m)$ is dense in $\mathscr{C}_f$.

Let $k \in\mathscr{C}_f$, $\phi(Y_m) = \phi(X_m)$. If $k_{Y_m} \neq
k_{X_m}$, $k \in \mathscr{O}(m, Y_m)$. If $k \not\in
\mathscr{O}(m,Y_m)$, let $h \in \mathscr{C}_f$ be so that $h_{X_m}
\neq h_{Y_m}$ ($h$ exists: for if $f$ has this property one may take
$h = f$ while if $fX_m = f_{Y_m}$ and $l$ is such that $l_{X_m} \neq
l_{Y_m}$ (hypothesis $(2^\circ)$) one may take $h = fl$). Then, if
$|\lambda|$ is small enough, $k + \lambda h $ defines different germs
at $X_m$ and $Y_m$, and is in the closure of the functions $k +
\lambda h$, $|\lambda|$ small, which are in $\mathscr{O}(m,Y_m)$ and
$k \in \overline{\mathscr{O} (m, Y_m)}$. This proves that
$\mathscr{O}(m, Y_m)$ is dense in $\mathscr{C}_f$ and thus completes
Step \ref{chap6:step3} and with Steps \ref{chap6:step1},
\ref{chap6:step2} and \ref{chap6:step3}, the proof of Theorem 
\ref{chap6:thm2} is complete.  

\begin{coro*}
Under the hypotheses of Theorem \ref{chap6:thm2} 
\begin{itemize}
\item[(i)] $(V, \phi)$ is a domain of holomorphy.

\item[(ii)] $(V, \phi)$ is the maximal continuation of $(V, \phi,
  \mathscr{C})$. 
\end{itemize}

Theorems \ref{chap6:thm1} and \ref{chap6:thm2} show that the following
are equivalent: 
\begin{itemize}
\item[(a)] $(V, \phi)$ is a domain of holomophy.

\item[(b)] If $\phi(z) = \phi(z')$, $z \neq z'$ there exists $f \in
  \mathscr{H}_V$ so that $f_z \neq f_{z'}$, and, if $K$ is a compact
  subset of $V$, $d(K) = d (\hat{K}_{\mathscr{H}_V})$. 
\end{itemize}

The\pageoriginale last condition may be replaced by the apparently
weaker condition that for any $z \in \hat{K}_{\mathscr{H}_V}$ the
maximal polydisc about $z$ contains points not in
$\hat{K}_{\mathscr{H}_V}$. Also, it follows from Theorems
\ref{chap6:thm1} and \ref{chap6:thm2} that 
if $V$ is a domain of holomorphy, then there exists a function $g \in
\mathscr{H}_V$ which separates points in the sense that if $z \neq
z'$, and $\phi(z) = \phi(z')$, then $g_z \neq g_{z'}$, such that $g$
cannot be continued outside $V$, i.e., if the family of all functions
on $V$ cannot be continued simultaneously outside $V$, there is one
function which itself cannot be continued. 
\end{coro*}
