
\chapter{Complex analytic manifolds}\label{chap3}

\begin{defi*}
A Hausdorff,\pageoriginale topological space $V^n$ is called a
\textit{topological manifold of dimension $n$} ($n \geq 0$ an integer)
if it has followsing property: every point $a \in V^n$ has a
neighbourhood homeomorphic to an open set $\Omega \subset R^n$.
\end{defi*}

(Note that a space having this property is not automatically
Hausdorff.)

$V^n$ is said to be countable at infinity, if $V^n$ is a countable
union of compact sets.

We next recall, without proof, the following two propositions.

\setcounter{proposition}{0}
\begin{proposition}\label{chap3:prop1}
If $V^n$ is connected the following three conditions are equivalent:
\begin{itemize}
\item[(1)] $V^n$ is countable at infinity.

\item[(2)] $V^n$ is paracompact (i.e. any open covering $(U_i)_{i\in
  I}$ admits a locally finite refinement, namelyl there is another
  open convering $(W_j)_{j\in J}$ each set of which is contained in at
  least one $U_i$ and such that any point has a neighbourhood
  intersecting only a finite number of the $W_j$)

\item[(3)] $V^n$ has a countable open base.
\end{itemize}
\end{proposition}

\begin{proposition}[Poincar\'e - Volterra theorem]\label{chap3:prop2}
If $V^n$, $W^n$ are two $n$ dimensional manifolds and if (1) $V^n$ is
connected, (2) $W^n$ is countable at infinity, (3) there exists a
continuous mapping $\phi: V^n \to W^n$ which is a local homeomorphism
(i.e. every $a \in V^n$ has an open neighbourhood which is mapped
homeomorphically on an open set of $W^n$). Then $V^n$ is\pageoriginale
countable at infinity.
\end{proposition}

\medskip
\noindent{\textbf{Differentiable manifolds.}}
Let $V^n$ be a topological manifold of dimension $n$. by an
(indefinitely) differentiable or $C^\infty$-structure on $V^n$ is
meant a family $\{\mathscr{O}_i\}_{i\in I}$ of open sets $\subset V^n$
which cover $V^n$, and mappings $\{f_i\}_{i\in I}$ such that $f_i$
maps $\mathscr{O}_i$ homeomorphically onto an open subset
$\tilde{\mathscr{O}_i} \subset R^n$ and such that the mappings $f_i
\circ f^{-1}_j$ and $f_j \circ f^{-1}_i$ are $C^\infty$-mappings of
$f_j(\mathscr{O}_i \cap \mathscr{O}_j)$, $f_i(\mathscr{O}_i \cap
\mathscr{O}_j)$  respectively, onto $f_i (\mathscr{O}_i \cap
\mathscr{O}_j)$, $f_j(\mathscr{O}_i \cap \mathscr{O}_j)$, i.e. if the
correspondence 
$$
f_i (\mathscr{O}_i \cap \mathscr{O}_j) \longleftrightarrow f_j
(\mathscr{O}_j \cap \mathscr{O}_j)
$$
is $C^\infty$.

(Note that it may happen that one can define more than one
differentiable structure on a manifold $V^n$, so that a
$C^\infty$-manifold is not a special topological manifold, but is a
topological manifold with an additional structure).

If $\{\mathscr{O}'_k, f'_k\}$ defines a $C^\infty$-structure on $V^n$,
we say that it defines the same structure as $\{\mathscr{O}_i, f_i\}$
if (and only if) the correspondence $f_i (\mathscr{O}_i \cap
\mathscr{O}'_k) \longleftrightarrow f'_k (\mathscr{O}_i \cap
\mathscr{O}'_k)$ is $C^\infty$. If the intersections $\mathscr{O}_i
\cap \mathscr{O}_j$ or $\mathscr{O}_i \cap \mathscr{O}'_k$ are empty
we take it that the condition is satisfied. We make a similar
convention whenever we speak of properties of mappings of
$\mathscr{O}_i \cap \mathscr{O}_j$ or $\mathscr{O}_i \cap
\mathscr{O}'_k$, without stating this explicitly.

If $\phi$ is a mapping of $V$ to $W$, where $V$, $W$ are
$C^\infty$-manifolds (not necessarily of the same dimension) with
$C^\infty$-structures $\{\mathscr{O}_i , f_i\}$, $\{\mathscr{O}'_j,
f'_j\}$ we say that $\phi $ is a \textit{differentiable or a
  $C^\infty$-mapping} if $f'_j \circ\phi \circ f^{-1}_i$ is a
$C^\infty$-mapping,\pageoriginale of $f_i (\mathscr{O}_i \cap
\varphi^{-1} (\mathscr{O}'))$. If $V$ and $W$ have the same dimension,
we say that $\phi$ is a \textit{diffeomorphism} if $\phi$ is a
homeomorphism of $V$ onto $W$ and if $\phi$ and $\phi^{-1}$ are
differentiable. 

\medskip
\noindent{\textbf{System of local coordinates}}

\begin{defi*}
Let $V$ be a $C^\infty$-manifold and let $a\in V$. If $(X_1, \ldots,
X_n)$ is a system of \textbf{real valued} functions in an open
neighbourhood $W$ of a such that the mapping $\phi:W \to R^n$ defined
by $b \to (X_1 (b), \ldots, X_n (b))$ for every $b\in W$, is a
diffeomorphism, then $(X_1, \ldots, X_n)$ are said to form a
\textit{system of local coordinates at $a$}.
\end{defi*}

If $Y_1, \ldots, Y_n$ are $C^\infty$-functions in a neighbourhood of
a, they form a system of local coordinates at $a$ if and only if the
Jacobian
$$
\frac{D(Y_1, \ldots, Y_n)}{D(X_1, \ldots, X_n)}
$$
does not vanish at $a$. The proof follows easily from the implicit
function theorem.

\medskip
\noindent{\textbf{Complex analytic manifolds.}}

\begin{defi*}
Let $\Omega$, $\Omega'$ be two open sets in $C^n$, $f_1, \ldots, f_n$
complex valued functions in $\Omega$. We say that $f = (f_1, \ldots,
f_n)$ is an \textbf{analytic isomorphism} of $\Omega$ onto $\Omega'$
if the mapping $f=(f_1, \ldots, f_n)$ of $\Omega \to C^n$ is a
diffeomorphism of $\Omega$ onto $\Omega'$ and the functions $f_1 ,
\ldots, f_n$ are holomorphic in $\Omega$.
\end{defi*}

The composite of analytic isomorphisms is an analytic
isomorphicm. Moreover, the inverse of an analytic isomorphism is also
one. This is seen as follows:

Let\pageoriginale $(f_1, \ldots, f_n)$ be $n$ holomorphic functions of
$z_1, \ldots, z_n$, and let $J = \det \left(\dfrac{\partial
  f_k}{\partial z_l} \right)$ and $f_k = f^{(1)}_k + i f^{(2)}_k$,
$z_k = x_k + i y_k$. Then the determinant $\begin{vmatrix}
  A & B\\ C & D \end{vmatrix}$, where 
$$
A = \left(\frac{\partial f^{(1)}_k}{\partial x_l} \right), \quad
B = \left(\frac{\partial f^{(2)}_k}{\partial x_l} \right), \quad 
C = \left(\frac{\partial f^{(1)}_k}{\partial y_l} \right), \quad 
D = \left(\frac{\partial f^{(2)}_k}{\partial y_l} \right) 
$$
is equal to $\begin{vmatrix}
A_1 & B_1\\ C_1 & D_1 \end{vmatrix}$, where 
$$
A_1 = \left(\frac{\partial f_k}{\partial z_l} \right), \quad 
B_1 = \left(\frac{\partial \bar{f}_k}{\partial z_l} \right), \quad 
C_1 = \left(\frac{\partial f_k}{\partial \bar{z}_l} \right), \quad 
D_1 = \left(\frac{\partial \bar{f}_k}{\partial \bar{z}_l} \right)
$$
and, by the Cauchy-Riemann equations, this equals $|J|^2$ so that,
since $(f_1, \ldots, f_n) = f$ is a diffeomorphism, $J\neq 0$. We can
now easily compute $\dfrac{\partial z}{\partial \bar{f}_k}$ and show
that it is zero which proves the statement.

\begin{defi*}
Let $V$ be a topological manifold of dimension $2n$. We shall identify
$R^{2n}$ with $C^n$. A system $\{\mathscr{O}_i , f_i\}_{i \in I}$,
where $\{\mathscr{O}_i\}_{i\in I}$ is an open covering of $V$ and
$f_i$ are mappings, $f_i : \mathscr{O}_i \to \widetilde{\mathscr{O}^{}}_i
\subset C^n$, is said to define a
\textit{complex analytic structure} if the mapping $f_i \circ
f^{-1}_k$ defines an analytic isomorphism of $f_j(\mathscr{O}_i \cap
\mathscr{O}_j)$ onto $f_i (\mathscr{O}_i \cap \mathscr{O}_j)$, i.e.,
if the correspondence
$$
f_i (\mathscr{O}_i \cap \mathscr{O}_j) \longleftrightarrow f_j
(\mathscr{O}_i \cap \mathscr{O}_j) 
$$
is an analytic isomorphism for every $i$, $j \in I$. Two systems 
$\{\mathscr{O}_i, f_i\}$, $\{\mathscr{O}'_k, f'_k\}$\pageoriginale
define the same complex analytic structure if the correspondence $f_i
(\mathscr{O}_i \cap \mathscr{O}'_k) \longleftrightarrow f'_k
(\mathscr{O}_i \cap \mathscr{O}'_k)$  is an analytic isomorphism for
every $i$, $k$.

We call $V$ a complex analytic manifold of (complex) dimension $n$.
\end{defi*}

\begin{defi*}
Let $V^n$, $W^m$ be two complex analytic manifolds, with structures
$\{\mathscr{O}_i, f_i\}$, $\{\mathscr{O}'_j, f'_j\}$ and $\phi$ a map
$V^n \to W^m$. $\phi$ is said to be an \textit{analytic mapping} if
the mappings $f'_j \circ \phi \circ f^{-1}_i$ are analytic mappings of
$\tilde{\mathscr{O}}_j$ to $C^m$, i.e., if the component functions are
holomorphic in $\tilde{\mathscr{O}}_i$. $\phi$ is called an
\textit{analytic isomorphism} of $V^n$ onto $W^m$ if it is an analytic
map and is, moreover, a diffeomorphism.
\end{defi*}

\medskip
\noindent{\textbf{Local coordinates.}}

Let $V$ be a complex, analytic manifold. A system of $n$ complex
valued functions $(z_1, \ldots, z_n)$ in a neighbourhood of a point $a
\in V$ is said to form a system of local coordinates at $a$ if there
exists an open set $W$, $a \in W$ such that the mapping $\phi : W \to
\tilde{W} \subset C^n$ defined by $b \to (z_1(b), \ldots, z_n (b))$ is
an analytic isomorphism $n$ holomophic functions $t_1, \ldots, t_n$
(i.e., analytic mappings into $C^1$) in a neighbourhood of $a$ form
system of local coordinates if and only if the determinant 
$$
J = \det \left[ \frac{\partial t_k}{\partial z_l}\right]_a \neq 0.
$$
This follows from the fact that Jacobian of the $\mathscr{R}t_i$,
$\mathfrak{J} t_j$ in terms of the $\mathscr{R} z_i$, $\mathfrak{J}
z_j$ equals $|J|^2 \neq 0$ and the remark on the inverse of an
analytic isomorphism. 

The question arises as to which theorems on holomorphic functions in
an open set $\Omega \subset C^n$ generalize to holomorphic functions
on a complex analytic\pageoriginale manifold $V$. The following
theorems, and their proofs, do. 
\begin{itemize}
\item[(1)] $\mathscr{H}_V$ is closed in $\mathscr{C}_V$ (the notation
  is the obvious one).

\item[(2)] A closed bounded set $\Phi$ in $\mathscr{H}_V$ is
  compact. The diagonal process fails if $V$ is not countable at
  infinity, but it is easy to prove that any ultrafilter in $\Phi$
  converges, by considering, for any compact $K \subset V$, the
  restrictions to $K$ of the functions in the elements of the
  ultrafilter.

\item[(3)] If $\Phi \subset \mathscr{H}_V$ (or $\mathscr{C}_V$) and
  $\sum\limits_{f\in\Phi} |f(a)| < + \infty$ for every $a\in V$, then
  there exists an open set $V' \subset V$ dense in $V$ such that
  $\Phi_{V'}$ (the set of restrictions to $V'$ of the functions of
  $\Phi$) is a bounded set in $\mathscr{H}_{V'} $ (respectively
  $\mathscr{C}_{V'}$).
\end{itemize}

This is because Baire's theorem (used in I, Prop.4) is true for open
subsets of locally compact spaces or complete metric spaces (and any
manifold is locally compact).

It is also sometimes of interest to apply Baire's theorem in
$\mathscr{H}_V$, but this necessitates the assumption that $V$ is
countable at infinity when $\mathscr{H}_V$ is a Fr\'echet space and so
a complete metric space.

\medskip
\noindent{\textbf{The principle of analytic continuation.}}

Let $V$ be a connected, $W$ an arbitrary, complex analytic manifold,
and let $f_1$, $f_2$ be two analytic mappings of $V$ into $W$. Let
$\{\mathscr{O}_i, \phi_i\}$, $\{\mathscr{O}'_j, \phi'_j\}$ define the
structures on $V$, $W$ respectively. 

Let $a \in \mathscr{O}_i$, $f_1(a) = f_2 (a) \in \mathscr{O}'_j$. We
say that $f_1$, $f_2$ have the same derivatives at $a$ if the
components of the mappings $\phi'_j \circ f_1 \circ \phi^{-1}_{i}$ and
$\phi'_j \circ f_2 \circ \phi^{-1}_j$ have the same derivatives (of
all orders) at $\phi_i(a)$. This definition is independent of the
$\mathscr{O}_i, \mathscr{O}'_j$ containing $a$, $f_1(a)$ respectively,
and of the systems chosen to define the analytic structures. 

The\pageoriginale following theorem then holds:

\begin{theorem*}
\begin{itemize}
\item[\rm (1)] (The strong principle of analytic continuation). 

If $f_1$ and $f_2$ and all their derivatives coincide at a point $a\in
V$, then $f_1 = f_2$ everywhere on $V$. 

\item[{\rm (2)}] (The weak principle of analytic continuation). 

If $f_1 = f_2$ in an open set on $V$, $f_1 = f_2$ everywhere on $V$. 
\end{itemize}
\end{theorem*}

\begin{proof}
The weak principle follows at once from the strong principle so that
we have only to prove the latter. Let $E$ be the set of points where
$f_1$ and $f_2$ and all their derivatives coincide. It is clear that
$E$ is closed. Suppose now that $b \in E$. Then all the components of
the functions $\phi'_k \circ f_1 \circ \phi^{-1}_l$ and $\phi'_k \circ
f_2 \circ \phi^{-1}_l (b \in\mathscr{O}_l, f_1(b) = f_2 (b) \in
\mathscr{O}'_k)$ and all their derivatives coincide at $\phi_l(b)$. By
expanding in power series about $\phi_l(b)$, it follows that these
components and their derivatives coincide in a neighbourhood of
$\phi_l(b)$ so that $E$ is open. Since $a \in E$ and $V$ is connected,
$E=V$ and the theorem follows.
\end{proof}
