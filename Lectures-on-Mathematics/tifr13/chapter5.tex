
\chapter{Envelopes of Holomorphy}\label{chap5}

Let\pageoriginale $V$ be a complex analytic manifold of (complex)
dimension $n$ and 
let $\phi$ spread $V$ in $C^n$. Let $F = (f_i)_{i\in I}$ be a subset
of the set of all holomorphic functions on $V$. We say that $(V, \phi,
(f_i)_{i \in I})$ is continuted to $(V', \phi', \psi, (f'_i)_{i\in
  I})$ if there exists a complex analytic manifold $V'$, a spread
$\phi'$ of $V'$ in $C^n$, a system $(f'_i)_{i \in I}$ of holomorphic
functions $f'_i$ on $V'$ and a local isomorphism $\psi$ of $V$ into
$V'$ such that $\phi = \phi' \circ \psi$ and $f_i = f'_i \circ \psi$
for $i \in I$. This process is called \textit{simultaneous
  continuation} of $(f_i)_{i \in I}$ to $(V', \phi')$ from $(V,
\phi)$. A \textit{maximal continuation} of $(f_i)_{i \in I}$ is a
continuation $(\tilde{V}, \tilde{\phi}, \tilde{\psi},
(\tilde{f}_i)_{i\in I})$ such that if $(V', \phi', \psi', (f'_i)_{i
  \in I})$ is any continuation of $(V, \phi, (f_i)_{i \in I})$, then
there is a local isomorphism $\chi$ of $V' \to \tilde{V}$ such that
for all $i \in I$, $f'_i = \tilde{f}_i \circ \chi$ and $\tilde{\psi} =
\chi \circ \psi'$. It follows that $\phi' = \tilde{\psi} \circ \chi$. 

We can prove the existence and uniquencess of a maximal (simultaneous)
continuation of a given system $(V, \phi, (f_i)_{i \in I})$ in a
manner similar to the proof of the theorem in IV.

Consider an open neighbourhood $U$ of a point $a \in C^n$. Let
$(g_i)_{i \in I}$ be a family of holomorphic functions in $U$ (indexed
by I). Let $(g'_i)_{i \in I}$ be another such family defined in a
neighbourhood $U'$ of $a$. Identify $(g_i)_{i \in I}$ and $(g'_i)_{i
  \in I}$ if there existsaneighbourhood $W$ of $a$, $W \subset U \cap
U'$ such that $g_i = g'_i$ in $W$ for every $i \in I$. Denote by
$(g_i)_a$ an equivalence class of the set of all families $(g_i)_{i
  \in I}$ of holomorphic functions, such that all functions of one
family are defined in a fixed neighbourhood of $a$ by the equivalence
relation defined by this identification. The set of all $(g_i)_a$ is
denoted by $\mathscr{O}_{I,a}$. Let $\mathscr{O}_{I} =
\bigcup\limits_{a \in C^n} \mathscr{O}_{I,a}$. Then
$\u{\mathscr{O}}_I$ is a sheaf. The topology on $\u{\mathscr{O}}_I$ is
defined\pageoriginale exactly as before: if $(a, (g_i)_a) \in
\u{\mathscr{O}}_I$ and $U$ is a neighbourhood of $a$, $(g_i)_{i \in
  I}$ a family of holomorphic functions in $U$ defining $(g_i)_a$,
then $\mathscr{U} = \bigcup\limits_{b \in U} (b,(g_i)_b)$ [$(g_i)_b$
  is the equivalence class defined by $(g_i)_{i \in I}$ at $b$] is an
open neighbourhood of $(a, (g_i)_a)$. Exactly as in IV, we put on
$\u{\mathscr{O}}_I$ a complex analytic structure and define a mapping
$\psi$ from $(V, \phi)$ to $\u{\mathscr{O}}_I: \psi(a) = (\psi(a),
(g_i \circ \phi^{-1})_{\phi(a)})$ and show that this indeed gives us a
maximal continuation. The uniqueness is proved in the same way as in
IV.

The most important case is that in which $F$ consists of all
holomorphic functions on $V$. In this case the maximal continuation
$(\tilde{V}, \tilde{\phi}, \tilde{\psi})$ is called the
\textit{envelope of holomorphy} of $(V, \phi)$. 

Let $V$ be a complex analytic manifold, $\phi_1$, $\phi_2$ two local
analytic isomorphisms of $V$ into $C^n$. Let $F = (f_i)_{i \in I}$ be
a family of holomorphic functions on $V$. Let $(\tilde{V}_1,
\tilde{\phi}_1, \tilde{\psi}_1, \tilde{f}_{1i})$, $(\tilde{V}_2,
\tilde{\phi}_2, \tilde{\psi}_2, \tilde{f}_{2i})$ be the maximal
continuations of $(V, \phi_1, f_i)$, $(V, \phi_2, f_i)$
respectively. The two continuations are said to be \textit{isomorphic}
if there exists an analytic isomorphism $\chi$ of $\tilde{V}_1$ onto
$\tilde{V}_2$ such that $\tilde{\psi}_2 = \chi \circ \tilde{\psi}_1$
and $\tilde{f}_{1i} = \tilde{f}_{2i} \circ \chi$ for all $i \in I$. 

We have the following 

\setcounter{thm}{0}
\begin{thm}\label{chap5:thm1}
Let $F = (f_i)_{i \in I}$ consist of all the holomorphic functions on
the complex analytic manifold $V$. Let $\phi_1$, $\phi_2$ be two maps
which spread $V$ in $C^n$. Let $(\tilde{V}_1, \tilde{\phi}_1,
\tilde{\psi}_1)$, $(\tilde{V}_2, \tilde{\phi}_2, \tilde{\psi}_2)$ be
the envelopes of holomorphy of $(V, \phi_1)$, $(V, \phi_2)$
respectively. Then $(\tilde{V}_1, \tilde{\phi}_1, \tilde{\psi}_1)$ and
$(\tilde{V}_2, \tilde{\phi}_2, \tilde{\phi}_2)$ are isomorphic.
\end{thm}

\begin{proof}
Consider\pageoriginale the components of the mapping $\phi_2 : V \to
C^n$. They are holomorphic functions on $V$ and they can be continued
to $\tilde{V}_1$ and this gives us a mapping $\phi'_2$ of
$\tilde{V}_1$ to $C^n$, such that $\phi_2 = \phi'_2 \circ
\tilde{\psi}_1$. Let $J$ be the Jacobian of $\phi_2$ with respect to
the local coordinates defined by $\phi_1$. Then $J$ is a holomorphic
function on $V$, and since $\phi_2$ is a local isomorphism, $J \neq 0$
so that $1 /J$ is holomorphic on $V$. Hence $1/J$ (resp. $J$) has a
continuation $(\widetilde{1/J})_1$ (resp. $\tilde{J}_1$) to
$\tilde{V}_1$. Clearly we have $\widetilde{(1/J)_1} X\tilde{J}_1 =1 $
on the image of $V$ by $\tilde{\psi}_1$ in $\tilde{V}_1$ and since
$\tilde{V}_1$ is connected, $(\widetilde{1/J})_1 X \tilde{J}_1 =1$
everywhere on $V_1$ so that $\tilde{J}_1 \neq 0$ throughout
$V_1$. Moreover $\tilde{J}_1$ is the Jacobian of $\phi'_2$ with
respect to the local coordinates defined by $\tilde{\phi}_1$ so that
$\psi'_2$ \textit{is a local analytic isomorphism}, and spreads
$\tilde{V}_1$ in $C^n$. The situation is explained by the following
diagram:
\[
\xymatrix@R=1.3cm{
 & & & V \ar[dl]^{\tilde{\psi}_1}\ar[dr]_{\tilde{\psi}_2}
  \ar[ddlll]_{\phi_1}\ar[ddrrr]^{\phi_2} & & &\\  
 &  & \widetilde{V}_{1}
  \ar@<.3em>[rr]^{\chi_1}\ar@{<-}@<-.3em>[rr]_{\chi_2} \ar[dll]^{\tilde{\phi}_1} 
  \ar[drrrr]_{\phi'_2} && \widetilde{V}_{2}\ar[drr]_{\tilde{\phi}_2} &
  &\\  
 C^n_1 & & & & & & C^n_2 
}
\]

Since $(\tilde{V}_2, \tilde{\phi}_2, \tilde{\psi}_2)$ is maximal, this
implies that there exists a local analytic isomorphism $\chi_1:
\tilde{V}_1 \to \tilde{V}_2$ such that $\tilde{\psi}_2 = \chi_2 \circ
\tilde{\psi}_1 $ and $\tilde{f}_{1i} = \tilde{f}_{2i} \circ
\chi_1$. In the same way, we prove that there exists a local analytic
isomorphism $\chi_2 : \tilde{V}_2 \to \tilde{V}_1$ such that
$\tilde{\psi}_1 = \chi_2 \circ \tilde{\psi}_2$. 

It follows\pageoriginale that $\tilde{\psi}_2 = \chi_1 \circ \chi_2
\circ \tilde{\psi}_2$ so that $\chi_1 \circ \chi_2 = I_{\tilde{V}_2}$
(the identity mapping of $\tilde{V}_2$) on the image of $V$ under
$\tilde{\psi}_2$ in $\tilde{V}_2$ and hence, by the principle of
analytic continuation, on all $\tilde{V}_2$. Similarly $\chi_2 \circ
\chi_1 = I_{\tilde{V}_1}$ on $\tilde{V}_1$, so that $\chi_1$ is an
analytic isomorphism of $\tilde{V}_1$ onto $\tilde{V}_2$. Since
$\tilde{\psi}_2 = \chi_1 \circ \tilde{\psi}_1$ and $\tilde{f}_{1i} =
\tilde{f}_{2i} \circ \chi_1$, this proves the theorem. 

We can prove also the following
\end{proof}

\begin{thm}\label{chap5:thm2}
Let $V$, $V'$ be complex analytic manifolds, and let $\phi$, $\phi'$
spread $V$, $V'$ in $C^n$. Let $g$ spread $V$ in $V'$. (It is not
required that $\phi = \phi' \circ g$. Let $(\tilde{V}, \tilde{\phi},
\tilde{\psi})$ , $(\tilde{V}', \tilde{\phi}', \tilde{\psi}')$ be the
envelopes of holomorphy of $(V, \phi)$, $(V', \phi')$. Then there
exists a local analytic isomorphism $\tilde{g}$ of $\tilde{V}'$ such
that the following diagram is commutative:
\[
\xymatrix@C=1.5cm@R=1.5cm{
V \ar[r]^g \ar[d]_{\tilde{\psi}} & V' \ar[d]^{\tilde{\psi}'}\\
\tilde{V} \ar[r]_{\tilde{g}}  &  \tilde{V}'
}
\]
\end{thm}

\begin{proof}
Consider the following diagram:
\[
\xymatrix{
& V \ar[rrr]^g \ar[dr]^{\tilde{\psi}} \ar[ddl]_{\phi} & & & 
  V'\ar[ddr]^{\phi'}\ar[dl]_{\tilde{\psi}'}  & \\
& & \tilde{V} \ar[r]^{\tilde{g}} \ar[dll]_{\tilde{\phi}}
  \ar[drrr]_{\bar{\phi}} & \tilde{V}' \ar[drr]^{\tilde{\phi}'} & &  
  \\
C^n_1 &  & & & & C^n_2
}
\]

$V$ is spread in $C^n$ by $\phi' \circ g$ and by Theorem \ref{chap5:thm1}, since
$\tilde{V}$ is maximal, there is a local analytic isomorphism
$\bar{\phi} : \tilde{V} \to C^n$ such that $\phi' \circ g = \bar{\phi}
\circ \tilde{\psi}$. Also,\pageoriginale $\tilde{V}'$ is a maximal
continuation for the functions on $V$ induced by $g$ from those on
$V'$, so that, since $\tilde{V}$ is maximal for \textit{all} functions on $V$,
$\tilde{V}$ is ``intermediate'' between $V$ and $\tilde{V}'$ and there
is a spread $\tilde{g}$ of $\tilde{V}$ in $\tilde{V}'$ with the
properties we require.

This has the following
\end{proof}

\begin{coro*}
Let $V$ be a complex analytic manifold and $\sigma$ an analytic
automorphism of $V$. Let $V$ be spread in $C^n$ by $\phi$, and
$(\tilde{V}, \tilde{\phi}, \tilde{\psi})$ be the envelope of
holomorphy of $(V, \phi)$. Then, there exists an analytic automorphism
$\tilde{\sigma}$ of $\tilde{V}$ such that the following diagram is
commutative: 
\[
\xymatrix@R=1.5cm@C=1.5cm{
V \ar[r]^\sigma \ar[d]_{\tilde{\psi}} & V \ar[d]^{\tilde{\psi}}\\
\tilde{V} \ar[r]_{\tilde{\sigma}} & \tilde{V}
}
\]
\end{coro*}

\begin{defn}\label{chap5:def1}
Let $V$ be a complex analytic manifold, $\phi$ a spread of $V$ in
$C^n$. Let $(\tilde{V}, \tilde{\phi}, \tilde{\psi})$ be the envelope
of holomorphy of $(V, \phi)$. 

$(V, \phi)$ is called a \textit{domain of holomorphy} of
$\tilde{\psi}$ is an analytic isomorphism of $V$ onto $\tilde{V}$.

Note that, in the definition above, $(V, \phi)$ is not called a domain
of holomorphy if $V$ and $\tilde{V}$ are isomorphic, but only if
$\tilde{\psi}$ is an isomorphism. Nevertheless, Theorem
\ref{chap5:thm1} shows that 
if $V$ is a complex analytic manifold, $\phi_1$, $\phi_2$, two
mappings which spread $V$ in $C^n$, $(V, \phi_1)$ is a domain of
holomorphy if and only if $(V, \phi_2)$ is. For, if $(\tilde{V}_1,
\tilde{\phi}_1, \tilde{\psi}_1)$, $(\tilde{V}_2, \tilde{\phi}_2,
\tilde{\psi}_2)$ are the envelopes of $(V, \phi_1)$, $(V, \phi_2)$,
then there is an analytic isomorphism $\chi$ of $V_1$ onto
$\tilde{V}_2$ such that $\tilde{\psi}_2 = \tilde{\psi}_1 \circ \chi$
and $\tilde{\psi}_1$ is an isomorphism if and only if $\tilde{\psi}_2$
is. This justifies the following 
\end{defn}


\begin{defn}\label{chap5:def2}
Let\pageoriginale $V$ be a complex analytic manifold which can be
spread in $C^n$. Then $V$ is called a \textit{domain of holomorphy},
if, for some spread $\phi$ of $V$ in $C^n$, $(V, \phi)$ is a domain of
holomorphy.
\end{defn}

Theorem \ref{chap5:thm1} is, in general, false, if, instead of
considering the family 
of all holomorphic functions on $V$, we take only a subfamily. The
following is a counter-example.

Let $\Gamma$ be the complex $\zeta$-plane, $C$ the complex
$z$-plane. Spread $\Gamma$ in $C$ by the two mappings $\phi_1 (\zeta)
= \zeta$ and $\phi_2 (\zeta) = e^\zeta$ respectively. Let $f(\zeta) =
e^{\zeta}$. The maximal continuation of $(\Gamma, \phi_1, f)$ is $(C,
\tilde{f}_1)$  where $\tilde{f}_1 (z) = e^z$. Since $\Gamma$ is the
universal convering surface of $C^\ast = C-(0)$ under the projection
$\phi_2 (\zeta)$ (which naturally has the same value at all points
lying over one point on $C^\ast$) and $f$ as a mapping $\Gamma \to C$
coincides with $\phi_2$, the maximal continuation of $(\Gamma, \phi_2,
f)$ is $(C, \tilde{f}_2)$ where $\tilde{f}_2(z)=z$. But (since
e.g. $z$ has zeros and $e^z$ has not) there is no isomorphism of $C$
into itself taking $z$ to $e^z$.
