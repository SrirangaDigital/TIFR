\chapter{Holomorphic Regular Matrices}\label{chap9}

In IX\pageoriginale and $X$, we shall prove analogues of the theorems
of VIII for holomorphic functions whose \textit{values are regular}
(invertible) \textit{matrices}, and give some applications of these
generalizations. We begin by restating Grothendieck's theorem in a
form which can be carried over.

Let $\alpha$ be an arbitrary differentiable $(0,1)$ form in a
neighbourhood of the cube $K$. The necessary and sufficient condition
that there exists a $C^\infty$-function $f$ in a neighbourhood of $K$
such that $f\neq 0$ at any point, and 
$$
f^{-1} d'' f = \alpha
$$
is that $d'' \alpha =0$. (We have only to find $g$ so that $d'' g
=\alpha$ and set $f= \exp g$). 

In what follows, the functions or forms considered have values or
coefficients in the space of $(m \times m)$ complex matrices or in the
full linear group $GL(m,C)$ or regular matrices.

Our aim will be to generalize the above result to the case when
$\alpha$ is a $(0,1)$ form whose coefficients are $(m \times m)$
complex matrices and $f$ is replaced by a mapping in $GL(m,C)$. We
shall need to generalize the lemma of VIII.

\setcounter{thm}{0}
\begin{thm}\label{chap9:thm1}
Let $K$ be a rectangle in the $C$-plane, $L$, $M$ compact sets in
$C^l$, $C^m$ respectively. Let $\alpha (z, \lambda, \mu)$ be a matrix
valued function, defined in a neighbourhood of $K\times L \times M$
such that it is a differentiable function of allits variables and a
holomorphic function of $\lambda$ in this neighbourhood. Then there
exists a $C^\infty$-function $f(z, \lambda, \mu)$ in a neighbourhood
of $K \times L \times M$ with values in $GL(M,C)$ which is
differentiable in all its variables, is holomorphic in $\lambda$ and
is such that 
$$
\frac{\partial f}{\partial \bar{z}} = f \cdot \alpha. 
$$\pageoriginale
We need some lemmas. The first two will not be proved here.
\end{thm}

\setcounter{lem}{0}
\begin{lem}\label{chap9:lem1}
Let $B$ be a Banach space, $\mathscr{L}, \mathfrak{m}$ open sets in
$C^l, C^m$ respectively and $U(\lambda, \mu)$ a continuous linear
operator $B \to B$ which has an inverse, for every $(\lambda, \mu) \in
\mathscr{L} \times \mathfrak{m}$. Suppose that $U(\lambda, \mu)$ is a
$C^\infty$-function of $\lambda$ and $\mu$ and is holomorphic in
$\lambda$. Suppose, moreover, that $X(\lambda, \mu)$ is a
differentiable function of $\lambda, \mu$ with values in $B$, which is
holomorphic in $\lambda$. 

Then $U^{-1} (\lambda, \mu) X (\lambda, \mu)$ has also these
properties. 
\end{lem}

\begin{lem}\label{chap9:lem2}
Let $\mathscr{O}$ be an open set in the plane and $H$ an open set in
$C^h$. Let $f(z, \eta)$ be a continuous function of the set of all
continuous functions of $z$ in $\mathscr{O}$ with values in the space
of differentiable functions of $\eta$ in $H$. Suppose that the
derivatives
$$
\frac{\partial k_f}{\partial \bar{z}^k}, \quad k =0, 1, 2, \ldots
$$
(in the sense of distributions) all exist and have the same
properties. 

Then $f(z, \eta)$ is an indefinitely differentiable function in
$\mathscr{O} \times H$. 

This is a particular case of a theorem on the ``regularity in the
interior'' of solutions of elliptic partial differential
equations. See, for example, Lions \cite{p2:key5} (also exercise 1). The
present situation involves vector functions with values in the space
of differentiable functions, but the proof remains valid.
\end{lem}

The proof of Theorem \ref{chap9:thm1} will be in three parts. 

\medskip
\noindent{\textbf{First Part:\pageoriginale Proof of Theorem
    \ref{chap9:thm1} in the     particular case when $\alpha$ is
    ``sufficiently near to zero''.}} 
[The last phrase means the following: if $K'$, $L'$, $M'$ are compact
  neighbourhoods of $K$, $L$, $M$ respectively such that $K' \times L'
  \times M'$ is contained in the domain of definition of $\alpha$,
  then $||\alpha|| < C (K', L', M')$ for a suitable $C$($||\alpha||$
  will denote, in what follows
  $\max\limits_{i,j}||\alpha_{ij}||_{K'\times L'\times M'}$ if $\alpha 
  = (\alpha_{ij})$)].

The lemma that follows is the crucial step in the first part. 

\begin{lem}\label{chap9:lem3}
Suppose that $\alpha (z, \lambda, \mu)$ satisfies the hypothesis of
Theorem \ref{chap9:thm1} and that $\alpha$ is ``sufficiently near
zero''. Then there exists a function $\beta (z, \lambda, \mu)$ in a
neighbourhood of $K \times L \times M$ which is $C^\infty$ in $z,
\lambda, \mu$ and holomorphic in $\lambda$ is such that  
$$
\frac{\partial \beta}{\partial \bar{z}} + [\alpha, \beta] =
\frac{\partial \alpha}{\partial z}.
$$
($[\alpha, \beta]$ stands, as usual, for $\alpha \beta - \beta \alpha$).
\end{lem}

\begin{proof}
Let $\gamma(z)$ be a $C^\infty$-function which is 1 in an open
neighbourhood of $K$ and is zero near the boundary of $K'$. Consider
the following integral equation (writing $[\alpha, \beta]$ for
$[\alpha, \beta] (\zeta, \lambda, \mu)$):
\begin{multline*}
\beta (z, \lambda, \mu) + \frac{1}{2\pi i} \iint\limits_{K'} \gamma
      [\alpha, \beta] \frac{1}{\zeta-z}\\ d \zeta \wedge d\bar{\zeta} =
      \frac{1}{2\pi i} \iint\limits_{K'}  \frac{\partial
        \alpha}{\partial \zeta} \frac{1}{\zeta -z}  d \zeta \wedge d
      \bar{\zeta} 
\end{multline*}
Consider the Banach space $B$ of all continuous functions on $K'$
whose values are $m \times m$ matrices, with norm $||\beta||$ (defined
as for $\alpha$) for $\beta \in B$. Let $A(\lambda, \mu)$ denote the
operator defined by
\begin{multline*}
A (\lambda, \mu) \beta (z) = \frac{1}{2 \pi i} \iint\limits_{K'}
\gamma (\zeta) \\
\{\alpha (\zeta, \lambda, \mu) \beta (\zeta) - \beta(\zeta) \alpha
(\zeta, \lambda, \mu)\} \frac{1}{\zeta-z} d \zeta \wedge d \bar{\zeta}
\end{multline*}
for $z \in K'$. The integral equation can then be written 
$$
(I + A (\lambda, \mu)) \beta = X (\lambda, \mu)
$$\pageoriginale
where 
$$
X(\lambda, \mu) (z) = \frac{1}{2\pi i} \iint\limits_{K'} \gamma \cdot
\frac{\partial \alpha}{\partial \zeta}  \frac{1}{\zeta -z} d \zeta
\wedge d \bar{\zeta}.
$$
Now, $X(\lambda, \mu)$ and $A(\lambda, \mu)$ are differentiable in
$\lambda, \mu$ and holomorphic in $\lambda$. It is clear from the
definition of $A(\lambda, \mu)$ that if $\alpha$ is sufficiently near
zero, $||A(\lambda ,\mu)|| \leq \theta < 1$ for $(\lambda, \mu) \in L'
\times M'$. Consequently, $I + A(\lambda, \mu)$ has an inverse for
every $(\lambda, \mu) \in \overset{\circ}{L'} \times
\overset{\circ}{M'}$, and so, by Lemma \ref{chap9:lem1}, $(I + A(\lambda, \mu))^{-1}
X (\lambda, \mu)$ is differentiable in $\lambda$, $\mu$ and
holomorphic in $\lambda$, and the integral equation has a solution
$\beta(z, \lambda, \mu)$ which has the following properties:
\begin{itemize}
\item[(1)] $\beta$ is a differentiable function of $(\lambda, \mu) \in
  \overset{\circ}{L'} \times \overset{\circ}{M'}$ with values in $B$; 

\item[(2)] $\beta$ is holomorphic in $\lambda$. 
\end{itemize}
From (\ref{chap9:eq1}) it follows that $\beta$ is a continuous function of $\zeta
\in \overset{\circ}{K'}$ with values in the space of all
differentiable function of $(\lambda, \mu)$ in $\overset{\circ}{L'}
\times \overset{\circ}{M'}$. 

Now, if $g(z) = \dfrac{1}{z}$, $\dfrac{1}{2\pi i} \iint\limits_{K'}
\gamma (\zeta) f(\zeta) \dfrac{1}{\zeta -z} d \zeta \wedge d
\bar{\zeta} = \dfrac{1}{\pi} g^\ast \gamma f$ ($\ast$ being
convolution). Since $\dfrac{\partial g}{\partial \bar{z}} = \pi
\delta_o$  ($\delta_o$ is the Dirac distribution at $0$; this is
essentially the lemma proved before Grothendieck's theorem), it
follows that 
$$
\frac{\partial \beta}{ \partial \bar{z}} =  - [\alpha, \beta] +
\frac{\partial \alpha}{\partial z}
$$
in $\mathscr{O} \times \overset{\circ}{L'} \times \overset{\circ}{M'}$
(in the sense of distributions). Since the terms on the right are
continuous functions of $z$ (with values in the space of
differentiable functions in $\overset{\circ}{L'} \times
\overset{\circ}{M'}$), so is $\dfrac{\partial\beta}{\partial \bar{z}}$
and so
$$
\frac{\partial^2 \beta}{\partial \bar{z}^2} = - \left[\frac{\partial
    \alpha}{\partial \bar{z}}, \beta \right]  - \left[\alpha,
  \frac{\partial \beta}{\partial \bar{z}} \right] + \frac{\partial^2
  \alpha}{\partial z \partial \bar{z}}
$$\pageoriginale
has the same property. By iteration
$$
\frac{\partial^k \beta}{\partial \bar{z}^k} \text{ is continuous for }
k \geq 0,
$$
and, by Lemma \ref{chap9:lem2}, $\beta$ is $C^\infty$ in a
neighbourhood of $K \times L \times M$. This proves Lemma \ref{chap9:lem3}.  
\end{proof}

\medskip
\noindent{\textbf{Proof of Theorem \ref{chap9:thm1} in the particular case.}} Let
$\mathscr{O}$ be an open rectangle $\subset C $, $K \subset
\mathscr{O}$ and let $\mathscr{L}, \mathfrak{m}$  be open
neighbourhoods of $L$, $M$ respectively, such that $\mathscr{O} \times
\mathscr{L} \times \mathfrak{m}$ is contained in the domain of
definition of $\alpha$. We shall find a differentiable, regular matrix
$f$ such that 
\begin{equation*}
\dfrac{\partial f}{\partial z} = f\cdot \beta, \quad
  \dfrac{\partial f}{\partial \bar{z}} = f \cdot \alpha, \quad f(0)
  = I \text{ (unit matrix).  } \tag{1}\label{chap9:eq1}
\end{equation*}
If we put $f(tz) = \phi_z (t)$, $\phi(t) = \phi_z (t)$ satisfies
\begin{equation*}
\begin{cases}
\phi(0) & = I \\
\dfrac{d\phi}{dt} & = z \phi(t) \beta (tz) + \bar{z} \phi (t) \alpha
(tz) = \phi(t) \cdot A,
\end{cases} \tag{2}\label{chap9:eq2}
\end{equation*}
if $f$ satisfies (\ref{chap9:eq1}). By the classical theorems on
systems of ordinary equations of the form (\ref{chap9:eq2}), a
solution $\phi(t)$ of (\ref{chap9:eq2}) exists, is  
$C^\infty$ in $z$, $\lambda$, $\mu$, holomorphic in $\lambda$ and is
unique; thus, $f$, if it exists, is uniquely given by $f(z) = \phi_2
(1)$. 

Let now $\phi$ be the solution of (\ref{chap9:eq2}); \textit{define} $f$ by $f(z) =
\phi_z(1)$. Then $f$ is $C^\infty$ in $\mathscr{O} \times \mathscr{L}
\times \mathfrak{m}$ and holomorphic as a function of $\lambda$. We
shall show that 
\begin{itemize}
\item[(i)] $f(z)$ is a regular matrix;

\item[(ii)] $f$ satisfies the equation (\ref{chap9:eq1}). 
\end{itemize}

\medskip
\noindent{\textbf{Proof of (i):}} $\phi$\pageoriginale satisfies
\begin{align*}
\phi(0) & = I, \\
\frac{d\phi}{dt} & = \phi \cdot A.
\end{align*}

Let $\psi$ be the unique solution of 
\begin{align*}
\psi(0) & = I, \\
\frac{d\psi}{dt} & = - A \cdot \psi.
\end{align*}
Then $\dfrac{d(\phi \cdot \psi)}{dt} =0$, $\phi \psi = \phi(0) \psi
(0)= I$ so that, in particular, $f(z)\cdot \psi (1) = I$. 

\medskip
\noindent{\textbf{Proof of (ii):}}
Let $g_z(t) = \partial \phi_z (t) / \partial \bar{z}$. We shall prove
that 
$$
g_z(t) = t\phi_z (t) \alpha (tz) = h_z(t)
$$
(the second equality is a definition) which implies (ii). Now, for
$t=0$, $g_z (0) = h_z (0) =0$, and $g_z (t)$ satisfies the equation
\begin{equation*}
\frac{dg_z}{dt} = g_z \{ z \beta (tz) + \bar{z} \alpha (tz)\} + \phi_z
\alpha (tz) + \phi_z \{zt \frac{\partial \beta}{\partial \bar{z}} (tz)
+ \bar{z} t \frac{\partial \alpha}{\partial \bar{z}} (tz)\}
\tag{3}\label{chap9:eq3}
\end{equation*}

Since $\dfrac{\partial \beta}{\partial \bar{z}} + [\alpha, \beta] =
\dfrac{\partial \alpha}{\partial z}$, the equation (\ref{chap9:eq3}) remains valied
if $g_z$ is replaced by $h_z$. Since $g_z(0) = h_z(0) (=0)$, it
follows from the uniqueness of a solution of an equation of the form
(\ref{chap9:eq3}) that $g_z (t) \equiv h_z (t)$ and this completes the proof of
(ii). This concludes the proof of Theorem \ref{chap9:thm1} in the particular case. 

\begin{coro*}
Under the hypothesis of Theorem \ref{chap9:thm1}, every point of $K$ has an open
neighbourhood $U$ with a function $f(z, \lambda, \mu)$, $C^\infty$ in
all its variables and holomorphic in $\lambda$ in a neighbourhood of
$U \times L \times M$ such that 
$$
\frac{\partial f}{\partial \bar{z}} = f\alpha \text{ in a
  neighbourhood of } U \times L \times M.  
$$
\end{coro*}

\medskip
\noindent{\textbf{Proof of the corollary.}} It\pageoriginale is enough
to prove thsi for the point $0 \in K$. 

Let
$$
\alpha_t = \alpha (tz, \lambda, \mu), \quad t \geq 0.
$$
It is clearly sufficient to find $t>0$ such that for $\alpha_t$ there
is a function $f_t$ in a neighbourhood of $K \times L \times M$ with
the required properties of regularity so that $\dfrac{\partial
  f_t}{\partial \bar{z}} = tf_t \alpha_t$ (by setting $f(z) = f_t
\left( \dfrac{z}{t} \right)$ in a neighbourhood of $z=0$).

It $t$ is small enough, $t \alpha_t$ is near zero and by the
particular case of Theorem \ref{chap9:thm1}, the matrix $f_t$ exists.

Before continuing with the proof of Theorem \ref{chap9:thm1}, we shall deduce from
the preceding results the following theorem of H. Cartan, which is all
that will be required in the later theory. 

\medskip
\noindent{\textbf{Second Part:}}
\textit{Theorem on holomorphic regular matrices.}

\begin{thm}\label{chap9:thm2}
Let $K$ be a rectangle in the complex plane and $L$, $M$ two compact
sets in $C^l$, $C^m$ respectively. Let $H$ be the intersection of $K$
with the line $\mathscr{R} z =0$. Let $C(z, \lambda, \mu)$ be a
$C^\infty$-function in a neighbourhood of $H \times L \times M$ which
is holomorphic in $z$ and in $\lambda$ with values in $GL(m,C)$. Let
$K_1 = K \cap \{z \in C \big| \mathscr{R} z \geq 0\}$, $K_2 = K \cap
\{z \in C \big| \mathscr{R} z \leq 0\}$. Then there exist functions
$C_1 (z, \lambda, \mu)$, $C_2 (z, \lambda, \mu)$ in neighbourhoods of
$K_1 \times L \times M$, $K_2 \times L \times M$ satisfying the same
regularity conditions as $C$ and such that, in a neighbourhood of $H
\times L \times M$,
$$
C = C_1 C^{-1}_2. 
$$
\end{thm}

\begin{proof}
The proof will be given first in the case when $C$ differs little from
the indentity matrix $I$ in a sense which is obvious. Let $H'$ be a
rectangle with sides parallel to the coordinate axes in the plane
containing $H$ such\pageoriginale that $H' \times L \times M$ is
contained in the domain of definition of $C$. Then $log C$ is defined
(as $\exp^{-1}(C)$) and is near zero if $C$ is near $I$ in $H' \times
L \times M$. Let $\phi$ be a $C^\infty$-function in a neighbourhood of
$K$ such that $\phi(z) =1$ if $\mathscr{R} z \geq \epsilon$,  $= 0$ if
$\mathscr{R} z \leq - \epsilon $ ($\epsilon$ so chosen that the intersection of
$K$ with the strip $|\mathscr{R}_z| \leq \epsilon $ is contained in $H'$).

Now define
$$
\gamma^{-1}_2 = \exp [\phi \log C]
$$
and $\gamma_1 = C \gamma_2$ in a neighbourhood of $H' \times L \times
M$. $\gamma_2$ is extended to a neighbourhood of $K_2 \times L \times
M$ by setting $\gamma^{-1}_2 = I$ for $\mathscr{R} z \leq - \epsilon$, and
$\gamma_1$ to a neighbourhood of $K_1 \times L \times M$ by setting
$\gamma_1 = I$ for $\mathscr{R}z \geq \epsilon$. Then we have 
$$
C = \gamma_1 \gamma^{-1}_2. 
$$
Also, if $C$ is near $I$, $\gamma_1, \gamma_2$ are near $I$, while
$\dfrac{\partial \gamma_1}{\partial \bar{z}}$ , $\dfrac{\partial
  \gamma_2}{\partial \bar{z}}$ are near $0$. Since $C$ is holomorphic,
we have 
$$
C \frac{\partial \gamma_2}{\partial \bar{z}} = \frac{\partial
  \gamma_1}{\partial \bar{z}},
$$
and 
$$
\gamma^{-1}_1 \frac{\partial \gamma_1}{\partial \bar{z}} =
\gamma^{-1}_2 \frac{\partial \gamma_2}{\partial \bar{z}} = \alpha. 
$$
Since $\alpha$ is near 0 if $C$ is near, $I$, there exists a
$C^\infty$-function $f(z, \lambda, \mu)$ holomorphic in $\lambda$ in a
neighbourhood of $K \times L \times M$ such that 
$$
f^{-1} \frac{\partial f}{\partial \bar{z}} = \alpha.
$$
If 
\begin{align*}
C_1 = \gamma_1 f^{-1}, \; C_2 = \gamma_2 f^{-1}, \;  \frac{\partial
  C_1}{\partial \bar{z}} & = - \gamma_1 f^{-1} \frac{\partial
  f}{\partial \bar{z}} f^{-1} + \frac{\partial \gamma_1}{\partial
  \bar{z}} f^{-1}\\
& = -\gamma_1 \alpha f^{-1} + \gamma_1 \alpha f^{-1} = 0,
\end{align*}
so that $C_1$ and similarly $C_2$ are holomorphic in $z$ and
$\lambda$. 

Clearly,\pageoriginale 
$$
C = C_1 C^{-1}_2
$$
in a neighbourhood of $H \times L \times M$.

To prove Theorem \ref{chap9:thm2} in the general case, we proceed as
follows. Let $C$ 
be any holomorphic regular matrix in a neighbourhood of $H \times L
\times M$. Then there exists a matrix $C'$ holomorphic in $z$ and
$\lambda$ in a neighbourhood of $K \times L \times M$ (and even in one
of $C \times L \times M$, $C$ being the complex plane) which
approximates to $C$ (one has only to approximate the entries of $C$),
and so, there is a $C'$ so that $C'^{-1} C$ is near $I$. By the
particular case proved above, there exist holomorphic regular matrices
$C'_1, C_2$ in neighbourhoods of $K_1 \times L \times M$, $K_2 \times
L \times M$ respectively so that 
$$
C'^{-1} C = C'_1 C^{-1}_2
$$
and so, we have 
$$
C = C_1 C^{-1}_2,
$$
where $C_1 = C' C'_1$. This concludes the proof of Cartan's theorem on
holomorphic regular matrices. 
\end{proof}

\medskip
\noindent{\textbf{Third Part: Proof of Theorem \ref{chap9:thm1} in the general
    case.}}

After the corollary to the particular case of Theorem
\ref{chap9:thm1}, we can divide 
$K$ into a finite number of closed rectangles $K_i$ with sides
parallel to the axes of coordinates and obtain functions $f_i$ in
neighbourhoods of $K_i$ so that $\dfrac{\partial f_i}{\partial
  \bar{z}} = f_i \alpha$ and $f_i$ has the required regularity
properties. We require a certain function $f$ in the whole of $K$. It
is easy to see that it is enough to solve the following problem: given
two functions $f_1$, $f_2$ in neighbourhoods of two adjacent
rectangles $K_1$, $K_2$ such that in neighbourhoods of $K_1 \times L
\times M$, $K_2 \times L \times M$, respectively,
$$
\frac{\partial f_1}{\partial \bar{z}} = f_1 \alpha, \quad
\frac{\partial f_2}{\partial \bar{z}} = f_2 \alpha
$$\pageoriginale
find a function $f$ in a neighbourhood of $(K_1 \cup K_2) \times L
\times M$ with $\dfrac{\partial f}{\partial \bar{z}} = f\alpha$. Now,
since $f^{-1}_1 \dfrac{\partial f_1}{\partial \bar{z}} = f^{-1}_2
\dfrac{\partial f_2}{\partial \bar{z}}$, the function $c = f_1
f^{-1}_2$ is $C^{\infty}$ in a neighbourhood of $H \times L \times M$
($H$ being the common side of ($K_1$ and $K_2$) and holomorphic in $z$
and $\lambda$.) Consequently, by Theorem \ref{chap9:thm2}, there exist
matrices $c_1$, 
$c_2$ in neighbourhoods of $K_1 \times L \times M$, $K_2 \times L
\times M$ holomorphic in $z$ and $\lambda$, so that $c = c_1 c^{-1}_2$
in a neighbourhood of $H \times L \times M$. Then $c = c_1 c^{-1}_2 =
f_1 f^{-1}_2$ and $c_1 f^{-1}_1 = c_2 f^{-1}_2$ in a neighbourhood of
$H \times L \times M$. If we define $f = c_1 f^{-1}_1$ in a
neighbourhood of $K_1 \times L \times M, = c_2 f^{-1}_2$ in a
neighbourhood of $K_2 \times L \times M$, then $\dfrac{\partial
  f}{\partial \bar{z}} = c^{-1}_1 \dfrac{\partial f_1}{\partial
  \bar{z}} = c^{-1}_1 f_1 \alpha  = f\alpha$ if $z$ belongs to a
neighbourhood of $K_1$, and the same equation holds also in a
neighbourhood of $K_2 \times L \times M$. This completes the proof of
Theorem \ref{chap9:thm1} in the general case. 
