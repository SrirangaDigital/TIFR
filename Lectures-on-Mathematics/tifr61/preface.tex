\thispagestyle{empty}
\begin{center}
{\Large\bf Lectures on}\\[5pt]
{\Large\bf Wave Propagation}
\vfill

{\bf By}
\medskip

{\large\bf G.B. Whitham}
\vfill

{\bf Tata Institute of Fundamental Research}
\medskip

{\bf Bombay}
\medskip

{\bf 1979}
\end{center}

\eject

\thispagestyle{empty}
\begin{center}
{\Large\bf Lectures on}
\medskip

{\Large\bf Wave Propagation}\\
\vfill

{\bf By}\\
\medskip

{\large\bf G. B. Whitham}
\vfill

Published for the
\medskip

{\bf Tata Institute of Fundamental Research, Bombay}
\medskip

{\bf Springer--Verlag}
\medskip

Berlin Heidelberg New York
\medskip

{\bf 1979}
\end{center}

\eject

\thispagestyle{empty}

\begin{center}
{\Large\bf Author}
\medskip

{\large\bf G.B. Whitham, F.R.S.}\\
Applied Mathematics 101-50\\
Firestone Laboratory\\
Galifornia Institute of Technology\\
Pasadena, California 91125, U.S.A.
\vfill

{\bf\copyright \quad Tata Institute of Fundamental Research, 1979}
\vfill

\noindent\rule{\textwidth}{1pt}

ISBN 3-540-08945-4 Springer-Verlag Berlin. Heidelberg.New York

ISBN 0-387-08945-4 Springer-Verlag New York.Heidelberg.Berlin

\noindent\rule{\textwidth}{1pt}
\vfill

\parbox{0.7\textwidth}{%
No part of this book may be reproduced in any form
by print, microfilm or any other means without written
permission from the Tata Institute of Fundamental
Research, Colaba, Bombay 400 005}
\vfill

Printed by M. N. Palwankar at the TATA PRESS Limited,\\
414, Veer Savarkar Marg, Bombay  400 025 and published by\\ H. Goetze,
Springer-Verlag, Heidelberg, West Germany\\
\vfill

Printed In India
\end{center}

\eject

\chapter{Preface}

\markboth{Preface}{Preface}

THESE ARE THE lecture notes of a course of about twentyfour lectures
given at the T.I.F.R. centre, Indian Institute of Science, Bangalore,
in January and February 1978. 


The first three chapters provide basic background on the theory of
characteristics and shock waves. These are meant to be introductory
and are abbreviated versions of topics in my book ``Linear and
nonlinear waves'', which can be consulted for amplification. 


The main content is an entirely new presentation. It is on water
waves, with special emphasis on old and new results for waves on a
sloping beach. This topic was chosen as a versatile one where an
enormous number of the methods and techniques used in applied
mathematics could be illustrated on a single area of application. In
the relatively short time availabel, I wanted to avoid spending time
on the formulation of problems in different areas. Waves on beaches
together with ramifications to islands, tsunamis, etc., is also a very
active field of research. 


In any current course on wave propagation, it seemed essential to
mention, at least, the quite amazing results being found on exact,
solutions for the Korteweg-de Vries equation and related
equations. Since this has now become such a huge subject, the choice
was to present a new approach we have developed (largely by
R. Rosales), rather than review the original and alternative
approaches. Since the Kortewegde Vries equation and its solutions
originated in water wave theory, this fits well with the other
material. Like the other topics, the mathematical results go far
beyond this original field and have many other applications.  
\eject

The enthusiasm and participation of the audience made this the most
enjoyable teaching experience I have ever had. I wish to thank the
students, faculty and N.A.L. participants for their kindness and
stimulation. 


Notes were taken by G. Vijayasundaram and P.S. Datti, and I thank them
for their devoted efforts.  

Professors K.G. Ramanathan and K. Balagangadharan gave most generously
of their time and energy to make all aspects of our visit smooth and
enjoyable. We are sincerely grateful. 
\bigskip

\begin{flushright}
{\large\bf G.B. Whitham}
\medskip

Pasadena, California\\

August, 1978
\end{flushright}
