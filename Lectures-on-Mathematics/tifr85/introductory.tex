\chapter*{Introductory Remarks}
\markboth{Introductory Remarks}{Introductory Remarks}

Riemann zeta-function\pageoriginale $\zeta(s) (s = \sigma + it)$ is defined in $\sigma >1$ by
\begin{equation*}
\zeta(s) = \sum\limits^\infty_{n=1} n^{-s} \left( = \prod\limits_p (1-p^{-s})^{-1} \right) \tag{1} \label{eq1}
\end{equation*}
where the product in the parenthesis is over all primes. The identity connecting the series in (\ref{eq1}) with the product is the well-known Euler product. Euler knew very much more about the series in (\ref{eq1}). He knew things like
\begin{align*}
\zeta(s) & = \sum\limits^{\infty}_{n=1} \left( n^{-s} - \int\limits^{n+1}_n \frac{du}{u^s}\right) + \sum\limits^\infty_{n=1} \int\limits^{n+1}_n \frac{du}{u^2} \\
& = \sum\limits_{n=1} \left(n^{-s} - \int\limits^{n+1}_n \frac{du}{u^s} \right) + \frac{1}{s-1}, \quad (\sigma > 0). \tag{2}\label{eq2}
 \end{align*}
Also by the repetition of the trick by which we obtained (\ref{eq2}) from the series in (\ref{eq1}) we can prove that the series in (\ref{eq2}) is an entire function (a fact known to Euler). Moreover he knew certainly bounds for the absolute value of the series in (\ref{eq2}) and its analytic continuation in the form
\begin{equation*}
|\zeta(s) - \frac{1}{s-1}| \ll_A (|t|+10)^{A+2} \;\;  (\sigma \geq -A, A \geq 0) 
\tag{3}\label{eq3}
\end{equation*}
and also things like
\begin{equation*}
|\zeta(s) -\frac{1}{s-1}| \ll (|t|+10)^{1-\sigma}\log (|t|+10), \; \; (0\leq \sigma \leq 1). \tag{4}\label{eq4}
\end{equation*}
Euler knew even the functional equation of $\zeta(s)$ (see A. Weil, \cite{Weil1}, p. 261-266). However the question of the distribution of the zeros of $\zeta(s)$ was raised by Riemann\pageoriginale who initiated some important researches. Riemann conjectured that
\begin{equation*}
\zeta (s) \neq 0, (\sigma > \frac{1}{2}), \tag{5}\label{eq5}
\end{equation*}
and from the functional equation it was an obvious deduction from this that
\begin{equation*}
\zeta(s) = 0 (0 \leq \sigma \leq 1) \text{ implies } \sigma = \frac{1}{2}. 
\tag{6}\label{eq6}
\end{equation*}
This is the famous Riemann conjecture. This being an intractable problem at present (it has withstood the attacks of many important mathematicians like G.H. Hardy for more than a century) we ask: What are some important consequences of (\ref{eq5})? Can we prove any of them without assuming (\ref{eq5})? I mention four outstanding unsolved problems which follow as consequences (\ref{eq5}).

\medskip
\noindent\textbf{First Consequence.} For every fixed $\epsilon >0$, we have
\begin{equation*}
\zeta(\frac{1}{2} + it) t^{-\epsilon} \to 0 \text{ as } t \to \infty.  \tag{7}\label{eq7}
\end{equation*}

\begin{remark}%%% 1
In fact J.E. Littlewood proved that (\ref{eq5}) implies things like
\begin{equation*}
\zeta \left(\frac{1}{2} + it \right) \Exp \left( -\frac{10 \log t}{\log \log t}\right) \to 0 \text{ as } t \to \infty. \tag{8}\label{eq8}
\end{equation*}
\end{remark}

\begin{remark}%%% 2
The latest unconditional result in (\ref{eq7}) is with $\epsilon > \dfrac{89}{570}$ due to M.N. Huxley. The truth of (\ref{eq7}) for every $\epsilon > 0$ is called Lindel\"of hypothesis. 
\end{remark}

\medskip
\noindent{\textbf{Second Consequence.}} Consider the rectangle
\begin{equation*}
\left\{\sigma \geq \alpha,   0 \leq t \leq T\right\} \quad \left( \frac{1}{2} \leq \alpha \leq 1, T \geq 10 \right). \tag{9}\label{eq9}
\end{equation*}
The number of zeros of $\zeta(s)$ in this rectangle does not exceed
\begin{equation*}
C(\epsilon) T^{(2+ \epsilon) (1-\sigma)} (\log T)^{100},\tag{10}\label{eq10}
\end{equation*}
for every $\epsilon > 0$, provided we assume (\ref{eq7}). (Consequences like this were deduced from (\ref{eq7}) for the first time by A.E. Ingham). The unconditional results $\epsilon = \dfrac{2}{3}, \dfrac{1}{2}$ and\pageoriginale $\dfrac{2}{5}$ were obtained by A.E. Ingham, H.L. Montgomery and M.N. Huxley respectively. The truth of (\ref{eq10}) for every $\epsilon > 0$ is called Density hypothesis.

\medskip
\noindent{\textbf{Third Consequence.}} Let $p_1 = 2, p_2 = 3, p_3 = 5, \ldots $ be the sequence of all primes. Then A.E. Ingham deduced from (\ref{eq10}) that
\begin{equation*}
p_{n+1} - p_n \leq C (\epsilon)p^{\frac{1}{2} + \epsilon}_n \tag{11}\label{eq11}
\end{equation*}
holds for every $\epsilon > 0 $ ($C(\epsilon)$ may be different from the one in (\ref{eq10})). His unconditional result $\epsilon = \frac{1}{8}$ in (\ref{eq11}) does not need the functional equation or the approximate functional equation. However all the results with $\epsilon < \frac{1}{8}$ which followed later need the functional equation. M.N. Huxley's result $\epsilon = \frac{2}{5}$ in (\ref{eq10}) implies an asymptotic formula for the number of primes in $(x,x + h)$ where $h = x^\lambda$ with $\lambda > \frac{7}{12}$. D.R. Heath-Brown has an asymptotic formula even when $h = x^{\frac{7}{12}} (\log x)^{-1}$ and slightly better results. All these results depend crucially on the deep result (\ref{eq13}) of I.M. Vinogradov. The latest unconditional result in (\ref{eq11}) is with $\epsilon > \frac{1}{22}$ due to S.T. Lou and Q. Yao, two Chinese students of H. Halberstam. The unconditional improvements from $\epsilon > \frac{1}{12}$ to $\epsilon > \frac{1}{22}$ are very difficult and involve ideas of H. Iwaniec, M. Jutila and D.R. Heath-Brown.

\medskip
\noindent{\textbf{Fourth Consequence.}} The consequence (\ref{eq7}) of (\ref{eq5}) implies
\begin{equation*}
\zeta(\sigma + it) t^{-\epsilon} \to 0 \text{ as } t \to \infty \tag{12}\label{eq12}
\end{equation*}
for all fixed $\sigma$ in $\frac{1}{2} \leq \sigma < 1$ and for every fixed $\epsilon >0$. (For $\sigma =1$ this is trivially true).

\setcounter{remark}{0}
\begin{remark}%%% 1
It is a pity that we do not know the truth of (\ref{eq12}) for any $\sigma, \left(\frac{1}{2} \leq \sigma < 1 \right)$.
\end{remark}

\begin{remark}%% 2
The most valuable and the most diffcult result in the whole of the theory of the Riemann zeta-function in the direction of (\ref{eq12}) is a result due to\pageoriginale I.M. Vinogradov (for reference see A.A. Karatsuba's paper \cite{Karatsuba1}) which states that for $\frac{1}{2} \leq \sigma \leq 1$ we have
\begin{equation*}
\left| \zeta(s) - \frac{1}{s-1} \right| \leq ((|t| + 10)^{(1-\sigma)^{\frac{3}{2}}} \;  \log  (|t| + 10)) \tag{13}\label{eq13}
\end{equation*}
where $A$ is a certain positive constant. Actually Vinogradov proved that in (\ref{eq13}) RHS can be replaced by
$$
\left( (|t| + 10)^{(1-\sigma)^{\frac{3}{2}}} + 10 \right)^A \log (|t| + 10).
$$
The inequality  (\ref{eq13}) implies that
\begin{equation*}
\pi (x) - li \; x = O (x \Exp (-c(\log x)^{\frac{3}{5}} (\log \log x)^{-\frac{1}{5}})), \tag{14}\label{eq14}
\end{equation*}
where $\pi (x) = \sum\limits_{p \leq x} 1$, $li \; x = \int\limits^x_2 \frac{du}{\log u}$ and $c$ is a positive numerical constant. This is the best known result to day as regards upper bounds for the LHS of (\ref{eq14}). Riemann's hypothesis (\ref{eq5}) implies in an easy way that LHS of (\ref{eq14}) is $O(x^{\frac{1}{2}} \log x)$. It must be mentioned that $O(x \Exp (-c (\log x)^{\frac{1}{2}}))$ is an easy result which follows from (\ref{eq13}) with $1 - \sigma$ in place of $(1-\sigma)^{\frac{3}{2}}$, which is a very trivial result. The inequality (\ref{eq13}) also implies that for $t \geq 200$,
\begin{equation*}
\zeta(1+ it) = O((\log t)^{\frac{2}{3}} (\log \log t)^{\frac{4}{3}}). 
\tag{15}\label{eq15}
\end{equation*}
Besides proving (\ref{eq13}) Vinogradov proved that in (\ref{eq15}) we
can drop $(\log\break\log t)^{\frac{4}{3}}$. As a hybrid of this result and
(\ref{eq13}) H.-E. Richert proved that the R.H.S. in (\ref{eq13}) can
be replaced by 
\begin{equation*}
O((|t| +10)^{100(1-\sigma)^{\frac{3}{2}}} (\log (|t| + 10))^{\frac{2}{3}}). \tag{16}\label{eq16}
\end{equation*}
See also the paper \cite{Ramachandra and Sankaranarayanan7} by K. Ramachandra and A. Sankaranarayanan.
\end{remark}

\begin{remark}%%% 3
Although the best known bound for $|\zeta(1+it)| (t \geq 1000)$ is $O((\log t)^{\frac{2}{3}})$, due to I.M. Vinogradov, we have still a long way to go since one can deduce in a simple way the bound $O(\log \log t)$ from (\ref{eq5}). In fact to deduce this result\pageoriginale it is enough to assume that the least upper bound $\theta$ for the real parts of zeros of $\zeta(s)$ is $<1$. The only information about $\theta$ available today is $\frac{1}{2} \leq \theta \leq 1$. An excellent reference article for many of the facts mentioned above is A.A. Karatsuba \cite{Karatsuba1}.

It must be mentioned that the results mentioned above serve as a motivation for many result proved in the theory of $\zeta(s)$. However $I$ concentrate on what I have called Titchmarsh's phenomenon (i.e. $\Omega$ theorems and mean-value theorems). I will also consider a few other problems like the proof (due to J.B. Conrey, A. Ghosh and S.M. Gonek) that $\zeta(s)$ has infinity of simple zeros in $t \geq 1$. In short this monograph is meant to be a short appendix to the famous book of E.C. Titchmarsh on the Riemann zeta-function. I do not deal with complicated results like N. Levinson's result on the critical  zeros, and the results of A. Selberg and D.R. Heath-Brown on Levinson's simple zeros, R. Balasubramanian's result on the mean square of $|\zeta (\frac{1}{2} + it)|$ and its latest improvements by D.R. Heath-Brown and M.N. Huxely, D.R. Heath-Brown's result on the mean fourth power and the mean twelfth power, the improvement of the error them (in the fourth power mean result of D.R. Heath-Brown) by N. Zavorotnyi, H.Iwaniec's contribution to the fourth and twelfth power moments and M. Jutila's approach to these and more general problems, the contribution to higher power moments on the lines $\sigma > \frac{1}{2}$ due to S. Graham, the results of N.V. Kuznetsov (to be corrected by Y. Motohashi), and the results of J.L. Hafner, A. Ivi\'c and Y. Motohashi. In short it is meant to be a readable appendix which is not too complicated. Before closing the introduction I would like to mention in connection with the second consequence that after Ingham's contribution $\epsilon = \frac{2}{3}$, a good amount of later researches were inspired by the conditional result (depending on Lindel\"of hypothesis) that for every fixed $\epsilon > 0$ and every fixed $\delta > 0$, there holds 
$$
\lim\limits_{T \to \infty} T^{-\epsilon} N \left(\frac{3}{4} + \delta, T \right) = 0
$$
where $N(\alpha, T)$ is the number of zeros of $\zeta(s)$ in the rectangle (\ref{eq9}). This conditional\pageoriginale result is due to G. Hal\'asz and P. Tur\'an. In connection with the second and the third consequences we have to mention the pioneering works of F. Carleson and G. Hoheisel. In connection with the fourth  consequence we have to mention the pioneering works of G.H. Hardy, J.E. Littlewood and H. Weyl. For these we refer to the work of A.A. Karatsuba cited above. Regarding the prime number theorems of Hadamard and de la Vall\'ee Poussin we refer the reader once again to the work of Karatsuba.
\end{remark}

\begin{center}
{\textbf{Notes at the end of Introductory remarks}}
\end{center}


The equation (\ref{eq8}) was proved with some unspecified constant in place of 10 by J.E. Littlewood. By using the method of A. Selberg, K. Ramachandra and A. Sankaranarayanan have shown that it is possible to replace 10  by a constant which is less than $\frac{9}{16}$ (see \cite{Ramachandra and Sankaranarayanan5}).

The inequality (\ref{eq10}) was first proved with $\epsilon =2$ by
F. Carleson. This result and the earlier results in the direction of
(\ref{eq13}) due to G.H. Hardy, J.E. Littlewood and (independently by)
H. Weyl were used by G. Hoheisel to prove (\ref{eq11}) with some
$\epsilon < \frac{1}{2}$. The earlier results just referred to,
implied that the RHS in (\ref{eq14}) could be replaced by 
$$O\left(x \Exp (-c\sqrt{\log x \log \log x})\right)$$ 
due to J.E. Littlewood which is deep but not very different from the
results of  J. Hadamard and de la Vall\'ee Poussin namely 
$$O\left(x\Exp (-c \sqrt{\log x})\right).$$ 
For these results see E.C. Titchmarsh,  
\cite{Titchmarsh1}. Hereafter we will refer to this as Titchmarsh's book. By looking at the proof of the results of G.H. Hardy, J.E. Littlewood, H. Weyl and also the great improvements by I.M. Vinogradov it will be clear that we do not need machinery like the functional equation or the approximate functional equation in the proof of things like (\ref{eq11}) with some $\epsilon < \frac{1}{2}$ (see K. Ramachandra, \cite{Ramachandra1}. The remark there in before \S 6 is not used in an essential way) and in fact with $\epsilon = \frac{1}{8}$. Among other important reference works are H.L. Montgomery \cite{Montgomery1}, H.E. Richert \cite{Richert1}, M. Jutila \cite{Jutila1}, K. Chandrasekharan \cite{Chandrasekharan1}, \cite{Chandrasekharan2}, Y. Motohashi \cite{Motohashi1} and A. Ivic \cite{Ivic1}, \cite{Ivic2}. One may also refer to the booklet by K. Ramachandra \cite{Ramachandra2}.  
