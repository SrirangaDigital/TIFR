
\chapter{Some Recent Progress}\label{c6}

\section{Introduction}\label{c6:sec6.1}\pageoriginale

In this chapter we shall state without proofs some difficult results (and related results) mentioned in the introductory remarks. Most of the references not mentioned are to be found in E.C.\@ Titchmarsh \cite{Titchmarsh1}.

\section{Hardy's Theorem and Further Developments}\label{c6:sec6.2}

G.H.\@ Hardy was the first to attack the problem of zeros of $\zeta(s)$ on the critical line. Of course the number $N$ of zeros of $\zeta(s)$ in $(0\leq \sigma\leq 1,0\leq t\leq T)$ is given by the Riemann-von Mongoldt formula
$$
N=\frac{T}{2\pi}\log \frac{T}{2\pi}-\frac{T}{2\pi}+O(\log T).
$$

Denote by $N_{0}$ the number of zeros of $\zeta(s)$ in
$(\sigma=\frac{1}{2}, 0\leq t\leq T)$. (Both $N$ and $N_{0}$ are
counted with multiplicity). Hardy proved that $N_{0}\to \infty$ as
$T\to \infty$. He and J.E.\@ Littlewood proved later that $N_{0}\gg
T$. A.\@ Selberg developed their method further and by using certain
mollifiers proved that $N_{0}\gg N$. On the other hand C.L.\@ Siegel
developed another method to prove $N_{0}\gg T$. By a deep variant of
this method N.\@ Levinson \cite{Levinson2} proved that limit of
$N^{-1}N_{0}\geq \frac{1}{3}$ as $T\to\infty$. The references to the
works of the other authors mentioned above are to be found in
Levinson's paper. A.\@ Selberg and D.R.\@ Heath-Brown (independent of
eacg other) pursued the method of Levinson and proved that if
$N^{*}_{0}$ denotes the number of simple zeros of $\zeta(s)$ in
$(\sigma=\frac{1}{2},0\leq t\leq T)$ then limit of
$N^{-1}N^{*}_{0}\geq \frac{1}{3}$ as $T\to \infty$. Next some work in
these directions was done by R.\@ Balasubramanian, J.B.\@ Conrey,
D.R.\@ Heath-Brown, A.\@ Ghosh and S.M.\@ Gonek and subsequently by
J.B.\@ Conrey who  refined the method of Levinson and proved that, as
$T\to \infty$, limit $N^{-1}N^{*}_{0}\geq \frac{2}{5}$. (Ref.\@ J.B.\@
Conrey \cite{Conrey1}). In another direction J.B.\@ Conrey improved the
previous results of A.\@ Selberg and M.\@ Jutila and proved (with
usual notation) that
$$
N(\sigma,T)\ll_{\epsilon}T^{1-(\frac{8}{7}-\epsilon)(\sigma-\frac{1}{2})}\log
T. 
$$

In\pageoriginale yet another direction J.B.\@ Conrey improved on the
previous results of A.\@ Selberg; R.\@ Balasubramanian, J.B.\@ Conrey,
{\small D.R.\@ Heath-Brown}. He proved that
$$
\int^{1}_{\frac{1}{2}}N(\sigma,T)d\sigma\leq (0.0806+o(1))T.
$$

The last three sults of J.B.\@ Conrey mentioned above were announced
in (J.B.\@ Conrey \cite{Conrey2}). In a completely different direction
(namely pair corelation of the zeros of $\zeta(s)$) H.L.\@ Montgomery
proved (on RH) that limit of $N^{-1}N^{*}_{0}\geq \frac{2}{3}$ as
$T\to \infty$, (see H.L.\@ Montgomery \cite{Montgomery4}).

\section{Deeper Problems of Mean-Value Theorems on
  $\sigma=\frac{1}{2}$}\label{c6:sec6.3}

Define $E(T)$ by
$$
\frac{1}{2\pi}\int^{T}_{0}|\zeta \left(\frac{1}{2}+it \right)|^{2}dt=\frac{T}{2\pi}\log
\frac{T}{2\pi}+(2\gamma-1)\frac{T}{2\pi}+E(T)
$$
where $\gamma$ is as usual the Euler's constant. Then A.E.\@ Ingham
was the first to show that $E(T)=O(T^{\frac{1}{2}+\epsilon})$. This
result was improved in a complicated way to E.C.\@ Titchmarsh who
proved that $E(T)=O(T^{\frac{5}{12}+\epsilon})$. After a lapse of
nearly 45 years R.\@ Balasubramanian \cite{Balasubramanian2} took up the problem
and building upon the ideas of Titchmarsh and adding his own ideas
proved that $E(T)=O(T^{\frac{1}{3}+\epsilon})$. The latest improvement
is due to D.R.\@ Heath-Brown and M.N.\@ Huxley namely
$E(T)=O(T^{\frac{7}{22}+\epsilon})$. The reference to their paper is
(D.R.\@ Heath-Brown and M.N.\@ Huxley \cite{Heath-Brown and Huxley1}). J.L.\@ Hafner and
A.\@ Ivic have proved (ref.\@ J.L.\@ Hafner and A.\@ Ivic \cite{Hafner and Ivic1})
some nice $\Omega_{\pm}$ theorems for $E(T)$. Their results read
$$
E(T)=\Omega_{+}\left\{(T\log T)^{\frac{1}{4}}(\log\log
T)^{\frac{1}{4}(3+\log 4)}\Exp(-c\sqrt{\log\log\log T})\right\}
$$
and
$$
E(T)=\Omega_{-}\left\{T^{\frac{1}{4}}\Exp \left(\frac{D(\log\log
  T)^{\frac{1}{4}}}{(\log\log\log T)^{\frac{3}{4}}}\right)\right\},
$$
where $c>0$ and $D>0$ are constants. Let
$$
E_{2}(T)=\int^{T}_{0}|\zeta \left(\frac{1}{2}+it \right)|^{4}dt-TP_{4}(\log T),
$$
where\pageoriginale $P_{4}(\log T)$ is a certain polynomial in $\log
T$ of degree $4$, Then D.R.\@ Heath-Brown was the first to prove (for
a certain explicit $P_{4}(\log T)$) that
$E_{2}(T)=O(T^{\frac{7}{8}+\epsilon})$. (Ref.\@ \cite{Heath-Brown3}). His
method also gave the result of R.\@ Balasubramanian mentioned
earlier. The final result $E_{2}(T)= O(T^{\frac{2}{3}+\epsilon})$ which
we can expect in the present state of knowledge was proved by N.\@
Zavorotnyi (Ref.\@ \cite{Zavorotnyi1}) It should be mentioned that
$T^{\epsilon}$ has been replaced by a constant power of $\log T$ by
A.\@ Ivi\'c and Y.\@ Motohashi (Ref.\@ A.\@ Ivi\'c \cite{Ivic2}). The
result $E_{2}(T)=\Omega(T^{\frac{1}{2}})$ (recently Motohashi has
proved that $E_{2}(T)=\Omega_{\pm}(T^{\frac{1}{2}})$) of considerable
depth is due to A.\@ Ivic and Y.\@ Motohashi (Ref.\@ \cite{Ivic and  Motohashi1}). The
deep result 
$$
\int^{T}_{0}|\zeta \left(\frac{1}{2}+it \right)|^{12}dt\ll T^{2}(\log T)^{17}
$$
was first proved by D.R.\@ Heath-Brown (Ref.\@ \cite{Heath-Brown2}). Later on,
H.\@ Iwaniec developed another method and proved a result on the mean
value of $|\zeta \left(\frac{1}{2}+it \right)|^{4}$ over short intervals, which
gave as a corollary the result of Heath-Brown just mentioned and also
$$
 \int^{T+H}_{T}|\zeta \left(\frac{1}{2}+it \right)|^{4}dt\ll
H^{1+\epsilon},H=T^{\frac{2}{3}}. 
$$

Afterwards M.\@ Jutila and Y.\@ Motohashi (independently) gave
different methods of approach to this problem of H.\@ Iwaniec. Thus
there are at present three different methods of approach to this
problem. (Ref.\@ H.\@ Iwaniec \cite{Iwaniec1}; M.\@ Jutila \cite{Jutila1};
M.\@ Jutila \cite{Jutila3}; M.\@ Jutila, \cite{Jutila4}; Y.\@ Motohashi
(several papers of which the following is one \cite{Motohashi2}). Of these
methods Jutila's method works very well for hybrid versions to
$L$-functions and so on. However we do not say more on such questions
in this monograph. In 1989, N.V.\@ Kuznetsov published (N.V.\@
Kuznetsov \cite{Kuznetsov1}) a proof of
$$
\int^{T}_{0}|\zeta \left(\frac{1}{2}+it \right) |^{8}dt\ll T(\log T)^{16+B},
$$
where $B>4$ is a certain constant. However his proof appears to
contain many serious errors. (Professor Y.\@ Motohashi of Japan is
trying to correct the\pageoriginale mistakes and the result that
$$
\int^{T}_{0}|\zeta \left(\frac{1}{2}+it \right) |^{8}dt\ll T^{\frac{4}{3}+\epsilon} 
$$
valid for every fixed $\epsilon>0$ which Motohashi hopes to obtain
should be called Kuznetsov-Motohashi theorem if at all Motohashi
succeeds in proving it. If however Motohashi succeeds in proving
$$\int^{T}_{0}|\zeta \left(\frac{1}{2}+it \right)|^{6}dt\ll
T^{1+\epsilon}$$
the full credit of such a discovery should go to Motohashi). Before leaving
this section it is appropriate that the following two results should
be mentioned (and as is common with all the results of this chapter we
do not prove them). Of course they have a place in Chapter \ref{c4} and we
have mentioned it there and we do not prove them. The first is the
result
$$
(\log T)^{k^{2}}\ll
\frac{1}{T}\int^{T}_{0}|\zeta \left(\frac{1}{2}+it \right)  |^{2k}dt\ll (\log
T)^{k^{2}}
$$
uniformly for all $k=\frac{1}{n}(n\geq 1$ integer) due to M.\@
Jutila. (Ref.\@ M.\@ Jutila, \cite{Jutila2}). This has application to
large values of $|\zeta(\frac{1}{2}+it)|$. Another result is due to
A.\@ Ivic and A.\@ Perelli. They have proved that if $k$ is any real
number with $0\leq k\leq (\psi(T)\log\log T)^{-\frac{1}{2}}$ where
$\psi(T)\to \infty$ as $T\to \infty$, we have,
$$
\frac{1}{T}\int^{T}_{0}|\zeta \left(\frac{1}{2}+it \right)|^{2k}dt\to 1.
$$
(Ref.\@ A.\@ Ivic and A.\@ Perelli, \cite{Ivic and Perelli1}).

\section{Deeper Problems on Mean-Value Theorems in
  $\frac{1}{2}<\sigma<1$}\label{c6:sec6.4}

For fixed $\sigma \left(\frac{1}{2}<\sigma<1 \right)$ define  
{\fontsize{10}{12}\selectfont
$$
E(\sigma,T)=\int^{T}_{0}|\zeta(\sigma+it)|^{2}dt-\zeta(2\sigma)T-\frac{\zeta(2\sigma-1)\Gamma(2\sigma-1)}{1-\sigma}\Sin
(\pi\sigma)T^{2-2\sigma}. 
$$}

This definition is due to A.\@ Ivic (Note that $\lim E(\sigma,T)=E(T)$
as $\sigma\to \frac{1}{2}+0$). K.\@ Matsumoto defines in a slightly
different way. K.\@ Matsumoto was the first to use the method of R.\@
Balasubramanian and he proved
$$
E(\sigma,T)\ll
T^{1/(4\sigma+1)+\epsilon},\left(\frac{1}{2}<\sigma<\frac{3}{4}\right). 
$$
(Ref.\@ \cite{Matsumoto1}\pageoriginale K.\@ Matsumoto. However in this
paper he uses a slightly different method namely the one which uses Atkinson's
formula). A.\@ Ivic (\cite{Ivic2} p.\@ 90), has shown that
$E(\sigma,T)\ll T^{1-\sigma}$. K.\@ Matsumoto has proved (in the paper
cited above, see also the A.\@ Ivi\'c \cite{Ivic2}) that
$$
\int^{T}_{0}(E(\sigma,T))^{2}dt=C(\sigma)T^{\frac{5}{2}-2\sigma}+O(T^{\frac{7}{4}-\sigma}),\left(\frac{1}{2}<\sigma<\frac{3}{4}\right), 
$$ 
(where $C(\sigma)>0$), which implies
$E(\sigma,T)=\Omega(T^{\frac{3}{4}-\sigma})$. There are many other
interesting results given in A.\@ Ivi\'c \cite{Ivic2} mentioned above
and the interested reader is referred to this LN. However we have to
mention a result of S.W.\@ Graham which seems to have missed the
attention of many mathematicians. Let $q\geq 1$ be an integer,
$R_{q}=2^{q+2}-2$, and $\sigma_{q}=1-(q+2)R^{-1}_{q}$. Then his result
reads
$$
\int^{T}_{0}|\zeta(\sigma_{q}+it)|^{14R_{q}}dt\ll T^{14}(\log
T)^{A(q)},
$$
where $A(q)>0$ is a certain constant depending only on $q$. In
particular when $q=2$, we have,
$$
\int^{T}_{0}|\zeta\left(\frac{5}{7}+it\right)|^{196}dt\ll T^{14}(\log
T)^{425}. 
$$
(Note that this implies that $\mu(\frac{5}{7})\leq \frac{1}{14}$ and
also $\mu(\sigma_{q})\leq R^{-1}_{q}$. These results are close to
Theorems 5.12 and 5.13 of E.C.\@ Titchmarsh \cite{Titchmarsh1}). Reference to
these results is (S.W.\@ Graham, \cite{Graham1}).

\section{On the Line $\sigma=1$}\label{c6:sec6.5}

Let $k$ be any complex constant,
$(\zeta(s))^{k}=\sum\limits^{\infty}_{n=1}d_{k}(n)n^{-s}$ when $Re\ s>2$. Put
$$
E(k,1,T)=\int^{T}_{1}|(\zeta(1+it))^{2k}|dt-T\sum^{\infty}_{n=1}|d_{k}(n)|^{2}n^{-2}.
$$

The function $E(1,1,T)$ was studied in great detail in (R.\@
Balasubramanian, A.\@ Ivic and K.\@ Ramachandra \cite{Balasubramanian Ivic and  Ramachandra1}). One of
the results proved in this paper is 
$$
E(1,1,T)=-\pi \log T+O\left((\log T)^{\frac{2}{3}}(\log\log
T)^{\frac{1}{3}}\right). 
$$ 

It\pageoriginale follows that $E(1,1,T)=\Omega_{-}(\log T)$. It is
also $O(\log T)$. In another paper (R.\@ Balasubramanian, A.\@ Ivic and
K.\@ Ramachandra \cite{Balasubramanian Ivic and Ramachandra2}) they have proved many results. A  sample
result is
$$
E(k,1,T)=O\left((\log T)^{|k^{2}|}\right).
$$

Finally we mention a result on the large value of $|\log \zeta(1+it)|$
(K.\@ Ramachandra \cite{Ramachandra25}). The result is this. Let
$\epsilon(0<\epsilon<1)$ be any constant, $T\geq 10000$,
$X=\Exp\left(\frac{\log\log T}{\log\log\log T}\right)$. Consider the
set of points $t$ for which $T\leq t\leq T+e^{X}$ and $|\log
\zeta(1+it)|\geq \epsilon\log\log T$. Then this set is contained in
$O_{\epsilon}(1)$ intervals of length $\frac{1}{X}$.


