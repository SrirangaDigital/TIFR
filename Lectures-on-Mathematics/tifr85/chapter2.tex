\chapter[Some Fundamental Theorems on TiTchmarsh Series and...]{Some
  Fundamental Theorems on TiTchmarsh Series and
  Applications}\label{c2} 

\section{Introduction}\label{c2:sec2.1}

In this\pageoriginale chapter we prove three fundamental theorems on
{\small ``TITCHMARSH SERIES''}. These concern lower bounds for  
$$
\frac{1}{H} \int^H_0 |F(it)| dt \text{ and } \frac{1}{H} \int^H_0 |F(it)|^2 dt, 
$$
where $H \geq 10$ and $F(s) (s = \sigma + it)$ is defined by
$$
F(s) = \sum\limits^\infty_{n=1} a_n \lambda^{-s}_n  \; (\sigma \geq A + 2), 
$$
where $A>0$ is an integer constant and the complex numbers $a_n$ are
subject to $a_1 = 1$, $|a_n| \leq (nH)^A \; (n \geq 2)$ and the real
numbers $\lambda_n$ are subject to $\lambda_1 =1$ and $\frac{1}{C}
\leq \lambda_{n+1} - \lambda_n \leq C \; (n \geq 1)$ where $C \geq 1$
is a constant. $F(it)$ is defined by the condition that $F(s)$ shall
be continuable analytically in $(\sigma \geq 0, 0 \leq t \leq
H)$. These conditions define a {\small ``TITCHMARSH SERIES''}. Some times as in
this chapter we impose a growth condition on certain horizontal
lines. But in a later chapter we will manage without the growth
condition at the const of imposing a more stringent condition than
$|a_n| \leq (nH)^A$. All these results have important
applications. Theorem 1 will be proved as a preparation to the proof
of a more complicated (but neat) Theorem 3. Both these deal with lower
bounds for the mean square of $|F(it)|$ while Theorem 2 deals with the
mean of $|F(it)|$. As handy results for application we state a
corollary below each of the three main theorems. We begin by stating a
main lemma. 

\section{Main Lemma}\label{c2:sec2.2}
Let $r$ be a positive integer $H \geq (r+5) U$, $U\geq  2^{70}(16B)^2$ and $N$ and $M$ positive integers subject to $N > M \geq 1$. Let $b_m (m \leq M)$ and $c_n(n\geq N)$ be complex numbers and $A(s) = \sum\limits_{m \leq M} b_m \lambda^{-s}_m$. Let $B(s) = \sum\limits_{n \leq N} c_n \lambda^{-s}_n$ be absolutely convergent in $\sigma \geq A +2$ and continuable analytically in $\sigma \geq 0$.\pageoriginale Write $g(s) = A(-s) B(s)$,
$$
G(s) = U^{-r} \int^U_0 du_r \ldots \int^U_0 du_1 (g(s+i\lambda))
$$
(here and elsewhere $\lambda = u_1 + u_2 + \ldots + u_r$). Assume that there exist real numbers $T_1$ and $T_2$ with $0 \leq T_1 \leq U$, $H - U \leq T_2 \leq H$, such that
$$
|g(\sigma + iT_1)| + |g(\sigma + iT_2)| \leq \Exp \Exp \left(\frac{U}{16 B} \right)
$$
uniformly in $0 \leq \sigma \leq B$. (As stated already $B = A + 2$). Let 
$$
S_1 = \sum\limits_{m \leq M, n \geq N} |b_m c_n| \left(\frac{\lambda_m}{\lambda_n} \right)^B 2^r \left( U \log \frac{\lambda_n}{\lambda_m}\right)^{-r},
$$
and
$$
S_2 = \sum\limits_{m \leq M, n \geq N} |b_m c_n| \left( \frac{\lambda_m}{\lambda_n}\right)^B.
$$
Then
\begin{align*}
|\int^{H - (r+3) U}_{2U} G(it) dt| & \leq |U^{-r} \int^U_0 du_r \ldots \int^U_0 du_1 \int^{H-(r+3)U+\lambda}_{2U+\lambda} g(it) dt|\\
& \leq 2B^2 U^{-10} +  54 BU^{-1} \int^H_0 |g(it)| dt\\
& \qquad + (H + 64 B^2) S_1  + 16 B^2 \Exp \left(-\frac{U}{8B} \right) S_2. 
\end{align*}

To prove this main lemma we need five lemmas. After proving these we complete the proof of the main lemma.

\begin{sublemma}\label{c2:lem2.2.1}
Let $z = x+ iy$ be a complex variable with $|x| \leq \frac{1}{4}$. Then, we have,
\begin{itemize}
\item[{\rm (a)}] $|\Exp ((\Sin z)^2)| \leq e^{\frac{1}{2}} < 2$ for all $y$

and

\item[{\rm (b)}] If $|y|\geq 2$,
$$
|\Exp ((\Sin z)^2)| \leq e^{\frac{1}{2}} (\Exp \Exp |y|)^{-1} < 2 (\Exp \Exp |y|)^{-1} .
$$
\end{itemize}
\end{sublemma}

\begin{proof}
We have\pageoriginale
\begin{align*}
\re (\Sin z)^2 & = -\frac{1}{4} \re \{(e^{i(x+iy)}  - e^{-i(x+iy)})^2\}\\
& = -\frac{1}{4} \re \{e^{2 i x - 2y} + e^{-2ix+ 2y} -2\}\\
& = \frac{1}{2} - \frac{1}{4} \left\{(e^{-2y} + e^{2y}) \cos (2x) \right\}.
\end{align*} 
But in $|x| \leq \frac{1}{4}$, we have $\cos (2x) = \cos (|2x|) \geq \cos \frac{1}{2} \geq \cos \frac{\pi}{6} \geq \frac{\sqrt{3}}{2}$. The rest of the proof is trivial since (i) $\cosh y$ is an increasing function of $|y|$ and (ii) for $|y| \geq 2$
$$
\Exp \left(-\frac{\sqrt{3}}{8} e^{2|y|} \right) \leq (\Exp \Exp |y|)^{-1}
$$
since $e^2 > (2.7)^2$ and $\frac{8}{\sqrt{3}} < \frac{8 \times 1.8}{3} = 4.8$ and so $e^2 > \frac{8}{\sqrt{3}}$. The lemma is completely proved.
\end{proof}

\begin{sublemma}\label{c2:lem2.2.2}
For any two real numbers $k$ and $\sigma$ with $0 < |\sigma| \leq 2B$, we have,
$$
\int^\infty_{-\infty} |\Exp \left(\Sin^2 \left( \frac{ik - \sigma - iu_1}{8B}\right) \right) \frac{du_1}{ik-\sigma - iu_1}| \leq 12 + 4 \log |\frac{2B}{\sigma}|. 
$$
\end{sublemma}

\begin{proof}
Split the integral into three parts $J_1, J_2$ and $J_3$ corresponding to $|u_1 - k| \geq 2B$, $|\sigma| \leq | u_1 - k |\leq 2B$ and $|u_1 - k| \leq \sigma$. The contribution to $J_1$ from $|u_1 -k| \geq 16 B$ is (by (b) of Lemma \ref{c2:lem2.2.1})
\begin{align*}
& \leq \frac{2e^{\frac{1}{2}}}{16B}\int^{\infty}_{16B} \Exp \left(-\frac{u_1}{8B} \right) du_1\\
& = e^{\frac{1}{2}} \int^\infty_{2} \Exp (-u_1) du_1 = \Exp \left(-\frac{3}{2} \right).
\end{align*}
The contribution to $J_1$ from $2B \leq |u_1 - k |\leq 16 B$ is (by (a) of Lemma \ref{c2:lem2.2.1})
$$
\leq e^{\frac{1}{2}} \int_{2B \leq |u_1 - k| \leq 16 B} |u_1 - k|^{-1} du_1 = 2 e^{\frac{1}{2}} \log 8 = 6e^{\frac{1}{2}} \log 2. 
$$
Now
\begin{gather*}
6e^{\frac{1}{2}} \log 2 + \Exp \left( -\frac{3}{2}\right) < 6 \left(
1+ \frac{1}{2} + \frac{1}{2\cdot 2^2} + \frac{1}{6  \cdot
  2^2}\right)\\
\qquad \left(\frac{1}{2} + \frac{1}{2\cdot 2^2} + \frac{1}{3\cdot 2^2} \right)
+ \left(\frac{1}{2 \cdot 7} \right)^{3/2} < 8.
\end{gather*}

\pageoriginale
Thus $|J_1| \leq 8$. Using (a) of Lemma \ref{c2:lem2.2.1} we have $|J_2| \leq 4 \log |\frac{2B}{\sigma}|$. In $J_3$ the integrand is at most $e^{\frac{1}{2}} \sigma^{-1}$ in absolute value and so $|J_3| \leq 2e^{\frac{1}{2}} \leq 4$. Hence the lemma is completely proved.
\end{proof}

\begin{sublemma}\label{c2:lem2.2.3}
If $n > m$, we have, for all real $k$, 
$$
|\int^U_0 du_r \ldots \int^U_0 du_1 \left(\frac{\lambda_m}{\lambda_n} \right)^{i(k+\lambda)} | \leq 2^r \left(\log \frac{\lambda_n}{\lambda_m} \right)^{-r}.
$$
\end{sublemma}

\begin{proof}
Trivial.
\end{proof}

\begin{sublemma}\label{c2:lem2.2.4}
For all real $t$ and all $D \geq B$, we have,
$$
|G(D+ it)| \leq S_1 \text{ and } |g(D+ it)| \leq S_2.
$$
\end{sublemma}

\begin{proof}
We have, trivially,
$$ 
|g(D+ it)| \leq \sum\limits_{m \leq M, n \geq N} |b_m c_n| \left( \frac{\lambda_m}{\lambda_n}\right)^D
$$
and the second result follows on observing that $\frac{\lambda_m}{\lambda_n} < 1$ and so $\left( \frac{\lambda_m}{\lambda_n}\right)^D \leq \left(\frac{\lambda_m}{\lambda_n} \right)^B$. Next
\begin{gather*}
G(D+it) = U^{-r} \int^U_0 du_r \ldots \int^U_0 du_1 (g(D+ it + i \lambda))\\
= U^{-r} \sum\limits_{m \leq M, n \geq N} b_m c_n \left( \frac{\lambda_m}{\lambda_n}\right)^D \int^U_0 du_r \ldots \int^U_0 du_1 \left(\frac{\lambda_m}{\lambda_n} \right)^{i(t+\lambda)}.
\end{gather*}
Using Lemma \ref{c2:lem2.2.3} and observing $\left(\frac{\lambda_m}{\lambda_n} \right)^D \leq \left( \frac{\lambda_m}{\lambda_n} \right)^B$ the first result follows. 
\end{proof}

\begin{sublemma}\label{c2:lem2.2.5}
Let $0 < \sigma \leq B $ and $2U \leq t \leq H - (r +3) U$. Then, for $H \geq (r+5)U$ and $U \geq (20)!(16B)^2$, we have,
\begin{align*}
|G(\sigma + it)| & \leq B U^{-10} + U^{-1} \left(2+ 4 \log \frac{2B}{\sigma} \right) \int\limits^H_0 |g(it)| dt\\
& + 16 S_1 \log (2B) + 8 B S_2 \Exp \left( -\frac{U}{8B}\right). 
\end{align*}\pageoriginale
\end{sublemma}

\begin{remark*}
$(20)! < 2^{70}$.
\end{remark*}

\begin{proof}
We note, by Cauchy's theorem, that
\begin{align*}
& 2 \pi i g (\sigma + it + i \lambda) = \int^{B+1+iT_1}_{iT_1} + \int^{B+1+iT_2}_{B+1+iT_1} - \int^{B+1+iT_2}_{iT_2} - \int^{iT_2}_{iT_1}\\
& \left\{g(w) \Exp \left(\Sin^2 \left(\frac{w-\sigma - it -i\lambda}{8B} \right) \right) \right\} \frac{dw}{w-\sigma - it -i\lambda}\\
& J_1 + J_2 - J_3 - J_4 say.
\end{align*}
We write
\begin{align*}
2 \pi i G (\sigma + it) & = 2\pi i U^{-r} \int^U_0 du_r \ldots \int^U_0 du_1 (g(\sigma + it + i\lambda))\\
& = U^{-r} \int^U_0 du_r \ldots \int^U_0 du_1 (J_1 + J_2 - J_3 - J_4)\\
& = J_5 + J_6 - J_7 - J_8 \text{ say}.
\end{align*}
Let us look at $J_5$. In $J_1$ (also in $J_3$) $|g(w)|\leq \Exp \Exp (\frac{U}{16B})$ (by the definition of $T_1$ and $T_2$). Also by using Lemma \ref{c2:lem2.2.1} (b) (since $|\re w-\sigma| \leq B + 1 \leq 2 B$, and $|\Iim (w - it - i \lambda)| \geq U \geq (20)!(16B)^2$), we have,
$$
|\Exp \left( \left( \frac{w-\sigma - it -i\lambda}{8B}\right)\right)| \leq 2 \Exp \left(- \frac{U}{8B}\right).
$$
Hence
\begin{align*}
|J_1| & \leq \frac{2(B+1)}{U} \Exp \left(\Exp \frac{U}{16B} - \Exp \frac{U}{8B} \right)\\
& \leq \frac{2(B+1)}{U} \Exp \left(- \left(\Exp \frac{U}{16B} \right) \; \left( \Exp \frac{U}{16B} -1\right) \right)\\
& \leq \frac{B}{2} U^{-10},
\end{align*}
since $U \geq (20)!(16 B)^2$ and so $\Exp \frac{U}{16B} -1 \geq 1$ and $\Exp \left(-\Exp \frac{U}{16B} \right) \leq \Exp \left( -\Exp U^{\frac{1}{2}}\right) \leq \Exp (-U^{\frac{1}{2}}) \leq (20)! U^{-10}$. Thus $|J_5| \leq \frac{1}{2} BU^{-10}$. 

Similarly, $|J_7| \leq \frac{1}{2} B U^{-10}$. Next
\begin{gather*}
J_8 = U^{-r} \int^{iT_2}_{iT_1} g(w) dw \int^U_0 du_r \ldots \int^U_0 du_2 \int^U_0\\
\Exp \left(\sin^2 \left(\frac{w - \sigma - it - i \lambda}{8B} \right) \right) \frac{du_1}{w - \sigma - it -i \lambda}.
\end{gather*}\pageoriginale
We note that $w - \sigma - it - i \lambda = i k - \sigma - iu_1$ where $k = \Iim w - t -u_2 \ldots - u_r$. Hence the $u_1$-integral is in absolute value (by Lemma \ref{c2:lem2.2.2})
$$
\leq 12 + 4 \log \frac{2B}{\sigma}.
$$
This shows that
\begin{align*}
|J_8| & \leq U^{-r} \int^{iT_2}_{iT_1} |g(w) dw| \left\{ U^{r-1} (12 + 4 \log \frac{2B}{\sigma})\right\}\\
& \leq U^{-1} (12 + 4 \log \frac{2B}{\sigma}) \int^H_0 |g(it)| dt
\end{align*}
Finally we consider $J_6$.
\begin{align*}
J_6 & = U^{-r} \int^U_0 du_r \ldots \int^U_0 du_1
\int^{B+1+iT_2}_{B+1+iT_1} g(w)\\ 
&\qquad\Exp \left(\Sin^2 \left(\frac{w-\sigma - it - i \lambda}{8B} \right) \right) \frac{dw}{w -\sigma - it -i\lambda}\\
& = U^{-r} \int^U_0du_r \ldots \int^U_0 du_1 \int^{B+1-\sigma + iT_2 - it -i\lambda}_{B+1-\sigma + iT_1 - it - i \lambda} g(w+ \sigma + it + i\lambda)
\end{align*}
Using Lemma \ref{c2:lem2.2.1} (b) we extend the range of integration of $w$ to $(B+1-\sigma - i \infty, B + 1 - \sigma + i\infty)$ and this gives an error which is at most
\begin{gather*}
U^{-r} \int^U_0 du_r \ldots \int^U_0 du_1 \int_{|\Iim w| \geq U, \re w
  = B + 1 -\sigma}\\ 
\qquad\quad\left|g(w+ \sigma + it + i \lambda) \Exp \left(\Sin^2 \left(\frac{w}{8B} \right)\right) \frac{dw}{w}\right|.
\end{gather*}
By Lemma \ref{c2:lem2.2.4} this is
$$
\leq S_2 U^{-r} \int^U_0 du_r \ldots \int^U_0 du_1 \int_{|\Iim w | \geq U, \re w =B+1 -\sigma} |\Exp \left(\Sin^2 \frac{w}{8B} \frac{dw}{w}\right)|.
$$
Here the innermost integral is (by Lemma \ref{c2:lem2.2.1} (b)) 
$$
\leq \frac{4}{U} \int^\infty_U\Exp \left( - \frac{u}{8B} \right) du \leq \int^\infty_U \Exp \left( -\frac{u}{8B}\right) du = 8B \Exp \left(-\frac{U}{8B} \right).
$$
Thus the error does not exceed $8BS_2$ $\Exp \left( -\frac{U}{8B}\right)$ and so
\begin{align*}
& |J_6| \leq U^{-r} \int^{U}_0 du_r \ldots \int^U_0 du_1 \int^{B+1-\sigma + i \infty}_{B+1-\sigma - i \infty} g(w+ \sigma + it + i \lambda)\\
&  \Exp \left(\Sin^2 \left(\frac{w}{8B}\right)\right) \frac{dw}{w}  + 8 B S_2 \Exp \left(-\frac{U}{8B}\right)\\
& = |U^{-r} \int^{B+1-\sigma + i \infty}_{B+1-\sigma - i \infty} \Exp \left(\Sin^2 \left(\frac{w}{8B} \right) \right) \frac{dw}{w} \int^{U}_0 du_r \ldots \\ 
\int^U_0 & du_1 g(w+\sigma + it + i\lambda)| + 8 B S_2 \Exp \left(-\frac{U}{8B} \right)\\
& = \left|\int^{B+1-\sigma + i \infty}_{B+1-\sigma - i\infty} G(w+\sigma +
it) \Exp \left(\Sin^2 \left(\frac{w}{8B} \right) \right)
\frac{dw}{w}\right|\\ 
&\qquad\qquad + 8BS_2 \Exp \left(-\frac{U}{8B} \right). 
\end{align*}\pageoriginale
Using the first part of Lemma \ref{c2:lem2.2.4} we obtain
\begin{align*}
|J_6| & \leq S_1 \int^{B+1-\sigma + i\infty}_{B+1-\sigma - i \infty} | \Exp \left(\Sin^2 \left(\frac{w}{8B} \right) \right) \frac{dw}{w}| + 8 BS_2 \Exp \left(-\frac{U}{8B} \right)\\
& \leq S_1 \left(12 + 4 \log \frac{2B}{B+1-\sigma} \right) + 8 B S_2 \Exp \left(-\frac{U}{8B} \right)
\end{align*}
by using Lemma \ref{c2:lem2.2.2}. Thus
$$
|J_6| \leq 16 S_1 \log (2B) + 8 BS_2 \Exp \Exp \left( -\frac{U}{8B}\right).
$$
This completes the proof of the lemma.
\end{proof}

We are now in a position to complete the proof of the main lemma. We first remark that
\begin{align*}
& 4 \int^B_0 \log \frac{2B}{\sigma} d \sigma = 4 B \log 2 + 4 \sqrt{2} \int^B_0 \left(\frac{B}{\sigma} \right)^{\frac{1}{2}} d \sigma\\
& < 4 \left(\frac{1}{4} + \frac{1}{2.2^2} + \frac{1}{3.2^2} \right) B + (8 \times 1.415) B < 15B.
\end{align*}
By Cauchy's theorem, we have,
\begin{align*}
\int^{H-(r+3)U}_{2U} G(it) idt & = \int^{i(H-(r+3)U)}_{i(2U)} G(s) ds\\
& = \int^{B+i(2U)}_{i(2U)} G(s) ds + \int^{B+i(H-(r+3)U)}_{B+i(2U)} G(s) ds -\\
& \qquad - \int^{B+i(H-(r+3)U)}_{i(H - (r+3)U)} G(s) ds\\
&  = J_1 + J_2 - J_3 \text{ say}. 
\end{align*}
Using\pageoriginale the estimate given in Lemma \ref{c2:lem2.2.5}, we see that
\begin{align*}
|J_1| & \leq \int^B_0 \left(BU^{-10} + \frac{(12+4 \log \frac{2B}{\sigma})}{U} \int^H_0 |g(it)| dt \right.\\
&  \qquad \left. + 16 (\log (2B)) S_1  + 8 BS_2 \Exp \left(-\frac{U}{8B}\right) \right) d \sigma\\
& \leq B^2 U^{-10} + \frac{12B+15B}{U} \int^H_0 |(g(it))| dt + 16 B S_1\log (2B)\\
& \qquad 8B^2 S_2 \Exp \left(-\frac{U}{8B} \right).
\end{align*}
The same estimate holds for $|J_3|$ also. For $|J_2|$ we use the estimate given in Lemma \ref{c2:lem2.2.4} to get
$$
|J_2| \leq HS_1.
$$
This completes the proof of the main lemma.

\section{First Main Theorem}\label{c2:sec2.3}
Let $A,B,C$ be as before $0 < \epsilon \leq \frac{1}{2}$, $r \geq [(200 A + 200) \epsilon^{-1} ]$, $|a_n| \leq n^A H^{\frac{r\epsilon}{8}}$. Then $F(s) = \sum^{\infty}_{n=1} a_n \lambda^{-s}_n$ is  analytic in $\sigma \geq A +2$. Let $K \geq 30$, $U = H^{1-\frac{\epsilon}{2}} +50 B \log \log K_1$. Assume that
$$
H \geq (120 B^2 C^{2A+4} (4rC^2)^r)^{\frac{400}{\epsilon}} + (100 rB)^{20} \log \log K_1, 
$$
and that there exist $T_1, T_2$ with $0 \leq T_1 \leq U$, $H - U \leq T_2 \leq H$ such that
$$
|F(\sigma + iT_1)| + |F (\sigma + iT_2| \leq K
$$
uniformly in $0\leq \sigma \leq B$ where $B(s)$ is assumed to be analytically continuable in $\sigma \geq 0$. Then
$$
\int^H_0 |F(it)|^2 dt \geq (H - 10 r C^2 H^{1-\frac{\epsilon}{4}} - 100 r B \log \log K_1) \sum\limits_{n \leq H^{1-\epsilon}} |a_n|^2,
$$
where 
$$
K_1 = \left( \sum\limits_{n \leq H^{1-\epsilon}} |a_n| \lambda^B_n \right)  K + \left(\sum\limits_{n\leq H^{1-\epsilon}} |a_n| \lambda^B_n \right)^2.
$$

\begin{coro*}
Let\pageoriginale $A$ and $C$ be as in the introduction \S\ \ref{c2:sec2.1}., $0 < \epsilon \leq \frac{1}{2}$, $r \geq [(200 A + 200) \epsilon^{-1}]$ , $|a_n | \leq n^A H^{\frac{r\epsilon}{8}}$. Then $F(s) = \sum\limits^\infty_{n=1} a_n \lambda^{-s}_n$ is analytic in $\sigma \geq A+2$. Let  $K \geq 30$, $U_1 = H^{1-\frac{\epsilon}{2}}$. Assume that $K_1 = (HC)^{12A} K$, 
$$
H \geq (120 (A+2)^2 C^{2A+4} (4rc^2)^r)^{\frac{100}{\epsilon}} + (100r(A+2))^{20} \log \log K_1,
$$
and the there exist $T_1,T_2$ with $0 \leq T_1 \leq U_1$, $H - U_1 \leq T_2 \leq H$, such that uniformly in $\sigma \geq 0$ we have
$$
|F(\sigma + iT_1)| + |F(\sigma + iT_2)| \leq K,
$$
where $F(s)$ is assumed to be analytically continuable in $(\sigma \geq 0, 0 \leq t \leq H)$. Then
$$
\frac{1}{H} \int\limits^H_0 |F(it)|^2 dt \geq (1-10 r C^2 H^{-\frac{\epsilon}{4}} - 100 r BH^{-1} \log \log K_1) \sum\limits_{n \leq H^{1-\epsilon}} |a_n|^2.
$$
\end{coro*}

\setcounter{remark}{0}
\begin{remark}\label{c2:rem1}
We need the conditions $H \geq (r+5) U$, $U\geq 2^{70}(16B)^2$ in the application of the main lemma. All such conditions are satisfied by our lower bound choice for $H$. We have not attempted to obtain economical lower bounds.
\end{remark}

\begin{remark}\label{c2:rem2}
Taking $F(s) = (\zeta(\frac{1}{2} + it + iT))^k$ in the first main theorem we obtain the following as an immediate corollary. Let $C(\epsilon, k) \log \log T \leq H \leq T$. Then for all integers $k\geq 1$.
$$
\frac{1}{H} \int\limits^{T+H}_T |\zeta \left(\frac{1}{2} + it \right)|^{2k} dt \geq (1-\epsilon)  \sum\limits_{n\leq H^{1-\epsilon}} (d_k(n))^2 n^{-1} \geq (C'_k -2\epsilon) (\log H)^{k^2},
$$
where
$$
C'_k = (\Gamma (k^2 +1))^{-1} \prod\limits_p \left\{ (1-p^{-1})^{k^2}  \sum\limits^\infty_{m=0} \left(\frac{\Gamma(k+m)}{\Gamma(k)m!} \right)^2 p^{-m}\right\}.
$$
(This is because it is well-known that
$$
\sum\limits_{n \leq X} (d_n (n))^2 n^{-1} = \left\{ C'_k + O \left( \frac{1}{\log X}\right)\right\} (\log X)^{k^2}).
$$
Our\pageoriginale third main theorem gives a sharpening of this. The third main theorem is sharper than the conjecture (stated by K. Ramachandra \cite{Ramachandra7} in Durham conference 1979). The conjecture (as also the weaker form of the conjecture proved by him in the conference) would only give
$$
 \frac{1}{H} \int^{T+H}_{T} |\zeta \left(\frac{1}{2} + it \right)|^{2k} dt \gg_k (\log H)^{k^2} \text{ in  } C(k) \log \log T \leq H \leq T.
$$
But the third main Theorem gives
{\fontsize{10}{12}\selectfont
$$ 
\frac{1}{H} \int^{T+H}_T |\zeta \left(\frac{1}{2} + it \right)|^{2k} dt \geq C'_k (\log H)^{k^2} + O \left(\frac{\log \log T}{H} (\log H)^{k^2} \right) + O (\log H)^{k^2}
$$}
where the $O$-constants depend only on $k$. 
\end{remark}

\begin{remark}\label{c2:rem3}
The first main theorem gives a lower bound for $\frac{1}{H}
\int^{T+H}_T\break |\zeta (\frac{1}{2}+ it)|^{2k} dt $ uniformly in $1
\leq k \leq \log H$, $T \geq H \geq 30$  and $C \log \log T \leq H
\leq T$. From this it follows (as was shown in R. Balasubramanian
\cite{Balasubramanian1}) that for $C \log \log T \leq H \leq T$ we
have uniformly 
$$
\max\limits_{T \leq t \leq T + H} |\zeta \left(\frac{1}{2} +it \right)| > \Exp \left(\frac{3}{4} \sqrt{\frac{\log H}{\log \log H}} \right)
$$
if $C$ is choosen to be a large positive constant. On Riemann hypothesis we can deduce from the first main theorem the following more general result. Let $\theta$ be fixed and $0 \leq \theta < 2\pi$. Put $z = e^{i\theta}$. Then (on Riemann hypothesis), we have,
$$ 
\max\limits_{T \leq t \leq T + H} | \left(\zeta \left(\frac{1}{2} + it \right) \right)^z| > \Exp \left(\frac{3}{4} \sqrt{\frac{\log H}{\log \log H}} \right)
$$
where is LHS is interpreted as $\lim\limits_{\sigma \to \frac{1}{2} + 0}$ of the same expression with $\frac{1}{2} + it$ replaced by $\sigma + it$. This result with $\theta = \frac{\pi}{2}$ and $\frac{3\pi}{2}$ gives a quantitative improvement of some results of J.H. Mueller \cite{Mueller1}.
\end{remark}

\begin{proof}
Write $M = [H^{1-\epsilon}]$, $N = M + 1$, $A(s) = \sum\limits_{m\leq M} \bar{a}_m \lambda^{-s}_m$, $\bar{A} (s) = \sum\limits_{m \leq M} a_m \lambda^{-s}_m$, $B(s) = \sum\limits_{n \geq N} a_n \lambda^{-s}_n$.\pageoriginale Then we have, in $\sigma \geq A +2$,
$$
F(s) = \bar{A} (s) + B(s).
$$
Also,
\begin{align*}
|F(it)|^2 & = |\bar{A}(it)|^2 + 2 \re (A (-it) B (it)) + |B(it)|^2\\
& \geq |\bar{A}(it)|^2 + 2 \re (g (it))
\end{align*}
where $g(s) = A(-s) B(s)$. Hence
\begin{align*}
\int^H_0 |F(it)|^2 dt & \geq U^{-r} \int^U_0 du_r \ldots \int^U_0 du_1 \int^{H-(r+3)U+\lambda}_{2U+\lambda} |F(it)|^2 dt\\
& \geq U^{-r} \int^U_0 du_r \ldots \int^U_0 du_1 \int^{H-(r+3)
  U+\lambda}_{2U + \lambda} (|\bar{A} (it)|^2\\ 
&\qquad {} + 2 \re g(it) ) dt 
 = J_1 + 2 J_2 \text{ say}.
\end{align*}
Now $\log \left(\frac{\lambda_{n+1}}{\lambda_n} \right) = - \log \left( 1-\left(1 - \frac{\lambda_n}{\lambda_{n+1}} \right)\right) \geq \frac{\lambda_{n+1} - \lambda_n}{\lambda_{n+1}} \geq (2nC^2)^{-1}$. Hence by Montgomery-Vaughan theorem,
\begin{align*}
J_1 & \geq \int^{H-(r+3)U}_{2U} |\bar{A} (it)|^2 dt\\
& \geq \sum\limits_{n\leq M} (H -(r+5)U-100C^2 n) |a_n|^2.
\end{align*}
We have
\begin{align*}
& |g(s)| = |A(-s) B(s)| = |A(-s) (F(s) -A(s))|\\
& \leq \left(\sum\limits_{n\leq H^{1-\epsilon}} |a_n|\lambda^B_n \right) K + \left(\sum\limits_{n \leq H^{1-\epsilon}} |a_n| \lambda^B_n\right)^2\\
& = K_1.
\end{align*} 
By the main lemma, we have,
\begin{align*}
|J_2| & \leq |U^{-r} \int^U_0 du_r \ldots \int^U_0 du_1 \int^{H-(r+3) U+\lambda}_{2U+\lambda} g(it) dt|\\
& \leq \frac{2B^2}{U^{10}} + \frac{54B}{U} \int^H_0 |g(it)| dt + (H + 64 B^2)S_1 + 16 B^2 S_2 \Exp (-\frac{U}{8B}) \tag{2.3.1}\label{c2:eq2.3.1}
\end{align*}
We simplify\pageoriginale the last expression in
(\ref{c2:eq2.3.1}). We can assume that 
$$\int^H_0 |F(it)|^2 dt \leq H \sum\limits_{n\leq H^{1-\epsilon}}
|a_n|^2 $$ 
(otherwise the result is trivially true). Hence
\begin{align*}
& \int^H_0 |g(it)| dt = \int^H_0 |A(-it)B(it)|dt\\
& \leq \int^H_0 |A(-it)|^2 dt + \int^H_0 |B(it)|^2 dt\\
& \leq \int^H_0 |A(-it)|^2 dt + \int^H_0 |F(it) - \bar{A}(it)|^2 dt\\
& \leq 3 \int^H_0 |A(-it)|^2 dt + 2 \int^H_0|F(it)|^2 dt\\
& \leq 3 \sum\limits_{n\leq M} (H+100C^2 n) |a_n|^2 + 2 H \sum\limits_{n\leq M} |a_n|^2\\
& \leq (300 C^2 + 5) H \sum\limits_{n \leq M} |a_n|^2.
\end{align*}
\begin{align*}
S_2 & \leq \sum\limits_{m\leq M, n\geq N} |b_m c_n| \left( \frac{\lambda_m}{\lambda_n}\right)^{A+2}\\
& \leq \sum\limits_{m\leq M, n \geq N} |a_m a_n| \left(\frac{\lambda_m}{\lambda_n} \right)^{A+2} \\
& \leq \sum\limits_{m\leq M, n\geq N} m^A H^{\frac{r\epsilon}{8}} n^A H^{\frac{r\epsilon}{8}} (C^2 mn^{-1})^{A+2}\\
& \leq H^{\frac{r\epsilon}{4}} C^{2A+4} \sum\limits_{m\leq M} m^{2A+2} \sum\limits_{n\geq N} n^{-2}\\
& \leq H^{\frac{r\epsilon}{4} + 2 A + 3}  C^{2A+4} \text{ since } \frac{\pi^2}{6} - 1 < 1.
\end{align*}
Now
$$
S_1 \leq \left(U \log \frac{\lambda_N}{\lambda_M}\right)^{-r} 2^r S_2
$$
and
\begin{gather*}
\log \frac{\lambda_N}{\lambda_M} \geq \frac{1}{2} \frac{\lambda_N -\lambda_M}{\lambda_M} \geq (2C^2 M)^{-1},\\
U \log \left( \frac{\lambda_N}{\lambda_M}\right) \geq (2C^2)^{-1} H^{\frac{\epsilon}{2}}.
\end{gather*}
Thus\pageoriginale
\begin{align*}
|J_2| & \leq \frac{2B^2}{U^{10}} + 54 B (300 C^2 + 5) HU^{-1} \sum\limits_{n \leq M} |a_n|^2\\
  & \quad (H + 64 B^2) H^{-\frac{r\epsilon}{4} + 2A + 3} 2^r (2C^2)^r C^{2A+4}\\
& \quad  + 16 B^2 \Exp \left(-\frac{U}{8B} \right) H^{\frac{r\epsilon}{4} + 2 A + 3} C^{2A+4}. (\text{Note } a_1 =\lambda_1 =1).
\end{align*}
So
\begin{align*}
& (r+5) U +100 C^2 H^{1-\epsilon}  + 2 |J_2| \left(\sum\limits_{n\leq M} |a_n|^2 \right)^{-1}\\
& \leq (r+5) H^{1-\frac{\epsilon}{2}} + 100 C^2 H^{1-\epsilon} + 100 Br \log \log K_1\\
&\qquad {} + \frac{4B^2}{H^5} + 108 B (300 C^2 +5) H^{\frac{\epsilon}{2}} + 128 (2^r) (2C^2)^r B^2 H^{2A+4 - 50A } C^{2A+4}\\
&\qquad {} + 32 B^2 C^{2A+4} r ! (8B)^r H^{2A +3 + \frac{r\epsilon}{2} -\frac{r}{2}}\\
& \leq 100 Br \log \log K_1 + r C^2 H^{1-\frac{\epsilon}{4}} \left\{\frac{r+5}{rC^2 H^{\frac{\epsilon}{4}}} + \frac{100 C^2}{H^{\frac{3\epsilon}{4}}}+ \frac{4B^2}{H^5} \right.\\
&\left.\qquad {}  + \frac{108B (300C^2
    +5)}{H^{1-\frac{3\epsilon}{4}}} + 128 (2^r) (2C^2)^r B^2 H^{-1}
  C^{2A+4}\right.\\ 
&\qquad{}\left.+ 32 B^2 C^{2A+4} r! (8B)^r H^{-1}  \right\}\\
& \leq 100 Br \log \log K_1 + 10 C^2 r H^{1-\frac{\epsilon}{4}}. 
\end{align*}
This completes the proof of the theorem. 
\end{proof}

\section{Second Main Theorem}\label{c2:sec2.4}

We assume the same conditions as in the first main theorem except that we change the definition of $U$ to $U = H^{\frac{7}{8}} + 50 B \log \log K_2$. Then there holds
$$
\int^H_0 |F(it)| dt \ge H - 10 r H^{\frac{7}{8}} - 100 r B \log \log K_2, 
$$ 
where $K_2 = K+1$.

\begin{coro*}
Let $A$ and $C$ be as in the introduction \S\ \ref{c2:sec2.1}, $|a_n| \leq (nH)^A$. Then $F(s) = \sum\limits^\infty_{n=1} a_n \lambda^{-s}_n$ is analytic in $\sigma \geq A +2$. Let $K \geq 30$,
$$
H \geq (3456000A^2 C^3)^{640000A} + (240000A)^{20} \log \log (K+1), 
$$
and that\pageoriginale there exist $T_1, T_2$ with $0\leq T_1 \leq H^{\frac{7}{8}}$, $H-H^{\frac{7}{8}} \leq T_2 \leq H$, such that uniformly in $\sigma \geq 0$ we have
$$
|F(\sigma+iT_1)| + |F(\sigma + iT_2)| \leq K
$$
where $F(s)$ is assumed to be analytically continuable in $(\sigma \geq 0, 0 \leq t \leq H)$. Then
$$
\frac{1}{H}\int^H_0 |F(it)| dt \geq 1 - 8000 A H^{-\frac{1}{8}} - 240000 A^2 H^{-1} \log\log (K+1).
$$

The corollary is obtained by putting $\epsilon = \frac{1}{2}$, $r = 800 A$  in the second main theorem. 
\end{coro*}

\begin{remark*}
Conditions like $H \geq (r+5) U$, $U \geq 2^{70} (16B)^2$ are taken care of by the inequality for $H$.
\end{remark*}

\begin{proof}
We have,
\begin{align*}
& \int^H_0 |F(it)| dt \geq U^{-r} \int^U_0 du_r \ldots \int^U_0 du_1 \int^{H-(r+3)U+\lambda}_{2U+\lambda} |F(it)| dt\\
& \geq  U^{-r} \re \left(\int^U_0 du_r \ldots\int^U_0 du_1 \int^{H-(r+3)U+\lambda}_{2U+\lambda} F(it) dt\right)\\
& = U^{-r} \re \left\{\int^U_0 du_r \ldots \int^U_0 du_1 \int^{H-(r+3)U+ \lambda}_{2U+\lambda} (1+ A (-it) B (it)) dt \right\}
\end{align*}
(where $A(s) \equiv 1$ (i.e. $a_1 = 1 = M$) and $B(s) = F(s) -1$) $=J_1 + \re J_2 $ say. Clearly $J_1 \geq H - (r+5) U$. For $J_2$ we use the main lemma.
\begin{equation*}
|J_2| \leq \frac{2B^2}{U^{10}} + \frac{54B}{U} \int^H_0 |g(it)| dt + (H+64 B^2) S_1 + 16B^2 \Exp \left( -\frac{U}{8B}\right) S_2. \tag{2.4.1}\label{c2:eq2.4.1}
\end{equation*}
As in the proof of the first main theorem we can assume $\int^H_0 |F(it)| dt \leq H$ and so $\int^H_0 |g(it)| dt \leq 2 H$. We have $|g(s)| \leq K+1 = K_2$. Now
$$
S_2 \leq H^{\frac{r\epsilon}{4} + 2 A + 3} C^{2A+4},
$$
and $U\log \left(\frac{\lambda_N}{\lambda_M} \right) = U\log \lambda_2 \geq (2C)^{-1} U$,
$$
S_1 \leq 2^r S_2 (U \log \lambda_2)^{-r} \leq 2^r S_2 ((2C)^{-1} U)^{-r}.
$$
This shows that
\begin{align*}
& (r+5) U + |J_2|\\
& \leq (r+5) U + \frac{2B^2}{U^{10}} + \frac{54B}{U} 2H + \frac{(H+ 64 B^2)}{(2C^{-1} U)^r} 2^r C^{2A+4} H^{\frac{r\epsilon}{4} + 2 A + 3}\\
& 16 B^2 \Exp \left(-\frac{U}{8B} \right) H^{\frac{r\epsilon}{4} + 2 A + 3} C^{2A+4}\\
& \leq 100r B \log \log K_2 + r H^{\frac{7}{8}} \left\{\frac{r+5}{r} + \frac{2B^2}{rH^{\frac{69}{8}}} + \frac{108B}{H^{\frac{3}{4}}} \right\}\\
& + (H + 64 B^2) 2^r C^{2A+4} H^{\frac{r\epsilon}{4} + 2 A + 3 + \frac{r}{16} - (r+1) \frac{7}{8}}  \\
& +16B^2 C^{2A+4} (8B)^r r !  H^{\frac{r\epsilon}{4} + 2A + 3 -\frac{7r}{8}}\\
& \leq 10 r H^{\frac{7}{8}} + 100 r B \log \log K_2,
\end{align*}\pageoriginale
when $H$ satisfies the inequality of the theorem.
\end{proof}

\section{Third Main Theorem}\label{c2:sec2.5}
Let $\{a_n\}$ and $\{\lambda_n\}$ be as in the introduction and $|a_n| \leq (nH)^A$ where $A \geq 1$ is an integer constant. Then $F(s) = \sum\limits^\infty_{n=1} a_n \lambda^{-s}_n$ is analytic in $\sigma \geq A +2$. Suppose $F(s)$ is analytically continuable in $\sigma \geq 0$. Assume that (for some $K \geq 30$) there exist $T_1$ and $T_2$ with $0\leq T_1 \leq H^{\frac{7}{8}}$, $H- H^{\frac{7}{8}} \leq T_2 \leq H$ such that $|F(\sigma + iT_1)| + |F(\sigma + iT_2)| \leq K$ uniformly in $0 \leq \sigma \leq A + 2$. Let 
$$
H \geq (4C)^{9000 A^2} + 520000 A^2 \log \log K_3. 
$$
Then
{\fontsize{10}{12}\selectfont
$$
\int^H_0 |F(it)|^2 dt \geq  \sum\limits_{n\leq \alpha H} (H -
(3C)^{1000A} H^{\frac{7}{8}} -  130000 A^2 \log \log K_3 - 100 C^2 n)
|a_n|^2,  
$$}
where $\alpha = (200 C^2)^{-1} 2^{-8A - 20}$ and
$$
K_3 = \left( \sum\limits_{n\leq H} |a_n| \lambda^B_n\right) K + \left( \sum\limits_{n\leq H} |a_n| \lambda^B_n\right)^2. 
$$

\begin{coro*}
Let $A$\pageoriginale and $C$ be as in the introduction \S\ \ref{c2:sec2.1}, $|a_n| \leq (nH)^A$. Then $F(s) = \sum\limits^\infty_{n=1} a_n \lambda^{-s}_n$ is analytic in $\sigma \geq A +2$. Let $K \geq 30$, $K_2 = (HC)^{12A} K$, 
$$
H \geq (4C)^{9000 A^2} + 520000 A^2 \log \log K_2, 
$$
and that there exist $T_1, T_2$ with $0\leq T_1 \leq H^{\frac{7}{8}}$, $H - H^{\frac{7}{8}}\leq T_2 \leq H$, such that uniformly in $\sigma \geq 0$ we have
$$
|F(\sigma + iT_1)| + |F(\sigma + iT_2)| \leq K,
$$ 
where $F(s)$ is assumed to be analytically continuable in $(\sigma \geq 0, 0 \leq t \leq H)$. Then
\begin{gather*}
\frac{1}{H} \int^H_0 |F(it)|^2 dt \geq \sum\limits_{n\leq\alpha H} (1-(3C)^{1000A} H^{-\frac{1}{8}} - 130000 A^2 H^{-1} \log \log K_2\\
- 100 C^2 H^{-1} n) |a_n|^2,
\end{gather*}
where $\alpha = (200 C^2)^{-1} 2^{-8A - 20}$.
\end{coro*}

To prove this theorem we need the following two lemmas.

\begin{sublemma}\label{c2:lem2.5.1}
In the interval $[\alpha H, (1600 C^2)^{-1} H]$ there exists an $X$ such that
$$
\sum\limits_{X \leq n \leq X + H^{\frac{1}{4}}} |a_n|^2 \leq H^{-\frac{1}{4}} \sum\limits_{n\leq X} |a_n|^2,
$$
provided $H \geq 2^{1000 A^2} C^{50A}$.
\end{sublemma}

\begin{proof}
Assume that such an $X$ does not exist. Then for all $X$ in $[\alpha
  H,\break (1600C^2)^{-1} H]$, 
\begin{equation*}
\sum\limits_{X \leq n \leq X + H^{\frac{1}{4}}} |a_n|^2 > H^{-\frac{1}{4}} \sum\limits_{n\leq \sum} |a_n|^2 \tag{2.5.1}. \label{c2:eq2.5.1}
\end{equation*}
Let $L = \alpha H$, $I_j = [2^{j-1} L, 2^j L]$ for $j =1,2, \ldots 8A + 17$. Also let $I_0 = [1,L]$. Put $S_j = \sum\limits_{n\in I_j} |a_n|^2 (j=0,1,2,\ldots, 8A +17)$. For $j \geq 1$ divide the interval $I_j$ into\pageoriginale maximum number of disjoint sub-intervals each of length $H^{\frac{1}{4}}$ (discarding the bit at one end). Since the lemma is assumed to be false  the sum over each sub-interval is $\geq H^{-\frac{1}{4}} S_{j-1}$. The number of sub-intervals is $\geq [2^{j-1} L H^{-\frac{1}{4}}] - 1 \geq 2^{j-2} L H^{-\frac{1}{4}}$ (provided $2^{j-1} LH^{-\frac{1}{4}} - 2 \geq 2^{j-2} LH^{-\frac{1}{4}}$, i.e. $2^{j-2} LH^{-\frac{1}{4}} \geq 2$ i.e. $\alpha H^{\frac{3}{4}} \geq 4$ i.e. $H \geq (4\alpha^{-1})^{\frac{4}{3}}$). It follows that $S_j \geq 2^{j-2} LH^{-\frac{1}{2}} S_{j-1}$. By induction $S_j \geq (\frac{1}{2} LH^{-\frac{1}{2}})^j S_0$. Since $S_0 \geq 1$ we have in particular
$$
S_{8A + 17} \geq \left(\frac{1}{2} \alpha H^{\frac{1}{2}} \right)^{8A + 17} \geq \left( \frac{1}{2} \alpha \right)^{8A + 17} H^{4A + \frac{1}{2} \cdot 17}.
$$
On the other hand 
$$
S_{8A + 17} = \sum\limits_{\alpha_1 H \leq n \leq \alpha_2 H} |a_n|^2 \leq \sum\limits_{n\leq \alpha_2 H} (nH)^{2A},
$$
where $\alpha_1 = 16^{-1} (200 C^2)^{-1}$ and $\alpha_2 = 8^{-1} (200C^2)^{-1}$. Thus $S_{8A +17} \leq H^{4A+1}$. Combining the upper and lower bounds we are led to 
\begin{equation*}
H^{\frac{1}{2} \cdot 15} \leq (2\alpha^{-1})^{8A + 17} \tag{2.5.2}\label{c2:eq2.5.2}
\end{equation*}
provided $H \geq (4\alpha^{-1})^{\frac{4}{3}}$ (the latter condition is satisfied by the inequality for $H$ prescribed by the Lemma). But (\ref{c2:eq2.5.2}) contradicts the inequality prescribed for $H$ by the lemma. This contradiction proves the Lemma.
\end{proof}

From now on we assume that $X$ is as given by Lemma \ref{c2:lem2.5.1}.

\begin{sublemma}\label{c2:lem2.5.2}
Let $\bar{A}(s) = \sum\limits_{n \leq X} a_n \lambda^{-s}_n$, $E(s) = \sum\limits_{X \leq n \leq X + H^{\frac{1}{4}}} a_n \lambda^{-s}_n$ and $B(s) = F(s) - \bar{A}(s) - E(s)$. Clearly in $\sigma \geq A + 2$ we have $B(s) = \sum\limits_{n \geq X + H^{\frac{1}{4}}} a_n\lambda^{-s}_n$. Let $H \geq 2^{1000 A^2} C^{50 A}$, $U = H^{\frac{7}{8}} + 100 B \log \log K_3$, $K_3 \geq 30$ and $H \geq (2r + 5) U$. Then we have the following five inequalities. 
\begin{itemize}
\item[{\rm (a)}] $\int^H_0 |\bar{A}(it)|^2 dt \leq 100 C^2 H \sum\limits_{n \leq X} |a_n|^2$, 

\item[{\rm (b)}] $\int^{H-(r+3)U}_{2U+rU} |\bar{A}(it)|^2 dt \geq \sum\limits_{n \leq X} (H - (2r + 5) U - 100 C^2 n)|a_n|^2$,

\item[{\rm (c)}]  $\int^{H}_0 |E(it)|^2 dt \leq 100 C^2 H^{\frac{3}{4}} \sum\limits_{n\leq X} |a_n|^2$,\pageoriginale 

\item[{\rm (d)}]  $\int^H_0 |B(it)|^2 dt \leq 1000 C^2 H \sum\limits_{n\leq X} |a_n|^2$,
and finally

\item[{\rm (e)}]  $\int^H_0 |A(-it) B( it)| dt \leq 400 C^2 H \sum\limits_{n\leq X} |a_n|^2$,
\end{itemize}
where (d) and (e) are true provided 
$$
\int^H_0 |F(it)|^2 dt \leq H \sum\limits_{n \leq X} |a_n|^2.
$$
\end{sublemma}

\begin{proof}
The inequalities (a) and (b) follow from the Montgomery - Vau\-ghan theorem. From the same theorem
\begin{align*}
\int^H_0 |E(it)|^2 dt & \leq \sum\limits_{X \leq n \leq X + H^{\frac{1}{4}}} (H+ 100 C^2 n) |a_n|^2 \\
& \leq 100 C^2 H \sum\limits_{X \leq n \leq X + H^{\frac{1}{4}}} |a_n|^2
\end{align*}
and hence (c) follows from Lemma \ref{c2:lem2.5.1}. Since
$$
|B(it)|^2 \leq 9 (|F(it)|^2 + |\bar{A} (it)|^2 + |E(it)|^2)
$$
the inequality (d) follows from (a) and (c). Lastly (e) follows from (a) and (d). Thus the lemma is completely proved.
\end{proof}

We are now in a position to prove the theorem. We write (with $\lambda = u_1 + u_2 + \ldots + u_r$ as usual)
$$
\int^H_0 |F(it)|^2 dt \geq U^{-r} \int^U_0 du_r \ldots \int^U_0 du_1 \int^{H-(r+3) U+\lambda}_{2U+\lambda } |F(it)|^2 dt
$$
(where $(r+5) U \leq H$ and $0 \leq u_i \leq U$. In fact we assume $(2r+5) U \leq H$).

Now\pageoriginale
{\fontsize{10}{12}\selectfont
$$
|F(it)|^2 \geq |\bar{A}(it)|^2 + 2 \re (A(-it) B(it)) + 2 \re (A (-it)
E (it)) + 2 \re (\bar{B} (-it) E (it)), 
$$}
where $\bar{B}(s)$ is the analytic continuation of $\sum\limits_{n \geq X + H^{\frac{1}{4}}} a_n\lambda^{-s}_n$. Accordingly
\begin{equation*}
\int^H_0 |F(it)|^2 dt \geq J_1 + J_2 + J_3 + J_4 \tag{2.5.3}\label{c2:eq2.5.3}
\end{equation*}
where 
\begin{gather*}
J_1 =\int^{H}_0 |\bar{A}(it)|^2 dt, J_2 = 2 \re \int^H_0 (A(-it) B(it))dt,\\
J_3 = 2 \re \int^H_0 (A(-it) E(it)) dt \text{ and } J_4 = 2 \re \int^H_0 (\bar{B}(-it) E(it))dt.
\end{gather*}
By Lemma \ref{c2:lem2.5.2} (b), we have,
$$
J_1 \geq \sum\limits_{n \leq X} (H - (2r + 5) U - 100 C^2 n) |a_n|^2.
$$
Also by Lemma \ref{c2:lem2.5.2} ((a) and (c)), we have,
$$
|J_3| \leq 2 \int^H_0 |A(-it) E(it)| dt \leq 200 C^2 H^{\frac{7}{8}} \sum\limits_{n \leq X} |a_n|^2. 
$$
Similarly by Lemma \ref{c2:lem2.5.2} ((c) and (d)),
$$
|J_4| \leq 800 C^2 H^{\frac{7}{8}} \sum\limits_{n\leq X} |a_n|^2.
$$
For $J_2$ we use the main lemma. We choose $U = H^{\frac{7}{8}} + 100 B \log \log K_3$. We have $g(s) = A(-s) B (s)$. We have 
$$
|g(s)| \leq \left( \sum\limits_{n\leq H} |a_n| \lambda^B_n \right) K + \left( \sum\limits_{n\leq H} |a_n| \lambda^B_n \right)^2 = K_3.
$$
By Lemma \ref{c2:lem2.5.2} ((e))\pageoriginale we have
$$
\int^H_0 |g(it)| dt \leq 400 C^2 H \sum\limits_{n\leq X} |a_n|^2.
$$
Again
\begin{align*}
S_2 & \leq \sum\limits_{m\leq X, n \geq X + H^{\frac{1}{4}}} | a_m ||a_n|  \left(\frac{\lambda_m}{\lambda_n} \right)^{A+2}\\
& \sum\limits_{m \leq X , n \geq  X+ H^{\frac{1}{4}}} (mH)^A (nH)^A (C^2 mn^{-1})^{A+2}\\
& \leq C^{2A+ 4} H^{4A+3}.
\end{align*}
Put $x = \frac{\lambda_N}{\lambda_M} - 1$ where $N = \left[X+H^{\frac{1}{4}} \right]$, $M = [X]$. Then $0 < x < \frac{2c(n-M)}{c^{-1}M} < \frac{3C^2 H^{\frac{1}{4}}}{\alpha H} < \frac{1}{2}$ under the conditions on $H$ imposed in the theorem. Hence
\begin{align*}
U \log\left(\frac{\lambda_N}{\lambda_M}\right) &\geq \frac{U}{2}
\left(\frac{\lambda_N - \lambda_M}{\lambda_M} \right)\\ 
&\geq \frac{U}{2} \left(\frac{N-M-3}{C^2 M} \right)\\ 
&\geq \frac{1}{2} H^{\frac{7}{8}} \left(\frac{H^{\frac{1}{4}} - 3}{C^2
  H} \right)\\
&\geq \frac{H^{\frac{1}{8}}}{3C^2},
\end{align*}
(under the conditions on $H$ imposed in the theorem). Thus
$$
S_1 \leq 2^r S_2 H^{-\frac{r}{8}} (3C^2)^r.
$$
We choose $r = 100 A + 100$ and check that $U \geq 2^{70} (16B)^2$, and that $H \geq (2r + 5) U$. Thus by applying the main Lemma we obtain
\begin{gather*}
|\frac{1}{2} J_2| \leq \left\{\frac{2B^2}{U^{10}} + \frac{54B}{U} (400 C^2 H) + \frac{(H + 64 B^2) 2^r C^{2A  + 4} H^{4A+3}}{((3C^2)^{-1} H^{\frac{1}{8}})^r} \right.\\
+ 16 B^2 \Exp \left(-\frac{U}{8B}\right) C^{2A+4} H^{4A+3} \sum\limits_{n \leq X} |a_n|^2.
\end{gather*}
Hence
$$
\int^H_0 |F(it)|^2 \geq \sum\limits_{n\leq \alpha H} (H - D - 100C^2 n) |a_n|^2,
$$
where 
\begin{align*}
& D = (2r+5) U + 1000 C^2 H^{\frac{7}{8}} + \frac{4B^2}{U^{10}} + \frac{43200C^2 BH}{U}\\
& + (H + 64 B^2) 2^{r+1} C^{2A+4} (3C^2)^r H^{4 A +3-\frac{r}{8}}\\
& + 32 B^2 \Exp \left(-\frac{U}{8B} \right)  C^{2A+4} H^{4A+3}\\
& < 130000 A^2 \log \log K_3 + 405 A H^{\frac{7}{8}} + 1000C^2 H^{\frac{7}{8}} + 36 A^2 H^{\frac{7}{8}}\\
& + 43200 C^2 (3A) H^{\frac{7}{8}}\\
& + 600 A^2 H (2^{100 A  + 101}) C^{2A+4} 3^{100A+100} C^{200A+200} H^{4A + 3 - 12A-12}\\
& + 300 A^2C^{2A+4} (720) (56) (24A)^8 H^{\frac{7}{8}}\\
& \leq 130000 A^2 \log \log K_3 + H^{\frac{7}{8}} \left\{405A + 1000 C^2 +36 A^2+129600 AC^2 \right.\\
& \left. + 600 A^2 C^{406 A} 3^{401 A} + 3^{58} A^{10} C^{6A} \right\}\\
& \leq 130000 A^2 \log \log K_3 + (3C)^{1000A}.
\end{align*}\pageoriginale
This proves the theorem completely.

\medskip
\begin{center}
\textbf{Notes at the end of Chapter II}
\end{center}

The previous history of the fundamental theorems proved in this chapter is as follows. In 1928 E.C. Titchmarsh proved (see Titchmarsh's book \cite{Titchmarsh1} p. 174 and also E.C. Titchmarsh \cite{Titchmarsh4}) that for every fixed integer $k \geq 1$ and any $\delta > 0$, we have
$$ 
\delta \int^\infty_0 |\zeta\left(\frac{1}{2} + it \right)|^{2k} e^{-\delta t } dt \gg \left( \log \frac{1}{\delta} \right)^{k^2},
$$
where the constant implied by $\gg$ depends only on $k$. Later these ideas were  developed by the author to prove $\Omega$-theorems for short intervals on the line $\sigma =\frac{1}{2}, \frac{1}{2} <\sigma < 1$, and on $\sigma =1$ (see K. Ramachandra \cite{Ramachandra6}). These were taken up again by the author who introduced ``TITCHMARSH SERIES'' and proved very general theorems which gave
$$
 \frac{1}{H} \int^{T+H}_T |\zeta \left(\frac{1}{2} + it \right)|^{2k} dt \gg (\log H)^{k^2}, \; (k \geq 1 \text{ fixed integer}),
$$
where\pageoriginale $H \gg \log \log T$ and at the same time gave $\Omega$-theorems for short intervals on the said lines. These results were presented by the author in Durham Conference (1979) (See K. Ramachandra, \cite{Ramachandra7}). The three fundamental  theorems proved in this chapter are due to R. Balasubramanian and K. Ramachandra. (See R. Balasubramanian and K. Ramachandra, \cite{Balasubramanian and Ramachandra5}, \cite{Balasubramanian and Ramachandra4} and also R. Balasubramanian, \cite{Balasubramanian1}). The word Titchmarsh's phenomenon is used to mean the swayings of $|\zeta(s)|$ as $t$ varies and $\sigma$ is fixed.
