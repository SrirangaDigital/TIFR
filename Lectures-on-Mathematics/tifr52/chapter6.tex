
\chapter{Finite Difference Schemes, Stability}\label{chap6}

\section{Introduction}\label{chap6:sec6.1}

In\pageoriginale this section we define what we mean by the stability
of difference schemes and view some conditions which are necessary
and/or sufficient for stability. In the study of stability, the
Fourier transform is a useful tool and so is it also in the study of
well-posedness.

\section{The Fourier Transform}\label{chap6:sec6.2}

For a well-posed problem we have a correspondence between the initial
data $\varphi_\circ(x)$ and the solution $\varphi(x,t)$. For a given
$t$, define
\begin{equation*}
\varphi(t) : x \mapsto \varphi (x,t).
\tag{6.1}\label{eq6.1}
\end{equation*}
If the problem is linear, then there exists a linear operator $G(t)$
such that 
\begin{equation*}
\varphi(t) = G(t) \varphi (0). \tag{6.2}\label{eq6.2}
\end{equation*}
The operator $G(t)$ acts on the space where we seek the solution and
where the initial data is also given. For instance we may have
$\varphi(0) \in L^2 _x(\mathbb{R})$ and we may look for the solution
in the same space. Then
\begin{equation*}
G(t): L^2_x (\mathbb{R}) \to L^2_x(\mathbb{R}). \tag{6.3}\label{eq6.3}
\end{equation*}

A fundamental question arising in the study of well-posedness is
whether the induced norm $||G(t)||_{L(X,X)}$ (where $X$ is the space
on which $G(t)$ operates) is bounded or not.

One way to answer this question is via the energy inequality. A second
method, which is essentially the same but more convenient (especially
when working in $L^2 (\mathbb{R})$), is that of the Fourier transform.

The Fourier transform, $\hat{\varphi}$, of a function $\varphi \in L^1
\mathbb{R})$ is defined by 
\begin{equation*}
\hat{\varphi} (\xi) = \frac{1}{\sqrt{2\pi}}
\int\limits^\infty_{-\infty} e^{{\rm ix }\xi} \varphi (x)
dx. \tag{6.4}\label{eq6.4} 
\end{equation*}\pageoriginale 
That this can be extended to the space $L^2 (\mathbb{R})$ and that the
map
$$
\mathscr{F} : L^2 (\mathbb{R}) \to L^2 (\mathbb{R})
$$
which maps $\varphi$ to $\hat{\varphi}$ is an isometry of $L^2
(\mathbb{R})$ are all well-known results of functional analysis. (See,
for instance, Rudin: {\em Functional Analysis}, McGraw-Hill). The
invers. Fourier transform is given by 
\begin{equation*}
\varphi(x) = \frac{1}{\sqrt{2\pi}}  \int\limits^\infty_{-\infty}
e^{-{\rm ix} \xi} \hat{\varphi} (\xi) d\xi. \tag{6.5}\label{eq6.5}
\end{equation*}

Now consider $\varphi (\cdot, t)$ and $\varphi (\cdot, 0)$ to be in
$L^2_x(\mathbb{R})$. Then $\mathscr{F}$ maps them to $\hat{\varphi}
(\cdot, t)$ and $\hat{\varphi} (\cdot, 0)$ respectively in
$L^2_{\xi}(\mathbb{R})$. The correspondence defined by (\ref{eq6.2}) induces a
relation between their Fourier transforms which we denote by 
\begin{equation*}
\hat{\varphi} (t) = \hat{G} (t) \hat{\varphi} (0).\tag{6.6}\label{eq6.6}
\end{equation*}

Since $\mathscr{F}$ is an isometry from $L^2_x(\mathbb{R})$ onto
$L^2(\mathbb{R})$ we see that $||G(t)|| = ||\hat{G}(t)||$. Thus it is
equivalent to checking either the boundedness of $||\hat{G}(t)||$ or
that of $||G(t)||$. The former is often easier to apply when we are in
the case of partial differential equations with constant
coefficients. 

\begin{exam}\label{chap6:exam6.1}
Consider the advection equation with $u$ constant:
\begin{equation*}
\varphi_t + u \varphi_x =  0. \tag{6.7}\label{eq6.7}
\end{equation*}
Applying the Fourier transform w.r.t. $x$, we have, for fixec $\xi
\in \mathbb{R}$,
\begin{equation*}
\frac{d}{dt} (\hat{\varphi} (\xi, t)) - \; {\rm iu } \xi \hat{\varphi}
(\xi,t) = 0.\tag{6.8}\label{eq6.8}
\end{equation*}
Integrating this equation, we get
\begin{equation*}
\hat{\varphi} (\xi,t) = \hat{\varphi} (\xi, 0) \exp ({\rm iu } \xi
t).\tag{6.9}\label{eq6.9}
\end{equation*}\pageoriginale 
Thus $G(t)$ is merely the multiplication by $\exp ({\rm iu} \; \xi
t)$. 
\end{exam}

\begin{exam}\label{chap6:exam6.2}
In the heat equation
\begin{equation*}
\varphi_t - \varphi_{xx} = 0,\tag{6.10}\label{eq6.10}
\end{equation*}
we get, on applying the Fourier transform,
\begin{equation*}
\frac{d}{dt} (\hat{\varphi} (\xi, t)) + \xi^2 \hat{\varphi} (\xi ,t) =
0, \tag{6.11}\label{eq6.11}
\end{equation*}
which gives 
\begin{equation*}
\hat{\varphi} (\xi, t) = \hat{\varphi} (\xi, 0) \exp (-\xi^2 t). 
\tag{6.12}\label{eq6.12}
\end{equation*}
Thus $\hat{G}(t)$ is just multiplication by $\exp (-\xi^2 t)$ . 
\end{exam}

\begin{exercise}\label{chap6:exer6.1}
Find $\hat{G}(t)$ for the wave equation system
$$
\begin{cases}
u_t + v_x & = 0\\
v_t + u_x & = 0.
\end{cases}
$$
\end{exercise}

To compute the norms of these we need two simple results:

\begin{lem}\label{chap6:lem6.1}
Let $a(\xi)$ be a bounded function on $\mathbb{R}$ and let $A: L^2_\xi
(\mathbb{R}) \to L^2_\xi (\mathbb{R})$ be multiplication by
$a(\xi)$. Then
\begin{equation*}
||A||_{L(L^2_\xi, L^2_\xi )} = \sup \mid a (\xi)\mid.
\tag{6.13}\label{eq6.13}
\end{equation*}
\end{lem}

\begin{proof}
Let $u(\xi) \in L^2_\xi (\mathbb{R})$. Then
$$
Au(\xi) = a(\xi) u(\xi). 
$$
One has 
\begin{align*}
||Au||^2_2 & = \int\limits^\infty_{-\infty} |a(\xi)|^2 \; |u(\xi)|^2 d
\xi\\
& \leq (\sup\limits_\xi |a(\xi)|)^2 \int\limits^\infty_{-\infty}
|u(\xi)|^2 d \xi.
\end{align*}
Thus\pageoriginale $||Au||_2 \leq (\sup\limits_\xi |a(\xi)|)
||u||_2$. Hence $A$ is bounded and its norm is 
$$
\leq \sup\limits_{\xi} |a(\xi)| = ||a||_\infty.
$$

To complete the proof, we show that for any $\varepsilon > 0$,
$||A||>||a||_\infty - \varepsilon$. Consider the set
$$
E = \{ \xi \in \mathbb{R} \mid |a(\xi) | \geq ||a||_\infty -
\varepsilon\}. 
$$
By definition of $||\cdot ||_\infty$, the above set has positive
Lebesgue measure. By properties of the Lebesgue measure one can find a
subset $F$ of $E$ which is measurable and such that 
$$
0< \mu (F) < \mu (E)
$$
where $\mu$ is the Lebesgue measure. Now take $u = \chi_{F} /
\sqrt{\mu(F)}$ where $\chi_F$ is the characteristic function of
$F$. This is clearly in $L^2 (\mathbb{R})$ and $||u||_2 =1$. Hence
\begin{align*}
||A||^2 \geq ||Au||^2_2 & = \int_\mathbb{R} |a(\xi)|^2
\frac{(\chi_F(\xi))^2}{\mu(F)} d \xi,\\
& = \frac{1}{\mu(F)} \int\limits_F |a(\xi)|^2 d \xi\\
& \geq \frac{1}{\mu(F)} \int\limits_F (||a||_\infty - \varepsilon)^2 d
\xi\\
& \geq (||a||_\infty - \varepsilon)^2.
\end{align*}
Thus $||A|| \geq ||a||_\infty - \varepsilon$ for each $\varepsilon >
0$ and this establishes (\ref{eq6.13}).
\end{proof}

We can extend this to the case of a finite product of $L^2_\xi
(\mathbb{R})$ with itself. We omit the proof and merely state the
result;

\begin{lem}\label{chap6:lem6.2}
If $U^T = (u_1 (\xi), \ldots , u_n (\xi))$ where $u_i (\xi) \in
L^2_\xi (\mathbb{R})$ and $B$ is defined by 
\begin{equation*}
BU(\xi) = A (\xi) U(\xi), \tag{6.14}\label{eq6.14}
\end{equation*}\pageoriginale 
$A(\xi)$ being an $n \times n$ matrix, then the norm of $B$ induced by
the vector norm 
\begin{equation*}
||U||^2 = \int\limits^\infty_{-\infty} |U(\xi)|^2 d \xi \tag{6.15}\label{eq6.15}
\end{equation*}
is given by
\begin{equation*} 
B = \sup\limits_\xi |A(\xi)|
\tag{6.16}\label{eq6.16}
\end{equation*}
where $|A(\xi)|$ is the Euclidean norm of the matrix $A(\xi)$.
\end{lem}

By virtue of these lemmas it is easy to see that in examples
\ref{chap6:exam6.1} and \ref{chap6:exam6.2}, $||\hat{G}(t)||
=1$. The case of exercise \ref{chap6:exer6.1} for the wave equation
is again left as an exercise. 

\section{Stability of two-level schemes}\label{chap6:sec6.3}

We now turn to finite difference schemes. Given a system of
differential equations over a domain, we discretize the system by
establishing a {\em mesh} of discrete points over the region and
replacing the differential operators by difference operators involving
these points.

Let us consider a {\em uniform} mesh of step $\Delta \; x$ in the
$x$-direction and step $\Delta$ $t$  in the $t$-direction. The nodes
of this mesh are thus the points $(j \Delta x, \; k \Delta t)$ where
$j$, $k \in \mathbb{Z}$, the set of integers with $k \geq 0$. The aim
of a difference scheme would be to express the value of the solution
at $u(x, n \Delta t)$ in terms of $u(y, \; k \Delta t)$ where
$k<n$. Let us denote by ($u^n(x)$) the value $u(x, n \Delta t)$. Then
a general {\em 2-level} finite difference scheme will take the form
\begin{equation*}
\sum\limits_{j\in\mathbb{Z}} b_j u^{n+1} (x+j \; \Delta x) =
\sum\limits_{j\in\mathbb{Z}} c_j u^n (x + j \; \Delta x).\tag{6.17}\label{eq6.17}
\end{equation*}

\begin{remark}\label{chap6:rem6.1}
Though these summations range over all $\mathbb{Z}$ in theory, we
only have, in practice, $j$ ranging over a finite set of values.
\end{remark}

If\pageoriginale $b_j=0$ for all $j \neq 0$, then we can explicitly
compute $u^{n+1}(x)$ in terms of $u^n$. Such a scheme is called
explicit. Otherwise the scheme is implicit.

\begin{remark}\label{chap6:rem6.2}
Though, on the face of it, it looks as if an explicit scheme is more
desirable compared to an implicit one, this is not always the case. As
we shall see in later examples, explicit schemes are not always
``unconditionally stable'' (i.e. stable for all values of $(\Delta x,
\; \Delta t)$) while implicit schemes may have this important property.
\end{remark}

In general, starting with a 2-level scheme, we can write, at least in
theory.
\begin{equation*}
u^{n+1} (x) = G(\Delta x, \Delta t) u^n (x), 
\tag{6.18}\label{eq6.18}
\end{equation*}
where $G(\Delta x, \Delta t): X \to X$, $X$ being the space where the
$u^n$,s belong. Recursively, one has 
\begin{equation*}
u^{n+1} (x) = G^{n+1} u^\circ(x). 
\tag{6.19}\label{eq6.19}
\end{equation*}
Then we have 

\begin{Definition}\label{chap6:def6.1}
The scheme given by (\ref{eq6.17}) is stable w.r.t. a given norm $||\cdot
||_X$ if and only if 
\begin{equation*}
|| G^n||_{L(X,X)} \leq \text{ a constant},
\tag{6.20}\label{eq6.20}
\end{equation*}
the constant being independent of $n$ for all $n>0$.
\end{Definition}

The simplest case, because of the use of the Fourier transform, is the
study of the $L^2$-stability of a scheme.

Starting from equation (\ref{eq6.17}) and applying the Fourier transform, we
get
$$
\sum\limits_j \int\limits^\infty_{-\infty} b_j u^{n+1} (x+j\Delta
x)e^{i\xi x} dx = \sum\limits_j \int\limits^\infty_{-\infty} c_j u^n
(x+ j\Delta x) e^{i\xi x} dx.
$$\pageoriginale 
Replacing $x + j \Delta x$  by $y_j$, we have
$$
\sum\limits_j b_j e^{-i \xi j \Delta x }  \hat{u}^{n+1} (\xi) =
\sum\limits_j c_j e^{-i\xi j \Delta x} \hat{u}^n(\xi).
$$
Setting
\begin{equation*}
\left.
\begin{aligned}
b (\xi) & = \sum\limits_j b_j e^{-i \xi j \Delta x}\\
c(\xi) & = \sum\limits_{j} c_j e^{-i \xi j \Delta x}
\end{aligned}
\right\}\tag{6.21}\label{eq6.21}
\end{equation*}
we have 
\begin{equation*}
b(\xi) \hat{u}^{n+1} (\xi) = c(\xi) \hat{u}^n (\xi). 
\tag{6.22}\label{eq6.22}
\end{equation*}

Thus $\hat{G}$ is merely multiplication by $a(\xi) = c(\xi) / b(\xi)$,
known as the {\em coefficient of amplification} of the scheme, and
$\hat{G}^n$ is multiplication by $(a(\xi))^n$.

Thus the scheme (\ref{eq6.17}) is stable if and only if there exists a
constant $C$, independent of $n$, such that
\begin{align*}
\max\limits_\xi |(a(\xi))^n| &  \leq C\\
\text{i.e. } \qquad (\max\limits_\xi |a (\xi)|)^n & \leq C. \qquad 
\end{align*}
or equivalently,
\begin{equation*}
\max\limits_{\xi} |a(\xi)| \leq 1.
\tag{6.23}\label{eq6.23}
\end{equation*}

Thus (\ref{eq6.23}) is a {\em necessary and sufficient} condition for the
$L^2$ - stability when we have a single {\em scalar equation}, and a
scheme given by (\ref{eq6.17}).

\section{Extension of systems}\label{chap6:sec6.4}
In the case of a system of $n$ equations, the $U^k$  are all
$n$-vectors. The quantities $b_j$ and $c_j$ of (\ref{eq6.17}) must now be
replaced by matrices\pageoriginale and hence we will have, on applying
the Fourier transform, matrices $B(\xi)$ and $C(\xi0$ playing the
roles of $b(\xi)$ and $c(\xi)$ in (\ref{eq6.22}). Thus the condition for
stability is 
\begin{equation*}
\max\limits_\xi |(B^{-1} (\xi) C(\xi))^n| \leq C,
\tag{6.24}\label{eq6.24}
\end{equation*}
$C$ being a constant independent of $n$.

One cannot reduce this to a neat condition which is necessary and
sufficient as in the scalar case. However, using the fact that
$|\cdot|$ is a matrix norm, a {\em sufficient} condition would be that
\begin{equation*}
\max\limits_\xi |B^{-1} C(\xi)| \leq 1.
\tag{6.25}\label{eq6.25}
\end{equation*}

One can obtain a necessary condition by argueing with the spectral
radius. Since, we have for any matrix $A$,
$$
(\rho (A))^n \leq |A^n|,
$$
where $\rho (A)$ is the spectral radius, a {\em necessary}
condition would be 
\begin{equation*}
\max\limits_\xi \rho (B^{-1} C(\xi)) \leq 1. 
\tag{6.26}\label{eq6.26}
\end{equation*}

Since for a normal matrix, the spectral radius is equal to the
Euclidean norm, we see that, if $B^{-1}C(\xi)$ is {\em normal} then
the condition (\ref{eq6.26}) is {\em necessary and sufficient}.

A host of necessary and/or sufficient conditions under various
hypotheses can be found in Richtmyer and Morton \cite{key32}.

Thomee \cite{key36} has studied the $L^p$-stability for $2 \leq  p
\leq + \infty$ using the Fourier transform.

\begin{remark}\label{chap6:rem6.3}
It must be observed that one can use Fourier transform only
when\pageoriginale the coefficients of the partial differential
equation are constants and when the mesh is uniform.
\end{remark}

Our subsequent study will be of numerical schemes for the heat and
advection equations. We will then turn to the general equations of
hydrodynamics in Section \ref{chap9}.
