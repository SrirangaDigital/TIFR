
\chapter{Energy Inequalities}\label{chap4}

\section{Introduction}\label{chap4:sec4.1}

The\pageoriginale question of well-posedness of a system of partial
differential equations with given initial or boundary conditions is
fundamental. This question is often answered using what are called
energy estimates or energy inequalities. From these estimates one
obtains the existence, uniqueness and continuous dependence on the
data of the solution of the problem, thus establishing well-posedness.

In this section, we take up the threee types of equations of section
1-the advection, wave and heat equations - and obtain energy estimates
in the linear case. We do not talk about the existence of solutions as
it entails some work in functional analysis. We only briefly sketch
how existence is proved, at the end of this section. 

Before we go forth, it is useful to put down our notations for the
various norms that we will be using. We write $|| \cdot ||{}_p$ for
the norm in $L^p (-\infty, \infty)$ where $1 \leq p \leq + \infty$. If
$U$ is a vector with $n$ components, its Euclidean norm is denoted by 
\begin{equation*}
|U|^2  = \sum\limits^n_{i=1} u^2_i. \tag{4.1}\label{eq4.1}
\end{equation*}
We also define 
\begin{equation*}
||U||^2 = \int^\infty_{-\infty} |U|^2 dx  = \int^\infty_{-\infty}
\sum\limits^n_{i=1} u^2_i dx, \tag{4.2}\label{eq4.2}
\end{equation*}
when each component $u_i$ is in $L^2 (-\infty, \infty)$. We also
consider only those functions in $L^2(-\infty, \infty)$ which vanish
at $\pm \infty$. 

\section{The advection equation}\label{chap4:sec4.2}
We consider\pageoriginale the simplest case where $u$ is a {\em constant}. Then we
have
\begin{equation*}
\frac{\partial \varphi}{\partial t} + \frac{\partial \varphi}{\partial
x} = 0 \tag{4.3}\label{eq4.3}
\end{equation*}
with $\varphi(x,0) = \varphi_0 (x)$, given,

Multiplying (\ref{eq4.3}) by $\varphi$ and integrating w.r.t. $x$ over the
entire real line, we get
\begin{equation*}
\int^\infty_{-\infty} \frac{\partial}{\partial t} \left(\frac{1}{2}
\varphi^2\right) dx + u \int^\infty_{-\infty} \frac{\partial}{\partial x}
\left(\frac{1}{2} \varphi^2\right) dx  =0. \tag{4.4}\label{eq4.4}
\end{equation*}
Since we look for a solution $\varphi$ such that $\varphi( \cdot, t)
\in L^2_x(-\infty, \infty)$ for each $t>0$ and which vanishes at $x =
\pm \infty$, the above equation gives 
\begin{equation*}
\frac{1}{2} \frac{d}{dt} (||\varphi (\cdot, t)||^2_2) = 0 \tag{4.5}\label{eq4.5}
\end{equation*}
which gives the {\em energy equality}
\begin{equation*}
||\varphi (\cdot , t) ||_2 = ||\varphi_\circ||_2.\tag{4.6}\label{eq4.6}
\end{equation*}


Observe that the uniqueness of the solution for given $\varphi_\circ$
is immediate from (\ref{eq4.6}). Indeed if, $\varphi_1$, $\varphi_2$ are two
solutions to (\ref{eq4.3}) then so is $\varphi_1 - \varphi_2$ with $(\varphi_1
- \varphi_2) (x, \circ) = 0$ for all $x$. Then (\ref{eq4.6}) shows that
$||(\varphi_1 - \varphi_2) (\cdot , t)||_2 = 0$ for all $t$ and hence
the uniqueness follows. 

\begin{exercise}\label{chap4:exer4.1}
Consider the equation
$$
\frac{\partial \varphi}{\partial t}+ \frac{\partial}{\partial x}
(\varphi u) = 0,
$$
where $u$ is not a constant. Assuming that $u$ and $\dfrac{\partial
  u}{\partial x}$ are bounded and looking for a solution $\varphi
(x,t)$ such that $\varphi (\cdot , t)$ is in $L^2_x(-\infty, \infty)$
for each $t$, vanishing for\pageoriginale $x = \pm \infty$, derive the
energy inequality
$$
||\varphi (\cdot, t) ||_2 \leq C ||\varphi (\cdot, 0)||_2, \; 0 <
t< \bar{t}. 
$$
\end{exercise}

\begin{remark}\label{chap4:rem4.1}
We have derived the energy inequality in the homogeneous case. It
can be shown that in case of linear equations, the estimate in the
homogeneous case also implies the existence of such an energy estimate
in the non-homogeneous case. For example, consider the equation 
\begin{equation*}
\frac{\partial \varphi}{\partial t} + u \frac{\partial
  \varphi}{\partial x} =f, \quad u \text{ constant. } \tag{4.7}\label{eq4.7}
\end{equation*}

Multiplying by $\varphi$ and integrating w.r.t. $x$, we get
$$
\frac{1}{2} \frac{d}{dt} (||\varphi (\cdot, t) ||^2_2) =
\int^\infty_{-\infty} f(x,t) \varphi(x,t) dx \leq ||f(\cdot, t)||_2
||\varphi(\cdot, t)||_2. 
$$
Hence 
\begin{equation*}
\frac{d}{dt} (||\varphi (\cdot, t) ||_2) \leq ||f(\cdot,
t)||_2. \tag{4.8} \label{eq4.8}
\end{equation*}
Integrating (\ref{eq4.8}) we get the energy inequality in the non-homogeneous
case as
\begin{equation*}
||\varphi (\cdot, t) ||_2 \leq ||\varphi (\cdot, 0) ||_2 +
\int^2_\circ ||f(\cdot, s) ||_2 ds.\tag{4.9}\label{eq4.9}
\end{equation*}

Again, this is a key-step in the proof of well-posedness of the
problem. 
\end{remark}

\begin{remark}\label{chap4:rem4.2}
One can also seek such `a priori' estimates in other spaces. For
instance in case of equation (\ref{eq4.3}) we know from section 1.6 that the
solution is given by
\begin{equation*}
\varphi (x,t) = \varphi_\circ (x-ut). 
\tag{4.10}\label{eq4.10}
\end{equation*}

Thus we can get the estimate in the $L^\infty$-norm, assuming
$\varphi_\circ \in L^\infty$ $(-\infty, \infty)$, as 
\begin{equation*}
||\varphi (\cdot , t)||_\infty = ||\varphi_\circ||_\infty. 
\tag{4.11}\label{eq4.11}
\end{equation*}\pageoriginale 
\end{remark}

\section{The wave equation}\label{chap4:sec4.3}

We will follow Friedrichs' argument for symmetric systems. We will
deal with the hyperbolic case. The system
\begin{equation*}
\frac{\partial U}{\partial t} + A \frac{\partial U}{\partial x} =0
\tag{4.12}\label{eq4.12}
\end{equation*}
where $U$ is an $n$-vector, is said to be symmetric if the matrix $A$
is symmetric. For  instance if
 $A = \begin{bmatrix}
0 & 1\\
0 & 0
\end{bmatrix}$, we get the wave equation. 

First of all we observe that 
\begin{equation*}
\sum\limits^n_{i=1} u_i \frac{\partial u_i}{\partial t} = \frac{1}{2}
\sum\limits^n_{i=1} \frac{\partial (u_i)^2}{\partial t} = \frac{1}{2}
\frac{\partial}{\partial t} (|U|^2). \tag{4.13}\label{eq4.13}
\end{equation*}

One also has
\begin{align*}
\frac{\partial}{\partial x} \left(\sum\limits_{i,j} a_{ij} u_i u_j\right) &
=\sum\limits_{i,j} \frac{\partial a_{ij}}{\partial x} u_i u_j +
\sum\limits_{i,j} a_{ij} \frac{\partial u_i}{\partial x} u_j +
\sum\limits_{i,j} a_{ij} u_i \frac{\partial u_j}{\partial x} = \\
& = \sum\limits_{i,j} \frac{\partial a_{ij}}{\partial x} u_i u_j + 2
\sum\limits_{i,j} u_i a_{ij} \frac{\partial u_j}{\partial x}
\end{align*}
by the symmetry of $A$. Hence one has 
\begin{equation*}
\sum\limits_{i,j} a_{ij} u_i \frac{\partial u_j}{\partial x} =
\frac{1}{2} \left[\frac{\partial}{\partial x} \langle U, AU\rangle -
  \langle U, \frac{\partial A}{\partial x} U \rangle \right] \tag{4.14}\label{eq4.14}
\end{equation*}
where $\langle \cdot, \cdot \rangle$ denotes the scalar product in
$\mathbb{R}^n$.

Using these, we get, by taking the scalar product with $U$ of the
equation (\ref{eq4.12}). 
\begin{equation*}
\frac{1}{2} \frac{\partial}{\partial t} (|U|^2) + \frac{1}{2}
\left[\frac{\partial}{\partial x} \langle U, A U\rangle  - \langle U,
  \frac{\partial A}{\partial x} U\rangle\right] = 0. \tag{4.15}\label{eq4.15}
\end{equation*}
Integrating w.r.t. $x$, and using the notation of section \ref{chap4:sec4.1}, we get
\begin{equation*}
\frac{1}{2} \frac{d}{dt} (|| U (\cdot, t)||^2) = \frac{1}{2} \int
\langle U, \frac{\partial A}{\partial x} U \rangle dx. \tag{4.16}\label{eq4.16}
\end{equation*}
Assume\pageoriginale that the $\dfrac{\partial a_{ij}}{\partial x}$
are all bounded, we get the inequality
\begin{equation*}
\frac{d}{dt} (||U (\cdot, t)||^2) \leq C ||U(\cdot, t)||^2.
\tag{4.17}\label{eq4.17}
\end{equation*}

It is a simple step to get the energy inequality from (\ref{eq4.17}). We leave
it as an 

\begin{exercise}\label{chap4:exer4.2}
Starting from (\ref{eq4.17}) deduce the inequality
$$
||U (\cdot, t) || \leq || U(\cdot , 0)|| \exp (Ct). 
$$
\end{exercise}

\section{The heat equation}\label{chap4:sec4.4}
We take the simplet case:
\begin{equation*}
\frac{\partial u}{\partial t} - \frac{\partial^2 u}{\partial x^3} =
0. \tag{4.18}\label{eq4.18}
\end{equation*}
Again, multiplying by $u$ and integrating w.r.t. $x$, we get
\begin{equation*}
\frac{1}{2} \frac{d}{dt} (|| u (\cdot, t) ||^2_2) -
\int^\infty_{-\infty} u \frac{\partial^2u}{\partial x^2} dx =
0. \tag{4.19}\label{eq4.19} 
\end{equation*}
Integrating the second term by parts, (\ref{eq4.19}) becomes 
\begin{equation*}
\frac{1}{2} \frac{d}{dt} (||u(\cdot , t)||^2_2) +
\int^\infty_{-\infty} (\frac{\partial u}{\partial x})^2 dx =0
\tag{4.20} \label{eq4.20}
\end{equation*}
Since the second term is non-negative, we can write 
\begin{equation*}
\frac{d}{dt} (|| u(\cdot ,t) ||^2_2) \leq 0\tag{4.21}\label{eq4.21}
\end{equation*}
which gives the energy inequality
\begin{equation*}
||u(\cdot, t) ||_2 \leq ||u(\cdot, 0)||_2. \tag{4.22}\label{eq4.22}
\end{equation*}

\section{Remarks on existence of solutions}\label{chap4:sec4.5}

A word about the existence of solutions. As is readily seen, the
uniqueness of the solution and its continuous dependence on the data
follows easily from the energy estimates. However, for the
existence\pageoriginale of solutions more work is necessary. In the
linear case we have the Galerkin method. We take a basis $\{v_1,
\ldots v_n, \ldots\}$ for $L^2 (-\infty, \infty)$ and then consider
the finite dimensional spaces $S_n$, spanned by $\{v_1, \ldots,
v_n\}$. We approximate the initial value function $u(x,0)$ by
$u_n(x,0)$ in $S_n$ and in the space $S_n$, the partial
differential equations give a system of ordinary differential
equations. To this we apply we existence theory available and get an
approximate solution $u_n(x,t)$. We then use the energy inequalities
to show that $u_n(\cdot, t)$ are bounded and we can extract a
subsequence converging (weakly) to a function $u$ which can be shown
to be a solution of the equations.

In the non-linear case,
\begin{equation*}
\frac{\partial u}{\partial t} + \frac{\partial}{\partial x} (F(u)) =
0, \tag{4.23}\label{eq4.23}
\end{equation*}
one not only has to show that $u_n \to u$ but also that $F(u_n) \to
F(u)$ for which a single `a priori' estimate is not enough. One does
not have general techniques for non-linear systems. Work has been done
only on a few specific examples.

\medskip
\noindent{\textbf{Reference:}}  Lions \cite{key26}.


