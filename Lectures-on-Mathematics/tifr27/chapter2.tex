\chapter{Theory of Valuations -II}\label{part1:chap2}

\section{Hensel's Lemma}\label{part1:chap2:sec1}

In\pageoriginale this section we give a proof of Hensel's lemma and deduce certain
corollaries which will be used quite often in the following. In this
section by a ring we mean a commutative ring with unity (It may have
zero divisors). 

\begin{defi*}%def
  Let $A$ be a ring. Two elements $x$ and $y$ in $A$ are said to be
  strongly relatively prime if and only if $Ax + Ay =A$ i.e. if and
  only if there exist two elements $u$ and $v$ in $A$ such that $ux+
  vy =1$.  
\end{defi*} 
 
In particular if $k[x]$ is the ring of polynomials over a filed $k$
then any two elements in $k[x]$ are strongly relatively prime if and
only if they are coprime in the ordinary sense. 
 
It is obvious that if $x$and $y$ are two strongly relatively prime
elements in a ring $A$, then for any $z$ in $A$ $x$ divides $y$ $z$
implies that $x$ divides $z$. 

\setcounter{Lemma}{0}
\begin{Lemma}\label{part1:chap2:sec1:lem1}%lem
  Let $P$ and $P'$ be two polynomials with coefficients in a ring $A$
  such that $P$ is monic and $P$ and $P'$ are strongly relatively
  prime. Let us assume that degree $P= d(P)=s$ and $d (P')=s'$. Then
  for every polynomial $Q$ in $A[x]$ there exists one and only one
  pair of polynomials $U$ and $V$ such that  
  $$
  Q=U P + V P' ~\text{with}~ d(V) < s 
  $$
  and\pageoriginale for every $t> s', d(Q)< t + s$ if and only if $d(U) <t$.
\end{Lemma}

\noindent \textit{Proof.}  
  The existence of one pair $U$ and $V$such that $Q=UP+V P'$ is
  trivial. If $d(V) >s$, we write $V=AP +B$ where $d(B)<s$, which is
  possible because $P$ is a monic polynomial, so we get  
  \begin{equation*}
  q = (U+A). P+B P' \text{with} d(B) <s.\tag*{$\Box$}
  \end{equation*}

Thus we can assume in the beginning itself that $d(V) < s$. If possible
let there exists another pair $U'$ and $V'$ such that 
$$
Q=U' P+V' P', d(V')< s.
$$

Then 

$U' P+V' P' = U P +V P'$ implies that $(U -U')P = (V'-V)P'$.

But $P$ and $P'$ are strongly relatively prime, therefore $P$ divides
$V'-V$. Since $d(V' -V)< s, V'-V=0$. This implies that $P(U-U')=0$. As
$P$ is monic we must have $U=U'$. Let $d(Q)< t+s$. Then $d(U P)= d(Q-V
P')$. But $d(V)<s$ and $d(P')= s' <t$, therefore $d(U P)< t+s$, which
implies that $d(U) <t$ because $P$ is a monic polynomial of degree
$s$. It is obvious that $d(V) < t \,(t>s')$ implies that $d(Q)< t+ s$. 

\begin{defi*}%def
  Let $A$ be a ring, the intersection of all the maximal ideals  is
  called the radical of $A$ and shall be denoted by $r(A)$. 
\end{defi*} 

 It can be easily proved that any element $x$ of $A$ belongs to $r(A)$
 if and only if 1-xy is invertible for all $y \in A$. 
 
\begin{Lemma}\label{part1:chap2:sec1:lem2}%lem
  Let\pageoriginale $A$ be a ring $\mathscr{O}$ an ideal in $A$ contained in
  $r(A)$. Then two polynomials $P$ and $P'$ in $A[x]$ ane of which
  (say P) is minic are strongly relatively prime if and only if
  $\bar{P}$ and $\bar{P}'$ (the images of $P$ and $P'$ in $A/
  \mathscr{O}[x]$) are strongly relatively prime. 
\end{Lemma} 
 
\begin{proof}%pro
  $P$ and $P'$ are strongly relatively prime implies $\bar{P}$ and
  $\bar{P}'$ are stron\-gly relatively prime is obvious. 
\end{proof} 
 
Suppose that $d(P')=s'$ and $d(P)=s$. Then $d(\bar{P})=d (P)=s$,
because $P$ is monic. Let $E= \{f | f \in A [x], d(f) < s+t$, for
  some $t> s' \}$. Then $E$ is a module of finite type over $A$. Let
$\bar{E}= E/ \mathscr{O} E$, since $\bar{P}$ and ${\bar{P}}'$ are
strongly relatively prime in $A/ \mathscr{O}[x]$, $\bar{E}'$ is
generated by the polynomials $X^u \bar{P}$ and $X^v \bar{P}'$ for $0
\leq u \leq s$. For, by Lemma \ref{part1:chap2:sec1:lem1} for every
polynomials $\bar{Q}$ in 
$\bar{E}$ there exists  one only one pair of polynomials $\bar{U}$ and
$\bar{V}$ in $A / \mathscr{O}[X]$ such that 
$$
\bar{Q} = \bar{U} \bar{P} +\bar{V}' \bar{P}' \quad d(\bar{V})< t +s.
$$
 
But $d(\bar{Q})< t+s$, therefore $d(\bar{U})<t$. Thus
\begin{align*}
  \bar{U} & = \sum_{\lambda =0}^{u}\bar{a}_{\lambda}X^{\lambda}, 0\leq u \leq t\\
  \bar{V} & = \sum_{\mu =0}^{v} \bar{b}_{\mu}X^{\mu}, 0 \leq v \leq s
\end{align*} 
and $\bar{Q} = \sum\limits_{\lambda=0}^{u}
\bar{a}_{\lambda}(X^{\lambda} \bar{P}) + \sum\limits_{\mu =0}^{v}
\bar{b}_{\mu} (X^{\mu}\bar{P}')$. 
 
By a simple corollary of Nakayama's lemma (For proof see Algebre by
$N$. Bourbaki chapter $8$ section $6$) which states that if $E$ is a\pageoriginale
module of finite type over a ring $A$ and $q$ and ideal in $r(A)$ then
if $(a_1,\ldots , a_n)$ generate $E$ module $qE$, they generate $E$ also, we
get $X^u P$ and $X^v P$ for $0 \leq u \leq t$ and $0 \leq v \leq s$ constitute a
set of generators for $E$. Therefore 
$$
1=\left(\sum_{r= 0}^{u} a_r X^r\right) P+ \left(\sum_{k= 0}^{v} b^k X^k\right) P'
$$
 because $1$ belongs to $E$. Hence $P$ and $P'$ are strongly
 relatively prime in $A[X]$. 
 
 Let $A$ be a ring with a decreasing filtration of ideals
 $(\mathscr{O}_n)_{n>0}$, defining a topology on $A$ for which $A$ is
 a complete Hausdorff space. If $f(X) = \sum\limits_{n=0}^{\infty} a_n
 X^n$ is a power series over $A$ converging everywhere in $A$ then
 $\lambda_n (f) = \sup\limits_{a_i \not\in \mathscr{O}_n} (i)$  
 
$(\lambda_n (f) < \infty$, because $a_n \to 0$ as $n\to \infty)$ is an
 increasing function of $n$ \iec  $\lambda_n (f) \leq
 \lambda_{n+1}(f)$ and $f(x)$ is a polynomial if and only if
 $\lambda_n (f)$ is constant for $n$ sufficiently large. 
 
We shall denote by $\bar{f}$ the image of $f$ in $A/ \mathscr{O}_1[X]$.
 
 \medskip
 \noindent
\textbf{Hensel's Lemma.} Let $A$ be a ring with a decreasing
filtration of ideals $(\mathscr{O}_n)_{n>0}$. Let $A$ for this
topology be a complete Hausdorff space. If $f(X) =
\sum\limits_{n=0}^{\infty} a_n X^n$ is an everywhere convergent power
series over $A$ and if there exist two polynomial $\varphi$ and $\psi$
an $A / \mathscr{O}_1[X]$ such that 
\begin{enumerate}[(1)] 
\item $\varphi$ is monic of degree $s$
\item $\varphi$ and $\psi$ are strongly relatively prime
\item $\bar{f}= \varphi \psi$  
 \end{enumerate}
then\pageoriginale there exists one and only pair $(g,h)$ such that 
\begin{enumerate}[(a)]
\item $g$ is a monic polynomial of degree $s$ in $A[X]$ and $\bar{g}=\varphi$.
\item $h$ is every where convergent power series over $A$ and $\bar{h}=\psi$.
\item $f=g h$
\end{enumerate}

Moreover $\lambda_n (h) = \lambda_n (f)-s$. If $f$ is a polynomial
then $h$ is a polynomial and $g$ and $h$ are strongly relatively
prime.  

\begin{proof}%pro
  {\em Existence} We construct two sequences of polynomials $(g_n)$
  and $(h_n)$ an $A[X]$ by induction on $n$ such that 
  \begin{align*}
    & (\alpha)\, g_n ~\text{is monic of degree s}, \bar{g}_n =\varphi
    ~\text{and}\\ 
    & g_n +1 \equiv g_n \pmod {\mathscr{O}_{n+1}} ~\text{for}~ n \geq 0\\
    & (\beta)\, \bar{h}_n = \psi,  h_{n+1} \equiv h_n \pmod
    {\mathscr{O}_{n+1}}  ~\text{and}\\ 
    & d(h_n)= \lambda_{n+1}(f) -s\\
    & (\gamma)\, f \equiv g_n h_n \pmod {\mathscr{O}_{n+1}}, n\geq 0
  \end{align*}

  For $n=0$, we take $g_o = \sum\limits_{r=0}^{s-1} a_r X^r + X^s$ if 
  \begin{align*}
     \varphi & = \sum_{r=0}^{s-1} \bar{a}_r X^r+ X^s ~\text{and}~ h_o =
    \sum_{u=0}^{t} b_u X^u ~\text{if}\\ 
     \psi &=  \sum_{u=0}^{t} \bar{b}_u X^u, ~\text{with}~ t=d (\psi
    )=d(\bar{f})- s =\lambda _1 (f)-s.  
  \end{align*}
  
  Let us assume that we have constructed the polynomials $g_1, g_2$
   $\cdots$ $g_{n-1}$ and $h_1, \ldots, h_{n-1}$ satisfying the conditions
  $(\alpha), (\beta)$, and $(\gamma)$. By lemma
  (\ref{part1:chap2:sec1:lem2}) $g_{n-1}$ and 
  $h_{n-1}$ are strongly relatively prime modulo $\mathscr{O}_q$ for
  every integer $q \geq 1$, because $g_{n-1}$ and $h_{n-1}$ are strongly
  relatively\pageoriginale prime in $A/ \mathscr{O}_1 [X]=A /
  \mathscr{O}_{q \bigg / 
    \mathscr{O}_1 / \mathscr{O}_g} [X]$ and $\mathscr{O}_1 /
  \mathscr{O}_q$ is contained in $ r(A/ \mathscr{O}_q)$, every element of
  $\mathscr{O}_1/ \mathscr{O}_g$ being nil potent. Therefore by lemma
  (\ref{part1:chap2:sec1:lem1}) there exist polynomials $X_n$ and
  $Y_n$ in $A(X)$ such that  
  $$
  f- g_{n-1}h_{n-1}\equiv Y_n g_{n-1}+ X_n h_{n-1} \pmod {\mathscr{O}_{n+1}}
  $$
  and $d(X_n)< s$.
\end{proof}

But by induction assumption $f-g_{n-1} h_{n-1}\equiv 0 \pmod
{\mathscr{O}_n}$ therefore $0 \equiv Y_n g_{n-1} + X_n h_{n-1} \pmod
{\mathscr{O}_n}$. Thus from the uniqueness part of lemma
(\ref{part1:chap2:sec1:lem1}) we get
$X_n \equiv 0 \pmod {\mathscr{O}_n}$ and $Y_n \equiv 0
(\mathscr{O}_n)$. We take $g_n = g_{n-1}+ X_n$ and $h_n = h_{n-1}+
Y_n$ obviously the polynomials $g_n$ and $h_n$ satisfy the conditions
$(\alpha), (\beta)$ and $(\gamma)$. Hence we get two sequences of
polynomials $(g_n)$ and $(h_n)$. The respective coefficients of $(g_n)$
and $(h_n)$ converge as $n$ tends to infinity because of the condition
$g_{n+1} \equiv g_n \pmod {\mathscr{O}_{n+1}}$ and $h_{n+1}\equiv h_n
\pmod{\mathscr{O}_{n+1}}$. Therefore
$\lim\limits_{n \to \infty} g_n =g$ is a monic polynomial of degree
$s$ and $\lim\limits_{n \to \infty}h_n = b$ is power series over $A$
which converges everywhere in $A$, because $h \equiv h_n \pmod
{\mathscr{O}_{n+1}}$. We see immediately that $f=gh$ $\bar{h}= \psi$
and $\bar{g}= \varphi$. Moreover $\lambda_n (h)= d (h_n)= \lambda_n
(f) -s$, because $h \equiv h_n \pmod {\mathscr{O}_{n+1}} \Longrightarrow
\lambda_{n+1} (h)=\lambda_{n+1}(h_n) =d (h_n)\leq \lambda_{n+1}(f) -s$ but $f= gh$
implies that $\lambda_{n+1}(f) \leq s+ \lambda_{n+1}(h)$, therefore\pageoriginale we
get our result. If $f$ is a polynomial then $\lambda_n (f)$ is
constant for $n$ sufficiently large implying $\lambda_n (h)$ is
constant for $n$ large, therefore $h$ is a polynomial. Since $g_n$ and
$h_n$ are strongly relatively prime modulo $\mathscr{O}_{n+1}$, there
exist by lemma (\ref{part1:chap2:sec1:lem1}) polynomials $a_n$ and
$b_n$ such that  
$$
\displaylines{\hfill 1 \equiv a_n g_n + b_n h_n \pmod
  {\mathscr{O}_{n+1}},  \hfill \cr
  \text{where}\hfill  d(b_n)< s \quad \text{and}\quad  d(a_n)< d(h_n)=
  \lambda_{n+1} (f)-s.\hfill }
$$ 
Similarly we have polynomials $a_{n+1}$ and $b_{n+1}$ such that 
$$
\displaylines{\hfill 
  1 \equiv a_{n+1} g_{n+1}+ b_{n+1} h_{n+1} \pmod
  {\mathscr{O}_{n+2}}\hfill \cr
  \text{where}\hfill d (b_{n+1})<s  ~\text{and}~ d(a_{n+1}) < d
  (h_{n+1})= \lambda_{n+2}  (f)-s.\hfill }  
$$
Combining these two we get
$$
(a_{n+1}-a_n)g_n + (b_{n+1}-b_n) h_n \equiv 0 \pmod {\mathscr{O}_{n+1}}
$$

Hence by uniqueness if lemma (\ref{part1:chap2:sec1:lem1}) we get $a_{n+1} \equiv a_n \pmod
{\mathscr{O}_{n+1}}$ and $b_{n+1} \equiv b_n \pmod
{\mathscr{O}_{n+1}}$. Since $d(b_n)< s$ for every $n$, we get that
$\lim\limits_{n \to \infty} b_n =b$ is a polynomial, moreover
$\lim\limits_{n \to \infty} a_n=a$  is everywhere convergent power
series in $A$; a is a polynomial if $f$ is a polynomial. Hence we get
$1 \equiv ag+bh \pmod{\mathscr{O}_{n+1}}$ for every $n \geq 1$, which
implies that $g$ and $h$ are strongly relatively prime in
$A\left[[X]\right]$.  

\medskip
\noindent \textbf{Uniqueness.} If possible let us suppose that there exists
another pair $(g',  h')$ satisfying the requirements of the lemma. Let
$V=h-h'$ and $U = g'-g$. Since $\bar{g}= \bar{g}'= \varphi$ and
$\bar{h}= \bar{h}'= \psi$, $U$ is in $\mathscr{O}[X]$ and $V$ is in
$\mathscr{O}_1[[X]]$. 

Let\pageoriginale us assume that $U$ belongs to $\mathscr{O}_n[X]$ and
$V$ belongs to $\mathscr{O}_n[[X]]$ for all $n< m, m>1$. We have  
$$ 
f=gh=g' h'= (U +g) (V +h)= UV +Uh+gV+gh
$$
which implies that $-UV = Uh + gV$. But $UV$ is in
$\mathscr{O}_{2n-2}[[X]]~ (2n -2> n, ~\text{as}~ n> 1)$, therefore 
$$
U h + gV \equiv 0 \pmod{\mathscr{O}_n} 
$$

Let $\rho_n$ be the canonical homomorphism from $\mathscr{O}_n A[[X]]$
onto $A/ \mathscr{O}_n$ $[[X]]$. Obviously we have 
\begin{equation*}
  \rho_n (U) \rho_n(h) + \rho_n (g) \rho_n (V)= 0, d(U)<
  s\tag{1}\label{part1:chap2:sec1:eq1} 
\end{equation*}

But $\rho_n (h)$ and $\rho_n (g)$ are strongly relatively prime in $A
\mathscr{O}_n [[X]]$, because they are so in $A/ \mathscr{O}_1[[X]]$
and $\mathscr{O}_1 / \mathscr{O}_n$ is contained in $r (A/
\mathscr{O}_n)$, therefore by uniqueness part of lemma
(\ref{part1:chap2:sec1:lem1}) we get from (\ref{part1:chap2:sec1:eq1}) 
$$
\rho_n(U) =0 ~\text{and}~ \rho_n (V) =0
$$

This means that $V = h-h' \equiv 0 \pmod {\mathscr{O}_n}$ and $U = g -
g' \equiv 0 \pmod {\mathscr{O}_n}$ for every $n$. But $\cap
\mathscr{O}_n =0$, because $A$ is a Hausdorff space, therefore $U=0$
and $V=0$. Hence the uniqueness of $g$ and $h$ is established. 

\begin{corollary}\label{part1:chap2:sec1:coro1}%coro
  Let $K$ be a complete filed for a real valuation $v$. Let $f(X) =
  \sum\limits_{n=0}^{\infty} a_n X^n$ be an every-where convergent
  powerseries with coefficient from $\mathscr{O}$. Let $\varphi$ and $
  \psi$ be two polynomials in $\mathscr{O}/ \mathscr{Y}[X]= k[X]$ such
  that 
  \begin{enumerate}[\rm(1)]
  \item $\varphi$\pageoriginale is monic of degree $s$.
  \item $\varphi$ and $\psi$ are strongly relatively prime in $k[X]$
  \item $\bar{f}$ (image of $f$ in $k[X]$)= $\varphi \psi$
  \end{enumerate}

  Then there exists one and only one pair $g$ and $h$ such that
  \begin{enumerate}[\rm(1)]
  \item $g \in \mathscr{O}[X]$, $g$ is monic of degree $s$ and
    $\bar{g}= \varphi$ 
  \item $h \in \mathscr{O}[X]$, $h$ converges everywhere in
    $\mathscr{O}$ and $\bar{h}= \psi$ 
  \item $f = g h $.
  \end{enumerate}
  and the radius of convergence of $h$ is the same as that of $f$. If
  $f$ is a polynomial, then $h$ is a polynomial. Moreover $g$ and $h$
  are strongly relatively prime. 
\end{corollary}

\begin{proof}%pro
  Suppose that $\varphi= \sum\limits_{r= o}^{s-1} \bar{a}_r X^r + X^s$
  and $\psi= \sum\limits_{u= 0}^{t} \bar{b}_u X^u$. 
\end{proof}

Let $\varphi_\circ= \sum\limits_{r=0}^{s-1} a_r X^r + X^s,
\psi_{\circ} u= \sum\limits_{u=0}^{t}\bar{a}_u X^u$. 

Obviously $\varphi_\circ$ is monic of degree $s$ and $\bar{f}=
\varphi_{\circ} \psi_{\circ}$, which implies that $f - \varphi_\circ
\psi_\circ$ belongs to $\mathscr{Y}[[X]]$ \iec  if $f - \varphi_\circ
\psi_\circ = \sum b_n X^n$, then $v(b_n)>0$ for every $n$. Let $\alpha
= \inf v(b_n), \alpha$ is obviously strictly positive. Let
$\mathscr{O}_1 =\{x^n / x\in \mathscr{O}, v(x) \geq \alpha\}$, then
$(\mathscr{O}_n)_{n>0}, \mathscr{O}_n = \mathscr{O}^n_1$ defines a
decreasing filtration on $\mathscr{O}$. Obviously
$\tilde{\varphi}_\circ$ and $\tilde{\psi}_\circ$ (images of $\varphi$
and $\psi$ in $\mathscr{O}/\mathscr{O}[X]$) are strongly relatively
prime modulo $\mathscr{O}_1$ and $\tilde{\varphi}_\circ$ and
$\tilde{\psi}_\circ$ satisfy all the requirements of Hensel's lemma,
therefore there exists one and only one pair $(g, h)$ such that 
\begin{enumerate}[(i)]
\item $g$\pageoriginale is monic polynomial of degree $s$ and $\tilde{g}=
  \tilde{\varphi}_\circ$ 
\item $h$ is an every where convergent powerseries in $\mathscr{O}$,
  $\tilde{h}= \psi_\circ$ and $\lambda_n (f) -s$. 
\item $f= g h$
\end{enumerate}

Form the choice of $\varphi_\circ$ and $\psi_\circ$ it is obvious that
this pair $(g,  h)$ satisfies the conditions $(1),(2)$ and $(3)$ of
the corollary. 

If possible let there exist another pair $(g',  h')$ satisfying the
conditions stated in the corollary. Let $g'' = g - g', h'' = h-h'$ Since
$\bar{g}''$ and $\bar{h}''$ are in $\mathscr{Y}[x]$, there exists
$\alpha' > 0$ such that $g''$ and $h''$ are in $\mathscr{O}'_1[[x]]$ where
$\mathscr{O}'_1=\{x | x \in \mathscr{O}, v(x) \geq \alpha'\}$. Let us
take in Hensel's lemma instead of $\mathscr{O}$ the ideal
$\mathscr{O}'_1$ and the filtration defined by $(\mathscr{O}_n)$
where $\mathscr{O}'_n=\mathscr{O}^n_1$. But then have two pairs $(g,
h)$ and $(g',  h')$ satisfying the conditions $(a),(b),(c)$ of the
lemma, which is not possible, therefore $g=g'$ and $h=h'$. 

If $f$ is a polynomial,  the result about radius of convergence is
obvious. Let us assume that $f$ is not a polynomial, then $\lambda_n
(f)$ tends to infinity as $n$ tends to infinity. It has  already been
proved that $t_f = \lim_i \inf \dfrac{v(a_i)}{i}$. Since $v$ is a real
valuation, for any $i$ we can find an integer $k$ such that$ (k-1)
\alpha \leq v(a_i) \leq k \alpha,  \Longrightarrow \lambda_k
(f)$. Therefore $\dfrac{v(a_i)}{i} \geq \dfrac{(k-1)\alpha}{\lambda_k
  (f)}$ and 
$$
\underset{i \to \infty}{\lim\inf} \frac{v(ai)}{i} \geq \underset{i \to
  \infty}{\lim\inf} 
\frac{k \alpha}{\lambda_k (f)} 
$$\pageoriginale
moreover for $i=\lambda_k (f)$ we have
$$
\frac{k \alpha}{\lambda_k (f)}\geq \frac{v(a_i)}{i}.
$$

Let $k \to \infty$, which implies that $i = \lambda_k (f) \to
\infty$. Then we get   
$$
\underset{i \to \infty}{\lim\inf}  \frac{k \alpha}{\lambda_k (f)} \geq
\frac{v(a_i)}{i} 
$$

Thus \qquad $\underset{i \to \infty}{\lim\inf} \dfrac{v (ai)}{i}=
\lim\limits_{n \to \infty} 
\dfrac{n \alpha}{\lambda_n (f)}= \lim\limits_{n \to \infty} \dfrac{n
  \alpha}{\lambda_n (h)}$  

This proves that $t_f = t_n$.

\begin{corollary}\label{part1:chap2:sec1:coro2}%coro
  Let $K$ be a complete valuated field with a real valuation $v$ and
  $f$ a polynomial in $\mathscr{O}[X]$. Then if $\alpha$ in $k$ is a
  simple root of $\bar{f}$, there exists one and only one element a in
  $\mathscr{O}$ such that a is a simple root of $f$ and $\bar{a}=
  \alpha$. 
\end{corollary}

\begin{proof}%pro
  Since $\alpha$ is a simple root of $\bar{f}$, we have $\bar{f}(X)=
  (X- \alpha)~ \psi (X)$ where $\psi (\alpha)=0$. Moreover $(X-\alpha)$
  and $\psi (X)$ are strongly relatively prime in $k[X], (X-\alpha)$
  being a prime element. Therefore form corollary 1, we have in one
  and only one way $f(X) = (X-a) h(X)$, where $\bar{h}= \psi$ and
  $\bar{a}= \alpha$. Moreover a is a simple root of $f$ because 
  $$
  \bar{h(a)}= \bar{h}(\alpha)= \psi (\alpha)=0.
  $$
  In particular if $K$ is a locally compact field such that
  characteristic $k \neq 2$, then we shall show that $K^*/ (K^*)^2$ is a
  group of order $4$. 
  
  $K$\pageoriginale locally compact implies that $v$ is discrete and $k$ is a
  finite. Let $\pi$ be a uniformising parameter and let $C \in
  \mathscr{O}^* = \mathscr{O}-\mathscr{Y}$ be an element such that
  $\bar{C}$ is not a square in $k$, such an element exists because $k^*
  /(k^*)^2$ is of order $2$. Then it can be seen that $1,C, \pi, C \pi$
  represent the distinct cosets in $K^*/ (K^*)^2$ and any element in
  $K^*$ is congruent to one  of them modulo $(K^*)^2$. 
\end{proof}

\section{Extension of Valuations - Transcendental case}\label{part1:chap2:sec2}

In order to prove that a valuation of a field can be extended to an
extension field it is sufficient to consider the following two cases: 
\begin{enumerate}[(i)]
\item When the extension field is an algebraic extension.
\item When the extension field is a purely transcendental extension.
\end{enumerate}

\begin{proposition}\label{part1:chap2:sec2:prop1}
  Let $L = K(X)$ be a purely transcendental extension of a field $K$
  with a valuation $v$, let $\Gamma'$ be any totally ordered group
  containing $\Gamma_v$. Then for $\xi$ in $\Gamma'$ there exists one
  and only one valuation $\omega_{\xi_n}$ of $L$ extending $v$ such that 
  $$
  w_\xi \left(\sum_{j=0}^{n} a_j x^j\right) = \inf\limits_{0 \leq j
    \leq n} \left(v  (a_j)+ j \xi\right) 
  $$ 
\end{proposition}

\begin{proof}%pro
  It is sufficient to verify that $w_\xi$ satisfies the axioms of a
  valuation for $K[X]$. The axioms $a_\xi (P) = \infty
  \Longleftrightarrow P=0$ and $w_\xi (P+Q) \geq \inf (w_\xi (P),
  w_\xi (Q))$ can be easily verified. To prove $w_\xi (PQ) = w_\xi
  (P)+ w_\xi (Q)$, where $P= \sum\limits_{j=0}^{n} a_j X^j$,  

  $Q = \sum\limits_{i=0}^{m} b_i X^i$ and $PQ \neq 0$, we write $P= P_1+
  P_2,Q= Q_1+Q_2, P_1$ being the\pageoriginale sum of all terms $a_j X^j$ of $P$ such
  that $w_\xi(P)=v(a_j)+j \xi$ and  
  $Q_1$ being the sum of those terms $b_iX^i $ of $Q$ for which
  $w_\xi(Q)=v(b_i) + i \xi$. Let $j_o$ and $k_o$ be the degree of $P_1$
  and $Q_1$ respectively. If $P_1 Q_1 = \sum C_r X^r$, then we have 
  \begin{align*} 
    w_\xi (P_1 Q_1) & =v(C_{j_o + k_o}) + \xi (j_o + k_o)\\
    & = v(a_{j_o})+\xi j_o + v(b_{k_o}) + \xi k_o = w_\xi (P_1) + w_\xi (Q_1).
  \end{align*}
  
  Now
  $$
  w_\xi (P Q) = w_\xi (P_1 Q_1 + P_1 Q_2 + Q_1 P_2 + P_2 Q_2) = w \xi
  (P_1 Q_1), 
  $$
  because the valuation of the other terms is greater than $w_\xi (P_1
  Q_1)$. 
\end{proof}

This implies that 
$$
w_\xi (P Q) = w_\xi (P_1 Q_1)= w_\xi (P_1) + w_\xi (Q_1) = w_\xi (P) +
w_\xi (Q).
$$
\begin{coro*}% coro 
  There exists one and only one valuation $w$ of $K(X)$ such that
  \begin{enumerate}[\rm(i)]
  \item $w$ extends $v$.
  \item $w(X) = 0$.
  \item The class $\bar{X}$ of $X$ in $k_w$ is transcendental over $k_v$.
  \end{enumerate}
  
  The valuation $w$ is the valuation $w_\xi$ for $\xi = 0$ and $k_w$
  is a purely transcendental extension of degree $1$ over $k_v$. 
\end{coro*}

It is obvious that $w_o$(i.e. $w_\xi$ for $\xi = 0)$ satisfies $(1)$
and $(2)$ and that $k_{w_o}= k_{v}(\bar{X}_{n})$.If $\bar{X}$ were
algebraic over $k_{v}$, then there exists a polynomial $\bar{P}(Y) =
\sum\limits^{n}_{j=0} \bar{a}_j Y_j$ such that at least one
$\bar{a}_j\neq 0$ and $\bar{P}(\bar{X})=0$, which means that
$P(X)=\sum\limits^n_{j = 0}a_j X^j$ is in $\mathscr{Y}_w$, where at
least\pageoriginale one $a_j$ is not in $\mathscr{Y}_v$ and all $a_j$ are in
$\mathscr{O}_v$. But this is impossible because $w(P(X))=\inf\limits_j
v(a_j)=0$. Conversely let $w$ be a valuation of $K(X)$ satisfying $1),
2), 3)$. Let $P=\sum\limits^m_{i=0}a_i X^i$ be a polynomial over
$K$. We have to prove that $w(P)=\inf v(a_i)$. Let
$P=\sum\limits^m_{i=0}a_i X^i$ be a polynomial over $K$. We can assume
without loss of generality that $a_i$ are in $\mathscr{Y}_v$ and at
least one of them is not in $\mathscr{Y}_y$, then $\inf\limits_i
v(a_i)=0$. If $w(P) > 0$, then $\bar{P}=0$ in $k_w$, which implies,
that $\bar{X}$ is algebraic over $k_v$, which is a contradiction. But
we know that 
$$
w\left(\sum\limits_i a_i X^i\right) \geq \inf\limits_i \{v(a_i) + iw(X)\}=0
$$
therefore $w(P)= \inf\limits_i  v(a_i)$.

\section{Residual Degree and Ramification Index}\label{part1:chap2:sec3} %\secc 3

Let $L$ be a field and $K$ a subfield of $L$. Let $w$ be a valuation
of $L$ and $v$ the restriction of $w$ on $K$. Since $\mathscr{Y}_w
\cap K = \mathscr{Y}_v$, the filed $k_v$ can be imbedded in the field
$k_w$. We shall say the dimension of $k_w$ over $k_v$ the \textit{residual
  degree of $w$ with respect to $v$} or of $L$ with respect to $K$. We
shall denote it by $f(w, v)$. 

The index of the group $\Gamma_v$ in $\Gamma_w$ is called the
\textit{ramification index of} $w$ with respect to $v$ or of $L$
with respect to $K$ and is denoted by $e(w, v)$. 

If no confusion is possible, we shall denote $f(w, y)$ by $f(L,K)$ and\pageoriginale
$e(w, v)$ by $e(L, K)$. 

If $e(w, v)= 1$, then $L$ is said to be an \textit{unramified
  extension} of $K$. 

If $f(w, v) = 1, L$ is said to be \textit{totally ramified} extension
of $K$. 

Since the group of values and residual field of $\hat{K}$ are the
same as that of $K$ we  have 
$$
e(\hat{L}, \hat{K}) = e (L, K) \text { and } f(\hat{L},
\hat{K})= f (L, K) 
$$

\begin{proposition}\label{part1:chap2:sec3:prop2}%\prop 2
  Let $L$ be a filed with a valuation $w$ and let $K$ be a field
  contained in $L$ and $v$ the restriction of $w$ on $K$. 
  Then $e(L, K) f(L, K)\leq (L:  K) =n$, where $(L:  K)$ is dimension
  of $L$ over $K$. 
\end{proposition}

\begin{proof}
  If $n$ is infinite, the result is trivial. Let us assume that $n$ is
  a finite number. Let $r \leq e$ and $s \leq f$ be two positive
  integers, then we can find $r$ elements $X_1, \ldots,  X_r$ in
  $L^\ast$ such that $w(X_i) \not\equiv w(X_j) \pmod {\Gamma v}$ for $i
  \neq j$ and $s$ elements $\bar{Y}_1, \ldots \bar{Y}_s$ in $k_w$ such
  that they are linearly independent over $k_v$. Let $Y_1, \ldots Y_s$
  be a system of representatives for $Y_1, \ldots Y_s$ in
  $\mathscr{O}^\ast_w$. Then the elements $(X_i Y_j, i=1, \ldots, r ;
  j=1, 2, \ldots s)$ are linearly independent over $K$. If they are
  not linearly independent, then there exists elements $a_{ij}$ in
  $K$ not all $0$ such that 
  $$
  \sum_{i, j}  a_{ij}  X_i  Y_j  = 0
  $$

  Let\pageoriginale $\alpha = \inf\limits_{i, j} w (a_{ij} X_i Y_j)$, obviously
  $\alpha$ is finite and belongs to $\Gamma w$. Therefore $w(a_{kl} X_k
  Y_l) = \alpha $ for some $k$ and $l$. We have 
  \begin{align*}
    w(a_{ij} X_i Y_j) & = w (a_{ij}) + w(X_i) + w(Y_j)\\
    & = w (a_{kl}) + w(X_k) + w(Y_l)\\
    \text{if}~ w(a_{ij } X_i Y_j) & = \alpha \text{ for some~ $i$ ~and~ $j$ }.
  \end{align*}
  But $w(Y_j) = w(Y_l) = 0$, therefore we get $w(X_i) \equiv w(X_k)
  \pmod {\Gamma_v}$, which is possible only if $i = k$. Thus we get 
  \begin{equation*}
    a_{kl} X_k Y_l + \sum\limits_{j \neq l} a_{kj} X_k Y_j \equiv 0
    \pmod {\mathscr{O}'} \tag{1}\label{part1:chap2:sec3:eq1}
  \end{equation*}
  where $\mathscr{O}' = \{X / X \in L, w(X) > \alpha \}$
  
  Multiplying the congruence (\ref{part1:chap2:sec3:eq1}) with
  $a^{-1}_{kl} X^{-1}_{k}$ we get 
  $$
  Y_l + \sum_{j \neq l} a^{-1}_{kl} a_{kj} Y_j \equiv 0 \pmod {\mathscr{Y}_w}.
  $$
  Therefore 
  $$
  \bar{Y}_l + \sum_{j \neq l} \overline{(a^{-1}_{kl}a_{kj})} \bar{Y}_j =
  0, \text { where } \overline{a^{-1}_{kl} a_{kj}} \text { are in }
  k_v. 
  $$
  But this is impossible, because $\bar{Y}_1 - - - \bar{Y}s$ are
  linearly independent over $k_v$, therefore $(X_i Y_j)$ are linearly
  independent over $K$. Since $(L:  K ) = n$, the number of linearly
  independent vectors in $L$ over $K$ cannot be greater than $n$. 
  
  Hence $e f \leq n$.
\end{proof}

\begin{corollary}\label{part1:chap2:sec3:coro1}% colo 1
  If $L$ is algebraic over $K$, then $k_w$ is algebraic over $k_v$ and
  $\Gamma_l / \Gamma _k$ is a torsion group of order $\leq (L:  K)$. 
\end{corollary}

The assertion is trivial when $(L:  K)$ is finite. When $(L:  K)$ is
infinite we can write $L = \bigcup\limits_{i \in I}L_i$ and $k_L = \cup
k_{Li}$, where $L_i$ is a finite algebraic extension of $K$. 

Then\pageoriginale $\Gamma_L / \Gamma_K$ is the union of the quotient groups
$\Gamma_{L_i} / \Gamma_K$ for $i$ in $I$ and therefore it is a torsion
group. 

\begin{corollary}\label{part1:chap2:sec3:coro2}%\coro 2
  Suppose that $L$ is algebraic over $K$, then $w$ is improper if and
  only if $v$ is improper. 
\end{corollary}

$v$ improper implies that $\Gamma _v = \{0\}$. Therefore by
corollary (\ref{part1:chap2:sec3:coro1}) $\Gamma_w$ is a torsion group. But $\Gamma_w$ is a
totally ordered and abelian group, therefore it consists of identity
only. 

\begin{corollary}\label{part1:chap2:sec3:coro3}%\coro 3
  Let $(L:  K)$ be finite. Then $w$ is discrete if and only $v$ is discrete.
\end{corollary}

$v$ discrete implies that $\Gamma_v$ is isomorphic to $Z$ and $(L: 
k)$ finite implies $\Gamma_w/\Gamma_v$ is of finite order. Moreover
$\Gamma_w$ is Archimedian, because if $\alpha$ and $\beta$ are in
$\Gamma_w$, then $n \alpha$ and $n \beta$ where n = order $\Gamma_w /
\Gamma _v$, are in $\Gamma_v$; therefore there exists an integer $q$
such that $qn \alpha > n \beta$, which shows that $q \alpha >
\beta$. There exists a smallest positive element in $\Gamma_w$. For,
each coset of $\Gamma_w / \Gamma _v$ has a smallest positive element,
the smallest among them is the smallest positive element fo
$\Gamma_w$. Hence $\Gamma_w$ is isomorphic to $z$. 

\begin{corollary}\label{part1:chap2:sec3:coro4}%coro 4
  If the valuation $v$ on $K$ is discrete, $K$ is complete and $
  (L:  K)$ is finite, then $ef = ( L:  K)$.
\end{corollary}

\begin{proof}
  Let $\pi$ be a uniformising parameter in $L$. Let $\bar{Y}_1, \ldots
 ,  \bar{Y}_f$ be a basis of $k_w$ over $k_v$ and $Y_1,\ldots,Y_f$
  their representatives in $\mathscr{O}^*_w$. Let $\mathscr{R}$ denote
  a system of representatives of $k_v$ in $\mathscr{O}^*_v$. 
\end{proof}

Then\pageoriginale any element $X$ in $\mathscr{O}_w$ can be written in the form
$\sum\limits^f_{i = 1} \alpha_i Y_i$ modulo $\mathscr{Y}_w$ with
$\alpha_i \in \mathscr{R}$ in one and only one way. Let $L'$ be vector
space over $K$ generated by $(Y_i \pi^j)$ for $i = 1, 2,\ldots,f$ and
$j=0, 1, 2,\ldots, e-1$. Since $L'$ is a finite dimensional
vectorspace over a complete field $K, L'$ is complete (for proof see
Espaces Vectoriels Topologiques by N. Bourbaki, chapter $I$ section
$2$) and therefore closed in $L$. But $L'$ is dense in $L$, because for
every element $X$ in $L$ and an integer $n$ there exists an element
$X_n$ is $L'$ such that $v(X-X_n) \geq n ~e$. For sufficiently small
$n$ the result is obviously true. Let us assume that it is true for
all integers $r \leq n$. Since n e is in $\Gamma_v$, there exists an
element $U$ in $K$ such that $w(U)=v(U)= n~ e$. Therefore
$U^{-1}(X-X_n)$ belongs to $\mathscr{O}_w$ and we have 
$$
U^{-1}(X-X_n) \equiv \sum_i \alpha_{io} Y_i \pmod {\mathscr{Y}_w
=\mathscr{O}_w \pi} 
$$
This means that $\pi^{-1}[(U^{-1}(X-X_n)-\sum_i \alpha _{io} Y_i]$
belongs to $\mathscr{O}_w$, therefore  
$$
\pi^{-1}\left[(U^{-1}(X-X_n)-\sum_i \alpha _{io} Y_i)\right] \equiv \sum_i
\alpha_{i1}Y_i \pmod {\mathscr{Y}_w} 
$$

Proceeding in this way we obtain  
$$
\displaylines{\hfill 
  U^{-1}(X-X_n)\equiv \sum_i \alpha _{io} Y_i + \cdots + \sum_i
  \alpha_{i e - 1} Y_i \pi^{e-1} \pmod {\mathscr{Y}^e_w} \hfill \cr
  \text{or}\hfill  (X-X_n) \equiv U\left[ \sum\limits^{e-1}_{j=0} \sum\limits_i
    \alpha_{ij} Y_i \pi^j\right] \pmod {\mathscr{Y}_w^{(n + 1)e}}\hfill} 
$$\pageoriginale
Let us take $X_{n + 1}= X_n + U \left[ \sum\limits^{e-1}_{j=0}
  \sum\limits_i \alpha_{ij} Y_i \pi^j\right]$. 

Then $w(X-X_{n+1}) \geq (n + 1) e$. Thus $L'$ is dense in $L$ and
therefore $L' = L$. So $n = (L:  K)\leq ef$. But we know that $ef \leq
n$, therefore $n = ef$. 

\section{Locally compact Fields}\label{part1:chap2:sec4}%\sec 4

\begin{proposition}\label{part1:chap2:sec4:prop3}%\prop 3
  If $K$ is a locally compact filed of characteristic $o$ with a
  discrete valuation $v$, then $K$ is a finite extension of $Q_p$
  where $p$ is characteristic of the residual field $k$. 
\end{proposition}

\begin{proof}
  Since characteristic $K = 0, K$ contains $Q$ the field of retinal
  numbers. We see immediately that $v$ is proper, because if $v$ is
  improper then $Q$ is contained in $k$ which is a finite field by
  theorem in \S\ 7.1 and this is impossible. The restriction of $v$ to
  $Q$ is $v_p$ for some $p$ because $p$ adic valuations are the only
  proper valuations on $Q$ and this $p$ is the characteristic of
  $k$. We have already proved in $\S7.1$ that $K$ is complete,
  therefore $K$ contains $Q_P$. The valuation $v$ on $K$ is discrete,
  therefore $\Gamma_v$ is isomorphic to $Z$,  but $\Gamma_{v_p}$ is
  also isomorphic to $Z$ and is contained in $\Gamma _v$, therefore $=
  (\Gamma_v:  \Gamma_{v_p})$ is finite. Moreover $f= (k_v:  k_{v_p})$ is
  also finite, because $k_v$ is a finite filed. Hence $(K:  Q_p)=e ~f$
  (by\pageoriginale corollary 4 of $\S 3.2)$ is finite. 
\end{proof}

\begin{proposition}\label{part1:chap2:sec4:prop4}%\prop 4
  Let $K$ be complete filed for a real proper valuation $v$ such that
  \begin{enumerate}[\rm(1)]
  \item characteristic $K$= characteristic $k$.
  \item $k$ and all its sub fields are perfect.
  \end{enumerate}
  Then there exists a subfield $F \subset \mathscr{O}$ which is a
  system of representatives of $k$ in $\mathscr{O}$. Moreover if $v$
  is discrete then $K$ is isomorphic to $k ((x))$. 
\end{proposition}

\begin{proof}
  Let $\Phi$ be the family of subfields $S$ of $\mathscr{O}$ such that
  the restriction of $\varphi$, the canonical homomorphism from
  $\mathscr{O}$ onto $k$ to $S$ is an isomorphism onto a subfield of
  $k$. The family $\Phi \neq \phi$, because the prime fields contained
  in $\mathscr{O}$ and $k$ are the same. Obviously $\Phi$ is
  inductively ordered by inclusion, therefore by Zorn's lemma it has a
  maximal element $F$. We shall prove that $k=\varphi(F)$. The field
  $k$ is algebraic over $\varphi(F)$. If possible let there exist an
  element $\bar{x}$ in $k$ transcendental over $\varphi(F)$. Let
  $\varphi(x) = \bar{x}$, where $x$ is in $\mathscr{O}$, then $x$ is
  transcendental over $F$. It is obvious that $F(x)$ is isomorphic to
  $\varphi(F) (\bar {x})$, which contradicts the maximality of $F$,
  therefore $k$ is algebraic over $\varphi(F)$. Suppose that
  $\varphi(F)$, then there exists one element $\bar{x}$ in $k$ and not
  in $\varphi(F)$. Since $\varphi(F)$ is a perfect field, $\bar{x}$ is
  a simple root of an irreducible monic polynomial $\bar{P}$ over
  $\varphi(F)$. Let 

  $\bar{P}=X^s + \bar{a}_{s-1} X^{s-1}+\cdots+ \bar{a}_o = (X-
  \bar{x})\bar{Q}$,\pageoriginale where $\bar{Q}$ is some polynomial over
  $\varphi(F)$ and $\bar{Q}(\bar{x})\neq 0$. 
  
  By Corollary (\ref{part1:chap2:sec3:coro2}) fo Hensel's lemma we obtain that the polynomial 
  $P= X^S +a _{s-1} X^{s-1} + \cdots+ a_0$ has a simple root $x$ in
  $\mathscr{O}$ such that $\varphi(x) = \bar{x}$ and $Q$ is an
  irreducible polynomial. Therefore $F(x)$ is isomorphic to
  $F[X]/(P)$. But $\varphi(F) (\bar{x})$ is isomorphic to $\varphi(F)
  [X]/ (\bar{P})$ therefore we see that $\varphi$ is still an
  isomorphism from $F(x)$ onto $\varphi(F) (\bar{x})$. But this is
  impossible, because $F$ is a maximal element of $\Phi$. Thus our
  theorem is proved. 
\end{proof}

When $v$ is discrete, we have seen that every element $x$ in $K$ is of
the form $\sum\limits^\infty_{i = m}r_i \pi^i$ with $r_i$ in $F$ and
conversely. Therefore the mapping $\sum\limits^\infty_{i = m}r_i \pi^i
\to \sum\limits^\infty_{i = m} \varphi(r_i) X^i$ is from $K$ onto
$k((x))$. It is trivial to see that it is an isomorphism. 

\begin{coro*}% corollary
  A non-discrete locally compact valuated field of characteristic $p >
  o$ is isomorphic to a field of formal power series over a finite
  field. 
\end{coro*}

Since we have already proved in $\S 7.1$ that a locally compact
valuated field $K$ is complete, its valuation is discrete and $k$ is
finite, our corollary follows from the theorem. 

\section[Extension of a Valuation to an Algebraic Extension..]{Extension of 
a Valuation to an Algebraic Extension  (Case of a Complete 
Field)}\label{part1:chap2:sec5} %\sec 5 

\setcounter{theorem}{0}
\begin{theorem}\label{part1:chap2:sec5:thm1}%\the I
  If\pageoriginale $L$ is an algebraic extension of a complete field $K$ with a real
  valuation $v$, then there exists one and only valuation $w$ on $L$
  extending $v$. 
\end{theorem}

\begin{proof}
  If $v$ is improper $w$ is necessarily improper. So we assume that
  $v$ is a proper valuation. Suppose that $L$ is a finite extension
  of $K$. If there exists a valuation $w$ on $L$ extending $v$, then
  $w$ is unique, because on $L$ any valuation defines the same
  topology as that of $K^{(L:  K)}$ and the topology on $L$ determines
  the valuation upto a constant factor and in this case the constant
  factor is also determined because the restriction of the valuation
  to $K$ is $v$. 
\end{proof} 

Let $L$ be a Galois extension of $K$. Then if $w$ is a valuation on
$L$ extending $v$, $w ~o~ \sigma$ for any $\sigma$ in $G(L/K)$ (the
Galois group of $L$ over $K$) is also a valuation extending
$v$. Therefore by uniqueness of the extension $w(x) = W o \sigma (x)$
for every $x$ in $L$. This shows that  
$$
\underset{L/K}{v \,(N (x))}= w ( \prod\limits_\sigma (x)) =
\sum\limits_\sigma w~ o~\sigma(x) = n~o~w(x) 
$$
where $(L:  K) = n$.
  
Thus
\begin{equation*}
  w(x) = \frac{1}{n} ~\underset{L/K}{v\,(N
    (x))}.\tag{1}\label{part1:chap2:sec5:eq1} 
\end{equation*}

Now\pageoriginale suppose that $L$ is any finite extension of $K$ of degree $n$. We
define a mapping $w$ on $L$ by (\ref{part1:chap2:sec5:eq1}) and prove that it satisfies the
axioms for a valuation. It is well known that (Bourbaki, algebra
chapter V, \S 8) that if $L$ is the separable closure of $K$ is $L$,
and if $p$ is the characteristic exponent of $K$ (\iec  $p=1$ if
characteristic $K = 0$ and p= characteristic $K \neq 0 $), then 
 $$
n= (L:  K) = q p^e
$$
with $q=(L':  K)$ and $p^e = (L:  L')$. Moreover $L$ is a purely
inseparable extension of $L'$, and for each K-isomorphism $\sigma_i\, (1
\leq i \leq q)$ of $L'$, in an algebraic closure $\Omega$ of $K$ there
exists one and only one $K$-isomorphism of $L$ which extends
$\sigma_i$. This extended isomorphism will also be denoted by
$\sigma_i$. Then 
$$
N_{L/K}(x) =\bigg[\prod\limits^{q}_{i=1} \sigma_i (x) \bigg]^{p^e}
$$
It is easy to prove that $w(x)=\infty$ if and only if $x=0$ and $w(xy)
= w(x) + w(y)$ for $x, y$ in $L$. To prove that $w(x+y) \geq \inf
(w(x), w(y))$, it is sufficient to prove that $w(\alpha ) \geq 0$
implies that $w(1 + \alpha ) \geq 0$ for any $\alpha $ in $L$. We know
that if $P(X) = X^r + a_{r-1}X^{r-1} + \cdots + a_o$ is the monic
irreducible polynomial of $\alpha$ over $K$, then $\underset{L / K} N
\alpha = (a_o)^{\frac{n}{r}}$ and $P(1-X)$ is the irreducible
polynomial of $1+\alpha $. Thus 
$$
\underset{L / K}{N} (1 + \alpha ) = (-1)^r \left\{(1 + a_{r-1} + \cdots +
(-1)^r a_o) \right\}^{\frac {n} {r} } = b_o 
$$

So\pageoriginale to prove our result we have to show that $b_o$ is in
$\mathscr{O}$ 
when $a_o$ is in $\mathscr{O}$, because
$w(\alpha)=\dfrac{v(a_o)^{n/r}}{n}$. This will follow from the
following theorem, which completely proves our theorem. 

\begin{theorem}\label{part1:chap2:sec5:thm2}%\theo 2
  Let $K$ be complete field with a real valuation $v$ and $x$ any
  element of an algebraic extension of $K$. If $f(X) =X^r
  +a_{r-1}X^{r-1}\cdots+a_o$ is the minimum polynomial of $x$ over
  $K$, then $a_o$ belonging to $\mathscr{O}$ implies that all the
  coefficients of $f(X)$ are in $\mathscr{O}$. 
\end{theorem}

\begin{proof}
  If possible suppose that all $a_i$ are not in $\mathscr{O}$, then
  $v(a_j) < 0$ for some $j, 0 < j < r$. Let $-\alpha = \inf v(a_j),
  \alpha > 0$ and $j$ the smallest index such that $v(a_j) =
  -\alpha$. We have $o < j < r$. Since $\alpha $ belongs to
  $\Gamma_v$, there exists an element $C$ in $K$ such $v(C) =
  \alpha$. Consider the polynomial $g= C f(X) = C X^k+\cdots +C_{a_j}
  X^j + \cdots + C_\circ a$. Because of the choice of $j, \bar{g}=
  \cdots+\bar{r}_j X^j$, where 
  $\bar{r}_j = \bar{ca}_j \neq 0$. Therefore $\bar{g}$ has $X^j$ as a
  factor which is a monic polynomial and if $\bar{g}=X^j \psi$, then
  $X^j$ and $\psi$ satisfy the requirements of Corollary
  (\ref{part1:chap2:sec3:coro2}) of
  Hensel's lemma, which gives that $g$ is reducible, which is a
  contradiction. Hence all $a_j$ are in $\mathscr{O}$. 
\end{proof}

When $L$ is infinite algebraic extension of $K$, we can express
$L=\bigcup\limits_{i \in I}L_i$ where each $L_i$ is a finite algebraic
extension of $K$ and the family $\{L_i\}_{i \in I}$ is a directed set
by the relation\pageoriginale of inclusion. We define the valuation $w$ for any $x$
in $L$ as $w(x)=w_i(x)$ if $x$ is in $L_i$ and $w_i$ is the extension
of $v$ on $L_i$. It is obvious that $w$ is the unique valuation on $L$
extending $v$.

\section{General Case}\label{part1:chap2:sec6}%\sec 6

We shall study now how a valuation of an incomplete field can be
extended to its algebraic extension. 

Let $K$ be field with a valuation $v$ and $L$ an algebraic extension
of $K$. If $w$ is a valuation of $L$ extending $v$, we can look at the
completion $\hat{L}$ of $L$. $\hat{L}$ contains $L$ and
$\hat{K}$, so it contains a well defined composite extension $M_w$
of $L$ and $\hat{K}$. Then there exist one and only one maximal
ideal $m_w$ in $L~ \underset{K}\otimes \hat{K}$ such that the
canonical mapping from $L~ \underset{K}\otimes \hat{K} \to M_w$
gives an isomorphism from $L~ \underset{K}\otimes \hat{K}/m_w$
onto $M_w$. So we get a map $\varphi$ from the set of the valuation
$w$  extending $v$ to the set of the maximal ideals of $L~
\underset{K}\otimes \hat{K}$. Conversely if we start from a
maximal ideal $\mathcal{M}$ in $L~ \underset{K}\otimes \hat{K}$,
then the corresponding composite extension $M=L \otimes \hat{K}/
\mathcal{M}$ is an algebraic extension of $K$ and there one and only
one valuation $w_M$ of $M$ which extends $v$ and the restriction of
$w_M$ to $L$ gives a valuation of $L$ extending $v$. So we get a map
$\psi$ from the set of the maximal ideals of $L~ \underset{K}\otimes
\hat{K}$ (or of the classes of complete extensions) to the set of
the valuations of $L$ extending $v$. 
 
Moreover\pageoriginale the completion $\hat{L}$ of $L$ with respect to $w_M$ is
exactly $\hat{M}$ and the composite extension of $L$ and $\hat{K}$
contained in $\hat{L}$ is $M$.So we have $\varphi \circ \psi =I$ (identity map) 

Now we have also $\psi \circ \varphi = I$, for if $w$ is any valuation of
$L$then the valuation $w_{M_w}$ is necessarily the same as $w$ by the
uniqueness of the extension to $M$ of the valuation $v$ of $\hat{K}$.  

Hence there exists a 1-1 correspondence between the set of valuations
on $L$ extending $v$ and the set of inequivalent composite extensions
of $L$ and $\hat{K}$. 

In particular if $(L:  K)=n<\infty$, then any composite extension of
$L$ and $\hat{K}$ is complete which means that $\hat{L} =L
\underset{K}{\otimes} K/ \mathcal{M}$, where $\mathcal{M}$ is some
maximal ideal of $L \underset{K} \otimes \hat{K}$. 

Suppose $L$ is an algebraic extension of an incomplete valuated field
$K$ with a valuation $v$. Let $(w_i)_{i \in I}$ be the set of
valuations on $L$ extending $v$. We shall denote by $L_i$ the field
$L$ with the valuation $w_i$, by $e_i$ the ramification index $e(L_i
:K) =e(\hat{L}_i:  \hat{K})$ by $f_i$ the residual degree $f
(\hat{L}_i: \hat{K})=f (L_i: K)$ and by $n_i$ the dimension of
$\hat{L}_i$ over $\hat{K}$. 

The sequence
$$
0 \to r (L \underset{K} \otimes \hat{K}) \to (L \underset{K} \otimes
\hat{K}) \to \prod_{i \in I } \hat{L}_i 
$$
is exact, because the radical of $ L \underset{K} \otimes \hat{K}$ is
the intersection of all the maximal ideals of $L \underset{K} \otimes
\hat{K}$, that is of all the kernels\pageoriginale of the map $L \underset{K}
\otimes \hat{K} \to L_i$. If the dimension of $L$ over $K$ is finite
we have the following result. 

\begin{theorem}\label{part1:chap2:sec6:thm3}
  If $L$ is a finite extension of degree $n$ of a field $K$ with a
  real valuation $v$, then there exist, only finitely many different
  valuations $(w_i)$ on $L$ extending $v$. Moreover we have $\sum n_i
  \le n $ and the sequence 
  $$
  0 \to r (L \underset{K} \otimes \hat{K}) \to L \underset{K} \otimes
  K \to \prod \hat{L}_i \to 0 
  $$
  is exact.
\end{theorem}

\begin{proof}
  We observe that $w_i$ is not equivalent to $w_j$ for $i \neq j$,
  because $w_i$ equivalent to $w_j$means that they differ by a
  constant factor and since their restriction to $K$ is same, we have
  $w_i =w_j$ 

  The number of different valuations $(w_i)$ on $L$ extending $v$ is
  finite because the number of inequivalent composite extensions of $L$
  and $\hat{K}$ is finite. To prove that the sequence is exact, we have
  to show that the mapping $\rho:  L \underset{K} \otimes \hat{K} \to
  \prod \hat{L}_i$ 
  is surjective.By the approximation theorem of valuations $\rho (L)$ is
  dense in $\prod \hat{L}_i$, therefore $\rho(L \underset{K} \otimes
  \hat{K})$ is dense in $\prod \hat{L}_i$, where $\rho$ is the canonical
  map from $L \to \prod \hat {L}_i$. But $\rho ( L \underset{K} \otimes
  \hat{K})$ is a finite dimensional vector space over $\hat{K}$
  therefore it is complete. Hence $\rho ( L \underset{K} \otimes
  \hat{K}) = \prod \hat{L}_i$ \iec  $\rho$ is onto. Obviously dim $
  \prod \hat{L}_i \le \dim L \underset{K}\otimes \hat{K} $ over
  $\hat{K}$,which means that $\sum n_i \le n$. 
\end{proof}

\begin{coro*}
  If\pageoriginale $\hat{K}$ or $L$ is separable over $K$, then we have $\sum n_i =n$.

  $\hat{K}$ or $L$ separable over $K$ implies that
  $r(\hat{K}\underset{K} \otimes L)=0$ (for proof see Algebre by
  $N$. Bourbaki chapter  8 section 7), therefore $\rho$ is an
  isomorphism. 
\end{coro*}

\section{Complete Algebraic Closure of a $p$-adic Field}\label{part1:chap2:sec7}

\begin{proposition}\label{part1:chap2:sec7:prop5}
  Let $K$ be a complete field with a real valuation $v$ and $\Omega$
  the algebraic closure of $K$.Then $\hat \Omega$ the completion of
  $\Omega$ by the valuation extending $v$ is algebraically closed. 
\end{proposition}

We shall denote the extended valuation also by $v$.

\noindent \textit{Proof.}
  To prove that $\hat{\Omega}$ is algebraically closed we have to show
  that any irreducible polynomial $f(X)$ in  $\hat{\Omega} [X]$ has a
  root in $\hat{\Omega}$. Without loss of generality we a can assume
  that $f(X)$ is in $\mathscr{O}_{\hat{\Omega}}[X]$ and leading
  coefficient of $f(X)$ is 1. Suppose that $f(X) = X^r + a_{r-1}
  X^{r-1} + \cdots +a_0$ then for every  integer $m$ there exists a
  polynomial $\varphi_m(X)=X^r +b^{(m)} X^{r-1}+\cdots + b_\circ^{(m)}$ in $
  \mathscr{O}_\Omega [X]$ such that for every $x$ in
  $\mathscr{O}^{r-1}_{\hat{\Omega}}$, $v(f(x) - \varphi_m (x)> rm $
  Let $\varphi_m(X)= \prod\limits^r_{j=1} (X-\alpha_{jm})$,
  $\alpha_{jm}$ are in $\mathscr{O}_\Omega $ as $\varphi_m(X)$ is in
  $\mathscr{O}_\Omega [X]$. Then 
  $$
  \displaylines{\hfill 
  \varphi_{m+1}(\alpha_{jm}) = \varphi_{m+1}(\alpha_{jm}) -f
  (\alpha_{jm})+f(\alpha_{jm})-\varphi_m(\alpha_{jm}) \hfill \cr
  \text{implies that}\hfill 
  v(\varphi_{m+1}(\alpha_{jm})) > \text{ rm }\hfill \cr
  \text{or}\hfill \sum \limits ^{r}_{t=1} v (\alpha_{jm}
  -\alpha_{tm+1})>  ~\text{rm}.\hfill \Box }
$$

Therefore there exists a root $\alpha_{tm+1}$ of $\varphi_{m+1}(X)$
such that  
$$
v(\alpha_{jm} - \alpha_{t_{m+1}})>m.
$$

So\pageoriginale we get a sequence $\bigg \{ \varphi_m(X)\bigg \}$ of polynomials
converging to $f$ and a sequence of elements $\bigg \{ \beta_m
\bigg\}$ converging to $\beta $ in $\hat{\Omega}$ and each $\beta_m$
is a root of $ \varphi_m (X)$. Since polynomials are continuous
functions, we have $\lim\limits_{m \to \infty}  f(\beta_m)= f(\beta)$ 

But $\lim\limits_{m \to \infty} f(\beta_m)=0$, because given integer
$N>0$, for $m>N$ we have $v(f(\beta_m))=v(f(\beta_m)-\varphi_m
(\beta_m))> rm >N$. 

Hence $\beta$ is a root of $f(X)$.

One can easily prove that the residual field of $\hat{\Omega}$ is the
algebraic closure of the residual field of $K$. In particular if $K =
Q_p$, then the residual field of $\hat\Omega$ \iec  $k_{\hat{\Omega}}$
is algebraic closure of $Z/(P)$.Thus 

$k_{\hat{\Omega}} =\cup F_i$, where each $F_i$ is a finite extension of $Z/(P)$

\section{Valuations of Non-Commutative Rings}\label{part1:chap2:sec8}

We define a valuation of a non-commutative ring $A$ without zero
divisors containing the unit element in the same way as of a
commutative ring. Almost all the results proved so far about valuated
can be carried over to division rings with valuations with obvious
modifications. WE mention the following facts far illustration. 

Let $L$ be a division ring with a valuation $v$.Then
\begin{enumerate}[(1)]
\item The set $\mathscr{O}_L = \bigg\{ x/x \in L,v(x) \ge 0 \bigg \}$
  is a non-commutative ring,which we call the valuation ring of $L$
  with respect to the valuation $v$. 
\item $\mathscr{Y}_L =\bigg\{ x/x \in L,v(x) > 0 \bigg \}$\pageoriginale is the
  unique two sided maximal ideal of $\mathscr{O}_L$. 
\item Any ideal in $\mathscr{O}_L$ is a two sided ideal. For, $v
  (x^{-1} y x)= -v(x)+v(y)+v(x)\ge 0$ for $x$ in $L$ and $y$ in
  $\mathscr{O}_L$ which means that $x^{-1} y x $ belongs to
  $\mathscr{O}_L$, therefore $yx=xz$ for some $z$ in
  $\mathscr{O}_L$. Hence $\mathscr{O}_L x = x \mathscr{O}_L$. 
\item The ideals of $\mathscr{O}_L$ are any one of the two kinds
  \begin{align*}
    I_\alpha & = \bigg\{ x|v(x)>\alpha \ge 0 \bigg\}\\
    I'_\alpha & = \bigg\{ x|v(x) \ge \alpha > 0 \bigg\}
  \end{align*}
\item The division ring $L$ is locally compact non-discrete division
  ring for the valuation $v$ if and only if $v$ is a discrete
  valuation, $L$ is complete and $ \mathscr{O} /\mathscr{Y}_L$ is
  finite 
\end{enumerate}

Regarding the extension of valuations to an extension division ring we
prove the following. 

\begin{theorem}\label{part1:chap2:sec8:thm4} % thm 4 
  Let $\widetilde{P}$ be a division algebra of finite rank over a
  complete valuated field $P$ with a valuation $v$ such that $P$ is
  contained  in the centre of $\widetilde{P}$. Then there exists one
  and only one valuation $w$ of $\widetilde{P}$ which extends $v$. 
\end{theorem}

\begin{proof}
  \textit{Existence} We define
  $ \underset{\widetilde{P}/P} N (x)$ = determinant of the endomorphism
  $$
  \rho_x y \to xy ~\text{of}~ \tilde{P}, ~\text{for any}~ x ~\text{
    in }~ \tilde{P}. 
  $$
  We shall prove that $w(x)= \dfrac{1}{r} v(\underset{\widetilde{P} |
    P}N (x)$ is a valuation of $\widetilde{P}$ if $r$ is the rank of
  $\widetilde{P}$ over $P$. The axioms $w(x)= \infty$ if and
  only\pageoriginale if 
  $x=0$ and $w(x y)= w(x)+w(y)$ are obviously true. 

To prove $w(x+y) \ge  \inf (w(x),w(y))$ it is sufficient to prove that
$w(x) \ge 0$ implies $ w(1+y)\ge 0$. Let $F=P(x)\, F$ is clearly a field
containing $P$ and $\widetilde{P}$ is a vector space over $F$ by left
multiplication. The mapping $\rho_x$ is an $F$ endomorphism. We know
that if $U$ is any $F$-endomorphism and $U_p$ the $P$-endomorphism
defined by $U$, then det $U_p = \underset {F/P} N$(det $U$) and we have
det $\rho _x =(x)^{(\tilde{P}: F)}$ if $\rho_x$ is considered as an
$F$-endomorphism. Therefore we have 
$$
w(x)=\frac{1}{r}(\widetilde{P}:F) =v (\underset{F/P}N(x)).
$$
\end{proof}

Now $w(x) \ge 0 \Longleftrightarrow v \underset{F/P}{v\,(N(x))} \ge 0
\Longrightarrow \underset{F/P}{v(N(1+x))} \ge 0$, because we have proved this
for commutative case.Hence $w$ is a valuation on $\tilde{P}$. 

\medskip
\noindent 
\textbf{Uniqueness.}  
Since $P$ is complete and $P$ is of finite rank $r$ over $P$, any
valuation defines the same topology on $\tilde{P}$ as that of
$P^r$. But the topology determines the valuation upto a constant
factor, If $w_1$ and $w_2$ are two valuations of $\widetilde{P}$
extending $v$ then $w_1 = C w_2$ for some $C$ in $P$.But restriction
of $w_1$ to $w_2$ to $P$ is $v$, therefore $C=1$ and $w_1 =w_2$. 
