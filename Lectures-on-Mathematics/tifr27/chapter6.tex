\chapter{Zeta-functions}\label{part3:chap2}

\section{}\label{part3:chap2:sec1}

It\pageoriginale is well known that the Riemann zeta function $\zeta (s) =
\prod\limits_{p} (1-p^{-s})^{-1}$, where $p$ runs over all prime
numbers, is absolutely convergent for Re $s > $. We can
generalise this definition for any commutative ring with unit element. 
In the case of ring of integers $p$ is nothing  but the generating
element of the  maximal ideal $(p)$ and it is also equal to the number
of elements in the field $Z/(p)$. Motivated by this we define  for any
commutative ring $A$ with identity 
\begin{equation} 
  \zeta_A (s) =\prod_{\mathcal{M}} (1-N(\mathcal{M})^{-s})^{-1}
  \tag{I}\label{part3:chap2:sec1:eqI} 
\end{equation}
where $\mathcal{M}$ runs over the set of all maximal ideals of $A$ and
$N(\mathcal{M})$ is the number of elements in the field $A/
\mathcal{M}$. But in general $N(\mathcal{M})$ is not finite and even
if $N(\mathcal{M})$ is finite the produce (\ref{part3:chap2:sec1:eqI}) is not convergent,
therefore we have to put some more restrictions on the ring.  In the
following we shall prove that if $A$ is finitely generated over $Z$
i.e., if there exist a finite number of elements $x_1, \ldots,  x_k$
in $A$ such that the homomorphism from $Z \bigg [ X_1, \ldots, 
  X_k\bigg]$ to $A$ which sends  $X_i$ to $x_i$ is surjective,  then
$N(\mathcal{M})$ is finite and the infinite product
(\ref{part3:chap2:sec1:eqI})  is
absolutely convergent fot Re $s > \dim A$, where the dimension
of $A$ is defined as follows. 

\begin{defi*}
  If $A$ is an integral domain,  the dimension  of $A$ is the
  transcendence degree (respectively transcendence degree $+1$) of the
  quotient field of  $A$ over $Z/(p)$ (respectively $Q$) if
  characteristic of $A$ is $p$ (respectively 0). In the general case
  $\dim A$ is the supremum of the dimension\pageoriginale of the rings $A/
  \mathscr{Y}$ where $\mathscr{Y}$ is any minimal prime ideal. 
\end{defi*}

It can be proved that dimension of $A$ is equal to the supremum of the
lengths of strict maximal chains of prime ideals. Before proving the
convergence of the zeta function we give some examples of finitely
generated rings of over $Z$. 
\begin{enumerate}
\item The ring $Z$ is finitely generated over itself. 
\item Any finite field $F_q$.
\item The ring of polynomials in a finite number of variables over
  $F_q$ i,e., the ring  $F_q [ X_1,  \ldots,  X_k ]$ 
\item The ring $F_q [ X_1,  \ldots,  X_r ] / \mathscr{U}$, where
  $\mathscr{U}$ is any prime ideal of  $F_q [ X_1$,  $\ldots,  X_r
]$. This is the set of regular functions defined over $F_q$ on the
  variety $V$ defined by the ideal $\mathscr{U}$ affine space. 
\item Let $K$ be any algebraic number field.  The ring  of integers
  $A$ in $K$ is finitely generated  over $Z$. 
\item Let $V$ be an affine variety defined over the algebraic number
  field $K$ and let  $\mathscr{O} \subset K [ X_1,  \ldots,  X_r ]$ be
  the ideal of $V$. Then the ring of regular functions on $V$ \iec  $K
  [ X_1,  \ldots,  X_r ] / \mathscr{O}$ is not finitely generated
  over $Z$. But the ideal $\mathscr{O}$ is generated by the ideal
  $\mathscr{O}_0 = \mathscr{O} \cap A [ X_1,  \ldots,  X_r ]$ of the
  ring  $A [ X_1,  \ldots,  X_r ]$ and we can associate to $V$ the
  quotient ring $A [ X_1,  \ldots,  X_r ] / \mathscr{O}$ which is
  obviously finitely generated over $Z$. It is to be noted that this
  ring is not intrinsic and depends on the choice of the coordinates
  in $K^r$ 
\end{enumerate}

\section{Fields of finite type over $Z$}\label{part3:chap2:sec2}

We shall require the following lemma in the course of our discussion. 
 
\noindent 
\textbf{Normalisation lemma of Noether.}\pageoriginale Let $K$ be a field. Let $R$
and $S$ be subrings of $K$ containing a unit elements such that $S$ is
finitely generated over $R$. Then there exists an elements $a \neq 0$
in $R$ and a finite number pf element $X_1, \ldots,  X_r$ in $S$ such
that 
\begin{enumerate}
\item $X_1,  \ldots,  X_r$ are algebraically independent over the
  quotient fields of $R$.  
\item  Any elements  of $S$ is integer over $R[ a^{-1}, X_1,  \ldots, 
  X_r ]$. 
\end{enumerate} 
 
\setcounter{proposition}{0}
\begin{proposition}\label{part3:chap2:sec2:prop1}
  Let $K$ be a field.  Let $R$ be a subring  of $K$ and $L$ the
  quotient field of $R$. If $K$ as a ring is finitely generated over
  $R$, then  $(K:L)$ is finite and there exists an element a in $R$
  such that  $L =R  [ a^{-1} ]$. 
 \end{proposition} 
 
 We first prove the following: If a field $K$ is integral over a
 subring $R$ then $R$ is a field. 
 
 Let $x$ be any element of $R$, then $x^{-1}$ belongs to $K$ and
 therefore satisfies an equation  
 $$
 X^n +a_1 X^{n-1}+ \cdots + a_n = 0, a_i \in R
 $$  
 
This implies that $x^{-1}$ is a polynomial in $x$ over $R$. But $R[
x]=R$, therefore $x^{-1}$ belongs to $R$. Hence $R$. Hence  $R$ is
a field 

\noindent \textbf{Proof of proposition 1.}
Since $K$ is finitely generated over $R$, by the normalisation
lemma,  there exists an element $a \neq 0$ in $R$ and a finite  family
$(x_1, \ldots,  x_r)$ in $K$ algebraically independent over $L$ such
that $K$ is integral over $R [ a^{-1}, x_1,  \ldots,  x_r ]$.  By the
remark above it follows that $R [ a^{-1}, x_1,  \ldots,  x_r ]$ is a field.  But
$x_1, \ldots,  x_r$ are algebraically independent over $L$, therefore
$r = 0$ and $L = R  [a^{-1} ]$. Since $K$ is finitely generated  and
integral over $L, (K:L)$ is finite.  

\begin{proposition}\label{part3:chap2:sec2:prop2}
  If a commutative ring $A$ is finitely generated over $Z$, then
  $W\mathfrak{N}(\mathfrak{M})$ is finite  for any maximal ideal
  $\mathfrak{M}$ of $A$.  
\end{proposition} 

\begin{proof}
  Since\pageoriginale $A$ is finitely generated over $Z$, the field  $K = A/
  \mathfrak{M}$ is finitely generated over  $Z$. If characteristic  of
  $K$ is zero  then $K$ contains $Z$. Therefore  by proposition
  (\ref{part3:chap2:sec2:prop1}) 
  $Q=Z(a^{-1})$ for some  $a \neq 0$ and  $a$ in $Z$, which is
  impossible.  Thus characteristic of $K$ is $p$ and by proposition
  (\ref{part3:chap2:sec2:prop1}) $K$ is a finite extension of $F_p$,
  hence $K$ is a finite field.   
\end{proof} 

\section{Convergence of the product}\label{part3:chap2:sec3}

\begin{proposition}\label{part3:chap2:sec3:prop3}
  The infinite product $\zeta_A(s)$ is a absolutely convergent for
  Re $s  > \dim A$ and uniformly convergent for Re $s > \dim  A+
  \varepsilon$ for every $\varepsilon > 0$. 
\end{proposition} 

\begin{proof}
We shall prove the result by induction on $r= \dim A$. If $r=0$ then
$A$ is a finite field.  Let us assume that $A= F_q$. Then  
$$
\zeta _A (s)= \frac{1}{1-q^{-s}}
$$   
is a meromorphic function in the plane with a simple pole at $s=0$. Let
us assume that the result is true for all those rings which are
finitely generated over $Z$ and dimension of which are less than
$r$. Before proving  the result for rings of dimension  $r$ we prove
the following result. 
\end{proof}

Let $A$ be a finitely generated ring over $Z$ and $B= A [ X ]$, the
ring of polynomials in one variable over $A$, then $\zeta_B(s)=
\zeta_A(s-1)$ in a suitable domain  of convergence.  In fact  if
$\zeta_A(s)$ is convergent for Re $s  > x$, then  $\zeta_B(s)$ is
convergent for Re $s > x+1$.  

If dim  $A= 0$, then  $A=F_q$ for some $q$ and  $B=F_q  [ X ]$. Since
he maximal ideals in $B$ are generated by irreducible polynomials,
which can be assumed to be monic,  we get 
$$
\zeta_B (s)= \prod_P (1-q^{sd(p)})^{-1}
$$  
where\pageoriginale $P$ runs over the set of monic irreducible polynomials over $A$. In
order to prove the absolute convergence of $\zeta_B(s)$, it is
sufficient to prove the convergence of the infinite series 
$$
S= \sum_P \bigg| q^{-d(P)} \bigg|^\sigma \text{ where } s= \sigma + it 
$$

Since the number of monic polynomials of degree $r$ is
$q^r$, we have   
\begin{align*}
  S= \sum_P |q^{-d(P)}|^\sigma & \le \sum^\infty_{r=1} q^r |q^{-r}|^\sigma\\
  &= \sum^\infty_{r=1}q(1-\sigma )r
\end{align*}

Obviously the series $S$ is convergent if $1-\sigma < 0$ \iec  $\sigma
> 1$.  
Moreover in this domain 
\begin{align*}
  \zeta_B(s) &= \sum_Q \frac{1}{q^{sd(Q)}} ~(Q \text{  a monic
    polynomial in} B)\\ 
  &= \sum_{k=0}^\infty \frac{q^k}{q^{sk}}= \sum_{k=0}^{\infty}
  \frac{1}{q^{k(s-1)}} =\frac{1}{1-q^{1-s}} 
\end{align*}
Hence 
$$
\zeta_B(s)= \zeta_A(s-1).
$$
Now let the dimension of $A$ be arbitrary and $B=A[ X ]$.

We shall denote by $\spm(B)$ the set of maximal ideals of $B$. For any
$\mathfrak{M}$ in $\spm(B), \mathfrak{M}\cap A$ is in $\spm (A)$,
because  $A/ \mathfrak{M} \cap A$, being a subring of the finite field
$B/\mathfrak{M}$, is a field.  Let  $\pi$ denote the mapping
$\mathfrak{M} \in \spm(B) \longrightarrow \mathfrak{M}\cap A \in
\spm(A)$. It can be easily proved that the set $\pi^{-1}\mathfrak{N}$
and $\spm(A/ \mathfrak{N}[X])$ are isomorphic,  where $\mathfrak{N}$ is
any maximal ideal of $A$. Therefore 
\begin{align*}
  \zeta_B(s) &= \prod_{\mathfrak{M}\in \spm (B)} [1- (N(\mathfrak{M}))^{-s}]^{-1}\\
  &= \prod_{\mathfrak{N}\in \spm (A)}  \prod_{\mathfrak{M}\in
    \pi^{-1}(\mathfrak{N})} ( 1-N(\mathfrak{M})^{-s})^{-1}\\ 
  &= \prod_{\mathfrak{N}\in \spm (A)} \zeta_{A/ \mathfrak{M}}[X]^{(s)}
\end{align*}\pageoriginale

But $A/ \mathfrak{N}$ is a finite  field,  therefore $\zeta_{A/
  \mathfrak{N}[X]^{(s)}} = \zeta_{A/ \mathfrak{N}}(s-1)$ 

So we get 
\begin{align*}
  \zeta_B(s) &= \prod_{\mathfrak{N}\in \spm (A)} (\zeta_{A/ \mathfrak{N}}(s-1))\\
  &=\prod_{\mathfrak{N}\in \spm (A)} (1-N(\mathfrak{N})^{1-s})^{-1}\\
  &=\zeta_A (s-1).
\end{align*}

It follows that $\zeta_{F_{q}} (s)_{[X_{1}, \ldots,  X_{k}]}=
\dfrac{1}{1-q^{k-s}}$ and $\zeta_{Z [X_1,  \ldots,  X_K]} = \zeta_Z
(k-s)$ where  $\zeta_Z$ is nothing but the Riemann zeta function. 

Now we shall prove our main proposition.  Assume that $A$ is an
integral domain. 

Let $K$ be the quotient field and $R$ the prime ring of $A$. 

Since $A$ is  finitely generated over $R$, by the normalisation lemma
we have the following: 

(1) If characteristic $A= p \neq 0$, then there  exist $r$ elements
$x_1, x_2, \ldots,\break  x_r$ in $A$ such that $A$ is integral over $R[x_1
 ,  \ldots,  x_r]$, where $x_1,  \ldots,  x_r$\pageoriginale are algebraically
independent over $R= F_p$. (ii)- If characteristic $A=0$, then there
exits an element a in $R =Z$ and $r-1$ elements  $x_1, \ldots, 
x_{r-1}$ in $A$ such that every  element of $A$ is integral over $Z [
  a^{-1}, x_1, \ldots x_{r-1} ]$  and the elements $x_1, \ldots, 
x_{r-1}$ are algebraically independent over $Q$.  

We get $r$ elements in the first case and  $r-1$ elements in the
second case because $r$ is the dimension  of $A$ which is equal to the
transcendence degree of $K$ over $F_p$ or transcendence degree of $K$
over $Q+1$ according as the characteristic of $A$ is non-zero or
not. It can be proved that $A$ (respectively $A' =A(a^{-1})$) is a
finite module over $B=F_p [x_1, \ldots,  x_r ]$ ( respectively $B' =Z
[a^{-1},x_1, \ldots,  x_{r-1} ]$) and the mapping  $\pi$ from
$\spm(A) \rightarrow \spm (B)$ (respectively from $\spm (A')
\rightarrow \spm (B')$) is onto. Let $A$ (respectively $A'$) be
generated by $k$ elements as a $B$ (respectively $B'$) module.  We
shall prove that 
$\pi^{-1}(\mathfrak{N})$ for any $\mathfrak{N}$ in $\spm(B)$ 
(respectively in $\spm (B'))$ has at most $k$ elements.  Let $C= A/A
\mathfrak{N}$. It is an algebra of rank $t \le k$ over  $B/
\mathfrak{N}$. Since $\pi^{-1}(\mathcal{M})$ is isomorphic to $\spm
(A/A \mathfrak{N})$ it is sufficient to prove that $C$ has at most $k$
maximal ideals. This will follow from the following. 

\begin{lemma*}[II]
  Let $A$ be any commutative ring with identity and  $(\mathcal{U}_{i_{1
      \le i \le m}}$ a finite set of prime ideals in $A$ such that 
  $$
  A= \mathcal{U}_i + \mathcal{U}_j  \text{ for } i \neq j
  $$ 
  Then the mapping  $\theta:A \to P = _i \prod\limits^m_{i=1} A
  \mathcal{U}_i$ is surjective 
\end{lemma*}

\begin{proof}
  It\pageoriginale is sufficient to prove that $1=\sum \limits_{i=1}^m a_i$ where
  $a_i$ belongs to $\mathcal{U}_j$ for $j \neq i$ because if $(t_1,\ldots,t_m )$
  is any element of $P$, then 

  $\theta\left(\sum \limits_{i=1}^{m}t'_i a_i\right)=(t_1,\ldots,t_m)$, where
  $t'_i$ is a representative of $t_i$ in $A$. 
  
  If $m=2$, the result is obvious \iec  $1=a_1+a_2$ where $a_1$ is in
  $\mathscr{O}_2$ and $a_2$ is in $\mathscr{O}_1$. Let us assume that
  it is true for less than $m$ ideals. 

  Then $1=\sum \limits_{i=1}^{m-1}v_i$ where $v_i \in \mathscr{O}_j$
  for $1 \leq j \leq m-1$ and $j \neq i$. Since
  $A=\mathscr{O}_i+\mathscr{O}_m$, we have $1=x_i + y_i$ for $1\leq i
  \leq m-1$ with $x_i \in \mathscr{O}_m$ and $\mathscr{Y}_i \in
  \mathscr{O}_i$. Clearly $\sum \limits _{i=1}^{m-1}x_i v_i+ \sum
  \limits _{i=1}^{m-1}v_i y_i=1$. 

  Let us take $u_i=v_i x_i$ for $i \leq i \leq m-1$ and $u_m=\sum
  \limits^{m-1}_{i~1} y_i v_i$, then $\sum \limits
  ^{m}_{i=1}u_i=1$ and $u_i \in \mathscr{O}_j$ for $j \neq i$. 
\end{proof}

Let $\mathfrak{M}_1,\mathfrak{M}_2,\ldots, \mathfrak{M}_t$ be any finite
set of distinct maximal ideals of $C$. Then by lemma
(\ref{part3:chap1:sec3:lem1}) $C/_i
\bigcap\limits^{t}_{i = 1} \mathfrak{M}_i$ is isomorphic to $
\overset{t}{\underset{i=1}{\oplus}} C/\mathfrak{M}_i (\oplus$ indicates
the direct sum). Thus $t \leq k$. 

Assume that the characteristic of $A$ is $0$. Let $\mathfrak{M}$ be any
maximal ideals of $A$. If a does not belong to $\mathfrak{M}$, then
$\mathfrak{M}A[a^{-1}]$ is a maximal ideal in $A[a^{-1}]$, because
$A[a^{-1}]/ \mathfrak{M}A[a^{-1}]$ is isomorphic to $A/\mathfrak{M}$. If
a belongs to $\mathfrak{M}$, then $\mathfrak{M}$ contains one and only
one prime $P_i$ occurring in the unique factorisation of $a$ and the
set of maximal ideals which contains $p_i$. is isomorphic to
$\spm (A/p_i A)$. Therefore if
$a=p_1^{\alpha_1},\ldots, p_t^{\alpha_t}$, then 
$$
\zeta_A (s) =\zeta_{A[a^{-1}]} (s) \prod_{i=1}^{t} \underset{A/p_i
  A}\zeta(s) 
$$

But\pageoriginale $\dim A/p_i A< \dim A$, therefore inorder to prove the convergence
of $\zeta_{A}(s)$ it is sufficient to consider
$\zeta_{A[a^{-1}]}(s)$. We have 
$$
\zeta_{A [ a^{-1} ] } (s) =\prod_{\mathfrak{N} \spm(B')} \prod
_{\mathfrak{N}\in^{-1}\pi(\mathfrak{N})} (1-(N \mathfrak{M})^{-s})^{-1} 
$$

Since $N(\mathcal{M})\geq N(\mathcal{M})$, we get
$$
\sum_{\mathfrak{N} \in \spm (B)} \sum_{\mathfrak{N}\in
  \pi^{-1}(\mathcal{M})}|N \mathcal{M}|^{-\sigma}\leq k
\sum_{\mathfrak{N}\in \spm(B')}|N \mathfrak{N}|^{-\sigma} \leq k
\zeta_z (r-\sigma-1) 
$$
Therefore $\zeta_{A[a^{-1}]}(s)$ is convergent for $\Re s > \dim A$.

If characteristic $A=p$, then we get
$$
\sum_{\mathfrak{N} \in \spm (B)} \sum_{\mathfrak{M}\in
  \pi^{-1}(\mathfrak{N})} |N \mathcal{M}|^{-\sigma} \leq k
\sum_{\mathfrak{N}\in \spm (B')} |N \mathfrak{N}|^{-\sigma} \leq k
\zeta_{F_p}(r-s) 
$$
which gives the same result as above, Now we have to prove our theorem
in the general case ($A$ is not an integral domain). But we shall prove
in the next $\S$ a more general result. 

\section{Zeta Function of a Prescheme}\label{part3:chap2:sec4}

Let $A$ be a commutative ring with unity. We shall denote by
$\Sp \,(A)$ the set of all prime ideals of $A$. On $\Sp \,(A)$ we define a
topology by classifying the sets $F(\mathscr{O})$ as closed sets,
where 
$$
F(\mathscr{O})=\{ \mathscr{Y} | \mathscr{Y} \supset \mathscr{O},
\mathscr{Y} \in \Sp \,(A) \}. 
$$
and $\mathscr{O}$ is any ideal in $A$. This topology is referred to as
the Jacobson Zariski topology. It is obvious that in this topology a
point is closed if and only if it is a maximal ideal of $A$. We
associate with every point $\mathscr{Y}$ of $\Sp (A)$ a local ring
$A_{\mathscr{y}}$ namely
the ring of quotient of $A$ with respect to the multiplicatively
closed set $A-\mathscr{Y}$. On $\mathscr{O}$ the sum of all these\pageoriginale
local rings we define a sheaf structure by giving``sufficiently many''
sections. For any $a, b, \in, A$ we consider the open subset 
$$
V(b)=\{\mathscr{Y}|\mathscr{Y} \in \Sp (A), \mathscr{Y} \notni b \}.
$$

For any $\mathscr{Y} \in v(b), \left(\dfrac{a}{b}\right)_{\mathscr{Y}}$ the,
fraction $\dfrac{a}{b}$, is an element of $A_\mathscr{Y}$. Then the
mapping $\mathscr{Y} \to \left(\dfrac{a}{b}\right)_{\mathscr{Y}}$ gives a section
$S(a, b)$ of $\mathscr{O}$. The pair $(X, \mathscr{O})$ together with
the sheaf of local rings $\mathscr{O}$ is called an \textit{affine
  scheme}, where $X=\Sp (A)$.  

\begin{defi*}
  Let $(X, \mathscr{O})$ be a ringed space. We say that $X$ is a
  prescheme if every point has an open neighbourhood which is
  isomorphic as a ringed space to $\Sp (A)$ for some ring $A$. Such a
  neighbourhood is called an affine neighbourhood. 
\end{defi*}

We shall assume that the pre-scheme $X$ satisfies the ascending chain
condition for open sets, then $X$ is quasi-compact and it can be
written as the union of a finite number of affine open sets $X_i$. We
shall denote by $A_i$ the ring such that $X_i$ is isomorphic to
$\Sp (A_i)$. Then the ring $A_i$ is Noetherian and has a finite number
of minimal prime ideals $\mathscr{Y}_{ij}$. Each prime ideal of $A_i$
contains a $\mathscr{Y}_{ij}$ and $X_i=\Sp (A_i)$ is the union of the
$s_{ij}=\Sp (A_{ij})$ (with $A_{ij})=A_i/\mathscr{Y}_{ij}$), each
$S_{ij}$ being a closed subset of $X_i$ and the $A_{ij}$ being
integral domains. Moreover the residue field of the local ring
associated to a point $x \in S_{ij}$ is the same for the sheaf of the
scheme $X$ and for the sheaf of the scheme $\Sp (A_{ij})$ 

We define the dimension of $X$ as the maximum of the dimensions of the
rings $A_i$(or of the rings $A_{ij}$). It can be proved that if $X$ is
irreducible (i.e. if $X$ cannot be represented as union of two proper
closed subsets). then $A_i=\dim A_j$ for $i \neq j$. 

A prescheme $S$ is a finite type over $Z$ if there exists a
decomposition of $S$ into a union of a finite number of open affine
sets $X_i$ such\pageoriginale that each $A_i$, the ring associated to $X_i$, is
finitely generated over $Z$. It can be proved that the same is true
for any decomposition into a finite number of affine open sets. In
particular, a ring $A$ is finitely generated over $Z$ if and only if
the scheme $\Sp (A)$ is of finite tyte over $Z$ and an open prescheme
of $S$ is also of finite type over $Z$. 

Let $S$ be a prescheme of finite type over $Z$. A point $x \in S$ is
\textit{closed} if and only if the residue field of the local ring
of $x$ is \textit{finite} (we shall denote by $N(x)$) the number of
elements of this field). In particular, if $S=U X_i'$, then a point $x
\in X_i$ is closed in $S$ if and only if it is closed in $X_i$ Now we
define the $\zeta$-function of $S$ by: 
$$
\zeta_S (s)=\prod (1-(N(x))^{-s})^{-1}
$$
where $x$ runs over the set of closed points of $S$. It is clear that
if $S=S_P(A)$, then $\zeta_S=\zeta_A$. As above, we can write $S$ as a
union of a finite number of subsets $S_i$, each $S_i$ being affine
open subset, with $S_i=\Sp (A_i)$, where $A_i$ is an integral domain
finitely generated over $Z$. Then it is obvious that: 
\begin{equation}
  \zeta_S=\frac{\left(\prod\limits_{\pi}\zeta_{S_i}\right)
    \left(\prod\limits_{i<j<k}\zeta_{s_i \cap S_j \cap
      S_k}\right)\cdots)}{\left(\prod\limits_{i<j} \zeta_{S_i \cap 
      S_j}\right)\cdots} \tag{I}\label{part3:chap2:sec4:eqI} 
\end{equation}
Now we shall prove the following generalisation of the
Theorem \ref{part3:chap1:sec3:thm1} 
bis:- \textit{ The } $\zeta$ \textit{ function of a prescheme } $S$
\textit{of finite type over } $Z$ \textit{is convergent for Re}
$s>\dim S$. 

Of course, theorem \ref{part3:chap1:sec3:thm1} bis implies theorem
\ref{part3:chap1:sec3:thm1}. Assume we have proved 
the theorem 1 bis for prescheme of dimension $<\dim S$. Then we get
as in the preceding $\S$ the convergence of $\zeta _A$ for any
integral domain $A$ finitely generated over $Z$ of dimension $\leq
\dim S$, and in particular the convergence of\pageoriginale the
$\mathcal{Z}_{S_i}$. After (\ref{part3:chap2:sec4:eqI}), we have just
to prove this: if 
$U$ (\resp $F$) is an open (\resp closed) subset of $X=\Sp (A)$ (with
$\dim A \leq \dim S$), 
then $\zeta_{U \cap F}$ is convergent for Re$(s)> \dim S$. But let
$G=X-U$; we have: 
$$
\zeta_{\cup \cap F}=\zeta_F/ \zeta _{F \cap G}
$$
and $F\cap G$ is closed in $X$. Hence we have just to prove the
convergence of $\zeta_F$. But $F$ is defined by an ideal of $A$ and
$F=\Sp (A/ \mathscr{O})$ and $\zeta_F=\zeta_{A/\mathscr{O}}$. If
$\mathscr{O}=\{0\}$, we have $\zeta_F=\zeta_A$ and if $\mathscr{O}\neq
\{0\}$ then the minimal prime ideals of $A/\mathscr{O}$ give non
trivial prime ideals of $A$ and we have $\dim A/ \mathscr{O}<\dim S:$
the induction hypothesis ensures the convergence of $\zeta_F$. Hence
we have completely proved the theorems $1$ and $1$ bis 

\section{Zeta Function of a Prescheme over $F_P$}\label{part3:chap2:sec5}

Let $S$ be a prescheme over $Z$ of finite type. We have a canonical
map from a prescheme $S$ to Sp$(Z)$ given by $\pi(x)=$ characteristics
of the residue field of local ring of $x$ for any $x$ in $S$. Suppose
that $\pi(x)=p$ for every $x$ in $S$. In this case each $A_i$ is of
characteristic $p$ and the canonical map from $Z$ into $A_i$ can be
factored through $F_P$. In this case we say that the prescheme $S$ is
over $F_P$. 

Let $S$ be a prescheme of finite type over $F_P$. Then the residue
field $k(x)$ of the local ring associated to a closed point $x$ is of
characteristic $P$ for every $x$ in $S$. Therefore
$k(x)=F_{P^{d(x)}}$ where $d(x)$ is a strictly positive
integer. thus 
$$
\zeta_S(s)=\prod_{\overline{x}=x \in S}\left(1-p^{-sd(x)^{-1}}\right)
$$

Let us take $t=p^{-s}$. Then
$$
\zeta_S(s)=\prod_{\overline{x}=x \in
  S}\left(1-t^{d(x)^-1} \right)=\overline{\zeta}_S(t). 
$$

The\pageoriginale function $\tilde{\zeta_s}(t)$ is also called a zeta function on
$S$. It is absolutely convergent in the disc $|t|<P^{-\dim(s)}$. we
have 
$$
\tilde{\zeta}_s(t)=\prod_{\overline{x}=x \in S} \sum ^{\infty}_{k=0}
t^{kd(x)}=\sum_{h=0}^{\infty}a_h t^h 
$$
with $a_0=1$ and $a_n \in Z$. The end of these lectures will be
devoted to the proof of the following theorem (Dwork's theorem): 

\setcounter{theorem}{0}
\begin{theorem}\label{part3:chap2:sec5:thm1}
  The function $\tilde{\zeta}_S(t)$ of a prescheme $S$ of finite type
  over $F_P$ is a rational function of $t$. 
\end{theorem}

\section{Zeta Function of a Prescheme over $F_q$}\label{part3:chap2:sec6}

In order to prove Dwork's theorem it is sufficient to prove it for an
affine scheme and open sets of an affine scheme because of the
equation (\ref{part3:chap1:sec4:eq1}). Then we have to look at the
zeta function of a ring $A$ 
finitely generated over $F_P$. Such a ring can be considered as the
quotient of $F_P[X_1,\ldots, X_k ]$ by some ideal $\mathscr{O}$ and we
can associate to $A$ the variety $V$ defined by $\mathscr{O}$ in $K^k$
where $K$ is the algebraic closure of $F_P$. It may be noted that $V$
is not necessary irreducible. We shall call $\zeta _A$ the zeta
function of the variety $V$. 

More generally we consider a variety $V$ over $F_q$, where
$q=p^{f}$. The variety $V$ is completely determined by the ring 

$A=F_q[X_1,\ldots X_n] / \mathscr{O}\cap F_q[X_1,\ldots, X_n]$ where
$\mathscr{O}$ is an ideal in $K[X_1$, $\ldots,X_n]$ generated by
$\mathscr{O}_0=F_q[X_1,\ldots,X_n ]\cap \mathscr{O}~ K$ being the
algebraic closure of $F_q$. We define 
$$
\zeta_v=\zeta_A \text{ and } \tilde{\zeta_v}=\tilde{\zeta_A}.
$$

For\pageoriginale every maximal ideal $\mathfrak{M}$ of $F_q[X_1,\ldots X_n]$ there
exists a maximal ideal $\mathfrak{M}$ in $K[X_1,\ldots, X_n]$ such that
$F_q[X_1, \ldots,X_n]\cap \mathfrak{M}'=\mathfrak{M}$. But $\spm (K[X_1,
  \ldots.X_n])$ is isomorphic to $K^n$, therefore a maximal ideal
$\mathfrak{M}$ of $F_q[X_1,\ldots,X_n]$ is determined by one point $x$
of $K^n$. Moreover this point $x$ belongs to $V$ if and only if
$\mathfrak{M} \supset \mathscr{O}$ However this correspondence between the
maximal ideals of $F_q[X]$ and the points of $K^n$ is not one-one. So
we want to find the condition when two points $x$ and $y$ of $K^n$
correspond to the same maximal ideal of
$F_q[X_1,\ldots,X_n]=F_q[X]$. Let $\mathfrak{M}_x$ and $\mathfrak{M}_y$
be the maximal ideals of $K[X]$ corresponding to $x=(x_1,\ldots, x_n)$
and $y=(y_1,\ldots,y_n)$ respectively such that $\mathfrak{M}_x \cap
F_q[x]=\mathfrak{M}_y \cap F_q[x]$. It is obvious that
$F_q[X]/\mathfrak{M}_x \cap F_q[x]=F[x]/\mathfrak{M}_y \cap F_q[x]$ is
isomorphic to $F_q[x_1,\ldots,x_n]=F_{q^f}$ for some $f>0$. We shall
show that the necessary and sufficient condition that 

$\mathfrak{M}_x \cap F_q[X]=\mathfrak{M}_y\cap F_q[X]$ is that there
exists an element $\sigma$ in $G(F_{q^f}/F_q)$ such that $\sigma
(x)=y$. For $n=1$ the existence of $\sigma $ is trivial. Let us assume
that there exists a $\sigma$ in $G(F_q f/F_q)$ such that $\sigma
(x_i)=y_i$ for $i=1,2,\ldots, r-1$ for $\leq n$. Let $\sigma(x_j)=z_j$
for $j\geq r$. Let $P(x)$ be the  polynomial of $z_r$ over
$F_q(y_1,\ldots,y_{r-1})$. Then $P(y_1,\ldots, y_{r-1} z_r)=0$, which
gives on applying $\sigma$ the equation
$P(x_1,\ldots,x_{r-1},y_r)=0$. Therefore $P$ is in $\mathfrak{M}_y
\cap F_q [X]$ \iec $P(y_1, \ldots y_{r-1}, y_r)=0$. 
Thus $y_r$ and $z_r$ are conjugate over $F_q(y_1,\ldots
,y_{r-1})$. Let $\tau $ be the automorphism of $K$ over
$F_q(y_1,\ldots,y_{r-1})$ such that $\tau (a_r)=y_r$. Then $\tau o
\sigma $ is an element of $G(F_{q^f}/F_q)$ such that $\tau o \sigma (x_i)
= y_i $ for $i = 1,2,\ldots,r$. Our result follows by induction.\pageoriginale The
converse is trivial. Hence we see that if $\mathfrak{M}$ is a maximal
ideal of $F_q[X]$ containing $\mathscr{O}$ with $N(\mathfrak{M}) =
q^f$, then there exist exactly $f$ points conjugate over $F_q$, in
$K^n\cap V$ and $f = (F_q(x):F_q)$ if and only if $f$ is the smallest
integer such that $x$ belongs to $(F_{q^f})^n$. Let 
\begin{align*}
  N_f & = \text{ number of points in } V\cap (F_{q^f})^n\\
  J_f & = \text{ number of points in } V\cap
  ({F_{q^f}}_{q^f})^{n}-\underset{f'<f}{U}(V \cap (F_{q^f})^n)\\ 
  I_f & = \text{ number of maximal ideals of $A$ of norm $q^f$}.
\end{align*}

We have proved that $J_f = fI_f$. By definition of the $\zeta$-
function of $V$ we have  
\begin{align*}
  \zeta_V(s) = \zeta_A(s) & = \prod_{\mathfrak{M}\in
    \spm(A)}(1-(n\mathfrak{M})^{-s})^{-1}\\ 
  & = \prod_{\mathfrak{M}\in \spm(A)}(1-q^{-s} f(\mathfrak{M}))^{-1}
\end{align*}
where $f(\mathfrak{M})$ is defined by the equation $N(\mathfrak{M}) =
q^{f(\mathfrak{M})}$  So we see that we can substitute $t=q^{-s}$ in
the zeta function (and not only $t=p^{-s}$ as in the general case) and
get a new zeta function. 
$$
\zeta_V (s)= \prod_{\mathfrak{M}\in \spm(A)}(1-t^{f(\mathfrak{M})})^{-1}
= \prod^{\infty}_{f=1} (1-t^f)^{-I_f} = \tilde{\zeta}_{v,q}(t) 
$$

Therefore
\begin{align*}
  \text{Log} \quad \tilde{\zeta}_{v,q} (t) & =
  \sum^{\infty}_{f=1}-I_f \log(1-t^f)\\ 
  & = \sum_{f=q}^{\infty} \sum_{k=1}^{\infty} I_f \frac{t^{kf}}{k}\\
  & = \sum_f \sum_k \frac{J_f}{f} \frac{t^{kf}}{k}\\
  & =\sum_{n=1}^{\infty}\left(\sum_{f/n}J_f\right)\frac{t^n}{n}\\
  & = \sum_{n=1}^{\infty} N_n \frac{t^n}{n}
\end{align*}\pageoriginale

Thus $\tilde{\zeta}_{v,q} (t) \exp \left(\sum\limits^{\infty}_{n=1} N_n
\dfrac{t^n}{n}\right)$, where $N_n$ is the number of points of $V$ in
$F^n_q$. We have already seen that this is a power series with
integral coefficients. 

\begin{theorem*}[$1'$]
  $\tilde{\zeta}_{V,q}(t)$ is a rational function of $t$.
\end{theorem*}

We shall show that in order to prove the rationality of
$\tilde{\zeta}_A(t)$ where $t = q^{-s}$, it is sufficient to prove the
rationality of $\tilde{\zeta}_A(t)$ where $t = p^{-s}$. Since
$\tilde{\zeta}_{v,q}(t)$ and $\tilde{\zeta}_v(t)$ are both convergent
in a neighbourhood of the origin, we have 
$$
\tilde{\zeta}_{v,q}(t^f) = \tilde{\zeta}_v(t) ~\text{with}~ q= p^f.
$$

Let $\mu $ be any $f$-th root of unity. Then
$$
\tilde{\zeta}_v(\mu t) = \tilde{\zeta}_{v,q}(\mu^{f} {t^f}) =
\tilde{\zeta}_{v,q}(t^f) = \tilde{\zeta}_v(t) 
$$
If we have
$$
\tilde{\zeta}_v(t)=\frac{\sum_{k=0}^{n}b_k t^k}{\sum _{k=0}^{n}C_k t^k}
$$
then also
\begin{align*}
  \tilde{\zeta}_v(t) &= \frac{\sum_\mu(\sum_{k=0}^{n}b_k \mu^k
    t^k}{\sum_{\mu} \sum_{k=0}^{n}C_k \mu^k t^k}\\ 
  & =\frac{\sum_{k=0}^n b_k(\sum _\mu
    \mu^k)t^k}{\sum_{k=0}^{n}C_k(\sum_\mu \mu^k)t^k}\\ 
  & =\frac{\sum_{0\leq k\leq [n/f]}b_{kf}t^{kf}}{\sum_{0\leq
      k\leq[n/f]}C_{kf}t^{kf}} 
\end{align*}\pageoriginale
because $\sum \limits _{\mu} \mu ^k =0 $ if $k \nequiv 0 \pmod f$

Thus we get
$$
\displaylines{\hfill 
  \tilde{\zeta}_v(t)=\frac{\sum\limits_{0 \leq k \leq
      [n/f]}b_{kf}t^{kf}}{\sum\limits _{0 
      \leq k \leq [n/f]}C_kf t^{kf}}=\tilde{\zeta}_{v, q}(t^f) \hfill \cr
  \text{\iec} \hfill \tilde{\zeta}_{V,q}(t)=\frac{\sum\limits_{0 \leq k
    \leq[n/f]}b_{kf}t^k}{\sum\limits_{0 \leq k \leq
      [n/f]}C_{kf}t^k}\hspace{2.5cm}\hfill }  
$$

Hence $\tilde{\zeta_{V, q}(t)}$ is a rational function of $t$.

\section{Reduction to a Hyper-Surface}\label{part3:chap2:sec7}

We shall show that to prove our theorem it is sufficient to consider
the zeta function of a hypersurface $V$ defined by a polynomial
$P(X_1, \ldots,X_n)$ in $F_P[X_1,\ldots,X_n]$. We know that we can
write $V=\bigcap \limits ^{r}_{i=1}V_i$ where each $V_i$ is a hyper
surface. Let $E$ be any subset of $\{1,2.\ldots,r\}$ and $V_E=\bigcap
\limits_{i \epsilon E}^{i}V_i$. Let $N_V$(respectively $N_{V_E}$) be
the number of points of $V$(respectively $V_E$) in any field
$F_{P^n}$. We now prove that 
\begin{equation}
  N_V=\sum_E(-1)^1+n(E)N_{V_E} \tag{I}\label{part3:chap2:sec7:eqI}
\end{equation}
where $n(E) $ is the number of elements in $E$.

Let any point $x$ in $V$ belong to $k$ hypersurface $V_i$ where $1
\leq k \leq r$. Then $x$ appears $l$ times in the right hand side of
equation (\ref{part3:chap2:sec7:eqI}), where 
\begin{align*}
  I & ={}^{r-k}C_0 {}^kC_1-\left({}^{r-k}C_0 {}^kC_2 +{}^{r-k}C_1\right)+
  \cdots +(-1)^{s+1}\\ 
  & \qquad \left({}^k
  C_s + {}^{r-k} C_1 {}^kC_{s-1}+ \cdots+ {}^k C_h
  {}^{r-k}C_{s-h}+\cdots\right)+\cdots \\ 
  & =\sum_{t=0}^{\infty} {r-k}_{C_t} \left[ \sum^{\infty}_{h=1}
    (-1)^{h+t-1}C_h^k\right]\\ 
  & =\sum^{\infty}_{t=0}(-1)^{t-1} ~ {}^{r-k}C_t
\end{align*}\pageoriginale

Thus $I=0$ or $1$ according as $r<k$ or $r=k$. Hence the equality
(\ref{part3:chap2:sec7:eqI}) is established. This proves that 
\begin{equation}
  \tilde{\zeta}_V(t)=\prod_E [\tilde{\zeta}_{V_E}(t)]^{(-1)^{1+n(E)}}
  \tag{2}\label{part3:chap2:sec7:eq2} 
\end{equation}
This proves that it is enough to prove theorem
\ref{part3:chap2:sec5:thm1} for a hypersurface. 

Let $V$ be a hypersurface defined by the polynomial $P(X_1,X_2,\ldots
X_n)$ in $F_P[X_1,\ldots X_n]$. Let $B$ be any subset of $\{1,2,\ldots
,n\}$. 
Let
\begin{align*}
  W_B& =\{x |x\in V, x_i=0 ~\text{ for }~ i ~\text{not in }~ B \}\\
  U_B& =\left\{x|x \in W_B, \prod_{i \in B} x_i=0 \right\}
\end{align*}

It is obvious that $V$ is union of disjoint subsets $W_B-U_B$ where
$B$ runs over all the subsets of $\{1,2,\ldots,n\}$. Hence the zeta
function of $V$ is the product of the zeta functions of the varieties
$(W_B-U_B)$ and the theorem \ref{part3:chap2:sec5:thm1} will be a
consequences of the following lemma.

\setcounter{Lemma}{1} 
\begin{Lemma}\label{part3:chap2:sec7:lem2}
  Let $P$ be a polynomial in $F_P[X_1,\ldots,X_n]$. then the zeta
  function of the open subset defined by $\prod
  \limits^{n}_{i=1}x_i=0$ in the hyper surface $W$ defined by $P$ is a
  rational function. 
\end{Lemma}

\section{Computation of $N_r$}\label{part3:chap2:sec8}

We\pageoriginale shall adhere to the following notation throughout our discussion.
\begin{align*}
  x & =(x_1.\ldots,x_{n+1}), x_i\in F_{P^r}.\\
  \alpha & = (\alpha_1, \ldots, \alpha_{n + 1}), \alpha_i \in Z.\\
  x^{\alpha} & = x_1^{\alpha_1}\cdots x_{n+1}^{\alpha_{n+1}}\\
  |\alpha| &= \alpha_1+\alpha_2+\cdots+\alpha_{n+1}.
\end{align*}

Let $\mathscr{X}$ be any additive character of $F_{p^r}$. Then we have 
\begin{align*}
  \sum_{U \in F_{P^r}}\mathscr{X}(UP(X_1,\ldots,X_n)) & =0   \text { if
  } P(x_1,\ldots,x_n) \neq 0\\ 
  & =p^r \text { if } P(x_1, \ldots, x_n )=0\\
\end{align*}
Therefore
$$
\displaylines{\hfill 
\sum _{x_1 \in F^*_{p^r}} \sum_{U \in F_{p^r}}
\mathscr{X}(UP(x_1,\ldots, X_n))=p^rN_r \hfill \cr
\text{where}\hfill 
p^rN_r=(p^r-1)^n+ \sum_{x \in(F^*_{P^r})}n+1
\mathscr{X}(x_{n+1}P(x_1,\ldots,x_n)) \hfill }
$$
Let
$$
x_{n+1}P(x_1,\ldots, x_n)=\sum_{\alpha}a_{\alpha} x^{\alpha}
$$
where only a finite number of $a_{\alpha}$ are nonzero. Then
$$
\displaylines{\hfill 
  \mathscr{X}(x_{n+1}P)=\prod_{\alpha}\mathscr{X}(a_{\alpha}x^{\alpha}).\hfill
  \cr 
  \text{Therefore}\hfill 
  p^rN_r=(p^r-1)^n+\sum_{x \in
    (F^*_{p^r})^{n+1}}\prod_{\alpha}\mathscr{X}(a_\alpha,\ldots,x^\alpha)
  \hfill }
$$

We\pageoriginale take the character $\mathscr{X}$ defined by $\mathscr{X}(t)=\prod
\limits_{k=0}^{r-1}\varphi(t'^{p^k})$. where $t' \in R_r$ such
that $\overline{t'}=t$ and $\varphi(y)=F(\zeta-1,y),\zeta$ being a
primitive p-th root of unity. Thus from equation
(\ref{part3:chap1:sec4:eq1}) we get
$$
p^rN_r=(p^r-1)^n+ \sum_{x \in (F^*_{P^r})^{n+1}} \prod_{\alpha}
\prod_{k=0}^{r-1}\varphi (b_{\alpha}\xi^{\alpha})^{p^k} 
$$ 
where $\overline{\xi_i}=x_i,\xi_i \in \mathscr{R}_r^*,
\overline{b}_{\alpha}=a_{\alpha}$ and $b_{\alpha}$ belongs to
$\mathscr{R}_1$. Let 
$$
G(\S)=\prod \varphi (b_{\alpha} \S^{\alpha}) \text{ and }
G_r(\S)=\prod_{k\quad 0}^{r-1} G(\xi^{p^k}). 
$$

Then
$$
p^r N_r = (p^r -1)^n + \sum_{\xi \in (\mathscr{R}_r^*)^{n+1}} G_r(\xi)  
$$
 
We have already proved that $G(\xi)$ is analytic for $\xi$
integral. Therefore  
$$
G_r(\xi) = \sum_{\alpha \in Z^{n+1}} g_{r\alpha} \xi^{\alpha}
$$
Then
\begin{align*}
  p^r N_r & = (p^r -1)^n + \sum_{\xi \in (\mathscr{R}_r^*)^{n+1}} G(\xi)\\
  & = (p^r -1)^n + \sum_{\alpha \in Z^{n+1}} g_{r \alpha} \sum_{\xi
    \in (\mathscr{R}_r^*)^{n+1}} \xi^{\alpha}\\ 
  & = (p^r-1)^n + \sum_{\alpha \in Z^{n+1}} g_{r_\alpha}
  \prod^{n+1}_{i=1}\left(\sum_i \xi^{\alpha_i}_{i}\right)  
\end{align*}

But $\sum\limits_{i} \xi_{i}^{\alpha_i}=0$ if $\alpha_i \not \equiv 0
\pmod {p^r-1}$ 
$$
= p^r -1 ~\text{if}~ \alpha_i \equiv 0 \pmod {p^r-1}
$$

Therefore
\begin{align*}
  p^r N_r &= (p^r-1)^n +  \sum_{\alpha = (p^r-1)} g_{r_{\alpha}} (p^r-1)^{n+1}\\
  &= (p^r-1)^n +  \sum_{\alpha} g_{p^r \alpha - \alpha } (p^r
  -1)^{n+1}  \tag{I}\label{part3:chap2:sec8:eqI} 
\end{align*}\pageoriginale

\section{Trace and Determinant of certain Infinite
  Matrices}\label{part3:chap2:sec9} 

Let  $K$ be any field and  $ A = K \bigg[ [ X_1, \ldots , X_{n+1}]\bigg]$ be the ring of formal power series in $n+1$ 
variables over $K$. Let $ H= \sum\limits_{\alpha} h_{\alpha} X^{\alpha}$ by any element of $A$. We define  an operator $T_H$ on $A$ as
follows   
$$
T_H (H') =  H H' ~\text{ for every }~ H' in A.
$$

For any integer $r$ we  define  an operator $\lambda_{r}$ Such that 
$$
\lambda_r \left(\sum_\alpha a_\alpha  X^\alpha\right) = \sum_{\alpha} a_{r
  \alpha} X^\alpha. 
$$

It can be easily proved that these two operators are continuous for
the topology given by the valuation on $A$ defined earlier. Let us set
$\Gamma_{H, r} = \lambda_r \circ T_H$. It is obvious that the  monomials
constitute a topological basis of $A$ and the operator $\Gamma _{H,r}$
has a matrix $(\gamma _{\alpha \beta})$ with respect to this basis,
where $\gamma_{\alpha \beta} = h_{r \alpha -\beta}$. It is trivial to
see that $T_{H H'}= T_H \circ T_H'$ for any two elements $H'$ and $H'$  of
$A$ and $\lambda_{r r'} = \lambda_r \circ \lambda_r'$ for any two integers
$r$ and $r'$. Moreover we have  
$$
\Gamma^{s}_{H,r} = \lambda_{r^{S}} \circ T_{H\cdot  H^{(r)} \cdots \cdot H^{r^{S-1}}}
$$
where $H^{(r)} (X) = H(X^r)$.

In order to prove the above identity it is sufficient to prove that
the action of the two sides is the same on the monomials. We have  

$T_H 0 \lambda_r (X^\beta) = 0$ if $\beta$ is not a multiple of $r$
\begin{align*}
  T_H \circ \lambda_r (X^\beta) &= T_H (X^{\frac{\beta}{r}}) \text{ if }
  \beta ~\text{is a multiple of }~r\\ 
  &= \sum h_\alpha X^{\alpha + \frac{\beta}{r}} = \sum h_{\alpha}X ~
  \frac{r \alpha + \beta}{r} 
\end{align*}\pageoriginale
with the convention that coefficient of $X^{\dfrac{r \alpha +
    \beta}{r}} = 0$ if $r$ does not divide $\beta$. 

Therefore 
\begin{align*}
  T_H \circ \lambda _r (X^\beta ) &= \lambda _r \left(\sum_\alpha h_\alpha X^{
    r \alpha}\right) X^\beta )\\ 
  &= \lambda _r \circ T_H^{(r)}
\end{align*}
Thus
\begin{align*}
  \Gamma ^2_{H,r} &= \lambda _r \circ T_H \circ \lambda _r \circ T_H \\
  &= \lambda _r  \circ \lambda _r  \circ T_{H^{(r)}} \circ T_H \\
  &= \lambda _{r^2} \circ T_{H, H^{(r)}}
\end{align*} 

Let us assume that we have proved that 
$$
\Gamma^{s}_{H,r} = \lambda _{r^S} \circ  T_{H \circ H^{(r)} \circ
  \ldots \circ H^{(r^{s-1})}} 
$$
Then
\begin{align*}
  \Gamma ^{s+1}_{H,r} &= \Gamma^{s}_{H,r} \circ \Gamma _{H,r} =
  \lambda_{r^S} T_{H \circ H^{(2)}} \dots \circ H^{r^{S-1}} \circ \lambda _r^\circ
  T_H\\ 
  &=  \lambda _{r^S} T_{H \circ} T_{H^{(2)}} T_{H (r^{S-1})} \circ
  \lambda _r \circ T_H\\ 
  &= \lambda_{r^{S+1}} T_H \circ T_{H^{(r)}} \circ \dots \circ T_{H^{(rs)}}
\end{align*} 
We see immediately that $\Gamma^s_{H,r}$ is an operator of the same
type as $\Gamma_{H,r}$ namely $\Gamma^s _{H,r} = \Gamma^s _{H' ~ \circ ~
  r'}$ where $r' = r^s$ and $H' = H ~ H^{(r)}\dots H^{(r^{S-1})} $. 

\begin{Lemma}\label{part3:chap2:sec9:lem3}%lemma 3.
  Let\pageoriginale us assume that $K = \Omega$ the complete algebraic closure of
  $Q_P $ and $r= p^f$. Let us further assume that the coefficients
  $h_\alpha$  tend  to $0$ as $|\alpha |$ tends to infinity. Then the
  series $\Tr (\Gamma^s_{H,r}) = \sum \limits_{\alpha} (\Gamma^s_{H,r})
    _{\alpha \alpha}$ giving the trace of $\Gamma^s _{H,r}$ with
  respect to the basis $(X^\alpha)$  is convergent and we have 
  $$
  \Tr (\Gamma^s_{H,r}) = \frac{1}{(r^{s} - 1)^{n+1}} = \sum_{\xi \in
    (\mathscr{R}^*_{fs})^{n+1}} H(\xi) \dots \dots \dots
  H \left(\xi^{r^{S-1}}\right) 
  $$
\end{Lemma} 

\begin{proof}%proof
  For any monomial $X^\beta$ in $K \big[ [X_1,\ldots,  X_{n + 1}] \big]$
  \begin{align*}
    \Gamma_{H,r} (X^\beta) &= \lambda_r \circ \sum_{\alpha} h_\alpha
    X^{\alpha + \beta} \\ 
    &= \sum_\alpha h_{\alpha r} X^{\alpha+ \beta}
  \end{align*}

  Therefore the matrix of the operator $\Gamma_{H,r}$ with respect to
  the basis $(X^\beta)$ is $(\gamma_{\alpha \beta})$
  with $\gamma_{\alpha \beta} = h_{r \alpha -\beta} $ and $T_r
  (\Gamma_{H,r}) = \sum \limits_{\alpha} h_{r \alpha-\alpha} $. 
  But $h_\alpha$ tends to $0$ as $\mid \alpha \mid $ tends to
  infinity, therefore the series $\sum \limits_{\alpha} h_{r \alpha -
    \alpha} $ is convergent in $K$. We have already proved that  
  $$
  \sum_{\substack{r-1\\ \rho = 1}} H (\rho) = (r-1)^{n+1} \sum_{\alpha
    \geq  0} h_{r \alpha - \alpha}  
  $$
  Therefore
  $$
  T_r (\Gamma_{H,r}) = \frac{1}{(r-1)^{n+1}}
  \sum_{\rho^{r-1}=1} H (\rho) 
  $$
  Hence our lemma is proved for $s = 1$ for $s > 1$,
  $\Gamma^{s}_{H,r}$ is of the same type as $\Gamma_{H,r}$.Thus our
  lemma is completely established. 
 \end{proof} 
 
 \begin{coro*}
   $p^s N_s = (p^s-1)^n +(p^s-1)^{n+1} \Tr \Gamma^s$\pageoriginale where $\Gamma =
   \Gamma_{G,p}$ we have already proved that 
   $$
   p^s N_s =(p^s-1)^n + \sum_{\xi \in (R^*_s)^{n+1}} \prod^{s-1}_{k=0}
   G(\xi^{p^{k}}) 
   $$

   Therefore the corollary follows from the lemma.
 \end{coro*} 
 
\section{Meromorphic character of $\xi_V (t) ~\textit{in}~
  \Omega$}\label{part3:chap2:sec10}  
 
 We have seen that 
\begin{gather*}
 \tilde{\zeta}_V (t) = \exp \left( \sum^{\infty}_{s=1} N_s
 \frac{t^s}{s}\right), \qquad  \text{where}\\ 
  N_s = \sum^{n}_{i=0} \binom{n}{i} (-1)^{n-i} p^{s(i-1)}+ \sum^{n+1}_{i=0}
  \binom{n+1}{i} (-1)^{n+1-i} p^{s(i-1)} \Tr \Gamma^s 
\end{gather*}
  
Therefore
\begin{align*}
  \sum^{\infty}_{s=1} \frac{N_s t^s}{s} &= \sum^{\infty}_{s=1}
  \sum^{n}_{i=0} (-1)^{n-i} \binom{n}{i} \frac{(p^{i-1}t)^s}{s}
  \\ 
  &=\sum^{\infty}_{s=1} \sum^{n+1}_{i=0} (-1)^{n+1-i}
  \binom{n+1}{i}  \frac{(p^{i-1}t)^s}{s} \Tr \Gamma^s \\ 
  \exp \left( \sum^{\infty}_{s=1} \frac{N_s t^s}{s}\right) &= \prod^{n}_{i=0}
  \exp \left( \sum^{\infty}_{s=1} \left[\frac{(p^{i-1}t)^s}{s}
    \right]\right){(-1)^{n+1-i}} \binom{n+1}{i}\\  
  &=  \prod^{n+1}_{i=0} \exp \sum^{\infty}_{s=1}
  \left[\frac{(p^{i-1}t)^s}{s} \Tr \Gamma^s \right]^{(-1)^{n+1-i}}
  \binom{n+1}{i}\\ 
  &=  \prod^{n}_{i=0} (1-p^{i-1}t)^{(-1)^{n-i}} \binom{n}{i})
  \prod^{n+1}_{i=0} \Delta (p^{i-1}t)^{(-1)^{n-i}} \binom{n+1}{i}) 
\end{align*}  
where $\Delta (t) = \exp  \left( -\sum \limits^{\infty}_{s=1}
\dfrac{t^s}{s} \Tr \Gamma^s \right) $   
 
 So\pageoriginale in order to prove that $\tilde{\zeta}_v (t)$ is meromorphic in $\Omega$, it
 is sufficient to prove that $\Delta (t)$  is every where  convergent
 in $\Omega$. 
 
 If $\Gamma$ were a finite matrix, then its trace is well defined. If
 the order of the matrix is $N$, then 
 $$
 \Tr \Gamma^s = \sum^{N}_{i=1} \lambda^s_i \text{are the eigen values
   of} ~\Gamma. 
 $$
 
Moreover
\begin{align*}
  \Delta (t) &= \exp \left(- \sum^{N}_{i=1}  \sum^{\infty}_{s=1}
  \frac{t^s}{s} \lambda^s_i \right) = \prod^{N}_{i=1} (1-t \lambda_i)\\ 
  &= \det (I-t \Gamma )
\end{align*}  

If $\Gamma$ is an infinite matrix, we define $\det (I-t \Gamma) = \sum
\limits^{\infty}_{m = 0} d_m t^m$, where 
  $$
  d_m =(-1)^m \sum_{1 \leq i_1 < < i_m}
  \sum_{\sigma}\varepsilon_{\sigma}\gamma_{i_{1}}
  \gamma_{i_{\sigma_{(1)}}}  \ldots \gamma_{i_{m}} i_{\sigma_{(m)}} 
  $$
 $\varepsilon _{\sigma}$  being the signature of any permutation
  $\sigma$ in $s_m$. 
 
 Then for $\Gamma = \Gamma_{G,p}$ we get
 $$
 d_m = (-1)^m \sum_{\alpha_i}, \sum_{1 \leq i \leq m} ~
 \sum_{\sigma \in s_m} \varepsilon_{\sigma} \gamma_{\alpha_1 \alpha_{\sigma (1)}}
 \ldots \gamma_{\alpha_{m} \alpha _{\sigma }(m)}  \alpha_i ~\text{being distinct.} 
 $$
 
 Let us assume that there exists a constant $M$ such that $v
 (g_{\alpha}) \geq M \mid \alpha \mid $. Then 
 \begin{align*}
   v(\gamma_{\alpha \beta})= v (g_{p \alpha -\beta}) &\geq  M \mid  p
   \alpha - \beta \mid \\ 
   &\geq M( p \mid \alpha \mid -\mid \beta \mid )
 \end{align*}
We consider one term of the   series giving $d_m$
\begin{align*}
  v \left( \prod^{m}_{j=1} \gamma_{\alpha_{j}} \gamma_{\sigma (j)}\right) &=
  \sum^{m}_{j=1} v( \gamma_{\alpha_{j}} \alpha_{\sigma (j)})\\ 
  &\geq M \sum_{j} p \mid \alpha_{j} \mid - \mid \alpha_{\sigma(j)} \mid \\
  &\geq M(p-1) \sum_{j} \alpha_j
\end{align*}\pageoriginale

Now there exist only a finite number of indices $\alpha_i$ such that
their length $\mid \alpha \mid $  is less than some constant,
therefore the series $d_m$  converges. Moreover we get $v(d_m) \geq M
(p-1) \inf  \sum \limits^{m}_{j=1} \alpha_j$ where  infimum is taken over all
the sequence $\alpha_1,\ldots,\alpha_m$. Let 
$\rho_m =\inf \sum \limits^{m}_{j=1} \mid \alpha_j \mid$. Now let us
order the sequence of indices $\alpha \in  Z^{n+1}_{+} $ in such a way
that $\mid \alpha_i \mid < \mid \alpha_{i+1} \mid$, then we have
$\rho_m = \sum \limits ^{m}_{i=1} \mid \alpha_i \mid $ and we see
immediately that 
$$
\lim_{m \rightarrow \infty}  \frac{\rho_m}{m} =
\sum^{m}_{\frac{i=1}{m}} \alpha_i = \infty 
$$
Therefore $\dfrac{v(d_m)}{m}  $ tends to infinity as $m$ tends to
infinity. Hence we get the following lemma. 

\begin{Lemma}\label{part3:chap2:sec10:lem4}
  If an element $G= \sum \limits_{\alpha \in Z^{n+1}_{+}}  g_\alpha
  {X^{\alpha}}$ satisfies the condition  
  $$
  (C)~ v (g_\alpha) \geq M \mid \alpha \mid 
  $$
  then the series $\det (I-t \Gamma )$ with $\Gamma = \Gamma _G$ is
  well defined as an element of $\Omega \big[ [t.]\big]$ and is an
  every where convergent power series in $\Omega$.  
\end{Lemma}  

 It is evident from the above discussion  that if we prove that
\begin{enumerate}[(i)]
\item The\pageoriginale function $G$ defined by = $\pi \varphi (a_\alpha
  \xi^\alpha)$ satisfies the condition $(C)$ 
\item The formal power series $\exp \left(- \sum \limits^{\infty}_{s=1}
  \dfrac{t^s \Tr \Gamma^s}{s}\right)$ and $\det (I-t \Gamma)$ are identical. 
\end{enumerate}

 Then $\Delta (t)$ is every where convergent in $\Omega$ which implies
 that $\tilde{\zeta}_v (t)$ is meromorphic in $\Omega$. 
 
 We have already proved the result (ii)  when $\Gamma$ is a finite
 matrix. Let $\Gamma_h$ denote the matrix of first $h$ rows and columns
 of $\Gamma$. 
 
 Then
 \begin{align*}
   \det ( I-t \Gamma_h ) &= \exp - \sum^{\infty}_{s=1} \frac{t^s Tr
     \Gamma^s_h}{s} \\ 
   &= \sum^{\infty}_{m=0} d^h_m t^m
 \end{align*} 
 where $d^h_m = (-1)^m \sum \limits_{ \leq i_1 \leq i_2 < \ldots < i_m
   \leq h} \gamma_{i_1 i_{\sigma (1)}} \ldots
 \gamma_{i_{m}i_{\sigma_{(m)}}} $ being an element of $S_m$. 
 
 Therefore
 $$
 \log \left( \sum^{\infty}_{m=0} d^h_m t^m \right) = -
 \sum^{\infty}_{s=1} t^s \frac{\Tr \Gamma^s_h}{s} 
 $$
 We shall show that $d^h_m$ converges to $d_m$ and $\Tr \Gamma^s_h$
 tends to $\Tr \Gamma^s $ as $h$ tends to infinity. We have 
 \begin{multline*}
   d_m -d^h_m = (-1)^m \sum_{\alpha_1,\ldots, \alpha_m} \sum_{\sigma \in
     S_m} \gamma_{\alpha_1} \alpha_{\sigma(1)}\ldots \gamma_{\alpha_m
     \alpha_{\sigma (m)}} \\
   -(-1)^m \sum_{\alpha_i \leq h} \sum_{\sigma \in S_m}
   \gamma_{\alpha_{1}} \alpha_{\sigma (1)} \ldots
   \gamma_{\alpha_{m}\alpha_{\sigma (*)}}\qquad  
 \end{multline*}
 
 Obviously\pageoriginale $v(d_m - d_m^h)$ tends to infinity as $h$ tends to
 infinity. Similarly one can prove that  
\begin{multline*}
  v \left(\Tr ~ \Gamma^s-\Tr ~ \Gamma_h^s\right) = v \left[
    \sum_{\alpha_1 \dots \alpha_s}
  g_{p \alpha_1 - \alpha_2} \dots g_{p \alpha_s - \alpha_1} \right] \\
  -\left(\sum_{\alpha_1 \dots \alpha_s ~ \leq h} g_{p \alpha_1 - \alpha_2}
  \dots g_{p \alpha_s - \alpha_1}\right)  
\end{multline*}
tends to infinity as $h$ tends to infinity. In order to prove that the
function $G$ satisfies (1) it is sufficient to prove that each term
$\varphi (a_\alpha \xi^\alpha)$ of the product satisfies (1). We
have 
$$
\varphi(t)= F(\zeta -1, t)
$$
But $F(Y,t)= \sum\limits_{m = o}^\infty A_m (Y) t^m$ with $A_m (Y) =
Y^m B_m (Y)$ and $B_m (Y)$ belongs to $\mathscr{O}\big[
  [Y]\big]$. Therefore  
\begin{align*}
  \varphi (a_\alpha \xi^\alpha) &= \sum_{m=0}^\alpha (\zeta - 1)^m B_m
  (\zeta -1) (a_\alpha \xi^\alpha)^m\\ 
  &= \sum_{\beta=0}^\alpha h_\beta \xi^\beta. 
\end{align*}

Thus $h_\beta = 0$ if $\beta \neq \alpha_m$
$$
=(\zeta -1)^{\frac{\beta}{\alpha}} B_{\beta_/{_\alpha}} (\zeta-1)
a_\alpha^{\frac{\beta}{\alpha}}. 
$$
which shows that 
$$
v(h_\beta) \geq \frac{|\beta|}{|\alpha|} ~ ~ \frac{1}{p-1} =
\left(\frac{1}{p-1} ~ \frac{1}{|\alpha|}\right) |\beta| 
$$
because $B_{\beta_/{_\alpha}} (\zeta -1)
a_\alpha^{\beta_{/{_\alpha}}}$ is of positive valuation. Hence $G$
satisfies $(I)$. 

We\pageoriginale have proved that $\tilde{\zeta}_v (t)$ is convergent in a
disc $|t| < \delta < 1$ as a series of complex numbers and is
meromorphic in the whole of $\Omega$, therefore by the Criterion of
rationality proved earlier we obtain that $\overset{\sim}{\zeta_v}(t)$
is a rational function of $t$. Hence the lemma
\ref{part3:chap2:sec7:lem2} of \S 7 is 
completely proved and also the theorem \ref{part3:chap1:sec3:thm1}. 


\begin{thebibliography}{99}
\bibitem{1} {E. Artin} Algebraic Theory of Numbers, Gottingen
  1960.\pageoriginale 
\bibitem{2} {E. Artin} Geometric Algebra.
\bibitem{3} {A. Borel} \textit{Groupes linearies algebriques,} Annals
  of Math., 64 ~ 1956, p.20-82. 
\bibitem{4} {N. Bourbaki} Algebre Ch.. VII
\bibitem{5} {N. Bourbaki} Algebre Ch.. VIII
\bibitem{6} {N. Bourbaki} Algebre Ch.. IX
\bibitem{7} {N. Bourbaki} Algebre commutative, Ch. I-VI (to appear).
\bibitem{8} {F. Bruhat} Sur les representations induties des groupes
  de Lie, Bull. Soc. Math. Fr., 84, ~ 1956, ~ p.97-205. 
\bibitem{9} {F. Bruhat} Lecture on Lie groups and representations of
  locally compact groups, Tata Institute 1958. 
\bibitem{10} {F. Bruhat} Distributions sur, les groupes localement
  compacts et applications aux representations des groupes p-adiques,
  Bull. Soc. Math. Fr., 89, 1961, p.43-76. 
\bibitem{11} {F. Bruhat} Sur les representations des groupes
  classiques p-adiques, Amer. Journal of Math., 84, 1961. 
\bibitem{12} {C. Chevalley} Sur certains groupes simples, Tohoku
  Math. JOur. 7, 1955, ~ p.14-66. 
\bibitem{13} {J. Dieudonne} Sur les groupes classiques, Hermann 1948.
\bibitem{14} {J. Dieudonne} La geometrie des groupes classiques,
  Springer 1955. 
\bibitem{15} {B. Dwork} Norm residue symbol in local number fields,
  Abh. Math. Sem. Hamburg, 22, 1958, p.180-190. 
\bibitem{16} {B. Dwork} On the retionality of zeta function of an
  algebraic variety, Amer. Jour. of Math., 82, 1960, p.631-648. 
\bibitem{17} {M. Eichler} Quadratische Formen and orthogonale
  Gruppen, Berlin 1952. 
\bibitem{18} {I. Gelfand and M. Naimark} Unitare Darstellungen der
  klassische Gruppen, Berlin 1957. 
\bibitem{19} {R. Godement} A theory of spherical functions,
  Trans. A.M.S., 73, 1952, p.496-556. 
\bibitem{20} {A. Grothendieck et J. Dieudonne} Elements de
  geometrique Algebrique, I, Publications Mathematiques de
  1'I. H.E.S. no4, Paris 1960.\pageoriginale 
\bibitem{21} {Harishchandra} Travaux de Harishchandra Seminaire
  Bourbaki February 1957. 
\bibitem{22} {Harishchandra} Representations of Semi-simple Lie
  Groups V Amer. Jour. of Math., 78, 1956, p.1-40. 
\bibitem{23} {H. Hasse} Zahlentheorie, Berlin 1949.
\bibitem{24} {I. Kaplansky} Groups with representations of bounded
  degree, Canad. Jour. of Math., 1, 1949, p.105-112. 
\bibitem{25} {I. Kaplansky} Elementary divisors and modules,
  Trans. A.M.S., 66,1949, p. 464-491. 
\bibitem{26} {M. Krasner} Theorie des corps values, C.R.Acad. Sci.,
  239, 1954, p.745-747. 
\bibitem{27} {G.W. Mackey} Induced representations of locally compact
  groups Annals of Math. 55, 1952, p.101-134. 
\bibitem{28} {F.I. Mautner} Spherical functions on $p$-adic fields,
  Amer. Jour. of Math., 80, 1958, p.441-457. 
\bibitem{29} {P. Samuel} Algebre locale, Memorial des Sci. Math.,
  123, Paris, 1953. 
\bibitem{30} {O. Schilling} Theory of valuations, New-York, A.M.S.,
  1950. 
\bibitem{31} {L. Schnirelmann} Sur les fonctions dans les corps
  normes algebriquement clos, Bull. Acad. Sci. U.R.S.S., 19 ~
  p.487-497. 
\bibitem{32} {J.P. Serre} Representations lineaires et espaces
  homogenes Kahleriens des groups de Lie Compacts. Seminaire Bourbaki
  Mai 1954. 
\bibitem{33} {R. Steinberg} The representations of GL(3, q), GL(4,
  q), PGL(3. q) and PGL(4, q), Canad. Jour. of Math., 3, 1951,
  p.225. 
\bibitem{34} {A. Weil} L'integration dans les groupes topologiques,
  Paris 1940. 
\bibitem{35} {A. Weil} Sur les courbes algebriques et les varietes
  qui s'en deduisent, Paris, 1948. 
\bibitem{36} {A. Weil} Numbar of solutions of equations in finite
  fields, Bull. A.M.S., 5, 1949, p.497. 
 
  {For Part \ref{part1}}, see\pageoriginale (1), (7), (23), (29), (30)

  {For Part \ref{part2}}, see (9), (19), (24), (27) and (34) for
  the theory of spherical functions in general. See (2), (6), (13),
  (14), (17), (24), for classical groups. See (10), (11), (17) and
  (28) for $p$-adic groups. 

  {For Part \ref{part3}}, see (15), (16), (26) and (31) for
  analytic functions and (16), (34), (36) for zeta-function. 
\end{thebibliography}
