\part{Representations of classical groups over $p$-adic Fields}\label{part2}

%\setcounter{chapter}{0}
\chapter[Representations of Locally Compact and Semi-Simple...]{Representations of Locally Compact and Semi-Simple Lie
  Groups}\label{part2:chap1} 

\section{Representations of Locally Compact Groups}\label{part2:chap1:sec1}

In\pageoriginale this section we give a short account of some definitions and
results about the representations of locally compact groups.We assume
the fundamental theorem on the existence and uniqueness (upto a
constant factor) of right invariant Haar measure on a locally compact
groups.For simplicity we assume that the locally compact groups in our
discussion are unimodular \iec  the Haar measure is both right and
left invariant.By $L(G)$ we shall denote the space of continuous
complex valued functions with compact support and by $L(G,K)$, where
$K$ is some compact set of $G$, the set of elements of $L(G)$ whose
support is contained in $K$. Obviously we have $L(G)=\underset{K
  \subset G}{\cup} L(G,K)$ and $L(G,K)$ is a Banach space under the norm
$f = \underset {x \in K} \sup |f(x)|$. 

The space $L(G)$ can be provided with a topology by taking the direct
limit of the topologies of $L(G,K)$. This topology makes $L(G)$ a
locally convex topological vector space. 

Let $G$ be a locally compact group and $H$ a Banach space

\begin{definition}\label{part2:chap1:sec1:def1}
  A continuous representation $U$ of $G$ in $H$ is a map $x \to U_x
  \in  \Hom (H,H)$ such that 
  \begin{enumerate}[(i)]
  \item $ U_{xy} = U_x\circ U_y $ for $x$; $y$ in $G$.
  \item The map $H \times G \to H$ defined by $(a,x) \to U_x $ a is continuous.
  \end{enumerate}
\end{definition}

\begin{defi*}
  Let\pageoriginale $H$ be a Hilbert space.The representation $U$ is said to be
  Unitary if $U_x$ is a unitary operator on $H$ for every $x$ in $H$. 
\end{defi*}

Let $M(G)$ be the space of measures on $G$ with compact support.The
space $M(G)$ is an algebra for the convolution product defined by  
$$ 
\mu * \nu (f)=\int \int f\,(x y )\,d \mu \,(x)\, d \nu\,(y)
$$

The space $L(G)$ can be imbedded into $M(G)$ by the map $ f \to \mu_f
= f(x)dx$. It is infact a subalgebra of $M(G)$ because $\mu_f * \mu _g
= \mu_{f *g}$ where 
$$
f* g(x)= \int f(xy^{-1}) g(y) dy.
$$

Moreover if $v$ is any element of $M(G)$, then $\mu_f * \nu $
belongs to $L(G)$, because for any $g \in L(G)$ we have 
\begin{align*}
  (\mu _f * \nu)(g) & = \iint  g(xy) f(x) dx d \nu (y)\\
  & = \int d \nu (y) \int g(x) f(xy^{-1}) dx.\\
  & = \mu_h(g) \text{~ where~ } h(x)=\int f(xy^{-1}) d \nu (y)
\end{align*}

Thus we define the convolution of a measure $\mu$ and function $f \in
L(G)$ by setting 
\begin{align*}
  (\mu  *f )(y) & = \int f(x^{-1}y) d \mu (x)\\
  (f * \mu )(y) & = \int f(y x^{-1}) d \mu (x)
\end{align*}

Let $U$ be a representation of $G$ in $H$. Then $U$ can be extended to
$M(G)$ by setting 
$$
U_\mu (a)= \int_G U_x a d \mu (x)~(\text{for a}~ \in H, \mu \in M(G)
$$

Now\pageoriginale let $H$ be a Hibert space and $U$ a Unitary representation. 

Then if $\mu$ and $\nu$ are any two elements in $M(G)$, we have
\begin{align*}
  \langle U_\nu U_\mu a,b \rangle & = \int \langle U_x U_\mu a,b
  \rangle d \nu (x)\\ 
  & = \int \langle U_\mu a, U_{x^{-1}} b \rangle d \nu (x)\\
  & = \int d \nu (x) \int \langle U_y a U_{x^{-1}} b \rangle d \mu(y)\\
  & = \int \langle U_x U_y a,b \rangle  d \nu (x) d \mu(y)
\end{align*}

This means that $U_{\mu * \nu} = U_\mu \circ U_\nu $\iec 

$U$ is a representation of the algebra 
$M(G)$. It can be easily verified
that map $\mu \to U_\mu$ is a continuous representation of the algebra
M(G). Moreover 
\begin{align*}
  \langle U^*_\mu a,b \rangle &= \langle \overline {U_\mu b, a} \rangle \\
  & = \int \langle U_x a,b \rangle d \overline{\mu (x^{-1})}
\end{align*}

This shows that $U^*_\mu = U_{\tilde{\mu}}$, where $d\tilde {\mu}(x)=d
\overline{\mu (x^{-1})}$. 

Thus the operator $U_{\mu * \tilde \mu }$ on $H$ is Hermitian.

In particular we get a representation of $L(G)$ in $H$ given by $f \to
U_{\mu _f} = U_f$, where 
$$
U_f(a)= \int_G U_x a f(x) d x.
$$

We can also get a representation of $M(G)$ by considering regular
representations of $G$ \iec representations $G$ by right or left
translations in $G$ in any function space connected with $G$ with some
convenient topology, for instance the space $L(G)$ or $L^2(G)$ (the
space of square integrable functions). 

We\pageoriginale shall denote by $\sigma_x$ the left regular representations and by
$\tau _x$ the right regular representations of $G$ \iec  for any
function $f$ on $G$ we have  
$$
\sigma_x (f)(y)=f(x^{-1}y), \tau_x (f)(y)=f(y x)
$$
we have for any $\mu $ in $M(G)$
\begin{align*}
  \sigma_\mu (f) &=\mu *f \\
  \tau _\mu (f) &= f * \overset{v}{\mu}~ \text{ where } d
  \overset{v}{\mu} (x) =d \mu (x^{-1}) 
\end{align*}

Let $K$ be a compact group, $M$ an equivalence class of (unitary)
irreducible representations of $K$. For any $x$ belonging to $G$, let
$M_x=(C^M_{ij} (x))$ be the matrix of $M_x$ with respect to some basis
of the representation  space. Let $r_M$  be the dimension of
$M$ and $\chi _M =\sum \limits^{r_m}_{i=1} C_{i~i}^{M}$ the character
of $M$. For any two irreducible unitary representations of $K$ we have
the following orthogonality relation, 
\begin{enumerate}[(1)]
\item $C^M_{ij} * C^{M'}_{kl}$  if $M \neq M'$
\item $C^M_{ij} * C^M_{kl} = \dfrac{l}{r_M} \delta_{jk} C^M_{il}$
\end{enumerate}
where the value of the convolution product at the unit element e of
$G$ is given by $C^M_{ij} * C^M_{kl} (e)= \int C^{\overline{M}_{ji}
(y)} C^M_{Kl}(y)dx$. 

When we write (1) and (2) in terms of characters we get 
\begin{enumerate}[(1)]
\item $\chi_M * \chi_{M'}= 0 \text{ if } M \neq M'$
\item $\chi_M * \chi_M =\dfrac{l}{r_M} \chi_M$
\end{enumerate}
Obviously\pageoriginale we have 
$$
(r_M \chi_M)* C^M_{ij} = C^M_{ij} * \chi_M r_M = C^M_{ij}
$$

Let $L_M(K)$ be the vector space generated by the coefficients
$C^{\bar M}_{ij}$, where $\bar M$ is the complex conjugate
representation of $M$. If $f$ is in $L^2(G)$, then by Peter-Weyl's
theorem, $f =\sum \limits_{i,j,N} \lambda_{ijN} C^N_{ij}$. Further if
$r_m \chi_{\bar M} *f=f$, then we have $ f =\sum \limits_{i,j}
\lambda_{ijM} C^{\bar M}_{ij}$, which means that $f$ belong to
$L_M(K)$. Conversely if $f$ belongs to $L_M(K)$, then $ f =\sum
\limits_{i,j} \lambda_{ij\bar M} C^{\bar M}_{ij}$. Therefore $r_M \chi
_{\bar M}*f=f$. Hence $f \in L^2(G)$ is in $L_M(K)$ if and only if $
r_M \chi _{\bar M}*f=f$. 

In this paragraph we give another interpretation of the space
$L_M(K)$. 

\begin{defi*}
Let $M$ be an irreducible unitary representation of $K$ and $U$ any
representation of $K$ in a Banach space $H$. 
\end{defi*}

We say that an element a $\in H $ is transformed by $U$ following $M$,
if a is contained in a finite dimensional invariant subspace $F$ of
$H$ such that the restriction of $U$ to $F$ is direct sum
representations of the equivalence class of $M$. 

Let\pageoriginale $H^U_M=H_M=\{a \mid \in H$, a transformed by $U$ following $M
\}$. It is easy to verify that $H_M$ is a vector space. 

\setcounter{proposition}{0}
 \begin{proposition}\label{part2:chap1:sec1:prop1}
   $L_M(K)$ is exactly the subspace of $L^2(K)$ formed by the elements
   which are transformed following $M$ (respectively following
   $\overline {M}$) by the left (respectively right) regular
   representation of $K$.		 
 \end{proposition} 
 
 \begin{proposition}\label{part2:chap1:sec1:prop2}
   If $U$ is a representation of $K$ in $H$, then $E_M=U_{r_M
     \overline{\chi}_M}$ is a continuous projection from $H\to H_M$. 
 \end{proposition} 
 
 In order to prove the proposition $1$ and $2$ prove the following results.
  \begin{enumerate}[(1)]
  \item Suppose that $\varphi$ belongs to $L_M(K)$, then $\varphi=
    \sum \limits_{i,j}\lambda_{ij}C^{\overline{M}}_{i,j} $. For  
    \begin{align*}
      x \in K, \text{we have} \left(\overset{\sigma}{x}
      C^{\overline{M}}_{ij}\right)~(y)&= C^{\overline{M}}_{ij}(x^{-1}y)= \sum
      _{k}C^{\overline{M}}_{ij}(x^{-1})C^{\overline{M}}_{ij}(y)\\ 
      &=\sum_{k}(C^M_{ki}(x))C^{\overline{M}}_{kj}(y).
    \end{align*}
    So the space $E_j$ generated by $C_{ij},\cdots, C_{rj}(r_M=r)$ is
    invariant by $\sigma$ and the restriction of $\sigma$ to $E_j$ is of
    class $M$. Therefore $L_M(K)$, which is direct sum of the $E_j$, is
    contained in $(L^2(G) \overset \sigma {\underset {M})}$.  

  \item If $\varphi$ belongs to $L_M (K)$ and a belongs to $H$, then
    we show that $U_\varphi a$ belong to $H_M$. 

    We have 
    $$
    U_x U_\varphi a=U_{\in _x * \varphi ^{a^p}}=U_{\underset {x}\sigma
      \circ\varphi^a}
    $$ 
    where $\varepsilon_x$ is the Dirac measure at the point $x$, and 
    $$
    U_{\varepsilon_x}b=\int_{K}  U_y b ~d\varepsilon_x (y)=U_x b.
    $$

    This\pageoriginale shows that $\varphi \in L_M(K) \to U_\varphi$ a $\in H$ is a
    morphism of representation $\sigma$ and $U$. Hence $U_\varphi a$
    is transformed by $U$ following $M$.

\item If a belongs to $H_M$, then $E_M a=a$. Since a belongs to some
  finite dimensional invariant subspace $F$ of $H$ and the restriction
  of $U$ to $F$ is the direct sum of representation of class $M$, we
  can find a basis $(e_{jk})$  of $F$ such that $U_x e_{jk}=\sum
  \limits_K C^M_{ij}(x)e_{ik}$ 

  Let a $= \sum \limits _{i,j}\lambda_{ij}e_{ij}$. Then 
  \begin{align*}
    E_M (a)&=r_M \int \sum_{i,j,k} \lambda_{jk} C^M_{jk}(x)e_{ik}
    \overline {\chi}_M(x)dx\\ 
    &=r_m \sum _{i,k} \bigg (\sum _{j}\lambda_{jk} \int C^M_{ij}(x)
    \overline{\chi}_M (x)dx \bigg ) e_{ik} \\ 
    &=\sum _{i,k}\lambda_{ik}e_{ik}=a.
  \end{align*}
  
  Moreover
  
  $\int r_M C^M_{ij}(x)\chi_M (x^{-1})dx = r_M \chi_M* C^M_{ij}(e)=\delta_{ij}$
  
\item In particular if $\varphi$ belongs to $L^2(G)$ it is transformed
  by $\sigma$ following $M$, then 
  $$
  \sigma_{r_M \chi _M} \varphi=r_M \,\overline{\chi}_M *\varphi=\varphi
  $$
\end{enumerate} 
  
  Therefore $\varphi$ belongs to $L_M(K)$.
  
  Clearly the results (1)and (4) imply proposition \ref{part2:chap1:sec1:prop1}.
  \begin{align*}
    \text{Since}\hspace{2cm} E_M\dot E_M & = U_{\underset{M}r \chi_M} 
    ~U_{\underset{M}r \chi_M} ~U_{\underset{M}r^2}\chi_M *\chi_M \\ 
    &=U_{r_M \chi_M}=E_M,
  \end{align*}
the proposition (\ref{part2:chap1:sec1:prop2}) is proved by result (3)

Similarly\pageoriginale we prove that $E_M \cdot E_{M'}=0$ for $M \neq M'$. Thus we get a
family of projections $E_M$ with $E_M(H)=H_M$. The sum $\sum H_M$ is
direct and is dense in $H$. It is sufficient to prove that if $a'$ is
a continuous linear form on $H$, which is zero on every $H_M$, then
$\langle a,a' \rangle=0$ for  every $a \in H$. Let us put $\varphi
(x)=\langle U_xa,a' \rangle$. Then  
\begin{align*}
  \langle \varphi, C^M_{ij} \rangle&=\int C^{\overline
    {M}}_{ij}(y)\langle U_ya,a' \rangle dy.\\ 
  &=\langle U_g a,a' \rangle ~\text{with}~ g=C^{\overline{M}}_{ij}
\end{align*}

But $U_g a'$ belongs to $H_{\bar{M}}$, therefore we get that
$\varphi$ is orthogonal to all the coefficients $C^{M}_{ij}$ for any
$M$, so $\varphi=0$. 

In particular if $U$ is unitary (for instance the regular
representation in $L^2(K)$), then the $E_M$ are orthogonal projections
and $H$ is exactly Hilbertian sum of the closed subspaces $H_M$. 

Let $G$ be a locally compact group, $K$ a compact subgroup of $G$.
Suppose that $U$ is a continuous representation of $G$ in $H$ and $M$
an equivalence class of unitary representation of $K$. By $H^U_M=H_M$
we shall mean the vector subspace of $H$ consisting of elements which
are transformed by the restriction of $U$ to $K$ following $M$. As in
the above case $E_M=U_{r_M \overline{\chi}_M}$ is a projection of $H$
to $H_M$. Let 
$$
L_M (G)= \bigg \{ f\mid f \in L(G), f*r_M
\overline{\chi}_M=r_M\overline{\chi}_M*f=f \bigg \} 
$$

It is easy to prove that $L_M(G)$ is a subalgebra of $L(G)$ and the
mapping $f\to r_M \overline{\chi}_M* \dot *r_H \overline{\chi}_M$ is a
projection from $L(G)$ to $L_M(G)$. 

If\pageoriginale $f$ belongs to $L_M(G)$ and a belongs to $H$, then  is in $H_M$. If
$b$ belongs to $H_M'$, then $U_f(a)=U_{r_M}(a)_{\chi_M}{}_{\ast~f}=E_M
~U_fa \Rightarrow U_f a$ is in $H_M$. If $b$ belongs to $H_{M'}$, then 
$$ 
 U_fb  =\underset{f~\ast~ r_M~\overline{\chi}_M}{U\qquad } \quad E_{M'} b= U_f
 E_M E_{M'} b=0
$$

This shows that $U$ is a representation of $L_M(G)$ in $H_M$ and
$U_f=E_M U_f E_M$. Moreover for $f \in L_M(G)$  
$$
f(y)=r_M \int\limits_{K} f(k^{-1}y) \overline{\chi}_M (k)dk.
$$

In particular if $M$ is the identity representation, then $\chi_M$ is
constant and $f$ is in $L_M(G)$ if and only if  
\begin{align*}
  f(y)&=r_M \int\limits_{K} f(ky)dk=r_M \int\limits_{K} f(yk)dk\\
  &\Longleftrightarrow f(hyk)=f(fyk)=f(y).
\end{align*} 
 
Such functions are called spherical function on $G$ with respect to
$K$. They can be considered as functions on $G/K$ which are left
invariant, provided we write $G/K=\{ K,aK,---\}$ 

 \section{Irreducible Representations}\label{part2:chap1:sec2}
 
 In this section we study how we can get some information about the
 representation of a group $G$ by studying the representation of the
 algebra $L_M(G)$.  

\setcounter{definition}{0}
\begin{definition}\label{part2:chap1:sec2:def1}
  A representation $U$ of a group $G$ in a vector space $V$ is said to
  be \textit {algebraically irreducible} if there exists no proper
  invariant subspace of $V$.  
\end{definition} 
 
\begin{definition}\label{part2:chap1:sec2:def2}
  A representation $U$ of a topological group $G$ in a
  locally\pageoriginale convex 
  space $E$ said to be \textit{topologically irreducible} if there
  exists no proper closed invariant subspaces of $E$. 
\end{definition} 
 
\begin{definition}\label{part2:chap1:sec2:def3}
  A representation $U$ of a topological group $G$ in a Banach space
  $H$ is said to be completely irreducible if $U(L(G))$ is dense in
  $Hom(H,H)$ in the topological of simple convergence \iec  given an
  operator $T$ on $H$ and element $a_1,a_2,\cdots, a_p$ in $H$, there
  exists for every $\in > 0$ an element $f$ in $L(G)$ such that  
 \end{definition}  
 
 $\parallel (U_f-T)a_i \parallel < \in$ \qquad for $i=1,2,\cdots, p$.
 
 It is obvious that $(1)\Rightarrow (2)$. To prove that
 $(3)\Rightarrow (2)$, suppose that $F$ is a proper closed invariant
 subspace of $H$. Let $a \neq 0$ be any element of $F$, then for every
 $b$ in $H$ there exists a $T \in \Hom (H, H)$ such that $T(a)=b$. But
 by definition for every $\varepsilon > 0$ there exists an
 element $f$ in $L(G)$ such that $\parallel U_f  a-T(a) \parallel
 <\varepsilon$. This means that $F$ is dense in $H$ which is a
 contradiction because $F$ was assumed to be a closed proper subspace
 of $H$.   
 
 The definitions (\ref{part2:chap1:sec2:def2}) and
 (\ref{part2:chap1:sec2:def3}) are equivalent for unitary
 representation 
 by Von Neumann and all the three representation are equivalent for
 finite dimension representations. The proof can be found in \cite{9}. The
 definition (\ref{part2:chap1:sec2:def1}) implies
 (\ref{part2:chap1:sec2:def3}) (for proof see annals of Mathematics,
 1954  Godement). 

\setcounter{Lemma}{0}
\begin{Lemma}\label{part2:chap1:sec2:lem1}
  If $U$ is a completely irreducible representation of $G$ in a Banach
  space $H$, then the representation $U^M$ of $L_M(G)$ is $H(M)$ is also
  completely irreducible. 
\end{Lemma}   
   
\begin{proof}
  Suppose that $T$ belongs to $\Hom (H(M), H(M))$. Extend $T$ to $H$ by
  setting $\tilde {T}=T$ on $H(M)$ and $O$ on
  $E^{-1}_M(0)$. Obviously $T$ is continuous on $H$. 
\end{proof}  
   
Since\pageoriginale $U$ is completely irreducible,  $\tilde{T}$ can be
approximated 
by $U_f$ for $f$ in $L(G)$ \iec  $\tilde {T}=\lim U_{f_i}$. Therefore  
\begin{align*}
  E_M \tilde{T} E_M &=\lim E_M U_{f_i} E_M\\
  &=\lim U_{r_M \overline{\chi}_M*f_i*r_M \overline{\chi}_M}
\end{align*}   
   
Hence in $H_M,T=\lim U_{r_M \overline{\chi}_M*f_i*r_M \overline{\chi}_M}$
   
where $r_M \overline{\chi}_M*f_i*r_M \chi _M$ is in $L_M(G)$. Thus
$U^M$ is completely irreducible. 
   
Let $U$ be a unitary irreducible representation of $G$ in a Hilbert
space $H$. \textit{By coefficient of $U$} we means positive definite
function $\langle U_x a$, $a \rangle$ on $G$. We state without proof the
following theorem about the coefficients of unitary representations. 

\setcounter{theorem}{0}
\begin{theorem}\label{part2:chap1:sec2:thm1}
  If two irreducible unitary representations have same non-zero
  coefficient associated to them, they are equivalent. 
  
  We have seen that the representation $U$ can be extended to the
  space $ M(G)$  and the operator $U_{\mu *\tilde{\mu}}$ for any $\mu$
  in $M(G)$ is Hermitian. In particular if we take $\mu=r_M
  \bar{\chi}_M dk$, we have $\mu=\tilde{\mu}$. There fore
  $U_{\mu*\mu}=E_M$ is Hermitian. 
\end{theorem} 
        
Moreover for any f in $ L(G)$ and a in $H_M$
\begin{align*}
  \langle U_f a,a \rangle &=\langle U_fE_Ma,E_Ma \rangle =\langle
  E_MU_fE_Ma,a \rangle \\ 
  &=\langle U_{f_0}a,a \rangle
\end{align*}   
where\pageoriginale $f_0=r_M ~\overline{\chi}_M *f*r_M \overline {\chi}_M$ belongs to
$L_M(G)$. Thus if we know nonzero coefficient associated to $U^M$, we
know coefficient associated to $U$ as a representation of $L(G)$,
which determines coefficient of $U$ as a representation of $G$. Thus a
unitary irreducible representation of $G$ is completely characterised
by its restriction $U^M$ to $L_G(G)$ if $U^M$ is not zero. 
    
\begin{defi*}
  A set $\Omega$ of representations of an algebra $A$ in a vector space
  is said to be complete if for every nonzero $f$ in $A$ there exists
  $U \in \Omega$ such that $Uf \neq O$. 
\end{defi*} 
      
\begin{proposition}\label{part2:chap1:sec2:prop3}
  If there exists a complete set $\Omega$ of representations of an
  algebra $A$ which are of dimension $\le K$ ($K$ a fixed integer), then
  every completely irreducible representation of $A$ in a Banach space
  is of dimension $\le k$. 
\end{proposition}   
           
We first prove a lemma due to Kaplansky. Let $A$ be any algebra. For
$x_1,\cdots,x_p$ in $A$ we define $[x_1, \cdots, x_p]=\sum
\limits_{\sigma \in S_p} \varepsilon_\sigma x_{\sigma_1} \ldots
x_{\sigma_p}$ where $S_p$ is the set 
of all permutations $\sigma$ on $1,2,\cdots, p$ and $\varepsilon_\sigma$ is
the signature of $\sigma$. Obviously if dim $A < p$, then
$[x_1,\cdots,x_p]=o$ for all $x_1,x_2,\cdots, x_p$ in $A$. 
           
In particular we take $A=M_n(C)$, algebra  of $n \times n$ matrix with
coefficient from $C$, the field of complex numbers, We define     
\begin{align*}
  r(n) & = \inf(p) \text {such that}\\
  [X_1,\cdots, X_p]&=0, X_i \in M_n(C).
\end{align*}  

Clearly $r(n)\le n^2 +1$. We shall prove that $r(n+1)\ge r(n)+2$. 
                 
We\pageoriginale have $r(n)-1$ elements $X_1, X_2,\cdots X_{r-1}$ in
$M_n(C)(r=r(n))$ such that $[X_1,\cdots,X_{r-1}]\neq 0$, Let $E_{kh}$
be the canonical basis of $M_n(C)$. Then 
$$
[X_1,\cdots,X_{r-1}]=\sum  ^{n}_{k,h=1} \lambda_{kh} E_{kh}.
$$               

Since $[X_1,\cdots, X_{r-1}]\neq 0$, there exists $k_0$ and $h_0$ such
that $\lambda _{k_0h_0} \neq 0$. Let $\tilde{X}_i$ be the matrix
obtained by adding a row and a column of zeros to $X_i$. Then               
\begin{align*}
  \left[\tilde{X}_1,\cdots,\tilde{X}_{r-1}E_{h_0,n+1} E_{n+1,n+1}\right] 
  &=\left[\tilde {X}_1,\cdots, \tilde{X}_{r-1}\right] E_{h_0, n+1}E_{n+1n+1}\\
  &=\sum _{h,k}\lambda_{kh} \tilde{E}_{kh} E_{h_0} n+1\\
&=\sum\lambda_{kh_0} E_{k,n+1}\neq 0. 
\end{align*}    
                 
Thus $r(n+1)\ge r(n)+2$.
                 
Now we prove the proposition. Suppose that $r(k)=r$ and $U$ is a
complete irreducible representation of $\dim > K$ in a Banach
space $H$. Let $F$ be a subspace of $H$ of $\dim k+1$. Since
$r(k+1)> r(k)$,  there exist operators $[A_1, \ldots, A_r]$ in $\Hom
(F, F)$ such that $[A_1,\ldots, A_r] \neq
0$. We extend each $A_i$ to the whole space $H$ by defining $A_i$ to
be zero on $F'$, where $F'$ is any closed subspace such that $H$ is
the topological direct sum of $F$ and $F'$. Suppose that $A_1=\lim
U_{f_{i_1}}$, where $f_{i_1}\in A$. We have 
$$
0\neq [\tilde{A}_1,\cdots, \tilde{A}_r]= \sum\limits_{\sigma \in
 \mathcal{S}_r} \tilde{A}_{\sigma_1}, \ldots \tilde{A}_{\sigma_r}=
\lim [U_{f_i}~\tilde {A}_2,\cdots, \tilde{A}_r].
$$ 
Therefore there exists $f_1$ in $A$
such that $[U_{f_1} \tilde {A}_2,\cdots, \tilde{A}_r]\neq 0$.
Repeating\pageoriginale this process we obtain that there exist elements
$f_1,\cdots,f_r$ in $A$ such that 

$U_{[f_1,\cdots,f_r]}=[U_{f_1},\cdots, U_{f_r}]\neq 0$. But this is a
contradiction because $[f_1,\cdots,f_r]=0$ if $[f_1,\cdots,f_r]\neq
0$, then there exists a $V$ in $\Omega$ such that
$V_{[f_1,\cdots,f_r]}(a)\neq 0 \Rightarrow$  
                                 
$[V_{f_1},\cdots, V_{f_r}](a)\neq 0$. But $r \ge r_k$ and $dim V\le
k$, therefore $[V_{f_1},\cdots$, $V_{f_r}]=0$. Hence $\dim U < k$. 
                 
\begin{coro*}
  Let $G$ be a locally compact group, $K$ a compact subgroup, $M$ a
  class of irreducible unitary representations of $K$ in a Banach space
  $H$. If there exists a system $\Omega$ of representations of $G$ in a
  Banach space such that (i) for every $U$ in $\Omega$, the
  representation $U^M$ of $L_M(G)$ is of $\dim \le p \dim
  M$. Equivalently $M$ occurs atmost $p$ times in each $U$.   
\end{coro*}    
 
(ii) The representations $U^M$ for $U$ in $\Omega$ form a complete
system of representation of algebra $L_M(G)$. 
                          
Then $M$ occurs atmost $p$ times in any completely irreducible
representation of $G$.   

\section{Measures on Homogeneous spaces}\label{part2:chap1:sec3}
 
Let $G$ be a locally compact group, $dx$ the right invariant Haar
measure and $\Delta (x)$ the modular function on $G$ \iec  $d(yx)=\Delta
(y)dx$. Let $\Gamma$  be a closed subgroup of $G$. We shall denote by
$\xi, \eta \ldots$ the elements of $\Gamma$ by $d\xi$ and
$\delta$ the Haar measure and the modular function on $\Gamma$. It is
well known that there exists a right  invariant Haar measure\pageoriginale on
$G/\Gamma$ if and only if $\Delta (\xi)=\delta (\xi)$. In general it
is possible to find a quasi-invariant measure on $G/ \Gamma$. In order
to show the existence, one shows that there exists a strictly positive
continuous function $\rho$ on $G$ such that $\rho(\xi
x)=\dfrac{\delta(\xi)}{\Delta(\xi)}\rho (x)$ for every $x$ in $G$ and
$\xi $ in $\Gamma$. Then the measure $\rho (x)dx$ gives rise to  a
measure $d \mu (x)$ on $G/\Gamma$ such that for any $f$ in $L(G)$ we
have  
\begin{equation*}
  \underset{G}\int f(x)\rho (x)dx=\underset{G/\Gamma}\int d \mu (x)
  \underset{\Gamma}\int f(\xi x)d\xi\tag{1}\label{part2:chap1:sec3:eq1} 
\end{equation*}   
                    
It is obvious from (\ref{part2:chap1:sec3:eq1})that                 
$$
d \mu (\overline{(xy)})=\frac{\rho(xy)}{\rho(x)} d \mu (x)
$$                
where $\dfrac {\rho(xy)}{\rho(x)}$ depends only on the cosets of $x$ modulo
$\Gamma$. Thus $\mu (x)$ is a quasi-invariant measure on
$G/\Gamma$. The details could be found in \cite{9}. 
                    
\section{Induced Representations}\label{part2:chap1:sec4}
                    
Let $L$ be a representation of $\Gamma$ in Hilbert space $H$. We shall
define two types of induced representation on $G$ given by $L$.  
\begin{enumerate}[(1)]
\item Assume that $L$ is unitary. Let $H^L$ be the spaces of functions
  $f$ on $G$ such that                                 
  \begin{enumerate}[(1)] 
  \item $f$ is measurable with values in $H$.
  \item $f(\xi x)=[p(\xi)]^{1/2} L_\xi f(x)$, for $\xi \in \Gamma$.
  \item $\underset {G/\Gamma}\int (\rho(x))^{-1} \parallel f(x)
    \parallel^2  dx < \infty$. 
  \end{enumerate}                     
\end{enumerate}

Since the function $(\rho(x))^{-1} \parallel f(x)\parallel ^2$ is
invariant on the left by $\Gamma$, it can be considered as a function
on $G/\Gamma$. Thus we define 
$$
\parallel f \parallel^2 =\underset {G/\Gamma} \int (\rho(x))^{-1}
\parallel f(x) \parallel^2 d \mu (x) 
$$

It\pageoriginale can be proved that $H^L$ is a Hilbert space with the scalar product  
$$
\langle f,g >, = \underset{G/\Gamma}\int (\rho(x))^{-1} \langle
f(x),g(x) \rangle d \mu (x). 
$$

Let $U^L$ be the map from $G$ to $H^L$ such that 
$$
U^L_x f(y)=f(xy)
$$ 

Obviously $U^L$ is continuous. Since we have 
\begin{align*}
  \parallel U^L_y f \parallel^2 &= \int\limits_{G/\Gamma} (\rho(x))^{-1}
  \parallel f(xy) \parallel^2 d\mu (x)\\ 
  &= \underset{G/\Gamma} \int (\rho(xy^{-1}))^{-1} \parallel f(x)
  \parallel^2 \frac{\rho(xy^{-1})}{\rho(x)} d \mu (x)\\ 
  &=\parallel f \parallel^2
\end{align*} 

It follows that $U^L$ is unitary. We say that $U^L$ is the unitary
representation induced by $L$. 

(ii) Let $L$ be any representation of $\Gamma$. Let us suppose that
there exists a compact subgroup $K$ of $G$ such that  $G=\Gamma
K$. Let $C^L$ be the space of functions $f$ such that  
\begin{enumerate}[(1)]
\item $f$ is continuous with values in $H$.
\item $f(\xi x)=(\rho(\xi))^{\dfrac{1}{2}} L_\xi (f(x))$ for $\xi \in \Gamma$.
\end{enumerate} 

 We define $\parallel f \parallel=\sup\limits_{x \in K} \parallel
 f(x) \parallel$. Clearly $C^L$ with this norm is a Banach
 space. Again right translation by elements of $G$ give rise to a
 representation of $G$ in $G^L$. We denote this also by $U^L$. 
 
 Let $ f \to $ restriction of $f$ to $K=f_0$ be the map from the
 $C^L$ to $C(K)$ (the set of continuous functions on $K$ with values in
 $H$). The image of $C^L$ by this map is the set of elements $f_0 \in
 C(K)$ which satisfy\pageoriginale condition (2) above for all $ \xi $ in $\Gamma
 \cap K$ and $x$ is $K$. But $\rho (\xi)=1$, because $\rho$ is a positive
 real character of $K \cap \Gamma$, therefore $f_0 (\xi
 x)=f_0(x)$. Through the space $C^L$ is identified with a subspace of
 $C(K)$ yet the representation $U^L$ cannot be defined on this
 subspace. However the restriction  of $U^L$ to $K$ and the
 representation induced by the restriction of $L$ to $\Gamma \cap K$
 are identical. 
 
 If $L$ is unitary then $f$ belongs to $H^L$ if and only if $f_0$
 belongs to $L^2(K)$. We can choose $\rho$ in such a way that $\rho (x
 k)=\rho (x)$ for $k \in K$. Since the group $K/K\cap\Gamma$ is
 compact homogeneous space, there exits one and only one invariant
 Haar measure on it. But $K/K\cap \Gamma$ is isomorphic to $G/ \Gamma$
 therefore with the above choice of $\rho$, the quasiinvariant
 measure on $G/ \Gamma$ gives rise to the invariant measure on $K/K
 \cap \Gamma$. We have 
 \begin{align*}
   \underset {G/ \Gamma}\int (\rho(x))^{-1} \parallel f(x) \parallel^2
   d\mu (x) &=\underset {K/K \cap \Gamma}\int \parallel f(k)
   \parallel^2 d \mu (k)\\ 
   &= \underset {K} \int \parallel f(k) \parallel^2 dk
   \tag{A}\label{part2:chap1:sec4:eqA} 
\end{align*} 
Our result is obvious from (\ref{part2:chap1:sec4:eqA}).

\section{Semi Simple Lie Groups}\label{part2:chap1:sec5}

Let $G$ be a semi simple Lie group worth a faithful representation. We
state here two theorems the proof of which could be found in \cite{19}. 
\begin{theorem}\label{part2:chap1:sec5:thm2}
  The group $G$ has a maximal compact subgroup and all the maximal
  compact subgroup are conjugates. 
\end{theorem} 
 
\begin{theorem}\label{part2:chap1:sec5:thm3}
  Suppose that $K$ is maximal compact subgroup of $G$, then there
  exists a connected solvable $T$ of $G$ such that $G=TK$. 
\end{theorem} 
 
 We\pageoriginale shall prove the following theorem about completely irreducible
 representation of $G$. 
  
\begin{theorem}\label{part2:chap1:sec5:thm4}
  Every irreducible representation $M$ of $K$ is contained atmost $\dim$
  $(M)$ times in every completely irreducible representation of $G$. 
\end{theorem} 
 
\begin{proof}
  (1) The finite dimensional irreducible representations of $G$ is a
  vector $H$ is a complete system of representations of $L(G)$. Let $x
  \to \rho _x$ be a representation of $G$ in a vector space $H$. We
  call the function $\theta (x)=\langle \rho _x a,a' \rangle$ where a
  belongs to $H$ and $a'$ belongs to $H^\ast$ (the conjugate space of
  $H$), a coefficient of the representation. Let $V$ denote the vector
  space generated by all coefficients of all finite dimensional
  irreducible representations of $G$. Since every finite dimensional
  representation of $G$ is completely reducible, $V$ contains all the
  coefficients of all finite dimensional representations of $G$. Let
  $\rho^1$ and $\rho^2$ be two finite dimensional irreducible
  representations of $G$. Then we have  
  $$
  \langle \rho^1_x a_1,a'_1 \rangle \langle \rho^2_xa_2,a'_2 \rangle=
  \langle \rho^1_x \otimes \rho^2_x ~ a_1 \otimes a_2, a'_1 \otimes a'_2
  \rangle
  $$ 
 showing that $V$ is an algebra.  Moreover $V$ is a self adjoint
 algebra, because if $\theta (x)=\langle \rho a, a' \rangle$ is in $V$,
 then $\bar {\theta}(x)=\langle \bar{\rho}_x \bar{a}, \bar {a'} \rangle$
 is also in $V$. Since $G$ has a finite dimensional faithful
 representation, $V$ separates points \iec  if $\theta (x)=\theta (x')$
 for every $\theta$ in $V$, then $x=x'$. Thus Stone- Weierstrass'
 approximation theorem every continuous function on $G$ can be
 approximated uniformly on every compact subset by elements of
 $V$. Hence if $f$  is a non-zero elements of $L(G)$, then $\int
 f(x)g(x)dx=0$ for every element $g$ of $C(G)$ (the set of all
 continuous\pageoriginale functions on G), because 
 $$
 \rho_f = \int \rho_x f(x) dx~ \text{and}~ \int< \rho_x a,a'> f(x)dx=0
 $$ 
 for every a in $H$ and $a'$ in $H^*$ and $\rho$. Therefore $f$ must be =0 

 (2) The representations of $G$ induced by all characters of $T$ form a
 complete system for $L(G)$ 
 
 Let $\rho$ be a finite dimensional irreducible representation of $G$
 and let $\overset{v}\rho = (\overset{t}\rho)^{-1})$ be the
 representation contragradient to $\rho$. By Lie's theorem \cite{19},the
 restriction of $\overset{v}\rho$ to $T$ has an invariant subspace of
 dimension 1, which implies that there exists a vector $b' \neq 0$ in
 $E^*$(the conjugate space of the representation space $E$ of, $\rho$)
 such that $\overset{v}\rho(t)=b'= \chi(t)b'$ for every $t \in
 T$. Consider the mapping a $\in E \to \widetilde{a} \varepsilon c^{\chi -1}$,
 where $ \widetilde{a}(x)=  < \rho_x a, b'>$. Since  
 $$
 \widetilde{a}(tx)= < \rho_{tx}a,b'> = < \rho_xa,  \rho^{r-1}_t  b'>
 =\chi^{-1}(t) < \rho_x a,b'> = \chi^{-1}(t) \widetilde{a}(x), 
 $$
 $\widetilde{a}(x)$ is covariant by left translation. Obviously the map
 $a \to \widetilde{a}$ is continuous. Let $U^{\chi-1}$ be the
 representation of $G$ induced by $\chi^{-1}$. The mapping $a \to
 \widetilde{a}$ is a morphism of representations $\rho \text{ and }
 U^{\chi-1}$, because 
 $$
 \widetilde{\rho}_y ~a(x) = \langle \rho_x ~\rho_y a, b' \rangle =
 \widetilde{a}(xy)= U^{\chi-1}_y(\widetilde{a}). 
 $$
 The mapping $a \to \tilde{a}$ is not zero. If $a \neq 0$, then
 $(\rho_x a)$ generates the 
 whole space $E$ because $\rho$ is irreducible, therefore for atleast
 on $x$ in $G \langle \rho_x a, b' \rangle \neq 0 \Rightarrow
 \widetilde{a} \neq 0$. Let $f$ be a non-zero element of
 $L(G)$. If $U^{\chi-1}_f = 0$. For every $\chi$ then $\rho_f = 0$ for
 every $\rho$ which means the $f=0$ by (1). This is a contradiction,
 hence our result is proved. 
 
 (3) We\pageoriginale shall show that if $\chi$ is a character of $T$, then $M$
 occurs atmost dim ($M$) times in $U^{\chi}$. Clearly
 $U^{\chi}/K$(restriction of $U^{\chi}to K)=U^{\chi / K \cap T}$.But
 the space of this representation is the space of continuous functions
 $f$ on $K$ such that  
 $$
 f(t k) = \chi (t)f(k) ~\text{ for }~ t \in K \cap T.
 $$
 Therefore $U^{\chi / K \cap T}$ is a subrepresentation of the right
 regular representation of $K$. Hence $(C^{\chi})_M \subset L_M(K)$ which
 is a space of $(dim M)^2$. Thus $M$ occurs at most dim(M) times in
 $U$. Our theorem follows from (\ref{part2:chap1:sec5:thm2}),
 (\ref{part2:chap1:sec5:thm3}) and proposition 1.3. 
\end{proof}
