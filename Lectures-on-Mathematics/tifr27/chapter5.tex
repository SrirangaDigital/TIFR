\part{Zeta-Functions}\label{part3}

%\setcounter{chapter}{0}
\chapter{Analytic Functions over $p$-adic Fields}\label{part3:chap1}

Unless\pageoriginale otherwise stated $K$ will denote a completed valuated field
with a real valuation $v$. We shall adhere to the notations adopted in
part $I$  throughout our discussion. 
 
\section{Newton Polygon of a Power-Series}\label{part3:chap1:sec1}

\begin{defi*}
  Let $f(x)=\sum\limits_{i= 0}^{\infty}a_{i}x^{i}$ be a
  power-series over $K$. Let $S$ be the set of points $A_{i} = (i,
  v(a_{i}))$ in the Cartesian plane. The convex hull of $S$ together
  with the point $y = \infty$ on the ordinate axis is called the
  \textit{Newton Polygon} of the power series $f$. 
\end{defi*}

It is obvious that the point $A_{i} = (i, v(a_{i}))$ lies on the line
$Y + v(x)X = v(a_{i}x^{i})$, where $v(a_{i}x^{i})$ is the intercept
cut off by the line on the $Y-axis$. If the series is convergent at
the point $x = t$ then intercepts cut off on the axis of $Y$ by the
lines through the points $A_{i}$ with the slope -$v(t)$ tend to
infinity as $i$ tends to infinity. Moreover it can be easily proved
that if $(m_{i})$ is the sequence of slopes of the sides of Newton
Polygon of $f$, then $(m_{i})$ is monotonic increasing and
$^{-\lim}_{i\to\infty} m_{i}=^{-\lim\inf}_{~~~i\to\infty}
\dfrac{v(a_{n})}{n}=\rho(f)$ (the order of convergence of $f$). 

\section{Zeroes of a power series}\label{part3:chap1:sec2}

Let $f=\sum\limits^{\infty}_{i=0} a_{i}x^{i}$ be a power series over
$K$.  Let $\rho(f)=^{-\lim\inf}_{i\to\infty} \dfrac{v(a_{i})}{i}$. We
have already proved that $f$  is convergent for\pageoriginale all
points $x$ in $K$ 
for which $v(x)> \rho(f)$. Let $r$ be a  real number greater than
$\rho (f)$. We shall try to find the zeroes of $f$ on the circle
$v(x)=r$. Let us assume that $a_{\circ}\neq 0$. 

(i)~ If there exists no side of the Newton Polygon of $f$ with
  slope-$r$, then there exists there exists one and only term of
  minimum valuation in $\sum a_{i}x^{i}$. For, if $v(x)=r$ and
  $a= v(a_{i}x^{i})= v(a_{j}x^{j})=^{\inf}_{k}v(a_{k}x^{k})$, then all the points
  $A_{k}$ are above the line $Y+rX=a$ and ${A_i A_j}$ is a side of the
  Newton Polygon of slope-$r$. This is contrary to the hypothesis. Thus
  $v(f(x)) =v(a_{i}x^{i})$ for some $i$ and for $v(x)=r$, which
  implies that there is no zero of $f$ on the circle $v(x)=r$. 

(ii)~ If there exists a side $A_p~ A_q$ of slope-$r$, then there exist
  at least two terms of minimum valuation. Therefore
  there may to be a zero of $f$ on the circle $v(x)=r$. Assume that
  $p<q$. Let $v(x_{\circ})=r$ for some $x_{\circ}$ in $K$ and
  $c=v(a_{q} x^{p}_{\circ})=v(a_{q} x^{q}_{\circ})$. Consider the
  power series  
  $$
  f_1(y)=a^{-1}_{q}x^{-q}_{\circ}f(x_{\circ}y)=\sum b_{i}y^{i}
  $$
  Obviously $v(b_{p})=v(b_{q})=0, v(b_{i})\geq 0$  for   $i\neq p,q$
  and $v(y)=0$  whenever $v(x)=r$. Hence without loss of generality we
  can take $r=0$, $v(a_{p})=v(a_{q})=0$, $v(a_{i})>0$ for $i<0$ and $i<p$ and
  $i>q$ and $a_{q}=1$. Therefore  
  \begin{align*}
    \overline{f(x)} &= x^{q}+\cdots+\overline{a_{p}} x^{p}
    &=x^{p}(x^{q-p}+\cdots+\overline{a}_{p})~\text{ where }~a_{p}\neq 0
  \end{align*}
  The polynomials $x^p$ and $(x^{q-p}+---+\overline{a}_{p})$ satisfy
  the requirements of Hensel's lemma, therefore there exists a monic
  polynomial $g$ of degree $q-p$ and a power series $h$, both with
  coefficients in $\mathscr{O}$, such that  
  $$
  \overline{g}=x^{q-p}+\cdots+\overline{a}_{p}, \overline{h}=x^{p}, f=gh
  $$\pageoriginale 
  and the radius of convergence of $h$ is equal to the radius of
  convergence of $f$. Let us assume that
  $g=x^{q-p}+\cdots+g_{\circ}$. Then
  $\overline{g_{0}}=\overline{a_{p}}\neq 0$. Let us further assume
  that $K$ is an algebraically closed field. Then $g$ has $q-p$ zeroes
  in $K$ which belong obviously to $\mathscr{O}^*$. Moreover $h$ has
  no zeroes on the circle $v(x)=0$. Thus $f$ has exactly $q-p$ zeroes
  on the circle $v(x)=0$ where $q-p$ is the length o the projection
  of the side of the Newton Polygon of $f$ with slope 0. If
  $\lambda_1, \lambda_2,\ldots,\lambda_{q-p}$
  are the zeroes of $f$, on $v(x)=0$ then $f=h\cdot
  \prod\limits^{q-p}_{i=1} 
 (x-\lambda_{i})$. We have also proved that if $f$ is a power
  series and $\lambda$ is its zero on a circle $v(x)=r>\rho(f)$, then
  $\dfrac{f(x)}{x-\lambda}$ is also a power series with the same
  radius of convergence. Regarding the zeroes of $f$ inside the circle
  $v(x)\geq r$ we prove the following. 

\setcounter{proposition}{0}
\begin{proposition}\label{part3:chap1:sec2:prop1}
  The power series $f$ has a finite number of zeroes
  $\lambda_1,\ldots,\break\lambda_{k}$ in the disc $v(x)\geq r>\rho (f)$
  and there exists a power series $h$ such that  
  $$
  f(x)=_{i} \prod^{k}_{1=1}(x-\lambda_i)\cdot h(x) ~\text{with}~ \rho
  (f)=\rho(h).    
  $$
\end{proposition}

\begin{proof}
  We have proved that $f(x)$ has zeroes on the circle
  $v(x)=r_{i}>\rho(f)$ if and only if there exists a side of the
  Newton Polygon of $f$ of slope $-r_{i}$. But we know that if
  $(m_{i})$ is the sequence of slopes of sides of the Newton Polygon of
  $f$, then $^{\lim}_{i\to \infty}m_{i}=-\rho(f)$. Therefore there
  exist only a finite number of sides of the Newton Polygon of slope
  $-r_1<-r<-\rho(f)$ \iec  there exists only a finite number of $r_1$
  such that $r_{1}>r>\rho (f)$ for which there are zeroes of $f(x)$ on
  $v(x)=r_{1}$. Hence the theorem follows.  
\end{proof}

If\pageoriginale $f(x)=\sum\limits_{i=0}^{\infty} a_{i}x^{i}$ is convergent in a
disc $v(x)>r$, then we shall say that f(x) is analytic $v(x)>r$. 

\begin{proposition}\label{part3:chap1:sec2:prop2}
  If $f(x)$ has no zeroes in the disc $v(x)\geq r > \rho (f)$ in
  particular $f(0)\neq 0$, then the power series $\dfrac{1}{f(x)}$ is
  analytic for $v(x)>r$. 
\end{proposition}

\begin{proof}
  Let us assume that $f(0)=1$. Since $f$ has no zeroes in $v(x) \geq
  r$, there exists no side of the Newton Polygon of $f$ of slope $
  \leq-r$. This implies that $\dfrac{v(a_i)}{i} \geq -r$ for every
  i. Considering $f$ as a formal power series over $K$ we get  
  $$
  \displaylines{\hfill 
  \frac{1}{f}=\frac{1}{1+\sum\limits_{i>0}
  a_{i}x^{i}}=\sum\limits^{\infty}_{k=0}(-1)^{k}\left(\sum_{i=0}a_{i}x^{i}\right)^{k}=
  \sum\limits^{\infty}_{j=0}b_{j}x^{j}  \hfill \cr
  \text{where}\hfill 
  b_{j}=\sum\limits_{k}(-1)^k
  \sum\limits_{i_{1}+i_{2}+\cdots+i_{k}=j}a_{i_{1}}\cdots a_{i_{k}} \hfill }
  $$
\begin{align*}
  \text{Therefore}\hspace{1cm}
   v(b_{j}) & \geq \inf_{\substack{k\\i, + -- + i_k =j}} \left(\sum^{k}_{l=1}
  v(a_{i1})\right)>-\sum_{l=1}^k r_{il}=-r_{j}\\
  & \Rightarrow \frac{v(b_j)}{j}>-r. 
\end{align*}
Hence ~$\rho \left(\frac{1}{f}\right)\geq r$.
\end{proof}

\begin{proposition}\label{part3:chap1:sec2:prop3}
  If $f$ \textit{is an entire
    function}(\iec  $\rho(f)=-\infty$) \textit{and has no zeroes, then
    $f$ is a constant.} 

  Let $f(x)=\sum^{\infty}_{i=0}a_{i}x^{i}$. As in the proof of the
  preceding proposition, we see that: 
  $$
  v(a_{j})\geq -rj\text{ for  any r. }
  $$
  Hence,\pageoriginale we have $a_{j}=0$ for $j\geq1$.
\end{proposition}

From these propositions, we can deduce the complete structure of
entire functions: 

\noindent \textbf{\textit{Weierstrass' Theorem}.} ~Let $K$ be an
algebraically closed 
complete field with a real valuation $v$. Let $f$ be an everywhere
convergent power series over $K$. Then the zeroes of $f$ different
from zero form a sequence
$(\lambda_{1},\lambda_{2},\ldots,\lambda_{n},\ldots,)$ such that
$v(\lambda_{n})$ is a decreasing sequence which tends to $- \infty$ if
the sequence $(\lambda_n)$ is infinite and we have  
\begin{equation*}
  f(x)=a_{\circ}x^{k}\prod^{\infty}_{i=1}
  \left(1-\dfrac{x}{\lambda_{i}}\right) \tag{1}\label{part3:chap1:sec2:eq1}
\end{equation*}
the infinite product being uniformly convergent in each bounded
subset of $K$. Conversely for any sequence $(\lambda_{n})$ such that
$v(\lambda_n)$ is a decreasing sequence tending to $-\infty$ as $n$
tends to infinity, the infinite product (1) is uniformly
convergent in every bounded subset of $K$ and defines an entire having
zeros at the prescribed points $\lambda_{n}$. 

\begin{proof}
  We shall  prove the latter part first. Consider 
  $$
  \displaylines{\hfill 
  \varphi_{N} (x) =\prod_{N}^{i=1}
  \left(1-\frac{x}{\lambda_{i}}\right) = \sum_{k=0}^{N}
  a_{kN} x^{k}, \hfill \cr
  \text{where}\hfill  
  a_{kN}=(-1)^k \sum\limits_{1\leq i_{1} < i_{2}<---<i_{k}\leq N}
  \frac{1}{\lambda_{i_1}\lambda_{i_2}\ldots \lambda_{i_k}}\hfill  }
  $$
  clearly $v(a_{kN})\geq
  +\left(v\left(\dfrac{1}{\lambda_{1}}\right)+\cdots + v
  \left(\dfrac{1}\lambda_{1}{\lambda_k} \right)\right)=
  \rho_{k}$. Since  
  $\lim\limits_{i\to \infty} v(\lambda_{i})=-\infty, \lim\limits_{k\to
    \infty} \frac{\rho k}{k} =\infty$.  Let  
  \begin{align*}
    \varphi(x) &=\prod_{i=1}^{\infty}
    \left(1-\frac{x}{\lambda_{i}}\right)=1+\sum^{\infty}_{k=1} a_{k}x^{k}, \text{
      where }\\ 
    a_k & =\sum_{1\leq i_1 < i_2 <\cdots < i_k}
    \frac{1}{\lambda_{i_1}\cdots\lambda_{i_k}+\lt\limits_{n\to \infty}
      a_{kn} } 
  \end{align*}\pageoriginale
  (obviously the series giving $a_{k}$ is convergent and
  $\dfrac{v(a_k)}{k} \geq \dfrac{\rho_k}{k}$). Therefore the series
  $\varphi(x)$represents an entire function. We have to show that the
  polynomials $\varphi_N$ converge to $\varphi$ uniformly on every
  bounded subset of $K$. Given two real numbers $M$ and $A$ there exists
  an integer $q$ such that $v(a_{kN}x^k)\geq M \text{ for }k \geq q$, 
  for all $x$ with $v(x)\geq A $and for all $N$, because
  $\dfrac{\rho_k}{k}\to \infty ~ as ~ k\to \infty $. This implies that
  for any $N$ 
  \begin{equation*}
    v \left(\varphi_N(x)-\sum^q_{k=0}a_{k_N}x^k\right)\geq M ~\text { for
    }v(x) \geq A.\tag{2}\label{part3:chap1:sec2:eq2}  
  \end{equation*}
  Similarly we get
  \begin{equation*}
    v\left(\varphi(x)-\sum^q_{k=0}a_{k_N}x^k\right)\geq M~\text { for
    }v(x) \geq A. \tag{3}\label{part3:chap1:sec2:eq3} 
  \end{equation*}
  Since $a_{kN}\to a_K $ as $N$ tends to infinity, combining (\ref{part3:chap1:sec2:eq2}) and (\ref{part3:chap1:sec2:eq3})
  we get $v(\varphi(x)-\varphi_N(k)) \geq M$ for $N$ sufficiently large. It can
  be easily proved that the $\lambda_i$ are the only zeroes of the
  function $\varphi(x)$.  
\end{proof}

Let us denote by $f_1$ the product given by
(\ref{part3:chap1:sec2:eq1}). Take a disc 
$v(x)\geq r$. In this disc $f(x)$ has only a finite number of
zeroes. Let the zeroes of $f$ in $v(x)\geq r$ be $0(k$ times) and
$\lambda_1,\lambda_2,\ldots\lambda_p$. Then  
$$
f(x)=x^k \prod^p_{i=1}\left(1-\frac{x}{\lambda_1}\right) g(x)
$$
where $g(x)$ has no zeroes in the disc $v(x)\geq r$. Therefore
$\dfrac{1}{g}$ is analytic in the disc $v(x)>r$. Consider
$\dfrac{f_1}{f}=\dfrac{g_1}{g}= \prod_{i=p+1}^\infty \left( 1-
\frac{x}{\lambda_i}\right) \frac{1}{g}$, where $g_1$ is analytic and has no
zeroes in the disc $v(x)>r$. Therefore $\dfrac{f_1}{f} is $ analytic
in the disc $v(x) > r$ and has no zeroes in it. Since it is true for
every $r, \dfrac{f_1}{f}$ is a constant function. Hence our theorem is
proved.  

Form  the proposition \ref{part3:chap1:sec2:prop2}, we can derive some
properties of the meromorphic functions: 
\begin{defi*}
  A\pageoriginale power series $\varphi =\sum\limits_{i=-m}^{\infty}
  a_ix^i$ over a 
  field $K$ is said to be a meromorphic function in a disc $v(x)\geq r$
  if and only if there exist two functions $f$ and $g$ analytic in the
  same disc such that $\varphi =\dfrac{f}{g}$. 
\end{defi*}

In any disc $v(x)\geq r' > r, g$ has a finite number of zeroes,
therefore $g=P g'$ where $P$ is a polynomial and $g'$ has no zeroes
 $ v(x) \geq r'$ which means that $\frac{1}{g'}$ is analytic in $v(x)>
r'$. Therefore we can can write $\varphi = \dfrac{f'}{P}$, where
$f'=f \dfrac{1}{g'}$ is a convergent power series in $v(x)>r'$. 

\section{Criterion for the Rationality of power-series}\label{part3:chap1:sec3}

Let $F$ be any field and $f=\sum^{\infty}_{k=0} a_kx^k$ an element in
$F\big[[x]\big]$. It can be easily proved that $f$ is a rational
function if and only if there exists a finite sequences $(q_i)_{\circ
  \leq i \leq h}$ of elements of $F$ at least one of which is non-zero
and an integer $k$ such that  
$$
a_n q_h+a_{n+1} q_{h-1} + \cdots +a_{n+h} q_o=0
$$
for all integers $n$ such that $n+h>k$. Let us denote by $A^{h+1}_n$
the determinant of the matrix $(a_{n+i+j})_{0\leq i, j\leq h}$. 

\setcounter{Lemma}{0}
\begin{Lemma}\label{part3:chap1:sec3:lem1}
  The power series $f$ is a rational function if and only if there
  exists integer $h$ and $n_{\circ}$ such that $A_n^{h+1}=0$ for all
  $v > n_\circ$. 
\end{Lemma}

\begin{proof}
  It is obvious that the condition is necessary. We shall prove that
  the condition is sufficient by induction on $h$. When $h=0$, we have
  $a_n=0$ for $n$ sufficiently large. Therefore $f$ is actually a
  polynomial. Let us assume that $A^{h+1}_n=0$ for
  $n>n_{\circ}$. Moreover we may assume that $A^{h}_n\neq0$ for
  infinitely many $n$, because if $A^{h}_n=0$ for $n$ large then by
  induction hypothesis we get that $f$ is a rational
  function. Since\pageoriginale 
  $A^{h+1}_n=0$ for $n>n_{\circ}, A_n^h ~ A^h_{n+2} =
  \left(A^h_{n+1}\right)^2$. So it follows that $A^h_n \neq 0$ for $n >
  n_\circ$. Consider the following system of linear equations 
  $$
  E_r=a_{n_{\circ+r}}
  x_1+a_{n_{\circ+1+r}} x_2+\cdots+a_{n_{\circ+h+r}} x_{h+1}=0
  ~\text{for}~ r=0,1,2,... 
  $$
  For any $q\geq n_{\circ}$ the system $\sum_q$of the $h$ if $h$
  equations $E_q, E_{q+1}, \ldots E_{q+h-1}$ is of rank $h$ (because
  $A_{q}^{h}\neq 0$). So has a unique solution upto a constant factor.
  But the system $\sum'_q$ of the $h+1$ equations $E_q,\ldots ,
  E_{q+h}$ is also of rank $h$ (because $A_{q+1}^h \neq 0$ and
  $A^{h+1}_{q}=0$) and therefore $\sum'_q$ and
  $\sum_{q+1}$ on the hand and $\sum'_q$ and $\sum q+1$ on the other hand
  have the same solution. Thus any solution of $\sum_q$  is a solution
  of $\sum_{q+1}$ and any solution of $\sum_{n_o}$ is a solution of
  $E_{q}$ for $q\geq n_\circ$. Thus we have found a finite sequence
  $(x_i)$ such that $a_{n_\circ+r}x_1+\cdots+a_{n_\circ+h+r}x_{h+1}=0$ for
  $r\geq 0$. Hence $f$ is a rational function. 
\end{proof}

\setcounter{theorem}{0}
\begin{theorem}\label{part3:chap1:sec3:thm1}
  Let $f(x)=\sum a_i x^i$ be a formal power series with coefficients in
  $Z$. Let $R$ and $r$ be two real numbers such that  
  \begin{enumerate}[\rm (1)]
  \item $R r >1$
  \item $f$\pageoriginale considered as a power series over the field of complex
    numbers is holomorphic in the disc $|x|<R$. 
  \item $f$ considered as a power series over $\Omega_p$ (the complete
    algebraic closure of $Q_p$) is meromorphic in the disc $|x|\leq r'$
    with $r'>r$. (where $||_p$) is the absolute value associated to
    $v_p$). Then $f$ is a rational function.  
  \end{enumerate}
\end{theorem}

\begin{proof}
  We can assume that $R\leq 1$, because $R>1$ implies that $f$ is a
  polynomial and we have nothing to prove.  Moreover $r>1$, because
  $Rr>1$. Since $f$ is meromorphic in $|x|_p \leq r'$, there exist two
  functions $g$ and $h$ analytic in $|x|_p \leq r$ such that $f =
  \frac{g}{h}$. If necessary by multiplying $f$ by a suitable power of
  $x$ we can assume that $f$ has
  no pole at $x=0$ and hence that $h$ is polynomial with
  $h(0)=1$. Let  
  $$ 
  g \sum^\infty_{i=0} g_i x^i ~\text{and}~ h=\sum^k_{i=0} h_i x^i
  $$
  \begin{equation}
    g_{n+k}=a_n h_k+a_{n+1}h_{k-1}+\cdots
    a_{n+k-1}h_1+a_{n+k}\tag{1}\label{part3:chap1:sec3:eq1} 
  \end{equation}
  By Cauchy's inequality we obtain the following
  \begin{enumerate}[(1)]
  \item $|a_s|\leq MR^{-s}$
  \item $|g_s|\leq Nr^{-s}$
  \end{enumerate}
  By taking $R $ and $r$ smaller if necessary we assume that $|a_s|\leq
  R^{-s}$ and $|g_s|_p \leq r^{-s}$ for $s>s_\circ$. Let 
  \begin{equation*}
    A^{m+1}_n=
    \begin{vmatrix}
      a_n    & a_{n+1}\cdots  & a_{n+k}     & a_{n+m}  \\
      a_{n+1} & a_{n+2} 	     & a_{n+k+1}   & a_{n+m+1} \\
      \cdots & \cdots	     &\cdots	    &\cdots\\
      a_{n+m} & a_{n+m+1}	     & a_{n+m+k}\cdots & a_{n+2m}
    \end{vmatrix}
  \end{equation*}
  where $m>k$. 

The\pageoriginale equation (\ref{part3:chap1:sec3:eq1}) gives 
  \begin{equation*}
    A^{m+1}_n=
    \begin{vmatrix}
      a_n & a_{n+1}\cdots & a_{n+k-1}  & g_{n+k} & g_{n+m} \\
      a_{n+1} & a_{n+2}& a_{n+k} & a_{n+k+1} & \cdots\\
      \cdots	& \cdots&\cdots  &\cdots & \cdots\\
      a_{n+m} & a_{n+m+1}	& a_{n+m+k-2} & g_{n+m-k}  & r_{2+2m}
    \end{vmatrix}
  \end{equation*}
  Obviously for $n>s_o$ we have 
  $$
  \displaylines{\hfill 
  |A^{m+1}_n| \leq (m+1)! (R^{-(n+2m)})^{m+1}\hfill \cr
  \text{and }\hfill 
  |A^{m+1}_n|_p \leq (r^{-n})^{m-k+1}\hfill }
  $$
  because $|a_n|_p \leq 1$ for every $n$. If $A^{m+1}_n  \neq 0$, then 
  $$
  1\leq |A^{m+1}_n| |A^{m+1}_n|\leq (m+1) ! R^{-2m(m+1)}
  r^{kn}[Rr]^{-n(m+1)} = k_1[(R~r)^{m+1}r^{-k}]^{-n} 
  $$
  Let $m$ be so chosen that $(R r)^{m+1} r^{-k}>1$. Then there exists
  an integer $n_\circ$ such that for $n>n_\circ$ 
  $$
  |A^{m+1}_n||A^{m+1}_n|<1.
  $$

  This is a contradiction. Therefore $A^{m+1}=0$ for
  $n>n_\circ$. Hence $f$ is a rational function. 
\end{proof}

\begin{coro*}
  If $f$ is a power series over $Z$ such that $f$ has a non-zero
  radius of convergence considered as series over the complex number
  field is meromorphic in $\Omega_{p}$, then $f$ is a rational
  function.  
\end{coro*}

\section{Elementary Functions}\label{part3:chap1:sec4}

We\pageoriginale consider the convergence of the exponential logarithmic and
binominal series in this section. We assume that the field $K$ is of
characteristic $0$ and the real valuation $v$ on $Q$ induces a
$p$-adic valuation.  

The exponential series
$e(x)=\sum\limits^{\infty}_{n=0}\dfrac{x^n}{n!}$. Converges in the
disc $v(x)>\dfrac{1}{p-1}$ and in the domain of convergence
$v(e(x)-1)=v(x)$. Let $n=a_\circ+a_1p+\cdots+a_r p^r$ where $p^r\leq n
\leq p^{r+1}$ and $0\leq a_i \leq p-1$. One can easily prove that  
$$
v(n!)=\left[\frac{n}{p}\right] +---+ \left[\frac{n}{p^r}\right]
=\frac{n-S_n}{p-1} 
$$
where $S_n=\sum\limits^r_{i=0}a_i$
Therefore 
$$
\therefore{\dfrac{v(\frac{1}{n!})}{n}} = \frac{-1}{p-1}+\frac{S_n}{n(p-1)}
$$
But $\dfrac{S_n}{p-1}\leq\left(\dfrac{\log ~n}{\log p}+1\right)$. Therefore
$\lim\limits_{n\to\infty} ~\dfrac{v\left(\frac{1}{n!}\right)}{n}
=\dfrac{-1}{p-1}$.  
Hence the series $e(x)$ converges for $v(x) > \dfrac{1}{p-1}$. If
$v(x)=\frac{1}{p-1}$, then
$\dfrac{v(x^n)}{n!}=\dfrac{1}{p-1}<\infty$ whenever $n$ is a power of
$p$. Thus the series does not converge for $v(x)=\dfrac{1}{p-1}$. The
latter part of the assertion is trivial. We see immediately that
$e(x+y)=e(x).e(y)$ and $e(x)$ has no zeroes in the domain of convergence.  

We define log $(1+y)=\sum\limits^\infty_{k=1}(-1)^{k-1}\dfrac{y^k}{k}$
as a formal power series over $K$. We shall show that the series
$\log(1+y)$ converges for $v(y)>0$ and\pageoriginale $v(\log (1+y))=v(y)$ for  $v(y)
>\dfrac{1}{p-1}$ 
we have 
$$
v \left(\frac{(-1)^ny^n}{n}\right)= n v(y)-v(n)
$$
But $v(n)\leq \dfrac{\log n}{\log p} $ therefore
$^v\bigg(\dfrac{(-1)^ny^n}{n}\bigg)$ tends to infinity as $n\to\infty$
whenever $v(y)>0$. On the other hand $v(n)=0$ if $(n,p)=1$, therefore
the series is not convergent for $v(y)\leq 0$. For $n~ >1$ and
$v(y)>\dfrac{1}{p-1}$, it can easily proved that
$v\bigg(\dfrac{(-1)^{n-1}y^n}{n}\bigg)>v(y)$, which proves our last
assertion. Moreover for $v(x)>\dfrac{1}{p-1}$ we have the equalities 
\begin{align*}
  & e(\log(1+x))=1+x \tag{1}\label{part3:chap1:sec4:eq1}\\
  & \log(e(x))= x\tag{2}\label{part3:chap1:sec4:eq2}
\end{align*}
Let
\begin{align*}
  G &= \left\{ x|x \in K, v(x) > \frac{1}{p-1}\right\}\\
  G &= \left\{ x + 1| x\in K, v(x)>\frac{1}{p-1}\right\}
\end{align*}
be subgroups of $K_+$ (the additive group of $K$) and $K^\ast$
respectively. The
mapping $x\to e(x)$ is an isomorphism of $G$ onto $G'$, the inverse
of which  is the mapping $1+x\to\log(1+x)$. In fact the mapping
$1+y\to\log (1+7)$ is a homomorphism of the group
$1+\mathscr{Y}_\Omega(\Omega$ begin the complete algebraic closure of
$K$) into the subgroup of $\Omega_+$, where $v(y)>0$.It is not an
isomorphism because it $\zeta$ is a $p-th$ root of unity, then
$v(\zeta-1)=\dfrac{1}{p-1}$ and log $\zeta=0$. 

We define $(1+Y)^Z=\sum\limits^\infty_{m=0}h(m,Z)Y^m=e(Z \log(1+Y))$
where\break $h(m,Z)=\dfrac{Z(Z-1)\cdots(Z-m+1)}{m!}$ as a formal power
series in the variables $Y$ and $Z$ over $K$. Since $h(m,Z)$ is a
polynomial in $Z$, we can substitute for $Z$ any element of $K$ to
get a power series in the one variable $Y$.  

\begin{proposition}\label{part3:chap1:sec4:prop4}
  For\pageoriginale any element $t$ in $K$ the power function $(1+x)^t$ defined
  above is analytic for $v(x)>\dfrac{1}{p-1}$ (respectively for
  $v(x)>-v(t)+\dfrac{1}{p-1}$) if $v(t) \geq 0$ (respectively if
  $v(t)<0$) Moreover if $t$ belongs 
  $Z_p$, then $(1+x)^t$ is analytic for $v(x)>0$. 
\end{proposition}

\begin{proof}
  When $v(t) < 0$  
  $$
  v(h(m,t))=m(v(t))-v(m!)\geq m v(t) - \frac{m-1}{p-1}
  $$

  Therefore $^{\lim \inf}_{m\to\infty} \dfrac{v(h(m,t))}{m} = v(t)-
  \dfrac{1}{p-1}$  
  Hence $(1+x)^t$ is analytic in $v(x)>\dfrac{1}{p-1}-v(t)$. Similarly
  one can prove the convergence when $v(t)\geq 0$.  
\end{proof}

Let $t$ be in $Z_p$. Then $h(m,t)$ is a $p$-adic integer. Suppose that
$v(m!)+1=\alpha$, then there exists an element $k_m$ in $Z$  such that  
$$
t\equiv k_m \pmod{p^k}
$$
Therefore 
$$
\displaylines{\hfill 
t(t-1)\ldots(t-m+1)\equiv k_m(k_m-1)\hfill \cr
\text{~or~}\hfill  h(m,t)\equiv h(k_m,m) \pmod{p}.\hfill }
$$
But $h(k_m, m)$ is a rational integer, therefore $v(h(m,t))\geq
0$. From this our assertion follows easily.  

\section{An Auxiliary Function}\label{part3:chap1:sec5}

Throughout our discussion $F_q$ shall denote a finite field consisting
of $q$ elements. Let us consider the infinite product  
\begin{equation*}
  F(Y,T)=(1+Y)^T (1+Y^P)^{\dfrac{T^P-T}{P}}
  (1+Y^{P^m})^{\dfrac{T^{p^m}-T^{p^{m-1}}}{p^m}} \tag{1}\label{part3:chap1:sec5:eq1} 
\end{equation*}

The product is well defined as formal power series in two variables $Y$
and $T$ over $Q$. Clearly (\ref{part3:chap1:sec5:eq1}) is convergent in $Q
\bigg[\big[Y,T\big]\bigg]$. Expressing $F(Y,T)$ as a power series over
$Q \bigg[\big[T\big]\bigg]$ and $Q \bigg[\big[Y\big]\bigg]$ we obtain 
\begin{align*}
  F(Y,T)&=\sum^\infty_{m=0} B_m(T)Y^m, d(B_m(T)) \le m\\
  &= \sum^\infty_{m=0} \alpha_m (Y)T^m, 
\end{align*}\pageoriginale
where $\alpha_m(Y)$ is a power series,  the terms being of degree
$\ge$ $m$. 

\begin{Lemma}\label{part3:chap1:sec5:lem2}
  The coefficients of $F(T,Y)$ are $p$-adic integers.
\end{Lemma}

\begin{Lemma}\label{part3:chap1:sec5:lem3}
  If $F$ is an element of $Q \bigg[ [ Y,Z ] \bigg]$ such that
  $F(0,0)=1$, then $F$ belongs to $Z_p\bigg[ [ Y,Z ] \bigg]$ 
  if only  if the coefficients of $\dfrac{(F(Y,Z))^p}{F(Y^p,Z^p)}$ are
  in $p Z_p$ excepts for the first.   
\end{Lemma}

\noindent \textbf{Proof of Lemma 3.}
Let us suppose that $F(Y,Z)=1- \sum\limits_{i+j>0} a_{ij}Y^i Z^j$,  then  
\begin{align*}
  G&= \frac{(F(Y,Z))^p}{F(Y^p,Z^p)} = F_1 x F_2  \quad \text{where} \\
  F_1 &= 1-p \sum\limits_{i+j>0} a_{ij}y^i Z^j +\cdots+ (^p_r)(-1)^r
  \left(\sum\limits_{i+j>0} a_{ij} Y^i Z^j\right)^r \\ 
  & \hspace{2cm}+ \cdots +(-1)^p \left(\sum\limits_{i+j>0} a_{ij}Y^i
  Z^j\right)^p.\\ 
  F_2 &=1 + \sum \limits_{k=1}^{\infty} \left(\sum_{i+j>0}
  a_{ij} Y^{pi}Z^{pj}\right)^k
\end{align*}
If $G=1+ \sum\limits_{i+j>0} b_{ij}Y^i Z^j$, then 

$b_{ij}= -pa_{ij}+$ (terms of the form $p X$ polynomials in a with
rational integers coefficients with  
\begin{align*}
  r+s <  i+  j) + \sum \limits_{k=1}^\infty \sum a_{i_1}& j_1 \ldots a_{ik}j_k\\
   & i_1 + \cdots + i_k  =i'\\
   &  j_1 + \cdots +j_k  = j'\\
   & i_r + j_r > 0\\
   + (-1)^p \sum \limits_{k=1}^\infty a^p_{i_1}j_1 a_{i_2} & j_2
   \cdots a_{i_k} j_k  i_1 + \cdots + i_k =i'\\
   & j_1 +\cdots + j_k =j'\\ 
   & i_r + j_r >  0
\end{align*}
where\pageoriginale the last two sums appear only if $i$ and $j$ are divisible by
$p$ and in this case $p i' = i, p j' =j$.

Assume that $b_{ij}$ belongs to $p Z_p$ for $i+ j > 0$. We shall prove
that $a_{ij}$ are in $Z_p$ by induction. Obviously $a_{00}$ is in
$Z_p$. Assume that $a_{rs}\in Z_p$ for $r+s < i+j$;  then in the
formula giving $b_{ij}$ all the terms except perhaps $- pa_{ij}$.  But
$a- a^p$ belongs to $p Z_p$ if a belongs  to $Z_p$, therefore
$pa_{ij}$ belongs to $p Z_p$ and $a_{ij}$ belongs $Z_p$. The other
part of the assertion  is trivial 

\textbf{Proof of Lemma 2.}
\begin{align*}
  \frac{(F(Y,T))^p}{F(Y^p, T^p)} & = \frac{(1+Y)^{pT}
  \prod\limits_{m=1}^{\infty} (1+Y^{p^m})^{\dfrac{T^{pm}-
        T^{p^{m-1}}}{p^{m-1}}}}{(1+Y^p)^{T^p}\prod\limits
    _{m=2}^{\infty}(1+Y^{p^m})^{\dfrac{T^{p^m}-
        T^{p^{m-1}}}{p^{m-1}}}}\\ 
  &= \left[\frac{(1+Y)^p}{(1+Y^p)} \right]^T\\
  &= \left[ a+p \sum\limits_{k=1}^\infty b_k Y^k \right]^T
\end{align*}
where $b_k$ are $p$-adic integers. 

Moreover 
$$
\left(1+p\sum_{k+1}^\infty b_k Y^k\right)^T= \sum_{m=0}^\infty h(m,T) p^m \left(\sum
_{k=1}^\infty b_k Y^k\right)^m 
$$

But  $\dfrac{v(p^m)}{m!}\ge m - \dfrac{m-1}{p-1} > 0$, therefore
$\dfrac{F(Y,T)}{F(Y^p, T^p)}  ^p -1$ has its coefficients in $p
Z_p$. Thus by lemma (\ref{part3:chap1:sec5:lem3}) the coefficients of
$F(Y,T)$ are $p$-adic integers. 

One\pageoriginale deduces from lemma (\ref{part3:chap1:sec5:lem2}) that $F(y,t)$ is analytic for $v(t)\ge 0$ 
and $v(y) > 0$, because if $v(t)\ge 0$, then $v(B_m(t)) \ge 0$
because $B_m (t)$ is a polynomial with coefficients from
$Z_p$. Therefore the series $\sum \limits _{m=0}^\infty B_m(t)y^m$
converges for $v(y) > 0$. 

\section[Factorisation of additive characters of a Finite
  Fields]{Factorisation of additive characters of a\hfil\break Finite
  Fields}\label{part3:chap1:sec6} 

Let $\mathscr{R}_s = \bigg\{x\mid x \in \Omega_p = \Omega, x^{P^s} = x
\bigg\}$. We have the canonical map from $\mathscr{R}_2$ to $F_{p^s}$
namely the restriction on the canonical homomorphism of
$\mathscr{O}_\Omega$ onto $k_\Omega$. In order t prove that this map
is  bijective,  it  is sufficient to prove  that is surjective;
because both $\mathscr{R}_s$ and $F_{p^s}$ have  $p^s$ elements.  If
$\bar{x} \neq 0$ is in $F_{p^s}$, then $\bar{x}^{p^s -1}-1=0$ and
$\bar{x}$ is a simple root of the polynomial $X^{p^s -1}-1
$. Therefore by Hensel's lemma there exists an element $\alpha$
belonging to $\Omega$ such that $\bar{\alpha}= \bar{x}$ and
$\alpha^{p^s -1}-1=0$, which proves that $\alpha$ is in
$\mathscr{R}_2$ and the mapping is onto.  Infact the canonical
homomorphism of $\mathscr{O} \Omega$ onto $k_\Omega$ when restricted
to $\mathscr{R}= \bigcup\limits_{s=1}^{\infty} \mathscr{R}_2$ is an
isomorphism onto $k_\Omega$. Finally Hensel's lemma shows that  $R_1$
is contained in $Q_p$.  

Let $U_s = Q_p (\mathscr{R}_s)$. Clearly $U_s$ is a Galois extension
of $Q_p$ and the Galois group is cyclic generated by the automorphism
$\sigma: \rho  \to \rho^p $, where $\rho $ is a primitive $p^s-1$ th
root of unity.  Moreover $U_s$ is an unramified extension  of $Q_p$,
because $[ U_s; Q_p] =  [ F_{p^s}; F_p]$. If we take  $U=
\bigcup \limits_{s=1}^\infty U_s$, then the completion of $U$  is the
maximum unramified extension of $Q_p$ in $\Omega$ and $\sigma$ is
called the Frobenius automorphism of $U$. If $t'$ is an elements is
$\mathscr{R}_2$, then  
$$ 
\underset{U_s}{\Tr} \underset{/ ~ Q_p}{t'} = t'+t^{'p}+ \cdots + t ^{'p^{s-1}}
$$
belongs\pageoriginale to $Z_p$. Thus the function $(1+Y)^{\Tr t'}$ is analytic for
$v(y) > 0$. Let $t'$ be the representative of $t \in F_{p^2}$ in
$\mathscr{R}_2$. If $y$ belongs $y$ belongs to $\mathscr{Y}_ \Omega$
then $(1+y)^{\Tr t'}$ belongs to $\Omega$. We shall choose $y$ in such
a way that mapping  $t \to (1+y)^{\Tr t'}$ is a character of the
additive group of $F_{p^s}$. Obviously for any $u$ and $v$ in
$F_{p^s}$ we have  
\begin{align*} 
  (u+v)' & \equiv u' +v' ~\pmod {\mathscr{Y}_\Omega}\\
  \Tr (u' + v' ) &\equiv \Tr u' + \tr v' ~\pmod {\mathscr{Y}_\Omega}\\
  &\equiv \Tr u' + \Tr v' ~\pmod {p Z_p}
\end{align*}
because $\Tr u'$ is a $p$-adic integer. Therefore
$$
(1+y)^{\tr(u+v)'}=(1+y)^{\Tr u'}(1+y)^{\Tr v'}(1+y)^a, 
$$
where a belongs to $p Z_p$. Let us take $1+y= \zeta$ where $\zeta^p
=1$ and $\zeta \neq 1$. It follows that $(1+y)^a=1$. Thus the mapping $u
\to \zeta^{\Tr~u'}$ is a character of  $F_{P^s}$. We shall show that it is a non
-trivial character. Firstly,  $\zeta^a =1$ if and only if a belongs to
$p Z_p$ proved that a already belongs to $Z_p$. For by choice of $y$
we have  $v(y)= \dfrac{1}{p-1} > 0$ and 
$$
\zeta^a = (1+y)^a = 1+ ay + \cdots + h(m,a)y^m + \cdots 
$$ 

Since a is $p$-adic integer,  $v(h(m,a) \ge 0$ and hence $v(h(m,a)y^m
\ge \dfrac{2}{p-1}$ for $m \ge 2, (a+y)^a \neq 1$ if $v(ay) <
\dfrac{2}{p-1}$. Therefore $v(ay)\ge \dfrac{2}{p-1}$, which  implies
that $v(a)\ge \dfrac{1}{p-1} < 0$, thus a belongs to $p
Z_p$. But the canonical image of  $\Tr u'$ in $F_p$ is the trace of $u$
as an element of $F_{p^s}$ over  $F_p$, therefore there exists as
least one $u'$ such that $\Tr u'$ is not in $p Z_p$. Hence the mapping
$u \rightarrow \zeta^{\Tr u'}$ is a non-trivial character of
$F_{p^s}$. By definition of the product $F(Y,T)$ we have 
\begin{align*}
  F(y, u') & = (1+y)^{u'} \cdots (1+ y^{p^m})^{\dfrac{u'^{p^m} -u'^{p^{m-1}}}{p^m}} \\
  F(y,u'^p) &= (1+ y)^{u'p} \cdots (1+ y^{p^m})^{\dfrac{u'^{p^{m+1-}-u'^{p^m}}}{-p^m}}\\
  F(y,u'^{p^{s-1}}) & = (1+y)^{u'^{p^{g-1}}} \cdots (1+
  y^{p^m})^{\dfrac{u'^{p^{m+s-1}}-u'^{p^{m+s-2}}}{p^m}} 
\end{align*}\pageoriginale
Since $u'^{p^s}= u'$, by multiplying these identities we get 
$$
\prod_{r=0}^{s-1} F(y,u'^{p^r})= (1+y)^{\Tr u'}
$$

Thus $\zeta^{\Tr u'}= \prod\limits_{k=0}^{s-1} \varphi(u'^{p^k})$ where
$\varphi(T)= F( \zeta -1, T)$, is the splitting of additive
characters of $F_{p^s}$ which we shall require later. 
