\thispagestyle{empty}

\begin{center}
{\Large\bf Lectures on Some Aspects of}\\[5pt] 
{\Large\bf {\boldmath $p$-}Adic Analysis}\\[5pt]
\vskip 1cm

{\bf  By}
\medskip

{\large\bf  F. Bruhat}
\vfill


{\bf  Notes by}
\medskip

{\large\bf  Sunder Lal}
\vfill


\parbox{0.7\textwidth}{
No part of this book may be reproduced in any form by print,
microfilm or any other means without written permission from the
Tata Institute of Fundamental Research, Colaba, Bombay-5}
\vfill

{\bf  Tata Institute of Fundamental Research}

{\bf  Bombay}

{\bf  1963}
\end{center}

\eject

\thispagestyle{empty}

\chapter{Introduction}


These lectures are divided in three parts, almost independent part $I$
is devoted to the more less classical theory of valuated  fields
(Hensel's lemma, extension of a valuation, locally compact fields,
etc.,) 

In the second part, we give some recent results about representations
of classical groups over a locally compact valuated field. We first
recall some facts about induced representations of locally compact
groups and representations of semi-simple real Lie groups (in
connexion with the theory of ``spherical functions"). Afterwards, we
construct a class of maximal compact subgroups $K$ for any type of
classical group $G$ over a $p$-acid field and the study of the left
coset and double coset modulo $K$ decomposition of $G$ allows us to
prove the first results about spherical functions on $G$. Some open
problems are indicated.  

Part $III$ is devoted to Dwork's proof of the rationality of the zeta
function of an algebraic variety over a finite field. We first need
some results (well known, but nowhere published) about analytic and
meromorphic functions on an algebraically closed complete valuated
field. Then we settle the elementary facts about the zeta function of
a scheme (in the sense of Grothendieck) of finite type over $Z$ and we
give,  following Dwork, the proof of the rationality of these zeta
functions 
