\part{Classical Theory of Valuated Fields}\label{part1}

\chapter{Theory of Valuations-I}\label{part1:chap1}

In\pageoriginale this and the next chapter we give a short account of the classical
theory of valuated fields. Unless otherwise stated by a ring we mean a
commutative ring with the unit element $1$ and without zero
divisors. 

\section{}\label{part1:chap1:sec1}

\begin{defi*}
  Let $A$ be a ring and $\Gamma$ a totally ordered comutative group
  \cite{1}. A valuation $v$ of the ring $A$ is a mapping from $A^*$ (the
  set of non-zero elements of $A$) into $\Gamma$ such that  
  \begin{enumerate}[\rm(I)]
  \item $ v(xy) = v(x) + v(y)$ for every $x$, $y$ in $A^*$.
  \item $v(x + y ) \geq \inf (v(x)$, $v(y))$ for every $x$, $y$  in $A*$.
  \end{enumerate}
  We extend $v$ to $A$ by setting $v(0)=\infty$; where $\infty$ is an
  abstract element added to the group $\Gamma$ satisfying the equation  
  $$
  \infty + \infty = \alpha + \infty = \infty + \alpha = \infty ~\text{
  for }~ \alpha ~\text { in }~ \Gamma.  
  $$
  We assume that $\alpha < \infty $ for every $\alpha$ in
  $\Gamma$. The valuation $v$ is said to be improper if $v(x)=0$ for
  all $x$ in $A^*$, otherwise $v$ is said to be proper. 
\end{defi*}

The  following are immediate consequences of our definition. 
\begin{enumerate}[(a)]
\item $v(1) = 0$. For, $v (x.1)=v(x)=v(x)+v(1)$, therefore $v(1)=0$ 
\item If\pageoriginale for $x$ in $A$, $x^{-1}$ is also in $A$, we have $v(x^{-1}) =
  -v(x)$, because $v(1)= v (x x^{-1})= v(x)+v(x^{-1})=0$ 
\item If $x$ is a root of unity, then $v(x) = 0$. In particular
  $v(-1)=0$, which implies that $v(-x) = v (x)$ 
\item Por $n$ in $Z$ (the ring of integers)
  $$
  v(n) = v(1+ ---+1) \geq \inf (v(1)) = 0.
  $$
\item If for $x$, $y$ in  $A$ $v (x)\neq v (y)$, then $v(x+y)= \inf (v(x),
  v(y))$. Let us assume that $v(x) > v (y)$ and $v(x +y )> v
  (y)$. Then $v(y)=v(x + y - x ) \geq \geq \inf (v(x+y), v(-x))>
  v(y)$, which is impossible.  
\end{enumerate}

If $x_i$ belongs to $A$ for $i =n, 1, 2, \ldots, n$, then one can prove
by induction on $n$ that $v(\sum\limits^n_{i=1} x_i) \geq
\inf\limits_{1\leq i \leq n}(v(x_i))$ and that the equality holds if
there exists only one $j$ such that $v(x_j)= \inf\limits_{1\leq i
  \leq n}(v(x_i))$. In particular if $\sum\limits^n_{i = 1} x_i = 0 \,(n
\geq 2)$ then $v(x_i)= v(x_j)= \inf\limits_{1\leq k \leq n}(v(x_k))$
for at least one pair of unequal indices $i$ and $j$. For, let $x_i$
be such that $v(x_i)\leq v(x_l)$ for $i \neq l$. Then $v(x_i) \geq
\inf\limits_{1\leq k \leq n~ k \neq i} (v (x_k)) = v(x_j)$, which
proves that $v(x_i)= v(x_j)$. 

Obviously we have  
\begin{prop}\label{part1:chap1:sec1:prop1}
  Let $A$ be a ring with a valuation $v$. Then there exists one and
  only one valuation $w$ of the quotient field $K$ of $A$ which
  extends $v$. 

  It is seen immediately that $w\left(\dfrac{x}{y}\right)= v (x)-v(y)$
  for $x,y$ in $A$. 
\end{prop}

So\pageoriginale without loss of generality we can confine ourselves
to a field. The 
image of $K^*$ (the set of non-zero elements of field $K$) by $v$ is a
subgroup of $\Gamma$ which we shall denote by $\Gamma_v$ 
\begin{prop}\label{part1:chap1:sec1:prop2}
  Let $K$ be a field with a valuation $v$. Then 
  \begin{enumerate}[\rm (a)]
  \item The set $\mathscr{O} = \{x | x \in \,K, v (x) \geq 0 \}$ is a
    subring of $K$, which we shall call the ring of integers of $K$
    with respect to the valuation $v$. 
  \item The set $\mathscr{Y} = \{x | x \in\, K, v (x) > 0 \}$ is an ideal
    in $\mathscr{O}$ called the ideal of valuation $v$. 
  \item $\mathscr{O}* = \mathscr{O} - \mathscr{Y}= \{x | x \in\,K, v (x) =
    0 \}$ is the set of inversible elements of $\mathscr{O}$ 
  \item $\mathscr{O}$ is a local ring (not necessarily Noetherian) and
    $\mathscr{Y}$ is the unique maximal ideal of $\mathscr{O}$. 
  \end{enumerate}
  We omit the proof of this simple proposition. The field $k =
  \mathscr{O}/ \mathscr{Y}$ is called the residual field of the valuation $v$. 
\end{prop}

It is obvious form the proposition \ref{part1:chap1:sec1:prop2} that
the valuation $v$ of $K$ 
which is a homomorphism form $K^*$ to $\Gamma $ can be
split up as follows  
$$
K^* \xrightarrow{v_1} K^* / \mathscr{O}^* \xrightarrow{v_2} \Gamma_v
\xrightarrow{v_3} \Gamma. 
$$
where $v_1$ is the canonical homomorphism, $v_2$ the map carrying an
element $x \mathscr{O}^* $ to $v (x)$ and $(v_3)$  the inclusion map
of $\Gamma_v$ into $\Gamma$. 

\begin{defi*}
  Two valuations $v$ and $v'$  of a field $K$ are said to be
  equivalent if there exists an order preserving isomorphism $\sigma$
  of $\Gamma_v$ onto $\Gamma_V'$ such that 
  $$
  v' = \sigma ~\circ~ v.
  $$
\end{defi*}

From\pageoriginale the splitting of the homomorphism $v$ it is obvious that a
valuation of a field $K$ is completely characterised upto an
equivalence by any one of $\mathscr{O}$, or $\mathscr{Y}$. 

A valuation of a field $K$ is said to be \textit{real} if $\Gamma_v$
is contained in $R$ (the field fo real numbers). Since any subgroup of
$R$ is either discrete \iec  isomorphic to a subgroup of integers or
dense in $R$, either $\Gamma_v$ is contained in $Z$ or $\Gamma_v$ is
dense in $R$. In the former case we say that $v$ is a discrete
valuation and in the latter non-discrete. Moreover $v$ is completely
determined upto a real constant factor, because if $v$ and $v'$ are
two non-discrete equivalent valuations of $K$, the isomorphism of
$\Gamma_v$ onto $\Gamma_v'$ can be extended to $R$ by continuity,
which is nothing but multiplication by a element of $R$. If $v$ and
$v'$ are discrete and equivalent, the assertion is trivial. If
$\Gamma_v = z$ we call $v$ a normed discrete valuation.  

\begin{defi*}
  Let $K$ be a field with a normed discrete valuation $v$. In $K$ we
  can find an element $\pi$ with $v(\pi)=1$. The element $\pi$ is
  called a uniformising parameter for the valuation $v$. 
\end{defi*}

Let $K$ be a field with a normed discrete valuation $v$ and
$\mathscr{O} \neq (0)$ an ideal in $\mathscr{O}$.  Let $\alpha =
\inf\limits_{x \in \mathscr{O}} (v(x))$. Such an $\alpha$ exists
because $v(x) > 0 $ for every $x$ in $\mathscr{O}$. Moreover there
exists an element $x_0$ in $\mathscr{O}$ such that $v
(x_\circ)= \alpha$, because the valuation is discrete. Then
$\mathscr{O} = \mathscr{O}x_0 = \mathscr{O} 
\pi^\alpha$ For,\pageoriginale $x$ belongs to $\mathscr{O}
\Longleftrightarrow v(x) 
\geq v(x_0)\Longleftrightarrow v (\dfrac{x}{x_0}) \geq 0
\Longleftrightarrow x/x_0$ belongs to $\mathscr{O} \Longleftrightarrow
x $ belongs to $\mathscr{O}x_0$. Since $v (\dfrac{x_o}{\pi^\alpha}) =
v(x_0)- \alpha v (\pi )= 0 $, we get that $x_0$ is in $\mathscr{O}
\pi^\alpha$, conversely $\pi^\alpha$ belongs to $\mathscr{O} x_0$ is
obvious. Therefore $\mathscr{O} = \mathscr{O} \pi^\alpha $. In
particular $\mathscr{Y}= \mathscr{O} \pi $. In general let $v$ be any
valuation of a field $K$.  Let $\mathscr{O}$ be any ideal of
$\mathscr{O}$ and $H_\mathscr{O} = \{ \alpha | \alpha \in
\Gamma_\vartheta$, such that there exists $x$ in $\mathscr{O}$ with
$v(x)= \alpha \}$. Then the map $\mathscr{O} \to H_{\mathscr{O}}$ is a
$1 - 1$ correspondence between the set of ideals $\mathscr{O}$ in
$\mathscr{O}$ and the subsets $H_{\mathscr{O}}$ of $\Gamma_v$ having
the property that if $\alpha$ belongs to $H_{\mathscr{O}}$ and $\beta$
belonging to $\Gamma_v$ is such that $\beta \geq \alpha$, then $\beta$
belongs to $H_\mathscr{O}$. In particular if $\Gamma_v$ is contained
in $R$, then the ideals of $\mathscr{O}$ are of one of the two  kinds 
\begin{enumerate}[(i)]
\item $I'_\alpha = \{ x | x \in \mathscr{O}, v (x) \geq \alpha \}$
\item $I_\alpha = \{ x | x \in \mathscr{O}, v (x) > \alpha \}$
\end{enumerate}
for any $\alpha > 0$.

\begin{examples*}
  (1)  Let $Q$ be the field of rational numbers. For any $m$ in $Q$
  we have $m = \pm p^{\alpha_1}_1 \cdots p^{\alpha_r}_r$ uniquely,
  where $\alpha_1, \ldots, \alpha_r$ are in $Z$ and $p_1, \ldots, p_r$
  are distinct primes. If $v$ is any valuation of $Q$, we have $v (m)
  = \sum\limits^r_{j=1} \alpha_j v (p_j)$. Therefore it is sufficient
  to define a valuation for primes in $Z$. We note that for a
  valuation $v$ there exists atmost one $p$ for which $v(p)> 0$. If
  possible let us suppose that there exist two primes $p_1$ and $p_2$
  such that $v(p_i)> 0$ for $i = 1, 2$. 
  
  Since\pageoriginale $(p_1, p_2)=1$, there exist two integers $a$ and
  $b$ such that 
  $ap_1 + bp_2 =1$. This implies that $0=v (1) \geq \inf (v(ap_1))$, $v
  (bp_2))>0$, which is impossible. Thus our assertion is proved, If
  there does not exist any prime $p$ for which $v(p)>0$, then $v$ is
  improper.  
  
  For  a prime $p$ we define $v_p(p) =1$ and $v_p(m)= \alpha$, where
  $\alpha$ is the highest power of $p$ dividing $m$. It is easy to
  verify that this is a valuation of $Q$ and any valuation of $Q$ for
  which $v(p)>0$ is equivalent to this valuation. It is a discrete normed
  valuation of $Q$. One can take $p$ as a uniformising parameter and
  prove that the residual field is isomorphic to $Z/(p)$   
  
  (2)  Let $K$ be any field, $K((x))$ the field of formal power series
  over $K$. For any element $f(x) = \sum\limits^\infty_{r = m} a_r x^r$
  of $K ((x))$ we define $v (f(x)) = t$, if $a_t$ is the first non-zero
  coefficient in $f(x)$. One an easily verify that $v$ is a normed discrete
  valuation of $K ((x))$. The ring of integers of  the valuation is the
  ring of formal power series with non-negative exponents and the ideal
  is the set of those elements in the ring of integers for which the
  constant term is zero. One can take $x$ as a uniformising parameter.  
\end{examples*}

\section{Valuation Rings and Places}\label{part1:chap1:sec2}

This section is added for the sake of completeness. The results
mentioned here will not be used in the sequel. 

\begin{remark*}
  Let\pageoriginale $K$ be a field with a valuation $v$ and ring of integers
  $\mathscr{O}$. Then for any $x$ in $K$, either $x$ belongs to
  $\mathscr{O}$ or $x^{-1}$ belongs to $\mathscr{O}$. 
\end{remark*}

Motivated by this we define

A subring $A$ of a field $K$ is called a \textit{valuation ring} of
$K$ if for any $x$ in $K$ either $x$ belongs to $A$ or $x^{-1}$
belongs to $A$. In general a ring $A$ is said to be a valuation ring
if it is a valuation ring for its quotient field. 

\begin{prop}\label{part1:chap1:sec2:prop3}
  A ring $A$ is a valuation ring if and only if the set of principle
  of $A$ is totally ordered by inclusion. 
\end{prop}

\begin{proof}
  Let $A$ be a valuation ring. Let $Ax$  and $Ay$ be two proper
  principle ideals of $A$. Consider $z = \dfrac{x}{y}$ belonging to $K$
  the quotient field of $A$. Since $A$ is a valuation ring, either $z$
  or $z^{-1}$ belongs $A$. But this implies that either $Ax \subset
  Ay$ or $Ay \supset Ax$. Therefore the set of principal ideals is
  totally ordered conversely let $x = \dfrac{y}{z}$, where $y$ and $z$
  belong to $A$ and $x \neq 0$, be an element of $K$ which is
  not in $A$. $x \notin A$ implies that $y$ does not belong to $Az$. But
  the set of principle ideals of $A$ is totally ordered, therefore we
  get $Az \subset Ay$ implying $z = ay$ for some a in $A$. But  $a =
  x^{-1}$, therefore $A$ is a valuation ring.  
\end{proof}

\begin{coro*}
  A valuation ring is a local ring.  

  If possible let $\mathcal{M}_1 \neq \mathcal{M}_2 $ be two maximal
  ideals in a valuation ring $A$. $\mathcal{M}_1 \neq \mathcal{M}_2 $
  implies that there exists  $x_1 \in \mathcal{M}_1$, $x_1 \notin
  \mathcal{M}_2$ and $x_2 \in \mathcal{M}_2$, $x_2 \notin
  \mathcal{M}_1$. 
  
  $x_1 \notin \mathcal{M}_2 \Longrightarrow Ax_1$\pageoriginale is not contained
  in $\mathcal{M}_2$ which implies that $Ax_1 $ is not contained in
  $Ax_1$. Similarly $x_2$ not belonging to $\mathcal{M}_1$ implies
  that $Ax_1$. But this is impossible, therefore $\mathcal{M}_1 =
  \mathcal{M}_2$. 
\end{coro*}

\begin{prop}\label{part1:chap1:sec2:prop4} 
  A ring $A$ is a valuation ring if and only if $A$ is the ring of a
  valuation of its quotient field $K$ determined upto an equivalence. 
\end{prop}

\begin{proof}
  Let $\mathcal{M}$ be the unique maximal ideal of the valuation ring
  $A$ and $A^* = A /\mathcal{M}$. For  $x,y$ in $K^*$ we define $x \geq
  y$ if and only if $x$ belongs to $Ay$. It is easy to verify that
  this relation among the elements of $K^*$ induces a total order in
  the group $K^* / A^*$ and the canonical homomorphism $K^*$ onto $K^*
  / A^*$ is a valuation of $K$ for which the ring of integers is
  $A$. The ring of integers of a valuation is a valuation ring has
  already been proved. 
\end{proof}

Let $k$ be a fields. By $k \cup \infty$ we mean the set of elements of $k$
together with an element $\infty$. We extend the laws of $k$ to (not
everywhere defined) laws in $k \cup \infty$ in this way  
\begin{enumerate}[(i)]
\item $\infty + a = a + \infty = \infty $ for a in $k^*$ 
\item $\infty \times a = a \times \infty = \infty \times \infty =
  \infty$, for a in $k^*$

  $0 \times \infty$ and $\infty + \infty$ are not defined.
\end{enumerate}

Let $K$ be a field with a valuation $v$ and let $k = \mathscr{O} /
\mathscr{Y}$ be the residual fields of $v$. Then the canonical
homomorphism $\rho$ of $ \mathscr{O}$ onto $k$ extended to $K$ by
setting $\rho (x) = \infty$ for $x$ not in $\mathscr{O}$ gives rise to
a map of $K$ onto $k \cup \infty$ called a place of $K$. 

In\pageoriginale general, we define 

A place of a field $K$ is a mapping $\rho$ form $K$ to $k \cup \infty$
such that  
\begin{enumerate}[(i)]
\item $\rho (a + b ) = \rho (a) + \rho (b)$
\item $\rho (a  b ) = \rho (a)  \rho (b)$
\end{enumerate}
for $a, b$ in $K$ and whenever the right hand side is meaningful. 

It is easy to prove that $\mathscr{O} = \rho^{-1} (K)$ is a valuation
ring with the maximal ideal $\mathscr{Y} = \rho^{-1}(0)$. 

Thus there exists a $1-1$ correspondence between the set of valuation
rings and the set of inequivalent places of a field (Two places $\rho_1$
and $\rho_2$ of a field $K$ carrying $K$ into $k \cup \infty$ and $k' \cup
\infty $) respectively are said to be equivalent if there exists an
isomorphism $\sigma$ of $k$ onto $k'$ such that $\rho_2 = \sigma \circ
\rho_1$, with $\sigma (\infty ) = \infty$. 

\section{Topology Associated with a Valuation}\label{part1:chap1:sec3}

Let $K$ be a field with a valuation $v$. For any $\alpha \geq 0$ in
$\Gamma_v$ consider the ideal 
$$
I_\alpha = \{ x | x \in K, v(x) > \alpha\}
$$

Then there exists one and only topology on $K$ for which 
\begin{enumerate}[(1)]
\item $I_\alpha$ for different $\alpha$ in $\Gamma_v$ form
  a fundamental system fo neighbourhoods of $0$.  
\item $K$ is a topological group for addition. 
\end{enumerate}

We see immediately that the operation of multiplication in $K$ is
continuous in topology. $I_\alpha$ for any $\alpha \geq 0$ in $\Gamma
_v$ is an open subgroup and hence a closed subgroup of $K$. Thus the  
residual\pageoriginale field $k$ is discrete for the quotient
topology. The topology 
of $K$ is discrete if and only if the valuation $v$ is improper ( if
$\Gamma_{v} = \{o\}$). In particular $K$ with a discrete and proper
valuation is not discrete as a topological space. The topology of $K$
is always Hausdorff, because if $x \neq 0$, then $x$ does note belong
to $I_{\alpha}$ with $\alpha = v(x)$, therefore
$\bigcup\limits_{\alpha \in \Gamma_v} I_\alpha {}_\alpha
  > 0 = (0)$ which proves our assertion.  

\begin{remark}\label{part1:chap1:sec3:rem1}% REM 1
  If $v$ is not improper, then the ideals $I'_{\alpha}$ for $\alpha
  \ge o$ in $\Gamma_{v}$ also constitute a fundamental system of
  neighbourhoods of $0$ for the topology of $K$. For, $I'_{\alpha}$
  and for $\alpha > 0~ I_{\alpha}$ contains $I'_{2 \alpha}$. 
\end{remark}

\begin{remark}\label{part1:chap1:sec3:rem2} %REM 2.
  Let $A$ be a ring a with a decreasing filtration by ideals
  i.e. there exists a sequence $(A_{n})_{n \geq 0}$ of ideals such
  that $A_{n} \supset A_{n + 1}$ and $A_{n} A_{m} \subset
  A_{m+n}$. Then there exists one and only one topology for which $A$
  is an additive topological group and $(A_{n})_{n \ge o}$ constitute
  a fundamental system of neighbourhoods of $0$. $A$ is a topological
  ring this topology. 
\end{remark}

Let $\mathcal{M}$ be any ideals of a ring $A$. Then $A$ can be made
into a topological ring by taking $A_{n} = \mathcal{M}^{n}$. We call
the topology defined by $\mathcal{M}$ on $A$ the $\mathcal{M}-$ adic
topology. In particular the ring of integers of a field $K$ which a
real valuation $v$ has the $\mathcal{M}-$ adic topology for every
$\mathcal{M} = \{x / v(x) \ge \alpha > 0\}$ We shall speak of this
topology of $K$ as the $\mathcal{M}-$adic topology.   

If\pageoriginale the valuation $v$ is discrete and normed. We can take $\alpha = 1$
and $\mathcal{M} = \mathscr{Y}$. 

\begin{remark}\label{part1:chap1:sec3:rem3} %REM 3.
  If $K$ is a field with a real valuation $v$, then the
  $\mathscr{Y}-$adic topology completely characterises the valuation
  upto a constant factor, because $x$ belongs to $\mathscr{Y}$ if and
  only if $x^{n}$ tends to zero as $n$ tends to infinity.     
\end{remark} 

\section{Approximation Theorem}\label{part1:chap1:sec4} % SEC 4
 
For the sake of simplicity we confine ourselves in this section to
real valuations though analogous results could be prove for any
valuation. In this section we deal with the question whether there
exists any connection between various inequivalent valuations of a
field. We first prove:-   
 
\begin{Lemma}\label{part1:chap1:sec4:lem1} % LEM1
  Let $K$ be a field with two valuations $v_{1}$ and $v_{2}$. Then
  $v_{1}$ and $v_{2}$ are inequivalent if an only if
  $\mathscr{O}_{1}$, the ring of integers of $v_{1}$, is not contained
  in $\mathscr{O}_{2}$, the ring of integers of $v_{2}$.  
\end{Lemma} 
 
\begin{proof}
  If $\mathscr{O}_{1} \subset \mathscr{O}_{2}$, then
  $K-\mathscr{O}_{1}$ contains $K-\mathscr{O}_{2}$ implying
  $\mathscr{Y}_{2} \subset\mathscr{Y}_{1} \subset \mathscr{O}_{1}
  \subset \mathscr{O}_{2}$. Therefore $\mathscr{Y}_{2}$ is a prime
  ideal in $\mathscr{O}_{1}$. Assume $\mathscr{Y}_{2} \neq
  \mathscr{Y}_{1}$, then there exists $x$ in $\mathscr{Y}_{1}$ which
  does not belong to $\mathscr{Y}_{2}$. Since $\mathscr{Y}_{2}$ is an
  ideal in $\mathscr{O}_{1}$, there exists $\alpha > 0$ in
  $\Gamma_{v_{1}}$ such that $\mathscr{Y}_{2}$ contains
  $I_{\alpha}$. Let $v_{1}(x) =\beta$.
\end{proof} 
 
 Then for large enough $q$ we have 
 $$
 v_{1}(x^{q}) = qv_{1}(x) = q \beta > \alpha, 
 $$
 which means that $x^{q}$ belongs to $\mathscr{Y}_{2}$, but
 $\mathscr{Y}_{2}$is a prime ideal, therefore $x$ belongs to
 $\mathscr{Y}_{2}$. Hence our assumption is wrong. 
 
 Therefore\pageoriginale $\mathscr{Y}_{2} = \mathscr{Y}_{1}$ and $v_{1}$ is
 equivalent to $v_{2}$. The converse is obvious. 
 
\begin{Lemma}\label{part1:chap1:sec4:lem2}%LEM 2.
  Let $K$ be a field with $v_{1},\ldots, v_{n}(n \ge 2)$ proper
  valuations such that $v_{i}$ is inequivalent to $v_{j}$ for $i \neq
  j$. Then there exists an element $z$ in $K$ such that $v_{1} (z) >
  0, v_{2}(z)< 0$ and $v_{i}(z) \neq 0$ for $i = 1, 2, \ldots,  n$.   
\end{Lemma} 
 
\begin{proof}
   We shall prove the results by induction on $n$. When $n = 2, v_{1}$
   inequivalent to $v_{2}$implies that $\mathscr{O}_{1}$ is not
   contained in $\mathscr{O}_{2}$ (lemma
   \ref{part1:chap1:sec4:lem1}). Therefore there exists 
   $x$ in $\mathscr{O}_{1}$ and not in $\mathscr{O}_{2}$. Moreover
   $\mathscr{O}_{2}$ not contained in $\mathscr{O}_{1}$ implies that
   $\mathscr{Y}_{1}$ is not contained in $\mathscr{Y}_{2}$.   
\end{proof} 
 
 Therefore there exists $y$ in $\mathscr{Y}_{1}$ and not in
 $\mathscr{Y}_{2}$. Then $z = xy$ is the required element.  
 
 When $n > 2$. By induction there exists an element $x$ in $K$ such
 that $v_{1}(x) > 0, v_{2}(x) < 0$ and $v_{i}(x) \neq 0$ for $i = 1,2,
 \ldots, n-1$. 
 If $v_{n}(x) \neq 0$, we have nothing to prove. If $v_{n}(x) = 0$, we
 take an element $y$ with $v_{n}(y) \neq 0$. Let $z = y x^{s}$, $s$ a
 positive integer. Then for sufficiently large $s,z$ fulfills the
 requirements of the lemma. 
 
\begin{theorem}\label{part1:chap1:sec4:thm1}
  Let $K$ be a field with $v_{1}, \ldots,  v_{r}$ proper valuations
  such that $v_{i}$ is inequivalent to $v_{j}$ for $i \neq j$. Let
  $K_{i}$ be the field $K$ with the topology defined by $v_{i}$ and
  $\rho$ the canonical map from $K \to  \prod\limits^{r}_{i=1} K_{i} = P$
  i.e. $\rho (a) = (a, a,\ldots,a)$. Then $\rho(K) = D$ is dense in $P$.  
\end{theorem}  

Equivalently stated if $a_{1}, \ldots, a_{r}$ are any $r$ elements in
$K$, then for every $\alpha_{1}, \ldots, \alpha_{r}$ in $R$ there
exists an element $x$ in $K$ such\pageoriginale that  
$$
v(x - a_{i}) > \alpha_{i} \text { for } i = 1, 2, \ldots,  r.
$$
\begin{proof}
  The theorem is trivial for $r = 1$. Let us assume that it is true in
  case the number of valuations is less then $r$.  
\end{proof}

By lemma \ref{part1:chap1:sec4:lem2} there exists an elements $x$ in $K$ such that $v_{1}(x) >
0, v_{r}(x) < 0$ and $v_{i}(x) \neq 0$ for $1 \le i \le r$, then
$y_{n} = \dfrac{x^{n}}{1 + x^{n}}$ tends to $0$ in $K_{1}$, to $1$ in
$K_{r}$ and to $0$ or $1$ in others as $n$ tends to infinity. Let the
notation be so chosen that $\rho (y_{n}) \to (0, 0, \ldots,  0, 1,
\ldots, 1)$ as $n$ tends to infinity, $0$ occurring in $s$ places where
$1 \le s \le r - 1$. Now $D$ is a subspace of $P$ over $K$, therefore   
$$
\lt_{ n \to \infty} x \rho (y_{n}) =  \lt_{ n \to \infty} \rho (x y_{n}) =
(0, \ldots0, ~x,\ldots,x) 
$$
and $(0,0,\ldots, 0, x, x,\ldots, x)$ is in $\bar{D}$. Consider the
product $\prod\limits^{r}_{i=s+1} K_{i}$, by induction assumption the
diagonal of $\prod\limits^{r}_{i=s+1} K_{i}$ which is imbedded in
$\bar{D}$ is dense in the product which implies that $(0, \ldots, 0,
a_{s+1}, \ldots, a_{r})$ belongs to $\bar{D}$ for $a_{i}$ in $K, s + 1
\le i \le r$. Similarly $(a_{1,}a_{2}, \ldots,  a_{s}, 0, \ldots, 0)$
belongs to $\bar{D}$. But $\bar{D}$ is a vector space over $K$,
therefore $(a_{1}, a_{2}, \ldots,  a_{r})$ is in $\bar{D}$. Hence
$\prod\limits^{r}_{i=1} K_{i} = \bar{D}$. 

\begin{coro*}%nonu
  Under the assumptions of the theorem for $\alpha_{j} \in
  \Gamma_{v_{j}} (j = 1, 2, \ldots,  r)$ there exists $x$ in $K$ such
  that $v_{j}(x) = \alpha_{j}$. 
\end{coro*}

For\pageoriginale $\alpha_{j}$ in $\Gamma_{v_{j}}$, there exists $a_{j} \in K$ such
that $v(a_{j}) = \alpha_{j}$. By approximation theorem there exists an
element $x$ in $K$ such that $v(x-a_{j}) > \alpha_{j}$. By definition
we have $v(x) = v(x - a_{j}+ a_{j}) = \inf v((x-a_{j}), v(a_{j})) =
v(a_{j}) = \alpha_{j}$. 

\section{Completion of a field with a valuation}\label{part1:chap1:sec5} %SEC 5

Let $K$ be a field with a valuation $v$. Since $K$ is a commutative
topological group for the topology defined by $v$, it is a uniform
space. Let $\hat{K}$ denote the completion $K$. The composition laws
of addition and multiplication can be extended by continuity to
$\hat{K}$, for which $\hat{K}$ is a topological ring. In fact
$\hat{K}$ is a topological field, because if $\Phi$ is a Cauchy filter
on $K$ converging to $ a \neq 0$, then $\Phi^{-1}$(the image of $\Phi$
by the map $x \to x^{-1}$ in $K$) is a Cauchy filter. For $\Phi$ not
converging to $0$ implies that there exists $\alpha \ge 0$ in
$\Gamma_{v}$ and a set $A$ in $\Phi$ such that $v(x) < \alpha$ for
every $x$ in $A$. Since $\Phi$ is a Cauchy filter, for every $\beta$
in $\Gamma_{v}$, there exists a set $B$ in $\Phi$ contained in $A$
such that 
$$
v(x-y) > 2 \alpha+ \beta ~~\text{for}~ x, y~ \text{in}~ B.
$$   

Then
$$
v(x^{-1}-y^{-1}) = v(x^{-1}y^{-1}(y-x)) = -v(x)-v(y)+v(y-x) > -\alpha
- \alpha +2 \alpha + \beta 
$$
which implies that $\Phi^{-1}$ is a Cauchy filter converging to
$a^{-1}$ in $\hat{K}$. The valuation $v$ can also be extended to be
valuation $\hat{v}$ of $\hat{K}$, in fact it is a continuous
representation of $K^{*}$ onto $\Gamma_{v}$ considered as a discrete
topological group, so $v$ can be extended as a continuous
representation $\hat{v}$ of $\hat{K^{*}}$ in $\Gamma$ and we get
$\hat{v} (x+y) \ge \inf (\hat{v} (x), \hat{v}(y))$\pageoriginale by
continuity. Moreover $\mathscr{O}_{\hat{K}}$ (the ring of integers of
$\hat{K})= \hat{\mathscr{O}}_{K} = \bar{\mathscr{O}}_{K}$, since
$\mathscr{O}_{\hat{K}}$ is open in $\hat{K}$ and $K$ is hence in
$\hat{K},\mathscr{O}_{\hat{K}} \cap K = \mathscr{O}_{K}$ is dense in
$\mathscr{O}_{\hat{K}}$, this implies that $\mathscr{O}_{\hat{K}}
\supset \bar{\mathscr{O}_{K}}$. But $\bar{\mathscr{O}_{K}} \supset
\mathscr{O}_{\hat{K}}$, therefore our result is proved. More generally  
$$
\hat{I}_{\alpha} = \bigg\{ x | \hat{v}(x) > \alpha,  x \in
\hat{K}\bigg\} = \bar{I}_{\alpha} = \overline{\bigg\{ x | v(x) >
  \alpha,  x \in K \bigg\}} 
$$
In particular $\mathscr{Y}_{\hat{K}} = \bar{\mathscr{Y}_{K}}$. We have
$\mathscr{Y}_{K} = \mathscr{O}_{K} \cap \mathscr{Y}_{\hat{K}}$, so we
may identify $\mathscr{O}_{K}/{_{\mathscr{Y}_{K}}}$ with a subset of
$\mathscr{O}_{\hat{K}}/_{\mathscr{Y}_{\hat{K}}}$, and
$\mathscr{O}_{K}/_{\mathscr{Y}_{K}}$ is dense in
$\mathscr{O}_{\hat{K}}/_{\mathscr{Y}_{\hat{K}}}$. But
$\mathscr{O}_{\hat{K}}/_{\mathscr{Y}_{\hat{K}}}$ is discrete,
therefore $\mathscr{O}_{\hat{K}}/_{\mathscr{Y}_{\hat{K}}} =
\mathscr{O}_{K}/_{\mathscr{Y}_{K}}$. 

\begin{remark*}
  Let $K$ be a field with a real valuation $v$, with $v$ we associate
  a map from $K$ to $R$. We defined for any $x$ in $K$ the absolute
  value $|x| = a^{-v(x)}$, where a is a real number $> 1$. The map $|
  |$ satisfies the following properties 
  \begin{enumerate}[(1)]
  \item $|x| = 0$ ~if and only if~ $x = 0$
  \item $|xy| = |x| ~~|y|$
  \item $|x + y| \le ~\sup ~( |x|, |y| ) \le |x| + |y|$.
  \end{enumerate}

  The absolute value of elements of $K$, which defines the same
  topology on $K$ as the valuation $v$. 
\end{remark*}

By $Q_{p}$ we shall always denote the completions of the field $Q$ for
$p$-adic valuation and by $Z_{p}$ the ring of integers in $Q_{p}$. For\pageoriginale
the absolute value associated to the $p$-adic valuation. We take $a=p$
so that $|x|_{p} = p^{-v}p^{(x)}$ 

\section{Infinite Series in a Complete Field}\label{part1:chap1:sec6}

Let $K$ be a complete field for a real valuation $v$. Since every
Cauchy sequence in $K$ has a limit in $K$, the definition of
convergence of infinite series and Cauchy criterium can be given in
the same way as in the case of real numbers. However in this case we
have the following. 

\begin{theorem}\label{part1:chap1:sec6:thm2}%THM 2
  A family $(u_{i})_{i \in I}$ of an infinite number of elements of
  $K$ is summable if and only if $u_{i}$ tends to $0$ following the
  filter of the complements of finite subsets of $I$. 
\end{theorem} 

\begin{proof}
  The condition is clearly necessary. Conversely for any $\alpha$ in
  $\Gamma_{v}$ we can find a finite subset $J$ of $I$ such that for
  $i$ not in $J, v(u_{i})>\alpha$, then for $i_{1}, \ldots,  i_{r}$
  not in $J$ we have $v \left(\sum \limits_{j=1}^{r}
  \cup_{i_j}\right)>\alpha$ which is 
  nothing but Cauchy Criterium. Hence the family is summable.
\end{proof} 

\begin{coro*}
  Let $\sum \limits_{n=0}^{\infty} u_{n}$ be infinite series of
  elements of $K$ Then the following conditions are equivalent. 
  \begin{enumerate}[\rm (a)]
  \item $\sum \limits_{n=0}^{\infty} u_{n}$ is convergent.
  \item $\sum \limits_{n=0}^{\infty} u_{n}$ is commutatively convergent.
  \item $u_{n}$ tends to $0$ as $n$ tends to infinity.
  \end{enumerate}
\end{coro*} 

\noindent \textbf{Application.}
  Let\pageoriginale $K$ be a complete field for a normed discrete real valuation $v,
  \pi$ a uniformising parameter for $K, \mathscr{R}$ a fixed system of
  representatives in $\mathscr{O}$ for the elements of the residual
  field $K$. Then the series $\sum \limits_{q=m}^{\infty} r_{q}
  \pi^{q}$, where $r_{q}$   belongs to $\mathscr{R}$ is convergent to
  an element $x$ in $K$ and conversely every $x$ in $K$ can be
  represented in this form in one and only one way. The series is
  convergent because $v(r_{q}\pi^{q})\ge q$ for $q \neq 0$ and
  therefore tends to infinity as $q$ tends to infinity. Conversely by
  multiplying with a suitable power of $\pi$  we can take $x$ in
  $\mathscr{O}$, then there exists a unique $r_{0} \in \mathscr{R}$
  such that $x \equiv r_{0} \pmod {\mathscr{Y}}$.
  
  This implies that $(x-r_{0}) \pi^{-1}$ is in $\mathscr{O}$. Therefore
  there exists unique $r_{1}$ in $\mathscr{R}$ such that 
  $$
  \displaylines{\hfill 
    (x - x_{0}) \pi^{-1} \equiv r_{1} \pmod {\mathscr{Y}}.\hfill \cr
    \text{or}\hfill  x \equiv r_{0} + r_{1} \pi \pmod
         {\mathscr{Y}^{2}}.\hfill }
  $$
  Proceeding in this way we prove by induction that 
  $$
  x \equiv r_o + r_1 \pi \cdots + r_m \pi^m \pmod {\mathscr{Y}^{m + 1}}
  $$ 
  Now it is obvious that the series $\sum\limits^{\infty}_{r = 0}r_m
  \pi^m$, is convergent and that $x = \sum\limits^{\infty}_{q = 0}
  r_q \pi^q$. The uniqueness of the series is obvious from the
  construction. 

In particular if, $K =Q_P$ then any $x$ in $Q_p$ can be represented in
the form $\sum\limits^{\infty}_{q = m}r_q p^q$, where $r_q \in \{0,
1,2 \ldots, p-1\}$. 

\section{Locally Compact Fields}\label{part1:chap1:sec7}%\sec 7

In\pageoriginale this section we give certain equivalent conditions for valuated
fields to be locally compact. Later on we shall completely
characterise the locally compact valuated fields. 

\begin{theorem}\label{part1:chap1:sec7:thm3}
  Let $K$ be a field with a proper valuation $v$. Then the following
  conditions are equivalent. 
  \begin{enumerate}[\rm(a)]
  \item $K$ is locally compact.
  \item $\mathscr{O}$ is compact.
  \item $K$ is complete, $v$ is a discrete valuation and $k$ is a finite field.
  \end{enumerate}
\end{theorem}

\begin{proof}
  (a) $\Longrightarrow$ (b). Since $(I'_\alpha)_{\alpha \in \Gamma _v}$
  form a fundamental system of closed neighbourhoods for $0$, there
  exists an $\alpha$ such that $I'_\alpha$ is compact. But $m$
  $I'_\alpha=\mathscr{O}_{x_o}$, if $v(x_0)=\alpha$, therefore
  $\mathscr{O}=x^{-1}_{0} I_\alpha$ is compact. 

  (b) $\Longrightarrow$ (a) is trivial, as $\mathscr{O}$ is a compact
  neighbourhood of $0$.  
  
  (a) $\Longrightarrow$ (c) $K$ is complete because it is a locally
  compact commutative group. For any $\alpha > 0$ in $\Gamma_v
  \,\mathscr{O}/I_{\alpha}$ is compact because $\mathscr{O}$ is
  compact. But $\mathscr{O}/I_\alpha$ is a discrete space, therefore
  it contains only a finite number of elements. In particular
  $k=\mathscr{O}/\mathscr{Y}$ is finite field. For any $\beta$ in
  $\Gamma_v, 0 < \beta < \alpha$, we have $I_\alpha \subset I_\beta
  \subset \mathscr{O}$, therefore $I_\beta / I_\alpha$ is a nontrivial
  ideal of $\mathscr{O}/I_\alpha$ and distinct elements give rise to
  distinct ideals. But $\mathscr{O}/I_\alpha$ is a finite set,
  therefore there exist only a\pageoriginale finite number of $\beta$ with $0 <
  \beta < \alpha$, so we get that  
  \begin{enumerate}[(i)]
  \item $\Gamma_v$ has a smallest positive element
  \item $\Gamma_v$ is Archimedian.
  \end{enumerate}

  Thus $\Gamma_v$ is isomorphic to $Z$ and the valuation $v$ is
  discrete. (c) $\Longrightarrow$ (b). We shall prove that discreteness
  of the valuation $v$ and finiteness of $k$ implies that $\mathscr{O}$
  is precompact, which together with the fact that $K$ is complete
  implies that $\mathscr{O}$ is compact. Let $V$ be any neighbourhood of
  $0$. Since $v$ is discrete, for some $n > 0~V$ contains
  $\mathscr{Y}^n$. We shall show by induction on $n$ that
  $\mathscr{O}/\mathscr{Y}^n$ is finite for $n > 0$. The result is true
  for $n=1 ;$ let us assume it to be true for all $r < n$. We have
  $\mathscr{O} / \mathscr{Y}^{n-1} \simeq \mathscr{O}/\mathscr{Y}^n
  / \mathscr{Y}^{n-1} / \mathscr{Y}^n$ But $\mathscr{O} /
  \mathscr{Y}^{n-1}$ is finite by induction hypothesis and
  $\mathscr{Y}^{n-1} / \mathscr{Y}^n$ is finite because it is isomorphic
  to $\mathscr{O}/\mathscr{Y}$, therefore $\mathscr{O}/\mathscr{Y}^n$ is
  finite. Hence there exist a finite number of elements $x_1 - -- x_r$
  in $\mathscr{O}$ such that $\mathscr{O}\subset \bigcup\limits^{r}_{i
    =1}(x_i + \mathscr{Y}^n) \subset \bigcup\limits^{r}_{i= 1}(x_i + V)$
  and since this is true for every neighbourhood of $0, \mathscr{O}$ is
  precompact. 
\end{proof}

\section{Convergent Power Series}\label{part1:chap1:sec8}%\sec 8

Let $K$ be complete field with a real valuation $v$. Then the power
series $f(x)=\sum\limits^{\infty}_{n = 0}a_n x^n$ with coefficients
from $K$ is said to be convergent at a point $x$ of $K$ if the series
$\sum\limits^{\infty}_{n = 0}a_n x^n$ is convergent. It has already
been proved that the series $\sum\limits^{\infty}_{n = 0}a_n x^n$\pageoriginale
converges if and only if 
\begin{equation*}
  v(a_{n}x^{n}) = v(a_{n}) + nv(x) \to \infty~ as ~n \to
  \infty\tag{1}\label{part1:chap1:sec8:eq1} 
\end{equation*} 
From (\ref{part1:chap1:sec8:eq1}) it is obvious that if take $t
=\lim\limits_n \inf 
\dfrac{1}{n} (v (a_{n}))$, then the series $f$ converges for all $x$
which $v(x)> -t$ and does not converge for those $x$ for which
$v(x)<-t$ and for those $x$ for which $v(x)= -t $ either the series
converges for all $x$ or does not converge at all. The number $-t$ is
called the order of convergence of the power series $f$ and the set
$\{x| v(x) > -t\}~ \text{or}~ \{x |v(x) \ge -t, $ if the series
converges at a point $x$ with $v(x) = -t\}$ is called the disc of
convergence, which we shall denote by $D_{f}$. If we consider the
absolute value associated to $v$ then the radius of convergence is  
$$
\displaylines{\hfill 
  \rho = a^{-t} = \left\{\lim_{n\to\infty}\sup
  (|a|_{n})^{1/n}\right\}^{-1} \hfill \cr 
  \text{and}\hfill D_{f} = \left\{x|~|x|< \rho\right\} \quad  \text{or} \quad
  \left\{x |~|x| \le \rho \right\}\hfill } 
$$
The mapping $x\to f(x)$ from $D_{f}$ to $K$ is continuous because it
is a uniform limit of polynomials namely the partial sums of the
series $\sum \limits_{n=0}^{\infty} a_{n} x^{n}$ in the disc $\{x |
v(x) \ge -t_1$, for all $t_{1} > t\}$ or in the disc $\{x | v(x)
\ge -t\}$ if the series converges on the disc. The classical results
about addition and multiplication, $\ldots$ of power series can be
carried over to the power series with coefficient in a complete
valuated field. For instance if $f(x) = \sum \limits_{n=0}^{\infty}
a_{n}x^{n}$ and $g(x) = \sum \limits_{n=0}^{\infty} b_{n} x^{n}$ are
two power series with $D_{f}$ and $D_{g}$ as their discs of
convergence respectively; then if\pageoriginale for one $x$  in $D_f$,  $a_i x^i$
belongs to $D_g$ for every $i$, $f(x)$ also belongs to $D_g$ and we
have 
\begin{align*}
   g(f(x)) & = \sum_{r=0}^{\infty} c_r x^r,  ~\text{where}\\
  c_r & = \sum_{q=0}^{\infty} b_q \sum_{i_1 + i_2 + \cdots +i_q =r}
  a_{i_1} a_{i_2}\ldots a_{i_q},
\end{align*}
all the series being convergent.

\setcounter{remark}{0}
\begin{remark}
If $k=\mathscr{O}/ \mathscr{Y}$ is an infinite field, then
$$
\inf_i (v(a_i x^i))= \inf_{v(y) = v(x)} (v(f(y))).
$$
\end{remark}

For, $v(f(x))\geq \inf_i (v(a_i x^i))$. We get equality, if there does
not exist any two terms of the same valuation. In the exceptional case
as the series as the series $\sum\limits_{n=0}^{\infty} a_n y^n$ is
convergent, we have 

$f(y) = \sum\limits_{r=i_\circ}^{j_\circ}a_r y^r+$ terms of higher valuation,
where $i_\circ \leq r \leq j_\circ < \infty$.

and without loss of generality we can assume that $v(x)=0$ and $\inf_i
v(a_i x^i)=0$. Now $v(f(y))>0$ if and only if
$\sum\limits_{r=i_\circ}^{j_\circ} a_r y^r$ belongs to
$\mathscr{Y}$\iec  if and only if the polynomial
$\sum\limits_{r=i_\circ}^{j_\circ} \bar{a}_r \bar{y}^r$ (the image in
$k)=0$. But $k$ has infinite  number of elements and the above
polynomial not being identically zero has only a finite number of
zeros, therefore there exists  
atleast\pageoriginale one $y$ for which $v(f(y))=0$ and  $v(x) = v(y)$. Thus in this
case whenever $x$ is in $D_f$ and $f(y)$ belongs to $D_g$ for all
those $y$ for which $v(x) = v(y)$, we have 
$$
\inf_i v(a_i x^i)= \inf_{v(y) = v(x)} v(f(y)).
$$

Then $f(g(x)) = \sum\limits_{r=0}^{\infty} c_r x^r$ with
$$
c_r = \sum_{r=0}^{\infty} b_q \sum_{v_1 + --+v_q=r} a_{v_1}\cdots a_{v_q}.
$$

\begin{remark}%rem
  Let $A$ be a ring with a topology defined by a decreasing filtration
  $(A_n)_{n \geq 0}$ of ideals for which $A$ is Hausdorff and complete
  space. Then the formal power series $\sum\limits_{n=0}^{\infty} a_n
  x^n$ converges at $x$ in $A$ if and only if $a_n x^n \to 0$ as $n$
  tends to infinity and obviously the series converges everywhere in
  $A$ if and only if $a_n$ tends to $0$ as $n$ tends to infinity. 
\end{remark}
