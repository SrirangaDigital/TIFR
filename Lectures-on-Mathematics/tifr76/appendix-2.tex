
\makeatletter

\def\@makechapterhead#1{%
  \vspace*{50\p@}%
  {\parindent \z@ \raggedright \normalfont
    \ifnum \c@secnumdepth >\m@ne
      \if@mainmatter
        \huge\bfseries Appendix\space \thechapter
       \par\nobreak
        \vskip 20\p@
      \fi
    \fi
    \interlinepenalty\@M
    \Huge \bfseries #1 \par\nobreak
    \vskip 40\p@
  }}

\c@tocdepth=1
\renewcommand\thesection{\@Alph\thechapter.\@arabic\c@section}

\def\@chapter[#1]#2{\ifnum \c@secnumdepth >\m@ne
                       \if@mainmatter
                         \refstepcounter{chapter}%
                         \typeout{\@chapapp\space\thechapter.}%
                         \addcontentsline{toc}{chapter}%
                                   {\protect\numberline{}
                          Appendix 2: #1}%
                       \else
                         \addcontentsline{toc}{chapter}{#1}%
                       \fi
                    \else
                      \addcontentsline{toc}{chapter}{#1}%
                    \fi
                    \chaptermark{#1}%
                    \addtocontents{lof}{\protect\addvspace{10\p@}}%
                    \addtocontents{lot}{\protect\addvspace{10\p@}}%
                    \if@twocolumn
                      \@topnewpage[\@makechapterhead{#2}]%
                    \else
                      \@makechapterhead{#2}%
                      \@afterheading
                    \fi}

\makeatother

\chapter{Complements of Chapter IV}


In\pageoriginale this appendix, we give complete proofs for several results we used in Chapter \ref{chap4}, $\S \S 5$ and 6 and which were omitted there for the sake of brevity. The notation is that of Chapter \ref{chap4} and our exposition follows E1 Hajji \cite{10}.

To begin with, we make precise a few definitions and elementary properties. Let $E$ be a real Banach space with norm $|| \cdot ||$ and $Q$ a compact space. The Banach space $C^{0} (Q, E)$ will be equipped with the norm $| \cdot |_{\infty, Q}$ defined by
$$
\psi \epsilon C^{0} (Q, E) \to |\psi|_{\infty, Q} = \sup_{x \epsilon
  Q} ||\psi(x)||. 
$$

Now, if $\Omega$ denotes an open subset of some other real Banach
space and $m \geq 0$ a given integer, we call
$C^{m}(\overline{\Omega}, E)$ the space of mappings $m$ times
differentiable in $\Omega$ whose derivatives of order $\leq m$ can be
continuously extended to the closure $\overline{\Omega}$. Note that if
the ambient space of $\Omega$ is infinite-dimensional,
$C^{m}(\overline{\Omega}, E)$ is not a normed space, even when
$\Omega$ is bounded. On the contrary, if $\Omega$ is bounded and its
ambient space is finite-dimensional, $C^{m}(\overline{\Omega}, E)$ is
a Banach space with the norm
$$
\psi \epsilon C^{m}(\overline{\Omega}, E) \to \sum\limits_{i=0}^{m}
\sup_{x \epsilon \overline{\Omega}} ||D^{i} \psi (x)|| =
\sum\limits_{i=0}^{m} |D^{i} \psi|_{\infty, \overline{\Omega}}.
$$

In other words, convergence in the space $C^{m}(\overline{\Omega}, E)$
is equivalent to convergence of the derivatives of order $\leq m$ in
the uniform norm.

Our\pageoriginale first task is to prove, for $r > 0$ and $\delta > 0$
small enough, that the sequence $(g_{\ell})$
(resp. $(D_{\widetilde{\xi}}g_{\ell})$) tends to $g$
(resp. $D_{\widetilde{\xi}}g$) in the space $C^{0}([-r, r] \times
\triangle, \mathbb{R}^{n})$ (resp. $C^{0} ([-r, r] \times \triangle,
\mathscr{L} (\mathbb{R}^{n+1}, \mathbb{R}^{n}))$). Since the results
we are looking for are local, it is not restrictive to assume that
$\mathscr{O}$ (in IV.5) is {\em convex} and $r > 0$ is chosen so that
$t \widetilde{\xi} \epsilon \mathscr{O}$ for every $t \epsilon [-r,
  r]$ and every $\widetilde{\xi} \epsilon \triangle$.

Due to the definition of $k$, the Taylor formula shows for $0 < |t| \leq
r$ and $\widetilde{\xi} \epsilon \triangle$ that
$$
g_{\ell} (t, \widetilde{\xi}) -g(t, \widetilde{\xi}) = k \int_{0}^{1}
(1-s)^{k-1} D^{k} (f_{\ell} - f) (st \widetilde{\xi}) \cdot
(\widetilde{\xi})^{k} ds.
$$

From the definitions, this remains true for $t = 0$ and, as
$||\widetilde{\xi}|| \leq 1 + \delta / 2$ we get
\begin{equation*}
|g_{\ell} - g|_{\infty, [-r, r] \times \triangle} \leq \left(1 +
\frac{\delta}{2}\right)^{k} |D^{k} (f_{\ell} - f)|_{\infty,
  \overline{\mathscr{O}}}.\tag{A2.1} \label{app-2-eqA2.1}
\end{equation*}

Also, for $t \neq 0$, we have
\begin{equation*}
D_{\widetilde{\xi}}(g_{\ell}-g) (t, \widetilde{\xi}) =
\frac{k!}{t^{k-1}} D(f_{\ell} - f) (t \widetilde{\xi}),\tag{A2.2}\label{app-2-eqA2.2}
\end{equation*}
while the relation $g_{\ell}(0, \cdot) = g(0, \cdot) = q$ for every
$\ell \geq 0$ yields 
\begin{equation*}
D_{\widetilde{\xi}}(g_{\ell} - g)(0, \widetilde{\xi}) = 0.\tag{A2.3}\label{app-2-eqA2.3}
\end{equation*}

Thus, if $k = 1$
\begin{equation*}
|D_{\widetilde{\xi}}(g_{\ell} - g)|_{\infty, [-r, r]} \times \Delta \leq k(1
+ \frac{\delta}{2})^{k-1} |D^{k}(f_{\ell}-f)|_{\infty,
  \overline{\mathscr{O}}}.\tag{A2.4} \label{app-2-eqA2.4}
\end{equation*}

In order to show that the above formula remains true also when $k \geq
2$, it suffices to apply the Taylor formula to the term $D(f_{\ell} -
f)$ and to substitute it in (\ref{app-2-eqA2.2}); we then find, for $0 < |t| \leq
r$\pageoriginale and $\widetilde{\xi} \epsilon \triangle$, that
$$
D_{\widetilde{\xi}}(g_{\ell}-g) (t, \widetilde{\xi}) = k(k-1)
\int_{0}^{1} (1-s)^{k-2} D^{k}(f_{\ell}-f) (st \widetilde{\xi}) \cdot
(\widetilde{\xi})^{k-1} ds,
$$
and this relation remains valid for $t = 0$, which establishs (\ref{app-2-eqA2.4})
immediately. The assertion follows from (\ref{app-2-eqA2.1}) and (\ref{app-2-eqA2.4}) and the
convergence of the sequence $(f_{\ell})$ to $f$ in the space
$C^{k}(\overline{\mathscr{O}}, \mathbb{R}^{n})$.

It is then easy to deduce that the sequence $(M_{\ell})$ (respectively\break
$(D_{\widetilde{\xi}} M_{\ell})$) tends to $M$ (respectively
$D_{\widetilde{\xi}}M$) uniformly in the set $[-r, r] \times C \times
\triangle$. More precisely, introducing
\begin{equation*}
w(\delta) = \sup_{\widetilde{\zeta}_0 \epsilon C}
||A(\widetilde{\zeta}_{0}) - A(\widetilde{\xi}_{0})|| = |A -
A(\widetilde{\xi}_{0})|_{\infty, C},\tag{A2.5}\label{app-2-eqA2.5}
\end{equation*}
one has
\begin{equation*}
|M_{\ell} - M|_{\infty, [-r, r] \times C \times \triangle} \leq
(||A(\widetilde{\xi}_{0})|| + w(\delta)) |g_{\ell} - g|_{\infty, [-r,
    r] \times \triangle}\tag{A2.6}\label{app-2-eqA2.6}
\end{equation*}

Besides, the same arguments leads to the inequality
\begin{equation*}
|M_{\ell} (t, \cdot , \cdot)-M(t, \cdot , \cdot)|_{\infty, C \times
  \triangle} < \left(||A(\widetilde{\xi}_{0})|| + w(\delta)\right)
|g_{\ell}(t, \cdot)-g(t, \cdot)|_{\infty, \triangle},\tag*{(A2.6)$'$}\label{app-2-eqA2.6'}
\end{equation*}
for every $t \epsilon [-r, r]$.

Now, for $(t, \widetilde{\zeta}_{0}, \widetilde{\xi}) \epsilon [-r, r]  
\times C \times \triangle$
$$ 
||N_{\ell}(t, \widetilde{\zeta}_{0}, \widetilde{\xi})-N(t,
\widetilde{\zeta}_{0}, \widetilde{\xi})|| \leq 2 \frac{||M_{\ell}(t,
  \widetilde{\zeta}_{0}, \widetilde{\xi}) -M(t, \widetilde{\zeta}_{0},
   \widetilde{\xi})||}{||M(t, \widetilde{\zeta}_{0}, \widetilde{\xi})||}
$$ 

Hence, from Lemma \ref{chap4-lem2.1} of Chapter \ref{chap4}
\begin{equation*}
|N_{\ell} - N|_{\infty, [-r, r] \times C \times \triangle} \leq
\frac{2}{1 - \delta} |M_{\ell} - M|_{\infty, [-r, r] \times C \times
  \triangle},\tag{A2.7} \label{app-2-eqA2.7}
\end{equation*}
from which it follows that the sequence $(N_{\ell})$ tends to $N$
uniformly in\pageoriginale the set $[-r, r] \times C \times \triangle$. The same
method shows that
\begin{equation*}
|N_{\ell}(t, \cdot, \cdot) -N(t, \cdot , \cdot)|_{\infty, C \times
  \triangle} < \frac{2}{1 - \delta} |M_{\ell} (t, \cdot , \cdot) -
M(t, \cdot, \cdot)|_{\infty, C \times \triangle},\tag*{(A2.7)$'$}\label{app-2-eqA2.7'}
\end{equation*}
for every $t \epsilon [-r, r]$.

The proof of Theorem \ref{chap4-thm5.1} of Chapter \ref{chap4} is
based on the following estimate. 

\begin{alphlemma}\label{app-2-lemA2.1}%[A2.1]
For every triple $(t, \widetilde{\zeta}_{0}, \widetilde{\xi}) \epsilon
[-r, r] \times C \times \triangle$ and every $\ell \epsilon
\mathbb{N}$, one has
\begin{align*}
& ||D_{\widetilde{\xi}}N_{\ell} (t, \widetilde{\zeta}_{0},
\widetilde{\xi})|| \leq \frac{3\delta}{(1 - \delta^{3})} +
\frac{2||D_{\widetilde{\xi}}g_{\ell} (0, \widetilde{\xi}_{0})||}{(1 -
  \delta)} w(\delta) +\\
& + 2 \left[\frac{||A(\widetilde{\xi}_{0})|| +
    w(\delta)}{(1-\delta)}\right] \left[\frac{3}{(1 - \delta)^{2}}||
  g_{\ell} (t, \widetilde{\xi})|| + ||D_{\widetilde{\xi}}g_{\ell} (t,
  \widetilde{\xi}) - D_{\widetilde{\xi}}g_{\ell} (0,
  \widetilde{\xi}_{0})||\right].\tag{A2.8} \label{app-2-eqA2.8}
\end{align*}
\end{alphlemma}


\begin{proof}
In this proof, it will be convenient to use the following natation:
$M_{\ell}^{*}$ (resp. $\Lambda_{\ell}^{*}$) will denote the value
$M(t, \widetilde{\zeta}_{0}, \widetilde{\xi})$
(resp. $D_{\widetilde{\xi}}M_{\ell} (t, \widetilde{\zeta}_{0},
\widetilde{\xi})$) at some arbitrary point $(t, \widetilde{\zeta}_{0},
\widetilde{\xi}) \epsilon [-r, r] \times C \times \triangle$ and
$M_{\ell}^{0}$ (resp. $\Lambda_{\ell}^{0}$) the particular value
$M_{\ell}(0, \widetilde{\xi}_{0}, \widetilde{\xi}_{0})$
(resp. $D_{\widetilde{\xi}}m_{\ell} (0, \widetilde{\xi}_{0},
\widetilde{\xi}_{0})$). Thus, by the definition of $M_{\ell}$
\begin{equation*}
M_{\ell}^{0} = \widetilde{\xi}_{0}.\tag{A2.9}\label{app-2-eqA2.9}
\end{equation*}

On the other hand, the calculation of $D_{\widetilde{\xi}}M(0,
\widetilde{\xi}_{0}, \widetilde{\xi}_{0})$ made in Chapter
\ref{chap4}, $\S 3$ can be repeated for finding
$D_{\widetilde{\xi}}M_{\ell} (0, \widetilde{\xi}_{0},
\widetilde{\xi}_{0}) = \Lambda_{\ell}^{0}$ so that
\begin{equation*}
\Lambda_{\ell}^{0} \widetilde{h} = (\widetilde{\xi}_{0} |
\widetilde{h}) \widetilde{\xi}_{0},\tag{A2.10}\label{app-2-eqA2.10}
\end{equation*}
for every $\widetilde{h} \epsilon \mathbb{R}^{n+1}$. As a result
\begin{equation*}
||M_{\ell}^{0}|| = 1,\tag{A2.11}\label{app-2-eqA2.11}
\end{equation*}

\begin{equation*}
||\Lambda_{\ell}^{0}|| = 1.\tag{A2.12}\label{app-2-eqA2.12}
\end{equation*}\pageoriginale

With the notation introduced above, an elementary calculation gives
\begin{equation*}
D_{\widetilde{\xi}}N_{\ell} (t, \widetilde{\zeta}_{0},
\widetilde{\xi}) \cdot \widetilde{h} = \frac{\Lambda_{\ell}^{*}
  \widetilde{h}}{||M_{\ell}^{*}||} - \frac{(\Lambda_{\ell}^{*} \widetilde{h} |
  M_{\ell}^{*})}{||M_{\ell}^{*}||^{3}} M_{\ell}^{*},\tag{A2.13}\label{app-2-eqA2.13}
\end{equation*}
for every $\widetilde{h} \epsilon \mathbb{R}^{n+1}$. In particular,
choosing $(t, \widetilde{\zeta}_{0}, \widetilde{\xi}) = (0,
\widetilde{\xi}_{0}, \widetilde{\xi}_{0})$, we get
\begin{equation*}
D_{\widetilde{\xi}}N_{\ell}(0, \widetilde{\xi}_{0},
\widetilde{\xi}_{0}) = \Lambda_{\ell}^{0} \widetilde{h} -
(\Lambda_{\ell}^{0} \widetilde{h} | M_{\ell}^{0}) M_{\ell}^{0}\tag{A2.14}\label{app-2-eqA2.14}
\end{equation*}

Next, from (\ref{app-2-eqA2.9})-(\ref{app-2-eqA2.10}), it follows that
$D_{\widetilde{\xi}}N_{\ell} (0, \widetilde{\xi}_{0},
\widetilde{\xi}_{0}) = 0$.\break Therefore the relation (\ref{app-2-eqA2.13}) is unchanged
by subtracting (\ref{app-2-eqA2.14}) from it, which means that
$$
D_{\widetilde{\xi}}N_{\ell} (t, \widetilde{\zeta}_{0},
\widetilde{\xi}) \cdot \widetilde{h} = \left(\frac{\Lambda_{\ell}^{*}
  \widetilde{h}}{||M_{\ell}^{*}||} - \Lambda_{\ell}^{0}
\widetilde{h}\right) - \left(\frac{(\Lambda_{\ell}^{*} \widetilde{h} |
  M_{\ell}^{*})}{||M_{\ell}^{*}||^{3}} M_{\ell}^{*} -
(\Lambda_{\ell}^{0} \widetilde{h} | M_{\ell}^{0})M_{\ell}^{0}\right)
$$
for every $\widetilde{h} \epsilon \mathbb{R}^{n+1}$. Hence the
inequality
\begin{equation*}
||D_{\widetilde{\xi}}N_{\ell} (t, \widetilde{\zeta}_{0},
\widetilde{\xi}) \cdot \widetilde{h}|| \leq
||\frac{\Lambda_{\ell}^{*}\widetilde{h}}{||M_{\ell}^{*}||} -
\Lambda_{\ell}^{0} \widetilde{h}|| +
||\frac{(\Lambda_{\ell}^{*}\widetilde{h} |
  M_{\ell}^{*})}{||M_{\ell}^{*}||^{3}} M_{\ell}^{*} -
(\Lambda_{\ell}^{0} \widetilde{h} | M_{\ell}^{0})M_{\ell}^{0}||.\tag{A2.15}\label{app-2-eqA2.15}
\end{equation*}

As a first step, we establish an estimate for the term
$||\frac{\Lambda_{\ell}^{*} \widetilde{h}}{||M_{\ell}^{*}||} -
\Lambda_{\ell}^{0} \widetilde{h}||$. One has
$$
\frac{\Lambda_{\ell}^{*} \widetilde{h}}{||M_{\ell}^{*}||} -
\Lambda_{\ell}^{0} \widetilde{h} = \frac{1}{||M_{\ell}^{*}||}
((\Lambda_{\ell}^{*})\widetilde{h} + (1 - ||M_{\ell}^{*})
\Lambda_{\ell}^{0}\widetilde{h}). 
$$

From (\ref{app-2-eqA2.11}), we first find a majorisation by
$$
\frac{1}{||M_{\ell}^{*}||} \left(||\Lambda_{\ell}^{*} -
\Lambda_{\ell}^{0}|| + |1 - ||M_{\ell}^{*}|| |\right)
||\widetilde{\mathscr{h}}||. 
$$

Now,\pageoriginale since $|1 - ||M_{\ell}^{*}||| \leq ||M_{\ell}^{*} -
M_{\ell}^{0}||$ (cf. (\ref{app-2-eqA2.11}))
\begin{equation*}
||\frac{\Lambda_{\ell}^{*} \widetilde{h}}{||M_{\ell}^{*}||} -
\Lambda_{\ell}^{0} \widetilde{h}|| \leq \frac{1}{||M_{\ell}^{*}||}
(||\Lambda_{\ell}^{*} - \Lambda_{\ell}^{0})|| + ||M_{\ell}^{*} -
M_{\ell}^{0}||) ||\widetilde{h}||.\tag{A2.16}\label{app-2-eqA2.16}
\end{equation*}

Next, we look for an appropriate majorisation of the second term
appearing in the right hand side of (\ref{app-2-eqA2.15}), namely 
$$||\frac{(\Lambda_{\ell}^{*} \widetilde{h} |
  M_{\ell}^{*})}{||M_{\ell}^{*}||^{3}} M_{\ell}^{*} -
(\Lambda_{\ell}^{0} \widetilde{h} | M_{\ell}^{0})
M_{\ell}^{0}||.$$ Here, we use the identity
\begin{align*}
\frac{(\Lambda_{\ell}^{*} \widetilde{h} |
  M_{\ell}^{*})}{||M_{\ell}^{*}||} M_{\ell}^{*} & - (\Lambda_{\ell}^{0}
\widetilde{h} | M_{\ell}^{0}) M_{\ell}^{0} =
\frac{1}{||M_{\ell}^{*}||^{3}} \{((\Lambda_{\ell}^{*} -
\Lambda_{\ell}^{0}) \widetilde{h} | M_{\ell}^{*}) M_{\ell}^{*} +\\
 & + (\Lambda_{\ell}^{0}  \widetilde{h} | M_{\ell}^{*}) (M_{\ell}^{*}
- M_{\ell}^{0}) + (\Lambda_{\ell}^{0}  \widetilde{h}, M_{\ell}^{*} -
M_{\ell}^{0}) M_{\ell}^{0} + \\
& + (1 - ||M_{\ell}^{*}||^{3}) (\Lambda_{\ell}^{0}
\widetilde{h}. M_{\ell}^{0})M_{\ell}^{0}\}. 
\end{align*}

As $|1 - ||M_{\ell}^{*}||^{3}| = |1 - ||M_{\ell}^{*}|| | (1 +
||M_{\ell}^{*}|| + ||M_{\ell}^{*}||^{2})$ and using the inequality $|1
- ||M_{\ell}^{*}|| | \leq ||M_{\ell}^{*} - M_{\ell}^{0}||$, we obtain
\begin{align*}
& || \frac{(\Lambda_{\ell}^{*}  \widetilde{h} |
    M_{\ell}^{*})}{||M_{\ell}^{*}||} M_{\ell}^{*} -
  (\Lambda_{\ell}^{0}  \widetilde{h} | M_{\ell}^{0}) M_{\ell}^{0}||
  \leq \tag{A2.17}\label{app-2-eqA2.17} \\
& \leq \{\frac{1}{||M_{\ell}^{*}||} ||\Lambda_{\ell}^{*} -
  \Lambda_{\ell}^{0}|| + \left(\frac{1}{||M_{\ell}^{*}||} +
  \frac{2}{||M_{\ell}^{*}||^{2}} +
  \frac{2}{||M_{\ell}^{*}||^{3}}\right) ||M_{\ell}^{*} -
  M_{\ell}^{0}||\} || \widetilde{h}||.
\end{align*}

Using (\ref{app-2-eqA2.16}) and (\ref{app-2-eqA2.17}) in (\ref{app-2-eqA2.15}), we arrive at
$$
||D_{\widetilde{\xi}}N_{\ell}(t,  \widetilde{\zeta}_{0},
\widetilde{\xi})|| \leq \frac{2}{||M_{\ell}^{*}||}
\{||\Lambda_{\ell}^{*} - \Lambda_{\ell}^{0}|| + \left(1 +
\frac{1}{||M_{\ell}^{*}} + \frac{1}{||M_{\ell}^{*}||^{2}}\right)
||M_{\ell}^{*} - M_{\ell}^{0}||\}.
$$

In view of Lemma \ref{chap4-lem2.1} of Chapter \ref{chap4} and from
$0 < 1 - \delta < 1$, this inequality\pageoriginale takes the simpler
form
\begin{equation*}
||D_{ \widetilde{\xi}}N_{\ell} (t,  \widetilde{\zeta}_{0},
\widetilde{\xi})|| \leq \frac{2}{1 - \delta}
\left(||\Lambda_{\ell}^{*} - \lambda_{\ell}^{0}|| +
\frac{3}{(1-\delta)^{2}} ||M_{\ell}^{*} - M_{\ell}^{0}||\right).\tag{A2.18}\label{app-2-eqA2.18}
\end{equation*}

At this stage, proving our assertion reduces to finding a suitable
estimate for the two terms $||\Lambda_{\ell}^{*} -
\Lambda_{\ell}^{0}||$ and $||M_{\ell}^{*} - M_{\ell}^{0}||$. We begin
with a majorisation of $||M_{\ell}^{*} - M_{\ell}^{0}||$. By (\ref{app-2-eqA2.9})
and the definition of the mapping $M_{\ell}$,
\begin{align*}
||M_{\ell}^{*} - M_{\ell}^{0}|| & \leq || \widetilde{\xi} -
\widetilde{\xi_{0}}|| + ||A( \widetilde{\zeta}_{0}) \cdot g_{\ell} (t,
 \widetilde{\xi})||\\
& \leq \frac{\delta}{2} + ||A( \widetilde{\zeta}_{0})|| ||g_{\ell}(t,
 \widetilde{\xi})||. 
\end{align*}

As $||A( \widetilde{\zeta}_{0})|| \leq ||A( \widetilde{\xi}_{0})|| +
w(\delta)$ by the definition of $w(\delta)$ (of. (\ref{app-2-eqA2.5}))
\begin{equation*}
||M_{\ell}^{*} - M_{\ell}^{0}|| \leq \frac{\delta}{2} + \left(||A(
\widetilde{\xi}_{0})|| + w(\delta)\right) ||g_{\ell}(t,
\widetilde{\xi})||.\tag{A2.19} \label{app-2-eqA2.19}
\end{equation*}

Finally, it is immediate that
$$
\Lambda_{\ell}^{*} - \Lambda^{0}  \widetilde{h} = A(
\widetilde{\zeta}_{0}) D_{ \widetilde{\xi}}g_{\ell} (t,
\widetilde{\xi}) \cdot  \widetilde{h} - A( \widetilde{\xi}_{0}) D_{
  \widetilde{\xi}}g_{\ell} (0,  \widetilde{\xi}_{0}) \cdot  \widetilde{h}.
$$

Hence
\begin{align*}
||\Lambda_{\ell}^{*} - \Lambda_{\ell}^{0}|| & < ||A(
\widetilde{\zeta}_{0}) - A( \widetilde{\xi}_{0})|| ||D_{
  \widetilde{\xi}}g_{\ell} (0,  \widetilde{\xi}_{0})|| +\\
& + ||A( \widetilde{\zeta}_{0})|| ||D_{ \widetilde{\xi}}g_{\ell} (t,
\widetilde{\xi}) - D_{\widetilde{\xi}} g_{\ell} (0,  \widetilde{\xi}_{0})||.
\end{align*}

From the inequality $||A( \widetilde{\zeta}_{0})|| < ||A(
\widetilde{\xi}_{0})|| + w(\delta)$ again and by the definition of
$w(\delta)$, we find then
\begin{align*}
||\lambda_{\ell}^{*} - \Lambda_{\ell}^{0}|| & < w(\delta) ||D_{
  \widetilde{\xi}}g_{\ell}(0)|| + \\
& + (||A( \widetilde{\xi}_{0})|| + w(\delta)) ||D_{
  \widetilde{\xi}}g_{\ell} (t,  \widetilde{\xi}) - D_{
  \widetilde{\xi}}g_{\ell}(0,  \widetilde{\xi}_{0})||.\tag{A2.20}\label{app-2-eqA2.20}
\end{align*}

The combination of inequalities (\ref{app-2-eqA2.18}) - (\ref{app-2-eqA2.20}) yields the desired extimate (\ref{app-2-eqA2.8}).
\end{proof}

\medskip
\noindent{\textbf{Proof of Theorem 5.1:}} As\pageoriginale a first
step, we prove that the mapping $N_{\ell}(t,  \widetilde{\xi}_{0},
\cdot)$ is Lipschitz continuous with constant $\gamma$. The fact that
$N(t,\break\widetilde{\zeta}_{0}, \cdot)$ maps the ball $\triangle$ into
itself will be shown afterwards. The starting point is the estimate
(\ref{app-2-eqA2.8}) of Lemma \ref{app-2-lemA2.1}: From the relation $D_{
  \widetilde{\xi}}g_{\ell}(0,  \widetilde{\xi}_{0}) = D_{
  \widetilde{\xi}}g(0,  \widetilde{\xi}_{0})$ ($= Dq(
\widetilde{\xi}_{0})$, Chapter \ref{chap2}, $\S 3$) and writing
$g_{\ell} = g_{\ell} - g + g$, it is easily seen that
\begin{align*}
||D_{ \widetilde{\xi}}N_{\ell} (t,  \widetilde{\zeta}_{0},
\widetilde{\epsilon})|| & \leq \frac{3\delta}{(1 - \delta)^{3}} +
\frac{2||D_{ \widetilde{\xi}}g(0,  \widetilde{\xi}_{0})||}{1 - \delta}
w(\delta) +
 + \frac{2(||A( \widetilde{\xi}_{0})|| + w(\delta))}{1 - \delta}\\
&\left[\frac{3}{(1 - \delta)^{2}} ||g(t,  \widetilde{\xi})|| + ||D_{
    \widetilde{\xi}}g(t,  \widetilde{\xi}) - D_{ \widetilde{\xi}}g(0,
  \widetilde{\xi}_{0})||\right] +\\
&{} + \frac{2(||A( \widetilde{\xi}_{0})|| + w(\delta))}{1 - \delta}\\
&\left[\frac{3}{(1-\delta)^{2}} ||(g_{\ell} - g)(t,  \widetilde{\xi})||
+ ||D_{ \widetilde{\xi}}(g_{\ell} - g)(t,  \widetilde{\xi})||\right].
\end{align*}

Given any constant $\gamma > 0$, the sum of the first three terms of
the above inequality can clearly be made $\leq \gamma / 2$ provided $r
> 0$ and $\delta$ with $0 < \delta < 1$ are taken small enough. Fixing
then $r$, and $\delta$ (for the time being), the last term is uniformly
bounded by
$$
\frac{2(||A( \widetilde{\xi}_{0})|| + w(\delta))}{1-\delta}
\left[\frac{3}{(1-\delta)^{2}} |g_{\ell}-g|_{\infty, [-r, r] \times
    \triangle} + |D_{ \widetilde{\xi}}(g_{\ell}-g)|_{\infty, [-r, r]
    \times \triangle}\right].
$$

For $\ell$ large enough, say $\ell \geq \ell_{0}$, it can then be made
$\leq \gamma / 2$ as well. By the mean value theorem, the mapping
$N(t,  \widetilde{\zeta}_{0}, \cdot)$ is then Lipschitz continuous
with constant $\gamma$ in the ball $\triangle$ for every pair $(t,
\widetilde{\zeta}_{0}) \epsilon [-r, r] \times C$ and every $\ell \geq
\ell_{0}$. This property is not affected by shrinking $r > 0$ and $0 <
\delta < 1$ arbitrarily. Applying then Theorem \ref{chap5-thm3.1} to
each mapping $N_{\ell}, 0 \leq \ell \leq \ell_{0} - 1$ (thus a finite
number of times), it is not restrictive to assume that $r > 0$ and
$\delta$ with $0 < \delta < 1$ are such\pageoriginale that the mapping $N_{\ell}(t,
\widetilde{\zeta}_{0}, \cdot)$ is a contraction with constant $\gamma$
in the ball $\triangle$ for every $\ell \epsilon \mathbb{N}$. Again,
this property remains true if $r > 0$ is arbitrarily diminished. Let us
then fix $\delta$ as above. We shall show that $N_{\ell}(t,
\widetilde{\zeta}_{0}, \cdot)$ maps the ball $\triangle$ into itself
for every $\ell \epsilon \mathbb{N}$ after modifying $r > 0$ if
necessary.

For every triple $(t,  \widetilde{\zeta}_{0},  \widetilde{\xi})
\epsilon [-r, r] \times C \times \triangle$ and since $N(0,
\widetilde{\zeta}_{0},  \widetilde{\xi}_{0}) =  \widetilde{\xi}_{0}$
(cf. relation (\ref{chap4-eq2.10}) of Chapter \ref{chap4}), one has
\begin{align*}
& ||N_{\ell}(t,  \widetilde{\zeta}_{0},  \widetilde{\xi}) -
\widetilde{\xi}_{0}||  \leq ||N_{\ell}(t,  \widetilde{\zeta}_{0},
\widetilde{\xi}) - N_{\ell}(t,  \widetilde{\zeta}_{0},
\widetilde{\xi}_{0})|| +\\
& \qquad + ||N_{\ell}(t,  \widetilde{\zeta}_{0},  \widetilde{\xi}_{0}) - N (t , \widetilde{\zeta}_0, \widetilde{\xi}_0) || +
||N(t,  \widetilde{\zeta}_{0},  \widetilde{\xi}_{0}) -N(0,
\widetilde{\zeta}_{0},  \widetilde{\xi}_{0})||.
\end{align*}

The first term in the right hand side is bounded by $\gamma ||
\widetilde{\xi} -  \widetilde{\xi}_{0}|| \leq \gamma \delta / 2$. As
$\gamma < 1$ and further the mapping $N(\cdot, \cdot,
\widetilde{\xi}_{0})$ is uniformly continuous on the compact set $[-r,
r] \times C$, we may assume that $r > 0$ is small enough fro the
inequality
$$
||N(t,  \widetilde{\zeta}_{0},  \widetilde{\xi}_{0}) - N(0,
\widetilde{\zeta}_{0},  \widetilde{\xi}_{0})|| \leq (1 - \gamma)
\frac{\delta}{4}, 
$$
to hold. The term $||N_{\ell}(t,  \widetilde{\zeta}_{0},
\widetilde{\xi}_{0}) - N(t, \widetilde{\zeta}_{0},
\widetilde{\xi}_{0})||$ is bounded by 
$$|N_{\ell} - N|_{[-r, r] \times
  C \times \triangle}$$ 
and can therefore be made $\leq (1 -
\gamma)\delta / 4$ for $\ell$ large enough, say $\ell \geq
\ell_{0}$. Thus
$$
||N_{\ell}(t,  \widetilde{\zeta}_{0},  \widetilde{\xi}) -
\widetilde{\xi}_{0}|| \leq \frac{\delta}{2},
$$
for every triple $(t,  \widetilde{\zeta}_{0},  \widetilde{\xi})
\epsilon [-r, r] \times C \times \triangle$ and every $\ell \geq
\ell_{0}$. This property notbeing affected by diminishing $r > 0$
arbitrarily (without (modifying $\delta$, of course) and observing
that $N_{\ell} (0,  \widetilde{\zeta}_{0},  \widetilde{\xi}_{0}) =
\widetilde{\xi}_{0}$ for every\pageoriginale $ \widetilde{\zeta}_{0}
\epsilon C$, let us write for $0 \leq \ell \leq \ell_{0} - 1$
\begin{align*}
||N_{\ell} (t,  \widetilde{\zeta}_{0},  \widetilde{\xi}) -
\widetilde{\xi}_{0}|| & \leq ||N_{\ell} (t,  \widetilde{\zeta}_{0},
\widetilde{\xi}) - N_{\ell}(t,  \widetilde{\zeta}_{0},
\widetilde{\xi}_{0}) || + \\
& + ||N_{\ell} (t,  \widetilde{\zeta}_{0},  \widetilde{\xi}_{0}) -
N_{\ell}(0,  \widetilde{\zeta}_{0},  \widetilde{\xi}_{0})||.
\end{align*}

The first term in the right hand side is bounded by $\gamma ||
\widetilde{\xi} -  \widetilde{\xi}_{0}|| \leq \gamma \delta /
2$. Owing to the uniform continuity if the mapping $N_{\ell} (\cdot,
\cdot,  \widetilde{\xi}_{0})$ on the compact set $[-r, r] \times C$,
we may assume that $r > 0$ is small enough for the inequality
$$
||N_{\ell} (t,  \widetilde{\zeta}_{0},  \widetilde{\xi}_{0}) -
N_{\ell} (0,  \widetilde{\zeta}_{0},  \widetilde{\xi}_{0})|| \leq (1 -
\gamma) \delta / 2,
$$
to hold for $0 \leq \ell \leq \ell_{0} - 1$ and, for these indices as
well, we get
$$
||N_{\ell} (t,  \widetilde{\zeta}_{0},  \widetilde{\xi}) -
\widetilde{\xi}_{0}|| \leq \frac{\delta}{2},
$$
which completes the proof.

In chapter \ref{chap4}, Theorem \ref{chap4-thm5.1} has been used for
proving the convergence of the sequence $ \widetilde{x}_{\ell} = t
\widetilde{\zeta}_{\ell}$ (where $ \widetilde{\zeta}_{\ell + 1} =
N_{\ell}(t,  \widetilde{\zeta}_{0},  \widetilde{\zeta}_{\ell})$) to $
\widetilde{x} = t \widetilde{\xi}$ (where $ \widetilde{\xi}$ is the
unique fixed point of the mapping $N(t,  \widetilde{\zeta}_{0},
\cdot)$ in the ball $\triangle$). We shall now give an estimate of the
rate of convergence of the sequence $( \widetilde{x}_{\ell})$ to $
\widetilde{x}$ under suitable assumptions on the rate of convergence
of the sequence $(f_{\ell})$ to $f$ in appropriate spaces.

\begin{alphtheorem}\label{app-2-thmA2.1}%[A2.1]
(i) Assume that the sequence $(f_{\ell})$ tends to f geometrically in
  the space $C^{k}(\overline{\mathscr{O}}, \mathbb{R}^{n})$. Then,
  there are constant $0 < \gamma' < 1$ and $K > 0$ such that, for
  every $t \epsilon [-r, r]$
\begin{equation*}
|| \widetilde{x}_{\ell} -  \widetilde{x}|| \leq K|t| \gamma'^{\ell},\tag{A2.21}\label{app-2-eqA2.21}
\end{equation*}
for every $\ell \geq 0$.

(ii) Assume\pageoriginale only that the sequence $(f_{\ell})$ tends to
f in the space $C^{k} (\overline{\mathscr{O}},\break \mathbb{R}^{n})$, the
convergence being geometrical in the space $C^{k-1}
(\overline{\mathscr{O}}, \mathbb{R}^{n})$. Then, there are constant $0
< \gamma' < 1$ and $K > 0$ such that, for every $t \epsilon [-r, r]$
\begin{equation*}
|| \widetilde{x}_{\ell} -  \widetilde{x} \leq K\gamma'^{\ell}.\tag{A2.22}\label{app-2-eqA2.22}
\end{equation*}
\end{alphtheorem}


\begin{proof}
From the relation
\begin{align*}
 \widetilde{\zeta}_{\ell + 1} -  \widetilde{\xi} & = N_{\ell}(t,
 \widetilde{\zeta}_{0},  \widetilde{\zeta}_{\ell}) - N(t,
 \widetilde{\zeta}_{0},  \widetilde{\xi})\\
& = N_{\ell} (t,  \widetilde{\zeta}_{0},  \widetilde{\zeta}_{\ell}) -
 N_{\ell} (t,  \widetilde{\zeta}_{0},  \widetilde{\xi}) + N_{\ell}(t,
 \widetilde{\zeta}_{0},  \widetilde{\xi}) - N(t,
 \widetilde{\zeta}_{0},  \widetilde{\xi}),
\end{align*}
and from Theorem \ref{chap4-thm5.1} of Chapter \ref{chap4}, we get
$$
|| \widetilde{\zeta}_{\ell + 1} -  \widetilde{\xi}|| < \gamma ||
\widetilde{\zeta}_{\ell} -  \widetilde{\xi}|| + ||N_{\ell} (t,
\widetilde{\zeta}_{0},  \widetilde{\xi}) - N(t,
\widetilde{\zeta}_{0},  \widetilde{\xi})||.
$$

Hence
\begin{equation*}
|| \widetilde{\zeta}_{\ell + 1} -  \widetilde{\xi}|| \leq \gamma ||
\widetilde{\zeta}_{\ell} -  \widetilde{\xi}|| + |N_{\ell}(t, \cdot,
\cdot) - N (t, \cdot, \cdot)|_{\infty, C \times \triangle},\tag{A2.23}\label{app-2-eqA2.23}
\end{equation*}
for every $t \epsilon [-r, r]$. To prove (i), observe that
$|N_{\ell} (t, \cdot, \cdot) - N (t, \cdot, \cdot)|_{\infty, C \times
  \triangle} \leq |N - N_{\ell}|_{\infty, [-r, r] \times C \times
  \triangle}$. From (\ref{app-2-eqA2.6}) - (\ref{app-2-eqA2.7}) and (\ref{app-2-eqA2.1})
\begin{align*}
|| \widetilde{\zeta}_{\ell + 1} -  \widetilde{\xi}|| & \leq \gamma ||
\widetilde{\zeta}_{\ell} -  \widetilde{\xi}|| + \frac{2(||A(
  \widetilde{\xi}_{0})|| + w(\delta))}{1 - \delta} |g_{\ell} -
g|_{\infty, [-r, r] \times \triangle}\\
& \leq \gamma || \widetilde{\zeta}_{\ell} -  \widetilde{\xi}|| +
\frac{2(||A( \widetilde{\xi}_{0})|| + w(\delta))}{1 - \delta} \left(1
+ \frac{\delta}{2}\right)^{k} |D^{k}(f_{\ell} - f)|_{\infty,
  \overline{\mathscr{O}}}. 
\end{align*}

As the sequence $(f_{\ell})$ tends to $f$ geometrically in the space
$C^{k} (\overline{\mathscr{O}}, \mathbb{R}^{n})$ by hypothesis and
after increasing $0 < \gamma < 1$ if necessary, there is a constant $C
> 0$ such that
$$
|| \widetilde{\zeta}_{\ell + 1} -  \widetilde{\xi}|| \leq \gamma ||
\widetilde{\zeta}_{\ell} -  \widetilde{\xi}|| + C \gamma^{\ell}.
$$

By\pageoriginale a simple induction argument, we find
$$
||\widetilde{\zeta}_{\ell + 1} - \widetilde{\xi}|| \leq \gamma^{\ell +
1} ||\widetilde{\zeta}_{0} - \widetilde{\xi}|| + (\ell + 1) C\gamma^{\ell}.
$$

Thus
$$
||\widetilde{\zeta}_{\ell} - \widetilde{\xi}|| \leq \gamma^{\ell}
\left(||\widetilde{\zeta}_{0} - \widetilde{\xi}|| + \frac{\ell C}{\gamma}\right),
$$
for every $\ell \geq 0$. As $\widetilde{\zeta}_{0} \epsilon C \subset
\triangle$ and $\widetilde{\xi} \epsilon C \subset \triangle$ and the
ball $\triangle$ has diameter $\delta < 1$
\begin{equation*}
||\widetilde{\zeta}_{\ell} - \widetilde{\xi}|| \leq \gamma^{\ell} \left(1 +
\frac{\ell C}{\gamma}\right).\tag{A2.24}\label{app-2-eqA2.24}
\end{equation*}

Multiplying by $|t|$, we see that
$$
||\widetilde{x}_{\ell} - \widetilde{x}|| \leq |t| \gamma^{\ell}
\left(1 + \frac{\ell C}{\gamma}\right).
$$

Choosing $\gamma < \gamma' < 1$, inequality (\ref{app-2-eqA2.21}) follows with
$$
K = \sup_{\ell \geq 0} \left(\frac{\gamma}{\gamma'}\right)^{\ell}
\left(1 + \frac{\ell C}{\gamma}\right) < + \infty.
$$

To prove (ii), we use the same method, replacing the relations (\ref{app-2-eqA2.6}) and (\ref{app-2-eqA2.7}) by \ref{app-2-eqA2.6'} and \ref{app-2-eqA2.7'}. With (\ref{app-2-eqA2.23}), we get
$$
||\widetilde{\zeta}_{\ell + 1} - \widetilde{\xi}|| \leq \gamma
||\widetilde{\zeta}_{\ell} - \widetilde{\xi}|| +
\frac{2(||A(\widetilde{\xi}_{0})|| + w(\delta))}{1 - \delta} |g_{\ell}
(t, \cdot) - g(t, \cdot)|_{\infty, \triangle}.
$$

Now, note that
$$
|g_{\ell} (t, \cdot) - g(t, \cdot)|_{\infty, \triangle} \leq
\frac{k}{|t|} \left(1 + \frac{\delta}{2}\right)^{k-1} |D^{k-1}
(f_{\ell} - f)|_{\infty, \overline{\mathscr{O}}}.
$$

Indeed, this is obvious if $k = 1$. If $k \geq 2$, the inequality
follows from the relation
$$
g_{\ell}(t, \widetilde{\zeta}) - g(t, \widetilde{\zeta}) = \frac{k(k -
  1)}{t} \int_{0}^{1} (1 - s)^{k-1} D^{k-1} (f_{\ell} - f)(st
\widetilde{\zeta}) \cdot (\widetilde{\zeta})^{k-1} ds.
$$

Arguing\pageoriginale as before, we find the analogue of (\ref{app-2-eqA2.24}),
namely
$$
||\widetilde{\zeta}_{\ell} - \widetilde{\xi}|| \leq \gamma^{\ell} \left(1 +
 \frac{\ell C}{\gamma |t|}\right),
$$
for every $\ell \geq 0$. Multiplying by $|t| > 0$, we obtain
$$
||\widetilde{x}_{\ell} - \widetilde{x}|| \leq \gamma^{\ell} \left(|t|
+ \frac{\ell C}{\gamma}\right) \leq \gamma^{\ell} \left(r + \frac{\ell
  C}{\gamma}\right). 
$$

Choosing $\gamma < \gamma' < 1$, the inequality (\ref{app-2-eqA2.22}) follows with
$$
K = \sup_{\ell \geq 0} \left(\frac{\gamma}{\gamma'}\right)^{\ell}
\left(r + \frac{\ell C}{\gamma}\right) < + \infty,
$$
and remains valide for $t = 0$.
\end{proof}

We shall now prove the results about convergence in the spaces of type
$C^{k}$ that we used in $\S 6$ of Chapter \ref{chap4}. Recall that
given a real Banach space $\widetilde{Z}$ and a mapping $\Phi(= \Phi
(\widetilde{x}, \widetilde{z})) \epsilon C^{k} (\overline{\mathscr{O}}
\times \overline{B} (0, \rho), \widetilde{Z})$, $k \geq 1$, where
$\overline{B}(0, \rho)$ denotes the closed ball with radius $\rho > 0$
centered at the origin of $\widetilde{Z}$, verifying $\Phi(0) = 0$ and
$D_{\widetilde{z}}\Phi(0) = 0$, it is possible to shrink $\rho > 0$
and the neighbourhood $\mathscr{O}$ so that, given any arbitrary
constant $0 < \beta < 1$, one has
\begin{equation*}
||D_{\widetilde{z}} \Phi(\widetilde{x}, \widetilde{z})|| \leq \beta\tag{A2.25}\label{app-2-eqA2.25}
\end{equation*}
for every $(\widetilde{x}, \widetilde{z}) \epsilon
\overline{\mathscr{O}} \times \overline{B}(0, \rho)$ and the mapping
$\Phi(\widetilde{x}, \cdot)$ is a contraction with constant $\beta$
from $\overline{B}(0, \rho)$ to itself for every $\widetilde{x}
\epsilon \overline{\mathscr{O}}$. If so, for $\widetilde{x} \epsilon
\overline{\mathscr{O}}$, the sequence
\begin{equation*}
\begin{cases}
\widetilde{\varphi}_{0} (\widetilde{x}) = 0,\\
\widetilde{\varphi}_{\ell + 1} (\widetilde{x}) = \Phi(\widetilde{x},
\widetilde{\varphi}_{\ell} (\widetilde{x})), \ell \geq 0,
\end{cases}
\end{equation*}
is well defined, each mapping $\widetilde{\varphi}_{\ell}$ being in
the space $C^{k}(\overline{\mathscr{O}}, \widetilde{Z})$ with values
in $\overline{B}(0, \rho)$. We already know that this sequence
converges pointwise to\pageoriginale the mapping $\widetilde{\varphi}
\epsilon C^{k} (\overline{\mathscr{O}}, \widetilde{Z})$ characterized
by
$$
\widetilde{\varphi} (\widetilde{x}) = \Phi(\widetilde{x},
\widetilde{\varphi}(\widetilde{x})), 
$$
for every $\widetilde{x} \epsilon \overline{\mathscr{O}}$. In $\S 6$
of Chapter \ref{chap4}, we used the fact that the sequence
$(\widetilde{\varphi}_{\ell})$ tends to $\widetilde{\varphi}$ in the
space $C^{k}(\overline{\mathscr{O}}, \widetilde{Z})$, the convergence
being geometrical in the space $C^{k-1} (\overline{\mathscr{O}},
\widetilde{Z})$. Proving this assertion will take us a certain amount
of time. To begin with, we show that


\begin{alphlemma}\label{app-2-lemA2.2}%[A2.2]
The sequence $(\widetilde{\varphi}_{\ell})$ tends to
$\widetilde{\varphi}$ in the space $C^{0} (\overline{\mathscr{O}},
\widetilde{Z})$. 
\end{alphlemma}

\begin{proof}
Let $\ell \geq 1$ be fixed. For every $\widetilde{x} \epsilon
\overline{\mathscr{O}}$ 
\begin{align*}
\widetilde{\varphi}_{\ell + 1} (\widetilde{x}) -
\widetilde{\varphi}_{\ell} (\widetilde{x}) & = \Phi(\widetilde{x},
\widetilde{\varphi}_{\ell}(\widetilde{x})) - \Phi(\widetilde{x},
\widetilde{\varphi}_{\ell - 1} (\widetilde{x})) \\
& = \int_{0}^{1} D_{\widetilde{z}}\Phi (\widetilde{x},
\widetilde{\varphi}_{\ell - 1} (\widetilde{x}) +
s(\widetilde{\varphi}_{\ell} (\widetilde{x}) -
\widetilde{\varphi}_{\ell - 1} (\widetilde{x}))) \cdot\\
&\hspace{4cm}(\widetilde{\varphi}_{\ell}(\widetilde{x}) - \widetilde{\varphi}_{\ell
- 1} (\widetilde{x})) ds.
\end{align*}

Due to (\ref{app-2-eqA2.25}) we obtain
$$
||\widetilde{\varphi}_{\ell + 1} (\widetilde{x}) -
\widetilde{\varphi}_{\ell} (\widetilde{x})|| \leq \beta
||\widetilde{\varphi}_{\ell} (\widetilde{x}) -
\widetilde{\varphi}_{\ell - 1} (\widetilde{x})||,
$$
for every $\widetilde{x} \epsilon \overline{\mathscr{O}}$. Thus
$$
|\widetilde{\varphi}_{\ell + 1} - \widetilde{\varphi}_{\ell}|_{\infty,
\overline{\mathscr{O}}} \leq \beta |\widetilde{\varphi}_{\ell} -
\widetilde{\varphi}_{\ell - 1}|_{\infty, \overline{\mathscr{O}}}.
$$

Hence, the sequence $(\widetilde{\varphi}_{\ell})$ tends
(geometrically) to a limit in the space $C^{0}(\overline{\mathscr{O}},
\widetilde{Z})$. Of course, this limit must be the pointwise limit
$\widetilde{\varphi}$ and the proof is complete.

For $j \geq 0$, let us denote by $\mathscr{L}_{j} (\mathbb{R}^{n+1},
\widetilde{Z})$ the space of $j$-linear mappings from $\mathbb{R}^{n+1}$
into $\widetilde{Z}$ with the usual abuse of notation
$\mathscr{L}_{0}(\mathbb{R}^{n+1}, \widetilde{Z}) =
\widetilde{Z}$. Also, recall the canonical isomorphism
\begin{equation*}
\mathscr{L}_{i+1} (\mathbb{R}^{n+1}, \widetilde{Z}) \simeq
\mathscr{L}(\mathbb{R}^{n+1}, \mathscr{L}_{i}(\mathbb{R}^{n+1},
\widetilde{Z})),\tag{A2.26} \label{app-2-eqA2.26}
\end{equation*}
which\pageoriginale will be repeatedly used in sequel.
\end{proof} 

We shall denote by $\lambda_{i}$ the generic element of the space
$\mathscr{L}_{i}(\mathbb{R}^{n+1}, \widetilde{Z})$. Due to the
identification $\mathscr{L}_{0} (\mathbb{R}^{n+1}, \widetilde{Z}) =
\widetilde{Z}$, this means, in particular, that we shall identify the
element $\widetilde{z} \epsilon \widetilde{Z}$ with $\lambda_{0}$ so
that the assumption $D_{\widetilde{x}}\Phi(0) = 0$ will be rewritten
as
\begin{equation*}
D_{\lambda_{0}} \Phi(0) = 0.\tag{A2.27}\label{app-2-eqA2.27}
\end{equation*}

Now, setting $\Phi = \Phi_{0}$, introduce the mappings
$$
\Phi_{j} : \overline{\mathscr{O}} \times \overline{B}(0, \rho) \times
\prod_{i=1}^{j} \mathscr{L}_{i} (\mathbb{R}^{n+1}, \widetilde{Z}) \to
\mathscr{L}_{j}(\mathbb{R}^{n+1}, \widetilde{Z}), 1 \leq j \leq k,
$$
by 
\begin{align*}
\Phi_{j} (\widetilde{x}, \lambda_{0}, \cdots, \lambda_{j}) & =
\frac{\partial \Phi_{j-1}}{\partial x} (\widetilde{x}, \lambda_{0},
\cdots, \lambda_{j-1}) + \tag{A2.28} \label{app-2-eqA2.28}\\
& + \sum\limits_{i=0}^{j-1} \frac{\partial \Phi_{j-1}}{\partial
  \lambda_{i}} (\widetilde{x}, \lambda_{0}, \cdots, \lambda_{j-1}) \lambda_{i+1},
\end{align*}
where the term $\frac{\partial \Phi_{j-1}}{\partial \lambda_{i}}
(\widetilde{x}, \lambda_{0}, \cdots, \lambda_{j-1}) \lambda_{i+1}$ is
the product of the linear mappings 
$$\frac{\partial
  \Phi_{j-1}}{\partial \lambda_{i}} (\widetilde{x}, \lambda_{0},
\cdots, \lambda_{j-1}) \epsilon \mathscr{L}
(\mathscr{L}_{i}(\mathbb{R}^{n+1}, \widetilde{Z}), \mathscr{L}_{j-1}
(\mathbb{R}^{n+1}, \widetilde{Z}))$$ and 
$$\lambda_{i+1} \epsilon
\mathscr{L} (\mathbb{R}^{n}, \mathscr{L}_{i}(\mathbb{R}^{n+1}, \widetilde{Z})).$$

\begin{alphrem}\label{app-2-remA2.1}%[A2.1]
Observe, in particular, that the only term involving $\lambda_{j}$ in
the definition of $\Phi_{j}, j \geq 1$, is the term $\frac{\partial
  \Phi_{j-1}}{\partial \lambda_{j-1}} (\widetilde{Z}, \lambda_{0},
\cdots, \lambda_{j-1}) \lambda_{j} $ it follows that $\Phi_{j}
(\widetilde{x}, \lambda_{0}, \cdots, \lambda_{j})$ is {\em linear with
respect to $\lambda_{j}$ when $j \geq 1$}.
\end{alphrem}

The importance of the mappings $\Phi_{j}, 0 \leq j \leq k$ lies in the
fact that
\begin{equation*}
D^{j}\widetilde{\varphi}(\widetilde{x}) = \Phi_{j}(\widetilde{x},
\widetilde{\varphi}(\widetilde{x}),
D\widetilde{\varphi}(\widetilde{x}), \cdots,
D^{j}\widetilde{\varphi}(\widetilde{x})), 0 \leq j \leq k\tag{A2.29} \label{app-2-eqA2.29}
\end{equation*}
for\pageoriginale every $\widetilde{x} \epsilon
\overline{\mathscr{O}}$ and, for every $\ell \geq 0$
\begin{equation*}
D^{j}\widetilde{\varphi}_{\ell + 1}(\widetilde{x}) =
\Phi_{j}(\widetilde{x}, \widetilde{\varphi}_{\ell}(\widetilde{x}),
D\widetilde{\varphi}_{\ell}(\widetilde{x}), \cdots,
D^{j}\widetilde{\varphi}_{\ell}(\widetilde{x})), 0 \leq j \leq k,\tag{A2.30}\label{app-2-eqA2.30}
\end{equation*}
for every $x \epsilon \overline{\mathscr{O}}$. These properties can be
immediately checked by induction. Also, it is clear that the mapping
$\Phi_{j}$ is of class $C^{k-j}$ and, for $1 \leq j \leq k, \Phi_{j}$
is of class $C^{\infty}$ with respect to $(\lambda_{1}, \cdots,
\lambda_{j})$. We now establish two simple preliminary lemmas.

\begin{alphlemma}\label{app-2-lemA2.3}%[A2.3]
Given any index $0 \leq j \leq k$, one has
\begin{equation*}
\frac{\partial \Phi_{j}}{\partial \lambda_{j}} (0, 0, \lambda_{1},
\cdots, \lambda_{j}) = 0,\tag{A2.31}\label{app-2-eqA2.31}
\end{equation*}
for every $(\lambda_{1}, \cdots, \lambda_{j}) \epsilon \prod\limits_{i=1}^{j}
\mathscr{L}_{i} (\mathbb{R}^{n+1}, \widetilde{Z})$.
\end{alphlemma}

\begin{proof}
For $ j = 0$, the result is nothing but (\ref{app-2-eqA2.27}). For $j \geq 1$, it
follows from Remark \ref{app-2-remA2.1} that for every $\mu_{j} \epsilon \mathscr{L}
(\mathbb{R}^{n+1}, \widetilde{Z})$
$$
\frac{\partial \Phi_{j}}{\partial \lambda_{j}} (\widetilde{x},
\lambda_{0}, \cdots, \lambda_{j}) \cdot \mu_{j} = \frac{\partial
  \Phi_{j-1}}{\partial \lambda_{j-1}} (\widetilde{x}, \lambda_{0},
\cdots, \lambda_{j-1}) \mu_{j}.
$$

Choosing $\widetilde{x} = 0, \lambda_{0} = 0$, the above relation
yields the desired result by an indeuction argument.
\end{proof}

\begin{alphlemma}\label{app-2-lemA2.4}%[A2.4]
Let the index $0 \leq  j \leq k $ be fixed. If the sequence
$(\widetilde{\varphi}_{\ell})$ tends to $\widetilde{\varphi}$ in the
space $C^{j}(\overline{\mathscr{O}}, \widetilde{Z})$ (which is already
known for $j = 0$), the set
\begin{equation*}
\Lambda_{j} = \bigcup_{\ell \geq 0} D^{j}\widetilde{\varphi}_{\ell}
(\overline{\mathscr{O}}) \subset \mathscr{L}_{j} (\mathbb{R}^{n+1},
\widetilde{Z}),\tag{A2.32} \label{app-2-eqA2.32}
\end{equation*}
is compact and one has
\begin{equation*}
D^{j}\widetilde{\varphi}(\overline{\mathscr{O}}) \subset \Lambda_{j}.\tag{A2.33}\label{app-2-eqA2.33}
\end{equation*}
\end{alphlemma}

\begin{proof}
A\pageoriginale sequence in the set $\bigcup\limits_{\ell \geq 0}
D^{j}\widetilde{\varphi}_{\ell} (\overline{\mathscr{O}})$ is of the
form $(\lambda_{j}^{(p)})_{p > 0}$ where for each $p \epsilon
\mathbb{N}$ there is an index $\ell = \ell(p)$ and an element
$\widetilde{x}_{p} \epsilon \overline{\mathscr{O}}$ such that
$\lambda_{j}^{(p)} = D^{j}\widetilde{\varphi}_{\ell(p)}
(\widetilde{x}_{p})$. If, for infinitely many indices $p$, namely for a
subsequence $(p_{m})$, the index $\ell(p_{m})$ equals some fixed value
$\ell$, one has
$$
\lambda_{j}^{(p_{m})} = D^{j} \widetilde{\varphi}_{\ell} (\widetilde{x}_{p_{m}}).
$$

As the set $\overline{\mathscr{O}}$ is compact and after extracting a
subsequence, we may assume that there is $\widetilde{x} \epsilon
\overline{\mathscr{O}}$ such that the sequence
$(\widetilde{x}_{p_{m}})$ tends to $\widetilde{x}$. By continuity of
the mapping $D^{j}\widetilde{\varphi}_{\ell}$, we deduce that the
sequence $\lambda_{j}^{(p_{m})}$ tends to
$D^{j}\widetilde{\varphi}_{\ell}(\widetilde{x})$ as $m$ tends to $+
\infty$.

Now, assume that the mapping $p \to \ell(p)$ takes only finitely many
times any given value $\ell$. Then, there is a subsequence
$\ell(p_{m})$ which is strictly increasing and thus tends to $+
\infty$ as $m$ tends to $+ \infty$. For every $m$, let us write
$$
D^{j}\widetilde{\varphi}_{\ell(p_{m})} (\widetilde{x}_{p_{m}}) =
D^{j}\widetilde{\varphi}_{\ell(p_{m})} (\widetilde{x}_{p_{m}}) -
D^{j}\widetilde{\varphi}(\widetilde{x}_{p_{m}}) +
D^{j}\widetilde{\varphi}(\widetilde{x}_{p_{m}}). 
$$

Again, in view of the compactness of the set $\overline{\mathscr{O}}$,
we may assume that the sequence $(\widetilde{x}_{p_{m}})$ tends to
$\widetilde{x} \epsilon \overline{\mathscr{O}}$. As the sequence
$(D^{j} \widetilde{\varphi}_{\ell})_{\ell \geq 0}$ tends to
$D^{j}\widetilde{\varphi}$ in the space $C^{0}
(\overline{\mathscr{O}}, \mathscr{L}_{j}(\mathbb{R}^{n+1},
\widetilde{Z}))$, by hypothesis, the same property holds for the
subsequence $(D^{j}\widetilde{\varphi}_{\ell(p_{m})})_{m \geq
  0}$. Therefore, in the space $\mathscr{L}_{j}(\mathbb{R}^{n+1},
\widetilde{Z})$ 
$$
\lim_{m \to + \infty} \left[D^{j}\widetilde{\varphi}_{\ell(p_{m})}
  (\widetilde{x}_{p_{m}}) -
  D^{j}\widetilde{\varphi}(\widetilde{x}_{p_{m}})\right] = 0
$$

On the other hand, by the continuity of the mapping
$D^{j}\widetilde{\varphi}$, we get
$$
\lim_{m \to + \infty} D^{j}\widetilde{\varphi}(\widetilde{x}_{p_{m}})
= D^{j}\widetilde{\varphi}(\widetilde{x}).
$$\pageoriginale

As a result, the sequence $(\lambda_{j}^{(p_{m})})_{m \geq 0}$ tends
to $D^{j}\widetilde{\varphi}(\widetilde{x})$.

To sum up, we have shown that every sequence $(\lambda_{j}^{(p)})_{p
  \geq 0}$ of the set $\bigcup\limits_{\ell \geq 0}
D^{j}\widetilde{\varphi}_{\ell} (\overline{\mathscr{O}})$ has a
cluster point in the space $\mathscr{L}_{j}(\mathbb{R}^{n+1},
\widetilde{Z})$ and hence is relatively compact, which proves
(\ref{app-2-eqA2.32}). The relation (\ref{app-2-eqA2.33}) is now obvious since
$D^{j}\widetilde{\varphi}(\widetilde{x})$ is the limit of the sequence
$(D^{j}\widetilde{\varphi}_{\ell} (\widetilde{x}))_{\ell \geq 0}$ for
every $\widetilde{x} \epsilon \overline{\mathscr{O}}$.
\end{proof}

In oure new notation, the condition (\ref{app-2-eqA2.25}) can be rewritten as
\begin{equation*}
||\frac{\partial \Phi_{0}}{\partial \lambda_{0}} (\widetilde{x},
\lambda_{0})|| \leq \beta,\tag{A2.34}\label{app-2-eqA2.34}
\end{equation*}
for every $(\widetilde{x}, \lambda_{0}) \epsilon
\overline{\mathscr{O}} \times \overline{B} (0, \rho)$. It will be
essential in the sequel to have an appropriate generalization of it,
which we establish in the following lemma.

\begin{alphlemma}\label{app-2-lemA2.5}%[A2.5]
Let the index $0 \leq j \leq k$ be fixed. If the sequence
$(\widetilde{\varphi}_{\ell})$ tends to $\widetilde{\varphi}$ in the
space $C^{j}(\overline{\mathscr{O}}, \widetilde{Z})$ (which is
  already known for $j = 0$) there is no loss of generality in
  assuming that
\begin{equation*}
||\frac{\partial \Phi_{j}}{\partial \lambda_{j}} (\widetilde{x},
\lambda_{0}, \cdots, \lambda_{j})|| \leq \beta,\tag{A2.35}\label{app-2-eqA2.35}
\end{equation*}
for every $(\widetilde{x}, \lambda_{0}, \cdots, \lambda_{j}) \epsilon
\overline{\mathscr{O}} \times \Lambda_{0} \times \cdots \times
\Lambda_{j}$.\footnote{For $j = 0$, (\ref{app-2-eqA2.35}) is nothing but (\ref{app-2-eqA2.34}) for the elements $(\widetilde{x}, \lambda_{0}) \epsilon
\overline{\mathscr{O}} \times \Lambda_{0}$, thus a weak form of (\ref{app-2-eqA2.34}).}
\end{alphlemma}

\begin{proof}
First from the previous lemma, the set $\overline{\mathscr{O}} \times
\Lambda_{0} \times \cdots \times \Lambda_{j}$ is compact and the
restriction of the continuous mapping $(\partial \Phi_{j} / \partial
\lambda_{j})$ to $\overline{\mathscr{O}} \times \Lambda_{0} \times
\cdots \times \Lambda_{j}$\pageoriginale is then {\em uniformly
  continuous}. Next, since the mappings $\widetilde{\varphi}$ and
$\widetilde{\varphi}_{\ell}, \ell \geq 0$, take their values in the
ball $\overline{B}(0, \rho)$, one has $\Lambda_{0} \subset
\overline{B}(0, \rho)$. Due to Lemma \ref{app-2-lemA2.3}, and the uniform continuity
of the mapping $(\partial \Phi_{j} / \partial \lambda_{j})$, there is
a neighbourhood $\mathscr{O}' \subset \mathscr{O}$ of the origin in
$\mathbb{R}^{n+1}$ and $0 < \rho' \leq \rho$ such that (\ref{app-2-eqA2.35}) holds
for
$(\widetilde{x}, \lambda_{0}, \cdots, \lambda_{j}) \epsilon
\overline{\mathscr{O}}' \times \left[\Lambda_{0} \cap \overline{B} (0,
  \rho')\right] \times \Lambda_{1} \times \cdots \times
\Lambda_{j}$. Let then $\rho'$ be fixed. Arguing as in $\S 6$ of
Chapter \ref{chap4}, we can shrink the neighbourhood
$\overline{\mathscr{O}}'$ so that the mappings $\widetilde{\varphi},
\ell \geq 0$, take their values in the ball $\overline{B}(0, \rho')$
for $x \epsilon \overline{\mathscr{O}}'$. Of course, the sequence
$(\widetilde{\varphi}_{\ell})$ still tends to $\widetilde{\varphi}$ in
the space $C^{j}(\overline{\mathscr{O}}, \widetilde{Z})$ and the sets
$\Lambda'_{i} = \overline{\bigcup\limits_{\ell \geq 0}
  D^{i}\widetilde{\varphi}_{\ell} (\overline{\mathscr{O}}')}, 0 \leq i
\leq j$, are compact (Lemma \ref{app-2-lemA2.4}) with $\Lambda'_{i} \subset
\Lambda_{i}$. Also, $\Lambda'_{0} \subset \overline{B}(0, \rho')$ so
that inequality (\ref{app-2-eqA2.35}) is a fortiori valid with $(\widetilde{x},
\lambda_{0}, \cdots, \lambda_{k}) \epsilon \overline{\mathscr{O}}'
\times \Lambda'_{0} \times \cdots \times \Lambda'_{j}$. In other
words, none of the properties we have proved is affected if we
restrict ourselves to the neighbourhood $\mathscr{O}'$ instead of
$\mathscr{O}$ and, in addition, inequality (\ref{app-2-eqA2.35}) holds.
\end{proof}

For $0 \leq j \leq k-1$ and $\widetilde{x} \epsilon
\overline{\mathscr{O}}$, we shall set
\begin{align*}
b^{j}(\widetilde{x}) & = \frac{\partial \Phi_{j}}{\partial
  \widetilde{x}} (\widetilde{x}, \widetilde{\varphi}(\widetilde{x}),
D\widetilde{\varphi}(\widetilde{x}), \cdots,
D^{j}\widetilde{\varphi}(\widetilde{x})),\tag{A2.36}\label{app-2-eqA2.36}\\
b_{\ell}^{i}(\widetilde{x}) & = \frac{\partial \Phi_{j}}{\partial
  \widetilde{x}} (\widetilde{x}, \widetilde{\varphi}_{\ell}
(\widetilde{x}), D\widetilde{\varphi}_{\ell}(\widetilde{x}), \cdots,
D^{j}\widetilde{\varphi}_{\ell}(\widetilde{x})) \text{ for } \ell \geq
0.\tag{A2.37} \label{app-2-eqA2.37}
\end{align*}

Similarly, for $0 \leq j \leq k-1$, $0 \leq i \leq j$ and
$\widetilde{x} \epsilon \overline{\mathscr{O}}$, set
\begin{align*}
B_{i}^{j}(\widetilde{x}) & = \frac{\partial \Phi_{j}}{\partial
  \lambda_{i}} (\widetilde{x}, \widetilde{\varphi}(\widetilde{x}),
D\widetilde{\varphi}(\widetilde{x}), \cdots,
D^{j}\widetilde{\varphi}(\widetilde{x})),\tag{A2.38}\label{app-2-eqA2.38}\\
B_{i \ell}^{j}(\widetilde{x}) & = \frac{\partial \Phi_{j}}{\partial
  \lambda_{i}} (\widetilde{x}, \widetilde{\varphi}_{\ell}
(\widetilde{x}), D\widetilde{\varphi}_{\ell}(\widetilde{x}), \cdots,
D^{j}\widetilde{\varphi}_{\ell}(\widetilde{x})) \text{ for } \ell \geq
0.\tag{A2.39} \label{app-2-eqA2.39}
\end{align*}
For\pageoriginale every $\widetilde{x} \epsilon
\overline{\mathscr{O}}, b^{j}(\widetilde{x})$
(resp. $b_{\ell}^{j}(\widetilde{x}))$ is an element of the space
$\mathscr{L} (\mathbb{R}^{n+1}$, $\mathscr{L}_{j} (\mathbb{R}^{n+1},
\widetilde{Z}))$ and $B_{i}^{j}(\widetilde{x})$ (resp. $B_{i \ell}^{j}
(\widetilde{x})$) is an element of the space
$\mathscr{L}(\mathscr{L}_{i} $ $ (\mathbb{R}^{n+1}, \widetilde{Z}),
\mathscr{L}_{j} (\mathbb{R}^{n + 1}, \widetilde{Z}))$.

\begin{alphlemma}\label{app-2-lemA2.6}%[A2.6]
(i) Assume that $k \geq 2$ and $0 \leq j \leq k-2$. Then, if the
  sequence $(\widetilde{\varphi}_{\ell})$ tends to
  $\widetilde{\varphi}$ geometrically in the space
  $C^{j}(\overline{\mathscr{O}}, \widetilde{Z})$, the sequence
  $(b_{\ell}^{j})_{\ell \geq 0}$ tends to $b^{j}$ geometrically in the
  space\break $C^{0}(\overline{\mathscr{O}}, \mathscr{L}(\mathbb{R}^{n+1},
  \mathscr{L}_{j}(\mathbb{R}^{n+1},  \widetilde{Z})))$ and the
  sequence $(B_{i\ell}^{j})_{\ell \geq 0}$ tends to $B_{i}^{j}$
  geometrically in the space $C^{0}(\overline{\mathscr{O}},
  \mathscr{L}(\mathscr{L}_{i}(\mathbb{R}^{n+1},  \widetilde{Z}),
  \mathscr{L}_{j}(\mathbb{R}^{n+1},  \widetilde{Z})))$.

(ii) Assume only $k \geq 1$ and $0 \leq j \leq k - 1$ and that the
  sequence $( \widetilde{\varphi}_{\ell})$ tends to $
  \widetilde{\varphi}$ in the space $C^{j}(\overline{\mathscr{O}},
  \widetilde{Z})$. Then, the sequence $(b_{\ell}^{j})_{\ell \geq 0}$
  tends to $b^{j}$ in the space $C^{0}(\overline{\mathscr{O}},
  \mathscr{L}(\mathbb{R}^{n+1}, \mathscr{L}_{j} (\mathbb{R}^{n+1},
  \widetilde{Z})))$ and the sequence $(B_{i\ell}^{j})_{\ell > 0}$
  tends to $B_{i}^{j}$ in the space $C^{0}(\overline{\mathscr{O}},
  \mathscr{L}(\mathscr{L}_{i}(\mathbb{R}^{n+1},  \widetilde{Z}),
  \mathscr{L}_{j} (\mathbb{R}^{n+1},  \widetilde{Z})))$.
  \end{alphlemma}

\begin{proof}
First, as the mapping $\Phi_{j}$ is of class $C^{k-j}$, the mappings
$b^{j}, b_{\ell}^{j}, B_{i\ell}^{j}$ and $B_{i}^{j}$ are of class
$C^{1}$ in the case (i) and of class $C^{0}$ in the case
(ii). Throughout this proof and for the sake of conveience, we shall
set for $ \widetilde{x} \epsilon \overline{\mathscr{O}}$
\begin{equation*}
V^{j} ( \widetilde{x}) = ( \widetilde{\varphi}( \widetilde{x}), D
\widetilde{\varphi}( \widetilde{x}), \cdots , D^{j} \widetilde{\varphi}( \widetilde{x}))\tag{A2.40}\label{app-2-eqA2.40}
\end{equation*}
and
\begin{equation*}
V_{\ell}^{j}( \widetilde{x}) = ( \widetilde{\varphi}_{\ell}(
\widetilde{x}), D \widetilde{\varphi}_{\ell}( \widetilde{x}), \cdots ,
D^{j} \widetilde{\varphi}_{\ell}( \widetilde{x})).\tag{A2.41}\label{app-2-eqA2.41}
\end{equation*}

This allows to write the relations (\ref{app-2-eqA2.36}) - (\ref{app-2-eqA2.39}) in the form
\begin{align*}
b^{j}( \widetilde{x}) & = \frac{\partial \varphi_{j}}{\partial
  \widetilde{x}} (\widetilde{x}, V^{j} (
\widetilde{x})),\tag{A2.42}\label{app-2-eqA2.42}\\
b_{\ell}^{j} (\widetilde{x}) & = \frac{\partial \Phi_{j}}{\partial
  \widetilde{x}} (\widetilde{x}, V_{\ell}^{j}
(\widetilde{x})),\tag{A2.43}\label{app-2-eqA2.43}\\
B_{i}^{j} (\widetilde{x}) & = \frac{\partial \Phi_{j}}{\partial
  \lambda_{i}} (\widetilde{x}, V^{j} (\widetilde{x})),\tag{A2.44}\label{app-2-eqA2.44}\\
B_{i\ell}^{j} (\widetilde{x}) & = \frac{\partial \Phi_{j}}{\partial
  \lambda_{i}} (\widetilde{x}, V-{\ell}^{j} (\widetilde{x})).\tag{A2.45}\label{app-2-eqA2.45}
\end{align*}\pageoriginale

Before proving (i), note that, for $\widetilde{x} \epsilon
\overline{\mathscr{O}}$ and $0 \leq s \leq 1$, the combinations
$$
V^{j}(\widetilde{x}) + s(V_{\ell}^{j}(\widetilde{x}) - V^{j}(\widetilde{x})),
$$
belong to the product $\Lambda_{0}^{c} \times \cdots \times
\Lambda_{j}^{c}$ of the closed convex hulls \footnote{Recall that the
  closed convex hull of a set is defined as the closure of its convex
  hull.} $\Lambda_{i}^{c}$ of the compact sets $\Lambda_{i}$
(cf. Lemma \ref{app-2-lemA2.4}). As the closed convex hull of a compact set is by a
classical result from topology, the product $\Lambda_{0}^{c} \times
\cdots \times \Lambda_{j}^{c}$ is compact. On the other hand, sinced
the mappings $\widetilde{\varphi}$ and $\widetilde{\varphi}_{\ell}$
for $\ell \geq 0$ take their values in the ball $\overline{B}(0,
\rho)$, we deduce that $\Lambda_{0}$ and hence $\Lambda_{0}^{c}$ is
contained in $\overline{B}(0, \rho)$. Therefore, for every $0 \leq i
\leq j \leq k-2$, an expression such as
$$ 
\frac{\partial^{2} \Phi_{j}}{\partial \widetilde{x} \partial
  \lambda_{i}} (\widetilde{x}, V^{j} (\widetilde{x}) +
s(V_{\ell}^{j}(\widetilde{x}) - V^{j}(\widetilde{x}))),
$$
is well defined for $\widetilde{x} \epsilon \overline{\mathscr{O}}$
and $0 \leq s \leq 1$. For $\widetilde{x} \epsilon
\overline{\mathscr{O}}$ and due to the Taylor formula (which can be
used in view of the above observation) we get
{\fontsize{10}{12}\selectfont
$$
b_{\ell}^{j}(\widetilde{x}) - b^{j}(\widetilde{x}) =
\sum\limits_{i=0}^{j} \int_{0}^{1} \frac{\partial^{2}
  \Phi_{j}}{\partial \widetilde{x} \partial \lambda_{j}}
(\widetilde{x}, V^{j} (\widetilde{x}) + s(V_{\ell}^{j}
(\widetilde{x})- V^{j}(\widetilde{x}))) \cdot
(D^{i}\widetilde{\varphi}_{\ell} (\widetilde{x}) -
D^{i}\widetilde{\varphi}(\widetilde{x})) ds.
$$}

From the compacteness of the set $\overline{\mathscr{O}} \times
\Lambda_{0}^{c} \times \cdots \times \Lambda_{j}^{c}$, the quantities
$||\frac{\partial^{2} \Phi_{j}}{\partial \widetilde{x} \partial
  \lambda_{i}} (\widetilde{x}, \lambda_{0}, \cdots,
\lambda_{j})||$\pageoriginale are bounded by a constant $K > 0$ for
$(\widetilde{x}, \lambda_{0}, \ldots,\break \lambda_{j}) \epsilon
\overline{\mathscr{O}} \times \Lambda_{0}^{c} \times \cdots \times
\Lambda_{j}^{c}$ and $0 \leq i \leq j$. Thus
$$
||b_{\ell}^{j}(\widetilde{x}) - b^{j}(\widetilde{x})|| \leq K
\sum\limits_{i=0}^{j} ||D^{i}\widetilde{\varphi}_{\ell}(\widetilde{x})
- D^{i}\widetilde{\varphi}(\widetilde{x})||.
$$

As the sequence $(D^{i} \varphi_{\ell})_{\ell \geq 0}$ tends to
$D^{i}\widetilde{\varphi}$ geometrically in the space
$C^{0}(\overline{\mathscr{O}}, \mathscr{L}_{i} (\mathbb{R}^{n+1},
\widetilde{Z}))$ for every $0 \leq i \leq j$ by hypothesis, there
exist q with $0 < q < 1$ and a constant $C > 0$  such that
$$
||D^{i}\widetilde{\varphi}_{\ell} (\widetilde{x}) -
D^{i}\widetilde{\varphi}(\widetilde{x})|| \leq Cq^{\ell},
$$
for every $0 \leq i \leq j$ and every $\widetilde{x} \epsilon
\overline{\mathscr{O}}$. Hence
$$
||b_{\ell}^{j}(\widetilde{x}) - b^{j}(\widetilde{x})|| \leq (j+1)KCq^{\ell},
$$
for every $\widetilde{x} \epsilon \overline{\mathscr{O}}$ and then
$$
\sup_{\widetilde{x} \epsilon \mathscr{O}}
||b_{\ell}^{j}(\widetilde{x}) - b^{j}(\widetilde{x})|| \leq (j+1) KCq^{\ell}.
$$

The above inequality proves that the sequence $(b_{\ell}^{j})_{\ell
  \geq 0}$ tends to $b^{j}$ geometrically in the space
$C^{0}(\overline{\mathscr{O}}, \mathscr{L}(\mathbb{R}^{n+1},
\mathscr{L}_{i}(\mathbb{R}^{n+1}, \widetilde{B}))$. Similar arguments
can be used for showing that the sequence $(B_{i \ell}^{j})_{\ell \geq
0}$ tends to $B_{i}^{j}$ geometrically in the space
$C^{0}(\overline{\mathscr{O},
  \mathscr{L}}(\mathscr{L}_{i}(\mathbb{R}^{n+1}, \widetilde{Z}),
\mathscr{L}_{j}(\mathbb{R}^{n+1}, \widetilde{Z}))), 0 \leq i \leq j$.

Now, we prove part (ii) of the statement. Here, the Taylor formula is
not available because of the lack of regularity for $j =
k-1$. Actually, as we no longer interested in the rate of convergence,
the resulta can be easily deducaed from the {\em uniform continuity}
of the continuous mapping $(\partial \Phi_{j} / \partial
\widetilde{x})$ and $(\partial \Phi_{j} / \partial \lambda_{i}), 0
\leq i \leq j$ on the compact set $\overline{\mathscr{O}} \times
\Lambda_{0} \times \cdots \times \Lambda_{j}$ together with the {\em
  uniform convergence} (i.e.\pageoriginale in the space
$C^{0}(\overline{\mathscr{O}}, \mathscr{L}_{i}(\mathbb{R}^{n+1},
\widetilde{Z}))$ of the sequence
$(D^{i}\widetilde{\varphi}_{\ell})_{\ell \geq 0}$ to the mapping
$D^{i}\widetilde{\varphi}$ for $0 \leq i \leq j$ from which the
uniform convergence of the sequence $(V_{\ell}^{j})_{\ell \geq 0}$ to
the mapping $V^{j}$ is derived.)
\end{proof}

$\S \qquad$ We are now in a position to prove an important part of the
results announced before. 
 
\begin{alphtheorem}\label{app-2-thmA2.2}%[A2.2]
After shrinking the meighbourhood $\mathscr{O}$ is necessary, the
sequence $(\widetilde{\varphi}_{\ell})$ tends to the mapping
$\widetilde{\varphi}$ geometrically in the space
$C^{k-1}(\overline{\mathscr{O}}, \widetilde{Z})$.
\end{alphtheorem}

\begin{proof}
Equivalently, we have to prove for $0 \leq j \leq k-1$ the sequence
$(D^{j}\widetilde{\varphi}_{\ell})_{\ell \geq 0}$ tends to
$D^{j}\widetilde{\varphi}$ geometrically in the space
$C^{0}(\overline{\mathscr{O}}, \mathscr{L}_{j}(\mathbb{R}^{n+1},
\widetilde{Z}))$. This result has already been proved for $j = 0$ in
Lemma \ref{app-2-lemA2.2} and we can thus assume $j \geq 1$ and $k \geq 2$. We
proceed by induction and therefore assume that the sequemce
$(D^{i}\widetilde{\varphi}_{\ell})_{\ell \geq 0}$ tends to
$D^{i}\widetilde{\varphi}$ geometrically in the space
$C^{0}(\overline{\mathscr{O}}, \mathscr{L}_{i}(\mathbb{R}^{n+1},
\widetilde{Z}))$ for $0 \leq i \leq j-1$.

Let $\ell \geq 0$ be fixed. From (\ref{app-2-eqA2.29}) and (\ref{app-2-eqA2.30}), one has for
$\widetilde{x} \epsilon \overline{\mathscr{O}}$
\begin{align*}
D^{j}\widetilde{\varphi}_{\ell + 1}(\widetilde{x}) -
D^{j}\widetilde{\varphi}(\widetilde{x}) &= \Phi_{j}(\widetilde{x},
\widetilde{\varphi}_{\ell} (\widetilde{x}),
D\widetilde{\varphi}_{\ell} (\widetilde{x}), \ldots,
D^{j}\widetilde{\varphi}_{\ell} (\widetilde{x}))\\ 
&- \Phi_{j}
(\widetilde{x}, \widetilde{\varphi}(\widetilde{x}),
D\widetilde{\varphi}(\widetilde{x}), \ldots,
D^{j}\widetilde{\varphi}(\widetilde{x})). 
\end{align*}

By definition of the mapping $\Phi_{j}$ (cf. (\ref{app-2-eqA2.48})) and in the
notations (\ref{app-2-eqA2.42}) - (\ref{app-2-eqA2.45}), this identity can be rewritten as
\begin{align*}
D^{j}\widetilde{\varphi}_{\ell + 1} (\widetilde{x}) -
D^{j}\widetilde{\varphi}(\widetilde{x}) &= b_{\ell}^{j-1}
(\widetilde{x}) - b^{j-1} (\widetilde{x})\\ 
&+ \sum\limits_{i=0}^{j-1}
\left[B_{i \ell}^{j-1} (\widetilde{x})
  D^{i+1}\widetilde{\varphi}_{\ell} (\widetilde{x}) -
  B_{i}^{j-1}(\widetilde{x}) D^{i+1}\widetilde{\varphi}(\widetilde{x})\right]. 
\end{align*}

Recall in this formula that the element
$D^{i+1}\widetilde{\varphi}_{\ell}(\widetilde{x})$ (resp. $D^{i+1}
\widetilde{\varphi}(\widetilde{x})$) is considered as an element of
$\mathscr{L}(\mathbb{R}^{n+1}, \mathscr{L}_{i}(\mathbb{R}^{n+1},
\widetilde{Z}))$ and the operation $B_{i\ell}^{j-1}(\widetilde{x})
D^{i+1}\widetilde{\varphi}_{\ell}(\widetilde{x})$\pageoriginale
(resp. $B_{i}^{j-1} (\widetilde{x})
D^{i+1}\widetilde{\varphi}(\widetilde{x})$) denotes composition of
{\em linear} mappings. Thus, we can as well write
\begin{align*}
D^{j}\widetilde{\varphi}_{\ell + 1}(\widetilde{x}) -
D^{j}\widetilde{\varphi}(\widetilde{x}) &= b_{\ell}^{j-1}
(\widetilde{x}) - b^{j-1}(\widetilde{x})\\ 
&+ \sum\limits_{i=0}^{j-1}
(B_{i\ell}^{j-1} (\widetilde{x}) - B_{i}^{j-1}(\widetilde{x}))
D^{i+1}\widetilde{\varphi}(\widetilde{x})\\
& + \sum\limits_{i=0}^{j-1} B_{i\ell}^{j-1}  
(\widetilde{x}) (D^{i+1} \widetilde{\varphi}_{\ell}(\widetilde{x}) -
D^{i+1}\widetilde{\varphi}(\widetilde{x})). 
\end{align*}

We deduce the inequality
\begin{align*}
||D^{j}\widetilde{\varphi}_{\ell + 1}(\widetilde{x}) -
D^{j}\widetilde{\varphi}(\widetilde{x})|| &\leq
||b_{\ell}^{j-1}(\widetilde{x}) - b^{j-1}(\widetilde{x})||\\ 
&+\sum\limits_{i=0}^{j-1} ||B_{i \ell}^{j-1}(\widetilde{x}) -
B_{i}^{j-1} (\widetilde{x})||
||D^{i+1}\widetilde{\varphi}(\widetilde{x})|| \tag{A2.46} \label{app-2-eqA2.46}\\
& + \sum\limits_{i=0}^{j-1} ||B_{i\ell}^{j-1} (\widetilde{x})||
||D^{i+1} \widetilde{\varphi}_{\ell} (\widetilde{x}) - D^{i+1}\widetilde{\varphi}(\widetilde{x})||.
\end{align*}

It is essential to pay special attention to the term corresponding
with $i = j-1$ in the last sum, namely $||B_{j-1, \ell}^{j-1}
(\widetilde{x})|| ||D^{j}\widetilde{\varphi}_{\ell}(\widetilde{x}) -
D^{j}\widetilde{\varphi}(\widetilde{x})||$. From (\ref{app-2-eqA2.39})
$$
B_{j-1,\ell}^{j-1} (\widetilde{x}) = \frac{\partial
  \Phi_{j-1}}{\partial \lambda_{j-1}} (\widetilde{x},
\widetilde{\varphi}_{\ell} (\widetilde{x}),
D\widetilde{\varphi}_{\ell}(\widetilde{x}), \cdots,
D^{j-1}\widetilde{\varphi}_{\ell}(\widetilde{x})). 
$$

As the sequence $(\widetilde{\varphi}_{\ell})$ tends to
$\widetilde{\varphi}$ in the space $C^{j-1}(\overline{\mathscr{O}},
\widetilde{Z})$ by hypothesis, we can apply Lemma \ref{app-2-lemA2.5} with $j-1$
(which amount to shrinking the neighbourhood $\overline{\mathscr{O}}$
if necessary) and hence
$$
||B_{j-1, \ell}^{j-1} (\widetilde{x})|| \leq \beta,
$$
for every $\widetilde{x} \epsilon \overline{\mathscr{O}}$. Inequality
(\ref{app-2-eqA2.46}) becomes
\begin{align*}
||D^{j}\widetilde{\varphi}_{\ell + 1}(\widetilde{x}) -
D^{j}\widetilde{\varphi}(\widetilde{x})|| &<
||b_{\ell}^{j-1}(\widetilde{x}) - b^{j-1}(\widetilde{x})||\\ 
&{} +\sum\limits_{i=0}^{j-1} ||B_{i\ell}^{j-1}(\widetilde{x}) -
B_{i}^{j-1}(\widetilde{x}) - B_{i}^{j-1}(\widetilde{x})||
||D^{i+1}\widetilde{\varphi}|| \tag{A2.47} \label{app-2-eqA2.47}\\
&{} + \sum\limits_{i=0}^{j-2} ||B_{i\ell}^{j-1} (\widetilde{x})||
||D^{i+1}\widetilde{\varphi}_{\ell} (\widetilde{x}) -
D^{i+1}\widetilde{\varphi} \widetilde{x} || + \\
&{} + \beta ||D^{j}\widetilde{\varphi}_{\ell} - D^{j}\widetilde{\varphi}(\widetilde{x})||.
\end{align*}

Now,\pageoriginale since $j \leq k - 1$, we can apply Lemma \ref{app-2-lemA2.6} (i)
with $j-1$: There are constant $q$ with $0 < q < 1$ and $C > 0$ such
that
\begin{align*}
& ||b_{\ell}^{j-1}(\widetilde{x}) - b^{j-1}(\widetilde{x})|| \leq
Cq^{\ell},\\
& ||B_{i\ell}^{j-1}(\widetilde{x}) - B_{i}^{j-1}(\widetilde{x})|| \leq
Cq^{\ell}, 0 \leq i \leq j-1,
\end{align*}
for every $\ell \geq 0$ and every $\widetilde{x} \epsilon
\overline{\mathscr{O}}$. As $\widetilde{\varphi} \epsilon
C^{k}(\overline{\mathscr{O}}, \widetilde{Z}) \subset
C^{j}(\overline{\mathscr{O}}, \widetilde{Z})$, there is a constant $K
> 0$ such that
$$
||D^{i+1} \widetilde{\varphi}(\widetilde{x})|| \leq K,
$$
for every $\widetilde{x} \epsilon \overline{\mathscr{O}}$ and $0 \leq
i \leq j-1$. Together with (\ref{app-2-eqA2.47}), these observations give
\begin{align*}
||D^{j}\widetilde{\varphi}_{\ell + 1}(\widetilde{x}) -
D^{j}\widetilde{\varphi}(\widetilde{x})|| &< (jK + 1)Cq^{\ell}\\  
&+\sum\limits_{i=0}^{j-2} ||B_{i\ell}^{j-1}(\widetilde{x})||
||D^{i+1}\widetilde{\varphi}_{\ell}(\widetilde{x}) -
D^{i+1}\widetilde{\varphi}(\widetilde{x})|| \tag{A2.48}\label{app-2-eqA2.48}\\
& + \beta ||D^{j}\widetilde{\varphi}_{\ell}(\widetilde{x}) - D^{j}\widetilde{\varphi}(\widetilde{x})||.
\end{align*}

From the convergence of the sequence $(\widetilde{\varphi}_{\ell})$ to
$\widetilde{\varphi}$ geometrically in the space
$C^{j-1}(\overline{\mathscr{O}}, \widetilde{Z})$, it is not
restrictive to assume that $q$ with $0 < q < 1$ and $C > 0$ are such
that
$$
||D^{i+1}\widetilde{\varphi}_{\ell}(\widetilde{x}) -
D^{i+1}\widetilde{\varphi}(\widetilde{x})|| \leq Cq^{\ell},
$$
for every $\widetilde{x} \epsilon \overline{\mathscr{O}}, 0 \leq i
\leq j-2$. Besides, we can suppose the constant $K$ large enough for the
inequality
$$
||B_{i\ell}^{j-1}(\widetilde{x})|| \leq K,
$$
to hold for every $\widetilde{x} \epsilon \overline{\mathscr{O}}, 0
\leq i \leq j-2$ and $\ell \geq 0$. Indeed, this follows from the
definition (\ref{app-2-eqA2.39}), namely
$$
B_{i\ell}^{j-1}(\widetilde{x}) = \frac{\partial \Phi_{j-1}}{\partial
  \lambda_{i}} (\widetilde{x},
\widetilde{\varphi}_{\ell}(\widetilde{x}),
D\widetilde{\varphi}_{\ell}(\widetilde{x}), \cdots,
D^{j-1}\widetilde{\varphi}_{\ell}(\widetilde{x})) 
$$
and\pageoriginale the compactness of the set $\overline{\mathscr{O}}
\times \Lambda_{0} \times \cdots \times \Lambda_{j-1}$ (Lemma \ref{app-2-lemA2.4}
with $j-1$). Inequality (\ref{app-2-eqA2.48}) then tales the simpler form 
$$
||D^{j}\widetilde{\varphi}_{\ell + 1}(\widetilde{x}) -
D^{j}\widetilde{\varphi}(\widetilde{x})|| \leq [(2j-1)K+1]Cq^{\ell} +
\beta ||D^{j}\widetilde{\varphi}_{\ell}(\widetilde{x}) -
D^{j}\widetilde{\varphi}(\widetilde{x})||. 
$$

Replacing $q$ by $\max(q, \beta) < 1$ and setting
$$
K_{0} = \frac{1}{q}[(2j-1)K + 1]C,
$$
the above inequality becomes
$$
||D^{j}\widetilde{\varphi}_{\ell + 1}(\widetilde{x}) -
D^{j}\widetilde{\varphi}(\widetilde{x})|| < K_{0}q^{\ell + 1} + q
||D^{j}\widetilde{\varphi}_{\ell}(\widetilde{x}) -
D^{j}\widetilde{\varphi}(\widetilde{x})|| 
$$
and yields
\begin{equation*}
|D^{j}(\widetilde{\varphi}_{\ell + 1} - \widetilde{\varphi})|_{\infty,
\overline{\mathscr{O}}} \leq K_{0}q^{\ell + 1} +
q|D^{j}(\widetilde{\varphi}_{\ell}-\widetilde{\varphi})|_{\infty,
  \overline{\mathscr{O}}}.\tag{A2.49}\label{app-2-eqA2.49} 
\end{equation*}

Let us define the sequence $(a_{\ell})_{\ell \geq 0}$ by
\begin{equation*}
\begin{cases}
a_{0} & = |D^{j}\widetilde{\varphi}|_{\infty,
  \overline{\mathscr{O}}},\\
a_{\ell + 1} & = K_{0}q^{\ell+1} + qa_{\ell}, \ell \geq 0
\end{cases}
\end{equation*}

It is immediately checked that
\begin{equation*}
|D^{j}(\widetilde{\varphi}_{\ell} - \widetilde{\varphi})|_{\infty,
  \overline{\mathscr{O}}} \leq a_{\ell},\tag{A2.50}\label{app-2-eqA2.50}
\end{equation*}
for every $\ell \geq 0$ (recall that $\widetilde{\varphi}_{0} =
0$). On the other hand, the sequence $(a_{\ell})$ is equivalently
defined by 
$$
a_{\ell} = (K_{0}\ell + a_{0})q^{\ell}, \ell \geq 0.
$$

Choosing $q < q' < 1$, this is the same as
$$
a_{\ell} = (K_{0}\ell + a_{0}) \left(\frac{q}{q'}\right) (q')^{\ell},
\ell \geq 0.
$$

As $0 < q/q' < 1$, the term $(K_{0}\ell + a_{0})(q/q')^{\ell}$ tends
to 0 and hence is uniformly\pageoriginale bounded by a constant $K_{1}
> 0$. Thus, 
$$
a_{\ell} \leq K_{1}(q')^{\ell}, \ell \geq 0
$$
and our assertion follows from (\ref{app-2-eqA2.50}).
\end{proof}

Replacing $k$ by $k + 1$ in Theorem \ref{app-2-thmA2.2}, we get

\begin{alphcorollary}\label{app-2-coroA2.1}%[A2.1]
Assume that the mapping $\Phi$ is of class $C^{k+1}$. Then, after
shrinking the neighbourhood $\mathscr{O}$ is necessary, the sequence
$(\widetilde{\varphi}_{\ell})$ tends to the mapping
$\widetilde{\varphi}$ geometrically in the space
$C^{k}(\overline{\mathscr{O}}, \widetilde{Z})$.
\end{alphcorollary}

Oue nect theorem shows that even when the mapping $\Phi$ is only of
class $C^{k}$, the sequence $(\widetilde{\varphi}_{\ell})$ tends to
$\widetilde{\varphi}$ in the space $C^{k}(\overline{\mathscr{O}},
\widetilde{Z})$ but the rate of convergence is not known in general
(compare with Corollary \ref{app-2-coroA2.1} above).

\begin{alphtheorem}\label{app-2-thmA2.3}%[A2.3]
After shrinking the neighbourhood $\mathscr{O}$ if necessary, the
sequence $(\widetilde{\varphi}_{\ell})$ tends to the mapping
$\widetilde{\varphi}$ in the space $C^{k}(\overline{\mathscr{O}},
\widetilde{Z})$. 
\end{alphtheorem}

\begin{proof}
From Theorem \ref{app-2-thmA2.2}, we know that the sequence
$(\widetilde{\varphi}_{\ell})$ tends to $\widetilde{\varphi}$ in the
space $C^{k-1}(\overline{\mathscr{O}}, \widetilde{Z})$ and it remains
to prove the convergence of the  sequence
$(D^{k}\widetilde{\varphi}_{\ell})_{\ell \geq 0}$ to
$D^{k}\widetilde{\varphi}$ in the space $C^{0}(\overline{\mathscr{O}},
\mathscr{L}_{k}(\mathbb{R}^{n+1}, \widetilde{Z}))$. To do this, we can
repeat a part of the proof of Theorem \ref{app-2-thmA2.2}. Replacing $j$ by $k$, the
observations we made up to and including formula (\ref{app-2-eqA2.47}) are still
valid and we then find
\begin{align*}
||D^{k}\widetilde{\varphi}_{\ell + 1}(\widetilde{x})  -
D^{k}\widetilde{\varphi}(\widetilde{x})|| &\leq
||b_{\ell}^{k-1}(\widetilde{x}) - b^{k-1}(\widetilde{x})||\\ 
&{}+\sum\limits_{i=0}^{k-1} ||B_{i\ell}^{k-1} (\widetilde{x}) -
B_{i}^{k-1}(\widetilde{x})|| ||D^{i+1}
\widetilde{\varphi}(\widetilde{x})|| \tag{A2.51}\label{app-2-eqA2.51}\\
&{} + \sum\limits_{i=0}^{k-2} ||B_{i\ell}^{k-1}(\widetilde{x})||
||D^{i+1}\widetilde{\varphi}_{\ell} (\widetilde{x}) -
D^{i+1}\widetilde{\varphi}(\widetilde{x})||\\ 
&{}+ \beta||D^{k}\widetilde{\varphi}_{\ell} (\widetilde{x}) -
D^{k}\widetilde{\varphi}(\widetilde{x})||, 
\end{align*}
for\pageoriginale every $\widetilde{x} \in
\overline{\mathscr{O}}$ and every $\ell \geq 0$. Here, we can no
longer use Lemma \ref{app-2-lemA2.6} (i) for finding an estimate of the terms
$||b_{\ell}^{k-1} (\widetilde{x}) - b^{k-1}(\widetilde{x})||$ or
$||B_{i\ell}^{k-1} (\widetilde{x}) - B_{i}^{k-1}(\widetilde{x})||$ but
Lemma \ref{app-2-lemA2.6} (ii) applies with $j = k-1$: Given $\epsilon > 0$ and
$\ell$ large enough, say $\ell \geq \ell_{0}$, one has
\begin{align*}
& ||b_{\ell}^{k-1}(\widetilde{x}) - b^{k-1}(\widetilde{x})|| \leq
\epsilon,\\
& ||B_{i\ell}^{k-1} (\widetilde{x}) - B^{k-1}(\widetilde{x})|| \leq
\epsilon, 0 \leq i \leq k-1,
\end{align*}
for every $\widetilde{x} \in \overline{\mathscr{O}}$. By the same
arguments as in Theorem \ref{app-2-thmA2.2} and from the regularity
$\widetilde{\varphi} \in C^{k}(\overline{\mathscr{O}},
\widetilde{Z})$, we get the existence of a constant $K > 0$ such that
$$
||D^{i+1}\widetilde{\varphi}(\widetilde{x})|| \leq K,
$$
for every $\widetilde{x} \in \overline{\mathscr{O}} \leq i \leq
k-1$ and 
$$
||B_{i\ell}^{k-1} (\widetilde{x})|| \leq K,
$$
for every $\widetilde{x} \in \overline{\mathscr{O}}, 0 \leq i
\leq k-2$ and $\ell \geq 0$ and $\ell_{0}$ can be taken large enough
for the inequality
$$
||D^{i+1}\widetilde{\varphi}_{\ell}(\widetilde{x}) -
D^{i+1}\widetilde{\varphi}(\widetilde{x})|| \leq \epsilon
$$
to hold for every $\widetilde{x} \in \overline{\mathscr{O}}, 0
\leq i \leq k-2$. Together with (\ref{app-2-eqA2.51}), we arrive at
$$
||D^{k}\widetilde{\varphi}_{\ell + 1}(\widetilde{x}) -
D^{k}\widetilde{\varphi}(\widetilde{x})|| \leq [(2k-1)K+1] \epsilon +
\beta ||D^{k}\widetilde{\varphi}_{\ell} (\widetilde{x}) -
D^{k}\widetilde{\varphi}(\widetilde{x})||. 
$$

Setting 
$$
K_{0} = (2k-1)K + 1,
$$
the above inequality becomes
$$
||D^{k}\widetilde{\varphi}_{\ell + 1}(\widetilde{x}) -
D^{k}\widetilde{\varphi}(\widetilde{x})|| \leq K_{0} \epsilon
\beta||D^{k}\widetilde{\varphi}_{\ell} (\widetilde{x}) -
D^{k}\widetilde{\varphi}(\widetilde{x})|| 
$$
and\pageoriginale yields
$$
|D^{k} (\widetilde{\varphi}_{\ell + 1} -
\widetilde{\varphi})|_{\infty, \overline{\mathscr{O}}} < K_{0}
\epsilon + \beta |D^{k} (\widetilde{\varphi}_{\ell} -
\widetilde{\varphi})|_{\infty, \overline{\mathscr{O}}}.
$$

Let us define the sequence $(a_{\ell})_{\ell \geq \ell_{0}}$ by
\begin{equation*}
\begin{cases}
& a_{\ell_{0}} = |D^{k}(\widetilde{\varphi}_{\ell_{0}} -
\widetilde{\varphi})|_{\infty, \overline{\mathscr{O}}},\\
& a_{\ell + 1} = K_{0}\epsilon + \beta a_{\ell}, \ell \geq \ell_{0}.\\
\end{cases}
\end{equation*}

It is immediately checked that
\begin{equation*}
|D^{k} (\widetilde{\varphi}_{\ell} - \widetilde{\varphi})|_{\infty,
  \overline{\mathscr{O}}} \leq a_{\ell},\tag{A2.52}\label{app-2-eqA2.52}
\end{equation*}
for every $\ell \geq \ell_{0}$. On the other hand, the sequence
$(a_{\ell})_{\ell \geq \ell_{0}}$ is equivalently defined by
$$
a_{\ell} = K_{0} \frac{1 - \beta^{\ell - \ell_{0}}}{1 - \beta}
\epsilon + \beta^{\ell - \ell_{0}} a_{\ell_{0}}, \ell \geq \ell_{0} + 1.
$$

For $\ell \geq \ell_{0}$ large enough, say $\ell \geq \ell_{1}$, one
has $\beta^{\ell - \ell_{0}} a_{\ell_{0}} \leq \epsilon$.

Thus
$$
a_{\ell} \leq \left[\frac{K_{0}}{1 - \beta} + 1\right] \epsilon,
$$
for $\ell \geq \ell_{1}$ and our assertion follows from (\ref{app-2-eqA2.52}).
\end{proof}

Here is now a result of a different kind that we shall use later on.

\begin{alphlemma}\label{app-2-lemA2.7}%[A2.7]
For every $\ell \geq 0$, one has
\begin{equation*}
D^{i} \widetilde{\varphi}_{\ell + k} (0) =
D^{i}\widetilde{\varphi}(0),\tag{A2.53} \label{app-2-eqA2.53}
\end{equation*}
for $0 \leq i \leq k$.
\end{alphlemma}

\begin{proof}
Again,\pageoriginale we shall proceed by induction and show for $0
\leq j \leq k$ and every $\ell \geq 0$
\begin{equation*}
D^{i}\widetilde{\varphi}_{\ell + j} (0) = D^{i}\widetilde{\varphi}(0),\tag{A2.54}\label{app-2-eqA2.54}
\end{equation*}
for $0 \leq i \leq j$. This result is true for $j = 0$ since
$\widetilde{\varphi}(0) = 0$ and $\widetilde{\varphi}_{\ell}(0) = 0$
as it follows from the definition. Assume then that $j \geq 1$ and the
result is true up to ranke $j - 1$, namely, that
\begin{equation*}
D^{i}\widetilde{\varphi}_{\ell + j-1} (0) =
D^{i}\widetilde{\varphi}(0),\tag{A2.55} \label{app-2-eqA2.55}
\end{equation*}
for $0 \leq i \leq j-1$. From the relations (\ref{app-2-eqA2.29}) and (\ref{app-2-eqA2.30})
\begin{align*}
& D^{i}\widetilde{\varphi}_{\ell + j}(0) = \Phi_{i} (0,
\widetilde{\varphi}_{\ell + j-1} (0), \cdots,
D^{i}\widetilde{\varphi}_{\ell + j - 1}(0)),\tag{A2.56}\label{app-2-eqA2.56}\\
& D^{i}\widetilde{\varphi}(0) = \Phi_{i} (0, \widetilde{\varphi}(0),
\cdots , D^{i}\widetilde{\varphi}(0)),\tag{A2.57}\label{app-2-eqA2.57}
\end{align*}
for $0 \leq i \leq k$ and hence for $0 \leq i \leq j$. Clearly, from
(\ref{app-2-eqA2.55}), the relation (\ref{app-2-eqA2.54}) holds for $0 \leq i \leq j-1$ but we must show that it also holds for $i = j$.

The fact that $\widetilde{\varphi}(0) = \widetilde{\varphi}_{\ell + j
  - 1}(0) = 0$ is essentail here because taking $i = j$ in (\ref{app-2-eqA2.56}) and
(\ref{app-2-eqA2.57}) and coming back to the definition (cf. (\ref{app-2-eqA2.28})), we get
\begin{align*}
& D^{j}\widetilde{\varphi}_{\ell + j}(0) = \frac{\partial
    \Phi_{j-1}}{\partial \widetilde{x}} (0, 0,
  D\widetilde{\varphi}_{\ell + j - 1}(0), \cdots,
  D^{j-1}\widetilde{\varphi}_{\ell + j - }(0)) + \tag{A2.58}\label{app-2-eqA2.58}\\
& + \sum\limits_{i=0}^{j-1} \frac{\partial \Phi_{j-1}}{\partial
    \lambda_{i}} (0, 0, D \widetilde{\varphi}_{\ell + j - 1} (0),
  \cdots, D^{j-1} \widetilde{\varphi}_{\ell + j - 1} (0))
  D^{i+1}\widetilde{\varphi}_{\ell + j - 1}(0)
\end{align*}
and
\begin{align*}
& D^{j}\widetilde{\varphi}(0) = \frac{\partial \Phi_{j-1}}{\partial
    \widetilde{x}} (0, 0, D\widetilde{\varphi}(0), \cdots,
  D^{j-1}\widetilde{\varphi}(0)) + \tag{A2.59}\label{app-2-eqA2.59}\\
& + \sum\limits_{i = 0}^{j-1} \frac{\partial \Phi_{j-1}}{\partial
    \lambda_{i}} (0, 0, D\widetilde{\varphi}(0), \cdots,
  D^{j-1}\widetilde{\varphi}(0)) D^{i+1}\widetilde{\varphi}(0).
\end{align*}

But,\pageoriginale in view of Lemma \ref{app-2-lemA2.3}, the term corresponding with
$i = j - 1$ vanishes in both expressions (\ref{app-2-eqA2.58}) and (\ref{app-2-eqA2.59}). It
follows that $D^{j}\widetilde{\varphi}_{\ell + j}(0)$ and
$D^{j}\widetilde{\varphi}(0)$ are given by the same formula involving
the derivatives of order $\leq j - 1$ of the mappings
$\widetilde{\varphi}_{\ell + j - 1}$ and $\widetilde{\varphi}$
respectively. As these derivatives coincide by the hypothesis of
induction, relation (\ref{app-2-eqA2.54}) at rank $j$ is established and the proof is complete.
\end{proof}

let us now consider the mapping $F \epsilon C^{k}
(\overline{\mathscr{O}} \times \overline{B}(0, \rho), \mathbb{R}^{n})$
introduced in $\S 6$ of Chapter \ref{chap4} verifying $F(0) = 0$. So
as to fully prove the results we used in $\S 6$ of Chapter
\ref{chap4}, we still have to show that the sequence of mapping
$$
\widetilde{x} \epsilon \overline{\mathscr{O}} \to
f_{\ell}(\widetilde{x}) = F(\widetilde{x}, \widetilde{\varphi}_{\ell +
k}(\widetilde{x})),
$$
of class $C^{k}$ from $\overline{\mathscr{O}}$ to $\mathbb{R}^{n}$
tends to the mapping
$$
\widetilde{x} \epsilon \overline{\mathscr{O}} \to f(\widetilde{x}) =
F(\widetilde{x}, \widetilde{\varphi}(\widetilde{x})),
$$
in the space $C^{k}(\overline{\mathscr{O}}, \mathbb{R}^{n})$, the
convergence being geometrical in the space
$C^{k-1}(\overline{\mathscr{O}}, \mathbb{R}^{n})$ (and also
geometrical in the space $C^{k}(\overline{\mathscr{O}},
\mathbb{R}^{n})$ if both mappings $\Phi$ and $F$ are of class
$C^{k+1}$). Besides, we used the relation
$$
D^{j}f_{\ell}(0) = D^{j}f(0), 0 \leq j \leq k,
$$
which is now immediate from Lemma \ref{app-2-lemA2.7} because, for a given index $0
\leq j \leq k$, the derivatives of order $j$ of the mappings $f_{\ell}$
and $f$ are given by the same formula involving the derivatives of order
$\leq j$ of the mappings $F$ and $\widetilde{\varphi}_{\ell + k}$ on the
one hand those of $F$ and $\widetilde{\varphi}$ on the other hand.

Still\pageoriginale denoting by $\lambda_{0}$ the generic element of
the space $\widetilde{Z}$ and setting $F_{0} = F$, define for $1 \leq
j \leq k$ the mappings
$$ 
F_{j} : \overline{\mathscr{O}} \times \overline{B}(0, \rho) \times
\prod_{i=1}^{j} \mathscr{L}_{i} (\mathbb{R}^{n+1}, \widetilde{Z}) \to
\mathscr{L}_{j} (\mathbb{R}^{n+1}, \mathbb{R}^{n}),
$$
by
\begin{align*}
F_{j}(\widetilde{x}, \lambda_{0}, \cdots, \lambda_{j}) & =
\frac{\partial F_{j-1}}{\partial \widetilde{x}} (\widetilde{x},
\lambda_{0}, \cdots, \lambda_{j-1}) + \tag{A2.60}\label{app-2-eqA2.60}\\
& + \sum\limits_{i=0}^{j-1} \frac{\partial F_{j-1}}{\partial
  \lambda_{i}} (\widetilde{x}, \lambda_{0}, \cdots, \lambda_{j-1}) \lambda_{i+1}.
\end{align*}

Of course, the mappping $F_{j}$ play with $F$ the role that the mappings
$\Phi_{j}$ play with $\Phi$ and have similar properties. For instance,
the mapping $F_{j}$ is of class $C^{k-j}$ and for $\widetilde{x}
\epsilon \overline{\mathscr{O}}$,
\begin{equation*}
D^{j}f_{\ell}(\widetilde{x}) = F_{j}(\widetilde{x},
\widetilde{\varphi}_{\ell + k}(\widetilde{x}),
D\widetilde{\varphi}_{\ell + k} (\widetilde{x}), \cdots,
D^{j}\widetilde{\varphi}_{\ell + k}(\widetilde{x}))\tag{A2.61}\label{app-2-eqA2.61}
\end{equation*}
for every $\ell \geq 0$, while
\begin{equation*}
D^{j}f(\widetilde{x}) = F_{j}(\widetilde{x},
\widetilde{\varphi}(\widetilde{x}),
D\widetilde{\varphi}(\widetilde{x}), \cdots,
D^{j}\widetilde{\varphi}(\widetilde{x})).\tag{A2.62} \label{app-2-eqA2.62}
\end{equation*}

For $0 \leq j \leq k-1$ and $\widetilde{x} \epsilon
\overline{\mathscr{O}}$, we shall set
\begin{align*}
c^{j}(\widetilde{x}) &   = \frac{\partial F_{j}}{\partial \widetilde{x}}
(\widetilde{x}, \widetilde{\varphi}(\widetilde{x}),
D\widetilde{\varphi}(\widetilde{x}), \cdots,
D^{j}\widetilde{\varphi}(\widetilde{x})),\tag{A2.63}\label{app-2-eqA2.63}\\
c_{\ell}^{j}(\widetilde{x}) & = \frac{\partial F_{j}}{\partial
  \widetilde{x}} (\widetilde{x}, \widetilde{\varphi}_{\ell +
  k}(\widetilde{x}), D\widetilde{\varphi}_{\ell + k} (\widetilde{x}),
\cdots, D^{j}\widetilde{\varphi}_{\ell + k}(\widetilde{x})) \text{ for
} \ell \geq 0.\tag{A2.64}\label{app-2-eqA2.64}
\end{align*}

Similarly, for $0 \leq j \leq k-1$, $0 \leq i \leq j$ and
$\widetilde{x} \epsilon \overline{\mathscr{O}}$, set
\begin{align*}
C_{i}^{j}(\widetilde{x}) & = \frac{\partial F_{j}}{\partial
  \lambda_{i}} (\widetilde{x}, \widetilde{\varphi}(\widetilde{x}),
D\widetilde{\varphi}(\widetilde{x}), \cdots,
D^{i}\widetilde{\varphi}(\widetilde{x})),\tag{A2.65}\label{app-2-eqA2.65}\\
C_{i\ell}^{j}(\widetilde{x}) & = \frac{\partial F_{j}}{\partial
  \lambda_{j}} (\widetilde{x}, \widetilde{\varphi}_{\ell +
  k}(\widetilde{x}), D\widetilde{\varphi}_{\ell + k}(\widetilde{x}),
\cdots, D^{i}\widetilde{\varphi}_{\ell + k}(\widetilde{x})) \text{ for
} \ell \geq 0\tag{A2.66}\label{app-2-eqA2.66}
\end{align*}

In what follows, we assume that the neighbourhood $\mathscr{O}$ has
been chosen so that Theorem \ref{app-2-thmA2.2} and \ref{app-2-thmA2.3} apply

\begin{alphlemma}\label{app-2-lemA2.8}%[A2.8]
(i) Assume\pageoriginale $k \geq 2$ and $0 \leq j \leq k-2$. Then, the
  sequence $(c_{\ell}^{j})_{\ell \geq 0}$ tends to $c^{j}$
  geometrically in the space $C^{0}(\overline{\mathscr{O}},
  \mathscr{L}(\mathbb{R}^{n+1}, \mathscr{L}_{j}(\mathbb{R}^{n+1},\break
  \mathbb{R}^{n})))$ and the sequence $(C_{i\ell}^{j})_{\ell \geq 0}$
  tends to $C_{i}^{j}$ in the space $C^{0} (\overline{\mathscr{O}},\break
  \mathscr{L}(\mathscr{L}_{i}(\mathbb{R}^{n+1}, \mathbb{R}^{n})$,
  $\mathscr{L}_{j}(\mathbb{R}^{n+1}, \mathbb{R}^{n})))$.

(ii) Assume only $k \geq 1$ and $0 \leq j \leq k-1$. Then, the
  sequence $(c_{\ell}^{j})_{\ell \geq 0}$ tends to $c^{j}$ in the
  space $C^{0}(\overline{\mathscr{O}}, \mathscr{L}(\mathbb{R}^{n+1},
  \mathscr{L}_{j}(\mathbb{R}^{n+1}, \mathbb{R}^{n})))$ and the
  sequence $(C_{i\ell}^{j})_{\ell \geq 0}$ tends to $C_{i}^{j}$ in
  the space $C^{0}(\overline{\mathscr{O}},
  \mathscr{L}(\mathscr{L}_{i}(\mathbb{R}^{n+1}, \mathbb{R}^{n}),
  \mathscr{L}_{i}(\mathbb{R}^{n+1}, \mathbb{R}^{n})))$.
  \end{alphlemma}

\begin{proof}
The proof of this lemma parallels that of Lemma \ref{app-2-lemA2.6}. Note only that
the (geometrical) convergence of the sequence
$(\widetilde{\varphi}_{\ell + k})$ to $\widetilde{\varphi}$ in the
space $C^{j}(\overline{\mathscr{O}}, \widetilde{Z})$ need not be
listed in the assumption since it is known from Theorem \ref{app-2-thmA2.2}
\end{proof}

\begin{alphtheorem}\label{app-2-thmA2.4}%[A2.4]
The sequence $(f_{\ell})$ tends to $f$ in the space
$C^{k}(\overline{\mathscr{O}}, \mathbb{R}^{n})$ and the convergence is
geometrical in the space $C^{k-1} (\overline{\mathscr{O}}, \mathbb{R}^{n})$.
\end{alphtheorem}

\begin{proof}
Once again, it is enough to prove equivalently that the sequence
$(D^{j} f_{\ell})_{\ell \geq 0}$ tends to $D^{j}f$ in the space
$C^{0}(\overline{\mathscr{O}}, \mathscr{L}_{j}(\mathbb{R}^{n+1},
\mathbb{R}^{n}))$ for $0 \leq j \leq k$, the convergence being
geometrical for $0 \leq j \leq k-1$. We shall distinguish the case $j
= 0$ and $j \geq 1$.

When $j = 0$, we must prove that the sequence $(f_{\ell})$ tends to $f$
geometrically in the space $C^{0}(\overline{\mathscr{O}},
\mathbb{R}^{n})$. By definition and with the Taylor formula (since $F$
is at least $C^{1}$), one has for $\widetilde{x} \epsilon
\overline{\mathscr{O}}$ and $\ell \geq 0$
{\fontsize{10}{12}\selectfont
\begin{align*}
f_{\ell} (\widetilde{x}) - f(\widetilde{x}) & = F(\widetilde{x},
\widetilde{\varphi}_{\ell + k}(\widetilde{x})) - F(\widetilde{x},
\widetilde{\varphi}(\widetilde{x})) \\
& = \int_{0}^{1} \frac{\partial F}{\partial \lambda_{0}}
(\widetilde{x}, \widetilde{\varphi}(\widetilde{x}) +
s(\widetilde{\varphi}_{\ell + k} (\widetilde{x}) -
\widetilde{\varphi}(\widetilde{x}))) \cdot (\widetilde{\varphi}_{\ell
  + k}(\widetilde{x}) - \widetilde{\varphi}(\widetilde{x})) ds.
\end{align*}}

By\pageoriginale the continuity of the mapping $(\partial F / \partial
\lambda_{0})$ on the compact set $\overline{\mathscr{O}} \times
\Lambda_{0}^{c}$, where $\Lambda_{0}^{c} \subset \overline{B}(0,
\rho)$ denotes the closed convex hull of the compact set $\Lambda_{0}$
(cf. Lemma \ref{app-2-lemA2.4}), there is a constant $K > 0$ such that
$$
||\frac{\partial F}{\partial \Lambda_{0}} (\widetilde{x},
\widetilde{\varphi}(\widetilde{x}) + s(\widetilde{\varphi}_{\ell +
  k}(\widetilde{x}) - \widetilde{\varphi}(\widetilde{x})))|| \leq K,
$$
for every $\widetilde{x} \epsilon \overline{\mathscr{O}}$ and $\ell
\geq 0$. Hence
$$
|f_{\ell} - f|_{\infty, \overline{\mathscr{O}}} \leq
K|\widetilde{\varphi}_{\ell + k} - \widetilde{\varphi}|_{\infty,
  \overline{\mathscr{O}}} 
$$
and the geometrical rate off convergence of the sequence $(f_{\ell})$
to $f$ in the space $C^{0}(\overline{\mathscr{O}}, \mathbb{R}^{n})$
follows from Theorem \ref{app-2-thmA2.2}.

When $1 \leq j \leq k$ and with (\ref{app-2-eqA2.60}) - (\ref{app-2-eqA2.62}) and the notation
(\ref{app-2-eqA2.63})-(\ref{app-2-eqA2.66}), the method we used in Theorem \ref{app-2-thmA2.2} and \ref{app-2-thmA2.3} leads to
the inequality
\begin{align*}
||D^{j}f_{\ell}(\widetilde{x}) - D^{j}f(\widetilde{x})|| &\leq
||c_{\ell}^{j-1} (\widetilde{x}) -C^{j-1}(\widetilde{x})||\\ 
&{}+||\sum\limits_{i=0}^{j-1} ||c_{i\ell}^{j-1} (\widetilde{x}) -
||C_{i}^{j-1}(\widetilde{x})|| ||D^{i+1}
||\widetilde{\varphi}(\widetilde{x})|| \\
& + \sum\limits_{i=0}^{j-1} ||C_{i\ell}^{j-1} (\widetilde{x})||
||||D^{i+1}\widetilde{\varphi}_{\ell + k}(\widetilde{x}) -
||D^{i+1}\widetilde{\varphi}(\widetilde{x})||, 
\end{align*}
for every $\widetilde{x} \epsilon \overline{\mathscr{O}}$. Due to the
compactness of the set $\overline{\mathscr{O}} \times \Lambda_{0}
\times \cdots \times \Lambda_{j-1}$ (Lemma \ref{app-2-lemA2.4}) and the regularity
$\widetilde{\varphi} \epsilon C^{k}(\overline{\mathscr{O}},
\widetilde{Z})$, there is a constant $K > 0$ such that
\begin{align*}
& ||D^{i+1} \widetilde{\varphi}(\widetilde{x})|| \leq K,\\
& ||C-{\i\ell}^{j-1} (\widetilde{x})|| \leq K,
\end{align*}
for $\widetilde{x} \epsilon \overline{\mathscr{O}}, 0 \leq i \leq j-1$
and $\ell \geq 0$. Thus, we first obtain
\begin{align*}
&||D^{j}f_{\ell}(\widetilde{x}) - D^{j}f(\widetilde{x})|| <
||c_{\ell}^{j-1} (\widetilde{x})-c^{j-1}(\widetilde{x})\\ 
&\qquad{}+ K\left[\sum\limits_{i=0}^{j-1} ||C_{i\ell}^{j-1} (\widetilde{x}) -
  C_{i}^{j-1}(\widetilde{x})|| + ||D^{i+1} \widetilde{\varphi}_{\ell +
k}(\widetilde{x}) - D^{i + 1} \widetilde{\varphi}(\widetilde{x})||\right]
\end{align*}
and\pageoriginale next
\begin{align*}
|D^{j}(f_{\ell} - f)|_{\infty, \overline{\mathscr{O}}} &\leq
|c_{\ell}^{j} - c^{j-1}|_{\infty, \overline{\mathscr{O}}}\\ 
&{}+ K|\left[\sum\limits_{i=0}^{j-1} |C_{i\ell}^{j-1} -
 C_{i}^{j-1}|_{\infty, \overline{\mathscr{O}}} + D^{i+1}
(\widetilde{\varphi}_{\ell + k} - \widetilde{\varphi})|_{\infty,
|\overline{\mathscr{O}}}\right].\tag{A2.67} \label{app-2-eqA2.67}
\end{align*}

For $k \geq 2$ and $1 \leq j \leq k-1$ (i.e. $0 \leq j-1 \leq
k-2$). this inequality yields the geometrical convergence of the
sequence $(D^{j}f_{\ell})_{\ell \geq 0}$ to $D^{j}f$ in the space
$C^{0}(\overline{\mathscr{O}}, \mathscr{L}_{j}(\mathbb{R}^{n+1},
\mathbb{R}^{n}))$ by applying Theorem \ref{app-2-thmA2.2} and Lemma \ref{app-2-lemA2.8} (i). For $k
\geq 1$ and $j = k$, convergence of the sequence
$(D^{k}f_{\ell})_{\ell \geq 0}$ to $D^{k}f$ (with no rate of
convergence in the space $C^{0}( \overline{\mathscr{O}},
\mathscr{L}_{k}(\mathbb{R}^{n+1}, \mathbb{R}^{n}))$ also follows from
(\ref{app-2-eqA2.67}) by applying Theorem \ref{app-2-thmA2.3} and Lemma \ref{app-2-lemA2.8} (ii).
\end{proof}

\begin{alphcorollary}\label{app-2-coroA2.2}%[A2.2]
Assume that both mappings $F$ and $\Phi$ are of class  $C^{k+1}$. Then,
the sequence $(f_{\ell})$ tends to the maping f geometrically in the
space $C^{k}(\mathscr{O}, \mathbb{R}^{n})$.
\end{alphcorollary}

\begin{proof}
When the mapping $F$ is of class $C^{k-1}$, the mappings $F_{j}$ $0 \leq
j \leq k$, are of class $C^{k+1-j}$and hence the mappings $c^{j},
c_{\ell}^{j}, C_{i}^{j}$ and $C_{i\ell}^{j}$ are of class $C^{1}$ at
least. This allows us to prove Lemma \ref{app-2-lemA2.8} (i) for the indices $0 \leq j
\leq k-1$ (instead of $0 \leq j \leq k-2$). On the other hand, if the
mapping $\Phi$ is of class $C^{k+1}$ too, the sequence
$(\widetilde{\varphi}_{\ell})$ tends to the mapping
$\widetilde{\varphi}$ geometrically in the space
$C^{k}(\overline{\mathscr{O}}, \widetilde{Z})$ (Corollary \ref{app-2-coroA2.1}) and so
does the sequence $(\widetilde{\varphi}_{\ell + k})$. The result
follows now from inequality (\ref{app-2-eqA2.67}) for $1 \leq j \leq k$ (i.e. $0
\leq j-1 \leq k-1$), the geometrical convergence of the sequence
$(f_{\ell})$ to $f$ in the space $C^{0}(\overline{\mathscr{O}},
\mathbb{R}^{n})$ having been already proved in Theorem \ref{app-2-thmA2.3}.
\end{proof}


\begin{thebibliography}{99}
\bibitem{1} {BERGER, M.S. :}\pageoriginale {\textit Nonlinearity and Functional
  Analysis},   Acad. Press (1977).

\bibitem{2} {BEREGER, M.S. - WESTRETICH, D. :} A Convergent Iteration
  Scheme for Bifurcation Theory in Banach Spaces, {\textit
    J. Math. Anal. Appl.,} 43, 136-144 (1973).

\bibitem{3} {BERS, L. :} {\textit Introduction to Several Complex
  Variables,} Courant Institute Lect. Notes (1964).

\bibitem{4} {BREZIS, H. :} {\textit Analyse Fonctionnelle,}  Masson
  (1983).

\bibitem{5} {BUCHNER, M. -MARSDEN, J. - SCHECTER, S. :} Application of
  the Blowing - Up Construction and Algebraic Geometry to Bifurcation
  Probelms, {\textit J. Diff Eq.,} 48, 404-433 (1983).

\bibitem{6} {CIARLET, P.G. - RABIER, P. :} {\textit Les Equations de
  von K\'{a}rm\'{a}n}, Lect. Notes in Math., 826, Springer (1980).

\bibitem{7} {CRANDAL, M, - RABINOWITZ, P. :} Bifurcation from Simple
Eignevalues, {\textit J. Funct. Anal.,} 8, 321-340 (1971).

\bibitem{8} {DANCER, E. :} Bifurcation Theory in Real Banach Spaces,
        {\textit Proc. London Math. Soc.,} 23, 699-734 (1971).

\bibitem{9} {DIEUDONN\'{E}, J. :} {\textit Foundations of Modern
  Analysis}, Acad. Press (1960).

\bibitem{10} {EL HAJJI, S. :} {\textit Th\'{e}se de 3eme Cycle},
Univ. P. et M. Curie, Paris (1983).

\bibitem{11} {GIRAUD, J. :} {\textit G\'{e}om\'{e}trie Alg\'{e}brique
  El\'{e}mentaire,} Publications Math\'{e}matiques d'Urasy (1977).

\bibitem{12} {GOLUBITSKY, M. -GUILLEMIN, V. :}\pageoriginale {\textit
  Stable Mappings and Their Singularities,} G.T.M. 14, Springer (1973).

\bibitem{13} {GOLUBITSKY, M. - KEYFITZ, B. :} A Qualitative Study of
Steady - State Solutions for a Continuous Flow Stirred Tank Chemical
Reactor,  {\textit STAM J. Math. Anal}., 11, 316-339 (1980).

\bibitem{14} {GOLUNITSKY, M. -MARSDEN, J. :} The Morse Lemma in
Infinite Dimensions vias Singularity Theorey (Preprint).

\bibitem{15} {GOLUBITSKY, M. - SCHAEFFER, D. :} A Theory for Imperfect
Bifurcation via Singularity Theorey, {\textit Comm. Pure Appl. Math.,}
32, 21-98 (1979).

\bibitem{16} {HODGE, W.V.D. - PEDOE, D. :} {\textit Methods of
  Algebraic Geometry,} Vol. I Cambridge University Press (1968).

\bibitem{17} {KATO, T. :} {\textit Perturbation Theory far Linear
  Operators,} Springer (1976).

\bibitem{18} {KENDIG, K. :} {\textit Elementary Algebraic Geometry,}
  GTM 44, Springer (1977).

\bibitem{19} {KRASNOSELSKII, M.A. :} {\textit Topological Methods in
  the Theory of Nonlinear Integral Equations}, The Macmillan Co.,
  Pergamon Press (1964).

\bibitem{20} {KREIN. M.G. - RUTMAN, M. A. :} Linear Operators Leaving
  a Cone in a Banach Space Invariant, {\textit Amer. Math. Soc
    Trans.,} Ser 1 (10) (1950).

\bibitem{21} {KUIPER, N.H. :} $C^{1}$ Equivalence of Functions Near
  Isolated Critical Points, in {\em Symposium on Infinite -
    Dimensional Topology}, R.D. Anderson Ed. Ann. of Math. Studies,
  69, Princerton University Press (1972).

\bibitem{22} {LUUSTERNIK, L.A. - SOBOLEV, V.J. :} {\textit Elements of
Functional Analysis}, Gordon and Breach (1961).

\bibitem{23} {Mc LEOD, J.B.-SATTINGER, D. :}\pageoriginale Loss of
  Stability and Bifurcation at Double Eigenvalues, {\textit J. Funct,
    Anal.,} 14, 62-84 (1973).

\bibitem{24} {MAGNUS, R.J. :} On the Local Structure of the Zero Set
  of a Banach Space Valued Mapping, {\textit J. Funct. Anal.,} 22,
  58-72 (1976).

\bibitem{25} {MARSDEN, J. :} Qualitative Methods in Bifurcation
  Theory, {\textit Bull. Amer. Math. Soc.,} 84, 1125-1148 (1978).

\bibitem{26} {MUMFORD, D. :} {\em Algebraic Geometry I. Complex
  Projective Varieties,} Frundlehren der Math. Wiss., 221, Springer (1976).

\bibitem{27} {NIRENBERG, L. :} {\textit Topics in Nonlinear Functional
Analysis}, Courant Institute Lecture Notes (1974).

\bibitem{28} {PALAIS, R. :} The Morse Lemma on Banach Spaces, {\textit
Bull. Amer. Math Soc.,}75, 968-971 (1969).

\bibitem{29} {RABIER, P. :} A Generalization of the Implicit Function
  Theorem for Mappings from $\mathbb{R}^{n+1}$ into $\mathbb{R}^{n}$
  and its Applications, {\textit J. Funct. Anal.,} 56, 145-170 (1984).

\bibitem{30} {RABIER, P. : } A Necessary and Sufficient Condition for
  the Coerciveness of Certain Functionsl, {\textit STAM
    J. MAth. Anal.,} Vol. 15, no.2, 367-388 (1984).

\bibitem{31} {RABIER, P. :} A Synthetic Study of One-Parameter
  Nonlinear Problems (Preprint).

\bibitem{32} {RABIER, P. :} A Desingularization Process in Degenerate
  Bifurcation Problems, Parts I and II (Preprint).

\bibitem{33} {RABIER, P.-EL HAJJI, S. :}\pageoriginale  New Algorithms for the
  Computation of the Branches in One-Parameter Nonlinear Problems (To
  appear in {\textit Num. Mat.}).

\bibitem{34} {SATTINGER, D. H. :} {\textit Group Theoretic Methods in
  Bifurcation Theory,} Lect, Notes in Math., 762, Springer (1979).

\bibitem{35} {SCHAEFFER, H. :} {\em Topological Vector Spaces,}
  Springer, 2nd ed. (1971).

\bibitem{36} {SCHECTER, M. :} {\textit Principles of Functional
  Analysis,} Acad. Press (1971).

\bibitem{37} {SEIDENBERG, A. :} A New Decision Method in Elementary
  Algebra, {\textit Ann. of Math.,} 60, 365-374 (1954).

\bibitem{38} {SHAFAREVICH, J. R. :} {\textit Basic Algebraic Geometry,
} Springer (1974).

\bibitem{39} {SHAERER, M. :} Bifurcation from a Multiple Eigenvalue in
  {\textit Ordinary and Partial Differentail Equation} (Dundee 1976),
  Lect. Notes in Math., 564, 417-427, Springer (1976).

\bibitem{40} {SZULKIN, A. :} Local Structure of the Zero Set of a
  Differentaible Mapping and Application to Bifurcation Theorey,
  {\textit Math, Scand.,} 45, 232-242 (1979).

\bibitem{41} {TROMBA, A. J. :} A Sufficient Condition for a Critical
  Point of a Functional to be a Minimum and its Application to
  Plateau's Problem (To appear in Math. Ann.).

\bibitem{42} {BEYN, W. J.} A Note on the Lyapunov-Schmidt Reduction
  (Private communcation).
\end{thebibliography}


 
