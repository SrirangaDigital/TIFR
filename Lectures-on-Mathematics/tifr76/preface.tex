\thispagestyle{empty}

\begin{center}
{\Large\bf Lectures on}\\[5pt]
{\Large\bf Topics In One-Parameter Bifurcation Problems}
\vfill

{\bf By}
\medskip

{\large\bf P. Rabier}
\vfill

{\bf Tata Institute of Fundamental Research}

{\bf Bombay}

{\bf 1985}
\end{center}

\eject

\thispagestyle{empty}

\begin{center}
{\Large\bf Lectures on}\\[5pt]
{\Large\bf Topics In One-Parameter Bifurcation Problems}
\vfill

{\bf By}
\medskip

{\large\bf P. Rabier}
\vfill

Lectures delivered at the\\
{\bf Indian Institute of Science, Bangalore}\\
under the \\
{\bf T.I.F.R.--I.I.Sc.} Programme In  Applications of\\
Mathematics
\vfill

{\bf Notes by}
\medskip

{\large\bf G. Veerappa  Gowda}
\vfill

Published for the 
\medskip

{\bf Tata Institute of Fundamental Research}
\vfill

Springer-Verlag\\
Berlin Heidelberg New York Tokyo\\
1985
\end{center}

\eject

\thispagestyle{empty}

\begin{center}
{\bf Author}\\
{\large\bf P. Rabier}\\
Analyse Num\'erique, Tour 55-65, $5^e$ \'etage\\
Universit\'e Pierre et Marie Curie\\
4, Place Jussieu\\
75230 Paris Cedex 05\\
France

\vfill

{\bf\copyright \quad Tata Institute of Fundamental Research, 1985}
\vfill

\noindent\rule{\textwidth}{1pt}
 
ISBN 3-540-13907-9 Springer-Verlag, Berlin. Heidelberg.\\ New York. Tokyo\\

ISBN 0-387-13907-9 Springer-Verla, New York. Heidelberg.\\ Berlin. Tokyo

\noindent\rule{\textwidth}{1pt}

\vfill

\parbox{0.7\textwidth}{No part of this book may be reproduced in any form
by print, microfilm or any other means without
 written permission from the Tata Institute of 
Fundamental Research, Colaba, Bombay 400 005}
\vfill

Printed by M. N. Joshi at The Book Centre Limited,\\
Sion East, Bombay 400 022 and published by H. Goetze,\\
Springer-Verlag, Heidelberg, West Germany\\
\vfill

Printed In India
\end{center}


\chapter{Preface}

\markboth{Preface}{Preface}

This set of lectures is intended to give a somewhat synthetic exposition for the study of one-parameter bifurcation problems. By this, we mean the analysis of the structure of their set of solutions through the same type of general arguments in various situations.

Chapter I includes an introduction to one-parameter bifurcation\break problems motivated by the example of {\em linear eiqenvalue problems} and step by step generalizations lead to the suitable mathematical form. The {\em Lyapunov-Schmidt reduction} is detailed next and the chapter is completed by an introduction to the mathematical method of resolution, based on the {\em Implicit function theorem} and  the {\em Morse lemma} in the simplest cases. The result by Crandall and Rabinowitz \cite{7} about {\em bifurcation from the trivial branch at simple characteristic values} is given as an example for application.

Chapter II presents a {\em generalization of the Morse lemma} in its\break ``weak'' form to mappings from $\mathbb{R}^{n+1}$ into $\mathbb{R}^n$. A slight improvement of one degree of regularity of the curves as it can be found in the literature, is proved, which allows one to include the case when the Implicit function theorem applies and is therefore important for the homogeneity of the exposition. The relationship with stronger versions of the Morse lemma is given for the sake of completeness but will not be used in the sequel. 

Chapter III shows how to apply the results of Chapter II to the study of one-parameter bifurcation problems. Attention is confined to two general examples. The first one deals with problems of bifurcation from the trivial branch at a multiple characteristic value. A direct application may be possible but, for higher non-linearities, a preliminary {\em change of scale} is necessary. The justification of this change of scale is given at an intuitive level only, because a detailed mathematical justification involves long and tedious technicalities which do not help much for understandig the basic phenomena, even if they eventually provide a satisfactory justification for the use of Newton diagrams (which we do not use however). The conclusions we draw are, with various additional information, those of McLeod and Sattinger \cite{23}. The second example is concerned with a problem in which {\em no particular branch of solutions is known a priori.} It is pointed out that while the case of a simple singularity is without bifurcation, bifurcation does occur in general when the singularity is multiple. Also, it is shown how to get further details on the location of the curves when the results of Chapter II apply after a suitable change of scale and how this leads at once to the distinction between ``turning points'' and ``hysteresis points'' when the singularity is simple.

Chapter IV breaks with the traditional exposition of the Lyapunov-Schmidt method, of little and hazardous practical use, because its assumed data are not known in the applications while the imperfection sensitivity of the method has not been evaluated (to the best of our knowledge at least). Instead, we present a new, more general (and we believe, more realistic) method, introduced in Rabier-El Hajji \cite{33} and derived from the ``almost'' constructive  proofs of Chapter II. Optimal rate of convergence is obtained. For the sake of brevity, the technicalities of \S\ 5 have been skipped but the first four sections fully develop all the main ideas.

Chapter V introduces a new method in the study of bifurcation problems in which the nondegeneracy condition of Chapter II is not fulfilled. Actually, the method is new in that it is applied in this context but similar techniques are classical in the desingularization of algebraic curves. We show how to find the local zero set of a $C^\infty$ real-valued function of two variables (though the regularity assumption can be weakened in most of the cases) verifying $f(0)=0$, $Df (0) = 0$, $D^2 f(0) \neq 0$ {\em but} $\det D^2 f(0)=0$ (so that the Morse condition fails). This method is applied to a problem of bifurcation from the trivial branch at a geometrically simple characteristic value when the nondegeneracy condition of Crandall and Rabinowitz is {\em not} fulfilled (i.e. the {\em algebraic multiplicity} as $> 1$). The role played by the generalized null space is made clear and the result complements Krasnoselskii's bifurcation theorem in the particular case under consideration. 

\newpage

\begin{center}
\textbf{Acknowledgement}
\end{center}


I wish to thank Professor M.S. NARASIMHAN for inviting me and giving me the opportunity to deliver these lectures at the Tata Institute of Fundamental Research Centre, Bangalore, in July and August 1984. I am also grateful to Drs. S. KESAVAN and M. VANNINATHAN who initially suggested my visit.

These notes owe much to Professor S. RAGHAVAN and Dr. S. KESAVAN who used a lot of their own time reading the manuscript. They are responsible for numerous improvements in style and I am more than thankful to them for their great help.
