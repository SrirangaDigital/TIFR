
 \chapter{Transmutation in the Irregular Case}\label{part1:chap4}%chap 4
 
 \heading{\bigg[Case of the general operator : $D^2 +
     \left(\dfrac{2p+3}{x}\right) + M(x) D + N(x)$\bigg]} 
  
\section{}\label{part1:chap4:sec1}%sec 1.
 
Let\pageoriginale $M$ and $N$ be two indefinitely differentiable functions, $M$ odd
(i.e., $M^{(2n)} (0) = 0$ for every $n \geq 0$) and $N$ even
(i.e., $N^{2n+1}(0) = 0$ for every $n \geq 0$ i.e., $N \in
\mathscr{E}_*$). 
 
We consider the two following problems. 

\setcounter{prob}{0}
\begin{prob}\label{part1:chap4:sec1:prob1}%problem 1.
  To find $u (x, y)$, indefinitely differentiable even in $x$ and $y$
  which is a solution of  
  \begin{equation*} 
    \frac{\partial^2 u}{\partial x^2} + \left(\frac{2p+1}{x} + M(x)\right)
    \frac{\partial u}{\partial x} + N(x) u - \frac{\partial^2
      u}{\partial y^2} = 0 \tag{1}\label{part1:chap4:sec1:eq1} 
  \end{equation*}
with
\begin{equation*}
    u(0, y) = g(y), g \in \mathscr{E}_* \tag{2}\label{part1:chap4:sec1:eq2}
\end{equation*}
\end{prob}

\begin{prob}\label{part1:chap4:sec1:prob2}%problem 2.
    To find $v(x, y)$ indefinitely differentiable, even in $x$ and
    $y$, solution of the same equation (\ref{part1:chap4:sec1:eq1}) with 
  \end{prob} 
  \begin{equation*}
    v(x, 0)= f (x), f \in \mathscr{E}_* \tag{3}\label{part1:chap4:sec1:eq3}
  \end{equation*} 

We intend to prove the following 
\begin{theorem}\label{part1:chap4:sec1:thm1}%them 1.
  For $p \neq -1, -2, \ldots$, each of the two problems stated above
  admits a unique solution, which depends continuously on $g$ (or
  $f$). 
\end{theorem} 

Once\pageoriginale we prove this theorem, we can define
\textit{Operators $X_p$ and $\mathfrak{X}_p$}.
 
\begin{defn}\label{part1:chap4:sec1:def1}%defini 1.
  For $g \in \mathscr{E}_{*}$, and $p \neq -1, -2, \ldots$
  $$
  \mathfrak{X}_p [g(x)] = u (x, 0),
  $$
  where $u$ is the solution of the Problem $1$.
\end{defn} 

\begin{defn}\label{part1:chap4:sec1:def2}%defini 2.
  For $f \in \mathscr{E}_{*}$, and $p \neq -1, -2, \ldots$
  $$
  X_p [ f(y)] = v(0, y)
  $$
  where $v$ is the solution of the second problem. Assuming Theorem
  \ref{part1:chap4:sec1:thm1}, definitions
  (\ref{part1:chap4:sec1:def1}) and (\ref{part1:chap4:sec1:def2}) give
  immediately  
 \end{defn}

\begin{theorem}\label{part1:chap4:sec1:thm2}%them 2.
  For $p \neq -1, -2, \ldots, X_p$ and $\mathfrak{X}_p$ are in
  $L(\mathscr{E}_{*}, \mathscr{E}_{*})$. 
\end{theorem}  
  
We now prove
\begin{theorem}\label{part1:chap4:sec1:thm3}%them 3.
  For $p \neq -1, -2, \ldots$, we have
  \begin{align*}
  D^2 X_p &= X_p \Lambda_p \tag{4}\label{part1:chap4:sec1:eq4}\\
  \mathfrak{X}_p D^2 & = \Lambda_p \mathfrak{X}_p
  \tag{5}\label{part1:chap4:sec1:eq5}\\ 
    \text{where}\hspace{2cm}
    \Lambda_p & = D^2 + \frac{2p+1}{x}D + M(x) D+ N(x)\hspace{1cm}
    \tag{6}\label{part1:chap4:sec1:eq6} 
  \end{align*}
   
  Applying the operator $\dfrac{\partial^2}{\partial y^2}$ to the two
  sides of  
  $$
  \frac{\partial^2 v}{\partial x^2} + \left(\frac{2p+1}{x} + M (x)\right)
  \frac{\partial v}{\partial x} + N (x) v - \frac{\partial^2 v}{\partial
    y^2} = 0 
  $$
  and\pageoriginale setting $\dfrac{\partial^2 v}{\partial y^2} = V$, we have
  $$
  (\Lambda_p)_x V - \frac{\partial^2 V}{\partial y^2}= 0,
  $$
  and $V(x,0) = \dfrac{\partial^2 v}{\partial y^2} (x,0) =
  (\Lambda_p)_x (v(x,0))= \Lambda_p [f(x)]$; and by the definition of
  $X_p,  V(0, y) = X_p \Lambda _p [f(x)]$. 
\end{theorem}

But we have also $\dfrac{\partial^2 V}{\partial y^2} (0,y) = D^2 _y
[X_p f(y)]$. Thus (\ref{part1:chap4:sec1:eq4}) is proved. Similarly
(\ref{part1:chap4:sec1:eq5}) can be proved. But
(\ref{part1:chap4:sec1:eq5}) follows from (\ref{part1:chap4:sec1:eq4})
and the  
\begin{theorem}\label{part1:chap4:sec1:thm4}%them 4.
  For $p \neq -1, -2, \ldots,$ the operators $X_p$, $\mathfrak{X}_p$
  are isomorphisms of $\mathscr{E}_{*}$ onto itself and $X_p
  \mathfrak{X}_p = \mathfrak{X}_p X_p=$ the identity. 
\end{theorem}  

Let $g \in \mathscr{E}_{*}$ and $u$ be the solution of Problem
\ref{part1:chap4:sec1:prob1}. Then $u(x,0) = \mathfrak{X}_p [g(x)]$. In
order to find $X_p 
\mathfrak{X}_{p}[g(x)]$ it is sufficient to determine the solution
$v(x, y)$ of 	 
$$
\frac{\partial^2 v}{\partial y^2} + (\frac{2p+1}{x}+ M(x))
\frac{\partial  v}{\partial x} + N (x) v - \frac{\partial^2
  v}{\partial y^2}=0 
$$
with $v(x,0) = \mathfrak{X}_p [g(x)] = u(x, 0)$.
  
Then, in view of Theorem \ref{part1:chap4:sec1:thm1}, $u (x, y) = v(x,
y)$. Hence $X_p 
\mathfrak{X}_p [g(y)] = u (0, y) = g(y)$ similarly\pageoriginale
$\mathfrak{X}_p X_p = $ the identity. 
  
\noindent \textbf{The Theorem 1.} We now proceed to solve
Problem \ref{part1:chap4:sec1:prob1}. Let
$(B_p)_x$ $[u (x,y)]= w(x,y)$ where $(B_p)_x$ denotes the operator $B_p$
relative to the variable  $x$. Applying $(B_p)_x$ to the two sides of
equation (\ref{part1:chap4:sec1:eq1}), i. e.,  of 
$$
(L_p)_x [u(x,y)] + M(x) \frac{\partial u}{\partial x} + N (x) u -
\frac{\partial^2 u}{\partial y^2}=0 
$$
  and using $D^2 B_p = B_p L_p$, we obtain
  $$
  \frac{\partial^2 w}{\partial x^2}-\frac{\partial^2 w}{\partial
    y^2}+ (B_p)_x [M(x) \frac{\partial  u}{\partial x} + N(x) u] = 0 
  $$
  and we have also $w(0,y) = g(y) = u (0,y)$ since $B_p f(0) = f(0)$.
  
  We intend to put this equation which is equivalent to equation
  (\ref{part1:chap4:sec1:eq1}) in a proper form, We first prove 

\setcounter{proposition}{0}
\begin{proposition}\label{part1:chap4:sec1:prop1}%proposi 1.
  $B_p A \mathscr{B}_p f(x) = A(x) f(x) + x \int_{o}^{x} T_p [A] (x,y)
  f(y) dy$ where $A(x)$ and $f(x) \in \mathscr{E}_{*}$ and $T_p$ is a
  map defined in $\mathscr{E}_{*}$ with values in the space of even
  indefinitely differentiable functions of the variables $x$ and $y$. 
\end{proposition}  
  
 For $-1 < \Rep< \dfrac{1}{2}$,
 $$
 \displaylines{\hfill 
   \bar{B}_p f(x) = \bar{b}_p \int_{o}^{x} (x^2 - y^2 )^{- p-
     \frac{1}{2}}y^{2p+1} f(y) dy\hfill \cr 
   \text{where}\hfill \bar{b}_p = \dfrac{\sqrt{\pi}}{\Gamma (p+1) \Gamma (-p +
   \dfrac{1}{2})}, \bar{B}_p f(0) = 0,  D\bar{B}_p = B_p\hfill \cr 
   \text{and for} \hfill \Rep > - \dfrac{1}{2},\hfill \cr 
   \hfill \mathscr{B}_p g(x) = \beta _p x^{-2p} \int_{o}^{x} (x^2 - y^2 )^{p -
     \frac{1}{2}} f(y) dy \hfill }
 $$
 where\pageoriginale \qquad $\beta _p = \dfrac{2 \Gamma
   (p+1)}{\sqrt{\pi}\Gamma (p + \dfrac{1}{2})}$. 
 
 Hence for $- \dfrac{1}{2} < \Rep< \dfrac{1}{2}$, we compute
\begin{align*}
  \bar{B}_p A \mathscr{B}_p f(x) & = \bar{b }_p \beta_p \int_o^x (x^2 -
  z^2)^{-p - \frac{1}{2}}z^{2p+1} A(z)\\ 
  & \hspace{2cm} \left[ z^{-2p} \int^z_o (z^2 - y^2 )^{p-\frac{1}{2}} f(y)
    dy \right] dz\\  
  & = \bar{b}_p \beta_p \int_o^x f(y) \bigg[ \int_y^x (x^2 - z^2 )^{-p
      - \frac{1}{2}} (z^2 - y^2 )^{p- \frac{1}{2}}A(z) zdz \bigg ]
  dy. 
\end{align*}
 
Setting $z^2 = x^2 \sin^2 \theta  + y^2 \cos^2 \theta $, we get
\begin{align*}
  \bar{B}_p A \mathscr{B}_p f(x) & = \bar{b}_p \beta_p \int_o \phi_p
  (x,y) f(y) dy\\ 
  \text{where},\hspace{.5cm}
  \phi_p (x,y) & = \int^{\frac{\pi}{2}}_o  \sin^{2p}\theta 
  \cos^{-2p}\theta  A \bigg[ \sqrt{x^2 \sin^2 \theta  + y^2 \cos^2
      \theta } \bigg ] d \theta  
\end{align*}
which converges for $-\dfrac{1}{2} < \Rep < \frac{1}{2}$. It follows that
$$
D\bar{B}_p A \mathscr{B}_p f(x) = \bar{b}_p \beta_p \phi_p (x, x) f(x)
+ \bar{b}_p \beta_p \int_{0}^x \frac{\partial}{\partial x} \phi_p (x,
y) f(y) dy 
$$
 
 But\pageoriginale we have  
 $$
 \displaylines{\hfill 
   \phi _p (x, x) = A(x) \frac{1}{2} \Gamma (p + \frac{1}{2}) \Gamma \left(-p
   + \frac{1}{2}\right) \hfill \cr
   \text{so that} \hfill 
   \bar{b}_p \beta_p \phi_p (x, x) = A(x).\hfill }
 $$
 
 Hence $B_p A \mathscr{B}_p f(x) = A(x) f(x) + \gamma_p \int_{0}^{x}
 \dfrac{\partial}{\partial x}\phi _p (x, y) f(y) dy$ where 
 $$
\gamma_p = \frac{2}{\Gamma (p + \frac{1}{2}) \Gamma (- p + \frac{1}{2})}.
 $$
 
 Now $\dfrac{\partial}{\partial x} \phi_p (x, y) =
 \int_{0}^{\dfrac{\pi}{2}} \sin^{2p+2} \theta  \cos^{-2p} \theta 
 \dfrac{A' (z)}{z}x d \theta $ so that if we give 

\begin{defn}\label{part1:chap4:sec1:def3}%defini 3.
  For $f \in \mathscr{E}_{*}$
  $$
  T_p [f] (x, y) = \gamma_p \int^{\frac{\pi}{2}}_0 \sin^{2p+2} \theta 
  \cos^{-2p}\theta  f_1 (z) d 
  $$
  where $f_1 (z) = \dfrac{f'(z)}{z}, z = \left[ x^2 \sin^2 \theta  +
    y^2 \cos^2 \theta  \right]^{\frac{1}{2}}, \gamma_p = \bar{b}_p
  \beta_p$, then we can write 	 
\end{defn} 

\begin{equation*}
  B_p A \mathscr{B} f(x) = A (x) f(x) + x \int^{x}_o T_p [A] (x, y)
  f(y) dy \tag{7}\label{part1:chap4:sec1:eq7} 
\end{equation*} 
(\ref{part1:chap4:sec1:eq7}) is valid for $| \Rep | < \dfrac{1}{2}$
and $f \in \mathscr{E}_{*}$. 

\begin{remark*}
  We\pageoriginale have $\gamma_p = 0$ for $p = \pm \dfrac{1}{2},  \pm \dfrac{3}{2},
  \ldots$ so that $T_{-\frac{1}{2}}= 0$. Again for $p = - \frac{1}{2},
  B_p = \mathscr{B}_p =$ the identity. Hence $(7)$ is valid for $p =-
  \frac{1}{2}$. Now $w(x, y) = (B_p )_x u(x, y)$ so that $u (x, y) =
  (\mathscr{B}_p)_x w(x, y)$ and $(B_p)_x M(x) \dfrac{\partial
    u}{\partial x} = (B_p)_x M(x) D_x (\mathscr{B_p})_x w(x, y)$. We
  shall denote by  $\phi (x)$, the function $w(x, y)$ considered as a
  function of $x$; let $(\mathscr{B}_p)_x w(x,y) = \mathscr{B}_p \phi
  = \psi$. Then $B_p \psi = \phi, B_p MD \mathscr{B}_p \phi = B_p MD
  \psi = B_p (D (M \psi))  B_p (M' \psi)= B_p [D (xM^* \psi )]- B_p
  (M' \psi) $ where $M^* (x) = \dfrac{M(x)}{x} \in \mathscr{E}_{*}$ and
  $B_p [MD \psi] = B_p [x DM^* \psi ] + B_p [(M^* - M') \psi]$. But for
  $-1 < \Rep <- \dfrac{1}{2}$, 
  $$
  B_p [x DF] = b_p x \int_{o}^{1 }t^{2p+2} (1-t^2)^{-p-3/2} F' (tx) dt
  = x D(B_p [F])  
  $$

  Hence $B_p (MD \psi) = x DB_p M^* \psi + B_p [(M^* - M) \psi]$

  Now by (\ref{part1:chap4:sec1:eq7}),
  \begin{multline*}
    B_p [MD \psi]  = xD \left\{ M^* \phi + x \int_o^x T_p [M^*] (x,y) \phi
    (y) dy \right\}\\ 
     \hspace{2cm}+ (M^* - M') \phi +x \int_{o}^{x} T_p [M^* - M ] (x, y)
    \phi (y) dy.\\ 
    = x^2 T_p [M^*] (x, x) \phi (x) + x \int_o^x S_p [M] (x, y) \phi (y)
    dy + M (x) \phi' (x) 
  \end{multline*}
  where\pageoriginale $S_p [M] (x,y) = T_p [2M^* - M] (x,y) + x
  \dfrac{\partial}{\partial x} T_p [M^*] (x,y)$ 
\end{remark*}  
As 
\begin{align*}
  T_p [M^*] (x, x) & = \gamma_p \dfrac{M^{*'} (x) }{x} \int_o
 ^{\dfrac{\pi}{2}} \sin^{2p+2} \theta  \cos^{-2p} \theta  d \theta  \\
  & \hspace{2cm}= \left(p+\dfrac{1}{2}\right) \dfrac{M^{*'} (x)}{x},\\ 
  B_p MD \mathscr{B}_p [\phi] & = M(x) \phi' (x) + \left(p +
  \frac{1}{2}\right) [M' (x) - M^* (x)]\phi (x)\\ 
  & \quad + x\int^{x}_o S_p [M](x, y) \phi (y) dy 
\end{align*}

Also $B_p [N (x) \psi] = B_p N \mathscr{B}_p \phi = N (x) \phi (x) + x
\int_{o}^x T_p [N] (x,y) \phi (y) dy$ 

Hence the differential equation in $w$ has the form
\begin{multline*}
  \frac{\partial^2 w}{\partial x^2}-\frac{\partial^2 w}{\partial
    y^2} + M(x) \frac{\partial  w}{\partial x }+ \left(P +
  \frac{1}{2}\right) [M' (x) - M^* (x) ] w \\  
  + N(x) w + x \int_o^x \left\{ S_p [M] (x,\xi) + T_p [N] (x, \xi )\right\} w
  (\xi,  y)d \xi = 0 
\end{multline*}

Let $Q_p (x) = N (x) + \left(p + \dfrac{1}{2}\right) \left(M' (x) -
\dfrac{M(x)}{x}\right) 
\in \xi_{*}$ and $R_p (x, \xi) = S_p [M] (x, \xi)+ T_p [N] (x,
\xi)$. We see that Problem \ref{part1:chap4:sec1:prob1} is equivalent to the determination of
the indefinitely differentiable solution  even in $x$ and $y$ the
integro differentiable equation 
$$
\frac{\partial^2 w}{\partial x^2}-\frac{\partial^2 w}{\partial y^2}
M(x) \frac{\partial w}{\partial x} + Q_p (x) w + x \int_o^x R_p (x,
\xi,) w (\xi, y) d \xi = 0 
$$
with the condition $w (0, y)= g(y)$.

It\pageoriginale is easy and classical to transform this problem to a purely
integral equation of Valterra type and the solution is obtained by the
method of successive approximation. It can be verified that all the
conditions of $w$ are verified . A process completely analogous  gives
the solution of Problem \ref{part1:chap4:sec1:prob2}. 

\setcounter{section}{2}
\section{Continuation of $T_p$}\label{part1:chap4:sec3}%sec 3

For $f \in \mathscr{E}_{*}$ fixed we define by induction the functions
$g_n (x, y, \theta )$ as follows 
\begin{multline*}
  g_o (x, y,  \theta  ) = f_1 (z) = \frac{f' (z)}{z} \text{ where } z =
  \sqrt{x^2 \sin^2 \theta  + y^2 \cos^2 \theta }\\ 
  \sin^{2p} \theta  \cos \theta  g_1 (x, y, \theta  ) =
  \frac{\partial}{\partial \theta } \left[\sin^{2p+1} \theta  g_o
    (x,y,\theta  )\right] 
\end{multline*}

In general
$$
\displaylines{\hfill 
  \sin^{2p - 2n +2} \theta  \cos \theta ~ g_n (x, y, \theta  ) =
  \frac{\partial}{\partial \theta } \Big[ \sin^{2p-2n+3} \theta 
    g_{n-1} (x, y,\theta  )\hfill \cr 
    \text{i.e.,}\hfill  g_1 (x, y, \theta  ) = (2p+1)f_1 (z) + (x^2 - y^2 )
    \sin^2 \theta  f_2 (z)\hfill \cr
    \text{where}\hfill 
    f_2 (z) = \frac{f'_1 (z)}{z} \in \mathscr{E}_{*}.\hfill }
  $$

In fact we can prove immediately  by induction on $n$ the following 
\setcounter{lem}{0}
\begin{lem}\label{part1:chap4:sec3:lem1}%lemma 1.
  $g_n (x, y, \theta)$ is a linear combination of $f_1,  f_2, \ldots,
  f_{n+1}$ with coefficients which are polynomials in $x^2, y^2 $ and
  $\sin^2 \theta $ where\pageoriginale $f_n (z) = \dfrac{1}{z} f_{n-1} (z) \in
  \mathscr{E}_{*}$. 
\end{lem}

\begin{coro*}
  The functions $g_n (x, y, \theta )$ are indefinitely  differentiable
  in $x, y$, $\theta $ and even in $x$ and $y$ for $(x, y) \in R^2$ and
  $\theta  \in [0,  \dfrac{\pi}{2}]$. 
\end{coro*}

\begin{defi*}
\begin{multline*}
  T_p^{(n)}[f] (x, y) =
  \frac{(-1)^n \gamma_p}{(2p-1)(2p-3) \cdots (2p - 2n+1)}
  \int^{\frac{\pi}{2}}_0 \\
  \cos^{-2p + 2n} \theta  \sin^{2p-2n+2} \theta 
  g_n (x, y,  \theta  ) d \theta  
\end{multline*}

The integral converges for $\dfrac{2n-3}{2}< \Rep < \dfrac{2n+1}{2}$
and $T^{(n)}_p \in\mathscr{L} (\mathscr{E}_{*}, E)$ where $E$ is the
space of indefinitely differentiable functions of two variables $x, y$
which are also even in $x$ and $y$ with the usual topology of uniform
convergence on every compact subset of $R^2$ of functions together
with their derivatives. It can be verified that $p \to T^{(n)}_{p}$ is
an analytic function for $p$ in the strip in view of the fact
that\hfill \break  $p
\to \dfrac{\gamma _p}{(2p -1) \cdots (2p - 2n + 1)}$ is an entire
function. 
\end{defi*}

\begin{lem}\label{part1:chap4:sec3:lem2}%lemma 2
  In each strip $\dfrac{2n-3}{2}< \Rep < \dfrac{2n -1}{2}$, we have
  \break $T^{(n)}_p = T^{(n-1)}_{p}$  
  \begin{multline*}
  T^{(n-1)}_p  [f] (x, y) = \frac{(-1)^{n-1} \gamma _p}{(2p-1)(2p-3
    )\dot \dot (2p - 2n +3)}\\ 
  \int^{\frac{\pi}{2}}_{0}\cos^{-2p + 2n -
    2} \theta  \sin^{2p-2n+4} \theta  g_{n-1} (x, y,  \theta ) d \theta  
  \end{multline*}
\end{lem}

The\pageoriginale integral can be written as
\begin{multline*}
  \int^{\frac{\pi}{2}}_0 \cos^{-2 p + 2n - 2} \theta \sin \theta
  \sin^{2p - 2n + 3} \theta g_{n - 1} (x, y, \theta) d \theta\\ 
  = \frac{-1}{(2 p - 2n + 1)} \int^{\frac{\pi}{2}}_0 \cos^{-2p +
    2n-1} \theta \frac{\partial}{\partial \theta} \left\{\sin^{+ 2 p
    - 2n + 3} \theta g_{n - 1} (x, y, \theta)\right\} d \theta 
\end{multline*}
(integrating by parts, the integrated part being zero for $\dfrac{2n -
  3}{2} < \Rep< \dfrac{2n - 1}{2})$. The lemma is now evident by the
recurrence formula for $g_n$. Thus we have   

\begin{prop*}
  The function $p \to T_p$ defined for $- \dfrac{3}{2} < \Rep<
  \dfrac{1}{2}$ admits an analytic continuation in the half plane $Rep
  > \dfrac{-3}{2}$, with values in $\mathscr{L} (\varepsilon_*,
  E)$. The explicit definition of $T_p$ is given by the formula for
  $T^{(n)}_P$ for suitable $n$. 
\end{prop*}

Analytic continuation of $T_p$ in the half plane $\Rep<
\dfrac{-3}{2}$ is obtained by exactly similar process. We introduce
functions 
\begin{gather*}
  h_o (x, y, \theta) = f_1 (z), z =  (x^2 \sin^2 \theta + y^2 \cos^2
  \theta)^{\frac{1}{2}},\\ 
  \cos^{-2p-2} \theta \sin \theta h_1 (x, y, \theta) =
  \frac{\partial}{\partial \theta} \bigg[\cos^{-2 p -1} \theta h_o (x,
    y, \theta)\bigg],
\end{gather*}
and by induction,
$$
\cos^{-2 p - 2n} \theta \sin \theta h_n (x, y, \theta) =
\frac{\partial}{\partial \theta} \bigg[\cos^{-2 p - 2n + 1} \theta
  h_{n - 1} (x, y, \theta)\bigg], 
$$

It\pageoriginale is easy to see by induction on $n$ that $h_n (x, y, \theta)$ is a
linear combination of $f_1, f_2, \ldots, f_{n + 1}$, with coefficients
which are polynomials in $x^2, y^2, \cos^2 \theta$ so that the
functions $h_n (x, y, \theta)$ are indefinitely differentiable in $x,
y, \theta$ and are even in $x$ and $y$. If we set 
\begin{multline*}
  (n) T_p f (x, y)  = \frac{(-1)^n \gamma_p}{(2p + 3)(2p + 5) \cdots (2p
    + 2n + 1)}\\ 
  \int\limits_0^{\frac{\pi}{2}} \sin^{2p + 2n + 2} \theta \cos^{-2 p
    -2n} \theta h_n (x, y, \theta) d \theta 
\end{multline*}
for $n \ge 1$, then ${}^{(n)}T_p \in \mathscr{L} (\varepsilon_*, E)$ for $-
n - \dfrac{3}{2} < \Rep < -  n + \dfrac{1}{2}$ and $p \to
^{(n)}T_p$ is analytic in this strip with values in
$\mathscr{L}(\varepsilon_*, E)$ and that for $- n - \dfrac{1}{2} < Rep
< - n + \dfrac{1}{2}, {}^{(n - 1)}T_p =^{(n)}T_p$. Finally we have the 

\begin{theorem*}
  The function $p \to T_p$ initially defined for $- \dfrac{3}{2} < Rep
  < \dfrac{1}{2}$, admits an analytic continuation into an entire
  function with values in $\mathscr{L} (\varepsilon_*, E)$ and we have
  the explicit formula for this continuation. 
\end{theorem*}

In particular, we have, by analytic continuation, $T_p [f] (x, x) = (p
+ \dfrac{1}{2}) f _1 (x)$ for any $p$. 

\begin{coro*}
  For $p \neq - 1, - 2, \ldots,$ the formula
  $$
  B_p A \mathscr{B}_p f (x) = A(x) f(x) + x \int^x_o T_p [A] (x, y) f (y) dy
  $$
  is valid.
\end{coro*}

The\pageoriginale first member $B_p A \mathscr{B}_p f (x)$ is defined and is
analytic except for $p = - 1, -2, \ldots$. The second member is an
entire function of $p$, and the two members are equal for $|Rep| <
\dfrac{1}{2}$ so that the corollary follows. 

\begin{remark*}
  If $A(x)$ is a constant, we have $T_p[A] (x, y) = 0$ and the formula
  of the corollary is equivalent to $B_p \mathscr{B}_p = $ the
  identity. 
\end{remark*}
