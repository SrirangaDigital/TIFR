\thispagestyle{empty}

\begin{center}
{\Large\bf Lectures on}\\[5pt]
{\Large\bf Topics in Mean Periodic Functions}\\[5pt]
{\Large\bf And The Two-Radius Theorem}
\vskip 1cm

{\bf  By}
\medskip

{\large\bf J. Delsarte}
\vfill

{\bf  Tata Institute of Fundamental Research,}

{\bf  Bombay}

{\bf  1961}
\end{center}

\eject


\chapter{Introduction}

\markboth{Introduction}{}

Three\pageoriginale different subjects are treated in these lectures.
\begin{enumerate}[1.]
\item In the first part, an exposition of certain recent work of
  J.L. Lions on the transmutations of singular differential operators
  of the second order in the real case, is given. (J.L. Lions-
  Bulletin soc. Math. de France, $84(1956)$pp. $9-95$) 
\item The second part contains the first exposition of several new
  results on the theory of mean periodic functions $F$, of two real
  variables, that are solutions of two convolution equations: $T_1 * F
  = T_2 * F = 0$, in the case of countable and simple spectrum. These
  functions can be, at least formally, expanded in a series of
  mean-periodic exponentials, corresponding to different points of the
  spectrum. Having determined the coefficients of this development, we
  prove its uniqueness and convergence when $T_1$ and $T_2$ are
  sufficiently simple. The result is obtained by using an
  interpolation formula, in $C^2$, which is analogous to the
  Mittag-Leffler expansion, in $C^1$. 

  The exposition and the proofs given here can probably later, be
  simplified, improved, and perhaps generalized. They should therefore
  be considered as a preliminary account only. 
\item Finally, in the third part, I state and prove the two-radius
  theorem, which is the converse of Gauss's classical theorem on\pageoriginale the
  spherical mean for harmonic functions. The proof is the same as that
  recently published, (Comm. Math. Helvetici, 1959) in collaboration
  with J.L. Lions; it uses the theory of transmutations of singular
  differential operators of the second order, and the fundamental
  theorem of mean-periodic functions in $R^1$. 
\end{enumerate}
\vskip 1cm

\hfill{\large\bf J. Delsarte}
