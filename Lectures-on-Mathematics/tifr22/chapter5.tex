
\part{Topics In Mean-Periodic Functions}\label{part2}%part II

\setcounter{chapter}{0}
\chapter{Expansion of a Mean-periodic Function in
  Series}\label{part2:chap1}%chap 1 

\heading{Introduction.} 

There\pageoriginale is no connection between
Part \ref{part1} and Part 
\ref{part2} of this lecture course. But both these parts are essential for the
two-radius theorem which will be proved in the last part. 

The theory of Mean periodic functions was founded in 1935 (refer to
``Functions Moyenne-periodiques'' by J. Delsarte in Journal de
Mathematique pures et appliques, Vol. 14, 1935). A mean periodic
function was then defined as the solution $f$ of the integral equation 
\begin{equation*}
  \int^b_a K(\xi) f (x + \xi) d \xi = 0 \tag{1}\label{part2:chap1:eq1}
\end{equation*}
where $K$ is `` density '' given by a continuous function $K(x)$,
given on a bounded interval $[a, b]$. It was obvious that the study of
an ordinary differential equation with constant coefficients or of
periodic functions with period $b - a$ was a problem exactly of the
same type as the study of equation (\ref{part2:chap1:eq1}). It was proved in the paper
mentioned above that if $K$ satisfied certain conditions and if $f$
were a solution of (\ref{part2:chap1:eq1}) i.e. $f$ were mean periodic
with respect to 
$K$ then $f$ is developable in a series of exponential
functions\pageoriginale which 
converges towards $f$ uniformly on each interval on which $f$ is
continuous. Equation (\ref{part2:chap1:eq1}) in modern notation is
essentially a convolution equation 
$$
K \ast f = 0
$$
if we replace $f$ by $ \overset{\vee}{f} (\overset{\vee}{f} (x) = f (-x))$.

L.Schwartz in 1947 applied the theory of distributions to the
theory of Mean periodic functions. He calls a mean periodic function
a continuous function $f$ which is a solution of  
$$
\mu \ast f = 0
$$
where $\mu$ is measure with compact support. More generally a
continuous function $f$ is mean periodic with respect to the
distribution $T$ with compact support if $T \ast f = 0$. He has also
given a new definition of mean periodic function which is important
from the topological point of view and the theory developed in his
paper (Annals of Mathematics, $1947$) is complete in the case of one
variable. J.P. Kahane has given a special and delicate development of
the theory which gives connection between Mean periodicity and almost
periodicity. The extension to the case of several variables is
certainly difficult and the first known result in $R^n$ is due to
B. Malgrange. 

For $x \in R^1, \lambda \in C$ we have,
$$
T \ast (e ^{\overset{\vee}\lambda x}) = e^{- \lambda x} M(\lambda)
$$
where\pageoriginale $M (\lambda) = \langle T, e^{\lambda \xi}\rangle$ is the Fourier
- Laplace transform of the distribution $T$. When $T$ is a density $K$
or a measure, $\mu$, 
$$
M(\lambda) = \int^b_a K (\xi) e^{\lambda \xi} d \xi ~~\text{ or }
~M(\lambda) = \int^b_a e^{\lambda \xi} d \mu (\xi) 
$$
respectively and for periodic function of period $a$, we have
$M(\lambda) = e^{a \lambda} - 1$. In the case of the differential
operator with constant coefficients, $M(\lambda)$ is the
characteristic polynomial of the operator. In all these cases
$M(\lambda)$ is an entire function of $\lambda$, of exponential type
and behaves like a polynomial if Re $\lambda$ is bounded and the
(simple) zeros of this function give the exponential functions (or the
exponential polynomials in the case of multiple zeros) which are mean
periodic with respect to the distribution $T$. It is clear that any
linear combination of mean periodic exponentials is also mean periodic
and the first problem is to determine, for any mean - periodic
function $f$, the coefficients of the mean periodic exponentials in
the expansion of the function. 

\section{Determination of the coefficients in the formal
  series}\label{part2:chap1:sec1}  %sec 1

Let $T$ be a distribution with compact support and $M(\lambda)$ its
Fourier-Laplace transform. The set of zeros of $M(\lambda)$ will be
called the \textit{ spectrum } of $M(\lambda)$ and will be denoted by
$(\sigma)$. 

For\pageoriginale the sake of simplicity we suppose that these zeros are simple. Let
$F$ be a function sufficiently regular for the validity of the scalar
product 
$$
\langle T, \int^x_o e^{\lambda (x - \xi)} F(\xi) d \xi
$$

If $T$ is a measure with compact support, $F$ need be only an
integrable function and if $T$ is a distribution with compact support
and order $m, F$ can be any $(m-1)$ times continuously differentiable
function. 

Let $F(x) = e^{\alpha x}$ where $\alpha \in C$ is fixed
$$ 
\displaylines{\hfill 
  \int^x_o e^{\lambda (x - \xi)} e^{\alpha \xi} d \xi =
  \dfrac{e^{\lambda x}-e^{\alpha x}} {\lambda - \alpha}\hfill \cr 
  \text {so that}\hfill \langle T, \int^x_o e^{\lambda (x - \xi)} e^{\alpha \xi} d
  \xi \rangle = \dfrac{ M(\lambda) - M(\alpha) }{\lambda - \alpha} =
  \tau_\alpha (\lambda) \cdot \tau_\alpha (\lambda)\hfill }
$$ 
is an integral function
of exponential type and for $\alpha, \beta \in (\sigma)$, 
\begin{align*}
  \tau_\alpha (\beta) & = 0 ~~\text { if } ~~\beta \neq \alpha\\
  & = M' (\alpha)~~ \text{ if } ~~\alpha = \beta.
\end{align*}

Let $t_\alpha (\lambda) = \dfrac{1}{M' (\alpha)} \tau_\alpha
(\lambda)$ so that $t_\alpha (\beta) =
\delta^\beta_\alpha 
\begin{matrix} 
  & = 1 \text {if } \alpha = \beta~\\ 
  & = 0 \text{ if } \alpha \neq \beta
\end{matrix}$ 

Then $t_\alpha (\lambda)$ is an entire function of exponential type
and therefore by the theorem of Paley-Wiener, there exists a
distribution $T_\alpha$\pageoriginale with compact support whose Fourier-Laplace
transform is 

 $t_\alpha (\lambda)$  i.e. $\langle T_\alpha, e^{\lambda x} \rangle
= t_\alpha (\lambda)$. 

Thus we see that the system of distributions $\{T_\alpha\}_{\alpha \in
  (\sigma)}$ and the functions $\{e^{\alpha x}\}_{\alpha \in
  (\sigma)}$ form a biorthogonal system relative to the distribution
$T$. 

If
\begin{align*}
 F(x) &= \sum_{\alpha \in (\sigma)} c_\alpha e^{\alpha x}
\text{ then } c_\alpha = \langle T_\alpha, F \rangle\\ 
& = \dfrac{1}{M' (\alpha)} \langle T, \int^x_o e^{\alpha (x - \xi)}  F
(\xi)d \xi \rangle.
\end{align*}

For any $F$ which satisfies suitable regularity
conditions (stated at the beginning) we consider the formal expansion 
$$
\displaylines{\hfill 
  F(x) \sim \sum_{\alpha \in (\sigma)}c_\alpha e^{\alpha x}\hfill \cr
  \text{where} \hfill c_\alpha = \dfrac{1}{M'(\alpha)} \langle T,
  \int^x_o e^{\alpha (x - \xi)} F (\xi) d \xi \rangle \hfill }
$$   
and we have immediately two problems.

\label{page68}
\setcounter{prob}{0}
\begin{prob}\label{part2:chap1:sec1:prob1}%problem 1
  If $F$ is mean periodic relatively to a distribution $T$ i.e. if $T
  \ast F = 0$ and if we compute the coefficients $(c_\alpha)$ and
  construct the series $\sum\limits_{\alpha \in (\sigma)} c_\alpha
  e^{\alpha x}$. 
\end{prob}

Then what is the significance of this series? If the series converges
(in some sense) what is the connection between the sum of the series
and the given function $F$? In particular does there exist a
one-to-one correspondence between the mean periodic\pageoriginale function $F$ and
the system of coefficients $(c_\alpha)_{\alpha \in \sigma}?$ 
\label{page69}
\begin{prob}\label{part2:chap1:sec1:prob2}%prob 2
  If $F$ is given one the smallest closed interval which contains the
  support of $T$ it is possible to compute the $c_\alpha$ by the
  formula. In this case is it possible to extend $F$ into a mean
  periodic function on $R' ?$ 
\end{prob}

\section{Examples}\label{part2:chap1:sec2}%sec 2

\begin{exam}\label{part2:chap1:sec2:exam1}%exam 1
  We consider a periodic function $F(x)$ with $[0, 1]$ as the interval
  of periodicity. In this case $T = \delta_1 - \delta_0 (\delta_a $
  being the Direct measure at $a)$ and $T \ast F = F (x + 1) - F(x) =
  0$ and $M(\lambda) = e^{\lambda} - 1$ so that the spectrum
  $(\sigma)$ is given by $= 2 k \pi i, k$ an integer; $M' (\lambda) =
  e^\lambda, M' (\alpha) = 1$ for every $\alpha \in (\sigma)$. We have
  \begin{multline*}
    c_\alpha =\langle T_\alpha, F \rangle = \langle T, \int^x_o e^{\alpha
      (x - \xi)} F (\xi) d \xi\\
    = \int^1_o e^{\alpha (1 - \xi)} F(\xi) d \xi = \int^1_o e^{-2 ik \pi
      \xi} F(\xi) d \xi. 
  \end{multline*}
\end{exam}

This is the classical formula for the coefficients of the Fourier
series for $F(x)$ on the interval $[0, 1]$ and the answers to Problem
\ref{part2:chap1:sec1:prob1} and Problem \ref{part2:chap1:sec1:prob2} are
classical.  

\begin{exam}\label{part2:chap1:sec2:exam2}%exp 2
  Let $T$ be a distribution in $R^1$ with compact support which has
  the property 
\end{exam}

\begin{equation*}
  T \ast F = F (x + 1) - k F(x) - \int^1_0 K(\xi) F (x + \xi)d \xi
  \tag{2}\label{part2:chap1:sec2:eq2} 
\end{equation*}
where\pageoriginale $k$ is a constant $\neq 0$ and $K(x)$ is continuously
differentiable and $T \ast F$ is a function defined by 
$$
T \ast F (x) = \langle T_\xi, F(x + \xi) \rangle = T_\xi. F(x + \xi).
$$

We shall study in this case the spectrum and the formal development in
series of exponentials for a function $F$ mean periodic with respect
to $T$. 

a)~  Let $K$ be continuous in $[0, 1]$.

The definition $T \ast e^{\lambda x} = e^{ + \lambda x} M(\lambda)$ gives
\begin{align*}
  M(\lambda) & = e^\lambda - k - \int^1_o K(\xi) e^{\lambda \xi} d \xi
  \tag{3}\label{part2:chap1:sec2:eq3}\\
  \text{or}\hspace{1cm} e^{-\lambda} M(\lambda) -1 & = - k
  e^{-\lambda} - \int^1_o K  (\xi) e^{\lambda (\xi  - 1)} d \xi\hspace{2cm}
\end{align*}

and if $\lambda = \lambda_o + i \lambda_1$,
$$
\big|e^{- \lambda} M(\lambda) - 1 \big| \leq \big| k \big| e^{-
  \lambda o} + \int^1_o \big| K (\xi) \big| e^{\lambda_o (\xi - 1)} d
\xi 
$$

The second member of this inequality $\to 0$ when $\lambda_0 \to +
\infty$ so that the real parts of the zeros of $M(\lambda)$ are
bounded above. 

Similarly\pageoriginale from $\big|M(\lambda) + k \big| \le e^{\lambda o} +
\int^{1}_o \big| K(\xi) \big| e^{\lambda o \xi} d \xi$, we see that
the real parts of the zeros are bounded below. Thus the spectrum lies in
a vertical band of complex numbers with real parts bounded. 

b)~  Let $K$ be once continuously differentiable in $[0,
  1]$. Integrating by parts the integral in (\ref{part2:chap1:sec2:eq3}), 
$$
M(\lambda) = e^\lambda - k - \frac{K(1)e^\lambda - K(0)}{\lambda} +
\frac{1}{\lambda} \int^1_o K' (\xi) e^\lambda d \xi = e^\lambda - k -
\frac{M_1 (\lambda)}{\lambda} 
$$
where $M_1(\lambda)$ is an entire function of exponential type which
remains bounded when the real part $\lambda_0$ of $\lambda$ remains
bounded. The zeros of $M(\lambda)$ are therefore asymptotic with the
zeros of $e^\lambda - k$ i.e. with $\alpha + 2 \pi ih$, where $\alpha$
is a determination of $\log \alpha$ and $h = 0, \pm 1, \pm 2,
\ldots$. Let $\alpha_h$ be the zero of $M(\lambda)$ near $\alpha + 2
\pi ih$. Then the convergence of the series $\sum\limits^{ +
  \infty}_{h = - \infty} \dfrac{1}{|\alpha + 2 \pi ih|^2}$ implies the
convergence of $\sum\limits_{h = - \infty}^{ + \infty }
\dfrac{1}{|\alpha_h|^2}$. 

c)~ Let $F$ be $(C, 2)$ in $[0, 1]$ and $K$ be $(C, 1)$. We
consider the expansion of $F(x)$ in terms of the exponentials
$\big\{e^{\alpha_h x} \big\}_{h = 0, \pm 1, \pm 2, \ldots} (\alpha_h
\in (\sigma))$, mean periodic with respect to $T$. 
\begin{align*}
  & F(x) \sim \sum_{h - - \infty}^{+ \infty} A_h e^{\alpha n^x}
  \tag{4}\label{part2:chap1:sec2:eq4}\\ 
  \text{where }\hspace{1cm}  A_h = & \frac{1}{M' (\alpha_h)} \langle T,
  \int^x_o e^{\alpha_ h(x - \xi)} F (\xi) d \xi\hspace{2cm}
  \tag{5}\label{part2:chap1:sec2:eq5} 
\end{align*}

We\pageoriginale have \qquad \qquad $M' (\lambda) = e^\lambda - \int^1_o K (\xi)
e^{\lambda \xi} d \xi$ 

since $M (\alpha_h) = 0$,
\begin{align*} 
  e^{\alpha_h} & = k + \int^1_o K (\xi) e^{\alpha_h \xi} d \xi
  \tag{6}\label{part2:chap1:sec2:eq6}\\  
  \text{and} \hspace{1cm} M' (\alpha_n) & = k + \int^1_o (1 - \xi) K
  (\xi)e^{\alpha_{h \xi}} d \xi\hspace{2cm}\\ 
\end{align*}

Integrating by parts integral and observing that $K'(\xi)$ is bounded in
$[0, 1]$ since it is continuous, it is clear that $M' (\alpha_h) = k +
\dfrac{M_2 (\alpha_h)}{\alpha_h}$ where $M_2 (\alpha_h)$ is bounded
when $Re \alpha_h$ is bounded. Also from $(a)$, we know that if $|h|
\to \infty, |\alpha_h| \to \infty$ with $|Re \alpha_h|$ bounded. Hence 
\begin{align*} 
  A_h & = \frac{B_h}{M' (\alpha_h)} \sim \frac{1}{K} B_h ~~\text { as }
  ~~|h| \to \infty \cdots \tag{7}\label{part2:chap1:sec2:eq7}\\ 
  \text{where}\hspace{1cm} B_h & = \langle  T, G_h (x) \rangle, G_h
  (x) = \int^x_o e^{\alpha_h (x - \xi)} F(\xi) d \xi\\
  \langle T, G_h (x) \rangle & = T \ast G_h (0) = G_h (1) - kG_h (0) -
  \int^1_o K(\xi) G_h (\xi) d \xi\\
  B_h & = \int^1_o e^{\alpha_h (1 - \xi)} F(\xi) d \xi - \int^1_o
  \int^\xi_o K(\xi) e^{\alpha_h (\xi - \eta)} F (\eta) d \eta  d\xi
\end{align*}
Now\pageoriginale  
\begin{align*}
\int^1_o e^{\alpha_h (1 - \xi)} F(\xi) d \xi & = -
  \frac{F(1)}{\alpha_h} + \frac{e^{\alpha_h} F(o)}{\alpha _h} +
  \frac{1}{\alpha_h} \int^{1}_o e^{\alpha_h (1 - \xi)} F' (\xi) d
  \xi\\ 
  & = - \frac{1}{\alpha_h} \big[ F(1) - e^{\alpha_{h}} F (0) \big] -
  \frac{1}{\alpha_h^2} \big[F' (1) - e^{\alpha_ h} F' (0) \big]\\ 
  & \hspace{1cm}+ \frac{1}{\alpha_h^2} \int^1_o e^{\alpha_h (1 - \xi)}
  F'' (\xi ) d  \xi\\ 
  \text{and}  \hspace{2.5cm}& \quad \qquad - \int^1_o \int^\xi_o K
  (\xi) e^{\alpha_h (\xi - \eta)} F (\eta) d \eta d \xi\hspace{1cm} \\ 
  =& \frac{1}{\alpha_h} \int^1_o K (\xi) F (\xi) d \xi - \frac{F
    (0)}{\alpha_h} \int^1_o e^{\alpha_h \xi} K (\xi) d \xi\\ 
  & - \frac{1}{\alpha_h} \int^1_o \int^\xi_o e^{\alpha_h (\xi - \eta)}
  K(\xi) F' (\eta) d \eta d \xi\\ 
  =& \frac{1}{\alpha_h} \int^1_o K (\xi) F (\xi ) d \xi - \frac{F
    (0)}{\alpha_h} \int^1_o e^{\alpha_h \xi} K (\xi) d \xi\\ 
  & + \frac{1}{\alpha_h^2} \int^1_o K (\xi) F' (\xi) d \xi - \frac{F'
    (0)}{\alpha_h^2} \int^1_o e^{\alpha_h \xi} K (\xi) d \xi\\ 
  & - \frac{1}{\alpha_h^2} \int^1_o \int^\xi_o e^{\alpha_h (\xi -
    \eta)} K (\xi) F'' (\eta) d \eta d \xi 
\end{align*}
whence\pageoriginale 
\begin{multline*}
  B_h = \dfrac{1}{\alpha_h} \Bigg\{- F (1) + e^{\alpha _h}  F
  (0) + \int^1_o K (\xi) F (\xi) d \xi - F (0) \int\limits^{1}_o
  e^{\alpha_h \xi} K (\xi) d \xi\\
  \frac{1}{\alpha_h 2} \Bigg\{ - F' (1) + e^{\alpha_h} F' (0) +
  \int\limits^{1}_o e^{\alpha_h (1 - \xi)} F'' (\xi) d  \xi\\ 
   + \int^1_o K (\xi) F' (\xi) d \xi - F' (0) \int^1_o e^{\alpha_h
    \xi} K (\xi) d \xi\\ 
   - \int^1_o \int^\xi_o e^{\alpha_h (\xi - \eta)} K(\xi)F '' (\eta)
   d \eta d \xi \Bigg\} 
\end{multline*}
substituting for $e^{\alpha_h}$ from (\ref{part2:chap1:sec2:eq6}) the
first bracket above becomes 

 $- F (1) + k F (0) + F (0) \int^1_o K (\xi) e^{\alpha_h \xi} d \xi +
\int^1_o K (\xi) F (\xi) d \xi$  
 
$- F(0) \int^1_o e^{\alpha_h \xi} K (\xi) d \xi = - (T \ast F)(0) =
0$ since $F$ is assumed to be mean periodic with respect to $T$. Hence
$B_h = o \left(\dfrac{1}{\alpha_h^2}\right)$ provided $F'$ and $F''$ are bounded
in $[0, 1]$. As $|Re \alpha_h|$ remains bounded when $|h| \to \infty$,
from (\ref{part2:chap1:sec2:eq7}) it is immediate that the series
(\ref{part2:chap1:sec2:eq4}) is comparable with 
$\sum\limits^{ + \infty}_{h = - \infty} \dfrac{1}{\alpha_h^2}$ and
therefore converges uniformly and absolutely on each compact subset of
$R^1$ to a continuous function $F_1(x) H(x) = F(x) - F_1 (x)$ is mean
periodic with respect to $T$ and the coefficients $c_{\alpha_h}
(\alpha_h \in \sigma)$ in the expansion for $H(x)$ in mean periodic
exponentials are all zero. Hence $H = 0$ by the uniqueness of
development. (Problem \ref{part2:chap1:sec1:prob1}). 
