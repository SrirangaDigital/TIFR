\part{The Two-Radius Theorem}\label{part3}



\setcounter{chapter}{0}
\chapter{}\label{part3:chap1}

\section{}\label{part3:chap1:sec1}

The\pageoriginale subject of this part is the two-radius theorem, which is the
converse of the classical theorem of Gauss on the spherical mean of
the harmonic functions in $R^n$ 

Let $k_n$ denote the area of the sphere $\sum_{n-1}$ of radius $1$ in
$R^n (\sum_{n-1} = \{x / x = (x_1,x_2,\ldots x_n) \varepsilon R^n, \sum
x^2_i = 1)$. Let $f(x)$ be a function which is $(C,2)$ in $R^n$. The
spherical mean of $f (x)$ on the surface of the sphere with center $x$
and radius $r$ is by definition  
\begin{equation}
  M(x,r) = \frac{1}{k_n r^{n-1}} \int_{S^{n-1} _r (x)} f (\xi) d
  \sigma \tag{1}\label{part3:chap1:sec1:eq1} 
\end{equation}
where $S_r^{n-1} (x)$ is the sphere of center $x$ and radius $r$ in
$R^n, \xi$ is the generic point of the sphere and $d \sigma$ the
element of area of the sphere. We have also   
\begin{equation}
  M (x,r) = \frac{1}{k_n} \int_{\sum_{n-1}} f (x+r ~\overrightarrow{u~})
  d \omega \tag{2}\label{part3:chap1:sec1:eq2} 
\end{equation}
where $d \omega$ is the element of the sphere $\sum_{n-1}$ and
$\overrightarrow{u}$ is the unit\pageoriginale vector at the origin, whose other
extremity describes $\sum_{n-1}$. 

\setcounter{proposition}{0}
\begin{proposition}[Poisson]\label{part3:chap1:sec1:prop1} 
  The function $M (r,x)$ is a solution of the partial differential
  equation 
  \begin{equation}
    \triangle_x \big[M (x,r)\big] = \frac{\partial^2 M} {\partial r^2} +
    \frac{n-1}{r} \quad \frac{\partial M} {\partial r}
    \tag{3}\label{part3:chap1:sec1:eq3}  
  \end{equation}
\end{proposition}

As $\triangle$ is a convolution operator in $R^n$, we have
$$
\triangle_x \big[ M (x,r)\big] = \frac{1}{k_n r^{n-1}}
\int_{S^{n-1}_r (x)} \triangle_{\xi} \big[ f (\xi)\big] d\sigma 
$$

From 2,
\begin{align*}
  \frac{\partial M} {\partial r} & = \frac{1}{k_n} \int_{\sum_{n-1}}
  \frac{\partial}{\partial r} \big[ f (x+r \overrightarrow{u})\big] d
  \omega \\
  & \frac{1}{k_n r^{n-1}} \int_{S_r^{n-1}{_{(x)}}} \frac{d}{d \nu} \big[
    f (\xi)\big] d \sigma 
\end{align*}
where $\nu$ is the exterior normal to $S_r^{n-1}{_{(x)}}$ at the point
$\xi$. By Green's formula, we also have, 
$$
\frac{\partial M}{\partial r} =\frac{1}{k_n^r{^{n-1}}} \int \int
\triangle_\xi \big[ f (\xi)\big] d V = \frac{1}{r^{n-1}} J 
$$
where $J= \dfrac{1}{k_n} \int \int \triangle_\xi \big[ f (\xi)\big] d V$
where\pageoriginale the integral is taken over the volume of the solid sphere in
$R^n$ with $S_r^{n-1}(x)$ as boundary. Then 
$$
\displaylines{\hfill 
  \frac{\partial^2 M} {\partial r^2}  = - \frac{(n-1)} {r^n} J +
  \frac{1}{r^{n-1}} \frac{\partial J}{\partial r}\hfill \cr
  \text{But}\hfill  \frac{\partial J} {\partial r}  = \frac{1}{k_n}
  \int_{S_r^{n-1}{_{(x)}}} \triangle_\xi [f (\xi)] d \sigma \hfill }
$$
since $d V =d \sigma dr$.

Thus
\begin{align*}
  \frac{\partial^2 M} {\partial r^2} + \frac{n-1}{r} \frac{\partial
    M} {\partial r} & = \frac{1} {k_nr^{n-1}} \int\limits_{S_r^{n-1}}
  \triangle_\xi [f(\xi)] d \sigma\\ 
  & =  \triangle_x \big[ M (x,r)\big].
\end{align*}

\begin{remark*}
  For $r=0$, obviously
  $$
  M (x,0) = f (x).
  $$
  
  And \qquad $\big[ \dfrac{\partial M} {\partial r}\big]_{r=0} = \lim_{r=0}
  \bigg\{\dfrac{1}{k_n r^{n-1}} \int \int_{S_r^{n-1}} \triangle_\xi
  \big[ f (\xi)\big] d V$ 
\end{remark*}
  
  The integral in this expression is of order $r^n$
  $$
  \left[ \frac{\partial M} {\partial r}\right]_{r=0} = 0
  $$
  
  Hence\pageoriginale the solution $M (x,r)$ of
  (\ref{part3:chap1:sec1:eq3}), verifies the
  Cauchy conditions 
  $$
  M (x,0) = f (x), \big[ \frac{\partial} {\partial r} M (x,r)\big]_{r=0} =0
  $$

\section{Study of certain Cauchy-Problems}\label{part3:chap1:sec2}%Sec 2

  Let $\mathscr{E}_*$ denote the vector space of  {\em{even}}
  functions indefinitely differentiable in $R$ with the usual topology
  (that induced by $\mathscr{E}$). Let 
  $$
  L = D^2 + \frac{q (x)} {x} D
  $$
be a differential operator where $D = \dfrac{d}{dx}, q \varepsilon
\mathscr{E}_*$ (with $q (0) \neq -\dfrac{n+1}{2},n$ integer $\ge
1$). Then (by the results obtained in part $I$) there exists an
isomorphism $B$ of $\mathscr{E}_*$ onto itself the property 
  $$
  D^2 B = B L \text{ and } B_0 \big[ f (\xi)\big] = f (0)
  $$

\setcounter{prob}{0}
\begin{prob}\label{part3:chap1:sec2:prob1}
  To find a function $F(x,y)$ which is twice differentiable in $R^2$
  and which is a solution of the Cauchy-problem 
  $$
  \displaylines{\hfill 
  \frac{\partial^2 F} {\partial x^2} + \frac{q (x)} {x}
  ~\frac{\partial F}{\partial x} =
  \frac{\partial^2 F} {\partial y^2} + \frac{q (y)} {y} \frac{\partial
    F}{\partial y} \hfill \cr
  \text{with}\hfill F (x,0) = f (x), \qquad f
  \in \mathscr{E}_*\hfill \cr 
  \text{and}  \hfill \left[ \dfrac{\partial F} {\partial y}\right]_{y=0} =
  0\hfill }
  $$
\end{prob}  

Suppose\pageoriginale that there exists a solution $F(x, y)$ which is an even
function of $x$ for $y$ fixed and also an even function of $y$ for $x$
fixed. 

Let 
$$
G(x, y) = B_x B_y [ F( \xi,  \eta)]
$$
where $B_x$ operates on $\xi$ and $B_y$ on $\eta$. $G(x, y)$ is an
element in $\mathscr{E}_*$ as a function of $y$ for $x$ fixed and of
$x$ for $y$ fixed. We have  
\begin{align*}
  \frac{\partial^r G} {\partial x^2} & = D^2_x B_{\xi} B_y[ F( \xi_1,
    \eta)] = B_y D^2_x B_{\xi}[ F( \xi_1,  \eta )]\\ 
  & = B_y B_xL_{\xi}[ F( \xi_1,  \eta )] = B_y B_x L_{\eta}[ F( \xi,
    \eta_1 )]\\ 
  & = B_x B_y L_{\eta}[ F( \xi,  \eta_1 )] = B_x D_y^2 B_{\eta}[ F(
    \xi,  \eta_1 )]\\ 
  & =  D_y^2  B_x B_{\eta}[ F( \xi,  \eta_1 )] = \frac{\partial 2_G}
       {\partial y^2}. 
\end{align*}

Also
\begin{gather*}
  G(x, 0) = B_x B_o [ F( \xi,  \eta )] = B_x [ F( \xi,  \eta )] = B_x
  [f(\xi ) ] = g(x).\\ 
  \left[ \frac{\partial G(x, y)}{ \partial y}\right]_{y =0} = 0.
\end{gather*}

Hence\pageoriginale Problem \ref{part3:chap1:sec2:prob1} is reduced to
the problem of finding a solution of the following Cauchy problem 
\begin{gather*}
  \frac{\partial^2 G}{\partial x^2} = \frac{\partial^2 G}{\partial y^2}\\
  G(x, 0)  = g (x) \in \mathscr{E}_*\\
  \left[ \frac{\partial G}{\partial y} \right]_{y = 0} = 0.
\end{gather*}

It is wellknown that this problem has the unique solution 
\begin{equation*}
  G(x, y) = \frac{1}{2}\left[g(x + y) + g(x - y)\right]
  \tag{4}\label{part3:chap1:sec2:eq4} 
\end{equation*}

Then there exists a unique solution of Problem
(\ref{part3:chap1:sec2:prob1}), defined by  
\begin{equation*}
  F(x, y) = \frac{1}{2} \mathscr{B}_x \mathscr{B}_y [ g(\xi + \eta ) +
    g(\xi - \eta )] \tag{5}\label{part3:chap1:sec2:eq5} 
\end{equation*}
where $\mathscr{B}= B^{-1}$ is an isomorphism of $\mathscr{E}_*$.

\begin{remark*}
  As $g$ is even, $G(x, y)$ is symmetric in $x$ and $y$. Then $F(x,
  y)$ is also symmetric in $x$ and $y$. 
\end{remark*}

\setcounter{defn}{0}
\begin{defn}\label{part3:chap1:sec2:def1}%defini 1.
  For any $f \in \mathscr{E}_*$ let, $M_{x,y}\left[ f(\xi)\right]$
  denotes the solution of problem (\ref{part3:chap1:sec2:prob1}): 
  \begin{equation*}
    M_{xy} [f(\xi )] = \frac{1}{2} \mathscr{B}_x \mathscr{B}_y \left\{
    B_{\xi + \eta} [f]+ B_{\xi - \eta }[f] \right\}
    \tag{6}\label{part3:chap1:sec2:eq6} 
  \end{equation*}
\end{defn}

\begin{prob}\label{part3:chap1:sec2:prob2}%probl 2.
  To\pageoriginale determine $F(x, y), (r \in R^1, r \geq 0)$ which
  satisfies the differential equation 
  $$
  A_x [F(\xi, r)] = L_r [F(x, \rho)],
  $$
  where $A$ is an elliptic differential operator in $R^n$ with
  indefinitely differentiable coefficients, and the Cauchy conditions
  $$
  F(x, 0) = f(x); \left[\frac{\partial F}{\partial r}\right]_{r=0}=0.
  $$
\end{prob}

The existence and uniqueness of the solution of
Problem \ref{part3:chap1:sec2:prob2} defines
the operator $\mathfrak{M}_r $ by 
$$
\mathfrak{M}_r [f(\xi)] = F(x, r).
$$

\begin{remark*}
  By Poisson's theorem, when $A=\Delta$ and $q(r) = n-1$, then the
  solution of Problem \ref{part3:chap1:sec2:prob2} is given by the
  spherical mean.  
\end{remark*}

\begin{proposition}\label{part3:chap1:sec2:prop2}%proposi 2.
  The operator $\mathfrak{M}$ commutes with $A$.
\end{proposition}

Let
$$
G(x, r) =A_x [F (\xi, r)] = A_x \mathfrak{M}_r [f(\xi) ]. 
$$
We have 
\begin{align*}
  A_x [G(\xi, r) ] & = A_x A_\xi [F(\xi_1, r) ]\\
  & = A_x L_r [F(\xi, \rho) ] = L_r A_x [F(\xi, \rho) ]  = L_r
  \left[G(x, \rho) \right]  
\end{align*}

Moreover, 
$$
G (x, 0) = A_x [F(\xi, 0 ]  = A_x [f(\xi)]
$$
and\pageoriginale $\left[\dfrac{\partial G}{\partial r} \right]_{r=0}=0$ since $G(x,
r) \in \mathscr{E}_*$, as a function of $r$ for $x$ fixed. 

Hence
$$
G(x, r) = \mathfrak{M}_r A_\xi [f(\xi_1) ]
$$
and the proposition is proved.

\heading{Iteration of the operator}

For $f \in \mathscr{E}(R^n)$, let $F(x, r) = \mathfrak{M}_r [f(\xi)]$
be the solution of Problem \ref{part3:chap1:sec2:prob2}. By iteration
we consider the Cauchy problem, 
$$
A_x [\mathscr{F}(\xi, s)] = L_s [\mathscr{F}(x, \sigma)] 
$$
with
\begin{gather*}
  \mathscr{F}(x, 0) = F(x, r) \\
  \left[\frac{ \partial \mathscr{F}}{\partial s }\right]_{s=0} = 0
\end{gather*}
in with $r$ is a positive parameter. The solution $\mathscr{F}$ is a
function of $x, r, s$, 
\begin{equation*}
  \mathscr{F}(x | r, s) = \mathfrak{M}_s \left[F(\xi, s)\right] =
  \mathfrak{M}_s \mathfrak{M}_r [f(\xi )] \tag{7}\label{part3:chap1:sec2:eq7} 
\end{equation*}

\begin{proposition}\label{part3:chap1:sec2:prop3}%proposi 3.
  For $x$ fixed in $R^n, ~\mathscr{F} (x | r, s)$ is a solution of
  Problem $1$, i. e. 
  $$
  \displaylines{\hfill 
  L_r \left[\mathscr{F}(x| \rho, s) \right]= L_s
  \left[\mathscr{F}(x|r,\sigma) \right] \hfill \cr
  \text{and}\hfill 
  \mathscr{F}(x | r, 0) = F(x, r)\hfill \cr
  \hfill \left[ \frac{ \partial \mathscr{F}}{\partial s}\right]_{s=0}
  =0. \hfill }
  $$
\end{proposition} 

We\pageoriginale compute
\begin{align*}
  L_r \left[\mathscr{F}(x| \rho, s) \right] &= \mathfrak{M}_s \Big\{ L_r
  \left[F (\xi, \rho) \right] \\
  & = \mathfrak{M}_s \left\{A_\xi \left[F(\xi_1, r) \right] \right\} =
  A_x \mathfrak{M}_s [F(\xi, r) ] \\ 
  & \hspace{2cm}\text{(by Proposition \ref{part3:chap1:sec2:prop2})} \\
  & = A_x \left[\mathscr{F}(\xi | r, s) \right] = L_s
  \left[\mathscr{F}(x | r, \sigma ) \right]. 
\end{align*}

By Definition \ref{part3:chap1:sec2:def1}, we have
$$
\mathscr{F} (x| r, s) = M_{r, s} [F(x, \theta)].
$$

Hence we obtain the formula 
\begin{equation*}
  \mathfrak{M}_s \mathfrak{M}_r = \mathfrak{M}_r \mathfrak{M}_s  =
  M_{r, s}[\mathfrak{M}_0] \tag{8}\label{part3:chap1:sec2:eq8}   
\end{equation*}

We now consider the solution
$$
F(x, r) = \mathfrak{M}_r [f(\xi)]
$$
of\pageoriginale Problem \ref{part3:chap1:sec2:prob2} and suppose that
the it satisfies for $a > 0$ fixed the condition 
\begin{equation*}
  F(x, a) = f(x) \tag{9}\label{part3:chap1:sec2:eq9}
\end{equation*}
for any $x$.

Condition (\ref{part3:chap1:sec2:eq9}) expresses the fact for $r = a$ fixed, the value $F(x,
a)$ reproduces the initial value $f(x)$ of the function for $r = 0$ in
the space $R^n$. 

\begin{remark*}
  If $A = \Delta$, and $L=D^2 + \dfrac{n-1}{r}D$ then $F(x, r) = M(x,
  r)$ and the condition (\ref{part3:chap1:sec2:eq9}) is 'Gauss's
  condition' for the fixed radius a. 
\end{remark*}

In view of (\ref{part3:chap1:sec2:eq9}) and (\ref{part3:chap1:sec2:eq7}) we have
$$
\mathscr{F} (x, a, s) = F(x, s)
$$
where $x$ and $s$ are arbitrary.

By definition (\ref{part3:chap1:sec2:def1}), 
\begin{equation*}
  M_{a, s} [F(x, \theta )] = F(x, s) \tag{10}\label{part3:chap1:sec2:eq10}
\end{equation*}
where the left-hand side the operator $M$ operates on the variable
$\theta$ (Equation (\ref{part3:chap1:sec2:eq10}) thus gives a
'transposition' from $R^n$ to 
$R^1$). Using (\ref{part3:chap1:sec2:eq6}),
(\ref{part3:chap1:sec2:eq10}) becomes,   
$$
2F(x, s) = \mathscr{B}_a \mathscr{B}_s \left\{B_{\alpha+ \sigma}F(x,
\theta) + B_{\alpha - \sigma} [F(x, \theta)]\right\} 
$$
which gives 
$$
2 B_s [F(x, \sigma)] = \mathscr{B}_a \left\{B_{\alpha+ s}F(x, \theta)
+ B_{\alpha - s} F(x, \theta)\right\} 
$$
since\pageoriginale $B$ is the inverse of $\mathscr{B}$.

Setting
$$
K(x, s) = B_s [F(x, \sigma )].
$$
we have
\begin{equation*}
  2 K(x, s) = \mathscr{B}_a \left\{ K(x, \alpha + s) + K(x, \alpha -
  s)\right\} \tag{11}\label{part3:chap1:sec2:eq11} 
\end{equation*}
in which a is fixed.

Since $K(x, s)$ is even in $s$, (\ref{part3:chap1:sec2:eq11}) can be written as 
\begin{equation*}
  2 K(x, s) = \mathscr{B}_a \left\{ K(x, \alpha + s) + K(x, s - \alpha
  ) \right\} \tag*{$(11)'$}\label{part3:chap1:sec2:eq11'} 
\end{equation*}
so that the function $K(x, s)$ is mean periodic in $s$ as the equation
\ref{part3:chap1:sec2:eq11'} is clearly an equation of convolution in $s$. 

\begin{remark*}
  In order to obtain the spectrum, we have to substitute $e^{\lambda
    s}$ in place of $K(x, s)$ in \ref{part3:chap1:sec2:eq11'} which leads to  
  $$
  1 = \mathscr{B}_a [ \cos h \alpha]
  $$
\end{remark*}

\section{The generalized two-radius theorem}\label{part3:chap1:sec3}%sec 3.

Let $a, b \in R^1$; $a, b, > 0 ; a \neq b$
\setcounter{defn}{1}
\begin{defn}\label{part3:chap1:sec3:def2}%defini 2.
  A function $f(x)$ which is $(C, 2)$ in $R^n$ possesses the two
  radius property with respect to the elliptic operator $A$ and
  the\pageoriginale singular operator $L$, if  
  \begin{equation*}
    F(x, a) = F(x, b) = f(x) \tag{12}\label{part3:chap1:sec3:eq12}
  \end{equation*}
  for $x \in R^n$, where
  $$
  F(x, r) = \mathfrak{M}_r [f(\xi)]
  $$
  is the solution of Problem \ref{part3:chap1:sec2:prob2}.
\end{defn}

\begin{remark*}
  If $A = \Delta$ and $L = D^2 + \dfrac{n-1}{r}D$. Condition
  (\ref{part3:chap1:sec3:eq12}) is
  the 'Gauss's condition' for two fixed radii $a$ and $b$. 
\end{remark*}

Now in place of \ref{part3:chap1:sec2:eq11'} we have two equations
\begin{align*}
  2 K(x, s) & = \mathscr{B}_a \left\{K(x, s+\alpha ) + K(x,
  \mathscr{S}-\alpha) \right\} \\ 
  2 K(x, s) & = \mathscr{B}_b \left\{K(x, s +\alpha ) + K(x, s -
  \alpha ) \right\} \\ 
\end{align*}
so that $K(x, s)$ is mean periodic in $s$ with respect to two
distribution, and by the classical result of the theory of mean
periodicity in $R^1$, the elements $\lambda$ in the spectrum
$\sigma(a, b)$ have to satisfy two equations 
\begin{equation*}
  \left.
  \tag{13}\label{part3:chap1:sec3:eq13}
  \begin{aligned}
    1= \mathscr{B}_a [\cos h \lambda \alpha ] \\
    1= \mathscr{B}_b [\cos h \lambda \alpha ] 
  \end{aligned}
  \right\}
\end{equation*}
Now we show that $\lambda = 0$ is a double solution of
(\ref{part3:chap1:sec3:eq13}).

By\pageoriginale definition $D^2 B = BL$.

If $B_s [1] = \varphi(s)~,~D^2[\varphi (s)] = 0$ since $L_s (1) = 0$.
As $\varphi(s)$ is even, $\varphi(s)$ has to be a constant equal to
$\varphi(0)$. But $\varphi(0)=B_0[1]=1$ so that $\varphi(s) \equiv
1$. Since $\mathscr{B}$ is the inverse of $B$, and $B_s[1]=1$, we have
$\underline{\mathscr{B}_s[1]\equiv 1}$. 

As $\mathscr{B}_a [\cos h \lambda \alpha]$ is an even function of
$\lambda$, (\ref{part3:chap1:sec3:eq13}) possesses the double solution
$\lambda =0$.  

We shall hereafter restrict ourselves to the following hypothesis

\noindent
\textit{Hypothesis $(H)$}- The equations
$$
1 = \mathscr{B}_a [ \cos h \lambda \alpha ] = \mathscr{B}_b [ \cos h
  \lambda \alpha] 
$$
have the only double solution $\lambda =0$. In this case $K(x, s)$ is
necessarily of the form 
\begin{equation*}
  K(x,s) = k_1(x) + sk_2(x) \tag{14}\label{part3:chap1:sec3:eq14}
\end{equation*}
by the fundamental theorem of the mean periodic functions in
$R^1$. But $K(x, s)$ is even in $s$ so that $k_2(x) \equiv 0$ in $R^n$
and we have 
\begin{equation*}
  K(x, s) = k_1(x) \tag{15}\label{part3:chap1:sec3:eq15}
\end{equation*}

By inversion, $K(x, s) = B_s [F(x, \sigma)]$ gives
$$
F(x, s) = k_1(x) \mathscr{B}_s [1] = k_1(x)
$$
and\pageoriginale for $s = 0, F(x, s) = f(x)$. Hence
\begin{equation*}
  F(x, s) = f(x) \tag{16}\label{part3:chap1:sec3:eq16}
\end{equation*}
for $x \in R^n$ and $s > 0$.

But $F(x, s)$ is a solution of
$$
A_x \left[F (\xi, s)\right] = L_s \left[F(x, \sigma)\right]
$$
and $F(x, s) = f(x)$ gives necessarily
\begin{equation*}
  A_x \left [f(\xi) \right] = 0 \tag{17}\label{part3:chap1:sec3:eq17}
\end{equation*}
Thus we have in conclusion the

\begin{theorem*}
  The hypothesis $(H)$ and the condition
  $$
  F(x, a)= F(x, b) = F(x),
  $$
  gives
  $$
  F(x, s)= f(x) \text{ for any }s \geq 0 
  $$
  and
  $$
  A_x  \left [f(\xi) \right ]= 0.
  $$
\end{theorem*}

\begin{coro*}
  $A$ is the Laplacian $\Delta$, then $f$ is a harmonic function.
\end{coro*}

\section{Discussion of the hypothesis $H$}\label{part3:chap1:sec4}%Sec 4

In general, the hypothesis $(H)$ is satisfied because the equations
$$
1 = \mathscr{B}_a \left [\cos h  \lambda \alpha \right ] =
\mathscr{B}_b \left [\cos h \lambda \alpha \right ] 
$$
are\pageoriginale a system of two equations with only one unknown $\lambda$. But it
is necessary to investigate certain exceptional values of $a, b (a
\neq b, a> 0, b > 0)$ for which the two-radius Theorem is false. The
question of existence of such exceptional couples $(a, b)$ is
difficult in the general case but in the case of $A = \Delta$ of $R^3$
and $L=D^2 + \dfrac{2}{r}D, (n=3)$, we can assert that there do not
exists any such exceptional couples and the two-radius is always true
in $R^3$. 

\begin{remark*}
  This discussion is completely independent of the function $f$ and
  consequently the couples of exceptional values $(a, b)$ are also
  independent of $f(x)$. 
\end{remark*}

\heading{Results of the discussion in the case $A = \Delta$.}

In this case, $\mathscr{B} = \mathscr{B}_p$ in the notation of part
\ref{part1} with $p= \dfrac{n-2}{2}$. 

We know that $\mathscr{B}\left [\cosh \lambda x \right ] = j_p
(\lambda ix)$ where $j_p(z) = 2^p \Gamma (p+1)$ $z^{-p}J_p(z)$.  The
function which assumes the value $1$ for $z = 0$. In this case the
equations under consideration are 
\begin{equation}
  j_P (\lambda ia) = j_p (\lambda ib) =1 \tag{18}\label{part3:chap1:sec4:eq18}
\end{equation}

Thus it is sufficient to consider in $C^1$, the equation
\begin{equation*}
  j_p (z) = 1 \tag{19}\label{part3:chap1:sec4:eq19}
\end{equation*}
and\pageoriginale to examine whether there exists two roots of
(\ref{part3:chap1:sec4:eq19}) with the same
argument. 

It is easy to see that the set of points $\zeta$ in $C^1$ which are
roots of (\ref{part3:chap1:sec4:eq19}) have for axes of symmetry the two axes $0 \xi$ and $0
\eta $ if $z = \xi + i \eta$. This set is countable and contains the
origin $(\xi = \eta = 0)$. By an intricate discussion based on the
asymptotic expansion of the Bessel functions, it is even possible to
prove that for a given $p$ (i.e. for a given dimension $n$ of the
space) the number of couples of roots of (\ref{part3:chap1:sec4:eq19})
which have the same 
argument is necessarily \textit{finite}. Hence for any dimension $n$
the number of exceptional ratios $\dfrac{a}{b}$ is necessarily
finite. 

In the case $n = 3$ i.e. $p = \dfrac{1}{2}$,
$$
j_p(z) = \frac{\sin z}{z}
$$
(for any odd $n$~, ~ $j_p(z)$ has an expression which depends
algebraically on $z$, $\sin z$ and $\cos z$). 

Thus $\sin z = z$ gives
\begin{align*}
  \sin \xi \cosh \eta & = \xi \\
  \cos \xi \sinh \eta & = \eta.
\end{align*}

Eliminating the hyperbolic functions and the circular functions, we
obtain respectively 
$$
\displaylines{\hfill 
  \frac{\xi^2}{\sin^2 \xi}  - \frac{\eta^2}{\cos^2 \xi} = 1 \hfill \cr
  \text{and}\hfill 
  \frac{\xi^2}{\cosh^2 \eta}  + \frac{\eta^2}{\sinh^2 \eta} = 1 
\hfill }
$$  
so that
\begin{equation*}
\eta  = \pm \xi \cot \xi \left [1-
    \frac{\sin^2
      \xi}{\xi^2}\right]^{\frac{1}{2}}\tag{A}\label{part3:chap1:sec4:eqA} 
\end{equation*}
and 
\begin{equation*}
\xi  = \pm \eta \coth \eta \left [\frac{\sinh^2
      \eta}{\eta^2} - 1 \right ]^ \frac{1}{2}
\tag{B}\label{part3:chap1:sec4:eqB} 
\end{equation*}

The\pageoriginale equations define two real curves in the plane $(\xi,
\eta)$, and 
the roots $z= \xi + i \eta$ of (\ref{part3:chap1:sec4:eq19}) are a
subset of the set points of intersection of two curves. 

As $0 \xi,  0 \eta$ are the axes symmetry of the two curves it is
sufficient to examine the situation for $\xi \geq 0, \eta \geq 0$
(\ref{part3:chap1:sec4:eqA}) 
can be written as 
\begin{equation*}
  \xi = f_1 (\eta) f_2 (\eta) \tag{B$'$}\label{part3:chap1:sec4:eqB'}
\end{equation*}
where $f_1(\eta) = \eta \coth \eta$,  $f_2 (\eta)= \left [
  \dfrac{\sinh^2 \eta}{\eta^2}-1 \right ]^{\dfrac{1}{2}}$. 

We have,
\begin{align*}
  f'_1(\eta) = \coth \eta - \frac{\eta}{\sinh^2 \eta} & = \frac{1}{\sinh^2
    \eta} \left[ \sinh \eta \cosh \eta - \eta \right] \\ 
  & = \frac{1}{ 2\sinh^2 \eta} \left[ \sinh 2\eta - 2 \eta  \right] > 0
\end{align*}
and
\begin{gather*}
  f'_2(\eta) = \frac{1}{f_2 (\eta)} \left[ \frac{\sinh \eta \cosh
      \eta}{\eta^2} -\frac{\sinh^2 \eta}{\eta^3} \right] 
  \frac{\sinh \eta \cosh \eta}{\eta^3 f_2 (\eta)} \left[ \eta - \tanh
    \eta \right] > 0. 
\end{gather*}
 
Thus\pageoriginale $f_1 (\eta)$ and $f_2 (\eta)$ are increasing functions of $\eta >
0$ and so is their product $f_1(\eta) f_2 (\eta)$. 
 $$
 \frac{d \xi}{d \eta}= f_1(\eta) \frac{\sinh \eta \cosh \eta}{\eta^3
   f_2 (\eta)} (\eta- \tanh \eta) + f_2 (\eta) \frac{1}{2 \sinh^2
   \eta}(\sinh 2 \eta - 2 \eta )  
 $$
 and $\xi/\eta = \coth \eta f_2 (\eta)$.
 
 Hence
 $$
 \frac{d \xi}{d \eta} / \frac{\xi}{\eta}=\frac{1}{[\eta f_2 (\eta)]^2}
 \sinh \eta \cosh \eta (\eta - \tanh \eta) + \frac{\sinh 2 \eta - 2
   \eta}{\sinh 2 \eta} 
 $$

 But $\left [ \eta f_2 (\eta)\right]^2 = \sinh^2 \eta - \eta^2$ and finally
 \begin{gather*}
   \frac{d \xi}{d \eta}/ \frac{\xi}{\eta}-1 = \sinh \eta \cosh \eta
   \frac{\eta - \tanh}{\sinh^2 \eta - \eta^2} - \frac{\eta}{\sinh \eta
     \cosh \eta} \\ 
   = \frac{\eta^3 + \eta \sinh ^4 \eta - \sinh^3 \eta \cosh
     \eta}{\sinh \eta \cosh \eta (\sinh^2 \eta - \eta^2)}. 
 \end{gather*} 
 
 As $\dfrac{d}{d \eta}(\eta \sinh \eta - \cosh \eta) = \eta \cosh \eta
 > 0,  \eta \sinh \eta - \cosh \eta $ and therefore $F(\eta)= \eta^3 +
 \sinh^3 \eta ( \eta \sinh \eta - \cosh \eta)$ is an increasing\pageoriginale
 function of $\eta$.  But $F(0)=0$ so that $F (\eta) > 0$ for $\eta >
 0$. Hence  
$$
\frac{d \xi}{d \eta} > \frac{\xi}{\eta} \text{ on B for } \xi > 0, \eta > 0
$$
from which it is clear that there does not exist any point $(\xi,
\eta)$ on \ref{part3:chap1:sec4:eqB}, $\xi > 0, \eta > 0$, the tangent at which passes through
the  origin. Then any chord through the origin can cut the curve only
in one point which is not the origin. But the roots of sinz = z lie on
the curve and it is impossible to find out distinct roots other than
zero which have the same argument. Finally, for $n=3$ there are no
exceptional ratios $\dfrac{a}{b}$ and the two-radius theorem is
completely proved in $R^3$.  

\begin{thebibliography}{11}
\bibitem {1} {N. Bourbaki},\pageoriginale Espaces vectoriels
  topogiques; Paris, Hermann. 
\bibitem {2} {N. Bourbaki}, Integration, Paris, Hermann 1952.
\bibitem {3} {J. Delsarte}, 'Les fonctions moyenne-periodiques',
  J. Math. Pures et Appl. 14 (1935), 403-453 
\bibitem {4} {L. Ehrenpreis}, Mean periodic functions, Amer. J.Math.,
  77 (1995), 293-326. 
\bibitem {5} {J.P. Kahane}, 'Sur quelques problems d'unicite et de
  prolongement relatifs aux fonctions approchables par des sommes
  d'exponentilles', Ann. Inst. Fourier, 5(1954),39-130. 
\bibitem {6} {\underline{~~~~~~~~~~~~}} 'Sur les fonctions
  moyenne-periodiques borne', Ann. Inst. Fourier, 7(1957), 293-314. 
\bibitem {7} {\underline{~~~~~~~~~~~~}} Lectures on mean periodic
  functions, Tata Institute of Fundamental Research, (1958). 
\bibitem {8}  {B. Malgrange}, 'Existence et approximations des
  solutions des equations aux derivees partielles et des 'equations
  deconvolutions', Ann. Inst. Fourier, (1956), 271-356. 
\bibitem {9} {L. Schwartz}, 'Theorie generale des fonctions
  moyenne-periodiques', Ann. of Math. 48(1947), 857-925. 
\bibitem {10} {\underline{~~~~~~~~~~~}} Theorie des distributions,
  Tome I et lI, Paris Hermann. 
\end{thebibliography}
