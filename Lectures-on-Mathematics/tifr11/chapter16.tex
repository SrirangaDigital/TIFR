
\chapter{Weak boundary value problems}\label{chap16}

Let\pageoriginale $E, F$ and $G$ be three Banach spaces and let
$B:E\times F\to G$ be a continuous bilinear map.
\setcounter{section}{16}
\setcounter{prop}{0}
\begin{prop}\label{chap16:prop16.1}
Let $\overrightarrow{S} \in \mathscr{D}_+'(E)$ and $\overrightarrow{T}
\in \mathscr{D}_+'(F)$ be two distributions such that
$\overrightarrow{S} \underset{Rl\; p>a}{\sqsupset} \overrightarrow{S}
(p)$ and $\overrightarrow{T} \underset{Rl\; p>a}{\sqsupset}
\overrightarrow{T} (p)$. Then $\overrightarrow{S} \underset{B}{*}
\overrightarrow{T}$  has a Laplace transform for $Rl p>a$ and the
Laplace transform is precisely $B(\overrightarrow{S}(p),
\overrightarrow{T} (p))$. 
\end{prop}

\begin{proof}
For $Rl\; p > a, e^{-pt} \overrightarrow{S} \in \mathscr{S}'(E)$ and
$e^{-pt} \overrightarrow{T} \in \mathscr{S}'(F)$. In fact, we know
that more is true. For $Rl \; p>a, e^{-pt} \overrightarrow{S} \in
\mathscr{O}_c'(E)$ and $e^{-pt} \overrightarrow{T} \in
\mathscr{O}_c'(F)$. We have $e^{-pt} \overrightarrow{S}
\underset{B}{*} e^{-pt} \overrightarrow{T} =
e^{-pt}(\overrightarrow{S} \underset{B}{*} \overrightarrow{T})$. The
convolutions $\mathscr{D}_+' \times \mathscr{D}_+' \to \mathscr{D}_+'$
and $\mathscr{O}_c' \times \mathscr{O}_c' \to \mathscr{O}_c'$ coincide
on the elements on which both are defined. Hence $e^{-pt}
(\overrightarrow{S} \underset{B}{*} \overrightarrow{T})= e^{-pt}
\overrightarrow{S} \underset{B}{*} e^{-pt} \overrightarrow{T} \in
\mathscr{O}_c'(G)$ for $Rl \; p>a$. Hence $\overrightarrow{S}
\underset{B}{*} \overrightarrow{T}$ has a Laplace transform for $Rl \;
p>a$ and the Laplace transform is the function
$B(\overrightarrow{S}(p), \overrightarrow{T} (p))$. 

We shall consider an application of the above theory to a problem in
differential equations. 

Let $Q'$ be a Banach space and $N$ a Hilbert space with $N \subset Q'$
with a continuous injection. Let $A:N\to Q'$ be a continuous linear
operator. Let $\overrightarrow{f} \in \mathscr{D}_+'(Q')$. To find, if
possible, $\overrightarrow{u} \in \mathscr{D}_+'(N)$ such that
$(\frac{d}{dt} + A) \overrightarrow{u}= \overrightarrow{f}$. This
problem can be restated as follows: To find $\overrightarrow{u} \in
\mathscr{D}_+'(N)$ such that 
$$
(\delta_t' I + \delta_t A) * \overrightarrow{u} = \overrightarrow{f}
$$
where $I : N \to Q'$ is the injection. We have $\delta_t' I$ and
$\delta_t A$ in $\mathscr{D}_+'(\mathscr{L}(N,\break Q'))$.  
\end{proof}
\noindent We\pageoriginale use the properties of the convolution operation to
get new operators.

Let $\mathscr{H}, \mathscr{K}$ and $\mathscr{L}$ be three nuclear
complete spaces with nuclear, complete strong duals. Let $B : E \times
F \to G$ be a continuous bilinear map, $E, F$ and $G$ being three
Banach spaces. Let $U : \mathscr{H} \times \mathscr{K} \to
\mathscr{L}$ be a bilinear map hypocontinuous with respect to the
bounded subsets of $\mathscr{H}$ and $\mathscr{K}$. As explained in
Theorem \ref{chap14:thm14.1} we can define a bilinear map
$\underset{B}{U} : \mathscr{H}(E) \times \mathscr{K}(F) \to
\mathscr{L}(G)$ hypocontinuous with respect to the bounded subsets.

Let $S \overrightarrow{e} \in \mathscr{H}(E)$ with $S \in \mathscr{H},
\overrightarrow{e} \in E$ and $\overrightarrow{T} \in
\mathscr{K}(F)$. We shall find an expression for $S \overrightarrow{e}
\underset{B}{U} \overrightarrow{T}$. For each $\overrightarrow{e} \in
E$, let $B_{\overrightarrow{e}} : F \to G$ be the continuous linear
map $B_{\overrightarrow{e}}(\overrightarrow{f}) =
B(\overrightarrow{e}, \overrightarrow{f})$. The continuous linear map
$B_{\overrightarrow{e}} : F \to G$ allows us to define a continuous
linear map $I_{\mathscr{L}} \varepsilon B_{\overrightarrow{e}}$ of
$\mathscr{L}(G)$. This linear map also we denote by
$B_{\overrightarrow{e}}$. Using the bilinear map $U: \mathscr{H}
\times \mathscr{K} \to \mathscr{L}$, hypocontinuous with respect to
the bounded subsets of $\mathscr{H}$ and $\mathscr{K}$, we can define
a bilinear map $\tilde{U} : \mathscr{H} \times \mathscr{K}(F) \to
\mathscr{L}(F)$ hypocontinuous with respect to the bounded subsets of
$\mathscr{H}$ and $\mathscr{K}(F)$ as explained in Theorem
\ref{chap7:thm7.1}. We have then $S \overrightarrow{e} \underset{B}{U}
\overrightarrow{T} = B_{\overrightarrow{e}} (S \tilde{U}
\overrightarrow{T})$. In fact, if $\overrightarrow{T}$ is of the form
$T \overrightarrow{f}, T \in \mathscr{K}, \overrightarrow{f} \in F$ we
have
\begin{align*}
B_{\overrightarrow{e}} (S \tilde{U}\overrightarrow{T}) &=
B_{\overrightarrow{e}} ((S U T) \overrightarrow{f})=(I_{\mathscr{L}}
\varepsilon B_{\overrightarrow{e}}) (( S U T) \overrightarrow{f})\\
&= (S U T). B_{\overrightarrow{e}} (\overrightarrow{f}) = (S U T) B
(\overrightarrow{e}, \overrightarrow{f}),
\end{align*}
and $S \overrightarrow{e} \underset{B}{U} T \overrightarrow{f} = S U T
B(\overrightarrow{e}, \overrightarrow{f})$. Hence from  the
approximation property for $\mathscr{K}$ it follows that $S
\overrightarrow{e} \underset{B}{U} \overrightarrow{T} =
B_{\overrightarrow{e}} (S \tilde{U} T)$ for any $\overrightarrow{T}
\in \mathscr{K}(F)$. 

$I$ and $A$ are two fixed elements of the vector space $\mathscr{L}(N,
Q')$. $\delta_t' I$ and $\delta_t A$ are distributions with values in
$\mathscr{L}(N, Q')$.
\noindent $\delta_t A * \overrightarrow{u}$\pageoriginale is, by the
formula $S \overrightarrow{e} \underset{B}{U} \overrightarrow{T} =
B_{\overrightarrow{e}} (S\tilde{U} \overrightarrow{T})$, the same as
$\underset{A}{B}(\delta_t * \overrightarrow{u})$ where
$\underset{A}{B}$ is the mapping $I_{\mathscr{D}_+'} \varepsilon
\Gamma_A, \Gamma_A : N \to Q'$ given by $\Gamma_A . n = A n$ for every
$n \in N$. Hence $\underset{A}{B} (\delta_t * \overrightarrow{u}) = A
\overrightarrow{u}$. Similarly $\delta_t' I * \overrightarrow{u} =
\frac{d}{dt} \overrightarrow{u}$. Thus we see that the equation
$(\frac{d}{dt} + A) \overrightarrow{u} = \overrightarrow{f}$ is the
same as $( \delta_t' I + \delta_t A) * \overrightarrow{u} =
\overrightarrow{f}$. What we do first is to look for
$\overrightarrow{G} \in \mathscr{D}_+' (t, \mathscr{L}(Q', N)$ such
that 
\begin{align*}
(\delta_t' I + \delta_t A) * \overrightarrow{G} &= \delta_t I_Q,\\
\text{and }  \overrightarrow{G} * (\delta_t' I + \delta_t
A) &= \delta_t I_N,
\end{align*}
where $I_{Q'}$ and $I_N$ are the identity mappings of $Q'$ and
$N$. Suppose such a $\overrightarrow{G}$ has been found out. Then our
contention is: $\overrightarrow{u} = \overrightarrow{G} *
\overrightarrow{f}$ is the only solution of the equation
$(\frac{d}{dt} + A) \overrightarrow{u} = \overrightarrow{f}$. To prove
this fact, we use the following associativity property of the
convolution. Let $*$ denote the convolution of $\mathscr{D}_+' \times
\mathscr{D}_+' \to \mathscr{D}_+'$. We have for $U, V$ and $W \in
\mathscr{D}_+'$ the following relation: $(U * V) * W = U * (V * W)$. 
\noindent Let $B_1 : L \times M \to P$ and $B_2 : P \times N \to Z$ be
bilinear continuous maps with $L, M, P, N$ and $Z$ Banach spaces. Let
$\alpha_1 : L \times Q \to Z$ and $\alpha_2 : M \times N \to Q$ be
bilinear continuous maps with $L, M$, $N, Q, Z$ Banach spaces. Let
$B_{1(l)} : M \to P$ be the linear map $B_{1(l)}(m) = B_1(l, m)$. Let
$B_{2, B_{1(l)^{(m)}}}: N \to Z$ be the linear map
$B_{2,B_{1(l)^{(m)}}}(n)= B_2(B_{1(l)}(m), n)$. Let $\mu:L \times
M\times N \to Z$ be the trilinear map defined by $\mu(l, m, n) =
B_{2,B_{1(l)^{(m)}}}(n)$. Similarly we can associate with $\alpha_2$
and $\alpha_1$ a trilinear continuous map, say $\gamma : L \times M
\times N \to Z$. If we assume that $\mu = \nu$ we have the following
equality: For\pageoriginale any $\overrightarrow{S} \in
\mathscr{D}_+'(L), \overrightarrow{T} \in \mathscr{D}_+'(M)$ and
$\overrightarrow{U} \in \mathscr{D}_+'(N)$,
$$
(\overrightarrow{S} \underset{B_1}{*} \overrightarrow{T})
\underset{B_2}{*} \overrightarrow{U} = \overrightarrow{S}
\underset{\alpha_1}{*} (\overrightarrow{T} \underset{\alpha_2}{*}
\overrightarrow{U}).
$$
The proof is, in fact, trivial. The convolution $\mathscr{D}_+' \times
\mathscr{D}_+' \to \mathscr{D}_+'$ satisfies the associativity. Hence
from the fact that $\mu = \nu$, we get 
\begin{align*}
(S \overrightarrow{e} \underset{B_1}{*} T \overrightarrow{f})
  \underset{B_2}{*} U \overrightarrow{g} &= (S * T * U) \mu
  (\overrightarrow{e}, \overrightarrow{f}, \overrightarrow{g})\\
&= (S * T * U) \nu (\overrightarrow{e}, \overrightarrow{f},
  \overrightarrow{g})\\
\text{and } S\overrightarrow{e} \underset{\alpha_1}{*} (T
\overrightarrow{f} \underset{\alpha_2}{*} U \overrightarrow{g}) &= (S
* T * U) \nu (\overrightarrow{e}, \overrightarrow{f},
\overrightarrow{g}) 
\end{align*}
for any $S, T, U \in \mathscr{D}_+'$ and $\overrightarrow{e} \in L,
\overrightarrow{f} \in M$ and $\overrightarrow{g} \in N$. 

\noindent Now, since $\mathscr{D}_+'$ has the approximation property,
we get the required associativity formula. Now, $\overrightarrow{u} =
\overrightarrow{G} * \overrightarrow{f}$ is a solution of $(\delta_t'
I + \delta_t A) * \overrightarrow{u} = \overrightarrow{f}$. For
$\hspace{1cm} (\delta_t' I + \delta_t A) * \overrightarrow{G} *
\overrightarrow{f} = \delta_t I_{Q'} * \overrightarrow{f} = I_{Q'}
(\delta_t * \overrightarrow{f}) = I \overrightarrow{f} =
\overrightarrow{f}$.

\noindent Hence $\overrightarrow{u} = \overrightarrow{G} *
\overrightarrow{f}$ is a solution. Suppose $\overrightarrow{v}$ is any
solution of 
$$
(\delta_t' I + \delta_t A) * \overrightarrow{u} = \overrightarrow{f}.
$$
We have $\hspace{2cm} \overrightarrow{G}* \{(\delta_t' I + \delta_t A) *
\overrightarrow{v}\} = \overrightarrow{G}* \overrightarrow{f}$. 
\begin{align*}
\text{But } \overrightarrow{G}* \{(\delta_t' I + \delta_t A)
* \overrightarrow{v}\} &= \{\overrightarrow{G}*(\delta_t' I + \delta_t A)\}
*\overrightarrow{v}\\
&= \delta_t I_N * \overrightarrow{v} = v.\\
\text{Hence }  \overrightarrow{v} &= \overrightarrow{G} *
\overrightarrow{f}.
\end{align*}
Thus $\overrightarrow{G} * \overrightarrow{f}$ is the only solution of
$(\delta_t' I + \delta_t A) * \overrightarrow{u} =
\overrightarrow{f}$. In applying the associativity formula, we should
take note of the following fact: The\pageoriginale obvious bilinear
maps $\mathscr{L}(N, Q')\times N \to Q'$, $\mathscr{L}(Q', N) \times
Q' \to N$ and $\mathscr{L}(Q', N) \times \mathscr{L}(N, Q') \to
\mathscr{L}(N, N)$ and $\mathscr{L}(N, N) \times N \to N$ satisfy the
condition which enables us to conclude that the trilinear maps $\mu$
and $\nu$ corresponding to these are the same. 

Hence the problem $(\frac{d}{dt} + A) \overrightarrow{u} =
\overrightarrow{f}$ will have one and only one solution
$\overrightarrow{u} \in \mathscr{D}_t'(t, N)$, if we can find a
$\overrightarrow{G} \in \mathscr{D}_+' (t, \mathscr{L}(Q', N))$ such
that 
\begin{align*}
(\delta_t' I + \delta_t A) * \overrightarrow{G} &= \delta_t I_{Q'}\\
\text{and }  \overrightarrow{G} * (\delta_t' I + \delta_t
A) &= \delta_t I_N.
\end{align*}
We, in fact, look for a $\overrightarrow{G}$ having a Laplace
transform. If at all such a $\overrightarrow{G}$ exists, it will
satisfy
$$
(pI + A) \overrightarrow{G} (p) = I_{Q'} \quad \text{and} \quad
\overrightarrow{G}(p) (pI + A) = I_N.
$$
Hence, if only we assume that for $Rl \; p> \mathscr{E}_\circ$
($\mathscr{E}_\circ$ some real number), the operator $(p+A)$ is
invertible and that the inverse is majorised uniformly in the half
plane $Rl\; p\geq \mathscr{E}_\circ + \varepsilon$ for any
$\varepsilon > 0)$ there exists a $\overrightarrow{G}(p)$ which is the
Laplace transform of a unique $\overrightarrow{G} \in \mathscr{D}_+'
(t, \mathscr{L}(Q', N))$ satisfying $(\delta_t' I + \delta_t A) *
\overrightarrow{G}=\delta_t I_{Q'},\overrightarrow{G}*(\delta_t' I +\delta_t A)=
\delta_t I_N$. Then the problem $(\frac{d}{dt} + A) \overrightarrow{u}
= \overrightarrow{f}$ has one and only one solution, namely
$$
\overrightarrow{u} = \overrightarrow{G} * \overrightarrow{f}.
$$  
