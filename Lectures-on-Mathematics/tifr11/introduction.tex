\chapter{Introduction}

Different kinds of problems can be put about partial differential equations.
\begin{itemize}
\item[a)] {\bf Local problems}, i.e. problems of regularity of solutions when we
 know the 
degree of regularity of the coefficients and the second member.

\item[b)]{\bf Boundary value problems}. These problems generally have a 
physical origin. As an example we have the first boundary value problem --
the famous Dirichlet problem -- for the Laplacian. We have a bounded domain
$\Omega$ with a smooth boundary in $R^n$; we are given a function $g$ in 
$\Omega$ and a function $h$ on the boundary of $\Omega$. The problem is to 
find a function $f$ in $\bar\Omega$ such that $\Delta f = g$ in $\Omega$ and
$f=h$ on the boundary of $\Omega$. $\left( \Delta = \sum\limits_{i=1}^n 
\dfrac{\partial^2}{\partial x_i^2}\right)$.Another problem is Neumann's
problem for $\Delta$: find $f$ such that $\Delta f=g$ in $\Omega$ and 
$\dfrac {\partial f}{\partial n}=h$ on the boundary. We can also consider the 
problem wherein $f$ is prescribed on a part of the boundary and
$\dfrac {\partial f}{\partial n}$ on the rest of the boundary. Under suitable
assumption on $g$ and $h$ these problems have one and only one solution. The
problems with a physical origin are usually well--posed.
\item [c)]{\bf Mixed problems or initial and boundary value problems}. Let 
$\Omega$ be a bounded domain with a smooth boundary $\mathcal{S}$. We 
consider the problem of heat conduction in $\Omega$. From a physical point of 
view, it is clear that the knowledge of the temperature in $\bar \Omega$ at
time $0$ and that of the temperature at the boundary at every time $t>0$
should completely determine the temperature in $\bar \Omega$ at any time $t$.
The corresponding problem is this: given a function $u_0(x)$ in $\bar \Omega$
and a function $h(x, t)$,  $t\geq0$, $x\in S$, find a function $u(x,t)$ such
that
     
\begin{itemize}
\item [i)] $\dfrac{\partial u(x, t)}{\partial t}= \Delta u(x,t)$
\item [ii)] $u(x,0)= u_0(x)$\; (initial condition)
\item [iii)] $u(x,t) = h(x,t)$ for every $t>0$ and $x\in S$ (boundary 
condition)
\end{itemize}
Another problem of this type arises when we know the initial temperature of the
body and the amount of heat that flows across the boundary at every subsequence
moment. The problem is to find $u(x,t)$ such that

\begin{itemize}
\item [i)] $\dfrac{\partial u(x,t)}{\partial t} = \Delta u(x,t)$
\item [ii)] $u(x,0) = u_0(x)$
\item [iii)] $\dfrac{\partial u(x,t)}{\partial n}=h(x,t)$ $(t>0, x\in S)$.
\end{itemize}
We shall formulate these problems, or rather weaker versions of these,
in the framework of spaces of distributions and solve them.
\end{itemize}
