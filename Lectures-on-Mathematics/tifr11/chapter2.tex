
\chapter{Vector valued Distributions (Contd.)}\label{chap2}

Let\pageoriginale $E$ and $F$ be two locally convex Hausdorff spaces
and $u:E \to F$ be a continuous linear map. If
$\overrightarrow{T}:\mathscr{D} \to E$ is an $E$-valued distribution
$u \circ \overrightarrow{T}:\mathscr{D} \to F$ is an $F$-valued
distribution. $u \circ \overrightarrow{T}$ is called the image of
$\overrightarrow{T}$ by $u$. The distribution $u\circ
\overrightarrow{T}$ has at least as simple properties as
$\overrightarrow{T}$. For example, if $\overrightarrow{T}$ is of
finite order, $u \circ \overrightarrow{T}$ is of finite order. The
support of $u\circ \overrightarrow{T}$ is contained in that of
$\overrightarrow{T}$. In particular, if $\overrightarrow{T}$ has a
compact support, $u\circ \overrightarrow{T}$ has a compact support. If
$\overrightarrow{T}$ is given by a function
$\overrightarrow{f}:R^N\to E$, then $u\circ \overrightarrow{T}$ is
given by the function $u\circ \overrightarrow{f}:R^N\to F$. This
follows immediately from the equality
$$
u\left(\int\limits_{R^N}\overrightarrow{f}(x) \varphi(x)\,dx\right) =
\int\limits_{R^N} u(\overrightarrow{f}(x)) \varphi(x)\,dx.
$$
Every distribution $\overrightarrow{T}:\mathscr{D} \to E$ is a
continuous image of the identity distribution for $\overrightarrow{T}
= \overrightarrow{T} \circ I$. In this sense the identity distribution
is the worst possible distribution. Suppose $\overrightarrow{T} \in
\mathscr{D}' (E)$ and $\overleftarrow{e}' \in E', E'$ denoting the
dual of $E$. Write for $\varphi\in\mathscr{D}$, $\langle\varphi
|\overrightarrow{T}|=\overrightarrow{T}(\varphi)$ and $\langle
\varphi |\overrightarrow{T}| \overleftarrow{e}'\rangle =
\overleftarrow{e}' (\overrightarrow{T} (\varphi))$. The mapping
$(\varphi,\overrightarrow{T},\overleftarrow{e}')\to \langle \varphi
|\overrightarrow{T}| \overleftarrow{e}'\rangle$ is a trilinear form
on $\mathscr{D}\times\mathscr{D}'(E)\times E'$. $\overleftarrow{e}' \circ
\overrightarrow{T}$ is a scalar distribution and we denote this
distribution by $|\overrightarrow{T}| \overleftarrow{e}'\rangle$.

\setcounter{section}{2}
\begin{prop}\label{chap2:prop2.1}
If $\overrightarrow{T}$ is an $E$-valued distribution, the map from
$E'$ to $\mathscr{D'}$ which takes $\overleftarrow{e}'$ into
$|\overrightarrow{T}|\overleftarrow{e}'\rangle$ is the transpose of
the map $\overrightarrow{T}:\mathscr{D}\to E$.
\end{prop}

\begin{proof}
We have\pageoriginale
\begin{align*}
|\overrightarrow{T}|\overrightarrow{e}'\rangle (\varphi) &=
\langle\varphi |\overrightarrow{T}|\overleftarrow{e}'\rangle \quad
\text{for every} \quad \varphi \in \mathscr{D}\\
& = \langle\overrightarrow{T}(\varphi), \overleftarrow{e}'
\rangle_{E,E'}\\
& =
\langle\varphi,\overset{t}{\overrightarrow{T}}\overleftarrow{e}'\rangle. 
\end{align*}
This proves the proposition.
\end{proof}

Let $E$ be a locally convex Hausdorff topological vector space, we
denote by $E_c'$ the dual of $E$ endowed with the topology of uniform
convergence on convex, compact, stable subsets of $E$. By Mackey's
theorem (Bourbaki $EVT$, Tome 2, Chapter IV, Theorem 2), the dual of
$E_c'$ is identical with $E$. Moreover if $E$ and $F$ are $ELC$ and
$u:E \to F$ is a continuous linear map, the transpose ${}^tu: F_c' \to
E_c'$ is continuous, because $u$ maps convex, compact, stable subsets
of $E$ into convex, compact, stable subsets of $F$.

Now, suppose $\overrightarrow{T}:\mathscr{D}\to E$ is an $E$-valued
distribution ${}^t{\overrightarrow{T}}: E_c' \to \mathscr{D}'$
is a continuous map for $\mathscr{D}'=\mathscr{D}_c'$. Conversely, we
have 
\begin{prop}\label{chap2:prop2.2}
If $u:E_c' \to \mathscr{D}'$ is a continuous linear map, it is the
transpose of a uniquely determined $E$-valued distribution.
\end{prop}

\begin{proof}
Let $u:E_c' \to \mathscr{D}'$ be a continuous linear map. Then ${}^tu:
(E_c')_c' \leftarrow (\mathscr{D}')_c'$ is a continuous linear
map. But $(\mathscr{D}')_c'=\mathscr{D}$, and $(E_c')_c'$ has a
topology finer than the topology of $E$. [$(E_c')_c'$ is algebraically
  the same as $E$ by Mackey's theorem]. To prove that the topology of
$(E_c')_c'$ is finer than that of $E$, we first remark that the
initial topology on $E$ is the topology of uniform convergence on
equicontinuous subsets of $E'$. To prove our assertion, we have only
to show that any equicontinuous subset of $E'$ is contained in a
convex, compact, stable subset\pageoriginale of $E_c'$. Let $A$ be any
equicontinuous subset of $E'$. Let $\overarc{A}$ be the convex, weakly
closed stable envelope of $A$. $\overarc{A}$ is then weakly compact
and equicontinuous. But on equicontinuous subsets the topology of
compact convergence and the weak topology coincide. Hence
$\overarc{A}$ is a compact subset of $E_c'$. Since the topology of
$(E_c')_c'$ is finer than that of $E$, $t_u:E\leftarrow\mathscr{D}$ is
also continuous. This proves our proposition, as ${}^t(t_u)=u$.    
\end{proof}
The above proposition shows that the vector spaces
$\mathscr{L}(\mathscr{D}, E)$ and $\mathscr{L}(E_c',\mathscr{D}')$
are algebraically isomorphic. 
