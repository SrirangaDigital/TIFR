\part{REPRESENTATIONS OF SEMI-GROUPS}

\chapter{Representations of semi-groups}\label{chap17}

\setcounter{section}{17}
\setcounter{definition}{0}
\begin{definition}\label{chap17:def17.1}
A\pageoriginale semi-group is a set $G_+$ with a binary associative
law of composition, having an identity element. That is to say, there
is defined a mapping $(x, y) \to x.y$ of $G_+ \times G_+$ in $G_+$
satisfying the following conditions: 
\begin{itemize}
\item [1)] $x.(y.z) = (x.y).z$ for every $x, y, z \in G_+$.
\item [2)] There exists an element $e \in G_+$ such that $e.x = x$ and
  $x.e=x$ for every $x \in G_+$. 
\end{itemize}
\end{definition}

\begin{definition}\label{chap17:def17.2}
A topological semi-group $G_+$ is a Hausdorff topological space with a
semi-group structure such that the mapping $(x, y) \to x.y$ of $G_+
\times G_+\to G_+$ is continuous. In this section, we deal only with locally
compact semi-groups. 
\end{definition}

\begin{definition}\label{chap17:def17.3}
A measure $\mu$ on $G_+$ is said to be summable if
$\int\limits_{G_+}|d\mu| < \infty$. One knows that, if $\mu$ is a
summable measure, for any continuous bounded complex valued function
$f$ the integral $\int\limits_{G_+} f(x) d \mu(x)$ can be defined. We
agree to denote this by $\mu (f)$.
\end{definition}

\begin{definition}\label{chap17:def17.4}
Let $\mu$ and $\nu$ be summable measures on $G_+$. The direct image of
the measure $\mu \otimes \nu$ on $G_+ \times G_+$ by the mapping $(x,
y) \to x.y$, which trivially exists, is defined to be the convolution
of the measures $\mu$ and $\nu$ and is denoted by $\mu * \nu$. 

If $\varphi$ is any continuous bounded function on $G_+$, the function
$\psi : G_+ \times G_+ \to C$ defined by $\psi (s, t) = \varphi(st)$
is a continuous bounded function. Hence the integral
$$
\int\limits_{G_+\times G_+}\psi d(\mu \otimes \nu)=
\int\limits_{G_+}\int\limits_{G_+} \psi(s, t) d \mu (s) d \nu (t)
$$\pageoriginale
has a meaning. We have the equality $\mu * \nu (\varphi)=
\int\limits_{G_+} \int\limits_{G_+} \varphi(st)d \mu(s)d \nu (t)$.
\end{definition}

\begin{definition}\label{chap17:def17.5}
For an integrable (or summable) measure $\mu, \int\limits_{G_+}
|d\mu|$ is defined to be the norm of $\mu$ and is denoted by
$\parallel \mu \parallel$.

One knows that if $\mu$ and $\nu$ are summable measures on $G_+, \mu *
\nu$ is also summable and that $\parallel \mu * \nu \parallel \leq
\parallel \mu \parallel \parallel \nu \parallel$ (Refer to Elements de
Mathematique, Integration, by N. Bourbaki). 
\end{definition}

\noindent {\bf Strict convergence.}

\begin{definition}\label{chap17:def17.6}
A sequence of measures $\{ \mu_j\}$ is said to strictly converge to
$0$ if $\mu_j(\varphi) \to 0$ for every fixed $\varphi$, continuous
with compact support and if there exists a positive real number
$\varepsilon (K)$ corresponding to each compact set $K$ such that
$\int\limits_{[ K} |d\mu_j| \leq \varepsilon (K)$ independent of $j,
  ([K$ is the complement of $K)$, with $\varepsilon(K) \to 0$
    according as the `filtrant set' of compact subsets of $G_+$
    ordered by inclusion, that is to say, given any $\varepsilon > 0$
    there exists a compact $K$ such that for any compact set $\Gamma$
    of $G_+$ with $\Gamma \supset K$ we have $\varepsilon
    (\Gamma)<\varepsilon$. In fact, it is sufficient if there exists a
    compact set $K$ such that $\varepsilon (K) < \varepsilon$.  
\end{definition}

\setcounter{section}{17}
\setcounter{lemma}{0}
\begin{lemma}\label{chap17:lem17.1}
If $\{\mu_j\}$ is a sequence of measures converging strictly to $0$,
for any fixed continuous bounded function $\varphi, \mu_j (\varphi)
\to 0$.
\end{lemma}
\begin{proof}
Let $\alpha$ be any continuous function with $0 \leq \alpha(x) \leq 1$
for every $x \in G_+$ and with compact support. We have $\mu_j
(\varphi)= \mu_j(\alpha \varphi) + \mu_j((1-\alpha)\varphi)$. We may
assume $\varphi \nequiv 0$. Since $\mu_j \to 0$ strictly, given any
$\varepsilon > 0$ we can find a compact set $K$ such that
$\int\limits_{[K} |d\mu_j| <
  \frac{\varepsilon}{2\parallel\varphi\parallel}$ where
  $\parallel\varphi\parallel= \underset{x\in
    G_+}{\sup}|\varphi(x)|$. For $\alpha$, choose a continuous
  function with compact support which is $1$ on $K$ and with $0 \leq
  \alpha(x)\leq 1$. Then $\alpha . \varphi$ is a continuous function
  with a compact support. Hence we can find a $j(\varepsilon)$ such
  that for $j \geq j(\varepsilon)$ we have $|\mu_j(\alpha \varphi)| <
  \frac{\varepsilon}{2}$. We\pageoriginale have $\mu_j((1-\alpha)
  \varphi) = \int\limits_{[K} (1-\alpha) \varphi d \mu_j$, since
    $1-\alpha = 0$ on $K$. Hence 
$$
|\mu_j ((1-\alpha)\varphi)| \leq \parallel \varphi \parallel
\int\limits_{[K} |d \mu_j| \leq \frac{\parallel \varphi \parallel.
 \varepsilon}{2\parallel\varphi\parallel}\; \text{for}\; j \geq j(\varepsilon).
$$   
Hence $|\mu_j(\varphi)| \leq | \mu_j (\alpha
\varphi)|+|\mu_j((1-\alpha)\varphi)| \leq \frac{\varepsilon}{2} +
\frac{\varepsilon}{2} = \varepsilon$ for $j \geq j(\varepsilon)$.
\end{proof}

\begin{lemma}\label{chap17:lem17.2}
If $\{ \mu_j\}, \{ \nu_j\}$ are two sequences of measures strictly
converging to $\mu$ and $\nu$ respectively, the sequence $\{\mu_j *
\nu_j\}$ strictly converges to $\mu * \nu$. 

We shall first show that if $\{\mu_j\}$ and $\{\nu_j\}$ tend to $0$
strictly, the sequence $\{\mu_j * \nu_j\}$ tends to $0$ strictly. If
$\varphi$ is a function in $\mathscr{C} (G_+ \times G_+)$ of
the type $\varphi(x, y) = \psi(x) \eta(y)$ where $\psi \in
\mathscr{C}(G_+), \eta \in \mathscr{C}(G_+)$ we have 
$$
(\mu_j \otimes \nu_j)(\varphi) = \mu_j (\psi). \nu_j(\eta).
$$
($\mathscr{C}(G_+)$ denotes the set of complex valued functions on
$G_+$ with compact support). Hence for a $\varphi$ of the above
mentioned form, $\mu_j \otimes \nu_j) (\varphi) \to 0$. The linear
combinations of elements of the form $\varphi(x, y)= \psi(x) \eta (y)$
form a dense subset of $\mathscr{C}(G_+ \times G_+)$. Since
$\{\mu_j\}$ and $\{\nu_j\}$ are strictly convergent sequen\-ces of
measures, $\{\mu_j\}$ and $\{\nu_j\}$ are bounded sequen\-ces of
measures and hence $\{\mu_j \otimes \nu_j\}$ is a bounded sequence of
measures on $G_+ \times G_+$. Hence the set consisting of the elements
$\{ \mu_j \otimes \nu_j\}$ is an equicontinuous set in the dual of
$\mathscr{C}(G_+ \times G_+)$. On this set the topology of simple
convergence on a dense subspace of $\mathscr{C}(G_+ \times G_+)$ by
Ascoli's Theorem. Hence for every $\varphi \in \mathscr{C}(G_+ \times
G_+)$ we have $\mu_j \otimes \nu_j(\varphi) \to 0$ as $j \to \infty$.
\end{lemma}

We shall prove the strict convergence of $\mu_j \otimes \nu_j$ to
$0$. We have already proved the `vague' convergence of $\mu_j \otimes
\nu_j$ to $0$, that is to say, for every fixed $\varphi \in
\mathscr{C}(G_+ \times G_+), \mu_j \otimes \nu_j(\varphi)$ tends to
$0$. To prove the strict convergence of $\{\mu_j \otimes \nu_j\}$ it
is sufficient to prove that there exist constants $\varepsilon (H
\times K)$ for compact sets of the form $H \times K$, $H$ and $K$
being compact\pageoriginale in $G_+$ such that given any $\varepsilon
> 0$ there exists a compact set $H \times K$ with $\varepsilon (H
\times K) < \varepsilon$ and $\int\limits_{[(H\times K)} |d \mu_j
  \otimes \nu_j | < \varepsilon (H\times K)$. Now 
\begin{align*}
\int\limits_{[(H\times K)} |d \mu_j \otimes \nu_j |&\leq
  \int\limits_{G_+\times[K} |d \mu_j \otimes \nu_j | +
    \int\limits_{[H\times G_+} |d \mu_j \otimes \nu_j |\\
&\leq A \int\limits_{[K} |d \nu_j| + B \int\limits_{[H}|d \mu_j|\\
&\leq A \varepsilon (K) + B \varepsilon (H),
\end{align*}
where $A \geq \int\limits_{G_+}|d\mu_j|$ and $B \geq
\int\limits_{G_+}|d \nu_j|$. (Such real numbers $A$ and $B$ exist). If
we choose $K$ and $H$ in such a way that
$\varepsilon(K)<\frac{\varepsilon}{2A}$ and $\varepsilon (H) <
\frac{\varepsilon}{2B}$ we have 
$$
\int\limits_{[H\times K}|d \mu_j \otimes \nu_j| <
  \frac{\varepsilon}{2} + \frac{\varepsilon}{2} = \varepsilon .
$$
Thus $\varepsilon (H \times K)=A\varepsilon(K) + B \varepsilon(H)$ are
constants satisfying 
$$
\int\limits_{[(H\times K)}|d\mu_j \otimes \nu_j| \leq \varepsilon(H
  \times K)
$$
and given any $\varepsilon > 0$ there exists $H \times K$ such that
$\varepsilon(H \times K) < \varepsilon$. We have thus proved that
$\mu_j \otimes \nu_j \to 0$ strictly. 

Let $u : E \to F$ be a continuous map of a locally compact space $E$
into a locally compact space $F$. Let $\{\lambda_j\}$ be a sequence of
summable measures strictly converging to $0$ on $E$. The direct image
of the $\lambda_j's$ by $u$ converges to $0$ strictly. First of all $u
\lambda_j \to 0$ vaguely. For, if $\varphi$ is any continuous function
with compact support on $F, u * \varphi$ is a continuous bounded
function on $E$ and $\lambda_j (u * \varphi) \to 0$. Hence $u
\lambda_j(\varphi)=\lambda_j(u * \varphi) \to 0$. It is sufficient to
prove the existence of constants $\varepsilon(H)$ for compact sets $H$
of the form $H=u(K)$, $K$ being compact in $E$ such that given any
$\varepsilon > 0$ there exists an $\varepsilon(H)$ with
$\varepsilon(H) < \varepsilon$ and $\int\limits_{[H}|d u \lambda_j|
  \leq \varepsilon(H)$. Now\pageoriginale 
$$
\int\limits_{[H}|d u \lambda_j| \leq \int\limits_{\bar{u}^1([H)} |d
    \lambda_j | \leq \int\limits_{[K} |d \lambda_j | \leq
      \varepsilon(K).
$$
Given any $\varepsilon > 0$ we know that there exists a compact set
$K$ in $E$ such that $\varepsilon (K) < \varepsilon$. We have already
proved that if $\mu_j$ and $\nu_j$ tend to $0$ strictly $\mu_j \otimes
\nu_j$ tends to $0$ strictly. $\mu_j * \nu_j$ being the direct image
of $\mu_j \otimes \nu_j$ by the map $(x, y) \to x.y$ of $G_+ \times
G_+$ in $G_+$ we have $\mu_j * \nu_j \to 0$ strictly.

Now suppose $\mu_j \to \mu$ and $\nu_j \to \nu$ strictly. Then $\mu_j
- \mu$ and $\nu_j - \nu$ are sequences of measures strictly converging
to $0$. Hence $(\mu_j - \mu) * (\nu_j - \nu)$ converges to $0$
strictly. We have
$$
\mu_j * \nu_j - \mu * \nu = (\mu_j - \mu) * (\nu_j - \nu) + \mu
*(\nu_j - \nu) + (\mu_j - \mu) * \nu
$$
Now $(\mu_j - \mu) * (\nu_j - \nu) \to 0$ strictly. To complete the
proof of the lemma we have only to prove the following: if $\Gamma_j$
is a sequence of measures strictly converging to zero and $\mu$ is a
fixed summable measure, then $\mu * \Gamma_j$ and $\Gamma_j * \mu$
tend to zero strictly. For this it is enough to prove that $\mu
\otimes \Gamma_j$ and $\Gamma_j \otimes \mu$ tend to zero
strictly. (see the general considerations given above). This is proved
the same way we proved that $\mu_j \otimes \nu_j \to 0$ strictly if
$\mu_j \to 0$ and $\nu_j \to 0$ strictly, using the following fact: if
$\mu$ is a summable measure, then $\int\limits_{[K}|d \mu| \to 0$
  following the filtered set of compact subsets $K$. 

Let $\lambda$, $\mu$ and $\nu$ be three summable measures on
$G_+$. Then $\lambda * \mu * \nu$ is defined to be the direct image of
the measure $\lambda \otimes \mu \otimes \nu$ by the map $(x, y, z)
\to x.y.z$ of $G_+ \times G_+ \times G_+$ in $G_+$. It is easily seen
that 
$$
\lambda * (\mu * \nu) = \lambda * \mu * \nu = ( \lambda * \mu) * \nu .
$$
Also\pageoriginale we have $\delta_x * \delta_y = \delta_{xy}$ and
$\delta_e * \mu = \mu * \delta_e=\mu$ where $\delta_x$ is the unit
mass (Dirac measure) at $x$ and $e$ is the identity element of $G_+$. 
\bigskip

\noindent {\bf Representation of semi-groups.}

\setcounter{definition}{5}
\begin{definition}\label{chap17:def17.6a}
Let $G_+$ denote a locally compact semi-group. Let $E$ be a complete
$E L C$. $A$ {\bf representation} of $G_+$ in $E$ is a map $U : G_+
\to \mathscr{L}_s(E, E)$ satisfying the following conditions:
\begin{itemize}
\item [(i)] $U(x.y)=U(x) \circ U(y), U(e)=I$ the identity map of $E$;
\item [(ii)] $U : G_+ \to \mathscr{L}_s(E, E)$ is continuous, and
\item [(iii)] For every compact set $K$ in $G_+$ the set of operators
  $\{U(k), k \in K\}$ is an equicontinuous set of linear maps of $E$
  in $E$. 
\end{itemize}

Property (iii) is called the property of local equicontinuity. We
shall consider here only representations satisfying the following
stronger condition of global equicontinuity. 

\noindent (iii)$'$ The set of operators $\{U(x), x \in G_+\}$ is an
equicontinuous set of linear maps of $E$ in $E$. 

Let $\mathscr{M}_{G_+}$ denote the set of all summable measures on
$G_+$. It is an algebra under the operations of addition and
convolution. 
\end{definition}
 \begin{lemma}\label{chap17:lem17.3}
The representation $U : G_+ \to \mathscr{L}_s(E, E)$ can be extended
into a map, which also we denote by $U$, of $\mathscr{M}_{G_+}$ in
$\mathscr{L}_s(E, E)$. When $\mu = \delta_x$, the unit mass at $x$,
$U(\mu)$ will be $U(x)$. 
 \end{lemma}

\begin{proof}
Since $\{ U(x), x \in G_+\}$ is an equicontinuous set of operators, it
is also a bounded set in $\mathscr{L}_s(E, E)$. Since $\mu$ is
summable, the integral $\int\limits_{G_+}U(x) d \mu (x)$ exists and is
an element of $(\mathscr{L}_s(E, E))^\wedge$, the completion of
$\mathscr{L}_s(E, E)$. (Refer to Integration, by
N. Bourbaki). Since\pageoriginale $E$ is complete, $(\mathscr{L}_s(E,
E))^\wedge \subset \wedge_s(E, E)$ where $\wedge_s(E, E)$ is the space
of all linear maps of $E$ in $E$ with the topology of simple
convergence. Now the set of elements $\{U(x)/x \in G_+\}$ is an
equicontinuous subset $\mathscr{U}$ of $\mathscr{L}_s(E, E)$. Let
$\overset{\frown}{\mathscr{U}}$ be the convex stable closed envelope of
$\mathscr{U}$ in $\wedge_s(E, E)$. $\overset{\frown}{\mathscr{U}}$ is also
equicontinuous and hence $\overset{\frown}{\mathscr{U}} \subset \mathscr{L}(E,
E)$. We have
$$
U(\mu) = \int\limits_{G_+} U(x) d \mu(x) \in
\overset{\frown}{\mathscr{U}} \int\limits_{G_+} d |\mu | \subset
\mathscr{L}_s (E, E).
$$
Hence $U(\mu) \in \mathscr{L}_s(E, E)$ for every $\mu \in
\mathscr{M}_{G_+}$. Trivially $U(\delta_x)=U(x)$.

If $\overrightarrow{e} \in E$, we have
$$
U(\mu) \overrightarrow{e} = \int\limits_{G_+} U(x) \overrightarrow{e}
d \mu (x),
$$
since the map $U \to U \overrightarrow{e}$ is continuous.

For any fixed $\theta \in \mathscr{L}_s(E, E)$, the maps $\Gamma
\to \Gamma \circ \theta$ and $\Gamma \to \theta \circ \Gamma$ of
$\mathscr{L}_s(E, E)$ in $\mathscr{L}_s(E, E)$ are continuous linear
maps. Hence
\begin{align*}
U(\mu) \circ \theta &= \int\limits_{G_+} (U (x) \circ \theta) d \mu
(x)\\
\text{and }  \theta \circ U(\mu) &= \int\limits_{G_+}
(\theta \circ U(x)) d \mu (x) 
\end{align*}
for every fixed $\theta \in \mathscr{L}(E, E)$. 
\end{proof}
