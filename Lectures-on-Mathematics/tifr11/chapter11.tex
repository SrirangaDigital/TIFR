
\chapter{Weak boundary value problems}\label{chap11}

\noindent We\pageoriginale first formulate the generalized weak
``boundary value problem''.
\noindent {\bf Problem 11.1.} $Q$ is a Banach space and $Q'$ is its
strong dual. $V$ is a Hilbert space satisfying $V\subset Q, V \subset
Q'$ with continuous injections and $a(u, v)$ is a continuous
sesquilinear form on $V$. Let $V$ be dense in $Q$. Then we can find a
subspace $N$ of $V$ and a continuous linear map $A : N \to Q'$ such
that $\langle Au, \bar{v}\rangle = a(u, v)$. (See Lions: {\bf``On
  Elliptic Partial Differential Equations''}, Tata Institute of
Fundamental Research, Lecture~5). Let $\overrightarrow{g} \in
\mathscr{D}_+'(t, Q')$. We look for $\overrightarrow{u} \in
\mathscr{D}_+'(t, N)$ such that $\frac{d\overrightarrow{u}}{dt}+
A\overrightarrow{u} = \overrightarrow{g}$. 
\setcounter{section}{11}
\setcounter{theorem}{0}
\begin{theorem}\label{chap11:thm11.1}
We follow the above notations. Assume that 
\begin{itemize}
\item [1)] $\overrightarrow{g}$ has a Laplace transform:
  $\overrightarrow{g}\sqsupset \overrightarrow{G} (p)$ for $Rl 
  p>a$.
\item [2)] $\langle u, \bar{u}\rangle V \geq 0$ for $u \in V$, where
  $\langle , \rangle_V$ denote the scalar product in $V$. 
\item [3)] There exists an $\alpha > 0$ and a real $\mathscr{E}_1$
  such that 
$$ 
a_1(u, u)+\mathscr{E} \langle u, \bar{u}\rangle \geq \alpha \parallel
u \parallel_v^2 \quad for \quad \mathscr{E} \geq \mathscr{E}_1
$$
where $a_1(u, v)$ is the real part of $a(u, v)$ and $\parallel \quad
\parallel_V$ denotes the norm in $V$. 
\end{itemize}

Then there exists a unique $\overrightarrow{u} \in \mathscr{D}_+'(t,
N)$ such that $\overrightarrow{u}$ has a Laplace transform and
satisfies $\frac{d\overrightarrow{u}}{dt}+ A
\overrightarrow{u} = \overrightarrow{g}$.
\end{theorem}

\begin{proof}
$\overrightarrow{G}(p)$ is a holomorphic function with values in
$Q'$. For every fixed $p$ such that $Rl p> \gamma = Max
(\mathscr{E}_1, a)$ consider the equation 
\begin{align*}
(p + A)\overrightarrow{U}(p) = \overrightarrow{G}(p)\tag{1}\label{c11:eq1}
\end{align*}
Since\pageoriginale $a_1(u, v)+\mathscr{E} \langle u, \bar{u}\rangle
\geq \parallel u \parallel_v^2$ for $\mathscr{E} \geq \mathscr{E}_1$
with $\alpha > 0$, it follows (See Lions: {\bf ``On Elliptic Partial
 Differential Equations''}, Tata Institute of Fundamental Research,
Lecture 5) that the operator $(p + A) : N \to Q'$ is an isomorphism
for $Rl p > \gamma$. Hence there exists a unique
$\overrightarrow{U}(p) \in N$ such that \eqref{c11:eq1} is valid. In fact,
$\overrightarrow{U}(p) = (p + A)^{-1} \overrightarrow{G}(p)$ for every
fixed $p$ with $Rl p > \gamma$.
\end{proof}

We shall next show that $\overrightarrow{U}(P)$ is a holomorphic
function with values in $N$. For this we need the following general

\begin{lemma*}
Suppose $N$ and $Q'$ are two Banach spaces and $L(P)$ a holomorphic
function in a domain $\Omega$ (in $C$) with values in $\mathscr{L}
(N,Q')$. Assume $L(P)$ has a continuous inverse $L^{-1}(P)$ at every
point $p$ of $\Omega$. Then the function $p \to L^{-1}(p)$ is a
holomorphic function in $\Omega$ with values in $\mathscr{L}(Q', N)$.  
\end{lemma*}

\begin{proof}
For $p, p+h \in \Omega$ we have
$$
L^{-1}(p+h)-L^{-1}(p) = L^{-1}(p+h) \left\{ L(P) - 
L(P+h)\right\} L^{-1}(P).
$$
The norm of $L^{-1}(p+h)$ remains bounded as $h \to 0$. For if not, 
we can find a sequence $\{h_n\} n = 1, 2, \ldots$ of complex numbers
tending to zero, and a sequence $X_n$ of element in $Q'$ such that 
$$
L^{-1}(p+h_n) X_n = \lambda_n \geq n \quad \text{and} \quad \parallel
X_n\parallel = 1\quad \text{in}\quad Q'.
$$
Let $Y_n = \frac{X_n}{\lambda_n}$. Then $\parallel Y_n \parallel_{Q'} \to
0$ and $\parallel L^{-1}(p+h_n)Y_n\parallel_V=1$. Let $Z_n =
L^{-1}(p+h_n)Y_n$. Then $\parallel Z_n\parallel = 1$. Now $\parallel
L(p+h_n)Z_n\parallel_{Q'}= \parallel Y_n\parallel_{Q'}$ and this tends
to $0$ as $n \to \infty$. Due to the continuity of $p \to L(p)$ we
have $\lim\limits_{n \to \infty}\parallel L(p+h_n) - L(p)\parallel =
0$. Hence $\lim\limits_{n \to \infty} \parallel L(p) Z_n \parallel_{Q'} =
0$. But $L(p)$ is invertible and $\parallel Z_n \parallel = 1$. This
cannot happen. Hence $\parallel L^{-1}(p+h) \parallel$ is bounded as
$h \to 0$. Hence $\lim\limits_{h \to \circ} \parallel L^{-1}(p+h) -
L^{-1}(p)\parallel = 0$. This\pageoriginale proves the continuity of
$p \to L^{-1}(p)$ for $p$ in $Rl p > \gamma$.
\end{proof}

We now prove the differentiability of the function $p \to L^{-1}(p)$
in $\Omega$. For $p$ and $(p+h) \in \Omega$, we have
$$
\frac{L^{-1}(p+h)-L^{-1}(p)}{h}= L^{-1}(p+h) 
\frac{L(p)-L(p+h)}{h}L^{-1}(p).
$$
Now, $\quad \lim\limits_{h \to 0} L^{-1}(p+h) = L^{-1}(p)$. Hence
$$
\lim\limits_{h \to 0}\frac{L^{-1}(p+h)-L^{-1}(p)}{h}=L^{-1}(p)L'(p)
L^{-1}(p),
$$
where $L'(p)$ is the derivative of $L(p)$. 

Continuing with proof of theorem~\ref{chap11:thm11.1}, we first remark that the
function $p \to p+A$ is a holomorphic function in $Rl p > \gamma$ with
values in $\mathscr{L}(N, Q')$. From
$\overrightarrow{U}(p)=(p+A)^{-1}\overrightarrow{G}(p)$ and the
holomorphic nature of $\overrightarrow{G}(p)$ and $(p+A)^{-1}$ in $Rl
p > \gamma$, it follows that $\overrightarrow{U}(p)$ is a holomorphic
function of $p$ with values in $N$ for $Rl p > \gamma$. 

The next step in proving our theorem is to show that
$\overrightarrow{U}(p)$ is the Laplace transform of a well determined
distribution $\overrightarrow{u} \in \mathscr{D}_+'(t, N)$ and that
this $\overrightarrow{u}$ satisfies the equation
$(\frac{d}{dt}+A)\overrightarrow{u}=\overrightarrow{g}$. For any $f
\in Q'$ we have $v \to \langle f, \bar{v}\rangle_{Q', Q}$ to be a
continuous linear functional on $V$. Hence
$$
\langle f, \bar{v}\rangle = \langle \tilde{J} f, v\rangle_V \quad
\text {where} \quad \tilde{J} : Q' \to V
$$
is some fixed continuous linear map. Let
$$
\alpha_p (u, v)= a (u, v) + p \langle u, \bar{v}\rangle.
$$
The map $v \to a (u, v)$ is also a continuous linear map of $V$ in
$V$. Hence there exists a fixed continuous linear map $\tilde{K} : V
\to V$ such  that 
$$
\hfill a(u, v) = \langle \tilde{K}u, v\rangle_V.  
$$
\begin{align*}
\text{Then} \quad \alpha_p(u, v) &= \langle \tilde{K}u, v\rangle_V + p
\langle u, \bar{v}\rangle\\
&= \langle \tilde{K}u, v\rangle_V + p \langle \tilde{J} u, v\rangle_V
\quad (\text{since} \quad V\subset Q')\\
&= \langle (\tilde{K} + p\tilde{J}) u, v \rangle_V.
\end{align*}\pageoriginale
For each fixed $p$ in $Rl p > \gamma$ the element
$\overrightarrow{U}(p)$ of $N$ is nothing but that element of $N$
satisfying
$$
\alpha_p (\overrightarrow{U}(p), v) = \langle \overrightarrow{G}(p),
\bar{v}\rangle_{Q', Q}.
$$
Hence $\overrightarrow{U}(p)$ is that element of $N$ satisfying 
\begin{align*}
\langle (\tilde{K} + p\tilde{J}) \overrightarrow{U} (p), v \rangle_V
&= \langle \tilde{J} \overrightarrow{G}(p), v \rangle_V.\\
\text{Hence}\qquad  (\tilde{K} + p\tilde{J}) \overrightarrow{U}(p) &=
\tilde{J} \overrightarrow{G} (p).
\end{align*}
If we show that for $Rl p > \gamma$ the operator $\tilde{K} + p
\tilde{J}$ is invertible, we will get 
$$
\overrightarrow{U}(p) = (\tilde{K} + p \tilde{J})^{-1} \tilde{J}
\overrightarrow{G}(p).
$$
Now $\qquad \left| \langle (\tilde{K} + p\tilde{J})u, u
\rangle_V\right| = \left| a(u, u) + p \langle u, \bar{u}\rangle
\right|$.

\noindent We have $Rl \left\{ a(u, u)+ p \langle u, \bar{u} \rangle
\right\}= a_1 (u, u) + \mathscr{E} \langle u, \bar{u} \rangle$ where
$\mathscr{E}= Rl p$. (This follows from the assumption that $\langle
u, \bar{u} \rangle$ is real). Hence
\begin{align*}
\left|a(u, u) + p \langle u, \bar{u}\rangle \right| &\geq a_1 (u, u)
+ \mathscr{E} \langle u, \bar{u}\rangle\\
&\geq \alpha \parallel u \parallel_V^2 \quad \text{for} \quad
\mathscr{E} > \nu.  (\text{ with } \alpha > 0).\\
\text{Hence } \parallel (\tilde{K}+p\tilde{J})
u\parallel_V \parallel u \parallel_V &\geq \alpha \parallel u
\parallel_V^2. \quad \text{Hence} \\
\parallel (\tilde{K} + p \tilde{J}) u \parallel_V &\geq \alpha
\parallel u \parallel_V \text{ with }  \alpha > 0.  
\end{align*}
This implies, since we are in a Hilbert space, that the operator
$\tilde{K}+p\tilde{J}$ is invertible and that $\parallel (\tilde{K} +
p\tilde{J})^{-1} \parallel \leq \frac{1}{\alpha}$ for $Rl p >
\nu$. Hence we have $\overrightarrow{U} (p) = (\tilde{K} +
p\tilde{J})^{-1} \tilde{J} \overrightarrow{G} (p)$.
\begin{align*}
\parallel \overrightarrow{U} (p) \parallel &\leq \parallel (\tilde{K}
+ p\tilde{J})^{-1} \parallel \parallel \tilde{J}\parallel
\parallel\overrightarrow{G} (p)\parallel\\
&\leq \frac{1}{\alpha} \beta 
\parallel \overrightarrow{G} (p)\parallel
\end{align*}
where\pageoriginale $\beta$ is some constant. Since
$\overrightarrow{G} (p)$ has a uniform polynomial majorisation in any
half plane $Rl p \geq \nu + \varepsilon, \varepsilon > 0$, the same is
true of $\overrightarrow{U} (p)$. Hence $\overrightarrow{U}(p)$ is the
Laplace transform of a certain distribution $\overrightarrow{u} \in
\mathscr{D}_+' (t, N)$, which is unique.
