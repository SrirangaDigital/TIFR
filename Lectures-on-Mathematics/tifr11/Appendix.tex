
\addcontentsline{toc}{chapter}{Appendix}
\chapter*{Appendix\\ Representations of the semi-group of positive
reals. Hille-Yosida Theorem for complete locally convex spaces} 


In\pageoriginale this section we take for $G_+$ the (additive)
semi-group of positive real numbers.

Let $U$ be a representation (equicontinuous) of $G_+$ in $E$. The
linear operator $U(-\delta')$ ($\delta'=$ first derivative of the
Dirac measure) is called the infinitesimal generator of the
representation. We shall show that every complex number $p$ with $Rl\;
p>0$ is in the resolvent set of the infinitesimal generator. We have
$pI-U(-\delta')=U(p\delta+\delta')$. We have further, in $R, (p\delta
+\delta')*Y(t) e^{-pt}=Y(t)e^{-pt}*(p\delta +\delta')=\delta, Y(t)$
denoting the Heaviside function. Now for $Rl\;p>0, e^{-pt}$ is a
summable measure in $G_+$ and hence $U(e^{pt})$ is a continuous linear
operator in $E$ (see Lemma \ref{chap17:lem17.3}).  Since
$U(\delta)=I$, using Proposition \ref{chap19:prop19.1'}$'$, we see that
$U(e^{-pt})$ is the inverse of $pI-U(-\delta')$. 

We shall now prove an equicontinuity property of the resolvent
operators of the infinitesimal generator $A$ of the one-parameter
semi-group $U(t)$. Since 
$$
(\delta +\frac{\delta'}{p})*p e^{-pt}=\delta(p>0),
$$
as before we see, using Proposition \ref{chap19:prop19.1'}$'$, that
$U(pe^{-pt})=(I-\frac{A}{p})^{-1}$. But as $\int\limits_\circ^\infty p
e^{-pt}dt=1$ we see that (see Lemma \ref{chap17:lem17.3})
$U(pe^{-pt})$ belongs to the convex closed stable envelope
$\overset{\frown}{\mathscr{U}}$ of the set $\mathscr{U}=\{U(t), t\geq
0\}$. In a similar way, we see that $(I-\frac{A}{p})^{-m} (p>0, m=1,
2,\ldots)\in\overset{\frown}{\mathscr{U}}$. Hence the set of operators
$\{(I-\frac{A}{p})^{-m}\}$, as $p$ runs through strictly positive
numbers and $m$ through positive integers, is equicontinuous (with
$\mathscr{U}$). 

We\pageoriginale shall now show that the equicontinuity condition we
proved for the resolvent operators of the infinitesimal generator of a
one-parameter semi-group is also sufficient to ensure that a densely
defined linear operator in $E$ be the infinitesimal generator of a
one-parameter semi-group. The problem here is to define the
exponential of such an operator. Before going into this problem we
shall first consider the question of defining the exponential of a
continuous linear operator. 

\noindent {\bf The exponential of a continuous linear operator.}

Let $E$ be a complete $E L C$ and $T$ a continuous linear operator of
$E$ into itself. We try to define $\exp tT$ as a continuous linear
operator by means of the series\footnote{If $E$ is a Banach space, the
  series $\sum\frac{(tT)^k}{k!_T}$ converges in the uniform topology
  for \emph{any} continuous linear operator $T$ of $E$ into $E$.}
 
$$
(\exp tT)x=\sum\limits_{k=\circ}^\infty\frac{(tT)^kx}{k!}\;(x\in
E)\;(t\geq 0). 
$$
The series will converge for every $x\in E$ and represent a continuous
linear operator of $E$ into itself, at least if $T$ and its iterates
$T^k(k=2, 3,\ldots)$ are equicontinuous. Actually the series
$\sum\limits_{k=\circ}^\infty \frac{(tT)^kx}{k!}$ will converge at
every $x\in E$ if the set $\{T, T^2,\ldots\}$ is weakly bounded. For
then if $q$ is any continuous semi-norm on $E$, we have 
$$
\sum\limits_{k=\circ}^m\frac{q((tT)^kx)}{k!}=\sum\limits_{k=\circ}^m
\frac{t^kq(T^kx)}{k!}\leq C\sum\limits_{k=\circ}^\infty
\frac{t^k}{k!},
$$
$C$ being a positive constant, so that the series
$\sum\limits_{k=\circ}^\infty\frac{q((tT)^kx)}{k!}$ is
convergent. Since $E$ is complete, it follows that
$\sum\limits_{k=\circ}^\infty\frac{(tT)^kx}{k!}$ is convergent in
$E$. To show, under the hypothesis that the set $\{T,T^2,\ldots\}$ is
equicontinuous, that $x\to (\exp T)x$ is a continuous operator, it is
sufficient to show that the operators $B_n=\sum\limits_{k=\circ}^n
\frac{(tT)^k}{n!}$ are equicontinuous since\pageoriginale the
pointwise limit of a sequence of equicontinuous linear operators is a
continuous linear operator. To show this we use the following
criterion for equicontinuity which will also be used later. Let
$\{B_\alpha \}$ be a family of linear operators of $E$ into $E$; in
order that $\{B_\alpha\}$ be equicontinuous, it is necessary and
sufficient that the following condition be satisfied: for every
continuous semi-norm $q$ on $E$ there exists a continuous semi-norm
$p$ on $E$ and a strictly positive number $\underline{a}$ such that 
$$
q(B_\alpha (x))\leq a p(x), \quad \text{for every}\quad \alpha \quad
\text{and} \quad x\in E.
$$
(see Espaces Vectoriels Topologiques, Ch. II, by N. Bourbaki). To
prove that the above $B_n$ are equicontinuous, let $q$ be a continuous
semi-norm on $E$. Since $\{T^k\}_{k=0, 1,\ldots}$ are equicontinuous
there exists a continuous semi-norm $p$ and $a>0$ such that
$q(T^kx)\leq a\, p(x)_k\;(k=0, 1,\ldots ,x\in E)$. Now 
\begin{align*}
q(B_n x)=q(\sum\limits_{k=\circ}^n \frac{(tT)^kx}{k!}) &\leq
\sum\limits_{k=\circ}^n \frac{t^kq(T^kx)}{k!}\\
&\leq (a. \exp t) p(x)
\end{align*}
which shows that $B_n$ are equicontinuous. One proves the same way the
following results. 
\begin{itemize}
\item [i)] Writing $T_tx=\exp (tT)x$, the map $t\to\exp tTx$ is a
  continuous function from $(0, \infty)$ in $E$, for every $x\in X$,
  and 
\begin{itemize}
\item [(a)] $\lim\limits_{h\downarrow\circ}\frac{T_hx-x}{h}$ exists
  for every $x\in E$ and in equal to $Tx$; 
\item [(b)] $\lim\limits_{h\to\circ}\frac{T_{t+h}x-T_tx}{h}$ exists for
  $t>0$ and $x\in E$ and equal to $T_tT x=TT_tx$.
\end{itemize}
\item [ii)] Let $T$ and $S$ be two continuous linear operators such
  that $ST=\break TS$ and such that $\{T^k\}$ and $\{S^k\}$ are
  equicontinuous. Then 
  $$\sum\limits_{k=\circ}^n \frac{t^k(T+S)^k}{k!}$$
  converges pointwise to a continuous linear operator $\exp (t(T+S)=
  \exp t T. \exp t S= \exp t S. \exp t T$ and 
$$
\lim\limits_{h\downarrow\circ}\frac{\exp h(T+S)x-x}{h}=(T+S)x, x\in E
$$\pageoriginale
\begin{align*}
&\lim\limits_{h\to\circ} \frac{\exp (h+t)(T+S)x-\exp t(T+S)x}{h}\\
&\qquad\qquad=\exp  t(S+T)\quad(S+T)x\\
&\qquad\qquad= (S+T)\exp t(S+T)x \quad (x\in E).
\end{align*}
\end{itemize}
We now prove the

\noindent {\bf Theorem (Hille-Yosida).} Let $E$ be a complete $E L
C$. Suppose that $A$ is a densely defined linear operator on $E$ such
that for every strictly positive $p,(I-\frac{A}{p})^{-1}$ exists and
such that the family $\mathscr{F}$ of operators
$\{(I-\frac{A}{p})^{-m}\}$ ($p$ strictly positive, $m=1, 2,\ldots)$ is
equicontinuous. Then there exists a uniquely determined representation
$T(t) (t\geq 0)$, which is equicontinuous with $\mathscr{F}$, whose
infinitesimal generator is $A$.

\begin{proof}
We follow Yosida's method of proof.

Writing $J_\lambda =(I-\lambda^{-1}A)(\lambda >0)$ we have
evidently: $A J_\lambda x=\lambda(J_\lambda -I)x, x\in E$ and $A
J_\lambda x=J_\lambda A x=\lambda (J_\lambda -I)x$, for
$x\in\mathscr{D}(A)$, where $\mathscr{D}(A)$ denotes the domain of
$A$. We shall prove that $J_\lambda x\to x$, as $\lambda \to\infty$,
for every $x\in E$. If $x\in \mathscr{D}(A), J_\lambda
x-x=\lambda^{-1} J_\lambda(Ax)$ and hence $J_\lambda x-x\to 0$ as
$\lambda\to \infty$, as the set $\{J_\lambda(Ax)\}$ is bounded. Since
$\mathscr{D}(A)$ is dense in $E$ and $\{J_\lambda\}_{\lambda >0}$ is
equicontinuous, it follows that $J_\lambda x\to x$ for every $x\in E$.

Set
$$
T_t^{(\lambda)}=\exp (t A J_\lambda)=\exp (t\lambda (J\lambda -I))=\exp
(-\lambda t)\exp(\lambda t J_\lambda).
$$
It is easily seen, using for example the criterion for equicontinuity
used earlier, that the operators $\{T_t^{(\lambda)}\} (\lambda >0,
t\geq 0)$ are equicontinuous with $\mathscr{F}$. We remark that
$J_\lambda J_\mu = J_\mu J_\lambda, \lambda, \mu >0$. We now prove
that as $\lambda\to\infty, T_t^{(\lambda)}$ converges in the topology
of simple convergence, to a continuous\pageoriginale linear operator
$T_t$ and for fixed $x$, $T_t^{(\lambda)}x\to T_tx$ uniformly when $t$
lies in a compact set. To prove this, let $q$ be a continuous
semi-norm on $E$. Since $\{T_t^{(\lambda)}\}$ are equicontinuous there
exist a continuous semi-norm $p$ and $a>0$ such that
$q(T_t^{(\lambda)}x)\leq$ a $p(x)$ for $\lambda >0, t\geq 0$ and every
$x\in E$. For $\lambda, \mu >0$ and $x\in\mathscr{D}(A)$
\begin{align*}
q(T_t^{(\lambda)}(x)-T_t^{(\mu)}x) &=q \left[\int\limits_0^t
  \frac{d}{ds}\{T_{t-s}^{(\mu)} T_s^{(\lambda)}x\}ds\right]\\
&=q\left[\int\limits_0^t T_{t-s}^{(\mu)}T_s^{(\lambda)}(AJ_\lambda-AJ_\mu)x\right]\\
&\leq t a^2 p\left[(J_\lambda A-J_\mu A)x\right],
\end{align*}
and $(J_\lambda A-J_\mu A)x\to 0$, as $\lambda, \mu\to\infty$ as $x\in
\mathscr{D}(A)$. So $\lim\limits_{\lambda, \mu\to\infty}
q(T_t^{(\lambda)}(x)-T_t^{(\mu)}x)=0$ uniformly when $t$ lies in a
compact set. Since $\mathscr{D}(A)$ is dense in $E$ and the set of
operators $\{T_t^{(\lambda)}\}$ is equicontinuous, we see that
$\lim\limits_{\lambda\to\infty} T_t^{(\lambda)}x\equiv T_tx$ exists
for every $x\in E$ and $t$ uniformly in any compact set that the set of
operators $\{T_t\}_t\geq 0$ is equicontinuous with $\mathscr{F}$. From
the uniform convergence, $t\to T_t$ is a continuous map of $t\geq
0$. To prove that $T_{t+s}=T_t T_s$, let $q$ be a continuous semi-norm
and let $\underline{a}$ and $p$ have the same meaning as before. Then,
using $T_{t+s}^{(\lambda)}=T_t^{(\lambda)}T_s^{(\lambda)}$, 
\begin{align*}
q((T_{t+s}-T_t T_s)x) &\leq q(T_{t+s}-T_{t+s}^{(\lambda)}x)+q
(T_{t+s}^{(\lambda)}x-T_t^{(\lambda)}T_s^{(\lambda)}x)\\
&+ q(T_t^{(\lambda)} T_s^{(\lambda)}-T_t^{(\lambda)}T_sx)+q
(T_t^{(\lambda)}T_sx-T_t T_sx)\\
&\leq q(T_{t+s}x-T_{t+s}^{(\lambda)}x)+a p(T_s^{(\lambda)}x-T_s x)\\
&+q\left[ (T_t^{(\lambda)}-T_t)(T_sx)\right]\to 0. 
\end{align*}
Since $q(T_{t+s}x-T_t T_sx)=0$ for every continuous semi-norm $q$, we
must have $T_{t+s}=T_t T_s$. 

Let\pageoriginale $A'$ be the infinitesimal generator of the
semi-group $T_t$. We have to show that $A'=A$. To show this, it is
sufficient to show that $A'$ is an extension of $A$ (i.e., $x\in
\mathscr{D}(A)$ implies $x\in\mathscr{D}(A')$ and $Ax=A'x$). For,
$t\to T_t$ being an equicontinuous representation,
$(I-\lambda^{-1}A'): \mathscr{D}(A')\to E$ is a bijection for $\lambda
>0$ and by hypothesis $(I-\lambda^{-1}A):\mathscr{D}(A)\to E$ is a
bijection, for $\lambda >0$, so that
$\mathscr{D}(A)=\mathscr{D}(A')$. To prove that $A'$ is an extension
of $A'$ let $x\in\mathscr{D}(A)$. Then $T_s^{(\lambda)} A J_\lambda
x\to T_s A I x$. For if $q$ is a continuous semi-norm, we have 

\begin{align*}
q(T_s A x-T_s^{(\lambda)}AJ_\lambda x) & \leq q(T_s A
x-T_s^{(\lambda)}Ax)+q(T_s^{(\lambda)}A x-T_s^{(\lambda)}A J_\lambda
x)\\
&\leq q\left[(T_s-T_s^{(\lambda)})(Ax)\right]+a p(Ax-J_\lambda Ax)\\
& \to 0, \quad\text{as}\quad \lambda
 \to \infty, \quad \text{(since}
 \quad J_\lambda Ax\to Ax).\\
\text{Now, }  T_tx-x &= \lim\limits_{\lambda\to\infty}
T_t^{(\lambda)}x-x\\
&= \lim\limits_{\lambda\to\infty}\int\limits_0^t T_s^{(\lambda)} A
J_\lambda x \; ds\\
&= \int\limits_0^t \lim\limits_{\lambda\to\infty} T_s^{(\lambda)} A
J_\lambda x\\
&= \int\limits_0^t T_s Ax
\end{align*}
so that $\lim\limits_{t\downarrow \circ} \frac{T_tx-x}{t}$ exists and
equal to $Ax$, i.e., if $x\in\mathscr{D}(A)$, then $x\in\mathscr{D}
(A')$ and $A'x=Ax$.

The uniqueness of $T_t$ follows from the following fact, which is\break
proved the same way as in the case of Banach spaces: If $t\to T_t$ is
a representation (equicontinuous) and $A$ is the infinitesimal
generator of $T_t$ then 
$$
T_tx=\lim\limits_{\lambda\to\infty}\exp(t A J_\lambda)x, \quad
\text{for every}\quad x\in E.
$$
\end{proof}

\begin{remarks*}
\begin{itemize}
\item [(i)] In a Banach space the condition of the theorem reads: there exists a
constant $M>0$ such that 
$$
\parallel (\lambda I-A)^{-m}\parallel \leq M/\lambda m \quad
\text{(for}\quad m=1, 2,\ldots, \lambda >0\text{)}
$$
\item [(ii)] For the proof of the theorem it is sufficient to assume
  that $E$ is quasi-complete.
\end{itemize}
\end{remarks*}
