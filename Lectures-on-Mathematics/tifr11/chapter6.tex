\chapter{Operations on Vector valued Distributions}\label{chap6}

$E$\pageoriginale will always denote a complete E L C.

\medskip
\noindent
{\bf Differentiation of vector valued distributions.}
\smallskip

Let $\overrightarrow{T} \in \mathscr{D}'(E)$. Let $D^p$ be the
operator $\dfrac{\partial^{p_1+\cdots+ p_n}}{\partial x_{1}^{p_{1}}\ldots
  \partial x_{n}^{p_{n}}}$ where $p$ is the $n$-tuple $(p_1,
p_2,\ldots , p_n)$. $D^p \overrightarrow{T}$ is defined to be the map,
which maps $\varphi \in \mathscr{D}$ into
$(-1)^{|p|}\,\overrightarrow{T}(D^p \varphi) \in E$ It is easily seen
that $D^p\overrightarrow{T}$ is an $E$-valued distribution.   


We have $\langle D^p\overrightarrow{T}, \overleftarrow{e'}\rangle =
D^p \langle \overrightarrow{T}, \overleftarrow{e'}\rangle$. This
follows from the very definition of $D^p$.

\smallskip
\noindent {\bf Scalar product}.

Let $\mathscr{H}$ be a normal space of distributions. Then
$\mathscr{H}_c'$ is a normal space of distributions (Proposition
\ref{chap4:prop4.1}). Let us denote by S.T the scalar product of any element 
$S \in\mathscr{H}$ and any element $T \in \mathscr{H}'$. Let now
$\overrightarrow{S} \in \mathscr{H} (E)$ and $T \in \mathscr{H}'$.

\setcounter{section}{6}
\setcounter{definition}{0}
\begin{definition}\label{chap6:def6.1}
$\overrightarrow{S} \in \mathscr{H}(E)$. Hence $\overrightarrow{S}$
  can be considered as a continuous linear map of $\mathscr{H}_c'$ in
  $E$. Hence if $T \in \mathscr{H}_c', \overrightarrow{S} (T) \in
  E$. We define $\overrightarrow{S}_{\dot{1}} T$ to be the element
  $\overrightarrow{S} (T)$ of $E$.
\end{definition}

\begin{definition}\label{chap6:def6.2}
Let $\overrightarrow{S} \in \mathscr{H} (E)$. Then
${}^t\overrightarrow{S}: E_c' \to \mathscr{H}$ is a continuous linear
map. We have agreed to denote by $\langle \overrightarrow{S},
\overleftarrow{e'} \rangle$ the element ${}^t \overrightarrow{S}
(\overleftarrow{e'})$ of $\mathscr{H}$. Let $T \in \mathscr{H}'$. Then
$\overrightarrow{S}_{\dot{2}} T$ is defined to be that element of $E$ which
satisfies 
\begin{align*}
\langle\overrightarrow{S}_2. T, \overleftarrow{e'} \rangle_E,
 E' &=\langle {}^t \overrightarrow{S} (\overleftarrow{e}')'^{T}\rangle_
{\mathscr{H}, \mathscr{H}'}\\
&= {}^t \overrightarrow{S} (\overleftarrow{e}')^{\cdot T}\\
&= \langle \overrightarrow{S}, \overleftarrow{e}'\rangle \cdot T 
\end{align*}
\end{definition}

\begin{definition}\label{chap6:def6.3}
Let\pageoriginale $\overrightarrow{S} \in \mathscr{H} \subset E$ and $T \in
\mathscr{H}'$. Then $T : \mathscr{H} \to C$ and $I : E \to E$ are
continuous linear maps. Hence $T \in I : \mathscr{H}
\subset E \to C \subset E = E$ is a continuous linear map
(Definitions~4.3 (\ref{chap4:def4.3(1)}), 4.3~(\ref{chap4:def4.3(2)}) and
4.3~(\ref{chap4:def4.3(3)})). We define
$\overrightarrow{S}_3. T$ to be the element $T \in I
(\overrightarrow{S})$ of $E$. 
\end{definition}

\setcounter{section}{6}
\setcounter{prop}{0}
\begin{prop}\label{chap6:prop6.1}
The elements $\overrightarrow{S}_{\dot{1}} T, \overrightarrow{S}_{\dot{2}} 
T$ and $\overrightarrow{S}_{\dot{3}}$ $T$ of $E$ are all equal.
\begin{align*}
Now \quad \langle \overrightarrow{S}_1. T, \overleftarrow{e}' \rangle _{E,
  E'} &= \langle \overrightarrow{S} (T), \overleftarrow{e}' \rangle =
\langle T, {}^t \overrightarrow{S}
(\overleftarrow{e}')\rangle_{\mathscr{H}',\mathscr{H}}\\
&= \langle \overrightarrow{S}_2. T, \overleftarrow{e}' \rangle_{E, E'}.
\end{align*}
\end{prop}

\noindent 
This proves\quad $\overrightarrow{S}_1. T=\overrightarrow{S}_2\cdot T$.

 \noindent Also \quad $\langle \overrightarrow{S}_3\cdot T,
\overleftarrow{e}'\rangle_{E, E'}= \langle T \in I
(\overrightarrow{S}), \overleftarrow{e}' \rangle_{E, E'}.$

\noindent Proposition~\ref{chap4:prop4.2} gives $T \in I
(\overrightarrow{S})$, considered as an element of $C \subset E$
or as an element of $\mathscr{L}_\varepsilon (C_C; E)$, is the same as
the composite of the maps ${}^t T : C \to \mathscr{H}_c',
\mathscr{H}_c'\xrightarrow{\overrightarrow{S}} E$ and $I : E \to E$.

\noindent Hence $\langle T \in I (\overrightarrow{S}),
\overleftarrow{e}' \rangle_{E, E'} = \langle I \circ
\overrightarrow{S} \circ t_T(1), \overleftarrow{e}' \rangle_{E, E'}$.

\noindent  Also $\qquad\overrightarrow{S} \circ t_T(1) =
\overrightarrow{S}(T). \quad \text{Hence we have}$
\begin{align*}
\langle I \circ \overrightarrow{S} \circ t_T (1),
\overleftarrow{e}'\rangle_{E, E'} &= \langle \overrightarrow{S} (T),
\overleftarrow{e}' \rangle\\
&= \langle \overrightarrow{S}_1. T, \overleftarrow{e}' \rangle.
\end{align*}
Hence $\qquad\overrightarrow{S}_{\dot{3}} T =
\overrightarrow{S}_{\dot{1}} T.$

\noindent As example, we see in the following situations we can define
a scalar product:

\begin{center}
\begin{tabular}{llllll}
1) & $\overrightarrow{T} \in \mathscr{D}'(E), \varphi \in \mathscr{D}$
& ; & & 2) & $\overrightarrow{T} \in \mathscr{D}_c'^m(E), \varphi \in
\mathscr{D}^m$;\\
3) & $\overrightarrow{T} \in \mathscr{S}'(E), \varphi \in \mathscr{S}$
& ; & & 4) & $\overrightarrow{T} \in \mathscr{D}(E), \varphi \in
\mathscr{D}'$;\\
5) & $\overrightarrow{\varphi} \in \mathscr{D}(E), T \in \mathscr{D}'$
& ; & and & 6) & $\overrightarrow{\varphi} \in \mathscr{D}^m(E), T \in
 \mathscr{D}'^m$. 
\end{tabular}
\end{center}

\noindent {\bf Properties\pageoriginale of the scalar product:}

\begin{prop}\label{chap6:prop6.2}
If $T$ belongs to an equicontinuous subset of $\mathscr{H}'$ and
$\overrightarrow{S}$ tends to $0$ in $\mathscr{H}(E),
\overrightarrow{S} \cdot T$ tends to zero uniformly in $E$.
\end{prop}

\begin{proof}
Now, $\overrightarrow{S} \in \mathscr{L}_\varepsilon(\mathscr{H}_c'
E)$, the $\varepsilon$-topology being the topology of uniform
convergence on equicontinuous sets of $\mathscr{H}'$. Hence when $T$
lies in an equicontinuous subset of $\mathscr{H}'$ and
$\overrightarrow{S}$ tends to $0$ in $\mathscr{H}(E),
\overrightarrow{S}(T)$ tends to $0$ uniformly in $E$. But
$\overrightarrow{S} \cdot T = \overrightarrow{S} (T)$. Hence
$\overrightarrow{S} \cdot T$ tends to $0$ in $E$ uniformly with respect to
$T$ in an equicontinuous subset of $\mathscr{H}'$, when
$\overrightarrow{S}$ tends to $0$ in $\mathscr{H} (E)$.
\end{proof}

\begin{prop}\label{chap6:prop6.3}
If $\overrightarrow{S}$ lies in a compact set of $\mathscr{H}(E)$ and
$T$ tends to $0$ in $\mathscr{H}_c'$ and if $\mathscr{H}$ is complete,
then $\overrightarrow{S} \cdot T$ tends uniformly to $0$ in $E$.
\end{prop}

\begin{proof}
Let $K$ be any compact subset of $\mathscr{H}(E) =
\mathscr{L}_{\varepsilon}(E_c' \mathscr{H})$. If $A$ is any
equicontinuous subset of $E_c', \underset{\vec{S} \in K}{U}
S(A)$ lies in a compact subset of $\mathscr{H}$. If
$\overrightarrow{S} \in K$ and $\overleftarrow{e}'$ lies in an
equicontinuous subset of $E'$, we have to show that $\langle
\overrightarrow{S} (T), \overleftarrow{e}'\rangle$ tends to $0$
uniformly as $T \to 0$ in $\mathscr{H}_c'$. We have
$$
\langle \overrightarrow{S}, \overleftarrow{e}' \rangle \cdot T = \langle
\overrightarrow{S} (T), \overleftarrow{e}' \rangle.
$$
That is to say $\langle \overrightarrow{S} (T), \overleftarrow{e}'
\rangle = T(\langle \overrightarrow{S}, \overleftarrow{e}'\rangle)$. 
From what has been said above $\langle \overrightarrow{S},
\overleftarrow{e}'\rangle$ lies in a compact subset of
$\mathscr{H}$. Since $\mathscr{H}$ is complete, the convex stable
envelope of a compact set is compact. Hence if $\overrightarrow{S}$
lies in a compact set of $\mathscr{H}(E)$ and $\overleftarrow{e}'$
lies in an equicontinuous subset of $E', \langle \overrightarrow{S},
\overleftarrow{e}' \rangle$ lies in a compact disc of
$\mathscr{H}$. Hence if $T \to 0$ in $\mathscr{H}_c', T(\langle
\overrightarrow{S}, \overleftarrow{e}' \rangle) \to 0$ uniformly. This
proves proposition \ref{chap6:prop6.3}.
\end{proof}

\begin{prop}\label{chap6:prop6.4}
If\pageoriginale $\overrightarrow{S}$ belongs to a bounded subset of
$\mathscr{H}(E)$ and $T$ tends to $0$ in the strong dual
$\mathscr{H}_\delta'$ then $\overrightarrow{S} \cdot T$ tends to $0$
uniformly in $E$.
\end{prop}

\begin{proof}
It suffices to show that when $\overleftarrow{e}'$ lies in an
equicontinuous subset of $E', \langle \overrightarrow{S} \cdot T,
\overleftarrow{e}' \rangle$ tends to $0$ uniformly. Let
$\overrightarrow{S}$ remain in the bounded set $B$ of
$\mathscr{H}(E)$. Now $\mathscr{H}(E) \approx
\mathscr{L}_\varepsilon(E_c', \mathscr{H})$. If $B$ is bounded in
$\mathscr{H}(E)$, then $\underset{\vec{S}\in B}{U}
\overrightarrow{S}(H)$ is a bounded set of $\mathscr{H}$, whatever be
the equicontinuous subset $H$ of $E_c'$. We have $\langle
\overrightarrow{S} \cdot T, \overleftarrow{e}' \rangle = \langle
\overrightarrow{S}, \overleftarrow{e}' \rangle . T = T (\langle
\overrightarrow{S}, \overleftarrow{e}' \rangle)$. When
$\overleftarrow{e}'$ lies in an equicontinuous subset of $E'$ and
$\overrightarrow{S}$ lies in a bounded set $B$ of $\mathscr{H}(E)$ we
have seen that $\langle \overrightarrow{S}, \overleftarrow{e}'
\rangle$ lies in a bounded set of $\mathscr{H}$. Since $T \to 0$ in
$\mathscr{H}_\delta'$ we have $T(\langle \overrightarrow{S},
\overleftarrow{e}' \rangle) \to 0$ uniformly with respect to
$\overrightarrow{S}$ in a bounded set of $\mathscr{H}(E)$ and
$\overleftarrow{e}'$ in an equicontinuous subset of $E'$. This proves
proposition \ref{chap6:prop6.4}.
\end{proof}


Combining propositions \ref{chap6:prop6.2} and \ref{chap6:prop6.4}, we
get the following

\begin{prop}\label{chap6:prop6.5}
The mapping $(\overrightarrow{S}, T) \to \overrightarrow{S} \cdot T$ of
$\mathscr{H}(E)\times \mathscr{H}_\delta' \to E$ is a bilinear map
hypocontinuous with respect to the bounded subsets of $\mathscr{H}(E)$
and equicontinuous subsets of $\mathscr{H}'$.
\end{prop}

\begin{prop}\label{chap6:prop6.6}
For any element of the form $S \overrightarrow{e}$ in $\mathscr{H}(E),
S \in \mathscr{H},\break \overrightarrow{e} \in E$, we have $S
\overrightarrow{e}\cdot T = S \cdot T \overrightarrow{e}$.
\end{prop}

\begin{proof}
To prove this we have only to verify that 
$$
\langle S \overrightarrow{e}\cdot T, \overleftarrow{e}' \rangle_{E, E'} =
\langle S\cdot T \overrightarrow{e}, \overleftarrow{e}' \rangle_{E, E'}
$$
for every $\overleftarrow{e}' \in E'$. Now
$$
\langle S \overrightarrow{e} \cdot T, \overleftarrow{e}' \rangle = \langle
S \overrightarrow{e} (T), \overleftarrow{e}' \rangle = \langle S\cdot T
\overrightarrow{e}, \overleftarrow{e}' \rangle.
$$
When\pageoriginale $\mathscr{H}$ satisfies the approximation property,
we have a characterization for the scalar product that we have introduced.
\end{proof}

\begin{prop}\label{chap6:prop6.7}
Let $\mathscr{H}$ satisfy the approximation property and $E$ be a
complete $E L C$. The bilinear map that we have defined is the only
bilinear map which is separately continuous and which satisfies $U
\overrightarrow{e} \cdot T = (U \cdot T) \overrightarrow{e}$, for every $U \in
\mathscr{H}, \overrightarrow{e} \in E$ and $T \in \mathscr{H}_c'$.
\end{prop}

\begin{proof}
We have already seen that the bilinear map defined by us is separately
continuous and satisfies $U \overrightarrow{e} \cdot T = (U\cdot T)
\overrightarrow{e}$. 

Suppose there exists two bilinear maps, say $\mu_1$ and $\mu_2$ of
$\mathscr{H}(E)\times \mathscr{H}_c' \longrightarrow E$ which are
separately continuous and satisfy $\mu_1(S \overrightarrow{e}, T) =
\langle S, T\rangle \overrightarrow{e}$ and $\mu_2(S
\overrightarrow{e}, T) = \langle S, T \rangle \overrightarrow{e}$ for
every $S \in \mathscr{H}, \overrightarrow{e} \in E$ and $T \in
\mathscr{H}'$. Since $\mathscr{H}\otimes E$ is dense in
$\mathscr{H}(E)$, the equality of $\mu_1$ and $\mu_2$ on $(\mathscr{H}
\otimes E) \times \mathscr{H}_c'$ and the separate continuity of
$\mu_1$ and $\mu_2$ give $\mu_1 = \mu_2$ on $\mathscr{H}(E) \times
\mathscr{H}_c'$. 
\end{proof}
