
\chapter{Multiplication of vector valued distributions}\label{chap14}

Let\pageoriginale $\mathscr{H}, \mathscr{K}$ and $\mathscr{L}$ be
three $E L Cs$. Let $U : \mathscr{H} \times \mathscr{K} \to
\mathscr{L}$ be a bilinear map which is hypocontinuous with respect to
the bounded subsets of $\mathscr{H}$ and $\mathscr{K}$. Let $E, F, G$
be three Banach spaces and $B : E \times F \to G$ be a continuous
bilinear map. We ask the question whether it will be possible to
define a bilinear map:

$\mathscr{H}(E) \times \mathscr{K}(F) \to \mathscr{L}(G)$ say
$\underset{B}{U}$ such that is satisfies the following conditions:
\begin{enumerate}
\item [1)] $\underset{B}{U}$ is hypocontinuous with respect to bounded
 sets of $\mathscr{H}(E)$ and $\mathscr{K}(F)$.
\item [2)] For decomposed elements, that is for elements of the type
  $S\overrightarrow{e}$ and $T \overrightarrow{f}$ of $\mathscr{H}(E)$
  and $\mathscr{K}(F)$ we have $S \overrightarrow{e} \underset{B}{U} T
  \overrightarrow{f} = (S U T) B(\overrightarrow{e},
  \overrightarrow{f})$ with $\overrightarrow{e} \in E,
  \overrightarrow{f} \in F, S \in \mathscr{H}$ and $T \in
  \mathscr{K}$.  
\end{enumerate}

In general it will not be possible to define such a map. We shall give
here, without proof, a certain example in which such a map
$\underset{B}{U}$ cannot be defined. Let $\mathscr{D}^\circ$ be the
space of continuous functions with compact support on $R^N$. Let
$\mathscr{D}_\delta^{\circ'}$ be the strong dual of
$\mathscr{D}^\circ$, that is to say $\mathscr{D}_\delta^{\circ'}$ is
the space of measures on $R^N$. Let $\overrightarrow{\mu} \in
\mathscr{D}_\delta^{\circ'}(E)$ and $\overrightarrow{\varphi} \in
\mathscr{D}^\circ (F)$ where $E$ and $F$ are two Banach spaces. Let $B
: E \times F \to G$ be a continuous bilinear map. The duality between
$\mathscr{D}_\delta^{\circ'}$ and $\mathscr{D}^\circ$ gives a bilinear
map of $\mathscr{D}_\delta^\circ x \mathscr{D}^\circ$ in $C$
hypocontinuous with respect to the bounded sets. But a bilinear map
$\underset{B}{U} : \mathscr{D}^{\circ'} (E) \times \mathscr{D}^\circ
(F) \to G$ cannot be defined to satisfy the conditions (1) and
(2). We\pageoriginale shall now prove the following

\setcounter{section}{14}
\setcounter{theorem}{0}
\begin{theorem}\label{chap14:thm14.1}
Let $\mathscr{H}, \mathscr{K}, \mathscr{L}$ be three locally convex
separated complete vector spaces all of which are nuclear. Let the
strong duals of the three spaces be also nuclear. Let $U : \mathscr{H}
\times \mathscr{K} \to \mathscr{L}$ be a bilinear map hypocontinuous
with respect to the bounded subsets of $\mathscr{H}$ and
$\mathscr{K}$. Let $E, F$ and $G$ be three Banach spaces with a
continuous bilinear map $B : E \times F \to G$. Then there exists one
and only one bilinear map $\underset{B}{U} : \mathscr{H}(E) \times
\mathscr{K}(F) \to \mathscr{L}(G)$ which satisfies 
\begin{itemize}
\item [1)] $S \overrightarrow{e} \underset{B}{U} T \overrightarrow{f}
  = (S U T) R(\overrightarrow{e}, \overrightarrow{f})$ for every
  $\overrightarrow{e} \in E, \overrightarrow{f} \in F, S \in
  \mathscr{H}$ and $T \in \mathscr{K}$.
\item [2)] $(\overrightarrow{S}, \overrightarrow{T}) \to
  \overrightarrow{S} \underset{B}{U} \overrightarrow{T}$ is separately
  continuous in $\overrightarrow{S}$ and
  $\overrightarrow{T}$. Moreover $\underset{B}{U}$ has the following
  supplementary properties.
\item [3)] $\underset{B}{U}$ is hypocontinuous with respect to the
  bounded subsets of\break $\mathscr{H}(E)$ and $\mathscr{K}(F)$. 
\item [4)] \begin{align*}
  \overrightarrow{S} \underset{B}{U} \overrightarrow{T} & =
(I_{\mathscr{L}} \varepsilon \tilde{B})
(U_{\overrightarrow{S}}^{\otimes} I_F) (\overrightarrow{T})\\
&= (I_{\mathscr{L}} \varepsilon \tilde{B}) (I_E \otimes
 U_{\overrightarrow{T}}) (\overrightarrow{S})
\end{align*} 
\end{itemize}
where $I_{\mathscr{L}}, I_E$ and $I_F$ are the identity mappings of
$\mathscr{L}, E$ and $F$ respectively, and $U_{\overrightarrow{S}},
U_{\overrightarrow{T}}$ and $\tilde{B}$ are defined as
follows. $U_{\overrightarrow{S}} : \mathscr{K} \to \mathscr{L}(E)$ on
any $T \in \mathscr{K}, U_{\overrightarrow{S}}(T) = \overrightarrow{S}
U T$ (for the definition of $\overrightarrow{S} U T$ refer to lecture
7). $U_{\overrightarrow{T}} : \mathscr{H} \to \mathscr{L}(E)$, on any
$S \in \mathscr{H}, U_{\overrightarrow{T}}(S) = S U
\overrightarrow{T}$ $B$ being a continuous bilinear map of $E \times
F$ in $G$, $B$ gives rise to a continuous linear map $B'$ of $\foprod{E}{F}{\pi}$
in $G$ which in its turn can be extended to a continuous linear map of
$\fohprod{E}{F}{\pi}$ in $G$. $\tilde{B}$ denotes this extended map. 
\end{theorem}

\begin{proof}
First, assuming the existence of a bilinear map satisfying (1) and
(2), we shall prove the uniqueness of the map. Suppose there are two
bilinear\pageoriginale maps $\underset{B}{U^1}$ and
$\underset{B}{U^2}$ satisfying (1) and (2). $\mathscr{H}$ and
$\mathscr{K}$ being nuclear, they have the approximation property and
$\foprod{\mathscr{H}}{E}{\pi} =\foprod{\mathscr{H}}{E}{\varepsilon}$. 
Therefore $\fohprod{\mathscr{H}}{E}{\pi} =
\mathscr{H}(E),\fohprod{\mathscr{K}}{F}{\pi} = \mathscr{K}(F)$, because
$\mathscr{H}(E)$ and $\mathscr{K}(F)$ are complete. Hence the sets
$\mathscr{H} \otimes E$ and $\mathscr{K}\otimes F$ are dense in
$\mathscr{H}(E)$ and $\mathscr{K}(F)$. On the product of the sets
$(\mathscr{H} \otimes E)\times (\mathscr{K} \otimes F)$ the bilinear maps
that we define are well determined because of (1). Now the separate
continuity of both $\underset{I}{U^1}$ and $\underset{B}{U^2}$ gives
$\underset{B}{U^1} = \underset{I}{U^2}$ on the whole of
$\mathscr{H}(E)\times \mathscr{K}(F)$. 

Now we shall prove the existence of a bilinear map satisfying (1) and
(2).

Given a bilinear map $U : \mathscr{H}\times \mathscr{K} \to \mathscr{L}$
hypocontinuous with respect to bounded sets we have already seen how
to define a bilinear map $\mathscr{H}(E) x \mathscr{K} \to
\mathscr{L}(E)$ hypocontinuous with respect to bounded sets satisfying
certain consistency conditions (see lecture 7). Each
$\overrightarrow{S} \in \mathscr{H}(E) = \fohprod{\mathscr{H}}{E}{\pi}$
defines a continuous linear map $U_{\overrightarrow{S}}: \mathscr{K}
\to \mathscr{L}(E) = \fohprod{\mathscr{L}}{E}{\pi}$ as follows:
$U_{\overrightarrow{S}}(T) = \overrightarrow{S} U
T$. $U_{\overrightarrow{S}}$ gives rise to a continuous linear map
$U_{\overrightarrow{S}}\otimes I_F :\foprod{\mathscr{K}}{F}{\pi}\to
(\fohprod{\mathscr{L}}{E}{\Pi})\otimes F$, which can be extended by 
continuity into a linear map, which also we denote by
$U_{\overrightarrow{S}} \otimes I_F$ of $\fohprod{\mathscr{K}}{F}{\pi}
\to (\fohprod{\mathscr{L}}{E}{\pi})\fohprod{}{F}{\pi}$;
that is to say, $U_{\overrightarrow{S}}\otimes I_F
:\fohprod{\mathscr{K}}{F}{\pi} \to\fohprod{\mathscr{L}}{}{\pi}
(\fohprod{E}{F}{\pi})$ is a continuous linear map. Since $\mathscr{L}$
is nuclear, we have  
$$
\mathscr{L} \hat{\otimes} (\fohprod{E}{F}{\pi}) =
\mathscr{L}(\fohprod{E}{F}{\pi}).
$$
$U_{\overrightarrow{S}} \otimes I_F : \mathscr{K}(F) \to
\mathscr{L}(\fohprod{E}{F}{\pi})$ is a continuous linear
map. Now $\tilde{B} :\fohprod{E}{F}{\pi} \to G$ is a
continuous linear map. Hence the map $I_{\mathscr{L}} \varepsilon
  \tilde{B}$ is a continuous linear map of $\mathscr{L}(\fohprod{E}{F}
{\pi}) \to \mathscr{L}(G)$. We define $\overrightarrow{S}
\underset{B}{U} \overrightarrow{T}$ to be the element
$(I_{\mathscr{L}} \varepsilon \tilde{B})((U_{\overrightarrow{S}} \otimes
I_F)(\overrightarrow{T}))$. We shall now show that $\underset{B}{U}
: \mathscr{H}(E) \times \mathscr{K}(F) \to
\mathscr{L}(G)$,\pageoriginale which is trivially bilinear, has the
  following properties: 
\begin{itemize}
\item [(i)] If $S \in \mathscr{H}, T \in \mathscr{K},
  \overrightarrow{e} \in E$ and $\overrightarrow{f} \in F, S
  \overrightarrow{e} \underset{B}{U} T \overrightarrow{f} = (S U T)\break
  B(\overrightarrow{e}, \overrightarrow{f})$:
\item [(ii)] If $\overrightarrow{S}$ remains in a bounded set in
  $\mathscr{H}(E)$, and $\overrightarrow{T} \to 0$ in
  $\mathscr{K}(F)$, then $\overrightarrow{S} \underset{B}{U}
  \overrightarrow{T}$ tends to $0$ uniformly in $\mathscr{L}(G)$; and 
\item [(iii)] If $\overrightarrow{S} \to 0$ in $\mathscr{H}(E)$ and
  $\overrightarrow{T}$ is fixed in $\mathscr{K}(F)$, then
  $\overrightarrow{S} \underset{B}{U} \overrightarrow{T} \to 0$ in
  $\mathscr{L}(G)$. 
\end{itemize}

We remark that proving (ii) and (iii) is more than proving separate
continuity.

\noindent {\bf Proof of (i).} Let $\overrightarrow{T} = T \overrightarrow{f}, T
\in \mathscr{K}$ and $\overrightarrow{f} \in F$. Then for any
$\overrightarrow{S} \in \mathscr{H}(E), (U_{\overrightarrow{S}}
\otimes I_F) (\overrightarrow{T})$ is nothing but $(\overrightarrow{S}
U T) \otimes \overrightarrow{f}$. If $\overrightarrow{S} = S
\overrightarrow{e}$, we see that 
$$
\overrightarrow{S} U T = (S \overrightarrow{e}) U T = (S U T)
\overrightarrow{e}.
$$
Hence $\hspace{1cm} (U_{\overrightarrow{S}}\otimes I_F)
(\overrightarrow{T}) = (S U T) \overrightarrow{e} \otimes
\overrightarrow{f}\quad \text{if}\quad \overrightarrow{S} = S \overrightarrow{e},
\overrightarrow{T} = T \overrightarrow{f}. \hspace{1cm}$
\begin{align*}
\text{Hence } (I_{\mathscr{L}} \varepsilon \tilde{B})
((U_{\overrightarrow{S}} \otimes I_F) (\overrightarrow{T})) &= (S U T)
\tilde{B} (\overrightarrow{e} \otimes \overrightarrow{f})\\
&= (S U T) B(\overrightarrow{e}, \overrightarrow{f}). 
\end{align*}
{\bf Proof of (ii).} Let $S$ remain in a bounded set of
$\mathscr{H}(E)$ and $\overrightarrow{T} \to 0$ in
$\mathscr{K}(F)$. We know that if $T \to 0$ in $\mathscr{K}$ and
$\overrightarrow{S}$ remains in a bounded set of $\mathscr{H}(E),
\overrightarrow{S} U T \to 0$ in $\mathscr{L}(E)$ uniformly, that is
to say $U_{\overrightarrow{S}}: \mathscr{K} \to \mathscr{L}(E)$ is an
equicontinuous set of linear maps when $\overrightarrow{S}$ lies in a
bounded set of $\mathscr{H}(E)$. Hence the set of operators
$U_{\overrightarrow{S}} \otimes I_F :\fohprod{\mathscr{K}}{F}{\pi} 
\to \mathscr{L}\fohprod{(E)}{F}{\pi}=
\mathscr{L}(\fohprod{E}{F}{\pi})$, $\overrightarrow{S}$ in a bounded
set of $\mathscr{H}(E)$, is an equicontinuous set. Hence if
$\overrightarrow{T} \to 0$ in $\mathscr{K}(F),
(U_{\overrightarrow{S}}\otimes I_F) (\overrightarrow{T}) \to 0$
uniformly. $I_{\mathscr{L}} \varepsilon \tilde{B}$ is a fixed,
continuous, linear map of $\mathscr{L}(\fohprod{E}{F}{\pi})$ 
in $\mathscr{L}(G)$ and hence $(I_{\mathscr{L}} \varepsilon
\tilde{B})((U_{\overrightarrow{S}} \otimes I_F) (\overrightarrow{T}))
\to 0$ uniformly in $\mathscr{L}(G)$ when $\overrightarrow{S}$ remains
in a bounded set of $\mathscr{H}(E)$.

\noindent {\bf Proof of (iii).} First\pageoriginale we show that if
$T$ remains in a bounded set $A$ of $\mathscr{K}$ and
$\overrightarrow{f}$ remains in a fixed ball, say $\parallel
\overrightarrow{v} \parallel \leq \alpha$ of $F$ and if
$\overrightarrow{S} \to 0$ in $\mathscr{H}(E)$, $\overrightarrow{S}
\underset{B}{U} T \overrightarrow{f}$ tends to $0$ in $\mathscr{L}(G)$
uniformly. If we show that $(U_{\overrightarrow{S}} \otimes I_F) T
\overrightarrow{f} \to 0$ in $\mathscr{L}(\fohprod{E}{F}{\pi})$
uniformly, the above result will 
follow. Now $(U_{\overrightarrow{S}} \otimes I_F) (T
\overrightarrow{f}) = (\overrightarrow{S} U T) \otimes
\overrightarrow{f}$. $\overrightarrow{S} U T \to 0$ in
$\mathscr{L}(E)$ uniformly if $\overrightarrow{S} \to 0$ in
$\mathscr{H}(E)$ because $T$ remains bounded in $\mathscr{K}$
(theorem \ref{chap7:thm7.1}), and since $\overrightarrow{f}$ remains
in a bounded sets,
$(\overrightarrow{S} U T) \otimes \overrightarrow{f} \to 0$ uniformly
as $\overrightarrow{S} \to 0$ in $\mathscr{H}(E)$. 

Now suppose that $\overrightarrow{T}$ lies in the closure of the
convex, stable, envelope $\Gamma (A, B_\alpha)$ of the product of a
bounded set $A$ in $\mathscr{K}$ and of a bounded set $B_\alpha$ in
$F$, the closure being taken in the sense of $\mathscr{K}(F)$. Let
$\overrightarrow{S} \to 0$ in $\mathscr{H}(E)$. We shall show that
$\overrightarrow{S} \underset{B}{U} \overrightarrow{T} \to 0$. Since,
convex, stable, closed neighbourhoods form a fundamental system of
neighbourhoods of $0$ in $\mathscr{L}(G)$, it is sufficient to prove
that $W$ being a closed disc of $\mathscr{L}(G)$ which is a
neighbourhood of $0$, there exists a neighbourhood $N$ of $0$ in
$\mathscr{H}(E)$ such that $\overrightarrow{S} \in N$ implies that
$\overrightarrow{S} \underset{B}{U} \overrightarrow{T} \in W$. Any
$\overrightarrow{T} \in \overline{\Gamma(A, B_\alpha)}$ can be got as
$\lim \overrightarrow{T}_j$, $\overrightarrow{T}_j$ being a filter of
sets in $\Gamma(A, B_\alpha)$. To any neighbourhood $W$ of $0$ in
$\mathscr{L}(G)$ there corresponds a neighbourhood $N$ of $0$ in
$\mathscr{H}(E)$ such that $\overrightarrow{S} \underset{B}{U}
\overrightarrow{T} \in W$ for every $\overrightarrow{S} \in N$ and
$\overrightarrow{T} \in A \otimes B_\alpha$. Since $W$ is convex,
stable we have $\overrightarrow{S}$ $\underset{B}{U}
\overrightarrow{T}_j \in W$ for any $\overrightarrow{T}_j \in
\Gamma(A, B_\alpha)$. Now if $\overrightarrow{T} = \lim,
\overrightarrow{T}_j$ with $\overrightarrow{T}_j \in \Gamma(A,
B_\alpha)$, we have $\overrightarrow{S} \underset{B}{U}
\overrightarrow{T}_j \to \overrightarrow{S} \underset{B}{U}
\overrightarrow{T}$. Hence $\overrightarrow{S} \underset{B}{U}
\overrightarrow{T} \in \overline{W}$. But since $W$ is closed, $W =
\overline{W}$. Hence $\overrightarrow{S}\; \underset{B}{U}\;
\overrightarrow{T} \in W$.

Now, suppose $\overrightarrow{T}$ is an element of
$\mathscr{K}(F)$. Since $\mathscr{K}$ is nuclear, bounded sets in
$\mathscr{K}$ are relatively  compact. Hence $\mathscr{K}_\delta' =
\mathscr{K}_c'$. We have assumed\pageoriginale that $\mathscr{K}_\delta'$ is
nuclear. $\mathscr{K}(F) = \mathscr{L}_\varepsilon(\mathscr{K}_c', F)
= \mathscr{L}_\varepsilon(\mathscr{K}_\delta', F)$. If
$\overrightarrow{T} \in \mathscr{K}(F)$, since $\mathscr{K}_\delta'$
is nuclear, and $F$ a Banach space, we see that $\overrightarrow{T}$
is a nuclear map. Hence $\overrightarrow{T} =
\sum\limits_n\lambda_nk_n \otimes \overrightarrow{f}_n, k_n \in
\mathscr{K} = (\mathscr{K}_\delta')'$ with $\sum\limits_n |\lambda_n
|< \infty, \; k_n$ being an equicontinuous set of $\mathscr{K}$
considered as the dual of $\mathscr{K}_c'$ and $\overrightarrow{f}_n$
lying in a bounded set of $F$. This just means that
$\overrightarrow{T} \in \overline{\Gamma(A, B)}$ with $A$ and $B$
bounded sets in $\mathscr{K}$ and $F$ respectively, closure in
$\mathscr{L}(\mathscr{K}_\delta', F) = \mathscr{K}(F)$. Hence
$\overrightarrow{S} U \overrightarrow{T} \to 0$ as $\overrightarrow{S}
\to 0$ in $\mathscr{H}(E)$. 

Thus we have seen that the bilinear map
$\underset{B}{U}:\mathscr{H}(E)\times \mathscr{K}(F) \to \mathscr{L}(G)$
satisfies (i), (ii) and (iii). In particular, $\underset{B}{U}$ is
separately continuous. Now, let $\underset{B}{U'}$ be defined by
$\overrightarrow{S} \underset{B}{U'} \overrightarrow{T} =
(I_{\mathscr{L}} \varepsilon \tilde{B}) (I_E \otimes
U_{\overrightarrow{T}}) (\overrightarrow{S})$. $\underset{B}{U'}$ can
be proved to satisfy 
\begin{itemize}
\item [$(i)'$] $S \overrightarrow{e} \underset{B}{U'} \; T
  \overrightarrow{f} = (S U T) B (\overrightarrow{e},
  \overrightarrow{f})$ when $S \in \mathscr{H}, T \in \mathscr{K},
  \overrightarrow{e} \in E$ and $\overrightarrow{f} \in F$. 
\item [$(ii)'$] When $\overrightarrow{T}$ remains in a bounded subset
  of $\mathscr{K}(F)$ and $\overrightarrow{S} \to 0$ in
  $\mathscr{H}(E), \overrightarrow{S} \underset{B}{U'}
  \overrightarrow{T} \to 0$ uniformly in $\mathscr{L}(G)$.
\item [$(iii)'$] If $\overrightarrow{T} \to 0$ in $\mathscr{K}(F)$ and
  $\overrightarrow{S}$ is a fixed element of $\mathscr{H}(E)$, the
  $\overrightarrow{S} \underset{B}{U'} \overrightarrow{T} \to 0$ in
  $\mathscr{L}(G)$.
\end{itemize}

In particular, $\underset{B}{U'}$ is separately continuous.   

\noindent Thus $\underset{B}{U}$ and $\underset{B}{U'}$ are separately
continuous bilinear maps satisfying the condition (1) of the
theorem. But from the uniqueness of such a map which we have proved
already, it follows that $\underset{B}{U} = \underset{B}{U'}$. 

Now, if we combine (ii), (iii), $(ii)'$, and $(iii)'$ we see that
$\underset{B}{U} = \underset{B}{U'}$ satisfies also the conditions (3)
and (4) stipulated in the theorem.
\end{proof}

\noindent {\bf A\pageoriginale particular case of the multiplicative product.}

Let $\mathscr{H}, \mathscr{K}$ and $\mathscr{L}$ be three nuclear
spaces with nuclear strong duals. Let $E$ be a Banach algebra. If $U :
\mathscr{H} \times \mathscr{K} \to \mathscr{L}$ is a bilinear map
hypocontinuous with respect to the bounded subsets of $\mathscr{H}$
and $\mathscr{K}$, by taking for $B$ the multiplication in $E$ we get
bilinear map $\underset{B}{U} : \mathscr{H}(E) \times \mathscr{K}(E)
\to \mathscr{L}(E)$ which is hypocontinuous with respect to the
bounded subsets of $\mathscr{H}(E)$ and $\mathscr{K}(E)$.  

