\chapter{}\label{chap4}%%% 4

This\pageoriginale chapter contains results related with the following kind of
problem: given a discrete group  of continuous transformations, use
information on the behaviour of a set of generators to prove that the
action of the group is proper. The solution of such a problem is based
here on a Lemma (Lemma \ref{chap4:lem2}) related to the methods of Chapter
\ref{chap3} as well as to a Theorem of Weil on discrete subgroup of
Lie groups (A. Weil \cite{key1}, \cite{key2}).   

\section{Criterion for proper action for groups of isometries}%sec 1

Let $G$ be a topological group acting on a \textit{connected} space $X$.

Let $S \subset G$ and $A \subset X$ be such that 
\begin{enumerate}[(i)]
\item $e \in S$
\item $S \subset G(A|A)$
\item $s, s' \in  S,  A\cap sA  \cap s' A \neq \phi$ imply $s^{-1} s' \in S$.
\end{enumerate}

Note that these conditions imply $S=S^{-1}$. 

On the product space $G \chi A$, consider the relation  $\mathscr{R}$
defined as follows: 
$$
(t,a) \mathscr{R} (t', a') \text{ if } ta= t' a' \text{ and } t^{-1}t'
\in S. 
$$

It is easily seen that $\mathscr{R}$ is an equivalence relation. Let
$y = (G \chi A)/ \mathscr{R}$, and let $q: G \chi A \to Y$ be the
canonical mapping. The mapping $(t,a) \rightsquigarrow ta $ of $G
\chi A$ into $X$ induces a mapping $f: Y \to X$ such that the diagram 
\[
\xymatrix{G \times A \ar[dr]  \ar[rr]^q && Y \ar[dl]^f  \\
& X &   
}
\]
is\pageoriginale commutative. $G$ acts $Y$ in the usual manner, and $f$ commutes
with the action of $G$. Our object is to give sufficient conditions on
$S$ and $A$ so that $f$ is a homeomorphism. 

\setcounter{lem}{0}
\begin{lem}\label{chap4:lem1}%lem1
  If $A$ is connected and $S$ generates $G$, then $Y$ is connected.
\end{lem}

\begin{proof}
  Let $Y_o$ be the connected component of $Y$ containing  $q(e \chi A)$. 
\end{proof}

Since $G q( e \times A)=Y$, we need only verify that $Y_o$ is
$G$-stable . This is clear since, for any $s \in  S,  q( e \times A)
\cap sq(e \chi  A) \neq \phi$, and $S$ generates $G$.    

\begin{lem}\label{chap4:lem2}%lem 2
  Suppose that: (i) there exists a $G$ invariant metric $d$ on $X$;
  (ii) $S$ is a neighbourhood of $e$ in $G$;  (iii) there exists a
  $\varrho > 0$ such that for any $a \in  A$ there is an $s \in S$ with
  $\bigg\{ x \in X | d(x,a) < \varrho \bigg\} \subset sA$.  
  
  Then $G(Int A)=X$, and $f: Y \to X$ is a covering . 
\end{lem}

\begin{proof}
  Since $X$ is connected and $G(\Int A)$ open in $X$, we will have
  $G(\Int A)=X$ if we show that $G(\Int A)$ is closed in $X$. Now let $x \in
  \overline{G(\Int A)}$. Then  there exist $t \in  G$ and a $a \in \Int
  A$ such that $d(x, ta)< \varrho$, i.e., $d(t^{-1} x,a)< \varrho$. Hence
  there is an $s \in S$ such that $t^{-1}x \in  s(\Int A)$; this implies
  that $x \in G(\Int A)$. 
\end{proof}

It follows from $G(\Int A)=X$ that $f$ is onto. We now prove that $f: Y
\to X$ is a covering. 
\begin{enumerate}
\item \textit{$f$ is locally injective}. It is sufficient to prove that
  $f$ is injective in a neighbourhood of any $q (e,a)$, with $a \in
  \Int A$. Now let $U$ be a neighbourhood of $e$ in $G$ such that
  $U^{-1} U \subset S$. Since $q^{-1}(q(U \chi \Int  A)) =
  \bigcup\limits_{s \in S} (Us \chi (A \cap s^{-1} \Int A)), q (U \chi
  \Int A)$\pageoriginale is a neighbourhood of $q(e \chi \Int A)$. We assert that $f$
  is injective on $q(U \chi\break \Int A)$. In fact let $(t,a),(t',a')\in U
  \chi \Int A$, and let $f(q(t,a))=f(q(t',a'))$, i.e. $ta =t'a'$. Then,
  since $t^{-1}t' \in U^{-1} U \subset S$, we have $q(t,a)= q(t',
  a')$. 
\item \textit{$f$ has a local section}. For any $x_o \in X$, let
  $B=\bigg\{ x \in X| d(x_o,x)< \varrho/_{2}  \bigg\}$, and let $N=
  \bigg\{t \in G \bigg | tA \supset B \bigg\}$. For each $t \in N$, we
  have a section  $\sigma_t :  B \to Y$ of $f$ defined by $\sigma_t(z)
  =q(t,t^{-1}z), z \in B$. Note that, for $t,t'$ in $N$, we have
  either $\sigma_t = \sigma_{t'}$ or $\sigma_t (B) \cap \sigma_{t'}(B)
  = \phi$. 
\item \textit{$f$ is a covering}. Let $x_o  \in X$. In view of 1)
  and 2), it is sufficient to show, with the notation  of 2), that
  $f^{-1}(B)= \bigcup\limits_{t \in N} \sigma_t (B)$. 
\end{enumerate}

Let $q(r,a) \in  f^{-1}(B)$, i.e. $ra \in B$. Let $s \in S$ be such
that $sA \supset \bigg\{x \in X \big| d(x,a)< \varrho \bigg\}$. Then
$rsA \supset \bigg\{ x \in X \big| d(x,ra) < \varrho \bigg\} \supset
B$, which means $rs \in N$. Then $\sigma_{rs}(ra) =q (rs, s^{-1}r^{-1}
ra)= q (rs,s^{-1}a)= q(r,a)$. 

This proves Lemma \ref{chap4:lem1}.

\setcounter{thm}{0}
\begin{thm}\label{chap4:thm1}%the 1
  Let $G$ be a topological group acting isometrically on a connected
  metric space $X$. Let $A$ be a  connected subset of $X$, and $S$ a
  neighbourhood of $e$ in $G$ generating $G$ such that the following
  conditions are satisfied: 
  \begin{enumerate}
  \item $S \subset G(A/A)$, 
  \item $s,s' \in S, A \cap sA \cap s' A\neq \phi$ imply  $s^{-1}s' \in S$;
  \item there exists $a \varrho > 0$ such that for any $a \in A$, we
    have $s \in S$ with $sA \supset \bigg\{ x \in X \big| d(x,a) <
    \varrho \bigg\}$; 
  \item any connected covering of $X$ admitting a section over $A$ is trivial. 
  \end{enumerate}
\end{thm}

Then\pageoriginale $S=G(A|A)$. If moreover $S$ is relatively compact in $G$, then
the action of $G$ on $X$ is proper. 

\begin{proof}
  By Lemmas \ref{chap4:lem1} and \ref{chap4:lem2}, $f: Y \to X$ is a
  connected covering; this 
  covering admits a section over $A$, given by a $\rightsquigarrow
  q(e,a)$. Hence $f$ is bijective. 
\end{proof}

We now prove that $G(A|A) \subset S$. Let $t  \in  G(A|A)$. Then there
exists $a, a' \in A$ such that $ta= a'$, i.e., $f(q(t, a))=
f(q(e,a'))$. Since $f$ 
is bijective, we have $q (t,a)= q(e,a')$, i.e., $t \in S$. 

The second assertion of the theorem now follows from the remark after
Lemma 1, Chapter \ref{chap3}. 

\section{The rigidity of proper actions with compact quotients}%sec 2

Let $G$ be a locally compact group, and $X$ a locally compact
metrisable space, We denote by $\mathscr{C}= \mathscr{C}(G \chi X,X)$
the space of all continuous mapping of $G \chi X$ into $X$, provided
with the compact open topology, and we denote by $\underline{M}$ the
subset of $\mathscr{C}$ consisting of continuous actions of $G$
on $X$ (with the induced topology). Also, we denote by
$\underline{M}_P$ the set of  \textit{proper} actions of $G$ on $X$,
and by $\underline{M}_I$ the set of isometric actions (i.e., an action
of $G$ on $X$ belongs to $\underline{M}_I$ if there exists a metric on
$X$ invariant under this action). By Theorem \ref{chap1:thm2}, chapter
\ref{chap1}, we have 
$\underline{M}_P \subset \underline{M}_I$ (at least when $X$ is
connected). 

\begin{thm}\label{chap4:thm2}%the 2
  Let $G$ be a locally compact group, and $X$ a connected, locally
  connected, locally compact metrisable space. Suppose that $X$ has a
  compact subset $K$ such that any connected covering of $X$ admitting a
  section over $K$ is trivial. Let $m_o \varepsilon \underline{M}_p$ be
  such that $m_o \  X$ is compact.   
\end{thm}

Then\pageoriginale there exists a neighbourhood $W$ of $m_o$ in $\underline{M}_I$
such that  
\begin{enumerate}[a)]
\item $W \subset \underline{M}_p$
\item for every $m \varepsilon W,  m \ X$ is compact
\item the action of $G$ on $W \times X$ defined by $(s, (m,x))
  \rightsquigarrow (m,m(s,x))$ is proper, 
\item if $G$ is a Lie group, then $\ker m \subset\ker m_o$ for any $m
  \varepsilon W$ (here, for any $m, \ker m = \bigg\{ g \varepsilon G |
  m (g,x) = x \text{for every} x \varepsilon X \bigg\}$. 
\end{enumerate}

\medskip
\noindent \textbf{Proof of a) and b)}. With the assumptions of the
theorem, we shall 
  prove that there exists a compact connected subset $A$ of $X$
  containing $K$, a relatively compact open neighbourhood $S$ of $e$
  in $G$, and a neighbourhood $W$ of $m_o$ in $\underline{M}_I$ such
  that, for every $m \varepsilon W, A$ and $S_m= S \cap G_m (A|A)$
  satisfy the conditions of Theorem \ref{chap4:thm1}. Then $W$ will satisfy a)
  and b). 

Let $C$ be a compact subset of $X$ such that $m_o (G,C)= X$. Since $X$
is locally connected, locally compact and connected, there exists a
connected compact neighbourhood $A$ of $C$ containing $K$. Let $B$
be an open relatively compact set in $X$, containing $A$. We set $S =
G_{m{_o}} (B|B)$. Clearly $S$ is a symmetric open relatively compact
neighbourhood of $e$ in $G$. For $m \varepsilon M$, we set $S _m = S
\cap G_m (A|A)$. Clearly $S_m$ is also a neighbourhood of $e$.   

\begin{enumerate}[(i)]
\item \textit{There exists a neighbourhood} $W_1$ {\em of} $m_o$ in
  M \textit{such, that, for any} $m \varepsilon W_1$, \textit{and any}
  $s,s' \varepsilon S_m, A \cap m (s, A) \cap m (s', A) \neq \phi $
  \textit{implies} $s^{-1} s' \varepsilon S_m$. 

   In fact, $L = \bar{S}^2 - S$ is compact, and $m_o (L,A) \cap A =
   \phi$. Hence there exists a neighbourhood $W_1$ of $m_o$ in
   $\underline{M}$ such that, for any\pageoriginale $m \in W, m(L, A) \cap
   A=\phi$. It is easily verified that $W_1$ has the required
   property. 

\item \textit{There exists a neighbourhood} $W_2$ \textit{of} $m_o$
  \textit{in $M$ such that, for any} $m \in W_2, S_m$ \textit{generates}
  $G$. 

  Let $C'$ be a compact neighbourhood  of $C$ contained in $\Int A$
  Then $T=G_{m_o}(C'|C)$ generates $G$; in fact, since $TC$ is a
  neighbourhood of $C$, and $T \supset G_{m_o}(C| C)$, the proof of
  Lemma 1, Chapter \ref{chap3} is valid. We shall now
  show that $T\subset 
  S_m$ is $m$ sufficiently close to $m_o$. 

  For each $t \in  T$, we have $a$ $c(t) \in C$ such that $m_o(t,  c(t))
  \in C' \subset \Int A$. Thus there exists a compact neighbourhood
  $V(t)$ of $t$ such that $m_\circ(V(t), c(t)) \subset \Int A$. Let
  $W(t)$ be a neighbourhood of $m_\circ$ in $\underbar{M}$  such
  that $m(V(t), c(t)) \subset \Int A$ for any  $m \in W(t)$. 

  Since $T$ is compact, there exists a finite subset $T'$ of $T$ such
  that $T \subset \bigcup\limits_{t \in  T'}V(t)$. If we take $W_2=
  \bigcap\limits_{t \in T'}W(t)$, we clearly have $T \subset G_m(A|A)$
  for any $m \in W_2$. Since $T \subset S$, we have $T\subset S_m$;
  hence $S_m$ generates $G$, for every $m \in W_2$.  

\item \textit{There exists a neighbourhood} $W_3$ 
  \textit{of} $m_o$ \textit{in M such that, for any} $m \in W_3,
  (S_m,\Int A) \supset A$. 
\end{enumerate}

We know that $m_o(S, \Int A) \supset A$. Thus for any $a \in A$, there
exists an $s_a \in S$ such that $m_o(s_a, \Int A) \ni a$.  Let $U_a$ be
a compact neighbourhood of a \textit{ in A} such that $U_a \subset m_o
(s_a, \Int A)$, i.e. $m_o (s^{-1}_a, U_a) \subset \Int A$. Since $A$ is
compact, we have a finite subset $F$ of $A$ such that
$\bigcup\limits_{a \in F} U_a =A$. For each $a \in F$, let $W_a$ be
neighbourhood of $m_o$ in $\underline{M}$ such that $m(s^{-1}_a, U_a)
\subset \Int A$ for every $m \in W_a$. Clearly $W_3 = \bigcap\limits_{a
  \in F} W_a$\pageoriginale has the required property. 

We now set $W =\underline{M}_1 \cap W_1 \cap W_2 \cap W_3$, and assert
that, for any $m \in W, A$ and $S_m$ satisfy the conditions of Theorem
\ref{chap4:thm1}. In view of the above considerations, our assertion  will follow
if we verify condition 3) of Theorem \ref{chap4:thm1}. Take any $m \in W$, and
choose an invariant metric $d$ on $X$ with respect to $m$. By (iii)
above, we have $A \subset \bigcup\limits_{s \in S_m}m(s, \Int A)= U$
say. Let  $\lambda= d(A, X-U)$, and let $A'=\bigg\{x \in X |
d(x,A) \le \lambda/2 \bigg\}$. Then for the $\rho$ of condition
$3)$ we can take the minimum of $\lambda/2$ and the Lebesgue number of
the covering $\{ m(s, \Int A)\}_{s \in S_m}$ of $A'$. 

Thus a) and b) are proved.

\medskip
\noindent{\textbf{Proof of c).}} 
We shall prove that the action of $G$ on $W \chi X$ is proper, where
$W$ is as above.  Since $W \chi X$  is Hausdorff,  it is enough to
verify  the condition $(P)$ of Chapter \ref{chap1} for the points $(m_o, x_1),
(m_o,x_2), x_1,x_2 \in X$. Now, given $x_1, x_2 \in X$, we may assume
by enlarging the $A$ of the above considerations if necessary, that
$x_1, x_2 \in \Int A$ . For this A we obtain  a neighbourhood $W'
\subset W$ of $m_o$ such that  $G_m(A|A)\subset S$ for every $m \in
W'$. $W' \chi A$ is a neighbourhood of $(m_o,x_1)$ and $(m_o,x_2)$
such that $G(W' \chi A | W' \chi A) \subset S$. Since $S$ Since $S$ is
relatively  compact, this proves c).  

\medskip
\noindent{\textbf{Proof of d).}} Let $K = \ker m_o$. Since the action
$m_o$ is proper, $K$ is  a \textit{compact} normal subgroup of $G$. Let
$q: G \to G/K$ be the canonical homomorphism. Let $V$ be an open
neighbourhood of $q(K)$ which contains no nontrivial subgroup of
$G/K$. Now $F=\bar{S}-q^{-1}(V)$ is a compact set in $G$ such that
$\ker m_o  \cap F=\phi$. hence there exists a neighbourhood $W' \subset
W$ of\pageoriginale $m_o$ such that ker $m \cap F=\phi$ for all $m \in W'$.  Since,
for any $m \in W$, we have ker $m \subset G_m (A|A) \subset S$, we
have, for $m \in W'$, $	q(\ker m) \subset V$, i.e. ker $m \subset
K$. This proves $d)$. 

\setcounter{rem}{0}
\begin{rem}%rem 1
  It is not in general true that every $m_o  \in \underline{M}_P$has a
  neighbourhood in $\underline{M}$ which is contained in
  $\underline{M}_P$, even if we suppose that $m_o \ X$ is compact. For
  instance, let $G=  \mathbb{Z}, X= \mathbb{R}$, and let $m_o \in
  \underline{M}_P$ be defined by $m_o(n,t) = t+n$. For any  $a \in
  \mathbb{R}$, let $\varphi_a : \mathbb{R} \mathbb{R}$ be a
  differentiable function such that  
  \begin{align*}
    \varphi_a (t) &=
    \begin{cases}
      1, &t \le a\\
      0, & t > a+2
    \end{cases}\\
    \varphi'_a (t)  &\ge -1.
  \end{align*}
\end{rem}    

 Let $m_a$ be defined by $m_a (x,t)= t+n \varphi_a(t)$. It is easy to
 check that $m_a \in \underline{M}$. It is also  clear that if a is
 large enough, $m_a$ is arbitrarily close to $m_o$.  However, $m_a \not\in
 \underline{M}_P$, since under this action $\mathbb{Z}$ leaves every
 point $\ge a+2$ fixed. 

\begin{rem}%rem 2
  In Theorem \ref{chap4:thm2}, the condition that $m_o  x$ is compact is essential
  For instance, let $G= \mathbb{Z},X= GL(2, \mathbb{C}). \mathbb{Z}$
  operates on $X$ by left multiplication, through the homomorphism $h$
  defined by  
  $$
  h(1)=
  \begin{pmatrix}
    1&1\\
    0&1
  \end{pmatrix}
  $$
\end{rem}  

    This action is proper. The action of $\mathbb{Z}$ on $X$ defined
    by the homomorphism $h_n: \mathbb{Z} \to GL(2, \mathbb{C})$ which
    maps $1$ on $\begin{pmatrix} e^{2 \pi i/n} &1\\ 0
      &1 \end{pmatrix}$ is arbitrarily near this action if $n$ is
    large, but is not proper. Note that all the above actions are in
    $\underline{M}_I$, since $GL(2, \mathbb{C})$ has a left-invariant
    metric.  
    
\section{Discrete subgroup of Lie group. Witt's Theorem}%sec 3

\begin{thm}[A. Weil \cite{key1}]\label{chap4:thm3}%the 3
  Let\pageoriginale $\Gamma$ be a discrete group, $G$ a connected
  Lie group,and $h_o: \Gamma \to G$ a homomorphism such that  
  \begin{enumerate}[\rm(i)]
  \item $\ker h_o$ is finite;
  \item $h_o (\Gamma)$ is discrete;
  \item $h_o(\Gamma)^G$ is compact.
  \end{enumerate}
\end{thm}

Then there exists a neighbourhood $W$ of $h_o$ in Hom\,$( \Gamma,G)$
(with the  finite open topology), such that for any  $h \in W$, (i),
(ii) and (iii) hold with $h_o$ replaced by $h$. 

\begin{proof}
  We may identify Hom\,$(\Gamma, G)$ with a subspace of
  $\underline{M}$. Then, since there exists a left invariant metric on
  $G$, Hom\,$(\Gamma,G) \subset \underline{M}_I$. Also, (i) and (ii)
  imply that $h_o \in \underline{M}_p$;  further $\prod_1(G)$  is
  finitely generated. Hence we may apply Theorem \ref{chap4:thm2} to obtain Theorem
  \ref{chap4:thm3}. 
  
  Let $G$ be a Lie group, and $X$  a defferential manifold. By a
  \textit{differentiable}(one-parameter) \textit{family of actions}of
  $G$ on $X$ we mean a differentiable mapping $m: \mathbb{R} 
  \chi G \chi X \to X$ such that for each  $t \in \mathbb{R},m_t:  (s,x)
  \rightarrow m(t,s,x)$ is an action of $G$ on $X$.  
\end{proof}

\begin{thm}\label{chap4:thm4}%the 4
  Let $G$ be a Lie group, and $X$ a connected differentiable manifold
  such that $\prod_1(X)$ is finitely generated. Suppose given a
  differentiable  family  $m:\mathbb{R} \times G \times X \to
  X$ of actions  of $G$ on $X$ such that $m_t \in \underline{M}_I$ for
  every $t \in \mathbb{R}$, and suppose that  $m_o$   is proper and
  $m_o^X$ compact. Then there exists a neighbourhood $W$ of $0$ in
  $\mathbb{R}$, and for each $t  \in  W$ a differentiable  automorphism
  $a_t$ of $X$ such that 
  $$
  m_t(s,x)= a_t(m_o(s, a^{-1}_t(x)))
  $$
  for\pageoriginale every $x \in  X, s \in G, t  \in W$.
\end{thm}

\begin{proof}
  In view of Theorem \ref{chap4:thm2}, we can find a neighbourhood $W_1$ of $0$ in
  $\mathbb{R}$, and  a compact set $A$ in $X$, such that the action of
  $G$ on $W_1 \times X$ defined by $s(t,x)=(t,m_t(s,x))$ is proper, and
  such that $m_t(G,A)=X$ for every $t \in W_1$. Then there exists a
  $G$-invariant Riemannian metric on $W_1 \times X$ (Theorem
  \ref{chap1:thm2}, Chapter \ref{chap1}). Let $p: W_1 \times X \to
  W_1$ be the natural projection, and let $H$ 
  be the vector-field on $W_1 \times X$ orthogonal to the fibres of $p$
  such that $p^{T_H}= \dfrac{d}{dt}$.  It is easily seen that  $H$ is
  $G$-invariant. 
\end{proof}

Let the differentiable mapping
$$
\varphi: \bigg\{ \tau \in \mathbb{R}|\tau | < \in \bigg\} x W_2
\times U \to W_1 \times X 
$$
be the local one-parameter group generated by the vector-field $H$ in
a neighbourhood $W_2 \times U$ of $\{0\} \times A$ in $W \times
X$. Since $H$ is $G$-invariant and $G(W_1 \times A) =W_1 \times X,
\varphi$ can be extended to a differentiable mapping 
$$
\varphi : \{ \tau \in \mathbb{R}|\tau|< \in \} \times W_2 \times X
\to  W_1 \times X 
$$
by means of the equation
\begin{equation}
  s \varphi_\tau(t,x) = \varphi_\tau(t,m_t(s,x)), t \in W_2. \tag{*}
\end{equation} 

Since $H$ projects on the vector- field $\dfrac{d}{dt}$, we have 
$$
\varphi_\tau(0,x)=(\tau, a_\tau(x))
$$
where $a_\tau: X \to X$ is a diffeomorphism. Using the fact that
$\varphi_\tau(0, m_0\break (s,x))= s \varphi_\tau(o,x)$\pageoriginale (which is $(*)$
with $t =0$), we see that the $a_t, |t| < \in$, satisfy the
conditions of the theorem. 

For other  applications, we need the following modification of Theorem
\ref{chap4:thm1}.  

\begin{thm}\label{chap4:thm5}%the 5
  Let $G$ be a discrete group acting isometrically on a connected
  locally connected, simply connected metric space $X$. Let $C$ be a
  connected compact subset of $X$, and $S$ a finite subset of $G(C|C)$
  such that 
  \begin{enumerate}[\rm(i)]
  \item $e \in S$,
  \item for any  $s, s' \in S, C \cap s C \cap s' C \neq \phi$ implies
    $s^{-1} s' \in S$,  
  \item $SC$ is a neighbourhood of $C$, 
  \item $S$ generates $G$.
  \end{enumerate}
  This $S =G(C|C)$, the action of $G$ on $X$ is proper, and  $GC=X$.
\end{thm}
 
\begin{proof}
  Since $C$ is compact, and $S$ is finite, there exists a neighbour $V$
  of $C$ such that $s, s' \in S, C \cap s C \cap s' C=\phi$, imply $V
  \cap sV \cap s' V= \phi$. Let  $A$ be the connected component of $V
  \cap SC$ which contains $C$. Since  $X$ is locally  connected, $A$ is a
  neighbourhood of $C$. $A$ and $S$ satisfy the conditions of Theorem
  \ref{chap4:thm1}. In  fact, it is clear we need only check the condition 3) of
  Theorem \ref{chap4:thm1},  and for the $\varrho$ of that condition we can take
  $d(C,X-A)$. Since $C \subset A \subset SC$, the  assertions of
  Theorem \ref{chap4:thm5} follows from Theorem \ref{chap4:thm1} (and
  Lemma \ref{chap4:lem1}).  
\end{proof}
 
\begin{thm}[E. Witt \cite{key1}]\label{chap4:thm6}%the 6
  Let $G$ be the group generated by the set $\{r_1,
  \ldots, r_n\}$ with the relations $(r_ir_j)^{p_{ij}}=e,1 \le i,j
  \le n$, where the\pageoriginale $p_{ij}$ are integers satisfying 
  $$
  P_{ii} =1, p_{ij} =p_{ji} > 1 \text{ if } j \neq i, 1 \le i, j \le n.
  $$ 
\end{thm} 

Then $G$ is finite if and only if the matrix $\left(- \cos
\dfrac{\prod}{p_{ij}}\right)$ is positive definite. 

\begin{proof}
  Let $(e_i)_{1 \le i \le n}$ denote the canonical basis of
  $\mathbb{R}^n$, and $B$  the symmetric bilinear form on $\mathbb{R}^n$
  defined  by $B(e_i, e_j)=- \cos \dfrac{\prod}{p_{ij}}$. We  define the
  \textit{standard representation} of $G$ in $\mathbb{R}^n$ by setting 
  $$
  r_i e_j= e_j - 2B(e_i,e_j)e_i.
  $$
  Clearly, $B$ is invariant under $G$.
\end{proof}  

\begin{enumerate}[a)]
\item \textit{$G$ is finite  $\Rightarrow B$ is positive definite}.
  
  We first prove that $B$  is non-degenerate. Let $N= \bigg\{ x \in
  \mathbb{R}^{n^n} |B(x,y)=0 \text{ for every } y \in
  \mathbb{R}^n\bigg\}$. Since $N$ is G-stable  and $G$ is finite,
  there exists a $G$-stable supplement $N'$ to $N$. Now, for every $i,
  r_i|N=$ identity, and $r_i$ is not identity on $\mathbb{R}^n$, hence
  there exists $a y_i \in N'$ such that $r_i y_i \neq y_i$,
  i.e., $B(e_i, y) \neq 0$. Since $r_i y_i -y_i=- 2B(e_i,y)e_i$, we
  have $e_i \in N'$. Hence $N'  = \mathbb{R}^n$, i.e. $N=0$. 
  
  Then prove the positive-definiteness, we consider any non-trivial
  irreducible G-subspace $L$ of $\mathbb{R}^n$. We see as above that
  there  exists an $e_i \in L$.On the other hand, there exists on $L$
  a $G$-invariant positive definite bilinear form, say $B_o$, and
  (Since  $L$  is irreducible) $a \lambda \in \mathbb{R}$ such that
  $B|L= \lambda B_o$. Since $B(e_i, e_i)=1$, we must have $\lambda >
  0$. Hence $B|L$ is positive definite. Since $B$  is non-degenerate,
  it follows\pageoriginale  that $B$ is positive definite. 

\item \textit{$B$ positive definite $\Rightarrow G$ is finite}.  (The
  following proof is based on Buisson [1]). Let $C \subset
  \mathbb{R}^n$ be defined by 
  $$
  C =\bigg\{x \in  \,\mathbb{R}^n \big| B(x,e_i) \ge 0 ~\text{ for
    every }~ i \bigg\}. 
  $$
\end{enumerate}

We shall prove the following  statements by induction on $n$. 
\begin{enumerate}[1)]
\item $G$ is finite
\item $GC= \mathbb{R}^n$
\item If $s \in G$ and $c \in C$ are such that $sc \in C$, then
  $sc=c$, and in fact $s$ belongs to the subgroup of $G$ generated by
  the $r_i$  belonging to $G(c)$. 
\end{enumerate}     

   If $n=2,B$ is automatically positive definite, and the above
   statements are easily verified.  Thus  let $n \ge 3$, and let us
   assume that 1), 2) and 3) are true for $n-1$. 
   
   Let $\sum = \bigg\{x \in  \mathbb{R}^n \big| B(x,x)=1\bigg\}$ ,
   and let $A= \sum \cap C$. For  each $i$, let $G_i$ be the subgroup
   of $G$ generated by $r_1, \ldots, r_{i-1}, r_{i+1}, \ldots ,  r_n$ Note that
   $N_i = \sum\limits_{j \neq i} \mathbb{R} e_j$ is $G_i$-stable, and
   that the  representation of $G_i$ in $N_i$ thus obtained  is the
   standard representation of $G_i$ in $\mathbb{R}^{n-1}$. By
   induction, each $G_i$ is finite, hence the set $S=
   \bigcup\limits_{1 \le  i \le n} G_i$ is finite. Clearly $e \in S$,
   and $S=S^{-1}$. Since $G_i$ operates trivially  on the orthogonal
   complement $N'_i$ of $N_i$ with respect to $B$, we have $G_i
   \subset G(A|A)$, hence $S\subset G(A|A)$. Also, $S$ generates $G$,
   since $r_i \in S,
   1 \le i \le n$. 
   
\begin{lemma*}
  If\pageoriginale $a \in  A$ and $s \in S$ are  such that $s a \in
  A$, then $sa=a;$ 
  in  fact $s$ belongs to the subgroup $G_a$ of  $G(a)$ generated by the
  $r_i \in G(a)$. 
\end{lemma*}   

\begin{proofofthelemma*}
  Let $s \in G_i$,a nd let $a=b+b'$, where  $b \in N_i, b' \in
  N'_i$. Then both $b$ and $sb$ belong to $N_i \cap(C+N'_i) \subset
  C_i$; hence $C_i = \bigg\{ y \in  N_i \big| B(y, e_j) \ge 0\text{ for
    every } j \neq i \bigg\}$. Hence, by induction,  $s$ belongs to
  the subgroup of 
  $G_i$ generated  by the  $r_j$ which leave  $b$ (and hence $a$) fixed .   
\end{proofofthelemma*}   

The lemma  implies in particular that if $s, s' \in S$ are such that
   $A \cap sA \cap  s' A \neq \phi$, then $s^{-1} s'  \in G_a$ for
   some $a \in  A$. But clearly $G_a \subset G_i$ for some $i$, and
   then $s^{-1} s' \in G_i \subset S$. 
   
   We prove finally that $SA$ is a neighbourhood of $A$ in $\sum$.
   Since, for any $ a \in A, G_a \subset S$, it is sufficient to check that
   $G_a A$ is a neighbourhood of a in $\sim$, or equivalently that
   $G_aC$ is a neighbourhood of a in $\mathbb{R}^n$.  
   
   Let $L = \sum\limits_{r_j  \in G(a)} \mathbb{R} e_j$, and $L'$ its
   orthogonal complement with  respect to $B$.  Clearly $a \in  L'$,
   and there exists a neighbourhood $V$ of a in $\mathbb{R}^n$ such
   that 
   $$
   V \cap C= V \cap  \bigg\{x \in \mathbb{R}^n\big| B(x, e_i) \ge 0
   \text{ for all } e_i \in L \bigg\}. 
   $$
   
   Then if
   $$
   C_a = \bigg\{y \in L | B(y, e_i) \ge 0 \text{ for all } e_i \in
   L \bigg\}, 
   $$ 
   we have $V \cap C= V \cap(C_a+L')$. Assuming as we may that $V$ is
   $G_a$-stable, we\pageoriginale have therefore 
   $$
   G_a(V \cap C) = V \cap (G_a  C_a +L').
   $$
   
By induction, $G_a C_a=L$, hence $G_a(V \cap C)=V$, and $G_a C$ is a
neighbourhood of $a$. 
   
Now, for the action of $G$ on $\sum$, all the conditions of Theorem
   $5$  are satisfied for $A$ and $S$; note that $A$ is connected and
   $\sum$ simply connected. Thus the action  of $G$ on $\sum$ is
   proper. Since $\sum$ is compact, this means that $G$
   finite. Moreover, since $GA= \sum$, we have  $GC=
   \mathbb{R}^n$. This proves the statements 1) and 2); 3) follows
   from the lemma since $S= G(A|A)$. Hence the proof of the theorem is
   complete. 

\begin{remark*}
  The proof of Theorem \ref{chap4:thm6} shows that show when $\left(-\cos
  \frac{\Pi}{p_ij}\right)$ is positive definite, the standard
  representation of $G$ in $\mathbb{R}^n$ is faithful. 
\end{remark*}   
