\chapter{}\label{chap2}%%% 2
 
 The\pageoriginale aim of this chapter is the description of the
 action of a group  of transformations in the neighbourhood of an
 orbit. For proper actions, the existence of ``slices'' reduces the
 general case to the case of a neighbourhood of a fixed point. For
 proper and differentiable actions, a description can be given in
 terms of linear representations of compact groups (Koszul \cite{key1}, Mostow
 \cite{key1}, Montgomery-Yang \cite{key1}, Palais \cite{key1}).  

\section{Slices}%%% 1 
Let $G$ be a topological group, and $H$ a subgroup 
 acting on a apace $Y$. We can then construct in a natural manner a
 topological space $X$ on which $G$ acts. In fact, we let $H$ operate
 on $G \times Y$ (on the right) by setting 
  $$
 (s, y)t=(st, t^{-1}y); s \in G, y \in Y, t \in H,
 $$
 and take $X=(G \times Y)/ H$. If $q:G \times Y \to X$ is the natural
 mapping, then the left action of $G$ on $X$ is defined by $sq (r,
 y)=q(sr, y)$. 
  
 Note that in the above situation, if we set $A=q(e \times Y)$, we
 have (i) $G(A | A)A=A$, (ii) $G(A | A)=H$, (iii) the mapping $(s, a)
 \rightsquigarrow sa$ of $G \times A$ into $X$ is open. The property
 (iii) follows trivially from the fact that the mapping $y
 \rightsquigarrow q(e, y)$ of $y$ onto $A$ is a homeomorphism.  

 Conversely, let $G$ be a transformation group of a space $X$, and $A$
 a subset of $X$ such that $G(A | A)A=A$. Then it is clear that $H=G(A
 |A)$ is a subgroup of $G$. By the above considerations, $G$ acts on
 $(G \times A)/ G(A | A)$. Let $F:G \times A \to X$ be the map $F(s,
 a)=sa$, and $q:G \times A \to G \times A/ G(A | A)$ the natural
 map. Then there is a map $f : G \times A/ G(A |A) \to X$ such that
 $F=f 0 q$. It is easy to verify that the mapping $f$ is injective,
 and commutes with the actions of $G$ on $X$ and $(G \times A)/ G(A
 |A)$. 

\begin{defi*}
  Let\pageoriginale $G$ be a group of transformations of a space $X$. $A$
  \it{slice} is a subset $A$ of $X$ such that (i) $G(A |A)A=A$, (ii)
  the mapping $(s, a) \rightsquigarrow sa $ of $G \times A$ into $X$ is
  open.   
\end{defi*} 
 
Condition (ii) means that the mapping $f :(G \times A)/_{G(A | A)}
 \to X$ defined above is a homeomorphism onto the $G$-stable open set
 $GA$ in $X$. 

\begin{defi*}
  Let $G$ be a transformation group of a space $X$.  A \textit{slice A
    at a point} $x \in X$ is a slice such that (i) $x \in A$, (ii) $G(A |
  A)= G(x)$. 
\end{defi*} 
Note that a slice need not be a slice at any of its points.
 
\begin{defi*}
  Let $G$ be a transformation group of a space $X$. A \textit{normal
    slice} is a slice $A$ such that $G(y)=G(A|A)$ for every $y \in
  A$. A \textit{regular} point of $X$ is a point at which a normal
  slice exists  
\end{defi*} 

A normal slice is characterised by the property that it is a slice at
 each of its points. It is clear that if $A$ is a normal slice, the
 orbit of each $s \varepsilon A$ is naturally homeomorphic to
 $G/_{G(x)} = G/_{G(A|A)}$, and the $G$-stable open set $GA$ is
 naturally homeomorphic to $A \times G/_{G(A|A)}$. Since, for every $s
 \varepsilon G$, $sA$ is also a normal slice, it is clear that the set
 of regular points is a $G$-stable open subset of $X$. 

\begin{examples*}
  \begin{enumerate}
    \renewcommand{\labelenumi}{\theenumi)}
  \item  Let $G$ be a topological group, and $H$ a subgroup acting on a
    space $Y$, and $q$ the natural mapping $G \times Y \to (G \times Y)/
    H$. Then $q(e \times Y)$ is a slice for the natural action of $G$ on
    $G \times Y/H$. In fact, this motivated our definition of slices. 

  \item  Let $G$ act without fixed points on a space $X$. Then for any $x
    \in X$, any slice at $x$ is a normal slice. If $X \to G/X$ is a
    locally trivial principal\pageoriginale fibre space, normal slices
    in $X$ are precisely the images of open sets in $G \backslash
    X$ by continuous sections.
  \end{enumerate}
\end{examples*} 


\section{General Lemmas}%%% 2

\setcounter{lem}{0} 
\begin{lem}\label{chap2:lem1}%%%% 1
  Let $G$ be a topological group, and $H$ a subgroup of $G$ acting
  continuously on a space $Y$. Let $X=(G \times Y)/H$; we suppose that
  $G$ acts on $X$ in the natural way. Let $q: G \times Y \to X$ be the
  natural mapping. Then we have; 
  \begin{enumerate}[\rm (i)]
  \item for any $B \subset Y, G(q(e \times B)| q (e \times B))=H(B |
    B)$,

  \item for any $y \in Y, G(q(e \times y))=H(y) $

  \item $B \subset Y$ is a slice in $Y$ if and only if $q(e \times B)$
    is a slice in $X$. 

  \item  $B \subset Y$ is a normal slice if and only if $q(e \times
    B)$ is a normal slice. 

  \item if $y \in Y$ is regular, then $q(e \times y)$ is regular;

  \item if $G$ is locally compact and $H$ is closed, and if $H$ acts
    properly on $Y$, then $G$ acts properly on $X$. 
  \end{enumerate}
\end{lem} 

\begin{proof}
  It is easy to verify (i), and (ii) is a special case. Also, once
  (iii) is proved, (iv) and (v) follows from (i) and (ii). We
  shall prove (iii) and (vi). 
\end{proof}
  
\medskip
\noindent{\textbf{Proof of (iii).}}
 Let $B \subset Y$ be a slice for the action of $H$. We shall prove 
  that the natural mapping $(G \times B)/ G(B | B) \to X$, which is
  clearly one-one and commutes with the action of $G$, is actually an
  open mapping; since $B$ is a slice for $(G \times B)/ G(B | B)$, it
  will follow that $q(e \times B)$ is a slice for $X$. 

To prove that the mapping $(G \times B)/G(B|B)\to X$ is open, it is
plainly sufficient to prove that for any neighbourhood $V$ of $e$ in
$G$, and any neighbourhood $W$ in $B$ of any $b \in B$, the
saturation by $H$ of $V \times W$\pageoriginale is a neighbourhood  of
$e \times b$ in $ G \times  Y $. Now, if  $U$ is a  symmetric  neighbourhood of
$e$ in $G$ such that  $ U^2  \subset V $, it is clear  that $( V
\times W )  H $ contains the neighbourhood  $ U X \{ (  H \cap U ) W
\} $.  

The converse assertion in  (iii) is easy to  verify. 

\medskip
\noindent{\textbf{Proof of (vi).}}
  Suppose that  $ H $  acts properly on $ Y $. Let $ q ( s,y ), q (
  s',y' ) \in  X $. Let $ V, V' $ be neighbourhoods of  $ y,y'$
  respectively  in $ Y$ such that  $H(V|V')$  is relatively
  compact. For any compact neighbourhoods $U, U'$ of $s,s' $ respectively in $
  G,  q ( U \times V)$, $q( U'\times V' ) $  are  neighbourhoods of $
  q ( s,y)  ~ q( s', y')   $ in  $X$. We assert that  $ G ( q ( U \times
  V )  \mid q ( U' \times V' )) $ is relatively  compact in $ G $. In
  fact, it is easily verified that $ G ( q ( U \times V ) \mid  q ( U'
  \times V' ))  <  U'  ~ H ( V' \mid V ) U^{-1} $. 

\begin{lem}\label{chap2:lem2}%lemma 2
  Let the topological topological group $G$  act  on two  spaces  $X$
  and $Y$, and let $ f: X \rightarrow Y $  be a continuous mapping
  commuting with the actions  of $G$. Then for any slice  $B$ in  
$Y, f^{-1} (B)$  is a slice  in  $X$.  
\end{lem}
  
\begin{proof}
  We may assume that  $f^{-1} (B)$ is non- empty. Since  $f$
  commutes with the actions of $ G $, we have  
  $$
  G ( f^{-1} (B) \mid ( f^{-1} (B))  =  G ( B \mid B ), G ( B \mid B ) 
  f^{-1} (B)= f^{-1} (B), 
  $$
  hence  we need  only prove  that the mapping $ G \times  f^{-1} (B)
  \rightarrow  X $  is  open. For this it is sufficient to prove that
  for  any $ x \in  f^{-1} (B) $, and for  any neighbourhoods $U$ of $e$
  in  $G$ and  $V$ of  $x$ in $ X, U ( V \cap f^{-1}  ( B)) $ is a
  neighbourhood of  $x$ in $X$. To do this, we  choose a neighbourhood
  $U'$ of $e$ in $G$, and neighbourhood $V'$ of $x$ in $X$, such that $
  U'V' \subset V $. Since  $B$ is  slice  in  $ Y, U' B $ is a
  neighbourhood  of  $ f (x) $. Since\pageoriginale $f$  commutes
  with the action of 
  $G, U' f^{-1} (B) \supset  f^{-1} (U' B)$  and  hence is
  neighbourhood of $x$ in $X$. It is easily verified that $U(V \cap
  f^{-1} (B))$ contains the neighbourhood $V'\cap(U' f^{-1} (B))$. 
\end{proof} 

\begin{remark*}
  If $B$ is  a slice at  $y =  f (x) \in X, f^{-1}  (B)  $ need  not
  be a  slice at $x$. 
\end{remark*} 

\section{Lie groups acting with compact isotropy groups}\label{chap2:sec3}%%% 3

We now  consider the case of a \textit{Lie group} $ G $ acting  on a
space $X$ such that  \textit{the isotropy groups are all compact}. We
wish to study the function  associating to any $ x \in X $  the
conjugacy class of  $G (x)$. 
  
  We denote  by $ \mathscr{C} = \mathscr{C} ( G ) $  the set of all
  conjugacy classes of compact subgroups of  $G$. For  $ T, T' \in
  \mathscr{C} $, we write  $ T < T' $ if  there exist  $ H \in T, H'
  \in T' $ such that  $ H \subset H' $. Since a compact Lie group
  cannot have proper Lie subgroups  isomorphic to it, we  see that  $
  T < T' < T $ implies $ T = T' $. For any $ x \in X $, we denote  by
  $ \tau (x) $ the  conjugacy class of  $ G ( x)$. Since, for  any $ x
  \in X $ and  $ s \in G $, $ G ( sx )  = sG ( x )s^{-1}, \tau $  can
  in fact be regarded  as a mapping of  $ G^{/X} $ into  $ \mathscr{C}
  (G) $. 

\begin{lem}\label{chap2:lem3}%lemma 3
  Let  $G$ be a Lie  group acting  on a topological  space  $X$ such
  that all the isotropy groups are  compact. Let  $x \in X $, and
  suppose there exists  a slice at  $x$. Then, (1)  there exists a
  neighbourhood  $V$ of  $x$  such that  $ \tau (y) < \tau ( x ) $ for
  every $ y \in V; ( 2 ) x $ is  regular if and  only if  $ \tau $ is
  constant in a neighbourhood of  $x$. 
\end{lem} 
 
\begin{proof}
  Let  $A$ be a slice (resp. normal slice)  at $x$. Then for any $ y
  \in A $, $ G (y) \subset G ( A \mid A ) = G( x ) ( resp. G (y) = G
  (x)) $. Hence it is clear that  $\tau (y) < \tau (x) ( resp. \tau
  (y) = \tau (x))$ for  all $ y$  belonging to   the
  neighbourhood\pageoriginale  $GA$ of  $x$.  
\end{proof}  
  
Now  suppose  that  $ \tau $  is constant in an open neighbourhood
$ V $ of  $x$. If $A$ is any slice at  $x$, we have, for  any $ y
\in  V \cap A$,  $G(y) \subset G(x)$ and  $ \tau (y) = \tau (x) $, which
implies  $ G (y) = G (x) $. Thus  $ V \cap A $ is a normal slice at
$x$, hence $x$ is regular. 

\begin{remark*}
  We have also  proved  that  if  $x \in  X$ is  regular, there exists a
  neighbourhood  $V$  of $x$  such that  \textit{every slice at}  $x$
  contained in  $V$ is normal. 
\end{remark*}  

\section{Proper differentiable action}%%% 4
 
In this article, we study the case of  a Lie Group $G$ acting
differentiably and properly on a  paracompact differentiable manifold
$X$ of dimension $n$. Note that, in this case the  orbits $Gx$ are
closed  submanifolds of $X$, naturally diffeomorphic with the $G/_{G
  (x) }$. 

\begin{lem}\label{chap2:lem4}%lemma 4
  Let $G$ be a Lie group acting properly and differentiably on a
  paracompact differentiable manifold of dimension $n$. Then for any $x \in X
  $, there exist a representation of $G(x)$ in a finite-dimensional real
  vector space $N$, and a differentiable mapping $f$ of a $G (x)$
  -stable neighbourhood $B$ of  $ 0 \in N $ in to $X$ such that  
  \begin{enumerate}[(i)]
  \item $f (0) = x$

  \item $f$ commutes with the actions of $G(x)$.

  \item $\dim N +  \dim  Gx = \dim X$

  \item $ G f (B) $ is open in $X$, and  the mapping $h : (s, b)
    \leadsto  sf (b)$ of  $ G \times B $ into $X$ passes down to  a
    diffeomorphism $\psi$ of $(G \times B)_{/  G (x)}$ onto $G f (B)$. 
  \end{enumerate}  
\end{lem}  
 
\begin{proof}
  By\pageoriginale Theorem \ref{chap1:thm2}, Chapter \ref{chap1}, we
  can choose a $G$-invariant   Riemannian  
  metric on $X$. Let $T(x)$ be the tangent bundle of $X$, and  let
  $\Omega $ be an  open neighbourhood of the zero section of $T(x)$ on
  which the exponential mapping  $ \exp : \Omega \rightarrow  X $  is
  defined  (Nomizu \cite{key1}). Since  $G$ acts  isometrically on  $X$,
  we may assume that $\Omega $ is stable for the induced action of $G$
  on $T(X)$; we denote  this  action by $(s, u) \leadsto s^T u$, $s \in
  G, u \in T( X)$. We have the relations 
  $$
  s \exp u = \exp ( s^T u );  s \leq G,  u \in T(X) 
  $$ 
    and
  $$
  d(x, \exp  u ) \leq u ;  ~ x \in X, u \in T_x (X), 
  $$
  where $d$ is the distance on $X$ induced by the Riemannian metric, and
  $|| u ||$ is the  length of $u$. 
\end{proof}   
    
Now let  $x \in X $, and let $ T_x(Gx)$ denote the  subspace of
$T_x (X)$ tangential to  $ Gx $. $G(X)$ leaves $T_x (X)$ invariant,
and clearly $T_x (Gx)$ is stable under this  action. Since $G$ acts
isometrically on  $X$, the  orthogonal complement  $N$ of  $T_x ( Gx)$
in $T_x (X)$ is also stable under $G_x$:   
$$
N =  \{ u \in  T_x (X) \mid  < u,v >  ~ =  0  \text{ for all }  v \in 
T_x ( G_x) \}.   
$$
  
\noindent
Clearly this  $N$ has  property (iii). Now for any $r > 0$, let $
B_r = \{ u \in N  \big|   || u || < r \} $.  Then $ B_r $ is $ G(x)
$-stable, and is  contained in $\Omega $ if $r$ is small. We set $f =
\exp \big| B_r $. Clearly $f$ has the properties (i) and (ii) of
the  lemma. We shall now show  that if $r$ is small enough, (iv) is
also valid with $B = B_r $. 

We\pageoriginale have as  usual the  commutative diagram
\[
\xymatrix{
\ar[dr]_q G \times B_r \ar[rr]^h && X\\
& (G \times B_r)/G(x) \ar[ru]_\psi & 
}
\]
 
\noindent
Here, $ ( G \times B_r, q, ( G \times B_r )_{/_G (x)}) $  is a
(locally trivial) differentiable  principal bundle, so that $\psi$
is  differentiable. Since $h$ is obviously of maximal rank at  $(e,
0)$, $\psi$ is of maximal rank at $q (e, 0)$. Since $ \dim ( G \times
B_r ) /_{G (x)} = \dim X $,  it follows that t $ \psi $  is a
diffeomorphism in  a  neighbourhood of  $q ( e, 0 ) $. Hence if $ W $
is a  suitable neighbourhood of  $ G (x) $ in  $ G $, and  $r$ is
small enough, we have that  $ \psi $  is a diffeomorphism  of  $ q (
W \times B_r )  $  onto an  open set in  $X$ and, if $ U = \{ z \in
X \big| d ( z, x ) < 2r \} $, $G ( U \mid U ) \subset W $ (Lemma
\ref{chap1:lem1}, Chapter \ref{chap1}). We set  $ B_r = B $, and
assert that  $ \psi $ 
is  a diffeomorphism of  $(G \times B) /_{ G (x)} $  onto an open
subset of  $X$. First, since $ \psi $ commutes with the  actions  of
$G$, and  $ G ( q ( W \times B ))  =  q ( G \times B ) $, it is
clear that  $ \psi $  is  everywhere of maximal rank. We shall now
show that it is injective. Equivalently we shall show that for $s,
s' \in G $  and  $ u, u' \in B, h ( s,u ) = h ( s', u' )  $  implies
$ q ( s,u ) =  q  ( s', u' ) $. In fact, let $ h (s, u ) = h ( s',
u')$, i.e., $s ~ \exp u = s' \exp u'$, or $s^{-1} s'  ~ \exp u' = \exp
u$. 

Then 
\begin{align*}
  d (x, s^{-1} s' x)  &\leq  d ( x, \exp u )  ~ + d ( s^{-1} s' x, \exp u ) \\
  & < 2r,  
\end{align*}
since $d( s^{-1} s' x, \exp u )  = d ( s^{-1} s' x,  ~ \theta^{-1} s'
\exp u' ) ~ = d ( x, \exp u')$.  
  
Hence\pageoriginale $  s^{-1} s' \in W $. Since $ \psi $  is  one -
one on  $ q ( 
W \times B ) $, it follows  easily that  $ q (s, u ) =  q ( s', u' )
$. 

In what follows, the hypothesis and  notation of Lemma
\ref{chap2:lem4} are retained. 

\setcounter{thm}{0}
\begin{thm}\label{chap2:thm1}%them 1
  For  every  $ x \in X $, there exists a  slice at  $x$. 
\end{thm}  

\begin{proof}
  With the notation of  Lemma \ref{chap2:lem4}, $f(B)$ is a slice at $x$. In fact
  $\psi : (G \times B)/_{G (x)} \rightarrow  G  f(B)$ is  a
  diffeomorphism of $(G \times B) /_{ G (x)} $ onto the $G$- stable open
  set $ G f(B)$  in $X$, commuting with the  action of $G$.  Since $ q (
  e \times B ) $  is a  slice in  $ ( G \times B ) /_{G (x)} $, it
  follows that $ h ( e \times B )  =  f (B) $  is a  slice in  $ X $. 
\end{proof}
  
\begin{thm}\label{chap2:thm2}%them 2
  $A$  point $ x \in X $  is regular if and  only the  action of  $ G
  (x) $ in  $ T_x (X) / _{T_x (G_x) } $ is trivial. 
\end{thm}  

\begin{proof}
  If we choose a  $G$-invariant Riemann metric on  $X$, and  use  the
  notation of   Lemma \ref{chap2:lem4}, we have to prove that $x$  is  regular if
  and only if the  action of $ G (x) $ on  $N$  is  trivial. Now we
  know, by (v) of  Lemma \ref{chap2:lem1},  and the remark after
  Lemma \ref{chap2:lem3}, that 
  $X $  is a regular point of  $X$  if and  only if, for sufficiently
  small $ \rho , B_\rho = \{ u \in B  \big|  || u || < \rho \}  $  is
  a normal slice for the action of  $G (X) $ in  $N, i. e $. if and
  only if  $ G (x) $ acts trivially on $N$. 
\end{proof}  

\begin{thm}\label{chap2:thm3}%them 3
  The set of  regular points is dense in  $X$.
\end{thm}  

\begin{proof}
  We proceed by  induction  on $ \dim  X $. If  $ \dim X = 0 $, every
  point of  $ X $ is regular. Now let  $ \dim X =  n >0 $, and  assume
  the theorem proved for all manifolds of   dimension  $ <  n $. Take
  any $ x \in X $. Since  the  theorem is of a  local nature, we may
  assume, with the notation of\pageoriginale Lemma \ref{chap2:lem4},
  that $ X = ( G   \times B )  
  /_{G (x)} $. Then, by $ (v) $ of Lemma \ref{chap2:lem1}, it is  sufficient to
  prove that the set of regular points in $ B$  for  the action of  $ H
  = G (x) $  on  $B$ is dense at  $ 0 \in B $.  
\end{proof}  

  For  any $ \rho, 0 <  \rho < r ( = $  radius of  $B ) $, let  $
  S_\rho $ be the  sphere  $ \{ v \in B \big| || V || = \rho \}
  $. Clearly  $ S_\rho $  is  $H$-stable. It is clear from Theorem 
  \ref{chap2:thm2}  that a  $ V \in S_\rho $  is regular for  the
  action of  $H$ 
  on $ B $  if  and only if  it is regular for the action of  $H$ on
  $S_\rho $. Since  $ \dim S_\rho < \dim B \leq \dim X $, it follows
  by the  induction  hypothesis that the  set of  $H$- regular  points
  of $ B $  is  dense in  $S_\rho $. Since   this is  true for all $
  \rho > 0 $, our assertion  follows. (We also note that if a  $v
  \in N $ is regular for  the action  of  $H$, so is  $ \lambda v$,
  for every $  \lambda > 0 $.)   

\begin{thm}\label{chap2:thm4}%them 4
  Let $G$ be a Lie group acting properly and  differentiably on a
  paracompact differentiable manifold of  dimension $n$. Let $ \tau :
  X  \rightarrow  \mathscr{C}  ( G)  $ be the function assigning to any
  $x$   in  $X$ the conjugacy class of  $ G (x) $ in $ G $, and let
  $\mathcal{R}$ be the  set  of  regular points  of $ X$. Then, 
  \begin{enumerate}[\rm(i)]
  \item  every $x \in X$ has a neighbourhood  $V$ such that  $ \tau
    (V) $  is a  finite set; 

  \item if  $ _G \backslash ^X $  is connected, $ _G \backslash
    ^\mathcal{R}$  is  connected; 

  \item if  $ _G \backslash ^X $ is  connected,  $ \tau $ is  constant
    on $ \mathcal{R} $; 

  \item if  $ _G \backslash ^X $ is  connected, a point $ x \in X $  is
    regular if  and only if  $ \tau (x) $ is minimal  (i.e. $\tau
    (x)  < \tau ( y)$ for every  $ y \in X$.  
  \end{enumerate} 
\end{thm}  

\medskip
\noindent \textbf{Proof of (i).} 
  We use  induction on $ \dim  X $; if  $ \dim X = 0 $, the  statement 
  is trivial. Let $ \dim X > 0 $, and assume that  (i) is proved
  for all manifolds\pageoriginale of  dimension $ < n $. On account of
  the  local 
  nature of (i), we may assume, with the notation of Lemma  \ref{chap2:lem4},
  that  $ X = ( G \times  B ) /_{G (x)}$. We assert now that  $\tau
  (X)$ is a finite set. In fact let  $ 0 < \rho < r $  (= radius  of
  $B$), 
  and let  $S = \{ u \in B \big| || u || = \rho \}$. $S$ is  stable for
  the action of $G(x)$ on $B$, and $\dim S< \dim X$. By the induction
  hypothesis and the compactness of $S$, we conclude that $\tau (S)$ is
  a finite set. However, since $G(x)$ operates linearly on $ N $, we
  have, for any $ u \in N $ and any $ \lambda  \in  \mathbb{R} - \{ 0 \}
  $, $ \tau ( u )  = \tau ( \lambda u  ) $. Hence $ \tau (B)  = \{  \tau
  (0) \} \bigcup  \tau (S) $. Thus $ \tau (B) $ is finite. By (ii)
  of Lemma \ref{chap2:lem1}, $ \tau ( q ( e \times B )) = \tau (B)
  $. Finally, since 
  $ G q ( e \times B )  =   X $, $ \tau (X)  =  \tau ( q ( e \times B ))
  $, hence $ \tau (X) $ is  finite as asserted. 
  
\medskip
\noindent \textbf{Proof of (ii).} 
  Again, we use  induction on  $ \dim X; $  if  $ \dim = 0 $, $
  \mathcal{R} =  X $, and (ii)  holds trivially. Let 
  $\dim X > 0$, and  assume (ii) proved for  manifolds of dimension 
  $< n$. We shall prove that every point of  $ _G \backslash^X $ has a
  neighbourhood  $ V $  such that  
  $V \cap  _G \backslash ^\mathcal{R}$  is connected. Since 
  $_G \backslash ^\mathcal{R} $  is dense in  
  $_G \backslash ^X $, it follows easily that if $ _G \backslash ^X $
  is  connected $ _G \backslash ^\mathcal{R} $  is also
  connected. Again, we may  assume, with the notation of   Lemma \ref{chap2:lem4},
  that $ X = ( G \times B ) \backslash _{G (x) }$; and we shall prove
  that  $ _G \backslash ^\mathcal{R} $  is connected.  

Let $\mathcal{R}' $ be the  set of regular points of  $B$ for  the
action  of  $ H = G (x) $. We assert that $_H  \backslash
\mathcal{R}' $  is  connected. If  $ \dim B = 1 $, or  if  $ x $ is
a regular point, this  is trivially verified. Thus let $ \dim B  >
1 $, and  $x$ be not  regular. Let $r$  be the radius of  $ B $, and
let $ S = \{ u \in B \big |  || u || = r/2  \} $. $S$  is $H$-
stable and connected. Hence, by induction, $ _{H} \backslash
\mathcal{R}" $  is  connected, where $ \mathcal{R}" $  is  the  set
of  regular points\pageoriginale of $S$. Since $ \mathcal{R}' =  \bigcup
\limits_{0 < \lambda < 2 } \lambda  \mathcal{R}'' $, it follows
easily that  $ _H  \backslash  \mathcal{R}' $ is  connected. 

Now, $q(e\times \mathcal{R}')$ is a dense  set of  regular
points in the slice $ q ( e  \times  B ) $,hence its in  $ _G
\backslash ^X  $  is  dense in  $ _G  \backslash  ^ \mathcal{R}
$. On the other hand, since $q ( e  \times^{\mathcal{R}})'$  is
contained in the slice  $ q ( e \times   \mathcal{B})'$ at $x$, it is  easy
to  verify that the mapping   $\mathcal{R}' \rightarrow _G
\backslash^X $   obtained by composing  the mappings  
$\mathcal{R}'  \rightarrow q ( e \times   \mathcal{R}' ) $ and  $ q (
e \times   \mathcal{R}' _G \backslash^X $, passes down to a mapping
$ _H \backslash^{\mathcal{R}}{'} \rightarrow _G \backslash ^X
$. Since   $ _G \backslash^ \mathcal{R}{'}$ is connected,   $_G
\backslash^ \mathcal{R} $  thus contains  a dense connected subset,
hence is connected. 

\medskip
\noindent \textbf{Proof of (iii).} 
  Use (ii), and  (ii)  of Lemma \ref{chap2:lem3}.

\medskip
\noindent \textbf{Proof of (iv).}  
  Let ${}_G \backslash^X $ be connected, and let $ y \in X $. By  (i)
  of Lemma \ref{chap2:lem3}, there exists a neighbourhood  $V$ of  $y$ such that 
  $ \tau (z) < \tau (y) $  for every  $ z \in V $. Now $ \mathcal{ R} $
  is  dense in  $ X $, and  $ \tau $, is constant on $  \mathcal{R} $,
  hence we have $ \tau (x)  <  \tau (y) $ for every  $ x \in
  \mathcal{R} $. The converse assertion  (even without  any assumption
  on  $_G \backslash X$) follows from Lemma \ref{chap2:lem3}, since we know that
  there exists a  slice  at every $x \in X$.

 \setcounter{rem}{0}
\begin{rem}%remark 1
  We see from (iv)  of  Theorem \ref{chap2:thm4} that for the proper
  differentiable action of  a Lie group on a connected paracompact
  manifold,  the orbits of regular points  are  of maximal
  dimension. The converse is  not true, even if the Lie  group is
  connected. For  instance, consider the  group  $ G = \text{  SO } (
  3, \mathbb{R})$ of  rotations of the two - sphere, acting on  it self
  by inner automorphisms. Then the regular points are the rotations of
  angle $ \neq 0 $ or $ \pi $; the  isotropy  group at such point is
  the one   parameter group through that point, consisting  of
  rotations about  the  same\pageoriginale axis, and the  orbit is a two
  sphere. For rotation of  angle $ \pi $ the isotropy group has
  \textit{two} connected components  (the identity component being the
  one  parameter group through that point), and  the orbit is a
  projective plane. 
\end{rem} 
 
\begin{rem}%remark 2.
  Let $G$ be a {\em connected} Lie group. For any compact subgroup  $H
  $ of  $ G$, let [ $ H $ ] denote its conjugacy class. Now suppose we
  are  given two conjugacy classes  $ T_1, T_2  \in  \mathscr{C} ( G )
  $. Then in the set  $ \{ [ H_1 \cap H_2 ],  H_1  \in T_1, H_2  \in
  T_2  \} $, there exists  a ( unique )  minimal class for the relation
  $ <  $.  In fact $ G $ acts in the  obvious manner on the connected
  space $ G \backslash  _{H{_1}} \times  G \backslash _{H{_2}}, H_1 \in
  T_1, H_2 \in T_2 $,  and the class we are looking for is the
  conjugacy class of  the isotropy   groups at regular points. Thus,
  given $ T_1, T_2 \in \mathscr{C} ( G ) $  we are able to associate
  with them an element  $T_1 \circ T_2$  of  $ \mathscr{C} (G)$
  characterised by the minimality property. 
\end{rem}  


\section{The discrete  case} %%% 5

Let  $G$ be  a discrete  group, acting properly on  a Hausdorff space
$X$. Then there exists a slice at every point of $X$. In fact for any
$ x \in X $, there  exists an open neighbourhood   $U$ of  $x$  such
that  $G( U \mid U) = G (x)$ (Lemma \ref{chap1:lem1}, Chapter \ref{chap1}).  Since $
G(x)$ is finite, $V = \bigcap\limits_{g \in G(x)} g U$ is an open
neighbourhood of $x$; clearly $G(V|V) = G(x)$ and $V$ is  
$G(x)$- stable. Since  $V$ is an open neighbourhood of $x$, it follows
that it is a  slice at  $x$. 

\begin{remark*}
  In the classical constructions  of  fundamental  domains for a group
  $G$ acting isometrically on a  metric space $ X$, one defines, for any
  $x \in X$, the set 
  $$
  A  = \{ z \in X \mid d ( z, x ) <  d ( z, sx )  \text{  for every }
  s \in G - G (x) \}. 
  $$
\end{remark*}
$A$\pageoriginale has the properties.

\begin{enumerate}[(i)]
 \item $G(A \mid A )  = G ( x ) $,
 \item $G(x) A= A $. 
\end{enumerate}

In fact let $t \in G(A|A)$, and  let $ z \in A $  be such that  $
tz \in A $. If  $t \not\in  G (x) $, we have  
$$
 d ( tz, tx )  >  d ( tz, x ) = d ( z, t^{-1} x )  >  d ( z, x ),
$$
which is impossible since $t$  is an isometry. Thus (i) is
proved, and (ii) is  easily verified. But $A$ is in general
not a slice. However, a slightly different construction  produces a
slice at $x$.  
   
Let $A$  be defined as above. Since $G$ is  discrete  and  acts
properly, $Gx$  is  discrete, hence  $ \lambda =  \inf \limits_{ s
  \in  G-G (x) } $  $ d ( sx, x ) > 0 $. Set  $   V = \{  z \in  X
\mid d ( z,x) <  \lambda  /_{2} \}  $. Clearly  $V $ is  stable under
$ G (x) $. On the other hand $V \subset A $,  hence $ G ( V  \mid  V )
< \subset G ( A|A )  = G (x) $. Since  $ V $ is  open, it follows that
$V$ is  a slice at $x$. 
   
   Our construction of a slice in the differentiable case (Lemma \ref{chap2:lem4})  
   is somewhat similar to the  construction given above, namely,
   the slice  $ q ( e \times B ) $ in Theorem \ref{chap2:thm1}  is the
   intersection of a  neighbourhood  of $ x $ with $ \{ y \in  X \mid
   d ( y, sx )  > d ( y, x)  $ for every  $ s \in  G - G (x) \}$. 
   

\section{}%%%% 6

   Let  $ G $ now  be a  \textit{compact lie group}, action continuously on
   a  \textit{completely regular} topological space. The following
   lemmas reduce  the  problem of  constructing a slice at a point of
   $X$  to  that of the differentiable\pageoriginale case. 

\begin{lem}\label{chap2:lem5}%lemma 5.
  Let $G$ be a compact Lie group. For any closed subgroup $H$ of $G$,
  there exists a representation of $G$ in a finite dimensional real
  vector space $E$, and a $u \in E$, such that $G(u) = H$. 
\end{lem}  

\begin{proof}
  We consider the left regular representation of $G$ in $L^2 (G)$. We
  know then, by the Peter-Weyl theorem, that $L^2 (G) = \sum\limits_{i
    \in I} E_i$, where the $E_i$ are finite dimensional, G-invariant,
  and pairwise orthogonal. 
\end{proof}

Let $q:G \to G/_H$ be the natural mapping, and let $f$ be a continuous
function on $G/_H$ such that $f(z) = 0$ if and only if $z = q(H)$. Let
$g = f ~ o ~ q$, and consider the decomposition $g = \sum g_i, g_i \in
E_i$, of $g$ in $L^2 (G)$. Since $g(y) = 0$ if and only of $y \in H$,
it is clear that $H = G (g)$, the isotropy group of $G$ at $g$. On the
other hand we have $G(g) = \bigcap\limits_{i \in I} G (g_i)$. Since
the $G(g_i)$ are compact Lie groups, we can find a finite subset $J$
of $I$ such that $H = \bigcap\limits_{i \in I} G (g_i)$. For the $E$
and $u$ of the lemma, we can take $E = \sum\limits_{i \in J} E_i, u =
\sum\limits_{i \in J} g_i$.  

\begin{lem}\label{chap2:lem6}%lemma 6
  Let $G$ be a compact Lie group acting on a completely regular space
  $X$. Then for any $x_o \in X$, there exists a finite dimensional
  representation of $G$ in a real vector-space $E$, and a mapping $f: X
  \to E$ commuting with the action of $G$, such that $G (f(x_o)) =
  G(x_o)$. 
\end{lem}

\begin{proof}
  By Lemma \ref{chap2:lem5}, we have a finite dimensional representation of $G$ in a
  real vector space $E$, and a $u \in E$, such that $G(u) =
  G(x_o)$. Hence the continuous mapping $s \rightsquigarrow su $  of $G$
  into $E$ passes down to a mapping of $G/_{G(x_o)}$ into $E$. Since $G$
  is compact, $G/_{G(x_o)}$ is canonically homeomorphic to $Gx_o$ and
  hence we get a continuous mapping $f : Gx_o \to E$
  with\pageoriginale the property 
  $f (sx_o) = su = s f(x_o)$. Since $X$ is completely regular, and
  $Gx_o$ is compact, $f$ cab be extended to a continuous mapping $f^* :
  X \to E$. The required $f$ is now given by $f(x)=\int\limits_{G} s
  f^* (s^{-1} x) ds$, where $ds$, is the Haar measure on $G$ with
  $\int\limits_{G} ds = 1$. 
\end{proof}


\begin{thm}[Mostow \cite{key1}]\label{chap2:thm5}%them 5.
  Let $G$ be a compact Lie group operating on a
  completely regular space $X$. Then there exists a slice at every $x
  \in X$. 
\end{thm}

\begin{proof}
  Let $f : X \to E$ be as in Lemma \ref{chap2:lem6}; thus $f$ commutes with $G$, and
  $G(f(x)) = G(x)$. By theorem \ref{chap2:thm1}, there exists a slice $B$ at $f
  (x)$. Then by Lemma \ref{chap2:lem2}, $f^{-1} (B)$ is a slice. Since $G(f^{-1} (B)
  \big| f^{-1} (B)) = G(f(x)) = G(x), f^{-1}(B)$ is a slice at $x$. 
\end{proof}

\setcounter{rem}{0}
\begin{rem}%remark 1.
  Because of theorem \ref{chap2:thm5}, the considerations of \S
  \ref{chap2:sec3} are valid in the 
  case of a compact Lie group acting on a completely regular space.  
\end{rem}

\begin{rem}%remark 2.
  By similar methods, Palais \cite{key1} has proved Theorem
  \ref{chap2:thm5} for arbitrary Lie groups acting properly on
  completely regular spaces.  
\end{rem}
