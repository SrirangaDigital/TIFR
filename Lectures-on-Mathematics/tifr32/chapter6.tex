\chapter{}\label{chap6}%%%% 6 

Discrete\pageoriginale linear groups acting properly on convex open cones in real
vector spaces are of special interest for the applications. In that
case, the existence of a stable lattice or, more generally, of
certain stable discrete subsets gives rise to special methods of
constructing fundamental domains. The material here is due to Koecher
\cite{key1} and Siegel \cite{key1}. 

\section{}%sec 1

Let $E$ be a real vector space, of dimension $n \geq 2$. A subset
$\Omega $ of $E$ is called a \textit{ cone } if $t \Omega \subset
\Omega $ for every real $t > 0$. The cone $\Omega^*$ in the dual $E^*$
of $E$, defined by  
$$
\Omega^* = \bigg \{ X^* \in E^* \big | \langle x^*, x \rangle > 0 ~\text { for all }~
x \in \Omega - \{ 0 \}\bigg \} 
$$
is called the \textit{ dual cone} of $\Omega . \Omega^*$ is always
open in $E^*$; in fact, if $\sum$ denotes the unit sphere in $E$ (with
respect to some norm on $E$), we have  
$ \Omega^* = \bigg \{ X^* \in E^* \bigg | \langle x^*, x \rangle > 0$ for all 
$x \in \bar{\Omega}\cap \sum - \bigg \} $. 

Assume $E$ to be a Euclidean vector space with scalar product $\langle
, \rangle $. If, under the canonical identification of $E^*$ with $E$,
we have $\Omega^* =  \Omega$, we say that $\Omega$ is a \textit{self-
  dual} cone (or a \textit{positivity domain}). Clearly, $\Omega$ is
self - dual if and only if $\Omega  = \bigg\{ x \in E \bigg | \langle
x, y \rangle > 0$ for all $y \in \bar{\Omega} - \{ 0\}\bigg\}$. 

\begin{examples*}
\begin{itemize}
\item[{\rm(i)}] % Examples 1
  In $\mathbb{R}^n$ (with the usual Euclidean structure), 
  $$
  \displaylines{\hfill 
  \Omega = \bigg\{ (t_1, \ldots, t_n )\big|t_i > 0  ~\text{for all}~
  i \bigg\}\hfill \cr 
  \text{and}\hfill  
  \Omega \bigg\{(t_1, \ldots, t_n ) |t^2_1 +\cdots + t^2_{n-1}  \langle
  t^2_n, t_n \rangle 0 \bigg\} \hfill }
$$ 
  are selfdual cones.  

\item[{\rm(ii)}] Let\pageoriginale $E$ be the vector of real $n \times
  n $ symmetric matrices, 
with the scalar product $ \langle A, B\rangle = T_r (AB)$. Then the
set $\Omega$ of positive definite matrices of $E$ is a selfdual
cone. To see this, we note first that $\Omega^* \subset \Omega $. In
fact, let $A \in \Omega^*$, and let $e_1, \ldots, e_n $ be an
orthonormal basis of $\mathbb{R}^n $ such that $Ae_i = \lambda_i e_i
, \lambda_i \in \mathbb{R}, i = 1. \ldots, n$. Let $P_i \in
E $  be defined by $P_i e_j = \delta _{ij} e_i$. Then $P_i \in
\bar{\Omega}- \{ 0 \}$, and $\langle A, P_i\rangle = \lambda _i$. Hence
$\lambda_i > 0$ for all $i$, thus $A \in \Omega$. Conversely, let $A
\in \Omega$, and $B \in
\bar{\Omega} -\{ 0\}$. Let $\sqrt{A} \in \Omega$ and $\sqrt{B} \in
\Omega - \{ 0 \}$ be the positive square roots of $A$ and $B$
respectively. Then  
\begin{align*}
  \langle A, B \rangle & = T_r (AB) = T_r (\sqrt{A}\sqrt{A}\sqrt{B}\sqrt{B}) \\
  & =  T_r (\sqrt{B}\sqrt{A}\sqrt{A}\sqrt{B}) \\
  & =  T_r ((\sqrt{A}\sqrt{B})' (\sqrt{A}\sqrt{B})) > 0.
\end{align*}
\end{itemize}
\end{examples*}

\section{}%Section 2

We now state elementary properties of cones and their duals. 
\begin{enumerate} [(i)]
\item For any cone $\Omega, \Omega^*$ is convex. 
\item If the cone $\Omega$ in $E$ contains a basis of $E$ the
  $\overline{\Omega^*}$ is a non-degenerate convex cone (A convex set
  \textit{non- degenerate} if it does not contain any straight
  line). In fact, let $x^*, y^* \in E^*$, and suppose $x^* +ty^* \in
  \Omega^*$ for every $t \in \mathbb{R}$. Then, for every $z \in
  \bar{\Omega} - \{ 0\}$, we have $0 \leq \langle x^* + ty^*, z \rangle =
  \langle x^*, z \rangle + t \langle y^*, z \rangle $, for every $t
  \in \mathbb{R}$.  

  Hence\pageoriginale $\langle y^*, z \rangle = 0$ for every $z \in \bar{\Omega} -
  \{ 0\}$, hence $y^* =0 $. 
\end{enumerate}

Using (i) and (ii) we have 
(iii) If $\Omega $ contains a basis of $E$, then 
$$
x^* \in \bar{\Omega^*}, -x^* \in \bar{\Omega^*} \textit{ imply } x^* = 0.
$$

\setcounter{lem}{0}
\begin{lem}\label{chap6:lem1} %Lemma 1
  Given any compact subset $K$ of $\Omega^*$, we have $\rho (K) > 0$
  such that $\langle x^*, y\rangle \geq \rho (K) |y|$ for every $x^*
  \in K$ and $y \in \bar{\Omega}$. Here $| |$ denotes some norm on $E$.  
\end{lem}

\begin{proof}
  Let $\sum$ be the unit sphere in $E$. Then the function $(x^*,
  y)\rightsquigarrow \langle x^*, y\rangle$ on $K x (\bar{\Omega} \cap \sum )$
  is continuous and $> 0$. We can take $\rho (K)$ to be the infimum
  of this function. 
\end{proof}

\begin{remark*}
  If $\Omega$ is open, the statement analogous to that of Lemma \ref{chap6:lem1},
  with the roles of $\Omega$ and $\Omega^*$ interchanged, is also true;
  the proof is the same.  
\end{remark*}

\section{} %SEction 3

Let $\Omega \subset E$ be a non-degenerate cone; we then have $\Omega^*
\neq \phi $.  Let $D$ be a discrete subset of $E$ contained in
$\bar{\Omega} - \{0\}$.  For any $x^* \in \Omega^*$, we define  
$$
\mu (x^*) = \inf_{d \in D} \langle x^*, d \rangle .
$$

We see by lemma \ref{chap6:lem1} that $\mu (x^*) < 0$, and that the set 
$$
M(x^*) = \bigg\{ d \in D \big| \langle x^*, d \rangle = \mu (x^*)\bigg\}
$$
is non-empty and finite. 

\begin{lem}\label{chap6:lem2} %Lemma 2
  For\pageoriginale any $x^* \in \Omega^*$ and $\in > 0$, there exists a
  neighbourhood $U \subset \Omega^*$ of $x^*$ such that, for any $y^*
  \in U, \big| \mu (y^*) - \mu (x^*) \big| < \in$ and $M (Y^*)
  \subset M (x^*)$. 
\end{lem}

\begin{proof}
  Let $K \subset \Omega^*$ be a compact neighbourhood of $x^*$. Let
  $\rho = \rho (K)$ be as in Lemma \ref{chap6:lem1}. Let $D' = \bigg\{\ d \in D
  \big| |d| \leq (\mu (x^*) + \in) / \rho \bigg\}$. Clearly $D'$ is
  finite, and for any $y^* \in K$ and $d \in D - D'$, we have  
  $$
  \langle y^*, d\rangle  \mu (x^*) + \in .
  $$
  
  In particular, we have $M(x^*) \subset D'$. Clearly there exists $a
  > 0$ such that  
  $$
  \langle x^*, d \rangle > \mu (x^*) + \frac{a}{2}
  $$
  for $d \in D'- M(x^*)$. (We may suppose that $\dfrac{a}{2} < \in $.)
  Thus, there exists a neighbourhood $V_i \subset \Omega ^*$ of $x^*$
  such that $y^* \in V_i$ implies  
  $$
  \langle y^* , d \rangle > \mu (x^* )+ \frac{a}{2} , d \in D' - M(x^*).
  $$

  Finally there exists  neighbourhood $V_2 \subset \Omega^*$ of $x^*$
  such that $y^* \in V_2$ implies 
  $$
  | \langle y^* , d \rangle - \mu (x^*) \big| < \frac{a}{2}; d \in M (x^*). 
  $$
  
  Clearly, $U = K \cap V_1 \cap V_2$ satisfies conditions of the lemma. 
\end{proof}

A point $x^*$ of $\Omega^*$ is called \textit{perfect} it $M(x^*)$
contains a basis of $E$. Since, for any $\lambda > 0, M( \lambda x^* ) =
M(x^*)$, we shall assume that, for a perfect point $x^*, \mu (x^*) =
1$.
 
\begin{lem}\label{chap6:lem3} %Lemma 3
  Let\pageoriginale $y^* \in \Omega^* $ be not perfect, and let $M
  \subset M (y^*) , M 
  \neq \phi$. Then, for every $x^* \in E^*$ with $\langle x^* , M(y^*)
  \rangle \geq 0$ and $\langle x^*,  M \rangle = 0$, we have either  
  \begin{enumerate} [\rm (i)]
  \item $\mu (y^* + tx^*) = \mu (y^*)$ for every $t \geq 0$ such that
    $y^*+ t x^* \in \Omega^*$, or 
  \item there exists $t_o > 0 $ such that 
    \begin{enumerate}[\rm (a)]
    \item $y^* + t_o x^* \in \Omega^*$
    \item $\mu (y^* + t_o x^*) = \mu(y^*)$
    \item $M \subset M (y^* + t_o x^*)$ 
    \item $\dim M(y^* + t_o x^*) > \dim M$,
    \end{enumerate} 
  \end{enumerate}
  (where, for any subset $S$ of $E, \dim ~ S$ denotes the dimensions of
  the subspace generated by $S$). 
\end{lem}

\begin{proof}
  Suppose that ($i$) does not hold. Since, for any $d \in M$ and any $t
  \in \mathbb{R}$, we have $\langle y^* + tx^* , d \rangle = \mu (y^*)$,
  it follows that $\mu (y^* + tx^*)\leq \mu (y^*)$ if $ y^* + tx^* \in
  \Omega^*$. Hence there exists $\theta > 0$ such that $ y^* + \theta x^*
  \in \Omega^*$, and $\mu (y^* + \theta x^*) < \mu (y^*)$. Let $\mathfrak{B} =
  \Bigg\{ d \in D \langle x^*, d\rangle < 0\Bigg\}$. $\mathfrak{B}$ is
  non-empty, since $ \mathfrak{B} \supset M(y^* + \theta x^*)$. For $d
  \in \mathfrak{B}$, we set  
  $$
  \varphi (d) = (\mu (y^*) - \langle y^* , d \rangle )/ \langle x^*, d
  \rangle \cdot 
  $$
  
  Clearly, $\varphi (d) > 0$, and for $d \in M(y^* + \theta x^*
  )$, we have $\varphi(d) < \theta$. On other hand, if $\varphi(d) <
  \theta $, we have $ \mu (y^*) - \langle y^* + \theta x^* , d\rangle
  >0$. Hence if $\rho = \rho (K)$ of Lemma \ref{chap6:lem1} with $K = \Big\{ y^*
  + x^* \Big\}$, we have $|d|\leq \mu(y^*)/\rho $. Hence $\varphi <
  \theta$ only on a (non-empty) finite subset of $\mathfrak{B}$. Hence
  $\varphi$ attaints its infimum in $\mathfrak{B}$, let $t_o=
  \inf\limits_{d \in \mathfrak{B}}\varphi (d)$, and\pageoriginale let
  $\varphi (d_o) =  t_0$. We assert $t_o$ has the properties stated in
  (ii) of the lemma.  
\end{proof}

Since $\Omega^*$ is convex, and $0 < t_0 < \theta$, we have $y^* +
t_o x^*\in \Omega^*$. We observe that for $d \in M, \langle y^* + t_o
x^* , d \rangle = \mu (y^*)$. Hence ($b$) and ($c$) of ($ii$) will be
proved if we show that  
\begin{equation*}
  \langle y^* + t_o x^* \rangle \geq \mu (y^*) \tag{I}
\end{equation*}
for every $d \in D$. This is obvious for $d \in D - \mathfrak{B}$. For
$d \in \mathfrak{B}$, we have $\varphi (d)\geq t_o$, i.e., $\mu (y^*)
- \langle y^* , d \rangle \leq t_o \langle x^*$, $d \rangle$. Hence ($I$)
follows, and (b), (c) are proved. Finally, it is clear that $d_o
\in M(y^* + t_o x^*)$; since $\langle x^*, M \rangle = 0$, while
$\langle x^*, d_\circ \rangle <0$, (d) follows.  

From now on, we shall suppose that $\Omega$ is an open non-degenerate
convex cone; we then have $(\Omega^*)^* = \Omega$. For any finite
subset $S$ of $\Omega$, the set $PS = \bigg\{ \sum t_i s_i | s_i \in
S, t_i \geq 0 \bigg \}$ is called the \textit{pyramid on} $S$. If $x^*
\in \Omega^*$ is a perfect point, then $PM(x^*)$ is called a 
\textit{perfect pyramid}. 

\begin{lem}\label{chap6:lem4} %Lemma 4
  For any $x^* , y^* \in \Omega^* $, we have 
  $$
  PM(x^*)\cap PM (y^*) = P(M(x^*)\cap M (y^*)) 
  $$
  and 
  $$
  \langle \mu (x^*) y^* - \mu (y^*) x^*, PM (x^*) \cap PM(y^*)\rangle = 0.
  $$
\end{lem}

\begin{proof}
  Obviously, $P(M(x^*)\cap M(y^*))\subset PM (x^*)\cap PM
  (y^*)$. Conversely, let $z \in PM (x^*)\cap PM (y^*)$. Let $z = \sum
  a_i x_i; x_i \in M(x^*), a_i >0$. Similarly, let\pageoriginale $z = \sum b_j y_j ;
  y_j \in M(y^*), b_j > 0$. We have  
  \begin{align*}
    \langle x^*, z \rangle &= \sum_i a_i \langle x^*, x_i \rangle = \mu
    (x^*) \sum_i a_i\\ 
    &= \sum_j b_j \langle x^*, y_j \rangle \geq (x^*) \sum_j b_j
  \end{align*}

  Since $\mu (x^*)\neq 0$, we have $\sum  a_i \geq \sum  b_j $, hence,
  by symmetry $\sum  a_i = \sum  b_j $. It follows that $\langle x^* ,
  y_j \rangle = \mu(x^*)$, i.e. $y_j \in M(x^*)$ for every
  $j$. Similarly, $x_i \in M(y^*)$ for every $i$, i.e. $z \in PM
  (x^*)\cap PM(y^*)$. The first assertion of the lemma is therefore
  proved. The second is then clear.  
\end{proof}

\section{}%Section 4

\begin{defi*}
  The discrete set $D$ in $E$ (contained in $\bar{\Omega} -\{0\}$) is
  said to satisfy the \textit{density condition} if, for each $z^* \in
  \bar{\Omega^*} -{\Omega^*} , \mu (x^*) \to 0 \text{ as } x^* (\in
  \Omega^*) \to z^*$.  
\end{defi*}

\begin{examples*}
  \begin{itemize}
  \item[{\rm(i)}] Let $\Omega C \mathbb{R}^2 $ be the (self - dual) cone defined
    by $\Omega = \left\{ (t_1, t_2) \big| t_1 , t_2 > 0 \right\}$. Then $D =
    \left\{(1, 0)\cup (0, 1) \right\}$ also
    satisfies the density condition. The set $D= \left\{ (\exp n, \exp
    \,(-n) \big| n \in \mathbb{Z} \right\}$ also satisfies the density
    condition.  
 
  \item[{\rm(ii)}] Let $\Omega$ be the (self - dual) cone of positive definite
    matrices in the space $E$ of real $n \times n$ symmetric
    matrices. Let $D$ be the set $\{ U U' | U \in \mathbb{Z}^n, U
    \neq 0 \}$. Clearly $D$ is a discrete set in $E$, and $D
    \subset \Omega - \{0\}$. $D$ satisfies the density condition. We
    shall prove in fact that for any $A \in \Omega $,  
    $$
    \mu (A)^n \leq (2^{2n}. \det A)/_{\rho^2_n}, 
    $$
    where\pageoriginale $\rho_n$ is the volume of the unit ball in
    $\mathbb{R}^n$. Let 
    $A \in \Omega$, and $U \in \mathbb{Z}^n - \{ 0 \}$. We have  
    $$
    \mu (A) \leq Tr (A U U' ) = Tr U' A U = U'B B U = |BU|^2,  
    $$
    where $B \in \Omega$ is the square root of $A$. Thus, the convex
    symmetric set $C = \Big\{ x \in \mathbb{R}^n\big| |B x|^2 <
    \rho(A) \Big\}$ does
    not contain any non-zero integral point. Hence, by a theorem of
    Minkowski, vol $C \leq 2^n$. However, the volume of $C$ is easily seen
    to be $\rho_n \mu (A)^{n/2}/(\det A)^{\frac{1}{2}}$, and we get the
    required inequality.  
  \end{itemize}
\end{examples*}

\begin{remark*}
  If $D$ satisfies the density condition, then 
  \begin{equation*}
    \bar{\Omega^*} = \Bigg\{ x^* \in E^* \big| \langle x^* , d \rangle
    \geq 0
    \text{ for all } d \in D  \Bigg  \}. \tag{I} 
  \end{equation*}
  
  In fact, it is clear that $\Omega$ is contained in the right hand side
  of ($I$). Now let $x^* \in E^* - \Omega^*$. Then, for any $y^* \in
  \Omega^*$, there exists $t_o, 0 < t_o < 1$, such that $t_o x^* + (
  1-t_o) y^* \in \bar{\Omega^*} - \Omega^*$. For any $d \in D$, we have  
  \begin{align*}
    \langle t_o x^* + ( 1- t_0) y^* , d \rangle & = t_o \langle x^* , d
    \rangle  + ( 1 -t_o) \langle y^* ,d \rangle \\ 
    & \geq t_o \langle x^* , d \rangle + (1- t_o) \mu (y^*).
  \end{align*}
  
  Since $\mu (tx^* + (1-t)y^* )\to 0$ as $t$ increases to $t_0$, it
  follows that $\langle x^*, d \rangle < 0$ for some $d \in D$, and
  (I) is proved. However, the density condition is not necessary for
  (I) to hold.  
\end{remark*}

\begin{lem}\label{chap6:lem5}%lemma 5
  If $D$ satisfies the density condition, then the set $\mathscr{P}$
  of perfect points is discrete in $E^*$. 
\end{lem}

\begin{proof}
  Let\pageoriginale $(x^*_i)$ be a sequence in $\mathbb{P}$ converging to $x^* \in
  E^*$. Clearly $x^* \in \bar{\Omega^*}$, and in view of the density
  condition, we must have $x^* \in \Omega^*$. Then $\mu (x^*)=1$, and
  $M(x^*_i) \subset M(x^*)$ for large $i$ (Lemma \ref{chap6:lem2}). Since $x^*_i$ is
  perfect, it follows by Lemma \ref{chap6:lem4} that $x^*=x^*_i$. 
\end{proof}

\begin{lem}\label{chap6:lem6}%lem 6
  If $D$ satisfies the density condition, every point of $\Omega$
  belongs to a perfect pyramid. 
\end{lem}

\begin{proof}
  We first note that, since $D$ satisfies the density condition, the
  first alternative of Lemma \ref{chap6:lem3} can never hold if $x^* \notin
  \bar{\Omega}^*$. Hence we see by Lemma \ref{chap6:lem3} that $\mathbb{P}\neq \phi
  $. Now let $z \in \Omega$, and let $y^*$ be any perfect point. If $z
  \in PM(y^*)$, there is nothing to prove. Let $z \notin PM(y^*)$. 
\end{proof}

Then there exists $x^* \in E^*$ such that (a) $\langle x^*,PM
(y^*)\rangle \geq O, (b) \langle x^*,z \rangle < O, (c)x^*$ vanishes
on a subset $M$ of $M(y^*)$ containing $n-1$ linearly independent
points. On account of (b), $x^* \notin \bar{\Omega}^*$. Hence the
second alternative of Lemma \ref{chap6:lem3} holds, and there exists $t_o > O$ such
$y^* +t_o x^* \in \Omega^*, \mu(y^* +t_o x^*) = 1,M \subset M(y^* +
t_o x^*)$ and $\dim M (y^* +t_o x^*) > \dim M$. Clearly $y^*_1 =y^* +
t_o x^*$ is perfect. Moreover, $\langle y^*_1,z \rangle < \langle
y^*z, \rangle$. 

If $z \in PM (y^*_1)$ we are through. Otherwise, we repeat the above
procedure with $y^*_1$, and obtain $y^*_2 \in \mathbb{P}$ such that
$\langle y^*_2,z \rangle < \langle y^*_1,z \rangle$. This process must
terminate after a finite number of steps, since $\mathbb{P}$ is
discrete, and since (Remark following Lemma \ref{chap6:lem1}) there is a constant
$\varrho = \varrho (z)$ such that 
$$
|y^*_i | \leq \varrho \langle y^*_i,z \rangle < \varrho \langle y^*,z \rangle
$$
for any $i$. We thus obtain a perfect pyramid containing $z$.

\begin{lem}\label{chap6:lem7}%lemma 7.
  Any\pageoriginale compact set $K$ in $\Omega$ is met by only
  finitely many perfect pyramids. 
\end{lem}

\begin{proof}
  Let $x^* \in \mathbb{P}$, and let $y \in K \cap PM (x^*)$. Then if
  $\varrho (K)$ is as in Lemma \ref{chap6:lem1}, we have $ \langle
  x^*,y \rangle \geq 
  \varrho (K) | x^*|, i.e.|x^*| \leq \dfrac{ \langle
    x^*,y\rangle}{\varrho (K)}$. On the other hand, the convex closure
  of $D$ does not contain $O$, and hence there exists $\varrho'(K) > O$
  such that for every $x^* \in \mathbb{P}, \langle x^*,y \rangle <
  \varrho' (K)$ on $K \cap PM (x^*)$. Since $\mathbb{P}$ is discrete,
  the lemma follows. 
\end{proof}

\begin{remark*}
  It follows from the above lemma that, for any $x^* \in \mathbb{P}$,
  the set $\{ y^* \in \mathbb{P}|PM (x^*) \cap PM(y^*) \cap \Omega
  \neq \phi \}$ is finite: in view of Lemma \ref{chap6:lem4}, this set is the set
  of $y^* \in \mathbb{P}$ such that $PM(y^*) \cap K \neq \phi $, where
  $K$ is, for instance the (finite) set in $\Omega$ consisting of
  those of the barycentres of the subsets of $M(x^*)$ which lie in
  $\Omega$. 
\end{remark*}

\section{}%sec 5

Let $\Omega$ be an open non-degenerate convex cone in a real vector
space $E$.  Let $G(\Omega)=G$ be the subgroup of $GL(E)$ which maps
$\Omega$ into itself. Then $G$ is a closed subgroup of $GL(E)$. For
any $x \in \Omega, G(x)$ is compact. In fact, the set $\Omega \cap \{
x-z|z \in \Omega \}$ is stable under the action of $G(x)$.  Since
$\Omega$ is non-degenerate, this is a bounded open set. Hence $G(x)$
is compact. 

$G=G(\Omega)$ also acts on $\Omega^*$: for $s \in G$ and $x^* \in
\Omega^*$, we define $sx^*$ by $\langle sx^* ,y \rangle =\langle
x^*,s^{-1}y \rangle $; this identifies $G$ with $G(\Omega^*)$. Let $D$
be a discrete subset of $\bar{\Omega} - \{O\}$, and let $\Gamma$ be a
subgroup of $G$ such that $\Gamma D =D$. Then clearly $\mu(sx^*)=\mu
(x^*)$ and $M(sx^*)=sM(x^*)$ for any $x^* \in \Omega^*$ and $s \in
\Gamma$. Thus $\Gamma $ also acts on the\pageoriginale set of perfect
points and the 
set of perfect pyramids. Note that if $x^* \in \mathbb{P}, \Gamma
(x^*)=\{ s \in \Gamma | PM (x^*)=s PM (x^*)\}$. 

Assume now that $D$ satisfies the density condition. Then for any
compact set $K$ in $\Omega$, we can find a finite subset $\mathbb{R}$
of $\mathbb{P}$ such that $K \subset \bigcup \limits_{x^* \in
  \mathbb{R}} PM(x^*)$ (Lemma \ref{chap6:lem6} and \ref{chap6:lem7}), thus 


$\Gamma (K|K) \subset \bigcup \limits_{x^*,y^* \in \mathbb{R}}
\Gamma(K \cap PM(x^*)|PM (y^*))$. Since, for any $x^* \in \mathbb{P},
\bigg\{y^* \in \mathbb{P}|K \cap PM(x^*)\cap PM(y^*) \neq \phi
\bigg\}$ is finite, it follows that $\Gamma (K \cap PM (x^*)|PM\break (y^*
))$ is finite for $x^*,y^*\in \mathbb{P})$; hence $\Gamma (K|K)$ is
finite. Thus $\Gamma$ is a discrete subgroup of $G(\Omega)$, and acts
properly on $\Omega$.  

\begin{remark*}
  It can be proved that $G(\Omega)$ itself acts properly on $\Omega$.
\end{remark*}

Also there is a natural $G(\Omega)$-invariant Riemannian metric on $\Omega$.
For any $ x \in \Omega $, we define
$$
(N(x))^{-1}= \int \limits_{\Omega^*} \exp (- \langle y^*,x \rangle ) dy^*.
$$

Then integral is finite, in view of Lemma \ref{chap6:lem1}. It is easy to verify
that for $s \in G(\Omega), (N(sx))^{-1} = | \det s | (N(x))^{-1}$. The
2-form $- \dfrac{\partial^2 \log N}{\partial x_i \partial x_j} dx_i dx_j$
gives a $G(\Omega)$-invariant Riemannian metric on $\Omega$. 

\setcounter{thm}{0}
\begin{thm}\label{chap6:thm1}%tthe 1
  Let $\Omega$ be an open convex non-degenerate cone in a real vector
  space $E, D \subset \bar{\Omega}- \{\circ\}$a discrete subset of $E$
  satisfying the density condition, and $\Gamma$ a discrete subgroup of
  $G(\Omega)$ such that $\Gamma D=D$. Let $\mathbb{P}(\subset \Omega^*)$
  be the set perfect points. Assume that there\pageoriginale exists a finite subset
  $L$ of $\mathbb{P}$ such that $\Gamma L =\mathbb{P}$. Then, if  
  $$
  A = \Omega \cap \bigcup_{x^* \in L}PM(x^*),
  $$
  we have 
  \begin{enumerate}[\rm (a)]
  \item $\Gamma A = \Omega$,
  \item $\Gamma (A|A)$ is finite,
  \item $\Gamma (A|A)A$ is a neighbourhood of $A$ in $\Omega$,
  \item $\Gamma $ is finitely presentable.
  \end{enumerate}
\end{thm}

\begin{proof}
  Using Lemma \ref{chap6:lem6} and the fact that $\Gamma L = \mathbb{P}$ (and since
  $sM(x^*)=M(sx^*), x^* \in \mathbb{P}, s \in \Gamma)$, it is easy to
  see that $\Gamma A = \Omega$. The proof of (b) is similar to that of
  the fact that the action of $\Gamma$ on $\Omega$ is proper; we have
  only to use the remark following Lemma \ref{chap6:lem7}. Since $
  \Gamma A = \Omega$, 
  (c) follows. (Note that $\bigg\{sA | s \in \Gamma \bigg\}$ is a
  locally finite family of closed sets in $\Omega$.) Since $\Omega$ is
  convex, it is connected, locally connected and simply connected. Hence
  the conditions of Theorem \ref{chap3:thm1}, Chapter \ref{chap3} are
  satisfied, and the   assertion (d) follows.  
\end{proof}

\begin{remark*}
  The condition in the above theorem that $_\Gamma\backslash\mathbb{P}$
  be finite is satisfied if we assume the following: there exists a
  finite subset $B$ of $D$ such that, for every $y^* \in \mathbb{P}$,
  there exists $s \in \Gamma$ such that convex envelope of $M(sy^*)\cap
  B$ meets $\Omega$. In fact, let $y^* \in \mathbb{P}$, and let $s$ be
  as above. Then $b = \dfrac{1}{r} \sum \limits_{ a \in M(sy^*) \cap B}
  a \in \Omega$, where $r=$number of elements of $M(sy^*)\cap B$. Now, 
  $$
  \langle sy^*,b \rangle = \frac{1}{r} \sum \langle sy^*,a\rangle =1.
  $$
\end{remark*}

Hence\pageoriginale $|sy^*| \leq \dfrac{1}{\varrho_b}$, where $\varrho_b =
\varrho(K)$ of Lemma \ref{chap6:lem1} with $K= \{b\}$. 

The number of points $b$ is finite. Since $\mathbb{P}$ is discrete,
our assertion follows. 

\section{}%sec 6

We now apply the preceding  results to the case of $GL(n, \mathbb{Z})$
acting on the space of symmetric positive definite matrices. Thus let
$E$ be the vector space of all real $n \times n$ symmetric matrices
with the scalar product $\langle A,B \rangle = Tr \,(AB)$, and let
$\Omega$ be the (self-dual) cone of positive definite matrices in
$E$. $\Gamma =GL(n,\mathbb{Z})$ acts on $\Omega : (S,A)\rightsquigarrow
SAS' =S[A]$. 

Let $D=\bigg\{UU'|U \in \mathbb{Z}^n,U \neq \circ \bigg\}$. We have seen
that $D$ satisfies the density condition. It is clear that $\Gamma
[D]=D$. 

For any $A \in \Omega$, let $\tilde{M}(A)=\bigg\{U \in
\mathbb{Z}^n|UU'\in M(A) \bigg\}$. For $S \in \Gamma $, we clearly
have $\tilde{M}(S[A])=S^* \tilde{M}(A)$, where $S^*=(S')^{-1}$. 

\begin{lem}\label{chap6:lem8}%lem 8
  If $A$ is perfect, then $\tilde{M}(A)$ contains $n$ linearly
  independent elements. 
\end{lem}

\begin{proof}
  Let $B$ be any matrix such that $BU=O$ for every $U \in
  \tilde{M}(A)$. Then $\langle B, UU' \rangle =Tr (BUU')=U'BU=O$ for
  every $U \in \tilde{M}(A)$, $i.e. \langle B, M(A) \rangle =O$. Since
  $A$ is perfect, this implies $B=O$. 
\end{proof}

\begin{lem}\label{chap6:lem9}%lem 9
  Let $A$ be perfect, and let $U_1, \ldots , U_n \in \tilde{M}(A)$ be
  linearly independent. Let $C=(U_1, \ldots , U_n)$ be the matrix whose
  $i$-th column is $U_i$. Then $| \det C| \leq 2^n/\varrho_n$, where
  $\varrho_n$ is the volume of the unit ball in $\mathbb{R}^n$. 
\end{lem}

\begin{proof}
  We\pageoriginale have $C'A C \in \Omega$, and the diagonal elements of $C'A C$ are
  equal to $1$. Hence by a known lemma we have $\det C' A \leq 1$,
  $i.e.(\det C)^2 \leq (\det A)^{-1}$. However, we have seen (Example
  (ii), p.82) that $(\det A)^{-1} \leq 2^{2n}/\varrho ^2_n$. 
\end{proof}

\begin{lem}\label{chap6:lem10}%lemm 10
  There exists a finite subset $L$ of $\mathbb{Z}^n$ such that, for any
  $A \in \mathbb{P}$, there exists $S \in \Gamma$ such that $\dim (L
  \cap S \tilde{M}(A))=n$. 
\end{lem}

\begin{proof}
  Let $L= \bigg\{(v_1,\ldots ,v_n) \in \mathbb{Z}^n-\{O\}|\circ \leq v_i
  \leq 2^n/_{\varrho_n}\bigg \}$.  Let $A \in \mathbb{P}$.  By Lemma
  \ref{chap6:lem8}, $\tilde{M}(A)$ contains $n$ independent elements $U_1 \, \ldots ,
  U_n$. 
\end{proof}

Now there exists $S \in \Gamma $ such that, for $1 \leq i \leq n$,
\begin{equation*}
  SU_i =
  \begin{pmatrix}
    a^1_i \\ \vdots \\ a^i_i \\ O \\ \vdots \\ O
  \end{pmatrix}
  ,O \leq a^j_i < a^i_i \text{ for }1 \leq  j < i
\end{equation*}

The proof of this fact is analogous to the Elementary Divisor Theorem
(see van der Waerden \cite{key1}). Clearly the $SU_i$ are independent. Since
$SU_i \in S \tilde{M}(A)= \tilde{M}(S^* [A])$, we have bu Lemma \ref{chap6:lem9}, 
$$
\det (SU_1, \ldots , SU_n)=a_{11} a_{12} \cdots a_{1n} \leq 2^n/{\rho_n}.
$$

Thus $a^j_i \leq 2^n/_{\varrho_n}, 1 \leq i,j \leq n$. Thus $SU_i \in
L$, and the lemma is proved. 

\begin{lem}\label{chap6:lem11}%lem 11
  There exists a finite subset $\mathbb{B}$ of $D$ such that for every
  $A \in \mathbb{P}$. there exists $S \in \Gamma$ such that the convex
  envelope of $\mathbb{B}\cap M(S^*[A])$ meets $\Omega$. 
\end{lem}

\begin{proof}
  Let\pageoriginale $\mathbb{B}=\bigg \{ UU'|U \in L \bigg\}$, where $L$ is as in
  Lemma \ref{chap6:lem10}. Let $A \in \mathbb{P}$. By Lemma
  \ref{chap6:lem10}, there exists $S \in 
  \Gamma$ such that $L \cap S \tilde {M}(A)$ contains $n$ independent
  elements $V_1, \ldots,V_n$. Then $V_i V'_i \in \mathbb{B} \cap M(S^*
  [A]), 1 \leq i \leq n$.
  Clearly $\dfrac{1}{n} \sum V_i V'_i$ is positive definite, and belongs 
  to the convex closure of $\mathbb{B} \cap M(S^*[A])$. 
\end{proof}

\begin{remark*}
  The above lemma and the remark following Theorem \ref{chap6:thm1}
  show that Theorem \ref{chap6:thm1} is applicable to the case of
  $GL(n, \mathbb{Z})$  acting on the positive definite matrices. It
  follows in particular that $GL(n, \mathbb{Z})$ is finitely
  presentable.  
\end{remark*}

\section{}%sec 7

Let $\Omega$ be an open non-degenerate convex cone in a real vector
space $E$. Let $G$ be a subgroup of $G(\Omega)$, and let $\chi : G \to
\mathbb{R}^+$ be a homomorphism ($\mathbb{R}^+$denotes the group of
real numbers $> O$). 

\begin{defi*}
  A {\em norm} on $\Omega$ (with respect to $\chi : G \to \mathbb{R}^+$)
  is a continuous map $\gamma : \Omega \to \mathbb{R}^+$ such that 
  \begin{enumerate}[\rm (i)]
  \item $ \nu  (sx) = \chi (s)\nu (x)$ for $s \in G, x \in \Omega$,
  \item $\nu (x)\to \circ$ as $x \to \bar{\Omega}-\Omega$
  \item for every $x \in \Omega $and $r \geq O$,
  \end{enumerate}
  $(x +\bar{\Omega}) \cap \bigg \{ x \in \Omega | \nu (x) \leq r \bigg
  \}$ is compact. 
\end{defi*}

\begin{examples*}
  \begin{enumerate}[i)]
  \item If $\Omega $ is the cone of positive definite real $n \times n$
    matrices and $G = GL(n, \mathbb{R})$, then $\nu (A)=\det A, A \in
    \Omega$, is a norm on $\Omega$ for $\chi : G \to \mathbb{R}^+$
    defined by $\chi (S) = (\det S)^2$. 

  \item For any $\Omega$, and $G =G (\Omega)$
  \end{enumerate}
  $$
  \nu (x)=\left(~ \int \limits_{\Omega^*} e^{- \langle
    x,y^*\rangle} dy^*\right)^{-1} 
  $$
  is\pageoriginale a norm for $\chi(s)= |\det s | $.
\end{examples*}

\begin{thm}\label{chap6:thm2}%the 2
  Let $\Omega$ be an open non-degenerate convex cone in a real vector
  space $E$ of dimension $n$. Let $L$ be a lattice in $E$, and let $D$
  be a subset of $L \cap (\bar{\Omega}- \{O\})$ satisfying the density
  condition. Let $\Gamma$ be a discrete subgroup of $G(\Omega)$ such
  that $\Gamma D=D$, and assume that ${}_\Gamma \backslash \mathbb{P}$ is
  finite. Then the subset $A$ of $\Omega$ constructed in Theorem
  \ref{chap6:thm1} has 
  the property: for any norm $\nu$ on $\Omega$ and any $r > \circ, A \cap L$
  contains only finitely many points $x$ with $\nu (x) \leq r$. 
\end{thm}

We first prove the following

\begin{lem}\label{chap6:lem12}%lem 12
  For any $y^* \in \mathbb{P}$, there exists $a \in \Omega$ such that
  $PM (y^*) \cap \Omega \cap L \subset a+ \bar{\Omega}$. 
\end{lem}

\begin{proof}
  We first remark that for any compact set $K \subset \Omega$, there
  exists $\vartheta \in \Omega$ such that $K \subset \vartheta +
  \Omega$. Now, for any $y^* \in \mathbb{P}, PM(y^*)$ is a finite union
  of pyramids $PM_i$, where the $M_i \subset M(y^*)$ consists of
  precisely $n$ independent elements. It is sufficient to find each $i$
  precisely $n$ independent elements. It is sufficient to find for each
  $i$ an $a_i \in \Omega $ such that $PM_i \cap \Omega \cap L \subset
  a_i + \bar {\Omega}$; for, by the remark above, there exists $a \in
  \Omega $ such that $a_i \in a+ \Omega $ for each $i$, and clearly a
  will satisfy the condition of the lemma. 
\end{proof}

Let $M=\bigg \{a_1,\ldots ,a_n \bigg \}$ be any one of the $M_i$. We
have $M \subset L$; let $L_o$ be the sublattice of $L$ generated by
$M$. Let $p>O$ be an integer such that $p L \subset L_o$. For any $ x
\in PM \cap \Omega \cap L$, let $px =\sum \lambda_i a_i, \lambda_i \in
\mathbb{Z}$. Since $x \in PM$, and the $a_i$ are independent,
we\pageoriginale have 
$\lambda_i \geq O$ for every $i$. Let $a_x =\dfrac{1}{p} \sum
\limits_{\lambda_i \neq O}a_i$. Since $x \in \Omega$, we see easily
that $a_x \in \Omega$. Also, $x-a_x = \dfrac{1}{p}(\sum \lambda_i a_i-
\sum \limits_{\lambda_i \neq O}a_i) \in PM \subset \bar{\Omega}$. The
set $\bigg \{ a_x | x \in PM \cap \Omega \cap L \bigg \}$ is clearly
finite. Hence we have $a \in \Omega $ such that $a_x \in a+\Omega$ for
every $x$. Clearly $PM \cap \Omega \cap L \subset a+ \Omega $, and
this proves the lemma.
 

\setcounter{proofofThm}{1}
\begin{proofofThm} %proof of theorem 2
  Let $A = \Omega \cap \bigcup \limits_{i \in I}PM (y^*_i)$ be as in
  Theorem \ref{chap6:thm1}. Let $a_i \in \Omega$ be such that $\Omega \cap
  PM(y^*_i)\cap L \subset a_i + \bar{\Omega}$. For any $r$, let $V_r =
  \bigg \{ z \in \Omega | \nu (z) \leq r \bigg \}$. Then $V_r \cap (a_i
  + \bar{\Omega)}$ is compact. Hence $V_r \cap (a_i + \bar{\Omega})\cap
  L$ is finite. It follows that $A \cap L \cap V_r$is finite. 
\end{proofofThm}

\begin{examples*}
  \begin{enumerate}[(i)]
  \item Let $\Omega$ be the cone or real positive definite $n \times n$
    matrices, $\Gamma=GL(n,\mathbb{Z}), D=\bigg \{ UU' | U \in
    \mathbb{Z}^n - \{\circ\}\bigg \},L=$ the lattice of all integral $n
    \times n$ matrices, $\nu (A)= (\det A)^2$ for $A \in \Omega$. Since
    $\nu $is constant on the orbits of $\Gamma$, Theorem
    \ref{chap6:thm2} implies in 
    particular that the number of orbits of $\Gamma$ in $A \cap L$ with
    determinant less than a given $r$ is finite. 

  \item Let $\Omega= \bigg \{ (x,y,z) \in \mathbb{R}^3 | z > O \text{
    and } x^2 + y^2 -3z^2 < O \bigg \}$. 
    Let $L =\mathbb{Z}^3$, and let $\Gamma$ be the subgroup of
    $GL(\mathbb{R}^3)$ generated by the matrices 
    \begin{equation*}
      \begin{pmatrix}
        0 & 1 & 0\\-1 & 0 & 0 \\0 & 0 & 1 
      \end{pmatrix}
      \text{and}
      \begin{pmatrix}
        2 & 0 & 3\\0 & 1 & 0 \\1 & 0 & 2 
      \end{pmatrix}
    \end{equation*}
    
    It\pageoriginale is easy to verify that $\Gamma \Omega = \Omega$. Let $D=\Gamma
    \{(0,0,1\}.D$ satisfies the density condition. The fact that $_\Gamma
    \backslash \mathbb{P}$ is finite is a consequence of the following
    remark: if $D \subset \Omega$ satisfies the density condition and $
    \Gamma\backslash D$ is finite, then $\Gamma \backslash \mathbb{P}$ is
    finite. In fact let $B$ be a finite subset of $D$ such that $\Gamma B
    =D$. Then for any $y^* \in \mathbb{P}$, there exists an $s \in \Gamma
    $ such that $M (sy^*)\cap A \neq \phi $. The remark follows, since $PM
    (z^*)\cap A \neq \phi $ for only finitely many $z^* \in
    \mathbb{P}$-note that by assumption, $A \subset \Omega$. 

  \item Let $K$ be a totally real extension of $\mathbb{Q}$ of degree $n$.
  \end{enumerate}
  
  Let $\Gamma $ be the group of totally positive units of $K$.  Let
  $\sigma_1, \ldots, \sigma_n$ be $n$ distinct isomorphisms of $K$ into
  $\mathbb{R}$. We make $\Gamma$ act on the self dual cone $\Omega =
  \bigg \{(t_1,\ldots,t_n) \in \mathbb{R}^n|t_i >O \text{ for all }i
  \bigg \}$ by setting 
  $$
  \varepsilon (t_1,\ldots ,t_n)=(\sigma_1 (\varepsilon)
  t_1,\ldots,\sigma_n(\varepsilon)t_n), 
  $$
  Let $D=\Gamma\bigg \{ (1,\ldots,1\bigg\}$. It is a classical result
  that for any $i, 1 \leq i \leq n$, there exists $\in \in \Gamma$ such
  that $\sigma_i(\varepsilon) > 1$, and $\sigma_j (\varepsilon)<1$ for
  $j \neq i$.  
  
  Using this, we verify that $D$ satisfies the density condition. Let
  $(t_1, \ldots,t_n)\in \bar{\Omega}$; let $t_i =\circ$. Let $\varepsilon$
  be chosen as above. Then  
  $$
  \langle \varepsilon^p (1, \ldots,1), (t_1,\ldots ,t_n)\rangle =\sum_{j
    \neq i} t_j (\sigma_j (\varepsilon))^p 
  $$
  which tends to zero as $p \to \infty$. It follows as in Example (ii)
  that $_\Gamma \backslash\mathbb{P}$ is finite. 
\end{examples*}


\begin{thebibliography}{99}
\bibitem{key1} H. Behr :   \"Uber\pageoriginale die endliche
  Defininerbarketit von Gruppen  J. f\"ur reine und
  angew. Math. 211 (1962) pp.116-122.   

\bibitem{key2} H. Behr : \"Uber die endliche Definierbarkeit verallgemeinerter
  Eineheitengruppen J. f\"ur reine und angew. Math.211 (1962) pp.123-135. 

\bibitem{key3} {A. Borel} : Seminar on Transformation groups Ann. of
  Math. Studies, $N^\circ$46, Princeton Un. Press (1960). 

\bibitem{key4} {N. Bourbaki} : Topologie G\'en\'erale, Chap. III, 3
  \`eme \'edition (1960). 

\bibitem{key5} {P. Buisson} : Lois d op\'eration propres (These 3 \`eme
  cycle, Strasbourg, 1964). 

\bibitem{key6} {H.S.M. Coxeter} : Regular Polytopes, Methuen \& C. London (1948).

\bibitem{key7} {M. Eichler} : Quadratische Formen und orthogole Grupper,
  (Springer, 1952).   

\bibitem{key8} {H. Freudenthal} : \"Uber die Enden diskreter Ra\.ume und Gruppen,
  Comm. Math. Helv. 17(1944) pp.1-38. 

\bibitem{key9} {M. Geretenhaber} : On the algebraic structure of
  discontinuous groups, Proe. of the
  Amer. Math. Soc. 4(1953) pp.745-750. 

\bibitem{key10} {H. Hopf} : Enden of fener R\"aume und endliche
  diskontinuierliche Gruppen, Comm. Math. Helv. 16(1943) pp.81-100. 

\bibitem{key11} {M. Koecher-R. Roelke} : Discontinuierliche und discrets
  Gruppen von Isometrien metrischer Raumen,
  Math. Zeit. 71(1959) pp.258-267. 

\bibitem{key12} {M. Koecher} : Beitr\"age zu einer reductionstheorie in
  Positivitatsbereichen, Math. Ann. 141(1960) pp.384-432.  

\bibitem{key13} {J.L.Koszul} : Sur certains espaces de transformation de Lie
  Coll. de Geometrie Diff. Strasbourg 1953, C.N.R.S. 

\bibitem{key14} {D Montgomery, H.Samelson, C.T. Yang} :  Exceptional
  orbits of Highest dimension, Ann. of Math., 64(1956) pp.131-141.  

\bibitem{key15} {D. Montgomery, C.T. Yang} : The\pageoriginale existence
  of slices Ann. of Math. 65(1957) pp.108-116. 

\bibitem{key16} {D. Montgomery, L. Zippin} : Topological transformation groups
  Interscience 1955. 

\bibitem{key17} {G.D. Mostow} : Equivariant embeddings in euclidean spaces
  Ann. of Math. 65(1957) pp.432-446. 

\bibitem{key18} {K. Nomizu} : Lie group and Differential Geometry
  Math. Soc. Japan Pub. Vol. 2(1956). 

\bibitem{key19} {R.S. Palais} : On the existence of slices for actions of non
  compact Lie group, Ann. of Math. 73(1961) pp.295-323. 

\bibitem{key20} {W. Roelke} : \"Uber Fundamentalbereiche diskontinuierlicher
  Gruppen Math. Nachr. 20(1959) pp.329-355. 

\bibitem{key21} {C.L Siegel} : Einheiten quadratischer Formen,
  Abh. Math. sem. Hansis. Univ. 13(1940), pp.209-239. 

\bibitem{key22} {C.L Siegel} : Discontinuous groups, Ann. of
  Math. 44(1943) pp.674-689. 

\bibitem{key23} {C.L Siegel} : Quadratic forms (TIFR Notes, 1957).  

\bibitem{key24} {Van der Waerden} : Moderne Algebra

\bibitem{key25} {A. Weil} : Discrete subgroups of Lie groups I, Ann. of Math.

\bibitem{key26} {A. Weil} : Discrete subgroups of Lie groups II Ann. of
  Math. 75(1962), pp.578-602. 

\bibitem{key27} {E. Witt} : Spiegelungsgruppen und Aufz\"ahlung halb-einfacher
  Lie' scher Ringe, Abh. Math. Sem. Univ. Hamburg 14
  (1941), pp.289-322. 

\bibitem{key28} {E. Witt} : \"Uber die Konstruktion von fundamental Bereichen
  Ann. Mat. Pura e Appl. 36(1954) pp. 215-221.      
\end{thebibliography}
