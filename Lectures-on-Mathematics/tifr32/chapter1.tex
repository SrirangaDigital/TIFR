\chapter{}\label{chap1}%%% 1

This\pageoriginale chapter collects some basic facts about proper actions of
topological groups on topological spaces; the existence of invariant
metrics is discussed in. \S\ \ref{chap1:sec4}  (Bourbaki \cite{key1},
Palais \cite{key1}).  

\setcounter{section}{-1}
\section{}%%% 0
Let $G$ ba a topological group, acting continuously on a topological
space $X$. We shall always suppose that the action is on the left, and
if $m : G \times X \to X$ defines the action, we shall write, for $s
\in G$ and $x \in X, m(s, x)=sx$. 


\medskip
\noindent{\textbf{Notation.}}
For $A, B \subset X$, we set
$$
G(A/B) = \bigg\{s \in G \big| sB \cap A \neq \phi \bigg\}.
$$
Clearly, we have, for any $A, B, C \subset X$,

\noindent
$G(A| B) = G(B|A)^{-1}, G(A \cup B|C) = G(A | C) \cup G(B|C), G (\{A
\cap B \} | C) \subset$ $G(A|C) \cap G (B | C)$ and for any $s, t, \in G$,  
$$
G(sA | tB) = s G(A | B)t^{-1}.
$$

We shall denote the orbit of $x \in X$ (i. e. the set $\{sx | s \in G
\}$) by $Gx$, and the space of all orbits by $G\slash X$. We shall
denote by $G(x)$ the isotropy group at $x \in X$; thus $G(x) =
G(\{x\}| \{x\})$. 

In what follows, we shall suppose that $G$ is locally compact, and that
$X$ is a Hausdorff space. 

\section{Proper groups of transformations}\label{chap1:sec1}%Sec $ 1.

\begin{defi*}
A\pageoriginale locally compact transformation group $G$ of a Hausdorff topological
space $X$ is \textit{proper} if the following condition is
satisfied. $(P)$ For any $x, y \in X$, there exist neighbourhoods $U$
of $x$ and $V$ of $y$ such that $G(U | V)$ is relatively compact. 
\end{defi*}

Clearly $(P)$ implies

\noindent
$(P_1)$ For any $x \in X$, there exists a neighbourhood $U$ of $x$
such that $G(U | U)$ is relatively compact. 

Although $(P_1)$ implies $(P)$ in many cases, it is not equivalent to
$(P)$, as the following example shows. 

\begin{example*}
  Consider the action of $\mathbb{Z}$ (with the discrete topology) on
  $\mathbb{R}^2-\{0\}$ defined by 
  $$
  n(x, y) = (2^n x, 2^{-n} y), (x, y) \in \mathbb{R}^2- \{ 0 \}, n \in
  \mathbb{Z}. 
  $$
\end{example*} 

Clearly $(P_1)$ is satisfied, but $(P)$ fails to hold, for instance
for the pair of points $(1, 0)$ and $(0, 1)$. 

Also, $\{P_1\}$ implies the condition

$(P_2)$ Let $\{s_n\}$ be any sequence in $G$, and suppose that for
some $x \in X, \{s_n~ x\}$ converges in $X$, then there exists a
compact set in $G$ which contains all the $s_n$. 

Again, $(P_2)$ implies $(P)$ in many cases.

\begin{rem}%remark 1.
  Let $G$ act on two spaces $X$ and $Y$, and let $f : X \to Y$ be a
  continuous mapping which commutes with the action of $G$,  i. e. we
  have $f (sx) = sf(x)$ for every $x \in X$ and $s \in G$. Then it is
  clear that if $G$ acts\pageoriginale properly on $Y$, it acts properly on $X$. This
  applies in particular to the natural action of $G$ on a subspace $X$
  of $Y$ which is stable under the action of $G$ (i. e. for which $Gx
  \subset X$ for all $x \in X$). 
\end{rem}

\begin{rem}%remark 2.
  It is easy to see that $(P_1)$ is equivalent to the condition: every
  point of $X$ has a $G$-stable open neighbourhood, on which the action
  of $G$ is proper. Thus $(P)$ is not a local property. On the other
  hand, it is easy to see that $(P_1)$ implies $(P)$ if the orbit space
  $G \backslash X$ is Hausdorff.  
\end{rem}

\section[Some properties of proper transformation...]{Some properties of proper transformation groups}\label{chap1:sec2}%Sec 2

In this article, it is assumed that $G$ is a proper transformation
group of the space $X$. 
\begin{enumerate}[(i)]
\item If $A, B \subset X$ are relatively compact (resp. Compact),$G(A
  | B)$ is relatively compact (resp. compact). (Note that $G(A | B)$
  is closed whenever $A$ is closed and $B$ is compact.) The proof is
  immediate. 

In particular, $G(x) = G(\{ x \}|\{ x \})$ is compact.

\item The orbit space $G | X$ is a Hausdorff space.
  \begin{proof}
    Since the equivalence relation defined on $X$ by $G$ is open, we
    have only to check that the graph 
  $$
    \Gamma = \{(x, y)\in X \times X |  x \in Gy \}
  $$
    of the relation is closed in $X \times X$. Thus let $(a, b)\in
    \bar{\Gamma}$. Then the family $\{G(U | V) | U $ a neighbourhood of
    $a, V$ a neighbourhood of $b \}$ generates a filter on $G$. Since $G$
    acts properly, this filter contains a compact set. Hence\pageoriginale there exists
    a $t \in G$ such that $t \in \overline{G(U |V)}$ for all the $U, V$,
    and it is easily seen that $tb = a$. This proves that $\Gamma$ is
    closed. 
  \end{proof}
  In particular, each orbit is closed in $X$.

\item For every $x \in X$, the mapping $m_x : s \rightsquigarrow sx$
  of $G$ onto $Gx$ is proper. (Since $Gx$ is closed in $X$, this is
  equivalent to saying that $m_x : G \rightarrow X$ is proper.) 

[We recall that a continuous mapping $f :X \to Y$ of Hausdorff spaces
  is \textit{ proper } if $(a)~f$  is closed, and $(b)$ for every $y
  \in Y, f^{-1}(y)$ is compact.] 
\begin{proof}
  For any $y = sx \in Gx, m_x^{-1}(y) = sG(x)$ is compact by $(i)$. We
  shall now show that $m_x$ is closed. Let $F$ be a closed set in $G$;
  we must show that $m_x(F)=Fx$ is closed. Let $y \in \bar{Fx}$, and let
  $U, V$ be neighbourhoods of $x, y$ respectively such that $G(V
  |U)\subset K, K$ compact. Then $F x \cap V=(F \cap K)x \cap V$  is
  closed in $V$, since $(F \cap K)x$ is compact. Thus $Fx$ is closed in
  a neighbourhood of every point of $\bar{Fx}$, hence $Fx$ is closed. 
\end{proof}
Thus in the canonical decomposition
$$
G \to G/_{G(x)} \xrightarrow{f}Gx \to X
$$
$f$ is a closed continuous bijection, hence a homomorphism. In other
words, the orbits (with the topology induced from $X$) are homogeneous
spaces of $G$.  

\item Let $G'$ be a locally compact group, and $h:G' \to G$ a
  continuous homomorphism. Then $G'$ also acts on $X$ in a natural way
  if we set, for $s' \in G'$ and $x \in X, s'x=h(s')x$. We have $:G'$
  acts properly\pageoriginale on $X$ if and only if the mapping $h$ is proper. 
\begin{proof}
  We have, for $A, B \subset X$,
  $$
  G'(A | B) = h^{-1}[G(A | B)];
  $$
  hence if $h$ is proper, $G'$ acts properly on $X$.
\end{proof}

For the converse, we first note that $G'$ also acts on $G$ by means of
$h$; we may set, for $s' \in G', s \in G, s's=h(s')s$. And the mapping
$m_x:G \to X$ commutes with the actions of $G'$ on $G$ and $X$. Hence
if $G'$ acts properly on $X$, it acts properly on $G$ (Remark $1,
1$). Hence by (iii) the mapping $s' \rightsquigarrow h(s')e_G=h(s')$
is proper. 
\end{enumerate}

In particular, every closed subgroup of $G$ acts properly on $X$.

\begin{example*}
  Let $G$ be a locally compact group, and $K$ a compact subgroup. Then
  the action of $G$ (by left multiplication) on the space $G/K$ of left
  cosets of $G$ modulo $K$ is a proper action. 
  
  In fact, let $q : G \to G/K$ be the natural mapping, and let $q(s), q
  (t) \in G/K$. If $U$ and $V$ are compact neighbourhoods of $s, t$
  respectively in $G, q(U), q(V)$ are neighbourhoods of $q(s), q(t)$
  respectively, and 
  \begin{align*}
    G (q(U) | q(V)) &=\left\{ s \in G | (sVK)\bigcap (UK) \neq \phi \right\}\\
    &=(UK)(VK)^{-1},
  \end{align*}
  which is compact.
\end{example*}

Using $(iv)$, we see that every closed subgroup of $G$ acts properly
on $G/K$. 

\section[A characterisation of proper transformation groups]{A characterisation of proper transformation\hfill\break groups}%sec 3.

\begin{thm}%them 1.
  Let\pageoriginale $G$ be a locally compact group of  transformations of the
  Hausdorff space $X$. In order that $G$ be proper, it is necessary and
  sufficient that the mapping $f : (s, x)\rightsquigarrow (sx, x)$ of
  $G \times X$ into $X \times X$ be proper. 
\end{thm}

\begin{proof}
  \textit{Sufficiency :} Let $x, y \in X$ be given.
\end{proof}


\begin{case}%Case 1.
  If $x \notin Gy$, then $(x, y) \notin f(G \times X)$. Since $f$ is
  proper, $f(G \times X)$ is closed in $X \times X$. Hence there exist
  neighbourhoods $U$ of $x$ and $V$ of $y$ such that $(U \times V) \cap
  f(G \times X) = $, i.e., $G(U | V) = $. Hence in this case,
  the condition $(P)$ is trivially satisfied. 
\end{case}

\begin{case}% case 2.
  Let $x \in Gy$. Then $f^{-1}((x, y)) = G(x | y)\times y$ is compact,
  since $f$ is proper. Hence $G(x |y)$ is compact; let $W$ be a compact
  neighbourhood of $G(x | y). ~ W \times X$ is a neighbourhood of
  $f^{-1}(x, y)$; since $f$ is proper, there exists a neighbourhood $U
  \times V$ of $(x, y)$ such that $f^{-1}(U \times V)\subset W \times
  X$. Then the projection of $f^{-1}(U \times V)$ on $G$ is contained in
  $W$. But this projection is precisely $G(U|V)$, and $W$ is compact,
  hence $(P)$ is verified for $(x, y)$. 
\end{case}

\medskip
\noindent{\textbf{Necessity.}}
 We first prove the
\begin{lem}\label{chap1:lem1}%lemma 1.
  Let $G$ be a proper transformation group of the space $X$. Then, for
  every $x \in X$ and every neighbourhood $W$ of $G(x)$ in $G$, there
  exists a neighbourhood $U$ of $x$ such that $G(U | U) \subset W$. 
\end{lem}

\begin{proofofthelemma*}
  $W$ may be assumed open. Let $V$ be a neighbourhood of $x$ such that
  $G(V | V)$  is relatively compact, and let $A=G(V | V)-W$. Then
  $\bar{A}\cap G(x)=\phi$ (note that  $G(x)\subset W$). Hence, for
  every $t \in \bar{A}$, there exist neighbourhoods $W_t$ of $t$ and
  $V_t$ of $x$ such that $(W_t~V_t)\cap V_t=\phi$. Since\pageoriginale
  $\bar{A}$ is 
  compact, we have a finite subset $F$ of $\bar{A}$ such that
  $\bar{A}\subset \bigcup \limits_{t \in F}W_t$. Let $U=V \cap
  \bigcap\limits_{t \in F}V_t$. Then clearly $G(U |U)\subset G(V |V)$
  and $G(U | U)\cap A \subset \bigg\{ \bigcap\limits_{t \in F}G(V_t |
  V_t)\bigg\}\cap  \bigcup\limits_{t \in F}W_t= \phi$, hence $G(U |
  U)\subset W$. 
\end{proofofthelemma*}

We now proceed with the proof of the theorem. Suppose that $G$ acts
properly on $X$. Then for any $(x, y) \in X \times X, f^{-1}((x, y)) =
G(x~y)\times y$ is compact. Hence we need only prove that $f$ is
closed. 

Let $F \subset G \times X$ be closed. since $f(G \times X)$ is the
graph of the relation defined by $G$, it is closed in $X \times X$ (\S\
\ref{chap1:sec2}, (ii)), so that $\bar{f(F)} \subset f(G \times X)$. Let $f(s, y) =
(x, y)\in \bar{f(F)}$. We must  show that $(x, y)\in f(F)$,
i.e., $f^{-1}((x, y)) \cap F \neq \phi$. Suppose this is false. since
$f^{-1}(x, y) = sG(y) \times y$, and $G(y)$ is compact, we then have
neighbourhoods $W$ of $G(y)$ and $V$ of $y$ such that $(sW \times V)
\cap F = \phi$ (recall that $F$ is closed). Now, by Lemma \ref{chap1:lem1}, there
exists a neighbourhood $U$ of $y$ such that $G(U | U) \subset W$;
clearly we may assume $U \subset V$. We then have 
$$
f^{-1} (sU \times U) \subset G (sU | U) \times U = sG(U~U) \times U
\subset sW \times V. 
$$
Hence $f^{-1} (sU \times U) \cap F = \phi$. It follows that $(sU
\times U) \cap f(F) = \phi$, which is a contradiction since $sU \times
U$ is a neighbourhood of $(x, y)$. 

\section{Existence of invariant metrics}\label{chap1:sec4}%sec 4.

If $G$ is a \textit{compact} Lie group operating differentiably on a
paracompact differentiable manifold $X$, it is well-known that there
exists a Riemannian metric on $X$, invariant under the action of
$G'$. We shall show now that similar results hold for proper
transformation groups of locally compact spaces. 

We\pageoriginale begin with the


\begin{lem}\label{chap1:lem2}%lemma 2.
Let $G$ be a locally compact group acting properly on a locally
compact space $X$, and suppose that $G \backslash X$ is
paracompact. Then there exists a closed set $A$ in $X$, and an open
neighbourhood $B$ of $A$ such that  
\begin{enumerate}[(i)]
\item $GA=X$,
\item for every compact set $K \subset X, G(B | K)$ is relatively compact.
\end{enumerate}
\end{lem}

\begin{proof}
Let $q : X \to_G \backslash X$ be the natural mapping; in the proof we
use the following statement, valid for any open mapping of a locally
compact space onto another; for any relatively compact open set $W$ in
$G \backslash X$ and any compact set $K \subset W$, there exists a
relatively compact open set $U$ in $X$ and a compact set $K_1 \subset
U$ such that $q(U) = W$ and $q(K_1) =K$. 
\end{proof}

Since $_G \backslash X$ is paracompact (and locally compact), we can
cover it by a locally finite family $(W_1)_{i \in I}$ of relatively
compact open sets. Let $(V_i)_{i \in I}$ be a covering og $_G
\backslash X$ such that $\bar{V}_i \subset W_i$ for every $i \in
I$. We now choose, for every $i \in I$, a relatively compact open set
$U_i$ in $X$ and a compact set $A_i \subset U_1$ such that $q(U_i) =
W_i$ and $q(A_i) = \bar{V_i}$. Let $A = \cup A_i, B = \cup U_i$. Now
$(U_i)_{i \in I}$ is a locally finite family on $X$. Hence $A$ is a
closed set in $X$, and clearly $GA = X$. Now, let $K$ be any compact
set in $X$. Since $G(U_i|K) =\phi$ implies $W_i \cap q (K) \neq \phi$,
and $(W_i)_{i \in I}$ is locally finite, $G(U_i | K)= \phi$ for only
finitely many $i \in 
I$. Since each $G(U_i| K)$ is relatively compact, it follows that $G(B
| K)$ is relatively compact. 

\begin{remark*}
  Suppose\pageoriginale a group $G$ acts on a locally compact
  paracompact space $X$, 
  such that $_G \backslash X$ is Hausdorff. Then $_G \backslash X$ is
  paracompact whenever the connected components of $X$ are open, or $X$
  is countable at infinitely, or $G$ is connected. 
\end{remark*}

\begin{thm}\label{chap1:thm2}%them  2.
  Let $G$ be a Lie group acting properly and differentiably on a
  paracompact differentiable manifold $X$. Then $X$ admits a
  Riemannian metric invariant under $G$.  
\end{thm}

\begin{proof}
  Since $X$ is paracompact, there exists a Riemannian metric $g$ on
  $X$. Further, if $A$ and $B$ are as in Lemma \ref{chap2:lem2}, there exists a
  differentiable function $f \geq 0$ on $X$, such that $f = 1$ on $A$ and
  $f = 0$ on $X-B$. 

  Let $x \in X$; let $T_X$ be the tangent space of $X$ at $x$, and $s^T
  = s_X^T : T_x \to T_{sx}$ the differential at $x$ of the mapping $y
  \rightsquigarrow sy$. Then for any $u, v \in T_X, s \rightsquigarrow
  f(sx) g (s^Tu, s^Tv)$ is a continuous function on $G$, whose support
  is compact since $f(sx) \neq 0$ implies $s \in G(B| \{ x \})$. Let
  $ds$ be a right-invariant Haar measure on $G$. If we set 
  $$
  g'_X(u^I, v) = \int\limits_G f(sx)~g ~ (s^Tu, s^Tv)ds,
  $$
  It is easily verified that $x \rightsquigarrow g'_X$ is a Riemannian
  metric on $X$, invariant under the action of $G$. 
\end{proof}

\begin{thm}%them 3.
  Let $G$ be a locally compact group acting properly on a locally
  compact metrisable space $X$ such that $_G \backslash X$ is
  paracompact. Then $X$ admits a $G$-invariant metric compatible with
  its topology. 
\end{thm}

\begin{proof}
  Let\pageoriginale $d$ be a metric on $X$, and let $B$ be as in Lemma
  \ref{chap2:lem2}; thus $B$ 
  is open, $GB=X$, and for any compact set $K \subset X, G(B | K)$ is
  relatively compact in $G$. Define 
  $$
  r(x)=d(x, X - B), x \in X.
  $$
  Clearly, for any $x, y \in X, r (x) - r(y) \leq d(x, y)$, and hence,
  for any $x, y, z, \in X$, 
  $$
  r(x) + r(z) \leq d(x, y) + \{r(y) + r(z) \}.
  $$
\end{proof}

Thus, if we define
$$
h(x, y)= \inf\{d(x, y), r(x)+r(y)\}, x,y \in X,
$$
it is clear that $h$ is a pseudo-metric on $X$; note that if $x \in B,
h(x, y)>0$ for $y \neq x$. Now the function $s \rightsquigarrow h(sx,
sy)$ is continuous. Its support is compact, since $h(sx, sy)\neq 0$
implies $s \in G(B|\{x, y\})$. Set 
$$
D(x, y) = \int\limits_G h (sx, sy)ds,
$$
with $ds$ a right-invariant Haar measure on $X$. Then clearly $D$ is a
continuous $G$-invariant distance function on $X$. We shall now verify
that it defines the topology of $X$. Since $GB = X$, and since $D$ as
well as the topology of $X$ is $G$-invariant, we have only to show
that, for every $x \in B$, every neighbourhood $W$ of $X$ contains a
$D$-neighbourhood of $x$. 

We choose an $r$, $0 < r < r(x)$, such that
$$
\mathscr{B} = \{z \in X | d (z, x) \leq r\} = \{z \in X | h(z, x) \leq
r \} 
$$
is\pageoriginale compact and contained in $W$. It is sufficient to find a compact
neighbourhood $V$ of $e$ in $G$ such that, for any $y \in X, h(x, y)>
r$ implies $h(sx, sy) > \dfrac{r}{2}$ for every $s \in V$. For then 
$$
\mathscr{B} \supset \{z \in X | D(x, z) < R\}, \text{ where }R =
\frac{r}{2}\int\limits_V ds. \text{ In fact, if } 
$$
$z \in X- \mathscr{B}, h(z, x)> r$, hence 
\begin{align*}
  D(x, z) &= \int\limits_G h(sx, sz)ds\\
  & \geq \int\limits_V h(sx, sz) ds \geq \frac{r}{2}\int\limits_V ds.
\end{align*}

We proceed to find such a $V$. Let $U$ be a compact symmetric
neighbourhood of $e$ in $G$ such that for $s \in U,h(x, sx) \leq
\frac{r}{2}$. Then, since the continuous function 
$$
(s, y)\rightsquigarrow h(sx, sy)-h(x, y)
$$
Vanishes on the compact set $\{ e \}\times U \mathscr{B}$ in $G \times
X$, we can find a compact  neighbourhood $V \subset U$ of $e$ such
that $| h(sx, sy)-h(x, y)|\leq \frac{r}{2}$ for $(s, y)\in V \times
U\mathscr{B}$. We claim that this $V$ suffices. In fact suppose for an
$s \in V$ and $y \in X$ that $h(sx, sy) \leq \frac{r}{2}$. Then $h(x,
sy) \leq h(x, sx)+h(sx, sy)\leq r$, so that $sy \in \mathscr{B}$,
i.e., $y \in V^{-1}\mathscr{B}\subset U \mathscr{B}$. Hence $|h(sx,
sy)-h(x, y)| \leq \frac{r}{2}$, and so $h(x, y) \leq r$. 

\setcounter{rem}{0}
\begin{rem}\label{chap1:rem1}%remark 1.
  If $G$ is a group of isometric transformations of a metric space
  $X$,   the condition $(P_1)$ and $(P)$ of \S \ref{chap1:sec1} are
  equivalent. In fact, 
  let\pageoriginale 
  $d$ be the metric on $X$, and suppose $(P_1)$ is satisfied.  Let $x, y
  \in X$, and let $W=\{z \in X | d(x, z)< r\}$ be a neighbourhood of $x$
  such that $G(W | W)$ is relatively compact in $G$. Let 
  
  $U=\{z \in X | d(z, x) < \frac{r}{3}\}, V = \{ z \in X | d (z, y)<
  \frac{r}{3} \}$. Then $G(V | U)$ is relatively compact. For let $s,
  s_o \in (V |U)$. Then there exist $z, z_o \in U$ such that $sz, s_o
  z_o \in V$, and we have 
  \begin{align*}
    d(s^{-1}s_0z_0,x) &=d(s_0z_0, sx)\\
    &\leq d(s_0 z_0, y)+d(y, sz)+d(sz, sx)\\
    &<\frac{r}{3}+\frac{r}{3}+\frac{r}{3}=r
  \end{align*} 
  so that $s^{-1}s_o \in G(W | W)$. Thus $G(V | U)\subset s_o ~ G(W | W)$.
\end{rem}

\begin{rem}%remark 2.
  Let $G$ be a locally compact group of isometric transformations of a
  metric space. Assume that $G$ is countable at infinite. Then the
  condition $(P_2)$ of \S\ \ref{chap1:sec1} implies $(P_1)$, and hence
  $(P)$ by Remark 
  \ref{chap1:rem1}. In fact let $G=\bigcup^{\infty}_{1}K_n, K_n$ compact and $K_n
  \subset K^0_{n+1}$. Suppose that $(P_1)$ fails at $x \in X$. Let $U_n=
  \{ z \in X | d (z, x)< \dfrac{1}{n}\}, n=1,2, \ldots $ since no $G(U_n
  |U_n)$ is relatively compact in $G$, we have, for every $n, a g_n
  \notin K_n$  and an $x_n \in U_n$ such that $g_n x_n \in U_n$. Then 
  \begin{align*}
    d(g_n x, x) &\leq d(g_nx, g_n x_n)+d(g_n x_n, x)\\
    &\leq \frac{1}{n}+\frac{1}{n}
  \end{align*} 
  so that $g_nx$ converges to $x$. However, for every $n > 0,g_n \notin
  K_n$, and every compact set in $G$ is contained in some $K_n$, so
  that $(P_2)$ fails. 
\end{rem}
