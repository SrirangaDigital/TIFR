\chapter{}\label{chap5}%%% 5
   
For\pageoriginale proper action a discrete group $\Gamma$ on a space with
compact orbit space $\Gamma / X$, there are rather strong
connections between the topological properties of $X$ and the
properties of $\Gamma$. The theory of ends, due to Freundenthal
\cite{key1} and Hopf \cite{key1} is the most  conspicuous example of such a
connection. 

\section{}%sec 1

Let $X$ be a connected topological space. We denote by
$\mathfrak{L}$ the set of all sequences $(a_i)$ of connected sets in
$X$ such that  
\begin{enumerate}[(i)]
\item for every $i, a_i  \neq \phi$,
\item $a_i \supset a_{i+1}$ for all $i$, 
\item each $a_i$ has compact boundary,
\item for every compact set $K$ in $X$, there exists an $i$ such that
  $a_i \cap K= \phi$. 
\end{enumerate} 

For $(a_i)$, $(b_i) \in \mathfrak{L}$, we write $(a_i) \sim (b_i)$
if for every $i$ there exists a $j$  such that $a_i \supset
b_j$. The relation $\sim$ is an equivalence relation
   $\mathfrak{L}$. Indeed we need only check that it is symmetric. Let
   $(a_i) \sim(b_i)$. For any $i$, there exists $a$ $j$ such that $a_j
   \cap \partial b_i = \phi$. Since $a_j$ is connected and $a_j
   \not\subset X -b_i$, it follows that $a_j \subset b_i$. 
   
   An equivalence class of $\mathfrak{L}$ with respect to the relation
   $\sim$ is called an \textit{end} of $X$. The set of all ends of
   $X$ is denoted by $\mathscr{E} (X)$. 
   
\begin{remark*}
  For  $(a_i)$, $(b_i) \in  \mathfrak{L}$ with $(a_i)\chi (b_i)$, there
  exists a $k$ such that $a_k \cap b_k = \phi$. In fact, there exists an
  $i$ such that, for every $j, a_i \not\supset b_j$. On the other hand,
  for a sufficiently large $j$, we have\pageoriginale  $a_i \cap \partial b_j =
  \phi$. Then for $k >  i,j$ we have $a_k \cap b_k = \phi$.   
\end{remark*}  
    
    Let $a \in  \mathscr{E}(X)$, and let $(a_i) \in \mathfrak{L}$
    represent $a$. By  a \textit{neighbourhood} of a we mean any
    subset of $X$ which contains $a_i$ for some $i$. If $V$  is a
    neighbourhood of $a$, it it clear that, for  \textit{any} $(a_i)$
    representing $a, V \supset a_i$ for some $i$. 
    
    We also need the notion of \textit{ends} of graphs. Let $X$ be a
    connected graph (see Chapter \ref{chap3}, \S\
    \ref{chap3:sec2}). For any $A \subset X$, we define the
    \textit{boundary} of $A$, denoted by $\partial A$, as the set  
    $$
    \bigg\{ x \in  X \big| \Sigma (x) \cap A \neq \phi, \Sigma (x)
    \cap (X-A) \neq \phi \bigg\}. 
    $$   
    
    It is easily seen that if $C \subset X$  is connected and $C \cap
    \partial A= \phi$, then either $C \subset A$ or $C \subset X-A$. Now let
    $\mathfrak{L}$ be the set all sequences $(a_i)$ of connected
    subsets of $X$ such that  
    \begin{enumerate}[(i)]
    \item $a_i \neq \phi$ for every $i$, 
    \item $a_i \supset a_{i+1}$ for every $i$, 
    \item $\partial a_i$ is finite for every $i$, 
    \item $\bigcap\limits_{i} a_i = \phi$.
    \end{enumerate}   
    
    We define the equivalence relation $\sim$ in $\mathfrak{L}$ as in
    the topological case, and the quotient set is the set of
    \textit{ends} of $X$, denoted by  $\mathscr{E}(X)$. The
    \textit{neighbourhoods} of points of $\mathscr{E}(X)$ are defined
    as in the topological case. 
    
    We note that a group which acts as a group of automorphisms on a
    connected space (or graph) $X$ also acts on $\mathscr{E}(X)$ in a
    natural way. 
    
    The following theorem will enable us to speak of the ``set of
    ends'' of any finitely generated group. 

\setcounter{thm}{0}
\begin{thm}\label{chap5:thm1}%the 1
  Let\pageoriginale $X$ and $Y$ be connected countable graphs finite at each point,
  and let $f:  X \to Y$ be a homomorphism. Suppose that  
  \begin{enumerate}
    \renewcommand{\labelenumi}{\rm (\theenumi)}
  \item for every $y \in  Y, f^{-1}(y)$ is finite.
    
    Then there exists a unique map $f^\varepsilon;\mathscr{E}(X) \to
    \mathscr{E}(Y)$ such that, for any $a \in \mathscr{E}(X)$ and any
    neighbourhood $V$ of $f^{\varepsilon}(a), f^{-1}(V)$ is a
    neighbourhood of $a$. If we further suppose that 
  \item $f$ is surjective, and for each connected $C' \subset Y$ with
    $C'$ finite, there exists a finite $H \subset Y$ such  that $H
    \supset \partial (f(C))$ for every connected component $C$ of
    $f^{-1}(C')$, then $f^{\varepsilon}$ is surjective. Finally, if we
    suppose in addition  that  
  \item for every $y \in Y, f^{-1} (y)$ is connected, then $f^\varepsilon$
    is bijective. 
  \end{enumerate}
\end{thm}
     
\begin{proof}
  Let $K_1\subset K_2 \subset \cdots$ be finite subsets of $Y$ such that
  $\bigcup K_i=Y$. Let $a \in  \mathscr{E}(X)$, and let $(a_i) \in
  \mathfrak{L}(X)$ represent $a$. Since, by (1), each $f^{-1}(K_i)$ is
  finite, there exists a $j(i)$ such that $f^{-1}(K_i) \cap a_{j(i)} =
  \phi$;  we assume  that $j(i)$  is the least
  integer with this property. Let $b_i$ be the connected component of
  $f(a_{j(i)})$ in $Y-K_i$. We assert that $(b_i) \in
  \mathfrak{L}(Y)$. It is clear $b_i \neq \phi$  and $b_{i+1} \subset
  b_i$ for every $i$. And since $b_i \subset Y-K_i, \cap b_i =
  \phi$. Also, $b_i$ being a connected component of $Y-K_i, \partial b_i
  \subset \partial (Y-K_i)= \partial K_i$ which is finite since $Y$  is
  finite at each point. Hence $(b_i) \in \mathscr{L} (Y)$. Let $b$ be
  the end of $Y$ defined by $(b_i)$. We set $f^\varepsilon(a) =b$. It is easily
  checked that $f^{\varepsilon}$ is a well-defined map from
  $\mathscr{E}(X)$ to $\mathscr{E}(Y)$.    
\end{proof}     
     
Now let $V$ be any neighbourhood of $b= f^\varepsilon(a)$. Then
$V \supset b_i \supset f(a_{j(i)})$ for some $i$. Thus $f^{-1}(V)
\supset a_{j(i)}$, and hence is\pageoriginale a neighbourhood of $a$. Suppose
$f^{\varepsilon_1}: \mathscr{E}(X) \rightarrow \mathscr{E}(Y)$ is
any map having this property. We assert that $f^{\varepsilon}_1=
f^{\varepsilon}$. Suppose in fact that $f^{\varepsilon}_1(a) \neq
f^{\varepsilon}(a)$ for some $a \in \mathscr{E}(X)$. Let $V,V_1$
be neighbourhoods of $f^\varepsilon(a), f_1^{\varepsilon}(a)$
respectively such that $V \cap V_1 = \phi$. Then $f^{-1}(V) \cap
f^{-1}(V_1)= \phi$, contradicting the assumption that $f^{-1}(V),
f^{-1}(V_1)$ are neighbourhoods of $a$.  

We now assume (2), and prove that $f ^{\varepsilon}$ is
surjective. Let $b \in \mathscr{E}(Y)$ and let $(b_i)$ represent
$b$. For every $i$, we choose a finite subset $H_i$ of $Y$ such
that $H_i \supset \partial (f(C_i))$ for every connected
component $C_i$ of $f^{-1}(b_i)$. Also let $j(i)$ be the least
integer such that $H_i \cap b_{j(i)}= \phi$. 

Let $a_1$ be any  connected component of $f^{-1}(b_1)$ which
meets $f^{-1}\break (b_{j(1)})$. Since $\partial(f(a_1)) \subset H_1$,
we have $\partial(f(a_1)) \cap b_{j(1)} = \phi$.  

Also, $f(a_1) \cap b_{j(1)} \neq \phi$, since $a_1 \cap
f^{-1}(b_{j(1)})  \neq \phi$. Hence $f(a_1) \supset b_{j(1)'}$
i.e., $f(a_1)$ is a neighbourhood of $b$. 

Assume inductively that we have a sequence $a_1 \supset a_2
\supset \cdots \supset a_n$ of subsets of $X$ such that each
$a_i$ is a connected component of $f^{-1}(b_i)$ and $f(a_i)$ is a
neighbourhood of $b$. Then we take for $a_{n+1}$ any  connected
component of $f^{-1}(b_{n+1})$ which meets $a_n \cap
f^{-1}(b_{j(n+1)})$; such a connected component exists since  $f (a_n),
b_{n+1}$ and  $b _{j (n+1)}$ are all neighbourhoods of $b$ so
that $f(a_n) \cap b_{n+1} \cap b_{j(n+1)} \neq  \phi$. It can be
verified as in the case of $a_1$ that $f(a_{n+1}) \supset
b_{j(n+1)}$ and hence is a neighbourhood of $b$. It is also clear
that $a_{n+1} \subset a_n$. Since $\partial a_i
\subset(f^{-1}(b_i)) \subset f^{-1}(\partial b_i), \partial a_i$
is finite for every $i$.  Also  $\bigcap_{a_i}= \phi$.\pageoriginale Thus the
sequence ($a_i$) defines an end  a in $X$. We have
$f^\varepsilon(a) = b$, since every neighbourhood of $f ^\in (a)$
is also a neighbourhood of $b$. Hence $f^\varepsilon$ is
surjective. 

With the same assumptions, we assert that for any $a \in
\mathscr{E}(X) $ and any neighbourhood $U$ of a, $f(U)$ is a
neighbourhood of $f^\varepsilon (a)$. Let $ b = f^\varepsilon (a)$,
and let ($a_i$),($b_i$) represent $a$ and $b$ respectively. Since, for
every $i, f ^{-1}(b_i)$ is a neighbourhood of a, there exists a $j(i)$
such that $a_{j(i)}\subset f^{-1}(b_i)$. Let $a	'_i$be the connected
component of $f^{-1}(b_i)$  which contains $a_{j(i)}$. Clearly, $(a'_i)
\varepsilon \mathscr{L}(X)$. Since $a'_i \supset a'_{j(i)}$, it
follows that $(a'_i)\sim (a_i)$, i.e. $(a'_i)$ represents $a$. We now
assert that $(f (a'_i))\in \mathscr{L}(Y)$ and represents $b$. In
fact, $(f (a'_i))\in \mathscr{L}(Y)$ since, by (2),$\partial f
(a'_i)$ is finite, and the other conditions are clearly
satisfied. Since $f(a'_i)\subset b_i$, we have $(f(a'_i)) \sim (b_i)$.
Thus every $f(a'_i)$ is a neighbourhood of $b$;  it follows that
$f(U)$ is a neighbourhood of $b$. 

Finally, we assume in addition that (3) holds and prove that $f$ is
also injective. Let $a, a' \in \mathscr{E} (X), a \neq a'$. Let $V,
V'$ be neighbourhoods of $a, a'$ such that $  V \cap V' = V \cap
\partial V' = \phi $. Then $f(V)\cap f(V')= \phi$, Since $f(V), f(V')$
are neighbourhoods of $f^{\varepsilon}(a), f^{\varepsilon}(a')$
respectively, we must have $f^\varepsilon (a) \neq f^\varepsilon
(a')$, and Theorem \ref{chap5:thm1} is proved.   

Let $G$ be a (discrete) group. Let $S$ be a set of generators for
$G$ such that $e \varepsilon S$, and $S = S^{-1}$. Then we know that
$S$ defines a left invariant connected graph structure $\sum_S$ on
$G$,  given by $\mathscr{E}_S (x) = x S, x \in G$, We denote by
$\mathscr{E}_S (G)$ the set of ends of $(G, \sum_S)$.  

\begin{thm} %theorem 2
  Let\pageoriginale $G$ be a finitely generated group, and let $S, S'$ be two finite
  symmetric sets of generators of $G$ which contain $e$. Then there is a
  unique natural bijection $\varphi_{S, S'} : \mathscr{E}_S (G) \to
  \mathscr{E}_S'(G)$ such that, for any $a \in \mathscr{E}_S (G)$, any
  neighbourhood of a is also a neighbourhood $\varphi _{S, S'}(a)$. 
\end{thm}

\begin{proof}
  The uniqueness of $\varphi _{S, S'}$, is obvious. To find $\varphi
  _{S,S'}$, we first assume that $S \subset S'$. Then the identity
  mapping of $G$ is a graph homomorphism $\varphi : (G, \sum_s)\to
  (G, \sum_S')$. We assert that the conditions of Theorem are satisfied for
  $\varphi$. In fact, we need only verify condition (2). Thus let
  $C'$ be an $S'$-connected set with $\partial _S' C'$ finite. Let $n$
  an integer such that $S' \subset S^n$. We take $H = \partial _S'
  C'. S^r$, and claim that for any $S$-connected component $C$ of $C',
  \partial_{S'} C \subset H$. In fact let $x \in \partial_{S'}
  C$. Then $\partial_{S'}x 
  S' \cap C \neq \phi \neq x S' \cap (G - C). x S^r $ is $S$-connected,
  hence $S'$- connected. Since $xS^r \cap C \neq \phi$, we must have $x
  S^r \not\subset C'$, for otherwise  $xS^r \subset C$, contradicting
  $xS' \cap (G - C) \neq \phi $. Hence $x S^r \cap \partial_{S'}, C' \neq
  \phi$, i.e., $x \in H$.  
\end{proof}

Hence $\varphi_{S, S'}$ is the $\varphi^\varepsilon$ of Theorem \ref{chap5:thm1}. 

If $ S \not\subset S'' $, let $S'' = S \cup S'$, then we can take
$\varphi_{S, S'} = \varphi^{-1}_{S'', S'} \circ \varphi_{S, S''}$.  

In view of the above theorem, ends and their neighbourhoods are
intrinsically defined for finitely generated groups.  

\begin{thm}\label{chap5:thm3} %Theorem 3
  Let $G$ be a discrete group, operating properly on a connn\-ected,
  locally connected locally compact space $X$ such that $_G\backslash^X$
  is compact (consequently $G$ is finitely generated). Then there exists
  a unique\pageoriginale map $f : \mathscr{E}(G) \to \mathscr{E}(X)$ such that for
  $a \in \mathscr{E}(G)$ and any neighbourhood $V$ of $f(a), G(V|\{ x\}
  )$ is a neighbourhood of a for any $x \in X$. Moreover, $f$ is
  bijective, and commutes with the operation of $G$.  
\end{thm}

\begin{proof}
  We first prove the uniqueness. Let $f_1, f_2$ be two maps $\mathscr{E}
  (G)\to \mathscr{E}(X)$ having the properties stated in the
  theorem. Let $a \in \mathscr{E}(G)$, and $f_i(a) = b_i$; let $V_i$ be
  any neighbourhood of $b_i (i = 1, 2)$. Then, for any $x \in X, G (V_1
  |\{ x\}) \cap G(V_2| \{ x\})$ is a neighbourhood of $a$, and hence
  non-empty. Hence $V_1 
  \cap V_2 \neq \phi $. It follows that $b_1 = b_2$. Hence $f_1 = f_2$.  
\end{proof}

We now prove the existence of $f$. There exists a compact connected
subset $K$ of $X$ such that $ GK = X $. Let $S = G(K|K)$. Then $S =
S^{-1}$is finite, contains $e$, and generates $G$; and $SK$ is a
neighbourhood of $K$. We put on $G$ the graph structure defined by
$S$. 

Let $a \in \mathscr{E}(G)$, and let ($a_i$) represent a. Let $b_i =
a_i K$. We want  to prove that $(b_i )\varepsilon
\mathscr{L}(X)$. Clearly, $b_i \supset b_{i +1}$, and each $b_i$ is
connected. Also, for any compact set $H$ in $X, G(H|K)$ is finite,
hence $a_i \cap G(K' | K) = \phi $ for all large
$i$, i.e., $b_i \cap K' = \phi$ for all large $i$. Now, for any $t \in
a_i -  \partial a_i $, we have $ tS \subset a_i$, hence $tK \subset
tSK \subset b_i$; since $SK$ is a neighbourhood of $K$, we have $tK
\subset \Int b_i$. Since $(tK)_{t \in a_i}$ is locally finite. $b_i
= a_i K $ is closed, hence it follows that $ \partial b_i \subset
\partial a_i K$. Since $\partial a_i$ is finite, we have finally that
 $\partial b_i$ is compact. Hence $(b_i) \in \mathscr{L}(X)$.  

Let $b$ denote the end defined by ($b_i$). We set $f(a) = b$. Clearly $f
:\mathscr{E}(G) \to \mathscr{E}(X)$ is then well defined. Now let
$V$ be any neighbourhood\pageoriginale of $b = f(a)$, and let $ x \varepsilon
X$. Since $S$ generates $G$, and $GK = X$, there exists an integer $n$
such that $x \in S^n K $. It is easily seen that $ (a_j S^n K)$
represents $b$. Thus $V \supset a_j S^n K$ for some $j$. Hence $a_j
\subset G(V |\{ x\}|)$, i.e., $G(V| \{ x\})$ is a neighbourhood of $a$.  

We now prove that $f$ is bijective. Let $b \in \mathscr{E}(X)$, and
let ($ b_i$) represent $b$. We set $a_i =  G(b_i| K)$. Clearly $a_i
\neq \phi, a_i \supset a_{i+1}$ and $\cap a_i = \phi $. Further, since $K$
and $b_i$ are connected, and the family $(gK)_{g \in G}$ is locally
finite, we see easily that the $a_i$ are connected. Now, if $tS \cap
a_i =\phi $, we have $t \in G (b_i | K)S = G( b_i| SK)$. Similarly $tS
\cap (G-a_i)\neq \phi $implies $ t \in G ((X - b_i)| SK)$. Since $SK$
is connected, it follows that $\partial a_i \subset G(\partial b_i| SK)
$, and hence is finite. Thus, ($a_i$) defines an end $f' (b)$ of
$G$. Clearly $b \rightsquigarrow f' (b)$ is a well - defined map of
$\mathscr{E}(X)$ into $\mathscr{E}(G)$, and $f'$ is easily seen to
be the inverse of $f$.  

Finally, for any $t \in G, t^{-1} \circ f \circ t : \mathscr{E}(G) \to
\mathscr{E}(X)$, also has the properties mentioned in the theorem,
hence we have, by the uniqueness, $t^{-1} \circ f \circ t = f$, i.e. $ f \circ t =
t \circ f$. This completes the proof of the theorem.  

\section{} %Section 2

\setcounter{lem}{0}
\begin{lem}\label{chap5:lem1}%lem 1
  Let $X$ be a connected graph, and let $A, B, H$ be connected subsets
  such that $\partial A \subset H$ and $\partial B \subset A - H$. Then
  either $B \subset A$ or $A \cup B = X$. 
\end{lem}

\begin{proof}
  Since $\partial B \cap H = \phi $ and $H$ is connected we have either
  $H \subset B$ or $H \subset X-B$. If $H \subset B$, we have $\partial
  (A \cup B) \subset \partial A \cup \partial B \subset H \cup A \subset
  B \cap A$.  
  
Since\pageoriginale $X$ is connected, we have $A \cap B = X$ or $\phi$. 
\end{proof}

If $H \subset X - B$, we have $\partial A \cap B = \phi $. Hence
either $B \subset A$ or $B\subset X -A $. But since $\partial B
\subset A$, we must have $B \subset A$ or $B \subset X -A $. But since
$\partial B \subset A$, we must have $B \subset A$.  

\begin{thm} %Theorem 3
  Let  $G$ be a finitely generated group, and let $a^{(1)}, a^{(2)},
  a^{(3)}$, their distinct ends of $G$. Then for every neighbourhood $V
  of a^{(3)}$, there exists a $t \in G$ such that $V$ is a neighbourhood
  of at least of $ta^{(1)}$, $ta^{(2)}$, $ta^{(3)}$.  
\end{thm}

\begin{proof}
  Let $S$ be a finite set of generators for $G$ defining a graph
  structure. Let $a_i^{(j)}$ represent $a^{(j)}, j = 1, 2, 3$. We may
  assume that, for every $i$, the $a^{(j)}, j = 1, 2, 3$, are mutually
  disjoint. Now let $V$ be a neighbourhood of $a^{(3)}$, say
  $V \supset a_i^{(3)}$. Let 
  $n$ be an integer such that $S^n \supset \bigcup\limits_j a_i^{(j)}$. Take any
  $t\in a_i^{(3)}- S^{2n}$. Since $tS^n$ is connected, and since $tS^n
  \cap \partial a_i^{(3)}\subset tS^n \cap S^n = \phi$, it follows that
  $tS^n \cap a_i^{(3)} - S^n$. Hence Lemma \ref{chap5:lem1} can be applied,
  with $A = a^{(3)}_i, H = S^n$, and $B = ta_i ^{(j)}$. Since the $ta_i
  ^{(j)}, j = 1, 2, 3$ are mutually disjoint, we must have $ta_i ^{(j)}
  \subset a_i ^{(3)}$ for at least two the $j's $. This proves the
  theorem.  
\end{proof}

\begin{corollary}%corollary 1
  Let $G$ be a finitely generated group. If $G$ has three distinct ends,
  then every neighbourhood of an end of $G$ is the neighbourhood of two
  distinct ends; in particular, the set of ends is finite. 
\end{corollary}

\begin{corollary}%corollary 2
  If the finitely generated group $G$ has two invariant ends, it has
  no other ends.  
\end{corollary}

\begin{proof}
  Let $a, b$ be two invariant ends of $G$. If possible let $c$ be
  another end of $G$. By Theorem \ref{chap5:thm3}, there exists, for every
  neighbourhood $V$ of\pageoriginale $ c, a t \in G$ such that $V$ is
  a neighbourhood of at least one of $ ta = a,  tb = b, $. Hence $c
  = a$ or $ b, a$ contradiction.  
\end{proof}

\begin{remark*}
  It is known whether a group with one invariant end can have
  infinitely may ends (Freundenthal \cite{key1}). 
\end{remark*}

\begin{examples*}
  \begin{enumerate}[1)]
  \item The group $\mathbb{Z}$ has two invariant ends. 
  \item The group $\mathbb{Z} X \mathbb{Z}$ has just one end. 
  \item The free product of the cyclic group of order 2 with the
    cyclic group of order 3 (which is isomorphic to the classical
    modular group) has infinitely many ends, none of which is
    invariant. This example shows incidentally that in Theorem \ref{chap5:thm3}, the
    assumption that $_G \backslash^X$ is compact cannot be dropped.  
  \end{enumerate}
\end{examples*}
