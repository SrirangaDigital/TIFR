\part[Axiomatic Theory of Harmonic...]{Axiomatic Theory of Harmonic and Superharmonic\break Functions -
  Potentials}\label{p4}%part IV 

\setcounter{chapter}{0}
\chapter{Generalised Harmonic Functions}\label{p4:chap1}%chap I

\section{The Fundamental axioms}\label{p4:chap1:sec1}%sec 1

The\pageoriginale following theory of harmonic and superharmonic
functions derives 
its inspiration from the earlier axiomatic theories of Tauts \cite{5} and
mainly from Doob's theory \cite{3}. Doob wished to include the study of
certain differential equations, not only of elliptic type but also of
parabolic type and his principal scope was the interpretation of the
behaviour of the solutions at the boundary by means of the notions of
probability theory. With other, partly weaker partly stronger
hypothesis we want to follow more closely but much farther the
classical potential theory, first as far as a kernel may be
avoided. We do not examine the probabilistic interpretations. 

\medskip
\noindent
\textbf{Fundamental space.} We shall consider a connected locally
compact (but not compact) Hausdorff space $\Omega$. We introduce an
Alexandroff point to get the compactification $\bar{\Omega}$. 

\medskip
\noindent
\textbf{Harmonic Functions.} To each open set $\omega$ of $\Omega$ is
assigned a real vector space of real valued continuous functions,
called the harmonic functions in $\omega$, defined on $\omega$. The
vector space satisfy the following local axioms: 

\begin{Axiom}\label{p4:chap1:sec1:axiom1} % axiom 1
  \begin{enumerate}[(i)]
  \item If $\omega_0$ is an open subset of $\omega$, the restriction
    to $\omega_0$ of any harmonic function in $\omega$ is harmonic in
    $\omega_0$. 
  \item If\pageoriginale $u$ is a function defined in an open set $\omega$, and
    harmonic in an open neighbourhood of every point of $\omega, u$
    harmonic in $\omega$. 
  \end{enumerate}
\end{Axiom}

In order to state the second axiom we need the following Definition of
regularity of an open set. 

\setcounter{defn}{0}
\begin{defn}\label{p4:chap1:sec1:def1}% definition 1
  An open set $\omega$ in $\Omega$ is called regular if 
  (i) it is relatively compact in $\Omega$ (i.e., its closure in the
  topology of $\bar{\Omega}$ is in $\Omega$) 
  (ii) for any finite continuous function $f$ on the boundary
  $\partial \omega$ of $\omega$ there exists an unique harmonic
  function $H^\omega_f$ (briefly $H_f$) on $\omega$ such that $H_f$
  tends to $f$ at each point of the boundary and 
  $(iii)$ for such a function $f \ge 0, H_f \ge 0$.
\end{defn}

\begin{Axiom}\label{p4:chap1:sec1:axiom2}% axiom 2
  The second axiom states that there exists a base of regular domains
  for the open sets of the topology of $\Omega$. 
\end{Axiom}

The axiom \ref{p4:chap1:sec1:axiom2} implies that the space is
locally connected. Note that 
if an open set $\omega$ is regular, any connected component $\delta$
of $\omega$ is also regular and $H^\omega_f = H^\delta_f$ in
$\delta$. That is the result of the possible finite continuous
extension in $\Omega$ of a finite continuous function on the boundary
of $\delta$. 

\begin{defn}\label{p4:chap1:sec1:def2}% definition 2
  Harmonic measure: Let $\omega$ be any regular open set. For any
  point $x$ of $\omega, H^\omega_f (x)$ is a positive linear
  functional on the space of finite continuous functions defined on
  $\partial \omega$. In other words, $H^\omega_f$ defines a positive
  Radon measure on $\partial \omega$, denoted by $\rho^\omega_x$ or $d
  ~ \rho^\omega_x$, called the harmonic measure relative to $\omega$
  and $x$; so that we may write $H^\omega_f (x) = \int ~ f ~ d ~
  \rho^\omega_x$. 
\end{defn}

If $\delta$ is the component of $\omega$ containing $x \in \omega, d ~
\rho^\omega_x$ and $d ~ \rho^\delta_x$ considered as measure on
$\Omega$ (see Part \ref{p3}) are identical. 

\begin{Axiom}\label{p4:chap1:sec1:axiom3}% axiom 3
  Any\pageoriginale family of harmonic functions defined on a domain $\omega$ and
  directed for natural increasing order (ordered increasing directed
  family) has an upper envelope (or a pointwise limit following the
  corresponding filter) which is $+ \infty$ everywhere in $\omega$ or
  harmonic in $\omega$. 
\end{Axiom}

In case the space $\Omega$ satisfies second axiom of
countability\footnote{Unnecessary restriction
  (Constantinescu-Cornea: See Add. Chapter)} Axiom
\ref{p4:chap1:sec1:axiom3} is 
equivalent to a similar one with increasing sequences replacing any
arbitrary family. It is an immediate consequence of the topological
lemma of Choquet [Part \ref{p1}, Lemma \ref{p1:chap2:sec3:lem3}]. 

\section{Examples}\label{p4:chap1:sec2}% sec 2 

\begin{enumerate}[(a)]
\item The classical harmonic functions on an open set of
  $\mathbb{R}^n$ satisfy the three axioms: Axiom
  \ref{p4:chap1:sec1:axiom1} is satisfied 
  because of the local character of the Laplacian operator. Axioms
  \ref{p4:chap1:sec1:axiom2} and \ref{p4:chap1:sec1:axiom3} follow as
  consequences of the properties of 
  Poisson's integral. 

\item More generally in any euclidean domain $\Omega$, we shall
  consider the functions with continuous second order derivatives
  (partial) which satisfy the elliptic equation 
  $$
  \sum a_{ik} ~ \frac{\partial^2 u }{\partial x_i \partial x_k} + \sum
  ~ b_i \frac{\partial u}{\partial x_i} + Cu = 0 
  $$
  where $\sum a_{ik} ~ X_i ~ X_k$ is a positive definite quadratic
  form. All the coefficients and their first derivatives are supposed
  to be continuous and to satisfy the condition of Lipshitz locally;
  $C \le 0$ \footnote{This condition on $C$ is not necessary
    (See Herve Thesis; add. chap.)} 
\end{enumerate}


Axiom \ref{p4:chap1:sec1:axiom1} is satisfied because of the local character of
integrals. Axioms \ref{p4:chap1:sec1:axiom2} and
\ref{p4:chap1:sec1:axiom3} result from a local integral 
representation similar to the Poisson integral. 

Various\pageoriginale generalisations are possible, for instance,
using varieties 
instead of $R^n$. Other examples would be desirable, particularly in
connection with the general potential theory with kernels. 

\section{First consequences}\label{p4:chap1:sec3}%sec 3

\setcounter{prop}{0}
\begin{prop}\label{p4:chap1:sec2:prop1} % proposition 1
  (Form $A$). Any function $u \ge 0$, harmonic in a domain is
  everywhere greater than zero or everywhere equal to zero in
  $\omega$. 
\end{prop}

For, the sequence $\{nu \}$ has a limit which is everywhere $+ \infty$
or harmonic in $\omega$. Therefore if $u = 0$ at some point $u = 0$
everywhere. 

This property is equivalent to 

\noindent (A$'$) a harmonic function $u$ cannot
have a minimum zero at a point without being zero in some
neighbourhood of it. 

Note that (A) is a consequence of axiom
\ref{p4:chap1:sec1:axiom3} alone, even restricted 
to sequences.%raghu

We have immediately from (A),

\setcounter{corollary}{0}
\begin{corollary}\label{p4:chap1:sec3:coro1} % corollary 1
  On any regular open set $\omega$ there exists a harmonic function
  greater than $k > 0$, for instance $\int ~ d ~ \rho^\omega_x$. 
\end{corollary}

\begin{corollary}\label{p4:chap1:sec3:coro2}% corollary 2
  Consider the regular open sets $\omega$ containing a point $x_0$ in
  $E$ ordered by inclusion. From this directed family we shall use the
  filter $\mathscr{F}$ of sections. Then $\int ~ d ~ \rho^\omega_{x_o}
  \xrightarrow[\mathscr{F}]{}1$  and $\int ~
  d~\rho^\omega_{x_0}\xrightarrow[\mathscr{F}]{}$ to the unit mass at
  $x_o$ vaguely. 
\end{corollary}

For if $h$ is harmonic $> 0$ in the neighbourhood $\int h(y)d
\rho^\omega_{x_o} (y) = h(x_o)$ then $\int ~ d \rho^\omega_{x_o}
\underset{\mathscr{F}}{\longrightarrow} 1$. 

\setcounter{thm}{0}
\begin{thm}\label{p4:chap1:sec3:thm1} % theorem 1
  If\pageoriginale axioms \ref{p4:chap1:sec1:axiom1} and
  \ref{p4:chap1:sec1:axiom2} 
   are satisfied, Axiom \ref{p4:chap1:sec1:axiom3} is equivalent to
  the following property of the harmonic measure: for any regular
  domain $\omega$ (or only for these of a base) the summability
  relative to $d ~ \rho^\omega_x$ independent of $x$ in $\omega$ and
  for such a summable of on $\partial \omega \int f ~ d^\omega_x$ is
  finite continuous in $\omega$ and harmonic. 
\end{thm}

Let us assume axioms \ref{p4:chap1:sec1:axiom1},
\ref{p4:chap1:sec1:axiom2} 
and \ref{p4:chap1:sec1:axiom3}. Let $\omega$ be a regular domain
and $\psi$ a lower bounded and lower semi continuous function on
$\partial\omega$: $\int \psi ~ d ~ \rho^\omega_x$ for $x \in \omega$
is the supremum of $\int \theta ~ d ~ \rho^\omega_x$ as $\theta$
ranges through the continuous functions on $\partial \omega$ such that
$\theta \le\psi$; therefore $\int \psi d ~ \rho^\omega_x$ is either
harmonic or $+ \infty$. 

Let now $f$ be any function on $\partial \omega$, introducing the
functions $\psi$ which are lower bounded and lower semi continuous and
which are such that $\psi \ge f$. We know that $\bar{\int} f ~ d ~
\rho^\omega_x $ is defined and is equal to $\inf. \int \psi~ d
~\rho^\omega_x$ for these $\psi$. Hence $\bar{\int} ~ fd ~
\rho^\omega_x$: is either harmonic or identically $+ \infty$ or $-
\infty$. Similar conclusions can be derived as regards
$\underline{\int} f d \rho^\omega_x$, which is $\leq \bar{\int} f d
\rho^\omega_x$ : now with the aid of the property $(A)$ we deduce the
given properties of $d ~ \rho^\omega_x$ (the summability - $d ~
\rho^\omega_x$ signifies that $\bar{\int}$ and $underline{\int}$ are equal and
finite). 

Conversely let us assume the properties of $d ~ \rho_x^\omega$ and the
first and second axioms. Suppose $\{ f_j\}_{i \in I}$ is any family of
harmonic functions on a domain $\delta$, directed for increasing
order, then we want to prove that the (pointwise) supremum of this
family is $+ \infty$ everywhere or a harmonic function. By considering
$f_i - f_{i_o}$, we may suppose\pageoriginale for the proof that all $f_i \ge 0$. 

Let $\omega$ be a regular domain of the base $\mathscr{B}$ such that
$\bar{\omega} \subset \delta$. Then,  
$$
f_i(x) = \int ~ f_i (y) ~ d ~ \rho^\omega_x (y) \text{ for every } x
\text{ in } \omega. 
$$

Taking the limits of both the sides following the corresponding
filter, (i.e.the supremum), we get 
$$
\lim.  f_i (x) = \int \lim.f_i(y) ~ d ~ \rho^\omega_x(y)
$$

If the right handside is $+ \infty$ for some $x$ in $\omega$, then it
is $+ \infty$ for every $x$ in $\omega$; if it is finite for some $x$
in $\omega$, it is finite everywhere in $\omega$. It follows
immediately that the set of points of $\delta$ at which $\lim f_i(x)$
takes the value $+ \infty$ and the set of points at which the limit is
finite are two disjoint open sets of $\delta$ and consequently one of
then is empty. If limit $f_i(x)$ is finite on $\delta$, it is $d
\rho^\omega_x$ summable and $\int \lim f_i(y)d ~ \rho^\omega_x(y)$ is
finite and continuous in any regular domain $\omega$ of $\delta$ such
that $\omega \subset \bar{\omega} \subset \delta$. Therefore limit
$f_i(x)$ is continuous everywhere in $\delta$; then $\int \lim f_i
(y) ~ d ~ \rho^\omega_x(y)$ being harmonic in any $\omega, \lim ~
f_i(x)$  is harmonic in $\delta$ itself. 

\begin{coro*}% corollary  0
  On $\partial \omega$, the sets of harmonic measure zero are
  independent of $x$ in $\omega$. 
\end{coro*}

This is an immediate consequence of $(A)$ and of the harmonicity of
$\bar{\int} \varphi_e ~d ~ \rho^\omega_x, \varphi_e$ being the
characteristic function of the set $e$ contained in $\partial
\omega$. 

\begin{prop}\label{p4:chap1:sec3:prop2}% proposition 2
  Let\pageoriginale $\omega$ be a regular domain. Any neighbourhood of any point on
  the boundary $\partial \omega$ has non-zero measure as regards $d
  ~\rho^\omega_x$. 
\end{prop}

Suppose the contrary is true of a neighbourhood $N$ of $x_o$ in
$\partial \omega$. There exists a finite continuous function $f$ on
$\partial \omega$ which is equal to $1$ at $x_o$ and zero outside
$N$. Now the harmonic function $H_f = \int ~ fd ~ \rho^\omega_x$ is
zero in $\omega$ and must tend to $1$ at $x_o$. This is clearly an
impossibility. 

\begin{prop}\label{p4:chap1:sec3:prop3}% proposition 3
  Let $f$ be a function on the boundary $\partial \omega$ of a regular
  open set $\omega$. If $f$ is bounded above, the function $\bar{\int}
  ~ f ~ d ~ \rho^\omega_x$ satisfies for any $x_o$ in $\partial
  \omega$ 
  $$
  \Lambda = \underset{x \in \omega, x \rightarrow x_o}{\lim.\sup} ~
  \int ~ f ~ d ~ \rho^\omega_x \le \underset{y \in \partial \omega, y
    \rightarrow x_o}{\lim. \sup.f(y)} = \lambda 
  $$
\end{prop}

\begin{proof}
  If $\lambda < + \infty$, let us choose $\lambda_1 > \lambda$, there
  exists then a neighbourhood $U$ of $x_o$ in which $f(y) <
  \lambda_1$; since $f$ is bounded above, we can find a finite
  continuous function $F \ge f$, such that $F \le \lambda_1$ in a
  neighbourhood of $x_o$.  Then 
  $$
  \bar{\int} ~ f ~ d ~ \rho^\omega_x = \int ~ F ~ d ~ \rho^\omega_x = H^\omega_F(x)
  $$
  and the last integral tends to $F(x_o)$ as $x \rightarrow x_o (x \in
  \omega)$. We conclude therefore $\Lambda \le \lambda_1$ and hence
  $\Lambda \le \lambda$. 
\end{proof}

We have immediately a corresponding result for $\lim.\inf.\underline{\int}$.

\section{The case where the constants are harmonic}\label{p4:chap1:sec4}%sec 4

\begin{prop}\label{p4:chap1:sec4:prop4}% proposition 4
  In\pageoriginale this case $\int ~ d ~ \rho^\omega_x = 1$. Now for any harmonic
  function $u$ in an open set $\omega, u \ge inf $ [$\lim.\inf ~u$ at
    the boundary points]. 
\end{prop}

If this were not true, $u$ would attain its minimum $k$ (finite and
smaller than the right hand side) at a point $y_o \in \omega$. In the
connected component $\delta$ of $\omega$ containing the point $y_o, u
- k$ would be zero; $u$ would be equal to $k$ on $\delta$ although the
$\lim.\inf$ at any boundary point of $\delta$ is $> K$. This is a
contradiction. 

We have therefore a minimum principle and a similar maximum principle.

An important case where the constants are harmonic is the following one.

\section{\texorpdfstring{$h$}{h}-harmonic functions}\label{p4:chap1:sec5}%sec 5

\begin{defn}\label{p4:chap1:sec5:def3}% definition 3
  Let us observe that if $h$ is a finite and continuous function $> 0$
  in $\Omega$ then the quotients $u/ h$ of all harmonic functions $u$
  in $\Omega$ satisfy the three axioms with the same regular open
  sets. Moreover if $h$ is harmonic, the new family of functions
  contains constants. These new functions are called h-harmonic
  functions. The new harmonic measure $d ~ \rho'^\omega_x$ is such
  that the h-harmonic function in $\omega$ taking continuous boundary
  values $f(x)$ is $\int f(y) ~ d ~ \rho'^\omega_x (y)$ but also 
  $$
  \displaylines{
    \frac{1}{h(x)} ~ \int ~ h(y) ~ f(y) ~ d  ~ \rho^\omega_x (y);\cr
    \text{ that is }\hfill 
    d ~ \rho'^\omega_x (y) = \frac{h(y)}{h(x)} ~ d ~ \rho^\omega_x
    (y).\hfill }
  $$
\end{defn}
