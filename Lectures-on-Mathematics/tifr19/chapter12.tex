\chapter[Nearly Hyper or Superharmonic Functions...]{Nearly Hyper or Superharmonic Functions - Reduced
  Functions}\label{p4:chap3} 

\setcounter{section}{10}
\section{}\label{p4:chap3:sec11} % section 11

\begin{defn}\label{p4:chap3:sec11:def7} % definition 7
  Let\pageoriginale $\mathscr{B}$ be a fixed base of regular domains of $\Omega$. A
  real functions $v$ is said to be a $\mathscr{B}$- nearly
  hyperharmonic function or a $S_\mathscr{B}$- function if  
  \begin{enumerate}[\rm (i)]
  \item $v$ is locally  bounded below
  \item for every $\omega \in \mathscr{B}, v(x) \ge \bar{\int} ~ v ~ d
    ~ \rho^\omega_x$ ~(for every ~$x \in \omega$) 
  \end{enumerate}
\end{defn}

When $\mathscr{B}$ is the family of all regular domains, we say that
$v$ is a \textit{nearly hyper-harmonic} function, or a $S$ -
function. 

Note that a $S_\mathscr{B}$-function is not generally a nearly
hyperharmonic function (as it is true when the function is lower semi-
continuous, because it is in this case hyperharmonic according to
Theorem \ref{p4:chap2:sec9:thm4}). For instance let us start from the
classical harmonic 
functions in $R^2$ and consider the function equal to zero except on
the boundary of a disc $\omega$ where it is equal to $1$. This
function is a $S_\mathscr{B}$-function for any base $\mathscr{B}$ of
discs which does not contain $\omega$, but is obviously not a
$S_{\mathscr{B}_1}$ function for $\mathscr{B}_1 = \mathscr{B} \cup \{
\omega\}$, it is not a nearly hyperharmonic function. 

The notion depends on $\mathscr{B}$, has no local character; there is
no local criterion as in Theorem \ref{p4:chap2:sec9:thm4}. 

We\pageoriginale define naturally the $S_\mathscr{B}$-functions in any open set of
$\Omega$ as $S_\mathscr{B}$ in any component considered as a
space. The importance of this notion is seen from the following two
theorems. The first one is immediate. 

\begin{thm}\label{p4:chap3:sec11:thm6}% theorem 6
  The lower envelope of any set of $S_\mathscr{B}$- functions that are
  locally uniformly bounded below is a $S_\mathscr{B}$ - function. 
\end{thm}

The second theorem needs the following remark. If $\omega_1$ and
$\omega_2$ are two regular domains of the given base such that
$\bar{\omega}_1 \subset \omega_2$ then the function $w(x) = \bar{\int}
~ v  ~ d ~ \rho^{\omega_2} ~ (x \in \omega_2)$ is equal in $\omega_1$
to $\int ~ wd ~ \rho^{\omega_1}_x$ which is $\le \bar{\int} ~ v ~ d ~
\rho^{\omega_1}_x$. 

\begin{thm}\label{p4:chap3:sec11:thm7}% theorem 7
  If $v$ is an $S_\mathscr{B}$ - function, the regularised function
  $\hat{v}(x)$ defined at every point $x$ as $\lim.\inf\limits_{y
    \rightarrow x}v(y)$ is hyperharmonic and  
  $$
  \hat{v}(x) = \sup_{\substack{\omega \ni x \\ \omega \in
      \mathscr{B}}} \bar{\int}  ~ v ~ d ~ \rho^{\omega}_x =
  \lim_{\mathscr{F}}. \bar{\int} ~ v ~ d ~ \rho^\omega_x 
  $$
\end{thm}

(Limit according to the filter $\mathscr{F}$ of sections of the
directed decreasing family of $\omega \in \mathscr{B}$ and $\omega$
containing $x$, ordered by inclusion.) 

\begin{proof} % proof
  By the definition of the function $\hat{v}(x)$, it is lower
  semicontinuous and $- \infty < \hat{v}(x) \le v(x)$. Now $\bar{\int}
  ~ v ~ d ~ \rho^\omega_x$ is continuous of $x $ in  $\omega,
  \hat{v}(x) \ge \bar{\int} ~ v ~ d ~ \rho^\omega_x \ge\bar{\int} ~
  \hat{v} ~ d ~ \rho^\omega_x$. Therefore $\hat{v}(x)$ is
  hyperharmonic and also $\hat{v}(x) \ge \sup\limits_{\substack{\omega
      \in\mathscr{B} \\ \omega \ni x}} \bar{\int} ~ v ~ d ~
  \rho^\omega_x$. 
\end{proof}

On\pageoriginale the other hand, given $\varepsilon > 0$, in a neighbourhood
$\delta$ of $x_o$,  
$$
v(x) > \hat{v}(x_o) - \varepsilon
$$

Therefore $\bar{\int} ~ v ~ d ~ \rho^\omega_x \ge (\hat{v}(x_o) -
\varepsilon) \int ~ d ~ \rho^\omega_x$ for $\omega \in \mathscr{B},
\omega \subset \delta, x \in \omega$. But we know that $\int ~ d ~
\rho^\omega_x \underset{\mathscr{F}}{\rightarrow} 1$. Hence the
theorem. 

\section{}\label{p4:chap3:sec12} %section 12

\textbf{Properties.} Some of the properties of the hyperharmonic
functions extend themselves to $S_\mathscr{B}$- functions. 
\begin{enumerate}[a)]
\item If $V_1$ and $V_2$ are $S_\mathscr{B}$- functions then
  $\lambda_1 ~ V_1, \lambda_1 ~ V_1 + \lambda_2 ~ V_2$ (for $\lambda_1
  > 0, \lambda_2 > 0) ~\text{and}~ \Inf. (V_1, V_2)$ are $S_\mathscr{B}$-
  functions. 
\item If $V_n$ is an increasing sequence of $S_\mathscr{B}$-functions
  then $\lim.v_n$ is an $S_\mathscr{B}$- function. 
\item If $v$ is a $S_\mathscr{B}$- function in a domain $\omega$ and
  $+\infty$ in the neighbourhood of a point, $v = + \infty$ everywhere
  in $\omega$, (because $\hat{v} = + \infty$). 

  If a $S_\mathscr{B}$ - functions is not equal to $+ \infty$
  everywhere, we say that it is $\mathscr{B}$ - nearly superharmonic
  or a $S^*_\mathscr{B}$- function. We have immediately the notion of
  nearly superharmonic or $S^*$- functions. 
\item If there exists a harmonic function $h > \varepsilon > 0$ on an
  open set $\omega_o$ the condition on the $S_\mathscr{B}$-function
  $v$ that $\lim.\inf.v \ge 0$ at all the boundary points of
  $\omega_o$ implies $v \ge 0$. [ Follows by considering $\hat{v}$.] 
\item If\pageoriginale $v$ is a $S_\mathscr{B}$ - function in $\Omega$ and $\omega
  \in \mathscr{B}$, the function $E^\omega_v$ equal to $v$ on $C
  \omega$ and to $\bar{\int} ~  v~ d \rho^\omega_x $ on $\omega$ is a
  $S_\mathscr{B}$ - function. 
\end{enumerate}

We have to prove that if $\omega' \in \mathscr{B}, E^\omega_v \ge
\bar{\int} ~ E^\omega_v ~ d ~ \rho^\omega_x$ for every $x$ in
$\omega'$. As $E^\omega_v\leq v $ everywhere and $E^\omega_v = v $ on
$C\omega$, the required inequality is true on $\omega' \cap C
\omega$. Let us introduce on $\partial \omega$ functions $f_n$ with
the following properties. 

$\alpha$) $f_n \ge v$, lower semi-continuous, $\{ f_n\}$ decreasing
and such that $\int ~ f_n ~ d ~ \rho^\omega_x \rightarrow \bar{\int} ~
v ~ d ~ \rho^\omega_x$ for any $x \in \omega$. 

Again a sequence of functions $g_n$ on $\partial \omega'$ satisfying,

$\beta$) $g_n$ a decreasing sequence of lower semi-continuous
functions $\ge E_v$ and $\int ~ g_n ~ d ~ \rho^{\omega'}_x \rightarrow
\bar{\int} ~ E^\omega_v ~ d ~ \rho^{\omega'}_x$ for any $x \in
\omega'$. 

On $\partial \omega' \cap \bar{\omega}$, replace $g_n$ by
$\inf.(g_n,f_n$ or $\int ~ f_n ~ d ~ \rho^\omega_x)$ and thereby get
another sequence $g'_n$ satisfying the conditions $(\beta)$. On
$\partial \omega \cap \omega'$ replace $f_n$ by $\sup.(f_n, \int ~
g_n' ~ d ~ \rho^{\omega'}_x)$ and get on $\partial \omega$ functions
$f'_n$ fulfilling $(\alpha)$. 

Now for any finite continuous functions $\theta \le g'_n$ on $\partial
\omega'$, $\int \theta ~ d ~ \rho^{\omega'}_x - \int ~ f'_n ~ d ~
\rho^\omega_x \le 0$ on $\omega \cap \omega'$ because at any boundary point the
$\lim.\sup.\le 0$. 

Hence 
$$
\displaylines{\hfill 
  \int ~ g'_n ~ d ~ \rho^\omega_x \le \int ~ f'_n ~ d ~\rho^\omega_x
  ~\text{ in }~ \omega \cap \omega' \hfill \cr
  \text{then}\hfill 
  \bar{\int} ~ E_v  ~ d ~ \rho^{\omega'}_x \le \bar{\int} ~ v ~ d ~
  \rho^{\omega}_x = E^\omega_x \text{ in } \omega \cap \omega'.\hfill } 
$$

This completes the proof. \footnote{For another proof see remark $n^o
  ~ 15$ and Seminar on potential theory II} 

\begin{defn}\label{p4:chap3:sec12:def8}% definition 8
  We call any set  - negligible (\resp negligible) in (or in any open
  set $\omega_o$) if its intersection with the boundary of any $\omega
  \in \mathscr{B}$\pageoriginale (\resp any regular domain) ($\bar{\omega} \subset
  \omega_o$ in the general case) has a $d ~ \rho^\omega_x$ - measure
  zero. (Then the complimentary set is dense in $\omega_o$). 
\end{defn}

We may use $\mathscr{B}$- nearly everywhere (nearly everywhere) in the
sense ``except on a $\mathscr{B}$-negligible (negligible) set''. 

\begin{remark*}~
  \begin{enumerate}
  \item If two hyperharmonic functions are equal $\mathscr{B}$-nearly
    everywhere they are equal everywhere. Any superharmonic function
    is finite nearly everywhere. 
  \item If a hyperharmonic (resp. superharmonic) function is
    majorised on a $\mathscr{B}$-negligible set we get $\mathscr{B}$-
    nearly (resp. superharmonic) hyperharmonic function. It would be
    interesting to compare any $\mathscr{B}$-nearly hyperharmonic
    function $v$ with $\hat{v}$ which is the greatest hyperharmonic
    minorant. 
  \end{enumerate}
\end{remark*}

\section{Applications}\label{p4:chap3:sec13}%sec 13 

{\em Introduction of the reduced function.}

\begin{defn}\label{p4:chap3:sec13:def9} % definition 9
  Let $E$ be a subset of $\Omega$ and $\varphi$ any function $\ge 0$
  on $E$. We denote by $R^E_\varphi$ the lower envelope of all
  hyperharmonic functions $w \ge 0$ on $\Omega$ which majorise
  $\varphi$ on $E$. It is a nearly hyperharmonic function. 
\end{defn}

In the case when $\varphi$ is the trace of a superharmonic function
$v$ on $\Omega$, we call $R^E_v$ the reduced function of $v$ relative
to $E$ and the regularised function $\hat{R}^E_v$ is called the
balayaged function of $v$ relative to $E$ (or extremised function
relative to $CE$). 

\medskip
\noindent
\textbf{Immediate Properties:}\pageoriginale
\begin{enumerate}[(i)]
\item $R^E_{\lambda\varphi} = \lambda R^E_{\varphi} ~(\lambda > 0);
  R^E_\varphi$ increases with $\varphi$ and $E$. 
  $$
  R^E_{\varphi_1+\varphi_2} + \varphi_2 \le ~ R^E_{\varphi_1} + R^E_{\varphi_2};
  R^E_{\varphi} ~ \le ~ R^{E_1}_\varphi + R^{E_2}_\varphi ~ (E_1 \cup
  E_2 = E) 
  $$ 
  The same properties hold good for $\hat{R}^E_\varphi$.
  
\item $R^E_\varphi \ge \varphi$ on $E$. If $E_1 \supset E,
  R^{E_1}_\psi = R^E_\varphi$ where $\psi = R^E_\varphi$. 
\item The most interesting case comes up when $\varphi$ is the trace
  of a superharmonic function $v \ge 0$ in $\Omega$. 
\end{enumerate}

$0 \le \hat{R}^E_\varphi \le R^E_v \le v$ everywhere. $R^E_v = v$ on
$E$ and $\hat{R}^E_v = v$ on $E$. 

In particular if $\omega$ is a regular open set $R^{C\omega}_v = \int
~ v ~ d ~ \rho^\omega_x$ on $\omega$ and $R^{C\omega}_v = \hat{R}^{C
  \omega}_v$ everywhere. For we know that $E^\omega_v \ge R^{C
  \omega}_v$. Conversely let $\omega$ be superharmonic $\ge 0$ in
$\Omega$ and majorise $v$ on $C \omega$. If $\theta$ is a continuous
function on $\partial \omega, \theta \le v$, the superharmonic
function $w - \int ~ \theta ~ d ~ \rho^\omega_x $ in $\omega$ is $\ge
0$ because of its behaviour on the boundary. Hence $R^{C \omega}_v \ge
\int ~ v ~ d ~ \rho^\omega_x$ in $\omega$. 

\setcounter{remark}{0}
\begin{remark}% remark 1
  We may as well replace the hyperharmonic functions (motorising
  $\varphi$ on $E$) by $S_\mathscr{B}$ - functions and study the lower
  envelope. But this new function is of little use when it is not
  identical with the first envelope. 
\end{remark}

\begin{defn}\label{p4:chap3:sec13:def10}% definition 10
  We\pageoriginale define immediately $(R^E_{\varphi})_{\omega_o}$ for a domain
  $\omega_o \subset \Omega$ and $E \subset \omega_o$ by considering
  $\omega_o$ in the place of $\Omega$. For an open set $\omega_o$, we
  define $(R^E_\varphi)_{\omega_o}$ equal in every component
  $\omega_i$ to $(R^{E \cap \omega_i}_\varphi)_{\omega_i}$. 
\end{defn}
