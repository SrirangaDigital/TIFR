\part{Introduction And Topological Preliminaries}\label{p1}

\chapter*{Introduction}\label{p1:chap1}
\addcontentsline{toc}{chapter}{Introduction}

\section{}\label{p1:chap1:sec1}

The\pageoriginale potential theory has been studied very much,
especially after the 
researches of Gauss in 1840, where he studied important problems and
methods which gave yet remained partly as basic ideas of modern
researchs in this field. For about thirty years many refinements of
the classical theory were given; later the axiomatic treatments
starting from different particular aspects of the classical
theory. About half a dozen of such axiomatic approaches to potential
theory, parts of which are not yet published with details, exist. It
would be necessary to compare these different approaches and to study
the equivalence or  otherwise of them. 

\textit{In the following we shall develop some results of such
  axiomatic theories principally some convergence theorems}; they may
be used as fundamental tools and applied to classical case as we shall
indicate sometimes. We do not presuppose anything of even classical
theory. 

A survey of the different developments of the potential theory has
been given by $M$. Brelot (Annales de 1'Institut Fourier t4, 1952-54)
with a historical view point and with a rather large bibliography. We
shall complete it with indication one some recent developments of the
theory. 

We\pageoriginale shall begin with some topological preliminaries that
are necessary for our development  

%\chapter{Topological Lemmas}\label{p1:chap2}

\section{}\label{p1:chap2:sec1}

\begin{Lemma}\label{p1:chap2:sec1:lem1} %lem 1
  Let $E$ be a connected topological space. $f$ be a lower
  semi-continuous function on $E$ such that $f$ cannot have a minimum
  (respectively a minimum $<0$) at any point of $E$ without being a
  constant in some neighbourhood of that point. Then if $f$ attains its
  lower bound (\resp, supposed to be $< 0)$, then $f$= constant. 
\end{Lemma}

\begin{proof}
  Consider the set of the points of $E$ where $f$ attains its lower
  bound and the set of points at which the function is different from
  its lower bound. These two are disjoint open sets of $E$ where union
  is $E$. The former
  set is non-empty by hypothesis. $E$ being connected this is the
  whole of $E$. Hence $f$ is a constant. 
\end{proof}

\begin{Lemma}\label{p1:chap2:sec1:lem2} %lem 2
  Let $E$ be a connected compact topological space, and $\omega$ an
  open set different from $E$. Let $f$ be a real valued lower
  semi-continuous function defined on $\omega$ and further satisfy the
  condition that it cannot have a minimum $<0$ at any point without
  being a constant in some neighbourhood of the point. If at every
  point of the boundary $\partial \omega$ of $\omega, \lim. \inf. f
  \geq 0$, then $f \geq 0$. 
\end{Lemma}

\begin{proof} % pro
  Define a new function $F$ which is equal to $f$ on $\omega$ and zero
  in the complement of $\omega$. The function $F$ is lower
  semi-continuous, and cannot have a minimum $< 0$ in $E$ without
  being constant in some neighbourhood of that point. Assume that $F <
  0$ at some point of $E$. Then the lower bound of $F$ in $E$ is less
  than zero and is attained in\pageoriginale $E$. Hence by Lemma
  \ref{p1:chap2:sec1:lem1}, $F$ is a constant; this is a contradiction. 
\end{proof}

Therefore $F \geq 0$. It may be observed that in the course of the
proof only the sequential compactness of $E$ has been made use of. 

\section{}\label{p1:chap2:sec3}

\setcounter{lemma}{2}
\begin{Lemma}[Choquet]\label{p1:chap2:sec3:lem3}% lem 3
  Let $E$ be a topological space satisfying  the
  second axiom of countability (i.e. possessing a countable base of open
  sets). 
\end{Lemma}

Let $(f_i)_{i \in I}$ be a family of real valued (finite or not )
functions on $E$. 

Denote 
$$
f_I(x) = \inf_{i \in I}. f_i(x)
$$
It is possible to extract a countable subset $I_o$ of $I$ such that if
$g$ is any lower semi-continuous function on $E$ with $g \leq
f_{I_o}$, then $g \leq f_I$. 

\begin{proof} %pro
  It is sufficient to consider functions with values in $[-1,
    1]$. (The general case may be deduced from this one with the aid
  of a transformation of the form $\dfrac{x}{1+|x|}$ on the real
  line). 
\end{proof}

Let $\omega_1, \omega_2, \ldots \ldots$ be a sequence of open sets,
forming a base for open sets of $E$, with the condition the each
$\omega$ appears in the  sequence infinitely many times. By this
arrangement, it is possible to choose same  $\omega$ with arbitrarily
large index. For every $n$ choose $i_n$ satisfying the following
inequality 
\begin{equation*}
  \Inf. f_{i_n}(y) - \Inf. f_{I} (y) <
  \frac{1}{n} \label{p1:chap2:sec3:eq1} \tag{1}
\end{equation*}
$$
y \in \omega_n \qquad y \in \omega_n
$$
We\pageoriginale have a sequence $ I_o = (i_n)$ a subset of $I$. We shall prove that
this choice of $I_o$ fulfills the required conditions. 

Suppose $g$ is a lower semi-continuous function on $E$ with $g \leq
f_{i_n}$ for every $i_n$, we want to show that $g \leq f_I$. For any
$x \in E$ and $\varepsilon >0$, there exists an open neighbourhood $N$
of $x$ such that for every $y$ belonging to $N, g(y)> g(x) -
\varepsilon/2$. There exists a $\omega_p$ containing $x$, at every
point of which the above inequality holds good, this choice being made
in such a way that $1/p < \varepsilon/ 2$. Hence, 
$$
\displaylines{\hfill 
  \inf_{y \in \omega_p}.g(y) \geq g(x) - \varepsilon/2\hfill \cr
  \text{i.e.,}\hfill g(x)- \inf_{y \in \omega_p}.g(y) \leq
  \varepsilon/2  \hfill (2)} 
$$
Further we have the inequality 
\begin{equation*}
  \inf_{y \in \omega_p}.g(y) - \inf_{y \in \omega_p}.f_{i_p}(y) \leq 0  \tag{3}
\end{equation*}
With this choice of $p$ the inequality $(1)$ takes the form 
\begin{equation*}
  \inf_{y \in \omega_p}.f_i{_p}(y) - \inf_{y \in \omega_p}.f_I(y) <
  \frac{1}{p} < \varepsilon / 2  \tag{1$'$}  
\end{equation*}
We get on adding $(1')$, (2) and (3)
$$
g(x)-  \inf_{y \in \omega_p}.f_I(y)\leq \varepsilon 
$$
$\varepsilon$\pageoriginale being arbitrary, it follows that $g(x) \leq f_I(x)$
i.e. the lemma. 

\begin{remark*}
  We shall denote the $\lim. \inf. \varphi$ for a function $\varphi$
  at every point by $\hat{\varphi}$ (called its lower semi-continuous
  regularisation). Making use of this notation Lemma
  \ref{p1:chap2:sec3:lem3} can be put in the form: 
\end{remark*}

If $f_I$ is the $\inf \limits_{i \in I}. f_i$ at each point of a
topological space (with countable base for open sets) of a family of
real valued functions (finite or not) $\{ f_i \}_{i \in I}$, then it
is possible to choose a countable subset $I_o \subset I$ such that
$\hat{f}_I{_o}= \hat{f}_I$.  
