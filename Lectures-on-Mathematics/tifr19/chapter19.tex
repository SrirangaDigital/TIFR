\chapter{Generalisation of the Riesz -Martin Integral
  Representation}\label{p4:chap10}%chap X 

\setcounter{section}{40}
\section{}\label{p4:chap10:sec41}%sec 41

We\pageoriginale shall give only a brief survey without detailed proofs.

For the developments see \cite{1} Seminaire $II$.

\noindent
\textbf{Choquet's theorem on extremal elements}

On a vector space $\mathscr{L}$, a point $x$ of a convex subset $A$ is
said to be an extremal element of $A$, if there exists no open segment
containing $x$ on $A$, defined by $\lambda x_1 +  \mu x_2, x_1, x_2
\in A$, $0 < \lambda < 1$ and $\lambda + \mu = 1 $ for all $\lambda$
and $\mu$. Let ${}^*A$ denote the extremal elements of $A$. 

If $\mathscr{L}$ is locally convex Hausdorff and $A$ is a convex
compact subset, the centre of gravity of a positive Radon measure
$\mu$ on $A$ may be defined as follows: when $\mu$ is defined by a
finite system $S$ of point masses $\ge 0 ~ (m_i $ at $z_i$; with $\sum
m_i = 1)$, the centre of gravity $G_S$ is the point $\sum m_i
z_i$. For any $\mu$ with $\mu (A) = 1$, there exists an ultra-filter
on the sets of the previous $S$, such that it converges vaguely to
$\mu$. Then $G_S$ converges to a point $G$ in $A$; this point $G$ is
independent of the filter and is called the centre of gravity of
$\mu$. This is a consequence of the fact that for any continuous
linear real functional $f(z), G$ satisfies $f(G) = \int f(z) d \mu (z)$.

\begin{thm}[Choquet]\label{p4:chap10:sec41:thm34} % the 34
  On a real, ordered, locally convex vector space $\mathscr{L}$ consider
  the convex cone $\mathscr{C}$ of positive elements. We suppose
  $\mathscr{C}$ has a compact base\pageoriginale $B$ (intersection of all rays of
  $\mathscr{C}$ with an affine manifold which does not pass through the
  origin). 
  \begin{enumerate} [\rm(i)]
  \item Suppose $\mathscr{C}$ is a lattice; then given any point $x \in
    B$, there exists at most one unitary measure $\mu \ge 0$ on $B, (\mu
    (B) = 1)$ supported by $*B$ and whose centre of gravity is $x$. 
  \item Suppose $B$ is metrizable; then there exists at least one such
    $\mu$. \textit{Integral representation of the harmonic functions}
    $\ge 0$ (generalised Martin representation). 
  \end{enumerate}
\end{thm}

\begin{lemma*} % lem
  For the set $H^+_{x_\circ}$ of the harmonic functions $u \ge 0$ on
  $\Omega$. satisfying $u(x_\circ) = 1$ (for fixed $x_\circ \in \Omega$), the
  property of compactness for the topology of the uniform convergence
  on every compact subset is equivalent to the following ``Harnack
  axiom''. 
\end{lemma*}

For every $x_1 \in \Omega$, $\dfrac{u(x)}{u(x_1)} \to 1 (x \to x_1)$
uniformly for all functions of the set $H^{'+} $ of the harmonic
functions $> 0$ on $\Omega$. (equivalent form : the functions of
$H^+_{x_1} $ are equicontinuous at $x_1$ for every $x_1 \in \Omega$). 

\begin{thm}\label{p4:chap10:sec41:thm35} % the 35
  Suppose $H^{' +}$ (if non-empty) satisfies the Harnack axiom. Let $H$
  be the vector space of difference of harmonic functions $\ge 0$ with
  the topology of uniform convergence on compact set of $\Omega$. 
\end{thm}

\begin{enumerate} [(i)]
\item Given any function $u$ of $H^+$ (the set of harmonic functions
  $\ge 0$ on $\Omega$) on the compact set $H^+_{x_\circ}$, there exists at
  most one Radon measure $\mu \ge 0$, supported by $^*H^+_{x_\circ} $ (set
  of extremal elements of $H^+_{x_\circ} \subset ~ H)$ such that  
  
  $u(x) = \int w(x) d \mu (w) ~ (x \in \Omega$; $w$\pageoriginale variable on the
  base $H^+_{x_\circ}$ of the cone $H^+$). 
\item If $\Omega$ is a countable union of compact sets, there exists
  one such measure. Moreover, for any $\mu \ge 0$ on $H^+_{x_\circ}$,
  $\int w(x) d \mu (w) \in H^+$. 
\end{enumerate}

Obviously $H$ is locally convex, $H^+$ is convex and further $H^+$ is
known to be a lattice for the natural order; with the hypothesis
$H^+_{x_\circ}$ is metrizable. We may therefore use the centre of gravity
and the linear functional $u(x)$ of $u$ (for fixed $x$). 

If we want such a representation for every domain of $\Omega$, we have
to suppose Harnack axiom for every domain. We emphasize that if axioms
\ref{p4:chap1:sec1:axiom1} and \ref{p4:chap1:sec1:axiom2}
are assumed, the axiom \ref{p4:chap1:sec1:axiom3} together with the
Harnack axiom for all domains of $\Omega$ is equivalent to the  

\medskip
\noindent \textbf{Axiom $3'$} 
  \footnote{Actually the Harnack  axiom is a consequence of axiom
    \ref{p4:chap1:sec1:axiom1}, \ref{p4:chap1:sec1:axiom2},
    \ref{p4:chap1:sec1:axiom3}
    and $1, 2,3 \quad 1, 2, 3$ (Mokobodski - Loeb-B/ Walsh; see
    add. Chapter).} There exists a base of domains of $\Omega$, such
  that for each one of them the set of harmonic functions $> 0$ is
  non-empty and further each satisfies the Harnack axiom (see another
  form in Sem $II$). 

\section[Integral representation of...]{Integral representation of the superharmonic\break functions $\ge
  0$}\label{p4:chap10:sec42}%sec 42 

Let us use the locally convex vector space $S (n^o. 19)$ of the
equivalence classes corresponding to the differences of superharmonic
functions $\ge 0$, the specific order and the positive cone $S^+$
(whose elements may be identified with superharmonic functions $\ge
0$). We know that $S^+$ is a lattice. 

We introduce the set $S^+_{x_\circ, \omega_\circ}$ of the superharmonic
functions of $S^+$ satisfying the condition $\int v d
\rho^{\omega_\circ}_{x_\circ} = 1$ for a regular domain $\omega_\circ$ and $x_\circ
\in \omega_\circ$. It is a base of the convex cone $S^+$. 

In\pageoriginale order to apply Choquet's Theorem, we need a topology in $S$ for
which $S^+_{x_\circ, \omega_\circ}$ becomes compact and metrizable. 

\begin{prop}\label{p4:chap10:sec42:prop27} % pro 27
  For the elements [$u, v$] of $S$, $\big | \int u d \rho^\omega_x -
  \int v d \rho^\omega_x \big |$ for a fixed regular domain $\omega$
  and a fixed point $x \in \omega$ is a seminorm. All these semi-norms
  define on $S$ a topology $\mathscr{C}$ which is locally convex,
  Hausdorff and compatible with the specific order. 
\end{prop}
 
We remark that when $(3')$ is satisfied, $\mathscr{C}$ on $H^+$ is
identical with the topology of uniform convergence on compact sets.  
 
\begin{defi*} % def
  A regular open set is said to be completely de terminating if for two
  superharmonic functions, $v_1 $ and $v_2 \ge 0$ on $\Omega$,
  harmonic on $\omega$, the condition $v_1 =  v_2$ on $C \omega$
  implies $v_1 = v_2$ on $\omega$. 
\end{defi*} 
 
We shall introduce a new axiom.
 
\begin{Axiom}\label{p4:chap10:sec42:axiom4} % axi 4
  There exists on $\Omega$ a base of completely determinating
  (regular) domains. In case $\Omega$ has a countable base, axiom $4$
  implies the existence of a countable base of completely
  determinating regular domains. 
\end{Axiom} 
 
It does not seem impossible that axiom $4$ which is satisfied (even
for all regular domains) and used in some proofs of classical theory,
is a consequence of axioms \ref{p4:chap1:sec1:axiom1},
\ref{p4:chap1:sec1:axiom2} and $3'$. 

\begin{prop}\label{p4:chap10:sec42:prop28} % pro 28
  Suppose axioms \ref{p4:chap1:sec1:axiom1},
  \ref{p4:chap1:sec1:axiom2} and $3'$ and
  $4$ and a countable base for open sets 
  on $\Omega$. For the topology $\mathscr{C}$ (introduced above ) the
  base $S^+_{x_\circ, \omega_\circ}$ of the cone $S^+$ is metrizable and
  compact. 
\end{prop}

Now with the set ${}^*S^+_{x_\circ, \omega_\circ}$ of the extremal points of
$S^+_{x_\circ, \omega_\circ}$, we may use Choquet's theorem, the linear
functional $\int v d \rho^\omega_x $ of $v$ and\pageoriginale the property $v \int d
\rho^\omega _x \xrightarrow[{\mathcal{F}}] {}v(x)$ (fixed $v \in S^+$;
fixed $x; \omega$ variable according to filter $\mathscr{F}$ of
Theorem \ref{p4:chap3:sec11:thm7}.) 

\begin{thm}\label{p4:chap10:sec42:thm36} %Theorem 36
  With the same hypothesis on $\Omega$ (as in
  Prop. \ref{p4:chap10:sec42:prop28}) and with the
  topology $\mathscr{C}$, there exists for every $v \in S^+$ an unique
  Radon measure $\mu \geq 0$ on $S^+_{x_0,\omega_0}$, supported by
  ${}^*S^+_{x_0, \omega_0}$ and such that $v(x)=\int w(x)d \mu (w)\, (w \in
  S^+_{x_0,\omega_0})$. 
\end{thm}

Note also that for any $\mu$ on $S^+_{x_0,\omega_0}, \int w(x)d\mu
(w)\in S^+$. 

\noindent \textbf{\textit{ Decomposition of this representation.}}

Let $P^+$ denote the set of potentials on $\Omega. P^+_{x_0,\omega_0}$
the subset in $S^+_{x_0,\omega_0}$ and $^*P^+_{x_0,\omega_0}$ the set
of external elements of $P^+_{x_0,\omega_0}$. Then $H^+_{x_0} \cap
P^+_{x_0,\omega_0}=\phi$. $S^+_{x_0,\omega_0}=^*H^+_{x_0,\omega_0}\cup
^*P^+_{x_0,\omega_0}=^*S^+_{x_0,\omega_0}\cap
P^+_{x_0,\omega_0}$. When $P^+$ is not empty,
$S^+=\bar{P}+,S^+_{x_0,\omega_0}=\bar{P}^+_{x_0,\omega_0},^*S^+_{x_0,\omega_0}
\subset ^*\bar{P}^+_{x_0,\omega_0}$. The sets considered with starts
are $G_\delta$-sets in $S^+_{x_0,\omega_0}$. 

\begin{thm}\label{p4:chap10:sec42:thm37} %thm {37}
  With hypothesis as in Theorem \ref{p4:chap10:sec42:thm36}, for
  any $v \in S^+$, 
  $$
  v(x)=\int w(x)d \mu_1(w)+ \int w(x)d \mu_2 (w)
  $$
  where $\mu_1$ and $\mu_2$ are unique Radon measures $\geq 0$ on
  $S^+_{x_0,\omega_0}$, with\pageoriginale supports $^*H^+_{x_0,\omega_0}$ and
  $^*P^+_{x_0,\omega_0}$ respectively. 
\end{thm}

The first integral is the greatest harmonic minorant of $v$ and the
second one is a potential. 

\medskip
\noindent
{\bf Classical case:} In the case of Green space, for instance a
bounded the domain of the Euclidean space, the extreme potentials
appears easily as the normalised Green function; $g_y(x)=
\dfrac{G_y(x)}{\int G_y(x)d \overset{\omega_0}\rho(x)}$ (Green
function $G_y(x)$ with pole $y$). The correspondence $y \to g_y(x)$
defines a homomorphism between $\Omega$ and $^*P^+_{x_0,\omega_0}$
(this set having topology $\mathscr{C}$) 

The measure $\mu_2$ of Theorem \ref{p4:chap10:sec42:thm37} may be
therefore considered as a 
measure on $\Omega$ and $\int w(x)d \mu_2 (w)$ is equal to
$\int G_y(x)d \nu (y)$ with another measure $\nu \geq 0$ on
$\Omega$. So we get the classical Riesz representation of
potentials. 

On the other hand $^*H^+_{x_0}$ is a subset of the compact $H^+_{x_0}
\cap \bar{^*P^+}_{x_0,\omega_0}$ which is subset of
$S^+_{x_0,\omega_0}$ and is a definition of the classical Martin
boundary  \footnote{ R.S. Martin
  Trans. Am. Martin. Soc. Vol. $42(194)$.} $\Delta$ (upto some
homeomorphism). It is measure on the compact set $\Delta$, supported
by the set of the minimal points and $\int w(x)d \mu_1 (w)$ gives the
Martin representation of the greatest harmonic minorant of $v$. 

\section{Further Theory-Kernel}\label{p4:chap10:sec43} %sec 43

R.M. Herve has succeeded in avoiding axiom
\ref{p4:chap10:sec42:axiom4} in the integral 
representation\pageoriginale with the help of a different topology on $S$. On the
other hand, she remarked that (with $1,2,3'$ countable base, pot
$>0$), the extreme potentials have a point support and the importance
of the ``case of proportionality'' where all potentials with same point
support are proportional. Such a proportionality allows the use of a
Green-type function. This is a kernel and the study becomes a
particular case of the theory initiated in Part \ref{p3}. 

\begin{thebibliography}{99}
\bibitem{p4key1}{M.Brelot}\pageoriginale ``Elements de la theore classique du Potential
  ``,(Paris, $CDU, 1959$) C.R. Acad. Sci P t.$245(157)$, p. $1688$
  t. $246(158),p.2334,2709$ Seminaire de theorie du potential (1'
  Institute Henri Poincare, Paris) $I 1^{ere}$ annee $1957$ $II 2^e$
  annee $1958$ $III 3^e$ annee $1958-59$ 
\bibitem {p4key2} {M.Brelot and R.M Herve}  C.R.Acad.Sci,t.$247 (1958)p.1956$
\bibitem {p4key3}{J.L.Dobo}  Proc. Third Symposium on probability
  (Brekeley) Vol $2$, published in $1956$ Illinois Journal $1958$ 
\bibitem {p4key4} {R.M. Herve}  C.R. Acad. Sci., $t.248(1959),p.179$
  Seminaire du Potential $III$
\bibitem {p4key5} {Tautz} Math. Nach. $t.2(1949)$
\end{thebibliography}
and see additional chapter dedicated to the further developments.

\chapter[Additional Chapter and Bibliography for Part IV...]{Additional Chapter and Bibliography for Part IV
  (2$^{\text{nd}}$ edition)} 

After\pageoriginale the publication of these lectures, very many researches were
made on the field of Part $IV$ and are mostly developed or mentioned in
$2$ courses (Brelot, Montreal \cite{21}, Bauer, Hamburg \cite{4}) and in a
summary of Constantinescu \cite{27}. Let us give a short survey with
mention of the latest results. 

Mokobodski proved (see \cite{21}) that $1,2,3 \Rightarrow 3'$ with a
restriction of countable base open sets, that was cancelled by
Loeb-Walsh \cite{53}. \footnote {See improved hypothesis in
  Constantinescu \cite{31}} He proved also, even in large conditions that
the points where proportionality does not hold form a polar
set. [still unpublished]. Mrs. Herve \cite{44} brought to $IV$ many
complements: first a topology on the space $S^+-S^+$, which allows
only with axioms \ref{p4:chap1:sec1:axiom1},
\ref{p4:chap1:sec1:axiom2}, \ref{p4:chap1:sec1:axiom3}, countable
base, $pot>0$, to find a compact 
metrizable base of $S^+$ and whose introduction was simplified by
Mokobodski \cite{60}; she discussed a balayage theory  and equivalences of
$D$, and by adding proportionality and a new axiom developed a theory
of adjoint harmonic functions which axiomatizes the adjoint elliptic
equations; she studied the linear elliptic equation of second order,
even later \cite{47} with discontinuous coefficients, showing that the
solution in a suitable sense considered by specialists like
Stampacchiea satisfies the axioms of Part
\ref{p4}. Boboc-Constantinescu-Cornea $[7\to 12,25 \to 34]$ brought also
many improvements or complements, even in more general conditions we
shall\pageoriginale mention later: axiom \ref{p4:chap1:sec1:axiom3} expressed
equivalently with sequences 
\cite{32}, the existence of a superharmonic function $>0$ implies (with
$1,2,3$) that $\Omega$ is a countable union of compact sets \cite{34};
weaker hypothesis for the properties of $R^e_v$; example where
countable base, proportionality, axiom $D$ are not satisfied. Similar
complements or improvements (for instance, cancellation of a countable
base or of $D$) were often given in paper like \cite{24}, \cite{49} or in
further developments or larger axiomatice we shall review now. 

Gowrisankaran \cite{38}, \cite{41} introduced and used an extension of classical
Naim's ``thinness at boundary'', first in a general abstract
setting. The harmonic function which are extreme points on some base
of $S^+$ are called minimal and form on some fixed base $\beta$ of
$S^+$ a set $\Delta_1$ (minimal boundary). A set $e$ is said to be
thin at $X \in \Delta_1$, if $R^e_X \neq X$ and the complementary sets
of thin sets at $X$ form a filter $\mathcal{F}_X$ according to which
limits are called ``fine''. Only with $1,2,3$, countable base and pot
$>0$, Gowrisankaran proved that $\dfrac{u}{v}$ (quotient of two
superharmonic function $>0$) have a finite fine limit at all points of
$\Delta_1$, except on a set of $\mu_v$ measure zero (measure
corresponding to $v$ in the integral representation) (Extension of a
Doob's result in classical case). Note that, by adding the
``proportionality'' Brolet [still unpublished] proved that there is a
unique topology on $\Omega \cup \Delta_1$, giving the fine topology on
$\Omega$, and for every $X \in \Delta_1$, neighbourhoods interesting
$\Omega$ according to $\mathcal{F}_X$; this topology induces on
$\Delta_1$ the discrete topology and may be called fine topology on
$\Omega \cup \Delta_1$ denoted\pageoriginale $\overset{v}{\Omega}$. Still with
``proportionality''`, Gowrisankaran considered on the base $\beta$ of
$S^+$, the set. of minimal potentials; it is homeomorphic to $\Omega$
and has a compact closure which gives a ``Martin space''
$\hat{\Omega}$ where $\Omega$ is dense and a ``Martin boundary''
$\Delta=\hat{\Omega}-\Omega$ where $\Delta_1$  is the ``minimal
part''. Dirichlet problems may be studied with adding $D$ for
h-harmonic function with $\Delta$ and $\hat{\Omega}$ or $\Delta_1$ and
$\overset{v}\Omega$ (same resolutivity and same solutions) or
without $D$ and proportionality, with $\Delta_1, \mathcal{F}_X$ and
boundary condition only $\mu_h$ a.e. By studying $\dfrac{u}{v}$, the
condition $u>0$ may be weakened \cite{39} as Doob did in in classical
case. Gowrisankaran extended also the Naim's comparison of all compact
boundaries which may allow a Dirichlet problem (unpublished). In
\cite{40}, he studied the doubly harmonic and superharmonic functions
$f(x,y)$ ($x$ and $y$ resp. In harmonic spaces). The uniform
integrability, introduced by Doob in potential theory, was used by
Brelot \cite{16}, Gowrisankaran, Mrs Lamer Na im \cite{56} in order to
characterize harmonic functions as solution of various Dirichlet
problems. Still with axioms \ref{p4:chap1:sec1:axiom1},
\ref{p4:chap1:sec1:axiom2}, \ref{p4:chap1:sec1:axiom3}, countable
base, potential $>0$,   
and constants harmonic, Mrs Lumer-Naim studied the complex harmonic
functions $f=u+iv$, such that for every $f$, the $L^p_j$-norms are
uniformly with respect bounded with respect to $j(1\leq p \leq
+\infty,$ fixed $p$) These norms are relative to the harmonic measure
$d \rho_{x_0}^{w_j}$ (relatively compact open sets $w_j \ni
x_0$). These functions have finite fine limits $\mu_1$ a.e on
$\Delta_1$. Similarly study for subharmonic functions. Extension by
changing the norm $|f|^p$ in $\Phi(f), \Phi$ positive convex
increasing. Various applications extending classical results. 

Brelot\pageoriginale \cite{17}, \cite{13}, then Fuglede \cite{36}, \cite{37}
discussed axiomatically the 
first idea of thinness.\footnote{see a survey of the notions of
  thinness in Brelot \cite{22} and the courses (Bombay $TIFR 1966$ ) on
  topologies and boundaries in potential theory}. Brelot extended
\cite{15} the functional Keldych's characterization of the ordinary
Dirichlet problem and \cite{16} various results on fine topology; he
compared \cite{15}, \cite{16} both the inner and minimal thinnesses and studied
\cite{20} $\hat{R}^e_w$ ($w$ superharmonic $>0$) for decreasing sets $e$
(that includes the capacities for decreasing set). thanks to the fine
topology; that led to the study of $R^\Omega_ \varphi$ for decreasing
fine upper semi-continuous functions. Doob, inspired by the fine
topology, developed \cite{35} \footnote{In \cite{20} \cite{35} see proof in
  various condition of a thinness of Getoor \cite{42} originally given in
  probability theory and that choquet (ann $IF$) discussed
  axiomatically. It says that for a measure which does not charge
  polar sets, there exists a fine closed support.}, a general theory
of ``small'' sets such that in large conditions, a union of open sets
in equal to a countable subunion upto a ``small'' set. Important
applications were given to potential theory and corresponding
probabilistic questions. 

Loeb \cite{52} compared two sheaves of harmonic functions on the same space
and studied the case where $1$ is superharmonic. A continuation was
developed by Loeb-Walsh \cite{55} with the use of compactify
boundaries. Constantinescu-Corna had studied compactifications of
$\Omega$ and corresponding Dirichlet problems. They also studied
\cite{38} the correspondence of two harmonic spaces $\Omega, \Omega'$
(continuous application $\varphi :\Omega \to \Omega'$) such that if
$u'$ is harmonic in $\omega' \subset \Omega', u' \circ \varphi$ is harmonic
in $\varphi^{-1}(\omega')$. Sibony \cite{63} developed independently\pageoriginale the
same idea in order to extend the results of Constantinescu-Cornea-Doob
on the analytic corresponding between hyperbola Riemann surfaces. 

A de la Pradella \cite{51} extended the classical theory of a Dirichlet
problem for compact sets and studied the property of quasi analyticity
(i.e. a harmonic function is zero in a domain when it is zero in a
neighbourhood of a point). Using the theory of adjoint harmonic
functions, he proved that the quasi-analyticity of these functions is
equivalent to an approximation property relative to the first sheaf
(approximation on the boundary of any relatively compact domain
$\omega$ of any real finite continuous function by a linear combination
of potentials with point support on a fixed neighbourhood of point
of $\omega$. 

Let us emphasize two important feature of this axiomatic theory, first
the bridge built by P.Meyer \cite{57} with semi-groups and Markov
processes, which allows to integrate nearly the theory (with $1,2,3,$
countable base, pot $>0$) in the Hunt's frame (see also the recent
books of Meyer \cite{58}, \cite{59}) for the general correspondence between
potential theory and Markov processes). Secondly the role of nuclear
spaces that Loeb-$B$ Walsh set in evidence \cite{54} and which implies
$3'$ from $1,2,3,$ countable base (the harmonic function on any open
set form a space which is provided with the topology of uniform
convergence on compact sets and then becomes a Frechet nuclear
space). Both these featuring may be extended to more general
axiomatics we  will consider now. 

In\pageoriginale order to include the heat equation and other parabolic equations in
the application of an axiomatic theory, as already did, Beuer
$[1,2,4]$ changed the axioms as follows: Same Hausdorff space
connectedness is not required (trivial extension); if local
connectedness and no compactness were not supposed, they would be
con-connectedness and no compactness were not supposed, they would be
con-sequence of the following axioms. Same axiom
\ref{p4:chap1:sec1:axiom1}. Same definition 
of regular open set (but with non-empty boundary) same axiom
\ref{p4:chap1:sec1:axiom2}  (base
of regular open sets); axiom \ref{p4:chap1:sec1:axiom3} becomes
weaker; for an increasing 
directed set of harmonic function on an open set $\omega$, the limit
is harmonic when $\mathcal{J}$ is superbounded (form $K_1$) when the
limit is finite everywhere (form $K_2$) or when it is finite on a
dense set (form $K_D$ used by Doob in a metrizable space). As that
does not allow a necessary minimum principle, a non-local separation
axiom is added. Hyperharmonic functions may be introduced and for a
harmonic function $h>0$, also $h$-harmonic and hyperharmonic
functions. Now ``axiom $T$'' means that a harmonic function $h>0$
exists on $\Omega$ and that the $h$-hyperharmonic functions separate
$\Omega$. A stronger form $T$ \footnote{For a summary of the last
  changes and improvements of the Bauer's theory, see \cite{6}.} means the
same for positive $h$-hyperharmonic functions. A similar form that
Bauer used in his final version \cite{4} is the existence of an $h>0$
and for any $x,y,x\neq y$ of two hyperharmonic functions $u_1,u_2 >0$
such that $u_1(x)u_2(y)\neq u_1 (y)u_2(x)\, (\circ. \infty =0)$. Bauer used
often and definitively (in \cite{4}) the existence of a countable base of
open sets. In this case and\pageoriginale with $K_D$ and $T^*$, there is an
equivalent form of this axiomatic setting which is more easily
comparable to Part \ref{p4} (see Brelot \cite{16}). For some questions, the
existence of potential which is $>0$ at any given point is also
supposed (and gives the so called strong harmonic spaces \cite{4}). Note
that even without a countable base a countable base, all these axioms
of Bauer are satisfied in the Brelot's axiomatic (with $1,2,3$ and
existence of $2$ non proportional harmonic functions in $\Omega$). Now
applications are possible to the heat equation and also to a large
class of parabolic equations (Guber \cite{43}). Little by little, with
more or less strong hypothesis, more of the results of
Brelot-Mrs. Herve's theory without $D$ were extended, adapted even
completed; the first steps of the ordinary  Dirichlet problems are
similar \footnote{ See complements in \cite{50}} and a more general
problem for non-relatively compact open sets was studied later (Bauer
\cite{5}); an integral representation is much more difficult by lacking
of a compact base of $S^+$, but Mokobodski [still unpublished]
succeeded to overcome the difficulties of using the most refined
results of Choquet on extreme elements. A precise convergence theorem
for superharmonic functions is not possible without $D$. But one may
use a weak general theorem for any family of real fictions $u_i$ in
any topology logical space, when all $u_i$ belong to a family $\Phi$
of functions $\geq 0$ for which a suitable corresponding notion of
weak thinness is defined; the exceptional set where $\hat{\inf} u_i
\neq \inf u_i$ is a countable union of sets which are weakly thin at
every point (Brelot \cite{16}); and in the strongest Bauer's\pageoriginale theory the
weak thinness of $e \notin x_0$ is identical to the thinness. Let us
emphasize another feature; i.e. the notion of absorbing set. (even
without the last axiom of strong harmonicity). Such a closed set $e$
is characterized by the existence of $v$ hyperharmonic $\geq 0$ which
is zero exactly on the set; or by the property of supporting the
harmonic measure $d \rho^{\omega_0}_{x_0}$ of any regular open
$\omega_0 \,(x_0 \in e, x_0 \in \omega_0)$. This allows to give a deeper
comparison of both Brelot and Bauer's axiomatics. Finally, for the
strongest Bauer's theory the property of nuclearity (even without the
last axiom) and the embedding in a Hunt type have been realized
(\resp Bauer \cite{4}, Boboc-Constantinescu-Corna \cite{10}). This nuclearity
was essentially used in the important thesis of Hinrichsen \cite{48}
which generalizes the fundamental Cauchy formula of function theory in
the Bauer's frame. 

A slight extension of the Bauer's theory was made by Boboc
Consta\-ninescu-Corna \cite{8}, \cite{9} in order to show that weak hyperothesis
which offer some new applications, are sufficient to get various
important results on the specific order (lattice properties and
applications), on reduce functions and balayage. In differs from
Bauer's theory essentially by changing  the separation axiom (which
furnished the minimum principle) into a weaker condition (close to this
principle): the existence of a harmonic positive function in a
neighbourhood of any fixed point and the covering property of $\Omega$
by a family of ``M.P open sets'' for which a suitable minimum
principle holds. 

The\pageoriginale question arises of determining or characterizing all sheaves
satisfying such systems of axioms. Striking results are already in
Bony \cite{14} $\alpha)$ for a domain of $\mathbb{R}^n$ and a sheaf
invariant by translation. With axiom $1,2$ and constants harmonic,
there exists $a)$ in case $\alpha$, an open set $\Omega _0$ dense in
$\Omega$ and a pre-elliptic operator A (quadratic form $\geq 0$) in
$\Omega_0$ with finite continuous coefficients, such that $A u=0
\Leftrightarrow u$ harmonic locally; axiom $K_1$ is even satisfied and
the case of $R^2$ may be depend. $b)$ in case $(\beta)$, there exists
a pre-elliptic operator $A$ with constant coefficients and $``Au=o$
(in the sense of distributions) is equivalent to harmonicity''; as
example of detailed study: ``axion $3 \Leftrightarrow A$ elliptic''. A
book on relation of these axiomatics and operators is under
preparation \cite{13}. 

By deeping these previous axiomatics, Mokobodski and $D$. Sibony
\cite{61} first considered the converse problem of determining a sheaf of
harmonic functions for which the superharmonic functions form a given
cone. More precisely, they start from a convex cone $Gamma$ of
lower semi-continuous bounded functions (on a locally compact space
whose relatively compact open sets $\omega$ have a non-void boundary)
which separate $\Omega$, and suppose essentially that $\Gamma$ is
\textit{maximal} (with respect to the inclusion order) among similar
cones whose functions satisfy the minimum principle (i.e. $u \geq$
constant $\lambda$ on $\partial \omega \Rightarrow  u \geq \lambda
\text{ on } \omega$) (this maximality is satisfied for the cone of
superharmonic functions in large axiomatics where constants are
harmonic). Under some conditions, one may define a sheaf of harmonic
functions (i.e. satisfying axioms $1,2$ and even $K_1$) such that any
bounded superharmonic function on an open\pageoriginale set $\omega$ is equal to a
function of the cone in any $\omega '\subset \bar{\omega '} \subset
\bar{\omega}$ upto a harmonic function. In further very important
researches \cite{62} after completing the Choquet theory of ``adapted
cones'' which is an essential tool,  they start from a convex cone of
continuous functions $\geq 0$ (even bounded) and develop a theory
generalizing Hunt's theory and where the given functions are
potentials. 

Let us finally mention the extension of the Poison integral to the
semi-simple Lie Groups by F\"uztenberg \cite{37bis} (see a lecture by
Delzart \cite{35bis}) and essay  by Monna \cite{62bis} of an axiomatic for
order elements which are more general that functions. 

\begin{thebibliography}{99}
\bibitem{1} {H.Bauer} Math. Ann. $146,1962$\pageoriginale
\bibitem{2}{-} Warscheinlichkeits theorie 1, 1963
\bibitem{3}{-} Ann I.F. 15/1 1965
\bibitem{4}{-} Harmonische Raime und ihre Potential theorie,
  Lecture Noter 22, Springer 1966 
\bibitem{5}{-}  Math. Ann. 164,1966
\bibitem{6}{-} Symposium of Loutraki 1966, Lecture Notes 31,
  Springer 1967 
\bibitem{7}{Bobc-Constantinescu conrna} Nagoya Math. Journal 23,1963
\bibitem{8}{-} Ann. I.F. 15/1 1965
\bibitem{9}{-} Ann. I.F. 15/1 1965
\bibitem{10}{-} Revue roumaine Math. Pure's appl. 1967
\bibitem{11}{Bobic and Cornea} C.R.Ac,Sc. 261, p.2564,1965
\bibitem{12}{-} Bull. Soc. Sc. Math. Phys. roumaine 1965
\bibitem{13}{Boboc and Mustata} Hormonic spaces (to appear Ac. Sc. Bucuresti)
\bibitem{14}{J.M. Bony}  Ann. I.F.17/1 1967
\bibitem{15}{M. Brelot}  J d' analyse Math, S,1960-61
\bibitem{16}{-} Sem. cheorie du potential 6/1 (three papers) 1961-62
\bibitem{17}{-} Annali di matematica 57,1962
\bibitem{18}{-} Ann. I.F.15/1 1965
\bibitem{19}{-} Ansi's Ac. Brazil $37 n^0 1,1965$
\bibitem{20}{-} Berkeley Symposium on prib. Summer 65, published 67
\bibitem{21}{M.Brelot} Aximoatic des fonctions harmoniques,
  Sem. Math. Sup. Univ. de Montreal Summer 1965, published 1966 \pageoriginale
\bibitem{22}{-} Symposium of Lutraki 66, Lecture  notes 31, Springer 1967
\bibitem{23}{-} Nagoya Math. Journal 1967
\bibitem{24}{B.Collin} Ann.I.F 14/2, 1964
\bibitem{25}{C.Constantinescu} Revue roumaine Math. pures and
  appl. 10, 1965 (two papers) 
\bibitem{26}{-} Revue roumaine Math.pures and appl. $11,n^o 8, 1966$ 
\bibitem{27}{-} Revue romaine Math.pures ans appl. $11, n^o o 8, 1966$
\bibitem{28}{-} Nagoya Math.Journal, 25, 1965
\bibitem{29}{-} Revue roumaine Math. pures and appl, (1967)
\bibitem{30}{-} Revue roumaine Math.pures and appl. (to appear) on balayage
\bibitem{31}{-} C R  Sc.Sc. Paris 262 A, p.1309 (1966)
\bibitem{32}{Constantinuescu Cornea} Ann I.F,13/2 (papers I and
  II), 1963 
\bibitem{33}{-} Nagoya Math.Journal 25,1965
\bibitem{34}{A.Cornea} C.R Ac.Sc.Paris, 264, A.p. 190, 1967
\bibitem{35}{J.L Doob} Bull. Amer. Math.Soc. 72 (1966) p. 579
\bibitem{35bis}{ Delzant} Sem.potentail 7
\bibitem{36}{B.Fuglede} Ann I.F 15/2,1965
\bibitem{37}{-}Sem. theorie du potential 10, 1965-66
\bibitem{37bis}{Fuztenberg} Ann Math 77,1963
\bibitem{38}{K.N Gowrishankaran} Ann I. F 13/2, 1963
\bibitem{39}{-} Math. Zeits 94, 1966
\bibitem{40}{-}Nagoya J.of Math.
\bibitem{41}{K.N Gowrishankaran} Ann.I.F. 16/2, 1966\pageoriginale
\bibitem{42}{Getoor} J. of Math. Analysis and appl. 13 1966, p.132
\bibitem{43}{Guber} Loutraki Symposium, Lecture Notes 31
\bibitem{44}{Herve (Mrs R.M)} Ann. I.F. 12, 1962
\bibitem{45}{-} Ann. I.F. 14, 1964
\bibitem{46}{-} Ann. I.F. 15/2, 1965
\bibitem{47}{-}Ann. I.F. 16, 1966
\bibitem{48}{Hinrichsen} Ann. I.F. 17/1, 1967
\bibitem{49}{Kohn} Math. Zeits 91. 1966 p. 50
\bibitem{50}{Kohn and Sieveking} Regulare and extremale Randpunkte in
  der Potential theorie (to appear in Revue roumaine de Mathematiques
  pure et appliqueis)  
\bibitem{51}{A de la Pradellc} Ann. I.F. 17/1, 1967
\bibitem{52}{P. Loeb} Ann. I.F. 16, 1966
\bibitem{53}{P.Loed and Walsh} Ann. I.F. 15/2, 1965
\bibitem{54}{-} Bull. Am Math. Soc. 72, p. 685, 1966
\bibitem{55}{-}(Continuation of [52] to appear)
\bibitem{56}{Lumer-Naim (Mrs L)} Ann. I.F. 17/2, 1967
\bibitem{57}{P.A. Mayer} Ann. I.F. 13/2, 1963
\bibitem{58}{-} Probabilities and Potentials (Blaisdell
  pub. co. Boston 1966, French version, Hermann Paris 1966) 
\bibitem{59}{-} Processes de Markov, Lecture Notes, 26, Springer 1967
\bibitem{60}{G.Mokobodski} Ann. I.F. 15/1, 1965
\bibitem{61}{Mokobodski- sibony} Ann. I.F. 17/1, 1967
\bibitem{62}{Mokobodski-Sibny} C R Ac.Sc.Pris 263(1966) A, p. 126 164
  (1967) A, p.(1967) A, p. 264, Development to appear in Ann I.F\pageoriginale 
\bibitem{62bis} {A.F.Monna} Nieuw Archief voor Wiskunde (3) XIV, pp
  213-221, 1966 
\bibitem{63}{D.Sibony} C R Ac Sc.Paris 260,1965 Imporved
  development to appear Ann I.F. 
\end{thebibliography}

\newpage
Complemetry\pageoriginale Ganaraly Bibliography on Potentional theory (revised in
$2^{nd}$ edition). 

Classical potential theory is being studied further (see for instance
the  treatise of Landkoff 1966). Various aspects of, modern
potential theory are to be seen in the ``colloque detheorie du
potential'' Paris-Orsay (1964) (Anaales T.F 15/2, 1965 ) and in
the Seminaire du potential and Sem Choquet. A general bibliography is
in the third edition of the French course : Brelot ``Elements de la
theorie classique du potential'' Paris 1965. 

Let us give here only short indications on modern potential theory,
for other directions then those studied in these lectures. 

\begin{enumerate}[i)]
\item Study and use of original Martin's boundary and similar boundaries

  \begin{tabular}{lp{5cm}<{\raggedright}}
    {M Parreau} & Ann I.F. 3 (1951-52)\\
    {M Brelot} & J. of Math.  35 (1956)\\
    {L.Naim} & Ann I.F. 7 (1957)\\
    {J.L Doob} & Ann I.F. 9 (1959)\\
    {Brelot and Doob} & Ann I.F. 13 1963\\
    {Dynkin} & Ann I.F. 15/1, 1965\\
    {Constantinescu -Cornea} & Ideale RAnder Riemannscher Flacher
    (Ergeb 32, Springer 1963)
  \end{tabular}
\item Relations with function theory

  \begin{tabular}{lp{5.8cm}<{\raggedright}}
  {Doob}&  J. of Math 5,1961 Anales I.F. 15/2, 1965 \\
  {Kuramochi} & J.Fac Sc. Hokkaido Univ., esp 16,17\\
  Constantinescu-Cornea & (see i))\\
  {Borelot} & Symposim of Erevan  (1965)
  \end{tabular}\pageoriginale
\item  Essential connections with semi-groups and Markov processes.\break
  Wide filed open chiefly by Doob (see previous bibl). Fundamental
  synthesis of Hunt (III J.) of Math. 1 and 2. 1957-58. See
  further researches in Meyer (previous bibl) in the treatise of Dynkin,
  in the reports of the Berkeley (1965) and Loutraki (1966) colloq.,
  in Deny  (Ann I.F.12 (1962), 15/1, (1965)) and Lion (Ann T. F 16/2
  (1966)). 
\item Axiomatistion of Dirichlet integral (by Seurling and Deny) and
  Dirichlet spaces of Deny. 

  \begin{tabular}{lp{6cm}<{\raggedright}}
    {Beurling -Deny} & Acta Math 99 (1958)\\
    {Beuriling} & Symposium on Banach Algebras (Stanford 1958)\\ 
    {Deny} & Sem.Potential (all volumes); Sem. Bourbaki 12, 1959-60\\
    {Thomas} & Sem. pot 9 (1964-65)
  \end{tabular}
\item Axiomatic approch to the Dirchlet problem with the ``Silov boundary''

  \begin{tabular}{lp{6cm}}
    {H. Bauer} & Ann. I. F 11 (1961)\\ 
    Further reserches by\\
    {Rogalski} & C R Ac. Sc. Paris 263 A, p. 664 and 726 and
    Seminaire Choquet
  \end{tabular}
  
\item Bessel potentials by Aronszajn, Smith Ann I. F. 11,
  15/1,17/2,18/2. 
\item Relations\pageoriginale with harmonic analysis, theory of games,ergodic
  theory. See papers of Carleson, Herz, Fuglede, Meyer
  (Ann. I.F. 15/1, 1965). 
\item Relations with partial differential equations. See Mrs. Herve's
  papers (additional bibl); on the Martin boundary for such equations
  see: S Ito J. Math Soc:, japan 16, 1964  

  G. Wildenhain Potential -theorie linearer elliptische Differential\break
  gleichungen beliebiger Ordnung (Dentscha Akod. Berlin 1967). 
\end{enumerate}

