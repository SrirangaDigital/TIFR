\chapter{General Dirichlet Problem}\label{p4:chap5}% chap 5 

\setcounter{section}{21}
\section{Fundamental envelopes}\label{p4:chap5:sec22} % sec 22

The\pageoriginale Perron - Wiener method, introduced in $1924$ to solve and
generalize the classical Dirichlet problem was further deeply studied
and then systematically extended to various ideal boundaries but,
using chiefly the classical harmonic functions. With the
axiomatisation of these functions, we have to develop the method in
the general set-up. 

\begin{defn}\label{p4:chap5:sec22:def15} % definition 15
  A saturated set $\sum$ if hyperharmonic functions on $\Omega$ will
  be called {\em additively}, (\resp {\em completely}) saturated, if
  sum of any two elements of $\sum$ belongs to $\sum$, (\resp any
  linear combination with coefficients $> 0$) and every hyperharmonic
  majorant of an element of $\sum$ belongs to $\sum$. 
\end{defn}

\begin{defn}\label{p4:chap5:sec22:def16} % definition 16
  Let $\mathscr{L}$ be a set of filters $\mathscr{F}$ on $\Omega$ such
  that one of them has any adherent point in $\Omega$. $\mathscr{L}$
  and a set $\sum_{0}$ of hyperharmonic functions in $\Omega$ are said
  to be {\em associated}, if for any $v \in \sum_0$, the condition 
  $$
  \liminf_{\mathscr{F}}. v\ge 0 ~\text{for every}~ \mathscr{F} \in
  \mathscr{L}, \text{implies} v \ge 0 
  $$
\end{defn}

\medskip
\noindent\textbf{Examples.}
  The set $\sum_1$ of all hyperharmonic functions is completely
  saturated. Again the set $\sum_2$ of all hyperharmonic functions
  such that each is bounded below is completely saturated. 

\begin{enumerate}[1)]
\item If\pageoriginale $\Omega_0$ is a relatively compact domain in the fundamental
  space $\Omega$, the intersection of $\Omega_0$ with the
  neighbourhoods of the boundary points of $\Omega_0$ forms a set of
  filters which is associated to $\sum_1$ or $\sum_2$ considered in
  the space $\Omega_0$ if there exists a harmonic function $h$ on
  $\Omega_0$, $h > \varepsilon > 0$. This is satisfied, for instance,
  when $\Omega_0$ is a regular domain or if there exists a potential
  $> 0$ on $\Omega$. (See Theorem $3 (ii)$). 
\item In a bounded euclidean domain (more generally in a ``Green
  space'') let us consider the Green lines (gradient curves of the
  Green functions $G_{x_0}(y)$ issued from the pole $x_0$. The sets on
  a regular line (where $\inf$. $G_{x_0} = 0$) where $G_{x_0} <
  \varepsilon$ forms for all $\varepsilon > 0$ a base of a filter. All
  such filters form a set associated to $\sum_2$. See \cite{2}. 
\end{enumerate}

\setcounter{thm}{15}
\begin{thm}\label{p4:chap5:sec22:thm16} % them 16
  Let $f$ be a real valued function (finite or not) on the set
  $\mathscr{L}$ of filters $\mathscr{F}$ associated to an additively
  saturated set $\sum$ of hyperharmonic functions. The function $v$ of
  $\sum$ satisfying,  
  $$
  \liminf_{\mathscr{F}}. v {\underset{> - \infty}{\ge f(\mathscr{F})}}
  $$
  for every $\mathscr{F} \in \mathscr{L}$, form a saturated set whose
  lower envelope $\bar{\mathscr{H}}_f$ is $+ \infty$, $- \infty$ or
  harmonic. Define $\underline{\mathscr{H}_f}= - \bar{\mathscr{H}}_{-f}$,
  then $\underline{\mathscr{H}}_f \le \bar{\mathscr{H}}_f$.  
\end{thm}

\begin{proof}
  Any $E^\omega_v$ (see $n^0 10$ ) for a regular domain $\omega $ has
  the same $\liminf \limits_{\mathscr{F}}$ as $v$. Therefore the set
  of $v$ under consideration is a saturated\pageoriginale one and we may apply the
  Theorem $8$. It remains to see the inequality. 
\end{proof}

Let $v$ and $w$ be any two functions in $\sum$ satisfying respectively
the conditions $\liminf \limits_{\mathscr{F}} v {\underset{> -
    \infty}{\ge f(\mathscr{F})}}, \lim.  \inf. w {\underset{> -
    \infty}{\ge f(\mathscr{F})}}$ 

Now
$$
\liminf \limits_{\mathscr{F}} (v+w) \ge \liminf \limits_{\mathscr{F}}v
+ \liminf \limits_{\mathscr{F}}.w \ge 0 
$$

As $v + w \in \sum,  v + w \ge 0$.

As $v$ and $w$ are $> - \infty$, $v \ge - w$, $\inf \limits_{v}$. $v
\ge \sup \limits_{w} (-w) = -\inf \limits_{w}$ or $\bar{\mathscr{H}}_f
\ge - \bar{\mathscr{H}}_{-f}$. 

\section{Properties of \texorpdfstring{$\bar{\mathscr{H}}_f$}{Hf}}\label{p4:chap5:sec23} % sec 23

From now we suppose essentially $\bar{\mathscr{H}}_f$ is defined by
means of a completely saturated set $\sum$ and of an associated set
$\mathscr{L}$. 

\begin{prop}\label{p4:chap5:sec23:prop13} % prop 13
  \begin{enumerate}[\rm (i)]
  \item $\bar{\mathscr{H}}_f$ is an increasing positively homogeneous
    fun\-ction of $f$ 
  \item If $\bar{\mathscr{H}}_f + \bar{\mathscr{H}}_g$ has a meaning
    (at a point and hence every where), then it is $\ge
    \bar{\mathscr{H}}_{f+g}$ where $f + g$ is arbitrarily chosen when
    ever it is not defined.	 
  \end{enumerate}
\end{prop}

\noindent 
\textbf{(Basic property) Theorem 17.}\label{p4:chap5:sec23:thm17} % them 17
\textit{Let $f_n$ be an increasing sequence of real functions on $\mathscr{L}$
converging to $f$ and further $\bar{\mathscr{H}}_{f_n} > - \infty$ for
every $n$. Then $\bar{\mathscr{H}}_{f_n} \rightarrow
\bar{\mathscr{H}}_f$. }

The theorem is obviously true if $\lim \bar{\mathscr{H}}_{f_n} = +
\infty$. Hence we assume $\lim$. $\bar{\mathscr{H}}_{f_n} < + \infty$;
and it is enough to show that $\varlimsup\limits_{n}
\bar{\mathscr{H}}_{f_n} \ge \bar{\mathscr{H}}_f$.\pageoriginale Let $x_0$ be any
point of $\Omega$. For an arbitrary $\varepsilon > 0$ choose
$\varepsilon_n$ such that $\sum \varepsilon_n = \varepsilon$ Let $v_n$
be a hyperharmonic function such that $\liminf
\limits_{\mathscr{F}}. v_n \ge {\underset{- \infty}{f (\mathscr{F})}}$
and $v_n (x_0) \le \bar{\mathscr{H}}_{f_n} + \varepsilon_n$ for every
$n$. Define $W = \lim \bar{\mathscr{H}}_{f_n} + \sum \limits^\infty_{n
  = 1}(v_n - \bar{\mathscr{H}}_{f_n})$. This hyperharmonic function
majorises any $v_n$, therefore belongs to $\sum$ and satisfies
$\liminf\limits_{\mathscr{F}}. W \ge f_n (\mathscr{F})$ then $\liminf
\limits_{\mathscr{F}}. W {\underset{> - \infty}{\ge f (\mathscr{F})}}$
we conclude $W \ge \bar{\mathscr{H}}_f$ and $W (x_0) \le \lim
\bar{\mathscr{H}}_{f_n} (x_0) + \varepsilon$. Therefore
$\bar{\mathscr{H}}_f(x_0) \le \lim
\limits_{n}. \bar{\mathscr{H}}_{f_n} (x_0)$. 

\noindent 
\textbf{Negligible sets: Definition 17.}\label{p4:chap5:sec23:def17} % def 17
\textit{A subset $\alpha \subset \mathscr{L}$ is said to be negligible if
$\bar{\mathscr{H}}_{\varphi \alpha} = 0$ for the characteristic
function $\varphi_\alpha$. Subsets of negligible sets are negligible
and any countable union of negligible sets is again negligible. The
term ``almost everywhere on $\mathscr{L}$'' is used as equivalence of
``except on a negligible set''.} 

\medskip
\noindent 
\textbf{Applications}
\begin{enumerate}[(i)]
\item If $f = 0$ almost everywhere, then $\bar{\mathscr{H}}_{f} = 0$
  (consider first theorem where $f$ is $+ \infty$ on a negligible set
  and zero else where).	 
\item If $\bar{\mathscr{H}}_{f}$ and $\underline{\mathscr{H}}_{f}$ ar
  finite, the set where $f$ is $\pm \infty$ is negligible. (Consider
  $f + (-f)$ and use Prop. $13 (ii)$). 
\item If $f_1 = f_2$ almost everywhere, then $\bar{\mathscr{H}}_{f_1}
  = \bar{\mathscr{H}}_{f_2}$.  
\end{enumerate}

(Consider first $f_1 = + \infty$ on a negligible set and $= f_2 $
elsewhere, use prop. $13 (ii)$). 

\section{Resolutivity}\label{p4:chap5:sec24} % sec 24

\setcounter{defn}{17}
\begin{defn}\label{p4:chap5:sec24:def18} % def 18
  If\pageoriginale $\bar{\mathscr{H}}_{f}$ and $\underline{\mathscr{H}}_{f}$ are
  equal at a point, they are equal everywhere. In case they are equal
  finite, therefore harmonic, $f$ is said to be resolutive and the
  common envelope $\mathscr{H}_f$ is called the {\em generalised
    solution}. 	 
\end{defn}

\noindent
\textbf{First Properties}
\begin{enumerate}[1)]
\item If $f$ is resolutive for any constant $\lambda \neq 0$, $\lambda
  f$ is resolutive and $\mathscr{H}_{\lambda f} = \lambda
  \mathscr{H}_f$. 
\item If $f_n$ is an increasing sequence of resolutive functions,
  $\lim. f_n = f$ is resolutive if ${\mathscr{H}}_{f_n}(x_0)$ is
  bounded at some point $x_0$. For ${\mathscr{H}}_{f_n} \le
  \underline{\mathscr{H}}_f \le \bar{\mathscr{H}}_{f}$ and
  $\bar{\mathscr{H}}_{f_n} \rightarrow \bar{\mathscr{H}}_{f}$. 
\item If $f_n$ are finite valued resolutive functions converging
  uniformly to $f$ and if $\bar{\mathscr{H}}_{1}$ is finite, then $f$
  is resolutive. For any $\varepsilon > 0$,	 
  $$
  f_n - \varepsilon \le f \le f_n + \varepsilon ~\text{for}~ n \ge
  N(\varepsilon). 
  $$

  Then
  $$
  \displaylines{\hfill 
    \underline{\mathscr{H}}_{f_n} + \underline{\mathscr{H}}_{ -
      \varepsilon} \le \underline{\mathscr{H}}_{f} \le
    \bar{\mathscr{H}}_{f} \le \bar{\mathscr{H}}_{f_n} +
    \bar{\mathscr{H}}_{\varepsilon}. \hfill \cr
    \text{or}\hfill 
    \underline{\mathscr{H}}_{f_n} - \varepsilon \bar{\mathscr{H}}_{1}
    \le \underline{\mathscr{H}}_{f} \le \bar{\mathscr{H}}_{f} \le
    \bar{\mathscr{H}}_{f_n} + \varepsilon {\mathscr{H}}_{1}.\hfill } 
  $$
\item If $f$ is resolutive and $f = f_1$ almost everywhere, then $f_1$
  is resolutive and ${\mathscr{H}}_{f_1} = \mathscr{H}_f$. For, if $f$
  is resolutive the set where $f$ is infinite is negligible. The
  property follows as a consequence of (ii) and (iii) of $\S 23$. 
\end{enumerate}


\noindent
\textbf{Equivalence classes of resolutive functions}.\pageoriginale

It is desirable to have the resolutive functions $f$ and
$\mathscr{H}_f$ as summable functions and the corresponding integral
in some suitable sense. 

We shall say that two  resolutive functions $f_1$ and $f_2$ are
equivalent $(f_1 \sim f_2)$ if $f_1 = f_2$ almost everywhere. Let
$\bar{f}$ denote the class containing $f$. 

The set $\Gamma$ of equivalence classes of resolutive functions is a
real vector space with obvious addition and scalar
multiplication. $\mathscr{H}_f$ is the same for every functions in the
same class. Hence every point of $\Omega$ defines a linear functional
on $\Gamma$, namely, the value of $\mathscr{H}_f(x_0)$ for the
respective classes. We may also introduce a natural order in $\Gamma$;
an equivalence class $\bar{f}$ is $\ge$ another class $\bar{g}$ if any
function in $\bar{f}$ is almost everywhere greater than or equal to
any function in $\bar{g}$. In order to see whether $\sup$ ($f$, $g$)
$\in \Gamma$, we have to study $\sup$ ($f$, $g$). We only prove: 

\begin{Basic Lemma} % lemma 1
  If $f $ and $g$ are resolutive, $\underline {\mathscr{H}}_{
  \sup. (f, g)} = \bar{\mathscr{H}}_{\sup. (f, g)}$  (both the
  sides are $= + \infty$ or harmonic and equal). Assume
  $\underline{\mathscr{H}}_{\sup (f, g)}<  + \infty$, otherwise
  the Lemma is obvious. Observe that $\sup (\mathscr{H}_f$,
  $\mathscr{H}_g) \le \underline{\mathscr{H}}_{\sup. (f, g)}$. There
  exists a least harmonic majorant $h_0$ for $\sup(\mathscr{H}_f$,
  $\mathscr{H}_g)$, hence $h_0 \le \underline{\mathscr{H}}_{
  \sup. (f, g)}$. For any $\epsilon > 0$ and a point $x_0 \in \Omega$
  make the choice of $v$ and $w$ such that for every $\mathscr{F}$, 
  \begin{gather*}
    {\rm(i)} \liminf \limits_{\mathscr{F}}.v \underset{> - \infty}{\ge f
      (\mathscr{F})}\quad {\rm(ii)}\liminf \limits_{\mathscr{F}}.w \underset{> -
      \infty}{\ge fg(\mathscr{F})}\\ 
    v(x_0) \le \mathscr{H}_f (x_0) + \frac{\in}{2}~  w(x_0) \le \mathscr{H}_g
    (x_0) + \frac{\in}{2}.  
  \end{gather*}

  Then\pageoriginale $w_1 = h_0 + (v - \mathscr{H}_f ) + (w - \mathscr{H}_g)$ is
  hyperharmonic $\geq v$ and $w$, is in $\sum$ and  
  $$
  \liminf _{\mathscr{F}}.w_1  \ge \sup_{> - \infty}(f (\mathscr{F}), g
  (\mathscr{F})) 
  $$

  Therefore $w_1 \ge \bar{\mathscr{H}}_{\sup. (f, g)}$ $\varepsilon
  + h_0 (x_0) \ge \bar{\mathscr{H}}_{\sup. (f, g)} (x_0)$ then
  $h_0 (x_0) \ge \bar{\mathscr{H}}_{\sup (f, g )} (x_0)$ and  
  $$
  \underline{\mathscr{H}}_{\sup.(f,g)}(x_0) \ge \bar{\mathscr{H}}_{\sup(f,g)^{x_0}}
  $$
\end{Basic Lemma}

\medskip
\noindent 
\textbf{Consequence. Proposition 14.}\label{p4:chap5:sec24:prop14} % prop 14
\textit{From the proof we see that if $\mathscr{H}_f$ and $\mathscr{H}_g$
 have a common superharmonic majorant, this function majorises
 $\bar{\mathscr{H}}_{\sup. (f, g)}$.} 

\noindent
\textbf{Corollaries.} 
If $f $ and $g$ are resolutive, then $\sup.(f,g)$ is resolutive if and
only if $\mathscr{H}_f,  \mathscr{H}_g$ have a common superharmonic
majorant. 

\begin{example*} 
  \begin{enumerate}[1)]
  \item if $f$ is resolutive, then $f^+$ is resolutive if and only if
    $\mathscr{H}_f$ has a superharmonic majorant $\geq 0$. 
  \item  If $f,g$ are resolutive and $\ge 0 \sup$. ($f$, $g$) and
    $\inf$. ($f$, $g$) are\pageoriginale resolutive. Therefore the set of the
    resolutive functions $\ge 0$, the set of their equivalence classes
    are lattices for the natural order. 
  \end{enumerate}
\end{example*}
 
If we want a subspace of $\Gamma$ which would contain the class of the
ordinary sup. of two functions or only which would be a Riesz space
for the natural order, and $f$ belonging to such a class would be
majorised by a resolutive function $\ge 0$; therefore $f^+$ would be
resolutive. Hence any subspace under consideration would be contained
in the subspace of the equivalence classes of the absolutely
resolutive functions defined as follows. 

\section{Absolute Resolutivity}\label{p4:chap5:sec25} % 25

\begin{defn}\label{p4:chap5:sec25:def19} % def 19
  A real function $f$ on $\mathscr{L} $ is said to be absolutely
  resolutive if $f^+$ and $f^-$ are both resolutive. It is equivalent
  to say that $f$ equals the difference of two resolutive functions
  $\ge 0$ almost everywhere or at every point where the difference has
  sense. 
\end{defn}

Now the finite absolutely resolutive functions and the equivalence
classes of absolutely resolutive functions contain \resp the ordinary
$\sup$. of two functions and the corresponding classes; and anyone of
these classes contains finite absolutely resolutive functions. 

We have the largest subspace of $\Gamma$ we wished to have. More over,
we give the following interpretation as a Daniell Integral of
$\mathscr{H}_f$ for an absolutely resolutive functions $f$. 

Starting with the Riesz space of the finite absolutely resolutive
functions $\alpha $ and the increasing linear functional
$\mathscr{H}_\alpha (x_0)$, we\pageoriginale see that they satisfy the Daniell
condition, viz : $\alpha_n$ decreasing to zero implies
$\mathscr{H}_{\alpha_n} (x_0)\break \to 0$ ($x_0$ fixed in $\Omega$). We
define the corresponding Daniell integral by continuation of the
functional as follows: (for fixed $x_0 \in \Omega$) if $\psi =
\lim. \alpha_n$ ($\alpha_n$  increasing), (we see that
$\lim. \mathscr{H}_{\alpha_n} (x_0)$ is the same for all sequences
$\alpha_n$ with limit $\psi$, and we denote $I
(\psi)=\lim$. $\mathscr{H}_{\alpha_n}(x_0)$. In the same way if
$\varphi = \lim. \alpha_n$ ($ \alpha_n$ decreasing) we denote $I
(\varphi) = \lim. \mathscr{H}_{\alpha_n}(x_0)$. We have immediately $I
(\psi) = \bar{\mathscr{H}}_\psi (x_0), I (\varphi) =
\underline{\mathscr{H}}_\varphi (x_\circ)$. Now for any function
$f(\mathscr{F})$, the superior and the inferior Daniell integrals are
defined as  
$$
\bar{I}(f) = {\substack{\Inf\\\psi \geq f}}, I (\psi), ~ \underbar{I} (f) =
\substack{\sup\\{\varphi \leq f}}. I (\varphi) 
$$
It is obvious that $\underbar{I}(f) \leq \underline{\mathscr{H}}_f
(x_0) \leq \bar{\mathscr{H}}_f (x_0) \leq \bar{I}(f)$. $f$ is said to
be $I$-summable if $\underbar{I}(f) = \bar{I}(f)$ (finite); in this
case, it is well known that $f^+$ and $\bar{f}$ are also $I$- summable
and therefore resolutive. 

Then $f$ is absolutely resolutive and $I(f) = \mathscr{H}_f (x_0)$.

Conversely, suppose $f$ is resolutive and $\geq 0$. Define $f_n$ as a
function equal to $f$ where $f$ is finite and to $n$ where $f$ is
infinite; $f_n$ is resolutive, i. e. a function of type $\alpha$,
therefore an $I$-summable function and
$I(f_n)=\mathscr{H}_{f_n}(x_0)$. As $f_n$ is increasing we get 
$$
I(f_n) = \mathscr{H}_{f_n} (x_0) \to \mathscr{H}_f (x_0)~ (\text{
  finite }). 
$$

But\pageoriginale the limit of $I$- summable functions $f_n$ with $I(f_n)$ bounded
is again $I$-summable and $I(f) = \lim. I(f_n) = \mathscr{H}_f(x_0)$. 

We may now consider any absolutely resolutive function $f$ and conclude,
\setcounter{thm}{17}
\begin{thm}\label{p4:chap5:sec25:thm18}%them 18
  For any $x_0 \in \Omega$, the absolutely resolutive functions
  $f(\mathcal{F})$ are the summable functions for a certain Daniell
  integral $I$ (positive linear functional) (or a certain abstract
  positive measure) on $\mathscr{L}$ and $I (f) = \mathscr{H}_f(x_0)$. 
\end{thm}

\begin{coro*}
  For a set $e \in \mathscr{L}$ and the characteristic function
  $\varphi_e$ the condition $\bar{I}(\varphi_e) = 0$ is equivalent to
  $\bar{\mathscr{H}}_{\varphi_e} = 0$, which does not depend on
  $x_0$. 
\end{coro*}

\setcounter{prop}{14}
\begin{prop}\label{p4:chap5:sec25:prop15}%proposi 15.
  Any resolutive function is absolutely resolutive in the following cases
  \begin{enumerate}[\rm(i)]
  \item $\sum $ is the set of all hyperharmonic functions such that
    every element has a harmonic minorant $\leq 0$. 
  \item $\sum$ is the set of all lower bounded hyperharmonic functions
    and moreover $\bar{\mathscr{H}}_1$ is finite. 
  \end{enumerate}
\end{prop}

In fact in both the cases, for any resolutive function $f$,
$\mathscr{H}_f$ has a superharmonic majorant $\geq 0$. 
\setcounter{remark}{0}
\begin{remark}%remark 1.
  For any Daniell integral (positive linear functional) let us start
  from all finite summable functions and apply the Daniell
  continuation with the same values for the integrals. We get same
  superior and inferior integrals and the set of summable functions
  remains unchanged.\pageoriginale 
\end{remark}

Therefore a Daniell integral is completely defined by the class of the
summable functions or of the finite summable functions and the value
of the corresponding integrals. On the other hand if we start with
some Riesz-space of finite summable functions the new summable
functions are summable in the old sense but the converse is not true. 

As for the $I$-integral of Theorem \ref{p4:chap5:sec25:thm18}, it is therefore interesting to
note that we may start with all bounded $I$-summable functions
(i. e. absolutely resolutive functions) and then get all summable
functions by means of the Daniell continuation, if we suppose that
there exists a resolutive function $\varphi$ which is bounded and $ >
0$. 

In fact if $f \geq 0$ is resolutive, $\psi_n = \inf.  (f, n \varphi)$
is bounded resolutive and $\mathscr{H}_{\psi_n} \leq
\mathscr{H}_f$. Hence the limit of $\psi_n$, i.e., $f$ is summable in
the sense of the integral we obtain by continuation. 
\begin{remark}%remark 2.
  The case where constants are resolutive is important, because in
  this case any $I$-summable function is $I$-measurable. 
\end{remark}

A particular case is the one where the constants are harmonic
in $\Omega$. Therefore, if there exists in some theory, a harmonic function $h
> 0$, it seems better to use the $h$-harmonic and $h$-super-harmonic
functions and to study the Dirichlet problems for these functions. 
