\chapter{Negligible Sets and Regular Kernels}\label{p3:chap2}%chap II

\setcounter{section}{3}
\section{Definitions and Fundamental Lemmas}\label{p3:chap2:sec4}%sec 4

In\pageoriginale the convergence that we are going to discuss, we
shall meet with 
exceptional set of points which may be ignored in some sense, such as
a set of measure zero in measure theory. We wish to introduce a notion
of smallness to describe these sets, without any appeal to the
capacity theory at the first instance. We remind that the kernel $G(x,
y)$ is always supposed to be lower semi-continuous $\ge 0$. 

\begin{defn}\label{p3:chap2:sec4:def4}%defi 4
  A compact set $K$ in $E$ is said to be G-negligible if for any Radon
  measure $\mu \neq 0$ on $K$, the potential $G \mu (x)$ is unbounded
  on $K$. 
\end{defn}

It is obvious that finite union of $G$-negligible compact sets is
$G$-negligible (whereas it may not be true of the intersection). Compact
subsets of $G$-negligible sets are not necessarily $G$-negligible. In
order to avoid such a disturbing possibility we shall introduce a
restriction on the kernel. 

\begin{defn}\label{p3:chap2:sec4:def5}%defi 5
  A kernel $G(x, y)$ is said to be `\textit{ regular }' (Choquet) if
  it satisfies that following \textit{continuity principle :} for any
  measure $\mu \in \mathfrak{m}^+$ with compact support $K$, if the restriction
  to $K$ of $G \mu (x)$ is finite and continuous, then $G \mu (x)$ is
  finite and continuous on the whole space. 
\end{defn}

Immediately follows from Lemma \ref{p3:chap1:sec3:lem1} the
\begin{Lemma}\label{p3:chap2:sec4:lem2}%lemma 2
  If $G$ is regular, for any non-zero measure $\mu \in \mathfrak{m}^{+}$ with
  compact support $K$ such that $G \mu (x)$ is finite on $K$, there
  exists a non-zero measure\pageoriginale $\mu' \leq \mu$ with compact
  support in a 
  subset of $K$, such that $G \mu' (x)$ is finite and continuous on
  the whole space. 
\end{Lemma}

Consequently for regular kernels, any compact subset of a $G$-negli\-gible
compact set is also $G$-negligi\-ble. This enables us to give a consistent
definition of general $G$-negligible sets. 

\begin{defn}\label{p3:chap2:sec4:def6}%defi 6
  For a regular kernel $G$, a set $A$ is defined to be $G$-negligible if
  every compact set contained in $A$ is negligible. 
\end{defn}

An equivalent definition for the G-negligibility in the case of a
relatively compact set $A$ is that for any non-zero measure supported
by $A $ (i.e. for which complement of $A$ has outer measure zero), $G
\mu (x)$ is unbounded on $A$. 

\noindent
\textbf{Fundamental Lemma (3)}. If $G$ is regular, on every compact
set $K$ which is not $G$-negligible, there exists a positive measure
$\mu \neq 0$, such that $G \mu (x)$ is finite and continuous on
$E$.\footnote[1]{In the classical case, it is a generally
  ignored result of de la Vallee Poussin (Le Potentiel Logarithmique :
  Paris et Louvain' 49): a consequence of already developed theory. By
  means of the Lusin property, Choquet introduced this and
  Lemmas \ref{p3:chap1:sec3:lem1} and \ref{p3:chap2:sec4:lem2}
   at the beginning, as powerful tools.}

\begin{prop}\label{p3:chap2:sec4:prop4}%prop 4
  $G$ being regular, given a (positive) measure $\mu$ on a compact set
  $K$ such that $G \mu (x)$ is finite on $K$, there exists an
  increasing sequence $\mu_n$ of measures on $K$ with the following
  properties: 
\end{prop}

\begin{enumerate}[(i)]
\item $\mu_n \le \mu$ for all $n$
\item $(\mu_n - \mu) (E) \to 0$ as $n \to \infty$. This implies $G
  \mu_n \to G \mu$.  
\item $G \mu_n$\pageoriginale is finite continuous on $E$. 
\end{enumerate}

\begin{proof}
  Let $\{ \varepsilon _n\}$ be a decreasing sequence of positive numbers
  tending to zero. There exists a measure $\nu_1 \leq \mu,  \int  d
  \mu - \int \limits_1 d \nu_1 < \varepsilon _1$ and $G\nu _1$ finite to
  continuous on $E$. Let $\mu = \nu_1 + \mu'_1$. A similar argument
  applied to  $\mu'_1$ gives a measure $\nu_2  \leq \mu'_1$ and such
  that $\int d \mu' _1 - \int d \nu_2 < \varepsilon _2$ and $G \nu_2$
  finite and continuous on $E$. Then induction assumption on $n$ and
  a similar argument for the passage from $n^{th} $ to $( n + 1)th$
  stage provide a sequence of measures $\{\nu_i\}$. The sequence $\{
  \mu_n\}$ of measures defined by $\mu _n = \sum\limits_{i = 1}^n \nu
  _i $ answers to our call. The first two result from the choice of
  $\mu' _n$ s and imply $\mu_n \to \mu$ vaguely. Because of lower
  semi- continuity of $G \mu (x)$ in $E \times \mathfrak{m}^+ : \lim
  ._{n} \inf . G\mu_n (x) \geq G(\mu)(x)$ for $x$ in $E$. But $G\mu
  _n \geq G \mu$. Hence $G\mu_n (x) \to G\mu(x)$.  
\end{proof}

\section{Associated kernel and Energy (Choquet)}\label{p3:chap2:sec5}

\begin{defn}\label{p3:chap2:sec5:def7} % definition 7
  The associated kernel $G^*$ of a kernel $G$ on $E$ is defined to be
  $G^* (x, y) =G (y, x )$.  
\end{defn}

\begin{defn}\label{p3:chap2:sec5:def8} % definition 8
  {\em Energy} The $G$- energy of a measure $\mu$ is the integral
  $\int G \mu (x) d \mu (x)$.  

  It is noted immediately that $G$-energy and $G^*$-energy of a
  measure are equal.  

  An immediate consequence of this remark is the 
\end{defn}

\begin{prop}\label{p3:chap2:sec5:prop5} %pro 5
  If $K$ is a compact set which is not $G$-negligible, there exists\pageoriginale a
  measure $\mu \neq 0$ on $K$ for which $G$-energy is finite.  
\end{prop}

\begin{prop}\label{p3:chap2:sec5:prop6} %pro 6
  If $G$ is a regular kernel on $E$ and $\mu \neq 0$ on $K$ with
  finite $G$-energy then $K$ is not a $G$-negligible set. 
\end{prop}

\begin{proof}
  Since the energy is finite, $G \mu (x)$ is finite valued almost
  everywhere (relative to the measure $\mu$) on $K$. It is possible to
  find (See lemma \ref{p3:chap1:sec3:lem1}) a compact set $K'\subset
  K$ for which the 
  restriction of $\mu'$ of $\mu$ is $\neq 0 $ and the restriction of
  $G\mu'$ finite and continuous ; $K'$ is not $G$-negligible. Now, $K$
  is not $G$-negligible follows because $G$ is regular.  
\end{proof}

\begin{coro*}
  For regular kernels $G$ and $G^*$, the negligible sets are the same. 
\end{coro*}

\section{Examples of Regular kernels}\label{p3:chap2:sec6} %section 6

\textbf{A. Newtonian kernel }. Let us consider for simplicity the space
$R^3$ and the kernel $G(x, y) = \dfrac{1}{|x - y|}$. Let $\mu \geq 0$
be a measure on a compact set $K$ in $R^3$. We shall prove that if the
function $G \mu (x) $ restricted to $K$ is finite continuous at a point
$x_\circ$ in $K$,  then $G \mu (x) $ is itself continuous at
$x_\circ$ as a point of the whole space (property of Evans
Vasilesco). Since $G \mu(x)$ is finite and continuous in $C_K$, the
regularity follows.  

In fact for every point $x$ and a projection $y$ on $K(| x - y |
\min.)$  we have for any $z \in K$,.  
$$
\big| y - z \big| \leq \big| x - y \big| +  \big| x - z \big| \leq 2
\big| x - z \big| 
$$

Hence for any measure $\nu$ on $K$,
$$
\int \frac{d \nu (z)}{| x - z |} \leq  2 \int \frac{d \nu (z)}{\big| y
  - z \big|} 
$$

Let\pageoriginale $V$ be a spherical domain of centre $x_\circ, \mu_V$ and $\mu
_{CV}$ be the restrictions of $\mu$ to $V \cap K$ and $CV \cap
K$. Since $G \mu (x_\circ)$ is finite there is no mass at $x_\circ$,
and $ C \mu_V (x_\circ) < \varepsilon$ if $V$ is sufficiently
small. Because of the hypothesis of continuity and because of the
obvious continuity in $V^\circ$ of $G \mu _{CV}$, the restriction of
$G\mu_V$ to $K$ is continuous at $x_\circ$ and smaller than $2
\varepsilon$ is another smaller spherical domain $V_1$. Hence $G \mu
_V < 4 \varepsilon$ in $V_1$.   
\begin{align*}
  \big| G \mu (x) - G \mu (x_\circ) \big|  & \leq \big|G \mu _{CV}(x)
  - G \mu_{CV} (x_\circ) \big| + \big| G \mu _{V}(x)\big|\\ 
  & \hspace{4cm}- \big| G \mu_{V} (x_\circ ) \big| \quad (x
  \in V_1)\\  
  & \leq \big| G\mu_{ CV} (x)  - G\mu_{ CV} (x_\circ ) \big| + 5
  \in\\ 
  &\leq  6 \in \text{ for } \big| x - x_\circ \big| \text{
    sufficiently small}. 
\end{align*}

This completes the proof. 

Such a proof can be immediately extended in $R^n \,(n \geq 3)$ with the
kernel $\big| x - y \big|^{  2-n}$ ; and to a bounded region $A$ of
the plane with the kernel $\log \dfrac{D}{| x - y |}$ ($D >$  diameter
of $A$). It is even obvious to make extension in metric space and much
more general kernels, for example $| x - y |^{- \alpha} (\alpha > 0 )$
. In fact many such examples are particular cases of the following
one.  

\medskip
\noindent
\textbf{$B$. General example of regular kernel }

\begin{defn}\label{p3:chap2:sec6:def9}%def 9
  {\em Maximum principle } A kernel $G$ is said to satisfy the maximum
  principle if for every measure $\mu \geq 0$ with compact support
  $K$, $G \mu (x) \sup\limits _{y \in K}$.  $G \mu (y)$ for every $x \in
  E$ 
\end{defn}

That\pageoriginale the Newtonian and more general kernels in the Euclidean spare or
in a metric space satisfy the maximum principle was proved by Maria
and Frostman. In what follows we shall use a Weaker condition.  

\begin{defn}\label{p3:chap2:sec6:def10}%def 10
  A Kernel $G$ is said to satisfy a {\em weak maximum principle } if
  there exists $a \,\lambda > 0$ such that for every measure $\mu \geq 0$
  with compact support $S \mu,  G \mu (x) \leq.  \lambda \sup
  \limits_{ y \in S} G \mu (y)$  for every $x \in E$.  
\end{defn}

\begin{thm}[Similar to a Choquet's theorem]\label{p3:chap2:sec6:thm2}%the 2
  Let $G$  be a continuous kernel, finite for $x \neq y$. If $G$
  satisfies the weak maximum principle in $E$ or only {\em locally} (
  that is to say in an open neighbourhood of each point), then $G$ is
  regula.  
\end{thm}

\begin{proof}
  We suppose that, the restriction of $G$ to the support $K$ (compact
  ) of $\mu$ is finite continuous at $x _\circ$ in $K$, and we shall
  prove that $G \mu$ is continuous at $x_\circ$. It is enough to
  consider $x_\circ$ on the boundary of $K$.  
\end{proof}

Suppose first $G(x _ \circ,  x_\circ)  = + \infty$ ,  then the mass of
$x _\circ$. If $V$ is an open
neighbourhood of $x_\circ$, let us introduce the restriction $\mu_V$
and $\mu_{CV}$ of the measure $\mu$ to $V$ and $CV$ respectively.  $G
\mu_{CV}$ is finite and continuous in $V$ because of the finiteness and
continuity of $G$ in the complement of the diagonal. Thus $G \mu_V$
has a restriction to $V \cap K$ which is continuous at $x_\circ$.  If $V$ is
small enough,  $G \mu _V (x_\circ) < \varepsilon$ and $G \mu _V (x) <
2 \varepsilon$ for $x \in V_1 \cap K$ where $V_1$ is a
compact neighbourhood ( small enough ) of $x_\circ,  \subset
V$.\pageoriginale Therefore $G \mu_{V_1}< 2 \epsilon$ on $V_1 \cap K$.
Applying the weak principle to $V$, we get  $G\mu_V < 2 \varepsilon
\lambda$ in $V$.    

Now 
$$
|G \mu (x) - G \mu (x _\circ)|  = | G \mu_{C V_1}(x_0) 
 - G \mu_{C V_1} (x)|+| G_{\mu_{V_1}}(x)|+|G \mu_{V_1}(x_\circ)|. 
$$

On
 the right the third term is less than 
$\varepsilon$, the second is less then $2 \lambda \varepsilon$ in
$V$. For a neighbourhood $V_\circ $ of $x_\circ$ we know a $\lambda$
which remains valid for any $V \subset V_\circ$. Then for any
$\varepsilon $ we determine $V$ in $V_\circ$, then a compact $V_1$ so
that the second member will be arbitrarily small if $| x - x_\circ| $
is small enough.  Suppose now $G(x_\circ,  x_\circ) \neq \infty;
\mu$. has a mass $m$ at $x_\circ$ whose potential is finite
continuous. The potential due to the other masses is finite continuous
at $x_\circ$ as in the previous case. Hence the same property for $G
\mu$.  
