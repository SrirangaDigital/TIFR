\chapter{Thins Sets - Some Applications\texorpdfstring{${}^1$}{1}}\label{p4:chap9}
\footnotetext[1]{This Chapter develops partly, with some improvements, a note
  published in collaboration with R.M. Herve \cite{2}.}  
 
\setcounter{section}{37}
\section{}\label{p4:chap9:sec38}%sec 38

To\pageoriginale start with we suppose the axioms \ref{p4:chap1:sec1:axiom1},
\ref{p4:chap1:sec1:axiom2} and \ref{p4:chap1:sec1:axiom3}  and the existence
of a potential $V > 0$ on $\Omega$. Without the last condition the
following definition would not allow the existence of thinsets. 

\begin{defn}\label{p4:chap9:sec38:def23} % def 23
  A set $E \subset \Omega$ is said to be thin at a point $x_\circ \notin
  E$, if $x_\circ \notin \bar{E} $ or if $x_\circ \in \bar{E} $ and if there
  exists a superharmonic function $v \ge 0$ on $\Omega$ such that
  $\underset{x \in E, x \to x_\circ}{\lim\inf} v(x) > v(x_\circ). (v$ is said to
  be associated to $E$ and $x_\circ$.) 
\end{defn}

A set $E \subset \Omega$ is said to be thin at $x_\circ \in E, $ if $\{
x_\circ\} $ is polar and $E - \{x_\circ\}$ is thin at $x_\circ$. 

It is therefore obvious that for any superharmonic function function
$v \ge 0$ in $\Omega$, if a set $E$ is not thin at $x_\circ$, then $x_\circ
\in \bar{E}$ and $\underset{x \in E, x \to x_\circ}{\lim, \inf,} v
v(x_\circ)$. A finite union of thin sets at $x_\circ$ is thin at $x_\circ$.  

\begin{prop}[Local character Proposition]\label{p4:chap9:sec38:prop24}
  %%%prop 24
  A set $E$ is thin at $x_\circ$ if and only if for every domain $\omega$
  containing $x_\circ, E \cap \omega$ is thin at $x_\circ$ in the subspace
  $\omega$. 
\end{prop}

This is an immediate consequence of the continuation theorem
(\ref{p4:chap4:sec21:thm14}) and Theorem \ref{p4:chap7:sec33:thm25}. 

\begin{thm}[General ciriterion]\label{p4:chap9:sec38:thm29}%thm29
  Let $v$ be a superharmonic \footnote{Superharmonic is
    unnecessary. Same proof.} function $> 0$ in $\Omega$, finite and\pageoriginale
  continuous at the point $x_\circ \in \Omega$. In order that a set $E
  \notni x_\circ$ is thin at $x_\circ$, it is necessary and sufficient that
  there exists a neighbourhood $\sigma$ of $x_\circ$ such that $R^{E \cap
    \sigma}_v ~ (x_\circ) < v(x_\circ)$. 
\end{thm}

\begin{proof} % pro 
  If $E$ is thin, the property is obvious when $x_\circ \notin \bar{E}$;
  if $x_\circ \in \bar{E}$, then there exists a superharmonic function $w
  > 0$ such that 
  \begin{align*}
    & \lim. \inf. w(x) > w (x_\circ)\\
    & x \in E, x \to x_\circ
  \end{align*} 
  
  Choose $\lambda > 0$ such that $\lambda v(x_\circ)$ lies strictly in
  between the two members of the above inequality; then a
  neighbourhood $\sigma$ of $x_\circ$ such that on $E \cap \sigma,
  w(x) > \lambda v(x)$. We deduce $w \ge R^{E \cap \sigma}_v$. Hence
  $R^{E \cap \sigma}_v (x_\circ) \le \dfrac{w (x_\circ)}{\lambda} < v
  (x_\circ)$. 
\end{proof}
 
Conversely, suppose $R^{E \cap \sigma}_v ~ (x_\circ)$ for a
neighbourhood $\sigma$. Then there exists a superharmonic function
$w > 0$ such that $w \ge v$ on $E \cap \sigma$, and $w(x_\circ) < v
(x_\circ)$. Therefore if $x_\circ \in \bar{E}$, 
\begin{align*}
  & \lim. \inf. w(x) \ge v(x_\circ) > w(x_\circ).\\
  & x \in E, x \to x_\circ
\end{align*}

\begin{prop}[Application Proposition]\label{p4:chap9:sec38:prop25} % pro 25
  For an open set $\omega \subset \bar{\omega} \subset \Omega$, if $C
  \omega$ is thin at a boundary point $x_\circ \in \sigma \omega$, then
  $x_\circ$ is irregular for $\omega$.  
\end{prop}

\begin{proof} % pro
  $\{x_\circ\}$ is polar and $C \omega - \{x_\circ \} =  E$ is thin at $x_\circ$;
  for any function $v$ of Theorem \ref{p4:chap9:sec38:thm29}, and a
  neighbourhood $\sigma$ 
  of $x_\circ$, we have $\hat{R}^{C \omega \cap \sigma - \{x_\circ\}}_v <
  v(x_\circ)$. By Theorem \ref{p4:chap7:sec33:thm24} (corollary),
  $\hat{R}^{C \omega \cap 
    \sigma- \{x_\circ\}} = \hat{R}^{C \omega \cap \sigma}_v$;
  therefore $\hat{R}_v^{\omega \cap \sigma} (x_\circ) < v(x_\circ)$. Now
  from Theorem \ref{p4:chap6:sec29:thm21}, it\pageoriginale follows that $x_\circ$ is
  not a regular point for $\omega$. 
\end{proof} 
 
\section{Fine topology}\label{p4:chap9:sec39}%sec 39


\begin{thm}\label{p4:chap9:sec39:thm30} % the 30
  The complements of the sets $E \notni x_\circ$ that are thin at $x_\circ$,
  form for all $x_\circ \in \Omega$ the filter of neighbourhoods for a
  topology on $\Omega$. This topology is called the `fine topology' on
  $\Omega$. Among the topologies on the set $\Omega$ which are finer (in
  the large sense) than the topology of the given space $\Omega, \Phi$
  be the coarsest for which all superharmonic functions $v \ge 0$ are
  continuous. Then $\Phi $ and the fine topology are identical. 
\end{thm}

\begin{proof} % pro
  If $V$ is a $\Phi$-neighbourhood of a point $x_\circ$ in $\Omega$, it
  contains an open set (of the initial topology) or is the
  intersection of such an open set with some sets of type $\alpha =
  \big \{x : v < \beta \big \}$ where $v$ is a superharmonic function
  $\ge 0$ with $v(x_\circ ) < \beta $. On $C \alpha, v \ge \beta$; hence
  the sets $C \alpha$ and their union are thin at $x_\circ$, therefore
  also $CV$. 
\end{proof}

Conversely $V$ be a set such that $x_\circ \in V, CV$ is thin at $x_\circ$. If
$x_\circ$ is in the interior (in the initial sense) of $V, V$ is also a
$\Phi$-neighbourhood of $x_\circ$ if $x_\circ$ is not in the interior (in the
initial sense) of $V$, there exists a superharmonic function $v \ge 0$
such that 
$$
v(x_\circ) < \beta < \substack{\lim.\inf.\\ v x \to x_\circ, x \in CV}.
$$

For\pageoriginale a suitable ordinary neighbourhood  $\delta$ of $x_\circ, v \ge \beta$
on $CV \cap \delta$. Therefore, $\delta \cap \big \{ x : v < \beta
\big \} \cap V$ and $V$ is a $\Phi$ -neighbourhood of $x_\circ$. The
various topological notions that are associated with $\Phi$- topology
are called `fine' : fine closure, fine limit, etc.; (the  name was
given by Cartan in the classical case). 

\begin{prop}\label{p4:chap9:sec39:prop26} % pro 26
  Let $E$ be a closed set thin at $x_\circ \in \partial E$ and $\omega$ an
  open neighbourhood of $x_\circ$. For any superharmonic function $v \ge
  0$ in $\delta' = \omega \cap CE$ there exists a fine limit at
  $x_\circ$. 
\end{prop}

\begin{proof} % pro
  $\{x_\circ\}$ is polar and the proposition is obvious if $x_\circ$ is
  isolated on $E$ (see $n^\circ$. 32 iv). 
\end{proof}

If not choose a superharmonic function $w \ge 0$ on $\Omega$
(associated to $E$ and $x_\circ$), such that 
\begin{align*}
  & \lim. \inf. w - w(x_\circ) = d > 0\\
  & x \in E - \{x_\circ\}\\
  & x \to x_\circ
\end{align*}

On a fine neighbourhood $V$ of $x_\circ,  w \le w(x_\circ) + \varepsilon ~(0 <
\varepsilon < d)$. 

If $v(x) \to + \infty \, (x \in V \cap \delta',  x \to x_\circ ), v$
on $\omega'$ has the fine limit $+ \infty$ at $x_\circ$. If not, let
$\lambda = \underset{x \to x_\circ, x \in V \cap \delta'}{\lim. \inf. } v
(x) < + \infty$.  

Let us compare
\begin{gather*}
   {\substack{\lim. \inf.\\y \in \delta E - x_\circ\\y \to x_\circ}}
  {\substack{(\lim. \inf. (v + kw))\\x \to y, x \in \delta'}} \ge
  kw(x_\circ) + d = k_1\\  
  \substack{\lim. \inf. (v + kw)\\x \in \delta',  x \to x_\circ}  \le
\substack{\lim. \inf.(v + kw)\\ x \in \delta' \cap V \\ x \to x_\circ}  \le
\lambda + k(w (x_\circ) + \varepsilon) = k_2 
\end{gather*}

We\pageoriginale shall choose $k_1$ and $k_2$ such that $k_1 > k_2$, then a finite continuous
potential $V_\circ$ such that $k_1 > V_\circ (x_\circ) > k_2$. 

Now the function equal to $\inf. (v + kw, V_\circ)$ on $\delta'$, and to
$V_\circ$ on $E - \{x_\circ\}$ is superharmonic in a suitable open
neighbourhood $\delta_1 \subset \delta$ of $x_\circ$, outside $x_\circ$. This
function $V_1$ may be therefore continued at $x_\circ$ in order to become
superharmonic in $\delta_1$. Therefore on $\delta_1 - \{x_\circ\}, V_1$
has a finite limit (at $x_\circ$), equal to $\underset{x \neq x_\circ, x \to
  x_\circ}{\lim. \inf.} V_1 < V_\circ (x_\circ)$. We deduce that $v + kw $ on
$\delta'$ has the same fine limit at $x_\circ$. As $w$ has a fine limit,
the same holds for $v$. This theory may be developed further, as in
the classical case, using first only a countable base of open sets in
$\Omega$ (thanks to some results of R.M.Herve). We will only  give the
following important theorem, with more restrictions, using the
convergence theorem. 

\section{Further development}\label{p4:chap9:sec40}%sec 40

with axioms \ref{p4:chap1:sec1:axiom1}, \ref{p4:chap1:sec1:axiom2},
\ref{p4:chap1:sec1:axiom3} $D$ a countable base in $\Omega$
and a potential $> 0$. 

\begin{thm}\label{p4:chap9:sec40:thm31} % the 31
  The subset of a set $E$ where $E$ is thin is a polar set.
\end{thm} 
 
Let $V_\circ$ be a finite continuous potential $> 0$ and $\{\omega_i\} $
a countable base of $\Omega$. If $E$ is thin at $x_\circ \in E$, $\{x_\circ\}$
is polar, $E - \{x_\circ\}$ is thin and there exists an $\omega_i \ni x_\circ$
such that 
$$
\hat{R}^{E \cap \omega_i}_{V_\circ} (x_\circ) = \hat{R}^{E \cap \omega_i -
  \{x_\circ\}}_{V_\circ} (x_\circ) = R^{E \cap \omega_i - \{x_\circ\}}_{V_\circ} (x_\circ) <
V_\circ (x_\circ) 
$$

Hence $x_\circ \in \big \{x : \hat{R}^{E \cap \omega_i}_{V_\circ} \big \}$ the
intersection of this set with $E \cap \omega_i$\pageoriginale is polar
$(n^o. 37)$. Therefore the set of all $x_\circ$ is polar. 
\begin{coro*} % coro
  A polar set is characterized as a set $e$ which is thin at any point
  of $\Omega$ or at every point of $e$. 
\end{coro*} 
 
\noindent
\textbf{Application to the Dirichlet problem for } $\omega \subset
\bar{\omega} \subset \Omega$ 
 
\begin{thm}\label{p4:chap9:sec40:thm32} % the 32
  With the previous hypothesis, the regularity of a point $x_\circ \in
  \partial \omega$ is equivalent to the non-thinness of $C \omega$ at
  $x_\circ$. Then follow: (i) the set of the irregular boundary points is
  polar \footnote{when $\omega$ is connected, the hypothesis of a
    countable base of open sets is unnecessary (quite different proof by
    R. willin, see add. bibliography).}, (ii) any bounded harmonic
  function on $\omega$, which tends to 0 at every regular boundary
  point, is equal to zero, (iii) there exists for any finite
  continuous function $\theta$ on $\partial \omega$, a unique bounded
  harmonic function on $\omega$ which tends to $\theta (x)$ at every
  regular boundary point $x$ (it  is $H^\omega_\theta$).   
\end{thm}

\begin{proof} % pro
  If $\{x_\circ\}$ is not polar, $C \omega$ is not thin at $x_\circ$, further
  we have by using a finite continuous superharmonic function $V > 0$,
  for every neighbourhood $\sigma$ of $x_\circ$, 
  $$
  \hat{R}^{C \omega \cap \sigma}_V (x_\circ) = R_v^{\sigma \cap C \omega}
  (x_\circ) = V (x_\circ) 
  $$
  Therefore $x_\circ$ is regular (Theorem \ref{p4:chap6:sec29:thm21}).

  If $\{x_\circ\}$ is polar,
  $$
  \hat{R}^{c \omega \cap \sigma}_V (x_\circ) = \hat{R}^{c \sigma \cap
    \omega - \{x_\circ\}}_V (x_\circ) = \hat{R}^{c \sigma \cap \omega -
    \{x_\circ\}}_V  
  $$
\end{proof} 

The equality of these members with $V (x_\circ)$ for every $\sigma$
implies (Theorems \ref{p4:chap6:sec29:thm21}
and \ref{p4:chap9:sec38:thm29}) both the regularity of $x_\circ$ and the
non-thinness of $C \omega$ at $x_\circ$. 

\begin{remark*} % Rem
  Theorem\pageoriginale \ref{p4:chap9:sec40:thm32} gives therefore, a
  criterion of non-thinness of a 
  closed set $E$ at a point $x_\circ \in \partial E$. It is interesting to
  prove it directly. (See bibliography)  
\end{remark*}

We recall now that a subset $e$ of the boundary of $\omega$ is defined
to be a negligible set if $\bar{H}^\omega_{\varphi_e} = 0$.
$(\varphi_ \varepsilon$ characteristic function of $e$), or as a set
of $d \mu^\omega_x$- measure zero for any $x \in \omega$. 

\begin{lemma*} % lem
  If for an open set $\delta, \delta \cap \omega = \delta \cap
  \omega_1 ~ (\omega$ and $\omega_1$ being two relatively compact open
  sets) the negligible sets of $\partial \omega \cap \delta$ or
  $\partial \omega_1 \cap \delta$ are the same relative to $\omega$ or
  $\omega_1$. 
\end{lemma*}

Consider the non-trivial case, $\omega \subset \omega_1$ and $e
\subset \partial \omega \cap \delta$ negligible for $\omega$. Then
$\bar{H}^{\omega_1}_{\varphi_e} = \bar{H}^\omega_\psi$ on $\omega$ for
a suitable $\psi$ equal to $\varphi_e$ on $\partial \omega \cap
\delta$. Now if we change $\psi$ into zero  on $e,
\bar{H}^\omega_\psi$ does not change. We deduce that
$\bar{H}^{\omega_i}_{\varphi_e}$ tends to zero at every regular point
of the boundary of $\omega_1$. Hence $\bar{H}^{\omega_1}_{\varphi_e}=
0$. 

\begin{thm}\label{p4:chap9:sec40:thm33} % the 33
  With the same hypothesis, the boundary points of $\omega \subset
  \bar{\omega}\subset \Omega$ where $\omega$ is thin form a negligible
  set (for $\omega)$. 
\end{thm}

\begin{proof} % pro
  Let us use a finite continuous superharmonic function $V > 0$.
\end{proof}

If $\omega$ is thin at $x_\circ$, there exists an open neighbourhood
$\sigma$ containing $x_\circ$, such that the potential $V_1$ defined by
$\hat{R}_V^{\omega \cap \sigma}$ or $R_V^{\omega \cap \sigma}$ is
smaller than $V (x_\circ)$ at the point $x_\circ$. Now on $\omega \cap \sigma\,
V$ and $V_1$ are equal, have the same greatest harmonic minorant
therefore (by axiom $D$) the same best harmonic minorant, i.e., 
$$
H^{\omega \cap \sigma}_V = H^{\omega \cap \sigma}_{V_1} \text {   or
} H^{\omega \cap \sigma}_{V- V_1} = 0 
$$

Hence\pageoriginale the set $\big \{x \in \partial \omega \cap \sigma  : V_1 (x) <
V(x) \big \}$ is negligible for $\omega \cap \sigma$, and therefore
for $\omega$. Now we can cover $\partial \omega$ by finite number of
such $\sigma$, hence we deduce that the subset of $\partial \omega$
where $\omega$ is thin is negligible for $\omega$. 
