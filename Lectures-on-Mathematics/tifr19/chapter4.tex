\part{Potentials with Kernels - Convergence Theorems}\label{p3}%part III

\setcounter{chapter}{0}
\chapter{Preliminaries on Measures, Kernels and
  Potentiels}\label{p3:chap1}%chap I 

\section{Radon Measures}\label{p3:chap1:sec1} %sec 1

Let\pageoriginale us recall some definitions and properties of Radon measures on a
locally compact Hausdorff space $E$. For the proofs we refer to
N. Bourbaki (Integration). 

Let $\mathcal{K}(E)$ be the real vector space of real valued (finite)
continuous functions on $E$ with compact support (closure of the set
where the function is $\neq 0$). $\mathcal{K}(E, K)$ stands for the
subspace of $\mathcal{K}(E)$ the elements of which have support in the
compact set $K$. The space $\mathcal{K}(E, K)$ is provided with the
topology of uniform convergence on $K$. 

We shall consider (unless the contrary is mentioned ) only
\textit{positive measures} called briefly measures; we recall their
definition. 

\setcounter{defn}{0}
\begin{defn}\label{p3:chap1:sec1:def1}%defi 1
  A positive Radon Measures on $E$ is a positive linear functional  on
  $\mathcal{K}(E)$. 
\end{defn}

A linear combination of measures $\alpha \mu_1 + \beta \mu_2$ is the
functional $\alpha \mu_1\break (f) + \beta \mu_2 (f)$. The order of two
measures is defined by $\mu_1 \le \mu_2 \Longleftrightarrow \mu_2 =
\mu_1 + $ positive measure. 

For any $f$ in $\mathcal{K}(E)$, the value $\mu (f)$ of the functional
is called the integral of $f$ and is also denoted by $\int f d
\mu$. This positive\pageoriginale linear functional is obviously continuous on every
$\mathcal{K} (E, K)$. 

The set $\mathfrak{m}^+$ of the positive Radon Measures is
\textit{provided} with the coarsest topology such that, for every
function $f$ in $\mathcal{K}(E)$, the mapping $\mu \to \int f d \mu$ is
continuous. This topology is called the ``topology of vague convergence
'' or simply ``\textit{ vague topology}''. This may otherwise be
described as the topology of simple convergence on $\mathfrak{m}^+$. 

\setcounter{prop}{0}
\begin{prop}\label{p3:chap1:sec1:prop1}%prop 1
  The set $E^1$ of unit measures (or Dirac measures) $\varepsilon _x$
  at the points $x$ in $E$ is a subset of $\mathfrak{m}^{+}$ which is
  homeomorphic to $E$. 
\end{prop}

\begin{prop}[Compactness]\label{p3:chap1:sec1:prop2}%prop 2
  A set of $\mu_i \in \mathfrak{m}^{+}$ such that for any $f \ge 0$ belonging to
  $\mathcal{K}(E), \int f d \mu_i$, is bounded, is relatively compact
  in $\mathfrak{m}^{+}$ for vague topology. 
\end{prop}

\begin{prop}\label{p3:chap1:sec1:prop3}%prop 3
  The function $\int f d \mu$ is continuous on $\mathcal{K}(E, K)
  \times \mathfrak{m}^{+}$ with the product topology. 
\end{prop}

Consider an element $(f_o, \mu_o)$ in the product space. Let $\varphi$
be in $\mathcal{K}(E)$ such that $\varphi = 1$ on $K$. Then, 
$$
\bigg| \int f_o d \mu_o - \int f d \mu \bigg| \le \bigg| \int f_o d
\mu_o - \int f_o d \mu \bigg| + \int f_o d \mu - \int f d \mu \bigg| 
$$

Given $\varepsilon > 0$, there exists a neighbourhood $V$ of $f_o$ in
$\mathcal{K}(E, K)$ where $|f - f_o | <^{'} \varepsilon \varphi$ and a
neighbourhood $W$ of $\mu_o n \mathfrak{m}^{+}$ where $\int \varphi d \mu \le
\int \varphi  d \mu_o + \varepsilon$ and $|\int f_o d \mu_o - \int
f_o d \mu | < \varepsilon$. Hence $ f \in V, \mu \in W$ implies $|
\int f_o d \mu_o - \int f d \mu | < \varepsilon$. The proposition is
an immediate consequence. 

\section{Radon Integrals of functions (for positive
  measures)}\label{p3:chap1:sec2} %sec 2

Let\pageoriginale $\psi \ge 0$ be a lower semi-continuous function defined on. $E$
The integral of $\chi$ with respect to the Radon measure $\mu$ is by
definition 
$$
\int \chi d \mu = \sup \int f d \mu \,(f \in \mathcal{K}(E), f \le \psi) 
$$

An important property is for any increasing directed set of such $\psi$
$$
\sup \int \psi_i d \mu = \int \sup \psi_i d \mu
$$

For any finite upper semi-continuous function $\varphi \ge 0$, with
$\varphi = 0$ outside a compact set, the integral is defined to be 
$$
\int \varphi d \mu = \Inf . \int f d \mu \,(f \in \mathcal{K} (E), f \ge
\varphi) 
$$

We now define for any valued function $g \ge 0$ on $E$ the upper and
lower integrals denoted respectively by $\bar{\int} _g  d \mu$ and
$\underline{\int} g d \mu$ 

\begin{tabular}{rcp{8cm}}
   $\bar{\int } g d \mu$ & = & $\inf \int _\chi  d \mu \,( \psi \text{
    lower semi-continuous}, \psi \ge g)$\\ 
  $\underline{\int } g  d \mu$ & = & $\sup \int \varphi d \mu \,( \varphi 
  \text {finite upper semi-continuous}, 0 \le \varphi \le g, \varphi =
  0 \text{ outside a compact set})$ 
\end{tabular}

It is easily seen that $\underline{\int} g d \mu \le \bar{\int} g d \mu$

Recall the fundamental property for a sequence $g_n \ge 0$
$$
\bar{\int} \lim \inf g_n d \mu \le \lim \inf \bar{\int} g_n d \mu
~\text{(Fatou's Lemma)} 
$$

\begin{defn}\label{p3:chap1:sec2:def2} %% 2
  A\pageoriginale function $g \ge 0$ is said to be $\mu-$ integrable (resp. in the
  large sense) if the upper and the lower $\mu-$ integrals of $g$ are
  equal and finite (resp. only equal); the common value is written as
  $\int g d \mu$ and it is by definition the $\mu$ integral of $g$
  (finite or not). 

  For any function $g$, utilizing the classical decomposition $g = g^+
  - g^-$, we define $\int g d \mu = \int g^+ d \mu - \int g^- d \mu$
  whenever the difference and its terms have meaning $(g$ is said to
  be $\mu$ - integrable). 
\end{defn}

\noindent
\textbf{Permutation of Integration}. We shall use the particular case
of a 	lower semi-continuous function $\psi (x, y ) \ge 0$ on $E \times
E$ for which 
$$
\int d \nu (y) \int \psi (x, y) d \mu (x) = \int \mu (x) \int \psi (x,
y) d \nu (y) 
$$

\noindent
\textbf{Inner and Outer measure - measurablility}. The inner and outer
measures of a set $\alpha$ are defined to be the lower and upper
integrals (respectively) of the characteristic function
$\varphi_\alpha$ of $\varphi$. If the two integrals are equal and
finite, $\alpha$ is said to be $\mu$- integrable, of measure $\mu
(\alpha)$, the common value. Then the intersection with any compact
set is also $\mu-$ integrable and $\mu (\alpha) = \sup. \mu (K)$ for
all compact sets $K$ contained in $\alpha$. Note that an open set of
measure zero is more directly characterised by the property that $\int
f d \mu = 0$ for any $f \in \mathcal{K} (E)$ with support in
$\omega$. The support $S_\mu$ of a measure $\mu$ is defined to be the
closed set complement of the largest open set whose measure is
zero. We say also that a measure $\mu$ is supported by a set $\alpha$
(which may not be closed) if the outer measure of $C \alpha$ is zero. 

A\pageoriginale set $\alpha$ is $\mu-$ measurable if $\alpha \cap K$ is $\mu-$
integrable for every compact set $K$. A set, $\mu-$ measurable for
every $\mu$ is defined to be a measurable set [Examples : closed and
  open sets]. 

A real valued function $f$ may be defined as $\mu$ measurable if the
sets where $f \ge \alpha, f \le \alpha$ are $\mu-$
measurable. [Examples : semi - continuous functions ]. A $\mu$
measurable function is $\mu-$ integrable if and only if $\bar{\int}
|f| d \mu < + \infty$. 

We recall the \text{Lusin's property} of $\mu$ -measurable functions
(taken as definition by Bourbaki, Integration Ch. $IV$) : for any
compact set $K$ and any $\varepsilon > 0$, there exists a compact set
$K_1 \subset K$ such that $\mu (K -K_1) \le \varepsilon$ and that the
restriction of $f$ on $K_1$ is continuous. 

\noindent
\textbf{Restriction of Measure.} The restriction of a measure $\mu $
on a $\mu-$ measurable set $\alpha$ may be defined as the measure
$\mu_\alpha$ determined by the functional $\int f \varphi_{\alpha} d
\mu \,[f \in \mathcal{K} (E), \varphi_\alpha]$ being the
characteristic function of $\alpha. f \varphi_\alpha$ is $\mu-$
integrable]. For any $\mu- $ integrable function $f$ it is seen that
  $f \varphi_\alpha$ is $\mu-$ integrable, that $f$ is $\mu_\alpha$
  integrable and that $\int f \varphi_\alpha d \mu$ denoted also by
  $\int_\alpha f d \mu$ is equal to $\int f d \mu_\alpha$. Note that
  when $\mu (C \alpha) = 0, \mu_\alpha \le \mu$. 

Suppose now that $\alpha$ is a compact set and $\mu$ a measure with
support contained in $\alpha$. We may associate a measure $\mu'$ on
the subspace $\alpha$ such that if $f$ is finite and continuous on
$\alpha$, and $F$ its continuation by zero, $\mu' (f) = \int F d
\mu$. Conversely, for any measure $\mu'$ on $\alpha$, there exists a
unique measure $\mu$ on $E$ such that $\mu (C \alpha) = 0$ and its
associated measure (by the above method) is $\mu'$ on $\alpha$. 

\noindent
For\pageoriginale such $\mu$ and $\mu'$ and any function $f$ on $\alpha$ (continued
arbitrarily to $E$) both $d \mu$ and $d \mu'$ integrals exist and are
equal. Hence we do not distinguish between $\mu$ and $\mu'$ and refer
to them as measure on $\alpha$. 

\section{Kernels and Potentials}\label{p3:chap1:sec3}%sec 3

A real valued function $G(x, y) \ge 0$ on $E \times E$, integrable in
$y$ in the large sense for every $x$ as regards and Radon measure
$\geq 0$ on $E$ is called a kernel on $E$. 

\begin{defn}\label{p3:chap1:sec3:def3}%defi 3
  The potential of a measure $\mu $ in $\mathfrak{m}^+$ with respect to the
  kernel $G(x, y)$ is defined as $G \mu (x) = \int G(x, y) d \mu
  (y)$. 
\end{defn}

In order to develop a large theory of potentials, it is easier to
assume that the function $G \mu (x)$ on $E$ is lower semi-continuous
for every $\mu$. We shall even suppose more. 

\setcounter{thm}{0}
\begin{thm}\label{p3:chap1:sec3:thm1}%the 1
  The potential $G \mu (x)$ is lower semi-continuous on $E \times
  \mathfrak{m}^+$ if and only if the kernel $G (x, y)$ is lower
  semi-continuous on $E \times E$. 
\end{thm}

\begin{proof}
  $E$ is homeomorphic to the subset $E^1$ of $\mathfrak{m}^+$ formed
  by the Dirac
  measures at the points of $E$. The potential for the Dirac measure
  $\varepsilon_y$ at $y$ in $E$ is $G(x, y)$. Hence the necessary part
  follows. 
\end{proof}

Conversely, the lower semi-continuous function $G(x, y)$ is the upper
envelope of an increasing directed (filtrante) family $\{G_i (x,
y)\}_{i \in I}$ of finite continuous function on $E \times E$ with
compact supports. Now $y \to G_i (x, y)$ is an element of some
$\mathcal{K}(E, K)$ which is a continuous function of $x$. Hence by
Proposition \ref{p3:chap1:sec1:prop3}, $\int G_i (x, y) d \mu (y)$ is
a finite continuous function on $E \times \mathfrak{m}^{+}$. 

The\pageoriginale limit $\int G(x, y) d \mu (y)$ according to the directed set is
also the upper envelop which is lower semi-continuous in $E \times
\mathfrak{m}^+$. 

We shall only use from now lower semi-continuous kernels.

Let us introduce a fundamental tool in our theory of potentials.

\setcounter{Lemma}{0}
\begin{Lemma}\label{p3:chap1:sec3:lem1}%lemma 1
  Let $G(x, y)$ be a lower semi-continuous kernel on $E$. Let $\mu$ be
  a Radon Measures with compact support $K$ in $E$ and $G \mu (x)$
  finite on $K$. Then for any $\varepsilon > 0$, there exists a
  compact set $K' \subset K$ such that the restriction $\mu'$ of $\mu$
  to $K'$ satisfies:  
  $$
  \displaylines{\text{\rm (i)}\hfill \int d \mu - \int d \mu' <
    \varepsilon\hfill } 
  $$
  and (ii) the restriction of $G \mu' (x)$ to $K'$ is continuous.
\end{Lemma}

\begin{proof}
  $G \mu (x)$ being lower semi - continuous, the application of
  Lusin's property $(\S 2)$ provides a compact set $K'$ contained in
  $K$ such that if $\mu'$ is the restriction of $\mu$ to $K'$ then $0
  \le \int d \mu -\int d \mu' < \varepsilon$ and $G \mu (x)$
  restricted to $K'$ is continuous. Now $\mu'' = \mu - \mu'$ being a
  positive Radon measure, $G \mu''(x)$ is lower semi-continuous on
  $E$ and further, 
  $$
  G \mu (x) = G \mu' (x) + G \mu'' (x).
  $$

  The above being a decomposition of the continuous function $G \mu (x)$
  into sum of two lower semi-continuous functions, the individual
  members of the right hand side are themselves continuous. This
  completes the proof. 
\end{proof}

The above lemma will be useful in the development of the convergence
theorems. This lemma enables us to discard a set of small $\mu-$\pageoriginale
measure, and consider continuous functions in the complement, instead
of lower semi-continuous functions. Here we shall prefer this approach
to the other ones of taking limits of continuous potentials. 
