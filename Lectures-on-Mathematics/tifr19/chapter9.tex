\chapter{An Application to The Balayage
  Principle\texorpdfstring{${}^1$}{1}}\label{p3:chap6} %chapter  VI 
\footnotetext[1]{Similar but perhaps more interesting
    developments were given later in a note ``Remarks sur le
    balayage''. Bull. Soc. royale des. Sc. Liege, $ 30^e$ ame. 1961,
    $p.210$.}

\setcounter{section}{14}
\section{}\label{p3:chap6:sec15} % section 15

\begin{defn}\label{p3:chap6:sec15:def18}% definition 18
  Let\pageoriginale $\alpha$ be any subset of a locally compact space $E$. Let $G$
  be a kernel on $E$ and $G \mu $ a potential. For any function
  $\varphi$ on $\alpha$  such that  $ 0 \leq \varphi \leq G \mu $, let
  $ R^\alpha_{\varphi} $ denote the lower envelope in $E$ of all
  potentials $ G \mu_i $ which majorise $\varphi$ on $\alpha$. 
\end{defn}

It is immediate that $R^\alpha_{G \mu} \leq G \mu$ everywhere and
$R^\alpha_{G \mu} = G \mu $ on $\alpha$. 

\begin{defn}\label{p3:chap6:sec15:def19} % definition 19
  $G$ is  said to satisfy the principle of lower envelope if  the
  infimum of two arbitrary potentials  is also a potential.  
\end{defn}

\begin{thm}\label{p3:chap6:sec15:thm14} % theorem 14
  Suppose
  \begin{enumerate}[\rm (i)]
  \item $E$ is a  compact space with a countable base for open sets.
  \item $G$ satisfies principle of lower envelope
  \item $G^*$ is regular 
  \item there exists a $G^*$-potential $ > 0 $ and bounded (for
    instance, this is satisfied if $G^* >0$ and $E$ is not
    $G^*$-negligible.) then 
    \begin{enumerate}[a)]
    \item $ R^\alpha_\varphi \big [ \alpha, \varphi, \mu $ as  in
      Def. \ref{p3:chap6:sec15:def18} $\big ] $ and  
    \item if in addition $G$ is regular and finite and continuous in
      the complement of the diagonal in $ E \times E $, then $
      R^\alpha_\varphi $ is equal to\pageoriginale a potential (which is again $
      \leq R^\alpha_\varphi$) $G^*$-quasi-everywhere. 
    \end{enumerate}
  \end{enumerate} 
\end{thm}

\begin{proof}
  Consider the family $\big \{\mu_i \big \} $ of measures such that $G
  \mu_i \geq \varphi$ on $\alpha$; this family is non-empty by
  hypothesis. The lower envelope of  $G \mu_i$ is not changed by
  keeping only those $\mu_i$ such that $ G \mu_i \leq G \mu $ ( $ \mu
  $ is in the family ) and that we shall suppose now. 
\end{proof}


The $G \mu_i$ form a directed set for the natural decreasing order (by
principle of lower envelope ). Let $ G^* \nu $ be a potential
satisfying $ 0 < G^* \nu < L $ ($ L$ finite). From the inequality  $ G
\mu_i \leq G \mu $ we get $ \int G^* \nu d \mu_i \leq \int G^* \nu d
\mu $ and therefore  
$$
(\inf. G^* \nu ) \mu_i (E) \leq L \mu (E). 
$$
This shows that $\big \{ \mu_i (E) \big \}$ is bounded. Applying  the
convergence theorems 5 (13), we deduce immediately $(a)$ [$
  \resp (b)$] 

\section{}\label{p3:chap6:sec16}% section 16.

With some more restriction on the kernel. we shall deduce a strong
principle of balayage (sweeping out process) from a weaker one.  

\begin{defn}\label{p3:chap6:sec16:def20} % Definition 20
  If for two measures $ \mu_1$ and $ \mu_2 $, $ G \mu_1 = G \mu_2$
  $G^*$-quasi everywhere implies $ \mu_1 = \mu_2,  G$ is said  to
  satisfy the {\em principle of uniqueness.} 
\end{defn}

If $G$ satisfies principles of uniqueness and lower envelope then\break $``G
\mu_1 \leq G \mu_2$ $G^* $-quasi everywhere ``implies'' $G \mu_1 \leq G
\mu_2 $ everywhere. 

\begin{defn}\label{p3:chap6:sec16:def21} % definition 21
  {\em (Weak Principle of balayage)}. It is the hypothesis that there
  exists a base $ \big \{ \omega_i \big \} $ of open sets with the
  following property: for\pageoriginale any measure $\mu$ on may compact set $ K
  \subset \omega_i $, there exists another measure $ \mu'$ such that $
  \mu' (\omega_i) = 0,   G \mu' \leq G \mu $ everywhere and $ G \mu'
  = G \mu $ on $ C \omega_i $. 
\end{defn}

\begin{thm}\label{p3:chap6:sec16:thm15} % theorem 15.
  In addition to the hypotheses of Theorem \ref{p3:chap6:sec15:thm14}
  (b) let $G$ satisfy the 
  principle of uniqueness. Given any set $\alpha$ and a measure $\mu$
  there is a {\em unique} measure $\mu_0$ such that $ R^\alpha_{G
    \mu} = G \mu_0 G^* $-quasi everywhere. This measure $\mu_0$ is
  such that $ G \mu_0 \leq G \mu $ everywhere and $ G \mu_0 = G \mu $
  on $\alpha$ $G^*$-quasi everywhere; and further if $G$ satisfies
  the weak balayage principle then  $ \mu_0 (C \bar{\alpha}) = 0 $
  (i.e.  support of $ \mu_0 \subset \bar{\alpha}$). 
\end{thm}

\begin{proof}
  We start with measures  $\mu_i$ such that $G \mu_i \leq G \mu,   G
  \mu_i = G \mu$ on $\alpha$ and arrive at a measure $\mu_0$ such
  that $ G \mu_0 = R^\alpha_{R \mu} G^* $-quasi everywhere; and this
  last property determines the measure $\mu_0$ uniquely. 
\end{proof}

Now assume that $G$ satisfies the weak balayage principle. In order to
prove that the support of $\mu_0 $ is contained in $\bar{\alpha}$  we
shall realise $\mu_0$ as the vague limit of a sequence of measures
whose supports are outside some neighbourhood of  (arbitrary)  point $
x \in C \bar{\alpha} $. Let $V_x$ be a neighbourhood of $x$ disjoint
from an open set containing $\bar{\alpha}$. There exists $W$ of the
given base (of open sets) with $ x \in ~ W \subset V_x $. Let $K$ be
a compact neighbourhood of $x$ contained in $W$. Now $\mu_i =
(\mu_i)_K + (\mu_i)_{CK}$. By the weak balayage principle, there exists
a measure $\mu_i'$ on $ C W $ such that  $ G \mu'_i \leq G (\mu_i)_K $
everywhere with equality holding on $ C \bar{W}$. This shows the
measures of the form $ \mu''_i = \mu'_i + (\mu_i)_{CK} $ is a subfamily
of $ \big \{ \mu_i \big \} $ and inf. $ G \mu'' = \inf.  G \mu_i =
R^\alpha_{G \mu}$. 

Consider\pageoriginale $\mu''_{i_{1}}$ and $\mu''_{i_{2}}. \inf ( G \mu''_{i_{1}}, G
\mu''_{i_{2}} )$ is a potential $ G \mu_{i_{3}} $. By taking the
corresponding $\mu''_{i_{3}}$, we see that $G \mu''_{i_{3}} \leq \inf. 
( G \mu''_{i_{2}},  G \mu''_{i_{2}} )$. We conclude (by
Theorem \ref{p3:chap5:sec14:thm13})
that the  lower envelope $ R^\alpha_{G \mu} $ of the family $G \mu''_i
$ is equal $G^*$-quasi everywhere to a potential $ G \mu_1 $ where $
\mu_1 $ is the vague limit of a suitable sequence from $\mu''_i $. If $
\overset{\circ}{K} $ is  the interior of $K$, since $ \mu''_i (
\overset{\circ}{K} )  = 0 $ for every $ i$, $ \mu_1 ( \overset{\circ}{K}) = 0
$. Now $ G \mu_1 = G \mu_0 $ $G^*$-quasi everywhere. Now it is
immediately seen that the support of $ \mu_0 $ is contained in $
\bar{\alpha}$.  

\begin{remark*}
  With various restrictions on $ \alpha,  \mu, G $ we could study the
  case of a locally compact space. It would be interesting to find
  conditions to give exceptional sets as  polar ones.  
\end{remark*}

We mention that all the principles that we studied in this part are
satisfied in the classical case. (Green's kernel in Green's space, for
example in a bounded euclidean domain:  see \cite{5}. The relations
between such principles are now being discussed \cite{14} ); a good base is
the complete study of a space containing a finite  number of points
\cite{10}. 

\begin{thebibliography}{99}
\bibitem{p3key1}{Anger}\pageoriginale- These Dresden 1957
\bibitem{p3key2}{Anger}- Habiletatiouschift, Rostock 1966
\bibitem{p3key3}{M.Brelot}- C.R. Acad. Sci. Paris  Oct. 1938
\bibitem{p3key4}{M.Brelot}- Annales de l'Institut Fourier $t6$ ( 1955 -56 ) 
\bibitem{p3key5}{M.Brelot} - Elements de la theorie classique du Potential
  (CDU, Paris, 1959) $3^d$ ed. 1965 
\bibitem{p3key6}{M.Brelot and G. Choquet} - Journal of Madras  Univ. 1957
\bibitem{p3key7}{H.Cartan} - Bull. Soc. Math. de France 1945, $p. 74 $

\bibitem{p3key8}{G.Choquet} - C.R. Acad. Sci. Paris t. 243 ('56)
  p.635-638   t. 244 (1957) p.  1606 - 1909  
\bibitem{p3key9}{G.Choquet} - Seminaire de theorie du Potentiel Paris
  1957 - 58 -59  
\bibitem{p3key10}{G.Choquet and J.Deny} - ``Modeles finis en Theorie du
  Potentiel'' Jounral d'Analyse Math. (Jersualem) 1956/57 
\bibitem{p3key11}{O.Frostman} - Thesis, Lund 1935
\bibitem{p3key12}{M.Kishi} - ``On a Theorem of Ugaheri'' Proc. Jap
  Aca. 32 (1956)  p.314  
\bibitem{p3key13}{Ninomiy} - Journ. Polytech. Osaka City Univ. 5 ( 1954 ) 
\bibitem{p3key14}{Ohtsuka} - Proc. Jap. Aca. 33  (1957)  p.37
\bibitem{p3key15}{Durier} - Sem. Pot   9 (64-65)  and  10 (65-66) 
\bibitem{p3key16}{Kishi} - Nagoy Math. J. 23 (1963) J.of Hiroshim
  Univ. Sect. A 28, 1964 Ann.  I.F . 14/2, 1964, 17/1, 1967 
\bibitem{p3key17}{Mokobodski} - Sem. Choquet 1, 1962
\bibitem{p3key18}{Meinhold} - Positive Linear formen in der Potential
  theorie (Diss Dresden 1959)   
\end{thebibliography}
