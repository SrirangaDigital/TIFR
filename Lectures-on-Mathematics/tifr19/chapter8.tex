
\chapter{Second Group of Convergence Theorems}\label{p3:chap5} % chapter V

\setcounter{section}{12}
\section{}\label{p3:chap5:sec13} %section 13.

We\pageoriginale start with a general kernel $G$ on a locally compact
space $E$. We 
shall introduce more restrictions than in Chapter \ref{p3:chap3} to get better
precision. 

\medskip
\noindent
\textbf{\textit{Reduction Method of Choquet}}

We shall prove an analogue of Lusin's property. The device  involved
is an important one. This concerns in throwing out an open set of
small capacity, in the complement of which the restriction of the
potential due to a measure is finite and continuous. 

\noindent 
\textbf{Decomposition lemma (5).}
Let $E$ be a  locally compact space, $\mu$ a positive measure on a
compact set $K$ and $G$ a regular kernel finite and continuous in the
complement of the diagonal in $ E \times E $. Then for any numbers  $
\varepsilon > 0 $ and $ \eta > 0 $, $\mu$ can be expressed as the sum
of two measures $ \pi \geq 0 $ and $ \nu \geq 0 $ such that  
\begin{enumerate}[(i)]
\item $ \pi (E) < \eta $
\item there exists an open set $\omega$ such that the inner $G^*$-cap
  $ (\omega) < \varepsilon $ and that the restriction of $C \nu$ to $C
  \omega $ is continuous and  $ \leq \dfrac{\mu (E)}{\varepsilon} $. 
\end{enumerate} 

\begin{proof}
  Let $\omega$ be the set of points $x$ where $ G \mu (x) > \dfrac{\mu
    (E)}{\varepsilon} $; $ G^*$-cap$ ( \omega ) \leq \varepsilon $ 
  (Prop \ref{p3:chap4:sec9:prop8}). There exists  a compact set $ K_1
  \subset \omega \cap K $ 
  such that $ \mu ( \omega ) - \mu (K_1) < \eta /2 $. On the other
  hand since $G \mu$ is  bounded\pageoriginale on $ C \omega $ there exists a
  measure  $\mu' \leq \mu$, with compact support  $K_2$ contained in $
  C \omega $ such that  $ G \mu' $ is finite and continuous every
  where and  $\mu(E)-\mu'(E) < \eta / 2$.  
\end{proof}

Now $\nu = \mu' + \mu_{K_{1}} \leq \mu $ and $ \pi = \mu \dots \nu $
fulfill the requirements of the lemma. 

\noindent 
\textbf{Fundamental lemma (6).} % fundamental lemma 6
The hypothesis being the  same on $E$, $G$ and $\mu$ (as in the
previous lemma)  for any $a > 0$, it is possible to find an open set
$\omega$ of $G^*-\text{ cap}. <$ a and such that $G \mu$ restricted to $C
\omega$ is  bounded and  continuous. 

\begin{proof}
  Let $\varepsilon_n > 0$, and $\eta_n > 0$ be two sequences  of
  numbers such that $ \sum \varepsilon_n <  a,  \eta_n \to 0 $ and $
  \sum \dfrac{\eta_{n-1}}{\varepsilon n} < + \infty$. Applying the
  decomposition lemma to $\mu$ we get two measures $ \pi_1 $ and
  $\nu_1$ and open set $\omega_1$ such that $\pi_1 (E) < \eta_1,  \mu
  = \nu_1 + \pi_1,  G^*$-cap. $ (\omega_1) < \varepsilon_1$ and the
  restriction of $G \nu_1$ to $ C \omega_1$  is less than $\mu (E) /
  \varepsilon_1$ and continuous. Taking the decomposition of $\pi_1 $,
  we get $\nu_2, \pi_2, \omega_2 $ satisfying  $ \pi_1 = \nu_2 + \pi_2
  \pi_2 (E) < \eta_2,  G^*-cap  (\omega_2 ) < \varepsilon_2
  $. Repetition of the process gives at the $ n^{th} $ stage 
  $$
  \pi_n =\nu_{n+1} + \pi_{n+1} \pi_{n+1} (E) < \eta_{n+1}, G^* -
  \text{ cap}. (\omega_{n+1} ) < \eta_{n+1} 
  $$
  and $G \nu_{n+1}$ restricted to $C \omega_{n+1} $ is continuous and
  $ < \dfrac{\pi_n (E)}{\varepsilon_{n+1}} <
  \dfrac{\eta_n}{\varepsilon_{n+1}}$ we assert that  $ \omega =
  \bigcup \omega_i $ answers to our need. Firstly 
  $$
  G^* - \text{cap}. \left( \bigcup \omega_i \right) \leq \sum G^*
  -\text{cap}. ( \omega_i ) < \sum \varepsilon_n < a.  
  $$
\end{proof}

And\pageoriginale $ \sum \limits^{n}_{1} \nu_p \to \mu $ in the strong sense $ \big[
  ( \mu - \sum \limits^{n}_{1} \nu_p ) (E) \to 0 \big ] $, hence as in
proposition 4, $ G \sum \nu_n \to G \mu $. Finally, because $ G \mu_n
\leq \dfrac{\eta_n}{\varepsilon_{n+1}} $ on  $ C \omega_n \supset C
\omega;  \sum \limits^{n}_{1} G \nu_p $ converges uniformly and is
bounded on $ C \omega $; the sum is bounded and continuous on $C
\omega$. 

\section{Convergence Theorems}\label{p3:chap5:sec14} % section 14.

\begin{defn}\label{p3:chap5:sec14:def17} % definition 17
  A property is said to hold $G$-quasi everywhere if the exceptional
  set where in the property does not hold good is of  outer
  $G$-cap. zero. 
\end{defn}

\begin{thm}\label{p3:chap5:sec14:thm11}%theorem 11.
  Let $E$ be a locally compact space, $G$ a kernel finite continuous
  in the complement of the diagonal in $ E \times E $ and further $G$
  and $G^*$ be regular. Let $\mu_n$ be a sequence of positive measures
  on a compact set $K$ tending to a measure $\mu$  vaguely. Then $\lim
  \inf.  G \mu_n= G\mu$, $ G^*$-quasi everywhere.  
\end{thm}

\begin{proof}
  We may, find for any real number $a > 0$, open sets $ \omega,
  \omega_n $ such that $G^*$-cap. $ (\omega) < \dfrac{a}{2}, 
  G^*-$-cap. $ (\omega_n) < \dfrac{a}{2^{n+1}}$ and the  potentials $
  G \mu, G \mu_n $ are finite and continuous respectively on $ C
  \omega, C \omega_n $. If  $ \Omega = \bigcup \limits_{n} \omega_n
  \cup \omega $, we have $ G^*$-cap. $ (\omega) < a,  G \mu_n,  G \mu$
  are finite and continuous on $C \Omega$ 
\end{proof}

Because of the hypothesis of the continuity of $G$, at any point of $C
K, G \mu_n \to G \mu$. Moreover, the lower semi-continuity implies
\break $
\lim._n \inf. G \mu_n \geq G \mu $ everywhere. The points where the
strict inequality  $( \lim_n. \inf G \mu_n > G \mu ) $ holds good are
characterised by the property that  there\pageoriginale exists integers $p$ and $q$
such that $G \mu_n \geq G \mu + \dfrac{1}{q} $ for $ n \geq p $. In
other words the set of  ``exceptional points" is  
\begin{multline*}
  \left\{ x : \lim_n. \inf. G \mu_n (x) > G \mu (x) \right\}\\ 
  = \bigcup_{p, q \in {\mathbb{Z}^+}} \left\{ x : G \mu_n (x) \geq G \mu (x) +
  \frac{1}{q} \text{ for  } ~ n \geq p \right \}. 
\end{multline*}

Let  $A_{p,q} = \big \{ x : G \mu_n (x) \geq G \mu (x) + \dfrac{1}{q}$
for $ n \geq p \big \} \cap C \Omega$. The sets $ A_{p,q}$ being
closed and contained in $K$ is compact. We assert  $G^*$-cap. $ (
A_{p,q} ) = 0 $. If not, there exist a non-zero measure $\nu$ on
$A_{p, q}$ such that $ G^* \nu $ is finite and continuous. This
follows because $G^*$ is  regular  (Lemma 2). Now  
$$
\int G \mu_n d \nu = \int G^* \nu d \mu_n 
$$
and the right hand side tends to $ \int G^* \nu d \mu = \int G \mu d
\nu $ as $n$ tends to infinity. By the nature of definition of
$A_{p,q}$ 
$$
\displaylines{\hfill 
  \int G \mu_n d \nu \geq \int G \mu d \nu + \frac{1}{q} \nu (E)
 . \hfill \cr
  \text{Hence} \hfill \int G \mu d \nu \leq \int G \mu d \nu +
  \frac{1}{q} \nu (E).\hfill } 
$$

This is impossible because $ \nu (E) \neq 0 $. Hence $G^*$-cap. $
(A_{p,q}) = 0 $ for all $p, q$ in $Z^+$. Hence it follows that in $C
\Omega $ the set of exceptional points  has outer $G^*$-capacity
zero. The above procedure is valid for arbitrary  $a > 0$ and the the
$ \Omega $ got from it. Now it is immediately verified that $
\lim. \inf.  G \mu_n = G \mu $ on $E$ $G^*$-quasi everywhere. 

The following two theorems follow on a line similar to the one used in
Chapter \ref{p3:chap3} in an analogous situation. 

\begin{thm}\label{p3:chap5:sec14:thm12} %\theorem 12.
  $E$\pageoriginale and $G$ satisfy the same conditions as in Theorem
  \ref{p3:chap5:sec14:thm11}. Let $ \big
  \{\mu_n \big \} $ be a sequence of positive measures on a compact
  set $K$ such  that $ \big \{\mu_n (E) \big\} $ is a bounded sequence
  and $G \mu_n (x)$ tends to $\varphi(x)$ pointwise,
  $G^*$-quasi-everywhere. Then $\varphi$ is equal $G^*$-quasi
  -everywhere to the potential of a measure $\mu$ which is the vague
  limit of a subsequence of the given sequence $\mu$. 
\end{thm}

\begin{thm}\label{p3:chap5:sec14:thm13}%theorem 13.
  $G$  being the same as int Theorem \ref{p3:chap5:sec14:thm11}, in
  addition let $E$ satisfy 
  the  second axiom of countability. Suppose $ \big \{ \mu_i \big
  \}_{i \in I}$ is any family of measures  $ \geq 0 $, on a compact
  set $K$, such that $\big \{\mu_i (E) \big \}$ is a  bounded family
  and $G \mu_i$ is directed for the natural decreasing order. Then the
  lower envelope of $G \mu_i$ is equal to the potential of a measure
  on $ K, G^*$-quasi-everywhere (this latter measure is the vague
  limit of a suitable subsequence from the given family ). 
\end{thm}

\begin{remark*} 
  The proof of the above convergence theorems  (of Chapters \ref{p3:chap3} and
  \ref{p3:chap5}),  are similar to those introduced in \cite{6}, but
  simpler; this  
  being rendered possible by the lemmas based on Lusin's property (or
  a similar one with capacity). There is another way of using
  continuous potentials to get analogous convergence theorems with the
  aid of functional analysis  (see \cite{1}, \cite{2}).  
\end{remark*}
