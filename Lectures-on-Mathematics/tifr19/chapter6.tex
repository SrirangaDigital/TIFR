\chapter{Convergence theorems (with exceptional \texorpdfstring{\boldmath${G^*}$}{G*}-negligible sets)\texorpdfstring{$^1$}{1}}\label{p3:chap3}

\setcounter{section}{6}
\section{Case of a compact space}\label{p3:chap3:sec7}%sec 7

We\pageoriginale shall study now some convergence theorems for potentials with
general kernels, first on a compact space $E$, and then on a locally
compact space. We shall see that the limits of potentials are
functions which differ from potentials only on sets, rare enough
according to the previous notion of negligibility or to a certain
capacity zero that we shall introduce for a second group of theorems.  

\begin{defn}\label{p3:chap3:sec7:def}%def 11 
  If $G$ is a regular kernel, a property $P$ is said to hold
  nearly $G$-everywhere if the set of points of $E$ at which $P$ does
  not hold is a $G$-negligible set.  
\end{defn}

\begin{thm}\label{p3:chap3:sec7:thm3}
Let $E$ be a compact space. $G$ a lower semi - continuous kernel $\geq
0$, with $G$\footnote{This Chapter improves and develops the paper \cite{6}
  of the bibliography} regular on $E$. If $\mu_n$ is a sequence of
measures in $\mathfrak{M}^+$ converging vaguely to $\mu$ then  
$$
\lim_n.  \inf_{ n \to \infty}. G \mu_n = C \mu \text{ nearly } G^* -
\text{everywhere} 
$$
\end{thm}

\begin{proof} 
  The lower semi-continuity of the kernel gives 
  $$
  \lim_n. \inf_{ n \to \infty}. G \mu \geq G \mu 
  $$
  Let\pageoriginale $K$ be a compact set in the set of points $A$ of $E$ at which
  strict inequality holds good. We assert that $K$ is $G^*$-
  negligible. Otherwise due to regularity of $G^*$, there exists a
  positive measures $\nu \neq 0$ on $K$ for which the $G^*$-potential
  is finite and continuous on $E$ (lemma $3$).  Since $\mu_n$ tends to
  $\mu$ in the vague topology,  
  $$
  \int G^* \nu d \mu_n \to \int G^* \nu d \mu  = \int G \mu d \nu.  
  $$

  On the other hand by Fatou's lemma, 
  \begin{multline*}
  \int \lim_n.  \inf. G \mu_n d \nu \leq. \lim_n \inf \int G \mu_n.  dv\\ 
  =\lim_n. \inf.  \int G \mu_n.  d \nu = \lim  _n. \inf.  \int G^* \nu
  d \mu_n  
  \end{multline*}
  therefore
  $$
  \int \lim_n.  \inf \mu _n d \nu = \int G \mu d \nu 
  $$

  This contradicts the assumption that $\lim _n.  \inf.  G \mu _n$ for
  points of $K$. Hence $A$ is a $G^*$-negligible set. The proof is
  complete.  
\end{proof}

\begin{thm}\label{p3:chap3:sec7:thm4} %the 4
  Let $E$ be a compact space and $G$ a kernel with $G^*$ regular. If
  $\{\mu _n\}$ is a sequence of measures in $\mathfrak{M}^+$ with $\mu
  _n (E)$ bounded further $G \mu_n (x)$ converges to a function $f(x)$
  then there exists a measure $\mu \geq 0$ (limits of a subsequence
  of $\mu'_n s$) such that $G \mu _n (x) = f(x)$ nearly
  $G^*$-everywhere.  
\end{thm}

\begin{proof}
  Since $\mu _n(E)$ are bounded the set $\{ \mu_n\}$ is relatively
  compact for the vague topology in $\mathfrak{M}^+$.  Hence we can
  choose a subsequence $\{ \mu_{\alpha _n}\}$ of $\{\mu _n\}$
  which converges vaguely to a measure $\mu \in \mathfrak{M}^+$. An
  application of theorem \ref{p3:chap1:sec3:thm1}\pageoriginale above shows that $G \mu (x) = f (x)$
  nearly $G^*$-everywhere.  
\end{proof}

\begin{thm}\label{p3:chap3:sec7:thm5} %the 5
  We again assume $E$ to be compact and $G^*$ regular.  If $\{ \mu
  _i\}_{i \in I}$ is a family of measures in $\mathfrak{M}^+$ with
  total mass $\mu_i (E)$ bounded for $i \in I$ such that  the family
  $\{ G \mu \}_{ i \in I}$ is a directed set for the increasing order,
  then the upper envelope $\varphi$ of $\{G \mu_i \}_{i \in I}$ is
  equal nearly $G^*$-everywhere to the potential of a measure (limits
  of $\mu _i $ according to a filter on $I$ finer than the filter $
  \mathscr{F}$ of sections with the order induced by the ordered
  family $\{ G \mu _i\} _{ i \in I}$.  
\end{thm}

\begin{proof}
  For any measure $\nu$
  $$
  \int G \mu_i d \nu = \int G^* \nu d \mu_i 
  $$
  The first integral tends for $\int \varphi d \mu$ according to the
  filter $\mathscr{F}$. 
\end{proof}

Let us introduce a filter finer than this filter for which $\{\mu_i\}$
converge to a measure $\mu \geq 0$. (This is possible because of the
relative compactness of the set $\{ \mu _i \} _{ i \in I}$  for which
$\mu _i (E)$ are bounded). Then if $G^* \nu $ is finite continuous the
second integral tends to $\int G^* \nu d \mu $, or what is the same to
$\int G \mu d \nu$.  

Suppose that $\varphi \neq G \mu$ on a non $G^*$- negligible set and
therefore on a non $G^*$-negligible compact set $K$. We may choose for
$\nu$, a measure on $K, \neq 0$ such that $G^* \nu$ is everywhere
finite continuous. Then our equality  
$$
\int \varphi d \nu = \int G \mu d \nu ~\text{gives a contradiction.}
$$ 

\begin{thm}\label{p3:chap3:sec7:thm6}%the 6 
  Let\pageoriginale the compact space $E$ possess a countable space  and $G$ be a
  kernel with $G^*$ regular.  If $\{\mu_i \} _{ i \in I}$ is any
  family of measures $\geq 0$ on $E$ such that $\{G \mu _i \}_{i \in
    I}$ is bounded for all $i \in I$, then the lower envelope
  $\inf\limits_{i}$. $G \mu_i$ is equal to a potential of a measure $\mu
  \leq 0$, nearly $G^*$-everywhere.  
\end{thm}

More precisely there exists a sequence $ \mu _{\alpha _n} \to \mu$
such that $\{ G \mu _{\alpha _n}\}$ is decreasing and $G \mu \leq
\inf\limits_i$. $G \mu_i \leq \inf\limits_{\alpha_n}$, these three
functions being equal $G^*$-nearly everywhere.  

\begin{proof}
  In virtue of Lemma \ref{p3:chap1:sec3:lem1}
  (\ref{p3:chap1}.\ref{p3:chap1:sec3}), there exists a  
  sequence $\{ G \mu _{ 
    \alpha _n }\}$\break $\{\alpha _{n}\}$  being a countable subset of $I$)
  of the family of such that for any lower semi continuous function
  $g$ on $E$ such that $g \leq \inf\limits_{\alpha _n }$. $G
  \mu_{\alpha_n}$ implies
  $g\leq \inf _{i \in I} G\mu_i$, we can choose in the family a
  decreasing sequence $\left\{G \mu_{\alpha'_n} \right\} G
  \mu_{\alpha'_n} \leq G\mu_{\alpha_n}$ therefore with the same
  property. As $\mu' 
  _{\alpha_n} (E)$ is bounded for every $\alpha'_n$ there exists a
  subsequence $\{ \beta _n \}$ such that $\{\mu _{\beta_n}\}$
  converges to a measure $\mu \geq 0$. By Theorem
  \ref{p3:chap3:sec7:thm3}, the potential 
  $G \mu = \lim _{\beta_n}.  \inf.  G \mu_{\beta _n}$ $G^*$- nearly
  everywhere. The fact that $\{G\mu_{\alpha'_n}\}$ has been choosen
  to be decreasing gives  
  $$
  G \mu = \inf_n.  G \mu _{\alpha _n} \text{ nearly}G^* \text{ - everywhere}
  $$
  
  But $G \mu \leq \inf_i G\mu_i$; therefore the three functions of the
  inequality $G \mu \leq \inf _i. G\mu^i _1 \leq \inf_n.  G \mu_{
    \alpha _n}$ are equal $G^* $- nearly everywhere.  
\end{proof}

\section[Extension of the converge...]{Extension of the converge theorems for locally\break compact
  space}\label{p3:chap3:sec8} %sce 8

The\pageoriginale previous convergence theorems allow extension to the locally
compact spaces with slightly more restrictions on the measures. In the
sequence of theorems that follows $E$ denote \textit{ a locally
  compact space}, $G$ a lower-semi continuous kernel with the
associated kernel $G^*$ regular on $E$.  

We shall say that a family $\{ \mu _i\}_{ i \in I}$ is $G^*$-admissible
if for any measure $\nu \geq 0$ whose support is compact and such that
$G^*\nu$ is finite continuous, $\int_{ C_k} G^* \nu d \mu _i \to 0$
\text{uniformly} with respect to $i \in I$, when $K \to E$ according
to the directed family of the compact sets $K$ of $E$.  

\medskip
\noindent
\textbf{General extension}

The theorems \ref{p3:chap3:sec7:thm3}, \ref{p3:chap3:sec7:thm4},
\ref{p3:chap3:sec7:thm5} and \ref{p3:chap3:sec7:thm6} of (\ref{p3:chap3} \S
\ref{p3:chap3:sec7}) are valid with 
$(\mu_i)$ or $(\mu _n)$ on $E$ and with the supplementary conditions
that  
\begin{enumerate}[1)]
\item the  family $(\mu_i)$ or $(\mu_n)$ is $G^*$-admissible 
\item in theorems \ref{p3:chap3:sec7:thm4}, \ref{p3:chap3:sec7:thm5}
  and \ref{p3:chap3:sec7:thm6} $\{ \mu _i (K)\}$ or $\{ \mu_n(K) \}$
  are 
  supposed to be bounded for every compact set $K$.  
\end{enumerate}

It will be sufficient to given in detail the extension of Theorem
\ref{p3:chap3:sec7:thm3}.  

\noindent \textbf{Theorem $\mathbf{3'}$.}
  $G^* $ being regular. Let $(\mu _n)$ be a sequence of
  $G^*$-admissible measure $\geq 0$ converging vaguely to the measure
  $\mu \geq 0$. Then \break $\lim_m.\inf / G \mu _n = G \mu,  G^*$- nearly
  everywhere.  


\begin{proof}
  As in the case of compact space we assume that $\lim\limits_n.  \inf
 .  G \mu _n > G \mu$ on a compact set $K_1$ which is not $G^*$ -
  negligible ; we will arrive at a contradiction\pageoriginale as follows. 
\end{proof}

There exists on $K_1$ a positive measure $\nu \neq 0$ such that $G^*\nu $
is finite and continuous on the whole space. We shall prove that $\int
G^* \nu d \mu_n \to \int G^* \nu d \nu $, and a contradiction follows
exactly as in the proof of theorem \ref{p3:chap3:sec7:thm3}. But in general $G^* \nu $ does
not have compact support. We may introduce for every compact set $K$ a
function $\varphi _K$ which is finite continuous, zero outside a
compact set, satisfying $0 \geq \varphi _K\geq  G^*  \nu $ everywhere and
$\varphi _K = G^* \nu $ on $K$.  The existence of such a function
$\varphi_K$ results from the normality of the space.  

Let us first observe that $\int G^* \nu d\mu _n $ is bounded: For any
compact set $K'$ it is equal to the sum $\int_{C_K'}. G^* \nu d \mu_n+ \int_{K'}
G^* \nu d \nu _n $.  The first integral is less than $\varepsilon (>
0)$ for a suitable $K'$ ; the second is bounded because $\mu _n (K')$
is bounded. If  
$$
\int G^* \nu d \mu _n\leq \lambda, \text{ then } \int \varphi _K d \mu
_n \leq \lambda \text{ and the limit } \int \varphi _K d \mu   
$$
is also $\leq \lambda$. Therefore $\int G^* d \mu$ ( the upper bound
of the integrals of all such $\varphi _K$ ) is $\leq \lambda$.  

Now
\begin{multline*}
  \big|  \int G^*\nu d \mu _n - \int G^* \nu d \mu \big|
  \leq \big|  \int (G^*\nu- \varphi _K) d \mu _n \big|\\  
  + \big| \int
  \varphi _K d \mu _n - \int \varphi_K d \mu \big|
  +\big|   \int (G ^* \nu - \varphi _K) d \mu \big| 
\end{multline*}

The first integral on the right, is $\leq  \int_{ C_K} G^* \mu _n $
which is smaller than $\varepsilon / 3$ for all $n$ and $a$ suitable
$K_1$ (say $K_\circ$); the third one is $\geq \int_{C_K} G^*_\nu d
\mu$ which is $< \epsilon/3$ for a suitable $K_1$ say $K^1_\circ$,
because $\int G^* \nu.  d \mu $\pageoriginale is 
finite. Taking for $K$  the union $K_\circ \cup K^1_\circ$, we
maintain these inequalities, with such a fixed $K$ the second integral
on the right is $< 
\varepsilon / 3$ for $N$ sufficiently large; therefore the left member
is smaller than $\varepsilon$ for $n > N$. Hence as $n \to \infty,
\int G^* \nu, d \mu _n \to G^*.  d \mu $. Hence we get a contradiction
as in the case of the compact space $E$.  

\begin{remark*}
  It would be interesting to have suitable criteria for $G^*$-
  admissibility. Let us indicate two important ones.  
  \begin{enumerate} [(i)]
  \item Supports of $\mu _n$'s are contained in a fixed compact set; 
  \item $G^* ( x, y ) \to 0 $ uniformly in $y$ on any compact set,
    when extends to the Alexndroff point of $E$.  
  \end{enumerate}
\end{remark*}
