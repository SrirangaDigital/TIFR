\part[Disintegration of a measure with respect to a single...]{Disintegration of a measure with respect to a single
  $\sigma$-Algebra}\label{part1}


\chapter{Conditional expectations and disintegrations}\label{part1:chap1}

\section{Notations}\label{part1:chap1:sec1}

Throughout\pageoriginale these Notes the following notations will be
followed. $\mathbb{R}$ will stand for the set of all real numbers,
$\bar{\mathbb{R}}$, for the set of all extended real numbers,
i.e. $\mathbb{R}$ together with the ideal points $-\infty$ and $+
\infty$, $\mathbb{N}$ for the set of all natural numbers, $\mathbb{Z}$
for the set of all integers and $\mathbb{Q}$ for the set of all
rational numbers. $\mathbb{R}^+$ (resp $\bar{\mathbb{R}}^+$) will
stand for the positive elements of $\mathbb{R}$ (resp
$\bar{\mathbb{R}}$). 

Let $X$ be a non-void set. Let $A$ be a subset of $X$. Them
$\complement A$ will stand for the complementary set of $A$,
i.e. $\complement A = \left\{ x \in X \mid x \notin A  \right\}$. 

Let $X$ be a non-void set and $\mathfrak{X}$ a $\sigma$-algebra of
subsets of $X$. Then the pair $(X, \mathfrak{X})$ is called a
\textit{measurable space}. If $f$ is a function on a measurable space
$(X, \mathscr{X})$, with values in a topological space $Y$, $Y \neq
\emptyset$, we say $f$  \textit{belongs to } $\mathfrak{X}$ and write
$f\in \mathfrak{X}$ if $f$ is measurable with respect to
$\mathfrak{X}$. (For measurablility concepts, we always consider on
topological spaces only their \textit{Borel $\sigma$-algebra}, i.e.,
the $\sigma$-algebra generated by all the open sets). 

By a \textit{measure space} $(X, \mathfrak{X}, \mu)$  we always mean a
measurable space $(X, \mathfrak{X})$ and a positive, $\not\equiv 0$
measure $\mu$ on $\mathfrak{X}$. 


Let $(X, \mathfrak{X}, \mu)$ be a measure space. Let $A$ be a subset
of $X$. We say $\mu$ is \textit{carried} by $A$ if $\mu (\complement
A) = 0$ in case $A \in \mathfrak{X}$ and in case $A \notin
\mathfrak{X}$, if $\mu(B)$ for every $B \in \mathfrak{X}$, $B \subset
\complement A$ is zero. $A \subset X$ is said to be a $\mu$-null set
if there exists a $B \in \mathfrak{X}$ with $\mu(B) = 0$ and $A
\subset B$. $\u{\mathscr{N}_\mu}$ will stand for the class of all
$\mu$-null sets of $X$. $\u{\hat{\mathfrak{X}}_\mu}$ will stand for
the $\sigma$-algebra generated by $\mathfrak{X}$ and
$\mathscr{N}_\mu$.


$\hat{\mathfrak{X}}_\mu$\pageoriginale is called the
\textit{completion} of $\mathfrak{X}$  with respect to $\mu$. If
$\mathfrak{X} = \hat{X}_\mu$, we say $\mathfrak{X}$ is
\textit{complete} with respect to $\mu$. If $f$ is a function on $X$
with values in a topological space $Y$, $Y \neq \emptyset$ and if $f
\in\hat{\mathfrak{X}}_\mu$, we say $f$ is
$\mu$-\textit{measurable}. If $\mathscr{Y}$ is any $\sigma$-algebra on
$X$, we denote by $\u{\mathscr{Y} V \mathscr{N}_\mu}$, the
$\sigma$-algebra generated by $\mathscr{Y}$ and
$\mathscr{N}_\mu$. Thus, $\hat{\mathfrak{X}}_\mu = \mathfrak{X} V
\mathscr{N}_\mu$. 

The symbol $\forall_\mu x$ will stand for `\textit{for $\mu$-almost
  all $x$}'. 

If $h$ is any non-negative function $\epsilon\hat{\mathfrak{X}}_\mu$,
$\u{h,\mu}$ will  stand for the measure on $\hat{\mathfrak{X}}_\mu$
given by $h \cdot \mu (B) = \int\limits_B h(w) d\mu (w)$ for all $B
\in \hat{\mathfrak{X}}_\mu$.


If $E$ is a Banach space over the real numbers, $\u{L^1 (X;
  \mathfrak{X} ; \mu ; E)}$ will stand for the Banach space of all
$\mu$-equivalence classes of functions on $X$ with values in $E$ which
belong to $\hat{\mathfrak{X}}_\mu$ and which are $\mu$-integrable. 

$\u{L^1(X; \mathfrak{X}; \mu)}$ will stand for the Banach space of all $\mu$-equivalence classes of functions with values in $\bar{\mathbb{R}}$ which belong to $\hat{\mathfrak{X}}_\mu$ and which are $\mu$-integrable.

Let $\u{\mathscr{A}(X; \mathfrak{X}; \mu)}$ denote the set of all extended real
valued functions on $X$ which belong to $\hat{\mathfrak{X}}_\mu$ and
let $\u{\mathscr{A}^+ (X; \mathfrak{X} ;\mu)}$ denote the set of all
non-negative elements of $\mathscr{A} (X; \mathfrak{X}; \mu)$. 

If $f \in \mathscr{A}^+ (X; \mathfrak{X}; \mu)$  or if $f \in L^l(X;
\mathfrak{X}; \mu)$ of if $f \in L^l(X; \mathfrak{X}; \mu; E)$, then
$\u{\mu(f)}$, $\u{\int f(w) d \mu (w)}$, $\u{\int f (w) \mu (dw), \;
  \int f d\mu}$ will all denote the integral of $f$ with
\textit{respect to $\mu$, over $X$}. 


\section[Basic definitions in the theory of...]{Basic definitions in
  the theory of integration for 
  Banach space valued functions}\label{part1:chap1:sec2}

The theory of integration for Banach space valued functions on a
measure space is assumed here. However, by way of recalling, we give
below a few basic definitions. 

Let\pageoriginale $(\Omega, \mathscr{O}, \lambda)$ be a measure
space. Let $E$ be a Banach space over the real numbers. If
$\mathcal{S}$ is any $\sigma$-algebra on $\Omega$, a function $f$ on
$\Omega$ with values in $E$ is said to be a \textit{step function}
belonging to $\mathcal{S}$, if there exist finitely many  sets
$(A_i)_{i=1,2,\ldots, n}, A_i \in \mathcal{S}$, $i = 1,\ldots n$,
$A_i \cap A_j = \emptyset$ if $i \neq j$ and finitely many points
$(x_i)_{i = 1,\ldots n}, x_i \in E \;\;  \forall i = 1, \ldots n$ such that
$\forall w \in\Omega$, $f(w) = \sum\limits^n_{i=1} \chi_{A_i}
(w)x_i$. A function $f$ on $\Omega$ with values in $E$ is said to be
\textit{strongly measurable} if there exists a sequence $(f_n)_{n\in
  \mathbb{N}}$ of step functions, $f_n \in\hat{\mathscr{O}}_\lambda
\forall n \in \mathbb{N}$, such that $\forall_\lambda w$, $f_n (w) \to
f(w)$ in $E$, as $n \to \infty$. Note that a strongly measurable
function belongs to $\hat{\mathscr{O}}_\lambda$. A function $f$ on
$\Omega$ with values in $E$ is said to be
$\lambda$-\textit{integrable} or \textit{integrable in the sense of
  Bochner} if $f$ is strongly measurable and if $\int|f| (w)
d\lambda(w) < \infty$, where $|f|$ is the real valued function on
$\Omega$ assigning to each $w \in \Omega$, the norm of $f(w)$ in
$E$. One can prove that if $\mathcal{S}$ is a $\sigma$-algebra
contained in $\hat{\mathscr{O}}_\lambda$ and if $f \in \mathcal{S}$
is $\lambda$-integrable, then one can find a sequence $(f_n)_{n \in
  \mathbb{N}}$ of step functions belonging to $\mathcal{S}$ and a
real valued non-negative $\lambda$-integrable function $g$ belonging
to $\mathcal{S}$ such that 
\begin{align*}
& \forall_\lambda w, \; f_n (w) \to f(w) \text{ in } E \text{ as } n
  \to \infty \text{ and }\\
& \forall n, \; \forall_\alpha w, |f_n| (w) \leq g (w). 
\end{align*}

For further properties of Bochner integrals, the reader is referred to
Hille and Phillips \cite{key1}. 

\section[Conditional Expectations and Disintegrations;...]{Conditional Expectations 
and Disintegrations;\hfil\break Basic definitions}\label{part1:chap1:sec3}

Let $(\Omega, \mathscr{O}, \lambda)$ be a measure space. Let
$\mathscr{C}$ be a $\sigma$-algebra contained in
$\hat{\mathscr{O}}_\lambda$. 

\begin{defn}\label{part1:chap1:def1}%%% 1 
Let $f$ be a $\lambda$-integrable function on $\Omega$ with values in
a Banach space $E$ over the reals (resp. $f$ extended real valued, $f
\geq 0$ and $f \in \hat{\mathscr{O}}_\lambda$). A
function\pageoriginale $f^\mathscr{C}$ on $\Omega$ with values in $E$
(resp. extended reals) is said to be a \textit{conditional
  expectation} of $f$ with respect to $\mathscr{C}$ if 

\noindent
(i) $f^\mathscr{C} \in \mathscr{C}$ and is $\lambda$-integrable
(resp. (i) $f^\mathscr{C} \in \mathscr{C}$ and is $ \geq 0$)

\noindent
and (ii) $\forall  \; A \in \mathscr{C}$, $\int\limits_A f^\mathscr{C}
(w) d\lambda (w) = \int\limits_A f(w) d \lambda (w) $. 
\end{defn}

Here, $\int\limits_A f(w) d \lambda (w)$ (resp. $\int\limits_A
f^\mathscr{C} (w) d \lambda (w)$) stands for the integral of
$\chi_A.f$ (resp. $\chi_A . f^\mathscr{C}$) with respect to
$\lambda$. These exist since $f$ and $f^\mathscr{C}$ are
$\lambda$-integrable (resp. $f$ and $f^\mathscr{C}$ are $\geq 0$). 


\begin{defn}\label{part1:chap1:def2}
A family $(\lambda^\mathscr{C}_w)_{w \in \Omega}$ of positive measures
on $\mathscr{O}$, indexed by $\Omega$ is called a system of
\textit{conditional probabilities} with respect to $\mathscr{C}$ or a
\textit{disintegration} of $\lambda$ with respect to $\mathscr{C}$ if
it has the following properties, namely 

\medskip
\noindent
(i) $\forall B \in\mathscr{O}$, the function $w \to
\lambda^\mathscr{C}_w (B)$ belongs to $\mathscr{C}$

\medskip
\noindent
and (ii) $\forall B \in \mathscr{O}$, the function $w \to
\lambda^\mathscr{C}_w (B)$ is a conditional expectation of $\chi_B$
with respect to $\mathscr{C}$. 
\end{defn}

We remark that (i) implies that for every function $f$ on $\Omega$, $f
\geq 0$, $f \in \mathscr{O}$, the function $w \to
\lambda^\mathscr{C}_w (f)$ belongs to $\mathscr{C}$ and that (ii)
implies that $\forall$ function $f$ on $\Omega$, $f \geq 0$, $f \in
\mathscr{O}$, the function $w \to \lambda^\mathscr{C}_w (f)$ belongs
to $\mathscr{C}$ and that (ii) implies that $\forall$ function $f$ on
$\Omega$, $f \geq 0$, $f \in \mathscr{O}$, the function $w \to
\lambda^\mathscr{C}_w (f)$ is a conditional expectation of $f$ with
respect to $\mathscr{C}$. From (ii), taking $B = \Omega$, we see that
$\forall_\lambda w$, $\lambda^\mathscr{C}_w$ is a probability
measure. 

Note that the existence of a disintegration of $\lambda$ with respect
to $\mathscr{C}$ implies immediately the existence of conditional
expectations with respect to $\mathscr{C}$ for non-negative functions
belonging to $\mathscr{O}$. We shall see below that if $\lambda$
restricted to $\mathscr{C}$ is $\sigma$-finite, conditional
expectation with respect to $\mathscr{C}$ for any non-negative
function belonging to $\hat{\mathscr{O}}_\lambda$ and for any
$\lambda$-integrable Banach space valued function exists. But a
disintegration of $\lambda$ need not always exist without any further
assumptions about $\Omega$ and $\mathscr{O}$ as can be seen by an
example due to J. Dieudonn\'e \cite{key1}. 

\section{Illustrations and motivations}\label{part1:chap1:sec4}

Let\pageoriginale $(\Omega, \mathscr{O}, \lambda)$ be a measure
space. Before we proceed to prove the existence theorems of
conditional expectations, we shall see the above notions when
$\mathscr{C}$ is given by a partition of $\Omega$. 

Let $A \in \hat{\mathscr{O}}_\lambda$ be such that $0 < \lambda (A) <
\infty$. We call the measure $\dfrac{\chi_A \cdot
  \lambda}{\lambda(A)}$ on $\mathscr{O}$, the \textit{conditional
  probability} of $\lambda$ given $A$. We call $\dfrac{\int\limits_A f
d \lambda}{\lambda(A)}$, the \textit{conditional expectation} of $f$
given $A$, where $f$ is either a Banach space valued
$\lambda$-integrable function or a non-negative function belonging to
$\hat{\mathscr{O}}_\lambda$. Note that the conditional probability is
always a probability measure whether $\lambda$ is or not. 

Suppose also that $0 < \lambda (\complement A) < \infty$. (This will
happen only if $\lambda$ is a finite measure and when $0 < \lambda (A)
< \lambda (\Omega)$). Then, consider the $\sigma$-algebra $\mathscr{C}
= (\emptyset, A, \complement A, \Omega)$. 

Let $E$ be a Banach space over the real numbers and let $f$ be a
$\lambda$-integrable function on $\Omega$ with values in $E$ (resp. $f
\geq 0$, $f \in\hat{\mathscr{O}}_\lambda$). Define
\begin{equation*}
f^\mathscr{C} (W) = 
\begin{cases}
\frac{\int\limits_A f d \lambda}{\lambda(A)}, \text{ if } w \in A\\[4pt]
\frac{\int\limits_{\complement A} f d \lambda}{\lambda(\complement
  A)}, \text{ if } w \in \complement A 
\end{cases}
\end{equation*}

Then $f^\mathscr{C}$ is with values in $E$, $f^\mathscr{C} \in
\mathscr{C}$ and is $\lambda$-integrable (resp. $f^\mathscr{C}$ is
$\geq 0$, $f^\mathscr{C} \in \mathscr{C}$).

We have 
\begin{itemize}
\item[{\rm (i)}] $\int\limits_A f^\mathscr{C} (w) d \lambda (w) =
  \int\limits_A f(w) d \lambda(w)$ \quad and 


\item[{\rm (ii)}] $\int\limits_A f^\mathscr{C} (w) d \lambda (w) =
  \int\limits_{\complement A} f(w) d \lambda (w)$. 
\end{itemize}

Thus, $f^\mathscr{C}$ is a conditional expectation of $f$ with respect
to $\mathscr{C}$. 

Consider\pageoriginale the family $(\lambda^\mathscr{C}_w)_{w \in
  \Omega}$ of measures on $\mathscr{O}$ defined as
$$ 
\lambda^\mathscr{C}_w = 
\begin{cases}
\dfrac{\chi_A \cdot \lambda }{\lambda(A)}, \text{ if } w \in  A\\[5pt]
\dfrac{\chi_{\complement A} \cdot \lambda}{\lambda(\complement A)}, \text{ if } w \in \complement A. 
\end{cases}
$$

This family $(\lambda^\mathscr{C}_w)_{w \in \Omega}$ satisfies the
conditions (i) and (ii) of definition (\ref{part1:chap1}, \S\ \ref{part1:chap1:sec3}, \ref{part1:chap1:def2}) and hence is a 
disintegration of $\lambda$ with respect to $\mathscr{C}$. 

Thus, we see that the conditional expectation of any Banach space
valued $\lambda$-integrable function, that of any non-negative
function belonging to $\hat{\mathscr{O}}_\lambda$ and a disintegration
of $\lambda$ with respect to $\mathscr{C}$, always exist when
$\mathscr{C}$ is of the form $(\emptyset, A, \complement A, \Omega)$
with $0 < \lambda(A) < \infty$ and $0 < \lambda (\complement A ) <
\infty$. 

Next, let us consider the following situation. 

Suppose $\Omega = \bigcup\limits_{n \in\mathbb{N}} A_n$, where
$\forall n $, $A_n \in\hat{\mathscr{O}}_\lambda$, $A_n \cap A_m =
\emptyset$, if $n \neq m$ and $0 < \lambda (A_n) < \infty \; \; \forall n
\in \mathbb{N}$. We call such a sequence $(A_n)_{n \in \mathbb{N}}$ of
sets, a \textit{partition} of $\Omega$. Let $I$ be any subset of
$\mathbb{N}$. Then, the collection of all sets of the form
$\bigcup\limits_{i \in I} A_i$, as $I$ varies over all the subsets of
$\mathbb{N}$  is a $\sigma$-algebra. Let us denote this
$\sigma$-algebra by $\mathscr{C}$. We say then that $\mathscr{C}$ is
given by the partition $(A_n)_{n \in\mathbb{N}}$.


If $f$ is a $\lambda$-integrable function with values in $E$ or if $f$
is $\geq 0$ and belongs to $\hat{\mathscr{O}}_\lambda$, consider the
function $f^\mathscr{C}$ defined as 
$$
f^\mathscr{C} (w) = \dfrac{\int\limits_{A_n} f d
  \lambda}{\lambda(A_n)}, \text{ if } w \in A_n. 
$$

Then $f^\mathscr{C}$ is easily seen to be a conditional expectation of
$f$ with respect to $\mathscr{C}$. 


Consider the family $(\lambda^\mathscr{C}_w)_{w \in \Omega}$ of
  measures on $\mathscr{O}$ defined as 
$$
\lambda^\mathscr{C}_w = \frac{\chi_{A_n} \cdot \lambda}{\lambda(A_n)}
\text{ if } w \in A_n . 
$$\pageoriginale

Then the family $(\lambda^\mathscr{C}_w)_{w \in \Omega}$ of measures
satisfies the condition (i) and (ii) of definition (\ref{part1:chap1},
\S\ \ref{part1:chap1:sec3}, \ref{part1:chap1:def2}) and 
hence is a disintegration of $\lambda$ with respect to $\mathscr{C}$. 

Thus, we see that when $\mathscr{C}$ is given by a partition $(A_n)_{n
\in \mathbb{N}}$ with $0 < \lambda(A_n) < \infty$, conditional
expectation of any Banach space valued $\lambda$-integrable function,
that of any non-negative function belonging to
$\hat{\mathscr{O}}_\lambda$ and a disintegration of $\lambda$ with
respect to $\mathscr{C}$, always exist. 

\section[Existence and uniqueness theorems of...]{Existence and uniqueness theorems of conditional expectations
for extended real valued functions; A few properties}\label{part1:chap1:sec5}


Let $(\Omega, \mathscr{O}, \lambda)$ be a measure space. Let
$\mathscr{C}$ be a $\sigma$-algebra contained in
$\hat{\mathscr{O}}_\lambda$. Let further, $\lambda$ restricted to
$\mathscr{C}$ be $\sigma$-finite. 

\begin{proposition}\label{part1:chap1:prop3}
Let $f \in L^1(\Omega; \mathscr{O}; \lambda)$ (resp. $f \in
\mathscr{A}^+ (\Omega ; \mathscr{O}; \lambda)$). Then a conditional
expectation of $f$ with respect of $\mathscr{C}$ exists. Moreover, it
is unique in the sense that if $g_1$ and $g_2$ are two conditional
expectations of $f$ with respect to $\mathscr{C}$, then
$\forall_\lambda w$, $g_1 (w) = g_2 (w)$. 
\end{proposition}

\begin{proof}
The proof of this proposition follows just from a straight forward
application of the Radon-Nikodym theorem. More precisely, for $A \in
\mathscr{C}$, define $\nu(A) = \int\limits_A f(w) d\lambda(w)$. Then
$\nu$ is a finite signed measure (resp. a positive measure) on
$\mathscr{C}$ and is absolutely continuous with respect to $\lambda$
which is $\sigma$-finite on $\mathscr{C}$. Hence, by the Radon-Nikodym
theorem, there exists a function $g \in \mathscr{C}$ which is
$\lambda$-integrable (resp. $g \in \mathscr{C}$ and is $ \geq 0$) and
which is unique upto a set of measure zero, such that 
$$
\nu(A) = \int\limits_A f(w) d \lambda (w) = \int\limits_A g (w) d
\lambda(w) \; \forall  A \in \mathscr{C}. 
$$
\end{proof}

Note\pageoriginale that for any given $f \in L^1(\Omega; \mathscr{O};
\lambda)$ (resp. $f \in \mathscr{A}^+ (\Omega; \mathscr{O}; \lambda)$,
we get a class of functions as conditional expectations of $f$ with
respect to $\mathscr{C}$, any two functions in the class differing by
a $\mathscr{C}$-set of measure zero at most. Also note that if $f_1$
and $f_2$ are any two functions belonging to $\mathscr{A}^+ (\Omega ;
\mathscr{O}; \lambda)$ and are equal almost everywhere, then also we
have $\forall_\lambda w$, $f^\mathscr{C}_1(w) = f^\mathscr{C}_2
(w)$. Thus, we have a map $u_{\mathscr{O}, \mathscr{C}}$
(resp. $v_{\mathscr{O}, \mathscr{C}}$) from $L^1 (\Omega; \mathscr{O};
\lambda)$ to $L^1 (\Omega; \mathscr{O}; \lambda)$ (resp. from
$\mathscr{A}^+ (\Omega; \mathscr{O}; \lambda)$ to
$\tilde{\mathscr{A}}^+_\mathscr{C}$ where $\tilde{A}^+_\mathscr{C}$
stands for the set of all $\lambda$-equivalence classes of
non-negative functions which belong to $\mathscr{C}$). Here, we are
making as abuse of notation by denoting by the same symbol $f, a$
function as well as the class to which it belongs.

The map $u_{\mathscr{O}, \mathscr{C}}$ is a continuous linear map from
the Banach space $L^1(\Omega; \mathscr{O}; \lambda)$ to $L^1 (\Omega;
\mathscr{C}; \lambda)$ with $||u_{\mathscr{O}, \mathscr{C}}|| = 1$
where $||u_{\mathscr{O}, \mathscr{C}}|| = \sup\limits_{f \neq 0}
\frac{\int |f^\mathscr{C}| d\lambda}{\int|f|d \lambda}$. 

We have the following properties of the maps $u_{\mathscr{O},
  \mathscr{C}}$ and $v_{\mathscr{O}, \mathscr{C}}$. 
\begin{itemize}
\item[{\rm (i)}] $u_{\mathscr{O}, \mathscr{C}} (f^\mathscr{C}) =
  f^\mathscr{C} \forall  f \in L^1 (\Omega; \mathscr{O}; \lambda)$
  i.e. $u_{\mathscr{O}, \mathscr{C}}$ is a projection onto the
  subspace $L^1(\Omega; \mathscr{C}; \lambda)$ of $L^1 (\Omega;
  \mathscr{O}; \lambda)$
$$
(v_{\mathscr{O}, \mathscr{C}} (f^\mathscr{C}) = f^\mathscr{C} \forall
  f \in \mathscr{A}^+ (\Omega; \mathscr{O}; \lambda))
$$

\item[{\rm (ii)}] $|u_{\mathscr{O}, \mathscr{C}} (f)| \leq
  u_{\mathscr{O}, \mathscr{C}} (|f|)$ in the sense that $\forall f \in
  L^1 (\Omega; \mathscr{O}; \lambda)$, $\forall_\lambda w$,
  $|u_{\mathscr{O}, \mathscr{C}} (f)| (w) \leq u_{\mathscr{O},
    \mathscr{C}} (|f|)(w)$. 


\item[{\rm (iii)}] $f \geq 0 \Rightarrow u_{\mathscr{O}, \mathscr{C}} (f) \geq 0$
  in the sense that $ \forall f \in L^1 (\Omega; \theta; \lambda)$ which is such that $\forall_\lambda w$, $f(w) \geq 0$, we have  $\forall_\lambda w, u_{\mathscr{O}, \mathscr{C}} (f) (w) \geq 0$. 

\item[{\rm (iv)}] If $g$ is non-negative and belongs to $\mathscr{C}$ and $f \in
  L^1(\Omega; \mathscr{O}; \lambda)$ and if $gf$ is
  $\lambda$-integrable or if $g$ is $\lambda$-integrable and belongs
  to $\mathscr{C}$  such that $gf$ is $\lambda$-integrable, we have 
$$
u_{\mathscr{O}, \mathscr{C}} (fg) = g \cdot u_{\mathscr{O},
  \mathscr{C}} (f). 
$$\pageoriginale 
(If $g$ is non-negative and belongs to $\mathscr{C}$) and if $f \in
\mathscr{A}^+ (\Omega; \mathscr{O}; \lambda)$, then
$$
(\forall_\lambda w, v_{\mathscr{O}, \mathscr{C}} (fg )(w) = g(w) \cdot
v_{\mathscr{O}, \mathscr{C}} (f) (w)).
$$

\item[{\rm (v)}] If $\mathscr{S}$ is any $\sigma$-algebra contained in
  $\hat{\mathscr{C}}_\lambda$, 
$$
u_{\mathscr{C}, \mathscr{S}} \circ u_{\mathscr{O}, \mathscr{C}} =
u_{\mathscr{O}, \mathscr{S}}
$$ 
i.e. the operation of conditional expectation is transitive
$$
(v_{\mathscr{C}, \mathscr{S}} \circ v_{\mathscr{O}, \mathscr{C}} =
v_{\mathscr{O}, \mathscr{S}})
$$

\item[{\rm (vi)}] \qquad $u_{\mathscr{O}, \mathscr{O}} =$ Identity. 
$$
(\forall f \in \mathscr{A}^+ (\Omega; \mathscr{O}; \lambda), \quad
  \forall_\lambda w, \; v_{\mathscr{O}, \mathscr{O}} (f) (w) = f(w))
$$

\item[{\rm (vii)}] \qquad $u_{\mathscr{O}, (\Omega, \emptyset)} (f) =
  \dfrac{1}{\lambda(\Omega)} \int f d \lambda$ \text{ if }
  $\lambda(\Omega)$  is finite

$(\forall f \in \mathscr{A}^+ (\Omega; \mathscr{O}; \lambda), \;
  \forall_\lambda w, v_{\mathscr{O}, (\Omega, \emptyset)} (f) (w) =
  \dfrac{1}{\lambda(\Omega)} \int f d \lambda$\break  if 
  $\lambda(\Omega) < \infty)$

\end{itemize}

\section[Existence and Uniqueness theorems of conditional...]{Existence and 
Uniqueness theorems of\hfil\break conditional expectation
for Banach space\hfil\break valued integrable functions}\label{part1:chap1:sec6}

Let $(\Omega, \mathscr{O}, \lambda)$ be a measure space, and let
$\mathscr{C}$ be a $\sigma$-algebra contained in $\hat{O}_\lambda$.
Let $\lambda$ restricted to $\mathscr{C}$ be $\sigma$-finite. Let $E$
be a Banach space over the real numbers. 

Now, we are going to prove the existence and uniqueness theorems of
conditional expectations for any $f \in L^1(\Omega; \mathscr{O};
\lambda; E)$. To prove the existence theorem, we have to adopt
essentially a different method than the one in
\S\ \ref{part1:chap1:sec5} for extended 
real numbers as the Radon-Nikodym theorem in general is not valid for
Banach\pageoriginale spaces. By the Radon-Nikodym theorem for Banach
spaces we mean the following theorem:

Let $(X, \mathfrak{X}, \mu)$ be a measure space. Let $E$ be a Banach
space over the real numbers. Let $\nu$ be a measure on $\mathfrak{X}$
with values in $E$ and let $\nu$ be absolutely continuous with respect
to $\mu$. Then, there exists a $\mu$-integrable function $g$ on $\chi$
with values in $E$ such that 
$$
\forall \; A \in \mathfrak{X}, \quad \nu (A) = \int\limits_A g d \mu. 
$$

\begin{thm}\label{part1:chap1:thm4}
Let $f \in L^1 (\Omega; \mathscr{O}; \lambda; E)$. Then, 
\begin{itemize}
\item[{\rm (i)}] {\em Existence: } A conditional expectation of $f$
  with respect to $\mathfrak{}$ exists. 

\item[{\rm (ii)}] {\em Uniqueness: } If $g_1$ and $g_2$ are two
  conditional expectations of $f$ with respect to $\mathscr{C}$, then
  $\forall_\lambda w, g_1(w) = g_2 (w)$. 
\end{itemize}
\end{thm}

\begin{proof}
\begin{itemize}
\item[{\rm (i)}] {\em Existence.} Let  $L^1(\Omega; \mathscr{O},
  \lambda) \bigotimes\limits_\mathbb{R} E$ be the algebraic tensor
  product of $L^1(\Omega; \mathscr{O}; \lambda)$ and $E$ over the real
  numbers. There is an injective linear map from $L^1(\Omega;
  \mathscr{O}; \lambda) \bigotimes\limits_\mathbb{R} E$ to
  $L^1(\Omega; \mathscr{O}; \lambda; E)$ which takes an element $f
  \otimes x$ of $L^1(\Omega; \mathscr{O}; \lambda)
  \bigotimes\limits_\mathbb{R} E$ to $f.x$ of $L^1(\Omega;\break
  \mathscr{O}; \lambda; E)$. Hence, we can consider $L^1 (\Omega;
  \mathscr{O}; \lambda) \bigotimes\limits_\mathbb{R} E$ as a subspace
  of $L^1 (\Omega; \mathscr{O}; \lambda; E)$. A theorem of Grothendieck says that $L^1 (\Omega; \mathscr{O}; \lambda; E)$ is the completion of $L^1 (\Omega; \mathscr{O}; \lambda) \bigotimes\limits_\mathbb{R}
  E$ for the `$\pi$-topology' on $L^1(\Omega; \mathscr{O}; \lambda)
  \bigotimes\limits_\mathbb{R} E$. 

The linear map $v$ from $L^1(\Omega;\mathscr{O};\lambda)
\bigotimes\limits_\mathbb{R} E$ to $L^1 (\Omega; \mathscr{C}; \lambda)
\bigotimes\limits_\mathbb{R} E$\break which takes an element $f \otimes x$
of $L^1 (\Omega; \mathscr{O};\lambda) \bigotimes\limits_\mathbb{R} E$
to $u_{\mathscr{O}, \mathscr{C}} (f) \otimes x$ of $L^1(\Omega;
\mathscr{C};\lambda) \bigotimes\limits_\mathbb{R} E$ is a contraction
mapping of these normed spaces under the respective `$\pi$-topologies'
and hence extends to a unique continuous linear mapping of
$L^1(\Omega; \mathscr{O}; \lambda; E)$ to $L^1 (\Omega; \mathscr{C};\break
\lambda; E)$, which we again denote by $v$, Now, it is easy to see
that $\forall f \in L^1 (\Omega; \mathscr{O}; \lambda; E)$, $v(f)$ is
a conditional expectation of $f$\pageoriginale with respect to
$\mathscr{C}$. 

\item[{\rm (ii)}] {\em Uniqueness. } To prove uniqueness, it is
  sufficient to prove that if $f$ is a $\lambda$-integrable function
  on $\Omega$ with values in $E$ and belongs to $\mathscr{C}$ and if
  $\int\limits_A f d \lambda = 0 \forall A \in \mathscr{C}$, then
  $\forall_\lambda w, f(w) = 0$. 

So, let $f \in L^1 (\Omega; \mathscr{C}; \lambda; E)$ with
$\int\limits_A f d \lambda = 0 \;\; \forall A \in \mathscr{C}$. 

Let $E'$ be the topological dual of $E$. If $x' \in E'$ and $x \in E$,
let $\langle x', x \rangle$ denote the value of $x'$ at $x$, $\forall
x' \in E'$, $\forall A \in \mathscr{C}$, we have $\int\limits_A
\langle x', f(w)\rangle$ $d\lambda(w) = 0$, and therefore, $\forall x'
\in E'$, $\forall_\lambda w$, $\langle x', f(w)\rangle =0$.
\end{itemize}
\end{proof}

Now, there exists a set $N_1 \in \mathscr{C}$ with $\lambda(N_1) = 0$,
and a \textit{separable} subspace $F$ of $E$ such that if $w \not\in
N_1$, $f(w) \in F$. This is because, $f$ is the limit almost
everywhere of step functions. Since $F$ is separable, there exists a
countable set $(x'_n)_{n \in \mathbb{N}}$, $x'_n \in F'$ $\forall n
\in \mathbb{N}$, such that 
$$
\forall x \in F, ||x|| = \sup\limits_n |\langle x'_n, x\rangle| 
$$

(See Hille and Phillips \cite{key1}, p.34, theorem 2.8.5). Therefore,
if $w \not\in N_1$,
$$
|f| (w) = \sup\limits_n |\langle x'_n, f(w)\rangle|
$$
where $|f|(w)$ is the norm of the element $f(w)$.

Since $\forall x' \in E'$, $\forall_\lambda w$, $\langle x', f(w)
\rangle = 0$, we can find a set $N_2 \in \mathscr{C}$ with
$\lambda(N_2) =0$ such that if $w \not\in N_2$, $\langle x'_n,
f(w)\rangle = 0 \; \; \forall n \in \mathbb{N}$. 

Therefore, if $w \not\in N_1 \cup N_2$, 
$$
|f| (w) = \sup\limits_n |\langle x'_n, f(w)\rangle| = 0. 
$$

This shows that  \quad $ \forall_\lambda w, \; f(w) = 0$. 


\section[Another way of proving the existence...]{Another way of proving 
the existence theorem of conditional
  expectations for Banach space valued integrable
  functions}\label{part1:chap1:sec7} 

The\pageoriginale existence of conditional expectation for Banach
space valued integrable functions can be proved in the following way
also. 

As before, let $(\Omega, \mathscr{O}, \lambda)$ be a measure
space. Let $\mathscr{C}$ be a $\sigma$-algebra contained in
$\hat{\mathscr{O}}_\lambda$. Let $\lambda$ restricted to $\mathscr{C}$
be $\sigma$-finite. Let $E$ be a Banach space over the real
numbers. Let $E'$ be the topological dual of $E$. If $x' \in E'$ and
$x \in E$, $\langle x', x\rangle $ will stand for the value of $x'$ at
$x$. If $f$ is a function on $\Omega$ with values in $E$ and if $\xi$
is a function on $\Omega$ with values in $E', \langle \xi, f\rangle$
will stand for the real valued function on $\Omega$ associating to
each $w \in \Omega$, the real number $\langle \xi (w),
f(w)\rangle$. If $f$ and $\xi$ are as above, $|f|$ (resp. $|\xi|$)
will denote the function on $\Omega$ associating to each $w \in
\Omega$, the norm in $E$ of the element $f(w)$ (resp. the norm in $E'$
of the element $\xi(w)$). 

Let $f$ be a function on $\Omega$ with values in $E$ such that $f
(\Omega)$ is contained in a finite dimensional subspace of $E$. We
call such a function a \textit{finite dimensional valued
  function}. Let $F$ be a finite dimensional subspace of $E$
containing $f(\Omega)$ and let $e_1, e_2, \ldots, e_n$ be a basis for
$F$. Then there exist $n$ real valued functions $\alpha_1, \alpha_2,
\ldots, \alpha_n$ on $\Omega$ such that $\forall w \in \Omega$, $f(w)
= \sum\limits^n_{i=1} \alpha_i (w) e_i$. If $\mathcal{S}$ is any
$\sigma$-algebra on $\Omega$, it is clear that $f \in \mathcal{S}$ if
and only if $\forall i = 1,2, \ldots n$, $\alpha_i \in \mathcal{S}$
and that if $f \in \mathcal{S}$ and if $\mathcal{S} \subset
\hat{\mathscr{O}}_\lambda$, then $f$ is $\lambda$-integrable if and
only if $\forall i = 1, \ldots n$, $\alpha_i$ is
$\lambda$-integrable. If $f$ is $\lambda$-integrable, then a
conditional expectation $f^\mathscr{C}$ of $f$ with respect to
$\mathscr{C}$ is defined as $f^\mathscr{C}(w) = \sum\limits^n_{i=1}
\alpha^\mathscr{C}_i (w) e_i$ $\forall w \in \Omega$, where $\forall i
= 1, \ldots n$, $\alpha^\mathscr{C}_i$ is a conditional expectation of
$\alpha_i$ with respect to $\mathscr{C}$. Note that
$\alpha^\mathscr{C}_i$ exist $\forall i = 1, \ldots, n$. It can be
easily seen that this definition is independent of the choice of the
basis of $F$ and\pageoriginale also in independent of the finite
dimensional subspace of $E$ containing $f(\Omega)$. 

\begin{rem}\label{part1:chap1:rem5}
If $f$ is a finite dimensional valued $\lambda$-integrable function
on $\Omega$ with values in $E$ and if $\xi$ is any function on
$\Omega$ with values in $E'$ such that $\xi \in \mathscr{C}$ (on $E'$
we always consider the Borel $\sigma$-algebra of the strong topology)
and $\xi$ is bounded in the sense that $\sup\limits_{w\in\Omega} |\xi|
(w)$ is a real number, then, 
$$
\forall_\lambda w, \langle \xi, f\rangle^\mathscr{C} (w) = \langle
\xi, f^\mathscr{C} \rangle  (w). 
$$

This is a consequence of the property (iv) of conditional expectations
of extended real valued functions listed in \S\ \ref{part1:chap1:sec5}
of this chapter.  
\end{rem}

The alternative proof of the existence of conditional expectations of
Banach space valued functions depends on the following theorem. 

\begin{thm}\label{part1:chap1:thm6}
Let $f$ be a finite dimensional valued $\lambda$-integrable function
on $\Omega$ with values in a Banach space $E$. Then,
$$
\forall_\lambda w, |f^\mathscr{C}| (w) \leq |f|^\mathscr{C} (w). 
$$

To prove this theorem, we need the following lemma. 
\end{thm}

\begin{lem}\label{part1:chap1:lem7}
Let $\mathcal{S}$ be a $\sigma$-algebra on $\Omega$. 
\begin{itemize}
\item[{\rm (i)}] Let $f$ be a step function on $\Omega$ with values in
  $E$, belonging to $\mathcal{S}$. Then there exists a function $\xi:
  \Omega \to E'$, $\xi \in\mathcal{S}$, $|\xi| \leq 1$ such that
  $\forall w \in \Omega$, $\langle \xi (w),f(w) \rangle = |f|(w)$. 

\item[{\rm (ii)}] If $\mathcal{S} \subset \hat{\mathscr{O}}_\lambda$
  and if $f$ is any $\lambda$-integrable function on $\Omega$ with
  values in $E$ belonging to $\mathcal{S}$, then there exists a
  sequence $(\xi_n)_{n \in \mathbb{N}}$ of functions on $\Omega$ with
  values in $E', \xi_n \in \mathcal{S}$ and $|\xi_n| \leq 1 \;
  \forall$ $n \in \mathbb{N}$, such that 
$$
\forall_\lambda w, \lim\limits_{n \to \infty} \langle \xi_n (w), \;
f(w)\rangle  = |f| (w). 
$$
\end{itemize}\pageoriginale
\end{lem}

\begin{proof}
\begin{itemize}
\item[{\rm (i)}] Let $f$, a step function be of the form $\sum\limits^n_{i=1}
\chi_{A_i} x_i$, $A_i \in \mathcal{S} $ $\forall i = 1, \ldots , n$,
$A_i \cap A_j = \emptyset$ if $i \neq j$, $x_i \in E$ $\forall i = 1,
\ldots ,n$. By Hahn-Banach theorem, there exists $\forall i$, $i = 1,
\ldots, n$, an element $\xi_i \in E'$ such that $\langle \xi_i,
x_i\rangle = ||x_i||$ and $||\xi_i|| \leq 1$. Let $\xi =
\sum\limits^n_{i=1} \chi_{A_i} \cdot x_i$. Then it is easily seen that
$\xi$ has all the required properties. 

\item[{\rm (ii)}] Let $f$ be an arbitrary $\lambda$-integrable
  function belonging to $\mathcal{S}$. Then there exist a sequence
  $(f_n)_{n \in \mathbb{N}}$ of step functions belonging to
  $\mathcal{S}$ such that 
$$
\forall_\lambda w, \; |f_n - f| (w) \to 0 \text{ as } n \to \infty. 
$$

By (i) $\forall n$, $\exists \xi_n : \Omega \to E'$, $\xi_n \in
\mathcal{S}$, $|\xi_n| \leq 1$ such that $\langle \xi_n (w),\break \; f_n
(w)\rangle = |f_n| (w) \forall w \in \Omega$. 
\begin{align*}
& \Big| \langle \xi_n (w), f(w) \rangle -|f|(w)\Big|\\
& \quad = \Big| \langle \xi_n (w), \; f(w) - f_n(w)\rangle + \langle
\xi_n (w), f_n (w)  \rangle - |f| (w) \Big|\\
& \quad = \Big| \langle \xi_n(w) , f(w) - f_n (w)\rangle + |f_n |(w) -
|f|(w)\Big| \\
& \quad \leq |f- f_n| (w) + |f_n -f| (w)\\
& \quad =  2 |f-f_n| (w). 
\end{align*}
\end{itemize}

Hence, $\forall _\lambda w$, $\lim\limits_{n \to \infty} \langle \xi_n
(w), f(w)\rangle $ exists and is equal to $|f|(w)$. 
\end{proof}

\medskip
\noindent{\textbf{Proof of the theorem 6.}}
Applying the above lemma (\ref{part1:chap1},
\S\ \ref{part1:chap1:sec7}, \ref{part1:chap1:lem7}) to $f^\mathscr{C}$, we see that
since $f^\mathscr{C} \in \mathscr{C}$, there exists a sequence
$(\xi_n)_{n \in \mathbb{N}}$ of functions on $\Omega$ with values in
$E'$ such that $\xi_n \in \mathscr{C} $ $\forall n \in \mathbb{N}$,
$|\xi_n | \leq 1$ $\forall n \in \mathbb{N}$ and $\forall_\lambda w$,
$\lim\limits_{n \to \infty} \langle \xi_n (w), f^\mathscr{C}
(w)\rangle = |f^\mathscr{C}|(w)$. 


By\pageoriginale remark (\ref{part1:chap1},
\S\ \ref{part1:chap1:sec7}, \ref{part1:chap1:rem5}),  
$$
\forall \; n \in \mathbb{N}, \; \forall_\lambda w, \langle \xi_n (w),
f^\mathscr{C} (w)\rangle = \langle \xi_n, f\rangle^\mathscr{C} (w). 
$$

Therefore, $\forall \; n \in\mathbb{N}$, $\forall_\lambda w$,
$|\langle \xi_n (w) , f^\mathscr{C} (w)\rangle | = | \langle \xi_n ,
f\rangle^\mathscr{C} (w)| $  and $\forall\; n \in \mathbb{N}$,
$\forall_\lambda w$, $|\langle \xi_n, f\rangle^\mathscr{C}| (w) \leq
|\langle \xi_n , f\rangle |^\mathscr{C} (w)$ by property (ii) of the
conditional expectations of extended real valued functions, listed in
\S\ \ref{part1:chap1:sec5} of this chapter. 

Now, $\forall \; n \in  \mathbb{N}$, $| \langle \xi_n, f \rangle | \leq
\xi_n | \cdot |f| \leq |f|$ and hence, $\forall \; n \in \mathbb{N}$,
$\forall_\lambda w$, $|\langle \xi_n, f\rangle |^\mathscr{C}(w) \leq
|f|^\mathscr{C}(w)$. Hence $\forall_\lambda w$, $\forall n \in
\mathbb{N}$, $| \langle \xi_n, f\rangle |^\mathscr{C} (w) \leq
|f|^\mathscr{C} (w)$. Hence, $\forall_\lambda w$, $\forall n \in
\mathbb{N}$, $\Big| \langle \xi_n (w), f^\mathscr{C} (w) \rangle
\Big| \leq |f|^\mathscr{C} (w)$. Hence, $\forall_\lambda w,
|f^\mathscr{C}|(w) \leq|f|^\mathscr{C} (w)$. 

\begin{rem}\label{part1:chap1:rem8}
Actually, one can prove that if $f$ is a finite dimensional valued
$\lambda$-integrable function, belonging to a $\sigma$-algebra
$\mathcal{S}$ contained in $\hat{\mathscr{O}}_\lambda$, then there
exists a function $\xi$ on $\Omega$ with values in $E'$, $\xi \in
\mathcal{S}$, $|\xi | \leq 1$ such that $\forall w \in\Omega$,
$\langle \xi(w), f(w)\rangle = |f|(w)$. But this is very
difficult. Note that once this is proved, the proof of theorem (\ref{part1:chap1},
\S\ \ref{part1:chap1:sec7}, \ref{part1:chap1:thm6}) follows more easily. 
\end{rem}

Now, let us turn to the proof of the existence theorem of conditional
expectations for Banach space valued $\lambda$-integrable functions. 

Let $\mathfrak{m}$ be the vector subspace of all finite dimensional
valued functions belonging to $\hat{\mathscr{O}}_\lambda$ and
$\lambda$-integrable. Then $\mathfrak{m}$ is dense in $L^1(\Omega;
\mathscr{O};\break \lambda; E)$, as we can approximate any $f \in
L^1(\Omega; \mathscr{O}; \lambda; E)$ by step functions. Consider the
linear map $u_{\mathscr{O}, \mathscr{C}}$ from $\mathfrak{m}$ to
$L^1(\Omega; \mathscr{C}; \lambda; E)$ given by 
\begin{align*}
u_{\mathscr{O}, \mathscr{C}} (f) & = f^\mathscr{C}. \\
\int |u_{\mathscr{O}, \mathscr{C}} (f)| (w) d \lambda (w) & = \int
|f^\mathscr{C}| (w) d \lambda(w). 
\end{align*}

By\pageoriginale theorem (\ref{part1:chap1},
\S\ \ref{part1:chap1:sec7}, \ref{part1:chap1:thm6}). $\forall_\lambda
w$, 
$|f^\mathscr{C}| (w) \leq |f|^\mathscr{C} (w)$. 

Hence
\begin{align*}
\int |f^\mathscr{C}| (w) d \lambda (w) & \leq \int |f|^\mathscr{C} (w)
d \lambda (w)\\
& = \int |f| (w) d \lambda (w). 
\end{align*}

Hence, $||u_{\mathscr{O}, \mathscr{C}} (f)|| \leq ||f||$ where
$||u_{\mathscr{O}, \mathscr{C}} (f)||$ (resp. $||f||$) denotes the
norm of $u_{\mathscr{O}, \mathscr{C}} (f)$ (resp. norm of $f$) in $L^1
(\Omega; \mathscr{C}; \lambda; E)$ (resp. in $L^1 (\Omega;
\mathscr{O};\break \lambda; E)$). 

Hence $u_{\mathscr{O}, \mathscr{C}}$ is a contraction linear map and
therefore, there exists a unique extension of this map to the whole of
$L^1 (\Omega; \mathscr{O}; \lambda; E)$ which we again denote by
$u_{\mathscr{O}, \mathscr{C}}$. It can be easily seen that
$u_{\mathscr{O}, \mathscr{C}} (f)$ is a conditional expectation of $f$
with respect to $\mathscr{C}$. 

\section[A few properties of conditional...]{A few properties of conditional 
expectations of Banach space valued integrable functions}\label{part1:chap1:sec8}

As before, let $(\Omega, \mathscr{O}, \lambda)$ be a measure space,
$\mathscr{C}$ a $\sigma$-algebra contained in $\hat{O}_\lambda$ and
let $\lambda$ restricted to $\mathscr{C}$ be $\sigma$-finite. Let $E$
be a Banach space over the real numbers. 

\begin{proposition}\label{part1:chap1:prop9}
 If $f\in L^1 (\Omega; \mathscr{O};
  \lambda; E)$, then 
$$
\forall_\lambda w, |f^\mathscr{C}| (w) \leq |f|^\mathscr{C} (w). 
$$
\end{proposition}


\begin{proof}
There exists a sequence $(f_n)_{n \in \mathbb{N}}$ of step functions
belonging to $\hat{\mathscr{O}}_\lambda$ such that 
\begin{itemize}
\item[{\rm (i)}] $f_n \to f$ in $L^1(\Omega; \mathscr{O}; \lambda; E)$ 

\item[{\rm (ii)}] $\forall_\lambda w$, $f_n(w) \to f(w)$ in $E$ \quad  and 

\item[{\rm (iii)}] $\forall_\lambda w$, $f^\mathscr{C}_n(w) \to
  f^\mathscr{C}(w)$ in $E$. 
\end{itemize}

Since $f_n \to f$ in $L^1 (\Omega;  \mathscr{O}; \lambda; E)$, $|f_n|
\to |f|$ in $L^1(\Omega; \mathscr{O}; \lambda)$. Hence
$|f_n|^\mathscr{C} \to |f|^\mathscr{C}$ in $L^1 (\Omega; \mathscr{C};
\lambda)$. Therefore, there exists a subsequence $f_{n_k}$ such that
$\forall_\lambda w$, $|f_{n_k}|^\mathscr{C} (w) \to |f|^\mathscr{C}
(w) $ in $E$. Since 
$$
\forall_\lambda w, \; |f^\mathscr{C}_{n_k}| (w) \leq |f_{n_k}|^\mathscr{C}(w),
$$\pageoriginale
passing to the limit, we see that 
$$
\forall_\lambda w, |f^\mathscr{C}| (w) \leq |f|^\mathscr{C} (w). 
$$
\end{proof}

\begin{proposition}\label{part1:chap1:prop10} 
Let $f_n$ be a sequence of functions on $\Omega$, belonging to
$\hat{\mathscr{O}}_\lambda$ with values in $E$ such that $f_n$
converges to a function $f$ in the sense of the dominated convergence
theorem, i.e., $\forall_\lambda w$, $f_n (w) \to f(w)$ in $E$ and
there exists a non-negative real valued $\lambda$-integrable function
$g$ on $\Omega$ such that $\forall_\lambda w$, $\forall n \in
\mathbb{N}$, $|f_n| (w) \leq g(w)$. Then $f^\mathscr{C}_n$ converges
to $f^\mathscr{C}$ also in the sense of the dominated convergence
theorem. 
 \end{proposition}


\begin{proof}
Without loss of generality let us assume that $f\equiv 0$, and $f_n(w)
\to 0$ for all $w \in\Omega$. 

Let $c^\circ (E)$ denote the vector space of all sequences $x =
(x_n)_{n \in \mathbb{N}}$, $x_n \in E$ $\forall \; n \in \mathbb{N}$
and $x_n \to 0$ in $E$ as $n \to \infty$. Define $\forall x \in^\circ
(E)$, $||| x||| = \sup\limits_{n}||x_n||$ where $||x_n||$ is the norm
of the element $x_n$ in $E$. Then, it is easily seen that $x \to
|||x|||$ is a norm in $c^\circ(E)$ and this norm makes $c^\circ(E)$, a
Banach space. 
\end{proof}

Let $h$ be a function on $\Omega$ with values in $c^\circ(E)$. Then
there exists a sequence $(h_n)_{n \in \mathbb{N}}$ of functions on
$\Omega$ with values in $E$ such that $\forall w \in \Omega$, 
$$
h(w) = (h_1 (w), h_2 (w), \ldots, h_n(w), \ldots). 
$$

It can be easily seen that if $\mathcal{S}$ is any $\sigma$-algebra on
$\Omega$, then $h \in \mathcal{S}$ if and only if $\forall n \in
\mathbb{N}$, $h_n \in \mathcal{S}$ and that if $\mathcal{S} \subset
\hat{\mathscr{O}}_\lambda$, then $h$ is $\lambda$-integrable if and
only if $\forall n \in \mathbb{N}$, $h_n$ is
$\lambda$-integrable. Moreover, it can be easily seen that if $h
\in\hat{\mathscr{O}}_\lambda$ and is $\lambda$-integrable, then
$\forall A \in\hat{\mathscr{O}}_\lambda$. 
$$
\int\limits_A hd\lambda = (\int\limits_A h_1 d\lambda, \;
\int\limits_A h_2 d \lambda , \ldots \int\limits_A d \lambda,
\ldots). 
$$

Hence\pageoriginale if $h \in\hat{\mathscr{O}}_\lambda$ and is
$\lambda$-integrable, then 
$$
\forall_\lambda w, h^\mathscr{C} (w) = (h^\mathscr{C}_1 (w),
h^\mathscr{C}_2 (w), \ldots , h^\mathscr{C}_n (w), \ldots )
$$

Consider the sequence $(\chi_n)_{n \in \mathbb{N}}$ of functions on
$\Omega$ with values in $c^\circ(E)$ given by 
$$
(\chi_n)_m(w) = 
\begin{cases}
0 \text{ if } m < n\\
f_m (w) \text{ if } m \geq n  
\end{cases}
$$
where $(\chi_n)_m (w)$ stands for the $m^{\rm th}$ coordinate of
$\chi_n(w)$. 

Then $\forall w \in \Omega$, $\chi_n(w) \to 0 $ in $c^\circ
(E)$. $\forall \; n \in \mathbb{N}$,
$\chi_n\in\hat{\mathscr{O}}_\lambda$ and is $\lambda$-integrable since
$\forall n \in \mathbb{N}$, $\forall_\lambda w$, $|\chi_n|(w) =
\sup\limits_{m \geq n} |f_m| (w) \leq g(w)$ and $g$ is
$\lambda$-integrable. 
$$
\int |\chi_n| d \lambda = \int \sup\limits_{m \geq n} |f_m| (w) d
\lambda(w) \downarrow  0 \; \text{ as } \; n \to \infty. 
$$

Hence
$$
\chi_n \to 0 \text{ in } L^1 (\Omega; \mathscr{O}; \lambda; c^\circ
(E)). 
$$

Therefore, 
$$
\chi^\mathscr{C}_n \to 0 \text{ in } L^1 (\Omega, \mathscr{C};
\lambda; c^\circ (E)). 
$$

Let us prove that $\forall_\lambda w$, $\chi^\mathscr{C}_n (w) \to 0$
in $c^\circ (E)$. Now, $\forall_\lambda w$, $|\chi^\mathscr{C}_n|(w)$
is a decreasing function of $n$ and hence $\lim\limits_{n \to \infty}
|\chi^\mathscr{C}_n| (w)$  exists $\forall_\lambda m$. Let $d_w =
\lim\limits_{n \to \infty} |\chi^\mathscr{C}_n|(w)$ when the limit of
$|\chi^\mathscr{C}_n|(w)$ exists. 
$$
\int |\chi^\mathscr{C}_n| (w) d \lambda (w) \downarrow \int d_w d
\lambda (w), \text{ as } n \to \infty. 
$$

Since $\chi^\mathscr{C}_n \to 0$ in $L^1 (\Omega; \mathscr{C};
\lambda; c^\circ (E))$, it follows that $\int d_w d\lambda(w) = 0$ and
hence $\forall_\lambda w$, $d_w =0$. 

Hence, 
$$
\forall_\lambda w, \chi^\mathscr{C}_n(w) \to 0 \text{ in }
c^\circ(E). 
$$

Hence, 
$$
\forall_\lambda w, \sup\limits_{m \geq n} |f^\mathscr{C}_m| (w) \to 0
\text{ as } n \to \infty. 
$$

This\pageoriginale means that $\forall_\lambda w$, $f^\mathscr{C}_n(w)
\to 0$ in $E$ as $n \to \infty$. Since  $|f_n| \leq g$,
$\forall_\lambda w$, $|f_n|\mathscr{C} (w) \leq g^\mathscr{C}(w)$ and
$g^\mathscr{C}$ is $\lambda$-integrable, since $g$ is. Hence
$f^\mathscr{C}_n$ converges to zero, in the sense of the dominated
convergence theorem. 


\begin{rem}\label{part1:chap1:rem11}
When $E = \mathbb{R}$, the above proposition (\ref{part1:chap1},
\S\ \ref{part1:chap1:sec8}, \ref{part1:chap1:prop10}) is proved
in Doob \cite{key1} in the pages 23-24. Though the theory of
conditional expectations for Banach space valued functions is not used
there as is done here, the idea is essentially the same. 
\end{rem}

The properties (v), (vi) and (vii) stated in
\S\ \ref{part1:chap1:sec5} of this chapter for 
functions belonging to $L^1(\Omega; \mathscr{O}; \lambda)$ are also
true for any $f \in L^1 (\Omega; \mathscr{O}; \lambda; E)$, as can be
easily seen. Property (iv) also is true with the same assumptions on
$g$ and on $gf$ as there.

Thus, in this chapter, we have proved the existence and uniqueness of
conditional expectations for Banach space valued $\lambda$-integrable
functions on a measure space $(\Omega, \mathscr{O}, \lambda)$. We
shall see in \S\ \ref{part1:chap3:sec1} of Chapter
\ref{part1:chap3}, that the existence of 
disintegration of $\lambda$ is linked to the existence of conditional
expectations for measure valued functions on $\Omega$, as defined in
chapter \ref{part1:chap2}. 

