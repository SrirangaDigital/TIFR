
\chapter{Measure valued Functions}\label{part1:chap2}

\section[Basic definitions: Fubini's theorem...]{Basic definitions: 
Fubini's theorem for extended real valued
  integrable functions}\label{part1:chap2:sec1}

In\pageoriginale  this chapter, we shall study the measure valued
functions. 

Let $(\Omega, \mathscr{O}, \lambda)$ be a measure space. Let $(Y,
\mathscr{Y})$ be a measurable space. Let $\mathfrak{m}^+ (Y,
\mathscr{Y})$ be the set of all positive measures on $\mathscr{Y}$. 

\begin{defn}\label{part1:chap2:def12} 
A {\em measure valued function} $\nu$ on $\Omega$ with values in
$\mathfrak{m}^+ (Y,\break \mathscr{Y})$ is an assignment to each $w \in
\Omega$, a positive measure $\nu_w$ on $\mathscr{Y}$. 
\end{defn}

If $(\lambda^\mathscr{C}_w)_{w \in \Omega}$ is a disintegration of
$\lambda$ with respect to a $\sigma$-algebra $\mathscr{C} \subset
\hat{\mathscr{O}}_\lambda$, it can be considered as a measure valued
function $\lambda^\mathscr{C}$ on $\Omega$ with values in
$\mathfrak{m}^+ (\Omega, \mathscr{O})$ taking $w \in\Omega$ to
$\lambda^\mathscr{C}_w$.  

If $\nu$ is a measure valued function on $\Omega$ with values in
$\mathfrak{m}^+(Y,\mathscr{Y})$ and if $f$ is any non-negative
function on $Y$ belonging to $\mathscr{Y}$, then $\nu(f)$ will denote
the function on $\Omega$ taking $w$ to $\nu_w (f)$. If $B$ is a set
belonging to $\mathscr{Y}$, $\nu(B)$ will stand for $\nu(\chi_B)$. 

\begin{defn}\label{part1:chap2:def13}
A measure valued function $\nu$ on $\Omega$ with values in
$\mathfrak{m}^+ (Y,\break \mathscr{Y})$ is said to be {\em measurable} with
respect to a $\sigma$-algebra $\mathcal{S}$ on $\Omega$ or is said to
{\em belong} to $\mathcal{S}$ if $\forall\; B \in \mathscr{Y}$, the
extended real valued function $\nu(B)$ belongs to $\mathcal{S}$. 
\end{defn}

If $\nu$ belongs to $\mathcal{S}$, we write $\u{\nu \in
  \mathcal{S}}$. 

If $\lambda^\mathscr{C}$ is a disintegration of $\lambda$ with respect
to a $\sigma$-algebra $\mathscr{C} \subset
\hat{\mathscr{O}}_\lambda$, note that $\lambda^\mathscr{C} \in
\mathscr{C}$. 

If $\nu \in \mathcal{S}$, we see immediately that $\forall$
non-negative function $f$ on $Y$, $f \in\mathscr{Y}$, $\nu(f) \in
\mathcal{S}$.\pageoriginale 

\begin{defn}\label{part1:chap2:def14}
If $\nu$ is a measure valued function on $\Omega$ with values in
$\mathfrak{m}^+ (Y, \mathscr{Y})$, belonging to
$\hat{\mathscr{O}}_\lambda$, the {\em integral} of $\nu$ with respect
to $\lambda$ is defined as the measure $J$ on $\mathscr{Y}$ given by
$\forall B \in \mathscr{Y}$, $J(B) = \int \nu_w (B) d \lambda (w)$. It
is written as $J = \int \nu_w d \lambda (w)$. If $A \in\mathscr{O}$,
the {\em integral of $\nu$ over } $A$ is defined as the integral of
$\nu$ with respect to the measure $\chi_A \cdot \lambda$. The integral
of $\nu$ over $A$ is written as $\int\limits_A \nu_w d\lambda(w)$. 
\end{defn}

Note that if $\lambda^\mathscr{C}$ is a disintegration of $\lambda$
with respect to a $\sigma$-algebra $\mathscr{C}\subset
\hat{\mathscr{O}}_\lambda$, the integral of $\lambda^\mathscr{C}$ with
respect to $\lambda$ is $\lambda$. i.e. $\lambda = \int
\lambda^\mathscr{C}_w d \lambda (w) $. 

Note also that from our definition, it easily follows that if $J$ is
the integral of $\nu$ with respect to $\lambda$, then for every
function $f$ on $Y$, $f \geq 0$, $f \in \mathscr{Y}$, $J(f) =
\int\nu_w (f) d \lambda (w)$. Hence, $\forall f \geq 0$, $f \in
\mathscr{Y}$, $J(f)=0$ implies that $\forall_\lambda w$, $\nu_w (f) =
0$. In particular, if $B \in \mathscr{Y}$ is such that its $J$-measure
is zero, then $\forall_\lambda w $ its $\nu_w$-measure is also zero. 

Note that it also follows easily from our definition that if $f$ is an
extended real valued function on $Y$, $f \in \mathscr{Y}$ and
$J$-integrable, then $\forall_\lambda w$, it is
$\nu_w$-integrable. Moreover, the function $w \to \nu_w (f)$ (defined
arbitrarily on the set of points $w$ where $f$ is not
$\nu_w$-integrable) belongs to $\hat{\mathscr{O}}_\lambda$ and is
$\lambda$-integrable. Further, $\int \nu_w (f)d \lambda (w) = J(f)$. 

If $A \in \mathscr{O}$, note that $\int\limits_A \nu_wd \lambda (w)
(f)$  is equal to $\int\limits_A \nu_w (f) d \lambda(w)$, for every
function $f$ on $Y$, $f\geq 0$ and $f \in \mathscr{Y}$. 

The following theorem and the corollary (\ref{part1:chap2},
\S\ \ref{part1:chap2:sec1}, \ref{part1:chap2:coro16}) contains as a
special case, as we shall see towards the end of this chapter, the
usual Fubini's theorem. Hence we shall call this also as Fubini's
theorem. 

\begin{thm}[Fubini]\label{part1:chap2:thm15}
Let\pageoriginale  $\mathcal{S}$ be a $\sigma$-algebra contained in
$\hat{\mathscr{O}}_\lambda$. Let $\nu$ be a measure valued function on
$\Omega$ with values in $\mathfrak{m}^+ (Y, \mathscr{Y})$ belonging to
$\mathcal{S}$ and with integral $J$. Let $f$ be an extended real
valued function on $Y$, $f \geq 0$ belonging to
$\hat{\mathscr{Y}}_J$. Then, 
\begin{itemize}
\item[{\rm (i)}] $\forall_\lambda w$, $f$ is $\nu_w$-measurable, i.e.,
   $f \in \hat{\mathscr{Y}}_{\nu_w}$. 

\item[{\rm (ii)}] The function $w \to \nu_w (f)$ (defined arbitrarily
  on the set of $w \in\Omega$ for which $f$ is not $\nu_w$-measurable)
  $\in \hat{\mathcal{S}}_\lambda$ and 

\item[{\rm (iii)}] \qquad $\int\nu_w (f) d \lambda (w) = J(f)$. 
\end{itemize}
\end{thm}

\begin{proof}
\begin{itemize}
\item[{\rm (i)}] Since $f \geq 0$ belongs to $\hat{\mathscr{Y}}_J$,
  there exist functions $f_i$ on $Y, f_i \geq 0$, $f_i \in
  \mathscr{Y}$, $i=1,2$, such that $f_1 \leq f \leq f_2$ everywhere
  and the set $B = \left\{  y \in Y \mid f_1 (y) \neq f_2(y) \right\}$
  is of $J$-measure zero. Since $J(B) = 0$, $\forall_\lambda w$,
  $\nu_w (B)  =0$. i.e., there exists a set $A \in
  \hat{\mathscr{O}}_\lambda$ such that $\lambda(A) = 0$ and if $w
  \not\in A$, $\nu_w (B) = 0$. Since $f_1 \in\mathscr{Y}$, it follows
  therefore that if $w \not\in A$, $f $ is  $\nu_w$-measurable i.e.,
  $f \in \hat{\mathscr{Y}}_{\nu_w}$. 

\item[{\rm (ii)}] Since for $i = 1,2, w \to \nu_w (f_i) \in
  \mathcal{S}$, it follows that $w \to \nu_w (f) \in
  \hat{\mathcal{S}}_\lambda$. 

\item[{\rm (iii)}] 
\begin{tabbing}
$\int \nu_w (f) d \lambda (w)$ \= = \= $\int \nu_w(f_i) d\lambda(w)$\\
\> = \> $J(f_i)$\\
\> = \> $J(f)$
\end{tabbing}
\end{itemize}
\end{proof}

\begin{corollary}[Fubini]\label{part1:chap2:coro16}
With $\mathcal{S}$, $\nu$ and $J$ as in the above theorem, if $f$ is a
$J$-integrable extended real valued function on $Y$, $f \in
\hat{\mathscr{Y}}_J$, then 
\begin{itemize}
\item[{\rm (i)}] $\forall_\lambda w$, $f$ is $\nu_w$-measurable,
  i.e., $f \in\hat{\mathscr{Y}}_{\nu_w}$, and is $\nu_w$-integrable 

\item[{\rm (ii)}] the function $w \to \nu_w (f)$ (defined arbitrarily
  on the set of $w \in \Omega$ for which $f$ is not
  $\nu_w$-integrable), belongs to $\hat{\mathcal{S}}_\lambda$ and is
  $\lambda$-integrable, and 

\item[{\rm (iii)}] \qquad $\int \nu_w (f) d\lambda(w) = J(f)$. 
\end{itemize}
\end{corollary}

\begin{proof}
If $f = f^+ - f^-$ with the usual notation, then the corollary
immediately follows from the above theorem by applying it to $f^+$ and
$f^-$. 
\end{proof}

\begin{corollary}\label{part1:chap2:coro17}
Let\pageoriginale $\mathscr{C}$ be a $\sigma$-algebra contained in
$\hat{\mathscr{O}}_\lambda$. Let $(\lambda^\mathscr{C}_w)_{w \in
  \Omega}$ be a disintegration of $\lambda$ with respect to
$\mathscr{C}$. Let $f$ be an extended real valued function on
$\Omega$, $f \geq 0$ (resp. $f$ $\lambda$-integrable) belonging to
$\hat{\mathscr{O}}_\lambda$. Then,
\begin{itemize}  
\item[{\rm (i)}] $\forall_\lambda w$, $f$ is
  $\lambda^\mathscr{C}_w$-measurable i.e. $f \in
  \hat{\mathscr{O}}_{\lambda^\mathscr{C}_w} $ (resp. $f$ is
  $\lambda^\mathscr{C}_w$-measurable and
  $\lambda^\mathscr{C}_w$-integrable). 

\item[{\rm (ii)}] the function $w \to \lambda^\mathscr{C}_w (f)$
  \{defined  arbitrarily on the set of $w \in \Omega$, for which $f$
  is not $\lambda^\mathscr{C}_w$-measurable (resp. $f$ is not
  $\lambda^\mathscr{C}_w$-integrable)\} belongs to
  $\hat{\mathscr{C}}_\lambda$ and is $\lambda$-integrable and 

\item[{\rm (iii)}] \qquad $\int \lambda^\mathscr{C}_w (f) d \lambda
  (w) = \lambda (f)$. 

\end{itemize}
\end{corollary}

\begin{proof}
When $f$ is $\geq 0$ (resp. $f$ is $\lambda$-integrable), this follows
immediately from theorem (\ref{part1:chap2},
\S\ \ref{part1:chap2:sec1}, \ref{part1:chap2:thm15}) (resp. from the
above 
Corollary (\ref{part1:chap2}, \S\ \ref{part1:chap2:sec1},
\ref{part1:chap2:coro16})) by taking $\lambda^\mathscr{C}$ instead of 
$\nu$ and observing that $\lambda^\mathscr{C} \in \mathscr{C}$ and
$\int \lambda^\mathscr{C}_w d \lambda (w) = \lambda $. 
\end{proof}

We saw is \S\ \ref{part1:chap1:sec3} of Chapter \ref{part1:chap1}, how
the existence of a disintegration 
$(\lambda^\mathscr{C}_w)_{w \in \Omega}$ of $\lambda$ with respect to
$\mathscr{C}$ implies immediately the existence of conditional
expectations for functions $f \geq 0$ on $\Omega$, $f \in\mathscr{O}$
and that $w \to \lambda^\mathscr{C}_w (f)$ is a conditional
expectation of $f$ with respect to $\mathscr{C}$. If $f$ is  $\geq 0$
on $\Omega$ and belongs to $\hat{\mathscr{O}}_\lambda$, we see from
the above Corollary (\ref{part1:chap2}, \S\ \ref{part1:chap2:sec1},
\ref{part1:chap2:coro17}) that $\forall_\lambda w$, $f$ is 
$\lambda^\mathscr{C}_w $-measurable and the function $w \to
\lambda^\mathscr{C}_w (f)$ belongs to $\hat{\mathscr{C}}_\lambda$. It
can be easily checked that the function $w \to
\lambda^\mathscr{C}_\lambda (f)$ is actually a conditional expectation
of $f$ with respect to $\hat{\mathscr{C}}_\lambda$ and hence is almost
everywhere equal to a conditional expectation of $f$ with respect to
$\mathscr{C}$. Similar result holds again when $f \in
\hat{\mathscr{O}}_\lambda$ and is $\lambda$-integrable. 

Thus, we see how the existence of a disintegration of $\lambda$ with
respect to a $\sigma$-algebra $\mathscr{C}$ contained in
$\hat{\mathscr{O}}_\lambda$, implies as well the existence of
conditional expectations with respect to $\mathscr{C}$ for 
$\lambda$-integrable extended real valued functions belonging to
$\hat{\mathscr{O}}_\lambda$. 

In \S\ \ref{part1:chap2:sec2},\pageoriginale we shall extend the
results of this section 
to Banach space valued $\lambda$-integrable functions. 

\section[Fubini's theorem for Banach space...]{Fubini's theorem for
  Banach space valued integrable 
  functions}\label{part1:chap2:sec2}

Throughout this section, we fix a measure space $(\Omega, \mathscr{O},
\lambda)$, a $\sigma$-algebra $\mathcal{S} \subset
\hat{\mathscr{O}}_\lambda$, a measurable space $(Y, \mathscr{Y})$, a
measure valued function $\nu$ on $\Omega$ with values in
$\mathfrak{m}^+ (Ym,  \mathscr{Y})$ belonging to $\mathcal{S}$, having
an integral $J$ with respect to $\lambda$ and a Banach space $E$ over
the real numbers. 

\begin{proposition}\label{part1:chap2:prop18}
Let $g$ be a step function on $Y$ with values in $E$ belonging to
$\hat{\mathscr{Y}}_J$ (resp. $\mathscr{Y}$). Let further, $g$ be
$J$-integrable. 

Then 
\begin{itemize}
\item[{\rm (i)}] $\forall_\lambda w$, $g \in \hat{\mathscr{Y}}_{\nu_w}
  $ and is $\nu_w$-integrable (resp. $\forall_\lambda w$, $g$ is
  $\nu_w$-integrable). 

\item[{\rm (ii)}] The function $w \to\nu_w(g)$ with values in $E$
  (defined arbitrarily, on the set of points $w \in \Omega$ where $g$
  is not $\nu_w$-integrable, in such a way that $\forall w$, $\nu_w
  (g)$ is still an element of $E$) belongs to
  $\hat{\mathcal{S}}_\lambda$ (resp. belongs to $\mathcal{S}$) and is
  $\lambda$-integrable and 

\item[{\rm (iii)}] \qquad $\int \nu_w (g) d \lambda(w) = J(g)$. 
\end{itemize}
\end{proposition}


\begin{proof}
\begin{itemize}
\item[{\rm (i)}] Let $g = \sum\limits^n_{i=1} \chi_{A_i} \cdot x_i$
  where for $i = 1, \ldots n$, $A_i \in \hat{\mathscr{Y}}_J$
  (resp. $A_i \in \mathscr{Y}$), $A_i \cap A_j = \emptyset$ if $ i
  \neq j$ and $x_i \in E$. By theorem (\ref{part1:chap2},
  \S\ \ref{part1:chap2:sec1}, \ref{part1:chap2:thm15}), $\forall i$, 
  $\forall_\lambda w$, $A_i \in \hat{\mathscr{Y}}_{\nu_w}$. Hence
  $\forall_\lambda w$, $\forall i$, $A_i \in
  \hat{\mathscr{Y}}_{\nu_w}$. Therefore, $\forall_\lambda w$, $g
  \in\hat{\mathscr{Y}}_{\nu_w}$.  

 Since $g$ is $J$-integrable, $\forall i$, $J(A_i)< +
  \infty$. Hence $\forall_\lambda w$, $\forall i$,
  $\nu_w(A_i)<+\infty$. Hence $\forall_\lambda w$, $g$ is
  $\nu_w$-integrable. 

\item[{\rm (ii)}] Let $A = \{ w \in \Omega \mid g  \text{ is }
  \nu_w - \text{ integrable }  \}$.  By (i) $\lambda$ is carried by $A$
  and $\forall w \in A$,
$$
\nu_w (g) = \sum\limits^n_{i=1} \nu_w (A_i) x_i
$$ 
$\forall n \in \mathbb{N}$, $\forall w \in \Omega$, let $g_n(w) =
\sum\limits^n_{i=1} \inf (\nu_w (A_i), n) x_i$. 

Then,\pageoriginale $\forall n \in \mathbb{N}$ $g_n$  is a finite
dimensional valued function belonging to $\hat{\mathcal{S}}_\lambda$
(resp. belonging to $\mathcal{S}$). Further, $g_n(w) \to \nu_w(g)$ in
$E \;\; \forall w \in A$. Since $\lambda$ is carried by $A$ and since,
$\forall n \in \mathbb{N} \; g_n \in\hat{\mathcal{S}}_\lambda$
(resp. $g_n \in \mathcal{S}$ and $A \in\mathcal{S}$) it follows that
the function $w \to \nu_w (g)$ belongs to $\hat{\mathcal{S}}_\lambda$
(resp. belongs to $\mathcal{S}$). 

$\forall i$, $w \to \nu_w (A_i)$ is $\lambda$-integrable, since
$J(A_i) < + \infty$, and hence $w \to \nu_w (g)$ is
$\lambda$-integrable, since for $w \in A$, 
$$
|\nu_w(g)| \leq \sum\limits^n_{i=1} | \nu_w (A_i) | \cdot || x_i ||. 
$$


\item[{\rm (iii)}] 
\begin{tabbing}
$\int \nu_w (g) d \lambda (w)$ \= = \=  $\int (\sum\limits^n_{i=1} \nu_w (A_i)
x_i) d \lambda(w)$\\
\> = \> $\sum\limits^n_{i=1} \int \nu_w(A_i) d \lambda (w) \cdot x_i$\\
\> = \> $\sum\limits^n_{i=1} J(A_i) x_i$ \\
\> = \> $J(g)$
\end{tabbing}
\end{itemize}
\end{proof}

\begin{corollary}\label{part1:chap2:coro19}
Let $\mathscr{C}$ be a $\sigma$-algebra contained in
$\hat{\mathscr{O}}_\lambda$ and let $(\lambda^\mathscr{C}_w)_{w \in
  \Omega}$ be a disintegration of $\lambda$ with respect to
$\mathscr{C}$. Let $g$ be a step function on $\Omega$ with values in
$E$, $g \in \hat{\mathscr{O}}_\lambda$ (resp. $g \in \mathscr{O}$) and
$\lambda$-integrable. Then 
\begin{itemize}
\item[{\rm (i)}] $\forall_\lambda w$, $g
  \in\hat{\mathscr{O}}_{\lambda^\mathscr{C}_w}$ and is
  $\lambda$-integrable (resp. $\forall_\lambda w$, $g$ is
  $\lambda^\mathscr{C}_w$-integrable). 

\item[{\rm (ii)}] The function $w \to \lambda^\mathscr{C}_w (g)$ with
  values in $E$ (defined arbitrarily, on the set of points $w \in
  \Omega$ where $g$ is not $\lambda^\mathscr{C}_w$-integrable, in such
  a way that $\forall w \in \Omega$, $\lambda^\mathscr{C}_w (g)$ is
  still an element of $E$) belongs to $\hat{\mathscr{C}}_\lambda$
  (resp. belongs to $\mathscr{C}$) and is $\lambda$-integrable and 

\item[{\rm (iii)}] \qquad $\int \lambda^\mathscr{C}_w (g) d \lambda(w)
  = \lambda (g)$. 
\end{itemize}
\end{corollary}

\begin{proof}
This follows immediately from the above proposition
(\ref{part1:chap2}, \S\ \ref{part1:chap2:sec2},
\ref{part1:chap2:prop18}), 
by taking $\lambda^\mathscr{C}$ instead of $\nu$ and observing that
$\lambda^\mathscr{C} \in \mathscr{C}$ and $\int\lambda^\mathscr{C}_w\break d
\lambda (w) = \lambda$. 
\end{proof}

In\pageoriginale the following theorem, let us consider the case of an
arbitrary $J$-integrable function on $Y$ with values in $E$. Since
this theorem contains as a special case, the usual Fubini's theorem,
as we shall see below, we call it also as Fubini's theorem. 

\begin{thm}[Fubini]\label{part1:chap2:thm20}
Let $f$ be a function on $Y$ with values in $E, f \in
\hat{\mathscr{Y}}_J$ (resp. $f \in \mathscr{Y}$) and
$J$-integrable. Then,
\begin{itemize}
\item[{\rm (i)}] $\forall_\lambda w$, $f \in\hat{\mathscr{Y}}_{\nu_w}$
  and is $\nu_w$-integrable (resp. $\forall_\lambda w$, $f$ is
  $\nu_w$-integrable). 

\item[{\rm (ii)}] The function $w \to \nu_w (f)$ on $\Omega$ with
  values in $E$ (defined arbitrarily on the set of points $w \in
  \Omega$ where $f$ is not $\nu_w$-integrable) belongs to
  $\hat{\mathcal{S}}_\lambda$ (resp. belongs to $\mathcal{S}$) and is
  $\lambda$-integrable  and 

\item[{\rm (iii)}] \qquad $\int \nu_w (f) d\lambda (w) = J(f)$. 
\end{itemize}
\end{thm}

\begin{proof}
Since $f \in \hat{\mathscr{Y}}_J$ (resp. $\mathscr{Y}$) and is
$J$-integrable, there exists a sequence $(g_n)_{n\in \mathbb{N}}$ of
step functions on $Y$ with values in $E$ and a non-negative real
valued function $g$ on $Y$ belonging to $\mathscr{Y}$ and
$J$-integrable such that $\forall n$, $g_n \in \hat{\mathscr{Y}}_J$
(resp. $\forall n$, $g_n \in \mathscr{Y}$) and $J$-integrable,
$\forall_J y$, $|g_n |(y) \leq g(y)$, $\forall_J y$, $g_n(y) \to f(y)$
in $E$ and $J(g_n) \to J(f)$ in $E$. 
\begin{itemize}
\item[{\rm (i)}] By proposition (\ref{part1:chap2},
  \S\ \ref{part1:chap2:sec2}, \ref{part1:chap2:prop18}), $\forall n$, 
  $\forall_\lambda w$, $g_n \in \hat{\mathscr{Y}}_{\nu_w}$. Since
  $\forall_J y$, $g_n (y) \to f(y)$ in $E$, $\forall_\lambda w$,
  $\forall_{\nu_w} y$, $g_n(y) \to f(y)$ in $E$. Hence,
  $\forall_\lambda w$, $f \in \hat{\mathscr{Y}}_{\nu_w}$. Further, by
  the same proposition (\ref{part1:chap2}, \S\ \ref{part1:chap2:sec2},
  \ref{part1:chap2:prop18}), $\forall n$, $\forall_\lambda 
  w $, $g_n$ is $\nu_w$-integrable. Hence, $\forall_\lambda w$,
  $\forall n $, $g_n$ is $\nu_w$-integrable. 

Also $\forall_\lambda w$, $g$ is $\nu_w$-integrable.

Since $\forall_J y$, $\forall n$, $|g_n| (y) \leq g(y)$, we have
$\forall_\lambda w$, $\forall_{\nu_w} y$, $\forall n$, $|g_n| (y) \leq
g(y)$ and hence, 
$$ 
\forall_\lambda w, \forall n \in \mathbb{N}, \; \nu_w (|g_n|) \leq
\nu_w (g) < + \infty.  
$$

Hence, by Fatou's lemma, $\forall_\lambda w$, $f$ is
$\nu_w$-integrable, and by the dominated  convergence\pageoriginale
theorem, 
$$
\forall_\lambda w, \nu_w (g_n) \to \nu_w (f) \text{ in } E. 
$$

\item[{\rm (ii)}] By proposition (\ref{part1:chap2},
  \S\ \ref{part1:chap2:sec2}, \ref{part1:chap2:prop18}), $\forall n
  \in 
  \mathbb{N}$, the function $w \to \nu_w (g_n)$ belongs to
  $\hat{\mathcal{S}}_\lambda$ (resp. belongs to $\mathcal{S}$) and
  since $\forall_\lambda w$, $\nu_w (g_n) \to \nu_w (g)$ in $E$
  (resp. since the set of $w$ where $\nu_w (g_n)$ converges to $\nu_w
  (f)$ belongs to $\mathcal{S}$ and carries $\lambda$) it follows that
  $w \to \nu_w(f)$ also belongs to $\hat{\mathcal{S}}_\lambda$
  (resp. belongs to $\mathcal{S}$). 

Since $\forall_\lambda w$,  $\nu_w(g_n) \to \nu_w (f)$ in $E$ and
since $\forall_\lambda w$, $\forall n \in \mathbb{N}$, $|\nu_w (g_n)|$
$\leq \nu_w (|g_n|) \leq \nu_w(g)$ and since $w \to \nu_w(g)$ is
$\lambda$-integrable, it follows by Fatou's lemma again that $w \to
\nu_w (f)$  is $\lambda$-integrable and again by the dominated
convergence theorem, 
$$
\int \nu_w (g_n) d\lambda(w) \to \int \nu_w (f) d \lambda(w). 
$$

\item[{\rm (iii)}] \qquad $J(g_n) \to J(f) $ in $E$. 
\end{itemize}

But $\forall n \in \mathbb{N}$, $J(g_n) =\int \nu_w (g_n) d\lambda(w)$
by proposition (\ref{part1:chap2}, \S\ \ref{part1:chap2:sec2},
\ref{part1:chap2:prop18}). Since $\int \nu_w (g_n) d\lambda (w)$ 
converges to $\int \nu_w (f) d\lambda (w)$, it follows that 
\begin{equation*}
J(f) = \int\nu_w (f) d\lambda(w). 
\end{equation*}
\end{proof}

\begin{corollary}\label{part1:chap2:coro21}
Let $\mathscr{C}$ be a $\sigma$-algebra contained in
$\hat{\mathscr{O}}_\lambda$, and let
$(\lambda^\mathscr{C}_w)_{w\in\Omega}$ be a disintegration of
$\lambda$ with respect to $\mathscr{C}$. Let $f$ be a function on
$\Omega$ with values in $E$, $f \in \hat{\mathscr{O}}_\lambda$
(resp. $f \in \mathscr{O}$) and $\lambda$-integrable. Then 
\begin{itemize}
\item[{\rm (i)}] $\forall_\lambda w$, $f
  \in\hat{\mathscr{O}}_{\lambda^\mathscr{C}_w}$ and is
  $\lambda^\mathscr{C}_w$-integrable

\item[{\rm(ii)}] The function $w \to \lambda^\mathscr{C}_w(f)$ with
  values in $E$ (defined arbitrarily on the set of $w \in\Omega$ where
  $f$ is not $\lambda^\mathscr{C}_w$-integrable in such a way that
  $\lambda^\mathscr{C}_w (f)$ still takes values in $E \forall w
  \in\Omega$) belongs to $\hat{\mathscr{C}}_\lambda$ (resp. belongs to
  $\mathscr{C}$) and is $\lambda$-integrable. and 

\item[{\rm (iii)}] \qquad $\int \lambda^\mathscr{C}_w (f) d \lambda
(w) = \lambda (f)$. 
\end{itemize}
\end{corollary}

\begin{proof}
This\pageoriginale follows immediately from the above theorem (\ref{part1:chap2},
\S\ \ref{part1:chap2:sec2}, \ref{part1:chap2:thm20}) by applying it to
$\lambda^\mathscr{C}$ instead of $\nu$ 
and observing that $\lambda^\mathscr{C} \in \mathscr{C}$  and
$\int\lambda^\mathscr{C}_w\break d\lambda(w) = \lambda$.
\end{proof}

We shall now deduce the usual Fubini's theorem from the above theorem
(\ref{part1:chap2}, \S\ \ref{part1:chap2:sec2}, \ref{part1:chap2:thm20}).

Let $(X, \mathfrak{X}, \mu)$  and $(Z, \mathfrak{z}, \nu)$ be two
measure spaces with $\mu$ (resp. $\nu$) $\sigma$-finite on
$\mathfrak{X}$ (resp. on $\mathfrak{z}$). Let $\forall x \in X$,
$\delta_x \otimes \nu$ be the product measure of $\delta_x$ and $\nu$
on the $\sigma$-algebra $\mathfrak{X} \otimes \mathfrak{z}$ on the set
$X \times Z$. Then $x \to \delta_x\otimes \nu$ is a measure valued
function on $X$ with values in $\mathfrak{m}^+ (X \times Z,
\mathfrak{X}\otimes \mathfrak{z})$, belonging to $\mathfrak{X}$ and
has the measure $\mu \otimes \nu$ for its integral with respect to
$\mu$.

Let $f$ be a function on $X \times Z$ with values in a Banach space
$E$, $f \in \widehat{\mathfrak{X} \hat{\otimes} \mathfrak{z}}_{\mu \otimes
  \nu}$. Then, by the above theorem (\ref{part1:chap2},
\S\ \ref{part1:chap2:sec2}, \ref{part1:chap2:thm20}).  
\begin{itemize}
\item[{\rm (i)}] $\forall_\lambda x$, $f \in \; \mathfrak{X}
  \hat{\otimes} \mathfrak{z}_{\delta_x \otimes \nu}$ and is $\delta_x
  \otimes \nu$ integrable; i.e., $\forall_\mu x$, the function $z \to
  f (x, z)$ belongs to $\hat{\mathfrak{z}}_\nu$ and is
  $\nu$-integrable and 
$$
\forall_\nu x, \int |f| (x, z) d \nu (z) < + \infty. 
$$

\item[{\rm (ii)}] $x \to \delta_x \otimes \nu(f)$ belongs to
  $\hat{\mathfrak{X}}_\mu$ and is $\mu$-integrable i.e., $x \to \int f
  (x, z)\break d \nu(z)$ is $\mu$-integrable and 

\item[{\rm (iii)}] \qquad $\int (\delta_x \otimes \nu) (f) d\mu (x) =
  \int f d \mu \otimes \nu$ 
$$
i.e. \int (\int f(x, z) d\nu (z)) d\mu (x) = \int f(x, z) d\mu \otimes
\nu (x,z). 
$$
\end{itemize}

When $E= \mathbb{R}$, this is the usual Fubini's theorem.

It is now clear how to deduce the Fubini's theorem for functions which
are non-negative (resp. integrable) and extended real valued from
theorem (\ref{part1:chap2}, \S\ \ref{part1:chap2:sec1},
\ref{part1:chap2:thm15}) (resp. Corollary (\ref{part1:chap2},
\S\ \ref{part1:chap2:sec1}, \ref{part1:chap2:coro16})).   


Form Corollary (\ref{part1:chap2}, \S\ \ref{part1:chap2:sec2},
\ref{part1:chap2:coro21}), we see that if $f$ is a $J$-integrable 
function on $Y$ with values in a Banach space $E$, $f \in
\hat{\mathscr{Y}}_J$ (resp. $f \in\mathscr{Y}$) and if $\mathscr{C}$
is a $\sigma$-algebra\pageoriginale contained in
$\hat{\mathscr{O}}_\lambda$ and if $(\lambda^\mathscr{C}_w)_{w \in
  \Omega}$ is a disintegration of $\lambda$ with respect to
$\mathscr{C}$, then the almost everywhere defined function $w \to
\lambda^\mathscr{C}_w (f)$ belongs to $\hat{\mathscr{C}}_\lambda$
(resp. $\mathscr{C}$) and is $\lambda$-integrable. One can easily
check that $w \to \lambda^\mathscr{C}_w (f)$ is actually a conditional
expectation of $f$ with respect to $\hat{\mathscr{C}}_\lambda$
(resp. with respect to $\mathscr{C}$) first by considering step
functions and then extending to $f$. Thus $w \to
\lambda^\mathscr{C}_w(f)$ is almost everywhere equal to a conditional
expectation of $f$ with respect to $\mathscr{C}$ (resp. is a
conditional expectation of $f$ with respect to $\mathscr{C}$). 

Thus, we see how the existence of a disintegration implies the
existence of conditional expectations for Banach space valued
integrable functions as well. Hence the importance of the existence of
disintegrations. In the next chapter, we shall give some sufficient
conditions for the existence of disintegration of a measure with
respect to a $\sigma$-algebra. 
