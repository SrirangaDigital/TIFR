
\chapter{Regular Disintegrations}\label{part2:chap6}

\section{Basic Definitions}\label{part2:chap6:sec1}


Throughout\pageoriginale this chapter, let us assume that $(\Omega,
\mathscr{O},\lambda)$ is a measure space, $(\mathscr{C}^t)_{t \in
  \mathbb{R}}$ is an increasing right continuous family of sub
$\sigma$-algebras of $\hat{\mathscr{O}}_\lambda$, $\lambda$ restricted
to $\mathscr{C}^t$ is $\sigma$-finite $\forall \; t\in \mathbb{R}$ and
that $\forall t$, $\lambda$ has a disintegration $(\lambda^t_w)_{w \in
\Omega}$ with respect to $\mathscr{C}^t$.

Then $(w,t) \to \lambda^t_w$ is a measure valued martingale on $\Omega
\times \mathbb{R}$ with values in $\mathfrak{m}^+
(\Omega,\mathscr{O})$, adapted to $(\mathscr{C}^t)_{t \in\mathbb{R}}$. 
 
\begin{defn}\label{part2:chap6:def84}
$(\lambda^t_w)_{w \in \Omega}$ is said to be a {\em regular
    disintegration} of $\lambda$ with respect to the family
  $(\mathscr{C}^t)_{t \in\mathbb{R}}$ if $(w,t) \to \lambda^t_w$ is a
  {\em regular} martingale adapted to $(\mathscr{C}^t)_{t \in
    \mathbb{R}}$. 

From the theorem (\ref{part1:chap3}, \S\ \ref{part1:chap3:sec6},
\ref{part1:chap3:thm44}), we see that if $\Omega$ is a 
topological space, $\mathscr{O}$ is its Borel $\sigma$-algebra and
$\Omega$ has the $\lambda$-compacity metrizability property, then
$\forall$ $t \in \mathbb{R}$, $\lambda$ has a disintegration
$(\delta^t_w)_{w \in\Omega}$ with respect to $\mathscr{C}^t  \cdot
(w,t) \to \delta^t_w$ is a measure valued martingale adapted to
$(\mathscr{C}^t)_{t \in \mathbb{R}}$. If further, $\lambda$ is finite,
then by theorem (\ref{part2:chap5}, \S\ \ref{part2:chap5:sec2},
\ref{part2:chap5:thm78}), this measure valued martingale has a 
regular modification, say $(\lambda^t_w)_{\substack{w \in \Omega\\ t
    \in \mathbb{R}}}$. Then, $(\lambda^t_w)_{\substack{w \in \Omega\\t
\in \mathbb{R}}}$ is a regular disintegration of $\lambda$ with
respect to $(\mathscr{C}^t)_{t \in \mathbb{R}}$. Hence, if we assume
that $\Omega$ is a topological space, $\mathscr{O}$ is its Borel
$\sigma$-algebra, $\Omega$ has the $\lambda$-compacity metrizability
property, $(\mathscr{C}^t)_{t \in \mathbb{R}}$ is an increasing right
continuous family of $\sigma$-algebras contained in
$\hat{\mathscr{O}}_\lambda$ and $\lambda$ is a finite measure on
$\mathscr{O}$, then $\lambda$ has a regular disintegration with
respect to $(\mathscr{C}^t)_{t \in \mathbb{R}}$. 
\end{defn}


\begin{defn}\label{part2:chap6:def85}
A set $F \subset \Omega \times \mathbb{R}$ is said to be
$\lambda$-{\em evanescent} if its projection on $\Omega$ is of
$\lambda$-measure zero. 
\end{defn}

Note\pageoriginale that if $F$ is a $\lambda$-evanescent set, then
there exists a set $B \in\hat{\mathscr{O}}_\lambda$ such that
$\lambda(B) =0$ and $F \subset B \times \mathbb{R}$. Hence, if a
property $P(w,t)$ holds except for a $\lambda$-evanescent set, then
there exists a set $B \in \hat{\mathscr{O}}_\lambda $ with $\lambda(B)
=0$ such that if $w \not\in B$, $P(w,t)$ holds for all  $t \in
\mathbb{R}$. Conversely, if there exists a set $B \in
\hat{\mathscr{O}}_\lambda$ with $\lambda(B)=0$, such that a property
$P(w,t)$ holds for all $t$, whenever $w \not\in B$, then the property
holds except for the $\lambda$-evanescent set $B \times \mathbb{R}$. 

\section{Properties of regular disintegrations}\label{part2:chap6:sec2}

Let us assume hereafter that $(\lambda^t_w)_{\substack{w\in
    \mathbb{R}\\t \in \mathbb{R}}}$ is a regular disintegration of
$\lambda$ with respect to $(\mathscr{C}^t)_{t \in \mathbb{R}}$. 

\begin{proposition}\label{part2:chap6:prop86}
Let $B \in \mathscr{O}$ with $\lambda(B) =0$. Then $\forall_\lambda
w$, $\forall t$, $\lambda^t_w(B)=0$. 
\end{proposition}

\begin{proof}
We have $\forall t$, $\forall_\lambda w$, $\lambda^t_w(B)=0$. Hence,
$$
\forall_\lambda w,\forall t \in \mathbb{Q}, \; \lambda^t_w(B)=0. 
$$
Since $\forall_\lambda w$, $t \to \lambda^t_w(B)$  is right
continuous, it follows that
$$
\forall_\lambda w, \forall t \in \mathbb{R}, \lambda^t_w(B) =0. 
$$
\end{proof}

\begin{proposition}\label{part2:chap6:prop87}
$\lambda^t_w$ is a probability measure except for a
  $\lambda$-evancescent set.
\end{proposition}

\begin{proof}
$\forall t$, $\forall_\lambda w$, $\lambda^t_w$ is a probability
  measure. Therefore, $\forall_\lambda w$, $\forall t \in
  \mathbb{Q}$, $\lambda^t_w$ is a probability measure. 

Since $\forall_\lambda w$, $t \to \lambda^t_w(\Omega)$ is right continuous, it follows that $\forall_\lambda w$, $\forall t \in \mathbb{R}$, $\lambda^t_w$ is a probability measure.
\end{proof}

\begin{proposition}\label{part2:chap6:prop88}
Let $A \in \mathscr{C}^t \forall t \in \mathbb{R}$. Then
$\forall_\lambda w$, $\forall t$, $\lambda^t_w$ is carried by $A$ or
by $\complement A$, according as $w \in A$ or $w \in \complement A$. 
\end{proposition}

\begin{proof}
By proposition (\ref{part1:chap3}, \S\ \ref{part1:chap3:sec7},
\ref{part1:chap3:prop49}),\pageoriginale we know that $\forall 
t$, $\forall B \in \mathscr{C}^t$, $\forall_\lambda w \lambda^t_w$ is
carried by $B$ or by $\complement B$ according as $w \in B$ or $w \in
\complement B$. i.e. $\forall \; t$, $\forall \; B \in \mathscr{C}^t$,
$\forall_\lambda w$, $\chi_B(w) \cdot \lambda^t_w(\complement B)$ and
$\chi_{\complement B} (w) \cdot \lambda^t_w(B)$ are both zero. 

Since $A \in\mathscr{C}^t$ for all $t$, we have therefore,
$\forall_\lambda w$, $\forall t \in \mathbb{Q} \chi_A(w) \cdot
\lambda^t_w(\complement A)$ and $\chi_{\complement A} (w) \cdot
\lambda^t_w(A)$ are both zero.

Since $\forall_\lambda w$, $t \to \lambda^t_w(\complement A)$ and $t
\to \lambda^t_w(A)$ are right continuous, it follows that
$\forall_\lambda w$, $\forall t \in \mathbb{R}$, $\chi_B(w) \cdot
\lambda^t_w(\complement A)$ and $\chi_{\complement A} (w)
\lambda^t_w(A)$ are both zero.
\end{proof}

\begin{proposition}\label{part2:chap6:prop89}
Let $f$ be a bounded regular supermartingale, adapted to the family
$(\mathscr{C}^t)_{t \in \mathbb{R}}$. Then, $\forall_\lambda w$,
$\forall s$, $t \to \lambda^B_w(f^t)$ is right continuous.
\end{proposition}

\begin{proof}
Since $f$ is a regular supermartingale, $\forall_\lambda w$, $f_w$ is
right continuous, i.e. $\exists B \in \mathscr{O}$ with $\lambda(B)
=0$ such that if $w \not\in B$, $t \to f^t(w)$ is right continuous.

Let $t$ be fixed. Let $t_n \downarrow t$. Then, if $w \not\in B$,
$f^{t_n}(w) \to f^t(w)$. 

By proposition (\ref{part2:chap6}, \S\ \ref{part2:chap6:sec2},
\ref{part2:chap6:prop86}),  
$$
\forall_\lambda w, \; \forall s\;  \lambda^s_w(B) = 0. 
$$
Hence, $\forall_\lambda w$, $\forall s$, $\forall_{\lambda^s_w} w'$,
$f^{t_n} (w') \to f^t(w')$. Since by proposition (\ref{part2:chap6},
\S\ \ref{part2:chap6:sec2}, \ref{part2:chap6:prop87}), 
$\forall_\lambda w$, $\forall \; s$, $\lambda^s_w$ is a probability
measure, and since $f$ is bounded, it follows by the dominated
convergence theorem, that 
$$
\forall_\lambda w, \forall s, \lambda^s_w(f^{t_n})  \to
\lambda^s_w(f^t). 
$$
\end{proof}

\begin{proposition}\label{part2:chap6:prop90}
Let $f$ be a  bounded regular supermartingale adapted to
$(\mathscr{C}^t)_{t \in \mathbb{R}}$. Then, $\forall_\lambda w$,
$\forall \; s$, $t \to \lambda^s_w(f^t)$ is a decreasing function in
$[s, + \infty)$. 
\end{proposition}

\begin{proof}
Let $\{t,t'\}$ be a fixed pair of real numbers, $t < t'$. Since $f$ is
a supermartingale adapted to $(\mathscr{C}^t)_{t \in \mathbb{R}}$, we
have 
$$
\forall \; s , \; s \leq t, \; \forall_\lambda w, \;
\lambda^s_w(f^{t'}) \leq \lambda^s_w(f^t). 
$$

Therefore,\pageoriginale $\forall_\lambda w$, $\forall s \in \mathbb{Q}$ $s \leq t$,
$\lambda^s_w(f^{t'}) \leq \lambda^s_w(f^t)$.

Now, $(w,s) \to \lambda^s_w(f^{t'})$ and $(w,s) \to\lambda^s_w(f^t)$
are regular supermartingales. Hence, $\forall_\lambda w$, $s \to
\lambda^s_w (f^{t'})$ and $s\to \lambda^s_w(f^t)$ are right
continuous. Therefore $\forall_\lambda w$, $\forall s < t$,
$\lambda^s_w(f^{t'}) \leq \lambda^s_w(f^t)$. 

Now, $\forall_\lambda w$, $\lambda^t_w(f^t) = f^t(w)$. Since $f$ is a
supermartingale, 
$$
\forall_\lambda w, \; \lambda^t_w(f^{t'}) \leq f^t(w).
$$

Hence, $\forall_\lambda w$, $\lambda^t_w(f^{t'}) \leq
\lambda^t_w(f^t)$. Therefore, 
$$
\forall_\lambda w, \; \forall s \leq t, \; \lambda^s_w(f^{t'}) \leq
\lambda^s_w(f^t). 
$$

Thus, for every pair $\{t, t'\} t < t'$,  $\forall_\lambda w$, $\forall s \leq t$, $\lambda^s_w(f^{t'}) \leq
\lambda^s_w(f^t)$. Hence, $\forall_\lambda w$, $\forall$ pair $\{t,
t'\}$, $t \in \mathbb{Q} \; t' \in \mathbb{Q} \; t < t'$, 
$$
\forall s \leq t, \; \lambda^s_w(f^{t'}) \leq \lambda^s_w(f^t).
$$

By the previous proposition (\ref{part2:chap6},
\S\  \ref{part2:chap6:sec2}, \ref{part2:chap6:prop89}),
$\forall_\lambda w$, 
$\forall s$, $t \to \lambda^s_w(f^t)$ is right continuous. Therefore,
$\forall_\lambda w$, $\forall$ pair $\{t,t'\}$, $t , t'$, $\forall s
\leq t$,
$$
\lambda^s_w(f^{t'}) \leq \lambda^s_w(f^t). 
$$ 
Thus, $\forall_\lambda w$, $\forall s$, $t \to \lambda^s_w(f^t)$ is a
decreasing function in $[s, + \infty)$. 
\end{proof}

\begin{thm}\label{part2:chap6:thm91}
Let $f$ be a regular supermartingale adapted to $(\mathscr{C}^t)_{t
  \in \mathbb{R}}$.  Then $\forall_\lambda w$, $\forall s$, $t \to
\lambda^s_w(f^t)$ is decreasing and right continuous in $[s, +
  \infty)$ and 
$$
\forall_\lambda w, \; \forall s, \; \lambda^s_w (f^s) = f^s(w). 
$$
\end{thm}

\begin{proof}
If $f$ is a bounded regular supermartingale, then from the previous
propositions, proposition (\ref{part2:chap6},
\S\ \ref{part2:chap6:sec2}, \ref{part2:chap6:prop89}) and proposition
(\ref{part2:chap6}, \S\ \ref{part2:chap6:sec2}, \ref{part2:chap6:prop90}), we see that 
$$
\forall_\lambda w, \; \forall s, t \to \lambda^s_w(f^t)
$$
is a\pageoriginale decreasing right continuous function in $[s, +
  \infty)$. Hence, $\forall_\lambda w$, $\forall s$, $t \to
  \lambda^s_w(f^t)$ is decreasing and lower semi-continuous in $[s, +
    \infty)$. (At $s$, we mean only right lower semi-continuity).

If $f$ is an arbitrary regular supermartingale, not necessarily
bounded, $\forall m \in \mathbb{N}$, $\inf (f,m)$ is a bounded regular
supermartingale. Hence,\break $\forall m\in \mathbb{N}$, $\forall_\lambda
w$, $\forall s$, $t\to \lambda^s_w[\{\inf (f,m)\}^t]$ is decreasing
and lower semi-continuous in $[s, + \infty)$. 

Since  $\forall t$, $\{\inf (f,m)\} {}^t \uparrow f^t$, we have 
$$
\forall s, \; \forall t, \; \forall w, \; \lambda^s_w(f^t) =
\lim\limits_{m \to \infty} \lambda^s_w [\{\inf (f,m)\}^t].
$$
Therefore, $\forall_\lambda w$, $\forall s$, $t \to \lambda^s_w(f^t)$
is decreasing and lower semi-continuous in $[s, + \infty)$. 

If a function, in an interval is decreasing and lower semi-continuous,
if is right continuous there.

Hence, $\forall_\lambda w$, $\forall s$, $t \to \lambda^s_w(f^t)$ is
decreasing and right continuous in $[s, + \infty)$.

Let us now prove that
$$
\forall_\lambda w, \; \forall s,\; \lambda^s_w(f^s) = f^s(w). 
$$
We have $\forall s$, $\forall_\lambda w$, \; $\lambda^s_w(f^s) =
f^s(w)$. Hence,
$$
\forall_\lambda w, \forall s \in \mathbb{Q}, \lambda^s_w(f^s) =
f^s(w). 
$$

Since $\forall_\lambda w$, $s \to f^s(w)$ is right continuous, to
prove the theorem, it is sufficient to prove that $\forall_\lambda w$,
$s \to \lambda^s_w(f^w)$ is right continuous. Now,
$$ 
\mathbb{R} = \bigcup\limits_{\substack{k \in \mathbb{Z}\\n\in
    \mathbb{N}}} \left[ \frac{k}{2^n}  \cdot \frac{k+1}{2^n}\right) .
$$
Define $\forall $ $n \in \mathbb{N}$, $w \in \Omega$
$$
f^s_n(w) = \lambda^s_w (f^{\frac{k+1}{(2^n)}} \text{ if } s \in \left[
  \frac{k}{2^n}, \frac{k+1}{2^n}   \right). . 
$$

We\pageoriginale claim that $\forall n$, $f_n$ is a regular
supermartingale adapted to $(\mathscr{C}^t)_{t\in
  \mathbb{R}}$. $\forall n$, the regularity of $f_n$ is clear, since
$(\lambda^s_w)_{\substack{w \in \Omega\\ s\in \mathbb{R}}}$ is a
regular disintegration of $\lambda$. We have to only prove that $\forall \; n \in  \mathbb{N}$,
$f_n$ is a supermartingale adapted to $(\mathscr{C}^t)_{t \in
  \mathbb{R}}$. 

Clearly, $f_n$ is adapted to $(\mathscr{C}^t)_{t \in \mathbb{R}}$,
$\forall n \in \mathbb{N}$. We have to only prove that $\forall \; n$,
$\forall \; t$, $t' \in \mathbb{R}$, $t < t'$, $\forall \; A \in
\mathbb{C}^t$,
$$
\int\limits_A f^{t'}_n (w) d \lambda(w) \leq \int\limits_A f^t_n(w) d
\lambda (w). 
$$

This is clear, if both $t$ and $t'$ belong to the same interval
$\left[ \dfrac{k}{2^n} , \dfrac{k+1}{2^n} \right).$  for some $t \in
  \mathbb{Z}$. Actually, in this case, we have even equality as $(s,w)
  \to \lambda^s_w(f^{\dfrac{k+1}{2^n}})$ is a martingale. 

It is sufficient to prove the above inequality when $t$ and $t'$
belong to two consecutive intervals, any $t \in \left[ \dfrac{k}{2^n},
  \dfrac{k+1}{2^n}\right)$ and $t'\in \left[ \dfrac{k+1}{2^n},
    \dfrac{k+2}{2^n} \right)$, because for other values of $t$ and
    $t'$, $t<t'$, the above inequality will then follow from the
    transitivity of the conditional expectations. 

Again to prove the inequality, it is sufficient to prove it when $t =
k/2^n$ for some $k \in \mathbb{Z}$ and $t' = \dfrac{k+1}{2^n}$,
because of the transitivity of the conditional expectations and
because of the fact that $(s,w) \to \lambda^s_w
(f^{\frac{k+1}{2^n}})$ is a martingale; i.e. we have to only prove
that $\forall \; n \in \mathbb{N}$, $\forall \; k \in \; \mathbb{Z}$,
$\forall \; A \in \mathscr{C}^{k/2^n}$
$$
\int\limits_A \lambda^{\dfrac{k+1}{2^n}}_w (f^{\dfrac{k+2}{2^n}}) d
\lambda(w) \leq \int\limits_A \lambda^{k/2^n}_w(f^{\dfrac{k+1}{2^n}})
d\lambda(w). 
$$

But this follows immediately from the fact that $f$ is a
supermartingale. It is easy to check that 
$$
\forall \; s, \forall n, \quad \forall_\lambda w, \; f^s_n(w) \leq
f^s_{n+1} (w). 
$$

Therefore, since $\forall n \in \mathbb{N}$, $f_n$ is regular, we have 
$$ 
\forall_\lambda w, \forall n , \forall s, f^s_n(w) \leq f^s_{n+l}
(w).  
$$\pageoriginale
Hence $\forall_\lambda w$, $\forall s$, $\lim\limits_{n \to \infty}
f^s_n(w)$ exists. But
$$
\forall_\lambda w, \forall s, t \to \lambda^s_w(f^t)
$$
is right continuous in $[s ,+ \infty)$. Hence
$$
\forall_\lambda w, \forall s, \lim\limits_{n \to \infty} f^s_n(w) =
\lambda^s_w(f^s). 
$$

Since $(f_n)_{n \in \mathbb{N}}$ is an increasing sequence of regular
supermartingales, it follows from the `Upper enveloppe theorem' that
$$
\forall_\lambda w, \; s \to \lambda^s_w(f^s)
$$
is right continuous. 
\end{proof}

We now state and prove another important theorem, which we call the
``theorem of trajectories''. In the course of the proof of this
theorem, we need the concepts relating to ``{\em Well-measurable
processes}'' and ``{\em $\mathscr{C}$-analytic sets}'' where
$\mathscr{C}$ is a $\sigma$-algebra on $\Omega$. We state about well
measurable processes what we need. For the definition and properties
of $\mathscr{C}$-analytic sets, we refer the reader to P.A. Meyer
\cite{key1}, Chapter, \ref{part1:chap3} and section
\ref{part1:chap3:sec1} ``Compact pavings and Analytic sets''.

By a {\em stochastic process} $X$ on $\Omega$, with values in a
topological spaces $E$, we mean a collection $(X^t)_{t \in
  \mathbb{R}}$ of mappings on $\Omega$ with values in $E$ such that
$\forall t$, $X^t \in\hat{\mathscr{O}}_\lambda$. A stochastic process
can be considered as a mapping $X$ from $\Omega \times \mathbb{R}$ to
$E$ such that $\forall t \in \mathbb{R}$, $X^t \in
\hat{\mathscr{O}}_\lambda$. A stochastic process $X$ is said to be
    {\em adapted} to $(\mathscr{C}^t)_{t \in \mathbb{R}}$ if $\forall t
      \in \mathbb{R}$, $X^t \in \mathscr{C}^t$. 

\begin{defn}\label{part2:chap6:def92}
A stochastic process $X$ on $\Omega$ with values in $\mathbb{R}$ is
said to be {\em well-measurable} if it belongs to the $\sigma$-algebra
on $\Omega \times \mathbb{R}$, generated by all\pageoriginale sets $A
\subset \Omega \times \mathbb{R}$ such that $(w,t) \to \chi_A(w,t)$
is regulated and right continuous and is adapted to
$(\mathscr{C}^t)_{t \in \mathbb{R}}$. 
\end{defn}

See definition D14 in page 156 and remark (b) in page 157 in
P.A. Meyer \cite{key1}. 

We need the following important fact.

If $X$ is a right continuous adapted process on $\Omega$ with values
in $\mathbb{R}$, then $X$ is well-measurable.

For a proof see remark (c) in page 157, in P.A. Meyer \cite{key1}.

\begin{thm}[The theorem of trajectories]\label{part2:chap6:thm93}
Let $X$ be a right continuous stochastic process on $\Omega$, adapted
to $(\mathscr{C}^t)_{t \in \mathbb{R}}$, with values in a topological
space $E$ which has a countable family of real valued continuous
functions separating its points (for example, a completely regular
Suslin space). Then, 
$$
\forall_\lambda w, \; \forall t \in \mathbb{R} \;
\forall_{\lambda^t_w} w', \forall s \leq t, \; X^s (w') = X^s (w). 
$$
i.e. $\forall_\lambda w$, $\forall t$, $\lambda^t_w$ is carried by the
set of all $w'$ whose trajectories coincide with those of $w$ upto the
time $t$. 
\end{thm}

\begin{proof}
If $\mathscr{B}_E$ is the Borel $\sigma$-algebra of $E$, it is easy to
see that $\mathscr{B}_E$ is countably separating. Let $s$ be any fixed
real number. Let $h$ be a function on $\Omega$ with values in $E$ such
that $h \in \mathscr{C}^t$ for all $t \geq s$. Then proceeding as in
the proof of proposition (\ref{part1:chap3},
\S\ \ref{part1:chap3:sec7}, \ref{part1:chap3:prop54}) and proposition
(\ref{part2:chap6}, \S\ \ref{part2:chap6:sec2},
\ref{part2:chap6:prop88}), we can prove that  
$$
\forall_\lambda w, \; \forall \; t \geq s, \; \forall_{\lambda^t_w}
w', \; h(w') = h(w).
$$

Now $X^s \in \mathscr{C}^t \;\; \forall t \geq s$. Hence,
$$
\forall s, \; \forall_\lambda w, \; \forall t \geq s , \; \forall
_{\lambda^t_w} w', \; X^s(w') = X^s(w). 
$$
Therefore, $\forall_\lambda w$, $\forall s \in \mathbb{Q}$, $\forall t
\geq s$, $\forall_{\lambda^t_w} w'$, $X^s(w') = X^s(w)$. Hence, 
$\forall_\lambda w$,\pageoriginale $\forall t$, $\forall_{\lambda^t_w}
w'$, $\forall s \in \mathbb{Q} s \leq t$, $X^s (w') = X^s (w)$, 

Now, $\forall_\lambda w$, $s \to X^s(w)$ is right continuous and hence
$\forall_\lambda w$, $\forall t$, $\forall_{\lambda^t_w} w'$, $s \to
X^s(w')$ is right continuous. Hence
\begin{equation*}
\forall_\lambda w, \; \forall t, \; \forall_{\lambda^t_w} w', \;
\forall s < t, X^s (w') = X^s(w). \tag{1}\label{part2:chap6:eq1}
\end{equation*}

To prove the theorem, we have to only prove that $\forall_\lambda w$,
$\forall t$, $\forall_{\lambda^t_w} w'$,  $X^t(w') = X^t(w)$. 

First let us consider the case when $E = \mathbb{R}$. 

Let us assume that $X$ is a regular supermartingale adapted to
$(\mathscr{C}^t)_{t \in\mathbb{R}}$. Let $M$ be any real number
$>0$. Let $X_M = \inf (X, M)$. 
 Then, $X_M$ is also a regular supermartingale. Therefore, by theorem
 (\ref{part2:chap6}, \S\ \ref{part2:chap6:sec2},
 \ref{part2:chap6:thm91}),
$$
\forall_\lambda w, \; \forall t, \; \lambda^t_w(X^t_M) = X^t_M(w).
$$

Let $w$ be such that $\lambda^t_w$ is a probability measure and
$\lambda^t_w(X^t_M) = X^t_M (w)$. If $X^t (w) \geq M$, then $X^t_M(w)
= M$ and $X^t_M(w') \leq M$ for all $w' \in \Omega$. Since
$\lambda^t_w(X^t_w) = X^t_M(w)$, it follows therefore that
$\forall_{\lambda^t_w} w'$,  $X^t_M (w') = M = X^t_M(w)$ and hence, 
$$
\forall_{\lambda^t_w} w', \; X^t(w') \geq M.
$$

Since $\forall_\lambda w$, $\forall t$, $\lambda^t_w$ is a probability
measure and since $\forall_\lambda w$, $\forall t$,\break
$\lambda^t_w(X^t_M) = X^t_M(w)$, we see that 
\begin{gather*}
\forall \; M \in \mathbb{R}, \; M>0, \; \forall_\lambda w, \; \forall
t, \; \text{ if } X^t(w) \geq M, \text{ then } \\
\forall_{\lambda^t_w} w', X^t(w') \geq M.
\end{gather*}
Hence, $\forall_\lambda w$, $\forall t$, $\forall M \in\mathbb{Q}$,
$M>0$, if $X^t(w) \geq M$, then $\forall_{\lambda^t_w} w'$, $X^t(w')
\geq M $. Therefore $\forall_\lambda w$, $\forall t$,
$\forall_{\lambda^t_w} w'$, $X^t(w') \geq X^t(w)$. But by theorem (\ref{part2:chap6},
\S\ \ref{part2:chap6:sec2}, \ref{part2:chap6:thm91}), 
$$
\forall_\lambda w, \; \forall t, \; \lambda^t_w(X^t) = X^t(w). 
$$
Hence, $\forall_\lambda w$, $\forall t$, $\forall_{\lambda^t_w} w'$,
$X^t (w') = X^t(w)$. 

This\pageoriginale proves the theorem, when $X$ is a regular
supermartingale. 

Since an adapted right continuous decreasing process with values in
$\mathbb{R}$, is a regular supermartingale, the theorem holds when $X$
is such a process. Therefore also, when $X$ is a bounded adapted right
continuous increasing process. By passing to the limit, the theorem
therefore also holds when $X$ is an adapted increasing right
continuous process. 

Now, let $X$ be any regulated right continuous process with values in
$\mathbb{R}$. 

Let $\forall w \in \Omega$ and $t \in \mathbb{R}$,
$$
\sigma (w,t) = X(w,t) - X (w, t-).
$$
$\sigma$ is again an adapted process with values in $\mathbb{R}$. Let
$\alpha >0$.

Since $\forall_\lambda w$, $X_w$ is a regulated function on
$\mathbb{R}$, $\forall_\lambda w$, the number of $t$'s in any
relatively compact interval of $\mathbb{R}$ for which $\sigma (w,t)
\geq \alpha$ is finite.

Let $n \in \mathbb{N}$ and $w \in \Omega$ and $t \in
\mathbb{R}$. Define $\overset{\alpha}{M}_n (w,t)$ as the number of
$\tau$'s, $\tau \in (-n, t]$ for which $\sigma (w, \tau) \geq \alpha$,
if  $t > -n$ and zero otherwise.

Then $\forall_\lambda w$, $\forall n$, $\overset{\alpha}{M_n} (w,t)$
is an integer for all $t$ and it is easy to see that $\forall w$, for
which $X_w$ is regulated, given any $t \in \mathbb{R}$, $\exists \;\; 
\delta >0$ such that $M_n(w,.)$ is constant in $[t, t +
  \delta)$. Hence $\forall_\lambda w$, $t \to M_n(w,t)$ is trivially a
  right continuous function. Moreover, $\forall_\lambda w$, $\forall
  n$, $M_n(w', \cdot)$ is an increasing function of $t$. 

Hence, $\forall \; n \in \mathbb{N}$, $\overset{\alpha}{M}_n$ is an
increasing right continuous process on $\Omega \times \mathbb{R}$. 

Let us show that it is adapted to the family
$(\hat{\mathscr{C}}^t_\lambda)_{t \in \mathbb{R}}$; i.e. let us show
that 
$$
\forall \; t \in \mathbb{R}, \; \forall n \in \mathbb{N}, \; w \to
\overset{\alpha}{M_n} (w,t) \in \hat{\mathscr{C}}^t_\lambda. 
$$

To understand the ideas involved in this prove, the reader is referred
to the proof of T52 in page 71 if P.A. Meyer \cite{key1}, where
similar ideas are used. 

Any right continuous or a left continuous process $Z$ adapted to\break
$(\mathscr{C}^t)_{t \in \mathbb{R}}$ is {\em progressively measurable}
with respect to $(\mathscr{C}^t)_{t \in \mathbb{R}}$ and hence
$\forall t$, the restriction of\pageoriginale $Z$ to $\Omega \times
(-n ,t]$ belongs to $\mathscr{C}^t \otimes \mathscr{B} (-n,t]$ where
  $\mathscr{B}(-n,t]$ is the Borel $\sigma$-algebra of $(-n,t]$. (see
$D45$ in page 68 and $T47$ in page 70 of P.A. Meyer \cite{key1}). Note
that $(w,t) \to X(w,t-)$ is a left continuous process and hence
$\sigma$ is progressively measurable with respect to
$(\mathscr{C}^t)_{t \in \mathbb{R}}$. 

Since $\overset{\alpha}{M} {}^t_n$ is an integer valued function, to
prove that $\forall \in \mathbb{N}$, $\overset{\alpha}{M}{}^t_n \in
\hat{\mathscr{C}}^t_\lambda $, it is sufficient to show that $\forall$
$m \in \mathbb{N}$, $\left\{ w: M^t_n (w) \geq m\right\}$ is a
$\mathscr{C}^t$-analytic set.

Let $A$ be the subset of $\Omega \times \mathbb{R}^{m+1}$ consisting
of points $(w, -n, s_1, s_2,\break\ldots,s_m)$ where $w \in \Omega$, $-n
< s_1 < s_2 \ldots  < s_m \leq t$ and 
$$
\sigma (w, s_1) \geq \alpha, \; \sigma (w, s_2) \geq \alpha, \ldots ,
\sigma (w , s_m) \geq \alpha. 
$$ 

This set $A \in \mathscr{C}^t \otimes \mathscr{B}^{m+1}(-n,t]$  where
$\mathscr{B}^{m+1} (-n,t] $ is the Borel $\sigma$-algebra of
$\underset{m+1 \text{ times}}{\underbrace{(-n,t] \times (-n,t] \ldots
\times (-n, t]}}$.  

This is because of the fact that $\sigma$ is progressively measurable
with respect to $(\mathscr{C}^t)_{t \in \mathbb{R}}$. 

The projection of $A$ on $\Omega$ is precisely the set $\{w:
\overset{\alpha}{M} {}^t_n (w) \geq m\}$. Hence $\{w: M^t_n (w) \geq
m\}$ is a $\mathscr{C}^t$-analytic set. Hence
$$
\forall \; n, \; \forall \; t, \; \overset{\alpha}{M}{}^t_n \in
\hat{\mathscr{C}}^t_\lambda. 
$$
Let $\forall t$, $\mathfrak{F}^t$ be the $\sigma$-algebra
$\bigcap\limits_{u>t} \hat{\mathscr{C}}^u_\lambda$. Then
$(\mathfrak{F}^t)_{t \in \mathbb{R}}$ is an increasing right
continuous family of $\sigma$-algebras on $\Omega$, contained in
$\hat{\mathscr{O}}_\lambda$ and $\forall n \in \mathbb{N}$,
$\overset{\alpha}{M}{}_n$ is adapted to $(\mathfrak{F}^t)_{t \in
  \mathbb{R}}$. Since $\overset{\alpha}{M}{}_n$ is an increasing right
continuous process adapted to $(\mathfrak{F}^t)_{t \in \mathbb{R}}$,
the theorem is valid for this process.

Hence $\forall \; n \in \mathbb{N}$, $\forall \alpha > 0$,
$\forall_\lambda w$, $\forall t$, $\forall_{\lambda^t_w} w'$, $\forall
s \leq t$, $\overset{\alpha}{M}{}_n (w',s) =
\overset{\alpha}{M}_n(w,s)$. 

This shows that $\forall n \in \mathbb{N}$, $\forall \alpha >0$,
$\forall_\lambda w$, $\forall t$, $\forall_{\lambda^t_w} w'$, for
all $s \leq t$, the number of jumps in $(-n, s]$ of magnitude greater
  than or equal to $\alpha$ are the same for $X_w$ and $X_{w'}$. 

Since\pageoriginale this is true for all rational $\alpha >0$ and the
same thing analogously for all rational $\alpha <0$, it follows that
$\forall$ $n \in \mathbb{N}$, $\forall_\lambda w$, $\forall t$,
$\forall_{\lambda^t_w} w'$, for all $s \leq t$, the number of jumps in
$(-n,s]$ of any given magnitude are the same for $X_w$ and $X_{w'}$. 

Hence $\forall$ $n \in \mathbb{N}$, $\forall_\lambda w$, $\forall t$,
$\forall_{\lambda^t_w } w'$, for all $s \leq t$, $X_{w'}$ and $X_w$
have jumps precisely at the same points in $(-n, s]$ and the
magnitudes of the jumps at each point of a jump are the same for
$X_{w'}$  and $X_w$. 

Since this is true for every $n \in \mathbb{N}$, we therefore have
that 
$$
\forall_\lambda w, \; \forall t,  \; \forall_{\lambda^t_w} w', \;
X_{w'} 
$$
has a jump at a point $s \leq t$ if and only if $X_w$ has one at $s$
and the magnitudes of the jumps for $X_w$ and $X_{w'}$ at $s$ are the
same.

Let $C = \{w \in\Omega \mid \forall t, \; \forall_{\lambda^t_w} w' ,
w'$ has a jump at a point $s \leq t $ if and  only if $X_w$ has
  one at $s$ and  the magnitudes of the jumps for $X_{w'}$ and $X_w$
  at $s$ are the same  $\}$. Then, by what we have seen, $\lambda$ is
carried by $C$. 

Let $B = \{ w \in \Omega \mid \forall t, \; \forall_{\lambda^t_w} w',
\forall s <t, X^s(w') = X^s(w) \}$. Then, from (\ref{part2:chap6:eq1}), $\lambda$ is
carried by $B$. 

Let $w \in B \cap C$. Let $t$ be any point of $\mathbb{R}$. Let $X_w$
be continuous at $t$. We have, since $w \in B$,
$$
\forall_{\lambda^t_w} w', \; \forall s < t, \; X^s(w') = X^s(w). 
$$

Hence, $\forall_{\lambda^t_w} w'$, $\lim\limits_{\substack{s \to t
    \\ s<t}} X^s (w')$ exists and is equal to $X^t(w)$, since $X_w$ is
continuous at $t$.

Let us prove that 
$$
\forall_{\lambda^t_w} w', \lim\limits_{\substack{s \to t\\s<t}}
X^s(w') = X^t(w'). 
$$

If\pageoriginale  not, the set $B^t_w = \{w' \in \Omega \mid
\sum\limits_{\substack{s\to t\\ s<t}} X^s (w') \neq X^t (w')\}$ is of
positive $\lambda^t_w$-measure. But, since $w \in C$ and since $X_w$
is continuous at $t$ and hence has no jump at $t$, on no set of
positive $\lambda^t_w$-measure, all the $w'$ can have a jump at
$t$. Hence $B^t_w$ must be of $\lambda^t_w$-measure zero. Hence, 
$$
\forall_{\lambda^t_w} w', \; \lim\limits_{\substack{s \to t\\s<t}}
X^s(w') = X^t (w') = X^t (w). 
$$

Therefore, $\forall_{\lambda^t_w} w'$, $X^t(w') = X^t(w)$. 

Now, let $w \in B\cap C$ and $t$ be a point at which $X_w$ is
discontinuous. Then $X_w$ has a jump at $t$ and since $X_w$ is right
continuous at $t$, the jump is equal to $X(w,t) - X(w, t-)$. Hence,
since $w \in C$, $\forall_{\lambda^t_w} w'$, $w'$ has also a jump at
$t$ and 
$$
X (w',t) - X(w', t-) = X(w,t) - X(w,t-).
$$
But since $w \in B$, 
$$
\forall_{\lambda^t_w} w', \forall s < t, \; X^s(w') = X^s(w)
$$
Hence
$$
\forall_{\lambda^t_w} w', \; X (w', t-) = X(w,t-). 
$$
Therefore,
$$
\forall_{\lambda^t_w} w', X(w',t) = X(w,t). 
$$

Since $\lambda$ is carried by $B \cap C$ and since $t \in \mathbb{R}$
is arbitrary, we have
$$
\forall_\lambda w, \forall t, \forall_{\lambda^t_w} w', \; X^t(w') =
X^t(w). 
$$

This combined with (\ref{part2:chap6:eq1}) gives that $\forall_\lambda w$, $\forall t$,
$\forall_{\lambda^t_w} w'$, $\forall s \leq t$, $X^s(w') = X^s(w)$. 

Hence the theorem is proved when $X$ is an adapted, regulated, right
continuous real valued process. 

Let $\mathscr{C}$\pageoriginale
$$
= \{ A \subset \Omega \times \mathbb{R} \mid \chi_A \text{ is adapted
  and the theorem is true for the process } \chi_A\}.
$$
Then $\mathscr{C}$ is a $\sigma$-algebra on $\Omega \times
\mathbb{R}$. 

$\mathscr{C}$ contains by the preceding, all the sets $A$ for which
$\chi_A$ is an adapted regulated right continuous process. Hence
$\mathscr{C}$ contains all the well-measurable processes.

Since a right continuous, adapted process is well-measurable, the
theorem is true for any right continuous adapted process with values
in $\mathbb{R}$.

Now, if $(f_n)_{n \in \mathbb{N}}$ is a countable family of continuous
functions on $E$, separating the points of $E$, and if $X$ is a right
continuous adapted process with values in $E$, then $\forall$ $n \in
\mathbb{N}$, $f_n \circ X$ is a real valued, adapted right continuous
process and hence the theorem is true for $f_n \circ X$, $\forall n\in
\mathbb{N}$. Since the $f_n$'s separate the points of $E$ and are
countable, the theorem is true for $X$. 
\end{proof}

\begin{thm}\label{part2:chap6:thm94}
Let $\mathscr{O}$ be countably generated. Then,
$$
\forall_\lambda w, \forall s , \forall t, \int \lambda^t_{w'}
\lambda^s_w(dw') = \lambda^{\Min(s,t)}_w.
$$
\end{thm}

\begin{proof}
$\forall B \in \mathscr{O}$, $\forall s$, $t$, $s \leq t$,
$$
\forall_\lambda w, \; \int \lambda^t_{w'} (B) \lambda^s_w(dw') =
\lambda^s_w(B) 
$$
since $(t,w') \to \lambda^t_{w'}(B)$ is a martingale. Hence, 
\begin{gather*}
\forall \; B\in \mathscr{O}, \; \forall t, \; \forall_\lambda w, \;
\forall s \in \mathbb{Q}, \;  s \leq t,\\
\int \lambda^{t}_{w'}(B) \lambda^s_w(dw') = \lambda^s_w(B).
\end{gather*}
Since $(\lambda^s_w)_{\substack{w \in \Omega\\s \in \mathbb{R}}}$ is a
regular disintegration, $\forall_\lambda w$, $s \to \lambda^s_w(B)$
and $s \to \int \lambda^t_{w'} (B) \lambda^s_w(dw')$ are right
continuous. Hence, 
\begin{gather*}
\forall \; B \in \mathscr{O}, \; \forall t, \; \forall_\lambda w, \;
\forall s < t,\\
\int \lambda^t_{w'} (B) \lambda^s_w (dw') = \lambda^s_w(B). 
\end{gather*}\pageoriginale 
Therefore, 
\begin{gather*}
\forall B \in \mathscr{O}, \; \forall_\lambda w, \; \forall t \in
\mathbb{Q}, \forall s < t,\\
\int \lambda^t_{w'} (B) \lambda^s_w(dw') = \lambda^s_w(B). 
\end{gather*}
Since $(\lambda^t_w)_{\substack{w \in \Omega\\ t \in \mathbb{R}}}$ is
a regular disintegration, $\forall B \in \mathscr{O}$,
$\forall_\lambda w$, $t \to \lambda^t_w(B)$ is right continuous. Hence
$\forall B \in \mathscr{O}$, $\forall_\lambda w$, $\forall s$,
$\forall_{\lambda^s_w} w'$, $t \to \lambda^t_{w'}(B)$ is right
continuous. Hence, 
\begin{align*}
& \forall B \in \mathscr{O}, \; \forall_\lambda w, \;  \forall t, \;
  \forall s < t, \\
& \int \lambda^t_{w'} (B) \lambda^s_w(dw') = \lambda^s_w(B) \tag{1}
\end{align*}

Since $(\lambda^s_w(B))_{\substack{w\in \Omega\\s \in \mathbb{R} }}$
is an adapted right continuous process with values in $\mathbb{R}$, by
the previous theorem (\ref{part2:chap6}, \S\ \ref{part2:chap6:sec2},
\ref{part2:chap6:thm93}),  
$$
\forall \; B \in \mathscr{O}, \; \forall_\lambda w, \; \forall s, \;
\forall_{\lambda^s_w} w', \forall t \leq s ,\; \lambda^t_{w'} (B) =
\lambda^t_w(B) . 
$$
Therefore, 
$$
\forall_\lambda w, \; \forall s, \; \forall t \leq s, \; \int
\lambda^t_{w'} (B) \lambda^s_w(dw') = \lambda^t_w(B),
$$
since $\forall_\lambda w$, $\forall s$, $\lambda^s_w$ is a probability
measure. \hfill{II}

From (I) and (II), we see therefore that 
$$
\forall B \in \mathscr{O}, \; \forall_\lambda w, \forall s, \forall t,
\; \int \lambda^t_{w'} (B) \lambda^s_w(dw') =
\lambda^{\Min(s,t)}_w(B). 
$$
Since $\mathscr{O}$ is countably generated, by the Monotone class
theorem, 
$$
\forall_\lambda w, \forall s, \forall t, \forall B \in \mathscr{O},
\int \lambda^t_{w'} (B) \lambda^s_w(dw') = \lambda^{\Min(s,t)}_w(B).
$$

Hence,
$$
\forall_\lambda w, \forall s, \forall t, \int \lambda^t_{w'}
\lambda^s_w(dw') = \lambda^{\Min(s,t)}_w.
$$
\end{proof}
