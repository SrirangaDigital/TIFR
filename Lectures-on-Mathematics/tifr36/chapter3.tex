\chapter[The Theorem of Grauert...]{The Theorem of Grauert (Mordell's conjecture for
  function fields)}\label{chap3} 

\section{Description of the method}\label{chap3:sec1}%\sec 1

In\pageoriginale this section we shall describe, often rather loosely,  the method
of attack in the proof of Grauert. (Paragraphs which do make a precise
mathematical sense will be starred). 

\noindent 
$(\ast)$~~ Let $k$ be an algebraically closed field and $K$, a
function field over $k$; let $C$ be a curve defined over $K$,  with
its absolute genus $g(\ge 2)$ equal to its relative genus over $K$. We
shall do the  geometry over a ``big'' universal domain
$\Omega$. Following Grauert (\cite{2}) we are going to analyse the cases
when the set $C_K$ of $K$- rational point of $C$ (i.e. the points of
$C$ having coordinates) in $K$ is \textit{infinite}.  The complete
results (at least in characteristic $0$) have been stated at the
beginning of these notes. Our first aim will be to prove the following  

\noindent $(\ast)$ ~ 
  \setcounter{theorem}{0}
  \begin{theorem}\label{chap3:sec1:thm1}%them 1.
    If $C_K$ is infinite then $C$ is birationally equivalent (over some
    extension of $K$) to a curve $C'$ defined over $k$.  
  \end{theorem}


If this is done, the theorems of Severi and be Franchis proved in
chapter II will enable us to study easily the rational points  of
$C'$ over $K$ (or some extension of $K$) and to obtain complete result
about the structure of $C_K$. 

\noindent
$(\ast)$~~\pageoriginale We shall first prove theorem \ref{chap3:sec1:thm1} under the additional
hypothesis that the transcendence degree of $K/k$ is 1. Then an easy
induction $d$ on will give theorem 1. It should be noted that this is
not really essential (Grauert studies case of arbitrary $d$). 

However, this makes the proof simpler and more understandable.

Gruert's method is the inverse of the so called ``Picard's Method''
whereas Picard liked to consider  a surface as  ``curve over a
function field'' here the curve $C$ will be interpreted as a surface
over $k$. Roughly speaking let $F(x,y) =0$ be a plane model of $C$ the
coefficients of $F$ are elements of $K,$ i.e. rational functions of
parameters $t=(t_1,\ldots,t_n)$ such that $K=k(t)$. Then (upto a
factor in $k(t))$ we may write $F(x,y)$ as a polynomial $G(t,x,y)$
over $k$. The surface will be $G(t,x,y)=0$. 

\medskip
\noindent
$(\ast)$~~ Let us be more precise. The field $K$ is k-isomorphic to
the function field $k(R)$ of an affine curve  $R$, which we may assume
to be nonsingular; thus we may write $K=k(r)$ where $r$ is a generic
point of $R$ over $k$. We may also assume that $C$ is a nonsingular
curve in some $\mathbb{P}_n(\large\Omega)$; let $x$ be a generic point
of $C$ over $K$. Then our surface $X$ will be the \textit{locus of the
  point} $(r,x)\in R \times \mathbb{P}_n$ \textit{over} $k$. The
projection on the first factor gives a fibration $\pi:X
\rightarrow R;$ for $t \in R$, we set $X_t=\pi ^{-1}(t)$ (a priori, this
is a cycle on $X)$. The generate fibre $X_r$ is essentially to curve $C$.  

\setcounter{lemma}{0}
\begin{lemma}\label{chap3:sec1:lem1}%lamma 1.
  Almost\pageoriginale all fibres $X_t , t \in   R,$ are irreducible nonsingular
  cur\-ves, having the same genus $g$ as $C$. 
\end{lemma}

Irreducibility is proved elements by taking a plane  model $C_1$ of
$C$, the corresponding surface $X_1 \subset R\times \mathbb{P}_2$ and by
noticing that the non-absolutely - irreducible homogeneous
polynomials in $3$ variables of a given degree $d$ form a closed
subset of the  space of all homogeneous polynomials of degree $d$. As
to nonsingularity, the singular set $X'$ of $X$ does not intersect
the generic fibre $X_r(X_r$ is nonsingular); Since $\pi  : X
\rightarrow R$ is proper, it mean that $X^{'}$ is in a finite union of
fibres say $\bigcup\limits^{q}_{i=1}X_{t_i};$ then for $t \neq t_i,
X_t$ is nonsingular. Finally all the nonsingular fibres $X_t$ have the
same genus according to a theorem of Igusa: 

\medskip
\noindent
\textit{\bf Sketch of the proof :} Consider the fibre product $X
\underset{R}{x}X$ and the iagonal $\Delta_t$ in $X_t \times X_t\subset
X\underset{R}{x}X$; the genus $g_t$ of $X_t$ is given by $2-2g_t$=
$(\Delta_t. \Delta_t)$ (Chapter II); this intersection number is
constant by the ``principle of conservation of the intersection
number''(of. The Theory of Chow coordinates).   

\noindent
$(\ast)$~~A rational point $(x')$ of $C$ over $K=k(r)$ corresponds to
a rational section $s:R\rightarrow X$ of $\pi : X\rightarrow R:$ the
coordinates of $(x'_i)$ are rational function $x'_{i}(r)$ (over $k$);
thus the section is given by $\mapsto$  $(x'_{i}(t))$. Since $R$ is
nonsingular 
and $X$ is complete $s$ is even a \textit{morphism} $R \rightarrow X$. 

Suppose\pageoriginale that we have prove that $C$ is birationally equivalent to a
curve defined over $k$. Then $X$ is birationally equivalent to $R\times
C'$ and the section $R \rightarrow X$ correspond to maps $R\rightarrow
C'$.  By the theorem of de Franchis (Chapter II) almost all these
maps are constant  or inseparable, i.e have a derivative equal to 0:
in other words, their graphs are tangent to the \textit{``horizontal
  directions field''} or $R\times C'$. Coming back to $X$ we see that
we shall have on $X$,a field  $E$ of tangent  directions,  everywhere
transversal  to the fibres and such that almost all sections $s:
R\rightarrow X$ are tangent to $E$. 

\medskip
\noindent 
$(\ast)$~~ More precisely, we replace $R$ by an open $R_o\subset R$
such that every fibre of $X\mid R_0$ is irreducible and nonsingular. We
denote $R_o$ by $R$ and $X \mid R_0$ by $X$, so that $X$ is
nonsingular. At each point $(t,x) \in   X$ the tangent  directions to
$X$ form a projective line, with a marked point  namely the tangent
to the fibre $X_t$ at $(t,x)$. We thus get a \textit{projective line
  bundle} $\theta:\hat{F}\rightarrow X$ (bundle of tangent direction)
with a section $F^\infty$ (the tangent to the fibres); we set
$F=\hat{F} -F^\infty$; this  is an affine \textit{Line
  bundle} on $X$. For $t \in   R$. We set $\hat{F}_t =\hat{F}\mid X_t,
F_t=F\mid X_t$ and $F^\infty_t=F^\infty \mid X_t$. 

Sine an affine space ``carries functionally in the its structure'' its vector
space of ``translations''  $F_t$ admits a \textit{vector-line-bundle}
of translation $S_t$. 

\begin{lemma}\label{chap3:sec1:lem2}%lamm 2
  The\pageoriginale translation bundle $S_t\rightarrow X_t$ of $F_t\rightarrow X_t$
  is isomorphic to the tangent bundle $T(X)$. 
\end{lemma}

\begin{proof}%proof
  Let $T(X), T(R)$ be the tangent bundles to $X,R$ and $\tilde{T}(R)$
  be the pull back $\pi^\ast T(R)$ of $T(R)$ to $X$. We have the exact
  sequence  
  \[
  \xymatrix{
    0\ar[r] & T(X_t)\ar[r] & T(X)\mid X_t \ar[r] & \tilde{T}(R)\mid X_t
    \ar[r] & 0. 
  }\tag{1}\label{chap3:sec1:eq1}
  \]
  Local sections $\alpha ,\beta$ of $F_t\rightarrow X_t$ give in each
  fibre $T(X)\mid X_t$ supplementary subspaces of the corresponding
  fibres  of $T(X_t)$. Hence they  be viewed as local  spitting
  $\alpha  , \beta :\tilde{T}(R)\mid X_t\rightarrow T(X)\mid X_t$ of
  the exact sequence (\ref{chap3:sec1:eq1}). Thus their \textit{difference}
  $\alpha -\beta $ is a local section of Hom $(\tilde{T}(R)\mid X_t,
  T(X_t))$. Now, for a fixed $t \in   R$, the bundle
  $\tilde{T}(R)\mid X_t$ is trivial whence Hom $(\tilde{T}(R)\mid X_t,
  T(X_t))\simeq T(X_t)$. \phantom{WWWWWWWWWWWW}\hfill{Q.E.D.}
\end{proof}

\medskip
If the ``horizontal direction field'' $E$ exists, it is a section of
$F\rightarrow X$. For fixed $t \in   R, E_t=E \mid X_t$ is a  section
of $F_t\rightarrow X_t$ which then identifies  $F_t$ with
$T(X_{t}$)(Lemma \ref{chap3:sec1:lem2}) (fixing a  point in an affine  space makes it a
vector space). Now since ($2g-2)>0, T(X_t)$ admits only the zero
section; thus, if $E$ exists,  it is \textit{unique}. Furthermore, we
have, by Grauert's criterion, (Chapter I) a morphism of $T(X_t)$
into an affine space which contracts the zero section to a point and which
is biregular elsewhere. Thus we should look for a nice morphism $\varphi
_t$ of $F_t$ into an affine space the curves on the surface $F_t$
which are contracted to points by $\varphi_t$.    

In\pageoriginale order to be ``coherent'' with respect to $t$''  we will look for a
nice morphism $\varphi$ of $F$ into an affine spaces. We shall
successively: 
\begin{enumerate}[a)]
\item construct the morphism $\varphi$
\item study the ``blowing down set $E$'' of $\varphi$ (i.e. the set of
  all $y  \in   F$ such that $\dim_y \varphi^{-1}(\varphi(y)>0$)  and
  prove that it is a section of $F \rightarrow X$ 
\item prove that almost all  sections s:$R\rightarrow X$ are tangent to $E$
\item prove that the existence of a direction field $E$ or $X$ enables
  us to ``descend'' to a smaller field of definition for the generic
  fibre $X_r$ namely $k(r^p)$ ($p=$ charac. $k$, so that $k(r^p)=k$ if
  $p=0$) 
\item in case $p$= charac $k\neq 0$ lower the field of definition  of
  $X_r$ successively to $k(r^{p^s}), s\ge   1$; then use a lemma (a
  construction analogous  to the one used by Mumford in his theory of
  module) to prove that $X_r$ is birationally equivalent to a curve
  $C'$ defined over $\bigcap\limits_{s\ge 1}k(r^{p^s})=k$. 
\end{enumerate}


\section{The Proof}\label{chap3:sec2}

\noindent
\textbf{(A)~Construction of a morphism.}

For $t \in R$, the dual $T(X_t)^\ast$ of the tangent bundle $T(X_t)
\xrightarrow{\mu}X_t$ corresponds to the canonical $k_{-t}$ on
$X_t$. As $(2g-2)>0$ this bundle is ample and it follows that
$\exists$ a should section  $s$ of $T(X_t)^*$ such that the divisor
class of $(s)$ is $k_t$ on $X_t$ (class are denoted  by the same symbol
as the divisors) on the bundle $L=T(X_t)$ one may then define a homogeneous
linear map by. 
$$
\tilde{s}(y)=\big<s(\mu (y))\cdot y\big>, y \in   L.
$$

Completing\pageoriginale $L$ to a projective line bundle $\hat{L}$ and extending
$\tilde{s}$ to $\hat{L}$, we complete immediately the
divisor of $ \overset{\sim}{s}$ or $ \hat{L}$: 
$$
(\tilde{s}) = L^{\circ} - L^{\infty}+ \mu^{-1}(\underline{k}_{t})
$$
where $L^{\circ}, L^{\infty}$ are the null and infinite sections of $
\hat{L}$ and $ \mu:\hat{L}\to X_{t}$ the canonical projection. 

Take now any rational section $\sigma$  of  $\hat{F}_{t}$ over
$X_{t}$, as $X_{t}$ is nonsingular and $\hat{F}_{t}$ complete
$\sigma$ is in fact a section. If $ x \in F_{t}, x- \sigma(\theta(x))$
is in the affine space $L$ for almost all $x$ and the rational function
$ x \longmapsto \overset{\sim}{s}(x -\sigma (\theta(x))$ on $ F_{t} $
extends to a rational function $\alpha$ on $ \hat{F}_{t}$: 
$$
\alpha (x) = \overset{\sim}{s}(x - \sigma(\theta (x)), x \in F_{t}.
$$

The divisor class of $(\alpha)$ is then
$$ 
(\alpha) = M - F^{\infty}_{t} + \theta^{-1}(\underline{k}_{t}) -
\theta^{-1} (M.F^{\infty}_{t}) 
$$
where $ M = Im. \sigma , \theta : \hat{F}_{t} \longrightarrow X_{t}$
the canonical projection, with an identification $ F^\infty _t
\underset{\rightarrow} \backsim X_{t}$. Under this identification the
self-intersection of the divisor $ D = F^{\infty}_{t}$ on the ruled -
surface $\hat{F}_{t}$ corresponds to the canonical class
$\underline{k}_{t}$ on $ X_{t}$; in fact, this class is given by  
$$
F^{\infty}_{t} .(F^{\infty}_{t} + (\alpha)) = M.F^{\infty}_{t} +
\underline{k}_{t} - M F^{\infty}_{t} =  \underline{k}_{t} 
 $$

Consider the divisor $ nD = nF^{\infty}_{t}$ on $
\hat{F}_{t}$ for large values of $n$;the corresponding line bundle
$ L_{nD}$ on $\hat{F}_{t}$ induces, as is seen above, the\pageoriginale line
bundle $ L{_n \underline{k}_{t}}$ on $ F^{\infty}_{t}$. On $
\hat{F}_{t}$ we have the exact sequence of coherent sheaves 
$$ 
\xymatrix{
 0 \ar[r]& \mathscr{L}((n-1)D) \ar[r] & \mathscr{L}(nD)
 \ar[r] & \mathscr{L}({n\underline{k}_{t}}) \ar[r] & 0 
}
 $$
which gives a cohomology exact sequence: 

\begin{multline*}
  0\longrightarrow H^{\circ}(\hat{F}_{t},(n-1)D) \longrightarrow
  H^{\circ}(\hat{F}_{t}, nD) \xrightarrow{\gamma}
  H^{\circ}(F^{\infty}_{t}, \underline{k}_{t}\longrightarrow\\
  \xrightarrow{\beta} H' (\hat{F}_{t},(n-1)D) \xrightarrow{\alpha}
  H' (\hat{F}_{t}, nD) \longrightarrow H' (F^{\infty}_{t},
  (1-n)\underline{k}_{t}). 
\end{multline*}

On $F^{\infty}_{t}$, one has  $H^{1}(F^{\infty}_{t},
n\underline{k}_{t}) \underline{\backsim} H^{\circ}(F^{\infty}_{t},
(1-n) \underline{k}_{t}$ (cf. Serre \cite{7}, ch II) and if $n$ is large
enough $H^{\circ}(F^{\infty}_{t},(1-n) \underline{k}_{t}) = 0$ so that
$\alpha$ is surjective. Now the $H'(\hat{F}_{t}, nD)$ are finite
dimensional vector spaces over $k$ and from the surjectivity of $
\alpha_\lambda$ it follows that $ h^1(nD) = \dim_{k} H^{1}
(\hat{F}_{t}, nD)$ is a decreasing positive integral valued
function of $n$ and has therefore to remain constant for $n$ large. It
follows that $\alpha$ is an isomorphism for $n$ large and therefore
that $\gamma$ is surjective. Fix such an $n$. 

The surjectivity of $\gamma$ means that the liner system $\mid nD \mid
= \{(f) + nD : (f) \geq - nD\}$ induces on $D = F^{\infty}_{t}$ the
complete linear system $\mid n \underset{-}k_t |;$ since
$|n\underset{-}k_t|$ has no base point on $F^{\infty}_t$ and as $nD$
as a member of $|nD|$, it follows that $|nD|$ has no base point. This
means that the rational map $\varphi_{nD}$ of $\hat{F}_t$ into a
projective space, defined by a basis for $L(nD)$ is a \textit{
  morphism } (Chapter I). On the other hand $n \underline{k}_{t}$ is
very ample for $n$ large (Chapter II)and the  surjectivity\pageoriginale of
$\gamma$ means that $\varphi_{n\underline{k}_{t}}$ is induced on
$F^{\infty}_{t}$ by $ \varphi_{nD}$. Let $1 = u_{\circ}, u_{1}, \ldots
, u_{r}$ be a basis of $L(nD)$. We set $g_t = \varphi_{nD}$; we have
thus proved a major part of 

\setcounter{lemma}{0}
\begin{lemma}\label{chap3:sec2:lem1}%lemma 1.
  \begin{enumerate}[\rm (i)]
  \item $ g_{t}(F_{t}^{\infty})\subset H$, the hyperplane at $\infty$
    in $\mathbb{P}_{r}$, for the affine coordinates $ (u_{1}(x),
    \ldots , u_r(x)); g_t \mid F^{\infty}_t$ is an imbedding.  
  \item  $g_t(F_t)\subset \mathbb{P}_r - H$.
  \end{enumerate}
\end{lemma}
 
\begin{proof}%proof
  If $ u_1$ is the function with a pole of maximum order along $
  F^\infty_t$ among the $ u_i$, it is clear that, for 
  $$ 
  x \in F^\infty_t, g_t(x) = \left(\dfrac{1}{u_1(x)}, \ldots
  ,\dfrac{u_r(x)}{u_1(x)}\right) \in H.
  $$

  Also, by choice, $ g_t\mid
  F^\infty_t$ is a multicanonical imbedding. This proves $(i)$ To
  prove $(ii)$ it is enough to observe that the polar varieties of
  $(u_i)$ lie in $F^\infty_t$. \phantom{WWWWWWWWWWWWWWWWWWWw}\hfill {Q.E.D}
\end{proof}

Fix a generic point $t_\circ$ of $R/K$. The varieties $
F^\infty_{t_\circ}, \hat{F}_{t_\circ}, F_{t_\circ}$ are all
defined over $k(t_\circ) = k(R) = K$ and $D = F^\infty_t$ is a
$K$-rational divisor on $F_{t_\circ}$. Thus, by the ``last theorem of
Weils Foundations'' the $ u_i$ can be assumed to be functions in $
k(t_\circ)(\hat{F}_{t_\circ}) = k\hat{(F)}$ and defined over $
K(t_\circ) = K$. As the homogeneous system $(u_i)$ of functions on
$\hat{F}$ does not have common zeros on the generic fibre $
\hat{F}_{t_\circ}$, by restricting $t$ to an open subset of $R$, we
may assume that the system $ (u_i)$ does not have common zeros on $
\hat{F}$. Then we and define a homomorphism $ \hat{F}
\rightarrow \mathbb{P}_r$ by $ x \longmapsto (1 = u_\circ(x), \ldots ,
u_r(x))$. 

The following two lemmas are then easy deductions from lemma 1.
\begin{lemma}\label{chap3:sec2:lem2}%lemma 2.
  \begin{enumerate}[\rm (i)] 
  \item $g(F^\infty) \subset H$,\pageoriginale the hyperplane at $\infty$ in
    $\mathbb{P}_r$ for the affine coordinates $(u_1(x), \ldots ,
    u_r(x))$; also, $g$ restricted to each $F^{\infty}_t$ is an
    imbedding. 
  \item $g(F) \subset \mathbb{P}_r -H.$
  \end{enumerate}
\end{lemma}

Thus, if we define a morphism $\hat{\varphi}=\pi \theta \times g :
\hat{F}\rightarrow R \times \mathbb{P}_r$, then: 
\begin{lemma}\label{chap3:sec2:lem3}%lemma 3.
  \begin{enumerate}[\rm (i)]
  \item  $\hat{\varphi} \mid F^\infty$ is biregular into $ R \times H$
  \item $\hat{\varphi} (F) \subset R \times (\mathbb{P}_r -H)$.
  \end{enumerate}
\end{lemma}

\noindent
\textbf{(B) The contraction set of the morphism {\boldmath $\hat{\varphi}$.}}

\begin{defi*}%defin
  Let $ Y \overset{g}\longrightarrow z$ be a dominant morphism of
  varieties; the {\em contraction} set $E(g)$ of $g$ is, by
  definition, the set 
  $$ 
  E(g) = \bigg\{ y \in Y : \dim_y(g^{-1} g(y))>0\bigg\}.
  $$ 
\end{defi*}

Our aim in this section will be to study the contraction set $ E
(\hat{\varphi)}$ of the morphism $\hat(\varphi)$ we have
constructed above. 
\begin{enumerate}[(i)]
\item \textit{$E(\hat{\varphi}$) \text{is a closed subset of}
  $\hat{F}$.} 

  Infact, we will prove that the contraction set $E(g)$ of an arbitrary
  dominant morphism  $Y \overset{g}\longrightarrow Z$ is closed; by
  an obvious reduction one may assume successively that $Y$ is normal,
  $Z$ is normal, and then by replacing $Z$ by its normalisation in
  $k(Y)$ that $g$ is birational (note that the case $\dim Z= \dim Y$
  is the only non-trivial one). But then, by $ZMT$ $g$ is a local
  isomorphism an all points $ y \notin E(g)$. Our assertion follows. 
\item  If\pageoriginale $\varphi = \hat{\varphi} \mid F$, then
  $E(\varphi) = E\hat{(\varphi)}$. 
  
  This follows from (i) of Lemma \ref{chap3:sec2:lem3}, (A). In particular. $E =
  E(\varphi) \subset F$. 

\item $E_t = E \mid F_t = E \cap F_t = E(\varphi \mid F_t)$.

  Follows from the fact that $\varphi$ separates fibres.

\item $E_t$  is complete.

  Follows from (iii) and (i).
\item  \textit{If $E_t \neq \phi$  then  $\theta \mid E_t: E_t
  \longrightarrow X_t$ is a bijection.} 
\end{enumerate}

Let $ \overset{\sim}{E}_t = \bigg\{e - e' \mid e,e' \in E_t$  such
that $\theta (e) = \theta (e')\bigg\}$. 

Then $ \overset{\sim}{E}_t \subset T(X_t) ; \overset{\sim}{E}_t$ is
one-dimensional, since $ E_t$ is one-dimen\-sional: Furthermore the map
$ E_t \underset{\times_t}{\times} E_t \longrightarrow T(X_t)$ given by
$ (e,e') \longmapsto (e-e')$ has $E_t$ for its image which is
therefore complete. 

Now, as the cotangent bundle $T (X_t)^\ast$ on $ X_t$ is ample $((
2g - 2) > 0),$ it follows, by Grauert's criterion of amplitude
(Chapter I, \S 2), that $\exists$ a morphism $ \tau :T(X_t)
\rightarrow U, U$ affine, such that $\tau$ contracts the null section
of $T(X_t)$ to a point and is biregular outside it; as $
\overset{\sim}{E}_t \subset T(X_t)$ is complete, its image under
$\tau$, which is affine, reduces to a finite subset of $U$. From the
one-dimensionality of $ \overset{\sim}{E}_t$ and the biregularity of
$\tau$ outside the null section of $T (X_t)$, one deduces that $
\overset{\sim}{E}_t$ is contained in the null section of  $T (X_t)$,
in other words, that $ e = e'$ if $\theta (e) = \theta (e'), e,e' \in
E.$ 

Consider\pageoriginale the morphism $ \varphi = \hat{\varphi} \mid F$; we have
seen that $\varphi (F) \subset R \times (\mathbb{P}_{r} -H)$ which is
affine; replacing the affine closure $ \overline{\varphi(F)}$ in $ R
\times (\mathbb{P}_r -H)$ by its normalisation in $k(F)$ we may assume
that $\varphi$ is a morphism of $F$ into an affine space $A$, which is
\textit{birational} onto $ \varphi(F)$, and therefore (by $ZMT)$
\textit{biregular outside the contraction set $E$}. 

\medskip
\noindent
\textbf{(C)~ Finiteness of sections of $ X / R $ not tangent to $E$. }

Let $s : R \rightarrow X $  be any section of $ X / R$. For $t \in R$,
the tangent to $s(R)$ at $s(t)$ is then well defined and not
``vertical'' and is thus in $F_t$; one can therefore define a section
$\overset{\sim}{s} : R \rightarrow F$ so that $ s = \theta
. \overset{\sim}{s}$. To say that ``$s$ is tangent to $E''$ means
precisely that $\overset{\sim}{s}(R) \subset E$. Denote by $\sum$ the
set of all sections $s$ of $ X / R$ and by $\sum'$ the subset of all $s$ such
that $ \overset{\sim}{s}(R) \not\subset E$. 

\setcounter{proposition}{0}
\begin{proposition}\label{chap3:sec2:prop1} %propo 1.
  {\em $\sum'$ is a finite subset of $\sum$.}
\end{proposition}

\begin{proof}%proof
  For any section s of $ X | R$ the composite $\varphi
  \circ\overset{\sim}{s}$ belongs to Mor $(R,A) = V$. 
\end{proof}

\begin{enumerate}[a)]
\item The map $s \longmapsto \varphi o \overset{\sim}{s}$ is a map of
  $\sum$ into $V$ and is \textit{ injective } on $\sum^{'}$. 

  The first assertion is trivial; to prove the second, take $s,s' \in
  \sum , s \ne s'$, such that $ \overset{\sim}{s} (R) \not \subset E $
  and $\tilde{s'} (R) \not \subset E;$ then    $\exists . t
  \in R$ such that $\overset{\sim}{s}(t)\ne \tilde{s'}(t)$ and
  $\overset{\sim}{s}(t),\tilde{s'}(t) \not\subset E$. By the
  biregularity  of $\varphi$ outside $E$, one gets $\varphi o
  \overset{\sim}{s} (t) \neq \varphi \circ \tilde{s'} (t)$. 
\item The elements $ \varphi o \overset{\sim}{s}, s \in \sum$, belong
  to a \textit{finite dimensional}  vector subspace $ V_{1}$ of $V$. 

  For\pageoriginale a large $q$, one has $F$ imbedded   in $ R \times \mathbb{P}_q$;
  let  $\overline{R}$ be a nonsingular projective closure of $R$ and
  $\overline{F}$ be the closure of $F$ in $\overline{R} \times
  \mathbb{P}_q$. Any section $s:  \overset{\sim}{R} \longrightarrow$
  extends to a rational section $\overline{s}: \overline{R}
  \longrightarrow \overline{F}$ in a natural way. Also, because
  $\overline{R}$ is non singular and $\overline{F}$ complete,
  $\overline{s}$ is a section; 
  $\overline{A}$ is a projective closure of $A$, the morphism
  $\varphi :F \to A$ extends to a rational map $\overline{\varphi}:
  \overline{F} \to \overline{A}$ and the composite
  $\overline{\varphi} o \overline{s} : \overline{R} \longrightarrow
  \overline{A}$ is a morphism.  

  Let $(\varphi_1, \ldots , \varphi_d)$  be the coordinate functions
  of $\varphi$; each  $\varphi_i$  is finite on $F$  but may have
  poles on $ \overline{F} - F \subset (\overline{R}-R) \times
  \mathbb{P}_q$. Chose a rational function $u$ on $\overline{R}$ such
  that each $ u \varphi_i$ is finite on $(\overline{R}-R) \times
  \mathbb{P}_q$; then $u \overline{\varphi}_i$ is
  finite on $(\overline{R}-R) \times \mathbb{P}_q$. Therefore the
  composite $u(\overline{\varphi} o \overline{s}): \overline{R}
  \longrightarrow \overline{A} $ is finite on $\overline{R}-R$; if
  $(t_1, \ldots , t_l)$ are the coordinate functions of $\varphi \circ
  \overset{\sim}{s}$, it follows that the $t_j$ which are rational
  functions on $\overline{R}$ have the property: 
  $$
  (t_i) \ge - h ~\text { on } ~\overline{R} - R
  $$
  where $h$ is the polar divisor of $u$ on $\overline{R}- R$ ; in
  other words, we have $ t_i \in L(h)$ on $\overline{R}$ and this is a
  finite dimensional vector space. Our assertion follows. 
\item The elements $\varphi o \overset{\sim}{s}, s \in \sum'$, form a
  closed subset of the linear variety $ V_{1}$. 
 
  In fact, if $ v \in V_1 (\subset ~ \text { Mor } ~ (R,A)) $ is of
  the form $\varphi \circ \overset{\sim}{s}, s \in \sum$, then it has the
  following properties: 
  
  \begin{enumerate}[(1)]  
  \item $ v(R) \subset \varphi (F)$\pageoriginale (More precisely
    $\forall  t \in R, v(t) \in \varphi(F_t)$). 
  \item if $ v(R) \not \subset \varphi (E)$ then $\varphi^{-1}.v$ is
    defined and is a  
    section $ R \rightarrow F$; also $\theta. \varphi^{-1}\cdot v$ is a
    section $ R \rightarrow X$; furthermore the ``direction function''
    involved in $\varphi^{-1}v$ must be the ``derivative'' (i.e must
    give the tangent direction) of the ``point function''
    $\theta \cdot \varphi^{-1}\cdot v$ with the above notation we can write as $
    \theta \widetilde{\varphi^{-1}v} = \varphi^{-1} v$. 
  \end{enumerate}

  Conversely, if $ v \in V_1$ satisfies $(1)$, we have either $ v(t) =
  \varphi (E_t) \forall t, $ so that $v$ comes from a section tangent
  to $E$ or $(2)$ holds so that $ v = \varphi o \underset{\sim}{s}$ ~
  with ~ $s = \theta \varphi^{-1}v$. Now notice that (1) and (2)
  are algebraic conditions on $v\in V_1$ (taking into account the
  fact that there is at most one $v_o \in V_1$ such that $v_1(t) =
  \varphi (E_t) \forall t \in R)$. 

  Therefore $Im  \sum, ~ Im \sum'$ are closed subs.of the linear
  variety $V_1$. 

  As the map $ \sum \rightarrow V_1$ is injective on $ \sum'$ (by a))
  the algebraic structure on $ Im \sum'(c))$ can be pulled to
  $\sum'$. To prove Proposition \ref{chap3:sec2:prop1}, it is then enough to prove that: 
\item $\dim. \sum'=0$

  We shall show that any morphism $ Q\overset{\psi} \rightarrow \sum'$
  of on irreducible curve $Q$ into $\sum'$ is constant. 
\end{enumerate}

We first remark that $Q$ may be assumed to be complete. Indeed for every
fixed t $\in$ R the morphism 
\begin{gather*}
  Q \rightarrow X_t\\
  q \longmapsto \psi (q) (t) 
\end{gather*}
admits as extension a morphism  $ \bar{Q} \overset{\psi} \rightarrow
X_t$ ($\bar{Q}$ a projective closure of $Q$) so that  $\psi _q=\psi(q)$ is a
section in $\sum'$, for all $ q \in \overline{Q}$. 

Now\pageoriginale consider, for a fixed $t\in R$. the morphism
\begin{gather*}
  \bar{Q} \longrightarrow A \\
  q  \longmapsto \varphi \circ \overset{\sim}\psi_q (t).
\end{gather*}

The image, being affine, is point ; this means that $\varphi o 
\tilde{\psi}_q(t)=\varphi o \tilde{\psi}_{q'}(t)$ for all
$q, q'\in \bar{Q}$. This is true for all $t \in R $ and as $\varphi$
is biregular outside $E$ while $ \tilde{\psi}_q (t)  \notin E $
for almost all $t$ it follows that $\overset{\sim}\psi_q
(t)=\tilde{\psi}_q (t)$ for almost all $t \in R$. One then
concludes that $\psi_q=\psi_{q'}$. \hfill Q.E.D.

\medskip
\noindent
\textbf{(D) Case of an infinity of sections.}

\begin{proposition}\label{chap3:sec2:prop2}%propo 2.
  If $X \overset{\pi} \rightarrow R$ admits an infinity of sections then 
  \begin{enumerate}[a)]
  \item  $E$ is an irreducible surface on $F$
  \item  $ \theta | E : E \rightarrow  X $ is biregular.
  \end{enumerate}
  In view of Proposition \ref{chap3:sec2:prop1}, it follows that there exists at least
  one section $s: R \rightarrow X$ which is tangent to $E$ i.e., $s(t)
  \in E_t $ for  all $t$; this means that $ E_t \neq \phi$ ~ for
  any~ $t \in R$. Assertion $a)$ is  proved. 
\end{proposition}

Regarding b), as we have already seen that $  \theta \mid E_\alpha$
bijective we will be through, by $ZMT$, if we prove the birationality
of $\theta | E$. 

Characteristic $0$ offers no trouble. \textit{We assume therefore that
  charac $ k=p,p\neq 0$, to prove the birationality of $\theta | E$.} 

We\pageoriginale have a chain
$$
k \subset k(R) \subset k (X).
$$

Let $k(X) =  k(x_1, x_2, x_3)$ and assume that $x_1,x_2$ form a
separating bases of $k (X)|k$. We consider the minimal equation 
\begin{enumerate}[1).] 
\item  $ G(x_1,x_2,x_3)=0$ with an  irreducible $ G \in k [X_1,
  X_2,X_3]$ with\break  $G'_{x_3} (x_1,x_2,x_3) \ne 0$. 
  
  A tangent direction at $(x) \in X$ is defined by homogeneous
  coordinates $ ( y_1,y_2,y_3)$ satisfying 
\item $  \sum\limits_{i=1}^{3}y_i G'_{x_i} (x)=0$.
  
  As $G'_{x_3}(x) \ne 0$ this direction is defined completely by
  $y=y_2/y_1$. 

  Now since $\theta | E$ is  bijective, $\theta | E$ is purely
  inseparable so that 
\item  the direction $E$ is defined by an equation
  $$
  y^{p^n}=H (x_1,x_2,x_3);
  $$
  we assume that $H$ is \textit {not } a $p^{th}$ power in $k(X)$.
 
  Any section $s$ of $X \rightarrow R$ defined by $s_1,s_2,s_3 \in
  k(R)$ such that $ G (s_1,s_2,s_3)=0$. Then the tangent to $s(R)$ the
  parameters $Ds_1$, $Ds_2$, $Ds_3$ (where $D$ is any non-trivial derivation
  of  $k (R)/k)$. To say that $ \overset{\sim} s  (R) \subset E $  is
  then equivalent to 
\item $\left(\dfrac{Ds_2} {Ds_1}\right)^{p^n}= H (s_1,s_2,s_3)$.
 
  If\pageoriginale $p^n=1$, then trivially $\theta \mid E$ will be birational;
  assume then that $ p^n \neq 1$. 

  Now deriving $4$ we obtain
\item $\hspace{1.5cm} 0 = \sum\limits^{3}_{i=1} H_{x_{i}}' (s)Ds_i$.

  Consider  now the locus $Y$ of the tangent direction

  $\left( (x_1, x_2, x_3), \dfrac{G'_{x_1} H'_{x_3} -H'_{x_1}
    G'_{x_3}}{G'_{x_2} H'_{x_3} + H'_{x_2} G'_{x_3}}  \right)$ at
  $(x_1,x_2,x_3)$,over $k$. Then $y$ is  
 
  an irreducible variety whose elements satisfy
\item $ \hspace{1.5cm} \sum\limits_{i=1}^{3} H'_{x_i}(x) y_i=0.$
\end{enumerate}

We claim that $Y$ is not the whole of $F$; in fact, if it were, for
any $k$-derivation $ \Delta $ of $K(X) =k(x_1,x_2,x_3)$, the tangent
direction $(\Delta x_1,\Delta x_2,\break\Delta x_3)$ will be in $Y$ so that  
$\sum H'_{x_i} (x)\Delta x_i=\Delta(H(x))=0$. This means that $H$ is a
$p^{th}$ power. Thus $Y$ is a surface.  

By the very definition of $Y, \theta \mid Y$ is birational  from $Y$
onto $X$. from 5), it follows that for any  $s  \not\in
\sum',\tilde{s}(R)\subset Y$. By hypothesis, there are an infinity of
$s \not\in\sum'$ (Proposition \ref{chap3:sec2:prop1});  thus, the two irreducible surfaces
$E$ and $Y$ have an infinity  of common curves, and therefore must
coincide: $E=Y$. In particular, it follows that $ \theta \mid E$ is
birational and $p^n = 1$ in 3).\hfill{Q.E.D} 

\medskip
\noindent
\textbf{(E)~Conclusion in charac. $p \neq 0$.} 
 
\begin{proposition}\label{chap3:sec2:prop3}%propo 3.
  With\pageoriginale the same notation as before, let $C$ be a curve defined over
  $K=k(R)$ such that 
  $$
  \text{ genus } {_K}^{C}=\text{ absolute genus C }\geq 2.
  $$
  If $C_K$ is infinite, then $C$ is birationally equivalent, over $K$,
  to a curve $C_1$ defined over $K^P$.    
\end{proposition}

\begin{proof}%proof
  Proceeding as in (D), we may first assume that $x_1 \in k(R)$. Let
  $D$ be the nontrivial derivation of  $k(R)/k$ such that $Dx_1=1$. We
  extend this to a derivation $D$ of $k(X)$ such that $
  Dx_2=H(x)$. (see 3) of the proof of proposition
  \ref{chap3:sec2:prop2}, (D). Then one has  
  \begin{align*}
    &D^P k(R)=0 \qquad \text{and}\\
    &G'_{x_1}+G'_{x_2} H +G'_{x_3} Dx_3 =0.
  \end{align*}
\end{proof}

Take any section $s_\alpha =(s^\alpha _{1},s^\alpha_{2},s^\alpha_{3})$
of $X/R$; then $s^\alpha_{1}=x_1$ and  
\begin{gather*}
  G(s^\alpha_{1},s^\alpha_{2},s^\alpha_{3})=0. \text{ Also },\\
  {\tilde{s} \alpha }(R)\subset  E \text{ means } Ds_{2}^{\alpha }=H
  (s^\alpha_{1},s^\alpha_{2},s^\alpha_{3}).  
\end{gather*}
 
Now $s^{\alpha }(R)$ is a curve on $X$ thus defines a discrete
valuation ring $\mathscr{O}_\alpha $ in $k(X);s^{\alpha }(R)$ is
k$(R)$-rational means that, for the canonical homomorphism  
$$
\sigma_{\alpha }:\mathscr{O}_{\alpha }\rightarrow k(R)=K
$$
of\pageoriginale $\mathscr{O}_\alpha $ onto its residue field, we have 
$$
\sigma_\alpha  (x_i)=s_{i}^{\alpha }. 
$$

The equations that we have obtained above mean that $\sigma_\alpha
D=D\sigma_\alpha $ or, more precisely, that if $D'=D\mid k(R)$ then 
$$
\sigma_\alpha  D=D'\sigma_\alpha.
$$ 

Iteration gives: $\sigma _\alpha  D^p=D'^{p} \sigma_{\alpha }=0$.  As
this is true for an infinity of $\alpha '$ s we conclude that
$D^p=0$.( if $y \in  k (X),\sigma_{\alpha }(D^p y) =0$  means that
$D^p y,$ which is a rational function, has $s^{\alpha }(R)$ as part of
its polar variety; recall that a rational function cannot have an
infinite number of polar varieties). 

Consider now the diagram of field extensions:
\[
\xymatrix{L^p \ar@{-}[r]\ar@{-}[dr] & \Ker D \ar@{-}[d] \ar@{-}[r] &
  \ar@{-}[d] k(X)=L\\
k \ar@{-}[r] & K^p \ar@{-}[r]^p & K= k(R).
}
\] 
 
As $K$ and $L$ are function fields of one and two variables
respectively over the algebraically closed $k$, it follows that
$[K:K^P]=P$ and $[L:L^p]=p^2 $. We also have $ {\dim}_{L_p}(\ker
D^p)\leq p. \dim_{L_p} (\ker D)$. (This can be proved easily as
follows: if $u, v\in $ End $V$, $V$ a vector space, then the sequence  
$$
\displaylines{\hfill 
0\longrightarrow {\Ker}~ v \longrightarrow {\Ker} ~uv \xrightarrow{v}
\Ker u \hfill \cr
\text{is exact so that}\qquad 
\dim (\ker ~uv)\leq \dim (ker ~u)+\dim (\ker~ v).\hfill }
$$

Recalling\pageoriginale that $\ker D^p=L$, one gets   
$$
p^2\leq p.\dim_{{L^p}} (\ker ~D)
$$  
i.e., $[\ker D:L^p]\leq  p$; as $\ker D\neq L$, it follows that $[\ker
  D:L^p]=[L: \ker D]= p$. Consequently, $K(=k(R))$ and ker $D$  are
$K^P$-linearly disjoint; thus if $C_1$ is a (nonsingular) curve over
$K^P$ such that  $K^P(C_1)= \ker D$, then $K(C_1)=K  (\ker D)
=L=k(X)=K(C)$.  \hfill{Q.E.D.}       

\setcounter{corollary}{0}
\begin{corollary}\label{chap3:sec2:coro1}%coroll 1.
  {\em $(C_1)_{k^P}$ is infinite.}
  With notations as in the proof of the above proposition, let
  $\mathscr{O}'_\alpha =\mathscr{O}_\alpha  \cap K^p (C_1);$ then the
  $\mathscr{O}'_{\alpha }$ are discrete valuation rings in   $ K^p
  (C_1)$   and are infinitely many in number (since $L/K^p (C_1))$ is
  purely inseparable, $\mathscr{O}_\alpha  \neq \mathscr{O}_{\alpha'}
  ,\Rightarrow \mathscr{O'}_{\alpha '}\neq \mathscr{O}'_{\alpha '}$.To
  prove that the $\mathscr{O}_\alpha  '$   define $ K^p$-rational
  sections of $C_1,$ one has to prove that $\sigma_{\alpha }
  (\mathscr{O'}_\alpha ) \subset K^p$.       
\end{corollary}

In fact, if $ \sigma_\alpha  (y)=\bar{y}, y \in K^p (C_1) =\ker D$,
then $D\bar{y}=D \sigma_\alpha  y =\sigma_\alpha  Dy=0, $ whence
$\bar{y} \in K^p$. 

\begin{corollary}\label{chap3:sec2:coro2}%coroll 2.
  \begin{align*}
    {\rm genus}_{{K}^p} C^1 & =  {\rm absolute ~ genus ~of}~  C_1\\ 
    & =  {\rm absolute ~genus ~of}~ C.
  \end{align*}
\end{corollary}

We shall prove the corollary by showing that $\exists$   a projective
model $D_1$ of $C_1$ over  $ K^p$, such that $D_1$ is
\textit{absolutely normal}. (We recall here that the genus drop of a
curve for extension of base  fields comes from nonabsolutely normal
points on it.) 

By\pageoriginale corollary \ref{chap3:sec2:coro1}, on $C_1$ there exist an infinity of $K^p$-rational
points; we form a divisor $\mathcal{O}_1$ on $C_1$, with such points,
and with such a large degree that $\mathcal{O}_1$ is very ample; in
view of proposition \ref{chap3:sec2:prop3}, this induces a very ample divisor
$\mathcal{O}$ on $C$. By the last theorem of Weil's Foundations, one
obtains an isomorphism over $K$   
$$ 
L_K(\mathcal{O},C)\overset{\sim}{\leftarrow} \underset{K^p} L
(\mathcal{O},C_1)\underset{K^p}\otimes K. 
$$
   
Thus, we may assume that there exist rational functions in $
K(C_1)=K(C)$, which give a projective imbeddings   $C
\overset{\eta}{\longrightarrow} C'$, $C_1\overset{\eta
}{\longrightarrow}C_1'$  in the same $\mathbb{P}_n$. We may assume in
addition that $\exists$ points $P_{1,0,}P_{1,1,\ldots,} P_{1,n,}
P_{1,n+1}$ in $C_1$, corresponding to some of the above defined
valuation rings $\mathcal{O}_\alpha $ such that 
$$ 
\eta_1(P_{1,i})=(0,0\ldots , \overset{i}{1},\ldots, 0)\in
\mathbb{P}_n, 0\leq i \leq n, 
$$
and $\eta_1(P_{1,n+1})=(1,1,\ldots, 1) \in \mathbb{P}_n$. After a projective transformation, we may also assume that the corresponding points $ P_{0,\ldots, }P_{n+1} \in C$ are such that 
\begin{align*}
  \eta & (P_i)  = (0,0,\ldots \overset{i}{1},\ldots, 0) \in
       {P}_n,0\leq i \leq n,\\ 
       \eta & (P_{n+1})  = (1,1,\ldots, 1)\in {P}_n.
\end{align*}

\noindent
\textbf{This follows from:}  $D \sigma_\alpha  =\sigma_\alpha  D$. But
as the national functions defining $\eta$ and $\eta_1$ generate the
same vector space there is a projective transformation $u$ of
$\mathbb{P}_n$ such that. 
\[
\xymatrix@C=2cm{C\ar[dd] \ar[r]^\eta & C'\ar[dd]^{u/C'}\\
  birational &\\
  C_1 \ar[r]^{\eta_1} & C'_1
}
\]
is\pageoriginale commutative. But as  $u$ fixes the base points $(0,\ldots ,
0,\overset{i}{1},0,\ldots ,0) $ and the unit point $(1,1,\ldots,1)$ in
$\mathbb{P}_n, u$ has be the identity in $\mathbb{P}_n$; so that
$C'=C'_1$. By hypothesis $C$ is absolutely normal and thus $C'_1=C'$
is the  absolutely normal model for $C_1$ that we were looking
for. \hfill{ Q.E.D.} 

This done, one may now suppose that curve $ C_1$ again satisfies the
conditions in the enunciation of proposition \ref{chap3:sec2:prop3}
(Corollaries \ref{chap3:sec2:coro1}, \ref{chap3:sec2:coro2}). Thus, one obtains, by iteration,  a sequence $(C_i)_{i\geq 0,}
C_\circ =C,$ of absolutely normal curves defined respectively over
fields $K_i/k$ and such that $C_i \underset{K_{i}} {\sim}
C_{i+1}$. The proof of the Grauert-Manin theorem will thus be
completed, in charac. $p\neq 0$, by the following two lemmas. 

\begin{lemma}\label{chap3:sec2:lem4}%lemma 4.
  If $K$ is an algebraic function field over an algebraically closed
  $K$, charac. $k=P\neq 0$, then $\bigcap\limits_{n\geq 0}
  K^{p^n}=k$. 
\end{lemma}

\begin{proof}%proof
  $F=\bigcap\limits_{n\geq 0} K^{p^n}$ is clearly perfect; secondly,
  as $F$ is a subfield of a finite type extension of $k, F$ is also of
  finite type over $k$. We claim that $F$ is purely  algebraic over $k$;
  in fact, for any finite type extension $L/k,[L:L^p]$ equals $p^d$ means
  $d=t \tr \deg_k^L$ as $F=F^P$ our lemma follows. 
\end{proof}
 
\begin{lemma}\label{chap3:sec2:lem5}%lemma 5.
  Let\pageoriginale $(C_n)_{n\geq 0}$ be a sequence of also normal curves, each
  $C_n$ defined over a $K_n$, such 
  that $k=\bigcap\limits_{n\geq 0} K_n$ is algebraically
  closed. Assume that, for each $n$, genus$_{K_n} C_n$(=absolute genus of
  $C_n) \geq 2$ and that $C_n \sim C_{n+1}$ over some field. Then each
  $C_n$ is birationally equivalent to a curve $D$ defined over $K$. 
\end{lemma}

\begin{proof}%proof
  Let $g=genus~ C_n(\forall n)$. As $g\geq 2$ by hypothesis, each
  $C_n$ is tricanonically  
  imbedded in $\mathbb{P}_{5g-6}$, as a positive one dimensional cycle
  of degree $(6g-6)$. Let $G=\mathbb{P}GL(5g-6)$. Then $G$ acts, in a
  natural way, on the Chow variety $Z$ of positive one  dimensional
  cycles  of degree $(6g-6)$ in $\mathbb{P}_{5g-6}$ ( we recall that
  this variety is defined over a ``small'' subfield of $k$ and in
  particular over $k$ itself.) This action is given by a morphism $G
  \times Z \longrightarrow Z$. As $C_n$ and $C_{n+1}$ are birationally
  equivalent, the Chow points $x_n$ of $C_n$ all lie in the same orbit
  $V=Gx_n$ for this action of $G$. 
\end{proof}

Let $\bar{V}$ be the closure of this orbit $V$ in $Z$. As $x_n$ is
$K_n$-rational, $\bar{V}$  is defined over $K_n$; thus the smallest
field of definition for $\bar{V}$ is contained in $K_n$ for each $n$
and hence in  $k$. 

By the Hilbert-Zero -Theorem it follows that $\exists$ an $x \in V$,
rational over $k$; then $x$ is the Chow point of a curve $D$ in
$\mathbb{P}_{5g-6}$ defined over $k; x \in V$ means that $C_n \sim
D$. \hfill{Q.E.D.}  

\begin{remark*}%remark
  We have made same construction above as Mumford has done in
  construction the moduli variety for curves. But when as Mumford
  naturally had to consider the entire orbit space we had to deal only  with  a
  single orbit; hence our result is quite elementary. 
\end{remark*}

\noindent
\medskip
\textbf{(F)~Conclusion in charac. 0.}\pageoriginale

\begin{proposition}\label{chap3:sec2:prop4}%propo 4.
  Let $C$ be e curve defined over function field of one variable $K/k$
  of genus $\geq 2$. If $C_K$ is infinite, then $C\sim D$. a curve
  defined over $k$.  
\end{proposition}
 
\begin{proof} %proof
  With the same notations and procedure as above, we obtain a tower
  $k\subset k(R)=K \subset k (X)=L$. A contrivial derivation $D'$ of
  $K/k$ is extended to $D$ on $L$; corresponding to an infinity of
  sections of $X\mid R$, we obtain, as before, valuation rings
  $\mathscr{O}_\alpha $ of $L/K$ such that the residue field of each
  $\mathscr{O}_\alpha $ is K and such that if $\sigma_\alpha
  :\mathscr{O}_\alpha  \longrightarrow K$ is the canonical
  homomorphism, then $\sigma_\alpha  D=D' \sigma_\alpha$. 
\end{proof}

Now observe that for any valuation ring $\mathscr{O}$ of $L/K,
D\mathscr{O}\subset \mathscr{O}$ (geometrically if $f$ is a  function
on $X$, which is finite along the divisor defined by $\mathscr{O}$,
then Df is also finite along that divisor; this is due to the fact
that the direction field $E$ is a morphism $X\longrightarrow F$, not
merely a rational map). Consider now that formal power   series fields
$K((T))\subset L ((T))$. We define an automorphism $S$ of $L((T))$(
that restricts to an automorphism of $K((T))$ as follows: 
\begin{align*}
  S(T) & = T\\
  S(y) & = y + T Dy + \cdots +T^n\frac{D^n y}{n!}+\cdots, y \in L.\\
\end{align*}

(One can divide by $n!$  in charac. $0$ !) The Leibnitz' formula
readily shows that $S(yy')=S(y)S(y')\forall y,y'\in L$. Also, as
$\sigma_\alpha  D=D' \sigma_\alpha $ one obtains  
$$ 
\sigma_\alpha  S=S \sigma_\alpha.
$$

Now\pageoriginale consider the curve $C$ as a over $K((T))$; its function field is
then $K((T))(x,y)$ where $L=K (x,y)$ is the function field of
$C/K$. The automorphism $C$ of $K((T))$ defines a curve $C^S$, the
$S$-conjugate of $C$ (replace the coefficients in the defining
equations of $C/K((T))$ by their $S$-conjugates). Then  
\begin{align*}
  \text{genus}_{K((T))^C} & = \text{genus~ (absolute)~ of~ C}\\
  & = \text{absolute~ genus~ of~ }C^S\\
  & = \text{genus}_{K((T))} C^S\geq 2.
\end{align*}  

Also the $K((T))$-function field of $C^S$ is $K((T))(Sx,Sy)$; we
claim that $C$ and $C^S$ are birationally equivalent over $K((T))$. In
fact, in view of the above relation on their genera and in view of
Hurwitz-Zenthen (Chapter II) it is enough to prove that  
 $$ 
 K((T))(Sx,Sy)\subset K ((T))(x,y).
 $$  
 
 To start with we know already that  $D\mathscr{O}_\alpha  \subset
 \mathscr{O}_\alpha $ for all $\alpha$; in addition if $\mu_\alpha  $
 is the maximal ideal of $\mathscr{O}_\alpha  $ then
 for $x'\in \mu _\alpha$ one has $\sigma_\alpha  (Dx')=D(\sigma_\alpha
 x')=0$, i.e. $Dx'\in \mu _\alpha$ so that  $D\mu_\alpha \subset \mu
 _\alpha $. Also we claim that the valuation $v_\alpha  $ on $L/K$
 defined by $\mathscr{O}_\alpha $ has the property 
 $$
 v_\alpha (Dz)\geq v_\alpha (z) \quad \forall z\in L.
 $$  
In fact, if $t$ is a uniformiser for $\mathscr{O}_\alpha $ then
$Dt=at$, $a \in \mathscr{O}_\alpha $ from above, and thus if $z  =
ur^n, u\in \vartheta_\alpha , n\in {Z}$, 
\begin{align*}
  DZ & = t^n Du+nut^{n-1} Dt\\
  & =  t^n (Dn+nua)
\end{align*}
whence $v_\alpha  (Dz)\geq v_\alpha  (Z)$,( since $(Du+nua)\in
\mathscr{O}_\alpha )$. 

Now\pageoriginale consider the divisor $P_\alpha $ on $X$ defined of some
$\mathscr{O}_\alpha ;$ for large $q \in \mathbb{Z},
\mathcal{O}_\circ=q p_\alpha$ is a divisor on $X$ such that
$L(\mathcal{O}_\circ)$ contains an $x$ transcendental over $K$. Then
we may write $L=K(x,y)$ with $y$ \textit {integral} on $K[X]$
$\begin{pmatrix}{\text {charac}.K=0}\\{\text{Thus}\; L/K
    \;\text{separable}}\end{pmatrix}$ If we write
$\mathcal{O}=\max(-v_\alpha (x),-v_\alpha (y)).P$, then $x,y\in L
(\mathcal{O})$. 
 
From the fact that $\vartheta_\alpha  (Dz)\geq v_\alpha  (z)$ it
follows that $Dx, Dy  \in L(\mathcal{O})$  and by iteration $D^nx, D^n
y \in L (\mathcal{O})$. 

Therefore,
\begin{align*}
   S(x) &= \sum_{n\geq 0} T^n
  \frac{D^n x}{n!}\\ 
  & \in L(\mathcal{O}) [[T]]\\ 
  &  =K[[T]][L] \\    
  &  \subset K((T))(x,y).  
 \end{align*}  
Similarly for $S(y)$. Our assertion is proved.
   
Now we take, as before, projective imbeddings of $C$ and $C^S$  over
$K((T))$; we fix, as before, base and unit points $P_{\alpha_
  1},\ldots,  P_{\alpha _{n+2}} \in C$ corresponding to some of the
valuation rings $\mathcal{O}_\alpha $ defined above; if $u:C
\rightarrow C^S$ is the birational correspondence over $K((T))$
constructed above, then the image $u(P_\alpha)$ of $P_\alpha $ is
$P^s_\alpha $ (defined by $\mathcal{O}^S_\alpha ) U$ follows easily
from the equality  $S\sigma_\alpha  =\sigma_\alpha  S$. One proves as
be for that $C=C^S$. 

We\pageoriginale project $C$ and $C^S$ now into the place, from a centre of
projection (linear variety of dimension $(n-3))$ which is
rational/$k$. If the images $C_1$ and $C_1^S$ are defined by
polynomial equations $\varphi (X,Y) = 0$ (coefficients $a_{ij}$) and
$\varphi^S (X,Y) = 0$ (coefficients $S(a_{ij}$)) then the equality of
$C$ and $C^S$ (hence of $C_1$ and $C_1^S$) gives that $S(a_{ij}$) are
proportional to $a_{ij}$; but since we may assume that some $a_{ij} =
1$ we obtain $S(a_{ij}) = a_{ij} \forall i,j$. This means the ${a_{ij}
  \in  ~\text{ker}~ D = k}$.  

The projection onto the plane is a birational map (defined over $k$)
from $C$ onto $C_1$ ; as $C_1$ is defined over $k$ the proposition is
proved.

\hfill{Q.E.D.} 

\section{Definite Results}\label{chap3:sec3}

We still have to remove the extra  hypothesis on $K$ that we made in
II namely, tr. $\deg_k K = 1$. We are now going to do it and later
we shall see over what fields $C$ and $C'$ are isomorphic and then
analyse $C_{K}$. 

\setcounter{proposition}{0}
\begin{proposition}\label{chap3:sec3:prop1}%propo 1.
  Let $K$ be any field $L$ a regular finite type extension of $K$. Let
  $C$ be a curve defined over $K$ such that genu$s_KC$ = absolute
  genus of $C \geq 2$. Then either $C_{L} - C_K$ is finite or $C \sim
  D$ a curve defined over a finite field.  
\end{proposition}

\begin{proof}%proof
  Each $x_{\alpha}\in(C_L - C_K)$  satisfies $K(x _ \alpha ) \simeq
  K(C)$ so that 
\begin{gather*}
  \underset{\smile}{ K \subset K}(x_\alpha)\subset K(X)= L.\\
  \tr.\deg.1
\end{gather*}

By\pageoriginale Severi $\exists$ only finitely many $K(x_\alpha)$ such that $L$ is
separable over $K(x_\alpha);$ and by Sobwarz-Klein, $\exists$ only
finitely many $x_\alpha$  such that $K(x_\alpha$) has a given
value. \textit{Therefore} $C_L -C_K$ \textit{infinite rules out the
  possibility}: \textit{charac}. $K = 0$. Let $p\neq 0$ be
charac. $K$. For each $x_\alpha$ , let $q(\alpha)$ be the largest
power of $p$ such that $K(x_\alpha)\subset K(L^{q^{(\alpha)}})$. 
\end{proof}

Then
\begin{align*}
  K\left(x_{\alpha}^\frac{1}{q(\alpha)}\right) & \subset L\\ 
  &\not\subset K(L^{p})
 \end{align*}
 and $L/K\left(x_{\alpha}^\frac{1}{q(\alpha)}\right)$ is separable. By
 Severi, therefore, the number of\break
 $K\left(x_{\alpha}^\frac{1}{q(\alpha)}\right)$ 
 is finite; moreover, we have 
  $$ 
 {K\left(x_{\alpha}^\frac{1}{q(\alpha)}\right)
   \underset{sep.}{\subset} L \Longleftrightarrow K(x_{\alpha})
   \underset{sep.}{\subset} K(L^{q(\alpha)})} 
  $$
so that for a given $q(\alpha)$, the number of $x_{\beta}$ such that
$K(x_\beta){\underset{sep.}{\subset}K(L^{q(\alpha)})}$ is finite
(Severi). If follows that $\exists q(\alpha),q(\beta),q(\alpha) \neq
q(\beta)$ such that $K \left(x_{\alpha}^\frac{1}{q(\alpha)}\right)$  =
$K\left(x_{\alpha}^\frac{1}{q(\alpha)}\right)$. If we write say $q(\alpha) =
p^n.q(\beta), n\geq 1$, then it follows that  
$$
{K(x_{\alpha}) = K \left(x^{p^n}_{\beta}\right) =
  K\left(x^{q}_{\beta}\right)\text{(\text { say } )}.} 
$$
(We remark that since $L$ and $K$ are linearly disjoint, to prove the
above equalities, we could have assumed that $K$ is algebraically
closed). This means that the curve $C^q$ conjugate to $C$ under the
isomorphism $x  \mapsto x^q$ of $K$, is K-birationally equivalent to
$C$. 

Thus,\pageoriginale to prove proposition \ref{chap3:sec3:prop1}, we shall now prove the 

\begin{proposition}\label{chap3:sec3:prop2}%propo 2.
  Let $C$ be a curve defined over $K$, charac $K = p \neq  0$. Let $q
  = p{^n, n\geq 1}$. Then $C {\sim C^q \Rightarrow C \sim D}$ a curve
  defined over a finite field. 
\end{proposition}

\begin{proof}%proof
  For $C$ we choose an absolutely normal model, defined over\break 
  $\overline{{F}}_p (z_1,\ldots,z_r)$ say (this is a finite type
  extension of $\overline{{F}}_p $); then $C^q$ is absolutely normal,
  defined over $\overline{{F}}_p (z^q)$ and so on; we also know from
  our computations preceding the proposition that $C \sim C^q$ an so
  on. Thus, by lemma \ref{chap3:sec2:lem5} II, it follows that $C \sim $
  D$'$ defined over $\overline{{F}}_p$; obviously, we can consider $D$
  as defined over a finite field. 
\end{proof}

\hfill{Q.E.D.}

\setcounter{theorem}{0}
\begin{theorem}\label{chap3:sec3:thm1}%thm 1.
  Let $k$ be an algebraically closed field, $K/k$ any function field,
  and $C$ be an algebraic curve defined over $K$ such that $K$-genus
  of $C$ = absolute genus $C \geq 2$. If $C_K$ is infinite then 
  \begin{enumerate}[1)]
  \item $C \sim  C'$ defined over $k$
  \item An isomorphism $C \rightarrow C'$ is defined over $K$ if $C$
    is not birationally equivalent to any curve defined over a finite
    field 
  \end{enumerate}
\end{theorem}

\begin{proof}%proof
  1) We shall prove 1) by induction on the transcendence degree
  $d$ of $K/k$. The case $d = 1$ has already been tackled (part
  II). If, at any stage of the inductive proof, we obtain $C \sim
  D, D$ defined over a finite field, we will be through; therefore we
  will rule out this possibility in the entire proof (so that we may
  Proposition \ref{chap3:sec3:prop1} effectively). 


  Let\pageoriginale $\tr.\deg_k K = d $; choose a $K_1, k \subset K_1 \subset K$ such
  that $\tr. \deg_{K_1} K = 1$. By part II, $C \sim C_1,C_1 $  defined
  over $\overline{{K}}_{1}$; we may assume that $C_1$ is defined over a
  finite extension $L_1$ of $K_1$ ; the birational correspondence $C
  \sim C_1$  is then defined over a finite extension $K'$ of $L_1(K)$;
  replacing $L_1$ by a finite extension $L_2$ of $L_1$, we may assume
  that $K'$ is separable over $L_1$; again replacing $L_1$ by its
  algebraic closure in $K'$ we may assume that $L_1$ is algebraically
  closed in $K'$. Thus $K'$ is a regular extension of $L_1$ and by
  hypothesis $(C_1)_{K'}$ is infinite; by the remark is made at the
  beginning and by Proposition \ref{chap3:sec3:prop1} we obtain: ${(C_1)_{K'} -
    (C_1)_{K'}}$ is finite so that $(C_1)_{K'}$ is infinite. The
  inductive assumption now completes the proof of 1). 
  
 2)~ Let $u : C \rightarrow C'$ be an isomorphism, defined over a
 finite extension $K'$ of $K$. We have ${C'}_{K'}$ infinite; then by
 hypothesis and by proposition \ref{chap3:sec3:prop1}, $({C'}_{K'} - {C'}_{K}) \subset
 (C'_{K'} - C'_{K})$ is finite. 
\end{proof}

 This means that there are an infinity of $x{_i\in C_K}$ such that
 $u(x_i)\in C'_K$. We now take tricanonical models $D$ and $D'$ in
 ${P}_r$ of $C$ and $C'$  respectively (these imbeddings are defined over $K$,
 by hypothesis). We may choose on $D$ base points and unit point ${x_1
   \ldots,x_{r+2}}$ among the ${(x_i)}$ all rational over $K$: and on
 $D'$ choose the base points and unit point as the K-rational points
 ${u(x_i)}$. The isomorphism $u : {C \rightarrow C'}$ then defined a
 projective transformation $u :  {D \rightarrow D'}$ which is
 necessarily defined over $K$ as the base points an unit point are
 $K$-rational. \hfill {Q.E.D.} 

\begin{remark*}%rema
  If\pageoriginale $v:C' \rightarrow C$ is an isomorphism defined  over $K$, then
  $v$ defines a bijection ${C'_{k}\rightarrow C_K}$; but ${C'_{K} -
    C'_k}$ is finite by Proposition \ref{chap3:sec3:prop1}, which means that almost all
  points in $C_K$ are in $\nu(C'_k)$. 
\end{remark*}

We had left out some ``exceptional cases'' in Theorem
\ref{chap3:sec3:thm1}, \ref{chap3:sec3:thm2}. Our aim 
now is to study the situation in this ``exceptional case''. (In the
following we shall use the term ``isomorphism'' for a ``birational
map''). 

\begin{theorem}\label{chap3:sec3:thm2}%them 2.
  Let $k$ be algebraically closed, $K$ any function field over $k; C$
  is a curve (as usual projective nonsingular) defined over $K$ such
  that $K$-genus of $C$ = absolute genus of $C \geq 2$ and $C_K$
  infinite. Assume that $C \sim C',C'$ defined over a finite field
  ${F}_p$ such that all the members of Aut $C'$ (by Schwarz-Klein
  these are all defined over ${F}_p$) are defined over ${F}_q$. Let
  $f$ be the automorphism $x\mapsto x^q$ of ${F}_q(C')$  giving an
  automorphism $f$ : $C'\mapsto C'$. Then  
  \begin{enumerate}[\rm (1)]
  \item $\exists$ a finite galois extension $K'$ of $K$, a $K'$-
    isomorphism $u : C \rightarrow C'$ and a monomorphism $\sigma
    \mapsto h_{\sigma}$ of $G = G(K'/K)$ into Aut $C'$ such that  
    $$ 
     {h_{\sigma} = u^{\sigma}\circ n^{-1}}
    $$
   \item $\exists$ a finite family $(z{_i})$ of transcendental points
     of $C'_{K'}$  such that $z^{\sigma}_{i} =
     h_{\sigma}(z_i)\forall\sigma\in G$. Also every $x\in C_K$ is
     either some $u^{-1}(f^{n}(z_i))$ or some $u^{-1}(z)$ with $z\in
     C'_K$ and  
     $$
     {z^{\sigma} = h_{\sigma}(z)\quad  \forall\sigma\in G.}
     $$
  \end{enumerate}
\end{theorem}

\begin{proof}%proof
  (1)\pageoriginale Let $w : C \rightarrow C'$ be the birational correspondence
  given; we may assume that $w$ is defined over a finite extension
  $K''$ of $K$. Let $(x_i)$ be the infinite family of points of $C_K$
  ; for each $i$, one has $k(w(x_i))\subset K''$ and by Severi almost
  all of the $k(w(x_i))$ are contained in $K''^p$ by iteration of this
  procedure, we may assume that $w$ is defined over a finite separable
  extension, whence also over a finite galois extension $K''/ K$. If
  $\sigma\in G(K''/K)$ then $g_{\sigma} = w^{\sigma}.w^{-1} \in $ Aut
  $C'$; we claim that $\sigma \mapsto g_{\sigma}$ is a homomorphism: in
  fact, that $g_{\sigma}$ is a cocycle follows from  
  \begin{align*}
    {g}_{\sigma\tauup} = { w^{\sigma\tauup}.w^{-1} } & =
    {(w^{\sigma})^{\tauup} w^{- \tauup}w^{\tauup}w^{-1}}\\ 
    & = {(g_{\sigma})^{\tauup}g_{\tauup}}\\
  \end{align*}
  and, as $g_{\sigma}$ is defined over $K$, we are through.
\end{proof}

Now consider the kernel of this homomorphism $\sigma\mapsto g_{\sigma}
$; it is the galois group of $K''$ over a galois extension $K'$ of
$K$. For $\sigma\in G(K''/K')$ then one has $w^{\sigma}= w$ and thus
$w$ is defined over $K'$, denote it by $u$. For $\sigma\in G(K'/K)$ if
we set $h_{\sigma} =u^{\sigma}.u^{-1}$ then $\sigma \mapsto
h_{\sigma}$ is a monomorphism. Hence (1). 

(2)~ Suppose $z\in C'$ is of the form $u(x), x \in C$ . Then 
$$
x \in C_{K}\Longleftrightarrow \{z\in C'_{K'} \text{ and }  z^{\sigma}
= h_{\sigma}(z),\forall \sigma \} 
$$

In fact $\Rightarrow$ is trivial ; on the other hand observing that
for every $\sigma\in G(K'/K)$ 
\begin{align*}
  z^{\sigma} = h_{\sigma}(z) & \Leftrightarrow {z^{\sigma} =
    u^{\sigma}u^{-1}(z)}\\ 
  & \Longleftrightarrow u^{-1}(z) = u^{-\sigma} (z^{\sigma})\\
  & \Longleftrightarrow u^{-1}(z) = (u^{-1}(z))^{\sigma}\\
\end{align*}
and\pageoriginale thus ${z\in C'_{K'},h_{\sigma}(z) = z^{\sigma},\forall \sigma,}$
imply that $x = u^{-1}{(z)\in C_{K}}$. 

Now by an easy iteration of Severi we prove that there are only
finitely many (transcendental) points ${y_1,\ldots,y_r \in C'_{K'}}$
such that $k(y_i) \not\subset { K'^q}$. If ${z \in (C'_{K'} -
  C'_{k'}})$ then it follows that for some $n$, 
$$
\displaylines{\hfill 
  k\underset{\neq}\subset k(z^{q^{\frac{1}{n}}})  \subset   K' \hfill \cr
  \hfill  \not\subset  K'^{q}\hfill \cr
  \text{i.~e.} \hfill z {^{q^{\frac{1}{n}}}} = y_i \qquad \text{ for
    some }~ i \hfill \cr
  \text{i.~e.} \hfill z = {f}^{n}({y_i}) \qquad \text{ for some }
  ~i.\hfill }
$$

Now if $z \in C'_{K'} - C'_{k}$ and if $z^{\sigma} = h_{\sigma}(z)
\forall \sigma$, then the $y_j$ for which $z = f^{n}(y_j)$ has the
property $y_{j}^{\sigma} = h_{\sigma}(y_j)\forall\sigma$: this follows
from the equality ${h_{\sigma}f^{n} = f^{n}h_{\sigma}}$ (recall that
$h_{\sigma}$ is defined over $F_{q}$). 

\noindent
 The proof of the theorem is complete.\hfill{Q.E.D.}

We shall end up by giving an example which will show that the part
(2) of Theorem \ref{chap3:sec3:thm2} cannot be strengthened. 

\begin{example*}%exam
  Let $k$ be an algebraically closed field of characteristic $p \neq 2$.
  Let\pageoriginale $C'$ be the plane curve defined over $k$ whose affine equation
  is ~$x^{4}+y^{4} + 1 = 0 ; C'$ is nonsingular: indeed, the
  derivatives $4x^3 , 4y^3, 4z^3$ of $x^4 + y^4 + z^4$ cannot all
  vanish at any point on $x^4 + y^4 + z^4 = 0 (\text{in}~ {P}_2)$, thus genus
  $C' = 3$, and the imbedding $C'\hookrightarrow{P}_2$ is canonical. 
\end{example*}

Let $K' = k(r,s), r$ transcendental over $k$ and $s$ such that at $r^4
+ s^4 + 1 = 0$. Let $\sigma$ be the automorphism of $K'/k$ such that
$\sigma{(r)= -r, \sigma(s) = -s;}$ then the fixed field $K$ of
$\sigma$ is $k(r^2, s^2, rs)$, and  $K'/K$ is a galois extension of
degree $2$ whose galois group is $G$ = $\{1,\sigma\} $. 

Let $C$ be the curve $X^4 + Y^4 + r^4 = 0$ defined over $K$. 

\begin{enumerate}[a)]
\item $C_K$ \textit{is infinite}.

  In fact, the infinity of points (${r^{1+p^{n}},rs^{p^n}})$ are $K$ -
  rational (if $p>2, {p^n +1}$ is even; also ${rs^{p^n}} =
  rs.{s^{p^n-1}}$ and ${p^n -1}$ is even) 
\item $C \underset{K'}{\sim} C'$

  In fact, the homothety $u : C' \rightarrow C$ given by $X = rx, Y =
  ry$ is a projective transformation in ${P}_2$, defined over $K'$. 

  The automorphism $h_{\sigma}$ of $C'$ is then the automorphism
  $(x,y) \rightarrow$ \break  $(-x,-y)$ 
\item $C$ \textit {is not} $K$ - isomorphic to any curve defined over
  $k$(in particular ${C \underset{K}{\nsim}C'})$. Indeed, if ${C
  \underset{K}{\sim} D}$, $D$ defined over $k$ then $D_{\tilde{K'}} C'$
  from $b$); as $D$ and $C'$ are both defined over $k$, $k$
  algebraically closed, it follows that ${D} \underset{k}{\sim} C'$;
  thus ${C}\underset{K}{\sim}C'$ , say through $ w : C $ $\rightarrow
  {C'}$ defined over $K$; then $w u \in $ Aut $C'$ is defined over $k$
  so that $u = {w^{-1}}({w} u)$ is defined over $K$. This is clearly
  false since $r \not\in K$. 
\item \textit{However},\pageoriginale $C$ \textit{is $K$ - isomorphic to}
  ${C^{p^n}}$ \textit{for all} $n \geq 1$. 
  
  Write
  \begin{align*}
    {X}' & = {r^{p^n -1.}X}\\
    {Y'} & = {r^{p^n -1.}Y.}
  \end{align*}

  Then this is a projective transformation defined over $K
  (p^n-1)$ is even) and transforms $C \equiv X^4 + Y^4 +
  r^4 = 0$ into  
  $$
  (X')^4 + (Y')^4 + (r^4)^{p^n} = 0
  $$
  Which is the curve $(C')^{p^n}$.
\item \textit{Rational points} $P'$ \textit{of} $C'_{K'}$ \textit{ and
} $P$ \textit{ of } $C_{K}$. 
\end{enumerate}

By theorem \ref{chap3:sec3:thm2}, part (2), the transcendental points of $C'_K$ are obtained
by applying the ``iterated Frobenius maps'' to points ($x_1,x_2$) such
that $k(x_1,x_2)$ is of transcendence degree $1$ over $k$ and
$k(x_1,x_2) \underset{\text{(separably)}}{\subset k(r,s)} = K'$. As these
fields have the same genus 3, it follows by Hurwitz$'$ theorem that
$k(x_1,x_2) = k(r,s) = K'$. Thus to find the $K'$ - rational points on
$C'$ one has to find and $k$-automorphisms of $C'$. Some of them, for
instance, are given by (see appendix 3 for a complete determination
of Aut $(C')$) 
\begin{enumerate}[i)]
\item $(r,s)  \mapsto (r, -s)$
\item $(r,s)  \mapsto (-r, -s)$
\item $(r,s)  \mapsto (\alpha r, \beta s), \alpha^4 = \beta^4 = 1$.
\item $(r,s)  \mapsto \left(\dfrac{1}{s},\dfrac{1}{r}\right)$.
\end{enumerate}

As we have seen in the proof of theorem \ref{chap3:sec3:thm2}, the rational points $P$
of $C_{K} - C_{k}$ are given by rational points $P'$ of $C'_{K'}-
C'_{K}$ which\pageoriginale have the property $P'^{\sigma} =
h_{\sigma}(P'),\forall\sigma\in G$. For instance, from among the above
four automorphisms it is clear that (i), (ii), (iii) satisfy this
requirement but the fourth does not. 

Finally, to get the $k$-rational points $P$ on $C$, we take
$k$-rational points $P'$ on $C'$ which have the property
$h_{\sigma}(P') = P'$ These are the points at infinity of $C'$, and
give the points at infinity on $C$.   
\newpage 

\thispagestyle{empty}
\begin{center}
  \textbf{\LARGE APPENDICES TO CHAPTER III}
\end{center}



\noindent \textbf{Appendix 1.} For\pageoriginale a purely aesthetic reason, we shall prove
here a stron\-ger form of Proposition \ref{chap3:sec3:prop2}, III of Chapter III. 
  
\begin{prop*}[ {\boldmath $\tilde{2}$}] 
  Let $C$ be a curve of absolute genus $\geq 2$
  in characteristic $ p \neq 0$.  Suppose $C$ is birationally
  equivalent to $C^q$ with $q = p^n,  n > 0$. Then $ C \sim D$ a curve
  defined over $\mathbb{F} _q$ (the finite field with $q$ elements). 
\end{prop*}

\begin{proof}%\proof  
  In  view of Proposition \ref{chap3:sec3:prop2}, III, we may now assume that $C$ is
  defined over an $\mathbb{F}_{q^n} $; by choosing $n$ large, we may
  also assume that the elements of Aut $C$, which are finitely many,
  are all defined over $\mathbb{F}_{q^n}$. We shall set $\mathbb{F} _q
  = F,\ldots ,\mathbb{F}_{q^r} = F_r{,\ldots}$ and $G_r$ =  the group
  of all F-automorphisms of  $F_r(C) (r$ large). We now define a
  homomorphism  
  \begin{align*}
    G_r & \longrightarrow G(F_r/F)\\
    \sigma & \longmapsto \sigma \vert F_r .
  \end{align*}
\end{proof}

If $r$ is large, (if $n \mid r )$ the kernel of this homomorphism is the
group of $F_r$ -automorphisms of $F_r(C)$  i. e. is Aut $C$. We
claim that this homomorphism is onto. In fact, if $\varphi$ denotes the
Frobenius automorphism $ x \mapsto x^q$ of $F_r, \varphi$ extends to a
$\varphi : F_r (C)\longrightarrow F_r (C^q)$ ; on the other hand, the
hypothesis $C \sim C^q$ (we may assume that this birational
correspondence is defined over $\mathbb{F}_{q^n} = F_n )$ gives an
isomorphism $\omega: F_r (C^q)\to F_r(C)$ so that $\omega \varphi \in
G ;$ if $n \mid r $, we have 
$$
\omega \varphi \vert F_r = \varphi  \vert F_r = \varphi.
$$

Thus,\pageoriginale we get an exact sequence (for $n \mid r)$
\begin{enumerate}[(1)]
\item $1 \rightarrow $ Aut $C \rightarrow G_r \rightarrow
  G(^Fr/F)\rightarrow 1$. 
\item We now assert that 
  \begin{center}
    $C \underset{F_\gamma}{\sim} D$ ~defined over $F$\\
    ($\Longleftrightarrow$ ~the above sequence ``splits'')\\
    $\Longleftrightarrow \exists~  r$-cyclic subgroup  $G'_r$ 
    of $G_r$ 
  \end{center}
  such that $G_r'\cap$ Aut $c = (1)$.
  $$
  \displaylines{
  \text{ i.e. } \hfill \Longleftrightarrow \exists \text{ a } \sigma
  \in G_r \hfill\cr 
  \text{with}\hfill \sigma^\gamma = 1 \qquad \text{ and } \qquad
  \sigma \mid F_r  = \varphi.\hfill } 
  $$
  $\Rightarrow$ : is quasi obvious $D$, whence for $C$.
\end{enumerate}

$\Leftarrow$: Suppose $\exists$ such a $ \sigma \in G_r$

Let $L$ be the fixed field of $\sigma$. By galois theory $[F_r(C) : L]
= r$ so that $F_r$ and $L$ are $F$-linearly disjoint. If $L = F(D)$,
$D$ a curve defined over $F$ then $F_r(C) = F_r (D)$\hfill{Q.E.D.} 
$$
\xymatrix@C=2cm{L \ar@{-}[d] \ar@{-}[r]^{r} & F_r(C)\ar@{-}[d]\\
F \ar@{-}[r]_r & F_r
}
$$

Our aim therefore will obviously be to make $(1)$ ``split'' for large
multiples $r$ of $n$. For large $r, r'$ with $n \mid r$  and  $n \mid
r'$ and $r' \mid r$, we have an obvious commutative diagram 
$$
\xymatrix{1 \ar[r] & Aut~ C \ar@{=}[d]\ar[r] & G_r\ar[r] &
  G(F_{r/F}) \ar[r] & 1\\
  1 \ar[r] & Aut ~ C \ar[r] & G_{r'} \ar[u] \ar[r] & G(F_{r'/F}) \ar[u]
  \ar[r] & 1   
}
$$
so\pageoriginale that we have an inverse system of exact sequences; as the so-called
Mittag-Leffler condition is trivially verified, in the limit we get
an exact sequence 
$$
1 \rightarrow \text{ Aut } C \rightarrow G \rightarrow G
(\bar{F}/_F)\rightarrow 1. 
$$

The group $ G(\bar{F}/F)$ is the limit of the inverse system $
(\mathbb{Z}/_{r\mathbb{Z}})_r$ and is thus the ``\textit{ universal
  pro-cyclic group }$\hat{\mathbb{Z}}$. (It is the completion of
$\mathbb{Z}$ for the topology where a fundamental system of
neighbourhoods of $U$ is $(r\mathbb{Z})_{r\neq o})$. It is
topologically one-generated (with the limit topology it is compact,
Hausdorff and totally disconnected) viz , by the Frobenius
automorphism $\varphi$ of $\bar{F}/F$. We take a $\sigma \in G$ which
maps onto $\varphi \in \hat{\mathbb{Z}}$. If $G'$ is the closed
topological subgroup of $G$ generated by $\sigma \in G$, then  $G'$
maps onto $\hat{\mathbb{Z}}$. But by a well-known property of
$\hat{\mathbb{Z}}$ (namely that for compact, totally disconnected
Hausdorff group $H$ and any  $h \in H, \exists$ a continuous homomorphism
$f_h : \hat{\mathbb{Z}} \longrightarrow H$, such that $\varphi
\longmapsto h, $ (see Corps Locaux, Sence Chapter XIII)) it
follows that $ G'\rightarrow \hat{\mathbb{Z}}$ is injective. 

Therefore, Aut $C\cap G' = (1)$

Let $\theta _r : G \rightarrow G_r$ be the canonical homomorphisms; set
$ G'_r = \theta -r( G') $ = subgroup generated by $\theta _r( \sigma
)$.  Obviously, one has  
$$
G'_r \cap \text{ Aut } C \supset G'_{r'}\cap  \text{ Aut } C \quad
\forall r'| r; 
$$
since all these groups are finite, the decreasing chain $(G'_r \cap ~\text{Aut}~
C)_r$ is stationary for large for $r$. We will through if we prove
that, for  large $\varphi G'_r \cap \text{Aut} C = (1)$. For this we
have merely to show that $\bigcap\limits_{r} (G'_r  \text{ Aut } C) =
(1)$;\pageoriginale in fact, if $ \alpha  \in\bigcap\limits_{r} (G'_r \cap$ Aut $ C)$
then considering $\alpha  $ as element of $G, \forall r, \theta _r( \alpha ) \in
G'_r = \theta _r (G')$ so that $\varphi \in \text{ Aut } C \cap G' =
(1)$.  \hfill{Q.E.D.}

\medskip
\noindent 
\textbf{Appendix 2. } Our aim in this appendix is to remove from
hypotheses  on our curve of investigation $C$  the condition  (in
charac  $p \neq 0)$ : $ genus_K  C$ = absolute genus $ C \geq 2 $ we
shall prove now the  
 
\begin{theorem*}%them 
  Let $k$ be an algebraically closed field of charac. $ p \neq 0, K$ a
  function field over $k$ , and  $C$ a curve defined over $K$, with
  absolute genus $ C \geq 2 $. If $C_K$ is infinite, then  $C$ admits
  an absolutely normal model defined on $K$ (so that $genus_K C$ =
  absolute genus of $C$).  \textit{( More precisely, the normalisation
  of $C$ is absolutely normal)}. 
\end{theorem*}

\begin{proof} %proof
  We may assume $C$ normal. The normalisation $C'$ of $C$ in
  $K^{p-\infty}$ is absolutely normal so that we  may assume that
  $\exists$ a finite, purely inseparable, extension $K'/K $ over which
  the normal model $C'$ of $C$ is \textit{absolutely normal}. By
  hypothesis, genus  $C'\geq 2$  and we may apply our results in
  section III. Let $u : C \rightarrow C'$ be the (natural)
  birational correspondence (defined over $K'$). By
  Theorem \ref{chap3:sec3:thm2} III ,
  of Chapter III, $\exists$ a curve $C''$ defined over $k$ and a
  birational correspondence $C' \overset{v}\rightarrow C'' $ defined
  over a finite galois extension ${k''}/{k'}$ such that for almost all
  points $ x \in C_K , v.u (x)$ is  in $K''^{p^n}$ ($n$ that large
  whence in the separable closure $L$ of  $K$ in $K''$. One may again choose\pageoriginale
  imbeddings and argue as before with base  points and unit point to
  prove that $v \cdot u$ is defined over  $L$ : as $L$ is separable over
  $K$ and as genus$_K C=$ genus$_L C =$ genus$_k C'' =$ genus$_{K'}
  c'=$ abs. genus of 
  $C$  it follows that $C$, which is normal/$L$, is already absolutely
  normal.  
\end{proof}
\hfill {Q.E.D.}

\chapter*{APPENDIX 3}

\addcontentsline{toc}{chapter}{Appendix 3}

\begin{center}
  \textbf{\Large Automorphisms of the curve  $x^4 + y^4 + z^4=0$ }
\end{center}

We\pageoriginale have seen that, in characteristic $\neq 2$, the plane curve $C:x^4
+ y^4 +z^4 = 0$ has genus $3$ and that imbedding $ C \hookrightarrow
{P}_2$ is the canonical one. Thus automorphisms  of $C$ are induced by
projective transformations. Among those we immediately see: 
\begin{enumerate}[a)]
\item The permutations of the variables $x,y,z$; these form a group
  $G_1$ of order 6; 
\item The multiplications of $x,y,z$ by arbitrary fourth roots of
  unity; these form a group $ G_2$ of order $16$. 
\end{enumerate}

Clearly $G_1$ and $G_2$ are permutable subgroups of $PGL(2)$ such that
$ G_1 \cap G_2 = \{ 1 \}$. Thus they generate a subgroup $ G \subset
PGL(2) $ of order $96$. We claim that: 

\medskip
\noindent
\textbf{The group $G$ is the group of automorphisms of $C$.}

To determine  Aut $(C)$ we may look for projective peculiarities of
$C$. Let us call a point $P$ of $C$ a \textit {superflex} if the
tangent to $C$ at $P$ intersects $C$ with multiplicity $4$ at $P$ (and
therefore has no other common point with $C$ ). Clearly the points of
$C$ on the coordinate axis $(e.g. (1,\alpha  ,0)$  with $\alpha^4 =
-1)$ are superflexes (the tangent at $(1, \alpha , 0)$ being $y-\alpha
x=0)$. We are going to find all superflexes of $C$. Disregarding the
points at infinity $(z = 0)$, and the points at which the tangent
passes through $(0,1,0)$ (i.e. the points on $y = 0)$, we may take
affine coordinates, consider a point $(a,b) \in C(a^4 + b^4 + 1 = 0)$,
and express that a line  
$$
x=a + \lambda y= b+t\lambda(b \neq 0)
$$
has\pageoriginale a quadruple intersection with $C$ at $(a,b)$; in other words
$\lambda = 0 $ must be a quadruple root of  
$$
(a+\lambda)^4 +(b+t\lambda)^4 +1 = 0. 
$$

This means
\setcounter{equation}{0}
\begin{equation*}
  a^4 + b^4 + 1 = 0, 4(a^3+tb^3) =0 , 0(a^2+b^2t^2) = 0 ,4(a+bt^3) = 0.
\end{equation*}

We deduce $t = -\dfrac{a^3}{b^3} $ whence $a-b \dfrac{a^9}{b^9}=
\dfrac{a}{b^8}(x-a^8) = 0$. 
We have solutions $a = 0 , b^4 = -1, t = 0$ on the axis $x = 0$, thus
we may assume $a \neq 0$. In characteristic $\neq 3$, the relations
$a^2 + b^2 t^2 = 0 , t = -a^3/b^3$ give $a^4 + b^4 = 0$ impossible. In
characteristic 3 the third relation  in (1) 
disappears. From $b^8 -a^8 = 0$ and $a^4 + b^4 =-1$ we deduce $a^4-
b^4 = 0$, whence $2a^4 = -1, a^4 = 1$, and $a,b$ are $4^{th}$ roots of
unity. Thus the superflexes of $C$ are:  
\begin{enumerate}[a)]
\item The $12$ points of $C$ on the coordinate axis (for any
  characteristic $\neq 2$) 
\item In  addition the $16$ points $(a,b,1)$ such that $a^4 = b^4 =
  1$, in characteristic $3$. 
\end{enumerate}

Notice that, in characteristic 3, the 23 tangents to $C$ at the
superflexes are the famous 28  
bitangents to $C$ (replaced with the 27 Lines on a cubic surface).

Through each base point pase $C$ lines in the general case (\resp 6
lines in characteristic 3) such that each line contains 4
superflexes. The base points are the only points with this property:
this is clear in the general case; a simple computation has to be made
in characteristic 3 (here the 28 superflexes are rational over
$\mathbb{F}_q$). 

Hence\pageoriginale the base points are characterized by an invariant projective
property of $C$. Therefore any automorphism of $C$ is induced by a
projective transformation $u$ which permutes the base points. Then
$uv^{-1}$, with some $v \in G_1$, leaves fixed the base points, i.e.,  
$uv^{-1}(x) = \lambda x, uv^{-1}(y)= \mu y, uv^{-1}(z) = \nu z$. Since
$C$ is globally invariant by $uv^{-1}$, this implies that $\lambda,
\mu, \gamma$, are proportional to $4^{th}$ roots of unity, i.e., $uv^{-1}\in
G_2$. Therefore $u \in G$. \hfill Q.E.D.

\begin{thebibliography}{99}
\bibitem{1}  {C. Chevalley}  Algebraic functions of one variable
  (A.M.S. Publication, 1951)
\bibitem{2} {H. Grauert} Mordell's Vermutung uber Punkte auf
  algebraischen kurven and Functionen-
  korper(I.H.E.S. Publication, 1965)
\bibitem{3} {J. Manin}  Rational points of algebraic curves
  over function fields (Izvestija Akad.
  Nauk. SSSR,Ser.Mat., t.27, 1963)
\bibitem{4}{P. Samuel}  Methodes dialgebre abstraite en geo
  -metrie algebrique (Ergebnisses der
  Mathematik, 1955)
\bibitem{5} {Seminaire Bourbaki talk} ($n^o 287, 1965$)
\bibitem{6}  Complements a un article de Hans
  Grauert sur la conjecture de mordell
  (I.H.E.S. Publication, $n^o 29$)
\bibitem{7} {J.P. Serre}  Groupes algebriques et corps de classes
  (Hermann, Paris, 1959)
\bibitem{8} {F. Severi}  Trattato di geometria algebrica (Vol I,
  Parte I, Zanichelli, Bologna, 1926)
\bibitem{9} {A. Weil}  Foundations of algebraic gemetry
  (Coll.No. 29, New York, 1946)
\end{thebibliography}
