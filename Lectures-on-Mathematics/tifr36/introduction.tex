\chapter*{Introduction}

\addcontentsline{toc}{chapter}{Introduction}

The\pageoriginale aim of the present course is to give a proof, due to Hans Grauert,
of an analogue of Mordell's conjecture. Mordell's conjecture says that
if $C$ is a curve, of genus $\geq 2$, defined over a number field $K$,
then set $C_K$ of $K$-rational points $C$ is \textit{ finite }. This
conjecture applies in particular to Fermat's curve $x^n + y^n = 1 (n
\geq 4)$. 

As a matter of notation if $V$ is an algebraic variety defined over a
field $K$, $V_K$ will denote the set of $K$-rational points of $V$. If
$V$ is an affine (\resp projective) variety, we mean, by a
$K-$rational point of $V$, a point whose affine coordinates
(resp. ratios of homogeneous coordinates) all lie in $K$. 

An analogue of Mordell's conjecture has recently been stated and
proved by Ju Manin and Hans Grauert. In this, number fields are
replaced by function fields. 

\begin{theorem*} %thm
  Let $k$ be an algebraically closed field of characteristic $0$, and
  $K$ a function field over $k$. If $C$ is a curve of genus $\geq 2$
  defined over $K$ such that $C_K$ is infinite, then 
  \begin{enumerate}[a)]
  \item $C$ is birationally equivalent, over $K$, to a curve $C'$
    defined over $k$ 
  \item $C'_K - C'_k$ is finite (in other words, almost all points of
    $C_K$ come from $k$). 
  \end{enumerate}
\end{theorem*}

The\pageoriginale algebraist Manin (\cite{3}) has given an analytical proof, in which $k
=\mathbb{C}$; the result, of course, remains valid for arbitrary $k$ by
the principle of Lefschetz. The analyst Grauert (\cite{2}) gives a purely
algebro-geometric proof, a large part of which is valid in
characteristic $p \neq 0$. The finishing touch in characteristic $p$
has been provided by the lecturer (\cite{5}, \cite{6}): 

Let $k$ be an algebraically closed field of characteristic $p\neq 0$,
$K$ a function field over $k$ and $C$ a curve defined over $K$ with
absolute genus $\geq 2$, such that $C_K$ is infinite. Then, 

(a) $C$ is birationally equivalent( over some field ) to a curve $C'$
defined over $k$. 
 
But unlike in characteristic $0$, (see the example at the end of the
course) this birational equivalence is in general \textit{not} defined
over $K$: one cannot expect $C'_K - C'_k$ to be finite; indeed, if $C$
is birationally equivalent to a curve $C'$ defined over a finite field
$\mathbb{F}_q$, then for any $x \in  C'_K-C'_k$, all the points
$x^{q^n}$ obtained from $x$ by iterated applications of the Frobenius
automorphism $x \to x^q$ (the point whose affine coordinates are the
$q^{th}$ powers of the affine coordinates of $x$) are again in $C'_K -
C'_k$. Interestingly enough, it turns out that these are the only
exceptional cases; more precisely, if $C$ is \textit{ not }
birationally equivalent to any curve defined over a finite field, then
the birational equivalence $C \thicksim C'$ ($C'$ defined over $k$) is
defined over $K$. One can also prove that, in any case, the birational
equivalence $C \thicksim C'$ is defined over a finite galois extension
$K'$ of $K$ and\pageoriginale that all the points of $C'_K - C'_k$ may be obtained
from a finite number among them, by applying the Frobenius process. 
 
We assumed that the \textit{absolute} genus of $C$ (i.e.. the genus of $C$
over the algebraic closure $\bar{K}$ of $K$) is $\geq 2$; this is
stronger than the assumption genus $_K C \geq 2$, since the genus of a
curve may very well drop by an inseparable base-extension; a classical
example is the curve  
$$ 
Y^2 = X^p - a 
$$
(with $p \geq 3$ and $a \in K-K^p$) whose relative genus is
$\dfrac{p-1}{2}$ and absolute genus is 0. At present, nothing is
known for curves of relative genus $\geq 2$ and absolute genus $0$ or
1. 

The key to the proof is \textit{ not } the inequality $g \geq 2$ but
the equivalent inequality $(2g-2) > 0$ which means that the canonical
divisor on $C$ is \textit{ample}. 

The proof of Grauert's theorem may be divided into two parts. 
\begin{enumerate}[1)]
\item Proving that $C$ is birationally equivalent to a curve $C'$
  defined over $k$. This is the hardest and most original part. 
\item Studying $C'_K - C'_k$. Here we are in the midst of nice old
  theorems on algebraic curves. For $K$ may be viewed as the function
  field $k(D)$ of an algebraic variety $D$ over $k$ then the points of
  $C'_K - C'_k$ correspond to nonconstant rational maps of $D$ in $C'$
  over $k$. The statement b) of
  the theorem says that these maps are finite in number in charac. 0; in
  charac. $p$. the separable ones are finite in number:\pageoriginale We will assume
  in our proof that $D$ 
  first a curve and then pass to the general case by a simple
  induction. We have a theorem of F. Severi (\cite{8} ,0. 291). the
  separable involutions of genus $\geq 2$ on a curve $D$ (i.e., the
  isomorphism classes of non-constant separable rational maps of $D$
  into curves of genus $\geq 2$) are finite in number. This is
  somewhat stronger than the finiteness of $C'_K - C'_k$ since the
  image curve $C'$ is \textit{not} fixed in Severi's theorem; but this
  stronger statement will be needed in clarifying the charac. $p$
  case. As a corollary we get a well known theorem of $H.A$. Schwarz
  and F. Klein: for a curve $D$ of genus $\geq 2$, Aut $D$ is
  finite. 
\end{enumerate}

The lecturer has felt that it will be more germane to the spirit of
Grauert's theorem (in which fields and rationality questions play a\break
prominent part)- or may be easier for himself - to use the older
algebro-geometric language of Weil and Zariski which, by now, is
barely distinguishable from the language of the ancient Italian
School. Many high powered classical tools of geometry will be used,
(eg. the intersection theory, Chow coordinates, Zariski's Main Theorem
etc.) An introductory chapter will give the necessary definitions and
state (mostly without proofs) the theorems that will be used. A second
chapter will give the theory of algebraic curves, with an emphasis on
correspondences. The last chapter will give Grauert's proof proper. 

We\pageoriginale remark, that, for the sake of ease, we will mostly be doing our
geometry over ``big'' (universal) fields (i.e. fields that have infinite
transcendence degree over the prime field). 
