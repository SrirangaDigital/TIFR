\thispagestyle{empty}
\begin{center}
{\Large\bf  Lectures on}\\[5pt]
{\Large\bf Numerical Methods In Bifurcation Problems}\\
\vskip 1cm

{\bf By}
\medskip

{\large\bf H.B. Keller}\\[5pt]
{Lectures delivered at the}\\
{\bf Indian Institute Of Science, Bangalore}\\[5pt]
under the\\[5pt]
{\bf T.I.F.R.-I.I.Sc. Programme In Applications Of}\\[5pt]
{\bf  Mathematics}\\
\vfill

{\bf  Notes by}\\[5pt]
{\large\bf  A.K.Nandakumaran and Mythily Ramaswamy}\\
\vfill


Published for the\\
{\bf  Tata Institute Of Fundamental Research}\\
Springer-Verlag\\
Berlin Heidelberg New York Tokyo
\end{center}

\eject

\thispagestyle{empty}
\begin{center}
{\bf  Author}\\
{\large\bf H.B. Keller}\\
  Applied Mathematics 217-50\\
  California Institute of Technology\\
  Pasadena, California 91125\\
  U.S.A.
\vfill

\textbf{\copyright Tata Institute Of Fundamental Research, 1986}
 \vfill

  ISBN 3-540-20228-5 Springer-verlag, Berlin, Heidelberg,\\ New
  York. Tokyo\\
  ISBN 0-387-20228-5 Springer-verlag, New  York. Heidelberg.\\ 
  Berlin. Tokyo
\vfill


\parbox{0.7\textwidth}{No part of this book may be reproduced in 
any form by print, microfilm or any other means without 
written permission from the Tata Institute of 
Fundamental Research, Colaba, Bombay-400 005.}
\vfill


Printed by\\
{\bf INSDOC Regional Centre.}\\
{\bf Indian Institute of Science Campus,}\\
Bangalore-560 012\\
and published by H.Goctzc, Springer-Verlag, \\
Heidelberg, West Germany\\
{\bf Printed In India}
\end{center}

\eject

\thispagestyle{empty}
\chapter*{Preface}


These lectures introduce the modern theory of continuation or path
following in scientific computing. Almost all problem in science and
technology contain parameters. Families or manifolds of solutions of
such problems, for a domain of parameter variation, are of prime
interest. Modern continuation methods are concerned with generating
these solution manifolds. This is usually done by varying one
parameter at a time - thus following a parameter path curve of
solutions.  

We present  a familiar, interesting and simple example in Chapter~\ref{chap1}
which displays most of the basic phenomena that occur in more complex
problems. In Chapter~\ref{chap2} we examine some local  continuation methods,
bases mainly on the implicit function theorem. We go on to introduce
concepts of global continuation, degree theory and homotopy invariance
with several important applications in Chapter~\ref{chap3}. In Chapter~\ref{chap4},
we discuss practical path following procedures, and introduce folds or
limit point singularities. Pseudo-arclength continuation is also
introduced  here to circumvent the simple fold difficulties. General
singular points and 
bifurcations are briefly studied in Chapter~\ref{chap5} where branch switching
and (multiparameter) fold following are discussed. We also very
briefly indicate how periodic solutions path continuation and Hopf
bifurcations are Incorporated into our methods. Finally in Chapter~\ref{chap6},
we discuss two computational examples and some details of general
methods employed in carrying out such computations.  

This material is based on a series of lectures I presented at the
Tata institute of Fundamental Research in Bangalore, India during
December 1985, and January 1986. It was a most stimulating and
enjoyable experienced for me, and  the response and interaction with
the audience was unusually rewarding. The lecture notes were
diligently recorded and written up by Mr. A.K. Nandakumar of T.I.F.R.,
Bangalore. The final chapter was mainly worked out with Dr. Mythily
Ramaswamy of T.I.F.R. Ramaswamy also completely reviewed the entire
manuscript, corrected many of the more glaring errors and made many
other improvements. Any remaining errors are due to me. The iteration
to converge on the final manuscript was allowed only one step due to
the distance involved. The result, however, is surprisingly close to
parts of my original notes which are being independently prepared for
publication in a more extend form.  

I am most appreciative to the Tata Institute of Fundamental Research
for the opportunity to participate in their program. I also wish to
thank the U.S. Department of Energy and the U.S.Army Research Office
who have for years sponsored much of the research that has resulted in
these lectures under  the grants: 0E-AS03-7603-00767 and
DAAG29-80-C-0041.  
\bigskip


\begin{flushright}
{\large\bf Herbert B. Keller}\\
Pasadena, California \\
December 29, 1986
\end{flushright}

