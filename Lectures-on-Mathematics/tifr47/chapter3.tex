

\chapter{Algebraic Tori}\label{chap3}

\section{Definitions and examples ([6],[7])}\label{chap3:sec3.1} 
\pageoriginale

\medskip
\noindent{\textbf{Definition of {\boldmath$G_m$}.}}
This is the algebraic group defined  over the prime field
 of the universal domain such that for any field $K$, the $K$-rational
 points of $G_m$ is $K^*$. 

\medskip
\noindent{\textbf{Notation}}
 If $G$ is an algebraic group and $L$ a field of definition we denote
 by $G_L$ the group of $L$-rational points of $G$. 

\begin{defi*}
An algebraic group $G$ is said to be a torus if it is isomorphic to a
product of copies of $G_m$; a field $L$ is said to be a splitting
field of $G$ if this isomorphism is defined over $L$. 
\end{defi*}

\begin{example*}
\begin{enumerate}[1)]
\item Let $L/K$ be a finite separable extension of fields. We define
  an algebraic group $L^*$ as follows: $(L^*)_{\bar{K}}= $ units of  
$ L \otimes_K \bar{K}$; choosing a basis of $L/K$ we get a basis for
$L \otimes_K \bar{K}$
  over $\bar{K}$ and with respect to this basis multiplication by a
  unit of $L \otimes_K \bar{K}$ is an element of $GL(n,\bar{K})$ where
  $n=[L:K]$. This makes $(L^*)_{\bar{K}}$ into a closed subgroup of
  $GL(n, \bar{K})$. Now $L \otimes_K  \bar{K} \widetilde{=}  \bar{K} \otimes
  \cdots \otimes \bar{K}$. (Here we use the fact that $L/K$ is
  separable). Hence $L^*_{\bar{K}} \widetilde{=}  (\bar{K}^*)^n$; this
  shows that $L^*$ is an algebraic torus. Any Galois extension of
  $K$ containing $L$ is a splitting field of $L^*$. 

\item Let $L/K$ be as above; define an algebraic group $G$ by the
  requirement $G_{\bar{K}}= \{ x \in (L \otimes_K \bar{K})^* /
  Nx=1\}$; this consists of those elements $(x_1, \ldots, x_n)$
  of $(\bar{K}^*)^n$ with $x_1 \cdots x_n =1$. Hence $G$ is an
  algebraic torus isomorphic to $G_m^{n-1}$ over $\bar{K}$. 
\end{enumerate}
\end{example*}

\begin{exam}\label{chap3:exam3}
$SO_2$\pageoriginale for a non-degenerate quadratic form $\mathscr
  {G}$ is an algebraic torus. 

Let $V$ be the corresponding quadratic space. Let $e_1,e_2$ be an
orthogonal basis of $V$ with $q(e_2)=cq(e_1)$. Then the orthogonal
group of $V$ consists of matrices of the form		 
$
\begin{pmatrix}
\lambda_1 &\lambda_2\\
-c \lambda_2 & \lambda_1
\end{pmatrix},
\begin{pmatrix}
\lambda_1 &\lambda_2\\
+c \lambda_2 & -\lambda_1
\end{pmatrix} 
\text{ with  }\lambda_1$, $\lambda_2 \in K \text{ and } \lambda_1^2 + c
\lambda_2^2=1;$ hence the special orthogonal group of  $V$ consists of
the matrices 
$
\begin{pmatrix}
\lambda_1 &\lambda_2\\
-c \lambda_2 & \lambda_1
\end{pmatrix}
\text{ with } \lambda_1^2 +c \lambda_2^2=1
$.
Such matrices are isomorphic to the group of elements of norm 1 in
the quadratic extension $K[X]/_{(x^2+c)}$ of $K$; hence $SO_2$ is an
algebraic torus. 
\end{exam}

Another example is the following.

\begin{exam}\label{chap3:exam4}
If $h$ is a skew hermitian form on the quaternion division algebra $C$
over $K$ then $SU_1(C/K,h)$, the special unitary group is an algebraic
torus.  

If $T$ is a torus defined over $K$ then there exists a finite Galois
extension $L$ of $K$ which is a splitting field of $T$. This is
because the rational functions defining the isomorphism $T
\widetilde{=} G_m^n$ are finite in number and we can adjoin the
coefficients of these to $K$ to get a field $L'$; the field we
require can be taken to be some finite normal extension $L$ of $K$
containing the field $L'$. 
\end{exam}

\begin{thm}\label{chap3:thm1}
Suppose $T$ is an algebraic torus defined over $K$; let $L$ be a
splitting field of $T$; then if $X= \Hom (G_m,T)$ we have $T_L
\widetilde{=} X \otimes_Z L^*$.  

Here\pageoriginale $\Hom (G_m,T)$ denotes the set of morphisms of
$G_m$ into $T$ defined over $L$ and which are also group
homomorphisms. 
 \end{thm} 

\begin{proof}
Consider the map $L^* \otimes_Z \Hom ( G_m,T)\rightarrow T_L$
defined by  $x \otimes f \rightarrow f(x)$. This makes sense since $f$
is defined over $L$ so that $f(x) \in T_L$. This is an
isomorphism. For if $T\widetilde{=} G_m^n$ over  $L$, $T_L \widetilde{=}
G_m (L)^n = (L^*)^n $; also $L^\ast \otimes_Z \Hom (G_m,T)\widetilde{=}L^*
\otimes_Z  Z^n \widetilde{=} (L^*)^n$; after these identifications the
map defined above becomes the identity map of $(L^*)^n$ onto
itself. This proves the theorem. 
\end{proof}

\begin{remark*}
If $L/K$ is a Galois extension then $g_{L/K}$ acts on $X = \Hom
(G_m,\break T)$ so that if $f: G_m \rightarrow T$ is an element of $X$,
$^sf(^sx)= {}^s (f(x))$; it therefore acts on $X \otimes_Z L^*$
too. Then the above isomorphism is an isomorphism of $g_{L/K}$-modules
$L^* \otimes_Z X$ and $T_L$. 
\end{remark*}

If $T$ is an algebraic torus and  $\widetilde{T}$ is a connected
commutative group which is a covering of $T$, then $\widetilde{T}$ is
also an algebraic torus. For if $0:\widetilde{T} \rightarrow T$ is the
covering an if $x \in \widetilde{T}$ is unipotent then $p(x)$ will be
unipotent and since $T$ is a torus $p(x)=1$; hence $\widetilde{T}_u$,
the unipotent part of $T$ is contained in the kernel of $p$ which  is
finite. Hence $\widetilde{T}_u$ being connected, it is $\{1\}$, and so
$\widetilde{T}$ consists only of semisimple elements and being
commutative and connected it is an algebraic torus. 


\section{Class Field Theory}\label{chap3:sec3.2} 

In this section we shall list some of the results of class field
theory we require without proofs. For proofs see the references $S_1,
T_a$ 

\medskip
\noindent{\textbf{a) Local class field Theory.}}\pageoriginale

Let $K$ be any field. We denote by $B_K$ the Brauer
group of $K$; it 
is defined to be the inductive limit of $H^2(g_{L/K}, L^*)$ where $L/K$
runs through the set of finite Galois extensions and the limit is
taken with respect to the inflation homomorphisms. Alternatively,
consider the class $\ell$ of simple algebras over $K$ with centre
$K$. If $A$, $A'$ are two such algebras we know by Wedderburn's theorem
that $A \cong M_n (D)$, $A' \cong M_{n'} (D')$ where $D$, $D'$ are division
algebras. The integer $n$ and the isomorphism class of $D$
characterise $A$ upto isomorphism; similarly $n'$ and the isomorphism
class of $D'$ characterise $A'$, upto isomorphism; We shall say $A
\sim A'$ if $D \cong D'$. Let $B_K = \ell / \sim$ be the set of
equivalence classes; $B_K$ is then made into a group as follows: if
$A$, $B$ are representatives of two classes of $B_K$ then the
equivalence class of $A \otimes_K B$ depends only on those of $A$ and
$B$ so that we get a map $B_K \times B_K \rightarrow B_K$. This
composition makes $B_K$ into a group, the identity element being the
equivalence class of $K$ and the inverse of a class with
representative $A$ is the class with representative $A^o$, the
opposite algebra of $A$. 

Now let $K$ be a $p$-adic field, i.e. a field complete under a discrete
valuation, with finite residue class field. 

\begin{thm}[cf. \cite{keyDe}]\label{chap3:thm2}
 There is a canonical isomorphism $B_K \cong Q/Z$. If
    $A$ is a simple algebra over $K$ with center $K$ then the image of
    its class under the above isomorphism is called the Hasse -
    invariant of $A$; if we denote this invariant by $inv_K (A)$ then
    for any finite extension $L$ of $K$. We have $inv_L (A \otimes_K
    L) = [ L : K] inv_K(A)$. We have further 
 \end{thm} 

 \begin{thm}\label{chap3:thm3}
If $A$\pageoriginale is a central simple algebra over $K$ then it is
split by an extension $L$ of $K$ with the property $[L : K]^2 = [ A:K ]$. 
 
 Another theorem which we will need is the following theorem of Tate
 and Nakayama (\cite{keyS1},  IX, \S 8): 
 \end{thm}

 \begin{thm}[Nakayama-Tate]\label{chap3:thm4}
 Let $G$ be a finite group, $A$ a $G$-module, an $(a)$ an element of
 $H^2(G,A)$. For each prime number $P$ let $G_P$ be a $p$-Sylow subgroup
 of $G$, and suppose that   
\begin{enumerate}[1)]
\item $H^1 (G_p , A) = 0$

\item $H^2 (G_p , A)$ is generated by Res $G/G_p(a)$ and the order of
  $H^2 (G_p , A)$ is equal to that of $G_p$. 
\end{enumerate}
 
 Then, if $D$ is a $G$-module such that for $(A,D) = 0$, the cup
 multiplication by $(a_g) = Res G/g(a)$ induces an isomorphism  
 $$
 \hat{H}^n(g,D) \rightarrow \hat{H}^{n+2} (g, A \otimes D)
 $$
 for every $n \in Z$ and every subgroup $g$ of $G$.
 
 In this theorem the Tate cohomology groups $\hat{H}$ are defined as
 follows (\cite{keyS1},  VIII, \S 1): 
\begin{align*}
 \hat{H}^n(G,A) & = H^n (G, A) \text{ for  } n \ge 1\\
\hat{H}^o (G,A) & = A^G /NA \text{ where } N:A \rightarrow A \\
&\quad\text{ is
  the norm homomorphism defined }\\
\text{ by } \; N(a) & = \sum_{s \in G} s_a\\ 
\hat{H}^{-1}(G,A) &  = {}_N A / IA\\
 \end{align*}
 where ${}_N A$ denotes the kernel of the norm mapping and $I$ is the
 augmentation ideal of $Z(G)$ generated by the elements $1-s$ with $s
 \in G$. 

$\hat{H}^{-n-1}(G,A) = H_n (G,A)$, the $n^{\rm th}$ homology
 group for $n \ge 1$.  
 
 \noindent
 If $L/K$\pageoriginale is a finite Galois extension of the adic field
 $K$ then $H^2(g_{L/K}, L^*)$  is a cyclic group generated by the `fundamental
 class' $(a)$ and is of order equal to $[L:K]$. Condition $2)$ of
 the present theorem for $g = g_{L/K}$, $A = L^*$ is an easy consequence
 of the property of Hasse invariant stated under theorem \ref{chap3:thm2}. 
 \end{thm}

 Using the theorem we can prove the following

 \begin{thm}\label{chap3:thm5}
Let $T$ be an algebraic torus over $K$ split by the finite Galois
extension $L$; with the notations of theorem \ref{chap3:thm1} we have  
$\hat{H}^{n+2} (g_{L/K}, T) \cong \hat{H}^{n} (g_{L/K}, X)$.
 \end{thm} 

 \begin{proof}
By theorem \ref{chap3:thm1} we know that $\hat{H}^{n+2} (g_{L/K}, T) \cong
\hat{H}^{n+2} (g_{L/K}, X \otimes_Z L^*)$. We shall apply theorem
\ref{chap3:thm4}. Condition 1) is Hilbert's theorem 90 (cf. \S
\ref{chap1:sec1.7}, example 
\ref{chap1:exam1}). We have just seen that conditions 2) is satisfied. Since $X$
is a free abelian group  $\Tor_l(X,L^*)$ is zero. Hence cup
multiplication by the fundamental class of $H^2(g_{L/K}, L^*)$ induces
by theorem \ref{chap3:thm3} an isomorphism  
$$
\hat{H}^{n+2} (g_{L/K}, X \otimes_Z L^*) \simeq \hat{H}^{n} (g_{L/K}, X)
$$
which proves the result.
 \end{proof}


\section{Global class field Theory.}\label{chap3:sec3.3}
 
 Let $K$ be a number field and $N$ a finite Galois extension with
 Galois group $g$; we shall employ the following notations: 
 
 \noindent
 $I = I_N$ - the idele group of $N$
 
 \noindent
 $C = C_N$ - the idele class group of $N$, i.e. $I_{N/N^*}$ where
 $N^*$ is imbedded in $I_N$ by the map $a \rightarrow (\ldots, a,
 \ldots)$. 
 
 \noindent
 $\bar{v}$ - places\pageoriginale of $N$
 
 \noindent
 $v$ - place of $K$
 
 \noindent
 $\infty$ - the set of infinite places.
  
 To any given place $v$ of $K$ we choose an extension $\bar{v}$ of $v$
 to $N$ and keep it fixed; this extension is denoted by $h(v)$;
$g_{h(v)}$ is then used to denote the Galois group of the extension
 $N_{h(v)}/K_v$, which is the decomposition group of $h(v)$. 
 
 \noindent
 $Z$ - the ring of rational integers
 
 \noindent 
 $Y$ - the free abelian group generated by the set of places of $N$.  
 
 \noindent
 If $s \in g$ and $\bar{v}$ a place of $N$ then ${}^s \bar{v}$
 denotes the place of $N$ defined by $|x|_{{}^s\bar{v}} =
 |x|_{\bar{v}}$; $g$ acts on $Y$ by the rule $^s( \sum n_{\bar{v}}
 \bar{v}) = \sum n_{\bar{V}} {}^s_{\bar{v}}$; $W$ - kernel of the
 surjective $g$-homomorphism $Y \rightarrow Z$ defined by  
 $$
\sum n_{\bar{v}}. \bar{v} \rightarrow \sum n_{\bar{v}}.
 $$
 With these notations we shall explain a result of Nakayama and Tate 
 which we shall use in the study of `Hasse Principle' in number
 fields.  
 
 We compare the two exact sequences of $g$-modules:
 \begin{gather*}
1 \longrightarrow N^* \longrightarrow I \longrightarrow C \longrightarrow
1  \tag{1}\label{c3:eq1} \\ 
1 \longrightarrow W \longrightarrow Y \longrightarrow Z
\longrightarrow 1  \tag{2}\label{c3:eq2}  
 \end{gather*} 
 Tate's theorem then asserts that we can find elements $\alpha_1 \in
 H^2 (g,\break \Hom (Z,C)), \alpha_2 \in H^2 (g, \Hom (Y, I))$ and $\alpha_3
 \in H^2 (g, \Hom (W, N^*))$ such that\pageoriginale cup
 multiplication by these 
 cohomology classes induce isomorphisms $\hat{H}^i(g,Z) \cong
 \hat{H}^{i+2}(g,C), \hat{H}^i (g,Y) \cong \hat{H}^{i+2}(g,I),
 \hat{H}^i(g,W) \cong \hat{H}^{i+2}\break(g,N^*)$; moreover there exists a
 commutative diagram with exact rows: 
\begin{equation*}
\vcenter{\xymatrix{
\ldots \ar[r] & \bar{H}^{i+2} (g, N^\ast) \ar[r] & \bar{H}^{i+2} (g,I)
\ar[r]& \bar{H}^{i+2} (g,C) \ar[r] & \cdots \\
\ldots \ar[r] & \hat{H}^i (g,X) \ar[r] \ar[u] & \hat{H}^i (g,Y) \ar[u]
\ar[r] &  \hat{H}^i (g,Z) \ar[r] \ar[u] & \cdots
}} \tag{3}\label{c3:eq3} 
\end{equation*}
 where the vertical maps are the isomorphisms mentioned above.
 
 Let $M$ be a torsian free $g$-module; tensoring (\ref{c3:eq1}) and
 (\ref{c3:eq2}) with 
 $M$ we get exact sequences of $g$-modules: 
 \begin{gather*}
1 \longrightarrow M \otimes N^* \longrightarrow M \otimes I
\longrightarrow M \otimes C \longrightarrow 1 \tag{$1'$}\label{c3:eq1'}\\  
1 \longrightarrow M \otimes N \longrightarrow M \otimes Y
\longrightarrow M \otimes Z \longrightarrow 1 \tag{$2'$}\label{c3:eq2'}  
 \end{gather*} 
 Then Cohomology classes $\bar{\alpha}_1 \in \hat{H}^2 (g, \Hom (M
 \otimes Z, M \otimes C))$, $\bar{\alpha}_2 \in \hat{H}^2 (g, \Hom (M
 \otimes Y, M \otimes I))$ and $\bar{\alpha}_3 \in \hat{H}^2 (g, \Hom
 (M \otimes W, M \otimes N^*))$ are constructed from $\alpha_1$,
 $\alpha_2$ and $\alpha_3$ such that the cup-multiplication by
 these cohomology classes give isomorphisms 
\begin{align*}
\hat{H}^i (g, M \otimes Z) & \cong \hat{H}^{i+2} (g, M \otimes C)\\
\hat{H}^i (g, M \otimes Y) & \cong \hat{H}^{i+2} (g, M \otimes I)
\end{align*}
 and $\hat{H}^i (g, M \otimes W) \cong \hat{H}^{i+2} (g, M \otimes
 N^*)$; moreover there exists a commutative diagram with exact rows: 
{\fontsize{9}{11}\selectfont
\begin{equation*}
\vcenter{\xymatrix{
\ldots \ar[r] & \hat{H}^{i+2} (g, M\otimes N^{\ast}) \ar[r] &
\hat{H}^{i+2} (g, M \otimes I)  \ar[r] & \hat{H}^{i+2} (g, M \otimes
C) \ar[r] & \ldots \\
\ldots \ar[r] & \hat{H}^i (g, M \otimes W) \ar[r] \ar[u] & \hat{H}^i
(g, M\otimes Y) \ar[r] \ar[u] & \hat{H}^i(g,M \otimes Z) \ar[r] \ar[u]&
\ldots 
}}\tag{$3'$}\label{c3:eq3'}
\end{equation*}}
where\pageoriginale the vertical maps are the isomorphisms quoted above. 

Another result which we shall need is the following isomorphisms also
proved in Tate's paper: 
\begin{align*}
 \hat{H}^{i} (g, M \otimes Y) & \cong \otimes_v \hat{H}^{i} (g_{h(v)},
 M \otimes Z) \tag{4}\label{c3:eq4}  \\
 \hat{H}^{i} (g, M \otimes I) & \cong \otimes_v \hat{H}^{i} (g_{h(v)},
 M \otimes N_{h(v)}) \tag{5}\label{c3:eq5}  
\end{align*}

With these preliminary discussion we shall go on to prove 


\begin{thm}[a]\label{chap3:thm6}
Let $T/K$ be an algebraic torus split by the finite Galois
extension $N$ of $K$; suppose $N/K$ is cyclic or there exists a place
$w$ of $K$ which is such that $N \otimes_K K_w$ is a field, i.e. there
exists a unique extension of $w$ to $N$; then for any $i$ the canonical
map 
\begin{equation*}
\hat{H}^i(g, T_N) \longrightarrow \prod_v \hat{H}^i(g_{h(v)}, T_{N_{h(v)}})
\tag{6}\label{c3:eq6} 
\end{equation*}
is injective; here the product on the right hand side is taken over
all places $v$ of $K$. 
\end{thm}

\setcounter{thm}{5}
\begin{thm}[b]\label{chap3:thm6b}%% 6
With the notations of theorem \ref{chap3:thm6} a) let $S$ be a finite set of
places of $K$ containing the infinite places. If the decomposition
groups $g_{h(v)}$, $v \in S$ are all cyclic then the canonical map 
\begin{equation*}
\hat{H}^i(g, T_N) \cdots \prod_{v \in S} \hat{H}^i(g_{h(v)},
T_{N_{h(v)}}) \tag{7}\label{c3:eq7} 
\end{equation*}
is surjective.
\end{thm}

\begin{coro*}
With the above notations the canonical map 
$$
H^1 (K,T) \longrightarrow  \prod_{v \in \infty} H^1 (K_v , T)
$$
is surjective.
\end{coro*}

\setcounter{proofoftheorem}{5}
\begin{proofoftheorem}%% 6
a). By theorem \ref{chap3:thm1},\pageoriginale we know that $T_N \cong
X \otimes N^*$ where $X = 
\Hom (G_m, T)$ this being a $g$-isomorphism; so that $\hat{H}^{i}
(g,T_N) \cong \hat{H}^{i} (g,X \otimes N^*)$. Similarly $\hat{H}^{i}
(g_{h(v)}, T_{N_{h(v)}}) \cong \hat{H}^{i} (g_{h(v)}, X \otimes
N^*_{h(v)})$. Hence we have only to prove that the mapping 
$$
\hat{H}^{i} (g, X \otimes N^*) \longrightarrow \prod_v \hat{H}^{i} 
(g_{h(v)}, X \otimes N^*_{h(v)}) 
$$
is injective. By the isomorphism (\ref{c3:eq5}) applied to $M = X$ we are
reduced to proving the injectivity of the map 
$$
\hat{H}^{i} (g,X \otimes N^*) \longrightarrow \hat{H}^{i} (g,X \otimes I).
$$
By (\ref{c3:eq3'}) this is equivalent to proving the injectivity of 
$$
\hat{H}^{i-2} (g,X \otimes W) \longrightarrow \hat{H}^{i-2} (g,X \otimes Y);
$$
again by (\ref{c3:eq3'}) the latter will follow if we prove that the map
$$
\hat{H}^{i-3} (g,X \otimes Y) \longrightarrow \hat{H}^{i-3} (g,X \otimes
Z) 
$$
is surjective. We shall prove that the map 
$$
\hat{H}^{i} (g,X \otimes Y) \longrightarrow \hat{H}^{i} (g,X \otimes Z)
$$
is surjective for all $i$ and this will establish our result. We know
by Frobenius theorem resp. by assumption there exists a place $w$ with
decomposition group $g_{h(w)} = g$; but then in (\ref{c3:eq4}) (with $M =X$ one
component on the right hand side is $\hat{H}^{i} (g,X \otimes
Z)$. This proves the result. 
\end{proofoftheorem}

\setcounter{proofoftheorem}{5}
\begin{proofoftheorem}%% 6
b). The right hand side of 7 is contained in $\otimes_v\break \hat{H}^{i}
(g_{h(v)}, T_{N_{h(v)}})$. Using (\ref{c3:eq3'}) and (\ref{c3:eq5}) we are reduced to
proving the following:\pageoriginale given elements $\alpha_v \in \hat{H}^i
(g_{h(v)},T_{N_{h(v)}})$ for $v \in S$ to find elements $\alpha_v \in
\hat{H}^i (g_{h(v)}, T_{N_{h(v)}} )$ for $v \in S$ such that the image
of the element $(\alpha_v)$ of $\otimes \hat{H}^i
(g_{h(v)},T_{N_{h(v)}} )$ under the map $\hat{H}^i (g,X \otimes I)
\rightarrow \hat{H}^i (g,X \otimes C)$ is zero; the latter question is
equivalent to the following: given finitely many components
corresponding to $v \in S$ in $\oplus \hat{H}^{i-2} (g_{h(v)}X
\otimes Z)$ to find other components so that the resulting element of
$\hat{H}^{i-2} (g,X \otimes Y)$ will have image zero under the map 
$$
\hat{H}^{i-2} (g,X \otimes Y) \longrightarrow \hat{H}^{i-2} (g,X \otimes
Z), 
$$
we shall prove that this is possible for every $i$.
\end{proofoftheorem}

By a theorem of Frobenius since all $g_{h(v)}$'s are cyclic for $v \in
S$ to any given $v \in S$ we can find a place $\bar{v}$ of $N$ not
dividing any place belonging to $S$ such that $g_{\bar{v}} = g_{h(v)}$; such a
choice is possible in infinitely many ways; let $\bar{v} = h(\hat{v})$
where $\hat{v}$ is a place of $K$. We can moreover assume that the
$\hat{v}$'s are all different; let $S^1$ be the set of places
$\hat{v}$ obtained in this way.  Suppose we are given elements $ \beta_v\in
\hat{H}^i(g_{h(v)}, X)$ for $v \in S$; define an element $(x_v) \in
\otimes \hat{H}^i(g_{h(v)}, X)  \cong \hat{H}^i(g, X \otimes Y)$ by the
requirements 

$\begin{aligned}
x_v & = \beta_v \text{ for } v \in S.\\ 
x_{\hat{v}} & = -\beta_v  \text{ if } \hat{v} \text{ is such
  that } h(\hat{v}) = \bar{v}, v \in S\\ 
x_v  & = 0 \text{ if } v \not\in S \cup S'
\end{aligned}$

\noindent
Then this element has image zero under the map $\hat{H}^i(g, X \otimes
Y) \rightarrow \hat{H}^i(g,\break X \otimes Z)$. This proves the result. 

\begin{thm}\label{chap3:thm7}
Let $K$\pageoriginale be a number field and $T$ an algebraic torus
defined over $K$ 
and split by the finite Galois extension; suppose there exists a place
$v$ of $K$ for which there exists no non-trivial homomorphism of
$G_m$ into $T$ defined over $K_v$. Then the canonical map 
$$
H^2 (g, T_N) \longrightarrow \prod_v H^2 (g_{h(v)}, T_{N_{h(v)}}) 
$$
is injective.
\end{thm}

\begin{proof}
With the notations of theorem \ref{chap3:thm6} $a)$ we have only to prove the
surjectivity of the map 
$$
\hat{H}^{-1}(g, X \otimes Y) \longrightarrow \hat{H}^{-1}(g, X \otimes 
Z). 
$$
By (\ref{c3:eq4}) we have
$$
\hat{H}^{-1}(g, X \otimes Y) \cong \oplus \hat{H}^{-1}(g_{h(v)}, X
\otimes Z) \cong N^X_v / I_v X 
$$
where $N_v$ denotes the norm mapping of $g_{h(v)}$, $I_v$ denotes the
augmentation ideal of $g_{h(v)}$ and ${}_{N_v} X$ denotes the kernel of the
norm mapping. Similarly $\hat{H}^{-1}(g, X \otimes Z) \cong {}_{N_g}X /
I_g X$, $N_g$ denoting the norm mapping of $g$  and $I_g$ the
augmentation ideal of $g$. If $\eta_v$ denotes the map
$\hat{H}^{-1}(g_{h(v)}, X ) \rightarrow \hat{H}^{-1}(g, X )$ got by
passage to quotients in the natural inclusion ${}_{N_v}X \rightarrow{}_{N_g}
X$ then the map $\hat{H}^{-1}(g, X \otimes Y) \rightarrow
\hat{H}^{-1}(g, X \otimes Z)$ is given after the above identification
by the rule 
$$
(x_v) \rightarrow \sum \eta_v x_v
$$
This shows that the proof of the theorem will be completed if we can
show the existence of a place $v$ for which ${}_{N_v} X = {}_{N_g} X$ holds; we
claim\pageoriginale that the place $v$ given in the statement of the
theorem will suffice; for by assumption $(X)^{g_{h(v)}} = (\Hom(G_m
,T))^{g_{h(v)}} = (1)$; since image $N_v \subset (X)^{g_{h(v)}}$ we
must have image $N_v = (1)$ which implies that ${}_{N_v} X =X$ but then
${}_{N_g}X=X$ and so we are through.  
\end{proof}

