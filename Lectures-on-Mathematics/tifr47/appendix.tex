

\chapter*{Appendix by T.A. Springer}

\section*{Hasse's theorem for skew-hermitian forms over\hfil\break quaternion
  algebras}\pageoriginale  

Theorem \ref{appen:thm3.3} below is equivalent to the proposition of
\S \ref{chap5:sec5.10}. The 
following proof was obtained in 1962. It differs from the proof
given in the lecture notes in that it uses besides algebraic and local
results, only Hasse's theorem for quadratic forms. 


\medskip
\noindent{\textbf{1. Algebraic results}}

Let $k$ be a field of characteristic not 2. Let $D$ be a quaternion
algebra with centre $k$. $D$ is either a division algebra or is
isomorphic to the matrix algebra $M_2(k)$. We denote by $\lambda \to
\bar{\lambda}$ the standard involution in $D$, by $\bar{D}$ the set of
all $\lambda \in D$ with $\lambda = - \bar{\lambda}$ and by $(D^*)^-$
the set of invertible elements in $D^-$. $N$ is the quadratic norm
form in $D$ and $N(x,y)=N(x+y)-N(x)-N(y)$ the corresponding symmetric
bilinear form. Let $V$ be a left $D$-module which is free and of
finite rank $n$. Let $F$ be a skew-hermitian form on $V \times V$ with
respect to the standard involution in $D$. So 
$$
F(x,y)= - \overline{F(y,x)}, \quad F(x,x) \in \bar{D} \quad  (x,y \in V). 
$$

Take any basis $(v_i)_{1 \le i \le n}$ of $V$ and consider the matrix
$M=(F(v_i,\break v_j))_{1 \le i,j \le n}$ with entries in $D$. The reduced
norm of $M$ in the matrix algebra $M_n(D)$ is an element of $k$, whose
class $\mod (k^*)^2$ is independent of the particular basis. $F$ is
called \textit{non-degenerate} if $M$ is non-singular. We assume
henceforth $F$ to be non-degenerate. Then the element of $k^*/(k^*)^2$
defined by the reduced norm of $M$ is called\pageoriginale the
\textit{discriminant} of $F$ and is denoted by $\delta(F)$.  

An element $x \in V$ is called \textit{non-singular} if the submodule
$Dx$ of $V$ is isomorphic to $D$. This is equivalent to the following:
if we express $x$ in terms of a suitable basis of $V$, then one of the
components of $x$ is invertible in $D$. We say that $F$
\textit{represents zero non-trivially} if $F(x,x)=0$ for some
non-singular $x$. One has the usual results about existence of
orthogonal bases in $V$ etc. 

We next describe some algebraic constructions. Let $\lambda \in D$ be
such that $K=k(\lambda)$ is a quadratic semi-simple subalgebra of
$D$. Let $\mu \in D^*$ be such that $N(\lambda,1)=N(\lambda, \mu
)=0$. Then any element $\rho$ of $D$ may be written in the form $\rho
= \xi +\eta\mu$ with $\xi, \eta \in K$. We have
$\bar{\rho}=\bar{\xi}-\eta \mu$ and $\rho \eta = \bar{\eta} \mu (\eta
\in K)$ 

$F$ and $V$ being as before, we can then write 
\begin{equation*}
F(x,y)=G(x,y)+H(x,y)\mu , \tag{1}\label{appen:eq1}
\end{equation*}
where $G(x,y)$, $H(x,y)\in K$. It readily follows that $H$ is a
non-degenerate symmetric bilinear form on $V$, considered as a
$K$-module and that\break $G(x,y)= -\overline{G(x,y)}= -H(x,\mu y)$. Define
the discriminant class $d(H) \in K^*/(K^*)^2$ in the obvious way. Let
$\varphi$ be the homomorphism $k^*/(k^*)^2 \to K^*/(K^*)^2$ induced by
the injection $k \to K$. 

\setcounter{section}{1}
\begin{lemma}\label{appen:lem1.1}%%% 1.1
$d(H) = \varphi (\delta(F))$.
\end{lemma}

\begin{proof}
Using an orthogonal basis of $V$ with respect to $F$ it follows that
it suffices to prove this if $V=D$, in which case an easy computation
establishes the assertion. 
\end{proof}

Suppose\pageoriginale now that $D=M_2(k)$. If $\lambda\in D$ we denote by
$\lambda_1$ and $\lambda_2$ the upper and lower row vectors of the
matrix $\lambda$. Identifying $V$ with $D^n$, we may write $x \in V$
in the form $x=(^{x_1}_{x_2})$, where $x_i \in k^{2n}$. It is easily
seen that if $F$ is any non-degenerate skew-hermitian form on $V$, we
have  
\begin{equation*}
F(x,x)=
\begin{pmatrix}
f(x_1,x_2) & -f(x_1,x_1) \\
& & \\
f(x_2,x_2) & -f(x_1,x_2)
\end{pmatrix},\tag{2}\label{appen:eq2}
\end{equation*} 
where $f$ is a symmetric bilinear form on $k^{2n}$, whose discriminant
class is $\delta(F)$, so that $f$ is non-degenerate. 

\begin{lemma}\label{appen:lem1.2}%%% 1.2
$F$ represents zero non-trivially if and only if $f$ has
  Witt-index $\ge 2$. If this is so, there exists $x$ in $D^n$ such
  that $F(x,x)=0$ and that any given component of $x$ is invertible in
  $D$. 
\end{lemma}

\begin{proof}
The first assertion is readily verified, using the relation (\ref{appen:eq2})
between $F$ and $f$, 
\end{proof}

Now suppose that the following holds: if $F(x,x)=o$, then the first
component of $x \in D^n$ is $0$. For $f$ this means that there exists
linearly independent vectors $a,b \in k^{2n}$ such that  
\begin{equation*}
f(x_1,a) f(x_2,b)-f(x_1,b)f(x_2,a)=0, \tag{3}\label{appen:eq3}
\end{equation*}
whenever $f(x_1,x_2)=f(x_1,x_1)=f(x_2,x_2)=0$.

If the index of $f$ is at least 2 we can find for any two $x_1,x_2
\in k^{2n}$ which are isotropic with respect to $f$, $a~ y \in k^{2n}$
such that\pageoriginale $f(y,y)=f(x_1,y)=f(x_2,y)=0$. This implies
that (\ref{appen:eq3}) holds 
for \textit{all} pairs of isotropic vectors $x_1,x_2$. Since the
isotropic vectors of $f$ span the whole of $k^{2n}$, it follows then
that (\ref{appen:eq3}) holds for all $x_1,x_2 \in k^{2n}$, which contradicts the
linear independence of $a$ and $b$. 

This establishes the second assertion.

\medskip
\noindent{\textbf{2. Local fields}}

Assume now that $k$ is a local field, i.e. complete for a
non-archime\-dian valuation, with finite residue field. With the same
notations as before, we then have the following result.   

\setcounter{section}{2}
\setcounter{lemma}{0}
\begin{lemma}\label{appen:lem2.1}%%% 2.1
\begin{enumerate}[(a)] 
\item  If rank  $ V > 3$, then $F$ represents 0 non-trivially; 

\item If is a division algebra, the equivalence classes of
  non-degenerate $F$ are characterised by their rank and
  $\delta (F)$;  

\item There is exactly one equivalence class of anisotropic forms
  of rank 3.  
\end{enumerate}
\end{lemma}

\begin{proof}
For the case of a division algebra $D$ see \cite{keyTs}. If $D$ is a matrix
algebra, this follows from the known properties of quadratic forms
over local fields, by using the relation (\ref{appen:eq2}) between
$F$ and $f$.   
\end{proof}

\begin{lemma}\label{appen:lem2.2}%% 2.2
If rank $F= 2$ then there exists $\alpha \in k^*$
  with the following property: any non-singular $\lambda \in D^-$ with
  $N(\lambda) \notin \alpha (k^*)^2$ is represented by $F$. 
\end{lemma}

\begin{proof}
This follows from lemma \ref{appen:lem2.1}, $(c)$.
\end{proof}

We now assume $D$ to be a matrix algebra. Let $\underbar{o}$ be a
maximal order in $D$. 

\begin{lemma}\label{appen:lem2.3}
 If\pageoriginale $\xi_1,\alpha_1 \bar{\xi}_1+ \xi_2 \alpha_2
 \bar{\xi}_2$ is a skew-hermitian form of rank 2 such that $\alpha_1$
 and $\alpha_2$ are unite in $\underbar{o}$ and that
 $N(\alpha_1)N(\alpha_2)$ is a square, then this form represents any
 non-singular $\lambda \in D^-$. 
\end{lemma}

\begin{proof}
$\underbar{o}$ is isomorphic to the subring of $M_2(k)$ whose entries
  are integers of $k$. using the corresponding representation (\ref{appen:eq2}) of
  our skew-hermitian form, we see that $f$ is now a quadratic form in
  $4$ variables over $k$, with integral coefficients, whose
  discriminant is the square of a unit of $k$. Such a form is known to
  have index 2 (see \cite{keyE3} \S 9). Lemma 2 implies that $F$
  represents $0$ non-trivially. It then represents all elements of
  $D^-$ (this is proved as in the division algebra case). 
\end{proof}


\medskip
\noindent{\textbf{3. Global fields}}%% 3

Now let $k$ be a global field of characteristic not 2, i.e. an
algebraic number field or a field of algebraic functions of dimension
1, with a finite field of constants. Let $D$ be a quaternion
division algebra with centre $k$. For any place $v$ of $k$ we denote
by $D_v$ the algebra $D \otimes_k k_v$ over the completion
$k_v$. $D_v$ is a division algebra for a finite number of places $v$,
by the quadratic reciprocity law this number is even. 

Let $V$ and $F$ be as in nr. 1. We put $V_v=V \otimes_k k_v$, this
is a free $D_v$-module. $F_v$ denotes the skew-hermitian form on $V_v$
defined by $F$. 

\setcounter{section}{3}
\setcounter{lemma}{0}
\begin{lemma}\label{appen:lem3.1}%%% 3.1
 Suppose that rank $F=2$ and $\delta (F)=1$. There exists a finite
  set $S$ of places of $k$ such that for any $v \in S$ and any
  $\lambda \in D^-$ the equation $F_v(x,x)=\lambda$ is solvable with
  $x \in V_v$.  
 \end{lemma} 

 \begin{proof}
Let\pageoriginale $F(x,x)= \xi_1 \alpha_1 \bar{\xi}_1 + \xi_2 \alpha_2
\bar{\xi}_2$. with respect to some basis of $V$. Let $\underbar{o}$ be
a maximal order in $D$ and take for $S$ the set places consisting of:
all infinite places; the finite places for which $\alpha_1, \alpha_2$
are not units in the maximal order $\underbar{o}_v$ of $k_v$, defined
by $\underbar{o}$; those for which $D_v$ is a division algebra . $S$
is finite. By lemma \ref{appen:lem2.3} it satisfies our requirements. 
  \end{proof} 

\setcounter{prop}{1}
 \begin{prop}\label{appen:prop3.2}
 Let $\lambda \in (D^*)^-$. Suppose that for all places $v$ there
  exists $x_v \in V_v$ such that $F_v(x_v,x_v)=\lambda$. Then there
  exists $x \in V$ and $\alpha \in k^*$ such that $F(x,x)=\alpha
  \lambda$. 
 \end{prop} 

 \begin{proof}
We put $K=k(\lambda)$. Writing $F$ in the form (\ref{appen:eq1}), our assumptions
imply that the quadratic form $H(x,x)$ represents 0 non-trivially in
all completions of $K$. By Hasse's theorem for quadratic forms
$H(x,x)=0$ has a non-trivial solution $x$ in the $K$-vector space
$V$. This is equivalent to the assertion. 
 \end{proof} 
 
 We come now to the proof of Hasse's theorem for skew-hermitian
 quaternion forms. 

\setcounter{theorem}{2}
 \begin{theorem}\label{appen:thm3.3}%%% 3.3
 Suppose that rank $F \ge 3$. If $F_v$ represents zero
  non-trivially for all places $v$, then $F$ represents zero
  non-trivially 
 \end{theorem} 

 \begin{proof}
Let $n= \rank F$. We distinguish several cases.
 \end{proof}

 $(i)~ n=3$. We first assert that under the assumptions of the theorem
there exists $\lambda \in D^-$ such that $\delta
(F)=N(\lambda)(k^*)^2$. Such a $\lambda$ is a solution of a ternary
equation $N(\lambda)= \alpha$. By Hasse's theorem for quadratic forms
it suffices to verify that this equation is everywhere locally
solvable. Solvability for the places $v$ where $D_v$ is a matrix
algebra is clear (since $N$ is then isotropic on $D^-_v$). If $D_v$ is
a\pageoriginale division algebra, the assumption that $F_v$ represents $0$
non-trivially implies that $F_v$ can be written, after a suitable
choice of coordinates, in the form $\xi_1 \alpha \bar{\xi}_1-\xi_2
\alpha \bar{\xi}_2+\xi_3 \beta \bar{\xi}_3$. It follows that $\delta
(F(k^*_v)^2 = N (\beta)(k^*_v)^2$. Since $\beta \in D^-_v$, this
implies the solvability for all places. 

We fix such a $\lambda \in D^-$ and let $K=k(\lambda)$. Write $F$ in
 the form (\ref{appen:eq1}). Our assumption imply that the Witt-index of the
 symmetric bilinear form $H$ is $\ge 2$ at all places $v$ of
 $K$. Hence the index of $H$ is $\ge 2$. by Hasse's theorem. Also $-
 \lambda^2=N(\lambda)\in \delta(F)$. by lemma \ref{appen:lem1.1} it follows that
 $d(H)=-(K^*)^2$. This combined with the fact that the index of $H$ is
 at least 2 implies that the index of $H$ equals 3, i.e. $H$ is
 of maximal index. 
 
 Consider now $V$ as a $6$-dimensional vector space over $K.\mu$ being
 as in nr. 1, put $Tx = \mu x$. This is a semi-similarity of
 $H$ (relative to the non-trivial $k$-automorphism of $K$) and $T^2 x =
 \mu ^2 x$. What we have to prove is that exists a non-trivial
 solution of  
 \begin{equation*}
H(x,x)=H(Tx,x)=0. \tag{4}\label{appen:eq4}
 \end{equation*} 
 
 Take a three-dimensional subspace $W$ of $V$, which is totally
 isotropic (for $H$). If $TW \cap W \neq (0)$, any non-zero $x$ in
 this intersection satisfies (\ref{appen:eq4}). Assume now that $TW \cap
 W=(0)$. Then $V$ is the direct sum of $W$ and $TW$. Let $H$ be the
 hermitian form on the $3$-dimensional $K$-vector space $W$ defined by  
 $$
 H_1(u,v)= \lambda H(u,Tu). 
 $$
 
 It\pageoriginale then follows that the solvability of (\ref{appen:eq4}) is
 equivalent to the existence of $u,v \in W$, not both $0$, such that   
 \begin{equation*}
H_1(u,v)=H_1(u,v)+\mu^2 H_1(v,v)=0. \tag{5}\label{appen:eq5}
 \end{equation*} 
 
 Let $\rho \in k^*$ be an element in the discriminant class $d(H_1)$
 of $H_1$. If   
  \begin{equation*}
H_1(w,w)=-\rho \mu^2  \tag{6}\label{appen:eq6}
 \end{equation*} 
  has a solution $w \in W$, then an orthogonal basis $u,v$ of the
  orthogonal complement of $w$ in $W$ will give a solution of (\ref{appen:eq5}). 
 
 So it suffices to prove that (\ref{appen:eq6}) is solvable. Now the values
 $H_1(w,w)$ are those of a 6-dimensional quadratic form $g$ over
 $k$, so that we can prove the solvability of (\ref{appen:eq6}) by using again
 Hasse's principle for quadratic forms. For finite places $v$ a
 6-dimensional form represents everything (since then the Witt-index
 is positive), so we have only to consider an infinite place $v$ (hence
 we may take $k$ to be a number field). If $v$ is complex there is no
 problem. If $v$ is a real place of $k$ which splits in $K$, $g_v$ is
 easily seen to be an indefinite quadratic form over $k_v$, which
 represents everything. So let $v$ be a real place of $K$ such that
 $K_v=K \otimes_k k_v$ is complex. If $H_1$ is indefinite (\ref{appen:eq6}) is
 solvable. If not, the solvability of (\ref{appen:eq5}) in $W_v$ (which follows
 from the assumptions) implies that $\mu ^2$ is negative in
 $k_v$. Then $- \rho \mu^2$ is positive or negative in $k_v$ if $H_1$
 is positive or negative definite, hence can be represented by
 $H_1$. This settles the case $n=3$. 
 

 \noindent
 $(ii) ~n \ge 4$.\pageoriginale Take $F$ in the form $\sum
 \limits^{n}_{i=1} \xi_i  \alpha_i \bar{\xi}_i$. We first prove a
 lemma.  

\setcounter{lemma}{3}
\begin{lemma}\label{appen:lem3.4}%%% 3.4
 Let $v$ be a place of $k$.
\begin{enumerate}[(a)]
\item There exists $\delta_v \in D^-_v$ such that the equations
\begin{gather*}
\xi_1 \alpha_1 \bar{\xi_1}+\xi_2 \alpha_2 \bar{\xi_2} =\delta_v
\tag{7}\label{appen:eq7}\\ 
\sum^{n}_{i=3} \xi_i \alpha_i \bar{\xi}_i= -\delta_v
\end{gather*}
have a solution in $(D_v)^n$.

\item The $\delta_v \in (D^*_v)^-$ for (\ref{appen:eq7}) is solvable form
  an open set in $ (D^*_v)^-$. 
 \end{enumerate} 
 \end{lemma}
 
\medskip
\noindent{\textbf{Proof of the lemma.}}

(a) ~ is a consequence of lemma \ref{appen:lem2.1} (a). to prove (b)
one observes 
that solvability of (\ref{appen:eq7}) depends only on the coset modulo $(k^*)^2$
of $N(\delta_v)$ and that $(k^*)^2$ is open in $k^*$. By proposition
\ref{appen:prop3.2} there exists $\lambda \in k^*$ such that  
$$
\sum^{n}_{i=2} \xi_i \alpha_i \bar{\xi}_i= \lambda \alpha_1
$$
is solvable in $D^{n-1}$. It follows that we may take $\alpha_2 =
\lambda \alpha_1$, so that in particular $N(\alpha_1)N(\alpha_2)$ is a
square. 

Now consider the global counterpart of (\ref{appen:eq7})
\begin{gather*}
\xi_1 \alpha_1 \bar{\xi}_1 +\xi_2 \alpha_2 \bar{\xi}_2 = \delta\\
\sum^n _{i=3} \xi_i \alpha_i \bar{\xi}_i = - \delta \tag{8}\label{appen:eq8}
\end{gather*}

We wish\pageoriginale to prove that there exists $\delta \in (D^*)^-$
such that (\ref{appen:eq8}) is solvable in $D^n$. This will prove the
theorem.  

Let $S$ be a finite set of places of $k$ with the property of lemma
\ref{appen:lem3.1} for the skew-hermitian form $\xi_1 \alpha_1
\bar{\xi}_1 +\xi_2 
\alpha_2 \bar \xi_2$. For each $v \in S$ we let $\xi_v$ be as in lemma
\ref{appen:lem3.4} (a). We take $\xi_{iv} \in D_v$ such that 
$$ 
\sum^n_{i = 3} \xi_{iv} \alpha_i \bar{\xi}_{iv} = - \delta_v \; (v \in S). 
$$

Approximate the $\xi_{iv}$ simultaneously by $\xi_i \in D$ such that,
putting 
$$
\delta = - \sum^n_{i = 3} \xi_{i} \alpha_i \bar{\xi}_{i}, 
$$
we have $\delta \in (D^*)^-$ and that the equations (\ref{appen:eq8}) are solvable
in $D^n_v$ for $v \in S$. In particular, the first equation (\ref{appen:eq8}) is
solvable in $D^2_v$ for $v \in S$. But by the choice of $S$, this
equation in also solvable in $D^2_v$ for $v \in S$, so that by the
case $n = 3$ of the theorem (using the second part of lemma \ref{appen:lem1.2})
this equation has a solution in $D^2$. Then we have a solution of
(\ref{appen:eq8}) since the second equation (\ref{appen:eq8}) is solvable by the construction
of $\delta$. 


\medskip
\noindent{\textbf{4. The case n = 2.}}%sec 4

For $n = 2$ Hasse's principle for skew-hermitian quaternion forms is
no longer true 			 	 

Representing 0 by a skew-hermitian form rank $2$ is tantamount (by
the second part of lemma \ref{appen:lem1.2}) to solving an equation
for $\rho \in 
D$ of the form 
$$
\rho \lambda \bar{\rho} = \lambda'  
$$
with\pageoriginale $\lambda$, $\lambda' \in (D^*)^-$.

Using proposition \ref{appen:prop3.2}, we see that in investigating Hasse's
principle we may assume that $\lambda'$ is a scalar multiple of
$\lambda$, so that we have to consider an equation 
\begin{equation*}
\rho \lambda \bar{\rho} = \tau \lambda \tag{9}\label{appen:eq9}
\end{equation*}
with $\lambda \in (D^*)^-$, $\in k^*$. We first investigate such an
equation for an arbitrary $D$. Put $K = k (\lambda)$, let $\mu$ be as
in nr. 1 (\ref{appen:eq9}) is then equivalent to 
$$
\varrho \lambda \bar{\varrho}^1  = N (\varrho)^{-1} \tau \lambda,
$$
which implies that either $\varrho \in K$, $N (\varrho) = \tau$, or
$\varrho \in K \mu$, $N(\varrho) = - N (\mu)^{-1} \tau$. One
concludes from this that (\ref{appen:eq9}) is solvable if and only if one of the
following equations is solvable in $K$ 
\begin{align*}
 N_{K/k} (\varrho) & = \tau \\ 
 N_{K/k} (\varrho) & = -N (\mu)^{-1} \tau .\tag{10}\label{appen:eq10}
\end{align*}

It is easily seen that $D$ is a matrix algebra if and only if $-N(\mu)
\in N_{K/k}(K^*)$. This implies that the two equations
(\ref{appen:eq10}) are both 
solvable or not if $D$ is a matrix algebra and are not both solvable
if $D$ is a division algebra. If $k$ is a local field or the field or
real numbers and $D$ a division algebra, the fact that $k^* /
N_{K/k}(K^*)$ has order 2 implies that precisely one of the
equations (\ref{appen:eq10}) is solvable. 

Form\pageoriginale the previous remarks one deduces the following
facts about the equation (\ref{appen:eq10}) for the case that $D$ is a division
algebra over a global field $k$. We denote by $(\alpha, \beta)_v$ the
quadratic Hilbert symbol. Let $S$ be the finite set of places such
that $D_v$ is a division algebra. Solvability of (\ref{appen:eq9}) in all $D_v$ is
equivalent to   
$$
(\tau, \lambda^2)_v = 1 \text{ for } v \notin S,
$$
solvability of (\ref{appen:eq9}) in $D$ is equivalent to:
\begin{align*}
\text{either } \qquad & \qquad (\tau, \lambda^2)_v  = 1 \text { for  all } v,\\
\text{or } \qquad & \qquad  (\tau , \mu^2 \lambda^2)_v  = 1 \text {
  for all } v.  
\end{align*}

From this it follows that the 1--dimensional skew-hermitian
forms over $D$, which are equivalent to a given non-degenerate one
at all places of $k$, form exactly $2^{s-2}$ equivalence classes ($s$
denoting the number of places in $S$). 

\begin{thebibliography}{99}
\bibitem{keyA} {A.A Albert},\pageoriginale  Structure of algebras, AMS
  Colloquium Publications, 1939, rev. ed. 1961. 

\bibitem{keyB} {N. Bourbaki}, Elements de Mathematique, Algebre 

\bibitem{keyB-T} {F. Bruhat et J}. Tits, C.R. Ac. Sci. Paris
  263(1966), 598 - 601, 766-768, 822-825, 867-869. 

\bibitem{keyC1} {C. Chevalley}, The algebraic theory of spinors, New
  York 1954. 

\bibitem{keyC2} {C. Chevalley}, Sur certains groupes simples, Tohoku
  Math. J. 7(1955), 14 - 66. 

\bibitem{keyC3} {C. Chevalley}, Classification des groupes de Lie
  algebriques, Seminaire ENS, Par is 1956 - 1958.

\bibitem{keyD-G} {M. Demazure et} A Grothendieck, Schemas en groupes,
  Seminaire IHES, Paris 1963 / 64. 

\bibitem{keyDe} { M. Deuring}, Algebren, Ergebnisse der Mathematik,
  Springer Varlag 1935, 2. Aufl. 1968. 

\bibitem{keyDi} {J. Dieudonne}, La geometrie des groupes classioues,
  Ergebnisse der Mathematik, Springer Varlag 1955, $2^e$ ed. 1963.    

\bibitem{keyE1} {M. Eichler}, Uber die Idealklassenzahl hyperkomplexer
  Systeme, Math. Z 43(1938), 481 - 494. 

\bibitem{keyE2} {M. Eichler}, Allgemeine kongruenzklasseneinteilungen
  der Ideale einfacher Algebren uber algebraischen Zahlkopern and ihro
  Z Reihen, J reine u. angew Math 179 (1938), 227-251. 

\bibitem{keyE3} {M. Eichler}, Quadratische Formen and orthogonale
  Gruppen, Springer - Verlag 1952. 

\bibitem{keyH} {G. Harder}, Uber die Galoiskohmologie halbeinfacher
  Matrizengruppen, Math, Z 90(1965), 404-428; 92(1966), 396 - 415. 

\bibitem{keyJ} {N. Jacobson}, Simple Lie algebras over a field of
  characteristic zero, Duke Math. J 4 (1938), 534 - 551. 

\bibitem{keyK1} {M. Kneser},\pageoriginale Starke Apporoximation in algebraishen
  Gruppen, J. reine u. angew Math. 218(1965), 190-203. 

\bibitem{keyK2} {M. Kneser}, Galois - Kohomologie halbeinfacher
  algebraischer Gruppen uber p - adischen Korpern, Math. Z. 88
  (1965), 40- 47, 89(1965), 250 -272. 

\bibitem{keyL} {W. Landherr}, Liesche Rings vom Typus A uber einem
  algebraischen Zahlkorper and hermitesche Formen uber einem
  Schifkorper, Abh. Math. Sem. Hamburg 12 (1938) 200 - 241. 

\bibitem{keyS1} {J.-P. Serre}, Corhomologie locaux,Paris, Hermann
  1962. 

\bibitem{keyS2} {J.-P. Serre}, Cohologie Galoisienne, Cours au
  College de France, 1962/63, ed, Springer - Varlag 1964. 

\bibitem{keySp} {T.A. Springer}, On the the equivalence of quadratic
  forms, Indag. Math. 21 (1959) 241-253. 

\bibitem{keyTa} {J. Tate}, The cohomology groups of tori in finite
  Galois extensions of number fields, Nagoya Math. J. 27(1966)
  709-719. 

\bibitem{keyTs} {T. Tsukamoto}, On the local theory of quaternionic
  antihermitian forms, J, Math. Soc. Japan 13(1961), 387-400. 

\bibitem{keyW1} {A. Weil}, The field of definition of a variety,
  Amer.J. Math. 78(1956) 509-524. 

\bibitem{keyW2} {A. Weil}, Algebras with involutions and the classical
  gruops, J.Ind. Maht. Soc 24(1961) 589-623. 
\end{thebibliography}

