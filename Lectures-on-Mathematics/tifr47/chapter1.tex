

\chapter{Galois Cohomology}\label{chap1}%chapter I

In\pageoriginale this chapter we shall collect the fundamental facts
about Galois Cohomology. Proofs are mostly straightforwards and
therefore omitted. Moreover they can be found in the basic reference
\cite{keyS2}.  


\section{Non-commutative cohomology}\label{chap1:sec1.1}

Let $G$ be a group operating on a set $A$; the image of $(s,a)$ under
the map $G \times A \rightarrow A$ defining the operation of $G$ on
$A$ will be denoted by ${}^sa$. A set is called a $G$-set if $G$ acts on
it. If the $G$-set $A$ has in addition the structure of group and if
the operation of $G$ respects this structure then we say $A$ is a
$G$-group. 

\medskip
\noindent
\textbf{Definition of $H^0(G,A)$:} If $A$ is any $G$-set then
$H^0(G,A)$ is defined to be the set $A^G$ of elements left fixed by
the operation of $G$ on $A$, i.e. $H^0(G,A) = A^G = \{a\in A | {}^s a = a
\; \forall s \in G \}$. 

\medskip
\noindent
\textbf{Definition of $H^1(G,A)$:} This definition will be given only
for a $G$-group $A$. Accordingly let $A$ be a $G$-group. $A$ mapping
$s\rightarrow a_s$ of $G$ into $A$ is said to be a 1-cocycle of $G$
in $A$ if the relation $a_{st} = a_s {}^s a_t$ holds for all $s,t \in
G$. Two 1-cocycles $(a_s)$ and $(b_s)$ are said to be equivalent if
for a suitable $c\in A$ the relation $b_s = c^{-1}a_s {}^s c$ holds for
all $s\in G$. This is an equivalence relation on the set of
1-cocycles of $G$ in $A$. $H^1(G,A)$ is by definition\pageoriginale the set of
equivalence classes of 1-cocycles. If $A$ is a commutative group we
have the usual definition of $H^i(G,A)$, $i=0,1,2,3,\ldots$ by means of
cocycles and coboundaries. We write it down here only for $i = 2$. If $A$
is commutative the definitions of $H^i (G,A)$, $i=0,1$ given above
coincide with the usual definition of these groups.  

\medskip
\noindent
\textbf{Definition of $H^2(G,A)$:} Here $A$ is any commutative
$G$-group. $A$ 2-cocycle of $G$ in $A$ is a mapping
$(s,t)\rightarrow a_{s,t}$ of $G \times G$ into $A$ satisfying the
relation $r_{a_{s,t}} ~a^{-1}_{rs,t}~ a_{r,st} ~a^{-1}_{r,s} = 1$ for
all $r,s,t \in G$. $A$ 2-coboundary is a 2-cocycle of the form
$s_{c_t}~c^{-1}_{st}~c_s$ with $c_s \in A$. If the product of two
cocycles $(a_{s,t})$ and $(b_{s,t})$ is defined as $(a_{s,t}~b_{s,t})$
the 2-cocycles will form a multiplicative group and the
2-coboundaries will form a subgroup; by definition the quotient
group is then $H^2(G,A)$. 

Let $A$ be a $G$-group and $B$ an $H$-group. Two homomorphisms $f: A
\rightarrow B$, $g: H \rightarrow G$ are said to be compatible if
$f(^{g{(s)}}a) = ^s(f(a))$ holds for every $s \in H$, $a \in A$; if $H =
G$ and $g$ is identity then $f$ is said to be a
$G$-homomorphism. Mapping a 1-cocycle $(a_s)$ of $G$ in $A$ onto the
1-cocycle $(b_s)$ of $H$ in $B$ defined by $b_s = f(a_{g(s)})$
induces a mapping $H^1(G,A) \rightarrow H^1(H,B)$; this will be a
group homomorphism in case $A$ and $B$ are commutative. If $H$ is a
subgroup of $G$ and $g$ is the inclusion the map $H^1(G,A) \rightarrow
H^1(H,B)$ defined above is called the restriction map. This definition
can be carried over to higher dimensional cohomology groups when $A$ and
$B$ are commutative. 

For\pageoriginale non-commutative $G$-group $A$, the set $H^1(G,A)$ is
not a group 
in general but it has a distinguished element namely the class of
1-cocycles of the form $b^{-1}{}^sb$, with $b \in A$. The existence of
this element enables one to define the kernel of a map $H^1(G,A)
\rightarrow H^1(H,B)$ such as we have obtained above and also to
attach meaning to the term `exact sequence of cohomology sets'. 


\section{Profinite groups}\label{chap1:sec1.2}


\begin{defi*}
A topological group $G$ is said to be profinite if it is a projective
limit of finite groups the latter carrying discrete topology. 
\end{defi*}

A profinite group is then compact and totally disconnected, it
possesses a base of neighbourhoods of the identity formed by open
normal subgroups. Conversely if a compact topological group $G$ has a
base of neighbourhoods of $1\in G$ formed by open normal subgroups $U$
then the quotients $G/U$ are finite and
$G\cong\underset{\leftarrow}{lim}G/U$; hence $G$ is profinite. 

Let $G$ be profinite and $A$ any $G$-group with discrete topology. In
such a case we shall always assume that the action of $G$ on $A$ is
continuous. This is equivalent to the requirement $A =
\bigcup\limits_U A^U$ the union being taken over the set of open
normal subgroups $U$ of $G$. We shall modify the definition of
$H^i(G,A)$ in this case by requiring the cocycles to be continuous. In
the sequel this new definition of $H^i(G,A)$ will be adhered to
whenever we consider profinite groups $G$ and $G$-sets. If $U \subset
V$ are open normal subgroups of\pageoriginale $G$ then the inclusion $A^V
\hookrightarrow A^U$  and natural projection $G/U \rightarrow G/V$ are
compatible, hence we get the induced map $\varrho^V_U : H^i(G/V,A^V)
\rightarrow H^i(G/U,A^U)$ called inflation. The sets $H^i(G/U,A^U)$
together with the maps $\varrho^V_U$ form an inductive system and
$H^i(G,A) \cong \underset{\rightarrow}{\lim}~ H^i(G/U,A^U)$. 


\section{Induction}\label{chap1:sec1.3}

Let $G$ be a profinite group, $H$ an open subgroup of $G$ and let $B$
be any $H$-set. Let $\bar{A}$ denote the set of mappings
$a:G\rightarrow B$ which map $s \in G$ onto $a(s)\in B$ satisfying the
condition $a(ts) =  {}^{t}a(s)$ for all $t \in H$, $s \in G$. $\bar{A}$
is made into a $G$-set by defining $r_a$ for $a \in \bar{A}$, $r\in G$
by the rule $({}^r a)(s) = a(sr)$; we then say that $\bar{A}$ is induced
from $B$; more generally, we call every $G$-set $A$ isomorphic to
$\bar{A}$ a $G-H$ induced set; if $B$ is an $H$-group, $A$ is a
$G$-group. Suppose $A$ is $G-H$ induced from $B$ so that we can
identify $A$ with $\bar{A}$. Mapping $a \in \bar{A}$ onto $a(1) \in B$
we get a mapping $A \cong \bar{A} \rightarrow B$ which is compatible
with the inclusion $H \hookrightarrow G$; passing to cohomology we get
a map $H^i(G,A) \rightarrow H^i(H,B)$ whenever $H^i(H,B)$ (and
therefore $H^i(G,A)$) is defined. We then have the  

\begin{lem}\label{chap1:lem1}%lemma 1
The mapping $H^i(G,A) \rightarrow H^i(H,B)$ defined
above is an isomorphism. 

For commutative $A$ and $B$, see \cite{keyS2} \ref{chap1} \S
\ref{chap2:sec2.5}. For 
completeness sake, we give the proof for $i = 0,1$ in the general
case. 
\end{lem}

\begin{proof}
 First\pageoriginale let $i=0$; if $a\in {\bar A}^G$, $(r_a)(s)=a(s)$ for
every, $r,s\in G$, taking $s=1$ we find that $a$ is a constant
function $G\longrightarrow B$. Hence the mapping
$H^{\circ}(G,A)\longrightarrow H^{\circ}(H,B)$ is injective. If $c\in
B$ the constant function  $G\longrightarrow B$ mapping every $s\in G$
onto $c$ is an element of ${\bar A}^G$; this shows that the mapping in
question is also surjective and hence bijective.   
\end{proof}

Now let $i=1$; suppose two elements $(a_r)$, $(b_r)$ of $H^1(G,A)$
have the same image under the mapping $H^1(G,A)\longrightarrow
H^1(H,B)$; then for a suitable $c\in B$ we must have
$a_r(1)=c^{-1}b_r(1)r_c$ for all $r\in H$. Now we find an element
$d\in\bar A$ with $d(1)=c$ by taking a set of representatives of
$g\mbox H$ including 1 and mapping 1 onto $c$ and the other
representatives onto arbitrarily chosen elements of $B$; replacing the
cocycle $b_r$ by the equivalent cocycle $d^{-1}b_rr_d$ we can assume
$a_r(1)=b_R(1)$ for all $r\in H$. Using the cocycle condition we have
for all $r,s,t\in G$ 
\begin{align*}
a_{rs}(t) & = a_r(t)r_{a_s}(t)=a_r(t)a_s(tr)\\
b_{rs}(t) & = b_r(t)r_{b_s}(t)=b_r(t)b_s(tr)
\end{align*}

Setting $r=t^{-1}$ in (1) and (2) we get
$a_{t^{-1}s}(t)=a_{t^{-1}(t)a_s(1)}$ and
$b_{t^{-1}s}(t)=b_{t^{-1}}(t)b_s(1)$; since $a_s(1)=b_s(1)$ for all
$s\in H$ we get
$a_{t^{-1}s}(t)b_{t^{-1}s}(t)=a_{t^{-1}}(t)b_{t^{-1}}(t)$ holding for
all $s\in H$; hence if we define $c(t)\in  B$ for every $t\in G$ by
the rule $c(t)=b_{t^{-1}}(t)a_{t^{-1}}(t)$ then by what precedes we
get for 
\begin{align*}
& s \in H, c(st)  = b_{t^{-1} s^{-1}} (st) a_{t^{-1}s^{-1}} (st)^{-1} = 
~^{s} b_{t^{-1}s^{-1}} (t) ~^{s} a_{t^{-1} s^{-1}} (t)^{-1} =\\
& \quad = ~^{s} (b_{t^{-1} s^{-1}}(t)a_{t^{-1}s^{-1}} (t)^{-1}) =
~^{s} (b_{t^{-1}}(t) a_{t^{-1}} (t)^{-1}) = ~^{s}c(t), 
\end{align*}\pageoriginale
 and hence $c$ belongs to $\bar{A}$. Given elements $r$ and $t$ in $G$
 choose $s \in G$ such that $trs = 1$; then we have
\begin{gather*}
a_{rs}(t) = a_{t^{-1}}(t) = c(t)^{-1}b_{t^{-1}}(t) = c(t)^{-1}b_{rs}(t)\\
\text{ and } a_s(tr)^{-1} = a_{r^{-1}t^{-1}}(tr)^{-1} =
b_{r^{-1}t^{-1}} (tr)^{-1} c(tr) = b_s(tr)^{-1}c(tr).
\end{gather*}

From equations (1) and (2) together with the
 foregoing it follows that $a_r(t) = a_{rs}(t)a_s(tr)^{-1} =$
 $c(t)^{-1}b_{rs}(t)b_s(tr)^{-1}c(tr) = c(t)^{-1}b_r(t)^r c(t)$
 i.e. $a_r = c^{-1}b_r r_c$, hence the cocycles $a_r$ and $b_r$ are
 equivalent; this means that the mapping $H^1(G,A) \rightarrow
 H^1(H,B)$ is injective. Now suppose $b = (b_s) \in H^1(H,B)$ is
 given. Let $V$ be a system of right representatives of $G$ 
 mod $H$, $v(s)$ the representative in $V$ of the coset $Hs$, and
 therefore $w(s) = sv(s)^{-1} \in H$. Define $a_s : G \rightarrow
 \bar{A}$ by $a_s(t) = ^{w(s)}b_{w(v(t)s)}$. It is straight forward to
 verify that $a_s(t)$ is indeed an element of $A$ and that $(a_s)$ is
 a 1-cocycle of $G$ in $\bar{A}$ whose image is $(b_s)$ under the
 mapping $H^1(G,A) \rightarrow H^1(H,B)$; this proves the surjectivity
 and so bijectivity of the mapping in question. 

\begin{remark*}
 If\pageoriginale $B$ is a $H$-group and $\bar{A}$ is $G-H$ induced
  from $B$ (so that $\bar{A}$ is a $G$-group) then the projection
  $\bar{A}\rightarrow B$ induces an isomorphism of $B$ with the
  subgroup of those $a\in \bar{A}$ which are equal to 1 outside $H$;
  if we identify $B$ with this subgroup, then $s_B$ depends only on
  the coset $Hs$ and we have $\bar{A}\cong \prod\limits_{s\in H/G}
 {}^s B$. The converse is also true, i.e. we have the lemma whose proof
  is straight forward and so omitted. 
\end{remark*}

\begin{lem}%lemma 2
A  $G$-group $A$ is $G-H$ induced if and only if there exists an
$H$-subgroup $B$ of $A$ such that $A$ is the direct product of the
subgroups $^{s} B(s\in H/G)$. 
\end{lem}


\section{Twisting}\label{chap1:sec1.4}

In the sequel we shall be considering a $G$-group $A$ and a $G$-set $E$
on which $A$ operates; for $x\in A$ and $a\in E$ we denote by $x.a$
the result of operating $x$ on $a$. We assume that the action of $A$
on $E$ satisfies the condition $^{s}(x.a) = s_{x.}s_a$ for $s \in
G$, $x\in A$ and $a \in E$. If $E,F$ are $G$-sets and $f:E\rightarrow F$ is
a map of sets not necessarily of $G$-sets we define the map $s_f : E
\rightarrow F$ for $s \in G$ by the rule $(^{s}f)(a) =
~^{s} (f(s^{s^{-1}} a))$ so that $(^{s}f)(^{s}a) = ~^{s}(f(a))$; taking
$F = E$ this definition makes Aut $E$ (the group of bijections of $E$
onto itself as a set) into a $G$-group. Associating to $x\in A$ the
automorphism of $E$ defined by the action of $x$ on $E$ we get a
$G$-homomorphism $A\rightarrow Aut~ E$ which induces a mapping of
cohomologies 
$$ 
H^i(G,A) \rightarrow H^i(G,Aut~ E). 
$$

If\pageoriginale $f: E \rightarrow F$ is a bijection of $G$-sets then
$a_s = f^{-1} \circ {}^{s}f$ is a 1-cocycle of $G$ in $Aut~E$;
changing $f$ through and 
automorphism of $E$ we get an equivalent cocycle. Hence any bijection
$f:E \rightarrow F$ modulo automorphism of $E$ defines an element of
$H^1(G,Aut~E)$. Moreover if $E$ and $F$ carry additional structures and
$f$ preserves these then so does $a_s$. Conversely starting with a
$G$-group $A$, a $G$-set $E$ on which $A$ operates $G$-compatibly and a
1-cocycle $(a_s)$ with values in $A$ we can construct a $G$-set $F$
and a bijection $f:E \rightarrow F$ such that if $b_s$ denotes the
image of $a_s$ under the mapping $A\rightarrow Aut ~E$ then $b_s =
f^{-1} \circ {}^{s}f$. To do this we take $F$ to be a copy of $E$ with a
bijection $f:E\rightarrow F$ namely the identity and define the
operation of $G$ 
on $F$ by ${}^{s}(f(x)) = f(a_s {}^{s}x)$ for $x\in E$ and $s \in G$;
with this 
operation $F$ is a $G$-set and $F$ together with the mapping
$f:E\rightarrow F$ solves our problem. We then say that $F$ is
obtained from $E$ by twisting with the cocycle $(a_s)$ and denote it
by ${}_{a}E$; replacing $a_s$ by an equivalent cocycle changes $f$ by an
automorphism of $E$. If $E$ has in addition algebraic structures and
$a_s$ preserves these then the twisted set ${}_{a}E$ will carry the same
algebraic structures. In particular taking $E = A$ and operating $A$
on itself by inner automorphisms we get a twisted group $_{a}A$; again
if $A$ carries additional algebraic structures and $a_s$ preserves
these then $_{a}A$ will carry the same algebraic structures. 

\begin{example*}
 (\cite{keyS1} X \S 2, \cite{keyS2} III \S 1.1). Let
  us consider the 
  universal domain  $\Omega$ in the sense of algebraic geometry and
  select a ground\pageoriginale field $K$; let $V$ be a
  $\Omega$-vector space. We 
  say that $V$ is defined over $K$ if there is given a $K$-subspace
  $V_K$ of $V$ such that $V \cong V_K \otimes_K \Omega$. As in
  algebraic geometry we shall say that $V_K$ is the space of
  $K$-rational points on $V$. Let $L$ be a finite Galois extension of
  $K$ with Galois group $g_{L/K}$; then $g_{L/K}$ acts on $V_L =
  V_{K}\otimes_K L$ by the rule $^{s}(a\otimes x) = a\otimes^s x$ if
  $a\in V_K$, $x\in L$ and $s\in g_{L/K}$. Let $(Aut~V)_L$ denote the
  group of $L$-linear automorphisms of $V_L$. Suppose $(a_s)$ is a
  1-cocycle of $g_{L/K}$ in $(Aut~V)_L$. We can then twist the
  vector space $V_L$ by the 1-cocycle $a=(a_s)$ to get a new
  $L$-vector space, say $V'$ with the same underlying set as $V_L$. If
  now $W$ is the fixed space of $V'$ under the twisted action of
  $g_{L/K}$ then it is well known that $W$ is a $K$-space and that
  $V'\cong W \otimes_K L$. Suppose $t,t',\ldots$ are given tensors
in the tensor space $T(V_K)$ of $V_K$. If these tensors are invariant
under the canonical extensions of all the $a_s$ to $T(V_L)\cong T(V_K)
  \otimes_K L$, denoting by $i:V\rightarrow V'$ the identity map and
  its extension to $T(V)$ the tensors $i(t), i(t')\ldots$ are then
  invariant under the twisted action of $g_{L/K}$ so that they belong
  to $T(W)$. This shows for example that if we have a hermitian or
  quadratic form on $V_L$ defined over $K$ and that if they are
  invariant under all the automorphisms $a_s$ then the twisted space
  $V'$ will also carry a hermitian or quadratic form defined over
  $K$. Again if $V$ is an algebra with involution and if all the
  automorphisms $a_s$ preserve the algebra structure and the
  involution then the twisted space will be an algebra with
  involution. 

Let\pageoriginale $A$ be a $G$-group, $(a_s)$ a 1-cocycle of $G$ in
$A$. 
\end{example*}

\begin{lemma*}
There exists a bijection of $H^1(G,_{a}A)$ onto $H^1(G,A)$ which maps
the class of (1) onto the class $(a_s)$. 
\end{lemma*}

\begin{proof}%proof
Identify the set $_{a}A$ with $A$ by means of the bijection $f: A
\rightarrow_{a}A$ and map the 1-cocycle $(b_s)$ of $G$ in $A$ onto
the 1-cocycle $(b_s,a_s)$ of $G$ in $A$. This gives a mapping of
$H^1(G,_{a}A)$ into $H^1(G,A)$ which takes the distinguished element
of $H^1(G,_{a}A)$ onto the cocycle $(a_s)$. This mapping has an
inverse defined by $(c_s)\rightarrow (c_s.a^{-1}_s)$ where $c_s$ is a
1-cocycle of $G$ in $A$. This proves the lemma. 
\end{proof}

This can be seen in another way as follows: Let $F$ be a set with some
structure whose automorphism group is $A$. The twisted group $_{a}A$
acts naturally on the twisted set $_{a}F$. Moreover $_{a}A$ will be
the automorphism group of $_{a}F$. Hence the elements of
$H^1(G,_{a}A)$ correspond bijectively with $G$-isomorphism classes of
$G$-sets $E$ (with structure) for which there is an isomorphism $h:
_{a}F\rightarrow E$; let $g : F \rightarrow_{a}F$ be the bijection
corresponding to $a$. Then $h~o~g : E \rightarrow E$ will determine an
element of $H^1(G,A)$, the corresponding cocycle will be
$(h o g)^{-1} o^s(h o g) = g^{-1} o h^{-1} o{}^{s}h o {}^{s}g =
g^{-1} o b_s  {}^{s}g = f^{-1}(b_s) \cdot a_s$ where $(b_s)$ is the
1-cocycle of $G$ in $^{a}A$  corresponding to $h: _{a}F \rightarrow
F$ and $f$ is the bijection $f:A \rightarrow {}_{a}A$ corresponding to
the twisting by $a$. This process in naturally reversible and so we get
a bijection of $H^1(G,{}_{a}A)$ onto $H^1(G,A)$. Evidently the
distinguished element (1) of $H^1(G,{}_{a}A)$ goes onto $(a_s)$ under
this mapping. 

Let\pageoriginale $A,B$ be $G$-groups and $g:A \rightarrow B$ be a
G-homomorphism. If $(a_s)$ is a $1$-cocycle of $G$ in $A$ define $b_s
= g(a_s)$; then $(b_s)$ is a $1$-cocycle of $G$ in $B$ and $g$ induces
a $G$-homomorphism also denoted by $g$ of $a^A$ into $b^B$. The
commutative diagram 
\[
\xymatrix@R=1.6cm@C=1.6cm{
A \ar[r]^{g} \ar[d]_f & B \ar[d]\\
{}_a A  \ar[r]_g 
& {}_b B
}
\]
gives rise to the following commutative diagram:
\[
\xymatrix@R=1.5cm@C=1.5cm{
H^1(G,A) \ar[r] \ar[d] & H^1 (G,B) \ar[d] \\
H^1 (G,{}_a A) \ar[r] & H^1 (G,{}_b B)
}
\]

\section{Exact Sequences}\label{chap1:sec1.5}

In what follows $A,B,C$ will be $G$-groups and homomorphisms will be
G-homomorphisms 
\begin{enumerate}[a)]
\item Let $A \rightarrow B$ be a monomorphism of $G$-groups. Let $B/A$
  be the homogeneous space of left cosets of $B$ in $A$; this is a
  $G$-set and one can define $H^0(G,B/A)$. Given an element in
  $H^0(G,B/A)$ choose a representative $b$ of it in $B$; define $a_s =
  b^{-1s}b$; then $a_s\in A$ and $(a_s)$ is a 1-cocycle of $G$ in
  $A$; moreover $(a_s)$ depends only on the element of $H^o(G,B/A)$
  under consideration and not on the particular representative
  chosen. In this way we get a map $\delta : H^0(G,B/A) \rightarrow
  H^1(G,A)$.\pageoriginale Then the following sequence with maps the
  natural ones and $\delta$ as defined above, is exact:  
{\fontsize{10}{12}\selectfont
$$
1\rightarrow H^o(G,A) \rightarrow H^o(G,B) \rightarrow
H^o(G,B/A)\overset{\delta}{\rightarrow} H^1(G,A) \rightarrow
H^1(G,B). 
$$}

\item Let $A$ be a normal subgroup of $B$ and let $C = B/A$; then $C$
  is a $G$-group and the exact sequence $1 \rightarrow A \rightarrow B
  \rightarrow C \rightarrow 1$ gives rise to an exact cohomology
  sequence 
\begin{align*}
1 \rightarrow H^o(G,A) &\rightarrow H^o(G,B) \rightarrow H^o(G,C)
\rightarrow H^1(G,A)\\
& \rightarrow H^1(G,B) \rightarrow H^1(G,C) 
\end{align*}
here the definition of $\delta$ is the same as before and all other
maps are natural ones. 

\item Let $A$ be a subgroup of the centre of $B$ so that $H^2(G,A)$ is
  defined. As in $b)$ we let $C = B/A$. We can then define a map
  $\partial : H^1(G,G) \rightarrow H^2(G,A)$ which will make the
  following sequence exact: 
\begin{align*}
1 \rightarrow H^o(G,A) &\rightarrow H^o(G,B) \rightarrow
H^o(G,C)\overset{\delta}{\rightarrow}H^1(G,A)\rightarrow H^1(G,B)\\
&\rightarrow H^1(G,C) \overset{\partial}{\rightarrow} H^2(G,A). 
\end{align*}
\end{enumerate}

\noindent
The map $\delta$ is the same as in $a)$ and the maps other than
$\delta$, $\partial$ are the natural ones. The definition of $\partial$
is as follows: let $c = (c_s) \in H^1(G,C)$ be given; lift $c$ to a
mapping $b:G\rightarrow B$ (if $G$ is profinite this lift can be
chosen to be continuous which we assume is done); define $a_{s,t} =
b_s {}^{s}b_t b^{-1}_{st}$; then $a_{s,t} \in A$ and is a 2-cocycle of
$G$ in $A$ whose class is by definition $\partial c$. 

The proofs of the above exact sequences and the propositions below
will be found in \cite{keyS2} $I \S ~5$. 

Let\pageoriginale $B$ be a $G$-group and $A$ a $G-$subgroup of $B$; the
injection of $A$ in $B$ gives rise to a  map $ H^1 (G,A) \rightarrow
H^1 (G,B)$; let $ (b_s)\in H^1 (G,B)$ be given. Then we have the
following proposition 

\begin{proposition}\label{chap1:prop1}%pro 1
In order that $(b_s)$ may belong to the image of $H^1 (G,A)$ under the
above map it is necessary and sufficient that the twisted homogeneous
space ${}_b(B/A)$ has an element invariant under $G$. 
\end{proposition}

Suppose $B$ is a $G-$group, $A$ a $G-$subgroup of $B$ contained in the
center of $B$; then we have the exact sequence 
$1 \rightarrow A\rightarrow B \rightarrow C \rightarrow 1$, where $C
= B/A$; let $(a_s)$ be a 1-cocycle of $G$ in $C$. The group $C$
operates on $B$ through inner automorphisms by a system of
representatives of $B/A$. Hence we can  twist both $C$ and $B$ by the
cocycle $(a_s)$; the twisted group  $_{a}A$ will be $A$ itself since
$A$ is central. The sequence  
$1 \rightarrow A \rightarrow _{a}B \rightarrow _{a}C \rightarrow 1$ is
then exact. 


\section{Galois Cohomology}\label{chap1:sec1.6}

Let $A$ be an algebraic variety defined over a field $K$; let $L/K$
be a Galois extension finite or infinite. If $A_{L}$ denotes  the set
of points of $A$ rational over $L$ then the Galois group  $g_{L/K}$ of
$L/K$ acts on $A_{L}$; this  action moreover is continuous, since
any $L-$rational point of $A$ generates a finite extension of $K$ and
so $A_{L} = \cup A_{M}$, the union being taken over the set of subfields
$M$ of $L$ containing $K$ such that $[M:K]< \infty$. If $A$ is an
algebraic group then since group multiplication is a morphism defined
over $K$, the set $A_{L}$ is\pageoriginale a group; we are interested
in the study 
of $H^{i} (g_{L/K},A_{L})$  which is also denoted by $H^{i} (L/K,A)$
or by $H^{i} (K,A)$ if $L$ is the separable  closure $K_{s}$ of
$K$. We shall be dealing only with fields of characteristic 0 so
that $K_s = \bar{K}$, the algebraic closure of $K$. Obviously $H^0
(L/K,A) = A_{K}$. If $A,B,C$  are algebraic groups  defined over
$K$ and if we have morphisms $f:A \rightarrow B$, $g:B
\rightarrow C$ defined over $K$ then we shall say that $1 \rightarrow A
\overset{f}{\rightarrow} B \overset{g}{\rightarrow} C \rightarrow 1$
is an exact sequence if the following sequence $1 \rightarrow
A_{\bar{K}} \rightarrow B_{\bar{K}} \rightarrow C_{\bar{K}}
\rightarrow 1$ induced by it is exact in the usual sense. One should
note that this is not a good definition in the case of characteristic
$\neq 0$. Suppose $1 \rightarrow A \rightarrow B \rightarrow C
\rightarrow 1$ is a sequence of morphisms of algebraic groups and that
the induced sequence   
$1 \rightarrow A_{L} \rightarrow B_{L} \rightarrow C_{L} \rightarrow
1$ is exact. Then we get an exact sequence  $1 \rightarrow A_{K}
\rightarrow B_{K} \rightarrow C_{K} \rightarrow H^{1} (L/K,B)
\rightarrow H^{1} (L/K,C)$. In particular if $1 \rightarrow A
\rightarrow B \rightarrow C \rightarrow 1$ is exact then the sequence
$1 \rightarrow A_{\bar{K}} \rightarrow B_{\bar{K}} \rightarrow
C_{\bar{K}} \rightarrow H^1 (K,A) \rightarrow H^{1} (K,B) \rightarrow
H^{1} (K,C)$ is exact. 

Let $A,B$ be algebraic varieties defined over $\bar{K}$ and let $f: A
\rightarrow B$ be an isomorphism defined over $L$. Then with the
usual notations $a_s = f^{-1} o^{s}f$ is a 1-cocycle of $G$ in the
group (Aut $A)_{L}$ of automorphisms of $A$ defined over $L$. Suppose
we fix $A$. An algebraic variety $B$ defined over $K$ and isomorphic
to $A$ over $L$ is called  a $L/K$-form  of $A$, or simply a
$K$-form. We have seen that any 
$L/K$-form of $A$ determines a 1-cocycle of $g_{L/K}$ in (Aut
$A)_{L}$. If two 


\vfill

\centerline{\LARGE Missing page}\pageoriginale

\vfill


\noindent
compatible\pageoriginale with the injection $A_{L}
\rightarrow A_{L}$, and so we get an induced map $H^1 (L/K,A)
\rightarrow H^1 
(L'/K'A)$. If now $L$ is the separable closure of $K$, $L'$ that of $K'$
there  exists an  injection $L \rightarrow L'$ and again $L$ can be
assumed to be contained in $L'$. By what precedes we then get a map
$H^1 (K,A) \rightarrow H^1 (K', A)$; it can be proved that this mapping
is independent of the particular imbedding of $L$ in $L'$ chosen
\cite{keyS1} $X \S ~ 4$, \cite{keyS2} $ II  ~\S ~ 1.1$. 


\section{Three Examples}\label{chap1:sec1.7}

\begin{exam}\label{chap1:exam1}%% 1
Let $A$ be a finite dimensional $K$-algebra; for any extension $L$ of
$K$ we denote by $ aA^{*}_{L}$ the group of units of the $L$-algebra
$A_L = A \otimes_{K} L$; if $L/K$ is Galois then $g_{L/K}$ acts on $A
\otimes_{K}L$ and hence it acts it acts on $A_{L}^{*}$. We claim that
$H^1 (L/K,A^{*}_{L}) = 1$. It is enough to give the proof for finite
Galois extensions, since $H^1 (L/K, A^{*}_{L}) =\lim
\limits_{\overset{\rightarrow}{M}} H^1 (M/K, A^{*}_{M})$ the inductive
limit is with respect to  inflation mappings and $M$ runs through
finite Galois  extensions of $K$ contained  in $L$. We treat  $A$ as a
right $A$-module  and consider the $L/K$ forms of $A$; the $L/K$ forms
are right $A$-modules  $B$ of finite dimension over $K$ such that
$A_{L} \cong B_{L} $ this being a right $A_{L}$-module is given by
left multiplication by an element of $A^{*}_{L}$ we know from \ref{chap1:sec1.6}
that $H^1 (L/K, A^{*}_{L})$ is bijective with the $K$-isomorphism
classes of $L/K$ forms  of the right $A$-module  $A$; hence we have
only to verify that any $L/K$ form\pageoriginale $B$ is isomorphic to
$A$ over $K$ 
i.e. $B_{K} \cong A_{K} =A$. Now  $B_{L} \cong B_{K} \otimes_{K} L
\cong A_{L}$ as right $A_{L}$-modules. The $A_{l}$-isomorphism $A_{L}
\cong B_{L}$ being also an $A_{K}$-module isomorphism we get the
$A_{K}$-isomorphism [$L:K$]$A_{K} \cong$[$L:B$]$B_{K}$[$L:K$] $A_{K}$
stands for the direct sum of [$L:K$] copies of $A_{K})$. Since
$A_{K}$ and $B_K$ are  Artinian $A_{K}$-modules Krull -Schmidt
theorem applies so that $A_{K}$ and $B_{K}$ must have isomorphic
in decomposable components. Hence $A_{K} \cong B_{K}$ as
$A_{K}$-modules. 
\end{exam}

\begin{exam}\label{chap1:exam2}
Let $K$ be a field and $A=K^{n}$, the direct sum of $n$ copies of $K$
considered as a $K$-algebra; the only $\bar{K}$-algebra automorphisms
of $A_{\bar{K}}$ correspond to permuting the components so that $(Aut
A)_{\bar{K}}= \gamma_{n}$  the symmetric group  on $n$ symbols;  the
action of $g_{\bar{K}/K}$ is trivial on $(Aut A)_{\bar{k}}$ so that
any $1$-cocycle of $g_{\bar{K}/K}$ in  $(Aut A)_{\bar{k}}$ is actually
a group homomorphism of $g_{\bar{K}/K}$ into $\gamma_{n}$. We contend
that  $H^1 (K,\gamma_{n})$ is bijective with the isomorphism classes
of commutative separable $K$-algebras of degree $n$. Because we noted
in \ref{chap1:sec1.6} that  $H^1 (K, Aut A)$ is isomorphic to the set of
$K$-isomorphism classes of $K$-forms of $A$, we have only to prove
that $K$-forms of the algebra  $A=K^n$ are exactly the commutative
separable $K$-algebras of degree $n$, $i.e$. that a commutative
$K$-algebra $B$ of degree $n$ is  separable if and only if $B \otimes
\bar{K} \cong \bar{K}^{n}$; but this is well known (\cite{keyB} Chap.~$8)$.  
\end{exam}

\begin{exam}\label{chap1:exam3}
Let $\mathscr{G}$ be a non-degenerate quadratic form on a finite
dimensional $K$-vector space $V$; then $\mathscr{G}$ corresponds to a
tensor of type $(2,0)$. In this case for any extension $L$ of $	K,
(Aut V)_{L}$ i.e. the\pageoriginale group of $L$-linear automorphisms
of $V$ fixing 
the tensor in the notation of \ref{chap1:sec1.6} is just the orthogonal group of
$\mathscr{G}$ considered as quadratic form over $V_{L}$; let us denote
the  latter by $o (\mathscr{G})_{L}$. $o (\mathscr{G})_{L}$.  is an
algebraic group defined over $K$. By the considerations of \ref{chap1:sec1.6} for
any Galois extension $L/K$, $H^1 (L/K,O (\mathscr{G}_{L}))$ is bijective
with the $K$-equivalent classes of quadratic forms which become
isometric to $\mathscr{G}$  when we extend the scalars to $L$.Over the
algebraic closure $\bar{K}$ of  $K$ any quadratic form $Q$ has the
orthogonal splitting $V_{\bar{K}} = \bar{K}x_1 \perp \bar{K}x_2 \perp
\ldots  \perp \bar{K}x_{n}$ with $Q (x_i) = 1$ for all $i$; this is
because  if we take any orthogonal splitting $V_{\bar{K}} = \bar{K}y_1
\perp \bar{K} y_2 + \cdots +  \bar{K} y_{n}$ with $Q(y_i) =
\alpha_i$ then $(\dfrac{y_1}{\sqrt{\alpha_1}},
\dfrac{y_2}{\sqrt{\alpha_2}},\ldots, \dfrac{y_n}{\sqrt{\alpha_n}})$
will be an orthogonal $\bar{K}$-basis with the desired
properties. Hence any two quadratic forms on $V_{\bar{K}}$ are
equivalent. This shows that $H^1 (K,O (\mathscr{G}))$  is just the set
of $K$-equivalent classes of quadratic forms. Again any nondegenerate
skew-symmetric bilinear form on $V$ is given by a tensor of type
$(2,0)$ and the corresponding algebraic group is the symplectic  group
$S_{p}$; here the dimension of $V$ must be even since the form is
assumed to be non-degenerate; let $\dim V = 2n$; then it is well known
that the matrix of the form can be brought to the form
$\begin{pmatrix}0 &I_n\\-I_n&0 \end{pmatrix}$ by a change  of
basis. Hence any two non-degenerate skew-symmetric bilinear forms on
$V$ are $K$-equivalent. This implies that $H^1 (K,Sp) = (1)$ or  $H^1
(L/K,Sp) = (1)$ for any Galois extension $L$ of $K$. 
\end{exam}

