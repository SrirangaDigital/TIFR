\thispagestyle{empty}

\begin{center}
{\Large\bf Lectures On}\\[5pt]
{\Large\bf Galois Cohomology of Classical}
{\Large\bf Groups}
\vskip 1cm


{\bf By}
\medskip

{\large\bf M. Kneser}
\vfill

{\bf Tata Institute of Fundamental Research, Bombay}
\medskip

{\bf 1969}
\end{center}
\eject


\thispagestyle{empty}
\begin{center}
{\Large\bf Lectures On}\\[5pt]
{\Large\bf Galois Cohomology of Classical}
{\Large\bf Groups}
\vskip 1cm


{\bf By}
\medskip

{\large\bf M. Kneser}
\vfill

{\bf Notes by}
\medskip

{\large\bf P. Jothilingam}
\vfill

\parbox{0.7\textwidth}{No part of this book may be reproduced
in any form by print, microfilm or any
other means without written permission
from the Tata Institute of Fundamental
Research, Colaba, Bombay 5}
\vfill

{\bf Tata Institute of Fundamental Research}

{\bf Bombay}

{\bf 1969}
\end{center}
\eject

\chapter{Preface}


These notes reproduce the contents of lectures given at the Tata
Institute in January and February 1967, with some details added which
had not been given in the lectures. The main result is the Hasse
principle for the one-dimensional Galois cohomology of simply
connected classical groups over number fields. For most groups, this
result is closely related to other types of Hasse principle. Some of
these are well known, in particular those for quadratic forms. Two
less well known cases are: 

i) Hermitian forms over a division algebra
with an involution of the second kind; here the result is connected
with (but not equivalent to) a theorem of Landherr. The simplified
proof of Landherr's theorem, given in \S 5.5, has been obtained
independently by T. Springer;

 ii) Skew-hermitian forms over a quaternion division algebra; here a
 proof by T. Springer, different from the one given in the lectures in
 \S 5.10, is reproduced as an appendix.

I wish to thank the Tata Institute for its hospitality and
P. Jothilingam for taking notes and filling in some of the details.
\bigskip


~\hfill{\large\bf Martin Kneser}
