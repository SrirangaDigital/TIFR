\chapter{Modular Forms of degree \texorpdfstring{$n$}{n} and differential equations}%%% chapter 15  

Let\pageoriginale $Z=(z_{\mu \nu})$ and $\bar{Z}=(\bar{z}_{\mu \nu})$
and consider  
the elements $z_{\mu \nu}$, $\bar{z}_{\mu \nu}$ as independent complex
variables. We shall require however that $z_{\mu \nu}+\bar{z}_{\mu
  \nu}$ and $i (z_{\mu \nu}$, $\bar{Z}_{\mu \nu})$ are real. Let $\alpha
, \beta $ be arbitrary complex numbers. By
$|CZ+D|^{-\alpha}$ and  $|C\bar{Z}+D|^{- \beta}$ where $Z \in
\mathscr{Y}$, and $(C,D)$ represents the second matrix row of a
symplectic substitution we always understand the functions $e^{-\alpha
  log |CZ+D|}$ and $e^{-\alpha \log |C \bar{Z}+D|}$ with the principal
value for the logarithm defined by  
$$
\log z = \log |z| + i \arg  z, \log |z| \text { real },-\pi < \arg z
\leq \pi 
$$
for complex numbers $z \neq o$

We now ask for differential operators $\Omega_{\alpha \beta}$ which
annihilate the\break Eisenstein series 
\begin{equation*}
g(Z,\bar{Z},\alpha,\beta)=\sum_{C,D} h(C,D)|CD+D|^{-\alpha} |C
\bar{Z}+D|^{-\beta} \tag{270}\label{eq270}  
 \end{equation*}
 for an arbitrary choice of the constants $h (C,D)$ where $(C,D)$
 denotes the second matrix row of a modular substitution or more
 generally of a symplectic substitution of degree, $n$. Thus we have
 to require that  
\begin{equation*}
\Omega_{\alpha \beta} |CZ+D|^{- \alpha}|C \bar{Z}+D|^{- \beta}=o
\tag{271}\label{eq271}   
\end{equation*}
for all pairs $(C,D)$ such that rank $(C,D)=n$, $CD'=DC'$ 

We shall also demand that the manifold of the functions  satisfying
\begin{equation*}
\Omega_{\alpha \beta}f (Z,\bar{Z})=o \tag{272}\label{eq272}  
\end{equation*}
is invariant\pageoriginale relative to the transformations
\begin{equation*}
f(Z,\bar{Z}) \to f(Z,\bar{Z})|M=|CZ+D|^{-\alpha}|C\bar{Z}+D|^{-
  \beta}f(M<Z>,M<\bar{Z}>) \tag{273}\label{eq273}   
\end{equation*}
for any $M=\begin{pmatrix}A & B \\ C & D \end {pmatrix} \in S$ In
other words if $f (z,\bar{z})$ is a solution of (\ref{eq271}) so is $f
(z,\bar{z})|M$. In view of this requirement, (\ref{eq272}) now takes a
particularly simple form. Prima facie, there is no loss of generality
if we assume (\ref{eq271}) to hold only for pairs $(C,D)$ with $|C| \neq O$
as the sub-manifold defined by $|C|=0$ in the manifold of all
$(C,D)'s$ occurring in (\ref{eq271}) is one of lower dimension. Then with
$P=C^{-1}_1 D_1$ we have $P=P'$ and (\ref{eq271}) is equivalent with 
\begin{equation*} 
\Omega_{\alpha \beta}|Z+P|^{-\alpha}| \bar{Z}+P|^{-\beta} =o
\tag{271}\label{eq271'}   
\end{equation*}

If we further assume that (\ref{eq272}) is left invariant by the
transformations (\ref{eq273}) as we did, and in particular by the
transformations 

 $f(z,\bar{z}) \to f (z,\bar{z})|M$ where $M= \begin{pmatrix} E & -P
  \\ o & E \end{pmatrix}$ then (\ref{eq271}) can be further simplified into  
 \begin{equation*}
\Omega_{\alpha \beta}|z|^{-\alpha}|\bar{z}|^{-\beta}=o \tag{274}\label{eq274}  
 \end{equation*} 
 and this is the equivalent form of (\ref{eq271}) we sought for.
 
 In order to construct the operators $\Omega_{\alpha \beta}$ with the
 desired properties we introduce the matrix operators 
 \begin{equation*}
\frac{\partial}{\partial z}=(e_{\mu \nu}\frac{\partial}{\partial
  z_{\mu \nu}}),\frac{\partial}{\partial \bar{z}}=(e_{\mu
  \nu}\frac{\partial}{\partial \bar{z}_{\mu \nu}}) \tag{275}\label{eq275}   
 \end{equation*} 
 with $e_{\mu \nu }=\dfrac{1}{2}(1+\delta_{\mu \nu})$ and  
 \begin{equation*}
K_\alpha = \alpha  E+(z-\bar{z})\frac{\partial}{\partial z}
\Lambda_\beta =-\beta E+(z-\bar{z})\partial/\partial \bar{z}
\tag{276}\label{eq276}    
 \end{equation*}
 Then\pageoriginale
 \begin{equation*}
\Omega_{\alpha \beta} = \Lambda_{\beta- \frac{1}{2}(n+1)}k_{\alpha +
  \alpha (\beta- \frac{1}{2}(n+1)) E} \tag{277}\label{eq277}   
  \end{equation*}  
  has the desired properties as we shall see presently. 

  We also show that $\Omega_{\alpha \beta}$ and  
  \begin{equation*}
\tilde{\Omega}_{\alpha \beta}=K_{\alpha -
  \frac{1}{2}(n+m)}\Lambda_\beta + \beta (d - \frac{1}{2}(n+m))E
\tag{278}\label{eq278}   
  \end{equation*}  
  annihilate the same functions.
  
  We first prove the following formulae.
  \begin{equation*}
\begin{cases}
\Omega_{\alpha \beta}=(Z-\bar{Z})((Z,\bar{Z})\frac{\partial}{\partial
  \bar{Z}})'\frac{\partial}{\partial Z}+ \alpha
(Z-\bar{Z})\frac{\partial}{\partial \bar{Z}}-\beta
(Z,\bar{Z})\frac{\partial}{\partial Z}\\ 
\tilde{\Omega}_{\alpha
  \beta}=(Z-\bar{Z})((Z,\bar{Z})\frac{\partial}{\partial
  Z})'\frac{\partial}{\partial \bar{Z}}+ \alpha
(Z-\bar{Z})\frac{\partial}{\partial \bar{Z}}-\beta
(Z,\bar{Z})\frac{\partial}{\partial Z} \tag{279}\label{eq279}   
\end{cases}
  \end{equation*}
  
  We may remark here concerning the use of $\dfrac{\partial}{\partial
    Z}f(Z,\bar{Z})=(\dfrac{\partial}{\partial z}_{\mu \nu}f)$ that as
  an operator $\dfrac{\partial}{\partial z}_{\mu \nu}f=f
  \dfrac{\partial}{\partial z}_{\mu \nu}+\dfrac{\partial f}{\partial
    z}_{\mu \nu}$ while in other places $\dfrac{\partial}{\partial
    z}_{\mu \nu}f=\dfrac{\partial f}{\partial z}_{\mu \nu}$. The
  meaning of the symbol will be clear from the context. 
  
  We take up the proof of (\ref{eq279}). We first need to establish the
  identities 
  \begin{align*}
\frac{\partial}{\partial z}(z-\bar{z}) & = \frac{n+1}{2}E
+((z-\bar{z})\frac{\partial}{\partial z})'\\ 
\frac{\partial}{\partial \bar{z}}(z-\bar{z}) & = -\frac{n+1}{2}E
+((z-\bar{z})\frac{\partial}{\partial \bar{z}})' \tag{280}\label{eq280}   
  \end{align*}  

  Indeed by a simple transformation we have
  \begin{align*}
\frac{\partial}{\partial z}(z - \bar{z}) & = (\sum_{\varrho =1}(e_{\mu
  \varrho}\frac{\partial}{\partial z}_{\mu \varrho}(z_{\varrho
  \nu}-\bar{z}_{\varrho \nu}))\\ 
& = (\sum^n_{\varrho =1}(z_{\varrho \nu}-\bar{z}_{\varrho \nu})e_{\mu
  \varrho}\frac{\partial}{\partial z}_{\mu
  \varrho}+(\sum^n_{\varrho=1}e_{\mu \varrho} \delta_{\mu \varrho})\\ 
& = (\sum^n_{\varrho =1}(z_{ \varrho \mu}-z_{\mu \varrho })e_{\varrho
  \nu }\frac{\partial}{\partial z}_{\varrho \nu })+ \frac{n+1}{2}
\delta_{\mu \nu})\\ 
& = (\sum^n_{\varrho =1}(z_{ \mu \varrho}-\bar{z}_{\mu \varrho
})e_{\varrho \nu }\frac{\partial}{\partial z}_{\varrho \nu })+
\frac{n+1}{2} E\\ 
& = ((z-\bar{z})\frac{\partial}{\partial z})'+\frac{n+1}{2}E
\end{align*}\pageoriginale  

The other half of (\ref{eq280}) follows on similar lines.

Now consider $\Omega_{\alpha \beta}$. By means of (\ref{eq280}) we have 
\begin{align*}
\Omega_{\alpha \beta} & =
\bigg\{-(\beta-\frac{n+1}{2})E+(z-\bar{z})\frac{\partial}{\partial
  \bar{z}} \bigg\}\bigg \{ \alpha
E+(z-\bar{z})\frac{\partial}{\partial \bar{z}} \bigg\}+\alpha(\beta
-\frac{n+1}{2})E \\ 
& = (z-\bar{z})\frac{\partial}{\partial
  \bar{z}}(z-\bar{z})\frac{\partial}{\partial z}+\alpha
(z-\bar{z})\frac{\partial}{\partial \bar{z}}-(\beta -
\frac{n+1}{2})(z-\bar{z})\frac{\partial}{\partial z}\\ 
& = (z-\bar{z})((z-\bar{z})\frac{\partial}{\partial \bar{z}})'
\frac{\partial}{\partial z}+\alpha (z-\bar{z})\frac{\partial}{\partial
  \bar{z}})-\beta (z-\bar{z})\frac{\partial}{\partial \bar{z}} 
\end{align*}
which is the first part of (\ref{eq279}). The proof of the other part is
exactly similar. 

We now show that $\Omega_{\alpha \beta}$ annihilates the Eisenstein
series (\ref{eq270}). A formula proof only requires that it annihilates each
term of (\ref{eq270}) separately, in other words that (\ref{eq271}) is true. Let
$A=(a_{\mu \nu})$ be a square matrix the elements of which belong to a
commutative ring, and $A_{\mu \nu }$ the algebraic minor corresponding
to $a_{\mu \nu}$. With $\tilde{A}=(A_{\nu \mu})$ we have in general $A
\tilde{A}=|A|E$. 

Then
\begin{align*}
\frac{\partial}{\partial z}|z|^{-\alpha} & = (e_{\mu
  \nu}\frac{\partial}{\partial z}_{\mu \nu}|z|^{- \alpha}) = - \alpha
|z|^{- \alpha -1}(e_{\mu \nu}\frac{\partial}{\partial z}_{\mu
  \nu}|z|)\\ 
& = -\alpha |z|^{- \alpha-1}(z_{\mu \nu})=-\alpha |z|^{- \alpha -1}\tilde{z}
\end{align*}\pageoriginale 
$\dfrac{\partial}{\partial \bar{z}}|\bar{z}|^{- \beta}$ can be
similarly computed and we have 
\begin{equation*}
\left. \tag{281}\label{eq281}   
\begin{aligned}
\partial/_{\partial z}|z|^{- \alpha} = - \alpha |z|^{- \alpha -1}_{\tilde{z}} \\
\partial/_{\partial \bar{z}} |z|^{-\beta} = - \beta |\bar{z}|^{-\beta
  -1}{\tilde{z}} 
\end{aligned}
\right \}
\end{equation*}
Since from (\ref{eq279})

\noindent
$(z-\bar{z})^{-1} \Omega_{\alpha
  \beta}=((z-\bar{z})\partial/_{\partial \bar{z}})'
\dfrac{\partial}{\partial z}+\alpha \dfrac{\partial}{\partial
  \bar{z}}- \beta \dfrac{\partial}{\partial z}$ 
it follows by means of (\ref{eq281}) that
\begin{align*}
(z-\bar{z})^{-1} \Omega_{\alpha \beta}|z|^{- \alpha}|\bar{z}|^{-
    \beta}= & - \alpha \beta |z|^{-\alpha }|\bar{z}|^{- \beta
    -1}\tilde{z}+\alpha \beta |z|^{- \alpha
    -1_{|\bar{z}|}-\beta_{\tilde{z}}}\\ 
& - \alpha((z-\bar{z})\frac{\partial}{\partial \bar{z}})'|z|^{-
    \alpha-1}\tilde{z}|\bar{z}|^{-\beta}. 
\end{align*}

The last term on the right can be rewritten as

$-\alpha((z-\bar{z})\dfrac{\partial}{\partial \bar{z}}|z|^{-\beta})'
|z|^{-\alpha-1}\tilde{z} $ with the terms outside the parenthesis not
depending on $\bar{z}$. Using (\ref{eq281}) this is easily seen to be equal
to $\alpha \beta |z|^{-\alpha-1}|\bar{z}|^{-\beta-1}\bar{z}(z-\bar{z})
\tilde{z}$ 

Thus
\begin{align*}
(z-\bar{z})^{-1}\Omega_{\alpha \beta}|z|^{- \alpha}|\bar{z}|^{-\beta}
  & = -\alpha \beta |z|^{- \alpha}|\bar{z}|^{-\beta-1}\bar{z}+\\ 
& + \alpha \beta |z|^{-\alpha-1}|\bar{z}|^{-\beta}\tilde{z}+\\
& + \alpha \beta
  |z|^{-\alpha-1}|\bar{z}|^{-\beta-1}\tilde{Z}(z-\bar{z})\tilde{z} 
\end{align*}
and the right side is now easily seen to vanish, there by
establishing\pageoriginale (\ref{eq271}). Since
$((z-\bar{z})\dfrac{\partial}{\partial
  \bar{z}})'\dfrac{\partial}{\partial z}$ is the transpose of
$((z-\bar{z})\dfrac{\partial}{\partial z})'\dfrac{\partial}{\partial
  \bar{z}}$ we verify without difficulty that  
$$
\tilde{\Omega}_{\alpha
  \beta}=(z-\bar{z})((z-\bar{z})^{-1}\Omega_{\alpha \beta})' 
$$
and this says that $\Omega_{\alpha \beta}$ and $\tilde{\Omega}_{\alpha
  \beta}$ annihilate the same functions as we wanted them to do. 

We have still to prove the invariance of the manifold formed by the
solutions of   
\begin{equation*}
\Omega_{\alpha \beta}\mathscr {F} (z,\bar{z})=o \tag*{$(272)'$}\label{eq272'}   
\end{equation*}
under the transformations
\begin{equation*}
f(z,\bar{z}) \to f^\ast(z,\bar{z})=|cz+D|^{- \alpha}|c
\bar{z}+D|^{-\beta}f(M < z > M<\bar{z}>) \tag*{$(273)'$}\label{eq273'} 
\end{equation*}
where $M= \begin{pmatrix} A & B \\ C & D \end{pmatrix} \in
\mathcal{S}$. The proof is rather long and mainly consists in securing
the following operator identity: 
\begin{equation*}
|cz+D|^{-\alpha}|c\bar{z}+D|^{-\beta} \Omega^*_{\alpha
  \beta}|cz+D|^\alpha
|c\bar{z}+D|^{\beta}=(zc'+D')^{-1}((cz+D)\Omega'_{\alpha \beta})'
\tag{282}\label{eq282}    
\end{equation*}
where $\Omega^*_{\alpha \beta}$ is the operator which results from
$\Omega_{\alpha \beta}$ on replacing $z$, $\bar{z}$ by $Z^*=M< Z
>$, $\bar{Z}=M< \bar{z}>$ respectively. 

Assuming (\ref{eq282}) for a moment, to obtain our main result, we argue as
follows:- 

We have to show that, under (\ref{eq272}), $\Omega_{\alpha \beta}
\mathfrak{f}^*(z,\bar{z}=o$ 

This is equivalent to showing that the right side of (\ref{eq282}), or
equivalently, the left side annihilates $\mathfrak{f}^* (z,\bar{z}$. 

The\pageoriginale left side of (\ref{eq282}) applied to
$f^*(z,\bar{z}$ gives 
$$
|cz+D|^{-\alpha}|c\bar{z}+D|^{-\beta}\Omega^*_{\alpha
  \beta}f(z^*,\bar{z}^*) 
$$
and this clearly vanishes under (\ref{eq272}). This proves the desired
result. We have then only to establish (\ref{eq282}). 

By means of the general rule relating to any three square matrices
$M_1$, $M_2$, $M_3$ which are such that the elements of $M_1$. commute
with those of $M_2$, viz. 
\begin{equation*}
\left. \tag{283}\label{eq283}   
\begin{aligned}
(M_1M_2)'=M_2'M'_2,(M_1(M_2 M_3)')' & =M_2(M_1M'_3)'\\ 
\sigma(M_1M_2) & =\sigma(M_2M_1)
\end{aligned}
\right \}
\end{equation*}
It is possible to reduce the proof of (\ref{eq282}) to showing that 
\begin{equation*}
\begin{cases}
((cz+D)\frac{\partial}{\partial z})'|cz+D|^\alpha = |cz+D|^\alpha
  ((cz+D)\frac{\partial}{\partial z})'+\alpha |cz+D|^\alpha c'\\[5pt] 
((cz+D)\frac{\partial}{\partial z})'|c \bar{z}+D|^\beta = |c
  \bar{z}+D|^\beta ((c \bar{z}+D)\frac{\partial}{\partial
    \bar{z}})'+\beta |c\bar{z}+D|^\beta c'
\end{cases}\tag{284}\label{eq284}   
\end{equation*}

It suffices of course to establish the first part of (\ref{eq284}) and that
too under the assumption $|C|\neq 0$ as in the alternative case, the
corresponding $(C,D)'s$ form a sub-manifold of lower
dimension as stated earlier. 

In this case, (\ref{eq284}) can be reduced to its special form corresponding
to $C=E$, $D=o$ \qquad viz. 
$$
(z \frac{\partial}{\partial z})' |z|^\alpha (z
\frac{\partial}{\partial z})+\alpha |z|^\alpha E 
$$
or equivalently
$$
(z \frac{\partial}{\partial z})' |z|^\alpha (z
\frac{\partial}{\partial z})+\alpha |z|^\alpha E 
$$
by a\pageoriginale substitution of the form $Z \to Z+S$ with an
appropriate symmetric matrix $S$, and the last relation is immediate
by a proof analogous to that of (\ref{eq281}).  

To deduce (\ref{eq282}) as a consequence of (\ref{eq283}) and
(\ref{eq284}) one need 
first to determine the transformation properties of the operators
$\dfrac{\partial}{\partial z}$, $\dfrac{\partial}{\partial \bar{z}}$
relative to symplectic substitutions. Let $\begin{pmatrix}A & B\\C &
  D \end {pmatrix}$ denote a symplectic matrix and let 
$$
z^*=(AZ+B)(cz+D)^{-1},\bar{z}^*=(A \bar{Z}+B)(c\bar{z}+D)^{-1} 
$$

From (\ref{eq122}) we deduce that
$$
\alpha z^* =(zc'+D')^{-1}dz(cz+D)^{-1}, \alpha \bar{z}^*
=(\bar{z}c'+D')^{-1}d\bar{z}(c\bar{z}+D)^{-1} 
$$

If $f=f(z,\bar{z})$ is an arbitrary function of $z$, $\bar{z}$ its total
differential $\alpha f$ can be represented in the form 
\begin{equation*}
df=\sigma (d z \frac{\partial}{\partial z}f)+\sigma (d
\bar{z}\frac{\partial}{\partial \bar{z}}f) \tag{285}\label{eq285}    
\end{equation*}
on the one hand, and on the other,
\begin{align*}
 df& =\sigma (dz^* \dfrac{\partial}{\partial z^*}f )+\sigma(d
  \bar{z}\dfrac{\partial}{\partial \bar{z}^*},f)\\ 
& =\sigma\bigg \{dz(cz+D)^{-1}(\frac{\partial}{\partial z}f)
  (zc'+D')^{-1}\bigg \}\\
& \qquad \qquad \qquad  +\sigma\bigg
  \{d\bar{z}(c\bar{z}+D)^{-1}(\frac{\partial}{\partial \bar{z}^*}f)
  (\bar{z}c'+D')^{-1}\bigg \} 
\end{align*}

Comparing the last with (\ref{eq285}) one deduce that
\begin{align*}
\frac{\partial}{\partial z}f & = (cz+D)^{-1}(\frac{\partial}{\partial
  z^*} f) (zc'+D')^{-1}\\ 
\frac{\partial}{\partial \bar{z}}f & =
(c\bar{z}+D)^{-1}(\frac{\partial}{\partial \bar{z}^*} f)
(\bar{z}c'+D')^{-1}
\end{align*}

Consequently we have the operator identities
\begin{equation*}
\left. \tag{286}\label{eq286}   
\begin{aligned}
\frac{\partial}{\partial z^*} = (cz+D)((cz+D)(\frac{\partial}{\partial
  z})'\\ 
\frac{\partial}{\partial \bar{z}^*} =
(c\bar{z}+D)((c\bar{z}+D)(\frac{\partial}{\partial \bar{z}})' 
\end{aligned} 
\right \}
\end{equation*}\pageoriginale

With these preliminaries we take up the proof of (\ref{eq282}). For
convenience we shall denote the product $|cz+D|^\alpha
|c\bar{z}+D|^\beta$ by $\varphi$. Then by means of (\ref{eq283}),
(\ref{eq284}), (\ref{eq286}) and (\ref{eq122}) we have  

\noindent
$\Omega^*_{\alpha \beta} \alpha =\bigg
\{(z^*-\bar{z}^*)(z^*-\bar{z}^*)(\dfrac{\partial}{\partial
  \bar{z}^*})'\dfrac{\partial}{\partial Z^*}+ \alpha
(z^*-\bar{z}^*)\dfrac{\partial}{\partial \bar{z}^*}-\beta
z^*-\bar{z}^*)\dfrac{\partial}{\partial \bar{z}^*}\bigg \} $ 

{\fontsize{9pt}{11pt}\selectfont
\begin{align*}
= & (zc'+D')^{-1}(z-\bar{z})(c\bar{z}+D)^{-1}
\bigg\{(z-\bar{z})((c\bar{z}+D)\frac{\partial}{\partial
  \bar{z}})'\bigg \}((cz+D)\frac{\partial}{\partial \bar{z}})'\alpha
\\ 
& +\bigg \{
\alpha(zc'+D')^{-1}(z,\bar{z})((c\bar{z}+D)\frac{\partial}{\partial
  \bar{z}})'-\beta
(\bar{z}c' \\
& \hspace{4cm} +D'^{-1}(z-\bar{z})((cz+D)\frac{\partial}{\partial
  \bar{z}})\bigg\}\alpha \\ 
= & |cz+D|^\alpha (zc'+D')^{-1}(z,\bar{z})(c\bar{z}+D)^{-1}\\
& \hspace{2cm} \bigg \{
(z,\bar{z})((c\bar{z}+D)\frac{\partial}{\partial \bar{z}})'\bigg
\}|c\bar{z}+D|^\beta ((cz+D) \frac{\partial}{\partial z})\\  
& +\alpha |cz+D|^\alpha (zc'+D')^{-1}(z,\bar{z})(c\bar{z}+D)^{-1}
\bigg \{ (z,\bar{z})((c\bar{z}+D)\frac{\partial}{\partial
  \bar{z}})'\bigg \}|c\bar{z}+D|^\beta c'\\ 
& + \alpha \varphi
(zc'+D')^{-1}(z,\bar{z})((c\bar{z}+D)\frac{\partial}{\partial
  \bar{z}})'+\alpha \beta \varphi (zc'+D')^{-1}(z,\bar{z})c'\\ 
& - \beta \varphi
(zc'+D')^{-1}(z,\bar{z})((c\bar{z}+D)\frac{\partial}{\partial
  \bar{z}})'\\ 
& - \alpha \beta \varphi (\bar{z}c'+D')^{-1}(z,\bar{z})c'\\ 
& = \varphi (zc'+D')^{-1}(z,\bar{z})(c\bar{z}+D)^{-1}\bigg
\{(z-\bar{z})((c\bar{z}+D)\frac{\partial}{\partial \bar{z}})'\bigg
\}((cz+D)\frac{\partial}{\partial \bar{z}})'\\ 
& + \beta \varphi (zc'+D')^{-1}(z,\bar{z})(c\bar{z}+D)(c
(z-\bar{z})((c\bar{z}+D)\frac{\partial}{\partial \bar{z}})'\\ 
& + \alpha \varphi (zc'+D')^{-1}(z,\bar{z})(c\bar{z}+D)\bigg \{(c
(z-\bar{z})((c\bar{z}+D)\frac{\partial}{\partial \bar{z}})'\bigg \}\\ 
& + \alpha \beta \varphi
(zc'+D')^{-1}(z,\bar{z})((c\bar{z}+D)^{-1}c(z,\bar{z})c'\\ 
& + \alpha \psi
(zc'+D')^{-1}(z,\bar{z})((c\bar{z}+D)\frac{\partial}{\partial
  \bar{z}})' 
 + \alpha \beta \varphi (z c' + D' ) ^{-1} (z - \bar{z}) c' + .. \\
& - \beta \varphi (\bar{z} C' + D' )^{-1} (z - \bar{z}) ((c z + D )
\frac{\partial}{\partial z })' 
 -\alpha  \beta \varphi (\bar{z} C' + D' )^{-1} (z - \bar{z}) C' 
\end{align*}}\pageoriginale

Collecting the like terms together it is seen that the terms involving
$\alpha \beta \varphi $ together and to zero while those involving
$\varphi$, $\alpha \varphi$ and $\beta \varphi$ by themselves, all
survive. These terms ultimately turn out to be respectively  
\begin{gather*}
(ZC' + D')^{-1} (Z - \bar{Z}) ((z - \bar{Z}) \frac{\partial }
  {\partial z})'  (( CZ + D)\frac{\partial } {\partial z})', \\ 
\alpha \varphi  ( ZC' + D')^{-1} (z - \bar{Z}) (( CZ +
D)\frac{\partial } {\partial z})' \text { and } \\ 
- \beta \varphi (Z C' + D')^{-1} (Z - \bar{Z}) (( CZ +
D)\frac{\partial } {\partial z})'  
\end{gather*}

Thus 
$$
\Omega^*_{\alpha \beta} \varphi = \varphi
(zc'+D'^{-1},(z-\bar{z})\bigg \{((z-\bar{z})\frac{\partial}{\partial
  \bar{z}})'+\alpha ((cz+D) \frac{\partial}{\partial
  \bar{z}})'-\beta((cz+D) \frac{\partial}{\partial \bar{z}})'\bigg \} 
$$
It then follows that
$$
\bigg \{cz-\bar{z})^{-1}(zc'+D', \zeta_{\alpha \beta} \varphi \bigg
\}'=\varphi (cz+D)\bigg \{ ((z-\bar{z}) \frac {\partial}{\partial
  z}\bigg \}'+ 
$$
\begin{align*}
+ & \alpha \varphi (cz+D)\frac{\partial}{\partial \bar{z}}- \beta
\varphi (cz+D)\frac{\partial}{\partial z}\\ 
= & \varphi (cz+D) \bigg\{((z \bar{z})'\frac{\partial}{\partial z})'
\frac{\partial}{\partial z}+\alpha\frac{\partial}{\partial z}-\beta
\frac{\partial}{\partial z}\bigg \}'\\ = &\varphi
(cz+D)\bigg\{(z-\bar{z})^{-1} \Omega_{\alpha \beta}\bigg \}' 
\end{align*} 

Consequently
$$
\bigg[ (cz+D)^{-1}\bigg\{(z-\bar{z})^{-1}(zc'+D')\Omega^*_{\alpha
    \beta} \varphi \bigg \}' \bigg ]' = \varphi
(z-\bar{z})^{-1}\Omega_{\alpha \beta} 
$$
or 
$$
\bigg[ (cz+D)^{-1}\bigg\{(zc'+D')\Omega^*_{\alpha \beta} \varphi \bigg
  \}' \bigg ]' = \varphi \Omega_{\alpha \beta} 
$$

Hence\pageoriginale finally
$$
\varphi^{-1}\Omega^*_{\alpha \beta} \varphi =(zc'+D')^{-1}\bigg(
(cz+D) \Omega'_{\alpha \beta}\bigg)' 
$$ 
and this precisely is the assertion of (\ref{eq282}). The proof is now
complete. 

Let us introduce the variables $X$, $Y$ as $x=\dfrac{1}{2}(z+\bar{z})$
and $\mathcal{Y}=\dfrac{1}{2 \ell }(Z,\bar{Z}$ and define
$\dfrac{\partial}{\partial x},\dfrac{\partial}{\partial y}$ as in
(\ref{eq275}) with $Z=(z_\mu)$ replaced by $X=(x_{\mu \nu})$ and
$\mathcal{Y}=(y_{\mu \nu})$ respectively. One then obtains the
transformation formulae 
\begin{equation*}
\left. \tag{287}\label{eq287}   
\begin{aligned}
\frac{\partial}{\partial z} & =\frac{1}{2}(\frac{\partial}{\partial
  x}-\frac{\partial}{\partial \mathcal{Y}}),\frac{\partial}{\partial
  z}=\frac{1}{2}(\frac{\partial}{\partial x}+\frac{\partial}{\partial
  \mathcal{Y}}),\\ 
\frac{\partial}{\partial z}_{\mu \nu} &
=\frac{1}{2}(\frac{\partial}{\partial x}_{\mu \nu}-\ell
\frac{\partial}{\partial y}_{\mu \nu}), \frac{\partial}{\partial
  z}_{\mu \nu}=\frac{1}{2}(\frac{\partial}{\partial x}_{\mu \nu}+\ell
\frac{\partial}{\partial y}_{\mu \nu}) 
\end{aligned} 
\right \}
\end{equation*}

Consider the operator $\Delta$ defined by  
\begin{equation*}
\Delta = - \sigma (z,\bar{z})\bigg((z,\bar{z})\frac{\partial}{\partial
  \bar{z}}\bigg)' \frac{\partial}{\partial z} \tag{288}\label{eq288}    
\end{equation*}
In terms of $X$ and $Y$, $\Delta$ takes the form 
\begin{equation*}
\Delta = - \sigma \bigg\{(\mathcal{Y}(y \frac{\partial}{\partial
  \bar{x}})' \frac{\partial}{\partial x}
+\mathcal{Y}(\mathcal{Y}\frac{\partial}{\partial
  \mathcal{Y}})'\frac{\partial}{\partial \mathcal{Y}} )\bigg\}
\tag{289}\label{eq289}    
\end{equation*}

This is immediate from the relations
\begin{gather*}
- \sigma (z-\bar{z})\bigg( (z-\bar{z})\frac{\partial}{\partial
  \bar{z}}\bigg)'\frac{\partial}{\partial z}=\sigma
\mathcal{Y}\bigg(\mathcal{Y}(\frac{\partial}{\partial
  x}+\mathcal{Y}\frac{\partial}{\partial
  y})\bigg)'\bigg(\frac{\partial}{\partial x}-
\frac{\partial}{\partial \mathcal{Y}})\bigg) \\ 
= \sigma \bigg\{ \mathcal{Y} (\mathcal{Y} \frac{\partial}{\partial
  x})'\frac{\partial}{\partial x} + \mathcal{Y} (\mathcal{Y}
\frac{\partial}{\partial y})'\frac{\partial}{\partial y}\bigg \} +
\sigma \bigg \{ \mathcal{Y}(\mathcal{Y} \frac{\partial}{\partial
  y})'\frac{\partial}{\partial x}-\mathcal{Y}(\mathcal{Y}
\frac{\partial}{\partial x})\frac{\partial}{\partial y}\bigg \} 
\end{gather*}
and the fact that

$S=(\mathcal{Y}\dfrac{\partial}{\partial
  \mathcal{Y}})'\dfrac{\partial}{\partial x}( \mathcal{Y}
\dfrac{\partial}{\partial x})'\dfrac{\partial}{\partial y} $ \; is a skew
symmetric matrix so that \break $\sigma (\mathcal{Y} S) =o$ 

One\pageoriginale interesting fact about $\Delta$ is its invariance
relative to the 
symplectic substitutions. Let the substitution $Z$, $\bar{Z} \to Z^*$,
$\bar{Z}^*$ carry $\Delta$ into $\Delta^*$ where $Z^*= (AZ + B)(CL +
D)$ and $\bar{Z}^* = (A \bar{Z} + B) (C \bar{Z} + D)^{-1}$ with
$\begin{pmatrix} A  B\\C D \end{pmatrix} \in S$. We observe from
(\ref{eq122}) that  
\begin{align*}
Z^* - \bar{Z}^* & = (Z C^1 + D')^{-1} (Z, \bar{Z}) (C, \bar{Z} +
D)^{-1} \\
& = (\bar{Z}C' + D)^{-1} (Z, \bar{Z})(CZ + D)^{-1}
\tag*{$(122)'$}\label{eq122'}  
\end{align*}

Consequently, by means of (\ref{eq286}) and (\ref{eq283}) we have
\begin{align*} 
\Delta^* \mathfrak{f} & =  - \sigma (Z^* - \bar{Z}^*) \big((Z^*
\bar{Z}^*)\frac{\partial}{\partial \bar{Z}^*} \big)
\frac{\partial}{\partial Z^*} \mathfrak{f}\\ 
& = - \sigma (ZC + D)^{-1} (Z - \bar{Z}) (C \bar{Z} + D)^{-1} \left\{(C
\bar{Z} + D')^{-1} (Z - \bar{Z}) \right.\\
& \hspace{2cm} \left. \big( (C \bar{Z} + D)
\frac{\partial}{\partial \bar{Z}}\big)' \}  \times \{(CZ + D)
(\frac{\partial}{\partial z} \mathfrak{f}) (ZC' + D')  \right\}\\ 
& = - \sigma (Z- \bar{Z})(C \bar{Z} + D)^{-1} \left\{(ZC' + D')^{-1} (Z -
\bar{Z}) ((C \bar{Z} + D) \frac{\partial}{\partial \bar{z}} )'
\right\} \\
& \hspace{4.3cm} \times \left\{ (CZ + D) (\frac{\partial}{\partial z}
\mathfrak{f}) \right\}\\  
& = - \sigma(Z - \bar{Z}) \big( (ZC' + D')^{-1} (Z - \bar{Z})
\frac{\partial}{\partial Z}  \big)' (CZ + D) (\frac{\partial}{\partial
  z} \mathfrak{f})\\ 
& = - \sigma (Z- \bar{Z}) \big( (ZC ' + D')^{-1} (Z -\bar{Z})
\frac{\partial}{\partial z} \mathfrak{f} \big) = \Delta \mathfrak{f} 
\end{align*} 
and it is immediate that $\Delta^* = \Delta$

We with to identify $\Delta$ with the Laplace Beltrami operator
associated with the symplectic metric. We may recall that the Laplace
Beltrami operator in a Riemannian space with co-ordinate systems
$\chi^1, \chi^2 \cdot \chi^N$ and the fundamental metric form $ds^2 =
\sum\limits_{\mu, \nu} g_{\mu , \nu} d \chi^\mu d \chi^\nu $ is given
by $\sum\limits_{\mu, \nu } \dfrac{1}{\sqrt{g}}
\dfrac{\partial}{\partial \chi^\mu} \bigg(\sqrt{g}  g^{\mu \nu}
\dfrac{\partial}{\partial \chi^\nu} \bigg)$ where\pageoriginale $g =
g(g_{\mu \nu})$ and $(g_{\mu \nu}) (g^{\mu \nu}) = E$. With respect to
a geodesic co-ordinate system at a given point, this operator takes in
this point the simple form  
\begin{equation*}
\sum_{\mu, \nu} g^{\mu, \nu} \frac{\partial^2}{\partial \lambda^N
  \partial \chi^\nu}  \tag{290}\label{eq290}    
\end{equation*}
and can be easily computed. The Laplace Beltrami operator in any
Riemannian space is invariant relative to the movements of the
space. In particular the Laplace Beltrami operator associated with the
symplectic metric is invariant under symplectic movements. As this has
been shown to be true of $\Delta$ also, to prove the equivalence of
the two, it suffices to verify that they define the same operator at
a special point. We choose this special point to be $Z = ( E =
\bar{Z})$. At this point let us introduce the co-ordinate system $W,
\bar{W}$ by $W = (Z - 1 E) (Z + 1 E)^{-1}$, $\bar{W} = (\bar{Z} + i E )
(\bar{Z} - (E)^{-1}$ which is a geodesic one relative to the
symplectic metric. Analogous (\ref{eq286}) we have 
\begin{equation*}
\frac{\partial}{\partial Z} = \frac{-1}{Z} (W-E) \bigg( (W -E)
\frac{\partial}{\partial W} \bigg)' \frac{\partial}{\partial \bar{Z}}
= \frac{1}{2} (\bar{W} - E) \bigg((\bar{W} -E)
\frac{\partial}{\partial \bar{W}} \bigg)' \tag{291}\label{eq291}     
\end{equation*}

It terms of $W$, $\bar{W}$ we have
$$
\Delta = \sigma (E - W \bar{W}) \bigg((E - W \bar{W})
\frac{\partial}{\partial \bar{W}} \bigg)' \frac{\partial}{\partial W} 
$$

The special point $Z = i E = - \bar{Z}$ transforms into $W = 0 =
\bar{W}$ and to compute the Laplace Beltrami operator at this point,
one needs only to determine the coefficients of the fundamental metric
form  
$$
ds^2 = 4 \sigma \bigg(d W (E - W \bar{W})^{-1} d \bar{W} (E - W
\bar{W})^{-1}\bigg) 
$$
at this\pageoriginale point. Then in (\ref{eq290}) one obtains the
same operator as 
$\Delta$. We proceed to consider modular forms associated with
arbitrary sub-groups of the symplectic group. Let $G$ be a group of
symplectic substitutions and $\nu(M)$ a \textit{multiplicator system}
of $G(M \in G) \alpha$, $\beta$ being arbitrary complex numbers, a
function $\mathfrak{f} (Z, \bar{Z})$ is said to be a modular form of
the type $(G, \alpha, \beta \vartheta )$ 
$$
\text{ if }
$$
\begin{enumerate}[(i)]
\item $\mathfrak{f} (Z, \bar{Z})$ is regular in the domain where $Z +
  \bar{Z}$ is real and $\dfrac{1}{2 i} (Z - \bar{Z})$ is real and
  positive 
\item $\Omega_{\alpha \beta} \mathfrak{f} (Z, \bar{Z})= 0$ 
\item 
\begin{equation*}
 \mathfrak{f} (Z , \bar{Z} | M = \nu (M)
  \mathfrak{f} (Z, \bar{Z}), M \in G,  \tag{292}\label{eq292}     
\end{equation*} 
\end{enumerate}

Some restriction on the behaviour of $\mathfrak{f}(Z, \bar{Z})$ on the
boundary of $\mathscr{Y}$ is perhaps necessary but we assume none. We
shall denote by $\{ G, \alpha, \beta, \nu \}$ the linear space of all
modular forms of the type $(G, \alpha, \beta, \nu)$. It is an easy
consequence of our definition that for every symplectic substitution
$M$, 
\begin{equation*}
\{ G, \alpha, \beta, \nu \} |  M = \{ M^{-1} G M  \alpha, \beta, \nu^*
\}\tag{293}\label{eq293}     
\end{equation*}
with an appropriate multiplicator system $\nu^{*}$ depending on $M$,
where the left side just means the set of all $\mathfrak{f} (Z,
\bar{Z}) | M$ for $\mathfrak{f} (Z, \bar{Z}) \in \{ G, \alpha, \beta,
\nu \}$ 

We would like to treat the question, whether it is possible to set up
a correspondence between $\{ G, \alpha, \beta, \nu \}$ and $\{ G,
\alpha \pm 1, \beta \mp 1, \nu \}$. Such a correspondence, we shall
\pageoriginale see, will be defined by certain differential
operators. Consider the Eisenstein series (\ref{eq270}), viz.  
\begin{equation*}
g (Z, \bar{Z}, \alpha, \beta) = \sum_{C, D} h (C, D) | CZ + D1 ^{-
  \alpha} | C \bar{Z} + D|^{-\beta}\tag*{$(270)'$}\label{eq270'} 
\end{equation*}
and introduce differential operators $M_{\alpha}$, $N_{\beta}$ with the
properties  
\begin{equation*}
\left.
\begin{aligned}
M_\alpha g(Z, \bar{Z}, \alpha, \beta)  = \varepsilon_n (\alpha) g (Z,
\bar{Z}, \alpha +1, \beta - 1)\\ 
N_\beta g(Z, \bar{Z}, \alpha, \beta)  = \varepsilon_n (\beta) g (Z,
\bar{Z}, \alpha +1, \beta - 1) 
\end{aligned}
\right \} \tag{294}\label{eq294}    
\end{equation*}
where
\begin{equation*}
\varepsilon_n (\lambda)  = \lambda (\lambda - \frac{1}{2}) \cdots
(\lambda - \frac{n -1}{2}). \tag{295}\label{eq295}     
\end{equation*}

The truth of (\ref{eq294}) for all $g (Z, \bar{Z})$ or equivalently for all
coefficients $h (C, D)$  in \ref{eq270'} clearly requires that  
$$
M_\alpha | CZ + D|^{-\alpha} |C \bar{Z} + D|^{- \beta} = \varepsilon_n
(\alpha) |C Z + D|^{- \alpha - 1} |C \bar{Z} + D|^{- \beta + 1}, 
$$
and as in earlier contexts, it suffices to suffices to require this
for $(C, D)$ with $|C| \neq 0$. Then by a replacement of the form $Z
\to Z + S$ with a suitable symmetric matrix $S$, the above can be
reduced, assuming that $M_\alpha$ is invariant for such a replacement,
to  
$$
M_\alpha |Z|^{-\alpha} |\bar{Z}|^{ - \beta} = \varepsilon_n (\alpha)
|Z|^{-\alpha - 1} |\bar{Z}|^{- \beta + 1} 
$$

If further we are sure that in $M_\alpha$ only
$\dfrac{\partial}{\partial Z_{\mu \nu}}$ appears explicitly and it is
independent of $\dfrac{\partial}{\partial Z_{\mu \nu}}$ as will be the
case, then the last relation simplifies further into 
\begin{equation*}
M_\alpha |Z|^{- \alpha} = \varepsilon_n (\alpha) |Z|
^{- \alpha - 1 } |\bar{Z}| \tag{296}\label{eq296}    
\end{equation*}

It is\pageoriginale now our task to construct $M_\alpha$ with these
required properties, viz:-  
\begin{enumerate}[(i)] 
\item $M_\alpha$ depends only on $\dfrac{\partial}{\partial Z_{\mu
    \nu}}$ and does not involve $\dfrac{\partial}{\partial
  \bar{Z}_{\mu \nu}}$ 

\item $M_\alpha$ is invariant relative to a replacement of $Z$ by $Z+
  S$ with an arbitrary symmetric matrix $S$. \hfill (297) 

\item $M_\alpha$ satisfies (\ref{eq296})
\end{enumerate}

For arbitrary integers $I \le L_1 < L_2 <  \cdots < l_h \le n, 1 \le
k_1 < k_2 < k_h \le n $ we get  
\begin{equation*}
\left.
\begin{aligned}
\begin{pmatrix} l_1, l_2 \cdots l_h\\ k_1, k_2 \cdots
  k_h \end{pmatrix}_z & = | Z_{i_\mu k \nu}|\\[4pt] 
\begin{bmatrix} l_1 l_2 \cdots l_h\\ k_1 k_2 \cdots
  k_h \end{bmatrix}_z & = |e_{i_\mu kk_\nu} \frac{\partial}{\partial
  z_{\nu_\mu} l_\nu}|  
\end{aligned}
\right \} \tag{298}\label{eq298}    
\end{equation*}

Also let
\begin{equation*}
s_h (Z - \bar{z}, \frac{\partial}{\partial z}) = \sum_{\substack{ 1
    \le l_1 < \cdots < l_h \le n \\ 1 \le k_1 < \cdots < k_h \le
    n}} \begin{pmatrix} l_1 l_2 \cdots l_h \\ k_1 k_2 \cdots
  k_h \end{pmatrix}_{z - \bar{z}} \begin{bmatrix} l_i l_2 \cdots l_l
  \\ k_1 k_2 \cdots k_h \end{bmatrix}_z \tag{299}\label{eq299}     
\end{equation*}
for $h = 1,2, \ldots n$ and $s_o (Z - \bar{z}$,
$\dfrac{\partial}{\partial  z}) = 1$ 

Finally we introduce 
\begin{equation*}
M_\alpha \sum^n_{h = 0} \frac{\varepsilon_n (\alpha)}{\varepsilon_h
  (\alpha)} s_h (Z - \bar{Z}, \frac{\partial}{\partial z})
(\varepsilon_o (\alpha) = 1) \tag{300}\label{eq300}     
\end{equation*}

That $M_\alpha$ satisfies the first two conditions in (297) is a
matter of easy verification. Besides, in the case $n =1$, $M_\alpha$ is
identical with $K_\alpha$ introduced earlier. The proof that
$M_\alpha$ satisfies (\ref{eq296}) is a consequence of the following result 
\begin{equation*}
\begin{bmatrix} l_1 l_2 \cdots l_h \\ k_1 k_2 \cdots
  k_h \end{bmatrix}_Z |Z|^{- \alpha} = (-1)^{h} \varepsilon_h |\alpha|
|Z|^{- \alpha -1} \overline{\begin{pmatrix} l_1 l_2 \cdots l_k \\ k_1
    k_2 \ldots k_h \end{pmatrix}_Z} \tag{301}\label{eq301}      
\end{equation*}\pageoriginale
where $\overline{ \begin{pmatrix} l_1 l_2 \cdots l_h \\ k_1 k_2 \cdots
    k_l   \end{pmatrix}_Z}$ denotes the algebraic minor of
$\begin{pmatrix} l_1 l_2 \cdots l_h \\ k_1 k_2 \cdot
  k_h \end{pmatrix}_Z$. 

This minor differs from the determinant of the submatrix which arises
from $Z$ on cancelling the rows and column corresponding to the
indices \; $l_1, l_2 , \ldots l_h$ \; and \; $k_1, k_2, \ldots k_h$
\; respectively by the sign \break $(-1)^{l_1 + l_2 + k_1 + k_2 \cdots
  k_h}$. In the case $h = n$, we define the algebraic minor to be
1. The relation (\ref{eq301}) can be proved by resorting to induction of
$\mathfrak{f}$ but we leave it here for fear of the length of such a
proof. Certainly a simpler proof of (\ref{eq301}) will be desirable. 

Assuming (\ref{eq301}) we proceed to establish (\ref{eq296}). It is
immediate from (\ref{eq301}) that  
$$
s_h (Z, \bar{Z}, \frac{\partial}{\partial z}) |Z|^{- \alpha} =
(-1)^{h} \varepsilon_h (\alpha) |Z|^{d - 1} \sum_{1 \le i_1 < l_h \le
  n} D_{l_1 l_2 \ldots l_h} 
$$
where
$$
D_{l_i l_2 \cdots l_h} = \sum_{1 \le k_1 < k_2 < \cdots
  k_h} \begin{pmatrix} l_1, l_2 \ldots l_h\\ k_1 k_2 \cdots
  k_h \end{pmatrix}_{Z - \bar{Z}} \overline{ \begin{pmatrix} l_1, l_2
    \ldots l_h\\ k_1 k_2 \cdots k_h \end{pmatrix}} 
$$

By a standard development of a determinant (Laplace's decomposition
theorem) it is clear that $D_{l_1, l_2 \cdots l_2}$ is the determinant
of the $n$- rowed matrix which arises from $Z$ on replacing its
rows with the indices\break $l_1, l_2, \ldots l_2$ by the corresponding rows
of $Z - \bar{Z}$. We split up this resulting matrix into a sum of
\pageoriginale matrices each row of which is upto a constant sign a
row of either $Z$ or $\bar{Z}$, and the summands which result are
$2^h$ is number.  

Let $\Delta_{g_1 g_2 \ldots g_r}$ denote the determinant of the matrix
which arises from $Z$ on replacing its rows with the indices $g_{''}
g_2, \ldots g_n$ by the corresponding rows of $\bar{Z}$. For a given
$\Delta_{g_1 g_2 \ldots g_n}$ to appear in the above decomposition of
$D_{i_1, i_2,\ldots i_h}$ it is necessary and sufficient that the
set $(f, r_2 \ldots 
g_r)$ is a subset of the set of indices $(l_1 l_2 \ldots l_0)$ so that
a fixed determinant $\Delta_{g_1 g_2 \ldots g_r}$ occurs in the
above decomposition of $D_{l_1 l_2 \ldots l_h}$ for a fixed $h$ exactly
$\begin{pmatrix} n- r \\ h - r \end{pmatrix}$ times, and the sign
which it takes is $(-i)^r$. Consequently we have 
\begin{align*}
\sum_{1 \le i_1 < \cdots < l_h \le n} D_{i_1,i_2, \ldots u_h} & = \sum^h_{r = 0}
\sum_{1 \le r_1 <} (-1)^r \begin{pmatrix} n-n \\h -r \end{pmatrix}
\Delta_{g_1 g_2, \ldots g}\\ 
& = \sum_{r =0}^h (-1)^r \begin{pmatrix} n-n \\h -r \end{pmatrix} A_n 
\end{align*}
with
$$
A_n = \sum_{1 \le g_1 < \cdots < g_n \le n} \Delta_{y, g_2}
\varphi r > 0 \text{ and } A_o = |Z| 
$$

Clearly $A_n = |\bar{Z}|$. Then
\begin{align*}
M_\alpha |Z|^{- \alpha} & = \sum^n_{h = 0} \frac{\varepsilon_n
  (\alpha)}{\varepsilon_h (\alpha)} s_h (Z - \bar{Z} ,
\frac{\partial}{\partial z}) |Z|^{- \alpha}\\ 
& = \varepsilon_n (\alpha) |Z|^{- \alpha -1} \sum^{n}_{h  =0} (-1)^h
\sum^h_{h = 0} (-1)^h \begin{pmatrix} n-n \\h -r \end{pmatrix} A_n\\ 
& = \varepsilon_r (a) |Z|^{- \alpha - 1} \sum^n_{r = 0}
\Bigg(\sum^{n}_{h = n} (-1)^{ h - r}
\begin{pmatrix}
n-n \\
h -r 
\end{pmatrix} \Bigg)A_n 
\end{align*}

The sum\pageoriginale within the parenthesis can be rewritten as
$\sum\limits^{r - n}_{h = 0} (-1)^h \big(^{n - 1}_h)$ and this latter
sum is equal to 0 or 1 according as $n - r > 0$ or $n - r =0 \qquad
0$. The above now reduces to  
$$
M_\alpha |Z|^{- \alpha} = \varepsilon_n (\alpha) |Z|^{- \alpha - 1}
A_n = \varepsilon_n (\alpha) |Z|^{-\alpha -l} |\bar{Z}| 
$$
as was desired. The deduction of the first half of (\ref{eq294}) at this
stage is similar to the deduction from (\ref{eq274}) of the corresponding
result for (\ref{eq270}). 

We have still to introduce the operator $N_\beta$. We introduce the
operator $\lambda$ by the requirement 
\begin{equation*}
\lambda \mathfrak{f} (z, \bar{z}) = \mathfrak{f} (- \bar{z} - z)
\tag{302}\label{eq302} 
\end{equation*}
and set $N_\beta = \lambda M_\beta \lambda$. It is immediate that for
$n =1$ we  have $N_\beta = - \wedge_\beta$ while we had $M_\alpha =
K_\alpha$ in this case. It is easily seen that $N_\beta$ satisfies the
second half of (\ref{eq294}) in view of $M_\alpha$ satisfying the first
half. In general it can be conjectured that 
\begin{gather*}
(N_{\beta - 1} M_\alpha - \varepsilon_n (\alpha) \varepsilon_n (\beta
  - 1)) \{G, \alpha, \beta, \nu \} = 0\\ 
(M_{\alpha - 1} N_\beta - \varepsilon_n (\beta) \varepsilon_n (\alpha
  - 1)) \{G, \alpha, \beta, \nu \} = 0 \tag{303}\label{eq303} 
\end{gather*}
\begin{align*}
M_{\alpha} \{G, \alpha, \beta, \nu \} & \subset \{G, \alpha +1, \beta
- 1, \nu \}\\ 
N_{\beta} \{G, \alpha, \beta, \nu \} & \subset \{G, \alpha +1, \beta -
1, \nu \} 
\end{align*}

This\pageoriginale has been proved however only in the case $n =1$ and
2 and the proof makes use of certain operator identities.  
 
We turn new to a different problem, viz. that of finding a Fourier
development for modular function defined in (\ref{eq292}). Net much progress
has so has been recorded in this direction. We assume that the given
group $G$ contains symplectic substitutions of the form
$\begin{pmatrix} E & O \\ 0 & E \end{pmatrix}$ ($S$ Symmetric), say, as
is the case with $\mathcal{M}$. If $\mathfrak{f} (z, \bar{z}) \in \{G,
\alpha, \beta, \nu \}$ and $\mathfrak{f} (z, \bar{z})  = \sum\limits_T
\alpha (y , < \pi T) \in^{2 \pi i \sigma (T \chi)}$, the
summation for $T$ being over all semi-integral matrices, the
hypothesis, viz. $\Omega_{\alpha, \beta} \mathfrak{f} = 0$ implies
that $\Omega_{\alpha \beta}$ annihilates each term of the above sum.  

Thus \; $\Omega_{\alpha \beta} 0 \alpha^{z 2 \pi T} e^{\pi i \sigma (T
  \chi)}$ \; which on replacing $2 \pi T$ by $T$ gives \break $\Omega_{\alpha
  \beta} e^{i \sigma (T \chi)} a(y, T) T 0$. This represents a
system of differential equations-here we can consider $T$ to be an
arbitrary real symmetric matrix-for the functions $a(y, T)$, and
we have to determine all the solutions of this system which are
regular in $y > 0$. Only in the case $n = 2$ some real
progress has been achieved and the main result in this case is that
linear space $\{ \alpha \beta T\}$ of the solutions is finite
dimensional for $T$ such that $|T| \neq 0$. Before we give a detailed
account of the results in this case it will be useful to make the
following remarks on a special parametric representation of the
matrices $y^{(z)} > 0$ 

Every 2 X 2 positive matrix $y$ has a parametric representation
of the form
\begin{equation*}
Y = \sqrt{|y|} \begin{pmatrix} (x^2 + y^2) y^{-1}    x
  y^{-1}\\ xy^{-1}   g^1 \end{pmatrix} \tag{304}\label{eq304} 
\end{equation*}\pageoriginale
and the requirement $y > 0$ is equivalent with $y > 0$. Let $Z =
X +i y$.Or the surface $|y| = L$, the correspondence $y \to
y$ is $y -1$. This surface can be considered as a Riemannian
surface, as in $y > 0$ we have the fundamental metric form $ds^2
= \sigma (y^{-1} \tau y)^2$. Relative to this metric, the
mappings $y \to \mathcal{U} y \mathcal{U}, \mathcal{U}$
unimodular, are movements (and the quadratic form $ds^2$  is invariant
relative to these movements). The space $|y| = 1$ is nothing
else than the hyperbolic plane in this metric. For if $\omega =
\sqrt{|y|}$ it can be shown that  
\begin{equation*}
ds^2 = Z \Bigg( \frac{d \omega^2}{\omega^2} + \frac{dx^2 + dy^2} {y^2}
\Bigg) \tag{305}\label{eq305} 
\end{equation*}

 On $|y| = 1$ we have $d \omega = 0$ and then in (\ref{eq305}) one has
 clearly the fundamental metric form for the hyperbolic plane. If
 $\mathcal{U}$ is a proper unimodular matrix $( \varphi | \mathcal{U}
 | = 1)$ then $y \to \mathcal{U} y \mathcal{U}'$ and $Z \to
 \mathcal{U} < Z >$ are representations of the same hyperbolic
 movement. We shall have more to say one the determinantal surface
 $|y| = 1$ later. 

For the special case $n=2$ we are considering, we state the following
specific results. Introduce $x,y$ by (\ref{eq304}) and set
$\mathcal{U} = ( \tau (y 0))^2 - 4 |y^\tau|$ 

\textbf{Case i}
$T = 0$.

In this\pageoriginale case the solutions $a(y, 0)$ can be written as   
\begin{equation*}
a (y, 0) = \varphi (x, y)|y|^{\frac{1}{2}(1 \alpha + \beta)}
+ C_1 |y|^{\frac{3}{2} - d - \beta} + 0_2 \tag{306}\label{eq306} 
\end{equation*}
where $\varphi(x, y)$ represents an arbitrary regular solution in $y >
0$ of the wave equation 
$$
y^2 (\varphi_{\chi \chi} + \varphi_{y y}) -(\alpha + \beta - 1)
(\alpha + \beta - 2) \varphi = 0 
$$
and $C_1$, $C_2$ are arbitrary constants. This is the general solution
for $\alpha + \beta \neq \dfrac{3}{2}, 2, 1$. The singular cases
$\alpha + \beta = \dfrac{3}{2}, 2,1$ are those at which at least two
of the three exponents $\dfrac{1}{2} (1- \alpha - \beta)$, $\dfrac{3}{2}
- \alpha - \beta$, $0$ occurring in (\ref{eq306}) are equal and require some
modifications. We exclude these cases here. 

\noindent
\textbf{Case ii} : $T \ge 0$, rank $\tau = T$. 

Here we shall have
\begin{equation*}
a (y, T) = \varphi (u) i |y|^{\frac{3}{2} - \alpha - \beta}
+ \psi (u) \tag{307}\label{eq307}
\end{equation*}
where $\varphi(u)$ and $\psi (u)$, denote confluent hypergeometric
functions satisfying the differential equations 
\begin{gather*}
\mathcal{U} \varphi^{''} + (3 - \alpha - \beta) \varphi' + (\alpha -
\beta -\mathcal{U} ) \varphi = 0\\ 
\mathcal{U} \psi^{''} + (\alpha + \beta) \psi' + (\alpha - \beta - u)
\psi = 0 
\end{gather*}

We have two independent solutions each for $\mathcal{U} \vartheta$ and
the dimension of $\{\alpha, \beta T \}$ is consequently equal to 4
in this case. Here again (\ref{eq307}) is the general solution for $\alpha +
\beta \neq \dfrac{3}{2}$\pageoriginale and we exclude the exceptional
case from our considerations. We remark that the condition $T \ge 0$
is irrelevant here. We can also take $T \ge 0$ and in general,  
$$
\{ \alpha, \beta ,T \} = \{ \beta, \alpha, -T \}.
$$



\noindent
\textbf{Case iii} $T > 0$

This is the first, general case we are dealing with in so far as the
solutions $a(y, T)$ here depend on functions of more than one
variable (unlike the $\varphi$  and $\psi$ of the earlier case). We
have in this case 
\begin{equation*}
a (y, T) = \sum^{\infty}_{\nu - 0} g_x (u) \vartheta^\nu , (|\nu|
< u^2)\tag{308}\label{eq308} 
\end{equation*}
the functions $g_\nu (\mathcal{U})$ are defined recursively by 
$$ 
4 (\nu +1)^2 \mu g_{\nu + 1} + \mathcal{U} g^{''}_\nu + 2 (2 \nu +
\alpha + \beta) g^{'}_\nu + (2 \alpha - 2 \beta - u) g_{\nu} = 1
\mathcal{U} \ge 0 
$$
and 
\begin{align*}
g_o (v) & = u^{1 - \alpha - \beta} \psi(\mathcal{U}),
\psi(\mathcal{U}) = \frac{1}{u} \varphi (u)\\ 
\varphi^{''} & = \bigg(1 + \frac{2 (\beta - \alpha)}{\mathcal{U}} +
\frac{(\alpha + \beta + 1) (\alpha + \beta -2)}{\mathcal{U}^2} \bigg)
\varphi. 
\end{align*}

Every possible function $g_\nu (\mathcal{U})$ leads to a series
(\ref{eq308}) converging in the whole domain $y > 0$ so that the
dimension of $\{\alpha, \beta, T \}$ is easily seen to be 3. As in
the earlier case we can also consider $T$ with $- T > 0$ 

\noindent
\textbf{Case iv} : \textit{ rank $T = 2$, $T$ indefinite}

This is the last case and we have here
\begin{equation*}
a(y, T) = \sum_{\nu = 0}^\infty h_\nu (\alpha) u^\nu,
(u^2 < \nu)\tag{309}\label{eq309} 
\end{equation*}

The functions\pageoriginale $h_\nu (\vartheta)$ are recursively
defined by  
$$
(\nu + 2) (u + 1) h_{\nu + 2} + 4 \nu h^{''}_\nu + 4 (\alpha + \beta +
\nu) h^1_\nu - \ell_\nu  = 0, \nu \ge 0 
$$
and 
$$
8 \nu^2 h^{''}_o + 4 (\alpha + 3 \beta + 3_o) \vartheta p^{''}_o (4
(\alpha + \beta)^3 + 2 (\alpha + \beta -1) -2 \nu) h^{'}_o + 
$$
\begin{align*}
- (\alpha + \beta) h_o & = (\alpha, \beta)h_i\\
8 \nu h^{'}_{i} + (\alpha - \beta) r_j & = (\beta - \alpha) h_o
\end{align*}

Every permissible $h_o, h_1$ leads to a series (\ref{eq309}) which converges
in $y > 0$ and the dimension of $\{ \alpha, \beta, T \}$ is
four. 

It remains to study these functions $a(y, T)$. These are
generalisations of the confluent hypergeometric functions of 2
variables. We can also obtain them in the following way.  We develop
the Eisenstein series. 

\noindent
$\sum |C Z + D|^{-\alpha} |C \bar{Z} + D|^{- \beta}$ as a power
series, 
$$
\sum |C Z + D|^{-\alpha} |C \bar{Z} + D|^{- \beta} = \sum_T a (y,
2 \pi T) \in^{k \pi i \sigma (T \chi)} m
$$

If we compute the coefficients by means of the Poisson summation
formula, we obtain representations of $a(y, 2 \pi T)$ by
integrals. It is desirable to give a characterisation of these special
functions $a (y 2 \pi T) \in \{\alpha, \beta 2 \pi T \}$ and this
has to be done by studying the behaviour of these functions in
neighbourhood of the boundary of the domain $y > 0$. 

For a\pageoriginale detailed account of some of the topics treated in
this section we refer to :-  

\medskip
\noindent
 \textbf{H. Maass,} Die Differentialgleichungen in der Theorie der
 Siegelachen Modul functionen, Math. Ann. 126 (1553), 44-68. 

