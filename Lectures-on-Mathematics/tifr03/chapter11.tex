

\chapter{The representation theorem} %%%% chapter 11

Our\pageoriginale  aim in this section is to establish a result we
promised earlier 
at the end of $\S 9$, viz. that the Poincare series actually generate
all modular form of degree $n$ under suitable assumptions. As a first
step we compute $\mathscr{Y}(Z, T) | \mathscr{Y}$. Let $z^* \in
\mathscr{Y}_{n - 1}$ and $x \in \mathscr{Y}$ so that $Z
= \begin{pmatrix} Z &  0 \\ 0 & r \end{pmatrix} c \mathscr{Y}_n$. For
fixed $Z*$ let $\varphi (z) = g (Z, T)$. Then $\varphi (z)$ is a
modular form of degree $1$ and weight $k$. For, if the matrix
$\begin{pmatrix} a & b \\ c & d \end{pmatrix}$ is a modular
substitution of degree 1 and we set  
\begin{align*}
A & = 
\begin{pmatrix}
E & 0\\
0 & a
\end{pmatrix}
 = A^{(n)}, B  = 
\begin{pmatrix}
0 & 0\\
0 & b
\end{pmatrix}
= B^{(n)}\\
C & = 
\begin{pmatrix}
0 & 0\\
0 & c
\end{pmatrix}
= C^{(n)}, D =
\begin{pmatrix}
0 & 0\\
c & d
\end{pmatrix}
= D^{(n)}
\end{align*}
 then $\mathcal{M} = \begin{pmatrix} A  & B \\ C & D \end{pmatrix} \in
 \mathcal{M}_n$ Also 
 \begin{align*}
M (Z) & = (AZ + B) (CZ + o) = \begin{pmatrix} Z & 0 \\ 0 & N
  <z> \end{pmatrix} \text { where } \tag{177}\label{eq177}\\ 
N(z) & = (az + b)(az + d)^{-1}
 \end{align*} 
 so that the mapping $z \to \mathcal{M}<z>$ is equivalent with the
 pair of mappings: 
$$
Z^* \to Z^*, z \to (az + b) (cz +d )^{-1} 
$$

Besides, we have also $|CZ + D| = cz + d$ so that from (\ref{eq177})
\begin{align*}
\varphi (z) = g(Z, T) = g(z, T) | M & = g (M <z> T) | CZ +
D|^{-\mathfrak{K}}\\ 
& = \varphi \left( \frac{az + b}{cz + d}\right) (cz + d)^{-
  \mathfrak{K}}\\ 
& = \varphi (z) \big| \begin{pmatrix} a & b \\ c & d \end{pmatrix}
\tag{178}\label{eq178} 
\end{align*}

Clearly\pageoriginale  $\varphi (Z)$ is regular in the fundamental domain
$\mathfrak{f}_1$ of $M_1$ It only remains then to verify that $\varphi
(Z)$ is bounded in $\mathfrak{f}_1$. This is obvious at least in the
case of $Z^* \in \mathfrak{f}_{n-1}$ since in this case $Z \in
\mathfrak{f}_n$ when the imaginary part $\mathscr{Y}$ of $z$ is
sufficiently large, and $g (z, T)$, being a modular form of degree
$n$, is bounded in $\mathfrak{f}_n$. 

The general case does not present any difficulty either, as in this
case there exists a substitution  

$M_1 = \begin{pmatrix} A_1 & B_1 \\ C_1 & D_1 \end{pmatrix} \in
M_{n-1}$ so that $M_1 <Z*> \in \mathfrak{f}_{n-1}$   
and if we set 

$A = \begin{pmatrix} A_1 & o \\ o & 1 \end{pmatrix},
B= \begin{pmatrix} B_1 & o \\ o & o \end{pmatrix}, C= \begin{pmatrix}
  C_1 & o \\ o & o \end{pmatrix} $ and $D = \begin{pmatrix} D_1 & o
  \\ o & 1 \end{pmatrix}$ 
 then $M = \begin{pmatrix} A & B \\ C & D \end{pmatrix} \in M_1,   M
 <z> = \begin{pmatrix} M_1 <z> & o \\ o & z \end{pmatrix}$  
 
 \noindent
 and $| C Z + D | = | c, z^* + D_1 |$.
\begin{align*}
\text {  Finally  } \varphi (Z) & = g (z, T) = g (z, T) | M\\
& = g (M <z> T) |c z + D |^{- \mathfrak{K}}\\
& = g 
 \left(
\begin{pmatrix}
M_I <Z*> & o \\ o & z
\end{pmatrix},
  T \right ) |C_1 z* + D_1 |^{-\mathfrak{K}}.
\end{align*}


The term to the extreme right is bounded as $M_1 <Z^*> \in
\mathfrak{f}_{n -1}$ and then $\varphi (z)$ is clearly bounded. It is
now immediate that $\varphi (z)$ is a modular form of  degree 1 and
weight $k$. 

Let $S = \begin{pmatrix} A & B \\ C & D \end{pmatrix}$ be an arbitrary
modular substitution of degree $n$\pageoriginale  (we shall stick to
this notation 
throughout this section). It is easy to see that the matrix $A^Z +B$
depends linearly in $z$ and further $z$ appears only in its last
column. The same is true of $(CZ + D)$ too so that $| C Z + D|$ is a
linear function of $z$, say $| c z + D| = cz + D$. It follows then
that $| CZ + D| (CZ + D)^{-1}$ too is a matrix whose elements are
linear in $z$ and which has its last row independent of $z$;
consequently 
\begin{equation*}
| CZ + D| S <Z> = | CZ + D | (AZ + B)(cz + D)^{-1} \tag{179}\label{eq179}
\end{equation*}
is also linear in $z$. Let then 

\noindent
$\sigma (T S <Z> = \dfrac{a z + b}{cz + d}$ with $| CZ + D| = cz + d$
where $a$, $b$, $c$, $d$ are all complex numbers not depending on $z$. We
shall denote by $L_{\mathscr{S}}$ the matrix $\begin{pmatrix} a & b
  \\ c & d \end{pmatrix}$ defined as above. Since $Z \in
\mathscr{Y}_n$ we have $| CZ + D | \neq o$ and hence $cz + d \neq
0$. Since $I m L_S <Z> = I m \dfrac{a x + b}{cz + d} \ge 0$ for $Z \in
\mathscr{Y}_1$ it follows that either $o = o$ in which case
$\dfrac{a}{d}$ is real and $\dfrac{a}{d} \ge 0$ or if $c \neq 0$ then
$I m - \dfrac{d}{c} \ge 0$. The Poincare' series (\ref{eq152}) now takes the
form  
\begin{equation*}
\varphi (z) = g (Z, T) = \sum_{S \in V(T)} e^{2 \pi i \dfrac{az +
    b}{cz + d}} (cz + d)^{- \mathfrak{K}}. \tag{180}\label{eq180} 
\end{equation*}

We construct a suitable $V (T)$ as follows. Let $\triangle$ be the
cyclic group generated by the modular substitution 
$$
G = \begin{pmatrix} E^{(n)} & \begin{pmatrix} o & o \\ o &
    1 \end{pmatrix} \\ o & E^{(n)}\end{pmatrix} 
$$

The mapping\pageoriginale  $Z \to G <Z>$ leaves $Z^*$ fixed and takes
$z$ into $z+1$. With the notation  
\begin{align*}
\omega_S(Z) & = e^{2 \pi i \sigma (T S <Z> )} | CZ  + D|^{-
  \mathfrak{K}}\\ 
& = e^{2 \pi i \sigma (T Z)} | S,  S = \begin{pmatrix} A & B \\ C &
  D \end{pmatrix} 
\end{align*}
we have $\omega_{SG} \in (Z) = e^{2 \pi i \sigma (T Z)} | S G$ 
\begin{align*}
& = e^{2 \pi i \dfrac{a z +b}{cz + d}} (cz + d)^{-\mathfrak{K}}
  | \begin{pmatrix} 1 & o \\ o & 1 \end{pmatrix}\\ 
& = e^{2 \pi i \dfrac{a(z+d)+d}{c(z+c)+d}} (c (z + b) +d)^{-\mathfrak{K}}
\end{align*}
where we recall $\begin{pmatrix} a & b \\ c & d \end{pmatrix} = L_S$. 

Let 
$$
M = \sum_{S* \in V* (T)} A (T) S^*.
$$

If $n_S$ denotes the set of the exponents $t$ with the property that\break
$(S^*G S^{* -1})^t = S* G^t S^{* - 1} \in \mathcal{A} (T)$, then it is
easily seen that the products $S^* G^t$ with $S^* \in V^* (T) t = t
\mod n_S$ constitute a set of the type $V (T)$. So we obtain from
(\ref{eq180}) that  
\begin{equation*}
\varphi (Z) = \sum_{S \in V* (T)} \sum_{t \mod n_S} \omega_{SG} t (z)
= \sum_{S \in V* (T)} \varphi (z) \tag{181}\label{eq181}  
\end{equation*}
where
\begin{align*}
\varphi_\mathfrak{K} (z) & = \sum_{t \mod n_S} \omega_{SG}t (Z)\\ 
& = \sum_{t \mod n_S} e^{2 \pi i \dfrac{a(z+t)+b}{a(z+t)+d}}
(c(z+b)+d)^{-\mathfrak{K}}\\ 
&= \sum_{t \mod n_S} e^{2 \pi c L_S< z + b>}
(c(z+t)+d)^{-\mathfrak{K}} \tag{182}\label{eq182}  
\end{align*}

\setcounter{pageoriginal}{141}
It is obvious that $\varphi_S (z + 1) = \varphi_S (z)$ in other words,
the functions\pageoriginale  $\varphi_S (z) ( S \in V* (T)$ and
consequently $\varphi 
(z)$ are periodic with period 1. Hence we can introduce the new
variable $S = \in ^{2 \pi i z}$ and then $\varphi_S (Z) =
\lambda_{(S)}$ and $\varphi (z) = \lambda (S)$ are single valued
functions of $S$. We shall see that they are also regular in $|S| <
1$. 

We first prove that $c \neq 0$ when and only when $n_S = 0$. Indeed
if $c = 0$ then $\dfrac{a}{d}$ is real and $> 0$ so that all the terms
occurring in the right side of (\ref{eq182}) have the same absolute value,
independent of $t$, and this in its turn means that the series in  the
right of (\ref{eq182}) is actually a finite sum as we know the series to
converge and consequently $n_S \neq 0$. Conversely, if $n_S \neq 0$
then $\omega _S (z) = \omega_{S^n_S}(z)$ or  
$$
e^{2 \pi i L_S <z>}(cz+d)^{-\mathfrak{K}} = e^{2 \pi L_S < + n_S} (c (z
+ n_S) +d)^{- \mathfrak{K}} 
$$
so that 
$$
e^{2 \pi i L_S <Z + n_S - \pi ** L_S<z>} = \left( \frac{c(z+ b_S+d}{cz
  + a} \right)^{\mathfrak{K}}. 
$$

The right side  of the last equality is meromorphic in $z$ and this
can be said of the left side only when 0 = 0. This proves our
assertion. 

We shall now appeal to Lemma \ref{chap9:lem16} to show that
\begin{equation*}
\lim\limits_{y \to r}
\varphi_S (z) = \sum\limits_{ t \mod n_S} \lim\limits_{y \to \infty}
e^{2 \pi i L_S<Z +b>} (c (z - t)+d)^{-\mathfrak{K}} \tag{183}\label{eq183}  
\end{equation*}\pageoriginale 
provided the limits in the right side all exist. This is trivially
true if $n_S>o$ as in this case we face only a finite sum in
(\ref{eq183}). In the case $n_S  = o$ the right side of (\ref{eq183})
is actually 
an infinite series, $t$ ranging from $-\infty$ to $+ \infty$, but in
this case we show that the right side of (\ref{eq182}) has a convergent
majorant not depending on $z$ in a domain of the type $| x | \le
\varphi, \mathscr{Y} \ge \delta > o$ where $x + i y = z$ and this will
validate our taking the limit in (\ref{eq183}) under the summation sign. 

Since $I m L_S < z + t> \ge o$ we have
$$ 
| e^{2 \pi i L_S< z + t>} |\le I.
$$

Also $| c (z + t) +d | \ge \in i c(c + t)+ d | \text { for } | x | \le
\mathscr{E}, y \ge \delta > o$ with a certain positive $\in = \in
(\mathscr{E}, \delta)$. For, since by assumption $n_S  = o$, we have
$c \neq 0$ and then the above inequality is equivalent with $ | z + t
+ \dfrac{d}{c} | \ge \in | c + c + \dfrac{d}{c} |$ or equivalent with
$| x + t + p + c (y + q) | \ge \in | c + p + c(i + q)$ where we write
$d/c = p + \nu $, i.e. with 
$$ 
\big | \frac{x + i (y + q)}{ 1 + q} + \frac{p + t}{1 + q} \big | \ge
\varepsilon \big | c + \frac{p + t}{1 + q} \big| 
$$\pageoriginale 
and the last inequality is clearly true by Lemma
\ref{chap8:lem15}. Thus we obtain     
$$
\big | e^{2 \pi i L_S< z + t>} (c (z + c) +d)^{- \mathfrak{K}} \big |
\le \in ^{-\mathfrak{K}} | c ( + () + d |^{- \mathfrak{K}} 
$$
and consequently $\sum\limits_{t = - \infty}^{\infty} \in ^{-
  \mathfrak{K}} | c (i + t)+ d|$ provides a convergent majorant we
were after. 

Reverting to our functions $\varphi_S(\zeta)= \varphi_S (z)$ and $x
(S) = \varphi(z)$ we observe that they are regular in $| \zeta | <
1$. The only point of doubt is the origin but all these functions are
bounded in a neighbourhood of the origin $- \varphi(z) $ is so because
it is a modular form and the $\varphi_S (z) r$ are so in view of their
having a convergent majorant not depending on $z$ as shown
earlier. Hence the origin is also free from being a singularity for any of
these functions. We can therefore conclude that 
\begin{equation*}
\chi (o) = \lim_{y \to \infty} \varphi (z), \chi_S(o) - \lim_{y \to
  \infty} \varphi_S (z) \tag{184}\label{eq184}  
\end{equation*}

Since the Poincare' series converges uniformly in every compact subset
of $\mathscr{Y}$, in particular 
$$
\chi (\zeta) = \sum_{S V*(T)} \chi_\infty (\zeta)
$$
converges uniformly on $| \zeta | = \rho (o < \rho < 1)$.

Cauchy's\pageoriginale  integration formula then yields that
\begin{align*} 
\chi (o) & = \frac{1}{2} \pi i \int\limits_{| \zeta | = S} \frac{\psi
  (\zeta)}{\zeta} o \zeta 
& = \sum_{S \in V* (T)} \frac{1}{2 \pi i} \int\limits_{| \zeta | =
  \rho} \frac{\psi_S (\zeta)}{\zeta} \alpha \zeta = \sum_{S \in V*
  (T)} \chi_S (o) \tag{185}\label{eq185}  
\end{align*}
The results (\ref{eq184}) and (\ref{eq185}) together imply that 
\begin{equation*} 
\lim_{y \to \infty} \varphi (z) = \sum_{S \in V*(T)} \lim_{y \to
  \infty} \varphi_S (z) \tag{186}\label{eq186}  
\end{equation*}
and coupled with (\ref{eq183}) this gives
\begin{align*}
g (Z, T) | \phi & = \lim_{y \to \infty} g (Z, T)\\
& = \sum_{S \in V (T)} \lim_{y \to \infty} e^{2 \pi i \sigma (T S
  <Z>)} | c z + D|^{-\mathfrak{K}} \tag{187}\label{eq187}  
\end{align*}
with $S = \begin{pmatrix} A & B \\ C & D \end{pmatrix}$ provided the
limits occurring in the right all exist and the series converges
absolutely. To compute $g (Z, T) | \phi$ therefore, and this was the
purpose with which we set out in the beginning of this section, we
need only compute  
$$
\lim_{y \to \infty} e^{2 \pi i \sigma (T S < 2 >)} | C Z + D|^{-
  \mathfrak{K}}, S \in V (T). 
$$

For this purpose we use the representation (\ref{eq162}) which can be
rewritten as 
\begin{equation*}
g(z, T) = \frac{1}{\sum (T,)} \sum_p e^{2 \pi i \sigma (T_1 S < z>
  [p])} |  c z + D|^{-\mathfrak{K}} \tag*{$(162')$}\label{eq162'}  
\end{equation*}
where $T = \begin{pmatrix} T_1 & o \\ o & o \end{pmatrix}, T_1 =
T_1^{\mathfrak{K}}> o$, $r$ being the rank of $T$ and\pageoriginale  $S
= \begin{pmatrix} A & B \\ C & D \end{pmatrix}$ and the summation for
$P$ runs over all primitive matrices $\rho = \rho^{(n, r)}$ while that
for $S$ runs over all the elements of $V_o$, viz. a complete set of
modular matrices whose second rows are non associated.  

A representation for $V_o$ is given by Lemma \ref{chap1:lem1} as follows.

Let $\{c_o, D_o\}$ run through a complete set of non associated
coprime symmetric pairs of square matrices of order $s(1 \le s \le n
)$ with $| c_o | \neq o$ and let $\{Q\}$ run through a complete set of
non right associated primitive matrices $Q  = Q^{(n, r)}$. Let $(C_o
D_o)$ be completed to a modular matrix $\begin{pmatrix} A_o & B_o
  \\ C_o & D_o \end{pmatrix} \in M_r$ and $Q$ to an unimodular matrix
$u = (Q R)$ in any one arbitrary way and set 
$$
A = 
\begin{pmatrix}
A_o & o \\ 
 o & E
\end{pmatrix}
u',\qquad B =
\begin{pmatrix}
B_o & o \\
o & o
\end{pmatrix}
u^{-1},\\
$$
$$
C = 
\begin{pmatrix}
C_o & o \\
o & o
\end{pmatrix}
u' ,  \qquad D =
\begin{pmatrix}
D_o & o\\
o & E
\end{pmatrix}
u^{-1}
$$
with $E = E^{n - r}$.

The resulting matrices $S = \begin{pmatrix} A & B \\ C &
  D \end{pmatrix}$ as $s$ runs through all integers $| \le r \le n$
together with the unit matrix $E = E^{(2 n)}$  corresponding to the
case $s = $ rank $C = o$ form a class of the desired type. We can now
write (\ref{eq162'}) using (\ref{eq69}) as $g (z, T) =
\frac{1}{\varepsilon (T,)} \sum_\rho e^{2 \pi i \sigma (T, Z [\rho])}$  
\begin{equation*}
\frac{1}{\sum (T,)} \sum_{r  = 1}^{n} \sum_\rho \sum_{\substack{\rho
    C_o, D_o \\ \{ Q \} }} e^{2 \pi  i \sigma(T, S <z> [\rho])} |c, z
     [Q] + D_o |^{-\mathfrak{K}} \tag{188}\label{eq188}  
\end{equation*}

We introduce\pageoriginale  again $Z = \begin{pmatrix} Z*  & o \\ o &
  z \end{pmatrix}$ with $Z = X + i y, z = x + i y, Z^* = X^* + i y^*$
and denote $Q$ as $Q = (\mathscr{Y}_{m \nu} = (\mathscr{Y}_1
\mathscr{Y}_2 \cdots \mathscr{Y}_r$. We assume without loss of
generality that $y [Q]$ is reduced. Let $Q^*$ denote the matrix
arising from $Q$ by deleting its last row and let $Q^* =
(\mathscr{Y}^*_1 \cdots \mathscr{Y}^*_r$. 

Then using (\ref{eq47} - \ref{eq49}) we have
\begin{equation*}
|| C_o Z [Q] + D_o || \ge | Y [Q] | \ge \frac{1}{c_1} \pi^r_{\nu = 1}
Y [\mathscr{Y}_\nu] = \frac{1}{c_1} \mathscr{Y}^r_{\nu = 1} (Y^* L
\mathscr{Y}_o^*) +y q^{z}_{n \nu}) \tag{189}\label{eq189}  
\end{equation*}
with a certain constant $c_1 > o$.


It is clear from (\ref{eq189}) that if $q_{n1} q_{n2} \cdots q_{n r}) \neq
(o o \cdots o)$ then $|| C_o Z\break [Q] + D_o || \to \infty$ as $y \to x$
and consequently 
$$
\lim_{y \to \infty} e^{2 \pi i \sigma (T_1 S <z> [P]} | C_o Z [Q] +
D_o |^{-\mathfrak{K}}= o 
$$

The alternative case, viz. $q_{n1} q_{n2} \cdots q_{n r}) = ( o o
\cdots o)$ can occur only if $ s < n$. For, if $s = n$, then $Q$ is a
primitive square matrix and hence is itself unimodular and its last
row cannot therefore consist of all zeros. Let then $s < n$ and
$q_{n1} q_{n2} \cdots q_{n r}) = ( o o \cdots o)$ so that $Q
= \begin{pmatrix} Q^*  \\ o   \end{pmatrix}$. Then $Q^*$ is itself
primitive and can be completed to a unimodular matrix $u^* = (Q ^*
R^*)$. The choice $R = \begin{pmatrix} R^* o \\ o & 1 \end{pmatrix} $
is permissible and then $u = \begin{pmatrix} u^* & o \\ o &
  1 \end{pmatrix}$ 

A simple\pageoriginale  computation yields that
\begin{align*}
s\langle Z \rangle &= (AZ +  B) (CZ + D)^{-1}\\ 
&=\left \{ \begin{pmatrix} A_0 & 0 \\ 0 &
  E \end{pmatrix} \begin{pmatrix}Q' \\ R' \end{pmatrix} Z
+ \begin{pmatrix} B_0 & 0 \\ 0 & 0 \end{pmatrix} (Q
R)^{-1}\right\}\\
& \hspace{3.2cm} \times \left \{ \begin{pmatrix} 0_0 & 0 \\ 0 &
  0 \end{pmatrix}\begin{pmatrix}Q' \\ R' \end{pmatrix} Z
+ \begin{pmatrix} D_0 & 0 \\ 0 & E \end{pmatrix} (Q
R)^{-1}\right\}^{-1}\\ 
&=\left \{ \begin{pmatrix} A_0 & 0 \\ 0 &
  E \end{pmatrix}\begin{pmatrix}Q' \\ R' \end{pmatrix}Z (Q
R)+\begin{pmatrix} B_0 & 0 \\ 0 & 0 \end{pmatrix}\right \}\\
& \hspace{3.5cm} \times \left
\{\begin{pmatrix} C_0 & 0 \\ 0 & 0 \end{pmatrix}\begin{pmatrix}Q'
  \\ R'\end{pmatrix}Z(Q R)+ \begin{pmatrix}D_0 & 0\\ 0 &
    E \end{pmatrix}\right \}^{-1}\\ 
&=\begin{pmatrix}A_0Z[Q]+B_0 & A_0 Q' ZR\\ R'ZQ &
  Z[A] \end{pmatrix}\begin{pmatrix}C_0Z[Q]+D_0 & C_0 Q' ZA \\ R'Z Q &
    E \end{pmatrix}^{-1}\\ 
&=\begin{pmatrix}
A_0Z^*[Q^*]+B_0 & A_0 Q^{*^1} Z^* R^* & 0\\ 
R^{*^1}Z^*Q^* &  Z^*[R^*] & 0 \\
 0 & 0 &  0
 \end{pmatrix}\\
&\qquad \begin{pmatrix}
C_0Z^*[Q^*]+D_0 & C_0{Q^{*^1}} Z^* R^* & 0\\
 0 & E & 0 \\ 
0 & 0 & 1
 \end{pmatrix}^2\\ 
&= \begin{pmatrix}
S^* \langle Z^* \rangle & 0\\ 
0 & Z
 \end{pmatrix} 
\end{align*}
where $S^*$ represents a modular substitution of degree $n-1$ which
has the same relationship to the classes $\{ C_0, D_0 \}$, $\{Q^*\}$ as
$S$ has, to $\{C_0, D_0\}$ and $\{Q \}$. This means that 

\noindent
$S^*=\begin{pmatrix}A^* & B^*\\ C^* & D^* \end{pmatrix}$ with \\

\noindent
$A^*=\begin{pmatrix}A_0 & 0\\ 0 & E \end{pmatrix}u^{*'},
\qquad B^*=\begin{pmatrix}B_0 & 0 \\ 0 & 0\end{pmatrix}
u^{*^{-1}}$\\ 

\noindent
$C^*=\begin{pmatrix}C_0 & 0\\ 0 & 0 \end{pmatrix}u^{*'},
\qquad D^*=\begin{pmatrix}D_0 & 0 \\ 0 & 0\end{pmatrix}
u^{*^{-1}}$, 

\noindent
where\pageoriginale  $u^* = (Q^* R^*)$  and  $E = E^{(n-1-s)}$. 

Writing $S \langle Z \rangle = X_3 + SY_S$  and  $S^* \langle Z ^*
\rangle = X^*_{S*} +  SY^*_{S^*}$  
we have

\noindent
$Y_S=\begin{pmatrix}Y_{S^*}^* & 0\\ 0 & y \end{pmatrix}$ and $| C_0 Z
    [Q]*  D_0 | = | C_0 Z^* [Q^*] + D_0|$ Let $P = (P_{\mu \nu}) =
    (y_n,y_2,\ldots,y_n) $ and $P^* = (y_n^*, y_2^*, \ldots, y_n^*)$
    be the matrix arising from $P$ by deleting its last row. From
    (\ref{eq102}) we have 
\begin{align*}
\sigma (\tau , Y_s[P])\geq y \sigma (Y_s [P]) &= y
\sum_{\nu=1}^r Y_S[Y_{\nu}]\\ 
&= y \sum_{\nu =1}^r (Y_{S^*}^*[Y_\nu^* ] + YP_{\mu\nu}^2) 
\end{align*}
with a positive constant $y = y(\tau_1)$. Hence
$\sigma(\tau, Y_s[P])\to \infty$ as $y \to \infty$ and consequently
$\lim_{y \to \infty} \in^{2 \pi s \in (T_1 S \langle Z \rangle[P])}=0$ 
provided $(P_{n1}, P_{n2},.P_{nr} ) \neq (0  0 ,. 0)$.

Thus in this case too, the general term

\noindent
$e^{2 \pi s \sigma(\tau_1S \langle Z \rangle [P])} | C_0 Z [Q] + D_0
|^{-k}$ of (\ref{eq162'}) 

\noindent
tends to zero as $y \to \infty$. The only case which remains to be
settled is when both $(q_{n1}, q_{n2}., q_{ns})$ and
$(p_{n1}, p_{n2}., p_{nr})$ are the respective zero vectors
simultaneously. In this case we have seen that $s < n$ and by an
analogous reasoning $r < n$ and $P = \begin{pmatrix}P^*
  \\ 0 \end{pmatrix}, Q = \begin{pmatrix}Q^* \\ 0 \end{pmatrix}$ so
that $P^*$ and $Q^*$ are themselves primitive. Further the general
term of the Poincare series (\ref{eq162'}) does not involve $Z$ in this case
as the relations $S \langle Z \rangle [P] = \begin{pmatrix}s^* \langle
  Z^* \rangle & 0 \\ 0 & Z \end{pmatrix} \begin{pmatrix}
  P^*\\ 0 \end{pmatrix} = S^* \langle Z^*, [P^*]$ and $|C_0 Z [Q] +
D_0| = |C_0 Z^*[Z^*]+D_0|$ show. Trivially then does the\pageoriginale limit
$\lim\limits_{y \to \infty}e^{2 \pi i \sigma (\tau_1 S \langle Z \rangle[P])}
|C_0Z[Q]+D_0|^{-k}$ exist in this case too. We have therefore shown
that when $Q \neq \begin{pmatrix}Q^*\\ 0 \end{pmatrix}$ or $P
\neq \begin{pmatrix}P^*\\ 0 \end{pmatrix}$ and in particular when
either $r = n$ or $s = n$, the general term of the Poincare' series
(\ref{eq162'}) tends to zero as $ y \to \infty$ while in the alternative
case viz. when $Q= \begin{pmatrix}Q^*\\ 0 \end{pmatrix},
P=\begin{pmatrix}P^* \\ 0 \end{pmatrix}$ simultaneously, whence $r <
n$ and $s < n$, it assumes as the limit 

\noindent
$e^{2 \pi i \sigma(\tau_1S^* \langle Z^* \rangle
  [P^*])}|C_{\circ}Z^*[Q^*]+D_{\circ}|^{-k}$, being in fact
independent of $z$ in this case. Since $g(Z, T)| \phi = \lim_{y \to
  \infty}g (Z, T)$  

\noindent
we can now state that $g(Z, T)| \phi=0$ in case $r=n > 1$ and
\begin{align*}
g(Z, T)| \phi &=\frac{1}{\varepsilon (T_1)}\sum e^{2 \pi i \sigma
  (T_1Z^*[P^*])}+\frac{1}{\varepsilon  (T_1)}\sum^{p^*
  n-1}_{\mathscr{S}=1}\sum_{p^*} \sum_{\substack{\{ C_\omega,
    0_0\}\\ \{ Q^*  \}}}\\
& \qquad e^{2 \pi i \sigma(T_1 S^* \langle
  Z^*\rangle[P^*])}|C_0Z^*[Q^*]+D_0|^{-k}\\ 
&=g(Z^*, T^*)
\end{align*}
in case $r <n$ where $T^*=T^{*(n-1)}=\begin{pmatrix}T_1 & 0 \\ 0 &
0 \end{pmatrix}$, viz. the matrix which arises from $T$ on depriving
it of its last row and column. $g(Z^*, T^*)$ is then just the
Poincare' series obtained from (\ref{eq162'}) on replacing $n$ by $n-1$. We
may define formally $g(Z, T)=1$ for $n=0$ so as to validate the above
results for all $n \geq 1$ and state 


\setcounter{thm}{11}
\begin{thm}\label{chap11:thm12}%%% 12
Let $T=\begin{pmatrix}T_1 & 0 \\ 0 & 0 \end{pmatrix}$,
$T_1=T_2^{(r)}>0$, $T=T^{(r)} n>0$ and $k \equiv 0(2)$, $k
>n+r+1$. Let  $g(Z, T)$ be a Poincare series  of degree 
$n$  and  weight  $k$ and let $Z^* \cdot T^*$   the matrix
  which arises from $Z$, $T$  by depriving them of their
  last\pageoriginale row and column. 
\end{thm}

 Then 
\begin{equation*}
g(Z, T)|\phi=
\begin{cases}
g({Z}^*, T^*)  &, \text{ for }  \quad r < n\\ 0 &, \text{ for } \quad
r=n\tag{190}\label{eq190}  
\end{cases}
\end{equation*}

The theorems \ref{chap10:thm11} and \ref{chap11:thm12} lead to
interesting consequences. An 
immediate consequence is that the Poincare' series $g(Z, T)$ with
$T>0$ generate the space  $\mathscr{P}^{(n)}_\mathfrak{K}$ of all cusp
forms. For, from (\ref{eq190}) 
we know that such a $g(Z, T)$ belongs to $\mathscr{P}^{(n)}_{\mathfrak{K}}$ If
$\mathfrak{f}_k^{(n)}$, denotes the space generated by $g(Z, T), T> 0$
and  $\mathfrak{K}_k^{(n)}$, the orthogonal space of
$\mathfrak{f}_\mathfrak{K}^{(n)}$ in $\mathscr{P}^{(n)}_\mathfrak{K}$, then
$\mathscr{P}^{(n)}_\mathfrak{K} =
\mathfrak{f}^{(n)}_\mathfrak{K}+\mathfrak{R}_\mathfrak{K}^{(n)}$,
the sum being  
direct. If $\mathfrak{f}(Z)\varepsilon  \mathfrak{R}_\mathfrak{K}^{(n)}$ then
$\mathfrak{f}(Z)\varepsilon   \mathscr{P}^{(n)}_\mathfrak{K}$ and is
therefore a cusp 
form. This means that $a(T)=0$ for $|T|=0$ in the standard notation by
(\ref{eq108}), while if $T>0$, $(\mathfrak{f}(Z), g(Z, T))=0$, as then
$\mathfrak{f}(Z)\in \mathfrak{K}_\mathfrak{K}^{(n)}$ and $g(Z, T)\in
\mathfrak{f}_\mathfrak{K}^{(n)}$ so that from Theorem
\ref{chap10:thm11} we have again  
$$
a(T) = \frac{1}{\mathscr{C}}(\mathfrak{f}(Z), g(Z, T)) = 0. 
$$

Thus all the Fourier coefficients $a (T)$ of $\mathfrak{f} (Z)$ vanish
and $\mathfrak{f}(Z) \equiv 0$. This means that
$\mathfrak{R}_\mathfrak{K}^{(n)} = 0$ and consequently
$\mathscr{P}^{(n)}_\mathfrak{K} = \mathfrak{f}_\mathfrak{K}^{(n)}$. We
now go a step further and prove 


\begin{thm}\label{chap11:thm13}%Thm 13
The Poincare' series $g(Z, T)$ with rank $T= r$
(fixed) generate $\mathscr{P}_{\mathfrak{K} r}^{(n)}(r \leq n)$ 
  provided $k \equiv 0(2)$  and  $k > n + r + 1$. 
\end{thm}

\begin{proof}
The\pageoriginale  case $r=n$ was just now settled. Assume than that
$r < n$. Let 
$\mathfrak{f}_{km}^{(n)}$ denote the space generated by $g(Z, T)$ with
rank $T= r < n$ and apply induction on $n$. From theorem \ref{chap10:thm11} it
easily follows that $\mathfrak{f}_{\mathfrak{K}r}^{(n)}\subset
\mathscr{M}_\mathfrak{K}^{(n)}$ and due to induction assumption
$\mathfrak{f}_{\mathfrak{K}r}^{(n)}|\phi=
\mathscr{P}^{(n-1)}_{\mathfrak{K}r}$. Since by 
construction, $\mathscr{P}^{(n)}_{\mathscr{K}r} | \phi \subset  
\mathscr{P}^{(n-1)}_{\mathfrak{K}r}$ and $\phi$ is $1-t$ on
$\mathscr{M}_\mathfrak{K}^{(n)}$ it 
follows that $\mathscr{P}_{kn}^{(n)}| \phi \subset
\mathscr{P}_{\mathfrak{K}r}^{(n-1)}=\mathfrak{f}_{\mathfrak{K}n}^{(n)}|
\phi$ and consequently 
$\mathscr{P}_{\mathfrak{K}r}^{(n)} \subset
\mathfrak{f}_{\mathfrak{K}n}^{(n)}$. The reverse 
inclusion is also immediate as is seen from the relation 
$$
\mathfrak{f}_{\mathfrak{K}r}^{(n)}| \phi \subset
\mathscr{M}_\mathfrak{K}^{(n)}| \phi \cap 
\mathscr{P}_{\mathfrak{K}r}^{(n-1)} = \mathscr{P}^{(n)}_{\mathfrak{K}r}| \phi 
$$
establishing thereby theorem \ref{chap11:thm13}. In conjunction with
Theorem \ref{chap10:thm11} this yields 
\end{proof}

\begin{thm}\label{chap11:thm14}%Thm 14
The Poincare' series $g(Z, T)$ generate
$\mathscr{M}_k^{(n)}$, the space of all modular form of degree
 $n$ and weight $k$ provided $k \equiv 0(Z)$, $k >
2n$ (the usual condition for the convergence of $g(Z, T)$). 
\end{thm}

This is precisely the representation theorem we were looking
for. Further since $\mathfrak{f}_\mathfrak{K}^{(n)}|
\phi=\mathscr{P}^{(n)}_\mathfrak{K}$ and we have shown that
$\mathfrak{f}_{\mathfrak{K}n}^{(n)} = \mathfrak{f}^{(n)}_{\mathfrak{K}
r} = \mathscr{P}_{\mathfrak{K}r}^{(n)}$ we have  
\begin{equation*}
\mathscr{P}_{\mathfrak{K}r}^{(n)}| P =
\mathscr{P}_{\mathfrak{K}r}^{|n-1|}(\mathfrak{K} \equiv 
0(Z),\mathfrak{K}>n+r+1) \tag{191}\label{eq191}   
\end{equation*}

Repeated application of (\ref{eq191}) yields that
$$
\mathscr{P}^{(n)}_{\mathfrak{K}r} | \phi^{n-r} =
\mathscr{P}^{(n)}_{\mathfrak{K}r} + \mathscr{P}^{(n)}_{\mathfrak{K}}
(\mathfrak{K} \equiv (2) \mathfrak{K} = n + r+1)
$$

We know\pageoriginale that $\phi^{n \cdot r}$ is $1-1$ on
$\mathscr{P}_{\mathfrak{K},r}^{(n)}$  and we therefore infer that  
$$
\text { rank } \mathscr{P}_{\mathfrak{K}r}^{(n)}= \text{ rank }
\mathscr{P}_\mathfrak{K}^{(r)} 
$$

In particular taking $r=0$, $\mathscr{P}_\mathfrak{K}^{(0)}$ is the
space of all constants so that rank
$\mathscr{P}^{(n)}_{\mathfrak{K}u}=$ rank
$\mathscr{P}_\mathfrak{K}^{(0)}=1  (k  
\equiv 0(2), k> n+1)$.  $\mathscr{P}_{\mathfrak{K}, u}^{(n)}$ is
thus generated by a 
single element $g(Z, 0)$, the so called \textit{Einstein Series}
which converges for $\mathfrak{K} > n+1 \equiv 0(2)$, and represents a
modular form not vanishing identically under these conditions. Setting now
$r=1$, $\mathscr{P}_\mathfrak{K}^{(1)}$ is the space of all cusp forms
of the (classical) elliptic modular forms and from a well known result  
$$
\text{ rank } \mathfrak{P}^{(1)}_\mathfrak{K}=
\begin{cases}
\left[\frac{\mathfrak{K}}{12} \right]-1, \text{ if } \mathfrak{K}
\equiv Z (12), \mathfrak{K}>n+2,  \\ 
\left[\frac{\mathfrak{K}}{12} \right], \text{ if } \mathfrak{K} \neq 2(12),
\mathfrak{K} \equiv 0(Z),\mathfrak{K}>n+2  
\end{cases}
$$
and the same is therefore true of rank $\mathscr{P}_{\mathfrak{K} l}^{(n)}$.

We proceed to generalize the fundamental metric formula given in
Theorem \ref{chap10:thm11}. 

Let $\mathfrak{f}(Z)\in \mathscr{P}_{\mathfrak{K}r}^{(n)}$, $0 \leq r
\leq n$. If rank $T=s$, then $g(\tau, T)\in
\mathscr{P}^{(n)}_{\mathfrak{K}, s}$ and we state that $\big(P(Z), g(Z,
T) \big)=0$   $y r  \neq s$ 

\noindent
while if $r=s$ we have in conformity with (\ref{eq172}),
$$
(f(Z), g(Z, T))= (f(Z)|\phi^{r-r}.g(Z)| \phi ^{n-r})
$$

We can apply theorem \ref{chap10:thm11} to the scalar product in the right side
above, as follows. 
Assume $r >0$ and assume for a moment that $T= \begin{pmatrix}f, & 0
  \\ 0 & 0 \end{pmatrix}, y_1=y_1^{(r)}>0$. 

Then,\pageoriginale as a result of theorem \ref{chap11:thm12}, $g(Z,
T)|\varphi^{n-r}=g(Z_1, T_1)$ 
where $Z_i$ denotes the matrix arising from $Z$ by deleting its last
$(n-r)$ rows and columns and $T_1$ is as defined above. Hence, by means
of theorem \ref{chap10:thm11}, 
\begin{align*}
 (\mathfrak{f}(Z), g(Z, T)) &= (\mathfrak{f}(Z)|\varphi^{n-r}, g(Z,
  T)|\varphi^{n-r})\\ 
 &=(\mathfrak{f}(Z)|\varphi^{n-r}, g(Z_1, T_1))\\
 &=\frac{2}{\in(T_1)} Q(T_1) \pi \frac{n(n-1)}{4  (4 \pi)}
  \frac{r(r+1)}{\mathfrak{K}}-r\mathfrak{K} \\
& \qquad \qquad \qquad \times \pi^{\mathfrak{K}}_{n=1}y
  (\mathfrak{K}-\frac{r+\nu}{2})|T_1|\frac{n+1}{2}-\mathfrak{K} 
 \end{align*}
  where the Fourier coefficient $a(T)$ of
  $\mathfrak{f}(Z)|\varphi^{n-r}$ is identical with\break
  $a \begin{pmatrix}T_1 & 0 \\ 0 & 0\end{pmatrix}=a(T)$, the Fourier
    coefficient of $\mathfrak{f}(Z)$ In the case of a general $T$ we
    can always assume that $T=\begin{pmatrix}T_1 & 0 \\ 0 &
    0 \end{pmatrix}[\mathcal{U}]$ with an appropriate unimodular
    matrix $\mathcal{U}$. Then $\triangle(T)=|T_2|$ and $\in (T_1)$
    are uniquely determined by $T$ and we have as above 
  
  $(\mathfrak{f}(Z), g(Z, T))= \dfrac{\mathfrak{K}}{\in (T)}a(T)\pi
    \dfrac{r(n-1)}{4} \underset{(4 \pi)}{} n(r+1)/\mathfrak{K}-r\mathfrak{K}_x
    \times \pi_{\nu=1}^{r} y
    (\mathfrak{K}-\dfrac{r+\nu}{\mathfrak{K}})(\triangle (T))
    \dfrac{n+1}{\mathfrak{K}-\mathfrak{K}}$, 
  
  \noindent
  our specific assumptions being $\mathfrak{K} \equiv 0(2)$,
  $\mathfrak{K}> n+r+1$, rank $T=r 
  \geq 1, \mathfrak{f}(Z)\in \mathscr{P}_{\mathfrak{K}r}^{(n)}$, $g(Z, T)\in
  \mathscr{P}_{\mathfrak{K} 
    n}^{(n)}$. The case when $r=s=0$ still remains. In this case
  $T=(0)$ and $(\mathfrak{f}(Z)$, $g(Z,
  0))=(\mathfrak{f}(Z)|\varphi^{(n)}$, $g(Z, 0)|\varphi^n)$ as an
  immediate consequence of our definition. Now
  $\mathfrak{f}(Z)|\varphi^n$ is a modular form of degree 0 and hence
  is equal to $C:(0)$ while with regard to $g(Z, 0)| \varphi^n$ we can
  be a bit more specific and state that $g(Z, 0)|\varphi^n=1$ as is
  easily verified by a consideration of the Eisenstein\pageoriginale
  series of degree 1 Hence $(\mathfrak{f}(Z), g(Z, 0))=a(0)$. We
  thus have  


   \begin{thm}\label{chap11:thm15}%Thm 15
  Assume $\mathfrak{K} \equiv 0(2)$, $\mathfrak{K}>n+r+1$  rank $T=r$ 
    and 
  $\mathfrak{f}(Z)\in \mathscr{P}_{\mathfrak{K} r}^{(r)}$. Then 
{\fontsize{10pt}{12pt}\selectfont
$$
(\mathfrak{f}(Z), g(Z, T)) = \begin{cases}
\frac{2}{\in(T)} a(T)\pi \frac{r(n-1)}{4}\underset{(4
  \pi)}{}\frac{r(r+1)}{2}n\mathfrak{K}_{\prod\limits_{\substack{\nu=1 \\ \text{
        for }  r>0}}}y (\mathfrak{K}-\frac{r+
  \nu}{\mathfrak{K}})(\triangle(T))^{\frac{r+1}{2}}\\a(0) \qquad
\text{ for }  \qquad  r=0 
\end{cases}
$$}\relax
where $a(T)$ denotes the Fourier coefficient of
  $\mathfrak{f}(Z)$ corresponding to the matrix $T$. 
   \end{thm}

