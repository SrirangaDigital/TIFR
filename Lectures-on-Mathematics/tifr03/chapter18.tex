
\chapter[The Dirichlet series corresponding ....]{The Dirichlet series corresponding to modular forms of
  degree $n$}%%% 18 

We wish\pageoriginale to determine a set of Dirichlet series which is
equivalent with a given modular form of degree $n$. We need as 
preliminaries the followings the following lemma and a host of other
operator identities.   

\setcounter{lem}{26}
\begin{lem}%%% 27 
Let $\mathfrak{f} (Z) = \sum_{ T ge O} a(T) e^{2 \pi i \sigma (T Z)}$ 
(T semi integral) (\ref{eq98}) be a modular from of degree $n$ and weight
$\mathfrak{K} \equiv (2)$. Then 
\begin{equation*}
|a (T)| \le \vartheta_o |T|^\mathfrak{K} \tag{409}\label{eq409}
\end{equation*}
for $T > o $ with a certain constant $\vartheta_o$ 
\end{lem}

\begin{proof} 
In view of (\ref{eq95}) both sides of (\ref{eq409}) are invariant relative to a
modular substitution $T \to T [u]$.Hence we need prove (\ref{eq409}) only
for $T > o$ which are further reduced. Clearly  $a(T)$ has the
integral representation, 
$$
a(T) =\int\limits_{\mathcal{H}} f (\times + i T^{-1}) e^{2 \pi i \sigma (T
  \times i ^{-1})} {_{[d \times]}} 
$$
as in (pp.64) and then  
$$
|a (T)| \le \max_{\times \varepsilon \mathcal{H}} |f (\times + i T
^{-1})| e^{2 \pi n} 
$$

Hence to infer (\ref{eq401}) we need only prove that 
$$
|f (\times + i T^{-1})| \le \mathscr{C}_1 |T|^\mathfrak{K} 
$$
and this again is ensured from (\ref{eq49}) as by assumption $T$ is reduced
provided\pageoriginale we show that  
\begin{equation*}
|\mathfrak{f} (\times + i T ^{-1})| \le \mathscr{C}_2 (t_{11} t_{22}
\dots t_{nn})^\mathfrak{K} \tag{410}\label{eq410} 
\end{equation*}
where we assume $T =(t_{\mu \nu})$
\end{proof}

We now prove (\ref{eq410}). Let $Z \in \mathscr{Y}_n$ and determine
$M \begin{pmatrix}  A & B\\ C & D\end{pmatrix} \in M_n$ such that $Z_1
  = M < Z > \in \mathfrak{f}_n$ Let $ N \begin{pmatrix}  S & -E
    \\ E & D\end{pmatrix} \in M_n$ and $Z_a N^{-1} < Z > = (-Z
    + S)^{-1}$. 

Then $Z_1 = MN <Z_o> $ and $ MN = \begin{pmatrix}  * & * \\ cs+d &
  -\varepsilon \end{pmatrix} $ by the property of modular forms we
have $\mathfrak{f} (Z_o) = |-z+s|^\mathfrak{K} \mathfrak{f} (z)$, and
$\mathfrak{f}(z_1) = |(cs+D)z_o - C_o| f (z_o)$ and consequently  
\begin{equation*}
\mathfrak{f} (z) = |-z +s|^{-\mathfrak{K}} |(cs +D) z_o-C| \mathfrak{f}
(z_1) \tag{411}\label{eq411} 
\end{equation*}

Choose $S = (\mathscr{S}_{\mu \nu})$ as an integral symmetric matrix
with $\pm \mathscr{S}_{\mu \nu} \le n$ and $|cs+D| \neq 0$. Such a
choice is clearly possible as $|CS+D|$ is a polynomial of degree almost
$n$. Not vanishing identically in the elements $\mathscr{S}_{\mu \nu}$
of $S$. Since $C$, $S$, $D$ are all integral matrices, we have then in
particular  
\begin{equation*}
\| CS +D \| \ge 1. \tag{412}\label{eq412}
 \end{equation*} 
 
 Let $Y, Y_o$ denote the imaginary parts of $Z, Z_o$. Since $||Z +S||
 \ge |Y|$ for any real symmetric matrix $S$ and $|Y_o| = |Y|| -Z
 +S|^{-2}$ for $z_o = (-Z+S)^{-1}$ as is seen from (\ref{eq72}) we conclude
 from (\ref{eq411}) - (\ref{eq412})  that\pageoriginale   
 \begin{align*}
|f(Z)| & \le  \mathscr{C}_3 ||-z+s||^\mathfrak{K} ||z_o-(cs+D)^{-1}c
||^{-\mathfrak{K}} \\ 
& \le \mathscr{C}_3 ||-z+s||^{-\mathfrak{K}} |y_o|^{-\mathfrak{K}}\\
& = \mathscr{C}_3 ||-Z+S||^\mathfrak{K} |y|^{- \mathfrak{K}}
|\tag{413}\label{eq413} 
 \end{align*} 
 
 If now elements of $X$ are reduced modulo 1 as in (\ref{eq410}) and
 $Y=T^{-1}, T > 0$, and semi integral, then the elements of $-X$ and
 of $B=-X+S$  are bounded , and   
 \begin{equation*}
(-Z +S)y^{-1} = (BT -iE), Z= \times + i Y \tag{414}\label{eq414}
 \end{equation*} 
 
 The elements of $(BT- iE)$ are of the form $\sum \mathscr{C}_{\mu \nu}
 t _{\rho \nu} -i \rho _{\mu \nu}$ and the absolute value of $\{ \sum
 \mathscr{C}_{\mu \nu} t _{\rho \nu} -i \rho _{\mu \nu}\}$ is at most
 equal to $\mathscr{C}_4 t_{\nu \nu}$ as $t_{\nu \nu} \ge 1$. Hence
 $||BT -iE|| \le \mathscr{C}_5 t_{11} t_{12} \dots t_{nn}$ and (\ref{eq410})
 now follows as a consequence of (\ref{eq413}) and (\ref{eq414}).  
 
 Let us split up the series on the right of \ref{eq98}$'$ according to the
 rank of $T$ and write 
 $$
 \mathfrak{f} (i y)= \sum^n_{r=0} \mathfrak{f}_r (y), f_n (y)
 =\sum_{\substack{t \ge 0 \\ r a n \mathfrak{K} T = r }} a (T) e ^{2
   \pi \sigma (T Y)}.  
 $$
 
 Set $Y = y Y_1, y > 0$ and $|Y_1| = 1$, and introduce the integral 
 \begin{equation*}
R (\mathscr{S}, Y_1) = \int\limits_o^\infty f_n (y Y_1) y^{n \mathscr{S}-1}
dy \tag{415}\label{eq415} 
 \end{equation*} 
 
 $R (\mathscr{S}, Y_1)$ is a function on the determinant surface $|Y_1|
 =1$ and is invariant relative to the unimodular substitutions, viz. 
 $$
 R(\mathscr{S}, Y_1 [u]) = R (\mathscr{S}, Y_1)
 $$
 for\pageoriginale any unimodular matrix $u$. Let $u(y)$ be any
 angular character  and let   
 \begin{equation*}
\xi (\mathscr{S}, u) =\int\limits_{\substack{Y_1 \varepsilon
    \mathfrak{K}\\ |Y_1|= 1}} R (\mathscr{S}, Y_1) u (Y_1) d\vartheta_1
\tag{416}\label{eq416} 
 \end{equation*} 
 where as usual, $\mathfrak{K}$ is the space of reduced matrices.  
 
 Since $y^{-1} dy d \vartheta_1 = \frac{1}{\sqrt{n}} d \vartheta$ from
(\ref{eq383}), we have
 \begin{align*}
\xi (\mathscr{S}, u) &  = \int\limits_{\substack{Y_1 \varepsilon
    \mathfrak{K} \\ |Y_1| = 1}} \int\limits_{y =0}^\infty \mathfrak{f}_n
(Y) u (Y)|Y|^{ n \mathscr{S} -1} y dy d \vartheta_1\\ 
 & = \frac{1}{\sqrt{n}} \int\limits_{Y \varepsilon \mathfrak{K}}
\mathfrak{f}_n (Y) u (Y) |Y|^\mathscr{S} d  \vartheta 
 \end{align*}
 
 In this we substitute for $\mathfrak{f}_n (Y)$ its representation by a
 series and integrate termwise.  All these and the subsequent
 transformations can be justified if $u (Y)$ is bounded and the real
 part of $s$ is sufficiently large, and we assume this to be
 the case. Then    
 \begin{equation*}
\xi (\mathscr{S}, u) = \frac{1}{\sqrt{n}} \sum_{T > o} a (T)
\int\limits_{Y \varepsilon \mathfrak{K}} e ^{- 2 \pi \sigma (T Y)} u (Y)
|Y|^\mathscr{S} d \vartheta \tag{417}\label{eq417} 
 \end{equation*} 
 
 In the right side (\ref{eq417}) we have a sum of the type $\sum\limits_{T >
   o} F (T)$ where $T$ runs over all semi integral matrices.  
 
 If \{ T \} denotes the class of all matrices $T [u]$ for a given $T
 (u)$\break (-unimodular) then in general we have 
 $$
 \sum_T > 0 F (T) = \sum_{\{T\}} \frac{1}{\varepsilon(T)} \sum_u F F(T
     [u]) 
 $$
 where $\in (T)$ denotes the number of units of $T$, i.e., of
 unimodular matrices $u$ with\pageoriginale $T[u]= T$ this number is
 finite as $T > 
 0$ and $u$ runs over all unimodular matrices while $\{ T\}$ runs
 over all distinct classes of positive semi integral matrices
 $T$. Since by assumption $\mathfrak{K} \equiv 0(2)$ we have from
(\ref{eq95}) that $a (T [u]) = a (T)$ and then from (\ref{eq417}),   
 \begin{align*}
\xi (\mathscr{S}, u) & = \frac{1}{\sqrt{n}} \sum_{\{T\}} \frac{a (T)}
    {\varepsilon (T)} \sum_u \int\limits_{ Y \varepsilon \mathfrak{K}} e
    ^{- 2 \pi \sigma (T [u] Y)} u (Y) |Y|^\mathscr{S} d \vartheta \\ 
& = \frac{1}{\sqrt{n}} \sum_{\{T\}} \frac{a (T)} {\varepsilon (T)}
    \sum_u \int\limits_{ Y \varepsilon \mathfrak{K}} e ^{- 2 \pi \sigma
      (T [u'] Y)} u (Y[u']) |Y[u']|^\mathscr{S} d \vartheta \\ 
& = \frac{1}{\sqrt{n}} \sum_{\{T\}} \frac{a (T)} {\varepsilon (T)}
    \int\limits_{Y > 0}  e^{- 2 \pi \sigma (T Y)} u (Y)
    |Y|^\mathscr{S} d \vartheta 
 \end{align*} 
 
 The substitution $Y [R] \to Y$ where $T = R R'$ then yields 
 \begin{equation*}
\xi (\mathscr{S}, u) = \frac{2}{\sqrt{n}} \sum_{\{ T \}} \frac{a (T)}
    {\varepsilon (T)} |T|^\mathscr{S} \int\limits_{Y > o}e^{- 2 \pi
      \sigma (Y)}{_{u_1 (Y) |Y|^\mathscr{S} d \vartheta}} \tag{418}\label{eq418} 
\end{equation*}
where
$$
u_1 (Y) = u (Y [R^{-1}])
$$

Where
\begin{equation*}
W (\mathscr{S}, u_1) = \int\limits_{ Y > o} e^{- 2 \pi \sigma
  (Y)}{_{u_1 (Y) |Y|^\mathscr{S} d \vartheta}} \tag{419}\label{eq419} 
\end{equation*}
that gives
\begin{equation*}
\xi (\mathscr{S}, u) = \frac{2}{\sqrt{n}} \sum_{\{ T \}} \frac{a (T)}
    {\varepsilon (T)} |T|^\mathscr{S} W (\mathscr{S}, u_1)
    \tag{420}\label{eq420}  
\end{equation*}

We may remark that $u_1$ is a solution of (\ref{eq362}) belonging to the
same set of eigen values as $u$ and if $u$ is bounded so is $u_1$. We
\pageoriginale have to compute $ W(\mathscr{S}, u_1)$ and before we do
this we need some preparations. For an arbitrary function matrix $A =
(a _{\mu   \nu})$ we prove that    
\begin{equation*}
(Y \frac{\partial} {\partial Y})' A Y = (Y(Y \frac{\partial}{\partial
    Y})'A)')' \to \frac{1}{2} A' Y + \frac{1}{2} \sigma (A Y) E
  \tag{421}\label{eq421} 
\end{equation*}

Indeed we know that 
$$
e_{\rho \mu} \frac{\partial y} {\partial y _{\rho \mu}} =\frac{1}{2}
(\delta _{\rho \tau} \delta_{\mu \nu} + \delta_{\rho \nu} \delta_{\mu
  \tau}) 
$$
and then 
\begin{align*}
(Y\frac{\partial} {\partial Y})' A Y  & = (\sum_\rho y_{\nu \rho} e
  _{\rho \mu} \frac{\partial}{\partial y_{\rho \mu}}) (\sum_ \tau a
  _{\mu \tau} y_{\tau \nu})\\ 
& = (\sum_{\rho \sigma \tau}) y_{\sigma \rho} e_{\sigma \mu}
  \frac{\partial}{\partial y_{\rho \mu}} a _{ \sigma \tau} y _{\tau
    \nu})\\ 
& = (\sum_{\rho \sigma \tau} y_{\mu \tau} y_{\sigma \rho} e_{\rho \nu}
  \frac{\partial}{\partial y_{\rho \nu}} a_{\sigma \tau})' +
  \frac{1}{2} (\sum_{\rho \sigma \tau} y_{\sigma \rho} a_{\sigma \tau}
  \delta_{\rho \tau} \delta_{\mu \nu})\\
& \hspace{2.5cm}  + \frac{1}{2} (\sum_{\rho
    \sigma \tau} y_{\sigma \rho} a_{\sigma \tau} \delta_{\rho \tau}
  \delta_{\mu \nu})\\ 
& = (Y((Y \frac{\partial} {\partial Y})'A)')' + \frac{1}{2} \sigma (A
  Y) E + \frac{1}{2} A'Y. 
\end{align*}

By induction on $\mathfrak{K}$ and by means of (\ref{eq421}) one easily
proves now that 
\begin{equation*}
(Y \frac{\partial}{\partial Y})'Y^\mathfrak{K} = \frac{\mathfrak{K}} {2}
  Y^\mathfrak{K} + \frac{1}{2} \sum^\mathfrak{K}_{\nu = 1} \sigma
  (Y^\nu) Y ^{\mathfrak{K}-\nu} (\mathfrak{K} \ge 1) \tag{422}\label{eq422} 
\end{equation*}

A direct\pageoriginale computation yields that
\begin{equation*}
(Y \frac{\partial}{\partial Y})' \sigma (Y^\mathfrak{K}) = \mathfrak{K}
  Y^\mathfrak{K} (\mathfrak{K} > 0 ) \tag{423}\label{eq423} 
\end{equation*}
and
\begin{equation*}
(Y \frac{\partial}{\partial Y})' e^{-2 \pi \sigma (Y)}{_{= -2 \pi
      e}}{^{-2 \pi \sigma (Y)}}_Y \tag{424}\label{eq424} 
\end{equation*}

By repeated applications of (\ref{eq422}) -- (\ref{eq424}) we obtain that 
\begin{gather*}
((Y \frac{\partial} {\partial Y})')^\mathfrak{K} e^{-2 \pi \sigma
    (Y)}{_{(- 2 \pi) ^\mathfrak{K} Y ^\mathfrak{K}}} + e^{- 2 \pi \sigma
    (Y)}\\
 \sum_{\nu_1 +\cdots+ \nu_r + \nu < \mathfrak{K}}
  a^\mathfrak{K}_{\nu_1 \nu_2 \dots \nu_r \nu} \nu \sigma (Y^\nu 1)
  \dots \sigma (Y^{\nu r}) Y^\nu \tag{425}\label{eq425} 
\end{gather*}
with certain constant coefficients $a^\mathfrak{K}{_{\nu_1 \nu_2 \dots 
    \nu_r \nu}}$. Taking traces, we have 
\begin{align*}
 \sigma ((Y \frac{\partial}
{\partial Y})')^\mathfrak{K} e^{-2 \pi \sigma (Y)} &=e^{-2 \pi \sigma
  (Y)} \{ (-2 \pi)^\mathfrak{K}\\[3pt]
&\quad  + \sum_{\nu_1 +\cdots+ \nu_\mathscr{S}
  < \mathfrak{K}}  \mathscr{C}^{\mathfrak{K}}_{\nu_1 \nu_2 \dots
    \nu_\mathscr{S}} \sigma (Y^\nu_1) \dots \sigma
(Y^{\nu_\mathscr{S}})\} 
\end{align*}
and this is easily generalised (by inductions
on $\rho$) to the following result, viz. 
\begin{align*}
& \sigma ((Y \frac{\partial} {\partial Y})')^{\nu_\rho} \sigma ((Y
\frac{\partial} {\partial Y})')^{\nu_{\rho-1}} \dots \sigma ((Y
\frac{\partial} {\partial Y})')^{\nu_1} e ^{-2 \pi \sigma (Y)} \\[4pt]
= & e^{2  \pi \sigma (Y)} \left\{ (-2 \pi)^{\nu_1 + \cdots+ \nu_\rho}
     {_{\sigma (Y^{\nu_1})} \dots \sigma (Y^{\nu_\rho})} +
     \mathscr{C}_{ \nu_1 \nu_2 \dots \nu_\rho }(Y) \right\} \tag{426}\label{eq426} 
\end{align*}
with
$$
\mathscr{C}_{\nu_1 \nu_2 \dots \nu_\rho}^{(Y)} = \sum_{\substack{\mu_1
    \dots \mu_r \\ \mu_1 +\cdots+ \mu_r < \nu_1 +\cdots+ \nu_\rho}}
C^{\nu_1 \nu_2 \dots \nu_\rho}_{\mu \mu_2 \dots \mu_r}
\sigma(Y^{\mu_1}) \sigma(Y^\mu_2) \dots \sigma (Y^\mu_r) 
$$

The\pageoriginale interesting fact is that the terms on the right side
of (\ref{eq426}) are all homogeneous (of different degrees) and the
first term is of the highest degree, viz $\nu_1 + \nu_2 + \cdots+
\nu_\rho$ while the degrees of the other terms are strictly less. The 
$\mathscr{C}^\mathfrak{K}_{\nu_1 \nu_2 \dots \nu_r}{'}\mathscr{S}$ and
$e^{\nu_1 \dots \nu_\rho}_{\mu_1 \dots \mu_r}{'}\mathscr{S}$ occurring
above are all constants.  

We now return to the integral formula (\ref{eq405}) and specialise that
functions $\varphi$ and $\psi$ occurring there as $\varphi = u_1 (Y),
\psi = \mathfrak{f}(Y) |Y|^\mathscr{S}$ where $u_1 (Y)$ is a bounded
solution of the differential equation system (\ref{eq362}) and $f$ is an
arbitrary function at our disposal. For $\mathscr{G}$ we now take the
whole domain $Y > 0$. By a suitable choice of $f (Y)$ we can obtain
the integral over the whole domain as a limit of integrals over
appropriate subdomains $\mathscr{G}$ such that in the limit, the
integrals over the boundary of $\mathscr{G}$ vanish. If $\lambda_1
\dots \lambda_n$ are the eigenvalues of $u_1 (Y)$, using (\ref{eq383}) we
obtain- from (\ref{eq405}) for $\mathfrak{K} \ge n$ that  
\begin{align*}
& -\lambda_\mathfrak{K} \int\limits_{ Y > o } \mathfrak{f} (Y
  |Y|^\mathscr{S} u_1 (Y) d \vartheta) =\\ 
& = (-1)^\mathfrak{K} \int\limits_{Y > o} u_1 (Y) \sigma ((Y
  \frac{\partial} {\partial Y})')^\mathfrak{K} |Y|^\mathscr{S} f (Y) d
  \vartheta\\ 
& = (-1)^\mathfrak{K} \sum_{\nu = o}^\mathfrak{K} \mathscr{S}^\nu
  (^\mathfrak{K}_\nu) \int\limits_{ Y > o} \{ \sigma ((Y
  \frac{\partial}{\partial Y})')^{\mathfrak{K}-\nu} \mathfrak{f} (Y) \}
  |Y|^\mathscr{S} u_1 (Y) d \vartheta\\ 
& = (-1)^\mathfrak{K} \sum_{\nu = o}^\mathfrak{K}
  \mathscr{S}^{\mathfrak{K}-\nu} (^\mathfrak{K}_\nu) \int\limits_{ Y > o} \{
  \sigma ((Y \frac{\partial}{\partial Y})')^{\mathfrak{K}-\nu}
  \mathfrak{f} (Y) \} |Y|^\mathscr{S} u_1 (Y) d \vartheta
  \tag{427}\label{eq427}  
\end{align*}
 
Finally\pageoriginale by induction on $\rho$ we obtain by means of
(\ref{eq427}) that for indices $\mathfrak{K}_1 ,\mathfrak{K}_2
,\ldots, \mathfrak{K}_\rho \ge n$  
\begin{align*}
& \lambda{_\mathfrak{K}}_{_1} \lambda{_\mathfrak{K}}_{_2} \dots
\lambda{_\mathfrak{K}}_{_\rho} \int\limits_{y > o} \mathfrak{f} (Y)
|Y|^\mathscr{S} u_1 (Y) d \vartheta = \\ 
& \hspace{4cm} (-1)^{\mathfrak{K}_1, \mathfrak{K}_2 +\cdots+
  \mathfrak{K}_\rho + \rho} \sum_{\nu_1 = 0}^{\mathfrak{K}_1}
\sum_{\nu_2 = 0}^{\mathfrak{K}_2} \dots \sum_{\nu_\rho =
  0}^{\mathfrak{K}_\rho} [ * ]  
\end{align*}
where
\begin{align*}
[*] & = \mathscr{S}^{\mathfrak{K}_1 +\mathfrak{K}_2 +\cdots+
  \mathfrak{K}_\rho -\nu_1 - \nu_2 \dots \nu_\rho} \left\{
(^{\mathfrak{K}_1}_{\nu_1}) (^{\mathfrak{K}_2}_{\nu_2}) \dots
(^{\mathfrak{K}_\rho}_{\nu_\rho})  \right.\\
\times & \int\limits_{Y > o} \{ \sigma ((Y
\frac{\partial} {\partial Y})')^{\nu_\rho} \sigma ((Y \frac{\partial}
     {\partial Y})')^{\nu_{\rho -1}}  \dots \sigma ((Y \frac{\partial}
     {\partial Y})')^{\nu_1} \mathfrak{f} (Y)\} |Y|^\mathscr{S} \\
& \hspace{5.5cm} \left. \times  u_1 (Y) d \vartheta
     \right\}\tag{428}\label{eq428}  
\end{align*}
choosing in particular $\mathfrak{f} (Y) = e^{-2 \pi \sigma (Y)}$ and
assuming that the real part of $\mathscr{S}$ is sufficiently large,
all the above steps can be justified, and using (\ref{eq426}) we finally
obtain that  
\begin{align*}
& \lambda{_\mathfrak{K}}_{_1} \lambda{_\mathfrak{K}}_{_2} \cdots
\lambda{_\mathfrak{K}}_{_\rho} e^{2 \pi \sigma (Y)}{_{|Y|^\mathfrak{K}
    u_1 (Y) d \vartheta}}  \\
= \; &  (-1)^{\mathfrak{K}_1 + \mathfrak{K}_2 +\cdots+ \mathfrak{K}_\rho - nu_1
  -\nu_2 -\cdots- \nu_\rho }\sum_{\nu_1 = 0} \sum_{\nu_2 = 0}^{K_2}
\dots \sum_{^\nu_\rho=0}^{\mathfrak{K}_\rho} \\
& \qquad (^{\mathfrak{K}_1}_{\nu_1})
(^{\mathfrak{K}_2}_{\nu_2}) \cdots (^{\mathfrak{K}_\rho}_{\nu_\rho})
\mathscr{S} ^{\mathfrak{K}_1 +\cdots+ \mathfrak{K}_\rho -\nu_1 \dots
  \nu_\rho [*]}  
\end{align*}
where
\begin{align*} 
[*]&  = (-2 \pi) ^{\nu_1 + \nu_2 +\dots+ \nu_\rho} \int\limits_{Y > o} e
^{- 2 \pi \sigma (Y)}{_{\sigma (Y^{\nu_1})}} \dots \sigma
(Y^{\nu_\rho}) |Y|^\mathscr{S} u_1 (Y) 0 \\
& \qquad + \sum_{\substack{\mu_1 \dots
    \mu_r \\ \mu_1 +\cdots+\mu_r < \nu_1 + \cdots \nu_\rho}} C^{\nu_1
  \nu_2 \dots \nu_\rho} _{\mu_1 \mu_2 \dots \mu_2} \int\limits_{Y>
  0}e^{- 2 \pi \sigma (Y)}{_{\sigma (Y ^{\mu_1})}} \dots\\[4pt]
& \hspace{4cm} \sigma(Y^{\nu_r}) |Y|^\mathfrak{K} u_1(y) d
\vartheta \tag{429}\label{eq429}   
\end{align*}\pageoriginale

Introduce again $y$ and $Y_1$ in place of $Y$ by
$$
Y = y Y_1 , |Y| = y^n , y > o
$$
and denotes as usual by $d \vartheta_1$ the invariant volume element
of the determinant surface $Y_1 > 0 |Y_1| =1$. Then from (\ref{eq383}) 
$$
d \vartheta = \sqrt{n}  y ^{-1} d y d \vartheta_1.
$$

In the sequel we consider only homogeneous functions $u_1 (Y)$ of
degree 0 i.e., we assume $\lambda_1 = 0$. In (\ref{eq429}), after the above
substitutions, the integral in $y$ is a $y$ integral. Carrying
out this integration over $y$ we obtain that  
\begin{align*}
& \int\limits_{Y > o} e^{- 2 \pi \sigma (Y)}{\sigma (Y^{\nu_1})}\dots (\sigma
(Y^{\nu_\rho}) |Y|^\mathscr{S} u_1 (y) d \vartheta\\
 =  & \sqrt{n}
y (n \mathscr{S} + \nu_1 +\cdots+ \nu_\rho) (2 \pi) ^{n
  \mathscr{S} - \nu_1 \cdots \nu_\rho } \\
&  \qquad \times \int\limits_{Y_1 > 0}
(\sigma(Y_1))^{-n \mathscr{S} -\nu_1 \cdots v\nu_\rho} \sigma
(Y_1^{\nu_1}) \dots \sigma (Y_1^{\nu_\rho}) u_1 (y_1) d \vartheta_1 
\end{align*}
Let
\begin{align*}
& \mathcal{J}_{\nu_1 \nu_2 \ldots \nu_\rho} (\mathscr{S},u_1) =
\frac{\Lambda (n \mathscr{S} + \nu_1 + \cdots + \nu_{\rho})}{\Lambda
  (n\mathscr{S})} \\
& \qquad   \times \int\limits_{Y_1 > O}
u_1(Y_1)(\sigma(Y_1))^{-n\mathscr{S}-\nu_1 - \cdots - \nu_{\rho}}
\sigma(Y^{\nu_1}_1) \ldots \sigma(Y^{\nu_\rho}_1) d\nu_1 \tag{430}\label{eq430} 
\end{align*}\pageoriginale
and 
\begin{equation*}
\mathcal{J}(\mathscr{S},u_1) = \int\limits_{Y_1 > O}  u_1(Y_1)
(\sigma(Y_1))^{-n\mathscr{S}}   d\nu_1 \tag{431}\label{eq431} 
\end{equation*}

Then (\ref{eq429}) implies that 
\begin{gather*}
\lambda_{\mathfrak{K}_1}   \lambda_{\mathfrak{K}_2} \ldots
\mathfrak{K}(\mathscr{S},u_1) = (-1)^{\mathfrak{K}_1 + \cdots +
  \mathfrak{K}_\rho + \rho} \sum_{\nu_1 = O}^{\mathfrak{K}_1}
\sum_{\nu_2 = O}^{\mathfrak{K}_2} \ldots\\ 
\ldots \sum_{\nu_\rho =
  O}^{\mathfrak{K}_\rho} \begin{pmatrix}\mathfrak{K}_1
  \\ \nu_1 \end{pmatrix}\begin{pmatrix}\mathfrak{K}_2
  \\ \nu_2 \end{pmatrix} \ldots \begin{pmatrix}\mathfrak{K}_\rho
  \\ \nu_\rho \end{pmatrix}   \mathscr{S}^{\mathfrak{K}_1} + \cdots +
\mathfrak{K}_\rho - \nu_1 - \cdots -\nu_\rho   \times \\ 
\times   \big\{ (-1)^{\nu_1 + \cdots + \nu_\rho}   \mathcal{J}_{\nu_1
  \cdots \nu_\rho}   (\mathscr{S},u_1) + \sum_{\substack{\mu_1 \ldots
    \mu_{r} \\ \mu_1 + \cdots + \mu_r    <   \nu_1 + \cdots +
    \nu_\rho}}   \\[4pt]
C^{\nu_1 \nu_2 \ldots \nu_\rho}_{\mu_1 \mu_2 \ldots
  \mu_r}   (2\pi)^{-\mu_1 - \cdots - \mu_r}   \mathcal{J}_{\mu_1
  \ldots \mu_r}   (\mathscr{S},u_1) \tag{432}\label{eq432} 
\end{gather*}

We therefore obtain for the integral corresponding to the indices\break
$(\nu_1,\nu_2 \ldots \nu_\rho) = (\mathfrak{K}_1,\mathfrak{K}_2,\ldots
\mathfrak{K}_\rho)$ a representation of the kind  
\begin{align*}
& \mathcal{J}_{\mathfrak{K}_1 \mathfrak{K}_2 \cdots \mathfrak{K}_\rho}
(\mathscr{S},u_1) = (-1)^{\rho}   \lambda_{\mathfrak{K}_1}
\lambda_{\mathfrak{K}_2}\ldots \lambda_{\mathfrak{K}_\rho}
\mathcal{J}(\mathscr{S},u_1) \\
&  \sum_{\substack{\mu_1 \ldots \mu_r
    \\ \mu_1 + \cdots + \mu_r   <   \mathfrak{K}_1 + \ldots +
    \mathfrak{K}_\rho }}   P^{\mu_1 \mu_2 \ldots \mu_r}_{\mathfrak{K}_1
  \mathfrak{K}_2 \ldots \mathfrak{K}_\rho}   (\mathscr{S})
\mathcal{J}_{\mu_1 \mu_2 \ldots \mu_r}   (\mathscr{S},u_1)
\tag{433}\label{eq433}  
\end{align*}
where\pageoriginale $P^{\mu_1 \ldots \mu_r}_{\mathfrak{K}_1 \ldots
  \mathfrak{K}_\rho}$ is either 0 or is a polynomial in $\mathscr{S}$
of degree atmost $\mathfrak{K}_1 + \mathfrak{K}_2 + \cdots +
\mathfrak{K}_\rho - \mu_1 - \mu_2 - \cdots \mu_r$. Since the
coefficients $C^{\nu_1 \nu_2 \ldots 
  \nu_\rho}_{\mu_1 \mu_2 \ldots \mu_r}$ by their very definition, do
not depend on the eigen values $\lambda_1, \lambda_2 \ldots
\lambda_n$, this is true also of the polynomials $P^{\mu_1 \ldots
  \mu_\rho}_{\mathfrak{K}_1 \ldots \mathfrak{K}_\rho}
(\mathscr{S})$. The method of induction on $\mathfrak{K}_1 +
\mathfrak{K}_2 + \cdots + \mathfrak{K}_\rho$ yields by virtue of (\ref{eq433})
that  
\begin{equation*}
\mathcal{J}_{\mathfrak{K}_1 \mathfrak{K}_2 \ldots \mathfrak{K}_\rho}
(\mathscr{S},u_1) = q_{\mathfrak{K}_1 \mathfrak{K}_2 \ldots
  \mathfrak{K}_\rho}   (\mathscr{S})   \mathcal{J}(\mathscr{S},u_1)
\tag{434}\label{eq434} 
\end{equation*}
where $q_{\mathfrak{K}_1\mathfrak{K}_\rho}(\mathscr{S})$ is again either
$0$ or is a polynomial in $\mathscr{S}$ of degree at most
$\mathfrak{K}_1 + \mathfrak{K}_2 + \cdots + \mathfrak{K}_\rho$. We now
show that  
\begin{equation*}
q_{\mathfrak{K}_1 \mathfrak{K}_2 \ldots \mathfrak{K}_{\rho}}
(\mathscr{S}) = n^{\rho}\mathscr{S}^{\mathfrak{K}_1 +  \cdots +
  \mathfrak{K}_\rho } + \text{ lower powers of }
\mathscr{S},\tag{435}\label{eq435}  
\end{equation*}
in other words that $q_{\mathfrak{K}_1 \cdots
  \mathfrak{K}_\rho}(\mathscr{S}$ is of degree exactly $\mathfrak{K}_1 +
\cdots + \mathfrak{K}_\rho$ and that the coefficient of the highest
degree term is precisely $n^{\rho}$. The proof is again by induction
on $\mathfrak{K}_1 + \cdots +  \mathfrak{K}_\rho$. If $\mathfrak{K}_1 +
\cdots +   \mathfrak{K}_\rho = O$ whence $\mathfrak{K}_1  =
\mathfrak{K}_2 = \cdots = \mathfrak{K}_\rho = O$ we have obviously
$q_{\mathfrak{K}_1 \ldots \mathfrak{K}_{\rho}} = n^\rho$, ensuring that
the start is good. Assume now that $\mathfrak{K}_1 + \cdots +
\mathfrak{K}_\rho   >   O$ and  
\begin{equation*}
q_{\mu_1 \mu_2 \ldots \mu_r}   (\mathscr{S}) = n^r
\mathscr{S}^{\mu_1 + \cdots + \mu_r} + \qquad \text{ lower powers of }
\mathscr{S}. \tag{436}\label{eq436} 
\end{equation*}
for
$$
\mu_1 + \mu_2 + \cdots \mu_r   <   \mathfrak{K}_1 + \mathfrak{K}_2 +
\cdots + \mathfrak{K}_\rho. 
$$

If 
\begin{equation*}
q_{\mathfrak{K}_1 \mathfrak{K}_2 \ldots \mathfrak{K}_\rho}
(\mathscr{S}) = C   \mathscr{S}^{\mathfrak{K}_1 + \cdots
  \mathfrak{K}_\rho +} \qquad \text{ lower powers of } \mathscr{S}
\tag{437}\label{eq437} 
\end{equation*}
we have to show that $C = n^\rho$.

The application of (\ref{eq434}) -- (\ref{eq437}) in (\ref{eq432})
yields a polynomial 
identity. The power $\mathscr{S}^{\mathfrak{K}_1 + \mathfrak{K}_2 +
  \cdots + \mathfrak{K}_\rho}$ appears only on one side of this
identity so that\pageoriginale the coefficient of this power must
necessarily vanish. This requires that   
\begin{align*}
& \sum_{\substack{\nu_1 = O \\ \nu_1 + \cdots + \nu_\rho   <
    \mathfrak{K}_1 + \cdots + \mathfrak{K}_\rho}}^{\mathfrak{K}_1} \cdots
\sum_{\nu_\rho = O}^{\mathfrak{K}_\rho}   (-1)^{\nu_1 + \cdots +
  \nu_\rho} \begin{pmatrix}\mathfrak{K}_1 \\ \nu_1 \end{pmatrix}
\cdots \begin{pmatrix}\mathfrak{K}_\rho \\ \nu_\rho \end{pmatrix}
n^{\rho} \\
& \hspace{5cm} + (-1)^{\mathfrak{K}_1 + \cdots + \mathfrak{K}_\rho}   C = O 
\end{align*}

Since this equation determines $C$ uniquely and if we substitute $C$
by $n^\rho$ this equation is satisfied - at least one
$\mathfrak{K}_\nu$ being greater than 0 - we conclude that $C =
n^{\rho}$. 

We observe more over that the coefficient of
$\mathscr{S}^{\mathfrak{K}_1 + \cdots + \mathfrak{K}_\rho -1}$ in\break
$q_{\mathfrak{K}_1\mathfrak{K}_2 \ldots \mathfrak{K}_\rho} (\mathscr{S}$
does not depend on the eigen values $\lambda_1, \lambda_2 \ldots
\lambda_n$. While this is evident in case $\mathfrak{K}_1 + \cdots +
\mathfrak{K}_\rho = 1$, as in this case $q_{\mathfrak{K}_1 \ldots
  \mathfrak{K}_\rho} (\mathscr{S}) = n^{\rho}\mathscr{S}$, the general
result follows by induction on applying (\ref{eq434}) in (\ref{eq433}).  

Now for any positive matrix $Y$, $|Y|$ is a rational function of the
sums of powers of the characteristic roots $\lambda_1, \lambda_2
\cdots \lambda_n$ of $Y$, and by a standard result on elementary
symmetric functions, it is then a rational function of $\sigma (Y^\nu)
( = \sum_{\mu = 1}^n   \lambda^\nu_\mu), \nu = 1,2,\ldots n$. In
particular $\sigma(Y^n)$ can be expressed as a function of $| Y | $
and $\sigma(Y^\nu)$, $\nu < n$. If we take $Y = Y_1$ with $| Y_1 | = 1$,
then $\sigma(Y_1^n)$ is a function of $\sigma(Y^\nu_1)$, $\nu < n$, and
this function can be determined to be of the form 
\begin{equation*}
\sigma(Y^n_1) = (-1)^{n+1}   n  + \sum_{\substack{\nu_1  \cdots \nu_n
    < n \\ \nu_1 + \cdots + \nu_r = n}}   d_{\nu_1\nu_2 \cdots \nu_r}
\sigma(Y^{\nu_1}_{1}) \cdots \sigma(Y^{\nu_1}_1) \tag{438}\label{eq438} 
\end{equation*}
with\pageoriginale constant coefficients $d_{\nu_1\nu_2 \ldots \nu_r}$.

Now by definition
\begin{equation*}
\mathcal{J}_n(\mathscr{S},u_1) = \frac{\Lambda (n\mathscr{S}+
  n)}{\Lambda (n\mathscr{S})} \int\limits_{Y_1 > O}
u_1(Y_1)(\sigma(Y_1))^{-n\mathscr{S}-r} \sigma(Y^n_1)d\nu_1
\tag{439}\label{eq439} 
\end{equation*}
and using (\ref{eq438}) we have 
\begin{align*}
\mathcal{J}_n(\mathscr{S},u_1) & = (-1)^{n+1}   n \frac{\Lambda(n
  \mathscr{S}+n)}{\Lambda{(n\mathscr{S})}} \mathcal{J}(\mathscr{S}+
1,u_1) \\ 
& \qquad + \sum_{\substack{\nu_1\nu_2 \ldots \nu_r < n \\ \nu_1 + \cdots
    + \nu_r=n}}   d_{\nu_1\nu_2 \ldots \nu_r}
\mathcal{J}_{\nu_1\nu_2 \ldots \nu_r} (\mathscr{S},u_1) 
\end{align*}

Hence
\begin{equation*}
\frac{\Lambda (n\mathscr{S}+ n)}{\Lambda (n\mathscr{S})}
\mathcal{J}(\mathscr{S}+1,u_1) = q(\mathscr{S})
\mathcal{J}(\mathscr{S},u_1) \tag{440}\label{eq440} 
\end{equation*}
with 
\begin{equation*}
q(\mathscr{S}) = (-1)^{n+1} \frac{1}{n}(q_n(\mathscr{S}) -
\sum_{\substack{\nu_1 \ldots \nu_r < n \\ \nu_\rho + \cdots + \nu_r =
    n}} d_{\nu_1 \cdots \nu_r} q_{\nu_1 \cdots \nu_r} (\mathscr{S}
\tag{441}\label{eq441} 
\end{equation*}
$q(\mathscr{S})$ is polynomial in $\mathscr{S}$ of degree $n$. The
power $\mathscr{S}^n$ appears in $q(\mathscr{S})$ with the coefficient 
$$
(-1)^{n+1}\frac{1}{n}\big( n - \sum_{\substack{\nu_1 \cdots \nu_r < n
    \\ \nu_\rho + \cdots + \nu_r = n}} d_{\nu_1\nu_2 \ldots \nu_r}
\quad n^r \big) 
$$
and it is immediate from (\ref{eq438}) (setting $Y_1 = E$) that this
coefficient is\pageoriginale exactly equal to 1. Hence
$q(\mathscr{S})$ can be factored as   
\begin{equation*}
q(\mathscr{S}) = (\mathscr{S}-d_1)(\mathscr{S}-d_2) \cdots
(\mathscr{S}-d_n) \tag{442}\label{eq442} 
\end{equation*}
with certain complex constants $d_1,d_2 \ldots d_n$, and from our
earlier statements concerning the coefficient of $\mathscr{S}^{\nu_1+
  \nu_2 + \cdots + \nu_\rho - S}$ in $q_{\nu_1\nu_2 \ldots
  \nu_\rho}(\mathscr{S})$ we infer that the coefficient of
$\mathscr{S}^{n-1}$ in $q(\mathscr{S})$ viz. $\pm \mathscr{L}_1 +
\mathscr{L}_2 + \cdots + d_n$ does not depend on the eigen values
$\lambda_1 \ldots \lambda_n$. By means of the functional equation for
$\Lambda (\mathscr{S})$, viz. $\Lambda (\mathscr{S}+1) =
\mathscr{S}\Lambda(\mathscr{S}$ we finally obtain from (\ref{eq440}) that  
\begin{equation*}
\mathcal{J}(\mathscr{S}+ 1, u_1) = \frac{(\mathscr{S}-\alpha_1)
  \cdots (\mathscr{S}-\alpha_n )}{n^n \mathscr{S}(\mathscr{S} +
  \frac{1}{n} )\cdots (\mathscr{S}+ \frac{n-1}{n})}
\mathcal{J}(\mathscr{S},u_1) \tag{443}\label{eq443} 
\end{equation*}

We shall use this transformation formula for computing
$\Lambda(\mathscr{S},u_1)$ explicitly, by a method which Huber
developed recently for the case $n = 2$. Clearly the function 
\begin{equation*}
G(\mathscr{S}) = \frac{\Lambda(\mathscr{S}-d_1)
  \Lambda(\mathscr{S}-d_2)\ldots
  \Lambda(\mathscr{S}-d_n)}{n^{n\mathscr{S}}\Lambda
  (\mathscr{S})(\mathscr{S}+ \frac{1}{n})\ldots \Lambda(\mathscr{S}+
  \frac{n-1}{n})} \tag{444}\label{eq444} 
\end{equation*}
satisfies the same transformation formula as $\Lambda
(\mathscr{S},u_1)$ does in (\ref{eq443}) so that, if we set  
\begin{equation*}
H(\mathscr{S},u_1) =
\frac{\mathcal{J}(\mathscr{S},u_1)}{G(\mathscr{S})} \tag{445}\label{eq445} 
\end{equation*}
then $H(\mathscr{S},u_1)$ is a periodic function of $\mathscr{S}$ and  
\begin{equation*}
H(\mathscr{S}+ 1, u_1) = H(\mathscr{S},u_1) \tag{446}\label{eq446}
\end{equation*}

For\pageoriginale values of $\mathscr{S}$ for which the real part of
$\mathscr{S}$ is sufficiently large, $\mathcal{J}(\mathscr{S},u_1)$
and $G(\mathscr{S})$ are regular functions of $\mathscr{S}$ and then
this is true also of $H(\mathscr{S})$. Since $H(\mathscr{S})$ is further
periodic, it is an entire function. We wish to show that this entire
function is actually a constant depending only upon $u_1(E)$. For this
we need some asymptotic formulae. 

It is well known that 
$$
\Lambda(\mathscr{S}-\alpha) \sim \sqrt{2\pi}
\mathscr{S}^{\mathscr{S}-\alpha - \frac{1}{2}}    e^{-\mathscr{S}}
\qquad (\mathscr{S}\rightarrow \infty) 
$$
and it follows that 
\begin{equation*}
G(\mathscr{S}) \sim   n^{-n\mathscr{S}} \mathscr{S}^{-\alpha_1 -
  \alpha_2 - \cdots - \alpha_n - \frac{n-1}{2}} (\mathscr{S}
\rightarrow \infty)I. \tag{447}\label{eq447} 
\end{equation*}

We shall now determine the value of the sum $\alpha_1 + \alpha_2 +
\cdots + \alpha_n$. We know that this sum is independent of the eigen
values $\lambda_1,\lambda_2 \ldots \lambda_n$, in other words it is
independent of the function $u_1$. We can then choose $u_1 \equiv 1$
and obtain by means of Lemma \ref{chap8:lem14} in view of (\ref{eq383}) that 
\begin{align*}
& (2\pi)^{-n\mathscr{S}+\frac{n(n-1)}{4}}
  \Lambda(\mathscr{S})\Lambda(\mathscr{S}-\frac{1}{2}) \cdots
  \Lambda(\mathscr{S}-\frac{n-1}{2}) = \int\limits_{Y > O}
  e^{-2\pi\sigma(Y)}   | Y |^{\mathscr{S}}    d\nu\\ 
 & = \sqrt{n} \int\limits_{Y_1 > O} \int\limits_{Y > O}   e^{-2\pi   y
    \sigma(Y_1)}   y^{n \mathscr{S}-1}   dy   d\vartheta_1 \\ 
 & = \sqrt{n} (2\pi)^{-n\mathscr{S}} \Lambda(\mathscr{S},1).
\end{align*}

Hence\pageoriginale 
$$
\mathcal{J}(\mathscr{S},1) = (2\pi)^{\frac{n(n-1)}{4}}
\frac{\Lambda(\mathscr{S}) \Lambda(\mathscr{S}-\frac{1}{2}) \cdots
  \Lambda(\mathscr{S}-\frac{n-1}{2})}{\sqrt{n}   \Lambda
  (n\mathscr{S})} 
$$

Since by the Gaussian multiplication formula,
\begin{equation*}
\Lambda ( n \mathscr{S} ) = \frac{n^{n\mathscr{S}-\frac{1}{2}}}{(2\pi
  )^{(n-1)/2}}   \Lambda(\mathscr{S}) \Lambda(\mathscr{S}_ \frac{1}{n}
) \cdots \Lambda(\mathscr{S}- \frac{n-1}{n}) \tag{448}\label{eq448} 
\end{equation*}
the above gives that 
\begin{equation*}
\mathcal{J}(\mathscr{S},1) = (2\pi)^{\frac{(n+2)(n-1)}{4}}
\frac{\Lambda(\mathscr{S})\Lambda(\mathscr{S}- \frac{1}{2}\cdots
  \Lambda(\mathscr{S}- \frac{n-1}{2})}
     {n^{n\mathscr{S}}(\mathscr{S})\Lambda(\mathscr{S}+ \frac{1}{2})
       \cdots \Lambda(\mathscr{S}+ \frac{n-1}{n})} \tag{449}\label{eq449} 
\end{equation*}
and consequently
$$
\frac{\mathcal{J}(\mathscr{S} + 1.1)}{\mathcal{J}(\mathscr{S},1)} =
\frac{\mathscr{S}(\mathscr{S}-\frac{1}{2})\cdots
  (\mathscr{S}-\frac{n-1}{n})}{n^n
  \mathscr{S}(\mathscr{S}+\frac{1}{n}) \cdots (\mathscr{S}+
  \frac{n-1}{n})} 
$$

A comparison with (\ref{eq443}) now shows that
\begin{equation*}
\alpha_1 + \alpha_2 + \cdots + \alpha_n =
\frac{n(n-1)}{4}. \tag{450}\label{eq450} 
\end{equation*}

Substituting this in (\ref{eq447}) we have
\begin{equation*}
G(\mathscr{S}) \sim n^{-n\mathscr{S}} \mathscr{S}^{-n+z)(n-1)/4}
\qquad (\mathscr{S}\rightarrow \infty) \tag{451}\label{eq451} 
\end{equation*}
and in particular, the asymptotic value of $G(\mathscr{S})$ is
independent of the eigen values $\lambda_1,\lambda_2 \ldots
\lambda_n$. 

Also it follows from (\ref{eq449}) that 
\begin{equation*}
\mathcal{J}(\mathscr{S},1) \sim (2\pi)^{n+2)(n-1)/4}
n^{-n\mathscr{S}}   \mathscr{S}^{-(n+2)(n-1)/4} \tag{452}\label{eq452} 
\end{equation*}
and\pageoriginale against (\ref{eq451}) this gives
\begin{equation*}
\mathcal{J}(\mathscr{S},1) \sim (2\pi)^{(n+2)(n-1)/4}   G(\mathscr{S})
(\mathscr{S}\rightarrow \infty) \tag{453}\label{eq453} 
\end{equation*}

It is also clear from (\ref{eq451}) that
\begin{equation*}
G(\mathscr{S}+\mathfrak{K}) \sim n^{-n\mathscr{S}}   G(\mathfrak{K})
\qquad (\mathfrak{K} \rightarrow \infty )  \tag{454}\label{eq454} 
\end{equation*}
where we let $\mathfrak{K}$ to tend to $\infty$ through all rational
values. With \break $\omega(Y) = u_1(Y)(\frac{\sigma(Y)}{n})^{-\mathscr{S}}$
we can now state from (\ref{eq446}) that $H(\mathscr{S}+\mathfrak{K},u_1) =
H(\mathscr{S},u_1)$ for every integer $\mathfrak{K}$. If now
$\mathscr{S} \rightarrow \infty$ through all rational \break values, say, it
is clear in view of (\ref{eq454}) that 
\begin{align*}
H(\mathscr{S},u_1) = \lim_{\mathfrak{K}\rightarrow \infty}
H(\mathscr{S}+\mathfrak{K},u_1) & = \lim_{\mathfrak{K} \rightarrow
  \infty}
\frac{\mathcal{J}(\mathscr{S}+\mathfrak{K},u_1)}{G(\mathscr{S}+
  \mathfrak{K})}\\ 
& = \lim_{\mathfrak{K}\rightarrow \infty}
\frac{\mathcal{J}(\mathscr{S}+\mathfrak{K},u_1)}{n^{-n\mathscr{S}}G(\mathscr{S})}
\tag{455}\label{eq455} 
\end{align*}

Now
\begin{gather*}
n^{n\mathscr{S}}   \mathcal{J}(\mathscr{S}+ \mathfrak{K},u_1) =
n^{n\mathscr{S}} \int\limits_{Y_1 > O}
\sigma(Y_1)^{-n\mathscr{S}-n\mathfrak{K}}   u_1(Y)   d\vartheta_1\\ 
= \int\limits_{Y_1 > O}   u_1(Y) (\frac{\sigma(Y)}{n})^{\mathscr{S}}
(\sigma(Y_1))^{-n\mathfrak{K}}   d\vartheta_1 =
\mathcal{J}(\mathfrak{K},\omega). 
\end{gather*}

Hence we conclude from (\ref{eq455}) by means of (\ref{eq454}) that 
\begin{equation*}
H(\mathscr{S},u_1) = \lim_{\mathfrak{K} \rightarrow \infty}
\frac{\mathcal{J}(\mathfrak{K},\omega)}{G(\mathfrak{K})} =
(2\pi)^{(n+2)(n-1)/4} \lim_{\mathfrak{K} \rightarrow \infty}
\frac{\mathcal{J}(\mathfrak{K},\omega)}{\mathscr{J}(\mathfrak{K},1)}
\tag{456}\label{eq456} 
\end{equation*}

We\pageoriginale now prove for continuous and bounded functions
$\omega(Y)$ that  
\begin{equation*}
\lim_{\mathfrak{K} \rightarrow \infty}
\frac{\mathcal{J}(\mathfrak{K},\omega)}{\mathcal{J}(\mathfrak{K},1)} =
\omega(E) \tag{457}\label{eq457} 
\end{equation*}

In view of the linearity of $\mathcal{J}(\mathfrak{K},\omega)$ in
$\omega$ we can further assume that $\omega(E) = 0$ as in the
alternative case we need only argue with the function $\omega(Y) -
\omega (E)$ in the place of $\omega (Y)$. Representing $\sigma (Y)$
and $|Y|$ in terms of the characteristic roots of $Y$ one easily sees
that $\sigma (Y) \ge n \sqrt{|Y|}$ for $Y > 0$. In particular, $\sigma
(Y_1) \ge n$ for $Y_1 > 0$, $| Y_1 | = 1$, and the equality is true
only for $Y_1 = E$. This implies that if $\sigma(Y_1) \rightarrow n$
and $Y_1 \rightarrow A$ then $\sigma (A) = n$ and consequently $A =
E$. Hence given $\varepsilon > 0$ we can find $\delta = \delta
(\varepsilon) > 0$ such 
that $|\omega (Y_1) | < \in $ for $\sigma (Y_1) < n(1+\delta)$. Let
$|\omega (Y_1)| \le \vartheta $ for all $Y_1$. (By assumption $\omega$
is bounded). 

Then,
$$
\frac{\mathcal{J}(\mathfrak{K},\omega)}{\mathcal{J}(\mathfrak{K},1)} =
\frac{\int\limits_{y_1 > O}\omega(Y_1)\sigma(Y_1)^{-n\mathfrak{K}}   d
  \vartheta}{\int\limits_{Y_1 > O} \sigma(Y_1)^{-n\mathfrak{K}}
  d\vartheta_1}, 
$$
and hence
{\fontsize{10pt}{12pt}\selectfont
\begin{align*}
\left|
\frac{\mathcal{J}(\mathfrak{K},\omega)}{\mathcal{J}(\mathfrak{K},1)}
\right| 
& \le    \varepsilon   \frac{\int\limits_{\sigma(Y_1)   \le   n(1 +
    \delta)}   (\sigma(Y_1)^{-n\mathfrak{K}}
  d{\vartheta_1}}{\int\limits_{Y_1 > O}(\sigma(Y_1))^{-n\mathfrak{K}}
  d\vartheta_1} + \vartheta   \frac{ \int\limits_{\sigma(Y_1) > n(1 +
    \delta)} (\sigma(Y_1))^{-n\mathfrak{K}}
  d\vartheta_1}{\int\limits_{Y_{1} > O} ( \sigma (Y_1)
  )^{-n\mathfrak{K}}   d \vartheta_1 } \cdots\\ 
& \le   \varepsilon + \vartheta(1 + \delta)^{-\frac{n\mathfrak{K}}{2}}
n^{-\frac{n\mathfrak{K}}{2}}
\frac{\mathcal{J}(\frac{\mathfrak{K}}{2},1)}{\mathcal{J}(\mathfrak{K},1)} 
\end{align*}}\relax\pageoriginale

From (\ref{eq452}) we have
$$
n^{-\frac{n\mathfrak{K}}{2}}
\frac{\mathcal{J}(\frac{\mathfrak{K}}{2},1)}{\mathcal{J}(\mathfrak{K},1)}
\sim   2^{(N+2)(N-1)/4}   (\mathfrak{K} \rightarrow \infty) 
$$
and then
$$
\left |
\frac{\mathcal{J}(\mathfrak{K},\omega)}{\mathcal{J}(\mathfrak{K},1)}
\right |   \sim   2 \varepsilon \text{ for } \mathfrak{K} \ge
\mathfrak{K}_O(\varepsilon) 
$$

In other words,
$$
\lim_{\mathfrak{K}\rightarrow \infty}
\frac{\mathcal{J}(\mathfrak{K},\omega)}{\mathcal{J}(\mathfrak{K},1)} = O
= \omega(E). 
$$

Applying (\ref{eq457}) in (\ref{eq456}) we obtain that 
$$
H(\mathscr{S},u_1) = (2\pi)^{(n+2)(n-1)/4} \omega(E).
$$

But $\omega(E) - \mu_1(E)$ and thus
\begin{equation*}
H(\mathscr{S},u_1) = (2\pi )^{(n+2)(n-1)/4} u_1(E). \tag{458}\label{eq458}
\end{equation*}

After all these, we are in a position to face the integral
$W(\mathscr{S},u_1)$ in (\ref{eq419}). We have 
\begin{align*}
W(\mathscr{S},u_1) & = \int\limits_{y > O} e^{-2\pi\sigma(Y)} u_1(Y) |
Y |^\mathscr{S}   d\vartheta\\ 
& = \sqrt{n}(2\pi)^{-n\mathscr{S}}  \Lambda (n \mathscr{S})
\mathcal{J}(\mathscr{S},u_1)\\ 
& = \sqrt{n}(2\pi)^{-n\mathscr{S}}  \Lambda
G(n\mathscr{S})H(\mathscr{S},u_1) \cdots\\ 
& = (2\pi )^{-n\mathscr{S}}   \Lambda (n\mathscr{S})
n^{-n\mathscr{S}} \frac{\Lambda(\mathscr{S} -d_1) \cdots
  \Lambda(\mathscr{S}-d_n)}{\Lambda(\mathscr{S})\Lambda(\mathscr{S}+
  \frac{1}{n}) \cdots \Lambda (\mathscr{S} + \frac{n-1}{n})}\\
& \qquad  (2n)
\frac{(n+2)(n)}{\begin{matrix}4 \\ \sqrt{n}   u_1 \end{matrix}} 
\end{align*}\pageoriginale

We know from (\ref{eq418}) that $u_1(E) = u(T^{-1}) = u^*(T)$ in the
notation in (p. \pageref{p.279}). Using again the Gaussian
multiplication formula 
(\ref{eq448}) the above gives that  
\begin{equation*}
W(\mathscr{S},u_1) = (2\pi )^{-n\mathscr{S}+ \frac{n(n-1)}{4}}
\Lambda(\mathscr{S} - d_1) \cdots   \Lambda(\mathscr{S} - d_n)   u^*
(T) \tag{459}\label{eq459} 
\end{equation*}

Then from (\ref{eq420})
\begin{align*}
\xi(\mathscr{S},u) & = \frac{2}{\sqrt{n}} \sum_{ \{ T \} }
\frac{a(T)}{\varepsilon(T)}   | T |^{-\mathscr{S}}
W(\mathscr{S},u_1)\\ 
& = \frac{2}{\sqrt{n}}(2\pi)^{-n\mathscr{S} + \frac{n(n-1)}{4}}
\Lambda{(\mathscr{S} - d_1)} \cdots \Lambda(\mathscr{S}-dn)
D(\mathscr{S},u) \tag{460}\label{eq460} 
\end{align*}
where
\begin{equation*}
D(\mathscr{S},u) = \sum_{\{ T \}} \frac{a(T)u^*(T)}{\varepsilon(T)}
| T |^{- \mathscr{S}}. \tag{461}\label{eq461} 
\end{equation*}

The next question is whether the function $D(\mathscr{S},u)$ defined
by the above Dirichlet series for values of $\mathscr{S}$ whose real
parts are sufficiently large, can be continued analytically in the
whole plane. We have 
$$
\xi (\mathscr{S},u) = \frac{1}{\sqrt{n}} \int\limits_{y \in
  \mathfrak{K}} \mathfrak{f}_n (Y) u(Y) |Y|^{\mathscr{S}}   d\vartheta. 
$$

We can take in the place of $\mathfrak{K}$ a domain which is invariant
relative to the transformation\pageoriginale $Y \to Y^{-1}$. Since the
volume element $d \vartheta$ is also invariant relative to this
transformation we can write  
\begin{align*}
 \xi (\mathscr{S}, u) & = \frac{1}{\sqrt{n}} \int\limits_{\substack{Y
    \in  \mathfrak{K} \\ | Y |  \ge   1}} \mathfrak{f}_n (Y^{-1}) | Y
|^{-\mathscr{S}} u^* (Y)   d   \vartheta\\
& \qquad  + \frac{1}{\sqrt{n}}
\int\limits_{\substack {Y   \in   \mathfrak{K} \\   | y |   \ge   1}}
\mathfrak{f}_n   (Y)   | Y |^{\mathscr{S}}   u(Y)   d \vartheta
\tag{462}\label{eq462}  
\end{align*}

 By the transformation formula for modular forms we have
 $\mathfrak{f}(-Z^{-1})=|Z |^{\mathfrak{K}} \mathfrak{f} (Z)$ and in
 particular, $\mathfrak{f}   (i Y^{-1}) = i^{n \mathfrak{K}}    | Y
 |^\mathfrak{K}   \mathfrak{f}   (i Y)$. Then,   
 $$
 \sum^n_{r=o}    \mathfrak{f}_{r}   (Y^{-1}) = \mathfrak{f}(iY^{-1})=
 i^{n \mathfrak{K}}   | Y |^{\mathfrak{K}}    \sum^n_{r=o}
 \mathfrak{f}_{r}(Y) 
$$
and consequently 
$$
\mathfrak{f}_n (Y^{-1})= i^{n \mathfrak{K}}   | Y |^{\mathfrak{K}}
\mathfrak{f}_n   (Y) + \sum^{n-1}_{r=o}   \big(i^{n \mathfrak{K}}   |
Y|^{\mathfrak{K}}  \mathfrak{f}_r  (Y)- \mathfrak{f}_r ( Y^{-1}) \big) 
$$

Substituting this in (\ref{eq462}) we have 
\begin{align*}
\xi   (\mathscr{S}, u ) & = \frac{1}{\sqrt{n}}   \int\limits_{\substack{Y
    \in   \mathfrak{K}\\ | Y |   \ge   1}} \mathfrak{f}_n   (Y)\{i^{n
  \mathfrak{K}}   |Y|^{\mathfrak{K}-\mathscr{S}}   u^*(Y) + |
Y|^{\mathscr{S}} u   (Y) \} d \vartheta + \sum^{n-1}_{r=o}
\frac{1}{\sqrt{n}}   \\
& \qquad \int\limits_{\substack{Y   \in   A \\ | Y |   \ge
    1}} \big( i^{n   \mathfrak{K}}   | Y |^{\mathfrak{K}}
\mathfrak{f}_r   (Y)- \mathfrak{f}_{r}   (Y^{-1})\big) |
Y|^{-\mathscr{S}}   u^* (Y)   d \vartheta \tag{463}\label{eq463} 
\end{align*}

Here we have assumed that the order of summation and integration can
be inverted. We now compute explicitly the integral corresponding to
$r=o$ in the sum on the right side of (\ref{eq463}). With
$\mathfrak{f}_o(Y)=a_o$, this reduces to  
\begin{align*}
&\frac{a(o)}{\sqrt{n}} \int\limits_{\substack{Y \in \mathfrak{K}  \\ | Y
      | \ge  1}} \big(i^{n   \mathfrak{K}}   y^{n   (\mathfrak{K}-
    \mathscr{S})} -y^{-n \mathscr{S}}\big)   u^*   (Y_1)   \sqrt{n}
  y^{-1}   dy  d \vartheta_1 \\ 
&= a(o) \int\limits_{\substack{Y_{1}   \in   \mathfrak{K} \\ | Y_{1} |
      = 1}}   u^*   (Y_1)   d \vartheta_1 \Bigg(\frac{i^{n
      \mathfrak{K}}}{n(\mathscr{S}-\mathfrak{K})}- \frac{1}{n
    \mathscr{S}}\Bigg) \\ 
& = -\frac{a(o)(1, u^*)}{n} \Bigg( \frac{1}{\mathscr{S}}+ \frac{i^{n
      \mathfrak{K}}}{\mathfrak{K}-\mathscr{S}} \Bigg) \tag{464}\label{eq464} 
\end{align*}\pageoriginale
where $(1, u^*)$ is the scalar product of the two angular characters
introduced in (\ref{eq408}). 

The first term on the right side of (\ref{eq463}) is an entire function of
$\mathscr{S}$. In the special cases when $n = 1, 2$, all the other
integrals occurring there can be shown to be meromorphic in
$\mathscr{S}$ and it will follow that in these cases $\xi
(\mathscr{S}, u)$ and consequently $D(\mathscr{S}, u)'$ can be continued
analytically in the whole domain. It is quite likely that this is true
for arbitrary $n$ too. Further, due to a transformation 
$$
\mathscr{S} \to \mathfrak{K}- \mathscr{S},   u \to u^* ,
$$
the\pageoriginale first integral on the right side of (\ref{eq463})
gets multiplied by a factor $i^{n \mathfrak{K}}$. For $n=1,2$, it is
known that all the integrals occurring on the right side of
(\ref{eq463}) have the above property so that in these cases we have  
\begin{equation*}
\xi (\mathfrak{K}-\mathscr{S}, u^*) = i^{n \mathfrak{K}} \xi
(\mathscr{S}, u) \tag{465}\label{eq465} 
\end{equation*}

It is to expected that this transformation formula is also valid in
the general case and our conjectures are further supported (1) by
the fact that one of the integrals computed in (\ref{eq464}) (corresponding
to $r=0$) verifies all these properties. 

\begin{thebibliography}{99}
\bibitem{1}{E. Hecke} \"Uber die Bestimmung Dirichletscher Reihen durch
  ihre Funktionalgleichung, Math. Ann. 112    (1936), 664-700. 

\bibitem{2}{H.Maass} Modulformen Zweiten Grades und Dirichletreihen,
  Math. Ann. 122   (1950), 90-108. 
\end{thebibliography}

\begin{thebibliography}{99}
\bibitem{key1}{H. Braun}: Zur\pageoriginale Theorie der Modulformen
  n-ten Grades, Math. Ann. 115 (1938), 507-517.  

\bibitem{key2}{H.Braun}: Konvergenz veralgemeinerter Eisensteinscher
  Reihen, Math. Z. 44  (1939), 387-397. 

\bibitem{key3}{M.Koecher}: Uber Thetareihen indefinitter quadratischer
  Formen Math. Nachr. 9.(1953)  51-85. 

\bibitem{key4}{M. Koecher}: Uber Dirichlet-Reihen mit
  Funktionalgleichung, Crelle Journal 192(1953), 1-23. 

\bibitem{key5}{M. Koecher}: Zur Theorie der Modulformen n-ten
  Grades.I. Math. Z. 59   (1954), 399-416. 

\bibitem{key6}{H. Maass}: Uber eine Metrik im Siegelschen Halbraum,
  Math. Ann. 118   (1942), 312-318. 

\bibitem{key7}{H. Maass}: Modulformen zwoiten Grades and
  Dirichletreihen, Math. Ann. 122   (1950), 90-108. 

\bibitem{key8}{H. Maass}: Uber die Darstellung der Modulformen n-ten
  grades durch Poincaresche Reihen, Math, Ann. 123
  (1951),121-151. 

\bibitem{key9}{H.Maass}: Die Primzahlen in der Theorie der Siegelschen
  Modulfunktionen, Math. Ann. 124   (1951), 87-122. 

\bibitem{key10}{H. Maass}: Die Differentialgleichungen in der Theorie
  der Siegelschen Modulfunktionen, Math. Ann. 126   (1953), 44-68. 

\bibitem{key11}{H. Maass}:\pageoriginale Die Bestimming der
  Dirichletreihen mit Grossen-charakteren zu den Modulformen n-ten
  Grades, J. Indian Math. Soc. (in print).  

\bibitem{key12}{C.L.Siegel}: Uber die analytische Theorie der
  quadratischen Formen,  Ann, of Math. 36   (1935), 527-606. 

\bibitem{key13}{C.L. Siegel}: Einfuhrung in die Theorie der
  Mokulfunktionen n-ten Grades, Math. Ann. 116   (1939), 617-657. 

\bibitem{key14}{C.L.Siegel}: Einheiten quadratischer Formen,
  Abh. a. d. Seminar d. Hamburger Univ. 13   (1940), 209-239. 

\bibitem{key15}{ C.L.Siegel}: Symplectic geometry, Amer. J. Math. 65
  (1943), 1-86. 

\bibitem{key16}{C.L. Siegel}: Indefinite quadratische Formen und
  Funktionen-theorie. I. Math. Ann. 124   (1952),17-54. 

\bibitem{key17}{C.L.Siegel}: Indefinite quadratische Formen und
  funktionen - theorie. II. Math. Ann. 124   (1952), 364-387. 

\bibitem{key18}{E. Witt}: Eine Identitate zwischen Modulformen zweiten
  Grades, Abh.a.d.Seminar d. Hamburger Univ. 14   (1941), 321-327. 
\end{thebibliography}
 
