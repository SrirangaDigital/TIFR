\chapter{The metrization of Modular forms of degree}%chap 10

In \pageoriginale \S. 5 we defined modular forms of degree  $n \ge
1$ and weight 
$k$ and we set up an operator $\phi$ which maps the linear space
$\mathscr{M}^{(n)}_{\mathfrak{K}}$  of such forms into $
\mathfrak{M}^{(n-1)}_{\mathfrak{K}}$ when $n >1$. With a view to
facilitating the statement of our subsequent results we agree to call
constants modular  forms of degree 0 and a given weight $k$ and
extend the domain of $\phi$ to include
$\mathscr{M}^{(1)}_{\mathfrak{K}}$ by  specifying that for modular
forms $f(Z) \in \mathscr{M}^1_{\mathfrak{K}}$, $f (Z) / \phi = a,
(o)$, viz, its `first'  Fourier coefficient . We can now state that  
$$ 
\mathscr{M}^{(n)}_{\mathfrak{K}} \bigg | \phi \subset
\mathscr{M}^{(n-1)}_{\mathfrak{K}} , n \ge 1. 
$$

It is of interest to investigate when the reverse inclusion is also
true, viz, when the inclusion relation above is actually a relation
of  equality. This turns  out to be true in all the cases where the
Poincare'  series  $g(Z,T)$ which depend on $k$ and $n$ have been
proved to exist  (as modular forms of degree  $n$ and weight $k$), as
we shall see  subsequently . A modular form of degree $n > 0$ shall be
called a  \textit{cusp form} if  
$$
f(Z)|  \phi \equiv o
$$  

Any modular form of degree $0$ shall by definition be a cusp form
too. If  $f(Z), g(Z)$ be two  modular forms of degree  $n \ge 0$ and
weight $k$, of which  at least one is a cusp form, we define a scalar
product $(f, g)$ by   
\begin{equation*}
(f, g) =
\begin{cases}
\int\limits_{f_n} f(Z) \overline{g(Z)} & |y| ^{\mathfrak{K}-n-1} [ d \times ]
    [d y] , f n  n > o, Z-x+y\\ 
f \bar{g},  & \text{ if  } n =0  \tag{169}\label{eq169}
\end{cases}x
\end{equation*}

We prove\pageoriginale  the existence  of the right  side of
(\ref{eq169}) to make our 
definition  meaningful. Let, of the two, $g(Z)$ be a cusp form. 

It follows from (\ref{eq111}) that 
$$
|g(Z) | \le \zeta e^{-2 \in \mathscr{Y} \overline{|y|}}  \text{ for } 
Z=  \chi + \in y \in F_n
$$
with certain  constants $\in, \zeta  0$. Clearly $f(Z)$ and $\in^{-\in
  \sqrt{| y|} } |y|^{\mathfrak{K}-n-1}$ are boun\-ded in $f_n$. We have
therefore only to prove the  existence of the integral  
$$
\int\limits_{f_{n}} e^{\in u \sqrt{|y|}} [d y] [d x].
$$

Since $f_n$ is contained in the set of  $z XXXXX$ defined by: 
$$
Z=  x+i y , \chi \in \mathscr{M}_n , y \in\mathfrak{K}_n  
$$
the above integral is majorised by   $\int\limits_{y \in \mathfrak{K}_n}
\varepsilon ^{- \in u \sqrt{|y|}}[d y]$ and this integral, we know,
exists by (\ref{eq149}). 

We also observe that the integrand  $f(Z, g, \bar{Z})
|y|^{\mathfrak{K} - n - 1} [dx][dy]$  is invariant with
respect to the modular substitutions, as the product $\overline{g(Z)}\break
f(Z) |y|^\mathfrak{K}$ and $dy = |y|^{-n-1}[d \chi][ dy]$ are so, and
consequently it is permissible to replace the domain of integration
$f_n$ occurring in the right side of (\ref{eq169}) by any other fundamental
domain for $M_n$  

The cusp forms of degree $n$ and weight $k$   constitute a linear
space $y^{(n)}_{\mathfrak{K}}$, a sub space of
$\mathscr{M}^{(n)}_{\mathfrak{K}}$. We define the orthogonal space
$y^{(n)}_{\mathfrak{K}}$ of $y^{(\mathfrak{K})}_{\mathfrak{K}}$
in $\mathscr{M}^{(n)}_{\mathfrak{K}}$ in the usual way, is the set of
forms in  $\mathscr{M}^{(\mathscr{N})}_\mathfrak{K}$, which are
orthogonal to all the forms\pageoriginale  in $y^{(n)}_\mathfrak{K}$
in the sense of 
our inner product. It is well known that
$\mathscr{M}^{(n)}_\mathfrak{K}$ is the direct sum of
$y^{(n)}_\mathfrak{K}$ and $^{(n)}_\mathfrak{K}$  i.e. 
$$
\mathscr{M}^{(n)}_\mathfrak{K}= y^{(n)}_\mathfrak{K}+
\mathscr{N}^{(n)}_\mathfrak{K}. 
$$

A much finer decomposition  of $\mathscr{M}^{(n)}_\mathfrak{K}$ is
given in  

\setcounter{thm}{9}
\begin{thm}\label{chap10:thm10}%the 10
There exists a uniquely determined representation of
$\mathscr{M}^{(n)}_\mathfrak{K}$ as a direct sum 
\begin{equation*}
\mathscr{M}^{(n)}_\mathfrak{K}= y^{(n)}_{\mathscr{R }0} +
y^{(n)}_{\mathfrak{K}1}+ \cdots +y^{(n)}_{\mathfrak{K}n}
\tag{170}\label{eq170} 
\end{equation*}
with
$$
y^{(n)}_{\mathfrak{K}n} \subset y^{(n)}_\mathfrak{K} 
$$
\begin{equation*} 
\text{ for } \nu < n \text{ and } y^{(n)}_{\mathfrak{K}\nu}
\subset \mathscr{N}^{(n)}_\mathfrak{K}, y^{(n)}_{\mathfrak{K}o} \&
\subset y^{(n-1)}_\mathscr{R ''} \tag{171}\label{eq171}  
\end{equation*}
\end{thm}

\begin{proof}
 We first remark that the mapping $\phi$ is $1-1$ on
 $\mathscr{N}^{(n)}_\mathfrak{K}$.  
\end{proof}

Indeed, due to linearity of $\phi$, this is ensured if $f(Z) \in
\mathscr{N}^{(n)}_\mathfrak{K}$, $f(Z) \phi \equiv 0$ together imply that
$f(Z)\equiv 0$, but this is immediate as,  in this case, $f(Z) \in
\mathscr{N}^{(n)}_\mathfrak{K} o y^{(n)}_\mathfrak{K}=(0)$. 

Also the theorem is innocuous in the case $n =0$. We now base our proof
on induction on $n$. We therefore assume given
$y^{(n-1)}_{\mathfrak{K}\nu}$, $\nu=1,\break 2 \cdots (n-1)$ satisfying
(\ref{eq170})- (\ref{eq171}) with $n$ replaced by $n-1$. Let
$y^{(n)}_{\mathfrak{K} \nu}(\nu < n)$ denote the linear space all
$f(Z) \in  \mathscr{N}^{(n)}_\mathfrak{K} $ such that $f(Z) \ \phi
y^{(n-1)}_{\mathfrak{K} \nu}$.  

Since $\phi$ is invertible on $\mathscr{N}^{(n)}_\mathfrak{K}$,
$\bar{y}^{(n)}_\mathfrak{K}$ can be characterized by the
relations: 
$$ 
\bar{y}^{(n)}_{\mathfrak{K}\nu} \subset
\mathscr{N}^{(n)}_\mathfrak{K}, \bar{y}^{(n)}_{\mathfrak{K}\nu} | 
\phi = \mathscr{N}^{(n)}_\mathfrak{K} | \phi \cap
y^{(n-1)}_{\mathfrak{K}\nu} 
$$
 
By assumption\pageoriginale 
$$
\mathscr{M}^{(n-1)}_{\mathfrak{K}}= \sum^{n-1}_{\nu  =o}
y^{n-1}_{\mathfrak{K} \nu} 
$$
so that
\begin{align*}
\mathscr{N}^{(n)}_{\mathfrak{K}} | \& &=\mathscr{N}^{(n)}_{\mathfrak{K}}
| \& \cap \mathscr{M}^{(n-1)}_{\mathfrak{K}} = \sum_{\nu =o}^{(n-1)}
\mathscr{N}^{(n)}_\mathfrak{K} | \& \cap y^{(n-1)}_{\mathfrak{K}
  \nu}\\ 
&=\sum^{n-1}_{\nu =o} y^{(n)}_{\mathfrak{K}\nu} | \phi 
\end{align*}
where all the sums occurring are direct.

Since $\phi$ is $1-1$ on $\mathscr{N}^{(n)}_{\mathfrak{K}}$ this means
that 
$$
\mathscr{N}^{(n)}_{\mathfrak{K}}= \sum^{n-1}_{\nu=o}
\bar{y^{(n)}}_{\mathfrak{K} \nu} 
$$
and then
$$
\mathscr{M}^{(n)}_{\mathfrak{K}}= \mathscr{N}^{(n)}_{\mathfrak{K}} +
y^{(n)}_{\mathfrak{K}} = \sum^{n-1}_{\nu =o}
\bar{y}^{(n)}_{\mathfrak{K}\nu} + y^{(n)}_{\mathfrak{K}} 
$$
again all the sums direct. The last decomposition has clearly all the
desired properties. 

It  remains to show that decomposition is unique . 

Let $\mathscr{M}^{(n)}_{\mathfrak{K}}= \sum^n_{\nu =o}
y^{(n)}_{\mathfrak{K}\nu}$ be another decomposition satisfying
(\ref{eq171}). Then we have for $\nu < n$,  
$$
y^{(n)}_{\mathfrak{K}\nu} | \phi \subset
\mathscr{N}^{(n)}_{\mathfrak{K}} | \phi \cap
\mathscr{N}^{(n-1)}_{\mathfrak{K}\nu} =
\bar{y}^{(n)}_{\mathfrak{K}} | \varphi 
$$
and both $y^{(n)}_{\mathfrak{K}\nu}$,
$\bar{y}^{(n)}_{\mathfrak{K} \nu}$ are contained in
$\mathscr{N}^{(n)}_{\mathfrak{K}}$ on which $\phi$ is $1-1$. Hence
$y^{(n)}_{\mathfrak{K}\nu} \subset
\bar{y}^{(n)}_{\mathfrak{K} \nu}$ for $\nu < n$, and by
assumption, $y^{(n)}_{\mathfrak{K} \nu} \subset
y^{(n)}_{\mathfrak{K}}$. Since $\sum^n_{\nu=o}
y^{(n)}_{\mathfrak{K} \nu} =\sum^n_{\nu=o}
\bar{y}^{(n)}_{\mathfrak{K} \nu}  + y^n_{\mathfrak{K}}$ and
both sums are direct, it easily follows that
$y^{(n)}_{\mathfrak{K} \nu}= \bar{y}^{(n)}_{\mathfrak{K} \nu}$
for $\nu < n$ and $y^{(n)}_{\mathfrak{K} \nu}=
y^{(n)}_{\mathfrak{K} \nu}$. 

The proof of theorem \ref{chap10:thm10} is now complete.

As a\pageoriginale  consequence of (\ref{eq171}) we deduce that 
$$
y^{(n)}_{\mathfrak{K} \nu} | \varphi^{n- \nu} \subset
y^{(n-1)}_{\mathfrak{K} \nu} | \varphi^{n- \nu-1} \subset
y^{(n-1)}_{\mathfrak{K} \nu} | \varphi^{n- \nu -1} \cdots \subset
y^{(n)}_{\mathfrak{K} \nu}= y^{(n)}_{\mathfrak{K}}. 
$$

This means in words that $y^{(n)}_{\mathfrak{K} \nu} |
\varphi^{n-\nu}$ consists of cusp forms of degree $\nu$. Further
$\phi^{n- \nu}$ is $1-1$ on $y^{(n)}_{\mathfrak{K} \mu}$ for $\mu
\le \nu \le n$ and in particular on $y^{(n)}_{\mathfrak{K}
  \nu}$. For, we need only show that the conditions $f(Z) \subset
y^{(n)}_{\mathfrak{K} \mu}$, $f(Z) | \phi^{n - \nu}=o$, $\mu \le
\nu \le n$, together imply that $f(Z) \equiv o$. Since in the case
$\nu= n$, the assertion is trivial, we assume $\nu < n$. We
resort to induction  on $n$. 

Let $g(Z) = f(Z) | \phi$. Then our assumption implies in view of
(\ref{eq171}) that  
$$
g(Z) \in y^{(n-1)}_{\mathfrak{K} \mu}, g(Z) | \phi^{n-1- \nu}=o
$$

The induction  assumption  then implies that $g(Z)=0$ which in its
turn means that $f(Z)=0$ as we know that $\phi$ is $1-1$ on
$\mathscr{N}^{(n)}_{\mathfrak{K}}$ and in particular on
$y^{(n)}_{\mathfrak{K} \mu} \subset
\mathscr{N}^{(n)}_{\mathfrak{K}}(\mu < n)$.  

We shall now introduce a scalar product for arbitrary pairs of modular
forms. In view of theorem \ref{chap10:thm10}, we have a decomposition  for any
modular form $f(Z) \in \mathscr{M}^{(n)}_{\mathfrak{K}}$ as  
$$
f(Z) = \sum^n_{\nu=o} f_\nu(Z), f(Z), f_\nu \in
y^{(n)}_{\mathfrak{K} \nu}, \nu=0.1.2 \cdots n, 
$$
and  this decomposition in unique. Let $g(Z)$ be another form in
$\mathscr{M}^{(n)}_{\mathfrak{K} \nu}$ and let $g(Z)= \sum^n_{\nu=o}
g_\nu(2)$, $g_\nu(2) \in y^{(n)}_{\mathfrak{K} \nu}$. 

Then we define the scalar product $(f(Z), g(Z))$ as
\begin{equation*}
 (f(Z), g(Z)) = \sum^n_{\nu=o} (f_\nu (Z) | \varphi^{n- \nu}
  g_{\nu}(2) | \varphi^{n- \nu})  \tag{172}\label{eq172}  
 \end{equation*}
 where\pageoriginale  the scalar product occurring in the right side
 have the meaning 
 given earlier. If $g(Z)$ is a cusp form, then $g(Z) =g_n(Z)$ and all
 but the last term vanish in the summation on the right side of
 (\ref{eq172}). Thus (\ref{eq172}) reduces in this case to: $(f(Z), g(Z))=
 (f_n(Z), g_n (Z))$, the right side being interpreted in the sense of
 (\ref{eq169}). This proves the consistency of our present definition of the
 scalar product $(f(Z), g(Z))$ with the earlier ones given by (\ref{eq169})
 whenever applicable. Also it is clear from (\ref{eq172})  that $(p,g)=
 (\overline{g,p})$ and that the assumption $(f,f)=0$ implies $p_\nu(Z)|
 \phi^{n- \nu}=0$ for $o \le \nu \le n$ and consequently $f_\nu, \nu
 \le n$ and $f= \sum^{n}_{\nu=o} f_\nu$ vanish identically.  
 
 We thus conclude that the metric which (\ref{eq172}) gives rise, to, is a
 \textit{positive hermitian} metric. 
 
 Let us compute the scalar product $((f,Z), g(Z))$ where $g(Z)$ is
 represented by the Poincare' series $g(Z,T)$ and $f(Z)$ is a cusp
 form, i.e. $f(Z) \in y^{(n)}_{\mathfrak{K}}$. We assume of course
 that the Poincare' series converges. Since $f(Z)$ is a cusp form, we
 have from (\ref{eq111}) that  
 $$
 f(Z) = 0(r^{-2m  {\sqrt[n]{|y|}})} \text{ for } Z= x+ y \in f_n 
 $$
 where $m$ is a suitable positive constant. Hence 
 $$
 |y|^{R/2}|f(Z) |\le \mathcal{M} e^{-m \sqrt[n]{|y|}}  \text{ for } Z
 \in f_n 
 $$
 with a sufficiently large $\mathcal{M}$. Since $|y|^{R/2} |f(Z)|=
 h(Z)$ is invariant by (\ref{eq101}) and $|y|$ does not increase when we
 replace $Z \in f_n$ by an equivalent point  with respect to $M_n$, 
it follows\pageoriginale  that
$$
| y |^{k/2} |f (Z) | \le  \mathcal{M} \in^{n \sqrt[n]{i y'}} \text
{for } > \in \mathscr{Y}_r. 
$$

We require these facts to prove that the function

$\rho (Z) \in^{2 \pi \sigma (T \varepsilon)} | y |^{k - r - 1}$
\qquad with $T = \begin{pmatrix} T^{(n)}_1 & o \\ o & o \end{pmatrix},
T^{(n)}_1 > o$  
is absolutely integrable over the fundamental domain $y_r$ of the
groups $\mathcal{A}_n $. This is certainly true as a result of our
above assertions if the integral. 
\begin{equation*}
G (T_\ell, m) = \int\limits_{y_n} e^{-Z \pi \sigma (\chi y) - m
  \sqrt[n]{| y |} } | y |^{k/2} - n - 1, k \times ] [dy]
  \tag{173}\label{eq173}   
\end{equation*}
exists. We shall show that this integral does exist.

First we consider the case $0 < r < n$. In this case we can use the
usual parametric representation for $y > o$ as  
\begin{equation*}
Y = 
\begin{pmatrix}
y_1 & o \\ o & y_w
\end{pmatrix}
\begin{bmatrix}
\begin{pmatrix}
E & v \\ o & E
 \end{pmatrix}
\end{bmatrix}
\end{equation*}


Transforming (\ref{eq173}) in terms of the new variables and carrying out
the integrations with suspect to $x$ and $v$ as in earlier contexts,
we have 
{\fontsize{10pt}{12pt}\selectfont
\begin{align*}
G (T, m) & = \int\limits_{\substack{y_1 > o \\ y_2 \in \mathfrak{K}_{n
      - r}}} \in^{-Z \pi \sigma (T_1 y_1 - m \sqrt[n]{| y_1 |}
  \sqrt[n]{|y_2|} | y_1 |^{k/2 - r - 1}} | y_2 |^{k/2 - r -1} [dy_1]
[dy_2]\\ 
& = \int\limits_{y_1 > o} e^{2 \pi \sigma < T_1 y_1>} |y_1|^{k/2 - r -
  1} \left ( \int\limits_{y_2 o \mathfrak{K}_{n-r}} e^{m
  \sqrt[n]{|y_1|} \sqrt[n]{|y_2|}} |y_2|^{k/2 - n - 1} [dy_2 ] \right)
[dy]\\ 
& = \frac{n (n - r + 1)}{2} \vartheta_{n - r} m^{\dfrac{-n (k - n -n
    -1}{2}} \chi \left( \frac{n (k - r - r - 1}{2} \right)\\
& \qquad \qquad \times
\int\limits_{y_1 > o} \in^{2 \pi \sigma < T_1 y_1 >} |y_2 |^{k - n
  -r}{2} [dy_1]\\ 
& = \frac{n(n-r+1}{2} \vartheta_{n - r} m ^{\frac{-n(k - n - r -1}{2}}
y \left( \frac{n(k - n - r - 1}{2} \right) \Pi^{r(r-o/4} \\
& \qquad \qquad (2
\Pi)^{r 0/2} | T |^{\frac{1}{2}} \mathscr{V}_{\nu - o}^{k - 2} y
\left( \frac{n - u}{2} \right) 
\end{align*}}

The last\pageoriginale  two relations which are consequences of
(\ref{eq149}) and (\ref{eq143}) are true provided $k > n + r + 1$.   

We shall now turn to the border cases $r = o$, $n$. In the first case,
viZ. $r = o$ 
\begin{align*}
G (T_1, m) & = \int\limits_{y \in \mathfrak{K}_n} e^{-m \sqrt[n]{| y
    |}} | y |^{k/2 - r - 1} [d y]\\ 
& = \frac{n (n + 1)}{2} \vartheta_n y \big(\frac{n (k - n -o}{2}
m^{-\frac{n (k - n - 1}{2}} 
\end{align*}
provided $k > n + 1$
and in the alternative case when $r = n$,
\begin{align*}
G (T_1 m) & = \int\limits_{y > o} e^{2 \pi \sigma (T y) - m \sqrt[n]{| y |}}
| y |^{k/2 - n - 1} [ d y]\\ 
& \le \int\limits_{y > o} e^{-2 \pi \sigma (T y)} | y |^{k/2 - n -1} [dy]\\
& = \Pi ^{\dfrac{n (n - 1}{4}} (2 \pi)^{\dfrac{n(n+1-k}{2}} | T
|^{\dfrac{n + 1 -k}{2}} ****_{\nu = 1}^{n} \chi( \frac{k - n - \nu}{2} 
\end{align*}
provided $k > 2n$.

The conditions in all these cases under which we have proved (\ref{eq173})
to exist are precisely those under which we proved the Poincare'
series to exist and in particular we have shown that the integral 
\begin{equation*}
H(T_1, \rho ) = \int\limits_{Z \in y_r} \rho (Z) e^{2 \pi i \sigma (T
  \bar{Z})} | y |^{k/2 - n - 1} [d x] [ d y] \tag{174}\label{eq174}  
\end{equation*}
exists provided $k \equiv 0 (2)$ and $k > n + 1 + r$, where $r = $
rank $T$. We observe that the integrand in $H(T_1, \mathfrak{f}$ is
invariant under the substitutions in $\mathcal{A}_r$ so that we can
choose for the domain of integration any convenient fundamental
domain\pageoriginale  of $\mathcal{A}_r$ in the place of $y_r$. In
fact we 
compute $H(T_2, \mathfrak{K})$ by choosing ``$ y_r$'' in two
different ways and this will lead us to an interesting result. 

From the definition of $\mathcal{A}_r (r \ge 0)$ it is clear that,
while $\mathcal{A}_0$ contains with every $\mathcal{M}$, -
$\mathcal{M}$ too, $\mathcal{A}_r$ for $r > o$ contains at most one of
the matrices $\mathcal{M}$, - $\mathcal{M}$. So we may assume that $ V_r
$ in the case $n > o$ contains $-S$ with $S$. Then $U_{S \in V_r} S
< S_n >$ is a fundamental domain for $\mathcal{A}_r$ which is covered
but once if $r = o$, while it is covered  twice if $r > 0$. 

Let us introduce
$$
\delta_r = 
\begin{cases}
1  .  \text{ if } r  =  0 \\
 2  . \text{ if } r >  0
\end{cases}
$$
and obtain from (\ref{eq174}), in view of the invariance of the integrand
appearing in it under the substitutions in $\mathcal{A}_r$, 
\begin{gather*}
H(T_1, \mathfrak{f}) = \frac{1}{\delta _r} \sum_{S \in V_r} \int\limits_{S \in
  S < \mathfrak{f}_n >} \mathfrak{f} (Z) \in^{- 2 \pi i \sigma (T
  \bar{Z})} | y |^{k - n - 1} [d x] [ d y]\\ 
= \frac{1}{S_r} \sum_{S \in V_r} \int\limits_{Z \in \mathfrak{f}_n}
\mathfrak{f} (S < Z >) e^{- 2 \pi \sigma ( T S < Z> )} | y_s |^{k - n -
  1} [d x_s] [d y _s] 
\end{gather*} 
where we assume $S < Z > = X_s +  i y_s$

Since $d v = | y |^{- n - 1} [ d x] [d y ]$ is the invariant volume
element, using (\ref{eq92}) and (\ref{eq160}) we get  
\begin{align*}
 H(T_1, \mathfrak{f}) & = \frac{1}{\delta_r} \sum_{S \in V_r} \int\limits_{Z
    \in \delta_n} \mathfrak{f} (S < Z > e^{- 2 \pi i \sigma (T
    \overline{S<Z>})} |c Z \\
& \hspace{2.6cm} \qquad \qquad + D |^{-2 k} | y |^{k - n - 1} [d x ][d
    y]\\ 
& = \frac{1}{\delta_r} \sum_{S \in V_r} \int\limits_{Z \in \delta_r}
  \mathfrak{f} (S <Z>) | c Z + d |^{-k} e^{-2 \pi i \sigma (T
    \overline{S <Z>})} | c \bar{Z}\\
& \hspace{2.6cm} \qquad \qquad  + D | |y |^{\mathfrak{K} - n - 1} [d
    x] [dy]\\ 
& = \frac{1}{\delta_r} \sum_{S \in V_r} \int\limits_{Z \in S_n} \rho (Z) e^{-
    2 \pi i \sigma (T \overline{S<Z>})} | c \bar{Z} + D
  |^{-\mathfrak{K}} | y |^{\mathfrak{K} - r - 1} [dx ]  [d y]\\ 
& = \frac{1}{\delta_r} \int\limits_{Z \in \mathfrak{f}_n} \mathfrak{f} (Z) | y
  |^{\mathfrak{K}-n - 1}  (\sum_{S \in V_r} e^{-2 \pi i \sigma (T
    \overline{S<Z>})} | c \bar{z} \\
& \hspace{2.6cm} \qquad \qquad + D |^{-\mathfrak{K}} [d x ] [ dy]\\ 
& = \frac{\varepsilon (T_1)}{\delta_r} \int\limits_{Z \in
    \mathfrak{f}_n} \mathfrak{f}(Z) \overline{ g (Z, T} |y|^{\mathfrak{K}
    - n - 1} [d x ] [d y]\\ 
& = \frac{\varepsilon(T_1)}{\delta_r} (\mathfrak{f} (Z), g (z, T))
  \tag{175}\label{eq175}  
\end{align*}\pageoriginale 

On the other hand we can computer $H (T_1, \mathfrak{f})$ directly as  
follows : 

The part of the integral involving $\times$ in (\ref{eq174}) is clearly  
\begin{align*}
\int\limits_{\times \in n 0_n} \mathfrak{f}(Z) e^{- 2 \pi i \sigma (T
  \bar{Z})} [ a x] & = e^{-4 \pi e (T y)} \int\limits_{\times \in \delta S_r
} \mathfrak{f} (z) e^{-2 \pi i \sigma (T Z) }\\ 
& = e^{-\triangle \pi \sigma (T y)} a (T)
\end{align*}
where $a (T)$ is the Fourier coefficient of $f(z)$ with respect to the
exponent - matrix $T$. 

But $a (T) = 0$ by (\ref{eq108}) if $| T | = 0$ since by assumption $f(z)$
is a cusp form so that, in particular if $\mathfrak{K} < \mathfrak{K}$,
$H(T_1, \mathfrak{f}$ and consequently the scalar product $(\mathfrak{f}
(Z) g (z, T))$ vanish. Assume then that $r = n$. Then  
\begin{align*}
H(T_1, \mathfrak{f}) & = \int\limits_{\substack{ \times \in xyw\\ y >
    o}} \mathfrak{f}(Z) e^{-2 \pi \sigma (T z)} [d x] e^{- \triangle
  \pi \sigma (T y)} | y |^{\mathfrak{K} - n -1} [d y]\\ 
& = a (T) \int\limits_{y > o} e^{- 4 \pi \sigma (T y)} | y
|^{\mathfrak{K} - n - 1} [d y] 
\end{align*}
\begin{equation*} 
= a(T) \pi ^{\dfrac{n (n - 1)}{4}} (4 \pi ) ^{\dfrac{n (n + 1}{2}- n
  \mathfrak{K}} \prod^n_{\nu = s} y (\mathfrak{K} - \frac{n +
  \nu}{2} | T |^{n + 1}{2} \tag{176}\label{eq176}   
\end{equation*}\pageoriginale 

Comparing with (\ref{eq175}) we get


\begin{thm}\label{chap10:thm11} %thm 11
 If $ \mathfrak{f} (Z) \in y_{\mathfrak{K}}^{(n)}$ and $\mathfrak{K}
 \equiv 0 (2), \mathfrak{K} > n + 1 + $ rank $T$, then $(f (z) g (z,
 T))$ is equal to 
 $$
\begin{cases}
\frac{2}{\varepsilon (T)} a(T) \pi^{\frac{n (n -1}{4}} (4
\pi)^{\frac{n (n + 1}{2} - n \mathfrak{K}} \prod_{\nu = 1}^{n}
y(\mathfrak{K} - \frac{n + \nu}{2} ) | T |^{\frac{n +1}{2}-
  \mathfrak{K}} \text {\em  for } T > o\\ 
o \qquad \text {  for  } T = o 
\end{cases}
$$
where $a(T)$ represents the Fourier coefficient of $f(z)$ with
  respect to the exponent matrix $T$. 
\end{thm}

We generalize the above result to the case of $f(z) \in
y_{\mathfrak{K}}   (\nu \le n)$ in the next section. 

