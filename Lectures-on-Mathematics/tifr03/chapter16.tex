
\chapter{Closed differential forms}% chapter 16

Our\pageoriginale aim in this section is to generalise to the case of
non-analytic 
forms some of the concepts associated with the analytic forms of
Grenzkreis group of the first kind considered by Petersson, and study
their applications to the space of symplectic geometry. Let $G$ denote
a Grenzkreis group of the first kind. With respect to a multiplicator
system $\vartheta$, let $< G, \mathfrak{K}, \vartheta >$ denote the space of all
analytic forms of $G$ of weight $\mathfrak{K}$-analytic in the usual sense,
viz. that the only singularities if any, are poles in the local
uniformising variables of the Riemann surface $\mathfrak{f}$ defined
by $G$ (We can assume that $\mathfrak{f}$ is a closed Riemann
surface). In conjunction with $< G, \mathfrak{K}, \vartheta >$ we have got to
consider also the space $< G, 2 -\mathfrak{K}, 1/ \vartheta >$, a sort of
\textit{adjoint} to the first. For $r \ge 2$ the weight of $g (z) \in
< G, 2 - \mathfrak{K}, 1/ \vartheta >$ is zero or negative so that its
dimension 
( = - weight) is positive or zero. Since it is known that an
automorphic form of zero or positive dimension which is everywhere
regular vanishes identically, it follows that in this case $g(z)$ is
uniquely determined by its principal parts if all the principal parts
are zero $g(z)$ itself is identically zero. The question now arises
whether there always exists an automorphic form of a given weight $2 -
\mathfrak{K}$ with arbitrarily given principal parts. While this is
true for $\mathfrak{K} = 
2$ in the case of the Gaussian sphere- a Riemann surface of genus
zero-this is not unreservedly true either if $\mathfrak{K} > 2$ or if the genus
is strictly positive. Nevertheless it is possible to give a necessary
and sufficient condition for the given principal parts to
satisfy,\pageoriginale in order that they should be all the principal
parts of an automorphic form of weight $\mathfrak{K}$ of $G$. We state
this condition presently. For $f (z) \in \langle G$, $\mathfrak{K}$,
$\mu$ and $g (z) \in \langle G$, $2-\mathfrak{K}$, $1/ \mu \rangle$
introduce  
\begin{equation*}
\omega (\mathfrak{f}, g ) = \mathfrak{f} (z) y (z) dz \tag{310}\label{eq310}
\end{equation*}
which is a closed meromorphic differential form on the Riemann surface
defined by $G$. That $\omega (\mathfrak{f}, g)$ is invariant relative to
$G$ is immediate if we observe that the product $\mathfrak{f} (z) g
(z)$ is an automorphic form of weight 2 (the sum of the weights of
$\mathfrak{f}$ and $g$) while $dz$ behaves as an automorphic
form of weight $- 2$, and the multiplicator systems for $\mathfrak{f},
g $ are $\nu$ and $1/\vartheta$ respectively. If $\mathfrak{f}$
denotes a fundamental domain of $G$ in $\mathscr{Y}_1$, by integrating
$\omega (\mathfrak{f}, g)$ over the boundary of $\mathfrak{f}$ it is
easily seen that  
\begin{equation*}
\sum_{Z \in \mathfrak{f}} \text { Residue } \omega (\mathfrak{f}, g ) =
0 \tag{311}\label{eq311} 
\end{equation*}

On the left side of (\ref{eq311}) we face only a finite sum as we assume
that the Riemann surface is closed and there is only a finite number
of singularities in the fundamental domain $\mathfrak{f}$. For
$\mathfrak{K} > 2$, as $\mathfrak{f}(z)$ runs over a basis of \textit {
  everywhere regular } form in $\langle G, \mathfrak{K}, \mu$ the left
side of (\ref{eq311}) depends only on the principal parts of $g (z) \in
\langle G$, $2-\mathfrak{K}$. $1/\mu  \langle$  and the conditions (\ref{eq311})
themselves are called the \textit{principal part conditions.} These
are necessary and sufficient for the existence of a form $g (z)$ with
assigned principal parts.  

We now wish to generalise the notion of $\omega (\mathfrak{f}, g)$ to
the case of the linear spaces $\{ G, \alpha, \beta, \mu\}$ of non
analytic automorphic forms of degree\pageoriginale $n$, and in
particular we have to settle when two such spaces can be considered
adjoint like the spaces $\langle G$, $\mathfrak{K}$, $\mu \rangle$ and
$\langle G$, $2-\mathfrak{K}$, $1/\mu \rangle$. The two important
properties of $\omega (\mathfrak{f}, g ) \langle$ are   
\begin{enumerate}[(i)]
\item it is completely invariant relative to $G$

\item it is closed. 
\end{enumerate}

The last condition is trivial in the earlier case of analytic forms
and is not so in the present case. 

First we have to introduce the concept of the dual differential forms
in the ring of exterior differential forms. Consider a Riemannian
manifold $R$ in which local complex coordinates $\mathscr{Y'} = (Z_1,
Z_2, \ldots, Z_n)$, $ \bar {\mathfrak{z}} = (\bar{z}, \bar{z}_z,
\ldots, \bar{z}_n) $ are introduced such that a neighbourhood of each
point is mapped pseudo conformally onto a neighbourhood of the origin
of the corresponding coordinate space. We consider $\mathfrak{z}$ and
$\bar{\mathfrak{z}}$  as formally independent complex variables, and
the above means that if  
$$
\mathfrak{z}^{*'} = (z^*_1 z^*_2,\ldots, z^*_n),
\bar{\mathfrak{z}}^{*'} = (\bar{z^*_1}, \bar{z^*_2},\ldots,
\bar{z^*_n}) 
$$
 is any other system of complex coordinates at the given point then we
 have  
 \begin{equation*}
\mathfrak{z}^{*} = \mathfrak{f}  (\mathfrak{z}), \bar{\mathfrak{z}^*} =
\bar{\mathfrak{f}} (\bar {\mathfrak{z}}) \tag{312}\label{eq312} 
 \end{equation*} 
 with the elements of the columns $\mathfrak{f} (\mathfrak{z})$,
 $\bar{\mathfrak{f}} \bar {\mathfrak{z}}$ representing regular functions
of $(z_1, z_2,\ldots, z_n)$, $(\bar{z_1}, \bar{z_2},\ldots, \bar{z_n})$
 respectively, which vanish at $\mathfrak{z} \bar{\mathfrak{z}} =
 0$. In the local co-ordinate system the metric fundamental form
 $ds^2$ is a Hermitian form and let us assume  
 \begin{equation*}
ds^2 = d \mathfrak{z}' G d \bar{\mathfrak{z}} \tag{313}\label{eq313}
 \end{equation*} 
 with a Hermitian metric $G$. 
 
 In the\pageoriginale ring of exterior differential forms let us
 introduce  
 \begin{align*}
[d \mathfrak{z}] & = d{z_1} d{z_2} \ldots d{z_n},\\ 
[d \bar{\mathfrak{z}}] & = d{\bar{z}_1} d{\bar{z}_2} \ldots
d{\bar{z}_n},\\ 
\omega_\nu  & = (-1)^{\nu - 1} dz_{1} \dots dz_{\nu - 1} 
dz_{\nu + 1} \ldots dz_n\\
\bar{\omega}_\nu & = (-1)^{\nu - 1} d\bar{z}_{1} \ldots d\bar{z}_{\nu
  - 1} \alpha \bar{z}_{\nu +1} \ldots d\bar{z}_n \tag{314}\label{eq314} 
 \end{align*} 
 
 Then 
 \begin{equation*}
[d \mathfrak{z}]   = dz_2 \omega_2; [ d\bar{\mathfrak{z}}] = d\bar{z},
\bar {\omega}_\nu \tag{315}\label{eq315} 
 \end{equation*} 
 
 Denote with $\mathscr{W}$, $\bar {\mathscr{W}}$ the columns with the
 elements $\omega_\nu$, $\bar{\omega}_\nu$ respectively, i,e. 
 \begin{equation*}
\mathscr{W}' = (\omega_1, \omega_2, \ldots \omega_n); \quad \bar
        {\mathscr{W}}' = (\bar {\omega}_1, \bar {\omega}_2, \ldots
        \bar {\omega}_n) \tag{316}\label{eq316} 
 \end{equation*} 
 
 Consider a differential form $\theta$ of degree 1. It can be
 represented as  
 \begin{equation*}
\theta = \mathscr{Y}' d \mathfrak{z} + \mathscr{U}' d
\bar{\mathfrak{z}} \tag{317}\label{eq317} 
 \end{equation*}  
 where $\mathscr{Y}$, $\mathscr{U}$ are columns with functions of
 $\mathfrak{z} \bar{\mathfrak{z}}$ as elements. We also introduce the
 dual form  
 \begin{equation*}
\tilde{\theta } = |G| (\mathscr{Y}' \bar{G}^{-1} \mathscr{W} [d
  \mathfrak{z}] + \mathscr{U}' G^{-1} \mathscr{W} [d
  \bar{\mathfrak{z}}]) \tag{318}\label{eq318} 
 \end{equation*} 
 
 We shall show that $\tilde{\theta}$ is uniquely defined by $\theta $
 or what is the same, it does not depend on the local coordinate
 system. We have only to prove that $\tilde{\theta}$ is invariant
 relative to the pseudo conformal mappings of the kind (\ref{eq313}), in the
 $\mathfrak{z}^*$, $\bar{\mathfrak{z}}^*$ system $\theta$ have the
 representation  
 $$
 \theta = \mathscr{Y}^{*'}  d \mathfrak{z}^*  + \mathscr{U}^{*'} d
 \bar{\mathfrak{z}}^* 
 $$
 
 From (\ref{eq313})\pageoriginale we have, with a certain non singular
 matrix $T$,  
 $$
 d \mathfrak{z}^* = T d \mathfrak{z}, T d \bar{\mathfrak{z}} ^* = \bar
 {T} d \bar{\mathfrak{z}} 
 $$
 and then 
 $$
 \mathscr{Y} = T' \mathscr{Y}^*  \text{ and } \mathscr{U} = \bar {T}'
 \mathscr{U} ^* 
 $$
 
 A simple computation shows that 
 \begin{align*}
T' \mathscr{W}^* & = |T| \mathscr{W},  \bar {T}' \bar {\mathscr{W}}^*
= |\bar {T}| \bar{\mathscr{W}}; \\ 
[T' \mathfrak{z}^*] & = |T| [d \mathfrak{z}],  [d \bar
  {\mathfrak{z}}^*] = |\bar {T}| [d \bar{\mathfrak{z}}]  
 \end{align*} 
 where the elements with a star denote the corresponding elements in
 the new coordinate system $\mathfrak{z}^*$, $\bar
 {\mathfrak{z}}^*$. Besides, we also have  
 $$
 d s^2 = d \mathfrak{z}' G d \bar {\mathfrak{z}} = d
 \mathfrak{z}^{*'} G^{*} d \bar {\mathfrak{z}}^* 
 $$
 and it is easy to deduce that 
 $$
 T' G^* \bar {T}  = G
 $$
 
 The invariance of (\ref{eq318}) relative to the transformations
 (\ref{eq313}) is  now immediate. If   
 \begin{equation*}
\frac{\partial}{\partial \mathfrak{z}}, (\frac{\partial}{\partial
  z_1}, \frac{\partial}{\partial z_2}, \cdots,
\frac{\partial}{\partial z_n}), \frac{\partial}{\partial \bar
  {\mathfrak{z}}}, (\frac{\partial}{\partial \bar{z}_1},
\frac{\partial}{\partial \bar{z}_2}, \cdots, \frac{\partial}{\partial
  \bar{z}_n}) \tag{319}\label{eq319} 
 \end{equation*} 
 we have 
 \begin{align*}
d \tilde{\theta} & = d \bar{\mathfrak{z}}' \frac{\partial}{\partial
  \bar {\mathfrak{z}}} |G| \mathscr{Y}' \bar {G} ^{-1} \bar
{\mathscr{W}} [ d \mathfrak{z}] +  d \mathfrak{z}'
\frac{\partial}{\partial z} |G| \mathscr{U}  G^{-1}{\mathscr{W}} [ d
  \bar {\mathfrak{z}}] \\ 
& = \frac{\partial}{\partial \bar {\mathfrak{z}}}' |G| G^{-1}
\mathscr{Y} [d \bar{\mathfrak{z}}] [ d \mathfrak{z}] +
\frac{\partial}{\partial \mathfrak{z}} |G| G^{-1} \mathscr{U} [d
  \mathfrak{z}] [d \bar{\mathfrak{z}}]\\ 
& = \{  (- \mathscr{U} \frac{\partial}{\partial \bar{\mathfrak{z}}}'
|G| G^{-1}) \mathscr{Y} + \frac{\partial}{\partial \mathfrak{z}}' |G|
G^{-1} \mathscr{U} \} [d \mathfrak{z}] [d \bar{\mathfrak{z}}]
\tag{320}\label{eq320} 
 \end{align*} 
 
 We\pageoriginale therefore infer that the condition for the form
 $\tilde{\theta}$  to be closed is that   
 \begin{equation*}
(-1)^n \frac{\partial} { \partial \bar {\mathfrak{z}}} '|G| G^{-1}
   \mathscr{Y} + \frac{\partial} { \partial  \mathfrak{z}}' |G| G^{-1}
   \mathscr{U} = 0 \tag{321}\label{eq321}  
 \end{equation*} 
 
 We now compute the invariant volume element in the metric
 (\ref{eq312}). Let $ z_\nu = r_2 + iy_\nu $,  $\bar {z} _\nu = r_\nu +
 iy_\nu  (\nu = 1.2\cdots n)$ and $\mathfrak{z} = \varepsilon + i
 \mathscr{Y}$, $\bar {\mathfrak{z}} = \mathcal{E} + i
 \mathscr{Y}$. Then $dz_\nu d\bar{z}_\nu =- zi dx_\nu dy$ so that $[d
   \mathfrak{z}] [d \bar {\mathfrak{z}}]$ and $[d \mathcal{E}][d
   \mathscr{Y}]$ differ only by a constant factor. Using the
 transformation formulae for these differential forms for a change of
 coordinate  systems and the transformation properties of $|G|$ it is
 easily seen 
 that $|G| [ d \mathfrak{z}] [ d \bar {\mathfrak{z}}]$ is invariant
 for the coordinate transformations. This is therefore true of  $|G| [
   d \mathcal{E}] [ d \mathscr{Y}]$ and this is precisely then the
 invariant volume element, viz.  
 \begin{equation*}
d \mathscr{U} = |G| [ d \mathcal{E}] [ d \mathscr{Y}]
\tag{322}\label{eq322}   
 \end{equation*} 
 
 We wish to interpret these results in the space $\mathscr{Y}$ of
 symplectic geometry. The fundamental metric form here is  
 $$
 d s^2 = \sigma (d z y^{-1} d \bar{z} y^{-1}), z = z + i y,
 \bar {z} = x - iy  
 $$
 
 If we set $|y| y^{-1} = (y_{\mu \nu })$, then $y_{\mu \nu}$ is just
 the algebraic minor of $\mathcal{Y}_{\mu \nu }$ in $y =
 (\mathcal{y}_{\mu \nu })$ and we have  
 \begin{align*}
d \mathscr{S}^2 & = |y|^{-2} \sum_{\substack{\mu, \nu \\ \varrho,
    \sigma }} dz_{\mu \nu} y_{\nu \varrho } d \bar {z}_{\varrho \sigma
} y_{\sigma \mu}\\ 
& = |y|^{-2} \sum_{\substack{\mu \leq  \nu \\ \varrho, \sigma }}
\frac{1}{e_{\mu \nu }}dz_{\mu \nu} y_{\nu \varrho } d \bar
     {z}_{\varrho \sigma } y_{\sigma \mu}\\ 
& = |y|^{-2} \sum_{\substack{\mu \leq  \nu \\ \varrho \leq  \sigma }}
     \frac{1}{2e_{\mu \nu } e_{\varrho \sigma }}(y_{\mu \sigma }
     y_{\nu \varrho } + y_{\nu \sigma } y_{\varrho \sigma })dz_{\mu
       \nu}  d \bar {z}_{\varrho \sigma } \\ 
& = \sum_{\mu \leq  \nu, \varrho \leq  \sigma } \mathcal{G}_{\mu
       \nu, \varrho  \sigma} dz_{\mu \nu}  d \bar {z}_{\varrho \sigma
     }  \tag{323}\label{eq323}  
 \end{align*} 
 with 
 \begin{equation*}
\mathcal{G}_{\mu   \nu, \varrho  \sigma} = \frac{1}{2e_{\mu \nu }
  e_{\varrho \sigma }} |y|^{-2} (y_{\mu \sigma } y_{\nu \varrho } +
y_{\nu \sigma } y_{\varrho \sigma }) \tag{324}\label{eq324}  
 \end{equation*} 
 
 If $G = (\mathcal{G}_{\mu \nu, \varrho \sigma })$ and $G^{-1} =
 (\mathcal{G}^{\mu \nu, \varrho \sigma })$, we find from the relation
 $G G^{-1} = E$ that  
 \begin{equation*}
\mathcal{G}^{\mu \nu, \varrho \sigma } = \frac{1}{2} (\mathcal{Y}_{\mu
  \varrho} \mathcal{Y}_{\nu \sigma} + \mathcal{Y}_{\mu \sigma}
\mathcal{Y}_{\nu \varrho}) \tag{325}\label{eq325}  
 \end{equation*}\pageoriginale 
 
 The transformations we have to consider are the symplectic
 substitutions  
 $$
 Z^* = (AZ + B ) (CZ + D ) ^{-1}, \bar{Z}^* = (A \bar{Z} + B )
 (C\bar{Z} + D ) ^{-1} 
 $$
 and these are pseudo conformal mappings of $\mathscr{Y}$ in the sense
 of (\ref{eq313}). Let us introduce as in the earlier case,  
 $$
 [dz] = \prod_{\mu \leq \nu} dz _{\mu \nu }, [d \bar z] = \prod_{\mu
   \leq \nu} d \bar {z} _{\mu \nu } 
 $$
 with the lexicographical order of the factors and  
 \begin{align*}
\omega_{\mu \nu } & = \pm \prod_{\substack{\varrho \leq \sigma
    \\ (\varrho, \sigma ) \neq (\mu, \nu )}} dz_{\varrho \sigma } =
\omega_{\nu \mu } 
 & &  \text { for } \mu \leq \nu, \\ 
\bar { \omega}_{\mu \nu } & = \pm \prod_{\substack{\varrho \leq \sigma
    \\ (\varrho, \sigma ) \neq (\mu, \nu )}} d\bar{z}_{\varrho \sigma
} = \bar{\omega}_{\nu \mu } 
 \end{align*} 
 the\pageoriginale ambiguous sign being fixed in each by stipulating
 that   
 \begin{equation*}
[dz] = dz_{\mu \nu } \omega_{ \mu \nu }; [d\bar {z}] = d\bar{z}_{\mu
  \nu } \bar{\omega}_{ \mu \nu } \tag{326}\label{eq326}  
 \end{equation*} 
 
 In the place of $\mathscr{W}, \bar{\mathscr{W}} $ we introduce  
 \begin{equation*}
\Omega = (e_{\mu \nu } \omega_{ \mu \nu }, \bar{\Omega} = (e_{\mu \nu
} \bar{\omega}_{ \mu \nu }. \tag{327}\label{eq327}  
 \end{equation*} 
 
 As before we start with a differential form of degree 1 given by  
 \begin{equation*}
\theta = \sigma (p d z ) + \sigma (\theta d \bar{z}) \tag{328}\label{eq328} 
 \end{equation*} 
 where $p$, $\theta $ are $n-$rowed symmetric with functions as
 elements. We have to compute the dual form $\tilde{\theta}$.  
 
 In the earlier notation, 
 \begin{equation*}
\tilde{\theta} = (\mathcal{Y}' \bar{G}^{-1} \bar{\mathscr{W}} [ d
  \mathfrak{z}] + \mathscr{U}' G^{-1} \mathscr{W} [ d
  \bar{\mathfrak{z}}]). \tag{$(318)'$}\label{eq318'} 
 \end{equation*} 
 $|G|$ can be fixed as follows. We know that the invariant volume
 element in symplectic geometry is  
 $$
 d \vartheta = |y|^{-n-1} [dx ] [dy]
 $$
 
 From (\ref{eq321}) it is also given by 
 $$
 d \vartheta = |G| [d \mathcal{E} ] [d\mathscr{Y}].
 $$
 
 A comparison\pageoriginale of the two shows that $|G|$ is equal to a
 constant  multiple of $|Y|^{-n-1}$ and this constant factor can be
 assumed to  be 1 for the purposes of $\tilde{\theta }$. Then $|G| =
 |Y|^{-n-1}$. We now compute the product $\mathscr{Y}' \tilde{\theta 
 }^{-1} \bar{\mathscr{Y}}$. 
 \begin{align*}
\mathcal{Y}' \bar{G}^{-1} \bar{\mathscr{W}} & =
\sum_{\substack{\mu\leq  \nu \\ \varrho \leq  \sigma }}
\frac{1}{e_{\mu \nu }} P_{\mu \nu } \frac{1}{2}(\mathcal{Y}_{\mu
  \varrho } \mathcal{Y}_{\nu \sigma } + \mathcal{Y}_{\mu \varrho }
\mathcal{Y}_{\nu \varrho  }) \bar {\omega}_{\varrho \sigma}\\ 
& = \sum_{\substack{\mu\leq  \nu \\ \varrho \leq  \sigma }}  P_{\mu
  \nu } \frac{1}{2}(\mathcal{Y}_{\mu \varrho } \mathcal{Y}_{\nu \sigma
} + \mathcal{Y}_{\mu \varrho } \mathcal{Y}_{\nu \varrho  }) \bar
        {\omega}_{\varrho \sigma}\\ 
& = \sum_{\mu,  \nu ; \varrho ,  \sigma } e_{\mu \nu } P_{\mu \nu }
        \mathcal{Y}_{\mu \varrho } \mathcal{Y}_{\nu \sigma }  \bar
                {\omega}_{\varrho \sigma} = \sigma (y p y \bar
                {\Omega})
 \end{align*} 
 
 The other product $\mathscr{U}' G^{-1} \mathscr{W}$ similarly reduces
 to $\sigma (y \theta y \bar {\Omega})$ and from (\ref{eq318}) we have  
 \begin{equation*}
\tilde{\theta} = |Y|^{-n-1} (\sigma (y p y \bar {\Omega}) [dz] +
\sigma (y \theta y  \Omega) [d\bar z]). \tag{329}\label{eq329}  
 \end{equation*} 
 
 Then form (\ref{eq320}) we will have for the differential of
 $\tilde{\theta}$,  
 \begin{equation*}
d \tilde{\theta} = \{  (-1)^{\frac{n (n + 1)}{2}} \sigma
(\frac{\partial}{\partial z} y p y |Y|^{-n-1}) + \sigma
(\frac{\partial}{\partial z} y \theta y |Y|^{-n-1})\} [dz ] [d
  \bar{z}]. \tag{330}\label{eq330}  
 \end{equation*} 
 
 We now specialise $\theta$ by appropriate choices of $P$ and
 $\mathcal{Q}$ and investigate in detail the cases in which $d
 \tilde{\theta}$ vanishes. 
 
 Let $\{ \alpha, \beta\}$ denote the linear space of all regular
 functions $\mathfrak{f} (z, \bar{z})$ in $\mathscr{Y}$ which satisfy
 the differential equation (\ref{eq272}). Let $\alpha$, $\beta$; $\alpha'$,
 $\beta'$ be complex numbers, arbitrary for the present, and choose  
 \begin{equation*}
\mathfrak{f} = \mathfrak{f} (z, \bar {z}) \in \{\alpha, \beta \},
\mathcal{G} = \mathcal{G} (z, \bar {z}) \in \{\alpha', \beta' \}
\tag{331}\label{eq331}   
 \end{equation*}\pageoriginale 
 
 We then set 
 \begin{equation*} 
p = \mathcal{E} |Y|^y y^{-1} \mathcal{G} K_{\alpha}
\mathfrak{f}, \mathcal{Q} = |Y|^y y^{-1} \mathcal{F}
\Lambda_{\beta} \mathcal{G} \tag{332}\label{eq332}   
 \end{equation*} 
 where we assume 
 \begin{equation*}
\mathcal{E}^2 = 1, y = \alpha + \alpha' = \beta + \beta'
\tag{333}\label{eq333}   
 \end{equation*} 
 and the differential operators $K_\alpha $ and $\Lambda_\beta$ are
 defined by (\ref{eq276}). With this $P$ and $Q$, (\ref{eq328})
 defines $\theta $,  and we set   
 \begin{equation*}
\omega (\mathfrak{f}, g) = \theta \tag{334}\label{eq334}  
 \end{equation*} 
 
 We shall study the behaviour of this differential form relative to
 the symplectic substitutions 
 $$
 z^* = (az + B ) (cz + B)^{-1}, \bar{z}^* = (a\bar{z} + B ) (c\bar{z}
 + B)^{-1}. 
 $$
 
 The replacement of $z$, $\bar{z}$ by $z^*$, $\bar{z}^*$ in any operator
 or function shall in general be denoted by putting a star $(*)$ over
 it and in particular  
 $$
 \mathfrak{f}^* = \mathfrak{f} (z^* , \bar{z}^*), \mathcal{G}^* =
 \mathcal{G} (z^*, \bar{z}^*) 
 $$
 
 If $M = \begin{pmatrix} A & B \\ C & D  \end {pmatrix}$ is any
 symplectic metric, we also introduce  
 \begin{align*}
\mathfrak{f} (z, \bar{z}) | M & = |c z + D|^{- \alpha } | C \bar{z} +
D|^{- \beta} \mathfrak{f} (z^*, \bar{z}^*)\\ 
\mathcal{G} (z, \bar{z}) | M & = |c z + D|^{- \alpha '} | C \bar{z} +
D|^{- \beta'} \mathcal{G} (z^*, \bar{z}^*) 
 \end{align*} 
 and for any differential form $\mu (z, \bar{z})$,
 \begin{equation*}
\mu (z, \bar{z}) | M = \mu (z,^* \bar{z}^*) \tag{335}\label{eq335}  
 \end{equation*}\pageoriginale 
 
 By means of (\ref{eq286}), \ref{eq122'} and (\ref{eq284}) we have 
 \begin{align*}
(z^* - \bar{z}^*) \frac{\partial}{\partial z^*} \mathfrak{f} & =
   (\bar{z}C' + D' ) ^{-1} (z , \bar{z}) ((c z + D )
   \frac{\partial}{\partial z})'\\
& \hspace{3.3cm} | e z + D^\alpha| |c \bar{z} +
   D|^\beta (\mathfrak{f} | M)\\ 
& = | e z + D^\alpha| |c \bar{z} + D|^\beta (\bar{z}C' + D' ) ^{-1} (z
   , \bar{z}) ((c z + D ) \\
& \hspace{3.3cm} \{ \alpha C' + ((c z + D )
   \frac{\partial}{\partial z})'\} (\mathfrak{f} | M)\\ 
& = | e z + D^\alpha| |c \bar{z} + D|^\beta (\bar{z}C' + D' ) ^{-1} [
     \{ \alpha (z c' + D' )\\
& \qquad  - \alpha (\bar{z} c' + D')\} (\mathfrak{f}
     | M) + (z - \bar{z} \frac{\partial}{\partial z} \mathfrak{f} |M )
     (z c' + D')]  
 \end{align*} 
 and 
$$
K_\alpha^ * \mathfrak{f} ^* = | C z + D^\alpha| |c \bar{z} + D|^\beta
(\bar{z}C' + D' ) ^{-1} (K_\alpha \mathfrak{f}|M) (z c' + D') 
$$

Further, 
\begin{align*}
d z^*  & = (zc' + D' )^{-1}  dz (zc + D )^{-1}\\
Y*^{-1} & = (cz + D )Y^{-1} (\bar{z}c' + D' )\\
| Y*^\nu | & = |Y|^y |c z + D |^{-y} |C \bar{z} + D|^{- y}
\end{align*}
and this yields 
$$
|Y^*|^y y*^{-1} g^* (K^*_\alpha \mathfrak{f}^*) dz^* = (cz + D)
|Y|^y Y^{-1} (g |M)(k_\alpha \mathfrak{f} |M ) dz (cz +D). 
$$

In our notation (\ref{eq355}) the left side is just
$$
(|Y|^y Y^{-1} \mathcal{G}(K_\alpha \mathfrak{f}) dz ) | M =
(\frac{1}{\mathcal{E}} p d z ) |M  
$$

Hence\pageoriginale we deduce that 
\begin{align*}
\sigma (\frac {1}{\mathcal{E}} p d z )|M &= \sigma \{ (c z + D)
|y|^y y^{-1} (g | M ) (K-\alpha \mathfrak{f} | M ) dz (cz +
D)^{-1}\\ 
& = \sigma \{ |y|^y y^{-1} (g i M) (K_\alpha \mathfrak{f} | M ) dz
\} 
\end{align*}

One proves analogously that 
$$
\sigma (\mathcal{Q} d \bar{z}) = \sigma \mathfrak{f} |Y|^y y^{-1}
\{ | M ) (\Lambda_\beta \mathcal{G} |M| d \bar{z}) \} 
$$
and then it is immediate that 
\begin{equation*}
\omega (\mathfrak{f}, \mathcal{G}) | M = \omega (\mathfrak{f} |
M. \mathcal{G} | M ). \tag{336}\label{eq336}   
\end{equation*}

This is naturally true of the dual form $\tilde{\omega } (\mathfrak{f},
\mathcal{G})$ also so that  
\begin{equation*}
\tilde{\omega} (\mathfrak{f}, \mathcal{G}) | M = \tilde {\omega}
(\mathfrak{f} | M. \mathcal{G} | M ). \tag{337}\label{eq337}   
\end{equation*}

Let us now compute $d \tilde{\omega} (\mathfrak{f}, \mathcal{G})$ form
(\ref{eq330}). In doing this we shall need the following identity  
\begin{equation*}
((z - \bar{z}) \frac{\partial }{\partial \bar{z}})' = \{((z -
  \bar{z})\frac{\partial }{\partial \bar{z}})' W  \} y - \frac{1}{2 }
  \sigma (y w ) E- \frac{1}{2} w' y \tag{338}\label{eq338}   
\end{equation*}
where $W = (\omega_{\mu \nu })$ denote an arbitrary matrix its
elements all functions. In fact, since  
$$
e_{\varrho \mu } \frac{\partial Y_{\nu, \mu }}{\partial
  \bar{z}_{\varrho, \mu } } = - \frac{1}{4i} (\partial_{\varrho \tau}
\partial_{\mu  \nu } + \partial_{\varrho \nu \partial_{\mu  \tau}}) 
$$
we have 
$$
((z - \bar{z})\frac{\partial }{\partial \bar{z} })' (w y ) = (\sum_
\varphi (z_{\nu \varrho }- \bar z_{\nu \varrho }) e_{\varrho \mu  }
\frac{\partial }{\partial \bar{z}_{\varrho, \mu } } (\sum_\tau
\omega_{\mu \tau } \mathscr{Y}_{\tau \nu }) 
$$
\begin{align*}
& = ( \sum_{\varrho \sigma \tau } (z_{\nu \varrho } - \bar z_{\sigma
    \varrho }) e_{\varrho \mu  } \frac{\partial }{\partial
    \bar{z}_{\varrho, \mu } } (\sum_\tau \omega_{\mu \tau }
  \mathscr{Y}_{\tau \nu })\\ 
 & = [((z - \bar{z}) \frac{\partial }{\partial \bar{z} })' W] y -
  \frac{1}{4i} (\sum_{\varrho \sigma \tau } (z_{\varrho \sigma } -
  \bar{z}_{\varrho \sigma }) \omega_{\sigma \tau } (\delta_{\varrho
    \tau }\delta_{\mu \nu  } + \delta_{\mu  \tau } \delta_{\varrho \nu
  })\\ 
& = [((z - \bar{z}) \frac{\partial }{\partial \bar{z} })' W] y -
  \frac{1}{4i} \sigma ((z - \bar{z}) W ) \mathcal{E} \frac{1}{4i} W'
  (z, \bar{z}) 
\end{align*}\pageoriginale
which is just another form of (\ref{eq338}). 

Similarly one obtains the analogous identity, viz. 
\begin{equation*}
((z - \bar{z}) \frac{\partial }{\partial z })' (W y) = [(z - \bar{z})
    \frac{\partial }{\partial z })' W y] y + \frac{1}{2} \sigma (y W )
  E + \frac{1}{2} W'Y \tag{339}\label{eq339}   
\end{equation*}

Finally one also has 
\begin{equation*}
\left.
\begin{aligned}
\mathcal {E} y P | y |^{-n-1} &= \alpha |y|^{\mu - n -1} y \mathfrak{f}
\mathcal{G} + |y| ^{\mu - n - 1 } \mathcal{G} (z - \bar{z})
(\frac{\partial}{\partial z } \mathfrak{f}) y \\ 
y \theta | y |^{-n-1} &= \beta |y|^{\mu - n -1} y \mathfrak{f}
\mathcal{G} + |y| ^{\mu - n - 1 } \mathcal{F} (z - \bar{z})
(\frac{\partial}{\partial z } \mathcal{G}) y  
\end{aligned}
\right \} \tag{340}\label{eq340}  
\end{equation*}

Using (\ref{eq280}) and (\ref{eq338}-\ref{eq340}) we can now state that 

$\varepsilon \dfrac{\partial}{\partial \bar{z}}y P|y|^{-n-1} =$
\begin{align*}
= & \alpha \bigg \{ - \frac{y-n-1}{2
  \ell}|y|^{y-n-1}y^{-1}+|y|^{y-n-1}\frac{\partial}{\partial
  \bar{z}}\bigg \}y fg +\\ 
 & \qquad + \bigg \{ - \frac{y-n-1}{2
  \ell}|y|^{y-n-1}y^{-1}+|y|^{y-n-1}\frac{\partial}{\partial
  \bar{z}}\bigg \}g.(z-\bar{z})(\frac{\partial}{\partial z} f)y\\ 
 & =  \frac{d \ell }{2}(y-n-1)|y|^(y-n-1)fg
E-(y-n-1)|y|^(y-n-1)g(\frac{\partial}{\partial z}f)+\\ 
 & \qquad - \frac{d \ell
}{2}|y|^{(y-n-1)}\bigg\{-\frac{n+1}{2}E+\bigg((z-\bar{z})\frac{\partial}{\partial  
  \bar{z}}\bigg ) \bigg \} fg+\\ 
 & \qquad +  |y|^{(y-n-1)}(\frac {\partial}{\partial \bar{z}}
g)(z-\bar{z})(\frac {\partial}{\partial \bar{z}}f)y \\ 
 & \qquad + g
|y|^{(y-n-1)}\bigg\{-\frac{n+1}{2}E+\bigg((z-\bar{z})\frac{\partial}{\partial 
  \bar{z}}\bigg )' \bigg \}(\frac{\partial}{\partial z}f)y \\
 & = \frac{di } {2} (y - \frac{n + 1 }{2}) |Y|^{y - n - 1 }
\mathfrak{f} \mathcal{G}  E - (y - \frac{n}{2}) | y |^{y n -
  1 } \mathcal{G} (\frac{\partial}{\partial z} \mathfrak{f}) + \\ 
 & = \frac{di } {2} (y - \frac{n + 1 }{2}) |Y|^{y - n - 1 }
\mathfrak{f} \mathcal{G}  E - (y - \frac{n}{2}) | y |^{y n -
  1 } \mathcal{G} (\frac{\partial}{\partial z} \mathfrak{f}) + \\ 
 & \qquad \alpha | y |^{y n - 1 } \mathcal {G } (\frac{\partial
}{\varrho z} \mathfrak{f}) y + \alpha |Y|^{y - n - 1 }
\mathfrak{f} (\frac{\partial}{\partial z } \mathcal{G}) Y + \\ 
 & \qquad 2 i |y |^{y - n- 1} (\frac{\partial } {\partial z }
\mathcal{G}) y (\frac{\partial } {\partial z } \mathscr{G}) Y \\ 
 & \qquad + \mathcal{G} |y|^{y - n -1}[\{ (( Z -\bar{Z},
  \frac{\partial}{\partial \bar{z}})' \frac{\partial}{\partial z}
  \mathfrak{f}\} y - \frac{1}{2} \sigma (y \frac{\partial}{\partial z}
  \mathfrak{f}) E]\\ 
 & = \frac{\alpha i}{2} (y - \frac{n+1}{2}) |y|^{y - n -1}
\mathfrak{f} \mathcal{G} E - (y - \frac{n}{2}- \beta )|y|^{y
  -n -1}\mathcal{G}( \frac{\partial} {\partial z} \mathfrak{f}) y +\\ 
 & \qquad + \alpha |y|^{y -n -1} \mathfrak{f}(\frac{\partial}{\partial z}
\mathcal{G}) y + 2i |y|^{y n-1} (\frac{\partial}{\partial
  \bar{z}}) y (\frac{\partial}{\partial z} \mathfrak{f}) y+\\ 
 & \qquad - \frac{1}{2} |y|^{y - n - 1} \mathcal{G}\sigma  (y
\frac{\partial}{\partial z} \mathfrak{f}) E \tag{341}\label{eq341}   
 \end{align*}\pageoriginale
 
 In stating the last relation we have used the fact that
 $$
 ((z - \bar{z})\frac{\partial}{\partial \bar{z}}\mathfrak{f})
 \frac{\partial}{\partial z} \mathfrak{f} =- \alpha
 \frac{\partial}{\partial \bar{z}} \mathfrak{f} + \beta
 \frac{\partial}{\partial z} \mathfrak{f} 
 $$
 which is easily proved.
 
 We have similar lines the dual formula also, viz.
 $\dfrac{\partial}{\partial z} y Q |y| ^{y -n -1}$
 \begin{gather*}
= (\beta' (y - \frac{n+1}{2}) |y|^{y - n - 1} \mathfrak{f}
\mathcal{G} E + (y - \frac{n}{2} - y) |y|^{y - n - 1}
\mathfrak{f} (\frac{\partial}{\partial \bar{z}}\mathcal{G}) y +)\\ 
- \beta ' \mathcal{G} |y|^{y - n - 1} ((\frac{\partial}{\partial
  z}\mathfrak{f}) y + 2i |y|^{y - n -1}((\frac{\partial}{\partial
  z} \mathfrak{f}) y ((\frac{\partial}{\partial \bar{z}}) y\\ 
+ \frac{1}{2}|y|^{y - n -1} \mathfrak{f} \sigma (y
(\frac{\partial}{\partial \bar{z}} \mathcal{G}) E \tag{342}\label{eq342}   
 \end{gather*} 
 
 In view\pageoriginale of our assumption (\ref{eq333}), setting
 $\mathcal{E} = - (-1)  \dfrac {n (n + 1 )}{2}$ we now obtain that   
 \begin{equation*}
d \tilde{\omega} (\mathfrak{f}, \mathcal{G}) = \frac{in } {2} (\beta' -
\alpha )(y - \frac{n + 1 } {2}) |y|^{y - n - 1} \mathfrak{f}
\mathcal{G} [dz ] [ d \bar{Z}] \tag{343}\label{eq343}   
 \end{equation*} 
 
 In particular we shall have 
 \begin{equation*}
d \tilde{\omega } (\mathfrak{f} \mathscr{G}) = 0 \tag{344}\label{eq344}  
 \end{equation*} 
 in the two cases 
 \begin{equation*}
y = \frac{n + 1 } {2} \alpha' = \frac{n + 1 } {2} - \alpha,
\beta' = \frac{n + 1 } {2} - \beta \tag{345}\label{eq345}   
 \end{equation*} 
 and 
 \begin{equation*}
y = \alpha + \beta, \alpha' = \beta, \beta' = \alpha \tag{346}\label{eq346}  
 \end{equation*} 
 
 We shall look into these cases a little more closely. 
 
 Let $\mathfrak{f} \in \{ G, \alpha, \beta, \mu \}$, $\mathcal{G} \in
 \{ G, \alpha', \beta', \mu' \} $ where we can assume $u' =
 1/\nu$. From (\ref{eq336}) and (\ref{eq337}) it is seen that the differential
 forms $\omega (\mathfrak{f}, g)$ and $\tilde{\omega}
 (\mathfrak{f}, g)$ are invariant relative to
 the substitutions of $G$. With regard to the case (\ref{eq346}) we can now
 make the following remark, viz. if $\alpha$, $\beta$ are real and
 $|\mu | = 1 $, the linear spaces $\{ G, \alpha, \beta, \mu\}$ and $\{
 G; \beta, \alpha, 1/\mu \}$ are mapped onto each other by the
 involution  
 $$
 \mathfrak{f} (z, \bar{Z}) = \mathcal{G} (z, \bar{z} ) =
 \bar{\mathfrak{f}} (\bar{z}, z). 
 $$
 
 Indeed, if $\mathfrak{f}(z, \bar{z}) \in \{ G, \alpha, \beta, \mu \}$
 then $\Omega_{\alpha \beta } \mathfrak{f}(z, \bar{z})=0$ so that,
 taking complex conjugates, $\Omega_{\beta \alpha } \mathcal{G} (z,
 \bar{z}) = 0 $ On the other hand, if $M = \begin{pmatrix} A & C\\ B &
   D \end {pmatrix} \in G$, then  
\begin{align*}
 \mathfrak{f} (M \langle \bar{z} \rangle) |cz + D |^{-\alpha} |c
 \bar{z} + D |^{-\beta} = \vartheta (M) \mathfrak{f} (z \bar{z}),\\ 
 \bar{\mathfrak{f}} (M \langle \bar{z} \rangle, M \langle \bar{z}
 \rangle) |C \bar{z} + D|^{- \alpha} |C \bar{z} + D|^{- \beta }&  =
 \bar{\mu }  (M) \bar{\mathfrak{f}} (\bar{z}, z). 
\end{align*}\pageoriginale
 
 This means that $\mathcal{G} (\bar{z}, z) |M = \frac{1}{\mu} (M)
 \mathcal {G} (z, \bar{z} )$ and our assertion is proved. We can now
 state if $\alpha, \beta $ are real and  $|\mu | = 1 $ then  
 \begin{equation*}
\left.
\begin{aligned}
\tilde{\omega} (f (z, \bar{z}), \bar{\mathcal{G}}(\bar{z}, z )) | M =
\tilde{\omega} (f (z, \bar{z}),  
\bar{\mathcal{G}}(\bar{z}), z)\\
d \tilde{\omega} (f (z, \bar{z}), \bar{\mathcal{G}}( \bar{z}, z))
\end{aligned}
\right \} \tag{347}\label{eq347}  
 \end{equation*} 
 for all $f (z,  \bar{z})$, $\mathcal{G}(z, \bar{z}) \in \{G, d \beta,
 \mu \}$. 
 
 Applications of (\ref{eq347}) have so far been made only in the case when
 $n = 1$. The next case of interest will be when $n = 2$. In this
 case, an explicit expression for $\tilde{\omega}(\mathfrak{f},
 \mathcal{G})$ can be given as follows. Let $\mathscr{L}_q$ denote the
 intersection of the fundamental domain $\mathfrak{f}$ of the modular
 group of degree $n$ with the domain defined by $|y| \leq q$.  
In either of the cases (\ref{eq345}), (\ref{eq346}) whence
$d\tilde{\omega} =0$, we will have
 $$
 \int\limits_{\mathfrak{K} q} \omega (\mathfrak{f}, \mathcal{G}) = 0
 $$
 where $R_q$ denotes the boundary of $\mathscr{Q}_q$. In view of the
 invariance properties of $\omega (\mathfrak{f}, \mathcal{G})$ set out
 earlier, this reduces to  
 \begin{equation*}
 \int\limits_{\mathfrak{f} q} \omega (\mathfrak{f}, \mathcal{G}) = 0
 \tag{348}\label{eq348}   
 \end{equation*}
 where $\mathfrak{f}_q$ denotes that part of $R_q$ which lies on $|y| =
 q$. While this\pageoriginale is all true for any $n$, in the case of
 $n = 2$, we  have the parametric representation (\ref{eq304}) for
 matrices $y > 0$, viz.  
$$
y = u \begin{bmatrix} (x^2 +  y^2) y^{-1}   x y^{-1} \\  xy^{-1}
  y^{-1} \end{bmatrix} , u > 0, y > 0 ; u = \sqrt{|y|} 
$$

One $7_q$ of course we shall have $du = 0$ and then it can be shown
that 
\begin{align*}
\tilde{\omega}(f , g) & = 4 u^{2 \nu - 2} \bigg\{ i(x^2 + y^2) y^{-1 }
\frac{\partial (f g)}{\partial x_1} + i y^{-1} \frac{\partial (f
  g)}{\partial x_{12}} + iy^{-1} \frac{\partial(f g)}{ \partial
  x_{22}} \\
& \qquad + g \frac{\partial f}{\partial u} - f \frac {\partial
  g}{\partial u} + z ( \alpha - \beta) u ^{-1} f g \bigg \} dx_u
dx_{12} y^{-2} dx dy \tag{349}\label{eq349}   
\end{align*}

The explicit computation of the integral on the left side of (\ref{eq348})
appears to be possible only by a detailed knowledge if the Fourier
expansions of the forms $f$ and $g$. 

The following reference pertain to the subject matter of this
section:- 
\begin{description}
\item[1. H. Mass] \"Uber eine neue Art von nichtanalytischen
  automorphen\break Funktionen and die Bestimming Dirichlet scher Reihen
  durch\break Funktionalgleichungen, Math. Ann. 121 (1949), 141 - 193. 

\item[2.  H. Petersson], Zur analytischen Theorie der Grenzkeris
  gruppen, Teil I bis TeilV, Math. Ann 115 (1938), 23 - 67, 175-204,
  518-572, 670-709, Math. Zeit 44(1939), 127-155.  
 
\item[3. H. Petersson]\pageoriginale Konstruktion der Modulformen and
  der zu   gewissen Grenzkreisgruppen gehorigen automorphen Formen von
  positiver reeller Dimension and die vollstandinge Bestimmning ihrer\break
  Fourie - koeffizienten, Sitz. Ber Akad. Wiss. Heidelberg 1950,
  Nr. 8.  
\end{description}

