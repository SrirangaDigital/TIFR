\chapter{The Poincare' Series}%chap 9

This\pageoriginale  section is devoted to the construction of modular
forms in the 
shape of the Poincare' series. Let $T \geq 0$ be a semigroup of modular
matrices of the form 
\begin{equation*}
\begin{pmatrix}
u & S_o u'^{-1} \\
o & u'^{-1}
\end{pmatrix}\tag{150}\label{eq150}
\end{equation*}
where $u$ is a \textit{unit} of $T$ meaning $T[u] = T$, and $S_o$ is
symmetric and integral. Let $V(T)$ denote a set of modular matrices
which constitutes a complete system of representatives of the left
cosets of $M$ modulo $\mathcal{A}(T)$. Then 
\begin{equation*}
M = \sum_{S \in V (T)} A (T) S \tag{151}\label{eq151}
\end{equation*}
With $k$ standing for an \textit{even integer}, we introduce the
Poincare' series as 
\begin{equation*}
g(Z, T) = \sum_{S \in V (T)} e^{2 \pi \sigma (TS < z >)} | cz + D
|^{-\mathfrak{K}} \tag{152}\label{eq152} 
\end{equation*}
where we assume $S = \begin{pmatrix} A & B \\ C & D \end{pmatrix}$ 

First we note that the Poincare' series does not depend on the choice
of $V (T)$. For, this only requires that (\ref{eq152}) is left invariant if
we replace the matrices $S$ occurring in (\ref{eq152}) by $M S$ where 

$M = \begin{pmatrix} u & S_o u'^{-1} \\ 0 & u'^{-1}\end{pmatrix}$
is a matrix of the type occurring in (\ref{eq150}). Due
to\pageoriginale  this replacement 
$\sigma (TS < Z >)$ will go over into $\sigma (TMS < Z >)$ and  
\begin{align*}
\sigma (TMS < Z >) &= \sigma \{ T (S < Z > [ u' ] + S_o ) \} \\
&= T(S < Z > [ u' ] + S_o ) \\
&=\sigma (T [ u ] S < Z >) + \sigma (TS_o) \\
&=\sigma  (TS < Z >) + \sigma (TS_o)
\end{align*}
since $u$ is a unit of $T$. Also
$$
\sigma (TS_o) = \sum_{\mu, \nu} t_{\mu \nu} s^o_{\nu \mu} = \sum 
t_{\mu \mu} s^o_{\mu \mu}  - \sum_{\mu < \nu} t_{\mu \nu} 2 s^o_{\mu
  \nu} 
$$
so that $\sigma (TS_o)$ is an integer as $S_o = (s^o_{\mu \nu})$ is
semi integral. It therefore follows that the exponential factors
occurring in (\ref{eq152}) are left unaltered by such a replacement. The
same is trivially true of the other factor $ | c Z + D |^\mathfrak{K} $
as it goes over into $| u^{-1} cZ + u^{-1} D |^{\mathscr{R}} = | u'
|\mathfrak{K} | cZ + D |^\mathfrak{K} = | cZ  + D |^\mathfrak{K}$, $u$
being unimodular and $k$ being an even integer. It is immediate that
$g (Z, T)$ is independent of the choice of $V (T)$. We now contend
that 
\begin{equation*}
g (Z, T) | M = g (Z, T) | \tag{153}\label{eq153}
\end{equation*}
for any $M = \begin{pmatrix} A_o  & B_o \\ C_o & D_o \end{pmatrix} \in
M$ where 
$$
g (Z, T) | M = g (M < Z >, T)  |C_0 Z + D_0|^{-\mathfrak{K}}
$$
For,
$g (Z, T) | M = \sum_{S \in V (T)} e^{2 \pi i (\sigma (TS) <Z>) } |C_0 Z +
D_0|^{-\mathfrak{K}}$  
\begin{align*}
&= \sum_{S \in V (T) M} e^{2 \pi (\sigma (TS) < Z > )} | cZ + D
  |^{-\mathfrak{K}} \\ 
&= g(Z, T)
\end{align*}\pageoriginale 
since $V (T) M$ represents a set of representatives of the left cosets
of $A (T)$ in $M$ in view of $V (T)$ doing so. What is more, $g (Z,
T)$ depends only on the equivalence class of $T$. In other words a
replacement of $T$ by $T [y]$ where $V$ is unimodular but arbitrary,
does not affect $g(Z, T)$. We may remark are this stage that all our
considerations are formal for the present. $V$ being any unimodular
matrix, the group $A (T [V])$ consists of all modular matrices 
$$
\begin{pmatrix}
u_1 & S_1 u_1'^{-1} \\
0 & u_1^{-1} \end{pmatrix} \text{ with } ( T [ v ] ) [ u_1 ] = T [ V ]
$$
and $S_1 = S_1'$ (integral). This means that if we introduce 
$$
u = v u_1 v^{-1}, S  =  v S_1 v' \text{ and } R = \begin{pmatrix} v & C
  \\ 0 & v^{-1} \end{pmatrix} 
$$
then $R \in M$ and, with $T [ u] = T$,
$$
 R \begin{pmatrix} u_1 & S-1 u_{1}^{-1}{'} \\
0 & u_1^{-1} \end{pmatrix} R^{-1} = \begin{pmatrix} u &S u{'}^{-1}
   \\ 0 & u^{-1}{'} \end{pmatrix}. 
$$

An argument in the reverse direction is also valid and thus we obtain  
\begin{equation*}
R A (T [ v ])R^{-1} = A (T) \tag{154}\label{eq154}
\end{equation*}

From (\ref{eq151}) we have
$$
M = \sum_{S \in V (T)} A (T) S = \sum_{S \in V (T [ v ])} A (T[ v ] S
) 
$$\pageoriginale  
so that using (\ref{eq154})
\begin{align*}
M = R M R^{-1} &= \sum_{S \in V (T [ v ])} R A (T [ v ]) R^{-1} R S
R^{-1} \\ 
&= \sum_{S \in R V (T [ v ]) R^{-1}  A (T)S}.
\end{align*}

Thus $R V (T [ v ]) R^{-1}$ is a set of the kind $V (T)$ and
multiplication on the right by a modular substitution will again lead
to a set of the same kind so that $ R V (R [ v ])$ is itself a set of
the type $V(T)$. Now  
$$
g (Z, T [ V ]) = \sum_{S \in V (T [ v ])}  e^{2\pi \sigma (T [ v ] S <
  Z >)} | cZ + D |^{\mathfrak{K}}. 
$$

Since
$$
\sigma (T [ v ] S < Z > ) = \sigma (T S < Z > [ v ]) = \sigma (TRS < Z
>) 
$$
we conclude by arguments as in earlier contexts that 
\begin{equation*}
g (Z, T [ v ]) = g (Z, T) \tag{155}\label{eq155}
\end{equation*}
where $V$ is an arbitrary unimodular matrix.

We are heading towards the proof of the absolute and uniform
convergence of the Poincare' series (\ref{eq152}) in every compact subset of
$\mathscr{Y}$ provided 
\begin{equation*}
\mathfrak{K} > \min (2n, n+1 + \text{ rank  } T) \tag{156}\label{eq156} 
\end{equation*}
and then for such $ R' s, g(Z, T)$ is a regular function in
$\mathscr{Y}$. The absolute convergence will in particular justify all
our earlier considerations which were hitherto formal and then by
means of (\ref{eq153}), we would have proved that $g ( Z, T )$ represents a
modular form of degree $n>1$ and weight $k$. The case $n = 1$ requires
some further consideration.\pageoriginale  

In view of (\ref{eq155}) and (\ref{eq106}), it suffices to consider the
convergence of the Poincare' series in the special case when  
\begin{equation*}
T = \begin{pmatrix} T_1 & 0 \\ 0 & 0\end{pmatrix}, T_1 = T_1^{(r)} >
  0. \tag{157}\label{eq157} 
\end{equation*}

If $u$ is a unit of $T$, we decompose $u$ analogous to $T$ as $u
= \begin{pmatrix} u_1 & u_2\\ u_3 & u_4 \end{pmatrix}$ and use the
relation $T [ u ] = T$ to infer that $u_2 = 0$ and $T_1 [ u_1 ] =
T_1$. Let us denote by $A_r (r \geq 0)$ the group of modular matrices 
\begin{equation*}
M = \begin{pmatrix} u & S_0 u'^{-1} \\ 0 & u'^{-1} \end{pmatrix}
\tag{158}\label{eq158} 
\end{equation*}
with $u = \begin{pmatrix} E^{(r)} & o \\ u_3 & u_4 \end{pmatrix} u$-
unimodular and $S_o$ symmetric integral. If $T$ be as in (\ref{eq157}) which
it is throughout our subsequent discussion, then $A_r$ is actually a
sub-group of $A(T)$ of finite index $(A (T): A_r)$. In fact, if
$\varepsilon (T_1)$ denotes the number of units of $T_1 = T_1^{(r)}$
for $r>0$ and denotes unity if $r = 0$, then 
$$
(A (T) : A_r ) = \varepsilon (T_1).
$$

Let $V_r$ be a set of representatives of the left cosets of $A_r$ in
$M$ so that  
\begin{equation*}
M = \sum_{S \in V_r} A_r S, r \geq 0. \tag{159}\label{eq159}
\end{equation*}

It is clear that to each $S \in V (T)$ there corresponds $\varepsilon
(T_1)$ elements of $V_r$ say $S_1, S_2, \ldots S_{\varepsilon (T_1)}$
all of which belong to the same left coset\pageoriginale  of $A (T)$
in $M$ and consequently, 
$$
g (Z, T) = \frac{1}{\varepsilon (T_1)} \sum_{S \varepsilon V_r} e^{2
  \pi \sigma (TS < Z > )} | cZ + D |^{-\mathfrak{K}}, S
= \begin{pmatrix} A & B \\ C & D \end{pmatrix} 
$$

By (\ref{eq159}), $M = \sum_{S \in V_o} A_0 S$ so that if we assume 

$A_o = \sum_{R} A_r R$, then $M = \sum_{R S} A_r R S$. Comparing this
with (\ref{eq159}) we conclude that $V_r$ is obtained as all possible
products $RS$ where $R$ runs through a set of representatives of left
cosets of $A_r$ in $A_o$ and $S$ run through all the elements of
$V_o$. In view of an earlier result (pp. 11) we can take for $V_o$
the set of all matrices whose second matrix rows constitute a class of
coprime symmetric mutually non-associated pairs. We now determine a
representative system for $R$. If $R_\nu = \begin{pmatrix} u_\nu & *
  \\ 0 & * \end{pmatrix}$, $\nu = 1, 2$ are two matrices which belong
to the same coset of $A$ in $A_o$ then there exists a matrix $M
= \begin{pmatrix} u & S_o u^{-1} \\ o u^{-1} \end{pmatrix}$ of the
type occurring in (\ref{eq158}) with $MR_1 = R_2$. This is easily seen to
mean that $u u_1 = u_2$ and consequently $u'_1 u' = u'_2$. Since
$u = \begin{pmatrix} E^{(r)} & u_3 \\ o & u_4 \end{pmatrix}$ the above
gives $ u_1{'} \begin{pmatrix} E^{(r)} & u_3 \\ o & u_4 \end{pmatrix}
= u_2{'}$. If  $u_{\nu}' = (p^{(r, r)} w)$, $\nu = 1. 2$, such an
equation can hold only when $p_1 = p_2$. It is now easy to infer that
if $P = P^{(r, r)}$ runs through all primitive matrices and to each
$P$ we make correspond a unique matrix $u^\ast$ obtained by
completing\pageoriginale  $P$ to a unimodular matrix in an arbitrary
way, then the class of all matrices  
\begin{equation*}
R = 
\begin{pmatrix}
u^* & o \\
o & u^{*'^{-1} }
\end{pmatrix} \tag{161}\label{eq161}
\end{equation*}
provides a compete representative system of the left cosets of $A_r$
in $A_o$. We now have 
$$
g (Z, T) = \frac{1}{\varepsilon (T_1)} \sum_{R} \sum_{** V_o} e^{2 \pi
  i \sigma (TRS <Z>)} | CZ + D |^{-\mathfrak{K}} 
$$
where $(C, D)$ which should strictly denote the second matrix row of
the product $RS$ can be assumed to be the second matrix row of $S$ in
view of the special form (\ref{eq161}) of $R$. Since to each primitive $P =
P^{(n, r)}$ we have corresponded a unique $R$, the summation in the
last series can be regarded as one over the set of $P's$ instead of
the $R' s$. In fact the general term of the last series depends only
on $P$ and not on the rest of the columns of $R$ as the equations 
\begin{align*}
\sigma (TRS < Z >) &= \sigma \{ T (S < Z >) [ u'^* ] \} \\ 
&= \sigma \{ T [u^*] S <Z> \} \\
&= \sigma ( T_1 [P']  S<Z>)
\end{align*}
show. We have thus shown that 
\begin{equation*}
g(Z, T) = \frac{1}{\in (T_1)} \sum_{P} \sum_{S \in V_o} e^{2
  \pi i \sigma (T_1 [ P'] S <Z>)} | CZ +S |^{-\mathfrak{K}}
\tag{162}\label{eq162}  
\end{equation*}
where $S = \begin{pmatrix} A & B \\ C & D \end{pmatrix}$

We proceed to construct a suitable fundamental domain for $A_r$ in
$\mathscr{Y}$. We use the parametric representation (\ref{eq132}) for $y >
o$ as  
\begin{equation*}
y = 
\begin{pmatrix}
y_1 & o \\ 
o & y_2
\end{pmatrix}
\begin{bmatrix}
\begin{pmatrix}
E & y \\
o & E
\end{pmatrix}
\end{bmatrix} \tag*{$(132)'$}\label{eq132'}
\end{equation*}\pageoriginale
where $y_1 = y_1^{(r)}, y_2 = y_2^{(n -r)}$ and $v = V^{(r, n-n)}$. 

The substitution (\ref{eq158}) takes $Z = X - i y \in \mathscr{Y}$ into  
\begin{align*}
Z[ u' ] + S_o &=X [ u' ] + S_o + i \begin{pmatrix} y_1 & o \\ o
  & y_2 \end{pmatrix} \begin{bmatrix}\begin{pmatrix} E & o \\ o &
    y_2 \end{pmatrix} \begin{pmatrix} E & u_3{'} \\ o &
    u_4{'} \end{pmatrix}\end{bmatrix} \\ 
&= X [ u' ] + S_o + i \begin{pmatrix} y_1 & o \\ o &
  y_2 \end{pmatrix} \begin{bmatrix} \begin{pmatrix} E & u_3{'} + v
    u_4{'}\\ o & u_4{'} \end{pmatrix}\end{bmatrix} \\ 
&= X[ u' ] + S_o + i \begin{pmatrix} y_1 & o \\ o & y_2 [
    u_1{'}] \end{pmatrix} \begin{bmatrix} \begin{pmatrix} E & u_3{'} +
    v u_4 \\ o & u_4{'} \end{pmatrix}\end{bmatrix} 
\end{align*} 

Comparing this with 
$$
Z = X + i y = \times + i \begin{pmatrix} y_1 & o \\ o &
  y_2 \end{pmatrix}  \begin{bmatrix} \begin{pmatrix} E & V \\ o &
    E \end{pmatrix} \end{bmatrix} 
$$
we deduce that a transformation of the type (\ref{eq158}) effects the
following changes. 
\begin{equation*}
\left.
\begin{aligned}
&\times \rightarrow \times [ u' ] + S_o && y_1 \rightarrow y_1\\
&Y_2 \rightarrow y_2 [u_4{'} ] . &&V \rightarrow u_3 + V u_4{'} 
\end{aligned}
\right \} . \tag{163}\label{eq163}
\end{equation*}

We need a few notations. We shall denote by 
\begin{enumerate}[1)]
\item $\mathfrak{K}_{\mathfrak{K}, n-n} $ the set of all matrices $V =
  V^{(\mathfrak{K}, n - n)} = (\vartheta_m \nu)$ satisfying the
  conditions: $-\dfrac{1}{2} \leq \vartheta_{\mu \nu} \leq +
  \dfrac{1}{2}$ for all $\mu , \nu$,  

\item $\mathfrak{K}_n$ the set of all matrices $\times = x^{(n)}$ with  
$$
\times = \times ', - \frac{1}{2} \leq \times_{\mu \nu} \leq +
\frac{1}{2}, \mu , \nu = 1,2,\ldots r 
$$

\item $\mathfrak{K}_{n'}$ the set of all reduced positive matrices $y
- y^{(n)} $ (in the sense of Minkowski) 
\end{enumerate}

By an\pageoriginale appropriate choice of $u_4$ in (\ref{eq158}), we
can obtain $y_2 [ 
  u_4{'}]$ in (\ref{eq163}) as a reduced matrix, in other words $Y_2 [
  u_4{'}] \in \mathfrak{K}_{n-n}$ and in general, $u_{\Lambda}$ is
uniquely determined. We can then choose an integral matrix $u_3$ such
that in (\ref{eq163}), $u_3 v u_4{'} \in \mathfrak{K}_{\mathfrak{K}, n-n}$ and
$u_3$ is also uniquely determined in general. The choice of $u_3$ and
$u_4$ fixes $u$ and then we can choose $S_o$ such that $[ u' ] + S_o
\in \mathfrak{K}_n$. In general $S_o$ is also uniquely
determined. These considerations prompt us to define a fundamental
domain $y_u$ for $A_2$ on $\mathfrak{K}$ as the set of $Z$.  

$Z = \times + i y = \times + i \begin{pmatrix}y_1 & o \\ o &
  y_2 \end{pmatrix} \begin{bmatrix}\begin{pmatrix} E & V \\ o &
    E \end{pmatrix}\end{bmatrix}$ where  
\begin{equation*}
\times \in y W_n, y_2 \in \mathfrak{K}_{n-n}, V \in
y_{n-n}. \tag{164}\label{eq164} 
\end{equation*}

For purposes of proving the convergence of the Poincare' series, we in
fact consider a majorant of it, viZ. the series  
\begin{equation*}
h (Z, T) = \frac{1}{\varepsilon (T_Z)} \sum_{S \in V_n} e^{-2 \pi
  \sigma (T y_S) } | y_2 |^{h/_{2}} \tag{165}\label{eq165} 
\end{equation*}
which arises from (\ref{eq160}) by replacing each term of (\ref{eq160}) by its
absolute value multiplied by a factor $| y |^{\mathfrak{K}} = | y
|^{\mathfrak{K}/_{2}} || CZ + D ||^{\mathfrak{K}}$ where we denote $S <
Z > = \times_S + y_2$. The absolute or uniform convergence of the
above series implies the corresponding fact for the Poincare'
series. In the sequel therefore, we shall concern ourselves with this
new series.  

We first prove the uniform convergence of $h (Z, T)$ on special
compact subsets of $\mathscr{Y}$, viz. the symplectic spheres and the
general case will immediately follow. Let $Z_0 \in \mathscr{Y}$ be
fixed and let $\mathfrak{K}_o$\pageoriginale  denote the symplectic
sphere $s (Z, Z) 
\leq \dfrac{1}{2}\rho$ Assume $\rho$ to be so chosen that
$\mathfrak{K}_o$ does not have a non empty intersection with any of the
$S (\mathfrak{K}_o) \mathscr{Y}$ without being identical with it, $S$
denoting an arbitrary modular substitution. The validity of this
assumption is an immediate consequence of the modular group being
discontinuous on $\mathscr{Y}$ (cf. Lemmas \ref{chap4:lem8}). Let $Z$, $Z^*$ be
arbitrary points of $\mathfrak{K}_\nu$. Then we have $s (Z, Z^*) \leq
\rho$ and $s (s <Z>)$, $S <Z^*> \leq S$, $S \in M$. With a view to
obtaining a convergent majorant for $h (Z^* T)$  which does not depend
on $Z^* \in \mathfrak{K}_o$ we work out the following estimations.  

If $Z = \times + i y$ and $y = \begin{pmatrix} r_n r \\ \ast
  y \end{pmatrix}$ with $y_1 = y^{(r)}_1$  then in view of the 
special form (\ref{eq157}) for $T$ we have 
$$
\sigma (T y) = \sigma (T_1 y_1). 
$$

Let $R = R^{(n)}$ be a real matrix such that $y = R R'$ and let $w_\nu
, u = 1, 2, \ldots r$ denote the columns of $R$. $\mathscr{E}$
denoting any real column, 
let $\mu_1 = \min_{\nu = 1} r_1 [\in]$ and $\mu_2 =
\mu_{\varepsilon=1} T_1 [\varepsilon]$, Then 
we have $\sigma (T_1 Y_2) = \sigma (T_1(\varepsilon)) =
\sum\limits^{\nu}_{\nu=1} \sigma (\Gamma_1 [W_\nu])$. 
Also
\begin{align*}
\mu_1 \sigma (y_1) & = \mu_1 \sum_{\nu = 1}^r   4'_\nu W_\nu \leq
\sum\limits^r_{\nu=1} T_1 [W_\nu]\\
& \leq \Gamma_2 \sum\limits^r_{\nu =1} W'_\nu W_\nu = \mu_2 \sigma y_1
\end{align*}
so that 
$$
\Gamma_1 \sigma (y_1) \leq (T y) \leq \mu_\nu \sigma (y_\lambda)
$$

Replacing $Z$ by $Z^\nu = x^r + 0 y^\nu, y^\nu = \left(
z^{\nu(\mu)}_\nu y_r\right)$.
 
\noindent
we have from the above
$$
\mu_1 \sigma (y^\nu_1) \leq \sigma (T j^2) \leq \mu_2 \sigma : y^\nu_1
$$

Since\pageoriginale $\mathscr{S}(Z, Z^* = \varrho)$, by means of Lemma
\ref{chap7:lem13}, we have 
$$
\frac{1}{m_1} \le \frac{\sigma(y^*_1)}{\sigma(y_1)} \le m_1
$$

Combining all these it results that
$$
\sigma (Ty^*) \ge \mu_1 \sigma (y^*_1) \ge \frac{\mu_1}{m_1} \sigma
(y_1) \ge m \sigma (Ty' , m \rangle 0 
$$

The interchange of $y$ and $y^*$ leads to an inequality in the
opposite direction and we have 
$$
m \le \frac{\sigma(Ty^*)}{\sigma (T y)} \le \frac{1}{m}
$$

The assumption on $Z, Z^*$ used to prove the above inequalities are
also true of  $S\langle Z\rangle$, $S\langle Z^*\rangle$,  $S \in
\mathscr{M}$ so that, writing 
$$
S\langle Z\rangle = x_s + tY_s , S\langle Z^*\rangle = \lambda^*_s + i
y^*_s 
$$
we also have
$$
m \le \frac{\sigma(Ty^*_s)}{\sigma (T y_s)} \le \frac{1}{m}  
$$

From Lemma \ref{chap7:lem13} we have an estimate for $|y^*_s|/|y_s|$ viz.
$$
\frac{1}{m_Z} \le \frac{|y^*_s|}{|y_s|} \le m_s
$$
and consequently $|y_s^*|^{k/Z}\le \mathscr{M}|x_s|^{k/Z}$ with
$\mathscr{M}= m_s^{k/Z}$. 

The general term of (\ref{eq165}) allows then the following estimation for
$Z \in \mathsf{E}_o$, viz 
$$
e^{-2\pi \sigma(Ty^*_s)} |y^*_s|^{k/2} \le \mathscr{M} e^{-2 \pi m
  \sigma (Ty_s,)} |y_s|^{k/2} 
$$

On integration, the above yields
$$
e^{-2\pi \sigma(Ty^*_s)} |y^*_s|^{k/2} \le \frac{\mathscr{M}}
{\daleth_o} \int\limits_{y \in R_o} e^{-2 \pi m \sigma (Ty_s,)} |y_s|^{k/2}
|y_s|^{n-1} dx To 
$$
where $\daleth_o$ is the volume of $\mathsf{E}_o$ and is given by 
$$
\daleth_o = \int\limits_{\daleth_o} |y|^{-n-1} [d x][dy] - \int\limits_{Z \in
  \mathsf{E}_o} |y_Z|^{n} [dx][dy] 
$$

Hence it\pageoriginale  follows that
\begin{align*}
h(Z^*, T) & \le \frac{\mathscr{M}}{\daleth_o} \frac{1}{\varepsilon
  (T_1)} \sum_{\varepsilon \in V_r} \int\limits_{Z \in E_2} \in^{-2 \pi m
  \sigma (T y_s)} |y_s|^{k/Z - n- 1}[dx_s][dy_s]\\ 
& =\frac{\mathscr{M}}{\daleth_o} \frac{1}{\varepsilon (T_1)} \sum_{ S
  \in V_r} \int\limits_{Z \in \in \langle E_2 \rangle } \in^{-2 \pi m \sigma
  (T y)} |y|^{k/Z - n- 1}[dx][dy]_{01 \in s}\tag{166}\label{eq166} 
\end{align*}

We now break the integral
$$
\int\limits_{s \langle E, o\rangle} e^{-2 \pi m \sigma (T y)} |y|^{k/Z - n-
  1}[dx][dy] 
$$
into a sum of integrals the domain for each of which can be brought
into a subset of the fundamental domain $'o'_n$ of $\mathscr{A}_n$ by
means of an appropriate substitution in $\mathscr{A}_n$. Noting that
the integrated above is invariant under the substitutions of
$\mathscr{A}_n$ we would have then expressed the above integral as a
sum of integrals, all taken over appropriate subsets of
$y_n^o$. This is done as follows. Since $y_n^o$ is a
fundamental domain for $\mathscr{A}_n$ in $\xi$ we have $\xi =
\bigcup\limits_{\mathscr{M} \in \mathscr{A}_n} \mathscr{M}
\langle y_n^O\rangle$. Hence 
$$
S\langle\mathfrak{K}_o\rangle = \varrho \langle\mathfrak{K}_o\rangle
\cap \mathscr{Y} \bigcup_{\mathscr{M} \in \mathscr{A}_n} \varrho
\langle\mathfrak{K}_o\rangle \cap M \langle y^o_n\rangle 
$$

Let $s\langle\mathfrak{K}_o\rangle \cap M^{\sim}_s
\langle y_n^o\rangle$ be non empty for  $\nu = 1, 2, \ldots ,
a_s$, the suffix $S$ in $M_s^{\infty}, a_s$ denoting that these depend
on $S$. (It is not difficult to show that $\mu_s$ is finite in each
case but we do not need this). Let further 
$$
\mathscr{L}_s^{(\nu)} = S \langle \tilde{\mathfrak{K}} \rangle \cap
M_\in^{(\nu)} \langle y_n^o \rangle, \nu = 1, R, \ldots , R_s. 
$$

Then $S \langle \mathscr{R}_o \rangle = \cup^{a}_{\nu = 1}
\mathscr{L}^{(\nu)}_S$. If $\mathscr{L}_s^{(\nu)} = M^{(\nu)^{-I}}
\langle \mathscr{L}_y^{(\nu)}\rangle$, then $\mathscr{L}^{(\nu)}_s
\subset \mathscr{L}$ for each $\lambda$ and we have clearly 
$$
\int\limits_{S\langle \mathfrak{K}_o \rangle} e^{-2 \pi m \sigma (Ty)}
|y|^{\frac{\mathfrak{K}}{2}-n-1} [dx][dy] 
$$
$$
= \sum_{\nu = I }^{a_s} \int\limits_{\mathscr{L}_{\mathscr{S}^{(\nu)}}}
e^{-2 \pi m \sigma (Ty)} |y|^{\frac{\mathfrak{K}}{2}-n-1} [dx][dy] 
$$\pageoriginale 
and consequently, from (\ref{eq166})
\begin{equation*}
h (Z^*, T) \le \frac{\mu}{\mathcal{J}_o \varepsilon(T_1)} \sum_{S
  \varepsilon V_r} \sum_{\nu = 1}^{a_s}
\int\limits_{\mathscr{L}_{\mathscr{S}}^{(\nu)}} e^{-2 \pi m \sigma (Ty)}
|y|^{\frac{\mathfrak{K}}{2}-n-1} [dx][dy] \tag{167}\label{eq167} 
\end{equation*}

We certainly know that the union $\underset{\substack{ S \in V_r
    \\ \nu = 1,2 \ldots a_s}}{\cup} \mathscr{L}_{\mathscr{S}}^{(\nu)}$
is contained in $y_r^o$. But something more is true, viz. this
union is actually contained in the set of $Z = \times + i y$ defined
by: 
$$
Z \in y_r, |y| \le t
$$
if $t$ is chosen sufficiently large. For $|y|$ is invariant under the
substitutions in $\mathscr{A}_r$ and then we need only prove that
$|y|\le t$ for  
$$
Z = \times + \tau y \in \cup_{\nu = 1}^{a_s}
\mathscr{L}_{\mathscr{S}}^{(\nu)} = S.  
$$

Consider now the set of points $\{S<Z_0>/ S \in V_r\}$.  Being a set
of equivalent points it has a highest point in it, say  $\mathscr{S}_i
<Z_0>$. 
 Then if $\mathscr{S}_i <Z_0> = \times _{os}+ i y_os $, we have $ ||
 y_{{os}_{i}}|| \geq ||y_{os}|| $ for every$ S \in v_r$ Let $Z$ be an
 arbitrary point is $S<\mathfrak{K}_0>$. 
 Then $\mathscr{S}(Z,s<Z_0>) \leq \dfrac{1}{2}  \mathcal{P}$ so that
 by Lemma \ref{chap7:lem13} 
 $$
 \frac{1}{m_2} \leq \frac{|y|}{|y_{os}|} \leq m_2, m_2 = n2 (\rho ,n) .
 $$ 
 
 Thus $|y| \leq m_2 |y_{os}| \leq m_2 |y_{{os}_i} | = t $ (say). Then
 with this $t$ (which is a fixed number) we have that
 $\bigcup\limits_{\substack {s \in V_r \\ u=1,2 \ldots
     0_{S}}}$\pageoriginale  is  contained in the set of $Z ' s$
 defined by:   
 $Z \in r_r$, $|y| \leq t$ If now $\in _0$ denotes the number of modular
 matrices $s$ having $Z_0$ as a fixed point, viz. $s<Z_0> =Z_0$, we
 know from Lemma \ref{chap7:lem11} that  
 $\in _0$ is finite and we want to show that every point of $y_r$
 appears at most $e^2_0$ times in the union $\bigcup\limits_{\substack
   {s \in V_r \\ {u=1,2,}}}  \mathcal{L}^\nu_{s_{a_s}  }$ In other
 words, we have got to show that if $\exists$ denotes the set of all
 pairs $(\nu, s)$ with $Z \in \mathcal{L}^{(\nu)}_s$ where $>$, is a
 fixed point in $\mathfrak{f}_r$, then this set has at most $e^2_0$
 elements. If $(\nu, s), (\nu_0, s_0)$ be two pairs in $\mathfrak{f}$,
 then $Z \in M^{\nu^{-1}}_s \langle\mathcal{L}^{(\nu)}_s\rangle
 \subset m_s^{{(\nu)}^{-1}}s \langle\mathfrak{K}_0\rangle$ and
 similarly  $Z \in M^{(\nu)_0^{-1}}_{s_0
 }\langle\mathcal{L}^{(\nu_0)}_{s_0}\rangle \subset
 M_{s_0}^{{(\nu)}^{-1}}s_0 \langle\mathfrak{K}_0\rangle$. This means
 that the two spheres $M^{(\nu)^{-1}}_s S \langle\mathfrak{K}_0\rangle,
 M^{(\nu_0)^{-1}}_{s_0} \langle\mathfrak{K}_0\rangle $ have one point
 in common and hence the spheres themselves are identical. In
 particular, their centres are equal so that  
 $$
 M^{{(\nu_0)}^{-1}}_{s_0} S_0 \langle Z_0 \rangle = M^{{(\nu)}^{-1}}_S
 S \langle Z_0 \rangle. 
 \text{i.e.}  Z_0 = S^{-1}_0 M^{(\nu_0)}_{s_0} M^{(\nu)^{-1}}_s S
 \langle Z_0 \rangle. 
 $$
 
Keeping $(\nu_0, S_0)$ fixed, we have now produced for each pair $
(\nu, S) \in \mathfrak{f} $ a substitution $ S^{-1} _0 M^{(\nu_0)}
_{s_0} M_{s}^(\nu)^{-1}_s S$ which has $Z_0$ as a fixed point and
which consequently belongs to a finite\pageoriginale  set of $e_0$
elements. It 
easily follows then that the possible choices for $(\nu, S)$ are at
most $e^2_0$ in number. Turning to our series $h(Z^*, T)$ and its
majorant in (\ref{eq167}), we can now state that  
\begin{align*}
h(Z^m,T) & \leq \frac{M}{\mathcal{J}_0 \in (T_1)} \sum_{s \in v_{r}}
\sum^{a_{s}}_{\nu =1} \int \limits_{\substack {Z \in
    \mathcal{L}^{(\nu)}_{s} \\ {|y| \leq t }}} e^{-2 \pi m \sigma
  (Ty)} |y|^{\frac{R}{2}}-n-1 [dx][dy] \\ 
&  \leq \frac{Me^2_0}{\mathcal{J}_0\in (T_1)} \int \limits_{\substack
  {Z \in y_r \\ {|y|\leq t }}} e^{-2\pi m \sigma
  (Ty)}|y|^{\frac{R}{2}}-n-1 [dx][dy] \\ 
& = \frac{Me^2_0}{\mathcal{J}_0\in (T_1)} F (m T_1, t)  \text{(say)} 
\end{align*}

It only remains therefore to investigate under what conditions the
last integral $F(m T_1, T)$ exists. First we treat the case $0<
r<n$. In this case we have the parametric representation for $y> 0$
from (\ref{eq132}) as 
\begin{equation*}
y=
\begin{pmatrix}
y_1 & o \\
o & y_2
\end{pmatrix}
\begin{bmatrix}
\begin{pmatrix}
E&v\\
o&E
\end{pmatrix}
\end{bmatrix}
\end{equation*}
where $y_1 =y_1{(r)}$, $y_2=y_2^{(n-r)}$, $\nu=\nu^{(r,n-r)}$ and if
$Z=\times + i y \in y_ {r}$ then $\times \in XXXXX \rho_n$, $\nu
\in \mathfrak{K}\rho_{r,n-r}$ and $y_2 \in \mathfrak{K}_{n-r}$ As with
(\ref{eq145}), we can show that  $\dfrac{\alpha(y)}{\alpha(y_1,y_2,
  \nu)}=|y_1|^{n-r}$ and consequently  
$$
[dy]=|y_1|^{n-r}[dy_1][dy_2][dv]
$$
(We may remark that in $[d\nu]$ we have the product of the
differentials of all the $r(n-r)$ elements of $V$ as $\nu$ is not in
general symmetric). 
Also $|y|=|y_1| |y_2| \text{and} \sigma(Ty)= \sigma(T_1y_1)$. Hence  
\begin{align*}
& F(m T_1, t) \equiv \int \limits_{\substack {Z \in \mathcal{L}_{r}
      \\ {|y| \leq t }}} \varphi^{-Z a m \sigma (T
    y)}|y|^{k/Z-n-1}[d \times ][dy] \\ 
& = \int \limits_{\substack {\times \in \mathfrak{K}\varphi_{n}.\nu
      \in\mathfrak{K}\varphi_{n}, n. r, |y_1||y_2| \leq t \\ {y_1>0,
        y_2 \in \mathfrak{K}_{n-r}|y_1||y_2| \leq t }}} e^{-2 \pi m
    \sigma (T_i y_i)} |y_i|^{\mathfrak{K}/2,n-1} [d x][d y_i][d y_i][d
    y]\\ 
& =\int \limits_{\substack {y_i>0, y_2 \in \mathfrak{K}_{n-r} \\ {|y_i|
        |y_2| \leq t }}} e^{-2 \pi m \sigma (T_i y_i)}
  |y_i|^{\mathfrak{K}/2 -r-1} |y_2|^{\mathfrak{K}/2-n-1}[d y_i][d y_2] 
\end{align*} 
since the above integrand is independent of $\times$, $\nu$ and the
volumes of $\mathfrak{K} \varphi_n$, $\mathfrak{K}\varphi_r$, $n-r$ which are
the domains for $\times$, $\nu$ are both unity. Thus, using (\ref{eq146}) and
(\ref{eq143}) we can now state that  
\begin{align*}
& F(m T_i, t) = \int \limits _{y_i>0} e^{-2 \pi m e(T,
    y_i)}|y_i|^{\mathfrak{K}/2-m-1} \bigg(  \int \limits_{\substack
    {y_2 \in \mathfrak{K}_{n-r} \\ {|y_2|  \leq t/|y_i| }}}
  |y_2|^{\mathfrak{K}/2 -r-e}[i y_2]\bigg)[d y_i]\\ 
&= \frac{n-r+1}{\mathfrak{K}-n-r-1} \vartheta_{n-r}
  t^{\frac{\mathfrak{K}-n-r-1}{2}} \int{y_i>0} e^{-2\pi m \sigma (T_i
    y_i)}|y_i| \frac{n-r-1}{2} [d y_i]\\ 
&= \frac{n-r+1}{\mathfrak{K}-n-r-1} \vartheta_{r-r}
  t^{\frac{\mathfrak{K}-n-r-1}{2}}\pi ^{\frac{n(r-1)}{4}}(2\pi
  m)^{\frac{-rt}{2}}|r| ^{-n/2}\prod ^{r-1}_{\nu= a}
  r(\frac{n-\nu}{2}) 
\end{align*}
provided $k>n+r	+1$.

In particular we have shown the existence of $F(m T_i t)$ under the
assumption $k> n + r+1 \; 0 <r <n$. We now discuss the border cases $r =
0, n$. 

Let\pageoriginale $r = 0$. Then using (\ref{eq146}) 
\begin{align*}
& F(m T_i, t) = \int  _{\times \in n \varphi_ n, y \in \mathfrak{K}n,
    |y| \leq t} |y|^{\mathfrak{K}/2-n-1} [d \times ][d y]\\ 
&= \int _{y \in \mathfrak{K}_n, |y| \leq t} |y|^{H/2 -n-1}[d y]\\ 
&= \frac{n+1}{\mathfrak{K}-n-1} \vartheta_n
  t^{\frac{\mathfrak{K}-n-r-1}{2}}  
\end{align*}
provided $k>n+1$ Let now $r=n$. Then  
\begin{align*}
& F(m T_i, t)= \int \limits _{\times \in  \mathfrak{K} \varphi n, y>0,
    |y| \leq t }e^{-2 \pi m \sigma (T y)}|y|^{\mathfrak{K}/2-n-1} [d
    \times ][d y]\\ 
&\leq \int _{y >0 } e^{-2 \pi m \sigma (T y)}|y|^{\mathfrak{K}/2-n-1}
  [d y]\\ 
&= \pi ^{\frac{n(n-1)} {4}}(2 \pi m) ^{\frac{n(n+1-
      \mathfrak{K}}{2}}|T| ^{\frac{n+1-\mathfrak{K}}{2}} \pi ^ n_{\nu
    =i} \lceil (\frac{\mathfrak{K}-n-\nu}{2}) 
\end{align*} 
provided $k > 2n$.

It follows  that in any case provided $k > \min (2n, n + r + 1)$, $r =$
rank $T$ the Poincare' series $g(Z, T)$ converges absolutely and
uniformly in a neighbourhood of every point $Z e y$ and
hence  also in every compact subset of $y$. 

Since by our assumption $k$ is an even integer, the above condition on
$k$ actually simplifies into  
$$
k>n + 1 + \text{rank} T.
$$ 

In the case $n= 1$, we can say something more. In fact, the Poincare'
series in this case has a convergent majorant which does not depend on
$Z \in \in _n$ so that $g(Z, T)$ is actually bounded in $\lceil
_n$. This result is a consequence of the following  

\setcounter{lem}{15}
\begin{lem}\label{chap9:lem16}%lem 16
 Let\pageoriginale $\mu, \vartheta, \times, y$ be real numbers, and let $|x|
   \leq b$ and $y \geq \delta >0$ for suitable constants $\mathscr{C},
   \xi$. If  $Z= x+ i y$, then  
 \begin{equation*}
|\mu Z + \vartheta| \geq \in |\nu i + \vartheta| \tag{168}\label{eq168}
\end{equation*}
with a certain positive constant $\in = \in (\delta , \mathscr{C}$.  
\end{lem}

If $\mu^2 + \nu^2 = 0$, (\ref{eq168}) is trivially true so that we can
assume $\mu^2 + \nu^2>0$. Then, due to the homogenity of (\ref{eq168}) in
$\mu$, $\nu$ we can in fact assume that $\mu^2 + \nu^2 = i$.  
The proof is by contradiction. If the Lemma were not true, there
exists sequences $Z_n= \mu _n + i y_n, u_n, \vartheta_n, n=1,2
\ldots,$ with $\mu^2_n+ \nu^2_n=|x_n| \leq \mathscr{E}$, $y_n \ge
\delta$ and finally  $|u_n 2_n  +  \vartheta_n|^2 = ( \nu _n y_n)^2
+(\nu_n x_n + \vartheta_n)^2 < 1/n$. So, in particular $u_n^2 < 1/n
y^2_n \le 1/n \delta^2$ so that $u_n \rightarrow 0 XXXXX n \to \infty
$ This implies, since $|x_n| \le \zeta$ that $u_n  x_n \to o$ and
consequently $\vartheta_n \to  o$. But  this is impossible as $u_n^2
+ \vartheta^2_n =1$ 1 for every $n$. This  proves our Lemma.    

Reverting to the Poincare'  series in the case $n=1$, we have
$$
g(Z,T)= \frac{1}{\varepsilon(T_1)} \sum\limits_{o_{1 V_n}} e^{Z ,
  *****} | e Z + D1 
$$

If  $r=1$, the above series is majorised by the series  
$$
\frac{1}{\varepsilon(t)} \sum\limits_{S o V_1} e ^{2 \pi t o y } |c2-
D1^{-R} 
$$
obtained by taking absolute values termwise and in the alternative
case, vie, $r=0$, it is majorised by $ \sum\limits_{S g V_o} |c L +D1$  

For\pageoriginale  $Z \in \delta_1$ , whence $|x| \le  \zeta = 1/2$
and $|y| \ge 
\delta = \sqrt{3}/2$, $Z= x+y$, either of these has the convergent
majorant $\zeta \sum |ei  + d 1^{-\mathfrak{K}}$ independent of $Z$ by
means of Lemma \ref{chap8:lem15}. 

We one summarize our results into the following


\begin{thm}%the 9
 The Poincare, series (\ref{eq152}), Viz. 
$$
g (Z,T) = \sum\limits_{S \in V(T)} e^{2 \pi \sigma(T S < Z>)}
|CZ|D1^{\mathfrak{K}}, S= \begin{pmatrix} A&B \\ C &D \end{pmatrix}  
$$
converges absolutely and uniformly in every compact subset of $
  \mathscr{Y}$  and represent a modular form of weight  $k$ for
$\mathfrak{K} \equiv o(2)$,  $k > n +  1+ rank T$, where  $T$
  stands  for an arbitrary semi integral semi positive matrix.  
\end{thm}


Later,  we shall prove that Poincare' series actually generates all
modular forms of degree $n$  and weight $k$  provided $k \equiv 0(2)$
and $k > 2n$. This proof requires a new tool, viz, the matrization of
modular forms and the next section is devoted  this topic.  

