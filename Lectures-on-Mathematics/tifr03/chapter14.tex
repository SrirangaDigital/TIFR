

\chapter{Indefinite Quadratic forms and modular forms}%%% 14 

We proceed\pageoriginale to investigate how far the results of the last
section can 
be carried over to the case of indefinite quadratic forms. The very
first concept we introduced in the last section was that of the
representation number $\alpha (S, T) $ which stood for the number of
integral solutions of the matrix equation $S[x] = T$ where $ S = S
^{(m)}$ and $ T = T^{ (m)}$ are given positive integral matrices. The
problem of determining the number of integral matrices $x^{(m, n)}$
satisfying the above equation in the case of an indefinite $S$ is 
much more difficult. In fact an indefinite rational symmetric matrix
$S$ is known to have in general an infinity of units, viz. unimodular
matrices $u$ with $ S[u] = s$. If now $G$ is one integral solutions of
the equation $S[X] = \tau$ so is $u G$ for every unit $u$ of $S$ so
that the number of integral solutions of this equation will in general
then be no more meaningful, and we therefore replace it by the
concept of \textit{representation measures}. If we divide the set of
all integral solutions $G$ of $S[G] = T$ into equivalence classes,
stipulating that two different $G' s$, say $G_1$, $G_2$ belong to the
same class if and only if $G_2 = u G_1$ for some unit $u$ or $S$, the
resulting number of distinct classes can be shown to be finite. To
each to these classes we attach a positive weight in a particular way
and then the sum of these weights for all the classes will yield the
representations measures $ \alpha _A (S, T)$. The following
considerations are more general in so far as they treat the notion of
representation measure $\alpha_A (S, T)$ for the set of solutions of
the inhomogeneous equation\pageoriginale $S[G +A] = T$.   

Let $S = S^{(m)}$ be a real non-singular matrix and let $(p, q )$ be
the signature of $S$. That is to say, for a suitable non-singular
matrix $C$ we have  
$$
S = S_ o [C] \text { with } S_o =\left ( \begin{smallmatrix} -
  \Sigma^{ (p)} & 0 \\ 0 & \Sigma^{ (p)} \end{smallmatrix} \right) 
$$

While $(p. q)$ is uniquely determined by $S$, $C$ is not necessarily
uni\-que. Let us set $ \mathscr{G} = C \in, \in = (x_u)$
and $\mathscr{G} = (y _m ) $ 

Then we have 
$ S [\in] = - y^2_1 - y^2_2 -- y^2_p + y^2_{p +1 } + y^2_{p +1
} , p + q, p + q = m $ Besides $S[\in]$, we also consider the
positive form $P [\in ] = y^2_1+ y^2_2 + quad y^2_{p +q }$ .  

We call the symmetric matrix $P$ defined by the above equation, a
majorant of $S$. This majorant is not unique as it evidently depends
on the choice of $C$. However, any majorant $P$ of $S$ is easily seen
to be characterized by the relation  
\begin{equation*}
PS^{ -1} P = S , P > 0 \tag{255}\label{eq255}
\end{equation*}

The equation (\ref{eq255}) is clearly invariant under the simultaneous
transformation $S \to S[V]$, $P \to P[V] $ where $V$ is an arbitrary non
singular real matrix. It is then immediate that the majorants $P_o$ of
$S_o$ defined by  
\begin{equation*}
P_o S_o^{ -1} P_o = S_o, P_o > 0 \tag*{$(255')$}\label{eq255'}
\end{equation*}
yield all the majorants $P$ of $S$ in the form $P = P _o [C]$ where $C$,
$S_o$ have the same meaning as before. To determine a parametric
representations for the solutions of (\ref{eq255}) it therefore suffices to
consider\pageoriginale the special case \ref{eq255'}. In the border
cases $p = 0 $ or $q 
= 0 $ we should have clearly $S_o = E$ and $S_o = -E$ respectively so
that the only possible solutions of (\ref{eq255}) are $P = S$ and $P = -S$
in the respective cases. For $pq > 0$, a parametric representation for
the solutions $P_o$ of \ref{eq255'} is given by  
\begin{equation*}
P_o = 2K - S_o, K = Y ^{-1}[x' S_0], y = S_o [x] > 0\tag{256}\label{eq256}
\end{equation*}
where $X$ is any real matrix of the type $x^{ \mu \nu}$ with rank $x =
q$ and the solutions of (\ref{eq255}) are given by $P = P_o [C]$, $P_o$, being
represented by (\ref{eq256}). Writing $x ' = (W_1   W_1)$ with and 
we observe that  
$$
y = S_o [X] = ( W', W'_2) S (^{W_1} _{W_2}) = W'_1W_1 +W'_2 W_2 
$$
so that the condition $Y > 0$ in (\ref{eq256}) in particular implies that $
W'_2 W_2 > 0$ and consequently $|W_2| \neq 0$. It is easy to see that
two matrices $X_1 , X_2$ yield the same point in (\ref{eq256}) if and only if 
$X_2 = X_1 H$ with a non-singular matrix $H$. We can
therefore assume in (\ref{eq256}) by replacing $X$ by $X W^1_2 = (^W_E)$
with $W = W_1W^1_2$ that $X$ is always of the normalized form $X =
(^W_E)$ Given, $X$, $P_o$ is uniquely determined in (\ref{eq256}) but not
necessarily is the converse true in general. However $P_o$ does
determine $W$ uniquely, in other words $X$ is uniquely determined by
$P_o$ if we stipulate that $X$ is always of the above normalized
form. What is more, the transformation from $W$ to $P = P_o [C]$ can
be easily shown to be bi rational. Specially with $x = (^W_E)$ we have  
\begin{equation*}
P = C 
 \left ( \begin{smallmatrix} ( \frac{E + WW'}{E - WW'} & - 2 w (E -
   W-W)^1 \\ - 2 w (E - W-W)^1 & {E + WW'}{E - WW'} \end{smallmatrix}
 \right ) C , E - WW > 0 \tag{257}\label{eq257} 
\end{equation*}

Let\pageoriginale $u = \left ( \begin{smallmatrix} A & B \\ C &
  D \end{smallmatrix} \right)$ with $A = A^{(p)}$ be a real matrix
which transforms $S_o$ into itself, i.e., $S_o [u] = S_o$. Then the
transformation $X\to u X$ has on $P_0$ and $W$ the effect   
\begin{equation*}
P_o \to P_o [ u^{-1} ], W \to u <w> = (AW + B)(CW +
D)^{-1}\tag{258}\label{eq258} 
\end{equation*}

The transformations $W \to W^* = u <W>$ constitute a group  $1 - 1$
transformations of the domains $E - W' W > 0$ onto itself and in this 
group we have a sub-group formed by the units of $S_o$ viz., those
among the $u's $ which are further unimodular. This group of units can
be shown to be a discontinuous group acting on the space $E - W' W >
  0$ and an immediate question is about a fundamental domain for this
  group or one of its sub groups and the volume of the fundamental
  domain in an appropriate sense. Towards this effect we note that the
  space $E - W'W > 0$ can be considered as a Riemannian space with a
  metric, invariant relative to the transformation (\ref{eq258}), given by  
\begin{align*}
8 ds^2 & = \sigma (p^{-1} d p)^2 = \sigma (P_o ^{-1} d P_o)\\
& = \sigma ((E - WW')^{-1} dW (E - W' W)^{-1} dW') \tag{259}\label{eq259}
\end{align*}

The invariant volume element in this metric can also be computed to be  
\begin{equation*}
d \vartheta = |E - WW' 1^{\frac{m}{2}} [dW] \tag{260}\label{eq260}
\end{equation*}

All these considerations are valid for every real matrix $S$with
signature $p$, $q$. We now restrict the domain of $S$ to suit our
needs.  

Consider\pageoriginale the inhomogeneous equation 
\begin{equation*}
S[ G + A ] - T = 0 , G = G ^{(m , n)} \text{ integral } (m \geq n )
\tag{261}\label{eq261} 
\end{equation*}
and require that the left side of this equation should represent an
integral valued matrix with the elements of $G$ as variables, via., a
matrix whose elements are polynomials in the elements of $G$ with
integral coefficients.  

This means the following requirements on $S$, $A$ and $T$
\begin{enumerate}[(i)]
\item $S$ is semi integral 

\item $SA$ is integral for $n > 1$ and $2 SA$ is integral for $n = 1$

\item $T \equiv S[A] (\mod 1) i.e.$, $S[A] - T$ is integral. 
\end{enumerate}

Confining ourselves to the above case, let us introduce the theta
series  
\begin{equation*}
f_A (Z, P) = \sum_{ B \equiv (\mod 1)} e^{ 2 \pi i \sigma ( S[B] x + i
  P[B]y}), z = x + iy \tag{262}\label{eq262} 
\end{equation*}
where $P$ denotes an arbitrary majorant of $S$. In the case $ p = 0$
we should have $P = S$ and then $\mathfrak{f}_A (Z, P)$ represents a
theta series in the usual sense. The case $q = 
0$ presents no new difficulty either so that we confine our attention
to the case $ p q > 0$. Since $\mathfrak{f}_A (Z, P)$ depends only
upon the coset $A (\mod 3)$ and $A[\varepsilon]$ has a bounded
denominator, it is clear that we only we have a finite number of
different series $\mathfrak{f} _A (Z, P)$ corresponding to a given
majorant $P$ of $S$. Let $\mathfrak{f} (Z, P) $ denote the column with
the $\mathfrak{f} (Z, P)'s$ as elements in a certain order. Due to an
arbitrary unimodular substitution\pageoriginale $M=\begin{pmatrix} A &
B \\ C & d \end {pmatrix} \in \mathcal{M}$ on $f (Z,P)$, one obtains   
\begin{equation*}
|CZ+D|^{- \frac{P}{2}}|C \bar{Z}+D|^{-\frac{q}{2}} f (M < Z >, \rho) =
\mathcal{L}_M f(Z, P) \tag{263}\label{eq263} 
\end{equation*}
with a non-singular matrix $\mathscr{L}_M$. The matrices
$\mathscr{L}_M$ belong to the modular group in case $p \equiv q \equiv
o (\mod 2)$, and (\ref{eq263}) is quite meaningful for positive definite $S$
too. Finally, it can be shown that $\mathscr{L}_M$ is the unit matrix
for $M \equiv \begin{pmatrix} E & o \\ o & E \end {pmatrix}
\pmod{\lambda}$ for a suitable integer $\lambda$, so that the matrices 
$\mathscr{L}_M$ define a representative system for the quotient group
$\mathcal{M}/\mathcal{M}(\lambda)$ where $M (\lambda)$ denotes the
main congruence group to the level $\lambda$ i.e. $M (\lambda)$
consists of the substitutions $M$ satisfying the congruence $M
\equiv \begin{pmatrix} E & o \\ o & E \end {pmatrix}$ \; $\pmod{\lambda}$   

In the group of units $\mathcal{U}$ of $S$, viz. unimodular matrices
$\mathcal{U}$ with $S[\mathcal{U}]=S$, let $y_A(S)$ be the sub
group defined by $\mathcal{U}A \equiv A(\mod 1)$. We denote by
$\mathcal{F}_A(S)$ a fundamental domain in $E-W'W >o$ relative to this
sub-group and introduce the volume 
\begin{equation*}
V_A(S)=\int \limits_{\mathcal{F}_A(S)}d \vartheta \tag{264}\label{eq264}
\end{equation*}
on $\mathcal{F}_A(S)$ computed with the volume element (\ref{eq260}). This
volume can be shown to be finite in all cases with one exception-the
exceptional case being one for which $m$ is 2 and $\sqrt{-|S|}$ is
rational. From (\ref{eq262}) it is easily seen that 
\begin{equation*}
\mathcal{F}_A(Z,P[\mathcal{U}])=\mathcal{F}(Z,P) \tag{265}\label{eq265} 
\end{equation*}
for $\mathcal{U} \in y_A (S)$ Therefore it makes sense to  form
\pageoriginale the integral mean value 
\begin{equation*}
g_A (Z,
S)=\frac{1}{V_A}_{(S)} \int\limits_{\mathcal{F}_A(S)}
f_\mathfrak{K}(Z,P)d V \tag{266}\label{eq266}   
\end{equation*}
This integral certainly exists provided $2 n < m-2$ 

It is remarkable that the column $\mathscr{G}(Z,s)$ with elements
$g_A(Z,s)$ (in some order) also satisfies a transformation formula of
the same kind as $\rho (Z,P)$ does in (\ref{eq263}). 

Specifically we have 
\begin{equation*}
|Cz+D|^{-\frac{p}{2}}|C\bar{Z}+D|^{-\frac{q}{2}}\mathscr{G}(M
<Z>,S)=\mathscr{L}_M \mathscr{G} (Z,S) \tag{267}\label{eq267}  
\end{equation*}

Developing $g_A(Z,S)$ into a Fourier series there results an expansion
which differs from the usual $\vartheta$- series expansion (\ref{eq241}) for
the definite case only in so far as the representation numbers
$\alpha(S,T)$ \qquad in (\ref{eq241}) are to be replaced by the
representation measures, $\alpha_A(S,T)$ and the exponential function
$e^{-2 \pi \sigma (t Y [Q]})$ by certain generalized confluent hyper
geometric functions. 

If we replace the representation measures in the Fourier series of
$g_A(z,s)$ by the products of the $p$-adic solution densities,
consequent on a main theorem of Siegel concerning indefinite quadratic
forms, we obtain a new representation of $g_A(z,S)$ which yields a
partial fraction decomposition of the kind 
\begin{equation*}
g_A(Z,S)=\sum_{C,D}h_A(C,D,S)|CZ+D|^{\frac{-p}{2}}| C \bar{Z} +
D|^{\frac{-q}{2}} \tag{268}\label{eq268}  
\end{equation*}
where $(C, D)$ runs over a complete set of non associated symmetric
coprime pairs and $h_A(C,D,S)$ denotes a certain sum of the kind
(\ref{eq254}). As against the definite case we have in (\ref{eq268})
not only a 
means of expressing Siegel's main theorem as an analytic identity but
actually it permits us to prove Siegel's result by analytical
means.\pageoriginale Towards this effect, one has to show that the
Eisenstein series 
on the right of (\ref{eq268}) have the same transformation properties
relative to the modular substitutions as $g_A(z,S)$ and that the
Eisenstein series can be developed into a Fourier series of the same 
kind as that of $g_A(z,S)$. These facts coupled with some special
properties of the generalized confluent hyper geometric function
yield, as M.Koecher could prove (unpublished) that the two sides of
(\ref{eq268}) are identical upto a constant factor, which factor becomes
1 after a suitable normalization. 

Now the question presents itself whether functions of the type (\ref{eq268})
which no longer depend analytically on the elements of $Z$ can be
characterized in any way. As we shall see in the next section, such a
characterization is possible by a system of partial differential
equations with the invariance properties we have got to require of
them. Another fact which appears at the outset in the case $n=1$ can
also be reasonably fused into this differential equation-theory. It
concerns the following problem:- 

In the case $n=1$ if we apply the Mellin's transform to Siegel's zeta
functions of indefinite quadratic forms with the signature $(p,q)$ we
obtain a function of the type 
\begin{equation*}
\sum_{(C,D)}h_A(C,D,S)|cz+D|^{-\alpha}|c\bar{z}+D|^{-\beta}
\tag{269}\label{eq269}   
\end{equation*}
instead of (\ref{eq268}), with certain exponents $\alpha,\beta$ which
satisfy the relations 
$$
\alpha \equiv P/2, \beta \equiv q/2 (\mod 1), d+\beta = 1/2 (p+q)  
$$

We now ask for a process which yields a direct correspondence between
\pageoriginale the type (\ref{eq268}) and (\ref{eq269}) without the
use of Dirichlit series. We  
shall in the following section that such a correspondence can be
defined through certain differential operators. 

For further details on some of the points raised in this section we
refer to  
\begin{enumerate}
\item H.Maass, Die Differentialgleichungen in der Theorie der
  elliptiochen Modulfunktionen, Math. Ann. 125(1953), 235-263. 

\item C.L.Siegel, On the theory of indefinite quadratic forms,
  Ann.\break Math. 45(1944), 577-622. 

\item $''$ \qquad $''$ , Indefinite quadratische Formen und Modulfunksionen,
  Studies Essays, Pres. to Courant, New York 1948, 395-406. 

\item $''$ \qquad $''$ , Indefinite quaderatische Formen und Funtionen
  theoie I, Math. Ann. 124(1951), 17-54. 
\end{enumerate}

