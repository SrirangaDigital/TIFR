

\chapter{Algebraic dependence of modular forms}%%%%% chpater 6

We are\pageoriginale interested here in the question: when  a modular form of
degree $n$ vanishes identically, in other words, when two when two
given modular forms  of the same degree are identical. The following
theorem provides a useful criterion in this direction. 

\setcounter{thm}{5}
\begin{thm}\label{chap6:thm6} % them 6
 Let $s_n$ denote the least upper bound of 
$\sigma(Y^{-1})$  for $Y^{-1}$ such that $X + i Y \in
f_n$ and let
$$
\mathfrak{f}(Z)=  \sum_{T  \ge 0} a(T) e^{2 \pi i \sigma(T Z)}
$$
 be a modular form of degree $n$ and weight  $k \ge 0$.  If
$a(T) =0$   for $T$  such that $\sigma(T) \le
\dfrac{\mathscr{R}}{4 \pi} s_n$ (i.e. if a certain finite set of
Fourier coefficients vanish), then $f(z)$ vanishes identically.  
 \end{thm} 

 \begin{proof}
First we note that $\{s_n\}, n=1,2, \ldots ,$ is an increasing
sequence. For, we know from (\ref{eq82}) that  
$$
\sigma(y^{-1}) \le \frac{na}{y_{M}} \le \frac{2n C_1}{\sqrt{3}} \text{
  for } Z=  x+o y \in \mathcal{F}_n . 
$$
 \end{proof} 
 
 In particular then, $S_n \le \dfrac{2nc_1}{\sqrt{3}} < \infty$
 Let now $Z_1 \in \mathcal{F}_{n-1}$. We claim that
 $Z= \begin{pmatrix} Z_1 & 0\\0& M\end{pmatrix}\in \mathcal{F}_n$
   provided $\lambda$ s sufficiently large. We prove this as
   follows. Let  $Z= x+ i yu, Z_1 = x_1+y_1 $ Then
   $y=  \begin{pmatrix} y_1& 0\\0 & \lambda \end{pmatrix} = (y_{m})$
   say. 
 
 We have to verify the reduction condition of Minkowski for $Y$. Let
 $\mathscr{Y}_\mathscr{R}= \begin{pmatrix}\mathscr{Y}^*_\mathscr{R}
   \\ g_n \end{pmatrix}$ be a primitive column with the integers $g_1,
 g_2, \cdots g_n$ as elements.  Then  we have 
 $$
 y[Y_{\mathfrak{K}}]= y_1[\mathscr{Y}_{\mathfrak{K}}] + \lambda g^2_n
 $$\pageoriginale
 
 If $g_n =0$, then  $g_{\mathfrak{K}}, g_{\mathfrak{k}+1} \cdots
 g_{n-1}$ are themselves coprime and then $Y_1[y_\mathfrak{K}] \ge y_{\mathscr{K}
   \mathscr{B}}$ as $y_1$ is reduced and $k$ is necessarily less than
 $n$. It then follows that
 $Y[\mathscr{Y}_{\mathfrak{K}}]=y_1[\mathscr{Y}_{\mathfrak{K}}] \ge
 y_{\mathfrak{K} \mathfrak{K}}$ If $y_n \neq 0$, then
 $y[\mathscr{Y}_{\mathfrak{K}}]=y[\mathscr{Y}_{\mathfrak{K}}]+ \lambda
 g_{\mathfrak{K}} +\lambda $, so that $y[\mathscr{Y}_{\mathfrak{K}}] \ge
 y_{\mathfrak{K}\mathfrak{K}}$ provided we choose $\lambda \ge g_{n-1},
 n-1$ Trivially $y_{\mathfrak{K}\mathfrak{K}+1} \ge o$ for all $k$ as
 this is true of  $y_1$. 
 
 It is now immediate that if  
 \begin{equation*}
\lambda \ge y_{n-1, n-1} \tag{104}\label{eq104}
 \end{equation*} 
 then $y= \begin{pmatrix} y_1 &0 \\ 0 & \lambda \end{pmatrix}$
 satisfies Minkowski's reduction conditions. Clearly, since  $y_1$ is
 reduced modulo 1, $\chi$ is also reduced modulo l. It only remains
 therefore to verify that $Z$ is a highest point in the set of all
 points which are equivalent to $Z$ relative to $M_n$ provided $\lambda$
 is sufficiently large; in other words  
 $$
 \| CZ + D\| \ge 1
 $$
 
 As in (\ref{eq71}) we have , in the usual  notations in (\ref{eq70}), 
 $$
 \| CZ+D\|^2= |C_1|^2 |T|^2 \prod\limits^r_{\nu=1} (1+h^2_\nu)
 $$
 and to infer that $\| CZ +D \|^2 \ge 1$  it suffices to ensure
 ourselves that $\| T \| \ge 1$.   

As usual,\pageoriginale we assume that $T$ is reduced. Then by
(\ref{eq49}), $C_1|T|> 
\prod\limits^r_{\nu=1} y [y_\nu]$, $Q = (y_1, y_2, \ldots y_r) =
(Q^\ast y')$
(say) where $Y' = (q_1, q_2, \ldots q_r)$ is the last row of $Q$ and
let $Q^\ast = (y^\ast_1, y^\ast_2, \ldots y^\ast_r)$ where 
$y_\nu = (y^\ast_\nu, q_\nu) $. Then $Y [y_\nu] = Y_1 [y^\ast_\nu] +
\lambda q^\ast_r$ and by (\ref{eq83}), since the smallest
characteristic root $\lambda$ 
of $\lambda_1$ is at most equal to $\dfrac{\sqrt{3}}{2(n-1) C_1} = C$,
we have $y_1 = y^\ast_\nu | \geq C y^\ast_\nu y^\ast_\nu$ with a 
positive constant $C = 1$ which depends only on $u$. Hence choosing
$\lambda \geq C$, we have  
$$
y [\mathscr{U}] = y_1 [\mathscr{U}*_\nu] + \lambda q^2_n \geq
C (y^\ast_\nu, y^\ast_\nu + y^2_\nu) = c y'_\nu y_\nu \geq C
$$ 
since $\mathscr{U}'_\nu \mathscr{U}_\nu \geq$ being a primitive vector.

Thus 
\begin{equation*}
y[\eta_{\nu}] \geq
\begin{cases}
c, \text{ in any case }\\
\lambda, \text{ if } q_\nu = C 
\end{cases}
\end{equation*}

Hence if $xxxx \neq o$ at least one $q_{\nu} \neq *****$ so that one
of the factors of the product ***** is greater than or equal to
$\lambda$ while the others are at least equal to $C$. Consequently
***** and a fortiri ***** if $\mathscr{U} \neq 0$. This means that $|
T | \geq xxxxxxx$ choosing $\lambda$ with $\lambda \geq \max *****$
and this in its turn implies that $|| CZ + 0 ||\geq 1$ as desired. If
however ***** then ***** is itself primitive and *****. The pair $\{
C_1, C_2 \}$ and ***** then determine a class ***** of coprime
symmetric pairs of order $n-1$. A simple consideration shows that
$$
||C_o Z_1 + D_o ||^2 = | C_o |^2 | T |^2 \prod\limits_{\nu = 1}^{r}
(1 + h^2_\nu) = ||CZ+D||^2.
$$\pageoriginale
 Since $Z_i \in  f_{n-1}, || C_o Z_1 + D_2|| \leq
1$- the same is therefore true of $|| CZ + D ||$ too. Thus if  $
\lambda \geq \max (y_{n-1, n_1}, C, C_1C^{1-n}$ and $Z_i \in $, then  
\begin{equation*}
Z=
\begin{pmatrix}
Z_i & 0 \\
0 & \lambda
\end{pmatrix}
\in f_n. \tag{105}\label{eq105}
\end{equation*}

Let $Z_1 \in \mathcal{F}_{n-1}$ be chosen such that $\sigma
(\gamma^{-1}_1) \geq s_{n-1}^{- \varepsilon}$ where $Z_1 = X_1 + i y_1
$ and $\varepsilon$ is a given positive number. 

Then, if $Z = \times + i y$ is determined by (\ref{eq105}) we have $y
= \begin{pmatrix} 
Z_i & 0 \\
0 & \lambda
\end{pmatrix} $ and $\sigma (y^{-1}) = \sigma^{-1}_1) = 1_{/
  \lambda}$. 

Now $\mathcal{S}_n \geq \sigma(y^{-1}) = \sigma(y^{-1}) +1_{/ \lambda}
\geq \mathcal{S}_{n-1}+ 1_{/ \lambda} - \varepsilon$. Since this is
true for all $\lambda \geq \max *****  $ letting $\lambda \to \infty$
we obtain that $s_n \geq s_{n-1} - \varepsilon$  

The arbitrariness of $\in$ now implies that $z_n \geq s_{n-1}$, in
other words, the sequence ***** is monotone increasing. 

We shall need one more fact for proving our theorem, viz. that if $T
= T^{(n)}$ is a semi positive matrix, and $0 < $ rank $ T = n < \eta$
then there exists a positive matrix $T_i = T^{(n)} = T_1$ and a
unimodular matrix $V$ such that  
\begin{equation*}
T[V] =
\begin{pmatrix}
T_1^{(n)} & 0 \\
0 & 0
\end{pmatrix}
T_1^{(r)}> 0. \tag{106}\label{eq106}
\end{equation*}

Indeed, due to our assumption we have $|T| = 0$ so that there exists
a rational column $\mathcal{W}$ with $T[\mathcal{W}]$. Clearly we can
assume $\mathcal{W}$ to be primitive. Then $\mathcal{W}$ can be
completed to a unimodular matrix\pageoriginale $V_o$ with
$\mathcal{W}$ as the last 
column and then $T[V_o] =  ( t^{(o)}_{\mu \nu})$ is a matrix all the
elements of whose last row and column vanish; in fact, $t^{(o)}_{ n n
} = 0$ since $T[\mathcal{W}] = 0$ and then $t^{(o)}_{n \nu} = 0 =
(t^{(o)}_{\nu n} = 1, 2 , \ldots n $ as $T \geq 0$. Let then  
$$ 
T[V]=
\begin{pmatrix}
T_0  & 0 \\
0 & 0
\end{pmatrix}
T_o = T'_o = t_o^{(n - 1)} \geq 0
$$

If $r = n - 1$, then clearly $T_o > 0$ and we are through. In the
alternative case we can repeat the above with $T_0$ in the place of
$T$ and this can be continued until we arrive at $T_1 =T^{(r)}_1$
satisfying (\ref{eq106}). 

We now take up the proof of the  main theorem. The proof is by
induction on $n$. We assume that either $n = 1$ or if $n > 1$ then the
theorem is true for all modular forms of degree $n-1$. With this
assumption on $n$ we shall prove that the theorem is true for
modular forms of degree $n$. 

Let
\begin{equation*}
f (Z) = \sum_{T \geq 0} a (T)  e^{2 \pi i \sigma (TZ)} \tag{107}\label{eq107}
\end{equation*}

In the second case, viz. when $n > 1$ we have 
$$
f (Z)| \phi = \sum_{T_1\geq 0} a(T_1) e^{2 \pi i \sigma (TZ)} 
$$
where 
$$ 
a (T_1) = a  
\begin{pmatrix} 
T_1  & 0 \\
0 & 0
\end{pmatrix}
$$
 
If $\sigma (T_i) < \dfrac{\mathfrak{K}} { 4 \pi } s_{n-i}$ then 
$\sigma \begin{pmatrix}
T_1 & 0\\
0 & 0 
\end{pmatrix} =  \sigma (T_1) < k_\nu S_{n-1}$
and our\pageoriginale hypothesis now implies that $a(T_1) = a   
\begin{pmatrix}
T_1  & 0 \\
0 & C
\end{pmatrix}
= 0$. In other words, $f (z)| \phi$ satisfies the conditions of
Theorem \ref{chap6:thm6} so that, due to our assumption $f (z) | \phi
\equiv 0$ as it 
is a modular form of degree $n-1$. This means that $a(T) = 0$ for $T$
such that $|T| = 0$ For, by (\ref{eq106}), $T = \begin{pmatrix} 
T_1  & 0 \\
0 & C
\end{pmatrix} [u]$, $\mathcal{U}$-unimodular, and by (\ref{eq95}), since
$T_1 \geq 0$, 
\begin{equation*}
\pm a (T) = a
\begin{pmatrix}
T_1  & 0 \\
0 & C
\end{pmatrix} = 
a(T_1) = 0 \tag{108}\label{eq108}
\end{equation*}

For $n =1$, the above is true by one of the assumptions in Theorem
\ref{chap6:thm6}. Thus in any case $|T| = 0$ implies $a (T) =0$ so
that in (\ref{eq107}) only those $T's$ with $T > 0$ survive; in other
words   
$$
f (Z) = \sum_{T > 0} a (T) e^{2 \pi i \sigma (TZ)}
$$

We now wish to prove a preliminary result, viz. if $Z =x, i y \in
\mathscr{F}_n$ then  
\begin{equation*}
\lim_{| y | \to \infty} | y |^{\mathfrak{K}/2} \int (Z) = 0
\tag{109}\label{eq109} 
\end{equation*}

Let $y = (y_{\mu \nu)}, T= (t_{\mu \nu}) > 0$ and introduce $y_1$, $T_1$
by requiring that $y = y_1 [K]$, $T_1 = T[K] = [K]$ where $K=
(\delta_{\mu \nu } \sqrt{y_{\mu \nu}})$. Then $\mathfrak{f} = y
[K^{-1}]$ is a matrix of the 
type $\begin{pmatrix} 1_1 * & \\ * & 1 \end{pmatrix}$ and is
positive. Hence $y_1$ is a bounded matrix with $|y_2|= |y|(y_{11},
y_{22},y_{nn})^1 \geq C^{-1}_1 > 0$ as a result of (\ref{eq49}). These show
that $y_1$ belongs to a compact subset $y$ of the space of all
positive symmetric matrices, which set depends only upon $n$. 

According to (\ref{eq102}) we then have 
\begin{align*}
\sigma (Ty) & = \sigma (T_1[k^{-1}] y_1[K]) = \sigma (T_1 y_1)\\
& \geq \gamma \sigma (T_1) = \gamma \sum^{\gamma}_{\nu = 1} y_{ \nu
  \nu } t_{\nu \nu} \tag{110}\label{eq110} 
\end{align*}
with\pageoriginale a positive constant $\gamma$ depending only on
$n$. Since the 
series (\ref{eq107}) converges everywhere and in particular at the point $z
= \frac{\sqrt{3}}{2} \frac {\gamma}{2} i E$ we get that $a (T)
e^{-\frac{\sqrt{3}}{2} \pi \gamma \sigma (T)}$ is bounded, and by
multiplying $\mathfrak{f}(Z)$ by a constant factor if necessary, we can
assume the bound to be 1. We may now estimate the general term of
(\ref{eq107}) as follows : 
\begin{align*}  
| a (T) e^{2 \pi i \sigma ( TZ)} | & \leq e^{\frac{\sqrt{3}}{2} \pi
  \gamma \sigma (T) - 2 \pi  \sigma ( TY)} \\ 
& =  e^{-\frac{\sqrt{3}}{2}} (t_{11} + t_{22} + \cdots t_{nn} - 2 \pi
\sigma (TY)\\ 
& \leq e^{\pi \gamma (\sum^n_{\nu =1} t_{\nu \nu }) -2 \pi \gamma
  (\sum^n_{\nu =1} t_{\nu \mu } y_{\nu \mu}})\\ 
& e^{-\pi \gamma (\sum^n_{\nu =1} t_{\nu \nu } y_{\nu \nu })}. 
\end{align*}

In the above we have made use of (\ref{eq75}) and (\ref{eq110}).

Since $\sum\limits^n_{\nu =1} t_{\nu \nu} y_{\nu \nu } \geq n
\sqrt[n]{\prod^n_{\nu =1}t_{\nu \mu } y_{\nu \nu }}$ and $|y| \leq
\prod\limits^{n}_{\nu =1} y_{\nu \nu}$ by (\ref{eq81}), $Y$ being positive,
we have from the above  
\begin{align*}
| a (T) e^{2 \pi i \sigma (TZ)} | \leq & e^{- \pi \gamma n \sqrt
  [n]{n \pi^n_{\nu =1}t_{\nu \nu } y_{\nu \nu }}}\\ 
& \leq e^{- \pi \gamma n } \sqrt[n]{|Y|} \sqrt[n ]{t_{11}
  t_{22}, \ldots t_{nn}} 
\end{align*}

Since $\mathfrak{K} \geq 0$ for a suitable constant $\mathscr{C} $ we
have  
$$
|Y|^{\frac{\mathfrak{K}}{2}} e^{ \pi n \frac{\gamma}{2}
  \sqrt[n]{|Y|}}< \mathscr{C} 
$$
and then $|Y|^{\frac{\mathfrak{K}}{2}} \mathfrak{f} (Z) = \sum
\limits_{ T > 0} | Y 
|^{\frac{\mathfrak{K}}{2}} a (T) e^{2 \pi i \sigma (TZ)}$ is majorised
by the series \break
$\mathscr{C} \sum\limits_{ T > 0} e^{\dfrac{- \pi \gamma
    n}{2} \sqrt[n]{| \gamma |} \sqrt[n]{t_{11} t_{22} \ldots
    t_{nn}}}$ 

The\pageoriginale last series converges by arguments as in pages
(\ref{eq66} - \ref{eq67}) since  
the number of semi integral positive matrices $T=(t_{\mu\nu})$ with $t_{11}
t_{22} \ldots t_{nn} = t$ for a given $t$, can be estimated by
$\mathscr{C}t^{\frac{n(n + 1)}{2}}$, viz. a fixed power of $t$ and then, the
last series is further majorised by the convergent series  
$$
\mathscr{C} \sum^{\infty}_{t=1} \mathscr{C}_1 t^{n(n + 1)/2} e^{-T
  \gamma /2 *****} 
$$

It now turns out that for $Z \in \mathcal{F}_n$
\begin{equation*}
|Y|^{\mathfrak{K}/2} \mathcal{F}(x) = 0 (e^{-e*****} \tag{111}\label{eq111} 
\end{equation*}
with some $\in > 0$ as $| y | \to \infty$ and this, in particular,
implies (\ref{eq109}). If $h(Z) = | y |^{\mathfrak{K}/2}| \mathcal{F}(Z)|$,
Theorem (\ref{eq4}) and (\ref{eq109}) imply that $h(Z)$ is bounded in the
fundamental domain. But we know from (\ref{eq101}) that $h (Z)$ is invariant
under modular substitutions and the $h (Z)$ is bounded throughout its
domain. Since further $h (Z)$ is continuous in $\mathscr{Y}$ it
attains its maximum; in other words there exists a point $Z_{\nu} \in
$ such that $h(Z) \leq h(Z_o)$ for $z \in \mathscr{Y}$. Consider
$h(Z)$ in a neighbourhood of $Z_o$. Let $z = x - l_y$ be a complex
variable $Z = Z_o - z E$ and $t = e^{z \pi i z}$. 

Let $g(t) = \mathcal{F}(z) e^{- \lambda \sigma (Z)}$ with $\lambda$
determined by  
$$
\frac{\eta \lambda}{2 \pi} = 1 + [k/_{4 \pi} s_n] \text{ where } [x] 
$$
denotes the integral part of $x$, viz. the largest integer not exceeding
$x$. Using our assumption that $a(T)  = 0$ if $\sigma (T)  <
\frac{\mathfrak{K}}{4 \pi} s_n$ we have  
\begin{align*}
g(t) & = \sum_{\sigma (T) > \frac{\mathfrak{K}}{4 \pi} s_n} a (T) e^{2
  \pi i \sigma (TZ) - i \lambda \sigma (z)}\\ 
& =\sum_{\sigma (T)  \frac{\mathfrak{K}}{4 \pi} \mathcal{S}_n} a (T)
e^{2 \pi i \sigma (Tz_o)}t^{\sigma (T)}e^{- i \lambda \sigma (z)}\\ 
& =\sum_{\sigma (T)  \frac{\mathfrak{K}}{4 \pi} \mathcal{S}_n} a (T)
e^{2 \pi i \sigma (Tz_o) - i \lambda \sigma(z_o)} t^{\sigma(T) -
  \frac{\lambda n}{2 \pi}} 
\end{align*}\pageoriginale

Here the exponent $\sigma (T) - \dfrac{\lambda n}{2 \pi} >
\dfrac{\mathfrak{K}}{4 \pi} \mathcal{S}_n -[\dfrac{\mathfrak{K}}{4 \pi}
  \mathcal{S}_n]-1 \geq -1$ which means, as $\sigma(T) -
\dfrac{\lambda n}{2 \pi}$ is an integer, that $\sigma(T) -
\dfrac{\lambda n}{2 \pi} \geq 0$. 

This shows that the function $g(t)$ is regular in a circle $| t | \leq
\rho $ and we can assume $\rho >1$ by choosing $y < 0$ with its
absolute value sufficiently small. 

By the maximum principle, there exists then a point $t_1$ with $|t_1|=
\rho > 1$ such that $|g(t_1) | \geq | g (1) |$. If the $z$-point
corresponding to $t_1$ is denoted by $z_1$, we have  
$$
| g (t) | = | \mathfrak{f}(Z) | e^{\lambda \sigma (y)} = h (z) | y
|^{\frac{\mathfrak{K}}{2}} e^{\lambda \sigma (y)} 
$$
and putting $t=1$ the above now implies that 
$$
h (z_o) | Y_o |^{\frac{\mathfrak{K}}{2}}e^{\lambda \sigma (y_o)} \leq
h(z_1) |Y_o|^{\frac{\mathfrak{K}}{2}}e^{\lambda \sigma (y_1)} 
$$
where $Z_1 = Z_o + z_1 E , y_1 = y_o + yE, \int^{-2 \pi y} = | t_1| =
\rho$. 
$$
y = - \frac{1}{2 \pi} \log \rho 
$$

Let $h (z_o) = \sup\limits_{z \in y} h(z) = M$ (say). Then the
above yields  
\begin{equation*}
M \leq M | y |^{\frac{\mathfrak{K}}{2}} |y_0 |^{\frac{\mathfrak{K}}{2}}
e^{\sigma (y_1 - y_o)} = M e^{\psi (y)} \tag{112}\label{eq112}  
\end{equation*}
where\pageoriginale 
\begin{align*}
\psi (y) & = \lambda n y - \frac{\mathfrak{K}}{2} \log (| y_1|
|y_o|^{-1}) \\ 
& =\psi (y) & \lambda n y - \frac{\mathfrak{K}}{2} \log |E |y
\gamma_o^{-1} 
\end{align*}

Now $\psi (0) = 0$ and $\psi' (o) = \lambda n \frac{\mathfrak{K}}{2}
\sigma (\gamma^{-1}_o)$ 
\begin{align*}
& \geq \lambda n - \frac{\mathfrak{K}}{2} s_n\\
&= 2 \pi (\frac{\lambda n}{2 \pi } - \frac{\mathfrak{K}}{4 \pi} s_n) > 0.
\end{align*}

Hence it follows that $\psi (y)$ is monotone increasing in a
neighbourhood of $y = 0$ so that it is negative for sufficiently
small $y < 0$. But $\psi (y) < 0$ implies from (\ref{eq112}) that $M = 0$
which in its turn means that $h (z)$ and consequently $f (z)$ vanishes
identically. The proof of theorem \ref{chap6:thm6} is now complete. 

The above theorem has several interesting consequences. In the first
instance it implies that \textit{all modular forms of weight $0
  \mathfrak{K} = 0)$ are necessarily constants}. For if $\mathcal{F}(z)
= a (o) + \sum\limits_{T \neq 0} a (T) e^{2 \pi i \sigma(TZ)}$ be such a
form, then $\mathcal{F}(Z) - a (0)$ is a form satisfying the
conditions of theorem \ref{chap6:thm6}. Hence it follows from theorem
\ref{chap6:thm6} that 
$\mathcal{F}(Z) - a (o) \equiv 0$, in there words $\mathcal{F} (Z) \equiv a
(o)$. We therefore assume in the sequel that $k > 0$. 
 
Now consider modular forms $\mathcal{F} (Z)$ of degree $1(n = 1)$ and
weight $\mathfrak{K} \leq \delta$. Then from (\ref{eq75}) we deduce that
$\mathscr{S}_1 \dfrac{2}{\sqrt{3}}$. Then $\sigma(T) \leq
\dfrac{\mathfrak{K}}{4 \pi}\mathscr{S}_1$ implies that $\sigma(T) \leq
\dfrac{2}{\pi} \dfrac{2}{\sqrt{3}} =\dfrac{4}{\pi \sqrt{3}}<1$ which
in its turn means that $\sigma (T) = 0$ as $sigma (T) $ is an integer,
and consequently $T =0$. Hence if  
$$
\mathfrak{f} (z) = a (0) + \sum_{T \neq 0} a (T) e^{2 \pi i \delta
  (TZ)} 
$$
and\pageoriginale if $a (0) = 0$, then the conditions of theorem
\ref{chap6:thm6} are satisfied 
so that by its conclusion, $\mathfrak{f} (z) \equiv 0$. In other words,
modular forms of degree 1 and weight $\mathfrak{K} \leq 8$ are
uniquely determined by their `first' Fourier coefficient $a(0)$. It is
now immediate that any two modular forms of degree 1 and weight
$\mathfrak{K} \leq 8$ are proportional and what is the same, a modular
form of degree 1 and weight $\mathfrak{K} \leq 8$ is unique, except
for multiplication by a constant. Indeed, the same result is true of
modular forms of degree 2 too. For if $Z = x + i y \in
\mathfrak{f}_2$ and $f(Z)$ is a modular form of degree 2 and weight
$\mathfrak{K} \leq 8$ , then writing $Y = (y_{\mu \nu})$ we have, by
Minkowski's reduction conditions, 
$$
0 \leq  2y_{12} \leq y_{y11} \leq y_{22}
$$
and then $|Y| = y_{11} y_{22 } - y^2_{12} \geq \dfrac{3}{4} y_{11} y_{22}$.

Since $y_{11} \geq \dfrac{\sqrt{3}}{2}$ by (\ref{eq75}), this means that  
$$
\sigma(Y^{-1}) = \frac{y_{11} + y_{22}}{|Y|} \leq
\frac{4}{3}(\frac{1}{y_{11}} + \frac{1}{y_{22}}) \leq \frac{8}{3}
\frac{1}{y_{11}} \leq \frac{16}{3 \sqrt{3}} 
$$
so that $s_2 \leq \dfrac{16}{3 \sqrt{3}}$. Also  
$$
\frac{\mathfrak{K}}{4 \pi} s_2 \leq \frac{2}{\pi} \leq \frac{32}{3
  \sqrt{3}} < 2 \text{ and then }, \sigma (T) < \frac{\mathfrak{K}}{4
  \pi} s_2 
$$
implies that $\sigma (T) < 2$ This means that at least one of the two
diagonal elements and consequently one of the elements $t_{12},
t_{21}$ where $T= (t_{\mu \nu})$ also vanish. Then it is clear that
$|T|=o$. Then assumption $a (o) =o$ will imply by our earlier result
on modular forms of degree 1 that $\mathfrak{f}(Z)| \phi \equiv 0$
and then as in (\ref{eq106}) and (\ref{eq108}), we have $a(T) =0$ for
$|T| =0$. In 
other words if we assume $a(0) =0$\pageoriginale then the conditions
of theorem \ref{chap6:thm6}  
are satisfied and we are able to conclude that $\mathfrak{f}(Z)
\equiv 0$. It now follows as in the earlier case that \textit{every
  modular form of degree 2 and weight $R \leq 8$ is uniquely determined
  by the first Fourier coefficient $a(0)$}. 

The above results are for the moment hypothetical. In other words,
their significance is based on the assumption that modular forms of
the desired degree, not vanishing identically actually exist. Their
existence we prove later, by constructing the so called \textit{
  Eisenstein Series}. 

An interesting application of theorem \ref{chap6:thm6} is to prove the
algebraic dependence of any set of sufficiently large number of
modular forms. Specifically we state  


\begin{thm}\label{chap6:thm7} %them 7
 Let $h = \frac{n(n+1)}{2}+2$ and let $\mathfrak{f}_{\nu}(Z)$ be
  a modular form degree $n$ and weight $\mathfrak{K}_{\nu} > 0, \nu =
  1,2, \ldots h$. Then  there exists an isobaric algebraic relation 
\begin{equation*}
\sum C_{\nu_1 \nu_2} , \ldots \nu_2 \mathfrak{f}^{1} \mathfrak{f}^{\nu
  2}_Z.\mathfrak{f}^{\nu h}_{h } = 0 \tag{113}\label{eq113}      
\end{equation*}
not all of whose coefficients vanish, the summation extending
   over all integers $\nu_{i} \geq o$ with the property 
\begin{equation*}
\sum^h_{l =1} \nu_{\varepsilon} \mathfrak{K}_i =
m\mathfrak{K}_i\mathfrak{K}_2,\mathfrak{K}_h \tag{114}\label{eq114}  
\end{equation*}
where $m$ is an integer which depends only upon $n$
\end{thm}

We may remark that the product of two modular forms of weight $k$ and
$\ell$ is a modular form of weight $\mathfrak{K} K$, their degrees being the
same . Consequently all the power products $\mathfrak{f}^{\nu_1}_1
\mathfrak{f}^{\mu_2}_2 \ldots , \mathfrak{f}^{\mu_h}_h$ that
occur in (\ref{eq113}) are modular forms of the same weight in view of
(\ref{eq114}) and hence the name \textit{isobaric} for the relation
(\ref{eq113}). 


\begin{proof}
\setcounter{pageoriginal}{85}
All\pageoriginale modular forms of degree $n$ and weight $k$ form a linear space
$m^{n}_{\mathfrak{K}}$. By means of theorem \ref{chap6:thm6} we infer that this
space is of finite dimension $\alpha_{n}(\mathfrak{K})$. In fact
$\alpha_{n}(\mathfrak{K})$. is at most equal to the number of solutions
of the relation $\sigma(T) \leq \frac{(\mathfrak{K})}{4 \pi}
\mathscr{S}_n$ with semi positive integral T's, a rough upper
  estimation of which is provided by $\frac{\mathfrak{K})}{4 \pi}
  \mathscr{S}_n^{*****}$ by arguments as at the end of page (\ref{eq66}); in
  other words we have  
\begin{equation*}
d_{\eta}(\mathfrak{K}) (\frac{\mathfrak{K})}{4 \pi} \mathscr{S}_n
+i)^{n(n+1)/2} \tag{115}\label{eq115}  
\end{equation*}

We now find a lower estimation of the number of possible power
products $F =\mathfrak{f}^{\nu_1}_1 \mathfrak{f}^{\nu_2}_2, \ldots
\mathfrak{f}^{\nu_h}_h$ subject to the conditions  
\end{proof}
\begin{equation*}
\left.
\tag{116}\label{eq116} 
\begin{aligned}
\mu_i \geq C , i = 1.2 \ldots 2 \text{ and integral},\\
\sum^h_{i=1} \nu_i \mathfrak{K}_i = \eta \mathfrak{K}_i \mathfrak{K}_2 ,
\ldots \mathfrak{K}_h m=m(n) 
\end{aligned}
\right \}
\end{equation*}

Let us assume for the moment that $m$ is divisible by $2h -2$.

Let $(x_1, x_2, *****)$ be a system of integers with 
\begin{equation*}
o \leq x_{\nu} \leq ***** \tag{117}\label{eq117} 
\end{equation*}
where $K = \mathfrak{K}_1 \mathfrak{K}_2 \cdots \mathfrak{K}_h$. The
number of such systems is clearly $H*****$ 

There exists then one coset modulo $\mathfrak{K}_1$ which contains the
sums $\mathfrak{K}_1 x_1 \mathfrak{K}_2 x_2 + \ldots \mathfrak{K}_{r-1}
x_{h-1} $(mod)$\mathfrak{K}_n$ for at least
$\frac{H-1}{\mathfrak{K}_h}+1$ different systems *** as otherwise,
the total number of different systems $(x_1, x_2, \ldots x_{h-1})$
could\pageoriginale be at most 
$H-1_{/\mathfrak{K}_h}$, $\mathfrak{K}_h= H-1$ while actually there are $H$
such systems. Consider then the systems $(x_1,x_2, \ldots x_{h-1})$
corresponding to this coset. Let $(\xi_1, \xi_2, \ldots \xi_{h-1})$
denote a fixed system among these and let $(\eta_1, \eta_2 , \ldots
\eta_{h-1})$ a variable system. Then clearly we have  
\begin{equation*}
\sum^{h-1}_{i=1} k_i\eta_i - \sum^{h-1}_{i=1}\mathfrak{K}_i  \xi_i \equiv o
\text{ mod }\mathfrak{K}_h \tag{118}\label{eq118}  
\end{equation*}

We shall further assume $o \leq  \xi_i  < \mathfrak{K}_i, l= 1,3 ,\ldots
h-1$ 

We introduce now $\nu_i = \eta_i \mathfrak{K}_h  - \xi_i, = 1,2, \ldots
h-1$ and $\nu_h = \frac{m-k}{\eta_n} - \frac{1}{h_k}(\sum^{h-1}_{i=1}
\mathfrak{K}_i \nu_i$  

We shall verify that the system $h\{\nu \}$ satisfies the
conditions (\ref{eq116}). Clearly $\nu_1, \nu_2, \ldots \nu_{h-1}$ are
non negative and integral. Also using (\ref{eq117}) we have   
\begin{align*}
\nu_h & = \frac{m K}{\mathfrak{K}_h} - \frac{1}{\mathfrak{K}_h}
\sum^{h-1}_{i=1} (\mathfrak{K}_i \eta_i + \mathfrak{K}_i(\mathfrak{K}_h -
\xi_i )) \tag{119}\label{eq119} \\ 
& \leq \frac{m K}{\mathfrak{K}_h} - \frac{1}{\mathfrak{K}_h}
\sum^{h-1}_{i=1} - \frac{m K}{2 h-2}-\sum^{h-1}_{i=1}
-\mathfrak{K}_i\\ 
& =m K_{/2 \mathfrak{K}_h}- \sum^{h-1}_{i=1} -\mathfrak{K}_i \\ 
& = \frac{(h-1)K}{\mathfrak{K}_h} -\sum^{h-1}_{i=1} -\mathfrak{K}_i \geq 0
\end{align*}


\setcounter{pageoriginal}{86}
The last but one of these relations is a consequence of the fact that
$m$ is divisible by $2 h -2$ and a fortiori $\frac{m}{2} \geq h-1$
while the last step is immediate by observing that $\frac{K}{k_h}\geq
\mathfrak{K}_i 0 = 1,2, \ldots h-1. \nu_{h} \nu_h$\pageoriginale is certainly
integral as seen from (\ref{eq119}) by means of (\ref{eq118}). We have
therefore 
shown that each system $(\eta_1 , \eta_2 , \ldots \eta_{h-1})$ leads
to a permissible system of exponents $\nu_1 , \nu_2 , \ldots
\nu_{h}$. Consequently,the number of possible power products
$\mathfrak{f}^{\nu 1}_1 \mathfrak{f}^{\nu 2}_2 \cdots \mathfrak{f}^{\nu
  h}_h$ is at least as great as the number of the systems  $(\eta_1 ,
\eta_2 , \ldots \eta_{h-1})$ which is at least
$\frac{H-1}{\mathfrak{K}_h}+1$. Denoting $\frac{H-1}{\mathfrak{K}_h}$ by
$q$ we have proved that there exist at least $q+1$ modular forms of
the kind $\mathfrak{f}^{\nu 1}_1 \mathfrak{f}^{\nu 2}_2 \cdots
\mathfrak{f}^{\nu h}_h$  

Therefore, in case 
\begin{equation*}
q \geq d_n (mK) \tag{120}\label{eq120} 
\end{equation*}
there exists a non trivial relation
$$
\sum C_{nu 1 \nu_2 }\cdots \nu_{h} \mathfrak{f}^{\nu 1}_1 \cdots
\mathfrak{f}^{\nu h}_h = 0 
$$
with constant coefficients $C_{\nu}$.

As a result of (\ref{eq115}), (\ref{eq120}) will be satisfied provided
we have $ q 
\geq ( \frac{m K}{\pi} s_n +1)^{2-2}$ which in turn will be true if  
\begin{equation*}
q \geq m^{h-8}( \frac{K}{\pi}\mathscr{S}_n +1)^{h-2} \tag{121}\label{eq121} 
\end{equation*}

Since $H = \prod \limits^h_{\nu =1}( 1+ \frac{mK}{(2,p, -2)
  \mathfrak{K}_{\nu}})$ we have  
\begin{align*}
q = \frac{H-1}{\mathfrak{K}_h} & \geq \prod \limits^h_{\nu =1}( 1+
\frac{mK}{(2,p, -2) \mathfrak{K}_{\nu}})\\ 
& = \frac{m^{h-1} K^{h-2}}{(2 h - 2 )^{h-1}}
\end{align*}

Hence (\ref{eq121}) is certainly satisfied if 
\begin{align*}
\frac{m^{h-1} K^{h-z}}{(2 h - 2)^{h-1}} & \ge m^{h-z} (\frac{K}{\pi}
\mathscr{S}_n + 1)^{h-z} \\ 
i.e. \qquad \text{ if } m & \ge (2h - 2)^{h-1}
(\frac{\mathscr{S}n}{\pi} + 1)^{h-2}, 
\end{align*} 
Obviously,\pageoriginale consistent with this requirement, the
assumption that $m$ is divisible by $2 h - 2$ is permissible and
theorem \ref{chap6:thm7} is established.  

 
