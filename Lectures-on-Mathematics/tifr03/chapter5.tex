\chapter{Modular Forms of Degree \texorpdfstring{$n$}{n}} % chap 5

The\pageoriginale possible of developing a modular form into a Fourier
series rests 
upon a general theorem in complex function theory. First of all we
have to deal with the following facts. 

Let $\mathscr{E}$ be the Gaussian plane. A subset $\mathfrak{K}$ of
direct product $\mathscr{E}^n=\mathscr{E} \times \mathscr{E} \times
\cdots \times \mathscr{E}$ ($n$ times) shall be called a
\textit{Reinhardt domain} with zero centre, when the following
conditions are satisfied. 
\begin{enumerate}[1)]
\item $\mathcal{R}$ is a domain in the sense of function theory,
  viz. an open   connected non empty set.  

\item $\mathcal{R}$ is invariant under the group of transformations
  $(Z_1,Z_2   \cdots Z_n) \to (Z_1 e^{i \varphi_1}, Z_2 e^{i
  \varphi_2}, \ldots   Z_1e^{i \varphi _n}) \varphi _ \nu$- real. We
  now stare  
\end{enumerate}

\begin{lem}\label{chap5:lem9} % lem 9
Let $R$ be a Reinhardt domain with \, $(o,o \cdots o)$ \, and \break
  $f(Z_1,Z_2 ,\ldots ,Z_n)^a$ functions regular in $\mathcal{R}$. Then we can
  develop $f$ into a power series 

$f(Z_1,Z_2 ,\ldots Z_n)= \sum ^\infty _{\nu _1,\nu _2, \cdot ,\nu
  _n=-\infty}o_{\nu _1 \nu _2 \cdots \nu _n}.0 Z_n^{\nu _1} Z_2^{\nu
  _2}\cdots Z_n^{\nu _n}$ 

This representation is valid in the whole domain $\mathcal{R}$ and the
  power series converges uniformly on the manifold $|Z_\nu|=r_\nu, \nu
  =1,2, \ldots n$ when $(r_1,r_2, \ldots ,r_n) \in \mathcal{R}$. 
\end{lem}

\begin{proof}
We first prove the Lemma in the special case of the domain
\begin{gather*}
\mathcal{R}:^{|Z_\nu|< \rho_\nu,1 \le \nu \le h}\\
{o<\sigma_\nu <|Z_\nu|< \rho_\nu, h< \nu \le n}.
\end{gather*}

Consider\pageoriginale the contracted domain
\begin{gather*}
|Z_\nu|< \rho'_\nu < \rho_\nu \;  (1 \le \nu \le h)\\
o <\sigma _\nu <\sigma' _\nu < |Z_\nu| < \rho'_\nu < \rho_\nu (h < \nu
\le n). 
\end{gather*}

Let $\mathcal{L}^1_\nu$ be the circle $|Z_\nu|< \sigma'_\nu,h < \nu
\le n$ assigned with the negative sense of rotation and let
$\mathcal{L}^1_\nu$ be the circle $|Z_\nu |=  \rho' _\nu (1 \le\nu \leq
n)$ assigned with the positive sense of rotation. Then we have by
mean of the Cauchy integral formula, $f(Z_1,Z_2, \ldots Z_n)$= 
\begin{align*}
&= (\frac{1}{2 \pi i})^n \int \limits_{\mathcal{L}^2_1} \cdots \int
  \limits_{\mathcal{L}^2_h}\int \limits_{\mathcal{L}^2_{h+1}} \cdots
  \int \limits_{\mathcal{L}^2_{n +\mathcal{L}^2_n}} \frac{f(\zeta _1,
    \ldots \zeta _1)}{(\zeta _1-Z_1)\cdot (\zeta _1-Z_n)} ds_1 \cdot
  d\\ 
&=\sum ^2_{\nu_{h+1}, \ldots \nu_n=1} ((\frac{1}{2 \pi i})^n )\int
  \limits_{\mathcal{L}^2_1} \int \limits_{\mathcal{L}^2_h} \int
  \limits ^{\nu_{h+1}}_{h+1} \cdot \int \limits _n ^{\nu
    ,h}\frac{f(\zeta _1, \ldots \zeta _1)}{(\zeta _1-Z_1)\cdot (\zeta
    _1-Z_n)} ds_1 \cdot d\zeta 
\end{align*}

By expanding the quotient $1/\rho_\nu - Z_\nu$ as a convergent power
series as in the one variable case of each $\nu$, a simple argument
shows that every integral in the above sum is expressible as a power
series, a typical term of which contains non negative powers of  

$Z_1,Z_2 \cdots Z_h, Z^{\pm 1}_{h+1},Z^{\pm 1}_{h+2}, \ldots Z^{\pm
  1}_{n}$ where in $ Z^{\pm 1}_{r}(r \ge h+ 1)$ the upper sign holds
in cases where $\nu_r = 2$ and the lower sign holds in cases where
$\nu_n=1$. Clearly, the coefficients of these power series do 
not depend upon the choice of $\rho^1 _\nu$, $\sigma ^1_ \nu$. The
uniform convergence as stated in the Lemma is immediate from a
consideration of the derived series as in the one variable case. 

In the\pageoriginale general case of any Reinhardt domain with center
$(0,0, \ldots 
0)$ by the result shown above, in every subset $R' \subset R$ of the
type $|Z_{\nu_i}|= \rho_i$, $ 1 \le i \le h$, 
$$
0< \sigma_\nu < |Z_\nu| < \rho_\nu ,\nu \neq \nu _1, \nu_2 \cdots \nu_h, 
$$ 
there exists a representation of $f(Z_1,Z_2, \ldots Z_n)$ as a power
series with the desired properties. We need then only prove that two
representations in two subsets $\mathcal{R'}_1$, $\mathcal{R'}_2$ are
identical in their intersection and then the representation in any two
subsets $\mathcal{R'}_1$, $\mathcal{R}''_1$ are identical as we can always
find a sequence of such subsets, viz. 
$$
\mathcal{R'}_1=R_1,R_2, \ldots R_m=\mathcal{R}'' 
$$
with $\mathcal{R}_i \subset \mathcal{R}_{i+1}$ non empty for any two
indices $i$, $i+1$. But 
for any two such subsets $\mathcal{R}'_i , \mathcal{R}'_2$ their
intersection 
$\mathcal{R'}_2 \cap \mathcal{R'}_2$ is again a subset of the same
type and it is then immediate that the power series representing
$f(Z_1,Z_2, \ldots Z_n)$ in the two sets are identical. Lemma
\ref{chap5:lem9} now follows. 

Let $\mathfrak{K}$ be given integer and $M= \begin{pmatrix} A&B
  \\C&D \end{pmatrix}$, a symplectic matrix. We introduce the
notation 
\begin{equation*}
f(Z)|M=f(M<Z>);CZ+D_1^{\mathcal{K}} \tag{92}\label{eq92}
\end{equation*}
where $f(Z)$ is an arbitrary function defined on $\mathscr{Y}$. It is
easy to see that $(f(Z)| M_1)|M_2=f(Z)|(M_1 M_2)$. We now fix the
conception of a modular form by the following. 
\end{proof}

\begin{defi*}
A modular form of degree $n \geq 1$ and weight $\mathfrak{K}$, is a 
function $f(Z)$ satisfying the following conditions.  
\begin{enumerate}[1)]
\item $f(Z)$\pageoriginale is defined in $\mathscr{Y}$ and is a
  regular function of   the $\dfrac{n(n+1)}{2}$ independent elements
  $Z_{\mu \nu}$  $(\mu \le   \nu)$ of $Z$.  

\item $f(Z)|M=f(Z)$ for $M\in M$

\item In this case $n=1,f(Z)$ is bounded in the fundamental domain
  $\mathcal{F}$ on $\mathscr{Y}$ relative to $M$.  
\end{enumerate}
\end{defi*}

We shall show later that the last condition is a consequence of  the
earlier conditions in the case $n> 1$. 

Considering a matrix $M$ of the form $M= \begin{pmatrix} E&S
  \\O&E \end{pmatrix} \in M$ where $\mathcal{S}$ is an integral
symmetric matrix, we infer from condition  (\ref{eq2}) above that
$f(Z+S)=f(Z)$ for an integral symmetric matrix $S$; in other words a
modular form is a periodic function of period $1$ in each of the
variables. Hence $g(\zeta _{\mu \nu})=f (e^{2pi i Z} \mu \nu)$ is a
single valued regular function in the domain into which $\mathscr{Y}$
is mapped by the mapping $\zeta _{\mu \nu}=f (e^{2pi i Z} \mu \nu$
viz. the domain defined by the inequalities $\zeta_{\mu \nu}=
\zeta_{\nu \mu}\neq O$, $(- \ell oc |\zeta_{\mu \nu}|)>O$. It is easy to
see that this is a Reinhard domain and then Lemma \ref{chap5:lem9}
assures us of a 
power series representation for $g(\zeta_{\mu \nu})$ valid throughout
this domain. This power series can be looked upon as a Fourier series
of $f(Z)$ and can be written in the form 
\begin{equation*}
f(Z)= \sum _T a (T) e^{2 \pi i \sigma(TZ)} \tag{93}\label{eq93}
\end{equation*}
where $T=(t_{\mu \nu})$ runs over all rational symmetric matrices such
that $t_{\mu \mu}$ and $2 b_{\mu \nu}(\mu \neq \nu)$ are
integral. Such matrices are called \textit{semi integral}. To verify
the above fact, we need only observe that 
$$ 
\sigma (TZ)=\sum ^n _{\mu ,\rho =1} t_{\mu \rho} Z_{\rho \mu}= \sum ^n
_{\mu  =1} t_{\mu \mu} Z_{\mu \mu} +2 \sum _{\mu <\nu} t_{\mu \nu}
Z_{\mu \nu} 
$$
so that\pageoriginale
\begin{align*}
e^{2 \pi i \sigma (T Z)} & = e^{2 \pi i (\sum^n_{\nu =1} t_{\mu\mu} + 2 \sum_{\mu
    < \nu} t_{\mu \nu}Z_{\mu \nu})}\\ 
&= \prod^n _{\mu=1} \zeta^{t_{\mu \nu}}_{\mu \nu} \prod\limits_{\mu < \nu}
\zeta^{zt_{\mu \nu}}_{\mu \nu} 
\end{align*}

The condition (\ref{eq2}) in the definition of a modular form yields in the
case of the modular substitution
$Z_1=Z[\mathcal{U}],\mathcal{U}$-unimodular the transformation formula 
\begin{equation*}
f(Z_1)= |\mathcal{U}|^{-\mathfrak{K}}f(Z). \tag{94}\label{eq94}
\end{equation*}

Observing that the trace of a matrix product is invariant under a
cyclic change in the succession of the factors and in particular
$\sigma (AB)=\sigma (BA)$ for any two matrices $A$, $B$, we obtain  
\begin{align*}
f(Z_1) & = \sum_T a (T)e^{2 \pi i \sigma (TZ[\mathcal{U}])}\\
&=\sum_T a (T)e^{2 \pi i \sigma (T \mathcal{U'}Z \mathcal{U})}\\
&=\sum_T a (T)e^{2 \pi i \sigma ( \mathcal{U} T  \mathcal{U'} Z)}\\
&=\sum_T a (T[\mathcal{U'}^{-1}]) e^{2\pi i \sigma (TZ)}
\end{align*}

On the other hand,
$$
|\mathcal{U'}|^{- \mathfrak{K}}f(Z)= \sum_T |\mathcal{U'}|^{-
  \mathfrak{K}} \alpha (T) e^{2\pi i \sigma (TZ)} 
$$

A comparison of coefficients by mean of (\ref{eq94}) now yields
$$
|\mathcal{U}|^\mathfrak{K}a (T[\mathcal{U'}^{-1}])= a (T)
$$
where\pageoriginale $\mathcal{U}$ is an arbitrary unimodular matrix. Replacing
$\mathcal{U}$ by $\mathcal{U}'^{-1}$ we get 
\begin{equation*}
a (T[\mathcal{U}])= |\mathcal{U}|^\mathfrak{K} a (T) \tag{95}\label{eq95}
\end{equation*}

The special choice $\mathcal{U}=-E$ leads to 
\begin{equation*}
a(T) \qquad = \qquad (-1)^{n \mathfrak{K}} a (T) \tag{96}\label{eq96}
\end{equation*}

This proves that modular forms not vanishing identically can exist
only in the case $n \mathfrak{K}\equiv 0(2)$. Also we deduce from
(\ref{eq95}) that for \textit{properly unimodular matrices}
i.e. $\mathcal{U}$ such that $|\mathcal{U}|=1$, 
\begin{equation*}
a (T[\mathcal{U}])= a (T) \tag{97}\label{eq97}
\end{equation*}

We apply these results to prove.

\begin{lem}\label{chap5:lem10}%lem 10
A modular form $f(Z)$ which is bounded in the fundamental domain
  $\mathcal{F}$ of $\mathscr{Y}$ relative to $M$ has a representation
  of the form  
\end{lem}
\begin{equation*}
f(Z)= \sum _{T \ge O} a (T)e^{2 \pi i \sigma (TZ)} \tag{98}\label{eq98}
\end{equation*}
and conversely any modular form representable by (\ref{eq98}) and in fact
any series (\ref{eq98}) which converges everywhere in $\mathscr{Y}$ is
bounded in $\mathcal{F}$. 

We prove the direct part first. We shall denote by $\mathcal{H}$ the
cube $- \dfrac{1}{2}\le x_{\mu \nu} \le \dfrac{1}{2} (\mu
\le \nu)$ \qquad and put $[dx]= \prod\limits_{\mu \le \nu} dx_{\mu
  \nu}$ 

Since the Fourier series of $f(Z)$ converge uniformly in
$\mathcal{H}$, we obtain 
$$
a (T)= {\underset{\mathcal{H}} {\int \cdots \int}} f(Z)e^{-2 \pi
  i \sigma (TZ)} [d x] 
$$ 
and\pageoriginale consequently
$$
a(T) e^{-2 \pi  \sigma (Ty)}={\underset{\mathcal{H}} {\int \cdots \int}}f
(Z)e^{-2 \pi i \sigma (Tx)} [dx] 
$$
where $Z= X+iY \in \mathscr{Y}$

Let $y_o$ be a reduced matrix such that $\sigma (Ty_o)<O$ for some
fixed $T$.  By (\ref{eq77}), if $\lambda$ is chosen sufficiently large, the
largeness depending only on $y_o$ then $Z=X+ i \lambda Y_o \in
\mathcal{F}$ for $x \in \mathcal{H}$ and $y$-reduced. With such a
choice of $\lambda$, the assumption $|f(Z)| < C$ for $Z \in
\mathcal{F}$ implies by means of the relation 
$$
a (T)e^{-2 \pi \gamma \sigma (Ty_o)}={\underset{\mathscr{X}} {\int
    \cdots \int}} f(Z)e^{-2 \pi i \sigma (Ty_o)} [dx] 
$$
that
$$
|a (T)|e^{-2 \pi \lambda \sigma (Ty_o)} \le C.\vartheta o \ell
\mathcal{H}=C. 
$$

As $\lambda \to \infty$ our assumption $\sigma (Ty_o)<0$ implies that  
\begin{equation*}
a(T)=0 \tag{99}\label{eq99}
\end{equation*}

Let now $Y$ be an arbitrary positive matrix with $\sigma (Ty)<0$ and
choose a unimodular matrix $\mathcal{U}$ such that $y=y_o
[\mathcal{U'}]$, $y_o \in \mathscr{R}$, the space of reduced
matrices. Then we have 
$$
0> \sigma (T y)= \sigma (Ty_o[\mathcal{U'}])= \sigma
(T[\mathcal{U}]y_o) 
$$
and (\ref{eq99}) and (\ref{eq95}) now imply that 
$$
a (T[\mathcal{U}])= \pm a (T)=0 
$$
for those $T$ for which there exist positive matrices $Y$ with
$\sigma(Ty)<0$. Hence it is sufficient that the matrix $T$ in the Fourier
series (\ref{eq93}) runs only over those matrices with the property $\sigma
(T y)\ge 0$ for all $y >0$. 

Let\pageoriginale $R=R^{(n)}$ be any nonsingular real matrix, $R=
(\mu_1,\mathscr{W}_2, \ldots \mathscr{W}_n)$ and let $y=R  R'$. Then
$Y>0$ and the above implies for the matrices $T$ occurring in (\ref{eq93})
that $\sigma (T y)= \sigma (T[R])= \sum^n_{\nu =1} T [W_\nu]\ge 0$. This is
true of all non-singular real matrices $R$ and $\sigma (T[R])$ is a
continuous function of $R$ so that it is true of singular matrices too
with real elements. In other words $T[\varepsilon]\ge 0$ for all real
column $\mathscr{E}$ and what is the same, $T\ge 0$. This settles the
first part of the Lemma. 

Conversely we have to prove that the sum of a Fourier series (\ref{eq98})
which converges everywhere in $\mathscr{Y}$ is bounded in
$\mathcal{F}$. The matrices $T$ occurring in (\ref{eq98}) are all semi
integral matrices which are further positive. Let
$T=RR',R=(\mathscr{W}_1,\mathscr{W}_2 \cdots \mathscr{W}_n)$-real and
let $Z=X+iY \in \mathcal{F}$. The characteristic roots of $y$ have by
(\ref{eq83}) a positive lower bound $C=
\dfrac{\sqrt{3}}{2nc_1},C_1=C_1(n)$. Hence  
\begin{align*}
\sigma (Ty)= \sigma (y[R])&= \sum ^n_{\nu =1}y [\mathscr{W}_\nu] \ge C
\sum ^n _{\nu =1} \mathscr{W}_\nu '\mathscr{W}_\nu \\ 
&= C \sigma (R'R)= C \sigma (T).
\end{align*}

The convergence of the series (\ref{eq93}) at the particular point $Z=
\dfrac{iC}{2} E$ implies that $| a (T)| e^{- \pi C \sigma (T)} <
\mathscr{C}$ \qquad for all $T$, where $\mathscr{C}$ is a suitable
constant. If now $Z \in \mathcal{F}$ then  
\begin{align*}
|a (T)e^{- \pi C \sigma (TZ)}| &=|a (T)|e^{- \pi C \sigma
  (Ty)}\\ 
& \le |a (T)|e^{- \pi C \sigma (T)}\\
& \le \mathscr{C}e^{- \pi C \sigma (T)}
\end{align*}
and\pageoriginale consequently,
\begin{align*}
|f(Z)|&= |\sum_{T\ge 0} a (T)e^{2 \pi i \sigma (TZ)}|\\
&\le |\sum_{T\ge 0}a (T)e^{2 \pi  \sigma (Ty)}|\\
&\le \sum_{T\ge 0} \mathscr{C}e^{- \pi  c\sigma (T)}
\end{align*}

The last sum is independent of $Z$ and we will be through if only we
show that this series is convergent. The convergence of this series is
a consequence of the fact that the number of semi integral matrices $T
\ge 0$ whose trace (which is always an integer) is equal to a given
integral value $t$ increases at most as a fixed power of $t$ as $t \to
\infty$. For it $\sigma (T)=t$, $T \ge 0-$ semi integral, then writing
$T=(t_{\mu \nu})$ we have $t_{\nu \nu} \le t$ for all indices $\nu$ so
that the number of possible choices for all $t_{\nu \nu}'s$
together is atmost $(t+1)^n$. Since $T \ge 0$ we have $t_{\nu \nu} \le
t_{\mu \mu}t_{\nu \nu} \le (t+1)^2$ so that $\pm 2t_{\mu \nu} \le 2 (t
  +1)$ and the number of $t_{\mu \nu}'s$ for a given $(\mu
,\nu)$ consistent with this inequality is at most $4t+1$. It
therefore follows that the number of $T'$ satisfying our
requirements can be majorised by 
$$
(t+1)^n (4t+1)^{n(n-1 )/2} \le \mathscr{C}_1 t^{n+n(n-1)/2}=
\mathscr{C}_1 t^{n(n+1 )/2} 
$$
$\mathscr{C}_1$ being suitable positive constant. We can now estimate 
\begin{align*}
\sum _{T \ge 0} \mathscr{C}e^{-\pi c \sigma (T)} &=
\mathscr{C}\sum^\infty_{t=0} \sum _{T \text{ semi integral,}} e^{-\pi
  ct}\\ 
&\le \mathscr{C}\mathscr{C}_1 \sum ^\infty _{t=0} e^{-\pi
  ct}t^{n(n+1)/2} 
\end{align*}
and\pageoriginale the last sum is clearly convergent. The proof of Lemma
\ref{chap5:lem10} is now complete. 

We now apply Lemma \ref{chap5:lem10} to show that every modular form
is boun\-ded in 
the fundamental domain $\mathcal{F}$ of the modular group. In case
$n=1$ this is true by the definition of a modular form. Assume
$n>1$. Since the Fourier series of a modular form may be considered as
a power series, the convergence of such a series is absolute. Hence
every partial series of (\ref{eq93}), viz $\sum _{T \in K} a (T)e^{2
  \pi i \sigma (T>.)}$ where $\mathscr{K}$ denotes an arbitrary set of
semi integral matrices, converges absolutely. Let $T$ be a fixed matrix
such that $a (T)\neq 0$ and $k_T$ the set of matrices
$T_1=T[\mathcal{U}]$ where $\mathcal{U}$ denotes an arbitrary proper
unimodular matrix. Then $a (T_1)= a (T)$ by (\ref{eq95}) and the
series 
\begin{equation*}
g(Z,T)= \sum _{T_1 \in k_T} e^{2 \pi i \sigma (T_1 Z)} \tag{100}\label{eq100}
\end{equation*}
converges absolutely. We shall show that this is possible only if $T
\ge 0$. Let $\nu (T, m)$ denote the number of matrices $T_1 \in k_T$
with $\sigma (T_1)=m$. We shall use the abbreviation $t$ for $e^{2
  \pi}$. 

From (\ref{eq101}) we get
\begin{align*}
g(\text{i.e, } T) & = \sum _{T_1 \in k_T}e^{-2 pi \sigma (T_1)}= \sum ^\infty
_{ m = - \infty} \nu (T ,m)t^{-m}\\ 
& \ge \sum ^\infty _{m=1} \nu (T,-m)t^m \ge \sum ^\infty _{m=1} \nu
(T,-m), 
\end{align*}

If $T \ngeq 0$ we shall show that $\nu (T,-m)\ge 1$ for an infinity of
$m's$ and this will provide a contradiction as we know that
the series (\ref{eq100}) is convergent. If $T \ngeq 0$ we can find an
integral column $\mathscr{Y}$ such
 that\pageoriginale $T[\mathcal{Y}]<0$. Let
$\mathcal{U}=E+ h(h_1 \mathcal{Y},h_2 \mathcal{Y} \cdots \beta _n
\mathcal{Y})$ with integers $h,h_i,i=12,\ldots n$. Since
$\mathcal{U}-E$ has rank 1, we contend  
\begin{align*}
\text{ that } |\mathcal{U}|&= 1+ \text{ trace } \{ h (h_1,
\mathcal{Y},h_2 \mathcal{Y}, \ldots h_n \mathcal{Y})\}\\ 
&= 1+h \sigma (h_1\mathcal{Y},h_2 \mathcal{Y}, \ldots, h_n y).
\end{align*}

To verify this we need only observe that while for any matrix $A$ we
have 
$$
|t E+A|=t^n+ \sigma (A)t^{n-1}+ \cdots,
$$
if $A$ is of rank 1, the coefficient of $t^{n-2}$ and lower powers
of $t$ which depend on subdeterminants of $A$ all vanish and
consequently $|t E +A|= t^n + \sigma (A)t^{n-1}$, 

In particular, choosing $t=1$ we obtain $|E+A| =1+ \sigma (A)$ which
is precisely what we desired. 

Now $\sigma (h_1 \mathcal{Y}, h_2 \mathcal{Y}, \ldots h_n
\mathcal{Y})$ is a linear form in the $h's$ and for $n >1$,
there exists a non trivial integral solution of the equation 
$$
\sigma (h_1 \mathcal{Y}, h_2 \mathcal{Y}, \ldots h_n \mathcal{Y})=0
$$
and then $\sum^n_{\nu= 1} h^2_\nu >0$.

With these $h_\nu's$ and the free variable $h$ we compute 
\begin{align*}
\sigma (T_1)&= \sigma (T[\mathcal{U}])\\
&=\sigma \{ T[E+h(h_1 \mathcal{Y},h_2 \mathcal{Y},\ldots h_n
  \mathcal{Y})]\}\\ 
&= \{ T+h T(h_1 \mathcal{Y},h_2 \mathcal{Y} ,\ldots h_\mu
\mathcal{Y})\\
& \qquad +h'k_i \mathcal{Y} \beta _\nu \mathcal{Y}, \ldots h_n
\mathcal{Y'} \pi + h^2 T[h_1 \mathcal{Y}, \cdot h_n \mathcal{Y}] \}\\ 
&= \sigma (T)+2 h \sigma (\lambda (h_1 \mathcal{Y},h_2 \mathcal{Y},
\ldots h_n \mathcal{Y}))+ h^2 \sigma T[(h_1 \mathcal{Y}, \ldots ,h_n
  \mathcal{Y})]\\ 
&= \sigma (T)+2 h \sigma (T(h_ \mathcal{Y},\ldots h_n
\mathcal{Y}))+\rho^2 T[\mathcal{Y}] \sum^n _{\nu=1} h^2_\nu 
\end{align*}

Choosing\pageoriginale $h$ suitable, this shows clearly that $\sigma (T
[\mathcal{U}])=-m,|\mathcal{U}|=1$ is solvable for an infinite number
of positive integers $m$ and this is equivalent to saying $\nu
(T,-m)\ge 1$ for an infinity of $m's$. Thus we have shown
that, for the series (\ref{eq101}) to converge absolutely, we must have $T
\ge 0$ and this in its turn implies that only such $T's$
occur in the series (\ref{eq93}) representing $f(Z)$. Lemma
(\ref{chap5:lem10}) implies that $f(Z)$ is bounded in
$\mathfrak{f}$. We have now proved  

\setcounter{thm}{3}
\begin{thm}\label{chap5:thm4}%the 4
 Every modular form is bounded in the fundamental domain $\mathcal{F}$ of
  the modular group acting on $\mathscr{Y}$. 
\end{thm}

We proceed to show that the modular forms of negative weight
$(\mathfrak{K}<0)$ must necessarily vanish identically. 

Let $\mathfrak{f}(Z)$ be a modular form of degree $n$ and weight
$\mathfrak{K}$. Then as a result of (\ref{eq72}) it is easy to see that
$h(Z)=|Y|^{\mathfrak{K}/2}|f(Z)|$ is invariant under all modular
substitutions, viz. 
\begin{equation*}
h (M< Z >)=h(Z) \tag{101}\label{eq101}
\end{equation*}
for $M \in M$. If we assume $\mathfrak{K}<0$, then by theorem
(\ref{chap5:thm4}) $ \mathfrak{f}(Z)$ 
is bounded in $\mathcal{F}$ and by (\ref{eq76}) $|y|$ has a positive bound in
$\mathcal{F}$. It follows therefore that $h(Z)$ is bounded in
$\mathcal{F}$ and hence also throughout $\mathscr{Y}$. Let then
$h(Z)\le \mathscr{C}$ for $Z \in \mathscr{Y}$.  

By means of the representation
$$
a (T)e^{-2 \pi \sigma (Ty)}= \underset {\mathscr{H}}{\int \cdots
  \int} f(Z)e^{-2 \pi i \sigma (Tx)}[dx] 
$$
we conclude that
$$
|a (T)|e^{-2 \pi \sigma (Ty)} \le \sup _{\substack{x \in
    \mathscr{H}\\{Z=x+ty}}}| f (Z)|=
\mathscr{C}|\lambda|^{-\mathfrak{K}/2}. 
$$
and\pageoriginale letting $\in \to 0$ the limit process yield that $a
(T)=0$. We then have 

\begin{thm}%the 5
 A modular form of negative weight vanishes identically. We can
therefore assume in the sequel that the weight $\mathfrak{K}$ of a
modular form $\mathscr{F}$ is non negative. Later on we shall show that
if $\mathfrak{K}=0$ then $\mathscr{F}$ is necessarily a constant. 

We now introduce an operator which maps the modular forms of degree
$n > 1$ into those of degree $n-1$ with the same weight. This
operators will be denoted by $\phi$ and the image of
$\mathscr{F}(Z)$ under will be denoted by $\mathscr{F}(Z)| \phi$. The
use of this operator will be particularly felt in such cases where
proofs are based on induction on $n$ 
\end{thm}

We write, in place like these, where we are concerned with modular
forms of different degrees, 
$$
\mathscr{Y}=\mathscr{Y}_n, \mathcal{F}=\mathcal{F}_n, M=M_n.
$$

It is straight forward verification that if $Z \in \mathscr{Y}_n$ the
matrix $Z_1$ arising from $Z$ by cancelling its last row and column
belongs to  $\mathscr{Y}_{n-t}$ and if $Z_1 \in \mathscr{Y}_{n-1}$ then
the matrix $Z= \begin{pmatrix}Z_1&0 \\ 0& i \lambda \end{pmatrix}
\in y_n$ provided $\lambda >0$ 

We can then form the function $f \begin{pmatrix} Z &0 \\0&i
  \lambda \end{pmatrix}$ define for every $Z_i  \in y_{n-1}$ and
$\mathfrak{f}$, a modular form of degree $n-1$. We shall show
that $\lim \limits _{\lambda \to \infty}\mathfrak{f} \begin{pmatrix} Z
  &C \\0&c \lambda \end{pmatrix}$ exists, denoted by $f(Z_t)$ and this
will be the modular form $f(z)| \phi$ of degree $n-1$ and weight
$\mathfrak{K}$. 

Let $\mathcal{L}$ be a compact subset of $y_{n-1}$ and $Z_1= X_1
 + i Y_1 \in \mathcal{L}$. 

We\pageoriginale show first that for any $T_1 \ge 0$,
\begin{equation*}
\sigma (T_1y_1) \ge \lambda \sigma (T_1) \tag{102}\label{eq102}
\end{equation*}
where $\gamma =\gamma (\mathcal{L})>0$. We need only consider the case
$\sigma (T_1)>0$ as otherwise $T_1=(0)$ and the inequality reduces to
a trivial equality. Then by reason of homogeneity of both sides in
$T_1$, we can assume $\sigma (T_1)=1$ The equations: 
$$
\sigma (T_1)=1, T_1 \ge 0,Z_1 \in \mathcal{L}\subset \mathscr{Y}_{n-1}
$$
define a compact $Z_1,T_1$-set and on this set, $\sigma (T_1Y_1)$ is a
function, continuous in $T_1$ and $Y_1$-If we show that this function
is positive at every point, then it has a positive minimum $\gamma$ in
this set and we would have proved (\ref{eq102}). Let $R=R^{n-1} \neq 0$ be
determined with $T_1=RR'$ and let
$R=(\mathscr{W}_1,\mathscr{W}_2,\ldots \mathscr{W}_{n-1})$. 

Then $\sigma (T_1Y_1)=\sigma(Y_1 [R])=\sum^{n-1}{\nu=1}
y_1[\mathscr{W}_\nu]$, and the last sum is positive as $Y_1>0$ and at
least one of the columns is nonzero. This settles our
contention. Since the Fourier series  
$$
f(Z)=\sum_{T\ge 0} a (T)e^{2 \pi i \sigma (TZ)}
$$
converges everywhere and in particular at the point
$Z=\dfrac{i}{2}\gamma E$, we have $|a(T)|\le \mathscr{C}e^{\pi
  \gamma \sigma (T)}$ for $T \ge 0$ and a certain positive constant
$\mathscr{C}$. 

Writing $Z = \begin{pmatrix} Z_i &0 \\ 0& t \lambda \end{pmatrix}, Z_1
\in \mathcal{L}$ \qquad and decomposing $T$ analogously as $T=(t_{\mu
  \nu})= \begin{pmatrix} T_1 &\mu \\ \nu & t_{nn}\end{pmatrix}$ we get
from the above that 
 $$
 |a(T)|e^{2\pi i \sigma (TZ)}\le \mathscr{C} e^{\pi\nu\sigma (T)}
 e^{-2 \pi \sigma (YT)}  
 $$
  \begin{align*} 
&=\mathscr{C}e^{\pi \lambda \sigma (T)} e^{2 \pi ( \sigma
      (T_1y_1)+\lambda tnn)}\\ 
&=\mathscr{C}e^{\pi \gamma \sigma (T_1)-2 \pi \sigma (T_1y_1)-\pi (2
      \lambda -\gamma)t_){nn}}\\ 
& \le \mathscr{C}e^{-pi \gamma \sigma(T_1)-\pi \gamma t_{nn}}\\ 
&=\mathscr{C}e^{-pi \gamma \sigma (T)}
 \end{align*} 
 assuming\pageoriginale $\lambda \ge \gamma$. Thus if $Z_1 \in
 \mathcal{L}, \lambda 
 \ge \gamma (\mathcal{L}),Z= \begin{pmatrix}Z_1 &0 \\ 0& t
   \lambda \end{pmatrix}$ then $|a(T)|e^{2 \pi i \sigma (TZ)}!\le
 \mathscr{C}^{-pi \gamma \sigma (T)}$ 
 
 It is now immediate that the series 
 $$
 f(Z)= \sum_{T \ge 0} \alpha (T)e^{2 pi i \sigma (Tz)}
 $$
  which is majorised by $\mathscr{C}\sum _{T \ge 0} e^{-pi \gamma
    \sigma(T)}$ independent of $Z$, converges uniformly for $Z_1$ in
  every compact domain $\mathcal{L}\subset\mathscr{Y}_{n-1}$ and $\lambda
  \ge \gamma (\mathcal{L})$. Then 
  \begin{align*}
\lim_{\lambda \to \infty} \mathfrak{f}(Z) & = \lim_{\lambda \to \infty}
\mathfrak{f}\begin{pmatrix} Z_1 &0 \\ 0& \ell \lambda \end{pmatrix}\\ 
& = \sum_{T \geq 0} a(T) \lim_{\lambda \to \infty}  e^{2 \pi i \sigma
  (TZ)}\\ 
& = \sum_{T \geq 0} a(T) \lim_{\lambda \to \infty}  e^{2 \pi i \sigma
  (T_1Z_2) - 2 \pi \lambda t n n} 
\end{align*}

\setcounter{pageoriginal}{72}
In the last series, the terms involving $T$ for which $t_{n n}>0$
vanish in the limit and only there terms for which $t_{ n n} = 0$
survive. We than obtain  
\begin{align*}
\lim_{\lambda \to \infty} \mathfrak{f}(Z)& = \mathfrak{f}_1(Z_1)\\
& = \sum_{T \geq 0 t_{n n } = 0} a(T) e^{2 \pi i 5(T_1 z_1)}
\end{align*}
\begin{equation*}
\sum_{T_{\nu} 0 \geq 0} a(T_1) e^{2 \pi i \sigma (T_1, z_1)}\tag{103}\label{eq103}
\end{equation*}\pageoriginale
where by definition $a(T_1) = a \begin{pmatrix} T_1 & 0 \\ 0&
  0 \end{pmatrix}$ 

It is clear that t $\mathfrak{f}_i (Z_i)$ is regular in
$\mathscr{Y}_{n-1}$ as the corresponding Fourier series converges
uniformly in every compact subset of $\mathscr{Y}_{n-1}$. Further
$\mathfrak{f}_1 (Z_2)$ is bounded in the fundamental domain
$\mathfrak{f}_{n-1}$ as in the series (\ref{eq103}) only those $T_1's$ occur
which are semi positive. It remains to show that 
$\mathfrak{f}_1(Z_1)$ is actually a modular form of degree $n-1$ and
weight $\mathscr{R}$ 

Let $M_1  =  \begin{pmatrix} A_1 & B_1 \\ C_1 & D_1 \end{pmatrix} \in
M_{n-1}$ We complete $M_1$ to a modular matrix $M$ of degree $n$ as
follows. 
\begin{align*}
& M = \begin{pmatrix} A & B\\ C & D \end{pmatrix} \text{ where } A
  = \begin{pmatrix} A_1 & o \\ o & 1 \end{pmatrix} , B
  =\begin{pmatrix} B_1 & 0 \\ 0 & 0 \end{pmatrix} \\ 
& C = \begin{pmatrix} C_1 &0 \\ 0& 0 \end{pmatrix} ,\text{ and }
  D=\begin{pmatrix} D_1 & 0 \\ 0 & 1 \end{pmatrix}  
\end{align*}

If the $Z= \begin{pmatrix} Z_1 &0 \\ 0& \ell \lambda \end{pmatrix} ,
L_1 \in \mathscr{Y}_{n-1}$ we have $M<Z> = (AZ+B) (CZ + D)^{-1}=$ 
\begin{align*}
&=\begin{Bmatrix}\begin{pmatrix}A_I &0\\ 0&
    1 \end{pmatrix} \begin{pmatrix}z_1 & 0 \\ 0 & i\lambda \end{pmatrix}
  + \begin{pmatrix}B_1 & 0 \\0 &
    0 \end{pmatrix} \end{Bmatrix}\begin{Bmatrix}
    \star \end{Bmatrix}^{-1}\\ 
&=  \begin{pmatrix}A_1 Z_1 +B_1 & 0 \\0 &
    i\lambda \end{pmatrix}  \begin{pmatrix} C_i Z_i+0_1 & 0\\ 0
    &1\end{pmatrix}^{-1}  \\ 
&=  \begin{pmatrix}A_1 Z_1 +B_1 & 0 \\0 &
      i\lambda \end{pmatrix}  \begin{pmatrix} (C_i Z_i+0_1)^1 & 0\\ 0
      &1\end{pmatrix}^{-1}  \\ 
&= \begin{pmatrix} M_1 < Z_1> & 0\\0 & i\lambda \end{pmatrix} 
\end{align*}\pageoriginale

Also $|CZ+D1= |C_1 Z_1 +D_1|$. Hence we obtain from the relation $f(M <
Z>)  |CZ+D1^{-\mathscr{R}}= f, z)$ that 

$\begin{pmatrix} M_1 < Z_1> & 0\\  0 & \lambda \end{pmatrix} |C_1 Z_1
+D_1|^{\mathscr{R}}=f \begin{pmatrix} Z_1 & 0\\ 0 &
  \lambda \end{pmatrix}$ 
which as $\lambda \to \infty$ yields 
$$
f_1 (M_1 \langle Z_1 \rangle)|C_1 Z_1+D_1|^{- \mathscr{R}}= f_1(Z_1),
$$

That is to say $f_1(Z_1) |M_1= f_1(Z_1)$ for $M_1 \in M_{n-1}$ and it
is immediate that $f_1(Z_1)$ is a modular form of degree $n-1$ and
weight $\mathscr{R}$. 

