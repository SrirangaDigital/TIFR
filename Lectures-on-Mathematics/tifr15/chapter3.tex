\chapter{Preliminaries (Continued)}\label{chap3}

\section[Fourier transforms of distributions with...]{Fourier transforms of distributions with compact support and
 the theorem of Paley-Wiener}\label{chap3:sec1}%Sec 1 

Let\pageoriginale $T$ be a distribution (in particular a measure ) with compact
support. We call the \textit{segment of support} of $T$ the smallest
closed interval [$a, b$] containing the support of $T$. Let $ d \mu$ be
a measure with compact support. Its Fourier-transform $\mathscr{C} (d
\mu) = M(w) = \int e^{-ixw} d \mu (x)$ is an entire function. Writing
$w = u + iv$ we have: 
\begin{align*}
 &\Big| M(w) \Big| < e^{bv} \int \Big| d \mu \Big|, v \geq 0;\\
 &\Big| M(w) \Big| < e^{av} \int \Big| d \mu \Big|, v \leq 0
\end{align*}
where [$a, b$] is the segment of support of $d \mu$. Thus $M(w)$ is an
entire function of exponential type bounded on the real line, i.e.,: 
\begin{equation*}
 \Big| M(w) \Big| \leq K e^{c|w|}; M(u) = 0(1) \tag{1}\label{chap3:sec1:eq1}
\end{equation*}
Moreover 
\begin{equation*}
 M(iv) = 0 (e^{bv}), v > 0 ~M(iv) = 0 (e^{av}), v < 0. \tag{2}\label{chap3:sec1:eq2}
\end{equation*}

We shall try to find, whether the condition (\ref{chap3:sec1:eq1}) is sufficient in
order that an entire function $M(w)$ be the Fourier-transform of a
measure $d \mu$ with compact support, and whether condition (\ref{chap3:sec1:eq2}) is
sufficient to prove that the support of $d \mu$ is contained in [$a,
 b$]. In fact, (\ref{chap3:sec1:eq1}) is not sufficient, and we must replace it by a
slightly stronger condition. 

\textit{Theorem of Paley-Wiener. Let $M(w)$ be an entire function of
 exponential type satisfying the condition:} 
\begin{equation*}
 \Big| M(w) \Big| \leq K e^{c | w |} ; M(u) = 0
 \left(\frac{1}{|u|^2}\right) \tag{$1'$} 
\end{equation*}

\noindent
\textit{Then\pageoriginale $M(w)$ is the Fourier transform $\mathscr{C}(d \mu)$ of a
 measure with compact support.} 
 
 The proof runs in three parts.
 \begin{enumerate}[(a)]
\item Theorem of Phragmen-Lindel of: let $\varphi (z)$ be a function,
 holomorphic for $\Big|\arg ~ z\Big| \leq \alpha <
 \dfrac{\pi}{2}$; we suppose that $\varphi(z) = 0(e^{\varepsilon
 |z|})$ for every $\varepsilon >0$ and that $| \varphi (z) | \leq
 B$ on $|\arg ~ z| = \alpha$. Then $|\varphi (z) |\leq B$, when
 $|\arg ~ z| \leq \alpha$ 

 We take $\varphi (z) e^{- \varepsilon' z} $ and to this we apply the
 principle of Maximum Modulus to get the result. (Titchmarsh 5.62) 
\item The result $(a)$ is translated into a theorem for a function
 $\psi(w)$ holomorphic in $0 \leq ~ \arg ~ w \leq \dfrac{\pi}{2}$. 

 Let $\psi (w)$ be holomorphic in this domain, bounded on the
 boundary, and let $(*)\, \psi (w) = 0 (e^{\eta |w|^{2-\varepsilon}})$
 for some $\varepsilon > 0$ and for every $\eta > 0$. Then $\psi (w)
 = 0(1)$ when $0 \leq $ arg $w \leq \dfrac{\pi}{2}$. 

 We take $\varphi (z) = \psi (w), z = w^{2- \varepsilon}$ and
 apply $(a)$. We can replace $(*)$ by $\psi (w) = 0
 (e^{|w|^{\alpha}})$ for some $\alpha < 2$. 
\item \textbf{Proof of the theorem}: by $(1')$, the function $w^2 M(w)
 e^{icw}$ satisfies the conditions of $(b) $ in $0 \leq \arg ~ w \leq
 \dfrac{\pi}{2}$ and $\dfrac{\pi}{2} \leq \arg w \leq \pi$ and the
 function $w^2 M (w) e^{-icw}$ satisfies the same conditions in the
 other quadrants and so we have $M (w) =
 0(\dfrac{e^{c|v|}}{|w|}^2)$. As $M(u) = 0(\dfrac{2}{u^2})$, $M(u)$
 admits a co-Fourier transform (Conjugate Fourier Transform), $f(x)$
 given by 

 $f(x) = \dfrac{1}{2 \pi} \int M(u) e^{ixu} du. M(w)e^{ixw}$ is an
 entire function. We apply Cauchy's theorem to this function along the
 rectangular\pageoriginale contour with sides $-R \leq u \leq R; R + iv, 0 \leq v
 \leq v_o; u + iv_o, -R \leq u \leq R; -R + iv, 0 \leq v \leq
 v_o$. Letting $R \to \infty$ we have the equation 

 $\dfrac{1}{2 \pi} \int M(u) e^{ixu} du = \dfrac{1}{2 \pi} \int M(u)
 e^{ixu} du, w = u + iv_o, \int = \int\limits^{\infty}_{- \infty}$
 Therefore $f(x) = \dfrac{1}{2 \pi} \int M(w) e^{ixu} du$. Suppose $x
 > c$. Then there exists $c'$ with $x > c' > c'$, 
 $$
 \Big|M(w) e^{ixw}\Big| K \frac{e^{c|v|}}{|w|^2} e^{-c'v} < K'
 \dfrac{e^{c|v|- c'v}}{1+u^2}. 
 $$

 This is true for every $v = v_0 > 0$. Allowing $v_0 \to \infty$ we
 find $f(x)= 0$. In a similar manner, if $x< -c$ we have $f(x) =
 0$. This means that the support of $f(x) \subset [-c, c]$, Then
 $M(w) = \int f(x) e^{-ixw} dx = \int e^{-ixw} d \mu (x), d \mu (x) =
 f (x) dx$. 
 \end{enumerate} 
 
\section[Refinements and various forms of...]{Refinements and various forms of the theorem of
  Paley-Wiener}\label{chap3:sec2}%Sec 2
 
 The second part of the theorem of Paley-Wiener consists in proving
 that the condition (\ref{chap1:sec2:eq2}) (which is merely a condition on the
 behaviour of $M(w)$ on the imaginary axis) is sufficient to know that
 the segment of support of $d \mu $ is [$a, b$]. This second part is
 due to Polya and Plancherel. (Plancherel). 
 \begin{theorem*}
 We suppose that the entire function $M (w) $ satisfies the conditions:
 \begin{align*}
  M(w) &\leq K e^{c|w|} \text{ and } M(u) = 0
  \left(\frac{1}{|u|^2}\right) \tag{$1'$}\\ 
  M(iv) &= 0(e^{bv}), v > 0; M(iv) = 0(e^{av}), v < 0. \tag{$2'$}
 \end{align*} 
 \end{theorem*}

\textit{Then $M(w) = \mathscr{C}(d \mu)$ and the segment of support of
 $d \mu$ is contained is [$a, b$].} 

\begin{proof}
 We\pageoriginale have seen, from the first part of Paley-Wiener theorem, that
 $(1')$ gives us that $M(w) = \mathscr{C} (d \mu)$, with $d \mu =
 f(x) dx$, and $f(x) = 0$ for $|x| >c$. Condition $(2')$ gives
 $\Big|M(w) e^{ibw}\Big| = 0(1)$ on the positive part of the
 imaginary axis. Then it is easy to see that $f(x) = 0$ for $x \geq
 b$ following the same line of argument as in part $(c)$ of the first
 part of the theorem of Paley-Wiener. In a similar fashion we prove
 that $f(x) = 0$ for $x < a$. 
\end{proof}

Now we study $d \mu = f(x) dx$, where $f \in \mathscr{D}(R)$ and the
segment of support of $f$ is [$a, b$]. Then $ M(w) = \mathscr{C} (d
\mu)$ satisfies: 

$(1'')$ ~ $M(w)$ is an entire function of exponential type with $M (u)
= 0 (\dfrac{1}{|u|^n})$ for every integer $n$; 

$(2'')$ ~ $\lim\limits_{v \to \infty} \sup \dfrac{\log |M(iv)|}{v}= b
,\lim\limits_{v \to \infty} \inf \dfrac{\log |M(iv)|}{v} = a$. 

For $M(w) = \dfrac{1}{2 \pi} f (x) e^{-ixw} dx = \dfrac{1}{2
 \pi}\dfrac{1}{(iw)^n} \int f^{(n)}(w)e^{-ixw} dx$ for every $n$. 

 So we have $(1'')$. Conversely if we have $1'')$, then $M(w) =
 \mathscr{C}(f)$ with $f \in \mathscr{D} (R)$, for we have already
 $f(x) = \dfrac{1}{2 \pi} \int M(u) e^{ixu} du$ and $1'')$ gives us that
 $f \in \mathscr{D} (R)$. 

 $(2'')$ ~ implies $(2')$ with $a'$ and $b'$ in $(2')$, $a' < a < b < b'$,
 and so we have the segment of support of $f$ actually [$a, b$]. Hence 

\begin{theorem*}
 Conditions $(1'')$ and $(2'')$ are necessary and sufficient in order
 in order that $M(w) = \mathscr{C} (f)$ with $f \in \mathscr{D} (R)$
 and support of $f= [a, b]$.
\end{theorem*}

Now we generalize the theorem to distributions. Let $\tau \in
\mathscr{E}' (R)$ be a distribution with the segment of support [$a,
 b$]. Its 
Fourier\pageoriginale transform is $T(w) = \langle \tau, e^{-ixw} \rangle$. We know
that $\tau$ is a finite linear combination of derivatives of measures
(Schwartz $4$, Chap.$III$, theorem $26$) $\tau = \sum
\dfrac{d^n}{dx^n} (d \mu_n)$. So $T(w)$ can be written as: 
\begin{equation*}
 T(w) = <\sum \frac{d^n}{dx^n} d \mu_n, e^{ixw} > =
 < d \mu_n, \pm \frac{d^n}{dx^n} e^{-ixw} > =
 \Sigma w^n i^n M_n (w) 
\end{equation*}
Thus we have $T(w)$ satisfying the following conditions:

\begin{itemize}
\item[$(1''')$] $T(w)$ is an entire function of exponential type with
 $\big | T(u) \big | = 0 (| u |^N) $ for some $N$. 
\item[($2''')$] $\lim \sup\limits_{v \rightarrow \infty}\dfrac{\log
 \big| T(iv) \big|}{v} = b$, $\lim \inf\limits_{v \rightarrow-
 \infty}\dfrac{\log |T(iv)|}{v}= a$, 
\end{itemize}

Conversely let $T(w)$ satisfy conditions $(1''')$ and $(2''')$. From
$(1''')$ we can write $T(w) = P (w) M(w)$, where $P(w)$ is a
polynomial and $M(w)$ satisfies $(1')$. This implies that $M(w) =
\mathscr{C} (d \mu)$ and $T(w) = \mathscr{C} (\sum\limits^{N}_o a_n$ 
$\dfrac{d^n}{dx^n} d \mu ) = \mathscr{C} (\tau)$. $\tau \in
\mathscr{E}' (R)$. We find as before that $(2')$ implies that the
segment of support of $\tau$ is [$a, b$]. 

\begin{theorem*}
 Conditions $(1''')$ and $(2''')$ are necessary and sufficient in
 order that $T(w) = \mathscr{C}(\tau) $ with $\tau \in \mathscr{C}'
 (R)$ and support of $\tau = [a, b]$. 
\end{theorem*}

The theorem of Paley-Wiener gives an easy characterisation of the
Fourier transforms of the $f \in \mathscr{D} (R)$ or $T \in
\mathscr{E}' (R)$. The original form of the theorem of Paley-Wiener
states that a necessary and sufficient condition in order a function
be the Fourier transform of a function $\in \mathscr{E}' \bigcap L^2$,
is that it should be of exponential type, and $\in L^2$ on the real
axis. This last statement results from the preceding one, concerning
the Fourier forms of $T \in \mathscr{E}'$, and from\pageoriginale the
invariance of $L^2$ under Fourier transformation. 

\section{Theorem of Hadamard}\label{chap3:sec3}%Sec 3

We recall the classical theorem of Hadamard (Titchmarsch), \textit{Let
 $M(w)$ be an entire function of exponential type with zeros}(other
than $0$) $\{ \lambda_n \}$. \textit{Then $M(w) = K e^{aw} w^k \prod
 \left(1 - \dfrac{w}{n}\right) e^{w/\lambda_n}$ and $\sum \dfrac{1}{|
 \lambda_n |^2} < \infty$} 

Later on we shall give more properties of functions of exponential type.

\section{Convolution product and its simple properties}\label{chap3:sec4}%sec 4

Let $f$ and $g$ be two functions (integrable, continuous of
$C^{\infty}$ functions) with segment of support [$a, b$] and [$c,
 d$]. The convolution of $f$ and $g$ is the function $h = f* g$
defined by 
$$
h(y) = \int f(y-x) g(x) dx = \int f(x) g(y - x) dx.
$$

The convolution product is commutative $(f* g = g * f)$, associative
$((f * g) * k = f * (g * k))$ and distributive with respect to
addition. 

For $y < a + c, h (y) = 0$ and for $y > b + d, h(y) = 0$. So the
segment of support of $f * g$ is contained in the segment of support
of $f + $ segment of support of $g$. 
$$
\text{ If } \ell \in \mathscr{C}, \int h(y) \ell (y) dy = \iint \ell
(x + x') f(x') g(x) dxdx'. 
$$

 This allows us to define the convolution of two measures with
 compact, support by means of duality. The convolution of two measures
 $d \mu_1$, $d \mu_2 \in \mathscr{C}'$ is a measure $d \nu \in
 \mathscr{C}', d\nu = d \mu_1 * d \mu_2$ defined by 
$$
\langle \ell, d\nu \rangle = \iint \ell (x + y) d\mu_1 (x) d\mu_2 (y)
$$
for\pageoriginale every $\ell \in \mathscr{C}$. The convolution product is again
associative, commutative and distributive with respect to addition. We
have segment of support of $d \nu \subset$ segment of support of $d
\mu_1 + $ segment of support of $d \mu_2$. Indeed if the support on
$\ell \subset (-\infty, a + c)$ or if the support of $\ell \subset (b
+ d, \infty)$ then $\int \ell d \nu = 0$. 

We have the same definition, by duality, for two distributions with
compact support, $S = T_1 * T_2$ is the distribution defined by
$\langle S, \ell \rangle = \langle \ell (x + y), T_{1x}$. $T_{2y}
\rangle$ for every $\ell \in \mathscr{E}$, $T_{1x}$. $T_{1y}$ being
the Cartesian product of $T_1$ and $T_2$. (Schwartz $4$ Chap. $IV$). 

By considering the duality between distributions (respectively
measures) with non-compact support and $\mathscr{D}(R)$ (respectively
the space of continuous functions with compact support), we can define
in the same way the convolution of a distribution $T_1$ (respectively
measure $d \mu_1$) and a distributive $T_2$ (respectively measure $d
\mu_2$) with compact support and $S = T_1 * T_2$ (respectively $d \nu =
d \mu_1 * d \mu_2$) will be in general a distribution (respectively
measure ) with non-compact support. For example 

$\delta_a * f = f(x - a) = f_a ; \dfrac{d}{dx} * f = f'$
etc. (Schwartz 4, Chap. VI). 

The convolution of several distributions (or measures), all but one of
which has a compact support, is associative and commutative. 

If $d \mu_1, d\mu_2, T_1, T_2$ have compact supports, we have
immediately 
$$
\displaylines{\hfill 
 \mathscr{C} (d \mu_1 * d \mu_2) = \mathscr{C} (d \mu_1 ) \mathscr{C}
 (d \mu_2), \mathscr{C} (T_1 * T_2) = \mathscr{C} (T_1) \mathscr{C}
 (T_2) \hfill \cr
 \hfill 
 e^{i \lambda x} T_1 * e^{i \lambda x} T_2 = e^{i \lambda x} (T_1 *
 T_2)\hfill }
$$
($\mathscr{C} = \langle T_x, e^{i \lambda x} \rangle = $ Fourier
transform of $T$). We can extend these equalities to more general
cases. For example, the first holds if $d \mu_1$ has a compact support
and $\int \bigg | d \mu_2 \bigg | < \infty ;$ the third holds whenever
$T_1$ has a compact support. 

The\pageoriginale convolution by a measure with compact support transforms a
continuous function into a continuous function, and also (see 
``$L'$ integration dans les groupes topologiques" by Andre Weil) a function
which is locally $\in L^p$ into a function which is locally $L^p (p
\geq 1)$. 
