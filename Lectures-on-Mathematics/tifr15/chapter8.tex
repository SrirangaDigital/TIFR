\chapter[Approximation by Dirichlet's polynomials and...]{Approximation by Dirichlet's polynomials and some
 problems of closure}\label{chap8} %chap 8 

\section{Dirichlet polynomials and approximation on an
  interval}\label{chap8:sec1}  %sec 1

\textit{A\pageoriginale Dirichlet polynomial} is a finite sum of the form $\sum a
(\lambda) e^{\lambda z}, \lambda \in C, z \in C$. Before studying the
approximation by such polynomials in a domain $\Omega$ of the complex
plane, we study the same problem in the space $\mathscr{C} (I)$. 

Let $\mathscr{C}_\Lambda (I)$ be the closed subspace of $\mathscr{C}
(I)$ spanned by $\Big\{e^{i \lambda x}\Big\}_{\lambda \in \Lambda}$
(if $\Lambda$ is simple) or $\Big\{x^p e^{i \lambda x}\Big\}
(\lambda)^{p + 1} \in \Lambda$ (let us recall that $(\lambda)^{p + 1}
\in \Lambda$ means $\lambda \in \Lambda$, at least $p + 1$ times) and
let $\mathscr{C}_\Lambda (I) \neq \mathscr{C} (I)$. We have the
following theorem: 
\begin{theorem*}
 Suppose $\mathscr{C}_\Lambda (I) \neq \mathscr{C}(I)$. Then
 $\Big\{x^p e^{i \lambda x}\Big\} (\lambda)^{p + 1} \in \Lambda$ form
 a basis of $\mathscr{C}_\Lambda (I)$ and each function in
 $\mathscr{C}(I)$ is characterised by its development. 

 For, suppose $I = [0, 1] f \in \mathscr{C}_\Lambda (I)$, and let $d
 \mu$ be a measure $\nequiv 0$, with support in $I$, such that
 $\int_I x^p e^{i \lambda x} d \mu (x) = 0 (\lambda)^{p + 1} \in
 \Lambda$. Let us put $f^* = f$ on $I, f^* = 0$ outside $I$, and $g-
 = f^* \ast d \mu$ on $I, g = 0$ outside $I$. Defining $F(w) =
 \dfrac{\mathscr{C}(g)}{\mathscr{C}(d \mu)}$, we get the same
 relation as in lect. 6 \S\ref{chap6:sec2} between the polar part of $F(w)$ and
 the Fourier expansion of $f$, when $f$ is a finite linear
 combination of $x^p e^{i \lambda x}$; then, taking a limit, we prove
 the existence of a Fourier expansion $S(f)$ for all $f \in
 \mathscr{C}_\Lambda (I)$, with the\pageoriginale same relation as in lecture 6,
 \S\ref{chap6:sec2}, between $S(f)$ and $F(w)$; hence $S(f) \equiv 0
 \Longrightarrow F \equiv 0 \Longrightarrow g \equiv 0
 \Longrightarrow f^{**} d \mu = 0$ on $I$. The same holds if we take
 $I = [- \ell, 0]$. Using these remarks, it is easy to see that $S(f)
 \equiv 0 \Longrightarrow f^* \ast d \mu \equiv 0 \Longrightarrow f^*
 \equiv 0 \Longrightarrow f \equiv 0$, and the theorem is proved. 
\end{theorem*}

We consider the same kind of problem in the complex plane; to this end
we first describe space analogous to $\mathscr{C}_\Lambda (I)$ and
obtain a theorem of closure. 

\section{Runge's theorem}\label{chap8:sec2}%sec 2

Let $\Omega$ be an open set in $C$ and $\mathscr{H}(\Omega)$ be the
space of holomorphic functions in $\Omega$ with the compact
convergence topology. $\mathscr{H}(\Omega)$ is an $\mathcal{F}-$
space. Let $\mathscr{H}' (\Omega)$ be its dual. Since
$\mathscr{H}(\Omega)$ is a closed subspace of $\mathscr{C}(\Omega)$,
by the Hahn - Banach theorem $\mathscr{H}' (\Omega) \subset
\mathscr{C}' \Omega$. Every vector of $\mathscr{H}'(\Omega)$ defines
an equivalence class of measures in $\mathscr{C}'(\Omega)$, which are
merely the extensions of this vector (by the Hahn - Banach theorem) to
a linear functional in $\mathscr{C}(\Omega)$. In other words, $d \mu_1
\sim d \mu_2$ if for every $f \in \mathscr{H}(\Omega), \int_\Omega f d
\mu_1 = \int_\Omega f d \mu_2$. Thus the dual $\mathscr{H}'(\Omega)$
of $\mathscr{H}(\Omega)$ is the quotient of $\mathscr{C}'(\Omega)$ by
the subspace orthogonal to $\mathscr{H}(\Omega)$. 

\medskip
\noindent
\textbf{Runge's theorem.} \textit{Suppose $\Omega$ is connected (but
 $\Omega$ need not be connected). In $\mathscr{H}(\Omega)$ the set
 $\{z^p\}, p = 0, 1,\ldots $ form a total set.} 

\textit{In other words, by the condition of Riesz, $\int z^p d \mu =
 0$ for $p = 0, 1, \ldots$ implies $\int f (z) d \mu = 0$ for every
 $f \in \mathscr{H}(\Omega)$ }. 

\begin{proof}
 Let\pageoriginale support of $d \mu ' \subset \Omega' \subset \bar{\Omega}'
 \subset \Omega (\Omega')$
 open and connected ) and let $|z_o|$ be large enough. Then
 $\dfrac{1}{z-z_o}= - \dfrac{1}{z_o} (1 + \dfrac{z}{z_o} +
 \cdots)$, the series in the right hand side being convergent in
 $\Omega'$. Thus $\dfrac{1}{z - z_o}$ belongs to the closed span of
 the monomials in $\mathscr{C}(\Omega')$. Moreover, $\dfrac{1}{(z- z_o)^{n +
  1}}$ belongs to this closed span ; then $\int \dfrac{d \mu
 (z)}{(z - z_o)^{n + 1}} = 0 (n = 0, 1, \ldots)$. But $\varphi
 (\zeta) = \int \dfrac{ d \mu (z)}{z - \zeta}$ is holomorphic outside
 $\Omega'$ (because $[\Omega'$ is connected) ; from $\varphi^{(n)}
 (z_o) = n ! \int \dfrac{ d \mu (z)}{(z - z_o)^{n + 1}} = 0 (n = 0,
 1, \ldots)$ results $\varphi \equiv 0$. Let $C$ be closed
 rectifiable curve around $\Omega'$ in $\Omega - \Omega'$. Then by
 Cauchy's theorem 
 \begin{align*}
  0 & = \frac{1}{2 \pi i} \int_C \varphi (\zeta) f (\zeta) d\zeta
  = - \int d \mu (z) \frac{1}{2 \pi i} \int_C f (\zeta) \frac{d
  \zeta}{\zeta - z}=\\ 
  & = - \int f (z) d\mu (z)
 \end{align*}
 \textit{Remarks about $\mathscr{H}' (\Omega)$}.
\end{proof}

\begin{remarks}\label{chap8:sec2:rem1}%remark 1.
 To each $d \mu \in \mathscr{C}' (\Omega)$ corresponds a $\varphi
 (\zeta) = \int \dfrac{d \mu (\zeta)}{z - \zeta}$ which is
 holomorphic outside the support $K$ of $d \mu$, and vanishing at
 infinity. $d \mu_1 \sim d \mu_2$ if and only if $\varphi_1 (\zeta) =
 \varphi_2 (\zeta)$, and the duality between $\mathscr{H} (\Omega)$
 and $\mathscr{H}'(\Omega)$ can be defined by $\int f d \mu = \int_C
 f (z) \varphi (z) dz, f \in \mathscr{H} (\Omega), C$: curve
 surrounding $K$, and contained in $\Omega$. Then it is convenient to
 represent the elements of $\mathscr{H}' (\Omega)$ as the functions
 vanishing at infinity and holomorphic outside a compact subset of
 $\Omega$. 
\end{remarks}

\begin{remarks}\label{chap8:sec2:rem2}%rema 2
 Let $\Omega$ be connected and $0 \in \Omega$. Suppose $\varphi (z) =
 \sum\limits^\infty_{0} \dfrac{a_n}{z^{n + 1}}$. Then $\dfrac{1}{2 \pi i} \int_c
 f (z) \varphi (z) dz = \sum\limits^\infty_0 \dfrac{a_n}{n} f^{(n)}
 (0)$. Thus one can represent\pageoriginale the linear functional on $\mathscr{H}
 (\Omega)$ as differential operators of infinite order with constant
 coefficients. In general, the relation between the coefficients
 $a_n$ and $\Omega$ is not simple. It is simple when $\Omega$ is a
 circle around origin and of radius $R$. Then $\sum \dfrac{a_n}{n!}
 f^{(n)} (0)$ is a linear functional if and only if $\lim \sup
 |a_n|^{1/n} < R$. 
\end{remarks}

We give without proof an extension of Runge's theorem due to Mer\-gelyan
and Lavrentie (Mergelyan). Let $K$ be a compact set of the complex
plane and let $\mathscr{H}(K)$ be the space of continuous functions on
$K$ which are holomorphic in the interior of $K$ with the topology of
Uniform Convergence. 

\medskip
\noindent
\textbf{Mergelyan's theorem.} \textit{ In order that the set $z^p, p =
 0, 1, \ldots$ be total in $\mathscr{H}(K)$ it is necessary and
 sufficient that the complement of $K$ should consist of one region
 (i.e., ``$K$ does not divide the plane '' ).} 

\section{Problems of Closure in the Complex Plane}\label{chap8:sec3} %sec 3

In terms of the duality between $\mathscr{H}(\Omega)$ and
$\mathscr{H}' (\Omega)$ we obtain a condition of closure. 

Let $\mathscr{H}_\Lambda (\Omega)$ be the closed span of $\{e^{\lambda
 z}\}_{\lambda \in \Lambda}$ ( or $\big\{z^p e^{\lambda z}\big\}
(\lambda)^{p + 1} \in \Lambda$ ) in $\mathscr{H} (\Omega)$. Then, by
the condition of Riesz, we have 
{\fontsize{10}{12}\selectfont
$$
\mathscr{H}_\Lambda (\Omega) = \mathscr{H} (\Omega)
\Longleftrightarrow \left[ \int e^{\lambda z} d \mu = 0, \lambda \in \Lambda
 \Longrightarrow \int fd \mu = 0, f \in \mathscr{H} (\Omega)\right] 
$$}\relax

Let $\varphi(\zeta) = \int \dfrac{d \mu (z)}{(z- \zeta)}, \zeta \not\in$
support of $d \mu$, then let $\Phi (w) = \int e^{wz} d \mu (z)$. We
may call $\Phi (w)$ the transform of $d \mu$. Then condition $\int
e^{\lambda z} d \mu (z) = 0, \lambda \in \Lambda$ is merely $\Phi
(\Lambda) = 0$ and if this implies $\int f d \mu = 0, f \in
\mathscr{H}(\Omega)$, by our duality, $\varphi(\zeta) \equiv 0$ and $\Phi
(w) \equiv 0$.\pageoriginale Conversely if $[\Phi (\Lambda) = 0 \Longrightarrow
 \Phi (w) \equiv 0]$, does it follow that $\mathscr{H}_\Lambda
(\Omega) = \mathscr{H}(\Omega)$, or again $\varphi (\zeta) \equiv 0 ?$
one can get the answer using Runge's theorem but we prefer to deduce
it from a relation between $\Phi$ and $\varphi$, which is interesting
in itself. 

We have the following equations: 
\begin{align*}
 \frac{1}{2 \pi i } \int_C \varphi (\zeta) e^{w \zeta d \zeta} 
 & =\frac{1}{2 \pi i} \int\limits_\Omega \int_C \frac{e^{w \zeta}}{z -
 \zeta} d \zeta d \mu (z) = \int e^{wz} d \mu (z)\\ 
 \Phi (w) & = \frac{1}{2 \pi i} \int_C \varphi (\zeta) e^{w \zeta} d
 \zeta. \tag{1}\label{chap8:sec3:eq1} 
\end{align*}

If $\varphi (z) = \dfrac{1}{z - \zeta}$, we have $\Phi (w) e^{\zeta
 w}$; but when Re $(z - \zeta) \ge 0, \dfrac{1}{z - \zeta} =
\int^\infty_0 e^{-u(z - \zeta)} d		u$. Heuristically we
can have a formula, reciprocal to formula (\ref{chap8:sec3:eq1}) in the
following form 
\begin{align*}
 \int^\infty_0 \Phi (u) e^{-uz} du &= \varphi_o (z)
 \tag{2}\label{chap8:sec3:eq2}\\ 
 \varphi_o (z) &= \varphi (z) \tag{3}\label{chap8:sec3:eq3}
\end{align*}

For the proof of (\ref{chap8:sec3:eq3}), let $\varphi (z)$ be regular
outside a compact 
set $K \subset \Omega$ and vanishing at infinity. Then
(\ref{chap8:sec3:eq1}) gives  
$$
\varphi (z) = \sum^\infty_{n = 0} \frac{a_n}{z{n +1}}, |a_n| < R^n
\Longrightarrow \Phi (w) = \sum \frac{a_n}{n!} w^n 
$$
and therefore $|\Phi (w)| < e^{R |w|}$. Conversely let $|\Phi (w)| <
e^{R |w|}$. Then (\ref{chap8:sec3:eq2}) has a meaning for Re $z > R$
and it is easily seen that 
$$
\int^\infty_0 \sum \frac{a_n}{n !} u^n e^{-uz} du = \sum
\frac{a_n}{z^{n + 1}} = \varphi_o (z) 
$$

\begin{defi*}%defi 0
 $\varphi_o (z) = \int^\infty_0 e^{-uz} \Phi (u) du $ is defined to
 be the {\em Laplace\pageoriginale transform} of the entire function $\Phi (w)$ of
 exponential type. 
\end{defi*}

Now if $\Phi (w) \equiv 0$, then $\varphi (z) \equiv 0$ and thus we
have the following closure theorem. 

\begin{theorem*}%the 
 $\mathscr{H}_\Lambda (\Omega) = \mathscr{H}(\Omega)
 \Longleftrightarrow [\Phi (w) = \int e^{wz} d \mu (z)]$, 

 \noindent
 $d \mu \in \mathscr{H}' (\Omega), \Phi (\Lambda) = 0 \Longrightarrow \Phi \equiv 0$.
\end{theorem*}

Thus the problem of closure is related to the problem of the
distribution of the zeros of an entire function of exponential type. 

\begin{defi*}
 Let $\Phi (w)$ be an entire function of exponential {\em type}. The
 {\em type} of $\Phi$ is the lower bound of $\tau$ such that $\Phi
 (w) = 0 (e^{\tau | w| })$. The {\em type } $ h (\theta)$ of $\Phi$ in
 the direction $\theta$ is defined by : 
 $$
 h(\theta) = \lim \sup\limits_{r \to \infty} \frac{\log \Phi (r e^i)}{r}
 $$
\end{defi*}
we shall see in the next lecture how the formula
(\ref{chap8:sec3:eq1}) helps us to find $h(\theta)$. 
