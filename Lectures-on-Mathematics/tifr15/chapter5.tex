\chapter[Harmonic synthesis for mean-periodic functions...]{Harmonic synthesis for mean-periodic functions
 on the real line}\label{chap5}%Lect 5 

\section{Solutions of the problem of harmonic
  synthesis}\label{chap5:sec1}%sec 1. 

We\pageoriginale give the solution of the problem of harmonic
synthesis by proving 
that if the spectrum of $f$ is void then $f$ is identically zero. 

Suppose that $(1) f * d \mu = 0$ and the spectrum of is void. First we
prove that $(1') f * xd \mu =0$. We shall suppose that $d \mu = pdx$
where $p$ is sufficiently differentiable (say, $N$-times
differentiable). This can be done by convoluting $d \mu$, if
necessary, with a suitable differentiable function. In this case, we
have 
$$
\mathscr{C} (d \mu) = M(w) = 0\left(\frac{1}{_uN}\right) \text{ for some }
N \geq 2. 
$$

The corollary of a theorem in $2$ of lecture $IV$, gives us the following lemma:
\begin{lemma*}
 Suppose $M(\lambda ) = 0$ and $F(w)$ has no pole at $\lambda$. Then
 $M(w)/ (w - \lambda ) = \mathscr{C} (p_\lambda )$, with $f *
 p_\lambda = 0$ and segment of support of $p_\lambda \subset$ segment
 of support of $d \mu$. 
\end{lemma*}

Now by the theorem of Hadamard, we have
\begin{align*}
 M(w) &= \mathscr{C}(p) = K e^{a w} w^k \prod \left(1- \frac{w}{\lambda_n}
 \right) e^{\frac{w}{\lambda n}} \text{ and } \sum \frac{1}{| \lambda n|^2}
 \\ 
 \frac{M' (w)}{M(w)} & = a + \frac{k}{w} + \sum \left(\frac{1}{w - \lambda}
 + \frac{1}{\lambda n}\right) 
\end{align*}

Since\pageoriginale $M'(w) = \mathscr{C} (-ixp)$ and $M(w) / (w - \lambda) =
\mathscr{C}(p_\lambda)$, we have 
\begin{equation}
 \mathscr{C} (- ixp ) = a \mathscr{C} (p) + k \mathscr{C} (p_0)
 +\sum_n (\mathscr{C}(p_{ \lambda n}) + \frac{1}{\lambda n}
 \mathscr{C} (p)). \tag{*} 
\end{equation}

From the equation $(*)$ we want to have the equation
\begin{equation}
 - ixp = a ~ p+ k ~ p_0 + \sum_n (p_{\lambda_n} + \frac{1}{\lambda_n} p) \tag{**}
\end{equation}
and the equation
\begin{equation}
 f * (-ixp) = a ~ f * p + k ~ f * p_0 + \sum_n (f * p_{\lambda_n} +
 \frac{1}{\lambda_n} f * p) = 0. \tag{***} 
\end{equation}

To pass from $(*)$ to $(**)$, it is sufficient to have convergence in
$L_1$ for the summation in $(*)$ and to pass form $(**)$ to $(***)$
it is sufficient to have uniform convergence in $(**)$, since each
term in it has support in the same interval. 

We write
\begin{align*}
X_n (w) & = \frac{M(w)}{w - \lambda_n} + \frac{M(w)}{\lambda n} =
\frac{w M(w)}{\lambda_n (w - \lambda_n)} = \frac{w^2 M(w)}{\lambda_n
 (w - \lambda_n)w} \\
 & \quad \big| w - \lambda_n \big| > \frac{\lambda_n}{2} \Rightarrow
 \big| X_n \big| < 2 \big| w ~ M(w) \big| / \big| \lambda_n
 \big|^2 \tag{1}\label{chap5:sec1eq1} \\
 & \quad \big| \frac{\lambda_n}{2}\big| > \big| w- \lambda_n \big|
 > 1 \Rightarrow \big| X_n \big| < 2 \big| w^2 M (w)
 \big| / \big| \lambda_n \big|^2 \tag{2}\label{chap5:sec1:eq2} \\
\big| w- \lambda \big| < 1 &\Rightarrow \big| \frac{M(w)}{w -
 \lambda_n} \big| \leq | \sup_{|w' - w| = 2} \big| \frac{M(w')}{w' -
 \lambda_n} \big| \leq | \sup_{|w'-w| = 2} \big| M(w') \big| \\
 & \Rightarrow \big| X_n \big| \leq \frac{4 | w |^2}{| \lambda_n
 |^2 } \sup\limits_{\big| w' - w\big| = 2} \big| M(w') \big| for
 \big| \lambda_n \big| > x_0. \tag{3}\label{chap5:sec1:eq3} 
\end{align*}

Since $\big| M (w) < K_i e^{a |v|} / |w|^N \big|$ for some $N$,
we have, in each of the above three cases $\big| X (u) \big| <
K/ \big| u \big|^{N-2} \big| \lambda_n \big|^2$. Now we have a
uniform majorization of each term in the summation of (*) so that
the sum\pageoriginale is absolutely convergent and we also have $\Big| P_{
 \lambda_n} + P/{\lambda_n}\Big| = 0 (\dfrac{1}{| \lambda_n|^2}
2)$. Thus we have $(***)$, i.e., $f * xp = 0$. By iterating this
process, we have $f * h (x) p(x) = 0$ for every polynomial
$h(x)$. This implies that $\int f (y - x) e^{iux} p (x) dx = 0$, since
on the support of $p(x)$ one can approach $e^{iux}$ uniformly by the
polynomials $h(x)$. In other words $\mathscr{C} (f(y -x) p (x)) = 0$
for every $y$. By the uniqueness theorem on Fourier transform, we have
$f(y - x) p (x) = 0$ for every $y$. Thus $f(x) = 0$, which gives the 
\begin{lemma*}
 If the spectrum of $f$ is void, $f \equiv 0$.
\end{lemma*}

In the last lecture we have seen that this lemma gives us the
following theorem: 
\begin{theorem*}
 Suppose $\tau (f) \neq \mathscr{C}$. $f$ belongs to the closed
 subspaces span\-ned by the ``polynomial exponentials'' in $\tau
 (f)$. Thus $\tau (f) $ is the closed span of ``polynomial
 exponentials'' contained in it. 
\end{theorem*}

In this theorem we have a solution of the problem of harmonic
analysis and synthesis of mean periodic functions. 

\section[Equivalence of all the definitions of...]{Equivalence of all the 
definitions of mean periodic functions}\label{chap5:sec2}%sec 2.

The above theorem enables us to prove the equivalence of our
definitions stated in the introduction. Using the condition of Riesz
we have already proved the equivalence of definition I and III
(lecture 4, \ref{chap4:sec1}). We have seen in the introduction that definition
Implies definition I. Our last theorem has just gives us the result
that definition III implies definition II. 
\begin{theorem*}
 Definitions I, II, III, of the lecture 1are equivalent.
\end{theorem*}

The\pageoriginale equivalence of definitions I and III allows us to define the mean
period of a mean periodic function. 
\begin{defi*}
 The mean period of a mean periodic function is defined as the
 infimum of the lengths of the segment of support of measures $d \mu$
 orthogonal to $\tau(f)$. 
\end{defi*}

\begin{remark*}
 This definition implies that if $f = 0$ on a segment $(a, a+ \ell +
 \varepsilon)$, $\varepsilon > 0$, then $f \equiv 0$. Let $\ell$ be
 the mean-period of $f$, and suppose $f =0$ on $(0, \ell +
 \varepsilon )$: then $g^+ \equiv 0$, and hence $F \equiv 0$, and so
 $f \equiv 0$. This means that a mean-periodic function cannot be
 zero on any interval of length larger than its mean-period, if it is
 not the zero-function. We shall see better results of this type
 (lecture 9). 
\end{remark*}

In the next lecture, using definition II, we shall develop various
equivalent forms of this definition. 

\section[Mean periodic \texorpdfstring{$C^\infty$}{Cinfty} - functions and...]{Mean periodic \texorpdfstring{$C^\infty$}{Cinfty} - functions and mean-periodic
  distributions}\label{chap5:sec3} %Sec 3

Just as one defines mean periodic functions by the intrinsic property
$\tau (f) \neq \mathscr{C}$, for $f \in \mathscr{C}$, one can
define mean periodic $C^\infty$- functions $f$ or distributions $T$ by
$\tau (f) \neq \mathscr{E}$, $f \in \mathscr{E}$ or $\tau (T) \neq
\mathscr{D}'$, $T \in \mathscr{D}'$. (Here $\tau (f)$) is the span of
the translates of $f$, \textit{in the space considered}; for example,
if $f \in \mathscr{E}$, ``$\tau(f)$'' in $\mathscr{C}$ is not ``$
\tau (f)$'' in ``$\mathscr{E}$'', which again is not ``$\tau (f)$ in $
\mathscr{D}$'', but, as $\mathscr{E} \subset \mathscr{C} \subset
\mathscr{D}'$, ``$\tau (f)$ in $\mathscr{E}$'' is dense in ``$\tau (f)$
in $\mathscr{C}$'', which is dense in ``$\tau (f)$ in $\mathscr{D}$''. We
can have definitions similar to definitions $I$ and $II$, by replacing
$d\mu$ with $T \in \mathscr{E}'$ when $ f \in \mathscr{E}$ and with
$\varphi \in \mathscr{D}$ when $T \in \mathscr{D}'$. It is possible to
develop the whole theorem in particular the equivalence of the
definitions, by considering harmonic\pageoriginale analysis and synthesis. In
obtaining the criterion for simple subspaces the same reasoning
applies, with $T \in \mathscr{E}'$ or $\varphi \in \mathscr{D}$
playing the role of $d \mu$. The proof of the theorem on synthesis
will-depend on the lemma in $1$. The same proof holds if one replaces
$M(w)$ by $\mathscr{C}(T)$ with $T \in \mathscr{E}'$ or $\mathscr{C}
(\varphi)$ with $\varphi \in \mathscr{D}$. In either case we get
$\mathscr{C} (f_y (-x) p(x))\equiv 0 (\mathscr{C}(T_y p(x)) \equiv 0)$
for every $y$, which gives us by the uniqueness theorem that $f \equiv
0 (T \equiv 0)$. 

If a mean periodic function (distribution) is also a $C^\infty$ -
function, it is a mean periodic $C^\infty$- function. In other words
$M.P. \mathscr{E} \equiv (M.P. \mathscr{C}) \cap \mathscr{E} =
(M.P. \mathscr{D}) \cap \mathscr{E}$. For if $f \in \mathscr{E}$ be
such that $f$ is a mean periodic distribution then there exists
$\varphi \in \mathscr{D}$ with $f * \varphi = 0$. But $\varphi$ is
again a distribution with compact support; so $f \in
M.P.\mathscr{E}$. In the same way, if $f \in (M.P.\mathscr{C}) \cap
\mathscr{E}$, $f \in M.P. \mathscr{E}$. Without confusion we can say
that a function is mean periodic and a $C^{\infty}$- function or a
mean periodic $C^{\infty}$-function. We have also $M.P. \mathscr{C} =
(M.P.\mathscr{D}) \cap \mathscr{C}$. 

\section{Other extensions}\label{chap5:sec4}%Sec 4

By the method we used in lecture \ref{chap4} and \ref{chap5}, it is possible to
develop the theory of harmonic analysis and synthesis for functions
which are ``mean-periodic on a half-line'', i.e., continuous functions
on [$0, \infty$] such that their negative translates $(f_\alpha (x) =
f (x - \alpha), \alpha < 0)$, restricted to [$0, \infty$], are not a
total set in the space of the continuous functions on [$0, \infty$],
with the topology of compact convergence (see Koosis (2)). 

Another\pageoriginale method seems to be necessary in order to study mean-periodic
functions on $R^n$ (see Malgrange, Ehrenpreis, and lect. \ref{chap23}
and \ref{chap24}) 
or mean-periodic functions (sequence) or $Z^n$ (whose theory was
given, quite recently, by Lefranc). 
