\chapter[Bounded mean periodic functions and their...]{Bounded mean periodic functions and their
 connection with almost periodic functions}\label{chap7}%Lect 7 

Let\pageoriginale $f$ be a bounded mean periodic function, i.e., $|f|<M, f \in
\mathscr{C}, \tau(f) \neq \mathscr{C}$. In this case the
Carleman transform is $F(w) = \int\limits_{-\infty}^0 f (x)
e^{-ixw}dx, w = u + iv, v > 0$, and $F(w) = -\int_0^\infty
f(x)e^{-ixw} dx, v< 0.$ As $\int\limits_{-\infty}^0 e^{vx}dx =
\dfrac{1}{v}$ for $v>0$, and $\int\limits_C^{-\infty} e^{vx}dx=\dfrac{1}{v}$
for $ v<0$, we have in both cases $|F(w)|< \dfrac{M}{|V|}$. This
implies the spectrum is real. Taking the polar development of $F (w)$
in $\lambda$ (that gives the principal part when $w$ tends to
$\lambda$), we see $1^{0}$) the spectrum is simple $2^{0}$) the
Fourier coefficients $A(\lambda)$ are bounded: $|A(\lambda)| <
M$. Thus we have proved: 
\begin{theorem*}%Thm
 A bounded periodic function has its spectrum real and simple and its
 Fourier coefficients are bounded. 
\end{theorem*}

Suppose $|f|<M, f''$ continuous and $|f''|<M''; f''$ is again
mean-periodic, and $f \sim \sum A (\lambda)e^{i \lambda x}\Rightarrow
f'' \sim - \sum A(\lambda)\lambda^2 e^{i \lambda x}$ and $f''$ is again
mean periodic. Thus $|\lambda^2 A(\lambda)| < M''$. Since $\lambda \in
S(f)$ are the zeros of an entire function of exponential type, $\sum
\dfrac{1}{|\lambda^2|}< \infty$. Thus the Fourier series of $f$, being
majorized by $\sum \dfrac{1}{|\lambda^2|}$, is absolutely convergent
and so $f$ is an almost periodic function. 

We recall the various definitions of almost periodic functions and
find their connection with bounded mean periodic functions. 

\medskip
\textbf{\textit{The definition of Bohr.}}\pageoriginale

We consider functions $f \in \mathscr{C} (R)$.
\begin{defi*}[(Bohr)]%Def
 Given $\varepsilon > 0$, the number $\tau$ is an
 \textit{almost period} (corresponding to $\varepsilon$) of $f$ if
 $\big|f_{\tau-f}\big| < \varepsilon$. A set is
 \textit{relatively dense} if there exists a length $\ell > 0$, such
 that in any interval of length $\ell$, there is at least one point
 of the set. 
\end{defi*}

A function is \textit{almost periodic} (in the sense of Bohr) if 

(a)~ for every $\varepsilon > 0$, the almost periods of $f$ form a
relatively dense set. This is equivalent to either of the following: 

(b) ~ The translates of $f$ form a relatively compact set of $L$.

(c) ~ $f$ is the uniform limit in $R$ of a sequence of trigonometric
polynomials $\sum a_n e^{i \lambda_n x}, \lambda_n$ real (see for
example (Besicovitch)). 
 
 \medskip
 \noindent
 \textit{Properties.}
\begin{enumerate}[(1)]
\item $a(\lambda ) = \lim\limits_{T \to \infty}
 \dfrac{1}{2T}\int\limits_{-T}^{T} f(x) e^{-i \lambda x} dx$ exists. 
\item $\sum \Big| a(\lambda) \Big|^2 = \lim\limits_{T \to
 \infty}\dfrac{1}{2T}\int\limits_{-T}^{T}\Big| f(x) \Big|^2 dx$
 (Perceval's relation). 
\end{enumerate} 
 
 Here only a countable number of a $(\lambda )\neq 0$.
 
 In order to find the analogue of the Riesz-Fischer theorem Besicovitch
 introduced the following norm. (Besicovitch). 
 $$
 D(f)=\lim\limits_{T \to \infty} \sup\dfrac{1}{2T}\int_{-T}^T|f(x)|^2 dx, D(f, z)=D(f-z).
 $$
 It may happen that $D(f)=0$ without $f=0$.
 
\begin{defi*}[of Besicovitch] 
  A\pageoriginale  bounded function $f \in
 \mathscr{C} (R)$ is \textit{almost periodic} (in the sense of
 Besicovitch) if it belongs to the closed subspace spanned by $\{e^{i
 \lambda x}\}, \lambda \in R$, the closure being taken in the metric
 $D(f)$. Then $f \sim \sum a(\lambda) e^{i \lambda x}$. 
\end{defi*}

 Besicovitch showed that if $\sum \big | a (\lambda) \big |^2 <
 \infty$, then there exists a function $f \sim \sum a (\lambda) e^{i
 \lambda x}$. 
 
\begin{defi*}[of Schwartz] 
  Consider the space $\mathscr{B}$ of
  $C^\infty$- functions all of whose derivatives are bounded. (Schwartz
  $3$). A function $f \in \mathscr{B}$ is $\mathscr{B}$- \textit{almost
    periodic} if the set of its translates from a relatively compact set
  in $\mathscr{B}$. These functions $f \in \mathscr{B}$ are such that
  they are almost periodic in the sense of Bohr and all their
  derivatives $f^{(p)}$ are also almost periodic in the sense of Bohr. 
\end{defi*}

Let $\mathscr{B}'$ be the dual of $\mathscr{B}$. $\mathscr{B}'$ is the
space of distributions which are finite sums of derivatives ( in the
sense of distributions) of bounded functions. A distribution $T \in
\mathscr{B}'$ is defined to be $\mathscr{B}'$- almost periodic if it
satisfies either the definition $(b)$ or $(c)$ of Bohr in the space
$\mathscr{B}'$. 
 There is a simple connection between the classes $M.P.,$ $\mathscr{B}
 A.P.$ and $\mathscr{B}' A.P.$ which consist respectively of
 mean-periodic distributions, $\mathscr{B}$- almost periodic functions
 and $\mathscr{B}'$ - almost periodic distributions. 
 \begin{theorem*}
 $M.P. \cap \mathscr{B} = M.P. \cap \mathscr{B}.A.P$. and $M.P. \cap
 \mathscr{B}' = M.P. \cap \mathscr{B}' A.P.$. 
 \end{theorem*} 

The first part results from the fact that $f$ is almost periodic\pageoriginale
(Bohr) whenever $f \in \mathscr{C}, f$ and $f"$ bounded; moreover, we
see that $f \in M.P \cap \mathscr{B} \Rightarrow f \sim \sum
a(\lambda) e^{i \lambda x}$, $a(\lambda) = 0 (\big | \lambda \big
|^{-n})$ for every $n > 0$. The second part results from: 
$$
f \in M.P. \cap \mathscr{B}' \Rightarrow f \sim \sum a (\lambda) e^{i
 \lambda x}, a (\lambda) = 0 (\big | \lambda \big |^N) \text { for
 one} N. 
$$

The corresponding result for Bohr almost periodic functions is the following:

\begin{theorem*}
 A uniformly continuous bounded mean periodic function is almost
 periodic (in the sense of Bohr). 
\end{theorem*}

We have seen that if $f$ has a bounded second derivative, $f$ is
almost periodic. In other cases we regularize $f$ with the help of
suitable functions. Let $\triangle_ \varepsilon$ be the conical
function defined by $\triangle_ \varepsilon (x) = \sup (0,
\dfrac{\varepsilon - | x |}{\varepsilon^2})$. $\triangle_ \varepsilon
(x)$ has its support in $(- \varepsilon, \varepsilon)$ and $\int^
\varepsilon_{- \varepsilon} \triangle_ \varepsilon (x) dx = 1$. Then
$f * \triangle_ \varepsilon * \triangle_ \varepsilon$ is a bounded
mean periodic function having a bounded second derivative and so
almost periodic. Since $f$ is uniformly continuous, $f * \triangle_
\varepsilon * \triangle_ \varepsilon \to f$ uniformly when
$\varepsilon \to 0$. Hence by the definition $(c)$ of Bohr $f$ is
almost periodic in the sense of Bohr. 

A natural question is to ask whether every bounded mean periodic
function is uniformly continuous. In fact, that this is not true in
general is seen from the following example. 

We take the Fejer Kernel $K_\nu (x) = (\nu \sin^2 \dfrac{\nu x}{2} /
(\dfrac{\nu x}{2})^2)$. It is possible to choose a increasing sequence
$\mu_n$ such that the function $\sum\limits_{n = 1}^\infty
\dfrac{1}{\mu_n} K_{\mu_n} (\dfrac{x}{2^n} - \pi)$ is bounded, and the sum
uniformly continuous\pageoriginale on each compact set. The spectrum of the Fejer
Kernel $K_{\mu_n} (\dfrac{x}{2^n})$ consists of $(2 \mu_n - 1)$ points
between $- \mu_n/2^n$ and $\mu_n / 2^n$.The spectrum of $e^{i \lambda
 n^x} K_{\mu_n} (\dfrac{x}{2^n} - \pi)$ consists of $(2 \mu_n - 1)$
points between $\lambda_n- \dfrac{\mu_n}{2^n}$ and $\lambda_n +
\dfrac{\mu_n}{n}$. Now it is possible to choose $\lambda_n$ satisfying
the following conditions: 
\begin{enumerate}[(1)]
\item Each term of the series $\sum\limits_{1}^{\infty} e^{i \lambda_n
 x}\dfrac{1}{\mu_n} K_{\mu_n} (\dfrac{x}{2^n} - \pi)$ consists of
 function whose spectra do not overlap. 
\item Denote by $\{v_n \}$ the spectrum of this sum, $\sum
 \dfrac{1}{v_n} < \infty$ 
\end{enumerate}

Then the function $f(x) = \sum\limits_{1}^{\infty} e^{i \lambda_n x}
\dfrac{1}{\mu_n} K_{\mu_n} \left(\dfrac{x}{2^n} - \pi\right)$ is not uniformly
continuous in $R$, but is bounded and (see theorem), lect. 6, \ref{chap6:sec1})
has mean-period zero. 

This example show that there are bounded mean periodic functions which
are not almost periodic in the sense of Bohr. Now one may ask for the
relation between bounded mean periodic functions and almost periodic
functions of Besicovitch. 

We shall prove that, if $f$ is a bounded mean periodic function then
$\sum \big|\Lambda (\lambda)\big|^2 < \infty$. In other words, its
Fourier coefficients are those of an almost periodic function in the
sense of Besicovitch. 

Let $f \varepsilon = f * \Delta_{\varepsilon }, f \sim \sum A
(\lambda) e^{i \lambda x}$.

Now $\mathscr{C} (\Delta_{\varepsilon}) = \dfrac{\sin^2
 \frac{\varepsilon x}{2}}{\left(\frac{\varepsilon x}{2}\right)^2} = \delta
_{\varepsilon} (x) ; f \sim \sum A (\lambda) \delta
_{\varepsilon} (\lambda ) e^{i \lambda x}$. 

Since $|\Lambda (\lambda)| \leq M$ and $\sum \dfrac{1}{|\lambda|^2} <
\infty$, we have 
$$
\sum |A (\lambda) \delta_{\varepsilon} (\lambda)|^2 = \lim_{T
 \to \infty} \frac{1}{2T} \int^{T}_{-T} |f_{\varepsilon}|^2 \leq
M^2. 
$$

When $\varepsilon \to 0$, $\delta_{\varepsilon} (\lambda)\to 1$ and so 

\begin{theorem*}
 If\pageoriginale $f$ is a bounded mean-periodic function,
 $$
 f \sim \sum A (\lambda) e^{i \lambda x}, |f| \le M, \text { then }
 \sum | A (\lambda) |^2 \le M^2 
 $$ 
  
 One has also the summation formula of Fejer, viz., 
 $$
 \sum_{|\lambda| < T} \left(1 - \frac{|\lambda|}{T}\right) A (\lambda) e^{i
 \lambda x} = f \ast \frac{1}{2 \pi } \frac{T \sin^{
  2\frac{Tx}{2}}}{\bigg(\frac{Tx}{2}\bigg)^2} \to f 
 $$
 uniformly if $f$ is uniformly continuous and other wise it is
 uniformly convergent on every compact set. 
\end{theorem*}

Instead of $\mathscr{C}$, let us consider the space $E^2$, consisting
of the functions which are locally $\in L^2. E^2$- mean-periodic
functions are defined, as usual, by the condition $\tau (f) \neq E^2$
 ($\tau (f)$ closed \textit{in $E^2$}). $E^2-$ bounded functions are
defined by $\int^{x + 1}_{x} |f|^2 = 0 (1)$ uniformly with respect to
$x$. 

The problem is to study the functions which are $E^2$-mean-periodic
and $E^2$-bounded. In particular $1^\circ)$ is it possible to get the
Parceval and Riesz-Fischer theorems ? $2^\circ)$ is it possible to
have the summation formula of Fej\'er? $3^\circ$ what are their
connections with Besicovitch almost-periodic functions and with
Stepanoff almost-periodic functions (which are analogous to Bohr
a.p. functions, with the norm $\sup\limits_{x} (\int^{x + 1}_{x}
|f|^2)^{\dfrac{1}{2}}$ instead of $\sup\limits_{x} |f (x) |$)? For
more details about this type of questions, see (Kahane, $2$). 
