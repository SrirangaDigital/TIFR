\chapter{New Quasi-analytic classes of functions}\label{chap20}%CHAP 20

We\pageoriginale study quasi-analyticity of $\mathscr{C}_\Lambda$ with conditions
involving $\Lambda$. First we recall a result of $S$. Mandelbrojt
about periodic functions (Mandelbrojt $1$). 

\begin{equation*}
  \left.
  \begin{aligned}
    f \sim \sum (a_n \cos \lambda_n x + b_n \sin \lambda_n x)&\quad \\
    \sum {1/ \lambda^\sigma_n} < \infty, 0 < \sigma < 1&\\
    \int^\alpha_o | f | < K e^{-(\alpha^{- \rho})}, \alpha <
    \alpha_o, \rho > \frac{\sigma}{1-\sigma} &
  \end{aligned}
  \right \}
  \Rightarrow f \equiv 0
\end{equation*}

This signifies that if the spectrum is very lacunary, then it is not
possible to have $\int^\alpha_o \cdots$ very small. This leads us to
find conditions involving $\{\lambda_n\}$ and $I (\alpha)$ such that
$\int^\alpha_o | f| < I (\alpha) \Rightarrow f \equiv 0$. Such
functions form a class $I (\alpha)$. 

\begin{defi*}%defi 0
 Given a positive function $I (\alpha) ~ (\alpha > 0, I (\alpha)
 \nearrow)$, a class of $C^\infty$ - functions is defined to be a $I
 (\alpha)$ quasi-analytic class when the only function of the class
 which satisfies the conditions $\int^\alpha_o | f| < I (\alpha)
 (\alpha \to 0)$ is the zero function. 
 
 There is a connection between $I(\alpha)$ quasi-analyticity and $D$
 quasi-analyticity. Indeed if $f \in C_I \{M_n\}, 0 \in I$ and
 $f^{(n)} (0) = 0$, for $n = 0, 1, \ldots, $ then by Taylor's
 formula 
 $$
 f(x) = \frac{x^n}{n !} f^{(n)} (\theta x),
 $$
 $0 \le \theta \le 1$ and so $\big | f (x) \big | \le \dfrac{x^n}{n
 !} M_n$ which gives that $\int^\alpha_o \big | f(x) \big | dx <
 \min\limits_{n} \dfrac{\alpha^{n+1}}{(n+1)!} ! M_n$. So
 quasi-analyticity $I (\alpha)$ implies quasi-analyticity $D$, when
 $I (\alpha) = \min\limits_{n}
 \left\{\dfrac{\alpha^{n+1}M_n}{(n+1)!}\right\}$. This function is
 similar to the function $T(r)$ introduced\pageoriginale 
 in the last lecture, but it seems not possible to obtain the
 Denjoy-Carleman theorem by considering $I(\alpha)$. 
\end{defi*}

Our problem will be to define a relation between $\Lambda$ and $I
(\alpha)$ such that $\mathscr{E}_\Lambda$ is an $I (\alpha)$
quasi-analytic class. First we formulate the method we shall use in
this lecture (other methods will be explained in the next
one). Suppose $f \in \mathscr{E}_\Lambda \neq \mathscr{E}$ and the
mean period corresponding to $\Lambda$ is zero; for convenience,
suppose $0 \notin \Lambda$. Then for every $\alpha > 0$ it is possible
to find a measure $d \mu_\alpha$ with support in $\big [- \alpha/ 2,
 \alpha/2 \big ]$ such that $f * d \mu_\alpha = 0$, and we can assume
the conditions: $d \mu_\alpha = \mu'_\alpha dx, \mu'_\alpha \in
L^\infty$, and $\int d \mu_\alpha = M_\alpha (0) = 1$. Let $g = - f *
d \mu_\alpha$. Then (notations of Lecture $4) G_\alpha (w) = F(w)
M_\alpha (w)$ and $G_\alpha (0) = F(0)$. Suppose we have
$\int^\alpha_o | f (x) | < I (\alpha)$. Then $|| g_x||_ \infty < I
(\alpha) || \mu'_\alpha ||_\infty$. As $\big | G_\alpha (0) \big |
\le \alpha || g_\alpha ||$, we have $\big | F(0) \big | < \alpha I
(\alpha) || \mu'_\alpha ||_\infty$. Suppose that, for an infinity of
$\alpha \to 0$, we can choose $d \mu_\alpha$ in such a manner that
$\alpha I (\alpha) || \mu'_\alpha ||_\infty \to 0$. Then $F(0) \equiv
0$. This being true for any $f \in \mathscr{C}_\Lambda$, we take
primitives of $f$ instead of $f (x)$ and $F'(0) = 0$ etc. Thus $F
\equiv 0$ and $f \equiv 0$. Thus we are able to formulate our
condition as follows: 

Suppose that to each $\alpha > 0$ we associate $\mu'_\alpha \in
L^\infty$ with support in $\big [ - \dfrac{\alpha}{2}, 
 \dfrac{\alpha}{2} \big]$ such that $\int \mu'_\alpha = 1$ and $\int
e^{i \lambda x} \mu'_\alpha dx = 0$ for every $\lambda \in
\Lambda$. If $\lim \inf_{\alpha \to 0} \alpha I (\alpha) ||
\mu'_\alpha ||_\infty = 0$, then $\mathscr{C}_\Lambda$ is an $I
(\alpha)$ quasi-analytic class. 

We shall use this condition in the following form.
\begin{lemma*}
 Suppose\pageoriginale that to each $\alpha > 0$ we associate an entire function
 $M_\alpha (w)$ of exponential type $\le \dfrac{\alpha}{2}$, such
 that $M_\alpha (u) \in L^1, M_\alpha (0) = 1,\break M_\alpha (\Lambda) = 0$. If
 $\underset{\alpha \to o}{\lim \inf} \alpha I (\alpha) \int |
 M_\alpha (u) | = 0$, then $\mathscr{C}_\Lambda$ is an $I (\alpha)$-
 quasi-analytic class. 
\end{lemma*}

We apply the above condition in the case when $\Lambda$ is a real
symmetric sequence, $\Lambda = \{ \pm \lambda_n\}$ such that $\sum
\dfrac{1}{\lambda_n} < \infty$. To construct the function $M_\alpha
(w)$ we take the canonical product $C(w) = \prod\limits_{1}^\infty
\left(1 
- \dfrac{w^2}{\lambda^2_n}\right)$ and take $M_\alpha (w) = C(w)
\prod\limits_{o}^\infty \left(\dfrac{\sin \alpha_j w}{\alpha_j^w}\right)$ with $4
\sum \alpha_j = \alpha$. we take this additional factor since $C(w)$
does not behave well real axis. $M_\alpha (w)$ is a function of
exponential type $\dfrac{\alpha}{2}$, since $C(w)$ is of type $0$. Now
we have to construct $\alpha_n$ in such a manner that $M_\alpha (u)
\in L^1$. To do this we first try to majorise $M_\alpha (w)$. Suppose
$\lambda_n \le u \le \lambda_{n+1}$. We have the following
calculations: 
\begin{align*}
 \big | M_\alpha (u) \big | & = \prod^n_1 \left(\frac{u^2}{\lambda^2_m}-1\right)
 \prod^\infty_{n+1} \left(1- \frac{u^2}{\lambda^2_m}\right) \prod^\infty_o
 \left(\frac{\sin \alpha_j u}{\alpha_j u}\right)^2\\ 
 & < \frac{u^{2n}}{\lambda^2_1 \cdots \lambda^2_n} ~
 \frac{1}{\alpha^2_1 \cdots \alpha^2_n u^{2n}} \min \left(1,
 \frac{1}{\alpha^2_o u^2}\right)
\end{align*}
(we already used such a majorization, for $\prod\limits^{\infty}_o ~
\dfrac{\sin \alpha_j u}{\alpha_j u}$, in Lect. 19, \S \ref{chap19:sec2}). 
$$
\int^\infty_o \big | M_\alpha (u) \big | du = \int^1_o + \int^\infty_1
<\left(1+ \frac{1}{\alpha^2_o}\right) (\max\limits_{n} (\lambda_1 \alpha_1, 
\ldots, \lambda_n \alpha_n ))^{-2}. 
$$

This majorization is not useful if $\sum \dfrac{1}{\lambda_n} =
\infty$, because the second member is $\infty$ (if not, we would have
$(\alpha_1 \cdots \alpha_n)^{1/n} > \dfrac{K}{(\lambda_1 \cdots
 \lambda_n)}^{1/n} > \dfrac{K}{\lambda_n}$, and the equiconvergence of $\sum
\alpha_n$ and $\sum (\alpha_1 \cdots \alpha_n)^{1/n}$, which we stated
in Lect. 19 \S \ref{chap19:sec2}, leads to a contradiction). To get a result, we
must suppose $\sum \dfrac{1}{\lambda_n} < \infty$. Choose a sequence
$\{l_n\}, l_n \to \infty$, such that $ \sum^\infty_1 \frac{l_n}{\lambda_n}
<\frac{l}{8}$ and take $\alpha_o = \alpha/8, \alpha_j = l_j
\alpha/\lambda_j$. Then $4 \sum^\infty_o \alpha_j < \infty$\pageoriginale and
$\max\limits_{n} (\lambda_1 \alpha_1 \cdots$ $\lambda_n \alpha_n) =
\max\limits_n (l_1 \cdots l_n \alpha^n)$. This is finite since $l_n
\to \infty$. (The expression $(\max\limits_n (l_1 \cdots l_n \alpha^n
))^{-2}$ is of the same form as $\max r^n/M_n$ which we have seen
already). Thus we have 
$$
\int^\infty_ {- \infty} | M_\alpha | < \const. (\max_{n}(l_1
\cdots l_n \alpha^n))^{-2} \alpha^{-2}; 
$$
$M_\alpha (w) = 0$ on $\Lambda $ and $M_\alpha (w)$ is of type $\le
\alpha$. Now the condition $\underset{\alpha \to 0}{\lim \inf}$ $\alpha
I(\alpha) \int M_\alpha (u) = 0$ follows from $\underset{\alpha \to
 0}{\lim \inf} I (\alpha) / \alpha \min\limits_{n} (l_1 \cdots l_n
\alpha^n)^2 = 0$. By changing the first $l_j$ if necessary and by
replacing $l_j$ by $k l_j (k > 1)$ for sufficiently large $j$ we have
the following condition for $\mathscr{C}_\Lambda$ to be $I (\alpha)$
quasi-analytic. 

\begin{theorem*}
 Suppose $\Lambda = \{\pm \lambda_n\}, 0 < \lambda_1 < \lambda_2
 \cdots \sum\limits_1^\infty \dfrac{
1_n}{\lambda_n}$ with $1_n
 \nearrow \infty$ and $\underset{\alpha \to 0}{\lim \inf} ~
 \dfrac{I (\alpha)}{\alpha \min\limits_n ((l_1 \cdots l_n
  \alpha^n)^{-2}}< \infty$. Then $\mathscr{C}_\Lambda$ is an $I
 (\alpha) $ quasi-analytic class. 
\end{theorem*}

When $\{\lambda_n\} $ is not a sequence of real numbers, but symmetric
and $\sum \dfrac{1}{| \lambda_n |} < \infty$, the same method can be
used with the additional hypothesis that $\big | \lambda_j \big | /
j^{1+ \in \nearrow}$ so as to have a good majorization of
$\prod \big | 1-u^2 / \lambda^2_j \big |$ and in this case $\big |
\prod (1+ \dfrac{u^2}{| \lambda_j |^2 } )\big | < k \max (k_2 u)^{2n}/
| \lambda_1 \cdots \lambda_n |^2$ (see (Kahane 1)). The condition $|
\lambda_j | / j^{1 + \in } \nearrow$ is a condition of
regularity. 

Our condition of $I (\alpha)$-quasi-analyticity gives
$S$. Mandelbrojt's theorem. Suppose $ \sum \dfrac{1}{\lambda^\tau_n} <
\infty, 0 < \sigma < 1$. Take $1_n = \lambda^{1- \sigma}_n$. Thus
$\sum\limits_{1_n > n^{\dfrac{1-\sigma}{\sigma}}} 1/1^{\sigma (1-
 \sigma)} < \infty$. Take $\lambda_n \nearrow$ and so $1_n \nearrow,
1^{\dfrac{\sigma}{1- \sigma}}_n / n\to \infty$ and 
$$
\max_{n} \frac{r^n}{l_1 \cdots l_n} < \max_{n} \frac{(( 
 \frac{\sigma}{r^{1-\sigma}})^n )^{1-\sigma) / \sigma}}{(n !)^{(1 -
 \sigma)/ \sigma}} < e^{\dfrac{(1-\sigma)}{r^o}r^{\sigma/1 -
 \sigma}} 
$$
Taking\pageoriginale $\alpha = 1/ r$ we have quasi-analyticity whenever $I (\alpha)
< \alpha^2 e^{\dfrac{1- \sigma}{\sigma}}$ $\alpha^{\sigma / 1 - \delta}$. 

Again, we can derive a condition of $D$-quasi-analyticity for $C \{
M_n\} \cap \mathscr{C}_\Lambda$. We have seen that $C_I \{M_n\} \cap
\mathscr{C}_\Lambda$ is D-quasi-analytic with $I (\alpha) =
\min\limits_{n} \dfrac{\alpha^{n+1}}{(n+1) !} M_n$. Thus it is
sufficient that $\min\limits_{n} (\alpha^{2 n} M_{2n}/ (2n+ 1)!) <
\min\limits_{n} (l_1 \cdots l_n \alpha^n)^2$ for an infinity of
$\alpha, \alpha \to 0$. This means that the reverse inequality does
not hold for $\alpha > \alpha_o$. In other words, taking $\alpha =
1/r, \max\limits_n (r^{2n} (M_{2n}/ (2 n + 1) ! )^{-1}) \le
\max\limits_n r^{2 n} (l_1 \cdots l_n)^{-2}$ does not hold for $r >
r_o$. Now we use the following lemma. 

\begin{lemma*}
 Let $\{ A_n\}$ and $\{ B_n \}$ be two positive sequences with
 $B_{n+1}/ B_n \nearrow$. Then 
 $$
 \{B_n \le A_n, (n > n_o ) \} \Longleftrightarrow \{\max_n r^n/ B_n
 \ge \max_n r^n/A_n \} (r > r_o). 
 $$
\end{lemma*}

For the proof $Cf$. (Mandelbrojt 3, p.7 and p. 18).

Taking $A_n = \dfrac{M_{2n}}{(2n + 1)!}, B_n = (l_1 \cdots l_n)^2$ our
condition becomes: $\dfrac{M_{2n}}{(2n + 1)!} \ge (l_1 \cdots l_n)^2$
does not hold for $n > n_o$ (whatever we choose $n_o$). Replacing
$M_n$ by $kM_n$, we get: 
\begin{theorem*}%theo 0
 We make the same assumptions about $\Lambda, \lambda_n, 1_n$ as in
 the\break above theorem; $I$ is an arbitrary interval. 
$$\mbox{\rm If } \lim\limits_{n
 \to \infty} \inf \dfrac{M_{2n}}{(2n + 1) ! (1_1 \cdots 1_n)^2}> 0~
 \mathscr{C}_\Lambda \cap C_I \{M_n\}$$ 
is a D-quasi-analytic class. 
\end{theorem*}
