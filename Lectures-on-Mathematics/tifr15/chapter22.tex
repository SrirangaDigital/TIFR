\chapter{Reciprocal theorems about quasi-analyticity \texorpdfstring{$D$}{D}
 and \texorpdfstring{$I(\alpha)$}{Ialpha}}\label{chap22}% Lecture 22 

\markright{\thechapter. Reciprocal theorems about quasi-analyticity D
 and $I(\alpha)$}

In\pageoriginale the Lectures \ref{chap20} and \ref{chap21}, we gave
sufficient conditions in order 
that $\mathscr{C}_\Lambda$ should be an $I(\alpha)$ quasi-analytic
class, or $\mathscr{C}_\Lambda \cap C_I \{M_n \}$ resp
$\mathscr{C}_\Lambda \cap C \{M_n \}$ a $D$-quasi-analytic class. We
stated that if $\Lambda$ is sufficiently lacunary, some of these
conditions are necessary. In order to know whether our sufficiency
conditions are good (i.e., whether it is not possible to relax them
very much), and, if possible, to find necessary and sufficient
conditions, we shall construct a function $f \in \mathscr{C}_\Lambda$
``as small as possible" near the origin. Actually, we want first to
have $f^{(n)}(0) = 0 (n=0,1,\ldots)$ and $\sup_x \{|f^{(n)} (x) | \}$
increasing as slowly as possible. 

We saw in the Lecture \ref{chap21} that the smallness at infinity of $F(w)$,
the Carleman transform of $f$, is related to the smallness of $f$ near
the origin. Therefore, it is natural to take $F(w)$ ``as small as
possible'' at infinity. Suppose $\Lambda = \{\pm \lambda_n \}$ to be
symmetric and real. Then the Carleman transform of every $f \in
\mathscr{C}_\Lambda $ is the product of $F_o (w) = \prod (1 -
\dfrac{w^2}{\lambda^2_n})^{-1}$ by an entire function. With convenient
hypothesis on $\Lambda$ we shall construct $f_o \in
\mathscr{C}_\Lambda$, whose Carleman transform is $F_o(w)$. It is
natural to expect that $f_0$ is the function we want. 

We suppose $\Lambda = \{\pm \lambda_n \}$ to be symmetric, real and
lacunary in the same that $\dfrac{\lambda n+1}{\lambda n} > K > 1$. We
denote by $\sum ( \dfrac{A_k}{w - \lambda_k} - \dfrac{A_k}{+
 \lambda_k})$ the polar part of $R_0(w) = \prod (1 -
\dfrac{w^2}{\lambda^2_j})^{-1}$; then $A_k = - \dfrac{\lambda k}{2}
\prod\limits_{j \neq k} (1 - \dfrac{\lambda_k}{\lambda^2_j})^{-1}$. We\pageoriginale
define $f_0 (x) = 2 \sum A_k \sin \lambda_k ~ x$; indeed, if $\sum |
A_k | < \infty$ (and the following calculation proves that it is
realised), $F_o(w)$ is the Carleman transform of $f_o$. We now try to
get a majorization for $| f^{(n)}_0 (x) |$. 
\begin{align*}
 |f_o^{(n)} | & < 2 \sum^\infty_1 | A_k | \lambda^n_k\\
 | A_k | & = \frac{\lambda_k}{2} \frac{\lambda^2_1 \cdots
 \lambda^2_k}{\lambda_k^{2k}} \prod\limits_{j=1}^{k-1} 
 | \frac{\lambda^2_j}{\lambda^2_k} - 1 |^{-1}
 | \prod\limits^{\infty}_{j=k+1} \left(1-
 | \frac{\lambda^2_k}{\lambda^2_j}\right)^{-1}\\ 
 | A_k | \lambda^{2k}_k & < C \lambda^2_1 \cdots \lambda^2_{k-1}
 | \lambda^3_k\\ 
 | f^{(2n-1)}_o | & < 2C ~ \sum_{k=1}^\infty \lambda^2_1 \cdots
 \lambda^2_k ~ \lambda^{(2(n-k)}_{k}\\ 
 & = 2C ~ \lambda^2_1 \cdots \lambda^2_n ~ \left(\sum^{n-1}_{1} ~
 \frac{\lambda^{2(n-k)}_k}{\lambda^2_{k+1}\cdots \lambda^2_n}\right)
 ~ + 1 + \sum^\infty_{n+1} ~ \frac{\lambda^2_{n+1}\cdots
 | \lambda^2_k}{\lambda^{2(k-n)}_k}\\ 
 & < C_1 ~ \lambda^2_1 \cdots \lambda^2_n
\end{align*}
and $| f^{(2n)}_0 | < C_1 ~ \lambda^2_n \cdots \lambda^2_1 ~
\lambda_{n+1}$ by a similar calculation. We take $M_{2n-1} =
\lambda^2_1 \cdots \lambda^2_n$ and $M_{2n} = \lambda_1^2 \cdots
\lambda_n^2 ~ \lambda_{n+1}$. Then we have $f_o \in C\{ M_n\}$. 

Moreover $f^{(n)}_0 (0) (n = 0,1, \ldots)$. For, if $N$ were the first
integer such that $f^{(N)}_0 (0) \neq 0$, we would have, for $v > 0$. 
$$
F_0(w) = \frac{f_o^{(N)}(0)}{(iw)^N} =
\frac{f_0^{(N+1)}(0)}{(iw)^{N+1}} + \int\limits_{- \infty}^0 ~
f_0^{(N+2)} (x) ~ (ix)^{-N-1} e^{-ixw} {dx} 
$$
and $\lim\limits_{v \rightarrow \infty} ~ (-v)^{N} F(-iv) = f^{(N)}(0)
\neq 0$; since $\prod(1 + \dfrac{v^2}{\lambda^2_k})$ increases more
rapidly than any polynomial, this is impossible. 

In Lecture \ref{chap21}, we found that $\lim\inf\limits_{n \rightarrow \infty}
\dfrac{M_{2n}}{\lambda^2_1 \cdots \lambda^2_n \lambda_{n+1}} = 0$ is a
sufficient condition for the $D$-quasi-analyticity of
$\mathscr{C}_\Lambda \cap C\{ M_n\}$. The properties of $f$ gives us
the following result: 
\setcounter{theorem}{0}
\begin{theorem}\label{chap22:thm1}% theorem 1
 If\pageoriginale $\Lambda = \{ \pm \lambda_n\}$ is real, symmetric and lacunary in
 the sense that $\dfrac{\lambda_{n+1}}{\lambda_n} > K > 1$, necessary
 and sufficient condition in order that $\mathscr{C}_\Lambda \cap
 C\{M_n \}$ should be $D$-quasi-analytic is that either 
 \begin{enumerate}[1)]
 \item $\lim\inf\limits_{n\rightarrow \infty}
 \dfrac{M_{2n}}{\lambda^2_1 \cdots \lambda^2_n \lambda_{n+1}} = 0,$
 or 
 \item $\mathscr{C}_\Lambda \cap C\{M_n \}$ does contain the function
 $f_o$ whose Carleman transform is $F_o (w) = \prod (1 -
 \dfrac{w^2}{\lambda^2_n})^{-1}$. 
 \end{enumerate}
\end{theorem}

Using Taylor's formula as in Lecture \ref{chap20}, \S ~ 1, we have
$$
\int\limits_o^\alpha ~ | ~ f_o ~ | ~ < C_1 \int\limits_{o}^\alpha ~
\frac{x^{2n-1}}{(2n -1)!}\lambda^2_1 \cdots \lambda^2_n ~ dx = C_1
\frac{\alpha^{2n}}{(2n)!} ~ \lambda^2_1 \cdots \lambda^2_n 
$$

Hence
\begin{equation}
 \int\limits_o^\alpha | f_o | < C_1 \min_n (\frac{\alpha}{(2n)!}
 ~\lambda^2_1 \cdots \lambda^2_n) \tag{1}\label{chap22:eq1} 
\end{equation}

We saw (Lect. \ref{chap20}) that $\mathscr{C}_\Lambda$ is an $I(\alpha)$
quasi-analytic class if $\sum \dfrac{l_n}{\bar{\lambda}} < \infty$ and 
\begin{equation}
 \lim\inf_{\alpha \rightarrow \infty } \frac{I(\alpha)}{\alpha \min
 (l_1 \cdots l_n \alpha^n)^2} < \infty \tag{2}\label{chap22:eq2} 
\end{equation}

(\ref{chap22:eq1}) shows that this condition cannot be very much relaxed; for
example, we cannot replace $\sum\dfrac{l_n}{\bar{\lambda}} < \infty$
by $l_n= \dfrac{\lambda_{n+1}}{2n}$. 

If $\lambda_{n+1}/\lambda_n$ is bounded, the condition is invariant by
a change of $\{ l_n\}$ into $\{l_{n+p} \}$, and also into $\{l_{n+p}
\}$; then, if it holds with $I(\alpha)$ it still holds with
$\alpha^{-2p} ~I(\alpha)$. That is no longer true if we assume 
\begin{equation} 
 \sum \frac{\lambda_n }{\lambda_{n+1}} < \infty \tag{3}\label{chap22:eq3}
\end{equation}

For, we can take in (\ref{chap22:eq2}) $l_n = l_{n-1}$; but, according
to (\ref{chap22:eq1}), we
cannot $l_n = \lambda_{n+1}$. This remark will lead us tu show that
$f_0$ is ``the smallest'' function $\in \mathscr{C}_\Lambda$ near the
origin, in the sense that 
\begin{equation}
 f \in \mathscr{C}_\Lambda,\lim\inf_{\alpha \rightarrow 0}
 \int\limits_o^\alpha | f | ~ / \int\limits_o^\alpha | f_o | < \infty
 \tag{4}\label{chap22:eq4} 
\end{equation}
implies\pageoriginale $f = K ~ f_o, K$ constant.

We suppose (\ref{chap22:eq3}) and (\ref{chap22:eq4}). Since $\Lambda$ is a sequence of Banach -
Szidon, $f$ is bounded. Let $F(w)$ be the Carleman transform of $f$;
we have $| F((u+iv) | < \dfrac{K}{|v|}$ (Lect. \ref{chap7}); hence 
$$
| F(w) ~ (w - \lambda_n) ~ (w - \lambda_{n+1}) | < K(\lambda_{n+1} - \lambda_n)
$$
on the circle of diameter $(\lambda_n, \lambda_{n+1})$, and $F(w)$ is
uniformly bounded outside the discs of radius $1$ around the $\pm
\lambda_n$. Suppose, $f/f_o$ is not a constant. Then $A(w) = F(w) /
F_o (w)$ is an entire function which is not a constant. Since
$F^{-1}_o$ is of exponential type zero, and $F$ is bounded outside our
discs, $A(w)$ is also of exponential type zero; therefore, it has at
least one zero $w_1$. Let $F(w) = (w - w_1) ~ F_1(w)$. Without
restriction, we can suppose $f(0) = 0$; then the differential equation
$f = if'_1 + w_1 f_1$ has one solution such that $f'_1(0) = f_1(0) =
0$ and, using the formulae of Lecture 6, \S ~ \ref{chap6:sec3}, we see that
$F_1(w)$ is the Carleman transform of $f_1$. Now, using the trivial
inequality $\int\limits_{o}^\alpha | f_1 | < \alpha
\int\limits_o^\alpha | f'_1 |$, we get, for $\alpha $ small enough, 
$$
\int\limits_o^\alpha | f | > \int\limits_o^\alpha |f'_1 | - | w_1 |
\int\limits_o^\alpha |f_1| > \frac{1}{2} \int\limits^\alpha_o | f'_1 |
> \frac{1}{2\alpha} \int\limits_o^\alpha | f_1 | 
$$

This inequality, together with (\ref{chap22:eq4}), shows that $f_1/f_o$ is not a
consistent. We can iterate our argument and get $f_2 \in
\mathscr{C}_\Lambda, f_3 \in \mathscr{C}_\Lambda, f_3 \nequiv 0$ 
$$
\int\limits_o^\alpha | f | > \frac{1}{2\alpha} \int\limits_o^\alpha |
f_1 | > \frac{1}{4 \alpha^2} \int\limits_o^\alpha | f_2 | > \frac{1}{8
 \alpha^3} \int\limits_o^\alpha | f_3 | 
$$

Taking into account (\ref{chap22:eq1}) and (\ref{chap22:eq4}), we get
$$
\lim\inf_{\alpha \rightarrow o} \Big( \int\limits_o^\alpha ~ \Big |
f_3 | ~\Big/ \min_n \Big( \frac{\alpha^{2n -3}}{(2n)!} ~ \lambda^2_1
\cdots \lambda^2_n \Big)\Big) = 0, 
$$
which\pageoriginale is in contradiction with the fact that $\mathscr{C}_\Lambda$ is
an $I(\alpha)$ quasi-analytic class whenever $\lim\inf_{\alpha
 \rightarrow o}\dfrac{I(\alpha)}{\alpha \min (\lambda_1 \cdots
 \lambda_n \alpha^{n+1})^2} = 0$. We conclude that $f/f_o$ must be a
constant. We express the result in the following way. 

\begin{theorem}\label{chap22:thm2}% theorem 2
 If $\Lambda = \{\pm \lambda_n \}$ is real, symmetric and very
 lacunary in the sense that $\sum \dfrac{\lambda_n}{\lambda_{n+1}} <
 \infty$, a necessary and sufficient condition in order that
 $\mathscr{C}_\Lambda$ should be an $I(\alpha)$ quasi-analytic class
 is 
$$\lim\inf_{\alpha\rightarrow \infty}\left(I(\alpha) ~ \Big/
 \int\limits_o^\alpha | f_o |\right) = 0, f_o $$ 
being defined in Theorem  $1$. Every time $f \in \mathscr{C}_\Lambda$ and 
$$\lim\inf_{\alpha
 \rightarrow o} \Big( \int\limits_o^\alpha | f | /
 \int\limits_o^\infty | f_o | \Big) < \infty, f/f_o$$ is a
constant. 
\end{theorem}

\begin{remark*}%remark
 It is easy to extend this result. Indeed, if $f \in
 \mathscr{C}_\Lambda$ and\break  $\lim\inf_{\alpha \rightarrow o} \Big(
 \alpha^p \int\limits_o^\alpha | f | \Big/ \int\limits_o^\alpha | f_o
 | \Big) < \infty, f$ is a linear combination (with constant
 coefficients) of $f_o,f_o^1,\ldots, f_o^{(p)}$. It means not only
 that $f_o$ is ``the smallest" function $\in \mathscr{C}_\Lambda$
 near the origin, but also the linear combinations of $f_o$ and its
 derivatives are ``smaller" than any other $f \in
 \mathscr{C}_\Lambda$. 
\end{remark*}
