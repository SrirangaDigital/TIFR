
\chapter{Some Preliminaries}\label{chap2}

\section{Topological vector spaces}\label{chap2:sec1}

Let\pageoriginale $E$ be a vector space over the field $C$ of complex numbers. $E$
is a topological vector space when a topology is given on $E$, such
that addition and scalar multiplication are continuous from $E \times
E$ and $E \times C$ to $E$. We will confine ourselves only to those
topological vector spaces which are locally convex and separated. 
The topology of a locally convex separated vector space $E$ is
specified by a family $\{p_i \}_{i \in I}$ of semi-norms such that for
every $x \in E, x \neq 0$, there is an $i \in I$ with $p_i (x) \neq
0$. The dual $E'$ of $E$ is the set of all continuous linear
functionals on $E$. If $x' \in E'$ we write $x'(x) = \langle x, x'
\rangle$. Then $E$ and $E'$ are in duality in the sense that 
\begin{align*}
 D_1) \quad \langle x, x' \rangle &= 0~ \text{ for every }~ x \in E ~\text{
 implies}~ x' = 0. \\ 
 D_2) \quad \langle x, x' \rangle &= 0 ~\text{ for every }~ x \in E'
 ~\text{ implies }~ x = 0. 
\end{align*}
$D_1)$ is the statement that $x'$ is the zero functional and $D_2)$ is
given by the theorem of Hahn-Banach. (Bourbaki, Chap.$II$ and $III$). 

\textit{Condition of $F$. Riesz. Let $F$ be a closed subspace of $E$
 generated by $\{ x_i \}_{ i \in I}$. In order that $x \in E$ belong
 to $F$ it is necessary and sufficient that $ < x_i, x' >
 = 0$ for every $i \in I$ should imply $\langle x, x' \rangle = 0$.} 

The necessity is obvious and the sufficiency is a consequence of the
Hahn-Banach theorem. 

We\pageoriginale recall some examples of classical vector spaces. First come the
very well-known Banach spaces $L^p (R^n)$ and $L^p (I^n), 1 \leq p
\leq \infty$. Let $K$ be a compact subset of $R^n. \mathscr{C}(K)$ is
the space of continuous (complex-valued ) functions on $K$ with the
topology of uniform convergence. It is a Banach space and its dual
$\mathscr{C}'(K)$ is the space of Radon measures on $K$. 

Let $\Omega$ be an open subset of $R^n. \mathscr{C}(\Omega)$ is the
space of continuous functions on $\Omega$ with the compact convergence
topology (i.e., uniform convergence on every compact set of
$\Omega$). It is an $\mathscr{H}$ -space (a Frechet space,
i.e., locally convex, metrisable and complete). The dual $\mathscr{C}'
(\Omega)$ is the space of Radon measures with compact support in
$\Omega$. The duality is denoted by $\int f d \mu$. 

$D(R^n)$ is the space of $C^{\infty}$-functions (indefinitely
differentiable functions ) with compact support. It is an
$\mathscr{L}\mathscr{H}$-space (inductive limit of
$\mathscr{H}$-spaces). Its dual $D' (R^n)$ is the space of
distributions. 

$\mathscr{E} (R^n)$ is the space of $C^{\infty}$-functions. It is an
$\mathscr{H}$- space and its dual $\mathscr{E}' (R^n)$ is the space of
distributions with compact support. (The support of a distribution is
the smallest closed set such that $< f, T > = T. f$ vanishes whenever
$f$ vanishes on this closed set.) 

These spaces were introduced by L. Schwartz(Schwartz $4$).

\section{Basis in a topological vector space}\label{chap2:sec2}%sec 2

Let $E$ be a locally convex separated vector space over $C$. A set $\{
x_i \}_{ i \in I}$ of elements of $E$ is a \textit{ total } set in $E$
if for every $x \in E$ there are sums $\sum \limits^N_{i=1 } \xi^i_N
x_i \longrightarrow x$ as $N \to \infty ~(\xi^i_N \in C)$. 

A set $\{ x_i \}_{ i \in I}$ of elements of $E$ is \textit{ free} if
$0 = \lim \limits_{N \to \infty} \sum \xi^i_N x_i$ implies $\lim
\limits_{N \to \infty} \xi^i_N =0$ for every $i \in I$. This implies
that if $x= \lim \limits_{N \to \infty} \sum \xi^i_N x_i \lim
\limits_{N \to \infty}$ $\sum \eta^i_M x_i$ then $\xi = \lim\limits_{N
 \to \infty} \xi^i_N = \lim\limits_{M \to \infty} \eta^i_M$. 

A\pageoriginale set $\{ x_i \}_{ i \in I}$ of elements of I is a \textit{basis} if
it is total and free. 

\begin{remark*}
 In order that $\{ x_i \}_{ i \in I}$ be a basis of $E$ it is
 necessary and sufficient that for every $x \in E$ we have $x = \lim
 \limits_{N\to \infty} \sum \xi^i_N x_i$ and for each $i, \lim
 \limits_{N \to \infty} \xi^i_N = \xi^i$, the $\xi^i$ being uniquely
 determined. Then the $\xi^i$ are called the components of $x$ with
 respect to the basis. 
\end{remark*}

\section{Problems of harmonic analysis and synthesis}\label{chap2:sec3}%Sec 3

Let $E$ be a topological vector space of functions defined on an
abelian group $G; \tau(f)$ the closed subspace spanned by the
translates of $f$. There may be in $E$ subspaces which are closed,
invariant under translations, of finite dimension $\geq 1$, and not
representable as a sum of two such subspaces (the subspaces generated
in $\mathscr{C}$ by $e^{i \lambda x}$ or by $x^p e^{i \lambda x}, p =
0, 1, \ldots n,$ are of this type); we shall call such subspaces
``simple subspaces''. 

\textit{The problem of harmonic analysis } can be formulated as the
study of simple subspaces contained in $\tau (f)$. The \textit{problem
 of spectral synthesis} is this : Is it possible to consider $f$ as
the limit of finite sums, $\sum f_n$, of $f_n$ belonging to simple
subspaces contained in $\tau{(f)}$? Practically, we know a priori a
type of simple subspaces, and we ask whether analysis and synthesis
are possible with only these simple subspaces; if it is possible, we
know perfectly the structure of $\tau (f)$, and we can recognize
whether the only simple subspaces are those we know already.\pageoriginale The
problems of analysis and synthesis are usually solved by means of the
theory of duality. We give here a well-known example for
illustration. 

Let $G$ be the one-dimensional torus $T$ (circle) and let $E =
\mathscr{C}(T)$	. Then we can write the equation: 
$$
\displaylines{\hfill 
 a_n = \frac{1}{2 \pi} \int_T f (x) e^{-inx}d \hfill\cr 
 \text{as}\hfill 
 a_n e^{iny} = \frac{1}{2 \pi} \int_T f(x) e^{-inx(x-y)} dx =
 \frac{1}{2 \pi} \int_T f(x + y) e^{-inx} dx \hfill }
$$ 

If $a_n \neq 0$ then $e^{iny}$ is the limit of linear combinations of
translates of $f$, as the following calculations, taking the uniform
continuity of $f (x) e^{inx}$ into account, show: 
\begin{multline*}
 \Bigg | a_n e^{iny}- \frac{1}{2 \pi} \sum (x_{i + 1} - x_i) f(x_i + y)
 e^{-inx_i}\Bigg| \\
 \leq \sum \frac{1}{2 \pi} \Bigg| \int^{x_{i +
  1}}_{x_i} \left[ f(x + y) e^{-inx} - f(x_i + y) e^{-inx_i}\right] dx \Bigg|
 < \varepsilon. 
\end{multline*}

Conversely if $e^{inx} \in \tau (f), a_n \neq 0$. For, given an
$\varepsilon > 0$ and $\varepsilon <1$ we have $\Bigg| e^{iny}- \sum
.\alpha_i f(x_i+y) \Bigg| < \varepsilon$ for all $y$ and 
$$
\left| 1-\left(\sum \alpha_i e^{inx_i}\right) a_n \right| \leq \frac{1}{2 \pi}
\int_T \Bigg| e^{iny}- \alpha_i f(x_i + y) \Bigg| dy. 
$$

This gives $a_n \neq 0$. So in order that $e^{inx} \in \tau (f)$, or
again, for $\tau (e^{inx}) \subset \tau (f)$, it is necessary and
sufficient that $a_n \neq 0$. Thus to specify the ``simple subspaces"
of $\tau (f) [$ viz. $\tau (e^{inx})$] we define the \textit{spectrum}
$S (f)$ of $f$ to be the set of integers $\lambda_n$ such that $e^{i
 \lambda_n x}\in \tau (f)$. Then $\lambda_n \in S(f)$ if and
only if a $\lambda_n \neq 0$. On the other hand the answer to the
problem of spectral synthesis is given by the theorem of
Fejer. (Titchmarsch\pageoriginale p. 414) (Zygmund 1). 

A second method for solving the problem of harmonic analysis is to
apply the condition of Riesz : $e^{inx} \in \tau (f)$ if and if every
linear functional vanishing over the translates of $f$ also vanishes
at $e^{inx}$. In other words $f (x + y) d \mu (-x) = 0$ for every
$y$ implies $\int e^{inx} d \mu (-x) = 0$. But if $f \sim \sum a_n
e^{inx}$ and $d \mu \sim \sum b_n e^{inx}$ are the Fourier
developments of $f$ and $d \mu$ then $\int f(x+y) d \mu(x) \sim \sum a_n
b_n e^{iny}$ and $\int e^{inx} d \mu (-x) = b_n$. Then, $``a_n b_n=0$
for every $n$ implies $b_n = 0"$ follows from $a_n \neq 0$. The
converse is easy. 

We can have such a theory for $G = Z$, the group of integers or $G =
R$ and $E$ the space $\mathscr{C}$ or $L^p, 1 \leq p \leq
\infty$. Obviously we cannot have harmonic synthesis for $L^{\infty}$
with the strong (normed) topology. For example in the case of $G=T$,
the circle, this will imply that every function in $L^{\infty} (T)$ is
continuous. But for the space $L^{\infty}(Z)$ or $L^{\infty}(R)$ with
the weak topology [ for this notion cf. (Bourbaki chap. $IV$)] the
problem of harmonic analysis and synthesis is not solved. For example
in $L^{\infty} (Z)$ let $f = \{f_n \}$ and define the spectrum $S (f)$
of f by $\lambda \in S (f) \Longleftrightarrow \{ e^{i \lambda n} \}
\in \tau (f)$. If for every $g \in L^1 (Z)$, such that $\sum g_n
f_{n+m} = 0$ for every $m$ we have $\sum g_n e^{i \lambda n} = 0$,
then $\{e^{i \lambda n} \} \in \tau (f)$. Conversely if, for every
$\lambda \in S (f), \sum g_n e^{i \lambda n} = 0$, does it follow
that $\sum g_n b_n =0$? This is a problem about absolutely convergent
Fourier series. 
