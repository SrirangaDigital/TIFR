\chapter[Mean Periodic Functions of Several Variables
 (Contd.)]{Mean Periodic Functions of Several Variables
 (Continuation)}\label{chap24}% Lecture 24 

\markright{\thechapter. Mean Periodic Functions of Several Variables
 (Contd.)}
We\pageoriginale complete the proof of the main theorem of the last
lecture. \ref{chap4}, $d) \Longleftrightarrow e$. 

\textbf{d)} $\mathscr{C}(\nu)/\mathscr{C}(\mu)$ is an entire function
of exponential type. 

\textbf{e)} $\nu \in \mu * \mathscr{E}'$, the closed ideal generated
by $\mu$ in $\mathscr{E}'$. 

Now $e) \Longleftrightarrow c)$ is evident and since $c)
\Longleftrightarrow d), e) \Rightarrow d)$. The proof of $d)
\Rightarrow e)$ is involved. 

Let $\mathbb{Q}\mathscr{Y}'$ be the space of Fourier-transforms of
distributions in $\mathscr{E}'$. Let $M, N \in \mathbb{Q}\mathscr{Y}'$
and suppose $N/M$ is of exponential type. Moreover suppose it is
possible to show that $\dfrac{\partial ~ M}{\partial ~ w_1} ~ N \in
\mathscr{C} (\mu * \mathscr{E}')$. Then we can prove that $d)
\Rightarrow e)$, because we can iterate the same process with each
variable and get $\dfrac{\partial ~ M }{\partial ~ w_2}
\dfrac{\partial ~ M}{\partial ~ w_1} ~ N \in \mathscr{C} (\mu *
\mathscr{E}')$ etc., and finally $(P(x) \mu) * \nu \in \mu *
\mathscr{E}'$, for every polynomial $P(x)$. Using the result that if
$\mu \neq 0$ it is possible to find a sequence $\{P(x) \}$ and
$\lambda \in R^n$ such that $P(x) \mu \rightarrow \delta_\lambda$ we
find that $\delta_\lambda * \nu \in \overline{\mu * \mathscr{E}'}$,
i.e. $\nu \in \overline{\mu * \mathscr{E}'}$. Now we can write
$\dfrac{\partial ~ M}{\partial ~ w_1} N = \dfrac{1}{M} \dfrac{\partial
 ~ M}{\partial ~ w_1}NM$. To prove that $\dfrac{\partial ~
 M}{\partial ~ w_1} N \in \mathscr{C} (\mu * \mathscr{E}')$ it is
sufficient to prove the following conjecture (true in the case of one
variable). 

\begin{conjecture}% conjecture
 $\dfrac{1}{M} \dfrac{\partial ~ M}{\partial ~ w_1} NM \in \mathbb{Q}
 \mathscr{Y}'$, when $M.N \in \mathbb{Q}\mathscr{Y}'$ and
 $\dfrac{N}{M}$ is entire. Unfortunately we are not able to prove
 this conjecture. But to prove that $d) \Rightarrow e)$ it is
 sufficient (as indicated by Malgrange) to prove only the following
 proposition: 
\end{conjecture}

\setcounter{proposition}{0}
\begin{proposition}\label{chap24:prop1}% proposition 1
  $P,R \in \mathbb{Q}\mathscr{Y}'$\pageoriginale and $P/Q$ entire implies that
  $\dfrac{1}{Q}\dfrac{\partial ~ Q}{\partial ~ w_1} ~ P ~ Q^2 $ $(0,w_2,
  \ldots, w_n) \in \mathbb{Q}\mathscr{Y}'$. 
\end{proposition}

The idea of the proof is to majorise $\dfrac{1}{Q}\dfrac{\partial ~
 Q}{\partial ~ w_1}$ in the same way as $\dfrac{M'(w)}{M(w)}$ in
Lecture 5, \S  \ref{chap5:sec1}, and to use Proposition
\ref{chap24:prop1} of the last lecture (B.Malgrange). 

\begin{defi*}% definition 
 $H(\mu)$ is the set of distribution $\tauup$ such that $\tauup * \nu
 \in \mu * \mathscr{E}'$, for every $\nu$ with $\mathscr{C} (\nu)/
 \mathscr{C}(\mu)$ entire. 
\end{defi*}

\begin{theorem*}
  $H(\mu)$ {\em{is dense in}} $\mathscr{E}'$.
\end{theorem*}

Suppose this theorem is proved. Allowing $\tauup \rightarrow \delta$
we have $\nu = \lim\limits_{\substack{\tauup \in H (\mu) \\ \tauup
 \rightarrow \delta}} ~ \tauup * \nu$ and $\nu \in\mu *
\mathscr{E}'$. Thus we have $d) \Rightarrow e)$. 

\begin{defi*}% definition
  Let $\sigma \in \mathscr{E}'$. $H(\mu,\sigma)$ is the set of
  distributions such that $\tauup * \sigma \in H(\mu)$. 
\end{defi*}

We say that $H(\mu,\sigma) \in (\delta_p)$(i.e. has the property
$\delta_p$) if there exists a set of distributions $\sigma_1, \ldots,
\sigma_r$ such that $\sigma_{1}*, \ldots * \sigma_r \in H(\mu,\sigma)
$ with $\mathscr{C}(\sigma_j)$ depending only on $p$ variables. 

\begin{proposition}\label{chap24:prop2}% proposition 2
 Let $0 \le p \le n$. If $H(\mu, \sigma) \in (\delta_p)$, then
 $H(\mu,\sigma)$is dense in $\mathscr{E}'$. 
\end{proposition}

This will prove our theorem, for $H(\mu) = H(\mu,\delta) \in
(\delta_n)$. We prove Proposition $2$ by induction. For this we need
the following proposition: 
\begin{proposition}\label{chap24:prop3}%proposition 3
 Let $\tauup \in \mathscr{E}'$ be such that $\mathscr{C} (\tauup)$
 depends only on $p + 1$ variables. Suppose $\mathscr{C}(\tauup) = T
 = T(w_1,\ldots, w_{p+1})$. If 
 $$
 H(\mu,\sigma * \tauup) \in (\delta_p), \text{ then } H(\mu,\sigma *
 x_1 \tauup) \in (\delta_p). 
 $$
\end{proposition}


\begin{proof}% proof
 $(\sigma_1 ~ * \cdots * ~ \sigma_r ~ * \tauup * \sigma * \nu) \in
 \mu * \mathscr{E}'$ gives, by taking the Fourier transform that
 $(S_1 \cdots S_r ~ T ~ S ~ N/M) \in \mathbb{Q}\mathscr{Y}'$. We take
 $Q = T$ and $P = T ~ S_1 \cdots S_r S N/M$\pageoriginale in
 Proposition \ref{chap24:prop1}, and
 get 
 $$
 \frac{\partial ~ T}{\partial ~ w_1} ~ T^2 (0,w, \cdots, w) S_1
 \cdots S_r ~ S \frac{N}{M} \in \mathbb{Q}\mathscr{Y}'. 
 $$

 Now $T^2 (0, w_2, \ldots, w_r)$ is the Fourier transform of a
 distribution and depends only on $p$ variables. So $H(\mu,\sigma * x_1
 \tauup) \in (\delta_p)$. 
\end{proof}

\setcounter{proofofprop}{1}
\begin{proofofprop}%Proof
 $H(\mu,\sigma) \in (\delta_o)$ means that $\delta \in H(\mu,\sigma)$
 and so, since $H(\mu,\sigma)$ is an ideal in $\mathscr{E}'$, it is
 dense in $\mathscr{E}'$. Suppose that the proposition is true for
 $p$. Let $H(\mu,\sigma) \in (\delta_{p+1})$. Then there exist
 $\mathscr{C}(\sigma_j)$ depending on $(p+1)$ variables such that
 $\sigma_1 * \cdots * \sigma_r * \sigma_r * \nu \in \mu *
 \mathscr{E}'$, for every $\nu$ satisfying the condition $\tauup
 (\nu)/\mathscr{C}(\mu)$ entire. Therefore $H(\mu,\sigma_1 * \cdots *
 \sigma_r * \sigma) = \mathscr{E}'$ and so $\in (\delta_p)$. Applying
 successively Proposition \ref{chap24:prop3}, we get that $H(\mu,P_1 \sigma_1 *
 \cdots * P_r \sigma_r * \sigma) \in (\delta_p)$, whatever be the
 polynomials $P_r$ depending only on those $x_j$ for which the $w_j$
 occurs in $\mathscr{C}(\sigma_r)$. According to the hypothesis,
$H(\mu, P_1 \sigma_1 * \ldots *P_r \sigma_r * \sigma)$ is dense in
 $\mathscr{E}'$; this means that $P_1 \sigma_1 * \cdots * P_r \sigma_r \in
 \overline{H(\mu,\sigma)}$. It is possible to choose $P_1, \cdots,
 P_r$ in such a manner that $P_j \sigma_r \rightarrow
 \delta_{\lambda_j}, j = 1, \ldots, r$. Therefore 
 $$
 \delta_{\lambda_1} * \cdots * \delta_{\lambda_r} \in
 \overline{H(\mu, \sigma)} \text{ and so } \overline{H(\mu,\sigma)} =
 \mathscr{E}'. 
 $$

 That achieves the proof of the theorem.
\end{proofofprop}

We give a few complements when $\mu$ is a partial differential
operator with constant coefficients. Solutions of $\mu * f = 0$ are
the solutions of the homogeneous equation $D * f = 0$. Consider
$\mathscr{E}(\Omega), \Omega$ an set of $R^n$. 

\begin{theorem*}% theorem 
 Let\pageoriginale $\Omega$ be an open convex set in $R^n$ and let $D * f =
 0$. Then $f \in $ span in $\mathscr{E}(\Omega)$ of the polynomial
 exponential $Q$ satisfying the equation $D * Q = 0$. 
\end{theorem*}

Indeed, $\mathscr{C}(D)$ is a polynomial $P(w_1,\cdots, w_n)$ and
$\mathscr{C} (\nu) = N (w_1, \ldots,$ $w_n)$. We can suppose (perhaps
after a change of variables) $P(w_1, \ldots,\break w_n) = w_1^{m} + A_1
w^{m-1} + \cdots + A_m, A_1, A_2, \ldots, A_m$ being functions of
$w_2, \ldots,\break w_n$. Then, by Cartan's lemma (Lect. 14, Lemma
\ref{chap14:lem2})  
$$
\Big|\frac{N}{P}(w_1,\ldots, w_n) \Big| < \sup_{|\zeta|\le 2c} \Big|
N(w_1 + \zeta, w_2, \ldots, w_n)\Big| 
$$
so that $\mathscr{C}(\nu)\Big/ \mathscr{C}(D) \in
\mathbb{Q}\mathscr{Y}'$ and hence $\nu = D * \mu, \mu \in
\mathscr{E}'$. Moreover, the support of $\mu$ is in the convex closure
of the support of $\nu$, which is in $\Omega$. Then $\nu * f = \mu * D
* f = 0$ and $\langle \nu, f \rangle = 0$. 

\setcounter{prob}{0}
\begin{prob}\label{chap24:prob1}% problem 1
 Is it possible to replace $\Omega$ convex by $\Omega$ connected or
 simply connected? 
\end{prob}

\begin{theorem*}% theorem
 $\{ \Omega \text{ convex, open and } D * f = 0\}\Rightarrow f \in
 ${\em{span of polynomials $P$ satisfying $D * P = 0$ is valid if and
  only if the irreducible factors of $\mathscr{C}(D)$ are all zero
  at 0}}. 
\end{theorem*}

For example, when $D = \triangle$, the Laplacian, every harmonic
function is the limit of polynomials. We indicate the idea of the
Proof (Malgrange). From the fact that $\mathscr{C}(\nu)\Big/
\mathscr{C}(D)$ is holomorphic at $0$, we will have
$\mathscr{C}(\nu)\Big/ \mathscr{C}(D)$ entire. Indeed if $R =
\mathscr{C}(D) = R_1 \ldots R_r$ and $V_j = \{w| R_j(w)= 0 \}$, the
polar manifold of $\mathscr{C}(V) / \mathscr{C}(D)$ is the union of
$V_j$. If $0 \notin \cup V_j$ and if $\mathscr{C}(\nu)/\mathscr{C}(D)$
is holomorphic at $0$, the polar manifold is void. 

We\pageoriginale conclude with the consideration of analytic mean periodic
functions. Instead of $\mathscr{E}(R^n)$ we take $\mathscr{H}(C^n)$
with the topology of compact convergence in $R^{2n}$. The dual
$\mathscr{H}'$ is a quotient space of the space of measures. 

Let $\mathscr{C}(\mu) = \langle \mu, e^{w_1z_1 + \cdots +
 w_nz_n}\rangle, \mu \in \mathscr{H}'$. $\mathscr{C}(\mu)$ is of
exponential type. 

\begin{theorem*}[(Malgrange - Ehrenpreis)]  
  If $M(w) = M(w_1 \cdots w_n)$ is an entire
  function of exponential type, then $M(w) = \mathscr{C}(\mu), \mu \in
  \mathscr{H}'$. 
\end{theorem*}

\begin{proof}% proof
 Let $M(w) = \sum a_{i_1, \ldots, i_n} w^{i_1}_1 \cdots
 w_n^{i_n}$. $|M(w)| < A ~ e^{B|w|}$ with $|w| = |w_1| + \cdots + |
 w_n |$ gives the following majorization: 
 $$
 | a_{i_1 \cdots i_n} | < \const. \frac{B^{i_1 + \cdots +
  i_n}}{i_1 ! \cdots i_n !}. 
 $$
\end{proof}

Consider the linear form $\langle D,f \rangle, f \in\mathscr{H}$ defined by 
$$
\langle D, f \rangle = a_{i_1 \cdots i_n} \frac{\partial^{i_1 + \cdots
 + i_n}}{\partial Z^{i_1}_1 \cdots \partial z^{i_n}_{n}} ~ f(0). 
$$

It is a continuous linear functional and so $\in \mathscr{H}'$. Moreover,
$$
M(w) = \langle D, e^{w_1 z_1 + \cdots + w_nz_n } \rangle.
$$

Now it is possible to extend $D$ to a measure such that $M(w) =
\mathscr{C}(\mu)$. Then it is easy to prove (Malgrange): 
\begin{theorem*}%theorem
 Let $f \in \mathscr{H}(C^n), \mu \in \mathscr{H}'(C^n)$. If $f * \mu
 = 0$, then $f$ belongs to the span of polynomial exponentials
 satisfying the same equation. 
\end{theorem*}

These considerations suggest the following problem. Consider $f * \mu
= 0$ as a class of partial differential equations of infinite
order. Then 
$$
\mu * f = \sum a_{i_1 \cdots i_n} \frac{\partial^{i_1 + \cdots +
 i_n}}{\partial z^{i_1}_1 \cdots \partial z^{i_n}_n} ~ f(z) = 0 
$$
gives\pageoriginale a homogeneous partial differential equation. This cannot be
defined for every $z \in \Omega$ when $f \in \mathscr{H}(\Omega)$. The
only case when $\mu * f$ is defined in the same open set as $f$ is
that when the Fourier transform of the operator is of type of zero. In
that case, we say that $\mu$ is of minimal type. 

\begin{prob}\label{chap24:prob2}% problem 2
 When $\mu$ is of minimal type, $f \in \mathscr{H} (\Omega)$ and $\mu
 * f = 0$, will this imply that $f \in $ span of exponential
 polynomials satisfying $\mu * Q = 0$ in its domain of existence? 
\end{prob}

In the case of one variable we know, by the method of A.F.Leontiev,
that the domain of existence is convex and $f \in \mathscr{H}_\Lambda
(\Omega)$. 
\newpage

\begin{center}
  \textbf{\Large APPENDIX I}\\ 
  \textbf{\large On the maximum density of Polya}
\end{center}
\medskip

Let\pageoriginale $\Lambda$ be a positive sequence and let $n(r) =
\sum\limits_{\lambda < r} 1, \lambda \in \Lambda$. 

\begin{defi*}% definition
 $D_{\max} \Lambda = \inf_{\Lambda' \supset \Lambda} \{D' (\Lambda')
 \}, \Lambda'$ being a sequence having density $D'$. 
\end{defi*}

We have the following relation:
$$
D' \ge D_{\max} \Longleftrightarrow \{n'(R) - n'(r) \ge n(R) - n(r), R
> r, \lim_{R \rightarrow \infty} \frac{n'(R)}{R} = D' \}. 
$$
\medskip


\noindent
\textbf{1. An inequality for $D_{\max}$.}

For $k > 1, \dfrac{n'(kr) - n'(r)}{kr - r} \ge \dfrac{n(kr) - n(r)}{kr
 - r}$. As the first member $\rightarrow D'$ when $r \rightarrow
\infty$, we have for every $k > 1, D_{\max} \ge \underset{r
 \rightarrow \infty}{\lim \sup} \dfrac{n(kr) - n(r)}{kr
 -r}$. Therefore we have following inequality: 
\begin{equation}
 D_{\max} \ge \underset{k \searrow 1}{\lim\sup} ~ \underset{r
 \rightarrow \infty}{\lim \sup} ~ \frac{n(kr) - n(r)}{kr-r} =
 \underset{k \searrow 1}{\lim \sup} ~ \varphi (k) \tag{*} 
\end{equation}
where $\varphi (k) = \underset{r \rightarrow \infty}{\lim \sup} ~
\dfrac{n(kr) - n(r)}{kr-r}$. 
\medskip

\noindent
\textbf{2. Study of $\varphi(k)$}

Using the fact that $p+ p' / q+ q'$ lies between $p/q$ and $p'/q'$ we have
{\fontsize{10}{12}\selectfont
$$
\frac{n(kr) - n(r)}{kr -r} \le \sup \left\{ \frac{n(\sqrt{kr}) -
 n(r)}{\sqrt{kr} - r}, \frac{n(kr) - n(\sqrt{kr})}{kr - \sqrt{kr}} \right\}
\text{ and so } \varphi (k) \le \varphi (\sqrt{k}). 
$$}\relax

Therefore we take a $k > 1$ and consider the quantity $\triangle$ defined by 
\begin{equation}
 \triangle = \lim_{p \rightarrow \infty} \varphi(k_p) \text{ where }
 k_p = k^{2^{-p}}. \tag{**} 
\end{equation}
$\triangle$ exists because $\varphi(k_p)$ is monotone when $p \rightarrow \infty$.
\medskip

\noindent
\textbf{3 Construction of $\Lambda^* \supset \Lambda, \Lambda^* $
 having density $\triangle$.} 

Suppose $\in_1, \in_2, \ldots, \in_q, \ldots$
is a sequence of positive numbers $\rightarrow 0$. In what follows we
use the following notation: $k^i_q = k^{i2^{-q}}$ and $k_q = k^1_q =
k^{2^{-q}}$.\pageoriginale Then $k^{i+1}_q = k_q. k^i_q$. Using the definition of
$\varphi (k)$ we are able to determine a sequence of even integers
$\{i_q \}$ with $i_{q+1} > 2i_q$ in such a manner that the following
conditions are satisfied: 
\begin{align*}
 \text{ for }\quad i & \ge i_1, \frac{n(k^{i+1})-n(k^i)}{k^{i+1} - k^i} <
 \varphi (k) + \in_1, (k^{i+1} - k^i)^{-1} < \in_1
 \tag{a}\\ 
 \text{ for }\quad i & \ge i_q, \frac{n(k^{i+1}_{q})-n(k^i_q)}{k^{i+1}_q -
 k^i_q} < \varphi (k_q) + \in_q, (k^{i+1}_q - k^i_q)^{-1}
 < \in_q \tag{b} 
\end{align*}

The set of segments $(k^i_q, k^{i+1}_q)$ for $i = i_q, \ldots,
(\dfrac{1}{2}i_{q+1} - 1), q = 1, 2, \ldots$ are disjoint and cover
the semi-axis [$k^{i_1}, \infty$]. 

In order to determine $\Lambda^*$ we define $n* (r)$ taking only
integral values and satisfying the following conditions: 
\begin{gather*}
 n* (R) - n*(r) \ge n(R) - n(r), R \ge r \tag{1}\\
 \frac{n* (k^{i+1}_q) - n* (k^i_q) + 0(1)}{k^{i+1}_q - k^i_q} =
 \varphi (k_q) + \in_q, 0 \le 0 (1) \le 1 \tag{2} 
\end{gather*}
whenever $i_q \le i < \dfrac{1}{2}i_{q+1}$. Indeed (1) is equivalent
to $\Lambda^* \supset \Lambda$ and (2) will show us how many points
we must add to $\Lambda$ in order to get $\Lambda^*$. Now in view of
the inequalities $(a)$ and $(b)$ above, the definition of $\Lambda^*$
(or again that of $n*(r)$) satisfying conditions (1) and (2) can
always be achieved. 

The density of $\Lambda*$ on an interval $(a,b)$ being $\dfrac{n*(b) -
 n*(a)}{b - 1}$, we set $D^*[k^i_q, k^{i+1}_q) - \dfrac{n*(k^{i+1}_q)
 - n*(k^i_q)}{k^{i+1} - k^i_q}$ and we have, by $(2)$ 
 \begin{equation}
 \varphi (k_q) \le D * [k^i_q, k^{i+1}_q) \le \varphi (k_q) +
  \in_q. \tag{***} 
 \end{equation}

Consider the density of $\Lambda^*$ on an interval $[k^{ip}_p,k^i_q) $
 with $i_q \le i < \dfrac{1}{2}i_{q+1}$ and $q > p$. It lies between
 the lower and upper bound of densities on the intervals\pageoriginale $[ k^i_s, 
 k^{i+1}_s)$ situated to the right of $k^{ip}_p$. But by $(***)$
 all these densities differ from $\Delta$ by a quantity which tends
 to zero with $1/p$. Finally, consider the density of $\Lambda^*$
 on an interval $[k^{ip}_p, r )$ with $k^i_q \le r < k^{i+1}_q$ and
 $i_q \le i < \dfrac{1}{2} i_{q+1}$. Let this density be denoted
 by $X_p (r)$. We have 
 $$
 \displaylines{\hfill 
  X_p (k^i_q) \frac{k^i_q-k^{ip}_p}{k^{i+1}_q - k^{ip}_p} \le X_p
  (r) \le X_p (k^{i+1}_q)
  \frac{k^{i+1}_q-k^{ip}_p}{k^i_q-k^{ip}_p} \hfill \cr
  \text{since}\hfill 
  \frac{n^*(k^i_q)-n^*(k^{ip}_p)}{k^{i+1}_q -k^{ip}_p} \le
  \frac{n^*(r)-n^*(k^{ip}_p)}{r-k^{ip}_p} \le
  \frac{n^*(k^{i+1}_q)-n^* (k^{ip}_p)}{k^i_q-k^{ip}_p}.\hfill } 
 $$

When $r \to \infty$, we have $\lim \sup X_p(r)< \Delta +
\in '_p, \lim \inf X_p(r)> \Delta - \in '_p$, with
$\underset{p \to \infty}{\lim} \in '_p=0$. Therefore
$\Lambda^*$ has density $\Delta$. 

From 1, 2, 3, we get the following result: 
$$
D_{\max} (\Lambda)= \Delta = \lim_{k \searrow} \sup_1 \varphi(k)
$$ 
\textit{ and there exists a sequence } $\Lambda^* \supset \Lambda$,
\textit{ having a density equal to } $D_{\max}(\Lambda)$. 
\medskip

\noindent
\textbf{4. A new expression for $D_{\max}$.} %section 4

We use two simple inequalities on $\varphi (k^{1-\alpha})-
\varphi(k)$, $\alpha$ being small enough: 
\begin{enumerate}[$1^o)$]
\item as $(k-1), \varphi(k)$ is an increasing function,
 $$
 \varphi(k)\ge \varphi (k^{1-\alpha}) \frac{k^{1-\alpha}-1}{k-1} \le
 \varphi(k^{1-\alpha})(1-k^\alpha) 
 $$
\item from the equality
 $$
 \displaylines{\hfill 
 n(kr)-n(r)= n(kr)-n(k^\alpha r)+ n(k^\alpha r)-n(r)\hfill \cr 
 \text{we get}\hfill 
 \varphi(k)\le \frac{k -k^\alpha}{k-1} \varphi (k^{1-\alpha})+
 \frac{k^\alpha-1}{k-1} \varphi (k^\alpha) \le \varphi (k^{1-\alpha})+
 \alpha \varphi (k^\alpha) \hfill \cr
 \text{and}\hfill 
 - \alpha \Delta \le \varphi (k^{1-\alpha}) - \varphi (k) \le k ~
 \alpha ~ \Delta. \hfill {(*)}} 
 $$
\end{enumerate}

Suppose\pageoriginale now $\underset{k \searrow 1}{\lim \inf} ~ \varphi (k)=
\Delta'$. There exists a sequence $k'_n \searrow 1$ such that
$\varphi (k'_n) \to \Delta'$. From the sequence $\Bigg\{ \dfrac{\log
 \log k'_n}{\log 2}\Bigg\}$ one can extract a sequence which is
convergent modulo $1$; we can suppose the sequence itself convergent
mod. $1$, viz. $\dfrac{\log \log k'_n}{\log 2}= -q_n + h+
\in_n (\in_n \to 0) $. Defining $k= \exp (2^h)$ and
$1+\alpha_n = 2^{\in_n}$, we have $ k'_n=
k^{1+\alpha_n}_{q_n}$. Using the inequality $(\otimes)$, we have
$\underset{n \to \infty}{\lim} (\varphi(k'_n)- \varphi(k_{q_n}))=0$,
then $\Delta' = \Delta$. That proves the following equality: 
$$
D_{\max}(\Lambda) =\lim\limits_{k \searrow 1 } \varphi
(k)=\lim\limits_{k \searrow 1 } \underset{r \to \infty}{\lim \sup} ~
\frac{n(kr)- n(r)}{kr-r} 
$$
\newpage

\begin{center}
  \textbf{\Large APPENDIX II} \\ 
  \textbf{\large Construction of a sequence with density zero
    and Mean-period infinity }
\end{center}

We\pageoriginale shall use the following fact, which is a simple consequence of
Carleman's theorem: If $F(w)$ is an entire function of exponential
type bounded on the real axis, then 
\begin{equation}
 \lim_{R \to \infty} \int^R_1 \log | F (u) F(-u) | \frac{du}{u^2} \tag{*}
\end{equation}
exists and is finite (see for example, (Levinson) $p.26$). We shall
construct the sequence $\Lambda$ in such a manner that no entire
function of exponential type, vanishing on $\Lambda$, can satisfy this
condition; thus the mean-period of $\Lambda$ is infinity. 

Let $\{ \mu_k \}$ be a rapidly increasing sequence of positive numbers
(we shall specify later), and $\{ \nu_k \}$ a sequence of integers
with $\nu_k=\in_k \mu_k=o(\mu_k)(k=1, 2, \ldots)$. We take as
$\Lambda$ the sequence $\{ \mu_k\}$, each $\mu_k$ counted $\nu_k$
times. Let $F$ be, if possible, an entire function of exponential type
bounded on the real axis; then, either $F(w)+ F(-w)$ or
$\dfrac{F(w)}{w}$ is even, and, by means of a trivial change, we can
suppose 
$$
F(w) = \prod^\infty_1 \left(1-\frac{w^2}{\mu_j^2}\right)^{\nu_j}_{j}
\prod^\infty_1 \left(1-\frac{w^2}{w^2_j}\right). 
$$

Whatever be $k$, we have (with $r=|w|$)
$$
|\left(1- \frac{w^2}{\mu^2_k}\right)^{-\nu_k} F(w)| \le \prod^\infty_1
\left(1+\frac{r^2}{\mu^2_j}\right) \prod^\infty_1
\left(1+\frac{r^2}{|w_j|^2}\right)<e^{Br} 
$$

The\pageoriginale last inequality holds if $B$ is large enough, according to the
calculation of Mandelbrojt (Lecture 10, \S \ref{chap10:sec1}); $B$ does not depend
on $k$. Thus 
$$
\log | F(\mu_k + t) | < B(\mu_k +t)+ \nu_k \log \frac{| 2 \mu_k ~ t +
 t^2 |}{\mu^2_k}. 
$$

Set $t_k=\dfrac{1}{2} \mu_k \exp \left(- \dfrac{2B}{\in_k}\right)
(=o(\mu_k))$ and $0 <t < t_k$; then 
$$
\nu_k \log \frac{2 \mu_k t +t^2}{\mu_k^2} \sim \nu_k \log
\frac{2t}{\mu_k} < -\frac{2B}{\mu_k} \nu_k =-2B \mu_k. 
$$

Hence
$$
\log|F(\mu_k +t)| < (\frac{1}{2}+ o(1)) \nu_k \log \frac{2t}{\mu_k} ~~
(0<t<t_k) 
$$
if $k>k_o$ large enough
\begin{align*}
 \int^{t_k}_o \log |F(\mu_k + t)| \frac{dt}{(\mu_k+t)^2} & <
 \frac{1}{4} \frac{\nu_k}{k} \int^{t_k}_{o} \log \frac{2t}{\mu_k}
 \frac{dt}{\mu_k}\\ 
 & = \frac{1}{4} \in_k \frac{t_k}{\mu_k} \log \frac{2t_k}{e \mu_k}\\
 & < - \frac{B}{4} \exp \left(- \frac{2B}{\in_k}\right).
\end{align*}

Now we shall choose $\{ \in_k\}$ in such a manner that
$\int^\infty_1 \dfrac{\log^{-}F(u)}{u^2} du$ $= -\infty$; as $\log^{+}
F(u)$ is bounded, we cannot have $(*)$ finite, and the existence of
$F$ leads to a contradiction. It is sufficient to take $\in_k
= (\log ~ (\log)k)^{-1}$; then 
\begin{multline*}
 \int^\infty_1 \frac{\log F(u)}{u^2}du\\ 
 < \sum^\infty_{k_o} \log^{-} |
 F(\mu +t)| \frac{dt}{(\mu_k +t)^2} < - \frac{B}{4} \sum^\infty_k \exp
 \left(-\frac{2B}{\in_k}\right) =-\infty 
\end{multline*}
(The first inequality holds if $\mu_{k+1} > \mu_k + t_k$, what is true
if $\mu_k$ is rapidly increasing). 

And\pageoriginale now we can choose $\{ \mu_k\}$ as rapidly increasing as we want;
in particular, we can choose it in order that the density of $\Lambda$
is zero, i.e.$ \sum^k_1 \nu_k = o (\mu_k)$ (for example, $ \nu_k =
\in_k \mu_k =2^k$). 
Thus $\Lambda$ has density zero, and mean period infinity. 

We have constructed $\Lambda$ as a sequence of multiple points. Now we
can replace the point $\mu_k$ counted $\nu_k$ times, by $\nu_k$ points
near enough to $\mu_k$, and the result still holds. 

It would be interesting to know if such an example can be constructed
with a ``regular" $\Lambda$; for example, for a sequence of distinct
integers. 

\begin{thebibliography}{99}
\bibitem{1}{Bernstein,V.}\pageoriginale Lecons sur les progres recents
  de la theorie des series de Dirichlet, Paris, 1933. 
\bibitem{2}{Besicovitch, A.S.} Almost periodic functions, Dover, 1954.
\bibitem{3}{Boas, R.P.} Entire functions, New York, 1954.
\bibitem{4}{Bohr,H.} Almost periodic functions, Chelsea, 1951.
\bibitem{5}{Bourbaki, N.} espaces vectoriels topologiques chap. I-V. Paris, 1955.
\bibitem{6}{Cartan,H.} Seminaire E.N.S.,1952-53 and M.I.T.1955.
\bibitem{7}{Carleman,T.} L' integrale de fourier et questions qui
 s' y rattachent, Uppasala, 1944. 
\bibitem{8}{Delsarte, J.} Les fonctions moyenne- periodiques, Journal
 de Math, Pures et Appl., 17 (1935), 403-453. 
\bibitem{9}{Ehrenpreis, L.} Mean periodic functions I,
 Amer. Jour. Math. 77(1955), 293-328. 
\bibitem{10}{Kahane, J.P.} 
 \begin{enumerate}
 \item Sur quelques problemes $d'$ unicite et de pro-longement,
 relatifs aux functions approchables par des sommes $d'$
 exponentials, Annales de $L'$ Institut Fourier, 5 (1953-54),
 pp.39-130. 
 \item Sur les fonctions moyenne-periodiques bornees, Annales de $L'$
 Institut Fouuier, 7 (1958). 
 \end{enumerate}
\bibitem{11}{Koosis,P.} 
 \begin{enumerate}
 \item Note sur les fonctions moyenne periodiques, Annales de $L'$
 Institut Fourier, 6 (1955-56), 357-360. 
 \item On functions which are mean periodic on a half line,
 Commun. pure and applied math., 10(1957), 133-149. 
 \end{enumerate}
\bibitem {12}{Lefranc,M.}\pageoriginale Analyse harmonique dans $Z^n$ Comptes Rendus
 Ac.Sc., Paris, 246(1958). 
\bibitem{13}{Leontieve, A.F.} Series of Dirichlet's polynomials, Trudy
 Matem. Inst. Stekloff, Moscow 39(1951) (Russian). 
\bibitem{14}{Levin, B.} 
 \begin{enumerate}
 \item Doklady C. R. U.S.S.R. 65 (1949), 265, and 70(1949), 757-760.
 \item Distributions of roots of entire functions, Moscow 1956 (Russian).
 \end{enumerate}
\bibitem{15}{Levinson, N.} Gap and density theorems,
 A.M.S. Colloquium, Vol. XXVI, 1940. 
\bibitem{16}{Malgrange, B.} Existence et approximations des solutions
 des equations aux derivees partielles et des equations de
 convolution, Annales de $L'$ Institut Fourier, 6 (1955-56),
 271-355. 
\bibitem{17}{Mandelbrojt,S.}
 \begin{enumerate}
 \item Series de Fourier et classes quasi-analytiques de fonctions, Paris (1935).
 \item Dirichlet's Series, Rice Institute Pamphlet (1944).
 \item Series adherents, regularisations des suites applications, Paris (1952).
 \end{enumerate}
\bibitem{18} {Mergelyan, S.N.} Uniform approximation to functions of a
 complex variable, $U$ spekhi matem, nauk 7 (1952), 31-122
 (Russian), or A.M.S. Russian translations,No. 101. 
\bibitem{19}{Paley, R.E.A.C. and Wiener, N.} fourier transforms in the
 complex domain, A.M.S. Colloquium, vol.XIX(1934). 
\bibitem{20}{Plancherel,M.} Int\'egrale de Fourier et fonctions enti\`eres
 Colloque analyse harmonique, C.N.R.S., Nancy (1947). 
\bibitem{21}{Schwartz, L.} 
 \begin{enumerate}
 \item sommes $d'$ exponentielles r\'eelles, Paris (1942).
 \item Approximation $d'$ une fonction par des sommes $d'$
 exponentielles imaginaires, Anna. Toulouse, 6 (1942), 111-174. 
 \item Fonctions moyenne periodiques, Annals of Mathematics, 48 (1947), 857-929.
 \item Theorie des distributions, tome 1 (1951), tome 2 (1951).
 \end{enumerate}
\bibitem{22}{Titchmarsh, E.C.} The theory of functions, Oxford 1952.
\bibitem{23}{Zygmund, A.} 
 \begin{enumerate}
 \item Trigonometrical series, Warsaw (1935).
 \item Quelques theoremes sur les series trigono-metriques, Studia
 math. 3(1931), 77-91. 
 \end{enumerate}
\end{thebibliography}

