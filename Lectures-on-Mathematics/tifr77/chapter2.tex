
\chapter{Arithmetic of Quadratic Forms}\label{c2}

IN\pageoriginale THIS CHAPTER, we exhibit several theorems on
representations of quadratic forms obtained by an arithmetical
approach. The only basic reference on quadratic forms here is

[S] J. -P. Serre, A Course in Arithmetic, Springer-Verlag, New
York-Heidelberg- Berlin, 1973.

\setcounter{section}{-1}
\section{Notation and Terminology}\label{c2:sec-2.0}

Let $k$ be a field with characteristic $\neq 2$, and $\ub{o}(\ni 1)$ a
ring contained in $k$ (with $k$ as quotient field).

Let $M$ be an $\ub{o}$-module and $Q$ a mapping from $M$ to $k$ such
that
\begin{enumerate}
\renewcommand{\labelenumi}{(\theenumi)}
\item $Q(ax)=a^{2}Q(x)$ for $a\in\ub{o}$ and $x\in M$

\item $Q(x+y)-Q(x)-Q(y)=2B(x,y)$ is a symmetric bilinear form. Then we
  call the triple $(M,Q,B)$ or simply $M$ a {\em quadratic module over
    $\ub{o}$}, and $Q$ (\resp $B$) the {\em quadratic form} (\resp the
  {\em bilinear form} associated with $M$.
\end{enumerate}

Hereafter, we consider only modules which are finitely generated and
torsion-free.


\subsection{}\label{c2:subsec2.0.1}
Let $M$ be a quadratic module over $\ub{o}$ and suppose that
$M$ has a basis $\{v_{i}\}$ over $\ub{o}$. Then we write
$M=<(B(v_{i},v_{j}))>$; $\det (B(v_{i},v_{j}))$ is determined up to
multiplication by an element of
$\ub{o}^{x^{2}}=\{x^{2}|x\in\ub{o}^{x}\}$. Now
$\det(B(v_{i},v_{j}))\ub{o}^{x^{2}}$ is called the {\em discriminant}
of $M$ and denoted by $d(M)$ (= disc $(Q)$ in $[S]$). If $d(M)\neq 0$,
then we say that $M$ is {\em regular} (= non-degenerate\pageoriginale
in $[S]$]. We write $d(M)=(\det (B(v_{i},v_{j}))$ if there is no
  ambiguity.


\subsection{}\label{c2:subsec2.0.2}
Let $M$, $M'$ be quadratic modules over $\ub{o}$. If $f$ is
  an injective linear mapping form $M$ to $M'$ which satisfies
$$
Q(f(x))=Q(x)\text{ \ for \ } x\in M,
$$
then $f$ is called an {\em isometry} from $M$ to $M'$ (= injective
metric morphism in $[S]$), and we say that $M$ is represented by
$M'$. If, moreover, $f$ is surjective, then $M$ and $M'$ are called
isometric (= isomorphic in $[S]$) and we write $f:M\cong M'$ (or $M\cong
M'$). The group of all isometries from $M$ onto $M$ is denoted by
$O(M)$; $0^{+}(M)$ stands for $\{x\in 0(M)|\det x=1\}$.


\subsubsection{}\label{c2:subsec2.0.3}
Let $M$ be a quadratic module over $\ub{o}$ and suppose that
$M$ is the direct sum of submodules $M_{1},\ldots,M_{n}$. If, for any
different indices $i$, $j$,
$$
B(x,y)=0\text{ \ for \ } x\in M_{i}\text{ \ and \ } y\in M_{j},
$$
then we write
$$
M=M_{1}\perp \ldots\perp M_{n}.
$$
($\hat{\oplus}$ is used in [S] instead of $\perp$).

\subsubsection{}\label{c2:subsec2.0.4}
Let $M$ be a quadratic module over $\ub{o}$ and $N$ a subset
  of $M$. We denote by $N^{\perp}$ ($=N^{0}$ in [S]) the orthogonal
  complement of $N$, \iec
$$
N^{\perp}=\{x\in M|B(x,N)=0\}.
$$

\subsubsection{}\label{c2:subsec2.0.5}
Let $V$ be a quadratic module over $k$ and $M$ an
$\ub{o}$-module on $V$. We call $M$ a $(\ub{o}-)$ {\em lattice} if
$kM=V$. 

\subsubsection{}\label{c2:subsec2.0.6}
Let\pageoriginale $M$ be a quadratic module over $\ub{o}$ and suppose
that $M$ contains a non-zero {\em isotropic vector} $x$, that is,
$M\ni x\neq 0$, $Q(x)=0$. Then $M$ is called an {\em isotropic}
quadratic module. (This definition is different from $[S]$.) If $M$
contains no (non-zero) isotropic element, $M$ is said to be {\em
  anisotropic}. 

\subsubsection{}\label{c2:subsec2.0.7}
Let $K\supset k$ be fields and $K\supset \ub{\widetilde{o}}$,
$k\supset\ub{o}$ rings and suppose that $\ub{\widetilde{o}}\supset
\ub{o}$. For a quadratic module $M$ over $\ub{o}$,
$\ub{\widetilde{o}}M$ denotes a canonically induced quadratic module
$\ub{\widetilde{o}}\mathop{\otimes}\limits_{\ub{o}}M$ over
$\ub{\widetilde{o}}$. Let $M$ (\resp $V$) be a quadratic module over
$\mathbb{Z}$ (\resp $\mathbb{Q}$). For a prime number $p$, we denote
by $M_{p}$, $V_{p}$ quadratic modules $\mathbb{Z}_{p}M$,
$\mathbb{Q}_{p}V$ respectively. For $p=\infty$, we write
$\mathbb{R}M$, $\mathbb{R}V$ for $M_{\infty}$, $V_{\infty}$.

\section{Quadratic Modules Over $\mathbb{Q}_{p}$}\label{c2:sec-2.1}

In this paragraph, $p$ is a prime number and we denote by
$\ub{o}=\mathbb{Z}_{p}$, $k=\mathbb{Q}_{p}$ the ring of $p$-adic
integers and the field of $p$-adic numbers.

\subsection{}\label{c2:subsec2.1.1}
Let $V$ be a regular quadratic module over $k$. Suppose
$$
V=<a_{1}>\perp\ldots\perp <a_{n}>(a_{i}\in k^{x}),
$$
that is, there is a basis $\{v_{i}\}$ such that $Q(v_{i})=a_{i}$,
$B(v_{i},v_{j})=0$ for $i\neq j$. Then $S(V)=\prod\limits_{i\leq
  j}(a_{i},a_{j})(=\varepsilon(V)(dV,-1)$ in the sense of $[S]$) where ( ,
) - the Hilbert symbol of $k$-is an invariant of $V$ and we quote the
following theorem from ($[S]$, p.39).

\begin{theorem}\label{c2:thm-2.1}
Regular quadratic modules over $k$ are classified by $d(V)$, $S(V)$,
$\dim V$.
\end{theorem}

\begin{coro*}
Let $V$, $W$ be regular quadratic modules over $k$. If\pageoriginale
$\dim V+3\leq \dim W$, then $V$ is represented by $W$.
\end{coro*}

\begin{proof}
Without loss of generality, we may assume $\dim V+3=\dim W$. Let $a$,
$b$, $c$ be non-zero elements of $k$ which satisfy
$$
\begin{cases}
ck^{x^{2}}=d(V)\cdot d(W),\\
-ac \not\in k^{x^{2}},\\
S(W)=(c,d(V))(a,c)(ab,ac)(bc,-1)S(V).
\end{cases}
$$
and put $W'=<a>\perp <ab>\perp <bc>\perp V$.

After simple manipulations, we get
$$
d(W)=d(W'), S(W)=S(W'), \dim W=\dim W'.
$$
The theorem implies that $W\cong W'$. 
\end{proof}

\subsection{Modular and Maximal Lattices}\label{c2:subsec2.1.2}

Let $M$ be a regular quadratic module over $\ub{o}$.

By the {\em scale} $s(M)$ (\resp the norm $\ub{n}(M)$) of $M$ we mean
an $\ub{o}$-module in $k$ generated by
$$
B(x,y)\quad\text{for}\quad x,y\in M(\text{\resp } Q(x) \quad\text{for}\quad
x\in M).
$$
$2s(M)\subset \ub{n}(M)\subset s(M)$ follows from
$Q(x+y)-Q(x)-Qy)=2B(x,y)$ and $Q(x)=B(x,x)$. Hence $\ub{n}(M)$ is
$s(M)$ or $2s(M)$.

If there exist $a\in k^{x}$ and a symmetric matrix
$A\in\mathscr{M}_{n}(\ub{o})$ with $\det A\in\ub{o}^{x}$ such that
$$
M=<aA>,
$$
then\pageoriginale we call $M$ ($(a)$-) {\em modular}. When
$a\in\ub{o}^{x}$, $M$ is said to be {\em unimodular.}

If $M$ is $(a)$-modular, then $s(M)$ is equal to $(a)$. We call $M$
($(a)$-) maximal $(a\in k^{x})$ if $n(M)\subset (a)$ and $M$ is the
only lattice $N$ which satisfies $M\subset N\subset kM$ and
$\ub{n}(N)\subset (a)$.

The fundamental fact on maximal lattices is the following

\begin{theorem}\label{c2:thm-2.2}
Let $V$ be a regular quadratic module over $k$ and $a\in k^{x}$. If
$M$, $N$ are $(a)$-maximal lattices on $V$, then $M$, $N$ are isometric.
\end{theorem}

To prove this, we need several lemmas.

\begin{lemma}\label{c2:lem-2.3}
Let $V$ be a regular quadratic module over $k$ with $\dim V=n$ and $M$
a lattice on $V$. If $\ub{n}(M)\subset (a)(a\in k^{x})$ and
$(2^{n}a^{-n}d(M))=\ub{o}$ or $(p)$, then $M$ is $(a)$-maximal.
\end{lemma}

\begin{proof}
Suppose that a lattice $N$ on $V$ contains $M$ and $\ub{n}(N)\subset
(a)$. Then $d(M)=[N:M]^{2}d(N)$, as is obvious. Since $(d(N))\subset
s(N)^{n}\subset (2^{-1}\break\ub{n}(N))^{n}\subset (a/2)^{n}$, we have
$(2^{n}a^{-n}d(N))\subset\ub{o}$. Then it implies
$$
\ub{o}\quad\text{or}\quad
(p)=(2^{n}a^{-n}d(M))=[N:M]^{2}(2^{n}a^{-n}d(N))\subset
   [N:M]^{2}\ub{o}.
$$
From this it follows that $[N:M]=1$ and $M$ is maximal.
\end{proof}

\begin{coro*}
If $M$ is a unimodular lattice with $\ub{n}(M)\subset (2)$, then $M$
is $(2)$-maximal. 
\end{coro*}

\begin{proof}
Since $\ub{n}(M)=(2)$ follows, Lemma \ref{c2:lem-2.3} yields immediately
the corollary.
\end{proof}

\begin{lemma}\label{c2:lem-2.4}
Let $V=<\left(\begin{smallmatrix} 0 & 1\\ 1 & 0
\end{smallmatrix}\right)>$ be a hyperbolic plane over $k$ and $M$ a
lattice on $V$. The following assertions are equivalent:
\begin{enumerate}
\renewcommand{\labelenumi}{\rm(\theenumi)}
\item $M$ is $(2a)$-maximal $(a\in k^{x})$,

\item $M$ is $(a)$-modular with $\ub{n}(M)\subset (2a)(=2s(M))$,

\item $M=<\left(\begin{smallmatrix} 0 & a \\ a & 0
\end{smallmatrix}\right)>$.\pageoriginale
\end{enumerate}
\end{lemma}

\begin{proof}
(3) $\Rightarrow$ (2) is trivial.

(2) $\Rightarrow$ (1)~: $n(M)=(2a)$, $(dM)=(a^{2})$ and Lemma
  1 complete this step.

(1) $\Rightarrow$ (3)~: Since any isotropic primitive vector of $M$ is
  extended to a basis of $M$, there exists a basis $\{e_{i}\}$ of $M$
  such that $(B(e_{i},e_{j}))=\left(\begin{smallmatrix} 0 & b\\ b & c
  \end{smallmatrix}\right)$, $b$, $c\in k$. $Q(e_{2})=c$,
  $Q(e_{1}+e_{2})=2b+c\in n(M)\subset (2a)$ imply $c\in (2a)$, $b\in
  (a)$. Suppose $bp^{-1}\in (a)$. Since
  $Q(a_{1}p^{-1}e_{1}+a_{2}e_{2})=2a_{1}a_{2}p^{-1}b+a^{2}_{2}c\in(2a)$
  for $a_{1}$, $a_{2}\in \ub{o}$,
  $M\subsetneqq L=\ub{o}[p^{-1}e_{1},e_{2}]$ and
  $n[L]\subset (2a)$. This is a contradiction. Therefore we have
  $a=bu(u\in \ub{o}^{x})$ and
  $M=\ub{o}[ue_{1},e_{2}-\dfrac{c}{2b}e_{1}]=<\left(\begin{smallmatrix}
    0 & a\\ a & 0  \end{smallmatrix}\right)>$.
\end{proof}

\begin{lemma}\label{c2:lem-2.5}
Let $V$ be a regular quadratic module over $k$ and $M$ a lattice on
$V$. Suppose that $L$ is a modular $\ub{o}$-module in
$M$. $B(L,M)\subset s(L)$ if and only if $M=L\perp K$ for some module $K$.
\end{lemma}

\begin{proof}
Let $s(L)=(a)$. Suppose that $M=L\perp K$. Then
$B(L,M)=B(L,L)=s(L)$. Conversely, suppose $B(L,M)\subset (a)$. We
define a submodule $L$ by $L^{\perp}=\{x\in M|B(L,x)=0\}$. Then
$L\perp L^{\perp}\subset M$ and $kL\perp kL^{\perp}=kM$. Take any
element $x\in M$ and decompose $x$ as $x=y+z(y\in kL, z\in
kL^{\perp})$. Then $B(L,y)=B(L,x)\subset B(L,M)\subset (a)$. Let
$\{v_{j}\}$ be a basis of $L$, then $(B(v_{i},v_{j}))=a(a_{ij})$,
$\det (a_{ij})\in\ub{o}^{x}$ for $a_{ij}\in\ub{o}$. Put $y=\sum
c_{j}v_{j}(c_{j}\in k)$ and $B(v_{i},y)=aa_{i}(a_{i}\in\ub{o})$. These
imply $(c_{1},\ldots,c_{n})a(a_{ij})=a(a_{1},\ldots,a_{n})$ and then
$c_{i}\in\ub{o}$. Hence we have $y\in L$, and $z\in L^{\perp}$ with
$L\subset M$. Thus $L\perp L^{\perp}=M$ follows.
\end{proof}

\begin{lemma}\label{c2:lem-2.6}
Let\pageoriginale $V$ be a regular quadratic module over $k$ and $M$
an $(a)$-maximal lattice on $V$. For an isotropic primitive element
$x$ of $M$, there is an isotropic element $y$ of $M$ such that
$M=\ub{o}[x,y]\perp \ast$, $\ub{o}[x,y]=<\left(\begin{smallmatrix} 0 &
  a/2\\ a/2 & 0\end{smallmatrix}\right)>$.
\end{lemma}

\begin{proof}
By definition, $B(x,M)\subset s(M)\subset \dfrac{1}{2}\ub{n}(M)\subset
\dfrac{1}{2}(a)$ holds. Suppose $B(x,M)\subset
\dfrac{1}{2}(pa)$. Then, for every $w\in M$, we have
$Q(w+p^{-1}x)=Q(w)+2p^{-1}B(w,x)\in(a)$. Hence
$\ub{n}(M+p^{-1}\ub{o}x)\subset (a)$ follows. This contradicts $M$
being $(a)$-maximal since
$M+p^{-1}\ub{o}x\supsetneqq M$. Taking an element $z\in
M$ such that $B(x,z)=\dfrac{1}{2}a$, we put $y=z-a^{-1}Q(z)x\in M$;
$\ub{o}[x,y]=<\dfrac{a}{2}\left(\begin{smallmatrix} 0 & 1\\ 1 & 0
\end{smallmatrix}\right)>\subset M$ is $(a/2)$-modular and
$B(\ub{o}[x,y],M)\subset s(M)\subset \left(\dfrac{a}{2}\right)$. We
may now apply Lemma \ref{c2:lem-2.5} to complete the proof.
\end{proof}

\begin{lemma}\label{c2:lem-2.7}
Let $V$ be an anisotropic quadratic module over $k$ and $M$ an
$(a)$-maximal lattice. Then we have
$$
M=\{x\in V|Q(x)\in(a)\}.
$$
\end{lemma}

\begin{proof}
We have only to prove $Q(x+y)\in(a)$ if $Q(x)$, $Q(y)\in(a)$. Suppose
that $2B(x,y)\not\in(a)$ for some $x$, $y\in V$ with $Q(x)$,
$Q(y)\in(a)$. Then $(2B(x,y)p^{n})=(a)$ for some $n\geq 1$. This
implies
$$
d(x,y)=Q(x)Q(y)-B(x,y)^{2}=-B(x,y)^{2}(1-Q(x)Q(y)/B(x,y)^{2}),
$$
and $(Q(x)Q(y)B(x,y)^{-2})=(Q(x)Q(y)a^{-2}4p^{2n})\subset
(4p^{2n})$. Hence\break $-d(x,y)\in k^{x^{2}}$ follows and then $k[x,y]$
is a hyperbolic plane and $V$ is isotropic. This is a
contradiction. Thus $2B(x,y)\in (a)$ and $Q(x+y)\in(a)$.
\end{proof}

\begin{lemma}\label{c2:lem-2.8}
Let $V$ be a regular quadratic module over $k$ and $M$ an
$(a)$-maximal lattice on $V$. Then there are hyperbolic planes
$H_{i}$, and an anisotropic submodule $V_{0}$ of\pageoriginale $V$
such that
\begin{align*}
& V = \perp H_{i}\perp V_{0},\\
& M = \perp (M\cap H_{i})\perp (M\cap V_{0}),\\
& M \cap H_{i}=<
\begin{pmatrix}
0 & a/2\\
a/2 & 0
\end{pmatrix}>,\\
& M\cap V_{0}=\{x\in V_{0}|Q(x)\in (a)\}.
\end{align*}
\end{lemma}

\begin{proof}
This follows inductively from Lemmas \ref{c2:lem-2.6} and \ref{c2:lem-2.7}.
\end{proof}

In Lemma \ref{c2:lem-2.8}, the number of hyperbolic planes and $V_{0}$ up
to isometry are uniquely determined by Witt's theorem. This proves the
theorem.

\begin{lemma}\label{c2:lem-2.9}
Let $V$ be a regular quadratic module over $k$ and $L$ an
$\ub{o}$-submodule in $V$ with $\ub{n}(L)\subset (a)(a\in
k^{x})$. Then there exists an $(a)$-maximal lattice on $V$
containing $L$.
\end{lemma}

\begin{proof}
Suppose that $\{v_{1},\ldots,v_{n}\}$ is a basis of $L$ over $\ub{o}$,
and $\{v_{1},\ldots,v_{n},\break\ldots,v_{m}\}$ is a basis of $V$ over
$k$. Put $M=\{v_{1},\ldots,v_{n},p^{t}v_{n+1},\ldots,p^{t}v_{m}\}$. It
is easy to see $\ub{n}(M)\subset (a)$ for a sufficiently large integer
$t$. Here we note the following two facts. (i) For lattice
$K\subsetneqq N$ on $V$, $d(K)/d(N)\equiv 0\mod
p^{2}$. (ii) For a lattice $K$ on $V$ with $n(K)\subset (a)$,
$d(K)\subset s(K)^{m}\subset (\dfrac{1}{2}\ub{n}(K))^{m}\subset
(a/2)^{m}$. If $M$ is not (a)-maximal, then there is a lattice $M_{1}$
on $V$ with $M\subset M_{1}$. If $M_{1}$ is not (a)-maximal, repeat
the preceding step and continue in this way. However, this process
must terminate at a finite stage, and the last lattice is (a)-maximal.  
\end{proof}

\setcounter{prop}{9}
\begin{prop}\label{c2:prop-2.10}
Let $V$, $W$ be regular quadratic modules over $k$ with $\dim V+3\leq
\dim W$, and $M$ a maximal lattice on $W$. Then every lattice $L$ on
$V$ is represented\pageoriginale by $M$ if $\ub{n}(L)\subset \ub{n}(M)$.
\end{prop}

\begin{proof}
From the Corollary to Theorem \ref{c2:thm-2.1}, $V$ is represented by
$W$. Theorem \ref{c2:thm-2.2} and Lemma \ref{c2:lem-2.9} imply the proposition.
\end{proof}

\subsection{Jordan Splittings}\label{c2:subsec2.1.3}

Let $L$ be a regular quadratic module over $\ub{o}$. We claim that $L$
is an orthogonal sum of modular modules of rank $1$ or $2$. Suppose
that there is an element $x\in L$ with $(Q(x))=s(L)$. Then, since
$\ub{o}x$ is $(Q(x))$-modular, Lemma \ref{c2:lem-2.5} implies
$L=\ub{o}x\perp \ast$. Next, suppose that $(Q(x))\neq s(L)$ for every
$x\in L$. Since $Q(x)=B(x,x)\in s(L)$ for $x\in L$, we have $Q(x)\in
ps(L)$ for $x\in L$. Hence, for $x$, $y\in L$ with $(B(x,y))=s(L)$, it
is obvious that $\ub{o}[x,y)$ is $s(L)$-modular. Again by the same
  lemma, $L$ is split by $\ub{o}[x,y]$. Grouping modular components of
  the above splitting, we have a Jordan splitting
$$
(\sharp)\quad L=L_{1}\perp\ldots\perp L_{t}.
$$
where every $L_{i}$ is modular and
$s(L_{1})\supsetneqq \ldots
\supsetneqq s(L_{t})$. 

For a quadratic module $M$ we put $M(a)=\{x\in M|B(x,M)\subset
(a)\}(a\in k)$. Suppose that $M$ is $(b)$-modular. Then it is easy to
see $M(a)=M$ or $ab^{-1}M$ according as $(b)\subset (a)$ or
$(b)\supsetneqq (a)$ respectively. Hence
$s(M(a))\subset (a)$; further $s(M(a))=(a)$ if and only if
$(a)=(b)$. On the other hand, we have $L(a)=L_{1}(a)\perp\ldots\perp
L_{t}(a)$ for $(\sharp)$. The above argument implies $s(L(a))=(a)$ if
and only if $(a)=s(L_{i})$ for some $i$. Thus the number $t$ and
$s(L_{i})$ in the decomposition $(\sharp)$ are uniquely
determined. Fix any $i$ and take $a\in k^{x}$ with
$(a)=s(L_{i})$. Then $B(L_{j}(a), L_{j}(a))\subset (pa)$ for $j\neq
i$; further $L_{i}(a)=L_{i}$ is $(a)$-modular. Set 
$V=L(a)/pL(a)$ and $B'(x,y)=a^{-1}B(x,y)\in\mathbb{Z}/(p)$ for $x$,
$y\in V$. Then $V$ is a vector space over $\mathbb{Z}/(p)$ and $B'$ is
a symmetric bilinear form. $B'$ is identically zero on the images of
$L_{j}(a)(j\neq i)$ on $V$ and gives a regular matrix on\pageoriginale
the image of $L_{i}(a)$ on $V$. Hence we get $\dim\{x\in
V|B'(x,V)=0\}=\sum\limits_{j\neq i}\rank L_{j}$. Thus rank $L_{i}$ is
also uniquely determined by $L$. If $\ub{n}(L_{i})\neq s(L_{i})$, then
$p=2$ and $2s(L_{i})=\ub{n}(L_{i})$, and it is the case if and only if
$B'(x,x)$ is identically zero for $x\in V$. This condition being
satisfied or not is determined by $L$ and $s(L_{i})$. Thus we have
proved

\begin{prop}\label{c2:prop-2.11}
Let $L$ be a regular quadratic module over $\ub{o}$. Then there is a
decomposition
$$
L=L_{1}\perp\ldots\perp L_{t},
$$
where every $L_{i}$ is modular and
$s(L_{1})\supsetneqq \ldots \supsetneqq 
s(L_{t})$. Moreover the number $t$, $s(L_{i})$, $\rank L_{i}$ and the
equality of $\ub{n}(L_{i})$ and $s(L_{i})$ or otherwise are uniquely
determined by $L$.
\end{prop}

\begin{prop}\label{c2:prop-2.12}
Suppose $p\neq 2$. Let $L$ be a unimodular quadratic module over
$\ub{o}$. Then $L=<1>\perp\ldots\perp<1>\perp <d(L)>$. If rank $L\geqq
3$, $Q(L)=\ub{o}$.
\end{prop}

\begin{proof}
For a unimodular module $M$, suppose $(Q(x))\neq \ub{o}(=s(M))$ for
every $x\in M$. Then we have $\ub{o}=s(M)\subset
\dfrac{1}{2}\ub{n}(M)\subset (p/2)$. This is a contradiction. Thus the
proof of the previous propositions shows $L=<u_{1}>\perp
L_{1}(u_{1}\in\ub{o}^{x})$. Since $L$ is unimodular $L_{1}$ is also
unimodular. Repeating this argument, we have
$L=<u_{1}>\perp\ldots\perp <u_{n}>(u_{i}\in\ub{o}^{x})$. Since the
Hasse invariant of $kL$ is $1$, $<u_{1}>\perp\ldots\perp <u_{n}>$ and
$<1>\perp\ldots\perp<1>\perp <d(L)>$ are isometric over $k$ after the
extension of coefficient ring from $\ub{o}$ to $k$, and they are
$\ub{o}$-maximal by Corollary to Lemma \ref{c2:lem-2.3}. Hence they are
isometric, by Theorem \ref{c2:thm-2.2}. If rank $L\geqq 3$, then $L\cong
<\left(\begin{smallmatrix} 0 & 1\\ 1 & 0\end{smallmatrix}\right)>\perp
  \ast$ holds. Therefore it follows that $Q(L)=\ub{o}$. 
\end{proof}

\begin{prop}\label{c2:prop-2.13}
Suppose\pageoriginale $p=2$. Let $L$ be a unimodular quadratic module
over $\ub{o}$. $L$ has an orthogonal basis if and only if
$\ub{n}(L)=s(L)$. Otherwise $L$ is an orthogonal sum of
$<\left(\begin{smallmatrix} 0 & 1\\ 1 & 0\end{smallmatrix}\right)>$,
  $<\left(\begin{smallmatrix} 2 & 1\\ 1 &
    2 \end{smallmatrix}\right)>$, $<\left(\begin{smallmatrix} 2 &
    1\\ 1 & 2\end{smallmatrix}\right)>\perp <\left(\begin{smallmatrix}
      2 & 1\\ 1 & 2    \end{smallmatrix}\right)>$ is isometric to
    $<\left(\begin{smallmatrix} 0 & 1\\ 1 & 0
    \end{smallmatrix}\right)>\perp <\left(\begin{smallmatrix} 0 &
      1\\ 1 & 0    \end{smallmatrix}\right)>$.
\end{prop}

\begin{proof}
As in the proof of Proposition \ref{c2:prop-2.13}, we have a
decomposition
$$
L=L_{1}\perp L_{2},
$$
where $L_{1}=<u_{1}>\perp\ldots\perp<u_{t}>(u_{i}\in\ub{o}^{x})$,
$L_{2}$ is an orthogonal sum of $<\left(\begin{smallmatrix} 2a_{i} &
  b_{i}\\ b_{i} &
  2c_{i}\end{smallmatrix}\right)>(a_{i},c_{i}\in\ub{o},b_{i},\in\ub{o}^{x})$. Moreover,
$n(L)=s(L)$ if and only if $\rank L_{1}\geqq 1$. Suppose
$M=\ub{o}x_{1}\perp \ub{o}[x_{2},x_{3}]$ and
$Q(x_{1})\in\ub{o}^{x}$,
$(B(x_{i},x_{j}))_{i,j=2,3}=\left(\begin{smallmatrix} 2a & b\\ b & 2c
\end{smallmatrix}\right)$, $a$, $b$, $c\in\ub{o}$,
$b\in\ub{o}^{x}$. Then $N=\ub{o}[x_{1}+x_{2},x_{3}]$ is unimodular
and $Q(x_{1}+x_{2})\in\ub{o}^{x}$. The proof of Proposition
\ref{c2:prop-2.11}  shows that $N$ has an orthogonal basis and $M$ is
isometric to $N\perp\ast$ by Lemma \ref{c2:lem-2.5}. Thus $M$ has an
orthogonal basis. This proves the first assertion. Let
$K=\ub{o}[v_{1},v_{2}]$, $(B(v_{i},v_{j}))=\left(\begin{smallmatrix}
  2a & b\\ b & 2c\end{smallmatrix}\right)(a,b,c\in\ub{o},b\in
  \ub{o}^{x})$. Then $kK$ is isometric to $<2a>\perp <2ad>$,
  $d=4ac-b^{2}\equiv -1\mod 4$. After easy manipulations, the Hasse
  invariant $S(kK)$ is $1$ (\resp $-1$) according as $d\equiv 3$
  (\resp 7) $\mod 8$. Hence by virtue of Theorem \ref{c2:thm-2.1}, $kK$
  is isometric to $<\left(\begin{smallmatrix} 2 & 1\\ 1 & 2
  \end{smallmatrix}\right)>$ (\resp $<\left(\begin{smallmatrix} 0 &
    1\\ 1 & 0  \end{smallmatrix}\right)>$) if $d\equiv 3$ (\resp 7)
  $\mod 8$. Since they are $(2)$-maximal by Lemma \ref{c2:lem-2.3}, they
  are isometric by Theorem \ref{c2:thm-2.2}. By Theorem \ref{c2:thm-2.1}
  again, it is easy to see $<\left(\begin{smallmatrix} 2 & 1\\ 1 & 2
  \end{smallmatrix}\right)>\perp<\left(\begin{smallmatrix} 2 & 1\\ 1 &
    2
  \end{smallmatrix}\right)>\perp<\left(\begin{smallmatrix} 0 & 1\\ 1 &
    0
  \end{smallmatrix}\right)>$ over $k$. Over $\ub{o}$, they are
  $(2)$-maximal by Lemma \ref{c2:lem-2.3} and then they are isometric.
\end{proof}

\subsection{Extension Theorems}\label{c2:subsec2.1.4}\pageoriginale

Let $V$, $W$ be quadratic modules over $k$ and $M$ a $(\ub{o})$-lattice on
$V$. Suppose that
$$
u:M\to W
$$
is a linear mapping over $\ub{o}$. Then, putting, for $w\in W$,
$$
B_{u}(w)(x)=B(u(x),w)\quad\text{for}\quad x\in M
$$
we obtain $B_{u}(w)\in \Hom(M,k)$. The following theorem is
fundamental.

\setcounter{theorem}{13}
\begin{theorem}\label{c2:thm-2.14}
Suppose that there is an $o$-submodule $G$ in $W$ such that, for
$k\in\mathbb{Z}$, the conditions
$$
(\sharp)_{k}
\begin{cases}
\Hom(M,\ub{o})=\{B_{u}(w)|w\in G\}+\Hom(M,po),\\
p^{k-1}n(G)\subset 2\ub{o},\\
Q(u(x))\equiv Q(x)\mod 2p^{k}\ub{o}\quad\text{for}\quad x\in M
\end{cases}
$$
are satisfied. Then there is an element $u'\in \Hom(M,W)$ satisfying
$$
u'(x)\equiv u(x)\mod p^{k}G\quad\text{for}\quad x\in M\quad{and}\quad (\sharp)_{k+1}.
$$
If, moreover, $V$ is regular, there is an isometry $u_{0}$ from $M$ to
$W$ satisfying
$$
u_{0}(x)\equiv u(x)\mod p^{k}G\quad\text{for}\quad x\in M.
$$
\end{theorem}

\begin{proof}
Let $\{v_{1},\ldots,v_{m}\}$ be a basis of $M$. Put
\begin{align*}
a\left(\sum x_{i}v_{i},\sum y_{i}v_{i}\right)
&= \frac{1}{2}p^{-k}\sum_{i}(Q(u(v_{i}))-Q(v_{i}))x_{i}y_{i}+p^{-k}\\
&\qquad\sum_{i<j}(B(u(v_{i}),u(v_{j}))-B(v_{i},v_{j}))x_{i}y_{j}.
\end{align*}
Since\pageoriginale $Q(\sum w_{i})=Q(w_{i})+2\sum\limits_{i<j}B(w_{i},w_{j})$, we
have
$$
2p^{k}a(x,x)=Q(u(x))-Q(x)\quad\text{for}\quad x\in M.
$$
It is obvious that
\begin{align*}
& \frac{1}{2}p^{-k}(Q(u(v_{i}))-Q(v_{i}))\in\ub{o}\quad\text{and}\\
 p^{-k}\{B(u(v_{i}),u(v_{j}))-B(v_{i},v_{j})\}
 &=\frac{1}{2}p^{-k}\{Q(u(v_{i}+v_{j}))-Q(u(v_{i}))\\
&\qquad-Q(u(v_{j}))-Q(v_{i}+v_{j})+Q(v_{i})\\
&\qquad\qquad +Q(v_{j})\}\in\ub{o}.
\end{align*}
Thus $a(x,y)$ is an $\ub{o}$-valued bilinear form on $M$. Therefore,
for each $i$, there exist $g_{i}\in G$ and $m_{i}\in\Hom(M,p\ub{o})$
such that
$$
a(x,v_{i})=B(u(x),g_{i})+m_{i}(x)\quad\text{for}\quad x\in M.
$$
Making use of $g_{i}$, we define $v\in\Hom(M,G)$ by
$$
v(\sum x_{i}v_{i})=-\sum x_{i}g_{i} \; (x_{i}\in\ub{o}).
$$
We put $u'(x)=u(x)+p^{k}v(x)$. Then $u'(x)\equiv u(x)\mod p^{k}G$ is
obvious for $x\in M$. We must verify the property $(\sharp)_{k+1}$ for
$u'$. For $x\in M$, $w\in G$, we have
\begin{align*}
B_{u'}(w)(x) &= B(u'(x),w)=B(u(x),w)+p^{k}B(v(x),w)\\
&= B_{u}(w)(x)+p^{k}B(v(x),w).
\end{align*}
Here the linear mapping $x\to p^{k}B(v(x),w)$ is in $\Hom(M,po)$ since
$p^{k}B(v(x),w)\in p^{k}s(G)\in\dfrac{1}{2}p^{k}\ub{n}(G)\subset
p\ub{o}$. Hence the first equation is valid for $u'$.

For $x=\sum x_{i}v_{i}\in M$, we have
\begin{align*}
Q(u'(x)) &=Q(u(x))+p^{2k}Q(v(x))+2p^{k}B(u(x),v(x))\\
&= Q(u(x))+p^{2k}Q(v(x))+2p^{k}(-\sum x_{i}B(u(x),g_{i}))\\
&= Q(u(x))+p^{2k}Q(v(x))-2p^{k}(a(x,x)-\sum x_{i}m_{i}(x))\\
&= Q(x)+p^{2k}Q(v(x))+2p^{k}\sum x_{i}m_{i}(x).
\end{align*}\pageoriginale
Here $p^{2k}Q(v(x))\in p^{2k}\ub{n}(G)\subset 2p^{k+1}\ub{o}$,
$2p^{k}\sum x_{i}m_{i}(x)\in 2p^{k+1}\ub{o}$ hold. Thus the third
property of $(\sharp)_{k+1}$ holds for $u'$, and the former part of
Theorem \ref{c2:thm-2.14} is proved. Repeating this argument inductively,
there is an element $u_{\ell}\in\Hom(M,W)(\ell\geq 1)$ satisfying
\begin{align*}
Q(u_{\ell}(x)) &\equiv Q(x)\mod 2p^{k+\ell}\ub{o}\quad\text{for}\quad
x\in M,\\
u_{\ell} (x) &\equiv u(x)\mod p^{k}G\quad\text{for}\quad x\in M. 
\end{align*}
Hence there is an element $u_{0}\in \Hom(M,W)$ such that
\begin{align*}
Q(u_{0}(x))=Q(x)\quad\text{and}\quad u_{0}(x)\equiv u(x)\mod
p^{k}G\quad\text{for}\quad x\in M.
\end{align*}
Suppose $u_{0}(y)=0$ for $y\in M$. Then we have
$$
B(y,M)=B(u_{0}(y),u_{0}(M))=0.
$$
If $V$ is regular, then $y=0$ follows. Hence $u_{0}$ is injective and
indeed an isometry. This completes the proof of Theorem \ref{c2:thm-2.14}.
\end{proof}

\begin{defi*}
Let $V$ be a quadratic module over $k$ and $M$ a lattice on $V$. Then
we denote by $M^{\sharp}$
$$
\{x\in V|B(x,M)\subset \ub{o}\}.
$$
\end{defi*}

\begin{cor}\label{c2:coro-1}
Let $V$, $W$ be regular quadratic modules over $k$ and $M$, $N$
lattices on\pageoriginale $V$, $M$ respectively. Let $h$ be an integer
such that
$$
p^{h}\ub{n}(M^{\sharp})\subset 2\ub{o}.
$$
If $u\in \Hom(M,N)$ satisfies
$$
Q(x)\equiv Q(u(x))\mod 2p^{h+1}\ub{o}\quad\text{for}\quad x\in M,
$$
then there exists an isometry $u'$ from $M$ to $N$ such that
\begin{align*}
& u'(M) = u(M)\\
& u'(x) \equiv u(x)\mod p^{h+1}u(M^{\sharp})\quad\text{for}\quad x\in M.
\end{align*}
In particular, we have $u':M\cong u(M)$.
\end{cor}

\begin{proof}
We claim that $u$ is injective. Suppose that $u(x)=0$ for $0\neq x\in
M$. Without loss of generality, we may assume that $x$ is primitive in
$M$. Hence there exists $x'\in M^{\sharp}$ satisfying
$B(x,x')=1$. $2s(M^{\sharp})\subset\ub{n}(M^{\sharp})\subset
2p^{-h}\ub{o}$ implies $B(p^{h}M^{\sharp},M^{\sharp})\subset\ub{o}$
and then $p^{h}M^{\sharp}\subset (M^{\sharp})^{\sharp}=M$. Thus
$p^{h}x'$ is in $M$. From
\begin{align*}
Q(x+p^{h}x') &\equiv Q(u(p^{h}x'))\mod 2p^{h+1}\ub{o}\\
&\equiv Q(p^{h}x')\mod 2p^{h+1}\ub{o}
\end{align*}
we have $0\equiv Q(x)+2p^{h}\equiv Q(u(x))+2p^{h}\equiv 2p^{h}\mod
2p^{h+1}\ub{o}$. This is a contradiction. Thus $u$ is injective. Let
$\varphi$ be an element of $\Hom(M,\ub{o})$. Then $\varphi(x)=B(x,z)$
for some $z\in M^{\sharp}$. We show that $(\sharp)_{h+1}$ holds for
$G=u(M^{\sharp})$. For $x\in M$, we have
$$
p^{h}\varphi(x)=B(x,p^{h}z)\equiv (B(u(x),p^{h}u(z))\mod p^{h+1}\ub{o}
$$
and then $\varphi(x)\equiv B(u(x),u(z))\mod
p\ub{o}$. Thus\pageoriginale the first condition holds. For $x\in
M^{\sharp}$, we have
\begin{align*}
& Q(p^{h}x)\equiv Q(p^{h}u(x))\quad\text{and}\quad 2p^{h+1}\ub{o}\\
\text{and}\qquad & p^{h+1}Q(x)\equiv p^{h+1}Q(u(x))\mod 2p^{2}\ub{o}.
\end{align*}
From the assumption $p^{h}\ub{n}(M^{\sharp})\subset 2\ub{o}$, it
follows that
$$
p^{h+1}Q(x)\equiv 0\mod 2p\ub{o}\quad\text{and then}\quad
p^{h+1}Q(u(x))\equiv 0\mod 2p\ub{o}.
$$
Thus $p^{h}\ub{n}(G)\subset 2\ub{o}$. By Theorem \ref{c2:thm-2.14}, there
exists an isometry $u'$ from $M$ to $W$ such that
$$
u'(x)\equiv u(x)\mod p^{h+1}u(M^{\sharp})\quad\text{for}\quad x\in M.
$$
Since $p^{h}M^{\sharp}\subset M$, $u'(x)\equiv u(x)\mod pu(M)$ for
$x\in M$. This implies $u'(M)=u(M)$.
\end{proof}

\begin{cor}\label{c2:coro-2}
Let $V$ be a regular quadratic module over $k$ and $M$ a lattice on
$V$. Let $h$ be an integer such that
$$
p^{h}\ub{n}(M^{\sharp})\subset 2\ub{o},
$$
and let $N$ be a submodule of $M$ which is a direct summand of $M$ as
a module, and suppose that $u_{0}$ is an isometry from $N$ to $M$
satisfying
$$
u_{0}(x)\equiv x\mod p^{h+1}M^{\sharp}\quad\text{for}\quad x\in N.
$$
Then $u_{0}$ extends to an isometry $u_{1}\in o(M)$ such that
$$
u_{1}(x)\equiv x\mod p^{h+1}M^{\sharp}\quad\text{for}\quad x\in M.
$$\pageoriginale
\end{cor}

\begin{proof}
We take a submodule $N'$ such that $M=N\oplus N'$. We define an
endomorphism $u$ of $M$ by
$$
u(x+x')=u_{0}(x)+x'\quad\text{for}\quad x\in N, x'\in N'.
$$
Put $G=M^{\sharp}$. By assumption, $p^{h}\ub{n}(G)\in 2\ub{o}$. For
$\varphi\in \Hom(M,\ub{o})$ there exists $z\in M^{\sharp}$ such that
$$
\varphi(x)=B(x,z)\quad\text{for}\quad x\in M.
$$
Then we have, for $x\in M$,
\begin{align*}
& \varphi(x)-B(u(x),z)\\
&= B(x-u(x),z)\in B(p^{h+1}M^{\sharp},M^{\sharp})\subset
  B(pM,M^{\sharp})\\
&\subset p\ub{o},
\end{align*}
since $p^{n}M^{\sharp}\subset M$ as in the proof of Corollary
\ref{c2:coro-1} to Theorem \ref{c2:thm-2.14}. Thus
$\Hom(M,\ub{o})=B_{u}(G)+\Hom(M,p\ub{o})$. For $x\in N$, $x'\in N'$,
we have
\begin{align*}
&\quad Q(u(x+x'))-Q(x+x')=Q(u_{0}(x)+x')-Q(x+x')\\
&=Q(u_{0}x))-Q(x)+2B(u_{0}(x)-x,x'),\text{ \ putting \ }
  u_{0}(x)-x=p^{h+1}y,\\
&= 2B(p^{h+1}y,x)+p^{2(h+1)}Q(y)+2B(p^{h+1}y, x')\in
  2p^{h+1}\ub{o}\text{ \ holds.}
\end{align*}
Thus the condition $(\sharp)_{h+1}$ in Theorem \ref{c2:thm-2.14} is
satisfied for $V=W$ and a linear mapping $u$ satisfying $u=u_{0}$ on
$N$. In the proof of this theorem, we assume that
$\{v_{1},\ldots,v_{n}\}$ (respectively $\{v_{n+1},\ldots,v_{m}\}$) is
a basis of $N$ (respectively $N'$). Then $Q(u(v_{i}))=Q(v_{i})$ for
$1\leqq i\leqq n$, and hence $a(x,y)=0$ holds for $x\in M$, $y\in
N$. Thus\pageoriginale we can take $g_{i}=u$, $m_{i}=0$ for $i\leqq
n$. This implies $v(N)=0$. Hence $u'$ constructed in Theorem
\ref{c2:thm-2.14} satisfies the condition $u'=u=u_{0}$ on $N$. Repeating
this argument, we obtain an isometry $u_{1}$ from $M$ to $V$ such that
\begin{align*}
& u_{1}(x)\equiv u(x)\equiv x\mod
  p^{h+1}M^{\sharp}\quad\text{for}\quad x\in M,\\
& u_{1}(x)=u_{0}(x)\quad\text{for}\quad x\in N.
\end{align*}
Now $p^{h}M^{\sharp}\subset M$ implies that $u_{1}(M)\subset M$ and
then $u_{1}(M)=M$, on comparing the discriminants.
\end{proof}

\begin{cor}\label{c2:cor-3}
Let $V$ be a regular quadratic module over $k$, $M$ a lattice on $V$,
and $\{u_{1},\ldots,u_{n}\}$ a set of linearly independent elements of
$V$. Then there is an integer $h$ such that for a set
$\{v_{1},\ldots,v_{n}\}$ of linearly independent elements satisfying
$B(u_{i},u_{j})=B(v_{i},v_{j})$ and $u_{i}-v_{i}\in p^{h}M$ for $1\leq
i\leq j\leq n$, there is an isometry $u\in o(M)$ such that
$u(u_{i})=v_{i}$ for $1\leq i\leq n$. 
\end{cor}

\begin{proof}
Without loss of generality, we may assume that $u_{i}\in M$ for $1\leq
in\leq n$, taking $p^{r}M$ instead of $M$. Put
$N=k[u_{1},\ldots,u_{n}]\cap M$ and let $\{w_{1},\ldots,w_{n}\}$ be a
basis of $N$ and
$$
(u_{1},\ldots,u_{n})=(w_{1},\ldots,w_{n})A\quad\text{for}\quad A\in
M_{n}(\ub{o}). 
$$
We define an isometry $u_{0}$ from $N$ to $M$ by
$u_{0}(u_{i})=v_{i}$. Then we have
\begin{align*}
&\quad
  (\ldots,u_{0}(w_{i})-w_{i},\ldots)=(\ldots,u_{0}(u_{i})-u_{i},\ldots)A^{-1}\\
& = (\ldots,v_{i}-u_{i},\ldots)A^{-1}.
\end{align*}
If $h$ is sufficiently large, then $u_{0}(x)\equiv x\mod
p^{h'+1}M^{\sharp}$ for $x\in N$ and a sufficiently large $h'$. From
the previous corollary our assertion follows.
\end{proof}

\begin{cor}\label{c2:coro-4}
Let $L$ be a regular quadratic module over $\ub{o}$ and
$x_{1},\ldots,\break x_{n}\in L$ a\pageoriginale set of elements of $L$
satisfying $\det(B(x_{i},x_{j}))\neq 0$. Then there exists an integer
$h$ such that for any $y_{1},\ldots,y_{n}\in L$ with $y_{i}\equiv
x_{i}\mod p^{h}L$, there is an isometry $\sigma\in 0(L)$ for which
$$
\sigma(\ub{o}[x_{1},\ldots,x_{n}])=\ub{o}[y_{1},\ldots,y_{n}]
$$
holds.
\end{cor}

\begin{proof}
Put $M=\ub{o}[x_{1},\ldots,x_{n}]$, $N=\ub{o}[y_{1},\ldots,y_{n}]$. We
take a sufficiently large $h$; then $\det (B(y_{i},y_{j}))/\det
(B(x_{i},x_{j}))\in \ub{o}^{x^{2}}$. Applying Corollary \ref{c2:coro-1}
to Theorem \ref{c2:thm-2.14} to $M$, $N$, $u:x_{i}\to y_{i}$, we see that
there are $z_{1},\ldots,z_{n}\in N$ such that
$B(z_{i},z_{j})=B(x_{i},x_{j})$ and $z_{i}\equiv y_{i}\mod p^{h'}L$
for a sufficiently large $h'$. From the previous corollary follows the
existence of an isometry $\sigma\in \ub{o}(L)$ such that
$\sigma(M)=\ub{o}[z_{1},\ldots,z_{n}]=N$. 
\end{proof}

\section{The Spinor Norm}\label{c2:sec2.2}

Let $k$ be a field with characteristic $\neq 2$ and $V$ a regular
quadratic module over $k$.

Let $T(V)=\mathop{\oplus}\limits_{n\geq
  0}\mathop{\otimes}\limits^{n}V(\mathop{\otimes}\limits^{0}V=k,\mathop{\otimes}\limits^{1}V=V)$
be the tensor algebra of $V$ and let $I$ be the two-sided ideal of
$T(V)$ generated by elements of the form $x\otimes x-Q(x)\in
T(V)$. Then $C(V)=T(V)/I$ is called the {\em Clifford algebra} of
$V$. It is easy to see that $C(V)$ is the direct sum of the images of
$T_{0}=\oplus (\mathop{\otimes}\limits^{n}V)$ ($n:$ even) and
$T_{1}=\oplus (\mathop{\otimes}\limits^{n}V)(n:\text{odd})$ since $I=(I\cap
T_{0})\oplus (I\cap T_{1})$.

\setcounter{lemma}{14}
\begin{lemma}\label{c2:lem-2.15}
Let $\{v_{1},\ldots,v_{n}\}$ be an orthogonal basis of $V$. Then the
centre of $C(V)$ is contained in $k+kv_{1}\ldots v_{n}$ (where
$v_{1}\ldots v_{n}$ is the product of $v_{1},\ldots,v_{n}$ in $C(V)$).
\end{lemma}

\begin{proof}
For $x$, $y\in V$, we have
$$
Q(x+y)=(x+y)(x+y)=Q(x)+Q(y)+xy+yx\quad\text{in}\quad C(V),
$$
and\pageoriginale then $xy+yx=2B(x,y)$. For a subset $S$ of
$\{1,\ldots,n\}$, we identify $S$ with $v_{i_{1}}\ldots
v_{i_{j}}(S=\{i_{1}<\ldots<i_{j}\})$. If $S=\phi$, then we take the
identity in $C(V)$. Then $x\in C(V)$ is written as
$$
x=\sum_{S}a(S)S\quad (a(S)\in k),
$$
where $S$ runs over all subsets of $\{1,\ldots,n\}$. Although it is
known that the expression is unique, \iec $\dim C(V)=2^{n}$, we do not
need to prove the lemma. Set $e(S)=1$ (\resp $-1$) if the cardinality
of $S$ is even (\resp odd). Since $v_{i}v_{j}=-v_{j}v_{i}$ for $i\neq
j$, we have, for $S\subset \{1,\ldots,n\}$,
$$
Sv_{i}=
\begin{cases}
e(S)v_{i}S\quad\text{if}\quad i\not\in S,\\
-e(S)v_{i}S\quad\text{if}\quad i\in S.
\end{cases}
$$
Hence it is easy to see that $1$ and $v_{1}\ldots v_{n}$ with $n$ odd
are in the centre of $C(V)$; moreover, $1$ and $v_{1}\ldots v_{n}$ for
odd $n$ are linearly independent, since $1\in T_{0}$ and $v_{1}\ldots
v_{n}\in T_{1}$. Let $\mathfrak{S}$ consist of all subsets of
$\{1,\ldots,n\}$, giving a basis of $C(V)$; we may assume that
$\mathfrak{S}\ni \phi$ and $\{1,\ldots,n\}$ if $n$ is odd. Suppose
that $x$ is an element of the centre of $C(V)$ and let
$$
x=\sum_{S\in\mathfrak{S}} a(S) \; S(a(S)\in k).
$$
Then $xv_{i}=v_{i}x$ implies
\begin{align*}
xv_{i} & \sum a(S)Sv_{i}\\
 &= \sum_{i\not\in S\in\mathfrak{S}}a(S)e(S)v_{i}S-\sum_{i\in
  S\in\mathfrak{S}}a(S)e(S)v_{i}S\\
&= \sum_{S\in\mathfrak{S}}a(S)v_{i}S.
\end{align*}
Multiplying\pageoriginale $v_{i}$ from the left, we have
$$
\sum_{\substack{i\not\in
    S\in\mathfrak{S}\\ e(S)=-1}}a(S)S+\sum_{\substack{i\in
    S\in\mathfrak{S}\\ e(S)=1}}a(S)S=0.
$$
Since $\mathfrak{S}$ gives a basis of $C(V)$, we have
$$
a(S)=0,
$$
if $\phi\neq S\subsetneq \{1,\ldots,n\}$, or $S=\{1,\ldots,n\}$ with
$n$ even. This completes the proof of Lemma \ref{c2:lem-2.15}.
\end{proof}

For any anisotropic vector $v\in V$ (\ie with $Q(v)\neq 0$), we define
an isometry $\tau_{v}$ of $V$ by
$$
\tau_{v}x=x-\frac{2B(x,v)}{Q(v)}v.
$$
It is called a {\em symmetry} (with respect to $v$)

\begin{lemma}\label{c2:lem-2.16}
Suppose $\tau_{u_{1}}\ldots \tau_{u_{m}}=1$. Then $Q(u_{1})\ldots
Q(u_{m})\in k^{x^{2}}$.
\end{lemma}

\begin{proof}
First, we note that $m$ is even, since $\det \tau_{u}=-1$. For an
aniso\-tropic $u\in V$ and all $x\in V$ we have
\begin{align*}
\tau_{u}x &= x-\frac{2B(u,x)}{Q(u)}u\\
&= x-Q(u)^{-1}(xu+ux)u\quad\text{in}\quad C(V)\\
&= -uxu^{-1} \; (u^{-1}=Q(u)^{-1}u\quad\text{in}\quad C(V)).
\end{align*}
Hence $\tau_{u_{1}}\ldots \tau_{u_{m}}=1$, implying that
$$
u_{1}\ldots u_{m}x=xu_{1}\ldots u_{m}\quad\text{for all}\quad x\in V.
$$
By\pageoriginale the previous lemma, we have
$$
u_{1}\ldots u_{m}=a+bv_{1}\ldots v_{n},
$$
where $a$, $b\in k$ and $\{v_{1},\ldots,v_{n}\}$ is an orthogonal
basis for $V$ and $b=0$ if $n$ is even. If $n$ is odd, then
$bv_{1}\ldots v_{n}$ is in the images of $T_{0}$ and $T_{1}$, since
$u_{1}\ldots u_{m}-a\in T_{0}$.

Hence $bv_{1}\ldots v_{n}$ is $0$ and $u_{1}\ldots u_{m}=a\in
k$. Since $x_{1}\otimes\ldots\otimes x_{t}\to
x_{t}\otimes\ldots\otimes x_{1}$ induces an anti isomorphism $f$ of
$C(V)$. Hence we have
\begin{align*}
Q(u_{1})\ldots Q(u_{m}) &= u_{1}\ldots u_{m}u_{m}\ldots u_{1}\\
&= u_{1}\ldots u_{m}f(u_{m})\ldots f(u_{1})\\
&= u_{1}\ldots u_{m}f(u_{1}\ldots u_{m})\\
&= a^{2}.\tag*{Q.E.D.}
\end{align*}
\end{proof}

The following theorem is implicitly proved in [S].

\setcounter{theorem}{16}
\begin{theorem}\label{c2:thm-2.17}
The group $O(V)$ is generated by symmetries.
\end{theorem}

Hence we can express $\sigma\in O(V)$ as a product of symmetries,
$$
\sigma=\tau_{u_{1}}\ldots \tau_{u_{m}}
$$
and denote by $\theta(\sigma)$ the element $Q(u_{1})\ldots Q(u_{m})\in
k^{x}/k^{x^{2}}$. By Lemma \ref{c2:lem-2.16}, this mapping is
well-defined and then it is obvious that $\theta$ is a group
homomorphism from $O(V)$ to $k^{x}/k^{x^{2}}\cdot \theta(\sigma)$ is
called the {\em spinor norm} of $\sigma$.

\begin{defi*}
$O'(V)=\{\sigma\in O^{+}(V)|\theta(\sigma)\in (k^{x})^{2}\}$.
\end{defi*}

\setcounter{prop}{17}
\begin{prop}\label{c2:prop-2.18}
Let $L$ be a modular or maximal regular quadratic module over
$\mathbb{Z}_{p}$ with $\rank L\geq 2$. Suppose $\rank L\geq 3$ unless
$L$ is modular with $p\neq 2$. Then $\theta(O^{+}(L))\supset
\mathbb{Z}^{x}_{p}$. 
\end{prop}

\begin{proof}
Suppose\pageoriginale that $L$ is $(a)$-modular. Let, first $p\neq
2$. Proposition \ref{c2:prop-2.12} implies
$$
<a_{1}>\perp\ldots\perp <a_{n}>\cong <b_{1}>\perp\ldots\perp <b_{n}>
$$
if $a_{i}$, $b_{i}\in\mathbb{Z}^{x}_{p}$ and $\Pi a_{i}=\Pi b_{i}$.

Hence, for each $b\in\mathbb{Z}^{x}_{p}$, there exists a decomposition
$$
L=\mathbb{Z}_{p}v\perp\ast, Q(v)=ab.
$$
Then $\tau_{v}$ induces an isometry of $L$ and
$\theta(\tau_{v})=ab\mathbb{Q}^{x^{2}}_{p}$. Therefore\break
$\theta(O^{+}(L))\supset \mathbb{Z}^{x}_{p}$. Suppose $p=2$. Let
$M=\mathbb{Z}_{2}[v_{1},v_{2}]$ and
$(B(v_{i},v_{j}))=a\left(\begin{smallmatrix} 0 & 1\\ 1 & 0
\end{smallmatrix}\right)$. For $u\in\mathbb{Z}^{x}_{2}$, it is clear
that $\tau_{v_{1}+uv_{2}}\in O(M)$ and
$\theta(\tau_{v_{1}+uv_{2}})=2au$. Hence $\theta(O^{+}(M))\supset
\mathbb{Z}^{x}_{2}\mathbb{Q}^{x^{2}}_{2}$. Next suppose that
$M=\mathbb{Z}_{2}[v_{1},v_{2}]$ and
$(B(v_{1},v_{j}))=a\left(\begin{smallmatrix} 2 & 1\\ 1 & 2
\end{smallmatrix}\right)$. Then $\mathbb{Q}_{2}M$ is anisotropic and
$M$ is $(2a)$-maximal, since $<\left(\begin{smallmatrix} 2 & 1\\ 1 & 2
\end{smallmatrix}\right)>$ is $(2)$-maximal by Lemma \ref{c2:lem-2.3}. Hence
$M=\{x\in\mathbb{Q}_{2}M|Q(x)\in(2a)\}$ and
$O^{+}(M)=O^{+}(\mathbb{Q}_{2}M)\cdot Q(v_{1}+bv_{2})=2a$, $2a.3$,
$2a.7$ and $2a.13$ according as $b=0,1,2$ and $3$ respectively. Hence
we have $\theta(O^{+}(M))\supset
\mathbb{Z}^{x}_{2}\mathbb{Q}^{x^{2}}_{2}$. Thus, to prove our
assertion, we have only to show $L=<a\left(\begin{smallmatrix} 2c &
  1\\ 1 & 2c\end{smallmatrix}\right)>\perp \ast (c=0\text{ \ or
    \ }1)$. From Proposition \ref{c2:prop-2.13}, it follows that $L$ has
  an orthogonal basis $\{v_{i}\}$ with $Q(v_{i})=au_{i}$,
  $u_{i}\in\mathbb{Z}^{x}_{2}$. Put
  $M=\mathbb{Z}_{2}[v_{1}+v_{2},v_{2}+v_{3}]=<a\left(\begin{smallmatrix}
    u_{1}+u_{2} & u_{2}\\ u_{2} &
    u_{2}+u_{3}  \end{smallmatrix}\right) >$. Then $M$ is $(a)$-modular,
  $M\subset L$ and hence $L=M\perp \ast$. Proposition \ref{c2:prop-2.13}
  now implies $<\left(\begin{smallmatrix} u_{1}+u_{2} & u_{2}\\ u_{2}
    & u_{2}+u_{3}  \end{smallmatrix}\right)>\cong
  <\left(\begin{smallmatrix} 0 & 1\\ 1 & 0
  \end{smallmatrix}\right)>$ or $<\left(\begin{smallmatrix} 2 & 1\\ 1
    & 2  \end{smallmatrix}\right)>$, and the previous assertion gives
  $\theta(O^{+}(L))\supset \theta(O^{+}(M))\supset
  \mathbb{Z}^{x}_{2}\mathbb{Q}^{x^{2}}_{2}$. 
\end{proof}

Suppose\pageoriginale that $L$ is maximal. By virtue of Lemma
\ref{c2:lem-2.8} and the previous results, we may assume that
$\mathbb{Q}_{p}L$ is anisotropic. By the same lemma, $L$ is fixed as a
set for every isometry of $\mathbb{Q}_{p}L$. Suppose rank $L\geq 4$;
then the corollary on page 37 in [S] implies
$Q(\mathbb{Q}_{p}L)=\mathbb{Q}_{p}$. Hence
$\theta(O^{+}(L))=\theta(O^{+}(\mathbb{Q}_{p}L))=\mathbb{Q}^{x}_{p}\supset
\mathbb{Z}^{x}_{p}$. Suppose rank $L=3$. From the same corollary, it
follows that $u\in Q(\mathbb{Q}_{p}L)$ if $-u\not\in
d(\mathbb{Q}_{p}L)$. Since every non-zero element in $\mathbb{Q}_{p}$
is a product of two elements $u$, $v$ with $u$, $v\in
-d(\mathbb{Q}_{p}L)$, we have
$\theta(O^{+}(L))=\theta(O^{+}(\mathbb{Q}_{p}L))=\mathbb{Q}^{x}_{p}$
again. 

\begin{prop}\label{c2:prop-2.19}
Let $V$ be a regular quadratic module over $\mathbb{Q}$ with $\dim
V\not\in 3$. Then we have
$$
\theta(O^{+}(V))=\{a\in\mathbb{Q}^{x}|a>0\quad\text{if}\quad
\mathbb{R}V\text{ \ is anisotropic}\}.
$$
\end{prop}

\begin{proof}
Suppose that $\mathbb{R}V$ is anisotropic. Then $\mathbb{R}V$ is
definite and $Q(V)\subset \{a\in \mathbb{Q}|a\geqq 0\}$ or
$\{a\in\mathbb{Q}|a\leq 0\}$. Hence the spinor norm is positive. Put
$\delta=-1$ if $\mathbb{R}V$ is positive definite, $\delta=1$
otherwise, and let $a$ be a rational number such that $a>0$ if
$\mathbb{R}V$ is anisotropic. By Theorem 6 on page
36 in [S]. $\mathbb{Q}_{p}V$ is isotropic except at a finite
number of primes. Hence we can choose $b\in\mathbb{Q}^{x}$ such that
$b>0$, and $\delta.a.b.d(V)\not\subset \mathbb{Q}^{x^{2}}_{p}$,
$\delta.b.d(V)\not\subset \mathbb{Q}^{x^{2}}_{p}$ for a prime $p$ if
$\mathbb{Q}_{p}V$ is anisotropic. Then $V\perp <\delta b>$, $V\perp
<\delta.a.b>$ are isotropic at every prime spot by the same theorem
and hence they are isotropic by the Hasse-Minkowski theorem on page
41 in [S]. By Corollary \ref{c2:coro-1} on page 33 in
        [S], $-\delta b$ and $-\delta ab$ are in $Q(V)$. Therefore
        $a\mathbb{Q}^{x^{2}}=(-\delta b)(-\delta
        ab)\mathbb{Q}^{x^{2}}\subset\theta(O^{+}(V))$. 
\end{proof}

\begin{prop}\label{c2:prop-2.20}
Let $V$ be a regular quadratic module over $\mathbb{Q}_{p}$ with $\dim
V\geqq 3$. Then $\theta(O^{+}(V))=\mathbb{Q}^{x}_{p}$.
\end{prop}

\begin{proof}
If $Q(V)=\mathbb{Q}_{p}$, the assertion is obvious. Otherwise, it
follows that $\dim V=3$\pageoriginale and if
$-a.d(V)(a\in\mathbb{Q}^{x}_{p})$ is not a square, then $V$ represents
$a$. Hence $\theta(O^{+}(V))=\mathbb{Q}^{x}_{p}$, as it is easy to see.
\end{proof}

\begin{prop}\label{c2:prop-2.21}
Let $V$ be a regular isotropic quadratic module over a field $k$ with
{\em characteristic} $\neq 2$. Then $O'(V)$ is generated by
$\tau_{x}\tau_{y}(x,y\in V, Q(x)=Q(y)\neq 0)$.
\end{prop}

\begin{proof}
Let $\Omega$ be the subgroup of $O(V)$ which is generated by
$\tau_{x}\tau_{y}(x,y\in V, Q(x)=Q(y)\neq 0)$. Then clearly
$\Omega\subset O'(V)$, and from
$\sigma\tau_{x}\tau_{y}\sigma^{-1}=\tau_{\sigma(x)}\tau_{\sigma(y)}(\sigma\in
0(V))$, it follows that $\Omega$ is a normal subgroup of $O'(V)$. Let
$V=H\perp W$ where $H=k[e_{1},e_{2}]$, $Q(e_{1})=Q(e_{2})=0$,
$B(e_{1},e_{2})=1$. Let $\sigma=\tau_{x_{1}}\ldots \tau_{x_{n}}\in
O'(V)$; then take $y_{i}\in H$ so that $Q(y_{i})=Q(x_{i})$. Since
$\tau_{x_{i}}\tau_{y_{i}}\in\Omega$, $\sigma=\tau_{y_{1}}\ldots
\tau_{y_{n}}$ in $O'(V)/\Omega$. Set $\eta=\tau_{y_{1}}\ldots
\tau_{y_{n}}$; then $\eta$ is identity on $W$, and hence $\eta|_{H}\in
O'(H)$. Since $\eta|_{H}\in O'(H)$, there exist $z_{1}$, $z_{2}\in H$
such that $\eta|_{H}=\tau_{z_{1}}\cdot \tau_{z_{2}}$ and
$Q(z_{1})Q(z_{2})=1$. Then $\eta=\tau_{z_{1}}\tau_{z_{2}}$ on
$V$. Thus we have $\sigma=\eta=1$ in $O'(V)/\Omega$ and so
$O'(V)\subset\Omega$. 
\end{proof}

\section{Hasse-Minkowski Theorem}\label{c2:sec-2.3}

This section is a complement to \S\ 3 of Chapter IV in [S].

\setcounter{theorem}{21}
\begin{theorem}\label{c2:thm-2.22}
$V$, $W$ be regular quadratic modules over $\mathbb{Q}$. If $V_{p}$,
  $V_{\infty}$ are represented by $W_{p}$, $W_{\infty}$ for every
  prime $p$, then $V$ is represented by $W$.
\end{theorem}

\begin{proof}
When $\dim V=1$, this is nothing but Corollary \ref{c2:coro-1} on page 43 in
[S]. We prove the theorem by induction on $\dim V$. Decompose $V$ as
$V=<a>\perp V_{0}$, $a\in \mathbb{Q}^{x}$. The inductive hypothesis
shows that $V_{0}$ is represented by $W$ and hence there is a
submodule $W_{0}$ in $W$ which is isometric to $V_{0}$. Since $V$ is
locally represented\pageoriginale by $W$, $<a>$ is locally represented
by $W^{\perp}_{0}:=\{x\in W|B(x,W)=0\}$, using Witt's theorem
(Corollary on page 32 in [S]). Hence $<a>$ is represented by
$W^{\perp}_{0}$. Thus $V$ is represented by $W$.
\end{proof}

\begin{coro*}
Let $V$, $W$ be regular quadratic modules over $\mathbb{Q}$ with $\dim
V+3\leq \dim W$. If $\mathbb{R}V$ is represented by $\mathbb{R}W$,
then $V$ is represented by $W$.
\end{coro*}

\begin{proof}
Corollary to Theorem \ref{c2:thm-2.1} and the above theorem yield the
assertion. 
\end{proof}

\section{Integral Theory of Quadratic Forms}\label{c2:sec2.4}

For a finite set $S=\{p_{1},\ldots p_{n}\}$ of prime numbers, we
define a $\ring \mathbb{Z}[S]$ by
$$
\mathbb{Z}[S]=\mathbb{Z}[p^{-1}_{1},\ldots,p^{-1}_{n}].
$$
If $S=\phi$, then $\mathbb{Z}[S]$ means the ring $\mathbb{Z}$ of
rational integers. We define the class, the spinor genus, and the
genus of quadratic modules.

Let $V$ be a quadratic module over $\mathbb{Q}$, $S$ a finite set of
primes, and $L$ a $\mathbb{Z}[S]$-lattice on $V$. Now we put
\begin{align*}
\cls  L &= \left\{
\begin{array}{@{}c|l}
\multirow{2}{*}{$K$} & \mathbb{Z}[S]-\text{lattice on $V$ such that
  $K=\sigma(L)$}\\
 & \text{for some $\sigma\in O(V)$}
\end{array}
\right\},\\[4pt]
\spn  L &= \left\{
\begin{array}{@{}c|l}
\multirow{3}{*}{$K$} & \mathbb{Z}[S]-\text{lattice on $V$ such that
  there exists}\\
 & \text{isometries $\sigma\in O(V)$, and $\sigma_{p}\in O'(V_{p})$}\\
 & \text{satisfying $\sigma(K_{p})=\sigma_{p}(L_{p})$ for every
  $p\not\in S$}
\end{array}\right\},\\[4pt]
\gen  L &= \left\{
\begin{array}{@{}c|l}
\multirow{2}{*}{$K$} & \mathbb{Z}[S]-\text{lattice on $V$ such that
  for every $p\not\in S$ there}\\
 & \text{is an isometry $\sigma_{p}$ satisfying $K_{p}=\sigma_{p}(L_{p})$}
\end{array}
\right\}.
\end{align*}
It\pageoriginale is obvious that $\gen L\supset \spn L\supset \cls
L$. When $K\in\cls L$, $\spn L$, $\gen L$ respectively, we say that
$K$ and $L$ belong to the same class, spinor genus, genus,
respectively.

Here we recall the fundamental relations between global lattices and
their localizations.

\begin{theorem}\label{c2:thm-2.23}
Let $V$ be a finite dimensional vector space over $\mathbb{Q}$, $S$ a
finite set of prime numbers, and $K$ a $\mathbb{Z}[S]$-lattice on
$V$. Suppose that a collection $\{L_{P}\}$ of a
$\mathbb{Z}_{p}$-lattice on $V_{p}(p\not\in S)$ is given and that
$L_{p}$ is equal to $K_{p}=\mathbb{Z}_{p}K$ for almost all ($=$ all
but a finite number of) prime numbers. Then
$M=\bigcap\limits_{p\not\in S}(V\cap L_{p})$ is a
$\mathbb{Z}[S]$-lattice on $V$ satisfying
$M_{p}=\mathbb{Z}_{p}M=L_{p}$ for every $p\not\in S$.
\end{theorem}

\subsection{}\label{c2:subsec2.4.1}

The most fundamental result is the following

\begin{theorem}\label{c2:thm-2.24}
Let $V$ be a regular quadratic module over $\mathbb{Q}$, $S$ a finite
set of prime numbers. For any $\mathbb{Z}[S]$-lattice $L$ on $V$,
$\gen L$ contains only a finite number of distinct classes.
\end{theorem}

\begin{proof}
Suppose that the assertion is proved for $S=\phi$. For $p\in S$, we
take and fix a $\mathbb{Z}_{p}$-lattice $M_{p}$ on $V_{p}$, and for
$K\in\gen L$ we put $K_{0}=\bigcap\limits_{p \not\in S}(V\cap
K_{p})\bigcap\limits_{p\in S}(V\cap M_{p})$. Then $K_{0}$ is a
$\mathbb{Z}$-lattice on $V$ and $K_{0}\in\gen L_{0}$ as is obvious. By
assumption, $\gen L_{0}$ contains only a finite number of distinct
classes $\cls K_{i}(i=1,\ldots,n)$. Hence there is an isometry
$\sigma\in O(V)$ such that $\sigma(K_{0})=K_{i}$ for some
$i=1,2,\ldots,n$, and then
$\sigma(K)=\sigma(\mathbb{Z}[S]K_{0})=\mathbb{Z}[S]K_{i}\in\gen
L$. Thus $\cls\mathbb{Z}[S]K_{i}(i=1,2,\ldots,n)$ are the only classes
contained in $\gen L$.
\end{proof}

Thus we have only to prove our assertion in case $S=\phi$. In the rest
of the proof, we assume $S=\phi$. For an integer $a\neq 0$, it is
obvious that if $\gen L=\{\cls K_{i}\}$\pageoriginale the $\gen
aL=\{\cls aK_{i}\}$. Thus we may assume $s(L)=\{\sum
B(x_{i},y_{i})|x_{i},y_{i}\in L\}\subset \mathbb{Z}$. If $K\in\gen L$,
then $d(L)=d(K)$ and $s(L)=s(K)$ since
$s(L)\mathbb{Z}_{p}=s(L_{p})$. Thus we have only to prove

\setcounter{prop}{24}
\begin{prop}\label{c2:prop-2.25}
Let $V$ be a regular quadratic module over $\mathbb{Q}$ and $d\neq 0$
an integer. Then there is only a finite number of $\cls L$ such that
$s(L)\subset \mathbb{Z}$ and $d(L)=d$.
\end{prop}

\setcounter{lemma}{25}
\begin{lemma}\label{c2:lem-2.26}
Let $V$ be a regular quadratic module over $\mathbb{Q}$ and $M$ a
$\mathbb{Z}$-lattice with $s(M)\subset \mathbb{Z}$ on $V$. Suppose
that $N$ is a regular quadratic submodule of $M$. Put
$K=N^{\perp}=\{x\in M|B(x,N)=0\}$. Then we have
$$
N\perp K\subset M\subset M^{\sharp}\subset N^{\sharp}\perp
K^{\sharp}\quad\text{and}\quad |d(K)|\Big||d(M)|\cdot  |d(N)|,
$$
where, for a quadratic module $L$ over $\mathbb{Z}$, we denote
$\{x\in\mathbb{Q}L|B(x,L)\subset\mathbb{Z}\}$ by $L^{\sharp}$. 
\end{lemma}

\begin{proof}
The relations on inclusions are trivial, since $L_{1}\subset L_{2}$
implies $L^{\sharp}_{1}\supset L^{\sharp}_{2}$. Let $x$ be an element
of $M$. Then there is an element $y\in N^{\sharp}$ such that
$B(x,z)=B(y,z)$ for all $z\in N$. This correspondence $\varphi$ is
linear and we claim that $\varphi^{-1}(N)=N\perp K$. Suppose that
$\varphi(x)\in N$; then $B(x-\varphi(x),z)=0$ for $z\in N$ and so
$x-\varphi(x)\in K$. If, conversely, $x=x_{1}+x_{2}$, $x_{1}\in N$,
$x_{2}\in K$, then $B(x-\varphi(x),z)=B(x_{1}-\varphi(x),z)=0$ for
$z\in N$ and $\varphi(x)=x_{1}\in N$. Thus we have $[M:N\perp
  K]=[\varphi(M):N]|[N^{\sharp}:N]=d(N)$, and $|d(N)\cdot
d(K)|=|d(M)|\cdot [M:N\perp K]^{2}||d(M)|\cdot |d(N)|^{2}$. 
\end{proof}

\begin{lemma}\label{c2:lem-2.27}
For a regular quadratic module $L$ over $\mathbb{Z}$, $\min
(L):=\min\{|Q(x)| \;\; |x\in L,x\neq 0\}\leq (4/3)^{(n-1)/2}|d(L)|^{1/n}$
where $n=\rank L$.
\end{lemma}

\begin{proof}
We use induction on rank $L$. In case $\rank L=1$, the assertion is
trivial. For rank $L>1$, we take $v_{1}\in L$ such that
$|Q(v_{1})|=m(L)$, $v_{1}\neq 0$. If $\min (L)=0$, then\pageoriginale
we have nothing to prove. Suppose $Q(v_{1})\neq 0$, and
$\{v_{1},\ldots,v_{n}\}$ is a basis for $L$. Define a linear mapping
$p$ by $p(v_{1})=v_{1}$,
$p(v_{i})=v_{i}-Q(v_{1})^{-1}B(v_{i},v_{1})v_{i}(i\geqq 2)$. Then the
determinant of $p$ is one, and hence
\begin{align*}
|d(L)|&=|\det(B(v_{i},v_{j}))|=|\det(B(pv_{i},pv_{j}))|\\
&=\min(L)|\det(B(pv_{i},pv_{j}))_{i,j\geq  2}|, 
\end{align*}
since $B(v_{1},pv_{i})=0$ for $i\geqq 2$. Put
$M=\mathbb{Z}[p(v_{2}),\ldots,p(v_{n})]$. By the inductive assumption,
we have
$$
\min(M)\leqq (4/3)^{(n-2)/2}|d(M)|^{1/n-1}.
$$
Take $y\in M$ and a rational number $r$ such that
$$
|Q(y)|=\min(M), y+rv_{1}=x(\text{say})\in L,|r|\leqq 1/2.
$$
Then we obtain $\min(L)\leqq |Q(x)|=|Q(y)+r^{2}Q(v_{1})|\leqq
\min(M)+\dfrac{1}{4}\min(L)$.  Hence
\begin{align*}
\min(L) & \leqq \dfrac{4}{3}\min(M)\\
        & \leqq (4/3)^{n/2}|d(M)|^{1/n-1}\\
        & =(4/3)^{n/2}|d(L)/\min(L)|^{1/n-1}
\end{align*}
implying that
$$
\min(L)\leqq (4/3)^{(n-1)/2}|d(L)|^{1/n}.
$$
\end{proof}

We prove the proposition by induction on $\dim V$. In the case of
$\dim V=1$, it is obvious.

Suppose that $M$ is a lattice on $V$ such that $s(M)\subset\mathbb{Z}$
and $d(M)=d$. If $\min(M)\neq 0$, then for $v\in M$ with
$|Q(v)|=\min(M)$, we put $N=\mathbb{Z}v$. If $\min(M)=0$, then there
is a primitive isotropic vector $v_{1}\in M$. We can take a basis
$\{v_{1},v_{2},\ldots\}$ of $M$ such that
$B(v_{1},M)=B(v_{1},v_{2})\mathbb{Z}$, $B(v_{1},v_{i})=0$ for $i\geqq
3$. Hence $a=|B(v_{1},v_{2})|$ divides $d$. Since
$Q(v_{2}+bv_{1})=Q(v_{2})\pm 2ba$, we may assume $|Q(v_{2})|\leqq
a$. In this case, we put $N=\mathbb{Z}[v_{1},v_{2}]$. If $\dim V=2$,
then $M=N$ and 
the\pageoriginale number of possible corresponding matrices is
finite. Hence, for a binary isotropic quadratic module $V$, the
assertion is proved. Otherwise, we have constructed a sub-module $N$
of $M$ such that $|d(N)|$ is bounded by a constant depending only on
$d(M)$ and $\dim V$. Put $K=N^{\perp}$. Then $rank K = rank M-1$ or
$rank M-2$, and $|d(K)| \leqq |d(M)| |d(N)|$ which is less than a
constant depending only on $d(M)$ and $\dim V$. By the inductive
assumption, the number of possible $K$ is finite and then the number
of possible $K$ is finite and then the number of possible $M$ for
which $N\perp K\subset M \subset N^{\sharp} \perp K^{\sharp}$ is also
finite. This completes the proof. 

\setcounter{section}{2}
\setcounter{theorem}{27}
\begin{theorem}\label{c2:thm-2.28}
Let $W$, $V$ be regular quadratic modules over $\mathbb{Q}$, $S$ a
finite set of prime numbers, and $M,L\mathbb{Z}[S]$-lattices on $W$,
$V$ respectively, and suppose that $M_p$ is represented by $L_p$ for
$p\not\in S$ and $W_p, W_{\infty}$ are represented by $V_p,
V_{\infty}$ for $p\in S$. Then there is a lattice
$K\in \gen L$ such that $M$ is represented by $K$. 
\end{theorem}

\begin{proof}
By the Hasse-Minkowski theorem, we may assume that $W$ is a submodule
of $V$. Then there is an isometry $\sigma_p \in 0(V_p)$ such
that $\sigma_p(M_p) \subset L_p$, and for almost all $p, M_p\subset
L_p$. Hence q $\mathbb{Z}[S]$-lattice
$K=\bigcap\limits_{M_1\not\subset L_p} (V\cap \sigma^{-1}_p
(L_p)) \bigcap\limits_{M_p\subset L_p} (V\cap L_p)$ contains $M$ and
obviously, $K\in \gen L$ .
\end{proof}

\subsection{}\label{c2:subsec2.4.2}
 In this paragraph, we give two
different kinds of approximation theorems which are necessary
latter. Before stating the results, we first describe the
topology. Let $V$ be a vector space over $\mathbb{Q}_p$ with $\dim
V=n<\infty$. Fixing a basis of $V$ over $\mathbb{Q}_p, V$ (resp. End
$V$) is isomorphic to $\mathbb{Q}^n_p
(resp. \mathfrak{M}_n(\mathbb{Q}_p))$. Using this isomorphism, we can
introduce a topology on $V$ or End $V$ which is independent of the
choice of bases. Take two bases $\{u_i\}$, $\{v_i\}$ of $V$. If $u$
and $v \in V$ or End $V$) are sufficiently close with respect
to the topology introduced by $\{u_i\}$, then\pageoriginale they are
also sufficiently close with respect to $\{v_i\}$. Hence we can use
``sufficiently close'' without ambiguity, when a finite number of
fixed bases are involved.

The first theorem is an approximation theorem for $0'(V)$. 

\begin{theorem}\label{c2:thm-2.29}
Let $V$ be a regular quadratic modular over $\mathbb{Q}$ with $\dim
V \geqq 3$ and suppose that $V_v=\mathbb{Q}_vV$ is isotropic for some
spot $v$. ($v$ may be finite or infinite). Let $L$ be a
$\mathbb{Z}$-lattice on $V$ and $S$ a finite set of prime numbers
with $S\not\ni v$. For a given $\sigma_p \in 0'(V_p)$ for
$p\in S$, there is an isometry $\sigma \in 0'(V)$ such
that 
\begin{align*}
& \sigma(L_p) = L_p \text{ for } p \not\in S \cup \{ v\} \text{ and}\\
& \sigma \text{ and } \sigma_p \text{ are sufficiently close in } End
V_p \text{ for } p \in S. 
\end{align*}

To prove the theorem, we need some preparatory lemmas.
\end{theorem}

\setcounter{lemma}{29}
\begin{lemma}\label{c2:lem-2.30}
Let $V$ be a regular quadratic module over $\mathbb{Q}$ and $S$ a
finite set of spots including $\infty$. For given
$\sigma_v \in 0^+(V_v)$ for $v\in S$, there are
vectors $x_1, \ldots, x_{2n} \in V$ such that $\sigma_v$ and
$\tau_{x_1} \cdots \tau_{x_{2n}}$ are sufficiently close for
$v\in S$. 
\end{lemma}

\begin{proof}
Put $\sigma_v= \tau_{x_1(v)} \cdots \tau_{x_{2n}(v)}
(x_i(v), \in V_v)$. Since the order of any symmetry is 2, we
may suppose that $n$ is independent of $v\in S$. We have only
to choose $x_i \in V$ so that $x_i$ and $x_i(v)$ are
sufficiently close in $V_v$ for $v\in S$.
\end{proof}

\begin{lemma}\label{c2:lem-2.31}
Let $W$ be a regular quadratic module of $\dim 
W \geqq 3$, over $\mathbb{Q}, S$ a finite set of sports, and $v$ a
spot $\not\in S$. For a $\mathbb{Z}$-lattice $K$ on $W$ there
is an integer $\mu$ such that 
\begin{itemize}
\item[\rm{(i)}] $\mu \in \mathbb{Z}^x_p$\pageoriginale if
$p\in S$. 

\item[\rm{(ii)}] if a rational number $a$ is represented by $W$, and 
\end{itemize}
$a\in Q(K_p) \cap \mu \mathbb{Z}_p$ for $p\neq v$, $W\ni y$
with $Q(y)=a$ and $y\in K_p$ for $p\neq v$.
\end{lemma}

\begin{proof}
Extending $S$, we may assume that if $p\not\in S$, $p\neq v$, then
$K_p$ is unimodular and $p\neq 2$. Let $K_1, \ldots, K_h$ be a
complete set of representatives of classes in $\gen K$. We show that
$K_i$ can be chosen so that $(K_i)_p =K_p$ for $p\in
S$. First, we note that every regular quadratic module $M$ over
$\mathbb{Z}_p$ has a symmetry, since, for $m\in M$ satisfying
$(Q(m))=\underline{n}(M)$, $\tau_m$ gives a symmetry of $M$ . Hence by
the definition of the genus, there is an isometry
$\sigma_{i,p}\in 0^+(W_p)$ such that
$\sigma_{i,p}((K_i)_p)=K_p$, and then by Lemma \ref{c2:lem-2.30} there is an
isometry $\sigma_i$ such that $\sigma_i$ and $\sigma_{i,p}$ are
sufficiently close for $p\in S$. As representatives we have
only to take $\sigma_i(K_i)$. Thus we may assume $(K_i)_p=K_p$ for
$p\in S$. Now we choose an integer $\lambda$ so that $\lambda
K_i \subset K$ for all $i$ and $\lambda \in \mathbb{Z}^x_p$
for $p \in S$, and put $\mu = \lambda^2$. The condition (i)
is satisfied. Suppose that a is a rational number as in (ii). If
$p\in S$, then $a/\mu\in Q(K_p)$. If $p\not\in S$,
$p\neq v$, then $p\neq 2$, and $K_p$ is unimodular, and then
$K_p \cong<\left(\begin{smallmatrix}
0 &1\\1&0
\end{smallmatrix}\right)>\perp\ast$. Since
$a/\mu\in \mathbb{Z}_p$, $a/\mu$ is represented by
$<\left(\begin{smallmatrix}
0&1\\1&0
\end{smallmatrix}\right)> \subset K_p$. If $v$ is a finite spot
associated with a prime number $q$, then $a/\mu \cdot q^{2t} \in
Q(K_q)$ for a sufficiently large integer $t$. Thus $a/\mu$ or
$a/\mu\cdot q^{2t}$ is locally represented by $K$ according as
$v=\infty$ or $q$. By Theorem \ref{c2:thm-2.28}, there is a vector $x$ in some
$K_i$ such that $Q(x)=a/\mu$ or $a/\mu \cdot q^{2t}$ according as
$v=\infty$ or $q$. Then $y=\lambda x$ or $\lambda q^{-t}x$ is what we
want. 
\end{proof}

\begin{lemma}\label{c2:lem-2.32}
Let $V$ be a regular quadratic module over $\mathbb{Q}$ of $\dim
v\geqq 4$ which is isotropic at a spot $v$, $L$ a $\mathbb{Z}$-lattice
on $V$, and $T$ a finite set of prime numbers with $T \not\ni v$. 

Suppose\pageoriginale that a non-zero rational number $a$ and
$z_p \in V_p(p\neq v)$ satisfy
\begin{itemize}
\item[\rm{(i)}] $Q(z_p) = a\in Q(V)$ for every $p\neq v$, and 

\item[\rm{(ii)}] $z_p\in L_p$ if $p\not\in T$.
\end{itemize}

Then there is a vector $z\in V$ satisfying
\begin{itemize}
\item[\rm{(i)}] $z$ and $z_p$ are sufficiently close if $p\in
T$,

\item[\rm{(ii)}] $z\in L_p$ if $p\not\in T \cup \{v\}$,

\item[\rm{(iii)}] $Q(z)=a$.
\end{itemize}
\end{lemma}

\begin{proof}
Multiplying the quadratic form by $a^{-1}$, we may assume $a=1$
without loss of generality. Extending $T$, we may assume that if
$p\not\in T \cup \{v\}$, then $L_p$ is unimodular and $p\neq 2$. If
$V_{\infty}$ is isotropic, then we have only to consider the case of
$v=\infty$. Thus we may assume that $a=1$, and $V_{\infty}$ is
anisotropic in case $v\neq \infty$. Since we can take
$(\prod\limits_{p\in T}p)^{-r}L$ instead of $L(r\geqq 0)$, we
may assume that $z_p \in L_p$ if $p\neq v$. Take and fix any
vector $x$ such that $Q(x)=1$. Take $\varphi_p\in 0^+(V_p)$ so
that $\varphi_p(x)=z_p$ for $p\in T$. From Lemma \ref{c2:lem-2.30}, follows
the existence of an isometry $\varphi \in 0^+(V)$ such that
$\varphi$ and $\varphi_p$ are sufficiently close for $p\in T$,
and $y\in L_p$ if $p\in T$. Choose an integer
$\lambda$ so that $\lambda$ and 1 are sufficiently close in
$\mathbb{Z}_p$ if $p\in T$, and $\lambda y \in L_p$ if
$p \not\in T \cup\{v\}$, and set $u=\lambda y$; then $u\in
L_p$ if $p\neq v$ and $Q(u)=\lambda^2$. Set $W=u^{\perp} = \{
w\in V|B(u,w)=0\}$, and we determine a lattice $K$ on $W$
under the following conditions:
$$
K_p = (L\cap W)_p = L_p \cup W_p \text{ if } p \not\in T,
$$
$K_p \subset p^r L_p$ for sufficiently large $r$ if $p\in
T$. $K\subset L$, as is obvious. Set\pageoriginale
$T_{\lambda}=\{p| \lambda \not\in \mathbb{Z}^x_p, p\neq v\}$; then
$T\cap T_{\lambda}=\phi$ since $\lambda \in \mathbb{Z}^x_p$ if
$p\in T$. Let $\mu$ be an integer in Lemma \ref{c2:lem-2.31} for $v\not\in
S=T\cup T_{\lambda}$ and $M$. Set
$T_{\mu}=\{p|\mu \not\in \mathbb{Z}^x_p, p \neq v\}$; then
$T_{\mu} \cap (T\cup T_{\lambda} \cup\{v\})=\phi$. We claim that
$(\sharp)$ there is a rational number $\beta$ so that
$$
1-\lambda^2 \beta^2 \in \mu \mathbb{Z}_p \cap Q (K_p) \text{
and } \beta \in \mathbb{Z}_p \text{ if } p \neq v,
$$
$\beta$ and 1 are sufficiently close if $p\in T$,
$$
1-\lambda^2 \beta^2 \in Q(W_v) \text{ and }
1-\lambda^2 \beta^2 \in Q(W_{\infty}).
$$
We return to the proof of this latter and first complete with its help
the proof of Lemma \ref{c2:lem-2.32}. By the Hasse-Minkowski Theorem,
$1-\lambda^2 \beta^2 \in Q(W)$. Applying the property (ii) in
Lemma \ref{c2:lem-2.31} to $a=1-\lambda^2\beta^2$, there is a vector $w\in
W$ such that $Q(w)=1-\lambda^2\beta^2$ and $w\in K_p$ for
$p\neq v$. We show that $z=\beta u + w$ is what we want. Suppose
$P\in T$; then $\beta$ and 1 are sufficiently close in
$\mathbb{Z}_p$ and $w\in K_p \subset p^r L_p$. Thus $z$ and
$u$ are sufficiently close in $V_p$. On the other hand, $z_p$ and $y$,
$u$ and $y$ are sufficiently close respectively. Hence $z$ and $z_p$
are sufficiently close for $p\in T$. If $p\not\in
T\cup \{v\}$, then $z=\beta\lambda y + w \in \beta L_p +
K_p \subset L_p$. Lastly $Q(z)=\beta^2\lambda^2+Q(w)=1$. Thus the
assertions (i), (ii)., (iii) are satisfied. It remains for us to prove
the existence of a rational number $\beta$. First, we construct
$\beta_p\in \mathbb{Q}_p$ which satisfies the condition
$(\sharp)$ locally with $1-\lambda^2\beta^2_p\neq 0$ for $p\in
T \cup T_{\lambda} \cup T_{\mu}$. Then we approximate $\beta_p$ by
$\beta$, noting that $Q(K_p)\backslash\{0\}$, $Q(W_v)\backslash\{0\}$,
$Q(W_{\infty})\backslash\{0\}$ are open sets.

Let $p\in T$; take a non-zero number $\alpha_p\in
Q(K_p)$ which is sufficiently close to $0$ and set $\beta_p
=\lambda^{-1} (1-\alpha_p / 2)$. Since $\lambda$ and 1 are sufficiently
close, $\beta_p$ and\pageoriginale 1 are also sufficiently
close. Clearly, $1-\lambda^2 \beta^2_p  =
1-(1-\alpha_p / 2)^2=\alpha_p(1-\alpha_p/4)\in \alpha_p \mathbb{Z}^{x^2}_p
\subset Q(K_p)$. Since $T \cap T_{\mu}=\phi$, we have
$\mu\in \mathbb{Z}^x_p$ and then
$1-\lambda^2\beta^2_p\in \mu \mathbb{Z}_p$. Thus the condition
$(\sharp)$ is satisfied for $\beta_p$ with $q-\lambda^2\beta^2_p \neq
0$. 

Let $p\in T_{\mu}$; take $\beta_p \in \mathbb{Q}_p$ so
that $\beta_p$ and $\lambda^{-1}$ are sufficiently close but
$\beta_p\neq \lambda^{-1}$. Since
$\lambda \in \mathbb{Z}^x_p$,
$\beta_p\in \mathbb{Z}^x_p$. Obviously $0\neq
1-\lambda^2 \beta^2_p \in \mu\mathbb{Z}_p$. Since $p\not\in T$
and $Q(u)=\lambda^2 \in \mathbb{Z}^x_p$, $K_p=u^{\perp}$ in
$L_p$ is unimodular, by virtue of Lemma \ref{c2:lem-2.5}. From
Proposition \ref{c2:prop-2.12}, it 
follows that $Q(K_p)=\mathbb{Z}_p$. Thus the condition $(\sharp)$ is
satisfied for $\beta_p$ with $1-\lambda^2 \beta^2_p\neq 0$.


Let $p\in T_{\lambda}$; first, we claim that $K_p$ contains a
unimodular submodule of $rank \geqq 2$. Let $\{v_i\}$ be a basis of
$L_p$ over $\mathbb{Z}_p$ and assume $v_1 = by$,
$b\in \mathbb{Q}_p$. Since $T\cap T_{\lambda} = \phi$, $L_p$
is unimodular and then $Q(v_1)=b^2\in \mathbb{Z}_p$. Suppose
$b\in \mathbb{Z}^x_p$; then $L_p =\mathbb{Z}_p
v_1 \perp(v^{\perp}_1 \text{ in } L_p) = \mathbb{Z}_pv_1 \perp K_p$ by
virtue of Lemma \ref{c2:lem-2.5} and the definition of $K$. Hence $K_p$ itself is
unimodular. Suppose $b\in p \mathbb{Z}_p$; since
$\mathbb{L}_p$ is unimodular, $B(v_1, L_p)=\mathbb{Z}_p$ and in view
of $Q(v_1) \in p\mathbb{Z}_p$, we may assume $B(v_1, v_2)=1$,
without loss of generality. Then $\mathbb{Z}_p[v_1,v_2]$ is unimodular
and so is $\mathbb{Z}_p[v_1, v_2]^{\perp}$ in $L_p(\subset K_p)$ by
Lemma \ref{c2:lem-2.5}. Thus our claim above has been proved, and then Proposition
\ref{c2:prop-2.12} implies that $Q(K_p) \ni 1$. For $\beta_p=0$, the condition
$(\sharp)$ is satisfied with $1-\lambda^2 \beta^2_p\neq 0$ since
$\mu\mathbb{Z}_p =\mathbb{Z}_p$.

Suppose $v=\infty$; then we choose a large number
$\beta\in \mathbb{Q}$ such that $\beta$ and $\beta_p$ are
sufficiently close for $p\in T \cup T_{\lambda} \cup T_{\mu}$,
and $\beta \in \mathbb{Z}_p$ otherwise. If $p\not\in T \cup
T_{\lambda} \cup T_{\mu}$, then $\mu \in \mathbb{Z}^x_p$ and
$K_p$ is unimodular since $L_p$ is unimodular and
$Q(u)\in \mathbb{Z}^x_p$. Hence $\mu\mathbb{Z}_p =Q(K_p)
=\mathbb{Z}_p$, and the condition $(\sharp)$
is\pageoriginale satisfied for each prime number. By assumption
$Q(W_{\infty})\supset \{a\in\mathbb{R}|a<0\}$, and then it is also
satisfied for $v=\infty$.

Suppose $v=q<\infty$. Set $\beta_q=q^{-r}$ for a sufficiently large
$r$; then $1-\lambda^2\beta^2_q = - \lambda^2 \beta^2_q(1-\lambda^{-2}
q^{2r}) \in Q (W_v)$, since $V_q=<\lambda^2> \perp W_q$ is
isotropic and $1-\lambda^{-2} q^{2r}$ is a square. We take a rational
number $\beta'$ so that $\beta'$ and $\beta_p$ are sufficiently close
for $p \in T \cup T_{\lambda} \cup T_{\mu} \cup \{q\}$ and
$\beta' \in \mathbb{Z}_p$ otherwise. Next we take a
sufficiently large integer $m$ such that $q^m$ and 1 are sufficiently
close for $p\in T \cup T_{\lambda} \cup T_{\mu}$, and set
$\beta = \beta' q^{-m}$. In the process, $V_{\infty}$ is positive
definite, and $1-\lambda^2 \beta^2$ is sufficiently close to 1 in
$\mathbb{R}$. It is easy to see that $\beta$ is the rational number
required in $(\sharp)$.
\end{proof}


\medskip
\noindent\textbf{PROOF of Theorem \ref{c2:thm-2.29} when {\boldmath{$\dim V \geqq 4$}}.}

Let $V$, $v$, $S$, $\sigma_p$ be as in Theorem \ref{c2:thm-2.29}.
\begin{enumerate}
\renewcommand{\theenumi}{\roman{enumi}}
\renewcommand{\labelenumi}{(\theenumi)}
\item Suppose that $\sigma_p = \tau_{x_p} \tau_{y_p}$, $Q(x_p) =
  Q(y_p) \; (x_p, y_p \in V_p)$ for any $p\in S$.

Take a vector $x\in V$ so that $x$ and $x_p$ are sufficiently
close for $p\in S$, and $x\in L_p$ otherwise, and take
$\eta_p \in 0(V_p)$ so that $y_p=\eta_p x_p$. Choose a finite
set $S'$ of prime numbers so that $S'\cap (S\cup \{v\})=\phi$ and if
$p\not\in S'$, then $\tau_x L_p =L_p$, $Q(x)\in
\mathbb{Z}^x_p$, $p\neq 2$, and $L_p$ is unimodular. Set
$z_p=\eta_px$ for $p\in S$, $z_p=x$ for $p\in S'$. If
$p \not\in S \cup S' \cup \{v\}$, then there exists $z_p \in
L_p$ such that $Q(z_p)=Q(x)$ since $L_p$ is unimodular $(p\neq 2)$ and
$Q(x)\in \mathbb{Z}^x_p$. Applying Lemma \ref{c2:lem-2.32} to $z_p,
T=S\cup S'$, $0\neq a=Q(x)\in Q(V)$, there is a vector
$z\in V$ with $Q(z)=Q(x)$ such that $z$ and $z_p$ are
sufficiently close for $p\in S \cup S'$, $z\in L_p$
for $p\not\in S \cup S'\cup \{v\}$. If \pageoriginale $p\in
S$, then $\tau_x \tau_z$ and $\tau_{x_p} \tau_{y_p} = \sigma_p$ are
sufficiently close. If $p\in S'$, then $\tau_x \tau_z$ and
$\tau_x \tau_x = id$ are sufficiently close and hence $\tau_x \tau_z
L_p = L_p$. Suppose $p\not\in S \cup S'\cup \{v\}$; then $\tau_x
L_p=L_p$ by the definition of $S'$, and further $\tau_z L_p=L_p$ since
$Q(z)=Q(x)\in \mathbb{Z}^x_p$ and $z\in
L_p$. Thus $\sigma =\tau_x \tau_z$ is what we want.

\item Suppose that $\sigma_p = \tau_{x_{1,p}} \tau_{y_{1,p}} \cdots
  \tau_{x_{r,p}} \tau_{y_{r,p}}$, $Q(x_{i,p}) = Q(y_{i,p})$ for each
  $p\in S$. 

In this case, we may assume that $r$ is independent of each
$p\in S$, since the order of any symmetry is 2. Applying (i)
to $\tau_{x_i} \tau_{y_i}$, we complete the proof.

\item 
\noindent
\textbf{General Case.}

Set $\sigma_p = \tau_{x_{1,p}} \cdots \tau_{x_{2r,p}}$ with $\Pi Q
(x_{i,p})=1$ and assume $r$ is independent of each $p\in S$ as
in (ii). Extending $S$, we may assume that $V_p$ is isotropic if
$p\not\in S$, by virtue of Theorem 6 on page 36 in $[S]$. On this
occasion, we set $\sigma_p=$ the identity mapping for $p$ which
belongs not to the originale $S$ but to the extended $S$. Take
$x_1,\ldots, x_{2r-1}\in V$ so that $x_i$ and $x_{i,p}$ are
sufficiently close for $p\in S, 1 \leqq i \leqq 2r-1$ and so
are $\prod\limits_{1\leqq i \leqq 2r-1} Q(x_i)$ and
$\prod\limits_{q\leqq i \leqq 2r-1} Q(x_{i,p}) (\neq 0)$ for
$p\in S$. Hence there is a unit
$\varepsilon_p \in \mathbb{Z}^x_p$ such that
$Q(x_{2r,p})^{-1} = \prod\limits_{1\leqq i \leqq 2r-1}Q(x_{i,p})
= \in^2_p \prod\limits_{1\leqq i \leqq 2r-1}Q(x_i)$, and
$\in_p$ is sufficiently close to 1. We claim that there is a
vector $x_{2r}\in V$ so that $Q(x_{2r})=\prod\limits_{1\leqq i
\leqq 2r-1} Q(x_i)^{-1}$, and $x_{2r}$ and $x_{2r,p}$ are sufficiently
close for $p\in S$. Set $a=\prod\limits_{1\leqq i \leqq 2r-1}
Q(x_i)^{-1}$; then $a=Q(\in_p x_{2r,p})$ for $p\in S$,
and since $V_p$ is isotropic for $p\not\in S$, a is represented by
$V_p$ for every prime \pageoriginale number $p$. If $V_{\infty}$ is
isotropic, then a is also represented by $V_{\infty}$. If $V_{\infty}$
is anisotropic, then the sign of a is equal to $Q(x_{2r-1})$, and
hence a is also represented by $V_{\infty}$. By virtue of
Hasse-Minkowski Theorem, $a$ is represented by $V$. Take a vector
$w\in V$ with $Q(w)=a$, and $\eta_p \in 0^+(V_p)$ with
$\eta_p w = \in_p x_{2r,p}$ for $p \in S$, and
approximate $\eta_p$ by $\eta \in 0^+(V)$ by Lemma \ref{c2:lem-2.30}. We
can take $\eta(w)$ as $x_{2r}$. Then $\prod\limits_{1\leqq i \leqq 2r}
Q(x_i)=1$ and $\tau_{x_1} \cdots \tau_{x_{2r}}$ and
$\tau_{x_{1,p}} \cdots \tau_{x_{2r,p}}$ are sufficiently close for
$p\in S$. Set $S'=\{p\not\in
S \cup \{v\}|\tau_{x_1} \cdots \tau_{x_{2r}} L_p \neq L_p\}$. Since
$V_p$ is isotropic for $p\in S'$, it follows that
$\tau_{x_1} \cdots \tau_{x_{2r}}$ is a product of
$\tau_x \tau_y(x,y \in V_p, Q(x) =Q(y))$ for $p\in
S'$. From (ii), follows the existence of $\sigma_1 \in 0'(V)$
such that $\sigma_1$ and $1$ (resp. $\tau_{x_1} \cdots \tau_{x_{2r}}$)
are sufficiently close for $p\in S$ (resp. $p\in S'$)
and $\sigma_1 (L_p)=L_p$ for $p\not\in S \cup S' \cup \{v\}$. Then
$\sigma = \sigma^{-1}_1 \tau_{x_1} \cdots \tau_{x_{2r}}$ is what we
want. Thus we have completed the proof of Theorem \ref{c2:thm-2.28} when $\dim
V \geqq 4$.
\end{enumerate}

Suppose now that $\dim V=3$. Multiplying the quadratic form by a
constant, we may assume $d(V)=1$, that is, $V=<a_1>\perp <a_2> \perp
<a_1 a_2>, a_i\in \mathbb{Q}^{x}$. Now we define a quaternion
algebra $C=\mathbb{Q} + \mathbb{Q} i + \mathbb{Q}j + \mathbb{Q}k$ by
$i^2 = -a_1, j^2=-a_2$, $k^2 = -a_1 a_2$ and $-ji=k$. The conjugate
$\overline{x}$ of $x=a+bi+cj+dk (a,b,c,d \in \mathbb{Q})$ is
defined by $a-bi-cj-dk$. Then the norm $N(x)$ of $x$ is, by
definition, $x\overline{x}=a^2+b^2a_1+c^2a_2+d^2a_1a_2$, and so it is
a quadratic form and the corresponding bilinear form $B(x,y)$ is
$\dfrac{1}{2} (x\overline{y} + y \overline{x})$. Thus $V$ is
isometric to the subspace $\mathbb{Q}i+ \mathbb{Q}j + \mathbb{Q}k$ and
we identify them. We note that $V$ and $\mathbb{Q} \subset \mathbb{C}$
are orthogonal. For $x\in V$ with $N(x)\neq 0$, we have 
\begin{align*}
\tau_x y & = y -\frac{(x\overline{y} + y \overline{x})}{N(x)}\\
& = -N(x)^{-1} x \overline{y} x = -x y x^{-1} \text{ for }
y \in V.
\end{align*}\pageoriginale 
Therefore $\varphi \in 0^+(V)$ is written, for some
$z\in C$, as 
$$
\varphi(y) = z y z^{-1} \text{ for } y \in V,
$$
and then the spinor norm $0(\varphi)$ is $N(z)\mathbb{Q}^{x^2}$. Set
$\widetilde{L} = \mathbb{Z} \perp L$ and extend the given
$\sigma_p \in 0'(V_p)$ to $\widetilde{\sigma}_p \in
0'(C_p)$ where $\widetilde{\sigma}_p(1)=1(\in C)$. Similarly to
the foregoing, there is a vector $z_p \in C_p$ so that
$\widetilde{\sigma}_p(y)=z_pyz^{-1}_p(p\in S)$. Since
$\widetilde{\sigma}_p(\in 0' (V_p), Nz_p = a^2_p,
a_p \in \mathbb{Q}^x_p$. Taking $a^{-1}_p z_p$ instead of
$z_p$, we may assume $N z_p=1$. If $p\not\in S$, then set $z_p=1$. Let
$T(\not\ni v)$ be a finite set of prime numbers such that $T\supset S$
and if $p \not\in T$, then $\widetilde{L}_p$ is unimodular and a
subring. Applying Lemma \ref{c2:lem-2.32}, there is a vector $z\in C$ so
that $N(z)=1,z$ and $z_p$ are sufficiently close if $p\in T$
and $z\in \widetilde{L}_p$ if $p\not\in T\cup \{v\}$. We
define an isometry $\widetilde{\sigma} 0'(C)$ by
$\widetilde{\sigma}(y)=zyz^{-1}$. Since $\widetilde{\sigma}(1)=1$,
$\widetilde{\sigma}(V)=V$ follows, and set
$\sigma=\widetilde{\sigma}|V$. If $p\in S$, then $z$ and $z_p$
are sufficiently close, and then $\widetilde{\sigma}$ and
$\widetilde{\sigma}_p$ are sufficiently close, and hence so are
$\sigma$ and $\sigma_p$ since
$\widetilde{\sigma}(1)=\widetilde{\sigma}_p(1)=1$. If $p\in
T\backslash S$, then $\widetilde{\sigma}$ and $\id$ are sufficiently
close, and then $\widetilde{\sigma}(\widetilde{L}_p)$, and hence
$\sigma(L_p)=L_p$. Suppose $p\not\in T$; then
$z \in \widetilde{L}_p$, and $\widetilde{L}_p$ is
unimodular. Hence $\tau_z(\widetilde{L}_p)=\widetilde{L}_p$, since
$N(z)=1$. From $\widetilde{L}_p=\tau_z \widetilde{L}_p =
z \widetilde{L}_p z = z \widetilde{L}_p z^2 z^{-1} \subset
z \widetilde{L}_p z^{-1}=\widetilde{\sigma}\widetilde{L}_p$, it
follows that $\widetilde{\sigma}$ preserves $\widetilde{L}_p$, and
then $\sigma(L_p)=L_p$. Thus $\sigma$ is what we wanted, and the proof
of Theorem \ref{c2:thm-2.29} is complete.

\setcounter{section}{2}
\setcounter{theorem}{32}
\begin{theorem}\label{c2:thm-2.33}
Let $V$ be a regular quadratic module over $\mathbb{Q}$ with $\dim
V=m\geqq 2$ and suppose that $V$ is not a hyperbolic plane, i.e.,
either $\dim V=2$ and $d(V)\neq -1$ or $\dim V\geqq 3$, and that
$V_{\infty}=\mathbb{R}V\cong
(\mathop{\perp}\limits_r<1>)\perp(\mathop{\perp}\limits_s<-1>)$. Suppose
that the following \pageoriginale are given:
\begin{itemize}
\item[\rm{(a)}] a $\mathbb{Z}$-lattice $M$ on $V$,

\item[\rm{(b)}] a finite set $S$ of prime numbers $p$ such that $S\ni
2$ and $M_p$ is unimodular for $p\not\in S$,

\item[\rm{(c)}] integers $r'$, $s'$ with $0\leqq r'\leqq r$, $0\leqq
s' \leqq s$,

\item[\rm{(d)}] $x_{1,p}, \ldots, x_{n,p} \in M_p
(r'+s'=\eta<m)$ for $p\in S$.
\end{itemize}

Then there are vectors $x_1,\ldots, x_n$ in $M$ satisfying 
\begin{itemize}
\item[\rm{(i)}] $x_i$ and $x_{i,p}$ are sufficiently close in $V_p$
for $p\in S$, $1\leqq i \leqq n$,

\item[\rm{(ii)}] for $p\not\in S$, $\det(B(x_i,
x_j)) \in \mathbb{Z}^x_p$ with precisely one exception $p=q$,
where\break $\det(B(x_i, x_j)) \in q \mathbb{Z}^x_p$,

\item[\rm{(iii)}] a subspace spanned by $\{x_i\}$ in $\mathbb{R}V$ is
isometric to $(\mathop{\perp}\limits_{r'}<1>) \perp
(\mathop{\perp}\limits_{s'} <-1>)$.
\end{itemize}
\end{theorem}

\begin{proof}
We use induction on $n=r'+s'$. First suppose $n=1$, $m=2$. This case
is fundamental. Let $V^a(a\in \mathbb{Q}^x)$ denote the vector
space provided with a new quadratic form $aQ(x)$. We shall use $L^a$
to denote the lattice $L$ when it is regarded as a lattice in
$V^a$. First, we show that if the theorem is true for $V^a$, then it
holds for $V$. Suppose that the theorem holds for
$V^a(a\in \mathbb{Q}^x)$ and that $M, S, r', s', x_{1,p}$ in
$(a), \ldots, (d)$ are given. Put
$S(a)=S\cup \{p|a \not\in \mathbb{Z}^x_p\}$. Then for a lattice $M^a$
and $S(a)$, the condition (b) is satisfied. For a prime number
$p\in S(a)\backslash S$, we can choose $x_{1,p} \in
M_p$ with $Q(x_{1,p}) \in \mathbb{Z}^x_p$ since $p$ is odd and
$M_p$ is unimodular. If a is positive, then we put $r''=r',
s''=s'$. Otherwise, put $r''=s', s''=r'$. From the assumption, it
follows that there exists $x\in M^a$ for which 
\begin{itemize}
\item[\rm{(i$'$)}] $x$ and $x_{1,p}$ are sufficiently close in $V_p$ for
$p\in S(a)$,

\item[\rm{(ii$'$)}] for $p\not\in S(a)$,
$aQ(x)\in \mathbb{Z}^x_p$ with precisely one exception $p=q$,
where $aQ(x)\in q \mathbb{Z}^x_q$, and

\item[\rm{(iii$'$)}] $aQ(x)$ is \pageoriginale positive (resp. negative)
if $r''=1$, $s''=0$ (resp. $r''=0$, $s''=1$).
\end{itemize}

(i$'$) (resp. (iii$'$)) implies (i) (resp. (iii)). If $p\not\in S(a)$,
$p\neq q$, then we have $aQ(x)\in \mathbb{Z}x_p$ and
$a\in \mathbb{Z}^x_p$ and therefore
$Q(x) \in \mathbb{Z}^x_p$. For $p=q$, $Q(x)\in
q \mathbb{Z}^x_q$ since $q \not\in S(a)$. For $p\in
S(a)\backslash S$, (i$'$) implies that $Q(x)$ and
$Q(x_{1,p}) \in \mathbb{Z}^x_p$ are sufficiently close. Hence
$Q(x)\in \mathbb{Z}^x_p$. Thus we get the assertions (i),
(ii), (iii). Therefore, we may assume that $V$ is a quadratic field
$k$ over $\mathbb{Q}$ and the quadratic form $Q$ is equal to the norm
$N$ from $k$ to $\mathbb{Q}$. We take a finite set $S'\supset S$ of
prime numbers so that for $p\not\in S'$, $M_p$ is equal to the
localization of the maximal order of $k$. We choose
$x_{1,p}\in M_p$ is equal to the localization of the maximal
order of $k$. We choose $x_{1,p}\in M_p$ for $p\in
S'\backslash S$ such that $N x_{1,p} \in \mathbb{Z}^x_p$. By
the approximation theorem, there exists $y\in k$ such that
$Ny$ is positive (resp. negative) for $r'=1$, $s'=0$ (resp. $r'=0$,
$s'=1$) and $y$ and $x_{1,p}$ are sufficiently close for $p\in
S'$. Decompose the principal ideal $(y)$ as
$(y)=\widetilde{m}\widetilde{n}$ where $\widetilde{m}$,
$\widetilde{n}$ are ideals of $k$ and the prime divisor
$\widetilde{p}$ appears in $\widetilde{m}$ if and only if
$\widetilde{p}$ divides some prime number $p$ in $S'$. Thus it is
known that there exists a number $z\in k$ for which $z$ and 1
are sufficiently close for $p\in S'$, $Nz$ is positive, and
$\widetilde{q} =\widetilde{n}z$ is a prime divisor with
$N\widetilde{q}=q$ prime.
\end{proof}

Put $x=yz$. Then the conditions (i), (iii) are obviously
satisfied. For $p\in S'\backslash S, y, x_{1,p}$ and $z,1$ are
sufficiently close respectively and
$Nx_{1,p} \in \mathbb{Z}^x_p$. Hence
$Q(x) \in \mathbb{Z}^x_p$, for $p\in S'\backslash
S$. Since $(x)=(yz)=\tilde{m}\tilde{q}$, we have $Q(x)=\pm
N(\widetilde{m})q$. By the assumption on $\widetilde{m}$, the
condition (ii) is satisfied. By the construction, it\pageoriginale is
easy to see that $x$ is contained in every localization of $M$ and
hence in $M$. Thus we have completed the proof for the case $n=1$,
$m=2$. Suppose now that $n=1$ and $m=\dim V\geqq 3$. Take any prime
$h \not\in S$. Then there exists a basis $\{v_i\}$ of $M_h$ such that
$(B(v_i, v_j))_{i=1,2} = \left(\begin{smallmatrix}
0&1\\1&0\end{smallmatrix}\right)$, $Q(v_3)\in \mathbb{Z}^x_h$
and $M_h = \mathbb{Z}_h[v_1, v_2]\perp \mathbb{Z}_h v_3 \perp \cdots$
by Proposition \ref{c2:prop-2.12}. We take $x_1 \in M$ so that $x_1$ and
$x_{1,p}$ are sufficiently close for $p\in S$, and $x_1$ and
$v_1+hv_2$ are sufficiently close for $h$. Put $T=\{p \not\in
S|Q(x_1) \not\in\mathbb{Z}^x_p\} \ni h$. There exists $x_2 \in
V$ such that $Q(x_2)>0$ (resp. $<0$) for $r'=1$, $s'=0$ (resp. $r'=0$,
$s'=1$), for $p\in T$, $x_2 \in M_p$ and
$Q(x_2)\in \mathbb{Z}^x_p$ and moreover, $x_2$ and $v_3$ are
sufficiently close for $p=h$. Then we have a natural number a such
that $ax_2 \in M$ and $p\not\mid a$ for $p\in T$. The
discriminant of $M'=\mathbb{Z}[x_1, ax_2]$ is divisible exactly by
$h$. Hence $W=\mathbb{Q}[x_1, ax_2]$ is not a hyperbolic plane. Put
$U=\{p \not\in S \cup T|d(M') \not\in \mathbb{Z}^x_p\}$ and $x'_{1,p}
=x_1$ if $p\in S\cup U$, $x'_{1,p}=x_2$ if $p\in T$,
and $S'=S\cup T \cup U$. Then $M'_p$ is unimodular for $p\not\in S'$,
since $d(M'_p)$ is a unit and $M'\subset M$. Applying the previous
result to this, we have an element $x\in M'$ such that 
\begin{itemize}
\item[\rm{(i)}] $x$ and $x'_{1,p}$ are sufficiently close for
$p\in S'$

\item[\rm{(ii)}] for $p\not\in S'$, $Q(x)\in \mathbb{Z}^x_p$
with precisely one exception $p=q$, where $Q(x)\in
q \mathbb{Z}^x_p$, 

\item[\rm{(iii)}] $Q(x) >0$ (resp. $<0$) if $r'=1$ (resp. $r'=0$).
\end{itemize}

It is easy to see that this $x$ is what we wanted. Now suppose
$1<n<m$. Applying the inductive assumption to $x_{1,p}, \ldots,
x_{n-1,p}, S$ and $M$, there exist $x_1, \ldots, x_{n-1} \in
M$ such that $x_i$ and $x_{i,p}$ are sufficiently close for $1\leqq
i \leqq n -1$, $p\in S$, $\det (B(x_i,
x_j))_{i,j<n} \in \mathbb{Z}^x_p$ \pageoriginale for $p\not\in
S \cup \{q_1\}$ for some prime $q_1 \not\in S$, $\det(B(x_i,
x_j))_{i,j<n} \in q_1 \mathbb{Z}^x_{q_1}$, and over
$\mathbb{R}$
$$
(<B(x_i, x_j))_{i,j<n} > \perp <\delta> \cong
(\mathop{\perp}\limits_{r'} <1>) \perp (\mathop{\perp}\limits_{s'}
<-1>) \text{ for } \delta =\pm 1.
$$

Put $U = \sum\limits^{n-1}_{i=1} \mathbb{Q}x_i$, $W=\{x\in
V|B(x,U)=0\}$. Then $V=U\perp W$. Put $A =\mathbb{Z}[x_1,\ldots,
x_{n-1}]$; then $d(A_{q_1}) \in q_1 \mathbb{Z}^x_{q_1}$. From
the local version of Lemma \ref{c2:lem-2.26}, it follows that $d(A^{\perp}_{q_1}$
in $M_{q_1}) \in \mathbb{Z}^x_{q_1}$. On the other hand,
$d(M_{q_1})\in \mathbb{Z}^x_{q_1}$ implies $d(A_{q_1})\cdot
d(A^{\perp}_{q_1}) \in \mathbb{Z}^x_{q_1} \mathbb{Q}^{x^2}_{q_1}/ 
\mathbb{Q}^{x^2}_{q_1}$. Thus $d(A^{\perp}_{q_1}$ in
$M_{q_1})\in q_1\mathbb{Z}^x_{q_1}$. Let $A_{q_1} = L_1 \perp
L_2$, $A^{\perp}_{q_1} = L_3 \perp L_4$ be Jordan splittings so that
$L_1, L_3$ are unimodular and rank $L_2 = $ rank $L_4=1$. Since
$M_{q_1}$ is unimodular, $L_2 \perp L_4$ is contained in the
unimodular module $(L_1 \perp L_3)^{\perp}$ in $M_{q_1}$, and then
$A_{q_1} \subset L_1 \perp (L_1 \perp L_3)^{\perp}$. From
$d(A_{q_1}) \in q_1 \mathbb{Z}^x_{q_1}$, it follows that
$A_{q_1}$ is a direct summand in $L_1 \perp (L_1 \perp L_3)^{\perp}$,
and hence there is an element $x_{n,q_1}\in M_{q_1}$ such that
$A_{q_1} + \mathbb{Z}_{q}x_{n,q_1}$ is a unimodular module
$L_1\perp(L_1 \perp L_3)^{\perp}$. Decompose $x_{n,p}$ as
$x_{n,p}=y_{n,p} + z_{n,p}(y_{n,p}\in U_p, z_{n,p} \in
W_p)$ for $p\in S \cup \{ q_1\}$. We can take $y_n \in
U$ so that $y_n$ and $y_{n,p}$ are sufficiently close for
$p\in S \cup \{ q_1\}$ and $y_n \in A_p$ for $p\not\in
S \cup \{ q_1\}$. We claim that 

$(\sharp)$ there exist an element $z_n$ in the projection $M'$ of $M$
to $W$ and a prime $q \not\in S \cup \{ q_1\}$ such that 
\begin{align*}
& z_n \text{ and } z_{n,p} \text{ are sufficiently close for }
p\in S \cup \{q_1\},\\
& Q(z_n) \in \mathbb{Z}^x_p \text{ for } p \not\in S \cup \{
q,q_1\} \text{ and } Q(z_n) \in q \mathbb{Z}^x_q,\\
& Q(x_n) \delta >0.
\end{align*}\pageoriginale 
We come to the proof of this later and first complete the proof of the
theorem with its help. put $x_n = y_n + z_n$. Then $x_n$ and $x_{n,p}
= y_{n,p} + z_{n,p}$ are sufficiently close for $p\in
S \cup \{ q_1\}$. Hence the condition (i) is satisfied, and
$x_n \in M_p$, for $p\in S \cup \{ q_1\}$. For
$p\not\in S \cup \{ q_1\}$, $M_p$, $A_p$ are unimodular and hence
$M_p=A_p \perp (\ast)$. Since $M'$ is the projection of $M$ to $W$, we
have $M_p=A_p \perp M'_p$. Hence we have $x_n = y_n + z_n \in
A_p + M'_p =M_p$ for $p\not\in S \cup \{q_1\}$. Thus $x_n \in
M$. We check the condition (ii). $d(\mathbb{Z}_p[x_1, \ldots, x_n])$
and $d(\mathbb{Z}_p[x_{1,p},\ldots, x_{n,p}])$ are sufficiently close
for $p=q_1$, and the latter is a unit by the definition of
$x_{n,q_1}$. Hence $d(\mathbb{Z}_p[x_1,\ldots,
x_n]) \in \mathbb{Z}^x_p$, for $p=q_1$. For $p\not\in
S \cup \{q_1\}$ , $d(\mathbb{Z}_p[x_1, \ldots, x_n]) =
d(\mathbb{Z}_p[x_1, \ldots, x_{n-1}, y_n + z_n]) = d(\mathbb{Z}_p[x_1,
\ldots, x_{n-1}, z_n])$ $(y_n \in A_p)=d(A_p)\cdot
Q(z_{n}) \in Q(z_n) \mathbb{Z}^x_p$. Thus from the property of
$z_n$ in $(\sharp)$ condition (ii) follows. Condition (iii) follows
from 
\begin{align*}
\mathbb{Q}[x_1,\ldots, x_n] &= \mathbb{Q} [x_1, \ldots
x_{n-1}] \perp \mathbb{Q} z_n\\
&= <(B(x_i, x_j))_{i,j <n}> \perp
<\delta> \text{ over } \mathbb{R}.
\end{align*}
It remains to show $(\sharp)$. For $\dim W \geqq 2$, this is
clear. Since $d(W_{q_1}) = d(A^{\perp}_{q_1}) \in
q_1 \mathbb{Z}^x_{q_1}$, $W$ is not the hyperbolic plane. As we have
seen, $M_p = A_p \perp M'_p$ for $p\not\in S \cup \{ q_1\}$ and then
$M'_p$ is unimodular for $p\not\in S\cup \{ q_1\}$. Also,
$z_{n,p} \in M'_p$, from the definitions of $z_{n,p}$ and
$M'$. Obviously, $W \cong <\delta> \perp (\ast)$ over $\mathbb{R}$, by
the definition of $\delta$. Applying the theorem for the case $n=1$,
we obtain the existence of $z_n$.

\subsection{}\label{c2:subsec2.4.3} 
In this paragraph, we give sufficient
conditions under which 
$\gen L=spn L$\pageoriginale or $spn L = cls L$, and also a result on
representation 
of indefinite quadratic forms.

\begin{theorem}\label{c2:thm-2.34}
Let $V$ be a regular quadratic module over $\mathbb{Q}$ with $\dim
V \geqq 3$, $S$ a finite set of prime numbers and $L$ a
$\mathbb{Z}[S]$-lattice on $V$.

If $\theta (0^+(L_p)) \supset \mathbb{Z}^x_p$ for every prime number
$p\not\in S$, then we have $\gen L=spn L$. 
\end{theorem}

\begin{proof}
Suppose $K\in \gen L$. Then from the definition, we have an isometry
$\sigma_p \in 0(V_p)$ such that $\sigma_p(K_p)=L_p$. For
$v\in K_p$ satisfying $Q(v)\mathbb{Z}_p=\underline{n}(K_p)$,
the symmetry $\tau_v(x)=x-\dfrac{2B(x,v)}{Q(v)}v$ belongs to
$0(K_p)$. Hence we may assume $\sigma_p\in 0^+(V_p)$, after
multiplying it by $\tau_v$, if necessary. Moreover, we assume
$\sigma_p=id$, if $K_p=L_p$. We can take a positive number $a$ so that
$a \theta (\sigma_p)$ contains a unit for $p\not\in S$. By Proposition
\ref{c2:prop-2.19}, there is an isometry $\sigma \in 0^+(V)$ such that
$\theta(\sigma)=a\mathbb{Q}^{x^2}$. For $M=\sigma^{-1}(K)$, we have
$\sigma_p\sigma(M_p) = L_p$ for every $p$, and $\theta(\sigma_p\sigma)
\subset \mathbb{Z}^x_p \mathbb{Q}^{x^2}_p$ for $p\not\in S$.

By assumption, there is an isometry $\eta_p \in 0^+(L_p)$ such
that $\theta(\eta_n \sigma_p\break \sigma)=\mathbb{Q}^{x^2}_p$. Thus we have 
$$
\sigma^{-1}(K_p) = (\eta_p \sigma_p \sigma)^{-1}
L_p, \quad \eta_p \sigma_p \sigma  \in 0'(V_p) \text{ for }
p \not\in S.
$$
This means $K \in spn L$.
\end{proof}

\begin{remark*}
Let $L_p = L_1 \perp \cdots \perp L_t$ be a Jordan splitting. If
either rank $L_i \geq 2$ (resp. 3) for some $i$ for $p\neq 2$
(resp. $p=2$), or $L_p$ is maximal and rank $L_p \geqq 3$, then the
condition $\theta (0^+(L_p)) \supset \mathbb{Z}^x_p$ is satisfied by
Proposition 1 in previous section. 
\end{remark*}

\begin{theorem}\label{c2:thm-2.35} 
Let\pageoriginale $V$ be a regular quadratic module over $\mathbb{Q}$
with $\dim V\geqq 3$, $S$ a finite set of prime numbers, and $L$ a
$\mathbb{Z}[S]$-lattice on $V$. If $V_{\infty} = \mathbb{R} V$ is
isotropic or $V_{p_o}$ is isotropic for some $p_0 \in S$, then
$spn L = cls L$ .
\end{theorem}

\begin{proof}
Suppose $K\in spn L$. Then there exist isometries
$\mu \in 0(V)$, $\sigma_p \in 0' (V_p)$ for $p\not\in
S$ such that 
$$
\mu(K_p) = \sigma_p (L_p).
$$
Put $T = \{ p\not\in S|\mu(K_p) \neq L_p\}$ (a finite set). Then by
Theorem \ref{c2:thm-2.29}, there is an isometry $\sigma \in 0'(V)$ such
that $\sigma (L_p) =L_p$ if $p \neq T$ or $T \cup \{p_0\}$ according
to the hypothesis, $\sigma$ and $\sigma_p$ are sufficiently close if
$p\in T$.

Hence for $p \in T$, $\sigma(L_p) = \sigma_p (L_p)$ and then
$\mu(K_p) =\sigma(L_p)$ for $p\not\in S$. This leads to
$K=\mu^{-1}\sigma(L)$.
\end{proof}

\setcounter{cor}{0}
\begin{cor}%%%% 1
Let $V$ be a regular quadratic module over $\mathbb{Q}$ with $\dim
V \geqq 3$, and suppose that $V_{\infty}$ is isotropic. If $L$ is a
$\mathbb{Z}$-lattice on $V$ that $s(L)\subset \mathbb{Z}$, $d(L)$ is
odd and square-free, $\gen L=cls L$.
\end{cor}

\begin{proof}
By assumption, $L_2$ is modular and $L_p$ is maximal for $p\neq
2$. Hence Theorems \ref{c2:thm-2.34}, \ref{c2:thm-2.35} and the Remark for
$S=\phi$ imply the 
corollary. 
\end{proof}

\begin{cor}
Let $V$, $W$ be regular quadratic modules over $\mathbb{Q}$ with $\dim\break
V + 3 \geqq \dim W$, and $L(\resp. M)$ a $\mathbb{Z}$-lattice on $V
(\resp W)$. Suppose that 
\begin{align*}
L_p & \text{ is represented by } M_p \text{ for all } p,\\
V_{\infty} & \text{ is represented by $W_{\infty}$, and}\\
W_{\infty} & \text{ is isotropic}. 
\end{align*}
Then\pageoriginale $L$ is represented by $M$.
\end{cor}

\begin{proof}
From the Corollary to Theorem \ref{c2:thm-2.1}, it follows that $V_p$ is
represented by $W_p$, and then the Hasse-Minkowski theorem implies
that $V$ is represented by $W$. We may assume $V \subset W$. By
assumption, there is an isometry $\sigma_p$ from $L_p$ to $M_p$. By
Witt's theorem, we may assume $\sigma_p \in
0(W_p)$. Multiplying a symmetry of $V^{\perp}_p$ from the right, we
may assume $\sigma_p \in 0^+(W_p)$. From Proposition \ref{c2:prop-2.20}
follows the existence of $\eta_p \in 0^+(V^{\perp}_p)$ and
that $\theta (\sigma_p)\theta(\eta_p)=1$. Multiplying $\sigma_p$ by
$\eta_p$ on the right, we may assume $\theta (\sigma_p)=1$. Then there
exists an isometry $\sigma \in 0'(W)$ such that 
\begin{align*}
& \sigma (M_p) = M_p \text{ if } L_p \subset M_p.\\
& \sigma \text{ is sufficiently close to } \sigma_p \text{ if }
L_p \not\subset M_p.
\end{align*}

Hence $\sigma(L_p) \subset M_p$ for every $p$ and so $\sigma
(L) \subset M$.

\subsection{}\label{c2:subsec2.4.4} 
The aim of this paragraph is to prove the fundamental theorem on
representations of positive definite quadratic forms. We mean by
a {\rm positive\break lattice} a quadratic module
$M=\mathbb{Z}[v_1, \ldots, v_m]$ over $\mathbb{Z}$ with basis
$\{v_i\}$ such that $(B(v_i,v_j))$ is positive definite. By
definition, every $B(v_i,v_j)$ is rational. 
\end{proof}


\begin{theorem}[\cite{key8}]\label{c2:thm-2.36}
Let $M$ be a positive lattice of $rank M\geqq 2n + 3$. There is a
constant $c(M)$ such that any positive lattice $N$ of $rank N = n$ is
represented by $M$ provided that 
$$
\min(N):= \min\limits_{0\neq x \in N} Q (x) \geqq c(M), \text{
and }
$$
$N_p$ is represented by $M_p$ for every prime $p$. 
\end{theorem}

The proof is based on several lemmas. 

Let\pageoriginale $\mathbb{N}$ be the set of non-negative integers and
we introduce a partial ordering in $\mathbb{N}^k$ defined by
$(x_1,\cdots, x_k)\leqq (y_1, \ldots, y_k)$ if $x_i \leqq y_i (1\leqq
i \leqq k)$. Then our first lemma is the following. 

\setcounter{lemma}{36}
\begin{lemma}\label{c2:lem-2.37}
Every subset $X$ of $\mathbb{N}^k$ contains only finitely many minimal
elements.
\end{lemma}

\begin{proof}
We use induction on $k$. The assertion is trivial for $k=1$. Write
$x=(x',x_k)$ with $x'=(x_1, \ldots ,
x_{k-1})\in \mathbb{N}^{k-1}$, and put $X'_n = \{
x' \in \mathbb{N}^{k-1}|(x',n) \in X\}$. Let $Y_n, Y'$
be the sets of minimal elements of $X'_n, \bigcup^{\infty}_{n=0}\break X'_n$
respectively. By the inductive assumption, $Y_n, Y'$ are finite
sets. For $y'\in Y'$ we choose and fix an element
$y\in X$ satisfying $y=(y', y_k)$, and denote by $Y$ the set
of such $y$. $Y$ is also a finite set and put $m=\max\{
y_k|y\in Y\}$. Suppose that $x\in X$ is minimal. Then
$x'\in X'_{x_k}$ from the definition and then there exist
$y \in Y$ such that $y'\leqq x'$. If $x_k \geqq y_k$, then
$x \geqq y$ and then $x=y\in Y$ in view of the minimality of
$x$. Suppose $x_k < y_k (\leqq m)$. Since $x$ is minimal in $X_{x_k}$,
$x$ is minimal in $X'_{x_k}$. Hence $x\in (Y_{x_k},
x_k) \subset \cup^m_{n=0} (Y_n, n)$. Thus every minimal element $x$ is
in a finite set $Y\cup \cup^m_{n=0} (Y_n,n)$. 
\end{proof}

\begin{lemma}\label{c2:lem-2.38}
Let $M_p$ be a regular quadratic module over $\mathbb{Z}_p$ of $rank\break
M_p = m \geqq n$. Then there are only finitely many regular submodules
$N_p(j)$ of $rank N_p(j) =n$ such that each regular regular submodule
$N_p$ of $rank N_p =n$ of $M_p$ is represented by some $N_p(j)$.
\end{lemma}

\begin{proof}
If is obvious that the assertion holds if it is true for $p^{\tau}M_p$
instead of $M_p$. Hence we may assume that
$s(M_p) \subset \mathbb{Z}_p$. Let $N_p$ be a regular quadratic module
of rank $n$ and $N_p =\mathop{\perp}\limits^t_{i=1} L_i$ a Jordan
splitting. Since $L_i$ is modular, $p^{-b_i} L_i =K_i$\pageoriginale
is unimodular or $(p)$-modular for some
$b_i \in \mathbb{N}$. By virtue of Propositions \ref{c2:prop-2.12} and \ref{c2:prop-2.13},
there are only finitely many isometric modules over $\mathbb{Z}_p$ of
unimodular or $(p)$-modular quadratic modules of fixed rank. Thus
there are only finitely many possibilities for the whole collection
$(rank L_i, K_i)$. Fix one of these and consider the corresponding
$(b_1, \ldots, b_t)$. By Lemma \ref{c2:lem-2.37}, there exist only finitely many
minimal ones. It is clear that if 
\begin{align*}
&N_p=\mathop{\perp}\limits^t_{i=1} L_i, \;\; N'_p
= \mathop{\perp}\limits^t_{i=1} L'_i,\\  
&rank L_i = rank
L'_i, \;\; p^{-b_i} L_i \cong p^{-b'_i} L'_i,\\
&b_i \leqq b'_i \;\; \text{for} \;\; 1\leqq i \leqq t, 
\end{align*}
then $N'_p$ is represented by $N_p$. Hence $N_p$, ranging over all
possible collections $(rank L_i, K_i)$ and minimal families $(b_i)$,
constitute a finite family with the required property.
\end{proof}

\begin{lemma}\label{c2:lem-2.39}
Let $L$ be a positive lattice of rank $L \geqq 3$ and suppose that
$L_p$ is maximal for all $p$, and let $q$ be a prime such that
$(\mathbb{Q}L)_p$ is isotropic. Then there is a natural number $s$
such that $L$ represents every positive lattice $N$ for which $q^sL_p$
represents $N_p$ for every prime $p$.
\end{lemma}

\begin{proof}
Let $\{L_i\}$ be a complete set of representatives of classes in $\gen
L$. From Theorem \ref{c2:thm-2.35}, it follows that $spn \mathbb{Z}\{q^{-1}\} L =
cls \mathbb{Z}[q^{-1}] L$. On the other hand, our assumption implies
$\gen L=spn L$ and then $\gen \mathbb{Z}[g^{-1}]\break L=spn \mathbb{Z}
[g^{-1}]L$ by virtue of Proposition \ref{c2:prop-2.18} and Theorem
\ref{c2:thm-2.34}. Thus we 
have $\gen \mathbb{Z}[q^{-1}] L = cls \mathbb{Z}[g^{-1}]L$. Hence
there is an isometry $\sigma_i \in 0(\mathbb{Q}L)$ such that
$\mathbb{Z}[q^{-1}] L=\mathbb{Z}[g^{-1}]\sigma_i(L_i)$. We determine
$s$ by
$$
q^s \sigma_i(L_i) \subset L \text{ for every } i.
$$
The lemma follows immediately from Theorem \ref{c2:thm-2.28}.
\end{proof}

\begin{lemma}\label{c2:lem-2.40}
Let\pageoriginale $L,q,s$ be as in Lemma \ref{c2:lem-2.39}, $rank L \geqq n + 3$,
$K$ a positive lattice. Then there is a constant $c$ such that $K\perp
L$ represents a positive lattice $N=\mathbb{Z}[v_1, \ldots, v_n]$ of
$rank n$ for which $N_p$ is represented by $K_p \perp q^s L_p$ for
every $p$, and $(B(v_i, v_j)) > c E_n$.
\end{lemma}

\begin{proof}
Let $S$ be a finite set of prime numbers such that $S\ni 2, q$ and for
$p\not\in S K_p, L_p$ are unimodular, and fix a natural number $r$
such that $p^rs(K_p) \subset \underline{n} (q^sL_p)$ for
$p \in S$. Choose vectors $v^h_i \in K(i=1,2, \ldots,
n, h=1, \ldots, t)$ so that for given $x_{1,p}, \ldots,
x_{n,p} \in K_p$, we have     
$$
v^h_i \equiv x_{i.p} \mod p^r K_p \cdots (\ast)
$$
for some $h(q\leqq h \leqq t)$ and every $i=1,2,\ldots, n$ and all
$p\in S$. We choose a positive number $c$ so that 
$$
cE_n -(B(v^h_i, v^h_j)) >0 \text{ for } h =1, \ldots, t.
$$
Let $N=\mathbb{Z}[v_1,\ldots, v_n]$ be a lattice which satisfies the
conditions in the lemma. By the first condition, there exist
$x_{i,p} \in K_p$, $y_{i,p} \in q^s L_p$ such that 
$$
B(v_i, v_j) = B(x_{i,p}, x_{j,p}) + B(y_{i,p}, y_{j,p}) \text{ for all
} p.
$$
For some $h$ satisfying $(\ast)$ for these $x_{i,p}$, we put 
$$
A=(B(v_i, v_j)) - (B(v^h_i, v^h_j)).
$$

We have only to prove that $A$ is represented by $L$. All the entries
of $A$ are rational and $A$ is positive definite, since $A=((B(v_i,
v_j)) -cE_n+(cE_n-(B(v^h_i, v^h_j))>0$. Let\pageoriginale
$H=\mathbb{Z}[u_1, \ldots, u_n]$ be a positive lattice such that
$(B(u_i, u_j))=A$. Put $x_{i,p} =v^h_i+ p^rz_{i,p}(z_{i,p}\in
K_p)$. Then 
\begin{align*}
A & = (B(x_{i,p}, x_{j,p})) + (B(y_{i,p}, y_{j,p})) - (B(v^h_i,
v^h_j))\\
& = (B(v^h_j, v^h_j) + p^r B(v^h_i, z_{j,p}) + p^r B(z_{i,p}, v^h_j) +
 p^{2r}B(z_{i,p}, z_{j,p})) +\\
& \qquad \quad + (B(y_{i,p}, y_{j,p}) -(B(v^h_i, v^h_j))
\end{align*}
holds.

By the choice of $r$, the $(i,j)$th entry of $A$ is congruent to
$B(y_{i,p}, y_{j,p})$ modulo $\underline{n}(q^sL_p)$ for $p\in
S$. It follows from $y_{i,p} \in q^s L_p$ that
$\underline{n}(H_p) \subset \underline{n} (q^s L_p)$ for $p\in
S$. Since $v_i \in N$, $v^h_i \in K$ and
$\underline{n}(N_p) \subset \underline{n} (K_p\perp q^s
L_p) \subset \mathbb{Z}_p$ for $p\not\in S$, we have
$\underline{n}(H_p) \subset \mathbb{Z}_p =\underline{n}(q^sL_p)$ for
$p\not\in S$. Thus we have proved
$\underline{n}(H_p) \subset \underline{n} (q^sL_p)$ for every
$p$. Proposition \ref{c2:prop-2.10} implies that $H_p$ is represented by $q^sL_p$
for all $p$. From Lemma \ref{c2:lem-2.39}, it follows that $H$ is represented by
$L$ and the proof is complete.
\end{proof}

\setcounter{proofofthm}{35}
\begin{proofofthm}%%% 2.36
Let $M$ be a positive lattice of $rank M \geqq 2n +3$. Let $S$ be a
finite set of prime numbers such that $S\ni 2$ and $M_p$ is unimodular
for $p\not\in S$ and $M_q$ is unimodular for some $q(\neq
2)\in S$. We construct a set of submodules $K(J)$, $L(J)$ of
$M$ as in Lemma \ref{c2:lem-2.40} and show that $N$ satisfies the condition in
Lemma \ref{c2:lem-2.40} for some $J$. For each $p\in S$, we choose finitely
many submodules $N_p(j_p)$ of rank $n in M_p$ according to Lemma
\ref{c2:lem-2.38} and to each collection $J=(j_p)_{p\in S}$ we take a
submodule $K(J)$ or rank $n \in M$ satisfying the conditions
$K(J)_p \cong N_p (j_p)$ and $d(K(J)) \in \mathbb{Z}^x_p$ or
$p\mathbb{Z}^x_p$ for $p\not\in S$ by Theorem \ref{c2:thm-2.33} and
Corollary \ref{c2:coro-4}  to
Theorem \ref{c2:thm-2.14}. We construct a submodule $L(J)$ of\pageoriginale $rank
L(J) =rank M-n \geqq n +3$ in $\{x\in M|B(x, K(J))=0 \}$ as
follows:
For $p\not\in S$, $L(J)_p=K(J)^{\perp}_p=\{ x\in M_p|B(x,
K(J_p))=0\}$. In this case, $L(J)_p$ is $(\mathbb{Z}_p-)$ maximal,
since $s(L(J)_p) \subset \mathbb{Z}_p$ and
$d(L(J)_p) \in \mathbb{Z}^x_p \cup p \mathbb{Z}^x_p$ by the local
version of Lemma \ref{c2:lem-2.26}. For $p\in S$, we take any maximal module in
$\{x\in M_p |B(x,K(J)_p)=0 \}$. From Proposition \ref{c2:prop-2.18} and
Theorem \ref{c2:thm-2.34}, it follows that $\gen L = spn L$. We show that $L(J)_q$
is isotropic. If $rank L(J)_q \geqq 5$, then $L(J)_q$ is
isotropic. Otherwise, we have $rank L(J)_q=4$, $n=1$, $rank M_q=S$. By
the assumption $q(\neq 2) \in S$, $M_q$ is unimodular. Hence
$M_q=<\left(\begin{smallmatrix}
0&1\\1&0
\end{smallmatrix}\right)> \perp <\left(\begin{smallmatrix}
0&1\\1&0
\end{smallmatrix}\right)> \perp <\ast>$. Unless $L(J)_q$ is isotropic,
$\mathbb{Q}_qM$ does not contain two copies of hyperbolic planes. Thus
$L(J)_q$ is isotropic. Let $N$ be a positive lattice of $rank \; n$ such
that $N_p$ is represented by $M_p$ for every $p$. Suppose $p\not\in
S$; then $M_p$ is unimodular. Hence
$\underline{n}(N_p)\subset \mathbb{Z}_p$. Since $L(J)_p$ is
$\mathbb{Z}_p$-maximal and $rank L(J)_p\geqq n+3$, $N_p$ is
represented by $L(J)_p =q^sL(J)_p$ by Proposition \ref{c2:prop-2.10} for every
$J$. For $p\in S$, $N_p$ is represented by $K(J)_p$ for some
$J$. By Lemma \ref{c2:lem-2.40}, there is a constant $c(J)$ so that $N$ is
represented by $K(J)\perp L(J)\subset M$ if $(B(v_i, v_j))>c(J)$ for
some basis $\{v_i\}$ of $N$. Put $c'=\max\limits_J c(J)$. By reduction
theory, there is a basis $\{v_i\}$ of $N$ such that 
$$
(B(v_i, v_j)) \in S_{4/3, 1/2}, \text{ and then } (B(v_i,
v_j)) \gg \left(\begin{smallmatrix}
Q(v_1) & & & & \\
& \cdot & & & \\
& & \cdot & & \\
& & & \cdot & \\
& & & & Q(v_n)
\end{smallmatrix} \right).
$$ 
If $\min\limits_{0\neq v \in N} Q (v)$ is sufficiently large,
then we have $(B(v_i, v_j)) > c'E_n$. 

This completes the proof.
\end{proofofthm}

\begin{remark*}
By the analytic considerations in \S \ref{c1:sec-1.7} of Chapter
\ref{c1}, the following 
assertion holds for $n=1$, $m\geq 4$ or $n=2$, $m\geq 7$.

Let $M$\pageoriginale be a positive lattice with $M=m$. There is a
constant $c(M)$ 
such that any positive lattice $N$ with rank $N=n$ is primitively
represented by $M$ provided that 
$$
\min(N) = \min_{0\neq x\in N} Q(x) \geqq c (M) \text{ and }
$$

$N_p$ is primitively represented by $M_p$ for every prime $p$.
\end{remark*}

\subsection{}\label{c2:subsec2.4.5}
 In this last subsection, we show that
there is a submodule of codim 1 which characterizes a given module.

Let $L=L_1\perp \cdots \perp L_k$ be a Jordan splitting of a regular
quadratic module $L$ over $\mathbb{Z}_p$, that is, every $L_i$ is
modular and
$s(L_1) \mathop{\supset}\limits_{\neq} \cdots \mathop{\supset}\limits_{\neq}
s(L_k)$. Then we put 
$$
t_p (L) = (\underbrace{a_1,\ldots,a_1}_{\text{rank }
L_1},\ldots, \underbrace{a_k,\ldots, a_k}_{\text{rank } L_k})
$$
where $a_i$ is defined by $p^{a_i} \mathbb{Z}_p =s(L_i)$ and then $a_1
<a_2 <\ldots <a_k$. For two ordered sets $a=(a_1,\ldots, a_n),
b=(b_1,\ldots, b_n)$, we define the ordering $a\leqq b$ by either $a_i
=b_i$ for $i<k$ and $a_k < b_k$ for some $k\leqq n$ or $a_i = b_i$ for
all $i$. For brevity, we denote $t_p(L_p)$ by $t_p(L)$ for a regular
quadratic module over $\mathbb{Z}$.

\begin{lemma}\label{c2:lem-2.41}
Let $L$ be a $\mathbb{Z}_p$-lattice on a regular quadratic module $U$
over $\mathbb{Q}_p$. Then $L$ contains a $\mathbb{Z}_p$-submodule $M$
satisfying the following conditions 1), 2):
\begin{enumerate}
\renewcommand{\labelenumi}{\theenumi)}
\item $d(M) \neq 0$, rank $M=$ rank $L-1$ and $M$ is a direct summand
of $L$ as a module.

\item Let $L'$ be a $\mathbb{Z}_p$-lattice on $U$ containing $M$. If
$d(L')=d(L)$ and $t_p(L')\geqq t_p(L)$, then $L'=L$. 
\end{enumerate}
\end{lemma}

\begin{proof}
First,\pageoriginale we assume that $L$ is modular. Multiplying the
quadratic form by some constant, we may suppose that $L$ is
unimodular, without loss of generality. Let $L'$ be a lattice as in
2). Then $t_p(L') \geqq t_p(L) = (0,\ldots, 0)$ implies $s(L') \subset
\mathbb{Z}_p$, and $d(L')=d(L)$ implies that $L'$ is
unimodular. Suppose that $L$ has an orthogonal basis, that is,
$L=\mathop{\perp}\limits^n_{i=1} \mathbb{Z}_pv_i$. Then we put $M
= \mathop{\perp}\limits^{n-1}_{i=1} \mathbb{Z}_p v_i$. The condition
1) is trivially satisfied. $L'$ is split by $M$, in view of Lemma
\ref{c2:lem-2.5}. Thus $L'=M\perp a \mathbb{Z}_p v_n
(a\in \mathbb{Q}^x_p)$. Further, $d(L') = d(L)$ implies
$a\in \mathbb{Z}^x_p$ and $L'=L$. Suppose that $L$ does not
have any orthogonal basis. Then, from Propositions \ref{c2:prop-2.12} and
\ref{c2:prop-2.13}, it 
follows that $p=2$ and 
\begin{align*}
&L = \mathop{\perp}^n_{i=1} \mathbb{Z}_2 [u_i, v_i],\\  
&\mathbb{Z}_2 [u_i,v_i] = <\begin{bmatrix}
0&1\\1&0\end{bmatrix}> \text{ for } i<k,\\ 
&\mathbb{Z}_2[u_k, v_k] = <\begin{bmatrix}
2c & 1\\
1 & 2c
\end{bmatrix}>
\end{align*}
$c=0$ or 1. Let $Q(u_k)=Q(v_k)=2c$, $B(u_k, v_k)=1$ and put
$M=\mathop{\perp}\limits^{k-1}_{i=1} \mathbb{Z}_2[u_i,v_i]\perp \perp \mathbb{Z}_2
[u_k+v_k]$. Then condition 1) is satisfied. From Lemma \ref{c2:lem-2.5}, it now
follows that $L'=\mathop{\perp}\limits^{k-1}_{i=1}\mathbb{Z}_2
[u_i,v_i] \perp L''$. Moreover, $L''$ is unimodular and $L''\ni u_k
+v_k$. Since $Q(u_k+v_k)=2(2c+1)$, $u_k + v_k$ is primitive in
$L''$. Hence $L''=\mathbb{Z}_2 [u+v, au+bv](u=u_k, v=v_k)$, for some
$a,b \in \mathbb{Q}^x_2$. Since $L''$ is unimodular, and
$Q(u+v)=2(2c+1)$, we have $B(u+v, au+bv)\in \mathbb{Z}^x_2$
and $Q(au+bv)\in \mathbb{Z}_2$. Thus $B(u+v,
au+bv)=(a+b)(2c+1)\in \mathbb{Z}^x_2$ and $Q(au+bv) = 2(2c-1)
a^2 -2(2c-1) ax + 2cx^2 \in \mathbb{Z}_2 (x=a+b)$. Hence $x\in
\mathbb{Z}^x_2$ and $2a(a-x) \in \mathbb{Z}_2$. This implies
$a\in \mathbb{Z}_2$ and $b=x-a\in \mathbb{Z}_2$. Thus
we have $L''=\mathbb{Z}_2[u,v]$ and $L'=L$. Returning to the general
case, let $L=\mathop{\perp}\limits^k_{i=1} L_i$, where $L_i$ is
$p^{a_i} \mathbb{Z}_p$-modular and $a_1 <\cdots <a_k$. Denote by $M_k$
a submodule of $L_k$ which satisfies 1), 2) for $L_k$, and put
$M=\mathop{\perp}^{k-1}_{i=1} L_i \perp M_k$. Then condition 1) is
obviously satisfied. For a lattice $L'$ as in 2), $L'$ contains a
modular module $L_1$ and\pageoriginale $t_p(L') \geqq t_p(L)$ implies
$s(L') \subset s (L_1)$. By Lemma \ref{c2:lem-2.5}, $L'=L_1 \perp L''$, and
$t_p(L'') \geqq t_p (\mathop{\perp}\limits_{1\geqq 2} L_i)$ and
clearly $L''\supset \mathop{\perp}\limits^{k-1}_{1=2} L_i \perp
M_k$. Repeating this argument, we get $L'
= \mathop{\perp}\limits_{i<k} L_i \perp \tilde{L}$,
$t_p(\tilde{L}) \geqq t_p (L_k)$, $\tilde{L} \supset M_k$,
$d(\tilde{L}) = d(L_k)$. Thus we have $L'=L$.

We call a submodule $M$ in Lemma \ref{c2:lem-2.41} a \textit{characteristic
submodule} of $L$. Obviously the images of a characteristic submodule
by $0(L)$ are also characteristic.
\end{proof}

\setcounter{theorem}{41}
\begin{theorem}\label{c2:thm-2.42}
Let $L$ be a $\mathbb{Z}$-lattice on regular quadratic module $U$ over
$\mathbb{Q}$; then $L$ contains a $\mathbb{Z}$-submodule $M$
satisfying the following conditions 1), 2): 
\begin{enumerate}
\renewcommand{\labelenumi}{\theenumi)}
\item $d(M) \neq 0$, $rank M=rank L-1$, and $M$ is a direct summand of
$L$ as a module.

\item Let $L'$ be a $\mathbb{Z}$-lattice on a regular quadratic module
$U'$ over $\mathbb{Q}$ satisfying $d(L')=d(L)$, $rank$ $L'=rank$ $L$,
$t_p(L') \geqq t_p(L)$ for every prime $p$. If there is an isometry
$u$ from $M$ to $L'$, then $L'$ is isometric to $L$. 
\end{enumerate}
\end{theorem}

\begin{proof}
We separate the case when $U$ is a hyperbolic plane.

Suppose that $U$ is a hyperbolic plane and further, let
$L=\mathbb{Z}[u_1, u_2]$, $(B(u_i,u_j)) = \left( \begin{smallmatrix}
0 & b'\\ b' & c'
\end{smallmatrix}\right)$. Multiplying the quadratic form on $U$ some
constant, we may assume $2|c'$, and $(b',c'/2)=1$ without loss of
generality. Since $Q(xu_1+ u_2) = 2(xb'+c'/2)$, there is an integer
$x$ such that $\dfrac{1}{2} Q(xu_1+u_2)$ is a prime number $q$ with
$(q, 2dL)=1$. Hence to $L$ corresponds the matrix
$\left(\begin{smallmatrix}
2q & b\\b& c
\end{smallmatrix}\right)$ with $0<b<q$. It is easy to see that $b,c$
are uniquely determined by $q$ and $d(L)$. We put
$M=\mathbb{Z}[xu_1+u_2]$, and let $L'$ be a lattice in 2). From the
hypothesis, it follows that $s(L'_p)\subset \mathbb{Z}_p$ for every
$p$ and hence to $L'$ corresponds the matrix
$\left(\begin{smallmatrix}
2q & b''\\b'' &c''
\end{smallmatrix}\right)$ with $0<b''<q$. Hence $b''=b$,
$c''=c$. As\pageoriginale a result, $L'\cong L$. From now on, we
suppose that $U$ is not a hyperbolic plane. Let $S$ be a set of prime
numbers such that $S\ni 2$, and $L_p$ is unimodular for $p\not\in S$,
and $\tilde{M}_p$ a characteristic submodule of $L_p$ for
$p\in S$. Suppose $\mathbb{Z}_px_p = \tilde{M}^{\perp}_p =\{
x\in L_p|B(x,\tilde{M}_p) =0\}$. Then $x^{\perp}_p
=\tilde{M}_p$, since $\tilde{M}_p$ is a direct summand of $L_p$. By
Theorem \ref{c2:thm-2.33}, there exists an element $x\in L$ such that $x$
and $x_p$ are sufficiently close for $p\in S$ and
$Q(x) \in \mathbb{Z}^x_p$ for $p\not\in S$ with precisely one
exception $p=q$, where $Q(x)\in q \mathbb{Z}^x_p$. We put
$M=x^{\perp}$. Then $M$ satisfies the condition 1). From Corollary \ref{c2:coro-4}
to Theorem \ref{c2:thm-2.14}, it follows that $\mathbb{Z}_p x$ and $\mathbb{Z}_p
x_p$ are transformed by $0(L_p)$ for $p\in S$. Thus
$\tilde{M}_p, M_p$ are also transformed by $0(L_p)$. Hence $M_p$ is a
characteristic submodule of $L_p$. If $p \not\in S$, $p \neq q$, then
$M_p$ is unimodular and then $M_p$ is a characteristic submodule of
$L_p$. Let $L'$ be a lattice as in 2). Then
$\mathbb{Q}L'=\mathbb{Q}u(M) \perp <d(L')d(M)>\cong \mathbb{Q}
L$. Hence we may suppose that $L'$ is a lattice on $U$ and $L'$
contains $M$. Since $M_p$ of Lemma \ref{c2:lem-2.26}, we have $d(M_q)\in
q \mathbb{Z}^x_p$. Hence there is a basis $\{ w_i\}$ of $M_q$ such
that $\mathop{\perp}\limits_{i\leqq n -2} \mathbb{Z}_qw_i$ is unimodular and
$Q(w_{n-1})\in q \mathbb{Z}^x_q$. Since
$\mathop{\perp}\limits_{i\leqq n -2} \mathbb{Z}_{q} w_i$ splits $L_q$,
and $M_q$ is a direct summand of $L_q$, there is $w_n \in L_p$
such that $\{w_1,\ldots, w_n\}$ is a basis of $L_q$. Since
$N=\mathbb{Z}_q [w_{n-1}, w_n]$ is unimodular,
$d(N)=Q(w_{n-1})Q(w_n)-B(w_{n-1},w_n)^2$ is a unit. From
$Q(w_{n-1})\in q\mathbb{Z}^x_q$, it follows that
$B(w_{n-1},w_n) \in \mathbb{Z}^x_q$ and $\mathbb{Q}_qN$ is
hyperbolic. By Lemma \ref{c2:lem-2.4}, there is a basis $\{e_1,e_2\}$ of $N$ such
that $Q(e_i)=0(i=1,2)B(e_1, e_2)=1$. Put $w_{n-1} =a_1 e_1+a_2 e_2
(a_i \in \mathbb{Z}_q)$; then $2a_1 a_2 \in
q \mathbb{Z}^x_q$. Multiplying $e_i$ by a unit and renumbering, we may
suppose $w_{n-1}=e_1+vqe_2(v\in \mathbb{Z}^x_q)$. Since $L'_q$ is
unimodular and $L'_q$ contains $M_q$, there is a unimodular submodule
$K_q$ such that $L'_q=\mathop{\perp}\limits_{i\leqq n -2} \mathbb{Z}_q
w_i \perp K_q, K_q \ni w_{n-1}$. Let\pageoriginale $\{w_{n-1},
ce_1+de_2\}$ be a basis of $K_q$. Since $K_q$ is unimodular, we have
$d+vqc \in \mathbb{Z}^x_q$ and
$cd \in \mathbb{Z}_q$. Then $c\in
q^{-1}\mathbb{Z}^x_q, d \in q \mathbb{Z}_q$ or
$c\in \mathbb{Z}_q$, $d\in \mathbb{Z}^x_q$. Thus we
have $K_q=\mathbb{Z}_q[q^{-1} e_1, qe_2]$ or $\mathbb{Z}_q [e_1,
e_2]$. Since $B(x,M)=0$ and $B(e_1 - vqe_2, M_q)=0$,
$\tau_x=\tau_{e_1-vqe_2}$. It is easy to see that
$\tau_{e_1-vqe_2}\mathbb{Z}_q[e_1,e_2]=\mathbb{Z}_q[q^{-1}e_1,qe_2]$. Thus
we have $L'_q =L_q$ or $\tau_x L_q$.
Since $\tau_xM_p = M_p$ and $M_p$ is a characteristic submodule of
$L_p$ for $p\neq q$, we have $L'_p = L_p = \tau_x L_p$. Thus we have
$L' = L$ or $\tau_x L$. 
\end{proof}

\begin{remark*}
Let $L$ be a regular quadratic module over $\mathbb{Z}$ and $S$ a
finite set of prime numbers such that $2\in S$ and $L_p$ is
unimodular for $p\not\in S$, and let $M$ be a submodule of $L$, of
rank $=rank L-1$, such that $M_p$ is characteristic for $p\in
S$ and for $p\not\in S$, $d(M_p)\in \mathbb{Z}^x_p$ with
precisely one exception $p=q$ and $d(M_q)\in
q \mathbb{Z}^x_q$. Let $u$ be an isometry from $M$ to $L$. Extend $u$
to an isometry of $\mathbb{Q}L$. Another extension is
$u\tau_x(x\in M^{\perp})$. The proof shows that $u^{-1}(L)=L$
or $\tau_xL$. Hence $u$ is uniquely extended to an isometry of $L$. In
particular, if $L$ is positive definite, then we have 
$$
r(M,L) = \sharp \{\text{isometries }: M\to L\} = \sharp 0(L).
$$
\end{remark*}

\setcounter{cor}{0}
\begin{cor}
Let $\{L_i\}^m_{i=1}$ be a set of regular quadratic modules over
$\mathbb{Z}$ such that $rank L_i = n, d(L_i) = d(1\leqq i \leqq m)$,
and $L_i \neq L_j$ if $i\neq j$. Then there is a regular quadratic
module $M$ over $\mathbb{Z}$ such that $rank M=n-1$ and there is
precisely one $i(1\leqq i \leqq m)$ for which $M$ is represented by
$L_i$. 
\end{cor}

\begin{proof}
Let $S$ be a finite set of prime numbers such that $2\in S$
and $(L_i)_p$ is unimodular for $1\leq i \leq m$, $p\not\in S$. Put
$S=\{p_1,\cdots,p_r\}$ and define $A_1,\ldots, A_r$ as follows: 
\begin{align*}
A_1 & = \{L_i ; t_{p_1}(L_i) \text{ is minimal in } \{t_{p_1} (L_j);
1\leqq j \leqq m\} \}, \ldots,\\
A_{k+1} & = \{L_i; t_{p_{k+1}} (L_i)\text{ is minimal in
} \{t_{p_{k+1}} (L_j); L_j \in A_k\} \}.
\end{align*}\pageoriginale 
Suppose $L_i \in A_r$, and $M$ is a submodule of $L_i$ which is
constructed in the proof of Theorem \ref{c2:thm-2.42}. Assume $M$ is represented by
$L_j$. Since $L_i \in A_r \subset A_1, t_{p_1} (L_i) \leqq
t_{p_1} (L_j)$. Further, $M_{p_1}$ is a characteristic submodule of
$L_i$. Hence $(L_i)_p \cong (L_j)_p$ and then $t_{p_1}(L_i) =
t_{p_1}(L_j)$. Thus $L_j$ belongs to $A_1$. Repeating this argument,
we have $L_j \in A_r$. Thus $t_p(L_i) = t_p(L_j)$ for every
$p$. From Theorem \ref{c2:thm-2.42}, it follows that $L_j$ is isometric to
$L_j$. This completes the proof.
\end{proof}

\begin{cor}[\cite{key9}]
Let $\{S_i\}^m_{i=1}$ be a set of positive definite rational symmetric
matrices such that $rank$ $S_i =n$, $|S_i|=d(1\leqq i \leqq m)$ and
there is no element $T\in GL_n(\mathbb{Z})$ which satisfies $S_i
[T]=S_j$ if $i\neq j$. Then $\theta(Z,S_i) = \sum e(\sigma(S_i[G]Z))$
are linearly independent where $G$ runs over
$\mathfrak{M}_{n,n-1}(\mathbb{Z})$ and 
$$  
Z \in H_{n-1} = \{Z\in \mathfrak{M}_{n-1}
(\mathbb{C})|Z = {}^tZ, Im Z>0\}.
$$
\end{cor}

\begin{proof}
This follows immediately from the previous corollary.
\end{proof}

\begin{thebibliography}{99}
\bibitem{key1} A.N. Andrianov: Spherical\pageoriginale functions for
$GL_n$ over local fields and summation of Hecke series, Math. Sbornik
12 (1970), 429-452.

\bibitem{key2} A.N. Andrianov and G.N. Maloletkin: Behaviour of
theta-series of degree $n$ under modular substituions, Math. USSR
Izv. 9(1975), 227-241.

\bibitem{key3} H. Braun: Darstellung hermitischer Modulformen durch
Poincaresche Reihen, Abhand. Math. Sem. Hamburg 22 (1958), 9-37.

\bibitem{key4} J.W.S. Cassels: Rational Quadratic Forms, Academic
Press, 1978.

\bibitem{key5} R. Carlsson and W. Johanssen: Der Multiplikator von
Thetareihen h\"oheren Grades $zu$ quadratischen Formen ungerader
Ordnung, Math, Zeit 177 (1981), 439-449.

\bibitem{key6} U. Christian: Uber Hilbert-Siegelsche Modulformen und
Poincaresche Reihen, Math. Annalen 148 (1962), 257-307.

\bibitem{key7} E. Hecke: Theorie der Eisensteinscher Reihen und ihre
Anwendung auf Funktionentheorie und Arithmetik, Abh Math. Sem. Hamburg
5 (1927), 199-224, Gesamm. Abhand. 461-486. 

\bibitem{key8} J. C Hsia, Y. Kitaoka and M. Kneser: Representations
of positive definite quadratic forms, Jour. reine angen. Math, 301
(1978), 132-141.

\bibitem{key9} Y. Kitaoka: Representations of quadratic forms and
their application to Selberg's zeta functions, Nagoya Math. J. 63
(1976), 153-162.


\bibitem{key10} Y. Ditaoka: Modular forms of degree $n$ and
representation by quadratic forms I, II, III Nagoya Math. J. 74
(1979), 95-122, ibid. 87(1982). 127-146, Proc. Japan. Acad. 57
(1981). 373-377. 

\bibitem{key11} Y. Kitaoka:\pageoriginale Fourier coefficients of
Siegel cusp forms of degree 2, Nagoya Math. J. 93 (1984), 149-171. 

\bibitem{key12} Y. Kitaoka: Dirichlet series in the theory of Siegel
modular forms, Nagoya Math. J. 95 (1984), 73-84.

\bibitem{key13} H. Klingen: Zum Darstellungssatz fur Siegelsche
Modulformen, Math. Zeit. 102 (1967), 30-43.

\bibitem{key14} H.D. Kloosterman: Asymptotische Foremeln fur die
Fourierkoeffizienten ganzer Modulformen, Math. Sem. Hamburg 5 (1927),
337-352. 

\bibitem{key15} M. Kneser: Quadratische Formen,
Vorlesungs-Ausarbeitung, Giottingen, 1973-'74.

\bibitem{key16} M. Koecher: Zur Theorie der Modulformen $n$-ten
Grades I, II, Math. Zeit. 59 (1954), 399-416, ibid. 61 (1955),
455-466. 

\bibitem{key17} H. Maass: Siegel's Modular Forms and Dirichlet
Series, Lecture Notes in Math. 216, Springer-Verlag, 1971.

\bibitem{key18} O.T.O'Meara: Introduction to Quadratic Forms,
Grundlehren Math. Wissen. 117, Springer-Verlag, 1973. 

\bibitem{key19} S. Raghavan: Modular forms of degree $n$ and
representation by quadratic forms, Annals Math. 70 (1959), 449-477.

\bibitem{key20} S. Raghavan: Estimates of coefficients of modular
forms and generalized modular relations, International Colloq. on
Automorphic forms, Representation theory and Arithmetic, Bombay 1979.

\bibitem{key21} J.P. Serre: A Course in Arithmetic, Springer-Verlag,
1973. 

\bibitem{key22} G. Shimura: On Eisenstein series, Duke Math. J. 35
(1984), 73-84. 

\bibitem{key23} C.L. Siegel:\pageoriginale \"Uber die analytische
Theorie der quadratischen Formen, Annals Math. 36 (1935), 527-607,
Gesamm. Abhand.I, 326-405.

\bibitem{key24} C.L. Siegel: Einf\"uhrung in der Theorie der
Modulfunktionen, Math. Annalen 116 (1939), 617-657,
Gesamm. Abhand. II, 97-137.

\bibitem{key25} C.L. Siegel: On the theory of indefinite quadratic
forms, Annals Math. 45 (1944), 577-622, Gesamm. Abhand II, 421-466.

\bibitem{key26} T. Tamagawa: On the $\varsigma$-functions of a
division algebra, Annals Math. 77 (1963), 387-405.

\bibitem{key27} W.A. Tartakowsky: Die Gesamtheit der Zahlen, die
durch eine positive quadratische Form $F(x_1, \ldots, x_s)(s\geq 4)$
darstellbar sind, Izv. Akad. Nauk SSSR u (1929), 111-122, 165-196.

\bibitem{key28} A. Weil: On some exponential sums,
Proc. Nat. Acad. Sci. USA, 34 (1948), 204-207.

\end{thebibliography}
