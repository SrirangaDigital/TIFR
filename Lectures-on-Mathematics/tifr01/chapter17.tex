\chapter{The Zeta Function of Riemann (Contd).}\label{chap17}

\setcounter{section}{11}
\section[The Prime Number Theorem]{The Prime Number Theorem \cite[pp.1-18]{key11}}\label{chap17:sec12}
 The number\pageoriginale of
primes is infinite. For consider any finite set of primes; let $P$
denote their product, and let $Q=P+1$. Then $(P,Q)=1$, since any
common prime factor would divide $Q-P$, which would be impossible if
$(P,Q) \neq 1$. But $Q>1$, and so divisible by some prime. Hence there
exists at least one prime distinct from those occurring in $P$. If
there were only a finite number of primes, by taking $P$ to be their
product we would arrive at a contradiction.

Actually we get a little more by this argument: if $p_n$ is the
$n^{\rm th}$ prime, the integer $Q_{n} = p_1 \ldots p_n+1$ is
divisible by some $p_m$ with $m>n$, so that
$$
p_{n+1} \leq p_m \leq Q_n
$$
from which we get, by induction, $p_n < 2^{2^n}$. For if this
inequality was true for $n=1,2,\ldots N$, then 
$$
p_{N+1} \leq p_1 \ldots p_N + 1 < 2^{2+4+\cdots + 2^N} + 1 < 2^{2^{N+1}},
$$
and the inequality is known to be true for $n=1$.

The fact that the number of primes is infinite is equivalent to the
statement that $\pi (x) = \sum\limits_{p\leq x} 1 \to \infty$ as $x \to
\infty$. Actually far more is true; we shall show that $\pi(x) \sim
\dfrac{x}{\log x}$. Again the relation $p_n< 2^{2^n}$ is far weaker
than the asymptotic formula: $p_n \sim n \log n$, which\pageoriginale
is know to be equivalent to the formula: $\pi (x) \sim \dfrac{x}{\log
  x}$ (The Prime Number Theorem!). We shall prove the Prime Number
Theorem, and show that it is equivalent to the non-vanishing of the
zeta function on the line $\sigma =1$. We shall state a few
preliminary results on primes.


%\setcounter{thm}{0}
\begin{thm}\label{chap17:thm1}
The series $\sum \dfrac{1}{p}$ and the product $\pi
\left(1-\dfrac{1}{p} \right)^{-1}$ are divergent.
\end{thm}

\begin{proof}
Let $S(x) = \sum\limits_{p\leq x} \dfrac{1}{p}$, \;\; $P(x) =
\prod\limits_{p\leq x} \left( 1-\dfrac{1}{p}\right)^{-1}$, \;\; $x>2$.

Then, for $0<u<1$,
$$
\frac{1}{1-u} > \frac{1-u^{m+1}}{1-u} = 1+ u + \cdots + u^m
$$
Hence
$$
P(x) \geq \prod\limits_{p \leq x} (1+p^{-1} + \cdots + p^{-m}),
$$
where $m$ is any positive integer. The product on the right is $\sum
\dfrac{1}{n}$ summed over a certain set of positive integers $n$, and
if $m$ is so chosen that $2^{m+1} > x$, this set will certainly
include all integers from $1$ to $[x]$. Hence
\begin{equation*}
P(x) > \sum\limits^{[x]}_1 \frac{1}{n} > \int\limits^{[x]+1}_1
\frac{du}{u} > \log x. \tag{1}\label{c17:eq1}
\end{equation*}
Since\pageoriginale
$$
-\log (1-u) - u < \frac{\frac{1}{2}u^2}{1-u}, \text{ for } 0 < u < 1,
$$
we have
\begin{align*}
\log P(x) - S(x) & < \sum\limits_{p\leq x} -
\frac{p^{-2}}{2(1-p^{-1})} \\
& < \sum\limits^\infty_{n=2} \frac{1}{2n(n-1)} =\frac{1}{2}
\end{align*}
Hence, by (\ref{c17:eq1}),
\begin{equation*}
S(x) > \log \log x -\frac{1}{2}. \tag{2}\label{c17:eq2}
\end{equation*}
(\ref{c17:eq1}) and (\ref{c17:eq2}) prove the theorem.

\medskip
\noindent{\textbf{The Chebyschev Functions. }}
Let 
$$ 
\vartheta (x) = \sum\limits_{p\leq x} \log p, \quad \psi (x) =
\sum\limits_{p^m \leq x} \log p, \;\; x>0.
$$

Grouping together terms of $\psi(x)$ for which $m$ has the same value,
we get
\begin{equation*}
\psi (x) = \vartheta (x) + \vartheta (x^{1/2}) + \vartheta (x^{1/3}) +
\cdots, \tag{3}\label{c17:eq3}
\end{equation*}
this series having only a finite number of non-zero terms, since
$\vartheta (x) = 0$ if $x <2$.

Grouping together terms for which `$p$' has the same value $(\leq x)$,
we obtain
\begin{equation*}
\psi(x) = \sum\limits_{p\leq x } \left[\frac{\log x}{\log p} \right]
\log p,\tag{4} \label{c17:eq4}
\end{equation*}
since the number of values of $m$ associated with a given $p$ is equal
to the\pageoriginale number of positive integers $m$ satisfying $m\log
p \leq x$, and this is $\left[\dfrac{\log x}{\log p} \right]$ 
\end{proof}

\begin{thm}\label{chap17:thm2}
The functions
$$
\frac{\pi(x)}{x/\log x}, \quad \frac{\vartheta(x)}{x}, \quad
\frac{\psi(x)}{x} 
$$
have the same limits of indetermination as $x \to \infty$.
\end{thm}

\begin{proof}
Let the upper limits (may be $\infty$) be $L_1$, $L_2$, $L_3$ and the
lower limits be $l_1, l_2, l_3$, respectively. Then by (\ref{c17:eq3}) and (\ref{c17:eq4}), 
$$
\vartheta(x) \leq \psi(x) \leq \sum\limits_{p\leq x} \frac{\log
  x}{\log p} \log p = \pi (x) \log  x.
$$
Hence
\begin{equation*}
L_2 \leq L_3 \leq L_1. \tag{5}\label{c17:eq5}
\end{equation*}
On the other hand, if $0< \alpha < 1$, $x>1$, then 
$$
\vartheta (x) \geq \sum\limits_{x^{\alpha} < p \leq x} \log p \geq
\{\pi(x) - \pi (x^{\alpha})\} \log x^{\alpha}
$$
and $\pi (x^{\alpha}) < x^{\alpha}$, so that
$$
\frac{\vartheta (x)}{x} \geq  \alpha \left(\frac{\pi(x) \log x}{x} -
\frac{\log x}{x^{1-\alpha}}  \right) 
$$
Keep\pageoriginale $\alpha$ fixed, and let $x\to \infty$; then
$\dfrac{\log x}{x^{1-\alpha}} \to 0$

Hence
$$
L_2 \geq \alpha L_1, 
$$
i.e.
$$
L_2 \geq L_1, \text{ as } \alpha \to 1 - 0.
$$
This combined with (\ref{c17:eq5}) gives $L_2 = L_3 = L_1$, and similarly for the
$1$'s. 
\end{proof}

\begin{coro*}
If $\dfrac{\pi (x)}{x/\log x}$ or $\dfrac{\vartheta(x)}{x}$ or
$\dfrac{\psi(x)}{x}$ tends to a limit, then so do all, and the limits
are equal.
\end{coro*}

\begin{remarks*}
If $\wedge(n)= \begin{cases}
\log p, \text{ if $n$ is a $+$ve power of a prime $p$}\\ 
0, \quad  \text{otherwise},
\end{cases}
$

\noindent
then 
\begin{gather*}
\psi(x) = \sum\limits_{p^m\leq x} \log p = \sum\limits_{n \leq x}
\wedge(n),  \text{ and }\\
- \frac{\zeta'(s)}{\zeta(s)} = \sum\limits^\infty_1
\frac{\wedge(n)}{n^s}, \;\; \sigma >1 \text{ [p.69]}
\end{gather*}
Also
$$ 
-\frac{\zeta'(s)}{\zeta(s)} = s\int\limits^\infty_1
\frac{\psi(x)}{x^{s+1}}  dx, \;\; \sigma >1. 
$$
\end{remarks*}

\medskip
\noindent{\textbf{The Wiener-Ikehara theorem. }}\cite[4]{key3}
Let $A(u)$ be a non-negative non-decreasing function defined for $0
\leq u < \infty$. Let 
$$
f(s) \equiv \int\limits^\infty_0 A(u) e^{-us} du, \quad s = \sigma + i
\tau,
$$
converge\pageoriginale for $\sigma >1$, and be analytic for $\sigma
\geq 1$ except at $s=1$, where it has a simple pole with residue
1. Then $e^{-u} A(u) \to 1$ as $u\to \infty$

\begin{proof}
If $\sigma >1$,
$$
\frac{1}{s-1} = \int\limits^\infty_0 e^{-(s-1)u} du
$$
Therefore
\begin{align*}
g(s) & \equiv f(s) - \frac{1}{s-1} = \int\limits^\infty_0 \{A(u)
-e^u\} e^{-us} du\\
& \equiv \int\limits^\infty_0 \{B(u)-1\} e^{-(s-1)u}du,
\end{align*}
where 
$$
B(u) \equiv A (u) e^{-u}
$$
Take 
$$
s = 1+\epsilon + i \tau .
$$
Then
$$
g(1+ \epsilon + i \tau) \equiv g_\epsilon (\tau) =
\int\limits^\infty_0 \{B(u) -1\} e^{-\epsilon u} \cdot e^{-i \tau
  u} du
$$
Now $g(s)$ is analytic for $\sigma \geq 1$; therefore
$$
g_\epsilon (t) \to g(1+it) \text{ as } \epsilon \to 0.
$$
\textit{uniformly in any finite interval} $-2\lambda \leq t \leq 2
\lambda$. 

We should like to form the Fourier transform of $g_\epsilon
(\tau)$, but since we do not know that it is  bounded on the whole
line, we shall introduce a smoothing kernel. Thus 
\begin{multline*}
\frac{1}{2}\int\limits^{2\lambda}_{-2\lambda} g_\epsilon (\tau)
\left(1-\frac{|\tau|}{2\lambda} \right) e^{iy\tau} d\tau \\
= \frac{1}{2} \int\limits^{2\lambda}_{-2\lambda} e^{iy\tau}
\left(1-\frac{|\tau|}{2\lambda} \right)  d\tau \int\limits^\infty_0
\{B(u)-1\} e^{-\epsilon u - i \tau u} du.
\end{multline*}
For a\pageoriginale fixed $y$, a finite $\tau$-interval, and an
infinite $u$-interval we wish to change the order of integration. This
is permitted if 
$$
\int\limits^\infty_0 \{B(u)-1\} e^{-\epsilon u} \cdot e^{-i \tau u}
du 
$$
converges uniformly in $-2\lambda \leq \tau \leq 2\lambda$. This is
so, because it is equal to 
$$ 
\int\limits^\infty_0 B(u) e^{-\epsilon u} e^{-i \tau u} du -
\int\limits^\infty_0 e^{-\epsilon u} \; e^{-\epsilon \tau u} du.
$$
For a fixed $\epsilon >0$, the second converges absolutely and
uniformly in $\tau$. In the first we have
$$
B(u) \cdot e^{-(\epsilon /2)u} = A(u) \cdot e^{-(1+\epsilon/2)u}
\to 0 \text{ as } u \to \infty
$$
Hence $B(u) = O (e^{(\epsilon/2) u})$, which implies the first integral converges
uniformly and absolutely. 

Thus 
\begin{align*}
& \frac{1}{2} \int\limits^{2\lambda}_{-2\lambda} e^{iy \tau} \left(
1-\frac{|\tau|}{2\lambda}\right)  g_\epsilon (\tau) d \tau\\
& = \left[\int\limits^\infty_0 \{B(u)-1\} e^{-\epsilon u} du
  \int\limits^{2\lambda}_{-2\lambda} \frac{1}{2} e^{i(y-u)\tau}
  \left(1-\frac{|\tau|}{2\lambda} \right) d\tau \right]\\
& = \int\limits^\infty_0 [B(u) -1]e^{-\epsilon u} \frac{\sin^2
  \lambda(y-u)}{\lambda(y-u)^2}  du
\end{align*}
Now\pageoriginale consider the limit as $\epsilon \to 0$. The left
hand side tends to 
$$
\frac{1}{2} \int\limits^{2\lambda}_{-2\lambda} e^{iy\tau} \left(1 -
\frac{|\tau|}{2\lambda} \right)  g(1+i\tau) d \tau
$$
since $g$ is analytic, or rather, since $g_\epsilon (\tau) \to
g(1+i\tau)$ uniformly. On the right side, we have
$$
\int\limits^{\infty}_0 e^{-\epsilon u}
\frac{\sin^2\lambda(y-u)}{\lambda(y-u)^2} du \to
\int\limits^{\infty}_0 \frac{\sin^2 \lambda(y-u)}{\lambda(y-u)^2} du. 
$$
Hence
\begin{multline*}
\lim\limits_{\epsilon \to 0} \int\limits^\infty_0 e^{-\epsilon
  u} B(u) \frac{\sin^2 \lambda (y-u)}{\lambda(y-u)^2} du =\\
\int\limits^\infty_0 \frac{\sin^2 \lambda(y-u)}{\lambda(y-u)^2} du +
\int\limits^{2\lambda}_{-2\lambda} e^{iyt}
\left(1-\frac{|t|}{2\lambda} \right) g(1+it) \; dt
\end{multline*}
The integrand of the left hand side increases monotonically as
$\epsilon \to 0$; it is positive. Hence by the monotone-convergence
theorem,
\begin{multline*}
\int\limits^\infty_0 B(u) \frac{\sin^2 \lambda(y-u)}{\lambda(y-u)^2}
du = \int\limits^\infty_0 \frac{\sin^2 \lambda (y-u)}{\lambda(y-u)^2}
du \\
+ \int\limits^{2\lambda}_{-2\lambda} e^{iyt}
\left(1-\frac{|t|}{2\lambda} \right) g (1+it) \; dt
\end{multline*}
Now let $y \to \infty$; then the second term on the right-hand side
tends to zero by the Riemann-Lebesgue lemma, while the first term is
$$
\lim\limits_{y\to\infty} \int\limits^{\lambda y}_{-\infty}
\frac{\sin^2 \nu}{v^2} dv = \int\limits^\infty_{-\infty} \frac{\sin^2
  v}{v^2} dv = \pi 
$$
Hence\pageoriginale
$$ 
\lim\limits_{y \to \infty} \int\limits^{\lambda y}_{-\infty}
B \left(y-\frac{v}{\lambda} \right) \frac{\sin^2v}{v^2} dv = \pi \text{ for every }
\lambda. 
$$
To prove that \quad $\lim\limits_{u \to \infty} B(u) =1$.
\end{proof}

\medskip
\noindent{\textbf{Second Part.}}
\begin{itemize}
\item[(i)] $\overline{\lim} B(u) \leq 1$.

For a fixed `$a$' such that $0<a< \lambda y$, we have
$$
\overline{\lim\limits_{y \to \infty}}\int\limits^a_{-a} B\left(y -
\frac{v}{\lambda} \right) \frac{\sin^2 v}{v^2} dv \leq \pi
$$
By definition $B(u) = A(u)e^{-u}$, where $A(u) \uparrow$, so that
$$ 
e^{y -\frac{d}{\lambda}} B \left(y - \frac{a}{\lambda} \right) \leq e^{y -
  \frac{v}{\lambda}} B \left(y -\frac{v}{\lambda} \right)
$$
or
$$
B \left(y -\frac{v}{\lambda} \right) \geq e^{(v-a)/\lambda}
B \left(y-\frac{a}{\lambda} \right). 
$$
Hence
\begin{align*}
& \overline{\lim\limits_{y\to\infty}} \int\limits^a_{-a} B \left(y -
\frac{a}{\lambda} \right) e^{(v-a)/\lambda} \frac{\sin^2 v}{v^2} dv \leq \pi
\\
\text{or }\qquad & \int\limits^a_{-a} e^{\frac{-2a}{\lambda}}
\frac{\sin^2v}{v^2}\;  \overline{\lim} B \left(y -\frac{a}{\lambda} \right) dv \leq \pi \qquad \\
\text{or } \qquad & e^{-2a/\lambda} \int\limits^a_{-a} \frac{\sin^2
  v}{v^2} \;\; \overline{\lim\limits_{y \to \infty}} B(y) \cdot dv \leq \pi 
 \qquad 
\end{align*}\pageoriginale
Now let $a\to \infty$, $\lambda\to\infty$ in such a way that
$\dfrac{a}{\lambda} \to 0$. 

Then
$$
\pi \overline{\lim\limits_{y \to \infty}} B(y) \leq \pi
$$

\item[(ii)] $\lim\limits_{u\to\infty} B(u) \geq 1$.
\begin{equation*}
 \int\limits^{\lambda y}_{-\infty} B \left(y -\frac{v}{\lambda} \right)
\frac{\sin^2 v}{v^2} dv = \int\limits^{-a}_{-\infty} +
\int\limits^a_{-a} + \int\limits^{\lambda y}_a  \tag*{$(\circledast)$}
\end{equation*}
We know that $|B(u)| \leq c$, so that
$$
\int\limits^{\lambda y}_{-\infty} \leq c
\left[\int\limits^{-a}_{-\infty} \frac{\sin^2 v}{v^2} dv +
  \int\limits^\infty_a \frac{\sin^2 v}{v^2} dv \right] +
\int\limits^a_{-a} 
$$
\end{itemize}

We know that $|B(u)| \leq c$, so that
$$
\int\limits^{\lambda y}_{-\infty} \leq c
\left[\int\limits^{-a}_{-\infty} \frac{\sin^2v}{v^2} dv +
  \int\limits^\infty_a \frac{\sin^2 v}{v^2} dv\right]  +
\int\limits^a_{-a} 
$$
As before we get
$$
e^{y+a/\lambda} B \left(y + \frac{a}{\lambda} \right) \geq e^{y-v/\lambda} B \left(y
-\frac{v}{\lambda} \right) 
$$
or
$$
B \left(y-\frac{v}{\lambda} \right) \leq B \left(y + \frac{a}{\lambda} \right)
e^{\frac{a+v}{\lambda}} \leq B \left(y + \frac{a}{\lambda} \right) e^{2a/\lambda} 
$$
Therefore 
\begin{gather*}
\int\limits^a_{-a} B \left(y - \frac{v}{\lambda} \right) \frac{\sin^2 v}{v^2} dv
\leq \int\limits^a_{-a} B \left(y + \frac{a}{\lambda} \right) e^{2a/\lambda}
\frac{\sin^2 v}{v^2} dv\\
\leq e^{2a/\lambda} B \left(y + \frac{a}{\lambda} \right) \int\limits^a_{-a}
\frac{\sin^2 v}{v^2} dv
\end{gather*}\pageoriginale
Hence taking the $\underline{\lim}$ in ($\circledast$) we get, since
$$ 
\lim\limits_{y \to \infty} \int\limits^{\lambda y}_{-\infty} = \lim\limits_{y
  \to \infty} = \pi
$$
and
$$
\underline{\lim} (c+\psi(y)) \leq c + \underline{\lim} \psi(y), 
$$
that
\begin{align*}
\pi & \leq c \left(\int\limits^{-a}_{-\infty} +
\int\limits^{\infty}_{a} \right) \frac{\sin^2 v}{v^2} dv +
\underline{\lim} \int\limits^a_{-a}  \\
& \leq \{\cdots\} + \underline{\lim} B \left(y+\frac{a}{\lambda} \right)
e^{2a/\lambda} \int\limits^a_{-a} \frac{\sin^2v}{v^2} dv\\
& \leq \{\cdots\} + e^{2a/\lambda} \int\limits^a_{-a} \underline{\lim}
B(y) \frac{\sin^2 v}{v^2} dv
\end{align*}

Let $a \to \infty$, $\lambda \to \infty$ in such a way that
$\dfrac{a}{\lambda} \to 0$. 

Then 
$$
\pi \leq \pi \underline{\lim} B(y)
$$
Thus
$$
\lim\limits_{u\to\infty} B(u) = \lim\limits_{u\to \infty} A(u) e^{-u}
=1. 
$$

\medskip
\noindent{\textbf{Proof of the Prime Number Theorem.}}\pageoriginale

We have seen [p.143] that 
\begin{align*}
\frac{-\zeta'(s)}{s\zeta(s)} & = \int\limits^{\infty}_1
\frac{\psi(x)}{x^{s+1}} dx, \quad \sigma >1\\
& = \int\limits^{\infty}_0 e^{-st} \psi (e^t) dt, \quad \sigma > 1.
\end{align*}

Since $\zeta(s)$ is analytic for $\sigma \geq 1$ except for $s=1$,
where it has a simple pole with residue 1, and has no zeros for
$\sigma \geq 1$, and $\psi(e^t) \geq 0$ and non-decreasing, we can
appeal to the Wiener-Ikehara theorem with $f(s) = -
\dfrac{\zeta'(s)}{s\zeta(s)}$, and obtain
\begin{gather*} 
\psi(e^t) \sim e^t \text{ as } t \to \infty\\
\text{or } \qquad \qquad \psi (x) \sim x. \qquad \qquad 
\end{gather*}

\section[Prime Number Theorem and the zeros of  $\zeta(s)$]{Prime Number Theorem and the zeros of  {\boldmath$\zeta(s)$}}\label{chap17:sec13}

We\pageoriginale have seen that the Prime Number Theorem follows from the
Wiener-Ikehara Theorem {\bf if} we assume that $\zeta(1+it)\neq 0$. On the
other hand, if we assume the Prime Number Theorem, it is easy to
deduce that $\zeta(1+it) \neq 0$. If
$$
\psi(x) = \sum\limits_{p^m \leq x} \log p = \sum\limits_{n \leq x}
\wedge (n),
$$
then, for $\sigma >1$, we have
$$
\int\limits^\infty_1 \frac{\psi(x) -x}{x^{s+1}} dx = -
\frac{\zeta'(s)}{s\zeta(s)} -\frac{1}{1-s} \equiv \phi (s),  \text{
  say. }
$$
Now $\phi(s)$ is regular for $\sigma >0$ except (possibly) for simple
poles at the zeros of $\zeta(s)$. Now, if $\psi(x) = x + O(x)$, [which
is a consequence of the p.n. theorem], then, given $\epsilon >0$,
$$
|\psi (x) - x| < \epsilon x, \text{ for } x \geq x_0 (\epsilon)
> 1. 
$$
Hence, for $\sigma >1$,
$$
|\phi(s)| < \int\limits^{x_0}_1 \frac{|\psi(x)-x |}{x^2} dx+
\int\limits^\infty_{x_0 }  \frac{\epsilon}{x^\sigma} dx < K +
\frac{\epsilon}{\sigma-1} 
$$
where $K = K (x_0) = K(\epsilon)$. Thus
$$
|(\sigma -1) \phi (\sigma + i t)| < K(\sigma -1) + \epsilon < 2
\epsilon, 
$$
for $1<\sigma < \sigma_0 (\epsilon , K) =
\sigma_0(\epsilon)$. Hence, for any fixed $t$, 
$$
(\sigma -1) \phi (\sigma + it) \to 0 \text{ as } \sigma \to 1 + 0. 
$$
This shows that $1+it$ cannot be a zero of $\zeta(s)$, for in that
case
$$
(\sigma -1) \phi (\sigma + it)
$$
would tend to a limit different from zero, namely the residue of
$\phi(s)$ at the simple pole $1+it$.

\section[Prime Number Theorem and the magnitude of
  $p_n$]{Prime Number Theorem and the magnitude of
  {\boldmath$p_n$}}\label{chap17:sec14}

It is\pageoriginale easy to see that the p.n. theorem is equivalent to the result:
$p_n \sim n \log n$. For if 
\begin{equation*}
 \frac{\tau(x) \log x}{x}  \to 1\tag{1}\label{c17:eqadd1}
\end{equation*}
then 
$$
\log \pi (x) + \log \log x - \log x \to 0,
$$
hence
\begin{equation*}
 \frac{\log \pi (x)}{\log x} \to 1\tag{2}\label{c17:eqadd2}
\end{equation*}
Now (\ref{c17:eqadd1}) and (\ref{c17:eqadd2}) give 
$$
\frac{\pi (x) \cdot \log \pi (x)}{x} \to 1
$$
Taking $x =p_n$, we get $p_n \sim n \log n $, since $\pi(p_n) = n$. 

Conversely, if $n$ is defined by : $p_n \leq x < p_{n+1}$, then $p_n
\sim n \log n $ implies
$$
p_{n+1} \sim (n+1) \log (n+1) \sim n \log n, 
$$
as $x \to \infty$. Hence
\begin{align*}
& x \sim n \log n \\
\text{or} \qquad \qquad  & x \sim y \log y, \quad y = \pi (x) =
n. \qquad \qquad  
\end{align*}
Hence
\begin{gather*}
\log x \sim \log y, \text{ as above,  so that }\\
y  \sim x / \log x
\end{gather*}

\begin{thebibliography}{99}
\bibitem{key1} L.Bieberbach:\pageoriginale Lehrbuch der Funktionentheorie I \& II,
  Chelsea, 1945

\bibitem{key2} S.Bochner: Vorlesungen \"uber Fouriersche Integrale,
  Leipzig, 1932

\bibitem{key3} S.Bochner: Notes on Fourier Analysis, Princeton, 1936-37

\bibitem{key4} S.Bochner: Math. Zeit. 37(1933), pp.1-9

\bibitem{key5} S.Bochner \& K.Chandrasekharan: Fourier Transforms,
  Princeton, 1949

\bibitem{key6} A.P.Calderon \& A.Zygmund: Contributions to Fourier
  Analysis, Princeton, 1950, pp. 166-188

\bibitem{key7} C.Caratheodory: Funktionentheorie, I \& II, Basel, 1950 

\bibitem{key8} K.Chandrasekharan: J.Indian Math. Soc. 5(1941),
  pp. 128-132 

\bibitem{key9} K.Chandrasekharan \& S.Minakshisundaram: Typical Means,
  Bombay, 1952

\bibitem{key10} G.H.Hardy \& M.Riesz: The General Theory of Dirichlet
  Series, Cambridge, 1915

\bibitem{key11} A.E.Ingham: The Distribution of Prime Numbers,
  Cambridge, 1932

\bibitem{key12} J.E.Littlewood: The Theory of Functions, Oxford, 1945

\bibitem{key13} C.L.Siegel: Math. Annalen, 86 (1922), pp.276-9

\bibitem{key14} C.L.Siegel: Lectures on Analytic Number Theory, New
  York, 1945

\bibitem{key15} E.C.Titchmarsh: The Theory of Functions, Oxford, 1932

\bibitem{key16} E.C.Titchmarsh: The Theory of the Riemann
  Zeta-Funcation, Oxford, 1951

\bibitem{key17} W.Thron: Introduction to   the Theory of Functions of a
  Complex Variable, Wiley, 1953
\end{thebibliography}
