\chapter{The Zeta Function of Riemann (Contd.)}\label{chap9}

\setcounter{section}{1}
\section[Elementary theory of Dirichlet Series]{Elementary theory of Dirichlet Series
  \cite{key10}}\label{chap9:sec2}\pageoriginale

The Zeta function of Riemann is the sum-function associated with the
Dirichlet series $\sum\frac{1}{n^s}$. We shall now study, very
briefly, some of the elementary properties of general Dirichlet series
of the form
$$ 
\sum\limits^\infty_{n=1} a_n e^{-\lambda_{n} s}, 0 \leq \lambda_1 <
\lambda_2 < \cdots  \to \infty
$$
Special cases are: $\lambda_n = \log n $; $\lambda_n = n$, $e^{-s} =
x$. We shall first prove a lemma applicable to the summation of
Dirichlet series.

\begin{lemma*}
Let $A(x) = \sum\limits_{\lambda_n \leq x} a_n$, where $a_n$ may be
real or complex. Then, if $x \geq \lambda_1$, and $\phi(x)$ has a
continuous derivative in $(0,\infty)$, we have
$$
\sum\limits_{\lambda_n \leq \omega} a_n \phi (\lambda_n) = -
\int\limits^\omega_{\lambda_1} A(x) \phi'(x) dx + A(\omega) \phi (\omega).
$$
If $A(\omega) \phi (\omega) \to 0$ as $\omega \to \infty$, then 
$$
\sum\limits^\infty_{n=1} a_n \phi (\lambda_n) = -
\int\limits^\infty_{\lambda_1} A(x) \phi'(x) dx,
$$
provided that either side is convergent.
\end{lemma*}

\begin{proof}
\begin{align*}
A(\omega)\phi(\omega) & - \sum\limits_{\lambda_n \leq \omega} a_n \phi
(\lambda_n)\\
& = \sum\limits_{\lambda_n \leq \omega} a_n \{\phi(\omega) - \phi
(\lambda_n)\} \\
& = \sum\limits_{\lambda_n \leq \omega} \int\limits^\omega_{\lambda_n}
a_n \phi'(x) dx \\
& = \int\limits^\omega_{\lambda_1} A(x) \phi'(x) dx 
\end{align*}\pageoriginale
\end{proof}

\setcounter{thm}{0}
\begin{thm}\label{chap9:thm1}
If $\sum\limits^\infty_{n=1} a_n e^{-\lambda_n s}$ converges for $s =
s_0 \equiv \sigma_0 + it_0$, then it converges uniformly in the
angular region defined by 
$$
|am(s-s_0)| \leq \theta < \pi/2, \quad \text{for} \quad 0 < \theta <
\pi/2  
$$
\end{thm}

\begin{proof}
We may suppose that $s_0=0$. For 
$$
\sum a_n e^{-\lambda_ns} = \sum a_n e^{-\lambda_n s_0}
e^{-\lambda_{n}(s-s_0)} \equiv \sum b_n e^{-\lambda_n s'}, \text{ say,}
$$
and the new series converges at $s'=0$. By the above lemma, we
get
\begin{align*}
\sum\limits^\nu_{\mu+1} a_n e^{-\lambda_n s} & =
\left(\sum\limits^{\nu}_1 - \sum\limits^\mu_1 \right) a_n e^{-\lambda
  n^s}\\
& = \int\limits^{\lambda_\nu}_{\lambda_{\mu}} A(x) \cdot e^{-xs} \cdot
s dx + A(\lambda_\nu) e^{-\lambda_\nu s} - A (\lambda_\mu)
e^{-\lambda_\mu s}\\
& = s \int\limits^{\lambda_\nu}_{\lambda_\mu} \{A(x) -
A(\lambda_\mu)\} e^{-xs} dx + [A(\lambda_\nu) - A (\lambda_\mu)]
e^{-\lambda_\nu s}
\end{align*}

We have assumed that $\sum a_n$ converges, therefore, given $\epsilon > 0$,
there\pageoriginale exists an $n_0$ such that for $x > \lambda_\mu
\geq n_0$, we have
$$
|A(x) - A (\lambda_\mu)| < \epsilon
$$
Hence, for such $\mu$, we have
\begin{align*}
\left|\sum\limits^\nu_{\mu+1} a_n e^{-\lambda_n s} \right| & \leq
\epsilon |s| \int\limits^{\lambda_\nu}_{\lambda_\mu} e^{-x\sigma}
dx + \epsilon e^{-\lambda_\nu \sigma}\\
& \leq 2 \epsilon \frac{|s|}{\sigma} + \epsilon\\
& < 2 \epsilon \; (\sec \theta + 1),
\end{align*}
\textit{if $\sigma \neq 0$}, since $\dfrac{|s|}{\sigma} < \sec
\theta$, and this proves the theorem. As a consequence of Theorem \ref{chap9:thm1},
we deduce
\end{proof}

\begin{thm}\label{chap9:thm2}
If $\sum a_n e^{-\lambda_ns}$ converges for $s = s_0$, then it
converges for $\re s > \sigma_0$, and uniformly in any bounded closed
domain contained in the half-plane $\sigma > \sigma_0$. We also have 
\end{thm}

\begin{thm}\label{chap9:thm3}
If $\sum a_n e^{-\lambda_ns}$ converges for $s = s_0$ to the sum
$f(s_0)$, then $f(s) \to f(s_0)$ as $s \to s_0$ along any path in the
region $|am (s-s_0)| \leq \theta < \pi/2$.
\end{thm}

A Dirichlet series may converge for \textit{all} values of $s$, or
\textit{some} values of $s$, or \textit{no} values of $s$.
\begin{itemize}
\item[{\bf{Ex.1.}}] $\sum a_n n^{-s}$, $a_n = \dfrac{1}{n!}$
converges for \textit{all} values of $s$.

\item[{\bf{Ex.2.}}] $\sum a_n n^{-s}$, $a_n = n!$ converges for
  \textit{no} values of $s$.

\item[{\bf{Ex.3.}}] $\sum n^{-s}$ converges for $\re s >1$, and
  diverges for $\re s \leq 1$.
\end{itemize}

If a\pageoriginale Dirichlet series converges for some, but not all,
values of $s$, and if $s_1$ is a point of convergence while $s_2$ is a
point of divergence, then, on account of the above theorems, we have
$\sigma_1 > \sigma_2$. All points on the real axis can then be divided
into two classes $\mathsf{L}$ and $\mathsf{U}$; $\sigma \in
\mathsf{U}$ if the series converges at $\sigma$, otherwise $\sigma
\in \mathsf{L}$. Then any point of $\mathsf{U}$ lies to the
right of any point of $\mathsf{L}$, and this classification defines a
number $\sigma_0$ such that the series converges for every $\sigma >
\sigma_0$ and diverges for $\sigma < \sigma_0$, the cases $\sigma =
\sigma_0$ being undecided. Thus the region of convergence is a
half-plane. The line $\sigma=\sigma_0$ is called the \textit{line of
  convergence}, and the number $\sigma_0$ is called the
\textit{abscissa of convergence}. We have seen that $\sigma_0$ may be
$\pm \infty$. On the basis of Theorem \ref{chap9:thm2} we can establish

\begin{thm}\label{chap9:thm4}
A Dirichlet series represents in its half-plane of convergence a
regular function of $s$ whose successive derivatives are obtained by
term wise differentiation.
\end{thm}

We shall now prove a theorem which implies the `\textit{uniqueness}'
of Dirichlet series.

\begin{thm}\label{chap9:thm5}
If $\sum a_n e^{-\lambda_ns}$ converges for $s=0$, and its sum
$f(s)=0$ for an infinity of values of $s$ lying in the region:
$$
\sigma \geq \epsilon >0, \quad |am \; s| \leq \theta < \pi/2, 
$$
then $a_n = 0$ for all $n$.
\end{thm}

\begin{proof}
$f(s)$ cannot have an infinite number of zeros in the neighbourhood of
  any \textit{finite} point of the given region, since it is regular
  there. Hence there exists an infinity of values 
$s_n = \sigma_n + it_n$, say, with $\sigma_{n+1} > \sigma_n$, $\lim
 \sigma_n = \infty$ such that $f(s_n) = 0$.

However,\pageoriginale
$$
g(s) \equiv e^{\lambda_1s} f(s) = a_1 + \sum\limits^\infty_2 a_n
e^{-(\lambda_n - \lambda_1) s},
$$
(here we are assuming $\lambda_1 > 0$) converges for $s=0$, and is
therefore uniformly convergent in the region given; each term of the
series on the right $\to 0$, as $s \to \infty$ and hence the right
hand side, as a whole, tends to $a_1$ as $s \to \infty$. Thus $g(s)
\to a_1$ as $s \to \infty$ along any path in the given region. This
would contradict the fact that $g(s_n)=0$ for an infinity of $s_n \to
\infty$, unless $a_1=0$. Similarly $a_2 = 0$, and so on.
\end{proof}

\begin{remark*}
In the hypothesis it is essential that $\epsilon >0$, for if
$\epsilon =0$, the origin itself may be a limit point of the zeros
of $f$, and a contradiction does not result in the manner in which it
is derived in the above proof.
\end{remark*}

The above arguments can be applied to $\sum a_n  e^{-\lambda_n s}$ so
as to yield the existence of an \textit{abscissa of absolute
  convergence } $\overline{\sigma}$, etc. In general, $\overline{\sigma} \geq
\sigma_0$. The strip that separates $\overline{\sigma}$ from $\sigma_0$ may
comprise the whole plane, or may be vacuous or may be a half-plane.
\begin{itemize}
\item[{\bf{Ex.1.}}] $\sigma_0 = 0$, $\overline{\sigma} = 1$
\begin{align*}
(1-2^{1-s}) \zeta(s) & = \left(\frac{1}{1^s} + \frac{1}{2^s} +
  \frac{1}{3^s} + \cdots  \right)  - 2 \left(\frac{1}{2^s} +
  \frac{1}{4^s} + \cdots \right)\\
& = \left(\frac{1}{1^s} - \frac{1}{2^s} + \frac{1}{3^s} -
  \frac{1}{4^s} + \cdots \right)
\end{align*}

\item[{\bf{Ex.2.}}] $\sum \dfrac{(-1)^n}{\sqrt{n}} (\log
  n)^{-s}$\pageoriginale converges for all $s$ but never
  absolutely. In the simple case $\lambda_n = \log n$, we have
\end{itemize}


\begin{thm}\label{chap9:thm6}
$\overline{\sigma} - \sigma_0 \leq 1$.

For if $\sum a_n n^{-s} < \infty$, then $|a_n|\cdot n^{-\sigma} = O
(1)$, so that $\sum \dfrac{|a_n|}{n^{s+1+\epsilon}} < \infty$ for
$\epsilon >0$

While we have observed that the sum-function of a Dirichlet series is
regular in the half-plane of convergence, there is no reason to assume
that the line of convergence contains at least one singularity of the
function (see Ex 1. above!). In the special case where the
coefficients are positive, however we can assert the following 
\end{thm}


\begin{thm}\label{chap9:thm7}
If $a_n \geq 0$, then the point $s = \sigma_0$ is a singularity of the
function $f(s)$.
\end{thm}

\begin{proof}
Since $a_n \geq 0$, we have $\sigma_0 = \overline{\sigma}$, and we may
assume, without loss of generality, that $\sigma_0=0$. Then, if $s=0$
is a regular point, the Taylor series for $f(s)$ at $s =1$ will have a
radius of convergence $>1$. (since the circle of convergence of a
power series must have at least one singularity). Hence we can find a
negative value of $s$ for which
$$
f(s) = \sum\limits^\infty_{\nu=0} \frac{(s-1)^\nu}{\nu!} f^{(\nu)} (1)
= \sum\limits^\infty_{\nu=0} \frac{(1-s)^{\nu}}{\nu !}
\sum\limits^{\infty}_{n=1} a_n \lambda^{\nu}_n e^{-\lambda_n} 
$$
Here every term is positive, so the summations can be interchanged and 
\pageoriginale we get
$$
f(s) = \sum\limits^\infty_{n=1} a_n e^{-\lambda_n}
\sum\limits^\infty_{\nu=0} \frac{(1-s)^\nu \lambda^\nu_n}{\nu!} =
\sum\limits^\infty_{n=1} a_n e^{-\lambda_ns}
$$
Hence the series converges for a negative value of $s$ which is a
contradiction. 
\end{proof}
