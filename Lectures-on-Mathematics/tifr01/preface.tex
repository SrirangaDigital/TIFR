\thispagestyle{empty}

\begin{center}
{\Large\bf Lectures on The}\\[5pt]
{\Large\bf Riemann Zeta--Function}\\
\vskip 1cm

{\bf By}
\medskip

{\large\bf K. Chandrasekharan}
\vfill

{\bf Tata Institute of Fundamental Research, Bombay}
\medskip

{\bf 1953}
\end{center}

\eject

\thispagestyle{empty}
\begin{center}
{\Large\bf Lectures on the Riemann Zeta-Function}
\vskip 1cm

{\bf By}
\medskip

{\large\bf K. Chandrasekharan}
\vfill

{\bf Tata Institute of Fundamental Research}
\medskip

{\bf 1953}
\end{center}
\eject

\thispagestyle{empty}


{\narrower\narrower
The aim of these lectures is to provide an 
intorduction to the theory of the Riemann
Zeta-function for students who might later
want to do research on the subject. The 
Prime  Number Theorem, Hardy's theorem on the
Zeros of $\zeta(s)$, and Hamburger's theorem are 
the principal results proved here. The 
exposition is self-contained, and required
a preliminary knowledge of only the elements
of function theory.\par}


