
\chapter{Vector Bundles and Derivation Laws}\label{chap5}

\section{}\label{chap5:sec1}%Sec 5.1

Let\pageoriginale $P$ be a differentiable principal bundle over a manifold $V$ with
group $G$. Consider a vector space $L$ of finite dimension over the
field $R$ of real numbers. Let $s \to s_L$ be a linear representation
of $G$ in $L.G$ acts differentiably on $L$ (regarded as a manifold )
to the right by the rule $vs = s^{-1}_L v$. 

Let $E$ be a fibre bundle associated to $P$ with fibre $L$ and $q$ the
map $ P \times L \to E$. For every $\xi \in P$, the map $v \to q(\xi,
v)$ is a bijection of $L$ onto the fibre of $E$ at $p \xi$, or a frame
of $L$. 
On each fibre $E_x$ at $x \in V$ of the associated bundle $E$, we may
introduce the structure of a vector space by requiring that the frame
defined by any point $\xi \in p^{-1}(x) $ be linear. It is clear that
such a structure does exist and is unique. Let $U$ be an open subset
of $V$ over which $P$ is trivial and $\rho$ a diffeomorphism $U \times
L \to p^{-1}(U)$ defined by $\rho(x, v) = q(\sigma(x) ,v)$ where
$\sigma$ is a differentiable cross section of $P$ over $U$. Then we
have  
\begin{enumerate}[1)]
\item $p ~\rho (x,y) =x$
\item $\rho (x,v+v') = \rho (x,v) + \rho (x,v')$
\item $\rho (x , \lambda v) = \lambda \rho (x,v)$
\end{enumerate} 
for every $x \in U, v, v' \in L$ and $\lambda \in R$.

Conversely, let $E$ be a differentiable manifold and $p: E \to V$a
differentiable\pageoriginale map. Assume each $p^{-1}(x) = E_x$ to be a vector space
over $R$. The manifold $E$ (together with $p$) is called a \textit{
  differentiable vector bundle} (or simply a vector bundle ) over $V$
if the following  condition is satisfied. For every $x_0 \in V$, there
exist an open neighbourhood $V$ of $x_0$ and a differentiable
isomorphism $\rho$ of $U \times R^n$ onto $p^{-1} (U)$ such that  
\begin{enumerate}[1)]
\item $p \rho (x,v) =x$
\item $\rho (x,v + v') = \rho (x ,v) + \rho (x,v')$
\item $\rho (x, \lambda v) = \lambda \rho (x,v)$
\end{enumerate}
for every $x \in V, v, v' \in L$ and $\lambda \in R$.

\setcounter{proposition}{0}
\begin{proposition}\label{chap5:sec1:prop1}%Prop 1
  Every vector bundle of dimension $n$ over $V$ is associated to a
  principal bundle over $V$ with group $GL(n, R)$. 
\end{proposition}

In fact, let $E$ be a vector bundle of dimension $n$ over $V$ and $q :
E \to V$ the projection of $E$. For every $x \in V$, define $P_x$ to
be the set of all linear isomorphisms of $R^n$ onto $E_x$ and $P =
\bigcup \limits_{x \in V} P_x$. The group $G = GL(n, R)$ acts on $P$
by the rule $(\xi s) v = \xi(sv)$ for  every $v \in R^n$. Define $p :
P \to V$ by $p(P_x)=x$. By definition of vector bundles, there exist
an open covering $U_{\alpha} \subset V$ and a family of
diffeomorphisms $\rho _{\alpha} : U_{\alpha} \times R^n \to q^{-1}
(U_{\alpha})$ satisfying conditions $1), 2)$ and $3)$. For every
$\alpha$, the map $\gamma_{\alpha}: U_{\alpha} \times G \to p^{-1}
(U_{\alpha})$ defined by $\gamma _{\alpha}(x, s) v = \rho _{\alpha}
(x, sv)$ for $x \in U_{\alpha}, \,s \in G$ is bijective. 
We put on $P$ a differentiable structure by requiring that all
$\gamma_{\alpha}$ be diffeomorphisms. It is easy to see that on the
overlaps the differentiable structures agree. Moreover, 
\begin{align*}
  \gamma_{\alpha}(x,st)v & = \rho_{\alpha} (x, (st) v)\\
  &=\rho_{\alpha} (x, s (tv) )\\  
  & = \gamma_{\alpha}(x,s) (tv)
\end{align*}
for\pageoriginale $x \in U_{\alpha}, s, t \in G$ and $v \in R^n$. Thus $P$ is a
principal bundle over $V$ with group $G$. Let $q'$ be the map of $P
\times R^n$ onto $E$ defined by $q'(\xi, v) = \xi v$ for $\xi \in P, v
\in R^n$. It is easy to see that $P \times R^n$ is a principal bundle
over $E$ with projection $q'$. Hence $E$ is an associated bundle of
$P$. 

\section{Homomorphisms of vector bundles}\label{chap5:sec2}%Sec 5.2

\setcounter{defn}{0}
\begin{defn}\label{chap5:sec2:def1}%%Def1
  Let $V$ and $V'$ be two differentiable manifolds, $E$ a vector
  bundle  over $V$ and $E'$ a vector bundle over $V'$. A \textit{
    homomorphism } $h$ of $E$ into $E'$ is a differentiable map $h : E
  \to E'$ such that such that, for every $x \in V, p' h (E_x)$ reduces
  to a point $x' \in V'$ and the restriction $h_x$ of $h$ to $E_x$ is a
  linear map of $E_x$ into $E'_x$. 
\end{defn}

The map $\underbar{h}: V \to V'$ defined by the condition $p'h =
\underbar{h} p $ is differentiable and is called the \textit{
  projection } of $h$. If $V =V'$, by a homomorphism of $E$ into $E'$,
we shall mean hereafter a homomorphism having the identity map $V \to
V$ as projection. In that case, $h :E \to E'$ is \textit{ injective }
or \textit{ surjective} according as all the maps $h_x : E_x \to E'_x$
are injective or surjective.    

A vector bundle of dimension $n$ over $V$ is said to be \textit{
  trivial } if it is isomorphic to the bundle $V \times R^n$ (with the
natural projection $V \times R^n \to V)$. Let $P$ be a principal
bundle over $V$ to which the vector\pageoriginale bundle $E$ is associated. If $P$
is trivial, so is $E$ and every section of $P$ defines an isomorphism
of $V \times L$ onto $E$. 

Let $E, E' , E''$ be vector bundles over $V$. Then a sequence of homomorphisms 
$$
E' \xrightarrow{h} E  \xrightarrow{k} E''
$$
is said to be \textit{ exact } if, for every $x \in V$, the sequence  
$$
E'_x \xrightarrow{h_x} E_x \xrightarrow{k_x} E''_x
$$
is exact 

\section{Induced vector bundles}\label{chap5:sec3}%Sec 5.3

Let $E$ be a vector bundle over a manifold $V$ and $q$ a
differentiable map of a manifold $Y$ into $V$. Let $E_q$ be the subset
of the product $Y \times E$ consisting of elements $(y, \eta)$ such
that $q(y) = p(\eta)$. 
Define $p :E_q \to Y$ by setting $p'(y, \eta) =y$. If $p' (y, \eta) =
p'(y', \eta')$, then $y = y'$ and $p'^{-1} (y) (y \in Y)$ by setting 
\begin{align*}
  & (y,\eta ) + (y, \eta') = (y, \eta + \eta')\\
  & \lambda (y, \eta) = (y , \lambda \eta)
\end{align*}
for $\eta, \eta' \in E$ and $\lambda \in R$. It is clear that this
makes of $E_q$ a differentiable vector bundle over $Y$. This is
called the \textit{ bundle } over $Y$ \textit{induced} by $q$. Let $P$
be a principal bundle over $X$ and $E$ a vector bundle associated to
$P$. If $P_q$ is the principal bundle over $Y$ induced\pageoriginale by
$q$ (Ch. \ref{chap2:sec4}) then $E_q$ is associated to $P_q$. 

\section{Locally free sheaves and vector bundles}\label{chap5:sec4}%Sec 5.4

Let $M$ be a differentiable manifold and $\varepsilon$ a sheaf over $M$. 
If $V \subset U$ are two open sets in $M, \varphi_{VU}$ denotes the
restriction map $\varepsilon (U) \to \varepsilon(V)$ and for every $x
\in U, \varphi_{x U}$ denotes the canonical map of $\varepsilon(U)$
into the stalk $\varepsilon_x$ at $x$. 

Let $\mathscr{U}$ be the sheaf of differentiable real valued functions
on $M$. For every open set $U \subset M, \varepsilon (U)$ is the
algebra of real valued differentiable functions on $U$. A sheaf
$\varepsilon$ over $M$ is called a \textit{ sheaf } of
$\mathscr{U}$-\textit{ modules} if, for every open set $U \subset M,
\mathscr{U} (U)$ is an $\mathscr{U}(U)$-module and if the restriction
maps satisfy the condition: 
$$
\varphi_{Vu}(f \sigma) = (\varphi_{VU}f) (\varphi_{VU} \sigma)
$$
whenever $V \subset U$ are open sets in $M, f \in \mathscr{U} (U)$ and
$\sigma \in \varepsilon (U)$., 

Let $\varepsilon$ and $\varepsilon'$ be two sheaves of
$\mathscr{U}$-modules over $M.A$ homomorphism $h$ of $\varepsilon$
into $\varepsilon'$ is a family of maps $h_U:
Hom_{\mathscr{U}(U)}(\varepsilon(U) , \varepsilon'(U)) (U$ open subset
of $M$) such that if $V \subset U$ are two open sets in $M, \varphi
_{VU} h_U = h_V  \varphi_{VU}$. 

Let $E$ be a differentiable vector bundle over $M$. The sheaf of
differentiable sections of $E$ is a sheaf $\varepsilon$ of
$\mathscr{U}$-modules over $M$; for every open set $U \subset M,
\varepsilon (U)$ is the $\mathscr{U}(U)$- module of differentiable
sections of $E$ over $U$. 

\begin{defn}\label{chap5:sec4:def2}%Def 2
  A sheaf $\varepsilon$ of $\mathscr{U}$-modules over $M$ is said to
  be \textit{ free } of rank $n$ if $\varepsilon$ is isomorphic to the
  sheaf $\mathscr{U}^n$ of differentiable maps $M \to R^n$.\pageoriginale A sheaf
  $\varepsilon$ of $\mathscr{U}$-modules over $M$ is said to be
  \textit{ locally free } of rank $n$ if every point $x \in M$ has a
  neighbourhood $U$ such that $\varepsilon$ restricted to $U$ is a
  free sheaf of rank $n$. 
\end{defn}

For every vector bundle $E$ over $M$, the sheaf $\varepsilon$ of
differentiable sections of $E$ is locally free of rank= dimension of
$E$. 

\begin{proposition}\label{chap5:sec4:prop2}%Prop 2
  A sheaf $\varepsilon$ of $\mathscr{U}$-modules over $M$ is locally
  free of rank $n$ if and only if for every $x \in M$, there exists a
  neighbourhood $U$ of $x$ and elements $\sigma_1, \sigma_2, \ldots
  \sigma_n \in \varepsilon (U)$ such that for every open set $V
  \subset U$, $(\varphi_{VU} \sigma_1, \varphi_{VU} \sigma_2, \ldots
  \varphi_{VU} \sigma_n)$ is 
  a base of $\varepsilon(V)$ over $\mathscr{U}(V)$. 
\end{proposition}

That a locally free sheaf has such a property is an immediate
consequence of the definition. Conversely, if such elements $\sigma_1,
\sigma_2, \ldots \sigma_n$ over $U$ exist, then, for every open set $V
\subset U$,the map $h_V$ of $\mathscr{U}(V)^n$ into $\varepsilon (V)$
defined by $h_V(f_1, f_2, \ldots f_n)= f_1 \sigma_1 + f_2 \sigma_2 +
\ldots f_n \sigma_n$ is bijective and the maps $h_V$ define an
isomorphism of the sheaf $\mathscr{U}^n$ of differentiable maps $U \to
R^n$ onto the sheaf $\varepsilon$ restricted to $U$. 

Let $E$ and $E'$ be two vector bundles over $M$. To every homomorphism
$h: E \to E'$ there corresponds in an obvious way a homomorphism $\tau
h : \varepsilon \to \varepsilon'$ and the assignment $E \to
\varepsilon$ of locally free sheaves to vector bundles is a functor
$T$ from the category of vector bundles over $M$ into the category of
locally free sheaves over $M$. 

Moreover, if 
$$
0 \to  E' \xrightarrow{h} E \xrightarrow{k} E'' \to 0
$$
is\pageoriginale an exact sequence, then the sequence 
$$
0 \to TE' \xrightarrow{ \tau h} Te \xrightarrow{ \tau k} TE'' \to 0
$$
is also exact (in the first sequence, 0 denotes the vector bundle of
dimension 0 over $M$). 

We shall now define a functor from the category of locally free
sheaves into the category of vector bundles over $M$. Let
$\varepsilon$ be a locally free sheaf over $M$. Let $\mathscr{U}_x$ be
the stalk at $x \in M$ for the sheaf $\mathscr{U}$ and $m_x$ the ideal
of germs of $f \in \mathscr{U}_x$ such that $f(x) =0$. Then we have
the exact sequence  
$$
0 \to m_x \to \varepsilon_x \to R \to 0.
$$
If $\varepsilon_x$ is the stalk at $x$ for the sheaf $\varepsilon$,
then we have correspondingly the exact sequence  
$$
0 \to m_x \varepsilon_x \to \varepsilon_x \to \varepsilon_x / m_x
\varepsilon_x \to 0. 
$$

Let $E_x = \varepsilon_x / m_x \varepsilon_x, E = \bigcup \limits_{x
  \in M} E_x$ and define $p : E \to M$ by the condition $p(E_x) =
(x)$. Let $U$ be an open subset of $M$ and $\sigma_1, \sigma_2, \ldots
\sigma_n \in \varepsilon (U)$ satisfying the condition of
prop.\ref{chap5:sec4:prop2}. Then, for every $x \in U, \varphi_{xU} \sigma_1,\break
\varphi_{xU} \sigma_2$, $\ldots \varphi_{xU} \sigma_n$ is a basis of
$\varepsilon_x$ over $\mathscr{U}_x$. Let $\sigma^*_i (x)$ be the
image of $\varphi_{xU} \sigma_i$ in the quotient space $E_x$. 
Then $\sigma^*_1 (x), \sigma^*_2 (x), \ldots \sigma^*_n (x)$ also is a
base of $E_X$ over $R$. The map $\rho_U : U \times R^n \to p^{-1}(U)
\subset E$ defined by $\rho(x, a_1, a_2, \ldots a_n)= \sum a_i
\sigma^*_i (x)$ is obviously a bijection. There exists on $E$ one and
only\pageoriginale one differentiable structure such that every map $\rho_U$ is a
diffeomorphism. With this structure, $E$ is a differentiable vector
bundle $T^* \varepsilon$over $M$. Let $\varepsilon, \varepsilon'$ be
two locally free sheaves of $\mathscr{U}-$ modules over $M$. To every
homomorphism $h : \varepsilon \to \varepsilon'$ corresponds in an
obvious way a homomorphism $T^* h :T^*\varepsilon \to T^* \varepsilon'$
and the assignment $\varepsilon \to T^*\varepsilon$ is a functor $T^*$
from the category of locally free sheaves to the category of vector
bundles over $M$. Moreover, if 
$$
0 \to \varepsilon' \to \varepsilon \to \varepsilon'' \to 0
$$
is an exact sequence of locally free sheaves, then  
$$
0 \to T^* \varepsilon'\to T^*\varepsilon \to T^*\varepsilon'' \to 0
$$ 
is an exact sequence of vector bundles.

\begin{proposition}\label{chap5:sec4:prop3}%Prop 3
  For  every vector bundle $E$ over $M$, there exists a canonical
  isomorphism of $E$ onto $T^*T E$. For every locally free sheaf of
  $\mathscr{U}-$ modules $\varepsilon$ over $M$, there exists a
  canonical isomorphism of $\varepsilon$ onto $TT^* \varepsilon$. (We
  shall later identify $E$ with $T^*TE$ and $E$ with $TT^*E$ means of
  these isomorphisms.) 
\end{proposition}

For instance, if $\varepsilon = TE$ and $E' =T^* \varepsilon$, the
isomorphism $E \to E'$ maps $u \in E_x$ into $\sigma^* (x)$ where
$\sigma$ is a differentiable section over an open neighbourhood of $x$
such that $\sigma(x) = u$. In particular $T$ defines a bijection of
the set of classes of isomorphic vector bundles over $M$ onto the set
of classes of isomorphic locally free sheaves of $\mathscr{U}-$modules
over $M$. 

Many\pageoriginale vector bundles are defined in differential geometry from a
locally free sheaf, using the functor $T^*$. For instance the \textit{
  tangent bundle } for a manifold $M$ corresponds the sheaf
$\mathscr{C}$ over $M$ such that $\mathscr{C}(U)$ is the module of
differentiable vector fields over the open set $U \subset M$, 

\section{Sheaf of invariant vector fields}\label{chap5:sec5}%Sec
   
Let $P$ be a differentiable principal bundle over a manifold $V$ with
group $G$. Let $\mathcal{J}$ be the  sheaf over $V$ such that for
every open set $U \subset V, \mathcal{J} (U)$ is the spec of
invariant differentiable vector fields on $p^{-1}(U) \subset
P$. Clearly each $\mathcal{J}(U)$ is a module over the algebra
$\mathscr{U}(U)$ of differentiable functions on $U$ and $\mathcal{J}$
is sheaf of $\mathscr{U}$-modules. 

Let $U$ be an open subset of $V$ such that $P$ is trivial over $U$ and
such that the module $\mathscr{C}(U)$ of vector fields on $U$ is a
free module over $\mathscr{U} (U)$. Let $\underbar{X}_1.\underbar{X}_2
, \ldots \underbar{X}_m$ be a base of $\mathscr{C}(U)$ over
$\mathscr{U}(U)$. Let $I_1, I_2, \ldots I_n$ be a base of right
invariant vector fields on $G$. Then $((\underbar{X}_i, 0), (0, I_j))$
for $i = 1, 2, \ldots m, j = 1, 2, \ldots n$ is a base of the
$\mathscr{U}(U)$-module of vector fields on $U \times G$ invariant
under $G$ acting to the right. This base satisfies the condition of
Prop.\ref{chap5:sec4:prop2}. 
Therefore the sheaf of invariant vector fields on $U \times G$ is a
free sheaf of rank = $m+n$ over the sheaf $\mathscr{U}$, restricted to
$U$. Hence,\textit{$\mathcal{J}$ is a locally free sheaf of
  $\mathscr{U}$ modules of rank equal to } $\dim V+ \dim G$. 

By Chap.\ref{chap5:sec4}, there exists a vector bundle $j = T^*\mathcal{J}$ of
dimension = $\dim V + \dim G$ canonically associated to
$\mathcal{J}$. Any point in the\pageoriginale fibre $J_x $ of $J$ at $x \in V$ may
be interpreted as a family of vectors $L$ along the fibre $P_x$
satisfying the condition 
$$
L_{\xi s} = L(\xi)s \text{ for } \xi \in P_x , s \in G.
$$

Let $ \mathscr{K}$ be the sheaf of $\mathscr{U}$-modulo over $V$ such
that, for every open subset $U \subset V, \mathscr{K} (U)$ is the
sub-module of $\mathcal{J}(U)$ consisting of invariant vector fields
\textit{tangential } to the fibres. Then $\mathscr{K}$ is locally free
of rank $= \dim G$ and there exists a canonical injection $\mathscr{K}
\to \mathcal{J}$. The vector bundle $K = T^* \mathscr{K}$ associated to
$\mathscr{K}$ is a vector bundle of dimension $= \dim G$ over $V$. Any
point in the fibre $K_x$ of $K$ at $x \in V$ may be regarded as a
vector field of the manifold $P_x$ invariant under the action of $G$. 

For every open set $U \subset V$, any $X \in \mathcal{J}(U)$ is
projectable and the projection defines a homomorphism $P_U\in $
Hom$_{\mathscr{U}(U)}(\mathcal{J}(U), \mathscr{C}(U))$. 
The family of homomorphisms $p_U$ gives rise to a homomorphism 
$p:\mathcal{J} \to \mathscr{C}$. Since $p_U$ is surjective when $U$ is
paracompact, 
$p:\mathcal{J} \to \mathscr{C}$ is surjective. On the other hand,
since the kernel of $p_U$ is $\mathscr{K}(U)$, the kernel of $p$ is
the image of $ \mathscr{K} \to \mathcal{J}$. Therefore we have the
exact sequence of locally free sheaves over $V$:   
$$
0\to \mathscr{K} \to \mathcal{J} \to \mathscr{C} \to 0.
$$
This gives rise to an exact sequence of vector bundles over $V$:
$$
0 \to K \to J \to C \to 0
$$
where\pageoriginale $C$ denotes the tangent bundle of $V$.

\setcounter{theorem}{0}
\begin{theorem}\label{chap5:sec5:thm1}%theorem 1
  The vector bundle $K$ is a vector bundle associated to $P$ with
  typical fibre $\mathscr{Y}$, the action of $G$ on $\mathscr{Y}$
  being given by the adjoint representation. 
\end{theorem}

Let $E = (P \times \mathscr{Y})/G$ the vector bundle over $V$
canonically associated to $P$ and $q$ the map $P \times \mathscr{Y}
\to E$. We shall define a canonical isomorphism of $E$ onto $K$. Let
$U$ be an open subset of $V$ and $\sigma \in \varepsilon (U)$ a
differentiable section of $E$ on $U$. We define a differentiable map
$\tilde{\sigma}$ of $p^{-1}(U)$ into $\mathscr{Y}$ by the condition
$q(\xi, \tilde{\sigma} \xi) = \sigma (p \xi)$ for every $\xi \in
p^{-1}(U)$. Then $\tilde{\sigma} (\xi s) = s^{-1} \tilde{\sigma} (\xi)s$ for
every $s \in G$ and $\xi \in p^{-1}(U)$. Therefore the vector field
$X^{\sigma}$ on $p^{-1}(U)$ defined by $X^{\sigma}_{\xi} = \xi
(\tilde{\sigma \xi})$ belongs to $\mathscr{K}(U)$. It is easy to
verify that the map $\sigma \to X^\sigma$ is an isomorphism
$\lambda_U$ of the $\mathscr{U} (U)$-module $\varepsilon (U)$ onto the
$\mathscr{U}(U)$-module $\mathscr{K}(U)$. The family $\lambda_U$ ($U$
open in $V$) is an isomorphism of the sheaf $\varepsilon$ of
differentiable sections of $E$ onto the sheaf $\mathscr{K}$. Hence
$(\lambda_U)$ defines an isomorphism $\lambda$ of $E$ onto $K$. Let
$(\xi , a) \in P \times \mathscr{Y}$. Then $\lambda q (\xi , a) =
\varphi_{x U} X$ where $U$ is an open neighbourhood of $x$ and $X \in
\mathscr{K} (U)$ a vector field satisfying the condition $X_{\xi} =
\xi a$. 

The isomorphism $\lambda$ will be used later to identify the bundle $E
= (P \times \mathscr{Y})/G$ with $K$. The vector bundle $K$ is called
the \textit{adjoint bundle of $P$}. 

\begin{defn}\label{chap5:sec5:def3}%definition 3
  Let\pageoriginale
  $$
  0 \to E' \xrightarrow{h} E \xrightarrow{k} E'' \to 0
  $$
  be an exact sequence of vector bundles over the manifold $V$. A {\em
    splitting} of this exact sequence is an exact sequence $0 \to E''
  \xrightarrow{\mu} E \xrightarrow{\lambda}E' \to 0$, such that
  $\lambda h$ is the identity on $E'$ and $k \mu$ the identity on
  $E''$. Any homomorphism $\lambda$ (resp $\mu$) of $E$ into $E'$
  (resp of $E''$ into $E$) such that $\lambda h$ (resp $k \mu$) is the
  identity determines a splitting. We shall now interpret the splitting
  of the exact sequence 
  \begin{equation}
    0 \to K \to J \to C \to 0. \tag{S}
  \end{equation}
\end{defn}

For every open set $U \subset V$, the module $\mathcal{J}(U)$
(resp. $\mathscr{K} (U)$) will be identified with the module of
differentiable sections of $J$ (\resp. $K$) over $U$. 

\begin{theorem}\label{chap5:sec5:thm2}%theorem 2
  There exists one and only one bijection $\rho$ of the set of
  connections on $P$ onto the set of splittings $J \to K$ of $(S)$
  such that, if $\Gamma$ is a connection on $P$, 
  $$
  \Gamma (X)=  (\rho \Gamma) \circ  X
  $$
\end{theorem}

If $\Gamma$ is a connection on $P$, then, for every open set $U
\subset V$, the restriction of the tensor $\Gamma$ to $U' = p^{-1}
(U)$ is a projection of the module $\mathscr{C}(U')$ of vector fields
on $U '$ onto the module $\mathfrak{N}(U')$ of\pageoriginale vector fields on $U'$
tangential to the fibres which is invariant under the action of $G$
and induces a projection $\lambda_U : \mathcal{J} (U) \to \mathscr{K}
(U)$. The family $(\lambda_U)$ is a homomorphism of the sheaf
$\mathcal{J}$ onto the sheaf $\mathscr{K}$ and defines a homomorphism
$\rho \Gamma : J \to K$ which is a splitting of $(S)$. For every
invariant vector field $X$ on $P$, we have $(\rho \Gamma) \circ X =
\lambda_V (X) = \Gamma (X)$. Since two homomorphisms $\lambda,
\lambda' : J \to K$ such that $\lambda \circ X = \lambda' \circ X$ for every
differentiable section $X$ of $J$ coincide, the map $\rho$ is
completely determined by the condition $\Gamma (X) = (\rho \Gamma) \circ
X$. Moreover, since two connections $\Gamma$ and $\Gamma '$ such that
$\Gamma(X) = \Gamma '(X)$ for every invariant vector field $X$
coincide, $\rho$ is injective. It remains to prove that $\rho$ is
surjective. Let $\lambda$ be a splitting of $(S)$. For every open set
$U \subset V$, Let $\lambda_U$ be the projection of $\mathcal{J} (U)$
onto $\mathscr{K} (U)$ defined by setting $\lambda_U X = \lambda \circ X$
for every $X \in \mathcal{J} (U)$. Assume $P$ to be trivial over $U$
and let $U' = p^{-1} (U)$. The injection $\mathcal{J} )(U) \to
\mathscr{C} (U')$ defines an isomorphism of the $\mathscr{U} (U
')$-module $\mathscr{U}(U') \bigotimes \limits_{\mathscr{U}(U)}
\mathcal{J} (U)$ onto $\mathscr{C}(U')$. The injection $\mathscr{K}
(U) \to \mathfrak{N} (U')$ defines an isomorphism of the $\mathscr{U}(U
')$-module $\mathscr{U} (U ') \bigotimes\limits_{\mathscr{U}(U)}
\mathscr{K} (U)$ onto
$\mathfrak{N} (U')$. Therefore $\lambda_U$ is the restriction of a
projection $\lambda$ of $\mathscr{C}(U')$ onto $\mathfrak{N} (U')$
which, regarded as tensor over $U'$ is invariant under the action of
$G$. Hence $\lambda_V$ is the restriction to $\mathcal{J}(V)$ of a
projection $\Gamma : \mathscr{C} (P) \to \mathfrak{N} (P)$ which is a
connection on $P$. For every invariant vector field $X$ on $P, \Gamma
(X) = \lambda_V x = \lambda \circ X$. Therefore $\lambda = \rho \Gamma$. 

\begin{remark*}%Remk
  Let $0 \leftarrow K \xleftarrow{\lambda} J \xleftarrow{\mu} C
  \leftarrow 0$ be the splitting of $(S)$ corresponding\pageoriginale to a
  connection $\Gamma$ on $P$. For every vector field $\underbar{X}$ on
  the base manifold, regarded as a section of $C$, the invariant
  vector field $\mu \circ \underbar{X}$ is the horizontal vector field on
  $P$ (for the connection $\Gamma$), having $\underbar{X}$ as
  projection (cf. Atiyah \cite{2}). 
\end{remark*}

\section{Connections and derivation laws}\label{chap5:sec6}% Sec 5.6

Let $V$ and $V'$ be two differentiable manifolds, and $E, E'$ two
differentiable vector bundles of dimension $n$ over $V$ and $V'$
respectively. Let $h$ be a homomorphism of $E'$ into $E$ with
projection $\underbar{h} : V' \to V$. Assume the map $h : E'_{x'} \to
E_{\underbar{h}(x')}$ to be bijective for every $x' \in V'$. If
$\sigma$ is a section 
 of $E$ over $V$, then exists one and only one section $\sigma '$ of
 $E'$ over $V'$ such that $h \sigma ' = \sigma \underbar{h}$. If
 $\sigma$ is differentiable, so is $\sigma '$. Since $\underbar{h}$ is
 a differentiable map of $V'$ into $V$, any module over the algebra
 $\mathscr{U}(V')$ of differentiable functions over $V'$ can be
 regarded as a module over the algebra $\mathscr{U}(V)$ of
 differentiable functions over $V$. Then the map $\sigma \to \sigma '$
 is a homomorphism of $\varepsilon (V)$ into $\varepsilon (V')$
 regarded as $\mathscr{U}(V)$-modules. 
 
 In particular, let $P$ be a principal differentiable bundle with
 group $G$ over $V$ and let $p : P \to V$ be the projection. Let $s
 \to s_L$ be a linear representation of $G$ in a vector space $L$. 
  Assume $E$ to be a vector bundle associated to $P$ with typical
  fibre $L$ and $q$ to be the map $P \times L \to E$. Taking $V' = P ,
  E' = P \times L, h = q , \underbar{h} = p$, for every section
  $\sigma$ of $E$ the section $\sigma '$ is defined\pageoriginale by $q \sigma
  '(\xi) = \sigma p(\xi)$ for $\xi \in P$. Let $\mathscr{L}(P)$ be
  the space of differentiable maps of $P$ into $L$. Since to any
  section $\sigma '$ of $E' = P \times L$ corresponds a map
  $\tilde{\sigma} \in \mathscr{L} (P)$ such that $\sigma'(\xi) = (\xi
  , \tilde{\sigma}(\xi))$, we obtain a \textit{homomorphism $\lambda :
    \sigma \to \tilde{\sigma}$ of the $\mathscr{U}(V)$-module
    $\varepsilon(V)$ into $\mathscr{L}(P)$ such that} 
  $$
  q(\xi , \tilde{\sigma} \xi) = \sigma_p (\xi)
  $$
  for every $\xi \in P$.

\begin{defn}\label{chap5:sec6:def4}%definition 4
  A differentiable function on $P$ with values in the vector space $L$
  is said to be a \textit{$G$-function} if $f(\xi s) = s^{-1}_L
  f(\xi)$ for every $\xi \in P, s \in G$. 
\end{defn}

We shall denote the space of $G$-functions $P \to L$ by $\mathscr{L}_G (P)$.

\begin{proposition}%Prop 4
  The homomorphism $\lambda : \varepsilon (V) \to \mathscr{L}(P)$ is
  injective and $\lambda \,\varepsilon(V) = \mathscr{L}_G (P)$. 
\end{proposition}

If $\sigma \in \varepsilon(V)$ and $\lambda \sigma = 0$, then $\sigma
(\xi) = 0$ for every $\xi \in P$. Therefore $\sigma = 0$ since $p$ is
surjective. On other hand, if $\sigma \in \varepsilon (V)$ and
$\tilde{\sigma} = \lambda \sigma$, we have $q(\xi s,
\tilde{\sigma}(\xi s)) = \sigma p (\xi s) = \sigma p (\xi) = q(\xi ,
\tilde{\sigma} \xi) = q(\xi s, s^{-1}_L \tilde{\sigma} (\xi))$ for every $\xi
\in P$ and $s \in G$. Hence $\tilde{\sigma} \in \mathscr{L}_G
(P)$. Conversely, let $f \in \mathscr{L}_G (P)$. The map $\xi
\to q(\xi , f \xi)$ maps each fibre into a point of $E_X$ and
therefore can be written $\sigma_p$, with $\sigma \in \varepsilon
(V)$. We have $f = \lambda \sigma$. 

We shall now show how a connection on $P$ gives rise to a derivation
law in the module of sections of the vector bundle $E$ over $V$. By
means of the map $\varepsilon (V) \to \mathscr{L} (P)$, the derivation
law in the $\mathscr{U}(V)$-module $\varepsilon (V)$ will be deduced
from a derivation law in the $\mathscr{U}(P)$-module
$\mathscr{L}(P)$.\pageoriginale 

To the representation $s \to s_L$ of $G$ in $L$ corresponds a
representation in $L$ of the Lie algebra $\mathscr{Y}$ of left
invariant vector fields on $G$. For every $a \in \mathscr{Y}$, define
$a_L$ by setting $a_L v = av-v$ for every $v \in L$, the vectors at
$0$ of $L$ being identified with elements of $L$. Then it is easy to
see that $[a, b]_L = a_L b_L - b_L a_L$. Hence $a \to a_L$ is a linear
representation of $\mathscr{Y}$ in $L$. 

Let $\mathscr{Y}(P)$ be the  space of differentiable functions on $P$
with values in $\mathscr{Y}$. The linear map $a \to a_L$ of
$\mathscr{Y}$ into $\hom_R (L, L)$ defines an $\mathscr{U}(P)$-linear
map $\alpha \to \alpha_L$ of $\mathscr{Y}(P)$ into
$\hom_{\mathscr{U}(P)} (\mathscr{L}(P), \mathscr{L}(P))$ where
$(\alpha_L f) (\xi) = \alpha (\xi)_L f(\xi)$ for $\xi \in P, \alpha
\in \mathscr{Y} (P)$ and $f \in \mathscr{L} (P)$. 

We have already seen (Ch. $1$) that there are canonical derivation
laws in the modules $\mathscr{Y}(P)$ and $\mathscr{L} (P)$ and hence
in $\hom_{\mathscr{U}(P)} (\mathscr{L}(P) , \mathscr{L}(P))$ also. It
is easy then to see that 
$$
(X \alpha)_L f = X(\alpha_L f) - \alpha_L (X f)
$$
for every $\alpha \in \mathscr{Y} (P)$ and $f \in \mathscr{L} (P)$. We
now define in $\mathscr{L}(P)$ a new derivation law in terms of a
given connection form $\gamma$ on the bundle by setting 
$$
D_X f = X f + \gamma (X)_L f 
$$ 
for every $X \in \mathscr{C}(P)$ and $f \in \mathscr{L}(P)$. This
differs from the canonical derivation law in $\mathscr{L}(P)$ by the
map $\gamma : X \to \gamma (X)_L$ of $\mathscr{C}(P)$ into
$\hom_{\mathscr{U}(P)} (\mathscr{L}(P), \mathscr{L} (P))$.\pageoriginale The
curvatures form the canonical derivation law has been shown to be zero
in Ch.\ref{chap1:sec9}. Hence if $K$ is the curvature form of the derivation law
and $\mathcal{K}$ the curvature form of the connection from $\gamma$, we
have 
\begin{align*}
  K(X,Y) & = D_X D_Y - D_Y D_X - D_{[X.Y]}\\
  & = X \gamma (Y)_L - Y \gamma (X)_L - \gamma ([X, Y])_L + [\gamma
    (X)_L , \gamma (Y)_L]\\ 
  & = X \gamma (Y)_L - Y \gamma (X)_L - \gamma ([X, Y])_L + [\gamma (X)
    , \gamma (Y)]_L\\ 
  & = (d \gamma (X, Y))_L + [\gamma (X), \gamma (Y)]_L\\
  & = K(X, Y)_L
\end{align*}

\begin{theorem}\label{chap5:sec6:thm3}%theorem 3
  For a given connection form $\gamma$ on $P$, there exists one and
  only one derivation law $D$ in the module of sections on the vector
  bundle $E$ on $V$ such that for every section $\sigma$ and every
  projectable vector field $X$ on $P$, we have  
  $$
  D^{\sim}_{pX} \sigma = X \tilde{\sigma} + \gamma (X)_L \tilde{\sigma}
  $$
\end{theorem}

The proof is an immediate consequence of the two succeeding lemmas

\setcounter{lem}{0}
\begin{lem}\label{chap5:sec6:lem1} %lemma 1
  If $X$ is a vector field on $P$ tangential to the fibres and $f
  \in \mathscr{L} (P)$, then $D_X f = 0$. 
\end{lem}

In fact, if $X = Z_a$ where $a \in \mathscr{Y}$, we have
$$
D_{Z_a} f = Z_a f + (\gamma(Z_a))_L f = Z_a f + a_L f
$$

Hence\pageoriginale
$$
D_{Z_a} f (\xi) =(\xi a) f + a_L f(\xi).
$$

If $g(s) = f (\xi s) = s^{-1}f (\xi)$, then $(\xi a) f = ag = - a_L f
(\xi)$. Therefore $D_{Z_a} f = 0$ and the lemma is proved since the
module of vector fields tangential to the fibres is generated by the
vector fields $Z_a (a \in \mathscr{Y})$. 

\begin{lem}\label{chap5:sec6:lem2}%lemma 2
  If $X$ is projectable and $f \in \mathscr{L}_G (P)$, then $D_X f \in
  \mathscr{L}_G (P)$. 
\end{lem}

By lemma \ref{chap5:sec6:lem1}, it is enough to consider the case when $X$ is a
horizontal projectable vector field. Then $\gamma (X) = 0$ and hence
we have 
$$
(D_X f) (\xi s) = (X f) (\xi s) = X_{\xi s} f = (X_\xi s) f = X_\xi h ,
$$
where $h(\xi) = f(\xi s) = s^{-1} f$. Therefore
$$
(D_X f) (\xi s) = s^{-1} (X_\xi f) = s^{-1} (D_X f) (\xi)
$$
for $s \in G$, i.e., $D_X f \in \mathscr{L}_G (P)$.

The converse cannot be expected to be true in general. For, if $E$ is
a trivial vector bundle over $V$ with group reduced to $\{e\}$, it is
clear that we can have only the trivial connections in $P$, whereas
there are nonzero derivation laws in $V$. However, we have the  

\begin{theorem}\label{chap5:sec6:thm4}%theorem 4
  Let $G$ be the group of automorphisms of a vector space $L, P$\pageoriginale a
  principal differentiable bundle with group $G$ over a manifold $V$
  and $E$ a differentiable vector bundle over $V$ associated to $P$
  with typical fibre $L$. Let $p$ be the projection of $P$ onto
  $V$. For every derivation law $D$ in the $\mathscr{U}(V)$-module
  $\varepsilon (V)$ of differentiable sections of $E$, there exists one
  and one connections form $\gamma$ on $P$ such that
  \begin{equation}
    D_{pX}^{\sim} \sigma = X \tilde{\sigma} + \gamma (X)_L \tilde{\sigma}\tag{C}
  \end{equation}
  for every $\sigma \in \varepsilon (V)$ and every projectable vector
  field $X$ on $P$. 
\end{theorem}

We denote as before by $q$ the map of $P \times L$ onto $E$. For every
$\sigma \in \varepsilon (V)$ and every vector $dx$ at a point $x \in
V$, let $D_{dx}$ be the value $(D_{\underbar{X}} \sigma) (x)$, where
$\underbar{X}$ is any vector field on $V$ such that $\underbar{X}_x =
dx$. Let $d \xi$ be a vector with origin at $\xi \in P_x$. We define a
map $r (d \xi)$ of $\varepsilon (V)$ into the fibre $E_x$ by setting 
$$
r(d \xi) \sigma = D_{pd \xi} \sigma - q (\xi , d \xi \tilde{\sigma})
$$
for every $\sigma \in \varepsilon (V)$. Since $r(d \xi) (f \sigma) =
f(x) (r(d \xi) \sigma)$ for every $f \in \mathscr{U}(V)$ and every
$\sigma \in \varepsilon (V) , r (d \xi) \sigma$ depends only upon
$\sigma (x)$ or upon $\tilde{\sigma} (\xi)$. Therefore, there exists
an endomorphism $\alpha$ of $L$ such that $r(d \xi) \sigma = q (\xi ,
\alpha \tilde{\sigma}(\xi))$ for every $\sigma \in \varepsilon
(V)$. Since $G$ is the group of automorphisms of $L$, there exists an
element $\gamma (d \xi)$ in the Lie algebra $\mathscr{Y}$ of $G$ such
that $r(d \xi) \sigma = q (\xi ,\gamma (d \xi)_L \tilde{\sigma}(\xi))$
for every $\sigma \in \varepsilon (V)$, and $\gamma$ is clearly a form
degree $1$ on $P$ with\pageoriginale values in $y$ satisfying the condition $(C)$. From
$(C)$ we deduce that $\gamma(X)$ is differentiable whenever $X$ is a
differentiable projectable vector field on $P$. Therefore $\gamma$ is
differentiable. It is the easy to verify that $\gamma$ is a connection
form on $P$. Any connection form on $P$ satisfying the condition $(C)$
coincides with $\gamma$ since the representation of $\mathscr{Y}$ in
$L$ is faithful and since a form on $P$ is determined by its values on
projectable vector fields. 

\section{Parallelism in vector bundles}\label{chap5:sec7}%Sec 5.7

Let $E$ be a vector bundle over $V$ associated to the principal bundle
$P$ with typical fibre $L, \gamma$ a connection form on $P, D$ the
corresponding derivation law in the module of sections $\varepsilon (V)$
of $E, q$ the usual map $P \times L \to E , \varphi$ a differentiable
path in $V$ with origin at $x \in V$, and $\hat{\varphi}$ an integral
lift of $\varphi$ in $P$ with respect to $\gamma$. For every $y \in
E_x$, there exists $v \in L $ such that $q (\hat{\varphi}(0), v) =
y$. 

Then it is easy to see that the path $q(\hat{\varphi}(t), v)$ in $E$
depends only on $y$ and $\varphi$. The path $q(\hat{\varphi}(t), v)$
is called the integral lift of $\varphi$ with origin $y$. The vectors
$q(\xi (t), v)$ are sometimes said to be parallel vectors along
$\varphi$. 

Let $\sigma$ be a section of the bundle $E$ and $dx$ a vector at $ x
\in V$. We would like to define the derivation of $\sigma$ with
respect to $dx$ in terms of parallelism. Let $\varphi$ be a
differentiable path with origin $x$ and $\varphi '(0) = dx$ and
$\hat{\varphi}$ an integral lift of $\varphi$ in $P$. 
Let $v(t) = \tilde{\sigma}(\hat{\varphi}(t))$ so that
$q(\hat{\varphi}(t), v(t)) = \sigma (\varphi(t))$. Define $y(t) =
q(\hat{\varphi}(0), v(t))$. Then we have 
$$
D_{dx} \sigma = \lim_{t \to 0} \frac{1}{t} \{y(t)-y(0)\}. 
$$

In\pageoriginale fact, if
\begin{align*}
  \xi &= \hat{\varphi} (0)\quad \text{ and
  }\quad d \xi = \hat{\varphi}' (0), \text{we have}\\ 
  \lim_{t \to 0} \left\{ \frac{1}{t} (y(t)-y(0))\right\} & = q(\hat{\varphi}(0),
  \lim_{t \to 0} \frac{1}{t} \left\{ v(t) - v(0)\right\} )\\ 
  & =  q(\hat{\varphi}(0), \lim_{t \to 0} \frac{1}{t} \left\{
  \tilde{\sigma}(\hat{\varphi}(t) - \tilde{\sigma}(\hat{\varphi}(0))\right\}
  )\\ 
  & = q(\hat{\varphi}(0), d \xi \tilde{\sigma})\\
  & = q(\hat{\varphi}(0), d \xi \tilde{\sigma} + \gamma (d \xi )_L
  \tilde{\sigma})\\ 
  & = D_{dx} \sigma. 
\end{align*}

For every $Y \in E_x$, let $(\xi, v) \in P \times L$ such that $q(\xi
, v)  = Y$. Then $q(d \xi , v) \in T_Y$. This gives in particular, a
map of the space $\zeta_\xi$ of horizontal vectors with origin at
$\xi$ into $T_Y$, which is clearly injective. If we compose this with
the projection, we obtain the bijection $p : \zeta_\xi \to T_x$. We
define a vector at $Y$ in $E$ to be \textit{horizontal} if it is of
the form $q(d \xi , v)$ with $q(\xi , v) = Y$ and $d \xi \in
S_\xi$. It is easy to see that this definition does not depend on the
particular pair $(\xi , v)$. With this definition, for every integral
$\hat{\varphi}$ in $P$ and every $ u \in L$, the path $Y(t) = q
(\hat{\varphi} (t), v)$ has horizontal agents at all points. The
space of horizontal vectors at $Y$ is a subspace of $T_Y$
supplementary to the subspace of vectors $\in T_Y$ tangents to the
fibres through $Y$. 

\begin{theorem}\label{chap5:sec7:thm5}%theorem 5
  Let\pageoriginale $\sigma$ be a section of $E$ over $V$. Then $\sigma(dx)$ is
  horizontal if and only if $D_{dx} \sigma = 0$. 

  Let $d \xi $ be a horizontal tangent vector at $\xi \in p^{-1}(x)$
  whose projection is $dx$. Then we have 
  \begin{align*}
    D_{dx} \sigma &= q (\xi , d \xi \tilde{\sigma} + \gamma (d \xi)
    \tilde{\sigma})\\ 
    & = q(\xi , d \xi \tilde{\sigma}).
  \end{align*}
\end{theorem}

But $\tilde{\sigma}$ has been defined by 
$$
q(\xi , \tilde{\sigma} (\xi)) = \sigma (p \xi). 
$$

Hence $q(\xi , \tilde{\sigma}(d \xi)) + q(d \xi , \tilde{\sigma}(\xi)) =
\sigma (pd \xi)= \sigma dx$. If $\sigma (dx)$ is horizontal, so is
$q(\xi , \tilde{\sigma} (d \xi))$. But $p_E. q(\xi , \tilde{\sigma} (d
\xi)) = x$, i.e., $\tilde{\sigma} (d \xi) = \tilde{\sigma}
(\xi)$. Since $d \xi \tilde{\sigma} = \tilde{\sigma} (d \xi) -
\tilde{\sigma} (\xi)$, it follows that $d \xi \tilde{\sigma} =
0$. Therefore $q(\xi, d \xi \tilde{\sigma}) = 0$. The converse is also
immediate. 

This theorem enables us to define integral paths in $E$ with respect
to a given derivation law in $\varepsilon (V)$. 

\section{Differential forms with values in vector bundles}\label{chap5:sec8}%Sec 5.8

\begin{defn}\label{chap5:sec8:def5}%definition 5
  Let $E$ be a differentiable vector-bundle over the manifold
  $V$. {\em A differential form} $\alpha$ of degree $n$ on $V$ {\em
    with values in the vector bundle} $\underbar{E}$ is an $n$-form on
  the module $\mathscr{C}(V)$ of differentiable vector fields on $V$
  with values in the module $\varepsilon(V)$ of differentiable\pageoriginale
  sections of the bundle $E$. 
\end{defn}

For every $n$-tuple $d_1 x , d_2 x , \ldots d_n x$ of vectors at $x
\in V$, the value $\alpha (d_1 x, d_2 x , \ldots d_n x)$ of $\alpha$
belongs to $E_x$. The $\mathscr{U}(V)$-module of forms of degree $n$
on $V$ with values in $E$ will be denoted by $\varepsilon^n (V)$. 

Let $P$ be a differentiable principal bundle over $V$,  with  group
$G$ and projection $p$. Assume $E$ to be associated to $P$, with typical fibre
$L$ and let $q$ be the map $P \times L \to E$. For every integer $n
\ge 0$, we shall define an isomorphism $\lambda$ of $\varepsilon^n (V)$
into the space $\mathscr{L}^n (P)$ of differential forms of degree
$n$ on $P$ with values in the vector space $L$. For $ n = 0, \lambda $
has been defined in \ref{chap5:sec6}. Assume $n > 0$. For every $\alpha \in
\varepsilon^n (V)$, we define a form $\tilde{\alpha} = \lambda \alpha$
on $P$ with values in $L$ by the condition 
$$
q(\xi , \tilde{\alpha}(d_1 \xi , d_2 \xi , \ldots d_n \xi)) = \alpha
(pd_1 \xi , pd_2 \xi , \ldots pd_n \xi) 
$$ 
for\pageoriginale every sequence $d_i \xi$ of $n$ vectors with origin at $\xi \in
P$. If $X_1, X_2, \ldots X_n$ are $n$ projectable vector fields on
$P$, then 
$$
(\lambda \alpha) (X_1, X_2, \ldots X_n) = \lambda(\alpha (pX_1, pX_2 ,
\ldots pX_n)) 
$$

Therefore, $\lambda \alpha$ is differentiable and belongs to
$\mathscr{L}^n (P)$. It is immediately seen that $\lambda$ is an
injective homomorphism of $\varepsilon^n(V)$ into $\mathscr{L}^n (P)$
regarded as $\mathscr{U}(V)$-modules. Moreover, if $\tilde{\alpha}$
belongs to the image of $\lambda$ in $\mathscr{L}^n (P)$, then : 
\begin{enumerate}[1)]
\item  $\tilde{\alpha} (d_1 \xi s , d_2 \xi s , \ldots d_n \xi s) =
  s^{-1} \tilde{\alpha}(d_1 \xi , d_2 \xi , \ldots d_n \xi) s$ 

  for every sequence $d_i \xi$ of $n$ vectors with origin at $\xi \in
  P$ and every $s \in G$, 
\item \qquad $\tilde{\alpha} (d_1 \xi , d_2 \xi , \ldots d_n \xi) = 0$

  for every sequence $d_i \xi$ of $n$ vectors with origin at $\xi \in
  P$ such that one of the $d_i \xi$ has projection $0$. 
\end{enumerate}

\begin{defn}\label{chap5:sec8:def6}%definition 6
  A form $\alpha \in \mathscr{L}^n (P)$ satisfying the conditions 1)
  and 2) is said to be a {\em $G$- form of degree} $n$ on $P$ with
  values in $L$. 
\end{defn}

Let $\mathscr{L}^n_G (P)$ be the set of $G$-forms of degree $n$. It is
easy to see that $\mathscr{L}^n_G (P)$ is a submodule of
$\mathscr{L}(P)$ over $\mathscr{U}(V)$. Moreover $\mathscr{L}^n_G(P)$
is the image of the homomorphism $\lambda : \varepsilon^n (V) \to
\mathscr{L}^n (P)$. 

\section{Examples}\label{chap5:sec9}%Sec 5.9

\begin{enumerate}
\item Let $\gamma , \gamma '$ be two connection forms on the principal
  bundle $P$ over $V$. Let $\beta = \gamma - \gamma '$. Then we have
  $\beta (d \xi \,s) = s^{-1} \beta (d \xi) s$; and $\beta(d \xi) =
  \gamma (d \xi) - \gamma'(d \xi) =0$ if $pd \xi =0$. In other words,
  $\beta \in \mathscr{L}^1_G (P)$ with respect to the adjoint
  representation of $G$ in $\mathscr{Y}$. Hence $\beta =
  \tilde{\alpha}$ where $\alpha$ is a differential form of degree $1$
  on $V$ with values in the adjoint bundle of $P$ which is a vector
  bundle associated to $P$ with typical fibre $\mathscr{Y}$. This
  gives a method of finding all connection forms from a given
  one. This is particularly useful when $G$ is abelian, in which case
  the adjoint representation of $G$ in $\mathscr{Y}$ is trivial and
  consequently $\alpha$ may be considered as a differential form on
  $V$ with values in $\mathscr{Y}$ 
\item Let\pageoriginale $K$ be the curvature form of the connection form
  $\gamma$ on a principal bundle $P$. Then 
\end{enumerate}
$$
\displaylines{\hfill 
   K (d_1 \xi s, d_2 \xi s) = s^{-1} K (d_1 \xi , d_2 \xi)
   s\hspace{1.4cm} \hfill\cr\\ 
   \text{ and }\hfill  
    K (d_1 \xi , d_2 \xi) = 0~ \text{ if either}~ d_1 \xi ~ or ~
    d_2 \xi \in \mathfrak{N}_\xi \hfill }
$$

  Therefore $K$ is an alternate $G$-form of degree $2$ on $P$ with
  values in $\mathscr{Y}$ and corresponds to a form of degree $2$ on
  $V$ with values in the associated adjoint bundle. 

\section{Linear connections and geodesics}\label{chap5:sec10}%Sec 5.10

Let $C$ be the vector bundle of tangent vectors on $V$. We consider
$C$ as a vector bundle associated to the principal bundle $P$ of
tangent frames on $V$ with typical fibre $R^n$. There exists an
one-one correspondence between connections on $P$ (which are called
\textit{linear connections} on $V$) and derivations  laws in the
module $\mathscr{C}(V)$ of vector fields on $V$. The torsion form is
an alternate form of degree $2$ with values in the tangent bundle
$C$. 

Given any linear connection on $V$, we have the notion of
\textit{geodesics} on $V$. (In Ch.\ref{chap5:sec10} to Ch.\ref{chap5:sec12}, we consider
paths with arbitrary intervals of definition). In fact, if $\varphi$
is a differentiable path in $V$, then there exists a \textit{canonical
  lift} of $\varphi$ in $C$ defined by $t \to \varphi'(t)$. We shall
denote this lift by $\varphi '$. 

\begin{defn}\label{chap5:sec10:def7}%definition 7
  A path $\varphi$ in $V$ is said to be a {\em geodesic} if the
  canonical lift $\varphi'$ is integral. (In other words, the vector
  $\varphi''(t)$ at\pageoriginale $\varphi'(t) \in C$ should be horizontal for the
  given linear connection). 
\end{defn} 

\begin{lem}\label{chap5:sec10:lem3}%lemma 3
  Let $\varphi$ be a path in $V$ and $\hat{\varphi}$ an integral lift
  of $\varphi$ in $P$. Then $\varphi$ is a geodesic if and only if
  there exists $v \in R^n$ such that $q(\hat{\varphi}(t), v) =
  \varphi'(t)$ for every $t \in I$. 
\end{lem}

The proof is trivial.

It will be noted that the notion of a geodesic depends essentially on
the parametrisation of the path. However, a geodesic remains a
geodesic for \textit{linear} change of parameters. 

Let $Y \in C$ be a vector at a point $x \in V$ and $\theta_Y$ the
horizontal vector at $Y$ whose projection is $Y$. Then $\theta$ is a
vector field on $C$ called the \textit{geodesic vector field}. The
geodesic vector field is \textit{differentiable} according to the
following computation of $\theta$ in local coordinates. 

\section{Geodesic vector field in local coordinates}\label{chap5:sec11}%Sec 5.11

Let $C$ be the tangent bundle of a manifold $V$ with a derivation law
$D$. Let $p$ be the projection $C \to V$. If $Y$ is a vector at $x = p
Y \in V$ and $U$ a neighbourhood of $pY$ wherein a coordinate system
$(x^1, \ldots x^n)$ is defined, we denote the vector fields
$\dfrac{\partial}{\partial x^i}$ on $U$ by $P_i$. Let $y^i = dx^i$ be
a family of differential forms on $U$ dual to $P_i$, i.e., $y^j(P_i) =
\delta^j_i$. The $y^j$ may also be regarded as scalar functions on
$p^{-1}(U)\subset C$. Thus for $Y \in P, (y^1, y^2, \ldots y^n ,x^1,
x^2, \ldots x^n)$ form a coordinate system in $p^{-1}(U)$. Set $Q_i =
\partial/ \partial y^i$. Then we assert that the geodesic vector field
$\theta$ is given in $p^{-1}(U)$ by 
$$
\theta = \sum_i y^i P_i - \sum_{i,j,k} \Gamma_{i,j}^k y^i y^j Q_K
$$
where\pageoriginale the $\Gamma^k_{i, j}$ are defined by
$$
D_{P_i}P_j = \sum_k \Gamma^k_{i,j} P_k. 
$$

In fact, using Th.\ref{chap5:sec7:thm5}, Ch.5.7, we see that the value of the
differential forms $dy^j + \sum \limits_{i, k} y^k
\Gamma^j_{i,k}(x)dx^i$ on the vector is zero. It is easy to observe
that $\theta_Y = \sum\limits_{i} y^i (P_i)_Y - \sum\limits_{i,j,k}
\Gamma^k_{i,j} y^i y^j Q_k$ is horizontal and that $p \theta_Y =
Y$. This shows that $\theta$ as defined above is the geodesic vector
field.	 

\section{Geodesic paths and geodesic vector field}\label{chap5:sec12}%Sec 5.12

\begin{proposition}\label{chap5:sec12:prop4}%proposition 4
  $\varphi$ is geodesic path in $V$ if and only if the canonical lift
  $\varphi'$ of $\varphi$ is integral for $\theta$, i.e.,
  $\varphi''(t) = \theta_{\varphi'(t)}$ for every $t$. 
\end{proposition}

In fact, if $\varphi$ is a geodesic, then we have $p\varphi'(t) =
\varphi(t)$ and $p\varphi''(t) = \varphi(t)$ for every $t \in
I$. Hence $\varphi ''(t) = \theta_{\varphi''(t)}$. The converse is
trivial in as much every path integral for $\theta$ is also integral
for the connection. 

\begin{proposition}\label{chap5:sec12:prop5}%proposition 5
  If $\psi$ is an integral path for $\theta$ then $\varphi =p\psi$
  is a geodesic $\varphi'(t) = \psi(t)$ 
\end{proposition}

By assumption $\psi'(t) = \theta_{\psi(t)}$ and if $\varphi(t) = p
\psi(t)$, we have $\varphi'(t) = p \psi'(t) = p \theta_{\psi(t)}$,
i.e., $\psi$ is the canonical lift of $\varphi$ and is integral. 

The\pageoriginale geodesic vector field $\theta$ on $C$ generates a local
one-parameter group of local automorphisms of $C$. We say that the
linear connection on $V$ is \textit{complete} if $\theta$ generates a
one-parameter group of global automorphisms of $V$. It is known that
in a compact manifold, any vector field generates such a one-parameter
group of automorphisms. On other hand, if the local one-parameter
group generated by $\theta$ in $C$ is represented locally by $2 n$
functions $\varphi_i (y_1 , \ldots y_n , x_1 , \ldots x_n, t)$ of
$(2n+1)$ variables, then it is easy to observe that the first $n$ of
these functions are linear in $y_1 , \ldots , y_n$. From this it is
immediate that any linear connection on a compact manifolds is
complete. 

The nomenclature `complete' is due to the fact that a Riemannian
connection is complete if and only if the Riemannian metric is
complete (\cite{27}). 

Let $\Gamma$ be a complete linear connection and $\theta$ the geodesic
vector field. Let $t_\theta$ be the automorphism of $C$ corresponding
to the parameter $t$ in the one-parameter group generated by
$\theta$. Paths which are integral for $\theta$ are of the form $t \to
t_\theta Y$ (i.e., the orbit of $Y$ under $t_\theta$). Given any
vector $Y$ at a point $x$ on the manifold $V, \varphi (t) = p(t_\theta
Y)$ is the \textit{geodesic curve} defined by $Y$. 
For every $x \in V$, the map $\rho : C_x \to V$ defined by setting
$\rho (Y) = p(1_\theta Y)$ is a differentiable map of maximal rank at
$0_x$ and  therefore defines a diffeomorphism of an open neighbourhood
of $0$ in the vector space $C_x$  onto an open neighbourhood of $x$ in
V(\cite{26}). 
