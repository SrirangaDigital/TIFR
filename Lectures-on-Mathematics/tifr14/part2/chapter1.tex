\chapter{Measures on locally compact spaces}\label{partII-chap1}

\setcounter{section}{1}
\setcounter{subsection}{0}
\subsection{Definition of a
measure.}\label{partII-chap1-sec1.1}%sec 1.1  

In\pageoriginale this chapter and the following, we shall give a brief summary of
certain results on measure theory, a knowledge of which is essential
in what follows. 

Let $X$ be a locally compact topological space and $\mathscr{C}_X$ the
algebra of continuous complex-valued functions on $X$ with compact
support. Let $K$ be a compact subset of $X$ and $\mathscr{C}_K$ the
subset $\{f$: support of $f \subset K\}$ of $\mathscr{C}_X$. Then
$\mathscr{C}_K$ is a Banach  space under the norm $|| f
||= \sup\limits_{x\in K} | f(x)|$.  

\begin{defi*}
 A {\em measure} on $X$ is a linear form on $\mathscr{C}_X$ such that
 the restriction to $\mathscr{C}_K$ is continuous for  every compact
 subset $K$ of $X$. A measure $\mu$  is said to be {\em positive} if
 $\mu(f) \geq 0$ for every $f\geq 0$.  
\end{defi*}

\setcounter{proposition}{0}
\begin{proposition}\label{partII-chap1-prop1}%prop 1
Every positive linear form on $\mathscr{C}_X$ is a measure on $X$. 
\end{proposition}

In fact, if $K$ is any compact subset of $X$, there exists a
continuous function $f$ on $X$ which $=1$ in $K$, $=0$ outside a compact
neighbourhood of $K$ and $0 \leq f \leq 1$. If $g$ is a function
belonging to $\mathscr{C}_K$, obviously $-|| g || f \leq g \leq || g ||
f$ and hence  
$$
- || g || \mu (f) \leq \mu (g) \leq || g || \mu (f). 
$$
i.e., 
$$
| \mu (g)| \leq || g || \mu (f), 
$$
which shows that $\mu$ is continuous when restricted to $\mathscr{C}_K$.

On the other hand, if $\mu$ is any real measure (i.e. a measure $\mu$
such that $\mu(f)$ is real whenever $f$ is real) it can be  expressed
as the difference of two positive measures. Moreover, a complex
measure $\mu$ can\pageoriginale 
be uniquely decomposed into $\nu + i \nu'$ where
$\nu$ and $\nu'$ are real.  Hence a measure can alternatively be
defined as a linear combination of positive linear forms on
$\mathscr{C}_X$. For a real measure  $\nu$, there exists a unique
`minimal' decomposition $\mu_1 -  \mu_2$ ($\mu_1,\mu_2$ positive) in
the sense that if $\nu  
=\mu'_1 - \mu'_2$ be any other decomposition, we have
$\mu'_1=\mu_1+\pi$, $\mu'_2=\mu_2+\pi$ with $\pi$ positive. 

\subsection{Topology on
$\mathscr{C}_X$.}\label{partII-chap1-sec1.2}%sec 1.2 

The space $\mathscr{C}_X = \bigcup\limits_{K} \mathscr{C}_K$ where $K$
runs through all the compact subsets of $X$ can be provided with the
topology obtained by taking the direct limit of the topologies on
$\mathscr{C}_K$. This topology makes of $\mathscr{C}_X$ a locally
convex  topological vector space. The fundamental property of this
space is that a linear map of $\mathscr{C}_X$ in a locally convex
space is continuous if and only if its restriction to each
$\mathscr{C}_K$ is continuous. (Bourbaki, Espaces vectoriels
topologiques, Chapter \ref{partII-chap2}). A measure is, by
definition, a continuous 
linear form on $\mathscr{C}_X$ with its topology of direct limit. The
space $\mathscr{M}_X$ of measures is none other than the dual of
$\mathscr{C}_X$. One can provide  $\mathscr{M}_X$ with several topologies, as
for instance, the weak topology in which $\mu \rightarrow 0
\Leftrightarrow$ for every $f \in \mathscr{C}_X$, $\mu(f)\rightarrow 0$.  

\subsection{Support of a measure.}\label{partII-chap1-sec1.3}%sec 1.3 

\begin{defi*}%def
 The {\em support} of a measure $\mu$ is the smallest closed set $S$
 such that for every function $f \in  \mathscr{C}_X$ whose support is
 contained in $X-S$, $\mu (f) =0$. 
\end{defi*}

Let $\mathscr{M}^c$ be the space of measures with compact support. If $f$ is a
continuous function on $X$, for every $\mu\in \mathscr{M}^c$, we can
define $\mu(f)=\mu(\alpha f)$\pageoriginale 
where $\alpha$ is a function 1 on a
neighbourhood of the support $K$ of $\mu$, and $0$ outside a compact
neighbourhood of $K$. It is obvious that the value of $\mu(f)$ does
not depend on $\alpha$. We shall denote by $\mathscr{E}^\circ_X$  the space of
continuous function on $X$. $\mathscr{E}^\circ_X$ with the topology of compact
convergence is a locally convex topological vector space. $\mu$
defined on $\mathscr{E}^\circ_X$ in the above manner is continuous with respect
to this topology. Conversely, let $\mu$ be a continuous form on
$\mathscr{E}^\circ_X$. The topology on $\mathscr{C}_X$ is finer than that
induced from the topology of $\mathscr{E}^\circ_X$. Hence $\mu$ restricted to
$\mathscr{C}_X$ is again continuous and is consequently a measure. We
now show that this has compact support. Since $\mu$ is a continuous
function on $\mathscr{E}^\circ_X$, we can find a neighbourhood $V$ of $0$ such
that $|\mu(f)|<1$ for every $f \in V$. $V$ may be taken  
to be of the form $\{f:| f | < \epsilon\text{~ on~ } K\}$ as the topology on
$\mathscr{E}^\circ_X$ is the topology of compact convergence. Let $g \in
\mathscr{C}_X$ be a function  $0$ on $K$. Then $| \mu(g)| < 1$. If
$\lambda$ is any complex number, $\mu (\lambda g)= \lambda \mu (g)$,
and $| \mu(\lambda g)| <1$. Hence $\mu(g)=0$, i.e. the support of
$\mu$ is contained in $K$. 

\subsection{Bounded measures.}\label{partII-chap1-sec1.4}%sec 1.4

Let $\mu$ ba a measure $\in \mathscr{M}_X$. We define a positive
measure $| \mu |$ in the following way:   

$| \mu | f=\sup\limits_{0 \leq | g | \leq f } | \mu (g)|$ for every
positive function $f$ and extend it by linearity to all functions $\in
\mathscr{C}_X$. If $\mu$ is a real measure with the minimal
decomposition (Ch. \ref{partII-chap1-sec1.1}) $\mu=\mu_1-\mu_2$, then
$|\mu|=\mu_1+\mu_2$.  

\begin{defi*}%def
 A measure $\mu$ is {\em bounded} if and only if there exists a real
 number $k$ such that $| \mu(f)| \leq  k|| f ||$ with $ || f || =
 \sup\limits_{x \in X}$ $| f(x)|$. 
\end{defi*}

 Obviously\pageoriginale $\mu$ is bounded if and only if $|\mu|$ is
 bounded, $\mu$ 
 is continuous for this norm and can be extended to the completion
 $\overline{\mathscr{C}}_{X} $ (which is only the space of continuous
 functions tending to zero at $\infty$). $\overline{\mathscr{C}}_{X}$
 is actually the adherence of $\mathscr{C}_X$ in the space of all
 continuous bounded functions. The space of bounded measures is a
 Banach space under the norm $||\mu|| = \sup\limits_{f \in
   \mathscr{C}_X} \dfrac{|\mu(f)|}{|| f ||}$. $||\mu||$ is the
 smallest number $k$ such that  $| \mu (f) | \leq k || f ||$. It can
 proved that every bounded continuous function is integrable with
 respect to a bounded measure, and we have still the inequality $| \mu
 (f) | \leq || \mu ||~ || f ||$ for bounded continuous functions $f$.  


\subsection{Integration of vector valued
 functions.}\label{partII-chap1-sec1.5}%sec 1.5  

We introduce here the notion of integration of a vector valued
function with respect to a scalar measure, a use  of which we will
have frequent occasions to resort to in the sequel. Let $K$ be a
compact space and $E$ a locally convex quasi-complete topological
vector space (i.e. every closed bounded subset is complete). We shall
provide the space $\mathscr{C} (K, E)$ of continuous functions of $K$
into $E$ with the topology of uniform convergence. 

\setcounter{thm}{0}
\begin{thm}\label{partII-chap1-thm1}%thm 1
 {Corresponding to every measure} $\mu$ {on} $K$, {there
   exists one and only one continuous linear map} $\tilde{\mu}$ of
 $\mathscr{C} (K,E)$ {in} $E$ {such that}  $\tilde{\mu} (f\cdot a)
 = \mu (f)\cdot a$ {for every continuous complex valued function} $f$
 {on} $K$ {and} $a \in E$. 
\end{thm}

$\mu$ can obviously be lifted to a linear map $\bar{\mu}$ of
$\mathscr{C} \otimes E$ in $E$ by setting $\bar{\mu}  (c \otimes
e) = \mu(c)e$ and extending by linearity. Also, $\mathscr{C} \otimes
E$ can be identified with a subset of the space $\mathscr{C}(K, E)$ of
continuous functions\pageoriginale of $K$ into $E$. We will now show that
$\bar{\mu}$ is continuous with respect to the induced topology on
$\mathscr{C} \otimes  E$  and that $\mathscr{C} \otimes E$ is dense in
$\mathscr{C}(K,E)$. We will in fact prove more generally that every
function $f \in \mathscr{C} (K,E)$ is adherent to a bounded subset of
$\mathscr{C} \otimes E$. 

Let $V$ be a convex neighbourhood of $0$ in $E$. Then there exists a
neighbourhood $A_x$ of each point $x \in K$ such that $f(y) - f(x) \in
V$ for every $y \in A_x$. Now the $A_x$ cover the compact space $K$
and let $ A_{x_{1}},\ldots, A_{x_{n}}$ be a finite cover 
extracted from it. Let $\varphi_i$  be positive continuous functions
on $K$ such that $\sum\limits_{i=1}^{n} \varphi_i=1$ and the support
of $\varphi\subset A_{x_i}$. If $g = \sum \varphi_i f({x_i})$,
then  $g \in \mathscr{C}\otimes E$, and we have  
\begin{align*}
g(y)-f(y) &= \sum \varphi_i(y)f{(x_i)} - \sum \varphi_i(y)f(y)\\
&= \sum\varphi_i(y)[f(x_i) - f(y)]\\
& \quad \in  V  \text{~ since $V$ is convex.}
\end{align*}

By allowing $V$ to describe fundamental system of neighbourhoods of
$0$, we see that $f$ is adherent to the set of such functions
$g$. This set is a bounded subset of $\mathscr{C}(K,E)$. For,
$\sum \varphi_i(y)f(x_i)$ is in the convex envelope of $f(K)$ for every
$y \in K$ and the convex envelope of a compact set is bounded. It
follows that $\sum\varphi_if(x_i)$ are uniformly bounded and hence
form a bounded subset of $\mathscr{C}(K,E)$. This proves, in
particular,that $\mathscr{C} \otimes E$ is dense in $\mathscr{C}
(K,E)$. 

Now let $g = \sum g_i a_i$ be a function $\in \mathscr{C} \otimes E$
tending to zero  
in the topology of $\mathscr{C} (K,E)$. Then $\langle \sum  g_i a_i,
a'\rangle \rightarrow  0$ uniformly on the compact set $K$ and on any
equicontinuous subset $H$ of the dual of $E$. 
\begin{align*}
\mu \langle g, a'\rangle &=  \mu \langle \sum{g_i} a_i, a' \rangle\\
&= \mu(\sum g_i\langle a_i, a'\rangle)  = \sum  \mu(g_i) \langle a_i,
a'\rangle\\ 
&= \langle\sum\mu(g_i) a_i, a'\rangle = \langle \overline{\mu}g, a'
\rangle
\end{align*}\pageoriginale

Since $\langle  g, a' \rangle \rightarrow  0$ uniformly on $K \times
H$, $ \mu  \langle  g, a'  \rangle   \rightarrow  0$ and hence
$\langle \overline{\mu}_g, a' \rangle  \rightarrow 0 $ uniformly on
any equicontinuous subset $H$ of $E$. Consequently $\overline{\mu} g
\rightarrow 0$. This shows that $\overline{\mu}$ is continuous on
$\mathscr{C} \otimes E$. 

Therefore $\overline{\mu}$ can be extended uniquely to a continuous
linear map of $\mathscr{C} (K,E)$ in the completion $\hat{E}$ of
$E$. But if $f \in \mathscr{C}(K,E)$, it is adherent to a bounded set
$B$ and $\mu(B)$ is also bounded in $E$. By the quasi-completeness of
$E$,  
the closure of $\mu(B)$ in $\hat{E}$ and $E$ are the same. Hence
$\mu$(f)$\in$ $\mu$ $(\overline{B})$ $\subset$
$\overline{\mu(B)}\subset E$. Thus we have extended $\mu$ to a
continuous linear map $\tilde{\mu}$ of $\mathscr{C} (K,E)\rightarrow
E$ and it is obvious  
this is unique. Now by Theorem \ref{partII-chap1-thm1}, if $G$ be any
locally compact space and $\mu$ a measure on $G$, we can define  $\int
f(x) d\mu = \tilde{\mu}(f)$ for every continuous function $f$ from $G$
to $E$ with compact support.  

\begin{remark*}
The measure with this extended meaning is factorial in character in
the following sense: Let $E$ and $F$ be two locally convex spaces and
f a continuous map of a compact space $K$ into $E$. If $A$ is a
continuous map of $E$ in $F$, we have $Af \in \mathscr{C} (K, F)$
and $\mu$ satisfies $\mu (Af) =A \mu (f)$. 
\end{remark*}

