
\chapter{Invariant measures}\label{partII-chap3}

\setcounter{section}{3}
\setcounter{subsection}{0}
\subsection{Modular function on a
  group.}\label{partII-chap3-sec3.1}\pageoriginale%sec  3.1 

We assume the fundamental theorem relating to measures on locally
compact groups, namely the existence and uniqueness (upto a positive
constant factor) of a right invariant positive measure. If $\mu$ is
such a measure, we have 
$$
(\epsilon_y * \mu) * \epsilon_x = \epsilon_y  * (\mu * \epsilon_x) =
\epsilon_y * \mu. 
$$

Hence $\epsilon_y * \mu$ is also a right invariant positive measure. By our
remark above, $\epsilon_y*\mu=k\mu$ where, of course, $k$ depends on
$y$. We shall denote $k$ by $\Delta (y)^{-1}$  where $\Delta  (y)$
is a positive real number. It is immediate that $\Delta (yz) =
\Delta (y) \Delta (z)$. $\Delta$ is therefore a
representation of $G$  in the multiplicative group of $R^+$. In fact,
the continuity of $\Delta (y) = \dfrac{\int f(y^{-1}x) d\mu (x)}
{\int f (x) d \mu (x)}$ follows at once from that of $\int f (y^{-1}
x) d \mu (x)$ (Lemma \ref{partII-chap2-lem1},
Ch. \ref{partII-chap2-sec2.3}). This representation $\Delta$ of a 
locally compact group is said to be its \textit{modular function}. 
  
\setcounter{proposition}{0}
\begin{proposition}\label{partII-chap3-prop1}%pro 1 
If a right invariant positive measure on $G$ is denoted by $dx$, then
 the following identity holds: $dx^{-1}=\Delta  (x^{-1} )dx$ 
\end{proposition}


In fact, if $d\mu$ stands for $\Delta (x^{-1}) dx$, we have
$$
d\mu(yx)= \Delta (x^{-1}y^{-1}) d(yx)=\Delta (x^{-1}) dx=d\mu.
$$

Hence $d\mu$ is left invariant. So also is $dx^{-1}$ for,
$$
\int f(yx) dx^{-1} = \int f(x) d(y^{-1} x)^{-1} = \int f(x) d(x^{-1}).
$$

So $k dx^{-1}=\Delta (x^{-1})dx$ where $k$ is a constant. We now
prove that $k=1$. 
When $x$ is near $e$, $\Delta (x^{-1})$ is arbitrarily near 1 and
if we take $g(x) = f(x) +f(x^{-1})$,\pageoriginale 
$f$ being a positive continuous function with sufficiently small
support, we have  
$$
\int g(x) dx = \int g(x) dx^{-1} = \frac{1}{k} \int g(x) \Delta (x^{-1}) dx
$$
$k$ being a fixed number and $\Delta (x^{-1})$ arbitrarily near 1;
it follows that $k = 1$. 

\begin{defi*} %def
 A locally compact group $G$ is said to be {\em unimodular} if its
 modular function is a trivial map which maps $G$ onto the unit
 element of $R$. 
\end{defi*}

The group of triangular matrices of the type
$
\left(
\begin{smallmatrix}
a_{11}\ldots 0\\
\ldots\ldots\\
\ldots a_{nn}
\end{smallmatrix}\right)
$
can be proved to be non-unimodular.

\medskip
\noindent
\textbf{Examples of unimodular groups.}
\begin{enumerate}
\renewcommand{\labelenumi}{(\theenumi)}
\item  A trivial example of unimodular groups is that of commutative
  groups. 

\item Compact groups are unimodular. This is due to the fact that
  $\Delta (G)$ is a compact subgroup of $R^+$ which cannot but be
  (1). 

\item If in a group the commutator subgroup is everywhere dense, then
  the group is unimodular. This again is trivial as $\Delta$ maps
  the commutator subgroup and consequently the whole group onto 1. 

\item  A connected semi-simple Lie group is unimodular. (A Lie group
  $G$ is said to be \textit{semi-simple} if its Lie algebra
  $\mathfrak{g}$ has no proper abelian ideals. Consequently, it dose
  not have proper ideals such that the quotient is abelian). The
  kernel $\mathscr{N}$ of the representation $d\Delta$ of the Lie algebra
  into the real number is an ideal in $\mathfrak{g}$ such that
  $\mathfrak{g}/\mathscr{N}$ is abelian and is therefore the whole Lie
  algebra. It follows that the map $d\Delta$ maps the Lie algebra
  onto $(0)$. This shows that group is unimodular. 
\end{enumerate}

\subsection{Haar measure on a Lie
  group.}\label{partII-chap3-sec3.2}\pageoriginale%sec 3.2  

Let $G$ be a Lie group with a coordinate system $(x_{1},\ldots,x_{n})$
in a neighbourhood of $e$. We now investigate the form of the right
invariant Haar measure on the Lie group. By invariance of a measure
$\mu$ here we mean that $\int f(xy^{-1}) d \mu (x)= \int f(x) d\mu(x)$ for $y$
which are sufficiently near $e$ and for $f$ whose supports are
sufficiently small. We set $d\mu(x)=\lambda (x) dx_1 \wedge
\ldots\wedge dx_n$ and enquire if integration with respect to this
measure is invariant under right translations. For invariance, we
require 
\begin{gather*}
\int f (x') \lambda (x' y) \bigg| \det \frac{\partial \varphi _
  i(x',y)} {\partial x'_j}  \bigg| x'_{1} \wedge \ldots
\wedge dx'_n\\  
=\int f(x) \lambda (x) dx_1 \wedge \ldots\wedge dx_n,\\
\text{or still}\qquad  \lambda(x) = \lambda(xy) J(x,y) \text{~ with~ }
J(x,y) = 
\bigg | \det \frac{\partial \varphi_{i}} {\partial x_j} {(x,y)}\bigg|. 
\end{gather*}

For this it is obviously necessary and sufficient to take
$\lambda(x)=J^{-1}(e,\break x)$. This gives an explicit construction of the
Haar measure in the case of Lie groups. 

\subsection{Measure on homogeneous
  spaces.}\label{partII-chap3-sec3.3}%sec 3.3 

If $G/H$ is the quotient homogeneous space of a locally compact group
$G$ by a closed subgroup $H$, we denote the elements of $G$ by letters
$x, y,\ldots$ those of $H$ by $\xi , \eta, \ldots $ the respective
Haar measure by $dx, dy,\ldots d\xi,\break d \eta \ldots $ and the
respective modular functions by $\Delta , \delta$. Also $\pi$ is the
canonical map $G \rightarrow G/H $. Let $f$ be a continuous function
on $G$ with compact support $K$. Then $f^o(x) =\int_H f(\xi x)d \xi$
is a continuous function on $G$ (as in lemma \ref{partII-chap2-lem1},
Ch. \ref{partII-chap2-sec2.3}) and we have
$f^o(\zeta x) = f^o(x)$ for every $\xi \in H$. Therefore $f^o$ may be
considered as a continuous function on $G/H$. Obviously it has support
$\pi(K)$ which is again compact. 

\begin{proposition}\label{partII-chap3-prop2}%pro 2
 The\pageoriginale map $f\rightarrow f^0$ is a homomorphism (in the sense of
 N. Bourbaki) of $\mathscr{C}_G$ onto $\mathscr{C}_{G/H}$. 
\end{proposition}

We prove this with the help of

\setcounter{lem}{0}
\begin{lem}\label{partII-chap3-lem1}%lem 1
There exists a positive continuous function $f$ on  $G$  such that
for every compact subset $K$ of $G$ the intersection of $HK$ and the
support $S$ of $f$ is compact and such that $\int_H f(\xi x) d\xi=1$
for every $x\in G$. 
\end{lem}

A locally compact group is always paracompact
(Prop. \ref{chap1-prop3}, Ch. \ref{chap1-sec1.2}. part \ref{partI}) 
and using the fact that the canonical map is open and continuous, we
see that G/H is also paracompact. Let $U$ be an open relatively
compact neighbourhood of $e$ in $G$. $\pi(U x)$ is a family of open
subsets covering $G/H$. Let $(V_i)$, $(V'_i)$ be two locally finite open
refinements of this covering such that $\bar{V}_i \subset
V'_i$. Then there exist open relatively compact subsets $(W_i),(W'_i)$  
such that $\overline{W}_i \subset W'_i$ and $\pi(W_i)= V_i$. In other
words these are families of subsets such that each point in $G$ has a
saturated neighbourhood which intersects only a finite number of the
subsets. We can moreover say that for every compact subset $K$ of $G$,
HK intersects only a finite number of $W'_i$. Now, let us define
continuous functions $g_i$ such that $g_i=1$ on $W_i$ and $0$ outside
$W'_{i}$, and set $g=\sum\limits_{i} g_{i}$. This last summation has a
sense as the summation is only over a finite indexing set at each  
point.This is continuous, as every point in $G$ has a neighbourhood in
which $g$ is the sum of a finite number of continuous functions. Let
$S$ be the support of $g$ and $K$ any compact subset of $G$. Then
HK$\cap$ $S$ is the union of a finite number of $\overline{W}_{i}$ and
is hence compact. 


Now let $g^0= \int_{H} g(\xi x) d \xi>0$. 

This\pageoriginale inequality is strict as at each point $x$, $xH$
intersects some 
$W_{i}$. Obviously $f=g/g^{0}$ is a continuous function of $G$ with
$S$ as its support. Trivially, $f^0=1$ and the proof of the lemma is
complete. 

\medskip
\noindent
{\bf Proof of the Proposition \ref{partII-chap3-prop2}:}~
That the map $f \rightarrow f^0$ is continuous from $\mathscr{C}_{G}
\rightarrow \mathscr{E}^0_{G}$ has already been proved (Lemma
\ref{partII-chap2-lem1}, 
Ch. \ref{partII-chap2-sec2.3}) and it is easy to see that this implies
that the map 
$\varphi : f\rightarrow f^0$  of $\mathscr{C}_{G} \rightarrow
\mathscr{C}_{G/H}$ is also continuous. We now exhibit a continuous map
$\psi : \mathscr{C}_{G/H} \rightarrow \mathscr{C}_{G}$ such that
$\varphi \circ \psi$ = Identity. For this one has only to define for
every $g \in \mathscr{C}_{GH}$, $\psi(g)$ to be $\psi(g)(x)= g
(\pi(x)) f(x)$ where $f$ is the function constructed in the
lemma. This has support $H K\cap S$ where $K$ is a compact subset of
$G$ canonical image in $G/H$ is the support of $g$. 
  \begin{align*}
 (\psi(g))^0(x) &= \int_{H} g(\pi(\xi x)) f(\xi x) d \xi\\
 & = g(\pi(x)) \int_{H} f(\xi x) d \xi\\
  & =g(\pi(x))
 \end{align*}
by the construction of $f$. Hence $\varphi(\psi(g)) = g\cdot \psi$ is of
course continuous.  
 
Every measure $\nu$ on $G/H$ gives rise to a measure $\nu^0$ on $G$ in
the following way $\nu^{0} (f) = V(f^{0})$ for every continuous
function $f$ on $G$ with compact support. 
 
\medskip
\noindent
{\bf Corollary to Proposition \ref{partII-chap3-prop2}:}~
 The image of $\mathscr{M}_{G/H}$ under the map $\nu \rightarrow
 \nu^{0}$ is precisely the set of all measure on $G$ which vanish on
 the kernel $\mathscr{N}$ of the map $f \rightarrow f^{0}$ of $\mathscr{C}_{G}
 \rightarrow \mathscr{C}_{G/H}$. 

This is an immediate consequence of the proposition.
 
\begin{proposition}\label{partII-chap3-prop3}%pro
 A measure $\mu$ on $G$ is zero on $\mathscr{N}$ if and only if $d \mu(\xi x) =
 \delta (\xi) d \mu(x)$ for every $\xi \in H$. 
\end{proposition}

By\pageoriginale the above corollary $\mu$ is of the form $\nu^\circ$
where $\nu$ is a measure on $G/H$. 

Hence
\begin{align*}
\int_G f(\xi^{-1} x) d\nu^0 (x)& =\int\limits_{G/H} f^0 (
\xi^{-1} x) d \nu (x) \\ 
&=\int\limits_{G/H}\int\limits_{H} f( \xi^{-1}\eta x) d \eta d \nu (x)\\
&= \int\limits_{G/H} \int\limits_{H} \delta ( \xi ) f (\eta x ) d \eta
d \nu (x)\\ 
&=\int\limits_{G/H}\delta ( \xi )f^0 (x) d \nu (x) =
\int_{G}\delta (\xi) f(x) d\nu^0 (x) 
\end{align*}

It follows that $d\mu (\xi x)=\delta (\xi)  d\mu (x)$.

Conversely let $d\mu (\xi x)=\delta (\xi) d\mu (x)$.

Let $f$, $g$ be any two continuous functions on $G$ with compact
support. Then
\begin{align*}
\mu (g^0 f) &= \int_{G} f(x)d \mu (x) \int_{H} g(\xi x) d \xi\\ 
				&= \iint f( \xi^{-1} x) g (x) d \mu
(\xi^{-1} x) d \xi\\ 
				&=\iint  f(\xi^{-1}x) g (x)
\delta(\xi^{-1}) d\mu (x) d \xi\\ 
				&=\iint  f(\xi x) g (x) \delta (\xi) d
\mu (x) \delta (\xi^{-1})d \xi\\ 
				& \hspace{2cm} \text{(by
  Prop. \ref{partII-chap3-prop1},   Ch. \ref{partII-chap3-sec3.1})}\\ 
				&=\mu (f^\circ g). 
\end{align*}

If $f$ is in $\mathscr{N}$, one can choose $g$ such that $g^0=1$ on the
support of $f$. Then $\mu(f)= \mu (fg^0)=\mu (f^0
g)=0$. Hence $\mu=0$ on $\mathscr{N}$. 

If there exists an invariant measure $\nu$ on $G/H$, then $\nu^0$
must be the Haar measure and conversely if the Haar measure is of the
form $\nu^0$  then $\nu$ is an \textit{invariant} measure on
$G/H$. Hence $\delta (\xi)=\Delta  (\xi)$ is a necessary and
sufficient condition for the existence of a right invariant measure on
$G/H$. 
 
\subsection{Quasi-invariant
  measures.}\label{partII-chap3-sec3.4}\pageoriginale%sec 3.4 

\begin{defi*} %def
Let $\Gamma$ be a transformation group acting on a locally compact
space $E$. We say that a positive measure $\mu$ on $E$ is {\em
  quasi-invariant} by $\Gamma$ if the transform of $\mu$ by 
every $\gamma \in \Gamma$  is {\em equivalent} to $\mu$
in the sense that there exists a positive function $\lambda
(x,\gamma)$ on $E\times \Gamma$ which is bounded on every
compact subset and measurable for each $\gamma$ such that $d\mu
(\gamma,x)=\lambda (x,\gamma) d \mu (x)$. 
\end{defi*}

If under the above conditions $\lambda (x,\gamma)$ is independent of
$x$, the measure is said to be \textit{relatively invariant}. 

\begin{proposition}\label{partII-chap3-prop4}%pro 4
There always exists a quasi-invariant measure on the homogeneous space
$G/H$. 
\end{proposition}

We prove this by making use of

\begin{lem}\label{partII-chap3-lem2}%lem 2
There exists a strictly positive continuous function $\rho$ on $G$
such that $\rho (\xi x) =\delta (\xi)/\Delta  (\xi)\rho(x)$ for
every $x\in G$ and $\xi \in H$. 
\end{lem}

Let $f$ be the function on $G$ constructed in Lemma \ref{partII-chap3-lem1},
Ch. \ref{partII-chap3-sec3.3}. Define 
$\rho (x)=\int_H \left( \dfrac{\delta(\xi)} {\Delta 
(\xi)}\right)^{-1}f(\xi x) d \xi$. 

Then it is an immediate verification to see that $\rho (\eta
x)=\dfrac{\delta (\eta)} {\Delta  (\eta)} \rho(x)$ and that $\rho$
is positive continuous. 

\medskip
\noindent
{\bf Proof of Proposition \ref{partII-chap3-prop4}:}~
Let $\mu$ be the measure $\rho (x)dx$ on $G$, where $\rho$ is the
function of Lemma \ref{partII-chap3-lem2}. Then 
\begin{align*}
d\mu (\xi x)&=\frac{\delta(\xi)}{\Delta(\xi)} \rho (x) \Delta 
(\xi) dx\\ 
&=\delta (\xi) \rho (x) dx=\delta (\xi) d \mu (x).
\end{align*}

By proposition \ref{partII-chap2-prop1},
Ch. \ref{partII-chap2-sec2.4}, there exists a 
measure $\nu$ on $G/H$ 
such that  
$$
d\mu(x)=d \nu^0(x).
$$

Now\pageoriginale
$$
d\mu(xy)=\frac {\rho (xy)}{\rho (x)} d\mu (x)
$$
and hence we get
$$
d \nu (\pi(xy))=\frac{\rho (xy)} {\rho (x)} d \nu (\pi (x)), 
$$ 
$\dfrac{\rho(xy)}{\rho (x)}$ depending only on the coset of $x$
modulo $H$. $\nu$ is therefore quasi-invariant. 

This incidentally gives also the following relation between the Haar
measure on $G$ and the quasi-invariant measure on $G/H$, viz. 
$$
\int_{G} f(x)\rho (x) dx= \int\limits_{G/H} d \nu (\pi (x))
\int\limits_{H} f(\xi x) d \xi.  
$$

If we relax the condition of continuity on $\rho$, then we can assert
that all the quasi invariant measures on $G/H$ can be obtained this
way \cite{key6}. So $\nu$ is relatively invariant if and only if there exists
a positive function $\rho$ on $G$ Such that  $\rho (xy)/\rho(x)=\rho
(y)/\rho (e)$. If we take  $\rho (e)=1$, we have $\rho
(xy)=\rho(x)\cdot \rho (y)$ with $\rho (\xi)=\delta (\xi)/\Delta 
(\xi) $ for every $\xi \in H$. In other words, the one dimensional
representation $\xi \rightarrow \delta (\xi)/ \Delta (\xi)$ of $H$
can be extended globally to a representation of $G$. 

\subsection{Some applications.}\label{partII-chap3-sec3.5}%sec 3.5
       
Let $G$ be the group product of two closed subgroups $A$ and $B$ such
that the map $(a,b)\rightarrow ab$ of $A\times B \rightarrow G$ is a
homeomorphism. Then the homogeneous space $G/A$ is homeomorphic to
$B$. We define a  function on $G$ by setting $\rho(ab)=\delta(a)/
\Delta  (a)$. To this function, there corresponds a quasi-invariant
measure on $G/A$ such that 
$$
\int_G f(ab) \delta (a)/ \Delta  (a)dx=\int_B d \mu (b) \int_A f(ab) d a.
$$

If\pageoriginale $x=ab'$, we have $\rho (xb)/{\rho (x)}=\rho (ab'b)/_{\rho 
  (ab')}=1$ by definition. Hence $d \mu (b)$ is right invariant, and
$d \mu (b)=db$. 

Let $d_r x$, $d_lx$  denote respectively the right and left
Haar measures. Then $\int_{G} f (ab) \delta (a)/_{\Delta  (a)}
d_{r}x= \int d_{r}b \int f (ab)d_{r}a$, or again 
\begin{align*}  
\int_{G}f(x)d_{r}x&=\iint\limits_{A x B} f (ab) \frac{\Delta 
  (a)} {\delta (a)} d_{r} a  d_{r}b\\ 
 &=\iint\limits_{A x B} f(ab) \Delta  (a) d_l a d_{r} b
\end{align*}

Thus we have got the right Haar measure on $G$ in terms of the Left
and right Haar measures on $A$, $B$ respectively and the modular
function on $G$. This dependence on the modular function can be done
away with if we restrict ourselves to unimodular groups.Thus in the
case of a unimodular group $G$, we have the simple formula 
$$
\int_{G}f(x)d_{r}x=\iint\limits_{A x B} f(ab) d_{l} a d_{r} b 
$$ 

Again when $A$ is a normal subgroup, we have $\delta (a) = \Delta 
(a) $ or $A$ and hence  
$$
\int_G f(x) d_r x= \iint\limits_{A x B}f (ab) d_r a d_r b. 
$$
   
\subsection{Convolution of functions}\label{partII-chap3-sec3.6}

\begin{defi*}%def
 A function  $f$ is said to be {\em locally summable} with respect to
 a measure $\mu$ if for every continuous function $\varphi$ with
 compact support, $\varphi f$ is $\mu$-integrable. 
\end{defi*}

This has the property that for compact set $K$, $\chi (K)f$ is
$\mu$-integrable. 

If $f$ is locally summable, the map $\varphi\rightarrow\int \varphi fd
\mu$ is a continuous linear form on $\mathscr{C}_{G}$ and hence
defines a measure denoted by $\mu_{f}$. Let now $f$ be locally
summable with respect to the Haar measure on $G$ and $\nu$
another\pageoriginale 
measure on $G$. We shall assume that $\mu_{f}*\nu$ exists. Then for
every continuous function $g$ with compact Support, we have 
\begin{align*}
\mu_{f}*\nu_{(g)}&= \iint g(xy)f(x) dx d\nu (y)\\
&=\int d \nu (y)\int g(xy) f (x)dx\\
&=\int d \nu (y) \int g (x)f (xy^{-1}) dx
\end{align*}

Now the map $(x,y)\rightarrow (xy^{-1},y)$ obviously preserves the
product measure $dx d \nu(y)$ on $G x G$ because for continuous
functions $u$ with compact support we have 
$$
\iint u (xy^{-1},y )\dxd \nu (y)= \int \int u (x,y) dxd\nu (y).
$$

Hence $\iint f(xy^{-1} g(x)\dxd \nu (y)$ exists and the theorem of
Lebesgue-Fubini can  be applied. It therefore results that $\int
f(xy^{-1})d \nu (y)$ exists for almost every $x$ and $g(x)$ $\int f
(xy^{-1})d \nu (y)$ is integrable, In other words,  $\int f (xy^{-1})d
\nu (y)$ is locally summable. If we denote by $h(x)$,$\int f
(xy^{-1})d\nu\break (y)$, this can be expressed by $\mu_{f}*\nu$ 
=$\mu_{h}$. We can now define the convolution of a measure and a
locally summable function $f$ by putting $h(x)$ = $f*\nu (x)$.One can
similarly define a convolution $\nu * \mu_{f}$  
=$\mu_{k}$ where $k(x)$=$\nu*f(x)$=$\int f(y^{-1}x) \Delta 
(y^{-1})d \nu (y)$. This is unsatisfactory in as much as it is necessary
to choose between left and right before one can identify functions
with measures. Thus the notion of convolution of a function and a
measure is not very useful in groups are not unimodular. 


Let now $f$, $g$ be two locally summable functions on $G$. Then we can
define convolution of $f$ and g such that
$\mu_{f} * \mu_{g} = \mu_{f^*g}$  by setting  
\begin{align*}
f*g(x)&=\int f(xy^{-1})g (y) dy\\
&=\int f(y)g (y^{-1}x) \Delta  (y^{-1})dy
\end{align*} 


But we have in Prop. \ref{partII-chap3-prop1},
Ch. \ref{partII-chap3-sec3.1} that $\Delta  (y^{-1})dy=dy^{-1}$.   

Thus\pageoriginale the convolution of $f$ and $g$ can be
satisfactorily defined even 
if the group $G$ is not unimodular. 


Note that the convolutions of two measures and of a measure and a
function are uniquely defined, whereas the convolution of two
functions is defined  only upto a constant factor, as it depends on
the particular Haar measure we consider. 


If $f$ and $g$ are integrable, $\mu_{f}$, $\mu_{g}$ are bounded and so is
$\mu _{f^*g}$.  
Consequently $f*g$ is also integrable. Thus the map $f\rightarrow
\mu_{f}$ is an imbedding of $L^1$ in $\mathscr{M}^1$ as a closed
subspace. It is linear and one-one and also preserves metric, for, 
$$
|| f ||_{1}  = \int| f(x)|  dx= \int| d \mu_f |\leq||\mu_f||\quad \text{and},
$$
on the other hand, $|| \mu_{f}||  \leq || f ||_{1}$,
trivially. Actually $L^1$ 
is a Banach subalgebra of $\mathscr{M}^1$. In $\mathscr{M}^1$, the
Dirac measure at the unit element acts 
as the unit element of the algebra but $L^1$ does not possess any unit
element, unless the group is discrete. 

\subsection{Convolutions of
  distributions}\label{partII-chap3-sec3.7}%sec 3.7 

We close this chapter with a brief discussion of convolutions of
distributions on a Lie group. 
A detailed account of distributions may be found in Schwartz's
`Th\'eorie des Distributions' and de Rham's  
`Vari\'et\'es diff\'erentiables'.

Let $G$ be Lie group and $\mathcal{D}_G$ the space of indefinitely
differentiable functions on $G$ which have compact support. Let
$\mathcal{D}_K$ be the subset of $\mathcal{D}_G$ consist ring of
functions whose supports are contained in the compact set $K$. One can
provide $\mathcal{D}_{K}$ with the topology of uniform convergence of
each derivative, and $\mathcal{D}_{G}$ with the topology of direct
limit of those on $\mathcal{D}_{K}$. 
 The\pageoriginale topology on $\mathcal{D}_{K}$can be characterised
 by the fact that
$f \rightarrow 0$ on $\mathcal{D}_K$ if for every differential operator
$D$ on  $G$ with  continuous coefficients, $Df\rightarrow 0$
uniformly on $K$. It is enough to consider only the left invariant
differential operators, or still the $\Delta_{\alpha}$ alone
(Ch. \ref{chap2-sec2.4}, Part \ref{partI}). This topology makes of
$\mathcal{D}_{K}$ a \textit{Fr\'echet space}  
(i.e. a locally convex  topological vector space which is metrisable
and complete). 

\begin{defi*}%def
 A distribution on $G$ is a continuous linear form on
 $\mathcal{D}_{G}$. 
\end{defi*}  
 
As in the case of measures, one can define the notion of the support
of a distribution, distributions with compact support, etc. Let $T$, $S$
be two distributions on $G$. If one of them has compact support we
define the convolution product as for measures:$T*S(\varphi)  =\iint
\varphi (xy)dT(x)dS(y)$. Let  $\xi'$ be the space of distributions
with compact support. It is an algebra with convolution as product and
the space of distributions is a module over $\xi'_{G}$. We denote by
$\xi'_{e}$ the space of distributions with support
$=\{e\}$. Let $(x_{1},\ldots x_n)$ be a coordinate system at $e$. Then
$T\in \xi'_e$ implies the existence of $\lambda_\alpha\in C$ such that
$T (\varphi)=\sum \lambda_\alpha \dfrac{\partial^{\alpha}\varphi}
{\partial x^{\alpha}} (e)$, $\lambda_{\alpha}$ being zero except for a
finite number of terms. If $f$ is a locally summable function, we can
identify $f$ with  the distribution $f(x)dx$ and we can define the
notion of convolution $f*T$ of $f$ and distribution $T$ under some
assumptions on $f$ and $T$. But this product even when it is defined,
is not in general a distribution of the form $g(x)dx$; however, if $f$
is \textit{indefinitely differentiable} with compact support (or if $f$
is indefinitely differentiable and $T$ has compact support) $f*T$ is a
distribution of the form $g(x)dx$ where $g(x)$ is an indefinitely
differentiable function. This function is, in virtue of the above
identification, given by $g(x)=f*T(x)=\int f(xy^{-1})dT(y)$. 

%raghu djview 78

If\pageoriginale  $T = \sum\limits_{\alpha} \lambda_\alpha
\dfrac{\partial^\alpha}{\partial  x^\alpha}(e),f * T(x) = \sum\limits_{\alpha}
\lambda_\alpha 
\big\{\dfrac{\partial^\alpha}{\partial y^\alpha} f(xy^{-1})\big\} y = e$. 

The map $f \rightarrow f*T$ is a differential operator which is left
invariant. 

We have already seen that $\epsilon_y *\mu = \sigma_y (\mu)$
(Ch. \ref{partII-chap2-sec2.5}). 
Hence
\begin{align*}
(\sigma_y f) * T & = \epsilon_y * f * T\\
& =\epsilon_y * (f * T)\\
& =\sigma_y (f * T)
\end{align*} 

In other words, $T$ commutes with the left translation. At $x=e$ we
have $f*T(e)=\sum\limits{\alpha} \lambda_\alpha
\big\{\dfrac{\partial^\alpha}{\partial y^\alpha} f(y^{-1})\big\}_ {y = e}$
i.e. every left invariant differential operator is obtained in the above
manner. This leads us to the  

\begin{proposition}\label{partII-chap3-prop5}%pro 5
 The algebra  $\mathcal{U} (G)$ is canonically isomorphic to the
 algebra of distributions with support $\{e\}$ 
with convolution as multiplication.
\end{proposition}
