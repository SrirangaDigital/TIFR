
\chapter{Convolution of measures}\label{partII-chap2}

\setcounter{section}{2}
\setcounter{subsection}{0}
\subsection{Image of a
  measure}\label{partII-chap2-sec2.1}\pageoriginale%sec 2.1 

\begin{defi*}%def
Let $X$, $Y$ be two locally compact topological spaces and $\pi$ a map
$X \rightarrow Y$. Let $\mu$ be a positive  measure on $X$. Then $\Pi
$ is said to be {\em $\mu-$ proper} if for every function $f \in
\mathscr{C}_Y$, $f\circ \pi$ is integrable with respect to $\mu$. The
value $\mu (f \circ \pi)$ depends linearly on $f$ and is therefore a
linear form on $\mathscr{C}_Y$. In other words, $\mu (f\circ \pi)$
defines a positive measure on $Y$, which we denote by $\pi(\mu)$. We
have, by definition, $\int_Y f(y) d  \pi (\mu)(y) = \int_X f \circ \pi (x)
d \mu (x)$. 
\end{defi*}

If $\mu$ is not positive, but is equal to $(\mu_1 - \mu_2) + i(\mu_3
- \mu_4)$, $\mu_1$, $\mu_2$, $\mu_3$, $\mu_4$ positive, and if $\pi$
is $|\mu|$ - proper, we can define the image measure 
$$ 
\pi (\mu)= \pi (\mu_1) - \pi(\mu_2) + i \pi (\mu_3) - i\pi(\mu_4).
$$

\noindent
\textbf{Examples.}
\begin{enumerate}
\renewcommand{\labelenumi}{(\theenumi)}
\item A continuous {\em proper} map of $X \rightarrow Y$ (i.e. a map
  such that inverse image of every compact set is compact) is $\mu$ -
  proper for every $\mu$. 

In fact, $f \in \mathscr{C}_{K} \rightarrow f \circ \pi \in
\mathscr{C}_{\pi-1(K)}$ and $\pi^{-1}(K)$ is compact. 

\item Let $\pi$ be a continuous map $G \rightarrow H$ and let $\mu$
  have compact support $K$. Then $\pi$ is $\mu$-proper; the support of
  $\pi (\mu) \subset \pi (K)$ and is hence compact. 

If $f$ is a continuous function on $H$ with compact support $K$, $f\circ
\pi$ is continuous and hence $\mu$-integrable (Ch. \ref{partII-chap1-sec1.3}). This
shows that $\pi$ is $\mu$-proper and if $f=0$ on $\pi (K)$, then $f
\circ \pi = 0$ on $K$ and therefore $\mu$($f\circ \pi) = 0$,
i.e. support of $\pi(\mu)\subset \pi (K)$. 

\item More\pageoriginale generally, when $\pi$ is continuous and $\mu$
  bounded, 
   $\pi$ is $\mu$-proper. Also $\pi(\mu)$ is bounded and $|| \pi (\mu)
  || \leq || \mu ||$. 
\end{enumerate}

In fact, $f \circ \pi$ is bounded and in view of the remark in
Ch. \ref{partII-chap1-sec1.4}, $f\circ \pi$ is integrable with respect
to $\mu$. Moreover,   
$$
|| \pi (\mu) || = \sup\limits_{g \in \mathscr{C}_Y}  \frac{| \pi \mu (g)
  |}{|| g ||} = \sup\limits_{g \in \mathscr{C}_Y}  \frac{|\mu(g \circ
  \pi)|}{|| g \circ \pi ||} \leq || \mu || 
$$



\subsection{Convolution of two
  measures.}\label{partII-chap2-sec2.2}%sec 2.2 

Let $G$, $H$ be two locally compact topological spaces and $\mu$,
$\nu$ measures on $G$, $H$ respectively. Then there exists one and
only measure  $\lambda$ on $G \times H$ such that if $f$, $g$ be
functions with compact support respectively on $G$, $H$ we have 
$$
\int f(x)  g(y) d \lambda (x,y) = (\int f(x)d\mu (x)) (\int g (y) d
\nu (y)). 
$$
$\lambda$ shall be called the product measure of $\mu$ and $\nu$. 

If $\mu$ and $\nu$ are two  measures on a locally compact group $G$,
we denote the product measure by $\mu \otimes \nu$ and, if the group
operation $\pi : G \times G \rightarrow G$  
defined by $(x,y) \rightarrow xy$ is $\mu \otimes \nu$ - proper, its
image in $G$ by $\mu*\nu$. The latter is said to be the
\textit{convolution product} of $\mu$ and $\nu$. The most general
class of measures for which convolution product can be defined are
those for which $f(xy)$ is integrable with respect to the product
measure for every function $f \in \mathscr{C}_G$. The following cases
are the particular interest to us: 
\begin{enumerate}
\renewcommand{\labelenumi}{(\theenumi)}
\item If $\mu$ and $\nu$ are bounded, the convolution product exists
  and is bounded. 

 This is almost obvious, $\pi$ being continuous and $\mu\otimes\nu$
 bounded (Example 3, Ch. \ref{partII-chap2-sec2.1}).   

\item If\pageoriginale $\mu$ and $\nu$ are measures on $G$ with
  compacts $K$, $K'$ 
  respectively, $\mu * \nu$ exists and has compact support. In fact,
  $\mu \otimes \nu$ has support $\subset K \times K'$. Hence,
  convolution product exists, and has support $\subset KK'$. 

\item If either $\mu$ or $\nu$ has compact support, $\mu * \nu$
  exists. Let $f$ be a continuous function on $G$ with compact 
support $K$ and let $\mu$ have compact support $K'$. Obviously 
$$
\iint f(xy) d \mu (x) d \nu (y) = \int_{K'} d \mu (x)
\int_{KK'^{-1}}f(xy)d \nu (y) 
$$
\end{enumerate}

 
Hence $f(xy)$ is integrable with respect to $\mu \otimes
\nu$. Consequently, $\mu * \nu$ exists. 

 We denote as usual by $\mathscr{M}^1$, $\mathscr{M}^c$,
 $\mathscr{E}^{0}_{G}$ the spaces of bounded 
 measures, the space of measures with compact support and the space of
 all continuous functions on $G$ respectively. Let $\lambda$, $\mu$,
 $\nu$ be three measures on $G$ such that either all three are bounded
 or two of them have compact support. In any case the function  
$(x, y, z) \rightarrow f(xyz)$ is integrable with respect to
 $\lambda \otimes \mu \otimes \nu$ and hence Fubini's theorem can be
 applied. 
\begin{align*} 
\iiint f(xyz) d\lambda (x) d \mu (y) d \nu (z) &= \int d \nu (z)
\int\int f(xyz) d  
\lambda (x) d \mu (y)\\ 
& = \int d \nu (z) \int f (tz) d (\lambda * \mu)(t)\\
& = \iint f(tz) d(\lambda * \mu) (t) d \nu (z)\\
& = \{(\lambda * \mu ) * \nu\} f\\
& = \{\lambda * (\mu * \nu)\} f
\end{align*}
by a similar computation. This shows that $m^1$ with the convolution
product is an associative algebra and that $\mathscr{M}^c$ acts on
$\mathscr{M}$ on both sides and\pageoriginale 
makes it a two-sided module. Moreover, $\mathscr{M}^1$ is actually a
Banach algebra under the usual norm, since we have $|| \mu * \nu ||
\leq || \mu ||~ || \nu ||$. 

\begin{remarks*}
\begin{enumerate}
\renewcommand{\labelenumi}{(\theenumi)}
\item It is good to point out here that the associativity does not
  hold in general. Take,for instance, $R$ to be the locally compact
  group and $\lambda$ the Lebesgue measure. Let $\mu$ be $\epsilon_1 -
  \epsilon_0$ ($\epsilon_a$ begin the Dirac measure at a - see
  Ch. \ref{partII-chap2-sec2.5}   
  and $\nu =
  \varphi(x) dx$ where $\varphi$ is the Heaviside function
  viz. $\varphi=0$ for $x<0$ and =$1$ for $x$ $\geq 1$. Then
  $(\lambda*\mu)*\nu=0$ and $\lambda*(\mu*\nu)=dx$. Again
  $\lambda*(\mu*\nu)$ may exist without $\lambda*\mu$ begin well
  defined. Let $R$ be the locally compact group, 
$\lambda$ and $\mu$ Lebesgue measures on $R$ and $\nu =
  \epsilon_1-\epsilon_0$. Then $\lambda*(\mu*\nu)=0$ but $\lambda*\mu$
  is not defined. However, when $f(xyz)$ is integrable with respect to
  $ \lambda \otimes \mu \otimes\nu$ then the convolution product is
  associative.  

\item The formula for the integration of functions with respect to the
  convolution of two measures is valid also for vector-valued 
function.
Thus we have 
$$
\int_G f(x) d(\mu*\nu) = \iint f(xy) d \mu(x) d \nu(y). 
 $$
\end{enumerate}
\end{remarks*}

\subsection{Continuity of the convolution
  product.}\label{partII-chap2-sec2.3}%sec 2.3 

That the convolution product is continuous in $\mathscr{M}^1$ is trivial in
virtue of our remark that it is a Banach algebra. Regarding the
continuity of the convolution product in the other cases, we have the
following  

\setcounter{lem}{0}
\begin{lem}\label{partII-chap2-lem1}%lem
 Let $f$ be a continuous function and $\mu$ a measure on $G$, one of
 them having compact support. Then {\em(i)} the function $g(x) =\int f
 (xy) d \mu (y)$ is continuous; {\em(ii)} the map $f\rightarrow g$ is a
 continuous linear map of $\mathscr{C}_G$ in\pageoriginale 
$\mathscr{E}^{0}_{G} $ (with the
 usual topologies); {\em(iii)} if $ \mu $ has compact support, then the
 above map $f  \rightarrow \int f (xy) d\mu (y)$ is also continuous
 from $\mathscr{E}_G^\circ \rightarrow\mathscr{E}_G^\circ$ and from $\mathscr{C}_G
 \rightarrow \mathscr{C}_G$. 
\end{lem}

\begin{enumerate}
\renewcommand{\labelenumi}{\roman{enumi}}
\renewcommand{\labelenumi}{(\theenumi)}
\item Let  $H$ and $K$  be two compact subsets of $G$. Then $H \times
  K$ is also compact and $f(xy)$ is uniformly continuous on $H\times
  K$. For every $\epsilon >0$, and for every $x  \in H$, there exists a
  neighbourhood $U$ of $x$  such that $|f(x'y) -f (xy)| < \epsilon$ for
  every $x' \in U \cap H$ and   $y \in   K$. If $f$ has compact support
  $S$, we choose a compact neighbourhood $H$ of $x$ and $K$ such that
  $H S^{-1}\subset K$. If $y \notin K$, then $xy$, $x'y\notin S$. Hence 
$$
\int\limits_{G-K}\bigg\{f(x'y)-f(xy)\bigg\}d\mu (y)=0.
$$

So, we have
$$
|g(x') -g(x)|\leq\int_K \bigg\{f(x'y)-f(xy)\bigg\}d |\mu| (y)
\leqslant \epsilon |\mu|(K).  
$$

This shows that $g$ is continuous. If however $\mu$ has compact
support $C$, we take $K=C$ and the same inequality as above results. 

\item Again, as in (i) if we assume that  $f$ has compact support $S$
  ans $ H S^{-1}\subset K$, it is immediate that 
\begin{align*}
|g(x)| &\leq \int_K | f (xy)| d | \mu| (y) \text{~ for every~ } x \in H\\
&\leq \int_K |f(y) | d|\mu|(x^{-1}y)\\
&\leq \sup |f| |\mu| (H^{-1}K).
\end{align*}

Hence we have $\sup\limits_{x \in H} |g(x)|  \leq \sup  |f||\mu|
(H^{-1}K)$. 

It follows that whenever $f  \rightarrow 0$ on $\mathscr{C} _S$,
$g(x)\rightarrow 0$  uniformly on the compact set $H$. A similar proof
holds when $\mu $ has compact support. 

\item Let\pageoriginale now $C$ be the support of $\mu$, and f has
  compact port $K$; 
  obviously $g \in \mathscr{C}_{KC}-1$. Since the map $\mathscr{C}_G
  \rightarrow \mathscr{E}_G^\circ$ is continuous, so also is the map
  $\mathscr{C}_K \rightarrow \mathscr{C}_{KC^{-1}}$ and by the properly
  of the direct limit topology,
  $\mathscr{C}_G\rightarrow\mathscr{C}_G$ is continuous. An analogous
  proof holds for the other part. 
\end{enumerate}

\subsection{Duality and convolution
  products}\label{partII-chap2-sec2.4}%sec 2.4 

Let $E$ be a locally convex topological vector space an $E'$ its
dual. Then $E'$ can be provided with several interesting topologies
(Bourbaki, Espaces vectoriels topologiques, Chapter
\ref{partII-chap4}). The following 
three are of fundamental importance: 
\begin{enumerate}
\renewcommand{\theenumi}{\roman{enumi}}
\renewcommand{\labelenumi}{(\theenumi)}
\item The \textit{weak topology}, in which $x' \in E' \rightarrow 0$
  if and only if $\langle x',x \rangle  \rightarrow 0$ for every
  $x\in E$. 

\item The \textit{convex compact topology}, in which $x' \in E'
  \rightarrow 0$ if and only if $\langle x',x \rangle \rightarrow 0$
  uniformly on every convex compact subset, and 

\item The \textit{strong topology}, where $x' \in E' \rightarrow 0 $
  if and only if $\langle x',x \rangle \rightarrow 0$ uniformly on
  every bounded set. 
\end{enumerate}

In general, these topologies are distinct. If $E$ is a Banach space,
the usual dual is the $E'$ with the strong topology. However, the convex
compact topology is often the most useful, in as much as it shares the
`good' properties of both the weak and the strong topologies. To
mention but one such, $(E')'=E$ is true for the weak, but not for the
strong, topology. The convex compact topology possesses this
property. We shall almost always restrict ourselves to the
consideration of this topology. 

In\pageoriginale particular, the spaces ${\mathscr{M}}$,
${\mathscr{M}}^c$ being duals of $\mathscr{C}_G$ 
and $\mathscr{E}^\circ_G$ respectively, they can be provided with the convex
compact topology. With reference to the convolution map we have the  

\setcounter{proposition}{0}
\begin{proposition}\label{partII-chap2-prop1}%pro 1
The convolution map $(\mu,v) \rightarrow \mu * \nu$ is continuous in
each variable separately in the following situations: 
$$
\mathscr{M}^c\times \mathscr{M}^c\rightarrow \mathscr{M}^c;
\mathscr{M}^c \times \mathscr{M} \rightarrow \mathscr{M};
\mathscr{M}\times \mathscr{M}^c 
\rightarrow \mathscr{M}. 
$$
\end{proposition}

In fact, let $\mu$ be fixed in $\mathscr{M}^c$ and $\nu \rightarrow 0$ in
$\mathscr{M}$. Then  
\begin{align*}
\mu * \nu (f) & = \iint f(xy) d \mu(x) d \nu(y)\\ 
& =\int d\nu(y) \int f(xy) d \mu(x).
\end{align*}

Denoting by $f_0(y)$ the function $\int f(xy) d \mu(x)$ the map
$f\rightarrow f_0$ is continuous from $\mathscr{C}_G \rightarrow
\mathscr{C}_G$ (Lemma \ref{partII-chap2-lem1},
Ch. \ref{partII-chap2-sec2.3}). The image of 
a convex compact 
subset being again a convex compact subset, $\mu* \nu \rightarrow 0$
uniformly on a convex compact subset. All other assertions in the
proposition can be demonstrated in an exactly similar manner. 

\subsection{Convolution with the Dirac
  measure}\label{partII-chap2-sec2.5}%sec 2.5 

If $x$ is a point of $G$, the Dirac measure $\epsilon_x$ is defined by
$\epsilon_x(f)=f(x)$. This is trivially a measure with compact support. Let
$\nu$ be any arbitrary measure. Then 
\begin{align*}
\epsilon_x * \nu(f) & = \iint f(yz) d \epsilon_x(y) d\nu(z)\\
&=\int f(xz) d \nu(z)\\
&=\nu(\sigma_x  -1 f).
\end{align*}

In a similar manner, $\nu* \epsilon_x (f) = \nu(\tau_x f)$. We may define
left and right translation of a measure by setting $d(\tau_x \nu)(y) =
d \nu(yx)$ and $d(\sigma_x \nu)(y)=d\nu (x^{-1}y)$. It requires a
trivial verification to establish that\pageoriginale $\tau_x\nu
(f)=\nu(\tau_{x^{-1}} f)$ and $\sigma_x\nu(f)=\nu(\sigma_{x^{-1}}
f)$. Hence we have 
$$
\epsilon_x *  \nu  =  \sigma_x \nu ,\quad \text{and}\quad  
\nu * \epsilon_x = \tau_x^{-1} \nu.
$$ 

In particular, $\epsilon_x * \epsilon_y = \sigma_x \epsilon_y =
\epsilon_{xy}$. In other 
words, the map $x\rightarrow \epsilon_x$ is a representation in the
algebraic sense of the group $G$ into the algebra $\mathscr{M}^c$ or
$\mathscr{M}^1$. As 
a matter of fact, this can be proved to be a topological isomorphism
(Bourbaki, Int\'eration, Chapter \ref{partII-chap3}).  




