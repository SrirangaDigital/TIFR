
\chapter{Topological groups}\label{chap1}

\setcounter{section}{1}
\subsection{}\label{chap1-sec1.1}%%% 1.1
We\pageoriginale here assemble some results on topological groups
which we need in the sequel. 

\begin{defi*} %\def 
A {\em topological group} $G$ is a topological space with a
composition law $G \times G \to G$, $(x ,y)\to x y$ which is  
\begin{itemize}
\item[\rm(a)] a group law, and 
\item[\rm(b)] such that the map $G \times G \to G $ defined by $(x ,
  y) \mapsto x^{-1} y$ is continuous. 
\end{itemize}
\end{defi*}

The condition (b) is clearly equivalent to the requirement that the
maps $G \times G \to G$ defined by $(x,y) \to xy$ and $G
\to G$ defined by $x \to x^{-1}$ be continuous. 

It is obvious that the translations to the right: $x \to xy$
are homeomorphisms of $G$. A similar statement is true for left
translations also. We denote by $A^{-1}$, $AB$ the subsets $\{a^{-1}:a
\in A\}$, $\{ab : a \in A, b \in B\}$ respectively. It is immediate from
the definition that the neighbourhoods of the identity element $e$
satisfy the following conditions: 

\begin{description}
\item[$(V_1)$] For every neighbourhood $V$ of $e$, there exists a
  neighbourhood $W$ of $e$ such that $W^{-1} W \subset V$. (This
  follows from the fact that $(x, y) \to x^{-1}y$ is continuous). 

\item[$(V_2)$] For every neighbourhood $V$ of $e$ and for every $y \in
  G$, there exists a neighbourhood $W$ of $e$ such that $y W y^{-1}
  \subset V$. (This is because $x \to yxy^{-1}$ is continuous). 
\end{description}
 
These conditions are also sufficient to determine the topology of the
group. More precisely, 

\begin{proposition}\label{chap1-prop1}%pro 1
Let\pageoriginale $\mathscr{V}$ be a family of subsets of a groups
$G$, such that 
\begin{enumerate}
\item[\rm(a)] For every $V \in \mathscr{V}$, $e \in V$. 

\item[\rm(b)] Any finite intersection of elements of $\mathscr{V}$ is
  still in $\mathscr{V}$. 

\item[\rm(c)] For every $V \in \mathscr{V}$, there exists $W \in
  \mathscr{V}$ such that $W^{-1}W \subset V$. 

\item[\rm(d)] For every $V \in \mathscr{V}$, and for every $y \in G$
  there exists $W \in \mathscr{V}$ such that $y W y^{-1}\subset V$. 
\end{enumerate}
\end{proposition}


Then $G$ can be provided with a unique topology $\Phi$ compatible with
the group structure such that the family $\mathscr{V}$ is a fundamental
system of neighbourhoods at $e$. 

Supposing that such a topology exists, a fundamental system of
neighbourhoods at $y$ is given by either $\mathscr{V} y$ or $y
\mathscr{V}$. It is, therefore, a natural requirement that these two
families generate the same filter. It is this that necessitates the
condition (d). 

We may now take the filter generated by $y \mathscr{V}$ and $\mathscr{V}
y$ as the neighbourhood system at $y$. This can be verified to satisfy
the neighbourhood axioms for a topology. It remains to show that $(xy)
\to x^{-1}y$ is continuous. 
 
Let $V x^{-1}_0 y_0$ be any neighbourhood of $x^{-1}_0 y_0$. By (c),
(d), there exist $W$, $W_1 \in \mathscr{V}$ such that $W^{-1}W \subset V$
and $x^{-1}_0 y_0 W_1 y^{-1}_0 x_0 \subset W$. If we take $x \in x_0
W$, $y \in y_0 W_1$, we have $x^{-1}y \in W^{-1} x^{-1}_0 y_0 W_1
\subset W^{-1}W x^{-1}_0 y_0 \subset V x^{-1}_0 y_0$. This shows that
$(x , y)\to x^{-1}y$ is continuous. 

\bigskip
\noindent{\textbf{Examples of topological groups.}}
\begin{enumerate}
\renewcommand{\labelenumi}{(\theenumi)}
\item Any group $G$ with the discrete topology.

\item The additive group of real numbers $R$ or the multiplicative
  group of non-zero real numbers $R^\ast$ with the `usual topology'. 

\item The\pageoriginale direct product of two topological groups with
  the product topology.
 
\item The general linear group $GL(n, R)$ with the topology induced by
  that of $R^{n^2}$. 

\item Let $X$ be a locally compact topological space, and $G$ a group
  of homeomorphisms of $X$ onto itself. This group $G$ is a topological
  group with the `compact open topology'. (The compact-open topology
  is one in which the fundamental system of neighbourhoods of the
  identity is given by finite intersections of the sets $u(K , U)=\{ f
  \in G : f(x) \in U, f^{-1}(x) \in U$ for every $x \in K\}$, $K$ being
  compact and $U$ an open set containing $K$). 
\end{enumerate}

\subsection{Topological subgroups.}\label{chap1-sec1.2}%subsec 1.2

Let $G$ be a topological group and $g$ a subgroup in the algebraic
sense. $g$ with the induced topology is a topological group which we
shall call a \textit{topological subgroup} of $G$. We see immediately
that the closure $\bar{g}$ is again a subgroup. 

If $g$ is a normal subgroup, so is $\bar{g}$. Moreover, an open
subgroup is also closed. In fact, if $g$ is open, $G-g=
\bigcup\limits_{x \notin g} xg$, which is open as left translations are
homeomorphisms. Hence $g$ is closed. 

Let now $V$ be a neighbourhood of $e$ and $g$ the subgroup generated
by the elements of $V$. $g$ is open containing as it does a
neighbourhood of every element belonging to it. Consequently it is
also closed. So that we have 

\begin{proposition}\label{chap1-prop2}%\pro 2
If $G$ is a connected topological group, any neighbourhood of $e$
generates $G$. 
\end{proposition}

On\pageoriginale the contrary, if $G$ is not connected, the connected
component $G_0$ of $e$ is a closed normal subgroup of $G$. Since $G_0
x G_0 \to 
G^{-1}_0 G_0$ is continuous and $G_0 x G_0$ connected, $G^{-1}_0 G_0$
is also connected, and hence $\subset G_0$. This shows that $G_{0}$ is
a subgroup. As $x \to y x  y^{-1}$ is continuous, $y G_0 y^{-1}$
is a connected set containing 
$e$ and consequently $\subset G_0$. Therefore, $G_0$ is a normal
subgroup. 

\begin{proposition}\label{chap1-prop3}%\pro 3
Every locally compact topological group is paracompact.
\end{proposition}

In fact, let $V$ be a relatively compact open symmetric neighbourhood
of $e$. $G'=\bigcup\limits^{\infty}_{n=1} V^n$ is an open and
hence closed subgroup of $G$. $G'$ is countable at $\infty$ and
therefore paracompact. Since $G$ is the topological union of left
cosets modulo $G'$, $G$ is also paracompact. 

\subsection{Factor groups.}\label{chap1-sec1.3}%subsec 1.3

Let $g$ be a subgroup of a topological group $G$. We shall denote by
$\dot{x}$ the right coset $gx$ containing $x$. On this set, we already
have the quotient topology. Then the canonical map $\pi : G
\to G/g$ is open and continuous. For, if $\pi(U)$ is the image
of an open set $U$ in $G$, it is also the image of $gU$ which is an
open saturated set. Hence $\pi(U)$ is also open. But this canonical
map is not, in general, closed. The space $G/g$ is called a
\textit{homogeneous space}. If $g$ is a normal subgroup, $G/g$ is a
group and is a topological group with the above topology. This is the
\textit{factor group} of $G$ by $g$. 

\subsection{Separation axiom.}\label{chap1-sec1.4}%subsec 1.4

\begin{thm}\label{chap1-thm1}%\the 1
The homogeneous space $G/g$ is Hausdorff if and only if the subgroup
$g$ is closed. 
\end{thm}

If\pageoriginale $G/g$ is Hausdorff, $g= \pi^{-1}(\pi (e))$ is closed, since
$\pi(e)$ is closed. Conversely, let $g$ be closed. Let $x$, $y \in G/g$
and $x \neq y$. Since $xy^{-1}\notin g$ and $g$ is closed, there
exists a symmetric neighbourhood $V$ of the identity such that $x
y^{-1} V \cap g= \phi$.  Hence $x y^{-1}\notin g V$. Now choose a
neighbourhood $W$ of $e$ such that $WW^{-1} \subset y^{-1} Vy$. We
assert that $g x W$ and $g y W$ are disjoint. For, if they were
not, $\gamma_1$, $\gamma_2 \in g$, $w_1$, $w_2 \in W$ exist such that 
$$ 
\gamma_1 x w_1 = \gamma_2 y w_2; \text{~ i.e.~  } \gamma^{-1}_2
\gamma_1 x= y w_2 w^{-1}_1 \in y W W^{-1} \subset V y. 
$$

Hence $\gamma^{-1}_2 \gamma_1 x y^{-1} \in V$,  or $x
y^{-1}\in \gamma^{-1}_1 \gamma_2 V \subset g V$. 

This being contradictory to the choice of $V$, $g x W$, $g y W$ are
disjoint or $\pi (g x W)\cap \pi (g y W)= \phi$. Hence $G/g$ is
Hausdorff. 

In particular, if $g=\{e\}$, $G$ is Hausdorff if and only if $\{e\}$ is
closed. On the other hand, if $\{e\}$ is not closed, $\overline{\{e\}}$
is a normal subgroup and $G/\overline{\{e\}}$ is a Hausdorff
topological group. We shall hereafter restrict ourselves to the
consideration of groups which satisfy Hausdorff's axiom. 

\subsection{Representations and
  homomorphisms.}\label{chap1-sec1.5}%subsec 1.5 

\begin{defi*} %def
A {\em representation} $h$ of a topological group $G$ into a
topological group $H$ is a continuous map : $G \to H$ which is an
algebraic representation. In other words, $h(xy)= h(x)\cdot h(y)$ for
every $x$, $y \in G$. 
\end{defi*}

Obviously the image of $G$ by $h$ is a subgroup of $H$ and the kernel
$N$ of $h$ is a closed normal subgroup of $G$. The canonical map
$\bar{\{h\}}:G/N \to H$ is a representation and is one-one. 

\begin{defi*}%\def
A representation $h$ is said to be a {\em homomorphism} if the induced
map $\bar{h}$ is a homeomorphism.
\end{defi*}

\begin{proposition}\label{chap1-prop4}% pro 4
Let\pageoriginale $G$ and $H$ be two locally compact groups, the former being
 countable at $\infty$. Then every representation $h$ of $G$ {\em
   onto} $H$ is a homomorphism. 
\end{proposition}

It is enough to show that for every neighbourhood $V$ of $e$ in $G/N$,
$\bar{h}(V)$ is a neighbourhood of $h(e)$ in $H$. Choose a
relatively compact open neighbourhood $W$ of $e$ such that
$\bar{W} \bar{W}^{-1}\subset V$. $G$ is a countable union of
compact sets and $\bigcup_{x \in G} W x = G$. Therefore, one
can find a sequence $\{x_j\}$ of points such that $G =
\bigcup_{j}  W x_j$. Since $h$ is onto, $H = \bigcup_{j}
h (W x_j)= \bigcup_{j} h(W)h(x_j)$. $H$ is a locally compact
space and hence a Baire space (Bourbaki, Topologie g\'en\'erale,
Ch.~9). There exists, therefore, an integer $j$ such that
$\overline{h(Wx_j)}$ has an interior point. $h(\bar{W})$ being
compact, $h(\bar{W})=\overline{h(W)}$. Consequently,
$h(\bar{W})h(x_j)$ and hence $h(\overline{W})$ has an interior
point $y$. There exists a neighbourhood $U$ of $e$ such that
$h(\bar{W})\supset Uy$. Now $h(V)\supset h(\bar{W}
\bar{W}^{-1}) = h(\bar{(W)}\cdot h \bar{(W)}^{-1}
\supset Uy(y^{-1}U^{-1})=UU^{-1}$. $h(V)$ is therefore a neighbourhood
of $h(e)$, which completes the proof of proposition \ref{chap1-prop4}. 
