
 \chapter{Relation between Lie groups and Lie algebras -
   II}\label{chap4}%\chap 4  

\setcounter{section}{4}
\setcounter{subsection}{0} 
\subsection{The enveloping
  algebras.}\label{chap4-sec4.1}\pageoriginale%\sec 4.1  

Let $\mathfrak{g}$ be a Lie algebra, and $T$ the tensor algebra of the
underlying vector space of $\mathfrak{g}$. Consider the two-sided ideal
$I$ generated in $T$  By the elements of the form  $x\otimes y-
 y \otimes x -[x,y]$. Then the associative algebra $T/I$ is said to
be the \textit{universal enveloping algebra} of the Lie algebra
$\mathfrak{g}$. 

\begin{defi*}% \def
A linear map $h$ of al Lie algebra $\mathfrak{g}$ into as associative
algebra $A$ is said to be a {\em linearisation} if
$h([x,y])=h(x)h(y)-h(y)h(x)$ for every $x$, $y \in \mathfrak{g}$. 
\end{defi*}

We have obviously a canonical map of $\mathfrak{g}$ in to $T/I$, which we
shall denote by $j$. 

\setcounter{proposition}{0}
\begin{proposition}\label{chap4-prop1}% \pro 1
 To any linearisation $f$ of $\mathfrak{g}$ in an associative algebra
 $A$, there corresponds one and only one representation
 $\bar{f}$ of $T/I$ in such that $\bar{f}\circ j=f$. 
\end{proposition}

In fact, $f$ being a linear map, it can be lifted uniquely into a
representation $\hat{f}$ of the tensor algebra  $T$ in $A$. Obviously,
the kernel of $\hat{f}$ contains elements of the form $x \otimes y - y
\otimes x - [x , y]$ and hence contains $I$. Hence this gives rise to
a map $\bar{f}$ with the required property.  

\subsection{The Birkhoff-Witt theorem.}\label{chap4-sec4.2}% sec 4.2

With reference to the universal enveloping algebra of a Lie algebra of
a Lie group, we have the following 

\setcounter{thm}{0}
\begin{thm}\label{chap4-thm1}%\the 1
 {The\pageoriginale algebra of left invariant differential operators}
 $\mathcal{U}$ 
 {is canonically isomorphic to the universal enveloping algebra
   of the Lie algebra.} 
\end{thm}

In fact,  the inclusion map of the Lie algebra $\mathfrak{g}$ into
$\mathcal{U}$ can be lifted to a representation of the enveloping
algebra $\mathcal{U}'$ of $\mathfrak{g}$ in $\mathcal{U}$ by
proposition \ref{chap4-prop1}. It only remains to show that this map $h:~~
\mathcal{U}' \rightarrow \mathcal{U}$ is an isomorphic. If $S_\alpha$
is the coefficient of $t^\alpha$ in the expansion of
$(\sum\limits^{n}_{i=1}t_i X_i)^{|\alpha|}$, it has been proved
(Th. \ref{chap3-thm3}, Ch. \ref{chap3-sec3.4}) that $S_\alpha$ form a
basis for 
$\mathcal{U}$. $S_\alpha$ is an operators of the form
$\sum(\ldots)X_{i_1}\cdots X_{i_{|\alpha|}}$, where $X_{i_1}\cdots
X_{i_{|\alpha|}}$ are obtained by certain permutations of
$x^{\alpha_i}_{1}\cdots X^{\alpha_n}_{n}$. Let $S'_\alpha$ denote the
element of the form $\sum(\ldots)X_{i_1}\otimes \ldots \otimes
X_{i_{|\alpha|}}$, where $S_\alpha = \sum (\ldots) X_{i_1}\ldots
x_{i_{|\alpha|}}$. By definition of $h$, we  have $h(S'_\alpha)=S_\alpha$. To
prove that $h$ is an isomorphism, it is therefore sufficient to show
that the $S'_{\alpha}$ generate $\mathcal{U}'$. We shall do this
showing that the $ S'_\alpha $  for $|\alpha|\le r$ generate  the
space $T_r$ of tensors of order $\le r$ modulo $I$. The statement being
trivially true for $r=0$, we shall assume it verified for ($r-1)$ and
prove it for $r$. Again,  it is enough to prove that $S'_\alpha$ for
$|\alpha| = r$ generate the space $T'_r$ of tensors of order = $r$,
modulo $I+T_{r-1}$. Let $X_{i_1}\otimes \cdots \otimes X_{i_r}$ be an
element $\in T'_r$. Then 
$$
X_{i_1}\otimes X_{i_2} \otimes \ldots \otimes  X_{i_r}\equiv X_{i_2}
\otimes X_{i_1}\otimes X_{i_3} \otimes \ldots  +
\big[X_{i_1},X_{i_2}\big] \otimes X_{i_3}\otimes \ldots \mod  I. 
$$

Hence, if $\sigma$ is a permutation of $(1,2,\ldots,r)$,
$$  
X_{i_1}\otimes X_{i_2}\otimes \ldots \otimes X_{i_r} \equiv 
X_{i_{\sigma(1)}} \ldots X_{i_{\sigma(r)}} \mod (T_{r-1} + I) 
$$
by  successive transpositions. Now,
$S'_\alpha=\sum_{\sigma}\mathcal{U}_{\sigma} X_{i_\sigma(1)}\ldots
x_{i_\sigma(r)}$, where\pageoriginale $\mathcal{U}_\sigma$ are
positive integers. Therefore 
$$
(\sum\limits_{\sigma} \mathcal{U}_\sigma)X_{i_{1}} \otimes \ldots
\otimes X_{i_r}\equiv S'_\alpha \mod (T_{r-1} + I ). 
$$

Since 
$$
(\sum\limits_\sigma \mathcal{U}_\sigma) \ne 0, X_{i_1} \otimes \ldots
\otimes X_{i_r} \equiv k S_\alpha \text{ mod } (T_{r-1}+I), 
$$
and hence the theorem is completely proved.

Incidentally we have proved that the Lie algebra $\mathfrak{g}$ can be
embedded in its universal enveloping algebra by the natural map
$h$. This is known as the Birkhoff-Witt theorem,  and it true in the
more general case when the Lie algebra is over a principal ideal
ring. 

\subsection{Group law in terms of structural
  constants.}\label{chap4-sec4.3}%\sec 4.3  

We now show that the Lie algebra of a Lie group completely
characterises the group locally.  In other words, the group laws of the
Lie group  can be expressed in terms of the \textit{structural
  constants} of its Lie algebra. (If $(X-\alpha)_{\alpha \in A}$ be a
basis of the Lie algebra,  and $[X_i,X_j]=\sum_{k}
C^k_{i,j}X_k,C^k_{i,j}$  are called the structural constants of the
Lie algebra). 

\begin{thm}\label{chap4-thm2}%\the 2
{Lie groups having isomorphic Lie algebras are locally isomorphic. If
  they are connected and simply connected, they are isomorphic}. 
\end{thm}

Choose a basis $X_1, \ldots , X_n$ of the Lie algebra
$\mathfrak{g}$. If $\theta_i(x)$ be the $i^{\text{th}}$ coordinates of $x$ in
the canonical of coordinates with respect  to the above basis  we
have. 
\begin{align*}
\varphi_i(x,y)& =\theta_i(xy)=\tau_y \theta_i(x)= \sum \frac{1}{\alpha
  !}y^\alpha\Delta_\alpha\theta_i(x)\\ 
\text{But~~ } \Delta_\alpha\theta_i(x) &=
\tau_x(\Delta_\alpha\theta_i)(e)\\ 
&=\sum\limits_{\beta} \frac{1}{\beta !}x^\beta \Delta_\beta
(\Delta_\alpha\theta_i)(e). 
\end{align*}

Hence\pageoriginale     
$$
\varphi_i(x,y)  = \sum\limits_{\alpha ,\beta} \dfrac{1}{\alpha !}
\dfrac {1}{\beta !} y^\alpha x^\beta (\Delta _\beta \Delta _\alpha
\beta_i ) (e). 
$$

If 
$$ 
\Delta_\beta \Delta_\alpha  = \sum_\gamma d^\gamma_{\beta,\alpha}
\Delta_\gamma = d_{\beta,\alpha}^\gamma
$$ 
are completely known, once the Lie algebra $\mathfrak{g}$ is given, because
$\Delta_\alpha =\frac{\alpha!}{|\alpha|!}S_\alpha$. Since
\begin{align*}
\Delta_r \theta_i (e) & =(\dfrac{\partial^r}{\partial x^r xi})_{x=e}  =
      {^{1  \text{ if } r= [i]}_{0 \text{ if } r \neq [i]}}\\ 
\text{we have~~ } \varphi_i(x,y)&  = \sum_{\alpha, \beta}
\dfrac{d^{[i]}_{\beta , \alpha}}{\alpha! \beta!}x^\beta
y^\alpha.
\end{align*}

Thus the group law is completely determined by the constants $ d
^{[i]}  _{\beta, \alpha}$.  
If two Lie groups have isomorphic Lie algebras, the constants $
d^{[i]}_{\beta, \alpha}$ are the same for both, and the group
operation is given  locally by the  above formula, which is to say the
groups are locally isomorphic. By Lemma \ref{chap3-lem1},
Ch. \ref{chap3-sec3.3}, if the groups are 
connected and simply connected, they are isomorphic. 

We can compute the constants $d^{[i]}_{\beta, \alpha}$  in terms
of the structural constants of the Lie  algebra and obtain a universal
formula (i.e. a formula which is the same for all Lie groups - the
Campbell--Hausdorff formula). For instance, if $\delta$ is a multi-index
of order 2 with 1 in the $j^{\text{th}}$ and $k^{\text{th}}$ indices and  $0$
elsewhere, it can easily be seen that $\Delta_\delta
=\frac{1}{2}S_\delta =\frac{1}{2}(X_j X_k+ X_k X_j)$  and if
$C^i_{j,k}$ are the structural constant of the Lie algebra,  
\begin{align*}
X_j X_k & = \dfrac{1}{2}\big[ X_j, X_k\big]+ \dfrac{1}{2}(X_j X_k +
X_k X_j)\\ 
& = \dfrac{1}{2}\sum_{i} C^i_{j,k} X_i+\dfrac{1}{2}(X_j X_k +
X_k X_j)\\ 
\text{and hence~ } d^{[i]}_{[j],[k]}& = \dfrac{1}{2} C^i _{j,k}.
\end{align*}

We have therefore
$$
\varphi_i (x,y) = x_i + y_i + \dfrac{1}{2} \sum C^i_{j,k} x_j x_k +
\text{ terms of order }  \ge 3.  
 $$

Again,\pageoriginale if $x = \exp X$, $y = \exp Y$ ($x_{i}$, $y_{i}$
begin canonical coordinates) and $xy = \exp Z$, we have 
$$
Z = X + Y + \dfrac{1}{2}[X, Y] + \text{ terms of order } \ge 3.
$$ 

\subsection{}\label{chap4-sec4.4}% 4.4

We have proved (Prop. \ref{chap3-prop4}, Ch. \ref{chap3-sec3.2}) that
the Lie algebra of a Lie 
subgroup can be identified with a subalgebra of the Lie algebra. We
now establish the converse by proving the following 

\begin{thm}\label{chap4-thm3}%\the 3
{To every subalgebra} $\mathscr{J}$ {of a Lie algebra} $\mathfrak{g}$
{of a Lie group} $G$, {there corresponds one and only one
  connected Lie subgroup} $H$, {having it for its Lie algebra}. 
\end{thm}

Let $X_1, \ldots , X_n$  be a basis of the Lie algebra $\mathfrak{g}$
such that $X_{1} ,\ldots,   X_{r}$ is a basis of $\mathscr{J}$. Let
$\vartheta$ be the subalgebra generated by $X_i$ with $i \le r$ in
$\mathscr{U}(G)$. We assert that the subspace $\vartheta$ of
$\mathscr{U}(G)$ generated by $S_{\alpha}$ with $\alpha = (\alpha_{1}
,\ldots$, $\alpha_{n} \circ \cdots)$ is the same as $\vartheta$. By
definition of $S_{\alpha}$, it is evident that $\mathscr{V}' \subset
\mathscr{V}$.  It is enough to show that elements of the form
$X_{i_1} \cdots  X_{i_s} \in \mathscr{U}'$ if $1 \le i_{k} \le
r$. We prove this by induction on the length of the product. Now, 
$$
 X_{i_1} X_{i_2} \ldots = X_{i_2} X_{i_1} \ldots  + \big[
   X_{i_1},  X_{i_2}\big]  X_{i_3} \ldots
$$
and since $\big[X_{i_1}, X_{i_2}\big] \in \vartheta$,
 $\mathscr{J}$ being a subalgebra, we have, by induction assumption 
$$
X_{i_1} X_{i_2} \ldots \equiv X_{i_2} X_{i_1} \ldots \mod
\vartheta'. 
$$ 

If $\sigma$ is any permutation of $(1,2, \ldots, s)$, we have 
$$
X_{i_1} \ldots X_{i_s} \equiv X_{i_{\sigma(1)}} \ldots
x_{i_{\sigma(s)}} (\mod \vartheta').
$$

It follows (as in Th. \ref{chap4-thm1}), that $\vartheta = \vartheta'$.

Now,\pageoriginale let $U$ be a symmetric neighbourhood of $e$ in which
the system of 
canonical coordinates with respect to $X_1,\ldots, X_n$ is valid. Let
$N$ denote the subset of $U$ consisting of points for  which
$x_{r+1}=\cdots=x_n=0$. $N$ is obviously a closed submanifold of $U$.
Let $x$, $y \in N$ sufficiently near $e$. 
$$
(xy)_i= \sum_{\alpha,\beta}
\dfrac{1}{\alpha!}\dfrac{1}{\beta !} x^\beta y^\alpha
d^{[i]}_{\beta,\alpha} 
$$

We now show that $(xy)_i=0$ for $i > r$. In the summation, unless both
$\alpha$ and $\beta$ are off the form $(\alpha_1,\ldots,\alpha_r, 0
\ldots)x^\beta y^\alpha = 0$. If both $\alpha$ and $\beta$ are of the
above form, $\Delta_\alpha, \Delta_\beta \in \vartheta$ and
$\vartheta$ being generated by $\Delta_\gamma, \gamma$ of the same
form $d^{[i]}_{\beta,\alpha}=0$ for $i>r$. Hence $(xy)_i=0 $ for
$i>r$. Thus, $x, y \in N \Rightarrow xy \in N$ and $x \in \Rightarrow
x^{-1} \in N$ for $x,y$ sufficiently near $e$.  

Finally, let $H$ be the subgroup algebraically generated by the 
connected component $N^1$ of $e$ in $N$. Then $H$ can be provided with
an analytic structure such that the map $H \rightarrow G$ is everywhere
regular. We define neighbourhoods of $e$ in $H$ by intersecting
neighbourhood of $e$ in $G$ with $N^1$. This system can easily be seen
to satisfy the neighbourhood axioms for a topological group (Prop \ref{chap1-prop1},
Ch. \ref{chap1-sec1.1}). For every $x \in H$, the neighbourhood $x N^1$ of $x$ 
can be provided with an analytic structure induced by that of $G$,
since $x N$ is a closed submanifold of $x U$. For $x,y \in H,$ these
analytic structure agree an $x N^1\cap yN^1$ because those of $x U$
and $y U$ agree on $x U\cap y U$. $H$ of course has $\mathscr{J}$ as
its Lie algebra.  

We now prove the uniqueness of such a group. Let $H^1$ be another
connected Lie subgroup having $\mathscr{J}$ for its Lie algebra. Exp
$\mathscr{J}$ is open in $H^1$, as\pageoriginale 
the map $h \rightarrow \exp  h$ is
open (Th. \ref{chap3-thm2}, Ch. \ref{chap3-sec3.4}). But $\exp
\mathscr{J}\subset  H$.   
Hence $H$ is open in $H^{1}$. As $H$ is open, it is also closed
(Ch. \ref{chap1-sec1.2}) and therefore =$H^{1}$. This completes the demonstration of
Theorem \ref{chap4-thm3}. 

\begin{remark*}%\rema
We have  incidentally proved that if a Lie subgroup has $\mathscr{J}$
for its  Lie algebra, it contains $H$ as an open subgroup. 
\end{remark*}

It has already been proved (Prop. \ref{chap3-prop1},
Ch. \ref{chap3-sec3.1}) that if $f : G 
\rightarrow H$ is a representation of Lie groups, there exists a
representation $df:  
\mathfrak{g} \rightarrow \mathscr{J}$ of Lie algebras. Now, we
establish the converse in the form of a   

\begin{coro*}%\coro
 Let $G$ and $H$ be two Lie groups having $\mathfrak{g}$ and
 $\mathscr{J}$ as their Lie algebras. If $G$ is  connected and simply
 connected, to every representation $\pi$ of $\mathfrak{g}$ in
 $\mathscr{J}$, there corresponds one and only one representation $f$
 of $G \rightarrow H$ such that $df = \pi$. 
\end{coro*}

If there exists one such representation, by Prop. \ref{chap3-prop5},
Ch. \ref{chap3-sec3.4}, it 
is unique. We shall now prove the existence of such an $f$. 

We first remark that if $f$ is a representation of $G$ in $H$, $K$ the
graph of $f$ in $G x H viz$. the set $\{(x,f(x)),x \in G\}$, and
$\lambda$ the restriction to $K$ of the projection of $G x H
\rightarrow G$, then $\lambda$ is an analytic isomorphism.  
Conversely, to every subgroup of $G x H$ the first projection
from which is an isomorphism to $G$, there corresponds one and only
one representation of $G$ in $H$. $\mathfrak{g} x \mathscr{J}$ is
evidently the Lie algebra of $G x H$. 

Now, Let $\pi$ be a representation of $\mathfrak{g}$ in
$\mathscr{J}$. Let $\mathcal{K}$ be the subset $\{(x,\pi (x)), x \in
\mathfrak{g}\}$ of $\mathfrak{g}\times \mathscr{J}$. It can easily be
seen that $\mathcal{K}$ is a subalgebra. Then there exists (Th. \ref{chap4-thm3}) a
connected Lie subgroup $K$ of $G x H$ whose Lie algebra is
isomorphic to $\mathcal{K}$. Let $\lambda$ be the  restriction
to\pageoriginale $K$
of the projection of $G \times H \rightarrow G$. Then $d \lambda$ is
the map $\mathcal{K}\rightarrow \mathfrak{g}$  defined by $d \lambda
(x, \pi (x))= x$. Obviously $d \pi $ is an isomorphism. Hence
$\lambda$ is a local isomorphism of $K$ in $G$. $K$ is therefore  a
covering space of $G$ and $G$ being simply connected, $\lambda$ is
actually an isomorphism. To this there corresponds (by our remark
above)  a representation $f$ of $G$ in $H$,the graph of whose
differential is $\mathcal{K}$, i.e. $df = \pi$. 

\subsection{}\label{chap4-sec4.5} %\sec 4.5

\begin{thm}[E. Cartan]\label{chap4-thm4}%\theorem 4
Every closed subgroup of a Lie group is a Lie subgroup.
\end{thm}

For proving this, we require the following

\setcounter{lem}{0}
\begin{lem}\label{chap4-lem1}%lemma 1
 Let $G$ be a Lie group with Lie algebra $\mathfrak{g}$. Let
 $\mathfrak{g}$ be the direct sum of vector subspaces  $\wp,
 \delta$. Then the map $f: (A,B) \rightarrow \exp A \exp B$ of
 $\mathfrak{g}\rightarrow G$ is a local isomorphism. 
 \end{lem}


It is obvious that $f$ is an  analytic map. To prove that it is  a
local isomorphism, it is enough to show that the Jacobian of the
$\map \neq 0$ in a neighbourhood of $(0,0)$. Let $(X_{1}, \ldots, X_{r})$
be a basis of $\mathcal{U}$ and $X_{r+1},\ldots,X_{n},$ a basis of
$\delta$. 
 Let $(y_{1},\ldots,y_{n})$ be the canonical coordinate system with
 respect to $(X_{1}\ldots,X_{n})$ and $y_{i} \circ  = f_{i}$. Then,  if
 $ X = \sum\limits^{n}_{i=1}  y_i  X_i$, 
\begin{align*}
f\{X\} & = \text{ exp } ( \sum\limits^{r}_{i=1} y_i X_i ) \text{ exp }
( \sum\limits^{n}_{j=r+1} y_j X_j )\\ 
 \frac{\partial f_k}{\partial y_l}(0) & = \dfrac{d}{dt} (\text{ exp }~
 t ~X_l)_k~(t=0)=(X_l.x_k)_{x=e}=\delta_{k,l}.
\end{align*}

Hence Jacobian $\neq 0$ at $(0,0)$ and by continuity, $\neq 0$ in a
neighbourhood of the origin.   This completes the proof of the lemma. 

Let\pageoriginale $H$ be a closed subgroup of a Lie group $G$. We
first construct a 
subgroup $\mathscr{J}$ of $\mathscr{G}$ and prove that the Lie subgroup
$H^1$ of $G$  with $\mathscr{J}$ as its Lie algebra is
relatively open in $H$. Then $H$  is the topological union of cosets
of $H$ modulo $H^1$ and is hence a closed submanifold of $G$. 
  
Let $\mathscr{J}$ be the set $\{ X \in \mathfrak{g}:$ exp $t X \in$ $H$
for every $t \in R\}$. We assert that $\mathscr{J}$ is a Lie
subalgebra of $\mathfrak{g}$. To prove this, we have to verify  
\begin{enumerate}
\renewcommand{\theenumi}{\roman{enumi}}
\renewcommand{\labelenumi}{(\theenumi)}
\item  $ X \in \mathscr{J}  \Rightarrow \alpha X \in \mathscr{J}$ for
  every $\alpha \in R$ 

\item  $X,Y \in  \mathscr{J} \Rightarrow X + Y \in \mathscr{J}$ 

\item $X, Y \in \mathscr{J} \Rightarrow [X, Y] \in \mathscr{J}$.

(a) is a trivial consequence of the definition.

(b)  Let now $ X,Y \in \mathscr{J}$. We have seen that (Ch. \ref{chap4-sec4.3})
\begin{align*}
  \exp  X \exp  Y  & = \exp  (X+Y
  +\dfrac{1}{2}[X,Y]+ \cdots)\\ 
\text{Hence }  (\exp \dfrac{t X}{n} \exp
\dfrac{tY}{n})^n & =(\exp  \{t\frac
      {X+Y}{n}+\frac{1}{2}\frac{t^2}{n^2}[X,Y]+0(\frac{1}{n^2}\})^n
      \qquad \\ 
&  = \exp \big\{t(X+Y)+\frac{t^2}{2n}[X,Y]+0(\frac{1}{n}).\big\}
\end{align*}


But  $\exp \dfrac{tX}{n}$, $\exp \dfrac{tY}{n}\in  H$ and $H$ is a
subgroup. Therefore $(\exp \dfrac{tX}{n}\break\exp \dfrac{tY}{n})^{n}{\in
  H}$ and since $H$ is closed, $\lim\limits_{n \rightarrow \infty}
(\exp \dfrac{tX}{n} \exp \dfrac{tY}{n})^n=\break\exp t(X + Y)$ (by the above
formula) $\in H$. Hence $X + Y \in \mathscr{J}$. 
 

\item As before, 
$$X,Y \in \mathscr{J} \Rightarrow \lim\limits_{n
  \rightarrow \infty} (\exp \dfrac{tX}{n} \exp \dfrac{tY}{n}\exp
  \dfrac{-tY}{n} \exp \dfrac{-tY}{n})^{n^{2}}\in H.$$
The right hand side in this case tends to exp $t^2[X,Y]$ as $n
\rightarrow \infty$.  

Hence $\exp t [X,Y] \in H$  for positive values of $t$, and since
$\exp(-t\break[X,Y])= (\exp t [X,Y])^{-1}$ for all values of $t$, i.e. $[x,y]
\in \mathscr{J}$. 
\end{enumerate}

Let\pageoriginale  $K$ be the connected Lie subgroup of $G$ having
$\mathscr{J}$ for 
its   Lie algebra (Th. \ref{chap4-thm3},  Ch. \ref{chap4-sec4.4}). We
now show that  $K$ is open 
in $H$.  It is obviously sufficient to prove that $K$ contains a
neighbourhood of $e$ in $H$. If $\mathcal{U}$ is a vector subspace of
$\mathfrak{g}$ supplementary to $\mathscr{J}$, by Lemma
\ref{chap4-lem1}, there exists a neighbourhood $V$ of $0$ in
$\mathfrak{g}$ such that the map 
$\lambda :(X, A)\rightarrow \exp  X \exp  A$, $ X \in
\mathscr{J}$, $A \in \mathcal{U} $ is an isomorphism of $V$ onto 
$\lambda(V)=W$. Suppose that $K$ does not contain any neighborhood of
$e$ in $H$. Then, we  can find a sequence of points $a_n \in H \cap W$
which are not in $K$ and which tend to $e$. There is no loss of
generality in assuming $a_n $ to be of the form exp $ A_n ,A_n \in
\mathcal{U} \cap V$, $A_n \ne 0$. For, If $a_n = \exp X_n  \exp A_n$,
then $(\exp X_n)^{-1} a_n = \exp A_n \notin K$. Let $ V^1$ be a compact
neighborhood of $0$ in  $\mathfrak{g} \subset V/2$. For  sufficiently
large $n$, $A_n \in V^1$. Let $r_n$  be the largest integer for which
$r_n A_n \in V^1$. i.e. $(r_n+1) A_n \notin V^1$. 

But  
$$
(r_n+1)A_n = r_n(A_n) +A_n \in V/2 + V/2=V\cdot a_n^{r_n} \in W^1=
\lambda(V^1) 
$$
and $a_n^{r_n+1} \notin W^1 $ but $\in W$. Since $W^1$ is compact,
we may assume (by taking a suitable subsequence) that $a_n ^{r_n}$
converges to an $ a \in W^1$. Now, we  assert that $ a \ne e$. For, if
$a =e, a_n^{r_{n^{+1}}}=a_n \quad ^{r_n}a_n \rightarrow e$. But
$a_n^{r_n}$ cannot tend to $e$. Hence $a \ne e \in
W^1$. Therefore $a = \lim a_n r_n = \exp A$ with $A \ne 0 $ and $A
\in \mathcal{U} \cap V^1$. 

We shall now show that $ A \in \mathscr{J}$, which will imply that
$\mathscr{J} \cap \mathcal{U} \ne (0)$ and hence will give the
contradiction we were seeking. It is enough to show that $\exp p/q  A
\in H$  for every rational number $p/q$. Now let $pr_n/q= s_n +
t_n/q,s_n $ an integer and $0 \le t_n \le q$. 
\begin{align*}
\exp \dfrac{pA}{q} & =\lim\limits_{n \rightarrow \infty}
\exp (\dfrac{pr_n }{q} A_n)\\ 
&=\lim\limits_{n \rightarrow \infty} \exp s_n A_n
\exp(\dfrac{t_nA_n}{q}) 
\end{align*}

Now,\pageoriginale $\exp  \dfrac{t_n}{q}  A_n \rightarrow  e$ as $n
\rightarrow \infty$,  and  $\lim\limits_{n \rightarrow \infty} \exp
s_n A_n = \lim\limits_{n \rightarrow \infty}  a_n {^{s_n}}\in H$ as
$H$ is closed. Hence $A \in \mathscr{J}$, and Theorem \ref{chap4-thm4}
is completely proved. 
  
\begin{remark*}%\rema
 The theorem is not true in the case of complex Lie groups. For
 instance, the space of real numbers is a closed subgroup of the
 complex plane, but is not a complex Lie group. 
\end{remark*}

\begin{corollary}\label{chap4-cor1}%\coro 1
Every continuous representation  $f$ of the underlying topological
group of a Lie group $G$ into that of another Lie group $H$ is an
analytic representation. 
\end{corollary}

In fact, the graph $K$ of $f$ is a closed subgroup of the Lie group
$G x H$, and hence is a Lie subgroup. Then $f$ is the composite of the
maps $G \rightarrow K$, and $K \rightarrow H$, and both of them can be
seen to be  analytic. As an immediate consequence, we have the
following  

\begin{corollary}\label{chap4-cor2}%corollary 2 
Lie groups with isomorphic underlying topological group structures
are analytically isomorphic. 
\end{corollary}

\subsection{Some examples.}\label{chap4-sec4.6}%sec 4.6

We have seen that the general linear group $GL(n,R)$ is a Lie group,
and by Theorem \ref{chap4-thm4}, every closed subgroup, and in
particular, the orthogonal and symmetric groups are Lie groups. A
group matrices defined by some polynomial identities in the
coefficients of the matrices is a Lie group.  

We\pageoriginale proceed to study $GL(n,R)$ in greater detail. If $x
\in GL(n,R)$ 
is the matrix $(a_{ij})$, $x_{i,j}=a_{i,j} - \delta_{i,j}$ is a coordinate
system which takes the unit matrix to origin in the space $
R^{n^2}$. Now,  
$$
\varphi_{i,j}(x,y) = x_{i,j}+y_{i,j}+ \sum_{k}x_{i,k} y_{k,j}
$$

Setting $u_{i,j}   = y_{i,j} + \sum_{k} x_{i,k}y_{k,j}$,
we have   
\begin{align*}
 \tau_{y}f(x) & =  f(x+u)\\
&   = f(x) + \sum\limits_{i,j} y_{i,j}(\frac{\partial  f}{\partial
   x_{i,j} }+\sum\limits_{k}^{x_{k,i}} \frac{\partial f}{\partial x_{k,j}})
\end{align*}

The left invariant differential operators of order $1$ are therefore
generated  
by $ x_{i,j} = \sum\limits^{n}_{k=1}a_{k.i} \dfrac{\partial}{\partial
  x_{k,j}}$. The $ X_{i,j}$ form a basis of the Lie algebra of
$GL(n,R). Y = \sum_{i,j}^{\lambda}X_{i,j}$ is a generic element of the
Lie algebra. We associate the matrix $\hat{Y}= (\lambda_{i,j})$
with this element $Y$. we now have 
the 

\begin{proposition}\label{chap4-prop2}%\pro 2
 The map $Y \rightarrow \hat{Y}$ of the Lie algebra of $GL(n,R)$
 into the algebra of all $n$-square matrices $\mathscr{M}_n(R)$ is a
 Lie algebra isomorphism.  
\end{proposition}

Let $Y$ be an element of the Lie algebra of $GL(n,R)$. We show that
the map $t \rightarrow \exp t  Y$ assigns to $t$ the  usual
exponential matrix exp $t Y$. It has been proved (Ch. \ref{chap3-sec3.3}) that $ x
=\exp t Y$ satisfies 
\begin{align*}
\frac{\partial x_{i.j}}{\partial t}& =\sum\limits_{k,l} \lambda_{k,l}
\frac{\partial \varphi_{i,j}}{\partial y_{k,l}} (x(t),e)\\ 
&=\sum\limits_{k,l} \lambda_{k.l} \delta_{j,l} (\delta
_{i,k}+x_{i,k})\\ 
&=\sum\limits_{k} \lambda_{k,j}(\delta_{i,k}+x_{i,k})
\end{align*}
i.e. $x$ is a matrix satisfying $\dfrac{d}{dt}x(t) =x(t) \hat{Y}$
with $x(O) =I$. These two conditions can easily be seen to be
satisfied by  $\exp t \hat{Y}$.  By the uniqueness 
theorem\pageoriginale on differential equations, $\exp tY= \exp t
\hat{Y}$, where the latter exponential is in the sense of the exponential
matrix. Let $Y$, $Z \in \mathfrak{g}$ the Lie algebra of $GL(n,R)$. The
map $X \rightarrow \hat{X} $ is trivially a vector space
isomorphism of $\mathfrak{g}$ onto $\mathscr{M}_n(R)$. Now, 
\begin{align*}
& \exp Y \exp Z = \exp (Y+Z+\frac{1}{2}[Y,Z]
  +\cdots) \quad\text{ by Ch. \ref{chap4-sec4.3}.}\\ 
\text{But} \qquad &   \exp Y \exp Z = \bigg(\sum
\frac{\hat{Y}^n}{n!}\bigg)\bigg(\sum
\frac{\hat{Z}^m}{m!}\bigg)\\ 
& \hspace {2cm }=1+\hat{Y}+\hat{Z}+\frac
     {\hat{Y}^2}{2!}+\cdots
\end{align*}
and
$$
\exp(Y+Z+\frac{1}{2}[Y,Z]+\cdots)
=1+\hat{Y}+\hat{Z}+\frac{1}{2} (\hat
{Y}^2+\hat{y}\hat{Z}+\hat{Z}\hat{Y}+\hat{Z}^2)+\cdots 
$$

Comparing the coefficients, we get
$$
[\hat{Y,Z}]=[\hat{Y},\hat{Z}].
$$

\begin{remarks*}
\begin{enumerate}
\renewcommand{\labelenumi}{(\theenumi)}
\item $Y = \dfrac{d}{dt}(\exp tY)_{t=0}$.

\item The Lie algebra of a closed subgroup $H$ of $GL(n,R)$ is simply
  the Lie algebra of matrices $Y$ such that $\exp t Y \in H$ for every
  $t \in R$ (by  Theorem \ref{chap4-thm4}, Ch. \ref{chap4-sec4.5}). 

\item Let  $B(a,b)$ be a bilinear form on $R^n$. 
\end{enumerate}
\end{remarks*}

Then, the set of all regular matrices $x$ which leave $B(a,b)$
invariant is a Lie subgroup $H$ of $GL(n,R)$. Then the Lie algebra of
this Lie Group consists of matrices $Y$ such that $B (Ya, b) +B(a,Yb)
=O$ for every $a,b \in R^n$. In fact, if $Y$ is in the Lie algebra,
exp   $t Y \in H , B$ being invariant under $H,B$ ($\exp tY a, \exp t
Y b ) = B (a,b)$. But  
$$
B(\exp t Y a, \exp t Y b ) =
\sum\limits_{p,Q}\dfrac{t^{p+q}}{p!q!}B (Y^p a, Y^qb).  
$$

Hence 
$$
B(Ya,b) + B(a,Yb) = 0.
$$


Conversely,\pageoriginale if $Y$ satisfies this condition, by
induction it can be 
seen that $\sum\limits_{p+q=n} \dfrac{t^{p+q}}{p!q!}$ $B(Y^P a \cdot Y^q
b)=0$, which proves that $B(\exp  t Y a, \exp\break  tYb) = B(a, b)$,
i.e. $\exp t Y \in H$ for every $t \in R$. This proves that $Y$ is
in the Lie algebra of $H$. 


\subsection{Group of automorphisms.}\label{chap4-sec4.7}%\sec 4.7

Let $G$ be a connected Lie group. Then the set of automorphisms of $G$
(continuous representations of $G$ onto itself), form a group. We
shall denote this group by $\Aut G$. 

Let $\alpha \in \Aut G$. This gives rise to a map $d \alpha:
\mathfrak{g} \rightarrow  \mathfrak{g}$ where $d \alpha$ is an
automorphism of the Lie algebra.  We thus have a map;  
 $\Aut G \rightarrow \Aut \mathfrak{g}$.  This map is one-one, and, if $G$
 is simply connected, onto. We have, in this case, an isomorphism of
 $\Aut G \rightarrow \Aut \mathfrak{g}$, for $d(\alpha_1, \alpha_2)=
 d \alpha _1 \circ d \alpha _2$ and  $d \alpha^{-1}
 =(d\alpha)^{-1}$. Now, $\Aut  \mathfrak{g}  \subset GL(\mathfrak{g}
 )$. $\Aut  \mathfrak{g}$  is actually a \textit{closed subgroup} of
 $GL(\mathfrak{g})$.  For, if $C^{k}_{i,j}$ be the structural
 constants of  $\mathfrak{g}$, 
 $A \in \Aut \mathfrak{g} \Leftrightarrow \big[A(x_i),
   A(x_j)\big] = \sum\limits_{k}  C^{k}_{i,j}A(x_k)$  for    every
 $i$, $j$ and $A \in GL(\mathfrak{g})$, $\{x_i\}$  being a basis of 
$\mathfrak{g}$. Since $\Aut \mathfrak{g}$ is determined  by these 
$n^2$ equations, it  is a closed subgroup of $GL( \mathfrak{g})$.
Hence, $\Aut  \mathfrak{g}$  is a Lie group. 

\begin{proposition}\label{chap4-prop3}%\prop 3
 Let $ \Gamma$ be the Lie algebra of $\Aut \mathfrak{g}$.  Then $X \in
 \Hom ( \mathfrak{g},   \mathfrak{g})$ (which is the Lie algebra of 
 $GL(\mathfrak{g} )$) is in $\Gamma$ if and only if $\exp tX  \in
 \Aut  \mathfrak{g}$ for every $t$ in $R$. This is obvious from the
 proof of Theorem \ref{chap4-thm4}, Ch. \ref{chap4-sec4.5}. 
\end{proposition}

\begin{proposition}\label{chap4-prop4}%\prop 4
$X  \in \Hom (\mathfrak{g}, \mathfrak{g})$ is in $\Gamma$ if and
  only if $X$ is a derivation. 
\end{proposition}

By\pageoriginale Proposition \ref{chap4-prop3}, 
\begin{align*}
(\exp tX) [Y,Z] & =[\exp tX\cdot Y,\exp 
  tX\cdot Z]\\ 
\text{i.e.,} \qquad \sum \frac{t^n X^n}{n!}\big([Y,Z]\big) &=
\sum\limits_{p,q}\frac{t^{p+q}}{p!q!}[x^pY,X^qZ]\\ 
X^n [Y,Z]&= \sum\limits_{p+q=n}\frac{n!}{p!q!}[X^pY, X^qZ]
\end{align*}
for every $n$. In particular
$$
X[Y,Z] =[XY,Z]+[Y,XZ]
$$
$X$ is therefore a derivation. Conversely,
$$
X[Y,Z]=[XY,Z]+[Y,XZ] \Rightarrow X^n [Y,Z]= \sum_{p+q=n}
\frac{n!}{p!q!} [X^pY,X^qZ]  
$$
by induction  on  $n \Rightarrow \exp  tX [Y,Z]= [\exp tX\cdot Y,
  \exp  tX\cdot Z]$. 

Hence   $X \in  \Gamma$.

In other words, the Lie algebra of $\Aut \mathfrak{g}$ is only the
{\em Lie algebra of  derivations of}  $\mathfrak{g}$ 
(it is a trivial verification to see that the derivations of $\mathfrak{g}$
form a Lie algebra and
the set of {\em inner derivations} (Remark 1, Ch. \ref{chap2-sec2.8})
form an ideal in that algebra). 
 
 Now, corresponding to every $y \in G$, there exists an inner
 automorphism $\rho_y : x \rightarrow y x y^{-1}$  of
 $G$. Obviously 
$y \rightarrow \rho_y$ is an algebraic representation of $G$ in $\Aut
 G$. $\rho_y$ induces an automorphism $d  \rho_y$ of  $\mathfrak{g}$.  
 We denote this by $ady$.
 
 $y \rightarrow  ady$ is an algebraic representation of $G$ in $\Aut
 \mathfrak{g} $. 
  We now show that this is an analytic representation. By Corollary to 
  Theorem \ref{chap4-thm4}, Chapter \ref{chap4-sec4.5}, it is enough
  to show that this is 
  continuous, i.e. if $y \rightarrow e$ then $adyX \rightarrow X$ for
  every $X \in \mathfrak{g}$. Since $G$ and  $ \mathfrak{g}$ are
  locally isomorphic, it suffices to prove that as $ y \rightarrow e$,
  $y\exp Xy^{-1}\rightarrow \exp X$ 
  but\pageoriginale this is obvious. This analytic representation of
  $G $ in $\Aut G=\Aut \mathfrak{g}$ is called the \textit{adjoint
    representation} of 
  $G$. Let $\theta$ be the differential of the representation $ y
  \rightarrow  ady$. We now show that this is actually the
  \textit{adjoint representation} (Ch. \ref{chap2-sec2.8}) of the Lie algebra
  $\mathfrak{g}$. 

\begin{thm}\label{chap4-thm}%\theo 5
$\theta (X)=ad~ X~$ for every  $X \in \mathfrak{g}$.
\end{thm}

By Remark 1, Prop. \ref{chap4-prop2},  Ch. \ref{chap4-sec4.6},
\begin{align*}
  \theta  (X)  &= \{\frac {d}{dt}(\exp~t~ \theta (X))\}_{t=0} \\
	&=\frac{d}{dt}~(ad~\exp~t~ x)_{t=0} 
\end{align*}
by definition of exponential. We have now to show that $\theta (X) Y
=[X,Y]$ for every $Y \in \mathfrak{g}$. Let $x=\exp t X\cdot \theta (X)Y=
\dfrac {d} {dt} (ad~Y)_{t=0}$.  But $(adx~Y~f) \circ \rho_x =
Y(f\circ \rho_x)$ where $f$ is any analytic function on $G$. 
It follows that
\begin{align*}
 \sigma_{x^{-1}} \circ \tau_{x^{-1}}~adx~Yf & =Y \sigma_{x^{-1}} \circ
   \tau_{x^{-1}} \circ f\\  
  or \qquad (adx~Y)\circ \tau_x  &= \tau_x\circ \sigma_x
  Y\sigma_{x^{-1}}\\ 
&  =\tau_x Y\\ 
\end{align*}
since $Y$ is left invariant
$$ 
adx~ Y =\tau_x Y\tau_{x^{-1}}.
$$

But
\begin{align*}
\tau_x~f(e)~=~f(\exp tx) & =\sum\limits_{n}\frac{t^n
  X^n}{n!}f(e)\quad \text{(Prop. \ref{chap3-prop6}, Ch. \ref{chap3-sec3.4})}\\
\text{Hence}\quad   adx ~Y & = \sum\limits_{m,n} \frac{t^m X^m}{m!}
\frac{(-t)^nX^n}{n!}\\ 
&  = \sum\limits_{m,n} (-1)^n\frac{t^{m+n}}{m!n!}X^m Y X^n
\end{align*}

Therefore 
$$
\theta (X)Y= \{\dfrac{d}{dt}(adx~ Y)\}_{t=0} = XY-YX=[X,Y].
$$

\begin{coro*}% \coro 
 Let $H$ be a connected Lie subgroup of $G$. $H$ is a normal subgroup
 if and only if its Lie algebra $\mathscr{J}$ is an ideal in
 $\mathfrak{g}$. 
\end{coro*}

In\pageoriginale fact, if $H$ is a normal subgroup, $\rho_y(H)\subset
H$  for every  $y \in G$, i.e. $ady \mathscr{J} \subset \mathscr{J}$  for
every $y \in G$. Let $X \in \mathfrak{g}$. Then ad $\exp t X
\mathscr{J} \subset \mathscr{J}$. 
$$
\dfrac{d}{dt} (ad~\exp tX)_{t=0} \mathscr{J} = ad~X \mathscr{J}\subset
 \mathscr{J} 
$$ 
i.e. $\mathscr{J}$ is an ideal.

Reciprocally, let $\mathscr{J}$ be an ideal. ad $X \mathscr{J}\subset
\mathscr{J}\cdot (\sum \dfrac{t^nad X^n}{n!}) \mathscr{J} \subset
\mathscr{J}$. 
But $\exp t\ ad\ X =ad\ \exp tX\cdot (ad \exp tX) H \subset H$ for every
$X$ in a neighborhood of $e$. Since $G$ is connected, $H$ is normal.  

\subsection{Factor groups.}\label{chap4-sec4.8}%\sec 4.8

\begin{thm}\label{chap4-thm6}% the 6
 Let $H$ {be a closed subgroup of a Lie group} $G$. {The homogeneous
   space $G/H$ is an analytic manifold in a canonical way}. {The
   operations by} $G$ on $G/H$ are isomorphisms. If $H$ is a normal
 subgroup, is a Lie group, and its Lie algebra is
   isomorphic to $\mathfrak{g}/\mathscr{J}$. 
\end{thm}

Let $\mathscr{J}$ be the Lie algebra of $H$, and let $\mathcal{U}$ be
a vector subspace of $\mathfrak{g}$ supplementary to $\mathscr{J}$. We
have seen in the proof of Theorem \ref{chap4-thm4},
Ch. \ref{chap4-sec4.5}, that there exists a 
neighborhood $V^1$ of $(0,0)$ in $\mathscr{J} \times \mathcal{U}$  such
that the  
map $\lambda:(X,A) \rightarrow \exp X \exp A$ is an isomorphism of
$V^1$ onto a neighborhood $W^1$ of $e$ in $G$. Let $U$ and $V$ be
neighborhoods of $0$ in $\mathscr{J}$ and $\mathcal{U}$ respectively
such that $U \times V \subset V^1$ and $W W^{-1} \subset W^{1}$ with
$W=\lambda(U\times V)$. 
We now show that $L = \exp V$ is a cross-section of the canonical map
$\eta : G \rightarrow G/H$ in the neighborhood $\dot{W} = \eta (W)$
of $\eta (e)$. In other words, $L\cap Hx$ contains one and only one
element for every $x\in  W$. For, we have $x=\exp X \exp  
A$ with $X \in U$ and $A \in V$  and $\exp A  \in  L \cap Hx$. On the
other hand, if $\exp A_1$ and $\exp A_2$ belong to $Hx$ (with $A_1$, $A_2
\in V$), then $\exp A_1 (\exp A_2)^{-1} \in H \cap W^1$;\pageoriginale 
hence there exists an $X \in V^1 \cap \mathscr{J}$ such that $\exp A_1= 
\exp X \exp A_2$  and this implies $X=0$, $A_1 = A_2$ because
$\lambda$ is an isomorphism from $V^1$ onto $W^1$. 

We can, therefore, provide $\dot{W}$ with a manifold structure induced
from that of $\mathcal{U}$. This can be extended globally by
translating that on $W$. It is easily seen that on the overlaps
$\dot{x} \dot{W}$, $ \dot{y} \dot{W}$, the analytic structures agree
because the analytic structure on $\dot{W}$ is induced from that of
$\mathcal{U}$. By the definition of the manifold $G/H$, it is obvious
that the operations by $G$ on $G/H$ are analytic isomorphisms.  

If $H$ is normal subgroup, $G/H$ has also a group structure and is a
Lie group with the above manifold. By Theorem \ref{chap4-thm6},
Ch. \ref{chap4-sec4.8}, 
$\mathscr{J}$ ia an ideal of $\mathscr{J}$ and if $\mathcal{K}$ is the
Lie algebra of $G/H$, the map $\eta :\rightarrow G/H$  
gives rise to a representation $d\eta :\mathscr{J} \rightarrow
\mathcal{K}$. The kernal of this map is $\mathscr{J}$ since
$\mathcal{K}$ is isomorphic as a vector space to the tangent space at
$e$ of $L$ which is $\mathcal{U}$. Hence $\mathscr{H}/\mathscr{J}$ is
isomorphic to $\mathcal{K}$ as a Lie algebra also, i.e. $G/H$ has its
Lie algebra isomorphic to $\mathfrak{g}/\mathscr{J}$. 

\begin{coro*}
 Let $f$ be a representation of a Lie group $G$ which is countable at
 $\infty$ in another  Lie group $H$. Then the image $f(G)$ is a Lie
 subgroup. If $N$ is the kernel of $f$, then $f$ can be factored into
 $G \xrightarrow{\pi} G/N \xrightarrow{\bar{f}}H$ where $\pi$ is the
 canonical map and $\bar{f}$ an injective regular map. 
\end{coro*}
 
The proof is an immediate consequence of the isomorphism theorem on
Lie algebras and Theorem \ref{chap4-thm6}. 
