
\chapter{Local study of Lie groups}\label{chap2}

\setcounter{section}{2}
\setcounter{subsection}{0}
\subsection{}\pageoriginale\label{chap2-sec2.1}%sec 2.1

\begin{defi*}
{\em A Lie group} $G$ is a real analytic manifold with a composition
law $(x,y)\to xy$ which is  
\begin{enumerate}
\renewcommand{\theenumi}{\alph{enumi}}
\renewcommand{\labelenumi}{\rm(\theenumi)}
\item  a group law, and
\item such that the map $(x,y)\to x^{-1}y$ is analytic.
\end{enumerate}

(b) is equivalent to the analyticity of the maps $(x,y) \to xy$ and $x
\rightarrow x^{-1}$. 
\end{defi*}

\begin{remarks*}
\begin{enumerate}
\renewcommand{\labelenumi}{(\theenumi)}
\item  A Lie group is trivially a topological group.

\item  We may replace `real' by `complex and define the notion of a
  complex Lie group. We shall not have occasion to study complex Lie
  groups in what follows, though most of the theorems we prove remain
  valid for them. 

\item  It is natural to inquire whether every topological group with
  the structure of a topological manifold is a Lie group. This problem
  (Hilbert's fifth problem) has been recently solved by Gleason
  \cite{key18} who has proved that a topological group $G$ which is
  locally compact, locally connected, metrisable and of finite
  dimension, is a Lie group. 
\end{enumerate}
\end{remarks*}

\section*{Examples of Lie groups.}

\begin{enumerate}
\renewcommand{\labelenumi}{(\theenumi)}
\item  $R$ - real numbers, $C$ - complex numbers,

$T$ - the one-dimensional torus and $R^n$, $C^n$ and $T^n$ in the
  usual notation are all Lie groups. 

\item  Product\pageoriginale of Lie groups with the product manifold
  structure is a Lie group. 

\item  $GL(n, R)$ - the general linear group.
\end{enumerate}

\subsection{Local study of Lie groups.}\label{chap2-sec2.2}%subsec 2.2

We shall assume that $V$ is a sufficiently small neighbourhood of $e$
in which a suitably chosen coordinate system, which taken $e$ into the
origin, is defined. 

The following notations will be adhered to throughout these lectures:

If $a \in V$, $(a_1,\ldots,a_n)$ will denote the coordinate of
$a$. $\alpha =(\alpha_1,\ldots,\break\alpha_n)$ is a multi-index with
$\alpha_i$, non-negative integers.  
$$
|\alpha|=\alpha_1+\alpha_2+ \cdots + \alpha_n
$$ 
[i] will stand for $\alpha$ with $\alpha_i =1$ and $\alpha_j=0$ for $j\neq i$. 
\begin{align*}
\alpha ! & = \alpha_1 ! \ldots \alpha_n !\\
x^{\alpha}  &= x_1^{\alpha_1} \ldots x_n^{\alpha_n}\\
\frac{\partial^\alpha}{\partial x^\alpha}  &=
\frac{\partial^{\alpha_1+ \cdots + \alpha_n}}{\partial x_1^{\alpha_1}
  \cdots \partial x_n^{\alpha_n}} 
\end{align*}

Let $x$, $y \in V$ be such that $xy \in V$. $(xy)_i$ are analytic
functions of the coordinate of $x$ and $y$. 
\begin{enumerate}
\renewcommand{\labelenumi}{(\theenumi)}
\item  $(xy)_i=\varphi_i(x_1, \ldots , x_n, y_1, \ldots , y_n)$ where
  the $\varphi_i$ are analytic functions of the $2n$ variables $x$, $y$
  in a neighbourhood of $0$ in $R^{2n}$. These $\varphi_i$ cannot be
  arbitrary functions, as they are connected by the group
  relations. These are reflected in the following equations: 
$$
(x\  e)_i=(e\ x)_i =x_i,\quad\text{or} 
$$ 

\item $\varphi_i(x, e)= \varphi_i(e,x)=x_i$.\pageoriginale

The $\varphi_i$ are analytic functions and so are of the form 
$$
\varphi_i(x,y)= \sum_{\alpha, \beta}\lambda^i_{\alpha_\beta}
x^{\alpha}y^{\beta}, i = 1, 2, \ldots , n  
$$

By equation (2), this can be written

\item $\varphi_i (x,y)=x_i+y_i+ \sum_{\substack{| \alpha
    | \ge 1 \\ | \beta | \ge 1}} \lambda^i_{\alpha,\beta}
  x^{\alpha} y^{\beta}$. 

By associativity,
\begin{align*}
\varphi_{i}(xy,z) = \varphi_{i} (x,yz),\quad\text{or}\\
\varphi_i(\varphi_1(x,y), \ldots , \varphi_n(x,y),z)=\varphi_i(x,
\varphi_1(y,z), \ldots , \varphi_n(y,z)). 
\end{align*}

These may also be written

\item $\varphi_i(\varphi(x,y),z)=\varphi_i(x, \varphi(y,z))$. 
\end{enumerate}

One is tempted to expect another equation in $\varphi_i$, due to the
existence of the inverse of every element. However, these two
equations are sufficient to characterise locally the Lie group, and
the existence of the inverse is, in a certain sense, a consequence of
the associative law and the existence of the identity. To be more
precise, 

\setcounter{proposition}{0}
\begin{proposition}\label{chap2-prop1}% pro 1
 Let $G$ be the semigroup with an identity element $e$. If it can be
 provided with the structure of an analytic manifold such that the map
 $(x,y)\rightarrow xy$ of $G x G \rightarrow G $ is analytic, then
 there exists an open neighbourhood of $e$ which is a Lie group. 
\end{proposition}

In fact, the existence of the inverse element of $x$ depends upon the
existence of the solution for $y$ of $\varphi_i(x,y)=0, 1=1, \ldots ,
n$. Now $\dfrac {\partial \varphi_i}{\partial x_j}=\delta_{ij}+$ terms
containing positive powers of the $y_i$. If we put $y = e$, the latter
terms vanish and 
$$
\left( \frac{\partial \varphi_i (x,y)}{\partial x_j} \right)_{y=e}=
\delta_{ij}. 
$$

Hence\pageoriginale 
$$
J=\det \left(\dfrac {\partial \varphi_i
  (x,y)}{\partial x_j} \right)_{y=e}=1
$$ 
$J$ being a continuous function of $x$ and $y$, $J \neq 0$ in some
neighbourhood $V'$ of $e$. Therefore there exists a neighbourhood $V$
of $e$ every element of which has an inverse. Then the neighbourhood
$W= \bigcup^{\infty}_{n=1}(V\cap V^{-1})^n$ is a group
compatible with the manifold structure. 

\subsection{Formal Lie groups.}\label{chap2-sec2.3}%subsec 2.3

\begin{defi*}
A {\em formal Lie group} over a commutative ring A with unit
elements, is a system of $n$ formal series $\varphi_i$ in $2n$
variables with coefficients in A such that 
$$
\varphi_i(x_1, \ldots , x_n, 0, 0, \ldots) = x_i = \varphi_i (0, 0,
\ldots , x_1, \ldots , x_n) 
$$ 
and
$$
\varphi_i(\varphi_1(x,y), \ldots , \varphi_n (x,y),z)= \varphi_i(x,
\varphi_1(y,z), \ldots , \varphi_n(y,z)). 
$$
\end{defi*}

Almost all that we prove in the next few lectures will be
valid for formal Lie groups over a field of characteristic zero
also. For a study of formal Lie groups over a field of characteristic
$p\neq 0$, one may see, for instance, \cite{key9}, \cite{key10}.  

\subsection{Taylor's formula.}\label{chap2-sec2.4}%subsec 2.4

Let $f$ be a function on an open neighbourhood of $e$, and let
$\tau_y$, $\sigma_z$ denote respectively the right and left translates of
$f$ defined by $\tau_y f(x)=f(xy)$; $\sigma_z f(x) = f(z^{-1}x)$ for
sufficiently small $y$ and $z$. These two operators commute, 

i.e. 
$$
\tau_y (\sigma_z f) = \sigma_z(\tau_y f)
$$

If $f$ is analytic in $V$, $\tau_y f$ is (for $y \in W$) analytic in
$W$, where $W$ is a neighbourhood of $e$ such that $W^2 \subset V$. 

Now,\pageoriginale 
$$
\varphi_i f(x)  = f(\varphi_1(x,y), \ldots , \varphi_n (x,y))
$$
with 
$$
\tau_i(x,y) = x_i + y_i + \sum _{\substack{| \alpha |
    \ge 1\\ | \beta |  \ge 1}}  \lambda^i_{\alpha, \beta}
x^{\alpha}y^{\beta}.
$$

If we set
$$
u_i = y_i + \sum _{\substack{|\alpha| \ge 1\\ |\beta| \ge 1 }}
\lambda_{\alpha, \beta}^i x^{\alpha}y^{\phi} 
$$
$\tau_y f(x) = f(x+u)$ in the usual notation. This can be expanded as
a Taylor series 
$$
\tau_y f(x) = \sum_{\alpha} \frac{1}{\alpha !} u^{\alpha} 
\frac{\partial^\alpha f}{\partial x^\alpha} (x). 
$$

We may now substitute for the $u_i$ in this convergent series.
\begin{align*}
u^\alpha &= u_1^\alpha \cdots u_n^{\alpha_n}\\
& = \bigg( y_1 + \sum\limits_{\substack{| \gamma |  \ge 1\\ |
    \delta |  \ge 1 }}  \lambda_{\gamma, \delta}^1
x^{\gamma}y^{\delta} \bigg)^{\alpha_1} \ldots\\ 
& = y^{\alpha} + \sum_{| \beta | \geq | \alpha |}
g^{\alpha}_{\beta}(x) y^\beta\\ 
\end{align*}
where the coefficient of powers of $y$ are analytic functions of $x$
and $g^{\alpha}_{\beta}(e)= 0$. If here we take $\alpha = 0$, $u^\alpha
= 1$, and hence $g^{0}_{\beta} = 0$ for every $\beta$. Thus 
$$ 
\tau_y f(x) = \sum_{\alpha}\dfrac{1}{\alpha !} \dfrac{\partial^\alpha
  f}{\partial x^\alpha}(x) (y^\alpha + 
    \sum_{| \beta |\ge | \alpha|}
    g^{\alpha}_{\beta} (x) y^\beta). 
$$

These are uniformly absolutely convergent in a suitable neighbourhood
of $e$, on the explicit choice of which we shall not meticulously
insist. Hence the above formula can be written as 
$$
\tau_y f(x) = \sum_{\alpha} y^\alpha(\dfrac{1}{\alpha !}
\dfrac {\partial^\alpha f} {\partial x^\alpha}(x) + \sum\limits_{|
  \beta | \leq | \alpha |}g^\beta_\alpha (x) \dfrac{1}{\beta
  !} \dfrac{\partial^\beta f}{\partial x^\beta}(x)). 
$$

This we shall denote by
$$
\tau_y f(x) = \sum_{\alpha}\dfrac{1}{\alpha!} y^\alpha
\Delta_\alpha f(x) 
$$
where\pageoriginale $\Delta_\alpha$ is a differential operator not
depending on $f$. This formula is the generalised Taylor's formula we sought
establish. 

\subsection{Study of the operator
  $\Delta_\alpha$.}\label{chap2-sec2.5}%subsec 2.5 

Now, $\Delta_\alpha = \dfrac {\partial^\alpha}{\partial x^\alpha} +
\sum\limits_{\mid \beta \mid \le \mid \alpha \mid} g^\beta_\alpha (x)
\dfrac {\partial^\beta}{\partial x^\beta} \dfrac {\alpha !}{\beta !}$
with $g^\beta_\alpha (e)=0$.  
Since at $x = e$, $\Delta_\alpha = \dfrac{\partial^\alpha}{\partial
  x^\alpha}$, the $\Delta_\alpha$ are linearly independent at the
origin. At $\alpha = 0$, $\Delta_\alpha$ is the identity operator and
if $\alpha \neq 0$, $\Delta_\alpha$ is without constant term as
$g^0_\alpha=0$. 

Let us denote $\Delta_{[i]}$ by $X_i$. Then $X_i =
\dfrac{\partial}{\partial x_1}+\sum_{j}
a_{ij}(x)\dfrac{\partial}{\partial x_j}$, with $a_{ij}(e)=0$. These
are vector fields in a neighbourhood of $e$. Now we shall use the fact
that, for every $y$, $z \in G$, the operators $\sigma_z$ and $\tau_y$
commute. We have 
$$
\sigma_z(\tau_y f)= \sigma_z (\sum \dfrac{1}{\alpha !} y^\alpha
\Delta_\alpha f) = \sum \dfrac{1}{\alpha !} y^\alpha (\sigma _z \circ
\Delta_\alpha)(f) 
$$
and, on the other hand, 
$$
\tau_y (\sigma_z f) = \sum \dfrac{1}{\alpha !} y^\alpha \Delta_\alpha
(\sigma_z f). 
$$

Therefore, by the uniqueness of the expansion in power-series in $y$
of $\sigma_z(\tau_y f)= \tau_y (\sigma_z f)$, we have $\sigma_z \circ
\Delta_\alpha = \Delta_\alpha \circ \sigma_z$, for every $z$ in a
sufficiently small neighbourhood of $e$. Otherwise stated,
$\Delta_\alpha$ is left invariant in this neighbourhood. This enables
us to define $\Delta_\alpha$ at every point $z$ in the Lie group by
setting $\Delta_\alpha f(z) = \sigma_{z^{-1}} \Delta_\alpha \sigma_z
f(z)$, so that the extended operator remains left invariant. 

\setcounter{thm}{0}
\begin {thm}\label{chap2-thm1}%thm 1
The linear differential operators $\Delta_\alpha$ form a basis for the
algebra of left invariant differential operators. 
\end{thm}

We have already remarked that the $\Delta_\alpha$ are linearly
independent at $e$. Let $D$ be any left invariant linear differential
operator on $G$. 
Then\pageoriginale $D=\sum_{| \alpha| \leq r} b_\alpha (x)
\dfrac{\partial^\alpha}{\partial x^\alpha}$, $r$ being some positive
integer. Define $\Delta=\sum_{| \alpha| \leq r}  b_\alpha
(e)\break\dfrac{\partial^\alpha}{\partial x^\alpha}$. Obviously, $\Delta\circ
\sigma_z=\sigma_z \circ \Delta$. At $e$, $D=\Delta$,  and by the left
invariance of both $D$ and $\Delta$, $D=\Delta$ everywhere. This proves
our contention that the $\Delta_\alpha$ form a basis of the algebra of
left invariant differential operators. 

We shall hereafter denote this algebra by $\mathcal{U}(G)$. 

\subsection{The Lie algebra of a Lie group
  G.}\label{chap2-sec2.6}%subsec 2.6 

Let $\mathscr{G}$ be the subspace of $\mathcal{U}(G)$ generated by the
$X_i$. This is the same as the subspace composed of vector fields
which are left invariant. This is obviously isomorphic as a vector space
to the tangent space at $e$. If there are two vector fields $X$, $Y$,
then $XY$ is an operator of order 2, as also $YX$. But $XY-YX$ is a
left invariant vector field, as can be easily verified. Lef $[X,Y]$
stand for this composition law. It is not hard to see that this
bracket operation satisfies 
$$
[X,Y]=0,\text{~  and~  } [[X,Y],Z]+[[Y,Z],X]+[[Z,X],Y]=0.
$$

This leads us to the following

\begin{defi*}%\def
A {\em Lie algebra} $\mathfrak{g}$ over a field, is a vector space
with a composition law $[X,Y]$ which is a bilinear map
$\mathfrak{g}\times \mathfrak{g} \to \mathfrak{g}$ satisfying
$[X,X]=0$, and the Jacobi's identity, viz.  
$$
[[X, Y],Z] + [[Y, Z], X] + [[Z, X],Y]=0.
$$
\end{defi*}

\begin{examples*}
\begin{enumerate}
\renewcommand{\labelenumi}{(\theenumi)}
\item The left invariant vector fields of a Lie group form a Lie algebra.

\item Any vector space $\mathcal{U}$ with the composition Law $[X,
  Y]=0$ for every $X, Y \in \mathcal{U}$ is a Lie algebra. 

Such\pageoriginale an algebra is called an \textit{abelian Lie algebra}.

\item Any associative algebra with the bracket operation 
$$
[X, Y]=XY-YX
$$
is a Lie algebra.
\end{enumerate}
\end{examples*}

In particular, the matrix algebra  $\mathscr{M}_n(K)$ over a field $K$
and the space of endomorphisms of a vector space $\mathcal{U}$ are Lie
algebras. 

\begin{defi*}%\def
 A subspace $\mathscr{J}$ of a Lie algebra $\mathfrak{g}$ is a {\em
   subalgebra} if for every $x$, $y \in \mathscr{J}$, $[x,y]\in
 \mathscr{J}$. 
\end{defi*}
 
A subspace $\mathcal{U}$ of a Lie algebra $\mathfrak{g}$ is an
 \textit{ideal}, if for every $x \in \mathcal{U}$, $y \in
 \mathfrak{g}$, $[x, y]\in  \mathcal{U}$. 

\begin{examples*}
\begin{enumerate}
\renewcommand{\labelenumi}{(\theenumi)}
\item The set of all matrices in $\mathscr{M}_n(K)$ whose traces are
  zero is an ideal of $\mathscr{M}_n(K)$. 

\item The set of all elements $z \in \mathfrak{g}$ such that $[z, x]=0$
  for every $x \in \mathfrak{g}$ is an ideal of $\mathfrak{g}$, called
  \textit{centre} of $\mathfrak{g}$.  
\end{enumerate}
\end{examples*}

If $\mathcal{U}$ is an ideal in $\mathfrak{g}$, the quotient space
$\mathfrak{g}/_{\mathcal{U}}$ can be provided with the structure of a
Lie algebra by defining $[(x+\mathcal{U}),(y+\mathcal{U})]=[x,
  y]+\mathcal{U}$. This is called the \textit{factor algebra of}
$\mathfrak{g}$ \textit{by} $\mathcal{U}$. 

\subsection{Representations of a Lie algebra.}\label{chap2-sec2.7}%subsec 2.7

\begin{defi*}%definition
 A {\em representation} of a Lie algebra $\mathfrak{g}$ in another Lie
 algebra $\mathfrak{g}'$ is a linear map $f$ such that $f([x,
   y])=[f(x), f(y)]$ for every $x$, $y \in \mathfrak{g}$.  
\end{defi*}

It can be verified that the image of $\mathfrak{g}$ by $f$ and the
kernel of $f$ are subalgebras of $\mathfrak{g}'$ and $\mathfrak{g}$
respectively. The latter is, in fact, an ideal and $\mathfrak{g}/ \ker
f$ is isomorphic to $f(\mathfrak{g})$. 

In\pageoriginale particular, $\mathfrak{g}'$ may be taken to the the space of
endomorphisms of a vector space $V$, leading us to the definition of
a linear representation of $\mathfrak{g}$. 

\begin{defi*}%\def
A {\em linear representation} of a Lie algebra $\mathfrak{g}$ in a vector
space $V$ is a representation of $\mathfrak{g}$ into the Lie algebra of
endomorphisms of $V$. 
\end{defi*}

\subsection{Adjoint representation.}\label{chap2-sec2.8}%subsec 2.8

Let $\mathfrak{g}$ be a Lie algebra and $x \in \mathfrak{g}$. The map
$\mathfrak{g} \rightarrow \mathfrak{g}$  defined by $y
\longrightarrow[x, y]$ is a linear map of $\mathfrak{g}$ into
itself. This map is denoted adx. Thus, adx $(y)=[x, y]$. 


\begin{remarks*}
\begin{enumerate}
\renewcommand{\labelenumi}{(\theenumi)}
\item Ad $x$ is a derivation of the Lie algebra. We recall here the
  definition of a derivation in an algebra $\mathfrak{g}$, associative
  or not. A linear map $D$ of $\mathfrak{g}$ into itself is a
  \textit{derivation} if for any two elements $x$, $y \in
  \mathfrak{g}$, 
  we have $D (xy)=x(Dy)+(Dx)y$. In a Lie algebra derivations of the
  type ad $x$ are called \textit{inner derivations}.  

\item $x \rightarrow ad\ x$ is a linear map of $\mathfrak{g}$ into
  the Lie algebra of endomorphisms of the vector space $\mathfrak{g}$. 
\end{enumerate}
\end{remarks*}

This is, moreover, a linear representation of $\mathfrak{g}$ in
$\mathfrak{g}$. The verification of the relation $ad[x, y]=[adx, ady]$
is an immediate consequence of Jacobi's identity. 

This linear representation will henceforth be referred to as the
\textit{adjoint representation} of $\mathfrak{g}$.  
