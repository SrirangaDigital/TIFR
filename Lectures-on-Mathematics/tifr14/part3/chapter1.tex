\chapter{Continuous sum of Hilbert Spaces-I}\label{partIII-chap1}

\setcounter{section}{1}
\setcounter{subsection}{0}
\subsection{Introduction}\label{partIII-chap1-sec1.1}\pageoriginale% 1.1 

In the general theory of unitary representations of a locally compact
group, two main problems are (i) to determine all the irreducible
unitary representations of a group, and (ii) to decompose a given
unitary representation into irreducible ones. The first of these has
been completely solved in certain cases (e.g. abelian groups, compact,
certain semisimple Lie groups), but it is to the latter that we
address ourselves in the following pages. We start by giving some
Examples. 

Let $U$ be the regular representation of the one dimensional torus
$T^1$ in the space $L^2$ of square summable functions. If $f$ belongs
to $L^2$, it can be expressed in Fourier series $\sum  a_n
e^{inz}$. If  $x = e ^{it}$, then $\sigma_x f = \sum\limits_n
\sigma_{x} a_{n} e^{inz} =   
\sum\limits_n (a_{n} e^{\int}) e^{inz}$  and we have decomposed a
unitary representation into a direct sum of irreducible
representations. 

If we take $R$ instead of $T^1$ and $F \in L^2$, then $\hat{f}(y) =
\int f(x)e^{ixy}dx$ also belongs to $L^2$. By the inversion formula,
when $f$ is sufficiently regular (we do not enter into these details)
we have $f(x) = \dfrac{1}{2 \pi} \int\hat{f}(y)e^{-ixy}dy$. Hence
\begin{align*}
\sigma_z f &= \dfrac{1}{2\pi} \int \hat{f}(y)e^{-i(x+z)y}dy\\ 
&=\dfrac{1}{2\pi}\int\hat{f}(y)e^{izy}e^{-ixy}dy\\
&= \dfrac{1}{2\pi} \int(\hat{\sigma f})(y)e^{-ixy}dy.
\end{align*}

Therefore $(\hat{\sigma_z
  f})(y) = e^{izy}\hat{f}(y)$. The regular representation has again
been decomposed into one-dimensional representations but this is not a
discrete sum but a `continuous sum' - a concept which we shall define
presently. 
 
Before proceeding with the formal definitions, we give one more
example which is more akin to the theory we are to develop. Let $E$ be
a locally\pageoriginale compact space with a positive measure $\mu$ and
$\mathscr{H}$ a Hilbert space. In the space
$\mathscr{C}(E,\mathscr{H})$ of continuous functions $f : E \to
\mathscr{H}$ with compact support, we introduce a semi-norm 
$$
|| f ||=\left( \int\ || f(z) ||^{2} d \mu (z)\right)^{\frac{1}{2}} <
\infty.
$$  

Let $\mathscr{L}^2 (\mathscr{H})$ be the completion of the
Hausdorff space associated with $\mathscr{C}(E,\mathscr{H})$. We have
also a scalar product in this space given by $\langle f,g \rangle =
\int \langle f(z),g(z)\rangle  d\mu(z)$. 

If $\mu$ is discrete (i.e. is a linear combination of Dirac measures
at certain points), $\mathscr{L}^2 (\mathscr{H})$ becomes a discrete
sum of Hilbert spaces associated to each of those points, all the
Hilbert spaces being isomorphic to $\mathscr{H}$. 

These considerations motivate some kind of a continuous sum of Hilbert
spaces indexed by points of a locally compact space. We therefore
assume the following data to start with: 
\begin{enumerate}
\renewcommand{\labelenumi}{(\theenumi)}
\item $\mathcal{Z}$, a locally compact space (which will be assumed
  for simplicity to be countable at $\infty$) with a positive measure
  $\mu$; 

\item For every $z \in \mathcal{Z}$, a Hilbert space
  $\mathscr{H}(z)$. In other words, we assume given at each point a
  `tangent space' which is a Hilbert space. For instance, in a
  Riemannian manifold, the metric assigns a scalar product to the
  tangent space at each point of $\mathcal{Z}$. Of course in this case
  the spaces are of finite dimension. Having in mind the example
  above, we seek to find an analogue of the concept of functions on
  $\mathcal{Z}$. This is served by the notion of a \textit{vector
    field} (in exactly the same sense as in manifolds) which is an
  assignment to each point, of an element of the associated Hilbert
  space. We would like to have a notion analogous to that of
  continuous functions in our example. To this end, we introduce a
  \textit{fundamental family}\pageoriginale 
   of vector fields with reference to which the continuity of an
  arbitrary vector field will defined. Thus we 
  suppose given 

\item \textit{A fundamental family} $\wedge$ of vector fields which
  satisfies the following conditions: 
\begin{enumerate}
\renewcommand{\theenumii}{\alph{enumii}}
\renewcommand{\labelenumii}{(\theenumii)}
\item $\wedge$ forms a vector space under the `usual' operations.

\item For every vector field $X \in \wedge$, the real valued function
  $|| X(z) ||$ is continuous. This in particular implies that the map
  $z \to \langle X(z),Y(z) \rangle $ is continuous for every $X,Y \in
  \wedge$. 

\item For every $z$, the vectors $X(z)$ for $X\in \wedge$ are
  everywhere dense in $\mathscr{H}(z)$. This only ensures that the
  system $\wedge$ is sufficiently large. Sometimes the fundamental
  family $\wedge$ will be supposed to satisfy the following stronger
  condition: 

\setcounter{enumii}{2}
\renewcommand{\labelenumii}{(\theenumii$'$)}
\item There exists a countable subset $\Lambda_0 = \{X_n \}$
  of $\Lambda$ such that for every $z \in \mathcal{Z}$, $X_n (z)$ are
  dense in $\mathscr{H}(z)$. In particular, this implies that all the
  Hilbert spaces $\mathscr{H}(z)$ are separable. (We will always
  assume that the stronger condition (c$'$) is valid though some of
  the results remain true without this supposition). 
\end{enumerate}
\end{enumerate}

\subsection{Notion of continuity}\label{partIII-chap1-sec1.2}%1.2

We proceed to construct the continuous sum of the spaces
$\mathscr{H}(z)$. In our axioms relating to the fundamental family, we
have not imposed any restrictions on its behaviour at
$\infty$. Consequently it cannot be asserted that the $|| X(z) ||$ are
square summable. Moreover, the class $\Lambda$ may be too small (as
they will be if we take them to be constants in our example) to be
dense in $\mathscr{L}^2(\mathscr{H})$. This necessitates the extension
of this family to the class of \textit{continuous} vector fields by
means of the  

\begin{defi*}
A\pageoriginale vector field $X$ is continuous at a point $\zeta_0$ if
  given an $\epsilon > 0$ there exists $Y \in \Lambda$ and a
  neighbourhood $V$ of 
  $\zeta_0$ such that $|| X(\zeta) - Y(\zeta) || < \epsilon$ for every
  $\zeta \in V$. 
\end{defi*}

\begin{remark*}
When we take $\Lambda$ to be constants in the example, this
corresponds to the usual continuity of functions. 
\end{remark*} 

\setcounter{proposition}{0}
\begin{proposition}\label{partIII-chap1-prop1}%prop 1
 $A$ vector field $X$ is continuous if an only if $|| X ||$ is
  continuous and $\langle X, X_n\rangle$ is continuous for every $X_n
  \in \Lambda_0$. 
\end{proposition}

If $X$ is continuous, trivially $|| X ||$ is continuous. Also if $X$
and $X'$ are continuous, then $\langle X,X'\rangle$ is continuous: in
fact, for every $\epsilon > 0$ and every $\zeta \in \mathcal{Z}$, there
exist a neighbourhood $V$ of $\zeta$ and $Y$, $Y' \in \Lambda$ such that
$|| X - Y|| < \epsilon$, $|| X' - Y' || < \epsilon$ in $V$. 


Hence  
$$
| \langle X,X' \rangle -  \langle Y,Y'\rangle |  \leq  |
\langle X - Y, X' \rangle | + | \langle Y,Y' - X'\rangle | 
$$ 
$\leq M \epsilon$ in $V$, where $M$ is some constant. As $\langle Y,Y'
\rangle$ is continuous, it follows that $\langle X,X'\rangle$ is also
continuous. To prove the converse, it is enough to show that $||
X(\zeta) - X_n (\zeta) ||$ is continuous. But $|| X(\zeta) -
X_n(\zeta) ||^{2}  = || X ||^{2} - 2Rl \langle X,X_n\rangle + || X_n
||^2 $, all continuous by our assumption. 

A continuous vector field can be multiplied by a scalar continuous
function without affecting its continuity. 

\begin{proposition}\label{partIII-chap1-prop2}%prop 2 
The vector fields $\sum \varphi_i(\zeta)Y_i(\zeta)$ with $\varphi_i \in
\mathscr{C}(\mathcal{Z})$ and $Y_i \in \Lambda$ are dense in the space
of continuous vector fields with the topology of uniform convergence
on compact sets. 
\end{proposition}

At each point $x$ in a compact set $K$, there exists a neighbourhood
$A_x$ in which $|| X - Y_x || < \epsilon $ for some $Y_x \in \Lambda$. We
can extract a finite cover $\{A_{x_i}\}$ from $\{A_x\}$ and take
the partition of unity with respect to\pageoriginale 
this cover. Hence there exist
continuous functions $\phi_i$ such that $|| X-\sum \phi_i Y_{x_i}||<
\epsilon$ on $k$. 

\subsection{The space $ L^2_\wedge$}\label{partIII-chap1-sec1.3} % sec 1.3

Our next step is to construct the space $L^2_\wedge$ of square
summable  vector fields. We shall say that $X$  belongs to
$L^2_\wedge$ if given an $\epsilon >  0$ there exists a continuous vector
field $Y$ 
 with compact support such that 
$$
\int\limits^{*} || X (\zeta)-Y(\zeta)||^2 d\mu(\zeta)<\epsilon. 
$$
(We do not know  a priori whether $|| X(\zeta)-Y(\zeta)||^2$ is 
 measurable or not and hence we can only consider the upper
 integral). In this space, we can define 
\begin{gather*}
|| X ||^2 =\int || X(\zeta)||^2 d\mu(\zeta) \text{~ and}\\
\langle X,X' \rangle = \int  \langle X(\zeta), X' (\zeta) \rangle
d\mu(\zeta). 
\end{gather*}

By passing to the quotient space modulo vector fields of norm $0$, we
get a Hilbert space (which again we denote by $L^2_\wedge$). As in the
case of the  theory of integration, we have of course to prove the
completeness of $L^2_\wedge$, but there is no trouble in imitating the
proof of the Riesz-Fisher theorem in this case. 

$L^2_\wedge$ is the \textit{continuous sum} of the
$\mathscr{H}(\zeta)$ that we wished to construct. 

\subsection{Measurablility of vector
fields}\label{partIII-chap1-sec1.4} % sec 1.4  

\begin{defi*}
A vector field $X$ is said to be {\em measurable} if for every compact
$K$ and positive $\epsilon$, there exists a set $K_1 \subset K$  such
that $\mu(k-k_1)< \epsilon$ and $X$ is continuous on $K_1$. 
\end{defi*} 

\begin{proposition}\label{partIII-chap1-prop3}% pro 3
A vector field $X$ is measurable if and only if $\langle X,X_n
  \rangle$ is measurable for every $X_n\in\Lambda_0$. 
\end{proposition}
 
If\pageoriginale $X$ is measurable, $\langle X,X_n \rangle $ is
 continuous on $K_1$ 
 and hence $\langle X,X_n\rangle$ is measurable. Conversely, let
 $\langle X,X_n\rangle $ be measurable. Then $|| X ||=\sup
 \dfrac{\mid\langle X,X_n\rangle\mid}{||  X_n ||}$ is also
 measurable. (Here we put $\dfrac{\langle X,X_n \rangle}{|| X_n ||}=0$
 if $|| X_n(\zeta)||=0)$. But $\langle X,X_n\rangle$ are continuous
 outside a set $\{K-K_n\} $  of measure $< \epsilon/2^n$. If $K_\infty=\cap
 K_n$, it is obvious that  $\mu(K-K_\infty)<\epsilon$ and all the functions
 $\langle X,X_n \rangle$ are continuous on $K_\infty$. Hence $||X||$
 is continuous on $K_\infty$. By prop. \ref{partIII-chap1-prop1},
 Ch. \ref{partIII-chap1-sec1.2}, prop. \ref{partIII-chap1-prop3}
 follows.  

In particular, this shows that strong measurablility and weak
measurablility are the same in a separable Hilbert space. 

\begin{proposition}\label{partIII-chap1-prop4} % pro 4
A vector field $X$ belongs to $L^2_\wedge$ if and only if $X$ is
  measurable and $\int || X(\zeta)||^2 d\mu(\zeta)<\infty$. 
 \end{proposition} 
 
The proof is exactly similar to that for ordinary integration of
scalar functions. 

\subsection{Orthogonal basis}\label{partIII-chap1-sec1.5} % sec 1.5

We now find an orthogonal basis for the space of vector fields by
Schmi\-dt's orthogonalisation process. We can of course define 
$$
e_1=\frac{X_1}{|| X_1 ||},\quad e_2=\frac{X_2-\langle X_2,e_1 \rangle
  e_1}{|| X_2-\langle X_2,e_1\rangle e_1 ||},\ldots 
$$
where we put $e_1=0$ whenever the numerator is zero  \cite{key20}. However,
the process is defective as the basic elements are not continuous, and
the following orthogonalisation seems preferable:  

Put 
$$
e_1=X_1;\quad e_2=X_2- \langle X_2, e_1\rangle e_1,
$$
$e_n = x_n$ - orthogonal projection of $X_n$ on the space generated by
$$
e_1,\ldots e_{n-1}
$$


These\pageoriginale are of course continuous vector fields. At each
point $\zeta$, 
the nonzero $e(\zeta)$ from an orthogonal basic for
$\mathscr{H}(\zeta)$. 
 
\subsection{Operator fields}\label{partIII-chap1-sec1.6}
 
Let $X$ be a square summable vector field and  $A(\zeta)$ an operator
on $\mathscr{H}(\zeta)$. Then we may
define $(AX)(\zeta)=A(\zeta)X(\zeta)$ and get another vector field
$AX$. The assignment to each $\zeta$ of an operator of
$\mathscr{H}(\zeta)$  is called an \textit{operator field}. However,
in order that $A$ may act as operator on $L^2_\wedge$, we have to make
sure that $AX$ is also square summable. Obviously some restrictions an
$A(\zeta)$ will be necessary to achieve this. In the particular case
when all the spaces 
$\mathscr{H}(\zeta)$ are the isomorphic, $A(\zeta)$ is a map of
$\mathcal{Z}$ into the set of operators of the Hilbert space
$\mathscr{H}$. Now, Hom $(\mathscr{H},\mathscr{H})$  can be provided
with uniform, strong or weak topologies and we may restrict $A$ to be
continuous for one of these topologies. However, the first topology is
too strong and consequently the space `good operator fields' will
become too restricted to be of any utility. Therefore, in this
particular case, we may require the map $A$ to be continuous in the
weak or strong topology as  
 the case may  be. To transport this definition to the general case,
 we have to reformulate this in a suitable way. Take for instance the
 requirement of strong continuity. This is equivalent to requiring
 that  
(a) $A(\zeta)$ is locally bounded (i.e. bounded on compact sets) and
(b) for every continuous function $f(\zeta)$ of $\mathcal{Z}$ with
values in $\mathscr{H}$, $A(\zeta)f(\zeta)$ is continuous. 

This motivates the following

\begin{defi*} 
An operator field $\zeta \to A(\zeta)$ is said to be {\em strongly
  continuous}\pageoriginale 
if (a) $A(\zeta)$ is locally bounded and (b) for every
continuous vector field $X(\zeta)$, $A(\zeta) X (\zeta)$ is also
continuous. 
\end{defi*}

Similar considerations for weak continuity give us the following 
\begin{defi*}
An operator field $A$ is said to be {\em weakly continuous} if (a) it
is locally bounded, and (b) for any two continuous vector fields 
 $X(\zeta),Y(\zeta)$, the map $\zeta \to \langle
A(\zeta)X(\zeta),Y(\zeta) \rangle$ is continuous.  
\end{defi*}
  
\begin{proposition}\label{partIII-chap1-prop5} % pro 5
An operator field $A$ is {\em strongly continuous} if and only if $A$ is
locally bounded and $\zeta \to A (\zeta)X_n(\zeta)$ is continuous for
every  $X_n \in \wedge_0$. 
\end{proposition}
 
This follows straight from the definition.
 
The strongly continuous operator fields form an algebra which is not
however self adjoint, while the weakly continuous operator fields  do
not even form an algebra. In order to ensure that our definitions are
good  enough, we should know if there exist sufficiently many
non-scalar continuous operator fields. This is answered by the
following 
 
\setcounter{thm}{0}
\begin{thm}\label{partIII-chap1-thm1}%thm 1
{Let} $K$ {be a compact subset of} $\mathcal{Z}$ {and}
$Y_1,\ldots Y_n, Z_1,\ldots Z_n$,  $2n$  {continuous vector fields
  such that} $Y_i(\zeta)$ {\rm are linearly independent for every}
$\zeta \in K.$  {Then there exists a continuous operator field}
$A$ {such that} $A (\zeta)Y_j(\zeta)=Z_j(\zeta)$ {for every}
$\zeta \in K$ {and such that} $A(\zeta)^*$ {is also
  continuous}. 
\end{thm}

We first remark that we can as well assume that $K=\mathcal{Z}$. For,
$Y_i(\zeta)$ are linearly independent if and only if $\Delta= \det
\langle Y_i(\zeta),Y_j(\zeta) \rangle \neq 0$. This being a continuous
function of $K$, there exists a compact neighborhood $V$ of $K$ such
that $| \Delta| \ge \alpha >0$ on $V$.  If the theorem were true
for a compact space, there exists an operator $A_\circ$ on $V$
satisfying the conditions above.\pageoriginale  
Set $A=\varphi A_0$ where $\varphi$ is a
continuous function 1 on $K$, $0$ outside $V$; then $A$ verifies all
the conditions. 
 
Now let $P(\zeta)$ be the space spanned by $Y_i(\zeta)$. We can then
define $A(\zeta)=0$ on the orthogonal complement of $P(\zeta)$ and
$A(\zeta)Y_i(\zeta)=Z_i(\zeta)$ and extend $A$ by linearity. If $\pi$
is the projection of  $\mathscr{H}(\zeta)$ onto $P(\zeta)$, we have $
\langle X,Y_j \rangle  =  \langle \pi X,Y_j \rangle $  and if $\pi
X=\sum \xi_k Y_k$, then $\langle X,Y_j \rangle=\sum \xi_k \langle
Y_k,Y_j \rangle$ with $| \det \langle Y_j,Y_k \rangle| = |
\Delta | \ge \alpha>0$. On solving the linear equations for
$\xi_k$, we get $\xi_k=\Delta_k/ \Delta$.  Since the functions
occurring in the linear equations are continuous, $\xi_k$ are
continuous functions and we have 
$|\xi_k|=|\dfrac{\Delta_k}{\Delta}| \le M|| X ||$ where
$M$ is a constant. Hence $A(\zeta)X(\zeta)=\sum\xi_k(\zeta)Z_k(\zeta)$
is continuous for every continuous  vector field $X$. $A$ is locally
bounded by virtue of the above remark and hence $A$ is a continuous
operator field. 
  
It is obvious that $A^*$ is also locally bounded. Now, $A^*$ maps the
whole of $\mathscr{H}(\zeta)$ onto $P(\zeta)$ and hence
$A^*(\zeta)X(\zeta)=\sum\eta_k(\zeta)Y_k(\zeta)$ 
where the $\eta(\zeta) $ are given by
\begin{align*}
\sum \eta_k(\zeta) \langle   Y_k,Y_j \rangle &= \langle \sum
\eta_k(\zeta)Y_k,Y_j \rangle\\ 
&=\langle A^*(\zeta)X(\zeta),Y_j(\zeta) \rangle\\
&=\langle X(\zeta),A(\zeta)Y_j(\zeta) \rangle\\
&=\langle X(\zeta),Z_j(\zeta) \rangle
\end{align*}

The Gram determinant in this case also is $\Delta$. Hence by the same
argument as before, $A^*$ is a continuous operator field. 

\subsection{Measurablility of operator
fields}\label{partIII-chap1-sec1.7}\pageoriginale % sec 1.7 

\begin{defi*}
An operator field $A$ is said to be {\em measurable} if (a) it is
almost everywhere locally bounded, and (b) for every compact $K$ and
positive  $\epsilon$, there exists $K_1\subset K$ such that
$\mu(K-K_1)<\epsilon$ and $A(\zeta)$ is continuous on $K_1$. 
\end{defi*}

If $(e_p)$  form an orthogonal basis (Ch. \ref{partIII-chap1-sec1.5}),
then any operator 
can be expressed by means of a matrix with respect to this base. The
matrix coefficients are only  $\langle Ae_p, e_q \rangle$. We then
have 

\begin{proposition}\label{partIII-chap1-prop6}% prop 6 
A locally bounded operator field $A$ is measurable if and only if
  the matrix coefficients of $A(\zeta)$ are measurable functions on  
$\mathcal{Z}$. 
\end{proposition}

That a measurable operator field satisfies the above condition is a
trivial consequence of the definition. Conversely, if $ \langle
Ae_p,e_q \rangle$ is measurable for every $q$, by
prop. \ref{partIII-chap1-prop3}, ch. \ref{partIII-chap1-sec1.4}, 
$Ae_p$ is measurable for every $p$. As in
Prop. \ref{partIII-chap1-prop3}, Ch. \ref{partIII-chap1-sec1.4}, we
can find 
a compact set $K_1$ such that the  $Ae_p$ are continuous on $k_1$ and
$\mu(K-K_1)<\epsilon$.    

\subsection{Decomposed operators}\label{partIII-chap1-sec1.8}% sec 1.8

Let $A(\zeta)$ be a measurable operator field bounded almost
everywhere. For every $X (\zeta)\in L^2_\wedge$, $A(\zeta) X (\zeta)$
is also a measurable vector field and $|| A (\zeta) X (\zeta) || \le||
A(\zeta)||_ \infty|| X(\zeta)||$. Hence $\int || A (\zeta)X(\zeta)||^2
d\mu$ exists and we have $|| AX ||_{L_\wedge^2} \le || A
(\zeta)||_\infty|| X ||_{L^2_\wedge}$. 
In other words, $A$ is a continuous operator on $L^2_\wedge$ and $||
A || \le || A (\zeta)||_ \infty$. 

If $A$ is an operator in $L^2_\wedge$ which arises from an operator
field, we say that it is a {\em decomposed operator} and write $A\sim
\int_{\mathcal{Z}}A(\zeta)$. In particular if we take
$A(\zeta)=f(\zeta)$.  
Identity where $f\in L^\infty(\mu)$ we obtain a decomposed operator
$M_f \sim \int f(\zeta) \Id$. This is called a {\em scalar
decomposed\pageoriginale 
  operator} on $L^{2}_{\wedge}$. We denote the space of all such
operators by $\mathscr{M}$. This is a self adjoint subalgebra of $\Hom
(L^{2}_{\wedge}, L^{2}_{\wedge})$. For, 
\begin{align*}
\langle{M_f X, Y} \rangle &= \int \langle f(\zeta)
X(\zeta),Y(\zeta)\rangle d \mu (\zeta) \\ 
&=  \int 
\langle X(\zeta), \overline{f(\zeta)}Y(\zeta)\rangle d\mu(\zeta)\\ 
&= \langle X, M_{\bar{f}}Y\rangle 
\end{align*}

Hence $M^*_f=M_{\bar{f}}$.
 
We now give a characterisation of decomposed operators in terms of this
algebra $\mathscr{M}$ by means of 

\begin{thm}\label{partIII-chap1-thm2}
The set of decomposed operators is precisely the commutator of
$\mathscr{M}$. If
$A \sim \int A (\zeta)$, then $|| A || = ||
A(\zeta)||_{\infty}$. 
\end{thm}

In fact, if $A$ is a decomposed operator, it commutes with all
the elements of $\mathscr{M}$ and hence belongs to $\mathscr{M}'$.

Conversely let $A$ be an operator in $L^{2}_{\wedge}$ which commutes
with $M_f$ for every $f \in L^\infty(\mu)$. Let $K$ be a compact subset
of $\mathcal{Z}$. Then $\chi_{_K}e_n(\zeta) \in L^{2}_{\wedge}$, $\{e_n\}$
being the orthogonal basis (Ch. \ref{partIII-chap1-sec1.5}). Let $H$ be a compact
nighbourhood of $K$. Then 
\begin{align*}
A(\chi_{_K}e_n) & = A (\chi_{_K}\chi_{_H}e_n)\\ 
& = AM_{\chi_{_K}}(\chi_{_H}e_n)\\ 
& = M_{\chi_{_K}}A(\chi_{_H}e_n) \text{~ by assumption}\\ 
& =\chi_{_K}A(\chi_{_H}e_n) \text{~ almost everywhere}. 
\end{align*}

It is obvious that at the intersection of any two compact sets $K,K^1$,
the vector fields $A (\chi_{_K} e_n)$ and $A (\chi_{_K^1} e_n)$ coincide
almost everywhere.\break Hence there exists a vector field $A(e_n)$ such
that $\chi_{_K}(e_n) A (e_n)(\zeta) =\break A(\chi_{_K}e_n)(\zeta)$ almost
everywhere. Thus we have a countable family of relations and hence
there exists a set $N$ of measure $0$ such that
$\chi_{_K}A(e_n)(\zeta)= A(\chi_{_K}e_n)(\zeta)$\pageoriginale for every
$\zeta \notin N$. Since the $X_n$ in the fundamental sequence $\wedge_0$ are finite
linear combinations of the $e_n$ we have $\chi_{_K}A(X_n)(\zeta)= A
(\chi_{_K}\break X_n)(\zeta)$ for every $\zeta\notin N$. Also $\int_{K} || A
(X_n(\zeta))||^2 d\mu(\zeta)=|| A(\chi_{_K}X)||^2 \le ||A||^2 \int_K ||
X_n ||^2\mu(\zeta)$ for every compact set $K$. Hence the set of
elements $\zeta$ such that $|| A(X_n)(\zeta)||$ is strictly greater
than $||A||~|| X_n(\zeta)||$ is of measure zero. 

We have thus proved that the algebra of all decomposed operators 
is $\mathscr{M}'$ and we know that
$\mathscr{M} \subset \mathscr{M}''$. We can moreover assert 

\begin{thm}\label{partIII-chap1-thm3}
$\mathscr{M}$ is a weakly closed algebra.
\end{thm}

In fact, let $(e_n)$ be an orthogonal basis of $L^{2}_{\wedge}$. Then
we define for any two integers $p$, $q$, a decomposed operator $H_{p,q}$
by setting 
\begin{align*}
& H_{p,q}(\zeta)e_n(\zeta)=0 if n \neq p \text { or }  q\\
& H_{p,q}(\zeta)e_p(\zeta)=|| e_p(\zeta) ||^{2}e_q(\zeta).\\
\text{and~ } & H_{p_q}(\zeta) e_q(\zeta)= || e_q(\zeta)||^{2}e_p(\zeta).
\end{align*}

If $B$ commutes with every $A\in \mathscr{M}'$, in particular, it
commutes with all 
the $H_{p,q}$ therefore $B(\zeta)$ commutes with $H_{p,q}(\zeta)$ for
every fixed $p$, $q$ and almost every $\zeta$. Since the $H_{p,q}$ are
countable, $B(\zeta)$ commutes with $H_{p,q}(\zeta)$ for almost every
$(\zeta)$ and for all $p$, $q$. Hence outside a set of measure zero, we
have 
\begin{align*}
\langle B(\zeta)e_p, e_p \rangle &= \frac {\langle Be_p,H_{n,q},e_n
  \rangle }{|| e_n ||^2}\quad \text{for every~ } p,q.\\ 
&=\langle BH_{n,q} \; e_p, \; \frac{e_n}{|| e_n||^2}\rangle\quad \text{(since
the~ } H_{n,q} \text{~ are Hermitian)}\\ 
&=0 \text{~ if~ } p \neq q. 
\end{align*}

On\pageoriginale the other hand,
\begin{align*}
\langle B(\zeta)e_p,e_p\rangle &= \langle
BH_{n,p}e_p,\frac{e_n}{|| e_n||^2}\rangle\\ 
&=\frac{|| e_p||^2}{|| e_n ||^2}\langle Be_n,e_n\rangle 
\end{align*}

This shows that $B(\zeta)$ is a scalar operator for almost every
$(\zeta)$. Hence $B=\int B(\zeta)d(\zeta)$ is a scalar decomposed
operator and hence $\epsilon \mathscr{M}$. This shows that
$\mathscr{M}=\mathscr{M}''$ and by Th. \ref{partII-chap5-thm2},
Ch. \ref{partII-chap5-sec5.6}, Part \ref{partII}, is weakly closed.  
