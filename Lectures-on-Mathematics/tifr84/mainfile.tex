\documentclass[a4paper,11pt]{book}
\usepackage{amsmath}
\usepackage{amsthm,mathrsfs,array}
\usepackage[dvips,breaklinks]{hyperref}
\usepackage{txfonts}
\usepackage[all]{xy}
\usepackage{multirow}
\usepackage{epsfig}
\usepackage{float,longtable,enumerate,amsxtra}
\usepackage{makeidx,multind}
\floatstyle{plain}
\floatplacement{Figure}{H}
\input macros
\textwidth=11cm
\textheight=17cm
\allowdisplaybreaks

\hfuzz=1pt

\makeatletter

\def\@makechapterhead#1{%
  \vspace*{50\p@}%
  {\parindent \z@ \raggedright \normalfont
    \ifnum \c@secnumdepth >\m@ne
      \if@mainmatter
        \huge\bfseries \@chapapp\space \thechapter
       \par\nobreak
        \vskip 20\p@
      \fi
    \fi
    \interlinepenalty\@M
    \Huge \bfseries #1 \par\nobreak
    \vskip 40\p@
  }}

\c@tocdepth=1
\makeatother

\makeindex{A}
\makeindex{B}

\begin{document}

\frontmatter

\input mainpage

\input preface

\tableofcontents

\mainmatter

\input chapter1
\input chapter2
\input chapter3
\input chapter4
\input chapter5
\input chapter6
\input chapter7
\input appendix-A

\clearpage

\printindex{A}{Index: First some notions for which we refer to textbooks, as indicated.}  
\printindex{B}{Index: Now the terms that are explained in the notes.}

\end{document}
