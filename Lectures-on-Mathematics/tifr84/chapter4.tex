\chapter{Donkin's Conjecture}\label{chap4}

Let\pageoriginale $k$\label{page33} now be an algebraically closed field of {\em
  positive characteristic $p$}, and let $G$ be our connected reductive
group over $k$. Let $M$ be $G$-module. A filtration $\F$ of $M$ is
called {\em good} if the successive quotients are isomorphic to a
direct sum of copies of $P(\mu)$ with $\mu\in X(T)^{+}$. In this
chapter we prove Donkin's conjecture for good filtrations. The best
known half of this conjecture is the (older) conjecture stating that
for $\lambda$, $\mu\in X(T)^{+}$, $P(\lambda)\otimes P(\mu)$ has good
filtration. The crucial idea (due to O. Mathieu) is to study the
$G$-modules which are embedded in a graded $B$-algebra with a
``canonical splitting''.

In the first section we give the definition and basic properties of
good filtration. We also give the relationship between the excellent
filtrations and good filtrations.

In teh second section we give a criterion for existence of a good
filtration for a $G$-module. This criterion works in a very
specialized case of a $G$-module embedded inside a graded $B$-algebra
each of whose graded components has an excellent filtration and only
one weight in its socle. However, as we will see in the last section,
this criterion gives us the proof of Donkin's conjecture.

This criterion leads us to study what we call Frobenius-linear
endomorphisms of a graded $k$-algebra $R$. A {\em splitting} $\sigma$
of $R$ is a Frobenius-linear endomorphism such that $\sigma(1)=1$. The
Frobenius splittings were introduced by Mehta and Ramanathan in
\cite{key24}. Following Mathieu, we then introduce the notion of a
canonical splitting of $R$ and prove the crucial proposition that the
image of a $B$-submodule of $R$ under a canonical splitting is again a
$B$-submodule. 

The criterion for good filtration relates the concept of canonical
splitting and\pageoriginale that\label{page34} of good filtration. This gives the
proof of Donkin's conjecture.

\section{Good Filtrations}\label{chap4-sec4.1}

\begin{definition}\label{chap4-defi4.1.1}
Let $M$ be a $G$-module. A filtration $\F=F_{0}\subset F_{1}\subset
\ldots$ of $M$ by $G$-submodules is said to be a 
{\em good filtration}\index{B}{good filtration} if
\begin{itemize}
\item[(i)] $\cup_{i}F_{i}=M$.

\item[(ii)] $F_{i}/F_{i-1}\simeq \oplus P(\mu_{i})$ with $\mu_{i}\in
  X(T)^{+}$. 
\end{itemize}
\end{definition}

The reader may have noticed the similarity between excellent
filtration of a $B$-module and good filtration of a $G$-module. Indeed
the questions of a $B$-module $M$ having excellent filtration and
$\ind^{G}_{B}(M)$ having good filtration are related. First we see
what happens if $\ind^{G}_{B}(M)=M$.

\begin{exercise}\label{chap4-exer4.1.2}
Let $M$ be a $G$-module. Show that the length-height filtration of $M$
is not just a filtration by $B$-submodules, but one by
$G$-submodules. (Hint: Consider a minimal counterexample and factor
out an irreducible $G$-submodule.)
\end{exercise}

\begin{exercise}\label{chap4-exer4.1.3}
Let $M$ be a $G$-module. Prove that the following are equivalent:
\begin{itemize}
\item[(i)] $M$ has a good filtration.

\item[(ii)] $M$ has an excellent filtration. (That is,
  $\res^{G}_{B}(M)$ has one, but recall from \ref{chap2-coro2.1.7}
  that we embed $\C_{G}$ in $\C_{B}$.)

\item[(iii)] The length-height filtration of $M$ is a good filtration.
\end{itemize}
\end{exercise}

\begin{remark}\label{chap4-rem4.1.4}
As the property of having excellent filtration is closed under
extension, we see that the property of having a good filtration is
also closed under extension.
\end{remark}

We also have a cohomological criterion for good filtration which is
analogous to the one for existence of an excellent filtration. (It is
much older.) 

\begin{proposition}[Donkin]\label{chap4-prop4.1.5}
(\cite[II 4.16]{key11}) Let\pageoriginale 
  $M$\label{page35} be a $G$-module. Then $M$ has a good
  filtration if and only if for every dual Weyl 
module\index{B}{dual Weyl module} $P(\lambda)$,
  $\lambda\in X(T)^{+}$, one has $H^{1}(G,M\oplus P(\lambda))=0$.
\end{proposition}

\begin{corollary}\label{chap4-coro4.1.6}
Let $M=M_{1}\oplus M_{2}$ be a direct sum of two $G$-modules. Then $M$
has good filtration if and only if both $M_{1}$ and $M_{2}$ have good
filtration. 
\end{corollary}

\begin{exercise}\label{chap4-exer4.1.7}
Use Lemma \ref{chap3-lem3.2.11} to show that if $M$ has excellent
filtration, $\ind^{G}_{B}(M)$ has good filtration.
\end{exercise}

\section{Criterion for Good Filtrations}\label{chap4-sec4.2}

In this section we give a criterion for existence of good
filtrations. Unlike the cohomological criterion, which is intrinsic,
this criterion depends upon an embedding of the given $G$-module into
a graded $B$-alge\-bra. To motivate this approach we look at Donkin's
conjecture. 

\begin{remark}\label{chap4-rem4.2.1}
Donkin's conjecture claims that for $\lambda$, $\mu$ two dominant
characters the module $P(\lambda)\oplus P(\mu)$ has good
filtration. Now geometrically $P(\lambda)\oplus P(\mu)$ can be
interpreted as $P(\lambda)\oplus P(\mu)=H^{0}(G/B\times G/B,\L)$ where
$\L$ is the line bundle $\L(w_{0}\lambda)\times \L(w_{0}\mu)$ on
$G/B\times G/B$. The variety $G\times^{B}G/B=G/B\times G/B$ contains
$Bw_{0}B\times^{B}G/B$ as an open subset. Therefore the natural
restriction map gives a natural embedding of $P(\lambda)\oplus P(\mu)$
into the graded $B$-algebra
$\oplus^{\infty}_{j=0}H^{0}(Bw_{0}B\times^{B}G/B,\L^{j})$. This
$B$-algebra is induced from the $T$-module
$\oplus_{j}H^{0}(Tw_{0}B\times^{B}G/B,\L^{j})$ and is therefore
injective. Hence by the cohomological criterion, it has excellent
filtration. 
\end{remark}

Motivated by this remark, we state the following criterion for good
filtration.\index{B}{cohomological criterion!for good filtration} First a definition.

\begin{definition}\label{chap4-defi4.2.2}
Let $A=\oplus_{i}A^{i}$ be a graded $B$-algebra. We define a
$B$-subalgebra $A_{\leq \lambda}$ of $A$ by
$A_{\leq\lambda}=\oplus_{i}A^{i}_{\leq i\lambda}$. (Recall that
$M_{\leq \mu}$ is the largest $B$-submodule of $M$ which is in the
category $\C_{\leq \mu}$.)
\end{definition}

\begin{theorem}[$p$-root closure and good filtration]\index{B}{p@$p$-root closure}\label{chap4-thm4.2.3}
Let $A=\oplus_{i}A^{i}$ be a graded $B$-algebra such that 
\begin{itemize}
\item[\rm(i)] $A^{0}=k$.\pageoriginale

\item[\rm(ii)] $A$\label{page36} has excellent filtration.

\item[\rm(iii)] There exists $\lambda\in X(T)^{+}$ such that in
  $\soc(A^{j})$ only $j\cdot\lambda$ occurs as weight.
\end{itemize}

Let $S$ be graded subalgebra which is a graded $G$-module and which is
$p$-root closed (i.e. $a^{p}\in S\Rightarrow a\in S$). Then $S$ has
good filtration.
\end{theorem}

\begin{proof}
We wish to prove that each $S^{r}$ has good filtration and we may
restrict attention to a given $r$. We know by the cohomological
criterion for excellent filtration (Theorem \ref{chap3-thm3.2.7}) that
each of $A^{j}$ has excellent filtration. Therefore for any $m$, the
rescaled $B$-algebra $A_{1}=\oplus_{i}A^{m-i}$ with $A^{i}_{1}=A^{im}$
also has excellent filtration. Therefore we may assume that $S^{1}\neq
0$.

The socle of $S^{i}$ contains only a single weight $i\lambda$. Further
as $S$ is a $G$-module, $i\lambda$ is an extremal weight of $S^{i}$
and all extremal weights are in the same Weyl group orbit as
$i\lambda$.

Therefore we have $S\subset A_{\leq \lambda}$.

The length-height filtration of $A$ is excellent. Further as the socle
of $A^{j}$ has no other weight than $j\cdot\lambda$, we see that the
first non-zero module in this filtration of $A^{j}$ is $(A^{j})_{\leq
  j\lambda}$. Therefore $(A_{\leq \lambda})^{j}$ is isomorphic to
$\oplus P(j\cdot \lambda)$.

To get a firm hold of the situation we need a technical sublemma that
gives more insight in the algebra structure of $A_{\leq
  \lambda}$. That will allow us to pass to convenient subalgebras. The
reader is advised to pass over this sublemma quickly.
\end{proof}

\begin{sublemma}\label{chap4-sublem4.2.4}
The graded $B$-algebra $A_{\leq \lambda}$ may be reconstructed from
its ``subalgebra of socles''\index{B}{subalgebra of socles} 
$\oplus_{j}\soc_{B}(A^{j})$. More
generally, any graded subalgebra of $\oplus_{j}\soc_{B}(A^{j})$ is the
subalgebra of socles of a suitable graded subalgebra $\widetilde{A}$
of $A_{\leq\lambda}$, with $\widetilde{A}$ having excellent filtration.
\end{sublemma}

\begin{proof}
We have seen already that $(A_{\leq \lambda})^{j}$ is isomorphic as a
$B$-module to a direct sum of cipies of $P(j\lambda)$, with the number
of copies equal to the dimension of $\soc((A_{\leq\lambda})^{j})$. To
say it more canonically-which one must, in view of the task at
hand--there is a canonical isomorphism of $B$-modules
$$
P(j\lambda)\otimes \soc ((A_{\leq \lambda})^{j})\otimes
k_{-j\lambda}\to (A_{\leq \lambda})^{j}.
$$

So\pageoriginale that\label{page37} is how we reconstruct $A_{\leq \lambda}$ as a
$B$-module. To get the ring structure, note that multiplication is
given by $B$-module maps
$$
(A_{\leq \lambda})^{r}\otimes (A_{\leq \lambda})^{s}\to (A_{\leq
  \lambda})^{r+s}.
$$

Thus we are done with the first half of the lemma if we show that
restriction defines an isomorphism from
$$
\Hom_{B}((A_{\leq \lambda})^{r}\otimes (A_{\leq \lambda})^{s},(A_{\leq
  \lambda})^{r+s}) 
$$
to
$$
\Hom_{B}(\soc((A_{\leq \lambda})^{r})\otimes
\soc((A_{\leq\lambda})^{s}),\soc((A_{\leq \lambda})^{r+s})).
$$

For surjectivity one uses Polo's theorem with $R$ equal to the length
of $(r+s)\lambda$. To see injectivity, consider a map
$$
f:(A_{\leq \lambda})^{r}\otimes (A_{\leq \lambda})^{s}\to (A_{\leq
  \lambda})^{r+s} 
$$
in the kernel. If $f$ is not zero, its image must hit the socle of
$(A_{\leq \lambda})^{r+s}$. But then it must be non-zero on the weight
space $((A_{\leq \lambda})^{r}\otimes (A_{\leq
  \lambda})^{s})_{(r+s)\lambda}$. And that is just
$\soc((A_{\leq\lambda})^{r})\otimes \soc((A_{\leq \lambda})^{s})$ as
one sees by looking at lengths and heights. The rest of the sublemma
follows similarly.
\end{proof}

Encouraged by the sublemma we let $I(S)$ denote the graded subalgebra
of $A_{\leq \lambda}$ whose $j^{\text{th}}$ component is the injective
hull of $S^{j}$ in the category $\C_{\leq j\cdot\lambda}$. The
subalgebra $I(S)\subset A_{\leq \lambda}$ clearly has excellent
filtration. In fact $I(S)$ is a direct summand of $A_{\leq \lambda}$
and thus the filtration from its grading is an excellent filtration!
Therefore we replace $A$ by $I(S)$.

We will prove that $S=A$.

We can assume, by rescaling again if necessary, that $S^{1}\neq
I(S^{1})$. Therefore there exists a copy of $P(\lambda)\subset
I(S^{1})$ such that $S^{1}$ does not contain $P(\lambda)$. We denote
by $A_{1}$ the algebra generted by this $P(\lambda)$. Phrased
differently, we let $A_{1}$ be the graded subalgebra with excellent
filtration whose socle algebra is generated by the socle of our chosen
copy of $P(\lambda)$. Let $A_{2}=A_{1}\cap S$. The $G$-algebra $A_{2}$
is again $p$-root closed in $A_{1}$. One may quickly dispense of the
case that $A^{j}_{1}=0$ for large $j$.

We choose a parabolic subgroup $P$ such that $\lambda$ extends as a
character to $P$ and $P$ is maximal for this property. The line bundle
$\L=\L(w_{0}\cdot\lambda)$ is very ample\index{B}{ample}\index{B}{very ample} on $G/P$ \cite[II
  8.5]{key11}. Further, we have $H^{0}(G/B,\L)=H^{0}(G/P,\L)$. 

Thus\pageoriginale we\label{page38} can restrict our attention to the situation
$$
S=A_{2}\subset A_{1}=A=\bigoplus_{i}H^{0}(G/P,\L^{i}).
$$

Consider the projective space $\mathbb{P}(A^{1})$ of one-dimensional
{\em quotients} of $A^{1}$. We have a rational map\index{A}{rational map [7]}
$f:\mathbb{P}(A^{1})\to \mathbb{P}(S^{1})$. However, the image of
$G/P$, under the canonical embedding, lies inside the domain of this
map. Therefore we get a morphism $f:G/P\to IMAGE\hookrightarrow
\mathbb{P}(S^{1})$. The space $IMAGE$ is a $G$-space (a homogeneous
space) and we claim the map $f$ is bijective from $G/P$ to
$IMAGE$. Indeed let us inspect the stabilizer $Q$ of the image of
$x=P/P$. This is the stabilizer in $(S^{1})^{*}$ of a line $L$
stabilized by $P$. So $Q$ is a parabolic subgroup containing $P$ and
by the classification of parabolic subgroups containing $B$ we only
have to check which elements of the Weyl group stabilize $L$. That is
easy, as $L$ has weight $-w_{0}\lambda$. Note that things would be
much more subtle if we needed the scheme theoretic stabilizer \cite[I
  2.6]{key11} of $x$. We do not need it as we do not claim our
bijection is an isomorphism of varieties.

Next, we recall a lemma from algebraic geometry. The lemma is not
stated in its full generality but only in a form which will be useful
to us. The proof is given in the Appendix (cf.\@ Sublemma \ref{sublem-A.5.1}). We
wish to apply it with the line bundle $\L\approx \O(1)$ on {\em
  IMAGE}. Alternatively one may apply Sublemma \ref{sublem-A.5.1} to the structure
sheaf of $\Spec(k[S^{1}])=$ the affine cone over {\em IMAGE}.

\begin{sublemma}\label{chap4-lem4.2.5}
Let $X$, $Y$ be two projective varieties over $k$ and let $f:X\to Y$
be a morphism which is bijective. Then for every ample line bundle
$\L$ on $Y$ and for $s\in H^{0}(X,f^{*}(\L))$ we have $s^{p^{n}}\in
H^{0}(Y,\L^{p^{n}})$ for some large $n$.
\end{sublemma}

Therefore (cf.\@ \cite[II 7]{key7}) for every $a\in A^{1}$, we have
$a^{p^{m}}\in S$ for some large $m$. Now using the $p$-root closure of
$S$ we see that $a\in S$. Thus $A^{1}\subset S$, a
contradiction.\hfill$\Box$

\begin{remark}\label{chap4-rem4.2.6}
There is another way to understand why $a^{p^{m}}\in S$ for some large
$m$. Namely, scheme theoretically the stabilizer $Q$ is generated by
$P$ and some connected infinitesimal subgroup. This connected
infinitesimal subgroup\pageoriginale is\label{page39} contained in a Frobenius
kernel \cite{key11} and thus acts trivially on $a^{p^{m}}$ for some
large $m$.
\end{remark}

\section{Frobenius Splittings}\label{chap4-sec4.3}

In this section we define Frobenius splittings and introduce the
canonical splittings.

Let $R$ be a $k$-algebra.

\begin{definition}\label{chap4-defi4.3.1}
A Frobenius-linear endomorphism of $R$ is a map $\sigma:R\to R$ such
that for $a$, $b\in R$,
\begin{itemize}
\item[\rm(i)] $\sigma(a+b)=\sigma(a)+\sigma(b)$

\item[\rm(ii)] $\sigma(a^{p}b)=a\cdot\sigma(b)$
\end{itemize}

We denote the space of Frobenius-linear\index{B}{Frobenius-linear} 
endomorphisms by $\End_{F}\break (R)$.
\end{definition}

\begin{definition}\label{chap4-defi4.3.2}
\begin{enumerate}
\item A Frobenius-linear endomorphism $\sigma$ is called a {\em
  splitting}\index{B}{splitting} 
if $\sigma(a^{p})=a$. This means $\sigma$ is a splitting
  if and only if $\sigma(1)=1$.

\item Let $I$ be an ideal of $R$. We say a $\sigma\in \End_{F}(R)$ is
{\em compatible} with $I$ if and only if $\sigma(I)\subset I$. We
denote the space of such endomorphisms by $\End_{F}(R,I)$.

\item We say $I$ is {\em compatibly split}\index{B}{compatibly split} 
in $R$ if there exists a
  splitting $\sigma$ of $R$ such that $\sigma\in \End_{F}(R,I)$.
\end{enumerate}
\end{definition}

\begin{definition}\label{chap4-defi4.3.3}
Let $R$ be a $k$-algebra. For $a\in R$ and $\sigma\in \End_{F}(R)$ we
define $a\ast \sigma$ by
$$
a\ast \sigma(b)=\sigma(a\cdot b).
$$
\end{definition}

\begin{definition}\label{chap4-defi4.3.4}
Let $A=\oplus_{i\geq 0}A^{i}$ be a graded $B$-algebra. A
$\sigma\in \End_{F}(A)$ is called {\em graded}\index{B}{graded splitting} if
$\sigma(A^{ip})\subset A^{i}$ and $\sigma(A^{i})=0$ if $i$ is not
divisible by $p$.
\end{definition}

In case $R$ is a $G$-module, we define a $G$ action on $\End_{F}(R,I)$
by
$$
(g\ast\sigma)(a)=g\cdot \sigma(g^{-1}\cdot a).
$$

Let\pageoriginale $R$\label{page40} be a $B$-algebra. Then under the $*$ action, the
module $\End_{F}(R)$ is a $B$-module, possibly not rational. Now $B$
is generated by the torus $T$ and the one-parameter subgroups
$U_{\alpha}=\{x_{\alpha}(t)\mid t\in k\}$ with $\alpha$ a simple
root. Every $\sigma\in \End_{F}(R)$ defines a map $B\to \End_{F}(R)$
be $b\mapsto b\ast \sigma$. If the $B$-module $\End_{F}(R)$ is finite
dimensional, one expects this to define a polynomial map on each of
the subgroups $U_{\alpha}$. A $T$-invariant splitting is canonical if
an even stronger condition is true.

\begin{definition}\label{chap4-defi4.3.5}
A splitting $\sigma\in \End_{F}(R)$ (or $\sigma\in \End_{F}(R,I))$ is
called {\em canonical} if for every simple root $\alpha$, there exist
$\sigma_{r,\alpha}\in \End_{F}(R)$ such that
\begin{itemize}
\item[(i)] $h\ast\sigma=\sigma$ for every $h\in T(k)$.

\item[(ii)] $x_{\alpha}(t)\ast
  \sigma=\sum\limits^{p-1}_{r-0}t^{i}\ast\sigma_{r,\alpha}$ for every
  simple root $\alpha$ and every $t\in k$.
\end{itemize}

Here it is important that the summation stops at $p-1$.
\end{definition}

\begin{remark}\label{chap4-rem4.3.6}
If $R$ is a $B$-algebra and $\sigma$ a canonical
splitting\index{B}{canonical splitting} on $R$, 
then $\sigma$ takes weight vectors of weight $p\lambda$ to weight
vectors of weight $\lambda$. Therefore $\sigma(R_{\mu})=0$ if
$\frac{1}{p}\mu$ is not a weight of $R$.
\end{remark}

The following proposition underlines the importance of canonical
splittings. 

\begin{proposition}[Key Proposition]\label{chap4-prop4.3.7}
Let $\sigma$ be a canonical splitting of the $B$-algebra $R$. Then the
image under $\sigma$ of a $B$-submodule of $R$ is again
$B$-invariant--and thus a $B$-submodule.
\end{proposition}

\begin{proof}
Let $v$ be in a $B$-submodule $N$. Recall that one may write
$x_{\alpha}(t)v$ as a polynomial $\sum_{i\geq
  0}t^{i}X^{(i)}_{\alpha}v$. This explains the notation
$X^{(i)}_{\alpha}$ in what follows.


We write $z(t)=(x_{\alpha}(-t^{p})\ast \sigma)(v)$ in two ways. On the
one hand we have
$$
z(t)=\sum_{i\geq 0}(-t^{p})^{i}X^{(i)}_{\alpha}\sigma\left(\sum_{j\geq
  0}t^{jp}X_{\alpha}^{(j)}v\right)=\sum_{i,j\geq
  0}t^{ip+j}(-1)^{i}X^{(i)}_{\alpha}\sigma(X^{(j)}_{\alpha}v). 
$$

On\pageoriginale the\label{page41} other hand, as $\sigma$ is canonical, $z(t)$
equals
\begin{align*}
\sum^{p-1}_{r=0}((-t^{p})^{r}\ast\sigma_{r,\alpha})(v) &=
\sum^{p-1}_{r=0}(\sigma_{r,\alpha}((-t^{p})^{r}v))\\ 
&= \sum^{p-1}_{r=0}(-t)^{r}(\sigma_{\alpha,r}(v)).
\end{align*}

Write $z(t)=\sum_{n\geq 0}z_{n}t^{n}$. Then $\sigma(v)=z_{0}$. From
the second expression oen sees that the other $z_{pn}$ vanish, so
$\sigma(v)=\sum_{n\geq 0}z_{pn}t^{pn}$. Now we use the first
expression to rewrite this as
$$
\sum_{i,s\geq 0}t^{ip+sp}(-1)^{i}X^{(i)}_{\alpha}\sigma(X^{(ps)}_{\alpha}v).
$$

But that is just
$$
x_{\alpha}(-t^{p})\sum_{s\geq
  0}\sigma((t^{p})^{ps}X^{(ps)}_{\alpha}v),
$$
whence the result that $x_{\alpha}(t^{p})\sigma(v)$ is in
$\sigma(N)$. Now just substitute $t$ for $t^{p}$. (We have $t$ vary
over an algebraically closed field). We conclude that $\sigma(N)$ is
invariant under all $x_{\alpha}(t)$ with $\alpha$ simple. It is more
or less built into the definition of canonical that $\sigma(N)$ is
also invariant under $T(k)$. Now use that $B(k)$ is generated by
$T(k)$ and the above $x_{\alpha}(t)$.
\end{proof}

This proposition together with Remark \ref{chap4-rem4.3.6} immediately
gives us the following corollary. Here $A_{<\lambda}$ is the obvious
variation on $A_{\leq \lambda}$. It equals
$\oplus_{i}A^{i}_{<i\lambda}$, where $A^{i}_{<i\lambda}$ is the
largest $B$-submodule of $A^{i}$ which is in the category
$\C_{<i\lambda}$ consisting of all $B$-modules with weights strictly
preceding $i\lambda$ in length-height order.

\begin{corollary}\label{chap4-coro4.3.8}
If $\sigma$ is a graded canonical splitting on $A$, then we have
$\sigma(A_{\leq \lambda})\subseteq A_{\leq\lambda}$ and
$\sigma(A_{<\lambda})\subseteq A_{<\lambda}$.
\end{corollary}

The Remark \ref{chap4-rem4.2.1} motivates us to look for geometric
examples of splitings and in particular canonical splittings. The
Frobenius-linear endomorphisms have the following geometric extension.

Let $X$ be a variety over $k$. Let $F:X\to X$ denote the {\em absolute
  Frobenius morphism, i.e.} the morphism which on $\O_{X}$ restricts
to the morphism induced by\pageoriginale taking\label{page42} $p^{\text{th}}$ power. This
morphism is identity on the underlying topological space. However, on
functions, it takes a given function to its $p^{\text{th}}$ power.

We define $\cEnd_{F}(X)$ -- sheaf of Frobenius-linear endomorphisms --
by assigning the abelian group $\End_{F}(\O_{X}(U))$ to each open
$U$. Let $F_{\ast}\O_{X}$ be the direct image\index{A}{direct image [7]} of $\O_{X}$. As a sheaf
of abelian groups, the sheaf $F_{\ast}\O_{X}$ is isomorphic to
$\O_{X}$. The $\O_{X}$-module structure of $F_{\ast}\O_{X}$ is via the
Frobenius morphism. We therefore have $a\cdot s=a^{p}s$ for $a\in
\O_{X}$ and $s\in F_{\ast}\O_{X}$. Thus,
$\cEnd_{F}(X)=(F_{\ast}\O_{X})^{*}$, the dual of
$F_{\ast}\O_{X}$. This gives an $\O_{X}$-module structure on
$\cEnd_{F}(X)$. We denote the space of global sections of
$\cEnd_{F}(X)$ by $\End_{F}(X)$. We get
\begin{align*}
\End_{F}(X) &= H^{0}(X,\cEnd_{F}(X))\\
&= H^{0}(X,(F_{\ast}\O_{X})^{\ast}).
\end{align*}

\begin{definition}\label{chap4-defi4.3.9}
A variety $X$ over $k$ is called 
Frobenius split\index{B}{Frobenius split variety} if there exists
$\sigma\in \End_{F}(X)$ which is a splitting.
\end{definition}

If $X$ is a $G$-variety, we can give a $G$-structure to $\End_{F}(X)$
by $(g\ast \sigma)(s)=g\cdot\sigma(g^{-1}\cdot s)$ for $s\in \O_{X}$.

The operation $\ast$ defined before gives another $\O_{X}$-module
structure on $\cEnd_{F}(X)$. We see that this $\O_{X}$-module
structure is obtained by using the isomorphism between
$F_{\ast}\O_{X}$ and $\O_{X}$ as abelian groups. If $X$ is smooth,
then the sheaf $\cEnd_{F}(X)$ is isomorphic to a line bundle under the
$*$ operation. This is best seen by passing to the completion at a
point, which makes things very computable. (Recall the completion of
the local ring at a smooth point is just a power series ring in a
number of variables.)

Let $Y$ be a closed subvariety\index{B}{compatibly split!subvariety} of $X$. Let $\I$ be the sheaf of ideals
defining $Y$. We define the sheaf of Frobenius-linear endomorphisms
which are {\em compatible} with $Y$ by assigning the abelian group
$$
\End_{F}(\O_{X}(U),\I(U))
$$ 
to any open subset $U$ of $X$. We denote
this sheaf by $\cEnd_{F}(X,Y)$ and its space of global sections by
$\End_{F}(X,Y)$. 

\begin{definition}\label{chap4-defi4.3.10}
A closed subvariety $Y$ is said to be compatibly split in $X$ if there
exists a splitting $\sigma\in \End_{F}(X,Y)$.
\end{definition}

We next list certain properties of splittings and canonical splittings
which are useful to us. 

\medskip
\noindent
{\bf Direct images:}\pageoriginale
\begin{enumerate}
\item Let\label{page43} $f:Z\to X$ be a morphism such that
  $f_{\ast}\O_{Z}=\O_{X}$. Suppose $\sigma$ is a splitting on $Z$ such
  that $\sigma$ compatibly splits $Y\subset Z$. Then there exists a
  splitting on $X$ which compatibly splits $f(Y)$.

\item If moreover $Z$, $Y$, $X$ are $B$-varieties, $f$ is a
  $B$-equivariant morphism and $\sigma\in \End_{F}(Z,Y)$ is canonical,
then the induced splitting in $\End_{F}(X,f(Y))$ is also canonical.
\end{enumerate}

\begin{lemma}\label{chap4-lem4.3.11}
Let $\sigma\in \End_{F}(X)$ be a splitting and $\L$ a line bundle on
$X$. Then $\sigma$ extends uniquely to a graded splitting of
$R(\L)=\oplus_{i\geq 0}H^{0}(X,\L^{i})$.
\end{lemma}

\begin{proof}
Let $V\subset X$ be such that $V=\Spec A$ is affine and $\L|_{V}$ is
trivial. Then $R(\L)$ is a polynomial ring $A[T]$. We first prove that
a splitting of $A$ extends uniquely to a graded splitting of
$A[T]$. We define $\tilde{\sigma}_{V}$ by
$\tilde{\sigma}_{V}(\sum_{i\geq 0}a_{i}T^{i})=\sum_{i\geq
  0}\sigma(a_{ip})T^{i}$. It is clear that any splitting of $A[T]$
which restricts to $\sigma$ on $A$ and which is graded has to satisfy
this equation. Therefore this extension is unique. It is this
uniqueness that allows us to patch these local sections
$\tilde{\sigma}_{V}$ to get a splitting of $R(\L)$. 
\end{proof}

\begin{remark}\label{chap4-rem4.3.12}
For a $B$-variety $X$ and equivariant line bundle $\L$, the extension
of a canonical splitting will again be canonical.
\end{remark}

Let $G$ be our reductive algebraic group over $k$, with $B$ (and
$T\subset B$) a Borel (and torus) subgroup of $G$. We now consider the
special case of $X=G/B$. We will prove that the Demazure
desingularisation $Z$ of $G/B$, introduced in the second section of
the first chapter, has a canonical splitting. Therefore using the
direct image property of splittings, $G/B$ itself will have a
canonical splitting.

Let $W$ be the Weyl group of $G$. Let $s_{1}\ldots s_{n}$ be a reduced
expression for the longest element $w_{0}$ in $W$. For each $s_{i}$,
we have a minimal parabolic subgroup $P_{i}$ of $G$. Then,
$Z_{n}=P_{1}\times^{B}P_{2}\times^{B}\cdots\times^{B}P_{n}/B$ is
called the Demazure desingularisation of $G/B$. The multiplication map
$m:P_{1}\times\cdots\times P_{n}\to G$ induces a morphism
$\varphi:Z_{n}\to G/B$. The morphism $\varphi$ is birational. Thus as
$G/B$ is a normal variety, we get $\varphi_{\ast}\O_{Z_{n}}=\O_{G/B}$
(\cite[II Lemma 14.5]{key11}).

\begin{remark}\label{chap4-rem4.3.13}
Later\pageoriginale we\label{page44} will also have use for $Z_{n}$ when $n$ is more
than the number of positive roots. Then of course $s_{1}\ldots s_{n}$
will not be a reduced expression for $w_{0}$. Much of the discussion
that follows applies to this more general situation.
\end{remark}

We define divisors
$\tilde{D}_{j}=P_{1}\times^{B}\cdots\times^{B}P_{j-1}\times^{B}B\times^{B}P_{j+1}\ldots
P_{n}/B$ of $Z_{n}$. Let $D_{n}=\cup^{n}_{j=1}\tilde{D}_{j}$. The
components of $D_{n}$ intersect transversally at their intersection
point $x=B\times^{B}\cdots\times^{B}B/B$.

Consider $\cEnd_{F}(Z_{n},D_{n})$, the sheaf of Frobenius-linear
endomorphi\-sms on $Z_{n}$ which leave the ideal of $D_{n}$
invariant. Since $D_{n}$ is a codimension one subvariety of the smooth
variety $Z_{n}$, the duality theory for the absolute Frobenius map
$F:Z_{n}\to Z_{n}$ tells us that $\cEnd_{F}(Z_{n},D_{n})\approx
\omega_{Z_{n}}(D_{n})^{1-p}$. (See also \ref{prop-A.3.5}, \ref{prop-A.4.6}). Here
$\omega_{Z_{n}}$ denotes the canonical line 
bundle\index{A}{canonical line bundle [7]}
$\Omega^{n}_{Z_{n}}$ of $Z_{n}$.

\begin{definition}\label{chap4-defi4.3.14}
Let $\V$ be a $B$-equivariant vector bundle\index{B}{equivariant vector bundle} on a variety $X$ with $B$
action. (That is, on the corresponding geometric vector
bundle\index{A}{geometric vector bundle [7; II Exercise 5.18]} there
is a $B$ action compatible with the action on $X$.) Then $\V[\lambda]$
denotes the same vector bundle, but with $B$ action twisted by
$\lambda:$ For $s\in H^{0}(U,\V)$, $b\in B$, we let $b.s$ be
$\lambda(b)$ times what it would be without the twist.
\end{definition}

\begin{proposition}\label{chap4-prop4.3.15}
The sheaf $\cEnd_{F}(Z_{n},D_{n})$ is $B$-equivariantly isomorphic
with $\varphi^{*}\L((1-p)\rho)[(p-1)\rho]$, so that if
$\varphi:Z_{n}\to G/B$ is surjective, its module of global sections
$\End_{F}(Z_{n},D_{n})$ is $B$-equivariantly isomorphic with
$k_{(p-1)\rho}\otimes H^{0}(G/B,\L((1-p)\rho))$. 
\end{proposition}

For the proof we refer reader to the Appendix (\ref{prop-A.4.6}).\hfill$\Box$

Restricting the above isomorphism to global sections, we get the
following corollary.

\begin{corollary}\label{chap4-coro4.3.16}
If the map $\varphi:Z_{n}\to G/B$ is surjective, then there exists a
$B$-equivariant isomorphism between $\End_{F}(Z_{n},D_{n})$ and
$k_{(p-1)\rho}\otimes H^{0}(G/B,\L((1-p)\rho))$. 
\end{corollary}

\begin{proposition}\label{chap4-prop4.3.17}
Let $\{s_{1},\ldots,s_{n}\}$ denote a sequence of simple reflections,
let $P_{i}$ be the corresponding minimal parabolic subgroups and let
$Z_{n}=P_{1}\times^{B}\cdots\times^{B}P_{n}/B$ be as above. Let
$\varphi:Z_{n}\to G/B$ be the ``multiplication'' map which we assume
to be surjective. Then there exists $\sigma\in \End_{F}(Z_{n},D_{n})$
which is a canonical splitting. 
\end{proposition}

\begin{remark}\label{chap4-rem4.3.18}
The\pageoriginale surjectivity\label{page45} is not really needed for the conclusion to hold.
\end{remark}

\noindent
{\bf Proof of Proposition \ref{chap4-prop4.3.17}:} (See also the
Appendix \ref{prop-A.4.7}.) To get a candidate for the canonical splitting we use
\cite{key24} to which we refer for details. As Mehta and Ramanathan
explain in \cite{key24}, one gets a splitting by taking the correct
scalar multiple of any element of $\End_{F}(Z_{n},D_{n})$ that does
not vanish at the intersection point $x$ of the components of
$D_{n}$. And such an element can be obtained by pulling back a section
of $\L((1-p)\rho)[(p-1)\rho]$ that does not vanish at $B/B$. We claim
the splitting may be taken to be $T$-equivariant so that it satisfies
the first condition for being canonical. Indeed, if it were not
$T$-equivariant we could simply take its weight zero component and we
would find that component is also a splitting (exercise). From
Proposition \ref{chap4-prop4.3.15} we see that the weight zero space
of $\End_{F}(Z_{n},D_{n})$ is one-dimensional, so in fact we end up
with a unique splitting this way. Now the extremal weights of
$H^{0}(G/B,\L((1-p)\rho))$ are in the Weyl group orbit of $(1-p)\rho$,
so for a simple root $\alpha$ the ladder $\{i\alpha\mid i\alpha$ is a
weight of $\End_{F}(Z_{N},D_{N})\}$ stops with
$(p-1)\alpha=(p-1)\rho-s_{\alpha}(p-1)\rho$. Thus the second condition
for being canonical is also satisfied.\hfill$\Box$

\section{Donkin's Conjecture}\label{chap4-sec4.4}

In this section we prove Donkin's conjecture.

\begin{theorem}[Canonical splittings and good
    filtrations]\label{chap4-thm4.4.1} 
Let $A$ be a connected (i.e. $A^{0}=k$), graded $B$-algebra with
excellent filtration. Let $\sigma$ be a graded canonical splitting of
$A$. If $S$ is a graded $\sigma$-invariant subalgebra which is a
$G$-module, then $S$ has a good filtration.
\end{theorem}

\begin{proof}
We concentrate on proving that $S^{1}$ has a good filtration. The
other degrees can be treated similarly, using rescaling as in the
proof of \ref{chap4-thm4.2.3}. (We ask the reader to figure out how a
graded canonical splitting on $A$ defines one on the rescaled algebra
$\oplus_{i}A^{m-i}$.)

The length-height filtration of $A$ is an excellent filtration,
therefore $A_{\leq \lambda}$ also has excellent filtration. For any
weight $\lambda$ of $A$, the $B$-subalgebra $A_{\leq\lambda}$ of $A$
is invariant under the canonical splitting, as is its ideal
$A_{<\lambda}$ (Corollary \ref{chap4-coro4.3.8}). Also\pageoriginale
note\label{page46} for $\lambda\in X(T)^{+}$, that the submodule $S\cap A_{\leq
  \lambda}$ is again $G$-invariant, as is its ideal $S\cap
A_{<\lambda}$ (cf.\@ \ref{chap4-exer4.1.2}). We therefore replace $A$
by $A_{\leq \lambda}/A_{<\lambda}$--with its induced canonical
splitting--and $S$ by $S\cap A_{\leq \lambda}/S\cap
A_{<\lambda}$. Then with these new choices $\lambda$ is such that
$i\lambda$ is the only weight in $\soc A^{i}$. Also $S$ is $p$-root
closed since $S$ is invariant under $\sigma$ and $\sigma(a^{p})=a$. We
now use Theorem \ref{chap4-thm4.2.3} to see that $S$ has good filtration.
\end{proof}

Next, we give a geometric implication of the above theorem. Note that
the motivating variety is $G\times^{B}G/B$.

\begin{lemma}\label{chap4-lem4.4.2}
Let $X$ be a $B$-variety and $Y$ a $B$-invariant subvariety. Let
$G\times^{B}X$ denote the associated fibre bundle over $G/B$ with
fibre $X$. Assume that there exists a canonical splitting $\sigma$ of
$G\times^{B}X$ compatible with $G\times^{B}Y$. Let $\L$ be a
$G$-equivariant line bundle on $G\times^{B}X$. Let $K(\L)$ denote the
kernel of the restriction morphism res: $H^{0}(G\times^{B}X,\L)\to
H^{0}(G\times^{B}Y,\L)$. Then the $G$-modules $H^{0}(G\times^{B}X,\L)$
and $K(\L)$ have good filtrations.
\end{lemma}

\begin{proof}
Let $\pi:G\times^{B}X\to G/B$ be the projection map. Now
$\oplus_{n}H^{0}(G\times^{B}X,\L^{n})\hookrightarrow
\oplus_{n}H^{0}(\pi^{-1}(Bw_{0}B/B),\L^{n})$. But
$\pi^{-1}(Bw_{0}B/B)\approx Bw_{0}T\times^{T}X$ in a $B$-equivariant
way and therefore
$$
H^{0}(\pi^{-1}(Bw_{0}B)/B,\L^{n})=\ind^{B}_{T}H^{0}(w_{0}T\times^{T}X,\L^{n}).
$$ 
Therefore
$\oplus H^{0}(\pi^{-1}(Bw_{0}B/B),\L^{n})$ is an injective
$B$-module. Thus by the cohomological criterion, (Theorem
\ref{chap3-thm3.2.7}), it has an excellent filtration. Now we extend
$\sigma$ to a graded canonical splitting on
$$
\oplus_{n}H^{0}(\pi^{-1}(Bw_{0}B),\L^{n}).
$$ 

This splitting leaves the
$G$-submodule $\oplus_{n}H^{0}(G\times^{B}X,\L^{n})$
invariant. Therefore, by Theorem \ref{chap4-thm4.4.1}, $\oplus
H^{0}(G\times^{B}X,\L^{n})$ has good filtration. Thereofre
$H^{0}(G\times^{B}X,\L)$ has good filtration.

Similar arguments show that the $G$-module $H^{0}(G\times^{B}Y,\L)$
has good filtration.

Consider next the following diagram:
\[
\xymatrix{
0\ar[d] & 0\ar[d]\\
\oplus_{n}K(\L^{n})\ar[r]\ar[d] & \oplus_{n}K'(\L^{n})\ar[d]\\
\oplus_{n}H^{0}(G\times^{B}X,\L^{n})\ar@{^(->}[r]\ar[d]^{\res} &
\oplus_{n}H^{0}(Bw_{0}B\times^{B}X,\L^{n})\ar[d]^{\res}\\
\oplus_{n}H^{0}(G\times^{B}Y,\L^{n})\ar@{^(->}[r] &
\oplus_{n}H^{0}(Bw_{0}B\times^{B}Y,\L^{n}) 
}
\]

Now\pageoriginale the\label{page47} splitting on
$\oplus_{n}H^{0}(Bw_{0}B\times^{B}X,\L^{n})$ restricts to a splitting
on the algebra $k\oplus\bigoplus_{n}K'(\L^{n})$. (We added $k$ in
degree zero to get an algebra rather than an ideal.) Further, this
splitting leaves $k\oplus\bigoplus_{n}K(\L^{n})$ invariant. Therefore,
by Theorem \ref{chap4-thm4.4.1}, $K(\L)$ also has good filtration.
\end{proof}

Now we are in a position to prove Donkin's conjecture.\index{B}{Donkin's conjecture} Like all main
results in these notes, it and its method of proof are due to
Mathieu. (The reader is invited to compare our exposition with that of
Mathieu, to see where the emphasis differs.)

\begin{theorem}[Donkin's Conjecture]\label{chap4-theo4.4.3}
Let $\lambda$, $\mu\in X(T)^{+}$.
\begin{enumerate}
\item $P(\lambda)\otimes P(\mu)$ has a good filtration.

\item {\bf(Restriction Conjecture)}\index{B}{Restriction Conjecture} 
  Let $L$ be the Levi factor\index{A}{Levi factor [9]} of a
  parabolic subgroup $P$ of $G$ and let $\lambda\in X(T)^{-}$. Then
  $\res^{G}_{L}(\Ind^{G}_{B}(\lambda))$ as an $L$-module has a good
  filtration. 
\end{enumerate}
\end{theorem}

\begin{proof}
We are now in a position to exploit Remark \ref{chap4-rem4.2.1}. We
have $P(\lambda)\otimes
P(\mu)=H^{0}(G\times^{B}G/B,(G\times^{B}\L(\mu))[\lambda])$ with
$G\times^{B}G/B\approx G/B\times G/B$. If $s_{1},\ldots,s_{n}$ is a
sequence of simple reflections such that--with suitable choice of
$\varphi$--the map
$Z_{n}=P_{1}\times^{B}\cdots\times^{B}P_{n}/B\xrightarrow{\varphi}G\times^{B}G/B$
is birational, then we have a canonical splitting on $Z_{n}$ inducing
one on $G\times^{B}G/B$ and by Lemma \ref{chap4-lem4.4.2} we get the
first result.

For the second result notice that similarly $P\times^{B}G/B\approx
P/B\times G/B$. The module we have to study is now the restriction of
$$
H^{0}(P\times^{B}G/B,(P\times^{B}\L(\lambda)).
$$ 
As $P/B$ is a
Schubert variety, it has its Demazure resolution just like $G/B$. It
is thus not difficult to come up with $s_{1},\ldots,s_{n}$, such that
$Z_{n}=P_{1}\times^{B}\cdots\times^{B}P_{n}/B\xrightarrow{\varphi}P\times^{B}G/B$
is birational. Thus $P\times^{B}G/B=L\times^{L\cap B}G/B$ has a
canonical splitting, which of course remains canonical with respect to
the Borel subgroup $B\cap L$ of $L$. Apply Lemma \ref{chap4-lem4.4.2}.
\end{proof}

\begin{exercise}\label{chap4-exer4.4.4}
Read the Appendix and fill in the detains in the above proof. 
\end{exercise}




