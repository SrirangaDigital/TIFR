\chapter{Joseph's Conjecture}\label{chap5}

In\pageoriginale the\label{page48} last chapter we proved that the tensor product of
two modules with good filtrations has good filtration. Now as the
reader will see, (Example \ref{chap5-exam5.3.1}), the tensor product
of two modules with excellent filtration need not have excellent
filtration. However in this chapter we prove Joseph's conjecture which
says that the tensor product of a module with good filtration and an
anti-dominant character has excellent filtration.

We will prove that $\lambda\otimes P(\mu)\otimes Q(\nu)$ is
$B$-acyclic for $\lambda\in X(T)^{-}$, $\mu\in X(T)^{+}$ and $\nu\in
X(T)$. This implies, by the cohomological criterion, that the tensor
product $\lambda\otimes P(\mu)$ has excellent filtration for $\lambda$
anti-dominant and $\mu$ dominant. From this the Joseph's conjecture
follows.

To prove the vanishing of $B$-cohomology, we first induce these
modules up to $G$ using the $\ind^{G}_{B}$ functor. We then prove that
the induced $G$-modules have good filtration and thereby are
$G$-acyclic. We then use the Frobenius reciprocity to prove the
$B$-acyclicity. The use of Frobenius reciprocity requires the
$\ind^{G}_{B}$-acyclicity of these modules and we use the method of
Frobenius splitting to prove the same.

\section{Double Schubert Varieties}\index{B}{double Schubert variety}\label{chap5-sec5.1}

Let $w$, $z\in W$ be two elements of the Weyl group of $G$. Let $P$
and $Q$ be two parabolic subgroups of $G$, containing $B$. Let $X_{w}$
and $X_{z}$ denote the Schubert varieties $\overline{BwP}/P\subset
G/P$ and $\overline{BzQ}/Q\subset G/Q$ respectively. Consider the
closed $B$-subvariety $X_{w}\times X_{z}$ of $G/P\times G/Q$.

\begin{definition}\label{chap5-defi5.1.1}
By\pageoriginale a double\label{page49} Schubert variety we mean the subvariety
$G\times^{B}(X_{w}\times X_{z})$ of $G\times^{B}(G/P\times G/Q)$.
\end{definition}

As the total space of the fibre bundle $G\times^{B}(G/P\times G/Q)$ is
isomorphic with $G/B\times G/P\times G/Q$, a double Schubert variety
is naturally embedded in the triple product $G/B\times G/P\times G/Q$.

\begin{proposition}\label{chap5-prop5.1.2}
There exists a canonical splitting of $G/B\times G/P\times G/Q$ such
that all double Schubert varieties are simultaneuously compatibly
split in the triple product.
\end{proposition}

For the proof we refer the reader to the
Appendix (Proposition\break \ref{prop-A.4.8}). 

Let $\mu\in X(T)^{-}$, and let $P_{\mu}$ be the parabolic subgroup
such that $\mu$ extends to a character on $P_{\mu}$ and it is maximal
for this property. Therefore on $G/P_{\mu}$, the line bundle $\L(\mu)$
associated to the character $\mu$ exists and is ample. Indeed we work
with $G/P_{\mu}$ instead of $G/B$ for precisely this reason. One
further notes that if $\pi:G/B\to G/P_{\mu}$ is the natural projection
map, then we have $\pi^{*}\L(\mu)=\L(\mu)$.

We have $P_{\mu}=B$ if and only if $\mu$ is regular in $X(T)^{-}$.

Let $\lambda$, $\mu$, $\nu$ be characters in $X(T)^{-}$ with $\lambda$
regular. Let $\L(\lambda,\mu,\nu)$ denote the line bundle
$\L(\lambda)\times \L(\mu)\times \L(\nu)$ on the product $G/B\times
G/P_{\mu}\times G/P_{\nu}$. Let $\sum_{1}$ and $\sum_{2}$ be unions of
double Schubert varieties in $G/B\times G/P_{\mu}\times G/P_{\nu}$
such that $\sum_{2}\subset \sum_{1}$. Then, we have

\begin{lemma}\label{chap5-lem5.1.3}
\begin{enumerate}
\renewcommand{\theenumi}{\roman{enumi}}
\renewcommand{\labelenumi}{\rm(\theenumi)}
\item $H^{0}(\sum_{1},\L(\lambda,\mu,\nu))$ has good filtration.

\item The restriction map $H^{0}(\sum_{1},\L(\lambda,\mu,\nu))\to
  H^{0}(\sum_{2},\L(\lambda,\mu,\nu))$ is surjective. Further, its
  kernel $K(\sum_{1},\sum_{2},\lambda,\mu,\nu)$ has good filtration.
\end{enumerate}
\end{lemma}

\begin{proof}
\begin{enumerate}
\renewcommand{\theenumi}{\roman{enumi}}
\renewcommand{\labelenumi}{\rm(\theenumi)}
\item By Corollary \ref{chap4-coro4.4.2} we see that
  $H^{0}(\sum_{1},\L(\lambda,\mu,\nu))$ and the kernel of the
  restriction map $K(\sum_{1},\sum_{2},\lambda,\mu,\nu)$ have good
  filtration. 

\item As $G\times^{B}\sum_{i}$ are compatibly split in
  $G\times^{B}(G/P_{\mu}\times G/P_{\nu})$ and the line bundle
  $\L(\lambda,\mu,\nu)$ is amply on $G/B\times G/P_{\mu}\times
  G/P_{\nu}$ the surjectivity of the restriction map follows from the
  Appendix (see Corollary \ref{coro-A.2.2}).
\end{enumerate}
\end{proof}

\begin{remark}\label{chap5-rem5.1.4}
Consider the short exact sequence
{\fontsize{10pt}{12pt}\selectfont
$$
0\to K(\sum_{1},\sum_{2},\lambda,\mu,\nu)\to
H^{0}(\sum_{1},\L(\lambda,\mu,\nu))\to
H^{0}(\sum_{2},\L(\lambda,\mu,\nu))\to 0.
$$}\relax

As\pageoriginale the\label{page50} kernel $K(\sum_{1},\sum_{2},\lambda,\mu,\nu)$ has
good filtration, it is $G$-acyclic. Thus
$H^{1}(G,K(\sum_{1},\sum_{2},\lambda,\mu,\nu))=0$. Therefore, by
writing out the long exact sequence of $G$-cohomologies corresponding
to the above short exact sequence, we get the surjectivity of the
restriction map on the $G$-invariants
$H^{0}(G,\sum_{1},\L(\lambda,\mu,\nu))\to
H^{0}(G,\sum_{2},\L(\lambda,\mu,\nu))$. 
\end{remark}

The double Schubert varieties arise naturally in the context of
filtrations of $B$-modules in the following manner:

Let $\lambda$, $\mu$, $\nu$ be characters. Let $M=\lambda\otimes
P(\mu)\otimes P(\nu)$. As a vector space $M$ is isomorphic with
$P(\mu)\otimes P(\nu)$ but the $B$ action on $M$ is shifted by the
character $\lambda$.

Let $\mu_{1}=w^{-1}_{\mu}\mu$ and $\nu_{1}=w^{-1}_{\nu}\nu$ be the
anti-dominant characters in the respective Weyl group orbits. We put
$P=P_{\mu_{1}}$ and $Q=P_{\nu_{1}}$.

Using the double Schubert varieties we get the following description
of $\ind^{G}_{B}(M)$.

Let $S$ be the product $X_{w_{\mu}}\times X_{w_{\nu}}$ in
$G/P_{\mu}\times G/P_{\nu}$. Consider the restricted fibration
$f=\pi\circ i$ on $G/B$ as given below.
\[
\xymatrix{
G\times^{B}S \ar@{^(->}[r]^-{i}\ar[dr]_{f}  &
G\times^{B}(G/P_{\mu}\times G/P_{\nu})\ar[d]^{\pi} \\
 & G/B
}
\]

If $\L(M)$ denotes the vector bundle on $G/B$ corresponding to the
$B$-representation $M$, we have
$\L(M)=f_{\ast}i^{*}\L(\lambda,\mu_{1},\nu_{1})$. Therefore we have
\begin{align*}
\ind^{G}_{B}(M) &=
H^{0}(G/B,f_{\ast}i^{\ast}\L(\lambda,\mu_{1},\nu_{1}))\\
&= H^{0}(G\times^{B}S,\L(\lambda,\mu_{1},\nu_{1}))
\end{align*}

If we assume $\lambda$ regular anti-dominant, the line bundle
$\L(\lambda,\mu_{1},\nu_{1})$ is ample on $G/B\times G/P\times
G/Q$. Further, $G\times^{B}S$ is compatibly split in $G/B\times
G/P\times G/Q$. Therefore, we have
\begin{align*}
R^{j}\ind^{G}_{B}(M) &= H^{j}(G/B,\L(M))\\
&= H^{j}(G\times^{B}S,\L(\lambda,\mu_{1},\nu_{1}))\q\text{by Remark
  \ref{rem-A.2.8},}\\
&=0\q \text{for~ } j>0\q \text{by Corollary \ref{coro-A.2.2}.}
\end{align*}

Thus we have the following lemma.

\begin{lemma}\label{chap5-lem5.1.5}
Let\pageoriginale 
$\lambda\in X(T)^{-}$\label{page51} be regular. Let $S$ be a union of products
of Schubert varieties in $G/P_{\mu}\times G/P_{\nu}$ with $\mu$,
$\nu\in X(T)^{-}$. Then, $M=\lambda\otimes H^{0}(S,\L(\mu)\times
\L(\nu))$ is $\ind^{G}_{B}$-acyclic.
\end{lemma}

\begin{proof}
The above reasoning also works for such a union.
\end{proof}

\section{Joseph's Conjecture}\label{chap5-sec5.2}

In this section we will prove Joseph's conjecture. Moreover, for a
regular, anti-dominant character $\lambda$ and any two characters
$\mu$ and $\nu$ we will prove the $B$-acyclicity of $\lambda\otimes
Q(\mu)\otimes Q(\nu)$.

Lemma \ref{chap5-lem5.1.5} gives us the following vanishing result.

\begin{lemma}\label{chap5-lem5.2.1}
Let $\lambda$, $\mu$, $\nu$ be anti-dominant with $\lambda$ being
regular. Let $S$, $S_{1}$, $S_{2}$ be unions of products of Schubert
varieties with $S_{2}\subset S_{1}$. Then
\begin{itemize}
\item[\rm(i)] $M=\lambda\otimes H^{0}(S,\L(\mu)\times \L(\nu))$ is
  $B$-acyclic.

\item[\rm(ii)] $M'=\Ker\{\lambda\otimes H^{0}(S_{1},\L(\mu)\times
  \L(\nu))\to \lambda\otimes H^{0}(S_{2},\L(\mu)\times \L(\nu))\}$ is
  $B$-acyclic. 
\end{itemize}
\end{lemma}

\begin{proof}
\begin{itemize}
\item[(i)] By Lemma \ref{chap5-lem5.1.3} we see that $\ind^{G}_{B}(M)$
  has good filtration. Further, by Lemma \ref{chap5-lem5.1.5} $M$ is
  $\ind^{G}_{B}$-acyclic. Therefore, we have
  $H^{i}(B,M)=H^{i}(G,\ind^{G}_{B}(M))=0$.

\item[(ii)] We know that both $\lambda\otimes
  H^{0}(S_{i},\L(\mu)\times \L(\nu))$ are $B$-acyclic. Further, using
  Remark \ref{chap5-rem5.1.4} and Frobenius reciprocity, we see that
  $H^{0}(B,\lambda\otimes H^{0}(S_{1},\L(\mu)\times\L(\nu)))\to
  H^{0}(B,\lambda\otimes H^{0}(S_{2},\L(\mu)\times \L(\nu)))$ is
  surjective. Now we write the long exact sequence of $B$-cohomo\-logy
  associated with $0\to M'\to \lambda\otimes
  H^{0}(S_{1},\L(\mu)\times\L(\nu))\to \lambda\otimes
  H^{0}(S_{2},\L(\mu)\times \L(\nu))\to 0$ to get the result.
\end{itemize}
\end{proof}

\begin{corollary}\label{chap5-coro5.2.2}
Let $\lambda\in X(T)^{-}$ be regular. Let $\mu$, $\nu\in X(T)$ and let
$Q(\mu)$, $Q(\nu)$ denote the relative Schubert modules with socle
$\mu$ and $\nu$ respectively. Then, $\lambda\otimes Q(\mu)\otimes
Q(\nu)$ is $B$-acyclic.
\end{corollary}

\begin{proof}
Recall\pageoriginale that\label{page52} the relative Schubert modules $Q(\mu)$ are
defined as kernels of the restriction map of $P(\mu)$ onto the
sections over the boundary of the Schubert variety defining
$P(\mu)$. We take $S_{1}=X_{w_{\mu}}\times X_{w_{\nu}}$ and $S_{2}=(\p
X_{w{\mu}}\times X_{\mu_{\nu}})\cup (X_{w_{\mu}}\times \p
X_{w_{\nu}})$. Then the kernel of the restriction map $\lambda\otimes
H^{0}(S_{1},\L(\mu_{1})\times \L(\nu_{1}))\to \lambda\otimes
H^{0}(S_{2},\L(\mu_{1})\times \L(\nu_{1}))$ is canonically isomorphic
with $\lambda\otimes Q(\mu)\otimes Q(\nu)$ where $\mu_{1}$ and
$\nu_{1}$ are the anti-dominant characters in the Weyl group orbit of
$\mu$ and $\nu$ respectively. Now the Lemma \ref{chap5-lem5.2.1} gives
the result.
\end{proof}

\begin{corollary}\label{chap5-coro5.2.3}
Let $\lambda\in X(T)^{-}$ be regular and let $\mu$ be any
character. Then $\lambda\otimes Q(\mu)$ has excellent filtration.
\end{corollary}

\begin{proof}
Apply the cohomological criterion for excellent filtration (Theorem
\ref{chap3-thm3.2.7}). 
\end{proof}

In order to prove Joseph's conjecture we now need the following lemma.

\begin{lemma}\label{chap5-lem5.2.4}
Let $\rho$ be the character corresponding to the half sum of positive
roots. Then, for $\lambda\in X(T)^{+}$ we have $k_{\rho}\otimes
P(\lambda)=Q(\lambda+\rho)$. 
\end{lemma}

\begin{proof}
We have a natural multiplication map from
$H^{0}(G/B,\L(w_{0}\lambda)\otimes H^{0}(G/B,\L(-\rho))$ to
$H^{0}(G/B,\L(w_{0}\lambda)\otimes \L(-\rho))$. Let $k_{\rho}$ be the
weight space of weight $\rho$ of $H^{0}(G/B,\L(-\rho))$. We restrict
the multiplication map to the sub-space
$H^{0}(G/B,\L(w_{0}\lambda))\otimes k_{\rho}$. This gives us a map
$m:P(\lambda)\otimes k_{\rho}\to P(\lambda+\rho)$. This map is
injective as it is injective on the one-dimensional socle of its
domain. (Use the geometric description of extremal weights.)

We claim that $m$ defines a natural isomorphism between
$P(\lambda)\otimes k_{\rho}$ and $Q(\lambda+\rho)\subset
P(\lambda+\rho)$.

To see this we first fix a non-zero element $f\in k_{\rho}\subset
H^{0}(G/B,\L(-\rho))$. Then $f$ vanishes on lower dimensional Schubert
varieties $X_{w}$. Thus the image of the multiplication map $m$ is
contained in $Q(\lambda+\rho)$.

To see the surjectivity, view $1/f$ as a rational section of
$\L(\rho)$. Notice that $1/f$ has pole of order $1$ along the
codimension one Schubert varieties 
(\ref{chap5-exer5.2.5}). Now if $\L$ is a line
bundle and $s$ any section of $\L$, we get a (possibly rational)
section $s/f$ of the line bundle $\L\otimes \L(\rho)$. Thus, for a
section $s$ of the line bundle $\L(w_{0}\lambda-\rho)$, the element
$s/f$ gives us a rational section of\pageoriginale
$\L(w_{0}\lambda)$.\label{page53} However, if we restrict this map to the subspace
$Q(\lambda+\rho)$ of
$P(\lambda+\rho)=H^{0}(G/B,\L(w_{0}\lambda-\rho))$ we get an algebraic
map as all the elements of $Q(\lambda+\rho)$ vanish on the codimension
one Schubert varieties. This map from $Q(\lambda+\rho)$ to
$P(\lambda)$ is injective (by its injectivity on  the
socle). Therefore the dimensions satisfy
$$
\dim_{k}P(\lambda)\otimes k_{\rho}=\dim_{k}P(\lambda)\geq
\dim_{k}Q(\lambda+\rho). 
$$

Therefore the multiplication map defined above is also surjective.
\end{proof}

The reader is advised to do the following illuminating exercise to see
the ``geometry'' involved in the apparently representation theoretic
lemma above. The exact formula for computing the degree of a line
bundle $\L(\lambda)$ on $G/B$ restricted to any line of the type
$P_{s}/B$ can be found in \cite{key3}.

\begin{exercise}[cf.\@ \cite{key14}]\label{chap5-exer5.2.5}
Let $f\in k_{\rho}\subset H^{0}(G/B,\L(-\rho))$ be as in the proof of
\ref{chap5-lem5.2.4}. Let $s$ be a simple reflection with
corresponding minimal parabolic $P_{s}$. Show
\begin{enumerate}
\renewcommand{\theenumi}{\roman{enumi}}
\renewcommand{\labelenumi}{(\theenumi)}
\item The restriction of $\L(-\rho)$ to the line $P_{s}/B$ has degree
  $1$, and the same is true for the restriction to any left translate
  of $P_{s}/B$ in $G/B$.

\item The line $w_{0}P_{s}/B$ intersects the zero set of $f$ only in
  the point $w_{0}sB/B$. 

\item $f$ vanishes to order one along the codimension one Schubert
  variety $X_{w_{0}s}$.
\end{enumerate}
\end{exercise}

We now prove Joseph's conjecture.\index{B}{Joseph's conjecture} The proof given here differs a
little from the one by Mathieu.

\begin{proposition}[Joseph's Conjecture]\label{chap5-prop5.2.6}
Let $\lambda\in X(T)^{-}$ and $\mu\in X(T)^{+}$. Then $\lambda\otimes
P(\mu)$ has excellent filtration.
\end{proposition}

\begin{proof}
We know that for $\lambda\in X(T)^{-}$ which is also regular,
$\lambda\otimes Q(\mu)$ has excellent filtration. Now,
\begin{align*}
\lambda \otimes P(\mu) &= (\lambda-\rho)\otimes \rho\otimes P(\mu)\\
&= (\lambda-\rho)\otimes Q(\mu+\rho).
\end{align*}

Further, $\lambda-\rho\in X(T)^{-}$ is regular. (In fact $\nu-\rho$ is
regular anti-dominant\index{B}{regular anti-dominant} if and only if $\nu$ is anti-dominant.)
Therefore by Corollary \ref{chap5-coro5.2.3} we get he result.
\end{proof}

\begin{corollary}[Joseph]\label{chap5-coro5.2.7}
Let\pageoriginale $\lambda\in X(T)$\label{page54} and $\mu\in X(T)^{+}$. Then,
$P(\lambda)\otimes P(\mu)$ has excellent filtration.
\end{corollary}

\begin{proof}
Let $w\in W$ be such that $w^{-1}\lambda=\nu\in X(T)^{-}$. We have
$P(\nu)=k_{\nu}$. Therefore, $P(\nu)\otimes P(\mu)$ has excellent
filtration. Since $\mu$ is dominant, $P(\mu)$ is a $G$-module and
therefore, by the tensor identity, $\ind^{P_{s}}_{B}(P(\tau)\otimes
P(\mu))=(\ind^{P_{s}}_{B}P(\tau))\otimes P(\mu)$ for any simple
reflection $s$ and weight $\tau$. Recall that we have $H_{s}\circ
H_{z}=H_{sz}$ for Joseph functors when the length of $sz$ is more than
the length of $z$. Therefore we see that $H_{w}(P(\nu)\otimes
P(\mu))=P(w\nu)\otimes P(\nu)$. Now recall that Proposition
\ref{chap3-prop3.2.11} states that $H_{w}$ sends a module with
excellent filtration to a module with excellent filtration. Therefore
the result.
\end{proof}

For an application f Joseph's conjecture see \cite[Theorem
  5.5]{key21},\break which gives the existence of a ``good basis'' in a
module with good filtration. One easily checks that although the proof
refers to Polo's conjecture (cf.\@ next chapter), it suffices to apply
Joseph's conjecture.

\section{An Example}\label{chap5-sec5.3}

In this section we give an example showing that the tensor product of
modules with excellent filtration need not have excellent filtration.

\begin{example}\label{chap5-exam5.3.1}
We take $G=SL(3,k)$, with $B$ the subgroup of upper triangular
matrices, $T$ the subgroup of diagonal matrices. Inside the $G$-module
$M_{3}$ of 3-by-3 matrices, upon which $G$ acts by conjugation, we
consider the five-dimensional $B$-submodule $E$ generated, as a
$B$-module, by the matrices
$$
C=
\begin{pmatrix}
1 & 0 & 0\\
0 & 0 & 0\\
0 & 0 & 0
\end{pmatrix},\qquad
D=
\begin{pmatrix}
0 & 0 & 0\\
0 & 1 & 0\\
0 & 0 & 0
\end{pmatrix}.
$$

If has a four-dimensional submodule $S$ generated by $D$, and the
extension
$$
0\to S\to E\to k\to 0
$$
does not split. So $H^{1}(B,S)\neq 0$. Now one checks that
$S=P(-s_{2}\omega_{1})\otimes P(-s_{1}\omega_{2})\otimes Q(\rho)$,
where $\omega_{1}$, $\omega_{2}$ denote the fundamental weights, cf.\@
\cite{key11}. (Recall\pageoriginale $Q(\rho)=k_{\rho}$.)\label{page55} So $S$ gives
an example of a tensor product of the form $P(\lambda)\otimes
P(\mu)\otimes Q(\nu)$ which is not $B$-acyclic. From the cohomological
criteria it then follows that $P(\lambda)\otimes P(\mu)$ does not have
excellent filtration and that $P(\mu)\otimes Q(\nu)$ does not have
relative Schubert filtration. 
\end{example}

\begin{exercise}[Polo]\label{chap5-exer5.3.2}
Computer the characters of the $P(\xi)$ for each weight $\xi$ of
$P(-s_{2}\omega_{1})\otimes P(-s_{1}\omega_{2})$ and show that
$P(-s_{2}\omega_{1})\otimes P(-s_{1}\omega_{2})$ does not even have
the character of any module with excellent filtration. Similarly show
that $P(-s_{1}\omega_{2})\otimes Q(\rho)$ does not even have the
character of any module with relative Schubert filtration.
\end{exercise}
