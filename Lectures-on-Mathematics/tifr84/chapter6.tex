\chapter{Polo's Conjecture}\label{chap6}

Let\pageoriginale $\zeta$\label{page56} be a character. We denote by $\zeta_{1}$ (by
$\zeta_{0}$) the anti-dominant (the dominant) character in the Weyl
group orbit of $\zeta$. The Joseph Conjecture states that for
$\lambda\in X(T)^{-}$ and $\mu\in X(T)$, the module $\lambda\otimes
P(\mu_{0})$ has excellent filtration. Here we study a generalization
of that conjecture, first stated by P.\@ Polo. It says that for
$\lambda\in X(T)^{-}$ and $\mu$ arbitrary, the module $\lambda\otimes
P(\mu)$ has excellent filtration. Equivalently, we need to prove that
$\lambda\otimes P(\mu)\otimes Q(\nu)$ is $B$-acyclic.

\section{Reformulating the Problem Repeatedly}\label{chap6-sec6.1}

We first look at the case when $\lambda$ is regular
anti-dominant. Consider the following exact sequence:
\begin{equation*}
0\to \lambda\otimes K\to \lambda\otimes P(\mu_{0})\to \lambda\otimes
P(\mu)\to 0\tag{6.1.1}\label{chap6-eq6.1.1}
\end{equation*}

By Joseph's conjecture $\lambda\otimes P(\mu_{0})$ has excellent
filtration. The module $K$ has a filtration by relative Schubert
modules and for $\lambda$ regular anti-dominant we already know that
$\lambda\otimes Q(\nu)$ has excellent filtration for any character
$\nu$. Therefore $\lambda\otimes K$ has excellent filtration. Now,
using the long exact sequence of $B$-cohomology associated to
\eqref{chap6-eq6.1.1}, we see that the module $\lambda\otimes P(\mu)$
also satisfies the cohomological criterion for excellent filtration.

However this method fails when $\lambda$ is not regular as it is no
longer true that $\lambda\otimes Q(\nu)$ has excellent
filtration. Indeed, when $\lambda$ is a trivial character, we see that
$Q(\nu)$ cannot have an excellent filtration unless $\nu$ is
anti-dominant. 

To\pageoriginale tackle\label{page57} the general case we again resort to the same
trick. We first induce the $B$-modules to $G$-modules and use the
canonical splitting to prove results. But first, we need the following
lemma.

\setcounter{theorem}{1}
\begin{lemma}\label{chap6-lem6.1.2}
Let $\lambda$, $\mu$, $\nu\in X(T)^{-}$ and $w\in W$. Let $S$ be a
union of Schubert varieties in $G/P_{\nu}$. Assume that we can prove
(for all such $\lambda$, $\mu$, $\nu$, $w$, $S$) that the natural
restriction map
{\fontsize{10pt}{12pt}\selectfont
\begin{equation*}
H^{0}(B,\lambda\otimes H^{0}(X_{w},\L(\mu))\otimes P(\nu_{0}))\to
H^{0}(B,\lambda\otimes H^{0}(X_{w},\L(\mu))\otimes
H^{0}(S,\L(\nu)))\tag{6.1.3}\label{chap6-eq6.1.3} 
\end{equation*}}\relax
is surjective. Then Polo's conjecture is true.
\end{lemma}

\begin{proof}
Let
$K=\ker(H^{0}(G/P_{\nu},\L(\nu))\xrightarrow{\res}H^{0}(S,\L(\nu)))$. Let
$M=\lambda\otimes K$. We know by Joseph's conjecture that
$\lambda\otimes P(\nu_{0})$ has excellent filtration. Therefore,
$H^{1}(B,\lambda\otimes P(\nu_{0})\otimes Q(\tau))=0$ for all
$\tau$. However, $H^{0}(X_{w},\L(\mu))$ has a filtration by relative
Schubert modules $Q(\tau)$. Hence, $H^{1}(B,\lambda\otimes
P(\nu_{0})\otimes H^{0}(X_{w},\L(\mu)))=0$. Therefore, for $\mu\in
X(T)^{-}$ and $w\in W$, the surjectivity in \eqref{chap6-eq6.1.3}
gives $H^{1}(B,M\otimes H^{0}(X_{w},\L(\mu)))=0$.

Thus $H^{1}$ ($B$, $M\otimes$ module with excellent filtration)
$=0$. Therefore, $M$ has filtration by relative Schubert modules by
the cohomological criterion for relative Schubert filtration (cf.\@
Exercise \ref{chap3-exer3.3.3}).

This in turn means that $M\otimes P(\tau)$ is $B$-acyclic for any
$\tau\in X(T)$.

For any $z\nu\in X(T)$, we have the following diagram:
\[
\xymatrix{
0\ar[r] & K_{1}\ar[d]\ar[r] & H^{0}(G/P_{\nu},\L(\nu))\ar[d]\ar[r] &
H^{0}(X_{z},\L(\nu))\ar[d]^{\res}\ar[r] & 0\\
0\ar[r] & K_{2}\ar[r] & H^{0}(G/P_{\nu},\L(\nu))\ar[r] & H^{0}(\p
X_{z},\L(\nu))\ar[r] & 0 
}
\]

Further, $K_{1}$ and $K_{2}$ satisfy the following exact sequence
(cf.\@ Exercise \ref{exer-A.2.9}):
$$
0\to K_{1}\to K_{2}\to Q(z\nu)\to 0.
$$

Now we may take $M=\lambda\otimes K_{i}$ in the above, so
$\lambda\otimes K_{i}\otimes P(\tau)$ is $B$-acyclic for any $\tau\in
X(T)$ $(i=1,2)$. But then the quotient $\lambda\otimes Q(z\nu)\otimes
P(\tau)$ is $B$ acyclic too, for all $\lambda$, $\nu\in X(T)^{-}$,
$z\in W$, $\tau\in X(T)$. This proves the lemma.
\end{proof}

\begin{remark}\label{chap6-rem6.1.4}
Note\pageoriginale that\label{page58} there is also a slightly different argument to
prove the $B$-acyclicity of $M\otimes P(\mu)$ in the above: If one has
that $H^{1}(B,M\otimes$ module with excellent filtration) $=0$, then
let $I_{0}\to I_{1}\ldots$ be an injective resolution of
$P(\mu)$. Consider the exact sequences
$$
0\to \ker(I_{n}\to I_{n+1})\to I_{n}\to \Iim(I_{n})\to 0
$$
of modules with excellent filtration. Tensoring with $M$ and taking
$B$-invariants gives many short exact sequences and thus
$H^{i}(B,M\otimes P(\mu))$ in fact vanishes for $i>0$. The advantage
of this argument is that it does not need the cohomological criterion
for relative Schubert filtrations.
\end{remark}

To prove surjectivity of \eqref{chap6-eq6.1.3}, we first induce both
modules up to $G$ and then prove that the map on $G$-invariants in
surjective. The Frobenius reciprocity then gives us the surjectivity
on $B$-invariants.

Recall that $\ind^{G}_{B}(\lambda\otimes H^{0}(X_{w},\L(\mu))\otimes
H^{0}(S,\L(\nu)))=H^{0}(G\times^{B}(X_{w}\times S),
\L(\lambda,\mu,\nu))$. Now the line bundle $\L(\lambda,\mu,\nu)$ is
not ample on $G\times^{B}(G/P_{\mu}\times G/P_{\nu})$, unless
$\lambda$ is regular. Therefore, for all we know now, the restriction
map $H^{0}(G\times^{B}(X_{w}\times G/P_{\nu}),\L(\lambda,\mu,\nu))\to
H^{0}(G\times^{B}(X_{w}\times S),\L(\lambda,\mu,\nu))$ need not be
surjective, even though $G\times^{B}(X_{w}\times S)$ is compatibly
split in the product $G/B\times G/P_{\mu}\times G/P_{\nu}$. However,
the line bundle $\L(\lambda,\mu,\nu)$ is ample on $G/P_{\lambda}\times
G/P_{\mu}\times G/P_{\nu}$. However, the line bundle
$\L(\lambda,\mu,\nu)$ is ample on $G/P_{\lambda}\times G/P_{\mu}\times
G/P_{\nu}$. Therefore, we consider the following diagram:
\[
\xymatrix{
Z=G\times^{B}(X_{w}\times S)\ar@{^(->}[r] &
G\times^{B}(G/P_{\mu}\times G/P_{\nu})\ar[d]^{\pi}\\
 & G/P_{\lambda}\times G/P_{\mu}\times G/P_{\nu}
}
\]

The map $\pi$ is defined by $(g,x,y)\mapsto
(\overline{g},\overline{gx},\overline{gy})$, where the ``bar'' denotes
the image of an element of $G$ in the corresponding quotient.

\setcounter{theorem}{4}
\begin{lemma}\label{chap6-lem6.1.5}
If $\pi_{\ast}\O_{Z}=\O_{\pi(Z)}$, then
\[
\xymatrix{
H^{0}(G,\ind^{G}_{B}(\lambda\otimes H^{0}(X_{w},\L(\mu))\otimes
P(\nu_{0})))\ar[d]^{\res}\\
H^{0}(G,\ind^{G}_{B}(\lambda\otimes H^{0}(X_{w},\L(\mu))\otimes
H^{0}(S,\L(\nu)))) 
}
\]
is surjective. 
\end{lemma}

\begin{proof}
We\pageoriginale have\label{page59} $\ind^{G}_{B}(\lambda\otimes
H^{0}(X_{w},\L(\mu))\otimes P(\nu_{0}))=H^{0}(G\times^{B}(X_{w}\times
G/P_{\nu}),\L(\lambda,\mu,\nu))$ and $\ind^{G}_{B}(\lambda\otimes
H^{0}(X_{w},\L(\mu))\otimes
H^{0}(S,\L(\nu)))=H^{0}(G\times^{B}(X_{w}\times
S),\L(\lambda,\mu,\nu))$. Consider the map of pairs
$$
(G\times^{B}(G/P_{\mu}\times
G/P_{\nu}),Z)\xrightarrow{\pi}(G/P_{\lambda}\times G/P_{\mu}\times
G/P_{\nu},\pi(Z)) 
$$

If $\pi|_{Z}$ has the direct image property
$\pi_{\ast}\O_{Z}=\O_{\pi(Z)}$, we have
\begin{enumerate}
\item
  $\pi_{\ast}(\L(\lambda,\mu,\nu)|_{Z})=\L(\lambda,\mu,\nu)|_{\pi(Z)}$
  and therefore,
  $H^{0}(Z,\L(\lambda,\mu,\nu))=H^{0}(\pi(Z),\L(\lambda,\mu,\nu))$. 

\item Further, the canonical splitting on the domain will give us a
  canonical splitting on $G/P_{\lambda}\times G/P_{\mu}\times
  G/P_{\nu}$, which compatibly splits $\pi(Z)$.
\end{enumerate}

Now, $\L(\lambda,\mu,\nu)=\L(\lambda)\times \L(\mu)\times \L(\nu)$ is
ample on $G/P_{\lambda}\times G/P_{\mu}\times G/P_{\nu}$. Therefore
the restriction map 
$$
H^{0}(G/P_{\lambda}\times G/P_{\mu}\times
G/P_{\nu},\L(\lambda,\mu,\nu))\to H^{0}(\pi(Z),\L(\lambda,\mu,\nu))
$$
will be surjective. Further, its kernel will have good
filtration. This allows us to apply the Remark \ref{chap5-rem5.1.4} to
see that the restriction map on $G$-invariants is surjective. From
this the claim follows as this surjective map factors through
$H^{0}(G,\ind^{G}_{B}(\lambda\otimes H^{0}(X_{w},\L(\mu))\otimes
P(\nu_{0})))$. 
\end{proof}

Therefore to prove Polo's conjecture, we only have to prove that
$\pi|_{Z}$ has the indicated direct image property. Now we remark
again that the map $\pi$ is defined on $X=G/B\times G/P_{\mu}\times
G/P_{\nu}$ and we have $\pi_{\ast}\O_{X}=\O_{G/P_{\lambda}\times
  G/P_{\mu}\times G/P_{\nu}}$. Therefore we can ``push forward'' the
canonical splitting of $X$ on to its image. This ``pushed splitting''
will split the image $\pi(Z)$ of $Z$.

Consider now the following proposition. The proof of this proposition
will be given in the Appendix (\ref{prop-A.5.2}). We have to explain first what
{\em separable}\index{B}{separable map} means for $f:X\to Y$. The relevant notion of
separability is somewhat fancy, as our varieties are not
irreducible. What it means is that there is a dense subset of $y$ in
$Y$ for which there is an $x\in f^{-1}(y)$ so that the tangent map at
$x$ is surjective. It is thus some kind of generic smoothness.

\begin{proposition}\label{chap6-prop6.1.6}
Let $f:X\to Y$ be a surjective, separable, proper morphism between two
varieties, with connected fibres. We assume that $Y$ is Frobenius
split. Then $f_{\ast}\O_{X}=\O_{Y}$. 
\end{proposition}

Let\pageoriginale us\label{page60} grant separability for the time being. Thus in
order to prove Polo's conjecture it only remains to prove that the
fibres of the map $\pi:Z\to G/P_{\lambda}\times G/P_{\mu}\times
G/P_{\nu}$ are connected. This topological problem will also be
reformulated repeatedly.

The reader is asked to be patient about this roundabout proof. The
fact is that, as he/she will come to know in Remark
\ref{chap6-rem6.1.10}, the statement we want to prove is very similar
to some false statements. We have to sneak around all these false
statements. 

First we note a result, which tells us that having connected fibres
and having the direct image property are really the same problem, so
that we may switch back and forth between the two at our
convenience. Indeed we will later turn around and go back all the way
to a problem similar to surjectivity of \eqref{chap6-eq6.1.3}.

\begin{lemma}[Corollary 11.3]\label{chap6-lem6.1.7}
Let $f:X\to Y$ be a proper morphism between two varieties and assume
$f_{\ast}\O_{X}=\O_{Y}$. Then all fibres of $f$ are connected.
\end{lemma}

Next note that $G\times^{B}(G/P_{\mu}\times
G/P_{\nu}){\displaystyle{\mathop{\approx}^{\phi}}}G\times^{P_{\mu}}G\times^{B}G/P_{\nu}$. The
map $\phi$ is defined on the product by
$\phi(g,\overline{x},\overline{y})=(gx,x^{-1},\overline{y})$. 

The image of $Z$ under $\phi$ is
$G\times^{P_{\mu}}(\overline{P_{\mu}w^{-1}B}\times^{B}S)$. 

We define
$\tilde{\pi}:G\times^{P_{\mu}}(\overline{P_{\mu}w^{-1}B}\times^{B}S)\to
G/P_{\mu}\times G/P_{\lambda}\times G/P_{\nu}$ by
$\tilde{\pi}((g,x,\overline{y})=(\overline{g},\overline{gx},\overline{gxy})$. Up
to the isomorphism $\phi$, this $\tilde{\pi}$ is just $\pi$.

So our aim is to prove that fibres of $\tilde{\pi}$ are
connected. Using the $G$-equivariance of $\tilde{\pi}$ we see that we
may restrict $\tilde{\pi}$ to the subspace
$Z'=(P_{\mu})\times^{P_{\mu}}(\overline{P_{\mu}w^{-1}B}\times^{B}S)=(e)\times
(\overline{P_{\mu}w^{-1}B}\times^{B}S)$.

All we need is that the fibres of that restricted map are connected.

The image of $Z'$ is contained in $G/P_{\lambda}\times
G/P_{\nu}$. Note that as $\overline{P_{\mu}w^{-1}B}$ is an irreducible
two-sided $B$-invariant closed subvariety of $G$, we have by Bruhat
decomposition some $y\in W$ such that
$\overline{P_{\mu}w^{-1}B}=\overline{ByB}$. 

Summing up, we have to show that the map $\overline{ByB}\times^{B}S\to
G/P_{\lambda}\times G/P_{\nu}$ has connected fibres.

A fibre of the map $\overline{ByB}\times^{B}S\to G/P_{\lambda}\times
G/P_{\nu}$ is simply an intersection of a fibre of
$\overline{ByB}\times^{B}S\to G/P_{\lambda}$ with a fibre of
$\overline{ByB}\times^{B}S\to G/P_{\nu}$. We first concentrate on the
projection towards $G/P_{\lambda}$.

\begin{proposition}\label{chap6-prop6.1.8}
Let\pageoriginale $P$\label{page61} be a parabolic, $X_{w}\subset G/B$ a Schubert
variety. The non-empty fibres of the projection $X_{w}\to G/P$ are
left translates of Schubert varieties.
\end{proposition}

\begin{proof}
Using the $B$-equivariance we may restrict attention to the fibre of
$zP/P$, where $z$ is a minimal representative in the Weyl group $W$ of
the coset $zW(P)$, if $W(P)$ denotes the Weyl group of $P$. Recall
from \cite[Proposition 1.10]{key10}, cf.\@ \cite[Ch.\@ IV, \S1
  Exercice 3]{key1} that $l(zu)=l(z)+l(u)$ if $u\in W(P)$, so that
$BzBuB=BzuB$. The fibre is thus a union of sets $zBuB/B$, where $u\in
W(P)$ is such that $zu\leq w$. Recall also (same source) that $w$
decomposes uniquely as $z'u'$ where $z'$ is a minimal representative
of the coset $z'W(P)$ and $u'\in W(P)$. Then a lemma of Deodhar (read
$w\in W_{Q}$ where it says $w\in W/W_{Q}$, in \cite[Lemma 4.4]{key16})
says there is a unique maximal $u$. Then the fibre is
$z\overline{BuB}/B$ for that maximal $u$.
\end{proof}

So what the proposition tells us is that we should prove that the
fibres of $g\overline{BuB}\times^{B}S\to G/P_{\nu}$ are connected for
$g\in G$, $u\in W$. And by $G$-equivariance we may forget $g$.

Thus we have to prove

\begin{proposition}\label{chap6-prop6.1.9}
The fibres of the multiplication map $m:\overline{BuB}\times^{B}S\to
G/P_{\nu}$ are connected.
\end{proposition}

\begin{remark}\label{chap6-rem6.1.10}
We can now point out a subtlety, which shows that one cannot get by
just with generalities about Frobenius splittings. Namely, the
proposition fails if $\overline{BuB}$ is replaced by a union of
$\overline{BvB}$'s $(v\in W)$. This is related to the fact that a
tensor product of two modules with excellent filtration need not have
an excellent filtration (see Example \ref{chap5-exam5.3.1}.)
\end{remark}

\section{The Proof of Polo's Conjecture}\label{chap6-sec6.2}

Clearly Proposition \ref{chap6-prop6.1.9} presents a smaller problem
than the one suggested by Lemma \ref{chap6-lem6.1.5}. In this section
we prove the Proposition \ref{chap6-prop6.1.9} and thus also:

\begin{theorem}[Mathieu; Polo's Conjecture]\label{chap6-thm6.2.1}
Let $\lambda\in X(T)^{-}$ and let $\mu\in X(T)$. Then $\lambda\otimes
P(\mu)$ has excellent filtration.
\end{theorem}

Apart\pageoriginale from\label{page62} Proposition \ref{chap6-prop6.1.9} one must
also must worry about separability. But fortunately this does not
require a thorough understanding of fibres. One only needs to show that
the source of our map is a finite union of pieces on which the map to
``image piece'' is separable. The pieces to take are the
$\overline{BuB}\times^{B}$ (component of $S$) of Proposition
\ref{chap6-prop6.1.9}, basically. One easily finds subvarieties that
actually map birationally to the image of the piece. We leave it at
this sketch for now and return to the proof of Proposition
\ref{chap6-prop6.1.9}. 

We first note that if $u=s_{1}\ldots s_{n}$ is a reduced expression of
$u\in W$, then the multiplication map $m:\overline{BuB}\times^{B}S\to
G/P_{\nu}$ can be lifted to the projection
$$
P_{s_{1}}\times^{B}\ldots P_{s_{n}}\times^{B}S\to G/P_{\nu}.
$$

The fibres of this projection map surjectively onto the fibres of
$m$. Further, the study may be broken up into little pieces like this:
$$
P_{s_{1}}\times^{B}\ldots P_{s_{n}}\times^{B}S\to
P_{s_{1}}\times^{B}\ldots P_{s_{n-1}}\times^{B}P_{s_{n}}S\to
G/P_{\nu}.
$$

So the trick is to show (cf.\@ Lemma \ref{chap6-lem6.1.7}) that
$\psi:P_{s}\times^{B}S\to P_{s}S$ does have the direct image property.

Say $C$ is the cokernel of the map $\O_{P_{s}S}\to
\psi_{*}\O_{P_{s}\times^{B}S}$. We need to show that
$H^{0}(P_{s}S,C\otimes \L(n\nu))$ vanishes for large $n$. (That will
show $C=0$ by ampleness, cf.\@ \cite[II 14.6 (4)]{key11}.)

Consider the following diagram
\[
\xymatrix{
P_{s}\times^{B}S\ar[d]^{\pi} \ar[r]^-{\psi} & P_{s}S\subset G/P_{\nu}\\
P_{s}/B 
}
\]

We have
\begin{align*}
H^{0}(P_{s}\times^{B}S,\psi^{*}\L(n\nu)) &=
H^{0}(P_{s}/B,\pi_{*}\psi^{*}\L(n\nu))\\ 
&= H_{s}(H^{0}(S,\L(n\nu))).
\end{align*}

Therefore, we have a natural injective map 
$$
H^{0}(P_{s}S,\L(n\nu))\to
H^{0}(P_{s}\times^{B}S,\psi^{*}\L(n\nu))=H_{s}(H^{0}(S,\L(n\nu))).
$$ 
By
Exercise \ref{exer-A.2.9} the proof of Proposition \ref{chap6-prop6.1.9} will be
finished once we have the following lemma.

\begin{lemma}\label{chap6-lem6.2.2}
For\pageoriginale any\label{page63} $B$-invariant closed subset $S$ of $G/B$, any simple
reflection $s$ and $\lambda\in X(T)^{-}$, the natural map
$$
H^{0}(P_{s}S,\L_{\lambda})\to H_{s}(H^{0}(S,\L_{\lambda}))
$$ 
is an isomorphism. 
\end{lemma}

\begin{proof}
We will prove the lemma by induction on ``size'' of $S$. Note that if
$S$ is irreducible, {\em i.e.} when $S$ is a Schubert variety $X_{w}$,
the image $P_{s}S$ is either $X_{sw}$ (when $sw>w$) or $X_{w}$. In
either case the lemma is true. Therefore we assume that the lemma is
true if we substitute for $S$ any of its proper $B$-invariant closed
subvarieties. 

Now we write $S$ as $X_{w}\cup S'$, and we may replace $S'$ by $S'\cup
\p X_{w}$ to make sure we understand $S'\cap X_{w}$ well. Indeed
$S'\cap X_{w}$ is now $\p X_{w}$ (even scheme theoretically by
Ramanathan). And of course we mean that $X_{w}$, $S'$ are really
smaller than $S$. By the Mayer-Vietoris Lemma \ref{chap2-lem2.2.11} we
have an exact sequence $0\to H^{0}(S,\L)\to H^{0}(X_{w},\L)\oplus
H^{0}(S',\L)\to H^{0}(\p X_{w},\L)\to 0$. This gives an exact sequence
$0\to H_{s}(H^{0}(S,\L))\to H_{s}(H^{0}(X_{w},\L))\oplus
H_{s}(H^{0}(S',\L))\to H_{s}(H^{0}(\p X_{w}))$.

Thus what remains to be checked is that $P_{s}S'\cap
P_{s}X_{w}=P_{s}\p X_{w}$, to make the computation go. If $sw<w$, then
$P_{s}\p X_{w}=X_{w}=P_{s}X_{w}$, and $X_{w}\subset P_{s}S'$.

If $sw>w$, then $P_{s}X_{w}=X_{sw}$ and we need that for $z\in W$,
$z\neq w$, $z\neq sw$, $sz\leq sw$ implies $z<w$. (The $z$ to be taken
are such that $BzB\subset S'$.) That is indeed so, and a reference is
\cite[5.9]{key10}. (The reader can take this as an exercise!) 
\end{proof}

We still have to explain how to handle the details of the separability
issue. We do this in a series of exercises. The reader is assumed to
be familiar with standard coordinates in Bruhat cells, as explained
for instance in \cite[Chapter 10]{key34}.

\begin{exercise}\label{chap6-exer6.2.3}
Let $g:Z\to X$, $f:X\to Y$ be maps between varieties, with $g$
surjective, so tht $fg$ is separable. Then $f$ is separable.
\end{exercise}

\begin{exercise}\label{chap6-exer6.2.4}
More generally, let $g_{i}:Z_{i}\to X$, $i=1,\ldots n$, $f:X\to Y$ be
maps between varieties, with $\cup_{i}g_{i}(Z_{i})=X$, so that each
$fg_{i}$ is separable to its image. Then $f$ is separable to its
image. 
\end{exercise}

\begin{exercise}\label{chap6-exer6.2.5}
Let\pageoriginale $f:X\to Y$\label{page64} be a separable $P_{\mu}$-equivariant
map. Then it induces a separable map $G\times^{P_{\mu}}X\to
G\times^{P_{\mu}}Y$.

\noindent
(Hint: Use that the fibrations $G\times^{P_{\mu}}X\to G/P_{\mu}$ and
$G\times^{P_{\mu}}Y\to G/P_{\mu}$ are locally trivial.)
\end{exercise}

\begin{exercise}\label{chap6-exer6.2.6}
Let $z$, $u$ be as in the proof of Proposition \ref{chap6-prop6.1.8},
with $P=P_{\lambda}$ and let $C$ be a component of $S$. Let $U_{z}$ be
the subgroup of $U$ generated by the root groups $U_{\alpha}$ with
$U_{\alpha}z\cap P=(e)$. Then $a\mapsto \overline{az}$ maps $U_{z}$
isomorphically to its image in $G/P$. Furthermore the rule
$(a,b,c)\mapsto (a,\overline{azbc})$ maps $U_{z}\times
\overline{BuB}\times C$ separably to its image in $U_{z}\times
G/P_{\nu}$.
\end{exercise}

Hint:~ Replace $\overline{BuB}$ and $C$ by suitable subvarieties to
make to make the map $(b,c)\mapsto \overline{bc}$ birational towards
$\overline{BuB}C$ and use the automorphism $(a,\overline{b})\mapsto
(a,\overline{ab})$ of $U_{z}\times G/P_{\nu}$.

\begin{exercise}\label{chap6-exer6.2.7}
Now check that the map needed in the proof of Polo's conjecture is
indeed separable.
\end{exercise}

\section{Variations and Questions}\label{chap6-sec6.3}

We start with an analogue of Donkin's restriction conjecture. Let $P$
be a parabolic subgroup corresponding with a subset $I$ of the simple
roots, so that $P$ is generated by $B$ and the $U_{-\alpha}$ with
$\alpha\in I$. Let $L$ be the Levi factor of $P$ with Borel subgroup
$B\cap L$ generated by $T$ and the $U_{\alpha}$ with $\alpha\in I$.

\begin{theorem}\label{chap6-thm6.3.1}
If $M$ is a $B$-module with excellent filtration, then
$\res^{B}_{B\cap L}\break M$ is a $B\cap L$-module with excellent filtration.
\end{theorem}

\begin{remark}\label{chap6-rem6.3.2}
Note that one may just as well restrict to $B\cap L'$, where $L'$ is
the commutator subgroup of $L:$ Any $B\cap L$-module breaks up into a
direct sum of weight spaces for the action of the center of $L$. These
weight spaces are $B\cap L'$-modules and they have excellent
filtration as $B\cap L'$-modules if and only if they have one as
$B\cap L$-modules. If you wish this is so by definition.
\end{remark}

\noindent
{\bf Proof of theorem:}~ We may assume $M$ is finite
dimensional. Choose an anti-dominant weight $\delta$ whose stabilizer
in $W$ is the Weyl group $W(L)$ of $L$. Thus $\delta$ lies in the
reflecting hyperplanes of the simple reflections corresponding with
the elements of $I$, but not in the other reflecting hyperplanes (see
\cite[1.12]{key9}).\pageoriginale Let\label{page65} $C$ be the closure of the
anti-dominant chamber. Then $\delta$ lies in the interior of
$\cup_{w\in W(L)}wC$. As this union is a cone, it follows that for $n$
sufficiently large $\mu+n\delta$ is in the cone for every weight $\mu$
of $M$. We proceed with such $n$ and study $M\otimes k_{n\delta}$,
which has excellent filtration by Polo's conjecture. Now for a $B\cap
L$-module having an excellent filtration it does not matter whether
one twiss by $\delta:$ all that changes is the action of the center of
$L$. So we may further assume that all weights of $M$ lie in
$\cup_{w\in W(L)}wC$. In other words, in the excellent filtration of
$M$ all the $P(\lambda)$ that occur have their $\lambda$ in the
$W(L)$-orbit of an element $\lambda_{1}$ of $X(T)^{-}$. Write
  $P(\lambda)=H_{s_{1}}H_{s_{2}}\ldots H_{s_{r}}(\lambda_{1})$ with
  the $s_{i}$ simple reflections that are in $W(L)$. Noting that
  $P_{s}/B=P_{s}\cap L/B\cap L$, we get $\res^{B}_{B\cap
    L}P(\lambda)=H^{L}_{s_{1}}H^{L}_{s_{2}}\ldots
  H^{L}_{s_{r}}(\lambda_{1})$, where $H^{L}_{s_{i}}$ is the analogue
  of $H_{s_{i}}$ in the context of $L:H^{L}_{s_{i}}=\ind^{P_{s}\cap
    L}_{B\cap L}$. So the restriction property holds for all relevant
  $P(\lambda)$.\hfill$\Box$
 
\begin{exercise}\label{chap6-exer6.3.3}
State and prove a similar result for relative Schubert filtrations.
\end{exercise}

Polo has introduced another notion, viz.\@ that of having a Schubert
filtration.\index{B}{Schubert filtration} We first give the definition, then relate it to other
concepts to show that the analogue of Polo's conjecture holds for
Schubert filtrations too. (This was proved by Polo under some
restrictions.) 

\begin{definition}\label{chap6-defi6.3.4}
A finite dimensional $B$-module $M$ has a {\em Schubert filtration} if
and only if there exists a filtration $0=F_{0}\subset F_{1}\subset
\ldots \subset F_{r}=M$ by $B$-modules such that
$F_{i}/F_{i-1}=H^{0}(S_{i},\L(\lambda_{i}))$ for some $\lambda_{i}\in
X(T)^{-}$. Here the $S_{i}$ are unions of Schubert varieties and
$r\geq 0$.
\end{definition}

In \cite{key27} Polo proves the following cohomological criterion for
having a Schubert filtration. If $\lambda\in X(T)^{-}$, $y\leq w$ in
$W$, put $K(w,y,\lambda)=\ker P(w\lambda)\to P(y\lambda)$. 

\begin{theorem}[Polo]\label{chap6-thm6.3.5}
Let $M$ be a finite dimensional $B$-module. Then $M$ has a Schubert
filtration if and only for all $\lambda\in X(T)^{-}$ and $y\leq w$ in
$W$ the module $M\otimes K(w,y,\lambda)$ is $B$-acyclic.
\end{theorem}

From this it follows that if
$$
0\to M'\to M\to M''\to 0
$$
is\pageoriginale exact,\label{page66} and $M'$, $M$ have Schubert filtration, then
so does $M''$.

Clearly, a module with Schubert filtration also has a filtration by
relative Schubert modules. Also, if $s$ is a simple reflection and $M$
is a module with Schubert filtration, then $M$ is acyclic for $H_{s}$
and $H_{s}(M)$ has Schubert filtation. This follows by imitating the
proof of Lemma \ref{chap3-lem3.2.11} with the help of Lemma
\ref{chap6-lem6.2.2}. From Lemma \ref{chap6-lem6.2.2} one then
concludes that in fact a relative Schubert module $M$ is already
acyclic for $H_{s}$. (Another way to see this is through the formula
$H^{i}_{s}(M)=H^{i}(B,H_{s}(k[B])\otimes M)$, see \cite[I
  4.10]{key11}. As $k[B]$ is injective, $H_{s}(k[B])$ has excellent
filtration and $H_{s}(k[B])\otimes M$ is $B$-acyclic.) This will be
used in the proof of

\begin{proposition}\label{chap6-prop6.3.6}
For a $B$-module $M$ the following are equivalent.
\begin{itemize}
\item[\rm(i)] $M$ has a Schubert filtration.

\item[\rm(ii)] The evaluation map $\ind^{G}_{B}(M)\to M$ is
  surjective, its kernel has a relative Schubert filtration and
  $\ind^{G}_{B}(M)$ has a good filtration.

\item[\rm(iii)] There is a module with good filtration $N$ and a
  surjective $B$-modu\-le map $N\to M$ whose kernel has relative
  Schubert filtration.
\end{itemize}
\end{proposition}

\begin{proof}
(Sketchy)

(i) $\Rightarrow$ (ii). If the Schubert filtration of $M$ has just one
  layer, (ii) follows easily. The general case then follows using
  acyclicity for induction.

(ii) $\Rightarrow$ (iii). Obvious.

(iii) $\Rightarrow$ (i). Let $K$ be the kernel of $N\to M$. We must
  show that $M\otimes K(w,y,\lambda)$ is $B$-acyclic. As $M$ has
  relative Schubert filtration, the problem is to show that
  $H^{0}(B,M\otimes P(w\lambda))\to H^{0}(B,M\otimes P(y\lambda))$ is
  surjective. It suffices to show that $H^{0}(B,N\otimes
  P(w\lambda))\to H^{0}(B,M\otimes P(y\lambda))$ is surjective. Now
  $H^{0}(B,N\otimes P(w\lambda))=H^{0}(G,\ind^{G}_{B}(N\otimes
  P(w\lambda)))=H^{0}(G,\ind^{G}_{B}(N\otimes
  P(y\lambda)))=H^{0}(B,N\otimes P(y\lambda))$. But $K\otimes
  P(y\lambda)$) is $B$-acyclic.
\end{proof}

\begin{corollary}\label{chap6-coro6.3.7}
Let $\lambda$ be a dominant or an anti-dominant weight and let $M$
have Schubert filtration. Then $P(\lambda)\otimes M$ has Schubert
filtration. 
\end{corollary}

\begin{proof}
Write $M$ as a quotient of a module with good filtration by one with
relative Schubert filtration and use that the analogue of the
corollary holds for those concepts.
\end{proof}

\begin{remark}\label{chap6-rem6.3.8}
Mathieu's\pageoriginale proof\label{page67} of this corollary (for anti-dominant
$\lambda$) was 
similar to his proof of Polo's conjecture. It did not rely on Polo's
conjecture, like ours does.
\end{remark}

We now list some open questions which are related to those answered in
these notes.

\smallskip
\noindent
{\bf Question 1} We know that excellent tensor excellent need not be
excellent, (see Example \ref{chap5-exam5.3.1}). No counterexamples are
known to the following question: Is excellent tensor excellent
relative Schubert? That is, is the tensor product of three modules
with excellent filtration $B$-acyclic?

\smallskip
\noindent
{\bf Question 2} 
Define that $M$ {\em preserves excellence} if $M\otimes$ excellent =
excellent. Using the cohomological criteria one sees this is
equivalent to ``preserving the existence of a relative Schubert
filtration''. (It is equivalent to $M\otimes P(\lambda)\otimes Q(\mu)$
being $B$-acyclic for all $\lambda$, $\mu$.) In particular, it implies
that $M\otimes Q(\rho)=M\otimes k_{\rho}$ has relative Schubert
filtration. Mathieu conjectures the converse: $M$ preserves excellence
if $M\otimes k_{\rho}$ has relative Schubert filtration.

\begin{remark}\label{chap6-rem6.3.9}
There are many related questions one may ask. We do not know for what
tensor products one should expect $B$-acyclicity. It undoubtedly has
to do with the facets the weights of the socles lie on. 
\end{remark}



