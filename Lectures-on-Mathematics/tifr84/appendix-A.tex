\makeatletter

\def\@makechapterhead#1{%
  \vspace*{50\p@}%
  {\parindent \z@ \raggedright \normalfont
    \ifnum \c@secnumdepth >\m@ne
      \if@mainmatter
        \huge\bfseries Appendix\space \@Alph\thechapter
       \par\nobreak
        \vskip 20\p@
      \fi
    \fi
    \interlinepenalty\@M
    \Huge \bfseries #1 \par\nobreak
    \vskip 40\p@
  }}

\c@tocdepth=1
\renewcommand\thesection{\@Alph\thechapter.\@arabic\c@section}

\def\@chapter[#1]#2{\ifnum \c@secnumdepth >\m@ne
                       \if@mainmatter
                         \refstepcounter{chapter}%
                         \typeout{\@chapapp\space\thechapter.}%
                         \addcontentsline{toc}{chapter}%
                                   {\protect\numberline{\@Alph\thechapter}
                         Appendix on #1}%
                       \else
                         \addcontentsline{toc}{chapter}{#1}%
                       \fi
                    \else
                      \addcontentsline{toc}{chapter}{#1}%
                    \fi
                    \chaptermark{#1}%
                    \addtocontents{lof}{\protect\addvspace{10\p@}}%
                    \addtocontents{lot}{\protect\addvspace{10\p@}}%
                    \if@twocolumn
                      \@topnewpage[\@makechapterhead{#2}]%
                    \else
                      \@makechapterhead{#2}%
                      \@afterheading
                    \fi}

\makeatother

\setcounter{chapter}{0}
\chapter{Geometry}\label{appndx-A}
%\addtocontents{toc}{chapter}{Appendix A: Geometry}

In\pageoriginale this\label{page75} appendix we give a more extensive discussion of
Frobenius spliting of varieties. Further we tie up some loose ends
that have more to do with algebraic geometry than with $B$-modules.

The notion of Frobenius split varieties was introduced by V.\@ Mehta
and A.\@ Ramanathan in 1984. We refer the reader to \cite{key32} for
historical remarks. Indeed, much of the material in this appendix is
copied from this source.


\section{Frobenius Splitting of Varieties}\label{sec-A.1}

In this section and the next some proofs are sketchy or absent. For
more information see \cite{key32}, \cite{key24}, \cite{key31}. Let $k$
be a algebraically closed field of characteristic $p>0$. Let $A$ be
any $k$-algebra. In this situation, we have the Frobenius ring
homomorphism $a\mapsto a^{p}$ of $A$. For a variety $X$ over $k$ we
have the absolute Frobenius morphism $F:X\to X$ which is induced by
the Frobenius ring homomorphism on any of its affine open
subsets. Note that the map $F$ is identity on the underlying
topological space of $X$ and on functions it is the $p^{\text{th}}$ power
map. By abuse of notation, we also use $F$ to denote the $p^{\text{th}}$ power
map $F:\O_{X}\to F_{\ast}\O_{X}$. If $\G$ is a coherent sheaf on $X$
then the direct image $F_{\ast}\G$ is the same as $\G$ as a sheaf of
abelian groups; only its $\O_{X}$-module structure $\circ$ is via the
Frobenius morphism, {\em i.e.} $f\circ g=f^{p}g$, of $f\in \O_{X}$ and
$g\in F_{\ast}\G$.

\begin{definition}\label{defi-A.1.1}
\begin{enumerate}
\item A variety $X$ over $k$ is called Frobenius split if the $p^{\text{th}}$
  power map $F:\O_{X}\to F_{\ast}\O_{x}$ has a splitting {\em i.e.} an
  $\O_{X}$-module morphism $\phi:F_{\ast}\O_{X}\to
  \O_{X}$\pageoriginale such\label{page76} that the composite $\phi F:\O_{X}\to
  \O_{X}$ is identity.

\item If $Y$ is a closed subvariety of $X$ with the ideal sheaf $\I$
  such that $\phi(F_{\ast}\I)=\I$ then we say $Y$ is compatibly split
  in $X$.

\item If $Y_{1},\ldots,Y_{n}$ are closed subvarieties which are all
  compatibly split by the same Frobenius splitting of $X$ then we say
  that the closed subvarieties $Y_{1},\ldots,Y_{n}$ are {\em
    simultaneously\index{B}{compatibly split!simultaneously} compatibly split} in $X$.
\end{enumerate}
\end{definition}

\begin{exercise}\label{exer-A.1.2}
Check that these definitions agree with those given earlier in
\ref{chap4-sec4.3}. 
\end{exercise}

The following remark was used by Ramanathan to study the scheme
theoretic intersection of two unions of Schubert varieties (cf.\@
proof of Mayer-Vietoris Lemma \ref{chap2-lem2.2.11}).

\begin{remark}\label{rem-A.1.3}
If $X$ is a scheme and $F:X\to X$ has a splitting then $X$ is
necessarily reduced. This is a consequence of the fact that the
Frobenius morphism is the $p^{\text{th}}$ power map on functions and if the
scheme is Frobenius split then this map is an injection.
\end{remark}

A Frobenius splitting of a variety $X$ is thus an element in the set
of global sections $H^{0}(X,(F_{\ast}\O_{X})^{*})$ of the dual of
$F_{\ast}\O_{X}$. Let us assume now that $X$ is a smooth variety of
dimension $n$. Let $\omega_{X}$ be its canonical bundle. 
Using duality\index{A}{duality [7]}
theory--an alternative will be discussed in section \ref{sec-A.3}--we
see that
\begin{align*}
H^{0}(X,(F_{\ast}\O_{X})^{\ast}) &= H^{n}(X,F_{\ast}\O_{X}\otimes
\omega_{X})\\
&= H^{n}(X,F_{\ast}(\O_{X}\otimes F^{\ast}\omega_{X}))\\
&= F_{\ast}H^{n}(X,\omega^{p}_{X})\\
&= H^{0}(X,\omega^{1-p}_{X}).
\end{align*}

The following proposition tells that a normal variety will be
Frobenius split if one of its desingularisation if Frobenius split.

\begin{remark}\label{rem-A.1.4}
Conversely, there are proofs of normality based on Fro\-benius
splittings, using Proposition \ref{prop-A.5.2}. See \cite{key25}. 
\end{remark}

\begin{proposition}\label{prop-A.1.5}
Let\pageoriginale $f:Z\to X$\label{page77} be a morphism of algebraic varieties. Assume that
$f_{\ast}\O_{Z}=\O_{X}$. (We will say that $f$ has the direct image
property.)\index{B}{direct image property} Then,
\begin{itemize}
\item[\rm(i)] If $Z$ is Frobenius split then $X$ is also Froenius
  split.

\item[\rm(ii)] If $Y$ is a closed subvariety of $Z$ which is
  compatibly split in $Z$ then its image $f(Y)$ is compatibly split in $X$.
\end{itemize}
\end{proposition}

\begin{proof}
\begin{itemize}
\item[(i)] For an open subset $U$ of $X$ the splitting gives an
  element of $\End_{F}(\O_{Z}(f^{-1}(U)))$ that sends the function $1$
to itself.

\item[(ii)] Let $\I\subset \O_{Z}$ be the ideal sheaf of $Y$. Then as
  $f_{\ast}\O_{Z}=\O_{X}$, the ideal sheaf of $f(Y)$ is
  $f_{\ast}\I$. Now it is an easy exercise to see that the ``pushed''
  splitting of $X$ splits $f(Y)$.
\end{itemize}
\end{proof}

\begin{lemma}\label{lem-A.1.6}
If a splitting of the variety $X$ is compatible with the subvarieties
$Y_{1}$ and $Y_{2}$ then it is also compatible with $Y_{1}\cap Y_{2}$
and $Y_{1}\cup Y_{2}$.

It is compatible with a subvariety $Y$ if and only if it is compatible
with each irreducible component of $Y$.
\end{lemma}

\begin{proof}
For the first part one uses that $\I_{Y_{1}\cap
  Y_{2}}=\I_{Y_{1}}+\I_{Y_{2}}$ and $\I_{Y_{1}\cup
  Y_{2}}=\I_{Y_{1}\cap \I_{Y_{2}}}$. For the second one shows that a
splitting $\sigma$ is compatible with a subvariety $Z$ if and only if
there is an open subset $U$ such that $U\cap Z$ is dense in $Z$ and
such that $\sigma|_{U}$ is compatible with $U\cap Z$.
\end{proof}

Now we give a criterion for a section of $\omega^{1-p}_{X}$ of a
smooth variety to be a splitting.

\begin{proposition}\label{prop-A.1.7}
Let $Z$ be a smooth projective variety of dimension $n$. Let
$Z_{1},\ldots,Z_{n}$ be smooth irreducible subvarieties of codimension
$1$ such that the scheme theoretic intersection
$Z_{i_{1}}\cap\ldots\cap Z_{i_{r}}$ is smooth irreducible and of
dimension $n-r$ for all $1\leq i_{1}<\ldots<i_{R}\leq n$. If there
exists a section $s\in H^{0}(Z,\omega^{-1}_{X})$ such that ${\rm
  div}\,(s)$, the divisor of zeroes of $s$, is $Z_{1}+\cdots+Z_{n}+D$
where $D$ is an effective divisor not passing through the point
$P=Z_{1}\cap\ldots\cap Z_{n}$ then the section $\sigma=s^{p-1}$ of
$\omega^{1-p}_{X}$ gives, by duality, a splitting of $Z$ (or a
non-zero multiple of one) which makes all the intersections
$Z_{i_{1}}\cap\ldots\cap Z_{i_{r}}$ compatibly split.
\end{proposition}

Note\pageoriginale that\label{page78} an element $\sigma$ of $\End_{F}(X)$ is a
splitting if and only if $\sigma(1)=1$. If $X$ is projective, then in
any case $\sigma(1)$ is a global function, hence constant. Thus it
suffices to check its value at a single point. In the case of the
proposition one uses the point $P$ and makes a computation in local
coordinates. 

Let $G$ be a connected simply connected semi-simple algebraic group
over $k$. (Or let it be as in 2.2.8.) Let $T$ be a maximal torus,
$B\supset T$ a Borel subgroup and $W=N(T)/T$ the Weyl group of
$G$. Let $w_{0}\in W$ denote the longest element of the Weyl group.

The homogeneous space $G/B$ is a projective variety. A closure of a
$B$-orbit in $G/B$ is called a Schubert variety. The $B$-orbits in
$G/B$ are indexed in a natural way by elements of $W$. If $P\supset B$
is a parabolic subgroup of $G$, then there are only finitely many
$B$-orbits in the projective variety $G/P$. We refer the reader to
Kempf's paper (\cite{key14}), for basic facts about the geometry of
Schubert varieties.

Let $D$ denote the divisor sum of all codimension one Schubert
varieties of $G/B$. Let $\tilde{D}$ denote the sum of $w_{0}$
translates of codimension one Schubert varieties. Then the divisor
$D+\tilde{D}$ gives the anti-canonical bundle $\omega^{-1}_{G/B}$ of
$G/B$. It is the image of a divisor in a Demazure resolution that
satisfies the criterion \ref{prop-A.1.7} for a splitting and by
pushing forward with Lemma \ref{lem-A.1.5} one gets a splitting which
simultaneously splits all the Schubert varieties of $G/B$. Therefore
we have the following theorem.

\begin{theorem}\label{thm-A.1.8}
Let $G$ be connected simply connected semi-simple algebraic group. Let
$P$ be a parabolic subgroup of $G$. Then the projective variety $G/P$
is Frobenius split. Further, all the Schubert varieties of $G/P$ are
simultaneously compatibly split.
\end{theorem}

\begin{proof}
One uses Lemma \ref{lem-A.1.6} to deal with Schubert varieties of
higher codimension.
\end{proof}

\begin{theorem}\label{thm-A.1.9}
\begin{enumerate}
\renewcommand{\labelenumi}{\rm\theenumi.}
\item The product $G/B\times G/B$ is Frobenius split. Further the
  diagonal $\Delta=\{(x,x)\mid x\in G/B\}$ is compatibly split in
  $G/B\times G/B$.

\item The variety $G\times^{B}(G/B\times G/B)$ is Frobenius
  split. Further all the double Schubert varieties are simultaneously
  compatibly split.
\end{enumerate}
\end{theorem}

This will be proved below (Propositions \ref{prop-A.4.9} and
\ref{prop-A.4.8}). 

\section{Applications of Frobenius
  Splitting}\label{sec-A.2}\pageoriginale

In\label{page79} this section we prove certain vanishing theorems for the Frobenius
split variety $X$.

First some remarks on the direct and inverse images of sheaves under
the absolute Frobenius morphism $F$. Let $\M$ be a sheaf of
$\O_{X}$-modules on $X$. Recall that the direct image sheaf
$F_{\ast}\M$ is the same as $\M$ as a sheaf of abelian groups, but the
$\O_{X}$-module structure is changed to $f\circ m=f^{p}m$, for $f\in
\O_{X}$ and $m\in \M$. As a way of notation, we will identify $\M$ and
$F_{\ast}\M$ as sets. The pullback $F^{\ast}M$ is by definition
$\M\otimes_{\O_{X}}F_{\ast}\O'_{X}$. Here the prime has been put in to
denote that the $\O_{X}$-module structure is given by the usual
multiplication on the second factor, {\em i.e.} $f(m\otimes
g)=fm\otimes g=m\otimes fg$ (and not $m\otimes f^{p}g$). The sheaf
$\M\otimes F_{\ast}\O_{X}$ with its $\O_{X}$-module structure coming
from $\M$, {\em i.e.} $f(m\otimes g)=fm\otimes g=m\otimes f^{p}g$, is
by definition $F_{\ast}F^{\ast}\M$. This gives us the projection
formula: $F_{\ast}F^{\ast}\M=\M\otimes_{\O_{X}} F_{\ast}\O_{X}$.

If we consider a line bundle $\L$ on $X$, we get a natural isomorphism
$F^{\ast}\L\approx \L^{p}$. Tensoring the Frobenius exact sequence
$$
0\to \O_{X}\to F_{\ast}\O_{X}\to \C\to 0
$$
by $\L$ and taking the cohomology, we get a natural map
$$
H^{i}(X,\L)\to H^{i}(X,\L\otimes
F_{\ast}\O_{X})=H^{i}(X,F_{\ast}F^{\ast}\L)=H^{i}(X,F,\L^{p}).
$$

\begin{proposition}\label{prop-A.2.1}
Let $X$ be a projective variety which is Frobenius split. Let $Y$ be a
closed subvariety of $X$ which is compatibly split. Let $\L$ be a line
bundle on $X$ such that $H^{i}(X,\L^{m})=H^{i}(Y,\L^{m})=0$ for some
$i$ and for all large $m$. Then $H^{i}(X,\L)=0=H^{i}(Y,\L)$.
\end{proposition}

\begin{proof}
We have a natural map $H^{i}(X,\L)\to
H^{i}(X,F_{\ast}\L^{p})$. Further as $F$ is affine ({\em i.e.} inverse
image of an affine open set is affine), it commutes with the
cohomology. Thus $H^{i}(X,F_{\ast}\L^{p})=H^{i}(X,\L^{p})$. Now as the
sequence
$$
0\to \L\to \L\otimes F_{\ast}\O_{X}\to \L\otimes \C\to 0
$$
is split exact this morphism is injective. Therefore, by iteration, we
have an injective morphism $H^{i}(X,\L)\to H^{i}(X,\L^{p^{\nu}})$ for
all $\nu$. Thus $H^{i}(X,\L^{p\nu})=0$ implies that
$H^{i}(X,\L)=0=H^{i}(Y,\L)$. 
\end{proof}

The\pageoriginale above\label{page80} proposition together with Serre vanishing
theorem gives us the following corollary.

\begin{corollary}\label{coro-A.2.2}
Let $\L$ be an ample line bundle on $X$. If $X$ is Frobenius split,
then $H^{i}(X,\L)=0$ for all $i>0$. Further, if $Y\subset X$ is
compatibly split, $H^{i}(Y,\L)=0$ and the restriction map
$H^{0}(X,\L)\to H^{0}(Y,\L)$ is surjective.
\end{corollary}

\begin{proof}
To see the surjectivity of the restriction map, we consider
\[
\xymatrix{
H^{0}(X,\L)\ar[d]\ar[r] & H^{0}(X,\L^{p^{\nu}})\ar[d]\\
H^{0}(Y,\L)\ar[r] & H^{0}(Y,\L^{p^{\nu}})
}
\]

As the horizontal arrows are split, it is enough to see the
surjectivity of the global sections for a high power of $\L$. Thus the
result. 
\end{proof}

For Schubert varieties Ramanathan proved something better than what
one can achieve with the above. He also deals with base point free
line bundles on $G/B$ that are not ample. So he deals with the
$\L(\lambda)$ with $\lambda$ anti-dominant, but not regular
anti-dominant. We need this stronger result. Therefore let us now
discuss a more refined notion of splitting (although we have no other
application than this stronger result of Ramanathan).

\begin{definition}\label{defi-A.2.3}
Let $\L$ be a line bundle on $X$ and $s:\O_{X}\to \L$ a non-zero
section of $\L$ with zeroes precisely on $D$.
\begin{enumerate}
\item We say $X$ is Frobenius $D$-split (or less precisely Frobenius
  $\L$-split) if there exists $\psi:F_{\ast}\L\to \O_{X}$ such that
  the composite $\phi=\psi F_{\ast}(s)$
\[
\xymatrix{
F_{\ast}\O_{X}\ar[dr]_{F_{\ast}(s)} \ar[rr]^{\phi} & & \O_{X}\\
 & F_{\ast}\L\ar[ur]_{\psi} & 
}
\]
is a Frobenius splitting of $X$.

\item If $Y$ is a closed subvariety of $X$ such that
\begin{enumerate}
\renewcommand{\theenumii}{\roman{enumii}}
\renewcommand{\labelenumii}{(\theenumii)}
\item no irreducible component of $Y$ is contained in the support
  $\supp D$, 

\item $\phi$\pageoriginale gives\label{page81} a compatible splitting of $Y$ in $X$,
\end{enumerate}
then we say $Y$ is compatibly $D$-split in $X$.

\item If all subvarieties $Y_{1},\ldots,Y_{r}$ are compatibly
  $D$-split by the same $D$-splitting of $X$ then we say that
  $Y_{1},\ldots,Y_{r}$ are simultaneously compatibly $D$-split in $X$.
\end{enumerate}
\end{definition}

\begin{remark}\label{rem-A.2.4}
\begin{enumerate}
\item We note that if $X$ is Frobenius split, it is also
  $\omega^{1-p}_{X}$-split, as any section which gives a splitting
  vanishes on a divisor whose associated line bundle is $\omega^{1-p}_{X}$.

\item Let $D'$ be another Cartier divisor with $0\leq D'\leq D$. Then
  if $X$ is $D$-split it is also $D'$-split.
\end{enumerate}
\end{remark}

We now see a consequence of $D$-splittings.

\begin{proposition}\label{prop-A.2.5}
If $X$ is $\L$-split with $\L$ ample and $\M$ is a line bundle without
base points (i.e. for every $x\in X$, there exists $s\in H^{0}(X,\M)$
such that $s(x)\neq 0$) then $H^{i}(X,\M)=0$ for $i>0$. If further $Y$
is compatibly $\L$-split then $H^{i}(Y,\M)=0$ for $i>0$ and the
restriction map $H^{0}(X,\M)\to H^{0}(Y,\M)$ is surjective.
\end{proposition}

For the proof we refer the reader to Ramanathan \cite{key32}.

Let us now consider the case when $X$ is the projective homogeneous
space $G/B$. In this case the divisor $(p-1)(D+\tilde{D})$ gives a
splitting. The line bundle corresponding to the divisor $D$ is ample,
in fact it is the line bundle given by the character $-\rho$. Also as
$G/B$ is homogeneous, any homogeneous line bundle with a non-zero
section is base point free. Therefore we get the following
proposition.

\begin{proposition}[Ramanathan, {\cite[Theorem 3]{key31}}]\label{prop-A.2.6}
Let $\L$ be a line bundle on $G/B$ such that $H^{0}(G/B,\L)\neq
0$. Then $H^{i}(X,\L)=0$ for any union of Schubert varieties $X$ and
for all $i>0$. Further the restriction map $H^{i}(G/B,\L)\to
H^{i}(X,\L)$ is surjective for all $i$.
\end{proposition}

\begin{remark}\label{rem-A.2.7}
The case $X=G/B$ is known as Kempf's vanishing\index{B}{Kempf vanishing} theorem. 
\end{remark}

\begin{remark}\label{rem-A.2.8}
Let\pageoriginale $S$\label{page82} be a union of product of Schubert varieties. Consider the
fibration
\[
\xymatrix{
G\times^{B}S\ar[d]^{\pi}\\
G/B
}
\]

It is locally trivial in the Zariski topology (Exercise
\ref{chap1-exer1.2.1}), so the structure sheaf of $G\times^{B}S$ is
certainly flat over the base $G/B$. The proposition above gives us
that $R^{i}\pi_{\ast}\O=0$ for $i>0$ because $R^{i}\pi_{\ast}\O$ is a
vector bundle on $G/B$ with fibre isomorphic with $H^{i}(S,\O)$ which
vanishes as $\O$ is base point free on $G/B$. (Use \cite[Grauert's
  corollary to Semicontinuity]{key7}.) Similarly, if $P$, $Q$ are
parabolics and $\L$ is an ample line bundle (or one without base
points) on $G\times^{B}(G/P\times G/Q)$ then for any union $S$ of
products of Schubert varieties in $G/P\times G/Q$ the higher
$R^{i}f_{\ast}(\L|_{G\times^{B}S})$ vanish, where $f:G\times^{B}S\to
G/B$. So
$H^{i}(G/B,f_{\ast}(\L|_{G\times^{B}S}))=H^{i}(G\times^{B}S,\L|_{G\times^{B}S})$
by Leray (\cite[III, Ex. 8.1]{key7}).
\end{remark}

\begin{exercise}\label{exer-A.2.9}
Let $P$ be a parabolic and let $S$ be a union of Schubert varieties in
$G/P$. Argue as in the remark above to show that if $\L$ is a line
bundle on $G/P$, then $H^{i}(S,\L)=H^{i}(\pi^{-1}(S),\pi^{\ast}\L)$,
with $\pi:G/B\to G/P$. Next assume $\L$ is base point free and let
$S_{1}$ be a union of Schubert varieties in $G/B$ with
$\pi(S_{1})=S$. Show that $H^{0}(S_{1},\pi^{*}\L)=H^{0}(S,\L)$. 
\end{exercise}

\section{Cartier Operators and Splittings}\label{sec-A.3}

We now give another approach to the isomorphism $\cEnd_{F}(X)\approx
\omega^{1-p}_{X}$. It does not make reference to duality theory, but
only to the Cartier operator. With this description it will be quite
feasible to make explicit computations with splittings in local
coordinates, if the splittings are given as sections of
$\omega^{1-p}_{X}$. 

Let $X$ be a variety of dimension $n$ over $k$, with $k$ algebraically
closed of characteristic $p$, as usual. We consider the DeRham complex
$$
0 \to \Omega^{1}_{X}\to \cdots \to \Omega^{n}_{X}\to 0
$$
with as differential $d$ the usual exterior differentiation. Because
this differential is not $\O_{X}$-linear, we twist the $\O_{X}$-module
structure on $\Omega^{i}_{X}$ by putting $f\ast \omega=f^{p}\omega$
for a section $f\in H^{0}(U,\O_{X})$ and a differential $i$-form
$\omega\in H^{0}(U,\Omega^{i}_{X})$. With this twisted module
structure the DeRham complex is a\pageoriginale complex\label{page83} of coherent
$\O_{X}$-modules, and the exterior algebra
$\Omega^{*}_{X}=\oplus^{n}_{i=0}\Omega^{i}_{X}$ is a differential
graded $\O_{X}$-algebra. We denote its cohomology sheafs
$\H^{i}_{dR}$. So if $U$ is an affine open subset, then
$H^{0}(U,\H^{i}_{dR})$ consists of all closed differential $i$-forms
on $U$ modulo the exact ones. Now consider the map $\gamma:f\mapsto$
class of $f^{p-1}df$ from $\O_{X}$ to $\H^{1}_{dR}$. 

\begin{lemma}\label{lem-A.3.1}
$\gamma$ is a derivation and thus induces an $\O_{X}$-algebra
  homomorphism $c:\Omega^{*}_{X}\to \H^{*}_{dR}$.
\end{lemma}

\begin{remark}\label{rem-A.3.2}
Note that one should put the ordinary $\O_{X}$-module structure on
$\Omega^{*}_{X}$ here, not the twisted one that is used for
$\H^{*}_{dR}$. 
\end{remark}

\noindent
{\bf Proof of Lemma \ref{lem-A.3.1}:} With
$$
\Phi(X,Y)=((X+Y)^{p}-X^{p}-Y^{p})/p\in \mathbb{Z}[X,Y]
$$
we get
\begin{align*}
(f+g)^{p-1}d(f+g) &= f^{p-1}df+g^{p-1}dg+d\Phi(f,g)\\
(fg)^{p-1}d(fg) &= g\ast f^{p-1}df+f\ast g^{p-1}dg,
\end{align*}
where the first equality is a consequence of the fact that
$$
p(X+Y)^{p-1}d(X+Y)=pX^{p-1}dX+pY^{p-1}dY+pd\Phi(X,Y)
$$
in the torsion free $\mathbb{Z}$-module
$\Omega^{1}_{\mathbb{Z}[X,Y]}$.\hfill$\Box$ 

\begin{proposition}\label{prop-A.3.3}
If $X$ is smooth, the homomorphism $c$ is bijective. The inverse map
$C:\H^{*}_{dR}\to \Omega^{*}_{X}$ is called the Cartier
operator\index{B}{Cartier operator} 
(cf.\@ \cite{key26}).
\end{proposition}

\begin{proof}
To check that a map of coherent sheafs is an isomorphism it suffices
to check that one gets an isomorphism after passing to the completion
at an arbitrary closed point. But then we are simply dealing with the
DeRham complex for a power series ring in $n$ variables over $k$ and
everything can be made very explicity (exercise).
\end{proof}

\begin{remark}\label{rem-A.3.4}
Here are some formulas satisfied by the Cartier operator, in sloppy
notation. In view of these formulas the connection with Frobenius
splittings is not surprising.
\begin{itemize}
\item[(i)] $C(f^{p}\tau)=fC(\tau)$\pageoriginale

\item[(ii)] $C(d\tau)=0$\label{page84}

\item[(iii)] $C(\dlog f)=\dlog f$, where $\dlog f$ stands for
  $(1/f)df$ if $f$ is invertible (or inverted).

\item[(iv)] $C(\xi\wedge \tau)=C(\xi)\wedge C(\tau)$
\end{itemize}

Here $f$ is a function and $\xi$, $\tau$ are forms.
\end{remark}

\begin{proposition}\label{prop-A.3.5}
If $X$ is smooth, we have a natural isomorphism
$$
\cEnd_{F}(X)\approx \omega^{1-p}_{X}=\cHom(\omega^{p}_{X},\omega_{X}),
$$
where $\omega_{X}$ is the canonical line bundle $\Omega^{n}_{X}$. If
$\tau$ is a local generator of $\omega_{X}$, $f$ a local section of
$\O_{X}$, $\phi$ a local homomorphism $\omega^{p}_{X}\to \omega_{X}$,
then the corresponding local section $\sigma$ of $\cEnd_{F}(X)$ is
defined by $\sigma(f)\tau=C(\text{class of~}
\phi(f\tau^{\otimes_{p}}))$. 
\end{proposition}

\begin{proof}
One checks that $C$ (class of $\phi(f\tau^{\otimes_{p}}))/\tau$ does
not depend on the choice of $\tau$, so that $\sigma$ depends only on
$\phi$. To see that the map $\phi\mapsto \sigma$ defines an
isomorphism of line bundles we may argue as in the previous proof.
\end{proof}

\section{Canonical splitting of the Demazure
  resolution}\label{sec-A.4}

We wish to study $\cEnd_{F}(Z_{n},D_{n})$ for an arbitrary sequence
$s_{i_{1}},\ldots,s_{i_{n}}$ of simple reflections. In particular, we
wish to prove Proposition \ref{chap4-prop4.3.15}. We start with the
problem of recognizing $B$-equivariant bundles on $Z_{n}$.

\begin{lemma}\label{lem-A.4.1}
Let $X$ be a connected projective variety with $B$ action and $x$ an
invariant point. Let $\E$, $\F$ be $B$-equivariant line bundles that
are isomorphic as line bundles. If there fibres over $x$ are
$B$-equivariantly isomorphic, then the line bundles themselves are
$B$-equivariantly isomorphic.
\end{lemma}

\begin{proof}
Tensoring with $\E^{*}$ we reduce to the case that $\E$, $\F$ are
trivial as line bundles. Then
$$
H^{0}(X,\E)\approx H^{0}(\{x\},\E)\approx H^{0}(\{x\},\F)\approx
H^{0}(X,\E)
$$
equivariantly.\pageoriginale But\label{page85} the global sections generate a
trivial sheaf everywhere, so the $B$ action on such a sheaf is
determined by what it does on global sections.
\end{proof}

This lemma takes care of recognizing the $B$ action, so let us now
look at the Picard group of $Z_{n}$.

\begin{lemma}\label{lem-A.4.2}
The isomorphism type of a line bundle on $Z_{n}$ is determined exactly
by the degrees of the restrictions to the $n$ embedded
$\mathbb{P}^{1}$'s of the form
$B\times^{B}\cdots\times^{B}P_{i}\times^{B}\cdots\times^{B}B/B$. 
\end{lemma}

\begin{proof}
This is clear for $Z_{n,n}\approx \mathbb{P}^{1}$, so we work our way
back to $Z_{n}$ by means of the fibrations
$\pi_{j}:Z_{j,n}=P_{j}\times^{B}P_{j+1}\ldots P_{n}/B\to
P_{j}/B\approx \mathbb{P}^{1}$ with fibre $Z_{j+1,n}$. Use
\cite[Ch.\@ II, Prop.\@ 6.5]{key7} with as divisor the fibre of the
point ``at infinity'' $s_{i_{j}}$ of $P_{j}/B$ and observe that the
complement of this fibre is a direct product of $Z_{j+1,n}$ with an
affine line. Apply \cite[Ch.\@ II, Prop.\@ 6.6]{key7} to this complement.
\end{proof}

\begin{corollary}\label{coro-A.4.3}
Under the standing hypothesis 2.2.8 all line bundles on $G/B$ come
from $G$-equivariant ones. The equivariant structure is unique up to a
twist by a character of $G$. In particular, if $G$ is its own
commutator subgroup then the equivariant structure is unique.
\end{corollary}

\begin{proof}
The regular representation\index{A}{regular representation [11]} of $G$ restricts to a faithful
representation of its commutator group, so the fundamental weights of
the commutator group are restrictions of weights of $B$. Therefore the
set of degrees of restrictions to the projective lines $P_{s}/B(s\in
S)$ runs through all possibilities as we vary the line bundle over all
$\L(\lambda)$, $\lambda\in X(T)$. And a line bundle is clearly
determined by its pullback to a Demazure resolution of $G/B$. To
finish, argue as in the proof of Lemma \ref{lem-A.4.1}.
\end{proof}

\begin{exercise}\label{exer-A.4.4}
Let $P$ be a parabolic and $X$ a space with $B$ action. Show that
every $P$-equivariant vector bundle on $P\times^{B}X$ is of the form
$P\times^{B}\V$ with $\V$ a $B$-equivariant bundle on $X$.
\end{exercise}

The following lemma may be used to pass between
$\cEnd_{F}(Z_{n},D_{n})$ and $\cEnd_{F}(Z_{n})\otimes
\I^{p-1}_{D_{n}}$. 

\begin{lemma}\label{lem-A.4.5}
Let\pageoriginale $A$\label{page86} be a domain of characteristic $p$ and $(f)$ a
principal ideal in it. Then $\End_{F}(A,(f))=(f)^{p-1}\ast \End_{F}(A)$.
\end{lemma}

\begin{proof}
That the left-hand side contains the right-hand side is clear. Let
$\sigma\in \End_{F}(A,(f))$. Then $\sigma(fa)=f\tau(a)$ defines a map
$\tau$ from $A$ to itself. One checks that $\tau\in \End_{F}(A)$ and
that $f\sigma=f(f^{p-1}\ast\tau)$.
\end{proof}

\begin{proposition}\label{prop-A.4.6}
The sheaf $\cEnd_{F}(Z_{n},D_{n})$ is $B$-equivariantly isomorphic
with $\varphi^{*}\L((1-p)p)[(p-1)\rho]$, so that if $\varphi:Z_{n}\to
G/B$ is surjective, $\cEnd_{F}(Z_{n},D_{n})$ is $B$-equivariantly
isomorphic with 
$$
k_{(p-1)\rho}\otimes H^{0}(G/B,\L((1-p)\rho)).
$$ 
\end{proposition}

\begin{proof}
By Lemmas \ref{lem-A.4.5} and \ref{prop-A.3.5} all we have to show for
the first statement is that $\omega_{Z_{n}}(-D_{n})\cong
\varphi^{*}\L(\rho)[-\rho]$, equivariantly. We argue again by
induction, using the fibration
$\pi_{j}:Z_{j,n}=P_{j}\times^{B}P_{j+1}\ldots P_{n}/B\to
P_{j}/B\approx \mathbb{P}^{1}$ with fibre $Z_{j+1,n}$. Let $D_{j,n}$
denote the analogue of $D_{n}$ in $Z_{j,n}$. Thus $D_{j,n}$ is a
divisor with $n-j+1$ components intersecting in a point $x$. The
required result is easy for $j=n$. Indeed if $\alpha$ is the simple
root corresponding with $P_{n}$, one gets a local coordinate $t$ on
$P_{n}/B\approx \mathbb{P}^{1}$ from $t\mapsto x_{-\alpha}(t)B/B$ and
the stalk at the ``origin'' $x$ of $\omega_{Z_{n,n}}(-D_{n,n})$ is
generated by $dt/t$ on which $T$ acts trivially. Further the degree of
the line bundle is $-1$, so by our recognition Lemma \ref{lem-A.4.1}
we must have $\omega_{Z_{n,n}}(-D_{n,n})\cong
\varphi^{*}_{n,n}\L(\rho)[-\rho]$, equivariantly.

Now assume such a result for $\omega_{Z_{j+1,n}}(-D_{j+1,n})$ and
consider $\omega_{Z_{j,n}}\break (-D_{j,n})$. It is the tensor product of two
line bundles. The first one, say $\R$, is the relative canonical
bundle
$\omega_{Z_{j,n/\mathbb{P}^{1}}}=\wedge^{n-j}\Omega_{j,n/\mathbb{P}^{1}}$,
  twisted by $\L(P_{j}\times^{B}D_{j+1,n})$. The second is the
  pullback of $\omega_{\mathbb{P}^{1}}(-\{x\})$, with $x$ ``as
  above''. Let us study $\R$ through its restrictions to the various
  copies of $\mathbb{P}^{1}$, cf.\@ Lemmas \ref{lem-A.4.1} and
  \ref{lem-A.4.2}. By base change for relative differentials, see
  \cite[II, 8.2]{key7}, the restriction of
  $\omega_{Z_{j,n/\mathbb{P}^{1}}}$ to $B\times^{B}Z_{j+1,n}$ is just
  $\omega_{Z_{j+1,n}}$. So $\R$ restricts to
  $\omega_{Z_{j+1,n}}(-D_{j+1,n})$, which we know. We also need the
  restriction of $\R$ to $P_{j}/B$. Now that is a $P_{j}$-equivariant
  sheaf whose fibre at $x$ has trivial $T$ action, so it must be the
  structure sheaf on $P_{j}/B$. The sheaf
  $\omega_{\mathbb{P}^{1}}(-\{x\})$ we have already found to be the
  pullback from $G/B$ of $\L(\rho)[-\rho]$, and its pullback to
  $Z_{j,n}$ is easy to understand in terms of its restrictions to the
  relevant $\mathbb{P}^{1}$'s. So we have all the ingredients to
  conclude $\omega_{Z_{j,n}}(-D_{j,n})\cong
  \varphi^{*}\L(\rho)[-\rho]$, equivariantly. To prove the last
  statement of the proposition, use Exercise \ref{exer-A.4.4} and the
  fibrations $\pi_{j}$ to se that
  $H^{0}(\varphi^{*}\L((1-p)\rho))=H_{s_{1}}\circ \cdots\circ
  H_{s_{n}}((1-p)\rho)$. 
\end{proof}


\begin{proposition}[Proposition \ref{chap4-prop4.3.17}]\label{prop-A.4.7}
There\pageoriginale exists\label{page87} $\sigma\in \End_{F}(Z_{n},D_{n})$ which is
a canonical splitting.
\end{proposition}

\begin{proof}
We have already described in \ref{chap4-prop4.3.17} how one proves
this with the criterion \ref{prop-A.1.7}. Let us tell it a little
differently now. Let 
$$
s\in H^{0}(G/B,\L((1-p)\rho)[(p-1)\rho])
$$ 
be a
weight vector of weight zero. It does not vanish at $B/B$. We wish to
show that its pullback defines $\sigma_{n}\in \End_{F}(Z_{n},D_{n})$
with $\sigma_{n}(1)\neq 0$. As $Z_{n}$ is complete, $\sigma_{n}(1)$ is
a constnat function. Call the constant $c_{n}$. We argue by induction,
the case $n=0$ being easy. Now an exercise in chasing duality, say
with the Cartier operator, shows that the restriction of
$\sigma_{j,n}(1)$ to $Z_{j+1,n}$ is just $\sigma_{j+1,n}(1)$ in
hopefully self-explanatory notation. (Use reasonable identifications,
choose a local coordinate $t$ on $P_{j}/B$ which vanishes at $B/B$ and
use that the fibration $Z_{j,n}\to P_{j}/B$ is trivial in a
neighborhood of $B/B$.) So $c_{j,n}=c_{j+1,n}$, which is non-zero by
inductive assumption. This proves that up to a scalar multiple we have
produced a splitting, and by construction it has weight $0$ so that it
must be canonical because of the position of the weights of
$\End_{F}(Z_{N},D_{N})$. (See proof of \ref{chap4-prop4.3.17}).
\end{proof}

\begin{proposition}[\ref{thm-A.1.9} part. 2.]\label{prop-A.4.8}
Let $P$ and $Q$ be parabolic subgroups. There is a canonical splitting
on $G\times^{B}(G/P\times G/Q)$ which is compatible with all double
Schubert varieties.
\end{proposition}

\begin{proof}
Choose a reduced expression of a minimal representative of $w_{0}$
modulo the Weyl group of $P$. Let it be followed by a reduced
expression for $w_{0}$ and let that finally be followed by a reduced
expression for a minimal representative of $w_{0}$ modulo the Weyl
group of $Q$. Together that is a long expression based on which one
gets a $Z_{n}$ which maps birationally onto
$G\times^{P}G\times^{B}G/Q$ by ``multiplication''. This proper
birational map has the direct image property because the target is
normal. One now takes the canonical splitting of \ref{prop-A.4.7}. It
is compatible with all unions of intersections of components of
$D_{n}$.

Next note that $G\times^{B}(G/P\times
G/Q){\displaystyle{\mathop{\approx}^{\phi}}}G\times^{P}G\times^{B}G/Q$. The
map $\phi$ is defined by
$\phi(g,\overline{x},\overline{y})=(gx,x^{-1},\overline{y})$ (cf.\@
\ref{chap1-rem1.2.2}). The image of $G\times^{B}(X_{v}\times X_{w})$
under $\phi$ is $G\times^{P}(\overline{Pv^{-1}B})\times^{B}X_{w}$,
which is clearly the image of an intersection of components of
$D_{n}$. So the splitting is compatible with it.
\end{proof}

We are also ready to prove 

\begin{proposition}[\ref{thm-A.1.9} part 1.]\label{prop-A.4.9}
The\pageoriginale 
product\label{page88}  $G/B\times G/B$ is Frobenius split. Further the diagonal
$\Delta=\{(x,x)|x\in G/B\}$ is compatibly split in $G/B\times G/B$.
\end{proposition}

\begin{proof}
Take $Q=G$, $P=B$ in the previous proof and recall
(\ref{chap1-rem1.2.2}) that
$G\times^{B}G/B{\displaystyle{\mathop{\approx}^{\phi}}}G/B\times G/B$
with $\phi(g,\overline{h})=(\overline{g},\overline{gh})$. We get a
splitting which is compatible with $G\times^{B}B/B$, and that subspace
is mapped to the diagonal by $\phi$.
\end{proof}

\section{Two Technical Results}\label{sec-A.5}

\begin{sublemma}\label{sublem-A.5.1}
Let $X$, $Y$ be two quasi-projective schemes over an algebraically
closed field $k$ of characteristic $p>0$. Let $f:X\to Y$ be a
bijective proper morphism. Then for every line bundle $\L$ on $Y$ and
for $s\in H^{0}(X,f^{*}(\L))$ we have $s^{p^{n}}\in \text{\rm
  image}(H^{0}(Y,\L^{p^{n}}))$ for some large $n$.
\end{sublemma}

\begin{proof}
As $f$ is proper and quasi-finite, it is finite and affine. We may
assume $X$ and $Y$ to be reduced, in which case $H^{0}(Y,\L^{n})$ may
be identified with its image. Then the problem is local on $Y$. Thus
we may assume that $Y$ and $X$ are affine and that the line bundles
are trivial. We identify them with the structure sheafs. Say $Y=\Spec
(A)$, $X=\Spec(B)$, $A\subset B$. As $B$ is finite over $A$, we have a
bound on the dimension of $B\otimes_{\phi}k$ for any point $\phi:A\to
k$. We may replace $B$ by $B^{p}A$. Repeating that if necessary we may
assume that for all points $\phi$ the local artin algebra
$B\otimes_{\phi}k$ is reduced. But then it must simply be $k$, as $k$
is algebraically closed. By Nakayama's Lemma the map $A\to B$ is now
surjective at all points, hence surjective.
\end{proof}

\begin{proposition}\label{prop-A.5.2}
Let $f:X\to Y$ be a surjective, separable, proper morphism between two
varieties, with connected fibres. We assume that $Y$ is Frobenius
split. Then $f_{\ast}\O_{X}=\O_{Y}$.
\end{proposition}

\begin{proof}
By Stein factorisation we may assume $f$ to be finite. Then it is
actually a bijection, so that our earlier Lemma \ref{sublem-A.5.1}
applies. We may assume again that $X=\Spec(B)$, $Y=\Spec(A)$,
$A\subset B$ and we have to show that $A$ is $p$-root closed in $B$. 

First\pageoriginale consider\label{page89} a smooth point $x$ of $X$, such that the
tangent map is surjective at $x$ and $f(x)$ is smooth in $Y$. As the
dimensions are the same at $x$ and $f(x)$, the surjectivity of the
tangent map implies an ``analytic'' isomorphism $\hat{\O}_{x}\approx
\hat{\O}_{f(x)}$. Thus $f(x)$ is outside the support of the $A$-module
$B/A$. Therefore there is $c\in A$ which annihilates that module--one
says $c$ is in the conductor of $B$ over $A$--such that $c(f(x))\neq
0$.

Return to the question of $p$-root closure. Let $b\in B$ with
$b^{p}\in A$ and let $\sigma:A\to A$ be the splitting. For $c$ in the
conductor we have $cb$, $c$, $b^{p}\in A$, so $c\sigma
(b^{p})=\sigma(c^{p}b^{p})=cb$. So $b$ equals $\sigma(b^{p})$ at all
points where $c$ does not vanish. Varying $c$ we get a dense set of
points where $b$ equals $\sigma(b^{p})$, so $b\in \sigma(A)\subset A$.
\end{proof}

\newpage

\addcontentsline{toc}{chapter}{Bibliography}

\begin{thebibliography}{}
\bibitem{key1} N.\@ Bourbaki,\pageoriginale Groupes et Alg\`ebres de
  Lie, Ch.\@ 4, 5 et 6, Paris: Hermann 1968.

\bibitem{key2} E.\@ Cline, B.\@ Parshall, L.\@ Scott, Finite
  dimensional algebras and highest weight categories, J.\@ reine
  angew. Math. 391 (1988), 85-99.

\bibitem{key3} M.\@ Demazure, D\'esingularisation de vari\'et\'es de
  Schubert g\'en\'eralis\'ees, Ann. Sci. \'Ecole Norm. Super. 7
  (1974), 53-88.

\bibitem{key4} S.\@ Donkin, Rational Representations of Algebraic
  Groups: tensor products and filtrations, Lecture Notes in
  Mathematics 1140, Berlin: Springer 1985.

\bibitem{key5} S.\@ Donkin, Good filtrations of rational modules for
  reductive groups, Proc. Symp. in Pure Math. 47 (1987), 69-80.

\bibitem{key6} A.\@ Grothendieck and J.\@ Dieudonn'e, EGA IV (3),
  Publ. Math. IHES 28 (1966).

\bibitem{key7} R.\@ Hartshorne, Algebraic Geometry, Graduate Texts in
  Mathematics, Berlin: Springer 1977.

\bibitem{key8} J.E.\@ Humphreys, Introduction to Lie Algebras and
  Representation Theory, Graduate Texts in Mathematics, Berlin:
  Springer 1972.

\bibitem{key9} J.E.\@ Humphreys, Linear Algebraic Groups, Graduate
  Texts in Mathematics, Berlin: Springer 1975.

\bibitem{key10} J.E.\@ Humphreys, Reflection Groups and Coxeter
  Groups, Cambridge Studies in Advanced Math. 29, Cambridge: Cambridge
  University Press 1990.

\bibitem{key11} J.-C.\@ Jantzen,\pageoriginale 
  Representations of Algebraic Groups,
  Pure and Applied Mathematics v. 131, Boston: Academic Press 1987.

\bibitem{key12} A.\@ Joseph, On the Demazure character formula,
  Ann. Sci. \'Ecole Norm. Super. 18 (1985), 389-419.

\bibitem{key13} G.\@ Kempf, Linear systems on homogeneous spaces,
  Ann. of Math. 103 (1976), 557-591.

\bibitem{key14} G.\@ Kempf, The Grothendieck-Cousin complex of an
  induced representation, Adv. in Math. 29 (1978), 310-396.

\bibitem{key15} S. Kumar, A refinement of the PRV conjecture,
  Invent. Math. 97 (1989), 305-311.

\bibitem{key16} V.\@ Lakshmibai and C.S. Seshadri, Geometry of
  $G/P$-5, J. Algebra 100 (1986), 462-557.

\bibitem{key17} P.\@ Littelmann, Good filtrations and decomposition
  rules for representations with standard monomial theory, J. reine
  angew. Math. 433 (1992), 161-180.

\bibitem{key18} O.\@ Mathieu, Formules de Caract\`eres pour les
  Alg\`ebres de Kac-Moody G\'en\'erales, Ast\'erisque 159-160 (1988).

\bibitem{key19} O.\@ Mathieu, Filtrations of $B$-modules, Duke
  Math. Journal 59 (1989), 421-442.

\bibitem{key20} O.\@ Mathieu, Filtrations of $G$-modules,
  Ann. Sci. Ecole Norm. Sup. 23 (1990), 625-644.

\bibitem{key21} O.\@ Mathieu, Good bases for $G$-modules,
  Geom. Dedicate 36 (1990) 51-66.

\bibitem{key22} O.\@ Mathieu, Bases des repr\'esentations des groupes
  simples complexes [d'apres Kashiwara, Lusztig, Ringel et al.],
  S\'eminaire Bourbaki, Ast\'erisque 201-202-203 (1991), 421-442.

\bibitem{key23} S.\@ MacLane, Homology, Berlin: Springer 1963.

\bibitem{key24} V.B.\@ Mehta\pageoriginale 
 and A.\@ Ramanathan, Frobenius splitting
  and cohomology vanishing for Schubert varieties, Annals of Math. 122
  (1985), 27-40.

\bibitem{key25} V.B.\@ Mehta and V.\@ Srinivas, Normality of Schubert
  varieties, American Journal of Math. 109 (1987), 987-989.

\bibitem{key26} J.\@ Oesterl\'e, D\'eg\'enerescence de la suite
  spectrale de Hodge vers De Rham, Expos\'e 673, S\'eminaire Bourbaki,
  Ast\'erisque 152-153 (1987), 67-83.

\bibitem{key27} P.\@ Polo, Un crit\`ere d'existence d'une filtration
  de Schubert, C.R. Acad. Sci. Paris, 307, s\'erie I (1988), 791-794.

\bibitem{key28} P.\@ Polo, Modules associ\'es aux vari\'et\'es de
  Schubert, C.R. Acad. Sci. Parix, 308, s\'erie I (1989), 123-126.

\bibitem{key29} P.\@ Polo, Modules associ\'es aux vari\'et\'es de
  Schubert, to appear in the proceedings of the Indo-French Geometry
  Colloquium (Bombay, Feb. 89).

\bibitem{key30} P.\@ Polo, Vari\'et\'es de Schubert et excellentes
  filtrations, Ast\'erisque 173-174 (1989), 281-311.

\bibitem{key31} A.\@ Ramanathan, Schubert varieties are arithmetically
  Cohen Macaulay, Invent. Math. 80 (1985), 283-294.

\bibitem{key32} A.\@ Ramanathan, Equations defining Schubert varieties
  and Frobenius splitting of diagonals, Publ. Math. IHES 65 (1987), 61-90.

\bibitem{key33} C.M.\@ Ringel, The category of modules with good
  filtrations over a quasi-hereditary algebra has almost split
  sequences, Math. Z. 208 (1991), 209-223.

\bibitem{key34} T.A.\@ Springer, Linear Algebraic Groups, Boston Basel
  Stuttgart: Birkh\"auser 1981. 

\bibitem{key35} W.\@ van der Kallen, Longest weight vectors and
  excellent filtrations, Math. Z. 201 (1989), 19-31.

\bibitem{key36} Wang Jian-Pan, Sheaf cohomology on $G/B$ and tensor
  products of Weyl modules, J. Algebra 77 (1982), 162-185.

\end{thebibliography}

\newpage





\thispagestyle{plain}
\section{Glossary of Notations}\pageoriginale

\markboth{Glossary of Notations}{Glossary of Notations}
%\addcontentsline{toc}{chapter}{Glossary of Notations}

\begin{longtable}[l]{@{}>{$}l<{$}p{7.65cm}}
(,) & $W$-invariant inner product, \pageref{page4}\\
w'\leq w & $w'$ precedes $w$ in the Bruhat order, {\em i.e.}
  $X_{w'}\subset X_{w}$, \pageref{page17}\\
s\mu>\mu & $(\rho-s\rho,s\mu-\mu)>0$, \pageref{page71}\\
\mu_{1} & anti-dominant weight in $W$-orbit of $\mu$, \pageref{page18}\\
\nu_{0} & dominant weight in $W$-orbit of $\nu$, \pageref{page17}\\
\ind^{G}_{P}\circ \ind^{P}_{B} & composite functor, \pageref{page11}\\
a\ast\sigma & $b\mapsto \sigma(a\cdot b)$ when $a$ is ring element,
  \pageref{page39}\\
g\ast \sigma & $b\mapsto g\cdot \sigma(g^{-1}\cdot a)$ when $g$ is
  group element, \pageref{page39}\\
m^{*}\L & pullback of $\L$, see \cite{key7}, \pageref{page14}\\
m_{\ast}\O & direct image of $\O$, \pageref{page14}\\
M^{*} & dual of $M$, \pageref{page25}\\
(F_{\ast}\O_{X})^{*} & $\cHom(F_{\ast}\O_{X},\O_{X})$, \pageref{page42}\\
G\times^{B}X & total space of associated fibre bundle, \pageref{page7}\\
\L|_{X} & restriction of bundle to subspace $X$, \pageref{page11}\\
A_{<\lambda} & $\oplus_{i}A^{i}_{<i\lambda}$ when $A$ is graded, \pageref{page41}\\
A_{\leq \lambda} & also in graded case: $\oplus_{i}A^{i}_{\leq
  i\lambda}$, \pageref{page35}\\
M_{<\lambda} & largest $B$-submodule of $M$ that is in
$\C_{<\lambda}$, \pageref{page23}\\
M_{\leq R} & largest $B$-submodule with weights of length $\leq R$,
\pageref{page10}\\
M_{\leq \lambda} & largest $B$-submodule of $M$ that is in $\C_{\leq
  \lambda}$, \pageref{page23}\\
M_{\mu} & weight space of weight $\mu$, \pageref{page2}, \pageref{page10}\\
G_{\mathbb{Z}} & $\mathbb{Z}$-form of $G$, \pageref{page68}\\
B & Borel subgroup, \pageref{page2}\\
BwB & double coset, \pageref{page4}\\
\C_{B} & the category of rational $B$-modules, \pageref{page10}\\
\C_{G} & the category of rational $G$-modules, \pageref{page9}\\
\C_{\leq \lambda} & category of $B$-modules whose weights precede
$\lambda$, \pageref{page23}\\
\C_{<\lambda} & subcategory with weights strictly preceding $\lambda$,
\pageref{page22}\\
\C_{\leq R} & subcategory of $\C_{B}$ with length of weights $\leq R$,
\pageref{page10}\\
\C_{<R} & subcategory of $\C_{B}$ with length of weights
$<R$, \pageref{page23}\\ 
\ch(M_{k}) & formal character of $M$, \pageref{page72}\\
\tilde{D}_{j} & irreducible divisor in $Z_{n}$, \pageref{page44}\\
D_{n} & divisor with normal crossing in $Z_{n}$, \pageref{page44}\\
\End_{F}(R) & space of Frobenius-linear endomorphisms of
$R$, \pageref{page39}\\ 
\End_{F}(R,I) & subspace of those compatible with $I$, \pageref{page39}\\
\End_{F}(X) & global sections of $\cEnd_{F}(X)$, \pageref{page42}\\
\End_{F}(X,Y) & global sectons of $\cEnd_{F}(X,Y)$, \pageref{page42}\\
\Ext^{i} & $i$-th $\Ext$ functor \cite[Ch.\@ III]{key23}, \pageref{page24}\\
\Ext^{i}_{B}(M,N) & $\Ext$ group in the category $\C_{B}$, \pageref{page12}\\
\Ext^{1}_{B,\lambda} & $\Ext$ in $\C_{\leq l(\lambda)}$, \pageref{page25}\\
\cEnd_{F}(X) & sheaf of Frobenius-linear endomorphisms, \pageref{page42}\\
\cEnd_{F}(X,Y) & subsheaf of those compatible with $Y$, \pageref{page42}\\
F & absolute Frobenius morphism, \pageref{page41}\\
F_{*}\O_{X} & the direct image of $\O_{X}$ under $F$, \pageref{page42}\\
\mathbb{G}_{a} & additive group, \pageref{page1}\\
\mathbb{G}_{m} & multiplicative group, \pageref{page1}\\
GL(n,k) & general linear group, \pageref{page1}\\
G & algebraic group, \pageref{page1}\\
 &\quad reductive connected, \pageref{page2}\\
 &\quad simply connected too, \pageref{page15}\\
H^{0}(B,M) & submodule of $M$ consisting of vectors fixed by
$B$, \pageref{page10}\\ 
H^{0}(X,\L) & global sections over $X$ of $\L$ or
$\L|_{X}$, \pageref{page11}, \pageref{page13}\\ 
H^{i}(B,M) & $i$-th cohomology of $M$ in $\C_{B}$
(\cite{key11}), \pageref{page25}\\ 
H_{w} & Joseph's functor $M\mapsto H^{0}(X_{w},\L(M))$, \pageref{page13}\\
H_{w}(\lambda) & $H_{w}(k_{\lambda})$, \pageref{page15}\\
\I_{S} & ideal sheaf of $S$, \pageref{page16}\\
\ind^{G}_{B} & induction functor $\C_{B}\to \C_{G}$, \pageref{page11}\\
k & algebraically closed field, \pageref{page1}\\
 &\quad of characteristic $p>0$, \pageref{page33}\\
k[B] & the ring of regular functions on $B$, \pageref{page23}\\
k_{\lambda} & one-dimensional $B$-module of weight $\lambda$, \pageref{page15}\\ 
K(w,y,\lambda) & $\ker P(w\lambda)\to P(y\lambda)$, \pageref{page65}\\
K(\sum_{1},\sum_{2},\lambda,\mu,\nu) &
$\ker:H^{0}(\sum_{1},\L(\lambda,\mu,\nu))\to
H^{0}(\sum_{2},\L(\lambda,\mu,\nu))$, \pageref{page49}\\
l(w) & length of $w$, \pageref{page13}\\
\L(M) & vector bundle $G\times^{B}M$ over $G/B$ with fibre $M$, \pageref{page8}\\
\L(\lambda) & $\L(k_{\lambda})$, \pageref{page15}\\
\L(\lambda,\mu,\nu) & line bundle
 $\L(\lambda)\times \L(\mu)\times \L(\nu)$, \pageref{page49}\\
n & has no fixed value, \pageref{page44}\\
\O(n) & power of twisting sheaf \cite{key7}, \pageref{page27}\\
\O_{X} & structure sheaf of $X$, \pageref{page14}\\
p & the characteristic, \pageref{page33}\\
\mathbb{P}^{n} & projective $n$-space \cite{key7}, \pageref{page26}\\
P(\mu) & dual Joseph module with socle $k_{\mu}$, \pageref{page18}\\
P_{i} & minimal parabolic $B_{s_{i}}B\cup B$, \pageref{page4}\\
P_{s} & minimal parabolic $B_{s}B\cup B$, \pageref{page5}\\
P_{\mu} & parabolic with $\L(\mu)$ very ample on $G/P_{\mu}$, \pageref{page49}\\
Q(\mu) & minimal relative Schubert module with socle
$k_{\mu}$, \pageref{page19}\\ 
Q(S,S',\lambda) & $\ker(\res:H^{0}(S,\L(\lambda))\to
H^{0}(S',\L(\lambda)))$, \pageref{page19}\\
R^{n}F & $n$-th derived functor of $F$, \pageref{page26}\\
R_{u}(G) & unipotent radical of $G$, \pageref{page2}\\
\res^{G}_{H} & restriction functor $\C_{G}\to \C_{H}$, \pageref{page10}\\
SL(n,k) & special linear group, \pageref{page1}\\
\soc M & socle of $M$, usually as $B$-module, \pageref{page10}\\
s_{i} & $i$-th simple reflection in a sequence, \pageref{page3}\\
T & maximal torus, contained in $B$, \pageref{page4}\\
U & unipotent radical of $B$, \pageref{page5}\\
U_{\alpha} & root subgroup $\{x_{\alpha}(t)\mid t\in k\}$, \pageref{page40}\\
W & Weyl group, \pageref{page3}\\
w_{0} & longest element, \pageref{page7}\\
X(G) & character group of $G$, \pageref{page2}\\
X(T)^{+} & the set of dominant weights in $X(T)$, \pageref{page17}\\
X(T)^{-} & the set of anti-dominant weights in $X(T)$, \pageref{page9}\\
X_{w} & the Schubert variety $\overline{BwB}/B$, \pageref{page6}\\
\p X_{w} & complement of the open Bruhat cell in $X_{w}$, \pageref{page19}\\
x_{\beta} & isomorphism $\mathbb{G}_{a}\to U_{\beta}$, \pageref{page4}\\
Z_{j} & Demazure resolution, \pageref{page6}\\
\rho & half sum of the roots of $B$, \pageref{page27}\\
\sum_{1} & union of double Schubert varieties, \pageref{page49}\\
\omega_{X} & $\Omega^{n}_{X}$ where $X$ is smooth of dimension $n$,
\pageref{page44}\\
\Omega^{q}_{X} & sheaf of $q$-forms on $X$, \pageref{page44}\\
\end{longtable}


