\usepackage{graphicx,xspace,fancybox}
\newcounter{pageoriginal}
\marginparwidth=10pt
\marginparsep=10pt
\marginparpush=5pt
%\renewcommand{\thepageoriginal}{\arabic{pageoriginal}}
\newcommand{\pageoriginale}{\refstepcounter{pageoriginal}\marginpar{\footnotesize\xspace\textbf{\thepageoriginal}}
} 
\let\pageoriginaled\pageoriginale




\newtheorem{theorem}{Theorem}[section]
\newtheorem{lemma}[theorem]{Lemma}
\newtheorem{snakelemma}[theorem]{Snake Lemma}
\newtheorem{sublemma}[theorem]{Sublemma}
\newtheorem{proposition}[theorem]{Proposition}
\newtheorem{corollary}[theorem]{Corollary}

\newtheoremstyle{starprop}{10pt}{10pt}{\it }%
{}{\bfseries}{$^*$.}{ }{}
\theoremstyle{starprop}
\newtheorem{starprop}{Proposition}[section]

\newtheoremstyle{remark}{10pt}{10pt}{ }%
{}{\bfseries}{.}{ }{}
\theoremstyle{remark}
\newtheorem{definition}[theorem]{Definition}
\newtheorem{remark}[theorem]{Remark}
\newtheorem{notation}[theorem]{Notation}
\newtheorem{exercise}[theorem]{Exercise}
\newtheorem{example}[theorem]{Example}


\newtheoremstyle{nonum}{}{}{\itshape}{}{\bfseries}{\kern -2pt{\bf.}}{ }{#1 \mdseries
{\bf #3}}
\theoremstyle{nonum}

\newtheorem{lemma*}{Lemma}	
\newtheorem{theorem*}{Theorem}	
\newtheorem{irreducibilitythm*}{IRREDUCIBILITY THEOREM}
\newtheorem{genirreducibilitythm*}{GENERIC IRREDUCIBILITY THEOREM}
\newtheorem{embedthm*}{EMBEDDING THEOREM}
\newtheorem{prop*}{Proposition}	
\newtheorem{claim*}{Claim}
\newtheorem{defi*}{Definition}
\newtheorem{conjecture*}{CONJECTURE}
\newtheorem{coro*}{Corollary}
\newtheorem{fact}{Fact}
\newtheorem{facts}{Facts}

\newtheoremstyle{mynonum}{}{}{ }{}{\bfseries}{\kern -2pt{\bf.}}{ }{#1 \mdseries
{\bf #3}}
\theoremstyle{mynonum}
\newtheorem{remark*}{Remark}	
\newtheorem{remarks*}{Remarks}	
\newtheorem{exer*}{Exercise}	
\newtheorem{example*}{Example}	
\newtheorem{examples*}{Example}	
\newtheorem{note*}{Note}




\def\p{\partial}
\def\la{\langle}
\def\ra{\rangle}
\def\ophi{\overset{o}{\phi}}

\def\oval#1{\text{\cornersize{2}\ovalbox{$#1$}}}


\newcommand*\mycirc[1]{%
  \tikz[baseline=(C.base)]\node[draw,circle,inner sep=.7pt](C) {#1};\:
}


\def\uub#1{\underline{\underline{#1}}}
\def\ub#1{\underline{#1}}
\def\oob#1{\overline{\overline{#1}}}
\def\ob#1{\overline{#1}}


\font\bigsymb=cmsy10 at 4pt
\def\bigdot{{\kern1.2pt\raise 1.5pt\hbox{\bigsymb\char15}}}
\def\overdot#1{\overset{\bigdot}{#1}}

\makeatletter
\renewcommand\subsection{\@startsection{subsection}{2}{\z@}%
                                     {-3.25ex\@plus -1ex \@minus -.2ex}%
                                     {-1.5ex \@plus .2ex}%
                                     {\normalfont}}%

\renewcommand\thesection{\thechapter.\@arabic\c@section}
\renewcommand\thesubsection{({\thechapter.\thesection.\@arabic\c@subsection})}

\renewcommand{\@seccntformat}[1]{{\csname the#1\endcsname}\hspace{0.3em}}
\makeatother

\def\fibreproduct#1#2#3{#1{\displaystyle\mathop{\times}_{#3}}#2}
\let\fprod\fibreproduct

\def\fibreoproduct#1#2#3{#1{\displaystyle\mathop{\otimes}_{#3}}#2}
\let\foprod\fibreoproduct


\makeatletter

\def\cleardoublepage{\clearpage\if@twoside \ifodd\c@page\else
    \thispagestyle{empty}\hbox{}\newpage\if@twocolumn\hbox{}\newpage\fi\fi\fi}

\renewcommand\tableofcontents{%
    \if@twocolumn
      \@restonecoltrue\onecolumn
    \else
      \@restonecolfalse
    \fi
    \chapter*{\contentsname
        \@mkboth{%
           \contentsname}{\contentsname}}%
    \@starttoc{toc}%
    \if@restonecol\twocolumn\fi
    }
\makeatother

\marginparsep=10pt
\marginparwidth=18pt


\renewcommand\chaptermark[1]{\markboth{\thechapter. #1}{}}
\renewcommand\sectionmark[1]{\markright{\thesection. #1}}


\newtheoremstyle{alph}{}{}{\itshape}{}{\bfseries}{\kern -2pt{\bf.}}{
      }{#1 \mdseries{\bf #2}
{\bf #3}}
\theoremstyle{alph}
\newtheorem{alphlemma}{Lemma}
\renewcommand{\thealphlemma}{\Alph{alphlemma}}


\def\q{\quad}
\def\ul#1{\underline{#1}}

\DeclareMathOperator{\Hom}{Hom}
\DeclareMathOperator{\soc}{soc}
\DeclareMathOperator{\res}{res}
\DeclareMathOperator{\ind}{ind}
\DeclareMathOperator{\Ext}{Ext}
\DeclareMathOperator{\End}{End}
\DeclareMathOperator{\cEnd}{\mathcal{E}nd}
\DeclareMathOperator{\cHom}{\mathcal{H}om}
\DeclareMathOperator{\Spec}{Spec}
\DeclareMathOperator{\Ind}{Ind}
\DeclareMathOperator{\Ker}{Ker}
\DeclareMathOperator{\Iim}{im}
\DeclareMathOperator{\Tor}{Tor}
\DeclareMathOperator{\eval}{eval}
\DeclareMathOperator{\ch}{ch}
\DeclareMathOperator{\supp}{supp}
\DeclareMathOperator{\dlog}{dlog}

\def\bfC{\mathbf{C}}
\def\bfP{\mathbf{P}}
\def\bfQ{\mathbf{Q}}
\def\bfR{\mathbf{R}}
\def\bfZ{\mathbf{Z}}



\def\A{\mathcal{A}}
\def\B{\mathcal{B}}
\def\C{\mathcal{C}}
\def\E{\mathcal{E}}
\def\F{\mathcal{F}}
\def\G{\mathcal{G}}
\def\H{\mathcal{H}}
\def\L{\mathcal{L}}
\def\M{\mathcal{M}}
\def\O{\mathcal{O}}
\def\R{\mathcal{R}}
\def\V{\mathcal{V}}
\def\X{\mathcal{X}}
\def\Y{\mathcal{Y}}
\def\I{\mathcal{I}}


\makeatletter

\def\printindex#1#2{\@restonecoltrue\if@twocolumn\@restonecolfalse\fi
  \columnseprule \z@ \columnsep 35pt
  \newpage \thispagestyle{plain}\twocolumn[{\large\bf #2 \vskip3ex}]
  \markboth{\it\bfseries Index}%
                        {\it\bfseries\indexname}%
  \addcontentsline{toc}{chapter}{#2}
  \@input{#1.ind}}

\makeatother
