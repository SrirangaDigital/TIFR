\chapter{Polo's Theorem}\label{chap3}

In\pageoriginale characteristic\label{page22} zero, the
representations of reductive 
algebraic groups are completely reducible. This means that the
irreducible representations are injective, as any extension of an
irreducible by an irreducible is split exact. The dual Joseph modules
introduced in the last chapter are not injective in the category of
$B$-modules. Due to this non-injectivity the excellent filtrations are
non-trivial filtrations of $B$-modules. However, in this chapter, we
prove certain injectivity theorems for $P(\lambda)$.

In the first section, we prove Polo's theorem which says that the dual
Joseph module $P(\lambda)$ is injective in a smaller category
$\C_{\leq l(\lambda)}$.

In the second section, using a strong version of Polo's theorem, we
give a cohomological criterion for a $B$-module to have an excellent
filtration. 

\section{Polo's Theorem}\label{chap3-sec3.1}

We choose as in Bourbaki a linear functional {\em height} on
$X(T)\otimes \mathbb{R}$ which is positive on all roots of $B$ and
injective on the lattice $X(T)$. We say that $\lambda$ precedes $\mu$
in {\em length-height order}\index{B}{length-height order} 
if either $l(\lambda)<l(\mu)$ or
$[l(\lambda)=l(\mu)$ and the height functional takes a higher value on
  $\mu$ then on $\lambda$]. This defines a total order on
$X(T)$-somewhat arbitrarily because of the freedom in the choice of
the height functional-which captures the ``highest weight category
structure'' corresponding with the dual Joseph modules. Rather than
explaining what this means we ask the reader to look how the
length-height order functions in proofs. For $\lambda\in X(T)$ we
define $\C_{<\lambda}$ to consist of the\pageoriginale $B$-modules\label{page23} all
of whose weights strictly precede $\lambda$ in length-height order. If
$M$ is a $B$-module then $M_{<\lambda}$ is the largest $B$-submodule
of $M$ that is in $\C_{<\lambda}$. Similarly one defines $\C_{\leq
  \lambda}$ and $M_{\leq \lambda}$. (For graded $B$-modules we will
give a slightly different meaning to these notations.)

If $R\geq 0$ then $\C_{\leq R}(\C_{<R})$ denotes the full subcategory
of $\C_{B}$ whose objects are the modules all of whose weights have
length not more than $R$ (strictly less than $R$).

In this section we prove that the dual Joseph module $P(\lambda)$ is
injective in $\C_{\leq l(\lambda)}$. 

If $R\geq 0$ then $\C_{\leq R}(\C_{<R})$ denotes the full subcategory
of $\C_{B}$ whose objects are the modules all of whose weights have
length not more than $R$ (strictly less than $R$).

In this section we prove that the dual Joseph module $P(\lambda)$ is
injective in $\C_{\leq l(\lambda)}$.

\begin{lemma}\label{chap3-lem3.1.1}
The category $\C_{B}$ of $B$-modules has enough injectives.
\end{lemma}

\begin{proof}
Recall that for any subgroup $H$ of a group $G$, the restriction
functor $\res^{G}_{H}$ is exact. Further by the Frobenius reciprocity
the induction functor $\ind^{G}_{H}$ is its right adjoint functor (see
Proposition \ref{chap2-prop2.1.6}). The induction functor thus sends
injective\index{A}{injective (module) [11]} 
$H$-modules to injective $G$-modules. This makes $k[B]$, the
ring of regular functions\index{A}{regular function [9]} 
on $B$, an injective $B$-module as
$k[B]=\ind^{B}_{\{e\}}(k)$, where the $\{e\}$ denotes the identity
subgroup of $B$. Similarly, if $M$ is a $B$-module, then
$\ind^{B}_{\{e\}}\res^{B}_{\{e\}}M$ is injective, and it contains $M$
as a submodule (exercise). Therefore $\C_{B}$ has enough injectives. 
\end{proof}

\begin{remark}\label{chap3-rem3.1.2}
A useful property of injectives in $\C_{B}$ is that if one tensors
them with any $B$-module, the result is again injective (\cite[I
  3.10]{key11}). 
\end{remark}

\begin{corollary}\label{chap3-coro3.1.3}
The subcategories $\C_{\leq R}$, $\C_{< R}$, $\C_{\leq \lambda}$,
$\C_{<\lambda}$ have enough injectives.
\end{corollary}

\begin{proof}
We prove the corollary for $\C_{\leq R}$. The proof is similar for the
other cases.

We denote by $M_{\leq R}$ the largest $B$-submodule of a $B$-module
$M$ whose weights have length less than or equal to $R$. Then
$M\mapsto M_{\leq R}$ is the right adjoint of the embedding functor
$\C_{\leq R}\to \C_{B}$, which is exact. So if $M$ is an injective
$B$-module, then $M_{\leq R}$ is injective in the category $\C_{\leq R}$.
\end{proof}

\begin{remark}\label{chap3-rem3.1.4}
Beware taht $M_{\leq R}$ is usually much smaller than the largest
$T$-submodule of $M$ whose weights have length less than or equal to
$R$. The latter would be simply the sum of those weight spaces whose
weight has length less than or equal to $R$.
\end{remark}

\begin{remark}\label{chap3-rem3.1.5}
The\pageoriginale description\label{page24} of the extremal weights of
$H_{w}(\lambda)$ says that $H_{w}(\lambda)\in \C_{\leq R}$ for
$k_{\lambda}\in \C_{\leq R}$. Therefore for $\mu$ with $l(\mu)\leq
l(\lambda)\leq R$, the module $P(\mu)$ (and hence $Q(\mu)$) is an
object of $\C_{\leq R}$.
\end{remark}

For a module $M$ to be injective in a category $\C$ we need to have
vanishing of the Ext functors\index{A}{Ext functor [23; Ch III]} 
for $M$ (\cite[Ch III]{key23}). Before
trying to prove such vanishing for a dual Joseph module we first make
some remarks.

\begin{remark}\label{chap3-rem3.1.6}
Note that given a $B$-module $N$ one may write it as a filtered union
$N=\lim_{j}N_{j}$ of finite dimensional submodules $N_{j}$. This
construction also has the property that the standard injective
resolutions \cite[Hochschild complex]{key11} of the $N_{j}$ converge
to an injective resolution of $N$. Thus to prove $\Ext^{i}(M_{0},?)=0$
for a fixed finite dimensional $M_{0}\in\C_{B}$ and fixed $i$, we need
only prove $\Ext^{i}(M_{0},N)=0$ for a {\em finite dimensional} $N$.
\end{remark}

\begin{remark}\label{chap3-rem3.1.7}
Further given a finite dimensional $B$-module $N$, using Borel's Fixed
Point Theorem we get a one-dimensional $B$-module $k_{\nu}$ with
weight $\nu$ such that $0\to k_{\nu}\to N\to Q\to 0$. 
\end{remark}

Writing its associated long exact sequence of $\Ext^{i}$ groups, we
see that $\Ext^{i}(M_{0},N)=0$ whenever
$\Ext^{i}(M_{0},Q)=\Ext^{i}(M_{0},k_{\nu})=0$. 

Therefore a $B$-module $M_{0}$ with $\Ext^{i}(M_{0},k_{\nu})=0$ for
all $\nu$, is injective.

\begin{remark}\label{chap3-rem3.1.8}
Let $\C$ be a category with sufficiently many injectives and let $\C'$
be a full subcategory of $\C$ with the following property: whenever
$M_{1}$, $M_{2}\in \C'$, then for every exact sequence $0\to M_{1}\to
M\to M_{2}\to 0$ in $\C$, $M$ also lies in $\C'$. Then for $M$ and $N$
in $\C'$, we have $\Ext^{1}_{\C}(M,N)=\Ext^{1}_{\C'}(M,N)$ (cf.\@
\cite[Ch III \S 1, \S 8]{key23}). This observation is useful in the
case of $C'=\C_{\leq R}$ and $C=\C_{B}$. Since $T$-modules are
semi-simple, every exact sequence of $B$-modules $0\to M_{1}\to M\to
M_{2}\to 0$ splits as $T$-modules and thus if $M_{1}$, $M_{2}\in
\C_{\leq R}$ then $M\in \C_{\leq R}$ indeed.
\end{remark}

\begin{definition}\label{chap3-defi3.1.9}
The {\em injective hull}\index{B}{injective hull} 
of a $B$-module $M$, is an injective
$B$-module containing $M$ whose socle is $\soc(M)$. It is unique up to
non-canonical isomorphism.
\end{definition}

\begin{theorem}[Polo's theorem]\index{B}{Polo's theorem}\label{chap3-thm3.1.10}
Let $\lambda\in X(T)^{-}$ with length $l(\lambda)$. Then
$$
H_{w}(\lambda)=H^{0}(X_{w},\L(\lambda))
$$
is the injective hull of $k_{w\lambda}$ in $\C_{\leq l(\lambda)}$. 
\end{theorem}

\begin{proof}
(After H.H. Andersen.)\pageoriginale  
The\label{page25} dual Joseph module $H_{w}(\lambda)$ has one-dimensional socle with
weight $w\lambda$. Thus it is enough to prove that $H_{w}(\lambda)$ is
injective in $\C_{\leq l(\lambda)}$.
\end{proof}

By a familiar application of Zorn's Lemma--cf.\@ proof of Prop. 7.2 in
\cite[Ch.\@ III]{key23}--it suffices to prove that $H_{w}(\lambda)$ is
injective in the full subcategory of $\C_{\leq l(\lambda)}$ consisting
of finite dimensional modules. Also note that for finite dimensional
$M$ (see \cite[I Ch.\@ 4]{key11})
\begin{align*}
\Ext^{i}_{B}(M,H_{w}(\lambda)) &= H^{i}(B,H_{w}(\lambda)\otimes
M^{*})\\
&= \Ext^{i}_{B}(H_{w}(\lambda)^{*},M^{*}).
\end{align*}

Therefore it is enough to prove that
$\Ext^{1}_{B,\lambda}(H_{w}(\lambda)^{*},M)=0$, where
$\Ext^{1}_{B,\lambda}$ denotes the first derived 
functor\index{A}{derived functor [7]} of the
functor $\Hom$ in the category $\C_{\leq l(\lambda)}$. We will prove
using induction on {\em length of $w$} that
$$
\Ext^{1}_{B,\lambda}(H_{w}(\lambda)^{*},k_{\nu})=0.
$$ 

When $w=e$, we have $H_{w}(\lambda)=k_{\lambda}$. Also
$\Hom_{B}(k_{-\lambda},M)=M^{U}_{-\lambda}$, the $U$-invariants in
$M_{-\lambda}$. But in $\C_{\leq l(\lambda)}$ we have
$M^{U}_{-\lambda}=M_{-\lambda}$ because $\lambda\in X(T)^{-}$
(exercise, cf.\@ proof of \ref{chap2-prop2.2.15}). Thus the $\Hom$
functor is identified with the functor $M\mapsto M_{-\lambda}$. This
functor is exact. Therefore $\Ext^{1}_{B,\lambda}(k_{-\lambda},M)=0$.

Let $H_{w}(\lambda)$ be the injective hull of $k_{\lambda}$. Let $s\in
W$ be a simple reflection such that $l(sw)=l(w)+1$. To complete the
inductive argument, we need to prove that $H_{sw}(\lambda)$ is
injective in $\C_{\leq l(\lambda)}$.

Recall that by Proposition \ref{chap2-prop2.2.5}
$$
H_{sw}=H_{s}\circ H_{w}=\ind^{P_{s}}_{B}\circ H_{w}.
$$

Further, by using the Frobenius reciprocity repeatedly, we obtain:
\begin{align*}
\Hom_{B}(H_{s}(H_{w}(M))^{*},N) &=
\Hom_{P_{s}}(H_{s}(H_{w}(M))^{*},H_{s}(N))\\
&= \Hom_{P_{s}}(H_{s}(N)^{*},H_{s}(H_{w}(M)))\\
&= \Hom_{B}(H_{w}(M)^{*},H_{s}(N)).
\end{align*}

Thus we get that
\begin{equation*}
\Hom_{B}(H_{sw}(M)^{*},N)=\Hom_{B}(H_{w}(M)^{*},H_{s}(N))\tag{*}
\end{equation*}

This proves that the functor $\Hom_{B}(H_{sw}(\lambda)^{*},?)$ is the
composition of the two functors $H_{s}:\C_{B}\to \C_{B}$ and
$\Hom_{B}(H_{w}(\lambda)^{*},?)$. Now recall the Grothendieck
spectral\pageoriginale sequence\label{page26} (\cite{key11}) 
for two functors\index{A}{acyclic for a functor [11]}
$F:\C\to \C'$ and $F':\C'\to \C''$ with $F$, $F'$ left exact and $F$
mapping injective objects in $\C$ to objects acyclic for $F'$. It says
that
$$
(R^{n}F')(R^{m}F)(M)\Rightarrow R^{n+m}(F'\circ F)(M)\q \forall M\in
\C.
$$

In particular, if $M$ is acyclic for $F$, {\em i.e.} if $(R^{m}F)M=0$
for $m>0$, then the spectral sequence degenerates to
$$
(R^{n}F')F(M)=R^{n}(F'\circ F)(M).
$$

The latter is all we will use about the Grothendieck spectral 
sequence\index{B}{Grothendieck spectral sequence}
and it can of course be proved directly--without spectral
sequen\-ces--by induction on $n$, using the long exact sequences
associated with the exact sequence
$$
0\to M\to Q_{M}\to Q_{M}/M\to 0,
$$
where $Q_{M}$ is the injective hull of $M$.

We want to use all this for $F=H_{s}:\C_{B}\to \C_{B}$ and
$$
F'=\Hom_{B}(H_{w}(\lambda)^{*},?).
$$ 
We have to check the
conditions. To this end we need

\begin{lemma}\label{chap3-lem3.1.11}
Let $M$ be a $B$-module which is a quotient of a $P_{s}$-module. Then
$M$ is $\ind^{P_{s}}_{B}$-acyclic. In particular, $H_{w}(\lambda)$ is $\ind^{P_{s}}_{B}$-acyclic.
\end{lemma}

\begin{proof}
Note that the restriction map $H^{0}(G/B,\L(\lambda))\to
H_{w}(\lambda)$ is surjective by Ramanathan (cf.\@ Proposition \ref{prop-A.2.6}),
so that $H_{w}(\lambda)$ is indeed a quotient of a $P_{s}$-module. Now
$P_{s}/B$ is a projective line $\mathbb{P}^{1}$, so there is no higher
cohomology than in degree $1$, and if $M$ is a quotient of the
$P_{s}$-module $N$ then
$R^{1}\ind^{P_{s}}_{B}(M)=H^{1}(P_{s}/B,\L(M))$ is a quotient of
$R^{1}\ind^{P_{s}}_{B}(N)$, which vanishes because $\L(N)$ is a
trivial bundle (see also \cite{key7}).
\end{proof}

Now for the spectral sequence to apply we must check the vanishing of
$\Ext^{m}_{B}(H_{w}(\lambda)^{*},H_{s}(N))=H^{m}(B,H_{w}(\lambda)\otimes
H_{s}(N))$ for $m>0$, when $N$ is an injective $B$-module. But then
$H_{s}(N)=\ind^{P_{s}}_{B}(N)$ is an injective $P_{s}$-module, and if
$M$ is a finite dimensional $B$-module,
$$
\Ext^{m}_{B}(H_{s}(M)^{*},H_{s}(N))=\Ext^{m}_{P_{s}}(H_{s}(M)^{*},H_{s}(N))
$$
by \ref{chap2-lem2.1.10}, so this vanishes and
$H^{m}(B,H_{S}(M)\otimes H_{s}(N))$ thus vanishes for any $B$-module
$M$. This means (use Remark \ref{chap3-rem3.1.2}) that\pageoriginale
at\label{page27} least we have a spectral sequence for the functors $F$ and $F''$,
with $F''=H^{0}(B,?\otimes H_{s}(N))$. The composite functor $F''\circ
F$ is just $H^{0}(B,?\otimes H_{s}(N))$, by Frobenius reciprocity and
\ref{chap2-lem2.1.10}. The lemma gives us that
$H^{m}(B,H_{w}(\lambda)\otimes H_{s}(N))=R^{m}(F''\circ
F)(H_{w}(\lambda))=R^{m}(F'')\circ F(H_{w}(\lambda))=0$ for $m>0$, as
required. 

We may thus state that
$$
\Ext^{i}_{B}(H_{w}(\lambda)^{*},R^{j}H_{s}(k_{\nu}))\Rightarrow
\Ext^{i+j}_{B}(H_{sw}(\lambda)^{*},k_{\nu}) 
$$
and finish the proof as follows.

\begin{enumerate}
\item The case when $\nu$ is anti-dominant with respect to $s$, {\em
  i.e.} $H_{s}(\nu)\neq 0$. In this case using Kempf's vanishing
  theorem we see that $k_{\nu}$ is acyclic for $H_{s}$. Therefore our
  spectral sequence degenerates and gives:
$$
\Ext^{i}_{B}(H_{sw}(\lambda)^{*},k_{\nu})=\Ext^{i}_{B}(H_{w}(\lambda)^{*},H_{s}(k_{\nu})) 
$$

Now we use the inductive hypothesis to get the required result.

\item $\nu$ is not anti-dominant with respect to $s$.

We put $\mu=s(\nu-\rho)$, where $\rho$ is the half sum of the roots of
$B$. Then $\mu$ is anti-dominant with reference to $s$ and moreover
$k_{\nu}$ is the socle of $\rho\otimes H_{s}(\mu)$. Also we have:
$R^{j}H_{s}(\rho\otimes H_{s}(\mu))=R^{j}H_{s}(\rho)\otimes
H_{s}(\mu)$ by the tensor identity (\cite{key11}). But
$R^{j}H_{s}(\rho)=0 \  \forall j\geq 0$ (cohomology of line bundle
$\O(-1)$ on $\mathbb{P}^{1}$, cf.\@ \cite[II 5.2]{key11}.


Thus we have $\Ext^{i}(H_{w}(\lambda), R^{j}H_{s}(\rho\otimes
H_{s}(\mu)))=0$ for all $i$ and $j$. Now consider 
$$
0\to k_{\nu}\to \rho\otimes H_{s}(\mu)\to Q\to 0.
$$
Writing down part of the associated long exact sequence of
$B$-cohomology gives $\Hom_{B}(H_{sw}(\lambda)^{*},Q)\to
\Ext^{1}_{B}(H_{sw}(\lambda)^{*},k_{\nu})\to 0$. But one can check
(cf.\@ \cite[II 5.2]{key11}) that all weights of $Q$ are strictly less
in length than $\nu$. As the socle of $H_{sw}(\lambda)$ has a weight
at least as long as $\nu$, one must have
$\Hom_{B}(Q^{*},H_{sw}(\lambda))=0$. This gives the vanishing of the $\Ext$.
\end{enumerate}

\begin{lemma}\label{chap3-lem3.1.12}
Let $M$ be a $B$-module with an excellent filtration. Then $M$ is
$\ind^{P_{s}}_{B}$-acyclic. 
\end{lemma}

\begin{proof}
Use Remark \ref{chap2-rem2.1.5} and Lemma \ref{chap3-lem3.1.11} to
prove this lemma.
\end{proof}

\section{Cohomological Criterion}\label{chap3-sec3.2}\pageoriginale

In\label{page28} this section we give a criterion for a $B$-module to have an
excellent filtration. First we prove a stronger version of Polo's
theorem.

\begin{remark}\label{chap3-rem3.2.1}
Polo's theorem and Remark \ref{chap3-rem3.1.8} preceding it says that:
$\Ext^{1}_{B}(H_{w}(\lambda)^{*},M)=0$, where $w\in W$ and $\lambda\in
X(T)^{-}$ and $M\in \C_{\leq \lambda}$.
\end{remark}

The following theorem proves that this equality is true in case of the
higher derived functors too.

\begin{theorem}[Strong form of Polo's Theorem]\label{chap3-thm3.2.2}
\index{B}{Polo's theorem!strong form}
Let $\lambda\in X(T)^{-}$ and $M\in \C_{\leq l(\lambda)}$. Then, for
$w\in W$, $i>0$,
$$
\Ext^{i}_{B}(H_{w}(\lambda)^{*},M)=0.
$$
\end{theorem}

\begin{proof}
We merely extend H.H.\@ Andersen's proof of Polo's theorem\break (Theorem
\ref{chap3-thm3.1.10}) to prove this extension. We go through the old
proof. This time we want to prove
$\Ext^{i}(H_{w}(\lambda)^{*},k_{\nu})=0$ for $i>0$ and $k_{\nu}\in
\C_{\leq l(\lambda)}$.

When $w=e$, the identity element of the Weyl group, we take the
minimal injective resolution $I^{*}(\lambda)$ of $k_{\lambda}$ in
$\C_{B}$, as in \cite[II 4.8-9]{key11}. We claim that all the weights
other than $\lambda$ occurring in $I^{1}(\lambda)$ are necessarily
{\em longer} than $\lambda$. Indeed $I^{1}(\lambda)=k_{\lambda}\otimes
k[U]$ and $\lambda$ has non-negative inner product with every non-zero
weight of $k[U]$ because $\lambda$ is anti-dominant. For higher values
of $i$ the weights of $I^{i}(\lambda)$ are in the same region (see
\cite[II 4.8-9]{key11}) and are thus also strictly longer than
$\lambda$. Therefore
$\Ext^{i}(k_{-\lambda},k_{\nu})=\Ext^{i}(k_{-\nu},k_{\lambda})=0$,
which proves the case when $w$ has length zero.

The rest of the proof of Theorem \ref{chap3-thm3.1.10} extends without
trouble to give this stronger version. As the end, where the weights
of $Q$ are all strictly shorter than $\nu$, use that we may assume by
induction on the length of weights that all
$\Ext^{i}(H_{sw}(\lambda)^{*},Q)$ vanish.
\end{proof}

\begin{exercise}\label{chap3-exer3.2.3}
Complete the above proof by filling in all the details.
\end{exercise}

Let $M$ be a finite dimensional $B$-module. Then,
$$
\Ext^{i}(M,N)=H^{i}(B,M^{*}\otimes N). 
$$
Thus the injectivity of
$H_{w}(\lambda)$ can be interpreted in terms of
$B$-acyclicity. (Recall that a $B$-module $M$ is 
$B$-acyclic\index{B}{B@$B$-acyclic} if
$H^{i}(B,M)=0$ for $i>0$.)

\begin{corollary}\label{chap3-coro3.2.4}
For\pageoriginale $\lambda$,\label{page29} $\mu\in X(T)$, $P(\lambda)\otimes P(\mu)$
is $B$-acyclic.
\end{corollary}

\begin{proof}
Let $(\mu,\mu)\leq (\lambda,\lambda)$, Then we have:
$$
H^{i}(B,P(\lambda)\otimes
P(\lambda))=\Ext^{i}_{B}(P(\lambda)^{*},P(\mu))
$$
Now the strong Polo's theorem gives the result.
\end{proof}

Recall that a $B$-module $M$ is said to have an {\em excellent
  filtration} if there exists a filtration $0=F_{-1}\subset
F_{0}\subset F_{1}\subset \ldots$ by $B$-modules such that $\cup
F_{i}=M$ and $F_{i}/F_{i-1}\approx \oplus P(\lambda_{i})$ for some
$\lambda_{i}\in X(T)$.

\begin{corollary}\label{chap3-coro3.2.5}
The tensor product of two modules with excellent filtrations is
$B$-acyclic. 
\end{corollary}

\begin{theorem}\label{chap3-thm3.2.6}
For $\lambda$, $\mu\in X(T)$, $P(\lambda)\oplus Q(\mu)$ is $B$-acyclic.
\end{theorem}

\begin{proof}
If $l(\mu)\leq l(\mu)$ then $Q(\mu)\in \C_{\leq l(\lambda)}$ and thus
$H^{i}(B,P(\lambda)\otimes Q(\mu))=0$ for $i>0$.

If $l(\mu)>l(\lambda)$ then we let $w_{\mu}$ denote the minimal
element of the Weyl group which takes $\mu$ to the anti-dominant
chamber. We will prove the theorem by induction on the length of
$w_{\mu}$. 

When $l(w_{\mu})=0$, we have $\mu\in X(T)^{-}$ and therefore
$Q(\mu)=P(\mu)$ and the result follows.

When $l(w_{\mu})>0$, we look at the short exact sequence
$$
0\to Q(\mu)\to P(\mu)\to H^{0}(\p X_{w_{\mu}^{-1}}, \L(w_{\mu}\mu))\to
0.
$$

The quotient has a filtration whose associated graded consists of
direct sums of relative Schubert modules $Q(\tau)$ with
$l(w_{\tau})<l(w_{\mu})$ and thus we can use an induction hypothesis
for the quotient. Now the associated long exact sequence of $\Ext$
gives the result.
\end{proof}

We now prove that a weaker condition than the one suggested by Theorem
\ref{chap3-thm3.2.6} is sufficient for a module to have an excellent
filtration. 

\begin{theorem}[Cohomological criterion for excellent filtration]\label{chap3-thm3.2.7}\index{B}{cohomological criterion!for excellent filtration}
Let $M$ be a $B$-module such that for every $\lambda\in X(T)$,
$H^{1}(B,M\otimes Q(\lambda))=0$. Then, $M$ has excellent filtration.
\end{theorem}

\begin{proof}
First,\pageoriginale we\label{page30} order the characters in the length-height
order. Let $\lambda_{1},\ldots$ be our enumeration of $X(T)$ according
to length-height order. Let $\{F_{i}\}$ be the length-height
filtration\index{B}{length-height filtration} of $M$, {\em i.e.} $F_{i}=M_{\leq \lambda_{i}}$ is the
largest $B$-submodule whose weights are in
$\{\lambda_{1},\lambda_{2},\ldots,\lambda_{i}\}$. 

We will prove that the length-height filtration of $M$ is an excellent
filtration of $M$. In fact we will show that $F_{i}/F_{i-1}\approx
\oplus P(\lambda_{i})$ for $i\geq 0$. If not, take $i$ to be minimal
so that it fails.

Consider the short exact sequence: $\mathcal{E}:0\to F_{i-1}\to M\to
R\to 0$. All the weights occurring in $\soc(R)$ are strictly larger
than $\lambda_{i-1}$ in the length-height order. Now for a character
$\eta$ such that $l(\eta)\leq l(\lambda_{i-1})$, we write the long
exact sequence of $B$-cohomology associated to $\mathcal{E}\otimes
Q(\eta)$. We get because of the Acyclicity Theorem
\ref{chap3-thm3.2.6} that $H^{1}(B,R\otimes Q(\eta))=0$. Therefore we
do not cheat if we replace $M$ by $R$ in the sequel. The effect of
this is that we may further assume that $H^{1}(B,F_{i}\otimes
k_{\eta})=0$. There are two cases. The first case is that the height
of $-\eta$ is at least that of $\lambda_{i}$. Then all weights of
$N:=F_{i}\otimes k_{\eta}$ are of negative or zero height, as the
socle of $N$ is of weight $\lambda_{i}-\eta$. But then $\Ext_{B}(k,N)$
clearly vanishes, cf.\@ \cite[II 4.10]{key11}.

The second case is that $-\eta$ precedes $\lambda_{i}$ in
length-height order, so that $\Hom(Q(\eta)^{*},M/F_{i})=0$. It follows
that $\Ext^{1}(Q(\eta)^{*},F_{i})=0$. Further, looking at
\begin{equation*}
0\to \eta\to Q(\eta)\to \frac{Q(\eta)}{\eta}\to 0,\tag{*}
\end{equation*}
with $\eta$ short for $k_{\eta}$, we get
$\Hom_{B}((Q(\eta)/\eta)^{*},F_{i})\to \Ext^{1}(k_{-\eta},F_{i})\to
0$. However $Q(\eta)/\eta$ has weights which are strictly less in
length than $\lambda_{i}$. Therefore we have
$\Hom_{B}((\frac{Q(\eta)}{\eta})^{*},F_{i})=0$ and the second case
follows too. Thus $F_{i}$ is injective in $\C_{\leq l(\lambda_{i})}$
with a socle purely of weight $\lambda_{i}$. This proves that $F_{i}$
is a direct sum of copies of $P(\lambda_{i})$, with as many copies as
the dimension of the socle of $F_{i}$.
\end{proof}

From the proof we actually get:

\begin{corollary}\label{chap3-coro3.2.8}
For a $B$-module with excellent filtration the length-height
filtration is an excellent filtration. 
\end{corollary}

This\pageoriginale corollary\label{page31} is important for checking that the
length-height order leads to a highest weight category structure in
the sense of Cline-Parshall-Scott. We will not get into that and just
tell everything in terms of the length-height order itself.

\begin{corollary}\label{chap3-coro3.2.9}
An injective $B$-module has an excellent filtraton.
\end{corollary}

\begin{corollary}\label{chap3-coro3.2.10}
The property of excellent filtration is closed under extension.
\end{corollary}

\begin{proof}
Let $M_{1}$ and $M_{2}$ be two $B$-modules (maybe infinite
dimensional) with excellent filtration. Let $M$ be a $B$-module such
that we have an exact sequence $0\to M_{1}\to M\to M_{2}\to 0$. Tensor
this exact sequence by $Q(\mu)$ and write its associated long exact
sequence of $B$-cohomology and use the cohomological criterion.
\end{proof}

\begin{lemma}\label{chap3-lem3.2.11}
Let $M$ be a $B$-module with excellent filtration and let $w$ be an
element of the Weyl group. Then the module $H_{w}(M)$ has excellent
filtration. 
\end{lemma}

\begin{proof}
We fix a set of generators $S=\{s_{1},\ldots,s_{l}\}$ of $W$ such that
each of its elements is a simple reflection. Let $w=s_{1}\ldots s_{n}$
be a reduced expression of $w$. By Proposition \ref{chap2-prop2.2.5},
we have $H_{w}(M)=H_{s_{1}}\circ\cdots\circ H_{s_{n}}(M)$. Therefore
it is enought to prove that $H_{s}(M)$ has excellent filtration for
every simple reflection $s\in S$. We first consider the case when
$M=P(\mu)$ for some character $\mu$. Let $\mu_{1}=w^{-1}_{\mu}\mu$
denote the anti-dominant weight in its Weyl group orbit. Then
$P(\mu)=H_{w_{\mu}}(\mu_{1})$ and $H_{s}(P(\mu))$ is by Proposition
\ref{chap2-prop2.2.5} either isomorphic to
$H_{w_{\mu}}(\mu_{1})=P(\mu)$ or to
$H_{sw_{\mu}}(\mu_{1})=P(s\mu)$. Therefore we have proved the claim
for $M=P(\mu)$. 

Now we will use induction to prove the claim for all of $M$.

Let $0\subset F_{1}\subset F_{2}\subset \ldots$ be an excellent
filtration of $M$. Note that $F_{1}$ is a direct sum of copies of
$P(\mu)$ for some $\mu$. Therefore we know that $H_{s}(F_{1})$ has
excellent filtration but $H_{s}(F_{m+1})$ does not have excellent
filtration. Consider the exact sequence
$$
0\to F_{m}\to F_{m+1}\to M_{1}\to 0.
$$

The\pageoriginale module\label{page32} $M_{1}$ is isomorphic to a direct sum of
copies of $P(\nu)$ for some character $\nu$. This exact sequence gives
rise to the exact sequence
$$
0\to H_{s}(F_{m})\to H_{s}(F_{m+1})\to H_{s}(M_{1})\to 0.
$$

The surjectivity of this exact sequence is due to the
$\ind^{P_{s}}_{B}$-acyclicity of $F_{m}$ (Lemma
\ref{chap3-lem3.1.12}). Now the cohomological criterion for excellent
filtration--or common sense if the modules are finite
dimensional--gives us the result.
\end{proof}

\section{Relative Schubert Modules}\label{chap3-sec3.3}

In this section we state and prove (in the form of exercises)
analogous statements for the relative Schubert modules.

\begin{definition}\label{chap3-defi3.3.1}
Let $\C_{\lambda}$ denote the full subcategory of $\C_{B}$ whose
objects are the modules $M$ such that if $\mu$ is a weight of $M$ then
either $\mu=\lambda$ or $l(\mu)<l(\lambda)$.
\end{definition}

Note that $\C_{<l(\lambda)}\subsetneq \C_{\lambda}\subsetneq \C_{\leq
  l(\lambda)}$ if $\lambda\neq 0$.

\begin{exercise}\label{chap3-exer3.3.2}
Prove that $Q(\lambda)$ is injective in $\C_{\lambda}$.
\end{exercise}

Hint: Use the injectiveness of $P(\lambda)$ in $\C_{\lambda
  l(\lambda)}$ and the proof of the Corollary \ref{chap3-coro3.1.3}.

The proof of the cohomological criterion for excellent filtration
extends easily to give us the following result.

\begin{exercise}[The cohomological criterion for relative Schubert
    filtration]\label{chap3-exer3.3.3} 
Prove that a $B$-module $M$ has relative Schubert
filtration\index{B}{cohomological criterion!for rel. Schubert filtration} if and
only if $H^{1}(B,M\otimes P(\mu))=0$ for all characters $\mu$.
\end{exercise}

Hint: This time order the weights a little differently, using the
negative of the height function. 









