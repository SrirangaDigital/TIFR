\chapter{Lecture 3}\label{chap3} %chapter 3

\section{The Conjugate space (dual) of a normed linear space}\label{chap3:sec1}

Let\pageoriginale $X$ be a normed linear space. Let $X^*$ be the totality of all
linear functionals on $X$. $X^*$ is a linear space with the operations
defined by:
\begin{align*}
 (f + g ) (x) &= f(x) + g(x) f, g \in X^*, x \in X\\
 (\alpha f )(x) & = \alpha. f(x). 
\end{align*}
$X^*$ is a \textit{ Banach space} with the norm 
$$
||f || = \sup_{ ||x||\leq 1} |f(x)| \quad (f \in X^*, x \in X). 
$$

We call the Banach space $X^*$ the \textit{ conjugate space } of $X$. 

\section{The Resonance Theorem}\label{chap3:sec2}

\begin{lemma*}[Gelfand ]
 Let $p(x)$ be a semi -norm on a \textit{ Banach space }$X$. Then
 there exists a number $\wp > o $ such that 
 $$
 p(x) \leq \wp ||x || \textit { for all } x \in X
 $$
 if and only if $p(x)$ is lower semi - continuous. (Lower semi -
 continuity means this); for any $x_\circ \in $ and any
 $\mathcal{E} > 0$, there exists a $\delta = \delta (x, \mathcal{E} )
 > 0$ such that $p(x) \geq p(x_o) - \mathcal{E} $ for $||x - x_o
 || \leq \delta$. 
\end{lemma*}

\begin{proof}
 \begin{enumerate}[i)]
 \item Suppose $p(x) \leq \wp || x || $ for all $x \in X$, $\wp > 0$; then 
  \begin{align*}
   p(x_o) = p(x_o -x+ x) & \leq p(x_o - x) + p(x) \\
   & \leq \wp || x - x_o || + p(x)\\
   & \leq p(x) + \epsilon, \text{ if }|| x - x_0|| \leq \mathcal{E} / \wp =
   \delta.  
  \end{align*}
 \item Conversely assume that $p(x)$ is lower semi - continuous. 
  
  To prove that there is a $\wp > 0$ such that $p(x)\leq \wp || x
  ||$ for every $x \in X$ it is sufficient to show that $p(x)$ is
  bounded, say by $\mathscr{P}_1$, in some\pageoriginale closed sphere $K$ of positive
  radius $(K = \big\{ x|~ || x - x_o|| \leq \delta \big\}$. For if $x
  \in X$ with $||x || \leq \delta$, then $x_o$ and $x_o + x$ both
  belong to $K$ and hence 
  \begin{align*}
   p(x) & = p(-x_o + x_o +x) \leq p(-x_0) + p(x_o +x)\\ 
   & = p(x_o) + p(x_o + x)\\
   & \leq 2 \wp_1 ; 
  \end{align*}
  if $x$ is an arbitrary element of $X$ 
  \begin{align*}
   p (x) & = p\left( \frac{||x||}{\delta} ~\frac{x
    \delta}{||x||}\right) = \frac{||x||}{\delta} p 
   \frac{x\delta}{|| x||}\\ 
   & \leq \frac{2 \mathscr{P}_1}{\delta}||x|| \left(as ||\frac{x \delta}{||
    x ||} ||= \delta \right) \text{ and choose } \wp = 2 \wp_1 /
   \delta. 
  \end{align*}
 \end{enumerate}
 
 Now we assume that $p(x)$ is unbounded in every closed sphere of
 positive radius and derive a contradiction. Let 
 $$
 K_o = \{ x \big| ||x - x_o || \leq \delta, \delta > 0 \}; 
 $$
 there exists in interior point $x_1$ of $K_o$ such that $p(x_1) >
 1$. By the lower semi - continuity of $p$, there exists a closed
 sphere $K_1 = \bigg\{ x ; || x - x_1 || \leq \delta_1 < 1, \delta
_1 > 0\bigg\}, K_1 \subset K_o$ such that $p(x) > 1$ for each $x
 \in K_1$. By a repetition of this argument we may choose a sequence
 of closed spheres $K_n = \big\{ x ; || x - x_n || \leq \delta_n < 1/n,
 \delta_n > 0 \big\}$, $n$ running through all positive integers,
 such that ${K_n \subset K_{n-1}}$ and $p(x) > n$ for each $x \in
 K_n $. For $m, m' > n$, Since $x_m, x'_m \in K_n $, we have
 $|| x_m - x'_m || \leq || x_m - x_n|| + ||x'_m - x_n || \leq 2
 \delta_n < 2/n$; so
 $x_n$ is a Cauchy sequence. Since $X$ is complete there exists an
 $x_\infty \in X$ such that $s - \lim\limits_{ n \to \infty} x_n =
 x_\infty$. As $||x_m-x_n|| \leq\delta_n$ for $m>n$, we have,
 passing to the limits, $|| x_\infty- x_n || \leq \delta_n$. So $x_\infty \in
 \bigcap\limits^\infty_{ n = 1} K_n$; this would mean that
 $p(x_\infty)$ (which is a real number) is greater than every
 positive integer $n$, which is absurd. 
\end{proof}

\noindent \textbf{The Resonance theorem:} 
Let\pageoriginale $X$ be a Banach space and $Y_n (n = 1, 2, \ldots)$ a
sequence of 
normed linear spaces. Let, for each $n$, $T_n$ be a linear operator
from $X$ to $Y_n$. Then the boundedness of the sequences $\big\{|| T_n x
||\big\}$ for every $x \in X$ impels the boundedness of the sequence
$\big\{ || T_n ||\big\}$. 

\begin{proof}
 For each $x \in X$, $\sup\limits_n ||
 T_n (x)||$ is finite as $\left\{ || T_n (x) || \right\}$ is
 bounded. Define $p(x) =\sup\limits_n \| T_n (x)\| ; p(x)$ is a semi-norm on
 $X$. $p(x)$ is also lower semi-continuous since it is the supremum of
 the sequence of continuous functions $\big\{ || T_n
 ||\big\}$. Consequently, by Gelfand's lemma, $p(x) \leq \wp ||x||$
 (for some $\wp > o$) for such $x \in X $; so $ || T_n (x)||\leq \wp
 ||x||$ for each $n$ and each $x \in X$. Thus $ || T_n || \leq \wp $. 
\end{proof}

\begin{coro*}
 Let $X$ be a Banach space $Y$ a normed linear space, and $\big\{
 T_n \big\}$ a sequence of linear operators form $X$ to $Y$. Assume
 that $s - \lim\limits_{n \to \infty }T_n (x) \in Y $ exists for each
 $x \in X $. If we define $T_x = s- \lim\limits_{n \to \infty} T_n
 (x) $ then $T$ is a linear operator from $X $ to $Y$ and $||T|| \leq
 \varliminf\limits_{ n \to \infty} || T_n ||$. 
\end{coro*}

$T$ is evidently additive. By the Resonance theorem, $|| T_n(x)|| \leq
\wp ||x||\break ( \wp > 0); so || T(x) || \leq \wp || x|| $, i.e., $T$ is
continuous. Further, $|| T_n x || \leq || T_n|| || x ||$; so $|| Tx||
\leq \varliminf || T_n || || x| $. Hence $|| T|| \leq
\varliminf\limits_{n \to \infty } || T_n || $. 

\section{Weak convergence}\label{chap3:sec3}
\begin{defi*}
 Let $X$ be a normed linear space; we say that a sequence. $\{x_n \}
 \subset X$ \textit{ converges weakly } to $x_{\infty} \in X$ (and
 write $w \lim\limits_{n \to \infty} x_n = x_\infty$) if, for every
 linear functional $f$ on $X$, we have
 $\lim\limits_{n \to \infty} f(x_n) = f(x_\infty)$. 
\end{defi*}

\begin{prop*}
 \begin{enumerate}[\rm i)]
 \item $w -\lim\limits_{ n \to \infty} x_n $, if it exists, is unique. 
 \item $s -\lim\limits x_n = x_\infty$ implies $w -\lim x_n = x_\infty$. 

  (The converse is not true in general). 
 \item if\pageoriginale $\underset{n \to \infty}{w-\lim\limits}~ x_n = x_\infty$
  then $\varliminf~ || x_n || \geq x_\infty$. 
 \end{enumerate}
\end{prop*}

\begin{proof}
\begin{enumerate}[\rm (i)]
 \item Let $w - \lim\limits x_n = x_\infty$, $w -\lim x_n =
  x_\infty, \neq x'_\infty$. By the Hahn -Banach theorem there exists a
  linear functional $f$ on $X$ such that $f(x_\infty - x'_\infty)
  \neq 0$ i.e., $f(x_\infty) \neq f(x'_\infty)$. But by the
  condition of weak limit we must have $f(x_\infty)= \lim\limits_{
   n \to \infty} f(x_n) = f(x'_\infty)$. 
 \item This follows form the inequality: 
  $$
  | f(x_\infty) - f(x_n) | = f(x_\infty - x_n) \leq || f||~ ||
  x_\infty - x_n||, 
  $$
  for each $f \in X^*$.
  
 \item Let $f_o \in X^*$ with $||f|| = 1$ and $f_o (x_\infty) = ||
  x_\infty||$. 

 Then 
 \begin{align*}
  || x_\infty|| = f_o (x_\infty ) & \leq \varliminf | f_o (x_n) |\\
  & \leq \varliminf || f_o || ~||x_n ||\\
  & = \varliminf \limits_{ n \to \infty}|| x_n||. 
 \end{align*}
\end{enumerate}
\end{proof}

\section{A counter-example}\label{chap3:sec4}

We shall now show by an example that weak convergence does not imply
strong convergence in general. Consider the sequence $\{\sin n \pi t
\}$ in $L_2 (0, 1 )$ (real). This sequence converges weakly to
zero. Since, by the Riesz theorem, any linear functional is given by
the scalar product with a function we have to show that $\int\limits_\circ^1 f(t)
\sin n \pi t dt \to 0$, for each $f \in L_2 (0,1)$. But By Bessel's
inequality, 
$$
\sum_{n=1}^\infty \big| \int_0^1 f(t) \sin n \pi t dt\big|^2 \leq
\int^1_0 |f(t)|^2 dt; 
$$
so $\int\limits_0^1 f(t) \sin n \pi t dt \to 0$ as $n \to \infty$. But $\{
\sin n \pi t \} $ is not strongly convergent, since 
\begin{align*}
 || \sin n \pi t - \sin m \pi t ||^2 & = \int_0^1 | \sin n \pi t - \sin m
 \pi t |^2 dt \\ 
 & = 2 \text { for } n \neq m.  
\end{align*}
