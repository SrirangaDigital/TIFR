\chapter{Lecture 11}\label{chap11} %Lect 11 

\section[Temporally homogeneous Markoff...]{Temporally homogeneous Markoff process on a locally compact
 topological space}\label{chap11:sec1} 

Let\pageoriginale $R$ be a locally compact topological space, countable at
infinity. We consider in $R~ {}'a$ probabilistic movement'. Suppose that
for each triple $(t, x, E)$ consisting of a real number $t > 0$, a
point $x \in R$ and Borel set $E \subset R$ there is given a real
number $P (t, x, E)$ such that the following conditions are
satisfied. 

\begin{enumerate} [i)]
\item $P(t, x, E) \ge 0$, $P(t, x, R) = 1$
\item for fixed $t$ and $x$, $P(t, x, E)$ is a countably additive set
 function on the Borel sets 
\item for fixed $t$ and $E, P(t, x, E)$ is a Borel measurable function
 in $x$ 
\item $P(t + s, x, E) = \int\limits_{R} P(t, x, dy)\, P(s, y, E) ~ t, s
 > 0$. (Chapman - Kolmogoroff relation). 

 The function $P(t, x, E)$ is called the \textit{transition
 probability}; this gives the probability that, in this process, a
 point $x \in R$ is transferred to the Borel set $E$ after $t$ units of
 time. We say then that there is given a \textit{temporally homogeneous
  Markoff process} on $R$ (temporal homogeneity means that the motion
 does not depend on the initial time but only on the time elapsed). 
 
\section{Brownian motion on a homogeneous Riemannian space}\label{chap11:sec2}
 
 Next, we wish to define the `spatial homogeneity' of the process. We
 assume that $R$ is an $n$-dimensional, orientable connected
 $C^\infty$\pageoriginale 
 Riemannian space such that the (full) group of isometries $G$ of $R$,
 which is a Lie group, is transitive on $R$ (i.e., for each pair $x, y
 \in R$ there exists an isometry $S^*$ such that $S^* x = y$. The process
 $P(t, x, E) $ is called \textit{spatially homogeneous} if 
 
\item $P (t, x, E) = P(t, S^*x, S^*E)$ for each $S^* \in G, x \in R,
 E \subset R$. A temporally and spatially homogeneous Markoff process
 on $R$ is called a \textit{Brownian motion} on $R$, if the following
 condition, known as the continuity condition of \textit{Lindeberg},
 is satisfied. 
\item $\lim\limits_{t \downarrow o} t^{-1} \int\limits_{dis (x, y)
 > \varepsilon} P(t, x, dy) = 0$, for every $\epsilon > 0$ and $x
 \in R$. 
\end{enumerate}

\begin{prop*}%Prop 
 Let $C [R]$ denote the Banach space of bounded uniformly continuous
 real valued functions $f(x)$ on $R$, with the norm 
 $$
 || f || = ~ \sup_{x \in R} | f(x) |.
 $$

 Define
 $$
 (T_t f) (x) =
 \begin{cases}
  \int\limits_R P(t, x, dy) f(y), &\text{if}~ t> 0.\\
  f(x), & \text{if}~ t = 0.
 \end{cases}
 $$
 
 Then $T_t$ defines a semi -group of normal type in $C[R]$.
\end{prop*}

\begin{proof}
 We have by condition $(i)$,
 $$
 | T_t f(x) | \le \sup_{y \in R} | f(y) |.
 $$
\end{proof}

If we define a linear operator $S$ by $(Sf)(x) = f(S^*x), S^* \in G$, we have
$T_t S = ST_t$. For, 
\begin{align*}
 (S T_t f) (x) & = (T_t f) (S^*x)\\
 & = \int P(t, S^* x, dy) f(y)\\
 & = \int P(t, S^*x, d (S^* y)) f(S^* y)\\
 & = \int P(t, x, dy) f(S^* y) = (T_t S f) (x).
\end{align*}

If\pageoriginale $S^* \in G$ be such that $S^* x = x^1 $, we have
\begin{align*}
 (T_t f ) (x) - (T_t f) (x') & = (T_t f) (x) - (S T_t f) (x)\\
 & = T_t (f- S f) (x).
\end{align*}

By the uniform of continuity of $f(x)$ and the above equality, we see
that $(T_t f) (x)$ is uniformly continuous and bounded. The semi-group
property follows easily from Fubini's theorem and the
Chapman-Kolmo\-gorff relation $(T_o = I$ by definition). 

To prove the strong continuity, it is enough by and earlier theorem,
to verify weak right continuity at $t = 0$. Since the conjugate space
of $C[R]$ is the space of measures of finite total variation, it is
enough to show that $\lim\limits_{t \downarrow o}(T_t f(x) ~) = f(x)$
boundedly in $x$. 

Now 
\begin{align*}
 & | (T_t f) (x) - f(x) | = \big | \int\limits_{R} P(t, x, dy) [f (y)
  - f(x) ] \big | ~by (i)\\ 
 & = \big | \int\limits^R_{d(x, y) \le \varepsilon} P(t, x, dy) [f(y)
  - f(x) ] \big | + \big | \int\limits_{dis (x, y) > \varepsilon}
 P(t, x, dy) [f (y) - f(x)] \big|\\ 
 & \le \big | \cdots ~ \cdots ~ \cdots ~ \cdots | + 2 || f ||
 \int\limits_{dis (x, y) > \varepsilon} P(t, x, dy)\\ 
 & \le 1. 
\end{align*}

The first term on the right tends to zero as $\varepsilon \rightarrow
0$ and, for fixed $\varepsilon$, the second term tends to zero
boundedly in $x$ as $t \downarrow 0$ (by $(vi)$, and the spatial
homogeneity). Thus $\lim\limits_{t \downarrow o} (T_t f) (x) = f(x) $
boundedly in $x$. 

\begin{theorem*}%Thm
 Let $x_o$ be a fixed point of $R$. Let us assume that the isotropy
 group $G_o = \big \{ S^* | S^* \in G, S^* x_o = x_o \big \}$ is
 compact. $(G_o$, being a closed sub-group of Lie group, is a Lie
 group). Let $A$ be the infinitesimal\pageoriginale generator of $T_t$. Then 
 \begin{enumerate}[\rm (i)]
 \item if $f \in \mathscr{D} (A) \cap C^2$ ($C^2$ denoting the set of
  twice continuously differentiable functions), then, for a
  coordinate system $(x^1 \cdots x^n)$ at $x_o$, 
  $$
  (Af) (x_o) = a^i (x_o)~ \frac{\partial f}{\partial x_o^i} + b^{ij}
  ~(x_o)~ \frac{\partial^2 f}{\partial x^i_o \partial x^j_o} 
  $$
  (adapting the summation convention), where
  \begin{gather*}
   a^i (x_o) = \lim_{t \downarrow o} t^{-1} \int\limits_{dis (x_o,
    x) \le \varepsilon} (x^i - x^i_o ) P(t, x_o, dx)\\ 
   b^{ij} (x_o) = \lim_{t \downarrow o} t^{-1} \int\limits_{dis
    (x_o, x) \le \varepsilon} (x^i - x^i_o) (x^j - x^j_o) P(t,
   x_o, dx) 
  \end{gather*}
  the limits existing independently of sufficiently small
  $\varepsilon > 0$. 
 \item The set $\mathscr{D} (A) \cap C^2$ is 'big' in the sense that,
  for any $C^2$ function with compact support there exists $f(x) \in
  \mathscr{D}(A) \cap C^2$ such that $f(x_o), \dfrac{\partial
   f}{\partial x^i_o}, \dfrac{\partial^2 f}{\partial x^i_o \partial
   x^j_o}$ are arbitrarily near respectively 

  $g(x_o), \dfrac{\partial g}{\partial x^i_o}$, $\dfrac{\partial^2
   g}{\partial x^i_o \partial x^j_o}$. 
\end{enumerate} 
\end{theorem*}

\begin{proof}
 \begin{Step} % step 1
  Let $g(x)$ be a $C^\infty$ function with compact support. 
  
  If $f \in \mathscr{D} (A)$, the convolution
  $$
  (f \otimes g) (x) = \int_G f(S^*_y x) g(S^*_y x) dy,
  $$
  ($S^*_y$ denotes a generic element of $G$ and dy a fixed right
  invariant Haar measure on $G$) is $C^\infty$ and belongs to
  $\mathscr{D} (A)$. (The integral exists since the isotropy group is
  compact and $g$ has compact support). By the uniform continuity of $f$
  and the compactness of the support of $g$ we can approximate the
  integral by Riemann sums $\sum\limits_{i = 1}^{k} f(S^*_{y_i} x) C_i $
  uniformly in $x: (f \otimes g) (x) = s-\lim\limits_{n \to
   \infty}\sum\limits_{i = 1}^{k} f(S^*_{y_i} x)C_i$. 

  Since\pageoriginale $T_t S=S T_t$, $S$ commutes with $A$, i.e., if $f \in \mathscr{D}(A)$,
  then $S f \in \mathscr{D} (A)$ and $A S f = S A f$. Putting $h(x) =
  (Af) (x), (h \in C [R])$. 
  \begin{align*}
   A\left(\sum^m_{i = 1} f (S^*_{y_i} x ) C_i\right) & = \sum^m_{i = 1} (A S_{y_i}
   f ) (x) C_i\\ 
   & = \sum^m_{i = 1} (S_{y_i} A f) (x) C_i\\
   & = \sum^m_{i = 1} h(S^*_{y_i} x) C_i
  \end{align*}
  and the right hand side tends to $(h \otimes g) (x) = (Af \otimes g)
  (x)$. Since $A$ is closed, it follows that $f \otimes g \in
  \mathscr{D}(A)$, and $A(f \otimes g) = Af \otimes g$. Since $R$ is a
  homogeneous space of the Lie group $G$ (by the closed subgroup $G_o)$
  we can find a coordinate neighbourhood $U$ of $x_o$ and for each $x
  \in U$ an element $S^* (x) \in G$ such that i) $S^* x = x_o$ ii) $S^*
  (x) x_o $ depends analytically on the coordinate functions $x^1 \cdots
  x^n$. by the right invariance of the Haar measure, 
  \begin{align*}
   (f \otimes g) ~ (x) & = \int_G f(S^*_y S^* (x) x_o ) g (S^*_y
   S^*(x) x_o ) dy\\ 
   & = \int_G f(S^*_y x_o) g(S^*_y S^* (x) x_o) dy, ~ x \in U.
  \end{align*}
  
  The function on the right side is $C^\infty$ in a neighbourhood of
  $x_o$ and 
  $$
  \frac{\partial^{q_1 + \cdots + q_n}}{\partial(x^1)^{q_1}\cdots
   (\partial x^n)^{q_n}} f \otimes g(x) = \int_G f(S^*_y x_o
  )\frac{\partial^{q_1 + \cdots + q_n} g(S^*_y S^* (x)
   x_o}{\partial(x^1)^{q_1}\cdots (\partial x^n)^{q_n}} dy 
  $$
 \end{Step}
\end{proof}
