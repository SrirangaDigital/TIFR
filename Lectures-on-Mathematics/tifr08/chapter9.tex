\chapter{Lecture 9}\label{chap9} % lecture 9
 
\section{Group of operators}\label{chap9:sec1}
 
We\pageoriginale add certain remarks which will be useful for the application of
semi-group theory to Cauchy's problem. The first of these relates to
conditions under which a semi-group becomes a group; this will be
useful in connection with the wave equation. 
 
\begin{defi*}%Def
 $A$ one parameter family $T_{t ~- \infty < t < \infty}$ of linear
 operators $T_t$ of a Banach space $X$ is called a \textit{group of
  linear operators of normal type} (or simply a group) if the
 following conditions are satisfies: 
 \begin{enumerate}[\rm i)]
 \item $T_t T_s= T_{t+s}, T_o =I$ (group property)
 \item $s-\lim\limits_{n\to t_0} T_t x = T_{t_o}x $ for each $x
  \in X$ and $t_o \in (- \infty, \infty)$ 
 \item there exists a $\beta \ge 0$ such that for all $t$
 \end{enumerate}
 $$
 || T_t || \le e^{\beta|t|}.
 $$
\end{defi*}  
\noindent  (The infinitesimal generator of a group is defined by: $Ax=
 \lim\limits_{t \downarrow o} \dfrac{T_t x-x}{t}$). 

\begin{theorem*}%Thm
 Let $A$ be an additive operator from a Banach space $X$ into $X$
 such that $\mathscr{D}(A)$ is dense in $X$. $A$ necessary and
 sufficient condition that $A$ be the infinitesimal generator of a
 \textit{group} $T_t$ is that there exists a $\beta \ge 0$ such that
 for every $n$ with $|n| > \beta$ the inverse $J_n = (I-N^{-1}
 A)^{-1}$ exists as linear operator with $|| J_n || \le \beta /(1-
 |n|^{-1} \beta)$.  
\end{theorem*} 
 
\begin{proof}%Prf
 {\em Necessity}. Let $\left\{ T_t \right\}$ be a group. Consider the
 two semi-groups $\left\{T_t\right\}_{t \ge o},
 \left\{\hat{T}_t\right\}_{t \ge o}$ where $\hat{T}_t = T_{-t}$. The
 infinitesimal generator of the semi-group $\left\{T_t\right\}_{t \ge
  0}$ coincides with the infinitesimal generator $A$ of the group;
 let $A'$ be the infinitesimal generator of
 $\left\{\hat{T}_t\right\}$ 
\end{proof}  
 
 If\pageoriginale we show that $A' =-A$ the proof of the necessity part will be
 complete. Let $x \in \mathscr{D}(A')$. Then 
 $$
 s-\lim\limits_{n \downarrow 0} \frac{\hat{T}_h -I}{h} x= A'x.
 $$
 
Putting $x_n= h^{-1}(\hat{T}_h -I)x$, we have 
\begin{align*}
 || T_h x_h - A' x|| &\le || T_h x_h -T_h A' x || + || T_h A' x-A'
 x || \\ 
 & \le || T_h || || x_h -A' x || + || T_h A' x-A' x ||.\\
 & \le (\exp \beta h) || x_h -A' x || + || T_h A' x-A' x || \\
 & \to 0 \text{ as } h \downarrow 0.
\end{align*} 
 
Thus for $x \in \mathscr{D}(A')$
\begin{align*}
 -A x= s-\lim_{n \downarrow 0}~ h^{-1}(I- T_h) &= s-\lim_{n
  \downarrow 0} ~T_h x_h \\ 
 &= A'x.
 \end{align*}
 
Hence $x \in \mathscr{D}(A')$ implies $x \in \mathscr{D}(A)$ and
$A'x=-Ax$. Similarly it is proved that if $x \in \mathscr{D}(A)$, then
$x \in \mathscr{D}(A')$ and $A'x=- -Ax$. So $A' =-A$. 

\medskip
\noindent \textbf{sufficiency: }
 We can construct two semi -groups $\left\{T_t\right\}_{t \ge o}$ and
 $\left\{ \hat{T}_t\right\}_{t \ge o}$ as follows: 
 \begin{align*}
  T_t x & =s-\lim_{n\to \infty}~T_t^{(n)}x= s-\lim_{n\to \infty}~ \exp
  ~(t A J_n) x\\
  & =- s-\lim_{n\to \infty} ~\exp (nt [(I-n^{-1}A)^{-1}-I]x)\\ 
  \hat{T}_t x &= s-\lim_{n\to \infty}~ \exp (t-A J_{-n}) x=
  s-\lim_{n\to \infty}~ \exp ~(nt[(I+n^{-1}A)^{-1} -I]x) 
 \end{align*} 
  
 If we show that $\hat{T}_t T_t = T_t \hat{T}_t =I$, then 
 \begin{equation*}
  \hat{\hat{T}}_t = 
  \begin{cases}
   T_t &\text{ for } t \ge 0 \\
   \hat{T}_{-t} &\text{ for } t \le 0
  \end{cases}
  (- \infty < t < \infty )
 \end{equation*} 
 will be a group with $A$ as the infinitesimal generator. 
   
Since\pageoriginale $J_n= (I- n^{-1}A)^{-1}$ commutes with $J_{-n}= (I +n^{-1}A)^{-1}$ we have 
\begin{align*}
 (I-n^{-1}A)^{-1} & + (I +n^{-1}A)^{-1}\\ 
 &= [(I+n^{-1}A)+
  (I-n^{-1}A)](I-n^{-1}A)^{-1}(I+n^{-1}A)^{-1} \\ 
 &= 2 (I- n^{-1}A)^{-1} (I +n^{-1}A)^{-1}\\
 &= 2(I -n^{-2} A^2)^{-1}.
\end{align*} 
    
Since $J_k$ maps $X$ onto the dense subspace $\mathscr{D}(A)$ of $X$,
$J_n J_{-n}= (I-n^{-1} A^2)^{-1}$ maps $X$ onto a dense subspace
$\mathscr{D}(A^2)$. Moreover 
\begin{align*} 
 || (I-n^{-2}A)^{-1} || \le || J_n || || J_{-n} || & \le (1-
 \beta/_n)^{-1} \left(1- \frac{\beta}{n}\right)^{-1}\\ 
 &= (1- \beta^2/ n^2)^{-1}.
\end{align*} 

Therefore $A^2$ is the infinitesimal generator of a semi-group $\exp (tA^2)$. 
\begin{align*}
 \exp (t A^2) x &= s-\lim_{m\to \infty}~ \exp (tA^2 (I- m^{-1} A^2)^{-1})x\\
 &= s-\lim_{m\to \infty} ~\exp (m^2 t [ (I -m^{-1} A^2 )^{-1} -I]) x
\end{align*}  
the convergence being uniform in $t$ in any finite interval of $t$.

We have 
\begin{align*}
 || T_t \hat{T}_t x- T^{(n)}_t \hat{T}^{(n)}_t x || &\le || T_t
 \hat{T}_t x- T^{(n)}_t \hat{T}_t x || + || T^{(n)}_t \hat{T}_t x-
 T_t^{(n)} \hat{T}_t^{(n)} x || \\ 
 &\le || \left(T_t - T^{(n)}_t \right) \hat{T}_t X || + \exp \left(
 \frac{\beta t}{1-n^{-1} \beta}\right) || \hat{T}_t x-
 \hat{T}_t^{(n)} x || \\ 
 & \to 0 \text{ as } n \to \infty,
\end{align*}
uniformly in $t$ in any bounded interval of $t$.

That the first on the right tends to zero uniformly in $t$ in any
bounded interval of $t$ may be proved as follows: Let $0 \le t \le t_o
< \infty$. $(t_o > 0)$. For any $\varepsilon > 0$, we can find $t_1,
\ldots, t_k, 0 \le t_1, \ldots, t_k \le t_o$ such that 
$$
\inf_{1 \le i \le k} || \hat{T}_k x- T_{t_i}x || \le \varepsilon, 
$$
(by\pageoriginale the strong continuity of $T_t$ in $t$).

Now
$$
|| (T_t- T_t^{(n)}) \hat{T}_{t_i} x || \to 0~ (i= 1,2, \ldots, k)
$$
uniformly in $t$ for $0 \le t \le t_o$, and hence, choosing $t_i $
properly for given $t$, we have 
\begin{align*}
 || (T_t - t^{(n)}_t ) \hat{T}_t X || \le || (T_t - T^{(n)}_t
 \hat{T}_{t_i} x || + || (T_t- T^{(n)}_t) (\hat{T})t -
 \hat{T}_{t_i}) x || \\ 
 \le || (T_t - T^{(n)}_t) \hat{T}_{t_i} x || + \left[\exp \beta t + \exp
  \frac{\beta t}{1-n^{-1}-\beta}\right] \varepsilon.  
\end{align*} 

So the right side tends to zero uniformly in $0 \le t \le t_o$.

Since 
\begin{align*}
 T^{(n)}_t \hat{T}_n^{(n)} x &= \exp \left(nt \left[(I-n^{-1}A)^{-1}+
  (I+n^{-1}A)^{-1}-2 I\right]\right)x \\ 
 &= \exp \left(\frac{st}{n}. n^2\left[(I- n^{-2} A^2)^{-1}
  -I\right]\right) x,
\end{align*}
we have
$$
T_t \hat{T}_t x= s-\lim_{n \to \infty} ~\exp ( \frac{2t}{n}. n^2[(I-
 n^{-2} A^2)^{-1} -I])x) 
$$
the convergence being uniform in any bounded interval of $t$. Thus 
$$
T_t \hat{T}_t x= \exp (0. A^2 x) =x.
$$ 

Similarly
$$
\hat{T}_t T_t x= x.
$$

\begin{remark*}%Remk
 For an alternative proof of the above theorem, see $E$. Hille: Une
 g\'en\'eralisation du probl\`em de Cauchy, {\em Ann. de 1' Institut
  Fourier}, 4 (1952), p.37 (Th\'eor\`eme 4). 
\end{remark*}
