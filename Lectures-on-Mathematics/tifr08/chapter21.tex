\chapter{Lecture 21}\label{chap21}%lecture 21

\section{Integration of the Fokker-Planck equation}\label{chap21:sec1}

We\pageoriginale consider the Fokker-Planck equation
\begin{gather*}
 \frac{\partial u(t,x)}{\partial t}=Au(t,x), \quad t \geq 0\\
 (Af)(x)=\frac{1}{\sqrt{g(x)}} \frac{\partial^2}{\partial x^i \partial
  x^j}
 (\sqrt{g(x)} a^{ij}(x)f(x))-\frac{1}{\sqrt{g(x)}}\frac{\partial}{\partial
  x^i}(g(x)b^i(x)f(x)) 
\end{gather*}
in a relatively compact subdomain $R$ (with a smooth boundary) of an
oriented $n$-dimensional Reimannian space with the metric
$ds^2=g_{ij}$ $(x)dx^i$ $dx^j$. As usual the volume element in $R$ is
defined by $dx=\sqrt {g(x)}dx^1$ $\cdots dx^n$. where $g(x)=\det
(g_{ij}(x))$. We assume that the contravariant tensor $a^{ij}(x)$ is
such that $a^{ij}\xi_i \xi_j >0$ for $\sum \limits^m_{i=1}\xi^2_i
>0,\xi_i$ real. The functions obey, for the coordinate transformation
$x\to \bar{x}$, the transformation rule 
$$
\bar{b}^i(\bar{x})=\frac{\partial \bar{x}}{\partial x^k}b^k +
\frac{\partial^2 \bar{x}}{\partial x^k \partial x^s} a^{ks}(x). 
$$

We assume that $g_{ij}(x),a^{ij}(x)$ and $b^i(x)$ are $C^\infty$
function of the local coordinates $x=(x^1 \cdots x^m)$. 

Suggested by the probabilistic interpretation of the Fokker-Planck
equation due to $A$. Kolmogorov, we shall solve the Cauchy problem in
the space $L_1(R)$. 

\section*{Green's integral formula:}

Let $A^*$ be the formal adjoint of $A$;

$$
A^*=a^{ij}(x)\frac{\partial^2}{\partial x^i \partial x^j}+a^i
(x)\frac{\partial}{\partial x^i}. 
$$

Let $G$ be a subdomain of $R$ with a smooth boundary $\partial
G$. Then we obtain by partial integration Green's formula: 
{\fontsize{10pt}{12pt}\selectfont
\begin{align*}
\int \limits_G \{ h(x)(Af)(x) & -f(x)(A^*h)(x) \} dx\\
 & =\int \limits_{\partial G} \left( \frac{\partial
  \sqrt{g(x)}a^{ij}(x)}{\partial x^j}-\sqrt{g(x)} b^i (x)\right) 
  \cos (n,x^i)f(x)h(x)dS\\ 
 & +\int \limits_{\partial G} \sqrt{g(x)}a^{ij}(x) \left(h(x) \frac{\partial
  f}{\partial x^j}-f (x)\frac{\partial h}{\partial x^j}\right) \cos (n,
 x^i)dS 
\end{align*}}\relax
where\pageoriginale $n$ denotes the outer normal at the point $x$ of $\partial G$
and $ds$ denotes the hypersurface area of $\partial G$. 

\begin{remark*}%Remk
 If $a^{ij}(x)\cos (n,x^i)\cos (n,x^j)>0$ at $x \in \partial G$ we
 may define the transversal (or conormal) direction $\nu$ at $x$ by 
 $$
 \frac{dx_i}{\sqrt{g(x)}a^{ij}(x)\cos (n,x^j)}= d \nu (i=1,2, \ldots, m)
 $$
 so that we have
 $\sqrt{g(x)}a^{ij}(x)(h(x)\dfrac{\partial f}{\partial
  x^i}-f(x)\dfrac{\partial h}{\partial x^j}) \cos (n,x^i)dS$ 
 
 $=(h(x)\dfrac{\partial f}{\partial v}-f(x)\dfrac{\partial
  h}{\partial v}) dS$. 
\end{remark*}

We consider $A$ to be an additive operator defined on the totality of
$D(A)$ of $C^\infty$ functions $f(x)$ in $R U \partial R$ with compact
supports satisfying the following boundary condition: 
\begin{align*}
 \sqrt{g(x)}a^{ij}(x)\frac{\partial f}{\partial x^j}\cos
 (n,x^i) & + \left(\frac{\partial \sqrt{g(x)} a^{ij} (x)}{\partial x^j}
 -\sqrt{g(x)} a^i(x)\right).\\ 
 &\quad . \cos (n,x^i) f(x) =0.
\end{align*}

(When $R$ is a subdomain of the euclidean space $E^m$ and $A$ is the
Laplacian $\Delta$ the above condition is nothing but the co called
``reflecting barrier condition'')  
$$
\frac{\partial f}{\partial n}=O,
$$
since $\nu$ and $n$ coincide in this case). $D(A)$ is dense in the
Banach space $L_1(R)$. 

To\pageoriginale discuss the resolvent of $A$ we begin with
\setcounter{Lemma}{0}
\begin{Lemma}\label{chap21:sec1:lem1}%Lem 1
 Let $f(x)\in D(A)$ be positive (or negative) in domain $G \subseteq R$
 such that $f(x)$ vanishes on $\partial G-\partial R$, (i.e., $f(x)$
 vanishes on the part of $\partial G$ not contained in $\partial
 R$). Then we have the inequality 
 $$
 \int \limits_G (Af)(x)dx \leq 0 \quad \left(\resp \int \limits_G
 Af(x)dx \geq 0\right). 
 $$
\end{Lemma}

\begin{proof}
 Taking $h \equiv 1$ in Green's formula and remembering the boundary
 condition on $f(x)$, we obtain 
 $$
 \int \limits_G (Af)(x)dx= \int \limits_{\partial G-\partial R}
 \frac{\partial f}{\partial \nu}ds 
 $$
 $$
 \leq 0.
 $$
\end{proof}

\begin{coro*}%Corlry
 For $f \in D(A)$ we have for any $\alpha >0 \parallel f-
 \alpha^{-1}Af \parallel \geq \parallel f \parallel$. 
\end{coro*}

\begin{proof}
 Let $h(x)=1,-1$ or $0$
 according as $f(x)$ is $>0, <0$ or $= 0$. Since the conjugate space of
 $L_1(R)$ is the space of essentially bounded function $k(x)$ with the
 norm 
 $$
 \parallel k \parallel^*= \text{ essential } \sup_{x \in R} |k(x)|,
 $$
 we have
 \begin{align*}
  \parallel f- \alpha^{-1} Af \parallel & \geq \int \limits_R h(x)
  \left\{ f(x)- \alpha^{-1} Af(x) \right\} dx\\ 
  &= \int |f(x)|dx- \alpha^{-1} \sum_i \int\limits_{P_i} (Af)(x)dx\\
  &+ \alpha^{-1}\sum_i \int \limits_{N_j} (Af)(x)dx
 \end{align*}
 where $P$ (\resp $N$) is connected domain in which $f(x)>0$ (\resp $<0$)
 such that $f(x)$ vanishes on $\partial P(\text{\resp} \partial N)$. 
\end{proof}

\begin{Lemma}\label{chap21:sec1:lem2}%Lem 2
 The smallest closed extension $\tilde {A}$ of $A$ exists and for any
 $\alpha > 0$ the operator $(I- \alpha^{-1} \tilde {A})$ admits of a
 bounded inverse, $J_\alpha=(I-\alpha^{-1} \tilde{A})^{-1}$ with
 norm $\leq 1$. 
\end{Lemma}

\begin{proof}
 The\pageoriginale existence of $\tilde{A}$ follows from Green's formula. For if
 $\{ f_k \} \subseteq D(A)$ be such that
 strong $\lim f_k=0$, strong $\lim Af_k=h$, then for $\varphi \in
 \mathscr{D}^\infty (R)$, 
 \begin{align*}
  \lim \int \limits_R & \left\{ \varphi Af_k-f_k A^* \varphi
  \right\} dx=0, ~\text(or)\\ 
  & \int \varphi h dx=0. ~\text{ So}~ h=0.
 \end{align*}
 
 The other part of the lemma follows from the corollary of
 lemma \ref{chap21:sec1:lem1}. 
\end{proof}

\begin{Lemma}\label{chap21:sec1:lem3}%Lem 3
 $\tilde {A}$ is the infinitesimal generator of a semi-group $T_t$ in
 $L_1(R)$ if and only if for sufficiently large $\alpha$ the range
 $\{ (I-\alpha^{-1} A)f,f \in D(A) \}$ of the operator $(I-
 \alpha^{-1}A)$ is dense in $L_1(R)$. Moreover, if this condition is
 satisfied, then $J_\alpha$ is a transition operator, i.e., if
 $f(x)\geq 0$ and $f \in L_1 (R)$, then $(J_\alpha f)(x)\geq 0$ and 
 $$
 \int \limits_R (J_ \alpha f)(x)dx= \int \limits_R f(x)dx.
 $$
\end{Lemma}

\begin{proof}
 The first part is evident. Then latter part may be proved as
 follows: For any $g(x) \geq 0$ of $L_1 (R)$ there exists a sequence
 $\{ f_k (x)\} \subset D(A)$ such that $s-\lim f_k=f$ exists and
 $s-\lim (f_k -\alpha^{-1}Af_k)=f- \alpha^{-1}\tilde {A}f=g$. By the
 boundary condition on $f_k$, we have 
 $$
 \int \limits_R (f_k -\alpha^{-1}Af_k)dx= \int \limits_R f_k dx,
 $$
 (Put $h(x)\equiv 1$ in Green's formula). So in the limit we have 
 $$
 \int \limits_R gdx= \int fdx.
 $$
 Also, by the Corollary to Lemma \ref{chap21:sec1:lem1},
 $$
 \displaylines{\hfill 
 \int |f_k- \alpha^{-1} A f_k| dx \geq \int |f_k|dx,\hfill \cr
 \text{and hence}\hfill 
 \int |g|dx \geq \int |f|dx. \hfill }
 $$
 Therefore\pageoriginale by the positivity of $g(x)$
 $$
 \int f(x)dx=\int g(x)dx= \int |g(x)|dx \geq \int |f(x)| dx
 $$
 proving that $J_ \alpha$ is a transition operator.
\end{proof}
Therefore the semi-group
\begin{align*}
 T_t u &= \strong_{\alpha \to \infty} \lim \exp (t \tilde{A}J_\alpha)u\\
 & = \strong_{\alpha \to \infty}\lim \exp (\alpha t(J_ \alpha -I)u)
\end{align*}
is a semi-group of transition operators.
