\chapter{Lecture 20} % lecture 20

\section{Cauchy problem for the wave equation (continued)}

\textbf{Fifth step:}\pageoriginale If $f$ and $g$ satisfy the conditions of the
theorem, i.e., if $A^k f \in H, A^k g \in H(k = 0, 1, \ldots)$, we
have 
$$
\bar{ \mathscr{O}}^k \binom{f}{g} = \mathscr{O}^k \binom{f}{g} \in
\bar{H}_1 \times H_o ( k = 0, 1, \ldots) ,  
$$
i.e., $(^f_g)$ is in the domain of every power of
$\bar{\mathscr{O}}^k$. So, referring to step 4, we find that vectors 
$$
\binom{v(t, x)}{v(t, x)}= T_t \binom{f(x)}{g(x)}
$$
are in the domain of every power of $\mathscr{O}:$
$$
\bar{\mathscr{O}}^k \binom{u (t, x)}{v(t, x)}
\in \bar{H}_1 \times H_o. ( k = 0, 1, 2, \ldots ) 
$$

Thus, for integral $k \geq 0, u (t,  x )$ is for $t$ fixed, a weak
solution of the equation 
$$
A^k u = f^{(k)}, ~\text{ with}~ f^k \in L_2 (E^m)
$$
$A^k $ is an elliptic operator of order $2k$ and $k$ may be taken
arbitrarily large. We see therefore by the Friedrichs-Lax-Nirenberg
theorem and Sobolev's lemma, that $u(t, x )$ is $C^\infty$ in $x$ (for
fixed $t$). And the same statement holds for $v(t, x )$. 

Since $|| T_t|| \exp \beta (|t|)$ we see that 
$$
|| u (t, x ) ||^2_1 +
|| v(t, x ) ||^2_o \leq \const. \exp (2 \beta |t|) \left\{|| f||^2_1 +
||g ||^2_0 \right\}.
$$ 

This, combined with the strong continuity of $T_t$ in $t$, shows that
$u(t, x )$ and $ v(t, x )$ are locally square summable in the product
space\break $ (- \infty < t < \infty )\times E^m$. And we have, for the
second order strong derivative $\partial^2_t$, 
$$
\partial^2_t u (t, x ) = A u (t, x ) 
$$
so\pageoriginale that $(\partial^2_t + A ) u = 2 A u, (\partial^2_t + A )^k u =
(2A)^k u$,.  

Since $\dfrac{\partial^2}{\partial t^2} + A $ is an elliptic operator in
$(- \infty < t < \infty) \times E^m$, we see that $u(t, x )$ is almost
everywhere equal to a function $C^\infty$ in $(t, x)$. 

The proof of the first step is obtained by the 
\begin{lemma*}
 Let $f, g \in H$ and $Af\in H_o$. Then 
 $$
 (Af, g )_o = -\int a^{ij} \frac{\partial f}{\partial x_i}
 \frac{\partial g}{\partial x_j} dx -
 \frac{\partial a^{ij}}{\partial x_j} \frac{\partial f}{\partial x_i}
 g \,d \,x + \int b^i 
 \frac{\partial f}{\partial x_i} g \,d \,x + \int c f g. 
 $$

 And we can also partially integrate the second and the third terms on
 the right, so that the first order derivatives of $\dfrac{\partial
  f}{\partial x_i}$ shall be eliminated, and the integrated terms are
 nought.  
\end{lemma*}

\begin{proof}
 From $Af \in H_o$ and $g \in H$ we see that $a^{ij} \dfrac{\partial
 ^2 f }{\partial x_i \partial x_j}. g$ is integrable over
 $E^m$. Thus, by Fubini theorem,  
 \begin{align*}
   & \int\limits_{ E^m} a^{ij} \frac{\partial^2 f }{\partial x_i
   \partial x_i \partial x_j} g \,d \,x 
   = \lim_{\delta \to + \infty}\int^{\infty}_{- \infty} dx_2 \cdots dx_m
  \int\limits^{\delta_1}_{ \varepsilon_1} a^{ij} \frac{\partial^2
   f}{\partial x_i \partial x_j} g \,d \,x_1\\ 
   & \quad \int\limits_{\varepsilon
  _1}^{ \delta_1} a^{ij} \frac{\partial^2 f}{\partial x_i \partial
   x_j} gdx_1 
   = \left[a^{ij} \frac{\partial f }{\partial x_i} g \right]^{x_1 =
   \delta_1}_{x_1= \varepsilon_1} 
   - \left[\int\limits_{\varepsilon_1}^{\delta} a^{ij}
         \frac{\partial f}{\partial x_j} \frac{\partial g}{\partial x_i} dx
        _1 \right.\\ 
         & \hspace{4cm} \left. + \int\limits_{\varepsilon}^{\delta} \frac{\partial
          a^{ij}}{\partial x_i} \frac{\partial f }{\partial
          x_j} g d x \right] 
   + \int\limits^{ \partial_1}_{\varepsilon_1} \sum_{i, j \neq 1} a^{ij}
        \frac{\partial^2 f}{\partial x_i \partial x_j} g d x_1. \\ 
        & = k_1 (\delta_1, \varepsilon_1, x_2, \ldots, x_m)
        + k_2 (\delta_1, \varepsilon_2, x_2, \ldots, x_m) + k_3
        (\delta_1, \varepsilon_1, x_2, \ldots, x_m) 
 \end{align*}
 
 By Schwarz's inequality, we have $\big| \int\limits^\infty_{ -
  \infty}dx_2 \ldots dx_m k_1 \big| $ 
{\fontsize{10pt}{12pt}\selectfont
 $$
 \leq \eta \sum_j \int\limits^{\infty}_{- \infty} dx_2 \ldots dx_m \big
 | \frac{\partial f (\delta_1, x_2 \ldots, x_m)^2}{\partial x_j}\big
 | \cdot \int\limits^{ \infty}_{- \infty} dx_2 \ldots dx_m | g (\delta
_1, x_2. x_m)^2 )^{\frac{1}{2}} 
 $$ }
 $+$\pageoriginale similar terms pertaining to $\varepsilon_1$ instead of $\delta_1$. 
 
 Since 
 \begin{gather*}
  \int\limits_{ E^m} g^2 dx = \int\limits^{\infty}_{- \infty} dx_1
  \int^{\infty}_{ - \infty} \cdots \int_{- \infty }| g(x_1, x_2,
  \ldots, x_m)^2 d x_2, \ldots, d x_m, \\ 
  \int\limits_{E^m} | \frac{\partial f}{\partial x_j}|^2 dx = \int
  dx_1 
  \int\limits^{ \infty}{-\infty} \int |
  |\frac{\partial f}{\partial x_j} (x_1, x_2, \ldots, x_m)|^2 dx_2
  \ldots dx_m, 
 \end{gather*}
 we see there exists $\{ \delta^{(n)}_1\}$ and $\{ \varepsilon_1^(n)\}$
such that  
 $$
 \lim\limits_{\substack{\delta_1^{(n)} \to
     \infty\\ \varepsilon_1^{(n)}\to \infty}} \int k_1
 (\varepsilon_1^{(n)}, \delta^{(n)}_{1}, x_2, \ldots, x_m) dx_2 \ldots dx_m = 0. 
 $$
 
 On the other hand, since $f, g \dfrac{\partial f}{\partial x_j}$,
 $\dfrac{\partial g}{\partial x_1} \in H_o$, we see that 
 \begin{gather*}
  \lim\limits_{\substack{\delta_1 \to \infty\\ \varepsilon_1 \to - \infty}}
  \int k_2 (\varepsilon_1, \delta_1, x_2, \ldots, x_m) d x_2,
  \ldots, dx_m \\ 
  = \int\limits_{ E^m} \big\{ - a^{ij} \frac{\partial f }{\partial
   x_j} \frac{\partial 
   g }{\partial x_1} - \frac{\partial^{ij}}{\partial x_1}
  \frac{\partial f }{\partial x_j} g \big\} d x = k_2 
 \end{gather*}
 is finite, Thus, 
 $$
 \int a^{ij} \frac{\partial^2 f}{\partial x_i \partial x_j} g d x =
 k_2 + \lim\limits_{\substack{\delta(n) \to \infty \\ \varepsilon
  ^{(n)}_1 \to-\infty}} \int^{\infty}_{- \infty} k_3
 (\varepsilon^{(n)}_1, \delta^{(n)}_{1}, x_2, \ldots, x ) dx_2
 \ldots dx_m 
 $$
 
 Hence   
 $$
 \int\limits^{ \delta_1^{(n)}}_{ \varepsilon^{(n)}_1} \sum_{ i, j \neq
  1} a^{ij} \frac{\partial^2 f }{\partial x_i \partial x_j} g dx_1 
 $$
 is integrable over $-\infty < x_i < \infty (i = 2, \ldots, m)$. 
\end{proof}

Hence 
\begin{align*}
 k_3 & = \lim\limits_{\substack{\delta^{(n)}_1 \to \infty
   \\ \varepsilon^{(n)}_1 } \to -\infty} \int^{\infty}_{- \infty}
 dx_2 \ldots dx_m k_3 (\varepsilon^{(n)}_1, \delta^{(n)}_1, x_2,
 \ldots, x_m)\\ 
& = \lim\limits_{\substack{ \delta^{(n)}_1 \to \infty
   \\ \varepsilon^{(n)}_1 } \to -\infty}  \lim\limits_{\substack{
   \delta_{2} \to \infty \\ \varepsilon_{2} } \to \infty} \int dx_3
 \ldots dx_m
 \left\{ \int\limits^{ \delta_2}_{ \varepsilon_2} dx_2
 \int\limits^{ \delta^{(n)}_1}_{ \varepsilon^{(n)}_1} \sum_{ i, j
  \neq 1} a^{ij} \frac{\partial^2 f}{\partial x_i \partial x_j} g
 dx_1 \right\}
\end{align*}

However\pageoriginale 
\begin{align*}
  \Big\{ \cdots \cdots \Big\} & =
  \int\limits^{\delta^{(n)}_{1}}_{ \varepsilon^{(n)}_1}
  dx_1 \int\limits_{\varepsilon_2}^{\delta_2} \sum_{i, j \neq 1}
  -a^{ij} \frac{\partial^2 f}{\partial x_i \partial x_{ij}} dx_2\\ 
  & = \int\limits^{\delta_1^{(n)}}_{\varepsilon_1^{(n)}} dx_1 \left[ \left[ a^{2j}
      \frac{\partial f}{\partial x_j} g \right]^{x_2=
          \delta_2}_{x_2=\varepsilon_2}+ \int\limits^{\delta_2}_{\varepsilon_2} -a^{2j}
    \frac{\partial f}{\partial x_j} \frac{\partial g}{\partial x_2}
    dx_2\right.\\
    &\hspace{1cm}\left.-\int\limits_{\varepsilon 2}^{\delta_ 2} \frac{\partial a^{2j}
    }{\partial x_2} \frac{\partial f}{\partial x_j}  g dx_2\right] 
  + \int\limits^{\delta^{(n)}_1}_{\varepsilon^{(n)}_1}
  \int\limits_{\varepsilon_2}^{\delta_2} \sum_{ i, j \neq 1, 2
  }\frac{\partial^2 f}{\partial x_i \partial x_j} g dx_1
  dx_2 
\end{align*} 
we have 
\begin{multline*}
  \left| \int^{\infty}_{-\infty} dx_3 \ldots dx_m \int^{ \delta_1
 ^{(n)}}_{\varepsilon^{(n)}_1} a^{2j} \frac{\partial
  f}{\partial x_j} g dx_1\right| \\ 
 \leq \eta \sum_j \left(\int dx_1 dx_3, \ldots dx_m \left|\frac{\partial
  f}{\partial x_j}\right|^2 \int g^2 dx_1 dx_3 \ldots dx_m \right)^{\frac{1}{2}}  
\end{multline*}
and so, by the integrability on $E^m$ of $\left|\dfrac{\partial
 f}{\partial x_j}\right|^2$ and $|g|^2$, there exists $\delta^{(1)}_{(2)},
\varepsilon^{(1)}_{(2)} $ such that 
$$
\lim_{\substack {\delta^{(1)}_2 \to \infty \\ \varepsilon^{(1)}_2 \to
  \infty}}\int dx_3 \ldots dx_m
\int\limits_{\varepsilon_1}^{\delta_1} \left[ a^{2j} \frac{\partial
  f}{\partial x_j}g\right]^{x^2 = \delta_1 (1)}_{x_2 = \varepsilon_2
 (1)} dx_1 = 0 
$$
uniformly with respect to $\delta_1$ and $\varepsilon_1$. 

We have also 
\begin{multline*}
 \lim\limits_ {\substack{\delta^{(n)}_1 \to
     \infty\\{\varepsilon^{(n)}_1 \to
    -\infty}}}\lim\limits_{\substack{\delta_2\to \infty\\{\varepsilon_2\to
    -\infty}}} \int dx_3 \cdots dx_m \int
^{\delta^{(n)}_1}_{\varepsilon^{(n)}_1}dx_1\\ 
 \left\{ \int \limits
^{\delta_2}_{\varepsilon 2} \left[-a^{2j}\frac{\partial f}{\partial
   x_j} \frac{\partial g}{\partial x_1}-\frac{\partial
  a^{2j}}{\partial x_2} \frac{\partial g}{\partial x_j} dx_2\right]
 \right\}\\ 
 = \int \limits_{E^m} \left(-a^{2j}\frac{\partial f}{\partial
  x_j}\frac{\partial g}{\partial x_2}-\frac{\partial
  a^{2j}}{\partial x_2} \frac{\partial g}{\partial x_j} g\right)dx. 
\end{multline*}

Therefore\pageoriginale
\begin{multline*}
 \int a^{ij} \frac{\partial^2 f}{\partial x_i \partial x_j} gdx=-
 \int \sum_{i \text{ or} j=1,2} a^{ij} \frac{\partial f}{\partial
  x_i} \frac{\partial g}{\partial x_j}dx
 -\int \sum_{i \text{ or}j=1,2} \frac{\partial a^{ij}}{\partial
  x_i} \frac{\partial f}{\partial x_j}gdx \\
  +\lim\limits_{\substack{\delta^{(n)}_1 \to \infty\\{\varepsilon^{(n)}_1\to
   -\infty}}}\lim\limits_{\substack{\delta_2^{(1)} \to \infty\\{\varepsilon_2^{(1)}\to
   -\infty}}} \int\limits^\infty_{-\infty} dx_3 \cdots dx_m \int
^{\delta^{(k)}_1}_{\varepsilon^{(k)}_1} \int
^{\delta^{(1)}_2}_{\varepsilon^{(1)}_2} \sum_{i,j \neq 1,2} a^{ij}
 \frac{\partial^2 f}{\partial x_i \partial
 x_j}g dx_1dx_2 
\end{multline*}

Repeating the process, we get the Lemma.
