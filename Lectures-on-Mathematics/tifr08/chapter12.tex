\chapter{Lecture 12}%Lect 12

\section[Brownian motion on a homogeneous...]{Brownian motion on a homogeneous Riemannian space (Contd.)}\label{chap12:sec1}
\begin{proof}
 \pageoriginale
 \begin{Step}
  Remarking that $\mathscr{D} (A)$ is dense in $C[R]$ and choosing $f$
  and $g$ properly we obtain 
  \begin{enumerate}[(a)]
  \item there exist $C^\infty$ functions $F^1 (x), \ldots, F^n (x)
   \in \mathscr{D} (A)$ such that the Jacobian $\dfrac{\partial (F^1
    (x), \ldots, F^n (x))}{\partial (x^1, \ldots, x^n)} > 0$ at
   $x_o$. 
  \item there exists a $C^\infty$ function $F_o (x) \in \mathscr{D}
   (A)$ such that 
  $$
  (x^i - x^i_o ) (x^j - x^j_o) \frac{\partial^2 F}{\partial x^i_o
  \partial x^j_o} \ge \sum^n_{i = 1} (x^i - x^i_o)^2. 
  $$
 \end{enumerate}
 \end{Step}

 We can use $F^1 (x), \ldots, F^n (x)$ as coordinate functions in a
 neighbourhood $d(x_o, x) < \varepsilon $; we denote these new local
 coordinates by $(x_1, \ldots, x_n)$. 
 
 Since $F^i (x) \in \mathscr{D} (A)$,
 $$
 s-\lim_{t \downarrow o} \frac{T_t F^i (x) - F^i (x)}{t}
 $$
 exists and $= A F^i (x)$
 \begin{align*}
  (A F^i) (x) & = \lim_{t \downarrow o} t^{-1} \int\limits_R P(t, x_o,
  dx) (F^i (x) - F^i (x_o))\\ 
  & = \lim_{t \downarrow o} t^{-1} \int\limits_{d (x, x_o) \le
   \varepsilon} P(t, x, dx) (F^i (x) - F (x_o)) 
 \end{align*}
 independent of $\varepsilon > 0$, by Lindeberg's
 condition. So, for the coordinate functions $x^1 \cdots x^n, (x^i =
 F^i)$, 
 $$
 \lim_{t \downarrow o} t^{-1} \int\limits_{d (x, x_o)\le \varepsilon}
 (x^i - x_o ) P(t, x_o, dx) = a^i (x_o) 
 $$
 independent of $\varepsilon > 0$. Since $F_o \in \mathscr{D} (A)$, we
 have, using Lindeberg's condition, 
 \begin{align*}
  (A F_o) (x_o) & = \lim_{t \downarrow o} t^{-1} \int\limits_{R} P(t,
  x_o, dx) (F (x) - F_o (x_o))\\ 
  & = \lim_{t \downarrow o} \int\limits_{d (x, x_o) \le \varepsilon}
  P(t, x_o, dx) (F (x) - F_o (x_o ))\\ 
  & = \lim_{t \downarrow o} \left[ t^{-1} \int\limits_{d(x, x_o) \le
   \varepsilon} (x^i - x^i_o) \frac{\partial F_o}{\partial x^i_o}
  P(t, x_o, dx)\right. \\ 
  & \quad + t^{-1} \int\limits_{d(x-x_o) \le \varepsilon} (x^i - x^i_o) (x^j
  - x^j_o) \left( \frac{\partial^2 F_o}{\partial x^i \partial x^j}
  \right) P(t, x_o, dx) \\ 
  & \hspace{5cm} x= x_o + \Theta (x - x_o 0 < \Theta 1. 
 \end{align*}
 
 The\pageoriginale first term on the right has a limit $a^i (x_o) \dfrac{\partial
  F_o}{\partial x^i_o}$; hence by the positivity of $P$, and $(b)$, 
 \begin{equation}
  \overline{\lim}_{t \downarrow o} t^{-1} \int\limits_{d(x, x_o) \le
   \varepsilon} \sum_{i = 1}^n (x^i - x^i_o)^2 P(t, x_o, dx) <
  \infty \tag{*} 
 \end{equation}
 
 \begin{Step} % step 3 
  Let $f \in \mathscr{D}(A) \cap C^2$. Then, expanding $f(x)- f(x_o)$,
  \begin{align*}
   \frac{T_t f(x_o) - f(x_o)}{t} & = t^{-1} \int\limits_ R f(x) -
   f(x_o) P(t, x_o, dx)\\ 
   & = t^{-1} \int\limits_{d(x, x_o)> \varepsilon} f(x) - f(x_o)) P
   (t, x_o, dx)\\ 
   & + t^{-1} \int\limits_{d(x, x_o ) \le \varepsilon} (x^i - x^i_o )
   \frac{\partial f}{\partial x^o_i} P(t, x_o, dx)\\ 
   & + t^{-1} \int\limits_{d(x, x_o ) \le \varepsilon} (x^i - x^i_o )
   (x^j -x^j_o) \frac{\partial^ 2 f}{\partial x^i_o \partial x^j_o}
   P(t, x_o, dx)\\ 
   & + t^{-1} \int\limits_{d(x, x_o ) \le \varepsilon} (x^i - x^i_o
   )(x^j - x^j_o) C_{i_j} (\varepsilon) P(t, x_o, dx)\\ 
   & = C_1 (t, \varepsilon) + C_2 (t, \varepsilon) + C_3 (t,
   \varepsilon) + C_4 (t, \varepsilon), \text { say }, 
  \end{align*}
  where\pageoriginale $C_{i_j} (\varepsilon) \to 0$ as $\varepsilon \downarrow 0$. We
  know that $\lim\limits_{t \downarrow o} C_1 (t, \varepsilon ) = 0$ for
  fixed $\varepsilon > 0$ (Condition (vi)) and $\lim\limits_{t
   \downarrow o} C_2(t, \varepsilon ) = a^i (x_o) \dfrac{\partial f}{\partial
   x_o^i}$, independently of small $\varepsilon$. By $(*)$ and
  Schwarz's inequality $\lim\limits_{t \downarrow o} C_4 (t,
  \varepsilon) = 0$, boundedly in $t > 0$. Also the left side has a
  finite limit as $t \downarrow 0$. So the difference 
  $$
  \overline{\lim}_{t \downarrow o} C_3 (t, \varepsilon) -
  \frac{\lim}{t ~ o} C_3 (t, \varepsilon) 
  $$
  can be made arbitrarily small by taking $\varepsilon > 0$ small. But
  by $(*)$, Sch\-warz's inequality and $(vi)$, the difference is
  independent of small $\varepsilon > 0$. Thus finite limit
  $\lim\limits_{t \downarrow o} C_3 (t, \varepsilon)$ exists
  independently of small $\varepsilon > 0$. Since we may choose $F \in
  \mathscr{D} (A) \cap C^\infty$ such that
  $$
  \frac{\partial^2 F}{\partial x_\circ^i \partial x_\circ^i} \quad (i,
  j=1, \ldots , n)
  $$
  is arbitrarily near $\alpha_{ij}$ \quad  $\alpha_{ij}$ being
  constants, it follows, by an argument similar to the one above that
  $$
  \displaylines{
  \text{finite limit}\hfill \int\limits_{d(x, x_o) \le \varepsilon} (x^i -
  x^i_o)\, (x^j - x^j_o) P(t, x_o, dx) = b^{ij} (x_o) \hfill \cr
  \text{exists and}\hfill
   \lim_{t \downarrow o} C_3 (t, \varepsilon) = b^{ij} (x_o)
  \frac{\partial^2 F}{\partial x^i_o \partial x^j_o}.\hfill } 
  $$
 \end{Step}
 This completes the proof of the theorem.
\end{proof}

\begin{remark*}%Remk
 \begin{enumerate} [i)]
 \item We have $b^{ij} (x) = ~ b^{ij } (x)$ and
  \begin{gather*}
   b^{ij} (x_o) \xi_i \xi_j \ge 0, (\xi_i \text {real}), \text { for },\\
   (x^i - x^i_o) (x^j - x^j_o) \xi_i \xi_j = \left(\sum (x^i -
   x^i_o) \xi_i \right)^2 
  \end{gather*}
 \item $b^{ij}(x)$\pageoriginale is a contravariant tensor:
  $$
  \displaylines{\hfill 
  \bar{b}^{ij} = b^{kl} \frac{\partial \bar{x}^i}{\partial x^k}. ~
  \frac{\partial \bar{x}^j}{\partial x^1} (x^1, \ldots, x^n) \to
  (\bar{x}^1, \ldots, \bar{x}^n) \hfill \cr
  \text{and}\hfill 
  \bar{a}^m = a^s \frac{\partial \bar{x}^m}{\partial x^s} + b^{kl}
  \frac{\partial^2 \bar{x}^m}{\partial x^k \partial x^l}.\hspace{3.2
   cm} \hfill }
  $$
 \end{enumerate}
\end{remark*}

This follows from the equality 
$$
\bar{b}^{ij} ~ \frac{\partial^2 f}{\partial \bar{x}^i_o \partial
 \bar{x}^j_o} + \bar{a}^m ~ \frac{\partial f} {\partial \bar{x}^m} =
~ b^{k1} \frac{\partial^2 f}{\partial x^k \partial^1} + a^s
\frac{\partial f}{\partial x^s} 
$$
[since each is $ = (Af ) (x_o) $].
