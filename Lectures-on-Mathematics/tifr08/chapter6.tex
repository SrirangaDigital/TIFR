\chapter{Lecture 6}\label{chap6}

\begin{theorem*}
 For\pageoriginale $ n > \beta $, the operator $ (I - n^{-1} A) $ admits of an
 inverse $ J_n = ( I -n^{-1} A )^{-1} $ which is linear and
 satisfies the relation 

 $ J_n x = n \int\limits^{\infty}_{0} $ $ e^{-ns} T_s x ds $, for $ x
 \in X $ (i.e., $J_n = C_{\varphi_{n}} $ Also $ \mid\mid J_n
 \mid\mid \leq ( 1-n^{-1} \beta )^{-1} $. 
\end{theorem*}

\begin{proof}
 We first show that $ (I -n^{-1} A )^{-1} $ exists [i.e., ~
  $( I - n^{-1} A)$ is one -one]. If $ (I - n^{-1} A) $ is
 not one-one, there will exist $ x_0 \in \mathscr{D} (A) $ such that
 $ \mid\mid x_0 \mid\mid = 1$ and $(I - n^{-1} A) x_0 ~ = 0 $, i.e.,
 $Ax_0=n x_0$. Let $ f_0 $ be a linear functional on $X$ such that $
 \mid\mid f_0 \mid\mid = 1 $ and $ f_0 (x_0) = 1 $. Define $ \varphi
 (t) = f_0 ( T_t x_0 )=1 $. Define $ \varphi (t) = f_0 (T_t x_0 )
 $. Since $x_0 \in \mathscr{D} (A) $, $ \varphi (t) $ is
 differentiable and 
 \begin{align*}
  \frac{d \varphi (t)}{dt} &= f_\circ ( D_t T_t x_\circ ) = f_\circ
  (T_t A x_\circ) = f_\circ (T_t n x_\circ) \\ 
  &= n f_\circ ( T_t x_\circ ) \\
  &= n \varphi (t).
 \end{align*}
\end{proof}

Solving this differential equation with the initial condition $
\varphi (0) = 1 $ we get $ \varphi (t) = e^{nt} $. On the other hand
we have 
\begin{align*}
 \mid \varphi (t) \mid = \mid f_0 ( T_t x_0 ) \mid &\leq \mid\mid
 f_0 \mid\mid ~ \mid\mid T_t \mid\mid ~ \mid\mid x_0 \mid\mid \\ 
 &\leq~e^{\beta t};
\end{align*}
since $ \varphi (t) = e^{nt} $ and $ n > \beta $ this is
impossible. So $ ( I - n^{-1} A )^{-1} $ exists. 

Since $ A ~ C_{\varphi_{n}} x = n ( C_{\varphi_{n}} ~ - I )x $, we
have $(I_n- n^{-1}A) C_{\varphi_{n}} ~ x = x $ for all $ x \in X $. So
$ ( I - n^{-1} A )$ maps $\mathfrak{W} (C_n) \subseteq \mathscr{D} (A) $
on to $X$; thus\pageoriginale $ (I - n^{-1} A ) $ maps $ \mathscr{D} (A) $ in a
one-one manner onto $X$. It follows that $ \mathscr{M} (
C_{\varphi_{n}} )~ \mathscr{D} (A)$ and $(I - n^{-1} A)^{-1} =
C_{\varphi_{n}}$. But $C_{\varphi_{n}}$ is a linear operator and we
have already proved that $ \mid\mid C_{\varphi_{n}} \mid\mid \leq (
1-n^{-1} \beta )^{-1} $. 

\begin{coro*}
 \begin{align*}
  \mathfrak{M} ( C_{\varphi_{n}} ) &= \mathscr{D} (A) \\
  A J_n x &= n ( J_n - I ) X, ~ x \in X. \\
  A J_n x &= J_n A x = n (J_n - I ) x, ~ x \in \mathscr{D} (A) \\
  s- \lim_{n \rightarrow \infty} J_n x &= x, ~ x \in X, \\
  D_t T_t x &= s - \lim_{h \rightarrow 0 } h^{-1} ( T_{t+h} - T_t )
  x = A T_t x = T_t A x, ~ x \in \mathscr{D} (A). 
 \end{align*}
\end{coro*}

\section[The resolvent set and...]{The resolvent set and the spectrum of an additive operator on
 a Banach space}\label{chap6:sec1} 

We may state our theorem in the terminology of spectral theory. 

Let $A$ be an additive operator ( with domain $ \mathscr{D} (A) ) $
from a Banach space $X$ into $X$. Let $ \lambda $ be a complex number
($\lambda $ is assumed to be real if $X$ is a real space
). Regarding the inverse of the additive operator $ (\lambda I - A )
$ there are various possibilities. 
\begin{enumerate}[(1)]
\item $ ( \lambda I -A ) $ does not admit of an inverse, $ i.e.$,
 there exists an $ x \neq 0 $ such that $ A x = \lambda x $. We then
 call $ \lambda $ an eigenvalue of $A$ and $x$ an
 \textit{eigenvector} belonging to the eigenvalue $ \lambda $. In
 this case we also say that $ \lambda $ is in the \textit{
  point-spectrum} of $A$. 
\item When $ ( \lambda I - A )^{-1} $ exists there are three
 possibilities: 
 \begin{enumerate}[(i)]
 \item $ \mathscr{D} (( \lambda I - A )^{-1} ) $ is not dense in
  $X$. Then $\lambda$ is said to be in the \textit{residual
  spectrum} of $A$. 
 \item $ \mathscr{D} (( \lambda I - A )^{-1})$\pageoriginale is dense in $ X $
  but $ ( \lambda I - A )^{-1} $ is not continuous. In this case $
  \lambda $ is said to be in the \textit{continuous} spectrum. 
 \item $ \mathscr{D} (( \lambda I - A )^{-1} ) $ is dense in $X$ and
  $ ( \lambda I - A )^{-1} $ is continuous in $ \mathscr{D} ((
  \lambda I - A )^{-1} ) $. Then $ ( \lambda I -A )^{-1} $ can be
  extended uniquely to a linear operator on the whole space $X$. In
  this case $ \lambda $ is said to be in the \textit{ resolvent
   set}; the inverse $ ( \lambda I - A )^{-1} $ is called the
  \textit{resolvent}. 
 \end{enumerate}
\end{enumerate} 

The complement of the resolvent set in the complex plane (or in the
real line if $X$ is real) is called the spectrum of $A$.

The first part of the theorem proved above says that if $ \{ T_t \} $
is a semi-group of normal type $(\mid\mid T_t \mid\mid \leq e^{\beta
 t})$ any number $\lambda > \beta $ is in the resolvent set of the
infinitesimal generator $A$. 

\section{Examples}\label{chap6:sec2}

 Using these results we now determine the infinitesimal generators of
 the semi-groups we considered earlier. 

$ I: ~ \underline{ C [ 0, \infty]: ~(T_t x) ~ (s) =~ x ( t + s ) } $
\begin{align*}
 \text{ Writing } \hspace{1cm}y_n (s) &= ( J_n x) (s) \quad \text{ we
  have }\hspace{1cm} \\ 
 y_n (s) &= n \int \limits^{\infty}_{0} e^{-nt} x (t +s) dt \\
 &= n \int \limits^{\infty}_{s} e^{-n (t-s)} x (t) dt: 
\end{align*}
\begin{align*}
 y'_n (s) &= -n e^{-n (s-s)} x (s) + n^2  \int \limits^{\infty}_{s}
 e^{-n (t-s)} x (t) dt \\ 
 &= -n x (s) + n y_n (s) 
\end{align*}

Comparing this with the general formula 
$$
\displaylines{\hfill 
 (A J_n x) (s) = n (( J_n - I ) x ) (s)\hfill \cr
 \text{or}\hfill A y_n (s) = n y_n (s) - n x (s)\hfill \cr 
 \text{we have}\hfill A y_n (s) = y'_n (s).\hspace{2.2cm}\hfill }
$$

For\pageoriginale $n > \beta,\mathfrak{W} (J_n) ~ = \mathscr{D} (A) $. So if $ y
\in \mathscr{D} (A) $, $ y' (s) $ exists and belongs to $C [0,\infty ]$
and  
$$
(Ay) (s) = y' (s).
$$ 

Conversely let $ y (s) $ and $ y' (s) $ both belong to $C [0,\infty ]
$; we shall show that $ y \in \mathscr{D} (A) $ and 
$ (Ay) (s) = y' (s) $. For define $ x (s) $ by 
$$
y' (s) - ny (s) = -nx (s).
$$

Putting $ ( J_n x ) (s) = y_n (s) $, we have, as shown above, 
$$
y'_n (s) - n y_n (s) = -n x (s).
$$

Writing $ \omega (s) = y(s) -y_n (s)$, we obtain
$$
\omega' (s) -n \omega (s) = 0 
$$ 
or $ \omega (s) = C e^{ns} $. But $ \omega (s) \in ~ C [ 0,
 \infty ] $ and this is possible only if $ C = 0 $. Hence $ y
(s) = y_n (s) \in \mathscr{D} (A) $ and so $ (Ay) (s) = y' (s) $. Thus
the domain of the infinitesimal generator $A$ is precisely the set of
functions $ y \in C [ 0, \infty ] $ and for such a function
$ Ay = y' $. We have thus characterized the differential operator $
\dfrac{d}{dt} $ as the infinitesimal generator of the semigroup
associated with the translation by $t$. 

II.~ In this we give the characterization of the second derivation as
 the infinitesimal generator of the semi-group associated with the
 Gaussian distribution. The space is $ C[-\infty, \infty ] $
 and 
 $$
 ( T_t x ) (s) =
 \begin{cases}
  \int \limits^{\infty}_{- \infty} \frac{1}{\sqrt{2 \pi t} }
  e^{-(s-v)^2/2t} x \,(v)\, dv ~\text{ if } ~ t > 0 \\ 
  x (s) \text{ if } ~ t = 0. \\
 \end{cases}
 $$ 

We\pageoriginale have
\begin{align*}
 y_n (s) = (J_n x) s &= \int \limits^{\infty}_{-\infty} x(v) ~ \left\{
 \int \limits^{\infty}_{0} \frac{n}{\sqrt{2 \pi t}} e^{-nt - (s-v)^2
  /2t} dt \right\} dv \\ 
 &= \int \limits^{\infty}_{-\infty} x(v) ~ \left\{ \int
 \limits^{\infty}_{0} \frac{2 \sqrt{n}}{\sqrt{2 \pi }} e^{-\sigma^2
  - (s-v)^2 n/2 \sigma^2} d \sigma \right\} dv ~\tag*{(\text{change}: $t =
 \sigma^2 / n$)}
 \end{align*} 
 
 Assuming for moment the formula 
 $$
 \int \limits^{\infty}_{0} e^{-(\sigma^2 + c/ \sigma^2)} d \sigma =
 \frac{\sqrt{\pi}}{2} e^{-2 c}, ~ c > 0, \text{ with } c = \sqrt{n}
 \frac{\mid s -v \mid}{\sqrt{2}}, 
 $$
 we get 
 \begin{align*}
  y_n (s) &= \int \limits^{\infty}_{-\infty} x(v) ~ \left( \sqrt{n}/2
  e^{-\sqrt{2 n} \mid s-v \mid}\right)\, dv \\ 
  &= \sqrt{n}/2 \int \limits^{\infty}_{-\infty} x(v) e^{-\sqrt{2 n}
   \mid s-v \mid} = \frac{\sqrt{2}}{2}~ \left(\int \limits^{s}_{-\infty}
  \cdots + \int \limits^{\infty}_{s} \cdots \right) 
 \end{align*} 
 $x (v)$ being continuous we can differentiate twice and we then obtain 
 \begin{align*}
  y'_n (s) &= n \left \{ \int \limits^{\infty}_{s} x (v) e^{-2
   \sqrt{n} (v-s)} dv - \int \limits^{s}_{-\infty} x (v) e^{-
   \sqrt{2 n} (s -v)} dv \right\} \\ 
  y''_n (s) &= n \left \{ -x (s) -x (s) + 2 \sqrt{n} \int
  \limits^{\infty}_{s} x (v) e^{-\sqrt{2}n (v-s)} dv\right.\\ 
  & \hspace{5cm} \left.+ \sqrt{2 n}
  \int \limits^{s}_{-\infty} x (v) e^{- \sqrt{2}n (s-v)} dv ~ \right
  \} \\ 
  &= -2n x (s) + 2n y_n (s).
 \end{align*} 
 
 Comparing this with the general formula
 \begin{align*}
  (A y_n) (s) = ( A J_n x ) (s) &= n \bigg\{ ( J_n - I ) x \bigg\}
  (s) \\ 
  &= n (y_n (s) - x (s))
 \end{align*} 
we find that $ A y_n (s) = \dfrac{1}{2} y''_n (s) $. For $ n > \beta,
\mathfrak{W} ( J_n ) = \mathscr{D} (A) $. Thus if $ y \in \mathscr{D}
(A), y'' (s) $ exists and belongs to $C [ - \infty, \infty ] $ and
further $ (Ay) (s) = \dfrac{1}{2} y'' (s)$. Conversely, let $y (s)$
and $y''(s)$ 
both belong to $ C [ - \infty, \infty ] $. Define $ x (s)
$ by 
$$
y'' (s) - 2 n y (s) = - 2 n x (s). 
$$ 
 
Putting\pageoriginale $y_n(s) = ( J_n x )(s) $, we have, as shown above, 
 $$
 y''_n (s) -2 n y_n (s) = -2 n x (s).
 $$
 
So, if $ \omega (s) ~= y_n (s) - y (s) $,
 $$
 \omega'' (s) -2n \omega (s) = 0.
 $$

This $ \omega (s) = C_1 e^{\sqrt{2 n} s} + C_2 e^{- \sqrt{2n} s}$.

This function cannot be bounded unless both $ C_1 $ and $ C_2 $ are
zero. Hence $ y (s) = y_n (s) $. So $ y (s) \in \mathscr{D} (A) $ and
$(Ay) (s) = \dfrac{1}{2} y'' (s) $. 

Thus the differential operator $ \dfrac{1}{2} \dfrac{d^2}{dt^2} $ is
the infinitesimal generator of the semi-group associated with the
Gaussian process. 

We now prove the formula
$$
\int \limits^{\infty}_{0} e^{-( \sigma^2 + c^2 / \sigma^2 )} d
\sigma = ~ \sqrt{^\pi}/_2 e^{-2c}, ~ c > 0. 
$$

We start with the formula
$$
\int \limits^{\infty}_{0} e^{-x^2} dx = ~ \sqrt{\pi} /2. 
$$

Putting $ x = \sigma - c/ \sigma $, we have
\begin{align*}
 \frac{\sqrt{\pi}}{2} &= \int \limits^{\infty}_{\sqrt{c}} e^{-(
  \sigma - c / \sigma )^2} ( 1 + c / \sigma^2 ) d \sigma \\ 
 &= e^{2 c} \int \limits^{\infty}_{\sqrt{c}} e^{-( \sigma^2 + c^2 /
  \sigma^2 )} ( 1 + c/ \sigma^2 ) d \sigma \\ 
 &= e^{2c} \left \{ \int \limits^{\infty}_{\sqrt{c}} e^{-( \sigma^2 +
  c^2 / \sigma^2)} d \sigma + \int \limits^{\infty}_{\sqrt{c}} e^{-(
  \sigma^2 + c^2 / \sigma^2 )} c / \sigma^2 d \sigma \right\}
\end{align*}

Setting $ \sigma = c / t $ in the last integral 
\begin{align*}
 \frac{\sqrt{\pi}}{2} &= e^{2c} \left\{ \int
 \limits^{\infty}_{\sqrt{c}} e^{- ( \sigma^2 + c^2 \sigma^2 )} d
 \sigma - \int \limits^{\infty}_{\sqrt{c}} e^{- ( c^2 / t^2 + t^2 )}
 dt \right.\\
 &= e^{2 c} \int \limits^{\infty}_{0} e^{( \sigma^2 + c^2 / \sigma^2
  )} d \sigma. 
\end{align*}
