\part[Application of the semi-group theory....]{Application of the semi-group theory\break to the Cauchy
 problem for the diffusion and wave\break equations}\label{chap18:p4}

\chapter{Lecture 18}\label{chap18} % LECTURE 18

\section{Cauchy problem for the diffusion equation}\label{chap18:sec1}

 Let\pageoriginale $R$ be a connected n-dimensional oriented Riemannian space with
 the metric 
$$
ds^2 = g_{i_j}(x) ~ dx^i ~ dx^j.
$$

Let $A$ be a second order linear partial differential operator in $R$
with $C^\infty$ coefficients: 
$$
(Af)(x) = b^{ij}(x) \frac{\partial^2 f}{\partial x^i ~ \partial x^j} +
a^i (x) \frac{\partial f} {\partial x^i} ~ (x) c(x) ~ f(x); 
$$
we assume that $b^{ij}$ is a symmetric contravariant tensor and
$a^i(x)$ satisfies the transformation rule 
$$
a^{-i} = a^k ~ \frac{\partial \bar{x}_i}{\partial x_k} + b^{kl} ~
\frac{\partial^2 \bar{x}^i}{\partial x^k \partial x^l} 
$$
$[(x_1,\ldots,x_n) \rightarrow (\bar{x}_1,\ldots,\bar{x}_n)]$ so that
the value $(Af)(x)$ is determined independent of the choice of the
local coordinates. We further assume that $A$ is \textit{elliptic} in
the strong sense that there exist positive constants $\mu$ and
$\lambda (0 < \lambda < \mu)$ such that 
$$
\mu ~ g^{ij}(x) ~ \xi_i ~ \xi_j \ge b^{ij}(x) ~ \xi_{i} ~ \xi_j
\geq \lambda  g^{ij}(x) \xi_i ~ \xi_j 
$$
for every real vector $(\xi_i, \dots,\xi_n)$ and every $x \in R$.

We consider the Cauchy problem in the large on $R$ for the diffusion
equation: to find $u(t,x) ~ (x \in R)$ such that 
\begin{equation}
 \begin{cases}
  \frac{\partial u}{\partial t} = A ~ u ~ (t,x), & t > 0\\
  u(0,x) = f(x), & f(x) ~\text{being a given function on}~ R.
 \end{cases}
 \tag{**}
\end{equation}

We\pageoriginale shall first give a rough sketch of our method of integration. We
wish to integrate the equation in a certain function space $L(R)$
which is a Banach space (i.e., we want to obtain $u(t,x)$ such that
$u(t,\ldots) \in L(R)$ for each $t \ge 0$); we assume that $L(R)$
contains $\mathscr{D}^\infty(R)$, the space of $C^\infty$ functions
with compact support, as a dense subspace. (Examples: $L_p(R), 1 \le p
< \infty; C(R)$ if $R$ is compact). We determine an additive operator
$A_o$ such that: $(i) ~ C^\infty (R) \supset \mathscr{D}(A_o) \supset
\mathscr{D}^\infty (R)$, if $f \in \mathscr{D}(A_o) ~ A_o f = Af.(ii)$
the smallest closed extension $\bar{A}_o$ of $A_o$ exists and
$\bar{A}_o$ is the infinitesimal generator of a semi group $T_t$ on
$L(R)$. We then have 
\begin{equation*}
 \begin{cases}
  D_t ~ T_t ~ f = s - \lim\limits_{h\rightarrow o} \frac{T_{t+h} -
   T_t ~ f}{h} = \bar{A}_o ~T_t ~ f (= T_t ~ \bar{A}_o ~ f), t \ge
  0\\ 
  T_o ~ f = f.
 \end{cases}
\end{equation*}

Thus $T_t f$ is a kind of solution of $(**)$. Next, we shall show
that, if the initial function $f(x)$ is prescribed suitably [e.g.,if
 $f \in \mathscr{D}^\infty (R)$ or more generally, if $A_o^k ~ f \in
 \mathscr{D}(A_o)$ for all integers $k \ge o$], there exists a
function $u(t, x) $ definitely differentiable in $t$ and $x$ such that $T_tf(x) =
u(t,x)$ almost everywhere in $(0,\infty] \times R$, the measure in $R$
being the one given by $\sqrt{g} ~ d ~ x_1,\ldots,d ~ x_n$, and
$u(t,x)$ will be a solution of $(**)$. 

In carrying out this procedure, we have to solve an equation of the
form $\left(u -\dfrac{A_o}{m}\right) ~ u = f, f$ is given and $u$ is
to be found 
from $\mathscr{D}(A_o)$. This is a kind of boundary value problem
connected with the elliptic differential operator $A$. 

\begin{theorem*}
 If\pageoriginale $R$ is compact, the equation
 \begin{equation*}
  \begin{cases}
   \frac{\partial u}{\partial t} = Au = b^{ij}(x) \frac{\partial^2
    u}{\partial x^i \partial x^j} + a^i (x) \frac{\partial
    u}{\partial x^i}, t>0\\ 
   u(0, x) = f(x) \in \mathscr{
    D}^\infty (R), (f(x) \text{ given })
  \end{cases}
 \end{equation*}
 admits of a solution $C^\infty$ in $(t,x)$. This solution can be
 represented in the form 
 $$
 u(t, x) = \int \limits_R P (t, x, dy) f(y)
 $$
 where $P(t, x, E)$ is the transition probability of a Markoff
 process on $R$. 
\end{theorem*}

The proof will be preceded by two lemmas.

We take for $L(R)$ the Banach space $C(R)$ of continuous functions
with $|| f || = \sup \limits_x | f(x)|. \mathscr{D}^\infty (R)$ is
dense in $L(R)$. The operator $A_\circ$ is defined as follows: 
$$
\mathscr{D} (A_\circ)=\mathscr{D}^\infty(R) \text{ and } A_\circ f =
Af \text{ for } f \in \mathscr{D}^\infty (R). 
$$

\setcounter{Lemma}{0}
\begin{Lemma}\label{chap18:sec1:lem1}
 For any $f \in \mathscr{D}^\infty (R)$ and and any $m > 0$, we have
 $$
 \displaylines{\hfill
 \max_x h (x) \geq f(x) \geq \min_{x \in R} h(x)\hfill \cr
 \text{where}\hfill h(x) = f(x)- \dfrac{(A_\circ f)(x)}{m}.\hfill }
 $$
\end{Lemma}

\begin{proof}
 Let $f(x)$ attain its maximum at $x_\circ$. We choose a local
 coordinate system at $x_\circ$ such that $b^{ij}(x_\circ)=
 \delta_{ij}$ (Kronecker delta). 
\end{proof}

(Such a choice is possible owing to the positive definiteness of\break
$b^{ij} \xi_i \xi_j$. Then 
\begin{align*}
 h(x_\circ ) & = f (x_\circ )-m^{-1}(A_\circ f) (x_\circ )\\
 & = f(x_\circ)-m^{-1} a^i (x_\circ) \frac{\partial f}{\partial
  x^i_\circ} - m^{-1} \Sigma^n_{i=1} \frac{\partial^2 f}{\partial
  (x^i_\circ )^2}\\ 
 & = f(x_\circ ).
\end{align*}
since\pageoriginale we have, at the maximum point $x_\circ$,
$$
\frac{\partial f}{\partial x^i_\circ}=0 \text{ and } \sum^n_{i=1}
\frac{\partial^2 f}{\partial x^{i^2}_\circ} \leq 0. 
$$ 

Thus $\max \limits_{x} h(x)\geq f(x)$. Similarly we have $f(x)\geq
\min \limits_x h(x)$. 

\begin{coro*}
 The inverse $( I -m^{-1} A_\circ)^{-1}$ exists for $m > 0$ and $||
 (I-m^{-1} A_\circ)^{-1}\break || \leq 1$. Further $((I-m^{-1} A_\circ
 )^{-1} h) (x)\geq 0$ if $h(x) \geq 0$. Also 
 $$
 (I -m^{-1}A_\circ)^{-1}. 1 =1.
 $$
\end{coro*}

\begin{Lemma}\label{chap18:sec1:lem2} %lemma 2
 The smallest closed extension $\bar{A}_\circ$ of $A_\circ$ exists.
\end{Lemma}

$\bar{A}_\circ f$ is defined and equal to $h$ if there exists a
sequence $\{f_k\} \subset \mathscr{D}^\infty (R)$ such that $s-\lim
\limits_{k \to \infty}f_k = f$ and $s-\lim \limits_{k \to
 \infty}A_\circ f_k = h$. 

$\bar{A}_\circ f$ is determined uniquely by $f$. For if $\{f_k\}
\subset \mathscr{D}^\infty (R)$ be such that $\lim \limits_{k \to
 \infty}f_k =0$ and $\lim \limits_{k \to \infty} A_\circ f_k =h$,
then we must $h = 0$. 

For by Green's integral theorem, $R$ being compact,
$$
\int \limits_R f_k A^* g dx = \int \limits_R g A f_k dx,
$$
for every $g \in \mathscr{D}^\infty (R)$ so that, in the limit,
 $$
 0 = \int \limits_R g h d x, \text{ for every }g \in
 \mathscr{D}^\infty (R); \text{ so } h =0. 
 $$
 
To prove that the resolvent $(I - m^{-1}A_\circ )^{-1}$ exists as a
linear operator in $C (R)$, for $m$ large, it will be sufficient to
show, in view of the Corollary to Lemma \ref{chap18:sec1:lem1} and the fact that
$\bar{A}_\circ$ is closed, that the range of $(I -m^{-1} A_\circ)$ is
dense in $C(R)$. We shall show that for any $h \in \mathscr{D}^\infty
(R)$ we can find $f \in \mathscr{D}(R)$ such that $(I -m^{-1} A_\circ)
f= h$ ($m$ large). To this purpose, we need 

\section{G\r{a}rding's inequality}\label{chap18:sec2} 
 
For\pageoriginale $u, v \in \mathscr{D}^\infty (R)$, define
\begin{align*}
 (u, v)_0 & = \int \limits_R u v dx \qquad (|| u ||^2_\circ = (u,
 u)_\circ)\\ 
 (u, v)_1 & = (u, v)_0 + \int \limits_R g^{ij} \frac {\partial
  u}{\partial x^i}\frac {\partial v}{\partial x^j} dx~(|| u ||^2_1 =
 (u, u)_1) 
 \end{align*} 
 
Then there exists $\gamma > 0$ and $\delta > 0$ such that for all
sufficiently large $m > 0$, 
 $$
 B' (u, v) = \left(\left(I- \dfrac{A*}{m}\right) u, v\right)
 $$
 satisfies
\begin{align*}
 | B' (u, v)| & \leq \gamma || u ||_1 ||v||_1 \\
 \delta || u ||^2_1 & \leq B' (u, u) ~\text{ for all }~ u, v \in
 \mathscr{D}^\infty (R). 
\end{align*}
 
This lemma can be proved by partial integration.
 
Let $H_\circ$ be the Hilbert space of square summable functions in
$R$. We have $\mathscr{D}^\infty (R) \subset H_\circ (R)$. Let $A_1$
be the operator in $H_\circ$ with domain $\mathscr{D}^\infty (R)$
defined by: $A_1 f = A_\circ f, f \in \mathscr{D}^\infty (R)$. As in
Lemma \ref{chap18:sec1:lem2}, the closure of $A_1$ in $H_0, \bar{A}_1$, exists. We show
now that the range of $(I - \dfrac{A_1}{m})$ is dense in $H_\circ$,
for $m$ large. If $(I - \dfrac{A}{m}) \mathscr{D}^\infty$ were not
dense in $H_\circ$, there will exists an element $f \neq 0$ in
$H_\circ$ which will be orthogonal to $(I - \dfrac{A}{m})
\mathscr{D}^\infty$. This mean that $f$ is a weak solution of $\left(I -
\dfrac{A*}{m}\right) f=0$. 
 
By the Weyl-Schwartz theorem, $f$ may be considered to be in\break
$\mathscr{D}^\infty(R)$. By G\r{a}rding's inequality, assuming $m$ to be
sufficiently large, 
$$
\delta || f ||^2_1 \leq \left(I - \frac{A^*}{m}f, f\right)=0. \quad
\text{ So }~f=0. 
$$
 
Since the range of $\left(I - \dfrac{A_1}{m}\right)$ is everywhere dense in
$H_\circ, \left(I -\dfrac {\bar{A}_1}{m}\right)^{-1}$ is defined everywhere in
$H_\circ$. So for every $h \in \mathscr{D}^\infty (R)$, we can find
$f_\circ \in  H_\circ 
$ such that $f_\circ$ is a weak solution of $\left(I -
\dfrac{A}{m}\right) f = h$. 
 
 Again\pageoriginale by the Weyl-Schwartz theorem, $f$ will be in
 $\mathscr{D}^\infty (R)$. Thus we see that for large $m$ the
 resolvent $J_m = \left(I - \dfrac{\bar{A}_\circ}{m}\right)^{-1}$
 exists as a 
 linear operator on $L(R)$ and satisfies $|| J_m || \leq 1$ (also,
 $(J_m h) (x) \geq 0$ if $h(x) \geq 0;~J_m.1 =1$). Consequently, (see
 Lecture 8) $\bar{A}_\circ$ is the infinitesimal generator of a
 uniquely determined semi-group 
 $$
 T_t = \exp (t \bar{A}_\circ) = s-\lim \exp (t m (J_m - I)).
 $$
 
 We have further
 $$
 || T_t || \leq 1, ~(T_t f) (x) \geq 0 \text{ if } f(x) \geq 0,~ T_t.1 =1.
 $$
 
 If $f \in \mathscr{D}^\infty (R)$, we have
 \begin{align*}
  D_t T_t f & = \bar{A}_\circ T_t f=T_t \bar{A}_\circ f= T_t A_\circ f \\
  D^2_t T_t f & = \bar{A}_\circ T_t A_\circ f=T_t A^2_\circ f \\
  &: \qquad: \\
  D^k_t T_t f & = T_t A^k_\circ f,\\
 \end{align*}
 for $k \geq 0$, since $A^k_\circ f \in \mathscr{D}^\infty (R)$ for
 integral $k \geq 0$. By making use of the strong continuity of $T_t$
 in $t$ we see that $(D^2_t + \bar{A})^k T_t f$ is locally square
 summable on the product space $(0, \infty) \times R$. Since
 $(\dfrac{\partial^2}{\partial t^2}+A)^k$ is an elliptic operator, it
 follows the Friedrichs-Lax-Nirenberg theorem that $(T_t f)(x)$ is
 almost everywhere equal to a function $u (t, x)$ indefinitely
 differentiable in $(t, x)$ for $t \geq 0$. 
 
\noindent \textbf{Proof of the latter part of the theorem:}
 
 $$
 | u (t, x)| = |(T_t f) x| \leq || T_t f || \leq || f ||
 $$
 
 Hence $u (t, x)$ is, for fixed $(t,x)$ a linear functional of $f \in
 L(R)$. 
 
Therefore there exists $P(t, x, E)$ such that
$$
u(t, x) = \int \limits_R P (t, x, dy) f(y).
$$
The non-negativity of $u(t, x)$ for $f(x) \geq 0$ implies that $P(t,x,
E)$ is $\geq 0$. Since $T_t 1=1$, we must have $P(t, x, R)=1$. 
