\chapter{Lecture 2}\label{chap2}

\section{Linear operators}\label{chap2:sec1}\pageoriginale

\begin{defi*}%def
 An additive operator $T$ from a normed linear space $X$ into a
 normed linear space $Y$ whose domain $\mathscr{D}(T)$ is the whole
 space $X$ and which is continuous is called a {\em linear operator}
 from $X$ to $Y$. The \textit{norm }$|| T ||$ of a linear operator is
 by definition: $|| T || = || T ||_X = \sup\limits_{x \in X, || x ||
  \le 1} || Tx ||$. If $Y$ is the real or complex numbers (according
 as $X$ is a real or a complex linear space) the linear operator $T$
 is called a {\em linear functional} on $X$. 
\end{defi*}

So far we have proved the existence of non-trivial linear
functionals. We shall prove the Hahn-Banach extension theorem which
will have as a consequence the existence of many linear functionals on
a normed linear space. 

\section{Hahn-Banach lemma}\label{chap2:sec2}

\begin{defi*}%def
 Let $X$ be a linear space (over real or complex numbers). A real
 valued function $p$ on $X$ will be called a {\em semi-group}(or a
 sub-additive functional) if it satisfies the following conditions: 
 \begin{enumerate}[\rm i)]
 \item $p(\alpha x) = | \alpha | p(x)$, for each $\alpha \in K$ and
  $x \in X$. 
 \item $p(x + y) \le p(x) + p(y)$ for all $x, y \in X$.

  Note that these conditions imply that $p(x) \ge 0$ for all $x \in X$.
 \end{enumerate}
\end{defi*}

\section{Lemma (Hahn-Banach)}\label{chap2:sec3}

Let $X$ be a real linear space and $p$ a semi-norm on $X$. Let $M$ be
a (real) subspace of $X$ and $f$ a real additive functional on $M$
such that $f(x) \le p(x)$ for all $x \in M$. Then there exists a real\pageoriginale
additive functional $F$ on $X$ such that $F$ is an extension of
$f$(i.e., $F(x) = f(x)$ for $ x \in M)$ and $F(x) \le p(x)$ for all $x
\in X$. 

\begin{proof}
 By the application of Zorn's lemma or transfinite induction, it is
 enough to prove the lemma when $X$ is spanned by $M$ and an element
 $x_0 \notin M$, i.e., when 
 $$
 X = \{ M, x_0\} = \{x | x \in X, x =m + \alpha x_0, m \in M, \alpha
 \text{ real }, x_0 \notin M \}.  
 $$

 The representation of an element $x \in X$ in the form $x = m +
 \alpha x_0, (m \in M, \alpha ~real)$ is unique. It follows that if,
 for any real number $c$, we define 
 $$
 F(x) = f(m) + \alpha c, 
 $$
 then $F(x) $ is an additive functional on $X$ which is an extension
 of $f(x)$. We have now to choose $c$ in such a way that $F(x)\le
 p(x), x \in X$, i.e., 
 $$
 f(m) + \alpha c \le p(m + \alpha x_0). 
 $$

 This condition is equivalent to the following two conditions: 
 $$
 \begin{cases}
  f\left(\frac{m}{\alpha}\right)+ c \le p \left(\frac{m}{\alpha} +
  x_0\right)
  &\text{for} \quad \alpha > 0\\[5pt]
  f\left(\frac{m}{-\alpha}\right) - c \le p \left(\frac{m}{-\alpha} -
  x_0\right) &\text{for} \quad \alpha < 0. 
 \end{cases}
 $$
 
 To satisfy these conditions, we shall choose $c$ such that
 $$
 f(m') - p(m' - x_0) \le c \le p(m'' + x_0) - f(m'')
 $$
 for all $m', m'' \in M$. Such a choice of $c$ is possible since
 \begin{align*}
  f(m') + f(m'') & = f(m' + m'')\\
  & \le p(m' + m'')\\
  & = p(m' - x_0 + m'' + x_0)\\
  &\le p(m' - x_0) + p(m'' + x_0).
 \end{align*}
 So \quad $f(m') - p(m' - x_0) \le p(m'' + x_0) - f(m''), m', m'' \in M$. 
 
 So\pageoriginale
 $$
 \sup_{m' \in M} \left\{f(m') - p(m' -x_0) \right\} \le \inf_{m'' \in
  M} \left\{ p(m'' + x_0) - f(m'') \right\} 
 $$
 and we can choose for $c$ any number in between.
\end{proof}

\section[Hahn-Banach extension theorem...]{Hahn-Banach extension theorem for real normed linear spaces}\label{chap2:sec4}

\begin{theorem*}%thm
 Let $X$ be a real normed linear space and $M$ a real subspace of
 $X$. Given a (real) linear functional $f$ on $M$, we can extend $f$
 to a (real) linear functional on the whole space $X$ in such a way
 that the norm is preserved: 
 $$
 || F || = || F ||_X = || f ||_M.
 $$
\end{theorem*}

\begin{proof}
 Take $p(x) = || f||_M ||x||$ in the Hahn-Banach lemma. We have
 $f(x) \le p(x)$ on $M$ and $p(x)$ is subadditive. We then have an
 additive functional $F(x)$ on $X$ which is an extension of $f$ with
 $F(x) \le || f ||_M || x ||$ for all $x \in X$. Also $-F(x) = F(-x)
 \le ||f||_M ||-x|| = || f||_M ||x||$. Hence 
 $$
 | F(x) | \le ||f||_M ||x||. 
 $$
 
 This shows that $F$ is a linear functional on $X$ and $|| F ||_X \le
 || f ||_M$. The reverse inequality, $|| F ||_X \ge || f ||_M$, is
 trivial as $F$ is an extension of $f$. 
\end{proof}

\section[Hahn-Banach extension theorem for...]{Hahn-Banach extension theorem for complex\hfill\break normed linear
 spaces (Bohnenblust-Sobczyk)}\label{chap2:sec5} 

\begin{theorem*}%thm
 Let $X$ be a complex normed linear space and $M$ a (complex)
 subspace. Given a complex linear functional $f$ on $M$ we can extend
 $f$ to a complex linear functional $F$ on $X$ in such a way that
 $||F||_X = ||f||_M$. 
\end{theorem*}

\begin{proof}
 A\pageoriginale complex normed linear space becomes a real normed linear space if
 scalar multiplication is restricted to real numbers and the real and
 imaginary parts of a complex linear functional are real linear
 functionals. If $f(x) = g(x) + ih(x)~ (g(x), h(x) \text{ real })$,
 $g$ and $h$ are real linear functionals on $M$ and $|| g ||_M \le
 ||f||_M$, $||h||_M \le ||f||_M$. Since, for each $x \in M$, 
 \begin{align*}
  g(ix) + ih(ix) & = f(ix)\\
  & = if(x)\\
  & = i(g(x) + ih(x))\\
  & = -h (x) + ig(x),
 \end{align*}
 we have $h(x) = -g(ix)$, for $x \in M $. 
 
 By the Hahn-Banach theorem for real linear spaces $g$ can be
 extended to a real linear functional $G$ on $X$ with the property
 $|| G ||_X = || g ||_M$. Now define 
 $$
 F(x) =G(x) - iG(ix).
 $$
 $F$ is then a complex linear functional on $X$. (For complex additivity notice that 
 $$
 F (ix) = G(ix ) - iG (-x ) = G(ix ) + iG(z) = iF (x )).
 $$
 $F$ is an extension of $f$; for, if $x \in M$, 
 $$
 F(x) = G(x) - iG(ix) = g(x) - ig(ix) = g(x) + ih(x) = f(x).
 $$
 
 We have now only to show that the norm is not changed. For this,
 writes, for $x \in X, F(x) = re^{ie}$. Then $E^{-i\theta}F(x)$ is
 real. So 
 \begin{align*}
  |F(x)| & = |e^{-i\theta} F(x)| = |F(e^{-i\theta} x)|\\
  & = |G(e^{-i\theta} x)| \quad (=\text{ since }~ e^{-i\theta} F(x)
  \text{is real}).\\ 
  & \le || G ||~ || e^{-i\theta} x||\\
  & = || g ||_M ||x||\\
  & \le ||f ||_M x. 
 \end{align*}
 
 So\pageoriginale $|| F ||_X \le ||f||_M$ and the reverse inequality holds since
 $F$ is an extension of $f$. 
\end{proof}

\section{Existence of non-trivial linear functionals}\label{chap2:sec6}

We consider some consequences of the Hahn-Banach extension theorem; we
prove the existence of plenty of linear functionals on a normed linear
space. 

\begin{prop*}%proposition
 Let $X$ be a normed linear (real or complex) and $x_o \neq 0$ be an
 elements of $X$. Then there exists a linear functional $f_o$ on $X$
 such that $f_o$ on $X$ such that $f_o (x_o)= || x_o || $ and $||
 f_o || =1$. 
\end{prop*}

\begin{proof}
 Let $M$ be the subspace spanned by $x_o$, i.e., $M= \{ x | x=
 \alpha x_o$ for some number $\alpha \}$. Define $f(x)= \alpha
 ||x_o ||$ for $x = \alpha x_o \in M$. This is a linear functional
 on $M$ and $|| f ||_M=1$. By the Hahn-Banach extension theorem there
 exists a linear functional $f_o$ on $X$ which extends $f$ in such a
 way that $|| f_o|| = || f ||_M=1$; $f_o (x_o)= f(x_o) = ||x_o ||$. 
\end{proof}

\begin{remark*}
 For a pre-Hilbert space the existence of such a linear functional is
 evident; we may take $f_o(x)= (x, \dfrac{x_o}{|| x_o ||})$. The
 additivity of $f_o$ follows from the homogeneity and distributivity
 of the scalar product. The continuity of $f_o$ is a consequence of
 Schwarz's' inequality. 
\end{remark*}

\begin{prop*}
 Let $X$ be a normed linear space. Let $M$ be a subspace and $x_o$
 an element $X$ such that $d= \inf\limits_{m \in M} || x_o -m ||
 >0$. Then there exists a linear functional $f_o$ on $X$ such that
 $f_o (x) =0$ for every $x \in M$ and $f_o (x_o)=1$. 
\end{prop*}

\begin{proof}
 Let $M_\circ = \{ x | x=m + \alpha x_\circ , m \in M \}$. Define $f(x)=
 \alpha$ for $x= m+ \alpha x_\circ \in M_\circ(m \in M)$. $f$ is
 additive on $M_\circ$, 
 vanishes on $M$ and $f(x_\circ)=1$. Also $f$ is continuous on $M_\circ$: if
 $\alpha \neq 0$, then 

 $x =m + \alpha x_\circ \neq 0 (m \in M)$,\pageoriginale and 
 \begin{align*}
  |f (x)| = |\alpha| & = \alpha ||x|| \big / || x|| \\
  & =|\alpha| ||x|| \big/ ||m+ \alpha x_o || \\
  & =||x|| \big / ||x_o -(-m/ \alpha) ||\\
  & \le d^{-1} ||x|| (-m/\alpha \in M);
 \end{align*}
 if $\alpha =0, f (x) = 0$ and the inequality $|f (x)| \le d^{-1}
 ||x||$ is still valid. If $f_o$ is a linear functional on $X$ which
 is an extension of $f$, then $f_o$ satisfies the requirements of the
 proposition. 
\end{proof}

\section[Orthogonal projection and...]{Orthogonal projection and the Riesz representation theorem}\label{chap2:sec7}

\begin{defi*} % def 
 Let $x$ and $y$ be two elements of a pre-Hilbert space $X$; we say
 that $x$ is orthogonal to $y$ (written $x \perp y$) if $(x,y) =
 0$. If $x \perp y$ then $y \perp x$; if $x \perp x$, then $x=0$. 
\end{defi*}

Let $M$ be a subset of a pre-Hilbert space; we denote by $M^\perp$ the
set of elements $x \in X$ such that $x \perp y$ for every $y
\in M$. 

\begin{theorem*}
 Let $M$ be a closed liner subspace of a Hilbert space $X$. Then any
 $x_o \in X$ can be decomposed uniquely in the form $x_o =
 m+n, m \in M, n \in M^\perp$. ($m$ is called the
 {\em{orthogonal projection}} of $x_o$ on $M$ and is denoted by $P_M
 x_\circ$). 
\end{theorem*}

\begin{proof}
 The uniqueness of the decomposition is clear from the fact that an
 element orthogonal to itself is zero. To prove the existence of the
 decomposition we may assume $M \neq X$ and $x_o \notin M$ (if $x_o
 \in M$ we have the trivial decomposition with $n =0$). Let
 $d = \inf\limits_{m \in M} || x_o - m||$; since $M$ is
 closed and $x_o \notin M,d > 0 $. Let $\{m_k\} \subset M$ be a
 minimizing sequence, i.e., $\lim\limits_{k \to \infty} || x_o -m_k
 || = d$. $\{m_k\} $ is a Cauchy sequence; for   
 \begin{align*}
  || m_k - m_n ||^2 & = || (x_o - m_n) - (x_o - m_k) ||^2 \\
  & = 2 (|| x_o - m_n||^2 +|| x_o - m_k||^2) - ||2 x_o - m_k - m_n||
 ^2 \tag{(Euclidean property )} \\ 
  & = 2 (|| x_o - m_n ||^2 + ||x_o- m_k||^2) - 4 || x_o -
  \frac{m_k + m_n}{2}||^2\\ 
  &\leq 2 (|| x_o - m_n ||^2 + ||x_o- m_k||^2 - 4d^2 (\text{ as
  }\frac{m_k + m_n}{2} \in M)\\ 
  & \to 2 (d^2 + d^2 ) - 4d^2 = \text{ as } m, n \to \infty. 
 \end{align*}

 By\pageoriginale the completeness of the Hilbert space there exists and element $m
 \in X$ with $\lim\limits_ { k \to \infty}|| m - m_k|| = 0 $;
 in fact $m \in M$, as $M$ is closed. Also $||x_o - m || = d$. Write
 $x_o = m + (x_o - m)$. Putting $n = x_0 - m$ we have to show that $
 n \in M^\perp$. Let $m'\in M$. Since, for any real $\alpha, m +
 \alpha m' \in M$ we have $d^2 \leq || x_o - m - \alpha
 m'||^2 = || n - \alpha m'||^2 = (n - \alpha m ', n - \alpha m ' )$ 
 $$
 = ||n ||^2 - \alpha (n, m') - \alpha (m', n ) + \alpha^2 ||m'||^2. 
 $$ 
 
 Since $||n||^2 = d^2 $, this gives, for any real $\alpha$, 
 $$
 0 \leq -2 \alpha \mathscr{R}(n, m')+ \alpha^2 ||m'||^2. 
 $$
 
 So $\mathscr{R} (n, m') = 0 $ for every $m ' \in
 M$. Replacing $m'$ by $im'$ we have $\mathscr{I} m\, (n, m') = 0$, for
 every $m ' \in M$. Thus $(n, m') = 0$ for each $m '
 \in M$. 
\end{proof}

\begin{remark*}
 If $x_o \notin M$, then $n \neq 0$ and $f_o (x) = (x,
 \dfrac{n}{||n||^2})$ satisfies the conditions of the last
 proposition. 
\end{remark*}

\begin{theorem*}[Riesz]
 Let $X$ be a Hilbert space and $f$ a linear functional on $X$. Then
 there exists a unique element $y_f$ of $X$ such that 
 $$
 f(x) = (x, y_f )
 $$
 for every $x \in X$.
\end{theorem*}

\begin{proof}
 \textbf{ Uniqueness:}\pageoriginale If $(x, y_1) = (x, y_2)$ for every $x, (x,
 y_1- y_2) = 0$ for every $x$; choosing $x=y_1-y_2$ we find
 $y_1-y_2=0$. 
 
 \textbf{Existence:}
 Let $M$ be the zero manifold of $f$, i,e,., $M = \{x | f(x) =
 0\}$. Since $f$ is additive, $M$ is a linear subspace and since $f$
 is continuous $M$ is closed. The theorem is evident if $M =
 X$. i.e., if $f(x) = 0 $ on $X$; in this case we need only take $y_f
 = 0$. So suppose $M \neq X$. Then there exists, by the last theorem,
 an element $y_0 \neq 0$ such that $y_o$ is orthogonal to every
 element of $M$. Define 
 $$
 y_f = \frac{\overline{ f (y_o)}}{|| y_o ||^2} Y_o. 
 $$
 $y_f$ meets the condition of the theorem. First, for $x \in
 M, f(x) = (x, y_f)$ since $f(x) = 0 $ for $x \in M$ and
 $y_f \in M^\perp$. For elements $x$ of the form $x = \alpha y_0$. 
 \begin{align*}
  (x, y_f ) & = (\alpha y_o, y_f) = \left(\alpha, \frac{\overline{
    f(y_0)}}{||y_0||^2} y_0\right)\\ 
  & = \alpha f(y_o) = f(\alpha y_o)\\
  & = f(x). 
 \end{align*}
 
 Since $f$ is linear and $(x, y_f)$ is linear and $(x, y_f)$ is
 linear in $x$, to show that $f(x) = (x, y_f)$ for each $x
 \in X $ it is enough to show that $X$ is spanned by $M$
 and $y_o$. If $x \in X$, write, noting that $f(y_f) \neq 0$, 
 $$
 x = \frac{f(x)}{f(y_f)} y_f + \left( x - \frac{f(x)}{f(y_f)} y_f \right). 
 $$
 $\dfrac{f(x)}{f(y_f)} f_y$ is of the form $\alpha y_o$. The second
 term is an element of $M$, since $ 
 f\left(x - \dfrac{f(x)}{f(y_f)} y_f\right) =  f(x) -
 \dfrac{f(x)}{f(y_f)} y_f = 0 $. 
\end{proof} 

\begin{remark*}
 $$
 || f || = || y_f ||. 
 $$
\end{remark*} 
