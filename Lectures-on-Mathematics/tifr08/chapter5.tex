\chapter{Lecture 5}\label{chap5}

\section{Some examples of semi-groups}\label{chap5:sec1}\pageoriginale

\begin{enumerate}[I]
\item In $C[o, \infty]$ [ the space of bounded uniformly continuous
 functions on the closed interval $[0, \infty]$] define $
 \big\{T_t\big\}_{t \geq 0}$ by 
 $$
 (T_t x)(s) = x(t +s)~ (x \in C). 
 $$
 $\big\{ T_t \big\}$ is a semi-group. Condition (1) is trivially
 verified. (2) follows from the uniform continuity of $x$, as 
 $$
 || T_t x - T_{t_o} x || = \sup_{s \geq 0} | x (t + s ) - x (t_o + s )|.
 $$

 Finally $|| T_t || = 1 $ and so $(3)$ is satisfied with $\beta = 0$. 
 
 In this example, we could replace $C[ 0, \infty]$ by $C[ -\infty,
  \infty] $. 
\item On the space $C [ o, \infty]$ (or $C[ -\infty, \infty]$),
 define $\big\{T_t \big\} t \geq 0$ 
 $$
 (T_t x ) (s) = e^{\beta t} x(s)
 $$
 where $\beta$ is a fixed non-negative number. Again (1) is
 trivial; for (2) we have $|| T_t x - T_{t_\circ} x || = \big|
 e^{\beta t } - e^{ \beta t_o}\big| \sup_s | x(s) | $. Trivially $||
 T_t || = e^{ \beta t}$. 
\item Consider the space $C[ -\infty, \infty]$. Let 
 $$
 N_t (u) = \frac{1}{\sqrt{ 2 \pi t}} e^{-u^{2}/2t}, =\infty < u <
 \infty, t > 0, 
 $$
 (the normal probability density). Define $\big\{ T_t \big\}_{t\geq
  0}$ on $C[ -\infty, \infty]$ by: 
 $$
 (T_t x )(s) = 
 \begin{cases}
  ~\int\limits_{-\infty}^{ \infty} N_t (s - u) x(u) du, &\text { for
  } t > 0\\ 
  ~x(s) & \text{ for } t = 0
 \end{cases}
 $$
 
 Each\pageoriginale $T_t$ is continuous:
 $$
 || T_t x || \leq || x || \int\limits^\infty_{- \infty} N_t (s - u )
 du = || x||, \text{ as } \int\limits^\infty_{- \infty} N_t (s - u )
 du = 1. 
 $$

 Moreover it follows from this that condition $(3)$ is valid with
 $\beta = 0$. By definition $T_o = I$ and the semi-group property
 $T_t T_s = T_{ t+s}$ is a consequence of the well -known formula
 concerning the Gaussian distribution. 
 $$
 \frac{1}{\sqrt{ 2 \pi (t +t ')}} e^{ -u^2 / 2 (t + t')} =
 \frac{1}{\sqrt{ 2 \pi t}} \frac{1}{\sqrt{ 2 \pi t'}} \int\limits_{ -
  \infty}^{ \infty} e^{\frac{-( u -v )^2}{2t} e^{ \frac{- v^2}{2t'}}}dv. 
 $$
 
 (Apply Fubini's theorem). To prove the strong continuity, consider
 $t, t_o > 0$ with $t \neq t_0$. (The case $t_o = 0$) is treated in a
 similar fashion). By definition 
 $$
 (T_t x )(s) - (T_{ t_o }x)(s) = \int\limits^{\infty}_{
  -\infty}\bigg\{ N_t (s-u) x(u) - N_{t_o}(s-u) x(u) \bigg\} du. 
 $$
 
 The integral $\int\limits^{ \infty}_{-\infty} \dfrac{1}{\sqrt{2
   \pi t}} e^ {-(s - u)^2 /2t} x(u) du$ becomes, by the change of
 variable $\dfrac{s-u}{\sqrt{t}} = z$, $\dfrac{1}{\sqrt{2 \pi}}
 \int\limits^{\infty}_{ - \infty} e^{- z^2 / 2} x ( s - \sqrt{t}z) dz$. Hence 
 $$
 (T_t x)(s) - (T_{t_0} x) ~
 (s)) = \int\limits_{-\infty}^{\infty} N_1 (z)\left\{ x\left(-s
 \sqrt{tz}\right)- x (s - \sqrt{t_o z} )\right\} dz~ x(s)
 $$ 
 being uniformly continuous
 on $-\infty,\infty$, for any $\varepsilon > 0$ there exists a
 number $\delta = \delta (\varepsilon) > 0$ such that $| x(s_1) -
 x(s_2) | \leq \varepsilon $ whenever $|s_1 - s_2|\leq \delta$. Now,
 splitting the last integral 
 \begin{align*}
  | (T_t x ) (s) & - (T_{t_\circ} x )(s)| \\
  & \leq \int\limits_{|\sqrt{tz}- \sqrt{t_o} z| \leq \delta} N_1
  (z) | x (s - \sqrt{tz}) - x (s - \sqrt{t_\circ z}dz\\ 
  & \hspace{4cm} +
  \int\limits_{|\sqrt{tz} - \sqrt{t_\circ}z |> \delta} N_1 (z)(\ldots)
  dz\\ 
  & \leq \mathcal{E} \int N_1 (z) dz + 2 ||x|| \int\limits_{ |
   \sqrt{tz} - \sqrt{t_o}z> \delta} N_1 (z) dz\\ 
  & = \mathcal{E} + 2 || x|| \int\limits_{|z| > |
   \frac{\delta}{\sqrt{t}-{\sqrt{t_0}}|}} 
 \end{align*}
 
 The\pageoriginale second term on the right tends to $0$ as $|t -t_o| \to 0$,
 because the integral $\int\limits_{ -\infty}^{ \infty} N_1 (z) dz$
 converges. Thus 
 $$
 \varlimsup_{ t \to t_o} \sup_{ -\infty < s < \infty}| (T_t x)(s) -
 (T_{t_0} x)(s)| \leq \mathcal{E}. 
 $$

 Since $\mathcal{E} > 0$ was arbitrary, we have proved the strong
 continuity at $ t =t_o$ of $T_t$. 
 
 In this example we can also replace $C[ o, \infty]$ by $L_p[ o,
  \infty]\, 1 \leq p < \infty$. Consider, for example $L_1[ o,
  \infty]$. In this case, $||T_t x|| \leq \int\limits^{\infty}_{ -
  \infty} \bigg[ \int\limits^{\infty}_{-\infty} N_t\break (s-u) |x(u)|
  ds\bigg] du \leq || x ||$, applying Fubini's theorem. 
 
 As for the strong continuity, we have 
 \begin{align*}
  (T_t x)(s) & - (T_{t_o} x) (s)||\\ 
  &= \int\limits^{ \infty}_{ -
   \infty}\bigg| \int\limits^{ \infty}_{ - \infty} N_1 (z) \left\{ x
  (s - \sqrt{tz}) - x (s - \sqrt{t_o z})\right\} dz | ds \\ 
  & \leq \int\limits^\infty_{-\infty} N_1 (z) \left[
   ~\int\limits^{ \infty}_{ - \infty} |x(s -\sqrt{t}z)-x( s -
   \sqrt{t_o} z)| ds \right]dz 
 \end{align*}
 
 Since $N_1 (z) \int\limits^{ \infty}_{ -\infty} | x(s - \sqrt{t}z )
 - x (s - \sqrt{t_o} z) | ds \leq 2 || x || N_1 (z)$, we may apply
 Lebesgue's dominated convergence theorem. We then have 
 \begin{multline*}
 \varlimsup_{ t \to t_\circ} || (T_t x) (S) - (T_{t_0} x ) (S)||\\
 \int\limits^{\infty }_{- \infty}N_1 (z) \left\{ \varlimsup_{ t \to
  t_o} \int\limits_{-\infty}^{\infty} | x( - \sqrt{t}z )- x(s -
 \sqrt{t_o} z)| ds \right\} dz = 0, 
 \end{multline*}
 by the continuity in mean of the Lebesgue integral. 
\item Consider $C[ - \infty, \infty]$. Let $\lambda > 0, \mu >
 0$. Define $\big\{T_t \big\}_{t \geq 0}$ 
 $$
 (T_t x )(s) = e^{- \lambda t}\sum^\infty_{ k = 0} \frac{(\lambda
  t)^k}{k!} x(s - k \mu ). 
 $$
 $\{ T_t \}$ is a semi-group. Strong continuity follows from: 
 $$
 \mid ( T_t X) (s) - T_{t_{0}} X) (s) \mid \leq \| x \|
 |e^{-\lambda t} \sum^{\infty}_{k=0} \frac{(\lambda t)^k}{k!}
 -e^{\lambda t_0} \sum^{\infty}_{k=0} \frac{(\lambda t_0)^k}{k !}
 | = 0. 
 $$
 (3)~\pageoriginale is satisfied with $ \beta = 0 $. To verify $ (1) $
 \begin{align*}
  (T_w (t_t x) ) (s) & = e^{\lambda w} \sum^{\infty}_{l=0} \frac{(
   \lambda w )^l}{l!} \left[ e^{- \lambda t} \sum^{\infty}_{k=0}
   \frac{( \lambda t)^k}{k!} f ( s-k \mu -1 \mu ) \right]\\ 
  &= e^{-\lambda ( w +t)} \sum^{\infty}_{p=0} \frac{1}{p!} \left[
   p! \sum^{p}_{1=0} \frac{(\lambda w)^1 ( \lambda t )^{p-1}}{1! ~
    p-1!} f (s-p \mu ) \right]\\ 
  &= e^{- \lambda ( w + t )} \sum^{\infty}_{p=0} \frac{1}{p!} (
  \lambda w + \lambda t)^p f ( \lambda + \lambda t)^p f ( s - p \mu
  ) \\ 
  &= ( T_{w+t} x) (s). \\
 \end{align*}
\end{enumerate}

\section{The infinitesimal generator of a semi-group}\label{chap5:sec2}

\begin{defi*}
 The \em{infinitesimal generator } $A$ of a semi-group $ T_t $ is
 defined by: 
 $$
 A x = s- \lim_{h \downarrow 0 } h^{-1} ( T_h - I) x, 
 $$
 i.e., as the additive operator $A$ whose domain is the set 
 \begin{align*}
 \mathscr{D} (A) &= \left\{ x \mid s- \lim \limits_{h \downarrow 0}
 h^{-1} (T_h- I ) x ~\text{exists }\right\} ~\text{ and for }~ 
 x \in \mathscr{D} (A),\\  
 A x & = s - \lim_{h \downarrow o} h^{-1} ( T_h -I ) x. 
 \end{align*}
\end{defi*} 

$ \mathscr{D} (A) $ is evidently non- empty; it contains at least
zero. Actually $ \mathscr{D} (A) $ is larger. We prove the 

\begin{prop*}
 $ \mathscr{D} (A) $ is dense in $X$ ( in the norm topology ).
\end{prop*}

\begin{proof}
 Let $ \varphi_n (s) = n e^{-ns} $. Introduce the linear operator $
 C_{\varphi_{n}} $ defined by 
 $$ 
 C_{\varphi_{n}} x = \int\limits^{\infty}_{0} \varphi_n (s) T_s x
 ds ~\text{ for } x \in X \text{ and } n > \beta, 
 $$
 the integral being taken in the sense of Riemann. (The ordinary
 procedure of defining the Riemann integral of a real or complex
 valued\pageoriginale func\-tions can be extended to a function with values in a
 Banach space, using the norm instead of absolute value ). The
 convergence of the integral is a consequence of the strong
 continuity of $ T_s $ in $s$ and the inequality, 
 $$
 \mid\mid \varphi_n (s) T_s x \mid\mid \leq n e^{( -n + \beta) s} \mid x \mid\mid.
 $$
\end{proof}

 The operator $ C_{\varphi_{n}} $ is a linear operator whose norm
 satisfies the inequality 
 $$
 \mid\mid \varphi_n \mid\mid \leq n ~ \int \limits^{\infty}_{0} e^{(
  -n + \beta) s} ds = 1/1 - \beta/n. 
 $$

We shall now show that $\mathfrak{W} ( C_{\varphi_{n}} ) \subseteq
\mathscr{D} (A) $ $ (\mathfrak{W} ( C_{\varphi_{n}} ) $ denotes the
range of $ C_{\varphi_{n}} ) $ for each $ n > \beta $ and that for
each $ x \in X $, $ s - \lim \limits_{n \rightarrow \infty} $  $
C_{\varphi_{n}} x = x $; then $ \bigcup \limits_{n >
 \beta}\mathfrak{W} C_{\varphi_{n}} ) $ will be dense in $X$ and
a-portion $\mathscr{D} (A)$ will be dense in $X$. We have 
$$
h^{-1} (T_h -I ) C_{\varphi_{n}} x = h^{-1} \int
\limits^{\infty}_{0} \varphi_n (s) T_h T_s x ds- h^{-1} \int
\limits^{\infty}_{0} \varphi_{n} (s) T_s x ds 
$$ 
(The change of the order $T_h \int \limits^{\infty}_{0} \cdot = \int
\limits^{\infty}_{0} T_h \cdots $ is justified, using the additivity
and the continuity of $ T_h $, by approximating the integral by
Riemann sums). Then 
\begin{align*}
 h^{-1} ( T_h - I) C_{\varphi_{n}} x &= h^{-1} \int
 \limits^{\infty}_{0} \varphi_n (s) T_{h+s} x ds -h^{-1} \int
 \limits^{\infty}_{0} \varphi_n (s) T_s x ds \\ 
 &= h^{-1} \int \limits^{\infty}_{0} \varphi_n (s-h) T_s x ds -
 h^{-1} ~ \int \limits^{\infty}_{0} \varphi_n (s) T_s x ds\\ 
 &\hspace{1.cm} \text{ ( by a change of variable in the first
  integral ).} \\ 
 &= h^{-1} \int\limits^{\infty}_{h} \{ \varphi_n (s -h) - \varphi_n
 (s) \} T_s x ds\\ 
 &= h^{-1} \int \limits^{h}_{0} \varphi_n (s) T_s x ds.
\end{align*}

By\pageoriginale the strong continuity of $ \varphi_n (s) T_s x $ in $s$, the
second term on the right converges strongly to $ - \varphi_n (0) T_0 x
= -nx $, as $ h \downarrow 0 $. 
\begin{align*}
 & h^{-1} \int \limits^{\infty}_{h} \{ \varphi_n (s-h) - \varphi_n
 (s) \} T_s x ds \\ 
 &= \int \limits^{\infty}_{h} - \varphi'_n ( s - \Theta h ) T_s x ds
 ~ ( 0 < \Theta < 1 ) \text{ ( by the mean value theorem )} \\ 
 &= \int \limits^{\infty}_{0} - \varphi'_n (s) T_s x ds +
 \int\limits^{h}_{0} \varphi'_n (s) T_s x ds + ~ \int
 \limits^{\infty}_{h} \{ \varphi'_n (s) - \varphi'_n ( s - \theta h )
 \} T_s x ds. 
\end{align*}

But, $ \int \limits^{h}_{0} \varphi'_n (s) T_s x ds \rightarrow 0 $
as $ h \downarrow 0 $ and 
\begin{align*}
 \mid\mid \int \limits^{\infty}_{h} & \left\{ \varphi'_n (s) - \varphi'_n
 ( s - \Theta h )\right\} T_s x ds \mid\mid \\ 
 &\leq n^2 ~ \int \limits^{\infty}_{h} \mid e^{-n ( s - \theta h )}
 -e^{-ns} \mid e^{\beta s} \mid\mid x \mid\mid ds \\ 
 &\leq n^2 ( e^{n \Theta h} -1 ) ~ \int \limits^{\infty}_{h} e^{(
  \beta-n) s} \mid\mid x \mid\mid ds \rightarrow 0 \text{ as } h
 \downarrow 0. ( \beta < n ). 
\end{align*}

Thus we have proved that $\mathfrak{W} ( C_{\varphi_{n}} ) \subseteq
\mathscr{D} (A) $ and 
 $$
\begin{fbox}
 { $A C_{\varphi_{n}} x = n ( C_{\varphi_{n}} - I ) x $}
\end{fbox}
$$
as $ \varphi'_n = -n \varphi_n $. Next, we show that $ s-\lim
\limits_{n \rightarrow \infty} C_{\varphi_{n}} (x) = x $ for each $
x \in X $. We observe that 
\begin{align*}
 C_{\varphi_{n}} x - x &= \int\limits^{\infty}_{0} n e^{-ns} T_s x
 ds - \int \limits^{\infty}_{0} n e^{-ns} x ds, ( \text{ as } \int
 \limits^{\infty}_{0} ne^{-ns} ds =1 ) \\ 
 &= n \int \limits^{\infty}_{0} e^{-ns} \big[ T_s x -x \big] ds.
\end{align*}

Approximating the integral by Riemann sums and using the triangle
inequality we have 
\begin{align*}
 \mid\mid C_{\varphi_{n}} x - x \mid\mid &\leq n ~ \int
 \limits^{\infty}_{0} e^{-ns} \mid\mid T_s x-x \mid\mid ds \\ 
 &= n \int \limits^{\delta}_{0} \cdots + n \int
 \limits^{\infty}_{\delta} \cdots, ~ \delta > 0 \\ 
 &= I_1 + I_2 , \text{ say }.
\end{align*}

Given\pageoriginale $ \mathcal{E} > 0 $, by strong continuity, we can choose a $
\delta > 0 $ such that $ \mid\mid T_s x - x \mid\mid < \mathcal{E} $
for $ 0 \leq s \leq \delta $; then 
$$
I_1 \leq \mathcal{E} n \int \limits^{\delta}_{0} e^{-ns} ds \leq
\mathcal{E} n ~ \int \limits^{\infty}_{0} e^{-ns} ds = \mathcal{E}.
$$

For a fixed $ \delta > 0 $, using the majorization condition in the
definition of a semi-group, 
\begin{gather*}
 I_2 \leq n ~ \int \limits^{\infty}_{\delta} e^{-ns} ( e^{\beta s} +
 1 ) \mid\mid x \mid\mid ds 
 = \mid\mid x \mid\mid \left[ n \frac{e^{( n + \beta )s}}{-n}
  \right]^{\infty}_{\delta} - \mid\mid x \mid\mid \left[ n
  \frac{e^{-ns}}{n} \right]^{\infty}_{\delta}
\end{gather*}

Each of the terms on the right tends to zero as $ n \rightarrow
\infty$. So $ I_2 \leq \mathcal{E} $, for $ n > n_0 $. Thus $
C_{\varphi_{n}} x \rightarrow x $ as $ n \rightarrow \infty $. 

\begin{remark*}
 That $ \mathscr{D} (A) $ is dense in $X$ can be proved more
 easily. But we need the considerations given in the above proof for
 later purpose. 
\end{remark*}

\begin{defi*}
 For $ x \in X $ define $ D_t T_t x $ by
 $$
 D_t T_t x = s - \lim_{h \rightarrow 0} h^{-1} ( T_{t+h} - T_t ) x
 $$
 if the limit exists.
\end{defi*}

\begin{prop*}
 If $ x \in \mathscr{D} (A) $ then $ x \in \mathscr{D} (D_t) $ and $
 D_t T_t x = A T_t x = T_t A x $. 
\end{prop*}

\noindent\textit{Proof.}
 If $ x \in \mathscr{D} (A) $, we have, since $T_t $ is a linear operator,
 \begin{align*}
  T_t A x &= T_t ~s-\lim_{h \downarrow 0} h^{-1} (T_h -I ) x \\
  &= s-\lim_{h \downarrow 0} h^{-1} ( T_t T_h - T_t ) x\\
  &= s-\lim_{h \downarrow 0} h^{-1} ( T_{t+h} - T_t ) x\\
  &= s-\lim_{h \downarrow 0} h^{-1} ( T_h - I ) T_t x ~ = A T_t
  x.\tag*{$\Box$}
 \end{align*}

Thus,\pageoriginale if $x \in \mathscr{D} (A) $, then $ T_t x \in \mathscr{D} (A)
$, and $ T_t A x = A T_t x = s- \lim \limits_{h \downarrow 0} h^{-1}
(T_{t+h} - T_t ) x $. We have now proved that the strong right
derivative of $ T_t x $ exists for each $ x \in \mathscr{D} (A) $. We
shall now show that the strong left derivative exists and is equal to
the right derivative. For this, take any $f\in X^* $. For fixed $ x,
f (T_t~ x ) $ is a continuous numerical function (real or complex -
valued ) on $ t \ge 0 $. By the above. $f(T_t~ x)$ has right
derivative $\dfrac{d^+ f(T_t~ x)}{dt}$ and 
$$
\frac{d^+ f (T_t~ x)}{dt} = f ( A T_t\, x ) = f ( T_t\, A \,x ).
$$ 

But $f(T_t \,A \,x )$ is a continuous function. It is well-known that
if one of the Dini-derivatives of a numerical function is ( finite and
) continuous, then the function is differentiable ( and the
derivative, of course, is continuous ). So $ f( T_t x ) $ is
differentiable in $t$ and 
\begin{align*}
 f (T_t x -x ) &= f (T_t x ) - f (T_0 x ) \\
 &= \int \limits^{t}_{0} \frac{d^+ f (T_s x)}{ds} ~ ds = ~ \int
 \limits^{t}_{0} f (T_s A x ) ds \\ 
 &= f \left(\int \limits^{t}_{0} A~ x~ ds\right).
\end{align*}

However, if every linear functional vanishes on an element $ x \in X
$, then $ x = 0 $ ( by Hahn - Banach theorem ). Consequently, 
$$
T_t ~ x -x = ~ \int \limits^{t}_{0} T_s ~ A x ds.
$$
for each $ x \in \mathscr{D} (A) $. Since $T_s$ is strongly
continuous in $s$, it follows from this, that $T_t$ is strongly
derivable: 
\begin{align*}
 D_t T_t x &= s- \lim_{h \rightarrow 0} h^{-1} ( T_{t+h} - T_t ) x\\
 &= s- \lim_{h \rightarrow 0} h^{-1} \int^{t+h}_t T_s A x ds \\
 &= T_t A x.
\end{align*}
