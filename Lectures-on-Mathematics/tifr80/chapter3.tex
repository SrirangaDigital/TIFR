
\chapter{Transformation Formulae for Exponential Sums}\label{chap3}

THE\pageoriginale BASIC RESULTS of these notes, formulae relating
exponential sums
$$
\sum\limits_{M_1\leq m\leq M_2}b(m)g(m)e(f(m)=d(m)\quad\text{or}\quad
a(m),
$$
or their smoothed versions, to other exponential sums involving the
same $b(m)$, are established in this chapter by combining the
summation formulae of Chapter \ref{chap1} with the theorems of
Chapter \ref{chap2} on exponential integrals. The theorems in
\cite{key16} and \cite{key17} concerning Dirichlet polynomials (which
will be discussed in \S~ \ref{chap4:sec4.1}) were the first examples
of such results. As will be seen, the methods of these papers work
even in the present more general context without any extra effort.

\section{Transformation of Exponential Sums}\label{chap3:sec3.1} 

To begin with, we derive a transformation formula for the above
mentioned sum with $b(m)=d(m)$. The proof is modelled on that of
Theorem 1 in \cite{key16}.

In the following theorems, $\delta_1, \delta_2,\ldots$ denote positive
constants which may be supposed to be arbitrarily small. Further, put
$L=\log M_1$ for short.

\begin{thm}\label{chap3:thm3.1}
Let $2\leq M_1<M_2\leq 2M_1$, and let $f$ and $g$ be holomorphic
functions in the domain
\begin{equation}\label{chap3:eq3.1.1}
D=\left\{z~ |z-x|<cM_1\quad\text{for some}\quad x\in[M_1,
M_2]\right\},
\end{equation}
where\pageoriginale $c$ is a positive constant. Suppose that $f(x)$ is
real for $M_1\leq x\leq M_2$. Suppose also that, for some positive
numbers $F$ and $G$,
\begin{align}
& |g(z)|\ll G,\label{chap3:eq3.1.2}\\
& |f'(z)|\ll FM_1^{-1}\label{chap3:eq3.1.3}
\end{align}
for $z\in D$, and that 
\begin{equation}\label{chap3:eq3.1.4}
(0<)f''(x)\gg FM_1^{-2}\quad\text{for}\quad M_1\leq x\leq M_2.
\end{equation}

Let $r=h/k$ be a rational number such that 
\begin{align}
& 1\leq k\ll M_1^{1/2-\delta_1},\label{chap3:eq3.1.5}\\
& |r|\asymp FM_1^{-1}\label{chap3:eq3.1.6}\\
\intertext{and}
& f'(M(r))=r\label{chap3:eq3.1.7}
\end{align}
for a certain number $M(r)\in(M_1,M_2)$. Write 
$$
M_j=M(r)+(-1)^jm_j,j=1,2.
$$

Suppose that $m_1\asymp m_2$, and that 
\begin{equation}\label{chap3:eq3.1.8}
M_1^{\delta_2}\max\left(M_1F^{-1/2},|hk|\right)\ll m_1\ll
M_1^{1-\delta_3}. 
\end{equation}

Define for $j=1,2$
\begin{align}
p_{j,n}(x) &= f(x)-rx+(-1)^{j-1}\left(2\sqrt{nx}/k-1/8
\right),\label{chap3:eq3.1.9}\\
n_j &= \left(r-f'\left(M_j\right)\right)^2k^2
M_j,\label{chap3:eq3.1.10} 
\end{align}
and for $n<n_j$ let $x_{j,n}$ be the (unique) zero of $p'_{j,n}(x)$ in
the interval $(M_1,M_2)$. Then 
\begin{gather}
\sum\limits_{M_1\leq m\leq M_2}d(m)g(m)e(f(m))\label{chap3:eq3.1.11}\\
=k^{-1}\left(\log M(r)+2\gamma-2\log k\right)g(M(r))f''(M(r))^{-1/2}
e(f(M(r))\notag\\
-rM(r)+1/8)+ +i2^{-1/2}k^{-1/2}\sum\limits_{j=1}^2(-1)^{j-1}\notag\\
\sum\limits_{n<n_j}d(n)
e_k\left(-n\bar{h}\right)n^{-1/4}x_{j,n}^{-1/4}g\left(x_{j,n}\right)
p''_{j,n}\left(x_{j,n}\right)^{-1/2}\times\notag\\
\times e\left(p_{j,n}(x_{j,n})+1/8\right)+o\left(FGh^{-2}km_1^{-1}L
\right)+o\left(G(|h|)^{1/2}m_1^{1/2}L^2\right)+\notag\\
+o\left(F^{1/2}G|h|^{-3/4}k^{5/4}m_1^{-1/4}L\right).\notag
\end{gather}\pageoriginale
\end{thm}

\begin{proof}
Suppose, to be specific, that $r>0$, and thus $h>0$. The proof is
similar for $r<0$.

The assertion \eqref{chap3:eq3.1.11} should be understood as an
asymptotic result, in which $M_1$ and $M_2$ are large. Then the
numbers $F$ and $n_j$ are also large. In fact,
\begin{gather}
F\gg M_1^{1/2+\delta_1}\label{chap3:eq3.1.12}\\
\intertext{and}
n_j\gg hkM_1^{2\delta_2}.\label{chap3:eq3.1.13}
\end{gather}

For a proof of \eqref{chap3:eq3.1.12}, note that by
\eqref{chap3:eq3.1.6} and \eqref{chap3:eq3.1.5}
$$
F\gg M_1r\geq k^{-1}M_1\gg M_1^{1/2+\delta_1}.
$$

Consider next the order of $n_j$. By \eqref{chap3:eq3.1.1} and the
holomorphicity of $f$ in the domain \eqref{chap3:eq3.1.1} we have
$f''(x)\ll FM_1^{-2}$, which implies together with
\eqref{chap3:eq3.1.4} that
\begin{equation}\label{chap3:eq3.1.14}
f''(x)\asymp FM_1^{-2}\quad\text{for}\quad M_1\leq x\leq M_2.
\end{equation}

Thus, by \eqref{chap3:eq3.1.7}, 
\begin{gather}
\left|r-f'\left(M_j\right)\right|\asymp
m_jFM_1^{-2},\label{chap3:eq3.1.15}\\
\intertext{so that by \eqref{chap3:eq3.1.10} and \eqref{chap3:eq3.1.6}
we have}
n_j\asymp F^2k^2m_j^2M_1^{-3}\asymp
F^{-1}h^3k^{-1}m_j^2.\label{chap3:eq3.1.16} 
\end{gather}\pageoriginale

This gives \eqref{chap3:eq3.1.13} owing to the estimates $m_j\gg
M_1^{1+\delta_2}F^{-1/2}$ and $F\ll M_1r$. Also, it follows that
$n_j\ll M_1^A$, for $h\ll M_1$ and $m_j\ll M_1$ by
\eqref{chap3:eq3.1.8}. 

The numbers $n_j$ are determined by the condition 
\begin{equation}\label{chap3:eq3.1.17}
p'_{j,n_j}\left(M_j\right)=0.
\end{equation}
Then clearly 
$$
(-1)^jp'_{j,n}\left(M_j\right)>0\quad\text{for}\quad n<n_j.
$$

On the other hand, by \eqref{chap3:eq3.1.7}
$$
(-1)^jp'_{j,n}(M(r))=-n^{1/2}M(r)^{-1/2}k^{-1}<0
$$
for all positive $n$. Consequently, for $n<n_j$ there is a zero
$x_{j,n}$ of $p'_{j,n}(x)$ in the interval $(M_1,M_2)$, and moreover
$x_{1,n}\in(M_1,M(r)), x_{2,n}\in(M(r),M_2)$. Also, it is clear that
$p'_{j,n}$ has no zero in the interval $(M_1,M_2)$ if $n\geq n_j$.

To prove the uniqueness of $x_{j,n}$, we show that $p''_{j,n}$ is
positive and thus $p'_{j,n}$ is increasing in the interval
$[M_1,M_2]$. In fact, 
\begin{equation}\label{chap3:eq3.1.18}
p''_{j,n}(x)\asymp FM_1^{-2}\quad\text{for}\quad M_1\leq x\leq M_2,
n\leq 2n_j,
\end{equation}
at least if $M_1$ is supposed to be sufficiently large. For by
definition 
$$
p''_{j,n}(x)=f''(x)+(-1)^j\frac{1}{2}n^{1/2}x^{-3/2}k^{-1},
$$
where by \eqref{chap3:eq3.1.16} and \eqref{chap3:eq3.1.8}
$$
n^{1/2}x^{-3/2}k^{-1}\ll Fm_1M_1^{-3}\ll FM_1^{-2-\delta_3},
$$\pageoriginale
so that by \eqref{chap3:eq3.1.14} the term $f''(x)$ dominates.

After these preliminaries we may go into the proof of the formula
\eqref{chap3:eq3.1.11}. Denote by 5 the sum under
consideration. Actually it is easier to deal with the smoothed sum 
\begin{gather}
S'=U^{-1}\int\limits_\circ^US(u)\,du,\label{chap3:eq3.1.19}\\
\intertext{where}
S(u)=\sum\limits_{M_1+u\leq m\leq M_2-u}d(m)g(m)e(f
(m)).\label{chap3:eq3.1.20} 
\end{gather}

The parameter $U$ will be chosen later in an optimal way; presently we
suppose only that 
\begin{equation}\label{chap3:eq3.1.21}
M_1^{\delta_4}\ll U\leq\frac{1}{2}\min\left(m_1,m_2\right). 
\end{equation}

Since
\begin{gather}
\sum\limits_{x\leq n\leq x+y}d(n)\ll y\log x\quad\text{for}\quad x^\epsilon
\ll y\ll x,\label{chap3:eq3.1.22}\\
\intertext{(see \cite{key25}), we have}
S-S'\ll GUL.\label{chap3:eq3.1.23}
\end{gather}

The summation formula \eqref{chap1:eq1.9.1} is now applied to the sum
$S(u)$, which is first written as 
$$
S(u)=\sum\limits_{a\leq m\leq b}d(m)g(m)e(f(m)-mr)e(mr),
$$
with $a=M_1+u, b=M_2-u$. We may assume that neither of the numbers $a$
and $b$ is an integer, for the value of $S(u)$ for the finitely many
other values of $u$ is irrelevant in the integral
\eqref{chap3:eq3.1.19}. Then by \eqref{chap1:eq1.9.1}
\begin{align}
S(u) &= k^{-1}\int\limits_a^b(\log x+2\gamma-2\log k)g(x)e(f(x)-rx)
\,dx\label{chap3:eq3.1.24}\\ 
&\qquad +k^{-1}\sum\limits_{n=1}^\infty d(n)\int\limits_a^b\left\{
-2\pi e_k\left(-n\bar{h}\right)Y_\circ\left(4\pi\sqrt{nx}/k\right)+4
e_k\left(n\bar{h}\right)\right.\notag\\
&\qquad \left. K_\circ\left(4\pi\sqrt{nx}/k\right)\right\}
g(x)e(f(x)-rx)\,dx\notag\\
&= k^{-1}\left\{I_\circ+\sum\limits_{n=1}^\infty d(n)\left(e_k\left(-n
\bar{h}\right)I_n+e_k\left(n\bar{h}\right)i_n\right)\right\},\notag
\end{align}\pageoriginale
say.

The integrals $i_n$ are very small and quite negligible. Indeed, by
\eqref{chap3:eq3.1.5} we have
$\sqrt{nM_1}/k\gg\sqrt{n}M_1^{\delta_1}$, so that by
\eqref{chap1:eq1.3.17}
\begin{align}
k^{-1}\sum\limits_{n=1}^\infty d(n)\left|i_n\right|& \ll k^{-1}GM_1
\sum\limits_{n=1}^\infty d(n)\exp\left(-A\sqrt{n}M_1^{\delta_1}
\right)\label{chap3:3.1.25}\\
&\ll G\exp\left(-AM_1^{\delta_1}\right).\notag
\end{align}

Consider next the integral $I_\circ$. We apply Theorem
\ref{chap2:thm2.1} with $\alpha = -r$ and $\mu(x)$ a constant function
$\asymp M_1$. The assumptions of Theorem \ref{chap2:thm2.1} are
satisfied in virtue of the conditions of our theorem. By
\eqref{chap3:eq3.1.7}, the saddle point is $M(r)$. Hence the
saddle-point term for $k^{-1}I_\circ$ equals the leading term in
\eqref{chap3:eq3.1.11}. 

The first error term in \eqref{chap2:eq2.1.9} is 
$$
\ll LGM_1\exp(-AF)
$$
which is negligible by \eqref{chap3:eq3.1.12}.

The last two error terms contribute
\begin{equation}\label{chap3:eq3.1.26}
\ll GL\left\{\left(\left|f'(a)-r\right|+F^{1/2}M_1^{-1}\right)^{-1}
+\left( \left|f'(b)-r\right|+F^{1/2}M_1^{-1}\right)^{-1}\right\}.
\end{equation}

For same reasons as in \eqref{chap3:eq3.1.15}, we have 
$$
\left|f'(a)-r\right|\asymp Fm_1M_1^{-2},
$$
and likewise for $|f'(b)-r|$. Hence the expression
\eqref{chap3:eq3.1.26} is $\ll F^{-1}\break Gm_1^{-1}M_1^2L$,\pageoriginale
which is further $\ll FGm_1^{-1}r^{-2}L$ by \eqref{chap3:eq3.1.6}. The
second error term in \eqref{chap2:eq2.1.9}, viz. $o(GM_1F^{-3/2}L)$, can
be absorbed into this, for 
$$
M_1F^{-3/2}\ll r^{-1}\ll FM_1^{-1}r^{-2}\ll Fm_1^{-1}r^{-2}.
$$

Hence the error terms for $k^{-1}I_\circ$ give together
$o(FGh^{-2}km_1^{-1}L)$,\break which is the first error term in
\eqref{chap3:eq3.1.11}. 

We are now left with the integrals 
\begin{equation}\label{chap3:eq3.1.27}
I_n=-2\pi\int\limits_a^bY_\circ\left(4\pi\sqrt{nx}/k\right)g(x)e(f(x)-rx)
\,dx. 
\end{equation}

By \eqref{chap1:eq1.3.9}, the function $Y_\circ$ can be written in
terms of Hankel functions as 
\begin{align}
Y_\circ(z) &= \frac{1}{2i}\left(H_\circ^{(1)}(z)-H_\circ^{(2)}(z)
\right),\label{chap3:eq3.1.28}\\
\intertext{where by \eqref{chap1:eq1.3.13}}
H_\circ^{(j)}(z) &= \left(\frac{2}{\pi z}\right)^{1/2}\exp\left(
(-1)^{j-1}i\left(z-\frac{1}{4}\pi\right)\right)\left(
1+g_j(z)\right).\label{chap3:eq3.1.29}
\end{align}

The functions $g_j(z)$ are holomorphic in the half-plane $Re\,z>0$,
and by \eqref{chap1:eq1.3.14}
\begin{equation}\label{chap3:eq3.1.30}
\left|g_j(z)\right|\ll |z|^{-1}\quad\text{for}\quad |z|\geq 1,
Re\,z>0. 
\end{equation}

By \eqref{chap3:eq3.1.27} - \eqref{chap3:eq3.1.29} we may write 
\begin{gather}
I_n=I_n^{(1)}-I_n^{(2)},\label{chap3:eq3.1.31}\\
\intertext{where}
I_n^{(j)}=i2^{-1/2}k^{1/2}n^{-1/4}\int\limits_a^bx^{-1/4}g(x)
\left(1+g_j\left(4\pi\sqrt{nx}/k\right)\right)e\left(p_{j,n}(x)
\right)\,dx.\label{chap3:eq3.1.32}
\end{gather}

For $n\leq 2n_j$ we apply Theorem \ref{chap2:thm2.1} to $I_n^{(j)}$,
again with $\alpha = -r$. The\pageoriginale function 
$$
f(x)+(-1)^{j-1}\left(2\sqrt{nx}/k-1/8\right)
$$
now stands for the function $f$, and moreover $\mu(x)\asymp M_1$ and
$F(x)=F$. The conditions of the theorem are satisfied, in particular
the validity of the condition (iv) on $f''$ follows from
\eqref{chap3:eq3.1.18}, and the condition (iii) on $f'$ can be checked
by \eqref{chap3:eq3.1.3} and \eqref{chap3:eq3.1.16}. The number
$x_{j,n}$ is, by definition, the saddle point for $I_n^{(j)}$, and it
lies in the interval $(M_1,M_2)$ if and only if $n<n_j$. However, in
$I_n^{(j)}$ the interval of integration is $[a,b]=[M_1+u,M_2-u]$, and
$x_{j,n}\in (a,b)$ if and only if $n<n_j(u)$, where
\begin{equation}\label{chap3:eq3.1.33}
n_j(u)=\left(r-f'\left(M_j+(-1)^{j-1}u\right)\right)^2k^2\left(M_j+
(-1)^{j-1}u\right)
\end{equation}
in analogy with \eqref{chap3:eq3.1.10}. But for simplicity we count the
saddle-point terms for all $n<n_j$, and the number of superfluous
terms is then 
\begin{equation}\label{chap3:eq3.1.34}
\ll 1+n_j-n_j(U)\ll 1+F^2k^2m_1M_1^{-3}U.
\end{equation}

The saddle-point term for $k^{-1}I_n^{(j)}$ is 
\begin{gather}
i2^{-1/2}k^{-1/2}n^{-1/4}x_{j,n}^{-1/4}g\left(x_{j,n}\right)\;\left(1+g_j
\left(4\pi\sqrt{nx_{j,n}}/k\right)\right)
\times\label{chap3:eq3.1.35}\\
\times p''_{j,n}\left(x_{j,n}\right)^{-1/2} e\left(p_{j,n}\left
(x_{j,n}\right)+1/8\right).\notag
\end{gather}

Multiplied by $(-1)^{j-1}d(n)e_k(-n\bar{h})$, these agree, up to
$g_j(\ldots)$, with the individual terms of the sums on the right of
\eqref{chap3:eq3.1.11}. The effect of the omission of $g_j(\ldots)$ is
by \eqref{chap3:eq3.1.30}, \eqref{chap3:eq3.1.16},
\eqref{chap3:eq3.1.18}, and \eqref{chap3:eq3.1.5}
\begin{align*}
&\ll F^{-1/2}Gk^{1/2}M_1^{1/4}\sum\limits_{n\ll M_1}d(n)n^{-3/4}\\
&\ll Gkm_1^{1/2}M_1^{-1/2}L\ll Gm_1^{1/2}L,
\end{align*}
which\pageoriginale can be absorbed into the second error term in
\eqref{chap3:eq3.1.11}. 

The extra saddle-point terms, counted in \eqref{chap3:eq3.1.34},
contribute at most
$$
\ll
\left(1+F^2k^2m_1M_1^{-3}U\right)F^{-1/2}Gk^{-1/2}M_1^{3/4+\epsilon}n_1^{-1/4},
$$
which, by \eqref{chap3:eq3.1.16} and \eqref{chap3:eq3.1.6}, is 
\begin{equation}\label{chap3:eq3.1.36}
\ll F^{1/2}Gh^{-3/2}k^{1/2}m_1^{-1/2}M_1^\epsilon+F^{-1/2}Gh^{3/2}k^{-1/2}
m_1^{1/2}M_1^\epsilon U.
\end{equation}

Now, allowing for these error terms, we have the same saddle-point
terms given in \eqref{chap3:eq3.1.11}, for all sums $S(u)$, and hence
by \eqref{chap3:eq3.1.19} for $S'$, too. 

Consider now the error terms when Theorem \ref{chap2:thm2.1} is
applied to $I_n^{(j)}$ for $n\leq 2n_j$. The first error term in
\eqref{chap2:eq2.1.9} is clearly negligible. Further, the contribution
of the error terms involving $x_\circ$ to $S(u)$ is 
$$
\ll F^{-3/2}Gk^{-1/2}M_1^{3/4}\sum\limits_{n\ll n_1}d(n)n^{-1/4},
$$
which, by \eqref{chap3:eq3.1.16}, \eqref{chap3:eq3.1.8} and
\eqref{chap3:eq3.1.5}, is 
$$
\ll Gkm_1^{3/2}M_1^{-3/2}L\ll Gkm_1^{1/2}M_1^{-1/2}L\ll Gm_1^{1/2}.
$$

This is smaller than the second error term in \eqref{chap3:eq3.1.11}.

The last two error terms are similar, so it suffices to consider
$o(E_\circ(a))$ as an example. By \eqref{chap2:eq2.1.8} and
\eqref{chap3:eq3.1.32}, this error term for $k^{-1}I_n^{(j)}$ is 
$$
\ll Gk^{-1/2}M_1^{-1/4}n^{-1/4}\left(\left|p'_{j,n}(a)\right|+
p''_{j,n}(a)^{1/2}\right)^{-1}.
$$

Consider the case $j=1$; the case $j=2$ is less critical since
$|p'_{2,n}(a)|$ cannot be small. Now $p'_{1,n_1(u)}(a)=0$ and
$p''_{1,n}(a)\asymp F^{-1}r^2$, so\pageoriginale it is easily seen
that 
{\fontsize{10}{12}\selectfont
\begin{equation*}
\left(\left|p'_{1,n}(a)\right|+p''_{1,n}(a)^{1/2}\right)^{-1}\ll
\begin{cases}
F^{1/2}r^{-1}\quad\text{for}\quad\left|n-n_1(u)\right|\ll F^{-1/2}
h^2m_1,\\
kM_1^{1/2}n_1^{1/2}\left|n-n_1(u)\right|^{-1}\quad\text{otherwise}
\end{cases}
\end{equation*}}
Note that by \eqref{chap3:eq3.1.8} and \eqref{chap3:eq3.1.6}
$$
F^{-1/2}h^2m_1\gg F^{-1}h^2M_1^{1+\delta_2}\gg hkM_1^{\delta_2}.
$$

Hence, by \eqref{chap3:eq3.1.22}, the mean value of $d(n)$ in the
interval $|n-n_1(u)|\break\ll F^{-1/2}h^2m_1$ can be estimated as $o(L)$. It
is now easily seen that the contribution to $S(u)$ of the error terms
in question is 
$$
\ll Gh^{1/2}k^{1/2}m_1^{1/2}L^2,
$$
which is the second error term in \eqref{chap3:eq3.1.11}.

The smoothing device was introduced with the integrals $I_n^{(j)}$ for
$n>2n_j$ in mind. By \eqref{chap3:eq3.1.19}, \eqref{chap3:eq3.1.24},
\eqref{chap3:eq3.1.31}, and \eqref{chap3:eq3.1.32}, their contribution
to $S'$ is equal to 
\begin{align}
& i2^{-1/2}k^{-1/2}\sum\limits_{j=1}^2(-1)^{j-1}\sum\limits_{n>2n_j}
d(n)e_k\left(-n\bar{h}\right)n^{-1/4}\times\label{chap3:eq3.1.37}\\
&\qquad \times \int\limits_{M_1}^{M_2}\eta_1(x)x^{-1/4}g(x)\left(1+g_j\left
(4\pi\sqrt{nx}/k\right)\right)e\left(p_{j,n}(x)\right)\,dx,\notag 
\end{align}
where $\eta_1(x)$ is a weight function in the sense of Chapter
\ref{chap2}, with $J=1$ and $U$ being the other smoothing
parameter. The series in \eqref{chap3:eq3.1.24} is boundedly
convergent with respect to $u$, by Theorem \ref{chap1:thm1.7}, so that
it can be integrated term by term. 

The smoothed exponential integrals in \eqref{chap3:eq3.1.37} are
estimated by Theorem \ref{chap2:thm2.3}, where $p_{j,n}(z)$ stands for
$f(z)$, and $\mu\asymp m_1$. To begin with,\pageoriginale we have to
check that the conditions of this theorem are satisfied. We have 
$$
p'_{j,n}(z)=f'(z)-r+(-1)^{j-1}n^{1/2}z^{-1/2}k^{-1}.
$$

Let $n>2n_j$, and let $z$ lie in the domain $D$, say $D_\circ$, of
Theorem \ref{chap2:thm2.3}. Then by \eqref{chap3:eq3.1.16}
$$
\left|n^{1/2}z^{-1/2}k^{-1}\right|\gg m_1FM_1^{-2}.
$$

On the other hand, since $f'(M(r))-r=0$ and $|f''(z)|\ll FM_1^{-2}$
for $z\in D_\circ$ by \eqref{chap3:eq3.1.3} and Cauchy's integral
formula, we also have 
$$
|f'(z)-r|\ll m_1FM_1^{-2}.
$$

Thus, the condition \eqref{chap2:eq2.2.3} holds with 
\begin{equation}\label{chap3:eq3.1.38}
M=k^{-1}M_1^{-1/2}n^{1/2}.
\end{equation}

Further, to verify the condition \eqref{chap2:eq2.2.2}, compare
$p'_{j,n}(x)$ with $p'_{j,n_j}(x)$, using \eqref{chap3:eq3.1.17} and
the fact that $p'_{j,n_j}(x)$ is increasing in the interval\break
$[M_1,M_2]$. 

We may now apply the estimate \eqref{chap2:eq2.2.4} in
\eqref{chap3:eq3.1.37}. The second term on the right of
\eqref{chap2:eq2.2.4} is exponentially small, for by
\eqref{chap3:eq3.1.38}, \eqref{chap3:eq3.1.16}, and
\eqref{chap3:eq3.1.8}
\begin{align*}
M\mu\gg k^{-1}M_1^{-1/2}n^{1/2}m_1&\gg\left(n/n_1\right)^{1/2} Fm_1^2
M_1^{-2}\\
&\gg \left(n/n_1\right)^{1/2}M_1^{2\delta_2}.
\end{align*}

Hence these terms are negligible.

The contribution of the terms $U^{-1}GM^{-2}$ in \eqref{chap2:eq2.2.4}
to \eqref{chap3:eq3.1.37} is 
\begin{align*}
& \ll Gk^{3/2}M_1^{3/4}U^{-1}\sum\limits_{n\gg n_1}d(n)n^{-5/4}\\
& \ll Gk^{3/2}M_1^{3/4}n_1^{-1/4}U^{-1}L\\
& \ll GF^{-1/2}kM_1^{3/2}m_1^{-1/2}U^{-1}L\\
& \ll GFh^{-3/2}k^{5/2}m_1^{-1/2}U^{-1}L.
\end{align*}\pageoriginale

Combining this with \eqref{chap3:eq3.1.23} and \eqref{chap3:eq3.1.36},
we find that \eqref{chap3:eq3.1.11} holds, up to the additional error
terms
\begin{align}
&\ll GUL+F^{1/2}Gh^{-3/2}k^{1/2}m_1^{-1/2}
M_1^\epsilon\label{chap3:eq3.1.39}\\ 
& +F^{-1/2}Gh^{3/2}k^{-1/2}m_1^{1/2}M_1^\epsilon U+GFh^{-3/2}k^{5/2}
m_1^{-1/2}U^{-1}L.\notag
\end{align}

Here the second term is superseded by the last term in
\eqref{chap3:eq3.1.11}. Further, the first and last term coincide with
the last term in \eqref{chap3:eq3.1.11} if we choose
\begin{equation}\label{chap3:eq3.1.40}
U=F^{1/2}h^{-3/4}k^{5/4}m_1^{-1/4}.
\end{equation}

Then, by \eqref{chap3:eq3.1.8}, the third term in
\eqref{chap3:eq3.1.39} is 
$$
\ll Gh^{3/4}k^{3/4}m_1^{1/4}M_1\ll G(hk)^{1/2}m_1^{1/2}M_1^{\epsilon
-\delta_2/4}, 
$$
which can be absorbed into the second error term in
\eqref{chap3:eq3.1.11}.

It should still be verified that the number $U$ in
\eqref{chap3:eq3.1.40} satisfies the condition
\eqref{chap3:eq3.1.21}. By \eqref{chap3:eq3.1.8} and
\eqref{chap3:eq3.1.6} we have 
$$
Um_1^{-1}\ll U\left(M_1^{1+\delta_2}F^{-1/2}\right)^{-1}\ll(hk)^{1/4}
m_1^{-1/4}M_1^{-\delta_2}\ll M_1^{-\delta_2}
$$
and also, in the other direction,
\begin{align*}
U &\gg F^{1/2}h^{-3/4}k^{5/4}M_1^{-1/4+\delta_3/4}\\
& \gg M_1^{1/4+\delta_3/4}h^{-1/4}k^{3/4}\gg M_1^{\delta_3/4}
\end{align*}

Hence\pageoriginale \eqref{chap3:eq3.1.21} holds, and the proof of the
theorem is complete.

The next theorem is an analogue of Theorem \ref{chap3:thm3.1} for
exponential sums involving the Fourier coefficients $a(n)$ of a cusp
form of weight $\kappa$. The proof is omitted, because the argument is
practically the same; the summation formula \eqref{chap1:eq1.9.2} is
just applied in place of \eqref{chap1:eq1.9.1}. Note that in
\eqref{chap1:eq1.9.2} there is nothing is correspond to the first term
in \eqref{chap1:eq1.9.1}, and consequently in the transformation
formula there are no counterparts for the first explicit term and the
first error term in \eqref{chap3:eq3.1.11}.
\end{proof}

\begin{thm}\label{chap3:thm3.2}
Suppose that the assumptions of Theorem \ref{chap3:thm3.1} are
satisfied. Then 
\begin{align}
&\quad \sum\limits_{M_1\leq m\leq
M_2}a(m)g(m)e(f(m))\label{chap3:eq3.1.41}\\
& = i2^{-1/2}k^{-1/2}\sum\limits_{j=1}^2(-1)^{j-1} \sum\limits_{n<n_j}
a(n)e_k\left(-n\bar{h}\right)n^{\kappa/2+1/4}\times\notag\\
&\quad \times x_{j,n}^{\kappa/2-3/4}g\left(x_{j,n}\right) p''_{j,n}
\left(x_{j,n}\right)^{-1/2}e\left(p_{j,n}\left(x_{j,n}\right)
+1/8\right)\notag\\
&\quad +o\left(G\left(|h|k\right)^{1/2}M_1^{(\kappa-1)/2}m_1^{1/2}L^2
\right)\notag\\
&\quad +o\left(F^{1/2}G|h|^{-3/4}k^{5/4}M_1^{(\kappa-1)/2}m_1^{-1/4}L
\right)\notag. 
\end{align}
\end{thm}

\section{Transformation of Smoothed Exponential
  Sums}\label{chap3:sec3.2}

We now give analogues of Theorem \ref{chap3:thm3.1} and
\ref{chap3:thm3.2} for smoothed exponential sums provided with weights
of the type $\eta_J(n)$. We have to pay for the better error terms in
these new formulae by allowing certain weights to appear in the
transformed sums as well.

\begin{thm}\label{chap3:thm3.3}
Suppose that the assumptions of Theorem \ref{chap3:thm3.1} are
satisfied. Let 
\begin{equation}\label{chap3:eq3.2.1}
U\gg F^{-1/2}M_1^{1+\delta_4}\asymp F^{1/2}r^{-1}M_1^{\delta_4},
\end{equation}\pageoriginale
and let $J$ be a fixed positive integer exceeding a certain bound
(which depends on $\delta_4$). Write for $j=1,2$
$$
M'_j=M_j+(-1)^{j-1}JU=M(r)+(-1)^jm'_j,
$$
and suppose that $m'_j\asymp m_j$. Let $n_j$ be as in
\eqref{chap3:eq3.1.10}, and define analogously
\begin{equation}\label{chap3:eq3.2.2}
n'_j=\left(r-f'\left(M'_j\right)\right)^2k^2M'_j.
\end{equation}

Then, defining the weight function $\eta_J(x)$ in the interval
$[M_1,M_2]$ as in \eqref{chap2:eq2.1.2}, we have
\begin{align}
&\qquad \sum\limits_{M_1\leq m\leq M_2}\eta_J(m)d(m)g(m)
e(f(m))\label{chap3:eq3.2.3}\\
&= k^{-1}\left(\log M(r)+2\gamma -2\log k\right) g
(M(r))f''(M(r))^{-1/2}\notag\\ 
& \hspace{5cm}e(f(M(r))-rM(r)+1/8)\notag\\
&\quad +i2^{-1/2}k^{-1/2}\sum\limits_{j=1}^2(-1)^{j-1}
\sum\limits_{n<n_j}w_j(n)d(n)e_k\left(-n\bar{h}\right)n^{-1/4}
x_{j,n}^{-1/4}\times\notag\\
&\quad \times \left(x_{j,n}\right)p''_{j,n}\left(x_{j,n}\right)^{-1/2}
e\left(p_{j,n}\left(x_{j,n}\right)+1/8\right)\notag\\
& \hspace{3cm}+o\left(F^{-1}G|h|^{3/2}
k^{-1/2}m_1^{1/2}UL\right),\notag
\end{align}
where
\begin{align}
w_j(n) &=1 \quad\text{for}\quad n < n'_j,\label{chap3:eq3.2.4}\\
w_j(n) &\ll 1\quad\text{for}\quad n < n_j,\label{chap3:eq3.2.5}
\end{align}
$w_j(y)$ and $w'_j(y)$ are piecewise continuous functions in the
interval $(n'_j,\break n_j)$ with at most $J-1$ discontinuities, and 
\begin{equation}\label{chap3:eq3.2.6}
w'_j(y)\ll \left(n_j-n'_j\right)^{-1}\quad\text{for}\quad n'_j<y<n_j 
\end{equation}
whenever $w'_j(y)$ exists.
\end{thm}

\begin{proof}
We\pageoriginale follow the argument of the proof of Theorem
\ref{chap3:thm3.1}, using however Theorem \ref{chap2:thm2.2} in place
of Theorem 
\ref{chap2:thm2.1}. Denoting by $S_J$ the smoothed sum under
consideration, we have by \eqref{chap1:eq1.9.1}, as in
\eqref{chap3:eq3.1.24}, 
\begin{align}
  S_J & = k^{-1}\int\limits_{M_1}^{M_2}\left(\log x+2\gamma-2\log
  k\right)\eta_J(x)g(x)e(f(x)-rx)\,dx\label{chap3:eq3.2.7}\\
  &\quad +k^{-1}\sum\limits_{n=1}^\infty d(n)\int\limits_{M_1}^{M_2}
  \left\{-2\pi e_k\left(-n\bar{h}\right)Y_\circ\left(4\pi\sqrt{nx}/k
  \right)+4e_k\left(n\bar{h}\right)K_\circ\right.\notag\\
  & \hspace{4cm}\left.\left(4\pi\sqrt{nx}/k\right) \right\}
  \eta_J(x)g(x)e(f(x)-rx)\,dx\notag\\ 
  &= k^{-1}\left\{I_\circ+\sum\limits_{n=1}^\infty d(n)(e_k(-n\bar{h})
  I_n+e_k(n\bar{h})i_n)\right\}.\notag
\end{align}

As in the proof of Theorem \ref{chap3:thm3.1}, the integrals $i_n$ are
negligible.

Consider next the integral $I_\circ$. We apply Theorem
\ref{chap2:thm2.2} choosing\break $\mu(x)\asymp M_1$ again. The saddle-point
is $M(r)$, as before, and the saddle-point term for $I_\circ$ is the
same as in the proof of Theorem \ref{chap3:thm3.1}. The first error
term in \eqref{chap2:eq2.1.11} is exponentially small. The error terms
involving $E_J$ are also negligible if $J$ is taken sufficiently large
(depending on $\delta_4$), since
$$
U^{-1}f''(x)^{-1/2}\ll M^{-\delta_4}\quad\text{for}\quad M_1\leq x\leq
M_2. 
$$

The contribution of the error term $o(G(x_\circ)\mu(x_\circ)F
(x_\circ)^{-3/2})$ to $k^{-1}I_\circ$ is by \eqref{chap3:eq3.2.1} and
\eqref{chap3:eq3.1.8} 
$$
\ll k^{-1}F^{-3/2}GM_1L\ll F^{-1}GUL\ll F^{-1}Gh^{-1/2}m_1^{1/2}U,
$$
which does not exceed the error term in \eqref{chap3:eq3.2.3}. 

Turning to the integrals $I_n$, we write as in \eqref{chap3:eq3.1.31}
and \eqref{chap3:eq3.1.32}
$$
I_n = I_n^{(1)}-I_n^{(2)},
$$
where\pageoriginale
{\fontsize{10}{12}\selectfont
\begin{equation}\label{chap3:eq3.2.8}
I_n^{(j)}=i2^{-1/2}k^{1/2}n^{-1/4}\int\limits_{M_1}^{M_2}\eta_J(x)^{-1/4}
g(x)\left(1+g_j\left(4\pi\sqrt{nx}/k\right)\right)e\left(p_{j,n}(x)
\right)\,dx. 
\end{equation}}

Let first $n>2n_j$. As in the proof of Theorem \ref{chap3:thm3.1}, we
may apply Theorem \ref{chap2:thm2.3} with $\mu\asymp m_1$ and $M$ as
in \eqref{chap3:eq3.1.38}. Observe that by \eqref{chap3:eq3.2.1},
\eqref{chap3:eq3.1.16}, and \eqref{chap3:eq3.1.8}
$$
U^{-1}M^{-1}\ll \left(n_1/n\right)^{1/2}F^{-1/2}m_1^{-1}
M_1^{1-\delta_4}\ll\left(n_1/n\right)^{1/2}M_1^{-\delta_2-\delta_4}
$$
whence we we may make the term $U^{-J}GM^{-J-1}$ in
\eqref{chap2:eq2.2.4} negligibly small by taking $J$ large enough. As
before, the second term in \eqref{chap2:eq2.2.4} is also negligible.

The terms for $n\leq 2n_j$ are dealt with by Theorem
\ref{chap2:thm2.2}. The saddle point terms occur again for $n<n_j$,
and they are of the same shape as in \eqref{chap3:eq3.1.35} except
that there is the additional factor 
\begin{equation}\label{chap3:eq3.2.9}
w_j(n)=\xi\left(x_{j,n}\right).
\end{equation}

The property \eqref{chap3:eq3.2.4} of $w_j(n)$ is immediate by
\eqref{chap2:eq2.1.12}, for $M_1+JU < x_{j,n} < M_2-JU$ if and only if
$n<n'_j$. Further, \eqref{chap3:eq3.2.5} follows from
\eqref{chap2:eq2.1.12} - \eqref{chap2:eq2.1.14} by
\eqref{chap3:eq3.2.1} and \eqref{chap3:eq3.1.18}. To prove the
property \eqref{chap3:eq3.2.6}, consider
$$
w_j(y)=\xi\left(x_{j,y}\right)
$$
as a function of the continuous variable in the interval
$(n'_j,n_j)$. Here $x_{j,y}$ is the unique zero of $p'_{j,y}(x)$ in
the interval $(M_1,M_1+JU)$ for $j=1$, and in the interval
$(M_2-JU,M_2)$ for $j=2$. Thus $z=x_{j,y}$ satisfies
$$
f'(z)-r+(-1)^{j-1}y^{1/2}z^{-1/2}k^{-1}=0.
$$

Hence,\pageoriginale by implicit differentiation,
$$
p''_{j,y}(z)\frac{dx_{j,y}}{dy}+\frac{1}{2}(-1)^{j-1}y^{-1/2}
z^{-1/2}k^{-1}=0,
$$
which implies that 
\begin{equation}\label{chap3:eq3.2.10}
\left|\frac{dx_{j,y}}{dy}\right|\asymp F^{-1}k^{-1}M_1^{3/2}n_1^{-1/2}
\asymp m_1n_1^{-1}\quad\text{for}\quad n'_j<y<n_j.
\end{equation}

The function $\xi_J(x)$ in \eqref{chap2:eq2.1.13} and
\eqref{chap2:eq2.1.14} is continuously differentiable except at the
points $a+JU$ and $b-jU, j=1,\ldots,J$, where terms appear or
disappear. By differentiation, noting that ${f''}'(x)\ll FM_1^{-3}$,
it is easy to verify that 
\begin{equation}\label{chap3:eq3.2.11}
\xi'_J(x)\ll U^{-1}
\end{equation}
elsewhere in the intervals $(a,a+JU)$ and $(b-JU,b)$. By
\eqref{chap3:eq3.1.10}, \eqref{chap3:eq3.2.2}, and
\eqref{chap3:eq3.1.14} we have 
\begin{equation}\label{chap3:eq3.2.12}
\left(n_j-n'_j\right)n_j^{-1}\asymp m_1^{-1}U.
\end{equation}

Now \eqref{chap3:eq3.2.6} follows from \eqref{chap3:eq3.2.10} -
\eqref{chap3:eq3.2.12} at those points $y$ for which $x_{j,y}$ is not
of the form $a+jU$ or $b-jU$ with $1\leq j\leq J-1$. 

As in the proof of Theorem \ref{chap3:thm3.1}, we may omit
$g_j(\ldots)$ in the saddle-point terms with an admissible error
$$
Gkm_1^{1/2}M_1^{-1/2}L\ll F^{-1}Gh^{3/2}k^{-1/2}m_1^{1/2}U,
$$
and after that these terms coincide with those in
\eqref{chap3:eq3.2.3}.

Consider finally the error terms in \eqref{chap2:eq2.1.11} for
$I_n^{(j)}$. The first of these is clearly negligible. Also, for the
same reason as in the case of $I_\circ$, the error terms involving
$E_J$ can be omitted if $J$ is chosen sufficiently large.

Finally,\pageoriginale the second error term in \eqref{chap2:eq2.1.11}
for $k^{-1}I_n^{(j)}$ is 
$$
\ll F^{-3/2}Gk^{-1/2}M_1^{3/4}n_1^{-1/4}\quad\text{for}\quad n<n'_j
$$
and
$$
\ll F^{-1}Gk^{-1/2}M_1^{3/4}n_1^{-1/4}\quad\text{for}\quad n'_j\leq
n<n_j. 
$$

The contribution of these to $S_J$ is 
\begin{equation}\label{chap3:eq3.2.13}
\ll F^{-3/2}Gk^{-1/2}M_1^{3/4}n_1^{3/4}L+F^{-1}Gk^{-1/2}M_1^{3/4}
n_1^{-1/4}\left(n_j-n'_j\right)L.
\end{equation}

Here we estimated the mean value of $d(n)$ in the interval $[n'_j,
n_j)$ by  $o(L)$, which is possible, by \eqref{chap3:eq3.1.22}, for by
\eqref{chap3:eq3.2.12}, \eqref{chap3:eq3.1.16}, \eqref{chap3:eq3.1.8},
\eqref{chap3:eq3.2.1}, and \eqref{chap3:eq3.1.6} we have 
\begin{align*}
n_j-n'_j &\ll m_1^{-1}n_1U\ll F^2k^2m_1M_1^{-3}U\\
&\ll F^2k^2\left(M_1^{1+\delta_2}F^{-1/2}\right)M_1^{-3}
\left(F^{-1/2}M_1^{1+\delta_4}\right)\asymp hkM_1^{\delta_2+\delta_4}.
\end{align*}

By \eqref{chap3:eq3.2.12}, the second term in \eqref{chap3:eq3.2.13}
is 
$$
\asymp F^{-1}Gk^{-1/2}m_1^{-1}M_1^{3/4}n_1^{3/4}UL,
$$
which is of the same order as the error term in \eqref{chap3:eq3.2.3}
and dominates the first term, since
$$
m_1^{-1}U\gg \left(F^{-1/2}M_1\right)M_1^{-1}=F^{-1/2}.
$$

The proof of the theorem is now complete.

The analogue of the preceding theorem for exponential sums involving
Fourier coefficients $a(n)$ is as follows. The proof is similar and
can be omitted.
\end{proof}

\begin{thm}\label{chap3:thm3.4}
With\pageoriginale the assumptions of Theorem \ref{chap3:thm3.3}, we
have 
\begin{align}
&\qquad \sum\limits_{M_1\leq m\leq
M_2}\eta_J(m)a(m)g(m)e(f(m))\label{chap3:eq3.2.14}\\
&= i2^{-1/2}k^{-1/2}\sum\limits_{j=1}^2(-1)^{j-1}\sum\limits_{n<n_j}
w_j(n)a(n)e_k\left(-n\bar{h}\right)n^{-\kappa/2+1/4}\notag\\
&\quad \times x_{j,n}^{\kappa/2-3/4}g\left(x-{j,n}\right)p''_{j,n}
\left(x_{j,n}\right)^{-1/2}e\left(p_{j,n}\left(x_{j,n}\right)+1/8
\right)\notag\\
&\quad +o\left(F^{-1}G|h|^{3/2}k^{-1/2}M_1^{(\kappa-1)/2}m_1^{1/2}UL
\right).\notag 
\end{align}
\end{thm}

\begin{remark*}
In practice it is of advantage to choose $U$ as small as the condition
\eqref{chap3:eq3.2.1} permits, \ie
\begin{equation}\label{chap3:eq3.2.15}
U\asymp F^{1/2+\epsilon}r^{-1}.
\end{equation}

Then the error term in \eqref{chap3:eq3.2.3} is 
\begin{equation}\label{chap3:eq3.2.16}
o\left(F^{-1/2+\epsilon}G\left(|h|k\right)^{1/2}m_1^{1/2}\right)
\end{equation}
and that in \eqref{chap3:eq3.2.14} is 
\begin{equation}\label{chap3:eq3.2.17}
o\left(F^{-1/2+\epsilon}G\left(|h|k\right)^{1/2}M_1^{(\kappa-1)/2}
m_1^{1/2}\right). 
\end{equation}
\end{remark*}
