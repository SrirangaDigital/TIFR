\chapter{Cyclic Homology and the Connes Exact Couple}\label{chap3}

WE\pageoriginale START WITH the standard Hochschild complex and study
the internal cyclic symmetry in this complex. This leads to the cyclic
homology double complex $CC_{\bigdot\bigdot}(A)$ for an algebra $A$
which is constructed from two aspects of the standard Hochschild
complex and the natural homological resolution of finite cyclic
groups. In terms of this double complex, we define cyclic homology as
the homology of the associated single complex, and since the
Hochschild homology complex is on the vertical edge of this double
complex, we derive the Connes' exact couple exploiting the horizontal
degree $2$ periodicity of the double complex.

The standard Hochschild complex comes from a simplicial object which
has an additional cyclic group symmetry, formalized by Connes when he
introduced the notion of a cyclic object. As introduction to cyclic
objects is given.

\section{The standard complex}\label{chap3-sec1}\index{standard complex}

In Chapter \ref{chap2} \S\ \ref{chap2-sec6}, we considered an
axiomatic characterization of Hochs\-child homology and then remarked
that it is a split Tor functor over $A\otimes A^{op}$. The Tor
functors are defined, and in some cases also calculated, using a
projective resolution which in this case is a split projective
resolution made out of extended modules. We consider a particular
resolution using the most natural extended $A$-bimodules, $A\otimes
A^{q\otimes}\otimes A=C'_{q}(A)$ made out of tensor powers of $A$. The
morphisms in the resolution are defined using the extended
multiplications $\phi_{i}:C'_{q}(A)\to C'_{q-1}(A)$ defined by
$$
\phi_{i}(a_{0}\otimes\cdots\otimes a_{q+1})=a_{0}\otimes \cdots\otimes
a_{i}\cdot a_{i+1}\otimes \cdots\otimes a_{q+1}\q\text{for}\q
i=0,\ldots,q. 
$$

The $A$-bimodule structure on $C'_{q}(A)$ is given by the extended
$A$-bimodule structure where for $a\otimes a'\in A\otimes A^{op}$ we
have
$$
(a\otimes a')(a_{0}\otimes\cdots\otimes a_{q+1})=(aa_{0})\otimes
\cdots\otimes (a_{q+1}a'),
$$
and\pageoriginale from this it is clear that $\phi_{i}$ is an
$A$-bimodule morphism. Finally, note that the morphism
$\phi_{0}:C'_{0}(A)\to A$ is the usual multiplication morphism on $A$.

\begin{definition}\label{chap3-defi1.1}
The standard split resolution of $A$ as an $A$-bimodule is the complex
$(C'_{\ast}(A),b')\to A$ of $A$-bimodules over $A$ where with the
above notations $b':C'_{q}(A)\to C'_{q-1}(A)$ is given by
$b'=\sum\limits_{0\leq i\leq q}(-1)^{i}\phi_{i}$.
\end{definition}

\begin{proposition}\label{chap3-prop1.2}
The standard split resolution\index{standare split resolution} of $A$ is a split projective resolution
of $A$ by $A$-bimodules.
\end{proposition}

\begin{proof}
By construction each $C'_{q}(A)$ is an extended $A$-bimodule. Next, we
have $b'b'=0$ because an easy check shows that
$$
\phi_{i}\phi_{j}=\phi_{j-1}\phi_{i}\q\text{for}\q i<j,
$$
and this gives $b'b'=0$ by an argument where $(q+1)q$ terms cancel in
pairs. Finally the complex is split acyclic with the following
homotopy $s:C'_{q}(A)\to C'_{q+1}(A)$ given by
$s(a_{0}\otimes\cdots\otimes a_{q+1})=1\otimes
a_{0}\otimes\cdots\otimes a_{q+1}$. Since $\phi_{0}s=1$ and
$\phi_{i+1}s=s\phi_{i}$ for $i\geq 0$, we obtain $b's+sb'=1$, the
identity. This proves the proposition.
\end{proof}

To calculate the Hochschild homology\index{Hochschild homology} with the resolution, we must
apply the functor $R$, where $R(M)=A\otimes_{(A\otimes A^{op})}M$, to
the complex of the resolution. Now, for an extended $A$-bimodule
$A\otimes X\otimes A$ the functor has the value $R(A\otimes X\otimes
A)=A\otimes X$ as a $k$-module.

\begin{definition}\label{chap3-defi1.3}
The standard complex $C_{\ast}(A)$ for an algebra $A$ over $k$ is,
with the above notation $C_{\ast}(A)=R(C'_{\ast}(A),b')$.
\end{definition}

In particular, we have $C_{q}(A)=A^{(q+1)\otimes}$ and for
$d_{i}=R(\phi_{i})$ the differential of the complex is
$b=\sum\limits_{0\leq i\leq q}(-1)^{i}d_{i}$ where $d_{i}:C_{q}(A)\to
C_{q-1}(A)$ is given by the following formulas
\begin{gather*}
d_{i}(a_{0}\otimes\cdots\otimes
a_{q})=a_{0}\otimes\cdots\otimes(a_{i}a_{i+1})\otimes \cdots\otimes
a_{q}\q\text{for}\q 0\leq i<q\\
d_{q}(a_{0}\otimes\cdots\otimes a_{q})=(a_{q}a_{0})\otimes
a_{q}\otimes\cdots\otimes a_{q-1}. 
\end{gather*}\pageoriginale

The last formula, the one for $d_{q}$, reflects how the identification
of $A\otimes X$ with $R(A\otimes X\otimes A)$ is made from the right
action of $A$ on $A$ becoming the left action on $A^{op}$. Again we
have $d_{i}d_{j}=d_{j-1}d_{i}$ for $i<j$.

Further, as a complex over $k$, we see clearly that
$C_{q}(A)=C'_{q-1}(A)$ with $d_{i}=\phi_{i}$ for $i<q$. If
$b'=\sum\limits_{0\leq i<q}(-1)^{i}d_{i}:C_{q}(A)\to C_{q-1}(A)$, then
from (\ref{chap3-prop1.2}) we deduce immediately that
$(C_{\ast}(A),b')$ is acyclic. In terms of $b'$, it is clear that
$b=b'+(-1)^{q}d_{q}$. 

\begin{remark}\label{chap3-rem1.4}
The Hochschild homology $HH_{\ast}(A)=H_{\ast}(A,A)$ of $A$ can be
calculated as $H_{\ast}(C_{\ast}(A))$, the homology of the standard
complex of $A$.
\end{remark}

\section{The standard complex as a simplicial
  object}\label{chap3-sec2}\index{simplicial object}

\begin{remark}\label{chap3-rem2.1}
Besides the operators $d_{i}$ on the standard complex $C_{\ast}(A)$,
there are operators $s_{j}$ where $s_{j}:C_{q}(A)\to C_{q+1}(A)$ for
$0\leq j\leq q$ defined by the following formula
$$
s_{j}(a_{0}\otimes \cdots a_{q})=a_{0}\otimes \cdots\otimes
a_{j}\otimes 1\otimes a_{j+1}\otimes\cdots\otimes a_{q}\q\text{for}\q
0\leq j\leq q.
$$

With both the operators $d_{i}$ and $s_{j}$, the standard complex
becomes what is called a simplicial $k$-module. We define now the
general concept of a simplicial object over a category.
\end{remark}

\begin{definition}\label{chap3-defi2.2}
Let $\C$ be a category. A simplicial object $X_{\ast}$ in the category
$\C$ is a sequence of objects $X_{q}$ in $\C$ together with morphisms
$d_{i}:X_{q}\to X_{q-1}$ for $q>0$ and $s_{j}:X_{q}\to X_{q+1}$ for
$0\leq i$, $j\leq q$ satisfying the following relations
\begin{enumerate}
\renewcommand{\labelenumi}{(\theenumi)}
\item $d_{i}d_{j}=d_{j-1}d_{i}$ for $0<j-1$,

\item $s_{j}s_{i}=s_{i}s_{j-1}$ for $0<j-i$,

\item $d_{i}s_{j}=
\begin{cases}
s_{j-1}d_{i} & \text{for~ } 0<j-i\leq q\\
\text{identity} & \text{for~ } -1\leq j-i\leq 0\\
s_{j}d_{i-1} & \text{for~ } j-i<-1.
\end{cases}
$
\end{enumerate}

A\pageoriginale morphism $f:X_{\ast}\to Y_{\ast}$ of simplicial
objects over the category $\C$ is a sequence $f_{q}:X_{q}\to Y_{q}$ of
morphisms in $\C$ such that $d_{i}f=fd_{i}$ and $s_{j}f=fs_{j}$,
i.e.\@ a sequence of morphisms commutating with the simplicial
operations. Composition of $f:X_{\ast}\to Y_{\ast}$ and $g:Y_{\ast}\to
Z_{\ast}$ is the sequence $g_{q}f_{q}$ defined $gf:X_{\ast}\to
Z_{\ast}$. 
\end{definition}

Simplicial objects in a category $\C$, morphisms of simplicial
objects, and composition of morphisms define the category $\Delta(\C)$
of simplicial objects in $\C$.

Originally, simplicial objects arose in the context of the singular
complex of a space which is an example of a simplicial set, and by
considering the $k$-module in each degree with the singular simplexes
as basis, we come to a simplicial $k$-module $C_{\ast}(A)$ associated
with an algebra $A$ over $k$.

Already, for the standard simplicial $k$-module $C_{\ast}(A)$ we have
associated a positive complex with boundary operator defined in terms
of the operators $d_{i}$. This can be done for any simplicial object
over an abelian category. Let $\C^{+}(\A)$ denote the category of
positive complexes over an abelian category $\A$.

\begin{notation}\label{chap3-not2.3}
For a simplicial object $X_{\ast}$ in an abelian category $\A$ we use
the following notations
\begin{alignat*}{2}
 & b=d= &\q& \sum_{0\leq i\leq q}(-1)^{i}d_{i}\\
 & b'=d'= &\q& \sum\limits_{0\leq i<q}(-1)^{i}d_{i},\\
 & s=(-1)^{q}s_{q}: &\q& X_{q}\to X_{q+1}.
\end{alignat*}
\end{notation}

\begin{remark}\label{chap3-rem2.4}
The functors which assign to a simplicial object $X_{\ast}$ in
$\Delta(\A)$ either the complex $(X_{\ast},d)$ or the complex
$(X_{\ast},d')$, and to morphisms in $\Delta(\A)$ the corresponding
morphisms of complexes, are each functors defined $\Delta(\A)\to
\C^{+}(\A)$. By a direct calculation, $s$ is a homotopy operator for
$d'$\pageoriginale of the identity to zero, that is,
$$
d's+sd'=1.
$$

This means that $(X_{\ast},d')$ is an acyclic complex or
equivalently\break $H_{\ast}(X_{\ast},d')=0$. 
\end{remark}

\begin{notation}\label{chap3-not2.5}
We define a filtration $F^{\ast}X$ and two subcomplexes $D(X)$ and
$N(X)$ of the complex $(X,d)$ associated with the simplicial object
$X$ in $\A$. For the filtration in degree $n$, we define
$$
F^{p}X_{n}=\bigcap_{n-p<i\leq n,0<i}\ker(d_{i}).
$$

The subcomplex of degeneracies\index{subcomplex of degeneracies} $D_{n}(X)$ in degree $n$ is the
subobject of $X_{n}$ generated by $\Iim(s_{i})$ for $i=0,\ldots,n-1$,
and the Moore subcomplex\index{Moore subcomplex} $N_{n}(X)$ in degree $n$ is $F^{n}X_{n}$. In
other words, the Moore subcomplex is the intersection of the
filtration $\displaystyle{N(X)=\bigcap_{q}F^{q}(X)}$, and the boundary $d$ is just
$d_{0}:N_{q}(X)\to N_{q-1}(X)$.
\end{notation}

The next theorem is proved by retracting $F^{p}X$ into $F^{p+1}X$ with
a morphism of complexes homotopic to the inclusion morphism of
$F^{p+1}X$ into $F^{p}X$. For the proof of the theorem, we refer to
MacLane 1963, VIII. 6.

\begin{theorem}\label{chap3-thm2.6}
Let $X$ be a simplicial object in an abelian category $\A$. The
following composite is an isomorphism
$$
N_{\ast}(X)\to X_{\ast}\to X_{\ast}/D_{\ast}(X),
$$
and the induced homology morphisms
$$
H_{\ast}(N(X))\to H_{\ast}(X)\q\text{and}\q H_{\ast}(X)\to
H_{\ast}(X/D(X))
$$
are each isomorphisms.
\end{theorem}

\noindent
{\bf Normalized standard complex 2.7.}\index{normalized standard complex}~ Let $A$ be an algebra over
$k$. The subcomplex of degeneracies in degree $q$ is $DC_{q}(A)$ and
is generated by all elements\pageoriginale $a_{0}\otimes\cdots\otimes
a_{q}$ such that $a_{i}=1$ for some $i$, $1\leq i\leq q$. Thus there
is a natural isomorphism of $\overline{C}_{q}(A)=C_{q}(A)/DC_{q}(A)$
with $A\otimes \overline{A}^{q\otimes}$. The graded $k$-module
$\overline{C}_{\ast}(A)$ has a quotient complex structure, and by 
(\ref{chap3-thm2.6}) the quotient morphism $C_{\ast}(A)\to
\overline{C}_{\ast}(A)$ induces an isomorphism in homology, i.e.\@
$HH_{\ast}(A)\to H_{\ast}(\overline{C}_{\ast}(A))$ is an
isomorphism. The complex $\overline{C}_{\ast}(A)$ is called the
normalized standard complex. In the case of the standard complex, the
fact that $C_{\ast}(A)\to \overline{C}_{\ast}(A)$ induces an
isomorphism in homology can be seen directly, by noting that
$\overline{C}_{\ast}(A)$ is obtained as $A^{op}\otimes_{(A\otimes
  A^{op})}\overline{C}'_{\ast}(A)$ in the quotient resolution of
$(C_{\ast}(A),b')$ where $\overline{C}'_{\ast}(A)$ is defined by
$$
\overline{C}'_{q}(A)=A\otimes \overline{A}^{q\otimes}\otimes A.
$$

The normalized complex is useful for comparing Hochschild homology
with differential forms. We treat this in greater detail later.

\section{The standard complex as a cyclic
  object}\label{chap3-sec3}\index{cyclic object}

\begin{remark}\label{chap3-rem3.1}
Besides the operators making $C_{\ast}(A)$ into a simplicial
$k$-module, there is a cyclic permutation operator $t:C_{q}(A)\to
C_{q}(A)$ defined by the following formula
$$
t(a_{0}\otimes \cdots\otimes a_{q})=a_{q}\otimes a_{0}\otimes
\cdots\otimes a_{q-1}.
$$

With the simplicial operators and this cyclic permutation in each
degree, the standard complex becomes what is called a cyclic
$k$-module. We now define the general concept of a cyclic object in a
category. 
\end{remark}

\begin{definition}\label{chap3-defi3.2}
Let $\C$ be a category. A cyclic object $X_{\bigdot}$ in the category
$\C$ is a simplicial object together with a morphism $t_{q}:X_{q}\to
X_{q}$ for each $q\geq 0$ satisfying:
\begin{enumerate}
\renewcommand{\labelenumi}{(\theenumi)}
\item The $(q+1)^{th}$-power $(t_{q})^{q+1}=X_{q}$, the identity on
  $X_{q}$,

\item As morphisms $X_{q}\to X_{q-1}$ we have
  $d_{i}t_{q}=t_{q-1}d_{i-1}$ for $i>0$ and $d_{0}t_{q}=d_{q}$, and 

\item As\pageoriginale morphisms $X_{q}\to X_{q+1}$ we have
  $s_{j}t_{q}=t_{q+1}s_{j-1}$ for $j>0$ and
  $s_{0}t_{q}=(t_{q+1})^{2}s_{q}$. 
\end{enumerate}

A morphism $f:X_{\bigdot}\to Y_{\bigdot}$ of cyclic objects in $\C$ is
a morphism of the simplicial objects $f:X\to Y$ associated with the
cyclic objects such that $t_{q}f_{q}=f_{q}t_{q}$ as morphisms of
$X_{q}\to Y_{q}$. The composition of cyclic morphisms as simplicial
morphisms is again a cyclic morphism. We denote the category of all
cyclic objects in $\C$ and their morphisms by $\Lambda(\C)$. 
\end{definition}

For each algebra $A$, we denote the cyclic object determined by the
standard complex by $C_{\bigdot}(A)$. We leave it to the reader to
check that the above axioms (1), (2) and (3) are satisfied. The
following discussion is carried out for $C_{\bigdot}(A)$, but in fact,
it holds for any cyclic object over an abelian category.

\begin{notation}\label{chap3-not3.3}
Let $T=(-1)^{q}t:C_{q}(A)\to C_{q}(A)$, and observe that both
$T^{q+1}$ and $t^{q+1}$ are equal to the identity map on
$C_{q}(A)$. Let $N:C_{q}(A)\to C_{q}(A)$ be defined by
$N=1+T+T^{2}+\cdots+T^{q}$, and observe that $N(1-T)=0=(1-T)N$. In
order to prove the next commutativity proposition, it is handy to have
the following operator $J=d_{0}T:C_{q}(A)\to C_{q-1}(A)$, because it
satisfies the relations
$$
\begin{cases}
T^{i}JT^{-i-1}=(-1)^{i}d_{i}\q\text{for}\q 0\leq i<q\\
T^{q}JT^{-q-1}=J
\end{cases}
$$
\end{notation}

\begin{proposition}\label{chap3-prop3.4}
For an algebra $A$ the following diagrams are commutative,
\[
\xymatrix{
C_{q}(A)\ar[d]_{b} \ar[r]^{N} & C_{q}(A)\ar[d]^{b'}\\
C_{q-1}(A)\ar[r]^{N} & C_{q-1}(A) 
}
\qquad\quad
\xymatrix{
C_{q}(A)\ar[d]_{b'}\ar[r]^{1-T} & C_{q}(A)\ar[d]^{b}\\
C_{q-1}(A)\ar[r]^{1-T} & C_{q-1}(A).
}
\]
\end{proposition}

\begin{proof}
We first note that
{\selectfont{\fontsize{10}{8}{
$$
b(1-T)=\left(\sum^{q}_{i=0}(-1)^{i}d_{i}\right)
\left(1-(-1)^{q}t_{q+1}\right) =
\sum^{q-1}_{i=0}(-1)^{i}d_{i}-(-1)^{q-1}\sum^{q-1}_{i=0}(-1)^{i}t_{q}d_{i},  
$$}}}
since\pageoriginale $d_{i}t_{q+1}=t_{q}d_{i-1}$ for $0<i\leq n$ and
$d_{0}t_{q+1}=d_{q}$. But the last expression is just $(1-T)b'$
proving that the second diagram is commutative.

For the commutativity of the first diagram, we use $NT^{i}=T^{i}N=N$
for all $i$. Using the operator $J$ introduced above in
(\ref{chap3-not3.3}), we have
\begin{align*}
b'N &= JT^{-1}N+TJT^{-2}N+\cdots+T^{q-1}JT^{-q}N\\
&= JN+TJN+\cdots+T^{n-1}JN=(1+T+\cdots+T^{q-1})JN=NJN,
\end{align*}
and similarly
\begin{align*}
Nb &= NJT^{-1}+NTJT^{-2}+\cdots+NT^{q}JT^{-q-1}\\
&=
NJT^{-1}+NJT^{-2}+\cdots+NJT^{-q-1}\\
&=NJ(T^{-1}+T^{-2}+\cdots+T^{-q-1})\\
&=NJN.
\end{align*}

This proves the proposition.
\end{proof}

\begin{remark}\label{chap3-rem3.5}
This proposition is the basis for forming a double complex in the next
section. Since $(C_{\ast}(A),b')$ is an acyclic complex, we consider
two complexes coming from the standard complex and each giving
Hochschild homology. From (\ref{chap3-prop3.4}) the double complexes
with two vertical columns
$(C_{\ast}(A),b)\xleftarrow{1-T}(C_{\ast}(A),-b')$ and
$(C_{\ast}(A),-b')\xleftarrow{N}(C_{\ast}(A),b)$ where
$(C_{\ast}(A),b)$ is in horizontal degree $0$ have associated total
single complexes with homology equal to Hochschild homology. Using the
spectral sequence of a filtered complex, we see by filtering on the
horizontal degree that we get Hochschild homology for the homology of
the associated total complex because $E^{1}_{0,q}=HH_{q}(A)$ and
$E^{q}_{p,q}=0$ otherwise.
\end{remark}

\section{Cyclic homology defined by the standard double
  complex}\label{chap3-sec4}\index{cyclic homology}\index{standard
  double complex}

\begin{definition}\label{chap3-defi4.1}
Let $C_{\bigdot}(A)$ denote the cyclic object associated with the
standard complex of an algebra $A$ over $k$. The standard double
complex $CC_{\bigdot\bigdot}(A)$ associated with this cyclic object
and hence also with $A$ is the first quadrant double complex which is
the sequence of vertical columns made\pageoriginale up of even degrees
by $(C_{\ast}(A),b)$ and odd degrees by $(C_{\ast}(A),b')$, with
horizontal structure morphisms given by $1-T$ and $N$ as indicated in
the following display
$$
C_{\ast}(A),b\xleftarrow{1-T}C_{\ast}(A),
-b'\xleftarrow{N}C_{\ast}(A), b\xleftarrow{1-T}C_{\ast}(A),-b'\xleftarrow{N}C_{\ast}(A),b\leftarrow_{\bigdot\bigdot}
$$
which is periodic of period 2 horizontally to the right, starting with
$p=0$ in the double complex. The corresponding cyclic complex
$CC_{\bigdot}(A)$ is the associated total complex of
$CC_{\bigdot\bigdot}(A)$. 
\end{definition}

Observe that by (\ref{chap3-prop3.4}), $CC_{\bigdot\bigdot}(A)$ is a
double complex, since we have already remarked that
$(1-T)N=0=N(1-T)$. This construction is made with just the cyclic
object structure, and thus can be made for any cyclic object in an
abelian category.

\begin{definition}\label{chap3-defi4.2}
Let $A$ be an algebra over $k$. The cyclic homology $HC_{\ast}(A)$ of
$A$ is the homology $H_{\ast}(CC_{\bigdot}(A))$ of the standard total
complex of the standard double complex of $A$.
\end{definition}

\begin{remark}\label{chap3-rem4.3}
The standard double complex $CC_{\bigdot\bigdot}(A)$, its associated
total complex $CC_{\bigdot}(A)$, and the cyclic homology
$HC_{\ast}(A)$, are all functors of $A$ on the category of algebras
over $k$, since the standard cyclic object $C_{\bigdot}(A)$ is
functorial in $A$ from the category of algebras over $k$ to the
category of cyclic $k$-modules $\Lambda(k)$.
\end{remark}

\noindent
{\bf Connes' exact couple 4.4.} From the $2$-fold periodicity of the
double complex $CC_{\bigdot\bigdot}(A)$, we have a morphism
$\sigma:CC_{\bigdot\bigdot}(A)\to CC_{\bigdot}(A)$ of bidegree
$(-2,0)$, giving a morphism $\sigma:CC_{\bigdot}(A)\to
CC_{\bigdot}(A)$ of degree $-2$ and a short exact sequence of
complexes
$$
0\to \ker(\sigma)\to
  CC_{\bigdot}(A)\xrightarrow{\sigma}CC_{\bigdot}(A)\to 0.
$$

The homology of $\ker(\sigma)$ was considered in (\ref{chap3-rem3.5})
and we have
$$
H_{\ast}(\ker(\sigma))=HH_{\ast}(A).
$$

The\pageoriginale homology exact triangle of this short exact sequence
of complexes is the Connes' exact triangle
\[
\xymatrix{
HC_{\ast}(A)\ar[rr]^{S} & & HC_{\ast}(A)\ar[dl]^{B}\\
 & HH_{\ast}(A)\ar[ul]^{I} & 
}
\]
where $S=H_{\ast}(\sigma)$ so $\deg(S)=-2$, $\deg(B)=+1$, and
$\deg(I)=0$. Moreover, this defines an functor from the category of
algebras over $k$ to the category of positively $\bfZ$-graded exact
couples $ExC(-2,+1,0)$ over the category of $k$-modules $(k)$.

\begin{remark}\label{chap3-rem4.5}
The entire discussion in this chapter could have been carried out with
$\Theta$-graded $k$-algebras $A$. The $\Theta$-grading plays no role
in any of the definitions. In particular, we have completed the
definition of cyclic homology and the Connes' exact couple introduced
in 1(\ref{chap3-rem3.5}) namely
$$
(HC_{\ast},HH_{\ast},S,B,I):\Alg_{\Theta,k}\to
ExC((k),\bfZ\times\Theta,(-2,0),(1,0),(0,0)). 
$$

Also, the fact that $I:HH_{0}(A)\to HC_{0}(A)$ is an isomorphism
holds, (see 1(3.6)), and if $f:A\to A'$ is a morphism of algebras,
then $HC_{\ast}(f)$ is an isomorphism if and only if $HH_{\ast}(f)$ is
an isomorphism, see 1(3.7).
\end{remark}

\section{Morita invariance of cyclic
  homology}\label{chap3-sec5}\index{Morita invariance}

Let $A$ and $B$ be two algebras, and let ${}_{A}\Mod_{B}$ denote the
category of bimodules with $A$ acting on the left and with $B$ acting
on the right. In other words ${}_{A}\Mod_{B}$ is the category of left
$A\otimes B^{op}$-modules or the category of right $A^{op}\otimes
B$-modules. 

\begin{definition}\label{chap3-defi5.1}
A Morita equivalence between two algebras $A$ and $B$ is given by two
bimodules, $P$ in ${}_{A}\Mod_{B}$ and $Q$ in ${}_{B}\Mod_{A}$
together with isomorphisms
$$
w_{A}:P\otimes_{B}Q\to A\q\text{and}\q w_{B}:Q\otimes_{A}P_{A}\to B
$$
in\pageoriginale the categories ${}_{A}\Mod_{A}$ and ${}_{B}\Mod_{B}$
respectively. Two algebras $A$ and $B$ are said to be Morita
equivalent provided there exists a Morita equivalence between $A$ and
$B$.

The bimodules $P$ and $Q$ define six different functors:
\begin{enumerate}
\renewcommand{\theenumi}{\alph{enumi}}
\renewcommand{\labelenumi}{(\theenumi)}
\item for left modules, $\phi_{P:B}\Mod\to_{A}\Mod$ and
  $\phi_{Q:A}\Mod\to_{B}\Mod$ defined by $\phi_{P}(M)=P\otimes_{B}M$
  and $\phi_{Q}(M')=Q\otimes_{A}M'$, 

\item for right modules $\psi_{P}:\Mod_{A}\to \Mod_{B}$ and
  $\psi_{Q}:\Mod_{B}\to \Mod_{A}$ defined by
  $\psi_{Q}(L)=L\otimes_{A}P$ and $\psi_{Q}(L')=L'\otimes_{B}Q$, and 

\item for bimodules $\phi_{P,Q}:_A\Mod_{A}\to_{B}\Mod_{B}$ and
  $\phi_{Q,P}:_B \Mod_{B}\to_{A}\Mod_{A}$ defined by
  $\phi_{P,Q}(M)=Q\otimes_{A}M\otimes_{A}P$ and
  $\phi_{Q,P}(N)=P\otimes_{B}N\otimes_{B}Q$. 
\end{enumerate}
\end{definition}

\begin{proposition}\label{chap3-prop5.2}
Let $A$ and $B$ be two algebras, and let $(P,Q,w_{A},w_{B})$ be a
Morita equivalence. Then the following hold:
\begin{enumerate}
\renewcommand{\labelenumi}{\rm(\theenumi)}
\item The functors $\phi_{P}:_B\Mod\to_A\Mod$ and
  $\phi_{Q}:_A \Mod\to_{B}\Mod$ are inverse to each other up to
  equivalence.

\item The functors $\psi_{P}:\Mod_{A}\to \Mod_{B}$ and
  $\psi_{Q}:\Mod_{B}\to \Mod_{A}$ are inverse to each other up to
  equivalence.

\item The functors $\phi_{P,Q}:_A \Mod_{A}\to_{B}\Mod_{B}$ and
  $\phi_{Q,P}:_B \Mod_{B}\to_{A}\Mod_{A}$ are inverse to each other up
  to equivalence.
\end{enumerate}

Also, there are natural isomorphisms induced by $w_{A}$ and $w_{B}$
between the functors defined on ${}_{A}\Mod_{A}\times_{A}\Mod_{A}$,
namely
$$
\phi_{P,Q}(M)\otimes_{B\otimes B^{op}}\phi_{P,q}(N)\to
M\otimes_{A\otimes A^{op}}N,
$$
and the corresponding derived functors
$$
\Tor^{B\otimes B^{op}}_{\ast}(\phi_{P,Q}(M),\phi_{P,Q}(N))\to
\Tor^{A\otimes A^{op}}_{\ast}(M,N). 
$$
\end{proposition}

\begin{proof}
As an indication of the proof, we consider an $A$-bimodule $M$. There
is a natural isomorphism
$$
\phi_{Q,P}(\phi_{P,Q}(M))=(P\otimes_{B}Q)\otimes_{A}M\otimes_{A}(P\otimes_{B}Q)\to
A\otimes_{A}M\otimes_{A}A=M,
$$
and\pageoriginale similarly there is a natural isomorphism
$\phi_{P,Q}\phi_{Q,P}\simeq \id$.

The isomorphism between two bimodule tensor products is just an
associativity law for tensor products. This canonical isomorphism
extends to the derived functors from uniqueness properties of the
derived functors. This proves the proposition.
\end{proof}

\begin{corollary}\label{chap3-coro5.3}
Morita equivalent algebras $A$ and $B$ have isomorphic Hochschild
homology.
\end{corollary}

\begin{example}\label{chap3-exam5.4}
The algebras $A$ and the matrix algebra $M_{n}(A)$ are Morita
equivalent. To see this, we observe that the module of $n$ by $q$
matrices $M_{n,q}(A)$ is a left $M_{n}(A)\otimes M_{q}(A)^{op}$-module
and matrix multiplication factors by a tensor product over $M_{q}(A)$
as follows
\[
\xymatrix{
M_{n,q}(A)\otimes M_{q,s}(A)\ar[dr] \ar[rr]^{\text{matrix
    multiplication}} & & M_{n,s}(A)\\
& M_{n,q}\otimes_{M_{q}(A)}M_{q,s}(A)\ar[ur]_{f} & 
}
\]
\end{example}

\noindent
{\bf Assertion.} The morphism $f$ in the previous diagram is an
isomorphism of $M_{n}(A)\otimes M_{s}(A)^{op}$-modules. Clearly $f$ is
a bimodule morphism. To see the isomorphism assertion, we can reduce
to the case $n=s=1$ and consider
$f:M_{1,q}(A)\otimes_{M_{q}(A)}M_{q,1}(A)\to M_{1,1}(A)=A$ and
calculate
\begin{gather*}
f\left((a_{1},\ldots,a_{q})\otimes
\begin{pmatrix}
b_{1}\\
\vdots\\
b_{q}
\end{pmatrix}
\right)=
f
\left(
(a_{1},\ldots,a_{q})\otimes
\begin{pmatrix}
b_{1} & 0 & \ldots & 0\\
\ldots & \ldots & \ldots & \ldots\\
\ldots & \ldots & \ldots & \ldots\\
b_{q} & 0 & \ldots & 0
\end{pmatrix}
\begin{pmatrix}
1\\
0\\
0
\end{pmatrix}
\right)=\\
=f((c,0,\ldots,0)\otimes 
\begin{pmatrix}
1\\
0\\
0
\end{pmatrix}
=c=a_{1}b_{1}+\cdots+a_{q}b_{q}.
\end{gather*}

It is clear from this computation that $f$ is a bijection.

The Morita equivalence between $A$ and $M_{q}(A)$ is given by
$(M_{1,q}(A),\break M_{q,1}(A),f,f)$. There is a morphism of cyclic sets from
the standard complex for $M_{n}(A)$ to\pageoriginale the standard
complex for $A$.

\begin{definition}\label{chap3-defi5.5}
The Dennis trace map
$$
\Tr:M_{n}(A)^{(q+1)\otimes}\to A^{(q+1)\otimes}
$$
is given by
$$
\Tr(a(0)\otimes\cdots\otimes a(q))=\sum\limits_{1\leq
  i_{0},\ldots,i_{q}\leq n} a_{i_{0}i_{1}}(0)\otimes\cdots\otimes
a_{i_{q}i_{0}}(q). 
$$
\end{definition}

\begin{theorem}\label{chap3-thm5.6}
The Dennis trace map\index{Dennis trace map} induces isomorphisms
$HH_{\ast}(M_{n}\break (A))\to
HH_{\ast}(A)$ and $HC_{\ast}(M_{n}(A))\to HC_{\ast}(A)$. 
\end{theorem}

\begin{proof}
It is an isomorphism on Hochschild homology by (\ref{chap3-coro5.3}),
and since this isomorphism is given by a morphism of cyclic objects,
the induced map is an isomorphism on cyclic homology by the criterion
1(3.7). This proves the theorem.
\end{proof}

\begin{remark}\label{chap3-rem5.7}
In McCarthy [1988], there is a proof that in general Morita equivalent
algebras have isomorphic cyclic homology.
\end{remark}

\noindent
{\bf Reference:}~ Compte Rend Acad Sci, 307 (1988), pp. 211-215.

