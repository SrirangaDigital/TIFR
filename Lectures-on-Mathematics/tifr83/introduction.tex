

\chapter{Introduction}

\markboth{Introduction}{Introduction}

This book is based on lectures delivered at the Tata Institute of
Fundamental Research, January 1990. Notes of my lectures and a
preliminary manuscript were prepared by R. Sujatha. My interest in the
subject of cyclic homology started with the lectures of A. Connes in
the Algebraic $K$-Theory seminar in Paris in October 1981 where he
introduced the concept explicitly for the first time and showed the
relation to Hochschild homology. In the year 1984-1985, I collaborated
with Christian Kassel on a seminar on Cyclic homology at the Institute
for Advanced Study. Notes were made on the lectures given in this
seminar. This project was carried further in 1987-1988 while Kassel
was at the Institute for Advanced Study and in 1988-1989 while I was
at the Max Planck Institut f\"ur Mathematik in Bonn. We have a longer
and more complete book coming on the subject. The reader is familiar
with functions of several variables or sets of $n$-tuples which are
invariant under the full permutation group, but what is special about
cyclic homology is that it is concerned with objects or sets which
only have an invariance property under the cyclic group. There are two
important examples to keep in mind. Firstly, a trace $\tau$ on an
associative algebra $A$ is a linear form $\tau$ satisfying
$\tau(ab)=\tau(ba)$ for all $a$, $b\in A$. Then the trace of a product
of $n+1$ terms satisfies
$$
\tau(a_{0}\ldots a_{n})=\tau(a_{i+1}\ldots a_{n}a_{0}\ldots a_{i}).
$$

We will use this observation to construct the Chern character of
$K$-theory with values in cyclic homology. Secondly, for a group $G$,
we denote by $N(G)_{n}$ the subset of $G^{n+1}$ consisting of all
$(g_{0},\ldots,g_{n})$ with $g_{0}\ldots g_{n}=1$. This subset is
invariant under the action of the cyclic group on $G^{n+1}$ since
$g_{0}\ldots g_{n}=1$ implies that $g_{i+1}\ldots g_{n}g_{0}\ldots
g_{i}=1$. This observation will not be used in these notes but can be
used to define the chern character for elements in higher algebraic
$K$-theory. This topic will not be considered here, but it is covered
in our book with Kassel. This book has three parts organized into
seven chapters. The first part, namely chapters \ref{chap1} and
\ref{chap2}, is preliminary to the subject of cyclic homology which is
related to classical Hochschild homology by an exact couple discovered
by Connes. In chapter \ref{chap1}, we survey the part of the theory of
exact couples and spectral sequences needed for the Connes exact
couple, and in chapter \ref{chap2} we study the question of
abelianization of algebraic objects and how it relates to Hochschild
homology. In the second part, chapters \ref{chap3}, \ref{chap4}, and
\ref{chap5}, we consider three different definitions of cyclic
homology. In chapter \ref{chap3}, cyclic homology is defined by the
standard double complex made from the standard Hochschild complex. The
first result is that an algebra $A$ and any algebra Morita equivalent
to $A$, for example the matrix algebra $M_{n}(A)$, have isomorphic
cyclic homology. In chapter \ref{chap4}, cyclic homology is defined by
cyclic covariants of the standard Hochschild complex in the case of a
field of characteristic zero. The main result is a theorem discovered
independently by \cite{Tsygan1983} and \cite{Loday1984} calculating
the primitive elements in the Lie algebra homology of the infinite Lie
algebra $\underline{g\ell}(A)$ in terms of the cyclic homology of
$A$. In chapter \ref{chap5}, cyclic homology is defined in terms of
mixed complexes and the Connes' $B$ operator. This is a way, due to
Connes, of simplifying the standard double complex, and it is
particularly useful for the incorporation of the normalized standard
Hochschild into the calculation of cyclic homology. The third part,
chapters \ref{chap6} and \ref{chap7}, is devoted to relating cyclic
and Hochschild homology to differential forms and showing how
$K$-theory classes have a Chern character in cyclic homology over a
field of characteristic zero. There are
 two notions of differential
forms depending on the commutativity properties of the algebra. In
chapter \ref{chap6}, we study the classical K\"ahler differential
forms for a commutative algebra, outline the proof of the classical
Hochschild-Kostant-Rosenberg theorem relating differential forms and
Hochschild homology, and relate cyclic homology to deRham
cohomology. In chapter \ref{chap7} we study non-commutative
differential forms for algebras and indicate how they can be used to
define the Chern character of a $K$-theory class, that is, a class of
an idempotent element in $M_{n}(A)$, using differential forms in
cyclic homology. In this way, cyclic homology becomes the natural home
for characteristic classes of elements of $K$-theory for general
algebras over a field of characteristic zero. This book treats only
algebraic aspects of the theory of cyclic homology. There are two big
areas of application of cyclic homology to index theory, for this, see
\cite{Connes1990}, and to the algebraic $K$-theory of spaces $A(X)$
introduced by F. Waldhausen. For references in this direction, see the
papers of Goodwillie.

I wish to thank the School of Mathematics of the Tata Institute of
Fundamental Research for providing the opportunity to deliver these
lectures there, and the Haverford College faculty research fund for
support. I thank Mr. Sawant for the efficient job he did in typing the
manuscript and David Jabon for his help on international transmission
and corrections. The process of going from the lectures to this
written account was made possible due to the continuing interest and
participation of R. Sujatha in the project. For her help, I express my
warm thanks. 
