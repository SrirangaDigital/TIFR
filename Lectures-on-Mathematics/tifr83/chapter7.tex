\chapter{Noncommutative Differential Geometry}\label{chap7}

IN\pageoriginale THE PREVIOUS chapter, we developed the close
relationship between differential forms and de Rham cohomology on one
hand and Hochschild and cyclic homology on the other hand, for
commutative algebras. In this chapter, we explore the relationship in
the general case, using the concept of the bimodule of
differential\index{bimodule differentials}
forms, which we denote by $\Omega^{1}(A/k)$. As before, these forms
are related to $I$, the kernel of the multiplication map
$\phi(A):A\otimes A\to A$, and in fact in this case, we have
$\Omega^{1}(A/k)=I$. 

\section{Bimodule derivations and differential
  forms}\label{chap7-sec1}\index{derivation with values in a bimodule}

In this section let $A$ denote an algebra over $k$.

\begin{definition}\label{chap7-defi1.1}
Let $M$ be an $A$-bimodule. A derivation $D$ of $A$ with values in $M$
is a $k$-linear map $D:A\to M$ such that
$$
D(ab)=aD(b)+D(a)b\text{~ for~ } a,b\in A.
$$

We denote by $\Der_{k}(A,M)$ or just $\Der(A,M)$ the $k$-module of all
bimodule derivations of $A$ with values in $M$.
\end{definition}

Unlike in the commutative case, $\Der(A,M)$ has no $A$-linear
structure, but $\Der(A,A)$ is a Lie algebra over $k$ with Lie bracket
given by $[D,D']=DD'-D'D$ for $D$, $D'\in \Der(A,A)$.

\begin{definition}\label{chap7-defi1.2}
The $A$-bimodule of bimodule differentials is a pair\break 
$(\Omega^{1}(A/k),d)$ where $\Omega^{1}(A/k)$, or simply
$\Omega^{1}(A)$ or $\Omega$, is an $A$-bimodule and the morphism
$d:A\to \Omega^{1}(A/k)$ is a bimodule derivation such that, for any
derivation $D:A\to M$ there exists a unique $A$-linear morphism
$f:\Omega^{1}(A/k)\to M$ such that $D=fd$. The bimodule derivation $d$
defines a $k$-linear morphism
$$
\Hom_{A}(\Omega^{1}(A/k),M)\to \Der_{k}(A,M)
$$
by\pageoriginale assigning to each morphism $f\in
\Hom_{A}(\Omega^{1}(A/k),M)$ of $A$-bimodules the bimodule derivation
$fd\in \Der_{k}(A,M)$. The universal property is just the assertion
that this morphism is an isomorphism of $A$-modules. As usual, the
universal property shows that two possible $k$-modules of
differentials are isomorphic with a unique isomorphism preserving the
derivation $d$. As in the previous chapter, there are two
constructions of the module of derivations $\Omega^{1}(A/k)$. The
first uses $I=\ker(\phi(A))$ and the second uses the relations coming
directly from the derivation property. They are tied together with an
acyclic standard resolution.
\end{definition}

\noindent
{\bf Construction of {\boldmath$\Omega^{1}(A/k)$} I. 1.3.} Let $I$
denote the kernel of the multiplication morphism $\phi(A):A\otimes
A\to A$. We define $\Omega^{1}(A/k)=I$ and $d:A\to I$ by
$d(a)=1\otimes a-a\otimes 1$ for $a\in A$ and check that it is a
derivation by
\begin{align*}
d(ab) &= 1\otimes ab-ab\otimes 1\\
&=(1\otimes a)(1\otimes b-b\otimes
1)+(1\otimes a-a\otimes 1)(b\otimes 1)\\
&= ad(b)+d(a)b
\end{align*}
where the left action of $A$ on $I\subset A\otimes A$ is given by
$ax=(1\otimes a)x$ and the right action by $xb=x(b\otimes 1)$ in $I$
for $x\in I$. To verify the universal property, we consider a
derivation $D:A\to M$. If $\sum_{i}a_{i}\otimes b_{i}\in I$ or in
other words $\sum_{i}a_{i}b_{i}=0$, then we have
$$
\sum_{i}a_{i}(Db_{i})+\sum_{i}(Da_{i})b_{i}=0
$$
from the derivation rule, and we define $f:I\to M$ by
$$
f\left(\sum_{i}a_{i}\otimes
b_{i}\right)=\sum_{i}a_{i}D(b_{i})=-\sum_{i}D(a_{i})b_{i}.
$$

Now $f(d(a))=f(1\otimes a-a\otimes 1)=1D(a)-aD(1)=D(a)$, and hence
$fd=D$. Thus $(\Omega^{1}(A/k),d)$ is a module of bimodule
differentials.

\medskip
\noindent
{\bf Construction of {\boldmath$\Omega^{1}(A/k)$} II. 1.4.}
Following the idea of 6(1.4), we should consider the $k$-submodule $L$
of $A\otimes A\otimes A$ generated by all elements of the form
$a_{0}a_{1}\otimes a_{2}\otimes a_{3}-a_{0}\otimes a_{1}a_{2}\otimes
a_{3}+a_{0}\otimes a_{1}\otimes a_{2}a_{3}$ which is just
$b'(a_{0}\otimes a_{1}\otimes a_{2}\otimes a_{3})$\pageoriginale for
the differential $b':C_{3}(A)\to C_{2}(A)$ in the standard acyclic
complex for the algebra $A$. Since $(C_{*}(A),b')$ is acyclic, we have
a natural isomorphism
$$
A^{3\otimes}/L=\coker(b':A^{4\otimes}\to A^{3\otimes})\to
\ker(b'=\phi(A):A\otimes A\to A)=I.
$$

To see the universal property for $d(a)=1\otimes a\otimes 1$ mod $L$,
we note first that $d$ is a derivation by the properties of the
generators of $L$ and for a derivation $D:A\to M$ we define a morphism
$f:A^{3\otimes}/L\to M$ by the relation $f(a\otimes b\otimes
c\text{\,mod\,}L)=aD(b)c$. 

\setcounter{theorem}{4}
\begin{remark}\label{chap7-rem1.5}
The module $\Omega^{1}(A/k)$ is generated by elements $adb$ for $a$,
$b\in A$ with the left $A$-module structure given by
$$
a'(adb)=(a'a)db
$$
and the right $A$-module structure given by
$$
(adb)a'=ad(ba')-(ab)da'
$$
for $a$, $a'$, $b\in A$.
\end{remark}

Now we proceed to define the bimodule of $q$-forms by embedding
$\Omega^{1}(A/k)$ in a kind of tensor algebra derived from the
$A$-bimodule structure. In this case, we factor tensor products over
$k$ as tensor products over $A$, but we do not introduce any
commutativity properties in the algebra since $A$ is not commutative.

\begin{definition}\label{chap7-defi1.6}
Let $M$ be an $A$-bimodule. The bimodule tensor algebra $T_{A}(M)$ is
the graded algebra where in degree $n$
$$
T_{A}(M)_{n}=M\otimes_{A}\ldots (n)\ldots \otimes_{A}M
$$
with algebra structure over $k$ given by a direct sum of the natural
quotients $T_{A}(M)_{p}\otimes T_{A}(M)_{q}\to T_{A}(M)_{p+q}$. In
particular $T_{A}(M)_{n}$ is generated by elements
$$
x_{1}\otimes_{A}\cdots\otimes_{A}x_{n}=x_{1}\ldots x_{n}\q\text{for}\q
x_{1},\ldots,x_{n}\in M,
$$
and\pageoriginale in degree zero $T_{A}(M)_{0}=A$.
\end{definition}

\section{Noncommutative de Rham cohomology}\label{chap7-sec2}

Now we apply the above constructions, not directly to the algebra $A$,
but to $k\oplus A$ viewed as a supplemented algebra with augmentation
ideal $A$ itself.

\begin{notation}\label{chap7-not2.1}
Let $A^{\sharp}$ denote the algebra $k\oplus A$ given by inclusion
$k\to A^{\sharp}=k\oplus A$ on the first factor. Since $A^{\sharp}$ is
supplemented, we have a splitting $s:A^{\sharp}\to A^{\sharp}\otimes
A^{\sharp}$, of the exact sequence
$$
0\to \Omega^{1}(A^{\sharp})\to A^{\sharp}\otimes A^{\sharp}\to
A^{\sharp}\to 0
$$
defined by $s(a)=a\otimes 1$. Thus there is a natural morphism
$\Omega^{1}(A^{\sharp})\to \coker(s)$ and we have the following result.
\end{notation}

\begin{proposition}\label{chap7-prop2.2}
We have a natural isomorphism
$$
\delta:A\oplus (A\otimes A)\to \Omega^{1}(A^{\sharp})
$$
where $\delta(a,0)=da$ and $\delta(0,a\otimes b)=adb=a(1\otimes
b-b\otimes 1)$. The right $A$-module structure is given by
$(a_{0}da_{1})a=a_{0}d(a_{1}a)-a_{0}a_{1}da$. Now we define the
algebra of all noncommutative forms.
\end{proposition}

\begin{definition}\label{chap7-defi2.3}
The algebra of noncommutative differential forms\index{noncommutative
  differential forms} is the following
tensor algebra $T(\Omega^{1}(A^{\sharp}))$ over $A^{\sharp}$. This is
a graded algebra and $d$ extends uniquely to $d$ on this tensor
algebra satisfying $d^{2}=0$. More explicitly, we have the following
description. 
\end{definition}

\begin{proposition}\label{chap7-prop2.4}
We have a natural isomorphism
$$
\delta:A^{\sharp}\otimes A^{p\otimes}=A^{p\otimes}\oplus
A^{(p+1)\otimes}\to \Omega^{p}(A^{\sharp})
$$
where $\delta(a_{1}\otimes\cdots\otimes a_{p})=da_{1}\ldots da_{p}$
and $\delta(a_{0}\otimes\cdots\otimes a_{p})=a_{0}da_{1}\ldots
da_{p}$. The right $A^{\sharp}$-module structure on
$\Omega^{p}(A^{\sharp})$ is given by the formula
\begin{align*}
(da_{1}\ldots da_{p})b &= da_{1}\ldots d(a_{p}b)-da_{1}\ldots
  d(a_{p-1}a_{p})db\\ 
&{} +da_{1}\ldots
  d(a_{p-2}a_{p-1})da_{p}+\cdots+(-1)^{p}a_{1}da_{2}\ldots da_{p}db. 
\end{align*}

Moreover,\pageoriginale $H^{*}(\Omega^{*}(A^{\sharp}))=k$ which is
illustrated with the following diagram
\[
\xymatrix@=.18cm{
k & & A & & A^{2\otimes} & & A^{(p-1)\otimes} & & A^{\otimes p} & \\
\oplus & \nearrow d & \oplus & \nearrow d & \oplus & \ldots\ \ldots &
\oplus & \nearrow d & \oplus & \ldots\\
A & & A^{2\otimes} & & A^{3\otimes} & & A^{p\otimes} & &
A^{(p+1)\otimes} &   
}
\]
\end{proposition}

\begin{definition}\label{chap7-defi2.5}
The noncommutative de Rham cohomology\index{noncommutative de Rham cohomology} of an algebra $A$ over a field
is $H^{*}_{NDR}(A)=H^{*}(\Omega^{*}(A^{\sharp})^{\alpha\beta})$, the
cohomology of the Lie algebra abelianization of the differential
algebra of noncommuative differential forms over $A^{\sharp}$. More
precisely, for $\omega\in \Omega^{p}(A^{\sharp})$ and $\omega'\in
\Omega^{q}(A^{\sharp})$ we form the (graded) commutator
$[\omega,\omega']=\omega-(-1)^{pq}\omega'\omega$ and denote by
$[\Omega^{*}(A^{\sharp}),\Omega^{*}(A^{\sharp})]$ the Lie subalgebra
generated by all commutators. The Lie algebra abelianization of the
algebra of differential forms is
$$
\Omega^{*}(A^{\sharp})^{\alpha\beta}=\Omega^{*}(A^{\sharp})/\{k\oplus
      [\Omega^{*}(A^{\sharp}), \Omega^{*}(A^{\sharp})]\}.
$$

To obtain an other version of $\Omega^{*}(A^{\sharp})^{\alpha\beta}$,
we use the following result.
\end{definition}

\begin{proposition}\label{chap7-prop2.6}
Let $S$ be a set of generators of an algebra $B$. For a $B$-module $M$
we have $[B,M]=\displaystyle{\sum_{b\in S}[b,M]}$. 
\end{proposition}

\begin{proof}
First, we calculate
\begin{align*}
[bb',x]&=(bb')x-x(bb')\\
&=b(b'x)-(b'x)b+b'(xb)-(xb)b'\\
&=[b,b'x]+[b',xb]. 
\end{align*}

Thus it follows that $[bb',x]\in [b,M]+[b',M]$. Hence the set of all
$b\in B$ with $\displaystyle{[b,M]\subset \sum_{b\in S}[b,M]}$ is a subalgebra of $B$
containing $S$, and therefore it is $B$. This proves the proposition.
\end{proof}

\begin{corollary}\label{chap7-coro2.7}
The abelianization of the algebra of differential forms is
$$
\Omega^{*}(A^{\sharp})^{\alpha\beta}=\Omega^{*}(A^{\sharp})/\{k+[A,\Omega^{*}(A^{\sharp})]+[dA,\Omega^{*}(A^{\sharp})]. 
$$
\end{corollary}

\begin{definition}\label{chap7-defi2.8}
Let\pageoriginale $A$ be an algebra over $k$. The noncommutative de
Rham cohomology of $A$ is
$$
H^{*}_{NDR}(A)=H^{*}(\Omega^{*}(A^{\sharp})^{\alpha\beta}).
$$

Since $\Omega^{*}(A^{\sharp})^{\alpha\beta}$ is a functor from the
category of algebras over $k$ to the category of cochain complexes
over $k$, the noncommutative de Rham cohomology is a graded
$k$-module, but is does not have any natural algebra structure.
\end{definition}

\section[Noncommutative de Rham cohomology and...]{Noncommutative de
  Rham cohomology and cyclic 
  homology}\label{chap7-sec3} 

Now we relate the noncommutative de Rham cohomology with cyclic
homology over a field $k$ of characteristic zero following ideas from
the theory of commutative algebras where the morphism $\mu$ is used.

\begin{notation}\label{chap7-not3.1}
Again we denote by
$$
\mu:C_{q}(A)\to \Omega^{*}(A^{\sharp})^{\alpha\beta}
$$
the morphism $\mu(a_{0}\otimes\cdots\otimes
a_{q})=(1/q!)a_{0}da_{1}\ldots da_{q}$. 
\end{notation}

\begin{proposition}\label{chap7-prop3.2}
The morphism $\mu$ satisfies the following identities
\begin{enumerate}
\renewcommand{\labelenumi}{\rm\theenumi.}
\item $\mu b(a_{0}\otimes\cdots\otimes
  a_{q+1})=((-1)^{q+1}/q!)[a_{q+1},a_{0}da_{1}\ldots da_{q}]$

\item $\mu(1-t)(a_{0}\otimes\cdots\otimes a_{q})\equiv
  (1/q!)[a_{0}da_{1}\ldots da_{q-1},da_{q}]{\rm mod\,}
  d\Omega^{q-1}(A^{\sharp})$. 
\end{enumerate}
\end{proposition}

\begin{proof}
The composite $\mu b$ is zero for a commutative algebra, see 6(5.3),
but this time the sum will not have the same cancellations in the last
two terms. We have
\begin{align*}
q!\mu b & (a_{0}\otimes\cdots\otimes a_{q+1}) = a_{0}a_{1}da_{2}\ldots
da_{q+1}+\\
&\q \sum_{0<i<q+1}(-1)^{i}a_{0}da_{1}\ldots d(a_{i}a_{i+1})\ldots
da_{q+1})\\
&\q +(-1)^{q+1}a_{q+1}a_{0}da_{1}\ldots da_{q}\\
&= (-1)^{q}a_{0}da_{1}\ldots
da_{q}a_{q+1}+(-1)^{q+1}a_{q+1}a_{0}da_{1}\ldots da_{q}\\
&= (-1)^{q+1}[a_{q+1},a_{0}da_{1}\ldots da_{q}].
\end{align*}

For\pageoriginale the second formula we have the calculation
\begin{align*}
\mu(1-t)& (a_{0}\otimes \cdots\otimes a_{q}) =
\mu(a_{0}\otimes\cdots\otimes a_{q})-(-1)^{q}\mu(a_{q}\otimes
a_{0}\otimes\cdots\otimes a_{q-1})\\
&= (1/q!)(a_{0}da_{1}\ldots da_{q}-(-1)^{q}a_{q}da_{0}\ldots
da_{q-1})\\
&\equiv (1/q!)(a_{0}da_{1}\ldots
da_{q}+(-1)^{q}da_{q}a_{0}da_{1}\ldots da_{q-1}){\rm
  mod\,}d\Omega^{q-1}\\ 
&\equiv (1/q!)[a_{0}da_{1}\ldots da_{q-1},da_{q}]{\rm
  mod\,}d\Omega^{q-1}. 
\end{align*}

From this proposition we state the following theorem of Connes' where
only the question of injectivity in the first assertion is not covered
by the above proposition. As for the second assertion, this is a
deeper result of Connes which we do not go into, see \cite{Connes1985}.
\end{proof}

\begin{theorem}\label{chap7-thm3.3}
The morphism $\mu$ induces an isomorphism
{\fontsize{10}{12}\selectfont
$$
\mu:A^{(q+1)\otimes}/((1-t)A^{(q+1)\otimes}+bA^{(q+2)\otimes})\to
\Omega^{q}/(d\Omega^{q-1}+[dA,\Omega^{q-1}]+[A,\Omega^{q}]) 
$$}
where, as usual, $\Omega^{q}=\Omega^{q}(A^{\sharp})$. The left hand
side has $HC_{q}(A)$ as a submodule and $\mu$ restricted to the
submodule
$$
\mu:\ker(B)=\Iim(S)\to H^{q}_{NDR}(A)
$$
is an isomorphism on the noncommutative de Rham cohomology of $A$
viewed as a submodule of
$\Omega^{q}/(d\Omega^{q-1}+[dA,\Omega^{q-1}]+[A,\Omega^{q}])$. 
\end{theorem}

\section[The Chern character and the suspension...]{The Chern
  character and the suspension in noncommutative de 
  Rham cohomology}\label{chap7-sec4}\index{Chern character}\index{suspension}

\begin{example}\label{chap7-exam4.1}
Let $A=ke$ where $e=e^{2}$ is the identity in the algebra $A$ and an
idempotent in $A^{\sharp}=k\oplus ke$. Then $\Omega^{1}(A^{\sharp}/k)$
is free on two generators $de$ and $ede$, and
\begin{align*}
\Omega^{q}(A^{\sharp}/k)^{\alpha\beta} &= k.e(de)^{q}\text{~ for~ }
q=2i\\
&=0\text{~ for~ }q \text{~ odd.}
\end{align*}
\end{example}

\begin{remark}\label{chap7-rem4.2}
With this calculation we can carry out the construction
of\pageoriginale $ch_{q}(e)$ for $e^{2}=e\in A$ for an arbitrary
algebra $A$ over $k$. Namely, we map the universal $e$ to the special
$e\in A$, and this lifts to $\Omega^{*}(ke^{\sharp})\to
\Omega^{*}(A^{\sharp})$ as differential algebras by the universal
property of the tensor product and hence to
$$
\Omega^{*}(ke^{\sharp})^{\alpha\beta}\to
\Omega^{*}(A^{\sharp})^{\alpha\beta}
$$
as complexes and to $H^{*}_{NDR}(ke)\to H^{*}_{NDR}(A)$. The image of
$d(de)^{2q}/q!$ is $ch_{q}(E)$. Now we consider the $S$ operator in
noncommuative de Rham theory which has the property that 
$$
\frac{S(e(de)^{2q})}{q!}=\frac{e(de)^{2q-2}}{(q-1)!}
$$
\end{remark}

\begin{remark}\label{chap7-rem4.3}
The natural isomorphism $A\to A\otimes ke$ extends to a morphism of
differential algebras
$$
\Omega^{*}(A^{\sharp})\to \Omega^{*}(A^{\sharp})\otimes
\Omega^{*}(ke^{\sharp})
$$
with quotient morphism
$$
\Omega^{*}(A^{\sharp})^{\alpha\beta}\to
\Omega^{*}(A^{\sharp})^{\alpha\beta}\otimes
\Omega^{*}(ke^{\sharp})^{\alpha\beta}
$$
which on degree $q$ is given by
$$
\Omega^{q}(A^{\sharp})^{\alpha\beta}\to
\oplus_{i}\Omega^{q-2i}(A^{\sharp})^{\alpha\beta}\otimes
\Omega^{2i}(ke^{\sharp})^{\alpha\beta}. 
$$

Now we consider the map picking out the coefficient of $e(de)^{2}$
which we call $S:\Omega^{q}(A^{\sharp})^{\alpha\beta}\to
\Omega^{q-2}(A^{\sharp})^{\alpha\beta}$. Observe that $S$ is
compatible with $d$ and we have the following formula.
\end{remark}

\begin{proposition}\label{chap7-prop4.4}
For $a_{0}da_{1}\ldots da_{q}\in \Omega^{q}(A^{\sharp})^{\alpha\beta}$
we have
$$
S(a_{0}da_{1}\ldots da_{q})=\sum_{1\leq i\leq q-1}a_{0}da_{1}\ldots
da_{i-1}(a_{i}a_{i+1})da_{i+2}\ldots da_{q}. 
$$
\end{proposition}

\begin{proof}
Let\pageoriginale $\tau:\Omega^{2}(ke^{\sharp})^{\alpha\beta}\to k$ be the linear
functional such that 
$$\tau((de)^{2})=0 \mbox{ and } \tau(e(de)^{2})=1.$$ 
Then
\begin{align*}
S(a_{0}da_{1}\ldots da_{q}) &= (1\otimes\tau)[(a_{0}\otimes
  e)(da_{1}\otimes e+a_{1}\otimes de)\cdots\\ 
&(da_{q}\otimes e+a_{q}\otimes de)]
 +(1\otimes \tau)\\ 
&\left[\left(\sum_{1\leq i\leq q-1}a_{0}da_{1}\ldots
  da_{i-1}(a_{i}a_{i+1})da_{i+2}\ldots da_{q}\right)\otimes e(de)^{2}\right]\\
&= \sum_{1\leq i\leq q-1}a_{0}da_{1}\ldots
da_{i-1}(a_{i}a_{i+1})da_{i+2}\ldots da_{q}. 
\end{align*}

This proves the proposition.
\end{proof}

\begin{corollary}\label{chap7-prop4.5}
We have $S(ch_{q})=ch_{q-1}$.
\end{corollary}

\begin{proof}
Using (\ref{chap7-prop4.4}) we calculate
\begin{align*}
S(e(de)^{2q}) &= e^{3}(de)^{2q-2}+e(de)ee(de)^{2q-2}+\cdots\\
&= qe(de)^{2q-2}
\end{align*}
and hence we have the result indicated above, that
$$
S\left(\frac{e(de)^{2q}}{q!}\right)=\left(\frac{e(de)^{2q-2}}{(q-1)!}\right). 
$$

This is the statement of the corollary.
\end{proof}

\backmatter

\chapter{Bibliography}

THIS\pageoriginale IS A short list of some of the basic references in
the subject. For further references, see P. Cartier [1985],
S\'eminaire Bourbaki 621 (f\'ev. 1984): Asterisque 121-122 (1985),
123-146, or the book of \cite{Connes1990}. Most of the references are
related to the algebraic aspect of the theory.


\begin{thebibliography}{99}
\bibitem[\protect\citeauthoryear{H. Cartan and S. Eilenberg}{1956}]{Cartan1956}
  H. Cartan and S. Eilenberg [1956], {\em Homological
  algebra}, Princeton University Press, 1956.

\bibitem[\protect\citeauthoryear{A. Connes}{1983}]{Connes1983}
  A. Connes [1983], {\em Cohomology cyclique et foncteurs 
  Ext$^{n}$}, C. R. Acad. Sc. Paris 296 (1983), 953-958.


\bibitem[\protect\citeauthoryear{Connes}{1985}]{Connes1985}
\quad-[1985], {\em Non-commutative differential geometry}, Publ. IHES 62
(1985), 41-144.

\bibitem[\protect\citeauthoryear{Connes}{1990}]{Connes1990}
\quad-[1990], {\em Geometrie non-commutative}, InterEditions, Paris.

\bibitem[\protect\citeauthoryear{Eilenberg and Moore}{1962}]{Eilenberg1962} 
S. Eilenberg and J. C. Moore
  [1961], {\em Limits and 
  spectral sequences}, Topology 1 (1961), 1-23.

\bibitem[\protect\citeauthoryear{Feigin and Tsygan}{1983}]{Feigin1983} 
    B. L. Feigin and B. L. Tsygan
  [1983], {\em Cohomology 
  of Lie algebras of generalized Jacobi matrices},
  Funct. Anal. Appl. 17 (1983), 153-155.

\bibitem[\protect\citeauthoryear{Feigin and Tsygan}{1985}]{Feigin1985} 
\quad-[1985], {\em Additive $K$-Theory and crystalline cohomology},
Funct. Anal. Appl. 19 (1985), 124-132.

\bibitem[\protect\citeauthoryear{Feigin and Tsygan}{1987}]{Feigin1987} 
\quad-[1987], {\em Additive $K$-Theory}, Springer Lecture Notes in
Math. 1289 (1987), 67-209.

\bibitem[\protect\citeauthoryear{Goodwillie}{1985}]{Goodwillie1985} 
  T. G. Goodwillie [1985], {\em Cyclic homology,
  derivations, and the free loopspace}, Topology 24 (1985), 187-215.

\bibitem[\protect\citeauthoryear{Goodwillie}{1985}]{Goodwillie1985a} 
\quad-[1985'], {\em On the general linear group and Hochschild homology},
Ann. of Math. 121 (1985), 383-407; Corrections: Ann. of Math. 124
(1986), 627-628.

\bibitem[\protect\citeauthoryear{Goodwillie}{1985}]{Goodwillie1986} 
\quad-[1986], {\em Relative algebraic $K$-theory and cyclic homology},
Ann. of Math. 124 (1986), 347-402.

\bibitem[\protect\citeauthoryear{Hochschild etal.}{1962}]{Hochschild1962} 
  G. Hochschild, B. Kostant and A. Rosenberg [1962], {\em
  Differential forms on regular affine algebras}, Trans. AMS 102
  (1962), 383-408.

\bibitem[\protect\citeauthoryear{Hsiang and Staffeldt}{1982}]{Hsiang1982} 
  W. C. Hsiang\pageoriginale and R. E. Staffeldt [1982],
  {\em A model for computing rational algebraic $K$-theory of simply
    connected spaces}, Invent. math. 68 (1982), 383-408.

\bibitem[\protect\citeauthoryear{Kassel}{1987}]{Kassel1987} 
  C. Kassel [1987] {\em Cyclic homology, comodules, and
  mixed complexes,} J. of Algebra, 107 (1987), 195-216.

\bibitem[\protect\citeauthoryear{Kassel}{1988}]{Kassel1988} 
\quad-[1988], {\em L'homologie cyclique des alg\`ebres enveloppantes},
Invent. math 91 (1988), 221-251.

\bibitem[\protect\citeauthoryear{Loday-Quillen}{1984}]{Loday1984} 
   J.-L. Loday and D. Quillen [1984], {\em Cyclic homology
  and the Lie algebra homology of matrices,} Comment. Math. Helvetici
  59 (1984), 565-591.

\bibitem[\protect\citeauthoryear{Quillen}{1969}]{Quillen1969} 
  D. Quillen [1969], {\em Rational homotopy theory},
  Annals of Math. 90 (1969), 205-285.

\bibitem[\protect\citeauthoryear{Quillen}{1985}]{Quillen1985} 
\quad-[1985], {\em Superconnections and the Chern character}, Topology 24
(1985), 89-95.

\bibitem[\protect\citeauthoryear{Quillen}{1989}]{Quillen1989} 
\quad-[1989], {\em Algebra cochains and cyclic homology}, Publ. Math. IHES,
68 (1989), 139-174.

\bibitem[\protect\citeauthoryear{Quillen}{1990}]{Quillen1990} 
\quad-[1990], {\em Chern Simons forms and cyclic homology}, The interface
of mathematics and particle physics, Clarendon Press, Oxford (1990).


\bibitem[\protect\citeauthoryear{Tsygan}{1983}]{Tsygan1983} 
  B. L. Tsygan [1983], {\em Homology of matrix algebras over
  rings and Hochschild homology, Russian Math. Surveys} 38:2 (1983), 198-199.

\bibitem[\protect\citeauthoryear{Tsygan}{1986}]{Tsygan1986} 
\quad-[1986], {\em Homologies of some matrix Lie superalgebras},
Funct. Anal. Appl. 20:2 (1986), 164-165.

\bibitem[\protect\citeauthoryear{Wodzicki}{1987}]{Wodzicki}
  M. Wodzicki [1987], {\em Cyclic homology of 
  differential operators}, Duke Math. J. 5 (1987), 641-647.
\end{thebibliography}


