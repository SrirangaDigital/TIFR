
\makeatletter
\def\@makechapterhead#1{%
  \vspace*{50\p@}%
  {\parindent \z@ \raggedright \normalfont
    \ifnum \c@secnumdepth >\m@ne
      \if@mainmatter
        \huge\bfseries Part\space \thechapter
       \par\nobreak
        \vskip 20\p@
      \fi
    \fi
    \interlinepenalty\@M
    \Huge \bfseries #1 \par\nobreak
    \vskip 40\p@
  }}
\makeatother

\chapter{Elkik's Theorems on Algebraization}\label{part2}

\section{Solutions of systems of equations}\pageoriginale\label{part2-sec1}

Let $A$ be a commutative noetherian ring with $1$ and $B$ a
commutative finitely generated $A$-algebra, i.e.,
$B=A[X_{1},\ldots,X_{N}]/(f_{1},\ldots,f_{q})$,
$F=(f_{1},\ldots,f_{q})$. Then finding a {\em solution} in $A$ to
$F(x)=0$, i.e., finding a vector $x=(a_{1},\ldots, a_{N})\in A^{N}$
such that
$$
f_{i}(x)=0\quad (1\leq i\leq q),
$$
is equivalent to finding a sectin for $\Spec B$ over $\Spec A$.

Let $J$ be the Jacobian matrix of the $f_{i}$'s defined by
$$
J=
\begin{bmatrix}
\dfrac{\p f_{1}}{\p X_{1}} & \ldots & \dfrac{\p f_{1}}{\p X_{N}}\\
\vdots & &\\
\dfrac{\p f_{q}}{\p X_{N}} & & \dfrac{\p f_{q}}{\p X_{N}}
\end{bmatrix}
\quad (q\times N)\text{~ matrix}.
$$

We recall that at a point $z\in \Spec B\hookrightarrow \mathbb{A}^{N}$
represented by a prime ideal $\mathscr{P}$ in $A[X_{1},\ldots,X_{N}]$,
$\Spec B$ is {\em smooth} over $\Spec A$ at $z$ if and only if: There
is a subset $(a)=(a_{1},\ldots,a_{p})$ of $(1,2,\ldots,q)$ such that
\begin{itemize}
\item[(i)] there exists a $(p\times p)$ minor $M$ of the $(p\times N)$
  matrix $\dfrac{\p f_{a_{i}}}{\p X_{j}}$ such that $M\nequiv 0(\text{mod~}
  \mathscr{P})$, and 

\item[(ii)] $(f_{1},\ldots,f_{q})$ and $(f_{a_{1}},\ldots,f_{a_{p}})$
  generate the same ideal at $z$.
\end{itemize}

If the conditions (i) and (ii) are satisfied at $z$, then $\Spec B$ is
of relative codimension $p$ in $\mathbb{A}^{N}$ (i.e., relative to
$\Spec A$). 

Let\pageoriginale $F_{(\alpha)}$ be the ideal
$(f_{a_{1}},\ldots,f_{a_{p}})$. The condition (ii) above is equivalent
to the following: There is a $g\in A[X_{1},\ldots,X_{n}]$ such that
$z \not\in V(g)$ and
$$
(F_{(a)})_{g}=(F)_{g}.
$$
(The subscript $g$ means localization with respect to the
multiplicative set generated by $g$. This implies that
$g^{r}(F)\subset (F_{(a)})$ for some $r$. Conversely suppose given a
$g\in A[X_{1},\ldots,X_{N}]$, such that 
$$
g(F)\subset F_{(a)}
$$
(i.e., $g\in$ conductor of $F$ in $F_{a}$, $(F_{(a)}:F)$).

Then at all points $z\in \Spec B$ such that $g(z)\neq 0$, $(F)$ and
$F_{(a)}$ generate the same ideal (since $F_{(a)}\subset F$). Hence
the condition (ii) above can be expressed as:

(ii)~ There is an element $g\in K_{(a)}=$ conductor of $F$ in
$F_{(a)}$ (i.e., the set of elements $g$ such that $gF\subset
F_{(a)}$) such that $g(z)\neq 0$.

Let $\Delta_{(a)}=$ ideal generated by the determinants of the
$(p\times p)$ minors of the $(p\times N)$ matrix $\left(\dfrac{\p
  f_{a_{i}}}{\p X_{j}}\right)$. Let $H$ be the ideal in
$A[X_{1},\ldots,X_{N}]$ defined by
$$
H=\sum_{(a)}K_{(a)}\Delta_{(a)}
$$
i.e., the ideal generated by the ideals $\{K_{(a)}\Delta_{(a)}\}$
where $(a)$ ranges over all subsets of $(1,\ldots,q)$. Then we see
that at a point $z\in \Spec B\hookrightarrow \mathbb{A}^{N}$, $\Spec
B$ is smooth over $\Spec A\Leftrightarrow H$ generates the unit ideal
at $z\Leftrightarrow z\not\in V(H)$. Hence we conclude:

$z\in \Spec B$ is {\em smooth} over $\Spec A$ if and only if $z\not\in
V(H)\cap \Spec B$. 

\section{Existence of solutions when $A$ is $t$-adically
  complete}\label{part2-sec2}\pageoriginale

With the notations as in the above \S\ \ref{part2-sec1}, we have

\begin{theorem}\label{part2-thm2.1}
Suppose further that $A$ is complete with respect to the $(t)$-adic
topology, $(t)$ = principal ideal generated by $t\in A$ (i.e.,
$A=\varprojlim A/(t)^{n}$). Let $I$ be an ideal in $A$. Then there is
a positive integer $q_{0}$ such that whenever
$$
F(a)\equiv 0(\text{mod~}t^{n}I),\ n\geq n_{0},\ n>r
$$
(i.e., $f_{i}(a)=0(\text{mod~}t^{n}I)$, $\forall 1\leq i\leq q$,
$n\geq n_{0}$, $n>r$) and
$$
t^{r}\in H(a)
$$
($H(a)$ is the ideal in $A$ generated by $H$ evaluated at $a$, or
equivalently, the closed subscheme of $\Spec A$ obtained as the
inverse image of $V(H)$ by the section $\Spec
A\xrightarrow{s}\mathbb{A}^{N}$ defined by $(a_{1},\ldots,a_{N})$),
then we can find $(a'_{1},\ldots,a'_{N})\in A^{N}$ such that
$$
F(a')=0\text{~ and~ } a'\equiv a(\text{mod~ } t^{n-r}I). 
$$
\end{theorem}

\begin{remark}\label{part2-rem2.1}
The condition $t^{r}\in H(a)$ implies that $V((t))\supset V(H(a))$. So
the section $s:\Spec A\hookrightarrow \mathbb{A}^{N}$ defined by
$(a_{1},\ldots,a_{N})$ does not pass through $V(H)$ except for the
points $t=0$. Roughly speaking, the above theorem says that a section
$s$ of $\mathbb{A}^{N}_{A}$ over $\Spec A$ which is an approximate
section of $\Spec B$ over $\Spec A$ and not passing through $V(H)$
except over $t=0$ can be approximated by a true section of $\Spec B$
over $\Spec A$.
\end{remark}

\noindent
{\bf Proof of Theorem \ref{part2-thm2.1}.} (1) {\em We claim that it
  suffices to prove the  following}: $\exists h\in A^{N}$ (represented
as a column vector) such that 
\begin{equation*}
\begin{cases}
F(a)\equiv J(a)h(\text{mod~}t^{2n-r}I),\quad\text{and}\\
h\equiv 0(t^{n-r}I).
\end{cases}\tag{*}
\end{equation*}\pageoriginale

To prove this claim we use the Taylor expansion
$$
F(a-h)=F(a)-J(a)h+O(h^{2})\quad\text{(error terms quadratic in $h$)}.
$$

Now $h^{2}\in t^{2(n-r)}I$. We note further that
$$
t^{2n-r}I=t^{r}\cdot t^{2n-2r}I\subset t^{(2n-2r)}I. 
$$

We have $F(a)-J(a)h\in t^{2n-r}I$ and hence
$$
F(a)-J(a)h\in t^{2n-2r}I,
$$
i.e., $(*)$ implies
$$
\begin{cases}
F(a-h)\in t^{2(n-r)}I,\quad\text{and}\\
h\in t^{(n-r)}I.
\end{cases}
$$

Set $a_{1}=a-h$, $a_{0}=a$. Then this gives
\begin{gather*}
F(a_{1})\in t^{2(n-r)}I,\\
a_{1}-a_{0}\in t^{(n-r)}I.
\end{gather*}

Hence by iteration we can find $a_{i}\in A$ such that
\begin{itemize}
\item[(i)] $F(a_{i})\in t^{2^{i}(n-r)}I$, $i\geq 0$, and

\item[(ii)] $(a_{i}-a_{i-1})\in t^{2^{i-1}(n-r)}I$, $i\geq 1$.
\end{itemize}

Now by (ii), $a'=\varprojlim a_{i}$ exists, and $a'\equiv
a(\text{mod~}t^{n-r}I)$. Further (i) implies that $F(a')=0$. This
completes the proof of the claim.

\smallskip
(2)~ {\em We claim that it suffices to prove that there is a $z\in
  A^{N}$ such that}
\begin{align*}
t^{r}F(a) &\equiv J(a)z(\text{mod~}t^{2n}I),\quad\text{and}\\
z &\equiv 0(t^{n}I).
\end{align*}\pageoriginale

For, we see easily that there is an $h\in A^{N}$ such that $t^{r}h=z$
and $h\in t^{n-r}I$. Then, if $J=J_{F}$ denotes the Jacobian for $F$,
we have
\begin{align*}
t^{r}(F(a)-J_{F}(a)h) &\equiv 0(\text{mod~}t^{2n}I),\\
h &\equiv 0(\text{mod~}t^{n-r}I).
\end{align*}

Set $t^{r}F=G$; then these relations can be expressed as
$$
\begin{cases}
G(a)-J_{G}(a)h\equiv 0(\text{mod~}t^{2n}I),\quad\text{and}\\
h\equiv 0(\text{mod~}t^{n-r}I).
\end{cases}
$$

In particular we certainly have
\begin{equation*}
\begin{cases}
G(a)-J_{G}(a)h\equiv 0(\text{mod~}t^{2n-r}I),\quad\text{and}\\
h\equiv 0(\text{mod~}t^{n-r}I).
\end{cases}\tag{**}
\end{equation*}

These are just the same as the relations $(*)$ as in Step (1) above,
with $F$ replaced by $G$. Hence we conclude (as for $F$) that there is
an $a'$ such that 
$$
\begin{cases}
G(a')=0,\quad\text{and}\\
a'\equiv a(\text{mod~}t^{n-r}I).
\end{cases}
$$

Thus we conclude that there is an $a'$ such that
$$
\begin{cases}
t^{r}F(a')=0,\quad\text{and}\\
a'\equiv a(\text{mod~}t^{n-r}I).
\end{cases}
$$

Note that $F(a')\in t^{n-r}I$ since
$$
F(a')=F(a)+J_{F}(a)(a'-a)(\text{mod~}t^{2(n-r)}I)
$$
and $F(a)\in t^{n}I$. Thus we have
$$
t^{r}F(a')=0,\quad\text{and}\quad F(a')\in t^{n-r}I.
$$\pageoriginale

We would like to conclude that $F(a')=0$; this may not be true;
however, we have

\begin{lemma}\label{part2-lem2.1}
Let $T_{q}=\{a\in A|t^{q}a=0\}$ and let $q_{0}$ be an integer such
that $T_{q_{0}}=T_{q_{0}+1}=T_{q_{0}+2}=\ldots$, etc., (note that
$T_{q}\subset T_{q'}$ for $q'\geq q$ and $A$ being noetherian, the
sequence $T_{1}\subset T_{2}\ldots$ terminates). Then
$$
T_{s}\cap (t^{m})A=(0)\quad\text{for}\quad m\geq q_{0}.
$$
\end{lemma}

\begin{proof}
Let $a\in T_{q_{0}}\cap t^{m}A$. Then
$a=t^{m}a'=t^{q_{0}}(t^{m-q_{0}}a')$. Since $t^{q_{0}}a=0$, we have
$t^{n+q_{0}}a'=0$. But by the choice of $q_{0}$, we have in fact
$t^{q_{0}}a'=0$. Hence
$a=t^{q_{0}}t^{m-q_{0}}a'=t^{m-q_{0}}t^{q_{0}}a'=0$. This proves the lemma
\end{proof}

By the lemma if $(n-r)\geq q_{0}$, then $t^{r}F(a')=0$, and $F(a')\in
t^{n-r}I$ implies that $F(a')=0$. Thus the claim (2) is proved.

\smallskip
(3)~ Let $(\beta)=(\beta_{1},\ldots,\beta_{p})$ denote a subset of $p$
elements of $(1,\ldots,N)$. Given
$(\alpha)=(\alpha_{1},\ldots,\alpha_{p})\subset (1,\ldots,q)$, let
$\delta_{\alpha,\beta}$, be the $p\times p$ minor $\left(\dfrac{\p
  f_{\alpha_{i}}}{\p X_{\alpha_{j}}}\right)$ of the Jacobian matrix
$J$. Then we claim that it suffices to prove the following:

Given $k\in K_{(\alpha)}$ and $\delta_{\alpha\beta}\in\Delta_{\alpha}$
($\delta_{\alpha\beta}$ defined as above) then there exists a $z\equiv
0(\text{mod~}t^{n}I)$ such that
$$
k(a)\delta_{\alpha\beta}(a)F(a)\equiv G(a)z(\text{mod~}t^{2n}I).
$$

For, we have
$t^{r}=\sum\limits_{\alpha,\beta}\lambda_{\alpha\beta}k_{\alpha}(a)\delta_{\alpha\beta}(a)$. Then
by hypothesis, given $k_{\alpha}$ and $\delta_{\alpha\beta}$, we have
$z_{\alpha\beta}$ such that $z_{\alpha\beta}=0(\text{mod~}t^{n}I)$ and
$$
k_{\alpha}(a)\delta_{\alpha\beta}(a)F(a)\equiv
J(a)z_{\alpha\beta}(\text{mod~}t^{2n}I); 
$$
then\pageoriginale we see that if we set
$z=\sum\limits_{\alpha\beta}\lambda_{\alpha\beta}z_{\alpha\beta}$, we
have
\begin{align*}
t^{r}\cdot F(a) &\equiv J(a)z(\text{mod~}t^{2n}I),\quad\text{and}\\
z &\equiv 0(\text{mod~}t^{n}I),
\end{align*}
which is the claim (2).

\smallskip
(4)~ We can assume without loss of generality that
$(\alpha)=(1,\ldots,p)$ and $(\beta)=(1,\ldots,p)$ so that
$\delta_{\alpha,\beta}=\det M$ where $M=\left(\dfrac{\p f_{i}}{\p
  X_{j}}\right)$, $\dfrac{1\leq i\leq p}{1\leq j\leq p}$. Then we have
$$
J=
\begin{bmatrix}
M & \ast\\
\ast & \ast
\end{bmatrix}.
$$

Let $N$ be the matrix formed by the determinants of the $(p-1)\times
(p-1)$ minors of $M$ so that we have
$$
MN=\delta\cdot
\text{Id},\ \delta=\delta_{\alpha\beta},\ \alpha,\beta\quad\text{as
  above.}
$$

We denote by $\left[\begin{smallmatrix} N\\ 0\end{smallmatrix}\right]$
the $(N\times p)$ matrix by adding zeros to $N$.

Let $k\in K_{(\alpha)}$, $(\alpha)$ as above and $\delta$ be as
above. {\em Then we claim that if $z$ is defined by}
$$
Z=k
\begin{bmatrix}
N\\
0
\end{bmatrix}
\begin{bmatrix}
f_{1}\\
\vdots\\
f_{p}
\end{bmatrix},\ 
Z\text{~ is an~ }(N\times 1)\text{~ matrix;}
$$
then if $z=Z(a)$, we have
\begin{align*}
k(a)\delta(a)F(a) &\equiv J(a)z(\text{mod~}t^{2n}I),\quad\text{and}\\
z &\equiv 0(\text{mod~}t^{n}I).
\end{align*}

By the foregoing, if we prove this claim, then the proof of the
theorem would be completed.

To\pageoriginale prove this claim, we observe that the relation
$z\equiv 0(\text{mod~}t^{n}I)$ is immediate. Since $k\in
K_{(\alpha)}$, $(\alpha)$ as above, we have
\begin{equation*}
kf_{j}=\sum^{p}_{i=1}\lambda_{ij}f_{i},\ 1\leq j\leq q.\tag{I}
\end{equation*}

This gives $k\dfrac{\p f_{j}}{\p
  X_{1}}=\sum\limits^{p}_{i=1}\lambda_{ij}\dfrac{\p f_{i}}{\p
  X_{1}}(\text{mod~}F)$, $1\leq j\leq q$.

Substituting (a), we get
\begin{align*}
k(a)\frac{\p f_{j}}{\p X_{1}}(a) &\equiv
\sum^{p}_{i=1}\lambda_{ij}\frac{\p f_{i}}{\p
  X_{1}}(a)(\text{mod~}F(a))\\
&\equiv \sum^{p}_{i=1}\lambda_{ij}(a)\frac{\p f_{i}}{\p
  X_{1}}(a)(\text{mod~}t^{n}I). 
\end{align*}

Let $u\in A^{N}$ be an element such that $u\equiv
0(\text{mod~}\mathcal{V})$, where $\mathcal{V}$ is some ideal of
$A$. Let $v=J(a)\cdot u\in A^{N}$. Then
\begin{align*}
k(a)v_{j} &= k(a)\sum^{N}_{\ell=1}\frac{\p f_{j}}{\p
  X_{\ell}}u_{\ell}\\
&\equiv \sum^{N}_{\ell=1}\left(\sum^{p}_{i=1}\lambda_{ij}(a)\frac{\p
  f_{i}}{\p X_{1}}(a)\right)u_{\ell}(\text{mod~}t^{n}I\mathcal{V})\\
&\equiv \sum^{p}_{i=1}\lambda_{ij}(a)\cdot
\left(\sum^{N}_{\ell=1}\frac{\p x_{i}}{\p
  X_{1}}(a)u_{\ell}\right)(\text{mod~}t^{n}I\mathcal{V})\\
&\equiv
\sum^{p}_{i=1}\lambda_{ij}(a)v_{i}(\text{mod~}t^{n}I\mathcal{V}). 
\end{align*}

In other words,
\begin{equation*}
k(a)v_{j}=\sum^{p}_{i=1}\lambda_{ij}(a)v_{i}(\text{mod~}t^{n}I\mathcal{V}),\quad\text{for}\quad
1\leq j\leq q,\tag{II}
\end{equation*}
where\pageoriginale $u\in A^{N}$, $v=J(a)u$, and $u\equiv
0(\text{mod~}\mathcal{V})$.

Now take for $u\in A^{N}$ and $v$ the elements
$$
u=
\begin{bmatrix}
N\\
0
\end{bmatrix}
\begin{bmatrix}
f_{1}\\
\vdots\\
f_{p}
\end{bmatrix}
(a),\quad v=J(a)
\begin{bmatrix}
N(a)\\
0
\end{bmatrix}
\begin{bmatrix}
f_{1}(a)\\
\vdots\\
f_{p}(a)
\end{bmatrix}.
$$

Then $u\equiv 0(\text{mod~}t^{n}I)$ and we have
$$
k(a)v_{j}=\sum^{p}_{i=1}\lambda_{ij}(a)v_{i}(\text{mod~}t^{2n}I).
$$

Moreover,
\begin{align*}
v &= J(a)
\begin{bmatrix}
N(a)\\
0
\end{bmatrix}
\begin{bmatrix}
f_{1}(a)\\
\vdots\\
f_{p}(a)
\end{bmatrix}
=
\begin{bmatrix}
M(a) & *\\
* & *
\end{bmatrix}
\begin{bmatrix}
N(a)\\
0
\end{bmatrix}
\begin{bmatrix}
f_{1}(a)\\
\vdots\\
f_{p}(a)
\end{bmatrix}\\
&= 
\begin{bmatrix}
M & N\\
* & N
\end{bmatrix}
(a)
\begin{bmatrix}
f_{1}(a)\\
\vdots\\
f_{p}(a)
\end{bmatrix}
=
\begin{bmatrix}
\delta(a)f_{1}(a)\\
\vdots\\
\delta(a)f_{p}(a)\\
\vdots\\
w_{p+1}\\
\vdots\\
w_{N}
\end{bmatrix}
\end{align*}

Hence $v_{i}=\delta(a)f_{i}(a)$, $1\leq i\leq p$. By (II),
\begin{align*}
k(a)v_{i} &\equiv
\sum^{p}_{i=1}\lambda_{ij}(a)v_{i}(\text{mod~}t^{2n}I),\text{~ so
  that}\\
k(a)v_{i} &\equiv
\sum^{p}_{i=1}\lambda_{ij}(a)f_{i}(a)\delta(a)(\text{mod~}t^{2n}I). 
\end{align*}

By (I),
$$
\sum^{p}_{i=1}\lambda_{ij}(a)f_{i}(a)\delta(a)=k(a)\delta(a)f_{j}(a).
$$

Hence it follows that
$$
k(a)J(a)
\begin{bmatrix}
N\\
0
\end{bmatrix}
\begin{bmatrix}
f_{i}\\
f_{p}
\end{bmatrix}
\equiv k(a)\delta(a)
\begin{bmatrix}
f_{1}(a)\\
f_{q}(a)
\end{bmatrix}
\quad(\text{mod~}t^{2n}I)
$$\pageoriginale
which is precisely the claim in Step (4) above. This completes the
proof of the theorem as remarked before.

\begin{remark}\label{part1-rem2.1}
A better proof of the theorem is along the following lines: Introduce
a set of relations for $F$ so that we have an exact sequence
$$
P^{\ell}\xrightarrow{R}P^{q}\xrightarrow{F}P\to P/F\to 0.
$$
(Here $F$ denotes $\left[\begin{smallmatrix}
    f_{1}\\ \vdots\\ f_{q}\end{smallmatrix}\right]$ as well as the
ideal generated by $f_{i}$ and $P=A[X_{1},\ldots,X_{N}]$, and $R$ is
the matrix of relations of $F$; it has entries in $P$.) In matrix
notation, we have $F\cdot R=0$. Differentiating with respect to
$X_{1}$, we obtain 
$$
JR\equiv 0(\text{mod~}F),
$$
where $J$ is the Jacobian matrix $\left(\dfrac{\p f_{i}}{\p
  X_{j}}\right)$ (it is an $N\times q$ matrix and is therefore the
transpose of the matrix $J$ introduced in the theorem above). Let
$\overline{P}=P/F$ and define $\overline{J}$, $\overline{R}$
similarly. Then we obtain a {\em complex}
\begin{equation*}
\overline{P}^{\ell}\xrightarrow{\overline{R}}\overline{P}^{q}\xrightarrow{\overline{J}}\overline{P}^{N}\tag{*} 
\end{equation*}
(or: $P^{\ell}\xrightarrow{R}P^{q}\xrightarrow{J}P^{N}$ is a {\em
  complex} mod $F$). It is not difficult to see that at a point $x\in
\Spec (P/F)$, the complex $(*)$ is homotopic to the identity (in
particular, is exact), i.e., denoting by a subscript $x$ the
localization at $x$, there is a diagram
\[
\xymatrix@C=.15cm@R=.1cm{
 & r & & j & & \\
 &  & \ar@/_/[ll] &   &\ar@/_/[ll] & \\
\overline{P}^{1}_{x}\ar[rr]^-{\overline{R}} & &
\overline{P}^{q}_{x}\ar[rr]^-{\overline{J}} & &
\overline{P}^{N}_{x}, & \text{with~ }
\overline{R}r+j\overline{J}=\text{Id}  
}
\]
if\pageoriginale and only if $\Spec (P/F)$ is smooth at $x$. For
example, suppose that $\Spec (P/F)$ is smooth at $x$. Then we can
assume without loss of generality that $\det\left(\dfrac{\p f_{i}}{\p
  X_{i}}\right)$, $\begin{smallmatrix} 1\leq i\leq p\\ 1\leq j\leq
  p\end{smallmatrix}$, is a unit in $\overline{P}_{x}$. From this it
  follows easily that there is a {\em direct summand}
  $\mathcal{Q}_{1}\hookrightarrow \overline{P}^{q}_{x}$,
  $\mathcal{Q}_{1}\approx \overline{P}^{p}_{x}$ such that $\Iim
  (\overline{J})=\Iim(\overline{J}|\mathcal{Q}_{1})$ and
  $\overline{J}|\mathcal{Q}_{1}:\mathcal{Q}_{1}\to \Iim(\overline{J})$
  is an isomorphism ($J|\mathcal{O}_{1}$ denotes the restriction to
  $\mathcal{Q}_{1}$). Indeed, we can take $\mathcal{Q}_{1}$ to be the
  submodule generated by the first $p$-coordinates. (Note we have
  $\overline{J}(e_{i})=\sum\limits^{N}_{\ell=1}\dfrac{\p f_{i}}{\p
    X_{1}}\xi_{1}$, with $(e_{i})$ a basis of $\overline{P}^{q}$, and
  $(\xi_{i})$ a basis of $\overline{P}^{N}$). We see also that
  $\Iim(\overline{R})$ is of rank $(q-p)$ and is a direct summand in
  $\overline{P}^{q}$. In fact the relations
$$
f_{j}=\sum^{p}_{i=1}\lambda_{ij}f_{i},\quad j\geq p+1
$$
give elements of the form
$$
e_{j}=\sum^{p}_{i=1}\lambda_{ij}e_{i},\quad j\geq p+1,
$$
in $\Iim\overline{R}$. Suppose now $\sum\limits^{p}_{i=1}\mu_{i}f_{i}$
is a relation; then as we have seen before (on relations for a
complete intersection), $\mu_{i}\in (f_{1},\ldots,f_{i})$ so that
$\overline{\mu}_{i}=0$. Hence we conclude that $(\Iim \overline{R})$
is precisely the submodule generated by
$(e_{j}-\sum\limits^{p}_{j=1}\overline{\lambda}_{ij}e_{i})$, $j\geq
p+1$, which shows that $\Iim \overline{R}$ is of rank $(q-p)$ and is a
direct summand in $\overline{P}^{q}$. {\em Now we see easily that the
  complex $(*)$ is homotopic to the identity at $x$} if and only if,
(i) $(*)$ is exact and (ii) $\Iim \overline{R}$ is a direct summand
which admits a complement $\mathcal{Q}_{1}$ such that
$\overline{J}|\mathcal{Q}_{1}$ is an isomorphism and
$\Iim\overline{J}\simeq \Iim(\overline{J}|\mathcal{Q}_{1})$. Now we
have checked these conditions when $\Spec(\overline{P})$ 
is\pageoriginale smooth at $x$. Conversely if (i) and (ii) are
satisfied (at $x$), it can be checked that $\Spec (\overline{P})$ is
smooth at $x$.
\end{remark}

Suppose more generally that we are given a complex
$$
\overline{P}^{1}\xrightarrow{\overline{R}}\overline{R}^{q}\xrightarrow{\overline{J}}\overline{P}^{N} 
$$
(we keep the same notation). Then we can define an ideal $\mathscr{H}$
in $\overline{P}$ which measures the nonsplitting of the complex as
follows: $\mathscr{H}$ is the ideal generated by elements $h$ such
that there are maps $r$, $j$:
\[
\xymatrix@C=.1cm@R=.1cm{
 & r & & j & & \\
 &  & \ar@/_/[ll] &   &\ar@/_/[ll] & \\
\overline{P}^{1}\ar[rr]^-{\overline{R}} & &
\overline{P}^{q}\ar[rr]^-{\overline{J}} & &
\overline{P}^{N}, & \text{such that~ }
\overline{R}r+j\overline{J}=\text{Id.}h.  
}
\]

It can be shown that $\mathscr{H}$ is the following ideal: Take a
$(p\times p)$ minor $M$ in $\overline{J}$ and a ``{\em
  complementary}'' $(q-p)\times (q-p)$ minor $K$ in $\overline{R}$
(involving ``complementary indices''); then $\mathscr{H}$ is the ideal
generated by the elements $(\det M)(\det K)$.

In our case, $\Spec \overline{P}-V(\mathscr{H})$ is the open subscheme
of smooth points. It follows then that $\mathscr{H}$ and the ideal
$\overline{H}$ ($H$ is the ideal defined before and $\overline{H}$
denotes the image of $H$ in $B=P/F$) have the same radical. In our
case we have then
$$
Rr+jJ\equiv \widetilde{h}(\text{mod~}F),\quad\text{for some}\quad
\widetilde{h}\in P\quad\text{with image}\quad h\text{~ in~
}\mathscr{H}.
$$

Multiplying by $F$ (where $F$ is a vector now), we have
$$
FRr+FjJ\equiv F\widetilde{h}(\text{mod~}F^{2}).
$$

Since $FR=0$, this gives
$$
FjJ\equiv F\widetilde{h}(\text{mod~}F^{2}).
$$

We have used the transposes of the original $J$; setting
$z=j^{t}F^{t}$ and taking ``evaluation\pageoriginale at $a$'', we get
$$
h(a)F(a)\equiv J(a)z(\text{mod~}t^{2n}I).
$$

Now a power of $h$ is contained in $\overline{H}$ since $\overline{H}$
and $\mathscr{H}$ have the same radical. This is the crucial step in
the proof of the theorem above. Hence from this the proof of the
theorem follows easily.

Before going to the next theorem, let us recall the following facts
about $\mathfrak{a}$-adic rings (cf.\@ Serre, {\em Algebre Locale},
Chap.\@ II, A). Let $(A,\mathfrak{a})$ be a Zariski ring, i.e., $A$ is
a noetherian ring, is an ideal contained in the Jacobson radical Rad
$A$ of $A$, and $A$ is endowed with the $\mathfrak{a}$-adic topology,
i.e., a fundamental system of neighborhoods of $0$ is formed by
$\mathfrak{a}^{n}$. Since $\bigcap\limits_{n}\mathfrak{a}^{n}=(0)$, it
follows that this topology is Hausdorff. Let $\overline{A}$ denote the
$\mathfrak{a}$-adic completion. Then $\overline{A}$ is noetherian. If
$M$ is an $A$-module of finite type we can consider the
$\mathfrak{a}$-adic topology on $M$ and similarly it is Hausdorff and
if $\overline{M}$ denotes the $\mathfrak{a}$-adic completion of $M$,
we have $\overline{M}=M\otimes_{A}\overline{A}$. In fact the functor
$M\mapsto \overline{M}$ is exact. Suppose now that
$M=\overline{M}$. Then we note that any submodule $N$ of $M$ is {\em
  closed} with respect to the $\mathfrak{a}$-adic topology (for, the
quotient topology in $M/N$ is the $\mathfrak{a}$-adic topology and
since $M/N$ is of finite type and $\mathfrak{a}\subset \Rad A$, this
topology is Hausdorff and hence $N$ is closed). If $A=\overline{A}$,
i.e., $A$ is complete, then any module of finite type is complete for
the adic topology so that we don't have to assume further that
$M=\overline{M}$. 

Let $A=\overline{A}$, $M$ be as usual and $t\in \mathfrak{a}$. Then
$M/tM$ is complete for the $\mathfrak{a}/t\mathfrak{a}$-adic topology,
for this is simply the $\mathfrak{a}$-adic topology on $M/tM$.

Let $A=\overline{A}$ and $\mathcal{V}$ be an ideal in $A$. Given an
$r$, suppose that the relation $\mathfrak{a}^{r}\subset
\mathcal{V}+\mathfrak{a}^{m}$, $m\gg 0$, holds. Then
$\mathfrak{a}^{r}\subset \mathcal{V}$, for our relation implies
that\pageoriginale
$\mathfrak{a}^{r}(A/\mathcal{V})\subset\bigcap\limits_{m}\mathfrak{a}^{m}(A/\mathcal{V})$. Since
$A/\mathcal{V}$ is Hausdorff for the $\mathfrak{a}$-adic topology, it
follows that $\mathfrak{a}^{r}A/\mathcal{V}=(0)$, which implies
$\mathfrak{a}^{r}\subset \mathcal{V}$. 

\section{The case of a henselian pair
  $(A,\mathfrak{a})$}\label{part2-sec3}

\begin{theorem}\label{part2-thm3.1}
With the same notations for $A$, $B$ as in the pages preceding Theorem
\ref{part2-thm2.1}, suppose further that $(A,\mathfrak{a})$ is a {\em
  henselian pair}\footnote[1]{{\bf Definition.}~ By a {\em henselian
    pair} we mean a ring $A$ and an ideal $\mathfrak{a}\subset \Rad A$
(= Jacobson radical of $A$) such that given $F=(f_{1},\ldots,f_{N}),N$
elements of $A[X_{1},\ldots,X_{N}]$ and $x^{0}\in A^{n}$,
$x^{0}=(x^{0}_{1},\ldots,x^{0}_{N})$ such that $F(x^{0})\equiv
0\text{mod~}(\mathfrak{a})$ and such that $\det\left(\dfrac{\p
  f_{i}}{\p X_{j}}\right)|_{x^{0}}$ is invertible
$\text{mod\,}\mathfrak{a}$, then $\exists x\in A^{N}$, $x\equiv
x^{0}\text{mod\,}(\mathfrak{a})$ with $F(x)=0$.} (in particular
$\mathfrak{a}\subset \Rad A$), and that $\overline{A}$ is the
$\mathfrak{a}$-adic completion of $A$. Suppose we are given
$\overline{a}\in \overline{A}^{N}$ such that $F(\overline{a})=0$
(i.e., a formal solution) and $\mathfrak{a}^{r}\cdot \overline{A}$ (or
briefly $\mathfrak{a}^{r}$) $\subset H(\overline{a})$ for some
$r$. (This means that the section of $\Spec \overline{B}$ over $\Spec
\overline{A}$ where $\overline{B}=\overline{A}|X|/(f_{i})$ represented
by $\overline{a}$ passes through smooth points of the morphism $\Spec
\overline{B}\to \Spec \overline{A}$ except over $V(\mathfrak{a})$.)
Then for all $n\geq 1$ (or equivalently for all $n$ sufficiently
large) there is an $a\in A^{N}$ such that $F(a)=0$, and $a\equiv
\overline{a}(\text{mod\,}\mathfrak{a}^{n})$. 
\end{theorem}

\begin{proof}
(1)~{\em Reduction to the case $\mathfrak{a}$ principal.}

Let $\mathfrak{a}=(t_{1},\ldots,t_{k})$. Let us try to prove the
theorem by induction on $k$. If $k=0$, then the theorem is trivial. So
assume the theorem proved for $(k-1)$. We observe that the couple
$(A/t^{\ell}_{k},(t_{1},\ldots,t_{k-1}))$ is again a henselian pair
$\forall \ell \geq 1$ (here $t_{1},\ldots,t_{k-1}$ denote the
canonical images of $t_{i}$ in $A/t^{\ell}_{k}$). Set\pageoriginale
$t=t_{k}$ and $A_{1}=A/(t^{\ell})$. We note also that the
$\mathfrak{a}$-adic topology on $A_{1}$ is the same as the
$(t_{1},\ldots,t_{k-1})=\mathfrak{a}_{1}$-adic topology. If
$\overline{A}$, $\overline{A}_{1}$ denote the corresponding
completions, we get a canonical surjective homomorphism
$\overline{A}\to \overline{A}_{1}$ whose kernel is $t^{\ell}\cdot
\overline{A}$. Let $\overline{b}$ be the canonical image of
$\overline{a}$ in $\overline{A}_{1}$. Then we have
$F(\overline{b})=0$. Besides, we see that $\mathfrak{a}^{s}_{1}\subset
H(\overline{b})$ for some $s$ (this follows from the fact that
$V(H)\cap \Spec B=$ locus of nonsmooth points for $\Spec B\to \Spec A$
and the set of smooth points behaves well by base change and for us
the base change is $\overline{A}\to \overline{A}_{1}$. We canonot say
that $s=r$, for the ideal $H$ (or rather $H(\text{mod\,}B)$) which we
have defined using the base ring $A$ does not behave well with respect
to base change. The ideal $\mathscr{H}$ does behave well with respect
to base change, and if we had used this ideal we could have got the
same integers). Hence by induction hypothesis, for all $m\geq t$,
there exists a $b\in A^{N}_{1}$ such that $F(b)=0$, and $b\equiv
\overline{b}(\text{mod~}\mathfrak{a}^{M}_{1})$. 

Lift $b$ to an element $a_{1}\in A^{N}$ and choose $\ell$ so that
$\ell \geq m$. Then we see that
\begin{align*}
a_{1} &\equiv
\overline{a}(\text{mod~}\mathfrak{a}^{m}),\quad\text{and}\\
F(a_{1}) &\equiv 0(\text{mod\,}(t^{\ell})).
\end{align*}
(The fact that $b\equiv
\overline{b}(\text{mod\,}\mathfrak{a}^{m}_{1})$ implies
$(a_{1}-\overline{a})+x\in\mathfrak{a}^{m}$ with $x\in \Ker (A\to
A_{1})=(t^{\ell})$. Now $t\in \mathfrak{a}$, and if $\ell\geq m$,
$(t)\in \mathfrak{a}^{m}$.) We claim that if $m\gg 0$ (and
consequently $\ell \gg 0$), we have $\mathfrak{a}^{r}\subset
H(\mathfrak{a}_{1})$. By hypothesis we have $\mathfrak{a}^{r}\subset
H(\overline{a})$, and from the relation $a_{1}\equiv
\overline{a}(\text{mod\,}\mathfrak{a}^{m})$, we get
$$
\mathfrak{a}^{r}\subset H(a_{1})+\mathfrak{a}^{m}
$$
(as ideals in $\overline{A}$; to deduce this we use the Taylor
expansion). Then, as we remarked\pageoriginale before the theorem for
$m\gg 0$, this implies that $\mathfrak{a}^{r}\subset H(a_{1})$. Since
$t\in \mathfrak{a}$ it follows that $t^{r}\in H(a_{1})$. 

Let $A_{t}$ denote the $t$-adic completion. Then the following
relations in $A$,
\begin{align*}
& \text{for all~ }\ell \geq 0,\ F(a_{1})\equiv
  0(\text{mod\,}(t^{I}))\\
&\text{there exists~ }a_{1}\in A^{N}\text{~ such that~ }t^{r}\in H(a_{1})
\end{align*}
hold {\em a fortiori} in $A_{t}$, and hence by Theorem
\ref{part2-thm2.1} we can find $a' \in A^{N}_{t}$ such that
\begin{align*}
F(a') &= 0,\quad\text{and}\\
a' &\equiv a_{1}(\text{mod\,}(t^{\ell-r})).
\end{align*}

Note that the pair $(A,(t))$ is also henselian. Hence if the theorem
were true for $k=1$, we would have
\begin{align*}
\forall n, \exists \; a\in A^{N}\text{~ such that~ }F(a) &= 0,\text{~
  and}\\
a &\equiv a'(\text{mod\,}t^{n}A).
\end{align*}

But we have
\begin{align*}
a' &\equiv a_{1}(\text{mod\,}(t^{s})A_{t}),\text{~ for~ } s \text{~
  sufficiently large, and}\\
a_{1} &\equiv \overline{a}(\mathfrak{a}^{m}\cdot
\overline{A}),\ m\text{~ sufficiently large.}
\end{align*}

These imply $a\equiv \overline{a}(\text{mod\,}\mathfrak{a}^{n})$
(since $t\in \mathfrak{a}$ and $A_{t}\subset \overline{A}$), which
implies the theorem. Hence we have only to prove the theorem in the
case $k=1$, i.e., $\mathfrak{a}$ principal.
\end{proof}

\smallskip
(2)~ {\em A general lemma:}

\setcounter{lemma}{1}
\begin{lemma}\label{part2-lem3.2}
Let $A$, $B$ be as in the pages preceding the theorem, i.e.,
$B=A[X_{1},\ldots,X_{N}]/(f_{1},\ldots,f_{q})$,
$F=(f_{1},\ldots,f_{0})$. Let $C$ be the symmetric
algebra\pageoriginale on $F/F^{2}$ over $B$, so that $\Spec C$ is the
conormal bundle over $\Spec B$ ($F/F^{2}$ as a $B$-module is the
conormal sheaf over $\Spec B$). Let $f$, $g$, $h$ denote the canonical
morphisms 
$$
\Spec C\xrightarrow[f]{}\Spec B\xrightarrow[h]{}\Spec A,\ g=h\circ f. 
$$


Let $V$ be the open subschemes of $\Spec B$ where $\Spec B\to \Spec A$
is smooth and $V'=f^{-1}(V)$. Then we have the following:
\begin{itemize}
\item[\rm(a)] $g:\Spec C\to \Spec A$ is smooth and of relative dimension
  $N$ (over $A$) on $V'$ and

\item[\rm(b)] $\exists$ an imbedding $\Spec C\hookrightarrow
  \mathbb{A}^{2N+q}_{A}$ ($A$-morphism) such that the restriction of
  the normal sheaf (for this imbedding) to every affine open subset
  $U\hookrightarrow V'$ is trivial.
\end{itemize}
\end{lemma}

\noindent
{\bf Proof of Lemma \ref{part2-lem3.2}.}~ We set
\begin{align*}
C &= B[Y_{1},\ldots,Y_{q}]/I\\
& A[X,Y]/(F,I),\ K=(F,I).
\end{align*}

On $V$ we have the exact sequence
\begin{equation*}
0\to F/F^{2}\to \Omega_{A[X]/A}\otimes_{A[X]}B\to \Omega_{B/A}\to
0.\tag{i} 
\end{equation*}
(This is an abuse of notation; strictly speaking we have to write\break
$(F/F^{2})|_{V}\ldots$, etc.) On $V$ we have the exact sequence
\begin{gather*}
0\to f^{*}(\Omega_{B/A})\to \Omega_{C/A}\to \Omega_{C/B}\to
0,\quad\text{and}\tag{ii}\\
\Omega_{C/B}\approx f^{*}(F/F^{2}).
\end{gather*}

Taking\pageoriginale $f^{*}$ of the first sequence, we get sequences 
\begin{align*}
& 0\to f^{\ast}(F/F^{2})\to (\text{Free})\to f^{\ast}(\Omega_{B/A})\to
  0\\
& 0\to f^{*}(\Omega_{B/A})\to \Omega_{C/A}\to f^{*}(F/F^{2})\to 0 
\end{align*}
which are exact on any affine open set $U$ in $V'$. These exact
sequences are split on $U$, so that we conclude
$$
\Omega_{C/A}\quad\text{is free on}\quad U.
$$

On $V'$ we have the exact sequence
$$
0\to K/K^{2}\to \Omega_{A[X,Y]/A}\otimes_{A[X,Y]}C\to \Omega_{C/A}\to
0.
$$

Thus it follows that on $U$
$$
K/K^{2}\oplus \underbrace{(\text{Free})}_{\text{of rank
    $N$}}=(\text{Free})
$$
(From the exact sequence (ii) it follows that $\Omega_{C/A}$ is of
rank $N$ over $C$ and this implies the assertion (a).) If we introduce
$N$ more indeterminates $Z_{1},\ldots,Z_{N}$, then
\begin{equation*}
C=A[X,Y,Z_{1},\ldots,Z_{n}]/(K,Z_{1},\ldots,Z_{n}).\tag{*}
\end{equation*}

Let $K'=(K,Z_{1},\ldots,Z_{N})$. Then we see easily that
$$
K'/{K'}^{2}=K/K^{2}\oplus (\text{Free of}~rk\ N).
$$

It follows that $K'/{K'}^{2}$ is free on $U$. Thus for the embedding\break
$\Spec C\hookrightarrow \mathbb{A}^{2N+q}_{A}$, the restriction of the
normal bundle $K'/{K'}^{2}$ to $U$ is trivial. This completes the
proof of the lemma.

\smallskip
(3)~ We\pageoriginale saw in (1) above, that for the theorem it
suffices to prove it in the case $\mathfrak{a}=(t)$. The condition
$\mathfrak{a}^{r}\overline{A}\subset H(\overline{a})$ becomes
$t^{r}\in H(\overline{a})$. (Note that $H(\overline{a})$ is the ideal
in $\overline{A}$ generated by evaluating at $\overline{a}$ elements
of $H$ and $H$ {\em is an ideal in} $A[X_{1},\ldots,X_{N}]$ not in
$\overline{A}[X_{1},\ldots,X_{N}]$.)

{\em We claim now that there is an $h\in H$ such that}
$$
h(\overline{a})=(\text{unit}) \cdot t^{r}.
$$

{\em A priori} it is clear there is an $\overline{h}\in H\cdot
\overline{A}[X]$ such that $\overline{h}(\overline{a})=t^{r}$. Since
$\overline{A}$ is the $t$-adic completion of $A$, we can find an $h\in
H$ such that
$$
h\equiv \overline{h}(\text{mod\,}(t^{r+1}))
$$
(i.e., the coefficients of $h$ and $\overline{h}$ differ respectively
by an element of $(t^{r+1})$). This implies that
\begin{align*}
& h(\overline{a})-\overline{h}(\overline{a})=\ast
  t^{r+1},\quad\text{hence}\\
& h(\overline{a})=t^{r}(1+*t).
\end{align*}

Now $(1+*t)$ is a unit in $\overline{A}$. This proves the required
claim.

\smallskip
(4)~ {\em The final step.}

Since $\Spec C$ is a vector bundle over $\Spec B$, we have the
$0$-section $\Spec C \overset{\curvearrowleft}{\to}\Spec B$. We are
given $\overline{a}\in\overline{A}^{N}$ such that
$F(\overline{a})=0$. Now $\overline{a}$ determines a section of $\Spec
(B\otimes_{A}\overline{A})$ over $\overline{A}$, or equivalently a
morphism $s:\Spec \overline{A}\to \Spec B$ forming a commutative
diagram
\[
\xymatrix{
\Spec \overline{A}\ar[rr]^-{s}\ar[dr] & & \Spec B\ar[dl]\\
 & \Spec A &
}
\]

Using\pageoriginale the $0$-section, $s$ can be lifted to a morphism
$s_{1}:\Spec \overline{A}\to \Spec C$:
\[
\xymatrix{
\Spec \overline{A}\ar[rr]^-{s_{1}}\ar[dr] & & \Spec C\ar[dl]\\
 & \Spec A &
}
\]

Now by (3), $s_{1}$ carries $(\Spec A-V(t))$ to $\Spec C[1/h]$, where
$h$ is as in (3) (here $h$ denotes the canonical image in $C$ of the
$h$ in (3)). We observe that
$$
\Spec C[1/h]\subset V',
$$
($V'=f^{-1}(V)$, $V=$ locus of smooth points of $\Spec B\to \Spec A$)
and $V'$ is contained in the locus of smooth points of the map $\Spec
C\to \Spec A$. We have $C=A[X,Y,Z]/K'$ where
$K'=(F,I,Z_{1},\ldots,Z_{N})$. The section $s_{1}$ defines a solution
$K'(\overline{a}')=0$, where $\overline{a}'\in \overline{A}^{2N+q}$
($\overline{a}'$ extends the section $\overline{a}$). We have
$t^{k}\in H(\overline{a}')$ for a suitable $k$. Hence the conditions,
similar to those of $\Spec B\to \Spec A$, are now satisfied for the
map $\Spec C\to \Spec A$. It is immediately seen that it suffices to
solve for the case $\Spec C\to \Spec A$, in fact if we get a solution
$a'\in A^{2N+q}$, we have to take for a the first $N$ coordinates of
$a'$.

Let $U=\Spec C[1/h]$. Then the restriction of the normal bundle of the
imbedding $\Spec C\hookrightarrow \mathbb{A}^{2N+q}_{A}$ is trivial on
$U$ and, $U$ being smooth over $A$, it follows easily that $U$ is open
in a global complete intersection. By this we mean there exist
$g_{1},\ldots,g_{N+q}\in A[X,Y,Z]$ such that $V(g_{1},\ldots,g_{N+q})$
has dimension $N$ and we have an open immersion
$$
(\Spec C[1/t])=\Spec C[1/h]\hookrightarrow \Spec
A[X,Y,Z]/(g_{1},\ldots,g_{n+q}).
$$

Let $G=(g_{1},\ldots,g_{N+q})$. Then we have
$$
K'_{t}=G_{t}.
$$
(We\pageoriginale denote the localization with respect to $t$ by a
subscript. Note that localization with respect to $t$ is the same as
localization with respect to $b$.) The given solution $\overline{a}'$
in $\overline{A}^{2N+q}$ is such that $K'(\overline{a}')$ gives rise
to a solution $G(\overline{a}')=0$ by changing the $g_{i}$,
multiplying them by suitable powers of $t$. Conversely, suppose we
have solved the problem for $G$, i.e., we have found $a'\in A^{2N+q}$
such that $G(a)=0$ and $a'\equiv
\overline{a}(\text{mod\,}t^{n})$. Then we see easily that there is a
$\theta$ such that
$$
t^{\theta}\cdot F(a')=0.
$$

Since $F(\overline{a})=0$ by Taylor expansion, it follows that
$F(a')\equiv 0\break(\text{mod\,}t^{n})$. Now if $n\gg 0$, by Lemma
\ref{part2-lem2.1}, it follows that $F(a')=0$. Thus it suffices to
solve the problem for $G$, i.e., for the morphism
$$
\Spec C'\to \Spec A,\quad C'=A[X,Y,Z]/G.
$$

We have seen that $\overline{a}'$ defines a section of this having the
required properties. Further, $\Spec C'$ is a {\em global} complete
intersection and\break smooth over $A$ in $\mathbb{A}^{2N+q}$, i.e.,
we have reduced the theorem to the following lemma. 

\section{Tougeron's lemma}\label{part2-sec4}

\begin{lemma}\label{part2-lem4.1}
Let $(A,\mathfrak{a})$ be a henselian pair and $f_{i}\in
A[Y_{1},\ldots,Y_{N}]$, $1\leq i\leq m$. Let $J=\left(\dfrac{\p
  f_{i}}{\p Y_{j}}\right)$ be the Jacobian matrix, $1\leq i\leq m$,
$1\leq j\leq N$. Suppose we are given
$y^{0}=(y^{0}_{1},\ldots,y^{0}_{N})\in A^{N}$ such that
$$
f(y^{0})\equiv 0(\text{mod\,}\Delta^{2}\mathcal{V})
$$
where $V(\mathcal{V})=V(\mathfrak{a})$ (or $\Leftrightarrow
\mathcal{V}$ is also a defining ideal for $(A,\mathfrak{a})$) and
$\Delta$ is\pageoriginale the annihilator of the $A$-module $C$
presented by the relation matrix (i.e., $C$ is the cokernel of the
homomorphism $A^{N}\to A^{m}$ whose matrix is $J(y_{0})$). Then there
is a $y\in A^{N}$ such that
$$
f(y)=0\quad\text{and}\quad y\equiv y^{0}(\text{\rm
  mod\,}\Delta\mathcal{V}). 
$$
\end{lemma}

\begin{proof}
The henselian property of $(A,\mathfrak{a})$ is used in the following
manner: Let $F=(F_{1},\ldots,F_{N})$ be $N$ elements of
$A[Y_{1},\ldots,Y_{N}]$ and $y^{0}=(y^{0}_{1},\ldots,y^{0}_{N})\in
A^{N}$ such that
\begin{align*}
& {\rm(i)~~ } F(y^{0})\equiv 0(\text{mod\,}\mathfrak{a})\\
& {\rm(ii)~ } \det \left(\frac{\p F_{i}}{\p Y_{j}}\right)_{y=y^0}\quad\text{is
    a {\em unit}}\quad (\text{mod\,}\mathfrak{a}).\tag{P}
\end{align*}

Then there is a $y\in A^{N}$ such that $F(y)=0$ and $y\equiv
y^{0}(\text{mod\,}\mathfrak{a})$. 

Let $\delta_{1},\ldots,\delta_{r}$ generate the annihilator of
$\Delta$. This implies that there exist $N\times m$ matrices such that
$$
JN_{i}=\delta_{i}I,\quad J=J(y^{0}),\quad I=\text{Id}_{(m\times m)}.
$$

Write
\begin{align*}
f(y^{0}) &= \sum_{i,j}\delta_{i}\delta_{j}\epsilon_{ij},\\
\epsilon_{ij} &=
(\epsilon_{ij1},\ldots,\epsilon_{ij\nu},\ldots,\epsilon_{ijm}),\ \epsilon_{ij}\in\mathcal{V}\\
&\hspace{3.7cm}\searrow\\[-4pt]
&\hspace{4.2cm}m \text{~ components.}
\end{align*}

We try to solve the equations
$$
f\left(y^{0}+\sum^{r}_{i=1}\delta_{i}U_{i}\right)=0
$$
for elements $U_{i}=(U_{il},\ldots,U_{iN})\in A^{N}$ (we consider
vectors in $A^{n}$ to be column\pageoriginale matrices). Expansion by
Taylor's formula in vector notation gives
$$
0=f(y^{0})+J\cdot (\sum
\delta_{i}U_{i})+\sum_{i,j}\delta_{i}\delta_{j}\mathcal{Q}_{ij}, 
$$
where
\begin{align*}
& J\text{~ is an~ }(m\times N)\ \text{ matrix},\\
& f(y^{0})\text{~ is an~ }(m\times 1)\ \text{matrix,}\\
& U_{i}\text{~ is an~ } (N\times 1) \text{~ matrix~ }(\text{not~}
  (1\times N)\text{~ matrix as it is written}),
\end{align*}
and
$$
\mathcal{Q}_{ij},\epsilon_{ij}\quad\text{are}\quad (m\times
1)\quad\text{matrices.} 
$$

Expanding, we get
\begin{align*}
 0 &=J\cdot
  \underbrace{(\sum\limits^{r}_{i=1}\delta_{i}U_{i})}_{(m\times
    N)(N\times 1)\text{~ matrix}} +
  \sum_{i,j}\delta_{i}\delta_{j}\underbrace{(\mathcal{Q}_{ij}+\epsilon_{ij})}_{(m\times
    1)\text{~matrix}}\\
&= \sum^{r}_{i=1}\delta_{i}(JU_{i})+\sum_{i,j}\delta_{i}\cdot
  JN_{j}\cdot (\mathcal{Q}_{ij}+\epsilon_{ij})\quad
  (\delta_{j}\text{Id}=JN_{j})\\ 
&= \sum^{r}_{i=1}(\delta_{i}J)\cdot
  U_{i}+\sum_{i}\delta_{i}J(\sum_{J}N_{j}(\mathcal{Q}_{ij}+\epsilon_{ij}))\quad
  (\delta_{i}\text{~ are scalars}).  
\end{align*}

Thus it suffices to solve the $r$ equations
\begin{equation*}
0=U_{i}+\sum_{j}N_{j}(\mathcal{Q}_{ij}+\epsilon_{ij}),\quad 1\leq
i\leq r.\tag{*} 
\end{equation*}

This is an equation for an $(N\times 1)$ matrix. Thus $(*)$ gives $Nr$
{\em equations in the $Nr$ unknowns} $U_{i\nu}$, $1\leq i\leq r$,
$1\leq \nu\leq N$.

We note that $\mathcal{Q}_{ij}$ are vectors of polynomials in
$U_{i\nu}$ all of whose terms are of degree $\geq 2$. Let
$F=F_{1},\ldots,F_{NR}\in A[U_{i\nu}]$ represent the right hand side
of $(*)$. Write $Z_{1},\ldots,Z_{Nr}$ for the indeterminates
$U_{i\nu}$. Then 
$$
\left(\frac{\p F_{k}}{\p Z_{\ell}}\right)=\text{Id}+M,\quad
M=(m_{\alpha\beta}),\quad (Nr\times Nr)\text{~ matrix}
$$
{\em where\pageoriginale $M$ is an $Nr\times Nr$ matrix of polynomials
  in $Z_{\ell}$ and every $m_{\alpha\beta}$ has no constant term.}

Let $x^{0}\in A^{Nr}$ represent the vector $(0,\ldots,0);$ then we
have
$$
F(z^{0})\equiv 0(\text{mod\,}\mathcal{V})\quad\text{since}\quad
\epsilon_{ij\nu}\in \mathcal{V}.
$$

Without loss of generality we can suppose $\mathcal{V}=\mathfrak{a}$
since $\mathcal{V}$ is also a defining ideal for $(A,\mathfrak{a})$
$$
F(Z^{0})\equiv 0(\text{mod\,}\mathfrak{a}).
$$

Further, we have $\left(\dfrac{\p F_{k}}{\p Z_{1}}\right)_{Z=z^{0}}$
is a unit in $A/\mathfrak{a}$. Hence by the henselian property of
$(A,\mathfrak{a})$, we have a solution $z$ of $(*)$ in $A$ such that
$z\equiv z^{0}\break(\text{mod\,}\mathfrak{a})$, i.e., $z\equiv
0(\text{mod\,}\mathfrak{a})$ since $z^{0}\equiv (0)$. Set 
$$
y=y^{0}+z.
$$

Then we have
$$
y\equiv y^{0}(\text{mod\,}\Delta\mathfrak{a})\quad\text{and}\quad
f(y)=0,
$$
which proves the lemma.
\end{proof}

\begin{remark}\label{part2-rem4.1}
It is possible to take $\mathcal{V}$ such that $\mathcal{V}\subset
\mathfrak{a}^{r}$, for if $(A,\mathfrak{a})$ is a henselian pair, the
henselian property is true for $\mathcal{V}$, $\mathcal{V}\subset
\mathfrak{a}$. Then the above proof also goes through for this case.
\end{remark}

\begin{coro*}
Let $(A,\mathfrak{a})$ be a henselian pair and
$$
f_{1},\ldots,f_{m}\in A[y_{1},\ldots,y_{N}]
$$
and $\varphi$ the canonical morphism
\[
\xymatrix@R=.5cm{
\varphi:\Spec A[y_{1},\ldots,y_{N}/(f_{1},\ldots,f_{m})\to \Spec
  A\ar@{=}[d]\\
B
}
\]

Suppose\pageoriginale that $\varphi$ {\em is smooth and a relative
  complete intersection}. Then given $\overline{y}\in \hat{A}$
($\mathfrak{a}$-adic completion of $A$) such that
$$
f(\overline{y})=0,
$$
there is a $y\in A$ such that $f(y)=0$ and
$$
y\equiv y(\text{mod\,}\mathfrak{a}^{c})
$$
for any given $c\geq 1$.
\end{coro*}

\begin{proof}
We can take $c=1$ for $\mathfrak{a}^{c}$ is also a defining ideal for
$(A,\mathfrak{a})$. We can {\em a fortiori} find $y^{0}\in A^{N}$ such
that
$$
f(y^{0})\equiv 0(\text{mod\,}\mathfrak{a})(\Leftarrow
f(\overline{y})=0).
$$
$y^{0}$ then defines $B$ a section of $\Spec (B)$ over $\Spec
A/\mathfrak{a}$. The morphism is a complete intersection and smooth at
points of this section. This implies that the Ideal generated by
canonical images in $A/\mathfrak{a}$ of the determinants of the
$(m\times m)$ minors of $J(y_{0})$ is the unit ideal in
$A/\mathfrak{a}$, i.e., the canonical image of $\Delta$ in
$A/\mathfrak{a}$ is the unit ideal. Since $\mathfrak{a}$ is in the
Jacobson radical, it follows that $\Delta$ is itself the unit
ideal. Indeed $\Delta$ being the unit ideal in $A/\mathfrak{a}$
implies there exists a $u\in \Delta$ such that $u\equiv
1(\text{mod\,}\mathfrak{a})$, hence $u-1\in \mathfrak{a}$, hence
$u=1+r$, $r\in \mathfrak{a}$. Since $\mathfrak{a}\subset \Rad A$, $u$
is a unit in $A$.
\end{proof}

\section[Existence of algebraic deformations of isolated...]{Existence
of algebraic deformations of isolated\hfil\break 
  singularities}\label{part2-sec5} 

\begin{definition}\label{part2-defi5.1}
A family of isolated singularities is a scheme $X\xrightarrow{\pi}S$
over $S=\Spec A$, $A$ a $k$-algebra such that (i) $\pi$ is flat, of
finite presentation and $X$ is affine, $X=\Spec \mathscr{O}$ and (ii)
if $\Gamma$ is the closed subset of $X$ where $\pi$ is not smooth,
then $\Gamma\to S$ is a {\em finite morphism} (for this let us say
that $\Gamma$ is endowed with the canonical structure of a reduced
scheme). 
\end{definition}

\begin{definition}\label{part2-defi5.2}
We\pageoriginale say that two families $X\to S$ and $X'\to S$ of isolated
singularities are {\em equivalent} or {\em isomorphic} if there is a
family of isolated singularities $X''\to S$ (note that $S$ is the
same) and \'etale morphisms $X''\to X'$, $X''\to X$ such that
\begin{itemize}
\item[(i)] the following diagram is commutative
\[
\xymatrix{
 & X''\ar[dl]_{\text{\'etale}}\ar[dr]^{\text{\'etale}}\ar[dd] & \\
X\ar[dr] & & X'\ar[dl]\\
 & S &
}
\]
and

\item[(ii)] these maps induce isomorphisms
\[
\xymatrix{
 & \Gamma''\ar[dl]_{\sim}\ar[dr]^{\sim} &\\
\Gamma & & \Gamma'.
}
\]
\end{itemize}
\end{definition}

An equivalence class of isolated singularities represented by $X\to S$
is therefore the {\em henselization of $X$ along $\Gamma$.}

Consider in particular a one point ``{\em family}'' $X_{0}\to \Spec k$
{\em with an isolated singularity.} (We could also take a finite
number of isolated singularities.) We see easily that the formal
deformation space of $X_{0}$ (in the sense of Schlessinger defined
before) depends only on the equivalence class of $X_{0}$. Let $A$ be
the {\em formal versal deformation space} associated to $X_{0}$ (we
can speak of the versal deformation space of $X_{0}$ by taking Zariski
tangent space of $A=\dim T^{1}_{X_{0}}$), i.e., $A$ is a complete
local ring, and we are given a sequence $\{X_{n}\}$ of deformations
over $A_{n}$
$$
X_{n}=\Spec \mathscr{O}_{n},\quad A_{n}=A/m^{n+1}_{A},\quad
\mathscr{O}_{A}\otimes A_{n-1}\cong \mathscr{O}_{n-1} 
$$
satisfying the versal property mentioned before. Note that by a versal 
deformation\pageoriginale it is not meant that there is a deformation
of $X_{0}$ over $A$. The following theorem, proved by Elkik, asserts
that in fact a deformation of $X_{0}$ over $A$ does exist (i.e., in
the case of isolated singularities). It is the crucial step for the
existence of ``algebraic'' versal deformations for $X_{0}$.

\begin{theorem}\label{part2-thm5.1}
Let $X_{0}=\Spec \mathscr{O}_{0}$,
$\mathscr{O}_{0}=k[X_{1},\ldots,X_{m}]/(f)$,
$(f)=(f_{1},\break\ldots,f_{r})$. Let $X_{n}=\Spec \mathscr{O}_{n}$ be as
above, but we suppose moreover that $X_{0}$ is equidimensional of
dimension $d$. Then there is a deformation $X'$ over $A$ such that
$X'\otimes A_{n}\approx \mathscr{O}_{n}$, and if $X'=\Spec
\mathscr{O}'$ then $\mathscr{O}'$ is an $A$-algebra of finite
type. (We do not claim that $\mathscr{O}'$ has the same presentation
as $\mathscr{O}_{0}$.)
\end{theorem}

\begin{proof}
Let $\overline{\mathscr{O}}=\varprojlim \mathscr{O}_{n}$,
$\mathscr{O}_{n}=A_{n}[X_{1},\ldots,X_{m}]/(f^{(n)})$, where
$f^{(n)}_{i}\in A_{n}[X]$ is a lifting of $f_{i}\in k[X]$. Let
$A[X]^{\wedge}$ denote the $m$-adic $(m=m_{A})$ completion of $A[X]$. We
see that $A[X]$ is the set of formal power series $\sum a_{i}X^{(i)}$
such that $a_{i}\to 0$ in the $m$-adic topology of $A$. Then
$\overline{f}_{i}=\lim f^{(n)}_{i}$ is in $A[X]$ and we see easily
that
$$
\overline{\mathscr{O}}=A[X]^{\wedge}/(\overline{f}).
$$
(For, we see that we have a canonical homomorphism
$$
\alpha:A[X]^{\wedge}/(\overline{f})\to \overline{\mathscr{O}}
$$
obtained from the canonical homomorphism $A[X]^{\wedge}/(\overline{f})\to
A_{n}[X]/(f^{(n)})$. It is easy to see that $\alpha$ is an isomorphism.)

The proof of the theorem is divided into the following steps:
\begin{itemize}
\item[(1)] {\em It is enough to find a (flat) deformation $X''$ over
  $A$ of $X_{0}$ such that $\mathscr{O}''\otimes_{A}A_{1}\approx
  \mathscr{O}_{1}$,\pageoriginale where $X''=\Spec \mathscr{O}''$
  (recall that $A_{1}=A/m^{2}$).} 
\end{itemize}

For, given a (flat) deformation $X''$ over $A$ we get deformations\break
$\{X''_{n}\}=X''\otimes A_{n}$ over $A_{n}$. For each $n$, we get then
by the versal property of $A$, a homomorphism
$$
\alpha_{n}:A\to A_{n}.
$$

Note that $\alpha_{n}$ is defined by
$$
(\alpha_{n})_{m}:A_{m}\to A_{n},\quad m\gg 0,
$$
so that $X_{m}\otimes_{A_{m}}A_{n}\approx X''_{n}$. These
$\{\alpha_{n}\}$ are consistent and hence $\{a_{n}\}$ patch up to
define a homomorphism of rings
$$
\alpha:A\to A.
$$

The hypothesis that $\mathscr{O}''\otimes_{A}A_{1}\approx
\mathscr{O}_{1}$ implies that
$$
\alpha\equiv \text{Id}(\text{mod\,}m^{2}_{A}).
$$

This condition on $\alpha$ implies that $\alpha$ is an isomorphism;
for it follows that $\alpha$ induces an isomorphism on the Zariski
tangent spaces, so that $\Iim \alpha$ contains a set of generators of
$m_{A}$, hence $\alpha$ is {\em surjective}; further, this condition
implies that the induced homomorphisms $\alpha_{n}:A/m^{n}_{A}\to
A/m^{n}_{A}$ are surjective, and these vector spaces being
finite-dimensional, it follows that $\alpha_{n}$ is an isomorphism for
all $n$ (in particular injective). It follows easily that $\Ker
\alpha\subset \bigcap\limits_{n}m^{n}_{A}=(0)$, i.e., $\alpha$ is
injective. Hence $\alpha$ is an isomorphism.

Now define the deformation $X'$ over $A$ as the pull back of $X''$
over $A$ by the isomorphism $\alpha-1$. It is easily checked that
$X'\otimes_{A}A_{n}\approx X_{n}$, and this proves (1).

Let us set $\overline{X} = \Spec \overline{\mathscr{O}}$. Consider the
Jacobian matrix $J = \left(\dfrac{\p \overline{f}_{i}}{\p X_{j}}\right)$
\;\; $1\leq i\leq r$,\pageoriginale $1\leq j\leq m \; 
((\overline{f})=(\overline{f}_{1},\ldots,\overline{f}_{r}))$. Let
$\overline{\Gamma}$ denote the locus of points in $\overline{X}=\Spec
\overline{\mathscr{O}}$ where $rk\ J<(m-d)$. Then $\overline{\Gamma}$
is a {\em closed subscheme} in $\overline{X}$. (We note that
$\overline{f}_{i}\in A[X]^{\wedge}$ and $\dfrac{\p \overline{f}_{i}}{\p
  X_{j}}\in A[X]^{\wedge}$ so that the Jacobian matrix $J$ is a matrix of
elements is $A[X]^{\wedge}$.) Hence if $x\in\Spec A[X]^{\wedge}$ (in
particular if $x\in \Spec
\overline{\mathscr{O}}=\overline{X}\hookrightarrow A[X]$), we can talk
of the rank of $J$ at $x$, i.e., the matrix $J(x)$ whose elements are
the canonical images in $k(x)$ (residue field at $x$) of the elements
of $J$. It follows then easily that the locus of points $x$ of
$\overline{X}$ where $J(x)$ is of rank $<(m-d)$ is closed in
$\overline{X}$; in fact we see that $\overline{\Gamma}=V(I)$ where $I$
is the ideal generated by the determinants of all the $(m-d)\times
(m-d)$ minors of $J'$ where $J'$ is $J$ with elements replaced by
their canonical images in $\overline{\mathscr{O}}$. It is clear that
$\overline{\Gamma}\cap X_{0}$ is precisely the set of singular points of
$X_{0}$, which is by our hypothesis a finite subset of $X_{0}$. It can
then be seen without much difficulty that $\overline{\Gamma}$ {\em is
  finite over} $A$ ($\overline{\Gamma}$ is endowed with the canonical
structure of a reduced scheme or a scheme structure from the ideal $I$
introduced above). The proof of this is similar to the fact:
quasifinite implies finite in the ``formal case'', i.e., in the
situation $A\to B$ where $A$, $B$ are complete local rings and $B$ is
the completion of a local ring of an $A$-algebra of finite type.

Let $\overline{\Gamma}=\Spec \overline{\mathscr{O}}/\Delta$ and let
$\Delta_{0}$ be the ideal in $\mathscr{O}_{0}$ defined by $\Delta$
(i.e., the canonical image of $\Delta \otimes k$ in
$\mathscr{O}_{0}$). By the Noether Normalization lemma we can find
$y^{0}_{1},\ldots,y^{0}_{d}$ in $\Delta_{0}$ such that
$\mathscr{O}_{0}$ is a finite $k[y^{0}_{1},\ldots,y^{0}d]$ module such
that the set of common zeros of $y^{0}_{i}$ is precisely the set of
singular points of $X_{0}$. Lift $y^{0}_{i}$ to elements
$y_{1},\ldots,y_{d}$ in $\Delta$ so that $y_{1},\ldots,y_{d}$ vanish
on $\overline{\Gamma}$. Then we have 
\begin{itemize}
\item[(2)] $\overline{\mathscr{O}}$\pageoriginale {\em is a finite}
  $A[y]^{\wedge}$ {\em module} (and $\overline{\Gamma}$ is precisely the
  locus of $y_{i}=0$).
\end{itemize}

This is again obtained by an argument generalizing ``quasifinite
implies finite in the ``formal'' cases.''
\begin{itemize}
\item[(3)] {\em The open subscheme $\overline{X}-\overline{\Gamma}$ of
  $\overline{X}$ is regular (over $A$) (i.e.,
  $\overline{X}-\overline{\Gamma}$ is flat over $A$ and the fibres are
  regular)}.  
\end{itemize}

This is a generalization of the Jacobian criterion of regularity to
the adic and formal case.
\begin{itemize}
\item[(4)] {\em Outside the set $\{y=0\}$ in $\Spec A[y]^{\wedge}$ (this
  is a section of $\Spec\break A[y]^{\wedge}$ over $\Spec A$ and
  $\overline{\Gamma}$ lies over this set), $\overline{\mathscr{O}}$ is
  locally free over $A[y]^{\wedge}$ say of rank $r$, i.e.,
  $p_{\ast}(\mathscr{O}_{\overline{X}-\overline{\Gamma}})$ is a
  locally free sheaf of $\mathscr{O}_{\Spec A[y]^{\wedge}-\{y_{i}=0\}}$
    modules ($p:\overline{X}\to \Spec A[y]^{\wedge}$ canonical morphism).}
\end{itemize}

For, $p_{\ast}(\mathscr{O}_{\overline{X}})$ is finite over $\Spec
A[y]^{\wedge}$. Now $\Spec A[y]^{\wedge}$ is regular over $\Spec A$. We have
seen that $\overline{X}-\overline{\Gamma}$ is regular over $\Spec A$,
so that it is in particular Cohen-Macaulay over $\Spec A$. Now a
Cohen-Macaulay module $M$ (of finite type) over a regular local ring
$B$ is free (cf.\@ Serre's ``Alg\`ebre locale'') and from this (4)
follows. (We can use this property only for the corresponding fibres,
but then the required property is an easy consequence of this.)
\begin{itemize}
\item[(5)] {\em Set $\hat{P}=A[y]^{\wedge}$. Then for
  $\overline{\mathscr{O}}$ considered as a $\hat{P}$ module we have a
  representation of the form}
$$
\hat{P}^{m}\xrightarrow{(a_{ij})}\hat{P}^{n}\to
\overline{\mathscr{O}}\to 0\quad\text{(exact as $\hat{P}$ modules)}
$$
with $rk(a_{ij})\leq (n-r)$ (i.e., determinants of all minors of
$rk(n-r+1)$ of $(a_{ij})$ are zero). 
\end{itemize}

More\pageoriginale generally, let us try to describe an $R$-{\em
  algebra} $B$ having a representation of the form
\begin{equation*}
\begin{cases}
R^{m}\xrightarrow{(a_{ij})}R^{n}\to B\to 0\quad\text{exact sequence}\\
rk(a_{ij})\leq (n-r).\qquad \text{of $R$-modules}
\end{cases}\tag{*}
\end{equation*}
The we have
\end{proof}

\begin{lemma}\label{part2-lem5.1}
$\exists$ an affine scheme $V$ (of finite type) over $\Spec
  \mathbb{Z}$ such that every $R$-algebra of the form $(*)$ is induced
  by an $R$-valued point of $V$.
\end{lemma}

\noindent
{\bf Proof of Lemma.}~To each $R$ we consider the functor
$\mathfrak{F}(R)$
\begin{equation*}
\mathfrak{F}(R)=
\left\{
\begin{array}{c}
\text{set of all commutative $R$-algebras $B$}\\
\text{with a representation of the form $(*)$}
\end{array}
\right\}.
\end{equation*}

One would like to represent the functor $\mathfrak{F}$ by an affine
scheme, etc. We don't succeed in doing this, but we will represent a
functor $G$ such that we have a {\em surjective} morphism $G\to
\mathfrak{F}$. 

An algebra structure on $B$ is given by an $R$-homomorphism
$$
B\otimes_{R}B\to B.
$$

Then in the diagram
\[
\xymatrix{
R^{n}\otimes{R}R^{n}\ar[r]\ar[d]_{\exists} & B\otimes_{R}B\ar[d]\\
R^{n}\ar[r] & B
}
\]
the homomorphism $R^{n}\otimes_{R}R^{n}\to R^{n}$ factors via a
homomorphism $R^{m}\otimes R^{n}\to R^{n}$ (of course not uniquely
determined). Now the following sequence
$$
(R^{m}\otimes_{R}R^{n})\oplus (R^{n}\otimes
R^{m})\xrightarrow{\psi}R^{n}\otimes R^{n}\to B\otimes B\to 0
$$
is\pageoriginale exact where $\psi=(a_{ij})\otimes \text{Id}\oplus
\text{Id}\otimes (a_{ij})$. [We have
{\fontsize{10}{11}\selectfont
$$
(\Ker (R^{n}\to B)\otimes R^{n})\oplus (R^{n}\otimes (\Ker R^{n}\to
  B))\xrightarrow{\text{can hom}}\Ker (R^{n}\otimes R^{n}\to B\otimes
  B)\to 0.
$$}
This implies exactness of the given diagram.] Again there exists a
lifting of the above commutative diagram
\begin{equation*}
\vcenter{
\xymatrix{
(R^{m}\otimes R^{n})\oplus (R^{n}\otimes
  R^{m})\ar[d]^c_{\exists}\ar[r]^<<<<{\psi} & R^{n}\otimes R^{n}\ar[r]\ar[d]^{b}_{\exists} &
  B\otimes_{R}B\ar[r] &  0\\
R^{m}\ar[r]^{(a_{ij})=a} & R^{n}\ar[r] & B\ar[r] & 0
}}\tag{$I_{0}$}
\end{equation*}


On the other hand, giving a commutative diagram
\begin{equation*}
\vcenter{
\xymatrix{
(R^{m}\otimes R^{n})\oplus(R^{n}\otimes R^{m})\ar[d]_{c}\ar[r]^{\psi}
  & R^{n}\otimes R^{n}\ar[d]_{b}\\
R^{m}\ar[r]^{(a_{ij})=(a)} & R^{n}
}}\tag{$I$}
\end{equation*}
where $\psi=(a_{ij})\otimes \text{Id}\oplus \text{Id}\otimes (a_{ij})$
determines the commutative diagram $(I_{0})$. 

The algebra structure on $B$ induced by $(I_{0})$ is associative if
the diagram
\[
\vcenter{\xymatrix@C=.45cm{
R^{n}\otimes R^{n}\otimes R^{n}\ar[dr]_{\text{Id}\otimes
  b}\ar[rr]^{\mathfrak{b}\cdot (\text{Id}\otimes b)} & & R^{n}\\
& R^{n}\otimes R^{n}\ar[ur]_{b}
}
}~,~
\vcenter{\xymatrix@C=.45cm{
R^{n}\otimes R^{n}\otimes R^{n}\ar[dr]_{b\otimes
  \text{Id}}\ar[rr]^{\mathfrak{b}\cdot (b\otimes \text{Id})} & & R^{n}\\
& R^{n}\otimes R^{n}\ar[ur]_{b}
}},
\]
and the map $R^{n}\otimes R^{n}\otimes R^{n\mathfrak{b}\cdot
  (\text{Id}\otimes b)-\mathfrak{b}\cdot (b\otimes \text{Id})}\to
R^{n}$ factorizes through $R^{m}\to R^{n}$, i.e.,
\begin{equation*}
\vcenter{\xymatrix{
R^{n}\otimes R^{n}\otimes
R^{n}\ar[dr]_{\exists\alpha}\ar[rr]^-{\mathfrak{b}\cdot(\text{Id}\otimes
  b)-\mathfrak{b}\cdot (b\otimes\text{Id})} & & R^{n}\\
 & R^{m}\ar[ur]_{(a_{ij})=a}
}}\tag{II}
\end{equation*}
or, $b\cdot (\Id\otimes b)-b\cdot(b\otimes \Id)$ is zero in $B$.

Let\pageoriginale $\widetilde{b}:R^{n}\times R^{n}\to R^{n}$ be the
homomorphism obtained by changing $b:R^{n}\times R^{n}\to R^{n}$ by
the involution defined by $x\otimes y\mapsto y\otimes x$ in $R\otimes
R$. Then the algebra structure is commutative if there is a
homomorphism $\delta:R^{n}\otimes_{R}R^{n}\to R^{m}$ such that
\begin{equation*}
\vcenter{
\xymatrix{
R^{n}\otimes_{R}R^{n}\ar[dr]_{\delta}\ar[rr]^{b-b} & & R^{n}\\
 & R^{m}\ar[ur]_{(a_{ij})=a} &
}}\tag{III}
\end{equation*}
commutes.

Finally the identity element $1\in B$ can be lifted to an element
$e\in R^{n}$ (determines a homomorphism $R\to R^{n}$) and there is a
map $\epsilon:R^{n}\to R^{m}$ such 
\begin{equation*}
\vcenter{
\xymatrix{
R^{ne}\otimes b-\Id \ar[dr]_{\epsilon}\ar[rr] & & R^{n}\\
 & R^{m}\ar[ur]_{a=(a_{ij})} &
}}\tag{IV}
\end{equation*}
commutes, i.e., $e\otimes b-\Id=0$ in $B$.

Let us then define the functor $G:$ (Rings) $\to$ (Sets) as follows:
\begin{description}
\item[$G(R)=${\rm(i)}] {\em Set of homomorphisms} $a:R^{m}\to R^{n}$
  such that $rk\ a\leq (n-r)$ (i.e., determinants of all minors of $a$
  of $rk(n-r+1)$ vanish), {\em together with} (0)

\item[{\rm(ii)}] {\em Set of homomorphisms} $b$, $c$
\[
\xymatrix{
R^{n}\otimes R^{n}\ar[d]_{b} \ar@{}[r]|-{,} & (R^{m}\otimes
R^{n})\oplus (R^{n}\otimes R^{m})\ar[d]_{c}\\
R^{n} & R^{m}
}
\]
such\pageoriginale that the following diagram is commutative:
\begin{equation*}
\vcenter{
\xymatrix@C=2cm{
(R^{m}\otimes R^{n})\oplus (R^{n}\otimes R^{m})\ar[d]_{c}
  \ar[r]^-{a\otimes \Id\oplus \Id\otimes a} & R^{n}\otimes
  R^{n}\ar[d]^{b}\\
R^{m}\ar[r]^-{a} & R^{n}
}}\tag{I}
\end{equation*}
together with

\item[{\rm(iii)}] an $R$-homomorphism $\alpha:R^{n}\otimes
  R^{n}\otimes R^{n}\to R^{m}$ such that the following is commutative
\begin{equation*}
\vcenter{
\xymatrix{
R^{n}\otimes R^{n}\otimes R^{n}\ar[dr]\ar[rr]^-{bo(\Id\otimes
  b)-bo(b\otimes \Id)} & & R^{n}\\
 & R^{m}\ar[ur]_-{a} &,
}}\tag{II}
\end{equation*}
and

\item[{\rm(iv)}] {\em an $R$-homomorphism $\delta:R^{n}\otimes
  R^{n}\to R^{m}$ such that}
\begin{equation*}
\vcenter{
\xymatrix{
R^{n}\otimes R^{n}\ar[dr]_{\delta}\ar[rr]^-{b-\tilde{b}} & & R^{n}\\
 & R^{m}\ar[ur]_-{a} &
}}\tag{III}
\end{equation*}
commutes, and

\item[{\rm(v)}] an $R$-homomorphism $e:R\to R^{n}$ and
  $\epsilon:R^{n}\to R^{m}$ such that
\begin{equation*}
\vcenter{
\xymatrix{
R^{n}\ar[dr]_{\epsilon}\ar[rr]^-{e\otimes b-\Id} & & R^{n}\\
 & R^{m}\ar[ur]_-{a} &
}}\tag{IV}
\end{equation*}
commutes.
\end{description}

It is now easily seen that $G(R)$ can be identified with a subset
$S\hookrightarrow R^{P}$ such that there exist polynomials
$F_{i}(X_{1},\ldots,X_{p})$ over $\mathbb{Z}$ such that
$s=(s_{1},\ldots,s_{p})\in S$ iff $F_{i}(s_{1},\ldots,s_{p})=0$, and
the set $\{F_{i}\}$ and $p$ are independent\pageoriginale of $R$. From
this it is clear that $G(R)$ is represented by a scheme $V$ of finite
type over $\mathbb{Z}$ and since $G(R)\to F(R)$ is surjective, Lemma
\ref{part2-lem5.1} follows immediately.

It follows in particular that $\overline{\mathscr{O}}$ is represented
by {\em a homomorphism} $\varphi:\Spec \hat{P}\to
V(\hat{P}=A[y_{1},\ldots,y_{m}]^{\wedge})$. 
\begin{itemize}
\item[(6)] {\em The image of $\Spec \hat{P}-\{y=0\}$ in $V$ lies in
  the smooth locus of $V$ over $\Spec \mathbb{Z}$.}
\end{itemize}

We have a representation
\begin{equation*}
\hat{P}[z_{1},\ldots,z_{s}]\to \overline{\mathscr{O}}\to 0\tag{*}
\end{equation*}
(homomorphisms of rings and homomorphisms as $P$ modules). Now
$\hat{P}[z_{1},\ldots,z_{s}]$ is {\em regular} over $A$ and
$\overline{X}-\overline{\Gamma}$ is {\em regular} over $\Spec
A-\{y=0\}$. Hence the immersion $\overline{X}\hookrightarrow \Spec
\hat{P}[z_{1},\ldots,z_{s}]\simeq \mathbb{A}^{s}_{\hat{P}}$, being an
$A$-morphisms, is a local complete intersection at every point of
$\overline{X}-\overline{\Gamma}$ (we use the fact that a regular local
ring which is the quotient of another regular local ring is a complete
intersection in the latter; we use this fact for the corresponding
local rings of the fibres and then lifting the $m$-sequence,
etc.). Now $\codim \overline{X}$ in $\mathbb{A}^{s}_{\hat{P}}$ is
$s$. Now take a closed point $x_{0}\in \Spec \hat{P}-\{y=0\}$. Then
tensoring $(*)$ by $k(x_{0})$ we get
$$
k(x_{0})[z_{1},\ldots,z_{s}]\to
\mathscr{O}\otimes_{\hat{P}}k(x_{0})\to 0\quad\text{exact.}
$$

Now $\Spec \hat{\mathscr{O}}\otimes_{\hat{P}}k(x_{0})$ is precisely
the fibre of $\overline{X}$ over $x_{0}$ for the morphism
$\overline{X}\hookrightarrow \Spec \hat{P}$. We {\em claim} that
$\Spec \overline{\mathscr{O}}\otimes_{\hat{P}}k(x_{0})$ is also a
local complete intersection in $k(x_{0})[z_{1},\ldots,z_{s}]$ wherever
$x_{0}\in \Spec P-\{y=0\}$ (it is a $0$-dimensional\pageoriginale
subscheme of $\Spec k(x_{0})[z_{1},\ldots,z_{s}]\hookrightarrow
\mathbb{A}^{s}_{k(x_{0})}$) and in fact that $\Spec
\overline{\mathscr{O}}\otimes_{\hat{P}}R_{0}\hookrightarrow \Spec
R_{0}[z_{1},\ldots,z_{s}]$ is a {\em morphism of local complete
  intersection} over $R_{0}$ for any base change $\Spec R\to \Spec
\hat{P}-\{y=0\}$ (i.e., flat and the fibres of the morphism over
$\Spec R$ is a local complete intersection). To prove this we note
that $\overline{\mathscr{O}}\otimes_{\hat{P}}R_{0}$ is locally free
(of rank $r$) ($\Spec R_{0}\to \Spec\hat{P}-\{y=0\}$) and $\Spec
R_{0}[z_{1},\ldots,z_{s}]\to \Spec R_{0}$ is a regular morphism. In
particular a Cohen-Macaulay morphism. Now the claim is an immediate
consequence of

\begin{lemma}\label{part2-lem5.2}
Let $B$, $C$, $R$ be local rings such that $B$, $C$ are $R$-algebras
flat over $R$ and $B$ is Cohen-Macaulay over $R$. Let $f:B\to C$ be a
surjective (local) homomorphism of $R$-algebras such that $C$ is a
complete intersection in $B$. Then for all $R\to R_{0}\to 0$, the
surjective homomorphism
$$
B\otimes R_{0}\to C\otimes R_{0}\to 0
$$
(the morphism $\Spec (C\otimes R_{0})\hookrightarrow \Spec (B\otimes
R_{0})$) is a morphism of complete intersection over $\Spec R_{0}$.
\end{lemma}

\begin{proof}
Since the flatness hypothesis is satisfied, it suffices to prove that
$C\otimes k$ is a complete intersection in $B\otimes
k(k=R/m_{R})$. Now we have
$$
0\to I\to B\to C\to 0\quad\text{exact,}
$$
$I=\ker B\to C$, and $I=(f_{1},\ldots,f_{s})$, $f_{i}$ is an
$m$-sequence in $B$. Now the codimension of $C$, $\Spec (C\otimes k)$
in $\Spec (B\otimes k)$, is equal to the codimension of $\Spec C$ in
$\Spec B$, which is $s$. (This follows by flatness of $B$, $C$ over
$R$ and\pageoriginale the fact that flat implies equidimensional.) Let
$\overline{f}_{i}$ denote the canonical images of $f_{i}$ in $B\otimes
k$. We have then $(B\otimes
k)/(\overline{f}_{1},\ldots,\overline{f}_{s})=C\otimes k$. Now
$(B\otimes k)$ is Cohen-Macaulay and the codimension of $\Spec
(C\otimes k)$ in $\Spec(B\otimes k)$ is $s$. It follows by Macaulay's
theorem that $\overline{f}_{1},\ldots,\overline{f}_{s}$ is an
$m$-sequence in $\overline{B}\otimes k$. This implies that
$\overline{C}\otimes k$ is a complete intersection in
$\overline{B}\otimes k$, and proves Lemma \ref{part2-lem5.2}.

\smallskip
\noindent
{\bf The complete intersection trick.}~
We go back to the proof of (5). Let $\lambda:\Spec R_{0}\to \Spec
\hat{P}$ be a morphism such that $R_{0}$ is Artin local and
$\lambda(\Spec R_{0})\subset \Spec \hat{P}-\{y=0\}$. Consider the
morphism $(\varphi\circ \lambda):\Spec R_{0}\to V$. Then
$(\varphi\circ \lambda)$ defines an $R_{0}$ algebra $B$ which is a
free $R_{0}$-module of rank $r$ (in particular flat over $R_{0}$), and
$B$ is a morphism of local complete intersection over $R_{0}$ ($B$ is
of relative dimension $0$ over $R_{0}$). Let $R\to R_{0}\to 0$ be such
that $\Spec R$ is an infinitesimal neighborhood of $\Spec R_{0}$. Then
the assertion (5) follows if we show that $(\varphi\circ\lambda):\Spec
R_{0}\to V$ factors through $\Spec R\to V$.

Now $B$ is defined by
$$
R^{m}_{0}\xrightarrow{a^{0}}R^{n}_{0}\to B\to 0,
$$
where $a^{0}=(a^{0}_{ij})$. Now $\Spec B\to \Spec R_{0}$ is a morphism
of local complete intersection for a suitable imbedding and $\Spec
B$. It is clear that $\Spec B\to \Spec R_{0}$ is in fact a morphism of
global complete intersection for the corresponding imbedding, for it
is easily seen that a $0$-dimensional closed subscheme of
$\mathbb{A}^{n}_{k}$ which is a local complete intersection, is in
fact a global intersection. We have seen in \S\ \ref{part1-sec4}, Part
\ref{part1}, That the {\em functor of global deformations of a
  complete intersection is unobstructed i.e., formally smooth.}
Hence\pageoriginale there is a {\em flat} $R$-algebra $B'$ such that
$B'\otimes_{R}R_{0}\approx B$. Hence the sequence
$R^{m}_{0}\xrightarrow{a_{0}}R^{n}_{0}\to 0$ can be lifted to an exact
sequence $R^{m}\xrightarrow{a}R^{n}\to B'\to 0$; the proof of this is
in spirit analogous to imbedding a deformation (infinitesimal) of
$X\hookrightarrow \mathbb{A}^{n}$ in the same affine space. A
homomorphism $(R^{n}_{0}\to B)$ over $R_{0}$ is given by $n$ elements
in $B$. Hence this homomorphism can be lifted to $R^{n}\to B'$ and it
becomes surjective. Now it is seen easily that $R^{m}_{0}\to
R^{n}_{0}$ can be lifted to $R^{m}\to R^{n}$. It follows that
determinants of all minors of $a$ of rank $(n-r+1)$ vanish. Thus the
relations (0) above can be lifted to $R$. 

Consider the relations (I). We are ginve $b_{0}$, $c_{0}$ such that
the following diagram is commutative:
\[
\xymatrix@C=2.5cm{
(R^{m}_{0}\otimes R^{n}_{0})\oplus (R^{n}_{0}\otimes
  R^{m}_{0})\ar[d]_{c_{0}}\ar[r]^-{a_{0}\otimes \Id \oplus \Id \otimes
  a_{0}=\psi_{0}} & R^{n}_{0}\otimes R^{n}_{0}\ar[d]^{b_{0}}\\
R^{m}_{0}\ar[r]^-{a_{0}} & R^{n}_{0}.
}
\]

With the above lifting of $(a_{0})$ to $a$ we get
\begin{landscape}
\begin{equation*}
\vcenter{
\xymatrix@C=1.5cm{
(R^{m}\otimes R^{n})\oplus (R^{n}\otimes
  R^{m})\ar[dd]^{c}_{\exists}\ar[r]^-{a\otimes\Id\oplus\Id\otimes a} &
  R^{n}\otimes R^{n}\ar@{-->}[dd]_{b}^{\exists}\ar[r]^-{\text{can}} &
  B'\otimes_{R}B'\to 0\ar[dd]^{\text{can}}\ar[r] & 0\\
\ar@{}[r]|{\rm(B)} & \ar@{}[r]|{\rm(A)} & \\
R^{m}\ar[r]^-{a} & R^{n}\ar[r] & B\ar[r] & 0.
}}\tag{*}
\end{equation*}
\end{landscape}
The two rows are exact. We claim {\em that there is a $b$ which lifts
  $b_{0}$, and such that the square {\em(A)} is commutative.} In fact this
  is quite easy, for it is clear that we can find $b':R^{n}\otimes
  R^{n}\to R^{n}$ such that (A) is commutative. Let $b'_{0}$ be the
  reduction mod $R_{0}$ of $b'$. Then we see that the composition of
  the\pageoriginale canonical map $R^{n}_{0}\to B$ with $b_{0}-b'_{0}$
  is zero. Hence
$$
b_{0}(z)\equiv b'_{0}(z)\quad (\text{mod\,}\Iim a_{0}, \text{~ or~
}\Ker R^{n}_{0}\to B).
$$

Now $b_{0}(z)-b'_{0}(z)$ can be lifted to elements in $\Iim a$, so
that taking for $\{z_{a}\}$ a canonical basis in $R^{n}\to R^{n}$ we
define $b:R^{n}\otimes R^{n}\to R^{n}$
$$
b(z_{a})=b'(z_{a})+\theta_{a},
$$
where $\theta_{a}\in \Iim a$ lifts $b_{0}(a)-b'_{0}(z)$. It is now
clear that $b$ lifts $b_{0}$ and the square (A) is commutative. This
proves the claim.

By a similar argument as above there is a $c$ such that $c$ reduces to
$c_{0}$ and the square (B) is commutative. (If necessary we can
prolong to the left the exact sequence of the second row in the
diagram (*).) Thus it follows that the relations in $I$ can be lifted
to $R$.

A similar argument shows that $a_{0}$, $\delta_{0}$, $e_{0}$ and
$\epsilon_{0}$ which are given representing the point $\Spec R_{0}\to
V$ can be lifted to $R$ so that the diagrams (II), (III) and (IV) 
are still commutative. This means that the $\Spec R_{0}\to V$ can be
lifted to an $R$-valued point of $V$. As we remarked before, the
assertion (5) now follows.

We go back to the usual notations in the theorem. Then: (7) {\em Let
  $\mathfrak{a}=m_{A}(y_{1},\ldots,y_{d})$ ideal in $A[y]$. Then
  $\hat{P}=A[y]^{\wedge}$ is also the $\mathfrak{a}$-adic completion of
  $A[y]$ (of course $\hat{P}$ is also the $m_{A}\cdot (A[y])$-adic
  completion of $A[y]$).}

The proof of this assertion is immediate for a convergent series
$\sum\limits_{i}f_{i}$, $f_{i}\in A[y]$ in the $\mathfrak{a}$-adic
topology is precisely one such that the coefficients of $f_{i}$ tend
to zero in the $m_{A}$-adic topology and the degree of the monomials
$\to \infty$.  This\pageoriginale implies that a convergent series is
precisely a formal power series in $\{y_{i}\}$ such that the
coefficients (in $A$) tend to $0$ in the $m_{A}$-adic topology. This
is the description of $\hat{P}$ we had and (7) is proved. (8) Now for
the morphism $\varphi:\Spec \hat{P}\to V$ we have that the image of
$\Spec \hat{P}-\{y=0\}$ is in the smooth locus of $V$ over
$\mathbb{Z}$. {\em We note that} $V(\mathfrak{a})=(\Spec k[y])\cup
  \{y=0\}$ ($k$ {\em residue field of $A$}), {\em i.e.,}
  $V(\mathfrak{a})$ {\em contains} $\{y=0\}$ so that the image of
  $\Spec \hat{P}-V(\mathfrak{a})$ (by $\varphi$) is also in the smooth locus
  of $V$ over $\mathbb{Z}$. Now the $\mathfrak{a}$-adic completion of
  $A[y]$ is $\hat{P}$. Let $\widetilde{P}$ denote the henselization of
  $(P,\mathfrak{a})$, $P=A[y]$. Now apply Theorem 2
  proved above. Hence we can find an \'etale map $\Spec R\to \Spec P$
  which is trivial over $V(\mathfrak{a})$ and morphism $\varphi':\Spec
  R\to V$ (note that $\hat{P}$ is also the $\mathfrak{a}$-adic
  completion of $\widetilde{P}$) such that
$$
\varphi'\equiv \varphi(\text{mod}~~{}^{N}),\quad\text{for any given~
}N.
$$
(Note that the $\mathfrak{a}$-adic (i.e., $\mathfrak{a}R$-adic,
$\mathfrak{a}$ is not an ideal in $R$) completion of $R$ is also
$A|y|^{\wedge}$). Let $\mathcal{O}'$ be the $R$-algebra defined by
$\varphi'$. Then we have
$\mathscr{O}\equiv\overline{\mathscr{O}}(\text{mod~}\mathfrak{a}^{N})$
(i.e.,
$\mathscr{O}'/\mathfrak{a}^{N}=\overline{\mathscr{O}}/\mathfrak{a}^{N}$). Now
$\mathscr{O}'$ becomes an $A$-algebra and then we have
$$
\mathscr{O}'\equiv \overline{\mathscr{O}}(\text{mod~}m^{N}_{A})  
$$
for $(m_{A}R)^{N}\supset \mathfrak{a}^{N}$. Take in particular
$N=2$. Thus we can find an $R$-algebra $\mathscr{O}'$ of finite type
and consequently of {\em finite type over} $A$ such that
$$
\mathscr{O}'\equiv\overline{\mathscr{O}}(\text{mod\,}m^{2}_{A}).
$$

Thus to conclude the proof of the theore, it suffices to prove that
$\mathscr{O}'$ is {\em flat} over $A$.

\smallskip
(9)~ {\em Choice\pageoriginale of $\mathscr{O}'$ such that
  $\mathscr{O}'$ is flat/$A$.}

We had a presentation of $\overline{\mathscr{O}}$ as follows:
$$
\hat{P}^{m}\xrightarrow{a}\hat{P}^{n}\to \overline{\mathscr{O}}\to 0.
$$

We claim that we have a presentation such that
\begin{equation*}
\begin{cases}
\hat{P}^{\ell}\xrightarrow{\theta}\hat{P}^{m}\xrightarrow{a}\hat{P}^{n}\to
\overline{\mathscr{O}}\to 0,\\
\text{where}\quad a\cdot \theta=0\quad\text{and}\quad
\hat{P}^{m}\xrightarrow{a}\hat{P}^{n}\to \overline{\mathscr{O}}\to
0\text{~ is exact, and}\\
\hat{P}^{\ell}\otimes_{A}k_{A}\xrightarrow{\theta\otimes
  k_{A}}\hat{P}^{m}\otimes k_{A}\xrightarrow{a\otimes
  k_{A}}\hat{P}^{n}\otimes k_{A}\to \overline{\mathscr{O}}\otimes
k_{A}\to 0\text{~ is exact,}
\end{cases}\tag{*}
\end{equation*}
where $k_{A}=A/m_{A}$.

This follows if we prove that $\overline{\mathscr{O}}$ is flat over
$A$ (cf., \S\ \ref{part1-sec3}, Part \ref{part1}). However, (*) can be
established directly as follows: We can find an exact sequence of the
form
$$
\hat{P}^{\ell}\otimes
k_{A}\xrightarrow{\overline{\theta}_{0}}\hat{P}^{m}\otimes
k_{A}\xrightarrow{\overline{a}_{0}}\hat{P}^{n}\otimes k_{A}\to
\overline{\mathscr{O}}\otimes k_{A}\to 0.
$$
(Note that $\Spec \overline{\mathscr{O}}\otimes
k_{A}=X_{0}=\Spec\mathscr{O}_{0}$ is the scheme whose deformation we
are considering and that $X_{n}=\overline{\mathscr{O}}\otimes
A/m^{n+1}_{A}=\Spec \mathscr{O}_{n}$ are infinitesimal deformations of
$X_{0}$.) The above exact sequence can be lifted to an {\em exact
  sequence}
$$
\hat{P}^{\ell}\otimes
A/m^{n+1}_{A}\xrightarrow{\overline{\theta}_{n}}\hat{P}^{m}\otimes
A/m^{n+1}_{A}\xrightarrow{\overline{a}_{n}}\hat{P}^{n}\otimes
A/m^{n+1}_{A}\to \mathscr{O}_{n}\to 0
$$
since $\mathscr{O}_{n}$ is flat over $A/m^{n+1}_{A}$. Passing to the
limit, we have an exact sequence
$$
\hat{P}^{\ell}\otimes
A/m^{n+1}_{A}\xrightarrow{\overline{\theta}_{n}}\hat{P}^{m}\otimes
A/m^{n+1}_{A}\xrightarrow{\overline{a}_{n}}\hat{P}^{n}\otimes
A/m^{n+1}_{A}\to \mathscr{O}_{n}\to 0
$$
since $\mathscr{O}_{n}$ is flat over $A/m^{n+1}_{A}$. Passing to the
limit, we have an exact sequence
$$
\hat{P}^{\ell}\xrightarrow{\overline{\theta}}P^{m}\xrightarrow{\overline{a}}P^{n}\to
\overline{\mathscr{O}}\to 0
$$
such\pageoriginale that $\overline{a}\cdot \overline{\theta}=0$, and
$\hat{P}^{m}\xrightarrow{\overline{a}}P^{n}\to
\overline{\mathscr{O}}\to 0$ is exact. This proves the existence of
(*).

Now define a functor $G'$ which is a modification of $G$ as follows:
$G'(R)=$ Set of $\{\theta,a,b,c,\alpha,\delta,e, \epsilon$ where
$a,b,c,\alpha,\delta,e,\epsilon$ are as in definition of  $G(R)$; and
$\theta$ is defined by
$R^{\ell}\xrightarrow{\theta}R^{m}\xrightarrow{a}R^{n}$ with
$a\cdot\theta=0\}$. 

Then as in the case of $G(R)$, we see that $G'$ is represented by a
scheme $V'$ of finite type over $\mathbb{Z}$. The given representation
for $\overline{\mathscr{O}}$ as in $(*)$ above gives rise to a
morphism $\psi:\Spec P\to V'$. We claim that as in the case of $V$,
the image of $\Spec P-\{y=0\}$ lies in the smooth locus of $V'$. With
the same notations for $R$, $R_{0}$ as in the proof of the statement
for the case $V$, it suffices to prove the following: Given
$$
R^{\ell}_{0}\xrightarrow{\theta_{0}}R^{m}_{0}\xrightarrow{a_{0}}R^{n}_{0}\to
B\to 0
$$
such that $a_{0}\cdot\theta_{0}=0$ and
$R_{0}^{m}\xrightarrow{a_{0}}R^{n}_{0}\to B\to 0$ is exact, and a flat
lifting $B'$ over $R$ (hence free over $R$), we have to lift this
sequence to $R$ (the proof of the lifting of the quantities involved
is the same as for $G(R)$). As we have seen before, for $G(R)$ we have
a lifting
$$
R^{m}\xrightarrow{a}R^{n}\to B'\to 0.
$$

Now $\Iim \theta_{0}$ are a set of relations. Since $B'$ is flat over
$R$ these relations can also be lifted, i.e., we have a lifting
$$
R^{\ell}\xrightarrow{\theta}R^{m}\xrightarrow{a}R^{n}\to B'\to 0
$$
such that $a\cdot \theta=0$ and $R^{m}\xrightarrow{a}R^{n}\to B'\to $
is exact. This proves the required claim\pageoriginale and hence it
follows $\psi(\Spec P-\{y=0\})$ lies in the smooth locus (over
$\mathbb{Z}$) of $V'$.

Applying Theorem 2, we can find an \'etale $\Spec R\to
\Spec P$, $P=A[y]$ and a morphism $\psi':\Spec R\to V'$ such that
$$
\psi'\equiv \psi (\text{mod\,}~\mathfrak{a}^{2}).
$$

Let $\mathscr{O}'$ be the $R$-algebra defined by $\psi'$. Then as we
have seen before, we have
$$
\mathscr{O}'\equiv\overline{\mathscr{O}}(\text{mod\,}m^{2}_{A}).
$$

We claim that $\mathscr{O}'$ is flat over $A$. For this we observe
that we have {\em a fortiori}
$$
\mathscr{O}'\equiv\overline{\mathscr{O}}(\text{mod\,}m_{A}).
$$

This implies that $\mathscr{O}'/m_{A}\cdot
\mathscr{O}'=\overline{\mathscr{O}}/m_{A}\overline{\mathscr{O}}\approx
\mathscr{O}_{0}$. Let
\begin{equation*}
\begin{split}
 & \qquad \hat{P}^{\ell}\xrightarrow{\theta'}\hat{P}^{m}\xrightarrow{a'}\hat{P}^{n}\to
\mathscr{O}'\to 0\\
&  a' \circ \theta'=0,\ \hat{P}^{m}\to \hat{P}^{n}\to \hat{P}^{n}\to
\mathscr{O}'\to 0\quad\text{exact}
\end{split}\tag{I$'$}
\end{equation*}
be a representation of $\mathscr{O}'$. Recall we have the
representation for $\overline{\mathscr{O}}$
\begin{equation*}
\hat{P}^{\ell}\xrightarrow{\theta}\underbrace{\hat{P}^{m}\xrightarrow{a}\hat{P}^{n}\to
\overline{\mathscr{O}}\to 0}_{\text{exact}},\quad\text{and}\quad
a\circ \theta=0.\tag{I}
\end{equation*}

Now tensoring (I$'$) with $A/m^{n}_{A}$ yields
$$
\hat{P}^{\ell}\otimes A/m^{n}_{A}\xrightarrow{\theta'\otimes
  A/m^{n}_{A}}\underbrace{\hat{P}^{m}\otimes
  A/m^{n}_{A}\xrightarrow{a'\otimes A/m^{n}_{A}}\hat{P}^{n}\otimes
  A/m^{n}_{A}\to \mathscr{O}'\otimes Am^{n}_{A}\to 0}_{\text{exact}},
$$
and
$$
(a'\otimes A/m^{n}_{A})\cdot (\theta'\otimes A/m^{n}_{A})=0.
$$

We\pageoriginale have $(I')\otimes A/m_{A}=(I)\otimes A/m_{A}$, as a
consequence of the fact that $\mathscr{O}'\equiv
\overline{\mathscr{O}}(\text{mod\,}m_{A})$. By $(I)$ $I\otimes
A/m_{A}$ is {\em exact}. This implies that $(I')\otimes A/m_{A}$ is
exact. So we have that $(I')\otimes A/m_{A}$ is {\em exact}, and
$(I')\otimes A/m^{n}_{A}$ has the property, $(a'\otimes
A/m^{n}_{A})\cdot (\theta'\otimes A/m^{n}_{A})=0$ and
$$
\hat{P}^{m}\otimes A/m^{n}_{A}\xrightarrow{a'\otimes
  A/m^{n}_{A}}\hat{P}^{n}\otimes A/m^{n}_{A}\to \mathscr{O}'\otimes
A/m^{n}_{A}\to 0,
$$
for all $n$. This implies, as we saw in the first few lectures, that
$\mathscr{O}'\otimes A/m^{n}_{A}$ is flat over $A/m^{n}_{A}$ for every
$n$.

Now $\mathscr{O}'$ is an $A$-algebra of finite type and so
$\mathscr{O}'\otimes A/m^{n}_{A}$ flat over $A/m^{n}_{A}$ for all $n$
implies that $\mathscr{O}'$ is flat over $A$ (cf.\@ SGA exposes on
flatness). 

The proof of the theorem is now complete.
\end{proof}

\begin{remark}\label{part2-rem5.1}
The fact that $\mathscr{O}'$ is flat over $A$ can also be shown in a
different manner. This can be done using only the functor $G$ (i.e.,
$V$), but a better approximation (i.e., better than $N=2$) for
$\mathscr{O}'$ may be needed. This uses the following result of
Hironaka: Let $B$ be a complete local ring, $b\subset m$ an ideal in
$B$ and $M$ a finite $B$-module locally free (of rank $r$) outside
$V(b)$. Then there is an $N$ such that whenever $M'$ is a finite
$B$-module locally free of rank $r$ outside $V(b)$ and
$M'=M(\text{mod\,}b^{N})$, then $M'\approx M$. Take in our present
case $B=A[[y]]$ so that we have
\[
\xymatrix{
A\ar@{-}[ddrr]\ar@{^{(}->}[rrrr] && & & A[y]^{\wedge} \;\; \ar@{^{(}->}[r] &
A[[y]]\\
 & && R\ar@{^{(}->}[ur] & &\\
&& A[y]\ar@{^{(}->}[ur] & & &
}
\]
$R$ \'etale over $A[y]$ such that $A[y]/\mathfrak{a}\approx
R/\mathfrak{a}$ and that all the extensions are faithfully flat. Take
a coherent sheaf on $\Spec R$; to verify that it is flat over
$A$\pageoriginale it suffices to verify that its lifting to $\Spec
A[[y]]$ is flat over $A$. Take $\underline{b}$ to be
$\mathfrak{a}\cdot A[[y]]$. Then by taking a suitable approximation
for $\mathscr{O}'$ it follows that the liftings to $A[[y]]$ of
$\mathscr{O}'$ and $\overline{\mathscr{O}}$ are isomorphic. This
implies that $\overline{\mathscr{O}}\otimes A/m^{n}\approx \mathscr{O}'\otimes
A/m^{n}$, hence that $\mathscr{O}'$ is flat over $A$, since $\Spec
(\overline{\mathscr{O}}\otimes A/m^{n})=X_{n}$ is flat over $A$. 
\end{remark}
