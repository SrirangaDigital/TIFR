\chapter{Random Time Changes}\label{chap28}

LET\pageoriginale
$$
L=\frac{1}{2}\sum\limits_{i,j}a_{ij}\frac{\p^{2}}{\p x_{i}\p
  x_{j}}+\sum_{j}b_{j}\frac{\p}{\p x_{j}}
$$
with $a:\mathbb{R}^{b}\to S^{+}_{d}$ and $b:\mathbb{R}^{d}\to
\mathbb{R}^{d}$ bounded measurable funcitons. Let $X(t,\cdot)$, given
by $X(t,w)=w(t)$ for $(t,w)$ in $[0,\infty)\times
  C([0,\infty):\mathbb{R}^{d})$ be an It\^o process corresponding to
    $(\Omega,\mathscr{F}_{t},Q)$ with parameters $b$ and $a$ where
    $\Omega=C([0,\infty);\mathbb{R}^{d})$. For every constant $c>0$
      define
$$
L_{c}\equiv c\left[\frac{1}{2}\sum_{i,j}a_{ij}\frac{\p^{2}}{\p x_{i}\p
    x_{j}}+\sum_{j}b_{j}\frac{\p}{\p x_{j}}\right]
$$

Define $Q_{c}$ by $Q_{c}=PT^{-1}_{c}$ where $(T_{c}w)(t)=w(ct)$. Then
one can show that $X$ is an It\^o process corresponding to
$(\Omega,\mathscr{F}_{t},Q_{c})$ with parameters $cb$ and $ca$ [Note:
  We have done this in the case where $a_{ij}=\delta_{ij}$].

Consider the equation
$$
\frac{\p u}{\p t}=L_{c}u\q\text{with}\q u(0,x)=f(x).
$$

This can be written as $\dfrac{\p u}{\p \tau}=Lu$ with $u(0,x)=f(x)$
when $\tau=ct$. Thus changing time in the differential equation is
equivalent to stretching time in probablistic language.

So far we have assumed that $c$ is a constant. Now we shall allow $c$
to be a function of $x$.

Let $\phi:\mathbb{R}^{d}\to \mathbb{R}$ be any bounded measurable
function such that 
$$
0<C_{1}\leq \phi(x)<C_{2}<\infty,\q \forall x\in \mathbb{R}^{d}
$$
and suitable constants $C_{1}$ and $C_{2}$. If
$$
L\equiv \left[\frac{1}{2}\sum a_{ij}\frac{\p^{2}}{\p x_{i}\p
    x_{j}}+\sum b_{j}\frac{\p}{\p x_{j}}\right]
$$\pageoriginale
we define
$$
L_{\phi}=\phi L\equiv \phi\left[\frac{1}{2}\sum a_{ij}\frac{\p^{2}}{\p
    x_{i}\p x_{j}}+\sum b_{j}\frac{\p}{\p x_{j}}\right].
$$

In this case we can say that the manner which time changes depends on
the position of the particle.

Define $T_{\phi}:\Omega\to \Omega$ by
$$
(T_{\phi}w)(t)=w(\tau_{t}(w))
$$
where $\tau_{t}(w)$ is the solution of the equation
$$
\int\limits^{\tau_{t}}_{0}\frac{ds}{\phi(w(s))}=t.
$$

As $C_{1}\leq \phi\leq C_{2}$ it is clear that
$\tau_{t}\dfrac{1}{C_{1}}\leq t\leq \tau_{t}\dfrac{1}{C_{2}}$. When
$\phi\equiv c$ a constant, then $\tau_{t}=ct$ and $T_{\phi}$ coincides
with $T_{c}$. 

As
$$
0<C_{1}\leq \phi\leq C_{2}<\infty,\q
\int\limits^{\lambda}_{0}\frac{1}{\phi(w(s))}ds 
$$
is continuous and increases strictly from $0$ to $\infty$ as $\lambda$
increases, so that $\tau_{t}$ exists, is unique, and is a continuous
function of $t$ for each fixed $w$.

\medskip
\noindent
{\em Some properties of $T_{\phi}$.}
\begin{enumerate}
\renewcommand{\theenumi}{\roman{enumi}}
\renewcommand{\labelenumi}{(\theenumi)}
\item If $l$ is the constant function taking the value $1$ then it is
  clear that $T_{l}=$ identity.

\item Let\pageoriginale $\phi$ and $\psi$ be two measurable funcitons
  such that $0<a\leq \phi(x)$, $\psi(x)\leq b<\infty$, $\forall x\in
  \mathbb{R}^{d}$. Then $T_{\phi}\circ T=T_{\phi\psi}=T_{\psi}\circ T_{\phi}$.
\end{enumerate}

\begin{proof}
Fix $w$. Let $\tau_{t}$ be given by
$$
\int\limits^{\tau_{t}}_{0}\frac{1}{\phi(w(s))}ds=t.
$$

Let $w^{*}(t)=w(\tau_{t})$ and let $\sigma_{t}$ be given by
$$
\int\limits^{\sigma_{t}}_{0}\frac{1}{\phi(w^{*}(s))}ds=t.
$$

Let $w^{**}(t)=w^{*}(\sigma_{t})=w(\tau_{\sigma_{t}})$. Therefore
\begin{align*}
((T_{\psi}\circ T_{\phi})w)(t) &= (T_{\phi}w^{*})(t)=w^{*}(\sigma_{t})\\
&= w^{**}(t)=w(\tau_{\sigma_{t}}).
\end{align*}

Hence to prove the property (ii) we need only show that
$$
\int\limits^{\tau_{\sigma_{t}}}_{0}\frac{1}{\phi(w(s))}\frac{1}{\psi(w(s))}ds=t. 
$$

Since
$$
\int\limits^{\tau_{t}}_{0}\frac{1}{\phi(w(s))}ds=t,\q
\frac{dt}{d\tau_{t}}=\frac{1}{\phi(w(\tau_{t}))} 
$$
and
$$
\frac{dt}{d\sigma_{t}}=\frac{1}{\psi(w^{*}(\sigma_{t}))}=\frac{1}{\psi(w(\tau_{\sigma_{t}}))} 
$$

Therefore
\begin{align*}
\frac{d\tau_{\sigma_{t}}}{dt}=\frac{d\tau_{\sigma_{t}}}{d\sigma_{t}}-\frac{d_{\sigma_{t}}}{dt}
&=\phi(w(\tau_{\sigma_{t}}))\phi(w^{*}(\sigma_{t}))\\
&=\phi(w(\tau_{\sigma_{t}})\psi(w(\tau_{\sigma_{t}}))\\
&=(\phi\psi)(w(\tau_{\sigma_{t}})).
\end{align*}\pageoriginale

Thus
$$
\int\limits^{\tau_{\sigma_{t}}}_{0}\frac{1}{(\phi\psi)(w(s))}ds=t.
$$

This completes the proof.
\end{proof}

\begin{enumerate}
\renewcommand{\theenumi}{\roman{enumi}}
\renewcommand{\labelenumi}{(\theenumi)}
\setcounter{enumi}{2}
\item From (i) and (ii) it is clear that $T^{-1}_{\phi}=T_{\phi-1}$
  where $\phi^{-1}=\dfrac{1}{\phi}$.

\item $(\tau_{t})$ is a stopping time relative to $\tau_{t}$. i.e.\@
$$
\left\{w:\int\limits^{\lambda}_{0}\frac{1}{\phi(w(s))}ds\geq
r\right\}\in \lambda \text{~ for each~ }\lambda\geq 0.
$$

\item $T_{\phi}(w)(t)=w(\tau_{t}w)=X_{\tau_{t}}(w)$.

Thus $T_{\phi}$ is
$(\mathscr{F}_{t}-\mathscr{F}_{\tau_{t}})$-measurable, i.e.\@
$T^{-1}_{\phi}(\mathscr{F}_{t})\subset \mathscr{F}_{\tau_{t}}$.
\end{enumerate}

Since $X(t)$ is an It\^o process, with parameters $b$, $a$, $\forall
f\in C^{\infty}_{0}(\mathbb{R}^{d})$,
$f(X(t))-\int\limits^{t}_{0}(Lf)(X(s))ds$ is a martingale relative to
$(\Omega,\mathscr{F}_{t},P)$. By the optional sampling theorem
$$
f(X_{\tau_{t}})-\int\limits^{\tau_{t}}_{0}(Lf)(X(s))ds
$$
is a martingale relative to $(\Omega,\mathscr{F}_{\tau_{t}},P)$, i.e.
$$
f(X_{\tau_{t}})-\int\limits^{t}_{0}(Lf)(X(\tau_{s}))d\tau_{s}
$$
is a martingale relative to $(\Omega,\mathscr{F}_{\tau_{t}},P)$. But
$\dfrac{d\tau_{s}}{dt}=\phi$. Therefore 
$$
f(X(\tau_{t}))-\int\limits^{t}_{0}(Lf)(X_{\tau_{s}})\phi(X_{\tau_{s}})ds
$$\pageoriginale
is a martingale.

Put $Y(t)=X_{\tau_{t}}$ and appeal to the definition of $L_{\phi}$ to
conclude that
$$
f(Y(t))-\int\limits^{t}_{0}(L_{\phi}f)(Y(s))ds
$$
is a martingale. $Y(t,w)=X_{\tau_{t}}(w)=(T_{\phi}w)(t)$. Let
$\overline{\mathscr{F}}_{t}=\sigma\{Y(s):0\leq s\leq t\}$. Then
  $\overline{\mathscr{F}}_{t}=T^{-1}_{\phi}(\mathscr{F}_{t})\subset
    \mathscr{F}_{\tau_{t}}$. Thus
$$
f(Y(t))-\int\limits^{t}_{0}(L_{\phi}f)(Y(s))ds
$$
is a martingale relative to
$(\Omega,\overline{\mathscr{F}}_{t},P)$. Define $Q=PT^{-1}_{\phi}$ so
that
$$
f(X(t))-\int\limits^{t}_{0}(L_{\phi}f)(X(s))ds
$$
is an $(\Omega,\mathscr{F}_{t},Q)$-martingale, i.e.\@ $Q$ is an It\^o
process that corresponds to the operator $\phi L$. Or,
$PT^{-1}_{\phi}$ is an It\^o process that corresponds to the operator
$\phi L$.

We have now proved the following theorem.

\begin{theorem*}
Let $\Omega=C([0,\infty);\mathbb{R}^{d});X(t,w)=w(t)$;
$$
L=\frac{1}{2}\sum_{i,j}a_{ij}\frac{\p^{2}}{\p x_{i}\p x_{j}}+\sum
b_{j}\frac{\p}{\p x_{j}}.
$$

Suppose that $X(t)$ is an It\^o process relative to
$(\Omega,\mathscr{F}_{t},P)$ that corresponds to the operator $L$. Let $0\leq
C_{1}\leq \phi\leq C_{2}$ where $\phi:\mathbb{R}^{d}\to
\mathbb{R}$\pageoriginale  is
measurable. If $Q=PT^{-1}_{\phi}$, then $X(t)$ is an It\^o process
relative to $(\Omega,\mathscr{F}_{t},Q)$ that corresponds to the
operator $\phi L$.
\end{theorem*}

As $0<C_{1}\leq \phi\leq C_{2}$, we get $0<1/C_{2}\leq 1/\phi<1/C_{1}$
with $T_{\phi-1}\circ T_{\phi}=I$. We have thus an obvious corollary.

\begin{coro*}
There exists a probability measure $P$ on $\Omega$ such that $X$ is an
It\^o process relative to $(\Omega,\mathscr{F}_{t},P)$ that
corresponds to the operator $L$ if and only if there exists a
probability measure $Q$ on $\Omega$ such that $X$ is an Ito process
relative to $(\Omega,\mathscr{F}_{t},Q)$ that corresponds to the
operator $\phi L$.
\end{coro*}

\begin{remark*}
If $C_{2}\geq \phi\geq C_{1}>0$ then we have shown that existence and
uniqueness of an It\^o process for the operator $L$ guarantees
existence and uniqueness of the It\^o process for the operator $\phi
L$. The solution is no longer unique if we relax the strict positivity
on $C_{1}$ as is illustrated by the following example.
\end{remark*}

Let $\phi\equiv a(x)=|x|^{\alpha}\wedge 1$ where $0<\alpha <1$ and let
$L=\dfrac{1}{2}a\dfrac{\p^{2}}{\p x}$. Define $\delta_{0}$ on
$\{C([0,\infty);\mathbb{R})\}$ by
$$
\delta_{0}(A)=
\begin{cases}
1, & \text{if}\q \theta\in A,\ \forall A\epsilon,\\
0, & \text{if}\q \theta\not\in A,
\end{cases}
$$
where $\theta$ is the zero function on $[0,\infty)$.

\begin{claim*}
$\delta_{0}$ is an It\^o process with parameters $0$ and $a$. For this
  it is enough to show that, $\forall f\in C^{\infty}_{0}(\mathbb{R})$
$$
f(X(t))-\int\limits^{t}_{0}(Lf)(X(s))ds
$$\pageoriginale
is a martingale, using $a(0)=0$, it follows easily that
$$
\int\limits_{A}\int\limits^{t}_{0}(Lf)(X(\sigma))d\sigma d\delta_{0}=0
$$
$\forall$ Borel set $A$ of $C([0,\infty);\mathbb{R})$ and
  $\int\limits_{A}f(X(t))d\delta_{0}=0$ if $\theta\not\in A$ and
  $$
\int\limits_{A}f(X(t))d\delta_{0}=f(0)
$$ 
if $\theta\in A$, and this
  is true $\forall t$, showing that $X(t,w)=w(t)$ is an It\^o process
  relative to $\delta_{0}$ corresponding to the operator $L$.
\end{claim*}

Next we shall define $T_{a}$ (as in the theorem); we note that $T_{a}$
cannot be defined everywhere (for example $T_{a}(\theta)$ is not
defined). However $T_{a}$ is defined a.e.\@ $P$ where $P=P_{0}$ is the
Brownian motion.
$$
E^{P}\left(\int\limits^{t}_{0}\frac{1}{|X(s)|^{\alpha}}ds\right)=\int\limits^{t}_{0}\int\limits^{\infty}_{0}\frac{1}{y^{\alpha}}\frac{1}{\surd
  (2\pi s)}e^{\frac{-y}{2s}}dy\ ds<\infty
$$
since $0<\alpha<1$. Thus by Fubini's theorem,
$$
\int\limits^{t}_{0}\frac{1}{|w(s)|^{\alpha}}ds<\infty\q\text{a.e.}
$$

Taking $t=1,2,3\ldots$, there exists a set $\Omega^{*}$ such that
$P(\Omega^{*})=1$ and
$$
\int\limits^{t}_{0}\frac{1}{|w(s)|^{\alpha}}ds<\infty,\q \forall t,\q
\forall w\in \Omega^{*}
$$

Observe that
$$
\int\limits^{t}_{0}\frac{1}{|w(s)|^{\alpha}}ds<\infty
$$
implies that
$$
\int\limits^{t}_{0}\frac{1}{|w(s)|^{\alpha}\wedge 1}ds<\infty,
$$\pageoriginale
for
\begin{gather*}
\int\limits^{t}_{0}\frac{ds}{|w(s)|^{\alpha}\wedge 1}=\\
=\int\limits_{[0,t]\{|w(s)|^{\alpha}>1\}}\frac{ds}{|w(s)|^{\alpha}\wedge
  1}+ \int\limits_{\{|w(s)|^{\alpha}\leq
  1\}[0,t]}\frac{ds}{|w(s)|^{\alpha}},\ldots\\
\leq
m\{(|w(s)|^{\alpha}>1)[0,t]\}+\int\limits^{t}_{0}\frac{1}{|w(s)|^{\alpha}}ds<\infty
\\
\text{($m$ = Lebesgue measure)}
\end{gather*}

Thus $T_{a}$ is defined on the whole of $\Omega^{*}$. Using the same
argument as in the theorem, it can now be proved that $X$ is an It\^o
process relative to $Q$ corresponding to the operator $L$. Finally, we
show that $Q\{\theta\}=0$. $Q\{\theta\}=PT^{-1}_{a}\{\theta\}$. Now:
$T^{-1}_{a}\{\theta\}=$ empty. For, let $w\in
T^{-1}_{a}\{\theta\}$. Then $w(\tau_{t})=0$, $\forall t$, $w\in
\Omega^{*}$. Since $|\tau_{t}-\tau_{s}|\leq |t-s|$, one finds that
$\tau_{t}$ is a continuous function of $t$. Further $\tau_{1}>0$, and
$w=0$ on $[0,\tau_{1}]$ gives
$$
\int\limits^{\tau_{1}}_{0}\frac{1}{|w(s)|^{\alpha}\wedge 1}ds=\infty.
$$

This is false unless $T^{-1}_{a}\{\theta\}=$ empty. Thus
$Q\{\theta\}=0$ and $Q$ is different from $\delta_{0}$.
