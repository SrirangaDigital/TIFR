\chapter{Cameron - Martin - Girsanov Formula}\label{chap29}

LET\pageoriginale US REVIEW what we did in Brownian motion with drift. 

Let $(\Omega,\mathscr{F}_{t},P)$ be a $d$-dimensional Brownian motion
with
$$ 
P\{w:w(0)=x\}=1.
$$

Let $b:\mathbb{R}^{d}\to \mathbb{R}^{d}$ be a bounded measurable
function and define
$$
Z(t)=\exp\left(\int\limits^{t}_{0}\langle b,dx\rangle
-\frac{1}{2}\int\limits^{t}_{0}|b|^{2}ds\right).
$$

Then we see that $Z(t,\cdot)$ is an
$(\Omega,\mathscr{F}_{t},P)$-martingale. We then had a probability
measure $Q$ given by the formula
$$
\frac{dQ}{dP}\Big|_{\mathscr{F}_{t}}=Z(t,\cdot).
$$

We leave it as an exercise to check that in effect $X$ is an It\^o
process relative to $Q$ with parameters $b$ and $I$. In other words we
had made a transition from the operator $\Delta/2$ to $\Delta/2+b\cdot
\nabla$. We now see whether such a relation also exists for the more
general operator $L$.

\begin{theorem*}
Let $a:\mathbb{R}^{d}\to S^{+}_{d}$ be bounded and measurable such
that $a\geq CI$ for some $C>0$. Let $b:\mathbb{R}^{d}\to
\mathbb{R}^{d}$ be bounded, $\Omega=([0,\infty);\mathbb{R}^{d})$,
  $X(t,w)=w(t)$, $P$ any probability measure on $\Omega$ such that $X$
  is an It\^o process relative to $(\Omega,\mathscr{F}_{t},P)$ with
  parameters $[0,a]$. Define $Q^{t}$ on $\mathscr{F}_{t}$ by the rule
$$
\frac{dQ^{t}}{dP}\Big|_{\mathscr{F}_{t}}=Z(t,\cdot)=\exp\left[\int\limits^{t}_{0}\langle
  a^{-1}b,dX\rangle -\frac{1}{2}\int\limits^{t}_{0}\langle
  b,a^{-1}b\rangle ds\right].
$$\pageoriginale

Then
\begin{itemize}
\item[\rm(i)] $\{0^{t}\}t\geq 0$ is a consistent family.

\item[\rm(ii)] there exists a measure $Q$ on
  $\sigma(||\mathscr{F}_{t})$:
$$
Q\Big|_{\mathscr{F}_{t}}=Q^{t}.
$$

\item[\rm(iii)] $X(t)$ is an It\^o process relative to
  $(\Omega,\mathscr{F}_{t},Q)$ with parameters $[b,a]$, i.e.\@ it
  corresponds to the operator
$$
\frac{1}{2}\sum\limits_{i,j}a_{ij}\frac{\p^{2}}{\p x_{i}\p
  x_{j}}+\sum_{j}b_{j}\frac{\p}{\p x_{j}}.
$$
\end{itemize}
\end{theorem*}

\begin{proof}
\begin{enumerate}
\renewcommand{\theenumi}{\roman{enumi}}
\renewcommand{\labelenumi}{(\theenumi)}
\item Let $A(t)=\int\limits^{t}_{0}\langle a^{-1}b, dX\rangle$. Then
  $A\in I[0,\langle b, a^{-1}b\rangle]$.

Therefore $Z(t)$ is a martingale relative to
$(\Omega,\mathscr{F}_{t},P)$ hence $\{Q^{t}\}_{t\geq 0}$ is a
consistent family.

\item Proof as in the case of Brownian motion.

\item We have to show that
$$
\exp[\langle \theta,X(t,\cdot)\rangle -\langle
  \theta,\int\limits^{t}_{0}bds\rangle
  -\frac{1}{2}\int\limits^{t}_{0}\langle\theta,a\theta\rangle ds]
$$
is a martingale relative to $(\Omega,\mathscr{F}_{t},Q)$.
\end{enumerate}

Now for any function $\theta$ which is progressively measurable and
boun\-ded
$$
\exp[\int\limits^{t}_{0}\langle \theta, dX\rangle
  -\frac{1}{2}\int\limits^{t}_{0}\langle \theta, a\theta\rangle ds]
$$
is an $(\Omega,\mathscr{F}_{t},P)$-martingale. Replace $\theta$ by
$\theta(w)=\theta+(a^{-1}b)(\chi(s,w))$,\pageoriginale where $\theta$
now is a constant vector. Then
$$
\exp [\int\limits^{t}_{0}\langle \theta+a^{-1}b,dX\rangle
  -\frac{1}{2}\int\limits^{t}_{0}\langle \theta+a^{-1}b,a\theta\rangle ds
$$
is an $(\Omega, \mathscr{F}_{t},P)$-martingale, i.e.
$$
\exp[\langle \theta, X(t)\rangle
  -\frac{1}{2}\int\limits^{t}_{0}\langle \theta, a\theta\rangle
  ds-\frac{1}{2}\int\limits^{t}_{0}\langle
  a^{-1}b,a\theta\rangle-\frac{1}{2}\int\limits^{t}_{0}\langle
  \theta,b\rangle ]
$$
is an $(\Omega,\mathscr{F}_{t},Q)$-martingale, and
\begin{align*}
\langle a^{-1}b,a\theta\rangle &= \langle a^{*}a^{-1}b,\theta\rangle\\
&= \langle aa^{-1}b,\theta\rangle \q (\text{since~ } a=a^{*})\\
&= \langle b,\theta\rangle.
\end{align*}

Thus
$$
\exp[\langle \theta, X(t)\rangle
  -\frac{1}{2}\int\limits^{t}_{0}\langle \theta,a\theta\rangle
  ds-\int\limits^{t}_{0}\langle \theta, b\rangle ds]
$$
is an $(\Omega,\mathscr{F}_{t},Q)$-martingale, i.e. $X$ is an It\^o
process relative to $(\Omega,\mathscr{F}_{t},Q)$ with parameters
$[b,a]$. This proves the theorem.
\end{proof}

We now prove the converse part.

\begin{theorem*}
Let
$$
L_{1}=\frac{1}{2}\sum\limits{i,j}a_{ij}\frac{\p^{2}}{\p x_{i}\p x_{j}}
$$
and 
$$
L_{2}\equiv \frac{1}{2}\sum\limits_{i,j}a_{ij}\frac{\p^{2}}{\p x_{i}\p
  x_{j}}+\sum_{j}b_{j}\frac{\p}{\p x_{j}},
$$
where $a:\mathbb{R}^{d}\to S^{+}_{d}$ is bounded measurable such that
$a\geq CI$ for some $C>0$; $b:\mathbb{R}^{d}\to \mathbb{R}^{d}$ is
bounded and measurable. Let $\Omega=C([0,\infty);\mathbb{R}^{d})$ with
  $\mathscr{F}_{t}$ as usual. Let $\theta$ be a probability measure on
  $\sigma(\cup \mathscr{F}_{t})$ and\pageoriginale $X$ a progressively
  measurable function such that $X$ is an It\^o process relative to
  $(\Omega,\mathscr{F}_{t},Q)$ with parameters $[b,a]$ i.e.\@ $X$
  corresponds to the operator $L_{2}$. Let 
$$
Z(t)=\exp [-\int\limits^{t}_{0}\langle a^{-1}b,dX\rangle
  +\frac{1}{2}\int\limits^{t}_{0}\langle b,a^{-1}b\rangle ds].
$$

Then
\begin{enumerate}
\renewcommand{\theenumi}{\roman{enumi}}
\renewcommand{\labelenumi}{\rm(\theenumi)}
\item $Z(t)$ is an $(\Omega,\mathscr{F}_{t},Q)$-martingale.

\item If $P^{t}$ is defined on $\mathscr{F}_{t}$ by
$$
\frac{dP^{t}}{dQ}\Big|_{\mathscr{F}_{t}}=Z(t),
$$

Then there exists a probability measure $P$ on $\sigma(\cup
\mathscr{F}_{t})$ such that 
$$
P\Big|_{\mathscr{F}_{t}}=P^{t}
$$

\item $X$ is an It\^o process relative to $(\Omega,
  \mathscr{F}_{t},P)$ corresponding to parameters $[0,a]$, i.e.\@ $X$
  corresponds to the operator $L_{1}$.
\end{enumerate}
\end{theorem*}

\begin{proof}
\begin{enumerate}
\renewcommand{\theenumi}{\roman{enumi}}
\renewcommand{\labelenumi}{(\theenumi)}
\item Let
$$
A(t)=\int\limits^{t}_{0}\langle -a^{-1}b,dX\rangle.
$$

Then $A(t)$ is an It\^o process with parameters $[\langle
  -a^{-1}b,b\rangle, \langle a^{-1}b,b\rangle]$.

Thus 
$$
\exp [A(t)-\int\limits^{t}_{0}\langle -a^{-1}b, b\rangle
  ds-\frac{1}{2}\int\limits^{t}_{0}\langle a^{-1}b,b\rangle ds]
$$
is an $(\Omega,\mathscr{F}_{t},Q)$-martingale, i.e.\@ $Z(t)$ is an
$(\Omega,\mathscr{F}_{t},Q)$ martingale.

\item By\pageoriginale (i), $P^{t}$ is a consistent family. The proof
  that there exists a probability measure $P$ is same as before.
\end{enumerate}

Since $X$ is an It\^o process relative to $Q$ with parameters $b$ and
$a$,
$$
\exp[\int\limits^{t}_{0}\langle \theta,dX\rangle
  -\int\limits^{t}_{0}\langle \theta, b\rangle
  ds-\frac{1}{2}\int\limits^{t}_{0}\langle \theta, a\theta\rangle ds]
$$
is a martingale relative to $Q$ for every bounded measurable
$\theta$. Replace $\theta$ by $\theta(w)=\theta-(a^{-1}b)(X(s,w))$
where $\theta$ now is a constant vector to get
\begin{gather*}
\exp [\langle \theta, X(t)\rangle -\int\limits^{t}_{0}\langle
  a^{-1}b,dX\rangle -\int\limits^{t}_{0}\langle \theta, b\rangle
  +\int\limits^{t}_{0}\langle a^{-1}b,b\rangle ds-\\
-\frac{1}{2}\int\limits^{t}_{0}\langle \theta-a^{-1}b,a\theta-b\rangle ds]
\end{gather*}
is an $(\Omega,\mathscr{F}_{t},Q)$ martingale, i.e.
\begin{gather*}
\exp[\langle \theta, X\rangle -\int\limits^{t}_{0}\langle
  a^{-1}b,dX\rangle -\int\limits^{t}_{0}\langle \theta,b\rangle
  ds+\int\limits^{t}_{0}\langle a^{-1}b,b\rangle ds-\\
-\frac{1}{2}\int\limits^{t}_{0}\langle \theta, a\theta\rangle
ds-\frac{1}{2}\int\limits^{t}_{0}\langle a^{-1}b,b\rangle
ds+\frac{1}{2}\int\limits^{t}_{0}\langle \theta,b\rangle ds+\\
+\int\limits^{t}_{0}\langle a^{-1}b,a\theta\rangle ds]
\end{gather*}
is an $(\Omega,\mathscr{F}_{t},Q)$ martingale. Let $\theta\in
\mathbb{R}^{d}$, so that
$$
\exp[\langle \theta,X\rangle -\frac{1}{2}\int\limits^{t}_{0}\langle
  \theta,b\rangle ds-\frac{1}{2}\int\limits^{t}_{0}\langle \theta,
  a\theta\rangle ds +\frac{1}{2}\int\limits^{t}_{0}\langle a^{-1}b,
  a\theta\rangle ds]Z(t)
$$
is an $(\Omega, \mathscr{F}_{t},Q)$-matringale and
$$
\langle a^{-1}b,a\rangle =\langle b,\theta\rangle \q(\text{since~ }
a=a^{*}).
$$

Therefore\pageoriginale
$$
\exp [\langle \theta, X\rangle -\frac{1}{2}\int\limits^{t}_{0}\langle
  \theta, a\theta \rangle ds]Z(t)
$$
is an $(\Omega,\mathscr{F}_{t},Q)$ martingale.

Using the fact that $\dfrac{dP}{dQ}\Big|_{\mathscr{F}_{t}}=Z(t)$, we
conclude that 
$$
\exp [\langle \theta, X\rangle -\frac{1}{2}\int\limits^{t}_{0}\langle
  \theta, a\theta \rangle ds]
$$
is a martingale relative to $(\Omega,\mathscr{F}_{t},P)$, i.e.\@ $X\in
I[0,a]$ relative to 
$$
(\Omega,\mathscr{F}_{t},P).
$$ 
This proves the theorem.
\end{proof}

\noindent
{\bf Summary.}~ We have the following situation
$$
L_{1},\Omega,\mathscr{F}_{t},\ \Omega=C([0,\infty);\mathbb{R}^{d}),\ L_{2},\ \Omega,\ \mathscr{F}_{t}.
$$
\begin{center}
\begin{tabular}{@{}lcl@{}}
$\left.\begin{array}{p{4.2cm}}
$P$ a probability measure such that $X$ is an It\^o Process relative
  to $P$ corresponding to the operator $L_{1}$. 
  \end{array}\right)$
& $\Longrightarrow$ &
$\left(\begin{array}{p{4.2cm}}
  $X$ is an It\^o process relative to a probability measure $Q$
  corresponding to $L_{2}$. $Q$ is given by
  $\dfrac{dQ}{dP}\Big|_{\mathscr{F}_{t}}=Z(t,\cdot)$
  \end{array}\right.$\\
& & \\
$\left.\begin{array}{p{4.2cm}}
$X$ is an It\^o process relative to $P$ corresponding to $L_{1}$ where
  $\dfrac{dP}{dQ}\Big|_{\mathscr{F}_{t}}=\dfrac{1}{Z(t,\cdot)}$ 
\end{array}\right)$
&
$\Longleftarrow$ 
& 
$\left(\begin{array}{p{4.1cm}}
$X$ is an It\^o process relative to $Q$
  corresponding to $L_{2}$.\\
\phantom{a}
\end{array}\right.$ 
\end{tabular}
\end{center}

Thus existence and uniqueness for any system guarantees the existence
and uniqueness for the other system.
\medskip

\noindent
{\bf Application. (Exercise).}\pageoriginale
\smallskip

Take $d=1$, $a:\mathbb{R}\to \mathbb{R}$ bounded and measurable with
$0<C_{1}\leq a<C_{2}<\infty$. Let $L=\dfrac{a}{2}\dfrac{\p^{2}}{\p
  x^{2}}+b\dfrac{\p}{\p x}$. Show that there exists a unique
probability masure $P$ on $\Omega=C([0,\infty);\mathbb{R})$ such that
  $X(t)$ is It\^o relative to $P$ corresponding to $L$. $(X(t,w)\equiv
  w(t))$ for any given starting point.


