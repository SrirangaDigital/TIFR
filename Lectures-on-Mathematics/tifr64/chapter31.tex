\chapter{Invariant Probability Distributions}\label{chap31}

\begin{defi*}
Let\pageoriginale $\{P_{x}\}_{x\in \mathbb{R}^{d}}$ be a family of
Markov process on 
$$
\Omega=C([0,\infty);\mathbb{R}^{d})
$$ 
indexed by the
  starting points $x$, with homogeneous transition probability
  $p(t,x,A)=P_{x}(X_{t}\in A)$ for every Borel set $A$ in
  $\mathbb{R}^{d}$. A probability measure $\mu$ on the Borel field of
  $\mathbb{R}^{d}$ is called an {\em invariant distribution} if,
  $\forall A$ Borel in $\mathbb{R}^{d}$.
$$
\int\limits_{\mathbb{R}^{d}}p(t,x,A)d\mu(x)=\mu(A).
$$

We shall denote $dp(t,x,y)$ by $p(t,x,dy)$ or $p(t,x,y)dy$ if it has a
density. 
\end{defi*}

\begin{prop*}
Let $L_{2}=\Delta/2+b\cdot\nabla$ with no explosion. Let $Q_{x}$ be
the associated measure. If $\{Q_{x}\}$ has an invariant measure $\mu$
then the process is recurrent.
\end{prop*}

\begin{proof}
It is enough to show that if $K$ is a compact set with non-empty
interior then
$$
Q_{x}(E^{K}_{\infty})=1
$$
for some $x$. Also $Q_{x}(E^{K}_{t})\geq Q_{x}(X_{t}\in
K)=q(t,x,K)$. Therefore
$$
\mu(K)=\int q(t,x,K)d\mu(x)\leq \int Q_{x}(E^{K}_{t})d\mu(x).
$$

Now, $0\leq Q_{x}(E^{K}_{t})\leq 1$ and $Q_{x}(E^{K}_{t})$ decreases
to $Q_{x}(E^{K}_{\infty})$ as $t\to \infty$. Therefore by the
dominated convergence theorem
$$
\mu(K)\leq \int Q_{x}(E^{K}_{\infty})d\mu(x).
$$\pageoriginale

If the process were transient, then $Q_{x}(E^{S_{n}}_{\infty})=0$,
$\forall n$, where $S_{n}=\{x\in \mathbb{R}^{d}:|x|\leq n\}$, i.e.\@
$\mu(S_{n})=0$, $\forall n$. Therefore $\mu(\mathbb{R}^{d})=0$, which
is false. Thus the process is recurrent.

The converse of this proposition is {\em not} true as is seen by the
following example.

Let $L=\dfrac{1}{2}\dfrac{\p^{2}}{\p x^{2}}$ so that we are in a
one-dimensional situation (Brownian motion). Then
$$
p(t,x,K)=\int\limits_{K}\frac{1}{\surd (2\pi
  t)}e^{\frac{-(x-y)^{2}}{2t}}dy\leq \frac{1}{\surd (2\pi t)}\lambda (K),
$$
where $\lambda$ denotes the Lebesgue measure on $\mathbb{R}$. If there
exists an invariant distribution $\mu$, then
$$
\mu(K)=\int p(t,x,K)d\mu(x)\leq \frac{1}{\surd(2\pi t)}\lambda(K)\int
d\mu(x)=\frac{\lambda(K)}{\surd (2\pi t)}
$$

Letting $t\to \infty$, we get $\mu(K)=0\ \forall$ compact $K$, giving
$\mu=0$, which is false.
\end{proof}

\begin{theorem*}
Let $L=\Delta/2+b\cdot \nabla$ with no explosion. Assume $b$ to be
$C^{\infty}$. Define the formal adjoint $L^{*}$ of $L$ by
$L^{*}=\Delta/2-\nabla\cdot b$ (i.e.\@ $L^{*}u=\dfrac{1}{2}\Delta
u-\nabla\cdot (bu)$). Suppose there exists a smooth function
$\phi(C^{2}$ - would do) such that $L^{*}\phi=0$, $\phi\geq 0$,
$\inf\phi dx=1$. If one defines $\mu$ by the rule
$\mu(A)=\int\limits_{A}\phi(y)dy$, then $\mu$ is an invariant
distribution relative to the family $\{Q_{x}\}$.
\end{theorem*}

\begin{proof}
We\pageoriginale assume the following result from the theory of
partial differential equations. 

If $f\in C^{\infty}_{0}(G)$ where $G$ is a bounded open set with a
smooth boundary $\p G$ and $f\geq 0$, then there exists a smooth
function $U_{G}:[0,\infty)\times \overline{G}\to [0,\infty)$ such that
\begin{align*}
& \frac{\p U_{G}}{\p t}=LU_{G}\q\text{on}\q (0,\infty)\times G,\\
& U_{G}(0,x)=f(x)\q\text{on}\q \{0\}\times \overline{G},\\
& U_{G}(t,x)=0,\q \forall x\in G.
\end{align*}

Let $t>0$. As $U_{G}$, $\phi$ are smooth and $G$ is bounded, we have
$$
\frac{\p}{\p
  t}\int\limits_{G}U_{G}(t,x)\phi(x)dx=\int\limits_{G}\frac{\p}{\p
  t}U_{G}\phi ds=\int\limits_{G}\phi LU_{G}dx
$$

Using Green's formula this can be written as
\begin{align*}
\frac{\p}{\p t}\int\limits_{G}U_{G}(t,x)\phi(x)dx &=
\int\limits_{G}U_{G}L^{*}\phi-\frac{1}{2}\int\limits_{\p
  G}\left[\phi\frac{\p U_{G}}{\p n}-U_{G}\frac{\p}{\p n}\right]dS+\\
&\q +\int\limits_{\p G}\langle b\cdot n\rangle U_{G}(t,x)\phi(x)dS
\end{align*}

Here $n$ is assumed to be the inward normal to $\p G$. So,
$$
\frac{\p}{\p
  t}\int\limits_{G}U_{G}(t,x)\phi(x)dx=-\frac{1}{2}\int\limits_{\p
  G}(x)\frac{\p U_{G}}{\p n}(t,x)dS
$$
(Use the equation satisfied by $\phi$ and the conditions on
$U_{G}$). Now $U_{G}(t,x)\break \geq 0$, $\forall x\in G$, $U_{G}(t,x)=0$,
$\forall x$ in $\p G$, so that
$$
\frac{\p U_{G}}{\p n}(t,x)\geq 0.
$$

This\pageoriginale means that
$$
\frac{\p}{\p t}\int\limits_{G}U_{G}(t,x)\phi(x)dx\leq 0,\ \forall t>0,
$$
i.e.\@ $\int\limits_{G}U_{G}(t,x)\phi(x)dx$ is a monotonically
decresing function of $t$. Therefore
\begin{align*}
\int\limits_{G}U_{G}(t,x)\phi(x)dx &\leq
\int\limits_{G}U_{G}(0,x)\phi(x)dx\\
&= \int\limits_{G}f(x)\phi(x)dx\\
&= \int\limits_{\mathbb{R}^{d}}f(x)\phi(x)dx.
\end{align*}

Next we prove that if $U:[0,\infty)\times \mathbb{R}^{d}\to
  [0,\infty)$ is such that $\frac{\p U}{\p t}=LU$, $\forall t>0$ and
    $U(0,x)=f(x)$, then
$$
\int\limits_{\mathbb{R}^{d}}U(t,x)\phi(x)dx\leq
\int\limits_{\mathbb{R}^{d}}f(x)\phi(x)dx. 
$$

The solution $U_{G}(t,x)$ can be obtained by using It\^o calculus and
is given by
$$
U_{G}(t,x)=\int f(X(t))\chi_{\{\tau_{G>t}\}}dQ_{x}.
$$

We already know that
$$
U(t,x)=\int f(X(t))dQ_{x}.
$$

Therefore
$$
\int U(t,x)\phi(x)dx=\iint f(X(t))\phi(x)DQ_{x}dx.
$$

Now 
\begin{align*}
& \iint f(X(t))\chi_{\{\tau_{G}>t\}}dQ_{x}\phi(x)dx\\
&\qq \int U_{G}(t,x)\phi(x)dx\leq \int\limits_{\mathbb{R}^{d}}f(x)\phi(x)dx.
\end{align*}\pageoriginale

Letting $G$ increase to $\mathbb{R}^{d}$ and using Fatou's lemma, we
get
$$
\iint f(X(t))\phi(x)dQ_{x}dx\leq \int f(x)\phi(x)dx
$$
\end{proof}

This proves the assertion made above.

Let
\begin{gather*}
\mu(A)=\int\limits_{A}\phi(X)dx,\\
\nu(A)=\int Q_{x}(X_{t}\in A)d\mu(x)=\int q(t,x,A)d\mu(x).
\end{gather*}

Let $f\in C^{\infty}_{0}(G)$, $f\geq 0$, where $G$ is a bounded open
set with smooth boundary. Now
\begin{align*}
\int f(y)d\nu(y) &= \iint f(y)q(t,x,y)d\mu(x)dy\\
&= \iint f(X(t))dQ_{x}d\mu(x)\\
&= \int U(t,x)d\mu(x)\\
&= \int U(t,x)\phi(x)dx\\
&\leq \int f(x)\phi(x)dx=\int f(x)d\mu (x).
\end{align*}

Thus, $\forall f\geq 0$ such that $f\in C^{\infty}_{0}$,
$$
\int f(x)d\nu(x)\leq \int f(x)d\mu(x).
$$

This\pageoriginale implies that $\nu(A)\leq \mu(A)$ for every Borel
set $A$. (Use mollifier $s$ and the dominated convergence theorem to
prove the above inequality for $\chi_{A}$ when $A$ is
bounded). Therefore $\nu(A^{c})\leq \mu(A^{c})$, or $1-\mu(A)\leq
1-\mu(A)$, since $\mu$, $\nu$ are both probability measures. This
gives $\mu(A)=\nu(A)$, i.e.
$$
\mu(A)=\int q(t,x,A)d\mu(x),\q \forall t,
$$
i.e.\@ $\mu$ is an invariant distribution.

We now see whether the converse result is true or not. Suppose there
exists a probability measure $\mu$ on $\mathbb{R}^{d}$ such that
$$
\int Q_{x}(X_{t}\in A)d\mu(x)=\mu(A),\q \forall A\text{~ Borel in~ }
\mathbb{R}^{d}\text{~ and~ } \forall t.
$$

The question we have in mind is whether $\mu(A)=\int\limits_{A}\phi
dx$ for some smooth $\phi$ satisfying $L^{*}\phi=0$, $\phi\geq 0$,
$\int \phi(x)dx=1$. To answer this we proceed as follows.

By definition $\mu(A)=\int q(t,x,A)d\mu(x)$. Therefore
\begin{gather*}
\iint f(X(t))dQ_{x}d\mu(x)\\
=\iint f(y)q(t,x,y)dy\ d\mu(x)\\
=\int f(y)d\mu(y)\forall f\in
C^{\infty}_{0}(\mathbb{R}^{d})||f||_{\infty}\leq 1.\tag{1}
\end{gather*}

Since $X$ is an It\^o process relative to $Q_{x}$ with parameters $b$
and $I$,
$$
f(X(t))-\int\limits^{t}_{0}(Lf)(X(s))ds
$$
is\pageoriginale a martingale. Equating the expectations at time $t=0$
and time $t$ we obtain
$$
E^{Q_{x}}(f(X(t))=f(x)+E^{Q_{x}}\left(\int\limits^{t}_{0}(Lf)(X(s))ds\right)
$$

Integrating this expression with respect to $\mu$ gives
$$
\iint f(X(t))dQ_{x}d\mu(x)=\int
f(x)d\mu(x)\iint\limits_{\mathbb{R}^{d}}\int\limits^{t}_{0}(Lf)(X(s))ds\ dQ_{x}d\mu. 
$$

Using (1), we get
$$
0=\int\limits_{\mathbb{R}^{d}}\int\limits_{\Omega}\int\limits^{t}_{0}(Lf)(X(s))dQ_{x}ds\ d\mu(x)
$$

Applying equation (1) to the function $Lf$ we then get
\begin{align*}
0 &= \int\limits_{\mathbb{R}^{d}}\int\limits^{t}_{0}(Lf)(y)d\mu(y)ds\\
&= t \int\limits_{\mathbb{R}^{d}}(Lf)(y)d\mu(y),\q \forall t>0.
\end{align*}

Thus
$$
0=\int\limits_{\mathbb{R}^{d}}(Lf)(y)d\mu(y),\q \forall f\in
C^{\infty}_{0}(\mathbb{R}^{d}).
$$

In the language of distributions this just means that $L^{*}\mu=0$.

From the theory of partial differential equations it then follows that
there exists a smooth function $\phi$ such that $\forall A$ Borel in
$\mathbb{R}^{d}$,
$$
\mu(A)=\int\limits_{A}\phi(y)dy
$$
with $L^{*}\phi=0$. As $\mu\geq 0$, $\phi\geq 0$ and since
$$
\mu(\mathbb{R}^{d})=1,\q \int\limits_{\mathbb{R}^{d}}\phi(x)dx=1.
$$\pageoriginale

We have thus proved the following (converse of the previous) theorem.

\begin{theorem*}
Let $\mu$ be an invariant distribution with respect to the family
$\{Q_{x}\}$ with $b:\mathbb{R}^{d}\to \mathbb{R}^{d}$ being
$C^{\infty}$. Then there exists a $\phi \in L'(\mathbb{R}^{d})$,
$\phi\geq 0$, $\phi$ smooth such that
$$
L^{*}\phi=0,\q \int \phi(y)dy=1
$$
and such that
$$
\mu(A)=\int\limits_{A}\phi(y)dy,\q \forall A\q\text{Borel in}\q
\mathbb{R}^{d}. 
$$
\end{theorem*}

\begin{theorem*}[(Uniqueness)]
Let $\phi_{1}$, $\phi_{2}$ be smooth on $\mathbb{R}^{d}$ such that 
$$
\phi_{1}, \phi_{2}\geq 0,
1=\int\limits_{\mathbb{R}^{d}}\phi_{1}dy=\int\limits_{\mathbb{R}^{d}}\phi_{2}dy,
L^{*}\phi_{1}=0=L^{*}\phi_{2}. 
$$

Then $\phi_{1}=\phi_{2}$.
\end{theorem*}

\begin{proof}
Let $f(x)=\phi_{1}(x)-\phi_{2}(x)$,
$$
\mu_{i}(A)=\int\limits_{A}\phi_{i}(x)dx,\q i=1,2.
$$

Then $\mu_{1}$, and $\mu_{2}$ are invariant distributions. Therefore
\begin{align*}
\int q(t,x,y)\phi_{i}(x)dx &= \int q(t,x,y)d\mu_{i}(x)\\
&= \phi_{i}(y),\q\text{(a.e.),}\q i=1,2.
\end{align*}

Taking the difference we obtain
$$
\int q(t,x,y)f(x)dx=f(y),\q\text{a.e.}
$$

Now 
\begin{align*}
\int |f(y)\ dy &= \int|\int q(t,x,y)f(x)dx|dy\\
&\leq \iint q(t,x,y)|f(x)|dx\ dy\\
&= \int |f(x)|dx\int q(t,x,y)dy\\
&= \int |f(\underline{x})|dx.
\end{align*}\pageoriginale

Thus
\begin{equation*}
\iint |f(x)|q(t,x,y)dx\ dy=\int|\int q(t,x,y)f(x)dx|dy\ \forall t.\tag{*}
\end{equation*}

We show that $f$ does not change sign, i.e.\@ $f\geq 0$ a.e.\@ or
$f\leq 0$ a.e. The result then follows from the fact that $\int
f(x)dx=0$. Now
$$
|\int q(1,x,y)f(x)dx|\leq \int q(1,x,y)|f(x)|dx
$$
and $(*)$ above gives
$$
\int|\int q(1,x,y)f(x)dx|dy=\iint q(1,x,y)|f(x)|dx\ dy.
$$

Thus
$$
|\int q(1,x,y)f(x)dx|=\int q(1,x,y)|f(x)|dx\text{~ a.e.~ } y,
$$
i.e.
\begin{align*}
& |\int\limits_{E^{-}}q(1,x,y)f(x)dx+\int\limits_{E^{-}}q(1,x,y)f(x)dx|\\
&\qq
  =\int\limits_{E^{+}}q(1,x,y)f(x)dx-\int\limits_{E^{-}}q(1,x,y)f(x)dx\text{~
    a.e.~ } y,
\end{align*}
where
$$
E^{+}=\{x:f(x)>0\},\q E^{-}=\{x:f(x)<0\},\q E^{0}=\{x:f(x)=0\}.
$$

Squaring both sides of the above equality, we obtain
\begin{equation*}
\left(\int\limits_{E^{+}}q(1,x,y)f(x)dx\right)\left(\int\limits_{E^{-}}q(1,x,y)f(x)dx\right)=0,\q\text{a.e.}\q y.\tag{**}
\end{equation*}\pageoriginale

Let $A$ be a set of positive Lebesgue measure; then
$p(1,x,A)=P_{x}(X(1)\in A)>0$. Since $Q_{x}$ is equivalent to $P_{x}$
on $\Omega$ we have $Q_{x}(X(1)\in A)=q(1,x,A)>0$. Therefore
$q(1,x,y)>0$ a.e.\@ $y$ for each $x$. By Fubini's theorem $q(1,x,y)>0$
a.e.\@ $x$, $y$. Therefore for almost all $y$, $q(1,x,y)>0$ for almost
all $x$. Now pick a $y$ such that $(**)$ holds for which $q(1,x,y)>0$
a.e.\@ $x$.

We therefore conclude from $(**)$ that either
$$
\int\limits_{E^{+}}q(1,x,y)f(x)dx=0,\q \text{in which case}\q f\leq
0\q \text{a.e.,}
$$
or
$$
\int\limits_{E^{-}}q(1,x,y)f(x)dx=0,\q \text{in which case}\q f\geq
0\q\text{a.e.} 
$$

Thus $f$ does not change its sign, which completes the proof.
\end{proof}

\begin{remark*}
The only property of the operator $L$ we used was to conclude
$q>0$. We may therefore expect a similar result for more general operators.
\end{remark*}

\begin{theorem*}
Let $L^{*}\phi=0$ where $\phi\geq 0$ is smooth and
$\int\phi(x)dx=1$. Let $K$ be any compact set. Then
$$
\sup\limits_{x\in K}\int|q(t,x,y)-\phi(y)|dy\to 0\q\text{as}\q t\to +\infty.
$$
\end{theorem*}

\setcounter{lemma}{0}
\begin{lemma}\label{chap31-lem1}
Let $b$ be bounded and smooth. For every $f:\mathbb{R}^{d}\to
\mathbb{R}^{d}$ that is bounded and measurable let
$u(t,x)=E^{Q_{x}}(f(X(t))$. Then for every fixed $t$, $u(t,x)$ is a
continuous function of $x$. Further, for $t\geq \epsilon >0$,
\begin{gather*}
|u(t,x)-\int u(t-\epsilon,y)\frac{1}{\surd (2\pi \epsilon)^{d}}\exp
-\frac{|x-y|^{2}}{2\epsilon}dy|\\
\leq ||f||_{\infty}\surd (e^{ct}(e^{c\epsilon}-1)),
\end{gather*}
where\pageoriginale $c$ is a constant depending only on $||b||_{\infty}$.
\end{lemma}

\begin{proof}
Let
\begin{equation*}
\begin{split}
(T_{t}f)(x) &= E^{Q_{x}}(f(X(t))=E^{P_{x}}(f(X(t))Z(\epsilon,t))+\\
&\q +E^{P_{x}}(f(X(t))(Z(t)-Z(\epsilon,t))),
\end{split}\tag{1}\label{chap31-eq1}
\end{equation*}
where
\begin{align*}
& Z(t)=\exp \left[\int\limits^{t}_{0}\langle b^{2},dx\rangle
  -\frac{1}{2}\int\limits^{t}_{0}|b|^{2}ds\right],\\
& Z(\epsilon,t)=\exp \left[\int\limits^{t}_{\epsilon}\langle
    b,dx\rangle -\frac{1}{2}\int\limits^{t}_{\epsilon}|b(X(s))|^{2}ds\right].
\end{align*}
\begin{align*}
E^{P_{x}}(f(X(t))Z(\epsilon,t)) &=
E^{P_{x}}(E^{P_{X}}(f(X(t))Z(\epsilon,t)|_{\epsilon}))\\ 
&= E^{P_{x}}(E^{P_{X}}\epsilon)(f(X(t-\epsilon))Z(t-\epsilon)))\\
&\qq\q \text{(by Markov property),}\\
&= E^{P_{x}}(u(t-\epsilon , X(\epsilon)).
\end{align*}
\begin{equation*}
=\int u(t-\epsilon, y)\frac{1}{(\surd (2\pi \epsilon))^{d}}\exp
\left[-\frac{|(x-y)|^{2}}{2\epsilon}\right]dy.\tag{2}\label{chap31-eq2} 
\end{equation*}

Now
\begin{align*}
& (E^{P_{x}}(|Z(t)-Z(\epsilon,t)|))=\\
&\qq =E^{P_{x}}(|Z(\epsilon)Z(\epsilon,t)-Z(\epsilon,t)|))^{2}\\
&\qq =E^{P_{x}}(Z(\epsilon,t)Z(\epsilon)-1|))^{2}\\
&\qq \leq
  (E^{P_{x}}((Z(\epsilon)-1)^{2}))(E^{P_{x}}(Z^{2}(\epsilon,t)))\\
&\qq\qq \text{(by Cauchy Schwarz inequality),}\\
&\qq \leq
  E^{P_{x}}(Z^{2}(\epsilon)-2Z(\epsilon)+1)E^{P_{x}}(Z^{2}(\epsilon,t))\\
&\qq \leq
  E^{P_{x}}(Z^{2}(\epsilon)-1)E^{P_{x}}(Z^{2}(\epsilon,t)),(\text{since~
  } E^{P_{x}}(Z(\epsilon))=1),\\
&\qq \leq E^{P_{x}}(Z^{2}(\epsilon)-1)E^{P_{x}}(\exp
  (2\int\limits^{t}_{\epsilon}\langle b,dX\rangle
  -\frac{2}{2}\int\limits^{t}_{\epsilon}|b|^{2}ds+\int\limits^{t}_{\epsilon}|b|^{2}ds))\\
&\qq \leq E^{P_{x}}(Z^{2}(\epsilon)-1)e^{ct},
\end{align*}\pageoriginale
using Cauchy Schwarz inequality and the fact that
$$
E^{P_{x}}(\exp (2\int\limits^{t}_{\epsilon}\langle b,dX\rangle
-\frac{2^{2}}{2}\int\limits^{t}_{\epsilon}|b|^{2}ds))=1. 
$$

Thus
$$
E^{P_{x}}(|Z(t)-Z(\epsilon,t)||^{2}\leq (e^{c\epsilon}-1)e^{ct}
$$
where $c$ depends only on $||b||_{\infty}$. Hence
\begin{align*}
& |E^{P_{x}}(f(X(t))(Z(t)-Z(\epsilon,t))|\leq
||f||_{\infty}E^{P_{x}}(|Z(t)-Z(\epsilon,t)|)\\ 
&\qq \leq ||f||_{\infty}\surd
((e^{c\epsilon}-1)e^{ct}).\tag{3}\label{chap31-eq3} 
\end{align*}

Substituting \eqref{chap31-eq2} and \eqref{chap31-eq3} in
\eqref{chap31-eq1} we get
\begin{gather*}
|u(t,x)-\int u(t-\epsilon y)\cdot \frac{1}{(\surd (2\pi
  \epsilon)^{d})}\exp \left[\frac{-|x-y|^{2}}{2\epsilon}\right]dy\\
\leq ||f||_{\infty}\surd ((e^{c\epsilon}-1)e^{ct})
\end{gather*}

Note that the right hand side is independent of $x$ and as
$\epsilon\to 0$ the right hand side converges to $0$. Thus to show
that $u(t,x)$ is a continuous function of $x$ ($t$ fixed), it is
enough to show that
$$
\int u(t-\epsilon,y)\frac{1}{(\surd (2\pi\epsilon)^{d}}\exp
\left[\frac{-|x-y|^{2}}{2\epsilon}\right]dy 
$$
is\pageoriginale a continuous function of $x$; but this is clear since
$u$ is bounded. Thus for any fixed $tu(t,x)$ is continuous.
\end{proof}

\begin{lemma}\label{chap31-lem2}
For any compact set $K\subset R^{d}$, for $r$ large enough so that 
$$
K\subset \{x:|x|<r\},\q x\to Q_{x}(\tau_{r}\leq t)
$$
is continuous on $K$ for each $t\geq 0$, where
$$
\tau_{r}(w)=\inf \{s:|w(s)|\geq r\}.
$$
\end{lemma}

\begin{proof}
$Q_{x}(\tau_{r}\leq t)$ depends only on the coefficient $b(x)$ on
  $|x|\leq r$. So modifying, if necessary, outside $|x|\leq r$, we can
  very well assume that $|b(x)|\leq M$ for all $x$. Let
\begin{align*}
\tau^{\epsilon}_{r} &= \inf \{s:s\geq \epsilon,|w(s)|\geq r\}.\\
&\q Q_{x}(\tau^{\epsilon}_{r}\leq t)=E^{Q_{x}}(u(X(\epsilon))),
\end{align*}
where 
$$
u(x)=Q_{x}(\tau_{r}\leq t-\epsilon)
$$

As $b$ and $u$ are bounded, for every fixed $\epsilon>0$, by Lemma
\ref{chap31-lem1}, $Q_{x}(\tau_{r}\leq t)$ is a continuous function of
$x$. As 
$$
|Q_{x}(\tau^{\epsilon}_{r}\leq t)-Q_{x}(\tau_{r}\leq t)|\leq
Q_{x}(\tau_{r}\leq \epsilon),
$$
to prove the lemma we have only to show that
$$
\displaystyle{\mathop{\text{limit}}_{\epsilon\to 0}}\sup\limits_{x\in
    K}Q_{x}(\tau_{r}\leq \epsilon)=0
$$

Now
\begin{align*}
& Q_{x}(\tau_{r}\leq \epsilon)=\int\limits_{\{\tau_{r}\leq
    \epsilon\}}Z()dP_{x}\\
& \leq (\int (Z(\epsilon))^{2}dP_{x})^{1/2}\cdot \surd
  P_{x}(\tau_{r}\leq \epsilon),
\end{align*}
by\pageoriginale Cauchy-Schwarz inequality. The first factor is
bounded because $b$ is bounded. The second factor tends to zero
uniformly on $K$ because
$$
\sup\limits_{x\in K}P_{x}(\tau_{r}\leq \epsilon)\leq
P(\sup\limits_{0\leq s\leq \epsilon}|w(s)|>\delta)
$$
where
$$
\delta=\inf\limits_{\substack{y\in K\\ |x|=r.}}|(x-y)|.
$$
\end{proof}


\begin{lemma}\label{chap31-lem3}
Let $K$ be compact in $\mathbb{R}^{d}$. Then for fixed $t$, $Q_{x}(\tau_{r}\leq
t)$ monotically decreses to zero as $r\to \infty$ and the convergence
is uniform on $K$.
\end{lemma}

\begin{proof}
Let $f_{r}(x)=Q_{x}(\tau_{r}\leq t)$. As $\{\tau_{r}\leq t\}$
decreases to the null set, $f_{r}(x)$ decreases to zero. As $K$ is
compact, there exists an $r_{0}$ such that for $r\geq r_{0}$,
$f_{r}(x)$ is continuous on $K$, by Lemma \ref{chap31-lem2}. Lemma
\ref{chap31-lem3} is a consequence of Dini's theorem.
\end{proof}

\begin{lemma}\label{chap31-lem4}
Let $b:\mathbb{R}^{r}\to \mathbb{R}^{d}$ be smooth (not necessarily
bounded). Then 
$E^{Q_{x}}(f(X(t)))$ is continuous in $x$ for every fixed $t$, $f$
being any bounded measurable function.
\end{lemma}

\begin{proof}
Let $b_{r}$ be any bounded smooth function on $R^{d}$ such that
$b_{r}\equiv b$ on $|x|\leq r$ and $Q^{r}_{x}$ the measure
corresponding to $b_{r}$. Then by Lemma \ref{chap31-lem1},
$E^{Q_{x}}(f(X(t)))$ is continuous in $x$ for all $r$. Further,
$$
|E^{Q^{r}_{x}}(f(X(t)))-E^{Q_{x}}(f(X(t)))|\leq 2||f||_{\infty}\cdot
Q_{x}(\tau_{r}\leq t).
$$

The\pageoriginale result follows by Lemma \ref{chap31-lem3}.
\end{proof}

\begin{lemma}\label{chap31-lem5}
With the hypothesis as the same as in Lemma \ref{chap31-lem1},
$(S_{1})$ is an equicontinuous family, where
$$
S_{1}=\{f:\mathbb{R}^{d}\to \mathbb{R},f\text{~ bounded measurable,~ }
||f||_{\infty}\leq 1\}
$$
\end{lemma}

\begin{proof}
For any $f$ in $S_{1}$, let $U(x)=U(t,x)\equiv E^{Q_{x}}(f(X(t)))$ and
$$
U_{\epsilon}(x)=U_{\epsilon}(t,x)=\int U(t-\epsilon,y)\frac{1}{(\surd
  (2\pi\epsilon)^{d})}\exp
\left[\frac{-|x-y|^{2}}{2\epsilon}\right]dy. 
$$

By Lemma \ref{chap31-lem1},
\begin{gather*}
|U(x)-U_{\epsilon}(x)|\leq (((e^{c\epsilon}-1)\epsilon^{ct}))^{1/2}\\
|U(x)-U(y)|\leq
|U(x)-U_{\epsilon}(x)|+|U_{\epsilon}(y)-U(y)|+|U_{\epsilon}(x)-U_{\epsilon}(y)|\\
\leq 2\surd ((e^{c\epsilon}-1)e^{ct})+|U_{\epsilon}(x)-U_{\epsilon}(y)|.
\end{gather*}

The family $\{U_{\epsilon}:f\in S_{1}\}$ is equicontinuous because
every $U$ occuring in the expression for $U_{\epsilon}$ is bounded by
1, and the exponential factor is uniformly continuous. Thus the right
hand side is very small if $\epsilon$ is small and $|x-y|$ is
small. This proves the lemma.
\end{proof}

\begin{lemma}\label{chap31-lem6}
Let $b$ be smooth and assume that there is no explosion ($b$ is not
necessarily bounded). Then $(S_{1})$ is an equi-continuous family
$\forall t>0$.
\end{lemma}

\begin{proof}
Let $r>0$ be given. Define $b_{r}\in C^{\infty}$ such that $b_{r}=0$
on $|x|>r+1$, $b_{r}=b$ on $|x|\leq r$, $b_{r}:\mathbb{R}^{d}\to
\mathbb{R}$. By Lemma \ref{chap31-lem2}, we have that
$$
\{E^{Q^{r}_{x}}(f(X(t))):f\in S_{1}\}
$$\pageoriginale
is equicontinuous, where $Q^{r}_{x}$ is the probability measure
corresponding to the function $b_{r}$.
\begin{equation*}
E^{Q_{x}}(f(X(t))\chi_{\{\tau_{r}>t\}})E^{Q^{r}_{x}}(f(X(t))\chi_{\{\tau_{r}>t\}}).\tag{1}\label{chap31-addeq1} 
\end{equation*}

Therefore
\begin{align*}
&\qq |E^{Q_{x}}(f(X(t)))-E^{Q^{r}_{x}}(f(X(t))|\\
&=
  |E^{Q_{x}}(f(X(t))\chi_{\{\tau_{r}>t\}})+E^{Q_{x}}(f(X(t))\chi_{\{\tau_{r}\leq
    t\}})\\ 
&-
  E^{Q^{r}_{x}}(f(X(t))\chi_{\{\tau_{r}>t\}})-E^{Q^{r}_{x}}(f(X(t)))\chi_{\{\tau_{r}\leq
    t\}})\\
&= |E^{Q_{x}}(f(X(t)\chi_{\{\tau_{r}\leq
    t\}})-E^{Q^{r}_{x}}(f(X(t))\chi_{\{\tau_{r}\leq t\}})|\\
&\leq ||f||_{\infty}(E^{Q_{x}}(\chi_{\{\tau_{r}\leq
    t\}})+E^{Q^{r}_{x}}(\chi_{\{\tau_{r}\leq t\}})\\
&\leq l[E^{Q_{x}}(\chi_{(\tau_{r}\leq
      t)})+E^{Q_{x}}(\chi_{(\tau_{r}\leq t)})](\text{use (1)  with~ }
  f=1)\\
&= 2E^{Q_{x}}(\chi_{(\tau_{r}\leq t)}).
\end{align*}

Thus
$$
\sup\limits_{x\in K}\sup\limits_{||f||_{\infty}\leq
  1}|E^{Q_{x}}(f(X(t))-E^{Q^{r}_{x}}(f(X(t))|\leq 2\sup\limits_{x\in
  K}(\chi_{\{\tau_{r}\leq t\}}).
$$

By Lemma \ref{chap31-lem3},
$$
\sup\limits_{x\in K}E^{0_{x}}(\tau_{r}\leq t)\to 0
$$
for every compact set $K$ as $n\to \infty$, for every fixed $t$.

The equicontinuity of the family $(S_{1})$ now follows
easily. For\pageoriginale fixed $x_{0}$, put
$u_{r}(x)=E^{Q^{r}_{x}}(f(X(t)))$ and $u(x)=E^{Q_{x}}(f(X(t)))$ and
let $K=s[x_{0},1]=\{x:|x-x_{0}|\leq 1\}$. Then
\begin{gather*}
|u(x)-u(x_{0})|\leq
|u(x)-u_{r}(x)|+|u(x_{0})-u_{r}(x_{0})|+|u_{r}(x)-u_{r}(x_{0})|\\
\leq 2\sup\limits_{y\in K}E^{Q_{y}}(\chi_{(\tau_{r}\leq
|t)})+|u_{r}(x)-u_{r}(x_{0})| 
\end{gather*}

By the previous lemma $\{u_{r}\}$ is an equicontinuous family and
since $\sup\limits_{y\in K}E^{Q_{y}}(\chi_{(\tau_{r}\leq 1)})\to 0$,
$\{u:||f||_{\infty}\leq 1\}$ is equicontinuous at $x_{0}$. This proves
the Lemma. 
\end{proof}

\begin{lemma}\label{chap31-lem7}
$T_{r}\circ T_{s}=T_{t+s}$, $\forall s$, $t\geq 0$.
\end{lemma}

\begin{remark*}
This property is called the {\em semigroup property}. 
\end{remark*}

\begin{proof}
\begin{align*}
& T_{r}(T_{s}f)(x)\\
& =\iint f(z)q(s,y,z)q(t,x,y)dy\ dz.
\end{align*}

Thus we have only to show that
$$
\int q(t,x,y)q(s,y,A)dy=q(t+s,x,A).
$$
\begin{align*}
q(t+s,x,A) &= E^{Q_{x}}(X(t+s)\in A)\\
&= E^{Q_{x}}(X(t+s)\in A|_{t}))\\
&= E^{Q_{x}}(E^{Q_{x}}X(t)(X(s)\in A))),\\
&\qq \text{by Markov property}\\
&= E^{Q_{x}}(q(s,X(t),A))\\
&= \int q(t,x,y)q(s,y,A)dy,
\end{align*}
which proves the result.

As\pageoriginale a trivial consequence we have the following.
\end{proof}

\begin{lemma}\label{chap31-lem8}
Let $\epsilon>0$ and let $S_{1}$ be the unit ball in
$B(\mathbb{R}^{d})$. Then $\bigcup\limits_{t\geq
  \epsilon}T_{t}(S_{1})$ is equicontinuous.
\end{lemma}

\begin{proof}
$\bigcup\limits_{t\geq \epsilon>0}T_{t}(S_{1})=T(\bigcup\limits_{t\geq
    0}T_{t}(S_{1}))$ (by Lemma \ref{chap31-lem7})
  $T_{\epsilon}(S_{1})$. 

The result follows by Lemma \ref{chap31-lem6}.
\end{proof}

\begin{lemma}\label{chap31-lem9}
Let $u(ttx)=E^{Q_{x}}(f(X(t)))$ with $||f||_{\infty}\leq 1$. Let
$\epsilon>0$ be given and $K$ any compact set. Then there exists a
$T_{0}=T_{0}(\epsilon,K)$ such that $\forall T\geq T_{0}$ and $\forall
x_{1}$, $x_{2}\in K$,
$$
|u(T,x_{1})-u(T,x_{2})|\leq \epsilon.
$$
\end{lemma}

\begin{proof}
Define
$q^{*}(t,x_{1},x_{2},y_{1},y_{2})=q(t,x_{1},y_{1})q(t,x_{2},y_{2})$
and let $Q_{(x_{1},x_{2})}$ be the measure corresponding to the
operator
$$
L=\frac{1}{2}(\Delta_{x_{1}}+\Delta_{x_{2}})+b(x_{1})\cdot
\nabla_{x_{1}}+b(x_{2})\cdot \nabla_{x_{2}}
$$
i.e., for any $u:\mathbb{R}^{d}\times \mathbb{R}^{d}\to \mathbb{R}$, 
\begin{align*}
Lu &= \frac{1}{2}\sum\limits^{2d}_{i=1}\frac{\p^{2}u}{\p
  x^{2}_{i}}+\sum\limits^{t}_{i=1}b_{i}(x_{1},\ldots,x_{d})\frac{\p
  u}{\p x^{2}_{i}}+\\
&\qq +\sum\limits^{d}_{i=1}b_{i}(x_{d+1,\ldots,x_{2d}})\frac{\p u}{\p x_{i+d}}.
\end{align*}

Then $Q_{(x_{1},x_{2})}$ will be a measure on
$C([0,\infty);\mathbb{R}^{d}\times \mathbb{R}^{d})$. We claim that
  $Q_{(x_{1},x_{2})}=Q_{x_{1}}\times Q_{x_{2}}$. Note that
$$
C([0,\infty);\mathbb{R}^{d}\times
  \mathbb{R}^{d})=C([0,\infty);\mathbb{R}^{d})\times
    C[(0,\infty);\mathbb{R}^{d}) 
$$\pageoriginale
and since $C([0,\infty);\mathbb{R}^{d})$ is a second countable metric
  space, the Borel field of $C([0,\infty)\mathbb{R}^{d}\times
    \mathbb{R}^{d})$ is the $\sigma$-algebra generated by
    
$$
\mathscr{B}=(C([0,\infty);\mathbb{R}^{d}))\times\mathscr{B}(C[0,\infty);\mathbb{R}^{d}).
$$ 

By going to the finite-dimensional distributions one can check that
$P_{(x_{1},x_{2})}=P_{x_{1}}\times P_{x_{2}}$.
\begin{align*}
& \frac{dQ_{(x_{1},x_{2})}}{dP_{(x_{1},x_{2})}}\Big|_{\mathscr{F}_{t}}=\exp
  \left[\int\limits^{t}_{0}\langle b^{(1)},dX_{1}\rangle
    -\frac{1}{2}\int\limits^{t}_{0}|b^{(1)}|^{2}ds\right]\times\\
&\q \times \exp \left[\int\limits^{t}_{0}\langle b^{(2)},dX_{2}\rangle
    -\frac{1}{2}\int\limits^{t}_{0}|b^{(2)}|^{2}ds\right], 
\end{align*}
where 
$$
b^{(1)}(x_{1}\ldots
x_{d})=b(x_{1}\ldots,x_{d})\cdot  
b^{(2)}(x_{d+1}\ldots
x_{2d})=b(x_{d+1},\ldots,x_{2d}),
$$ 
so that
$Q_{(x_{1},x_{2})}=Q_{x_{1}}\times Q_{x_{2}}$.

It is clear that if $\phi$ defined an invariant measure for the
process $Q_{x}$, i.e.
$$
\int\limits_{A}\phi(x)dx=\int \phi(y)Qy(X_{t}\in A)dy,
$$
then $\phi(y_{1})\phi(y_{2})$ defines an invariant measure for the
process $Q_{(x_{1},x_{2})}$. Thus the process $Q_{(x_{1},x_{2})}$ is
recurrent.

Next we show that $u(T-t,X_{1}(t))$ is a martingale $(0\leq t\leq T)$
for any fixed $T$ on $C([0,T];\mathbb{R}^{d})$.
\begin{align*}
& E^{Q_{x}}(u(T-t,X(t)|\mathscr{F}_{s}))\\
& =[\int u(T-t,y)q(t-s,x,dy)]_{x=X(s)}\\
& =[\iint f(z)q(T-t,y,dz)q(t-s,x,dy)]_{x=X(s)}\\
& =[\int f(z)q(T-s,x,dz)]_{x=X(s)}\\
& =u(T-s,X(s)),\q s<t. 
\end{align*}\pageoriginale

It now follows that $u(T-t,X_{1}(t))$ is a martingale on
$C([0,\infty);\mathbb{R}^{d})\times
  C([0,\infty);\mathbb{R}^{d})$. Hence
    $u(T-t,X_{1}(t))-u(T-t,X_{2}(t))$ is a martingale relative to
    $Q_{(X_{1},x_{2})}$. 

Let $V=S(0,\delta/2)\subset \mathbb{R}^{d}\times \mathbb{R}^{d}$ with
$\delta<1/4$. If $(x_{1},x_{2})\in V$, then
$$
|x_{1}-x_{2}|\leq |(x_{1},0)-(0,0)|+|(0,0)-(0,x_{2})|<\delta.
$$
\end{proof}

\setcounter{claim}{0}
\begin{claim}\label{chap31-claim1}
$Q_{(x_{1},x_{2})}(\tau_{V}\leq T)\to 1$ as $T\to \infty$, where
  $\tau_{V}$ is the exit time from $R^{d}-V$.
\end{claim}

\begin{proof}
If $w$ is any trajectory starting at some point in $V$, then
$\tau_{V}=0\leq T$, $\forall T$. If $w$ starts at some point outside
$V$ then, by the recurrence property, $w$ has to visit a ball with
centre $0$ and radius $\delta/2$; hence it must get into $V$ at some
finite time. Thus $\{\tau_{V}\leq T\}\uparrow$ to the whole space as
$T\uparrow \infty$. Next we show that the convergence is uniform on
compact sets.

If $x_{1}$, $x_{2}\in K$, $(x_{1},x_{2})\in K\times K$ (a compact
set). Put $g_{T}(x_{1},x_{2})=Q_{(x_{1},x_{2})}(\tau_{V}\leq T)$. Then
$g_{T}(x_{1},x_{2})\geq 0$ and $g_{T}(x_{1},x_{2})$ increases to $1$
as $T$ tends to $\infty$.
\begin{gather*}
g_{T}(x_{1},x_{2})=Q_{(x_{1},x_{2})}(\tau_{V}\leq T)\\
Q_{(x_{1},x_{2})}(\tau^{1}_{V}\leq T),
\end{gather*}\pageoriginale
where 
$$
\tau^{1}_{V}=\inf \{t\geq 1:(x_{1},x_{2})\in V\}.
$$

Therefore
\begin{align*}
g_{T}(x_{1},x_{2}) &\geq
E^{Q}(x_{1},x_{2})(E^{Q}(x_{1},x_{2})((\tau^{1}_{V}\leq T)|_{1}))\\
&= E^{Q}(x_{1},x_{2})(Q_{(X_{1}(1),X_{2}(1))}\{\tau^{1}_{V}\leq
T)\})\\
&= E^{Q}(x_{1},x_{2})(\psi_{T}(X_{1}(1),X_{2}(1))),
\end{align*}
where $\psi_{T}$ is a bounded non-negative function. Thus, if 
\begin{align*}
h_{T}(x_{1},x_{2}) &= Q_{(x_{1},x_{2})}(\tau^{1}_{V}\leq T)=\\
&= E^{Q}(x_{1},x_{2})(\psi_{T}(X_{1}(1),X_{2}(1))),
\end{align*}
then by Lemma \ref{chap31-lem4}, $h_{T}$ is continuous for each $T$,
$g_{T}\geq h_{T}$ and $h_{T}$ increases to $1$ as $T\to
\infty$. Therefore, $h_{T}$ converges uniformly (and so does $g_{T}$)
on compact sets.

Thus given $\epsilon>0$ chose $T_{0}=T_{0}(\epsilon,K)$ such that if
$T\geq T_{0}$,
$$
\sup\limits_{x_{2}\in K}\sup\limits_{x_{1}\in
  K}Q_{(x_{1},x_{2})}(\tau_{V}\geq T-1)\leq \epsilon.
$$

By Doob's optional stopping theorem and the fact that
$$
u(T-t,X_{1}(t))-u(t-t,X_{2}(t))
$$ 
is a martingale, we get, on equating
expectations, 
{\fontsize{10pt}{12pt}\selectfont
\begin{align*}
& |u(T,x_{1})-u(T,x_{2})|\\
&\qq =|E^{Q_{(x_{1},x_{2})}}[u(T-0,X_{1}(0)-u(T-0,X_{2}(0)]|\\
&\qq =|E^{Q_{(x_{1},x_{2})}}[u(T-(\tau_{v}\wedge
    (T-1)),X_{1}(T-(\tau_{v}\wedge (T-1))-\\
&\qq\q -u(T-(\tau_{v}\wedge T(-1)),X_{2}(T-(\tau_{v}\wedge (T-1)]|\\
&\qq |\int\limits_{\{\tau_{v}\geq
    T-1\}}[u(1,X_{1}(1))-u(1,X_{2}(1))]dQ_{(x_{1},x_{2})}+\\
&\qq +\int\limits_{\{\tau_{v}<(T-1)\}}[u(T-\tau_{v},X_{1}(T-\tau_{v}))-u(T-\tau_{v}),X_{2}(T-\tau_{v}))dQ_{(x_{1},x_{2})}|. 
\end{align*}}\relax\pageoriginale

Therefore
{\fontsize{10pt}{12pt}\selectfont
\begin{align*}
& |u(T,x_{1})-u(T,x_{2})|\\
& \leq\int\limits_{\{\tau_{v}\geq
    (T-1)\}}|[u(1,X_{1}(1))-u(1,X_{2}(1))]|dQ_{(x_{1},x_{2})}+\\ 
&
  +|\int\limits_{\{\tau_{v}<(T-1)\}}[u(T-\tau_{v},X_{1}(T-\tau_{v}))-u(T-\tau_{v},X_{2}(T-\tau_{v}))dQ_{(x_{1},x_{2})}|\\
& \leq 2\epsilon + |\int\limits_{\{\tau_{v}<(T-1)\}}[u(T-\tau_{v},X_{1}(T-\tau_{v}))-u(T-\tau_{v},X_{2}(T-\tau_{v}))]dQ_{(x_{1},x_{2})}|,
\end{align*}}\relax
since $u$ is bounded by $1$.

The second integration is to be carried out on the set $\{T-v\geq
1\}$. Since $\bigcup\limits_{t\geq 1}T_{t}(S_{1})$ is equicontinuous
we can choose a $\delta>0$ such that whenever $x_{1}$, $x_{2}\in K$
such that $|x_{1}-x_{2}|<\delta$
$$
|u(t,x_{1})-u(t,x_{2})|\leq \epsilon,\q \forall t\geq 1.
$$\pageoriginale

Thus $|u(T,x_{1})-u(T,x_{2})|\leq 3\epsilon$ whenever $x_{1}$,
$x_{2}\in K$ and $T\geq T_{0}$. This proves the Lemma.
\end{proof}

\noindent
{\bf Corollary to Lemma \ref{chap31-lem9}.}
$\sup\limits_{x_{1},x_{2}\in K}\int|q(t,x_{1},y)|dy$ {\em converges to
  $0$ as $t\to \infty$.}

\begin{proof}
Since the dual of $L^{1}$ is $L^{\infty}$, we have
\begin{gather*}
\int |q(t,x_{1},y)-q(t,x_{2},y)|dy\\
=\sup\limits_{||f||_{\infty}\leq 1}|\int [q(t,x_{1},y)-q(t,x_{2},y)]f(y)dy|
\end{gather*}
and the right side converges to $0$ as $t\to \infty$, by Lemma
\ref{chap31-lem9}.

We now come to the proof of the main theorem stated before Lemma
\ref{chap31-lem1}. Now
\begin{align*}
&\qq \int |q(t,x,y)-\phi(y)|dy\\
&= \int|q(t,x,y)-\int \phi(x^{1})q(t,x^{1},y)dx^{1}|dy\\
&\hspace{3.5cm} \text{(by invariance property)}\\
&= \int |\int q(t,x,y)\phi(x^{1})dx^{1}-\int
  \phi(x^{1})q(t,x^{1},y)dx^{1}|dy\\
&\hspace{4.5cm} (\text{since~ }\int\phi(x^{1})dx^{1}=1)\\
&\leq \iint |q(t,x,y)-q(t,x^{1},y)|\phi(x^{1})dx^{1}dy\q (\text{since~
  } \phi\geq 0)\\
&= \int \phi(x^{1})dx^{1}\int |q(t,x,y)-q(t,x^{1},y)|dy\\
\end{align*}

Since
$$
\int \phi (x^{1})dx^{1}=\Lt\limits_{n\to
  \infty}\int\limits_{|x^{1}|\leq n}\phi(x^{1})dx^{1},
$$
choose\pageoriginale a compact set $L$ $K$ such that
$\int\limits_{\mathbb{R}^{d}-L}\phi(x^{1})dx^{1}<\epsilon$. Then
\begin{align*}
& \int\phi(x^{1})dx^{1}\int |q(t,x,y)-q(t,x^{1},y)|dy\\
& =\int\limits_{L}\phi(x^{1})dx^{1}\int |q(t,x,y)-q(t,x^{1},y)|dy+\\
& +\int\limits_{\mathbb{R}^{d}-L}\phi(x^{1})dx^{1}\int
  |q(t,x,y)-q(t,x^{1},y)|dy\\ 
&\leq \iint\limits_{L}\phi(x^{1})dx^{1}\int
  |q(t,x,y)-q(t,x^{1},y)|dv+2\epsilon. 
\end{align*}

Chose $t_{0}$ such that whenever $t\geq t_{0}$,
$$
\int |q(t,x,y)-q(t,x^{1},y)|dy\leq
\frac{\epsilon}{1+\int\limits_{L}\phi(x^{1})dx^{1}} 
$$
$\forall x$, $x_{1}$ in $L$. (Corollary to Lemma
\ref{chap31-lem9}). Then
$$
\int |q(t,x,y)-\phi(y)|dy\leq 3\epsilon,
$$
if $t\geq t_{0}\forall x\in K$ completing the proof of the theorem.
\end{proof}
