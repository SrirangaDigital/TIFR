\chapter{Properties of Brownian Motion in One Dimension}\label{chap9}

WE\pageoriginale NOW PROVE the following.

\begin{lemma*}
Let $(X_{t})$ be a one-dimensional Brownian motion. Then
\begin{itemize}
\item[\rm(a)] $P(\varlimsup X_{t}=\infty)=1$; consequently
  $P(\varlimsup X_{t}<\infty)=0$.

\item[\rm(b)] $P(\varliminf X_{t}=-\infty)=1$; consequently
  $P(\varliminf X_{t}>-\infty)=0$. 

\item[\rm(c)] $P(\varliminf X_{t}=-\infty);\ \varlimsup X_{t}=\infty)=1$.
\end{itemize}
\end{lemma*}

\medskip
\noindent
{\bf SIGNIFICANCE.}~ By (c) almost every Brownian path assumes each
value infinitely often.

\begin{proof}
\begin{align*}
\{\varlimsup X_{t}=\infty\} &=
\bigcap\limits^{\infty}_{n=1}(\varlimsup X_{t}>n)\\
&= \bigcap\limits^{\infty}_{n=1}(\varlimsup_{\theta\text{~
    rational}}X_{\theta}>n)\q \text{(by continuity of Brownian paths)}
\end{align*}

First, note that
\begin{align*}
P_{0}\left[\sup\limits_{0\leq s\leq t}X(s)\leq n\right] &=
1-P_{0}\left[\sup\limits_{0\leq s\leq t}X(s)>n\right]\\
&= 1-21/\surd(2\pi t)\int\limits^{\infty}_{n}e^{-y^{2}/2t}dy\\
&= \surd(2/\pi t)\int\limits^{n}_{0}e^{-y^{2}/2t}dy.
\end{align*}

Therefore, for any $x_{0}$ and $t$,
$$
P\left[\sup\limits_{t_{0}\leq s\leq t}X(s)\geq
  n|X(t_{0})=x_{0}\right]=P_{0}\left[\sup\limits_{0\leq s\leq
    t-t_{0}}X(s)\geq n-x_{0}\right]
$$
(independent increments) which tends to $1$ as $t\to
\infty$. Consequently,
\begin{align*}
P_{0}\left[\sup\limits_{t\geq t_{0}}X(t)\geq n\right] &=
EP\left[\sup\limits_{t\geq t_{0}}X(t)\geq n|X(t_{0})\right]\\
&= E1=1.
\end{align*}\pageoriginale

In other words,
$$
P_{0}\left[{\displaystyle{\mathop{\lim\sup}_{t\to \infty}}}X(t)\geq n\right]=1
$$
for every $n$. Thus
$$
P(\varlimsup X_{t}=\infty)=1.
$$
\begin{itemize}
\item[(b)] is clear if one notes that $w\to -w$ leaves the probability
invariant.

\item[(c)] \begin{tabbing}
 $P(\varlimsup X_{t}$ \== $\infty,\varliminf X_{t}=-\infty)$\\[4pt]
\>= $P(\varlimsup X_{t}=\infty)-P(\varliminf X_{t}>-\infty,\varlimsup
X_{t}=\infty)$.\\[4pt]
\>$\geq  1-P(\varliminf X_{t}>-\infty)$\\[4pt]
\>= $1$.
\end{tabbing}
\end{itemize}
\end{proof}

\begin{coro*}
Let $(X_{t})$ be a $d$-dimensional Brownian motion. Then 
$$
P(\varlimsup |X_{t}|=\infty)=1.
$$
\end{coro*}

\begin{remark*}
If $d\geq 3$ we shall see later that $P(\Lt\limits_{t\to
  \infty}|X_{t}|=\infty)=1$. i.e.\@ almost every Brownian path
``wanders'' off to $\infty$.
\end{remark*}

\begin{theorem*}
Almost all Brownian paths are of unbounded variation in any interval.
\end{theorem*}

\begin{proof}
Let $I$ be any interval $[a,b]$ with $a<b$. For $n=1,2,\ldots$ define
$$
V_{n}(wQ_{n})=\sum\limits^{n}_{i=1}|w(t_{i})-w(t_{i-1})|\,
(t_{i}=a+(b-a)i/n,i=0,1,2,\ldots n),
$$

The\pageoriginale variation corresponding to the partioin $Q_{n}$
dividing $[a,b]$ into $n$ equal parts. Let
$$
U_{n}(w,Q_{n})=\sum\limits^{n}_{i=1}|(w(t_{i})-w(t_{i-1})|^{2}.
$$

If 
$$
A_{n}(w,Q_{n})\sup\limits_{1\leq i\leq n}|w(t_{i})-w(t_{i-1})|,
$$
then
$$
A_{n}(w,Q_{n})V_{n}(w,Q_{n})\geq U_{n}(w,Q_{n}).
$$

By continuity $\Lt\limits_{n\to \infty}A_{n}(w,Q_{n})=0$.
\end{proof}

\begin{claim*}
$\Lt\limits_{n\to \infty}E[(U_{n}(w,Q_{n})-(b-a))^{2}]=0$.
\end{claim*}

\begin{proof}
\begin{align*}
& E[(U_{n}-(b-a))^{2}]\\
&=E\left\{\sum\limits^{n}_{j=1}[(X_{t_{j}}-X_{t_{j-1}})^{2}-(b-a/n)]\right\}^{2}\\
& E[(\sum (Z^{2}_{j}-b-a/n))^{2}],\ Z_{j}=X_{t_{j}}-X_{t_{j-1}},\\
&= nE[(Z^{2}_{1}-b-a/n)^{2}]
\end{align*}
(because $Z_{j}$ are independent and identically distributed).
$$
=n[E(Z^{4}_{1})-(b-a/n)^{2}]=2(b-a/n)^{2}\to 0.
$$

Thus a subsequence $U_{n_{i}}\to b-a$ almost everywhere. Since
$A_{n_{i}}\to 0$ it follows that $V_{n_{i}}(w,Q_{n})\to \infty$ almost
everywhere. This completes the proof.
\end{proof}

\begin{note*}
$\{w:w$\pageoriginale is of bounded variation on $[a,b]\}$ can be shown to be
  measurable if one proves
\end{note*}

\begin{exer*}
Let $f$ be continuous on $[a,b]$ and define $V_{n}(f,Q_{n})$ as
above. Show that $f$ is of bounded variation on $[a,b]$ iff
$\sup\limits_{n=1,2,\ldots}V_{n}(f,Q_{n})<\infty$. 
\end{exer*}

\begin{theorem*}
Let $t$ be any fixed real number in $[0,\infty)$, $D_{t}=\{w:w$ is
  differentiable at $t\}$. Then $P(D_{t})=0$.
\end{theorem*}

\begin{proof}
The measurability of $D_{t}$ follows from the following observation:
if $f$ is continuous then $f$ is differentiable at $t$ if and only if
$$
\Lt\limits_{\substack{r\to 0\\ r\text{~
      rational}}}\frac{f(t+r)-f(t)}{r}.
$$
exists. Now
$$
D_{t}=\bigcup\limits^{\infty}_{m=1}w:|\frac{w(t+h)-w(t)}{h}|\leq
M,\text{~ for all~ } h\neq 0,\text{~ rational}\}
$$
and 
$$
P\left\{w:|\frac{X_{t+h}-X_{t}}{h}|\leq M\;\forall h\in Q, h\neq 
0\right\}\leq 2\inf\limits_{h}\int\limits^{M\surd
  h}_{0}\frac{1}{\surd(2\pi)}e^{-|y|^{2/2}}dy=0 
$$
\end{proof}

\begin{remark*}
A stronger result holds:
\begin{gather*}
P\left(\bigcup\limits_{t\geq 0}D_{t}\right)=0.\\
\text{Hint:~ }\bigcup\limits_{0\leq t\leq
  1}D_{t}\bigcup\limits_{=1}\bigcup\limits_{m=1}\bigcup\limits^{n+2}_{n-m\ i=1\ k=i+1,i+2,i+3}\left\{w:w\left(\frac{k}{n}\right)-w\left(\frac{k-1}{n}\right)|\leq \frac{71}{n}\right\}
\end{gather*}
and
$$
P\left(\bigcup\limits^{n+2}_{i=1\ k=i+1,\ldots,i+3}\left\{w:w\left(\frac{k}{n}\right)-w\left(\frac{k-1}{n}\right)|\leq\frac{71}{n}\right\}\q
  \text{const}/\surd n\right)
$$

This construction is due to A.\@ Dvoretski, P.\@ Erdos \& S.\@ Kakutani.
\end{remark*}





