\chapter{Uniqueness of Diffusion Process}\label{chap26}

IN\pageoriginale THE LAST section we proved that
$$
\xi(t,w)=x_{0}+\int\limits^{t}_{t_{0}}\langle \sigma(s,\xi(s,w)),d\beta(s,w)+\int\limits^{t}_{t_{0}}b(s,(s,w)ds
$$
has a solution under certain conditions on $b$ and $\sigma$ where
$\sigma\sigma^{*}=a$. The measure
$P_{t_{0},x_{0}}=P\xi^{-1}_{t_{0},x_{0}}$ was constructed on
$(C([t_{0},\infty);\mathbb{R}^{d}),\mathscr{F}_{t_{0}})$ so that the
  map $X(t,w)=w(t)$ is an It\^o process with parameters $b$ and
  $a$. We now settle the uniqueness question, about the diffusion
  process. 

\begin{theorem*}
Let
\begin{itemize}
\item[\rm(i)] $a:[0,\infty)\times \mathbb{R}^{d}\to S^{+}_{d}$ and
  $b:[0,\infty)\times \mathbb{R}^{d}\to \mathbb{R}^{d}$ be bounded
    measurable functions;

\item[\rm(ii)] $\Omega=C([0,);\mathbb{R}^{d})$;

\item[\rm(iii)] $X:[0,\infty)\times \Omega\to \mathbb{R}^{d}$ be
  defined by $X(t,w)=w(t)$;

\item[\rm(iv)] $X_{t}=\sigma\{X(s):0\leq s\leq t\}$;

\item[\rm(v)] $P$ be any probability measure on
$$
=\sigma\left(\bigcup\limits_{t\geq 0}X_t\right)
$$
such that $P\{X(0)=x_{0}\}=1$ and $X$ is an It\^o process relative to
$(\Omega,X_{t},P)$ with parameters $b(t,X_{t})$ and $a(t,X_{t})$;

\item[\rm(vi)] $\sigma:[0,\infty)\times \mathbb{R}^{d}\to M_{d\times
    n}$ be a bounded measurable map into the set of all real $d\times n$
  matrices such that $\sigma\sigma^{*}=a$ on
  $[0,\infty)\times\mathbb{R}^{d}$. 
\end{itemize}

Then\pageoriginale there exists a generalised $n$-dimensional
Brownian motion $\beta$ on $(\overline{\Omega},\sum_{t},Q)$ and a
progressively measurable a.s.\@ continuous map
$\xi:[0,\infty)\times\overline{\Omega}\to \mathbb{R}^{d}$ satisfying
  the equation
\begin{equation*}
\xi(t,\overline{w})=x_{0}+\int\limits^{t}_{0}\langle
\sigma(s,\int(s,\overline{w})),d\beta(s,\overline{w})\rangle
+\int\limits^{t}_{0}b(s,\xi(s,\overline{w}))ds\tag{1}\label{chap26-eq1}
\end{equation*}
with $Q\xi^{-1}=P$, where $\xi:\overline{\Omega}\to \Omega$ is given
by $(\xi(\overline{w}))(t)=\xi(t,\overline{w})$. 
\end{theorem*}

Roughly speaking, any It\^o process can be realised by means of a
diffusion process governed by equation \eqref{chap26-eq1} with
$\sigma\sigma^{*}=a$. 

\begin{proof}
{\bf Case (i).}~ Assume that there exist constants $m$, $M>0$ such
that $mI\leq a(t,x)\leq MI$ and $\sigma$ is a $d\times d$ matrix
satisfying $\sigma\sigma^{*}=a$. In this case we can identify
$(\overline{\Omega},\sum_{t},Q)$ with
$(\Omega,\mathscr{F}_{t},P)$. Since $D(t,\cdot)$ is an It\^o process,
$$
\exp\langle \theta,X(t)\rangle -\int\limits^{t}_{0}\langle
\theta,b(s,X(s,\cdot))\rangle ds-\frac{1}{2}\int\limits^{t}_{0}\langle
\theta,a(s,X(s,\cdot))\theta\rangle ds 
$$ 
is a $(\Omega,\mathscr{F}_{t},P)$-martingale. Put
$$
Y(t,w)=X(t,w)-\int\limits^{t}_{0}b(s,X(s,w))ds-x_{0}.
$$
Clearly $Y(t,w)$ is an It\^o process corresponding to the parameters
$$
[0,a(s,X_{s})],
$$ 
so that
$$
\exp\langle \theta, Y(t,w)\rangle
-\frac{1}{2}\int\limits^{t}_{0}\langle
\theta,a(s,X(s,\cdot))\theta\rangle ds
$$
is a $(\Omega,\mathscr{F}_{t},P)$-martingale. The conditions $m\leq
a\leq M$ imply that $\sigma^{-1}$ exists and is bounded. Let
$$
\eta(t)=\int\limits^{t}_{0}\sigma^{-1}dY=\int\limits^{t}_{0}\sigma^{-1}(s,X(s,\cdot))dY(s,\cdot),
$$\pageoriginale
so that (by definition of a stochastic integral) $\eta$ is a
$(\Omega,\mathscr{F}_{t},P)$-It\^o process with parameters zero and
$\sigma^{-1} a (\sigma^{-1})^{*}=1$. Thus $\eta$ is a Brownian
motion relative to $(\Omega,\mathscr{F}_{t},P)$. Now by change of
variable formula for stochastic integrals,
\begin{gather*}
\int\limits^{t}_{0}\sigma
d\eta=\int\limits^{t}_{0}\sigma\sigma^{-1}dY\\
=Y(t)-Y(0)=Y((t),
\end{gather*}
since $Y(0)=0$. Thus
$$
X(t)=x_{0}+\int\limits^{t}_{0}\sigma(s,X(s,\cdot))d+\int\limits^{t}_{0}b(s,X(s,\cdot))ds. 
$$

Taking $Q=P$ we get the result.

\medskip
\noindent
{\bf Case (ii).}~ $a=0$, $b=0$, $x_{0}=0$, $\sigma=0$ where $\sigma\in
M_{d\times n}$. Let $(\Omega^{*},\mathscr{F}^{*}_{t},P^{*})$ be an
$n$-dimensional Brownian motion. Define
$$
(\overline{\Omega},\sum_{t},Q)=(\Omega\times
\Omega^{*},\mathscr{F}_{t}\times \mathscr{F}^{*}_{t},P\times P^{*}).
$$

If $\beta$ is the $n$-dimensional Brownian motion on
$(\Omega^{*},\mathscr{F}^{*}_{t},P^{*})$, we define $\overline{\beta}$
on $\overline{\Omega}$ by
$\overline{\beta}(t,w,w^{*})=\beta(t,w^{*})$. It is easy to verify
that $\overline{\beta}$ is an $n$-dimensional Brownian motion on
$(\overline{\Omega},\sum_{t},Q)$. Taking $\xi(t,w,w^{*})=x_{0}$ we get
the result.
\end{proof}

Before we take up the general case we prove a few Lemmas.

\setcounter{lemma}{0}
\begin{lemma}\label{chap26-lem1}
Let\pageoriginale $\sigma:\mathbb{R}^{n}\to \mathbb{R}^{d}$ be linear
$\sigma\sigma^{*}=a:\mathbb{R}^{d}\to \mathbb{R}^{d}$; then there
exists a linear map which we denote by $\sigma^{-1}:\mathbb{R}^{d}\to
\mathbb{R}^{n}$ such that
$\sigma^{-1}a\sigma^{-1*}=\pi_{N^{\perp}_{\sigma}}$, where $\pi$
denotes the projection and $N_{\sigma}$ null space of $\sigma$.
\end{lemma}

\begin{proof}
Let $R_{\sigma}=$ range of $\sigma$. Clearly
$\sigma:N^{\perp}_{\sigma}\to R$ is an isomorphism. Let
$\tau:R_{\sigma}\to N^{\perp}_{\sigma}$ be the inverse. We put
$$
\sigma^{-1}=\tau\oplus 0:R_{\sigma}\oplus R^{\perp}_{\sigma}\to
N^{\perp}_{\sigma}\oplus N_{\sigma}. 
$$
\end{proof}

\begin{lemma}\label{chap26-lem2}
Let $X$, $Y$ be martingales relative to $(\Omega,\mathscr{F}_{t},P)$
and $(\overline{\Omega},\overline{\mathscr{F}}_{t},\overline{P})$
respectively. Then $Z$ given by
$$
Z(t,w,\overline{w})=X(t,w)Y(t,\overline{w})
$$
is a martingale relative to
$$
(\Omega\times\overline{\Omega},\mathscr{F}_{t}\times
\overline{\mathscr{F}}_{t},P\times \overline{P}).
$$
\end{lemma}

\begin{proof}
From the definition it is clear that for every $t>s$
$$
\int\limits_{A\times \overline{A}}Z(t,w,\overline{w})d(P\times
\overline{P})|_{\mathscr{F}_{s}\times
  \overline{\mathscr{F}_{s}}}=\int\limits_{A\times\overline{A}}Z(s,w,\overline{w})d(P\times\overline{P}) 
$$
if $A\in \mathscr{F}_{s}$ and $\overline{A}\in
\overline{\mathscr{F}}_{s}$. The general case follows easily.
\end{proof}


As a corrollary to Lemma \ref{chap26-lem2}, we have

\begin{lemma}\label{chap26-lem3}
Let $X$ be a $d$-dimensional It\^o process with parameters $b$ and $a$
relative to $(\Omega,\mathscr{F}_{t},P)$ and let $Y$ be a
$\overline{d}$-dimensional It\^o process relative to
$(\overline{\Omega},\overline{\mathscr{F}}_{t},\overline{P})$ relative
to $\overline{b}$ and $\overline{a}$. Then
$Z(t,w,\overline{w})=(X(t,w),Y(t,\overline{w}))$ is a
$(d+d)$-dimensional It\^o process with parameters
$B=(b,\overline{b})$, $A=\left[\begin{smallmatrix} a & 0\\ 0 &
    \overline{a}\end{smallmatrix}\right]$ relative to $(\Omega\times
\overline{\Omega}, \mathscr{F}_{t}\times \overline{\mathscr{F}}_{t},P\times
\overline{P})$. 
\end{lemma}

\begin{lemma}\label{chap26-lem4}
Let\pageoriginale $X$ be an It\^o process relative to
$(\Omega,\mathscr{F}_{t},P)$ with parameters $0$ and $a$. If $\theta$
is progressively measurable such that
$E(\int\limits^{t}_{0}|\theta|^{2},ds)<\infty$, $\forall t$ and
$\theta a\theta^{*}$ is bounded, then $\int\limits^{t}_{0}\langle
\theta, dX\rangle \in I[0,\theta a\theta^{*}]$.
\end{lemma}

\begin{proof}
Let $\theta_{n}$ be defined by
$$
\theta^{i}_{n}=
\begin{cases}
\theta^{i}, & \text{if~ } |\theta|\leq n,\\
0, & \text{otherwise;}
\end{cases}
$$

Then $\int\limits^{t}_{0}\langle \theta_{n},dX\rangle \in
I[0,\theta_{n}a\theta^{*}_{n}]$. Therefore
$$
X_{n}(t)=\exp(\lambda\int\limits^{t}_{0}\langle \theta_{n},dX\rangle
-\frac{\lambda^{2}}{2}\int\limits^{t}_{0}\langle
\theta_{n},a\theta_{n}\rangle ds
$$
is a martingale converging pointwise to
$$
X(t)=\exp \left(\lambda\int\limits^{t}_{0}\langle \theta, dX\rangle
-\frac{\lambda^{2}}{2}\int\limits^{t}_{0}\langle \theta,
a\theta\rangle ds\right).
$$

To prove that $\int\limits^{t}_{0}\langle \theta, dX\rangle$ is an
It\^o process we have only to show that $X_{n}(t)$ is uniformly
integrable. Without loss of generality we may assume that
$\lambda=1$. Let $[0,T]$ be given
\begin{align*}
E(X^{2}_{n}(t,w)) &= E\left(\exp\left[2 \int\limits^{t}_{0}\langle
  \theta_{n},dX\rangle -\int\limits^{t}_{0}\langle
  \theta_{n},a\theta_{n}\rangle ds\right]\right)\\
&= E\left(\exp \left[ 2\int\limits^{t}_{0}\langle \theta_{n},dX\rangle
  -2\int\limits^{t}_{0}\langle \theta_{n},a\theta_{n}\rangle
  ds\right.\right.\\
&\qq\left.\left. +\int\limits^{t}_{0}\langle
  \theta_{n},a\theta_{n}\rangle ds\right]\right).\\
&\leq e^{T}\sup\limits_{0\leq t\leq T}\langle
\theta_{n},a\theta_{n}\rangle. 
\end{align*}

But\pageoriginale $\langle \theta, a\theta^{*}\rangle$ is bounded and
therefore $\langle \theta_{n},a\theta_{n}\rangle$ is uniformly bounded
in $n$. Therefore $(X_{n})$ are uniformly integrable. Thus
$X(t,\cdot)$ is a martingale.

\medskip
\noindent
{\bf Case (iii).}~ Take $d=1$, and assume that
$$
\int\limits^{t}_{0}a^{-1}\chi_{(a>0)}ds<\infty,\forall t
$$
with $a>0$; let $\sigma=+$ve squareroot of $a$. Define
$1/\sigma=1/\sigma$ if $\sigma>0$, and $1/\sigma=0$ if $\sigma=0$. Let
$$
Y(t)=X(t)-x_{0}-\int\limits^{t}_{0}b(s,X(s))ds.
$$

Denote by $Z$ the one-dimensional Brownian motion on
$(\Omega^{*},\mathscr{F}^{*}_{t},P^{*})$ where
$\Omega^{*}=C([0,\infty),R)$. Now
$$
Y\in I[0,a(s,X(s,\cdot))],\ Z \in I[0,1].
$$

By Lemma \ref{chap26-lem3},
$$
(Y,Z)\in I\left((0,0);
\begin{pmatrix}
a & 0\\
0 & 1
\end{pmatrix}
\right).
$$

If
$$
\eta(t,w,w^{*})=\int\limits^{t}_{0}\langle
\left(\frac{1}{\sigma(s,X(s,\cdot))}\chi_{(\sigma>0)},\chi_{\sigma=0},d(Y,Z)\rangle\right) 
$$
then Lemma \ref{chap26-lem4} shows that
$$
\eta \in I[0,1].
$$

Therefore $\eta$ is a one-dimensional Brownian motion on
$\overline{\Omega}=(\Omega\times \Omega^{*},\mathscr{F}_{t}\times
\mathscr{F}^{*}_{t},P\times P^{*})$. Put
$$
\overline{Y}(t,w,w^{*})=Y(t,w)\q\text{and}\q
\overline{X}(t,w,w^{*})=X(t,w); 
$$
then\pageoriginale
\begin{align*}
\int\limits^{t}_{0}\sigma d\eta &=
\int\limits^{t}_{0}\sigma\frac{1}{\sigma}\chi_{(\sigma>0)}dY+\int\limits^{t}_{0}\sigma\chi_{(\sigma=0)}dZ\\
&= \int\limits^{t}_{0}\chi_{(\sigma>0)}dY.
\end{align*}

Since
$$
E\left(\left(\int\limits^{t}_{0}\chi_{(\sigma=0)}dY\right)^{2}\right)=E\left(\int\limits^{t}_{0}\sigma^{2}\chi_{(\sigma=0)}ds\right)=0, 
$$
it follows that
$$
\int\limits^{t}_{0}\sigma
d\eta=\int\limits^{t}_{0}dY=Y(t)=\overline{Y}(t,w,w^{*}). 
$$

Thus,
\begin{align*}
\overline{X}(t,w,w^{*}) &=
x_{0}+\int\limits^{t}_{0}\sigma(s,\overline{X}(s,w,w^{*})d\eta+\\ 
&\qq +\int\limits^{t}_{0}b(s,\overline{X}(s,w,w^{*})ds 
\end{align*}
with $\overline{X}(t,w,w^{*})=X(t,w)$. Now
$$
(P\times P^{*})\overline{X}^{-1}(A)=(P\times P^{*})(A\times
\Omega^{*})=P(A). 
$$

Therefore
$$
(P\times P^{*})\overline{X}^{-1}=P\q\text{or}\q Q\overline{X}^{-1}=P.
$$

\noindent
{\bf Case (iv).}~ (General Case). Define
$$
Y(t,\cdot)=X(t,\cdot)-x_{0}-\int\limits^{t}_{0}b(s,X(s,\cdot))ds.
$$

Therefore $Y\in I[0,a(s,X(s,\cdot))]$ relative to
$(\Omega,\mathscr{F}_{t},P)$. Let $Z$ be the\pageoriginale
$n$-dimensional Brownian motion on
$(\Omega^{*},\mathscr{F}^{*}_{t},P^{*})$ where
$$
\Omega^{*}=C([0,\infty);\mathbb{R}_{n}).
$$ 
$$
(Y,Z)\q I\left[(0,0);
\begin{bmatrix}
a(s,X_{s}), & 0\\
0 & I
\end{bmatrix}
\right]
$$

Let $\sigma$ be a $d\times n$ matrix such that $\sigma\sigma^{*}=a$ on
$[0,\infty)\times \mathbb{R}^{d}$. Let
\begin{align*}
\eta(t,w,w^{*}) &=
\int\limits^{t}_{0}\sigma^{-1}(s,X(s,w))dY(s,w)+\int\limits^{t}_{0}r_{N_{\sigma}}(s,Z(s,w^{*}))dZ(s,w^{*})\\
&= \int\limits^{t}_{0}\langle
(\sigma^{-1}(s,X(s,w)),\pi_{N_{\sigma}}(s,Z(s,w^{*}))), d(Y,Z)\rangle.
\end{align*}

Therefore $\eta$ is an Ito process with parameters zero and
\begin{align*}
A &= (\sigma^{-1},\pi_N)
\begin{pmatrix}
a & 0\\
0 & I
\end{pmatrix}
\begin{pmatrix}
\sigma^{-1^{*}}\\
\pi_{N^{*}_{\sigma}}
\end{pmatrix}\\
&=
\sigma^{-1}a(\sigma^{-1})^{*}+\pi_{N_{\sigma}}\pi_{N^{*}_{\sigma}}.\\
&= \pi_{N_{\sigma}}+\pi_{N_{\sigma}}\q\text{(for any projection
  $PP^{*}=PP=P$)}\\
&= I_{\mathbb{R}^{n}}.
\end{align*}

Therefore $\eta$ is $n$-dimensional Brownian motion on
\begin{align*}
&
  (\overline{\Omega},\overline{\mathscr{F}_{t}},\overline{P})=(\Omega\times
    \Omega^{*},\mathscr{F}_{t}\times \mathscr{F}^{*}_{t},P\times
    P^{*}).\\
&\qq \int\limits^{t}_{0}\sigma(s,X(s,w))d\eta(s,w,w^{*})\\
&=
    \int\limits^{t}_{0}\sigma(s,\overline{X}(s,w,w^{*}))d\eta(s,w,w^{*}),\text{~
      where~ }\overline{X}(s,w,w^{*})=X(s,w),\\
&=
    \int\limits^{t}_{0}\sigma\sigma^{-1}dY+\int\limits^{t}_{0}\sigma\pi_{N_{\sigma}}dZ.\\
&= \int\limits^{t}_{0}\pi_{R_{\sigma}}dY,\text{~ since~ }
    \sigma\sigma^{-1}=\pi_{R_{\sigma}}\text{~ and~ }
    \sigma\pi_{N_{\sigma}}=0,\\
&= \int\limits^{t}_{0}(1-\pi_{R_{\sigma}})dY.
\end{align*}\pageoriginale
\end{proof}

\begin{claim*}
$\int\limits^{t}_{0}\pi_{R_{\sigma}}dY=0$.

For
\begin{align*}
E\left[\left(\int\limits^{t}_{0}\pi_{R_{\sigma}}dY\right)^{2}\right]
&=
\int\limits^{t}_{0}a\pi_{R_{\sigma}}ds=\int\limits^{t}_{0}\sigma\sigma^{*}\pi_{R_{\sigma}}ds\\
&= \int\limits^{t}_{0}\sigma(0)ds=0.
\end{align*}

Therefore we obtain
$$
\int\limits^{t}_{0}\sigma(s,X(s,w))d\eta(s,w,w^{*})=\int\limits^{t}_{0}dY=Y(t)-Y(0)=Y(t) 
$$
putting $\overline{Y}(t,w,w^{*})=Y(t,w)$, one gets
\begin{align*}
\overline{X}(t,w,w^{*}) &=
x_{0}+\int\limits^{t}_{0}\sigma(s,\overline{X}(s,w,w^{*}))d\eta(s,w,w^{*})\\
&\qq +\int\limits^{t}_{0}b(s,\overline{X}(s,w,w^{*}))ds.
\end{align*}

As in Case (iii) one shows easily that
$$
(P\times P^{*})\overline{X}^{-1}=P.
$$

This completes the proof of the theorem.
\end{claim*}

\begin{coro*}
Let $a:[0,\infty)\times \mathbb{R}^{d}\to S^{+}_{d}$, and
  $b:[0,\infty)\times \mathbb{R}^{d}\to \mathbb{R}^{d}$ be bounded,
    progressively measurable functions. If for some choice of a
    Lipschitz function $\sigma:[0,\infty)\times \mathbb{R}^{d}\to
      M_{d\times n}$, $\sigma\sigma^{*}=a$ then the It\^o process
      corresponding to $[b,a)$ is unique.
\end{coro*}

To\pageoriginale  state the result precisely, let $P_{1}$ and $P_{2}$
be two probability measures on $C([0,\infty);\mathbb{R}^{d})$ such
  that $X(t,w)=w(t)$ is an It\^o process with parameters $b$ and
  $a$. Then $P_{1}=P_{2}$.

\begin{proof}
By the theorem, there exists a generalised $n$-dimensional Brownian
motion $\beta_{i}$ on $(\Omega_{i},\sum^{i}_{t},Q_{i})$ and a map
$\xi_{i}:\Omega_{i}\to \Omega$ satisfying (for $i=1,2$)
$$
i^{(t,w)}=x_{0}+\int\limits^{t}_{0}\sigma(s,\xi_{i}(s,w))d\beta_{i}(s,w)+\int\limits^{t}_{0}b(s,\xi_{i}(s,w))ds. 
$$
and $P_{i}=Q_{i}\xi^{-1}_{i}$.

Now $\sigma$ is Lipschitz so that $\xi_{i}$ is unique but we know that
the iterations converge to a solution. As the solution is unique the
iterations converge to $\xi_{i}$. Each iteration is progressively
measurable with respect to
$$
{}^{i}_{t}=\sigma\{\beta_{i}(s);0\leq s\leq t\}\text{~ so that~
}\xi_{i}\text{~ is also progressively}
$$
measurable with respect to $\mathscr{F}^{i}_{t}$. Thus we can restate
the result as follows: There exists
$(\Omega_{i},\mathscr{F}^{i}_{t},Q_{i})$ and a map
$\xi_{i}:\Omega_{i}\to \Omega$ satisfying 
\begin{align*}
\xi_{i}(t,w) &=
x_{0}+\int\limits^{t}_{0}\sigma(s,\xi_{i}(s,w))d\beta_{i}(s,w)\\
&\qq +\int\limits^{t}_{0}b(s,\xi_{i}(s,w))ds,
\end{align*}
and $P_{i}=Q_{i}\xi^{-1}_{i}$.

$(\Omega_{i},\mathscr{F}^{i}_{t},Q_{i},\beta_{i})$ can be identified
with the standard Brownian motion
$(\Omega^{*},\mathscr{F}^{*}_{t},Q,\beta)$. Thus
$P_{1}=Q\xi^{-1}=P_{2}$, completing the proof.
\end{proof}












