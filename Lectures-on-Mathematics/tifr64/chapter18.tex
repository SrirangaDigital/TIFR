\chapter{The Feynman-Kac Formula}\label{chap18}

WE\pageoriginale NOW CONSIDER the modified heat equation
\begin{equation*}
\frac{\p u}{\p t}+\frac{1}{2}\Delta u+V(x)u(t,x)=0,\q 0\leq t\leq T,\tag{*}
\end{equation*}
where $u(T,x)=f(x)$. The Feynman-Kac formula says that the solution
for $s\leq T$ is given by
\begin{equation*}
u(s,x)=E_{s,x}(e^{\int^{T}_{s}V(X(s))dS}f(X(T))).\tag{**}
\end{equation*}

Observe that the solution at time $s$ depends on the expectation with
respect to the process starting at time $s$.

\begin{note*}
(**) is to be understood in the following sense. If (*) admits a
  smooth solution then it must be given by (**). We shall not go into
  the conditions under which the solution exists. Let
$$
Z(t,\cdot)=u(t,X(t,\cdot))e^{\int^{t}_{s}V(X(\sigma,\cdot)d\sigma)},\q
t\geq s.
$$

By Ito's formula (see Example 2 of section
\ref{chap16}), we get 
$$
Z(t,\cdot)=Z(s,\cdot)+\int\limits^{t}_{s}e^{\int^{t}_{s}V(X(\sigma,\cdot)d\sigma)}
\langle \nabla u(\lambda,X(\lambda)),dX(\lambda)\rangle,
$$
provided that $u$ satisfies (*). Assume tentatively that $V$ and
$\nabla u$ are bounded and progressively measurable. Then $Z(t,\cdot)$
is a martingale. Therefore
$$
E_{s,x}(Z(T,\cdot))=E_{z,x}(Z(s,\cdot)),
$$
or
$$
E_{s,x}(u(T,X(T)))e^{\int^{T}_{s}V(X(\sigma,\cdot))d\sigma}=u(s,x).
$$

This proves the result.
\end{note*}


We\pageoriginale shall now remove the condition $\nabla_{u}$ is
bounded and prove the uniqueness of the solution corresponding to (*)
under the assumption that $V$ is bounded above and
$$
|u(t,x)|\leq e^{A+|x|^{\alpha}},\q \alpha<2,\q \text{on}\q [s,T).
$$

In particular, the Feynman-Kac formula extends the uniqueness theorem
for the heat equation to the class of unbounded functions satisfying a
growth condition of the form given above.

Let $\phi$ be a $C^{\infty}$ function such that $\phi=1$ on $|X|\leq
R$, and $\phi=0$ outside $|x|>R+1$. Put $u_{R}(t,x)=u(t,x)\phi$,
$$
Z_{R}(t,x)=u_{R}(t,x)e^{\int^{t}_{s}V(X(\sigma)d\sigma)}.
$$

By what we have already proved, $Z_{R}(t,\cdot)$ is a martingale. Let
$$
\tau_{R}(\omega)=\inf \{t:t\geq s\omega(t)\in S(0;R)=\{|x|\leq R\}\}.
$$

Then $Z_{R}(t\wedge \tau_{R},\cdot))$ is also a martingale, i.e.
$$
u_{R}(t\wedge \tau_{R},X(t\wedge \tau_{R},\cdot))e^{\int^{t\wedge
    \tau_{R}}_{s}V(X(\sigma))d\sigma}
$$
is a martingale. Equating the expectations at time $t=s$ and time
$t=T$ and using the fact that
$$
u_{R}(t\wedge \tau_{R},X(t\wedge
\tau_{R},\cdot))=u(t\wedge\tau_{R},X(t\wedge \tau_{R},\cdot))),
$$
we conclude that
\begin{align*}
u(s,x) &= E_{s,x}[u(\tau_{R}\wedge T,X(\tau_{R}\wedge
  T,\cdot))e^{\int^{\tau_{R}\wedge T}_{s}V(X(s))ds}]\\
&= E_{s,x}[X_{(\tau_{R}\wedge T)}f(X(T))e^{\int^{T}_{s}V(X(s))ds}]+\\
&\q +E_{s,x}[X_{(\tau_{R}\leq T)}u(\tau_{R},X_{(\tau_{R})}e^{\int\limits^{\tau_{R}}_{s}V(X(s))ds}]
\end{align*}\pageoriginale

Consider the second term on the right:
\begin{gather*}
|E_{s,x}[X_{(\tau_{R}\leq
    T)}u(\tau_{R},X(\tau_{R}))e^{\int^{\tau_{R}}_{s}V(X(s))ds}]|\\
\leq A'e^{R^{\alpha}}P[{}_{R}\leq T]
\end{gather*}
(where $A'$ is a constant given in terms of the constants $A$ and $T$
and the bound of $V$)
$$
=A'e^{R^{\alpha}}P[\sup\limits_{s\leq \sigma\leq T}|X(\sigma)|\geq R].
$$
$P[\sup\limits_{s\leq \sigma\leq T}|X(\sigma)|\geq R]$ is of the order
of $e^{-c(T)R^{2}}$ and since $\alpha<2$, the second term on the right
side above tends to $0$ as $R\to \infty$. Hence, on letting $R\to
+\infty$, we get, by the bounded convergence theorem,
$$
u(s,x)=E_{s,x}[f(X(T))e^{\int^{T}_{s}V(X(s))ds}]
$$

\noindent
{\bf Application.}~ Let $\beta(t,\cdot)$ be a one-dimensional Brownian
motion. Recall (Cf.\@ Reflection principle) that
$P\{\sup\limits_{0\leq s\leq t}|\beta(s)|\leq 1\}$ is of the order of
$\dfrac{4}{\pi}e-\dfrac{\pi^{2}t}{8}$. The Feynman-Kac formula will be
used to explain the occurance of the factor $\dfrac{\pi^{2}}{8}$ in
the exponent. First observe that
$\dfrac{\pi^{2}}{8}=\dfrac{\lambda^{2}}{2}$ where $\lambda$ is the
first positive root of $\Cos \lambda=0$. Let
$$
\tau(w)=\inf \{t:|\beta(t)|\geq 1\}.
$$

Then 
$$
P\{\sup\limits_{0\leq s\leq t}|\beta(s,\cdot)|\leq 1\}=P\{\tau\geq
t\}.
$$

Let\pageoriginale  $\phi(x)=E_{x}[e^{\lambda\tau}]$, $\lambda<0$. We
claim that $\phi$ satisfies
\begin{equation*}
\begin{split}
& \frac{1}{2}\phi''+\lambda\phi=0,\q |x|<1,\\
& \phi=1,\q |x|=1.
\end{split}\tag{*}
\end{equation*}
 
Assume $\phi$ to be sufficiently smooth. Using It\^o's formula we get
$$
d(e^{\lambda t}\phi(\beta(t))=e^{\lambda
  t}\phi'(\beta(t))d\beta(t)+[\lambda\phi(\beta(t))+\frac{1}{2}\phi''(\beta(t))]e^{\lambda t}dt.
$$

Therefore
\begin{align*}
& e^{\lambda
  t}\phi(\beta(t))-\phi(\beta(0))=\int\limits^{t}_{0}e^{\lambda
  s}\phi'(\beta(s))d\beta(s)+\\
&\qq
  +\int\limits^{t}_{0}[\lambda\phi(\beta(s))+\frac{1}{2}\phi''(\beta(s))]e^{\lambda s}ds,
\end{align*}
i.e.
$$
e^{\lambda
  t}\phi(\beta(t))-\phi(\beta(0))-\int\limits^{t}_{0}[\lambda\phi(\beta(s))+\frac{1}{2}\phi''(\beta(s))]e^{\lambda
  s}ds
$$
is a martingale. By Doob's optional sampling theorem we can stop this
martingale at time $\tau$, i.e.
$$
e^{\lambda_{(t\wedge \tau)}}\phi(\beta(t\wedge
\tau))-\phi(\beta(0))-\int\limits^{t\wedge
  \tau}_{0}[\lambda\phi(\beta(s))+\frac{1}{2}\phi''(\beta(s))]e^{\lambda
  s}ds
$$
is also a martingale. But for $s\leq t\wedge \tau$,
$$
\lambda \phi+\frac{1}{2}\phi''=0.
$$

Thus we conclude that $\phi(\beta(t\wedge \tau))e^{\lambda(\tau\wedge
  t)}$ is a martingale. Since $\lambda<0$ and $\phi(\beta(t\wedge
\tau))$ is bounded, this martingale is uniformly integrable. Therefore
equating the expectation at $t=0$ and $t=\infty$ we get (since
$\phi(\beta(\tau))=1$) 
$$
\phi(x)=E_{x}[e^{\lambda \tau}].
$$

By\pageoriginale uniqueness property this must be the
solution. However (*) has a solution given by
$$
(x)=\frac{\Cos (\surd(2\lambda x))}{\Cos (\surd(2\lambda))}.
$$

Therefore
\begin{equation*}
E_{0}[e^{\lambda \tau}]=\frac{1}{\Cos(\surd(2\lambda))}(\lambda<0),\tag{1}
\end{equation*}

If $F(t)=P(\tau\geq t)$, then
$$
\int\limits^{\infty}_{0}e^{\lambda t}dF(t)=E_{0}(e^{\lambda \tau}).
$$

A theorem on Laplace transforms now tells us that (1) is valid till we
cross the first singularity of $\dfrac{1}{\Cos
  (\surd(2\lambda))}$. This occurs at $\lambda=\dfrac{\pi^{2}}{8}$. By
the monotone convergence theorem
$$
E_{0}[e^{\tau\pi^{2}/8}]=+\infty
$$

Hence $\int\limits^{\infty}_{0}e^{\lambda t}dF(t)$ converges for
$\lambda<\dfrac{\pi^{2}}{8}$ and diverges for $\lambda\geq
\dfrac{\pi^{2}}{8}$. Thus $\dfrac{\pi^{2}}{8}$ is the supremum of
$\lambda$ for which $\int\limits^{\infty}_{0}e^{\lambda t}dF(t)$
converges, i.e.\@ sup $[\lambda:E_{0}(e^{\lambda \tau})]$ exists,
i.e.\@ the decay rate is connected to the existence or the non
existence of the solution of the system (*). This is a general feature
and prevails even in higher dimensions.

