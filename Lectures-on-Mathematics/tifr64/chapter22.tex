\chapter{Large Deviations}\label{chap22}

LET\pageoriginale $P_{\epsilon}$ BE THE Brownian motion starting from
zero scaled to Brownian motion corresponding to the operator
$\epsilon\dfrac{\Delta}{2}$. More precisely, let
$$
P_{\epsilon}(A)=P\left(\frac{A}{\sqrt{\epsilon}}\right)
$$
where $P$ is the Brownian motion starting at time $0$ at the point
$0$.

\begin{interpretation}\label{chap22-inter1}
Let $\{X_{t}:t\geq 0\}$ be Brownian motion with $X(0)=x$. Let
$Y(t)=X(\epsilon t)$, $\forall t\geq 0$. Then $P_{\epsilon}$ is the
measure induced by the process $Y(t)$. This amounts to stretching the
time or scaling time.
\end{interpretation}

\begin{interpretation}\label{chap22-inter2}
Let $Y(t,\cdot)=\surd\epsilon X(t,\cdot)$. In this case also
$P_{\epsilon}$ is the measure induced by the process
$Y(t,\cdot)$. This amounts to `looking at the process from a distance'
or scaling the length.
\end{interpretation}

\begin{exer*}
Make the interpretations given above precise.

\noindent
(Hint: Calculate (i) the probability that $X(\epsilon t)\in A$, and
(ii) the probability that $\surd\epsilon X(t,)\in A$).
\end{exer*}

\begin{problem*}
Let
$$
I(w)=\frac{1}{2}\int\limits^{1}_{0}|\dot{w}(t)|^{2}dt
$$
if $w(0)=0$, $w$ absolutely continuous on $[0,1]$. Put $I(w)=\infty$
otherwise. We would like to evaluate
$$
\int\limits_{\Omega}e^{\frac{F(w)}{\epsilon}}dP_{\epsilon}(w)
$$
for\pageoriginale small values of $\epsilon$. Here $F(w):C[0,1]\to
\mathbb{R}$ is 
assumed to the a bounded and continuous function.
\end{problem*}


\begin{theorem*}
Let $C$ be any closed set in $C[0,1]$ and let $G$ be any open set in
$C[0,1]$. Then
\begin{align*}
&\mathop{\lim\sup}\limits_{\epsilon\to 0}\in\log P_{\epsilon}(C)\leq
  -\int\limits_{w\in C}I(w),\\
&\q \mathop{\lim\inf}_{\epsilon\to 0}\epsilon \log P_{\epsilon}(G)\geq
  -\inf\limits_{w\in G}I(w).
\end{align*}

Here $P_{\epsilon}(G)=P_{\epsilon}(\pi^{-1}G)$ where
$\pi:C[0,\infty)\to C[0,1]$ is the canonical projection.
\end{theorem*}

\noindent
\textbf{\textit{Significance of the theorem }}.~ If
\begin{enumerate}
\item
$$
dP_{\epsilon}=e^{\frac{-I(w)}{\epsilon}},
$$
then
$$
P_{\epsilon(A)}=\int\limits_{A}e^{\frac{-I(w)}{\epsilon}}dP_{\epsilon}
$$
is asymptotically equivalent to
$$
\exp[-\frac{1}{\epsilon}\inf\limits_{w\in A}I(w)].
$$

\item If $A$ is any set such that
$$
\inf\limits_{w\in A^{0}}I(w)=\inf\limits_{w\in \overline{A}}I(w),
$$
then by the theorem
$$
\Lt\limits_{\epsilon\to 0}\log P_{\epsilon}(A)=\inf\limits_{w\in
  A}I(w).
$$
\end{enumerate}

\noindent
{\bf Proof of the theorem.}\pageoriginale

\setcounter{lemma}{0}
\begin{lemma}\label{chap22-lem1}
Let $w_{0}\in \Omega$ with $I(w_{0})=\ell<\infty$. If
$S=S(w_{0};\delta)$ is any sphere of radius $\delta$ with centre at
$w_{0}$ then $\varliminf\limits_{\epsilon\to 0}\in \log
P_{\epsilon}(S)\geq -I(w_{0})$. 
\end{lemma}

\begin{proof}
\begin{align*}
& P(S)=\int \chi^{(w)}_{S(w_{0},\delta)}dP_{\epsilon}\\
&\qq =\int\chi_{S(0;)}^{(w-w_{0})}dP_{\epsilon}\\
&= \int \chi_{S(0;\delta)}(\lambda
  w)dP\left(\frac{w}{\sqrt{\epsilon}}\right),\quad\text{where}\quad
  \lambda(w)=w-w_{0},\\
&= \int \chi_{S(0;\delta)}(\lambda(\surd\epsilon w))dP(w)\\
&= \int\chi_{S(0,\delta)}(\surd \epsilon w)\exp
  \left[\int\limits^{1}_{0}\langle w_{0},dX\rangle
    -\frac{1}{2}\int\limits^{1}_{0}|\dot{w}_{0}|^{2}d\sigma\right]dP(w)\\
&=\int\chi_{S(0;\delta)}(\surd \epsilon w)\exp
  \left[\int\limits^{1}_{0}\langle \dot{w}_{0},dX\rangle
    -I(w_{0})\right]dP(w)\\
&= \int\chi_{S(0;\delta)}(w)\exp
  \left[-\frac{1}{\epsilon}\int\limits^{1}_{0}\langle
    \dot{w}_{0},dX\rangle
    -\frac{1}{\epsilon}I(w_{0})\right]dP_{\epsilon}(w)\\
&=
  e^{\frac{-I(w_{0})}{\epsilon}}P_{\epsilon}(S(0;\delta))\frac{1}{P_{\epsilon}(S(0;\delta))}\int\limits_{S(0;\delta)}\exp\left[-\frac{1}{\epsilon}\int\limits^{1}_{0}\langle
    \dot{w}_{0},dX\rangle\right]dP_{\epsilon}\\
&\geq
  e^{\frac{-I(w_{0})}{\epsilon}}P_{\epsilon}(S(0;\delta))e\left[-\frac{1}{\epsilon}\frac{1}{P_{\epsilon}(S(0;\delta))}\int\limits_{S(0;\delta)}\int\limits^{1}_{0}\langle
    \dot{w}_{0},dX\rangle dP_{\epsilon}\right]\\
&\qq\qq \text{by Jensen's inequality,}\\
&=
  e^{\frac{-I(w_{0})}{\epsilon}}P_{\epsilon}(S(0,\delta))e^{0}(\text{use~
  } P_{\epsilon}(w)=P_{\epsilon}(-w)\text{~ if~ }w\in S(0;\delta))\\
&= e^{\frac{-I(w_{0})}{\epsilon}}P_{\epsilon}(S(0,\delta)).
\end{align*}\pageoriginale

Therefore
$$
P_{\epsilon}(S(w_{0};\delta))\geq
e^{\frac{-I(w_{0})}{\epsilon}}P(S(0;\frac{\delta}{\sqrt{\epsilon}}))
$$
or,
$$
\epsilon\log P_{\epsilon}(S(w_{0};\delta))\geq -I(w_{0})+\epsilon \log
P(S(0;\frac{\delta}{\sqrt{\epsilon}}));
$$
let $\epsilon\to 0$ to get the result. Note that the Lemma is
trivially satisfied if $I(w_{0})=+\infty$.
\end{proof}

\noindent
{\bf Proof of Part 2 of the theorem.}

Let $G$ be open, $w_{0}\in G$; then there exists $\delta>0$ with
$S(w_{0},\delta)\subset G$. By Lemma \ref{chap22-lem1}
$$
\varliminf\limits_{\epsilon\to 0}\in \log P_{\epsilon}(G)\geq
\varliminf\limits_{\epsilon\to 0}\in \log
P_{\epsilon}(S(w_{0};\delta))\geq -I(w_{0}).
$$

Since $w_{0}$ is arbitrary, we get
$$
\varliminf \in \log P_{\epsilon}(G)\geq -\inf \{I(w_{0}):w_{0}\in G\}.
$$

For part 1 we need some more preliminaries.

\begin{lemma}\label{chap22-lem2}
Let $(w_{n})\in C[0,1]$ be such that $w_{n}\to w$ uniformly on
$[0,1]$, $I(w_{n})\leq \alpha< \infty$. Then $I(w)<\alpha$, i.e. $I$
is lower semi-continuous.
\end{lemma}

\begin{proof}
\setcounter{step}{0}
\begin{step}%1
$w$ is absolutely continuous. Let $\{(x'_{i},x''_{i})\}^{n}_{i=1}$ be
  a collection of mutually disjoint intervals in $[0,1]$. Then
\begin{align*}
&\sum^{n}_{i=1}|w_{m}(x'_{i})-w_{m}(x''_{i})|\leq
\sum\limits^{n}_{i=1|}|x''_{i}-x'_{i}|^{1/2}[\int\limits^{x''_{i}}_{x'_{i}}|w_{m}|^{2}]^{1/2}\\
&\hspace{3cm} \text{(by H\"older's inequality)}\\
&\q \leq
\left(\sum\limits^{n}_{i=1}\int\limits^{x''_{i}}_{x'_{i}}|w_{m}|^{2}\right)^{1/2}\left(\sum\limits^{n}_{i=1}|x''_{i}-x'_{i}|\right)^{1/2}\q\text{(again
  by H\"older)}\\
&\q \leq \surd(2\alpha)(\sum |x''_{i}-x'_{i}|)^{1/2}.
\end{align*}\pageoriginale

Letting $m\to \infty$ we get the result.
\end{step}

\begin{step}%2
Observe that $w_{m}(0)=0S_{0}w(0)=0$. Therefore
\begin{gather*}
|\frac{w_{n}(x+h)-w_{n}(x)}{h}|^{2}=|\frac{1}{h}\int\limits^{x+h}_{x}w_{n}dt|^{2}\leq
\frac{1}{h^{2}}\left(\int\limits^{x+h}_{x}|w_{n}|dt\right)^{2}\\
\leq \frac{1}{h}\int\limits^{x+h}_{x}|w_{n}|^{2}dt.
\end{gather*}

Hence
\begin{align*}
\int\limits^{1-h}_{0}|\frac{w_{n}(x+h)-w_{n}(x)}{h}|^{2}dx
&\leq
\frac{1}{h}\int\limits^{1-h}_{0}dx\int\limits^{h}_{0}|(\dot{w}_{n}(x+t))|^{2}dt\\ 
&\leq
\frac{1}{h}\int\limits^{h}_{0}dt\int\limits^{1-h}_{0}|\dot{w}_{n}(x+t)|^{2}dx\\ 
&\leq \frac{1}{2}2\int\limits^{h}_{0}dt=2\alpha
\end{align*}
letting $n\to \infty$, we get
$$
\int\limits^{1-h}_{0}|\frac{w(x+h)-w(x)}{h}|^{2}dx\leq 2\alpha.
$$

Let $h\to 0$ to get $I(w)\leq \alpha$, completing the proof.
\end{step}
\end{proof}

\begin{lemma}\label{chap22-lem3}
Let\pageoriginale $C$ be closed and put
$C^{\delta}=\bigcup\limits_{w\in C}S(w;\delta)$; then
$$
\lim\limits_{\delta\to 0}(\inf\limits_{w\in
  C^{\delta}}I(w))=\inf\limits_{w\in C}I(w).
$$
\end{lemma}

\begin{proof}
If $\delta_{1}<\delta_{2}$, then $C^{\delta_{1}}\subset
C^{\delta_{2}}$ so that $\inf\limits_{w\in C^{\delta}}I(w)$ is
decreasing. As $C^{\delta}\supset C$ for each $\delta$,
$$
\lim\limits_{\delta\to 0}(\inf\limits_{w\in C^{\delta}}I(w))\leq
\inf\limits_{w\in C}I(w)
$$

Let $\ell=\lim\limits_{\delta\to 0}(\inf\limits_{w\in
  C^{\delta}}I(w))$. Then there exists $w_{\delta}\in C^{\delta}$ such
that $I(w_{\delta})\to \ell$, and therefore $(I(w_{\delta}))$ is a
bounded set bounded by $\alpha$ (say).


\begin{claim*}
\begin{align*}
|w_{\delta}(t_{1})-w_{\delta}(t_{2})| &=
|\int\limits^{t_{2}}_{t_{1}}w_{\delta}dt|\leq \surd
|(|t_{1}-t_{2}|)(\int|w_{\delta}|^{2})^{1/2}\\
&\leq \surd (2\alpha|t_{1}-t_{2}|).
\end{align*}

The family $(w_{\delta})$ is therefore equicontinuous which, in view
of the fact that $w_{\delta}(0)=0$, implies that it is uniformly
bounded and the claim follows from Ascoli's theorem. Hence every
subfamily of $(w_{\delta})$ is equi\-continuous. By Ascoli's theorem
there exists a sequence $\delta_{n}\to 0$ such that $w\delta_{n}\to w$
uniformly on $[0,1]$. It is clear that $w\in C$. By lower
semicontinuity of $I(w)$,
$$
\lim\limits_{\delta\to 0}\inf\limits_{w\in C^{\delta}}I(w)\geq
\inf\limits_{w\in C}
$$
completing the proof.
\end{claim*}
\end{proof}

\noindent
{\bf Proof of Part 1 of the theorem.}~ Let\pageoriginale $X$ be continuous in
$[0,1]$. For each $n$ let $X_{n}$ be a piecewise linear version of $X$
based on $n$ equal intervals, i.e.\@ $X_{n}$ is a polygonal function
joining the points $(0,X(0))$, $(1/n,X(1/n)),\ldots,\break (1,X(1))$.
\begin{gather*}
P_{\epsilon}(||X_{n}-X||\geq \delta), (||\cdot||=||\cdot||_{\infty})\\
\leq P\left(\bigcup\limits_{n}\sup\limits_{i\leq j\leq n}\cdot
\sup\limits_{\frac{j-1}{n}\leq t\leq
  \frac{j}{n}}|X_{r}(t)-X_{r}\left(\frac{j-1}{n}\right)|\geq
\frac{\delta}{2\surd d}\right),\\
\text{where}\q X=(X_{1},\ldots,X_{d}).
\end{gather*}
$\leq nd P_{\epsilon}(\sup\limits_{0\leq t\leq 1/n}|X(t)-X(0)|\geq
\dfrac{\delta}{2\surd d}$ (Markov property; here $X$ is one-dimensional).
\begin{align*}
&\leq nd P_{\epsilon}\left(\sup\limits_{0\leq t\leq 1/n}|X_{t}|\geq
  \frac{\delta}{2\surd d}\right)\q(\text{since~ } X(0)=0)\\
&\leq 2nd\ P_{\epsilon}(\sup\limits_{0\leq t\leq 1/n}X_{t}\geq
  \frac{\delta}{2\surd d})\\
&=2dn\ P(\sup\limits_{0\leq t\leq 1/n}X_{t}\geq \frac{\delta}{2\surd
    \epsilon d})\\
&= 4dn\ P(X(1/n)\geq \frac{\delta}{2\surd\epsilon d})\q\text{(by the
    reflection principle)}\\
&=4dn\ \int\limits^{\infty}_{\delta\surd n/2\surd \epsilon
    d}\frac{1}{\surd 2\pi /n}e^{-ny^{2}/2}dy\\
&= 4d\ \int\limits^{\infty}_{\delta\surd n/2\surd \epsilon
    d}\frac{1}{\surd 2\pi}e^{-x^{2}/2}dx
\end{align*}

Now, for every $a>0$,
$$
a\int\limits^{\infty}_{a}e^{-x^{2}/2}dx\leq
\int\limits^{\infty}_{a}xe^{-x^{2}/2}dx=e^{-a^{2}/2}. 
$$

Thus\pageoriginale
$$
P_{\epsilon}(||X_{n}-X||\geq \delta)\leq
\frac{4dn\ e^{-n\delta^{2}/(8\in d)}}{\delta\surd n/2\sqrt{\epsilon}
  d}=C_{1}(n)\frac{\surd \epsilon}{\surd\delta}e^{-n\delta^{2}/(8\in
  d)},
$$
where $C_{1}$ depends only on $n$. We have now
\begin{align*}
P_{\epsilon}(X_{n}\in C^{\delta}) &\leq P_{\epsilon}(I(X_{n})\geq
\ell_{\delta})\text{~ where~ }\ell_{\delta}=\inf \{I(w)w\in
C^{\delta}\}.\\
&=
P\left(\frac{1}{2}\sum\limits^{n-1}_{j=0}n|X\left(\frac{j+1}{n}\right)-X\left(\frac{j}{n}\right)|^{2}\geq
\ell_{\delta}\right)\\
&= P\left(Y^{2}_{1}+Y^{2}_{2}+\cdots+Y^{2}_{nd}\geq
\frac{2\ell_{\delta}}{\epsilon}\right), 
\end{align*}
where $Y_{1}=\surd n(X_{1}(1/n)-X_{1}(0))$ etc.\@ are {\em
  independent} normal random variables with mean $0$ and variance
$1$. Therefore,
\begin{align*}
& P(Y^{2}_{1}+\cdots+Y^{2}_{nd}\geq \frac{2\ell_{\delta}}{\epsilon})\\
&=\int\limits_{y^{2}_{1}+\cdots+y^{2}_{nd}}\frac{2\ell_{\delta}}{\epsilon}e^{-(y^{2}_{1}+\cdots+y^{2}_{nd})^{1/2}}dy_{1}\ldots
  dy_{nd}.\\
&=
  C(n)\int\limits^{\infty}_{\surd(2\ell_{\delta}/\epsilon)}e^{-r^{2}/2}r^{nd-1}dr, 
\end{align*}
using polar coordinates, i.e.
$$
P(Y^{2}_{1}+Y^{2}_{2}+\cdots+Y^{2}_{nd}\geq
\frac{2\ell_{\delta}}{\epsilon})=C'(n)\int\limits^{\infty}_{(\ell_{\delta}/\epsilon)}e^{-s}s^{\frac{nd}{2}-1}ds
$$
(change the variable from $r$ to $s=\dfrac{r^{2}}{2}$). An integration
by parts gives
$$
\int\limits^{\infty}_{\alpha}e^{-s}s^{k}ds=e^{-\alpha}(\alpha^{k}+\frac{k!}{(k-1)!}\alpha^{k-2}+\cdots). 
$$

Using this estimate (for $n$ even) we get
$$
P((Y^{2}_{1}+\cdots+Y^{2}_{nd})\geq
\frac{2\ell_{\delta}}{\epsilon})\leq
C_{2}(n)e^{-\ell_{\delta}/\epsilon}(\frac{\ell_{\delta}}{\epsilon})^{\frac{nd}{2}-1},
$$
where $C_{2}$ depends only on $n$. Thus,
\begin{align*}
P_{\epsilon}(C) &\leq P_{\epsilon}(||X_{n}-X||\geq
\delta)+P_{\epsilon}(X_{n}\not\in C^{\delta})\\
&\leq C_{1}(n)\surd\left(\frac{\epsilon}{\delta}\right)e^{-n\delta^{2}/(8\in
  d)}+C_{2}(n)e^{-\ell_{\delta}/\epsilon}\left(\frac{\ell_{\delta}}{\epsilon}
\right)^{\frac{nd}{2}-1}\\ 
&\leq 2\max [C_{1}(n)\surd
  \left(\frac{\epsilon}{\delta}\right)e^{-n\delta^{2}/(8\in
    d)},C_{2}(n)e^{-\ell_{\delta}/\epsilon}
\left(\frac{\ell_{\delta}}{\epsilon}\right)^{\frac{nd}{2}-1} \\
&\in \log P_{\epsilon}(C)\leq \epsilon \log 2+\epsilon
\max[\log(C_{1}(n)\surd\left(\frac{\epsilon}{\delta}\right)e^{-n\delta^{2}/(8\in
    d)}\\
&\qq \log
  C_{2}(n)e^{-\ell_{\delta}/\epsilon}(\frac{\ell_{\delta}}{\epsilon})^{\frac{nd}{2}-1}] 
\end{align*}\pageoriginale


Let $\epsilon\to 0$ to get
$$
\varlimsup \in \log P_{\epsilon}(C)\leq \max
\left\{\frac{-n\delta^{2}}{8d},\frac{-\ell_{\delta}}{1}\right\}.
$$

Fix $\delta$ and let $n\to \infty$ through even values to get
$$
\varlimsup \in \log P_{\epsilon}(C)\leq -\ell_{\delta}.
$$

Now let $\delta\to 0$ and use the previous lemma to get
$$
\varlimsup\limits_{\epsilon\to 0}\in \log P_{\epsilon}(C)\leq
-\int\limits_{w\in C}I(w).
$$

\begin{prop*}
Let $\ell$ be finite; then $\{w:I(w)\leq \ell\}$ is compact in $\Omega$.
\end{prop*}

\begin{proof}
Let $(w_{n})$ be any sequence, $I(w_{n})\leq \ell$. Then
$$
|w_{n}(t_{1})-w_{n}(t_{2})|\leq \surd(\ell|t_{1}-t_{2}|)
$$
and since $w_{n}(0)=0$, we conclude that $\{w_{n}\}$ is equicontinuous
and uniformly bounded.
\end{proof}

\noindent
{\bf Assumptions.}~ Let $\Omega$ be any separable metric space,
$\mathscr{F}=$ Borel $\sigma$-field on $\Omega$. For every
$\epsilon>0$ let $P_{\epsilon}$ be a probability measure. Let
$I:\Omega\to [0,\infty]$ be any function such that
\begin{itemize}
\item[(i)] $I$ is lower semi-continuous.

\item[(ii)] $\forall$\pageoriginale finite $\ell$, $\{w:I(w)\leq
  \ell\}$ is compact.

\item[(iii)] For every closed set $C$ in $\Omega$,
$$
\lim\limits_{\epsilon\to 0}\sup \in \log P_{\epsilon}(C)\leq
-\inf\limits_{w\in C}I(w).
$$

\item[(iv)] For every open set $G$ in $\Omega$
$$
\lim\limits_{\epsilon\to 0}\inf \in \log P_{\epsilon}(G)\geq
-\inf\limits_{w\in G}I(w).
$$
\end{itemize}

\begin{remark*}
Let $\Omega=C[0,1]$, $P_{\epsilon}$ the Brownian measure corresponding
to the scaling $\epsilon$. If
$I(w)=\dfrac{1}{2}\int\limits^{1}_{0}|w|^{2}dt$ if $w(0)=0$ and
$\infty$ otherwise, then all the above assumptions are satisfied.
\end{remark*}

\begin{theorem*}
Let $F:\Omega\to \mathbb{R}$ be bounded and continuous. Under the
above assumptions the following results hold.
\begin{enumerate}
\renewcommand{\theenumi}{\roman{enumi}}
\renewcommand{\labelenumi}{\rm(\theenumi)}
\item For every closed set $C$ in $\Omega$
$$
\lim\limits_{\epsilon\to 0}\sup \epsilon \log \int\limits_{C}\exp
\frac{F(w)}{\epsilon}dP_{\epsilon}\leq \sup\limits_{w\in C}(F(w)-I(w)).
$$ 

\item For every open set $G$ in $\Omega$
$$
\lim\limits_{\epsilon\to 0}\inf\in \log\int\limits_{G}\exp
\frac{F(w)}{\epsilon}dP_{\epsilon}\geq \sup\limits_{w\in
  G}(F(w)-I(w)).
$$
In particular, if $G=\Omega=C$, then
$$
\lim\limits_{\epsilon\to 0}\in \log \int\limits_{\Omega}\exp
\frac{F(w)}{\epsilon}dP_{\epsilon}=\sup\limits_{w\in
  \Omega}(F(w)-I(w)).
$$
\end{enumerate}
\end{theorem*}

\begin{proof}
Let $G$ be open, $w_{0}\in G$. Let $\delta\to 0$ be given. Then there
exists a neighbourhood $N$ of $w_{0}$, $F(w)\geq F(w_{0})-\delta$,
$\forall w$ in $N$. Therefore
$$
\int\limits_{G}\exp\frac{F(w)}{\epsilon}dP_{\epsilon}\geq
\int\limits_{N}\exp \frac{F(w)}{\epsilon}dP_{\epsilon}\geq
e^{\frac{F(w_{0})-\delta}{\epsilon}}P_{\epsilon}(N). 
$$\pageoriginale

Therefore
$$
\epsilon \log\int\limits_{G}\exp\frac{F(w)}{\epsilon}dP_{\epsilon}\geq
F(w_{0})-\delta+\epsilon \log P_{\epsilon}(N).
$$

Thus
\begin{gather*}
\varliminf \log \int\limits_{G}\exp
\frac{F(w)}{\epsilon}dP_{\epsilon}\geq F(w_{0})-\delta+\varliminf
\epsilon \log P_{\epsilon}(N).\\
\geq F(w_{0})-\delta-\inf\limits_{w\in N}I(w)\geq F(w_{0})-I(w_{0})-\delta.
\end{gather*}

Since $\delta$ and $w_{0}$ are arbitrary $(w_{0}\in G)$ we get
$$
\varliminf \in \log \int\limits_{G}\exp
\frac{F(w)}{\epsilon}dP_{\epsilon}\geq \sup\limits_{w\in
  G}(F(w)-I(w)).
$$

This proves Part (ii) of the theorem.
\end{proof}

\noindent
{\bf Proof of Part (i).}~ 
\setcounter{step}{0}
\begin{step}%1
Let $C$ be compact; $L=\sup\limits_{w\in G}(F(w)-I(w))$. If
$L=-\infty$ it follows easily that
$$
\lim\limits_{\epsilon\to 0}\sup \in
\log\int\limits_{C}e^{F/\epsilon}dP_{\epsilon}\leq -\infty.
$$
(Use the fact that $F$ is bounded). Thus without any loss, we may
assume $L$ to be finite. Let $w_{0}\in C$; then there exists a
neighbourhood $N$ of $w_{0}$ such that $F(w)\leq F(w_{0})+\delta$ and
by lower semi-continuity of $I$,
$$
I(w)\geq I(w_{0})-\delta,\ \forall w\in N(w_{0}).
$$

By regularity, there exists an open set $G_{w_{0}}$ containing $w_{0}$
such that $G_{w_{0}}\overline{G}_{w_{0}} N(w_{0})$.\pageoriginale
Therefore
$$
\int\limits_{\overline{G}_{w_{0}}}\exp
\frac{F(w)}{\epsilon}dP_{\epsilon}\exp
\left(\frac{F(w_{0})+\delta}{\epsilon}\right)P_{\epsilon}(\overline{G}_{w_{0}}). 
$$

Therefore
\begin{align*}
& \lim\limits_{\epsilon\to 0}\sup \in \log
\int\limits_{\overline{G}_{w_{0}}}\exp
\frac{F(w)}{\epsilon}dP_{\epsilon}\leq
F(w_{0})+\delta+\epsilon\varlimsup\limits_{\epsilon\to
  0}P_{\epsilon}(\overline{G}_{w_{0}})\\ 
&\qq \leq F(w_{0})+\delta-\inf\limits_{w\in
  \overline{G}_{w_{0}}}I(w)\\
&\qq F(w_{0})+\delta-I(w_{0})+\delta\\
&\qq \leq L+2\delta.
\end{align*}

Let $K_{\ell}=\{w:I(w)\leq \ell\}$. By assumption, $K_{\ell}$ is
compact. Therefore, for each $\delta>0$, there exists an open set
$G_{\delta}$ containing $K_{\ell}\cap C$ such that
$$
\lim\limits_{\epsilon\to 0}\sup \in
\log\int\limits_{G_{\delta}}e^{\frac{F(w)}{\epsilon}}dP_{\epsilon}\leq L+2\delta.
$$

Therefore
\begin{gather*}
\lim\limits_{\epsilon\to 0}\sup \in \log \int\limits_{G_{\delta}\cap
  C}e^{\frac{F(w)}{\epsilon}}dP_{\epsilon}\leq L+2\delta,\\
\int\limits_{G^{c}_{\delta}\cap C}e^{\frac{F(w)}{\epsilon}}dP_{\epsilon}\leq
e^{M/\epsilon}P(C\cap G^{c}_{\delta}).
\end{gather*}

Therefore
\begin{gather*}
\lim\limits_{\epsilon\to 0}\sup \in \log
\int\limits_{G^{c}_{\delta}\cap
  C}e^{\frac{F(w)}{\epsilon}}dP_{\epsilon}\leq
M+\lim\limits_{\epsilon\to 0}\sup \epsilon \log
P_{\epsilon}(C^{c}_{\delta}\cap C)\\
\leq M-\inf\limits_{w\in C\cap G^{c}_{\delta}}I(w).
\end{gather*}

Now\pageoriginale
$$
G^{c}_{\delta}\subset K^{c}_{\ell}\cap C^{c}.
$$

Therefore
$$
C\cap G^{c}_{\delta}\subset C\cap K^{c}_{\ell}
$$
if $w\in C\cap G^{c}_{\delta}$, $w\not\in K_{\ell}$. Therefore
$I(w)>\ell$. Thus
$$
\lim\limits_{\epsilon\to 0}\sup \in \log
\int\limits_{G^{c}_{\delta}\cap C}e^{F(w)/\epsilon}dP_{\epsilon}\leq
M-\ell \leq L\leq L+2\delta.
$$

This proves that
$$
\lim\limits_{\epsilon\to 0}\sup \in \log \int\limits_{C}\exp
\frac{F(w)}{\epsilon}dP_{\epsilon}\leq L+2\delta.
$$

Since $C$ is compact there exists a finite number of points
$w_{1},\ldots,w_{n}$ in $C$ such that
$$
C\subset \bigcup\limits^{n}_{i=1}G_{w_{i}}
$$

Therefore
\begin{gather*}
\varlimsup \in \log \int\limits_{C}\exp
\frac{F(w)}{\epsilon}dP_{\epsilon}\leq \varlimsup \epsilon \log
\int\limits_{\bigcup^{n}_{i=1}G_{w_{i}}}e^{F(w)/\epsilon}dP_{\epsilon}\\
\leq \varlimsup (\epsilon \log n\Max\limits_{1\leq i\leq
}\int\limits_{G_{w_{i}}}\exp \frac{F(w)}{\epsilon}dP_{\epsilon})\\
\leq L+2\delta.
\end{gather*}

Since $\delta$ is arbitrary.
$$
\varlimsup \in \log \int\limits_{C}\exp
\frac{F(w)}{\epsilon}dP_{\epsilon}\leq \sup\limits_{w\in
  C}(F(w)-I(w)).
$$
\end{step}

The\pageoriginale above proof shows that given a compact set $C$, and
$\delta>0$ there exists an open set $G$ containing $C$ such that
$$
\varlimsup \in \log\int\limits_{G}\exp
\frac{F(w)}{\epsilon}dP_{\epsilon}\leq L+2\delta.
$$

\begin{step}%2
Let $C$ be any arbitrary closed set in $\Omega$. Let
$$
L=\sup\limits_{w\in C}(F(w)-I(w)).
$$

Since $F$ is bounded there exists an $M$ such that $|F(w)|\leq M$ for
all $w$. Choose $\ell$ so large that $M-\ell\leq L$. Since $\delta$ is
arbitrary 
$$
\lim\limits_{\epsilon\to 0}\sup \in\log \int\limits_{C}\exp
\frac{F(w)}{\epsilon}dP_{\epsilon}\leq \sup\limits_{w\in C}(F(w)-I(w))
$$

We now prove the above theorem when $P_{\epsilon}$ is replaced by
$Q^{\epsilon}_{x}$. Let $P^{\epsilon}_{x}$ be the Brownian motion
starting at time $t=0$ at the space point $x$ corresponding to the
scaling $\epsilon$. Precisely stated, if
$$
\tau_{\epsilon}:C([0,\infty);\mathbb{R}^{d})\to
  C([0,\infty);\mathbb{R}^{d})
$$
is the map given by $(\tau_{\epsilon}w)(t)=w(\epsilon t)$, then
$P^{\epsilon}_{x}{\displaystyle{\mathop{=}_{\text{def}}}}P_{x}\tau^{-1}_{\epsilon}$. Note
$T''_{1}\tau_{\epsilon}=T_{\epsilon}$ and $T_{\epsilon}$ is given by
$$
T_{\epsilon}w=y\text{~ where~ } y(t)=w(\epsilon
t)+\int\limits^{t}_{0}b(y(s))ds. 
$$

Hence
$$
P_{x}T^{-1}_{\epsilon}=P_{x}(T_{1},\tau_{\epsilon})^{-1}=P_{x}\tau^{-1}_{\epsilon}T^{-1}_{1}=P^{\epsilon}_{x}T^{-1}_{1};
$$
either of these probability measures is denoted by $Q^{\epsilon}_{x}$.
\end{step}

\begin{theorem*}
Let $b:\mathbb{R}^{d}\to \mathbb{R}^{d}$ be bounded measurable and
locally Lipschitz. Define
$$
I(w):\frac{1}{2}\int\limits^{1}_{0}|X(t)-b(X(t))|^{2}dt
$$\pageoriginale

If $w\in C([0,\infty);\mathbb{R}^{d})$, $w(0)=x$ and $x$ absolutely
  continuous. Put $I(w)=\infty$ otherwise. If $C$ is closed in
  $C[(0,1];\mathbb{R}^{d})$, then
$$
\varlimsup\limits_{\epsilon\to 0}\in \log Q^{\epsilon}_{x}(C)\leq
-\inf\limits_{w\in C}I(w).
$$

If $G$ is open in $C([0,1];\mathbb{R}^{d})$, then
$$
\varliminf\limits_{\epsilon\to 0}\in \log Q^{\epsilon}_{x}(G)\geq
-\inf\limits_{w\in G}I(w).
$$

As usual $Q^{\epsilon}_{x}(C)=Q^{\epsilon}_{x}\pi^{-1}(C)$ where
$$
\pi:C([0,\infty); \mathbb{R}^{d})\to C([0,1];\mathbb{R}^{d})
$$ 
is the canonical projection. 
\end{theorem*}

\begin{remark*}
If $b=0$ we have the previous case.
\end{remark*}

\begin{proof}
Let $T$ be the map $x(\cdot)\to y(\cdot)$ where
$$
y(t)=x(t)+\int\limits^{t}_{0}b(y(s))ds.
$$

Then
$$
Q^{\epsilon}_{x}=P^{\epsilon}_{x}(T^{-1}).
$$

If $C$ is closed
$$
Q^{\epsilon}_{x}(C)=P^{\epsilon}_{x}(T^{-1}C).
$$

The map $T$ is continuous. Therefore $T^{-1}(C)$ is closed. Thus
\begin{equation*}
\begin{split}
& \lim\limits_{\epsilon\to 0}\sup \in \log
Q_{x}(C)=\lim\limits_{\epsilon\to 0}\sup \in \log
P^{\epsilon}_{x}(T^{-1}C)\\
& \leq -\inf\limits_{w\in
  T^{-1}(C)}\frac{1}{2}\int\limits^{1}_{0}|X|^{2}dt\q \text{(see
  Exercise \ref{chap22-exer1} below)}\\
&= -\inf\limits_{w\in C}\frac{1}{2}\int\limits^{1}_{0}|T^{-1}w|^{2}dt.
\end{split}\tag{*}
\end{equation*}

Now\pageoriginale
$$
y(\cdot)\xrightarrow{T^{-1}}y(t)-\int\limits^{t}_{0}b(y(s))ds.
$$

Therefore
$$
(T^{-1}y)=y-b(y(s)).
$$

Therefore
$$
\lim\limits_{\epsilon\to 0}\sup \in \log Q^{\epsilon}_{x}(C)\leq
-\inf\limits_{w\in C}I(w).
$$

The proof when $G$ is one is similar.
\end{proof}

\setcounter{exercise}{0}
\begin{exercise}\label{chap22-exer1}
Replace $P_{\epsilon}$ by $P^{\epsilon}_{x}$ and $I$ by $I_{x}$ where
\begin{align*}
I_{x}(w) &= \frac{1}{2}\int\limits^{1}_{0}|w|^{2},\ w(0)=x,w\text{~
  absolutely continuous},\\
&= \infty \text{~ otherwise.}
\end{align*}

Check that (*) holds, i.e.
$$
\lim\limits_{\epsilon\to 0}\sup \in \log P^{\epsilon}_{x}(C)\leq
-\inf\limits_{w\in C}I_{x}(w),\text{~ if~ } C\text{~ is closed},
$$
and
$$
\lim\limits_{\epsilon\to 0}\inf \in \log P^{\epsilon}_{x}(G)\geq
-\inf\limits_{w\in G}I_{x}(w).
$$

Let $G$ be a bounded open set in $\mathbb{R}^{n}$, with a smooth
boundary $\Gamma=\p G$. Let $b:\mathbb{R}^{d}\to \mathbb{R}^{d}$
be a smooth $C^{\infty}$ function such that
\begin{itemize}
\item[(i)] $\langle b(x),n(x)\rangle 0$, $\forall x\in \p G$ where
  $n(x)$ is the unit inward normal.

\item[(ii)] there exists a point $x_{0}\in G$ with $b(x_{0})=0$ and
  $|b(x)|>0$, $\forall x$ in $G-\{x_{0}\}$.

\item[(iii)] for\pageoriginale any $x$ in $G$ the solution
$$
\xi(t)=x+\int\limits^{t}_{0}b(\xi(s))ds,
$$
of the vector field starting from $x$ converges to $x_{0}$ as $t\to
+\infty$. 
\end{itemize}
\end{exercise}

\begin{remark*}
\begin{itemize}
\item[(a)] (iii) is usually interpreted by saying that ``$x_{0}$ is
  stable''.

\item[(b)] By (i) and (ii) every solution of (iii) takes all its
  values in $G$ and ultimately stays close to $x_{0}$.
\end{itemize}

Let $\epsilon>0$ be given; $f:\p G\to \mathbb{R}$ be any continuous
bounded function. Consider the system
\begin{gather*}
L_{\epsilon}u_{\epsilon}=\frac{1}{2}\Delta u_{\epsilon}+b(x)\cdot
\Delta u_{\epsilon}=0\text{~ in~ }G\\
u_{\epsilon}=f\text{~ on~ } \p G.
\end{gather*}

We want to study $\lim\limits_{\epsilon\to 0}u_{\epsilon}(x)$. Define
$$
I^{T}_{0}(X(t))=\frac{1}{2}\int\limits^{T}_{0}|X(t)-b(X(t))|^{2}dt;\ X:[0,T]\to
\mathbb{R}^{d} 
$$
whenever $X$ is absolutely continuous, $=\infty$ otherwise.
\end{remark*}

\begin{remark*}
Any solution of (iii) is called an integral curve. For any curve $X$
on $[0,T]$, $I^{T}_{0}$ gives a measure of the deviation of $X$ from
being an integral curve. Let
$$
V_{T}(x,y)=\inf\{I^{T}_{0}(X):X(0)=x;\ X(T)=y\}
$$
and 
$$
V(x,y)=\inf\{V_{T}(x,y):T>0\}.
$$
$V$ has the following properties.
\begin{itemize}
\item[(i)] $V(x,y)\leq V(x,z)+V(z,y)\;\forall x,y,z$.\pageoriginale

\item[(ii)] Given any $x$, $\exists \delta\to 0$ and $C>0$ such that
  for all $y$ with $|x-y|\leq \delta$.
$$
V(x,y)\leq C|x-y|
$$
\end{itemize}
\end{remark*}

\begin{proof}
Let $X(t)=\dfrac{t(y-x)}{|y-x|}+x$.

Put
\begin{gather*}
T=|y-x|,\ X(0)=x,\ X(T)=y,\\
I^{T}_{0}(X(t))=\frac{1}{2}\int\limits^{T}_{0}\left|\frac{y-x}{T}-b(X+\frac{S}{T}(y-x))\right|^{2}ds. 
\end{gather*}

Then
$$
I^{T}_{0}\leq
\frac{1}{2}\int\limits^{T}_{0}2\left(\frac{|y-x|^{2}}{T^{2}}+||b||^{2}_{\infty}\right)ds,
$$
where
$$
||b||_{\infty}=\sup\limits_{|\lambda-x|\leq |y-x|}b(\lambda),
$$
or,
$$
I^{T}_{0}\leq (1+||b||^{2}_{\infty})|y-x|.
$$

As a consequence of (ii) we conclude that
$$
V(x,y)\leq \left(1+\sup\limits_{|\lambda-x\leq |y-x|}|b(\lambda)|^{2}\right)|y-x|,
$$
i.e.\@ $V$ is locally Lipschitz.

The answer to the problem raised is given by the following.
\end{proof}

\begin{theorem*}
$$
\lim\limits_{\epsilon\to 0}u_{\epsilon}(x)=f(y_{0})
$$
where $y_{0}$ is assumed to be such that $y_{0}\in \p G$ and
$$
V(x_{0},y_{0})<V(x,y),\ \forall y\in \p G,\ y\neq y_{0}.
$$\pageoriginale
\end{theorem*}

We first proceed to get an equivalent statement of the theorem. Let
$P^{\epsilon}_{x}$ be the Brownian measure corresponding to the
starting point $x$, and corresponding to the scaling $\epsilon$. Then
there exists a probability measure $Q^{\epsilon}_{x}$ such that
$$
\frac{dQ^{\epsilon}_{x}}{dP^{\epsilon}_{x}}\Big|\mathscr{F}_{t}=Z(t)
$$
where
$$
Z(t,\cdot)=\exp \int\limits^{t}_{0}\langle b^{*}(X(s)),\ dX(s)\rangle
-\frac{1}{2}\int\limits^{t}_{0}b^{*}(X(s)) ds;
$$
$b^{*}$ is any bounded smooth function such that $b^{*}=b$ on
$G$. Further we have the integral representation
$$
u_{\epsilon}(x)=\int\limits_{\p G}f(X(\tau))dQ^{\epsilon}_{x}
$$
where $\tau$ is the exit time of $G$, i.e.
\begin{gather*}
\tau(w)=\inf \{t:w(t)\not\in G\}.\\
|u_{\epsilon}(x)-f(y_{0})|=|\int\limits_{\p
  G}(f(X(\tau))-f(y_{0}))dQ^{\epsilon}_{x}|\\
\leq |\int\limits_{N\cap \p
  G}(f(X(\tau))-f(Y_{0}))dQ^{\epsilon}_{x}|+\\
+|\int\limits_{N^{c}\cap \p
  G}(f(X(\tau))-f(Y_{0}))dQ^{\epsilon}_{x}|\\
\,\hspace{2cm} (N\text{~ is any neighbourhood of~ } y_{0}).\\
\leq Q^{\epsilon}_{x}(X(\tau)\in N\cap \p G)\sup\limits_{\lambda \in
  N\cap \p G}|f(\lambda)-f(y_{0})|+\\
+2||f||_{\infty}Q^{\epsilon}_{x}(X(\tau)\in N^{c}\cap \p G).
\end{gather*}\pageoriginale

Since $f$ is continuous, to prove the theorem it is sufficient to
prove the

\begin{theorem*}
$$
\lim\limits_{\epsilon\to 0}Q_{x}(X(\tau)\in N^{c}\cap \p G)=0
$$
for every neighbourhood $N$ of $y_{0}$.
\end{theorem*}

Let $N$ be any neighbourhood of $y_{0}$. Let
$$
V=V(x_{0},y_{0}), V'=\inf\limits_{y\in N^{c}\cap \p G}V(x,y).
$$

By definition of $y_{0}$ and the fact that $N^{c}\cap \p G$ is
compact, we conclude that $V'>V$. Choose $\eta=\eta(N)>0$ such that
$V'=V+\eta$. For any $\delta>0$ let
$D=S(x_{0};\delta)=\{y:|y-x_{0}|<\delta\}$, $\p
D=\{y:|y-x_{0}|=\delta\}$.

\begin{claim*}
We can choose a $\delta_{2}$ such that
\begin{itemize}
\item[(i)] $V(x,y)\geq V+\dfrac{3\eta}{4}$, $\forall x\in \p D_{2}$,
  $y\in N^{c}\ \p G$.

\item[(ii)] $V(x,y_{0})\leq V+\dfrac{\eta}{4}$, $\forall x\in \p D_{2}$.
\end{itemize}
\end{claim*}

\begin{proof}
\begin{itemize}
\item[\rm(i)] $V(x_{0},y)\geq V+\eta$, $\forall y\in N^{c}$ $\p
  G$. Therefore
\begin{align*}
& V+\eta\leq V(x_{0},y)\leq V(x_{0},x)+V(x,y)\\
& \leq C|x-x_{0}|+V(x,y).
\end{align*}

Choose $C$ such that $C|x-x_{0}|\leq \dfrac{\eta}{4}$. Thus
$$
V+\frac{3\eta}{4}\leq V(x,y)\text{~ if~ } C|x-x_{0}|\leq
\frac{\eta}{4},\ \forall y\in N^{c}\ \p G.
$$\pageoriginale
$C$ depends only on $x_{0}$. This proves (i).


\item[(ii)] $|V(x_{0},y_{0})-V(x,y_{0})|\leq V(x_{0},x)\leq
  C|x_{0}-x|\leq \dfrac{\eta}{4}$ if $x$ is close to $x_{0}$. Thus
$$
V(x,y_{0})\leq V(x_{0},y_{0})+\frac{\eta}{4}=V+\frac{\eta}{4}
$$
if $x$ is close to $x_{0}$. This can be achieved by choosing
$\delta_{2}$ very small.
\end{itemize}
\end{proof}

\begin{claim*}[(iii)]
We can choose $\delta_{1}<\delta_{2}$ such that for
  points $x_{1}$, $x_{2}$ in $\p D_{1}$ there is a path $X(\cdot)$
  joining $x_{1}$, $x_{2}$ with $X(\cdot)\in D_{2}-D_{1}$, i.e.\@ it
  never penetrates $D_{1}$; and
$$
I(X)\leq \frac{\eta}{8}.
$$
\end{claim*}

\begin{proof}
Let $C=\sup \{|b(\lambda)|^{2}:|\lambda-x_{0}|\leq
\delta_{2}\}$. Choose $X(\cdot)$ to be any path on $[0,T]$, taking
values in $D_{2}$ with $X(0)=x_{1}$; $X(T)=x_{2}$ and such that
$|X|=1$ (i.e. the path has unit speed). Then
\begin{align*}
& I^{T}_{0}(X)\leq \int\limits^{T}_{0}(|X|^{2}+C)dt\leq
  CT+\int\limits^{T}_{0}|X|dt\\
&\q =(c+1)T=(C+1)|x_{2}-x_{1}|.
\end{align*}

Choose $\delta_{1}$ small such that $(C+1)|x_{2}-x_{1}|\leq
\dfrac{\eta}{8}$.

Let $\Omega\delta_{1}=\{w:w(t)\in \overline{G}-D_{1},\forall t\geq
0\}$, i.e.\@ $\Omega\delta_{1}$ consists of all trajectories in
$\overline{G}$ that avoid $D_{1}$.
\end{proof}

\begin{claim*}[(iv)]
\begin{gather*}
\inf\limits_{T>0} \inf\limits_{X\in \Omega_{\delta_{1}},X(0)\in \p
  D_{2}}I^{T}_{0}(X(\cdot))\geq V+\dfrac{3\eta}{4}\\
X(T)\in N^{c}\cap \p G
\end{gather*}\pageoriginale
\end{claim*}

\begin{proof}
Follows from Claim (i) and (ii).
\end{proof}

\begin{claim*}[(v)]
\begin{gather*}
\inf\limits_{T>0}\inf\limits_{\substack{X\in \Omega_{\delta_{1}},X(0)\in \p
  D_{2}\\ X(T)=y_{0}}}I^{T}_{0}(X(\cdot))\leq V+\dfrac{3\eta}{8}.
\end{gather*}
\end{claim*}

\begin{proof}
By (ii) $V(x,y_{0})\leq V+\dfrac{\eta}{4}\forall x\in \p D_{2}$, i.e.
$$
\inf\limits_{T>0}\inf\limits_{X(0)=x,X(T)=y_{0}}I^{T}_{0}(X(\cdot))\leq
V+\dfrac{\eta}{4}. 
$$

Let $\epsilon>0$ be arbitrary. Choose $T$ and $X(\cdot)$ such that
$I^{T}_{0}(X)\leq V+\dfrac{\eta}{4}+\epsilon$ with $X(0)=x$,
$X(T)=y_{0}$, $X(\cdot)\in \overline{G}$. If $X\in\Omega_{\delta_{1}}$
define $Y=X$. If $X\not\in \Omega_{\delta_{1}}$ define $Y$ as follows:

Let $t_{1}$ be the first time that $X$ enters $D_{1}$ and $t_{2}$ the
last time that it gets out of $D_{1}$. Then $0<t_{1}\leq t_{2}<T$. Let
$X^{*}$ be a path on $[0,s]$ such that (by Claim (iii))
$I^{s}_{0}(X^{*})\leq \dfrac{\eta}{8}$ with $X^{*}(0)=X(t_{1})$ and
$X^{*}(s)=X(t_{2})$. Define $Y$ on $[0,T-(t_{2}-t_{1})+s]$
$[T-(t_{2}-t_{1})+s,\infty)$ by
\begin{align*}
Y(t) &= X(t)\text{~ on~ } [0,t_{1}]=X^{*}(t-t_{1})\text{~ on~
}[t_{1},t_{1}+s]\\
&= X(t-t_{1}-s+t_{1}),\text{~ on~ }[t_{1}+s,T-(t_{2}-t_{1})+s]\\
&= X(t_{2}),\text{~ for~ }t\geq T-(t_{2}-t_{1})+s.
\end{align*}

Then
\begin{gather*}
I^{T-t_{2}+t_{1}+s}_{0}=\frac{1}{2}\int\limits^{t_{1}}_{0}|X-b(X(s))|^{2}ds+\frac{1}{2}\int\limits^{s}_{0}|X^{*}-b(X^{*}(s))|^{2}ds\\
+\frac{1}{2}\int\limits^{T}_{t_{2}}|X(s)-b(X(s))|^{2}ds\\
\leq V+\frac{\eta}{4}+\epsilon+\frac{\eta}{8}
\end{gather*}\pageoriginale
by choice of $X$ and $X^{*}$. As $Y\in \Omega_{\delta_{1}}$, we have
shown that
\begin{gather*}
\inf\limits_{T>0}\inf\limits_{\substack{X\in
    \Omega_{\delta_{1}},X(0)\in \p
    D_{1}\\ X(T)=y_{0}}}I^{T}_{0}(X(\cdot))\leq V+\frac{3\eta}{8}+\epsilon.
\end{gather*}

Since $\epsilon$ is arbitrary we have proved (v).
\end{proof}

\begin{lemma*}
$I^{T}_{0}$ is lower semi-continuous for every finite $T$.
\end{lemma*}

\begin{proof}
This is left as an exercise as it involves a repetiti on of an
argument used earlier.
\end{proof}

\begin{lemma*}
Let $X_{n}\in \Omega_{\delta_{1}}$. If $T_{n}\to \infty$ then
$I^{T_{n}}_{0}(X_{n})\to \infty$.
\end{lemma*}

This result says that we cannot have a trajectory which starts outside
of a deleted ball for which $I$ remains finite for arbitrary long
lengths of time.

\begin{proof}
Assume the contrary. Then there exists a constant $M$ such that
$I^{T_{n}}_{0}(X_{0})\leq M$, $\forall n$. Let $T<\infty$, so that
$M_{T}=\sup\limits_{n}I^{T}_{0}(X_{n})<\infty$. 

Define $X^{T}_{n}=X_{n}|_{[0,T]}$.
\end{proof}

\begin{claim*}
$\{X^{T}_{n}\}_{n=1}$\pageoriginale is an equicontinuous family.
\end{claim*}

\begin{proof}
\begin{align*}
& |X^{T}_{n}(x_{2})-X^{T}_{n}(x_{1})|^{2}=|\int\limits^{x_{2}}_{x_{1}}X^{T}_{n}(t)dt|^{2}\\ 
& \leq |x_{2}-x_{1}|^{2}\int\limits^{x_{2}}_{x_{1}}|X^{T}_{n}|^{2}dt\\
&\leq 2|x_{2}-x_{1}|^{2}\int\limits^{x_{2}}|X^{T}_{n}-b(X^{T}_{n})|^{2}ds+\int\limits^{T}_{0}b(X^{T}_{n})^{2}ds]\\
&\leq 2|x_{2}-x_{1}|^{2}[2M_{T}+T||b||^{2}_{\infty}].
\end{align*}

Thus, $\{X^{T}_{n}\}_{n}$ is an equicontinuous family. Since
$\overline{G}$ is bounded, $\{X^{T}_{n}\}_{n}$ is uniformly
bounded. By Arzela-Ascoli theorem and a ``diagonal procedure'' there
exists a subsequence $X_{n_{k}}$ and a continuous function to $X$
uniformly on compact subsets of $[0,\infty)$. As $X_{n_{k}}(\cdot)\in
  \overline{G}-D_{1}$, $X\in \overline{G}-D_{1}$. Let $m\geq
  n$. $I^{T_{n}}_{0}(X_{m})\leq M$. $X_{n}\to X$ uniformly on
  $[0,T_{n}]$. By lower semi-continuity $I^{T}_{0}(X)\leq M$. Since
  this is true for every $T$ we get on letting $T$ tend to $\infty$,
  that
$$
\frac{1}{2}\int\limits^{\infty}_{0}|X-b(X(s))|^{2}ds\le M.
$$

Thus we can find a sequence $a_{1}<b_{1}<a_{2}<b_{2}<\ldots$ such that
$$
I^{b_{n}}_{a_{n}}(X(\cdot))=\frac{1}{2}\int\limits^{b_{n}}_{a_{n}}|X(t)-b(X(t))|^{2}dt
$$
converges to zero with $b_{n}-a_{n}\to \infty$. Let
$Y_{n}(t)=X(t+a_{n})$. Then
$$
I^{b_{n}-a_{n}}_{0}(Y_{n})\to 0\q\text{with}\q b_{n}-a_{n}\to
+\infty,\q Y_{n}\in \Omega_{\delta_{1}}.
$$

Just\pageoriginale as $X$ was constructed from $X_{n}$, we can
construct $Y$ from $Y_{n}$ such that $Y_{n}\to Y$ uniformly on compact
subsets of $[0,\infty)$. 
$$
I^{b_{n}-a_{n}}_{0}(Y)\leq \inf\limits_{m\geq
  n}I^{b_{n}-a_{n}}_{0}(Y_{m})=0
$$
(by lower semi-confirmity of $I^{T}_{0}$). Thus
$I^{b_{n}-a_{n}}_{0}(Y)=0$, $\forall n$, showing that
$$
\int\limits^{\infty}_{0}Y(t)-b(Y(t))|^{2}dt=0
$$

Thus $Y$ satisfies $Y(\cdot)=x+\int\limits^{t}_{0}b(Y(s))ds$ with
$Y(t)\in\overline{G}-\p D_{1}$, $\forall t$.

\medskip
\noindent
{\bf Case (i).}~$Y(t_{0})\in G$ for some $t_0$. Let $Z(t)=Y(t+t_{0})$
so that $Z$ is an integral curve starting at a point of $G$ and
remaining away from $D_{1}$ contradicting the stability condition.
\medskip

\noindent
{\bf Case (ii).}~$Y(t_{0})\not\in G$ for any $t_{0}$, i.e. $Y(t)\in \p
G$ for all $t$. Since $Y(t)=b(Y(t))\langle Y(t)$, $n(Y(t))\rangle$ is
strictly positive. But $Y(t)\in \p G$ and hence $\langle Y(t),
n(Y(t))\rangle =0$ which leads to a contradiction. Thus our assumption
is incorrect and hence the lemma follows.
\end{proof}

\begin{lemma*}
Let $x\in \p D_{2}$ and define
\begin{align*}
& E=\{X(t)\text{~exits from $G$ before hitting $D_{1}$ and it exits from
  $N$}\}\\
& F=\{X(t)\text{~exists from $G$ before hitting $D_{1}$ and it exits
    from $N^{c}$}\}
\end{align*}

Then
$$
\frac{Q^{\epsilon}_{x}(F)}{Q^{\epsilon}_{x}(E)}\leq \exp
\left(-\frac{3\eta}{8\epsilon}+0\left(\frac{1}{\epsilon}\right)\right)\to
0\text{~ uniformly in~ } x(x\in \p D_{2}).
$$
\end{lemma*}

\noindent
{\bf Significance.}~ $Q^{\epsilon}_{x}(E)$\pageoriginale and $Q^{\epsilon}_{x}(F)$
are both small because diffusion is small and the drift is large. The
lemma says that $Q^{\epsilon}_{x}(E)$ is relatively much larger than
$Q^{\epsilon}_{x}(F)$. 

\begin{proof}
$Q^{\epsilon}_{x}(E)\geq Q^{\epsilon}_{x}\{X(t)$ exists from $G$
  before hitting $D_{1}$, and exists in $N$ before time $T\}$,
  $=Q_{x}(B)\geq \exp [-\dfrac{1}{\epsilon}\inf I^{T}_{0}(X(\cdot))]$
  where the infimum is taken over the interior of $B$,
$$
\geq \exp \left[-\frac{1}{\epsilon}\left(V+\frac{3\eta}{8}\right)+0\left(\frac{1}{\epsilon}\right)\right].
$$

Similarly,
$$
Q^{\epsilon}_{x}(F)\leq \exp
\left[-\frac{1}{\epsilon}\left(V+\dfrac{3\eta}{4}\right)+0\left(\frac{1}{\epsilon}\right)\right]. 
$$

Therefore
$$
\frac{Q^{\epsilon}_{x}(F)}{Q^{\epsilon}_{x}(E)}\leq \exp
\left[-\frac{3\eta}{8\epsilon}+0\left(\frac{1}{\epsilon}\right)\right]\to
0\q\text{as}\q \epsilon\to 0.
$$

We now proceed to prove the main theorem. Let
\begin{align*}
& \qq \tau_{0}=0,\\
& \tau_{1}=\text{first time~ } \p D_{1}\text{~ is hit},\\
& \tau_{2}=\text{next time~ }\p D_{2}\text{~ is hit},\\
& \tau_{3}=\text{next time~ }\p D_{1}\text{~ is hit},\\
& \ldots\ldots\ldots\ldots\ldots\ldots 
\end{align*}
and so on. Observe that the particle can get out of $G$ only between
the time intervals $\tau_{2n}$ and $\tau_{2n+1}$. Let
$E_{n}=\{$between $\tau_{2n}$ and $\tau_{2n+1}$ the path exits from
$G$ for the first time and that it exits in $N\}$,\pageoriginale 
$F_{n}=\{$between
$\tau_{2n}$ and $\tau_{2n+1}$ the path exits from $G$ for the first
time and that it exists in $N^{c}\}$.
$$
Q^{\epsilon}_{x}(X(\tau)\in N)+Q_{x}(X(\tau)\in N^{c})=1.
$$

Also
\begin{gather*}
Q^{\epsilon}_{x}(X(\tau)\in
N^{c})=\sum\limits^{\infty}_{n=1}Q^{\epsilon}_{x}(F_{n}),\\
Q^{\epsilon}_{x}(X(\tau)\in
N)=\sum\limits^{\infty}_{n=1}Q^{\epsilon}_{x}(E_{n}),\\ 
\sum\limits^{\infty}_{n=1}Q^{\epsilon}_{x}(F_{n})=\sum\limits^{\infty}_{n=1}E^{Q^{\epsilon}_{x}}(Q^{\epsilon}_{x}(F_{n}|\mathscr{F}_{\tau_{2n}}))\\
\leq
\sum\limits_{n=1}E^{Q^{\epsilon}_{x}}[\chi_{(\tau>\tau_{2n})}\sup\limits_{x\in
    \p D_{2}}Q^{\epsilon}_{x}(F)]\q\text{(by the Strong Markov
  property)}\\
\leq
0(\epsilon)\sum\limits^{\infty}_{n=1}E^{Q^{\epsilon}_{x}}[\chi_{(\tau>\tau_{2n})}\inf\limits_{x\in
    \p D_{2}}Q^{\epsilon}_{x}(E)]\q(\text{as~
}\frac{Q^{\epsilon}_{x}(F)}{Q^{\epsilon}_{x}(E)}\to 0)\\
\leq
0(\epsilon)\sum\limits^{\infty}_{n=1}Q^{\epsilon}_{x}(E_{n})=0(\epsilon)Q_{x}(X(\tau)\in N^{c}). 
\end{gather*}

Therefore
$$
Q^{\epsilon}_{x}(\chi(\tau)\in N)\to 1,\q Q_{x}(X(\tau)\in N^{c})\to
0.
$$
\end{proof}

\begin{exer*}
Suppose $b(x)=\nabla u(x)$ for some $u\in C^{1}(G\cup
\p G,R)$. Assume that $u(x_{0})=0$ and $u(x)<0$ for $x\neq
x_{0}$. Show that
$$
V(x_{0},y)=-2u(y).
$$
[Hint:~ For any trajectory $X$ with $X(0)=x_{0}$,
$$
X(T)=y, I^{T}_{0}(X)=\frac{1}{2}\int\limits^{T}_{0}|X+\nabla
u(X)|^{2}dt-2\int\limits^{T}_{0}\nabla u(X)\cdot X(t)dt\geq -2u(y) 
$$\pageoriginale
so that $V(x_{0},y)\geq -2u(y)$. For the other inequality, let $u$ be
a solution of $X(t)+\nabla u(X(t)=0$ on $[0,\infty)$ with
  $X(0)=y$. Show that because $\dfrac{duX(s)}{ds}0$ for $X(s)\neq 0$
  and $x_{0}$ is the only zero of $u$,
  $\displaystyle{\mathop{\rm limit}_{t\to \infty}}X(t)=x_{0}$. Now
  conclude that $V(x_{0},y)\leq -u(y)$].
\end{exer*}

