\chapter{The One Dimensional Random Walk}\label{chap3}

BEFORE\pageoriginale WE TAKE up Brownian motion, we describe a one
dimensional random walk which in a certain limiting case possesses the
properties of Brownian motion.

Imagine a person at the position $x=0$ at time $t=0$. Assume that at
equal intervals of time $t=\tau$ he takes a step $h$ either along the
positive $x$ axis or the negative $x$ axis and reaches the point
$x(t)=x(t-\tau)+h$ or $x(t)=x(t-\tau)-h$ respectively. The probability
that he takes a step in either direction is assumed to be
$1/2$. Denote by $f(x,t)$ the probability that after the time
$t=n\tau$ ($n$ intervals of time $\tau$) he reaches the position
$x$. If he takes $m$ steps to the right (positive $x$-axis) in
reaching $x$ then there are ${}^{n}C_{m}$ possible ways in which he
can achieve these $m$ steps. Therefore, the probability $f(x,t)$ is
${}^{n}C_{m}(\dfrac{1}{2})^{n}$.

$f(x,t)$ satisfies the difference equation
\begin{equation*}
f(x,t+\tau)=\frac{1}{2}f(x-h,t)+\frac{1}{2}f(x+h,t)\tag{1}\label{chap3-eq1}
\end{equation*}
and
\begin{equation*}
x=h(m-(n-m))=(2m-n)h.\tag{2}\label{chap3-eq2}
\end{equation*}

To see this one need only observe that to reach $(x,t+\tau)$ there are two
ways possible, viz.\@ $(x-h,t)\to (x,t+\tau)$ or $(x+h,t)\to
(x,t+\tau)$ and the probability for each one of these is $1/2$. Also
note that by definition of $f$,
\begin{equation*}
f(h,\tau)=\frac{1}{2}=f(-h,\tau),\tag{3}\label{chap3-eq3}
\end{equation*}\pageoriginale
so that
\begin{equation*}
f(x,t+\tau)=f(h,\tau)f(x-h,t)+f(-h,\tau)f(x+h,t).\tag{4}\label{chap3-eq4}
\end{equation*}

The reader can identify \eqref{chap3-eq4} as a ``discrete version'' of
convolution. By our assumption,
\begin{equation*}
f(0,0)=1,\quad f(x,0)=0\quad\text{if}\quad x\neq 0.\tag{5}\label{chap3-eq5}
\end{equation*}

We examine equation \eqref{chap3-eq1} in the limit $h\to 0$, $\tau\to
0$. To obtain reasonable results we cannot let $h$ and $\tau$ tend to
zero arbitratily. Instead we assume that
\begin{equation*}
\frac{h}{\tau}\to 1\quad\text{as}\quad h\to 0\quad\text{and}\quad
\tau\to 0.\tag{6}\label{chap3-eq6}
\end{equation*}

The physical nature of the problem suggests that \eqref{chap3-eq6}
should hold. To see this we argue as follows. Since the person is
equally likely to go in either direction the average value of $x$ will
be $0$. Therefore a reasonable measure of the ``progress'' made by the
person is either $|x|$ or $x^{2}$. Indeed, since $x$ is a random
variable (since $m$ is one) one gets, using \eqref{chap3-eq2},
$$
E(x)=2E(m)-n=0,\quad E(x^{2})=h^{2}E((2m-n)^{2})=h^{2}n.
$$
(Use
$\sum\limits^{n}_{m=0}m{}^{n}C_{m}\left(\dfrac{1}{2}\right)^{n}=\dfrac{n}{2}$,
$\sum\limits^{n}_{m=0}{}^{n}C_{m}\left(\dfrac{1}{2}\right)^{n}=\dfrac{n(n+1)}{4}$)

\medskip

Thus
$$
E\left\{\frac{x^{2}}{t}\right\}=\frac{1}{t}E(x^{2})=\frac{h^{2}n}{n\tau}=\frac{h^{2}}{\tau},
$$
and as $t$ becomes large we expect that the average distance covered
per unit time remains constant. (This constant is chosen to be $1$
for\pageoriginale reasons that will become apparent later). This
justifies \eqref{chap3-eq6}. In fact, a simple argument shows that if
$\dfrac{h^{2}}{\tau}\to 0$ or $+\infty$, $x$ may approach $+\infty$ in
a finite time which is physically untenable.

(1) now gives
$$
f(x,t+\tau)-f(x,t)=\frac{1}{2}\{f(x-h,t)-f(x,t)+f(x,h,t)-f(x,t)\}.
$$

Assuming sufficient smoothness on $f$, we get in the limit as $h$,
$\tau\to 0$ and in view of \eqref{chap3-eq6},
\begin{equation*}
\frac{\p f}{\p t}=\frac{1}{2}\frac{\p^{2}f}{\p x^{2}}\tag{7}\label{chap3-eq7}
\end{equation*}
(to get the factor $1/2$ we choose $\dfrac{h^{2}}{\tau}\to 1$). This
is the equation satisfied by the probability density $f$. The particle
in this limit performs what is known as {\em Brownian motion} to which
we now turn our attention.

\vskip 1cm

\noindent
{\Large\bf References.}
\begin{description}
\item[\mbox{[1]}] GNEDENKO: {\em The theory of probability},
  Ch.~10.

\item[\mbox{[2]}] {\em The Feynman Lectures on physics}, Vol.~1, Ch.~6.
\end{description}
