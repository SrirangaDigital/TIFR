\chapter{Whitney's Theorems}\label{chap3}
 
\section{Tangent Cones}\label{chap3-sec1}\pageoriginale

In what follows $G_{n,r}$ will denote the Grassmann manifold of
$r-planes$ through $\mathscr{O}$ in $ \mathbb{C}^n$. We shall assume
the classical result that $G_{n,r}$ is a compact $
\mathbb{C}$-analytic manifold. $G_{n,1}= \mathbb{P}^{n-1}$ is the
complex proiective space. If $T$ is an $r$-plane in $\mathbb{C}^n$,
$T^{*} = K(T)$ will denote the corresponding point in $G_{n,r}$ and for
a vector $v \neq 0$ in $ \mathbb{C}^n$, $K(v)$ will denote the
corresponding point in $\mathbb{P}^{n-1}$. If $\alpha \in G_{n,r},
T(\alpha)$ will denote the $r$-plane in $ \mathbb{C}^n$ such that
$K\cdot T(\alpha) = \alpha$. If $r_{1}< r_{2}$ and if $\alpha_{1}\in
G_{n,r_{1}}$, $\alpha_{2}\in G_{n, r_2}$, $\alpha_{1} \subset \alpha_{2}$
will mean that $T(\alpha_{1}) \subset T(\alpha_{2})$. 

\setcounter{definition}{0}
\begin{definition}\label{chap3-defin1} %def 1
  Let $V$ be an analytic set and $a$, $a$ point in $V$; the tangent
  cone at $a$, denoted by $C(V,a)$ is difined to be $\big\{v \in
  \mathbb{C}^n |$. There is a sequence $(b_\nu)$ in $V$,
  $b_{\nu}\neq a$, and $\lambda_{\nu}$ in  $\mathbb{C}$ 
    such that $\underset{\nu \to
      \infty}{\text{\rm Lim}}\lambda_{\nu}(b_{\nu} -a) = v\big\}$.  
\end{definition}

\setcounter{remark}{0}
\begin{remark}\label{chap3-rem1} % rem 1
  It follows trivially that if $a$ is a simble point of $V$, $C(V,a)
  = T (V,a)$, i.e. the tangent space to $V$ at $a$. 
\end{remark}

\begin{definition}\label{chap3-defin2} %Def 1
With the above notation, we define $C^{\ast} (V,a) =
K[C(V,a)-0]\subset 
  \mathbb{P}^{n-1}$. 
\begin{enumerate}[(1)]
\setcounter{enumi}{2}
\item If $v_{1},\ldots,v_{r}$ are vectors in $\mathbb{C}^n$ we write
  dep. $(v_{1},\ldots,v_{r})$
  when $v_{1},\ldots,\break v_{r}$ are $\mathbb{C}$-linearly dependent. 

\item If $a \in  \mathbb{C}^n$, we define  $\tilde{\mathbb{C}}^{n}_{a}
  = \{(z, v)\mid z\in
  \mathbb{C}^n, v \in \mathbb{P}^{n-1}$ and if $K(\omega) =
  v$, dep. $(\omega,z-a)\}$. Clearly $\mathbb{C}^n \times
  \mathbb{P}^{n-1} \supset \tilde{\mathbb{C}}^{n}_{a}
  \supset \{a\} \times \mathbb{P}^{n-1}$ and $\Pi_{1}:
  (\tilde{\mathbb{C}}^{n}_{a}- \{a\} \times
  \mathbb{P}^{n-1}) \to  \mathbb{C}^n$ is injective. Also
  $\tilde{\mathbb{C}}^{n}_{a} = \text{clos}
  (\tilde{\mathbb{C}}^{n}_{a} - \{a\} \times
  \mathbb{P}^{n-1})$. 
\end{enumerate}
\end{definition} 

\begin{remark}\label{chap3-rem2} 
$\widetilde{\mathbb{C}}^{n}_{a}$\pageoriginale is an analytic manifold
  of dimension 
  $n$ and $\{a\}\times \mathbb{P}^{n-1}$ is a submanifold of dimension
  $n-1$ of $\widetilde{\mathbb{C}}^{n}_{a}$. 
\end{remark}

\begin{proof} %pro
 If $(\omega_{1},\ldots,\omega_{n})$ denote homogeneous coordinates on
 $\mathbb{P}^{n-1}$, $\widetilde{\mathbb{C}}^{n}_{a}=\{(z,\omega)\in
 \mathbb{C}^{n}\times\mathbb{P}^{n-1}\mid
 z_{i}\omega_{j}=\omega_{i}z_{j}\}$. Let $(z^{0},\omega^{0})\in
 \widetilde{\mathbb{C}}^{n}_{a}$. We may assume
 $\omega^{0}=(\omega^{0}_{1},\ldots,\omega^{0}_{n})$ where
 $\omega^{0}_{1}\neq 0$, Choose a neighbourhood $U$ of
 $(z^{0},\omega^{0})$ such that if $(z,\omega)\in U$, then
 $\omega_{1}\neq 0$. Then for any $(z_{1},\omega)\in U\cap
 \widetilde{\mathbb{C}}^{n}_{a}$, we have
{\fontsize{10pt}{12pt}\selectfont
$$
z_{1}\dfrac{\omega_{j}}{\omega_{1}}=z_{j}, j\geq 2, \text{~ i.e. ~}
U\cap \widetilde{\mathbb{C}}^{n}_{a}=\{(z_{1},\omega)\in
\mathbb{C}^{n}\times \mathbb{P}^{n-1}\mid
z_{j}=\dfrac{\omega_{j}}{\omega_{1}}\cdot z_{1},j\geq 2\},
$$}\relax
and
$(z_{1},\frac{\omega_{2}}{\omega_{1}},\ldots,\frac{\omega_{n}}{\omega_{1}})$
give the local coordinages in $U\cap \widetilde{\mathbb{C}}^{n}_{a}$
and this proves the remark.
\end{proof}

\setcounter{definition}{4}
\begin{definition}\label{chap3-defin5} % Def 5
If $V$ is an analytic set and $a \in V$, we define $ V^{\ast \ast}_{a}
= \widetilde{\mathbb{C}}^{n}_{a}\cap (V\times \mathbb{P}^{n-1})$. 
\begin{enumerate}[(1)]
\setcounter{enumi}{5}
\item $V^{\ast}_{a}=$ closure of
  $[V^{\ast\ast}_{a}-\{a\}\times\mathbb{P}^{n-1}]$ in
  $\mathbb{C}^{n}\times \mathbb{P}^{n-1}$. 
\end{enumerate}
\end{definition}

\begin{remark}\label{chap3-rem3} % rem 3
 Since $\{a\} \times \mathbb{P}^{n-1}\cap V^{\ast\ast}_{a}$ is an
  analytic set, it follows from Proposition \ref{chap1-prop4}, Chapter
  \ref{chap1} that $ V^{\ast}_{a}$ is an analytic set.  
\end{remark} 

\begin{remark}\label{chap3-rem4} % rem 4
 $\Pi_{1}: (V^{\ast\ast}_{a} - \{a\} \times \mathbb{P}^{n-1})
  \to V$ is injective and $\dim_{(a,v)}V^{\ast}_{a} =
  \dim_{a} V$,  where  $(a,v) \in V^{\ast}_{a}$ This is obvious
  since 
$$ 
  V^{\ast \ast}_{a}- \{a\} \times \mathbb{P}^{n-1} =
  \left\{(z, K(z,a)) \mid z \neq a, z \in V \right\}.
  $$ 
\end{remark}

\setcounter{proposition}{0}
\begin{proposition}\label{chap3-prop1} %pro 1
 $V^{\ast}_{a}\cap \{a\} \times \mathbb{P}^{n-1} = a \times C^{\ast}(V,a)$.
\end{proposition}

\begin{proof}
  Let $ v \in C^{\ast}(V,a)$ and $ v= K(\omega)$. Then there is a
  sequence $\{z_{\nu}\}$ in $V$, $z_{\nu} \neq a$  and  $z_{\nu}
  \to a$ and a sequence $\{\lambda_{\nu}\}$ in $\mathbb{C}$ 
such that $\lambda_{\nu}(z_{\nu} - a) \to
  \omega$. Consider the sequence $(z_{\nu},K(z_{\nu}-a))$
   in  $V^{\ast}_{a}$. 
 Obviously\pageoriginale $(z_{\nu}, K(z_{\nu} -a)) \to (a,v)$.
 Conversels if 
 $ (z_{\nu}, v_{\nu}) \to (a,v). v_{\nu} = K(\omega_{\nu})$,
 then dep $(z_{\nu} -a, \omega_{\nu})$ and hence we have a sequence
 $\{\lambda_{\nu}\}$ in $ \mathbb{C}$ such that 
$\lambda_{\nu}(z_{\nu} -a) \to \omega$, where $K(\omega) =v$. 
\end{proof}

\begin{proposition}\label{chap3-prop2} %pro 2
$C^{\ast} (V, a)$ is an analytic set and $\dim_{a} V = \dim
  \Omega^{\ast}(V, a)+1$. 
\end{proposition}

\begin{proof} %pro
  Since the problem is local, we may assume $ \dim.  V = \dim_{a}
  V=p$. Then by Remark \ref{chap3-rem4} above, $\dim V^{\ast}_{a} = p$ and $\dim
  C^{\ast}(V,a) \leq p-1$, by Proposition \ref{chap1-prop4} of Chapter
  \ref{chap1}. Also 
  $\dim \{a\} \times \mathbb{P}^{n-1} = n-1$ and hence it follows
  from Proposition \ref{chap1-prop3} of Chapter \ref{chap1} that $\dim
  C^{\ast}(V,a) \geq p + (n-1) -n = p-1$, i.e. $\dim. C^{\ast}(V,a) = p-1$ and
  this proves the Proposition. 


In fact we shall use the following theorem and prove that $C^{\ast}(V,a)$ is an algebraic variety in $\mathbb{P}^{n-1}$. [See \cite{key2}
  for a proof of the following theorem.] 
\end{proof}

\begin{theorem*}[(Remmert-Stein).] %The
  If $\Omega \subset  \mathbb{C}^n$ is an open set and if $A \subset
  \Omega$ is an analytic set, $\dim A \leq k-1$ and if  $B \subset
  \Omega-A$ is an analytic set of constant dimension $k$, then 
  $\bar{B}$ is an analytic set in $\Omega$ and $\dim
  \overline{B}= k$. 
\end{theorem*}

\begin{theorem*}[(Chow).] %The  
Any analytic set in $\mathbb{P}^{n-1}$ is an algebraic set.
\end{theorem*}

\begin{proof} %pro
 Let $\Pi : \mathbb{C}^n - \{0\} \to \mathbb{P}^{n-1}$ be
  the natural map. Then if $V$ is an analytic set in $\mathbb{P}^{n-1}$,
  $\dim  V\geq 0$, then $W = \Pi^{1}(V)$ is an analytic set in 
$\mathbb{C}^n -\{0\}$ and $\dim W = \dim V+1>0$. Hence by the theorem
  of Remmert and Stein stated above, $\bar{W}$ is analytic in $\mathbb{C}^n$. 
  Obviously $0 \in \bar{W}$. Let $U$ be a convex neighbourhood of
  $0$ and $f^{1},\ldots, f^{k}$ be homomorphic functions on $U$ such
  that $U \cap \bar{W} = \{z \in U \mid f^{i}(z) = 0, 1 \leq i
  \leq k\}$. Let $f^{i}(z) = \sum^{\infty}_{r=1}
  P^{i}_{r}(z)$, where $P^{i}_{r}(z)$ is\pageoriginale a homogeneous
  polynomial of 
  degree $r$. Since $z \in \overline{W}\Rightarrow \lambda z \in
  \bar{W}$, we have, 
  \begin{align*}
    U \cap \overline{W} &= \bigg \{z \in U \mid \lambda z \in U \cap
    \overline{W}, \mid \lambda \mid \leq 1 \bigg \}\\ 
    &= \bigg\{z \in U \mid \sum^{\infty}_{r=1}
    \lambda^{r} P^{i}_{r}(z) =0, 1 \leq i \leq k, |\lambda| \leq 1
    \bigg\}.\\ 
    &= \bigg \{ z \in U \mid P^{i}_{r}(z) = 0, 1 \leq i \leq k, r
    \geq 1 \bigg\}. 
  \end{align*}

Now by Hilbert's basis theorem, there exist a finite number of
polynomials, $P_{1}, \ldots , P_{m}$ among $\bigg\{P^{i}_{r} \bigg\}$,
$1 \leq r < \infty$, $1 \leq i \leq k$, such that  
\begin{align*}
  \bigg\{ z \in U \mid P^{i}_{r}(z) &= 0, 1\leq i \leq k, r \geq 1 \bigg\}\\
  &= \bigg\{z \in U \mid P_{j}(z) = 0, 1 \leq j \leq m\bigg\}.
\end{align*}

Thus $U \cap \bar{W}$ is the set of zeros of a finite number of
homogeneous polynomials. Hence $\Pi (W) = V$ is an algebraic set in
$\mathbb{P}^{n-1}$. 
\end{proof}

\begin{coro*} %Col
In particular, $C^{\ast}(V,a)$ is an algebraic set in $\mathbb{P}^{n-1}$.
\end{coro*}

\setcounter{definition}{6}
\begin{definition}\label{chap3-defin7}
  If $f$ is a holomorphic function in a neighbourhood of $a  \in
  \mathbb{C}^n$, we have the series 
  $$
  f(a+z) = f^{0} + f^{1}(z) + f^{2}(z) + \cdots
  $$
  where $f^{i}(z)$ is a homogeneous polynomial of degree $j$ in
  $z_{1},\ldots,z_{n}$. if $m$ is the smallest number such that 
  $f^{m}(z) \equiv 0$, then $f$ is said to have order $m$ at $a$ and for
  any such $f$, we define $f^{\ast}_{a}(z) = f^{m}(z)$ where $m$ is
  the order of $f$ at $a$. 
\end{definition}

\begin{remark} %rem 5
  In fact Whitney \cite{key6} has proved that if $a \in V$. $V$ being an
  analytic set in  $\mathbb{C}^n$ and  if  $I_{a}$ is the ideal
  of holomorphic germs vanishing\pageoriginale on $V_{a}$, then there is a
  neighbourhood $U$ of a such that 
  $$ 
  \bigg\{z \in U \mid
  f^{\ast}_{a}(f) = 0  ~\text{ for }~ f \in f_{a}\bigg\}= C(V,a) \cap U.
  $$  
\end{remark}

\section{Wings}\label{chap3-sec2}

\begin{definition}\label{chap3-defin8} % Def 8
  Let $V$ be an analytic set, $M$, a manifold, $M \subset V$. Let 
  $W\subset V$ be an analytic set with $\dim  W < \dim V$ and $U$, an open
  set in $M$ and $l$, a positive real number. Let $\widetilde{Z} =
  U \times [0,1[Z = U \times]0,1[$. Then we define a wing
      stretching from $U$ into $V-W$ to be a set $B \subset V$ and a
      homeomorphism $F$ of $\widetilde{Z}$ onto $B$ for same $1>
      0$, where $F$ satisfies the following conditions. 
\begin{enumerate}[(1)]
\item For every $\lambda$, $0 \leq \lambda < l$, $F_{\lambda} (z) = F(z,
  \lambda)$ is a biholomorphic map from $U$ onto $ F_{\lambda}(U)$. 

\item $F$ is differentiable in $\lambda$ and $\dfrac{\partial
  F}{\partial \lambda}$ is continuous in $z$. 

\item If $z_{1}, \ldots , z_{m}$ are local coordinates in $U$, $z_{j}
  = x_{j} + iy_{j}$, where $x_{j}$ and $y_{j}$ are real, then the
  vectors $\dfrac{\partial F}{\partial x_{1}}$, $\dfrac{\partial
    F}{\partial y_{1}},\ldots, \dfrac{\partial F}{\partial
    x_{n}}$, $\dfrac{\partial F}{\partial y_{n}}$, $\dfrac{\partial
    F}{\partial \lambda}$ in $Z$ are linearly independent over
  $\mathbb{R}$. 

\item $F_{0} \mid U =$ identity map and $F_{\lambda}(U) \subset
  V - W$ for $\lambda >0$. 
\end{enumerate}
\end{definition}

\begin{remark}\label{chap3-rem6} %rem 6
  If $\widetilde{F} : U \times [0,1[\to V$ defines a wing
      and if $(z_{1}, \ldots , z_{m})$ are local coordinates in $U$,
      $\dfrac{\partial F}{\partial z_{k}}$ is continuous in
      $\widetilde{Z}$, for $0 \leq \lambda < l$. 
\end{remark}

\begin{proof}%pro
  We have only to check the continuity at points on $U \times
  \{0\}$. Since $\widetilde{F}$ is continuous on $U \times [0,1
    [$, it is uniformly continuous on $U' \times [0,\delta]$, where $
      \bar{U}' \subset U$ and $0 < \delta < l$, i.e. $F_{\lambda}
      \to F_{0}$ uniformly on $ U'$ as $\lambda
      \to 0$. Hence by Weierstrass' theorem it follows that
      $\dfrac{\partial F_{\lambda}}{\partial z_{k}} \to
      \dfrac{\partial F_{0}}{\partial z_{k}}$ for $1 \leq k
      \leq n$. 
\end{proof}

\begin{remark}\label{chap3-rem7} %rem 7
  Let\pageoriginale $z_{i} \in F_{\lambda_{i}} (U)$ and  $z_{i} \to
  z \in F_{0}(U)$. Then   
  $T(F_{\lambda_{i}} (U), z_{i}) \to T(F_{0} (U), a)$.
\end{remark}

\begin{proof}%pro
  It follows from conditions (1) and (4) in the definition of a wing, that
  $$ 
  T(F_{\lambda_{i}}(U), z_{i}) = dF_{\lambda_{j}}\big[T(U, z'_{i})\big]
  $$
  where $F_{\lambda_{i}}(z'_{i}) = z_{i}$.


From the Remark \ref{chap3-rem6} above, it follows that $dF_{\lambda}$ is
continuous on $[0,l[$ and hence follows the proof. 
\end{proof}

\setcounter{lemma}{0}
\begin{lemma}\label{chap3-lem1}
  Let $V$ be an analytic set, $0 \in W \subset V$, $W$ being an
  analytic subset of $V$, such that $W_{0}$, $V_{0}$ are
  irreducible and $\dim_{0} W = m< \dim_{0} V = r$. 
  Then there exists a neighbourhood $U$ of $0$ and a basis $(z_{1},
  \ldots , z_{n})$ in $U$ such that the basis is proper for $V_{0}$ as
  well $W_{0}$.  
\end{lemma}

\begin{proof} %pro
  Recalling Proposition \ref{chap1-prop3} of Chapter \ref{chap1}, we
  have only to find a basis 
  $(z_{1},\ldots, z_{m},\ldots,z_{r},\ldots,z_{n})$ in a neighbourhood
  $U$ of $0$ such that  
\begin{equation*}
  \begin{cases}
    & \{z \in U \mid z_{1} = 0, \ldots , z_{m} =0 \} \cap W = \{0\}\\
    & \text{and \ \ } \{z \in U \mid z_{1} = 0, \ldots , z_{r} =0 \}
    \cap V = \{0\}. 
  \end{cases}\tag{1}\label{eq1}
\end{equation*}

Let $a^{0}\in V$, $b^{0}\in W$, $a^{0} \neq 0$, $b^{0}
\neq 0$  and $a^{0}$, $b^{0}$ simple points of $V$ and $W$
respectively. Choose a linear form $l_{1}(z) (=
\sum^{n}_{j=1} \lambda_{i} z_{j})$   such that 
 $l_{1}(a^{0}) \neq 0$, $l_{1} (b^{0}) \neq 0$. Then by a
 holomorphic change of coordinates, we may suppose $l_{1}(z) = z_{1}$
 and we have for some neighbourhood $U_{1}$ of $0$, $W_{1} = \{z \in
 U_{1} \mid z_{1} = 0\} \cap W$ is an analytic set of dimension $m-1$
 and\pageoriginale $V_{1}= \{z \in U_{1} \mid z_{1} = 0\} \cap V\}$
 is an analytic 
  set of dimension $n-1$. Let $W_{1}= \bigcup_{\alpha}
 W^{1}_{\alpha}$, $V_{1} =\bigcup_{\beta} V^{1}_{\beta}$,
 $W^{1}_{\alpha}$ and $V^{1}_{\beta}$ being irreducible components of
 $W_{1}$ and $V_{1}$ respectively.
 Choose $a^{1}_{\alpha}$, $b^{1}_{\beta}$, simple points of
 $W^{1}_{\alpha}$, $V^{1}_{\beta}$ respectively and a linear form
 $l_{2}(z)$ such that $l_{2}(a^{1}_{\alpha}) \neq 0$,
 $l_{2}(b^{1}_{\beta}) \neq 0$ for all $\alpha$ and $\beta$ and
 $z_{1}$ and $l_{2}(z)$ 
 are linearly independent. By a change of coordinates let $l_{2}(z) =
 z_{2}$ and then there exists  a neighbourhood $U_{2} \subset U_{1}$
 of $0$ such that $W_{2} = \{z \in U_{2} \mid z_{1} = z_{2} = 0\} \cap
 W$ is an analytic set of dimension $m-2$ and 
 $ \{z \in U_{2} \mid z_{1} = z_{2} = 0\} \cap V$  is an analytic set
 of dimension $r-2$. Proceeding this way, we finally have a basis 
$(z_{1}, \ldots , z_{n})$ in a neighbourhood $U$ of $0$ such that
 conditions \eqref{eq1} are satisfied. 
\end{proof}

\begin{remark}\label{chap3-rem8}% rem 8
   In the above lemma, if $0$ is a simple point of $W$, then there
   exists a basis $(z_{i}, \ldots , z_{n})$ in a neighbourhood $U$ of
   $0$ such that the basis is proper for $V_{0}$ and  
   $$ 
   U \cap W = \big\{z \in U \mid z_{m+1} = \ldots = z_{n} = 0\big\}.
   $$
 \end{remark}

\begin{lemma}\label{chap3-lem2} % lem 2
  Let $0 \in M \subset V$, where $M$ is a manifold of dimension $m$, $V$
  is an analytic set with the germ $V_{0}$ irreducible and $\dim_{0} V
  = r> m$. Let $W\subset V$ be an analytic set with $\dim W < r$. Then
  there exists a neighbourhood $U$ of $0$, an 
  analytic set $V' \subset V$ in $U$ such that  
  \begin{enumerate}[\rm (1)]
  \item $U \cap M \subset U \cap V'$
  \item $\dim V' = m+1$
  \item $\dim V' \cap W \leq m$.
  \end{enumerate}
\end{lemma} 

\begin{proof} %pro
  By\pageoriginale Remark \ref{chap3-rem8}, there is a neighbourhood $U$ and
  coordinates $(z_{1},\break \ldots, z_{n})$ which are proper for $V_{0}$ and
  all irreducible 
  components of $W_{0}$ and $M \cap U = \big\{ z \in U \mid
  z_{m+1} = \ldots = z_{n} = 0\big\}$. Let $\Pi_{r}$ denote the
  projection $(z_{1}, \ldots , z_{n}) \to (z_{1}, \ldots ,
  z_{r})$. Let $z_{0} \in \Pi_{r}(U)$ be such that $z_{0}
  \notin M$, $z_{0} \notin \Pi_{r}(W)$. (This is possible since 
  $\dim  W < r$, $m< r$). 
  Let $N$ be the $(m+1)$-plane spanned by $M$ and the complex line
  defined by $0$ and $z_{0}$. Then $\Pi^{-1}_{r} (N) \cap V$ is
  an analytic set. Also since $\Pi^{-1}_{r}(a) \cap V$ is finite for
  every $a$ in $\Pi_{r}(U)$, $\dim \Pi^{-1}_{r}(N) \cap V \leq m+1$. 
  Hence if $V' = \Pi^{-1}_{r}(N) \cap V$, $\dim V' = m+1$ and clearly 
  $V'$ satifies the conditions (1), (2) and (3) of the lemma. 
\end{proof}

\begin{proposition}\label{chap3-prop3} %pro 3
  Let $a \in M \subset V$, $M$ being a manifold of dimension $m$ and
  $V$, an analytic set with $V_{a}$ irreducible and $\dim_{a} V= r
  >m$. Let $W \subset V$ be an analytic set, $\dim  W <r$. Then for
  any neighbourhood 
  $\Omega$ of $a$ in $M$, there is an open set $U \subset \Omega
  \subset M$ ($U$ note necessarily a neighbourhood of $a$) and a wing
  stretching from $U$ into $V-W$. 
\end{proposition}

\begin{proof} %pro
  For simplicity, we may assume $a$ to be $0$, By Remark
  \ref{chap3-rem8} and Lemma \ref{chap3-lem2} 
   above, there is neighborhood $U^{n}_{1}$ of $0$ and coordinates
  $z_{1}, \ldots ,\break z_{n}$ in $U$, which are proper for $V_{0}$
  such that 
  $M \cap U^{n}_{1} = \{z \in U^{n}_{1} \mid z_{m+1} = \ldots =
  z_{n} = 0\}$, and an analytic set $V'$ in $U^{n}_{1}$ such that 
\begin{enumerate}[(1)]
\item  $\dim V' = m+1$, 
\item $U^{n}_{1} \cap M \subset U^{n}_{1} \cap V'
  \subset U^{n}_{1} \cap V$ and 
\item $\dim V' \cap W \leq m$. 
\end{enumerate}

We shall prove that there is a wing stretching from an open set in
  $\Omega$ into $V' - W$. 

We assume that $V'_{0}$ is irreducible and that the basis
$(z_1 , \ldots , z_n)$ is proper for $V'_{0}$ and satisfies the
condition of Remark \ref{chap3-rem8}. Then $U^{n}_{1} \cap M$ 
is\pageoriginale the analytic set
given by $\{z \in U^n_1 \mid z_{m-1} =\ldots= z_n =
0\}$. Let $I$ be the ideal of germs at $0$ of holomorphic functions
vanishing on $V'_0$ and let $\eta: \theta^n \to \theta^{n}/I$ 
be the natural projection. Then with the notation of Theorem \ref{chap1-thm5}
of Chapter \ref{chap1}, there exists a distinguished polynomial $P_{m+2}[x]$ in
$\theta^{m+1}[x]$ such that $P_{m+2}$ is the minimal polynomial of
$\eta(z_{m+2})$ over $\theta^{m+1}$, $\eta(z_{m+2})$ generating the
quotient field of $\theta^{n}/I$ over the quotient field of
$\theta^{m+1}$. Let $\delta$ be the discriminant of $P_{m+2}$. Let $C$
in $\Pi_{m-1}(U^n_1)$ be the analytic set given by 
$$
C=\Bigg\{z\in
\Pi_{m+1}(U^n_1)\mid \delta (z) = 0 ~\text{~ or~ }~ z\in \Pi_{m+1}(V'\cap W
\cap U^n_1)\Bigg\}.
$$ 

Then dimension of $C = m$ and if $D =
(\overline{C-M}) \cap M$, by Proposition \ref{chap1-prop4} of Chapter
\ref{chap1}, $\dim D < m$. Hence given an open set $\Omega < M$, there
is an open set $U^m_1 \subset \Omega$ such that $U^m_1 \cap D = \phi$,
i.e. $(U^m_1 \times 
\{0\}) \cap (\overline{C-M}) = \phi$. Hence there is an open set
$U^1_1$ in  $\mathbb{C}$, $0 \in U^1_1$, such that $(U^m_1
\times U^1_1)\cap (\overline{C-M}) = \phi$. This implies that (i)
if $(z_1,\ldots,z_{m+1})\in U^m_1 \times U^1_1$ and $z^{m+1} \neq
O$, then $\delta(z_1,\ldots,z_{m+1})\neq 0$ and
$(z_1^{0},\ldots,z_{m+1}) \notin (\overline{C-M})$.  

Let $\widetilde{z}^{0}= (z^{0}_1,\ldots,z^{0}_m)
\in U^m_1$. Let $z_0 = (z^0_1,\ldots,z^0_m,
O,\ldots,O)\in \mathbb{C}^n$. By Proposition \ref{chap1-prop1} of Chapter
\ref{chap1}, $V'_{z_{0}} = \bigcup_{i=1}^{k} V^i_{z_{0}}$  where
$V^i_{z_{0}}$ are irreducible germs of analytic sets and
$V^i_{z_{0}} \not\subset \bigcup_{j \neq i}
V^j_{z_{0}}$ for any $i$. We assume that $V^i_{z_{0}}$ are
germs of analytic sets, defined by analytic sets $V^i$ in a
neighbourhood $U^n_2$ of $z_0$, $\Pi_{m=1}(\Pi^n)\subset \Pi^m_1 \times
\Pi^1_1$ and that $\Pi_{m+1}(V'\cap U^n_2) = \Pi_{m+1}(\Pi^n_2)$. Now
$z_0$ is an isolated point of $U^n_2 \cap V^1 \cap
\Pi^{-1}_{m+1}(\widetilde{z}^{0}, O)$. Hence there is an open set
$\Pi^{n-m-1}_1$ in  $\mathbb{C}^{n-m-1}$, $0 \in U^{n-m-1}_1$,
such that $(U^m_1 \times U^1_1 \times \supset \Pi^{n-m-1}_1)\cap V^1
\cap \Pi^{-1}_{m+1}(\widetilde{z}^{0},0)$ is\pageoriginale 
empty and hence there is an open
set $U^m_2 \times \Pi^1_2 = U^m_1 \times U_1$ such that 
\begin{itemize}
\item[(i)] $(\widetilde{z}^0,0)\in U^m_2 \times U^1_2$ and
$\Pi^{-1}_{m+1}(z)\cap (U^m_2 \times U^1_2 \times
\partial U^{n-m-1}_1)\cap V^1 = \phi$ if $z\in U^m_2 \times
U^1_2$, 

\item[(ii)] $\Pi_{m+1}$; $(U^m_2 \times U^1_2 \times U^{n-m-1}_1)\cap
V^1 \to U^m_2 \times U^1_2$ is surjective. 
It follows that
$\Pi_{m+1}: (U^m_2 \times U^1_2 \times U^{n-m-1}_1)\cap V^1
\to U^m_2 \times U^1_2$ is proper and surjective. 
\end{itemize}

Let  
$$
\displaylines{\hfill 
X = (U^m_2 \times U^1_2 \times U^{n-m-1}_1)\cap V^1\quad\text{and}\hfill\cr 
\hfill U^m_2 \times U^1_2 = U^{m+1} = \bigg\{ z \in  \mathbb{C}^{m+1}
\big|~\big| z_i-z^0_i\big| < \rho_i, 1 \leq i \leq m,\big|
z_{m+1}\big| < \rho_{m+1}\bigg\}.\hfill} 
$$ 

Since 
{\fontsize{10pt}{12pt}\selectfont
$$
z\in(U^{m+1}-M)\Rightarrow \delta(z) \neq
O, \Pi_{m+1}:\bigg[X-\Pi^{-1}_{m+1} \left(M\cap U^{m+1}\right)\bigg]\to
(U^{m+1}-M)
$$}\relax
is a covering of $p$ sheets say. Moreover, since
$V^1_{z^0}$ is irreducible, we may assume that
$\bigg[X-\Pi^{-1}_{m+1} \left(M \cap U^{m+1}\right)\bigg]$ is connected. 

Let 
\begin{align*}
  Y_0 & =\bigg\{ z\in\mathbb{C}^{m+1}\big|~\big| z_i-z^0_i\big| <
  \rho_i, 
   1 \leq i \leq m, 0 < \big|
  z_{m+1}\big| < \rho^{1/p}_{m+1}\bigg\}\\ 
  \text{and}\quad  Y &=\bigg\{
  z\in \mathbb{C}^{m+1}\big|~\big| z_i-z^0_i\big|
  < \rho_i , 1 \leq i \leq m,  \big| z_{m+1}\big| <
  \rho^{1/p}_{m+1}\bigg\}, 
\end{align*}
and consider the covering $(\Pi' \big|
Y_0): Y_0 \to (U^{m+1}-M)$, where $\Pi' : Y\to
U^{m+1}$ is given by  
$$
\Pi'(z_1,\ldots,z_{m+1}) = (z_1,\ldots,z_m, z^p_{m+1}). 
$$


Then there is a map $f_{0}:Y_{0}\to X-\Pi^{-1}_{m+1}(U^{m+1}\cap M)$
such that $\Pi'=\Pi_{m+1}\circ f_{0}$ on $Y_{0}$. By Riemann's
extension theorem, $f_{0}:Y_{0}\to U^{m+1}\times U^{n-m-1}_{1}$ can be
extended to a holomorphic function on $Y$, the extention being denoted
by $f$, and since $X$ is closed in $U^{m+1}\times U^{n-m-1}_{1}$ and
$Y_{0}$ is dense in $Y$, it follows that $f(Y)\subset X$ and
$\Pi'=\Pi_{m+1}\circ f$ on $Y$. Also, since $\Pi'$ and $\Pi_{m+1}|X$
are proper, $f$\pageoriginale is proper and $f(Y)=X$. this implies
that
$$
X\cap
\Pi^{-1}_{m+1}(z_{1},\ldots,z_{m},0)=(z_{1},\ldots,z_{m},0,\ldots,0)=f(z_{1},\ldots,z_{m},0)\text{~in}~X.
$$

Now consider $U^{m}_{2}\times [0,\delta_{m+1})$ and let
  $g:U^{m}_{2}\times[0,\delta_{m+1})\to Y$ be given by
    $g(z_{1},\ldots,z_{m},\lambda)=(z_{1},\ldots,z_{m},\lambda^{1/p})$
    where $\lambda^{1/p}$ is the positive $p^{\text{th}}$ root of
    $\lambda_{0}$ for $\lambda>0$.

Let $\widetilde{Z}=U^m_2 \times (0, \delta_{m+1})$ and $Z = U^m_2 \times
(0, \delta_{m+1})$ and $\widetilde{F}\cdot\widetilde{Z}\to V'$ be
defined by $\widetilde{F}= f \circ  g$. Then we claim that $\widetilde{F}$
defines the wing with the required properties. It is obvious that
$\widetilde{F}$ is a homeomorphism and that $\widetilde{F}(z_1,\ldots,z_m,0) =
(z_1,\ldots,z_m, 0,\ldots,0)$. Also for every $\lambda \geq 0
\widetilde{F}_{\lambda}: U^m_2 \to F_\lambda (U^m_2)$ is
biholomorphic. In fact, $\widetilde{F}$ is analytic in $\lambda$ on $Z$
and hence $\dfrac{\partial F}{\partial \lambda}$ is continuous on
$Z$. Also $\Pi_{m+1} (F(z,\lambda)) = (z,\lambda)$, hence condition 3
in the definition of wing is trivially verified. Also because of 
(1), for $\lambda > 0$, $F_\lambda (U^m_2) \subset V' - W$. 
\end{proof}

\begin{remark}\label{chap3-rem9} %rem 9
  If the open neighbourhood $\Omega \subset M$ of a contains a simple
  point of $\overline{(V'-W)}$, the proposition is trivial.  
\end{remark}

\begin{remark}\label{chap3-rem10} % 10
  In fact the wing that we obtained in Proposition \ref{chap3-prop3}
  stretches into 
  $\bigg\{ z \in V' \mid z$ is a simple point of $V'$ and $z
  \notin W \bigg\}$ i.e. $F_\lambda (U^m_2) \subset \{ z\in
  V'\mid z$ is a simple point of $V'$ and $z\notin W\}$ for
  $\lambda > 0$. 
\end{remark}

\section{The singular set $S_a$}\label{chap3-sec3}

Let $\Omega$ be an open set in  $\mathbb{C}^n$ and $V\subset \Omega$
be an irreducible analytic set of dimension $r$. Let $W \subset V$ be
an irreducible analytic set and $\dim W=m<r$. We shall prove that
there is an analytic set
$S_a\varsubsetneqq W$ such that for
every simple point $z$ of $W$ with $z \notin S_a$, the pair $(W,V)$
is (a) regular at $z$. 

In\pageoriginale what follows $G$ will denote the Grassmann manifold
$G_{n,r}$, $G'$ 
will denote the Grassmann manifold $G_{n,m}$ and $\overset{\circ}{V}$,
$\overset{\circ}{W}$ will denote the sets of simple points of $V$ and $W$
respectively and $\overset{.}{V}$, $\overset{.}{W}$, the sets of
singular points of $V$ and $W$ respectively. If $\alpha \in G$,
$T(\alpha)$ will be the $r$-plane corresponding to $\alpha$. Consider
$C^* (\overset{\circ}{V}) = \{(z,\alpha)\mid z\in
\overset{\circ}{V}, T(\alpha) = T(V,z)\}$. Clearly $C^*
(\overset{\circ}{V}) \subset \overset{\circ}{V} \times G$ is an
analytic set. Let $C^*(V) = $ closure of $C^* (\overset{\circ}{V})$ in
$\Omega \times G$. For $z\in V$, we define $C^*(V,z)$ as
follows. 
$$
z\times C^* (V,z) = C^* (V)\cap \{z\}\times G. 
$$

\begin{proposition}\label{chap3-prop4} % pro 4
$C^* (V)$ is an analytic set in $\Omega \times G$ and $C^*(V,z)$ is
  an analytic subset of $G$. 
\end{proposition}

\begin{proof} % pro 
  Let $z\in V$. By Lemma \ref{chap2-lem5} of Chapter \ref{chap2} there exists a
  neighbourhood $U \subset \mathbb{C}^n$ of $z$ and holomorphic
  vector fields $v^1,\ldots,v^q$ on $U$ such that $v^i(z) = 0$, $1 \leq i
  \leq q$, for $z\in \overset{.}{V} \cap U$ and
  $\{v^i(z)\}$, $1\leq i\leq q$ generate $T(V,z)$ if
  $z\in \overset{\circ}{V}\cap U$.  

Now for any $\alpha \in G$, the $r$-plane $T (\alpha)$ defines upto
a complex non-zero factor, an $r$-vector $\widehat{\alpha}$ in the exterior
algebra of  $\mathbb{C}^n$. Moreover there exists a neighbourhood
$U'$ of $\alpha$ such that the co-ordinates of $\widehat{\alpha}$ are
holomorphic on $U'$. For any vector $v$, if
$\widehat{\alpha}=(\alpha_1,\ldots,\alpha_r)$, we define $v \wedge
\widehat{\alpha}= v \wedge\alpha_1\wedge\ldots \wedge \alpha_r$ and the
equation $v\wedge\widehat{\alpha}=0$ is independent of the choice of
holomorphic coordinates of $\widehat{\alpha}$. Hence if we define 
$$
C_U^{**}(V) = \bigg\{(z,\alpha)\mid z \in V \cap U, \alpha
\in G, v^i(z)\wedge \widehat{\alpha} = 0,1\leq i \leq q\bigg\} 
$$ 
where\pageoriginale $\widehat{\alpha}$ has the above meaning,
$C^{**}_U (V)$ is an 
analytic set in $U\times G$. Further, $v^i(z)\wedge \alpha = 0$, $1\leq
i\leq q$, if and only if all the vectors $v^i(z)\in
T(\alpha)$. Since $\{v^i(z)\}$, $1\leq i\leq q$ span $T(V,z)$ if
$z \in \overset{\circ}{V}$ and $\dim T(V,z)=\dim T(\alpha) = r$, we
have, for $z\in \overset{\circ}{V}$, $(z,\alpha) \in
C^{**}_U(V)$ if and only if $T(\alpha) = T(V,z)$. It follows from
Proposition \ref{chap1-prop4} of Chapter \ref{chap1} that $U \cap
C^{*}(V)=$  closure of 
$[C^{**}_U(V)-\overset{.}{V}\times G]$ in $U\times G$ is an
analytic set and further, 
\begin{align*} 
                 & \dim C^\ast (V) = \dim V = r\\
  \text{and}\quad & \dim C^\ast (V)\cap (\overset{.}{V}\times G)\leq r-1.
\end{align*}

It follows that $C^\ast(V,z)$ for any $z\in V$ is an analytic set
in $G$.  
\end{proof}

\begin{lemma}\label{chap3-lem3} % lem 3
If $z$ is a simple point of $W$, then the following are equivalent
  \begin{enumerate}[\rm(1)]
  \item $\alpha \in C^*(V,z)\Rightarrow T(\alpha)\supset
    T(\overset{\circ}{W},z)$ 

  \item $(W,V)$ is (a) regular at $z$.
  \end{enumerate}
\end{lemma}

\begin{proof} %pro  
  The proof is trivial. We assume that condition (1) holds. If $q_i
  \in \overset{\circ}{V}$, $q_i\to z$ and if
  $T(\overset{\circ}{V}, q_i) \to T$, then clearly $T^* \in C^*
  (V,z)$ where $T^*$ is the element in $G$, corresponding to $T$. It
  follows from (1) that $T\supset T(\overset{\circ}{W}, z)$,
  i.e. $(W,V)$ is (a) regular at $z$. Conversely, if we assume that
  $(W,V)$ is (a) regular at $z$ and if $\alpha \in C^*(V,z)$,
  then there is a sequence $\{q_i\}$ in
  $\overset{\circ}{V}$, $q_i \to z$ and
  $T^*(\overset{\circ}{V},q_i)\to \alpha$. Then
  $T(\alpha)\supset T(\overset{\circ}{W},z)$ and the condition (1)
  is satisfied. 

Consider the set $C^*$ in $\Omega \times G \times G'$, given by 
$$
C^*=\bigg\{(z,\alpha,\alpha')\mid z\in W, \alpha \in C^*(V,z),
\alpha'\in T^*(W,z)\bigg\}.
$$ 

Then if  
\begin{align*}
  A & = \bigg\{(z,\alpha,\alpha')\mid z\in W, \alpha \in
  C^*(V,z), \alpha' \in G'\bigg\} \\
\text{and}\quad  B & = \bigg\{(z,\alpha,\alpha')\mid z\in
W, \alpha \in G,  \alpha' \in C^*(W,z)\bigg\}, 
\end{align*}\pageoriginale
it follows from Proposition \ref{chap3-prop1} above, that $A$ and $B$
and hence $C^* = 
A \cap B $ are analytic sets. Let  
$$
R^* = \bigg\{(z,\alpha,\alpha')\mid z \in W, \alpha \in
C^*(V,z), \alpha'\in C^*(W,z),T(\alpha')\subset
T(\alpha)\bigg\}. 
$$

Then $R^*$ is an analytic set and it follows from Proposition
\ref{chap1-prop4} of 
Chapter \ref{chap1} that $S^*_a =$ closure of $(C^*-R^*)$ in $\Omega \times G
\times G'$ is an analytic set. Let $\Pi_1:\Omega \times G \times
G'\to \Omega$ be the projection $\Pi_1 (z,\alpha,\alpha') =
z$, and let $\Pi_1 (S^*_a) = S_a$. We shall prove in the following two
propositions that $S_a$ is an analytic set with  
\begin{enumerate}[\rm(1)]
\item $\dim S_a < \dim W$ and
\item if $z\in(\overset{\circ}{W}-S_a)$, $(W,V)$ is (a) regular
  at $z$. 
\end{enumerate}
\end{proof}

\begin{proposition} % pro 5
With the above definition of $S_a$, if $z\in(\overset{\circ}{W}-S_a)$,
$(W,V)$ is (a) regular at $z$.  
\end{proposition}

\begin{proof} %pro
  If $z\in(\overset{\circ}{W}-S_a)$ and if $\alpha\in
  C^*(V,z)$, then $(z.\alpha,\alpha')\in C^*$ where $T(\alpha') =
  T(\overset{\circ}{W},z)$. Since $z\notin S_a$,
  $(z,\alpha,\alpha')\notin S^*_a$ and hence $(z,\alpha,\alpha')\in
  R^*$, i.e. $T(\alpha') = T(\overset{\circ}{W},z)\subset T (\alpha)$.
  The Proposition now follows from Lemma \ref{chap3-lem1} above. 
\end{proof}

To prove the next proposition, we shall use the following 

\begin{theorem*}[(Remmert).] %the
  If $V$ is an analytic space and $f\cdot V \to \Omega' \subset
  \mathbb{C}^m$ is a holomorphic, proper map, then 
  \begin{enumerate}[\rm(1)]
  \item $f(V)$ is an analytic set in $\Omega'$
  \item $\dim f(V)= \max\limits_{\substack{z\text{ simple}
      \\ \text{point of }V}}$  (rank $(df)$ $(z)$).  
  \end{enumerate}
\end{theorem*}

\begin{proposition} %pro 6
$S_a$\pageoriginale is an analytic set and $\dim S_a < \dim W = m$. 
\end{proposition}

\begin{proof} % pro
  Since $G$ and $G'$ are compact, $\Pi_1 : \Omega \times G \times G'
  \to \Omega$ is proper and hence $S_a = \Pi_1 (S^*_a)$ is an
  analytic set by (1) of the theorem above. Also if $\dim S_a = m$,
  it follows from (2) of the same theorem that there exists a simple
  point $z^{**}_0$ of $S^*_a$ such that rank $(d\Pi_1)(z^{**}_0)
  = m$ and hence by the constant rank theorem stated in Chapter \ref{chap1},
  there is neighbourhood $U^{**}$ of $z^{**}_0$ such that $U^{**}
  = U_0 \times U \times U'$ and $\Pi_1 (U^{**}\cap S^*_a)$ is a
  submanifold of dimension $m$, i.e. if $z^{**}_0 =
  (z_0,\alpha_0,\alpha'_0)$, $z_0$ is a simple point of
  $S_a$, of dimension $m$. Since $z_0$ is a simple point of $W$,
  we may assume that $\Pi_2 : U^{**}\cap C^*$ is an isomorphism onto
  $\Pi_2 (U^* \cap C^*) = \widehat{C}$ where $\Pi_2 : \Omega \times G
  \times G'\to \Omega \times G$ is the projection $\Pi_2
  (z,\alpha,\alpha') = (z,\alpha)$. Let $\Pi_2(U^{**}\cap S^*_a) =
  \widetilde{S}_a$. 
\begin{enumerate}[(1)]
\item Since $z^{**}_0 \notin R^*$, there is a vector
  $v_0 \in T(\overset{\circ}{W},z_0)$ such that $v_0
  \notin T(\alpha_0)$. Let for simplicity $z_0 = 0$. 
  
  With a suitable change of coordinates we can assume that $S_a \cap
  U_0 = \{(z_1,\ldots,z_n)\mid z_{m+1} =\ldots= z_n =
  O\}$ and $V_0 = \dfrac{\partial}{\partial z_1}$. Consider
  the analytic set $L^*= \{(z,\alpha)\mid \alpha=\alpha_0,
  z_2 = \ldots = z_n = 0\}$ in $U_0 \times U$. This is of
  dimension 1 and rank $(d\Pi_1)(z^*_0) = 1$. It follows from
  (1) and from the constant rank theorem that there is a
  neighbourhood $U^*_2 = U_1\times U_2$ of $z^*_0$, $U_1$ and $U_2$
  being neighbourhoods of $0$ and $\alpha_0$ respectively such
  that 

\item $(\Pi_1 \mid (U^*_2 \cap L^*) : L^*\to U_1 \cap
  L$ is an analytic isomorphism and 
 
\item if $z^* \in L^* \cap U^*_2$ and $z^* = (z,\alpha)$, we have 
  $$
  T(L,z) \not\subset T(\alpha). 
  $$
\end{enumerate}

Now\pageoriginale $U^*_2 \cap L^* \subset C^*(V)$ and 
$$
\dim (\overset{.}{V}\times G) \cap C^* (V)< \dim C^*(V) = r.
$$

Hence by Proposition \ref{chap3-prop3} of \S \ref{chap3-sec2}, there
is an open set $U^*_3 \subset 
U^*_2 \cap L^*$ and a wing $B^*$ defined by $F^*: U^*_3 \times
[O,\delta)\to C^*(V)$ such that $F^*_\lambda (t)\in
(C^*(V)-\overset{.}{V}\times G)$ for $\lambda > O$. Let $\Pi_1
(U^*_3)=U_3$. Define $F:U_3 \times (0,\delta)\to V$ by $F =
\Pi_1 \circ F \circ \Pi^{-1}_1$ (since by (2), $\Pi_1:U^*_2 \cap L^*$ 
is an analytic isomorphism). Since, for $\lambda > 0$, $F^*_\lambda
(U^*_3)\subset (C^*(V) - \overset{.}{V}\times G)$, ${F}_\lambda :
   {U_3}\to {F}_\lambda ({U_3})$ is an analytic isomorphism
   for $\lambda > 0$ and it is  easy to varify that $B=F({U}_3
   \times [0,\delta))$ is a wing  which  is homeomorphice with
   $B^\ast$. Set  ${B}_\lambda={F}_\lambda {U_3}$. Choose a sequence
   $q_i$ in  ${B}_{\lambda_i}$ such that ${q}_i \to p$ in
   ${U}_3= {B}_0$. Then  by remark \ref{chap3-rem7} of \S \ref{chap3-sec2},
   $T({B}_{\lambda_i},{q}_i)\to {T}(B_0,{p})={T}({L,p})$. Let
   ${q}_i={F}(t_i,\lambda_i)$ and let ${q}_i^\ast =({q_i, \alpha_i})=
   {F}^\ast(t_i, \lambda_i)$. Then  $q_i^\ast \to
   (p,\alpha)\in {U}_3^\ast$. Now  ${T(L,p)}= \text{Lim}
   T(B_{\lambda_i},q_i)$ and  ${T}(B_{\lambda_i},q_i) \subset
   {T}(\alpha_i)$, since ${q}_i \in \overset{\circ}{V}$; hence
   $T(L,p)\subset {T}(\alpha)$, $(p, \alpha)\in U_3^\ast$. But this
   contradicts the condition (3) above and hence it follows that 
   $$
   \dim S_a < m.
   $$

   From  Proposition  \ref{chap3-prop2} and  Proposition
   \ref{chap3-prop3} follows the  
\end{proof}

\begin{theorem*}[(a)(Whitney).]
  If $V$ is an irreducible analytic set in an open set $\Omega
  \subset \mathbb{C}^n$ and  if $W\underset{\neq}{\subset} 
  V$ is an  irreducible  analytic  subset, then there  exists  an
  analytic set $S_a \underset{\neq}{\subset} W$ such that for any $z
  \in \overset{\circ}{W} -S_a$, $(W,V)$ is  (a) regular at $z$. 
\end{theorem*}

\section{Theorem (b)}\label{chap3-sec4}\pageoriginale

\begin{lemma}\label{chap3-lem4} %lem 04
  Let ${z_0}\in {W}\subset {V,W,V}$ being  analytic sets such that
  $W_{z_0}$ and $V_{z_0}$ are irreducible  and ${\dim}_{z_0} W = m <
  {\dim}_{z_0} V=r$. Then there  exists  a neighbourhood  $U$ of $z_0$
  and an analytic  set  $X$ of  dimension  $1$ in $U$ such  that $z_0
  \in X$ and  
  $$
  U\cap (X-\{z_0\})\subset U \cap (V-W).
  $$
\end{lemma}

\begin{proof} %pro
  Let  for the sake of simplicity $z_0=0$. We  have  only  to
  recall  the proof  of Lemma \ref{chap3-lem1}, \S
  \ref{chap3-sec2}. We  have  linear forms  $l_1, 
  \ldots , l_m$ and  a neighbourhood $U'$ of $0$ such  that $\{ z
  \in U' \mid l_i(z)=0,1 \leq i \leq m \}\cap W$ is an
  analytic  set of dimension  $0$ and $V' = \{z \in
  U' \mid l_i (z)=0,1 \leq i \leq m \} \cap V$ is an
  analytic set of dimension $r-m$. Let  $X$ be a one dimensional
  analytic  subset of $V'$, $0 \in X$. Then  clearly there  exists a
  neighbourhood $U$ of $0$ such that $U\subset U'$ and $U\cap
  [X-\{z_0\}) \subset V-W$.  

In what  follows, $V$  is an  irreducible  analytic set  of dimension
$r$ in  $\Omega$, $\Omega$ an open  set in  $\mathbb{C}^n$, $W$ is an
analytic  subset  of $V$. For any anlytic  set
$A$, $\overset{\circ}{A}$ is the  set of simple points of $A$  and
$\overset{.}{A}$ is the  set  of singular points of $A$. $G$ will  denote  the
Grassmann  manifold of $r$ planes  in   $\mathbb{C}^n$ and
$\mathbb{P}^{n-1}=R$  will  denote  the complex  projective  space. Let
$0 \in W  \subset V$, ${\dim}_0 W=m < r$. By Lemma  \ref{chap2-lem5} of Chapter
\ref{chap2}, there is a neighbourhood $U$ of $0$ and  holomorphic vector
fields $v^1,\ldots,v^k$ in  $U$ such  that $v^i(z)= 0$, $1 \leq i \leq
k$, if $z \in U$ is a singular point of $V$ and $(v^i(z))$ span
$T(v,z)$ is  $z$ is a simple point of $V$. We  now define  an analytic set
$C_0 \subset W \times  V \times \mathbb{P}\times G$ as follows. 
\begin{multline*}
  C_0 =\bigg\{(\zeta,z,v^*,\alpha)\mid \zeta \in W,z \in V,v^*
  \in \mathbb{P}, \alpha \in G. \text{~ if~ }  K(v)=v^*,\\
  \text{ dep. } (z-\zeta, v) ~\text{ and }~  v^i(z) \wedge \alpha =0, 1
  \leq i \leq k \bigg\}. 
\end{multline*}\pageoriginale
(For notation, see \S \ref{chap3-sec1})

Clearly $C_0$ is an analytic set. Let $C^{\ast\ast}=$ closure  of
$\bigg[C_0-(W \times \dot{V} \cup W \times W)\times
  \mathbb{P}\times G \bigg]$ in $W \times V \times \mathbb{P} \times
G$. By Proposition \ref{chap1-prop4} of Chapter \ref{chap1},
$C^{\ast\ast}$ is an analytic set 
and $(\zeta, z,v^\ast,\alpha)\in C^{\ast\ast}$ if and  only if
there  are sequences $z_\nu \in \overset{\circ}{V}$, $z_\nu \notin W$,
$\zeta_\nu \in W$, $\lambda_\nu \in\mathbb{C}$ such that  $z_\nu \to z$,
$\zeta_\nu \to \zeta$, $\lambda_\nu(\zeta_\nu-z_\nu)\to v$
where  $K(V)=V^\ast$ and $T(V, \zeta_\nu)\to T(\alpha)$. 

Let  $\Delta$ be the  diagonal  in the set $W \times W$ and let
$\tilde{C}^\ast= C^{\ast\ast} \cap \Delta \times \mathbb{P}\times
G$. If $\Pi_2 : W \times V \times \mathbb{P} \times G \to V
\times \mathbb{P} \times G$ is the  projection  $\Pi_2
(\zeta,z,v^\ast, \alpha)=(z,v^\ast,\alpha)$, let  $C^{\ast}=\Pi_2
\tilde{C}^\ast$ Clearly $C^\ast$ is an analytic set in $\Omega
\times \mathbb{P} \times G$. Now, let  $0 \in W \subset V$ and
$W_0$ and $V_0$ be irreducible such that $0$ is a simple
point of $W$, $\dim_0 W=m < \dim_0 V=r$. Then  we remark  that
we can choose a neighbourhood $U$ of $0$ and a  basis $(z_1, \ldots ,
z_n)$ in $U$ such that  $W \cap U\bigg\{z \in U \mid z_{m+1}= \ldots=
z_n =0\bigg\}$ and moreover, if  $\Pi_m :  \mathbb{C}^n \to
\mathbb{C}^m$ is the projection $\Pi_m (z_1, \ldots ,z_n)=(z_1, \ldots
, z_m)$ then  $\Pi_m^{-1}(z)\cap V \not\subset \dot{V}$ for  $z \in
\Pi_m(U)$. We  have only to choose a basis $(z_1, \ldots ,z_n)$ such
that  $W \cap U =\bigg\{z \in U \mid z_{m+1}=\ldots= z_n =0\bigg\}$
and  $\Pi_m^{-1} (\cap) \cap V \not\subset  \dot{V}$. (Since the set of
simple  points is open in $V$, by shrinking $U$ if  necessary, we then
have, $\Pi_m^{-1}(z)\cap V \not\subset \dot{V}$  for  $z \in
\Pi_m(U)$). Such a choice of basis is possible since the set of simple
points is dense in $V$. With respect to such a basis if  
$z^0 =(z^0_1,\ldots,z^0_m,0,\ldots,0)\in  U \cap
W = M$, $M_{z^0}$\pageoriginale will denote the transverse 
plane at $z^0$, i.e $\{ z \in  U \mid z_i = z^0_i, 1 \leq i
\leq m \}$. Let $\mathbb{P}'$ denote the projective space of  
$\mathbb{C}^{n-m} = \bigg\{ (z_{m+1},\ldots,z_n) \mid
(z_1,\ldots,z_n) \in \mathbb{C}^n \bigg \}$. We 
then define $\sigma_0 \subset (V \cap U) \times   \mathbb{P}'
\times G$ as follows 
$$
\sigma_0 =\bigg \{ (z, K (z - \Pi_m z), \alpha ) \mid z  \in
\overset{\circ}{V} \cap U, z \notin  W, T ( \alpha) = T(V,z) \bigg \}.   
$$

Let $\sigma = $ closure of $\sigma_0$ in $U \times \mathbb{P}'\times G$. 

In the proof of the Theorem (b) we shall use the sets and notations
introduced above. 
\end{proof}

\begin{theorem*}[Whitney]
  If $V$ is an analytic set, $W$ its analytic subset, $V$, $W$ being
  irreducible and $\dim W = m < r = \dim V$, then there exists an
  analytic subset $S_b$ of $W$ such that $\dim S_b < \dim W$ and if
  $z \in  \overset{\circ}{W}$, $z \notin S_b$,  then the pair $(W,V)$ is
  (b) regular at $z$.   
\end{theorem*} 

\begin{proof} %pro 
  Consider the analytic set $C^\ast$ as defined above. Let
  $R^\ast \subset V \times  \mathbb{P} \times G$  be the analytic set
  defined by  
  $R^\ast = \bigg\{ (z,v^\ast,\alpha) \mid z \in V,  v^\ast \in
  \mathbb{P},  \alpha \in G,  V^\ast \subset \alpha \bigg \}$. Here
  $v^\ast \subset \alpha$ means that if $v \in \mathbb{C}^n$
  such that $K(v) = v^\ast$, $v \subset T(\alpha)$. Then by
  proposition \ref{chap1-prop4} of Chapter \ref{chap1}, $S^\ast_b =$
  closure of $C^\ast - 
  R^\ast$ in $V \times  \mathbb{P} \times G$ is an analytic set. Let
  $\Pi: V \times  \mathbb{P} \times G \to V$ be the
  projection $\Pi(z, v^\ast, \alpha) = z$. Then $\Pi$ is proper
  and hence by Remmert's proper mapping theorem stated in \S \ref{chap3-sec3}, 
  $\Pi (S^\ast_b) = S_b$ is an analytic set, $S_b \subset W$. We
  claim that if $z \in \overset{\circ}{W}$ and $z \notin S_b$ then
  $(W,V)$ is (b) regular at $z$. This is obvious for if $(W,V)$ is
  not (b) regular at $z$, there are sequences $\zeta_\nu \in W$,
  $z_\nu \in \overset{\circ}{V} - W$, $\lambda_\nu \in \mathbb{C}$
  such that $\zeta_\nu \to z$,  $z_\nu
  \to z$,  $\lambda_\nu (z_\nu - \zeta_\nu) \to v$,
  $T(V,z_\nu)  \to  T$ and $v \notin T$. But then $z^{*} =
  (z,K(v),K(T)) \in C^\ast - R^\ast$ and $z = \Pi z^\ast \in
  S_b$, a contradiction. 
\end{proof}

We\pageoriginale now proceed to prove that $\dim S_{b}<m$. If possible
let $\dim S_{b}=m$. Since $\Pi:S^{\ast}_{b}\to S_{b}$ is proper, by
Remmert's theorem stated in \S \ref{chap3-sec3}, there exists a simple
point $z^{*}_{0}$ of $S^{\ast}_{b}$ (in particular $z^{\ast}_{0}\in
C^{*}-R^{*}$) and a neighbourhood $U^{*}_{0}$ of $z^{\ast}_{0}$ such
that $\Pi(U^{\ast}_{0}\cap S^{*}_{b})$ is a manifold of dimension $m$
and $\Pi(U^{*}_{0}\cap S^{*}_{b})=U_{0}\cap W=U_{0}\cap S_{b}$, $\Pi
z^{*}_{0}=z_{0}$ being a simple point of $S_{b}$ and
$\Pi(U^{*}_{0})=U_{0}$. 

By Theorem (a) of Whitney, there exists an analytic set $S_{a}\subset
W$ such that $\dim S_{a}<m$ and if $z\in (\overset{\circ}{W}-S_{a})$,
then $(W,V)$ is (a) regular at $z$. Hence we may assume that for
$U^{*}_{0}$ obtained above, $U_{0}\cap S_{a}=\emptyset$.

We assume, without loss of generality, that $\dot{V}\subset W$ and
that $V_{z_{0}}$ is irreducible. By the remark made above we put
$z_{0}=0$ and obtain a neighbourhood $U\subset U_{0}$ of $0$ and a
basis $(z_{1},\ldots,z_{n})$ such that $U\cap W=M=\{z\in U\mid
z_{m+1}=\ldots=z_{n}=0\}$ and for any $z\in \Pi_{m}(v)$, if $N_{z}$ is
the transverse plane, $N_{z}\cap V\not\subset V$. We not construct
$\sigma_{0}$ and $\sigma$ as above. Consider the holomorphic map
$\psi:V\times \mathbb{P}'\times G\to W\times V\times \mathbb{P}\times
G$ given by
$\psi(z,v^{\ast},\alpha)=(\Pi_{m}z,z,v^{\ast},\alpha)$. Now,
$(z,v^{\ast},\alpha)\in
\sigma\Rightarrow(\Pi_{m}z,z,v^{\ast},\alpha)\in C^{\ast}$. Hence
$\psi^{-1}(C^{*})=\sigma$ and $\sigma$ is an analytic set. Moreover,
the set $\{(z,K(z-\Pi_{m}z),T(V,z))\mid z$ is a simple point of $V$ in
$U\}$ is a connected set of simple points of $\sigma$ and is dense in
$\sigma$. Hence $\sigma$ is irreducible.

We now prove that $\Pi(\sigma-R^{*})\supset M\cap U$. If $z\in M$,
$z=\Pi(z^{*})$, $z^{*}=(z,v^{*},\alpha)\in C^{*}-R^{*}$, then there
are sequences $\zeta_{\nu}\in W$, $z_{\nu}\in(\overset{\circ}-W)$,
$\lambda_{\nu}\in \mathbb{C}$ such that $z_{\nu}$, $\zeta_{\nu}\to z$,
$\lambda_{\nu}(z_{\nu}-\zeta_{\nu})\to v$ and $K(v)=v^{*}$ and
$T(V,z_{\nu})\to T(\alpha)$.\pageoriginale Consider
$z_{\nu}-\zeta_{\nu}=z_{\nu}-\Pi_{m}z_{\nu}+\Pi_{m}z_{\nu}-\zeta_{n}$. Since
$|\Pi_{m}z_{\nu}-\zeta_{\nu}|\leq |z_{\nu}-\zeta_{\nu}|$, there exists
a subsequence $\{\lambda_{\nu_{k}}\}$ of $\{\lambda_{\nu}\}$ such that
$\lambda_{\nu_{k}}(\Pi_{m}z_{\nu_{k}}-\zeta_{\nu_{k}})$ converges to
$v''$ say. ($v''$ may be the zero vector). Clearly $v''\in T(M,z)$ and
since by our assumption $(W,V)$ is (a) regular at any point in $M$,
$v''\in T(\alpha)$. Hence if
$\lambda_{\nu_{k}}(z_{\nu_{k}}-\Pi_{m}z_{\nu_{k}})\to v'$, $v'\in
T(\alpha)$, i.e. $v'\neq 0$ and
$(z,K(v'),\alpha)\in\sigma-R^{*}$. Hence we have proved that
$\Pi(\sigma-R^{*})\supset M\cap U$.


Let $\sigma'=$ closure of $(\sigma\cap \Pi^{-1}(M)-R^{*})$ in $V\times
\mathbb{P}'\times G$. Then $\sigma'$ is an analytic set and $\Pi
\sigma'=M$. Again, using Remmert's proper mapping theorem, there
exists a simple point $z^{*}_{1}$ of $\sigma'$ (in particular
$z^{*}_{1}\not\in R^{*}$) and a neighbourhood $U'_{1}$ of $z^{*}_{1}$
in the set of simple points of $\sigma'(U'_{1}\cap R^{*}=\emptyset)$
such that if $\Pi_{1}=\Pi|\sigma'$, $\rank(d\Pi_{1})(z^{*}_{1})=m$ for
$z^{*}\in U'_{1}$ and $\Pi_{1}(U'_{1})=M_{1}$, $M_{1}$ being an open
set in $M$. Hence, using the constant rank theorem and assuming
$U'_{1}$ to be sufficiently small, we obtain an analytic set
$M'\subset U'_{1}$, $z^{*}_{1}\in M'$ such that $(\Pi_{1}|M'):M'\to
M_{1}$ is an analytic isomorphism. Consider $W_{1}=\Pi^{-1}(W)$. Then
$\dim W_{1}<r=\dim \sigma$. Hence by Lemma \ref{chap3-lem2}, there
exists a neighbourhood $U^{*}_{1}$ of $z^{*}_{1}$ in
$\mathbb{C}^{n}\times \mathbb{P}'\times G$, $U^{*}_{1}\cap
R^{*}=\emptyset$, and an irreducible analytic set $\sum^{*}\subset
U^{*}_{1}$, $\sum^{\ast}\subset \sigma$ such that $\dim \sum^{*}=m+1$,
$\sum^{*}\supset M'$ and $\dim\sum^{*}\cap W_{1}=m$. Now consider a
point $z^{*}_{2}$ in $M'$ such tht $z^{*}_{2}$ does not lie on other
components of $\sum^{*}\cap W_{1}$. If $\Pi_{2}=\Pi|\sum^{*}$, there
exists a neighbourhood $U^{*}_{2}$ of $z^{*}_{2}$ such that $z^{*}$ is
an isolated point of $\Pi^{-1}_{2}(\Pi_{2}z^{*})$ if $z^{*}\in
U^{*}_{2}$. In fact $\Pi^{-1}_{2}(\Pi_{2}z^{*})=z^{*}$. Hence
$\Pi_{2}(\sum^{*}\cap U^{*}_{2})=\sum$ is an irreducible
analytic\pageoriginale set in $\Pi_{2}U^{*}_{2}=U_{2}$ if $U^{*}_{2}$
is sufficiently small. Also $\sum\subset V$ and $\dim \sum =m+1$. Now
$N_{z_{2}}\cap \sum\cap W=\{z_{2}\}$ and if $B=N_{z_{2}}\cap \sum$,
$\dim B\geq 1$ by Proposition \ref{chap1-prop5} of Chapter
\ref{chap1}.

If $A=N_{z_{2}}\cap \sum \cap W$, then, if $A^{*}=\Pi^{-1}_{2}(A)$,
$B^{*}=\Pi^{-1}_{2}(B)$ then $\dim A^{*}=0$ and $\dim B^{*}\geq 1$. In
fact $A^{*}=\{z^{*}_{2}\}$. [If $\dim_{z^{*}_{2}}B^{*}<1$, we may
  choose $U^{*}_{2}$ sufficiently small and then $\dim B\cap
  \Pi(U^{*}_{2})<1$, which is a contradiction. Hence
  $\dim_{z^{*}_{2}}B^{*}\geq 1$.] Hence there exists a point
$z^{*}_{3}$ in $A^{*}\subset B^{*}$, such that
$\dim_{z^{*}_{3}}B^{*}\geq 1$ and since we assumed above that
$U^{*}_{1}\cap R^{*}=\emptyset$, $z^{*}_{2}\in \sigma-R^{*}$. We may
assume tht $B^{*}_{z^{*}_{2}}$ is irreducible and then, by Lemma
\ref{chap3-lem4} above, there exists a neighbourhood $U^{*}_{3}$ of
$z^{*}_{2}$ and a one-dimensionla analytic set $X^{*}$ in $U^{*}_{3}$,
$z^{*}_{2}\in X^{*}$, such that $X^{*}-z^{*}_{2}\subset
B^{*}-A^{*}$. Then if $\Pi X^{*}=X$, $X$ is an analytic set in $\Pi
U^{*}_{3}$ such that $\dim X=1$ and $X-z_{2}\subset
(\overset{\circ}{V}-W)$. Let $z^{*}_{2}=(z_{2},v^{*},\alpha)$, then
$v^{*}\not\in \alpha$ since $z^{*}_{2}\not\in R^{*}$. Let
$\zeta^{*}_{\nu}$ be a sequence in $B^{*}-A^{*}$, $\zeta^{*}_{\nu}$
simple points of $B^{*}$, $\zeta^{*}_{\nu}\to z^{*}_{3}$,
$\zeta^{*}_{\nu}=(z_{\nu},v^{*}_{\nu},\alpha_{\nu})$. Then $z_{\nu}\to
z_{2}$, $K(z_{\nu}-\Pi_{m}z_{\nu})=K(z_{\nu}-z_{2})=v_{\nu}$ and
$T(V,z_{\nu})=T(\alpha_{\nu})$ where $v^{*}_{\nu}\to v^{*}$ and
$T(V,z_{\nu})\to T(\alpha)$. Hence $v^{*}\in C(X,z_{2})$. Also $\dim
C^{*}(X,z_{2})=1$ (see \S \ref{chap3-sec1} for notation). Hence
$v^{*}=C^{*}(X,z_{2})$. Also since $\dim X=1$, and $z_{\nu}$ are
simple points of $X$, if $KT(X,z_{\nu})\to T^{*}$,
$T^{*}=C^{*}(X,z_{2})$. Hence $v^{*}\in
{\displaystyle{\mathop{\Lim}_{\nu\to \infty}}} K(T(X,z_{\nu}))\subset
{\displaystyle{\mathop{\Lim}_{\nu\to
      \infty}}}K(T(V,z_{\nu}))=\alpha$. Hence we have a contradiction
and this proves that the assumption that $\dim S_{b}=m$ is false. 
