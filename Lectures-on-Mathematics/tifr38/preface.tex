\chapter{Foreword}


The principle material of this course is taken from a paper of whitney
\cite{key7}. In the first chapter we recall some classical theorem (sec
\cite{key1} and \cite{key2}), explain the problem solved in
\cite{key7} and give several 
examples. In chapter II we study stratifications of an analytic set
having different properties. In chapter III we prove the theorem a)
and b) of Whitney. The main lines of the proofs are taken form \cite{key7}
but for the theorem b) our demonstrationis rather different (from the
application of therorem a) a result on field of frames tangent to the
strata of a stratification of an analytic set, along certain
skeletons. 

I have been lukcy enough to have the collaooration of Miss
M.S. Rajwade and Dr. Raghavan Narasimhan who had ideas for many
improvements and worte the present notes. I thank them very much for
their help. 

\bigskip
~\hfill{\large\bf A-H. Sehwartz}
