


\chapter{Appendix}

\begin{center}
{\huge\bf Evaluation of Zeta Functions for}
\medskip

{\huge\bf Integral Values of Arguments}
\end{center}
\bigskip

\centerline{{\large\bf\em By} \ \large\bf Carl Ludwig Siegel}
\bigskip

Euler\pageoriginale presumably enjoyed the summation of the series of
reciprocals of squares of natural numbers in the same way as Leibnitz
did the determination of the sum of the series which was later named
after him. In the nineteenth century, generalizations of these results
to the $L$-series introduced by Dirichlet were studied; these were
important for the class number formulae of cyclotomic fields;
corresponding questions for the $L$-series of imaginary quadratic
fields got elucidated by Kronecker's limit formulae.

In one of his most beautiful papers, Hecke has obtained an analogous
result for the $L$-series of real quadratic fields, which led him to
the assertion that the value of the zeta function of every ideal class
of a real quadratic number field of arbitrary discriminant $d$ is, for
the values $s=2,4,6,\ldots,$ of the variable, equal to the product of
$\pi^{2s}\sqrt{d}$ and a rational number. He further gave a brief
indication of two possible methods of proof for his assertion. The
first of these has recently been carried out in different ways
\cite{app-key1}. The second method of proof leads, after suitable
modifications to the generalization, published by Klingen \cite{app-key2}, to
arbitrary totally real algebraic number fields. This second method is
indeed quite clear in its underlying general ideas but, however, in
the form available till now, it admits of no effective determination
of the said rational number. By means of a theorem on elliptic modular
forms, which in itself seems not uninteresting, we shall, in the
sequel, present a proof by a constructive procedure meeting the usual
requirement of Number Theory.


\setcounter{section}{0}
\section{Elliptic Modular Forms}\label{app-sec1}

Let $z$ be a complex variable in the upper half plane and 
\begin{equation*}
F_{k}(z)=\frac{1}{2}\mathop{{\sum}'}_{l,m}(lz+m)^{-k} \;\;
(k=4,6,8,\ldots),\tag{1}\label{app-eq1} 
\end{equation*}\pageoriginale
where $l$, $m$ run over all pairs of rational integers except $0$,
$0$. If we set
$$
\sum_{t|n}l^{g}=\sigma_{g}(n)\quad (g=0,1,2,\ldots), e^{2\pi iz}=q
$$
then for this Eisenstein series we have the Fourier expansion
\begin{equation*}
F_{k}(z)=\zeta(k)+\frac{(2\pi
  i)^{k}}{(k-1)!}\sum^{\infty}_{n=1}\sigma_{k-1}(n)q^{n}\tag{2}\label{app-eq2} 
\end{equation*}
which, in view of
\begin{equation*}
\zeta(k)=-\frac{(2\pi i)^{k}B_{k}}{2\cdot k!},\tag{3}\label{app-eq3}
\end{equation*}
can also be expressed by the formulae
\begin{equation*}
F_{k}(z)=\zeta(k)G_{k}(z),
G_{k}(z)=1-\frac{2k}{B_{k}}\sum^{\infty}_{n=1}\sigma_{k-1}(n)q^{n}
\;\; (k=4,6,\ldots).\tag{4}\label{app-eq4}
\end{equation*}
In particular, here, we have the Bernoulli numbers
$$
B_{4}=-\frac{1}{30},B_{6}=\frac{1}{42},B_{8}=-\frac{1}{30},B_{10}=\frac{5}{66},B_{14}=\frac{7}{6};
$$
and hence
\begin{align*}
G_{4} &= 1+240\sum^{\infty}_{n=1}\sigma_{3}(n)q^{n},
G_{6}=1-504\sum^{\infty}_{n=1}\sigma_{5}(n)q^{n}\\
G_{8} &= 1+480\sum^{\infty}_{n=1}\sigma_{7}(n)q^{n},
G_{10}=1-264\sum^{\infty}_{n=1}\sigma_{9}(n)q^{n}\tag{5}\label{app-eq5}\\
G_{14} &= 1-24\sum^{\infty}_{n=1}\sigma_{13}(n)q^{n}
\end{align*}
with rational integral coefficients and constant term $1$. We set
further
$$
G_{0}=1.
$$
Let $h$ be a non-negative even rational integer. If we denote the 
dimension\pageoriginale of the linear space $\mathfrak{M}_{h}$ of
elliptic modular forms of weight $h$ by $r_{h}$, where for fixed $h$
we write, for brevity, also $r$ for $r_{h}$, then it is well known
that
$$
r_{h}=\left[\frac{h}{12}\right]+1(h\not\equiv 2 (\rm{mod} \;  12)),
r_{h}=\left[\frac{h}{12}\right](h\equiv 2(\rm{mod} \;  12)).
$$

We have, the particular,
\begin{equation*}
\begin{split}
G^{2}_{4} &= G_{8}, G_{4}G_{6}=G_{10}, G^{2}_{4}G_{6}=G_{14}\\
G_{l}G_{14-l} &= G_{14}(l=h-12r_{h}+12=0,4,6,8,10,14)
\end{split}\tag{6}\label{app-eq6} 
\end{equation*}
and for the modular form
$$
\Delta=q\prod^{\infty}_{n=1}(1-q^{n})^{24}
$$
of weight 12,
$$
1728\Delta= G^{3}_{4}-G^{2}_{6}.
$$

If 
\begin{equation*}
j(z)=G^{3}_{4}\Delta^{-1}=q^{-1}+\cdots\tag{7}\label{app-eq7}
\end{equation*}
denotes the absolute invariant, then
$$
\Delta^{2}\frac{dj}{dz}=3G^{2}_{4}\frac{dG_{4}}{dz}\Delta-G^{3}_{4}\frac{d\Delta}{dz}=\frac{1}{1728}G^{2}_{4}G_{6}\left(2G_{4}\frac{dG_{6}}{dz}-3G_{6}\frac{dG_{4}}{dz}\right), 
$$
and the expression in the brackets yields a modular form of weight 12
and indeed a cusp form which can therefore differ from $\Delta$ at
most by a constant factor. Comparing the coefficients of $q$ in the
Fourier expansions, we get
\begin{equation*}
\frac{dj}{d\log q}=-G_{14}\Delta^{-1}.\tag{8}\label{app-eq8}
\end{equation*}

Let hereafter, $h>2$ and hence $r_{h}>0$. The $r$ isobaric
power-products $G^{a}_{4}G^{b}_{6}$ where the exponents $a$, $b$ run
over all non-negative rational integer solutions of 
$$
4a+6b=h,
$$\pageoriginale
form a basis of $\mathfrak{M}_{h}$. It follows from this that, for
every function $M$ in $\mathfrak{M}_{h}$, $MG^{-1}_{h-12r+12}$ always
belongs to $\mathfrak{M}_{12r-12}$. Since $\Delta^{r-1}$ is a modular
form of weight $12r-12$, not vanishing anywhere in the interior of the
upper half-plane,
\begin{equation*}
MG^{-1}_{h-12r+12}\Delta^{1-r}=W\tag{9}\label{app-eq9}
\end{equation*}
is an entire modular function and hence a polynomial in $j$ with
constant coefficients.

Let
\begin{equation*}
T_{h}=G_{12r-h+2}\Delta^{-r}\tag{10}\label{app-eq10}
\end{equation*}
with the Fourier expansion
\begin{equation*}
T_{h}=C_{hr}q^{-r}+\cdots+C_{h1}q^{-1}+C_{h0}+\cdots\tag{11}\label{app-eq11}
\end{equation*}
and first coefficient $C_{hr}=1$. Since
\begin{equation*}
\Delta^{-1}=q^{-1}\prod^{\infty}_{n=1}(1+q^{n}+q^{2n}+\cdots)^{24},\tag{12}\label{app-eq12} 
\end{equation*}
all the Fourier coefficients of $T_{h}$ turn out to be rational
integers.

\setcounter{thm}{0}
\begin{thm}\label{app-thm1}
Let
\begin{equation*}
M=a_{0}+a_{1}q+a_{2}q^{2}+\cdots \tag{13}\label{app-eq13}
\end{equation*}
be the Fourier series of a modular form $M$ of weight $h$. Then
$$
C_{h0}a_{0}+C_{h1}a_{1}+\cdots+C_{hr}a_{r}=0
$$
\end{thm}

\begin{proof}
For $l=0,1,2,\ldots$
$$
j^{l}\frac{dj}{dz}=\frac{1}{l+1}\frac{dj^{l+1}}{dz}
$$
and hence, by \eqref{app-eq7}, it has a Fourier series without
constant term. Since the function $W$ defined by \eqref{app-eq9} is a
polynomial in $j$, the product $W\dfrac{dj}{dz}$ is, similarly, a
Fourier series without constant term.
\end{proof}

Because\pageoriginale of \eqref{app-eq6} and \eqref{app-eq8}, we have
$$
-\frac{1}{2\pi
  i}W\frac{dj}{dz}=MG^{-1}_{h-12r+12}\Delta^{1-r}G_{14}\Delta^{-1}=MG_{12r-h+2}\Delta^{-r}=MT_{h} 
$$
from which the theorem follows on substituting the series
\eqref{app-eq11} and \eqref{app-eq13} for $T_{h}$ and $M$
respectively.

Let us put for brevity,
$$
c_{h0}=c_{h}.
$$
It is important, for the entire sequel, to show that $c_{h}$ does not
vanish.

\begin{thm}\label{app-thm2}
We have
$$
c_{h}\neq 0.
$$
\end{thm}

\begin{proof}
First, let $h\equiv 2(\rm{mod} \;  4)$, so that $h\equiv 2t(\rm{mod} \;  12)$ with
$t=1,3,5$. Then correspondingly $12r=h-2$, $h+6$, $h+2$; hence
$12r-h+2=0,8,4$ and
$$
G_{12r-h+2}=G_{0},G^{2}_{4},G_{4}.
$$
Since by \eqref{app-eq5}, $G_{4}$ has all its Fourier coefficients
positive and the same holds for $\Delta^{-r}$ as a consequence of
\eqref{app-eq12}, we conclude from \eqref{app-eq10} that all the
coefficients in the expansion \eqref{app-eq11} are positive. Therefore
in the present case, the integers $c_{h0}$, $c_{h1},\ldots,c_{hr}$ are
all positive and, in particular, $c_{h}=c_{h0}>0$ \ie $c_{h}\neq 0$.
\end{proof}

Let now $h\equiv 0(\rm{mod} \;   4)$ so that $h\equiv 4t(\rm{mod} \;  12)$ with
$t=0,1,2$ whence $12r=h-4t+12$, $h-12r+12=4t$ and
$$
G_{h-12r+12}=G_{4t}=G^{t}_{4}.
$$
Further we have now
\begin{align*}
T_{h} &= -G_{12r+h+2}\Delta^{1-r}G^{-1}_{14}\frac{dj}{d\log
  q}=-G^{-t}_{4}\Delta^{1-r}\frac{dj}{d\log q}\\
&= \frac{3}{t-3}\Delta^{1-r-t/3}\frac{dj^{1-t/3}}{d\log q}\\
&= \frac{3}{t-3}\frac{d(G^{3-t}_{4}\Delta^{-r})}{d\log
  q}+\frac{3r+t-3}{(3-t)r}G^{3-t}_{4}\frac{d\Delta^{-r}}{d\log q};
\end{align*}\pageoriginale
hence $c_{h0}$ is also the constant term in the Fourier expansion of
the function
$$
V_{h}=\frac{3r+t-3}{(3-t)r}G^{3-t}_{4}\frac{d\Delta^{-r}}{d\log q}.
$$
In view of the assumption $h>2$, we have $3r+t-3>0$. The series for
$G^{3-t}_{4}$ begins with $1$ and has again all its coefficients
positive. Further, by \eqref{app-eq12}, the coefficients of the
negative powers $q^{-r},\ldots,q^{-1}$ of $q$ in the derivative of
$\Delta^{-r}$ with respect to log $q$ are all negative while the
constant term is absent. Hence the constant term in $V_{h}$ is
negative and $c_{h}=c_{h0}<0$ \ie $c_{h}\neq 0$. The proof is thus
complete.

A most important consequence of the two theorems is the fact that, for
every modular form $M$ of weight $h$, the constant term $a_{0}$ in its
Fourier expansion is determined by the $r$ Fourier coefficients
$a_{1},\ldots,a_{r}$ that follow, namely by the formula
\begin{equation*}
a_{0} = c^{-1}_{h}(c_{h1}a_{1}+\cdots+c_{hr}a_{r}).\tag{14}\label{app-eq14}
\end{equation*}
If, in particular, $a_{1},\ldots,a_{r}$ are rational integers, then
$a_{0}$ itself is rational and the denominator of $a_{0}$ divides
$c_{h}$.

As a further remarkable result, we note that the vanishing of $a_{0}$
follows from the vanishing of the Fourier coefficients
$a_{1},\ldots,a_{r}$ and hence the vanishing of the modular form $M$
itself, since then the entire modular function $W$ defined in
\eqref{app-eq9} has a zero at $q=0$.

From \eqref{app-eq10} and \eqref{app-eq11}, it follows that the
numbers $c_{hl}(l=0,1,\ldots,r)$ are the coefficients of $q^{-l}$ in
the product of $G_{12r-h+2}$ with $\Delta^{-r}$ as is indeed seen by
direct calculation, which can be slightly simplified by using the
formula
$$
\prod^{\infty}_{m=1}(1-q^{m})^{3}=\sum^{\infty}_{n=0}(-1)^{n}(2n+1)q^{n(n+1)/2}
$$
Now $c_{hr}=1$ always. Also we easily obtain the values
\begin{xalignat*}{2}
c_{4} &= -240=-2^{4}\cdot 3\cdot 5; &  c_{6}&=504=2^{3}\cdot
3^{2}\cdot 7;\\
c_{8} &= -480=-2^{5}\cdot 3\cdot 5; & c_{10} &= 264=2^{3}\cdot 3\cdot
11;\\
c_{12} &= -196560=-2^{4}\cdot 3^{3}\cdot 5\cdot 7\cdot 13, &
c_{12,1}&=24=2^{3}\cdot 3;
\end{xalignat*}\pageoriginale
$$
c_{14}=24=2^{3}\cdot 3;
$$
\begin{xalignat*}{2}
c_{16} &= -146880=-2^{6}\cdot 3^{3}\cdot 5\cdot 17, & c_{16,1}&=
-216=-2^{3}\cdot 3^{3};\\
c_{18} &= 86184=2^{3}\cdot 3^{4}\cdot 7\cdot 19, & c_{18,1} &=
528=2^{4}\cdot 3\cdot 11;\\
c_{20} &= -39600=-2^{4}\cdot 3^{2}\cdot 5^{2}\cdot 11, & c_{20,1} &=
-456=-2^{3}\cdot 3\cdot 19;\\
c_{22} &= 14904=2^{3}\cdot 3^{4}\cdot 23, & c_{22,1} &= 288=2^{5}\cdot
3^{2};\\
c_{24} &= -52416000=-2^{9}\cdot 3^{2}\cdot 5^{3}\cdot 7\cdot 13, &&\\
c_{24,1} &= -195660=-2^{2}\cdot 3^{2}\cdot 5\cdot 1087, & c_{24,2} &=
48=2^{4}\cdot 3;
\end{xalignat*}
$$
c_{26}=1224=2^{3}\cdot 3^{2}\cdot 17,\qquad c_{26,1}=48=2^{4}\cdot 3.
$$

From this table, we see for example, with $h=12$ and $h=24$, that the
$r$ numbers $c_{hl}(l=0,\ldots,r-1)$ need not all have the same sign,
if $h$ is a multiple of $4$. Such a change of sign always occurs,
moreover, for $h>248$ since, in particular, for this $h$,
$c_{h,r-1}>0$ whereas $C_{h0}<0$. Further, since the number
$c_{h,r-1}$ is $0$ for $h=214$ and $h=248$, the assertion of Theorem
\ref{app-thm2} does not hold, without exception, for all
$c_{hl}(l=1,2,\ldots,r-1)$.


For number-theoretic applications of theorem \ref{app-thm1}, the
prime-factors of $c_{h}$ are of interest. We get some information
regarding these prime factors if in the statement of Theorem
\ref{app-thm1} we consider the Fourier coefficients of
$G_{h}(h=4,6,\ldots)$ given by \eqref{app-eq4} instead of the modular
form $M$. Then we have from \eqref{app-eq14},
\begin{equation*}
\frac{c_{h}B_{h}}{2h}=\sum^{r}_{l=1}c_{hl}\sigma_{h-1}(l),\sigma_{h-1}(l)=\sum_{t|l}t^{b-1}(h=4,6,\ldots)\tag{15}\label{app-eq15} 
\end{equation*}

Since the first sum here is an integer, $c_{h}$ must be divisible by
the denominator of $\dfrac{B_{h}}{2h}$; in any case, the denominator
of the Bernoulli number\pageoriginale $B_{h}$ is a divisor of
$c_{h}$. By the Staudt-Clausen theorem, however, this denominator is
equal to the product of those primes $p$, for which $p-1$ divides $h$.
Thus, for example, for $h=24$, one sees the reason for the presence of
the prime divisors $2$, $3$, $5$, $7$, $13$ of $c_{24}$. On the other
hand, for $h=26$, the factor $17$ of $c_{26}$ cannot be explained in
this simple manner. By \eqref{app-eq15}, we have however explicitly
\begin{align*}
\frac{2\cdot 3^{2}\cdot 17\cdot B_{26}}{13}-\frac{c_{26}B_{26}}{2\cdot
  26} &= 2^{4}\cdot 3\cdot 1^{25}+(1^{25}+2^{25})=33554481\\
&= 3\cdot 17\cdot 657931
\end{align*}
where thus the factor 17 appears on the right side also and we obtain
$$
B_{26}=\frac{13\cdot 657931}{2\cdot 3}
$$
agreeing with the known value.

\section{Modular Forms of Hecke}\label{app-sec2}

Let $K$ be a totally real algebraic number field of degree $g$ and
discriminant $d$ and let further $k$, be a natural number which is,
in addition, even if there exists in $K$ a unit of norm $-1$. We denote
the norm and trace by $N$ and $S$. Let $\mathfrak{u}$ be an ideal,
$\mathfrak{d}$ the different of $K$ and
$\mathfrak{u}^{\ast}=(\mathfrak{u}\mathfrak{d})^{-1}$, the complementary
ideal to $\mathfrak{u}$.

We assign to every one of the $g$ fields conjugate to $K$, one of the
variables $z^{(1)},\ldots,z^{(g)}$ in the upper half-plane and form
with them the Eisenstein series
$$
F_{k}(\mathfrak{u},z)=N(\mathfrak{u}^{k})\mathop{{\sum}'}_{\mathfrak{u}|(\lambda,\mu)}N((\lambda
z+\mu)^{-k})
$$
where $\lambda$, $\mu$ run over a complete system of pairs of numbers
in the ideal $\mathfrak{u}$, different from $0$, $0$ and not differing
from one another by a factor which is a unit. The series converges
absolutely for $k>2$. For the cases $k=1,2$, in order to obtain
convergence, the general term of the series is multiplied, as usual,
by the factor $N(|\lambda z+\mu|^{-s})$, where the real part of the
complex variable $s$ is greater than $2-k$ and $F_{k}(\mathfrak{u},z)$
is defined as the\pageoriginale value of the analytic continuation at
$s=0$. A comparison with (1) shows that $F_{k}(\mathfrak{u},z)$ gives
for $g=1$ and $k=4,6,\ldots$ the function denoted there as
$F_{k}(z)$. Let hereafter $g>1$.

The generalization of (2) gives for the Fourier expansion of
$F_{k}(\mathfrak{u},z)$, the formula
\begin{equation*}
F_{k}(\mathfrak{u},z)=\zeta(\mathfrak{u},k)+\left(\frac{(2\pi
  i)^{k}}{(k-1)!}\right)^{g}d^{1/2-k}\sum_{\mathfrak{d}^{-1}|v \succ 0}\sigma_{k-1}(\mathfrak{u},v)e^{2\pi
  iS(vz)},\tag{16}\label{app-eq16} 
\end{equation*}
where $v$ runs over all totally positive numbers in $\mathfrak{d}^{-1}$ and
\begin{align*}
\zeta(\mathfrak{u},k) &=
N(\mathfrak{u}^{k})\sum_{\mathfrak{u}|(\mu)}N(\mu^{-k}),\\
\sigma_{k-1}(\mathfrak{u},v) &=
\sum_{\mathfrak{d}^{-1}|(\alpha)\mathfrak{u}|v}\sign
(N(\alpha^{k}))N((\alpha)\mathfrak{u}\mathfrak{d})^{k-1}. 
\tag{17}\label{app-eq17}
\end{align*}
Here the summation is over principal ideals $(\mu)$, $(\alpha)$ under
the conditions given. For even $k$, \eqref{17} can be put in the form
\begin{equation*}
\sigma_{k-1}(\mathfrak{u},v)=\sum_{\mathfrak{u}\mathfrak{d}\sim \mathfrak{t}|(v)\mathfrak{d}}N(\mathfrak{t}^{k-1})(k=2,4,\ldots)\tag{18}\label{app-eq18}
\end{equation*}
where $\mathfrak{t}$ runs over all integral ideals in the class of
$\mathfrak{u}\mathfrak{d}$ dividing $(v)\mathfrak{d}$. As a departure from the
case $g=1$, formula \eqref{app-eq16} is valid also for $k=2$, while
for $k=1$, an additional term is to be tagged on. For this reason, it
is provisionally assumed that $k>1$.

The function $F_{k}(\mathfrak{u},z)$ is a modular form of Hecke of
weight $k$, since under the substitutions
$$
z\to \frac{\alpha z+\beta}{\gamma z+\mathfrak{d}}
$$
of the Hilbert modular group, it takes the required factor $N((\gamma
z+\mathfrak{d})^{k})$. If we set all the $g$ independent variables
$z^{(1)},\ldots,z^{(g)}$ equal to the same value $z$ in the upper
half-plane, then form $F_{k}(\mathfrak{u},z)$ one obtains an elliptic
modular form, which we denote by $\Phi_{k}(\mathfrak{u},z)$. Its
weight is clearly $h=kg$. Let its Fourier expansion in powers of
$q=e^{2\pi iz}$ be 
$$
\Phi_{k}(\mathfrak{u},z)=\sum^{\infty}_{n=0}b_{n}q^{n}
$$
where\pageoriginale now the Fourier coefficients $b_{0}$,
$b_{1},\ldots$ are determined, in view of \eqref{app-eq16} by the
formulae
\begin{equation*}
b_{0}=\zeta(\mathfrak{u},k), b_{n}=\left(\frac{(2\pi
  i)^{k}}{(k-1)!}\right)^{g}d^{1/2-k}\sum_{S(v)=n}\sigma_{k-1}(\mathfrak{u},v)
\;\; (n=1,2,\ldots)\tag{19}\label{app-eq19} 
\end{equation*}
Here $v$ runs over all totally positive numbers with trace $n$ in the
ideal $\mathfrak{d}^{-1}$ and $\sigma_{k-1}(\mathfrak{u},v)$ is defined by
\eqref{app-eq17}. For brevity, let us put
\begin{equation*}
\sum_{S(v)=n}\sigma_{k-1}(\mathfrak{u},v)=s_{n}(\mathfrak{u},k) \;\;
(n=1,2,\ldots);\tag{20}\label{app-eq20}  
\end{equation*}
this is a rational integer. Explicitly,
\begin{equation*}
s_{n}(\mathfrak{u},k)=\sum_{\substack{\mathfrak{u}^{\ast}|(\alpha),\mathfrak{u}|\beta\\ \alpha\beta
    \succ 0,S(\alpha\beta)=n}}\sign
(N(\alpha^{k}))N((\alpha)\mathfrak{u}\mathfrak{d})^{k-1}\tag{21}\label{app-eq21} 
\end{equation*}
where the summation is over the principal ideals $(\alpha)$ and the
numbers $\beta$ in the field, subject to the given conditions.

On applying \eqref{app-eq14} with $M=\Phi_{k}(\mathfrak{u},z)$, it now
follows
\begin{equation*}
((k-1)!)^{g}(2\pi
  i)^{-h}d^{k-1/2} \zeta(\mathfrak{u},k) = -c^{-1}_{h} \sum^{r}_{l=1}
  c_{h} s_{l}(\mathfrak{u},k);\tag{22}\label{app-eq22}  
\end{equation*}
here we have to take $h=kg$ and $r=r_{h}$. We note further that
$N(-1)=(-1)^{g}$ and therefore $h$ is always an even number. These
formulae permit the calculation of $\zeta(\mathfrak{u},k)$, if the
ideal $\mathfrak{u}$ and the natural number $k$ are given. They show,
in particular, that the value
$$
\pi^{-kg}d^{k-1/2}\zeta(\mathfrak{u},k)=\psi(\mathfrak{u},k)
$$
is rational and more precisely, its denominator divides
$((k-1)!)^{g}c_{h}$. It is to be stressed here that $c_{h}$ depends
only on $h=kg$ but not on the other properties of the field $K$.

In the special case $g=2$ and thus for a real quadratic field, it was
already known that the number on the left side of \eqref{app-eq22} is
rational and that its denominator divides $4((2k)!)^{2}v^{2k}$ where
however $v$ still depends on $K$, being given by the smallest positive
solution $y=v$ of Pell's equation\pageoriginale $x^{2}-dy^{2}=4$. On
the other hand, with the help of a computing machine, Lang has listed
the values of $\psi(\mathfrak{u},2)$ in 50 different examples for
$k=g=2$ and remarked, in addition, that in every one of the cases
considered, only a divisor of 15 appears as the denominator. It is now
easy to explain this phenomenon since $-2^{-4}c_{4}=15$.

For every fixed totally real algebraic number field $K$, we can,
following \eqref{app-eq3}, look upon the positive rational numbers
$\psi(\mathfrak{u},k)(k=2,4,\ldots)$ as the analogues of the values
$2^{k-1}(k!)^{-1}|B_{k}|$ formed with the Bernoulli
numbers. Unfortunately, however, one cannot expect that for this
generalisation, corresponding recursion-formulae and divisibility
theorems will exist as for the Bernoulli numbers themselves. For, they
are connected with the properties of the function $(e^{2\pi
  iz}-1)^{-1}$ whose partial fraction decomposition also leads to
\eqref{app-eq3}. Indeed, similar to \eqref{app-eq16} we have even the
simpler formulae
\begin{align*}
\sum_{\mathfrak{u}|\mu} N((z+\mu)^{-k})
&=\left(\frac{(2\pi
    i)^{k}}{(k-1)!}\right)^{g}d^{-1/2}N(\mathfrak{u}^{-1})\\
&\quad\sum_{\mathfrak{u}^{\ast}|v>0}N(v^{k-1})e^{2\pi  iS(vz)} \;\; 
(k=2,3,\ldots);
\end{align*}
but in contrast with the case of one variable, the analytic function
appearing here for $g>1$ does not have, any more, important properties
such as the addition theorem and the differential equation enjoyed by
the exponential function. Further by an important result of Hecke, it
is always singular for all real values of any one of the $g$
variables, so that one cannot as in the case $g=1$ obtain the numbers
$\zeta(\mathfrak{u},k)$ by expansion in powers of $z$. We get these
numbers by going to the modular forms $F_{k}(\mathfrak{u},z)$ of
Hecke, which in spite of their complicated construction are really
connected with one another by algebraic relations.

To the result \eqref{app-eq22} we may add a remark which relates to a
lower bound for $d$ discovered by Minkowski. The left side of
\eqref{app-eq22} is clearly different from $0$ for even $k$, since
then $\zeta(\mathfrak{u},k)$ is positive. Consequently among the $r$
sums $s_{l}(\mathfrak{u},k)$ given for $l=1,2,\ldots,r$ by
\eqref{app-eq21}, at least one must be non-empty and hence there
always exist two numbers $\alpha$, $\beta$ in $K$ which satisfy the
four conditions
$$
\mathfrak{u}^{\ast}|\alpha,\mathfrak{u}|\beta,\alpha\beta>0,S(\alpha\beta)\leq
r
$$\pageoriginale
with $r=r_{h}=r_{kg}$. For this assertion, it is optimal to choose
$k=2$ and hence $h=2g$; accordingly
$$
r=\left[\frac{g}{6}\right]+1(g\not\equiv 1(\rm{mod} \;  6)),
r=\left[\frac{g}{6}\right](g\equiv 1(\rm{mod} \;  6))
$$
so that in every case
\begin{equation*}
r\leq \frac{g}{6}+1.\tag{23}\label{app-eq23}
\end{equation*}

From the decompositions
$$
(\alpha)=\mathfrak{u}^{\ast}\mathfrak{t},(\beta)=\mathfrak{ub}
$$
with integral ideals $\mathfrak{t}$, $\mathfrak{b}$, it follows that
there exist two integral ideals $\mathfrak{t}$ and $\mathfrak{b}$ (in
every ideal class and its complementary class), such that 
\begin{equation*}
\mathfrak{tbd}^{-1}=(v),v \succ 0,S(v)\leq r\tag{24}\label{app-eq24}
\end{equation*}

We therefore have
\begin{equation*}
\begin{split}
d^{-1}N(\mathfrak{tb}) &= N(\mathfrak{tbd}^{-1})=N(v)\leq
(g^{-1}S(v))^{g}\leq \left(\frac{r}{g}\right)^{g}\\
N(\mathfrak{tb}) &\leq \left(\frac{r}{g}\right)^{g}d
\end{split}\tag{25}\label{app-eq25}
\end{equation*}
and hence
$$
\Min (N(\mathfrak{t}), N(\mathfrak{b}))\leq
\left(\frac{r}{g}\right)^{g/2}\sqrt{d}
$$
Since especially for every real quadratic field, the ideal
$\mathfrak{d}=(\sqrt{d})$ belongs to the principal class, the classes
of $\mathfrak{t}$ and $\mathfrak{b}$ are inverses of each other
because of the first formula \eqref{app-eq24} and hence the conjugate
ideal $\mathfrak{b}'=(N(\mathfrak{b}))\mathfrak{b}^{-1}$ is equivalent
to $\mathfrak{t}$; further, $g=2$ and so $r=1$. Since
$N(\mathfrak{b})=N(\mathfrak{b}')$, there exists, in every ideal class
of a real quadratic field, an integral ideal of norm $\leq
\dfrac{1}{2}\sqrt{d}$ always.

For arbitrary totally real fields of degree $g\geq 2$, we always have
by \eqref{app-eq25} 
$$
1\leq N(\mathfrak{t})\leq \left(\frac{r}{g}\right)^{g}d,
$$\pageoriginale
from which, by \eqref{app-eq23} the inequality
\begin{equation*}
d\geq \left(\frac{g}{r}\right)^{g}\geq
6^{g}\left(1+\frac{6}{g}\right)^{-g}>6^{g}e^{-6}\tag{26}\label{app-eq26} 
\end{equation*}
follows. Since in the cases $g=2,3,4,5,7$ the number $r=1$, we have
then
$$
d\geq g^{g}
$$
and this is a new result for field degrees $5$ and $7$. Since
$e^{2}>6$, the estimate
$$
d\geq \left(\frac{g^{g}}{g!}\right)^{2}
$$
found by Minkowski is better than \eqref{app-eq26} for all
sufficiently large $g$, but not for a few small $g$ to which set the
given values $5$ and $7$ belong. For $g=4$, it is well-known that 725
is the smallest discriminant of a totally real field while
$$
4^{4}=256, \frac{4^{8}}{(4!)^{2}}=113+\frac{7}{9}.
$$

The special case $g=2$ yields also a connection with the reduction
theory of indefinite quadratic forms due to Gauss. Namely, let
$[\kappa,\lambda]$ be a basis of the ideal $\mathfrak{u}$ and hence
$$
\frac{\lambda}{\kappa}=\frac{b+\sqrt{d}}{2a}, d=b^{2}-4ac,
N(\mathfrak{u})\sqrt{d}=|\kappa \lambda'-\lambda \kappa'|
$$
with rational integers
$$
a=N(\kappa)N(\mathfrak{u}^{-1}),
b=S(\kappa\lambda')N(\mathfrak{u}^{-1}),
c=N(\lambda)N(\mathfrak{u}^{-1}).
$$
Then the conjugate ideal is $\mathfrak{u}'=(\kappa',\lambda')$ and
$(\kappa\lambda'-\lambda\kappa')^{-1}[\lambda',-\kappa']$ gives the
basis of $\mathfrak{u}^{\ast}=(\mathfrak{ud})^{-1}$ complementary to
$[\kappa,\lambda]$. If we now put
$$
\alpha=\frac{\kappa'x_{2}+\lambda'y_{2}}{\kappa\lambda'-\lambda\kappa'},\beta=\kappa
x_{1}+\lambda y_{1}
$$
with\pageoriginale rational integers $x_{1}$, $y_{1}$, $x_{2}$,
$y_{2}$, then
\begin{align*}
\alpha\beta &=
\frac{2\kappa\kappa'x_{1}x_{2}+(\kappa\lambda'+\lambda\kappa')(x_{1}y_{2}+x_{2}y_{1})+2\lambda\lambda'y_{1}y_{2}}{2(\kappa\lambda'-\lambda\kappa')}+\frac{x_{1}y_{2}-x_{2}y_{1}}{2}\\
&=
\frac{2ax_{1}x_{2}+b(x_{1}y_{2}+x_{2}y_{1})+2cy_{1}y_{2}}{2\sign(\kappa\lambda'-\lambda\kappa')\sqrt{d}}+\frac{x_{1}y_{2}-x_{2}y_{1}}{2} 
\end{align*}
and the two conditions
$$
S(\alpha\beta)=1,\alpha\beta \succ 0
$$
go over into
$$
x_{1}y_{2}-x_{2}y_{1}=1,|2ax_{1}x_{2}+b(x_{1}y_{2}+x_{2}y_{1})+2cy_{1}y_{2}|<\sqrt{d}.
$$
But this precisely means that the binary quadratic form
$au^{2}+buv+cv^{2}$ of discriminant $d$ goes over, under the linear
substitution $u\to x_{1}u+x_{2}v$, $v\to y_{1}u+y_{2}v$ of determinant
$1$, into a properly equivalent form whose middle coefficient is
smaller than $\sqrt{d}$ in absolute value. This means again that in
the transformed form both the outer coefficients are of opposite signs
and this is just the essence of reduction theory.

We have still to consider the case $k=1$ excluded hitherto. In this
case \cite{app-key3}, in contrast to \eqref{app-eq19}, the constant term is
$$
b_{0}=\zeta(\mathfrak{u},1)+\zeta(\mathfrak{u}^{\ast},1)
$$
where we have to define
$$
\zeta(\mathfrak{u},1)=N(\mathfrak{u})\lim\limits_{s\to
  0}\sum_{\mathfrak{u}|(\mu)}N(\mu^{-1})N(|\mu|^{-s})
$$
and also change \eqref{app-eq22} correspondingly. This way we do not
obtain the individual numbers $\zeta(\mathfrak{u},1)$ and
$\zeta(\mathfrak{u}^{\ast},1)$ but only their sum. By the way, this is
always $0$ if either $g=2$ or $\mathfrak{u}^{2}\mathfrak{d}$ is a
principal ideal $(\gamma)$ and $N(\gamma)<0$.

\section{Examples}\label{app-sec3}

In conclusion, we give three simple examples for evaluating
$\zeta(\mathfrak{u},k)$ using\pageoriginale \eqref{app-eq22}, where
the result can also be checked in another way.

With $k=g=2$, we have $h=4$, $r=1$ and
$$
N(\mathfrak{u}^{2})\sum_{\mathfrak{u}|(\mu)}N(\mu^{-2})=-\frac{(2\pi)^{4}}{c_{4}d^{3/2}}s_{1}(\mathfrak{u},2)=\frac{\pi^{4}}{15d\sqrt{d}}s_{1}(\mathfrak{u},2);
$$
here $s_{n}(\mathfrak{u},k)$ for $k=2,4,\ldots$ and $n=1,2,\ldots$ is
given by \eqref{app-eq18} and \eqref{app-eq20}. If we take, in
particular, $d=4.79$, $\mathfrak{u}=\mathfrak{b}=(1)$, then
$\mathfrak{d}=(2\delta)$, $v=\dfrac{a+\delta}{2\delta}$ with
$\delta=\sqrt{79}$ and $a=0$, $\pm 1,\ldots,\pm 8$, $-N(a+\delta)=79$,
$2\cdot 3\cdot 13$, $3\cdot 5^{2}$, $2\cdot 5\cdot 7$, $3^{2}\cdot 7$,
$2\cdot 3^{3}$, $43$, $2\cdot 3\cdot 5$, $3\cdot 5$. We now note that
$2=N(9+\delta)$, while the cube of the prime ideal
$\mathfrak{p}=(3,1+\delta)$ gives a principal ideal. For the
determination of the principal ideal divisors $(\tau)$ of
$(a+\delta)$, we have to consider besides the trivial decomposition
$(a+\delta)=(1)\cdot (a+\delta)$, only
$$
(a+\delta)=(9+\delta)\left(\dfrac{a+\delta}{9+\delta}\right)
$$
with odd $a$; hence, for
$$
\sigma_{1}(\mathfrak{o},v)=\mathfrak{o}(a)=\sum_{\mathfrak{o}|(\tau)|(a+\delta)}N((\tau)),
$$
we have the values
\begin{xalignat*}{2}
\sigma(0) &= 1+79=80, & \sigma(\pm 1) &= 1+78+2+39=120,\\
\sigma(\pm 2) &= 1+75=76, & \sigma(\pm 3) &= 1+70+2+35=108,\\
\sigma(\pm 4) &= 1+63=64, & \sigma(\pm 5) &= 1+54+2+27=84,\\
\sigma(\pm 6) &= 1+43=44, & \sigma(\pm 7) &= 1+30+2+15=48,
\end{xalignat*}
$$
\sigma(\pm 8)=1+15=16.
$$
Accordingly, in the present case,
\begin{gather*}
s_{1}(\mathfrak{o},2)=80+2(120+76+108+64+84+44+48+16)=1200\\
\sum_{\mathfrak{o}|(\mu)}N(\mu^{-2})=\frac{10\pi^{4}}{79\sqrt{79}}
\end{gather*}
in\pageoriginale agreement with the value found by me elsewhere. For
$\mathfrak{u}=\mathfrak{p}=(3,1+\delta)$, we obtain correspondingly
\begin{align*}
s_{1}(\mathfrak{p},2) &= 0+(3+6+13+26)+(3+5+15+25)+\\
&\quad +(5+10+7+14)+(3+7+21+9)+\\
&\quad +(3+6+9+18)+0+(3+6+5+10)+(3+5)=240\\
&\qquad\quad
N(\mathfrak{p}^{2})\sum_{\mathfrak{p}|(\mu)}N(\mu^{-2})=\frac{2\pi^{4}}{79\sqrt{79}} 
\end{align*}
and the same for $\mathfrak{u}=\mathfrak{p}'=(3,1-\delta)$. Since the
$3$ ideal classes of the field are represented by $1$, $\mathfrak{p}$,
$\mathfrak{p}'$, we have therefore for the zeta function
$\zeta_{K}(s)$ of the field,
$$
\zeta_{K}(2)=\frac{\pi^{4}}{79\sqrt{79}}(10+2+2)=\frac{14\pi^{4}}{79\sqrt{79}},
$$
while from the law of decomposition, we get, likewise, on using the
formula
\begin{align*}
\sum_{n=1}\left(\frac{d}{n}\right)n^{-k} &= -\frac{(2\pi
  i)^{k}}{2 \cdot
  k!\sqrt{d}}\sum^{d}_{l=1}\left(\frac{d}{l}\right)p_{k}\left(\frac{l}{d}\right),\\
P_{k}(x) &=
\sum^{k}_{l=0}\left(\frac{k}{l}\right)B_{l}x^{k-l}\;\; (k=2,4,\ldots) 
\end{align*}
that
\begin{align*}
\zeta_{K}(2) &=
\zeta(2)\sum^{\infty}_{n=1}\left(\frac{316}{n}\right)n^{-2}\\
&= \frac{\pi^{2}}{6}\frac{(2\pi)^{2}}{2\cdot
  2!\sqrt{316}}\sum^{316}_{l=1}\left(\frac{316}{l}\right)P_{2}\left(\frac{l}{316}\right)\\
&=
\frac{\pi^{4}}{12\sqrt{79}}\frac{1}{2}\sum^{39}_{l=1}\left(\frac{79}{2l-1}\right)\left(1-\frac{2l-1}{79}\right)=\frac{14\pi^{4}}{79\sqrt{79}}. 
\end{align*}
We now take $k=12$, $g=2$; then $h=24$, $r=3$ and 



\begin{align*}
N(\mathfrak{u}^{12})\sum_{\mathfrak{u}|(\mu)}N(\mu^{-12})&=-
\frac{(2\pi)^{24}}{(11!)^{2}c_{24}d^{23/2}}(c_{24,1}s_{1}(\mathfrak{u},12)
+c_{24,2}s_{2}(\mathfrak{u},12)\\
&\qquad+s_{3}(\mathfrak{u},12))\\
&= \frac{\pi^{24}}{2\cdot 3^{10}\cdot 5^{7}\cdot 7^{3}\cdot
    11^{2}\cdot 13d^{11}\sqrt{d}}\\
&\quad(-195660s_{1}(\mathfrak{u},12)+48s_{2}(\mathfrak{u},12)+s_{3}(\mathfrak{u},12)). 
\end{align*}\pageoriginale
In the special case $d=5$, there exists only one class
$\mathfrak{d}=(\sqrt{5})$ and the totally positive numbers $v$ of the
ideal $\mathfrak{d}^{-1}$ with trace $1$, $2$, $3$ are given by
$$\dfrac{\pm 1+\sqrt{5}}{2\sqrt{5}},
\dfrac{a+\sqrt{5}}{\sqrt{5}}(a=0,\pm 1,\pm 2),
\dfrac{a+3\sqrt{5}}{2\sqrt{5}}(a+\pm 1,\pm 3,\pm 5).$$ 
Since
$\dfrac{\pm 1+\sqrt{5}}{2}$, $\pm 2+\sqrt{5}$ are units and also
$$\sqrt{5}, \pm 1+\sqrt{5}, \dfrac{\pm 1+3\sqrt{5}}{2},
\dfrac{\pm 3+3\sqrt{5}}{2}, \dfrac{\pm 5+3\sqrt{5}}{2}$$ are
in-decomposable, we have
\begin{align*}
s_{1}(\mathfrak{o},12) &= 2\cdot 1^{11}=2,
s_{2}(\mathfrak{b},12)=1^{11}+5^{11}+2(1^{11}+4^{11})+2\cdot 1^{11}\\
&= 57216738,\\
s_{3}(\mathfrak{o},12) &=
2(1^{11}+11^{11})+2(1^{11}+9^{11})+2(1^{11}+5^{11})\\
&=633483116696 
\end{align*}
and thus
$$
\sum_{\mathfrak{o}|(\mu)}N(\mu^{-12})=\frac{2^{4}\cdot 691\cdot
  110921\pi^{24}}{3^{10}\cdot 5^{16}\cdot 7^{3}\cdot 11^{2}\cdot
  13\sqrt{5}}
$$

On the other hand, in the present case too, we have
\begin{align*}
\sum_{\mathfrak{o}|(\mu)}N(\mu^{-12}) &= \zeta_{K}(12)\\
&=
\zeta(12)\sum^{\infty}_{n=1}\left(\frac{5}{n}\right)n^{-12}=\frac{(2\pi)^{24}B_{12}}{(2\cdot
  12!)^{2}\sqrt{5}}\sum^{4}_{l=1}\left(\frac{5}{l}\right)P_{12}\left(\frac{l}{5}\right) 
\end{align*}
with the same result, where this time the factor 691 comes in through
the numerator of $B_{12}$.

Finally\pageoriginale let $k=2$, $g=3$ so that $h=6$, $r=1$ and
$$
N(\mathfrak{u}^{2})\sum_{\mathfrak{u}|(\mu)}N(\mu^{-2})=\frac{(2\pi)^{6}}{c_{6}d^{3/2}}s_{1}(\mathfrak{u},2)=\frac{8\pi^{6}}{63d\sqrt{d}}s_{1}(\mathfrak{u},2). 
$$
In particular, let $d=49$ and hence $K$, the totally real cubic
subfield of the field of $7^{\text{th}}$ roots of unity. It is
generated by
$$
\rho=2-\epsilon-\epsilon^{-1}=(1-\epsilon)(1-\epsilon^{-1})
$$
with $\epsilon=e^{2\pi i/7}$ and has class number $1$, as is also
evident from \eqref{app-eq25}. Since $[1,\rho,\rho^{2}]$ is a basis of
$\mathfrak{o}$ and $N(\rho)=7$, we have $(\rho^{3})=7$ and so
$\mathfrak{d}=(\rho^{2})$. We have hence to determine for
$$
v=x+y\rho^{-1}+z\rho^{-2},
$$
the solutions of $S(v)=1$, $v>0$ with rational integers $x$, $y$,
$z$. A short computation gives
\begin{gather*}
\rho^{3}=7(\rho-1)^{2},S(\rho)=7,S(\rho^{2})=21,S(\rho^{-1})=2,S(\rho^{-2})=2,\\
S(\rho^{-3})=2+\frac{3}{7},\\
S(v) =3x+2y+2z=1
\end{gather*}
By the substitution $x=1+2u$ with integral $u$, it follows that
$y=-1-y-3u$ and the condition $v \succ 0$, and consequently $\rho^{2}v
\succ 0$,
goes over into
\begin{equation*}
(2\rho^{2}-3\rho)u+(1-\rho)z+\rho^{2}-\rho>0.\tag{27}\label{app-eq27}
\end{equation*}

If we set again
$$
\epsilon^{k}+\epsilon^{-k}=2\cos \frac{2k\pi}{7}=\lambda_{k}\quad
(k=1,2,3),
$$
then
\begin{align*}
& 1=2\cos \frac{\pi}{3}<\lambda_{1}<2\cos
\frac{\pi}{4}=\sqrt{2}<\frac{3}{2},\\
&
-\frac{1}{2}<-\frac{\pi}{7}<-2\sin\frac{\pi}{14}=\lambda_{2}<2\cos\frac{\pi}{2}=0\\
& -2=2\cos
\pi<\lambda_{3}=-2\cos\frac{\pi}{7}<-2\cos\frac{\pi}{6}=-\sqrt{3}<-\frac{3}{2}. 
\end{align*}\pageoriginale

With this, we obtain from \eqref{27} by a simple modification, the
three conditions
\begin{align*}
& z+\lambda_{1}-2>(\lambda_{2}+5)u, z+\lambda_{2}-2<(\lambda_{3}+5)u,\\
& z+\lambda_{3}-2<(\lambda_{1}+5)u
\end{align*}
whence, it follows, in particular, that
$$
\lambda_{3}<u<\lambda_{1}
$$
Therefore, for $u$, only the values $0$, $\pm 1$ have to be
considered. Of these however, $u=1$ leads to the contradiction
$$
5<\lambda_{2}-\lambda_{1}+7<z<\lambda_{3}-\lambda_{2}+7<6.
$$
For $u=0$, we have
$$
2-\lambda_{1}<z<2-\lambda_{2}<2-\lambda_{3},
$$
and hence either $z=1$, $y=-2$, $x=1$ or $z=2$, $y=-3$, $x=1$ and for
$u=-1$,
$$
\lambda_{3}-2<z<\lambda_{2}-2<\lambda_{1}-2
$$
and hence $z=-3$, $y=5$, $x=-1$. The first of the three solutions
formed above, gives
$$
\rho^{2}v=(\rho-1)^{2},v=\frac{\rho}{7}.
$$
and the other two $v$ must therefore be conjugate to this, as we can
also check directly. In all the three cases, $\rho^{2}v$ proves to be
a unit and hence $(v)\mathfrak{d}=\mathfrak{o}$,
$\mathfrak{t}=\mathfrak{o}$, $s_{1}(\mathfrak{o},2)=3$ and
$$
\sum_{\mathfrak{o}|(\mu)}N(\mu^{-2})=\frac{8\pi^{6}\cdot 3}{63\cdot
  49\cdot \sqrt{49}}=\frac{2^{3}\cdot \pi^{6}}{3\cdot 7^{4}}.
$$

In order to verify this result in another way, we note from the law of
decomposition for the cyclotomic field, that
\begin{align*}
\sum_{\mathfrak{o}|(\mu)}N(\mu^{-2}) &=
\zeta_{K}(2)=(1-7^{-2})^{-1}\prod^{2}_{l=0}(w_{1}+\eta^{l}w_{2}+\eta^{-l}w_{3})\\
&= \frac{7^{2}}{2^{4}\cdot
  3}(w^{3}_{1}+w^{3}_{2}+w^{3}_{3}-3w_{1}w_{2}w_{3})
\end{align*}\pageoriginale
with
$$
w_{k}=\sum^{\infty}_{n=-\infty}(7n+k)^{-2}(k=1,2,3),\eta=e^{2\pi i/3}.
$$
But now
\begin{align*}
& w_{k}=\left(\frac{\pi}{7\sin
  \dfrac{k}{7}}\right)^{2}=\left(\frac{2\pi}{7}\right)^{2}(2-\epsilon^{k}-\epsilon^{-k})^{-1}=\left(\frac{2\pi}{7}\right)^{2}\rho^{-1}\\
&
  w^{3}_{1}+w^{3}_{2}+w^{3}_{3}=\left(\frac{2\pi}{7}\right)^{6}S(\rho^{-3})=\left(\frac{2\pi}{7}\right)^{6}\left(2+\frac{3}{7}\right)\\
&
  w_{1}w_{2}w_{3}=\left(\frac{2\pi}{7}\right)^{6}N(\rho^{-1})=\left(\frac{2\pi}{7}\right)^{6}\cdot\frac{1}{7}  
\end{align*}
and indeed we have again
$$
\sum_{\mathfrak{o}|(\mu)}N(\mu^{-2})=\frac{7^{2}}{2^{4}\cdot
  3}\left(\frac{2\pi}{7}\right)^{6}\cdot
2=\frac{2^{3}\cdot\pi^{6}}{3\cdot 7^{4}}
$$

\begin{thebibliography}{99}
\bibitem{app-key1} \textsc{Meyer, C:} \"Uber die Bildung von
  elementar-arithmetischen Klasseninvarianten in reell-quadratischen
  Zahlk\"orpern {\em Berichte.\@ math.\@ Forsch-Ist.\@ Oberwalfach} Z,
  Mannheim 1966, 165-215.

\textsc{Lang, H:} \"Uber eine Gattung elementar-arithmetischer
Klasseninvarianten reell-quadratischer Zahkk\'orper.\@ Inaug Diss.\@
K\"oln. 1967 VII u.\@ 84 S.

\textsc{Siegel, C.\@ L.:} Bernoullische Polynome und quadratische
Zahlk\"orper. {\em Nachr.\@ Akad.\@ Wiss.\@ Gottigen, Math-Phys.}
K1. 1968, 7-38.

\textsc{Barner, K:} \"Uber die Werte der Ringklassen-$L$-Funktionen
reell-quadratischer Zahlk\"orper an nat\"urlichen
Argumentstellen. {\em J.\@ Numbr.\@ Th.\@ } 1. 28-64 (1969).


\bibitem{app-key2} \textsc{Klingen, H:} \"Uber die Werte der Dedekindschen
  Zetafunktionen, {\em Math.\@ Ann.} 145, 265-272 (1962).

\bibitem{app-key3} \textsc{Gundlach, K.\@ B:} Poincar\'esche und
  Eisensteinsche Reihen 
  zur Hilbertschen Modulgruppe.\@ {\em Math.\@ Zeitschr.} 64, 339-352
  (1956); insbes 350-351.
\end{thebibliography}
