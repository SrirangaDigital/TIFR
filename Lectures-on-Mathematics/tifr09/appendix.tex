\backmatter
\chapter{Appendix I}

\textbf{1.} We\pageoriginale shall, in this appendix, discuss a
certain algebraic 
cohomology and obtain a mapping forms of degree $-2f$ on certain
cohomology classes.


Let $\varphi(\tau)$ be an automorphic form of degree $-2f$ with
respect to a group $\Gamma$ of matrices $\begin{pmatrix}
  a&b\\c&d\end{pmatrix} (a,b,c,d \text{ real})$. 

Consider $\Phi (\tau)=\dfrac{1}{(2f-2)!}\int ^\tau
_{\tau_0}(\tau-\sigma)^{2f-2}\varphi (\sigma)d \sigma$ where $\tau_
\circ$ is a fixed point in $\im (\tau)>0$. The integral exists and by
direct verification $\dfrac{d^{2f-1}\Phi (\tau)}{d\tau ^{2f-1}}=\varphi
(\tau)$. 

Let now $\alpha=\begin{pmatrix} a&b\\c&d\end{pmatrix}\in
\Gamma$. Then, we have, on putting $\sigma=\alpha (\sigma)$, 
\begin{align*}
  \Phi (\alpha(\tau)). \frac{(c\tau
    +d)^{2f-2}}{(ad-bc)^{f-1}}&=\frac{1}{(2f-2)!} \int
  \limits^\tau_{\alpha^{-1}(\tau_ \circ)}\bigg (\frac{a \tau+b}{c \tau
    +d}-\frac{a \sigma' +b}{c \sigma +d} \bigg )^{2f-2}\cdot \\ 
  & \qquad \varphi
  (\sigma)\frac{(c \sigma' +d)^{2f}}{(ad-bc)^f} \frac{(c \tau
    +d)^{2f-2}}{(ad-bc)^{f-1}} \frac{ad-bc}{(c \sigma' +d)^2}d
  \sigma'\\ 
  &=\frac{1}{(2f-2)!}\int \limits^\tau_{\alpha^{-1}(\tau_ \circ)}
  (\tau -\sigma)^{2f-2}\varphi (\sigma')d \sigma'\\ 
  &=\frac{1}{(2f-2)!}\left[~\int \limits^\tau_{\alpha^{-1}(\tau_ \circ)}+
    \int \limits^ \tau_{\tau_ \circ}\right]\\ 
  &=C(\alpha; \tau)+\Phi (\tau)
\end{align*}
where $C(\alpha ; \tau)=\frac{1}{(2f-2)!} \int \limits^
\tau_{\alpha^{-1}(\tau_ \circ)}(\tau -\sigma')^{2f-2}\varphi (\sigma')d
\sigma'$ and hence a polynomial of degree $\leq 2f-2$. 

Defining\pageoriginale \quad $\Phi (\tau). \alpha = \Phi(\alpha(\tau))\dfrac{(c
  \tau+d)^{2f-2}}{(ad-bc)^{f-1}}$ and  
\begin{equation}
  C(\tau).\alpha=C(\alpha(\tau))\frac{(c \tau+d)^{2f-2}}{(ad-bc)^{f-1}} \tag{*}
\end{equation}
we may write the above as
$\Phi(\tau).\alpha=C(\alpha;\tau)+\Phi(\tau)$. In other words, 
$$
C(\alpha, \tau)= \Phi (\tau). \alpha-\Phi(\tau),
$$
and consequently for $\alpha, \beta \in \Gamma$,
\begin{align*}
  C(\alpha \beta; \tau)= \Phi.\alpha \beta -\Phi
  &=(\Phi. \alpha-\Phi). \beta +(\Phi.\beta-\Phi)\\ 
  &=C(\alpha; \tau).\beta +u (\beta; \tau ).
\end{align*}

We may now assume for the same of simplicity that the elements of
$\Gamma$ have determinant $1$. 

We shall now go into the algebraic meaning behind these formulae. 

Let $\mathfrak{M}$ be a representation module of the group $\Gamma$. A mapping
$\alpha \to C(\alpha)$ of $\Gamma$ into the representation module $m$
we shall call a \textit{ $1$-cochain, } and if this cochain satisfies
the equation 
$$
C(\alpha \beta)=C(\alpha). \beta +C(\beta),
$$
then it is \textit{closed} or a \textit{$1$-cocycle}. Special cocycles
$C(\alpha)=m.(\alpha-1)$, ($m$ being fixed element of $\mathfrak{M}$ and $1$
denotes the unit matrix) are called \textit{coboundaries}. Now, the
cocycles forms an additive group $Z$ and the co-boundaries from a
subgroup $B$ of $Z$ and the quotient group $Z/B$ is the \textit{first
  cohomology group} of $\Gamma$ in $\mathfrak{M}$ and its elements are called
\textit{cohomology classes}. 

Take for $\mathfrak{M}$, the vector space of polynomial of degree $\leq
2f-2$ and\pageoriginale the representation of $\Gamma$ in $m$ defined as above by
$(*)$. Then, associated with every $\varphi (\tau)$ we have the
following mapping $\varphi (\tau)\to \{ C(\alpha, \tau)\}$ and the
cochain $C(\alpha):\alpha \to C(\alpha; \tau)$ is a cocycle, as we had
seen already. Thus, to a modular form of degree $-2f$, there
corresponds a cocycle $C(\alpha)$. This cocycle still depends on the
constant $\tau_0$ occurring in the definition of $\Phi (\tau)$. A
change of $\tau _ \circ$ would add a coboundary to $C(\alpha)$, for,
suppose 
\begin{align*}
  \Phi_1 (\tau) &= \frac{1}{(2f-2)!} \int \limits^ \tau _{\tau_1}
  (\tau- \sigma)^{2f-2} \varphi (\sigma)d \sigma \\ 
  &= \frac{1}{(2f-2)!} \left(\int \limits^ {\tau_ \circ} _{\tau_1}+\int
  \limits^ \tau _{\tau_\circ} \cdots\right)\\ 
  &=p(\tau)+\Phi (\tau) \text{ where } p(\tau)=\frac{1}{(2f-2)!}\int
  \limits^ {\tau_ \circ} _{\tau_1} (\tau -\sigma)^{2f-2} \varphi
  (\sigma)d \sigma 
\end{align*}
is a polynomial in $\tau$ of degree $\leq 2f-2$. Therefore
$$
\displaylines{\hfill 
\Phi_1 \circ \alpha=p(\tau). \alpha+ \Phi (\tau). \alpha\hfill \cr
\text{and if}\hfill 
\Phi_1 \circ \alpha- \Phi_1=C_1(\alpha; \tau),\qquad \hfill }
$$
then $C_1(\alpha ;\tau)=p(\tau).(\alpha-1)+C(\alpha ;\tau)$ or
$C_1(\alpha; \tau)-C(\alpha; \tau)=p(\tau).(\alpha-1)$ which shows
that the cocycle $C_1(\alpha)-C(\alpha)$ is a coboundary. 

Consequently, we have now a well-defined mapping $\varphi(\tau)\to (a$
cohomology class). We may even prove that this mapping is onto. For, let
$C$ be a representative of the cohomology class $\bar{C}$ and
$C(\alpha)$ (for $\alpha \in \Gamma)=C(\alpha; \tau)$, a polynomial of
degree $\leq 2f-2$. 

Consider now the series $\psi (\tau)= \sum\limits_{\alpha \in
  \tau}\dfrac{c(\alpha 
  ;\tau)}{(c\tau+d)^{2n}} e^{2 \pi i \mu \alpha (\tau)}$,
($\mu$ a positive integer). Then the series converges for $n$
sufficiently large\pageoriginale (only if $\Gamma$ is the unit group of
$\mathcal{J}= \begin{pmatrix} \mathscr{O}&\mathscr{O}\\q_2\mathscr{O}&
  \mathscr{O} \end{pmatrix}$), and if
$\beta=\begin{pmatrix}a'&b'\\c'&d' \end{pmatrix} \in \Gamma$, we have 
\begin{align*}
  (\psi(\tau).\beta)(c' \tau+d')^{-2n}&= \psi (\beta
  (\tau)). (c'\tau+d')^{2f-2-2n}\\ 
  &= \sum_{\alpha \in \Gamma} \frac{C(\alpha ;\tau).\beta}{(C''\tau
    +d'')^{2n}}e^{2 \pi i\mu \alpha \beta (\tau)} 
\end{align*}
where $\alpha \beta=\begin{pmatrix}a''&b''\\c''&d'' \end{pmatrix} \in
\Gamma, C$ being a cocycle, 
\begin{align*}
  (\psi(\tau).\beta)(c' \tau+d')^{-2n} & = \sum_{\alpha \beta \in
    \Gamma}\frac{C(\alpha \beta;\tau)}{(C''\tau + d'')^{2n}}e^{2 \pi i\mu
    \alpha \beta (\tau)}\\
  & \hspace{2cm}-C(\beta; \tau)\sum_{\alpha
    \beta}\frac{e^{2 \pi i \mu \alpha \beta (\tau)}}{(C''
    \tau+d'')^{2n}}\\ 
  &= \psi(\tau)-C(\beta ;\tau)\chi (\tau)
\end{align*}
$\chi (\tau)$ being a genuine Poincare series, is modular form of
degree $-2n$. Putting $\Phi (\tau)= \dfrac{\psi (\tau)}{\chi (\tau)}$,
we obtain 
$$
\frac{\psi(\beta (\tau))}{\chi (\beta(\tau))} (c' \tau +d')^{2f-2}
=\Phi (\tau)-C(\beta; \tau). 
$$

In other words,
$$
\Phi (\tau) \cdot \beta= \Phi (\tau)-C(\beta; \tau).
$$

On differentiating $\Phi (\tau),(2f-1)$ times and calling $\varphi
(\tau)=\dfrac{d^{2f-1}\Phi (\tau)}{d \tau ^{2f-1}}$ we have the
required modular form $\varphi (\tau)$ of degree $-2f(\varphi (\tau))$
will have poles in general) and we easily see that $C$ is the cocycle
associated with $\varphi(\tau)$. Again, it is to be noted that the
form $\varphi(\tau)$ is independent of the representative cocycle $C$
in the class $\bar{C}$. 

We\pageoriginale have thus established a two-way mapping (not necessarily one-one),
between the space of modular forms of degree $-2f$, and with have the
property that they have no logarithmic singularities and the
cohomology classes defined above. 

\begin{remarks}
  \begin{enumerate}[\rm (1)]
  \item  In the above correspondence, though the mapping $\varphi \to
    \bar{C}$ is unique, the converse mapping $\bar{C}\longrightarrow
    \varphi$ is not uniquely defined. For securing one-one nature, we
    take the space of classes of forms of degree $-2f$ modulo
    $(2f-1)\underset {\bar{..}}{th}$  derivatives of forms of degree
    $+(2f-2)$, because the periods of the integrals of these
    derivatives are $0$. The rank of the modulo of integral forms
    taken modulo the space of $(2f-2)th$ derivatives of forms of
    degree $2f-2$ can be calculated by application of Riemann-Roch
    theorem (Refer M.Eichler, Eine Verallgemienerung der Abelsche
    Integral, Mathenatische Zeitschrift, $1957$). 
    
    \item The above procedure can be generalized to forms not
    necessarily integral but the singularities must be such that no
    logarithmic terms can occur. (Ibid). 
  \end{enumerate}
\end{remarks}

\textbf{2.} We now study the behaviour of $C(\alpha)$ under the
correspondences $T_n$. Let $\nu _i$ be defined as usual and $\alpha
\in \Gamma _{\mathcal{J}}$. Then $\nu_i \alpha =\alpha' \nu_j,
\alpha' \in \Gamma _{\mathcal{J}}$, with $j$ and $\alpha'$ depending
on $i$ and $\alpha$. We defined now,  
\begin{align*}
  \Psi (\tau)= \Phi(\tau).T_n &= \sum^{d_n}_{i=1} \Phi (\tau). \nu_i\\
  &=\sum^{d_n}_{i=1} \frac{(C_i \tau+ d_i)^{2f-2}}{(a_i d_i-b_i
    c_i)^{f-1}} \Phi \left( \frac{(a_i \tau +b_i)}{(c_i \tau +d_i)}
  \right). 
\end{align*}

To the pair $i,\alpha$, there exist $j=j(i,\alpha)$ and $\alpha'
=\alpha' (i, \alpha)$, such that $\nu_i \alpha=\alpha'
\nu_j$. Consequently, 
\begin{align*}
  \Psi (\tau).\alpha &= \sum^{d_n}_{i=1}(\Phi (\tau).\alpha').\nu_j=C'
  (\alpha;\tau)+\Psi (\tau) \text{ with }\\ 
  C'(\alpha)&= \sum^{d_n}_{i=1} C(\alpha'). \nu_j= C(\alpha).T_n
  \text{ (definition)}. 
\end{align*}

Thus\pageoriginale $T_n$ are made endomorphisms of the first cohomology group of
$\Gamma_{\mathcal{J}}$ in the module $\mathfrak{M}$ of polynomial of degree $\leq
2f-2$. We have only to show that $C'(\alpha)$ is closed if $C(\alpha)$
is closed and that coboundaries are mapped onto coboundaries. Indeed,
if $C(\alpha)=C(\alpha-1)$, then  
$$
C'(\alpha)=\sum_i C. (\alpha'-1)\nu_j =\sum_i C. (\nu_i \alpha
-\nu_j)= \left(\sum_i C.\nu_i\right)(\alpha-1) 
$$

Now $\nu_i \alpha \beta=\alpha' \nu_i \beta=\alpha' \beta' \nu_k$ (say). Then
\begin{align*}
  C'(\alpha \beta)&= \sum_i C(\alpha' \beta'). \nu_k \\
  &=\sum_i (C(\alpha'). \beta' + C(\beta')).\nu_k\\
  &= \sum_i C(\alpha')\nu_i. \beta +\sum_i C(\beta')\nu_k
\end{align*}
or in other words, $C'(\alpha \beta)=C'(\alpha).\beta+C'(\beta)$ where
one has to bear in mind that $j$ and $k$ are function of $i$ which
assume all values from $1$ to $d_n$. 

It is easy to verify that the above-defined endomorphisms are in fact
independent of the special choice of the $\nu_i$. 

Then, the natural question is to ask for the trace of this
endomorphisms and this has been calculated even for more general
discontinuous groups $\Gamma$, by $M$.Eichler (Verallegemeinerung der
Ablesche Integrale, Mathematische Zeitschrift, $1957$) 

\chapter{Appendix II}

\textbf{3.} We\pageoriginale shall now speak about some ideas of Heche
and their possible generalizations. 

Let $\Gamma (Q)=\left\{ \begin{pmatrix}a&b\\c&d \end{pmatrix}\right.$ with $c
\equiv 0(Q)$ and $\Gamma
_1(Q)=\left\{ \begin{pmatrix}a&b\\c&d \end{pmatrix}\right. \equiv
\begin{pmatrix}1&0\\0&1 \end{pmatrix}\pmod Q \bigg \}$
be subgroups of the modular group. It can easily be seen that $\Gamma
(Q)/\Gamma_1(Q)\cong$ multiplicative groups of prime residue
classes modulo $Q$. (This quotient group can also be interpreted as
$G(K_1/K)$ where $G(K_1/K)$ denotes the Galio group of the field $K_1$
of functions invariant under $\Gamma_1(Q)$ over $K$, field of
functions invariant under $\Gamma (Q)$). 

Consider the space of modular forms of degree $-2f$ for the groups
$\Gamma_1(Q)$. Let $\varphi_1(\tau), \ldots, \varphi_\alpha
(\tau)$ form a basis of this space. Then, for every $\alpha \in \Gamma
(Q)$, 
$$
\varphi_i (\tau).\alpha =\sum^d_{j=1}C_{ij}(\alpha)\varphi_j(\tau)
$$
gives a representation $\alpha \to c_{ij}(\alpha)$ of the quotient
group $\Gamma(Q)/ \Gamma_1(Q)$. This finite group being
abelian, this representation splits into one- dimensional ones so that
for a suitable basis $\psi_1 (\tau),\ldots, \psi_d (\tau)$, we
have  
$$
\psi_i(\tau).\alpha =\chi_i(\alpha).\psi_i(\tau),
$$
and the representation is given by $\alpha \to \chi
(\alpha)=\begin{pmatrix}\chi (\alpha)\cdots 0\\ \vdots\\ 0 \cdots
  \chi_\alpha (\alpha) \end{pmatrix}; \chi_i-s$ denoting characters of
the group $\Gamma (Q)/\Gamma_1(Q)$. 

In this connection, the study of forms $\varphi$ with a given
character, was made by Hecke (Uber Modulfunktionen und die
Dirichlerschen Reihen mit Eulerscher Produktentwicklug, $I$ and $II$,
Math. Annalen, $1937$.) For example, it can be shown that the
theta-series 
$$
\vartheta_F (\tau) = \sum_{\underbar{X}}e^{\pi i \tau \underbar{X}' F
  \underbar{X}} 
$$
where\pageoriginale $\underbar{X}' F\underbar{X}$ is a positive definite even
quadratic form, is a modular form with the character $\chi(a)=\bigg(
\dfrac{(-1)^f|F|}{a}\bigg)$ (we denote $\chi (\alpha)$ by $\chi (a)$
where $\alpha= \begin{pmatrix}a&b\\c&d \end{pmatrix}$). 

The modular correspondences $T_n$ are defined as in Hecke's paper and
this ring of operator takes holomorphic forms to holomorphic forms and
cusp forms to cusp forms. 

\textbf{4.} Let $\varphi_1(\tau),\ldots,\varphi_\alpha (\tau)$ be basis of
forms of degree $-2f$ with respect to the the group $\Gamma_1(\Omega)$
with character $\chi$. Then we have 
$$
\begin{pmatrix}
  \varphi_1 (\tau)\\ \vdots\\ \varphi_d (\tau).
\end{pmatrix}
T_n=R_f(T_n).
\begin{pmatrix}
  \varphi_1 (\tau)\\ \vdots\\ \varphi_d (\tau).
\end{pmatrix}
\text{ or if } \varphi(\tau)=
\begin{pmatrix}
  \varphi_1 (\tau)\\ \vdots\\ \varphi_d (\tau).
\end{pmatrix}
$$
$$
\varphi{\tau}.T_n=R_f(T_n).\varphi (\tau).
$$

Now, let the Fourier expansion of $\varphi(\tau)$ be $\sum \limits
^\infty _{n= \circ} c_n e^{2 \pi i n \tau/Q}\, c_n$ being column
vectors. 

Consider $\varphi(\tau).T_p,p\,a$ prime. By definition, $T_p=\sum
\limits_i \Gamma(Q). \nu_i$ and for $\nu_i$, we may take 
$$
\displaylines{\hfill 
\left\{
\begin{pmatrix}  p&0\\ 0&1 \end{pmatrix},
\begin{pmatrix}  1&r\\0&p \end{pmatrix}
0 \leq r< p \right\}\hfill \cr
\begin{aligned}
\text{so that}\qquad 
  \varphi(\tau).T_n & = \bar{\chi}(p) \sum^\infty_{n=0} p^fc_n e^{2\pi
    i~np\tau/Q}+1 \sum ^{p-1}_{r= \circ}\sum^\infty_{n=0}
  \frac{1}{p^{f^cn}} e^{2\pi i \frac{n}{\alpha} \frac{n}{Q}\frac{t+r}{p}}\\ 
  &=\bar{\chi}(p)p^f \sum^\infty_{n=0} \frac{c_n}{p} e^{2\pi
    i~np\tau/Q}+ p^{1-f}\sum^\infty_{n=0} c_{np} e^{2\pi in\tau/Q} 
\end{aligned}}
$$
with\pageoriginale the prescription that $c_{n/p}=0$ if $p \nmid n$.

But $\varphi (\tau).T_p =R_f(T_p).\varphi (\tau)=\sum \limits ^\infty
_{n=0}R_f(T_p).c_n e^{2\pi i~np\tau/Q}$ and on comparing the
coefficients of $e^{2\pi i \tau/Q}$ on both sides, we obtain the famous
formula fo Hecke, 
$$
R_f(T_p).c_1=p^{1-f}.c_p \text{ or } c_p=p^{f-1} R_f(T_p).c_1
$$

Using the product formula for $T_n$, we may obtain the same for
arbitrary $n$, as: $C_n=n^{f-1}R_f(T_n).c_1$. 

This helps us to pass from zeta-functions associated with $R_f(T_n)$
to those associated with modular forms; for, if 
$$
\zeta_R(s)=\sum^\infty_{n=1} \frac{R_f(T_n)}{n^s} \text{ and
}\zeta_M(s)=\sum^\infty_{n=1}\frac{c_n}{n^s}, \text{ then
}\zeta_M(s)=n^{f-1} c_1\zeta_R(s). 
$$

In the case of subgroup of the modular group we have a functional
equation for $\zeta _M(s)$ obtained from the behaviour of $\vartheta
(\tau)$ (in this particular case) under the substitution $\tau \to
\frac{-1}{\tau}$ and this gives a functional equation for
$\zeta_R(s)$. But for groups of orders of division algebras, this is
an open problem. 

\textbf{5.} We now make one more application of our modular correspondences.

Consider the theta-series $\vartheta(\tau)=\sum
\limits_{\underline{X}}e^{\pi i \tau \underline{X'}F \underline{X}}$
with a positive form $\underline{X'}F\underline{X}$. This is a modular
form for a suitable subgroup of the modular group. There is a nature
question whether every modular form can be expressed as a linear
combination of such theta-series. The general question is still
unsolved. But, in a special case, Hecke conjectured in $1936$ that all
integral modular forms of degree $-2$ and stufe $q(a$ prime) can be
expressed as a linear combination of theta-series associated\pageoriginale with
quaternary forms. We succeeded in proving this conjecture, a couple of
years ago, (M.Eichler, Crelle's Journal, $1956$, Uber Darstellbarkeit
von Modulformen durch Thetareihen) and proof is based on the equality
of the trace of the representation of correspondence $T_n$, by means
of cusp forms of degree $-2$ and stufe $q$ and by means of
$\vartheta$-series associated with norm forms of a difinite quaternion
algebra. Even in case $2f >2$, Hecke's conjecture can be generalized
by considering the $\vartheta$-series with spherical harmonies and if
we use Selber's trace formula and compare the two traces, we obtain
the following result; all cusp forms of stufe $q$ (a prime) are
representable as a sum of cusp forms for the whole modular group and
generalized $\vartheta$-series with spherical harmonics for definite
quaternion algebras of discriminant $q^2$. 

