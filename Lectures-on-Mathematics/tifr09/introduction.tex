\chapter*{Introduction}
\addcontentsline{toc}{chapter}{Introduction}

\markboth{Introduction}{}

In the theory of modular forms, certain linear operators $T_n$ have
been used by Hecke, Petersson, Maass and others for deter-mining the
coefficients of modular forms and of corresponding Dirichlet
series. Although this idea has already brought rich success, one has
to bear in mind that they are only some special correspondences of
certain algebraic varieties. 

The correspondences which are represented by Hecke's $T_n$ are the
so-called modular correspondences. The latter have also applications
in the theory of complex multiplication, but we shall not speak about
that here. What is common to all these theories, is the general concept
of a correspondence which appears as a connection between certain
subgroups of the modular group. This observation at once leads to a
vast generalization of modular correspondences by replacing modular
groups by other groups, say, by groups of units of an order of a
normal simple algebra or of certain quadratic forms. But here, we
shall restrict ourselves to a very special case, that of units of
orders in a quaternion algebra. Our chief task in this connection will
be to determine the traces of the representations of $T_n$ and in some
case, we shall give them explicitly. 

The first three articles are devoted to the necessary algebraic
background and in \S \ref{chap2:sec4}, we study the group of units of a maximal
order $\mathcal{J}$ in an indefinite quaternion algebra, by exhibiting
it as a group of transformations of the upper half plane onto
itself. It is proved that this group is finitely generated by using
the finite sided nature of the fundamental domain $F$. The hyperbolic
area of this fundamental domain is then computed by using the residue
of the zeta function of $\mathcal{J}$ and from this the genus of $F$
by the application of Gauss Bonnet formal. 

The second part starts with the definition of correspondences in
general and modular correspondences in particular. We prove here the
Euler product formula of the Zeta function of the representation of
modular correspondence $T_n$. We then make a study of the
representations of $T_n$ by Betti groups of a certain Riemann surface
$S_\mathcal{J}$ in \S \ref{chap3:sec8} and herein we give a proof of Lefschetz'
fixed point theorem under certain restrictive assumptions, the
application of which is required later. \S \ref{chap4:sec9} deals with the
connections between the ideal theory of quadratic subfields and this
leads to applications in \S \ref{chap4:sec10}, especially the calculation of the
number of fixed points of $T_n$, with due multiplicity which
essentially reduces to the calculation of the trace of the
representation of $T_n$ as an endomorphism of the first Betti groups
of $S_{\mathcal{J}}$, by the application of Lefschetz, fixed 
point theorem. From this trace formula follow a host of relations
between class numbers of binary quadratic forms. 

We then suggest some problems of interest in an appendix on
automorphic forms in which we also give the formula for the trace of
representation of $T_n$ in the space of modular forms, by using
Riemann matrix. This leads to a proof of Hecke's conjecture on the
representation of modular forms by $\vartheta$-series. 
