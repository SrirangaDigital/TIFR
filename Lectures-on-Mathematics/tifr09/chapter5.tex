\chapter{Automorphic Forms}\label{chap5}

\setcounter{section}{10}
\section{Automorphic Forms}\label{chap5:sec11}

\textbf{1.} Let\pageoriginale $\Gamma$ be a group of (2, 2) real
matrices $\begin{pmatrix} 
  a & b \\ c & d \end{pmatrix}$ such that $ad-bc> 0$. Furthermore,
assume that $\Gamma$ considered as a group of transformations on the
upper half plane, is a ``discontinuous group of the first kind'', i.e.,
it possesses a fundamental domain $F$, which is bounded by a finite
number of sides and has only a finite number of parabolic cusps. 

\begin{defi*}
  An automorphic form $\varphi (\tau)$ belonging to the group
  $\Gamma$ and of degree $-2f(2f$ integral) is a meromorphic function
  $\varphi (\tau)$ in $\im  \tau > 0$, with the property that 
  $$
  \varphi(\tau) = \varphi \left(\frac{a \tau + b}{c \tau +
    d}\right)\left(\frac{ad - bc}{(c \tau + d)^2} \right)^f ~\text{ for
  }~ \begin{pmatrix} a & b \\ c & d \end{pmatrix} \in \Gamma. 
  $$
  $\varphi(\tau)$ is an \textit{integral form} if it is regular in the
  upper half plane including the parabloiccusps of $F$. 
\end{defi*}

By virtue of the transformation formula of $\varphi (\tau)$, one can
associate to $\varphi(\tau)$, the differential $\varphi(\tau).d
\tau^f$, of degree $f$, on the Riemann surface obtained by
identifying the sides of $F$ and adjoining the cusps. 

We shall now give some examples of automorphic forms.

\noindent 
\textbf{1. Poincare Series}%sec 1

The series $\varphi (\tau) = \sum\limits_{\begin{pmatrix} a & b \\ c
    & d \end{pmatrix}\in \Gamma} \dfrac{1}{(c \tau +d)^{2f }}\circ f
\left(\dfrac{a \tau + d}{c \tau + d}\right)$ for a suitable $f
(\tau)$, is called 
a \textit{ Poincare theta-series} and is an automorphic form of degree
$-2f$ for the group $\Gamma$. 

Some\pageoriginale special cases of Poincare Series are discussed in 
\begin{enumerate}[(i)]
\item H. Weyl - Die Idee der Riemannschen Fl\"ache, $1955$, (Page $151$)
\item Ford - Automorphic Functions
\item C. L. Siegel - Ausgewahlte Fragen der Funktionentheorie-$II$\break
  (G\"ottingen) (Page $50$). One of the important special cases of
  Poin\-care Series is the so-called Einstein series, for which
  $f(\tau) \equiv 1$. It was first studied by Hecke (Theorie der
  Eisensteinschen Reihe Hamburger Abhandlungen, $1927$), in the case
  of principal congruence subgroups of the modular group. 
\end{enumerate}

\noindent 
\textbf{2. Theta-series.}%sec 2

\textbf{2.} Let $F(\underbar{X})= \sum \limits^{4f}_{i, k=1} f_{ik}X_i X_k$ be
a positive definite 	quadratic form in an even number $4f$ of
variables, with integral coefficients and further $f_{ii} \equiv
0 \pmod 2, (-1)^{2f}|f_{ik}| =a$ square. Consider now the series 
$$
\vartheta_F (\tau) = \sum_{\underline{X}\text{ integral }} e^{\pi i
  \tau F(\underbar{X})}. 
$$

Then this convergent series defines an automorphic form of degree
$-2f$ for the group $\Gamma(q): (\begin{pmatrix} a & b \\ c &
  d \end{pmatrix}$ modular and $c \equiv\pmod q)$ where $q$ is the
smallest integer for which $q.(f_{ik})^{-1}$ is integral with even
diagonal elements. That it is a modular from was proved by Hermite. 

The above series can obviously be written in the from: $\vartheta
(\tau) = \sum \limits^{\infty}_{n=1} c_n e^{2 \pi \text{ in }c},
c_n$ being the number of representations of $n$ by $\dfrac{1}{2}
F(\underbar{X})$. When $F$ is definite, $c_n$ is finite, and from the
theory of modular correspondences, we can get some information about
$c_n$. 

\textbf{3.} One\pageoriginale can also form a theta series with weights which are
homogeneous polynomials of a certain type. Consider 
$$
\vartheta(\tau, P_\nu, F) = \sum_{\underbar{X}\text{ integral }}P_\nu
(\underbar{X}) e^{\pi i \tau F(\underbar{X})} 
$$
where $P (X_1 \ldots X_{4f})$ is a homogeneous polynomial of degree
$\nu$ in $4f$ variables and satisfies the Laplacian $\sum
\limits_{i,k} f_{ik} \dfrac{\partial^2 P_\nu}{\partial X_i \partial
  X_k}= 0 $ associated with the form $F(\underbar{X})$. Then it was
proved by B. Schoeneberg (Math. Annalen, 1939, Vol. 116, Das Verhalten
von mehrfachen thetareihen fur Modulsubstitutionen) that the series
represent a modular form of degree -$(2f+ \nu)$ for the group
$\Gamma(q)$, $q$ being defined as before. 

\textbf{3.} We may also define theta-series associated with $F(X)$ with
integral coefficient from a totally real algebraic number field $K$. 

But now, this will be an automorphic form with respect to a subgroup
of the corresponding Hilbert group. For example, in a particular case,
when $K =k(\sqrt{d}), d > 0$, 
$$
\vartheta(\tau_1, \tau_2) = \sum_{\underbar{X} \text{ integral in
  }K}e^{2 \pi i (\tau_1 F (\bar{X})+ \tau_2 \overline{F
    (\underbar{X})}} 
$$
($\underbar{X}$ being a vector with $4f$ components and $\nu \to
\bar{\nu}$ denoting the non identity automorphism of $K/k$). 

It can be shown then that $\vartheta(\tau_1, \tau_2)$ converges
absolutely and uniformly in $\im \tau_1 > 0, \im  \tau_2 > 0$. 
\begin{defi*}
  An analytic function $\varphi (\tau_1, \tau_2)$of $2$ complex
  variables, meromorphic in $\im \tau_1 > 0, \im  \tau_2 > 0$, is {\em
    automorphic form} of degree $-2f$ for the group $\begin{pmatrix} a
    & b \\ c & d \end{pmatrix}$ with $a, b, c, d$ integers in $K$ and\pageoriginale
  $ad - bc$, $a$ unit of $K$)), if  
  \begin{align*}
    \varphi (\tau_1, \tau_2) & =\varphi (\varepsilon (\tau_1),
    \bar{\varepsilon} (\tau_2)). \frac{1}{(\gamma \tau_1 +
      \delta)^{2f} (\bar{\gamma} \tau_2 + \bar{\delta})^{2f}}\hspace{2cm}\\ 
    \text{where}\hspace{1cm}\varepsilon & = \begin{pmatrix} \alpha & \beta
      \\ \gamma & \delta \end{pmatrix} \in \Gamma. 
  \end{align*}
\end{defi*}

It can be shown then that the above defined theta-series is an
automorphic form of degree $-2f$ for a subgroup of the Hilbert
modular group of $K$. Such forms are called Hilbert modular forms. 

\textbf{4.} Now, let $\vartheta (\tau_1, \tau_2)$ be an arbitrary Hilbert
modular form of degree $-2f$ (with respect to the whole Hilbert
modular group). 

We shall now reduce $\vartheta (\tau_1, \tau_2)$ to an automorphic
form $\varphi (\tau)$ in one variable $\tau$, by the following
procedure. 

Putting $\tau_1 = \tau$ and $\tau_2 = \dfrac{-1}{\tau}, r > 0$, a
rational number so that $\im (\tau_1) > 0$ implies $\im (\tau_2) > 0$,
we shall replace $\tau_1, \tau_2$ by $\tau' = \varepsilon (\tau)$ and
$\tau'' = \bar{\varepsilon}(\dfrac{-1}{r \tau})$. We then seek for
conditions on $\varepsilon$ such that $\tau'' = - \dfrac{1}{r
  \tau'}$. In other words, 
$$
\frac{\bar{\alpha} \left(- \frac{1}{\gamma
    \tau}+\bar{\beta}\right)}{\bar{\gamma}(- \frac{1}{\gamma
    \tau})+\bar{\delta}}= \frac{- (\gamma \tau +
  \delta)}{\gamma(\alpha \tau +\beta)} or, \pm \begin{pmatrix} \gamma
  \bar{\beta} & - \bar{\alpha} \\ \gamma \bar{\delta} &
  -\bar{\gamma} \end{pmatrix} = \begin{pmatrix} -\gamma & -\delta
  \\ \gamma \alpha & \gamma \beta \end{pmatrix} 
$$
so that $\varepsilon = \begin{pmatrix} \alpha & \bar{\beta} \\ \gamma
  \bar{\beta} & - \bar{\alpha} \end{pmatrix}$. If $r$ is an integer
and if $\alpha, \beta$ are also integers, then $\varepsilon$ is a unit
in some order of an indefinite quaternion algebra defined by  
$$
Q = [1, \omega, \Omega, \omega \Omega]
$$
with $\omega^2 = d > 0$ and $\Omega^2 = r > 0$. We shall call the unit
group of this order by $\Gamma$. Define now $\varphi (\tau ) =
\vartheta (\tau, \dfrac{-1}{\gamma \tau}).~ \tau^{-2f}$, then $\varphi
(\tau)$ is an automorphic from of degree - $4f$with respect to the
group\pageoriginale $\Gamma$, for 
\begin{align*}
  \varphi(\varepsilon(\tau)) & = \vartheta \left(\varepsilon(\tau), -
  \frac{1}{\gamma \varepsilon(\tau)}\right) (\varepsilon(\tau))^{-2f}\\ 
  & = \vartheta \left(\varepsilon (\tau),
  \bar{\varepsilon}\left(-\frac{1}{\gamma \tau}\right)\right)
  (\varepsilon (\tau))^{-2f}\\  
  & = \vartheta\left(\tau, -\frac{1}{\gamma \tau}\right) (\gamma \tau +
  \delta)^{2f} \left(\bar{\gamma} \left(\frac{-1}{\gamma \tau}\right) +
  \bar{\delta}\right)^{2f}(\varepsilon(\tau ))^{-2f}\\ 
  & = \vartheta \left(\tau, -\frac{1}{\gamma \tau}\right) (\gamma \tau +
  \delta)^{2f}\left(\frac{\beta}{\tau}+\alpha\right)^{2f}\left(\frac{(\gamma
    \tau+\delta)}{(\alpha \tau + \beta)}\right)^{2f}\\ 
  &= \vartheta \left(\tau,  -\frac{1}{\gamma \tau}\right) \left(
  \gamma \tau + \delta \right)^{4f} \tau^{-2f}\\ 
  &= \varphi(\tau) \left( \gamma \tau + \delta \right)^{4f}
\end{align*}

The Hilbert modular form $\vartheta(\tau_1, \tau_2)$ being periodic of
periods $\alpha, \bar{\alpha}(\alpha $ being an integer in $K =
k(\sqrt{d}) (d > 0))$ it has a series expansion of the form  
$$
\vartheta (\tau_1, \tau_2) = \sum_{\substack{ \nu, \bar{\gamma}> 0
    \\ v \in \theta^{-1}}} c_\gamma e^{2 \pi i (\tau_1 \gamma + \tau_2
  \bar{\nu})}(\theta \text{ is the different of } K) 
$$
so that on replacing $\tau_1 = \tau$, ~ $\tau_2 = -\dfrac{1}{r \tau}$,
we have 
$$
\vartheta \bigg(\tau, - \frac{1}{\gamma \tau}\bigg) = \sum
_{\substack{v, \bar{v} > o \\ { \gamma \in \theta^{-1}}}} C_\gamma
e^{2 \pi i(v \tau - \bar{\nu}(\frac{1}{\gamma \tau}))} 
$$

Consider the substitution $\tau \to \lambda (\tau)$ where $\lambda
= \begin{pmatrix} \varepsilon & 0 \\ 0 &
  \varepsilon^{-1} \end{pmatrix}$. Then $\vartheta(\varepsilon^2
\tau_1, \bar{\varepsilon}^2 \tau_2)= \vartheta(\tau_1, \tau_2)$
implies that 
\begin{align*}
  \sum_\nu c_\nu e^{\pi i (\tau_1 \varepsilon^2 \nu + \tau_2
    \bar{\varepsilon^2} \nu)} & = \sum_\nu c_{\nu \varepsilon^{-2}} e^{2
    \pi i (\tau_1 \nu  + \tau_2 \bar{\nu)}}  \\ 
  & = \sum_\nu c_\nu  e^{2 \pi i (\tau_1 \nu  + \tau_2 \bar{\nu)}}
\end{align*}
and by uniqueness of the expansion, on comparison of coefficients,
$c_\nu = c_{\nu \varepsilon^{-2}}$ for every unit $\varepsilon$ of
$K$. 

We\pageoriginale may therefore write
$$
\vartheta\left(\tau, - \frac{1}{r \tau}\right) = \sum_{(\nu)}c_{(\nu)}P_\nu (\tau)
$$
where $(\gamma)$ denotes the class of all $\nu \varepsilon^{-2}$ and
$P_\nu (\tau) = \sum \limits_{\varepsilon} e^{2 \pi i \left( \nu
  \varepsilon^2 \tau - \tau \varepsilon^2 \frac{1}{r \tau}\right)}$
the summation over 
$\varepsilon$ running over all powers of $\varepsilon_\circ$, the
fundamental unit of $K$. 

From the definition of $P_\nu (\tau)$, we have the following product formula,
$$
P_\nu (\tau). P_\mu (\tau) = \sum^{\infty}_{s=-\infty}P_{\nu + \mu
  \varepsilon_\circ}2s(\tau) 
$$
where $\varepsilon_\circ$ is the fundamental unit of $K$. For,
$$
\displaylines{\hfill 
  P_\nu (\tau) = \sum^{\infty}_{n= -\infty} e^{2 \pi i} \left(\nu
  \varepsilon^{2n}_\circ \tau -
  \bar{\nu}\varepsilon^{2n}_\circ\left(\frac{1}{r \tau}\right)\right)\hfill \cr
  \text{and}\hfill 
  P_\mu (\tau) = \sum^{\infty}_{m = -\infty} e^{2 \pi i} \left(\mu
  \varepsilon^{2m}_\circ \tau - \mu
  \varepsilon^{2m}_\circ. \left(\frac{1}{r \tau}\right)\right),\quad \hfill }
$$
so that
\begin{align*}
  P_\nu (\tau)P_\mu (\tau)  & = \sum_{n,m}e^{2 \pi i ((\nu
    \varepsilon^{2n}_\circ + \mu \varepsilon^{2m}_\circ) \tau -
    \overline{(\nu \varepsilon^{2n}_\circ + \mu
      \varepsilon^{2m}_\circ)}(\frac{1}{r \tau}))}\\ 
  & = \sum^{\infty}_{s=-\infty} \sum_n e^{2 \pi i ((\nu + \mu
    \varepsilon^{2s}_\circ) \varepsilon^{2n}_\circ (\tau)
    -\overline{(\nu + \mu
      \varepsilon^{2s}_\circ})\varepsilon^{-2n}_\circ (\frac{1}{r
      \tau}))}\\ 
  & = \sum^\infty_{s=-\infty} P_{\nu+ \mu \varepsilon^{2s}_\circ}(\tau)
  \text{ where } \varepsilon^{2s}_\circ. \varepsilon^{2n}_\circ =
  \varepsilon^{2m}_\circ. 
\end{align*}

If $\Gamma$ denotes the unit group of an order of the quaternion
algebra $Q$ we introduced before, then we had seen that the
function $f(\tau) = \tau^{-2f} \vartheta\left(\tau, - \dfrac{1}{r \tau}\right)$
is an automorphic form of degree $-4f$ for $\Gamma$. 

Now, the above expression for $\vartheta \left(\tau, \dfrac{-1}{r \tau}\right)$
leads at once to the following: 
$$
\varphi (\tau)= \tau^{-2f} \sum_{(\nu)}c_{(\nu)}p_\nu (\tau)
$$
with\pageoriginale $P_\nu-s$ having the above property. A natural question arises
now namely, is it true that \textit{any} from $\psi (\tau)$ for the
group $\Gamma$ has an expansion of the above type and if so, is such
an expansion unique? This question is still unsolved. 

\section*{Petersson metric.}

\textbf{5.} We shall now introduce the inner product $(\varphi,\psi)$ in the
space of automorphic forms of degree $-2f$, which was first defined by
Peterson in $1939$ (Math. Annalen, Vol. $116$, page $406$). If $Q$ is a
division algebra, $\varphi$ and $\psi$ may be arbitrary, but in case
of $Q$ being a matrix algebra or when $\Gamma$ is a subgroup pf the
module  of the modular group, either $\varphi$ or $\psi$ must be a
cusp form. (This condition becomes necessary for the convergence of
the integral we are going to define). For two such forms, $\varphi,
\psi$, 
$$
(\varphi,\psi)= \int _F \varphi (\tau) \,\overline{\psi (\tau)}~ y^{2f}~
\frac{dx dy}{y^2} 
$$
where $F$ is a fundamental domain in the hyperbolic plane for the
concerned group. 

Now, if $F_1$ is any domain in the hyperbolic plane and if we replace
$F_1$ by $\varepsilon F_1$, then the above integral taken over $F_1$
or $\varepsilon F_1$ is the same where $\varepsilon
= \begin{pmatrix}a&b\\ c&d \end{pmatrix}\in \Gamma $. For 
\begin{align*}
  (\varphi, \psi)&= \int_{F_1}\varphi (\tau). \overline {\psi
    (\tau)}. y^{2f} \frac{dx~dy}{y^2}\\ 
  &= \int_{F_1}\varphi (\varepsilon(\tau))
  \left(\frac{ad-bc}{(c\tau+d)^2}\right)^f \overline {(\psi
    (\varepsilon)(\tau))} 
  \left(\frac{\overline{ad-bc}}{(c \tau+d)^2}\right)^f y^{2f}\\
  & \hspace{5cm} .\left(\frac{|c \tau
    +d|^2}{ad-bc}\right)^f \frac {dx' ~dy'}{y'^2}  
\end{align*}
\noindent 
$
\begin{aligned}
\text{where}   \qquad  y' &= \im(\varepsilon (\tau))=\im \tau. \frac{ad-bc}{|c
    \tau +d|^2}=y.\frac{ad-bc}{|c \tau +d|^2}\\ 
  &=\int \limits _{\varepsilon F_1}\varphi (\tau)\overline{\psi
    (\tau)}. y^{2f}. \frac{dx' ~dy}{y'^2}, \tau '=\varepsilon (\tau). 
\end{aligned}
$\pageoriginale

Thus, if the fundamental domain is split up as $F=F_1 \cup \cdots \cup
F_n$ and if $F'=\varepsilon_1F_1 \cup \cdots \cup \varepsilon_n F_n$
where $\varepsilon_i \in
\Gamma$, then $(\varphi,\psi)$ taken over $F$ or $F'$ is the same. 

\section{Representation of \texorpdfstring{$T_n$}{Tn} by Automorphic
  Forms}\label{chap5:sec12}%sec 12 

\textbf{6.} Let $T_n=\sum \limits _i \Gamma _j \nu_i, \nu_i =\begin{pmatrix}
a_i&b_i\\ c_i &d_i \end{pmatrix}$ with $n(\gamma_i)=n$. If $\varphi
(\tau)$ is an automorphic form fo degree $-2f$ for the group $\Gamma
_\tau$, we define 
$$
\varphi (\tau).T_n=n^f. \sum_i \varphi (\frac{a_i \tau +b_i}{c_i \tau
  +d_i}).(c_i \tau +d_i)^{-2f}=\psi (\tau)\text{ (say)}. 
$$

Then $\psi (\tau)$ is again an automorphic form of degree $-2f$ for
the group $\Gamma_\tau$. Further one can show that an integral forms
goes to an integral forms by $T_n$ and a cusp form to a cusp form. The
cusp forms are of special interest to us. Now, since the cusp forms of
degree $-2f$ form a finite dimensional vector space, let $\varphi
_1(\tau),\ldots,  \varphi _d (\tau)$ be one basis for the same.Then 
\begin{align*}
  \varphi _i (\tau). T_n & = \sum^d_{j=1} c_{ij} \varphi_j(\tau)
  ~\text{(say)  or }\\ 
  \begin{pmatrix}\varphi_1 (\tau)\\\vdots\\\varphi _d (\tau)\end{pmatrix}
  T_n &= (c_{ij})
  \begin{pmatrix} \varphi_1 (\tau)\\\vdots\\\varphi _d (\tau). \end{pmatrix}
\end{align*}

In order words, this gives rise to a representation of the ring of
modular correspondences $\\mathfrak{R}= \{ T_n \}$, namely $T_n\to
(c_{ij})=R_f(T_n)$. 

We\pageoriginale shall now prove that the above representation matrix is hermitian
if we choose the basis $\varphi_1, \ldots, \varphi_d$ to be
orthonormal with respect to the Petersson metric. It is enough to show
that 
$$
(\varphi \circ T_n,\psi)= (\varphi,\psi \circ T_n).
$$

For, then $\varphi_1, \ldots, \varphi_d$ being orthonormal,
$(\varphi_i,\varphi_j)=\delta_{ij}$, so that if 
\begin{align*}
  \varphi_i \circ T_n & = \sum^d_{j=1} m_{ij} \varphi_j, \quad \text{ then }\\
  (\varphi_i \circ T_n, \varphi_k) & = (\varphi_i,\varphi_k \circ T_n) \quad
  \text{implies that}\\ 
  \left(\sum_j m_{ij}\varphi_j,\varphi_k\right)& = \left(\varphi_i,\sum_j
  m_{m_{kj}}\varphi_j\right) \quad \text{ or } \quad m_{ik}= \bar{m}_{ki}. 
\end{align*}
i.e., the matrix $(m_{ij})$ is hermitian so that all its eigen-values
are real. Now, because of the product formula for $T_n$,  it is
sufficient to prove that 
\begin{align*}
  (\varphi \circ T_p, \psi) &= (\varphi, \psi \circ T_p)\\
  (\varphi \circ T_p, \psi) &= \sum_i \int_F \varphi (\nu_i
  (\tau)). \overline{\psi(\tau)}. y^{2f} \frac{dxdy}{y^2}(c_i \tau
  +d_i)^{-2f} 
\end{align*}

where $F$ is a fundamental domain for $\Gamma$.

If $\Gamma (p)$ be the group of matrices $\begin{pmatrix}\alpha &\beta
  \\ \gamma &\delta \end{pmatrix} \in \Gamma $ with the property
$\begin{pmatrix}\alpha &\beta \\ \gamma
  &\delta \end{pmatrix}\equiv \begin{pmatrix} 1&0\\0&1 \end{pmatrix}
\pmod p$ 
$$
(\varphi \circ T_p,\psi)_ \Gamma =\frac{1}{g(p)}(\varphi \circ
T_n,\psi)_{\Gamma(p)} 
$$
where $g(p)$ is the index of $\Gamma(p)$ in $\Gamma$.

Now $\varphi_i (\tau)=\varphi(\nu_i(\tau)).(c_i \tau+d_i)^{-2f}$ is an
automorphic forms of degree $-2f$ for the group $\nu^{-1}_i \Gamma
\nu_i$ and since $g(p)$ is also the\pageoriginale index of $\Gamma(p)$ is each
$\nu^{-1}_i \Gamma \nu_i$, we have 
$$
(\varphi \circ T_p,\psi)= \frac{1}{g(p)}\sum_i (\varphi_i, \psi)_{\Gamma (p)}.
$$

On replacing $\tau$ by $\nu^{-1}_i (\tau)=\bar{\nu}_i (\tau)$ and
observing that for a suitable system of $\nu_i$, when $\nu_i$ runs
over a system of non-associated integral elements of $\mathcal{J}$ of
norm $p,\bar {\nu}_i$ also does the same, we obtain 
$$
\frac{1}{g(p)}\sum_i (\varphi_i,\psi)_{\Gamma(p)}=\frac{1}{g(p)}\sum_i
(\varphi,\psi_i)_{\Gamma(p)} 
$$
where $\psi_i=\dfrac{\psi (\bar{\nu}_i(\tau))}{(-c_i\tau
  +a_i)^{2f}}$. Then 
$$
\frac{1}{g(p)}\sum_i (\varphi,\psi_i)_{\Gamma(p)}=\frac{1}{g(p)}
(\varphi,\psi \circ T_p)_{\Gamma(p)}=(\varphi,\psi \circ T_p) 
$$

Consider integral modular forms $\varphi(\tau)$ of degree
$-2f=-2$. Then $\varphi(\tau)d \tau=du$ becomes a differential
holomorphic at all points of the surface $S_\mathcal{J}$ except
perhaps at the vertices and cusps. In the neighbourhood of an elliptic
vertex of order $n$, for the differential $du$ to be holomorphic, we
required that $\dfrac{\varphi(\tau)}{(\tau-\tau_ \circ)^{n-1}}$ be
holomorphic, since 
$$
\varphi (\tau). d(\tau-\tau_\circ)=\frac{\varphi(\tau).d (\tau-\tau_
  \circ)^n}{n(\tau-\tau _ \circ)^{n-1}}. 
$$

In the neighbourhood of the cusps, for example, at $\infty, \varphi
(\tau)d \tau=\psi (\tau)d(e^{2\pi i \tau})\,(e^{2\pi i \tau}$ is the
local uniformizer) and for the differential $du$ to be holomorphic we
require that $\psi(\tau)$ be holomorphic or $\varphi(\tau)e^{2\pi i
  \tau}$ be holomorphic. In other words, $\varphi(\tau)$ must have a
zero at $\infty$. 

Similarly\pageoriginale at all cusps. Therefore we can look upon the space of cusp
forms of degree $-2$ belonging to the group $\Gamma_ \mathcal{J}$ as
the space of differentials of the first kind on the surface
$S_{\mathcal{J}}$. The space of cusps forms being invariant under
$T_n$, we have a representation of $T_n$ in the space of differentials
of the first kind on $S_{\mathcal{J}}$. We shall now find explicitly
the trace of this representation. 

Let $\varphi_1 (\tau)d \tau,  \ldots,  \varphi_g (\tau)d \tau$ be a
base for the space of differentials of the first kind and $c_1,
\ldots,c_{2g}$ be a system of representatives of a basis for the first
homology group. Then, setting $ \nu_j = \begin{pmatrix}
  a_j&b_j\\ c_j&d_j,\end{pmatrix}$ 
\begin{align*}
  \int_{c_i}\varphi_k (\tau). T_n d\tau &= \sum_j \int_{c_i}\varphi_k
  \left(\frac{a_j\tau+b}{c_j \tau+ d_j}\right).d
  \left(\frac{a_j\tau+b}{c_j \tau+ d_j}\right)\\ 
  &=\sum_i \int \limits_{\nu_j(c_i)}\varphi_k (\tau)d \tau=\int
  \limits_{c_i.T_n}\varphi_k (\tau)d \tau. 
\end{align*}

Let $\varphi_k (\tau) \circ T_n= \sum \limits^g_{t=1}m_{kt} \varphi_t$ and
$c_i \circ T_n=\sum\limits^{2g}_{j=1} c_j M_{ij}$ with $M_{ji}$ integers. Then
$\sum\limits^g_{t=1}m_{kt}r_{ti}=\sum \limits^{2g}_{j=1}r_{kj}M_{ji}$
where $R= \bigg ( \int \limits_{c_i} \varphi_k (\tau)d \tau) \bigg
)=(r_{ki})$ is the period matrix (Riemann matrix) of $S_
\mathcal{J}$. From above we have 
$$
mR=RM \text{ or } \overline{mR}= \bar{R}M
$$
since the elements $M$ are integers.

On denoting the $(2g,2g)$ matrix $\left(\dfrac{R}{R}\right)(\bar {R}$ denoting
the conjugate of $R$) by $p$, we have $\begin{pmatrix} m&0 \\0& 
  \bar{m}\end{pmatrix} P=PM \text{ or } 2tr (m)=tr(M)$, 
since\pageoriginale $P$ is known to be non-singular. In other words,
$$
\fbox{$2tr_1 (T_n)= \tr^1(T)_n$.}
$$

\begin{note} 
  Even when $2f>2$ (but $\Gamma$ has to be the full
  modular group), a formula for $tr(R_f(T_n))$ has been found by Selberg
  (Report of the International Colloquium on Zeta-sanctions, page
  $85$). In our case, this method gives
  \begin{multline*}
    \tr(R_f(T_n))=\frac{1}{n^{f-1}} \Bigg[ -2^{k_2+1} \sum_{\substack {d|n
          \\{d\leq \sqrt{n}}}} d^{2f-1_-}\sum_{t,f} \prod _{p|q_1}\left(1-\left\{
      \frac{\Delta}{p}\right\} \right)\\
      \prod _{p|q_2}\left(1+\left\{ \frac{\Delta}{p}\right\}\right)
      \frac{h(\Delta)}{\omega(\Delta)}\cdot
      \frac{\eta^{2f-1_-}\bar{\eta}^{2f-1}}{\eta-  \bar{\eta}} 
  \end{multline*}
  where $\eta =\dfrac{1}{2}(t+\sqrt{\Delta})$.
\end{note}
  \begin{center}
    (The trace need not be an integer).
  \end{center}
