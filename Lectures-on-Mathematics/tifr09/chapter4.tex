\chapter{Ideal Theories in \texorpdfstring{$Q$}{Q} and in Quadratic Subfields}\label{chap4}%chap IV

The\pageoriginale following considerations serve as a tool for calculating the
number of fixed points of correspondences. Besides, there are other
applications. 
\setcounter{section}{8}
\section[Connections Between Ideals in...]{Connections Between Ideals in \texorpdfstring{$Q$}{Q} and in Quad\-ratic
  Subfields}\label{chap4:sec9}%sec 9

\textbf{1.} We consider an order $\mathcal{J}$ of the type $(q_1, q_2)$ for
which we know that the class number is $1$. Further, for such an
order, any left integral ideals $\mathfrak{M}$ such that
$(n(\mathfrak{M}), q_2) = 1$ can be written in the form $\mathfrak{M}
= \mathcal{J} \begin{pmatrix} n_1~ ~n_2 \\ 0 ~~ n_3 \end{pmatrix}$
(here $q_1 = 1), n_2$ being reduced $\mod n_3$. But, suppose
$(n(\mathfrak{M}), q_2) \neq 1$. Then $n (\mathfrak{M}) = p^r. u, p |
q_2$ and $u, a$ p-adic unit. We consider in the following structure of
$\mathfrak{M}_p$ for $p | q_2$. 

\setcounter{theorem}{0}
\begin{theorem}\label{chap4:sec9:thm1}%\the 1
  {\em For $\mathfrak{M}_p = \mathcal{J}_p.  \nu$, we have the
    following normal forms.} i.e., 
  \begin{enumerate}[\rm (1)]
  \item There exists a unit $\varepsilon$ of $\mathcal{J}_p$ such that 
    
    $\varepsilon  \nu = \begin{pmatrix} p^a & c \\ 0 &
    p^b \end{pmatrix}, r = a + b $ and $c$ is reduced $\mod p^a$. 
    
    or     
  \item There exists a unit $\varepsilon'$ of $\mathcal{J}_p$ so that
    $\varepsilon ' \nu = \begin{pmatrix} 0& p^a  \\  p^{b + 1}&
    c \end{pmatrix}, r = a + b  + 1$ and $c$ is reduced $\mod p^{a + 
    1}$. 
  \end{enumerate}
\end{theorem}

\begin{proof}
  Let the generator $\nu$ be $\begin{pmatrix} n_{11} ~~ n_{12}
    \\ pn_{21}~~ n_{22} \end{pmatrix}$. We now distinguish two cases:  
  \begin{enumerate}[(i)]
  \item $\dfrac{n_{21}}{n_{11}}$ is a p-adic integer, or
  \item $\dfrac{n_{21}}{n_{11}}$ is not a p-adic integer.
  \end{enumerate}
\end{proof}

\begin{case}[(i) ]\label{chap4:sec9:thm1:case1}%cae i
  Our\pageoriginale object is to find a unit $\varepsilon = \begin{pmatrix}
    e_{11}~~e_{12} \\ pe_{21} ~~ e_{22} \end{pmatrix}$ such that
  $\varepsilon \nu$ is of type $1)$. 
\end{case}

Now
$$
\varepsilon \nu = \begin{pmatrix} e_{11}~n_{11} + e_{12} ~ pn_{21}
  \qquad e_{11} n_{12} + e_{12} n_{22} \\ p(e_{21} n_{11} + e_{22}
  n_{21}) \qquad pe_{21} ~ n_{12}  + e_{22} n_{22} \end{pmatrix} 
$$
Letting $n_{11} = p^s. u_1$ and putting $e_{21} = \dfrac{n_{21}}{p^s}$
we have $e_{22} = - \dfrac{n_{11}}{p^s}$. Since $pe_{21}$ and $e_{22}$
are coprime as p-adic integers ($e_{22}$ being a unit) this row can be
completed to a p-unimodular matrix $\varepsilon$. Now 
$$
\varepsilon \nu = \begin{pmatrix} p^a.  u^/_1&  c^{'} \\ 0 & p^b.
  u'_2 \end{pmatrix}. 
$$

Again
$$
\begin{pmatrix} u'^{- 1}_1 &  0 \\ 0 &  u'^{-1}_2 \end{pmatrix}
\varepsilon \nu = \begin{pmatrix} p^a &  c \\ 0 &  p^b \end{pmatrix}, 
$$
$a + b = r$. For reducing $c \mod p^a$, we may multiply
$ \begin{pmatrix} p^a & c \\ 0 & p^b \end{pmatrix}$ to the left by a
unit $ \begin{pmatrix} 1 & t \\ 0 & 1 \end{pmatrix}$ for a suitable
$t$.
 
\begin{case}[(ii) ]\label{chap4:sec9:case2}%case ii 
  $n_{11} = p^s.  u_1$ and $n_{21} = p^1.  u_2$ (say). Now, by
  hypothesis, $s > 1$. Putting 
  $$ 
  e_{11} = p.  \frac{n_{21}}{p^{l+1}},  e_{12} = - n_{11} \frac{1}{p^{l+1}},
  $$
  we may complete $(e_{11}\, e_{12})$ to a unimodular matrix
  $\varepsilon$ belonging to $\mathcal{J}_p$ and we have 
  $$
  \varepsilon \nu =  \begin{pmatrix} 0 & u'_2.  p^a \\ u'_1.  p^{b +
      1} & c' \end{pmatrix},  a + b + 1 = r. 
  $$
\end{case}

As\pageoriginale before, $\begin{pmatrix}  u'^{- 1}_2 & 0  \\ 0 &
  u'^{-1}_2 \end{pmatrix} \varepsilon \nu = \begin{pmatrix} 0 & p^a
  \\  p^{b + 1} & c \end{pmatrix} $  and on multiplication to the left
by a unit of the form $\begin{pmatrix} 1& 0 \\ pt & 1 \end{pmatrix}$,
we can reduce $c \mod p^{a + 1}$. 

\noindent \textbf{Note:}
  As a consequence, the number of integral ideals with respect to the
  order $\mathcal{J}_p, (p | q_2)$ with norm $p^r = $ number of ideals
  in the first normal form $+$ number in the second normal form, and
  this is evidently given by 
{\fontsize{10pt}{12pt}\selectfont
  $$
  (p^r + p^{r - 1} + \cdots + 1) + ( p + p^2 + \cdots + p^r) = 2 (1 +
  p + \cdots + p^r) - 1 = 2.  \frac{1 - p^{r+ 1}} {1 - p} -1. 
  $$}\relax
(This result was assumed in $\S 2$).

Using the above normal form, we shall now find the structure of all
integral ambiguous ideals for the order $\mathcal{J}_p ; p | q_2$. 
\begin{theorem}\label{chap4:sec9:thm2}%theo 2
  If $\mathcal{J}_p \pi = \pi \mathcal{J}_p$ is an ambiguous
    ideal for $\mathcal{J}_p$, then either $\mathcal{J}_p \pi =
    \mathcal{J}_p \begin{pmatrix} p^r & 0 \\ 0 & p^r \end{pmatrix}$ or
    $\mathcal{J}_p \pi = \mathcal{J}_p \begin{pmatrix} 0 &  p^r
      \\ p^{r + 1} & 0 \end{pmatrix} $ according as $n (\pi)$ is an
    even or odd power of $p$. 
\end{theorem}

\begin{proof}
  Let $\pi_o = \begin{pmatrix}0 & 1 \\ p & 0 \end{pmatrix} $. Then
  $\mathcal{J}_p \pi_o = \pi_o \mathcal{J}_p$, for, if $x \in
  \mathcal{J}_p$, 
  $$
  \pi^{-1}_o x \pi_o = \begin{pmatrix}  0 & \frac{1}{p} \\1&  0
    ~ \end{pmatrix} \begin{pmatrix} x_{11} & x_{12} \\ px_{21} &
    x_{22} \end{pmatrix} \begin{pmatrix} 0 & 1 \\ p & 0 \end{pmatrix}
  = \begin{pmatrix} x_{22} & x_{21} \\ px_{12}& x_{11} \end{pmatrix}
  \in \mathcal{J}_p. 
  $$
  
  Further, since
  $$
  \begin{pmatrix} p^r & 0 \\ 0 & p^r \end{pmatrix} =
  p^r \begin{pmatrix}1 & 0 \\ 0 & 1 \end{pmatrix} \text { and
  } \begin{pmatrix} 0& p^r \\ p^{r + 1} & 0  \end{pmatrix}  =
  p^r \begin{pmatrix} 0 & 1 \\  p & 0 \end{pmatrix}, 
  $$
\end{proof}
the\pageoriginale corresponding ideals $\mathcal{J}_p \begin{pmatrix} p^r & 0 \\ 0 &
  p^r \end{pmatrix} $ and $\mathcal{J}_p \begin{pmatrix} 0 & p^r
  \\  p^{r + 1} & 0 \end{pmatrix} $ are ambiguous. 

I)~ Let $\pi$ be of the first normal form, i.e., $\pi
= \begin{pmatrix} p^a & c \\ 0 & p^b \end{pmatrix}, c $ reduced $\mod
p^b$, and $a + b = s$. Then we shall prove that $a = b$ so that $s = 2
r$ and $c = 0$. 
$$
\mathcal{J}_p \pi = \pi \mathcal{J}_p \Longrightarrow \pi^{-1} x \pi
\in \mathcal{J}_p 
$$ 
for any $x \in \mathcal{J}_p$. i.e.,
\begin{multline*}
\begin{pmatrix} p^a & -p^{-a}c^{-b} \\ 0 & p^{-b} \end{pmatrix} 
\begin{pmatrix} x_{11} & x_{12} \\ px_{21}&x_{22} \end{pmatrix}
\begin{pmatrix} p^a & c \\ 0 & p^b \end{pmatrix}\\
= \begin{pmatrix} 
  x_{11}- p^{1 - b} x_{21}c & p(x_{11} c - p^{1-b}
  c^2 x_{21} + p^b x_{12} - cx_{22}) \\ 
  p^{1 + a - b} x_{21} & p^{1- b} x_{21} c + x_{22} 
\end{pmatrix} 
\end{multline*}
is in $\mathcal{J}_p$.

Consequently, we have
\begin{enumerate}[i)]
\item $(1 + a - b) \ge 1 \Longrightarrow a \ge b$
\item $x_{11} - p^{1-b} x_{21} c= \lambda$, a p-adic integer and also
  $p^{-a} c (x_{11} - p^{1 - b} c x_{21}) + p^{-a} (p^b x_{12} -
  cx_{22}) = \lambda_1$, a p-adic integer. i.e., $c \lambda + p^b
  x_{12} - cx_{22} = p^a. \lambda_1 \Longrightarrow c (\lambda -
  x_{22}) = p^b.  \lambda'_1$ since $a \ge b (\lambda'_1$ an integer
  ). 
  
  Choosing $x_{11}, x_{21}$ and $x_{22}$ in such a manner that
  $\lambda - x_{22}$ is a p-adic unit, we conclude that $p^b | c
  \Longrightarrow c = 0$, since $c$ is reduced $\mod p^b$. 
\item Putting $c = 0, p^{b - a}x_{12}$ an integer $\Longrightarrow b \ge a$.
\end{enumerate}

From (i) and (iii) we conclude $a = b = r$.

II)~ If $\pi$ is of the second normal form, $\pi = \begin{pmatrix}0
  & p^a \\ p^{b + 1} & c \end{pmatrix}, c$ reduced $\mod p^{a + 1} a +
b + 1 = s$. 
$$ 
\mathcal{J}_p \pi = \pi \mathcal{J}_p \Longrightarrow \pi_o
\mathcal{J}_p \pi = \pi_o \pi \mathcal{J}_p. 
$$

But\pageoriginale
$$
\pi_o \mathcal{J} = \mathcal{J}_p \pi_o \text { so that }
$$
$$
\mathcal{J}_p \pi_o \pi = \pi_o \pi \mathcal{J}_p \text { and } \pi_o
\pi = \begin{pmatrix} 0 & 1 \\ p & 0 \end{pmatrix} \begin{pmatrix} 0 &
  p^a \\ p^{b + 1} & c \end{pmatrix} = \begin{pmatrix} p^{b + 1} & c
  \\ 0 & p^{a + 1} \end{pmatrix} 
$$
which is of the first normal form, since $c$ is reduced module $p^{a
  + 1}$, and $(a + 1) + (b + 1) = s + 1$. By $(i) s + 1$ is even and $
=  2r + 2$ (say) so that $s = 2r + 1$ and $a = b = r, c = 0$. In other
words, 
$$
\begin{pmatrix} 0 ~&~ p^r \\ p^{r + 1} ~&~ c \end{pmatrix}.
$$

\begin{note}
  We had assumed this lemma in \S \ref{chap3:sec7} for defining $T_p$
  when $p | q_2$. 
\end{note}

\textbf{2.} Now, we shall consider quadratic subfields $K$ of $Q$ and connect
the ideal theory of $K$ with that of $Q$. 

Let $K = k (\sqrt{d})$ ($d$ without square factor) be a quadratic
sub-field of $Q$ and $\mathscr{O}_o$ the maximal order in $K$ (the
ring of all integers). Then we know that $\mathscr{O}_o = [1,
  \omega_o]$ where 
$$ 
\omega_o = 
\begin{cases}
  \sqrt{d} &\text { if } d \nequiv 1 \pmod 4\\
  \frac{1 + \sqrt{d}}{2} &\text { if } d \equiv \pmod 4
\end{cases}
$$

If $\mathscr{O}$ be any order in $K$ (i.e., a subring of
$\mathscr{O}_o $), then $\mathscr{O}$ has a basis $[1, f \omega_\sigma], f,
$ a rational integer. This $f$, we call the \textit{conductor} of the
order $\mathscr{O}$. If $D_o$ denotes the discriminant of the order
$\mathscr{O}_o$ and $D$ that of $\mathscr{O}$, then $D = f^2 D_o$ and
since 
$$
D_o
\begin{cases}
  4d &\text{ if }  d \nequiv 1 \pmod 4\\
  d &\text { if } d \equiv 1 \pmod 4 
\end{cases}
$$
i.e.,\pageoriginale $D_o \equiv 0, 1 \pmod 4, D \equiv 0, f^2 \pmod 4
\Longrightarrow D \equiv 0, 1 \pmod 4$ (since $f^2 \equiv 0, 1 \pmod
4$). 

We now prove the following : 
\setcounter{lemma}{0}
\begin{lemma}\label{chap4:sec9:lem1}%lema 1
  Given a quadratic subfield $K \subset Q$, we can find an order
  $\mathcal{J}$ of $Q$ such that $\mathcal{J} \supset \mathscr{O}_o$. 
\end{lemma}

\begin{proof}
  Let $\mathscr{O}_o = [1, \omega_o]$. Consider an element $\Omega \in
  Q$ and $\not\in K$, so that $(1, \omega_o, \Omega, \Omega \omega_o)$
  are linearly independent over $k$. Then we prove that $\mathcal{J} =
  [1, \omega_o, m \Omega, m \Omega \omega_o]$ is an order for a
  sufficiently large integer $m$. 

  For, $m$ can be chosen so large that $tr (m \Omega), tr (m \Omega
  \omega_o)$ and $n(m \Omega)$ are integers. Now, it is enough to show
  that $m \omega_o \Omega\, m \omega_o \Omega \omega_o$ and $m \Omega
  \omega_o \Omega$ lie in $\mathcal{J}$. This can be shown as follows: 
  \begin{align*}
    \overline {m \omega_o \Omega} = \overline{\Omega} ~\overline{(m
      \omega_o)} & = (\tr (\Omega) - \Omega) (\tr (m \omega_o) - m
    \omega_o)\\ 
    & = \tr (\Omega)- \Omega \tr (m \omega_o) - \tr (m \omega_o) - \tr
    (\Omega) m \omega_o + m \Omega \omega_o 
  \end{align*} 
  is an element of $\mathcal{J}$, since each component lies in
  $\mathcal{J}$. Hence the conjugate of $\overline{m \omega_o \Omega}$
  also lies in $\mathcal{J}$. Similarly others. 
\end{proof}

\begin{note}%note 0
  The above lemma can be proved in the same way even if $k$ is
  replaced by $\bar{k}_p$ and $Q$ by $Q_p$. If $K_p = K \otimes
  \bar{k}_p$, then the max. order in $K_p = [1, \omega_o]$ (i.e., that
  $\mathscr{O}_p$-module generated by $1, \omega_o$). 
\end{note}

\begin{defi*}%defi 
  1)~ If $K \subset Q$ and $\mathscr{O} \subset \mathcal{J},
  \mathscr{O}$ is {\em optimally imbedded } in $\mathcal{J}$ if
  $\mathcal{J} \cap K = \mathscr{O}$. 
\end{defi*}

It follows immediately from the definition that $\mathscr{O}$ is
optimally imbedded in $\mathcal{J}$ if and only if $\mathscr{O}_p$ is
so in $\mathcal{J}_p$ for every $p$. 

\begin{defi*}%defi 0
  Consider\pageoriginale rational integers, $D \equiv 0, 1 \pmod 4$. We define a
  modified Legendre symbol for these as follows:  
  $$
  \bigg\{\frac{D}{p}\bigg\} = 
  \begin{cases}
    1 & \text { if } \frac{D}{p^2} \text { integral and} \equiv 0, 1 \pmod 4\\
    0 & \text { if } p|D \text{ but not the former case }\\
    \left(\frac{D}{p}\right) & \text { if } p \nmid D.
  \end{cases} 
  $$
\end{defi*}

\begin{theorem}\label{chap4:sec9:thm3}%theo 3
  Let $K$ be a quadratic subfield of $Q$ and $\mathscr{O}$, an order
  in $K$ with discriminant $D$. Then there exists an order
  $\mathcal{J}$ in $Q$ of type $(q_1, q_2)$ in which $\mathscr{O}$ is
  optimally imbedded, if and only if, 
  $$
  \prod_{p | q_1} \left( \left\{ \frac{D} {p}\right\} - 1 \right)~
  \prod_{p | q_2} \left( \left\{ \frac{D} {p}\right\} + 1 \right) \neq 0. 
  $$
\end{theorem}

We shall first prove the theorem in the local case and then extend it
to the global case, i.e., for every $p$, there exists and order
$\mathcal{J}_p$ of type $(q_1, q_2)$ containing $\mathscr{O}_p$
optimally, under the above condition and vice-versa. 

We split the proof into there parts.

$(i)~ p | q_1, ~(ii)~ p \nmid q_1 q_2, ~(iii) ~p | q_2 $.

\begin{case}[(i) ]\label{chap4:sec9:thm3:case1}
$P / q_1.$ (a) Given that $\mathscr{O}_p \subset
  \mathcal{J}_p$ optimally, to prove $ \left\{\dfrac{D} {p}\right\} \neq
  1 $.  
\end{case}

Let $\mathscr{O}_o$ be the unique maximal order in $K$, so that
$\mathscr{O} \subset \mathscr{O}_\circ$. Since $\mathscr{O}_p \subset
\mathscr{O}_{op} \subset \mathcal{J}_p, (\mathcal{J}_p$ here is the
unique maximal order in $Q_p$) and both the imbeddings being optimal,
$\mathscr{O}_p = [1, f \omega_o]_p = \mathscr{O}_{op} = [1,
  \omega_o]_p $ or $p \nmid f$. Now, $\dfrac{D}{p^2} =
\dfrac{f^2D_o}{p^2} = \dfrac{f^2 d }{p^2}$ or $\dfrac{4f^2 d}{p^2}$,
according as $d \equiv 1 \pmod 4$ or $d \nequiv 1 \pmod 4$. ($D_o$
is the discriminant of $\mathscr{O}_o$ and $K = k (\sqrt{d})), d$
being square-free, in either case, (except for the latter one, when $p
= 2 ) \dfrac{D}{p^2}$ is not an integer. In case $p = 2$, when $d
\nequiv 1 \pmod 4$, $\dfrac{D}{p^2} = f^2 d \nequiv 1, 0 \pmod 4$
for $2 \nmid f  = > f^2 \equiv 1 \pmod 4$\pageoriginale  and $f^2 d \equiv 0 \pmod
4 \Longrightarrow 4 | d$ which is impossible. 

Hence we have $\left\{\dfrac{D}{p}\right\} = 0$ or
$\left(\dfrac{D}{p}\right)$. 

Now that $\left(\dfrac{D}{p}\right) \neq 1$, in case
$\left\{\dfrac{D}{p}\right\} = \left(\dfrac{D}{p}\right)$. \footnote{For,
  $\left(\dfrac{D}{p}\right) = \left(\dfrac{D_o}{p}\right) \neq 1$, in case
  $\left\{\dfrac{D}{p}\right\} = \left(\dfrac{D}{p}\right)$.} 

b)~ Conversely, if $\left\{\dfrac{D}{p}\right\} \neq 1$, to prove that
$\mathscr{O}_p \subset \mathcal{J}_p$ optimally. Since we know that
$\mathscr{O}_p \subset \mathscr{O}_{op} \subset_{opt} \mathcal{J}_p$, it
is sufficient to prove that $\mathscr{O}_p = \mathscr{O}_{op}$ or $p
\nmid f$. 

Suppose $p | f, \dfrac{D}{p^2} = \dfrac{f^2 D_o}{p^2} \equiv 0, 1
\pmod 4$ for $D_o \equiv 0, 1 \pmod 4$. But this would mean that
$\left\{\dfrac{D}{p}\right\} = 1$, which contradicts our hypothesis. 

\begin{case}[(ii) ]\label{chap4:sec9:thm3:case2}%case ii
  $p \nmid q_1 q_2$. Given an order $\mathcal{J}_p
  \sim \begin{pmatrix}  \mathscr{O}_p& \mathscr{O}_p\\ \mathscr{O}_p &
    \mathscr{O}_p \end{pmatrix}$, to show that there always exists an
  order $\mathcal{J}'_p \sim \mathcal{J}_p$ such that $\mathscr{O}_p
  \subset \mathcal{J}'_p$ optimally. 
\end{case} 

\begin{proof}
  Let $\mathscr{O}_p = [1, \omega]_p, \omega = f \omega_o$. Without
  loss of generality we may take $\omega = \begin{pmatrix}
    0~~b\\c~~e \end{pmatrix} jb, c, e \in \bar{k}_p$. 
\end{proof}

If $(b, c, e) = 1, \mathscr{O}_p \subset \mathcal{J}_p$ optimally
vice-versa. In order to secure this, we will consider the element 
$$
\omega' = \pi^r \begin{pmatrix} 0~&~b\\c~&~e \end{pmatrix}  \pi^{-r}
= \begin{pmatrix} 0~&~p^{-rb}\\p^rc~&~e \end{pmatrix} \text{ where }
\pi = \begin{pmatrix} 1~&~0\\0~&~p \end{pmatrix} 
$$
and $r$ being chosen a positive or negative integer so as to make one
among $p^{-r} b$, $p^r c$, a unit. Thus $(p^{-r}b, p^rc, e) = 1$, implies
that $[1, \omega']_p \subset \mathcal{J}_p$ optimally or $[1,
  \omega]_p \subset \pi^{-r} \mathcal{J}_p \pi^r (\sim \mathcal{J}_p)$
optimally. 

\begin{case}[(iii) ]\label{chap4:sec9:thm3:case3}%case iii
  (a)\pageoriginale $p | q_2$. Given that $\mathcal{J}_p \sim \begin{pmatrix}
    \mathscr{O}_p~&~\mathscr{O}_p\\p\mathscr{O}_p~&~\mathscr{O}_p \end{pmatrix}$
  and $\mathscr{O}_p \subset \mathcal{J}_p$ optimally. To show that
  $\Bigg\{\dfrac{D}{p} \Bigg\} \neq - 1$. 
\end{case}

\begin{proof}
  If $\mathscr{O}_p = [1, \omega]_p$ and $\omega = \begin{pmatrix}
    0~&~b\\pc~&~e \end{pmatrix}, \omega$ satisfies the equation
  $\lambda^2 - e \lambda - pbc = 0$. Then $D = e^2 + 4 pbc$.  
\end{proof}

We have now two cases to consider 
$$
\alpha)p|D, \qquad \beta) p  \nmid D.
$$

In $\alpha), \left\{ \dfrac{D}{p} \right\}$ is either $1$ or $0$, so
that we are through. 

In $\beta) \left\{ \dfrac{D}{p} \right\} = \left( \dfrac{D}{p}
\right)$ and $D=e^2+4$ pbc 
$\equiv e^2 \pmod p$) shows that $\left(\dfrac{D}{p}\right)= 1$, i.e., $\left\{
\dfrac{D}{p} \right\}\neq -1$, in either case. 

Before going to the converse part of the above, we shall prove a lemma
which will be useful in the sequel. 

\begin{lemma}\label{chap4:sec9:lem2}%lem 2
  Let $\omega', \omega'' (\in Q_p)$ satisfy the same quadratic
  polynomial over $\bar{k}_p$. Then, if $[ 1, \omega'']_p \subset
  \mathcal{J}_p$ optimally, $[1, \omega']_p$ is optimally imbedded in
  an order isomorphic with $\mathcal{J}_p$. 
\end{lemma} 

\begin{proof}
  \begin{enumerate}[a)]
  \item By Wedderburn's Theorem, there exists an $\alpha \in Q_p$ such
    that $\alpha^{-1} \omega'' \alpha = \omega'$ so that $[ 1,
      \omega'']_p \subset \mathcal{J}_p$ optimally implies that $[ 1,
      \omega']_p \subset  \alpha^{-1} \mathcal{J}_p \alpha =
    \mathcal{J}'_p \sim \mathcal{J}_p$ optimally. 
  \item Conversely, let us suppose that $\left\{ \dfrac{D}{p}\right\} \neq
    -1$. To prove that the order $\mathscr{O}_p \subset \mathcal{J}_p
    \sim \mathcal{J}_p = \begin{pmatrix} \mathscr{O}_p ~&~ \mathscr{O}_P \\ p
      \mathscr{O}_p ~&~ \mathscr{O}_p \end{pmatrix}$ optimally.   
  \end{enumerate}
\end{proof} 
 
 Let $\mathscr{O}_p = [1, \omega]_p$ where $\omega
 = \begin{pmatrix}0~&~b\\c~&~e \end{pmatrix}, b,c,e \in
 \bar{k}_p$. Here again, we have two cases to distinguish: 
 $$
 (\alpha) p \nmid D, (\beta) p | D.
 $$
 
 In case $(\alpha), \Bigg\{\dfrac{D}{p}\Bigg\} = \left(\dfrac{D}{p}\right) = 1$
 and in $(\beta), \Bigg\{\dfrac{D}{p}\Bigg\} = 1$ or $0$. 
 
 $(\alpha) ~~D$\pageoriginale being a quadratic residue mod $p, \sqrt{D}$ is a
 $p$-adic integer. Now, $\omega$ satisfies the equation 
 $$
 \left(\omega - \frac{e - \sqrt{D}}{2}\right) \left(\omega - \frac{e +
   \sqrt{D}}{2}\right) =0, 
  $$
 since $D = d^2 + 4bc$. If $p \neq 2, \dfrac{e}{2}$ and
 $\dfrac{\sqrt{D}}{2}$ being p-adic integers $(e = tr \omega \in
 \mathscr{O}_p). \omega'  = \omega  - \dfrac{e - \sqrt{D}}{2} \in
 \mathscr{O}_p$. 
 
 In $p = 2, D$ being a quadratic residue mod $8, \dfrac{\sqrt{D}}{2}$
 is a 2-adic integer and so is $\dfrac{e}{2}$ so that $\omega'  =
 \omega  - \dfrac{e - \sqrt{D}}{2} \in \mathscr{O}_2$. 
 
 Hence the above equation can be written in the form $\omega'  =
 (\omega'  -  \sqrt{D})=0$. Now, if $\omega'' = \begin{pmatrix}0&0
   \\ pc'& \sqrt{D} \end{pmatrix}$ where $c'$ is a unit then $[1, \omega'']_p
 \subset \mathcal{J}_p$\pageoriginale optimally and further $\omega'', \omega'$
 satisfy the same equation, so that by our lemma, we are through. 
 
 $(\beta) p | D$. i) $p \nmid e$, in which case, $[1, \omega ]_p \subset
 \mathcal{J}_p$ optimally. 
 
 ii)~ $p | e$. Firstly, let us suppose $\Bigg\{\dfrac{D}{p}\Bigg\}  =
 0$. Now, since $D= e^2 + 4 bc, p | e \Longrightarrow p | 4 bc$, or $p
 | b. c$ if $p \neq 2$. Let $e =  p.  s$ and $b.c = p.n$, so that if
 $\omega' =\begin{pmatrix} 0 ~&~ b' \\ pc' ~&~ p. s \end{pmatrix} $
 where $b' c'  = n$ and one among $b', c'$ is a unit. Then $[1,
   \omega']_p \subset \mathcal{J}_p$ and $\omega', \omega$ satisfy the
 same equation. Our lemma is applicable and we are through. 
 
 In case $p = 2, p^2 | D$ so that this will be discussed in the following : 
 
 Secondly, $\Bigg\{\dfrac{D}{p}\Bigg\} = 1$ or $\dfrac{D}{p^2}$
 integral and $\equiv 0, 1 \pmod 4$. Now, $\dfrac{D}{p^2} =
 \dfrac{e^2 + 4 bc}{p^2}$ is an integer and $p | e$ imply that $p^2 |
 4bc$ or $p^2 | bc$ if $p \neq 2$. Let $bc = p^2. n$ and $e =
 p.s$. Then the element $\omega'' = \begin{pmatrix}
   0~&~b\\pc~&~p.s \end{pmatrix}$ where one among $b, c$ is a unit, is
 such that $[1, \omega'']_p \subset \mathcal{J}_p$ optimally and
 $\omega'', \omega$ satisfy the same equation. The application of our
 lemma gives the required result. Even if $p = 2$, the above argument
 can be applied, for 
 $$
 \frac{D}{4} = \frac{e^2}{4} + bc = 0, 1 \pmod 4 \Longrightarrow bc
 \equiv 0, 1 \pmod 4. 
 $$
 
 Since $p | bc$, being even, $bc \equiv 0. \pmod 4$ or $p^2 | bc$,
 which is essentially what we require in the above. 
 
 Thus the proof is complete for the local case. For going from the
 local to the global case, we distribute the primes into $2$ classes. 
 
 1) \quad p$ | q_1 q_2 f$, \qquad 2) \quad $p \nmid q_1 q_2 f$.
 
 
\medskip
\noindent
\textbf{Class (1).} Let these primes being finite in number be $p_1,
\ldots, p_m$. Then, by what has already been proved, there exists an
order $\mathcal{J}_{p_\nu}$ for each $\nu$ such that $\mathscr{O}_{p_
  \nu} \subset \mathcal{J}_{p_\nu}$ optimally. If we call
$\mathcal{J}_{p_\nu} \cap Q = \mathcal{J}_{\nu}$, then
$\mathcal{J}_{\nu p_\nu} =  \mathcal{J}_{p_\nu}$ so that we may write
$\mathscr{O}_{p_\nu} \subset \mathcal{J}_{\nu_{p_\nu}}$. 

\medskip
\noindent
\textbf{Class (2).} These primes which constitute almost all $p$, we
denote by $p_o$. By our previous lemma, there is a maximal order
$\mathcal{J}_o$ such that $\mathscr{O}_o \subset \mathcal{J}_o (
\mathscr{O}_o$ being the maximal order in $K)$. Therefore
$\mathscr{O}_{op_o} \subset \mathcal{J}_{op_o}$ and since $p_o \nmid
f, \mathscr{O}_{p_o} = \mathscr{O}_{op_o} \subset \mathcal{J}_{op_o}$
optimally. $\mathcal{J}_{op_o} = \mu^{-1}_{p_o} \mathcal{J}_{p_o}
\mu_{p_o}$ (say). 

Consider now $\mathcal{J} = Q \bigcap\limits_{p_\nu }
\mathcal{J}_{\nu_{\mathfrak{p}_o}}$. $\mathcal{J}$ is an order of $Q$ for which
$(\mathcal{J})_{p_\nu} = \mathcal{J}_{p_\nu}$ and $(\mathcal{J})_{p_o}
= \mathcal{J}_{p_o}$, i.e., $\mathcal{J}$is an order of type $(q_1, q_2)$ and
$\mathscr{O}_p \subset (\mathcal{J})_p$ optimally for every $p$,
implies that $\mathscr{O} \subset \mathcal{J}$  optimally, by a
previous lemma.  

Thus our theorem is completely established.

\begin{theorem}\label{chap4:sec9:thm4}%theorem 4
  Let\pageoriginale $ \mathcal{J}_1 $ and  $ \mathcal{J}_2 $ be $2$ orders in
    $Q$, of type  $  ( q_1, q_2), \mathscr{O}$ an order of a subfield
    $K$ of $Q$, optimally imbedded in both $\mathcal{J}_1 $ and $
    \mathcal{J}_2  $. Then there  exists an ideal $\mathscr{U} $  of
    $\mathscr{O} $ such that $ \mathcal{J}_1 \mathscr{O} = \mathscr{O}
    \mathcal{J}_2 $.  
\end{theorem}

Conversely, \textit{if} $ \mathscr{O} \subset \mathcal{J}_1 $
\textit{optimally and if } $ \mathcal{J}_1 \mathscr{U} = \mathscr{U}
\mathcal{J}_2 $, \textit{then} $ \mathscr{O} \subset \mathcal{J}_2 $
\textit{optimally}. 

\begin{proof}
  The second part is rather easy and we shall do it first. 
\end{proof}

Now, $ \mathcal{J}_1 \mathscr{U} ~= \mathcal{J}_1 ( \mathscr{U}
\mathscr{O} ) ~= ( \mathcal{J}_1 \mathscr{U} ) \mathscr{O} ~ =
\mathscr{U} \mathcal{J}_2 \mathscr{O} ~= \mathscr{U} \mathcal{J}_2  $
implies that $ \mathscr{O} \subset \mathcal{J}_2 $. If this imbedding
were not optimal, $ \mathscr{O}_p \subset \mathscr{O}'_p
\subs\limits_{\0pt} ~ ( \mathcal{J}_2 )_p $ for at least one $p$. 

$ \mathscr{U}_p $  being principal is  is $ \mathscr{O}_p \alpha_p $ (
say ) and $ ( \mathcal{J}_1 \mathscr{U}_p = (\mathcal{J}_1)_p \alpha_p
$ and similarly $ ( \mathscr{U} \mathcal{J}_2)_p  = \alpha_p  (
\mathcal{J}_2 )_p $ so that  
$$
( \mathcal{J}_1 \mathscr{U} )_p = ( \mathscr{U} \mathcal{J}_2 )_p
\Rightarrow \alpha^{-1}_p (\mathcal{J}_1 )_p \alpha_p = (
\mathcal{J}_2 )_p. 
$$

Then, $ \mathscr{O}'_p \subset ( \mathcal{J}_2 )_p $ optimally $
\Rightarrow \alpha_p \mathscr{O}'_p \alpha^{-1}_p \subset (
\mathcal{J}_1 )_p $ optimally which is a contradiction to the fact
that $ \mathscr{O} \subset \mathcal{J}_1$ optimally, since $
\mathscr{O}_p \neq \alpha_p\break \mathscr{O}'_p \alpha^{-1}_p $. 

For the first part, we observe that it is sufficient to prove it in
the local case, for, then it would imply that there exist $ \beta_p
\in \mathscr{O}_p $ such that $ ( \mathcal{J}_1 )_p \beta_p = \beta_p
( \mathcal{J}_2 )_p $ for every $p$ and $ \beta_p $ are units for
almost all $p$. Then the required ideal will be given  $ \mathcal{U}
= \bigcap \limits_{p} \mathscr{O}_p \beta_p $. 

For proving the theorem in the local case, we split the primes into
three parts. 

 i)~~  $ p \mid q_1 $,  \qquad ii)~~  $ p \nmid q_1 q_2 $, \qquad  
iii)~~   $ p \mid q_2 $.

\begin{case}[(i) ]\label{chap4:sec9:thm4:case1} % case 1
  $ p \mid q_1$.\pageoriginale This in trivial, for $ ( \mathcal{J}_1 )_p = (
  \mathcal{J}_ 2)_p =  \mathcal{J}_p $ so that  $ \beta_p = 1 $. 
\end{case}

\begin{case}[(ii) ]\label{chap4:sec9:thm4:case2}
  $ p \nmid q_1 ~ q_2 $. Without loss of generality we may assume that
  $ ( \mathcal{J}_1 )_p ~ = $ $  \begin{pmatrix} \mathscr{O}_p &
    \mathscr{O}_p \\ \mathscr{O}_p & \mathscr{O}_p \\  \end{pmatrix}
  $; now since  $ \mathcal{J}_{2p} \simeq \mathcal{J}_{1p} $,there
  exists  $ \alpha \in  Q_p $ such that $ \mathcal{J}_{2p} =
  \alpha^{-1} \mathcal{J}_{1p} \alpha $. We will reduce  $ \alpha $
  to a normal form $ \alpha' =  \begin{pmatrix} 1 & 0 \\ 0 & p^r
    \\  \end{pmatrix} $ by multiplication by units of $
  \mathcal{J}_{1p} $ on the left and right, say $ \alpha' =
  \varepsilon^{-1}_1 \alpha \varepsilon^{-1}_{2} $,  i.e., $ \alpha
  = \varepsilon_1 \alpha' \varepsilon_2 $ then  $ \mathcal{J}_{1p}
  \alpha = \mathcal{J}_{1p} \alpha' \varepsilon_2 $ and $
  \mathcal{J}_{2p} = \varepsilon^{-1}_{2} ( \alpha^{1-1}
  \mathcal{J}_{1p} \alpha' ) $  $\varepsilon_2 = \varepsilon^{-1}_2
  \mathcal{J}'_{2p} ~ \varepsilon_2 $; where $ \mathcal{J}'_{2p} =
  \alpha'^{-1} \mathcal{J}_{1p} \alpha' $.  
\end{case}

Let $ \mathscr{O}'_p = \varepsilon_2 \mathscr{O}_p \varepsilon^{-1}_2 $, then
\begin{gather*}
  \mathscr{O}_p \subs_{\0pt}  \mathcal{J}_{1p} \Rightarrow
  \mathscr{O}'_p ~ \subs_{\0pt} \varepsilon_2 \mathcal{J}_{1p}
  \varepsilon^{-1}_2 = \mathcal{J}_{1p}, \\ 
  \mathscr{O}_p \subs_{\0pt}  \mathcal{J}_{2p} \Rightarrow
  \mathscr{O}'_p ~ \subs_{\0pt} \varepsilon_2 \mathcal{J}_{2p}
  \varepsilon^{-1}_2 = \mathcal{J}'_{2p},
\end{gather*}
The conditions of the theorem being satisfied for $
\mathcal{J}_{1p}, \mathcal{J}'_{2p}, \mathscr{O}'_p $, we show that
there exists an $ \mathscr{O}_p $ -ideal $ \mathscr{U}'_p $ such that  
$$
\mathcal{J}_{1p} \mathscr{U}'_p = \mathscr{U}'_p  \mathcal{J}'_{2p},
$$
from this we would obtain on putting  $ \mathscr{U}'_p = \varepsilon_2
\mathscr{U}_p \varepsilon^{-1}_2 $ that $ \mathcal{J}_{1p}
\varepsilon_2 \mathscr{U}_p \varepsilon^{-1}_2$  = $\varepsilon_2
\mathscr{U}_p \varepsilon^{-1}_2 $. $ \varepsilon_2 \mathcal{J}_{2p}
\varepsilon_{2}^{-1} $;  i.e.,  $ \mathcal{J}_{1p} \mathscr{U}_p =
\varepsilon_2 \mathscr{U}_p \mathcal{J}_{2p} $, or  $ \mathcal{J}_{1p}
\mathscr{U}_p = \mathscr{U}_p \mathcal{J}_{2p} $ since $ \varepsilon_2
$ is a unit in $ \mathcal{J}_{1p} $. 

Therefore\pageoriginale we may assume that $ \alpha $ itself is of the form
$$
 \alpha =\begin{pmatrix} 1 & 0 \\ 0 & p^r \\ \end{pmatrix},
 \mathcal{J}_{2p} = \begin{pmatrix} \mathscr{O}_p & p^r \mathscr{O}_p
   \\ p^{-r} \mathscr{O}_p & \mathscr{O}_p \\ \end{pmatrix}.  
$$

Now, 
$$
\mathscr{O}_p = \big[ 1, \omega \big ]_p \text{ with }  \omega
= \begin{pmatrix} 0 & b \\ c & d \\ \end{pmatrix}.  
$$

Since $ \omega $ is optimally contained in $ \mathcal{J}_{1p}$ (
i.e., $\mathscr{O}_p \subset \mathcal{J}_{1p} $ optimally) we have
$(b, c, d) = 1$ and since $ \omega $ is optimally contained in $
\mathcal{J}_{2p} $, we have $ ( p^{-1}b, p^rc, d ) =1 $. It is enough
to find $ \beta \in \mathscr{O}_p $ such that  
\begin{equation}
  \mathcal{J}_{1p} \beta = \beta \mathcal{J}_{2p} \tag{1}
\end{equation}

Now (1) will be satisfied if $ \beta \alpha^{-1} = \varepsilon $
is a unit in $ \mathcal{J}_{1p} $. For, then  
$$
\mathcal{J}_{1p} \beta = \mathcal{J}_{1p} \alpha = \beta \alpha^{-1}
\mathcal{J}_{1p} \alpha = \beta \mathcal{J}_{2p}.  
$$

So let $ \beta = u + v \omega $, $ u, v \in \mathscr{U}_p $, then 
$$
\displaylines{\hfill 
  \beta = \begin{pmatrix} u & 0 \\ 0 & u \\ \end{pmatrix}
  + \begin{pmatrix} 0 & vb \\ vc & vd \\ \end{pmatrix} = \begin{pmatrix}
  u & vb \\ vc & u+vd \\ \end{pmatrix} \hfill \cr
  \text{so that }\hfill 
  \beta \alpha^{-1 } = 
  \begin{pmatrix} u & vb \\ vc & u+vd\\ \end{pmatrix} 
  \begin{pmatrix} 1 \\ o & p^{-r} \\ \end{pmatrix}= 
  \begin{pmatrix} u &  p^{-r}vb  \\ vc & p^{-r}(u+vd)\end{pmatrix}
  \quad   \hfill }
$$
Put $ v = 1 $, $ u = -d + p^r $. $u_1$, then 
$$
|\beta \alpha^{-1}| = ( -d + p^r u_1 ) u_1 - p^{-r} bc. 
$$

If
\begin{enumerate}[(i)]
\item $ ( d, p ) = 1 $, we can choose $u_1$ such that $|\beta
  \alpha^{-1}|=$ unit. 
\item $(d,p) \neq 1$, i.e., $ p \mid d $, then $ (p^{-r} bc, p ) =
  1 $, and taking\pageoriginale $u_1 = 0 $, $ \mid \beta \alpha^{-1} \mid $ is
  again a unit.  
\end{enumerate}

\begin{case}[(iii) ]\label{chap4:sec9:thm4:case3}
  $ p \mid q_2 $. We assume $ \mathcal{J}_{1p} $ $ = \begin{pmatrix}
    \mathscr{O}_p & \mathscr{O}_p \\ p\mathscr{O}_p & \mathscr{O}_p
    \\ \end{pmatrix} $, and since $ \mathcal{J}_{2p} \cong
  \mathcal{J}_{1p} $ there is an $ \alpha \in Q_p $ such that $
  \mathcal{J}_{1p} \alpha = \alpha \mathcal{J}_{2p} $. We have
  already seen that by multiplication on the left by a unit in $
  \mathcal{J}_{1p}, \alpha $ can be reduced to one of the  normal
  forms 
  $$
  \displaylines{\hfill \eta_1 \alpha = 
  \begin{pmatrix} p^a & c \\ 0 & p^b  \end{pmatrix}
  c \text{ reduced} \mod p^b,\hfill \cr 
  \text{or} \hfill \begin{pmatrix} 0 & p^a \\ p^{b+1} & 0 \end{pmatrix}  c 
  ~\text{reduced}   \mod p^{a+1}, \eta_1 -  ~\text{unit.}\hfill }
  $$
\end{case}

It is enough to consider the first normal form, for if $ \eta_1
\alpha $ is in the second normal form, by multiplication on the left
by  
\begin{align*}
  \pi & = 
  \begin{pmatrix} 
    0 &  1 \\ p & 0 \\ 
  \end{pmatrix} \text{ we obtain} \\
  \pi \eta_1 \alpha & = \begin{pmatrix} 0 &  1 \\ p & 0
    \\ \end{pmatrix} \begin{pmatrix} 0 &  p^a \\ p^{b+1} & c
    \\ \end{pmatrix} = \begin{pmatrix} p^{b+1} &  c \\ 0 & p^{a+1}
    \\ \end{pmatrix} 
\end{align*}
so that by suitable multiplication on the left and right by powers of
$ \pi $, we may assume that $\alpha = \pi^{r_1} \eta_1 \alpha
\pi^{r_2} $ has the same norm as $ \alpha $, and that it is in the
first normal form. 

We have now to consider the following cases:

\begin{enumerate}[(i)]
\item  $ b = 0 $. If $ b = 0 $, then $ c = 0 $, and 
  $$
  \pi^{r_1} \eta_1 \alpha \pi^{r_2} = \begin{pmatrix} p^a &  0 \\ 0 &
    1 \\ \end{pmatrix}.  
  $$

  By multiplication on the right by $ \pi $ and on the left by $
  \pi^{-1} $, we have  
  $$
  \begin{pmatrix} 0 &  \frac{1}{p} \\ 1 & 0
    \\ \end{pmatrix} \begin{pmatrix} p^a &  0 \\ 0 & 1
    \\ \end{pmatrix} \begin{pmatrix} 0 &  1 \\ p & 0 \\ \end{pmatrix}
  = \begin{pmatrix} 1 &  0 \\ 0 & p^a \\ \end{pmatrix} 
  $$
\item ${a = 0} $. $ \pi^{r_1} \eta_1 \alpha \pi^{r_2}
  =  \begin{pmatrix} 1 &  c \\ 0 & p^b  \\ \end{pmatrix} $,\pageoriginale and we
  have  
  $$
  \pi^{r_1} \eta_1 \alpha \pi^{r_2} 
  \begin{pmatrix} 1 &  -c \\ 0 & p^b  \end{pmatrix} = 
  \begin{pmatrix} 1 &  0 \\ 0 & p^b \end{pmatrix} 
  $$
\item $ {a > 0} $, ${ b > 0} $. We may assume that  $ (c,p) = 1 $,
  otherwise  $ \alpha |p $ instead of $ \alpha $ would serve the
  same purpose. Multiplying on the left by a unit, and on the right by
  $ \pi^{-1} $, we obtain $ \begin{pmatrix} 1 &  c' \\ 0 &
    p^{b'}  \end{pmatrix} $, and again multiplying on the right by
  $ \begin{pmatrix} 1 &  -c' \\ 0 & 1 \\ \end{pmatrix} $, we arrive at
  $ \begin{pmatrix} 1 &  0 \\ 0 & p^{b'} \\ \end{pmatrix} $. 
\end{enumerate}

Hence in all cases, $ \alpha $ is of the form $ \alpha =
\varepsilon_1 \pi^{r_1} \begin{pmatrix} 1 &  0 \\ 0 & p^r
  \\ \end{pmatrix} \pi^{r_2} \varepsilon_2 $. The transformed orders
will now be of the form 
\begin{gather*}
  \mathcal{J}'_{2p} = \varepsilon_2 \pi^{r_2}
  \mathcal{J}_{2p}. \pi^{-r_2} \varepsilon^{-1}_{2}, \varepsilon_2
  \text{ unit in } \mathcal{J}_{1p}. \\ 
  \mathscr{O}'_p = \varepsilon_2 \pi^{r_2} \mathscr{O}_p \pi^{-r_2}
  \varepsilon^{-1}_{2}, \text{ then since } \\ 
  \mathscr{O}_p \subs_{\0pt}  \mathcal{J}_{1_p},  \text{ we have }
  \mathscr{O}'_p \subs_{\0pt} \mathcal{J}'_{2p}, \text{ also } \\    
  \mathscr{O}'_p \subs_{\0pt}  \mathcal{J}_{1p} \text{ since }
  \mathcal{J}_{1p} = \varepsilon_2 \pi^{r_2} \mathcal{J}_{1_p}
  \pi^{-r_2} \varepsilon^{-1}_2  
\end{gather*}
As in case (ii)  we change notations and prove the theorem under
the assumption 
$$
\alpha = \begin{pmatrix} 1 &  0 \\ 0 & p^r \\ \end{pmatrix} ;
$$
 $ \mathscr{O}_p = \big[ 1, \omega \big ] _p $ is optimally imbedded in
$ \mathcal{J}_{1p} = \begin{pmatrix} \mathscr{O}_p &  \mathscr{O}_p
  \\ \mathscr{O}_p & \mathscr{O}_p \\ \end{pmatrix} $ and  in  $
\mathcal{J}_{2p} = \begin{pmatrix} \mathscr{O}_p &  p^r &
  \mathscr{O}_p \\ p^{1-r}  & \mathscr{O}_p & \mathscr{O}_p
  \\ \end{pmatrix} $. $ \omega $ is optimal in $ \mathcal{J}_{1p}
\Rightarrow ( b,c,d ) = 1 $, again\pageoriginale since $\omega $ is optimal in $
\mathcal{J}_{2p} \Rightarrow ( p^{-r}b, p^r.c, d ) =  1 $. As before
we take $ \beta = u + \omega, \omega = \begin{pmatrix} 0 &  b \\ pc &
  d \\ \end{pmatrix} $ and show that  $ \beta $ can be so chosen that
$\beta \alpha^{-1} $ is a unit of $ \mathcal{J}_{1p} $. Now as
before  $\mid \beta \alpha^{-1} | = (-d + p^r u_1)u_1- p^{1-r} b. c $;
where  $ u = -d + p^r u_1 $. 
\begin{enumerate}[(i)]
\item If $ ( p,d ) = 1 $, then $u_1 $ can be chosen such that $ \mid
  \beta \alpha^{-1} \mid = $ unit. 
\item If $ ( p, d ) = p,i.e.,p \mid d $; we consider $ \beta ( \pi
  \alpha )^{-1} $, if we put  $ u_1 = 0 $, $ (-d + \omega) ( \pi
  \alpha )^{-1}  = \begin{pmatrix} -d&  b \\ pc & 0
    \\ \end{pmatrix}  \begin{pmatrix} 1 &  0 \\ 0 & \dfrac{1}{p}
    \\ \end{pmatrix}  \begin{pmatrix} 0 &  \dfrac{1}{p} \\ 1 & 0
    \\ \end{pmatrix} $ =  $ \begin{pmatrix} p^{-r} b &  \dfrac{-d}{p}
    \\ 0 & c \\ \end{pmatrix} $ or $ | ( -d + \omega ) ( \pi \alpha
  )^{-r}|=p^{-r} b.c $.  
\end{enumerate}  
  
  Now since $ ( p^{-r} b, p ) = 1 $, and $ (c,p) = 1 $, we have $ (
  p^{-r} bc, p ) = 1 $,  i.e., $ p^{-r} bc $ is a $ p- $adic unit, so
  that $ \beta = \epsilon \pi \alpha,  \epsilon$, unit in $
  \mathcal{J}_{1p} $. In this case, again we have $ \mathcal{J}_{1p}
  \beta = \mathcal{J}_{1p} \pi \alpha  = \pi \mathcal{J}_[1p] \alpha  =
  \pi \alpha \mathcal{J}_{2p} = \beta \mathcal{J}_{2p}$. In order to
  complete the proof of the theorem, we have only to show that  $
  \beta $ is  a unit of $ \mathscr{O}_p $ for almost all $p$ and it is
  enough to verify this in the case  $ p \nmid q_1 q_2 $. 
  
This is secured by taking among the  primes $ p \nmid q_1 q_2 $, only
those for which both $b$ and $c$ are $p$-adic units. Then  
$$
\mid \beta \mid = | \begin{pmatrix} u &  b \\ c & u+d \\ \end{pmatrix}
| = p^r.u_1 ( p^r. u_1 -d ) -bc 
$$
is always a $p-$ adic unit. Or, $ \beta $ is a unit of $ \mathscr{O}_p
$. 

Thus in $ \mathcal{J}_{1p} \beta_p = \beta_p \mathcal{J}_{2p},
\beta_p $ is a unit for almost all $p$  so that the global ideal $
\mathscr{O} = \bigcap \limits_{p} \mathscr{O}_p \beta_p $ serves our
purpose. 

\begin{remark*}
  The\pageoriginale theorem proved above is a very important one for our further
  applications and it has been proved in various connections and in
  various forms by several mathematicians dating from Legendre, Gauss,
  Minkowski and Emmy  Noether to Hasse, Chevalley and
  Siegel. Chevalley and Hasse proved an analogous result in the
  theory of algebras while Siegel required a similar form for the
  theory of quadratic forms. 
\end{remark*}

\textbf{3.} Let $ \mathscr{O} $ be an order of a quadratic subfield $K$  of
  $Q$ optimally imbedded in an order $ \mathcal{J} $  of $Q$  of type
  $ ( q_1,q_2 ) $. Let $D$ denote  the discriminant of $ \mathscr{O}
  $. We have then the following  


\begin{theorem}\label{chap4:sec9:thm5}%theorem 5
  An ambiguous prime ideal $ \nu $ with norm $p$ is generated by an $
  \mathscr{O} $ -ideal $ \mathscr{U} $ if and only if $ \left\{
  \dfrac{D}{p} \right\} = 0 $. 
\end{theorem} 

\begin{proof}
  Since all global ideals are defined as intersections of local ones,
  it is sufficient to prove the theorem for the $p$-adic case. 
\end{proof}

For the proof in the local case, we have three possibilities :

(i)~~ $p \nmid q_1 q_2$,  \qquad (ii)~~ $p|q_1$, \qquad (iii)~~ $p | q_2$. 

\setcounter{case}{0}
\begin{case}[(i) ]% case 1
  $p+q_1 q_2$. In this case, we shall prove that there do not
  exist any proper $ ( i.e., \vartheta_p \neq \mathcal{J}_p. p^r ) $
  ambiguous ideals $ \vartheta_p $ at all so that the above problem
  does not arise.  
\end{case}

$ \mathcal{J}_{p} \cong \begin{pmatrix} \mathscr{O}_p &
  \mathscr{O}_p \\ \mathscr{O}_p & \mathscr{O}_p \\ \end{pmatrix} $;
if $ \vartheta_p = \mathcal{J}_p \pi $ is ambiguous, and if we assume
$ \pi $  to be without loss of generality, of the  form, $ \pi
=  \begin{pmatrix} p^a & c \\ 0 & p^b \\ \end{pmatrix} a + b = s; c $
reduced  $ mod p^p $, then $ \mathcal{J}_p \pi = \pi \mathcal{J}_p $
implies\pageoriginale that  

\begin{align*}
  &\qquad\begin{pmatrix} p^{-a} &  -cp^{-(a+b)} \\ 0 & p^{-b}
  \\ \end{pmatrix} \begin{pmatrix} x_1 &  x_2 \\ x_3 & x_4
    \\ \end{pmatrix} \begin{pmatrix} p^a &  c \\ 0 & p^b
    \\ \end{pmatrix} \\ 
  &= \begin{pmatrix} p^{-a} x_1 -cx_3 p^{-(a+b)} &  p^{-a}x_2-cx_4
    p^{-(a+b)} \\ p^{-b}x_3 & p^{-b}x_4
    \\ \end{pmatrix} \begin{pmatrix} p^a &  c \\ 0 & p^b
    \\ \end{pmatrix} \\ 
  &= \begin{pmatrix} x_1-cx_3 p^{-b} & cp^{-a} x_1 -c^2x_3 p^{-(a+b)}
    + p^{b-a}x_2 -cx_4 p^{-a} \\ p^{a-b}x_3 & cp^{-b} x_3  +x_4
    \\ \end{pmatrix} 
\end{align*}
must be an integral matrix for all $ x_1, x_2, x_3,x_4 \in
\mathscr{O}_p $. Choosing $x_3 $ to be a unit, $ a \ge b $ and  $ x_1
$ to be divisible by $ p, p^b | c $, which implies  $ c = 0 $ since
$c$ is reduced modulo $ p^b $. Hence, if we further choose $ x_2 $ to
be a unit, $b\geq a $ so that $ a = b $ and $ c = 0, i.e. $, $ \pi
= \begin{pmatrix} p^r & 0 \\ 0 & p^r \\ \end{pmatrix} $ where $ a = b
= r $ and $ 2r = s $. Hence  $ \vartheta_p = \mathcal{J}_p.  p^r $. 

\begin{case}[(ii) ]
  $ p | q_1 $. If $ \mathscr{O}_\circ $ is the maximal order of $K$,
  $$
  \mathscr{O}_p \subset \mathscr{O}_{op} \subset \mathcal{J}_p
  \Rightarrow \mathscr{O}_p = \mathscr{O}_{op}  ~\text{ and }~ \left\{
  \frac{D}{p} \right\} = \left( \frac{D}{p} \right) \neq 1,  
  $$
  by theorem \ref{chap4:sec9:thm1}. Therefore if $ \left( \dfrac{D}{p} \right) = \dfrac{D_o}{p} = 1
  $, where $D_o $ is the discriminant of  $ \mathscr{O}_0 $ so that
  $K_p$  is unramified over $ \bar{k}_p $ which implies that there
  cannot exist any  $ \mathscr{O}_jp $-ideal generating  $
  \vartheta_p $. On the other hand, if $ \left( \dfrac{D_0}{p}\right)
  = 0 $, then 
  $K_p $ is ramified over $ \bar{k}_p $, there exists an  $
  \mathscr{O}_p $ -ideal $ \mathscr{U}_p $ such that $ p \mathscr{O}_p
  = \mathcal{U}^2_p $  and $ N ( \mathscr{U}_p ) = p $ and by the
  uniqueness of such an ideal $ \vartheta_p = \mathcal{J}_p $. $
  \mathscr{U}_p $. 
\end{case}

\begin{case}[(iii) ]
  $ p | q_2 $. Again, by theorem \ref{chap4:sec9:thm1}, $\bigg\{ \dfrac{D}{R}\bigg \}  \neq -1$. 
\end{case}

\begin{enumerate}[a)]
\item $ \dfrac{D}{p^2} $ integral and  $ \equiv 0, 1  \pmod {4}$. $
  \mathscr{O}_p = \big [ 1, p^{r_\omega} \big ] \subs\limits_{\0pt}
  \mathcal{J}_p $ where\pageoriginale $ \mathscr{O}_{op} = \big[ 1, \omega \big ] $
  ( say ). Let $ \vartheta_p = \mathcal{J}_p \delta $, $ \delta \in
  \mathscr{O}_{p} $, then $ n ( \delta ) = p $ and  $ \delta = a +
  bp^{r_\omega} $; $ a, b \in \mathscr{O}_{p} $. Hence $ n (\delta) $
  is either a unit or $ \equiv 0 \pmod {p^2}$ in either case we
  obtain a contradiction. 
\item $ \left\{ \dfrac{D}{p} \right\} = \left(\dfrac{D}{p}\right) = 1 $. 

  Let $ \vartheta_p = \mathcal{J}_p \alpha_p, \alpha_p \in \mathscr{O}_p $; we,
  may assume without loss of generality $ \mathscr{O}_p = \left[
    1, \begin{pmatrix} 0 & b \\ pc & d \\ \end{pmatrix} \right]
  $. Since there is only one ambiguous prime ideal of norm $p$, namely
  $\mathcal{J}_p \pi,  \mathcal{J}_p \alpha_p = \mathcal{J}_p \pi $
  or $ \alpha_p \pi^{-1} $ is a unit of $ \mathcal{J}_p $.  
  
  Now,  $ D = d^2  -4 pbc $ and $ p \nmid D \Rightarrow p \nmid d $ so
  that if $ \alpha_p \in \mathscr{O}_p $, $ i.e., \alpha_p = u + v
  \omega $ where $ \omega = \begin{pmatrix} 0 & b \\ pc & d
    \\ \end{pmatrix} $ then $ \alpha_p \pi^{-1} = \begin{pmatrix} bv
    & u/p \\ u+vd & cv \\ \end{pmatrix} $ is a unit $ \Rightarrow  p |
  u $ and $ p | u + vd, i. e., p| vd $ or $ p | v $. In other words, $
  | \alpha_p \pi^{-1} | \equiv 0 ( mod p ) $ so that $ \alpha_p
  \pi^{-1} $ cannot be a unit.  
\item $ \left\{\dfrac{D}{p}\right\} = 0 $. $ \left( \dfrac{D}{p^2}
  \not \equiv 0, 1 \pmod 4 \right)$.
\end{enumerate}

$ D = d^2 - 4 pbc \equiv 0 \pmod p  \Rightarrow d \equiv 0 \pmod p 
$. Further $ p \nmid b c $ for otherwise, $ \dfrac{D}{p^2} \equiv
\dfrac{d^2}{p^2} \pmod 4 $ and  $ \equiv 0, 1 \pmod 4$ which is a
contradiction to the hypothesis. 

Consider $ \omega \pi^{-1} =  \begin{pmatrix} 0 & b \\ pc & d
  \\ \end{pmatrix} \begin{pmatrix} 0 & 1/p \\ 1 & 0 \\ \end{pmatrix} $
$ = \begin{pmatrix} b & o \\ d & c \\ \end{pmatrix} $ is a unit of $
\mathcal{J} _p $ by virtue of $ p | d, $ $ p \nmid bc $ so that $
\mathcal{J}_p \omega =\mathcal{J}_p \pi$. In other words, the unique
ambiguous prime ideal 
of norm $p$ is  $ \vartheta_p = \mathcal{J}_p \omega $. 

\begin{note}
  Let $ \vartheta $ be an ambiguous ideal with norm $n$ and let  $ n =
  p_1^{\varrho_1} \cdots p_k ^{\varrho_k} $. Then we know that $
  \vartheta  = \bigcap \limits_{p} \varepsilon_p $; $
  \vartheta_p = \mathcal{J}_p $ for almost all $p$ and for the rest  $
  n ( \vartheta_{pi} ) = p_i^{\mathscr{U}} $. 
\end{note}

 $ \vartheta_p$\pageoriginale being ambiguous for all primes $ p \nmid
q_1 q_2 $, 
$ \vartheta_p $ is a rational ideal;  i.e., $\vartheta_p =
\mathcal{J}_p \lambda_p $; $ \lambda_p $, a $p$-adic number and since
$ \vartheta_p = \mathcal{J}_p $ for almost all $ p, \lambda_p $ is a
p-adic unit for all but a finite number of  $ p \nmid q_1 q_2 $. Among
the primes $p_i$ that occur in the factorization of $ n$, let $ p_1
\cdots p_l $ (suitably rearranged) by those which divide  $ q_1 q_2 $
and for these, $ \vartheta_{pi} = \mathcal{J}_{pi} \pi_i^{\varrho_i}$ $ (
i = 1 $ to $l$)  where $ n ( \pi_i ) = p_i $ so that we may write
symbolically $ \vartheta = \mathcal{J}. r. \delta_1^{\varrho_1} \cdots
\delta_l^{\varrho_l} $ where $ \delta_i = \mathcal{J}_{p_i} \pi_i (p_i/
q_1 q_2 n) $ and  $r$  is a rational number. Since $
\mathcal{J}_{p_{i}} \pi^2_i = \mathcal{J}_{p_{i}}. p_i $, the indices
$ \varrho_1 \ldots \varrho_l$ can be reduced $\mod 2$;  i.e., $
\varrho_i = 0, 1 $. 
 
 From the above factorization we immediately deduce that the total
 number of proper ambiguous ideals i.e., upto multiplication by a
 rational number ) is $ 2^x,  x $ being the number of primes $ p |
 q_1 q_2 $. 
 
 From theorem \ref{chap4:sec9:thm3}, deduce that $ 2^{x'} $ is the number of ambiguous
 ideals generated by means of  $ \mathscr{O} -$ ideals where $ x' $ is
 the number of primes  $p$ for which $ \left\{ \dfrac{D}{p} \right\}
 = 0 $  among those which divide $ q_1 q_2 $. Therefore the order of
 the quotient group is  $ 2^{x-x'} = \prod \limits_{p \mid q_1}  \left( 1-
 \left\{ \dfrac{D}{p} \right\} \right)  \prod \limits_{p \mid q_2}  \left( 1
 + \left\{ \dfrac{D}{p} \right\} \right) $.  
 
\textbf{4.}~  We shall now prove the last theorem of this section, which sums
  up previous ones and which will be applied later for computing the
  traces of correspondences.  

If $ \mathscr{O} $ is an order of a subfield  $ K \subset Q $ such
that if $ \mathscr{O}  \subs\limits_{\0pt} \mathcal{J} $ and $
\varepsilon $, a unit of $ \mathcal{J} $, then $ \varepsilon^{-1}
\mathscr{O}  \varepsilon  \subs\limits_{\0pt} \mathcal{J} $. Thus,
with each order  $ \mathscr{O}  \subs\limits_{\0pt} \mathcal{J} $, a
whole class of orders  $ \big \{ \varepsilon^{-1} \mathscr{O}
\varepsilon \big \} $ is optimally contained  in  $ \mathcal{J} $. Of
course,  $\varepsilon^{-1} \mathscr{O} \varepsilon \subset
\varepsilon^{-1} ~ K \varepsilon \sim K $. We shall restrict our
attention to only proper classes of orders $ \big \{ \varepsilon^{-1}
\mathscr{O} \varepsilon ; n ( \varepsilon ) = + 1 \big \} $. 

\begin{theorem}\label{chap4:sec9:thm6}%\theorem 6
  The\pageoriginale number of proper classes of orders $ \big \{ \varepsilon^{-1}
  \mathscr{O} \varepsilon \big \} $ optimally imbedded in an order $
  \mathcal{J} $ of type  $ ( q_1,  q_2 ) $ (where class number of
  ideals is  $1 ) $  and  isomorphic to a given order $
  \mathscr{O}_\circ \subset K \subset Q $ $ ( Q $ indefinite  and $K$,
  imaginary )  is equal to the  following product   
  $$
  \prod_{p \mid q_1}  \left( 1 - \left\{ \frac{D}{p} \right\} \right) ~ \prod_{p
    \mid q_2}  \left( 1 + \left\{ \frac{D}{p} \right\} \right) h (D),  
  $$
  $D$  being the discriminant of $ \mathscr{O}_\circ  $ and $ h (D)
  $, the class number of  $ \mathscr{O}_\circ$-ideals. 
\end{theorem}

\begin{proof}
  $ \mathscr{O}_\circ $ is the given fixed order, optimally imbedded
  in $ \mathcal{J} $ and  $ \mathscr{O} $, any other order, isomorphic
  to   $ \mathscr{O}_\circ $, and optimally imbedded in $ \mathcal{J}
  $. Since the class number of $ \mathcal{J} $-ideals is $1$, there
  exists an $ \alpha \in Q $ such that  $ \mathscr{O} = \alpha
  \mathscr{O}_\circ \alpha^{-1} $. 
\end{proof}

Now, defining $ \mathcal{J}' = \alpha^{-1} \mathcal{J} \alpha $, we find that 
$$
\mathscr{O} \subs_{\0pt} \mathcal{J} \Rightarrow \alpha^{-1}
\mathscr{O}\alpha  \subs_{\0pt} \mathcal{J}'  \text{ or }
\mathscr{O}_\circ \subs_{\0pt} \mathcal{J}'.  
 $$
$ \mathscr{O}_\circ $ being contained optimally in both  $ \mathcal{J}
$ and $ \mathcal{J}' $ of type $ ( q_1, q_2 ) $, by our previous
theorem, there exists an $ \mathscr{O}_\circ $ -ideal $ \mathscr{U} $
such that $ \mathcal{J} \mathscr{U} = \mathscr{U} \mathcal{J}' $. In
other  words, $ \mathcal{J} \mathscr{U} \alpha^{-1} \alpha =
\mathscr{U} \alpha^{-1} \mathcal{J} \alpha $ or $ \mathcal{J}
\mathscr{U} \alpha^{-1} = \vartheta $ is an ambiguous  $ \mathcal{J}-
$ideal. Without loss of generality, we may assume that $ \vartheta $
contains in its  decomposition, no ideal generated by an $
\mathscr{O}_\circ $  -ideal, for if $ \vartheta = b \vartheta' $ $ b $
is generated by an $ \mathscr{O}_\circ $ -ideal, we may combine  $b$
with $ \mathscr{U} $ with  have $ \vartheta' $ instead of $ \vartheta
$.  
 
We now make correspond to the pair $ ( \mathscr{O}, \mathscr{O}_\circ
) $, the pair of ideals  $ ( \vartheta, \mathscr{U} ) $. $ \alpha $
is not uniquely determined by the condition $ \alpha
\mathscr{O}_\circ \alpha^{-1} = \mathscr{O} $. In fact 
 
\begin{enumerate}[i)]
\item $ \alpha $ can be replaced by $ \alpha \mu $, $ \mu \in K $,
  in which case  $ \mathscr{U} \rightarrow \mathscr{U} \mu  $ and  $
  \vartheta \rightarrow \vartheta $. 
\item $ \alpha$\pageoriginale can even be replaced by $ \alpha \mu \omega $
  where  $ \omega \in Q $ and $ \infty^{-1} K \omega = K^\sigma $, $
  \sigma $ being the only automorphism of  $ K / k $ different from
  the identity, so that $ \omega^{-1}  \mathscr{O}_\circ \omega =
  \mathscr{O}_\circ $ and  $ \mathscr{O} \rightarrow \mathscr{O} $.  
\end{enumerate} 
 
 But here for  $ \mathscr{U} $, we cannot take the ideal $
 \mathscr{U}' \omega^{-1} \mu^{-1} $ ( where $ \mathcal{J}
 \mathscr{U}' = \mathscr{U}' \mathcal{J} '' $;  $ \mathcal{J}''  = (
 \alpha \mu \omega ) ^{-1} \mathcal{J} ( \alpha \mu \omega )) $
 since it is no longer an $ \mathscr{O}_\circ $ -ideal. 
 
 $\mathscr{U} \sim \mathscr{U}/ \Rightarrow $ there exists $ \mu'\in
 K $ such that $ \mathscr{U} \mu' = \mathscr{U} $, in which case  
 
 $ \vartheta' = \mathcal{J} \mathscr{U}' $. $ \omega^{-1} \mu^{-1}
 \alpha^{-1} = \mathcal{J} \mathscr{U} \mu'$. $ \omega^{-1} \mu^{-1}
 \alpha^{-1} = \vartheta = \mathcal{J} \mathscr{U} \alpha^{-1} $ if
 and only if $ \eta = \mu' \omega^{-1} \mu^{-1} $ is a unit of $
 \alpha^{-1} \mathcal{J} \alpha = \mathcal{J}' $.  
 
 Consequently, if we now make correspond  to every pair of classes of
 orders $(( \mathscr{O} ), ( \mathscr{O}_\circ )) ~ ( \mathscr{O} $
 optimally imbedded in $ \mathcal{J} $ and isomorphic to  $
 \mathscr{O}_\circ ) $ the pair $ ( \vartheta, ( \mathscr{U} )) ~( (
 \mathscr{U} ) $  denoting the class of $  \mathscr{O}_\circ $ -ideals
 equivalent to $ \mathscr{U} $, and $ \vartheta $ is  an integral
 ambiguous  $ \mathcal{J} $ -ideal not divisible by an $ \mathcal{J}
 $-ideal generated by an $ \mathscr{O}_\circ $ -ideal ), then to a
 pair $ (( \mathscr{O}), ( \mathscr{O}_\circ )) $ there correspond
 exactly one or two pairs $ ( \vartheta, ( \mathscr{U} )) $  according
 as there does or does not exist a unit of the type $ \mu' \omega^{-1}
 \mu^{-1} $ in $ \mathcal{J}' $.  
 
 We shall now consider the converse map. Let $ ( \vartheta, (
 \mathscr{U} )) $ be a pair, $ \vartheta $ ambiguous and $ (
 \mathscr{U} )$, an $ \mathscr{O}_\circ $ -ideal class such that there
 exists an $ \alpha \in Q $ such that $ \mathcal{J} \mathscr{U} =
 \vartheta \alpha $. We may suppose that $ \vartheta $ does not
 contain any $ \mathscr{O}_\circ $ - ideal. Then, we associate to the
 pair $ ( \vartheta, ( \mathscr{U} )) $ the pair  $ ( \mathscr{O},
 \mathscr{O}_\circ ) $ where $ \mathscr{O} = \alpha \mathscr{O}_\circ
 \alpha^{-1} $. Now, $ \vartheta \alpha = \mathcal{J} \mathscr{U} $,
 holds even if we replace $ \alpha $ by $ \varepsilon \alpha $, $
 \varepsilon $ being a unit of $ \mathcal{J} $. But then $
 \mathscr{O} \rightarrow \varepsilon \alpha \mathscr{O}_\circ
 \alpha^{-1} \varepsilon^{-1} = \varepsilon \mathscr{O}
 \varepsilon^{-1} $ so\pageoriginale that $ \varepsilon \mathscr{O} \varepsilon^{-1}
 $ would properly be  equivalent with $ \mathscr{O} $ only if there
 exists a proper unit $ \varepsilon_+ $ such that $ \varepsilon
 \mathscr{O} \varepsilon^{-1} =  \varepsilon_+ \mathscr{O}
 \varepsilon_+^{-1} $. Therefore, the mapping $ (
 \vartheta,\mathscr{U} ) ) \rightarrow  (( \mathscr{U}, (
 \mathscr{O}_\circ ))  (( \mathscr{O} ) $ being the proper class of  $
 \mathscr{O} ) $ is in general two - valued or single-valued if and
 only if there does not exist or does exist a unit $ \varepsilon \in
 \mathcal{J} $ of norm $-1$ with the property that  $ \varepsilon
 \mathscr{O} \varepsilon^{-1} = \varepsilon_+ \mathscr{O}
 \varepsilon^{-1}_+ $  for a unit  $ \varepsilon_+ $ of norm $1$. 
 
 We now observe the following :

 The existence of $ \eta = \mu' \omega^{-1} \mu^{-1} $, a unit in $
 \mathcal{J}' \Longleftrightarrow $ the  existence of  $ \varepsilon
 $, a unit of $ \mathcal{J} $ with $ n (\varepsilon) = -1 $ such that
 $ \varepsilon \mathscr{O} \varepsilon^{-1} = \varepsilon_+
 \mathscr{O} \varepsilon_+^{-1} $ for some $ \varepsilon_+ $  a unit
 of $ \mathcal{J} $ with norm $1$. 
 
Now, $ n ( \eta ) = n (\mu ')  (\mu') (n ( \omega))^{-1} n (
  \mu^{-1} ) = n(\mu'')$. $(n ( \omega ))^{-1} $ if $ \mu'' =
  \mu'$. $ \mu^{-1} \in  Q $. Therefore $ n ( \eta ) $ and $ n (
  \omega) $ are of the same sign. But $ n ( \omega) $ is $ < 0 $ for
  otherwise, $ K = k (\delta) $ with $ \delta^2 < 0 $ implies that  $
  n (\delta) > 0 $ and if $ n ( \omega) > 0 $, then $ Q = k \big [ 1,
    \delta,  \omega, \delta \omega \big ] $ would be definite,
  contradictory to our hypothesis. Therefore $ n ( \eta )= -1  $. Let
  $\varepsilon$ be any unit of $ \mathcal{J} $ of  norm $-1$. Then, if
  $ \varepsilon_+ = \varepsilon \alpha \eta \alpha^{-1} $, we have  
  \begin{align*}
    \varepsilon_+ \mathscr{O} \varepsilon_+^{-1} &=  \varepsilon
    \alpha \eta \alpha^{-1} \mathscr{O} \alpha \eta^{-1}
    \alpha^{-1} \varepsilon^{-1} = \varepsilon \alpha \eta
    \mathscr{O}_\circ \eta^{-1} \alpha^{-1} \varepsilon^{-1} \\ 
    &= \varepsilon \mathscr{O} \varepsilon^{-1} ( \text { since }  ~
    \eta\mathscr{O}_\circ \eta^{-1} = \mathscr{O}_\circ ).
  \end{align*} 
 
 Conversely, if for an $ \varepsilon $ of norm $ -1$, $ \varepsilon
 \mathscr{O} \varepsilon^{-1} = \varepsilon_+ \mathscr{O}
 \varepsilon_+^{-1} $ then take for $ \eta = \alpha^{-1} \varepsilon
 \varepsilon_+ \alpha $. 
 
 We may therefore conclude the following:
 
 The direct mapping $ (( \mathscr{O}), ( \mathscr{O}_\circ ))
 \rightarrow ( \vartheta, (\mathscr{U} )) $ is single valued if and
 only if the inverse mapping is single-valued. The sets  being finite,
 combining this fact with both the mappings by an enumerative
 argument,\pageoriginale we obtain a $1-1$  correspondence in the above. The classes
 of orders $ ( \mathscr{O} ) $ which are optimally imbedded in $
 \mathcal{J} $ and isomorphic with  $ ( \mathscr{O}_\circ ) $ is equal
 to the number of pairs $ (( \mathscr{O}), ( \mathscr{O}_\circ )) $
 which in  turn is thus equal to the number of pairs, $ ( \vartheta, (
 \mathscr{U} )) $. But by the  deduction from
 Theorem \ref{chap4:sec9:thm5}, the number 
 of ambiguous ideals not containing $ \mathscr{O}_\circ $ - ideals is
 given by  
 $$
 \prod_{p \mid q_1}  \left( 1 - \left\{ \frac{D}{p} \right\} \right) ~ \prod_{p
   \mid q_2}  \left( 1 + \left\{ \frac{D}{p} \right\} \right) 
 $$ 
 and the number of ideal classes $ \left\{ \mathscr{U} \right\} $ is $ h
 (D) $, so that the required number is given by  
  $$
 \prod_{p \mid q_1}  \left( 1 - \left\{ \frac{D}{p} \right\} \right) ~ \prod_{p
   \mid q_2}  \left( 1 + \left \{ \frac{D}{p} \right \}\right) h (D). 
 $$
 
\begin{note}
  If $Q$ is definite, the above arguments have to be slightly modified
  and the number of units of $ \mathcal{J} $ being finite, we can show
  by a slightly different argument, that the number of classes is given
  by 
$$
\dfrac{\text{number of units in}
      \mathcal{J}}{2}   \prod 
  \limits_{p \mid q_1}  \left( 1 - \left\{ \dfrac{D}{p} \right \} \right) ~ \prod
  \limits_{p \mid q_2}  \left( 1 -+ \left \{ \dfrac{D}{p}
  \right\}\right) h (D) 
$$ 
(if again, the class number of $ \mathcal{J} $ -ideals is 1). 
\end{note}

\section[Applications, Especially to the...]{Applications, Especially to the Calculation of the Number of
   Fixed Points of a Correspondence \texorpdfstring{$T_n $}{Tn}}\label{chap4:sec10}%SCE 10 

\markright{10. Applications, Especially to the Calculation of the\ldots}

\noindent \textbf{5.}~
 We shall first take up the number of elliptic vertices of the
  fundamental domain of the proper unit group of an order $
  \mathcal{J}$ of type $(q_1, q_2)$ of an indefinite quaternion
  algebra $Q$  over the rational number field $k$. We will take up the
  calculation of parabolic cusps, later, since this does not require
  any of the theorems we have proved so far. 


\textbf{1.}~
If\pageoriginale $ \tau $ is an elliptic vertex and $ \varepsilon ( \tau ) =
  \tau $, then $ \varepsilon $ is a transformation of finite order, $
  \varepsilon^n = 1 $ (say). But $ \varepsilon \in Q $ satisfies  $
  \varepsilon^2 - s $. $ \varepsilon  + 1 = 0 $. Now, if $D$ is the
  discriminant of this equation, then $ \varepsilon $ lies in the
  field  $ K = k ( \sqrt{D})$  and  $ D < 0 $ implies that  $ n = 3,
  4$ or $6$. When $ n = 4, D = -4 $  and when $n =3$ or 6, $D=-3$.  If $
  \mathscr{O} = \big [ 1, \varepsilon \big ] $, then  $ \mathscr{O} $
  is a maximal order optimally contained in $\mathcal{J}$ and $D(
  \mathscr{O}) = D$. 

As we have already seen in  \S \ref{chap2:sec5}, to an elliptic vertex
$ \tau $ of 
$ \delta $ (actually $\Gamma. \tau ) $,  there corresponds a class $
\eta^{-1} \varepsilon \eta\, ( n ( \eta ) = 1, \eta \in \Gamma ) $ and
to each such class, there corresponds the proper class of orders $
\eta^{-1} \mathscr{O} \eta $. It is easily seen that this
correspondence is one-one so that the number of elliptic vertices is
equal to the number of isomorphic classes of orders $ ( \mathscr{O} )
$, optimally contained in $ \mathcal{J} $. By theorem \ref{chap4:sec9:thm4}, this number
is given by  
$$
 \prod_{p \mid q_1}  \left( 1 - \left\{ \frac{D}{p} \right\} \right) ~ \prod_{p
   \mid q_2}  \left( 1 + \left\{ \frac{D}{p} \right\} \right) h (D), D = D (
 \mathscr{O} ).  
$$

\begin{enumerate}[i)]
\item If  $ n = 4 $, $ D = -4 $ and this number is 
  $$
  \prod_{p \mid q_1} \left( 1 - \left( \frac{-4}{p}\right)\right)
  \prod_{p \mid q_2} \left(1 + \left( \frac{-4}{p} \right)\right).  1  
  $$
  since the modified Legendre symbol is the ordinary Legendre symbol
  and $ h (D) = 1 $. 
\item If  $ n = 3 $ or $6$, $D = -3 $ and this number is then 
  $$
  \prod_{p \mid q_1} \left( 1 - \left( \frac{-3}{p}\right)\right)
  \prod_{p \mid q_2} \left(1 + \left( \frac{-3}{p} \right)\right).  1 
  $$
  for the same reason as before.
\end{enumerate}

\section*{Representation of a Natural Number as a Sum of Three Squares}

\begin{enumerate}
\renewcommand{\labelenumi}{\bf\theenumi.}
\setcounter{enumi}{5}
\item Let $d$ be a negative rational integer and $ K= k ( \sqrt{d} ) $
  $ ( -d $ squarefree ). Let $ Q = k ( 1, i, j, k ) $ be the
  Hamiltonian quaternion algebra over the\pageoriginale rational number field
  $k$. Then $Q$ is definite and $K \subset Q$. 
\end{enumerate}

Now, the order $\mathcal{J} = (i, j, k, \dfrac{1+i+j+k}{2})$ is
maximal in $Q$ so that $q_2 = 1$. The only characteristic prime of $Q$
is $2$ and the class number of $\mathcal{J}$ is $1$. The units in
$\mathcal{J}$ are $24$ in number and are given by  
$$
\pm 1, \pm i, \pm j, \pm k, \frac{\pm 1 \pm i \pm j \pm k}{2}. 
$$

For the maximal order $\mathscr{O}_\circ = [ 1, \omega] \subset K, \omega =
\sqrt{d}$ if $d \not \equiv 1 \pmod 4$ and $ \omega =
\dfrac{1+\sqrt{d}}{2}$ if $d \equiv 1 \pmod 4$. Taking the basis
representation of $\omega$, we have 
\begin{enumerate}[i)]
\item $\omega = X_1 i + X_2 j + X_3 k$~,  if $d \not \equiv 1 \pmod 4$, or
\item $\omega = \dfrac{1+X_1 i + X_2 j + X_3 k}{2}$, if $d \equiv 1 \pmod 4$.
\end{enumerate}

But, in either case,
$$
X^2_1 +X^2_2 +X^2_3 = -d.
$$

It is easily seen that every representation of $-d$ as a sum of three
squares as above as above, is in one-one correspondence with a class
of orders $(\mathscr{O})$, isomorphic with the class
$(\mathscr{O}_\circ)$ and optimally contained in $\mathcal{J}$, so
that this number is given by (from Note to Theorem
\ref{chap4:sec9:thm4}, \S \ref{chap4:sec9}) 
$$
12 (1 - (\frac{D}{2})) h(D),~D=D(\mathscr{O}_\circ) = 4d \text{ or }d
$$
according as $d \not \equiv 1 \pmod 4$ or $d \equiv 1 \pmod 4$.

\textbf{7.} ~We shall now take up the third application, namely the
calculation of fixed points of a correspondence $T_n$. The fixed
points are of two typed, i) finite and ii) infinite. Firstly, we
shall consider the finite ones, i.e., points $\Gamma_\mathcal{J}. \tau
$ on $S_\mathcal{J}$ where $\im \tau > 0$. In case 
$$
n= m^2, T_n = \sum_{n (\mathcal{J} \nu_i)=m} \Gamma_\mathcal{J}~. \nu_i =
T^*_n + \Gamma_\mathcal{J}. m \text{ where } T^*_n = \sum_{\mathcal{J}
  \nu_i \neq \mathcal{J}m} \Gamma_\mathcal{J} \nu_i. 
$$

For\pageoriginale $\Gamma_\mathcal{J}. m$, all points are fixed points and Lefschetz'
theorem is not applicable, so that we consider only $T^*_n$. In case
$n \neq m^2$, if $\Gamma_\mathcal{J} \tau_\circ$ is a fixed point, then
$\Gamma_\mathcal{J}. \nu (\Gamma_\mathcal{J}. \tau_\circ) =
\Gamma_\mathcal{J}\nu (\tau_\circ) = \Gamma_\mathcal{J}(\tau_\circ)$ or $\nu
(\tau_\circ) = \varepsilon (\tau_\circ)$ for some $\nu$ of norm $n$
and $\varepsilon\in \Gamma_\mathcal{J}$. In other words,
$\varepsilon^{-1} \nu(\tau_\circ) = \tau_\circ$. Since
$\Gamma_\mathcal{J}. \varepsilon^{-1}\nu = \Gamma_\mathcal{J} \nu$, without
loss of generality, we may take $\nu (\tau_\circ) = \tau_\circ$. 

Let $\nu = \begin{pmatrix} a & b \\ c & d \end{pmatrix}; a, b, c, d$
are all real and $n (\nu)=n$. $t(\nu) = a + d = + t$ (say). 

Now, $\nu (\tau_\circ) = \tau_\circ \Rightarrow \dfrac{a \tau_\circ +
  b}{c \tau_\circ +d} = \tau_\circ$ or $\tau_\circ$ satisfies the
equation $\tau^2 - \dfrac{a - d}{c} \tau -b/c = 0$. Since
$\im \tau_\circ > 0$, it follows that the discriminant of this equation
$\dfrac{t^2 -4n}{c^2} < 0$ so that $t^2 -4n < 0$ is a necessary
condition for a solution of $\nu (\tau) = \tau$ existing in the finite
part of the upper half plane. If $\nu' = u + v \nu$, $u$ and $v$ rational,
then $\nu' (\tau) = \tau$ has the same set of solutions as $\nu (\tau
) = \tau$. 

Associate to any $\nu ( \in Q)$ of norm $n$ such that $\nu
(\tau_\circ) = \tau_\circ$ the order $\mathscr{O}= \mathcal{J} \cap k
(\nu)$, which, by definition, is optimally imbedded in $\mathcal{J}$. 

Now, 
$$
\displaylines{\hfill 
  \nu \tau_\circ = \tau_\circ \Rightarrow \eta^{-1_\nu}(\eta
  (\tau_\circ))= \eta ( \tau_\circ) ~\text{if}~  \eta \in
  \Gamma_\mathcal{J}\hfill \cr
  \text{so that} \hfill
  \left( \Gamma_\mathcal{J}. \eta^{-1_{\nu \eta }} \right) \left(
  \Gamma_\mathcal{J}. \tau_\circ\right) = \Gamma_\mathcal{J}. \eta \tau_\circ
  = \Gamma_\mathcal{J}\tau_\circ\hfill }
$$ 
Therefore to one fixed point
$\Gamma_\mathcal{J}. \tau_\circ$, we may make correspond a whole class
of orders $\bigg \{ \eta^{-1} \mathscr{O} \eta; \eta \in
\Gamma_\mathcal{J}\bigg \}$, which are optimally imbedded in
$\mathcal{J}$, and which contain a $\nu $ of norm $n$. 

Conversely,\pageoriginale given a class of orders $\bigg \{  \eta^{-1} \mathscr{O}
\eta ; \mathscr{O}= \mathcal{J} \cap k(\nu)\bigg\}$ (for which
$D(\mathscr{O})= \Delta < 0$ and $\mathscr{O}$ containing a $\nu$ of
norm $n$), then the solution of $\nu (\tau) = \tau $ is a fixed
point. 

So we have now a one-one correspondence between finite fixed points
$\Gamma_\mathcal{J}. \tau_\circ$ and classes of quadratic subfields
$K$ of $Q$ for which there is a $\nu \in K \cap \mathcal{J}$, of
norm $n$ and discriminant of $K$ is $< 0$. 

Let $\mathscr{O}= \mathcal{J} \cap K = \left[1, \dfrac{x +
    \nu}{f}\right]; x, f$ integers and $f > 0$. 

Here $t = tr(\nu)$ satisfies $(t^2 - 4n) < 0$ and $\Delta =
\dfrac{t^2- 4n}{f^2} \equiv 0$, $1 \pmod 4$. By $t, f$ and $n$, the
class of $\mathscr{O}$ is uniquely determined and we have as many
fixed points as there are such isomorphic classes with a negative
discriminant. By Theorem \ref{chap4:sec9:thm4} of \S \ref{chap4:sec9},
the number of such classes is 
given by the following sum over all admissible $\Delta$, 
$$
\sum _{\substack {t, f \\ {\Delta < 0 }}} \prod_{p|q_1} \left(1-
\left\{\frac{\Delta}{p}\right\}\right) \prod_{p|q_2} \left(1+
\left\{\frac{\Delta}{p}\right\}\right) h(\Delta). 
$$

Let $W(\Delta)$ denote the number of units $in
\mathscr{O}. (\mathscr{O} \subset K$ and $K$ being imaginary, this is
finite). Then, for all units $\varepsilon \in \mathscr{O}, \nu$ and
$\varepsilon \nu$ have different traces but correspond to the same
fixed point (except for $tr (\nu) = 0, \varepsilon = -1)$ so that we
would have counted each fixed point $W(\Delta)$ times in the above
sum, with the exception of those $\nu$ for which $tr (\nu) =0$ in
which case the fixed points belonging to $\nu$ would have been counted
only $\dfrac{1}{2}W (\Delta)$ times (since here for the unit
$\varepsilon = -1,~\varepsilon \nu = \bar{\nu}$). Even with this
correction the above sum would not yet be the number of fixed points,
as the following consideration shows: 

The sum would be the correct number of fixed points, if a
$\Gamma_\mathcal{J} \tau_\circ$ occurs only in one branch
$\Gamma_\mathcal{J} \nu_i$ of the correspondence $T_n = \sum
\limits_i \Gamma_\mathcal{J}. \nu_i$ 

But,\pageoriginale in fact $\Gamma_\mathcal{J}. \tau_\circ$ may be fixed by more
than one branch $\Gamma_\mathcal{J}.  \nu_i$. Let
$\Gamma_\mathcal{J}. \nu_1, \Gamma_\mathcal{J}. \nu_2$ be two branches
fixing $\Gamma_\mathcal{J}. \tau_\circ$. Let $\mathscr{O}_1= \bigg [
  1, \dfrac{x_1 + \nu_2}{f_2}\bigg]$ and $\mathscr{O}_2= \bigg [1,
  \dfrac{x_2 + \nu_2}{f_2}\bigg]$ be the orders associated with these
two branches. Then, if $(t_2, f_2) \neq (t_1, f_1)$, the fixed point
$\Gamma_\mathcal{J}. \tau_\circ$ would have been counted twice in the
above sum, as it should be. But it may happen that $(t_1, f_1) = (t_2,
f_2), i.e., t_1 =t_2$ and $f_1 = f_2$ and yet
$\Gamma_\mathcal{J}. \nu_1 \neq \Gamma_\mathcal{J}. \nu_2$. In other
words, $\nu_1 = \bar{\nu}_2$ and $\nu_1 \neq \varepsilon \nu_2 (
\varepsilon$, a proper unit). Therefore the number of fixed points of
$T_n$ is twice the above sum except for the terms $t = 0$ which should
be kept unchanged. So the number of fixed points is finally given by, 
\begin{equation}
  F = \sum_{t, f} \prod_{p|q_1} \left(1- \left \{ \frac{\Delta}{p}\right\}\right)
  \prod _{p/q_2} \left(1+ \left\{\frac{\Delta}{p}\right \}\right)
  \frac{h(\Delta)}{\omega (\Delta)} \tag{*} 
\end{equation}
$(\Delta = (T^2 -4n)f^{-2} \equiv 0, 1 \pmod 4$ and where $\omega
(\Delta)= \dfrac{1}{2}$ number of units of $\mathscr{O}(\Delta) =
\dfrac{1}{2}W (\Delta)$. 

ii)~ $n=m^2$. In this case, we have $T_n = T^*_n + \Gamma_\mathcal{J}.
m$. We only calculate the number of fixed points of $T^*_n$. The
considerations are the same as above with the exception that those
$\nu = \varepsilon. m$ must not be counted for $\varepsilon$, a unit
of $\mathscr{O}$. 

$\varepsilon = \pm 1$ would that $t = \pm 2m \Rightarrow t^2 - 4n = 0$
which had already been excluded. But $\varepsilon$ may be a third,
fourth or sixth root of unity. 
\begin{enumerate}[a)]
\item $ \varepsilon^4 = 1. t(\nu) = m t(\varepsilon) = 0$ since either
  $\varepsilon^2 -1 = 0$ or $\varepsilon^2 + 1 = 0$. $f = m$ and $(t^2
  - 4n)f^{-2} = -4$ so that $h(-4) = 1$ and $\omega (-4) =2$.\pageoriginale The
  number of fixed points of order $4$ is given by $(*)$ with these
  special values. 
  $$
  \displaylines{\text{i.e.,} \hfill = \frac{1}{2} \prod_{p|q_1} \left(1
    -\left(\frac{-4}{p}\right)\right) \prod_{p| q_2}   \left(1 +
    \left(\frac{-4}{p}\right)\right)\hfill } 
  $$
\item $\varepsilon^3 = 1$ or $\varepsilon^6 = 1$. Either
  $\varepsilon^2 + \varepsilon + 1 = 0$ or $\varepsilon^2 -
  \varepsilon + 1 = 0$, so that $t(\nu) = mt (\varepsilon) = \pm m$
  and $f =m; (t^2 -4n)f^{-2} = -3$. Further $h(-3)=1$ and $\omega
  (-3)=3$ in either case. The number of fixed points of order $3$ and
  $6$ is given by $2(*)$ since both are equal. 
  $$
  \displaylines{\text{i.e.,}\hfill  = \frac{2}{3} \prod_{p|q_1} \left(1 -
    \left(\frac{-3}{p}\right)\right) \prod_{p|q_2}\left(1+
    \left(\frac{-3}{p}\right)\right). \hfill } 
  $$
\end{enumerate}

Hence the total number of fixed points of $T^*_n$ is obtained by
subtracting the above terms from $(*)$. In other words, it is given by 
\begin{multline*}
\sum_{\substack {t,f \\ { f > 0 \text{ integral}}}} \prod_{p|q_1} \left(1-
\left\{ \frac{\Delta}{p}\right\}\right) \prod_{p/q_2} \left(1+ \left
\{\frac{\Delta}{p}\right\}\right) \frac{h(\Delta)}{\omega (\Delta)}-
\frac{1}{2} \prod_{p/q_1}\left(1-\left(\frac{-4}{p}\right)\right).\\ 
\Delta = (t^2 -4n)f^{-2}\equiv 0, 1 \pmod 4
\prod_{p/q_2}\left(1+\left(\frac{-4}{1p}\right)\right)\qquad \\
\qquad - \frac{2}{3}
\prod_{p|q_1}\left(1-\left(\frac{-3}{p}\right)\right)\prod_{p|q_2}
\left(1+\left(\frac{-3}{p}\right)\right) 
- 2 \sqrt{n} < t < 2 \sqrt{n} 
\end{multline*}

\textbf{8.} We shall now compute all the parabolic cusps of fundamental
domain $D$ of the proper unit group $\mathscr{O}_\mathcal{J}$ of an
order $\mathcal{J}$ of the type $(q_1, q_2)$ (here $q_1 = 1$ since
otherwise, $D$ is bounded). In fact, if $k$ is the number of prime
divisors of $q_2$, these cusps are the points $\tau = i \infty,
\dfrac{1}{q'}, q'| q_2 (q' \neq q_2); 2^k$ in number. 

The proof consists of two parts: (i) these points are inequivalent
with respect to the group $\mathscr{O}_\mathcal{J}$ or
$\Gamma_\mathcal{J}$. (ii) any rational cusp is equivalent to one
these by means os elements of $\Gamma_\mathcal{J}$. 

(i)~a)~\pageoriginale The points $\tau = i\, \infty$ and $\dfrac{1}{q'}, q'|q_2 (q'
\neq q_2)$ are inequivalent. 

If not, there will exist $\begin{pmatrix} a & b \\ q_2c &
  d \end{pmatrix} \in \mathscr{O}_\mathcal{J}$ such that $\dfrac{a
  \tau + b}{q_2 c \tau + d}= \dfrac{1}{q'}$ ($(\tau$ being the points
$i \infty$), $i.e$., $\dfrac {a}{q_2c}= \dfrac{1}{q'}$ which is
impossible since $q'$ is a proper divisor of $q_2$ and $a, c$ are
integers. 

b)~ The points $\dfrac{1}{q'}$ and $\dfrac{1}{q^*}(q' \neq q^*)$ are
inequivalent. Let $q_2 = q'q''$. Supposing $\dfrac{a+ bq'}{q_2 c+ dq'}
= \dfrac{1}{q^*}$ which $\Rightarrow \dfrac{a + bq'}{cq'' +d}=
\dfrac{q'}{q^*}, i.e., q^*(a+ bq') = q' (cq''+d)$. Therefore there
exists $p | q^*$ which divides either $q'$ or $cq'' +d$. But $(cq_2,
d) =1 \Rightarrow (cq'', d) = 1$, so that if $p|q'$ in which case
$p|q''$ and $p|cq'' + d$, which is not possible simultaneously. 

Hence $q^*|q'$- Without loss of generality, we could have assumed $q^*
\geq q'$ or else we can argue with the inverse transformation. 
Thus we arrive at a contradiction unless $q^* = q'$. 

(ii)~ Any parabolic cusp (which is of the form
$\dfrac{\alpha}{\beta},~ (\alpha, \beta) = 1)$ is equivalent to $i
\infty, \dfrac{1}{q'}; q'|q_2$. 
\begin{enumerate}[(a)]
\item Let $\dfrac{\alpha}{\beta}$ be a parabolic cusp $(\alpha,
  \beta)=1$. If $q_2|\beta$ let $\beta = q_2 c$. Then
  $\dfrac{\alpha}{\beta} \sim i \infty$ for $\dfrac{\alpha \tau +
    b}{q_2 c \tau + d} = \dfrac{\alpha}{q_2c}$ if $\tau = i \infty;
  (b,d)$ being chosen in such a way that $\begin{pmatrix} \alpha & b
    \\ q_2c & d \end{pmatrix} \in \Gamma_\mathcal{J}$ which is
  possible, since $(\alpha, \beta) = (\alpha, q_2 c) =1$. 
\item $q_2 \not{\mid} \beta$. Let $(\beta, q_2) = q'$ and $\beta = q'
  \beta''$. Then $(\beta'', q_2) = 1; q_2 = q'q''$. Our object is to
  find $\begin{pmatrix} a & b \\ q_2c & d \end{pmatrix} \in
  \Gamma_\mathcal{J}$ such that $\dfrac{a \alpha + b \beta}{q_2 c
    \alpha + d \beta}= \dfrac{1}{q'}$. ~ $(\alpha, \beta) = 1$ and $(
  \alpha q'', \beta'') = 1 \Rightarrow$ there exists integers
  $a_\circ, b_\circ, c_\circ, d_\circ$ such that $a_\circ \alpha +
  b_\circ \beta = 1$ and $c_\circ q'' \alpha + d_\circ \beta''=1$ or\pageoriginale
  $q_2 c_\circ \alpha + d_\circ \beta = q'$. Now $q'' c \alpha + d
  \beta'' = 1$ if $q'' c = q'' c_\circ + q'' t \beta''$ and $d =
  d_\circ - q'' t \alpha$, with arbitrary $t$. We determine $t$ in
  such a way that $d \equiv \alpha \pmod {q'}$. This is possible
  because $(\alpha q'', q')=1$. 
\end{enumerate}

Now, $a_\circ \alpha \equiv 1 \pmod {q'}$ and $d \equiv \alpha \pmod
{q'}$ imply that $a_\circ d \equiv 1 \pmod {q'}$. 

Similarly $a_\circ, b_\circ$ may be replaced by $a =a_\circ + s \beta$
and $b = b_\circ - s \alpha$.  

We choose $s$ in such a way that $\begin{vmatrix} a & b \\ q_2c &
  d \end{vmatrix} = 1$. For the same, we have 
\begin{align*}
  \begin{vmatrix} a & b \\ q_2c & d \end{vmatrix}  & = \begin{vmatrix}
    a & q'b \\ q''c & d \end{vmatrix} = \begin{vmatrix} a_\circ + s
    \beta & q'(b_\circ - s \alpha) \\ q''c & d \end{vmatrix}\\
  & \hspace{2cm} =\begin{vmatrix} a_\circ & q'b_\circ \\ q''c & d \end{vmatrix} +
  s \begin{vmatrix} \beta & -q\alpha \\ q''c &
    d \end{vmatrix}\\ 
  &  = a_\circ d - q_2 b_\circ c + q's (\text{ since } q_2 c \alpha +
  d \beta =q'). 
\end{align*}
$a_\circ d \equiv 1 \pmod {q'} \Rightarrow a_\circ d - q_2 b_\circ c
\equiv 1 \pmod {q'} =1- sq'$ (say). Thus $s$ can be determined. 

Collecting the above results, we have $\begin{vmatrix} \alpha & b
  \\ q_2c & d \end{vmatrix} =1$; $a \alpha + b \beta =1$ and $q_2 c
\alpha + d \beta = q'$ and thus 
$$
\frac{a \alpha + b \beta}{q_2 c \alpha + d \beta} = \frac{1}{q'}.
$$

After having found the parabolic cusps, we have still got to compute
the multiplicity of each such fixed point. 

a)~ The point $\tau =i \infty$. In this case, the local uniformiser
is given by $\zeta = e^{2 \pi i \tau}$ 

\begin{enumerate}[i)]
 \item $\sqrt{n} \not \equiv 0 \pmod 1$. Let $T_n = \sum \limits_{i}
   \Gamma_\mathcal{J} \nu _i$, where $\nu_i = \begin{pmatrix} n_1 &
     n_2 \\ 0 & n_3 \end{pmatrix} n_1, n_3 > 0; n_1 n_3 = n$ and $0
   \leq n_2 < n_3$. 
   
   Now,\pageoriginale $\nu_i (\tau ) = \dfrac{n_1}{n_3} \tau + \dfrac{n_2}{n_3}$ and
   the local uniformiser $\zeta$ is mapped into $\zeta_i = \zeta
   (\nu_i (\tau))= e^{2 \pi i \dfrac{n_1 \tau + n_2}{n_3}} =
   \zeta^{\dfrac{n_1}{n_3}} e^{2 \pi i n_2/n_3}$ Let $(n_1, n_3) = d$
   and $n_1 = n'_1 d, n_3 = dn'_3$. Then $\zeta_i =
   \zeta^{\dfrac{n'_1}{n'_3}}. e^{2 \pi i n_2/dn_3}$ Consider the
   functions $e^{2 \pi \gamma/n'_3}. \zeta^{n'_1/n'_3}(0 \leq r <
   n'_3)$. These are all analytic continuations of one another and
   they represent one branch of the correspondence $T_n$ and we have
   for the multiplicity for this branch $\min (n'_1,n'_3)$. But $0 \leq
   n_2 < d.n'_3$ implies that we have $d$ branches with the same
   $n'_1, n'_3$ and the total multiplicity corresponding to these
   branches is $d \min (n'_1, n'_3)= \min (n_1, n_3)$. This being the
   same for all branches whenever $n_1, n_3$ range through divisors of
   $n$ such that $n_1 n_3 = n$, we obtain for the totally multiplicity
   of the fixed point $\Gamma_\mathcal{J}. \tau, 2 \sum
   \limits_{\substack{ d|n \\ {d < \sqrt{n} }}} d$. 
 \item $\sqrt{n} \equiv 0 \pmod 1$. If $n = m^2,~T_n = T^*_n + \Gamma
   _\mathcal{J}m$. The above argument applies for $T^*_n$ and we have
   multiplicity $2 \sum \limits_ {\substack {d | n \\ {d < \sqrt{n}}}}
   d$ and corresponding to the term $\Gamma_\mathcal{J}. m$, we have
   $\zeta_i = e^{2 \pi i n_2/m.} \zeta^{m/m}$ and a similar argument
   as in $(i)$ shows the multiplicity to be $m$, but it is actually
   only $m-1$ since we should not take into account the term
   corresponding to $n_2 = 0, i.e., \begin{pmatrix} m & 0 \\ 0 &
     m \end{pmatrix}$ this being the identity map. 
\end{enumerate}

Hence the total multiplicity is here given by, 
$$
2 \sum_{\substack { d|n \\ {d < \sqrt{n}}}} d+ \sqrt{n}-1.
$$

(b)~ $\tau = \dfrac{1}{q'}, q'| q_2 ; q' \neq q_2$.

Firstly,\pageoriginale we shall calculate the local uniformising parameter at
$\Gamma _\mathcal{J}. \dfrac{1}{q'}$ and then see that the
multiplicity is exactly the same as that for $i  \infty$. 

If $\lambda = \begin{pmatrix} 1 & 0 \\ -q' & 1 \end{pmatrix}$, then
$\lambda \left(\dfrac{1}{q'}\right)= \infty$. We wish to fine a primitive
transformation $\rho \in \Gamma_\mathcal{J}$ such that $\rho
(\dfrac{1}{q'})= \dfrac{1}{q'}$. Let $\sigma$ be a transformation
fixing $\infty$, i.e., $\sigma = \begin{pmatrix}1 & s \\ 0 &
  1 \end{pmatrix}$ and all the transformations $\lambda^{-1} \sigma
\lambda$ fix $\dfrac{1}{q'}$. Now,  
$$
\lambda^{-1} \sigma \lambda = \begin{pmatrix}1 & 0 \\ q' &
  1 \end{pmatrix}\begin{pmatrix}1 & s \\ 0 &
  1 \end{pmatrix} \begin{pmatrix}1 & 0 \\ -q' & 1 \end{pmatrix}
= \begin{pmatrix}1-q's & s \\ -q'^2s & 1+q's \end{pmatrix} 
$$
is an element of $\Gamma_\mathcal{J}$ implies that $q''|s$ (where $q''
= \dfrac{q_2}{q'}$) and since we require a primitive transformation of
this type, we may take $s = q''$. 

Consider now $\zeta = e^{2 \pi i \dfrac{\lambda (\tau)}{q''}}; \tau$,
a point in a neighbourhood of $\dfrac{1}{q'}$. Now, $\dfrac{1}{q''}
\lambda \rho (\tau) = \dfrac{1}{q''} \sigma \lambda (\tau) =
\dfrac{1}{q''} \dfrac{\lambda (\tau) + q''}{1}= \dfrac{ \lambda
  (\tau)}{q''}+1$, by our choice of $\sigma$, so that $\zeta$ remains
unaltered. In other words, $\zeta$ takes any neighbourhood of
$\Gamma_\mathcal{J}. \dfrac{1}{q'}$ in $\mathcal{S}_\mathcal{J}$ to
the unit circle and hence is a local uniformiser at
$\Gamma_\mathcal{J}. \dfrac{1}{q'}$.  

Let $T_n = \sum \limits{i} \Gamma_\mathcal{J}. \nu_i; \nu_i
= \begin{pmatrix} n_1 & n_2 \\ 0 & n_3 \end{pmatrix}; n_1, n_3 > 0$;
$n_1 n_3 = n. 0 \leq n_2 < n_3$. 

(In case $n$ is a square, $T^*_n$ has to be considered). If
$\Gamma_\mathcal{J}. \nu_i \left(\dfrac{1}{q'}\right)=
\Gamma_\mathcal{J}. \dfrac{1}{q'}$ then $\nu_i \left(\dfrac{1}{q'}\right) =
\varepsilon_i \left(\dfrac{1}{q'}\right)$; $\varepsilon^{-1}_i \nu_i
\left(\dfrac{1}{q'}\right) = \dfrac{1}{q'}$. 

Call $\nu'_i = \varepsilon^{-1}_i \nu_i$. Since now $\lambda
\left(\dfrac{1}{q'}\right) = \infty$, then $\lambda \nu'_i \lambda^{-1}= \mu_i$
fix $\infty$, and hence are of the form $\begin{pmatrix} s_1 & s_2
  \\ 0 & s_3 \end{pmatrix}$ 
\begin{align*}
  \nu'_i & = \mu_i \lambda = \begin{pmatrix} 1 & 0 \\ q' &
    1 \end{pmatrix}\begin{pmatrix} s_1 & s_2 \\ 0 &
    s_3 \end{pmatrix}\begin{pmatrix} 1 & 0 \\ -q' & 1 \end{pmatrix}\\ 
  & = \begin{pmatrix} s_1-s_2q' & s_2 \\ q'(s_1-s_2q')-q's_3 &
    s_3+q's_2 \end{pmatrix} 
\end{align*}
in\pageoriginale $\mathcal{J}$ implies that $(s_1 - s_2q') - s_3 \equiv 0 \pmod
q''$. Such an $s_2$ can always be found out since $(q', q'')=1$. and
$s_2$ takes each residue class $\mod s_3$ exactly once. 

We shall prove that as $\bigg\{\mathcal{J}\nu'_i \bigg \}$run through a
system of distinct integral left ideals of norm
$n,\bigg\{\mathcal{J}\nu'_i \bigg \}$also run through a system of
distinct integral ideals of norm $n$; for
$\mathcal{J}\nu'_i=\mathcal{J}\nu'_j \Rightarrow$ for primes
$p|n$ (and hence $p \not {\mid} q_1 q_2$), 
$$
(\mathcal{J}\mu_i)_p = \mathcal{J}_p \lambda \nu'_i \lambda^{-1} =
\mathcal{J}_p \nu'_i \lambda^{-1} = \mathcal{J}_p \lambda \nu'_j
\lambda^{-1} = \mathcal{J}_p \mu_j 
$$
(for $\lambda$ is a unit of $\mathcal{J}p$).

For primes $p \not{\mid}n,~(\mathcal{J}\nu'_i)_p = \mathcal{J}_p =
(\mathcal{J}\nu'_j)_p$ and also $(\mathcal{J}\mu_i)_p = \mathcal{J}_p
= (\mathcal{J}\mu_j)_p$. 

Hence in all cases
\begin{multline*}
  \mathcal{J}\nu'_i = \mathcal{J}\nu'_j \Longleftrightarrow
  (\mathcal{J}\nu'_i)_p \\
  = (\mathcal{J}\nu'_j)_p
  \Longleftrightarrow (\mathcal{J}\mu_i)_p = (\mathcal{J}\mu_j)_p
  \Longleftrightarrow \mathcal{J}\mu = \mathcal{J}\mu_j. 
\end{multline*}

The local uniformising parameter $\zeta$ at $\dfrac{1}{q'}$ goes over
to $ \nu'_i \left(\dfrac{1}{q'}\right)$ and in fact 
\begin{align*}
  \nu'_i (\zeta_q') = e^{2 \pi i \lambda (\nu'_i (\tau))/q''}  & =
  e^{2 \pi i  \mu_i \lambda (\tau)/q''} \\ 
  & = e^{2 \pi i  \frac{\mathscr{S}_1}{\mathscr{S}_3}
    \frac{\lambda(\tau)}{q''}} e^{2 \pi i
    \frac{\mathscr{S}_1}{\mathscr{S}_3q''}}= \zeta^{\mathscr{S}_1 /
    \mathscr{S}_3}_{q'} e^{2 \pi i \mathscr{S}_2 /\mathscr{S}_3q''}   
\end{align*}
So, $\zeta_{q'}$ behave similar to $\zeta_\infty$ under the
correspondences and the sum of multiplicities of the branches is the
same as that for $\zeta_\infty$. 

The\pageoriginale parabolic cusps being $2^{k_2}$ in number, the number of parabolic
cusps with due multiplicity is given by $2 \sum \limits_{\substack {
    d| n \\ {d < \sqrt{n}}}} d + \gamma_n$  

$  \text{ where } \gamma_n =
  \begin{cases}
    \quad 0 & \text{ if } \sqrt{n} \not\equiv 0 \pmod 1 \\ 
    \sqrt{n}-1 & \text{ if } \sqrt{n} \equiv 0 \pmod 1. 
  \end{cases}
$
 
\begin{enumerate}[(i)]
\item If $\sqrt{n} \not\equiv 0 \pmod 1$ we had obtained by
  Lefschetz' fixed point theorem, $tr^1(T_n)= 2 \sum \limits_{d/n}d$-
  (number of finite fixed points of $T_n$) - (number of infinite fixed
  points of $T_n$). 
  \begin{multline*}
  \text{i.e.,}~ \tr^1(T_n) = 2 \sum_{d|n} d - \sum_{\substack {t,r \\ {f > o,
        \Delta = (t^2-4n)f^{-2}\equiv 0,1 \pmod 4 }\\ {-2 \sqrt{n} < t
        < 2 \sqrt{n}}}}\\ 
  \prod_{1-|q_1}
  \left(1-\left\{\frac{\Delta}{\mu}\right\}\right)
  \frac{h(\Delta)}{\omega(\Delta)} -\left(2^{k_2+1}
  \sum_{\substack{\alpha /n \\ { \alpha < \sqrt{n}}}}\right) 
  \end{multline*}
\item If $\sqrt{n} \equiv 0 \pmod 1, T_n = T^*_n +
  \Gamma_\mathcal{J}. m$ so that \qquad (in case $q_1 =1$). 
\end{enumerate}

$tr^1(T_n) = tr^1 (T^*_n) + tr^1 (T^*_n) + 2g$. By the application of
Lefschetz' fixed point theorem for the mapping $T^*_n$, 
$$
\left.
\begin{minipage}{6cm}
    total number of fixed points 
    (finite and infinite) of $T^*_n$    
    with due multiplicity 
\end{minipage}\quad  \right \}
= 2\left(\sum_{d|n}d-1\right) - tr^1(T^*_n).
$$

Therefore
\begin{align*}
\tr^1& (T^*_n)  = 2 \sum \limits_{d|n} d-2 
 - \text{(finite fixed points) - (infinite fixed points)} \\
  & = 2 \sum_{d|n} d-2 - \sum_{t, f} \prod_{p|q_1} \left(1-
  \left\{\frac{\Delta}{p}\right\}\right) \prod_{p|q_2}\left(1+
  \left\{\frac{\Delta}{p}\right\}\right) \frac{h}{\omega}(\Delta)\\ 
  & + \frac{1}{2} \prod_{p|q_1}
  \left(1-\left(\frac{-4}{p}\right)\right)  \prod_{p|q_2} (1 +
  \left(\frac{-4}{p}\right) + \frac{2}{3} \prod_{p|q_1}
  \left(1-\left(\frac{-3}{p}\right) \right)
  \prod_{p|q_2}\left(1+\left(\frac{-3}{p}\right)\right) \\ 
  & - 2^{k_2}\left(2 \sum _{\substack { d|n \\ { d < \sqrt{n}}}}d+
  \sqrt{n}-1\right) (\text{ in case } q_1 = 1). 
\end{align*}

From\pageoriginale \S \ref{chap2:sec6}, we have a formula for genus and using the expression for
the number of elliptic vertices we obtain for $g$, 
\begin{multline*}
2g = 2 - \frac{1}{2} \prod_{p|q_1} \left(1-\left(\frac{-4}{p}\right)\right)
\prod_{p|q_2}\left(1+\left(\frac{-4}{p}\right)\right)\\
-\frac{2}{3}\prod_{p|q_1}\left(1-\left(\frac{-3}{p}\right)\right)
\prod_{p|q_2}\left(1+\left(\frac{-3}{p}\right)\right) -2^{k_2} +
\frac{1}{6} \prod_{p|q_1}(p - 1) \prod_{p|q_2}(p+1). 
\end{multline*}

On adding the above two, we obtain finally
\begin{multline*}
  \tr^1 (T_n) = 2 \sum_{d|n}d-(2^{k_2 +1} \sum_{\substack { d_n \\ { d<
        \sqrt{n}}}}d, \text{ in case } q_1 = 1)\\ 
  -\sum \prod_{p|q_1}(1-
  \left\{\frac{\Delta}{p}\right\}) \prod_{p|q_2}\left(1+\left\{
  \frac{\Delta}{p}\right\}\right) \frac{h}{\omega}(\Delta) + \gamma_n 
\end{multline*}
$ 
\text{where}~ \gamma_n =
  \begin{cases}
    0 \text{ if } \sqrt{n} \not{\equiv} \pmod 1\\
    (-2^k 2 \sqrt{n}, \text{ in case } q_1 = 1) + \frac{1}{6} \prod
    \limits_{p|q_1}(p-1) \prod \limits_{p|q_2}(p+1),  \\
\hspace{5cm} \text{if}~ \sqrt{n}\equiv 0\pmod 1.
  \end{cases}$


Of course, all these formulae hold good only for $n$ such that\break $(n,
q_1. q_2) = 1$.  

\begin{note}
\begin{enumerate}[\rm (i)]
\item In case $g = 0$, this trace is always $0$. So the trace formula
  implies relations between the class numbers $h(\Delta)$ of different
  imaginary quadratic fields. An example is given in the case $q_1 =
  q_2 = 1$, where $\Gamma$ is the full modular group. 
  \begin{multline*}
    2 \sum_{\substack { d|n \\ d \geq \sqrt{n}}} - \sum_{\Delta = (t^2 -
      4n)f^2} \frac{h (\Delta)}{\omega(\Delta)}+ \gamma_n = 0,\\ 
    ~\text{where}~ \gamma_n = 
    \begin{cases} 
      \quad 0 & \text{ if } \sqrt{n} \not{\equiv} 0 \pmod 1 \\ 
      -\sqrt{n}+ \frac{1}{6} & \text{ if }\sqrt{n} \equiv 0 \pmod 1 
    \end{cases} 
  \end{multline*}

\item One may obtain other class number relations among which the
  following one is most remarkable: 
  $$
  \sum_{t \equiv \frac{n+1}{2}\pmod 2} \frac{h}{\omega}(t^2 -
  4n)f^{-2})= \frac{1}{3} \sum_{d|n}d - \sum_{\substack{d|n \\{ d \leq
        \sqrt{n}}}} *d (n \equiv 1 \pmod 2) 
  $$
  (M. Eichler, Jour. of the Ind. Math. Soc., 1956).
  
  Using\pageoriginale the above two relations, one can compute the
  class numbers of 
  imaginary quadratic fields, by a recursion scheme. 
\item These and similar class number relations are quoted in Dickson's
  ``History of the Theory of Numbers'', Vol. $III$, Chap. $VI$. They
  have been proved by application of the theory of elliptic modular
  function. Here we have seen that they originate from the topological
  background of that theory. The topological methods are even more
  powerful since they lead also to class number relations derived from
  quaternion algebras $Q \ncong \mathfrak{M}_2 (k)$ which
  cannot be obtained from the theory of modular functions. 
\end{enumerate}
\end{note}

Our consideration from a natural branch of algebraic number theory,
which contains yet a number of open problem. One of these is the
investigation of the algebraic geometric aspects of the
correspondences $T_n$ of those algebraic curves which are uniformized
by the groups $\Gamma_\mathcal{J}$. In case of $\Gamma_{\mathcal{J}}$
being the classical modular group, these investigations lead to the
theory of complex multiplication. 

Another question is the following one: the eigenvalues of the
representations of $T_n$ by the first homology group, are integers
from a totally real algebraic number field. What is the meaning of
this field? 

In case of $Q \cong \mathfrak{M}_2(k)$ it has been proved
(M. Eichler, Quaternare Formen und die Rienannsche Vermutung fur die
Kongruenzzetafunktion, Archiv der Mathematik, $V$, $1957$,
$p. 355-366$) that the absolute values of the eigenvalues of $T_n$,
for $n$, a prime are $\leq 2n^{\dfrac{1}{2}}$, up to a finite number
of exceptions at most. For an arbitrary $n$, there exists for every
$\varepsilon > 0$, a constant $C_\varepsilon$ such that values are
$\leq C_\varepsilon n^{\dfrac{1}{2}+ \varepsilon}$. 
