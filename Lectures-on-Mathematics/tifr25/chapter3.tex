
\chapter{Prolongation of Exterior Differential Systems}\label{chap3} % Chapter III

\section{}\label{chap3:sec3.1}%Sec 3.1

In\pageoriginale the chapter, we shall introduce the notion of jets of mappings of
one manifold into another and give the construction of prolongation of
a given differential system. For this purpose we make use of the
notion  of $\ell $- jets of mappings. 

Let $M'$ and $M$ be two infinitely differentiable $(C^\infty)$
manifolds and let $x'$ and $x$ denote points of $M'$ and  $M$
respectively. Let $f$ be a $C^\infty$ mapping  of an open
neighbourhood of $x'$ in $M'$ into $M$. We shall introduce an
equivalence relation in the set of all such $C^\infty$ maps $f$. The
open neighbourhood of $x'$ may depend on the function $f$. Let $(w_1,
\ldots , w_{n'})$ and $(x_1,  \ldots , x_n)$ be coordinate systems at
$x'$ in $M'$ and $x$ in $M$ respectively. Let $\ell$ be an integer
$\ge 0$. 

\begin{defi*}
  Two  $C^\infty$ mappings $f$ and $g$, of open neighbourhood of $x' $
  in $M'$ into $M$, are said to be $\ell$-equivalent, and is denoted
  by $f \tilde{\ell} g$, if , for every $h \le \ell$, 
  $$
  \frac{\partial^h f_i}{\partial w_{i_1} \cdots \partial w_{i_h}} (x')
  = \frac{\partial^h g_i}{\partial w_{i_1} \cdots \partial w_{i_h}}
  (x') 
  $$
  for all $(i_1,  \ldots , i_h)$ where $f_i$ and $g_i$ are the
  components of  $f$ and $g$ respectively. 
\end{defi*}   

Clearly $\tilde{\ell}$ is an equivalence relation and this definition
of equivalence is independent of the choice of the coordinate systems
as\pageoriginale can be easily verified. 
   
\begin{defi*}
  An equivalence class of $C^\infty$ mappings such as above under
  $\tilde{\ell}$ is called an $\ell${\em jet of mappings} at $x'$. 
  
  An  $\ell$-jet of mappings at $x'$ containing a mapping $f$ is
  denoted by $j^\ell_{x'}(f)$.  
\end{defi*} 
    
\begin{example*}
  Let $M'$ be the real line $R$ and $x'$ the origin. Then any
  $C^\infty$ mapping $f$ of an open neighbourhood of $0$ into  $M$ is
  a $C^\infty$ curve through the point $x=  f(0)$ in $M$. A jet is
  thus a generalisation of the notion of high order of contact of two
  curves. 
\end{example*}  

Let $J^\ell (M',M)= \cup \left\{ j^\ell_{x'} (f): x' \in M' \right\}$
be the set of all $\ell$-jets of $C^\infty$ mappings of $M'$ into
$M$. Let $\alpha$ (\resp  $\beta$) be the mapping of $J^\ell (M', M)$
onto $M'$(\resp  $M$) which associates to every jet $j^{\ell}_{x'}(f)$
in $J^\ell (M', M)$ the point $x'$ of $M'$ (\resp  $x$ of $M)$. the
point $x'$ is called the source and the point $x=  f(x')$ the  target
of the jet $j^\ell_{x'}(f)$. 

We can provide $J^\ell(M',M)$ with the structure of a $C^\infty$
manifolds as follows: Let $X \in J^\ell(M', M)$ with $\alpha (X)= x'
\beta (X)=x$, and suppose $V'$ and $V$ be coordinate neighbourhoods of
$x'$ and $x$ in $M'$ and $M$ respectively. Let $(w_1,  \ldots ,
w_{n'})$ and $(x_1,  \ldots , x_n)$ be the coordinate systems at $x'$
and $x$ in $V'$ and $V$ respectively. Denote by $\mathscr{H}$ the set 
$$
\mathscr{H} = \left\{ X' \in J^\ell (M', M): \alpha (X') \in V' ~\text{
  and }~ \beta (X') \in V \right\} 
$$
$\mathscr{H}$ can be taken as a coordinate neighbourhood of $X$ in
$J^\ell (M', M)$  defining\pageoriginale  the manifold structure. The explicit
coordinate system at $X$ in $\mathscr{H}$ can be given as follows: 

Suppose $X'= j^\ell_y (f) \in \mathscr{H}$ with $\alpha (X') = y$. The mapping 
$$
X' \to \left( \alpha(X'), \beta (X'), \ldots ,\frac{\partial^h
  f_j}{\partial w_{i_1} \cdots \partial w_{i_h}} (y), \ldots \right). 
$$
of $\mathscr{H}$ into $V' \times V \times \left\{ ( \ldots ,P^{i_1
  \cdots i_h}_j  , \ldots) \right\}$, where $h \le \ell, 1 \le i_1,
\ldots , i_h \le n, j=1, \ldots , m)$, is objective. Here the
functions $P^{i_1,  \ldots i_h}_{j}$ are assumed to be symmetric with
respect to $i_1 \cdots i_h$. Clearly this mapping is well defined
independent of the choice of the respective $f$ of $j^{\ell}_y
(f)$. Thus a coordinate system at $X'$ is  
$$
\alpha (X') , \beta (X') , \ldots , \frac{\partial^h f_j}{\partial
  w_{i_1} \cdots \partial w_{i_h}} (y), \ldots  
$$
  
The change of coordinate can again be verified to be
$C^\infty$. Therefore, this defines a $C^\infty$ manifold structure on
$J^\ell(M',M)$. 
  
When $M'$ and $M$ are real analytic manifolds, $J^\ell (M', M)$ can
also be made a real analytic manifold in the same way. This is the
case in which we will be interested in. It is, now, easy to see that
$(J^\ell(M',M), M' \times M, \alpha \times \beta)$ is a fibre bundle
over $M' \times M$ with projection mapping $\alpha \times \beta$ and
the structure group a linear group. 
  
Let $(M, M', \tilde{\omega})$ be a fibred manifold. Let us denote by
$J^\ell (M,M', \tilde{\omega})$ the set of all jets $X=
j^\ell_{x'}(f)$ in $J^\ell (M',M)$ of cross-sections $f$ of $(M,M',\break
\tilde{\omega})$ over open neighbourhoods of $x'$ in $M'$. $J^\ell
(M,M', \tilde{\omega})$  is a real analytic submanifold of  $J^\ell
(M',M)$ as\pageoriginale is clear from the following: 

Let $(x, y)$ be a coordinate system in $(M, M' , \tilde{\omega})$, and
let $\mathcal{V}$ be the coordinate neighbourhood in $J^\ell (M', M)$
associated with $(x, y)$. Then $\mathcal{V} \cap J^\ell (M,M',
\tilde{\omega})$ is a submanifold. In fact, let $X \in \mathcal{V}$
with $\alpha (X)= x^o$ be represented by a mapping $(X) \to (f(x),
g(x))$. $X$ has the coordinate $x^0,f(x^0), g(x^0) , \ldots
,\dfrac{\partial^h f_j}{\partial {x_{i_{1}}}\cdots
  \partial{x_{i_{h}}}} (x^o) , \ldots , \dfrac{\partial^h
  g_\lambda}{\partial{x_{i_{1}}}\cdots \partial{x_{i_{h}}}} (x^o) ,
\ldots)$ in $\mathscr{V}$. Then $X$ is in $J^\ell (M, M',
\tilde{\omega})$ if and only if $f(x^o)= x^o, \dfrac{\partial
  f_j}{\partial x_i}(x^o) = \delta^j_i,
\dfrac{\partial^{h_{f_{j}}}}{\partial{x_{i_{1}}}\cdots
  \partial{x_{i_{h}}}}$ $(x^o) =0$ for $h \ge 2$. Thus $\mathscr{V}
\cap J^\ell (M,M', \tilde{\omega})$ is a submanifold. Moreover, it has
a coordinate system $(x,y,y^{i_1 \cdots i_h)}$. More explicitly $X=
j^\ell_x(g)$, where $g(x') = (x', g_\lambda (x')$ is a cross-section,
has the coordinates: $y_\lambda = g_\lambda (x), y^{i_{1}\cdots i_{h}}
= \partial^{h_{g_\lambda}}/  \partial {x_{i_{1}}}\cdots
\partial x_{i_{h}}$. Thus coordinate system in $J^\ell(M,M',
\tilde{\omega})$ is called the coordinate system corresponding to the
coordinate system $(x,y)$ of $(M,M',\tilde{\omega})$. 
  
Let $U'$ be an open set of a manifold $M'$ and $M$ be another
manifold. For a mapping $f$ of  $U'$ into $M$ we denote by $j^\ell
(f)$ the submanifold of $J^\ell (M',M)$ defined by the mapping   
\begin{gather*}
  U' \to J^\ell(M', M) \\
  x' \to j^\ell_{x'}(f) 
\end{gather*}  
    
This mapping is injective because $\alpha o j^\ell_{x'}(f) = x'$. If
$f$ is a  cross-section\pageoriginale of a fibred manifold $(M, m', \tilde{\omega})$
over an open set $U'$ of $M'$ then $j^\ell (f)$ is a cross-section of
the fibred manifold $(J^\ell (M, M' \tilde{\omega}, ), M', \alpha)$. 

\section{}\label{chap3:sec3.2}% sec 3.2

Consider now a fibred manifold $(D, D', \tilde{\omega})$ where $D$ and
$D'$ are domains in Euclidean spaces $\underbar{R}^p \times
\underbar{R}^m$ and $\underbar{R}^p$ respectively, and where
$\tilde{\omega}$ is the projection. We shall denote by $\prod ^{[1]}
(\ell)$ the set of all differential forms $\omega$ of degree $1$ on
$J^\ell (D, D', \tilde{\omega})$ such that for any cross- section $f$
of $(D, D', \tilde{\omega})$ over an open set of $D'$, the restriction
$\omega | j^{\ell_{(f)}}$ is zero. This can equivalently be expressed
by saying $j^\ell (f)^* (\omega) = 0$ when $j^\ell (f)$ is regarded as
cross- section of $(J^\ell (D, D', \tilde{\omega}), D', d)$. 

\setcounter{proposition}{0}
\begin{proposition}\label{chap3:sec3.2:prop1}%prop 1
  $\prod ^{[1]} (\ell)$ is finitely generated over the ring of real
  analytic functions $\wedge^0 (J^\ell (D, D', \tilde{\omega}))$. 
\end{proposition}

More precisely, if $(x, y)$ is a coordinate system in $(D, D',
\tilde{\omega})$ and if $(x', x, y, \ldots , y^{i_1 \ldots i_h},
\ldots)$ is the corresponding coordinate system of $J^\ell (D,\break D',
\tilde{\omega})$ then $\prod^{[1]}(\ell)$ is generated over $\wedge^0
(J^\ell(D, D', \tilde{\omega}))$ by  
\begin{align*}
  \omega_\lambda & = dy_{\lambda} - y^i_\lambda dx_i \\
  \omega^{i_1 \cdots i_h}_\lambda & = dy^{i_1 \cdots i_h}_{\lambda} -
  y^{i_1 \cdots i_h i}_\lambda dx_i ~(h \leq \ell - 1) 
\end{align*}

\begin{proof}
  First of all we shall show that $\omega^{i_1 \cdots i_h}_\lambda$
  are in $\prod^{[1]} (\ell)$. Let $f$ be a cross-section of $(D, D',
  \tilde{\omega})$ over an open set $U'$ of $D'$ represented by
  $y_\lambda = f_{\lambda} (x_1, \ldots , x_q)$ then $j^\ell (f)$ is
  represented\pageoriginale by  
  \begin{align*}
    y_{\lambda} & = f_{\lambda} (x); \\
    y^{i_1 \cdots i_h}_\lambda & = \frac{\partial^h f_\lambda} {\partial
    x_{i_1} \cdots \partial x_{i_h}} 
  \end{align*}
\end{proof}

Hence 
\begin{align*}
  j^\ell (f)^* \omega_{\lambda} &  = df_{\lambda} - \left(\frac{\partial
    f_{\lambda}}{\partial x_i}\right) dx_i = 0; \\ 
  j^\ell (f)^* \omega^{i_1 \cdots i_h}_{\lambda} &= d
  \left(\frac{\partial^h f}{\partial x_i \cdots \partial
    x_{i_h}}\right) - \left[
    y^{i_i \cdots i_h.i}_{\lambda} j^\ell (f) \right] dx_i \\ 
  & = d   \left(\frac{\partial^h f}{\partial x_{i_1} \cdots \partial x_{i_h}}\right)
  - \frac{\partial^{h + 1} f_{\lambda}}{\partial x_{i_1} \cdots
    \partial x_{i_1} \partial x_i} dx_i = 0. 
\end{align*}

Conversely, let $\omega \in \prod^{[1]} (\ell)$. We know that $\bigg\{
dx_i , dy_\lambda^{i_1 \cdots i_h}\bigg\}$ form a basis of Pfaffian forms on
$D$. Hence any $\omega \in \prod^{[1]} (\ell)$ can be expressed as 
$$
\omega = a^i dx_i + b^{\lambda}_{i_1 \cdots i_\ell} dy^{i_1 \cdots
  i_\ell}_\lambda + c^{\lambda}_{i_1 \cdots i_h} \omega^{i_1 \cdots
  i_h}_{\lambda} (h \leq \ell -1) 
$$

Let $f$ be any cross - section of $(D, D', \tilde{\omega})$
represented by $y_{\lambda} = f_{\lambda} (x)$. Then we obtain 
$$
j^\ell (f)^* (\omega) = [a^i o j^\ell (f) dx_i] + \bigg[
  b^{\lambda}_{i_1 \cdots i_\ell} o j^{\ell}(f) \bigg]
\frac{\partial^{\ell + 1}f}{\partial x_{i_1} \cdots \partial x_{i_\ell}
  \partial x_i}dx_i 
$$

These $j^\ell (f)^* (\omega)$ are differential forms on $D'$. A
necessary and sufficient condition for $j^\ell (f)^* (\omega) = 0$ is
that 
$$
\bigg[ a^i o j^\ell (f) \bigg] dx_i + \bigg[ b^\lambda_{i_1 \cdots
    i_\ell} oj^\ell (f) \bigg] \frac{\partial^{\ell + 1}f}{\partial
  x_{i_1} \cdots \partial x_{i_\ell} \partial x_i} dx_i = 0. 
$$

This\pageoriginale equality holds for any cross-section $f$ if and only if $a^i = 0$,
$b^{\lambda}_{i_1 \cdots i} = 0$. Therefore we have $\omega =
c^{\lambda}_{i_1 \cdots i_h} \omega^{i_1 \cdots i_h}_{\lambda}$ so
much so that $\omega^{i_1 \cdots i_h}_{\lambda}$ generate $\prod^{[1]}
(\ell)$.  

Let us denote by $\prod (\ell)$ or by $\prod (\ell; (D, D'
\tilde{\omega}))$ if there is any possibility of confusion, the
differential system generated by $\prod^{[1]} (\ell)$ over the ring
$\wedge^0 (J^\ell (D, D' , \tilde{\omega}))$. 

\begin{proposition}\label{chap3:sec3.2:prop2} % \propo 2
  Let $F$ be a cross-section of the fibred manifold $(J^\ell (D$, $D',
  \tilde{\omega}), D', \alpha)$ over an open set $U$ of $D'$. Then
  there exists a cross-section $f$ of $(D, D' \tilde{\omega})$ over
  $U$ such that $F = j^\ell (f)$ if and only if $F$ is an integral of
  $\prod (\ell)$. 
\end{proposition}

\begin{proof}
  It is immediate that $F$ is an integral of $\prod (\ell)$ if $F =
  j^{\ell}(f)$ because of the definition of $\prod (\ell)$. 
\end{proof}

Conversely, $F$ being a cross-section it can be expressed by
$y_{\lambda} = F_{\lambda}(x), y^{i_1 \cdots i_h}_{\lambda} = F^{i_1
  \cdots i_h}_{\lambda} (x)$ so that we can write  
$$
F^* \omega^{i_1 \cdots i_h}_\lambda = dF^{i_1 \cdots i_h}_{\lambda} -
F^{i_1 \cdots i_h i}_{\lambda} (x) dx_i. 
$$

$F$ being an integral of $\prod (\ell), F^* \omega^{i_1 \cdots
  i_h}_{\lambda} = 0$ and therefore we obtain that $F^{i_1 \cdots i_h
  i}_{\lambda} (x) = \dfrac{\partial F^{i_1 \cdots
    i_h}_{\lambda}}{\partial x_i} (x)$. Therefore $F = j^\ell (f)$
where $f$ is represented by $y_{\lambda} = f_{\lambda}(x)$ over the
open set $U$ of $D'$ and this proves the existence of a section $f$ of
$(D,D', \tilde{\omega})$  such that $F = j^\ell (f)$. 

Suppose $(D, D', \tilde{\omega})$ is a fibred manifold with $\dim D' =
p$. Then we can identify $J^1 (D, D' , \tilde{\omega})$ with of
$\mathscr{G}^p (D, D'. \tilde{\omega})$ canonically by means of the
following map: Let $X \in  J^1 (D, D', \tilde{\omega})$. If $X = J^1_z
(f)$ we\pageoriginale associate to $X$ the element $(d f)_{z'} ((D')_{z'})$ of
$\mathscr{G}^p (D, D', \tilde{\omega})$. Here we observe the fact that
$(df)_{z'}$, is injective. It is clear that this canonical identification
is independent of the choice of the representative section $f$ of the
jet $X$. 

\section{}\label{chap3:sec3.3} %sec 3.3

We shall define the notion of $\ell$ jets of differential forms on a
$C^\infty$ manifold $M$. Let $z \in M$ and $(x_1, \ldots , x_n)$ be a
coordinate system at $z$ in $M$. Let $\varphi$ and $\theta$ be two
differential forms of the same degree (say a ) having the following
expressions with respect to the coordinate system $(x_1, \ldots ,
x_n)$: 
$$
\displaylines{\hfill 
  \varphi = \sum_{i_1 < \cdots < i_a} \varphi^{i_1 \cdots i_a} dx_{i_1}
  \wedge \cdots \wedge dx_{i_a} \hfill \cr 
  \text{and} \hfill \theta = \sum\limits_{i_1 < \cdots < i_a} \theta^{i_1 \cdots i_a}
  dx_{i_1} \wedge \cdots \wedge dx_{i_a}\phantom{and}\hfill} 
$$
respectively. 

\begin{defi*}%defi 0
  $\varphi$ is said to be $\ell -$ equivalent to $\theta$, and is
  denoted by $\varphi \tilde{\ell}$, if 
  $$
  j^{\ell}_z \left(\varphi ^{i_1 \cdots i_a}\right) = j^{\ell}_z
  \left(\theta ^{i_1 \cdots i_a}\right) 
  $$
  It is easy to verify that $\tilde{\ell}$ is an equivalence relation and is
  independent of the choice of the coordinate system. 
\end{defi*}

\begin{defi*}%defi 0
  An equivalence class of differential forms is called an $\ell-$ {\em
    jet of differential forms } on $M$ at $z$ and is denoted by
  $j^{\ell}_z (\varphi)$.  
\end{defi*}

The following are almost immediate consequences of this definition.
\begin{enumerate}[(1)]
\item $j^{\ell + 1}_z (\varphi) = j^{\ell + 1}_z (\theta)$\pageoriginale implies
  $j^{\ell}_z ( d \varphi) = j^{\ell}_z (d \theta)$. 
\item If $M'$ and $M$ are two $C^\infty$ manifolds, $f$ and $g$ are
  two $C^\infty$ maps of $M'$ into $M$, and $\varphi$ and $\theta$ are
  two differential forms on $M$ such that $ j^\ell_{z'}(f) =
  j^{\ell}_{z'} (g)$ and $j^{\ell}_z (\varphi) = j^{\ell}_z (\theta)$
  where $z = f(z') = g(z')$ then $j^{\ell - 1}_{z'} (f^* \varphi) =
  j^{\ell -1}_{z'} (g^* \theta)$. 
\end{enumerate}

Now consider an exterior differential system $[\sum , (D, D',
  \tilde{\omega})]$ with independent variables. We pose the following
definition. 

\begin{defi*}%defi 0
  An $\ell -$ jet $X \in J^\ell (D, D', \tilde{\omega})$ is an
  integral $\ell -$ jet of the system $[ \sum , (D, D',
    \tilde{\omega}]$ if the jet $X = j^{\ell}_z (f)$ satisfies
  $j^{\ell - 1}_{z'} (f^* \varphi) = o_{z'}$ the zero $\ell -$ jet of
  differential forms at $z'$, for every $\varphi \in (\sum)$. 
\end{defi*}

This is again independent of the choice of $f$. By the canonical
identification of $J^1 (D, D' \tilde{\omega})$ and $\mathscr{Y}^p (D,
D', \tilde{\omega})$, the notion of integral 1-jets is equivalent to
the notion of $p$-dimensional integral elements. 

Fix a coordinate system $(x, y)$ of $(D, D' , \tilde{\omega})$. For
any $\varphi$ in $\wedge^a (D)$ and for set of integers $i_1 < \cdots
< i_a$ define a mapping $F^{i_1 \cdots i_a}_\varphi$ of $J^\ell (D, D'
, \tilde{\omega})$ into $J^\ell (D', R)$ by setting $F^{i_1 \cdots
  i_a}_\varphi (X) = j^{\ell - 1}_{\alpha (X)} (\varphi^{i_1 \cdots
  i_a}_f)$ where $X = j^{\ell}_{\alpha (X)} (f) \in J^\ell (D, D
\tilde{\omega})$ and $\varphi^{i_1 \cdots i_a}_f$ are the coefficients
in $f^* \varphi = \sum \varphi^{i_1 \cdots i_a}_f dx_{i_1 \wedge
  \cdots \wedge dx_{i_a}}$. Let $(x, \ldots ,w^{i_1 \cdots i_r}, \ldots) r
\leq \ell - 1$, be a coordinate system in $J^{\ell -1} (D' R)$ where
$w^{i_1 \cdots i_r}$ is given by $w^{i_1 \cdots i_r} (L) =
\dfrac{\partial r_g}{\partial x_i \cdots \partial x_{i_r}}$ for any
$j^{\ell -1 }_z (g) = L \in j^{\ell - 1} (D'$, $R)$. Denote $w^{i_1
  \cdots i_r} (F^{k_1 \cdots k_a}_\varphi (X))$ by
$F^{k_1\cdots k_a ; i_1 \cdots i_r}_\varphi (X)$. 

In\pageoriginale particular $F^{k_ 1\cdots k_a}_{\varphi} = \varphi^{k_ 1\cdots k_a
}_f$ (the case $r = 0$). It can be verified that $F^{k_ 1\cdots k_a ;
  i_1 \cdots i_\varphi}$ is a real analytic function on $J^\ell (D, D'
\tilde{\omega})$, when the form $\varphi$ is real analytic. 

\begin{remarks*}%rem 1
\begin{enumerate}[\rm (1)]
\item Given a differential system $[\sum ,  (D, D' \tilde{\omega})]$
  with independent variables, an $\ell -$ jet $X$ in $J^\ell  (D, D',
  \tilde{\omega})$ is an integral $\ell -$ jet of the system if and
  only if each $F_\varphi^{k_ 1\cdots k_a ; i_1 \cdots i_r} (X) = 0$ for
  any $\varphi \in \sum ^{(a)} (0 \leq k_1, \ldots , k_a \leq p ; 1
  \leq i_1, \ldots i_r \leq p; a = 0,1, \ldots ; r \leq l -1$). This is
  an immediate consequence of the definition	of an integral $\ell
  -$ jet of such a differential system. 
\item $F_\varphi^{k_ 1\cdots k_a ; i_1 \cdots i_r}$ is symmetric with
  respect to $i_1, \ldots i_r$ and anti-symmetric with respect to
  $k_1 \cdots k_a$. 
  
  We shall denote by $\mathscr{F}^\ell (\sum)$ the set of all $F_\varphi^{k_
    1\cdots k_a ; i_1 \cdots i_r}$ where $\varphi \in \sum^{(a)}, (a =
  0, 1, \ldots $ and $r \leq \ell -1)$. Therefore an $\ell -$ jet $X$
  is an integral of $[\sum , (D, D', \tilde{\omega})]$ if and only if
  $F(X) = 0$ for every $F \in \mathscr{F}^\ell (\sum)$. 
\item If $\varphi , \psi$ are two differential forms of degree a, then
  $F^{k_ 1\cdots k_a ; i_1 \cdots i_r }_{\varphi + \psi} = F^{k_
  1\cdots k_a ; i_1 \cdots i_r }_{\varphi } + F^{k_ 1\cdots k_a ; i_1
  \cdots i_r }_{\psi} $. 

\item If $\varphi$ is a differential form of degree $a_1$ and $\psi$
  is a differential form of degree $a_2$, then, setting 
$$
a = a_1 + a_2
  F^{k_ 1\cdots k_a ; i_1 \cdots i_r }_{\varphi + \psi} = 0 \mod F^{h_
    1\cdots h_{a_2} ; j_1 \cdots j_s }_{\psi}, s \leq r).
$$  
\end{enumerate}
\end{remarks*}

Denote by $\rho^{\ell'}_{\ell} (\ell' \geq \ell)$ the natural projection of
$J^{\ell '} (D, D', \tilde{\omega})$ onto $J^\ell (D, D',
\tilde{\omega})$. If we write  ${}^{(\ell)}  F^{k_ 1\cdots k_a ; i_1 \cdots
  i_r }_{\varphi}$ the function $F^{k_ 1\cdots k_a ; i_1 \cdots i_r
}_{\varphi}$ on\break $J^\ell (D, D' , \tilde{\omega})$ for the sake of
precision, then 
$$
(\ell') F^{k_ 1\cdots k_a ; i_1 \cdots i_r }_{\varphi} = {}^{(\ell)} F^{k_
  1\cdots k_a ; i_1 \cdots i_r }_{\varphi} o \rho^{\ell'}_\ell 
$$

Because\pageoriginale of this relation, there will be no confusion
when we omit the 
index $(\ell)$ in ${}^{(\ell)} F^{k_ 1\cdots k_a ; i_1 \cdots i_r
}_{\varphi}$. Thus $F^{k_ 1\cdots k_a ; i_1 \cdots i_r }_{\varphi}$ is
a function on $J^\ell\break (D, D' \tilde{\omega})$ for $\ell \geq r+1$. 

\begin{proposition}\label{chap3:sec3.3:prop3}% propos 3
  If $\varphi$ is a differential form of degree $a$, then for $r \leq
  \ell - 2$ we have the identity 
  $$
  dF^{k_ 1\cdots k_a ; i_1 \cdots i_r }_{\varphi} \equiv F^{k_ 1\cdots
    k_a ; i_1 \cdots i_r i}_{\varphi} dx_i \pmod{\prod (\ell)}. 
  $$
\end{proposition}

\begin{proof}
  Suppose $j^\ell_z (f) = x \in J (D, D', \tilde{\omega})$; then we have 
  $$
  f^*\varphi = \sum_{k_1 < \cdots < k_a} \varphi_f^{k_1 \cdots k_a}
  dx_{k_1} \wedge \cdots \wedge dx_{k_a} 
  $$
\end{proof}

Therefore, $F^{k_ 1\cdots k_a ; i_1 \cdots i_r }_{\varphi} (X) = \left(
\dfrac{\partial \varphi^{k_1 \cdots k_a}_f}{\partial x_{i_1} \cdots
  \partial x_{i_r}}\right)_{z = \alpha(X)}$ On the otherhand, because
$dx_i, \omega_{\lambda}, \ldots \omega^{i_1 \cdots i_a}_{\lambda} (a
\leq \ell - 1), dy_\lambda^{i_1 \cdots i_\ell}$ form a base for Pfaffian forms on
$J^\ell (D, D', \tilde{\omega})$ and $\omega^{i_1 \cdots i_a}_{\lambda}$
generate $\prod (\ell)$, we can write 
$$
dF^{k_ 1\cdots k_a ; i_1 \cdots
  i_r }_{\varphi}\equiv A_j dx_j + A^{\lambda}_{j_i \cdots j} dy^{j_1
  \cdots j_1}_\lambda
$$ 
modulo $\prod (\ell)$. But for $r = \ell - 2,
F^{k_ 1\cdots k_a ; i_1 \cdots i_r }_{\varphi}$ are functions only of
the arguments $x, y, \ldots ,y^{h}_{\lambda}, \ldots , y^{h_1 \cdots
  h_{r + 1}}_{\lambda}$. Hence the terms $A^\ell_{j_1 \cdots j_\ell}
dy^{j_1 \cdots j_\ell}_{\lambda}$ do not appear in the expression of
$dF^{k_ 1\cdots k_a ; i_1 \cdots i_r }_{\varphi}$, i.e., $dF^{k_
  1\cdots k_a ; i_1 \cdots i_r }_{\varphi} \equiv A_j dx_j $ ~modulo~
$\prod(\ell)$. 

Now\pageoriginale $j^{\ell}(f)$ being a cross-section of $J^\ell (D, D' ,
\tilde{\omega})$ over an open neighbourhood  of $z = \alpha (X)$ we
obtain 
$$
j^{\ell} (f)^* \left(dF^{k_ 1\cdots k_a ; i_1 \cdots i_r }_{\varphi}\right)
\equiv \left[A^j o j^\ell (f)\right] dx_j ~\text{modulo}~ [j^{\ell}(f)^*
  \left(\prod(\ell)\right)] 
$$ 

Since  $j^\ell (f)^* (\prod (\ell)) = 0$ the above congruence becomes
an inequality,  
$$
(j^{\ell}(f))^* \left(dF^{k_ 1\cdots k_a ; i_1 \cdots i_r }_{\varphi}\right) =
[A^j o j^\ell (f)] dx_j. 
$$

\noindent
On the other hand
\begin{align*}
  j^\ell(f)^* (dF_\varphi^{k_ 1\cdots k_a ; i_1 \cdots i_r }) & =
  d(F^{k_ 1\cdots k_a ; i_1 \cdots i_r }_{\varphi} o j^\ell(f)) \\ 
  & = d \bigg( \frac{\partial^r \varphi_\ell^{k_1 \cdots k_a}}{\partial x_{i_\ell}
    \cdots \partial x_{i_r}} \bigg) \\ 
  & = \frac{\partial^{r+1} \varphi^{k_1 \cdots k_a}_f}{\partial
    x_{i_1} \cdots \partial x_{i_r} \partial x_j} dx_j 
\end{align*}
so much so that we obtain
\begin{align*}
  A^j o j^\ell (f) & = \frac{\partial^{r+1} \varphi^{k_1 \cdots
      k_a}_f}{\partial x_{i_1} \cdots \partial x_{i_r}} \\ 
  &= F^{k_ 1\cdots k_a ; i_1 \cdots i_r,  j}_{\varphi} o j^\ell (f)
\end{align*}

Therefore $j^\ell(f)^* (dF^{k_ 1\cdots k_a ; i_1 \cdots i_r
}_{\varphi} - F^{k_ 1\cdots k_a ; i_1 \cdots i_r j }_{\varphi} dx_j) =
0$ for any cross-section $f$. Hence, by the definition of
$\prod(\ell)$ 
$$
dF^{k_ 1\cdots k_a ; i_1 \cdots i_r }_{\varphi}  \equiv F^{k_ 1\cdots
  k_a ; i_1 \cdots i_r }_{\varphi} dx_j \pmod{ \prod (\ell)}. 
$$

\begin{proposition}\label{chap3:sec3.3:prop4}% propos 4
  For\pageoriginale any $\varphi \in \wedge' (D)$ we have the relation 
  $$
  F^{i ;k_ 1\cdots k_\nu j }_{\varphi} - F^{j ;k_ 1\cdots k_\nu i
  }_{\varphi} = F^{ji ;k_ 1\cdots k_\nu  }_{\varphi} (\nu \leq \ell -
  2) 
  $$
\end{proposition}

\begin{proof}
  Let $f$ be a cross-section of the fibred manifold $(D, D',
  \tilde{\omega})$; we can then write 
  $$
  f^* \varphi = \varphi^i_f dx_i \text{ and } f^* (d \varphi) =
  \frac{1}{2} \bigg( \frac{\partial \varphi^i_f} {\partial x_j} -
  \frac{ \partial \varphi^j_f} {\partial x_i}\bigg) dx_j \wedge dx_i 
  $$
\end{proof}

But, on the otherhand we have $f^* (d \varphi) = \dfrac{1}{2} (d
\varphi)_f^{j i} dx_j \wedge dx_i$ and therefore it follows that 
$$
(d \varphi)^{ji}_{f} =\frac{\partial \varphi^i_f} {\partial x_j}-
\frac{\partial \varphi^j_f} {\partial x_i}.  
$$

But 
\begin{align*}
  F^{ji; k_1 \cdots k}_{d \varphi} \left(j^\ell_{\alpha (X)} (f )\right) & =
  \dfrac{\partial^\nu}{\partial x_{k_1} \cdots x_{k_\nu}} \left[(d
    \varphi)^{ji}_f\right]\\
  & = \frac{\partial^{\nu + 1}}{\partial x_{k_1} \cdots \partial
    x_{k_\nu} \partial x_j} \left[\varphi^i_f \right] - \frac{\partial^{\nu
      +1}}{\partial x_{k_1} \cdots \partial x_{k_\nu} \partial x_i}
  \left[\varphi^j_f\right] \\ 
  & = F^{i;k_1 \cdots k_{\nu}j}_{\varphi} -F^{j;k_1 \cdots k_{\nu}i}_{\varphi}.
\end{align*}

\noindent
Hence the required identity.

By the same method as in the proof of Proposition \ref{chap3:sec3.3:prop3}, we prove the
following proposition: 
\begin{proposition}\label{chap3:sec3.3:prop5} % props 5
  If $\varphi$ is a form of degree a, then on $J^1 (D, D',
  \tilde{\omega})$ we have 
  $$
  \varphi \equiv F^{k_1 \cdots k_a}_{\varphi} dx_{k_1} \wedge \cdots
  \wedge dx_{k_a} \left(\mod \prod (1)\right). 
  $$
\end{proposition}

Consider\pageoriginale an exterior differential system $[\sum , (D,
  D'\tilde{\omega})]$ with independent variables. Since $(\sum)$ is
finitely generated as an ideal in $\wedge (D)$ (closed for the
operator $d$ of exterior derivation), we see that $\mathscr{F}^\ell
(\sum)$ is also finitely generated. Let $\prod (\ell)$ be the exterior
differential system constructed on $(J^\ell (D, D' , \tilde{\omega}),
D' , \alpha$). These considerations lead to the following
definition. 

\begin{defi*}%defi 0
  The differential system generated by $\left\{\prod (\ell) , \mathscr{F}^\ell,
  \beta^* \sum\right\}$ on $J^\ell (D, D', \tilde{\omega})$ is called the
       {\em standard prolongation} of $(\sum)$ to the space of $\ell
       -$ jets and is denoted by $P^\ell_S [ \sum ,(D, D' ,
         \tilde{\omega})]$. 
\end{defi*}

The following is a consequence of this definition and the Proposition
\ref{chap3:sec3.2:prop2}. (Remark that $\alpha = \tilde{\omega} \circ \beta$). 

\begin{proposition}\label{chap3:sec3.3:prop6} % props 6
  For an integral $f$ of the system $[\sum, (D, D' , \tilde{\omega})]$
  the cross-section $j^\ell(r)$ of $(J^\ell (D, D' , \tilde{\omega}),
  D', \alpha)$ is an integral of $P^\ell_S [\sum; (D, D' ,
    \tilde{\omega})]$. Conversely for any integral $F$ of $P^\ell_S
  [\sum ; (D, D' , \tilde{\omega})]$ there exists a unique cross-
  section  $f$ of $(D, D' , \tilde{\omega})$ such that $F =
  j^\ell(f)$; when this is so $f$ is an integral of $(\sum)$. 
\end{proposition}

\section{Admissible Restriction}\label{chap3:sec3.4} % sec 3.4

Let $D, D_1$ be two domains in Euclidean spaces such that $D_1
\subseteq D$ and $(\sum)$ be a differential system on $D$. 

\begin{defi*}%defi 0
  A differential system $(\sum_1)$ on $D_1$ is said to be an
  admissible restriction of $(\sum)$ to $D_1$ when 
\end{defi*}

\begin{enumerate}[(i)]
\item $(\sum_1)$ is generated by $i^* (\sum), i$ being the injection
  map of $D_1$ into $D$ 
\item there\pageoriginale exist functions $f_1, \ldots , f_a$ in $\sum^{(0)}$ such
  that $df_1, \ldots , df_a$ are linearly independent at each point of
  $D$ and $D_1$ is the set of common zeros of $f_1, \ldots , f_a$. 
\end{enumerate}

\begin{proposition}\label{chap3:sec3.4:prop7} %props 7
  If $(D_1, \sum_1)$ is an admissible restriction of $(\sum)$ on $D$
  to $D_1$ and $(D_2, \sum_2)$ is an admissible restriction of
  $(\sum_1)$ on $D_1$ to $D_2$, then $(D_2, \sum_2)$ is an admissible
  restriction of $(\sum)$ to $D_2$. 
\end{proposition}

This follows immediately from the above definition.

\begin{remarks*}%rem 0
\begin{enumerate}[\rm (1)]
\item The condition $(ii)$ of the above definition implies that
  $\ell^0 \sum$ is subset of $D_1$. 

  Suppose we denote by $di_q$ the injective mapping of
  $\mathscr{G}^q(D_1)$ into $\mathscr{G}^q(D)$ induced by the
  injection $i$ of $D_1$ into $D, di_q$ defines an isomorphism of
  $\ell^q \sum_1$ onto $\ell^q \sum_1$ onto $\ell^q \sum$. 
\item Any integral of $(\sum)$ is contained in $D_1$. A submanifold of
  $D_1$ is an integral of $(\sum_1)$ if only if it is an integral of
  $(\sum)$. 
\end{enumerate}
\end{remarks*}

\begin{proposition}\label{chap3:sec3.4:prop8} % props 8
  An integral element $E$ of $(\sum_1)$ is an ordinary (\resp \break  regular)
  with respect to $(\sum_1)$ if and only if $di_q(E)$ is ordinary
  (\resp  \break regular) with respect to $(\sum)$. 
\end{proposition}

\begin{proof}
  The proof is by induction on the dimension $q$ of $E$. The
  proposition is trivial in the case $q = 0$ because of the definition
  of an ordinary (\resp  regular) integral point. Let us suppose that
  the proposition holds for all $q' < q$. If $E \in \theta^q \sum_1$ then
  $E$ contains a $(q - 1)$ dimensional regular integral element $E'$
  of $(\sum_1)$. By induction assumption $E' \in \mathscr{R}^{q - 1}
  \sum_1$ if and only if\pageoriginale $di_q (E') \in \mathscr{R}^{q -1}
  \sum$. Hence $E \in \theta ^q \sum_1$ if and only if $di_q (E) \in
  \theta^q \sum$. Since $\ell^q \sum \subseteq \mathscr{G}^q D_1
  \subseteq \mathscr{G}^q D, i^* (J(E; \sum)) = J(i_q E; \sum)$, and
  $J(E; \sum) \ni (d f_1)_z, \ldots , (d f_a)_z$, where $z$ is the
  origin of $E$ and $f_1, \ldots , f_a \in \sum^o$ such that $D_1$ is
  defined by $f_1 = \cdots = f_a = 0$, it follows by the definition of
  regular elements that $E$ is regular if and only if $i_q E$ is
  regular. 
\end{proof}

\setcounter{propdash}{7}
\begin{propdash}\label{chap3:sec3.4:propdash8}% props 8' 
  Let $(D, D', \tilde{\omega})$ be a fibered manifold. Let $D_1$ be a
  submanifold of $D$ such that $(D, D' , \tilde{\omega})$ is a fibered
  manifold. Assume that $\sum$ is a differential system on $D$ such
  that its restriction to $D_1$ is an admissible restriction. Denote
  by $i'$ the canonical injection of $J^\ell (D_1, D' , \tilde{\omega})$ into
  $J^\ell (D, D' , \tilde{\omega})$, where $p = \dim D'$. Then $i'$
  induces an isomorphism of $P^\ell_S [\sum_1, (D, D' ,
    \tilde{\omega})]$ onto an admissible restriction of $P^\ell_S
  [\sum , (D_1, D' , \tilde{\omega})]$. 
\end{propdash}

\begin{proof}
  We can take a coordinate system $(x, y_1, \ldots , y_m)$ of $(D, D'
  , \tilde{\omega})$ such that $y_1 , \ldots ,y_s \in \sum{(0)}$ and
  $D_1$ is defined by the equation: $y_1 = \cdots = y_s = 0$. Then our
  assertion follows easily by direct verification. 
\end{proof}

Let $(D, D', \tilde{\omega})$ be a fibred manifold then we can define
a mapping of $J^{\ell + m} (D, D' , \tilde{\omega})$ into $J^m (J^\ell
(D, D' , \tilde{\omega}), D', \alpha)$ by associating to each jet $X =
j^{\ell + m}(f)$ in $J^{\ell +m}(D, D' , \tilde{\omega})$ the jet
$j^m_{z'} (j^\ell(f))$. Since $j^{\ell + m}(f)$ is a cross-section of
$(J^\ell (D, D' , \tilde{\omega}), D' , \alpha)$ the following
diagram is commutative. 
$$
J^{\ell+m} (D, D' , \tilde{\omega}) \rightarrow J^m(J^\ell(D, D' ,
\tilde{\omega}), D', \alpha) 
$$

\[
\xymatrix{J^{\ell+m} (d, D', \tilde{w}) \ar[rr] \ar[dr] & & J^{m} (J^{\ell} (D, D',
    \tilde{w}), D', \alpha)\ar[dl]\\
& D'& }
\]
\[
\xymatrix{j^{\ell+m}_{z'}(f) \ar[rr] \ar[dr] & & j^m_{z'} (j^{\ell}(f) )
    \ar[dl]\\
& z'&}
\]\pageoriginale

Then we claim that the standard prolongation $\prod (\ell + m; (D, D'
, \tilde{\omega})$) is an admissible restriction $p^m_S [\prod (\ell ;
  (D, D' , \tilde{\omega})) ; J^\ell (D, D' , \tilde{\omega})]$. For
simplicity, we set $\prod (\ell) = \prod (\ell ; (D, D' ,
\tilde{\omega})), \prod (\ell + m) = \prod (\ell+m; (D, D' ,
\tilde{\omega}))$, $J^r = J^r (D, D' , \tilde{\omega})$. 

Let $(x, y)$ be a fixed coordinate system of $(D, D' ,
\tilde{\omega})$. Then $(x, y, \ldots$, $y^{i_1 \cdots i_a}_{\lambda} ,
\cdots)\, a \leq \ell$, is a coordinate system in $J^\ell (D, D' ,
\tilde{\omega})$ and $(x, y, \ldots$, $y^{i_1 \cdots i_b})~ b \leq \ell
+ m$, is a coordinate system in $J^{\ell + m} (D, D' ,
\tilde{\omega})$. Let $w _\sigma$ denote $y^{i_1 \cdots i_a}_{\lambda}
(a \leq \ell)$; then a coordinate system in $J^m (J^\ell(D, D' ,
\tilde{\omega}), D', \alpha)$ will be $(x, y, \ldots , w_\sigma ,
\ldots w^{j_1 \cdots j_c}_\sigma , \ldots) (c \leq m)$. We shall write
$y^{i_1 \cdots i_a ; j_1 \cdots j_c}_{\lambda}$ instead of $w^{j_1
  \cdots j_c}_{\sigma}$. Then the canonical injection mapping i is
defined by: $y^{i_1 \cdots i_a ; j_1 \cdots j_c}_{\lambda} oi =
y^{i_1 \cdots i_a ; j_1 \cdots j_c}_{\lambda}$ Since $P^m_S [\prod
  (\ell), (J^\ell , D', \alpha)]$ is generated by $\prod (m;
(J^\ell , D' , \alpha ))$, which we shall denote by $\widetilde{\prod (m)},
\beta^* \prod (\ell)$ and $\mathscr{F}^m \break (\prod (\ell))$, we shall
compute each of these. 

$\prod (\ell)$\pageoriginale is generated by $\{ dy_{\lambda } - y^i_{\lambda} dx_i
, dy^{i_1 \cdots i_a}_{\lambda}- y^{i_1 \cdots i_a i}_{\lambda} dx_i
~(a \leq \ell - 1)$ and $\widetilde{\prod_i (m)}$ is generated by $d y^{i_1 \cdots
  i_h}_{\lambda} - y^{i_1 \cdots i_h:i} dx_i ; (h \leq \ell-1), dy^{i_1 \cdots i_h ; j_1
  \cdots j_c}_\lambda - y^{i_1 \cdots i_h; j_1 \cdots j_c, j}_{\lambda} dx_j$
 $(h \leq \ell, c \leq m-1)$. 

$\beta^* \prod (\ell)$ is generated by
$$
\omega^{i_1 \cdots i_a}_{\lambda} = dy^{i_1 \cdots i_a}_{\lambda} -
y^{i_1 \cdots i_a i}_{\lambda} dx_i ~(0 \leq a \leq \ell - 1) 
$$

We shall now calculate $\mathscr{F}^m (\prod (\ell))$. Take a jet $X
\in J^m (J^\ell , D' \alpha)$, say $X = j^m_x (g)$ where $g$ is
represented by $(x, \ldots , g_{\lambda}(x), \ldots , g^{i_1 \cdots
  i_a}_{\lambda} (x), \ldots )(a \leq \ell)$. We have 
\begin{align*}
  g^* (d \omega^{i_1 \cdots i_a}_{\lambda}) & = - dg^{i_1 \cdots i_a k}_{\lambda} dx_k \\
  & = - \frac{1}{2} \bigg( \frac{\partial g^{i_1 \cdots i_a
      k}_{\lambda}}{\partial x_j} - \frac{\partial g^{i_1 \cdots i_a
      j}}{\partial x_k} \bigg) dx_j \wedge dx_k 
\end{align*}

Therefore we obtain $F_{\omega_\lambda^{i_1 \cdots i_a}}^{i; 
  j_1 \cdots j_c} = y^{i_1 \cdots i_a ; ij_1 \cdots j_c}_{\lambda} -
y^{i_1 \cdots i_a  i;  j_1 \cdots j_c}_{\lambda}$, 
\begin{align*}
  2 F^{jk}_{d \omega} i_1 \cdots i_a (X) & = - y^{i_1 \cdots i_a k;j
  }_{\lambda} + y^{i_1 \cdots i_a j;k }_{\lambda}, \\ 
  2 F^{jk; j_1 \cdots j_c}_{d \omega \lambda} i_1 \cdots i_a (X) & = -
  y^{i_1 \cdots i_a k;j  j_1 \cdots j_c}_{\lambda} + y^{i_1 \cdots i_a
    j;k j_1 \cdots j_c }_{\lambda} 
\end{align*}

Now it can be verified that $\prod (\ell + m)$ is an admissible
restriction of $p^m_S (\prod (\ell); (J^\ell , D', \alpha))$ to the
submanifold defined by the equations  
\begin{gather*}
  y^{i_1 \cdots i_a j;j, j_1 \cdots j_b }_{\lambda} - y^{i_1 \cdots i_a
    j;i, j_1 \cdots j_b}_{\lambda} = 0\\ 
  y^{i_1 \cdots i_a ;i j_1 \cdots j_c }_{\lambda} - y^{i_1 \cdots i_a
    i; j_1 \cdots j_c }_{\lambda} = 0 
\end{gather*}

Similarly\pageoriginale it can be proved that, given a differential system $[\sum ,
  (D$, $D', \tilde{\omega})] ,P^{\ell +m}_S [\sum , (D, D',
  \tilde{\omega})]$ is an admissible restriction of $P^m_S(P^\ell_S
(\sum))$. 

\section{}\label{chap3:sec3.5} %sec 3.5

We consider certain special types of exterior differential systems and
their prolongation. 

\begin{defi*}%defi 0
  An exterior differential system $[\sum , (D, D', \tilde{\omega})]$
  with independent variables is called a {\em normal differential
    system } if it satisfies the following conditions. 
\end{defi*}

\begin{enumerate}[(1)]
\item $\sum ^{(0)} = 0$; 
\item there exist Pfaffian forms $\theta_{1, \ldots ,} \theta_a$ on
  $D$ which form a system of generators of $\sum^{(1)}$ and which are
  such that $\theta_{1, \ldots ,} \theta_a , dx_{1, \ldots ,} dx_p$
  are linearly independent at each point of $D$, where $(x_{1}, \ldots
  , x_p)$ is a coordinate system in $D'$; 
\item $\sum^{(2)} \equiv 0 \mod (\sum^{(1)}, dx_1 , \ldots , dx_p)$;
\item $\sum$ is generated as an ideal, without the operator $d$, by
  $\sum^{(1)}$ and $\sum^{(2)}$ over $\wedge^0 (D)$. 
\end{enumerate}

Let $(x, y) = (x_1, \ldots , x_p , y_1 , \ldots , y_m)$ be a
coordinate system of $(D, D'$, $\tilde{\omega})$. By a linear change of
coordinates $y$ and by restricting to a neighbourhood of a given point
if necessary, we can assume that $dx_1, \ldots , dx_p, dy_1$, $\ldots ,
dy_m , \theta_1, \ldots , \theta_a (m' = m-a)$ are linearly
independent at each point of $D$. Therefore we can write	 
$$
dy_{m' + b} = c^{b'}_b \theta_{b'} + A^{\lambda}_b y_{\lambda} + B^i_b
dx_i (1 \leq i \leq p, 1 \leq \lambda \leq m', 1 \leq b, b' \leq a). 
$$

Then\pageoriginale the determinant of the matrix $(C_b^{b'})$ is non zero. Hence
$C_b^{b'} \theta_{b'}$ $\in \sum ^{(1)}$ and generate $\sum^{(1)}$. Therefore
we can assume without loss of generality  that 
$$
\theta_b = dy_{m'+b} - A_b^{\lambda}dy_{\lambda} - B^i_b dx_i.
$$

Now we calculate $F_{\theta_{b}}^{k;i_1, \ldots i_r} = F_b^{k;i, \ldots i_r}$. Because 
$$
\theta_b \equiv (y^i_{m'+b} - A^{\lambda}_b y^i_{\lambda} - B^i_b ) dx_i \mod
(dy_{\sigma} - y^j_{\sigma} dx_j ) , (\sigma = 1 , \ldots , m) 
$$

Proposition \ref{chap3:sec3.3:prop5} implies that
$$
F^k _b = y^k_{m' +b}- A ^{\lambda}_b y_{\lambda}^k - B^k_b.
$$

Hence there are functions $E^{k;i}_{b}$ on $J' (D, D',
\tilde{\omega})$ such that 
$$
dF^k _b \equiv \left(y^{ki}_{m'+b}- A ^{\lambda}_b y_{\lambda}^{ki} -
B_b^{k;i}\right) dx_i \pmod {\prod (2)} 
$$

Therefore Proposition \ref{chap3:sec3.3:prop3} implies that
$$
F^{k;i}_b = y^{ki}_{m'+b}- A ^{\lambda}_b y_{\lambda}^{ki} - B_b^{k;i} 
$$

By the repetition of the same argument , we find that, for $r \leq \ell -1$
$$
F^{k; i_1 \cdots i_r}_b= y^{k i_1 \cdots i_r}_{m'+b} - A^{\lambda}_b
y_{\lambda}^{ki_1 \cdots i_r}-B^{k; i_1 \cdots i_r}_b 
$$
where $B^{k; i_1 \cdots i_r}_b$ are functions on $J^r$. Since $F^{k;
  i_1 \cdots i_r}_b$ is symmetric with respect to $i_1 \cdots i_r$, so
is $B^{k; i_1 \cdots i_r}_b$. Take $\varphi$ in $\sum^{(2)}$. By the
condition $(3)$ of the normal systems, 
$$
\varphi \equiv \varphi' = A^{ i\lambda}_{\varphi} dx_i \wedge
dy_{\lambda} + \frac{1}{2} B^{ij}_{\varphi} dx_i \wedge dx_j \pmod
{\Sigma^{(1)}}
$$
where\pageoriginale $\lambda = 1, \ldots, m'$ and $B^{ij}_\varphi + B^{ji}_{\varphi}
= 0$. Then $\varphi' \in \sum^{(2)}$ and  
$$
F^{kj; i_1 \cdots i_r}_{\varphi} \equiv F^{kj; i_1 \cdots
  i_r}_{\varphi} \mod (F^{h; j_1 \cdots j_\nu}_b , \nu \leq r) 
$$

We find, by the same argument as we used to calculate $F^{k; i_1
  \cdots i_r}_b$, that 
$$
F^{kj; i_1 \cdots i_r}_{\varphi'} = A^{k \lambda}_{\varphi} y^{ji_1
  \cdots i_r}_ \lambda - A^{j \lambda}_{\varphi} y^{ki_1 \cdots
  i_r}_{\lambda}+ B^{kj; i_1 \cdots i_r}_{\varphi} 
$$
where $B^{kj; i_1 \cdots i_r}_{\varphi}$ are functions on $J^r$ and
symmetric with respect to $i_1 \cdots i_r$. We set  
$$
'F^{kj; i_1 \cdots i_r}_{\varphi} = F^{kj; i_1 \cdots i_r}_{\varphi'}.
$$

Now, $P^{\ell}_S (\sum)$ is generated by $\{ F^{k_1 \cdots k_a ; j_1
  \cdots j_r}_{\psi} (\psi \in \sum^{(a)} , r \leq \ell -1), \prod
(\ell)$, $(\sum) \}$ and their exterior derivatives. By the remark
$(4)$ on the functions $F^{k_1 \cdots k_a ; i_1 \cdots i_r}_{\psi}$ we
can restrict $\psi$ to a system of generators of the ideal
$(\sum)$. By Proposition \ref{chap3:sec3.3:prop5}, we can omit $(\sum)$ for $\ell \geq
1$. Therefore, the condition (4) of the definition of normal system
shows that $p^{\ell}_S (\sum)$ is generated by  
$$
\begin{cases}
  F^{k;i_1 \cdots i_r}_{b} &(r \leq \ell -1, b=1 , \ldots, a),\\
  F'^{kj;i_1 \cdots i_r}_{\varphi} &(r \leq \ell -1 ; \varphi \in \sum^{(2)}),\\
  \prod(\ell)
\end{cases}
$$
and their exterior derivatives. By Proposition \ref{chap3:sec3.3:prop4} we have
$$
F^{k;ji_1 \cdots i_{r-1}}_{b} - F^{j;ki_1 \cdots i_{r-1}}_{b} =
F^{jk;i_1 \cdots i_{r-1}}_{d \theta _{b}} 
$$

Since\pageoriginale $F^{k;i_1 \cdots i_r}_{b}$ is symmetric with
respect to $i_1\ldots i_r$ and since $d \theta_b \in \sum^{(2)}$, we
have proved the following 

\begin{proposition}\label{chap3:sec3.5:prop9}%prop 9
  When $[ \sum , (D, D', \tilde{\omega})]$ is a normal exterior
  differential system, $P^{\ell}_S \sum$ is generated by  
  $$
  \begin{cases}
    F^{k;i_1 \cdots i_r}_{b} &(1 \leq k \leq i_1 \leq \cdots \leq i_r
    \leq p, 0\leq r \leq \ell -1 , 1 \leq b \leq a),\\ 
    'F^{k;i_1 \cdots i_r}_{\varphi} &(1 \leq k, i_1 \ldots , i_r \leq
    p,0 \leq r \leq \ell -1, \varphi \in \sum^{(2)}),\\ 
    \prod(\ell)
  \end{cases}
  $$
  and their exterior derivatives.
\end{proposition}

\section{}\label{chap3:sec3.6} %sec 3.6

Let $[\sum, (D,D', \tilde{\omega})]$ be an exterior differential
system with independent variables. Let it be normal. If $X$ in $J
(D,D', \tilde{\omega})$ is an integral  point of the standard
prolongation $P^{\ell}_S (\sum)$, let $J(X)$ be the space of polar
forms of $X$ with respect to $P^{\ell}_S (\sum)$. By definition $J(X)$
is the linear subspace of the dual of $(J^{\ell})_X$, the tangent
vector space of $J^{\ell}(D,D', \tilde{\omega})$ at $X$, generated by
$\{(\psi)_X : \psi \in (P^{\ell}_S \sum_k)^{(1)} \}$. This is
equivalent to say that $J(X)$ is generated by $\{(dF^{k;i_1 \cdots
  i_r}_B)_X, 0 \leq b \leq a, (dF _{\varphi}^{kj; i_1 \cdots i_r})_X,
(\prod^{(1)} (\ell))_X (r \leq \ell -1)\}$. But by Proposition
\ref{chap3:sec3.3:prop3} we
have for $r \leq \ell -2$ 
$$
dF^{k_1 \cdots k_a ; i_1 \cdots i_r}_{\psi} \equiv F^{k_1 \cdots k_a ;
  i_1 \cdots i_r i}_{\psi} dx_i \pmod {\prod (\ell)}. 
$$
$X$ being an integral point, $(F^{k;i_1  \cdots i_r}_b)_X= 0$ and so
$(dF^{k;i_1  \cdots i_r}_b)_X \equiv$\break  $\pmod {\prod (\ell)}$ for\pageoriginale $r \leq
\ell -2$. Also, for any  
\begin{align*}
  \varphi \in {\textstyle\sum^{(2)}}, (dF^{kj;i_1  \cdots i_r}_\varphi)_X & \equiv
  (F^{kj;i_1  \cdots i_r i}_\varphi )_X dx_i\\ 
  & \equiv 0 \pmod {\prod (\ell)}.
\end{align*}

From this we may conclude that $J(X)$ has for generators the set $\Big\{
(dF^{k;i_1  \cdots i_{\ell -1 }}_b)_X (k \leq i_1 \leq \cdots \leq i_{\ell
  -1} \leq p)$ and $\left(d'F^{kj;i_1  \cdots i_{\ell -1}}\right)_X$ $(1 \leq k ,
j, i_1 \cdots i_{\ell -1}$ $\leq p), (\prod ^{(1)} (\ell))_X\Big\}$. 

\begin{proposition}\label{chap3:sec3.6:prop10}%pro 10
  For any $\theta_b (1 \leq b \leq a)$. We have
  $$
  dF^{k;i_1  \cdots i_{\ell-1}}_b \equiv \left\{ (dy)^{ki_1 \cdots i_{\ell
      -1}}_{m'+b}- A^{\lambda}_b y_{\lambda}^{ki_1 \cdots i_{\ell-1}}
  + B_b ^{k;i_1 \cdots i_{\ell-1} i} \right\} dx_i \pmod {\prod (\ell)} 
  $$
  and for any $\varphi \in \sum ^{(2)}$
  $$
  d'F^{kj;i_1  \cdots i_{\ell-1}}_\varphi \equiv \bigg (A^{k \lambda
    ji_1 \cdots i_{\ell -1}}_{\varphi y _{\lambda}} - A^{j \lambda
    ki_i \cdots i_{\ell -1}}_{\varphi y _{\lambda}} + B^{kj; i_1
    \cdots i_{\ell-1}}_{\varphi}  \bigg ) dx_i \pmod {\prod (\ell)}. 
  $$
\end{proposition}

\begin{proof}
  $J^{\ell +1}$ can be considered as a fibre space over $J^{\ell}$
  with the natural projection $\rho^{\ell +1}_{\ell}$. Hence, by
  lifting, $dF_{b}^{k ; i_1 \cdots i_{\ell -1}}$ can be considered as
  a pfaffian form on $J^{\ell +1} (D, D', \tilde{\omega})$. Then we
  can write  
  \begin{align*}
    dF_{b}^{k;i_1 \cdots i_{\ell -1}} & \equiv F_{b}^{k;i_1 \cdots
      i_{\ell -1}i} dx_i \pmod {\prod (\ell +1)}\\ 
    & = \left\{y_{m'+b}^{ki_1 \cdots i_{\ell -1}i}-A^{\lambda ki_1 \cdots
      i_{\ell-1}i}_{b y \lambda} + B_{b}^{k;i_1 \cdots i_{\ell -1}i}
    \right\} dx_i \pmod {\prod (\ell+1)}\\ 
    & = dy_{m'+b}^{ki_1 \cdots i_{\ell -1}} - A^{\lambda}_{b}
    dy_{\lambda}^{ki_1 \cdots i_{\ell -1}} + B_{b}^{k;i_1 \cdots
      i_{\ell -1}i} dx_i \pmod {\prod (\ell +1)} 
  \end{align*}
  because\pageoriginale $dy_{\sigma}^{ki_1 \cdots i_{\ell -1}} - y_{\sigma}^{ki_1
    \cdots i_{\ell -1}i} dx_i \in \prod (\ell +1)$. Thus if $\Omega$
  denotes  
  $$
  dy_{m'+b}^{ki_1 \cdots i_{\ell -1}} -A^{\lambda}_{b}
  dy_{\lambda}^{ki_1 \cdots i_{\ell -1}} + B_{b}^{k;i_1 \cdots i_{\ell
      -1}i} dx_i, 
  $$
  then $dF_{b}^{k;i_1 \cdots i_{\ell -1}} - \Omega$ is a form on
  $J^{\ell}$ and is in $\prod (\ell +1)$. Therefore $J^{\ell} (f)^*
  (dF_{b}^{k;i_1 \cdots i_{\ell -1}}- \Omega) =0$ for any
  cross-section $f$ of $(D,D', \tilde{\omega})$. As proved before, it
  follows that $dF_{b}^{k;i_1 \cdots i_{\ell -1}} - \Omega \in \prod
  (\ell)$ and this completes the proof of the first assertion. The
  second assertion can also be proved on the same lines. 
\end{proof}

Let us denote by $G^{(\ell)}$ the subspace of Pfaffian forms on
$J^{\ell}$ generated by 
\begin{multline*}
  \eta_{b}^{k;i_1 \cdots i_{\ell -1}} = dy_{m'+b}^{ki_1 \cdots i_{\ell
      -1}} - A^{\lambda}_b dy_{\lambda}^{ki_1 \cdots i_{\ell -1}} +
  B_{b}^{k;i_1 \cdots i_{\ell -1}i} dx_i\\
  (1 \leq k \leq i_1 \leq\cdots \leq i_{\ell-1} \leq p , b=1 ,
  \ldots,a )
\end{multline*}
and by $A_o^{(\ell)}$ the subspace generated by
\begin{multline*}
  \xi_{\varphi}^{kj;i_1 \cdots i_{\ell -1}} = A^{k
    \lambda}_{\varphi} dy_{\lambda}^{ji_1 \cdots i_{\ell -1}} - A^{j
    \lambda}_{\varphi} y_{\lambda}^{ki_1 \cdots i_{\ell -1}} +
  B_{\varphi}^{kj;i_1 \cdots i_{\ell -1}i} dx_i\\ 
  \left(1 \leq k,j,i_1, \ldots, i_{\ell -1} \leq p , \varphi \in
  {\textstyle \sum^{(2)}}\right) 
\end{multline*}

By the remark preceding Proposition \ref{chap3:sec3.6:prop10} by
Proposition \ref{chap3:sec3.6:prop10} it 
follows that $J(X) = (G^{(\ell)})_X + (\prod (\ell))_X +
(A_o^{(\ell)})_X$. On the otherhand $dy_{\sigma}^{i_1 \cdots i_{r}}-
y_{\sigma}^{i_1 \cdots i_{r} i} dx_i (1 \leq i_1 \leq \cdots \leq i_r
\leq p.r \leq \ell -1 , \sigma =1 , \ldots m)$,  which\pageoriginale form a system
of generators of $\prod (\ell)$, together with the generators of
$G^{(\ell)}$ are linearly independent modulo $dx_1 , \ldots, dx_p,
\ldots, dy_{\lambda}^{i_1 \cdots i_{\ell}}, \ldots (\lambda =1, \ldots
, m')$. Moreover $\xi_{\varphi}^{kj;i_1 \cdots i_{\ell-1}}$ are linear
combinations of $dx_1, \ldots , dx_p, dy_{\lambda}^{i_1 \cdots i_{\ell
}}$.\break  Therefore $J(X) = (G^{(\ell)})_X + (\prod (\ell))_X +
(A_o^{(\ell)})_X$ is a direct sum decomposition of the vector space
$J(X)$. Moreover it is clear that $\dim (G^{(\ell)})_X= aC^{p+\ell
  -1}_{\ell}$ and $\dim (\prod (\ell))_X = m 
\sum\limits_{r=0}^{\ell} C^{p+r-1}_{r}$, where $C^{p+r-1}_{r}$ are the
Binomial coefficients. 

Now, we shall show a similar decomposition of the space of polar
forms $J(E)$ of a $q$-dimensional integral element $E$ of the standard
prolongation $P^{\ell}_S \sum$. Let $X$ be the origin of $E$. Let us
first recall the definition of $J(E)$. Take any system of generators
$\psi_1, \ldots, \psi_n$ of the ideal $P^{\ell}_S \sum$, where
$\psi_\tau$ is homogeneous of degree $a_\tau$. Let $L^1 , \ldots, L^q$ be
a base of the vector space $E$. Then $J(E)$ is generated by
$f_{\tau}^{h_1 \cdots h_a} \tau^{-1} (1 \leq h_1, \ldots,
h_{a_{\tau}-1} \leq q , \tau = 1, \ldots , N)$ defined by
$$
f_{\tau}^{h_1 \cdots h_a} \tau^{-1} (L) = \langle \psi_{\tau}, L^{h_1}
\wedge \cdots \wedge L^{h_a} \tau^{-1} \wedge L \rangle. 
$$  

Since $P^\ell_S \sum$ is generated by $(P^\ell_S \sum)^{(1)}$ and
$\prod^2 (\ell), J(E)$ is generated by $J(X)$ together with all the
$f$ defined by  
$$
f(L) = \langle dy_{\sigma}^{i_1 \cdots i_r i} \wedge dx_i , L' \wedge
L \rangle, (\sigma = 1 , \ldots, m; 1 \leq i_1, \ldots, i_r \leq p, L'
\in E). 
$$ 

If $r \leq \ell -2$, then $dy_{\sigma}^{i_1 \cdots i_r i} \wedge dx_i
\equiv y_{\sigma}^{i_1 \cdots i_r i j} dx_j \wedge dx_i =0 \pmod{\prod
(\ell)}$, because $y_{\sigma}^{i_1 \cdots i_r ij} = y_{\sigma}^{i_1
  \cdots i_r ji}$. Thus $J(E)$ is generated\pageoriginale by $J(X)$ together with 
$$
\zeta_{\sigma}^{i_1 \cdots i_{\ell -1},L} = \langle dx_i, L \rangle
dy^{i_1\cdots i_{\ell -1} i} - \langle dy^{i_1 \cdots i_{-1}i}, L
\rangle dx_i  
$$
where $L \in E$.

Let $\mathscr{G}^q J^{\ell} (dx_1, \ldots dx_q)$ denote the subspace
of $\mathscr{G}^q J^{\ell} (D, D', \tilde{\omega})$ consisting of all
the elements $E$ such that the restrictions $dx_1 |E, \ldots , dx_q|
E$ are linearly independent.  
$\mathscr{G}^q J^{\ell} (dx_1, \ldots, dx_q)$ is an open submanifold
of $\mathscr{G}^q J^{\ell} (D, D', \tilde{\omega})$. Let $L^1 (E) ,
\ldots , L^q (E)$ be a dual base in $E$ of $dx |E, \ldots$,
$dx_q|E$. We introduce the following functions on $\mathscr{G}^q
J^{\ell} (dx_1, \ldots, dx_q)$: for any $E \in \mathscr{G}^q J^{\ell}
(dx_1, \ldots, dx_q)$ let $w_{i.q}^{q'}(E) = \langle dx_i , L^{q'} (E)
\rangle$, 
$$
w_{\sigma~ q}^{i_1 \cdots i_{\ell}; q'} (E) = \langle dy_{\sigma}^{i_1
  \cdots i_\ell}, L^{q'} (E) \rangle. 
$$

Now  if the integral element $E$ is in $\mathscr{G}^q J^{\ell} (dx_1,
\ldots , dx_q)$ the above argument proves that $J(E)$ is generated by
$J(X)$ together with  
\begin{gather*}
  \zeta_{\sigma, q}^{i_1 \cdots i_{-1}; q'} = \zeta_{\sigma, q}^{i_1
    \cdots i_{-1};  L^{q'}(E)}= w^{q'}_{i,q} (E) dy_{\sigma}^{i_1 \cdots
    i_{\ell -1} i; q'} dx_{i}\\ 
  (\sigma = 1,\ldots , m ; q'=1, \ldots , q)
\end{gather*} 

We shall introduce the following notation to facilitate the writing of
the above identities. We shall denote by $I_r$ (or by $I$ when there
is no possible confusion) any set of indices $(i_1, \ldots , i_r)$ for
$r=0, 1,\ldots,\ell$. We can now write all the identities above in the\pageoriginale
compact form as follows: 

\begin{gather*}
  F_b^{k;I} = y_{m'+b}^{kI} - A^{\lambda}_b y_{\lambda}^{kI} +
  B_b^{k;I};\\ 
  dF_b^{k;I} \equiv dy_{m'+b}^{kI} - A^{\lambda}_b dy_{\lambda}^{kI} +
  B_b^{k;Ii} dx_i  \pmod {\prod (\ell )} 
\end{gather*}
where $B_b^{k ; I}$ are functions on $J^{r+1} (D,D', \tilde{\omega})$.
\begin{align*}
  'F_\varphi^{k;I} & =  A^{k\lambda}_\varphi y^{j I}_{\lambda} A^j_\lambda y^{k
    I}_{\lambda} + B_\varphi^{kj;I};\\ 
  d'F_\varphi^{k;I} & = A_{\varphi }^{k\lambda} dy^{jI}_\lambda
  -A_{\varphi}^{j \lambda} dy^{kI}+ B_\varphi^{kj;Ii} dx_i \pmod{\prod
  (\ell)} 	 
\end{align*}
where $B_{\varphi}^{kj;I}$ are functions on $J^r (D, D',
\tilde{\omega})$. The generators of $G^{(\ell)}$ are 
\begin{align*}
  \eta_b^{k;I} & = dy_{m'+b}^{kI} - A_b^\lambda dy_{\lambda}^{kI}+ B_b^{k;Ii} dx_i,\\
  (1 \leq b \leq a; I & = (i_1 , \ldots , i_{\ell -1}); 1 \leq k \leq
  i_1 \leq \cdots \leq i_{\ell -1} \leq p) \text{ and } \\ 
  \xi_{\varphi}^{kj,I} & = A_{\varphi}^{k \lambda} dy_{\lambda}^{jI} -
  A_{\varphi}^{j \lambda} dy_{\lambda}^{kI} + B_{\varphi}^{kj;Ii} dx_i 
\end{align*}
$(\varphi \in \sum^{(2)}, I = (i_{1},\cdots , i_{\ell -1}), 1 \leq k,
j,i_1 \cdots \leq p)$ are generators of the subspace $A_0^{(\ell)}$ of
Pfaffian forms. 

If $X$ is an integral point, it is clear that the dimensions of the
spaces $(G^{(\ell)})_X$ and $(\prod (\ell))_X$ do not change when $X$ is
moved in a sufficiently small neighbourhood of $\ell^o p_S^{\ell}
\sum$. The direct sum decomposition $J(X) = (G^{(\ell)})_X + (\prod
(\ell))_X + (A_o^{(\ell)})_X$ shows that\pageoriginale any change in the dimension
of $J(X)$ is due only to the change in the dimension of
$(A_o^{(\ell)})_X$. 

If $E$ is any $q$-dimensional integral element of $p_S ^{\ell} \sum$
and if $X$ is the origin of $E$, it has already been proved that
$J(E)$ is generated by $J(X)$ together with 
$$
\zeta_{\sigma , q}^{I;q'} = w_{i,q}^{q'} dy_{\sigma}^{Ii} - w_{\sigma,
  q}^{Iiq'} dx_i 
$$
on $\mathscr{G}^q J^\ell (dx_1, \ldots, dx_q) , I = (i_1 \cdots
i_{\ell -1})$. 

\begin{remark*}
  For any $q_1$ with $q' \leq q_1 < q$ let $E^{q_1}$ be the subspace
  of $E^q$ spanned by $L^1 (E), \ldots, L^{q_1} (E)$ and let $\eta$ be
  the natural projection of $E^q$ onto $E^{q_1}$. Clearly we have $w_{i,q}^{q'} =
  w_{i,q_1}^{q'} o \eta, w_{\sigma, q}^{i_1 \cdots i_{\ell; q'}} =
  w_{\sigma, q_1}^{i_1 \cdots i_{\ell; q'}} o \eta $ and hence we can
  simply  write $w_i^{q'} w_{\sigma}^{i_1 \cdots i_{\ell ; q^1}}$ in
  place of $w^{q'}_{i,q}$ and $w_{\sigma, q}^{i_1 \cdots i_{\ell; q'}}$
  without any ambiguity . Also, we can write $\zeta_{\sigma}^{I;q'}$
  instated of $\zeta_{\sigma, q}^{I;q'}$. For any $\ell \geq 2$ we
  have 
  \begin{multline*}
    dy_{m'+b}^{i_1 \cdots i_{\ell -1}i} \wedge dx_i \equiv
    A_b^{\lambda} dy_\lambda^{i_1 \cdots i_{\ell -1}i} \wedge dx_i -
    B_b^{i_1; i_2 \cdots i_{\ell -1}ij} dx_i \wedge dx_j\\ 
    \pmod {\left(p_S^\ell \sum\right)^{(1)}, \left(p_S^\ell \sum \right)^{(0)}} 
  \end{multline*}
\end{remark*}

But $B_b^{i_1 \cdots i_{\ell -1}ij} dx_i \wedge dx_j =0$ since $B_b^{
  i_1 \cdots i_{\ell -1}ij}$ are symmetric in $i,j$. Therefore $J(E)$
is generated by $J(X)$ and $\zeta_{\lambda}^{I;q'} (\lambda= 1, \ldots
, m')$. 

Let $A_q^{(\ell)}$ denote the subspace of 1-forms on $\mathscr{G}^q
J^{\ell} (dx_1, \ldots dx_q)$ generated by $A_o^{(\ell)},
\zeta_{\lambda}^{I;1}, \ldots \zeta_{\lambda}^{I;q} (I = (i_1, \ldots
, i_{\ell -1}); \lambda = 1, \ldots , m')$. 

Then\pageoriginale we prove the following:
\begin{proposition}\label{chap3:sec3.6:prop11}%Prop 11
  Let $E$ be a $q$-dimensional integral element of the standard
  prolongation to the space of $\ell$- jets of a normal system $[\sum,
    (D,\break D', \tilde{\omega})]~ (0 \leq q \leq p = \dim D')$. Denote by
  $X$ the origin of $E$. Then we have the direct sum decomposition 
  $$
  J(E) = \left(G^{(\ell)}\right)_X + \left(\prod (\ell)\right)_X +
  \left(A_q^{(\ell)}\right)_E 
  $$
  $L^1 (E) , \ldots L^q (E)$ being a basis of $E$ dual to $dx_1 |E,
  \ldots dx_q | E$ we see that $w_i ^{q'} = \langle dx_i, L^{q'} (E)
  \rangle = \delta ^{q'}_i (i \leq q)$ so that the Pfaffian forms
  $\zeta_{\sigma}^{I;q'}(I = (i_1, \ldots, i_{\ell -1}))$ have the
  reduced form $\zeta_{\sigma}^{I;q'} = dy_{\sigma}^{Iq'} +
  w_{q+u}^{q'} dy_{\sigma}^{Iq +u} - w_{\sigma}^{Ii; q'} dx_i$. Denote
  by $'A_q^{(\ell)}$ the subspace of Pfaffian forms generated by the
  following set:  
  \begin{align*}
    '\xi_{\varphi}^{kj ; I} & = A_{\varphi}^{k \lambda}
    dy_{\lambda}^{jI} - A_{\varphi}^{j \lambda}dy_{\lambda}^{kI},\\ 
    '\zeta_{\sigma}^{I; q'} &= dy_{\sigma}^{Iq'} + w_{q+u}^{q'}dy_{\sigma}^{I q+u}
  \end{align*}
  where $\varphi \in \sum^{(2)}; \lambda , \sigma = 1 ,\ldots , m';
  q'=1, \ldots , q; I
  = (i_1, \ldots , i _{\ell -1})$ with $1 \leq i_1, \ldots , i_{\ell -1}$
  $j,k \leq p$. Let $t'(E)$ denote the dimension of
  $('A^{(\ell)}_q)_E$. If we denote by $\Omega$ the space generated by
  $(dx_1)_X, \ldots (dx_p)_X$, $X$ being the origin of $E$, then the
  definition shows that $t' (E) = \dim ((A_q^{(\ell)})_E + \Omega/
  \Omega)$. Thus $t'(E)$ is defined independent of the choice of the
  coordinate system (because of Proposition \ref{chap3:sec3.6:prop11}).  
\end{proposition}

\setcounter{section}{7}
\section{}\label{chap3:sec3.8}%sec 3.8

In this section we establish an inequality regarding $t'(E)$. If $E$
is a $q$-dimensional integral element of $p_S^{\ell} \sum$ and if $L^1
(E), \ldots, L^q (E)$ is a basic of $E$ dual to $dx_1 |E, \ldots ,
dx_q|E$, denote by $E'$ the subspace\pageoriginale of $E$ spanned by $L^1 (E),
\ldots L^{q-1} (E)$. Suppose $\rho$ is the natural projection of
$J^\ell (D,D', \tilde{\omega})$ onto $J^{\ell -1} (D,D',
\tilde{\omega})$. It is clear that $E'' = d \rho .E'$ is an integral of
$P_S^{\ell -1} \sum$ on $J^{\ell -1} (D,D', \tilde{\omega})$. Then we
have the  

\begin{proposition}\label{chap3:sec3.8:prop12}%prop 12
  If $E$ is a $q$-dimensional integral element of $P_S^\ell \sum$ then
  the following inequality holds: 
  \begin{align*}
    t'(E) \leq & \dim (A_q^{(\ell)} )_E\\
    & \dim (A_{q-1}^{(\ell)} )_{E'} + n_{\ell -1} - t' (E''),
  \end{align*}
  where $n _{\ell -1} = m c^{p+\ell-2}_{\ell -1}$ where $c^r_s$
  denotes the Binomial coefficient. 
\end{proposition} 
 
\begin{proof}
  In general we denote by $I$ or $I_{\chi}$ indices $(i_1, \ldots,
  i_{\ell -1})$. Let $X$  be the origin of $E$. Let $\left(dy_{\lambda_
    \chi}^{I_\chi}\right)_{\rho (X)}$ be the maximum number of linearly
  independent elements $dy_{\lambda}^I$ modulo $(A^{(\ell
    -1)}_{q-1})_{E''}$. We denote by $J$ indices $(j_1, \ldots, j_{\ell
    -2})$. We can write  	 
  $$
  (dy_{\lambda}^{I}) _{\rho (X)} = a^\chi dy_{\lambda_\chi}^{I \chi}
  + b_{kj;J}^\varphi \xi_{\varphi}^{kj;J} + c_{J, q'}^{\mu '}
  \zeta_{\mu}^{J;q'},  
  $$
  $(q' = 1, \ldots, q-1 ; \mu = 1,\ldots, m')$. We claim that, for any
  $s(1 \leq s \leq p)$  
  $$
  (\#) ~(dy_{\lambda}^{Is})_X = a^\chi (dy_{\lambda \chi}^{I \chi s})_X
  + b_{kj;J} ('\xi_{\varphi}^{kj;Js})_{E'} + c_{J';q'}^\mu('\zeta
  _{\mu}^{Js, q'})_{E'} 
  $$
  To see this, we consider a linear mapping of the vector space
  generated by all $(dy_\mu^{I'})_{\rho (X)}$ onto the space generated
  by all $(dy_\mu^{I's})_X$, sending $(dy_\mu^{I'})_{\rho (X)}$ to
  $\left(dy^{I's}_{\mu} \right)_x$. It is clear that this map maps
  $'\xi_\varphi ^{kj;J}$ upon 
  $'\xi_\varphi ^{kj;Js}$. Since $w_{q+u}^{q'} (E) = w_{q+u}^{q'} (
  d\rho,E) $ it follows  that\pageoriginale $'\zeta _\mu^{J ; q'}$ is mapped upon
  $'\zeta _\mu^{Js ; q'}$. Thus we have proved the equality $(\#)$. $E$
  being an integral element of $P_S^\ell \sum$, taking the value of
  $(dy_\lambda^{Is})_X$ at $L^q (E)$ we obtain by $(\#)$ that  
  \begin{multline*}
    \langle (dy_\lambda^{Is})_X , L^q (E) \rangle  = w_{\lambda}^{Is;q} (E)
    = a^\chi w_{\lambda \chi}^{I \chi s;q} (E)\\  
    - b^{\varphi}_{kj;J}
    (B_{\varphi}^{kj; Jsi} (X) \langle dx_i, L^q (E) \rangle ) +
    c_{J;q'}^\mu (w_{\mu}^{Jsi ; q'} \langle dx_i, L^q (E) \rangle ), 
  \end{multline*}
 since $\xi_{\varphi}^{kj;J} = {'\xi}_{\varphi}^{kj ; J} +
 B_{\varphi}^{kj ; J}$ vanishes at $L^q (E)$. Using the relations
 $\langle dx_i$, $L^q (E) \rangle = \delta _i^q$ we can write 
 \begin{multline*}
   w_{\lambda}^{Is ; q}(E) = a^{\chi} w_{\lambda \chi}^{I \chi s ; q}(E)
   - b^{\varphi}_{kj ; J} (B _{\varphi}^{kj ; Jsq} (X) + B
   _{\varphi}^{kj;Js' (q+u)} w^{q}_{q+u} (E))\\ 
   + c^{\mu}_{J;q'}
   \left(w_{\mu}^{Jsq ; q'} + w_{\mu}^{Js'(q+u) ; q'} w^q_{q+u} (E)\right).   
 \end{multline*}
 
 But
 \begin{align*}
   \zeta_{\lambda}^{I;q} &= dy_{\lambda}^{Iq} + w^q_{q+u} (E)
   dy_{\lambda}^{Iq+u} - w_{\lambda}^{Ii ; q}(E) dx_i\\ 
   &= a^\chi \zeta _{\lambda \chi}^{I \chi ; q} + b_{kj;J}^{\varphi}
   (\xi _{\varphi}^{kj; Jq} + w^q _{q+u} (E)
   \xi_{\varphi}^{kj;J,(q+u)})\\ 
   & \hspace{3cm}+ c^\mu_{J; q'} (\zeta _\mu ^{Jq;q'} +
   w_{q+u}^q (E) \zeta _{\mu}^{J'(q+u);q'}) 
 \end{align*}
 where $q' =1 , \ldots, q - 1$. Thus $\zeta_{\lambda}^{I;q}$ is in the
 space generated by $\zeta_{\lambda \chi}^{I \chi ;q}$ and
 $(A_{q-1}^{(\ell)})_E$. Hence we obtain 
 \begin{align*}
   \dim (A^{(\ell)} _{q})_{E} &\leq \dim (A_{q-1}^{(\ell)})_{E'} + \text{the
     number of indices} ~\chi\\ 
   & = \dim (A_{q-1}^{(\ell)})_{E'}+ m.c_{\ell -1}^{p+ \ell -2} - t'(E'').
 \end{align*}
This\pageoriginale completes the proof of the proposition.
\end{proof}

\section{}\label{chap3:sec3.9} %sec 3.9

Let $[\sum, (D, D', \tilde{\omega})]$ be a differential system with
independent variables which is a normal system. Then we pose the
following definition. 

\begin{defi*}
  For any point $z \in D$, a pair $(z, E^{\chi})$ of $z$ and a
  $q$-dimensional contact element $E^\chi$ to $D'$ at $\tilde{\omega}
  (z)$ is called a $q$-dimensional reduced  contact element of $D$ and
  $z$ is called the origin of it. 
\end{defi*}

Let $\chi_{\mathscr{G}} q = \chi_{\mathscr{G}} q (D, D',
\tilde{\omega})$ be the set of all reduced contact elements $(z,
E^\chi) (z \in D)$. $\chi_{\mathscr{G}}$ is a submanifold of $D \times
\mathscr{G}^q D'$. For, let $\rho$ denote the mapping of
$\chi_{\mathscr{G}}$ into $D'$ which assigns to every element $(z,
E^\chi)$ the origin of $E^\chi$. We observe that $\chi_{\mathscr{G}} q
= \{ (z, E^\chi) \in D \times \mathscr{G}^q D' : \tilde{\omega} (z) =
\rho (E^{\chi})\}$. This condition defines the structure  of a real
analytic submanifold of $D \times \mathscr{G}^q D'$. 

To every reduced contact element $(z, E^\chi)$ we associate a certain
homogeneous ideal $A(z, E^\chi)$ in a certain symmetric algebra
$\underbar{R}(V)$  on a module $V$ over $\underbar{R}$. We shall first
construct the symmetric algebra $\underbar{R}(V)$. Consider the tangent
vector space $(D')_{\tilde{\omega} (z)}$ which is a $p$-dimensional
vector space over $\underbar{R}$. Denote this by  $V^p$. Let $V^{m'}$
denote the quotient module $[\wedge^1_z (D) / \tilde{\omega}^*
  (\wedge^1 D')_{\tilde{\omega}(z)} + (\sum^{(1)} )_z ]$ over the ring
$\wedge_z^o (D) = \underbar{R}$, where $\wedge^{1}_z (D)$ is the
conjugate space of $(D)_z$. Let $V$ denote the direct sum $V^p \oplus
V^{m'}$. Now we denote by $\underbar{R}(V)$ the symmetric algebra over
$V$. If we choose a coordinate\pageoriginale system $(x_1, \ldots, x_p , y_1,
\ldots, y_m)$ in $(D,D', \tilde{\omega})$ such that $dy_{\lambda}$ are
linearly independent modulo $(dx_1 , \ldots, dx_p ; \sum^{(1)})$, then
$V^p$ is spanned by $\dfrac{\partial}{\partial x_1}, \ldots,
\dfrac{\partial}{\partial x_p}$ and $V^{m'}$ by $[dy_1], \ldots,
      [dy_{m'}]$ where $[dy_{\mu}]= dy_{\mu} \mod \{
      \tilde{\omega}^* (\wedge^1 (D'))_{\tilde{\omega}(z)} + (\sum^{(1)}
      _{\tilde{\omega}(z)}) \}$ $(\mu = 1, \ldots,m')$. If we set
      $\dfrac{\partial}{\partial x_i} = X^i$ and $[dy_{\mu}] =
      Y_{\mu}$ the elements of the symmetric algebra
      $\underbar{R}(V)$ can be expressed as polynomials in $X^1 ,
      \ldots , X^p, Y_1, \ldots, Y_{m'}$. In view of the fact that
      $\underbar{R}(V)$ depends on $z \in D$, we may, when there is
      any possible ambiguity write $\underbar{R}(z)$. 

      $(\sum)$ being a normal differential system any $\varphi \in \sum
      ^{(2)}$ can be written as  
$$
\varphi \equiv A^{i \lambda} _{\varphi} dx_i \wedge dy_{\lambda} +
\frac{1}{2} B^{ij}_{\varphi} dx_i \wedge dx_j \pmod{\textstyle \sum ^{(1)}}  
$$
where $A^{i \lambda}_{\varphi}, B^{ij}_{\varphi}$ are real analytic
functions on $D$. Then we have that 
$$
('\widetilde{\xi}_{\varphi}^{kj})_z = A_{\varphi}^{k \lambda} (z)
dy_{\lambda}^j - A_{\varphi}^{j \lambda} (z) dy_{\lambda}^k. 
$$

\noindent
We associate to every $\varphi \in \sum ^{(2)}$ an element
$$
\tilde{\xi}_{\varphi}^{kj}=\tilde{\xi}_{\varphi}^{kj} (z) =
A_{\varphi}^{kj} (z) Y_{\lambda} X^j - A_{\varphi}^{j \lambda} (z)
Y_{\lambda} X^k 
$$
of $\underbar{R}(z)$. By a change of coordinate   systems $(x,y) \to
(x', y')$ such that $dy'_{\lambda}$ are linearly independent modulo
$(dx'_1 ,\ldots, dx'_p, \sum^{(1)})$ we have $(dx'_i)_z = a_i^j
(dx_j)_z (dy'_{\lambda})_z \equiv b_{\lambda}^{\mu} (dy_{\mu})_z \mod
((\sum^{(1)})_z (dx_i)_z)$.  

Hence obtain
$$
\displaylines{\hfill 
X^i = a_j^i {'X}^j , Y'_{\lambda}= b_{\lambda}^{\mu} Y_{\mu},\hfill \cr
\text{where} \hfill {'X}^j = (\dfrac{\partial}{\partial x'_j})_z , Y'_{\lambda} =
(dy'_{\lambda}) \mod (({\textstyle \sum^{(1)}})_z, (dx_i)_z).\hfill } 
$$

If\pageoriginale we express $\varphi \in \sum^{(2)}$ in the new coordinate system as 
$$
\varphi \equiv {'A}_{\varphi}^{i \lambda} dx'_i \wedge dy' +
\frac{1}{2} {'B}_{\varphi}^{ij} dx'_i \wedge dx'_j \pmod{\textstyle\sum^{(1)}} 
$$
then we obtain $A_{\varphi}^{i \lambda} (z) = {'A}_{\varphi}^{j \lambda}
a^i_j b_{\mu}^{\lambda}$. In the new coordinate  system the element of
the symmetric algebra associated to $\varphi \in \sum^{(2)}$ is  
$$
{'\tilde{\xi}}_{\varphi}^{kj} = {'A}_\varphi^{k \lambda} Y' _{\lambda'}
{'X}^j - {'A}^{j\lambda}_\varphi Y'_\lambda {'X}^k. 
$$

We shall obtain the relation between the $\tilde{\xi}_{\varphi}^{kj}$
and the $'\tilde{\xi}_{\varphi}^{kj}$. 
\begin{align*}
  \tilde{\xi}^{kj}_\varphi & = A_{\varphi}^{k \lambda}Y_{\lambda} X^{j} -
  A_{\varphi}^{j \lambda} Y_{\lambda} X^k\\ 
  & = {'A} _{\varphi}^{k' \lambda'} a_{k'}^k b_{\lambda'}^{\lambda}
  Y_{\lambda} X^j - {'A}_{\varphi'}^{j'\lambda'} a_{j'}^j
  b_{\lambda'}^{\lambda}Y_{\lambda} X^k\\ 
  & = {'A}_{\varphi}^{k' \lambda'} a_{k'}^k a_{j'}^j Y'_{\lambda'} {'X}^{j'}
  - {'A}_{\varphi}^{j'\lambda'} a_{j'}^j a_{k'}^k Y_{\lambda'} {'X}^{k'}.  
\end{align*}

Therefore $\tilde{\xi}_{\varphi}^{kj} = a_{j'}^j a_{k'}^k
{'\tilde{\xi}}_{\varphi}^{k'j'}$. Therefore the ideal $A(z)$ in $\underbar{R}(v)$
generated $\tilde{\xi}_{\varphi}^{kj} (\varphi \in \sum^{(2)})$ does
not depend on the choice of the coordinate system. 

Consider a $q$-dimensional reduced contact element $(z,E^{\chi})$.

We remark that any $L^\chi$ in $E^{\chi}$ is in $(D')_{\tilde{\omega}
  (z)} = V^p \subseteq \underbar{R}(z)$. Therefore, for any $Y \in V^{m'}
\subseteq \underbar{R}(z)$, the multiplication $L^{\chi}Y \in
\underbar{R}(z)$ is defined. Now  let $A(z,E^\chi)$ be the ideal in
$\underbar{R}(z)$ generated by $A(z)$ and by all $L^{\chi}. Y
(L^{\chi} \in E^{\chi}, y \in V^{m'})$. Let $(x_1, \ldots, x_p)$ be a
coordinate system in $D'$. Set $X^i = (\dfrac{\partial}{\partial
  x_i})_z$. Then $L^{\chi}$ is expressed as 
$$
L^{\chi} = \sum_{j =1}^{p} \langle dx_j, L^{\chi} \rangle X^j.
$$\pageoriginale

\begin{remark*}
  The symmetric algebra $\underbar{R}(z)$ being a polynomial algebra
  in the indeterminates $X^1, \ldots, X^p, Y_1, \ldots,Y_m$ we can
  decompose it into direct sum of submodules of bidegree $(\ell,
  h)$($\ell=$ degree in $X^k$ and $h=$ degree in $Y_{\lambda}$) and
  written $\underbar{R}(z) = \sum \underbar{R}^{(\ell , h)} (z)$. An
  ideal $I$ in $\underbar{R}(z)$ is said to be a homogeneous ideal if $I
  = \sum I ^{(\ell , h)}$ where $I^{(\ell, h)} = I \cap
  \underbar{R}^{(\ell, h)}$ where $\underbar{R}^{(\ell, h)}(z)$ is the
  submodule of all homogeneous polynomials of bidegree $(\ell ,h)$. $A
  (z,E^{\chi})$ is a homogeneous ideal, so we have  $A (z,. E^{\chi}) =
  \sum A^{(\ell , h)} (z,E^{\chi})$.	 
\end{remark*}

\begin{proposition}\label{chap3:sec3.9:prop13}%pro 13
  Let $E$ be a $q$-dimensional integral element of the system
  $P_S^{\ell}[\sum , (D,D',\tilde{\omega})]$ with independent
  variables. If $E^{\chi}$ denotes $d \alpha (E)$ then $t'(E)=$
  dimension of $A^{(\ell, 1)} (\beta (X) , E^{\chi})$ where $X$ is the
  origin of $E$. 
\end{proposition}

\begin{proof}
  Since $\alpha = \tilde{\omega} o \beta , (\beta (X), E^{\chi})$ is a
  reduced contact element. The proposition asserts that $t' (E) = \dim
  A^{(\ell ,1)} (z,E^{\chi}), z = \beta (X)$. Let $X_1, \ldots,\break X_p$
  be a coordinate  system at $\tilde{\omega}(z)$. Then the generators
  of $A^{(\ell,1)} (z,E^{\chi})$ are  
  \begin{multline*}
  \begin{cases}
    \tilde{\xi}^{kj;i_1 \cdots i_{\ell -1}}_\varphi = A_{\varphi}^{k \lambda}
    (z) Y_{\lambda} X^j X^{i_1} \cdots X^{i \ell -1 } - A_{\varphi} ^{j
      \lambda} (z) Y_{\lambda} X^k X^{i_1} \cdots i_{\ell -1} X;\\ 
    L^{\chi} Y_{\lambda} X^{i_{\ell-1}} \cdots X^{i \ell -1} = \sum_{i=1}^{p}
    \langle dx_i , L^{\chi}\rangle Y_{\lambda} X^{i_1} \cdots X^{i_{\ell-1}} (L
    ^{\chi} \in E^{\chi}) 
  \end{cases}\\
  (\varphi \in {\textstyle \sum^{(2)}}, 1 \leq k,j,i_1, \ldots,i_{\ell -1} \leq p,
  \lambda =1, \ldots , m). 
  \end{multline*}
  
  We can assume if necessary, by a change of coordinate system that
  $dx_i |E^{\chi},\ldots , dx_q|E^{\chi}$ are linearly independent. Let
  $L^1 (E^{\chi}),\ldots , L^q (E^{\chi})$ be\pageoriginale a basis of $E^{\chi}$ dual
  to $dx_1 |E^{\chi} , \ldots, dx_q| E^{\chi}$. Then  
  \begin{multline*}
    L^{q'} (E^{\chi}) Y_{\lambda}X^{i_1}\cdots X^{i_{\ell-1}} = Y_{\lambda}
    X^{i_1}\cdots X^{i_{\ell-1}} X^{q'}\\ 
    + \sum_{h=1}^{\ell-q}
    \langle dx_{q+h}, L^{q'} (E^{\chi})\rangle
    Y_{\lambda}X^{i_1}\cdots X^{i \ell-1} X^{q+h}\\ 
    = Y_{\lambda}X^{i_1}\cdots X^{i_{\ell-1}} X^{q'} + \sum_{h=1}^{\ell
      -q} \omega^{q'}_{q+h} (E^{\chi}) Y_{\lambda} X^{i_1}\cdots X^{i_{\ell-1}}
    X^{q+h} 
  \end{multline*}
  On the other hand the generators of $'A_q^{(\ell)}$ are
  \begin{align*}
    {'\xi}_{\varphi'}^{kj;i_1 \cdots i_{\ell -1}} &= A_{\varphi}^{k
      \lambda} dy_{\lambda}^{jii_1 \cdots i_{\ell -1}}  - A_{\varphi}
    ^{j \lambda} dy_{\lambda}^{ki_1 \cdots i_{\ell -1}}\\ 
    {'\zeta}_{\mu'}^{i_{1} \cdots i_{\ell -1} q'} &= dy_{\mu}^{i_1 \cdots
      i_{\ell -1} q'} + w^{q'}_{q+h} dy_{\mu}^{i_1 \cdots i_{\ell -1}
      q+h}\\
    & (\varphi \in {\textstyle \sum^{(2)}}, 1 \leq k , j, i_1 \cdots
    i_{\ell -1} \leq p, \lambda=1, \ldots , m) 
  \end{align*}
  and $t'(E)=\dim ('A_q^{(\ell)})_E$ by definition. 
\end{proof}

Setting $(dy_{\lambda}^{i_1 \cdots i_{\ell}} )_X =
X_{\lambda}^{i_1\cdots j_{\ell}}$, where $X$ is the origin of $E$, we
see that $x_\lambda^{i_1 \cdots i_\ell}= X_\lambda^{j_1\cdots j_\ell}$
for any permutation $j_1, \ldots, j_{\ell}$ of 
$i_1, \cdots , i_{\ell}$. The generators of $('A^{(\ell)}_q)_E$ are,
therefore, 
\begin{gather*}
  '\xi_{\varphi}^{k j;i_1 \cdots i_{\ell -1}} (E) = A^{k
    \lambda}_{\varphi} (z) X_{\lambda}^{ ji_1 \cdots i_{\ell -1}} -
  A_{\lambda}^{j \lambda} (z) X_{\lambda}^{ki_1 \cdots i_{\ell -1}}\\ 
  {'\zeta}_{\mu}^{i_1 \cdots i_{\ell -1}q'} (E) = X_{\lambda}^{ i_1
    \cdots i_{\ell -1}q'} + w_{q+h}^{q'} X_{\lambda}^{i_1 \cdots
    i_{\ell -1}q+h}. 
\end{gather*}

Let $f$ denote a homomorphism of the submodule $('A^{(\ell)} _q)_E$ of
the dual of $(J^\ell)_z$ generated by $\{'\xi_{\varphi}^{k j;i_1
  \cdots i_{\ell -1}}(E), {'\zeta}_{\mu}^{i_1 \cdots i_{\ell -1}q'}
(E)\}$ into the module $\underbar{R}^{(\ell ,1)} (z)$ defined by 
$$
f(X_{\lambda}^{i_1 \cdots i_\ell}) = Y_{\lambda}{X^{i_1} \cdots X^{i_\ell}}.
$$

It\pageoriginale is easy to verify that $f$ is an isomorphism of
$('A^{(\ell)}_q)_E$ 
onto $A^{(\ell,1)}\break (z, E^{\chi})$ and hence $t' (E) = \dim A^{(\ell,
  1)} (z, E^{\chi})$. 

\section{Some results from the theory of ideals in polynomial
  rings}\label{chap3:sec3.10}%sec 3.10 
 
Let $X_1 \cdots X_p$ be $p$ indeterminates. Order the set of all
monomials in $X_1 \cdots X_p$ lexicographically. Denote by $M$ the set
of first $k$ elements in the set of all monomials of degree
$\ell$. Let $Q_{\ell}(k,p)$ denote the number of elements in the set
$MX_1 U \cdots U MX_p$. 

We shall state two theorems, without proof, regarding the function
$Q_{\ell}(k, p)$. For proof one can refer to ``Lectures on commutative
algebra'' by S.Bochner (1938). 

\setcounter{theorem}{0}
\begin{theorem}\label{chap3:sec3.10:thm1}%the 1
  Given $p$ and $k$, there exists an integer $Q_\ell (k, p)$ such that
  $Q_\ell (k, p) = Q(k, p)$ for sufficiently large $\ell$. 
\end{theorem}

\begin{defi*}
  $Q(k, p)$ is called Macauly function.
\end{defi*}

If $I$ is any homogeneous ideal in a polynomial ring $K[X_1 \cdots
  X_p]$, let $I^{(\ell)} = I \cap K^{(\ell)}[X_1 \cdots X_p]$ where
$K^{(\ell)}[X_1 \cdots X_p]$ is the submodule of homogeneous
polynomials of total degree $\ell$. Set $\phi ^{\ell} (I) = $ the
dimension of $I^{(\ell)}$. 

\begin{theorem}[Hilbert]\label{chap3:sec3.10:thm2}%the 2
  For any homogeneous ideal $I$ of $K[X_1 \cdots X_p]$
  there  exists an integer $\ell_0 (I)$ satisfying the following
  conditions: 
  \begin{enumerate}[\rm (i)]
  \item $\phi (I^{(\ell +1)}) > Q(\phi ^{(\ell)}, p)$ for $\ell < \ell_0 (I)$,
  \item $\phi (I^{(\ell +1)}) = Q(\phi ^{(\ell)}, p)$ for $\ell \geq
    \ell _0 (I)$. 
  \end{enumerate} 
\end{theorem}

Let\pageoriginale us write the symmetric algebra $\underbar{R}(z)$ as a direct sum
of the submodules $\underbar{R}^{(\ell, h)}(z)$ of homogeneous
elements of bidegree $(\ell , h)$ (degree $\ell$ in $X^1 \cdots X^p$
and degree $h$ in $Y_1 , \ldots, Y_{m'}$). $\underbar{R}(z) = \sum
\underbar{R}^{(\ell,h)} (z)$. Set $\underbar{R}^{(\ell)}(z) =
\sum\limits_{\ell' +h= \ell} \underbar{R}^{(\ell' , h)} (z)$ for $\ell
\geq 0$. 

\begin{defi*}
  To every reduced contact  element $(z, E^\psi)$ associate an ideal
  $B(z, E^{\chi})$ defined to be $\underbar{R}^{(0, 1)}(z) A(z,
  E^{\chi})$, the product being in the sense of multiplication in the
  symmetric algebra.  
\end{defi*} 

\begin{defi*}
  For any integer $\ell \geq 0$ define
  \begin{gather*}
    \phi^{\ell} (z, E^{\chi}) = \dim (A(z, E^{\chi}) \cap
    \underbar{R}^{(\ell +1)} (z))\\ 
    \psi^{\ell}(z, E^{\chi}) = \dim (B( z,E^{\chi}) \cap \underbar{R}^{(\ell +1)}(z))
  \end{gather*}
  By definition it follows that $\phi^{\ell} (z, E^\chi) - \psi
  ^{\ell}(z, E^\chi)$ is the dimension of $A^{(\ell ,1)} (z,
  E^{\chi})$. 
\end{defi*}

\begin{proposition}\label{chap3:sec3.10:prop14}%pro 14
  There exists a proper subvariety $S_{\ell}$ of $\chi_{\mathscr{G}}q$
  and a constant $\phi ^{\ell}_q (\sum, (D, D', \tilde{\omega})) =
  \phi_{q}^{\ell}(\sum)$ depending only on $(\sum)$ and $(D,
  D',\tilde{\omega})$ such that 
  \begin{gather*}
    \phi^{\ell} \left(z, E^\chi \right) < \phi^{\ell}_q \left(\sum \right)
    ~\text{for any}~ (z,  E^\chi ) \in S_{\ell}\\ 
    \phi^{\ell}(z, E^\chi ) = \phi_q^{\ell} \left(\sum \right) ~\text{for any}~ (z,
    E^\chi ) \not\in S_{\ell}. 
  \end{gather*}
\end{proposition}

\begin{proof}
  Let $S_\ell$ be the set of all reduced contact element $(z, E^\chi)$
  such that the function $\phi^\ell (z', E'^{\chi})$ in a nighbourhood
  of $(z, E^\chi)$ in $\chi_{\mathscr{G}} q$ is not a constant. Let
  $(x_1, \ldots , x_p, y_1, \ldots , y_m)$ be a coordinate system in a
  neighbourhood of $z$ in $(D, D', \tilde{\omega})$. Then we know that 
  $$
  \bigcup_{1 \leq i_1 < \cdots < i_q \leq p} \chi_{\mathscr{G}} q_{D'}
  (dx_{i_1} , \ldots , dx_{i_q}) = \chi_{\mathscr{G}} q 
  $$
  and\pageoriginale that $\bigcap \chi_{\mathscr{G}} q D' (dx_{i_1}, \ldots ,
  dx_{i_q}) \neq \phi$. Therefore it is sufficient to show that
  $S''_{\ell} = S_{\ell} \bigcap \chi \mathscr{G}^q D' (dx_1, \ldots
  dx_q)$ is a proper subvariety, outside of which the function
  $\phi^\ell$ is a constant. For $(z, E^\chi) \in \chi \mathscr{G} ^q
  D'(dx_1, \ldots, dx_q)$ the set of generators of the ideal $A^\ell
  (z, E^{\chi})$ are 
  $$
  \begin{cases}
    \xi_{\varphi}^{kj;i_1 \cdots i_{\ell -1}} = A_{\varphi}^{k
      \lambda} (z) Y_{\lambda} X^j X^{i_1} \cdots X^{i_{\ell-1}}
    -A_{\varphi}^{j \lambda} Y_{\lambda} X^k X^{i_1} \cdots X^{i_{\ell
        -1}},\\ 
    L^\chi Y_{\lambda} = \sum_{i=1}^p \langle dx_i , L^\chi \rangle X^i Y_{\lambda}
    (L^\chi \in E^\chi).  
  \end{cases}
  $$
\end{proof}

$\phi^\ell (z, E^{\chi})$ is the number of linearly independent such
generators. Let $N$ be the maximum of the function $\phi^\ell (z,
E^{\chi})$ on $^\chi \mathscr{G}^q D'(dx_1, \ldots, dx_q)$. Take a
subset $g_1 , \ldots, g_N$ of the above system of generators and let
$f$ be the determinant of the submatrix  of degree $N$ in the matrix
of coefficients in $g_1, \ldots , g_N$. Let $f_1, \ldots , f_r$ be the
set of all $f$ obtained by this process. Then it is clear that
$S''_\ell $ is the set of common zeros of $f_1, \ldots ,f_r$ and that
$S''_\ell$ has the required properties. Therefore $S_{\ell}$ is a
proper subvariety. Then our assertion follows easily. 

\begin{proposition}\label{chap3:sec3.10:prop15}%pro 15
  There exists a proper subvariety $S'_{\ell}$ of $^\chi \mathscr{G}^q
  D'$ and a constant $\psi_q^{\ell} (\sum ,(D, D', \tilde{\omega})) =
  \psi^{\ell}_q (\sum)$ depending only on $\sum$ and $(D, D',
  \tilde{\omega})$ such that  
  \begin{gather*}
    \psi^\ell_q (z, E^\chi ) < \psi^{\ell}_{q} \left(\sum \right) ~\text{for any}~
    (z, E^\chi ) \in S',\\ 
    \psi^\ell_q (z, E^\chi ) = \psi^{\ell}_{q} \left(\sum \right) ~\text{for any}~
    (z, E^\chi ) \not\in S'_{\ell}, 
  \end{gather*}
  The\pageoriginale proof is on the same lines as for Proposition \ref{chap3:sec3.10:prop14}.
\end{proposition}
$$
\text{ Set }\qquad   t^\ell_q (\Sigma) = \phi^\ell_q (\Sigma)-
\psiup^\ell_q (\Sigma).  
$$

\begin{proposition}\label{chap3:sec3.10:prop16}% Prop16
  If $ \phi^\ell (z, E^\chi)= \phi^\ell_q (\Sigma)$, then $t^\ell(z,
  E^\chi)= t^\ell_q(\Sigma)$. 
\end{proposition}

\begin{proof}
  By definition, $B(z, E^\ell)= Y_1 A(z, E^\chi) + \cdots + Y_m A(z,
  E^\chi)$ and\break $A (z, E^\chi)$ is generated by elements of type (1,
  1). It follows, then, that $B^{(\ell, j)} (z, E^\chi)= A^{(\ell,
    j)} (z, E^\chi)$ for $j \ge 2$. By the same argument as in the
  proof of Proposition \ref{chap3:sec3.10:prop14}, it follows that $\dim A^{(\ell, j)}(z,
  E^\chi)$ is upper- semi continuous. Hence the fact that 
  $\dim ~\phi^\ell(z', E^\chi)= \sum\limits_{k +j = \ell +1} \dim A^{(k,
    j)}$ $(z', {'E}^\chi )$, is constant on an open set implies that $\dim
  A^{(A +1-j , j)} (z', {'E}^\chi)$ is also a constant. Therefore the
  function $\psi^\ell (z', {'E}^\chi)= \sum\limits_{1 \le j \le \ell-1} \dim$
   $(A^{(\ell+ 1-j,j)}) (z', {'E}^\chi )$ is a constant on a
  neighbourhood of $(z, E^\chi)$. Hence   
  $$
  \psiup^\ell (z, E^\chi)= \psiup^\ell_q(\Sigma)~~ \text{ and so }
  ~~t^\ell (z, E^\chi)= t^\ell_q (\Sigma) 
  $$
\end{proof}

\begin{proposition}\label{chap3:sec3.10:prop17}%Prop 17
  $\phi^{\ell + 1}_q (\Sigma) \ge Q (\phi^\ell_q (\Sigma), P+m')$ for
  any $\ell$. There is an integer $\ell_0 (\Sigma)$ such that  
  $$
  \phi^{\ell +1}_q (\Sigma)= Q (\phi^\ell_q (\Sigma ),  p+m') ~~\text{
    for } ~~\ell \ge \ell_0 (\Sigma) 
  $$
\end{proposition}

\begin{proof}
  It is sufficient, by Theorem 2, to show the following: There is an
  ideal $I$ in a ring of polynomials in $p+m$ variables over a field
  $K$ such that $\phi^\ell_q (\Sigma)$ is equal to the dimension of
  the vector space $I^{(\ell)}$ over $K$. To construct such a field $K$
  and $I$, take a coordinate system $(x_1,\ldots, x_p, y_1, \ldots,
  y_m)$ in $(D, D', \tilde{\omega})$ and let $x^i, Y_\lambda$ be as
  before. For a connected open set $\mathscr{D}$ of $^{\chi}
  \mathscr{G}^q (dx_1, \ldots, dx_q)$, denote\pageoriginale by $k(\mathscr{D})$ the
  field of quotient of the ring of real analytic functions of
  $\mathscr{D}$. We remark that 
  $$
  \xi^{kj}_\varphi= A^{k \lambda}_\varphi Y_\lambda X^j- A^{j
    \lambda}_\varphi Y_\lambda X^k, \eta^{q^1}_\lambda= Y_\lambda
  X^{q^1} + w^{q^1}_{q + h} Y_\lambda X^{q+h} 
  $$
  can be considered, by restriction, as elements in the polynomial
  ring $K(\mathscr{D})[X^1, \ldots, X^p, Y_1, \ldots , Y_m,]$. Let
  $I(\mathscr{D})$ be the ideal generated by $\xi^{kj}_\varphi$,
  $\eta^{q^1}_\lambda$ in $K(\mathscr{D})[X,Y]$. For $ \mathscr{D}
  \subset \mathscr{D}^1$, we have the canonical injective mapping
  $k(\mathscr{D}^1) \rightarrow K(\mathscr{D})$. It is clear by
  definition that the image of $I(\mathscr{D}^1)$ generate
  $I(\mathscr{D})$ over $K (\mathscr{D})$. Then it follows easily that
  $\dim_{K(\mathscr{D'})} (I(\mathscr{D}')^{(\ell)})\break =
  \dim_{K(\mathscr{D^1})} (I(\mathscr{D})^{(\ell)})$. Now, set $K=K
  ({\chi \mathscr{G}^q} (dx_1, \ldots ,dx_q)), I=I (^{\chi}
  \mathscr{G}^q$ \break $(dx_1, \ldots , dx_q))$. For any fixed $\ell$, take
  $(z, E^\chi)$ such that $\phi^\ell (z, E^\chi)= \phi^\ell_q
  (\Sigma)$. Take a sufficiently small open connected neighbourhood
  $\mathscr{D}$ of $(z, E^\chi)$. Then it is easy to verify that
  $\dim_{K(\mathscr{D})} I(\mathscr{D})^{(\ell)} = \phi^\ell_q
  (\Sigma)$. By the above remark this implies that
  $\dim_k(I^{(\ell)})= \phi^\ell_q (\Sigma)$. This finishes the proof
  of Proposition \ref{chap3:sec3.10:prop17}. 
\end{proof}

\section{}\label{chap3:sec3.11}%Sec 3.11

In this section we introduce the notions of $\ell$- stable and $\ell$-
regular reduced constant elements and $P$-regular points, and prove
some of their properties. 

\begin{defi*}
  A reduced contact element $(z, E^\chi)$ is said to be $\ell -$
  stable (with respect to $[\Sigma, (D, D', \tilde{\omega})]$) if the
  function $\phi^\ell$ remains a constant in a neighbourhood of $(z,
  E^\chi)$ in $^{\chi} \mathscr{G}^q$. 
\end{defi*}

\begin{proposition}\label{chap3:sec3.11:prop18}%Prop 18
  A reduced contact element $z, E^\chi$ is $\ell-$ stable if and only
  if $\phi^\ell(z, E^\chi)= \phi^\ell_q (\Sigma)$, where $q=\dim
  E^\chi$. The set of\pageoriginale non-$\ell$-stable $q-$dimensional reduced
  contact elements is a proper subvariety $S^q_\ell $ of $^{\chi}
  \mathscr{G}^q$. If $(z, E^\chi)$ is $\ell$-stable then $t^\ell (z,
  E^\chi)= t^\ell_q (\Sigma)$. 
\end{proposition}

\begin{proof}
  The first two assertions are immediate corollaries of Proposition
  \ref{chap3:sec3.10:prop14}. The last assertion follows from the
  first and Proposition \ref{chap3:sec3.10:prop16}. 
\end{proof}

\begin{proposition}\label{chap3:sec3.11:prop19}%Prop 19
  There exists an integer $\ell_0 (\Sigma) $ depending only on
  $(\Sigma)$ with the following property: If $(z, E^\chi)$ is $\ell$-
  stable and if $\ell \geq \ell_0 (\Sigma)$ then $(z, E^\chi)$ is
  $(\ell + 1)$ stable. 
\end{proposition}

\begin{proof}
  Take $\ell_0(\Sigma)$ as in Proposition \ref{chap3:sec3.10:prop17}. Then 
  \begin{align*}
    \phi^{\ell+1}_q (\Sigma)&= Q (\phi^\ell_q (\Sigma), p+m^1)\\
    &= Q(\phi^\ell_q (z,E^\chi), p+m^1) \text{ by } \ell- \text{
      stability of }(z, E)\\ 
    &\leq \phi^{\ell + 1}_q (z, E) \text{ by the theorem of Hilbert .}
  \end{align*}
  This inequality, together with the inequality $\phi^{\ell + 1}_q
  (\Sigma) \geq  \phi^{\ell + 1}_q (z, E^\chi)$\break (Proposition
  \ref{chap3:sec3.10:prop14}). shows that $(z, E^\chi)$ is $(\ell +
  1)$- stable.  
\end{proof}

The Proposition \ref{chap3:sec3.11:prop19} simply states that for sufficiently large integer
$\ell, \ell$- stability implies $(\ell + 1)$- stability. In other
words for sufficiently large $\ell, S^q_{\ell + 1} \subseteq
S^q_\ell$. 

Let us denote by $S(\ell)= S(\ell; [\Sigma, D, D', \tilde{\omega}])$
the set of all $z$ in $D$ such that any reduced contact element with
origin at $z$ is not an $\ell$-stable element. 

\begin{proposition}\label{chap3:sec3.11:prop20}%20
  $S(\ell)$ is a proper subvariety of $D$. 
\end{proposition}

This\pageoriginale follows from the following much more general lemma.

\begin{lemma*}
  Let $(M, M', \rho )$ be a fibred manifold, $M''$ be a proper
  subvariety of $M$. Assume that, for every $x \in M', \rho^{-1} (x)$
  be connected. Then the set $M''= \left\{ x \in M' : \rho^{-1} (x)
  \subset M'' \right\}$ is a proper subvariety of $M'$. 
\end{lemma*}

\begin{proof}
  Let $U$ be an open subset of $M$ and set $V = \rho (U)$. We claim
  first that $x \in M'' \cap V$ if and only if $\rho^{-1}(x) \cap V
  \subset M''$. Namely, if $\rho^{-1}(x) \cap V \subset
  M'',\rho^{-1}(x) \cap M''$ contains interior points of
  $\rho^{-1}(x)$. Then, $\rho^{-1} (x) \cap M''$ being a subvariety
  and $\rho^{-1}(x)$ being connected, it follows that $\rho^{-1}(x)
  \cap M''= \rho^{-1}(x)$, i.e.. $x \in M''' \cap V$. converse is
  trivial. 
\end{proof}

Take $z_0$ in $M'$. Let $x_0 \in M$ be such that $\rho
(x_0)=z_0$. Take an open neighbourhood $U$ of $x_0$ with the following
conditions; 
\begin{enumerate}[(i)]
\item A coordinate system $(x,y)$ of $(M, M', \rho)$ is defined on $U$.
\item there exist real analytic functions $f_1, \ldots , f_k $ on $U$
  such that $M'' \cap U$ is equal to the common zeros of $f_1, \ldots
  , f_k$. For each fixed $y$ and $\lambda=1 , \ldots , k ,$ we define
  a function $f^y_\lambda$ on $\rho(U)=V$ by $f^y_\lambda(x)=
  f_\lambda (x,y)$. Then by the remark made at the beginning, $M'''
  \cap V$ is equal to the common zeros of $f^y_\lambda$. Therefore
  $M'''$ is a subvariety. Since $\rho (M-M'') \subseteq M'- M'''$ and
  since $M-M''$ is not empty, $M'''$ is a proper subset. 
\end{enumerate}

\noindent 
\textbf{Proof of Proposition 20.} 
Denote by $S_q (\ell)$ the set of point $z$ such that any
$q-$dimensional reduced contact element at $z$ is not
$\ell-$stable. Set $M = D, M' = D', M'' = S_q(\ell)$ . Applying the
lemma, we find that $S_q(\ell)$\pageoriginale is a subvariety. Therefore $S(\ell)=
\bigcup\limits^P_{q=0} S_q (\ell)$ is a subvariety  

\begin{defi*}
  Given an integer $\ell > 0 $, a reduced contact element $(z, E^\chi)$
  is said to be $\ell$- regular if $(z, E^\chi)$ is $\ell'-$ stable for
  $\ell' \geq \ell$.  

  Let $\mathscr{V}(\ell, q)$ be the set of all non $\ell$- regular
  reduced contact elements of dimension $q$. Clearly $\mathscr{V}
  (\ell, q)= \bigcup\limits_{\ell' \geq \ell} S^q_{\ell'}= \bigcup\limits_{0
    \leq h \leq \max (0, \ell_0 (\Sigma)-\ell)} S^q_{\ell + h}
  $(cf. Proposition \ref{chap3:sec3.11:prop19}). This shows that
  $\mathscr{V}(\ell, q)$ is 
  again a proper subvariety of $^{\chi} \mathscr{G}^q D'$. 
\end{defi*}

Now let us introduce the notion of reduced flags.

\begin{defi*}
  A reduced flag $^{q} F^\chi$ on a $q$-dimensional reduced contact
  element $(z, E^{\chi})$ is a finite sequence of reduced contact
  elements $\Big\{z, (z, E^{\chi}_1)$, $\ldots , (z, E^{\chi}_q)
  \Big\}$ such that $E^{\chi}_0 \subset E^{\chi}_1 \subset \cdots
  \subset E^{\chi}_q = E^{\chi}$ is a flag on $E^{\chi}$. 
\end{defi*}

Let $^{q} \mathscr{F}^\chi$ denote the set of all reduced flags on
$q$-dimensional contact elements of $(D, D',
\tilde{\omega})$. $q\mathscr{F}^\chi$ is contained in the product
space $D \times $(space of all flags on $q$-dimensional contact
elements of $D'$) $=D \times M$. 

Let $\rho $ be the map which associates to each flag its origin. Then
$^{q}\mathscr{F}^\chi= \left\{ q F^\chi : \tilde{\omega} (z) = \rho
(^{q} F^\chi ) \right\}$ . This defines the structure of a real
analytic submanifold of $D \times M$ on $^{q} \mathscr{F}^\chi$. 

\begin{defi*}
  A reduced flag $^{q} F^\chi= \left\{ (z, E^\chi_k) (k=0 , 1,\ldots ,
  q)\right\} $ is said to be $\ell$- regular if each component $(z,
  E^\chi_k)$ is $\ell$- regular. 
\end{defi*}

Let $S^q \mathscr{F}^\chi (\ell)$ be the set of all non $\ell$-
regular reduced flags $^{q}F^\chi $ on $(D,D', \tilde{\omega})$. $S^q
\mathscr{F}^\chi (\chi)$ is a proper real analytic subvariety\pageoriginale of
${q}\mathscr{F}^\chi$. For if $\rho_k$ denotes the projection
${q}\mathscr{F}^\chi \rightarrow \chi_{\mathscr{G}}^k$ associating to
each flag in $^q \mathscr{F}^\chi$ its $kth$ component then $S^q
\mathscr{F}^\chi (\ell)= \bigcup\limits^q_{k=0} \rho^{-1}_{k}(\gamma
(\ell, k)) $. 

\begin{defi*}
  A point $z \in D$ is said to be $P$- regular of weight $\ell$ with
  respect to $(\Sigma)$ if there exists a reduced flag $F^\chi$ on
  $(z,(D') \tilde{\omega}(z))$ which is $\ell -$ regular (with respect
  to $(\Sigma)$). 
\end{defi*}

Let $S[\ell]$ denote the set of all points $z$ in $D$ which are not
$P$-regular of weight $\ell$. Denote by $p$ the dimension of $D'$. Let
us denote the map $\rho : p_{\mathscr{F}}\chi \rightarrow D$ which
associates to every reduced flag its origin. Then since $ S(\ell)=
\left\{ z \in D : \rho^{-1} (z) \subset S^\chi \mathscr{F}^p
(\ell)\right\}$, it follows by the lemma that $S[\ell]$ is a proper
subvariety. We remark that $S(\ell)\supseteq S(\ell + 1 ) \supseteq
\cdots $ 

\begin{defi*}
  A point $z$ in $D$ is said to be $P$-regular of if there exists an
  integer $\ell \geq 0$ such that $z$ is $P$-regular of weight
  $\ell$
  with respect to $(\Sigma)$ if there exists a reduce flag $F^\chi$ on
  $(, (D') \tilde{\omega}(z))$ which is $\ell$ which is $\ell-$
  regular (with respect to $(\Sigma)$). 
\end{defi*}

Let $S[\ell]$ denote the set of all points $z$ in $D$ which are not
$P$- regular of weight $\ell$. Denote by $p$ the dimension of
$D'$. Let us denote the map $\rho : ^{q} \mathscr{F}^\chi \rightarrow
D$ which associates to every reduced flag its origin. The since $
S(\ell)= \left\{ z \in D :  \rho^{-1} (z) \subset S^\chi \mathscr{F}^p
(\ell) \right\}$, it follows by the lemma that $S[\ell]$ is a proper
subvariety. We remark that $S(\ell) \supseteq S (\ell + 1) \supseteq
\dots $ 

\begin{defi*}
  A point $z$ in $D$ is said to be $P$- regular if there exists an
  integer $\ell \geq 0$ such that is $z$ is $P-$ regular of weight
  $\ell$. 
\end{defi*}

Let $S=S(\Sigma, (D, D', \tilde{\omega}))$ be the set of all points
$z$ in $D$ which are not $P-$ regular. Clearly $S=
\bigcap\limits_\ell S(\ell)$. Hence the set of all non $P$-regular
points $z$ in $D$ is a proper subvariety of $D$. For any $z \notin S$ we
can construct a reduced flag $F^\chi= \left\{ (z,
E^\chi_q)\right\}_{q=0, 1, \ldots, p}$ on $(z,
(D')_{\tilde{\omega}(z)})$ such that $\phi^\ell (z, E^\chi_q)=
\phi^\ell_q (\Sigma)$ for sufficiently large $\ell$. 

\section{}\label{chap3:sec3.12}%Sec 3.12
 
Let $(z, E^\chi)$ be a fixed reduced contact element and let $L^\chi$
be in $(D')_{\tilde{\omega}(z)}= \underbar{R}^{(1, 0)}(z)$. Then the
multiplication $L^\chi f \in \underbar{R} (z)$ is defined for any $f
\in \underbar{R}(z)$. We now pose the following definition: 

\begin{defi*}
  $L^\chi $ is called $\ell_1$-prime to $(z, E^\chi)$ if the
  conditions $f \in \underbar{R}^{(\ell, 1)}(z)$, $l \geq \ell_1$\pageoriginale and
  $L^\chi f \in A(z, E^\chi)$ imply $f \in A (z, E^\chi)$. 
\end{defi*} 
 
\begin{proposition}\label{chap3:sec3.12:prop21}%21
  Let $(z, e^\chi ) \in \chi_{\mathscr{G}} q$ be a reduced contact
  element. Then there exists an integer $\ell_1= \ell_1 (z, E^\chi)$
  satisfying the following condition: the set of vectors $L^\chi \in
  \underbar{R}^{1,0}(z)$ which are $\ell_1$- prime to $(z, E^\chi)$ is
  everywhere dense in $\underbar{R}^{1,0}(z)$ 
\end{proposition} 
 
\begin{proof}
  since $\underbar{R}(z)$ is a Noetherian ring we can write $A (z,
  E^\chi)= A =\mathscr{G}_1 \cap \cdots \cap \mathscr{G}_a $ where
  $\mathscr{G}_1, \ldots , \mathscr{G}_a$ are primary ideals in
  $\underbar{R}(z)$. We shall denote by $R$ the algebra
  $\underbar{R}(z)$, for simplicity. It is known from the theory of
  ideals in polynomial rings that the set $\mathscr{G}_i$ of all
  elements $u$ in $R$ for which there exists an integer $\ell$ such
  that $u^\ell \in \mathscr{u}_i$ is a prime ideal in $R$. We can also
  assume that there exists an integer $n_i$ such that
  $\mathscr{U}^{n_i}_i \subseteq \mathscr{G}_i$ for $i=1, \ldots,
  a$. We can assume further that $\mathscr{G}_1,\ldots, \mathscr{G}_b
  $ are the primary ideals such that for any integer $\ell,
  \mathscr{G}_i \nsupseteq R^{(\ell, 1)}(1 \leq i \leq b)$; but for
  each $\mathscr{G}_i$ among $\mathscr{G}_{b + 1}, \ldots,
  \mathscr{G}_a$ there exists an integer $\ell_i$ such that
  $\mathscr{G}_i \supseteq R^{(\ell_i, 1)}$ and hence $\mathscr{G}_i
  \supseteq R^{(~,1)}$ for any $\ell \geq \ell_i$. 

  $b$ may be zero and in this case take $\tilde{\ell}= \max (\ell_1,
  \ldots, \ell_a)$. Then $A \supseteq R^{(\ell, 1)}$ for any $\ell
  \geq \tilde{\ell}$ and the proposition follows in this case. Hence
  we may assume $b >0 $, define $\tilde{\ell}= \max (\ell_{b+1,
    \ldots, \ell_a})$ if $b < a $ and $\tilde{\ell}=1$ if $b=a$. For
  any $\ell \geq \tilde{\ell}$ we have $\mathscr{G}_1 \cap \cdots \cap
  \mathscr{G}_b \cap R^{(\ell,1)}= A \cap R^{(\ell, 1)}$. We claim
  that $\mathscr{G}_i \cap R^{(1, 0)}$ is a proper subspace of $R^{(1,
    0)}= (D')_{\tilde{\omega}(z)}$ for $i =1, \ldots, b$. For let, if
  possible, $\mathscr{G}_i \supseteq R^{(1, 0)}$ so that
  $\mathscr{G}_i \supseteq \mathscr{G}_i ^{n_i} \supseteq R^{(n_i,
    0)}$ which is a contradiction to the choice of $\mathscr{G}_1,
  \ldots , \mathscr{G}_b$. 
\end{proof} 

Now\pageoriginale take a vector $L^\chi$ in $R^{(1, 0)}$ such that $L^\chi \notin
\cup \mathscr{G}_i (1 \leq i \leq b)$. Fix $\ell \geq
\tilde{\ell}$. Then the condition $L^\chi f \notin A^{(\ell+1, 1)}$
implies $L^\chi. f \in \ell_i$ $(i \leq i \leq b)$ because we can write
$A^{(\ell +1, 1)}= \mathscr{G}_1 \cap \cdots \cap \mathscr{G}_b
R^{(\ell +1, 1)}$. Then since $L^\chi \in \mathscr{G}_i$ it is known
from the theory of ideals in polynomial rings that $f \in
\mathscr{G}_i$ $(1 \leq i \leq b)$ which means that $f \in \mathscr{G}_1
\cap \cdots \cap \mathscr{G}_b \cap R^{(\ell, 1 )} =A^{(\ell,
  1)}$. Therefore the complementary set of $(\mathscr{G}_1 \cup \cdots
\cup\mathscr{G}_b  ) \cap (D')_{\tilde{\omega}(z)}$, which is
everywhere dense in $(D')_{\tilde{\omega}(z)}$ consists of all vectors
$\tilde{\ell}$- prime to $(z, E^\chi)$. 

Let $(D, D', \tilde{\omega})$ be a fibred manifold and $(\Sigma)$ a
normal differential system on it. We pose the following definition. 

\begin{defi*}
  If $z$ is a point $D$, a coordinate system $(x_1, \ldots , x_p)$ of
  $D'$ at $\tilde{\omega}(z)$ is said to be $\ell_1$- regular when the
  following conditions are satisfied (at $z$ with respect to
  $\Sigma$): 
  \begin{enumerate}[\rm (i)]
  \item $z$ is a $P$-regular point of weight $\ell_1$;
  \item the reduced flag $F^\chi = \left\{ (z, E^\chi_q ); (q=0, 1,
    \ldots , p-1)\right\}$ where each $E^\chi_q$ is the subspace of
    $(D')_{\tilde{\omega)} (z)}$ consisting of vectors $L$  such that
    $\langle dx_{q+1}, L \rangle= \cdots = \langle dx_p, L \rangle=0 $,
    is an $\ell_1$-regular reduced flag; 
  \item If $(L^1, \ldots, L^p)$ denote a base of
    $(D')_{\tilde{\omega}(z)}= M$ dual to $dx_1| M, \ldots$, $dx_p |M$,
    each $L^q$ is $\ell_1$- prime to $(z, E^\chi_{q-1}) (q=1, \ldots,
    p)$. 
  \end{enumerate}
\end{defi*} 
 
\begin{theorem}\label{chap3:sec3.12:thm3}%Thm 3
  If $z$ is a $P$-regular point of weight $\ell_0$(with respect
  to\break 
  $[\Sigma, (D, D', \tilde{\omega})]$), there exists an $\ell_1$-
  regular coordinate system of $D'$ at $\tilde{\omega (z)}$ (with
  respect to $[\Sigma , (D, D', \tilde{\omega})]$) for sufficiently\pageoriginale
  large $\ell_1$.  
\end{theorem} 

\begin{proof}
  Since $z$ is a $P$-regular point of weight $ \ell_0 $ it is also a
  $P$-regular point of weight $ \ell $ for $  \ell \geq \ell_0 $ so
  much so that we can assume without loss of generality that
  $\ell_o\geq \ell_o (\Sigma)$ .There exists an $\ell_0 $ - regular flag
  $F^\chi = \{ (z,E^\chi _q): (q=1, \ldots ,p)\} $ on $ (z,
  (D')_{\tilde\omega(z)})$ by definition. The set of non $\ell_0$
  -regular reduced flags being a proper real analytic subvariety, the
  set of $\ell_0$-regular reduced flags  is open in the manifold of
  all reduced flags. Hence there exists a neighbourhood $U_q$ of
  $(z, E^\chi _q)$ in $\chi_\mathscr{G} q $ such that any flag $'F^\chi $
  with its qth component in $U_q$ for each q is also
  $\ell_0$-regular. Consider the vectors $L^1,\ldots , L^p$ such that
  $L^q \in E^\chi _q $ and $E^\chi _q $ is generated by $E^\chi _{q-1}
  $ and $L^q$. we can choose a neighbourhood $U_q $ of $ (z,E^\chi _q
  ) $ and a neighbourhood $V_q$ of $L^q $ in such a way that the space
  ${''E}^\chi_q $ spanned by ${'E}^\chi_{q-1} \in U_{q-1}$  and  $'L^q
  \in  V_q $ is in $U_q$. Let ${'E}^\chi_q
  = \tilde{\omega}(z)$. Then by Proposition \ref{chap3:sec3.12:prop21}
  there exists a vector 
  $'L^1 \in V_1 \cap (D')_{\tilde\omega (z)}$  which is $\ell^0 $
  -prime to $ (z,{''E}^\chi_0)$ and which is such that ${''E}^\chi_1$
  spanned by ${''E}^\chi_0$  and $'L^1$ is in $U_1$, where
  $\ell^0=\ell_1(z,{''E}^\chi_0)$. Proceeding in this manner inductively
  we can choose $'L^q \in V_q \cap (D')_{\tilde\omega(z)}$ such that  
\end{proof}

\begin{enumerate}[(i)]
\item $\{(z,{''E}^\chi_q )\}$  is a reduced flag on $
  (z,(D')_{\tilde\omega (z)})$, which is $\ell_0 $ -regular , 
\item $'L^q$ is $\ell^{q-1}$-regular to $(z,{''E}^\chi_{q-1}$ where
  $\ell^{q-1}= \ell_1(z,{''E}^\chi_{q-1}),$ and  
\item ${''E}^\chi_q$ is generated by ${''E}^\chi_{q-1}$ and ${'L}^q$.
\end{enumerate}

Set\pageoriginale $\ell_1 = \max (\ell^0 , \ldots , \ell^{p-1})$. Then we have a
reduced flag $\{(z, {''E}_q^\chi)\}$ such that  

(i)~ it is $\ell_1$ regular

(ii)~ ${''E}^\chi_q$ is spanned by ${''E}^\chi_{q-1}$ and $'L^q$; ${'L}^q$ is
$\ell_1$- prime to $(z, {''E}^\chi_{q-1})$.  

Take a coordinate system $(x_1,\ldots , x_p)$ such that $dx_{q+1} |
{''E}^\chi_q = \cdots  = dx_p | {''E}^\chi_q = 0$. Let $L', \ldots ,L^p$ be
a basis of $(D')_{\tilde{\omega}(z)}$ dual to $dx_1, \ldots ,dx_p$. Now we
have $L^q = a^{q'} L^q +v^q (v^q \in {''E}^\chi_{q-1}), a^q \neq
0$. Because ${'L}^q$ is $\ell_1$- prime to $(z, E^\chi_{q-1})$, it is
easy to see  that $L^q$ is $\ell_1$-prime to ${''E}^\chi_{q-1}$ and this
completes the proof. 

\begin{theorem}\label{chap3:sec3.12:thm4}%Thm 4
  There exists an integer $\tilde{\ell} \left(\sum\right)$ such that
  for $\ell\geq \tilde{\ell} \left(\sum\right)$  
  $$
  \displaylines{\hfill 
  t^\ell_q \left(\sum\right) = t_{q-1}^\ell \left(\sum\right) +
  n_{\ell-1} - t ^{\ell-1}_{q-1} \left(\sum\right)  \hfill \cr
  \text{where}\hfill   
  n_{\ell-1} = \dim \underbar{R}^{(\ell-1,1)} (z) , 1\leq q \leq
  p.\hfill }
  $$
\end{theorem}

\begin{proof}
  There exists point $z$ which is $P$-regular of weight $\ell_1$ for a
  sufficiently large $\ell_1$. By Theorem \ref{chap3:sec3.12:thm3}
  there exists a 
  coordinate system $(x_1, \ldots ,\break x_p)$ of $D'$ at $\tilde{\omega}
  (z)$ which is $P$-regular. Hence there exists an
  $\tilde{\ell}$-regular reduced flag $F^\chi$ for an integer
  $\tilde{\ell}$ satisfying the following conditions. 
\end{proof}

(i)~ if $(z,E^\chi_q)$ is the qth component of $F^\chi, E^\chi_q$ is
the space of all vectors $L$ such that $\langle dx_{q+1}, L \rangle =
\ldots = \langle dx_p, L\rangle =0$;  

(ii)~ the base $X^1,\ldots ,X^p$ of $(D')_{\tilde{\omega}{(z)}}$ dual
to $dx_1, \ldots ,dx_p$ is such that $X^p$ is $\tilde{\ell}$- prime to
$E^\chi_{q-1}$. Then $\phi^\ell (z, E^\chi_q) =
\phi^\ell_q\left(\sum\right)$  
and\pageoriginale  by Proposition \ref{chap3:sec3.10:prop16} $\psi^\ell (z,E_q^{\chi})=
\psi^\ell_{q}(\Sigma)~\text{for}~ \ell \ge  \tilde{\ell} $. Hence $
t^\ell_{q} (\Sigma) = \phi ^\ell_{q} (\Sigma) - \psi^\ell_{q}(\Sigma)
= \text{dim } A^{(\ell,1)} (z, E^\chi_{q}) $ for $\ell \ge \tilde
{\ell}. A^{(\ell,1) } (z,E^\chi _{q}) $is generated as a vector space
by  
\begin{equation*}
  \begin{cases}
    \tilde{\xi}^{kj}_\varphi(z) X^{i_1}\ldots X^{i_{\ell -1}},\\
    L X^{i_1} \ldots X^{i_{\ell-1}} Y_\lambda , L\in E^\chi_{q}
    \subseteq \underbar{R}^{(1,0)}(z) 
  \end{cases}
\end{equation*}

Where $ \xi^{kj}_\varphi (z) = A^{k\lambda}_\varphi (z)Y_\lambda X^j -
A^{j\lambda} _\varphi (z) Y_\lambda X^k $ and $ X^i = \left(\dfrac
{\partial}{\partial x_i}\right)_z \in (D')_z = \underbar{R}^{(1,0)}(z)
$. Hence  
$$
\xi^{kj}_\varphi X^{i_i} \ldots X^{i_{\ell -1}}, Y_\lambda  X^{q'}
X^{i_1}\ldots X^{i_\ell-1}(q'= 1,\ldots,q) 
$$
generate $A^{(\ell,1)} (z,E^\chi_{q}) $ and hence $ A^{(\ell,1)}
(z,E^\chi_{q}) = A^{(\ell,1)}(z,E^\chi _{q-1}) + X^q$ \break
$\underbar{R}^{(\ell-1,1)} (z)$. But this need not be a direct sum. Let
$v_1,\ldots,v_\gamma$ be a set 
of maximum number of linearly independent elements in
$\underbar{R}^{(\ell-1,1)}(z)$ modulo $A^{(\ell-1,1)} (z,
E^\chi_{q-1})$ .Then  
$$
A^{(\ell,1)} (z,E^\chi _{q}) = A^{(\ell,1) }(z,E^\chi_{q-1}) +
X^q\underbar{R} V_1 + \ldots + X^q \underbar{R}V_\nu  
$$
where $\underbar{R}$ is the field of real numbers. We claim that this
is a direct sum decomposition. For, let a non-trivial relation  
$$
X^q b_1 V_1 + \ldots + X^q b_\nu V_\nu + a = 0 , a \in A^{(\ell , 1)}
(z, E^\chi _{q-1})  
$$
hold. That is, $ X^q (b_1 V_1 + \ldots + b_\nu V_\nu ) \in
A^{(\ell,1)} (z,E^\chi _{q-1}) \text{ Since } X^q \text{ is }
\tilde{\ell}-$ prime to $(z,E^\chi _{q-1}),$ it follows then that $
b_1 V_1 + \ldots + b_\nu v_n \in  A^{(\ell -1,1)},(z,E^\chi_{q-1})$. Then
by the choice of $ V_i, b_i=0$, and so $a =0$. Therefore
$t^\ell_{q-1} (\Sigma) + \nu = t^\ell_{q} (\Sigma)$. But by definition
$\nu =n_{\ell-1}- t^{\ell-1}_{q-1} (\Sigma)$. Therefore\pageoriginale we obtain the
required equality  
$$
t^\ell_{q}(\Sigma) = t^\ell _{q-1} (\Sigma) + n_{q-1} -
t^{\ell-1}_{q-1}(\Sigma) ~~\text{ for } ~~\ell \geq \tilde{\ell}~
\quad \text{q.e.d.} 
$$

By the same argument employed in the last part of the above proof, we
have the following :  
\begin{proposition}\label{chap3:sec3.12:prop22}%Prop 22
  Let $(z,E^\chi _{q-1})$ be a $(q-1)-$dimensional reduced contact
  element. Assume that $L^\chi \in (D')_{\tilde{\omega}_{(Z)}}$ is
  $\ell_1$-prime to $(z,E^\chi_{q-1})$. Denote by $E^\chi _q$ the
  subspace generated by $E^\chi_{q-1}\text{and } L^\chi$. Assume that
  $E^\chi _q$ is $q$-dimensional. Then for $\ell \geq \ell_1$ 
\end{proposition}
$$
t^\ell(z,E^\chi_q) = t^\ell (z,E^\chi_{q-1})+ n_{\ell-1} - t^{\ell-1}
(z,E^\chi_{q-1}). 
$$

\section{}\label{chap3:sec3.13}%Sec 3.13

\begin{defi*}
  We say that a reduced flag $ \big\{ (z,E^\chi _{q}) ; q=0 , 1,\ldots
  p\big\}$ is weakly $\ell $- stable when $t^\ell_q (z,E^\chi _{q})=
  t^\ell_q (\Sigma) \text { for } q=0,1,\ldots,p-1$.  

  Clearly, the set of weakly $\ell$ -stable reduced flags is open, and
  contains the set of $\ell$-stable reduced flags. 
\end{defi*}

\begin{proposition}\label{chap3:sec3.13:prop23}%Prop 23
  Let $F= \big\{E_q\big\}$ be an integral flag of $[p_S^{\ell}\Sigma,
    (j^\ell,D',\alpha)]$ satisfying the following conditions:  
\end{proposition}

\begin{enumerate}[(i)]
\item $\ell \geq\ell_1 (\Sigma) ; $
\item $(p_S^{\ell}\Sigma)_X^{(1)} \cap \Omega_ X = \{0\}, $ Where $X$
  is the origin of $F$ and 
$$
\Omega_X = \alpha^* (\wedge^1_{\alpha (X)}
  (D'));
$$
 
\item the reduced flag $F^\chi = \{ (z,E^\chi _{q}): q=0,1,\ldots,p\}$
  is weakly $\ell$-stable and weakly $(\ell-1)-$ stable, where
  $z=\beta (X) ~\text{and}~ z,E^\chi _{q} = d \alpha E_q$. Then $\dim
  (A^\ell)_E{_q} = f_q^\ell (z, E^\chi) = t^\ell_q (\Sigma)$ for
  $q=0,1,\ldots,p-1$ (cf. Proposition \ref{chap3:sec3.6:prop11}).  
\end{enumerate}

\begin{proof}
  We\pageoriginale write $A^\ell(E_q)$ instead of $(A^\ell_q)E_q$. The proof is by
  induction on the dimension $q$. When $X$ is an integral point, we
  have $A^\ell(X) \subseteq J(X) = (P_S^\ell \Sigma)_X^{(1)}$. Hence by
  (ii) $\dim A^\ell (X) = \dim(A^\ell(X) + \Omega_X/\Omega_X = t' (X)=
  t_o^\ell (x)$ (cf. Proposition \ref{chap3:sec3.9:prop13}).  The proposition is therefore
  proved in the 
  case $q=0$ . Let us assume that the proposition holds in the case
  $(q-1)$. Now since $ (z,E^\chi _{q})$is weakly-$\ell $-stable, 
  \begin{gather*}
    t^\ell_q (\Sigma) = t^\ell_q(z,E^\chi _{q})=t' (E_q) \quad
    \text{(by Proposition \ref{chap3:sec3.9:prop13})}\\ 
    \leq \dim A^\ell (E_q) \\
    \dim A^\ell(E_{q-1)} + n_{\ell-1} - t' (d
    \rho^\ell_{\ell-1} E_{q-1}) \quad \text{(by
      Proposition \ref{chap3:sec3.8:prop12})} 
  \end{gather*}
  where $n_{\ell-1} = \dim \underbar{R}^{(\ell-1,1)}_(z)$ and
  $\rho^\ell_{\ell-1}$ is the projection of $J^\ell \text{ onto }
  J^{\ell-1} $. By induction assumption and
  Proposition \ref{chap3:sec3.9:prop13}  the
  latter member of the above inequality is equal to  
  $$
  t^\ell_{q-1}(\Sigma) + n_{\ell - 1 }- t^\ell_{q-1}(z,E^\chi _{q-1})
  $$ 
  Hence, since the flag is weakly -($\ell-1)$-stable we obtain
  $t^\ell_q(\Sigma) \leq \dim A^\ell$ $(E_q) \leq t^\ell_{q-1} (\Sigma)
  + n_{l-1}-t^{\ell-1}_{q-1} (\Sigma)$. But for $\ell
  \geq\ell_1 (\Sigma) , t^\ell_q (\Sigma) = t^\ell_{q-1} (\Sigma) +
  n_{\ell-1} - t^{\ell-1}_{q-1} (\Sigma) $ by Theorem \ref{chap3:sec3.12:thm4}. Therefore
  $t^\ell_q(\Sigma) = \dim A^\ell (E_q) $ and this proves the
  proposition.  
\end{proof}

Let [$\Sigma,(D,D',\tilde\omega ) $] be a differential system with
independent variables. Suppose that $\dim D' = p$. We pose the
following definition: 
\begin{defi*} 
  The\pageoriginale system [$ \Sigma,(D,D',\tilde\omega)$] is said to be in
  involution at an integral point  $z \in D $ if the following
  conditions are satisfied: 
\end{defi*}

\begin{enumerate}[(i)]
\item any integral element, of dimension p of
  $[\Sigma,(D,D',\tilde\omega)]$ with origin at $z$, is an ordinary
  integral element,  
\item there exists a $p$-dimensional integral element with origin at $z$
  of $[\Sigma,(D,D',\tilde\omega)]$. 
\end{enumerate}

\begin{defi*}
  An integral point $z$ of $[\Sigma,(D,D',\tilde\omega)] $ is a normal
  integral point if the following conditions are satisfied:  
\end{defi*}

\begin{enumerate}[(i)]
\item  $z$ is an ordinary integral point;
\item there exists a neighbourhood $U$  of $z $ in $ D$ such that  for
  any $z'\in U \cap v^\circ \Sigma $, we have  
  $$
  (\Sigma^{(1)})_{z'}\cap \tilde{\omega}^*
  (\wedge^{(1)}_{\tilde{\omega}(z)} (D')) = \{0\}. 
  $$
\end{enumerate}

\begin{proposition}\label{chap3:sec3.13:prop24}%Prop 24
  If $[\Sigma,(D,D',\tilde\omega)]$ is in involution at an integral
  point $Z$ in $D$ then $Z$ is a normal integral point. 
\end{proposition}

\begin{proof}
  By definition $z$ is an ordinary integral point and there exists an
  integral element $E$ of dimension $p$ at $z$. Let $(x_1,\ldots,x_p)$
  be a coordinate system of $D'$ at $\tilde{\omega}(z)$ . Then $E$
  being an integral element of $[\Sigma,(D,D',\tilde\omega)]$ $dx_1
  |E,\ldots,dx_p| E $ are linearly independent. Because of corollary
  to proposition \ref{chap2:sec2.4:prop8} (Chapter II) there exists a neighbourhood $U$
  of $Z$ in $D$ such that for any $Z'\in \cup \cap \vartheta^0\Sigma $
  there is an integral element $E'$ of $Z'$ such that
  $dx_1|E',\ldots,dx_p|E'$ are linearly independent, Take an element $
  a_1 (dx_1)_{z'} + \ldots + a_p(dx_p)_{z'}$\pageoriginale in $(\Sigma^{(1)})_{z'},
  \cap \tilde{\omega}^* (\wedge_{\tilde\omega(z)}D') |E'$. But since
  $ E'$ is an integral element of $(\sum), \Sigma^{(1)}| E' = 0 $ so
  much so that $ a_1(dx_1|E') + \ldots + a_p(dx_p |E') = 0,$  that is
  $a_1 = \ldots= a_p = 0$.  
\end{proof}
 
\begin{defi*}
  $Z$ in $D$  is called $P$-weakly -regular of weight $\ell_0$ with
  respect to $[\Sigma,(D,D',\tilde\omega)])$  when there is a reduced
  flag over $(z,(D')_{\tilde{\omega} (z)}) $ which is $\ell $ - weakly
  -stable for $\ell\geq \ell_0$. Therefore $z$ is a normal point.  
  
  Clearly , a $P$-regular point of weight $\ell_0$ is a $\ell$-weakly
  - regular point of weight $\ell_0 $ . 
\end{defi*} 

\section{}\label{chap3:sec3.14}%Sec 3.14

\begin{theorem}\label{chap3:sec3.14:thm5} % theorem 5
  Let $\left[\Sigma,(D,D',\tilde\omega)\right]$ be a normal
  differential system with independent variables and $P^\ell_S \Sigma
  $ be its standard prolongation to  $J^\ell(D,D',\break \tilde{\omega})$.  
  Let $z \in D $ be a $P$-weakly -regular point of weight $\ell_0$
    and $X$ be an integral point of [$P_S^\ell \Sigma, (J^\ell,
      D', \alpha)$] such that $\beta(X) = z$. Then,if $\ell \geq
    \max [\ell_0 + 1 , \ell_1(\Sigma)]$ the system $[P_S^\ell
      \Sigma, (J^\ell, D', \alpha)]$ is in involution at $X$ if and
    only if $X$ is normal with respect to $P_S^\ell \Sigma$.  
\end{theorem}

\begin{proof}
  In view of the Proposition \ref{chap3:sec3.13:prop24}, it is sufficient to prove that if
  $X$ is a normal integral point then $P_S^\ell \Sigma, $ is in
  involution at $X$.  
  
  Take a $p$-dimensional integral element $E$ of $[P_S^\ell \Sigma,
    (J, D', \tilde{\omega})]$ at $X$. Since $z$ is a $P$-regular point
  of weight $\ell_0$ there  exists an $\ell_0$ -regular reduced flag
  $F^\chi= \{(z,E^\chi_q)\}; $ that is each component is
  $\ell'$-stable for any $\ell'\geq \ell_0$. Since $d \alpha $ is an
  isomorphism of $E$ onto $(D')_{\tilde\omega(z)} $ there exists a
  subspace $E_q \subset E $ such that $D \alpha$ is an isomorphism of
  $E_q$ onto $E^\chi_q$. Now  
  $ F= \{E^q\}$\pageoriginale is a flag on $E$ at $X$. We claim that  each component
  $E_q$ of $F$ is regular; that is there a neighbourhood
  $\mathcal{U}_q$ of $E_q$ in $\mathscr{G}^q
  J^\ell(D,D',\tilde{\omega)}$  such that for any element $ E'_q\in
  \mathscr{U_q} \cap \vartheta^q (p^\ell_S \Sigma)$ we have $ \dim
  J(E'_q)$. Let $\Omega(X) = \alpha^* (\wedge^{(1)}_{\alpha (X)}
  (D'))$. 
\end{proof}

By condition (ii) of normality of $X$ there exists a neighbourhood
$U$ of $X$ in $J^\ell $ such that for any $X'$ in $ \mathscr{U}\cap
\vartheta^0 (P_S^\ell \Sigma)$ we have $ ((P_S^\ell \Sigma)^{(1)}\cap
\Omega(X') = \{0\}$. Choose $\mathcal{U}_q$ so small that  
\begin{enumerate}[i)]
\item for any $E'_q \in \mathcal{U}_q, $ the origin $X'$ of $E'_q$ is in $U$; 
\item $(\beta (X'') , d\alpha  (E'_q))$ is $\ell_0 $ -weakly -regular . 
\end{enumerate} 

Hence all the conditions of Proposition \ref{chap3:sec3.13:prop23} are satisfied and so 
$$
\dim J(E'_q) = \dim (G^\ell)_X + \dim (\pi^{(\ell) })_X + t^\ell_q (\Sigma) 
$$

The right member of this equation is independent of the choice of \break
$E'_q$ in $\mathcal{U}_q: $ This proves that $E$ is an ordinary
integral element.  

Now it only remains to show that there exists a  $P$-dimensional
integral element of $P_S^\ell \Sigma$ and or origin at $X$. Since $X$
is normal $J(X)\cap \Omega(X) = \{0\}$ and therefore dim $J(X) = dim
[(J(X) + \Omega (X)) / \Omega (X) ]$. 
 
Let $(x_1, \ldots, x_p) $ be a coordinate system of $D'$ at $\alpha
(X)$ such that $E^\chi_q$ is generated by the first $q$ elements in
the base of $E^\chi_q$ to $dx_1, \ldots , dx _p. \text{ Let } L_1 $ be
a solution of the system of  equations  
\begin{align*}
  \langle J(X) , L\rangle & = 0,\\
  \langle dx_1, L\rangle &= 1, \langle dx_2,L\rangle  =\ldots =\langle
  dx_p,L\rangle  = 0.  
\end{align*}
Let\pageoriginale $E_1$ denote the one dimensional contact element spanned by
$L_1$. $E_1$  is an one dimensional integral element of $P^\ell_S \Sigma
$. Again 
$$
J(E_1) = (G^\ell)_X + (\pi ^{(\ell)})_X + A^\ell(E_1) 
$$ 
$\dim A^\ell(E_1) = t^\ell_1 (\Sigma)$  and so  $J(E_1) \cap \Omega
(X) + \{0\}$. Let $L^2 $ be the solution of the equations 
 \begin{gather*}
   \langle J(E) ,L \rangle= 0,\\
   \langle dx_1,L\rangle = 0, \langle dx_2, L \rangle =1, \langle
   dx_3, L \rangle = \ldots = \langle dx_p,L\rangle = 0.   
 \end{gather*} 
 Repeating this process we construct a $p$-dimensional integral element
 of $ P_S^\ell \Sigma $. This completes the proof of the theorem. 
 
\begin{remark*}
In Theorem \ref{chap3:sec3.14:thm5}, the assumption that $z$ is
$P$-regular can be 
   replaced by the following assumption: There is a reduced flag on
$$
(z,(D')_{\tilde\omega(z)}) 
$$ 
such that each of its component
   $\ell$-weakly -stable and $ (\ell-1)$ -weakly stable. The reason is
   that the former assumption is used in the proof of
   Theorem \ref{chap3:sec3.14:thm5} only
   to the existence of a reduced flag having the property in the
   latter assumption.  
\end{remark*}  
 
\section{}\label{chap3:sec3.15}%Sec 3.15
 
Let $ [\Sigma, (D, D', \tilde{\omega})]$ be a differential system with
independent variables. We set $J^\ell = J^\ell(D, D',
\tilde{\omega)}$. The standard prolongation $P^\ell_S \Sigma$ on
$J^\ell$ is generated by $(P^\ell_S \Sigma)^{(0)}, \pi ^{(\ell)} $ as
an ideal\pageoriginale closed  under $d$. Here $\prod(\ell)$ and the operator $d$ do
not depend on the system given. Thus $P^\ell_S \Sigma$ is completely
determined by $(P^\ell_S \Sigma)^{(0)}$. More generally, let us
consider a submodule $F$ of $\wedge^0 J^\ell$ .We shall construct a
submodule $F'$ of $\wedge^0j^{\ell+1}$ out of  $F$ having the
following property : When this construction is applied to ($P^\ell_S
\Sigma)^{(0)}$ , we obtain $(P^{\ell+1}_S \Sigma)^{(0)}$. To make the
matter more general, we will consider subsheaves of the sheaf of germs
of (real analytic) functions, instead of submodules. 
 
Let ($X_1,\ldots , X_p,Y_1l,\ldots,Y_m) = (X,Y) $ be a coordinate system in 
$ (D, D'$, $\tilde{\omega}). J^\ell$ has the coordinates system ($
X,Y,\ldots,Y_\lambda ^{i,\ldots i_\nu},\ldots)~(\nu \leq \ell)$
associated with $(X,Y) $. Let $f$ be a function defined on an open set
$U$ in the domain of the coordinate system. Consider $f$ as a function
on $(\rho^{\ell+1}_\ell)^{-1}(U)=U' $. Then we have $df \equiv f^j
dx_j \pmod{\prod(\ell + 1)}$ where $f^j$ are functions on $U'$
This follows from the facts that $df$ is a linear combination of
$dx_j,dy_\lambda, dy_\lambda^{i_1\ldots i_\nu}(\nu \leq \ell)$ and
  that $dy^{i_,\ldots i_\nu }_\lambda \equiv y_\lambda ^{i_1\ldots i_\nu j }
dx_j \pmod {\prod (\ell+1)}$. We set $D^j f= f^j$. Clearly 
$$
D^j(\alpha f+ \beta g) = \alpha D^j (f) + \beta D^j(g) , D^j (fg)=
(D^jf) g + f(D^jg)   
$$
Where $\alpha , \beta \in \underbar{R}$. When $(X,Y) $ is changed to
$(X',Y') , $ have $dx_j = a^k _j dx'_k$. Since $\prod (\ell+1) $ does
not depend upon the choice of the coordinate system, df$
\equiv(D^jf)dx_j = (D^j f ) a^k _j dx'_k (mod \prod (\ell+1))$. 
Therefore 
$$
D'^j (f) = a^j_k (D^k(f)). 
$$

Let\pageoriginale $F$ be an ideal of $\wedge^0(U)$. Denote by $P(F)$ the ideal in
$\wedge^0(U') $ generated by $F o \rho^{\ell+1}_\ell$ and $D^j (F) ,
$ where $j=1,\ldots,P$ and $f$ runs through $F$. The above rule for
change of $D^j$ under coordinate transformation shows that $P(F) $ is
independent of the choice if coordinate system employed to construct
$P(F)$.  

Let ($M, M',\tilde{\omega}) $ be a fibered manifold. Denote by
$\mathscr{O}J^\ell $ the sheaf of germs of real analytic functions on
$J^\ell$. $\mathscr{O}J^\ell$ is a sheaf of rings. Let $\mathscr{U}$ be
an open set of $J^\ell$. Let $ \Phi $ be a subsheaf of  ideals of $
\mathscr{O}J^\ell|\mathscr{U}$, the restriction of $\mathscr{O}J^\ell
$ to $\mathscr{U} $. For each open set $ U \subset \mathscr{U}$ of $J^\ell
$ , denote by $\Gamma(U,\Phi) $ the ring of cross -sections of $\Phi$
over.$U$. For each open set $V$ of $ J^{\ell+1} $ such that
$\rho^{\ell+1}_\ell (V) = W\subset \mathscr{U}, $ denote by $ \Psi (V) $
the ideal in $\wedge^0 (V) = \Gamma (V, \mathscr{O}J^{\ell+1})$
generated by the restriction of  $P(\Gamma((\rho^{\ell+1}_\ell)^{-1}(W)
, \Phi))$. If $V'\subset V$, the restriction mapping sends $\Psi(V)$
into $ \Psi (V')$.  Hence the system $\Psi(V)$ defines a subsheaf of
ideals of $\mathscr{O }J^{\ell+1} |(\rho^{\ell+1}_\ell)^{-1}
(\mathscr{U},)$ which will be denoted by $ P(\Phi)$. $P(\Phi) $ is
called the standard  prolongation of $\Phi $ . Let us assume now that
$M= D, M'= D'$. Let $\left[(P^\ell_S \Sigma)^{(0)}\right] $ the
subsheaf of ideals in $\mathscr{O} J^\ell$ generated by
$[(P^\ell_S\Sigma ) ^{(0)}] $ the subsheaf of ideals in $
\mathscr{O}J^\ell $ generated by $(P^\ell_S \Sigma )^{(0) }$ . Then 

\begin{proposition}\label{chap3:sec3.15:prop25}%Prop 25
  $P([(P_S^\ell \Sigma )^{(0)}]) = [(P^{\ell+1}_S \Sigma)^{(0)}]. $
\end{proposition}

\begin{proof}
  By definition, $ (P^\ell_S \Sigma)^{(0)}$ is generated by 
  $ F^{k_1,\ldots k_a; i_1\ldots i_r}_\varphi (\varphi\in \Sigma
  ^{(a)}; 1\leq k_1,\ldots,k_a; i_1,\ldots,i_r\leq p;r\leq \ell-1)$.  
  
  By Proposition \ref{chap3:sec3.3:prop3}, $ D^j(F_\varphi^{k_1\ldots k_a ; j_1\ldots i_r
  }) = F_\varphi ^{k_1\ldots k_a ; i_1\ldots i_r j } $. Therefore
  then ideal $(P^{\ell+1}_S \Sigma)^{(0)}$ is generated by $ D^jf $
  and $f$, where  
  $f$\pageoriginale runs through \break $(P_S^\ell \sum)^{(0)}$. Hence our equality follows
  from the definition of $P$. 
\end{proof}

Now we pose the following 
\begin{defi*}
  By a partial differential equation of order $k$ on $(M, M'$,
  $\tilde{\omega})$, we mean an open set $\mathcal{U}$ in $J^k(M,M',
  \tilde{\omega})$ and a subsheaf of ideals $\Phi$ in $\mathscr{O} J^k
  | \mathcal{U}$ such that $\Phi$ is locally finitely generated. By
  the $\ell$ -th standard prolongation of the partial differential
  equation $\Phi$, we mean the open set $(\rho^{\ell+1}_\ell)^{-1}
  (\mathcal{U})$ and the subsheaf of ideals $P(\cdots (P(\Phi))\cdots)
  = P^\ell (\Phi)$, where we operate $P$ $\ell$-times. 
\end{defi*}

It is clear by the definition that the standard prolongation of a
partial differential equation is again a partial differential
equation. It will be easy to see that our definition of partial
differential equations is equivalent to the usual one when
$(M,M',\tilde{\omega})=(D,D', \tilde{\omega})$. Also it will be
clear by Proposition \ref{chap3:sec3.15:prop25} that the notion of partial differential
equations and their prolongations includes the notion of exterior
differential systems and their prolongations.  

\section{}\label{chap3:sec3.16}%Sec 3.16

Let $X$ be a point of $\mathscr{U}$. If $f(X)=0$ for any $f$ in
$\Gamma (U, \Phi)$ and for any open neighbourhood $U$ of $X$ in
$\mathscr{U}$, then we say that $X$ is an integral jet of the equation
$\Phi$. Denote by $\vartheta^o \Phi$ the set of integral jets of $\Phi$. It
is clear that $\vartheta^{o}\Phi$ is a subvariety of $\mathscr{U}$. Take an
open set $D$ in $\mathscr{U}$ such that a coordinate system in $(M,M',
\tilde{\omega})$ is defined on $D$. Set $D' =\tilde{\omega} (D)$. Then
$(D,D', \tilde{\omega})$ is a fibered manifold. Assume that $\Gamma (D,
\Phi)$ is finitely generated\pageoriginale and that $\Phi | D$ is generated by
$\Gamma (D, \Phi)$. For any $X$ in $J^k$ we can choose such a $D$
containing $X$, because $\Phi$ locally finitely generated. Denote
$\sum (D, \Phi)$ the exterior differential system generated by $\Gamma
(D, \Phi)$ and $\prod (k). \sum (D,\Phi)$ is called a defferential
system associated with $\Phi$. Thus $P^k\sum$ is associated with
$\left[(P^k \sum)^{(0)}\right]$. Let $g$ be a cross-section of
$(M,M', \tilde{\omega})$ over an open set in $M'$. We say that $g$ is an
integral of $\Phi$ when $j^k(g)\subset \vartheta^{0} \Phi$. It is
equivalent to say that, for any $\sum(D,\Phi)$ such that $D$
intersects with the image of $g$, the restriction of $g$ is an
integral of $\sum(D, \Phi)$. This follows from Proposition
\ref{chap3:sec3.2:prop2}. By the
proposition we also have the following: Let $G$ be an integral of an
associated system $\sum(D,\Phi)$. Then there is an integral $g$ of
$\Phi$ such that $G=j^k(g)$. Thus the problem of finding integrals of
associated differential systems.  

\begin{proposition}\label{chap3:sec3.16:prop26}%Prop 26
  Let $\Phi$ be a partial differential equation of order $k$. Let
  $\left[\sum, (D, D', \tilde{\omega})\right]\sum(D,\Phi)$ be an
  associated differential system. Then $\sum((\rho^{k+\ell}_{k} )^{-1}
  (D), P^\ell (\Phi))$ is isomorphic to an admissible restriction of
  \break $P^\ell_S \left[\sum, (D,D',\tilde{\omega})\right]$. 
\end{proposition}

\begin{proof}
  We can assume without loss of generality that $M'=D'$. Also it is
  easy to reduce the proof to the case $\ell=1$, by Proposition
  \ref{chap3:sec3.4:propdash8}$'$. So we assume that $\ell=1$. As mentioned in $p.96$, there is a
  canonical injection $\ell$ of $J^{k+1} (M,D',\tilde{\omega})$ into
  $J'(J^k(M,D',\tilde{\omega}), D', \alpha)$. $\ell$ induces an
  isomorphism of $\prod (k+1;(M,D'\tilde{\omega}))$ to an admissible
  restriction of $P'_S\left[\prod (k; (M,D',\tilde{\omega})),
    (D,D',\alpha)\right]$. $P'_S\left[\sum, (D,D',\alpha)\right]$ is\pageoriginale
  generated (with $d$) by $\Gamma(D,\Phi) o \rho^{\ell+1}_\ell,
  F^k_{d\varphi} (\varphi \in \Gamma (D,\Phi))$, and by $P^1_S
  [\prod (k; (M,D',\tilde{\omega}))$, \break $(D,D',\alpha)]$. By
  Proposition \ref{chap3:sec3.3:prop5} 
  $$
  d \varphi\equiv F^k_{d\varphi} dx_k \mod \prod (1; (J^k (M,D',
  \tilde{\omega}),D',\alpha)). 
  $$
\end{proof}

Then since $\vartheta*(\prod (1; (J^k (M,D',\tilde{\omega}), D',\alpha)) =
\prod (k+1; (M,D', \tilde\omega))$ as proved in p.97, it follows
that $F^k_{d\varphi}o \ell = D^k \varphi$. Therefore $\vartheta$ induces
an isomorphism of $\sum((\rho ^{k+1}_{k})^{-1} (D),P(\Phi))$ to an
admissible restriction $P^1_S \left [ \sum, (D,D',\alpha)\right]$. 

We say that an integral jet $X$ of $\Phi$ is ordinary when $X$ is
ordinary with respect to an associated differential system
$\sum(D,\Phi)$ (such that $D \ni X$). When this is so, $X$ is an
ordinary integral point of any associated system $\sum(D_1, \Phi)$
such that $D_1 \ni X$. The definition implies immediately the
following: When $X$ is an ordinary integral jet of $\Phi$ and $U$ is a
suitable open neighbourhood of $X$, $\ell^0 \Phi \cap U$ is
submanifolds of $U$ and $\Gamma (U, \Phi)=0$ is its regular local
equation. We say that $\Phi$ is in involution at an integral jet $X$,
when an associated differential system is in involution at $X$.  

Let $\Phi,\Psi$ be partial differential equations of order $k$ on $(M,
M', \tilde{\omega})$. Denote by $\mathscr{U}$, $\omega$ the sets in
$J^k$ on which $\Phi,\Psi$
are given, respectively. We say $\Phi \subset \Psi$, when
$\mathscr{U}\subset \omega$ and $\Phi_x \subset \Psi_x$ for any
$x$ in $\omega$. We say that $\Phi$ is a restriction of $\Psi$,
when $\mathscr{U} \subset \omega$ and $\Phi_x = \Psi_x$ for any $x$ in
$\mathscr{U}$. 

Our main purpose is to prove the following

\begin{theorem}\label{chap3:sec3.16:thm6}%Thm 6
  Let $(M,M', \tilde{\omega})$ be a fibered manifold. Assume that a
  partial\pageoriginale differential equation $\Phi^\ell$ of order $\ell$ on
  $(M,M',\tilde{\omega})$ is given for any $\ell \geq \ell_0$. Let
  $g^0$ be a cross-section of $(M,M',\tilde{\omega})$ over an open
  neighbourhood of a point $x^0$ in $M'$. Assume the following: for
  any $\ell \geq \ell_0$  
  \begin{enumerate}[\rm (i)]
  \item $g^0$ is an integral of $\Phi^\ell$,
  \item $\Phi^{\ell+1} \supseteq p (\Phi^\ell)$ on a neighbourhood of
    $X^\ell = j^{\ell}_{x^0} (f^0)$, 
  \item $X^\ell$ is an ordinary integral jet of $\Phi^\ell$,
  \item for a suitable open neighbourhood $U$ of $X^{\ell_0}$,
    $(\ell^o \Phi^{-o} \cap U, \alpha (U),\alpha)$ is a fibered manifold,
  \item $(\ell^0 \Phi ^{\ell+1} \cap V, \ell^0 \Phi ^\ell \cap V',
    \rho^{\ell+1}_{\ell})$ is a fibered manifold for suitable open
    neighbourhoods $V,V'$ of $X^{\ell+1},X^\ell$, respectively.  
  \end{enumerate}
\end{theorem}

Then there is an integer $\ell_1$ such that $\Phi^{\ell+1}$ and
$P(\Phi^\ell)$ are equal in a neighbourhood of $X^{\ell+1}$ and such
that $\Phi^\ell$ is in involution at $X^\ell$ for any $\ell\geq
\ell_1$  
 
\section{}\label{chap3:sec3.17}%Sec 3.17

In this article, we keep the notations in the above theorem and assume
that the assumption is satisfied. Choose a coordinates system $(x,y)$
in $(M,M',\tilde{\omega})$ defined on a neighbourhood of $g^0(x^0)$
such that $x^0=(0)$ and the cross-section $g^0$ is represented by
$y_\lambda =0$. Let $f$ be a function defined in a neighbourhood of
$X^\ell$. Expanding $f$ in the power series in $x,y,\ldots ,y_\lambda
^{i_1\cdots i_\nu},\ldots (1 \leq \nu \leq \ell)$, we have  
$$
f = a^\lambda_{i_1\cdots i_\ell} (x,y)y_\lambda^{i_1\cdots i_\ell} +
f' o \rho^\ell_{\ell-1} + h 
$$
where $f'$ is a function on $J^{\ell-1}$ and $h$ is a function on
$J^\ell$, each\pageoriginale of whose terms is of atleast degree two in
$y^{i_1\cdots i_\nu}_{\lambda}$. The function $a^\lambda_{i_1\cdots
  i_\ell} (x,y) y^{i_1\cdots i_\ell}_\lambda$ is called the principal
part of $f$ and is denoted by $R(f)$, or $R^\ell(f)$.  
  
\begin{lemma*}
  Under the above notations,
  $$
  R^{\ell+1} (D^j f) = a^\lambda_{i_1\cdots i_\ell} (x,y) y^{i_1\cdots i_\ell j}
  $$
\end{lemma*}   

\begin{proof}
  Because of the above expansion of $f$, we have
  \begin{align*}
    df&= a^\lambda_{i_1\cdots i_\ell}(x,y) dy^{i_1\cdots
      i\ell}_\lambda +y^{i_1 \cdots i\ell}_\lambda +
    da^\lambda_{i_1\cdots i\ell} + d(f'o \rho^\ell_{\ell-1})+dh\\ 
    &\equiv (a^\lambda_{i_\ell \cdots i\ell} y^{i_1\cdots i_\ell j}_\lambda
    + y_\lambda ^{i_1\cdots i_\ell} D^j (a^\lambda_{i_1\cdots i_\ell}) +
    (D^j f')o \rho^{\ell+1}_\ell\\ 
    & \hspace{3cm}+ (D^jh) dx_j \pmod {\prod (\ell+1)}. 
  \end{align*}
\end{proof}

Therefore
$$
D^jf = a^\lambda_{i_1\cdots i_\ell} (x,y)y^{i_1\cdots i_\ell j}_\lambda +
y^{i_1\cdots i_\ell}_\lambda D^j (a^\lambda_{i_1 \cdots i_\ell}) +
(D^jf') o \rho^{\ell+1}_\ell +D^j h. 
$$
Then our conclusion follows immediately

Introduce indeterminates $Z^1 , \ldots , Z^p, Y_1,\ldots , Y_m$, where
$p=\dim M'$, $m=\dim M-p$. For each $f$ as above, we set 
$$
F^\ell_f = a^\lambda_{i_1\cdots i_\ell} (0,0) Y_\lambda Z^{i_\ell} \cdots
Z^{i\ell} \in \underbar{R} [Z,Y] 
$$
where $\underbar{R}[Z,Y]$ is the ring of polynomials in $Z^1 , \ldots
, Z^p , Y_1,\ldots ,Y_m$. Denote by $A^\ell$ the ideal in
$\underbar{R}[Z,Y]$ generated by all $F^{\ell'}_f$ where $\ell \geq
\ell'$ and $f$ is function defined on a neighbourhood of $X^\ell$
which is a cross-section of $\Phi^\ell$. Clearly $A^0 \subseteq \cdots
\subseteq A^\ell \subseteq A^{\ell+1} \subseteq \cdots
\underbar{R}[Z,Y]$ being a Noetherian ring there exists on integer
$\ell_2$ such that\pageoriginale $A^{\ell+1} = A^\ell$ for $\ell\geq \ell_2$. This
together with the above lemma means the following: For any
cross-section $f$ of $\Phi^{\ell+1}$ defined on a neighbourhood of
$X^{\ell+1}$ there exists $h_j$ which is a cross section of
$\Phi^\ell$ such that $F^{\ell+1}_f = \sum Z^j
F^{\ell}_{h_j}$. Therefore by the above lemma, it follows that the
principal part of $f-\sum D^j h_j$ vanishes at $X^{\ell+1}$. Since
$X^{\ell+1}$ has the coordinates $x=y= \cdots = y^{i_1\cdots
  i_\nu}_\lambda = \cdots =0$, this means that $d(f-\sum D^j h_j)_X
\ell+1$ is in $\rho^{\ell+1*}_{\ell} (\wedge _x \ell (J^\ell))$. Then
the condition (iii) and (v) imply that $\Phi^{\ell+1}$ is
generated by $\Phi^\ell$ and $D^j (\Phi^\ell)$ locally at
$X^{\ell+1}$. Therefore $\Phi^{\ell+1}$ is equal to $P(\Phi^\ell)$ on
a neighbourhood of $X^{\ell+1}$ for $\ell \ge \ell_2$. 

\section{}\label{chap3:sec3.18}%3.18

Take a sufficiently small open neighbourhood $D^\ell$ of $X^\ell$
(which we will change if necessary), and set $W^\ell = \ell ^0 \Phi
^\ell \cap D^\ell$. $(D^\ell, \alpha (D^\ell),\alpha)$ and $(W^\ell ,
\alpha (D^\ell), \alpha)$ are fibered manifolds by $(iv)$ and
(v). Denote by $\sum_\ell$ the restriction of $\sum (D^\ell ,
\Phi^\ell)$ to $W^\ell$. By (iii) there are $f_1,\ldots ,f_a$ in
$\Gamma (D^\ell, \Phi^\ell)$ such that $df_1,\ldots , df_a$ are
linearly independent mod. $d(\Gamma (D^{\ell-1},\Phi^{\ell-1})o$ 
$\rho^{\ell}_{\ell-1}=0$ is a regular local equation of $W^\ell$ on
$D^\ell$. If $h$ is in $\Gamma (D^{\ell-1},\Phi^{\ell-1})$, then the
definition of $D^jh$ together with (ii) imply that $d(h \circ
\rho_{\ell-1}^\ell)_X \in \prod (\ell)_X$ for any $X$ in
$W^\ell$. Therefore $(\sum(D^\ell ; \Phi^{(\ell)})) ^{(1)}$ is
generated by $(df_1) _{X,\ldots ,}(df_a)_X, (\prod(\ell))_x$. Denote
by $\ell$ the injection of $W^\ell$ into $D^\ell$. Then the
conclusion just reached shows that  
$$
\left({\textstyle \sum^{(1)}_\ell}\right)_X = \ell^*\left(\prod(\ell)\right)_X
$$

Denote\pageoriginale by $\Omega_X$ the subspace of $\wedge^1_X(D)$ generated by
$(dx_1)_{X,\ldots,} (dx_p)_X$. We claim that $\ell^* (\prod(\ell))_X
\cap \ell^* \Omega_X = 0$. Namely, if $\ell^* (\prod (\ell))_X \cap
\ell^* \Omega_X \neq 0$, then there is no integral element $E$ of
$\sum_\ell$ with origin $X$ such that $\dim(d\alpha(E))=p$. This means
that $\ell^0 P^1_S (\sum; (W^\ell, \alpha(D^\ell),\alpha))$ has no
points with origin $X$. By Proposition \ref{chap3:sec3.16:prop26}, it follows that $X\notin
\rho^{\ell+1}_\ell (\ell^0 p(\Phi^\ell))$. By (ii) this contradicts to
(v). Thus $\ell^*(\prod(\ell))\cap \ell^* \Omega = \sum ^{(1)}_\ell
\cap \ell^* \Omega =0$. This shows in particular that $X^\ell$ is a
normal integral point of $\sum_\ell$. 

We will show that $\left[\sum_\ell, (W^\ell, \alpha (D^\ell),
  \alpha)\right]$ is a normal differential system. Since $\prod
(\ell)$ is normal. the conditions (1), (3), and (4) in the
definition of normal system (p.99) is trivial. As for the condition
(2), we already showed that $\sum^{(1)}_\ell \cap \ell^* \Omega
=0$. Since $\prod (\ell)$ is generated by $dy^{i_1\cdots
  i_\nu}_{\lambda} - y^{i_1\cdots i_\ell i}_\lambda dx_i (0 \leq \nu
\leq \ell-1)$, it is clear that $(\prod (\ell))_X +\Omega_X =
(\rho^\ell_{\ell -1})^* \wedge'_X (D^{\ell -1})$. Therefore
$(\sum^{(1)}_{\ell})_x + \ell^* (\Omega_X) = (\rho ^{\ell}_{\ell-1})^*
(\wedge'_X (W^{\ell -1}))$. Hence $\dim(\sum^{(1)}_\ell) _X = \dim
(w^{\ell-1})-p$, which is independent of $X$ in $W^\ell$. This proves
that the condition $(2)$ is satisfied. Thus $\left[\sum , (W^\ell,
  \alpha (D^\ell),\alpha)\right]$ is a normal system.  


We will show that $X^{\ell_2}$ is $P$-weakly-regular with respect
to\break 
$\left[\sum_2,(W^{\ell_2},\alpha (D^{\ell_2}, \alpha)\right]$,where
$\ell_2$ is the integer chosen in \S  \ref{chap3:sec3.17}. Take $z$ in $W^{\ell_2}$
sufficiently near $X^{\ell_2}$. Take an integral point $Y$\pageoriginale of
$P_s^{\ell} \sum\ell_2$ over $z$. $Y$ is in $J^\ell(W^{\ell_2} ,\alpha
(D^{\ell_2}),\alpha) \subseteq J^\ell (J^{\ell_2}
(M,M',\tilde{\omega}),M',\alpha)$. Denote by $\ell$ the canonical
injection of $J^{\ell_2+\ell}(M,M',\tilde{\omega})$ into
$J^\ell(J^{\ell_2 } (M,M',\tilde{\omega},M'\alpha)$. \break Then by Proposition
\ref{chap3:sec3.16:prop26} we can choose $Y$ in such a way that there is $X$ in $W^{\ell_2
  +\ell}$ near $X^{\ell_2+\ell}$ such that $Y=\ell(X)$. By Proposition
\ref{chap3:sec3.6:prop11} and \ref{chap3:sec3.9:prop13}, applied to the case $\dim E=0$, it follows that
$t^{\ell}_0 (z;\sum_{\ell_2}) =\dim ((P^{\ell}_S (\sum_{\ell_2})
^{(1)})_Y + \Omega _Y / \Omega _Y )-c$,where $c$ is a constant
independent of $z$. Therefore , by Proposition
\ref{chap3:sec3.16:prop26} and \ref{chap3:sec3.4:propdash8}$'$, $t_0  ^\ell 
(z; \sum_{\ell_2})=\dim ((\sum (P^\ell \Phi ^{\ell_2})))^{(1)}_X +
\Omega _X / \Omega_X) -c'$. By the choice of and by Proposition
\ref{chap3:sec3.4:propdash8}$'$,
it follows that $t_0^\ell(z; \sum_{\ell_2} = \dim
(\sum^{(1)}_{\ell_2+\ell})_{\bar{X}} c''$, because
$(\sum^{(1)}_{\ell_2+\ell})_X \cap \Omega_X =0$. As is already shown
$\dim (\sum^{(1)}_{\ell_2}+\ell)_X$ is a constant independent of $X$
in $W^{\ell_2 + \ell}$. Hence $t^\ell_0 (X^{\ell_2}$; 
$\sum\ell_2)=t^\ell_0 (\sum\ell_2)$. Let us assume as an induction
assumption that there is a sequence $X^{\ell_2} = E^{\chi}_0 \subset
E^\chi _1 \subset \cdots \subset E^\chi_q$ such that $t^{\ell}_r ((
X^{\ell_2}, E^\chi_r) ; \sum\ell_2)= t^\ell_r (\sum\ell_2)$ for
$r=0,1, \ldots ,q$ and for sufficiently large $\ell$. By Proposition
\ref{chap3:sec3.12:prop22}, there is $E^\chi_{q+1} \supseteq E_q$ such that
$t^{\ell}_{q+1}((X^{\ell_2}; E^\chi _q); \sum_{\ell_2}) = t^\ell_q
((X^{\ell_2} , E^\chi _q )$; $\sum_{\ell_s}) + n_{\ell-1} - t^{\ell-1}
((X^{\ell_2}, E^\chi_q); \sum_{\ell_2})$ for large $\ell$. Hence by
theorem \ref{chap3:sec3.12:thm4} for such $E^\chi_{q+1}$ we have the equality  $t^{\ell}_{q+1}
((X^{\ell_2}, E^\chi_{q+1}); \sum\ell_2) = t^{\ell}_{q+1}
(\sum\ell_2)$ 
for large $\ell$. Thus $X^{\ell_2}$ is $P$-weakly regular of weight
$\ell_1$, for sufficiently large $\ell_1 (\geq \ell_2)$ with respect to
$\sum_{\ell_2}$. By Proposition \ref{chap3:sec3.4:prop8}, a differential system is in
involution if and only if its admissible restriction is in
involution. Therefore by\pageoriginale Theorem \ref{chap3:sec3.14:thm5}, $\Phi^\ell$ is in involution at
$X^\ell$ for $\ell\geq \ell_1$. Thus Theorem \ref{chap3:sec3.16:thm6} is completely
proved. 
 
\begin{thebibliography}{99}\pageoriginale
\bibitem {1} {E. cartan, Les} systems differentials exterieurs et leur
  applications geometriques,Paris, 1945 
\bibitem {2} Sur 1'integration des systems d'equations aux
  differentielles totales, Annales Scientifiques del'Ecole Normale
  Superieure, vol. 18 (1901), pp.241-311. 
\bibitem {3} {E. Kahler} Einfuhrung in die Theorie der Systeme von
  Differentialgleichungen, Hamburger Mathematische Einzerschriften,
  vol. 16 (1934) 
\bibitem {4} {J.A. Schouten and W.v.d. Kulk}, Pfaff's Problem and its
  generalization, Oxford, 1949 
\bibitem {5} {M. Kuranishi,} On E. Cartan's prolongation theorem of
  exterior differential systems, American Journal of Mathematics,
  vol. 79 (1957), pp. 1-47 
\end{thebibliography} 
