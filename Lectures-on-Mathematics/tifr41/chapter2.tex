 \chapter{Categories of modules and their equivalences}\label{chap2}
 % \chapter II  
  
In\pageoriginale this chapter we first characterize (up to
equivalence) categories 
of modules as abelian categories  with arbitrary direct sums and
having a faithfully projective object. Then we show that any
equivalence from the category $A$-mod of left modules over a ring
$A$ into  the category $B$-mod for another ring $B$, is of the form
$P \otimes_A$, where $P$ is a $B$-$A$-bimodule, unique up to
isomorphism. We deduce a number of consequence of the existence of
such an equivalence, and we characterize the modules $P$ that can
arise in this manner. $A$ detailed account of the Wedderburn
structure theory for semi-simple algebras is obtained in this
context, Finally, for an algebra $A$ over a commutative ring $k$,
the study of autoequivalences of $A$-mod leads to the introduction
of a group $\Pic_k (A)$ for a $k$-algebra $A$, which  generalizes
the usual Picard group  $\Pic (k) = \Pic_k(k)$. 
  
  Most of this material is folklore. The main sources are Gabriel \cite{key1}
  and Morita \cite{key1}. I have borrowed a great deal from an unpublished
  exposition of S.Chase and S.Schanuel. 
  

  \section{Categories of modules; faithfully projective modules}%%% 1 
  
  Let $\mathscr{A}$ and $\mathscr{B}$ be two categories. We recall
  that $\mathscr{A}$ and $\mathscr{B}$ are said to be
  \textit{equivalent} if there exist functors $T:\mathscr{A} \to
  \mathscr{B}$ and $S: \mathscr{B} \to \mathscr{A}$ such that $ST$ and
  $TS$ are isomorphic to the identity functors of $\mathscr{A}$ and
  $\mathscr{B}$ respectively. By abuse of language we shall say that
  $T$ is an equivalence. 


  We\pageoriginale call  a functor $T: \mathscr{A} \to \mathscr{B}$
  \textit{faithful} (resp. \textit{full, fully faithful}) if the map 
  \begin{equation*}
T: \mathscr{A} (X, Y) \to \mathscr{B} (TX, TY)  \tag{1.1}  
   \end{equation*}   
   is injective (resp surjective, bijective) for all $X$, $Y \in \obj
   \mathscr{A}$, where $\mathscr{A} (X,Y)$ denotes the set of
   morphisms from $X$ into  $Y$. If $T$ is an equivalence, then
   obviously it is fully faithful; also, given $Y \in \obj
   \mathscr{B}$, there exists $X \in \obj \mathscr{A}$, such that $TX
   \approx Y$. Conversely, these two conditions together imply that
   $T$ is an equivalence. This gives us a 
   
(1.2) \textit{Criterion for equivalence: Let $T : \mathscr{A} \to
     \mathscr{B}$ be a functor satisfying the following conditions:}  
   \begin{enumerate} [(i)]
\item $T$ \textit{is fully faithful}

\item \textit{Given $Y \in  \obj \mathscr{B}$, there exists
  $X \in obj \mathscr{A}$ with $TX \approx Y$.}
   \end{enumerate}   
   \textit{Then $T$ is an equivalence.}

   \begin{proof}
Using (ii) we can choose, for each $Y \in \obj \mathscr{B}$, an $SY
\in \obj \mathscr{A}$ and an isomorphism 
$$
f(Y) :Y  \to TS Y.
$$

These induce bijections $\mathscr{B} (Y,Y') \to  \mathscr{B} (TSY,
TSY')$, and by (i), we have bijections $\mathscr{A}(SY, SY') \to
\mathscr{B} (TSY, TSY')$. The first map, followed by the inverse of
the second, defines a bijection  
$$
S: \mathscr{B}(Y,Y') \to \mathscr{A} (SY, SY').
$$
It is easy to see that $S$, so defined, is a functor satisfying our
requirements. 
   \end{proof}

We\pageoriginale shall now consider abelian categories. We shall
discuss them only 
provisionally, mainly for the purpose of characterizing categories of
modules. Definitions can be found in Gabriel \cite{key1}, Freyd
\cite{key1}, and Mitchell \cite{key1}. 

A functor $T: \mathscr{A} \to \mathscr{B}$ between abelian categories
is called \textit{additive} if the maps (1.1) are homomorphisms. $T$
is \textit{left exact} if it preserves kernels, \textit{right exact}
if it preserves cokernels, and \textit{exact} if it does both. We call
$T$ \textit{faithfully exact} if it is faithful, exact, and
preserves arbitrary direct sums. We shall often call direct sums
\textit{coproducts}, and use the symbol $\coprod$ in place of the
more familiar $\oplus$.  


Let $P$ be an object of the abelian category $\mathscr{A}$. Then 
$$
h^p = \mathscr{A}(P, ~)
$$
defines a functor from $\mathscr{A}$ to the category of abelian
groups. We call $P$ a \textit{generator of} $\mathscr{A}$ if $h^P$ is
faithful, \textit{projective} if $h^P$  is exact, and
\textit{faithfully projective} if $h^P$ is faithfully exact. 

\setcounter{lemma}{2}
\begin{lemma} % lem 1.3
Let $\mathscr{A}$ be an abelian category with arbitrary direct sums.
\begin{enumerate}[(a)]
\item An object $P$ of $\mathscr{A}$ is a generator of
  $\mathscr{A} \Leftrightarrow$ every object of $\mathscr{A}$ is a
  quotient of a direct sum of copies of $P$. 

\item $A$ class of objects of $\mathscr{A}$ which contains a
  generator is suitable under arbitrary direct sums, and which
  contains the co-kernel of any morphism between its members, is the
  whole of obj $\mathscr{A}$. 
\end{enumerate}
\end{lemma}

\begin{proof}
(a). $\Rightarrow$.\pageoriginale Let $X$ be any object of
  $\mathscr{A}$ and let $S 
  = \coprod_{f \in \mathscr{A} (P, X)}P_f$, where $P_f = p$, with
  inclusion $i_f : P \to S$. There is a morphism $F : S \to X$ such
  that $Fi_f = f $ for all $f$. Let $g: X \to$ coker $F$. We want to
  show that $g = 0$, and, by hypothesis, if suffices to show that
  $h^p (g) = \mathscr{A}(P, g) = 0 $. But $h^p (g) (f) = gf = gFi_f =
  0$.  
\begin{enumerate}[(a)]
\item $\Leftarrow$ Suppose $g : X \to Y$ be a non-zero morphism. We
  want $h^p (g) \neq 0$, i.e. $gf \neq 0$ for some $f: P \to
  X$. Choose a surjection $F : S \to X$ with $S =
  \coprod\limits_{i}P_i$, each $P_i = P$. The morphism $F$ is
  defined by a family of morphisms $f_i : P\to X$, and since $gF \neq
  0$, we must have $gf_i \neq 0$ for some $i$.  

\item is a trivial consequence of $(a)$. 
\end{enumerate}
\end{proof}

The theorem below gives a characterization of categories of
modules. We shall denote by  
$$
A - \mod \text{(resp}. \mod - A) 
$$
the category of left (resp. right) modules over a ring $A$. 

\setcounter{theorem}{3}
\begin{theorem}[See Gabriel \cite{key1} of Mitchell \cite{key1}]%theo 1.4
Let $\mathscr{A}$ be an abelian category with arbitrary direct
sums. Suppose $\mathscr{A}$ has a faithfully projective object
$P$. Let $A = \mathscr{A}(P,P)$. Then  
$$
h^p = \mathscr{A}(P. ~ ) : \mathscr{A} \to \mod - A 
$$
is an equivalence of categories, and $h^p(P) = A$.
\end{theorem}

\begin{proof}
Clearly $h^p(P) = A$, and since $h^p$ is faithful, 
\begin{equation*}
h^p : \mathscr{A} (X, Y) \to \Hom_A (h^pX , h^pY)
\tag{1.5}\label{chap2-eq1.5} 
\end{equation*}
is\pageoriginale a monomorphism. Using the criterion for equivalence
(1.2), it remains to show that   
\begin{enumerate}[(i)]
\item $h^p$ is full (that is, that \eqref{chap2-eq1.5} is surjective), and 

\item each $A$-module is isomorphic to some $h^pX$. 
\end{enumerate}
\end{proof}

For $X = P$ we see easily that \eqref{chap2-eq1.5} is the standard isomorphism
$h^p(Y) \to \Hom_A (A, h^P Y)$. As contravariant functors in $X$,
the two side of \eqref{chap2-eq1.5} are both left exact and convert direct sums into
direct products. This follows for the functor on the right, because
$h^p$ is faithfully exact. It follows from these remarks and the
5-lemma that the collection of $X$ for which \eqref{chap2-eq1.5} is an isomorphism
satisfies the hypothesis of (1.3)(b), and hence is the whole of obj
$\mathscr{A}$. This proves (i).  

If $M$ is an $A$-module, there is an exact sequence $F_1
\xrightarrow{d}F_0 \to M \to 0$ with $F_i$ free. Up to isomorphism we
can write $F_i = h^P G_i$, with $G_i$ a direct sum of copies of $P$. By
(i), we can write $d = h^P g $ for some $s : G_1 \to G_0$. Then, from
exactness, $M \approx$ coker $h^P g \approx h^P$ coker $g$.  

\setcounter{prop}{4}
\begin{prop}\label{chap2:prop1.5}%prop 1.5
Let $P$ be a right module over a ring $A$. The following statements
are equivalent: 
\begin{enumerate}[(i)]
\item $P$ is faithfully projective. 

\item $P$ is finitely generated, projective, and is a generator of
  $\mod A$.  
\end{enumerate}
generator of $\mod - A$.
\end{prop}

\begin{proof}
In view of the definition of faithful projectivity, we
have\pageoriginale only to 
show if $P$ is {\em projective}, then $P$ is finitely generated if and
only the functor $\Hom_A(P, ~ )$ preserves coproducts.  
\end{proof}

Suppose $P$ is finitely generated. Any homomorphism of $P$ into a
coproduct has its image in a finite coproduct (a finite number of
factors is enough for catching the non-zero coordinates of the images
of a finite system of generators of $P$). Thus such a homomorphism is a
(finite) sum of a homomorphisms of $P$ into the factors.  

Conversely, suppose $\Hom_A(P,~)$ preserves coproducts. Consider a
homomorphism $e : P \to \coprod \limits_i A_i$ (each $A_i = A$) with a
left inverse (such a map exists since $P$ is projective). By
hypothesis, $e$ is a finite sum of homomorphisms $e_i : P \to A_i$,
$i \in S$, $S$ a finite set. Thus $P$ is a direct summand of
$\coprod\limits_{i\in S} A_i$ and hence finitely generated.  

\begin{remark*}%rema 
If $P$ is not projective, then finite generation is no longer
equivalent with $\Hom_A (P, ~)$  preserving coproducts. For, we have
obviously,  
\end{remark*}
\begin{itemize}
 \item [(1.6)] $P$ is finitely generated $\Leftrightarrow$ the proper
   submodules of $P$ are inductively ordered by inclusion. 

On the other hand

\item [(1.7)]  {\em  $\Hom_A (P,~ )$ preserves coproducts
  $\Leftrightarrow$ the union of any ascending sequence of proper
  submodules of $P$ is a proper submodule.}  
\end{itemize}


If\pageoriginale $P$ is the maximal ideal of a valuation ring, where
the value group has a suitably pathological order type, then $P$ will
satisfy (1.7) but not (1.6).  

\noindent
\textbf{Proof of (1.7)} $\Leftarrow$. If $f : P \to \coprod \limits_{
  i \in I}M_i$  is a homomorphism such that $f(P)$ is not contained in
a finite direct sum of the $M'_is$, then we can choose a countable
subset $J$ of $I$ such that if $g : \coprod\limits_{i \in I} M_i \to
\coprod\limits_{j \in J}M_j$ is the projection, then $gf(P)$ is like wise
not in a finite sum. Letting $S$ expand through a sequence of finite
subsets of $J$, with $J$ as their union, we find that the submodules
$(gf)^{-1}(\sum\limits_{j \in S}M_j )$ violate the assumed chain
condition on $P$.  

$\Rightarrow$. Suppose $P_1 \subset P_2  \subset \cdots \subset P_n
\subset \cdots$ are proper sub - modules of $P$ with $\bigcup\limits_{
  n \geq 1} P_n = P$. The projections $f_n: P \to P/ P_n$ define a map
of $f : P \to \prod\limits_n P / P_n$, whose image is clearly in
$\coprod P / P_n$, but not in a finite sum of the $P/ P_n$. 

\section{$k$-categories and $k$-functors} % section 2 

Let $A$ be a ring and let $M$ be a right $A$-module. For an element $a
\in $ centre $A$, the homothetie $h(a)_M: M \to M$ (defined by
$h(a)_M (x) = xa$) is A-liner. These homomorphisms define an
endomorphism of the identity functor $\Id_{{\rm mod}-A}$.  

\begin{prop}\label{chap2:prop2.1}% proposition 2.1
The homothetie map 
$$
h : \text{ centre } A \to \text{ End } (\Id_{{\rm mod} - A})   
$$
is an isomorphism of rings.
\end{prop}

\begin{proof}
If\pageoriginale $h(c) = 0 $, then $c = h (c)_A (1) = 0 $. Let $f$ be an
endomorphism of the functor $\Id_{{\rm mod} - A}$. $f_A$ is the left
multiplication in $A$ by $c = f_A (1)$. The element $c$ belongs to the
centre of $A$. This follows from the fact that $f_A$ commutes with all
left multiplications in $A$, since $f$ is a natural
transformation. Set $f' = f - h(c)$. We shall show that $f' = 0$. Let
$M$ be a right $A$-module. For an $x \in M$, consider the $A$-linear map
$t : A \to M$ given by $t(a) = xa$. We have $f'_M \circ t = t \circ f'_A$. It
follows that $f'_M(x) = 0 $. Thus $f' = 0 $.  
\end{proof}

The proposition suggests the definition 
$$
\text{ centre }\mathscr{A} = \text{ End } (\Id_{\mathscr{A}})  
$$
for any abelian category $\mathscr{A}$. Let $k$ be a commutative ring
and $k \to$ centre $\mathscr{A}$ a homomorphism. This converts
the $\mathscr{A}(X, Y)$ into $k$-modules so that the composition is
$k$-bilinear. Conversely, given the latter structure, we can clearly
reconstruct the unique homomorphism $k \to \text{ centre }\mathscr{A}$
which induces it. An abelian category $\mathscr{A}$ with a
homomorphism $k \to \text{ centre }\mathscr{A}$ will be called a $k$-
\textit{category}. A functor $T : \mathscr{A} \to \mathscr{B}$ between
two such categories will be called a $k$-\textit{functor} if the maps
(1.1) are $k$-linear. The $k$-functors forms a category, which we shall
denote by $k$-Funct ($\mathscr{A}, \mathscr{B}$).  

If $A$ is a $k$-algebra, then by virtue of (2.1), mod-$A$ is a
$k$-category. Let $A$ and $B$ be $k$-algebras and suppose $M$ is a left
$A-$, right $B$-module. If $B$-module. If $t \in k$ and $x \in M$, then
$tx$ and $xt$ are both defined. The following statement is easily
checked:


(2.2) \quad $tx = xt$\pageoriginale for  all $t \in k$, $x \in <
\Leftrightarrow \otimes_A 
M : \mod - A \to \mod - B$ is a $k$-functor. 

This condition simply means that $M$ can be viewed as left module over
$A \underset{k}\otimes B^0$. We will often follow the Cartan-Eilenberg
convention of writing $~_A M_{B}$ to denote the fact that $M$ is left
$A-$, right $B$-bimodule, and when a ground ring $k$ is fixed by the
context, it will be understood that $M$ satisfies (2.2).  

\setcounter{prop}{2}
\begin{prop}\label{chap2:prop2.3}% proposition 2.3
$h(~_AM_{B})  \otimes_A M : \mod - A \to \mod - B$ defines a fully
  faithful functor  
$$
h : (A \otimes _k B^0)- \mod \to k-\text{Funct} (\mod-A, \mod-B).  
$$
In particular, $~_A M_B \approx _A N_B$ as bimodules $\Leftrightarrow
\otimes_A M \approx  \otimes_A N$ as functors from $\mod - A$ to
$\mod - B$. 
\end{prop}

\begin{proof}
If $f : ~_A M_B \to ~_AN_B$ is a bimodule homomorphism, then $h(f) =
\otimes_Af$ is a morphism of functors. Thus $h$ is a functor. If
$h(f)= 0$, then $1_A \otimes_A f = h(f) (1_A) = 0$, i.e, $f = 0$. So
$h$ is faithful.  
\end{proof}

Suppose $t : hM \to hN$ is a natural transformation. We will conclude
by showing that $t = h(f)$, where $f$ is the unique $B$-morphisms
rendering 
\[
\xymatrix{
M \ar[r]^f \ar[d]_{\approx} & N \ar[d]^{\approx}\\
A \otimes_A M \ar[r]_{t_A} & A \otimes_A N
}
\]
commutative. The vertical maps are bimodule
isomorphisms. Since\pageoriginale left 
multiplications in $A$ are right $A$-linear, $t_A$ must respect it, by
naturality. Thus $t_A$, and hence also $f$, is a bimodule homomorphism,
so $h(f)$ is defined. Let $ s = t - h(f)$ : $h M \to hN$. The class
$\mathscr{C}$ of $X$ in $\obj \mod - A$  for which $s_X = 0$ contains
$A$. Since $hM$ and $hN$ are right exact and preserve coproducts, it
follows from (1.3) $(b)$, that $\mathscr{C} = \obj \mod - A$.  


\section{Right continuous functors} % Section 3

We will here describe the image of the functor of proposition
\ref{chap2:prop2.3}. Functors of the type $\otimes _A M : \mod- A \to
\mod -B$ are (i) 
right exact, and (ii) preserve arbitrary coproducts. It follows that
they also preserve direct limits. A functor satisfying (i) and (ii)
will be called \textit{right continuous.} The next theorem says that
they are all tensor products.  


\begin{theorem}[Eilenberge-Watts] \label{chap2:thm3.1}% Theorem 3.1
 The correspondence $~_A M_B \mapsto \otimes_A M$
  induces a bijection from the isomorphism classes of left $A \otimes
  _k B^0$- modules to the isomorphism classes of right continuous
  k-functors from $\mod - A$ to $\mod - B$. In the situation $(~_A
  M_B, ~_BN_C)$, $_A(M \otimes_B N)_C$ corresponds to the composite
  of the respective functors.  
\end{theorem}

\begin{proof}
The last statement follows from 
\begin{align*}
(( \otimes_B N) \circ  (\otimes_A M)) (X) & = (\otimes_B N) (X
  \otimes_A M) \\ 
& = (X \otimes_A M) \otimes_B N \\
& =  X \otimes_A (M \otimes_BN) \\
& = \otimes_A (M \otimes_B N) (X). 
\end{align*}
Injectivity\pageoriginale is just the last part of proposition
\ref{chap2:prop2.3}.  
\end{proof}

Let $T: \mod- A  \to \mod - B$  be a right continuous $k$-functor. The
composite  
$$
A \to \Hom_A (A, A) \to \Hom _B (TA, TA), 
$$
where the first map is given by left multiplications, is a
homomorphism of $k$-algebras. This makes $M = TA$ into a left $A
\otimes_k B^0$ - module. We will conclude by showing that the
functors $T$ and $\otimes_A M$ are isomorphic. If $X$ is a right
$A$-module, we have maps  
$$ 
X \xrightarrow{\approx} \Hom_A (A, X ) \xrightarrow{T} \Hom_B (TA, TX) =
\Hom_B (M, TX),  
$$
and the composite $f_X$ is $A$-linear (for the action of $A$ on $M$ just
constructed). Now, there is a canonical isomorphism 
$$
\Hom_A (X, \Hom_B (M, TX)) \approx \Hom_B (X \otimes_A M, TX),
$$
and $f_X$ is an element of the first member. Let $g_X$ be the
corresponding element in the second member. The homomorphisms $g_X$
define a natural transformations of functors $g : \otimes_A M \to
T$. For $X = A$, we have $g_A$ as the obvious isomorphism $A \otimes
_A M \to TA = M$. Using the right continuity of $T$ and $\otimes_A M$,
we now see that the class of objects $x$ for which $g_x$ is an
isomorphism, satisfies the conditions of (1.3) $(b)$. Thus $g$ is an
isomorphism of functors.  

\setcounter{definition}{1}
\begin{definition}\label{chap2:def3.2}% definition 3.2
We shall call a bimodule $~_AM_B$ {\em invertible}, if the functor
$\otimes_A M : \mod - A \to \mod -B$ is an equivalence.  
\end{definition}

This\pageoriginale equivalence is evidently right continuous (indeed, any
equivalence is). It therefore follows from theorem
\ref{chap2:thm3.1}. that the 
invertibility of $M$ is equivalent to the existence of a bimodule
$~_BN_A$ such that $M \otimes_B N \approx A$ and $N \otimes_A M
\approx B$ as bimodules (over appropriate rings). This shows that the
definition of in vertibility is left-right symmetric. In particular,
$M \otimes_B : B - \mod \to A - \mod$ is also equivalence.  


\section{Equivalences of categories of modules} % Section 4

We have just seen that an equivalence is, up to isomorphism, tensoring
with an invertible bimodule. We now summarize.  

\begin{prop}\label{chap2:prop4.1}% proposition 4.1
Let $A$ and $B$ be a $k$-algebras and suppose  
$$
\mod - A^{\xrightarrow{T}}_{\xleftarrow[S]{~}} \, {\rm mod} - B
$$
are $k$-functors such that $ST$ and $TS$ are isomorphic to the identity
functors of $\mod-A$ and $\mod-B$ respectively. Set $P = TA$ and $Q =
SB$. Then we are in the situation $(~_A P_B, ~_BQ_A)$, and :  
\begin{enumerate}[(1)]
\item $T \approx \otimes_A P$, and $S \approx \otimes_B Q$.
\item There are bimodule isomorphisms 
$$
f : P \otimes _B Q \to A \text{ and } g : Q \otimes _A P \to B. 
$$

\item $f$ and $g$ may be chosen to render the diagrams  
\[
\vcenter{\xymatrix{
P \otimes_B Q \otimes_A P \ar[r]^{f\otimes 1_P} \ar[d]_{1_P \otimes g}
& A \otimes_A P \ar[d] \\
P \otimes_B B \ar[r] & P
}}
\quad \text{and} \quad 
\vcenter{
\xymatrix{
Q \otimes_A P \otimes_B Q \ar[r]^{g \otimes 1_Q} \ar[d]_{1_Q \otimes
  f} & B \otimes_B Q \ar[d] \\
Q \otimes_A A \ar[r] & Q
}}
\]
commutative. 
\end{enumerate}
\end{prop}

\begin{proof}
Statements\pageoriginale (1) and (2) follow immediately from theorem
\ref{chap2:thm3.1}, since an equivalence is automatically right
continuous. To prove 
the statement (3), we first note that all the intervening maps are
isomorphisms of bimodules. If $ a : A \otimes_A P \to P$ and $ b : P
\otimes_B B \to P$ are the natural maps, then we have $b(1 \otimes g)
= ua(f \otimes 1 )$ for some $A -B$ - automorphism $u$ of $P$. In
particular, $u \in \Hom_B (P, P) = \Hom_B(TA, TA) \approx \Hom_A(A, A) =
A$. So $u$ is a left multiplication by a unit in $A$, which we shall
denote by the same letter $u$. Since $u$ is also an $A$-homomorphism,
we must have $u \in \text{ centre }A$. Now, evidently $u a = a (u
\otimes 1_P)$. So if we replace $f$ by $uf $ we have made the first
square commutative. Assume that this has been done.  
\end{proof} 

Write $f(p \otimes q) = pq $ and $g(q \otimes p) = q p $ for $p \in
P$, $q \in Q$. We have arranged that $(pq)p' = p(qp')$, and we will
prove that the desired equality $(qp)q' = q(pq')$ follows
automatically. For, if $p$, $p' \in P$, $q$, $q' \in Q$ we have  
\begin{align*}
((qp)q')p' & = (qp) (q'p') \qquad (g \text{ is left $B$-linear})\\
& = q(p(q'p'))\qquad (g \text{ is right $B$-linear})\\
& = q((pq')p')\qquad (\text{ by assumption})\\
& = (q(pq')p')\qquad (q \otimes a p = qa \otimes p, a \in A).
\end{align*}
Hence, if $d = (qp) q' - q (pq')$, then $dp'=0$ for all $p' \in
P$. Let $h : A \to Q$ be defined by $h(a) = da$. Then $h \otimes 1_p :
A \otimes_A P \to Q \otimes _A P$ followed by the isomorphism $g$ is
zero. So $h \otimes 1_P = 0$. But $\otimes _A P$ is a fully faithful
functor. Therefore $h = 0$, that is $d = 0$.  

\setcounter{definition}{1}
\begin{definition}\label{chap2:def4.2}% Definition 4.2
{\em A set\pageoriginale of pre-equivalence data $(A, B, C, P, f,
  g)$}
 consists of $k$-algebras $A$ and $B$, bimodules $~_A P_B$ and
 $~_BQ_A$, bimodule homomorphisms   
$$
f : P \otimes_B Q \to A \text{ and } g: Q \otimes_A P \to B, 
$$
which are ``associative'' in the following sense: Writing $f(p \otimes
q) = pq$ and $g(q \otimes p) = qp$, we require that   
$$
(pq)p' = p (qp') \text{ and } (qp)q' = q (pq') \; p, p' \in P, q, q'
\in Q. 
$$
We call it a {\em set of equivalence data} if $f$ and $g$ are isomorphisms. 
\end{definition}

\setcounter{theorem}{2}
\begin{theorem}\label{chap2:thm4.3}% theorem 4.3
Let $(A, B, P, Q, f, g)$ be a set of pre-equivalence data. If $f$ is
surjective, then  
\begin{enumerate}[(1)]
\item $f$ is an isomorphism 

\item $P$ and $Q$ are generators as $A$-modules 

\item $P$ and $Q$ are finitely generated and projective as $B$-modules. 

\item $g$ induces bimodule isomorphisms 
$$
P \approx \Hom_B (Q, B) \text{ and } Q \approx \Hom_B(P, B)
$$

\item The $k$-algebra homomorphisms 
$$
\Hom_B (P, P) \leftarrow A \to \Hom_B (Q, Q)^0
$$
induced by the bimodule structures, are isomorphisms. 
\end{enumerate}
\end{theorem}

\begin{proof}
The hypothesis on $f$ means that we can write 
$$
1 = \sum p_i q_i \text{ in }A. 
$$
\begin{enumerate}[(1)]
\item Suppose\pageoriginale $\sum p'_j \otimes q'_j \in \ker f$. Then  
\begin{align*}
\sum p'_j \otimes q'_j & = \sum_{j , i} (p'_j\otimes q'_j) p_iq_i =
\sum_{j,i} p'_j \otimes ((q'_j p_i) q_i)=\\
& = \sum_{j , i} (p'_j(q'_jp_i)) \otimes q_i =  (\sum_{j, i} (p'_j q'_j)
(p_i \otimes q_i ) \\ 
& = (\sum_j p'_j q'_j)(\sum_i p_i q_i) = 0, \text{ since } \sum p'_j
q'_j = 0.    
 \end{align*} 

 \item We have $A$-linear maps $h_i : P \to A_i$ given by $h_i (p) =
   pq_i $. These define an $A$-linear map $h : \coprod\limits_i P_i
   \to A$ (each $P_i = P$), which is surjective. It follows by (1.3)
   (a), that $P$ is a generator of $A-\mod$, since $A$ is so. The
   argument for $Q$ is similar.  

 \item Define $ P^{\xrightarrow{e}}_{\xleftarrow{h}} \coprod
   \limits_i B_i$ (each $B_i = B$), by $e(p) = (q_i p)$ and $h((b_i))
   = \sum p_i b_i $. Then $he(p) = \sum_i p_i (q_i p) = (\sum\limits_i
   p_i q_i )p = p $. Thus $P$ is finitely generated and
   projective. Similarly $Q$ also is finitely generated and
   projective.  

 \item $g$ induces an $A$-$B$-bimodule homomorphism $h: P \to \Hom_B
   (Q, B)$, given by $h(p)(q) = qp$. If $h(p) = 0$, then  $p =
   \sum\limits_i (p_i q_i) p = \sum\limits_i p_i\break (q_i p) = 0$. If $f :
   Q \to B$ is $B$-linear, then $f(q) = f(\sum\limits_i q(p_i q_i)) =
   f(\sum\limits_i (qp_i)q_i) = \sum_i (qp_i)f (q_i) =
   \sum\limits_{i}q(p_if (q_i))$, so $f = h (\sum\limits_i p_i f
   (q_i))$. Similarly $Q \approx \Hom_B (P,B)$.  

 \item Define\pageoriginale $h : A \to \Hom_B (P, P)$ by $h(a)p =
   ap$. If $h(a) = 
   0$, then $a = \sum\limits_i a(p_i q_i ) = \sum\limits_{i} (ap_i
   )q_i = 0$. If $f : P \to P $ is a $B$-linear, then $f(p) = f
   (\sum\limits_i (p_i q_i)p) = f(\sum\limits_i p_i (q_i p)) =
   (\sum\limits_i f(p_i) q_i)p$, so
   that $f = h (\sum\limits_i f(p_i)q_i)$. Similarly $A \approx \Hom_B
   (Q, Q)^0$ via right multiplication.  
\end{enumerate}
\end{proof}

\begin{theorem}\label{chap2:thm4.4}% theorem 4.4
Let $(A, B, P, Q, f, g)$ be a set of equivalence data (see definition
\ref{chap2:def4.2}). Then  
\begin{enumerate}[(1)]
\item The functors $P \times_B, \otimes_A P, Q \otimes_A$, and
  $\otimes_B Q$ are equivalences between the appropriate categories
  of $A$-modules and $B$-modules.  

\item $P$ and $Q$ are faithfully projective both as $A$-modules and
  $B$-modules.  

\item $f$ and $g$ induce bimodule isomorphisms of $P$ and $Q$ with each
  others duals with respect to $A$ and to $B$.  

\item The $k$-algebra homomorphisms 
$$
\Hom_{B}(P, P) \leftarrow A \rightarrow \Hom_B (Q, Q)^0 
$$
and
$$
\Hom_A (P, P)^0 \leftarrow B \rightarrow \Hom_A (Q, Q),
$$
induced by the bimodule structures on $P$ and $Q$, are isomorphisms.  

\item The bimodule endomorphism rings of $A, B, P$ and $Q$ are all
  isomorphic to the centres of $A$, $B \mod-A$ and $\mod - B$.  

\item The\pageoriginale lattice of right $A$-ideals is isomorphic, via
  $\mathscr{U} 
  \mapsto \mathscr{U}P$, with the lattice of $B$-submodules of $P$, the
  two sided ideals corresponding to $A - B$-submodels, or equivalently,
  to fully invariant $B$-submodules. Similar conclusions apply with
  appropriate permutations of $(A, B)$, $(P, Q)$, (left, right). In
  particular, by symmetry, $A$ and $B$ have isomorphic lattices of
  two-sided ideals.  
\end{enumerate}
\end{theorem}

\begin{proof}
(1) is immediate.
 
(2), (3) and (4) follow immediately from (2), (3), (4) and
(5) of theorem \ref{chap2:thm4.3}. 

We have isomorphisms 
\begin{align*}
\text{centre }\quad & A \approx \Hom_{A - A} (A,
A)\xrightarrow[\approx]{\otimes_AP} \Hom_{A - B} (P, P), \\ 
\text{ centre }\quad  & B \approx \Hom_{B - B} (B,
B)\xrightarrow[\approx]{P\otimes_B} \Hom_{A - B} (P, P),  
\end{align*}
and similarly for $Q$ also. The statement (5) follows from these
isomorphisms plus proposition \ref{chap2:prop2.1}.  
\end{proof}

We now prove (6). Since $P$ is $A$-projective, the canonical map
$\mathscr{U} \otimes_A P \to \mathscr{U}P$ is an isomorphism. That
$\mathscr{U} \mapsto \mathscr{U} P$ is an isomorphism of the lattice
of right ideals of $A$ onto the lattice of $B$-submodules of $P$, now
follows from the fact that $\otimes_A P : \mod- A \to \mod - B$ is an
equivalence. The fully invariant right $A$-submodules of $A$, i.e., the
two-sided ideals of $A$, correspond to the fully invariant
$B-$submodules of $P$, which, by virtue of (4), are just the $A - B$-
submodules of $P$. 

The\pageoriginale remaining assertions in (6) are clear. The
isomorphism between the 
lattices of two-sided ideals of $A$ and $B$ can be made explicit:
$\mathscr{U} \leftrightarrow \mathfrak{b}$ if $\mathscr{U} P =
P\mathfrak{b}$, where $\mathscr{U}$ and $\mathfrak{b}$ are two-sided
ideals in $A$ and $B$ respectively. The conclusion above show that
given $\mathscr{U}$, the ideal $\mathfrak{b}$ exists and is unique.  

\section{Faithfully projective modules}% section 5

Let $B$ be a $k-$alegbra and let $P$ be right $B$-module. From $B$ and
$P$ we will construct a set of pre-equivalence data and then
determine in terms of $B$ and $P$ alone, what it means for them to be
equivalence data.  

We set
$$
A = \Hom_B (P, P), 
$$
and 
$$
Q = \Hom_B(P, B). 
$$
Then $A$ is a $k-$algebra and $P$ is an $A - B$-bimodule, that is, a
left $A \otimes_k B^0$-module. Moreover, $Q$ is a $B-A$ - bimodule
with the following prescription:  
\begin{equation*}
(bq)p = b(qp)\tag{5.1}
\end{equation*}
and 
\begin{equation*}
(qa)p = q (ap), \tag{5.2}
\end{equation*}
$a \in A$, $b \in B$, $p \in P$, $q \in Q$. Next we define $pq \in A$
for $p \in P$ and $q \in Q$, by requiring that  
\begin{equation*}
 (pq) p' = p (qp'), \qquad  p' \in P\tag{5.3}
\end{equation*}
This permits us to define a homomorphism of $A - A$-bimodules  
$$
f_p : P \otimes _B Q \to A, \text{ by } f_P(p \otimes q)  = pq,
$$
and\pageoriginale a homomorphism of $B - B$-bimodules 
$$
g_p : Q \otimes _A P \to B, \text{ by }g_p (q \otimes p ) = qp.
$$
Finally, we claim that 
\begin{equation*}
(qp) q' = q (pq'), \tag{5.4}
\end{equation*}
for $p \in P$, $q$, $q' \in Q$. Since these are linear maps $P \to B$,
we need only show that they have the same value at any $p' \in P$. But  
\begin{align*}
((qp)q')p' & = (qp) (q' p') \tag*{\text{by }(5.1)} \\
& = q(p(q' p')) \tag*{\text{by B - linearity of q}}\\
 & = q((pq')p') \tag*{\text{by } (5.3)}\\
 & = (q(pq')p') \tag*{\text{by } (5.2).}
\end{align*}

We have now proved 

\setcounter{prop}{4}
\begin{prop}\label{chap2:prop5.5}% proposition 5.5
Let $B$ be a $k$-algebra, $P$ a right $B$-module, and $f_p$ and $g_p$
be as constructed above. Then  
$$
( \Hom_B (P, P), B, P, \Hom_B (P, B), f_p. g_p)
$$
is a set of pre-equivalence data. 
\end{prop}

\begin{example*} %Exam 0
Let $P = eB$, where $e$ is an idempotent. Then $B = P \oplus (1
-e)B$. Any $B$-linear map $ f: P \to B$ can be extended to a $B-$linear
map $\bar{f} : B \to B$ by setting $\bar{f} (1 - e) = 0$. Thus we have
inclusions $\Hom_B (P, P) \subset \Hom_B(P, B) \subset \Hom_B(B, B)$. With 
this identification, $\Hom_B (P, P) = eBe $ and $\Hom_B (P, B) = Be$.  
\end{example*}

\begin{prop}\label{chap2:prop5.6}% proposition 5.6
In\pageoriginale the notation of proposition \ref{chap2:prop5.5}: 
\begin{enumerate}[(a)]
\item $ \im f_P = \Hom_B (P, P) \Leftrightarrow P$ is a finitely
  generated projective $B$-modu\-le, in which case $f_p$ is an
  isomorphism.  

\item $\im g_p = B \Leftrightarrow P$ is a generator of $\mod -B $, in
  which case $g_p$ is an isomorphism. 

\item $(\Hom_B (P, P)$, $B, P, \Hom_B (P, B)$, $f_p$, $g_p)$ is a set
  of equivalence data $\Leftrightarrow P$ is faithfully projective. 
\end{enumerate} 
\end{prop}

\begin{proof}
(c) follows from (a), (b) and proposition \ref{chap2:prop1.5}.

In view of theorem \ref{chap2:thm4.3}, it remains only to show the implications
$\Leftarrow$ in (a) and (b).  

Suppose $P$ is a finitely generated projective $B-$module. We can find
a free $B$-module $\coprod_{e_i} B$ with a basis $e_1, \ldots, e_n$,
and $B-$linear maps $P \xrightarrow{h_1} \coprod e_i B
\xrightarrow{h_2}P$ such that $h_2 h_1 = 1_P$. If $q_i : P \to B$
denotes the composite of $h_1$ and the $i^{\rm th}$ coordinate linear
form on $\coprod e_i (B)$, we can write $h_1 (p) = \sum e_i (q_i
p)$. Let $p_i = h_2 (e_i)$. Then $p = h_2h_1 p = h_2 (\sum e_i (q_i
\; p)) = \sum 
p_i (q_i p) = (\sum p_i q_i ) p$. So $1_p = \sum p_i q_i \in \im f_p$,
and the latter is a two-sided ideal in $\Hom_B (P, P)$. Hence  $\im f_p
= \hom_B (P. P)$.  

Next, suppose $P$ is a generator of $\mod - B$. Then $B$ is a quotient
of a sum (which we mau take finite) of copies of $P$. This means that
we can find $q_i \in \Hom_B (P, B)$ such that $\sum q_i P = B$. Hence
$g_p$ is surjective. 
\end{proof}

\setcounter{lemma}{6}
\begin{lemma}\label{chap2:lem5.7}% lemma 5.7
A right\pageoriginale $B-$module $P$ is projective $\Leftrightarrow$
there exist 
$p_i \in P$, $q_i \in \Hom_B(P, B)$, $i \in I$, such that 
\begin{enumerate}[(i)]
\item given $p \in P$, $q_i p = 0$ for almost all $i$, and 

\item $\sum_i p_i (q_i p) = p$, $p \in P$.
\end{enumerate}
 
The family $(p_i)$ which arise in this manner are precisely the
generating systems of $P$. If $\mathscr{U} = \im g_p$, then
$\mathscr{U}$ is generated, as a two-sided ideal, by the $q_i
p_j$. Moreover, $ P \mathscr{U} = P$ and $\mathscr{U}^2 =
\mathscr{U}$. 
\end{lemma}

\begin{proof}
Projectivity of $P$ is equivalent to the existence of a free
$B$-module $\coprod_{i \in I} e_i B$ and $B$-linear maps $P
\xrightarrow{h_1} \coprod\limits_{i \in I} e_i B \xrightarrow{h_2} P$
such that $h_2 h_1 = 1_P$. The latter condition, in turn, is
equivalent to the existence of the $p_i$ and $q_i$. For, given $p_i$
and $q_i$, one can construct $h_1$ and $h_2$ in an obvious fashion. On
the other hand, given $h_1$ and $h_2$, we can take $p_i$ to be
$h_2(e_i)$, and $q_i$ to be the composite of $h_1$ with the $i^{\rm th}$
coordinate linear form on $\coprod\limits_{ i \in I} e_i B$. If $P$
is projective, it is clear that families $(p_i)$ are precisely the
systems of generators for $P$.  
\end{proof}

Setting $Q = \Hom_B (P, B)$ we can write $\mathscr{U} = QP$ (the set
of sums of elements of the form $qp$, $q \in Q$, $p \in P$). But $qp=
q \sum\limits_i p_i (q_i p) = \sum\limits_{ i, j} q (p_j(q_j p_i))
(q_i p) = \sum\limits_{i, j}(qp_j)(q_j p_i) (q_i p)$, which shows
that $\mathscr{U}$ is generated, as a two-sided ideal, by the $q_j
p_i$. Moreover (ii) shows that $P = P \mathscr{U} = P Q P$, and
therefore $\mathscr{U}= QP = QPQP = \mathscr{U}^2$.  

\begin{lemma}% lemma 5.8
Let $B$\pageoriginale be a commutative ring, $M$ a finitely generated
$B$-modu\-le, and $\mathscr{U}$ an ideal of $B$ such that $M \mathscr{U}
= M$. Then $M(1 - a  ) = 0$ for some $a \in \mathscr{U}$.   
\end{lemma}

\begin{proof}
If $x_1, \ldots, x_n$ generate $M$, we can find $a_{ij}
\in\mathscr{U}$ such that $x_i = \sum\limits_j x_i a_{ij} $, that is,
$\sum\limits_j x_i (\delta_{ij} - a_{ij}) = 0$, $i = 1, \ldots, n$. It
follows by a well-known argument, that  $x_i \det ( \delta _{ij} -
a_{ij} ) = 0$, that is, $M \det (\delta_{ij} - a_{ij} ) = 0$. But
$\det (\delta_{ij} - a_{ij})$ is of the form $1- a$ for some $a \in
\mathscr{U}$. 
\end{proof}

\setcounter{prop}{8}
\begin{prop}% proposition 5.9
Let $B$ be a commutative ring and $P$ a projective $B$-modu\-le. If
either $B$ is noetherian or $P$ is finitely generated, the ideal im
$g_p$ of $B$ is generated by an idempotent $e$, and ann $P = (1
-e)B$. Hence $P$ is a generator of $\mod - B$ if and only if $P$ is
faithful ( i.e., $ann P = 0$). 
\end{prop}

\begin{proof}
The hypotheses guarantee that $\mathscr{U} = im g_p$ is a finitely
generated ideal of $B$, using (5.7) in the second alternative. From
(5.7) we also have $P \mathscr{U} = P$ and $\mathscr{U}^2 =
\mathscr{U}$. Taking $M = \mathscr{U}$ in (5.8) we find an $e \in
\mathscr{U}$ such that $\mathscr{U} (1 - e) = 0$. So $\mathscr{U} =
\mathscr{U} e$ and $e^2 = e$. Moreover, $P = Pe$ so $ P(1 - e)=0$. If
$Pa = 0$, then, since $e = \sum q_j p_j$, we have $ea = \sum q_j p_j a
= 0$ and thus $a = (1 - e)a$. Hence ann $P = (1 - e)B$. Finally, $P$
is a generator $\Leftrightarrow \im \, g_p = B \Leftrightarrow e = 1
\Leftrightarrow ann P =  0$.  
\end{proof}

The following corollary shows that for a \textit{commutative} ring,
$B$, the concept of a faithfully projective object of $\mod - B$ is
the same as that of faithfully projective $B$- modules (as defined in
\S \ref{chap1:sec6} of Chapter \ref{chap1}).  

\setcounter{coro}{9}
\begin{coro}\label{chap2:coro5.10}% corollary 5.10
Let\pageoriginale $P$ be a module over a commutative ring, $B$. Then
$P$ is a faithfully projective object of $\mod - B \Leftrightarrow P$
is finitely generated, projective, and faithful.  
\end{coro}

\begin{exam}% example 1
Let $k$ be a field and let $B$ be the ring of matrices of the form
$\left( \begin{smallmatrix}  a & b \\ 0 & c \end{smallmatrix}\right)$,
$a$, $b$, $c \in k$. Let $e =\left(  \begin{smallmatrix} 1 & 0 \\ 0
  & 0\end{smallmatrix} \right)$. The right ideal $P = eB$ is a
  finitely generated, projective, faithful $B-$module. However, $\im
  g_p = P \neq B$, so $P$ not a generator of $\mod - B$. Of course, $B$
  is not commutative.  
\end{exam}

\begin{exam}[Kaplansky]% example 2
Let $B$ be the (commutative) ring of continuous
real valued functions on the interval $[ 0, 1]$, and let $P$ be the
ideal of all functions vanishing in a neighbourhood of 0. It is
known that $P$ is projective, and clearly it is faithful. However, it
is easy to show that $\im g_p \neq B$, so $P$ is not a generator of
$\mod - B$. Of course, $P$ is not finitely generated.   
\end{exam}


\section{Wedderburn structure theory} % Section 6

Given a ring, $B$, we shall denote by $\mathbb{M}_n (B)$, the ring of
$n \times n$ matrices with entries in $B$. If $P$ is a $B$-module, we
shall write $p^{(n)}$ for the direct sum of $n$ copies of $P$. There
is a natural isomorphism  
$$
\Hom_B (P^{(n)}, P^{(n)}) \approx \mathbb{M}_n (\Hom_B (P, P)).  
$$

We\pageoriginale recall

\medskip
\noindent
\textbf{Schur's lemma.}
\textit{A homomorphism from a simple module over a ring into another
  simple module is either an isomorphism or the zero map.}  

\begin{theorem}\label{chap2:thm6.1}%theorem 6.1
Let $P$ be a faithfully projective right module over a ring
$B$. Suppose further that $P$ is simple (This is rare !). Then  
\begin{enumerate}[(1)]
\item $A = \Hom_B (P, P)$ is a division ring  

\item $P$ is a finite dimensional left vector space over $A$, say  $P
  \approx A^{(n)}$.  

\item $B \approx \Hom_A (P, P)^0  \approx \mathbb{M}_n(A)$ (via right
  multiplication ). 

\item $B$ is a simple ring  whose lattice of left ideals is
  isomorphic, via $\mathfrak{b} \mapsto P \mathfrak{b}$, to the
  lattice of $A$-subspaces of $P$.  

\item Centre $B \approx \Hom_{A- B} (P, P) \approx$ centre $A$, and these
  are fields.  

\item $P \otimes_B: B - \mod  \to A- \mod $ is an equivalence of
  categories.  
\end{enumerate}

Conversely, if $P \neq 0$ is a finite dimensional left vector space
over a division ring $A$, and if $B= \Hom_A (P, P^0)$, then $P$ is a
faithfully projective simple right $B$-module, and $A \approx
\Hom_B(P, P)$ (via left multiplication).  
\end{theorem}

\begin{proof}
(1)~ follows from Schur's lemma (2), (3), (4), (5) and (6) follows
  from theorem \ref{chap2:thm4.4} and proposition \ref{chap2:prop5.6} (c).  

If\pageoriginale $P \neq 0$ is a finite dimensional left vector space
over a division 
ring $A$, then evidently $P$ is a finitely generated projective
generator of $A$-mod, that is, a faithfully projective
$A$-module. Moreover, $B = \Hom_A(P,P)^0$ operates transitively on the
non-zero elements of $P$, so that $P$ is a simple $B$-module. It
follows, as before, form theorem \ref{chap2:thm4.4}(4) and proposition
\ref{chap2:prop5.6} (c), that $A \approx \Hom_B(P, P) $.   

We now describe the classical method for finding a $P$ as above. 
\end{proof}

\setcounter{lemma}{1}
\begin{lemma}\label{chap2:lem6.2}% 6.2
 If $P$ is a minimal right ideal in a ring $B$, and if $P^2 \neq 0$,
 then $P=eB$ for some idempotent $e$.  
\end{lemma}

\begin{proof}
Since $P^2 \neq 0$, there exists $x \in P$ such that $xP \neq
0$. Schur's lemma then implies that $P \xrightarrow{x} P$ (left
multiplication by $x$) is an isomorphism, so that $x = xe$ for a
unique $e \in P$. But this implies $x = xe^2$, so $e^2 = e$. In
particular, $0 \neq eB \subset P$, and thus $P = eB$. 
\end{proof}

\setcounter{prop}{2}
\begin{prop} %6.3
Let $B$ be a ring having no idempotent two-sided ideals other than
0 and $B$, and let $P$ be a minimal right ideal such that $P^2 \neq
0$. Then $P$ is a faithfully projective and simple $B-$module, so we
have the consequences of theorem \ref{chap2:thm6.1}. 
\end{prop}

\begin{proof}
$P$ is finitely generated projective thanks to lemma
  \ref{chap2:lem6.2}. Moreover $0 
  \neq P \subset \im g_p$ is, according to lemma \ref{chap2:lem5.7},
  an idempotent two 
  sided ideal. The hypothesis therefore implies that $\im g_p = B$,
  that is, $P$ is a generator of mod-$B$. Thus $P$ is faithfully
  projective. Also $P$ is simple by hypothesis. 
\end{proof}

\begin{example*}%exa 0
A right\pageoriginale artinian ring $B$ having no two-sided ideals
other than $0$ 
and $B$ satisfies the hypothesis of the above proposition. For, it has
a minimal right ideal $P \neq 0$ and $P^2$ cannot be zero (otherwise
the two-sided ideal $BP \neq 0$ would be distinct from $B$ since it
would be nilpotent). 
\end{example*}

We now generalize these results to the semi-simple case. Recall that a
module is called \textit{semi-simple} if it is a direct sum of simple
modules. 

\setcounter{lemma}{3}
\begin{lemma} % lem6.4 
Suppose a module $M$ is the sum of a submodule $N$ and a family
$(S_i)_{i \in I}$ of simple submodules. Then there is a subset $J$ of
$I$ such that the map 
 $$
  f_J : N \coprod(  \coprod_{j \in J}S_j ) \rightarrow M, 
 $$
 induced by inclusions, is an isomorphism.
\end{lemma}

\begin{proof}
Among the subsets $J$ for which $f_j$ is a monomorphism, we can choose
a maximal one, say $J_0$, by Zorn's lemma. If $f_{J_0}$ is not
surjective, there exists $j \in I - J_0$ such that $S_j \not \subset
\im f_{J_0}$. Since $S_j$ is simple, im $f_{J_0} \cap S_j = 0$. Thus
$J_0 \cup \{ j \}$ contradicts the maximality of $J_0$. 
\end{proof}

\begin{coro*}%coro 
A submodule of a semi-simple module is a direct summand.
\end{coro*}

\setcounter{prop}{4}
\begin{prop} %\props 6.5 
 Suppose $B$ has a faithfully projective right $B-$ module $P$ which
 is semi-simple. Then 
 $$
 P \approx S^{(n_1)}_1 \oplus \cdots \oplus S^{(n_r)}_r, 
 $$
 where\pageoriginale $S_1, \ldots , S_r$ are a complete set of
 non-isomorphic simple 
 $B-$ modules, and each $n_i > 0$. If $D_i = \Hom_B(S_i, S_i)$, then
 $D_i$ is a division ring, and  
 $$
 \Hom_B (P, P) \approx \prod_{1 \leq i \leq r} \mathbb{M}_{n_i}
 (D_i). 
 $$
 Moreover, $B$ is itself a semi-simple $B-$ module. 
\end{prop}

\begin{proof}
Since $P$ is finitely generated and semi-simple, it is a finite direct
sum of simple modules, and we can write $P \approx S^{(n_1)}_1 \oplus
\cdots \oplus S^{(n_r)}_r$, where each $S_i$ is simple, $S_i$ not
isomorphic to $S_j$ for $i \neq j$, and each $n_i > 0$. If $S$ is any
simple module, then $S$ is a quotient of a coproduct of copies of $P$
and this clearly implies that $S$ is isomorphic to some $S_i$. Since
$\Hom_B (S_i, S_j) = 0$ for $i \neq j$ (Schur's lemma), we have $\Hom_B
(P, P) \approx \prod \limits_{1 \leq i \leq n} \Hom_B (S^{(n_i)}_i,
S^{(n_i)}_i) \approx \prod \limits_{1 \leq i \leq r} \mathbb{M}_{n_i}
(D_i)$. Since $B$ is a quotient, and hence a direct summand of a
coproduct of copies of $P$, $B$ is also semi-simple. 
\end{proof}


\begin{prop}%props 6.6
Let $B$ be right artinian and let $B$ have no nilpotent two-sided
ideals $\neq 0$. Then $B$ is a semi-simple right $B-$ module. As a
ring, $B$ is a finite direct product of full matrix rings over
division rings. In particular, the center of $B$ is a finite product
of fields. 
\end{prop}

\begin{proof}
Once we know that $B$ is a semi-simple right $B-$ module, the
remaining conclusions follow from (6.5), since $B$ is obviously
faithfully $B-$projective and the ring of endomorphisms of the right
$B$-module $B$ is isomorphic to $B$. 
\end{proof}

If\pageoriginale $\mathfrak{b}$ is a minimal (i.e. simple) right ideal
of $B$, then $\mathfrak{b} = eB$ with $e^2 = e$. This follows from
lemma \ref{chap2:lem6.2}, provided $\mathfrak{b}^2 \neq 0$. But
$\mathfrak{b}^2 = 0$ 
implies that $B \mathfrak{b} \neq 0$ is a nilpotent two-sided ideal
contradicting our hypothesis. We note that $\mathfrak{b}$, being a
direct summand of $B$, is a direct summand of any right ideal which
contains $\mathfrak{b}$. 

Now, if $B$ is not semi-simple we can find a right ideal $\mathscr{O}$
minimal with the property that $\mathscr{O}$ is not
semi-simple. Choose a simple right ideal $\mathfrak{b}$ in
$\mathscr{O}$. Then $\mathscr{O} = \mathfrak{b} + \mathscr{O'}$
(direct sum) for some right ideal $\mathscr{O'} \underset{\neq}\subset
\mathscr{O}$. Then $\mathscr{O}'$ is semi-simple and thus
$\mathscr{O}$ also is semi-simple, which is a contradiction. 

\begin{prop}\label{chap2:prop6.7}% props 6.7
$B$ is a semi-simple $B-$module $\Leftrightarrow$ every $B-$ module is 
  projective. 
\end{prop}

\begin{proof}
$\Rightarrow$ Let $P$ be a right $B-$ module. Then $P$ is a quotient 
  of a free right $B- $module $F$ which is semi-simple by
  assumption. It follows from the corollary to (6.4), that $P$ a
  direct summand of $F$. 
\end{proof}

$\Leftarrow$ Let $\mathscr{O}$ be the sum of all simple right ideals
of $B$. Then $\mathscr{O}$ is semi-simple, by (6.4). By hypothesis,
$B/\mathscr{O}$ is projective, so that $B= \mathscr{O} \oplus
\mathfrak{b}$ for some right ideal $\mathfrak{b}$. If $\mathfrak{b}
\neq 0$, then, being finitely generated, it has a simple quotient
module and hence a simple submodule (because the simple quotient is
projective). This contradicts the defining property of $\mathscr{O}$,
and hence $\mathscr{O} = B$. Thus $B$ is semi-simple. 

\setcounter{definition}{7}
\begin{definition}% defin 6.8
We\pageoriginale call a ring $B$ {\em semi-simple} if it is semi
simple as a right module over itself.  
\end{definition}

The results above show that is equivalent to $B$ being a finite
product of matrix rings over division rings. In particular, the
definition of semi-simplicity of a ring is left-right symmetric. 


\section{Autoequivalence classes; the Picard group}%sec 7

If $\mathscr{A}$ is a $k-$category, $k$ a comutative ring, we define 
$$
\Pic_k (\mathscr{A})
$$	
to be the group of isomorphism classes $(T)$ of $k-$equivalences $T :
\mathscr{A} \to \mathscr{A}$. The group law comes from composition of
functors. 

If $A$ is a $k-$algebra, we define 
$$
\Pic_k (A)
$$
to be the group of isomorphism classes $(P)$ of invertible $A - A -$
bimodules (see definition \ref{chap2:def3.2}) with law of composition induced by
tensor product: $(P) (Q) = (P \otimes_A Q)$. It follows from
proposition \ref{chap2:prop4.1} and theorem \ref{chap2:thm4.4}(3),
that this is indeed a group with 
$(P)^{-1} = (\Hom_A(P, A))$. In the latter we can use either the left
or the right $A-$module structure of $P$. 

According\pageoriginale to theorem \ref{chap2:thm3.1}:

\begin{prop}\label{chap2:prop7.1} %props 7.1
$(P) \mapsto (P \otimes_A)$ and $(T) \mapsto (TA)$ define
    inverse isomorphisms 
$$
\Pic_k(A )^{\longrightarrow}_{\longleftarrow} \Pic_k (A - \mod).
$$
\end{prop}

Let $P$ be an invertible $A- A$ bimodule. If $\alpha$, $\beta \in
Aut_k(A)$ are $k-$algebra automorphisms, write 
$$
\alpha^P \beta
$$
for the bimodule with additive group $P$ and with operations
$$
a \cdot p = \alpha(a)p, p \cdot a = p \beta(a) \quad  (p \in P, a\in A).
$$
Thus $P =_1 P_1$.

Suppose $f: P \to Q$ is a \textit{left} $A-$isomorphism of invertible
$A-A-$bimodules. Since, via right multiplication, $A = \Hom_A (_A P,
_AP)^0$, we can define $\alpha \in Aut_k(A)$ by 
$$
p \alpha(a) = f^{-1} (f(p)a) 
$$
or
$$
f(p \alpha(a)) = f(p)a,  \qquad p \varepsilon P, a \in A. 
$$
Then $f : _1P_\alpha \rightarrow Q$ is a bimodule isomorphism. This
proves, in particular, the statement (4) in the following 

\setcounter{lemma}{1}
\begin{lemma}\label{chap2:lem7.2}% lem 7.2
For\pageoriginale  $\alpha$, $\beta$, $\gamma \in Aut_k (A)$ we have 
\begin{enumerate}[(1)]
\item $\alpha^A \beta \approx \gamma \alpha^A \gamma \beta$

\item $1^A \alpha \otimes_A 1^A \beta \approx 1^A \alpha \beta$

\item $_1 A_\alpha \approx _1A_ 1 \Leftrightarrow \alpha \in$ In Aut
  (A), the group of inner automorphisms of $A$. 

(In all cases above the symbol $\approx$ denotes bimodule isomorphism.)

\item If $P$ is an invertible $A- A-$bimodule and if $P \approx A$ as
  left $A-$ modules, then $P \approx_1 A_\alpha$ as bimodules for some
  $\alpha \in Aut_k (A)$. 
\end{enumerate}
\end{lemma}

\begin{proof}
\begin{enumerate}[(1)]
\item The map $_\alpha A _\beta \rightarrow_{\gamma \alpha}A_{ \gamma
  \beta}$ given by $x \mapsto \gamma (x)$ is the required
  isomorphism. 

\item Using (1) we have ${_1} A{_\alpha} \otimes_{A}{ _1}A_{\beta}
  \approx { _{\alpha^{-1}}} A_1 \otimes_{A1} A_{\beta} \approx
          {_{\alpha{-1}}}{_ \beta} \approx {_1} A_{\alpha \beta}$. 

\item If $f: {_1} A_\alpha \rightarrow _1 A_1$ is a bimodule
  isomorphism, then as a left $A$-auto\-morphism $f(x) = xu$, where $u
  = f(1)$ is a unit in $A$. Moreover, $f(\alpha(a)) = f(1.a) =
  f(1)a$, which gives $\alpha(a) u = ua$, that is, $\alpha(a) =
  uau^{-1}$ for all $a \in A$. 
\end{enumerate}
\end{proof}

Conversely, if $\alpha (a) = uau^{-1}$ for some unit $u \in A$, then
$f(x) = xu$ defines a bimodule isomorphism $_1 A_\alpha \to _1 A_1$. 

The group $\Pic_k (A-\mod) \approx \Pic_k(A)$ operates on the
isomorphism classes of faithfully projective left $A-$modules. We now
describe the stability group of a faithfully projective module under
this action. 

\setcounter{prop}{2}
\begin{prop}\label{chap2:prop7.3}% props 7.3
Let\pageoriginale $Q$ be a faithfully projective left $A-$module, and
let $B$ denote the $k-$algebra $\Hom_a(Q, Q)^0$. Then there is an
exact sequence  
\begin{equation*}
1 \to \text{ In Aut } (B) \to  \text{ Aut} _k(B)
\xrightarrow{\varphi_Q} \Pic_k (A) \tag{*} 
\end{equation*}
with
$$
\im \varphi_Q = \{ (P) \in \Pic_k (A) \big|  P \otimes_A Q \approx Q
\text{ as left A- modules} \}. 
$$
\end{prop}

\begin{proof}
Suppose first that $Q = A$, so that $B = A$. Define $\varphi_A
(\alpha) = (_1 A_{\alpha})$. Lemma \ref{chap2:lem7.2} tells us that
this is a homomorphism with kernel InAut $(A)$, and with the indicated
image.  
\end{proof}

In the general case, we set $Q^* = \Hom_A (Q, A)$. Then the functor $T =
\Hom_A (Q, ~)\approx Q^* \otimes_A : A- \mod \to B- \mod$ is an
equivalence, with $TQ = B$. This induces an isomorphism $\Pic_k(A-
\mod) \rightarrow \Pic_k(B- \mod)$. By proposition
\ref{chap2:prop7.1}, we obtain an 
isomorphism $\Pic_k (A) \to \Pic_k (B)$, and this maps $(P) \in \Pic_k
(A)$ into $(Q^* \otimes_A P \otimes_A Q) \in \Pic_k (B)$. 

We define now $\varphi_Q : Aut_k(B) \to \Pic_k(A)$ as the composite
$Aut_k(B) \to \Pic_k (B) \xrightarrow{\approx} \Pic_k (A)$, where the
first map is defined as in the special case treated in the beginning,
and the second is the inverse of the isomorphism just mentioned. The
exactness of $(*)$ follows from the special case. Also, if $(P) \in
\Pic_k (A)$, then $P \otimes_A Q \approx Q$ as left $A-$ modules
$\Leftrightarrow Q^* \otimes_A P \otimes_A Q \approx Q^* \otimes_A Q$
as left  $B-$modules. Since $Q^* \otimes_A Q \approx B$ as
$B-B-$bimodules, the last statement in the proposition follows from
the special case.  

Let\pageoriginale $C = $ center A. If $P$ is an invertible $A-A-$
bimodule, we can define a map  
$$
\alpha_P : C \to C
$$
by requiring that  
$$
pt = \alpha_P(t)p, \quad p \in P, t \in C. 
$$
This is possible because, the map $p \mapsto pt$, being a bimodule
endomorphism of $P$, is the left multiplication by a unique element in
the centre. Now $\alpha_P$ is a $k-$algebra homomorphism $(tp = pt$
for $t \in k)$. If $p \otimes q \in P \otimes_A Q$ and $t \in C$, then
$(p \otimes q) t = p \otimes \alpha_Q (t) q = p \alpha_Q (t) \otimes q
= \alpha_p \alpha_Q (t) (p \otimes q)$. Thus 
$$
\alpha_{P \otimes Q} = \alpha_P \alpha_Q. 
$$
Since, evidently $\alpha_A = \Id_C$, it follows from the invertibility
of $P$, that $\alpha_P$ is an automorphism of $C$, and that $(P)
\mapsto \alpha_P$ is a homomorphism $\Pic_k(A) \to Aut_k(C)$. The
kernel is clearly $\Pic_C(A)$. Summarization gives 

\begin{prop} %7.4
If $A$ is a $k$-algebra with center $C$, then there is an exact
sequence  
$$
0 \to \Pic_C (A) \to \Pic_k (A) \to Aut_k (C).
$$
If $A$ is commutative, then
$$
0 \to \Pic_A (A) \to \Pic_k (A) \to Aut_k (A) \to 1
$$
is exact and splits.
\end{prop}

\begin{proof}
The\pageoriginale map $\alpha \mapsto (_1 A_\alpha)$ (see lemma
\ref{chap2:lem7.2}) 
gives the required splitting $Aut_k (A) \to \Pic_k (A)$.  
\end{proof}

\begin{example*}
Let $A$ be the ring of integers in an algebraic number field $k$, and
let $G(k/\mathbb{Q})$ be the group of automorphisms of $k$. ($k$ need not be
Galois.) Evidently $Aut_{\mathbb{Z}}(A) \approx G(k/\mathbb{Q})$, and
$\Pic_A(A)$ is just the ideal class group of $A$. Thus
$\Pic_{\mathbb{Z}}(A)$ is the semi-direct product of the ideal class
group of $A$ with $G(k/\mathbb{Q})$, which operates on the ideal group, and
hence on $\Pic_A (A)$. This is also the group of autoequivalences of
the category $A-$mod. In particular, $\Pic_{\mathbb{Z}}(A)$ is finite
(finiteness of class number) and $\Pic_{\mathbb{Z}} (\mathbb{Z}) = \{ 1
\}$. Thus any autoequivalence $\mathbb{Z} - \mod \to \mathbb{Z}- \mod$
is isomorphic to the identity functor. 
\end{example*}

