\chapter{}\label{chap1}

\section{Preliminaries and function spaces}\pageoriginale\label{chap1-sec1}%%% 1

We will be concerned with functions and differential operators defined
on the $n$-dimensional  Euclidean space $\underbar{R}^n$. The points of
$\underbar{R}^n$ will be denoted by  $x = (x_1, \ldots, x_n)$, $\xi =
(\xi_1,  \ldots,  \xi_n)$, etc. and we will use the following
abbreviations: 
$$
|x|=\left(\sum x_j^2\right)^{\frac{1}{2}}, \lambda x =(\lambda x_1, \ldots, 
\lambda x_n), x\cdot \xi = \sum_j x_j \xi_j; 
$$
$S$ will denote the sphere $|x|=1$, $dS_x$ the element of surface area
on $S$, and $dx$ will denote the standard volume element in
$\underbar{R}^n$. If $\nu=(\nu_1, \ldots, \nu_n)$ is a multi-index of
non-negative integers $|\nu | =  \nu_1+ \cdots + \nu_n$ is called the
(total) order of $\nu$. We will also use the following standard
notation: 
\begin{align*}
\left(\frac{\partial}{\partial x}\right)^\nu & =\left(\frac{\partial}{\partial
    x_1}\right)^{\nu_1}\cdots\left(\frac{\partial}{\partial
  x_n}\right)^{\nu_n},  \; \xi^\nu 
  = \xi_1^{\nu_1} \ldots \xi_n^{\nu_n}, \\ 
  a_{\nu} (x) & = a_{\nu_1 \cdots \nu_n} (x).
\end{align*}

In general $a_\nu(x)$ will be complex valued functions on
$\underbar{R}^n$,  unless otherwise mentioned. We will also have
occasion to use vectors and matrices of complex valued functions. The
notation will be obvious from the context. 

A general linear partial differential operator can be written in the
form 
\begin{equation}
  a \left(x, \frac{\partial}{\partial x}\right) = \sum_\nu a_{\nu} (x)
\left(\frac{\partial}{\partial x}\right)^{\nu}. \tag{1.1} \label{chap1-eq1.1}
\end{equation}

The maximum $m$ of the total orders $| \nu |$ of multi-indices
occurring in (1) for which $a_{\nu}(x)\not \equiv 0$ is called the
order of the operator\pageoriginale $a \left(x,  \dfrac{\partial}{\partial
  x}\right)$. The 
transpose or the formal adjoint of $a\left(x,\dfrac{\partial}{\partial
  x}\right)$ is defined by 
\begin{equation*}
t_{a} \left(x, \frac{\partial}{\partial x}\right) [u] =\sum_{|\nu | \leq
  m}(-1)^{|\nu|} \left( \frac{\partial}{\partial x}\right)^{\nu}[a_\nu (x)
  u]. \tag{1.2} \label{chap1-eq1.2}
\end{equation*}

The adjoint of $a \left(x,  \dfrac {\partial}{\partial x} \right)$ in
$L^2$ is defined by 
\begin{equation}
a^* \left(x,  \frac{\partial}{\partial x}\right) [u] = \sum _{|\nu | \leq
  m}(-1)^{|\nu|}\left(\frac{\partial}{\partial
  x}\right)^{\nu}[\overline{a_{\nu}(x)}u]. \tag{1.3}  \label{chap1-eq1.3}
\end{equation}

In most of our considerations we will be considering systems of linear
differential equations of the first order. We refer to these as first
order. We refer to these as first order systems. A first order system
can therefore be written in the form: 
\begin{equation}
\left(A \left(x, \frac{\partial}{\partial x}\right)u\right)_j 
  = \sum^{N}_{K=1} A_{jk}\left(x,
  \frac{\partial}{\partial x}\right)u_{k},\qquad j=1, \ldots ,  N,
  \tag{$1.1'$} 
\end{equation}
where $A_{jk}\left(x, \dfrac{\partial}{\partial x}\right)=\sum\limits
^{n}_{\rho=1} 
a_{jk, \;  \rho} (x) \dfrac{\partial}{\partial x_{\rho}} +b_{jk}(x)$ and
$u= (u_1, \ldots,  u_N)$. The formal adjoint of $A \left(x,
\dfrac{\partial}{\partial x}\right)$ is defined by 
\begin{equation}
\left(^t A\left(x,  \frac{\partial}{\partial x}\right) v\right)_j =
\sum\limits_j{^t}  A_{jk} \left(x, 
  \frac{\partial}{\partial x}\right)v_j,\quad k=1, \ldots , N,
  \tag{$1.2'$}  
\end{equation}
where $^t A_{jk} \left(x, \dfrac{\partial}{\partial x}\right)u_j = \sum 
^{n}_{\rho=1} (-1) \dfrac{\partial} {\partial x_\rho}(a_{jk,  \rho}
(x) u_j) +  b_{jk}(x)u$, and the adjoint in $L^2$ of $A \left(x,  \dfrac
{\partial}{\partial x}\right)$ is defined by 
\begin{equation}
\left(A^*\left(x,  \frac{\partial}{\partial x}\right)v\right)_k =
\sum\limits_j A^*_{jk} \left(x, 
  \frac{\partial} {\partial x}\right) v_j,  \quad k=1,  \ldots ,  N
  \tag{$1.3'$}  
\end{equation}
where $A^*_{jk} \left(x,  \dfrac{\partial}{\partial x} \right)v_j =
\sum\limits_{\rho} (-1) ( \dfrac{\partial}{\partial x_{\rho}})
(\overline{a_{jk, \rho}   (x)} v_j ) + \overline{b_{jk}(x) }{v_j}$.   

We shall now introduce some function spaces used in the
sequel.\pageoriginale $U$ 
will denote an open set in $\underbar{R}^n$. $\mathscr{D} (U)$,
$\mathscr{E}(U)$, $\mathscr{E}^m(U)$, $\mathscr{D'}(U)$,
$\mathscr{E'}(U)$, $\mathscr{S} (\underbar{R}^n)$, $\mathscr{S}'
(\underbar{R}^n)$ will denote the function spaces of Schwartz
provided with their usual topologies. The space of $m$ times
continuously differentiable functions which are bounded together with
all their derivatives up to order $m$ in $U$ will be denoted by
$\mathscr{B}^m(U)$. $\mathscr{B}^m(U)$ is provided with the topology
of convergence in $ L^ \infty (U)$ of all the derivatives up to order
$m. \mathscr{E}^m_{L^p}(U)$ stands for the space of functions in
$L^{p} (U)$ whose distribution derivatives up to order $m$ are
functions in $L^p (U)$. For $f \in \mathscr{E}^m_{L^p} (U)$ we define 
$$
||f ||_{\mathscr{E}^m_{L^p}(U)} = || f ||_{p, m} = (\sum\limits_{|
  \nu | \leq m} || \left(\frac{\partial}{\partial x}\right)^\nu f||^p_{L^p
  (U)})^{1/p}. 
$$
$\mathscr{E}^m_{L^{p}}(U)$ is a Banach space with this norm. Clearly
$\mathscr{E}^m_{L^p} (U) \subset \mathscr{E}^k_{L^p} (U)$ for $k \leq
m$ and the inclusion mapping is continuous. The space of distributions
$f \in \mathscr{D}' (U)$ which are in $\mathscr{E}^m_{L{p}}(U')$ for
every relatively compact subset $U'$ of $U$ is denoted by
$\mathscr{E}^m_{L^p(loc)}(U)$. This space is topologized by the
following sequence of semi-norms. If $\{U_n\}$ is a sequence of
relatively compact subsets of $U$,  covering $U$,  we define 
$$
p_n (f) = ||f||_{\mathscr{E}^{m}_{L^p}(U_n)}  \text{ for }  f
\in \mathscr{E}^m_{L^p (\loc)} (U).  
$$
$\mathscr{E}^m_{L^p(\loc)} (U)$ is a Frechet space with this
topology. This space can also be considered as the space of 
distributions $f \in  \mathscr{D'} (U)$ such that  $\alpha f
\in \mathscr{E}^m_{L^p} (U)$\pageoriginale for every  $\alpha
\in \mathscr{D}(U)$. Evidently $\mathscr{E}^m_{L^p} (U) \subset
\mathscr{E}^m_{L^p(\loc)} (U)$ with continous inclusion for $m \geq
0$. The closure of $\mathscr{D}(U)$ in $ \mathscr{E}^m_{L^p}(U)$ is
denoted by $\mathscr{D}^m_{L^p}(U)$ and is provided with the induced
topology. As before  $\mathscr{D}^m_{L^p}(U) \subset
\mathscr{D}^k_{L^p}(U)$ for every $k \leq m$ with continuous
inclusion. In general $\mathscr{D}^m_{L^p} (U) \neq
\mathscr{E}^m_{L^p}(U)$  (for a detailed study of these spaces see
Seminaire Schwartz 1954 for the case $p=2$). However
$\mathscr{D}^m_{L^p} (\underbar{R}^n) = \mathscr{E}^m_{L^p}
(\underbar{R}^n)$. 

When we consider spaces of vectors or matrices of functions we use the
obvious notations,  which,  however will be clear from the
context. For instance,  if $f = (f_1,  \ldots,  f_N)$ where $f_j
\in \mathscr{E}^m_{L^2}(U)$ then $||f||_{\mathscr{E}^m_{L^2}}$ stands for
$\left(\sum \limits_j || f_j ||^2_{\mathscr{E}^2_{L^{2}}(U)}\right)^{\frac{1}{2}}$. 

When $U=\underbar{R}^n$ we simply write $\mathscr{D}$, $\mathscr{E}$,
$\mathscr{E}^m \mathscr{D}^m_{L^2}$ etc. for $\mathscr{D}(U), \ldots$, 

We will denote the space of all continuous functions of $t$ in an
interval [$0, T$] with values in the topological vector space
$\mathscr{E}^m$ by $\mathscr{E}^m [0, T]$. It is provided with the
topology of uniform convergence (uniform with respect to $t$ in $[0,
  T]$) for the topology of $\mathscr{E}^m$. Similar definitions hold
for $\mathscr{E}^m_{L^2}[0, T]$,  $\mathscr{D}^m_{L^2} [0, T]$,  
$\mathscr{D}^m_{L^2(loc)} [0, T]$,  $\mathscr{B}^m[0, T]$,  etc. 

We now recall, without proof, a few well-known results on the
spa\-ces $\mathscr{E}^m_{L^p}(U)$ and  $\mathscr{E}^m_{L^p (\loc)}(U)$. 

\begin{proposition}[Rellich]\label{chap1-sec1-prop1}
Every\pageoriginale bounded set in $\mathscr{E}^m_{L^p}(U)$ is 
relatively compact in $\mathscr{E}^{m-1}_{L^p (\loc)} (U)$ for $m \geq 
1$. 

In other words,  the proposition asserts that the inclusion mapping of
$\mathscr{E}^m_{L^p}(U)$ into $\mathscr{E}^{m-1}_{L^p (\loc)}(U)$ is
completely continuous. 

The following is a generalization due to Sobolev of a result of
F. Riesz. 
\end{proposition}

\begin{proposition}\label{chap1-sec1-prop2} %proposition 2
Let $g \in L^p$, $h \in L^q$ for $p$, $q > 1$  such
that $\dfrac{1}{p}+\dfrac{1}{q} > 1$. Then the following inequality
holds: 
\begin{equation*}
\left|  ~ \iint\limits_{\underbar{R}^n \times\underbar{R}^n}
\frac{g(x) h (y)}{|x - y|^{\lambda}} dx \; dy \right| \leq K || g
||_{L^p} \cdot || h ||_{L^p} \tag{1.4}\label{chap1-eq1.4}
\end{equation*}
where $\lambda = n\left(2-\dfrac{1}{p}-\dfrac{1}{q}\right)$ and $K$ is a
constant depending only on $p$, $q$, $n$ but not on $g$ and $h$. 
\end{proposition}

\begin{proposition}[Sobolev]\label{chap1-sec1-prop3} %proposition 3
 If $h \in L^{p}$ for $p> 1$ then the function 
\begin{equation*}
f(x) = \int \frac{h(y)}{|x-y|^{\lambda}}  dy,  \tag{1.5}\label{chap1-eq1.5}
\end{equation*}
where $n > \lambda > \dfrac{n}{p'}= n(1-\dfrac{1}{p})$,  is in 
$L^{q}$ where  $\dfrac{1}{q} = \dfrac{1}{p} + \dfrac{\lambda}{n}-1$. 
\end{proposition}

\begin{theorem}[Sobolev]\label{chap1-thm1} %theorem 1
  Let $U$ be an open set with smooth boundary $\partial U$ (for instance
  $\partial U \in C^2$). Then any function $\varphi \in
  \mathscr{E}^m_{L^p}(U)$ with $pm \leq n$ itself belongs to $L^q (U)$
  where $q$ satisfies $\dfrac{1}{q} = \dfrac{1}{p} -
  \dfrac{m}{n}$. Further we have an estimate 
\begin{equation}
|| \varphi ||_{L^q (U)} \leq C || \varphi ||_{\mathscr{E}^m_{L^p
    (U)}} \tag{1.6} \label{chap1-eq1.6}
\end{equation}

The\pageoriginale contant $C$ depends only on $p$, $q$, $r$ and $n$
but not on the function $u$. 
\end{theorem}

For the study of this inequality and delicate properties of the
inclusion mapping see S. Sobolev: Sur un Th\'eor\`eme d'analyse
fonctionnelle,  Mat. Sbornik,  4(46), 1938. 

\section{Cauchy Problem}\label{chap1-sec2} %sec 2 

In this section we formulate the Cauchy problem for a linear
differential operator $a\left(x,  \dfrac {\partial}{\partial
  x}\right)$. To begin with we make a few formal reductions. 

Let $S$ be a hypersurface in $\underbar{R}^n$ defined by an equation
$\varphi (x) = 0$ where $\varphi$ is a sufficiently often continuously
differentiable function with its gradient $\varphi_x (x_0)\equiv
\left(\dfrac{\partial \varphi}{\partial x_1} (x_0), \ldots ,
\dfrac{\partial \varphi}{\partial x_n}(x_0)\right) \neq 0$ at every point
$x_0$ of $S$. Let $n$ denote the normal at the point $x_0$ to $S$ and
$\dfrac{\partial}{\partial n}$ denote the derivation along the normal
$n$. 

Suppose $x_0$ is a point on $S$;  let $u_0,  \ldots , u_{m-1}$ be
functions on $S$ defined in a neighbourhood of $x_0$. A set $\psi =
(u_0,  \ldots,  u_{m-1})$ of such functions is called a set of Cauchy
data on $S$ for any differeential operator of order $m$. The Cauchy
data $\psi$ are said to be  analytic (resp. of class $\mathscr{E}^m$,
resp. of class $\mathscr{E}$) if each of the functions $u_0$, $u_1,
\ldots , u_{m-1}$ is an analytic (resp. $m$ times continuously
differentiable function resp. infinitely differentiable function) in
their domain of definition. 

Let there be given a function $f$ defined in a neighbourhood $U$ in 
$\underbar{R}^n$ of a point $x_0$ of $S$ and Cauchy data $\psi$ in a
neighbourhood $V$ of $x_0$ on $S$.\pageoriginale 

The Cauchy problem for the differential operator  $a \left(x, \dfrac
{\partial}{\partial x}\right)$ with the Cauchy data $\psi$ on $S$ consists
in finding a function $u$ defined in a neighbourhood $U'$ of $x_0$ in
$\underbar{R}^n$ satisfying 
\begin{equation}
a \left(x, \frac{\partial}{\partial x}\right) u =f \text{ in }
U'\tag{2.1}\label{chap1-eq2.1}  
\end{equation}
and $u(x) =u_0 (x)$, $\dfrac{\partial}{\partial n} u(x) = u_1(x)$;
$\ldots,  \left(\dfrac{\partial}{\partial n}\right)^{m-1} u(x) =u_{m-1} (x)$ for
$x \in V \cap U'$. When such a $u$ exists we call it a
solution of the Cauchy problem. 

In the study of the Cauchy problem the following questions arise: the
existence of a solution $u$ and its domain of definition,  uniqueness
when the solution exists,  dependence of the solution on the Cauchy
data and the existence of the solution in the large. The answers to
these questions will largely depend on the nature of the differential
operator and of the surface $S$  (supporting the Cauchy data) in
relation to the differential operator besides the Cauchy data $\psi$
and $f$. In order to facilitate the formulation and the study of the
above questions we first make a preliminary reduction. 

By a change of variables
$$
(x_1,  \ldots,  x_n) \to (x'_1,  \ldots,  x'_n) 
$$
with $x'_1 = x_1, \ldots, x'_{n-1}=x_{n-1}$ and $x'_n = \varphi (x)$ the
equation 
\begin{equation}
a \left(x, \frac {\partial}{\partial x}\right) u = f
\tag{2.1}\label{addeq2.1}  
\end{equation}
is transformed into an equation of the form
$$
h(x,  \varphi_x) \left(\frac{\partial}{\partial x'_n}\right)^m u+\sum
\cdots = f  
$$\pageoriginale
where $\varphi_x = \left(\dfrac{\partial \varphi}{\partial x_1}, \ldots,
\dfrac{\partial \varphi}{\partial x_n}\right)$ and $h(x, \xi) =
\sum\limits_{|\nu|=m} a_\nu (x) \xi^\nu$, $\xi = (\xi_1, \ldots,
\xi_n)$. The 
summation above contains derivatives of $u$ of orders $< m$ in the
$x'_n$-direction. 
\begin{enumerate}
\renewcommand{\labelenumi}{(\theenumi)}
\item If $h(x,  \varphi_x (x) ) \neq 0$ in a neighbourhood of the
  point under consideration we can divide the above expression for the
  equation by the factor $h(x, \varphi_x)$ and write 
\begin{equation*}
\left(\frac{\partial}{\partial x'_n}\right)^{m} u + \sum_{\substack {|\nu|\leq m\\ 
{\nu_n \leq m-1}}} a'_\nu (x') 
  \left(\frac{\partial}{\partial x'}\right)^\nu
  u=\frac{f}{h(x,\varphi_x)}. \tag{2.2}\label{chap1-eq2.2}  
\end{equation*}

This is called the normal form of the equation.
$$
a\left(x,  \frac{\partial}{\partial x}\right) u = f.
$$

The Cauchy problem is now given by 
$$
\left(\frac{\partial}{\partial x'_n}\right)^j a(x'_1\ldots,  x'_{n-1},
0) =  u_j (x'_1, \ldots,  x'_{n-1}) \text{ for } j=0, 1, \ldots,  m-1.  
$$

\item In the case in which $h (x,  \varphi_x) = 0$ at a point $x_0$ of
  $S$ the study  of the Cauchy problem in the neighbourhood of $x_0$
  becomes considerably more difficult. In what follows we only study
  the case (1) where the equation can  bebrought to the normal form by
  a suitable change of variables. This motivates the following 
\end{enumerate}

\begin{defi*}
A surface $S$ defined by an equation $\varphi(x) = 0$ ($\psi$ being once
continuously differentiable) in $\underbar{R}^n$ is said to be a
characteristic variety or characteristic hypersurface of the operator
$a\left(x,  \dfrac{\partial}{\partial x}\right)$ if $h(x,  \grad  \varphi (x)) =
0$\pageoriginale for all the points $x$ on $S$. 
\end{defi*}

A vector $\xi \in \underbar{R}^n$ is said to be a
characteristic direction at $x$ with respect to the differential
operator $a\left(x,  \dfrac{\partial}{\partial x}\right)$ if $h(x,
\xi) = 0$.  

Clearly,  if $S$ is a characteristic variety of a differential
operator $a \left(x,  \dfrac{\partial}{\partial x } \right)$ then the
vector normal to $S$ at any point on it will be a characteristic
direction at 
that point. For any point $x \in S$ the set of vectors $\xi$
which are characteristic directions at $x$ form a cone in the
$\xi$-space with vertex at the origin called the characteristic cone
of the operator $a \left(x,  \dfrac{\partial}{\partial x} \right)$ at
the point 
$x$. In the following we restrict ourselves to the case where $S$ is
not characteristic for the differential operator at any point and
hence assume the operator to be in the normal form. 

\section{Cauchy - Kowalevsky theorem and Holmgren's
  theorem}\label{chap1-sec3} % \section 3 

The first general result concerning the Cauchy problem (local) is the
following theorem due to Cauchy and Kowalevsky. This we recall without
proof. For a proof see for example Petrousky \cite{key1}. 

From now on we change slightly the notation and denote a point of
$\underbar{R}^{n+1}$ by $(x,  t) = (x_1,  \ldots,  X_n,  t)$ and a
point of $\underbar{R}^n$ by $x = (x_1,  \ldots,  x_n)$. 

Let
\begin{equation*}
L \equiv \left(\frac{\partial}{\partial t}\right)^m + \sum_{\substack{|\nu | + j
    \leq m \\ j \leq m-1}} a_{\nu, j}(x, t)  \left(\frac{\partial}{\partial
  x}\right)^\nu \left(\frac{\partial}{\partial t}\right)^j
\tag{3.1}\label{chap1-eq3.1}  
\end{equation*}
be a differential operator of order $m$ written in the normal form
with variable coefficients. 

\setcounter{theorem}{0}
\begin{theorem}[Cauchy-Kowalevsky]\label{chap1-sec3-thm1} %theorem 1
Let\pageoriginale 
the coefficients $a_{\nu,  j}$ of $L$ be defined and analytic in
 a neighbourhood $U$ of the origin in the $(x, t)$ space. Suppose that
 $f$ is an analytic function on $U$ and $\psi$  is an analytic Cauchy
 datum in a neighbourhood $V$ of the origin in the $x$-space. Then
 there exists a neighbourhood $W$ of the origin in the $(x,  t)$-space
 and a unique solution $u$ of the Cauchy problem 
\begin{equation*}
\begin{split}
& Lu = f \text{ in } W \text{~ and }\\
& \left(\frac{\partial}{\partial t}\right)^{j_{u}} =u_j \text{~ on~ } W \cap
  \{ t = 0 \}   \text{~ for~ } j=0, 1, \ldots ,  m-1, 
\end{split}\tag{3.2}\label{chap1-eq3.2}
\end{equation*}
which is defined and analytic in $W$.
\end{theorem}

\begin{remark*}
The domain $W$ of existence of $u$ depends on $U$, $V$ and the
maximum moduli of $a_{\nu,j}$. 
\end{remark*}

It is not in general,  possible to assert the existence of a solution
of the Cauchy problem when the Cauchy data are only of class
$\mathscr{E}$. Howevr for a certain class of differential
operators-such as Hyperbolic operators - the existence (even in the
large) of solutions of the Cauchy problem can be established under some
conditions. This will be done in the subsequent sections. 

If $u_1$ and  $u_2$ are two  analytic solutions of the Cauchv problem
in a neighbourhood of the origin with the same analytic Cauchy data
the theorem of Cauchy-Kowalevsky asserts that $u_1\equiv
u_2$. Holmgren showed that for an operator with analytic coefficients
the solution is unique,  if it exists, in the class $\mathscr{E}^m$
($m$,  we recall., is the order of $L$). More precisely we have the 

\begin{theorem}[Holmgren]\label{chap1-sec3-thm2} %theorem 2
 If the\pageoriginale coefficients $a_{\nu, j}$ of the
 differential operator $L$ are analytic functions in a neighbourhood
 $U$ of the origin then there exists a number $\varepsilon_0 > 0$
 satisfying the following: for any $0 <\varepsilon < \varepsilon_0$ if
 the Cauchy data $\psi$ vanish on $(t=0) \cap D_\varepsilon$ then any
 solution $u \in \mathscr{E}^m$ of the Cauchy problem 
\begin{align*}
& Lu = 0 \text{~ in~ } D_\varepsilon \text{~ and}\\ 
& \left(\frac{\partial}{\partial t}\right)^j u=0 \text{~ on~ } (t=0) \cap
D_\varepsilon \text{~ for~ } j = 0,  1,  \ldots,  m-1,  
\end{align*}
itself vanishes identically in $D_\varepsilon$,  where
$D_{\varepsilon}$ denotes the set 
$$
\left\{(x,  t)\in \underbar{R}^{n-1} \bigg| | x |^2 + | t | <
\varepsilon \right\}. 
$$
\end{theorem}

\begin{proof}
By a change of variables $(x, t) \rightarrow (x',  t')$ where $x'_k =
x_k (k=1,  \ldots,  n)$ and $t' = t + x^2_1 + \cdots + x^2_n$ the half
space $t \geq 0$ is mapped into the domain 
$$
\Omega =\left\{ (x',  t') \in \underbar{R}^{n+1} \bigg|  t'- |x'|^2  
\geq 0\right\} 
$$
in the $(x'_1, t')$ space. The transformed function $u'(x',  t')$ and
its derivatives upto order $(m-1)$ in the direction of the interior
normal to the hypersurface $\{t' - |x'|^2 =0\}$ vanish identically
on the hyper-surface. Hence extending $u'$ by zero outside the domain
$\Omega$ we obtain a function in $\mathscr{E}^m$,  which we again
denote by $u$, with support contained in $\Omega$. The differential
operator is transformed into another differential operator of order
$m$ with analytic coefficients.  
\end{proof}

Thus we may assume that $u$ is a solution of an equation 
\begin{equation}
L u \equiv \left(\frac{\theta}{\partial t}\right)^m u+ \sum_{\substack{| \nu |
    + j \leq m\\ j\leq m-1}} a_{\nu,  j} (x,
t)\left(\frac{\partial}{\partial x}\right)^\nu
\left(\frac{\partial}{\partial t}\right)^j 
u=0 \tag{3.3} \label{chap1-eq3.3}
\end{equation}\pageoriginale
with support contained in $\Omega$. Let ${}^tL$ be the transpose
operator of $L$ and $V$ be a solution of ${}^tL [v] =0$ in $\Omega_h =
\Omega \cap \{0 \leq t \leq h\}$ satisfying the conditions  
\begin{equation}
v(x,  h) = \frac{\partial}{\partial t} v(x,  h) = \ldots =
\left(\frac{\partial}{\partial t}\right)^{m-2} v(x,  h) = 0
\tag{3.4} \label{chap1-eq3.4} 
\end{equation}
on the hyperplane $(t = h)$. Then we have
\begin{equation}
\int_{\Omega_h}(u  ^t L[v] - v\ L [u]) dx ~ dt = 0.\tag{3.5}\label{chap1-eq3.5}
\end{equation}

On the other hand,  integrating by parts with respect to the variables
$t$ and $x$ yields 
$$
\int_{\Omega_h}(u{}^t L [v] - v L [u]) dx ~dt = \int_{t=h} (-1)^m u(x,
t)\left(\frac{\partial}{\partial t}\right)^{m-1} v(x,  t) dx 
$$
because of the conditions  \eqref{chap1-eq3.4}.
\begin{equation}
\text{Hence } \quad \int\limits_{t=h} (-1)^m
u(x,t)\left(\frac{\partial}{\partial t}\right)^{m-1} v(x,  t)
dx=0. \tag{3.6}\label{chap1-eq3.6}   
\end{equation}

Now consider the Cauchy problems
\begin{align*}
& t_L [v] = 0\\
& \left(\frac{\partial}{\partial t}\right)^j v(x,  0) = 0,  j = 1,  \ldots m,  ~
  \left(\frac{\partial}{\partial t}\right)^{m-1} v (x,  0) = P (x),  
\end{align*}
$P(x)$ running through polynomials. By the Cauchy Kowalevsky
Theorem, there exists solutions $v(x)$, in a \textit{fixed
  neighbourhood} $| t | \leq h$ satisfying the above\pageoriginale Cauchy
problems. Hence there is a $h > 0$ such that, for every polynomial
$P(x)$ there exist $\underbar{v}$ in $\Omega_h$ satisfying
\eqref{chap1-eq3.4} with 
$\left(\dfrac{\partial}{\partial t}\right)^{m-1} u (x,  h) = P(x)$. Hence by
\eqref{chap1-eq3.6} $u(x, t)$ is orthogonal to every polynomial $P(x)$
for $t \leq h$. Hence $u(x,  t) \equiv 0$ for $0 \leq t \leq h$. Replacing $t$,
by $-t$ we obtain $u(x,  t) \equiv 0$ for $-h \leq t \leq 0$. Hence
$u(x,  t) \equiv 0$ in $D_\varepsilon$ which finishes the prove of the
theorem. 

Further general results on the uniqueness of the solution of the
Cau\-chy problem were proved by Calderon \cite{key1}. We restrict ourselves to 
stating one of his results (\cite{key3}). 

\begin{theorem}[Calderon]\label{chap1-sec3-thm3} %theorem 3
 Let $L$ be an operator of the form \eqref{chap1-eq3.1} with real
 coefficients. Assume that in a neighbourhood of the origin all the
 coefficients $a_{\nu,  j}(x, t)$, for $|\nu | + j = m$,  belong to
 $C^{1+\sigma}(\sigma > 0)$ and the other coefficients are
 bounded. Further suppose that the  characteristic equation at the
 origin 
\begin{equation}
P(\lambda,  \xi ) \equiv \lambda^m + \sum_{|\nu | + j = m} a_{\nu ,
  j} (0,  0) \xi^\nu \lambda^j = 0 \tag{3.6} \label{chap1-addeq3.6}
\end{equation}
has distinct roots for any real $\xi \neq 0$. If the solution $u$
belong to $C^m$ and has zero Cauchy data (more precisely,  Cauchy
data,  zero in a neighbourhood of the hyperplane $t = 0$) then $u
\equiv 0$ in a neighbourhood of the origin. 
\end{theorem}

\section{Solvability of the Cauchy problem in the class
  $\mathscr{E}^m$}\label{chap1-sec4}%%% 4 

In this section we make a few remarks on the existence of solutions of
the Cauchy problem in the  class $\mathscr{E}^m$ under weaker
regularity conditions on the coefficients of the differential
operator. We\pageoriginale begin with the following formal definition. 

Let
\begin{equation}
L\equiv \left(\frac{\partial}{\partial t}\right)^m + \sum_{\substack{|\nu |+ j
    \leq m\\ j \leq m-1}}  a_{\nu,  j}(x,  t)\left(\frac{\partial}{\partial
  x}\right)^\nu \left(\frac{\partial}{\partial t}\right)^j
\tag{4.1}\label{chap1-eq4.1}  
\end{equation}
be a differential operator of order $m$ in the normal form. 

\begin{defi*}
The Cauchy problem for $L$ is said to be solvable at the origin in
the class if for any given $f \in \mathscr{E}_{x,  t}$ and any
Cauchy datum $\psi$ of class $\mathscr{E}_x$ there exists a
\textit{neighbourhood} $D_{\psi, f}$ of the origin in the $(x, t)$
space and a solution $u \in \mathscr{E}_{x,  t} (D_{(\psi,
  f)})$ of the Cauchy problem for $L$ with $\psi$ as the Cauchy
datum. 
\end{defi*}

\begin{remark*}
The Cauchy problem for a general linear differential operator $L$ is
not in general solvable in the class $\mathscr{E}$ as is shown by the
following counter example due to Hadmard. 
\end{remark*}

\noindent
\textbf{Counter example (Hadamard)}.
Let $L$ be the Laplacian $\Delta$ in  $\underbar{R}^3$ 
\begin{equation}
\Delta \equiv \left(\frac{\partial}{\partial x}\right)^2 +
\left(\frac{\partial}{\partial y}\right)^2 +
\left(\frac{\partial}{\partial z}\right)^2 \tag{4.2} \label{chap1-eq4.2}
\end{equation}
and $(z = 0)$ be the hyperplane supporting the Cauchy data. Consider
for the Cauchy data the conditions 
$$
u(x,y, 0)=u_0 (x, y)\text{ and } \frac{\partial u}{\partial z} (x,
y,  0) = 0. 
$$

Suppose $u(x, y, z) \equiv u$ is a solution of $\Delta u = 0$ in $z
\geq 0$ with the Cauchy data $(u_0,  0)$. Extend $u$ to the whole of
$\underbar{R}^3$ by setting 
\begin{align*}
\tilde{u} (x,  y,  z) & = u(x,  y,  z) \text{~ for~ } z \geq 0 \text{~
  and}\\ 
&= u(x,  y,  -z) \text{~ for~ } z \leq 0.
\end{align*}
$\tilde{u}$\pageoriginale satisfies the equation $\Delta \tilde{u} =
0$ in the sense 
of distributions. In fact, for any $\varphi \in
\mathscr{D}(\underbar{R}^3)$ we have 
\begin{align*}
\langle \tilde{u}, \Delta \varphi \rangle &=\int\limits_{\underbar{R}^3}
\tilde{u}(x,  y,  z) \Delta \varphi (x,  y,  z) dx \, dy \, dz\\ 
&= \lim\limits_{\varepsilon \to 0} \bigg\{-\int\limits_{|z|\geq \varepsilon}
\frac{\partial \varphi}{\partial z}\frac{ \partial \tilde{u}}{\partial
  z} dx \, dy \, dz + \int\limits_{|z|\geq \varepsilon}\left(\frac{ \partial^2
  \tilde{u}}{\partial x^2}+ \frac{\partial^2 \tilde{u}}{\partial y
  ^2}\right) \varphi dx \, dy \, dz  
\end{align*}
and
$$
\int\limits_{|z|\geq \varepsilon}\frac{\partial \tilde{u}}{\partial z} 
\frac{\partial \varphi}{ \partial z} dx~dy~dz = \int \left[\varphi
  \frac{\partial \tilde{u}}{\partial z}\right]^\varepsilon_{-\varepsilon}
dx \, dy- \int_{|z|\geq \varepsilon} \frac{ \partial^2 \tilde{u}}{\partial
  z^2} \varphi dz \, dx \, dy 
$$

Hence
{\fontsize{10pt}{12pt}\selectfont
\begin{align*}
\langle \tilde{u},  \Delta \varphi \rangle  &= \lim \limits_
        {\varepsilon \to 0} \left\{ \int \varphi(x,  y,  \varepsilon)
        \frac{\partial \tilde{u}} {\partial z}(x,  y,  \varepsilon)
        dx \, dy -\int \varphi (x,  y,  -\varepsilon) \frac{\partial
          \tilde{u}}{ \partial z} (x,  y,  -\varepsilon) dx \, dy \right
        \}\\ 
& = 0
\end{align*}}\relax

By the regularity of solutions of elliptic equations $u$ is an
analytic function of $x$, $y$, $z$ in $\underbar{R}^3$. Since $u_0 (x,
y) = u(x,  y,  0)= \tilde{u}(x,  y,  0),  u_0$ is an analytic function
of $(x,  y)$. Thus,  if $u_0$ is  taken to be in $\mathscr{E}_x$ but
non analytic there does not exist  a solution of the Cauchy problem
for $\Delta u = 0$ with the Cauchy data $(u_0,  0)$. 

As far as the domain of existence of a solution of the  Cauchy problem
is concerned we know by the Cauchy Kowalevsky theorem that,  whenever
the coefficients of $L$, $f$ and the Cauchy data $\psi$ are of
analytic classes,  there exists a neighbourhood of the origin and an
analytic function $u$ on it satisfying $L[u] = f$ with Cauchy data
$\psi$. However\pageoriginale it is not in general possible to
continue this local 
solution $u$ to the whole space as a solution of $L[u]=f$. This is
domonstrated by the following counter example which is again due to
Hadamard. 

\medskip
\noindent
\textbf{Counter example}. Let the differential operator be
$$
L \equiv \left(\frac{\partial}{\partial x}\right)^2+
\left(\frac{\partial}{\partial   y}\right)^2. 
$$

A solution of $L[u] = 0$ is provided by
$$
u(x,  y) = Re \frac{1}{z-a} = \frac{x-a}{(x-a)^2 + y^2} \text{~ where~ }
a>0.  
$$

Clearly $u(0,  y)$ and $\dfrac{\partial u}{\partial x}(0,  y)$ are
analytic functions of $Y$. However this solution can not be continued
to the half plane $x \geq a$ as can be easily seen. 

For a class of differential operators the existence of soluctions in
the large has been established by Hadamard,  Petrowsky,  Leray,
Garding and others. We shall  prove some of these results later by
using the method of singular integral operators introduced by Calderon
and Zygmund.
