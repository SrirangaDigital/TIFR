\chapter{}\label{chap2}

In\pageoriginale this chapter as well as in the next chapter we will be mainly
concerned with the study of the Cauchy problem for systems of
differen\-tial equations of the first order,  which will be referred to
as first order systems. 

\section{}\label{chap2-sec1}

If $u(x,  t) = (u_1(x,  t),  \ldots ,  u_N(x,  t))$ and  $f(x,  t) =
(f_1(x,  t),  \ldots, f_N(x , t))$ denote vector valued functions with
$N$ components,  a first order system of equations can be written in
the form 
\begin{equation}
M [u] \equiv \frac{\partial}{ \partial t} u-\sum^n_1 A_K (x,  t)
\frac{\partial}{ \partial x_k} u - B(x, t) u = f \tag{1.1}\label{chap2-eq1.1} 
\end{equation}
where $A_k (x, t)$, $B(x, t)$ are matrices of order $N$ of functions
whose rigularity conditions will be made precise in each of the
problems under consideration. 

\begin{defi*}
The Cauchy problem for a first order system $M [u] = 0$ is said to be
locally solvable at the origin in the space $\mathscr{E}$ (resp. $\mathscr{B}$,
resp. $D^\infty_{L^2}$) if for any given $\psi \in
\mathscr{E} (U)$ (resp. $\mathscr{B}(U)$, resp. $D^\infty_{L^2}(U))U$
being an arbitrary open set in the $x$-space containing the origin 
there exists a neighbourhood $V$ of the origin in $\underbar{R}^{n+1}$
and a function $u \in \mathscr{E}(V)$ (resp. $\mathscr{B}(V)$,
resp $D^\infty_{L^2}(V))$ satisfying 
$$
M[u] = 0 \text{ and } u(x, 0) = \psi (x)
$$
($V$ may depend on $\psi$).
\end{defi*}

The following proposition shows that when the system $M$ has analytic
coefficients the local solvability of the Cauchy problem
implies\pageoriginale the 
existence of a neighbourhood $V$ independent of $\psi$ such that for
any $\psi \in \mathscr{E}_x$  there exists a unique solution
$u \in \mathscr{E}^1 (V)$. 

We define a family of open sets $D_\varepsilon$ of
$\underbar{R}^{n+1}$ by 
\begin{equation}
D_\varepsilon = \left\{ (x, t) \in \underbar{R}^{n+1}\Big| |t| +
|x|^2 < \varepsilon\right\}. \tag{1.2}\label{chap2-eq1.2} 
\end{equation}

\setcounter{proposition}{0}
\begin{proposition}[P.D. Lax]\label{chap2-sec1-prop1}  %proposition 1
\cite{key1}. Assume that the coefficients of $M$ are analytic and the Cauchy
 problem for $M$ is locally solvable at the origin. Then there exists
 a  $\delta > 0$ such that for any given $\psi \in
 \mathscr{E}_x(U)$ there exists a unique solution $u \in
 \mathscr{E}^1 (D_\delta)$  of $M[u] = 0$, $u(x, 0)=\psi (x)$. 
\end{proposition}

\begin{proof}
By Holmgren's theorem there exists an $\varepsilon_0 > 0$ such that
for $0 < \varepsilon \leq \varepsilon_0$ a solution,  $u \in
\mathscr{E}^1_{x, t}$ with $u(x, 0) = \psi (x)$ on $D_\varepsilon \cap
(t=0)$ is uniquely determined in $D_\varepsilon$. Let $\varepsilon_0
> \varepsilon_1 \ldots $ be a sequence of positive numbers
$\varepsilon_n \to 0$. Denote by $A_{k, m}$  the set of all $\psi
\in\mathscr{E}_x (U)$ such  that the solution $u$ of $M
            [u]=0$ with $u(x, 0)=\psi (x)$ for $x \in
            D_{\varepsilon_{k}} \cap (t = 0)$ is in
            $\mathscr{E}^{[\frac{n}{2}]+2}_{L^2} (D \varepsilon _k)$
            and satisfies 
$$
|| u ||_{[\frac{n}{2}]+2 } \leq m. 
$$


The sets $A_{k, m}$ are symmetric and convex.  Further $\mathscr{E}
(U) = \bigcup\limits_{k, m} A_{k, m}$, by the local solvability at
the origin. We shall now show tht $A_{k, m}$ is \textit{closed} for
every $k, m$. 

Let $\psi_j$ be a sequence in $A_{k, m}$ converging to $\psi_0$ in 
$\mathscr{E}(U)$. The corresponding sequence of solutions $u_j$ is a
bounded set in
$\mathscr{E}^{[\frac{n}{2}]+2}_{L^2}(D_{\varepsilon_{k}})$ and hence
has a subsequence $u_{j_p} (x, t)$ weakly convergent in
$\mathscr{E}^{[\frac{n}{2}]+2}_{L^2} (D_{\varepsilon_k})$. In view of
the Prop.~\ref{chap1-sec1-prop1} of Chap.~\ref{chap1} \S\ \ref{chap1-sec1} 
we can,\pageoriginale if necessary by choosing a
subsequence, 
assume that $ u_{j_p}(x,  t)$ converges in $\mathscr{E}_{L^2
  (\loc)}^{[\frac{n}{2}]+1} (D_{\in _k})$. Let this limit be
$u_0$. Since $u_{j_p } \to u_0$ weakly in
$\mathscr{E}_{L^2}^{[\frac{n}{2}]+2} (D_{\in _k})$ we have $|| u_0 ||
_{[\frac{n}{2}]+2} \leq m$. By prop \ref{chap1-sec1-prop4} of Chap
\ref{chap1} \S\ \ref{chap1-sec1} (Sobolev's 
lemma) $u_0 \in \mathscr{E}^1 (D _{\in_k} )$ and further $ M [u_0] =
0$. Again $ u_{j_p} \to u_0$ in $\mathscr{E}_{L^2
  (\loc)}^{[\frac{n}{2}]+1} (D_{\in _k})$ implies that this
convergence is uniform on every compact subset of $D_{\in_k}$ and
hence $ u_0 (x,  0) = \psi_0 (x)$. Thus $A_{k, m}$ is a closed subset
of $\mathcal{E}_x (U)$.  

Now by Baire's category theorem one of the $ A_{k, m}$, let us say $
A_{k_0,  m_0}$,  contains an open set of $\mathscr{E}_x (U)$.
$A_{k_0,  m_0}$,  being symmetric and convex contains therefore a
neighbourhood of 0 in $\mathscr{E}_x (U)$. Since any $ \psi \in
\mathscr{E}_x (U) $ has a homothetic image $\lambda \psi$ in this
neighbourhood,  there is a unique solution $u \in
\mathscr{E}_{L^2}^{[\frac{n}{2}]+2}(D_{\in _{k_0}})$, a fortiori,
in $\mathscr{E}^1 (D_{\mathscr{E}_{k_{0}}})$ of $ M [u] = 0$ with $ u(x,  0) =
\psi (x)$. $\in_{k_0}$ can be taken to be the required $ \delta$.   
\end{proof}

\setcounter{theorem}{0}
\begin{theorem}\label{chap2-sec1-thm1} %theorem 1
Let the coefficients $ A_k (x, t)$, $B(x, t)$ of $M $ be analytic. If
the Cauchy problem is locally solvable at the origin in the space
$\mathscr{E}$ then the linear mapping $\psi (x) \to u (x,  t)$ is
continuous from $\mathscr{E} ( U) $ in to $\mathscr{E}^1 (D_{\in _0})$. 
\end{theorem}

\begin{proof}
The graph of the mapping $ \psi \to u$ is closed in $ \mathscr{E} ( U)
x \mathscr{E}^1 (D_{\in _0} )$ because of the uniqueness of the
solution of $ M [u ] = 0$,  with $ u(x,  0) = \psi (x)$ in $ D_{\in
  _0}$. Hence by the closed graph theorem of Banach the mapping is
continuous. 
\end{proof}

This\pageoriginale leads us to the notion of well-posedness of the
Cauchy problem in the sence of Hadamard. This we consider in the
following section.   

\section[Well-posedness and uniform-well
  posedness.....]{Well-posedness and uniform-well posedness
  of\hfill\break the Cauchy problem}\label{chap2-sec2} 

By a $k$-times differentiable function on a closed interval $ [0,  h]$
we mean the restriction to $ [0,  h]$ of a $k$-times continuously
defferentiable function on an open interval containing $ [0,  h]$.  

The space of continuous functions of $t$ in $ [0,  h]$ with values in
the space $\mathscr{E}^m _x$ is denoted by $ \mathscr{E}^m [ 0,
  h]$. It is provided with the topology of uniform convergence in the
topology of $ \mathscr{E}^m_x$ (uniform with respect to  $t$ in $[0,
  h]$). In other words,  a sequence $ \varphi_n \in \mathscr{E}^m [ 0,
  h]$ converges to 0 in the topology of $\mathscr{E}^m [ 0, h]$ if 
$\varphi_{n}(t)=\varphi_n (x,  t) \to 0$ in $\mathscr{E}^m_x$
uniformly with respect 
to $t$ in $ [0, h]$. A vector valued function $ u = (u_1,  \ldots ,
u_N)$ is said to belong to $\mathscr{E}^m [0,  h]$ if each of its
components $u_j$ belong to $\mathscr{E}^m [0,  h]$.  

Similarly one can define the spaces $\mathscr{B}^m [ 0,  h] \cdot D^s
_{L^2} [0,  h]$,  $L^2 [0, h] = D^0_{L^2} [0,  h]$ etc. These will be
the spaces which we shall be using in our discussions hereafter. We
also write $ \mathscr{B} [ 0,  h]$, $\mathscr{E} [ 0,  h]$, $D _{L^2}
[0,  h]$ instead of $ \mathscr{B}^\infty [ 0,  h]$,
$\mathscr{E}^\infty [ 0,  h]$, $D^\infty _{L^2} [0,  h]$. Following
Petrowesky \cite{key2} we give the  

\begin{defi*}
The forward Cauchy problem for a first order system $M$ is said to be
well posed in the space $\mathscr{E}$ in an interval $ [0,  h]$  if  
\begin{enumerate}[(1)]
\item for any given function $ f$ belonging to $ \mathscr{E} [0,  h]$
  and any Cauchy data $ \psi \in \mathscr{E}_x$ there exists a unique
  solution $u$ belonging to $ \mathscr{E} [0,  h]$ and once\pageoriginale
  continuously differentiable with respect to $t$ in $[0, h]$ (with
  its first derivative w.r.t. $t$ having its values in $
  \mathscr{E}_x$) of $M[u]=f$ with $u(x, 0)=\psi (x)$;  and  

\item the mapping $ (f,  \psi ) \to u $ is continuous from $
  \mathscr{E} [0,  h] \times  \mathscr{E}_x$ into $\mathscr{E} [0 ,
    h]$.  
\end{enumerate}
\end{defi*}

\begin{defi*}
The forward Cauchy problem for a first order system $ M$  is said to
be uniformly well posed in the space $\mathscr{E}$ if for every $t_0
\in [0,  h]$ the following condition is satisfied:  
\begin{enumerate}[(1)]
\item for any given function $f$ belonging to $\mathscr{E} [0, h]$
  and any Cauchy data $ \psi \in \mathscr{E}_x$ there exists a unique
  solution $u = u (x, t, t_0)$ belonging to $\mathscr{E} [t_0,  h]$
  and once continuously differentiable with respect to $t$ in $[t_0,
    h] $ (the first derivative having its values in $ \mathscr{E}_x $)
  of $M[u]=f$ with $u(x, t_0,  t_0) = \psi (x)$; and  

\item the mapping $ (f,  \psi) \to u$ is uniformly continuous from
  $\mathscr{E} [0, h]$, $\mathscr{E}_x $ into $ \mathscr{E} [t_0,
  h]$.  
\end{enumerate}
\end{defi*}

The condition of uniform continuity can also be analytically
described as follows: given an integer $l$ and a compact set $K$ of
$\underbar{R}^n$ there exists an integer $l'$,  a compact set $K'$ of $
\underbar{R}^n$ and a constant $C$ \textit{(all independent of $t_0$
  in $[0, h]$) such that}   
\begin{equation}
\sup\limits_{t_0 \leq t \leq h} |  u(x, t, t_0) |_{\mathscr{E}_K^l}
\leq C( |\psi (x)|_{\mathscr{E}_{K'}^{l'}} + \sup\limits_{0 \leq t
  \leq h} | f (x,  t)| _{\mathscr{E}_{K'}^{l'}}  \tag{2.1}\label{chap2-eq2.1}  
 \end{equation}
 where $|g(x) |_{\mathscr{E}_{K}^{r}}  = \sup\limits_{\substack {x
     \in K \\ 0 \leq | \nu | \leq r}} |\left(\dfrac{\partial}{\delta
   x}\right)^\nu g(x) |$. 

Similar\pageoriginale statements hold also for the spaces
$\mathscr{B}$ and $D^\infty _{L^2}$.   

We shall now give some criteria for the well posedness of the forward
Cauchy problem for first order systems $M$. For this purpose we
introduce the notions of characteristic equation and of the
characteristic roots of a first order system $M$.  

The polynomical equation 
\begin{equation}
\det \left(\lambda I - i \sum A_k (x, t) \xi_k - B(x, t)\right) = 0
\tag{2.2}\label{chap2-eq2.2}    
\end{equation}
is called the characteristic equation of $M$ and the roots $\lambda_1
(x, t, \xi) , \ldots ,\break \lambda_N (x, t, \xi)$ of this 
equation are called the characteristic roots of $M$.  

It will be useful for our future considerations to introduce the
notions of characteristic equation and of characteristic roots for a
single equation of order $m$ of the form  
\begin{equation}
L= \left(\frac{\partial}{\partial t}\right)^m +\sum_{\substack {|\nu | +j \leq m
    \\ j \leq m-1}} a_{\nu, j} (x, t) \left(\frac{\partial}{\partial
  x}\right)^\nu 
\left(\frac{\partial}{\partial t}\right)^j. \tag{2.3}\label{chap2-eq2.3}   
\end{equation}

Consider the principal part of $L$ and write it in the form  
\begin{equation}
\left(\frac{\partial}{\partial t}\right)^m + \sum^{m-1}_{j=0} a_j \left(x, t,
  \frac{\partial}{\partial x}\right) \left(\frac{\partial}{\partial
    t}\right)^j 
  \tag{2.4}\label{chap2-eq2.4}   
\end{equation}
where $ a_j (x, t, \xi) = \sum\limits_{|\nu| = m-j} a_{\nu,  j} (x, t)
\xi^\nu$ is a homogeneous polynomial in $\xi$ of degree $m-j$. The
characteristic equation of $L$ is defined to be  
\begin{equation}
\lambda^m + \sum^{m-1}_{j = 0} a_j (x, t, \xi) \lambda^j = 0
\tag{2.5 }\label{chap2-eq2.5}  
\end{equation}
and\pageoriginale its roots are called the characteristic roots of
$L$.   

We remark here that if we take $\left(u, \dfrac{\partial  u}{\partial  t},
\ldots,  \left(\dfrac{\partial }{\partial t}\right)^{m-1}u\right) $ as
a system of unknown functions,  say $(u_1,  u_2,  \ldots,  u_m)$,  we have  
\begin{equation}
\frac{\partial}{\partial  t}\begin{pmatrix}u_1 \\ \vdots
  \\ u_m \end{pmatrix} = \begin{pmatrix} 0 & 1 & 0 \ldots & 0\\ 0 & 0
  & 1 \ldots & 0\\  & & \ddots & \\  0 & 0 & 0 \ldots & 1\\ -a_0  -a_1
  & -a_2 & \ldots & -a_{m-1} \end{pmatrix} \begin{pmatrix}u_1
  \\ \vdots \\ u_m \end{pmatrix} \equiv H\left(x, t,
\frac{\partial}{\partial x}\right) \begin{pmatrix}u_1 \\ \vdots
  \\ u_m \end{pmatrix}\tag{2.6}\label{chap2-eq2.6}   
\end{equation}
and $\det (\lambda I - H(x, t, \xi)) = \lambda^m + \sum^{m-1}_{j=0}
a_j (x, t, \xi) \lambda^j$. Thus the characteristic roots of $L$ are
the same as those of the system \eqref{chap2-eq2.6}.  

We now obtain necessary and sufficient condition for the well
posedness of the Cauchy problem for first order systems in the case
where the coefficients depend only on $t$:  
\begin{equation}
\frac{\partial u}{\partial t} = \sum A_k (t) \frac{\partial
  u}{\partial  x_k} + B( t ) u. \tag{2.7}\label{chap2-eq2.7}   
\end{equation}

These conditions depend on the nature of the roots of its
characteristic equation   
\begin{equation}
\det (\lambda I - i \sum A_k (t) \xi_k - B(t)) = 0
\tag{2.8}\label{chap2-eq2.8}    
\end{equation}
In the case where $A_k$ and $ B$ are constant matrices, we have the
following proposition. 

\setcounter{proposition}{0}
\begin{proposition}[Hadamard]\label{chap2-sec2-prop1} %proposition 1
 Let the coefficients $A_k$ and $B$ of $M$ be constants. A necessary
 condition in order that the forward Cauchy 
  problem for $M$ be well posed in the space $\mathscr{B}$ is that
  there exist constants $c$ and $p$ such that  
\begin{equation}
\re \lambda _j (\xi ) \leq p  \log (1 + |\xi | ) + c \quad (j=1,  \ldots
,  N). \tag{2.9}\label{chap2-eq2.9}   
\end{equation}\pageoriginale
\end{proposition}

\begin{proof}
Assume that the forward Cauchy problem for $M$ is well posed but the
condition \eqref{chap2-eq2.9} is not satisfied. First of all we observe that, if
$\lambda (\xi)$ is any characteristic root of $M$  there exists a
non-zero vector $P (\xi) \in \underbar{C}^N$ with $|P(\xi)|= 1$
such that  
$$
\left( \lambda (\xi) I - i \sum A_k \xi_k - B\right) P (\xi ) = 0.  
$$

Then $u(x,  t) = exp (\lambda (\xi ) t + ix \cdot \xi )$. $P (\xi)$ is a 
solution of $M [ u ] = 0$. By assumption for any $p  > 0$ there
exists a vector $\xi$, $| \xi | \geq 2$,  and a characteristic root
$\lambda (\xi)$ such that,   
$$
\re  \lambda (\xi) \geq p \log (1+ | \xi |). 
$$

For this $ \lambda (\xi )$ we have 
\begin{enumerate}[(i)]
\item $M [ u ] \equiv M [ \exp (\lambda (\xi) t + ix. \xi 
  )\cdot P (\xi ) ] = 0$; 

\item $| u (x,  t) | = \exp (\re \lambda (\xi) t)\cdot |P (\xi) | \geq ( 1 
  + | \xi | )^{pt}$ for $t > 0$; and  

\item $\sum_{| \nu | \leq l} | \left(\dfrac{\partial}{\partial
  x}\right)^\nu u (x, 0) | \leq C (1) (1+ |\xi | )^l$. 
\end{enumerate}

The inequalities (ii) and (iii) show that the forward Cauchy problem
is not well posed which contradicts the assumption. Hence Proposition
\ref{chap2-sec2-prop1}  is proved.  
\end{proof}

For a smooth function $u$ (for instance a function in $L^2$ or
$\mathscr{S}$) 
the Fourier transform $ \widehat{u}$ with respect to $x$ is defined by    
\begin{equation}
\widehat{u} (\xi ,  t) = \int u(x, t) \exp (-2 \pi ix. \xi )
dx. \tag{2.10}\label{chap2-eq2.10}   
\end{equation}

More precisely if $ u$ belongs to $\mathscr{S}'$ then its Fourier
image is
denoted by $\widehat{u}$ and $\widehat{u}$ belongs to $\mathscr{S}'$. 

Let\pageoriginale us now assume that the coefficients $A_k$ and $B$ of
$M$ are continuous functions of $t$ in $[0,  h] $ but do not depend on 
$x$. Consider the system of ordinary differential equations  
\begin{equation}
\frac{d}{dt} \widehat{u} (\xi ,  t) = \left(2 \pi i \sum_{k} A_k (t) \xi_k
+ B(t) \right) \widehat{u} (\xi,  t). \tag{2.11}\label{chap2-eq2.11}   
\end{equation}

If $v^j_0$ denotes the vector in $\underbar{R}^N$ whose $ j^{\rm th}$
component is $1$ and the other companents are $0$,  let $v^j (\xi, t,
t_0)$ be the fundamental system of solutions of the system \eqref{chap2-eq2.11}
(defined in $[t_0, h]$) with the initial conditions $ v^j (\xi,
t_0,  t_0) = v_0^j$. Then we have the  

\begin{proposition}[Petrowsky]\label{chap2-sec2-prop2}
 Let the coefficients $A_k$ and $B$ of $M$ be
 continuous functions of $t$ in $ [0, h]$. A necessary condition in
 order that the forward Cauchy problem for $ M$ be uniformly well
 posed in the spaces $ \mathscr{B} $ and $ \mathscr{D}^\infty _{L^2}$
 is that there exist constant $c$ and $p$,  both independent of $t_0$
 in $[0, h]$,  such that  
\begin{equation}
| V^j (\xi , t,  t_0 ) | \leq c(1 +|\xi | )^p. \tag{2.12}\label{chap2-eq2.12}   
\end{equation}
\end{proposition}

\begin{proof}
{\em Necessity in the space} $ \mathscr{B}$. Assume that the forward
Cauchy problem is uniformly well posed in the  space $ \mathscr{B}$
but the condition \eqref{chap2-eq2.12} is not fulfilled. Then for any $p$,  one can
find $ \xi^*$, $t^*$,  $t^*_0$ and $k$ such that we have the inequality  
$$
| V^k (\xi^*,  t^*,  t^*_0) | \geq p(1+ |\xi^*| )^p.
$$

The function $u(x, t, t^*_0) = (u_1( x,  t, t^*_0), \ldots,  u_N (x,
t, t^*_0))$ with  
\begin{equation}
u (x, t, t^*_0) = \exp (ix. \xi^* ) \cdot v^k (\xi ^* , t, t^*_0),  t \in [
  t_0, h] \tag{2.13}\label{chap2-eq2.13}   
\end{equation}
is a\pageoriginale solution of $ M [ u] = 0$ and satisfies the inequalities  
\begin{enumerate}[(i)]
\item $ | u (x, t^*,  t^*_0) | \geq p(1+ |\xi ^*|)^p$ where $ t^*_0
  \leq t^* \leq h$,  and  

\item $\sum_{| \nu | \in 1} \left|\left(\dfrac{\partial}{\partial
  x}\right)^\nu u \left(x,  t^*_0,  t^*_0\right)\right| \leq c(l) (1+ |\xi ^*| )^l$,  
\end{enumerate}
$c(l)$ being a constant depending only on $l$ which again show that the
forward Cauchy problem is not uniformly well posed,  thus arriving at
a contradiction to the assumption.  
\end{proof}

\smallskip
\noindent
\textbf{Necessity in the space $\mathscr{D}^\infty _{L^2}$}. Again
assume that the forward Cauchy problem is uniformly well posed in 
$\mathscr{D}^\infty_{L^2}$ but the condition \eqref{chap2-eq2.12} 
does not hold. We
can therefore assume that for any $p$,  there exist $\xi^*$, $t^*$,
$t^*_0$ and $k$ such that we have the inequality  
$$
|V^k (\xi, t^*, t^*_0) | \geq p (1+ |\xi| )^p ,  t^* \geq t^*_0.  
$$
holds for all $\xi$ in a neighbourhood $U$ of $\xi_0^*$ in
$\underbar{R}^n$. Let $f \in L^2$ with its support contained in $U$ 
and $|| f || =1$. Then the function $u (x, t, t^*_0) = (u_1 (x, t,
t^*_0),  \ldots , u_N (x, t, t^*_0)$,  with  
\begin{equation}
u(x, t, t^*_0) = \int exp (ix. \xi ) v^k (\xi ,  t, t_0^*) f (\xi )
d\xi \text{ for } t \geq t^*_0,  \tag{2.14}\label{chap2-eq2.14}   
\end{equation}
is a solution of $M[ u ] = 0$. By Plancheral's theorem we have  
\begin{align*}
|| u ||  &= (2 \pi )^{n/2} (\int |v^k(\xi , t, t_0^*) |^2 |f (\xi)|^2
d \xi )^{\frac{1}{2}}\\   
& \geq (2\pi)^{n/2} p(1 +| \tilde{\xi}^*| )^p\tag{2.15}\label{chap2-eq2.15}  
\end{align*}
where $| \tilde{\xi}^*| =$ dist $(0, \supp. f)$. On the other hand
again by applying Plancheral's theorem we have,  for any $1$, that  
\begin{gather*}
\sum\limits_{| \nu | \leq 1} || \left(\frac{\partial}{\partial
  x}\right)^\nu u (x,t^*_0, t^*_0)|| = \sum\limits_{\substack{| \nu |
    \leq \ell\\ \leq c(l)(1+|\xi^*|^l}}(2\pi )^{n/2} \left( \int|
\xi^\nu v^k (\xi, t^*_0, t^*_0)^2 |f (\xi) |^2d\xi\right)^{\frac{1}{2}}
\tag{2.16}\label{chap2-eq2.16}  
\end{gather*}\pageoriginale
where $c(l)$ is a constant depending only on 1. The two inequalities
\eqref{chap2-eq2.15}, \eqref{chap2-eq2.16} 
together show that the forward Cauchy problem is
not uniformly well posed leading to a contradiction to the assumption.  

\begin{proposition}[Petrowsky]\label{chap2-sec2-prop3}%3 
 Let the coefficients $A_k$ and $B$ of $M$ be continuous functions of
 $t$. Then the condition \eqref{chap2-eq2.12} is sufficient in order that the
 forward Cauchy problem be uniformly well posed in the spaces $
 \mathscr{D}^\infty _{L^2}$, $\mathscr{B}$ and $ \mathscr{C}$. 
\end{proposition}

\begin{proof}
\textbf{Sufficiency in the space $\mathscr{D}^\infty _{L^2}$}. The
 inequality \eqref{chap2-eq2.12} 
$$
|v^j (\xi, t,  t_0) | \leq c(1+| \xi |)^p 
$$
shows that there exists a $\sigma$ such that $ (1+ | \xi |)^\sigma
v^j (\xi, t, t_0) \in \mathscr{B}^{0}_{\xi} $ and this depends
continuously on $(t,t_0)$. In fact, $ v^j (\xi,  t,  t_0)$
satisfies  \eqref{chap2-eq2.11}
\begin{equation*}
\frac{d}{dt} v^j (\xi,  t,  t_0) = (i \wedge \cdot \xi + B) v^j (\xi,
t,  t_0),  A. \xi = \sum A_k \xi _k   
\end{equation*}
consider
$$
V^j (\xi, t, t_o) -v^j (\xi,  t_0,  t_0) = \int\limits^t_{t_0} (iA
(s) \cdot \xi + B (s)) v^j (\xi,  s, t_0)ds.  
$$

This implies that $(1+ |\xi |)^{-p-1} v^j (\xi, t, t_0)$ is
continuous in $(t, t_0) $ in the space $\mathscr{B}^0_\xi
$. Hence the inverse Fourier image $R^j _x (t,  t_0)$ of $ V^j (\xi,
t, t_0)$ with respect to $\xi$ belongs to $ \mathscr{S'}$ and the
operator $ R^j_x(t, t_0) *_{(x)}$ has the following properties: 
\begin{enumerate}[(1)]
\item  for\pageoriginale any $ \varphi \in \mathscr{D}_{L^2}^\mathscr{S}$, $
  R^j_x(t, t_0) * _{(x)} \varphi \in \mathscr{D}_{L^2}^{s+ \sigma}
  [t_0,  h]$ and  

\item for any $ f \in \mathscr{D}_{L^2}^s [0, h]$, the integral
$$
 \int \limits^t_{t_0} R^j_x(t, \tau) *_{(x)}f(x,  \tau)d \tau 
 $$
belongs to $\mathscr{D}_{L^2}^{s+ \sigma} [t_0,  h]$. Further the
linear mappings  
 \begin{equation}
\varphi \to R^j_x(t, t_0) * _{(x)} \varphi ,  f \to \int
\limits^t_{t_0} R^j_x(t, \tau) * _{(x)}f(x,  \tau )d \tau
\tag{2.17}\label{chap2-eq2.17}    
 \end{equation} 
are continuous. Now given $ \psi = (\varphi _1,  \ldots,  \varphi_N)$
with $ \varphi_j \in \mathscr{D}_{L^2}^s$ and\break $f= (f_1,  \ldots ,
f_N)$ with $ f_j \in \mathscr{D}_{L^2} [0,  h]$ define $ u(x,  t,
t_0) = (u_1 (x, t, t_0),\break  \ldots,  u_N(x, t, t_0))$ by  
\begin{equation}
u(x, t, t_0) = \sum\limits_j R^j_x (t,  t_0) *_{(x)} \varphi_j (x)
+\int 
\limits^t_{t_0} R^j_x(t, \tau) *_{(x)}f_j(x,  \tau )d \tau .
\tag{2.18}\label{chap2-eq2.18}   
\end{equation}
\end{enumerate}

Then $ u (x, t, t_0)$ is a solution of $ M[u] = f $ with the Cauchy
data $ u (x, t_0, t_0) = \psi (x)$. In view of \eqref{chap2-eq2.18} we conclude
that the forward Cauchy problem is uniformly well posed in the space 
$\mathscr{D}_{L^2}^\infty$. 

\smallskip
\noindent
 \textbf{Sufficiency in the space $\mathscr{B}$}. We recall that 
$(v^j (\xi,  t,  t_0))$ is a fundamental 
system of solutions of the system \eqref{chap2-eq2.11}  
$$
\frac{d}{dt} v= (2 \pi i \sum A_k(t) \xi_k + B(t)) V. 
$$

Hence each $v^j (\zeta, t, t_0)$ is an entire function of exponential
type for complex $ \zeta \in \underbar{C}^n$. In fact,  if $ | v(J,
t, t_0)|^2$ stands for $ \sum\limits _{j=1 }^{N}v^j (\zeta , t,
t_0)|^2$,  we have since $ A_k (t)$ and $ B(t)$ are bounded 
\begin{equation}
|(2\pi i \sum A_k (t) \zeta _k + B(t)) v (\zeta,  t, t_0) | \leq c
(1+|\zeta|)| v (\zeta, t, t_0)| \tag{2.19}\label{chap2-eq2.19}   
\end{equation}\pageoriginale
with a constant $c$ independent of $\zeta$ and $v$,  Further  
\begin{align*}
\frac{d}{dt} | v (\zeta, t, t_0) |^2 &= \sum_j (\frac{dv^j}{dt} (
\zeta,  t, t_0). \overline{v^j (\zeta, t, t_0)} + v^j(\zeta, t, t_0)
\overline{\frac{dv^j}{dt} (\zeta,  t, t_0))}\\ 
&\leq 2 \Big| \frac{d}{dt} v (\zeta, t, t_0) | |v (\zeta,  t, t_0)\Big|\\
& = 2 | (2 \pi i \sum A_k (t) \zeta + B (t)) v (\zeta,  t, t_0)|| v
(\zeta , t, t_0)|\\ 
&\leq 2c' | v (\zeta, t, t_0)|^2 (1+|\zeta | ).
\end{align*}

Hence $|v (\zeta, t, t_0 ) | \leq c'' e^{c' (1+ |\zeta |)|t-t_{0}|}$
and consequently for large $\zeta$,  we have,  for each 
$j=1, \ldots , N$ the inequality 
$$
|v^j (\zeta, t, t_0)| \leq c_1 e^{c_2 |\zeta| |t-t_0|} 
$$

Hence by Paley-Wiener's theorem $R^j_x (t, t_0)$ is a distribution
with compact support contained in $\{(x, t) \in \underbar{R}^{n+1}|
|x| < c_2 |t-t_0| \}$ and depends continuousuly on $(t, t_0)$. By the
structure of distribution with compact supports we can wrte  
\begin{equation}
R^j_x(t, t_0) = \sum\limits_{|\nu| \leq s_j} (\frac{\partial}{\partial
  x}) [g^j_\nu (x, t, t_0)] (j=1, \ldots, N), \tag{2.20}\label{chap2-eq2.20}   
\end{equation}
where $g^j_\nu(x, t, t_0) \in \mathscr{B}^0_x [t_0, t]$ with support
contained in $\{x \big| |x|< c_3\}$ and the derivatives are taken in
the sense of distributions. This implies that  
\begin{itemize}
\item[(1)] for any $\varphi \in \mathscr{B}$ we have $R^j_x(t, t_{0}) *_{(x)} 
\varphi \in \mathscr{B}[t_0, h]$, 
 
\item[(2)] for any $f \in \mathscr{B} [0, h]$ the integral
$$
\int\limits^{t}_{t_0} R^j _x (t,  \tau)^* _{(x)} f (x,  \tau) d\tau
\in \mathscr{B} [t_0 ,  h]. 
$$\pageoriginale
\end{itemize}

Further the linear maps 
\begin{equation}
\varphi \to R^j_x (t,  t_0) *_{(x)} \varphi ,  f \to
\int\limits^{t}_{t_0} R^j_x (t,  \tau) *_{(x)} f(x,  \tau ) d \tau
\tag{2.21}\label{chap2-eq2.21}   
\end{equation}
are continuous. Now the same argument as in the first part of the
proposition shows that the Cauchy problem is uniformly well posed in
the base $\mathscr{B}$.

\smallskip
\noindent
\textbf{Sufficiency in the space $\mathscr{E}$}. In the above proof
we observe that, since $R^j_x (t,  t_0)$ is a distribution with
compact support,  we have  
\begin{enumerate}[(1)]
\item for any $\varphi \in \mathscr{E}$, $R^j_x (t, t_0) *_{(x)}
  \varphi \in \mathscr{E} [t_0, h]$,  

\item for any $f \in \mathscr{E} [O,  h]$ the integral  
$$
\int\limits^{t_0}_{t} R^j_{x} (t,  \tau) *_{(x)} f (x, \tau) d\tau 
$$
belongs to $\mathscr{E} [t_0, h]$. Again the linear maps
$$
\varphi \to R^j_x (t,  t_0) *_{(x)} \varphi,  f \to
\int\limits_{t_0}^{t} R^j_x (t, \tau) *_{(x)} f (x, \tau) d\tau 
$$
are continuous and an argument similar to the one used earlier shows
that the forward Cauchy problem is uniformly well posed in the space
$\mathscr{E}$. 
\end{enumerate}

This completes the proof of the proposition.
\end{proof}

\section{Cauchy problem for a single equation of order
  $m$}\label{chap2-sec3} %%% 3 

By an argument similar to thye ones used in the previous section we
shall presently prove a necessary and sufficient condition in order
that the forward Cauchy problem for a single equation of order $m$ be
uniformly well posed in the space $\mathscr{E}$. Let 
\begin{equation*}
L\equiv \left(\frac{\partial}{\partial t}\right)^m + \sum_{\substack{|
    \nu | +j \leq  m \\ j \leq m-1}} a_{\nu ,  j} (t)
\left(\frac{\partial}{\partial x}\right)^{\nu}
\left(\frac{\partial}{\partial t}\right)^j \tag{3.1}\label{chap2-eq3.1}    
\end{equation*}
be a\pageoriginale linear differential operators of order $m$ whose
coefficients 
$a_{\nu, j}(t)$ are $(m-1)$ times continuously differentiable
functions of $t$ in an interval $[0, h]$. By Fourier transforms in the
$x$-space we are lead to the following oridinary differential equation
of order $m$ with $(m-1)$-times continuously differentiable
coefficients in $t$: 
\begin{equation*}
\tilde{L} [V] \equiv \left(\frac{d}{dt}\right)^m v(\xi,  t) +
\sum_{\substack{|\nu| + j \leq m\\j \leq m-1}}a_{\nu,j}(t) (i \xi
)^\nu \left(\frac{d}{dt}\right)^j v (\xi, t) =
0. \tag{3.2}\label{chap2-eq3.2}      
\end{equation*}

Let $v(\xi, t, t_0)$ be a solution of $\tilde{L}[v] = 0$ satifying the
initial conditions on $(t=t_0)$. 
$$
v(\xi, t_0, t_0) = 0, \ldots ,  (\frac{d}{dt})^{m-2} v(\xi,  t_0, 
t_0) = 0,  (\frac{d}{dt})^{m-1} v(\xi,  t_0 ,  t_0) =1. 
$$

Then we have the 

\setcounter{proposition}{0}
\begin{proposition}\label{chap2-sec3-prop1} %proposition 1
If the coefficients $a_{\nu, j}$ of $L$ are $m-1$ times continuously
diffenentiable functions of $t$ in an interval $[0, h]$ the forward
Cauchy problem for $L$ is uniformly well posed in the space
$\mathscr{E}$ if and only if there exist constants $c$ and $p$ both
independent of $t_0$ such that  
\begin{equation*}
| v (\xi,  t, t_0)| \leq c(1+|\xi|)^p. \tag{3.3}\label{chap2-eq3.3}    
\end{equation*}
\end{proposition}

\begin{proof}
Suppose the Cauchy problem for $L$ is uniformly well posed for the
future in the space $\mathscr{E}$ but the condition \eqref{chap2-eq3.3} does not
hold. Then for any given $p>0$ there exist $\xi^*$, $t^*_0$ and $t$,
$t\geq t^*_0$,  such that we have the inequality 
$$
|v (\xi^*,  t, t^*_0 ) | \geq p(1+ | \xi^* |)^p.
$$

Then  The function $u(x, t, t^*_0) = \exp (ix. \xi^*) v (\xi^*, t,
t_0^*)$ is a solution of $Lu=0$ and has the properties. 
\begin{enumerate}[(i)]
\item $u(x, t, t^*_0) \in \mathscr{E} [t_0^*, h]$ and once
  continuously differentiable in $t$ with values in $\mathscr{E}_x$,  

\item $|u (x, t, t^*_0) | = | v (\xi^* ,  t,  t^*_0)| \geq p
  (1+|\xi^*|)^p$, and\pageoriginale 

\item $\sum\limits_{|\nu| \leq 1}\left| \left(\dfrac{\partial}{\partial
  x}\right)^\nu u (x, t^*_0,  t^*_0) \right| = \sum\limits_{|\nu|\leq 1}
  \left| 
  (i \xi^*)^\nu v (\xi^*,  t^*_0,  t^*_0) \right |  \leq c(l)
  (1+|\xi^*|)^l$ 
\end{enumerate}

The last two inequalities together show that forward Cauchy problem is
not uniformly well posed in the space $\mathscr{E}$ which
contradiction the assumption.  

Conversely,  assume that the condition \eqref{chap2-eq3.3} is
satisfied. The forward Cauchy problem is uniformly well posed in 
the space $\mathscr{E}$. First of all we prove that the
condition \eqref{chap2-eq3.3} implise that $v (\xi, t, t_0)$ and all
its derivatives 
upto order $(m-1)$ with respect to $t$ are uniformly majorized in
$[t_0, h]$ by polynominals in $\xi$. For this purpose we rewrite the
equation $\tilde{L}[V] =0$ in the form  
\begin{equation}
(\frac{d}{dt})^m v (\xi, t, t_0) + \sum\limits^{m-1}_{j=0} a_j (t,
  \xi ) (\frac{d}{dt})^j v(\xi, t, t_0) = 0 \tag{3.4}\label{chap2-eq3.4}     
\end{equation}
where $a_j(t,\xi)= \sum\limits_{| \nu | = m-j} a_{\nu,  j} (t)
(i\xi)^\nu $ for $j=0, 1, \ldots, (m-1)\, a_j (t, \xi)$ are hence
polynominals of degree at most $(m-j)$ in $\xi$ with coefficients
which are $(m-1)$-times continuously differentiable functions of $t$ in
the interval $[0, h]$. Hence we may assume that there exists a
constant $c$ such that 
\begin{equation*}
|a_j (t,  \xi ) | \leq c (1+|\xi | )^{m-j},  j=0, 1, \ldots , (m-1)
\text{~ for~ } t \in [0, h] \tag{3.5}\label{chap2-eq3.5}     
\end{equation*}

Integrating \eqref{chap2-eq3.4} once with respect to $t$ over the
interval [$t_0$, $h$] we obtain,  after using the initial conditions
at $t=t_0$, 
$$
\left(\frac{d}{dt}\right)^{m-1} v (\xi, t, t_0) -1 = - \sum\limits^{m-1}_{j=0}
\int\limits^{t}_{t_0} a_j (\tau,  \xi) \left(\frac{d}{d\tau}\right)^j v (\xi,
\tau ,  t_0) d \tau. 
$$

Integrating\pageoriginale by parts the terms in the right hand side in
view of the initial conditions satisfied by $v(\xi, t, t_0)$ we
obatain  
{\fontsize{10pt}{12pt}\selectfont
\begin{align*}
\left(\frac{d}{dt}\right)^{m-1} v(\xi,  t,  t_0) - 1 &= - \sum\limits^{m-1}_{j=0}
\left\{\sum\limits^{j-1}_{p=0} (-1)^p \left(\frac{\alpha}{dt}\right)^p (a_j (t,
\xi)) \left(\frac{d}{dt}\right)^{j-1-p} v(\xi,  t, t_0)\right.\\
&\left. + (-1)^j
\int\limits^{t}_{t_0} \left(\frac{d}{d\tau}\right)^j (a_j (\tau,  \xi )) v (\xi,
\tau, t_0 ) d \tau\right\}. 
\end{align*}}\relax

By successive integration with respect to $t$ over the interval
[$t_0$, $h$] $(m-1)$-times,  using the initial conditions and the
inequality \eqref{chap2-eq3.5} we show that $\dfrac{d}{dt} v (\xi,  t,
t_0), \ldots , 
\left(\dfrac{d}{dt}\right)^{m-1} v (\xi, t, t_0)$ are all majorized by
polynominals of the form $c_j (1+|\xi|)^{p_j} (j=1, 2, \ldots , m)$,
$C_{j}, p_{j}$ being independent of $t_0$. 

Thus it follows that there exist $\sigma_0, \ldots,  \sigma_m$ such
that $(1+|\xi|)^{\sigma j} \left(\dfrac{d}{dt}\right)^j\break v (\xi, t, t_0) \in
\mathscr{B}^0_\xi [t_0,  h]$ for $j=0, 1, \ldots , (m-1)$. Let
$R^j_x(t,  t_0)$ denote the inverse Fourier image of
$\left(\dfrac{d}{dt}\right)^j v (\xi,  t, t_0)$ in the $\xi$-space. 

We shall show that each $R^j_x (t, t_0)$ has compact support in the
$x$-space. In view of the theorem of Paley-Wiener we have only to show
that each $\left(\dfrac{d}{dt}\right)^j v (\zeta,  t, t_0)$ are of
exponential type for complex $\zeta \in \underbar{C}^n$. 

Denoting $(1+|\zeta|)$ for $\zeta \in \underbar{C}^n$ by $K$ we have $|a_j
(t, \zeta) | \leq c K^{m-j}$ for all $j=0, 1, \ldots, m-1$. The equation
\eqref{chap2-eq3.4} can now be written in the form 
\begin{gather*}
\left(\frac{d}{dt}\right)^m v (\zeta, t, t_0) + a_{m-1} (t, \zeta) 
\left(\frac{d}{dt}\right)^{m-1} v(\zeta, t, t_0) + \frac{a_{m-2}}{K}
K\left(\frac{d}{dt}\right)^{m-2} v (\zeta,  t, t_0)\\
+\ldots + \frac{a_0(t,
  \zeta)}{k^{m-1}} K^{m-1} v (\zeta ,  t ,  t_0) =0 
\end{gather*}

Taking for the new set of function $w=(w_0, w_1, \ldots,  w_{m-1})$
where  
\begin{align*}
w_0 (\zeta, t, t_0) &= K^{m-1} v (\zeta,  t, t_0),\\ 
w_1 (\zeta,  t,  t_0) &= K^{m-2} \frac{dv}{dt} (\zeta ,  t,  t_0)\\ 
w_{m-2} (\zeta, t, t_0) &= K\left(\frac{d}{dt}\right)^{m-2} v (\zeta, t , t_0)\\
w_{m-1} (\zeta, t, t_0) &= \left(\frac{d}{dt}\right)^{m-1} v (\zeta, t, t_0).
\end{align*}
the above equation can be written as a system of oridinary
differential equations in the following way: 
\begin{equation}
\frac{d}{dt} \begin{pmatrix} w_0 \\ w_1 \\ \vdots
  \\ w_{m-1} \end{pmatrix} = K \begin{pmatrix} 0 & 1 & 0 & \ldots & 0
  & 0\\ 0 & 0 & 1 & \ldots & \cdot & \cdot \\ \cdot & \cdot & \cdot &
  \ldots & \cdot & 
  \cdot\\ -\frac{a_0}{K^m} &- \frac{a_1}{K^{m-1}} & -\frac{a_2}{K^{m-2}} &
  \ldots & -\frac{a_{m-2}}{K^2} &
  -\frac{a_{m-1}}{K} \end{pmatrix} \begin{pmatrix}  w_0 \\ w_1
  \\ \vdots \\w_{m-1} \end{pmatrix} \tag{3.6}\label{chap2-eq3.6}     
\end{equation}

Denote the matrix of the system \eqref{chap2-eq3.6} by $H(t,
\zeta)$. Since $|a_j(t,\zeta)| \leq c K^{m-j}$ the elements of the matrix $H(t, \zeta)$ are
bounded in modulus by a constant $C_1$ independent of $\zeta$ in
$\underbar{C}^n$ and hence $H(t, \zeta)$ as a linear transformation in
an $m$-dimensional vector space is bounded in norm by a constant $C_2$
which depends only on $m$ but not on $\zeta$ in
$\underbar{C}^n$. Denoting by $|w (\zeta, t, t_0)|^2$ the sum
$\sum\limits_{j} |w_j(\zeta, t, t_0)|^2$ and by $w(\zeta,
  t, t_0) \cdot \overline{w' (\zeta, t, t_0)}$ the sum
    $\sum\limits_{j} w_j (\zeta, t, t_0)\cdot 
\overline{w'_j (\zeta, t, t_0)}$ we have  
\begin{align*}
\frac{d}{dt} | w (\zeta,  t, t_0) |^2 &= \frac{d}{dt} w (\zeta,  t,
t_0)\cdot \overline{w (\zeta,  t, t_0)} + w (\zeta ,  t,  t_0)
\overline{\frac{d}{dt} w(\zeta,  t, t_0)}\\ 
&= K(H(t, \zeta ) + \overline{H(t, \zeta )})| w (\zeta, t, t_0)|^2 
\end{align*}
on account of the system of equation \eqref{chap2-eq3.6} satisfied by $w(\zeta, t,
t_0)$. Hence\pageoriginale  
$$
\left(\frac{d}{dt}\right) | w(\zeta,  t, t_0) |^2 \leq 2 C_2 K| w (\zeta,  t,
t_0)|^2 
$$
which, by integration with respect to $t$ over the interval $[t_0,
  t]$ implies that  
$$
|w (\zeta, t, t_0)|^2 \leq \exp (2 C_2 K| t-t_0|) = \exp 2C_2
(1+|\zeta)|t-t_0|  
$$
since $|w (\zeta,  t_0, t_0)|=1$ consequently we have, since $k\geq
1$,   
$$
\left| \left(\frac{d}{dt}\right)^j v(\zeta,  t,  t_0 ) \right| \leq
\exp [C_2 (1+|\zeta|)| t-t_o |]. 
$$

Hence,  by the theorem of Paley-Wiener it follows that $R^j_x (t, 
t_0)$ are distributions with compact support in the $x$-space and depend
continuously on $(t, t_0)$. 

Let $\Psi = (\varphi_0, \ldots , \varphi_{m-1})$ with $\varphi_j \in
\mathscr{E}_x$ and $f \in \mathscr{E} [0, h]$ be given. 

The above argument can be modified a little in order to get
convolution operators $\tilde{R}^j_x (t, t_0)$ similar to $R^j_x(t,
t_0)$. This we do as follows:   

Let $v_j (\xi, t, t_0)$ be the solution of $\tilde{L}[v_j] = 0$ with
the initial values given by  
$$
\left(\frac{\partial}{\partial t}\right)^i v_j (\xi, t, t_0)
\big|_{t=t_0} = \delta^j_i. 
$$ 
($\delta^j_i$ are Kronecker's symobls). We see that $v_j (\xi, t,
t_0)$ is connected with the solution $v (\xi, t, t_0)$ in the
following way. 

Let $w_j (\xi, t, t_0) = v_j (\xi,  t, t_0) - \dfrac{(t-t_0)}{j!}$, $t
\geq t_0$. Then $w_j$ vanishes at $t=t_0$ together with derivatives
upto order $(m-1)$. Now $w_j$ satisfies the equation. 
$$
\tilde{L} \left[w_j + \frac{(t-t_0)^j}{j!}\right] = 0 \text{ or } 
\tilde{L} [w_j] = - \frac{1}{j!} \tilde{L} [(t-t_0)^j] = ^\mu_j (\xi, 
t, t_0).  
$$\pageoriginale
$\mu_j (\xi, t, t_0)$ are obviously polynomials in $\xi$ and we have
$$
| \mu_j (\xi, t, t_0) | \leq c_3 (1+|\xi|)^m \text{~ for~ } 0\leq t_0
\le t \le h,  
$$ 
here $c_3$ is a constant. Hence 
$$
w_j (\xi,  t, t_0) = \int\limits^t_{t_0} v (\xi, t, \tau) \mu_j (\xi,
\tau, t_0) d\tau. 
$$ 

This implies that 
\begin{align*}
|w_j(\zeta, t, t_0)| &\leq \int\limits^t_{t_0} | v (\zeta, t, \tau) |
| \mu_j (\zeta,  \tau, t_0) d\tau \\ 
&\leq c_3 (t-t_0) (1+|\zeta|) ^m \exp [c_4 (1+|\zeta|) (t-t_0)]. 
\end{align*}

Hence the inverse Fourier image $\tilde{R}^j_x (t, t_0)$ of $v_j(\xi, 
t, t_0) = w_j (\xi, t, t_0) + \dfrac{(t-t_0)^j}{j!}$ has its support
in $|x| \leq c'_4 (t-t_0)$. 

Then the function
\begin{equation}
u(x, t, t_0) = \sum\limits^{m-1}_{j=0} \tilde{R}^j_x (t, t_0) *_{(x)}
\varphi_j + \int\limits^{t}_{t_0} R_x (t, \tau ) *_{(x)} f (x, \tau )
d\tau \tag{3.7}\label{chap2-eq3.7} 
\end{equation}
is a solution of $L[u] = f$ with Cauchy data $\Psi$ on $t=t_0$. (Here
$R_x (t, t_0)$) stand for the inverse Fourier image of $v(\xi, t,
t_0))$. The linear mappings 
\begin{equation}
\varphi_j \to R^j_x(t, t_0)*_{(x)} \varphi_j ,  f \to
\int\limits^{t}_{t_0} R_x (t, \tau ) *_{(x)} f (s, \tau ) d\tau
\tag{3.8}\label{chap2-eq3.8} 
\end{equation}
being continuous the forward Cauchy problem is uniformly well posed in
the space $\mathscr{E}$. This completes the proof of the proposition.  
\end{proof}

\section{}\pageoriginale\label{chap2-sec4}%% 4

\setcounter{proposition}{0}
\begin{proposition}\label{chap2-sec4-prop1}%%% proposition 1
Let the coefficients $A_k$ and $B$ of a first order system of
differential operators $M$ be continuous functions of $t$ in an
interval $[0, h]$. If the forward Cauchy problem is well posed in the
space $\mathscr{E}$ then it is uniformly well posed in $\mathscr{E}$. 
\end{proposition} 

\begin{proof}
In view of Prop. \label{chap2-sec3-prop3} of \S\ \ref{chap2-sec2} it
is sufficient to prove that if 
$v^j(\xi, t, t_0)$ is the fundamental system of solutions of the
system of oridinary differential equations 
\begin{equation}
\frac{d}{dt} v (\xi,  t, t_0) = (iA(t) \xi + B (t)) v (\xi,  t, t_0),
A(t) \cdot \xi = \sum A_k (t) \xi_k \tag{4.1}\label{chap2-eq4.1}  
 \end{equation} 
with intial conditions $v^j(\xi,  t_0, t_0) = v^j_0$ then $v^j(\xi,
t, t_0)$ are majorized by polynominals in $|\xi|$. (We recall that
$v^j_0$ denotes the vector in $\underbar{R}^N$  having 1 for the
$j^{\rm th}$ component and 0 for the others). If $\xi^0 =
\dfrac{\xi}{|\xi |}$ we can write the above system as 
\begin{equation}
\frac{d}{dt} v (\xi, t, t_0) = (i|\xi | A (t) \cdot  \xi^0 + B (t)) v (\xi,
t, t_0). \tag{$4.1'$} 
\end{equation}  

The element $a_{kl}(t, \xi^0)$ of the matrix $A(t) \cdot \xi^0$ are
homogeneous functions of $\xi^0$ of degree one having for coefficients
continuous functions of $t$ in $[0, h]$. We remark that $v^j(\xi, t,
0)$ define the columns of the Wronskian $W(t, \xi)$ of the above system
of differential equations. From the theory of  linear ordinary
differential equations we know that  
\begin{equation}
w (t, \xi ) = W(0, \xi ) \exp \bigg\{i |\xi | \sum\limits_j
\int\limits^{t}_{0} a_{jj} (\tau, \xi^0 )d\tau + \sum\limits_j
\int\limits^t_{0} b_{jj} (\tau) d \tau. \tag{4.2}\label{chap2-eq4.2} 
\end{equation}

The forward Cauchy problem being well posed we can assume that
$\sum\limits_{j} \int\limits^{t}_{0} a_{jj} (\tau,  \xi^0) d \tau
$\pageoriginale is 
real for every $(t, \xi^0)$, $\xi^0$ real. For otherwise we may
assume, if necessary by changing $\xi^0$ to - $\xi^0$ that  
$$
\re i \sum\limits_{j}\int\limits^{t}_{0} a_{jj} (\tau, \xi ^0) d\tau > 
0. 
$$ 

By the assumption of the well posedness of the forward Cauchy problem 
it follows that  
\begin{equation}
| v^j (\xi,  t, 0)| \leq c (1+|\xi|)^p \tag{4.3}\label{chap2-eq4.3} 
\end{equation}
for suitable constants $c$ and $p$, and so $W(t, \xi)$ is majorized
by a polynomial in $|\xi|$. On the other hand, as $\rho \to +
\infty$,  
$$
|w (t, \xi) | \sim |W(0, \xi)| \exp \bigg\{\rho |\xi^0 | \sum\limits_j
\re i \int\limits^{t}_{0} a_{jj} (\tau,  \xi^0)d \tau \bigg\}, \; 
\xi=\rho \xi^0. 
$$

Thus $W(t, \xi)$ tends to $+\infty$ exponentially as $\rho \to +
\infty$ contradicting the inequality \eqref{chap2-eq4.3}. Hence it follows that
$\sum\limits_{j} \int\limits^{t}_{0} a_{jj} (\tau,  \xi^0) d \tau $ is
real for every $(t, \xi^0)$ with real $\xi^{0}$. We now have  
$$
| W (t, \xi) | = | W(0, \xi) | \exp \bigg\{\sum\limits_{j} \re
\int\limits^{t}_{0} b_{jj} (\tau)d\tau 
$$
and hence 
$$
|W (t, \xi) | \geq | W(0, \xi) |\exp \left\{-\sum\limits_{j}
\int\limits^{t}_{0} | b_{jj} (\tau) | d \Gamma\right\} \geq \delta > 0 \text{~
  for all~ } (t, \xi). 
$$
$\xi$ real. Further we observe that, as $v^j (\xi, t, 0)$ form a basis 
for the solutions of the system of ordinary differential equations
$$
v^i (\xi, t, t_0) = \sum c^i_j (\xi) v^j (\xi ,  t, 0).
$$\pageoriginale  

Putting $t=t_0$ solving for $c^i_j (\xi)$ we see that,  since $ 
\det (v_j^i(\xi , t_0 ,  0))$ is the Wronskian $W(t_0, \xi)$
which is minorized by a polynomial in $|\xi|$ and since $v^j (\xi ,
t, 0)$ are majorized by polynomials in $|\xi|$, $c^i_j (\xi)$ are
themselves majorized by polynomials. Hence $v^j (\xi, t, t_0) $ are
majorized by polynomials in $|\xi|$ independently of $t$ and $t_0$
which implies that the forward Cauchy problem is uniformly well posed
for $M$. Hence proposition \ref{chap2-sec4-prop1} is proved.  

Correspondingly we have the following result for a single differential
equation of order $m$. Let  
\begin{equation}
  L \equiv \left(\dfrac{\partial}{\partial t}\right)^m + \sum _{\substack{|\nu| +
      j \leq m\\j\leq m-1 }} a_{\nu, j} (t)\left(\dfrac{\partial}{\partial
    x}\right)^\nu \left(\dfrac{\partial}{\partial t}\right)^j
  \tag{4.4}\label{chap2-eq4.4}  
\end{equation}
be a linear differential operator of order m with the oefficients
depending only on $t$ in the interval [$0, h$]. 
\end{proof}

\begin{proposition}\label{chap2-sec4-prop2}% proposition 2
Let the coefficients $ a_{\nu, j}$ of $L$  be $(m-1)$ times 
continuously differentiable fucntions of $t$ in an interval $[0, h]$.   
If the forward Cauchy problem for $L$ is well posed then it is
uniformly well posed for the future for $L$. 
\end{proposition}

\begin{proof}
Writing the operator $L$ in the form 
\begin{equation*}
\left(\dfrac{\partial}{\partial t}\right)^m +
\sum\limits^{m-1}_{j=0}a_j
\left(t,  \dfrac{\partial }{\partial x}\right) \left(\dfrac{\partial}{\partial
    t}\right)^j\tag{4.5}\label{chap2-eq4.5}  
\end{equation*}
where $a_j(t, \xi) =\sum\limits_{|\nu|=m-j} a_{\nu, j} (t) (i\xi)^\nu\
(j=0, 1, \ldots, m-1)$, we are lead to the following oridinary
differential equation of order $m$:  
\begin{equation}
 \left( \dfrac{d}{dt}\right)^m v (\xi,  t) + \sum\limits^{m-1}_{j=0} a_j(t, \xi)
  \left(\dfrac{d}{dt}\right)^j v(\xi, t) = 0.\tag{4.6}\label{chap2-eq4.6} 
\end{equation}\pageoriginale

Denoting the Wronskian of the equation \eqref{chap2-eq4.6} by $w(t,
\xi)$ we have 
from the theory of ordinary differential equations 
\begin{equation}
 W(t, \xi) = W(0, \xi ) \exp \left\{-\int\limits^t_0 a_{m-1} (\tau, \xi ) d
 \tau \right\}.  \tag{4.7}\label{chap2-eq4.7} 
 \end{equation} 

Write $a_{m-1}(\tau, \xi) = a^{(1)}_{m-1}(\tau, \xi) + b(\tau)$ where
 $a_{m-1}^{(1)}(\tau, \xi)$ is homogeneous in $\xi$ of degree one
 with coefficients continuous functions of $t$ in $[0, h]$. Then   
 $$
 a_{m-1}^{(1)} (\tau, \xi) = |\xi| a^{(1)}_{m-1} (\tau, \xi^0),  \xi =
 |\xi | \xi^0  
 $$ 
 and so we can write 
 $$
  W(t, \xi) = W(0, \xi) \exp \left\{-|\xi| \int\limits^t_0
  a^{(1)}_{m-1} (\tau, \xi^0 d \tau - \int\limits^t_0 b
  (\tau)d\tau\right\} .
  $$
\end{proof}

Now arguing as in the proof of the
proposition \ref{chap2-sec4-prop1} one can show that  
the Cauchy problem is uniformly well posed using again the
prop. \ref{chap2-sec3-prop3} of 
\S\ \ref{chap2-sec2}. Finally we shall show that for first order
systems with constant 
coefficients the condition of Hadamard implies the condition of
Petrowsky. This will prove that for first order systems with constant
coefficients these two conditons are equivalent. For this we need the  

\setcounter{lemma}{0}
\begin{lemma}[Petrowsky]\label{chap2-sec4-lem1} % lemma 1
 Let a system of differential equations with constant coefficients 
\begin{equation}
\frac{d}{dt}v(t) = A v(t) \tag{4.8}\label{chap2-eq4.8}
\end{equation}
where $A = (a_{jk})$ and $v(t) = \left(\begin{smallmatrix} v_1 (t)
  \\ \vdots \\ v_N (t) \end{smallmatrix} \right)$ with $|a_{jk}| \leq
K $ be given. Then,  given\pageoriginale any positive number
$\varepsilon $ such that $\varepsilon \leq (N-1) ! 2^N K$ we can find
a non-singular matrix $C$ such that   
\begin{equation}
CA = DC \text{~ where~ } D =
\begin{pmatrix}
a^*_{11} &  & & 0\\ 
 & a^*_{22}  & & \\ 
& & \ddots  & \\ 
a^*_{jk} & & & a^*_{nn} 
 \end{pmatrix}
\tag{4.9}\label{chap2-eq4.9} 
\end{equation}
where all $ a^*_{jk }$, $k < j$  satisfy  $|a^*_{jk}| <
 \varepsilon$. Moreover  
\begin{equation}
| \det C | =\left [\frac{(N-1) ! 2^N K}{\in}
  \right]^{\frac{N(N-1)}{2}}\tag{4.10}\label{chap2-eq4.10}  
\end{equation}
and the elements $c_{jk }$ of $C$ satisfy 
\begin{equation}
|c_{jk}| \leq \left[\frac{(N-1) ! 2^N K }{\varepsilon}
  \right]^{(N-1)}.  \tag{$4.10'$} 
\end{equation} 

For a proof see Petrowesky \cite{key2}.
\end{lemma}

\begin{proposition}\label{chap2-sec4-prop3} %proposition 3 
Let the coefficients $A_k$  and $B$ of $M$ be constants. Then the
condition 9 of \S\ \ref{chap2-sec2} 
of Hadamard implies the condition 12 of \S\ \ref{chap2-sec2}. 
\end{proposition}

\begin{proof}
Consider the system of ordinary differential equations
\begin{equation}
\frac{d}{dt}v(\xi, t) = (iA.\xi + B) v(\xi,  t). \tag{4.11}\label{chap2-eq4.11}
\end{equation}

Let us fix $\xi^0$. Taking ($iA. \xi^0  + B)$ as the given matrix in
the lemma \ref{chap2-sec4-lem1} there exist constants $c_0$, $c_1$ such that  
\begin{equation}
 |i a_{jk}(\xi^0 ) + b_{jk} |< c_0 |\xi |^0 + c_1 \tag{4.12}\label{chap2-eq4.12}
\end{equation}
($c_0, c_1$ are independent of $\xi^0$). We take $K=c_0 |\xi |^0 +
c_1$ and $\xi = (N-1) ! 2^N K = (N-1) ! 2^N (c_0|\xi|^0 + c_1)$. Then,
by the lemma \ref{chap2-sec4-lem1},  we can find a matrix $ C(\xi^0) $
such that $ (|\det C(\xi^0)| = 1 $   and its elements $c_{jk} (\xi^0)$ satisfy
$|c_{jk}(\xi^0) | \leq 1$. So denoting $c(\xi^0) v$ by $w$ we have  
\begin{equation}
\frac{d}{dt} w(\xi^0, t) = 
\begin{pmatrix}  
\lambda_1(\xi^0) & & &\\
 & \lambda_2(\xi^0) & & \\
 & & \ddots & \\ 
a^*_{jk}(\xi^0) & & & \lambda_{N}(\xi^0) 
\end{pmatrix} 
w(\xi^0 , t) \tag{4.13}\label{chap2-eq4.13} 
\end{equation}\pageoriginale
where  $\lambda_1 (\xi^0), \ldots ,  \lambda_N (\xi^0) $ are the roots
of the equation 
\begin{equation}
\det(\lambda I - i A. \xi^0 - B) = 0 \tag{4.14}\label{chap2-eq4.14}
\end{equation}
and $\big|a^*_{jk}(\xi) \big| \leq (N-1)! 2^N (c_0 | \xi^0 | + c_1)$. by 
Hadamard's condition we have  
$$
\re \lambda_j(\xi^0)< p log ( 1+ | \xi ^0 | + log c.
$$

Now since $w(\xi^0, t, t_0) $ is a solution of the above system it
follows that  
$$
|w(\xi^0, t, t_0) | \leq c' (1+ |\xi^0|)^{p_o h} \text{ for } 0\leq
t_0 \leq t \leq h  
$$
with the constants $c'$, $p_0$ independent of $t$, $t_o$, $\xi^0$.
Finally since $v(\xi^0, t, t_0)\break = c(\xi^0)^{-1} w ( \xi^0 , t, t_0)$
we have desired property.  
\end{proof}

\section{Hyperbolic and strongly hyperbolic systems}\label{chap2-sec5}%%% 5

The notion of well posedness of the Cauchy problem is closely related
to the nature of the given system of differential equations. In this
section we introduce hyperbolic and strongly hyperbolic systems of
differential equations. We give criteria,  in order that a given
system of differential operators be of this type, in terms of the
characteristic roots of the system.  

$A_k \equiv A_k(x, t)$, $B \equiv B (x, t) $ will be matrices of order
$N$ of functions on $\underbar{R}^n \times [0, h]$ the regularity
conditions of which will be prescribed later in each case. Consider
the first order system of  differential operators  
\begin{equation}
M \equiv \frac{\partial}{\partial t} - \sum\limits_{k} A_k(x, t)
\frac{\partial}{\partial x_k} \tag{5.1}\label{chap2-eq5.1} 
 \end{equation}\pageoriginale 

\begin{defi*}
A system of differential operators $M$ is said to be hyperbolic if the
forward and backward Cauchy problems are well posed.  
\end{defi*}

\begin{defi*}
A first order system of differential operators $M$ is said to be
strongly hyperbolic if for any choice of the matrix $B(x, t)$ the
Cauchy problem (forward as well as backward) is well posed for the
system  
\begin{equation}
\frac{\partial}{\partial t}- \sum \limits_{k}A_k (x, t)
\frac{\partial}{\partial x_k} - B(x, t)  \tag{5.2}\label{chap2-eq5.2} 
 \end{equation} 
 
 
 Let $\lambda_1 (x,\xi, t), \ldots, \lambda_N (x,\xi,t)$ be the roots of the
 equation 
 \begin{equation}
\det (\lambda I - A(x, t) \cdot  \xi ) = 0  \tag{5.3}\label{chap2-eq5.3}
 \end{equation} 
 where $A(x, t)\cdot \xi$ denotes the matrix $\sum\limits_k A_k (x,
 t) \cdot \xi_k$.  
\end{defi*}

\setcounter{proposition}{0}
 \begin{proposition}\label{chap2-sec5-prop1}%%% proposition 1
 If the coefficient matrices $A_k$ of $M$ are constant matrices then a
 necessary condition in order that $M$ be strongly hyperbolic is that  
 \begin{enumerate}[\rm(1)]
\item $\lambda_j(\xi)$ is real for all real $\xi \neq 0\ (j = 1,
  \ldots, N)$ 

\item the matrix $A$. $\xi$ is diagonalizable for all $\xi$. 
\end{enumerate} 
 \end{proposition}

We shall actually prove a slightly stronger result: If one of the
$\lambda_j (\xi )$ is not real for some real $\xi \neq 0$, then for
any choice of $B$ (a constant matrix) the Cauchy problem for  
$$
 \frac{\partial}{\partial t} - \sum\limits_{k} A_k
 \frac{\partial}{\partial x_k}- B  
 $$
 is not well posed. 

\begin{proof}
If the\pageoriginale condition (1) is not satisfied for same real
$\xi^* \neq 0$, there exists a root, say $\lambda_1 (\xi^*)$, with
non vanishing imaginary part of the equation $\det(\lambda I - A. \xi
) = 0$. For 
$\xi = \tau \xi^*$,  $\lambda = \tau \lambda'$ we can write    
$$
\det(\lambda I - iA. \xi - B) = \tau^N \det \left(\lambda'I - iA. \xi^* -
\frac{B}{\tau}\right)  
$$
for any matrix $B$. Denoting $\det (\lambda' I - i A. \xi^*)$ by
$P(\lambda')$ we have  
 \begin{equation}
\det (\lambda I  - iA \cdot \xi - B ) = \tau^N \left\{ P(\lambda' ) +
\frac{1}{\tau } Q (\lambda', \tau )\right\}  \tag{5.4}\label{chap2-eq5.4} 
 \end{equation} 
 where $Q (\lambda', \tau )$ is a polynomial in $\lambda'$ of degree
 at most $N-1$ having for coefficients polynomials in
 $\tau^{-1}$. Since $\lambda_1(\xi^*)$ is not real we may, without
 loss of generality, assume that $\text{Im\,} \lambda_1 (\xi^*) < 0$ (if
 necessary after changing $\xi^*$ by $-\xi^*$ in the equation).
 Then $ i\lambda_1 (\xi^*)$ is a root of $P(\lambda') =0 $. By
 continuity of the roots there exists a root of $P(\lambda' ) +
 \dfrac{1}{\tau} Q (\lambda', \tau)=0 $ in a neighbourhood of $i
 \lambda_1 (\xi^*) $ in the complex plane. More precisely there exists
 a root 
 $\lambda'_1(\tau)$ for large $\tau$ of the equation $P(\lambda') +
 \dfrac{1}{\tau}Q (\lambda', \tau) = 0$ such that $\lambda'_1(\tau) =
 i \lambda_1 (\xi^* ) + \in (\dfrac{1}{\tau})$ where $\in (\tau
 )\rightarrow 0$ as $\tau \rightarrow +  \infty$. Hence Re $\lambda'_1
 (\tau )\geq \dfrac{1}{2} (-\text{Im\,} \lambda_1 (\xi^\ast))$ for large
 $\tau$. In other words there exists a root $\lambda_1 (\tau)$ of the
 equation
$$
\det(\lambda I - iA. \xi - B ) = 0 
$$ 
such that $\re \lambda_1 (\tau) \leq c \tau $ (with a positive constant
$c$) ,  which tends to $+ \infty$ as $\tau \rightarrow \infty$. Hence
the forward Cauchy problem is not well posed for the system $M - B $
by prop. \ref{chap2-sec2-prop2} of \S\ \ref{chap2-sec2}.   

 (2)~ Assume\pageoriginale again that the system $M$ is strongly
 hyperbolic,  but  
 that for a certain $\xi^\ast$ the matrix A. $\xi^*$ is not
 diagonalizable. There exists a non-singular matrix $N_0$ such that 
 $N_0 (A. \xi^*)N_0^{-1}$ has the Jordan canonical form  
\begin{equation*}
\begin{pmatrix} 
\lambda_1 &  0 \ldots &  0 \\
1 & \lambda_1 \ldots & 0 \\
 & *   & \ddots 
 \end{pmatrix}  \tag{5.5}\label{chap2-eq5.5}
\end{equation*}
 
Consider for $ B$ a matrix determined by 
$$
N_0 B N^{-1}_0 =
\begin{pmatrix} 
0 & 1 & 0 & \ldots & 0\\ 
0 & 0 & 0 & \ldots & 0\\ 
\cdot & \cdot & \hdotsfor{3} & &  \\ 
0 & 0 & 0 & \ldots  & 0\\ 
\end{pmatrix} 
$$

We shall show that the Cauchy problem is not well posed for the
system  of differential operators  
$$
\frac{\partial}{\partial t} - \sum A_k \frac{\partial} {\partial x_k}
- B. 
$$

Consider the characteristic equation of this system, namely 
$$
\det(\lambda I - iA. \xi - B  ) = 0.
$$

Taking for $\xi$ the vector $\tau \xi^*$ ($\tau$ a real parameter
 $\rightarrow \infty$) this equation becomes  
\begin{gather*}
\det(\lambda I - i\tau A. \xi^* - B ) = \det(\lambda I - i\tau N_0
(A.\xi^*) N_0^{-1}-N_0 B N^{-1}_0)  \\ 
= ~ \begin{vmatrix}
 \lambda-i \tau \lambda_1 & -1 & 0 \ldots 0\\ 
-i  \tau \lambda_1 & \lambda-i \tau \lambda_1 & 0 \ldots 0\\
\ldots\ldots & \ldots\ldots & \ldots X \\ 
\ldots\ldots & \ldots\ldots & \ldots X 
\end{vmatrix} 
\end{gather*}

Hence\pageoriginale $(\lambda - i \tau \lambda_1)^2 - i \tau = 0$, the
roots of 
which  are $\lambda (\tau) = i \tau \lambda_1 \pm \sqrt{i\tau}$ whose
real part $\mathcal{R} e \lambda (\tau) \to \infty $ along with
$\tau$. Hence the Cauchy problem for the system $M-B$ is not well
posed by prop \ref{chap2-sec2-prop2} of \S\ \ref{chap2-sec2}, 
which contradicts the assumption. 
\end{proof}

\begin{proposition}\label{chap2-sec5-prop2}% proposition 2
A sufficient condition in order that the system $M$ be  strongly
hyperbolic is that one of the following two conditions is satisfied:  
\begin{enumerate}[\rm(i)]
\item the characteristic roots $\lambda_i (\xi ) $ are real and
  distinct for all real $\xi \neq 0$; 

\item  $A_k$ are Hermitian. 
\end{enumerate}
\end{proposition}

\begin{proof}
Supposing the condition (i) is satisfied. We shall show that this
implies that the Cauchy problem is well posed for the system $M-B$ for
any choice of $B$.  Consider the equation $\det(\lambda I - iA. \xi -
B) = 0$. Denoting the projection $\dfrac{\xi}{|\xi|}$ of $\xi $ on
the unit sphere by $\xi^0$ and $\dfrac{\lambda}{|\xi |}$ by $\lambda'
(\xi)$ we can write this equation in the form  
$$
\det (\lambda' I - iA. \xi^0 - \frac{B}{|\xi|}) = 0. 
$$

If $ \lambda_1 (\xi^0),  \ldots,  \lambda_N(\xi^0) $ are the roots of
the equation  $(\det \lambda I - A. \xi^0 ) = 0 $ we can write  
\begin{equation}
\det (\lambda' I - iA. \xi^0 - \frac{B}{|\xi |}) = \prod
\limits^N_{j=1}(\lambda' - i \lambda_j (\xi^0 ))  + \frac{Q (\lambda'
  ,  \xi^0 ) }{|\xi | } = 0 , \tag{5.6}\label{chap2-eq5.6}  
\end{equation} 
where $Q(\lambda',  \xi)$ is a polynomial $a_0 (\xi) \lambda'^{N-1} +
\cdots + a_{N-1} (\xi )$ with  coefficients bounded for $|\xi |  \geq
1$. If $\Omega_0$ is the projection of $\Omega $ on the unit sphere we
have  
$$
\inf_{\substack{\xi^0 \in \Omega^0\\ j \neq k}} | \lambda_j 
(\xi^0) - \lambda_k (\xi^0) | \geq d > 0 
$$\pageoriginale
since $\lambda_{1}(\xi^0) \ldots \lambda_N (\xi^0) $ are all distinct.

Let $ K= \sup\limits_{\substack{\xi^0 \in \Omega^0 \\ 1\leq j\leq N}}  |
\lambda_j (\xi^0)$ and $m=\sup\limits_{\substack{|\xi \geq 1|\\ |\lambda|
    \geq K+1}} |Q(\lambda',  \xi)|$.  

Let $C$ be a positive number such that $C\left(\dfrac{d}{2}\right)^{N-1} \geq 2m
$ and $\Gamma_1 ,  \ldots ,  \Gamma_N$ be circles in the complex plane
of radic $\dfrac{C}{|\xi|}\left(\leq \dfrac{d}{2}\right)$ with centres
$\lambda_1 (\xi^0),\break  \ldots , \lambda_N (\xi^0)$ respectively. On $\Gamma_K$ we
have  
$$
\left|\prod_j (\lambda'- i \lambda_j (\xi^0))\right| \geq 
\frac{C}{|\xi|}\left(\frac{d}{2}\right)^{N-1} \geq \frac{2m}{|\xi|}
\text{~ and~ } \frac{|Q (\lambda',\xi)|}{|\xi|} \leq \frac{m}{|\xi|}. 
$$

Hence by Rouche's theorem there exists a unique root of 
$$
\prod_j (\lambda'- i \lambda_j (\xi^0)+ \frac{Q (\lambda',
  \xi)}{|\xi|}=0 
$$
in the dise enclosed by $\Gamma_k$. More precisely there exists a root
$\lambda'_j (\xi)$ of  $\det(\lambda'I - i A.  \xi^0-
\dfrac{B}{|\xi|}) = 0$ such that  
$$
\left| \lambda'_j (\xi)- i \lambda_j (\xi^0) \right|< \frac{C}{|\xi|} 
$$
or, what is the same,  there exists a root $\tilde{\lambda}_j (\xi)$ of  
$$
\det(\lambda I - i A. \xi - B) = 0
$$
such that $ |\tilde{\lambda}_j (\xi)- i \lambda_j (\xi)| < C$. Since
$\lambda_j(\xi)$ are real it therefore follows that  
$$
\re \tilde{\lambda}_j {\xi} \leq C\ (j=1,  \ldots ,  N) 
$$
and by prop. \ref{chap2-sec2-prop1} of \S\ \ref{chap2-sec2} the
forward Cauchy problem is well posed for the  
system\pageoriginale $M-B$. This proves that $M$ is strongly
hyperbolic.  

Next let us assume that the matrices $A_k$ are Hermitian. By Fourier
transforms in the $x$-space we obtain the first order system of ordinary
differential equations. 

Now consider
\begin{align*} 
\frac{d}{dt}| v (\xi , t)|^2 & = \frac{d}{dt} v (\xi, t) \cdot
\overline{v(\xi, t)} + v (\xi,  t) \overline{\frac{d}{dt} v(\xi, t)}\\ 
& = (iA. \xi + B ) v (\xi, t) \cdot \overline{v(\xi, t)} + v (\xi,  t)
\overline{(iA. \xi + B ) v (\xi , t)} 
\end{align*}

Since the $A_k$ are Hermitian, we obtain, $B$ being bounded,  
$$
\frac{d}{dt}|v(\xi, t)|^2 = 2 \re B v (\xi, t) \cdot \overline{v(\xi, t)}
\leq 2 c | v (\xi,  t)|^2.  
$$

We obtain therefore 
\begin{equation}
\left| v(\xi, t) \right|^2 \leq \left| v(\xi, 0) \right|^2 e
^{2ct}. \tag{5.7}\label{chap2-eq5.7} 
\end{equation}
which shows that the forward Cauchy problem is well posed for the
system $M-B$ and so $M$ is strongly hyperbolic. This completes the
proof of the proposition. Let us now remark the following fact:  
\begin{align*}
\frac{d}{dt} || v(\xi, t) ||^2 & = \frac{d}{dt}\langle v(\xi, t),
\overline{v(\xi, t) \rangle} \\ 
& = \langle \frac{d}{dt} v (\xi, t),  \overline{v(\xi, t)} +
\rangle v(\xi, t),  \overline{\frac{d}{dt} v(\xi, t)} \rangle\\ 
& = \langle (iA. \xi + B ) v (\xi, t),  \overline{v(\xi, t) }\rangle +
\langle v(\xi, t),  \overline{(iA.\xi + B ) v (\xi, t)}\rangle.   
\end{align*}

Since $A_k$ are Hermitian we obtain 
$$
\frac{d}{dt}||v(\xi, t) ||^2 = 2 \re \langle B v(\xi, t),
\overline{v(\xi, t)} \rangle \leq 2 c || v (\xi, t)||^2 
$$
with\pageoriginale a constant $c$ independent of $\xi$. Integrating
both sides of the inequality over $[0,  t]$ we obtain  

$|| v(\xi,  t)||^2 \leq || v(\xi,  0 )||^2 e^{2ct}$. Hence
\begin{equation*}
||u|| \leq || u (x,  0)||e^{ct}. \tag{5.8}\label{chap2-eq5.8}
\end{equation*}

We remark that the notions of hyperbolicity and strong hyperbolicity
can be anologously defined for a single differential operator of order
$m$. Consider a differential operator of order $m$ 
\begin{equation}
 L= \left(\frac{\partial}{\partial t}\right)^m + \sum_{\substack{| \nu | + j =m
     \\  j \leq m-1}} a_{\nu,  j} (x, t) \left(\frac{\partial}{\partial t}
 \right)^\nu \left(\frac{\partial}{\partial t}\right)^j.
 \tag{5.9}\label{chap2-eq5.9} 
\end{equation}
$L$ is said to be hyperbolic if the Cauchy problem (both the forward
and the backward) is well posed for $L$. It is said to be strongly
hyperbolic if the Cauchy problem (both the forward and the backward)
is well posed for $L-B$ for any choice of the lower order operator
$B$. Let  
 \begin{equation}
P(\lambda,  \xi ) = \lambda^m + \sum_{\substack{| \nu | + j = m \\  j
    \leq m-1}} a_{\nu,  j} (x, t ) \xi^\mu \lambda^j
\tag{5.10}\label{chap2-eq5.10} 
 \end{equation} 
\end{proof}

\begin{proposition}\label{chap2-sec5-prop3}%%% proposition 3
A necessary and sufficient condition in order that a differential
operator $L$ of order $m$ with constant coefficients be strongly
hyperbolic is that for every real vector $\xi (\neq 0)$ in $\underline{R}^n$
all the roots of the equation $P(\lambda,  \xi) = 0$ are real and
distinct. 
\end{proposition} 

\begin{proof}
The proof of the fact that the roots of $P(\lambda, \xi) = 0$ for all
real $\xi (\neq 0)$ are real runs on the same lines as in
Prop. \ref{chap2-sec5-prop1}. We shall 
now show that for all real $\xi \neq 0$ these roots are all distinct. 

It the roots of $P(\lambda, \xi ) = 0 $ are not distinct for all real
 $\xi \neq 0$ let us suppose that for some real $\xi^* \neq 0$ at
 least two roots of $P(\lambda,  \xi^*) = 0$\pageoriginale
 coincide. Writing  $P(\lambda,  \xi^*)$ explicitly 
 $$
 P(\lambda,  \xi^*) = (\lambda- \lambda_1 (\xi^*))^p
 \prod^{m-p+1}_{j=2}(\lambda- \lambda_j (\xi^*)),  P \geq 2,  
 $$
 where $\lambda_2(\xi^*), \ldots, \lambda_{m-p+1}~(\xi^*)$ are real,
 and different from $\lambda_1( \xi^*)$. Take for $\xi$ the vector
 $\tau \xi^*$ with a real parameter $\tau$ and set $\lambda'
 =\dfrac{\lambda}{\tau} - i \lambda_1 (\xi^*)$. Now consider the
 equation  
 $$
 P(\lambda,  i \tau \xi^*) + C \tau^{m-1} = 0 
 $$
 with a constant $C$ to be chosen later suitably. From this equation 
 we obtain 
\begin{align*}
& \lambda'^{p} \prod\limits^{m-p}_{j=2} \left\{ \lambda' + i (\lambda_1 
 (\xi^*) - \lambda_j (\xi^*))\right\}+ \dfrac{C}{\tau}\\
&\quad   = \lambda'^{p}(a_0 (\xi^*) + a_1 (\xi^*) \lambda' + \cdots
  +a_{m-p-1} (\xi^*) 
 \lambda'^{m-p-1} + \lambda'^{m-p}) + \frac{C}{\tau}=0 
\end{align*}
 where $a_0 (\xi^*) \neq 0$. Expanding this in a Puiseux series in a
 neighbourhood of $\tau = \infty $ we see that there exist $p$ roots 
$$
\lambda'_K (\tau) = \exp \left(\frac{2 \pi i}{p}k\right) \cdot 
\left(\frac{-C}{a_0
  (\xi^*)}\right)^{\frac{1}{p}} \tau^{- \frac{1}{p}} + 0
\left(\tau^{-\frac{1}{p}}\right)\ (k=1,  \ldots ,  p) 
$$ 
$p$ being at least 2 we can choose the constant $C$ such that there
exists a root with positive real part; that is there exists a $k_0$
such that  
 $$
 \re \lambda'_{k_0}(\tau ) \geq C_0 \tau^{-1/p} \text{~ for large~ }
 \tau.  
 $$
 ($C_0$ being a positive constant). Hence
 $$
 \re \lambda_{k_0} (\tau) \geq C_0 \tau^{1- \frac{1}{p}} \text{~ for~
   large } \tau. 
 $$

There\pageoriginale exist constants $b_\nu $ such that $C =
\sum\limits_{ | \nu | =  m-1} b (i  \xi^*)^\nu$. Thus it follows from
prop. \ref{chap2-sec2-prop2} \S\ \ref{chap2-sec2} that the 
Cauchy problem is not well posed for the operator 
$$
L + \sum\limits_{| \nu| = m-1} b_\nu\left(\frac{\partial}{\partial
  x}\right)^\nu. 
$$

This contradicts the assumption that the operator $L$ is strongly
hyperbolic. 
 
The sufficiency follows as in the proof of the prop. \ref{chap2-sec5-prop2}(i).
 
Finally we mention the following fact: Consider the following equation
with coefficients in $\mathscr{E}$. 
$$
M[u]= \frac{\partial}{\partial t} u- \sum A_k (x,  t)
\frac{\partial}{\partial x_k} u - B(x, t)u = 0. 
$$

If, at the origin, for some $\xi^*$ real $\neq 0$,  one of the
characteristic roots of $\det(\lambda I - A (0, 0) \xi^*) = 0$ is not
real,  then the Cauchy problem for $M$ is never well posed in
$\mathscr{E}$ in any small neighbourhood of the origin. (See Mizohata
\cite{key3}). We shall prove this fact later, in a simple case. Here we add
an important remark: Garding has shown in his paper
(G$\ring{\text{a}}$rding \cite{key1}), 
that the condition 9 of \S\ \ref{chap2-sec2} of Hadmard is equivalent
to the following: 
 
$\re \lambda_j (\xi)$ is bounded from above when $\xi$ runs through
$\underbar{R}^n$ for $j=1, \ldots , N$. 
 
 Next H\"ormander has systematized such inequalities by using
 Seidenberg's lemma (see H\"ormande \cite{key1}). 
\end{proof}

\begin{proposition}\label{chap2-sec5-prop4}%\proposition 4
Let\pageoriginale the coefficients $A_k$ and $B$ of $M$ be continuous
functions of 
$t$ in an interval $[0, T]$. If the forward Cauchy problem is
uniformly well posed then the backward Cauchy problem is also
uniformly well posed. 
\end{proposition} 

\begin{proof}
As before denoting $\dfrac{\xi }{|\xi|}$ by $\xi^0$ let $v^j(\xi, t,
t_0)$ be a fundamental system of solutions of the system of ordinary
differential equations  
$$
\frac{d}{dt}v(\xi,  t)= (i |\xi | A(t) \xi^0 + B(t)) v(\xi,  t),  0 
\leq t \leq t_0 
$$
with initial conditions $v^j (\xi,  t_0,  t_0)= v^j \equiv (v^j_1 , 
\ldots ,  v^j_N)$ where $v^j_j = 1$ and $v^j_k = 0$ for $k \neq
j$. First of all we remark that if $W(t,  \xi)$ is the Wronskian of
this system then $v^j (\xi,  t,  t_0) $ define its colums. Since the
forward Cauchy problem is uniformly well posed we have  
$$
|v^j (\xi,  t,  t_0)| \leq C(1 + |\xi |)^p,  j=1,  \ldots ,  N. 
$$

Hence $W (t, \xi )$ is also majorized by a polynomial in $|\xi 
|$. From the theory of ordinary differential equations we know that  
$$
 W(T, \xi ) = W (t,  \xi ) \exp \left\{ i | \xi | \sum_j
 (\int\limits^T_t a_{jj}  (s,  \xi^0 )ds + \int\limits^T_t b_{jj}(s)ds
 \right\}.  
 $$
 
Now as in Prop. \ref{chap2-sec5-prop1} it follows that $\sum\limits_j
\int\limits^T_t a_{jj}(s,  \xi^0) ds$ is real for any $t$ and
$\xi^0$. Thus we have  
 $$
 |W(T,  \xi)| \geq |w(t,  \xi ) | \exp \left\{ - \sum\limits_{j}
 \int\limits^t_t b_{jj} (s) ds \right\}.   
 $$

 That is,  $|W(T,  \xi)| \geq \delta > 0$ for all $t$ and
 $\xi$. Further we observe that as $v^j(\xi,  t,  t_0)$ form a basis
 for solutions of the system of equations\pageoriginale we can write 
 $$
 v^j (\xi, t, t_0) = \sum_{k} c^j_k (\xi) v^k (\xi, t, T). 
 $$

 Putting $t = t_0$ and solving for $c^j_k (\xi) $ we see that $c^j_k
 (\xi)$ are majorized by polynomials in $|\xi |$ since the determinant
 of this system of linear equations is the Wronskian $W(\xi,  T)$
 which is minorized by $\delta > 0$ and $v^j (\xi,  t,  t_0) $ are
 majorized by polynomials in $|\xi|$. Hence $v^j (\xi,  t,  t_0)$ are
 majorized by polynomials in $|\xi|$ independent of $t$ and $t_0$ in
$[0, T]$ which proves that the backward Cauchy problem is
 uniformly well posed. This completes the proof of the proposition. 
\end{proof}
