\chapter{}\label{chap5}

In\pageoriginale this chapter we study non-linear equations. Much of
this chapter is 
inspired by the recent monograph of S.L. Sobolev: Sur les equations
aux derivees particlles hyperboliques non-lineaires (Cremonese, Roma
1961). 

\section[Preliminaries to the study of semi-linear
  equa\-tions]{Preliminaries to the study of semi-linear\hfill\break
  equa\-tions}\label{chap5-sec1} 

In this section we recall, without giving the proofs, a few results
of Sobolev concering the differentiability properties of functions
belonging to the spaces $\mathscr{D}^m_{L^2}$. More precisely we give
estimates in the $L^p$ norm for the derivatives of these functions in
terms of their norms in the space $\mathscr{D}^m_{L^2}$. We shall also
introduce the functions spaces $\mathscr{D}^s_{L^2}$ for any
aribitrary real number $s \geq 0$ and obtain $L^2$ estimates of some
non-linear functions of derivatives of functions belonging to the
spaces $\mathscr{D}^s_{L^2}$.  

To begin with the state the following important result due to Sobolev
\cite{key1}.  

\setcounter{proposition}{0}
\begin{proposition}[Sobolev's lemma]\label{chap5-sec1-prop1}  %proposition 1.1
 Let $p$ and $q$ be positive numbers with $\underset{q>1}{p>1}$, and
 $\dfrac{1}{p} + \dfrac{1}{q}>1$. If $g \in L^p $ and $h
 \in L^q$ then  
\begin{equation*}
\left| \iint \frac{g(x)h(y)}{|x-y|^\lambda}  dx \; dy \right| \leq K
|| g ||_{L^p} || h ||_{L^q},  \tag{1.1}\label{chap5-eq1.1}
\end{equation*}
where $\lambda = n (2 - \dfrac{1}{p} - \dfrac{1}{q})$ and $K$ is a
constant depending only on $p, q, n$.  

Suppose $u \in L^p$ and a number $\lambda$ such that $0 <
\lambda < n$ and $\dfrac{\lambda}{n} > 1 - \dfrac{1}{p}$ are
given. Then the above inequality implies that the linear mapping  
\begin{equation*}
h\to \int (u (y) * \frac{1}{|y|^\lambda}) \cdot h (y) \; dy
\tag{1.2}\label{chap5-eq1.2}  
\end{equation*}\pageoriginale
is a continuous linear functional on the space $L^q$ for $q > 1$ with
$\dfrac{1}{q} = \left(2 - \dfrac{1}{p} - \dfrac{\lambda}{n}\right)$. 
Hence $u *\dfrac{1}{|x|^\lambda} \in L^{q'}$ where $q'$ satisfies
$\dfrac{1}{q'} =  1 - \dfrac{1}{q} = \dfrac{\lambda}{n} +
\dfrac{1}{p} - 1$. This proves the following
\end{proposition}

\setcounter{corollary}{0}
\begin{corollary}\label{chap5-sec1-coro1}  %corollary 1
Let $u \in L^p$ for a $p > 1$ and $\lambda$ be a positive
number such that $0 < \lambda < n$ and $\dfrac{\lambda}{n} > 1 -
\dfrac{1}{p}$. Then $u * \dfrac{1}{|x|^\lambda}\in L^{q'}$ 
where $\dfrac{1}{q'} = \dfrac{\lambda}{n} + \dfrac{1}{p} - 1$.  
\end{corollary}

In corollary \ref{chap5-sec1-coro1} taking $p = 2$ and $\lambda$ a
number such that $\dfrac{n}{2} < \lambda < n$ we have the following  

\begin{corollary}\label{chap5-sec1-coro2}  %corollary 2
If $u \in L^2$ then for any positive number $\lambda$ such
that $\dfrac{n}{2} < \lambda < n$ we have $u * \dfrac{1}{|x|^\lambda}
\in L^q$ where $\dfrac{1}{q} = \dfrac{\lambda}{n} -
\dfrac{1}{2}$ and  
\begin{equation*}
|| u * \frac{1}{|x|^\lambda} ||_{L^q} \leq K || u ||_{L^2}
\tag{1.3}\label{chap5-eq1.3} 
\end{equation*}
where $K$ is a constant depending on $n, \lambda$. 
\end{corollary} 
 
We shall now introduce the function space $\mathscr{D}^s_{L^2}\equiv
\mathscr{D}^s_{L^2} (\underbar{R}^n) $ for any arbitrary real number
$s > 0$.  
 
 Let $\Omega$ be an open set in $\underbar{R}^n$ and $m$ be a non-negative
 integer. We recall that $\mathscr{E}^m_{L^2} (\Omega)$ denotes the space
 of all square integrable functions $f$ on $\Omega$ for which all the
 derivative $\left(\dfrac{\partial}{\partial x}\right)^\nu  f$ 
(in the sense of distributions) of orders $|\nu|\leq m$ are again
 square integrable 
 functions on $\Omega$. $\mathscr{E}^m_{L^2} (\Omega)$ is provided
 with the scalar product  
{\fontsize{10pt}{12pt}\selectfont
 \begin{equation*}
(f, g )_{\mathscr{E}^m_{L^2}} (\Omega) \equiv (f, g)_m =
   \sum_{|\alpha|\leq m } \left(\left( \frac{\partial}{\partial
     x}\right)^\alpha \right) f, 
   \left(\frac{\partial}{\partial x}\right)^\alpha g)_{L^2 (\Omega)}
   \text{~ for~ } f, g \in \mathscr{E}^m_{L^2}(\Omega)
   \tag{1.4}\label{chap5-eq1.4}   
 \end{equation*}}\relax\pageoriginale 

Here $\left(\dfrac{\partial} {\partial x}\right)^\alpha$ denotes a
derivation in the sense of
distributions. $\mathscr{E}^m_{L^2}(\Omega)$ is a Hilbet 
space for this scalar product. Clearly $\mathscr{D} (\Omega) \subset
\mathscr{E}^m_{L^2} (\Omega)$. The closure of $\mathscr{D} (\Omega)$
in $\mathscr{E}^m_{L^2}(\Omega)$ is deoted by
$\mathscr{D}^m_{L^2}(\Omega) \cdot \mathscr{D}^m_{L^2}(\Omega)$, with the
scalar product which is the restriction of that in
$\mathscr{E}^m_{L^2}(\Omega)$, is again a Hilbert space. In general
$\mathscr{D}^m_{L^2}(\Omega) \neq
\mathscr{E}^m_{L^2}(\Omega)$. However when $\Omega = \underbar{R}^n$
we have $\mathscr{D}^m_{L^2} (\underbar{R}^n) =
\mathscr{E}^m_{L^2}(\underbar{R}^n)$. We write
$\mathscr{D}^m_{L^2}(\underbar{R}^n) =
\mathscr{E}^m_{L^2}(\underbar{R}^n) = \mathscr{D}^m_{L^2}$ for
abbreviation. The elements of $\mathscr{D}^m_{L^2}(\Omega)$ can be
considered as functions vanishing upto order $(m-1)$ (in a generalized
sense) on the boundary of $\Omega$.  
  
 We observe that  $\mathscr{D}^m_{L^2} \subset \mathscr{L'}$. Hence by
 Plancheral's theorem we have  
 $$
 || f ||^2_m = || f ||^2_{\mathscr{D}^m_{L^2}} =
 \sum\limits_{|\alpha|\leq m} || \left(\frac{\partial}{\partial
   x}\right)^\alpha  f||^2_{L^2}  = \sum_{|\alpha|\leq m} || (2 \pi i
 \xi)^\alpha \hat{f}||_{L^2}  
 $$
 where $\hat{f}$ is the Fourier image of $f$. Now there exist
 constants $c_1$, $c_2 > 0$ such that  
 $$
 c^2_1 (1 + |\xi |)^{2m} \leq \sum_{|\alpha|\leq m}| (2\pi i \xi )^\alpha
 |^2\leq c^2_2 (1 + |\xi |)^{2m}. 
 $$

Thus, if $f \in\mathscr{D}^m_{L^2}$ then $(1 + |\xi |)^m f
\in L^2$ and further  
$$
C_1||(1+|\xi|)^m \hat{f} ||_{L^2} \leq ||f||_m \leq  C_2 || (1 + |\xi|)^m
\hat{f}||_{L^2}.  
$$

Hence\pageoriginale $\mathscr{D}^m_{L^2} $ can also be defined as the
space of all 
tempered distributions $f$ such that $(1+|\xi|)^m \hat{f}
\in L^2$ where $\hat{f}$ denotes the Fourier image of
$f$. This motivatives the following.  

\begin{defi*}
For any real $s, \mathscr{D}^s_{L^2}$ is the space of tempered
distributions $f$ such that $(1+|\xi|)^s \hat{f} \in L^2$.  
 
 $\mathscr{D}^2 _{L^2}$ is provided with the scalar product 
 \begin{equation}
( f, g )_s \equiv (f, g )_{\mathscr{D}^s_{L^2}} = ((1 + |\xi |)^s
   \hat{f}, (1 + |\xi|)^s \hat{g})_{L^2}\tag{1.5}\label{chap5-eq1.5} 
 \end{equation} 

 For this scalar product $\mathscr{D}^s_{L^2}$ is a Hilbert space. It
 is clear that if $s \geq s'$ then $\mathscr{D}^s_{L^2} \subset
 \mathscr{D}^{s'}_{L^2}$ and the inclusion mapping is continuous.  
 \end{defi*}

\begin{remarks*}
\begin{enumerate}[(1)]
\item The dual space of $\mathscr{D}^s_{L^2}$ is
  $\mathscr{D}^{-s}_{L^2}$: $(\mathscr{D}^s_{L^2})' =
  \mathscr{D}^{-s}_{L^2}$.    

\item The mapping $u \to \dfrac{\partial u}{\partial x_j}$ from
  $\mathscr{D}^s_{L^2}$ into $\mathscr{D}^{s-1}_{L^2}$ is continuous.  

\item The mappings defined by 
$$
(a(x), u) \to a(x) u 
$$
(i) from $\mathscr{B}^m \times \mathscr{D}^m_{L^2}$ into
  $\mathscr{D}^m_{L^2}$ and (ii) from $\mathscr{B}^m \times
  \mathscr{D}^{-m}_{L^2}$ into $\mathscr{D}^{-m}_{L^2}$ are
  continuous for $m = 0, 1, 2, \ldots$ 
\end{enumerate}
\end{remarks*} 

\setcounter{lemma}{0} 
\begin{lemma}\label{chap5-sec1-lem1}  %lemma 1
Let $s$ be a real number $\geq 0$  
\begin{enumerate}[(i)]
\item If $u \in \mathscr{D}^s_{L^2}$ for $0 \leq s <
  \dfrac{n}{2}$ then $u \in L^p$ where $\dfrac{1}{p} =
  \dfrac{1}{2} - \dfrac{s}{n} > 0 $ and  
\begin{equation*}
|| u ||_{L^p} \leq c (s, n) || u ||_s  \tag{1.6}\label{chap5-eq1.6}
\end{equation*}
where\pageoriginale the constant $c(s, n)$ depends only on $s$ and $n$; 

\item If $u \in\mathscr{D}^s_{L^2}$ for $s > \dfrac{n}{2}$ 
  then $u \in \mathscr{B}^0$ and  
\begin{equation*}
|| u ||_{\mathscr{B}^0} \leq c (s, n) || u ||_s \tag{1.7}\label{chap5-eq1.7}
\end{equation*}
where the constant $c(s, n)$ depends only on $s, n$. 

More precisely, for any $\sigma \leq 1$ with $ 0 < \sigma < s -
\dfrac{n}{2}$ we have  
 \begin{equation*}
|| u ||_{ \mathscr{B}^\sigma} \leq c (s, n, \sigma ) || u ||_s
\tag{1.8}\label{chap5-eq1.8}  
 \end{equation*} 
 where the constant $c(s, n, \sigma)$  depends only on $s, n$,
 $\sigma$. 
\end{enumerate}
\end{lemma}

\begin{remark*}
We recall that $\dfrac{1}{|x|^\lambda}$ is tempered
distribution and we have the formulae giving its Fourier image.   
\begin{equation*}
\begin{split}
& \mathscr{F} \left(\frac{1}{|x|^m}\right) = \frac{1}{\Pi^{\frac{n}{2}- m}}
\frac{\Gamma(\frac{m-n}{2})}{\Gamma(\frac{m}{2})}
\left(\frac{1}{|\xi|^{n-m}}\right) \text{~ for~ } \frac{n}{2}
 \leq m < n \text{~ and}\\  
& \mathscr{F}\left(\frac{1}{|x|^{n-m}}\right) = \frac{1}{\Pi^{m-\frac{n}{2}}}
 \frac{\Gamma(\frac{m}{2})}{\Gamma(\frac{n-m}{2})}\left(\frac{1}{|\xi|^m
 }\right)\text{~ for~ } 0 < m < \frac{n}{2}. 
\end{split}
\tag{1.9}\label{chap5-eq1.9}   
\end{equation*}

For proof of these formulae we refer to L. Schwartz, Theorie des
distributions, Vol.~II, p.~113. 
\end{remark*} 

\noindent
{\bf Proof of Lemma \ref{chap5-sec1-lem1} :}~ 
(i) The assertion (i) is trivial when $s=0$. Hence we may assume that
   $ 0 < s <\dfrac{n}{2}$. Let $u \in
   \mathscr{D}^s_{L^2}$. Writing $\hat{u}$ as $|\xi|^{-s} (|\xi|^s
   \hat{u})$ we have  
$$
u = c \cdot \frac{1}{|x|^{n-s}} * (\wedge^s u) 
$$
by taking the inverse Fourier images and using the above remark (we
note that $c$ is a positive constant depending only on $n, s$). It
follows now, from\pageoriginale cor. \ref{chap5-sec1-coro2} after
Prop. \ref{chap5-sec1-prop1}, that 
$u\in L^p$ and    
$$
|| u ||_{L^p} = c || \frac{1}{|x|^{n-s}} * (\wedge^{s} u) ||_{L^p}
\leq c (s, n ) || \wedge^s u ||_{L^2}  
$$
where $\dfrac{1}{p} = \dfrac{1}{2} - \dfrac{s}{n}$ (the constant $c(s,
n)$ depends only on $s,n$) . By Plancheral's theorem we have  
$$
|| \wedge^s u ||_{L^2} = || \; |\xi |^2 \hat{u} ||_{L^2} \leq || (1 +
|\xi |)^s \hat{u} ||_{L^2}= || u ||_s.  
$$

This proves the inequality \eqref{chap5-eq1.6}. 


(ii) Let $u \in \mathscr{D}^s_{ L^2}$ for $s >
\dfrac{n}{2}$. We have, using Cauchy-Schwarz inequality  
$$
|u (x) | \leq \int |\hat{u} (\xi)| d \xi \leq || ( 1 +|\xi |)^s
\hat{u}||_{L^2} || (1 + |\xi |)^{-s}||_{L^2} 
$$
which implies that $|u(x)| \leq c(s, n ) ||u||_s$ where $c(s, n)$ is a
constant depending only on $s, n$.  
 
 
We shall now prove H\"older continuity of $u$. Consider 
\begin{align*}
u(x) - u(x') & = \int \exp (2 \pi i\ x.\xi \cdot \xi ) \hat{u} (\xi) d \xi - \int
\exp (2 \pi i x'. \xi ) \hat{u} (\xi) d \xi \\ 
& = \int \exp (2 \pi i x . \xi ) \{1 - \exp (2 \pi i (x' - x). \xi  \}
\hat{u} (\xi) d \xi.  
\end{align*} 

For any real number $\sigma$ such that $0  < \sigma \leq 1$ let  
\begin{equation}
M_\sigma = \sup_{-\infty < \lambda < \infty} \left| \frac{e^{i \lambda} -
  1}{\lambda^\sigma}\right|\tag{1.10} \label{chap5-eq1.10} 
\end{equation} 

Clearly $M_\sigma < \infty$. Taking $\lambda'  = 2 \Pi (x -
x') \cdot \xi$ we obtain  
$$
|1 - \exp (2 \pi i (x' - x) \cdot \xi ) | \leq M_\sigma (2 \pi |x- x'|
\; |\xi |)^\sigma. 
$$

Hence\pageoriginale
\begin{align*}
\frac{| u (x) - u(x')|}{|x-x'|} & \leq  (2 \pi )^\sigma M_\sigma \int |
\xi |^\sigma | \hat{u} (\xi ) |d \xi \\ 
& \leq (2 \pi )^\sigma M_\sigma || (1 + | \xi | )^s \hat{u} ||_{L^2} ||
(1+ |\xi|^{\sigma -s}) ||_{L^2}. 
\end{align*}

We know that $\sigma - s < - \dfrac{n}{2}$ implies $|| (1+ | \xi
|)^{\sigma -s} ||_{L^2} < \infty$ and this proves the Holder
continuity of $u$. Thus $u \in \mathbb{B}^\sigma$ for any
$\sigma \leq 1$ with $0 < \sigma < s - \dfrac{n}{2}$. 

\begin{proposition}\label{chap5-sec1-prop2} % proposition 2
If $u \in \mathscr{D}^{[\frac{n}{2}]+1}_{L^2}$ then, for $1
\leq | \nu | \leq \left[\dfrac{n}{2}\right] + 1$,
$\left(\dfrac{\partial}{\partial x}\right)^\nu u \in L_p$ 
where $\mathfrak{p}$ is a positive number such that  
\begin{enumerate}[\rm(a)]
\item $\dfrac{1}{p} \in \left[ \dfrac{|\nu|}{n} -
  \dfrac{1}{n}, \dfrac{1}{2} \right] - \{ 0 \}$ when $n$ is even,  

\item $\dfrac{1}{p} \in \left[ \dfrac{|\nu|}{n} -
  \dfrac{1}{2n}, \dfrac{1}{2}\right]$\quad when $n$ is odd 
\end{enumerate}

Further the mapping $u \to \left(\dfrac{\partial}{\partial
  x}\right)^\nu u$ is continuous from $\mathscr{D}^{[\frac{n}{2}] +
  1}_{L^2}$ into $L^2$ and we have the inequality 
\begin{equation}
|| \left(\frac{\partial}{\partial x}\right)^\nu u ||_{L^p} \leq c (\nu, n, p) ||
u ||_{[\frac{n}{2}]+1}.\tag{1.11}\label{chap5-eq1.11} 
\end{equation}

The constant $c(\nu, n, p)$ depends only on $\nu$, $n$, $p$.

Before proceeding with the proof of this proposition we introduce the
following
\end{proposition}

\begin{defi*}
{\em The operator} $\Lambda^s$. For any $u \in 
\mathscr{D}^\sigma_{L^2}$ with $-\infty < \sigma < \infty$ the
operator $\Lambda^s$ is defined by the condition that $\Lambda^s u$ is
the inverse Fourier\pageoriginale image of $| \xi |^2 \hat{u}$. 
\end{defi*}

\begin{proof}
For any real $s \geq 0$ such that $s \leq \left[\dfrac{n}{2}\right]+ 1$, $u
\in \mathscr{D}^{[\frac{n}{2}]+1}_{L^2}$ implies that $u
\in \mathscr{D}^s_{L^2}$. Since the inverse Fourier image of
$\dfrac{1}{|\xi|^{s-| \nu |}}$ is $c(n, \nu)
\dfrac{1}{|x|^{n-(s-|\nu|)}}$ we can write   
$$
\left(\frac{\partial}{\partial x}\right)^\nu u = c(n, \nu)
\frac{1}{|x|^{n-(s-|\nu |)}}*
\left(\Lambda^{s-|\nu|}\left(\frac{\partial}{\partial x}\right)^\nu u\right)  
$$
by taking inverse Fourier image of 
$$
\left(\left( \frac{\widehat{\partial}}{\partial x}\right)^\nu u\right)
= 
(2 \pi i \xi)^\nu
\hat{u} = \frac{1}{| \xi|^{s-| \nu |}} \{ | \xi |^{s-| \nu |}(2 \pi i
\xi)^\nu \hat{u} \}. 
$$

Hence it follows, from Cor. \ref{chap5-sec1-coro2} of
Prop. \ref{chap5-sec1-prop1}, that 
\begin{gather*}
|| \left(\frac{\partial}{\partial x}\right)^\nu u ||_{L^p} = c(n, \nu )\|
\frac{1}{|x|^{n-(s-| \nu |)}}* \Lambda^{s-|\nu|}\left(\frac{\partial}{\partial
  x}\right)^\nu u||_{L^p}\\  
\leq c(s,n, \nu ) || \Lambda^{s-| \nu |}\left(\frac{\partial}{\partial
  x}\right)^\nu u ||_{L^2}  
\end{gather*}
for $\dfrac{1}{p} = \dfrac{n-(s-|\nu|)}{n} - \dfrac{1}{2}=
\dfrac{1}{2}- \dfrac{s-|\nu|}{n}$. On the other hand we know that 
$$
|| \Lambda^{s-|\nu|}\left(\frac{\partial}{\partial x}\right)^\nu u ||_{L^2} \leq
|| u ||_s \leq || u ||_{[\frac{n}{2}]+1} 
$$
which proves that 
$$
|| \left(\frac{\partial}{\partial x}\right)^\nu u ||_{L^p} \leq c(s, n, \nu)||
u||_{[\frac{n}{2}]+1}. 
$$

Using the fact that $|\nu|\leq s \leq \left[\dfrac{n}{2}\right]+1$ we have,
since 
$$
\frac{1}{p}\in \left[ \frac{1}{2}- \frac{[\frac{n}{2}]+1-|\nu 
    |}{n}, \frac{1}{2}\right]- \{0\}, 
$$\pageoriginale
that $ \dfrac{1}{p} \in \left[ \dfrac{| \nu |}{n}-
  \dfrac{1}{n}, \dfrac{1}{2} \right]- \{0\}$ when $n$ is even and
similarly $\dfrac{1}{p} \in \left[ \dfrac{| \nu |}{n}-
  \dfrac{1}{n}, \dfrac{1}{2} \right]$  when $n$ is odd. 

An entirely analogous proof will yield 
\end{proof}

\setcounter{dashprop}{1}
\begin{dashprop}\label{chap5-sec1-dashprop2}%proposition 2'
If $u \in \mathscr{D}^{[\frac{n}{2}]+N}_{L^2}$ we have
$\left(\dfrac{\partial}{\partial x}\right)^\nu u \in L^p$ where $p$ is a
positive number such that 
\begin{enumerate}[\rm(a)]
\item $\dfrac{1}{p}\in \left[ \dfrac{|\nu |}{n}-\dfrac{N}{n},
  \dfrac{1}{2}\right]-\{0\}$ when $n$ is even and  

\item $\dfrac{1}{p}\in \left[ \dfrac{|\nu
    |}{n}-\dfrac{2N-1}{2n}, \dfrac{1}{2}\right]$ when $n$ is odd,
  where  $1\leq N \leq |\nu|\leq \left[\dfrac{n}{2}\right] + N$. 
\end{enumerate}

Fourther the mapping $u \to \left(\dfrac{\partial}{\partial
  x}\right)^\nu u$ is continuous from
$\mathscr{D}^{[\frac{n}{2}]+N}_{L^2}$ into $L^p$ and 
we have the inequality 
\begin{equation}
|| \left(\frac{\partial}{\partial x}\right)^\nu u ||_{L^p} \leq c(\nu,
|| n, N, p) || u ||_{[\frac{n}{2}]+N} \tag{1.12}\label{chap5-eq1.12} 
\end{equation}
where the constant $c(\nu, n, N, p)$ depends only $\nu$, $n$, $N$, $p$.

The following result gives estimates in the $L^2$ norm of some
non-linear functions of the derivatives of functions belonging to
$\mathscr{D}^s_{L^2}$. The proofs are based essentially on the above
result and a generalization of Holder's inequality which we recall
without proof. 
\end{dashprop}

\begin{proposition}[Generalized H\"older's
    inequality]\label{chap5-sec1-prop3}% proposition  3 
 Let $\lambda_1, \ldots, \lambda_p$
  be positive numbers $> 1$ such that $\sum
  \dfrac{1}{\lambda_j}=1$. If $f_1, \ldots, f_p$ are functions
  belonging to $L^{\lambda_1}, \ldots, L^{\lambda_p}$ respectively
  then 
\begin{equation*}
\int | f_1(x) \ldots f_p (x) | dx \leq || f_1||_{L}\lambda_1, \ldots ||
f_p ||_L\lambda_p. \tag{1.13}\label{chap5-eq1.13} 
\end{equation*}
\end{proposition}

\begin{proposition}\label{chap5-sec1-prop4}%propositon 4
Let $l$\pageoriginale be an arbitrary integer $\geq 1$ and $\nu_1,
\ldots, \nu_1$ denote multi-indices  
\begin{enumerate}[(i)]
\item If $u \in \mathscr{D}^{[\frac{n}{2}]+1}_{L^2}$ and
  $\sum |\nu_j|\leq \left[\dfrac{n}{2}\right] + 1$ then
  $\left(\dfrac{\partial}{\partial x}\right)^{\nu_1}u \ldots
  \left(\dfrac{\partial}{\partial x}\right)^{\nu_1}u \in L^2$ and
  satisfies 
\begin{equation*}
|| \left(\frac{\partial}{\partial x}\right)^{\nu_1}u \ldots
\left(\frac{\partial}{\partial x}\right)^{\nu_l}u ||_{L^2} \leq c || u
||^l_{[\frac{n}{2}]+1} \tag{1.14}\label{chap5-eq1.14} 
\end{equation*}
where $c$ depends on $n$, $\nu_1, \ldots, \nu_1$ only.

\item If $u \in \mathscr{D}^{[\frac{n}{2}]+2}_{L^2}$ and
  $\sum | \nu_j | \leq \left[\dfrac{n}{2} \right] + 2$ then
  $\left(\dfrac{\partial}{\partial x}\right)^{\nu_1}u \ldots
  \left(\dfrac{\partial}{\partial x}\right)^{\nu_l} u \in L^2$ and
  satisfies 
\begin{equation}
|| \left(\frac{\partial}{\partial x}\right)^{\nu_1}u \ldots
\left(\frac{\partial}{\partial x}\right)^{\nu_1}u ||_{L^2} \leq c || u
||^{l-1}_{\left[\frac{n}{2} \right]+1} || u
||_{\left[\frac{n}{2}\right]+2} ;\tag{1.15}\label{chap5-eq1.15}  
\end{equation}
the constant $c$ depends only on $n$, $\nu_1, \ldots, \nu_l$. 

\item If $u \in \mathscr{D}^{\left[\frac{n}{2}
    \right]+N+1}_{L^2}$ and 
  $\sum | \nu_j | \leq \left[\dfrac{n}{2} \right]+N+1$ then
  $\left(\dfrac{\partial}{\partial x}\right)^{\nu_1}u \ldots
  \left(\dfrac{\partial}{\partial x}\right)^{\nu_l}u \in L^2$ and
  satisfies 
\begin{equation}
|| \left(\frac{\partial}{\partial x}\right)^{\nu_1}u \ldots
\left(\frac{\partial}{\partial x}\right)^{\nu_l}u ||_{L^2} \leq c || u
||^{l-1}_{\left[\frac{n}{2} \right]+N} || u ||_{\left[ \frac{n}{2}
      \right]+N+1}; \tag{1.16}\label{chap5-eq1.16}  
\end{equation}
the constant $c$ depend only on $n$, $N$, $\nu_1, \ldots, \nu_l$.
\end{enumerate}
\end{proposition}

\begin{proof}
The case $l=1$ is trivial. If $\nu_j = 0$ for some $j$ one can
majorize $u$ in the maximum norm by $|| u
||_{\left[\frac{n}{2} \right]+1}$. Hence we may assume that $l \leq
2$ and $|\nu_j|\geq 1$. 
\begin{enumerate}[\rm(i)]
\item Since\pageoriginale $u \in
\mathscr{D}^{\left[\frac{n}{2} \right]+1}_{L^2}$ it follows, from
Prop. \ref{chap5-sec1-prop2}, that  
$\left(\dfrac{\partial}{\partial x}\right)^{\nu_j}u \in L^{pj}$ for $1
\leq | \nu_j | \leq \left[\dfrac{n}{2} \right]+1$ where $p_j$ is a
real number such that 
\begin{enumerate}[(a)]
\item $\dfrac{1}{p_j} \in \left[\dfrac{|\nu_j|}{n}-
  \dfrac{1}{n}, \dfrac{1}{2}\right] - \{0\}$ when $n$ is even and  

\item $\dfrac{1}{p_j} \in \left[\dfrac{|\nu_j|}{n}-
  \dfrac{1}{2n}, \dfrac{1}{2}\right] $ when $n$ is odd. 
\end{enumerate}

Further we have
$$
|| \left(\frac{\partial}{\partial x}\right)^{\nu_j} u ||_{L^{pj}} \leq
c (\nu_j, n, p_j) || u ||_{\left[\frac{n}{2} \right]+1} (j = 1, \ldots, l). 
$$

Let $\dfrac{1}{P_j}$ denote the infimum of $\dfrac{1}{p_j}$ in this
range. 

If $n$ is even (a) implies that 
$$
\sum \frac{1}{P_j} = \sum \left( \frac{| \nu_j|}{n} -
\frac{1}{n}\right) \leq  \frac{\frac{n}{2}+1}{n} - \sum \frac{1}{n} <
\frac{1}{2} + \frac{1}{n}  
$$
and so. One can choose $p_1,
\ldots, p_l$ satisfying (a) and such that $\sum \dfrac{1}{p_j}=
\dfrac{1}{2}$. Similarly if $n$ is odd (b) implies that 
$$
\sum \frac{1}{P_j}= \sum \left(\frac{|\nu_j|}{n}- \frac{1}{2n}\right)
\leq \frac{\frac{n-1}{2}+1}{n} - \sum \frac{1}{2n} \leq \frac{1}{2}+ 
\frac{1}{2n}. 
$$

Again one can choose $p_1, \ldots, p_1$ such that $\sum
\dfrac{1}{p_j}= \dfrac{1}{2}$ and satisfies (b). Applying the
generalized H\"older's inequality with these $p_1, \ldots, p_l$ we
obtain 
{\fontsize{9pt}{11pt}\selectfont
\begin{align*}
\int | \left(\frac{\partial}{\partial x}\right)^{\nu_1}u \ldots
\left(\frac{\partial}{\partial x}\right)^{\nu_l}u \Big|^2 dx & \leq
\Pi_j \left(\int | 
\left(\frac{\partial}{\partial x}\right)^{\nu_j}u |^{2. \frac{p_j}{2}}
dx\right)^{2/p_j} \\ 
& = \prod\limits_j \|\left(\frac{\partial}{\partial x}\right)^{\nu_j}u ||^2_L p_j
\leq c(\nu_1, \ldots, \nu_1, n) || u ||^{2l}_{\left[\frac{n}{2}
    \right]+1}  
\end{align*}}\relax

\item  Since\pageoriginale $u \in
\mathscr{D}^{\left[\frac{n}{2}\right]+2}_{L^2}$ it follows, from
Prop. \ref{chap5-sec1-dashprop2}$'$ that  
$\left(\dfrac{\partial}{\partial x}\right)^{\nu_j} u \in L^{p_j} (j=1,
\ldots, l)$ and  
\begin{align*}
& || \left(\frac{\partial}{\partial x}\right)^{\nu_1} u ||_{L^{p_1}}
\leq c (\nu_1, n, p_1) || u ||_{\left[\frac{n}{2}\right]+2'} \\ 
& || \left(\frac{\partial}{\partial x}\right)^{\nu_j} u ||_{L^{p_j}}
\leq c (\nu_j, 
 n, p_j) || u ||_{\left[\frac{n}{2} \right]+1} (j=2, \ldots, l) 
\end{align*}
for $1 \leq | \nu_1 | \leq \left[\dfrac{n}{2}\right]+2$, $1 \leq | \nu_j
| \leq \left[\dfrac{n}{2} \right]+1$ where $p_1, \ldots, p_l$ are real
numbers such that   
\begin{itemize}
\item[(a$_1$)] $\dfrac{1}{p_1} \in \left[ \dfrac{|
    \nu_1|}{n}- \dfrac{2}{n}, \dfrac{1}{2}\right]-\{0\}$ when $n$ is
  even, 

\item[(b$_1$)] $ \dfrac{1}{p_1} \in \left[ \dfrac{|
    \nu_1|}{n}- \dfrac{3}{2n}, \dfrac{1}{2}\right]$ when $n$ is odd 

and 
\item[(a$_j$)] $\dfrac{1}{p_j} \in \left[ \dfrac{| \nu_j|}{n}-
  \dfrac{1}{n}, \dfrac{1}{2}\right] -\{0\}$ when $n$ is even,

\item[(b$_j$)]  $\dfrac{1}{p_j} \in \left[ \dfrac{| \nu_j|}{n}-
  \dfrac{1}{2n}, \dfrac{1}{2}\right]$ when $n$ is odd $(j = 2, \ldots,
l)$. 
\end{itemize}

We may without loss of generality assume that $| \nu_1 | \geq |
\nu_j | $ for $j=2, \ldots, l$. 
\begin{itemize}
\item[(1)] Suppose $| \nu_1| = 1$. Since $\sum^l_2 | \nu_j | \leq
  \left[\dfrac{n}{2} \right]+1$ we have from lemma \ref{chap5-sec1-lem1} that 
\begin{align*}
||\left(\frac{\partial}{\partial x}\right)^{\nu_1} u \ldots
\left(\frac{\partial}{\partial x}\right)^{\nu_l} u ||_{L^2}  & \leq \sup |
\left(\frac{\partial}{\partial x}\right)^{\nu_1} u| \cdot ||
\left(\frac{\partial}{\partial x}\right)^{\nu_2} u \ldots
\left(\frac{\partial}{\partial x}\right)^{\nu_l} u ||_{L^2} \\ 
& \leq c(n) || u ||_{\left[\frac{n}{2} \right]+2} \cdot ||
\left(\frac{\partial}{\partial  x}\right)^{\nu_2} u \ldots
\left(\frac{\partial}{\partial x}\right)^{\nu_l} u  ||_{L^2} 
\end{align*}
\end{itemize}

\item Suppose $|\nu_j | \geq 2 (j = 2, \ldots, l)$ then we have the
  estimates of the type \eqref{chap5-eq1.11}. As before we denote the
  infimum of  $\dfrac{1}{p_j}$ by $\dfrac{1}{P_j}(j =1, \ldots, l)$. 
\end{enumerate}

If\pageoriginale $n$ is even (a$_1$), (a$_j$) imply that
$$
\sum \frac{1}{P_j} = \frac{|\nu_1|}{n} - \frac{2}{n} + \sum\limits^l_2 +
\left(\frac{|{}^\nu j|}{n}- \frac{1}{n}\right) < \frac{1}{2} 
$$
and if $n$ is odd (b$_1$), ($b_j$) imply that $\sum \dfrac{1}{P_j} <
\dfrac{1}{2}$. In either of the cases we can choose $p_2, \ldots, p_l$
such that $\sum \dfrac{1}{p_j}= \dfrac{1}{2}$. 

Again applying the generalized H\"older's inequality we obtain
\begin{align*}
|| \left(\frac{\partial}{\partial x}\right)^{\nu_1} u \ldots ||
\left(\frac{\partial}{\partial x}\right)^{\nu_1} u ||_{L^2} & \leq \prod\limits_j
||\left(\frac{\partial}{\partial x}\right)^{\nu j} u ||_{L^{p_j}} \\ 
&  \leq c(n, \nu_1, \ldots, \nu_1 ) || u ||^{l-1}_{\left[\frac{n}{2}
    \right]+1}|| u ||_{\left[\frac{n}{2}\right]+2}.  
\end{align*}

\item As before we may assume that $|\nu_1|\geq|\nu_j|$ for
$j=2, \ldots, l$. Let $u \in
\mathscr{D}^{\left[\frac{n}{2} \right]+N+1}_{L^2}$. We distinguish
the following three different cases:  
$$
(\alpha) | \nu_1 | \leq N-1, \;\; (\beta) | \nu_1| = N, \;\; (\gamma)
| \nu_j | \geq N.  
$$
\end{proof}

\noindent
\textit{Case $(\alpha)$.} Since $|\nu_j | \leq | \nu_1 | \leq N-1$ by
Sobolev's lemma we have 
$$
\sup |\left( \frac{\partial}{\partial x}\right)^{\nu_j} u | \leq c || u
||_{\left[\frac{n}{2} \right]+N}. 
$$

Therefore we have
\begin{align*}
|| \left(\frac{\partial}{\partial x}\right)^{\nu_1} u \ldots
\left(\frac{\partial}{\partial x}\right)^{\nu_l} u ||_{L^2} & \leq ||
\left(\frac{\partial}{\partial x}\right)^{\nu_1} u ||_{L^2} \cdot
\prod\limits^l_{j=2} \sup | 
\left(\frac{\partial}{\partial x}\right)^{\nu_j}u| \\ 
& \leq C || u ||_{| \nu_1 |} \cdot || u
||^{l-1}_{\left[\frac{n}{2}\right]+N} \\  
& \leq C || u ||_{\left[\frac{n}{2} \right]+N+1} \cdot || u
|^{l-1}_{\left[\frac{n}{2} \right]+N} . 
\end{align*}

\noindent
\textit{Case $(\beta)$}.~ $| \nu_1| = N$\pageoriginale implies that $\sum
\limits^l_{j=2} | \nu_j \leq \left[\dfrac{n}{2} \right]+1$ and we have
from lemma \ref{chap5-sec1-lem1} that 
$$
|| \left(\frac{\partial}{\partial x}\right)^{\nu_1} u \ldots
\left(\frac{\partial}{\partial x}\right)^{\nu_l} u ||_{L^2} \leq \sup |
\left(\frac{\partial}{\partial x}\right)^{\nu_1} u |~ ||
\left(\frac{\partial}{\partial x}\right)^{\nu_2} u \ldots
\left(\frac{\partial}{\partial x}\right)^{\nu_l} u ||_{L^2}. 
$$

By Sobolev's lemma we have
$$
\sup | \left(\frac{\partial}{\partial x}\right)^{\nu_1} u | \leq c(n,
N, \nu_1) || u ||_{\left[\frac{n}{2} \right]+N+1} 
$$
and on the other hand $\left(\dfrac{\partial}{\partial x}\right)^{\nu_j} u
\in L^{p_j}$ with 
$$
|| \left(\frac{\partial}{\partial x}\right)^{\nu_j} u ||_L^{p_j} \leq c(n, N,
\nu_j, p_j) || u ||_{\left[\frac{n}{2} \right] + N} 
$$
for  $\dfrac{1}{p_j} \in \left[ \dfrac{| \nu_j|}{n} -
  \dfrac{N}{n}, \dfrac{1}{2}\right]- \{ 0 \} $ if  $n$ is even and
$\dfrac{1}{p_j} \in \left[ \dfrac{| \nu_j|}{n} -
  \dfrac{2N-1}{n}, \dfrac{1}{2}\right]$ if $n$  is odd (from
Prop. \ref{chap5-sec1-dashprop2}$'$). 

Denoting $\inf \dfrac{1}{p_j}$ by $\dfrac{1}{p_j}$ we see
that $\sum\limits^l_{j=2} \dfrac{1}{P_j} = \sum \left(\dfrac{| \nu_j|}{n}-
\dfrac{N}{n}\right) < \dfrac{1}{2}$ if $n$ is even and $\sum\limits^l_{j=2}
\dfrac{1}{P_j} = \sum \left(\dfrac{|\nu_j|}{n} - 
\dfrac{2N-1}{n}\right) < \dfrac{1}{2}$ if $n$ is odd. One can choose $p_2,
\ldots , p_l$ such that $\sum\limits^l_{j=2} \dfrac{1}{p_j} =
\dfrac{1}{2}$ in both the cases. An application of the generalized
H\"older's inequality with these $p_2, \ldots, p_l$ gives 
\begin{align*}
|| \left(\frac{\partial}{\partial x}\right)^{\nu_2} u \ldots ||
\left(\frac{\partial}{\partial x}\right)^{\nu_l} u ||_{L^2}  & \leq
\prod\limits^l_{j=2} 
|| \left(\frac{\partial}{\partial x}\right)^{\nu_j} u ||_{L^{pj}} \\ 
&  \leq c(n, \nu_2, \ldots \nu_l, N, p_2, \ldots, p_l) ||
 u||^{l-1}_{\left[\frac{n}{2}\right]+N} 
\end{align*}
$(\gamma)$~\pageoriginale If $| \nu_j| \geq N$ for $j=2, \ldots, l$ we
have from  
Prop. \ref{chap5-sec1-dashprop2}$'$ that 
$$
||\left(\frac{\partial}{\partial x}\right)^{\nu_1} u ||_{L^{p_1}} \leq
c(\nu_1, p_1, N, n) || u ||_{\left[\frac{n}{2} \right] + N + 1} 
$$
and $||\left(\dfrac{\partial}{\partial x}\right)^{\nu_j} u ||_{L^{pj}}
\leq c (\nu_j, p_j, N, n) || u ||_{\left[\dfrac{n}{2} \right]+N}$ 
where $p_1, \ldots , p_l$ are real numbers such that
\begin{align*}
& \begin{cases}
\frac{1}{p_1} \in \left[\frac{|\nu_1|}{n}- \frac{N+1}{n},
  \frac{1}{2} \right] - \{ 0 \}, \\[5pt] 
\frac{1}{p_j} \in \left[\frac{|\nu_j |}{n}- \frac{N}{n},
  \frac{1}{2}\right] - 0~ \text{ for even $n$ and} 
\end{cases}\\
& \begin{cases}
\frac{1}{p_1} \in \left[ \frac{|\nu_1 |}{n}- \frac{2N+1}{2n},
  \frac{1}{2}\right] \\[5pt] 
\frac{1}{p_j} \in \left[ \frac{|\nu_j |}{n}- \frac{2N-1}{2n},
  \frac{1}{2}\right] \text{ for odd $n$}.  
\end{cases}
\end{align*}

If $\dfrac{1}{P_j}$ denotes $\inf \dfrac{1}{p_j}$ we have 
\begin{gather*}
\sum^l_{j=1}\frac{1}{P_j}= \sum^l_{j=1} \left(\frac{| \nu_j |}{n}\right) -
\frac{N+1}{n} - \sum^l_{j=2} \frac{N}{n}< \frac{1}{2} \text{~ for even
$n$ and } \\ 
\sum^l_{j=1}\frac{1}{P_j}= \sum^l_{j=1} \left(\frac{|\nu_j|}{n}\right) -
\frac{2N+1}{2n} - \sum^l_{j=2}< \frac{2N-1}{2n}< \frac{1}{2}
\text{~ for odd $n$}.   
\end{gather*}

Once again choosing $p_2, \ldots,p_l$ such that $\sum\limits^l_{j=2}
\dfrac{1}{p_j}= \dfrac{1}{2}$ we obtain the desired inequality after
applying the genearlized H\"older's inequality to $||
\left(\dfrac{\partial}{\partial x}\right)^{\nu_1} u \ldots
\left(\dfrac{\partial}{\partial x}\right)^{\nu_l} u ||_{L^2}$ with these $p_1,
\ldots, p_l$ and using the estimates of the form \eqref{chap5-eq1.11}. 

By an\pageoriginale argument completely analogous to the one in the
prop. \ref{chap5-sec1-prop4} one can establish the following more
general result.   

\begin{proposition}\label{chap5-sec1-prop5}% proposition 5
Let $l$ be an arbitrary integer and $\nu_1, \ldots, \nu_1$ be $l$
multi-indices. 
\begin{enumerate}[\rm(i)]
 \item If $u_1, \ldots, u_l \in
   \mathscr{D}^{\left[\frac{n}{2} \right]+1}_{L^2}$\quad and\quad
   $\sum\limits^l_{j=1} | 
   \nu_j | \leq \left[\dfrac{n}{2} \right] + 1$\quad then

   $\left(\dfrac{\partial}{\partial x}\right)^{\nu_1} u_1 \ldots
   \left(\dfrac{\partial}{\partial x}\right)^{\nu_l} u_l 
   \in L^2$. Further 
\begin{equation*}
|| \left(\frac{\partial}{\partial x}\right)^{\nu_1} u_1 \ldots
\left(\frac{\partial}{\partial x}\right)^{\nu_l} u_l ||_{L^2} \leq c
\prod\limits^l_{j=1} || u_j ||_{\left[\frac{n}{2} \right]+1}
\tag{1.17}\label{chap5-eq1.17}  
\end{equation*} 
where the constant $c$ depends only on $n$, $\nu_1, \ldots, \nu_l$.  

\item Let $| \nu_1 | \geq | \nu_j $ for $j = 2, \ldots, l$. If $u_1
  \in \mathscr{D}^{\left[\frac{n}{2} \right]+N+1}_{L^2}$, $u_2,
  \ldots, u_l 
  \mathscr{D}^{\left[\frac{n}{2}\right]+N}_{L^2}$ and $ \sum | \nu_j | \leq
          \left[\dfrac{n}{2}\right] + N + 1$ then
          $\left(\dfrac{\partial}{\partial x}\right)^{\nu_1} u_1
          \ldots \left(\dfrac{\partial}{\partial 
            x}\right)^{\nu_l} u_l \in L^2$ and 
 \begin{equation*}
|| \left(\frac{\partial}{\partial x}\right)^{\nu_1} u_1 \ldots
\left(\frac{\partial}{\partial x}\right)^{\nu_l} u_l ||_{L^2} \leq c || u_1
||_{\left[\frac{n}{2} \right] + N + 1} \prod\limits^l_{j=2} || u_j
||_{\left[\frac{n}{2} \right]+N} 
\tag{1.18} \label{chap5-eq1.18}
 \end{equation*} 
where $c$ depends only on $n, \nu_1, \ldots, \nu_1, N$.
\end{enumerate}
\end{proposition}

\section{Regularity of some non-linear functions}\label{chap5-sec2} % Section 2 

Here we make a few remarks on the local properties of certain smooth
non-linear functions of $x, t, u$ which will be required for the
study of some quasi-linear differential equations. Let $\Omega$ denote
the set 
$$
\left\{(x, t) \Big| x \in \underbar{R}^n, 0 \leq t \leq T
\right\}.
$$

Let $f(x, t, u)$ be a function belonging to
$\mathscr{E}^{\left[\frac{n}{2}\right] + 2}(\Omega \times \mathbb{C})$. For a
 fixed function $\alpha \in \mathscr{D}(\underbar{R}^n)$ we
denote $\alpha(x) f (x, t, u)$\pageoriginale by
$\tilde{f}(x,t,u)$. $\alpha$  
localizes $f(x, t, u)$ in the $x$-space. We use the following
abbreviations $\left(\dfrac{\partial}{\partial x},
\dfrac{\partial}{\partial u}\right)^\beta$ stands for a derivation of order
$|\beta|$ with respect to $x$ and $u$; $F(x,t)$, $\tilde{F}(x, t),
G(x, t), \ldots$ stand respectively for 
$$
f(x, t, u(x, t)),\ \tilde{f}
(x, t, u(x, t)),\ g(x, t, u(x, t))\ldots. 
$$

Let $U$ be the subset of
$\Omega \times \mathbb{C}$ defined by 
\begin{equation*}
U= \{ (x, t, u) | (x, t) \in \Omega, | u | \leq \sup_{\Omega}
|u(x, t) | \}. \tag{2.1}\label{chap5-eq2.1} 
\end{equation*}

Throughout this section $c_1(n)$, $c_2(n), \ldots$ denote constants
depending only on $n$. 

\setcounter{lemma}{0}
\begin{lemma}\label{chap5-sec2-lem1}% lemma 1
If $u \in \mathscr{D}^{\left[\frac{n}{2} \right]+1}_{L^2} [0, T]$ then
$\tilde{F} = \tilde{F} (x, t) \in
\mathscr{D}^{\left[\frac{n}{2} \right]+1}_{L^2} [0, T]$ and 
\begin{equation*}
|| \tilde{F} ||_{\left[\frac{n}{2} \right]+1} \leq c_1 (n) M \left\{
||1+ || u ||^{\left[\frac{n}{2}\right]+1}_{\left[\frac{n}{2}\right]+1}
\right\} \tag{2.2} \label{chap5-eq2.2} 
\end{equation*}
where $M= \max\limits_{|\beta|\leq\frac{n}{2}+1} \sup\limits_U \Big|
\left(\dfrac{\partial}{\partial x}, \dfrac{\partial}{\partial u}\right)^\beta
\tilde{f}(x, t, u) \Big|$. 
\end{lemma}

Before proceeding with the proof of the lemma \ref{chap5-sec2-lem1} we
make the following two remarks. Let $u \in
\mathscr{D}^{\left[\frac{n}{2}\right]+1}_{L^2} [0, T]$. Let
$\varphi_\varepsilon$ be the mollifiers in the $x$-space 
and let $u_\varepsilon (x, t) = u(x, t) *_{(x)} \varphi_\varepsilon
(x)$; then 
\begin{enumerate}[(i)]
\item $u_\varepsilon \in \mathscr{B}^0_x [0, T]$ and
\begin{equation*}
| u_\varepsilon (x, t) |_{\mathscr{B}^0_x} \leq | u (x,
t)|_{\mathscr{B}^0_x}. \tag{2.3}\label{chap5-eq2.3}  
\end{equation*}

This is an immediate consequence of lemma  \ref{chap3-sec1-lem1} \S\
\ref{chap3-sec1} of Chap. \ref{chap3}. 

\item $u_\varepsilon \in \mathscr{D}^s_{L^2} [ 0,
  T]$\pageoriginale and 
$$
|| u_\varepsilon ||_s \leq || u ||_s ~ \text{ for }~ 0 \leq s \leq
\left[\frac{n}{2} \right]+1. 
$$
\end{enumerate}

In fact, we observe that $\hat{\varphi}_\varepsilon (\xi) =
\hat{\varphi}(\varepsilon \xi) \to \hat{\varphi}(0)= 1$ as
$\varepsilon \to 0$. Consider 
\begin{align*}
|| u_\varepsilon - u ||_s & = || (1+ | \xi |)^s (\hat{u}_\varepsilon
(\xi, t) - \hat{u} (\xi, t)) ||_{L^2} \\ 
& = || (1+ | \xi |)^s \hat{u} (\xi, t) - (\hat{\varphi}_\varepsilon
(\xi ) -1) ||_{L^2}  
\end{align*}
which converges to 0 as $\varepsilon \to 0$. Hence
$$
|| u_\varepsilon || \leq || u || + || u_\varepsilon - u ||
$$
implies the assertion.

\begin{proofofthelemma}% proof of the lemma
 Through out the proof the derivatives with respect to $x$ are taken
 in the sense of distributions. Denoting $\tilde{f}(x, t,
 u_\varepsilon (x, t))$ by $\tilde{F}_\varepsilon (x, t)$ we see that 
 $\tilde{F}_\varepsilon (x, t) \to F(x, t)$ as $ \varepsilon \to
 0$. For,  
$$ 
|| \tilde{F}_ \varepsilon (x, t) - F (x, t) ||_{L^2} = || \left[
  \frac{\partial \tilde{f}}{\partial u} \right] (x, t, u(x, 
t)) \cdot (u_\varepsilon (x, t) - u(x, t)) ||_{L^2} 
$$
which tends to 0 as $\varepsilon \to 0$. Now, for $1 \leq j \leq n$,  
$$
\frac{\partial}{\partial x_j} F(x, t) = \lim_{\varepsilon \to 0}
\frac{\partial}{\partial x_j} F_\varepsilon (x, t) 
$$
where the limit is taken in the space $L^2$. In fact, we can write 
$$
\frac{\partial}{\partial x_j} \tilde{F}_\varepsilon (x, t) =
     \left[\frac{\partial \tilde{f}}{\partial x_j}\right] (x, t, u_\varepsilon
     (x,t)) + \left[ \frac{\partial \tilde{f}}{\partial u}\right] (x,
     t, u_\varepsilon (x, t)) \cdot \frac{\partial
       u_\varepsilon}{\partial x_j} (x, t) 
$$
in the sense of distributions. This function tends to 
$$
\left[\frac{\partial \tilde{f}}{\partial x_j}\right] (x, t, u(x, t)) + \left[
  \frac{\partial \tilde{f}}{\partial u}\right] (x, t, u(x,
t))\cdot \frac{\partial u}{ \partial x_j} (x, t). 
$$\pageoriginale
in the space $L^2 [ 0, T]$, because $u \in
\mathscr{D}^{\left[\frac{n}{2}\right]+1}_{L^2} [0, T]$ implies that
$\left[\dfrac{\partial \tilde{f}}{\partial x_j}\right]\break (x, t, u(x, t))$,
        $\left[\dfrac{\partial \tilde{f}}{\partial u}\right](x, t,
u(x, t))$ belong to the space $\mathscr{B}^0_x [0, T]$.  
\end{proofofthelemma}

For a multi-index $\nu$ with $| \nu  | \leq [\dfrac{n}{2}] + 1$ we have
\begin{equation*}
\left(\frac{\partial}{\partial x}\right)^\nu \tilde{F}_\varepsilon (x, t) =
  \sum_{\substack{|\rho_j |\leq|\nu| \\ {l\leq|\nu|}}}
  C_{\rho_1 \ldots \rho_l} g_{\rho_1 \ldots \rho_l}(x, t,
  u_\varepsilon (x, t)) \prod\limits^l_{j=1} \left(\frac{\partial}{\partial 
    x}\right)^{\rho_j} u_\varepsilon (x, t) \tag{2.4}\label{chap5-eq2.4}
\end{equation*}
where $C_{\rho_1 \ldots \rho_l}$ are constants and $g_{\rho_1 \ldots
  \rho_l}(x, t, u)$ is one of the derivatives
$\left(\dfrac{\partial}{\partial x},\dfrac{\partial}{\partial
  u}\right)^\beta \tilde{f} (x, t, u)$ of orders $| \beta | \leq | \nu
|$. This identity 
is again taken in the sense of distributions in the $x$-space. In view
of the prop. \ref{chap5-sec1-prop4} \S\ \ref{chap5-sec1}, the function. 
\begin{equation*}
g_{\rho_1 \ldots \rho_l} (x, t, u (x, t))\prod\limits^l_{j=1}
\left(\frac{\partial}{\partial x}\right)^{\rho_j}u (x, t) \equiv G_{\rho_1 \ldots
  \rho_l} (x, t) \prod\limits^1_{j=1} \left(\frac{\partial}{\partial
  x}\right)^{\rho_j}u (x, t) \tag{2.5} \label{chap5-eq2.5}
\end{equation*}
belongs to $L^2[0, T]$. Setting
\begin{equation*}
J_\varepsilon (x) = G_{\rho_1 \ldots \rho_l,\varepsilon} (x, t)\prod\limits^l_{j=1}
\left(\frac{\partial}{\partial x}\right)^{\rho_j}u_\varepsilon (x, t)-
G_{\rho_1 \ldots \rho_l} (x, t) \prod\limits^l_{j=1}
\left(\frac{\partial}{\partial x}\right)^{\rho_j}u (x, t)
\tag{2.6}\label{chap5-eq2.6}  
\end{equation*}
we have
{\fontsize{9pt}{11pt}\selectfont
\begin{align*}
|| J_\varepsilon ||_{L^2} \leq &  M \left\{ || (u_\varepsilon - u)
(x, t) \prod\limits^l_{j=1} \left(\frac{\partial}{\partial
  x_j}\right)^{\rho_j} u (x, t) ||_{L^2}\right. \\ 
 & + \sum^l_{j=1} || u (x, t)\left(\frac{\partial}{\partial
  x}\right)^{\rho_l} u 
 (x, t) \ldots \left(\frac{\partial}{\partial x}\right)^{\rho_{j}} u (x,
 t)\left(\frac{\partial}{\partial x}\right)^{\rho_{j+1}} (u_\varepsilon - u) (x,
 t) \left(\frac{\partial}{\partial x}\right)^{\rho_l} \\
& \hspace{6cm} u_\varepsilon (x, t)  ||_{L^2}.  
\end{align*}}\relax

The\pageoriginale  prop. \ref{chap5-sec1-prop4} of \S\ \ref{chap5-sec1}
implies that 
\begin{equation}
|| J_\varepsilon ||_{L^2} \leq c_2 (n) M || (u_\varepsilon - u)
||_{\left[\frac{n}{2} \right]+1} || u ||^{1-1}_{\left[\frac{n}{2}
||\right]+1} \tag{2.7}\label{chap5-eq2.7}  
\end{equation}
which tends to 0 as $\varepsilon \to 0$. This proves that
\begin{equation*}
\left(\frac{\partial}{\partial x}\right)^\nu \tilde{f} ((x, t), u(x, t)) = \sum
  c_{\rho_l \ldots \rho_l} G_{\rho_1 \ldots \rho_l}(x, t)
  \prod\limits^l_{j=1} \left(\frac{\partial}{\partial x}\right)^{\rho_j} u(x,
  t). \tag{2.8}\label{chap5-eq2.8} 
\end{equation*}

Again applying Prop. \ref{chap5-sec1-prop4} \S\ \ref{chap5-sec1} to
\eqref{chap5-eq2.8} it is easy to see that the 
estimate \eqref{chap5-eq2.2} holds. The continuity in $t$ of $F$ is proved as
before. This completes the proof of the lemma. 

The following results are proved in exactly the same manner as the
lemma \ref{chap5-sec1-lem1}. 

\setcounter{corollary}{0}
\begin{corollary}\label{chap5-sec2-coro1}%%% corollary 1
If $f(x, t, u) \in
\mathscr{E}^{\left[\frac{n}{2}\right]+N+1}(\Omega \times 
  \in )$ and $ u \in
  \mathscr{D}^{\left[\frac{n}{2}\right]+N+1}_{L^2}[0, T]$ 
then 
\begin{equation*}
|| \tilde{F} (x, t) ||_{\left[\frac{n}{2}\right] + N + 1} \leq C_3 (n) M_1
\left\{ 1+ (1+ || u
||^{\left[\frac{n}{2}\right]+N}_{\left[\frac{n}{2}\right]+N+1}) || u
||_{\left[\frac{n}{2}\right]+N+1} \right\} \tag{2.9}\label{chap5-eq2.9}  
\end{equation*}
where $M_1 = \max\limits_{| \beta | \leq \left[\frac{n}{2}\right] + N
  + 1} \sup\limits_U \left| \left(\dfrac{\partial}{\partial x},  \;
\dfrac{\partial}{\partial u}\right)^\beta \tilde{f}(x, t, u) \right|$. 
\end{corollary}

\begin{corollary}\label{chap5-sec2-coro2}% corollary 2
If $f(x, t, u_1, \ldots, u_s ) \in
\mathscr{E}^{\left[\frac{n}{2}\right] 
  + 2}( \Omega \times \mathbb{C}^s)$~ and~ $u_j \in
\mathscr{D}^{\left[\frac{n}{2}\right] + 1}\break [0, T] (1 \leq j \leq s)$
then $\alpha 
(x) \in \mathscr{D}$ implies that 
$$
\alpha(x) f(x, t, u_1(x,t), \ldots , u_s (x, t)) \in 
\mathscr{D}^{\left[\frac{n}{2}\right]+1}_{L^2} [0, T] 
$$
and 
\begin{gather*}
|| \alpha(x) f (x, t, u_1 (x, t), \ldots , u_s(x, t))
||_{\left[\frac{n}{2}\right]+1}  \\ 
\leq C_4 (n) M_2 \{ 1+ \sum\limits^s_{j=1} || u_j (x, t)
||^{\left[\frac{n}{2}\right]+1}_{\left[\frac{n}{2}\right]+1}\tag{2.10} 
\label{chap5-eq2.10}   
\end{gather*}
where\pageoriginale $M_2 = \max\limits_{| \beta|\leq
  \left[\frac{n}{2}\right]+1} \sup\limits_{U_s} |
\left(\dfrac{\partial}{\partial x}, \; 
\dfrac{\partial}{\partial u}\right)^\beta [\alpha(x) f(x, t, u_1(x, t),
  \ldots, u_s (x, t))]|$. 
\end{corollary}

Here $U_s$ is the subset of $\Omega \times \mathbb{C}^s$ defined by 
\begin{equation*}
U_s = \left\{ (x, t, u_1, \ldots , u_s) \bigg| | u_j | \leq
\sup\limits_\Omega | u_j (x, t)|, ~ 1 \leq j \leq s
\right\}. \tag{2.11}\label{chap5-eq2.11}   
\end{equation*}

\begin{corollary}\label{chap5-sec2-coro3}% corollary 3
If $f(x, t, u)$ is a vector valued function
$$ 
\begin{pmatrix}
f_1(x, t, u) \\ 
\vdots \\
f_m(x, t, u)
\end{pmatrix}
$$
with $f_k \in \mathscr{E}^{\left[\frac{n}{2}\right] + 2}
(\Omega \times 
\mathbb{C})$ for $1 \leq k \leq m$ and $u \in
\mathscr{D}^{\left[\frac{n}{2}\right]+1}_{L^2} [0, T]$ then $\alpha
\in \mathscr{D}$ implies that $\alpha (x) f_k (x, t, u(x, t))$
belong to the space $\mathscr{D}^{\left[\frac{n}{2}\right]+1}_{L^2}
[0, T]$ and   
\begin{align*}
|| \alpha (x) f (x, t, u(x, t)) ||_{\left[\frac{n}{2}\right]+1} & =
\sum_k || \alpha (x) f_k (x, t, u(x, t)) ||_{\left[\frac{n}{2}\right]+1}\\ 
&  \leq C_5 (n) M_3(1+ || u (x, t)
  ||^{\left[\frac{n}{2}\right]+1}_{\left[\frac{n}{2}\right]+1})
\tag{2.12}\label{chap5-eq2.12}  
\end{align*}
where $M_3 = \max\limits_{k,|\beta | \leq \left[\frac{n}{2}\right]+1}
\sup\limits_U | \left(\dfrac{\partial}{\partial x},
\dfrac{\partial}{\partial  u}\right)^\beta [ \alpha (x) f_k (x, t, u)]|$.  
\end{corollary}

Similar results hold when $u$ is a vector $(u_1, \ldots, u_s)$ and
when $u_j \in \mathscr{D}^{\left[\frac{n}{2}\right]+N+1}_{L^2} [ 0,
  T]$. 

Finally we state a result which is a consequence of these and will be
of importance. 

\begin{corollary}\label{chap5-sec2-coro4}% corollary 4
Let\pageoriginale $f(x, t, u_1, \ldots, u_s ) \in
\mathscr{E}^{\left[\frac{n}{2}\right]+2} (\Omega \times \mathbb{C}^s)$
and $\nu_1, \ldots, \nu_s $ denote multi-indices. If $u
\in\mathscr{D}^{\left[\frac{n}{2}\right]+m+1}[0, T]$ and
$|\nu_1| + \cdots + |\nu_s| \leq m$ then  
$$
\alpha (x) f \left(x, t, \left(\dfrac{\partial}{\partial
  x}\right)^{\nu_1}  (u (x, t)), \ldots,
\left(\dfrac{\partial}{\partial x}\right)^{\nu_s} u(x, 
t)\right)\in \mathscr{D}^{\left[\frac{n}{2}\right]+1}_{L^2} [0, T] 
$$
and 
\begin{gather*}
|| \alpha (x) f\left(x, t, \ldots, \left(\frac{ \partial }{\partial
  x}\right)^{\nu_1} 
u (x, t ), \ldots \right)||_{\left[\frac{n}{2}\right] + 1}\\
 \leq M' c (n, m)\left\{ 1 +
|| u
||^{\left[\frac{n}{2}\right]+1}_{\left[\frac{n}{2}\right]+m+1}\right\},
\tag{2.13}\label{chap5-eq2.13}  
\end{gather*}
where $M' = \max\limits_{|\beta| \leq \left[\frac{n}{2}\right] + 2}
\sup\limits_{U'_s} \left(\dfrac{\partial }{\partial x},
\dfrac{\partial}{\partial u_1 }, \ldots, \dfrac{\partial }{\partial
  u_s}\right)^\beta [\alpha (x)f (x, t, u_1, \ldots, u_s)]|$. 

Here again  
\begin{equation*} 
U'_s = \left\{(x, t, u_1, \ldots, u_2) \bigg|(x, t) \in
\Omega, |u_j| \leq \sup | \left(\frac{\partial }{\partial
  x}\right)^{\nu_j} u (x, t)| , 1 \leq j \leq s \right
\}. \tag{2.14}\label{chap5-eq2.14}  
\end{equation*}
\end{corollary}

\begin{proof}
From Prop. \ref{chap5-sec1-prop4} \S\ \ref{chap5-sec1} we have that,
if $u_1, \ldots, u_s \in 
\mathscr{D}^{\left[\frac{n}{2}\right] + 1}_{L^2}[0, T]$ and if $
\nu_1, \ldots, \nu_s$  are multi-indices with $\sum |\nu_j| \leq \left[
  \dfrac{n}{2}\right]+ 1$ then   
\begin{equation*}
|| \prod\limits^s_{j = 1} \left(\frac{\partial}{\partial
  x}\right)^{\nu_j} u ||_{L^2} 
\leq C(n, \nu_1, \ldots, v_s) \prod\limits^{s}_{j=1}|| u_j
||_{\left[\frac{n}{2} \right]+1 } .\tag{2.15}\label{chap5-eq2.15}  
\end{equation*}

Taking $u_j = \left(\dfrac{ \partial}{\partial x}\right)^{\nu_j} u$ we
apply this inequality and the rest of the proof is the same as in the previous
corollaries.  
\end{proof}

\section{An example of a semi-linear equation}\label{chap5-sec3} % section 3

In this section we consider an example of a semi-linear partial
differential equation of the second order and we recall a
theorem\pageoriginale on 
the existence of solutions of the Cauchy problem for such an
equation. This result is due to K. J\"orgens (see: Das
Anfangswertproblem in Grossen fur eine Klasse nichtlinearer
Wellengleichungen, Math.Zeit., 77 (1961), 295-308). This theorem will be
proved in \S \ref{chap5-sec5}.  

Let $u \to f(u)$ be a real valued infinitely differentiable function
defined in $- \infty < u < \infty$. We consider the following
semi-linear wave equation  
\begin{equation*}
\left(\frac{\partial}{\partial t}\right)^{2} u - \Delta u + f(u) =
0. \tag{3.1}\label{chap5-eq3.1} 
\end{equation*}

We assume that $f(0) = 0$. We shall show that, under certain
conditions on the function $f$, for a given smooth initial data
$(u_0, U_1)$ on the hyperplane $t=0$ there exists a unique solution
$u$ of \eqref{chap5-eq3.1} in $t \geq 0$ with $u(x, 0) = u_0 (x),
\dfrac{\partial}{\partial t} u (x, 0) = u_1 (x)$. For instance, we
shall show that if $u_0 \in \mathscr{D}^{\left[\frac{n}{2}\right]+2}_{L^2} \cap
\mathscr{E}^1$, $u_1 \in \mathscr{D}^{
  \left[\frac{n}{2}\right]+1}_{L^2} 
\cap \mathscr{E}^1$ then there exists a unique solution $u$ of
\eqref{chap5-eq3.1} such that  
$$
u \in \mathscr{D}^{\left[\frac{n}{2}\right]+2}_{L^2} \cap
\mathscr{E}', \; 
\frac{\partial u}{\partial t} \in
\mathscr{D}^{\left[\frac{n}{2}\right]+1}_{L^2} \cap \mathscr{E}'
$$
both depnding continuously on $t$ in $0 \leq t \leq \infty $ and such
that  
$$
u(x, 0) = u _0 (x), \frac{\partial u}{\partial t} (x, 0) = u_1 (x).  
$$

Under the assumption $f(0)=0$ one can also show that if the
supports of $u_0$ and $u_1$ are contained in $\{ |x| \leq R_0\}$ then
the supports of $u, \dfrac{\partial u}{\partial t}$ are contained in
$\{ |x| \leq R_0 + t\}$.  

Let\pageoriginale $u_0 \in \mathscr{D}^{\left[\frac{n}{2}\right] +
  3}_{L^2} \cap 
\mathscr{E}'$, $u_1 \in \mathscr{D}^{\left[\frac{n}{2}\right] +
  2}_{L^2} \cap \mathscr{E}'$ be given with their supports contained
in $\{|x| \leq R_0\}$. Assume that a solution of \eqref{chap5-eq3.1}
with the initial 
data $(u_0, u_1) $ on $t = 0$ exists locally. More precisely we assume
that there exists a $t_0 > 0$ such that there exists a solution $u$ of
\eqref{chap5-eq3.1} defined in $\{ x  \in \underline{R}^n , 0 \leq t \leq
t_0 \}$ with the property that  
\begin{enumerate}
\item[(1)] $u \in (\mathscr{D}^{\left[\frac{n}{2}\right] +
  3}_{L^2} \cap 
  \mathscr{E}^1) [0, t_0]\dfrac{\partial u}{\partial t}
  \in (\mathscr{D}^{\left[\frac{n}{2}\right] + 2}_{L^2} \cap
  \mathscr{E}^1) [0, t_0]$,

\noindent
 $\left(\dfrac{\partial}{\partial t}\right)^2 u
  \in(\mathscr{D}^{\left[\frac{n}{2}\right]+1}_{L^2} \cap
  \mathscr{E}^1) [ 0, t_0])$ and  

\item $u(x, 0) = u_0 (x)$, $\dfrac{\partial u}{\partial t} (x, 0) =
  u_1 (x)$.  
\end{enumerate}

We say that an \textit{a priori} estimate in the $L^2$-sense for the
solution of the Cauchy problem for \eqref{chap5-eq3.1} of order
$\left[\dfrac{n}{2}\right]+ 1$ holds if the following conditions is
satisfied: for any given initial data $(u_0 , u_1)$ with $u_0 \in
\mathscr{D}^{\left[\frac{n}{2}\right]+3}_{L^2}  \cap \mathscr{E}^1$, $u_1
\in \mathscr{D}^{\left[\frac{n}{2}\right]+2}_{L^2} \cap \mathscr{E}'$
and a number $T > 0$ there exists a constant $c \equiv c(T, u_0, u_1)$
such that  
$$
|| u (t) ||_{\left[\frac{n}{2} \right]+1} \leq c 
$$ 
for all $0 \leq t \leq T$. where $u$ exists an $u(x, 0) = u_0$,
$\dfrac{\partial u}{\partial t} (x, 0) = u_1 (x)$. $c$ is called an a
priori bound.  

The following is a special case of a theorem that will be proved in \S\
\ref{chap5-sec5}. We state it here to motivate
Prop. \ref{chap5-sec3-prop1}.   

\setcounter{theorem}{0}
\begin{theorem}\label{chap5-sec3-thm1} %theorem 1
Let\pageoriginale $f$ be an infinitely differentiable function in 
$-\infty < u < \infty $ with $f(0) = 0$. Assume that a priori estimate in the
$L^2$-sense for the solution of the Cauchy problem for
\eqref{chap5-eq3.1} of order 
$\left[\dfrac{n}{2}\right]+ 1$ holds. Then, for any intial data $(u_0,
u_1)$ with 
$u_0 \in \mathscr{D}^m_{L^2} \cap  \mathscr{E}^1$, $u_1
\in D^{m -1}_{L^2} \cap \mathscr{E}^1 (m \geq
\left[\dfrac{n}{2} \right]+ 3)$ there exists a unique solution $u$ of
\eqref{chap5-eq3.1} such that  
\begin{enumerate}[\rm(1)]
\item $u \in \mathscr{D}^m_{L^2} \cap \mathscr{E}^1,
  \dfrac{\partial u}{\partial t}\in \mathscr{D}^{m-1}_{L^2}
  \cap \mathscr{E}^1$, $\left(\dfrac{\partial}{\partial t}\right)^2 u
  \in\mathscr{D}^{m -2}_{L^2} \cap \varepsilon'$ all depending
  continuously on $t$,  

\item $u(x, 0) = u_0(x)$, $\dfrac{\partial u}{\partial t}(x, 0) = u_1
  (x)$.  
\end{enumerate}
\end{theorem}

\setcounter{proposition}{0}
\begin{proposition}\label{chap5-sec3-prop1} %proposition 1 
Let $f$ be an infinitely differentiable function in $- \infty < u <
\infty$ with $f(0) = 0 $. Then  
\begin{enumerate}[\rm(i)]
\item for $n = 1$ an a priori estimate of order one for the solutions
  of the Cauchy problem for \eqref{chap5-eq3.1} holds when  

{\rm(a)}~ $\int\limits^u_0 f(v) dv \equiv F(u) > - L_0$ ($L_0$ a
  positive constant),  


\item assume further that $f(u)$ satisfies the condition  

{\rm(b)}~ if $n = 2 $ there exist $\alpha$ and $k$ such that 
$$
|\frac{df(u)}{du}| \leq \alpha (1+|u|)^k  
$$
and if $n = 3$ there exists an $\alpha$ such that 
$$
|\frac{df(u)}{du}|\leq \alpha (1+u^2).  
$$

Then an a priori estimate of order 2 for solutions of the Cauchy 
problem for \eqref{chap5-eq3.1} holds.  
\end{enumerate}
\end{proposition}

\begin{proof}% proof 
Assume\pageoriginale that $u_0 \in \mathscr{D}^m_{L^2} \cap
\varepsilon^1$, $u_1 \in \mathscr{D}^{m-1}_{L^2}\cap
\mathscr{E}^1 (m \geq \left[\frac{n}{2} \right] + 3)$ are
given and also 
that there exists a solution $u$ of the Cauchy problem for
\eqref{chap5-eq3.1} with initial data $(u_0, u_1)$ such that  
{\fontsize{10pt}{12pt}\selectfont
$$
u \in (\mathscr{D}^{m}_{L^2} \cap \mathscr{E}') [0, T],
\frac{\partial u}{\partial t} \in(\mathscr{D}^{m-1}_{L^2}
\cap \mathscr{E}') [0, T], \; \left(\frac{\partial}{\partial
  t}\right)^2 u \in (\mathscr{D}^{m-2}_{L^2} \cap 
\mathscr{E}')[0, T].  
$$}\relax

Let $R$ be a number such that $R_0 + t < R$ for $t \leq T$.  
\begin{equation*}
{\rm(i)}\quad \text{ Set } E_1 (t) = \int\limits_{|x|<R}
  \left[\dfrac{1}{2} \left\{\left(\frac{\partial u}{\partial t}\right)^2 +
    \sum\limits^n_{j=1} \left(\frac{\partial 
    u}{\partial x_j}\right)^2 \right\} + F(u) + c \right] dx 
 \end{equation*}
 where $c$ is a constant to be chosen later. Differentiating with
 respect to $t$  
$$
\frac{d}{dt} E_1 (t) = \int\limits_{|x|\leq R} \left\{ \frac{\partial 
  u}{\partial t} \left(\frac{\partial }{\partial t}\right)^2 u + \sum_{j}
   \frac{\partial u}{\partial x_j }\left(\frac{\partial
  u}{\partial x_j }\right)  \left(\frac{\partial}{\partial t}\right) u + f (u)
\frac{\partial u}{\partial t}\right\} dx.  
$$

Since $\dfrac{\partial u}{\partial x_j}$, $\left(\dfrac{\partial}{\partial
  x_j}\right) \left(\dfrac{\partial}{\partial t}\right)u $ have
compact supports the 
second term in the right hand side becomes after integration by parts  
$$
\int \Delta^u \cdot\frac{\partial u}{\partial t} dx 
$$
and so we have 
$$
\frac{d}{dt} E_1 (t) = \int\limits_{|x|\leq R} (\Box u + f (u))
\frac{\partial u}{\partial t} \cdot dx = 0 
$$
(where $\Box = \left(\dfrac{\partial }{\partial t}\right)^2 - \Delta$)
since $\Box u + f (u) = 0 $. Hence $E_1 (t)$ is a constant $= E_1 0$.  
\end{proof}

Taking $c>L_0$ we have $F(u)+ c>0$ and so 
\begin{equation*}
\int \frac{1}{2} \left\{ \left(\frac{\partial u}{\partial t}\right)^2
+ \sum_j \left(\frac{\partial u}{\partial x_j}\right)^2 \right\} dx
\leq E_1 (t) = E_1 (0). \tag{3.2}\label{chap5-eq3.2} 
\end{equation*}\pageoriginale

Since the support of $u$ is compact there exists $c_1$ such that  
\begin{equation*}
|| u ||_{L^2}\leq c_1 \sum\limits_{j} || \frac{\partial u}{\partial
  x_j}||_{L^2}. \tag{3.3}\label{chap5-eq3.3}  
\end{equation*}

In fact, $u \in \mathscr{D}^m_{L^2} \subset
\mathscr{D}^{\frac{n}{2}+3}_{L^2}$ implies that $u$ is in
$\mathscr{E}^1$. We can hence write  
$$
u(x, t) = \int^{x_j}_{- \infty} \frac{\partial u}{\partial x_j} (y, t)
dy_j, j = 1, \ldots, n.   
$$

Using Cauchy-Schwarz inequality and calculating the norm of $u$ in
$L^2$ we obtain \eqref{chap5-eq3.3}. The estimates
\eqref{chap5-eq3.2}, \eqref{chap5-eq3.3} together show that 
an a priori estimate of order one holds thus proving (i).  

\noindent
(ii) Differentiating \eqref{chap5-eq3.1} with respect to $x_j$ we have 
\begin{equation*}
\Box  u_j + \frac{df}{du} u_j = 0 \text{~ where~ }  u_j = \frac{\partial
  u}{\partial x_j}.  \tag{3.4}\label{chap5-eq3.4}
\end{equation*}

Denoting $\dfrac{\partial}{\partial x_j} \dfrac{\partial }{\partial
  x_k} u$ by $u_{jk}$ and $\dfrac{\partial }{\partial
  x_j}\dfrac{\partial }{\partial t} u$ by $u_{jt}$ we define  
$$
E_2 (t) = \sum_{j=1}^{n} \int \frac{1}{2} \left(u^2_{jt} + \sum^n_{k=1} 
u^2_{jk}\right) \, dx.  
$$

Differentiating $E_2 (t)$ with respect to $t$ 	 
\begin{align*}
\frac{d \, E_2}{dt} (t) & = \sum_j \int \left(u_{jt} \cdot u_{jtt} + \sum_k
u_{jk} \cdot u_{jkt}\right) \, dx \\ 
& = \sum_j \int (\Box u_j ) \cdot u_{jt} \, dx 
\end{align*}
since\pageoriginale $\sum\limits_k \int u_{jk}\cdot u_{jkt} dx = -
\sum\limits_k \int u_{jkk} \cdot u_{jt} \, dx$ by integration by
parts. using the equation \eqref{chap5-eq3.4} we obtain   
$$
\frac{d E_2}{dt} (t) = - \sum_j \int \frac{df}{du} \cdot u_j u_{jt} \,
dx.   
$$

From the generalized H\"older's inequality it follows that  
$$
| \int \frac{df}{du} \cdot  u_j \cdot u_{jt}\,dx | \leq || u_{jt}
||_{L^2} || u_j ||_{L^6} \cdot || \frac{df}{du}||_{L^3}.  
$$

If $n = 2$ by Prop. \ref{chap5-sec1-prop2} \S\ \ref{chap5-sec1} we see that 
$$
|| u_j ||_{L^6} \leq c_1 (n) || u ||_2  
$$
where $c_1(n)$ is a constant depending only on $n$. From (b) we
have, with a suitable constant $\alpha'$ depending on $\alpha$, since
$u$ has compact support in $|x| < R$ 
\begin{align*}
\int\limits_{|x|<R} |\frac{df}{du}|^3 \, dx & \leq \alpha'^{3} \int
(u^6 + 1) dx \leq \alpha'^3 ||u||^6_{L^6} + C_2(\alpha', R, n) \\ 
& \leq C_3 (n, \alpha', R) (1 + || u ||^6_1). 
\end{align*}

These estimates together show that 
$$
\dfrac{dE_2}{dt}(t) \leq \gamma_1 E_2 (t).  
$$

Multiplying by $e^{-\gamma_1t}$ and integrating with respect to $t$ we
obtain  
\begin{equation*} 
E_2 (t) \leq E_2 (0) \cdot e^{\gamma_{1}t}. \tag{3.5}\label{chap5-eq3.5}
\end{equation*}

This proves that there is an a priori bound of order 2. A similar
argument holds for the case $n=3$. This completes the proof of the
proposition.  

\medskip
\noindent
\textbf{Exercise.}~ Consider\pageoriginale the semi-linear hyperbolic
equation  
\begin{equation*}
M[u] + f(u) = 0 \tag{3.6}\label{chap5-eq3.6}
\end{equation*}
where 
\begin{gather*}
M = \left(\dfrac{\partial}{\partial t}\right)^2 - \sum a_{jk}(x, t)
\frac{\partial^2}{\partial x_j \partial x_k} - \sum a_j (x,
t)\dfrac{\partial}{\partial x_j} - a_0 (x, t)
\dfrac{\partial}{\partial t}\\ 
\text{with~ } (1^\circ)~ a_{jk} \in B^1 [0, T],
\; \dfrac{\partial}{\partial t} a_{jk} \in B^0 [0, T], a_{0} a_j
\in B^0 [0, T],\\ 
 (2^\circ)~ \sum a_{jk} (x, t) \xi_j \xi_k \geq \delta |\xi |^2, \; \delta > 0
\text{~ is a constant.}  
\end{gather*}

Prove, under the same hypothesis on $f$ as in
Prop. \ref{chap5-sec3-prop1}, that an a 
priori estimate of order 2 holds and consequently there exists a
global solution of \eqref{chap5-eq3.6}.  

\section[Existence theorems for first order systems of.....]{Existence theorems for first order systems of\hfill\break semi-linear
  equations}\label{chap5-sec4} % section 4  

In this section we establish theorems on the existence of local and
global solutions of the Cauchy problem for semi-linear regularly
hyperbolic first order systems of differential equations.  

Let $\Omega$ be the set $\{ (x, t) | x \in \underline{R}^n, 0
\leq t \leq  T \}$. Consider the semi-linear first order system of
equations  
\begin{equation*}
M[u] = \frac{\partial u}{ \partial t} - \sum^n_{k = 1} A_k (x, t)
\frac{\partial u}{ \partial x_k} = f(x, t, u), \tag{4.1}\label{chap5-eq4.1} 
\end{equation*}
where we assume that the coefficients $A_k$ of $M$ and $f$ satisfy the
following regularity conditions:  
\begin{enumerate}[(a)]
\item $A_k \in B^{[\frac{n}{2}] +2} [0, T]$, $\dfrac{\partial
  A_k}{\partial t} \in \mathscr{B}^0 [0, T]$~ and  

\item $f \in \mathscr{E}^{[\frac{n}{2}]+3}$ in $\Omega \times
  \underline{C}$.  
\end{enumerate}

We also\pageoriginale assume that $M$ is regularly hyperbolic. As we
shall show 
later that under stronger differentiabililty conditions on the
coefficients $A_k$ and $f$ the Cauchy problem has more regular
solutions: For instance we assume  
\begin{enumerate}
\item[(a$'$)] $A_k \in B^m [0, T]$, $\dfrac{\partial A_k}{\partial t}
  \in B^0 [0, T]$~ and  

\item[(b$'$)] $f \in \mathscr{E}^{m+1}$  in $ \Omega \times
  \underline{C}$,   
\end{enumerate}
where $m \geq \left[\dfrac{n}{2}\right] + 2$. 

Although we are interested here mainly in the local existence theorem
we consider the following equation $(4.1)'$ instead of
\eqref{chap5-eq4.1} in order 
to elucidate our construction. We decompose $f$ into two parts  
$$ 
f(x,t,u) = f(x,t,0) + (f(x,t,u) - f (x,t,0)) = f (x,t,0) + 
g(x,t,u) 
$$	
where
\begin{equation*}
 g(x,t,u ) = f(x, t, u) - f(x, t, 0). \tag{4.2}\label{chap5-eq4.2}
\end{equation*}

We remark that $g(x,t,0) \equiv 0$. Define the function
$\tilde{f}\in \mathscr{E}^{\left[\frac{n}{2}\right] + 3} $ in
$\Omega \times \underline{C}$ by setting   
$$
\tilde{f}(x, t, u) = \alpha (x) g (x, t, u) + \beta (x) f(x, t, 0)
$$
where $\alpha, \beta \in \mathscr{D}$, and consider the first
order system of semi-linear equaions  
\begin{equation*}
M[u] = \tilde{f}. \tag*{$(4.1)'$}
\end{equation*}

Clearly $\tilde{f} = f$ whereever $\alpha (x) = 1 = \beta (x)$. If the
initial data $u_0 \in \mathscr{E}'$\pageoriginale has compact
support then, 
since $\beta (x) \tilde{f} (x, t, u)$ has compact support in the
$x$-space, the solution $u$ also has a fixed compact support for all
$0 \leq t \leq T$.  

Now we find a sequence of fucntions $\{u_j\}$ which will converge to a
limits $u$ giving the solution. Let $\psi$ be the solution of Cauchy
problem  
\begin{equation*}
M[\psi] = \beta (x) f(x, t, 0) \text{ with }\psi (0) = u_0
. \tag{4.3}\label{chap5-eq4.3} 
\end{equation*}

Hence by the theory of linear equations, there exists a constant
$\gamma_0$ depending on $T$ such that  
\begin{gather*}
|| \psi (t)||_{\left[\frac{n}{2}\right]+2} \leq \gamma_0 \{ || u_0
||_{\left[\frac{n}{2} \right] + 2}+ \sup\limits_{0 \leq t \leq T} ||
||\beta_f (x, t, 0)||_{\frac{n}{2}}+2 \\ 
|| \psi (t) ||_{\left[\frac{n}{2}\right]+1} \leq \gamma_0 \{ || u_0 ||
||_{\left[\frac{n}{2}\right]+1} + \sup_{0 \leq t \leq T}|| \beta f
(x, t, 0)||_{\left[\frac{n}{2}\right]+1}. \tag{4.4}\label{chap5-eq4.4} 
\end{gather*}

The Cauchy problem for $(4.1)'$ is therefore reduced to the following
problem: to find a solution $u \in
\mathscr{D}^{\left[\frac{n}{2}\right]+2}_{L^2} [0, T]$ of  
$$
M[u]= \tilde{g}(x, t, \psi + u)
$$
with the initial data $u_0$. Here 
$$
\tilde{g}(x, t, \psi + u) = \alpha (x) (f(x, t, u + \psi) -f (x, t,
0)). 
$$

Our main interest here is to determine how does the domain of
existence $\underline{R}^n \times \{0 \leq t \leq h \}$ of the
solution depend on the initial data $u_0$, after fixing\pageoriginale
$\alpha$, $\beta \in \mathscr{D}$. The functions $u_j$ are
defined inductively as solutions of the Cauchy problem for the first
order system of equations:   
\begin{align*}
M[u_1] & = \tilde{g} (x, t, \psi ), \; u_1 (0) = 0, \\
M[u_2] & = \tilde{g} (x, t, u_1 + \psi ), \; u_2 (0) = 0, \\
& \cdots \cdots \cdots \cdots \cdots \\
M[u_j] & = \tilde{g}(x, t, u_{j - 1} + \psi ), \; u_j(0) = 0,\\
& \cdots \cdots \cdots \cdots \cdots 
\end{align*}

Now since $\psi \in
\mathscr{D}^{\left[\frac{n}{2}\right]+2}_{L^2} [0, 
  T]$ we have $\tilde{g}(x, t, \psi(t)) \in
\mathscr{D}^{\left[\frac{n}{2}\right] + 2}_{L^2} [0, T]$ and hence by
the theory of linear equations there exists a solution $u_1$ of the
Cauchy problem  
$$
M[u_1]  = \tilde{g} (x, t, \psi), \;\; u_1 (0) = 0,  
$$
and $u_1 \in \mathscr{D}^{\left[\frac{n}{2}\right]+2}_{L^2}[0,
  T]$. Again we have $\tilde{g}(x, t, (\psi + u_1) (x,t)) \in 
\mathscr{D}^{\left[\frac{n}{2}\right]+2}_{L^2}\break [0, T]$ and hence there
exists a solution $u_2$ of   
$$
M[u_2]= \tilde{g} (x, t, u_1 +\psi), \; u_2 (0) = 0 
$$
and $u_2 \in \mathscr{D}^{\left[\frac{n}{2}\right]+2}_{L^2}
[0, T]$. This proceedure can be used to obtain $u_j$ inductively.   

Now we have the 

\setcounter{proposition}{0}
\begin{proposition}\label{chap5-sec4-prop1} % Proposition 1
There exists a positive, non-increasing functions $\varphi (\xi )$ of
$\xi > 0$ such that  
$$
h = \varphi (|| u_0||_{\left[\frac{n}{2}\right]+1})> 0 
$$
and\pageoriginale the set $\left\{\sup\limits_{0 \leq t \leq h} || u_j (t)
||_{\left[\frac{n}{2}\right]+1} \right\}$ is bounded.  
\end{proposition}

\begin{proof}
Let $\gamma$ denote the $\sup\limits_{(x, t) \in \Omega} | \psi
(x, t) |$. In view of \eqref{chap5-eq4.4} $\gamma$ is less than or
equal to $c_0 + c_1 || u_0 ||_{\left[\frac{n}{2} \right]+1} $ where
$c_0$, $c_1$ are constants 
depending on $T$. If $b$ is a positive number let $F$ be the set  
$$
F= \{(x, t, u) | (x, t) \in \Omega, |u | < b + \gamma \} 
$$
and put 
\begin{equation*}
M = \sup\limits_{F,|\alpha|\leq \left[\frac{n}{2}\right]+2} |
\left(\frac{\partial}{\partial x}, \frac{\partial }{\partial
  u}\right)^\alpha \tilde{g} (x, t, u)| \tag{4.5}\label{chap5-eq4.5} 
\end{equation*}
where $\left(\dfrac{\partial}{\partial x}, \dfrac{\partial }{\partial
  u}\right)^\alpha$ denotes a derivation of order $|\alpha|$ with respect to
$x$ and $u$. $M = M(b+ \gamma)$ is an increasing function of the
parameter. If $u \in \mathscr{D}^{\left[\frac{n}{2}\right] +
  1}_{L^2} [0, T]$ with $| u(x, t) | \leq b$ for $(x, t)\in
\Omega$ then we have   
\begin{equation*}
||\tilde{g}(x, t, (u + \psi) (x,
t))||_{\left[\frac{n}{2}\right]+1}\leq M c \{ 1 
+ || u(t) ||^k_{\left[\frac{n}{2}\right]+1} \},
\tag{4.6} \label{chap5-eq4.6} 
\end{equation*}
$k = \left[\dfrac{n}{2} \right] + 1$. Now, since $u_j (0) = 0$, we
have by the energy inequality 
$$
|| u_j(t) ||_{\left[\frac{n}{2}
    \right]+1}\leq c (T) \int\limits^T_0 || \tilde{g} (x, s, (u_{j-1}
+  \psi) (x, s))||_{\left[\frac{n}{2}\right]+1}ds.
$$
  
Hence from \eqref{chap5-eq4.6} we obtain 
\begin{equation*}
||u_j (t)||_{[\frac{n}{2}] + 1}\leq M c c (T) \int\limits_0^T (1 + ||
u_{j - 1}(s) ||^k _{[\frac{n}{2}]+1}) ds. \tag{4.7}\label{chap5-eq4.7} 
\end{equation*}

We recall that this was derived with the assumption that $|u_{j -
  1}(x, t)| < b$ which,\pageoriginale we shall show, holds when $h$ is
small and $0 \leq t \leq h$. Put   
\begin{align*}
c_2 & = M c \cdot c (T) \\
\gamma_1 & =  1 + 2^k \sup_{ 0 \leq t \leq T} || \psi (t)
||^k_{\left[\frac{n}{2} \right]+1} .\tag{4.8}\label{chap5-eq4.8}  
\end{align*}

Since $|| (\psi + u_{j-1}) (t) ||^k_{\left[\frac{n}{2}\right]+1}\leq
2^k \left\{||u_{j-1}(t)||^k_{[\frac{n}{2}]+1} + || \psi
(t)||^k_{\frac{n}{2}+1} \right\}$ \eqref{chap5-eq4.7} can be written as   
$$
|| u_j (t) ||_{[\frac{n}{2}] + 1}\leq 2^Kc_2\int\limits_0^t
\{\gamma_1 + || u_{j-1}(s) ||^k_{[\frac{n}{2}]+1} ds,  
$$
where $u_0 (t) \equiv 0$. Putting again $2^kc_2= c_3$ we have 
\begin{equation*}
||u_j (t) ||_{\left[\frac{n}{2}\right]+1}\leq c_3 \int\limits^t_0 \{\gamma_1 + ||
u_{j-1}(s) ||^k_{\left[\frac{n}{2}\right]+1} \}
ds. \tag{4.9} \label{chap5-eq4.9} 
\end{equation*}

Let $c_s(n)$ denote the Sobolev's constant, namely the constant in the
inequality  
$$
\sup |\varphi (x) |\leq c_s (n) || \varphi ||_{\left[\frac{n}{2}\right]+1}.  
$$

Define $b'$ by 
\begin{equation*}
b' = \frac{b} {c_s(n)}\tag{4.10}\label{chap5-eq4.10}
\end{equation*}
and denote $c_3 (\gamma_1 + b'^k)$ by $\tilde{M}$. Take  
\begin{equation}
h = \frac{b'}{\tilde{M}} = \frac{b'}{c_3 (\gamma_1 + {b'}^k
  )}.\tag{4.11}\label{chap5-eq4.11}  
\end{equation} 

Consider\pageoriginale the sequence $y_j (t)$ defined by the sequence
of integral equations   
$$
y_j (t) = c_3 \int\limits^t_0 \left\{\gamma_1 + y_{j-1}(s)^k
\right\}ds \text{~ for~ } t \geq  0, \; y_0 (t) \equiv 0. 
$$

Then we assert that 
$$
0 \leq y_j (t) \leq b' \text{ for } 0 \leq t \leq h, \; j = 1, 2, \ldots   
$$
\begin{align*}
\text{In fact,~ } y_1 (t) & \leq c_3 \gamma_1 t \leq \tilde{M} t \leq
\tilde{M} h = b',\\ 
y_2 (t) &\leq \tilde{M} t \leq \tilde{M} h = b' \text{~ and so on}.  
\end{align*}

Evidently $|| u_j (t)||_{\left[\frac{n}{2}\right]+1}\leq y_j (t)$ and 
\begin{equation*}
||u_j (t) ||_{\left[\frac{n}{2}\right]+1} \leq b' \text{ for } 0 \leq t \leq h
\tag{4.12}\label{chap5-eq4.12} 
\end{equation*}
which, a fortiori, implies (by using Sobolev's lemma) that 
$$
\sup |u_j (x, t) | \leq b' c_s (n) = b~~\text{ (see \eqref{chap5-eq4.10})}.  
$$

From \eqref{chap5-eq4.11} we obtain 
\begin{align*}
\frac{1}{h} & = \frac{c_3 (\gamma_1 + b'^k)}{b'} = 2^k c \cdot c (T)
\frac{{b'}^k + \gamma_1}{b'}M \\ 
& \leq c_0 (n, T) \frac{b^k + C'_0 (n) + c''_0(n) || \psi (t)
  ||^k_{\left[\frac{2}{n}\right]+1}}{b} M,  
\end{align*}
where $M = M (\gamma + b )$. $M (\xi)>0$ is an increasing function of
$\xi > 0$. So, if $||u_0||_{\left[\frac{n}{2} \right]+1}$ runs through a bounded
set, fixing $b, h$ has a positive infimum ($M$ is taken to be a fixed
positive number). This completes\pageoriginale the proof. 
\end{proof}

\begin{remark*}
Instead of taking hte initial data to be given at  $t = 0$ we can take
the initial data to be given at an arbitray $t_0 (0 \leq  t_0 \leq
T)$. We define $\psi (t, t_0)$ corresponding to $\psi (t)$ in the
above arguments. Here $|| \psi (t, t_0)||_{\left[\frac{n}{2}\right] + 1}$ is
majorized by $C_0 + C_1 || u_0 ||_{\left[\frac{n}{2}\right]+1}$, $C_0, C_1$
can be taken independently. The expression for $\dfrac{1}{h}$ shows
that $h$ has a positive infimum independent of $t_0$ if the initial
data $u_0$ runs through a bouded set in
$\mathscr{D}^{\left[\frac{n}{2}\right]+1}_{L^2}$.  
\end{remark*}

Next we prove that the sequence $\{u_j(t) \}$ is a Cauchy sequence in
$\mathscr{D}^{\left[\frac{n}{2} \right]+2}_{L^2}[ 0, h]$. First of all we shall
show that $\{\sup\limits_{0 \leq t \leq h}|| u_j
(t)||_{\left[\frac{n}{2}\right]+2} \}$ is boun\-ded. In fact, we have 
\begin{gather*}
||u_j(t)||_{\left[\frac{n}{2}\right]+2} \leq c(T) \int\limits^t_0 || \tilde{g}
(x, s (u_{j-1} + \psi) (x, s ))||_{\left[\frac{n}{2}\right]+2} ds \\ 
 \leq cM' \int\limits^t_0 \{ 1 +( 1+ || (\psi + u_{j-1}) (s)
 ||^k_{[\frac{n}{2}] + 1}) || u_{j-1} + \psi
 (s))||_{[\frac{n}{2}]+2}ds, \\ 
  k = \left[\frac{n}{2}\right]+1.
 \end{gather*} 
 \begin{align*}
& || u_2 (t) - u_1 (t) ||_{[\frac{n}{2}]+2} \leq K c' t, \\ 
& || u_3 (t) - u_2 (t) ||_{[\frac{n}{2}]+2} \leq K \frac{(c'
     t)^2}{2!}, \ldots,  \\ 
& || u_{j + 1}  - u_j (t) ||_{[\frac{n}{2}]+2} \leq K \frac{(c'
     t)^j}{j!}, \ldots  \\ 
\end{align*} 

Hence $\{u_j (t)\}$ is a Cauchy sequence in
$\mathscr{D}^{[\frac{n}{2}]+2}_{L^2}[0, h]$ and therefore
converges\pageoriginale to 
a limit $u(t)$ in $\mathscr{D}^{[\frac{n}{2}]+2}_{L^2}[0, h]$.  

If $m \geq [\dfrac{n}{2}] + 3 $ we now assume that $A_k \in
\mathscr{B}^m [0, T]$, $\dfrac{\partial A_k}{\partial t} \in
\mathscr{B}^0 [0, T]$ and $f \in \mathscr{E}^{m+1} (\Omega
\times \underline{C})$. Let $u_0 \in \mathscr{D}^m_{L^2}$ be
given. Then the limit $u(t)$ in  $\mathscr{D}^{[\frac{n}{2}] +
  2}_{L^2} [0, h]$ of the sequence $\{ u_j (t) \}$ obtianed above
itself belongs to $\mathscr{D}^m_{L^2}[0, h]$. In fact, it is enough
to prove that $\left\{\sup_{0 \leq t \leq h} || u_j (t)||_m\right\}$ is bounded
and $\{u_j (t)\}$ is a Cauchy sequence in $\mathscr{D}^m_{L^2}[0,
  h]$. For this we have only to use the following lemma which results
by arguments similar to those used in \S \ref{chap5-sec2}.  

\setcounter{lemma}{0}
\begin{lemma}\label{chap5-sec4-lem1} %lemma 1
Let $u \in \mathscr{D}^m_{L^2 } [0, T]$ and $f \in
\mathscr{G}^{m+1} (\Omega \times \underline{C})$ for an $m \geq
        \left[\dfrac{n}{2}\right]+2$. Then there exists constants
        $C_m$, $M_m$ such that   
$$
||\left(\frac{\partial}{\partial x}\right)^\nu f (x, t, u(x, t))||_m
C_m M_m\Big\{1 + (1 + || u (t)||^{m-1}_{m-1})  || u (t) ||_m.  
$$
\end{lemma}

Thus we have proved the following: 

\setcounter{theorem}{0}
\begin{theorem}[local existence theorem]\label{chap5-sec4-thm1}
Given any intial data $u_0 \in \mathscr{D}^m_{L^2}$, $m \geq
 \left[\dfrac{n}{2}\right] + 2$ and any initial time $t_0$, $0 \leq
 t_0 \leq T$ there exists a unique solution $u(t) \in
 \mathscr{D}^m_{L^2} [t_0, t_0 + h]$ of the equation  
\begin{equation*}
M[u] = \tilde{f}(x,t,u) = \beta (x) f(x, t, 0) + \alpha (x) \{ f(x,
t, u) - f (x, t, 0) \} \tag*{$(4.1)'$} 
\end{equation*}
with $u(t_0)= u_0$. Moreover $h$ can be chosen to be independent of
$t_0$ in $[0, T]$ when $||u_0||_{\left[\frac{n}{2}\right]+2}$ runs
through a bounded set.   
\end{theorem}

Now\pageoriginale we obtain a global existence theorem for solutions
of the Cau\-chy problem for regularly hyperbolic first order systems of
semi-linear equa\-tions. For this we assume that an a \textit{priori
  estimate} of the following type holds.  

If $\beta \in \mathscr{D}$ consider the regularly hyperbolic
first order system of equations 
\begin{equation*}
M[u] = \beta f (x, t, 0) + (f(x, t, u) - f(x, t,
0)). \tag{4.13}\label{chap5-eq4.13}  
\end{equation*}

By A \textit{priori estimate} we mean the following: For any initial
data $u_0$ in $\mathscr{D}^{\left[\frac{n}{2}\right] +2}_{L^2} \cap
\mathscr{E}'$ 
and any $t_0 (0 \leq t_0 \leq T)$ the solution $u (t) \in
\mathscr{D}^{\left[\frac{n}{2}\right] +2}_{L^2} [t_0, T] $ of
\eqref{chap5-eq4.13} 
satisfies the following condition: there exists a constant $c = c(T)$
such that   
\begin{equation*} 
|| u(t) ||_{\left[\frac{n}{2}\right]+1} \leq  c \text{ for all } t_0
\leq t \leq T. \tag{4.14}\label{chap5-eq4.14} 
\end{equation*}

\begin{theorem}[global existence theorem]\label{chap5-sec4-thm2} %theorem 2
 Suppose an a priori estimate of the type \eqref{chap5-eq4.14} holds
 for solutions  of \eqref{chap5-eq4.13}. Then, given any initial data $u_0 \in
 \mathscr{E}^m_{L^2(\loc)}$, $m \geq \left[\dfrac{n}{2}\right] + 2$
 there exists a  unique solution $u(t)$ of   
\begin{equation*}
M[u] = f \text{ with } f \in \mathscr{E}^{m+1} (\Omega \times
\underline{C}) \tag{4.1} 
\end{equation*}
for $0 \leq t \leq T$ such that $u(0) = u_0$, $u \in
\mathscr{E}^m_{L^2(\loc)} [0, T]$ and $\dfrac{\partial u}{\partial t}
\in \mathscr{E}^{m-1}_{L^2(\loc)} [0, T]$. 
\end{theorem}

\begin{proof}
As we have seen in the section on dependence domain there exists a
retrograde cone $K$ such that the value of a solution $u$ of $M [u] =
f$ at a point $(x_0, t_0) \in \Omega$ depends only on the
second member in the set $(x_0, t_0) + K$ and on the value of the
initial data in the intersection  of this\pageoriginale translated
cone with $(t=0)$. Let $D$ be the subset of $\Omega$ swept by $(x, T) + K$ as
$x$ runs through a ball $|x | < R$ and $D_0$ be the set $D \cap \{ t =
0 \}$. Let $\beta \in \mathscr{D}$ such that $\beta(x) \equiv 1$
for $x \in D_0$. Given any initial data $u_0  \in
\mathscr{E}^m_{L^2(\loc)}$ we consider the Cauchy problem  
\begin{align*}
M[u_1] & = \beta (x) f (x, t, 0) + (f(x, t, u) - f(x, t, 0))\\  
\text{ with } \quad u_1 (x, 0) & = \beta (x) u_0 (x) \in
\mathscr{D}^m_{L^2}. \tag{4.15}\label{chap5-eq4.15}  
\end{align*}

This solution $u_1(x, t)$  has an a priori estimate $||u_1
(t)||_{\left[\frac{n}{2}\right]+1} \leq C$. On the other hand this
solution $u_1$ 
has compact support as far as the solution exists. Hence, if we take
$\alpha \in \mathscr{D}$ such that $\alpha (x) \equiv 1$  for
$|x| \leq R$, \eqref{chap5-eq4.15} is equivalent to  
\begin{equation*}
M[u_1] = \beta (x) f(x, t, 0) + \alpha (x) (f(x, t, u) - f (x, t,
0)). \tag*{$(4.1)'$} 
\end{equation*}

Now since $u_1$ has an a priori estimate $||u_1 (t)
||_{\left[\frac{n}{2} \right]+ 1}\leq C$, it follows, by using
theorem \ref{chap5-sec4-thm1} to continue the solution 
step by step, that there exists a solution $u_{1} (x, t)$ for $0 \leq
t \leq T$. Clearly $u(x, t) = u_1 (x, t)$ for $(x, t)\in D$
and this completes the proof of theorem \ref{chap5-sec4-thm2}.  
\end{proof}

\section[Existence theorems for a single semi-linear.....]{Existence
  theorems for a single semi-linear equation of higher
  order}\label{chap5-sec5}%%% 5  

In this section we obtain theorems on existence of solutions, local
and global, of the Cauchy problem for a single semi-linear equation of
order $m$.  

As before $\Omega$ be the set $\{ (x, t) | x \in
\underline{R}^n$, $0 \leq t \leq T$ and  
\begin{equation*}
 M = \left(\frac{\partial}{\partial t}\right)^m + \sum_{\substack{j +
     |\nu | \leq m \\j < m }} a_{j, \nu} (x,
 t)\left(\frac{\partial}{\partial t}\right)^j 
 \left(\frac{\partial}{\partial x}\right)^\nu \tag{5.1}\label{chap5-eq5.1} 
\end{equation*}\pageoriginale
be a regularly hyperbolic operator in $\Omega$. Consider the quasi-linear
equation 
\begin{equation*}
M[u] = f\left(x, t, \left(\frac{\partial}{\partial
  t}\right)^{j_1}\left(\frac{\partial}{\partial
  x}\right)^{\alpha_1} u, \ldots, 
\left(\frac{\partial }{\partial t}\right)^{j_s} \left(\frac{\partial }{\partial
  t}\right)^{\alpha_s}u\right) \tag{5.2}\label{chap5-eq5.2}
\end{equation*}
where $j_k + |\alpha _k| \leq m-1 (k = 1,  \ldots,  s)$. We make the
following assumptions on the coefficients of $M$ and $f$: 
$$
a_{j,\nu} \in \mathscr{B}^{[\frac{n}{2}]+2}[0, T],
\frac{\partial} {\partial t} a_{j,\nu}  \in \mathscr{B}^0 [ 0, T]
\text{ and } f \in \mathscr{E}^{ [\frac{n}{2}]+3} (\Omega
\times \underline{C}^s).  
$$ 

When we consider the regularity properties of higher degrees. We
assume for $N \geq [\dfrac{n}{2}] + 3$  
$$
a_{j, \nu} \in \mathscr{B}^N  [0, T], \frac{\partial}{\partial
  t} a_{j, \nu} \in \mathscr{B}^0 [0, T] \text{~ and~ } 
f \in \mathscr{E}^{N+1}(\Omega \times \underline{C}^s).  
$$

The reasoning used in the case of the first order system (see
\S\ \ref{chap5-sec4}) 
can be applied to this case without any significant change. We will
indicate the necessary modifications very briefly.  

The space of all functions $u$ such that 
$$
u \in \mathscr{D}^{k + m-1}_{L^2} [0, T], \frac{\partial
  u}{\partial t} \in \mathscr{D}^{k+m-2}_{L^2} [0, T], \ldots, 
\left(\frac{\partial}{\partial t}\right)^{m-1} u \in
\mathscr{D}^{k}_{L^2} [0, T]   
$$
is denoted by $\tilde{\mathscr{D}}^k_{L^2} [0, T]$. We introduce a
topology on $\tilde{\mathscr{D}}^k_{L^2} [0, T]$ by a norm $|| u (t)
|||_k$ defined by  
\begin{equation*}
||| u |||^2_k = || u (t)||^2_{k+m-1} + \cdots +
\left(\frac{\partial}{\partial t}\right)^{m-1} u (t)||^2_k.
\tag{5.3}\label{chap5-eq5.3} 
\end{equation*}

Now\pageoriginale we recall the result in the linear case. Given the
equation 
\begin{equation*}
M[u]= f \tag{5.4}\label{chap5-eq5.4}
\end{equation*}
with $f_\varepsilon \mathscr{D}^{\left[\frac{n}{2}+1\right]}_{L_2} [ 0, T]$
(resp. $f \in \mathscr{D}^{\left[\frac{n}{2}\right]+2}_{L^2} [0, T]$)
and the initial data $ u(0) \in
\tilde{\mathscr{D}_{L^2}}^{\left[\frac{n}{2}\right]+1}$ (resp. $u(0)
\in  
\tilde{\mathscr{D}}^{\left[\frac{n}{2}\right]+2}_{L^2}$) the solution
$u(t)$ of the Cauchy problem belongs to
$\tilde{\mathscr{D}}^{\left[\frac{n}{2}\right]+1}_{L^2} [0,T]$
(resp. to 
$\tilde{\mathscr{D}}^{\left[\frac{n}{2}\right]+2}_{L^2} [0,T]$) and further
we have the energy inequality  
\begin{gather*}
||| u(t)|||_{\left[\frac{n}{2}\right]+1}  \leq c(T) \left\{ || u(0)
||_{\left[\frac{n}{2}\right]+1} + \int\limits^t_0 || f(s)
||_{\left[\frac{n}{2}\right]+1} 
ds \right\}\\ 
\left(\text{resp. } ||| u(t) |||_{\left[\frac{n}{2}\right]+2}  \leq  c(T)
\left\{||u(0) ||_{\left[\frac{n}{2} \right]+2} + \int^t_0 ||f(s)
||_{[\frac{n}{2}+2]}ds\right\}\right)  
\end{gather*}
for $0 \leq t \leq T$.

In the semi-linear case we use the following 

\setcounter{lemma}{0}
\begin{lemma}\label{chap5-sec5-lem1}% lemma 1
If $u(t) \in \tilde{\mathscr{D}}^{\left[\frac{n}{2}
     \right]+1}_{L^2} [0,T]$ then for any $ \alpha \in
 \mathscr{D}$ the function  $\tilde{f}= \alpha f$ satisfies   
 $$
  \tilde{f}\left(x,t, \left(\frac{\partial}{\partial t}\right)^{j_1}
 \left(\frac{\partial}{\partial x}\right)^{\alpha_1} u(x, t), \ldots,
 \left(\frac{\partial}{\partial t}\right)^{j_s}\left(\frac{\partial}{\partial
   x}\right)^{\alpha_s} u(x, t)\right)
 \in{\mathscr{D}}^{\left[\frac{n}{2} \right]}_{L^2} [0,T] 
 $$
 and
 \begin{gather*}
|| \tilde{f}(x, t, \left(\frac{\partial}{\partial
  t}\right)^{j_1}\left(\frac{\partial}{\partial x}\right)^{\alpha_1}
u(x, t), \ldots, 
\left(\frac{\partial}{\partial t}\right)^{j_s}\left(\frac{\partial}{\partial
  x}\right)^{\alpha_s} u(x, t)) ||_{\frac{n}{2}+1} \\ 
\leq  C ~M \left\{ 1+ |||
u(t)|||^{\left[\frac{n}{2}\right]+1}_{\left[\frac{n}{2}\right]+1}\right\}.
 \tag{5.5} \label{chap5-eq5.5}   
 \end{gather*} 
\end{lemma}

\begin{proof}
We write\pageoriginale $v_k(t)$ for $\left(\dfrac{\partial}{\partial
  t}\right)^{j_k} 
\left(\dfrac{\partial}{\partial x}\right)^{\alpha_k} u(x, t)$ and
$\tilde{f} (x, t, v_1 (t), \ldots, v_s(t))$ for $\tilde{f} \left(x, t,
\left(\dfrac{\partial}{\partial t}\right)^{j_1} \left(\dfrac{\partial}{\partial
  x}\right)^{\alpha_1} u(x, t), \ldots \right)$. Now we see that
$||v_k(t)||_{\left[\frac{n}{2}\right] + 1} \leq c ||| u (t)
|||_{\left[\frac{n}{2}\right]+1} (k = 1, \ldots, s)$. In fact,  
$$
||v_k (t)||_{\left[\frac{n}{2}\right]+1}= || \left(\frac{\partial}{\partial
  t}\right)^{j_k}\left(\frac{\partial}{\partial x}\right)^{\alpha_k}
u(t)||_{\left[\frac{n}{2}\right]+1} \leq  c || \left(\frac{\partial}{\partial
  t}\right)^{j_k} u(t) ||_{\left[\frac{n}{2}\right]+|\alpha_k | +1}. 
$$

Since $ j_k + | \alpha_k | \leq m-1$ we have $\left[\dfrac{n}{2}\right] + |
\alpha_k | + 1 \leq \left[\dfrac{n}{2}\right]+1 + (m-1-j_k)$ and hence  
$$
|| v_k (t) ||_{[\frac{n}{2}]+1} \leq c ||\left(\frac{\partial}{\partial
  t}\right)^{j_{k_u}}||_{[\frac{n}{2}]+1 + (m-1-j_k)} \leq c ||| u
|||_{[\frac{n}{2}]+1}. 
$$

The assertion follows from this by an application of
Cor. \ref{chap5-sec2-coro2} after lemma \ref{chap5-sec2-lem1} of
\S\ \ref{chap5-sec2}.  

The following lemma is proved on the same lines and we omit the
proof. 
\end{proof}

\begin{lemma}\label{chap5-sec5-lem2}% lemma 2
If $u \in \tilde{\mathscr{D}}^{\left[\frac{n}{2}\right]+1+N}_{L^2}
[0,T]$ for an integer $N \geq 1$ then for any $ \alpha \in
\mathscr{D}$ 
$$
 \tilde{f} \left(x, t, \left(\frac{\partial}{\partial t}\right)^{j_1}
\left(\frac{\partial}{\partial x}\right)^{\alpha_1} u(x,t), \ldots,
\left(\frac{\partial}{\partial t}\right)^{j_s}\left(\frac{\partial}{\partial
  x}\right)^{\alpha_s} u(x,t)\right) \in
\tilde{\mathscr{D}}^{\left[\frac{n}{2}\right]+1+N}_{L^2} [0,T] 
$$
and
\begin{gather*}
|| \tilde{f}\left(x, t, \left(\frac{\partial}{\partial t}\right)^{j_1}
\left(\frac{\partial}{\partial x}\right)^{\alpha_1} u(x, t), \ldots
\left(\frac{\partial}{\partial t}\right)^{j_s} \left(\frac{\partial}{\partial
  x}\right)^{\alpha_s} u(x, t)\right)||_{\left[\frac{n}{2} \right]+1+N} \\ 
\leq c M_n \left\{ 1+ \left(1+|||
u(t)|||^{\left[\frac{n}{2} \right]+N}_{\left[\frac{n}{2}\right]+N}\right) ||| u(t)
|||_{\left[\frac{n}{2}\right]+N+1} \right. \tag{5.6}\label{chap5-eq5.6} 
\end{gather*}
\end{lemma}

As in\pageoriginale the local existence theorem for the first order
systems we define  
\begin{multline*}
\tilde{f}(x, t, v_1, \ldots, v_s)= \beta(x)f(x, t, 0, \ldots, 0)\\
+\alpha (x)\{f(x, t,v_1, \ldots,v_s)- f(x, t, 0, \ldots, 0)\} 
\end{multline*}
where $\alpha$, $\beta \in \mathscr{D}$. Then the same
arguments as in the first order systems prove the following 

\setcounter{theorem}{0}
\begin{theorem}[local existence theorem]\label{chap5-sec5-thm1}
 For fixed $\alpha$, $\beta \in \mathscr{D}$ and $T$ let  
\begin{equation*}
M [u] = f \left(x, t, \left(\frac{\partial}{\partial
  t}\right)^{j_1}\left(\frac{\partial}{\partial t}\right)^{\alpha_1} u
(x, t), \ldots \right) 
\tag{5.7} \label{chap5-eq5.7}
\end{equation*}
be a semi-linear regularly hyperbolic equation of order $m$. Given any
initial 
data $u^{(0)} \in \mathscr{D}^N_{L^2}$, $N \geq
\left[\dfrac{n}{2}\right]+2$ (more precisely, given $$
(u_0, u_1,\ldots, u_{m-1})
$$ 
with $u_j \in \mathscr{D}^{N+m-j}_{L^2}$) and the initial time
$t_0 (0 \leq t_0 \leq T)$ there exists a unique solution $u(x, t) =
u(t)$ for $t_0 \leq t \leq t_0 + h$ of \eqref{chap5-eq5.7} such that $u
\in \tilde{\mathscr{D}}^N_{L^2} [t_0, t_0 + h]$,
$\dfrac{\partial u}{\partial t} \in
\tilde{\mathscr{D}}^{N-1}_{L^2} [ t_0, t_0 +h]$ taking the initial
value $u^{(0)}$ at $t= t_0$. $h$ can be taken to be a fixed number
independent of $t_0$ when $\left\{ ||| u^{(0)}
|||_{\left[\frac{n}{2}\right]+1} \right\}$ is 
a bounded set. More precisely, there exists a non-increasing function
$\alpha(\xi) > 0$ of $\xi > 0$ such that 
$$
h = \varphi \left(|| |u^{(0)}||_{\left[\frac{n}{2}\right]+1}\right).
$$
\end{theorem}

Now we state a global existence theorem for a single semi-linear
regularly\pageoriginale hyperbolic equation of order $m$. We assume an
a \textit{priori estimate} of the following type holds:  

For any initial data $u^{(0)} \in
\mathscr{D}^{\left[\frac{n}{2} \right] +
  2}_N  \cap \mathscr{E}', \beta \in \mathscr{D}$ the solution  
$u(t)$ of  
\begin{equation*}  
 M[ u ] = \beta f(x, t, 0, \ldots, 0) + \alpha (f,x, t, v_1,
\ldots, v_s) - f(x, t, 0, \ldots, 0)) \tag{5.8}\label{chap5-eq5.8} 
\end{equation*}
(where $v_k = \left(\dfrac{\partial}{\partial t}\right)^{j_k}
\left(\dfrac{\partial}{\partial t}\right)^{\alpha_k} u$) satisfies  
\begin{equation*}
|| u (t) ||_{[\frac{n}{2}]+m} + || \frac{\partial}{\partial t} u(t)
||_{[\frac{n}{2}]+ m-1}+ \cdots + || \left(\frac{\partial}{\partial
  t}\right)^{m-1} u(t) ||_{[\frac{n}{2}]+1} \leq
||c. \tag{5.9}\label{chap5-eq5.9}  
 \end{equation*} 

 \begin{theorem}[global existence theorem]\label{chap5-sec5-thm2}% theorem 2
 under the assumption that there exists an a priori estimate of the
 above type, given any initial data $(u_0, u_1, \ldots, u_{m-1})$ with
 $u_k \in \mathscr{E}^{N + m- k-1}_{L^2  (loc)}$, $N \geq \left[
   \dfrac{n}{2}\right] + 2$, there exists a unique solution $u(t) = u (x,
 t)$ for $0 \leq t \leq T$ of \eqref{chap5-eq5.2} such that  
$$
u \in \mathscr{E}^{N + m- 1}_{L^2  (\loc)} [0, T],
\frac{\partial u}{\partial t}  u \in \mathscr{E}^{N + m-
  2}_{L^2  (\loc)} [0, T], \ldots, \left(\frac{\partial u}{\partial
}\right)^{m} u \in \mathscr{E}^{N- 1}_{L^2  (\loc)} [0, T].| 
$$
\end{theorem} 

\setcounter{remark}{0}
\begin{remark}\label{chap5-sec5-rem1}% Remark 1
As a particular case of the Theorem we have Theorem
\ref{chap5-sec3-thm1} of \S\ \ref{chap5-sec3}.  
\end{remark}

\begin{remark}\label{chap5-sec5-rem2}% Remark 2
We assumed an a priori estimate \eqref{chap5-eq5.9} for the theorem of
exstence of 
global solutions. If in $f (x, t, v_1, \ldots, v_s) \left(v_k =
\left(\dfrac{\partial }{\partial t}\right)^{jk}\left(\dfrac{\partial }{\partial
  x}\right)^{\alpha k} u\right)$ the orders $j_k + |\alpha_k|$ are less than
$(m-1)$ the following remark will be useful. If we have an estimate of
derivatives of $u$ of the form  
$$
||\left(\frac{\partial }{\partial t}\right)^{j_k}\left(\frac{\partial }{\partial
  x}\right)^{\alpha_k} u(t)||_{[\frac{n}{2}] + 1} \leq c\ (k = 1, \ldots, s) 
$$
then\pageoriginale we have an a priori estimate of the form
\eqref{chap5-eq5.9}. In fact, first of 
all we have, if $g (x, t, v_1, \ldots, v_s)$ denotes $f(x, t, v_1,
\ldots, v_s) - f (x, t, 0, \ldots, 0)$ then for any $\alpha \in
\mathscr{D}$ the function $\tilde{g} = \alpha  g $ satisfies the
inequality  
$$
|| \tilde{g} \left( x, t, \left(\frac{\partial }{\partial
  t}\right)^{j_1} u (x, t), \ldots, \left(\frac{\partial }{\partial t}
^{j_s}\right)\left(\frac{\partial}{\partial 
  x}\right)^{\alpha _s} u (x, t)\right)|||_{[\frac{n}{2}]+1}\leq c' 
$$
with a constant $c'$. Now as in the case of first order systems this
inequality, together with the energy inequality in the linear case,
implies \eqref{chap5-eq5.9}.  
\end{remark}

We illustrate this by the following simple example. Take for $M$ the
operator $\square = \dfrac{\partial}{\partial}^2  \Delta$ and consider
the semi-linear equation  
$$
\square u + f (u) = 0. 
$$

We assume $f(0) = 0$. We show that it is enough to obtain an estimate
of $|| u (t)||_{[\frac{n}{2}] + 1}$. in order to get an a priori
estimate of $||u(t)||_{\left[\frac{n}{2}\right]+2}+||\frac{\partial
  u}{\partial t}(t)||_{\left[\frac{n}{2}\right]+1}$. First we obeserve that the
conditon $f(0) = 0$ can be removed. In fact, if $C_0 = f (0)$ we
consider the equation  
$$
\square u + ( f(u) - f(0))'+ \beta (x) f (0) f (0)= 0; 
$$
that is, 
$$
\square u + C_0 \beta (x) + (f(u) - C_0 ) = 0, 
$$
where $\beta \in \mathscr{D}$.  

It is enough to obtain an a priori estimate for solutions of this
equation. If $u_0$, $u_1$, $\in \mathscr{E}'$ then we know
that for $0 \leq t \leq T$ the solution\pageoriginale $u(t)$ has its
support contained in some compact set: say in $|x | < R$.  

Define
$$
E_1 (t) = \int\limits_{|x|<R} \left[ \frac{1}{2}\left\{ \left(\frac{\partial
    u}{\partial t}\right)^2 + \sum_j \left(\frac{\partial u}{\partial
    x_j}\right)^2 \right\} + F(u) - c_0 u + \gamma (u^2 + 1) \right] dx 
$$ 
where $F(u) = \int\limits_0^u f(\tau) d \tau$ and $\gamma$ is chosen so
large that $F(u) - c_0 u + \gamma (u^2 + 1) \geq 0$ for any $u$. This
is always possible if we assume $F(u) > - L$. Differentiating $E_1(t)$
with respect to $t$ and using integration by parts we have  
\begin{align*}
\frac{d}{dt} E_1 (t)  &= \int\limits_{|x| < R} \left\{ \frac{\partial
  u}{\partial t}. ~ \square  u + (f (u) - c_0 ) \frac{\partial
  u}{\partial t}  + 2 \gamma u. \frac{\partial u }{\partial t}
\right\} dx\\ 
&= \int 2\gamma u. \frac{\partial u}{\partial t} - c_0 \beta (x)
\frac{\partial u}{\partial t} dx \text{ since } \square u + f (u) -
c_0 = -\beta (x) c_0.\\ 
& \leq C E_1 (t). 
\end{align*}

Hence $E_1 (t) \leq e^{ct}\leq e^{cT} = c'$. This, together with the
expression for $E_1 (t)$, shows that we have the assertion. 

By considering the equation obtained by differentiating the equation
$\square u + c_0 \beta (x) + (f (u) - c_0) = 0$ with respect to $x_j$ 
$$
\square \frac{\partial u}{\partial x_j} + f' (u) \frac{\partial
  u}{\partial x_j} + c_0 \frac{\partial \beta}{\partial x_j} = 0\ (j =
1, 2, \ldots, n), 
$$
we can obtain an estimate for $E_2 (t)$ in an analogous way. Thus we
have the following result: 

Suppose\pageoriginale the function $f$ satisfies the conditions  
\begin{enumerate}[(1)]
\item $F(u) > - L$,

\item $| f' (u) | < \alpha (u^2 + 1)$ for $n = 3$
$$
\leq \text{  a polynomial for } n = 2.
$$ 
\end{enumerate}

For any initial data $(u_0, u_1)$ with $u_0 \in
\mathscr{E}^m_{L^2 (\loc)}$, $u_1 \in \mathscr{E}^{m -1}_{L^2
  (\loc)}$, $m \geq [\dfrac{n}{2}] + 3$, there exists a unique solution
$u(t) = u (x, t)$ for $0 \leq t < \infty$ such that 
$$
u \in \mathscr{E}^m_{L^2 (\loc)} [ 0, \infty), \frac{\partial
    u}{\partial t} \in \mathscr{E}^{m-1}_{L^2 (\loc)} [ 0,
    \infty ), \left(\frac{\partial }{\partial t}\right)^2 u \in
    \mathscr{E}^{m-2}_{L^2 (\loc)} [ 0, \infty ). 
$$

\begin{thebibliography}{99}
\bibitem{key1}{A.P. Calderon {[1]}}:\pageoriginale Uniqueness in the
  Cauchy problem for 
  partial differential equations, Amer. J. of Math. Vol. 80, 1958,
  p. 16-35. 

\bibitem{key2}{A. P. Calderon and A. Zygmund} :
\begin{enumerate}
 \item On singular  integrals, Amer. J. of Math. Vol. 78, 1956,
   p. 289-309. 

 \item Singular integral operators and differential equations,
   Amer. J. of Math. Vol. 79, 1957, p. 901-921. 
 \end{enumerate}

 \bibitem{key3}{K. O. Friedrichs {[1]}} : Symmetric hyperbolic linear
   differential equations, Comm. Pure Appl. Math. Vol.7, 1954,
   pp. 345-392.  

 \bibitem{key4}{L. Garding {[1]}} : Linear hyperbolic partial differential
   equations with  constant coefficients, Acta Math. Vol. 85, 1951,
   p. 1- 62. 

{[2]}: Hyperbolic equations Lecture Notes. University of
   Chicago, 1957. 

 \bibitem{key5}{J. Hadamard}{[1]} : Lectures on Cauchy's problem, Dover. 

 \bibitem{key6}{L. Hormander}{[1]}: Linear partial differential operators,
   Springer Verlag, Berlin, 1963. 

 \bibitem{key7}{F. John {[1]}} : On linear partial differential equations
   with analytic coefficients-unique continuation of deta- comm Pure
   and Appl. Math. Vol. 2(1949) pp. 209-253. 

 \bibitem{key8}{K. J\"orgens {[1]}} :\pageoriginale Das
   Anfangswertproblem in grossen fur 
   eine klasse nichtlineare Wellengleichungen, Math. Zeit. 77(1961),
   pp. 295-308. 

 \bibitem{key9}{P. Lax [1]} : Assymptotic solution of oscillatory initial
   value problems, Duke Math. Jour. Vol. 24 1957, pp. 627-646. 

 \bibitem{key10}{J. Leray {[1]}} : Hyperbolic differential
   equations. Lecture Notes, Institute for Advanced Study, Princeton,
   1952. 

 \bibitem{key11}{ W. Littmann {[1]}} : The wave operator and $L_p$  norms -
   jour. Math. Mech. Vol 12 (1963) pp. 55-68. 

 \bibitem{key12}{ S. Mizohata {[1]}} : Systemes
   hyperboliques. J. Math. Soc. Japan, 11,(1959), pp. 205-233. 

{[2]} : Le probleme de Cauchy pour la systemes
   hyper-boliques et paraboliques, Memoirs of the  College of Science,
   University of Kyoto, Vol. 32, (1959), pp. 181-212. 

{[3]} : Some remarks on the Cauchy problem, Jour. of
   Math. of Kyoto Univ. Vol. 1(1961), pp. 110-112). 

{[4]} : Unicte du prolongement des solutions pour quelques
   operateurs differentiels paraboliques. Memoirs of the College of
   Science, Univ. of Kyoto, 1958, pp. 219-239. 

 \bibitem{key13}{I. Petrowsky {[1]}} :\pageoriginale  Lectures on
   partial differential 
   equations, Interscience Publ. 1954. 

{[2]} : Uber des Cauchysche Problem$\ldots \ldots $
   Bull. I'Univ de Moscow, 1938, p. 1-74. 

 \bibitem{key14}{L. Schwartz {[1]}} : Theorie des distributions, Vols. 1
   and 2. Hermann et cie Paris, 1950-51. 

 \bibitem{key15}{S. Sobolev {[1]}} : Sur un Theoreme d'analye
   fonctionnelle, Mat. Sbornik, 4 ( 46), 1938, p 471-497. 

{[2]} : Sur les equations aux derives partielles
   hyperboliques non-lineaires Cremonese, Roma 1961. 

 \bibitem{key16}{H. F. Weinberger {[1]}} : Remarks on the preceding paper
   of Lax, Comm Pure and Applied Math. Vol. 11 (1958) p. 195-196. 
\end{thebibliography}

