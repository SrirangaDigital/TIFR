\chapter{Lecture 23}

\section*{Cousin's Problem}\pageoriginale

Cousin's problem for meromorphic functions in the complex plane is
Mittag-Leffler's problem. Mittag-Leffler's problem in the plane is as
follows: Given a discrete set of points in the plane and polar
developments at each of these points, construct a function meromorphic
in the whole plane having the given points as poles and the given
developments as the polar developments. We know that this problem
always admits of a solution. Let $a_{i}$ be the given points and
$$
P_{i}(1/z-a_{i})=\frac{C_{i,1}}{(z-a_{i})}+\frac{C_{i,2}}{(z-a_{i})} +
\cdots+\frac{C_i, p_i}{(z-a_{i})^{p_{i}}} 
$$
be the polar development at $\ub{a_{i}}$. Then we can find polynomial
$Q_{i}(z)$ such that the series
$$
\sum(P_{i}\left(\frac{1}{z-a_{i}}\right)-Q_{i}(z))
$$
coverges absolutely and uniformly on every compact set not containing
anyone of the $\ub{a_{i}}$. The limit function gives a solution of the
problem. The solution is unique upto an additive entire function. The
indeterminacy in the problem is thus an entire function.

Let us consider the corresponding problem on the Riemann sphere. We
have a finite number of points $\ub{a_{i}}$ and polar developments
$P_{i}\left(\dfrac{1}{z-a_{i}}\right)$ at $\ub{a_{i}}\neq \infty$ and a polar
development $P(z)$ (a polynomial without constant term) at
$\infty$. Then the function
$$
P_{i}(1/z-a_{i})+P(z)
$$\pageoriginale
is meromorphic on the Riemann sphere and has the prescribed
development at the given points. The solution is unique upto an
additive constant. The method of construction of the solution
corresponds to the classical decomposition of a rational function into
partial fractions.

We want to consider similar problems on complex analytic manifolds. We
shall first define a meromorphic function.

Suppose we consider in $C^{n}$ a `function'
$\varphi=\dfrac{f(z_{1},\ldots,z_{n})}{g(z_{1},\ldots,z_{n})}$ which
is the quotient of two holomorphic functions $f$ and $g$. The variety
of poles is the set of zeros of $g(z_{1},\ldots,z_{n})$. The variety
of poles is not, in general, a true manifold; if all the partial
derivatives of $\ub{g}$ vanish at a point this point is a singular
point for this variety. (For example the variety defined by
$z^{2}_{1}+z^{2}_{2}+z^{2}_{3}=0$ has a singularity at the origin). We
shall call the set of zeros of an analytic function an analytic
subset. At points where the analytic subsets defined by the zeros of
the functions $f$ and $g$ intersect the quotient $f/g$ is
indeterminate. (In the case $n=1$, if $f$ and $g$ are coprime at a
point $\ub{a}$ it is impossible to have $f(a)=0$ and $g(a)=0$
simultaneously. However for $n\geq 2$ this is possible; for example
the functions $z_{1}$ and $z_{2}$ are coprime at the origin and vanish
simultaneously at the origin). So the function $f/g$ is\pageoriginale
defined only in the complement of certain analytic subsets. It can be
shown that an analytic subset is a set of measure zero. It is not good
to define a meromorphic function on $V^{(n)}$ to be the quotient of
two holomorphic functions on $V^{(n)}$; as then the only meromorphic
functions on a compact complex analytic manifold would be constants!
These considerations lead to the following definition of a meromorphic
function on a complex analytic manifold.

A meromorphic function, $\varphi$, on a complex analytic manifold is a
complex valued function defined almost everywhere on the manifold such
that for every point $\ub{a}$ of the manifold there exists a
neighbourhood $U_{a}$ of $\ub{a}$ such that, in $U_{a}$, $\varphi$ is
almost everywhere equal to the quotient of two holomorphic functions
defined everywhere on $U_{a}$.

A meromorphic form of degree $p$ is a field of covectors defined
almost everywhere on the manifold such that every point has a
neighbourhood in which the form is almost everywhere equal to the
quotient of a holomorphic form of degree $p$ by a holomorphic
function.

Suppose we are given an open covering $(U_{i})$ of the manifold and in
every $U_{i}$ a meromorphic differential form
$\overset{p}{\omega_{i}}$ of degree $p$ such that the form
$\omega_{ij}=\omega_{i}-\omega_{j}$ is almost everywhere in $U_{i}\cap
U_{j}$ equal to a holomorphic form defined everywhere in
$U_{ij}=U_{i}\cap U_{j}$. These constitute a Cousin's datum. Cousin'
s problem is to find a meromorphic form $\overset{p}{\omega}$ on the
whole manifold such that $\omega-\omega_{i}$ is a holomorphic form in
$U_{i}$ (\iec $\omega-\omega_{i}$) is almost everywhere equal to a
holomorphic form defined everywhere in $U_{i}$). The indeterminacy of
the problem\pageoriginale is evidently an additive holomorphic form
(defined over the whole manifold). $\omega_{i}$ are ``singular parts''
of $\omega$ in $U_{i}$. Cousin's datum for the Mittag-Leffler's
problem is the following: With each $a_{i}$ we associate an open set
$U_{i}$ such that $U_{i}\cap U_{j}$ is empty for $i\neq j$. In $U_{i}$
we take for $\omega_{i}$ the meromorphic function given by the polar
development. In the complementary set $U_{0}$ of the points $a_{i}$ we
take the function $\omega_{0}\equiv 0$. All these together constitute
the Cousin datum for the Mittag-Leffler problem.

We now wish to formulate the Cousin datum and problem in terms of
currents. If $n=1$, a meromorphic function (or a form) can be
considered as a current. In $C^{1}$ the function $\dfrac{1}{(z-a)}$
defines a current in the usual manner. But the function
$\dfrac{1}{(z-a)k}$, $k>1$, is not summable in any neighbourhood of
$\ub{a}$. However if we take the Cauchy principal value,
$\dfrac{1}{(z-a)k}$ defines a current: for every
$\varphi\in\mathscr{D}$ 
$$
\langle \frac{1}{(z-a)^k},\varphi\rangle =\lim\limits_{\epsilon\to
  0}\iint\limits_{|z-a|\geq \epsilon}\frac{\varphi dx\ dy}{(z-a)k} 
$$
This enables us to consider meromorphic functions or forms as currents
when $n=1$; considering meromorphic functions and forms as currents in
this way leads to good solutions of problems on a compact Riemann
surface. However, it becomes difficult to associate canonically a
current with an arbitrary meromorphic function in higher
dimensions. If the analytic subset defined by the singularities is a
true manifold it is possible to associate canonically a current with
the meromorphic\pageoriginale function. In the case of a general
meromorphic function it has not yet been possible to associate
canonically a current with the meromorphic function.

We shall introduce currents in the problem in some other way. We can
find currents $\overset{p,0}{T_{i}}$ (of bidegree $(p,0)$) in $U_{i}$
such that $T_{i}-T_{j}=\omega_{ij}$ (as a current) in $U_{ij}$. The
indeterminacy in the choice of the $T_{i}$ is a current defined on the
whole manifold. If $S$ is a current on the whole manifold
$T'_{i}=T_{i}+S$ also possess the same property:
$T'_{i}-T'_{j}=\omega_{ij}$ in $U_{ij}$, conversely if $T_{i}$ and
$T'_{i}$ are two systems of currents such that $T_{i}$ and $T'_{i}$
are defined in $U_{i}$ and 
\begin{align*}
& T_{i}-T_{j}=\omega_{ij}\quad\text{in}\quad U_{ij},\\
& T'_{i}-T'_{j}=\omega_{ij}\quad\text{in}\quad U_{ij},
\end{align*}
there exists a current $S$ on the whole manifold such that
$T'_{i}=T_{i}+S$. We define the current $S$ by `piecing' together the
currents $S_{i}$ defined in $U_{i}$ by $S_{i}=T'_{i}-T_{i}$. The
currents $S_{i}$ define a single current on the whole manifold as we
have, in $U_{ij}$,
\begin{align*}
S_{i}-S_{j} &= (T'_{i}-T_{i})-(T'_{j}-T_{j})\\
           &= (T'_{i}-T'_{j})-(T_{i}-T_{j})\\
           &= \omega_{ij}-\omega_{ij}\\
           &= 0
\end{align*}

To\pageoriginale find the currents $T_{i}$ we proceed as follows. We
choose a partition of unity $\{\alpha_i\}$ subordinate to the covering
system $\{U_{i}\}$. We put
$$
T_{i}=\sum \alpha_{k}\omega_{ik}
$$
(the summation being over all $k$ for which $U_{i}\cap U_{k}$ is
non-empty) where $\alpha_{k}\omega_{ik}$ is the $C^{\infty}$ form
defined in $U_{i}$ as:
$$
\alpha_{k}\omega_{ik}=
\begin{cases}
\alpha_{k}\omega_{ik},\text{ \  in \ } U_{i}\cap U_{k}\\
0 \text{ \  in \ } U_{i}\cap \text{ (complement of support of $\alpha_{k}$)}
\end{cases}
$$
The definition of $\alpha_{k}\omega_{ik}$ is consistent, as 
$$
\alpha_{k}\omega_{ik}=0 \text{ \  on \ } U_{i}\cap U_{k}\cap \text{
  (complement of support of $\alpha_{k}$)}.
$$
$T_{i}$ is a $C^{\infty}$ form in $U_{i}$. Now
$$
T_{i}-T_{j}=\sum \alpha_{k}(\omega_{ik}-\omega_{jk})\text{ \  in \ }
U_{ij}.
$$
$\alpha_{k}(\omega_{ik}-\omega_{jk})$ is zero outside
$U_{ijk}=U_{i}\cap U_{j}\cap U_{k}$.

In $U_{ijk}$ we have the relation
$$
\omega_{ij}+\omega_{jk}+\omega_{ki}=0
$$
and, in $U_{ik}$, the relation
$$
\omega_{ik}+\omega_{ki}=0
$$
so\pageoriginale that
\begin{align*}
T_{i}-T_{j} &= \sum_{k}\alpha_{k}(\omega_{ik}-\omega_{jk})\\
 &= \sum_{k}\alpha_{k}\omega_{ij}\\
 &= \left(\sum_{k}\alpha_{k}\right)\omega_{ij}\\
 &= \omega_{ij}, \text{ \  in \ } U_{ij} \text{ \  as \ }
\sum_{k}\alpha_{k}=1. 
\end{align*}

This result is a particular case of a more general one. Suppose we
have an open covering $\{U_{i}\}$ and a system of currents
$\widetilde{\omega}_{ij}$ defined in $U_{ij}=U_{i}\cap U_{j}$ such
that
\begin{align*}
 \widetilde{\omega}_{ij}
 +\widetilde{\omega}_{jk}+\widetilde{\omega}_{ki}=0 \text{ \  in \ }
 U_{i}\cap U_{j}\cap U_{k}\\
\text{and}\quad &\widetilde{\omega}_{ij}+\widetilde{\omega}_{ji}=0
\text{ \  in \ } U_{ij}.
\end{align*}
Then it is possible to find a system of currents $T_{i}$ in $U_{i}$
such that
$$
T_{i}-T_{j}=\widetilde{\omega}_{ij}\quad\text{in}\quad U_{ij}
$$
and $T_{i}$ is given explicitly by
$$
T_{i}=\sum \alpha_{k}\widetilde{\omega}_{ik}
$$
where $\alpha_{k}$ is a partition of unity subordinate to the covering
$U_{i}$. If $\widetilde{\omega}_{ij}$ are $C^{\infty}$ forms, $T_{i}$
also can be chosen to be $C^{\infty}$ forms. 

[One\pageoriginale can also consider the following problem which is
  more general: given an open covering $\{U_{i}\}$ and a system of
  holomorphic forms $\omega_{ij}$ in $U_{ij}$ such that
$$
\omega_{ij}+\omega_{jk}+\omega_{ki}=0\quad\text{in}\quad U_{ijk}
$$
and
$$
\omega_{ij}+\omega_{ji}=0\quad\text{in}\quad U_{ij},
$$
to find holomorphic forms $h_{i}$ in $U_{i}$ such that
$$
h_{i}-h_{j}=\omega_{ij}\quad\text{in}\quad U_{ij}]
$$

By means of the currents $T_{i}$ we define a current of bidegree
$(p,1)$ defined on the whole manifold. We put $R_{i}=d_{\ob{z}}T_{i}$
in $U_{i}$. In $U_{ij}$,
$$
R_{i}-R_{j}=d_{\ob{z}}(T_{i}-T_{j})=d_{\ob{z}}\omega_{ij}=0
$$
(as $\omega_{ij}$ is holomorphic). So the currents $R_{i}$ define a
single current defined on the whole manifold, which we denote by
$\overset{p,1}{R}$. Since the currents $R_{i}$ are $\ob{z}$ closed,
the current $\overset{p,1}{R}$ is also closed. ($\overset{p,1}{R}$ is
locally a coboundary but need not be a coboundary in the large). If we
replace $\{T_{i}\}$ by a system $\{T'_{i}\}$ having the same
properties as $T_{i}$, $T'_{i}=T_{i}+S$, $S$ a current defined on the
whole manifold, the `$R$' corresponding to $T'_{i}$ would be
$\overset{p,1}{R}+d_{\ob{z}}S$. So we can associate canonically with
the Cousin datum a whole $\ob{z}$ cohomology class of bidegree
$(p,1)$. This $\ob{z}$ cohomology class is the `obstruction' to the
solution of Cousin's problem.

We\pageoriginale shall prove that in order that Cousin's problem be
solvable it is necessary and sufficient that the $\ob{z}$ cohomology
class associated with the Cousin datum is the zero class. Suppose
Cousin's problem is solvable and let $\omega$ be a solution. Let
further $\omega-\omega_{i}=h_{i}$, $h_{i}$ holomorphic in
$U_{i}$. Take $T_{i}=-h_{i}$. In $U_{ij}$,
\begin{align*}
T_{i}-T_{j} &= -h_{i}+h_{j}\\
           &= -(\omega-\omega_{i})+(\omega-\omega_{j})\\
           &= \omega_{i}-\omega_{j}\\
           &= \omega_{ij}.\\
d_{z}-T_{i} &= -d_{\ob{z}}h_{i}=0 
\end{align*}
So the current of bidegree $(p,1)$ associated with the system $T_{i}$
is zero, or the $\ob{z}$ cohomology class associated with the Cousin
datum is the zero class. Conversely if the associated $\ob{z}$
cohomology class is the zero class, Cousin's problem is solvable. In
this case we can find $T_{i}$ such that the associated
$\overset{p,1}{R}$ is zero (we may have to adjust the $S$
suitably). That is, we find $T_{i}$ such that
$$
T_{i}-T_{j}=\omega_{ij}\text{ \  in \ } U_{ij}\quad\text{ \  and \ }
d_{\ob{z}}T_{i}=0.
$$
Then by the ellipticity of $d_{\ob{z}}$ (on
$\overset{0}{\mathscr{D}'}$), $T_{i}$ is a holomorphic form
$h_{i}$. Then a solution is given by the form
$\omega=\omega_{i}-h_{i}$. In $U_{ij}$,
$\omega_{i}-h_{i}=\omega_{j}-h_{j}$ so that $\omega$ is a meromorphic
form well-defined on the manifold.

Let\pageoriginale $\overset{p,1}{R}$ be the current defined by:
$\overset{p,1}{R}=d_{\ob{z}}T_{i}$ in $U_{i}$. If $R=d_{\ob{z}}S$, a
solution of the Cousin's problem is given by the form $\omega$:
$$
\omega=\omega_{i}+S-T_{i}\quad\text{in}\quad U_{i}.
$$

In a compact K\"ahlerian manifold $d$ and $\ob{z}$ cohomologies
coincide and with a Cousin datum we have an ordinary cohomology class.

As an example let us consider the Riemann sphere. Here $b'=0$; so
$b^{0,1}=0$. So Cousin's problem for meromorphic functions is solvable
for any Cousin datum.


