\chapter{Lecture 21}

\section*{Compact Manifolds with a K\"ahlerian
  structure}\pageoriginale

Let $V$ be a compact complex analytic manifold. We shall assume that
there exists a K\"ahlerian metric on $V$. From this assumption we
shall derive some intrinsic properties of $V$ \iec properties of $V$
which depend only on the complex analytic structure of $V$ but not on
any particular K\"ahlerian metric.

Let $H^{(r,s)}$ denote the quotient space of the space of the
$d$-closed forms of bidegree $(r,s)$, by the space of forms of
bidegree $(r,s)$ which are coboundaries (not necessarily of
homogeneous forms). Then the $p$th cohomology space $H^{p}$ is the
direct sum of the spaces $H^{(r,s)},r+s=p$. To prove this we observe
that, as $P^{(r,s)}$ commutes with $\Delta$, $P^{(r,s)}$ operates
canonically (\iec independent of K\"ahlerian metric) on $H^{p}$ (see
lecture 15). To define this map, we choose in each cohomology class
the harmonic form, $\omega$, belonging to this class; since $\Delta$
and $P^{(r,s)}$ commute, each homogeneous component of the harmonic
form is harmonic and hence closed. Thus the homogeneous components of
the harmonic form define cohomology classes and $P^{(r,s)}$ is the map
which maps the cohomology class of $\omega$ into the cohomology class
determined by the $(r,s)$ component of $\omega$. If $p^{r,s}H$ is the
image of $H$ by $P^{(r,s)}$, $P^{(r,s)}H$ may be identified with the
space of harmonic forms of bidegree $(r,s)$ and 
$$
H^{p}=\sum P^{(r,s)}H.
$$
But\pageoriginale $P^{(r,s)}H$ is just the space $H^{(r,s)}$ defined
above and hence
$$
H^{p}=\sum_{r+s=p}H^{(r,s)}
$$
This decomposition is intrinsic; but we have used harmonic forms to
prove this.

If we define the double Betti number $b^{(r,s)}$ to be the dimension
of the space $H^{(r,s)}$ we have
$$
b^{p}=\sum_{r+s=p}b^{(r,s)}
$$
Since the mapping $\omega\to \ob{\omega}$, which assigns to every form
its complex conjugate, induces an isomorphism of $H^{(r,s)}$ onto
$H^{(s,r)}$, we find that $b^{(r,s)}=b^{(s,r)}$; so, when $p$ is odd
$b^{p}$ is the sum of numbers which are pairwise equal and hence
$b^{p}$ is even. So the Betti numbers for odd dimensions are
even. Moreover $H^{(n-r,n-s)}$ is dual to the space $H^{(r,s)}$ so
that $b^{(n-r,n-s)}=b^{(r,s)}$. Thus we have
$$
b^{(r,s)}=b^{(s,r)}=b^{(n-r,n-s)}=b^{(n-s,n-r)}
$$

We can also prove that $b^{p}$ is even for odd $p$ by introducing a
canonical complex structure on the $p$th cohomology space $H^{p(R)}$
formed from the {\em real} forms alone. [$H^{p}$ is the
  complexification of $H^{p(R)}$ and the complex dimension of $H^{p}$
  is equal to the real dimension of\pageoriginale $H^{p(R)}$]. Since
$J$ commutes with $\Delta$, $J$ operates canonically on
$H^{p(R)}$. Since $p$ is odd $J^{2}=-I$ and $J$ gives a complex
structure on $H^{p(R)}$. So the (real) dimension of $H^{p(R)}$ is
even; and hence $b^{p}$ is even.

The next result on the Betti-numbers is the following:
$$
b^{(r,s)}\geq b^{(r-1,s-1)}\text{ \  if \ } r+s\leq n+1
$$
(From this it follows at once that $b^{p}\geq b^{p-2}$ if $p\leq
n+1$). To prove this we need the following result: the map
$$
\Omega:\overset{r-1,s-1}{\Lambda}T^{\ast}_{a}(V)\to
\overset{r,s}{\Lambda}T^{\ast}_{a}(V)
$$
(multiplication by $\Omega$) is one to one (\iec
$\Omega\wedge\omega=0$ if and only if $\omega=0$) provided $r+s\leq
n+1$. Since $L$ commutes with $\Delta$, $L$ gives a map of the space
of harmonic forms of bidegree $(r-1,s-1)$ into the space of harmonic
forms of bidegree $(r,s)$; by the algebraic result stated above this
map is one to one if $r+s\leq n+1$; consequently $b^{(r-1,s-1)}\leq
b^{r,s}$, if $r+s\leq n+1$.

The map $\Omega:\overset{r-1,s-1}{\Lambda}\to \overset{r,s}{\Lambda}$

We shall now prove that the map
  $\Omega:\overset{r-1,s-1}{\Lambda}\to \overset{r,s}{\Lambda}$ is one
to one for $r+s\leq n+1$. Since $\Omega$ is an operator of type
$(1,1)$ it is sufficient to prove that
$$
\Omega:\overset{q-2}{\Lambda}\to \overset{q}{\Lambda}
$$
is one to one for $q\leq n+1$. This would follow if we prove that
the\pageoriginale map
$$
\Omega^{p}:\overset{n-p}{\Lambda}\to \overset{n+p}{\Lambda}
$$
is one to one for $p\geq 1$. For, if $q\leq n+1$, $q-2=n-p$, for some
$p\geq 1$ and
$$
\Omega\wedge\omega=0\Rightarrow \Omega^{p}\wedge\omega=0\Rightarrow
\omega=0
$$
since
$$
\Omega^{p}:\overset{n-p}{\Lambda}\to \overset{n+p}{\Lambda}
$$
is one to one. Since $\overset{n-p}{\Lambda}$ and
$\overset{n+p}{\Lambda}$ are of the same dimension it is enough to
show that the map $\Omega^{p}$ is onto. Let
$(x_{1},y_{1},\ldots,x_{n},y_{n})$ be a basis for the space of real
differentials at $\ub{a}$ such that
$$
\Omega=\sum x_{i}\wedge y_{i}
$$
Put $x_{i}\wedge y_{i}=\alpha_{i}$. The elements of
$\overset{n+p}{\Lambda}$ are generated by elements of the form
$x_{A}\wedge y_{B}$, where the set of indices $A$ and $B$ have at
least $p$ indices in common. Consequently it is sufficient to prove
that these elements $\omega=x_{A}\wedge y_{B}$ are divisible by
$\Omega^{p}$. We may assume that
$$
\omega=\alpha_{1}\wedge\ldots\wedge \alpha_{p+s}\wedge x_{C}\wedge
y_{D}
$$
where the indices $C$ and $D$ have no elements in common. Since the
transformation $x_{k}\to y_{k}$, $y_{k}\to -x_{k}$ for indices $k$ in
$D$ does not affect $\Omega$ we may assume that $\omega$ is of the
form
$$
\alpha_{1}\wedge\ldots\wedge \alpha_{p+s}\wedge
x_{p+s+1}\wedge\ldots\wedge x_{n-s}.
$$
Since\pageoriginale
$$
(p+s)!\alpha_{1}\wedge\ldots\wedge
\alpha_{p+s}=(\alpha_{1}+\cdots+\alpha_{p+s})^{p+s}  
$$
(the exponent represents power with respect to the exterior product),
we have
\begin{align*}
\frac{\omega}{(p+s)!} &= (\alpha_{1}+\cdots+\alpha_{p+s})^{p+s}\wedge
x_{p+s+1}\wedge\ldots\wedge x_{n-s}\\
&=
(\alpha_{1}+\cdots+\alpha_{p}+\alpha_{p+1}+\cdots+\alpha_{p+s} +
\cdots+\alpha_{n-s})^{p+s}\\
& \hspace{5cm} \wedge x_{p+s+1}\wedge\ldots\wedge x_{n-s}
\end{align*}
If we put $\gamma=\alpha_{n-s+1}+\cdots+\alpha_{n}$
\begin{align*}
\frac{\omega}{(p+s)!} &= (\Omega-\gamma)^{p+s}\wedge
x_{p+s+1}\wedge\ldots \wedge x_{n-s}\\ 
&= [\Omega^{p+s}-(\underset{1}{p+s})\Omega^{p+s-1} \gamma+\cdots+
  (-1)^s\binom{p+s}{s}\Omega^{p}\gamma^{s}].\\   
&\qquad .\wedge x_{p+s+1}\wedge\ldots\wedge x_{n-s}
\end{align*}
as $\gamma^{s+1}=0$. The left side containing $\Omega^{p}$ as a
factor.

\section*{The space $H^{(p,0)}$}

We shall now show that the space $H^{(p,0)}$ is just the space of
holomorphic differential forms of degree $p$. A closed differential
form of degree $(p,0)$ is $z$ closed (by homogeneity) and hence
holomorphic; and a holomorphic form (of degree $p$) is harmonic and
hence closed. On the other hand, since a holomorphic differential form
is harmonic, a closed differential form of bidegree $(p,0)$ cannot 
be\pageoriginale a coboundary unless it is the zero form.

From this we derive at once a majorant (in terms of the $p$th
Betti-number) for the number of linearly independent holomorphic
$p$-forms. Since
$$
b^{p}=b^{(p,0)}+\cdots+b^{(0,p)}\quad\text{and}\quad
b^{(p,0)}=b^{(0,p)}
$$
we have
$$
2b^{(p,0)}\leq b^{p}(\text{for } p\neq 0)
$$
For differential forms of degree $1$ we have
\begin{align*}
b^{1} &= b^{(1,0)}+b^{(0,1)}\\
&= 2b^{(1,0)}
\end{align*}
\iec the dimension of the space of holomorphic differential forms of
degree $1$ is equal to half the first Betti number.

\section*{Compact Riemann Surfaces}

A complex analytic manifold of complex dimension $1$ is usually called
a Riemann surface. We can always introduce a K\"ahlerian metric on a
Riemann surface. Let $V^{(1)}$ be a compact, connected Riemann
surface; $b^{2}=b^{0}=1$. Let $b^{1}=2g$. The number $g$ ($=$ half the
first Betti number) is called the genus of the Riemann surface. The
number of linearly independent holomorphic forms of degree $1$ is
equal to the genus of the surface, by what we have seen. Since there
are $2g$ independent $1$-cycles and $g$ independent holomorphic
$1$-forms the periods of a holomorphic $1$-form cannot be prescribed
arbitrarily on a basis of $1$-cycles. However it can be
proved\pageoriginale that there exists a unique holomorphic
differential form with prescribed real parts of the periods.

The Riemann sphere, $S^{2}$, is of genus zero. So there are no
holomorphic differential forms of degree $1$ apart from the
$0$-form. Of course, this can be proved directly. Let $\omega$ be a
holomorphic $1$-form on $S^{2}$. $\omega$ can be written as $f(z)dz$
in the plane, where $f(z)$ is an entire function in the plane. Using
the map given by $1/z$ at $\infty$ we find that $f(1/z)$. $1/z^{2}$
should be holomorphic at the origin.

If
$$
f(z)=\sum a_{n}z^{n}\quad\text{then}\quad f(1/z)1/z^{2}=\sum
\frac{a_{n}}{z^{n+2}}
$$
so that $f(1/z)$. $1/z^{2}$ has a pole at the origin unless $f\equiv
0$.

Next we consider a torus with the complex structure induced from
$C^{1}$. Here the genus is $1$. So the differential $dz$ (which is
well defined on the torus) is, but for a constant multiple, the only
holomorphic $1$-form on the torus.

All other compact Riemann surfaces can be considered as the quotient
spaces of the unit circle by certain Fuchsian groups.


