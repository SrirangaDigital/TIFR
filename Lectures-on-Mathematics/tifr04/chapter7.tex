\chapter{Lecture 7}

\section*{Some examples of currents}\pageoriginale

\begin{enumerate}
\renewcommand{\labelenumi}{\theenumi)}
\item A $C^{\infty}p$-form $\omega$ defines a current of degree
  $p$. For every $\overset{N-p}{\varphi}\in
  \overset{N-p}{\mathscr{D}}$ we define
$$
\langle
\overset{p}{\omega},\overset{N-p}{\varphi}\rangle=\int\limits_{V}
\overset{p}{\omega}\wedge \overset{N-p}{\varphi}
$$
($\overset{p}{\omega}\wedge \overset{N-p}{\varphi}$ has compact
support).

We have to verify that if $\{\varphi_{j}\}$ is a sequence of $N-p$
forms whose supports are contained in the same compact set $K$ and if
$\varphi_{j}\to 0$ in the sense of $C^{\infty}$ then
$\int\limits_{V}\omega \wedge \varphi_{j}\to 0$. By using a partition
of unity we may assume that $K$ is contained in the domain of a
map. Then continuity follows from well-known properties of Riemann
integrals on $R^{N}$.


If a $p$-form defines the zero current it can be proved easily that
the form itself is zero. (This is a consequence of the existence of
numerous $N-p$ forms). We can therefore identify a $p$-form with the
current it gives rise to.

More generally a locally summable $p$-form defines a current of degree
$p$. A differential form of degree $p$ is said to be locally summable
if, for every compact set $K$ contained in the domain $U$ of a map
$(U,\varphi)$, the coefficients of the differential form (expressed in
terms of the map $(U,\varphi)$) are summable on $\varphi(K)$. If
$\overset{p}{\omega}$ is a locally summable $p$-form and
$\overset{N-p}{\varphi}$ an $N-p$ form with compact support
the\pageoriginale integral $\int\limits_{V}\overset{p}{\omega}\wedge
\overset{N-p}{\varphi}$ can be defined. Then the scalar product
$$
\langle \overset{p}{\omega}, \overset{N-p}{\varphi}\rangle
=\int\limits_{V}\overset{p}{\omega}\wedge \overset{N-p}{\varphi}.
$$
defines a current of degree $p$. It can be proved that if a locally
summable $p$-form defines the zero current the differential form is
zero almost every where. (Though we have no notion of a Lebesgue
measure on a manifold, the notion of a set of measure zero has an
intrinsic meaning. A set on the manifold will be said to be of measure
zero if its image by every map has Lebesgue measure zero on
$R^{N}$). So there is a $(1,1)$ correspondence between the space of
currents of degree $p$ defined by locally summable $p$-forms and the
classes of locally summable $p$ forms, a class being the set of all
$p$-forms almost everywhere equal to the same form.

\item The second example of a current is of quite a different
  character. An $N-p$ chain $\Gamma_{N-p}$ defines a current of degree
  $p$. We define, for $\varphi\in\overset{N-p}{\mathscr{D}}$
$$
\langle \Gamma_{N-p},\varphi\rangle=\int\limits_{\Gamma_{N-p}}\varphi
$$
(Since $\varphi$ has compact support
$\int\limits_{\Gamma_{N-p}}\varphi$ is defined).

In this case it can happen that a chain $\Gamma_{N-p}$ is not the zero
chain, nevertheless the integral of every $N-p$ form on $\Gamma_{N-p}$
is zero. For this reason we shall consider two chains equivalent if
for any $N-p$ form with compact support the integrals of the form on
the two chains are\pageoriginale equal. There is a $(1,1)$
correspondence between these equivalence classes and the space of
currents of degree $p$ defined by $N-p$ chains.

\item {\bf Currents of Dirac.}

This is the generalization of Dirac's distribution on $R^{N}$. Let
$\ub{a}$ be a fixed point on $V$. The Dirac $N$-current at the point
$a$, $\delta_{(a)}$, is defined by
$$
\langle \delta_{(a)},\varphi\rangle=\varphi(a)
$$
where $\underset{N-p}{\varphi}$ is a $C^{\infty}$ function with
compact support. More generally, let $X$ be a fixed $N-p$ vector
tangent to the manifold at $\ub{a}\cdot \varphi$ being an $N-p$ form
with compact support we define
$$
\left\langle \delta^{p}_{\binom{N-p}{X}},
\overset{N-p}{\varphi}\right\rangle = \left\langle
\overset{N-p}{X}, \varphi(a)\right\rangle
$$
($\varphi(a)$ is the value of the form at $a$). The scalar product on
the right is given by the duality between
$\overset{N-p}{\Lambda}T_{a}(V)$ and
$\overset{N-p}{\Lambda}T^{\ast}_{a}(V)$. This defines a current of
degree $p$. This is called a Dirac current of degree $p$.
\end{enumerate}

\section*{Partition of Unity}

Suppose $\{\Omega_{i}\}$ is an open covering of $V$. Then there exists
a system of $C^{\infty}$ scalar functions $\{\alpha_{i}\}$ defined on
$V$ such that
\begin{itemize}
\item[(i)] $\alpha_{i}\geq 0$

\item[(ii)] support of $\alpha_{i}\subset \Omega_{i}$

\item[(iii)] $\{\alpha_{i}\}$\pageoriginale are locally finite \iec
  only a finite number of supports of $\alpha_{i}$ meet a given
  compact set. (All functions $\alpha_{i}$ except a finite number
  vanish on a given compact set)

\item[(iv)] $\sum \alpha_{i}=1$ (This sum is finite at every point by
  condition (iii)).
\end{itemize}

[If $\Omega_{i}$ are relatively compact, then $\alpha_{i}$ have
  compact supports.]

The functions $\alpha_{i}$ constitute a partition of unity subordinate
to the covering $\{\Omega_{i}\}-a$ partition of the function $1$ into
non-negative $C^{\infty}$ functions with small supports.

This theorem on partition of unity proves the existence of numerous
non-trivial $C^{\infty}$ functions and forms on $V$. 

\section*{Support of a current}

Let $T$ be a current. We say that $T$ is equal to zero in an open set
$\Omega$ if $\langle T,\varphi\rangle=0$ for every form $\varphi$ with
compact support contained in $\Omega$.

Suppose $\Omega_{i}$ is a system of open sets and
$\underset{i}{\cup}\Omega_{i}=\Omega$. If a current $T$ is zero in
every $\Omega_{i}$ then $T=0$ in $\Omega$. For, let $\varphi$ be a
form with compact support contained in $\Omega$. Applying the theorem
on partition of unity we can decompose the form $\varphi$ into a
finite sum of forms having supports in $\Omega_{i}$, as the support of
$\varphi$ is compact:
$$
\varphi=\sum \alpha_{i}\varphi=\sum\varphi_{i}, \,\Supp
\varphi_{i}\subset \Omega_{i}.
$$
consequently
$$
\langle T,\varphi\rangle =\sum\langle T,\varphi_{i}\rangle =0.
$$

This\pageoriginale result shows that there exists a largest open set
in which a current $T$ is zero, namely the union of all open sets in
which $T$ is zero. (This set may be empty). The complement of this set
will be called the support of $T$. The support of the current defined
by a form $(C^{\infty})\omega$ coincides with the support of $\omega$;
the support of the current defined by a chain is not always identical
with the support of the chain.

\section*{Main operations on currents}

\begin{enumerate}
\renewcommand{\labelenumi}{\theenumi)}
\item {\bf Addition of two currents and multiplication of a current by
  a scalar:}

If $T_{1}$ and $T_{2}$ are two currents of degree $p$ we define
$T_{1}+T_{2}$ and $\lambda T_{1}$, ($\lambda$ a constant) by:
\begin{align*}
\langle T_{1}+T_{2},\varphi\rangle &= \langle T_{1},\varphi\rangle
+\langle T_{2},\varphi\rangle\\
\langle T,\varphi\rangle &= \langle T,\varphi\rangle
\end{align*}

\item {\bf Multiplication of a current by a form}

Just as in the case of distributions, we can not multiply two
currents. However we can multiply a current by a form. If
$\overset{p}{\omega}$ is a $p$ form and $\overset{q}{\alpha}$ a $q$
form, we have for $\varphi\in \overset{N-p-q}{\mathscr{D}}$
\begin{align*}
\langle\omega \wedge\alpha,\varphi\rangle &=
\int\limits_{V}(\omega\wedge\alpha)\wedge\varphi\\
&=\int\limits_{V}\omega\wedge (\alpha\wedge\varphi)\\
&=\langle \omega,\alpha\wedge \varphi\rangle
\end{align*}
and $\langle
\omega\wedge\alpha,\varphi\rangle=(-1)^{pq}\langle\alpha\wedge\omega,\varphi\rangle$. 

Now\pageoriginale for any current $\overset{p}{T}$ of degree $p$ and
any $q$-form $\overset{q}{\alpha}$ we define $\overset{p}{T}\wedge
\overset{q}{\alpha}$ by:
$$
\langle
\overset{p}{T}\wedge\overset{q}{\alpha},\overset{N-p-q}{\varphi}\rangle=\langle
\overset{p}{T},\overset{q}{\alpha}\wedge
\overset{N-p-q}{\varphi}\rangle,~
\overset{N-p-q}{\varphi}\in\overset{N-p-q}{\mathscr{D}} 
$$
We define $\overset{q}{\alpha}\wedge \overset{p}{T}$ as
$(-1)^{pq}\overset{p}{T}\wedge\overset{q}{\alpha}$. 

\item {\bf The coboundary of a current.}

Suppose $\omega$ is a $p$ form and $\varphi$ and $N-p-1$ form with
compact support. Since $V$ is a manifold without boundary Stokes'
formula yields
$$
\int\limits_{V}d(\omega\wedge \varphi)=0
$$
But $d(\omega\wedge\varphi)=d\omega\wedge \varphi+(-1)^{p}\omega\wedge
d\varphi$ so that
$$
\int\limits_{V}d\omega\wedge\varphi=(-1)^{p+1}\int\limits_{V}\omega\wedge
d\varphi
$$
or 
$$
\langle d\omega,
\varphi\rangle=(-1)^{p+1}\langle\omega,d\varphi\rangle
$$
Now for any current $T$ of degree $p$ we define the coboundary $dT$,
which is a current of degree $p+1$, by:
$$
\langle d,T,\varphi\rangle =(-1)^{p+1}\langle T,d\varphi\rangle,~
\varphi\in \overset{N-p-1}{\mathscr{D}}
$$
Let $\mathscr{D}'(V)$ denote the direct sum of the spaces of currents
of degree $0$, $1,\ldots,N\cdot \mathscr{D}'(V)$ is a graded vector
space. $d$ is a linear map $d:\mathscr{D}'(V)\to \mathscr{D}'(V)$
which raises the degree of every homogeneous element by $1$. Moreover
$d^{2}=0$. For, $T$ being a current of degree $p$ we have, for
$\varphi\in \overset{N-p-2}{\mathscr{D}}$, 
\begin{align*}
\langle dd\ T,\varphi\rangle &= (-1)^{p+2}\langle dT,d\varphi\rangle\\
 &= (-1)^{p+1}(-1)^{p+2}\langle T,dd\varphi\rangle\\
&=0.
\end{align*}\pageoriginale
\end{enumerate}

Let us consider the coboundary of a current given by a chain
$\Gamma_{N-p}$
\begin{align*}
\langle d\Gamma_{N-p},\varphi\rangle &= (-1)^{p+1}\langle\Gamma,
d\varphi\rangle\\
&= (-1)^{p+1}\int\limits_{\Gamma}d\varphi\\
&= (-1)^{p+1}\int\limits_{\Gamma}\varphi\text{ \ by Stokes'
  formula,}\\
&= (-1)^{p+1}\langle b\Gamma, \varphi\rangle.
\end{align*}
So the coboundary of a current given by a chain is the current given
by the boundary of the chain, but for sign.

Thus the coboundary operator of differential forms and the boundary
operator of chains appear as particular cases of the coboundary
operator of currents.

If $T$ is a current of degree $p$ and $\alpha$ a form, then
$$
d(T\wedge\alpha)=dT\wedge\alpha+(-1)^{p}T\wedge d\alpha.
$$
