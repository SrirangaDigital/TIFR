\chapter{Lecture 6}

\section*{Integration on chains.}\pageoriginale

Let $\underset{N}{V}$ be an oriented $C^{\infty}$ manifold without
boundary. We shall now define integrals of $p$-forms on some kind of
$p$-dimen\-sional submanifolds of $\underset{N}{V}$.

An elementary $p$-chain on $V$ is a pair $(\underset{p}{W},\Phi)$
where $\underset{p}{W}$ is a $p$-dimen\-sional oriented $C^{\infty}$
manifold with boundary and $\Phi:\underset{p}{W}\to\underset{N}{V}$ is
a $C^{\infty}$ map which is continuous at infinity (\iec the inverse
image of every compact set of $V$ is a compact set of $W$). To avoid
certain logical difficulties we shall assume that all the manifolds
$\underset{p}{W}$ are contained in $R^{\mathscr{N}_{0}}$. A $p$-chain
on $V$ is a finite linear combination of elementary $p$-chains with
real coefficients. The $p$-chains evidently form a vector space over
$R$.

The support of a $p$-form $\omega$ is the smallest closed set outside
which the form in zero. Thus it is the closure of the set of points
$\ub{a}$ such that $\omega(a)\neq 0$. The support of an elementary
chain $(\underset{p}{W},\Phi)$ is the image of the $\map \Phi$. The
support of an arbitrary chain 
$$
\Gamma_{p}=a_{1}\Gamma_{p^{1}}+\cdots+a_{k}\Gamma_{p^{k}}
$$
(where $\Gamma_{p^{i}}$ are distinct elementary chains) is defined to
be the union of the supports of $\Gamma_{i}$, $i=1,2,\ldots,k$. The
support of a chain is always a closed set (since the image of every
closed set by a map continuous at infinity is a closed set).

Let\pageoriginale $\overset{p}{\omega}$ be a $p$-form on $V$ and
$\Gamma_{p}=\left(\overset{W}{p},\Phi\right)$. Let us suppose that the
intersection of the supports of $\overset{p}{\omega}$ and $\Gamma_{p}$
is compact. Then we define the integral of $\overset{p}{\omega}$ on
$\Gamma_{p}$, $\int\limits_{\Gamma_{p}}\overset{p}{\omega}$, by the
formula
$$
\int\limits_{\Gamma_{p}}\overset{p}{\omega}=\int\limits_{\overset{W}{p}}
\Phi^{-1}\overset{p}{\omega}.   
$$
The integral on the right is defined as
$\Phi^{-1}(\overset{p}{\omega})$ has compact support. If
$\Gamma_{p}=\sum a_{i}\Gamma_{p^{i}}$ ($\Gamma_{p}i$ elementary
chains) is an arbitrary $p$-chain such that the intersection of the
supports of $\Gamma_{p}$ and $\overset{p}{\omega}$ is compact define
the integral of $\overset{p}{\omega}$ on $\Gamma_{p}$ by:
$$
\int\limits_{\Gamma_{p}}\overset{p}{\omega}=\sum
a_{i}\int\limits_{\Gamma_{p^{i}}}\omega^{p}
$$

\section*{Stockes' Formula}

Let $(\overset{W}{p},\Phi)$ be an elementary $p$-chain. Let
$\overset{\bigdot}{\underset{p-1}{W}}$ denote the boundary of
$\underset{p}{W}$ oriented canonically and
$\Phi|\overset{\bigdot}{\underset{p-1}{W}}$ the restriction of $\Phi$
to $\overset{\bigdot}{\underset{p-1}{W}}$. Then
$(\overset{\bigdot}{\underset{p-1}{W}},
\Phi|\overset{\bigdot}{\underset{p-1}{W}})$ 
is a $p-1$ chain. We define this chain to be the boundary of the
$p$-chain $(\underset{p}{W},\Phi)$. We define the boundary of an
arbitrary $p$-chain by linearity. We denote the boundary of
$\Gamma_{p}$ by $b\Gamma_{p}$. If $b$ is the operator which maps a
chain to its boundary then $b^{2}=0$, since the boundary of a manifold
with boundary is\pageoriginale a manifold without boundary.

Let $\Gamma_{p}$ be a $p$-chain and $\overset{p-1}{\omega}$ a $p-1$
form such that the supports of $\Gamma_{p}$ and
$\overset{p-1}{\omega}$ have a compact intersection. Stokes' formula
asserts that
$$
\int\limits_{b\Gamma_{p}}\overset{p-1}{\omega}=
\int\limits_{\Gamma_{p}}d\,\overset{p-1}{\omega}.  
$$

\section*{Currents}

Currents were introduced by de Rham to put the chains and the forms in
the same stock. Currents are generalizations of distributions on
$R^{N}$. Currents are related to differential forms just the same way
distributions are to functions.

Let $\overset{p}{\mathscr{E}}(V)$ denote the space of $p$-forms on
$V$. We shall introduce a topology in $\overset{p}{\mathscr{E}}(V)$ so
that $\overset{p}{\mathscr{E}}(V)$ becomes a topological vector space:
We say that the sequence of forms $\{\overset{p}{\omega}_{j}\}$ tends
to zero as $j\to \infty$, if for every map $(U,\varphi)$ and for every
compact set $K\subset U$ each derivative of each coefficient of
$\overset{p}{\omega_{j}}$ [expressed with the help of the coordinate
  functions $(x_{1},\ldots,x_{N})$ of the map $(U,\varphi)$] tends
uniformly to zero on $\varphi(K)$ as $j\to \infty$. We may simply
describe this topology as the topology of uniform convergence of the
``coefficients'' of the forms along with all their derivatives on
every compact subset of $V$. Convergence in the sense described above
is called convergence in the sense of $\mathscr{E}$ or in the sense of
$C^{\infty}$.

Let $\overset{p}{\mathscr{D}}$ denote the space of $p$-forms with
compact support. It is difficult to introduce a topology on
$\overset{p}{\mathscr{D}}$ adapted to the $C^{\infty}$\pageoriginale
structure. Let 
$\overset{p}{\mathscr{D}_{K}}$ denote the space of $p$-forms whose
supports are contained in the compact set $K$. We consider
$\overset{p}{\mathscr{D}_{K}}$ as a topological space with the
topology induced from $\overset{p}{\mathscr{E}}$.

A current, $T$, of degree $p$ (or of dimension $N-p$) is a linear form
on $\overset{N-p}{\mathscr{D}}$ the restriction of which to every
$\overset{N-p}{\mathscr{D}}_{K}$ ($K$ compact) is continuous. If $\varphi_{1}$
and $\varphi_{2}$ are $N-p$ forms with compact supports
\begin{align*}
& T(\varphi_{1}+\varphi_{2})=T(\varphi_{1})+T(\varphi_{2})\\
& T(\lambda \varphi_{1})=\lambda T(\varphi_{1}),\lambda \text{ \ a constant}
\end{align*}
if $\varphi_{j}\in \mathscr{D}^{N-p}$ have their supports in the same
compact set $K$ and if $\varphi_{j}\to 0$ in the sense of $C^{\infty}$
as $j\to \infty$ then $T(\varphi_{j})\to 0$ as $j\to \infty$.

We shall write $\langle T,\varphi\rangle$ instead of $T(\varphi)$. 


