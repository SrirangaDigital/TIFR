\chapter{Lecture 5}

\section*{Manifolds with boundary}\pageoriginale

The upper hemisphere (with the rim) is an example of a manifold with
boundary. Here the boundary is regular. But in the case of a closed
triangle the boundary has singularities at the vertices. We consider
only manifolds with good boundary. The manifolds that we have defined
earlier are special cases of manifolds with boundary and will be
referred to as manifolds without boundary.

An $N$-dimensional $C^{\infty}$ manifold with boundary is a locally
compact space $V^{N}$, countable at infinity, for which is given two
kinds of maps with the following properties. A map of the first kind
maps an open subset of $V$ homeomorphically onto an open subset of
$R^{N}$, exactly as in the case of the ordinary manifolds. A map of
the second kind maps an open subset of $V$ homeomorphically onto an
open subset of the half-space
$$
\left\{(x_{1},\ldots,x_{N}),x_{1}\geq 0\right\},\text{ in } R^{N}.
$$
The domains of the maps cover $V$. In the overlaps the maps are
related by $C^{\infty}$ functions. (We calculate the
$x_{1}$-derivatives at points on the hyperplane $x_{1}=0$ in the
positive direction). As in the case of ordinary manifolds we require
the family of maps to be complete.

The notions of $C^{\infty}$ functions, tangent vectors, differential
forms etc.\@ can be defined as in the case of manifolds without
boundary. 

Let $a\in V$. An interior tangent, $L$, at $\ub{a}$ is defined to be a
positive derivation: if $f\geq 0$ is of class $C^{1}$ in a
neighbourhood of\pageoriginale $\ub{a}$ and $f(a)=0$, then $L(f)\geq
0$ and there exists at least one function $f$ such that $f\geq 0$,
$f(a)=0$ and $L(f)>0$. An exterior tangent at $\ub{a}$ is a negative
derivation: if $f\geq 0$, $f(a)=0$ then $L(f)\leq 0$ and there exists
at least one $f$ such that $f\geq 0$, $f(a)=0$ and $L(f) < 0$.

Suppose $\ub{a}$ is a point which is in the interior of the domain of
a map of the first kind, say $(U,\varphi)$. Let $f$ be a function of
class $C^{1}$ in a neighbourhood of $\ub{a}$ with $f\geq 0$ and
$f(a)=0$. Since $f$ attains the minimum at $\ub{a}$, by considering
the function $f\circ \varphi^{-1}$ at $\varphi(a)$ we find that $f$ is
stationary at $a$. So $L(f)=0$ for any tangent vector $L$ at
$\ub{a}$. So at a point which is in the interior of the domain of at
least one map of the first kind there are no interior and exterior
tangents. On the other hand, let $\ub{a}$ be a point which is mapped
by a map of the second kind onto a point in the hyperplane $x_{1}=0$:
let $(U,\varphi)$ be a map of the second kind at $a$,
$(x_{1},\ldots,x_{N})$ the corresponding coordinate functions and
$\varphi(a)$ a point on the hyperplane $x_{1}=0$. Then the tangent
vector
$$
L=\xi_{1}\left(\frac{\partial}{\partial
  x_{1}}\right)_{a}+\cdots+\xi_{N}\left(\frac{\partial}{\partial
  x_{N}}\right)_{a} 
$$
is an interior tangent vector at $\ub{a}$ if $\xi_{1}>0$. (and
exterior tangent vector if $\xi_{1}<0$). For, if $f\geq 0$ and
$f(a)=0$, $L(f)=\xi_{1}\left(\dfrac{\partial f}{\partial
  x_{1}}\right)_{a}$ since the $f \circ \varphi^{-1}$ function is
stationary at $\varphi(a)$ on the hyperplane $x_{1}=0$; $L(f)\geq
0$ so if $\xi_{1}>0$ and for the function $f=x_{1}(x_{1}\geq
0,x_{1}(a)=0)$, $L(x_{1})=\xi_{1}>0$.

It\pageoriginale follows that a point which is in the interior of the
domain of a map of the first kind is never mapped by a map of the
second kind onto a point in the hyperplane $x_{1}=0$ and that a point
which is mapped by a map of the second kind onto a point in the
hyperplane $x_{1}=0$ is not in the interior of a map of the first
kind.

A point which is in the domain of at least one map of the first kind
is called an interior point. A point which is mapped by a map of the
second kind into a point on the hyperplane $x_{1}=0$ is called a
boundary point.

Let $\overset{\bigdot}{V^{N-1}}$ denote the set of boundary points of
$V$. $\overset{\bigdot}{V^{N-1}}$ is called the boundary of
$V$. $\overset{\bigdot}{V^{N-1}}$ is an $N-1$ dimensional manifold
without boundary; the maps of $\overset{\bigdot}{V^{N-1}}$ are given
by the restriction of the maps of the second kind to
$\overset{\bigdot}{V^{N-1}}$. Moreover the tangent space to
$\overset{\bigdot}{V^{N-1}}$ at a point $\ub{a}\in
\overset{\bigdot}{V^{N-1}}$ can be canonically identified with a
subspace of the tangent space at $\ub{a}$ to $V^{N}$.

\section*{Oriented manifolds.}

Let $E^{N}$ be an $N$-dimensional vector space over real numbers. The
space $\overset{N}{\Lambda}E^{N}$ is one dimensional and is isomorphic
to $R$, but not canonically. To orient $E^{N}$ is to decide which
elements of $\overset{N}{\Lambda}E^{N}$ should be considered positive
and which negative. If $A\neq 0$ and $B\neq 0$ are two elements of
$\overset{N}{\Lambda}E^{N}$, we have $A=aB$, $\ub{a}$ a real
number. The non-zero elements of $\overset{N}{\Lambda}E^{N}$ fall into
two classes defined as follows: two elements $B$ and $A=aB$ belong to
the same class if $a>0$ and to opposite classes if $a<0$. Selecting
one of these two classes as the class of positive $N$-vectors is
called orienting the vector\pageoriginale space $E^{N}$. A vector
space for which a choice of one of the two classes has been made is
said to be oriented. If $E^{N}$ is oriented we may orient $E^{\ast N}$
in a natural way: $\overset{N}{\Lambda}E^{\ast N}$ is the dual of
$\overset{N}{\Lambda}E^{N}$; we say that a non-zero element in
$\overset{N}{\Lambda}E^{\ast N}$ is positive if its scalar product
with any positive vector in $\overset{N}{\Lambda}E^{N}$ is positive.

We can orient only vector spaces over ordered fields.

Let $V^{N}$ be a $C^{\infty}$ manifold (with or without boundary). By
an orientation at a point $\ub{a}\in V^{N}$ we mean an orientation of
the tangent space $T_{a}(V)$. $V^{N}$ is said to be oriented if we
have chosen at every point of $V^{N}$ an orientation satisfying the
following coherence condition. Let $(U,\varphi)$ be any connected map
(\iec a map whose domain is connected) and $a\in U$. The differential
of the map $\varphi$ at $\ub{a}$ gives rise to an isomorphism of
$T_{a}(V)$ onto $T_{\varphi(a)}(R^{N})$, which can be identified with
$R^{N}$ itself. Since we have oriented $T_{a}(V)$ this isomorphism
induces an orientation on $R^{N}$. We require this induced orientation
on $R^{N}$ to be the same for every point $a\in U$.

If a manifold can be oriented it is said to be orientable; otherwise,
non-orientable. Not every manifold is orientable. For example the
Mobius-band (with or without the boundary) is non-orientable. The
difficulty in non-orientable manifolds is this: Let $(U,\varphi)$ be a
connected map. We can always choose a coherent orientation in $U$. If
$(U',\varphi')$ is another connected map such that $U\cap U'$ is
non-empty we can extend the orientation in $U$ to $U'$. In
non-orientable manifolds it so happens that when we continue the
orientation like this along certain\pageoriginale paths and come back
to $U$ we arrive at the opposite orientation.

We can give a description of the orientation which uses only the
maps. Suppose $V^{N}$ is oriented. Let $(U,\varphi)$ be a map and
$(x_{1},\ldots,x_{N})$ the coordinate functions of $(U,\varphi)$. Then
this map determines a unique orientation in $R^{N}$. If this induced
orientation is not the canonical orientation of $R^{N}$ (canonical
orientation in $R^{N}$ is the orientation for which
$e_{1}\wedge\ldots\wedge e_{N}>0$ where $(e_{1},\ldots,e_{N})$ is the
canonical basis for $R^{N}$) the map given by
$(x_{2},x_{1},x_{3},\ldots,x_{N})$ induces the canonical orientation
in $R^{N}$. Thus it is possible to cover $V^{N}$ by the domains of
maps which induce on $R^{N}$ the canonical orientation; if
$(U,\varphi)$ and $(U',\varphi')$ are two maps which induce on $R^{N}$
the canonical orientation in $R^{N}$ the coherence maps $\varphi\circ
{\varphi'}^{-1}$ and $\varphi'\circ \varphi^{-1}$ have a positive
Jacobian. Conversely if we have a covering of $V^{N}$ by the domains
of maps all of whose coherence maps have a positive Jacobian, these
maps determine an orientation of $V^{N}$. 

Let $V^{N}$ be a manifold with boundary and
$\overset{\bigdot}{V^{N-1}}$ its boundary. If $V^{N}$ is oriented,
$\overset{\bigdot}{V^{N-1}}$ is canonically oriented: the $N-1$ vector
$e_{N-1}$ tangent to $\overset{\bigdot}{V^{N-1}}$ at a point $a\in
\overset{\bigdot}{V^{N-1}}$ will be said to be positive if, $e_{1}$
being an exterior tangent vector to $V^{N}$ at $a$, the $N$ vector
$e_{1}\wedge e_{N-1}$ tangent to $V^{N}$ at $\ub{a}$ is positive.

We consider only orientable manifolds and assume further that a
definite orientation has been chosen for the manifold.

\section*{Integration on a Manifold.}

Let $V^{N}$ be an oriented manifold (with or without
boundary). Let\pageoriginale $\omega$ be a continuous $N$-form on
$V^{N}$ which vanishes outside a compact set. With $\omega$ we can
associate a real number, $\int\limits_{V}\omega$ called the integral
of $\omega$ on $V$ possessing the following properties:
\begin{enumerate}
\renewcommand{\labelenumi}{\theenumi)}
\item Let $K$ be a compact set outside of which $\omega$ is zero. Let
  $U$ be an open set containing $K$. Then
$$
\int\limits_{U}\omega=\int\limits_{V}\omega.
$$
(The orientation on $U$ is the one induced from the orientation on
$V$) 

\item If $\omega$ and $\omega'$ are two continuous $N$-forms vanishing
  outside compact sets, 
$$
\int\limits_{V}\omega+\omega'=\int\limits_{V}\omega+\int\limits_{V}\omega';\quad
\int\limits_{V}\lambda \omega=\lambda\int\limits_{V}\omega, \,\lambda
$$
a constant.

\item If $\Phi:V^{N}\to W^{N}$ is an orientation preserving
  diffeomorphism of $V^{N}$ onto $W^{N}$ and $\ob{\omega}$ a
  continuous $N$-form on $W^{N}$, then
$$
\int\limits^{-1}_{V}\Phi\omega=\int\limits_{W}\omega.
$$

\item If $V$ is an open subset of $R^{N}$ with the canonical
  orientation of $R^{N}$ and $\omega=f\ dx_{1}\wedge\ldots\wedge
  dx_{N}$, $f$ a continuous function vanishing outside a compact set,
$$
\int\omega=\int\ldots\int f\ dx_{1}\ldots dx_{N}
$$
where the term in the right side denotes the ordinary Riemann integral
of $f$.
\end{enumerate}

It can be proved that these four properties determine the integral
uniquely.


