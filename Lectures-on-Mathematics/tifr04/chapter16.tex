\chapter{Lecture 16}

\section*{Real vector spaces with a $J$-Structure}\pageoriginale

Suppose $G^{(n)}$ is a vector space over the complex numbers. $G$ can
also be considered as a vector space over the real numbers. If
$e_{1},\ldots,e_{n}$ is a basis of $G^{n}$ over $C$, then
$e_{1},\ldots,e_{n}$, $ie_{1},\ldots,ie_{n}$ is a basis of $G$ over
$R$. The multiplication by $i$ is a linear transformation of $G$,
considered as a vector space over $R$, whose square is $-I$, where $I$
is the identity map. The vectors $e$ and $ie$ are dependent when $G$
is considered as a vector space over $C$ but are independent over
$R$. To avoid confusion we shall introduce the notion of a real vector
space with a $J$-structure.

A $2n$-dimensional real vector space $G$ along with a linear
transformation $J:G\to G$ with $J^{2}=-I$ will be a called a real
vector space with a $J$-structure.

(A real vector space $G$ with a $J$-structure can be considered as a
vector space over complex numbers by defining
$$
(\alpha+i\beta)e=\alpha e+\beta Je
$$
where $\alpha$ and $\beta$ are real numbers and $e$ is an element of
$G$).

Let $G$ be a real vector space with a $J$-structure and $G+iG$ its
complexification; the operator $J$ is extended canonically to $G+iG$
by defining:
$$
J(X+iY)=JX+iJY, X, Y\in G
$$
$J$\pageoriginale is a linear transformation on the complex vector
space $G+iG$. (We notice that the operations $J$ and multiplication by
$i$ are different on $G+iG$; if $x$ is a vector in $G$, $Jx$ is a
vector in $G$ while $ix$ is a vector in $iG$). The operator $J$ on
$G+iG$ has the eigen values $i$ and $-i$.

\footnotetext{We now define the operators $\dfrac{\partial}{\partial
    z_{j}}$ and $\dfrac{\partial}{\partial\ob{z}_{j}}$ on
  differentiable functions on $C^{n}/(z_{1},\ldots,z_{n})$,
  $z_{j}=x_{j}+iy_{j}$ are the coordinate functions on $C^{n}$). A
  priori they do not make sense. We define
$$
\frac{\partial}{\partial
  z_{j}}=\frac{1}{2}\left(\frac{\partial}{\partial
  x_{j}}-i\frac{\partial}{\partial y_{j}}\right), 
$$}
$J$ is an isomorphism of $G+iG$ onto itself. Canonically we have an
isomorphism of the dual space of $G+iG$, $(G+iG)^{\ast}$, onto itself
contragradient to $J$ (this isomorphism is the inverse of the
transpose of $J$). We denote this operator on the dual space also by
$J$. If $\alpha\in G+iG$ and $\beta\in (G+iG)^{\ast}$ then
\begin{gather*}
\langle\alpha,\beta\rangle=\langle J\alpha, J\beta\rangle\\
\langle J\alpha,\beta\rangle=\langle\alpha,J^{-1}\beta\rangle=-\langle
\alpha,J\beta\rangle. 
\end{gather*}
$J$ is also defined canonically on the exterior products of
$(G+iG)^{\ast}$. On the $p$-th exterior product we have
$$
J^{2}=(-1)^{p}I.
$$
The operators $\dfrac{\partial}{\partial z_{j}}$ and
$\dfrac{\partial}{\partial\ob{z}_{j}}$ 
$$
\frac{\partial}{\partial
  \ob{z}_{j}}=\frac{1}{2}\left(\frac{\partial}{\partial
  x_{j}}+i\frac{\partial}{\partial y_{j}}\right) 
$$\pageoriginale 
We have
\begin{gather*}
\frac{\partial}{\partial x_{j}}=\frac{\partial}{\partial
  z_{j}}+\frac{\partial}{\partial \ob{z}_{j}}\\ 
\frac{\partial}{\partial y_{j}}=i\left(\frac{\partial}{\partial
  z_{j}}-\frac{\partial}{\partial \ob{z}_{j}}\right) 
\end{gather*}
The reason for such a definition is as follows. If we take an analytic
function of $2n$ real variables $x_{1}$, $y_{1},\ldots,x_{n}$, $y_{n}$
it can be extended to an analytic function of $2n$ complex variables,
which we still denote by $x_{1}$, $y_{1},\ldots,x_{n}$,
$y_{n}$. Consider
$$
z_{j}=x_{j}+iy_{j}, \ob{z}_{j}=x_{j}-iy_{j}
$$
as $2n$-independent complex variables (Here $\ob{z}_{j}$ does not mean
the complex conjugate of $z_{j}$; this is so only if $x_{j}$ and
$y_{j}$ are real). By changing the variables $x_{j}$, $y_{j}$ to
$z_{j}$, $\ob{z}_{j}$ we get the above expressions for the partial
derivatives $\dfrac{\partial}{\partial z_{j}}$ and
$\dfrac{\partial}{\partial \ob{z}_{j}}$. Thus these relations are true
for analytic functions of real variables prolonged into the complex
field. So we take these as general definitions.

We have
$$
\frac{\partial}{\partial z_{j}}(z_{k})=\delta_{jk},
\frac{\partial}{\partial\ob{z}_{j}}(z_{j})=0.
$$
$\dfrac{\partial}{\partial_{z_{j}}}$\pageoriginale
$\dfrac{\partial}{\partial \ob{z}_{j}}$ are the complex
derivations. $\dfrac{\partial f}{\partial z_{j}}$ and $\dfrac{\partial
  f}{\partial \ob{z}_{j}}$ are defined for any differentiable function
$f$.

Suppose $f$ is a $C^{\infty}$ function on $C^{n}$, We have
$$
df-\sum_{j}\frac{\partial f}{\partial
  x_{j}}dx_{j}+\sum_{j}\frac{\partial f}{\partial y_{j}}dy_{j}.
$$

But this may be written as
$$
\df=\sum_{j}\frac{\partial f}{\partial
  z_{j}}dz_{j}+\sum_{j}\frac{\partial f}{\partial
  \ob{z}_{j}}d\ob{z}_{j}
$$
(Here $\dfrac{\partial f}{\partial z_{j}}$ and $\dfrac{\partial
  f}{\partial \ob{z}_{j}}$ are the complex derivatives of $f$ defined
above. $dz_{j}$ and $d\ob{z}_{j}$ are the differentials of the
functions $z_{j}$ and $\ob{z}_{j}$). Thus we have an expression for
$\df$ as though $z_{j}$ and $\ob{z}_{j}$ were independent
variables. Similarly the formula for the coboundary of a differential
form continues to hold as though $z_{j}$ and $\ob{z}_{j}$ were
independent variables.

\section*{Holomorphic functions on $C^{n}$}

If $f$ is a complex valued function defined on an open subset $\Omega$
of $C'$ we say that $f$ is holomorphic in $\Omega$ if
$$
\lim\limits_{\zeta\to 0}\frac{f(z+\zeta)-f(z)}{\zeta},
$$
exists at every point $z$ of $\Omega$. We know that if $f$ is
holomorphic it satisfies the Cauchy rule; the Cauchy rule can be
written as $\dfrac{\partial f}{\partial \ob{z}}=0$. Conversely, if $f$
is $C'$ and $\dfrac{\partial f}{\partial \ob{z}}=0$, $f$ is
holomorphic. So a holomorphic function of one complex variable may be
defined\pageoriginale as a $C^{1}$ function of $x$ and $y$ for which
$\dfrac{\partial f}{\partial \ob{z}}=0$. (We may say that a
holomorphic function is independent of $\ob{z}$). For functions of
several variables we adopt a similar definition. We say that a complex
valued function $f$ defined on an open subset of $C^{n}$ is
holomorphic if $f$ is a $C^{\infty}$ function with respect to the $2n$
real coordinates and
$$
\frac{\partial f}{\partial \ob{z}_{j}}=0(j=1,2,\ldots,n).
$$

\section*{Transformation formulae}

Suppose we have a diffeomorphism
$$
(z_{1},\ldots,z_{n})\to (\zeta_{1},\ldots,\zeta_{n})
$$
between two open subsets of $C^{n}$ given by the functions
$\zeta_{k}=\zeta_{k}(x_{1},y_{1},\break\ldots,x_{n},y_{n})$,
$k=1,\ldots,n$. We then have the following formulae:
\begin{align*}
d\zeta_{k} &= \sum_{j}\left(\frac{\partial \zeta_{k}}{\partial
  z_{j}}dz_{j}+\frac{\partial \zeta_{k}}{\partial
  \ob{z}_{j}}d\ob{z}_{j}\right)\\
d\ob{\zeta}_{k} &= \sum_{j}\left(\frac{\partial
  \ob{\zeta}_{k}}{\partial z_{j}}dz_{j}+\frac{\partial
  \ob{\zeta}_{k}}{\partial \ob{z}_{j}}d\ob{z}_{j}\right). 
\end{align*}
In particular if the $\zeta_{i}$ are holomorphic functions of
$z_{1},\ldots,z_{n}$ we have
$$
d\zeta_{k}=\sum_{j}\frac{\partial \zeta_{k}}{\partial z_{j}}dz_{j}
$$
(Here\pageoriginale $\dfrac{\partial \zeta k}{\partial z_{j}}$ is the ordinary
partial derivative of $\zeta_{k}$ with respect to $z_{j}$ defined in
the theory of holomorphic functions; and thus our notation is
coherent). In this case we also have
\begin{align*}
d\ob{\zeta}_{k} &= \sum_{j}\left(\frac{\ob{\partial
    \zeta_{k}}}{\partial z_{j}}\right)d\ob{z}_{j}\\
\frac{\partial}{\partial z_{j}} &= \sum_{k}\frac{\partial
  \zeta_{k}}{\partial z_{j}}\frac{\partial}{\partial \zeta_{k}}\\
\frac{\partial}{\partial \ob{z}_{j}} &= \sum_{k}\frac{\partial
  \ob{\zeta}_{k}}{\partial \ob{z}_{j}}\frac{\partial}{\partial
  \ob{\zeta}k}. 
\end{align*}

\section*{Canonical complex structure on $R^{2n}$}

Let $(x_{1},\ldots,x_{2n})$ be the coordinate functions on
$R^{2n}$. We shall identify $R^{2n}$ with $C^{n}$ by the map
$$
(x_{1},\ldots,x_{2n})\to (z_{1},\ldots,z_{n})
$$
where
$$
z_{j}=x_{2j-1}+ix_{2j}(j=1,2,\ldots,n)
$$
We thus have a canonical complex structure on $R^{2n}$. If
$(e_{1},\ldots,e_{2n})$ is the canonical basis for $R^{2n}$ then the
canonical complex structure on $R^{2n}$ is given by the $J$ operator
defined by:
$$
Je_{2j-1}=e_{2j}, Je_{2j}=-e_{2j-1}(j=1,\ldots,n)
$$

Let $U$ be an open subset of $R^{2n}$ and $\Phi:U\to R^{2n}$
be\pageoriginale a $C^{\infty}$ map. In terms of the canonical complex
coordinates on $R^{2n}$ we may give this map by:
$$
(z_{1},\ldots,z_{n})\to (\zeta_{1},\ldots,\zeta_{n})
$$
We say that the map $\Phi$ is complex analytic (with respect to the
canonical complex structure on $R^{2n}$) if
$\zeta_{1},\ldots,\zeta_{n}$ are holomorphic functions of
$(z_{1},\ldots,z_{n})$. 

\section*{Complex analytic manifolds}

A complex analytic manifold, $V^{(n)}$, of complex dimension $n$ is a
$C^{\infty}$ manifold of real dimension $2n$ with an atlas
$\{(U_{i},\varphi_{i})\}$ (which is incomplete with respect to the
$C^{\infty}$ structure) having the following property: for any two
maps $(U_{i},\varphi_{i})$ and $(U_{j},\varphi_{j})$ of the atlas, the
map
$$
\varphi_{j}\circ \varphi^{-1}_{i}:\varphi_{i}(U_{i}\cap U_{j})\to
\varphi_{j}(U_{i}\cap U_{j})
$$
is complex analytic (with respect to the canonical complex structure
on $R^{2n}$). We assume that the atlas is complete with respect to the
complex analytic structure.

\section*{Some examples of complex analytic manifolds}

\begin{enumerate}
\renewcommand{\theenumi}{\roman{enumi}}
\renewcommand{\labelenumi}{\theenumi)}
\item $C^{n}$. The simplest example of a complex analytic manifold is
  $C^{n}$ itself.

\item The Riemann sphere $S^{2}$. Consider $S^{2}$ as the one point
  compactification of $C^{1}:S^{2}=C^{1}\cup \infty$. Take for one map
  the identity map of $C^{1}$. For the second map take the map $\zeta$
  defined by 
\begin{gather*}
\zeta(z)=1/z,\quad z\neq \infty\\
\zeta(\infty)=0
\end{gather*}\pageoriginale
in the complementary set of $0$. The intersection of these two maps
is the complement of the points $0$ and $\infty$ and here $\zeta=1/z$
is a holomorphic function of $z$.

It is known that the only spheres on which we may have a complex
analytic structure are $S^{2}$ and $S^{6}$. On $S^{2}$ we have a
complex analytic structure. For $S^{6}$ we do not know.

\item {\bf The complex projective space $PC^{n}$}

The right generalization of the Riemann sphere is the $n$-dimen\-sional
complex projective space, $PC^{n}$. The $n$-dimensional complex
projective space is defined as follows. We take $C^{n+1}$ and omit
$0$. In $C^{n+1}-(0)$ we introduce an equivalence relation two points
$z=(z_{1},\ldots,z_{n+1})$ and $z'=(z'_{1},\ldots,z'_{n+1})$ are
equivalent if $z'_{i}=\lambda z_{i}(i=1,\ldots,n+1)(z'=\lambda z)$ for
some $\lambda\neq 0$. The quotient space of $C^{n+1}-(0)$ by this
relation (with the quotient topology) is the $n$-dimensional complex
projective space, $PC^{n}\cdot PC^{n}$ is a compact complex analytic
manifold of complex dimension $n$. We introduce complex analytic
coordinate systems in $PC^{n}$ as follows. For a fixed $i$ consider
the set of points $\ub{a}$ in $PC^{n}$ whose representatives in
$C^{n+1}$ are of the form $(z_{1},\ldots,z_{n+1})$, $z_{i}\neq 0$.


Then\pageoriginale the mapping
$$
a\to
\left(\frac{z_{1}}{z_{i}},\ldots,\frac{z_{i-1}}{z_{i}},\frac{z_{j+1}}{z_{i}},\ldots,\frac{z_{n+1}}{z_{i}}\right).  
$$
gives a map. The maps obtained for $i=1,2,\ldots,n+1$ cover $PC^{n}$
and are related by holomorphic functions on the overlaps.

\item The complex torus.

\end{enumerate}
