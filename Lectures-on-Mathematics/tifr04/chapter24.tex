\chapter{Lecture 24}

\section*{Cousin's Problem on $C^{n}$}\pageoriginale

It has been proved recently that for $C^{n}$ (the complex $n$-space)
$H^{(p,q)}_{\ob{z}}=0$ for $q>0$. So Cousin's problem for meromorphic
forms is solvable in $C^{n}$ for any Cousin datum.

\section*{Cousin's problem on a compact K\"ahlerian manifold:
  pseudo-solution}

We can give a more explicit solution in the case of a compact
K\"ahlerian manifold. A necessary and sufficient condition for
$\overset{p,1}{R}$ to be a $\ob{z}$ coboundary is that
$\pi_{1}R=0$. (This gives a system of $b^{p,1}$ linear conditions for
$R$). Now
\begin{align*}
R &=
\pi_{1}R+2d_{\ob{z}}\partial_{\ob{z}}GR+2\partial_{\ob{z}}d_{\ob{z}}GR\\
 &= \pi_{1}R+2d_{\ob{z}} \partial_{\ob{z}}GR.
\end{align*}
We can choose $S=2\partial_{\ob{z}}GR$.

Then 
$$
\omega=\omega_{i}+2\partial_{\ob{z}}GR-T_{i}\quad\text{in}\quad U_{i}
$$
is a solution of the Cousin's problem. Even in the case when Cousin's
problem has no solution, $\omega$ defined as above makes sense. Even
if $T_{i}$ are currents, $(2\partial_{\ob{z}}GR-T_{i})$ is a
$C^{\infty}$ form; for
\begin{align*}
d_{\ob{z}}(2\partial_{\ob{z}}GR-T_{i}) &=
2d_{\ob{z}}\partial_{\ob{z}}GR-d_{\ob{z}}T_{i}\\ 
&= (2d_{\ob{z}}\partial_{\ob{z}}GR-R)\\
 &= -\pi_{1}R. 
\end{align*}
$\pi_{1}R$\pageoriginale is a $C^{\infty}$ form and
$(2\partial_{\ob{z}}GR-T_{i})$ is a current of bidegree $(p,0)$; by
the ellipticity of $d_{\ob{z}}$, $(2\partial_{\ob{z}}GR-T_{i})$ is a
$C^{\infty}$ form. $\omega$ is thus a sum of a meromorphic form and a
$C^{\infty}$ form. When Cousin's problem is solvable
$(2\partial_{\ob{z}}GR-T_{i})$ is a holomorphic form and $\omega$ is a
solution of Cousin's problem. In any case, we call $\omega$ a
pseudo-solution. When Cousin's problem is solvable, the
pseudo-solution is actually a solution.

\section*{Cousin's problem on $PC^{n}$}

For $p\neq 1$ Cousin's problem has a solution for any Cousin datum,
since $b^{p,1}=0$ for $p\neq 1$. The solution is unique if $p\geq 2$
and is determined upto an additive constant when $p=0$, as holomorphic
differentials of degree $p\geq 1$ are zero and those of degree zero
are constants. For $p=1$, for Cousin's problem to be solvable it is
necessary and sufficient that $R$ be orthogonal to all closed
$C^{\infty}$ forms of bidegree $(n-1,n-1)$. But $\Omega^{n-1}$ is a
generator of the cohomology classes of bidegree $(n-1, n-1)$. So the
condition is
$$
\langle R,\Omega^{n-1}\rangle=0.
$$
When the solution exists, the solution is unique.

\section*{Cousin's problem on a compact Riemann Surface}

We shall now consider Cousin's problem on a compact Riemann surface,
which we assume to be connected. We shall first consider the Cousin's
problem for meromorphic differential forms of degree $1$. For Cousin's
problem to be solvable, it is necessary and sufficient that
$\overset{1,1}{R}$ ($\overset{1,1}{R}$ is a\pageoriginale current of
the associated cohomology class) be orthogonal to all closed zero
forms \iec constants: or
$$
\langle \overset{1,1}{R},1\rangle=0.
$$
We shall now interpret this condition in terms of residues. Let
$\omega$ be a meromorphic differential form of degree $1$. We define
the residue of $\omega$ at a pole $\ub{a}$ of $\omega$, denoted by
$\Res_{a}\omega$, as follows: We choose a local coordinate system
$(z)$ at $\ub{a}\cdot \omega=f(z)dz$, where $f(z)$ is a meromorphic
function of $z$. Then $\Res_{a}\omega$ is defined to be the residue of
$f(z)dz$ at $z(a)$. This definition is intrinsic. If we choose a
regular curve $C$, contained in the domain of a map, winding around
$\ub{a}$ once in the positive sense
\begin{align*}
\Res_{a}(f(z)dz) &= \frac{1}{2\pi i}\int\limits_{C}f(z)dz\\
&= \frac{1}{2\pi i}\int\limits_{C}\omega.
\end{align*}
Since the Riemann surface has no boundary, the sum of the residues of
a meromorphic differential of degree $1$ is zero. This gives a trivial
necessary condition on the Cousin datum, for Cousin's problem to be
solvable. We shall see that this condition is also sufficient.

Let $a_{1},\ldots,a_{m}$ be a finite number points given on the
Riemann surface. At each $\ub{a}$ we have a map $(W_{a},\varphi_{a})$
such that $W_{a}\cap W_{b}$ is empty for $a\neq b$. Let $U_{a}$ be a
neighbourhood of $\ub{a}$ such that $\ob{U}_{a}\subset W_{a}$ and
$\varphi_{a}(\ob{U}_{a})$ is a closed disc. In each
$W_{a}$\pageoriginale we are given a meromorphic form of degree $1$,
$\omega_{a}$, having a pole only at $\ub{a}$. Let $U_{0}$ be the
complement of the set of points $a_{1},\ldots,a_{m}$. In $U_{0}$ we
put $\omega_{0}\equiv 0$. With this Cousin datum we associate currents
$T_{a}$ in $U_{a}$ and $T_{0}$ in $U_{0}$ defined as follows:
\begin{align*}
T_{a} &= 0\text{ \  in \ } U_{a}\\
T_{0} &= -\sum \widetilde{\omega}_{a}\text{ \  in \ } U_{0}
\end{align*}
where
$$
\widetilde{\omega}_{a}=
\begin{cases}
\omega_{a}\text{ \  in \ } \ob{U}_{a}\\
0 \text{ \ outside \ } \ob{U}_{a}.
\end{cases}
$$
The form $\widetilde{\omega}_{a}$ has discontinuities along the
boundary of $U_{a}$; but $\widetilde{\omega}_{a}$ is locally summable
in $U_{0}$ and defines a current in $U_{0}$. In $U_{a}\cap U_{0}$,
$$
T_{a}-T_{0}=\widetilde{\omega}_{a}=\omega_{a}
$$
The current
$$
R=
\begin{cases}
d_{\ob{z}}T_{a} \text{ \ in \ } U_{a}\\
d_{\ob{z}}T_{0} \text{ \  in \ } U_{0}
\end{cases}
$$
is the ``obstruction''. $R=0$ in $U_{a}$. Since $T_{0}$ is of bidegree
$(1,0)$, $d_{\ob{z}}T_{0}=dT_{0}=$. If $\varphi$ is a $C^{\infty}$
function with compact support {\em in $U_{0}$} (\iec if the support of
$\varphi$ does not contain the points $a_{i}$),
\begin{align*}
\langle d_{\ob{z}}T_{0},\varphi\rangle &= \langle d
\overset{1,0}{T_{0}},\varphi\rangle\\
&= \langle T_{0},d\varphi\rangle\\
&= \langle -\sum\widetilde{\omega}_a, d\varphi\rangle\\
&= -\sum_{a}\iint\limits_{U_{a}}\omega_{a}\wedge d\varphi\\
&= \sum_{a}\iint\limits_{U_{a}}d(\omega_{a}\varphi)\\
&= \sum_{a}\int\limits_{bU_{a}}\omega_{a}\varphi.
\end{align*}\pageoriginale
Consequently
$$
R=\sum (bU_{a})\wedge \omega_{a}
$$
in $U_{0}$ and this relation is also true in $U_{a}$. [$\omega_{a}$ is
  a $C^{\infty}$ form in a neighbourhood of the support of
  $bU_{a}$. So multiplication of $bU_{a}$ by $\omega_{a}$ is
  possible]. Now the necessary and sufficient condition for Cousin's
problem to have a solution is
$$
\langle R,1\rangle =0
$$
or
$$
\sum_{a}\int\limits_{bU_{a}}\omega_{a}=0
$$
\iec
$$
\sum_{a}\Res_{a}\omega_{a}=0.
$$
In particular if we take $\omega_{a}$ without residues a solution of
the problem always exists.

The indeterminacy in the solution is a holomorphic differential
form\pageoriginale\ of degree $1$. Thus in this problem we have one
condition of compatibility and $g$ degrees of indeterminacy.

If we choose a K\"ahlerian metric on the Riemann surface, an explicit
solution is given by
$$
\omega=
\begin{cases}
\omega_{a}+2\partial_{\ob{z}}GR\text{ \  in \ } U_{a}\\
2\partial_{\ob{z}}GR+\sum\widetilde{\omega}_{a}\text{ \ in \ } U_{0}.
\end{cases}
$$
We may write
$$
\omega=\sum_{a}\widetilde{\omega}_{a}+2\partial_{\ob{z}}G[\sum_{a}\{bU_{a}\wedge\omega_{a}\}]
$$
as $\omega_{a}=\widetilde{\omega}_{a}$ in $U_{a}$. This expression has
always a meaning and gives a solution of the Cousin's problem when the
problem is solvable. The general solution of Cousin's problem is given
by
$$
\omega=\sum_{a}\widetilde{\omega}_{a}+2\partial_{\ob{z}}G[\sum_{a}(bU_{a}\wedge\omega_{a})]+h 
$$
where $h$ is a holomorphic differential form of degree $1$ on the
Riemann surface.

\section*{Cousin's problem for meromorphic functions\protect\hfil
\protect\break (Compact Riemann Surface)}

Here in each $U_{a}$ we have a meromorphic function $f_{a}$ instead of
a meromorphic differential form. In $U_{0}$ we take the function
$f_{0}=0$. Let $\tilde{f}_{a}$ be the function defined on the surface by
$$
\widetilde{f}_{a}=
\begin{cases}
f_{a} \text{ \  in \ } \ob{U}_{a}\\
0 \text{ \  outside \ } \ob{U}_{a}
\end{cases}
$$
We\pageoriginale consider the currents $T_{a}=0$ in $U_{a}$ and
$T_{a}=-\sum\widetilde{f}_{a}$ in $U_{0}$. Let $\overset{0,1}{R}$ be
the associated current of bidegree $(0,1)$. If
$\overset{1,0}{\varphi}$ is a $C^{\infty}$ form with compact support
in $U_{0}$
\begin{align*}
\langle d_{\ob{z}}T_{0},\overset{1,0}{\varphi}\rangle &= -\langle
T,d_{\ob{z}}\varphi\rangle\\ 
&= \langle\sum_{a}\widetilde{f}_{a},d_{\ob{z}}\varphi\rangle\\
&= \sum_{a}\iint\limits_{U_{a}}d_{\ob{z}}(f_{a}\varphi)\\
&= \sum \iint\limits_{U_{a}}d(f_{a}\varphi)\\
&= \sum\int\limits_{bU_{a}}f_{a}\varphi.
\end{align*}
Consequently,
$$
\overset{0,1}{R}=\sum_{a}(bU_{a})f_{a}.
$$
A necessary and sufficient condition for this Cousin's problem to have
a solution is that $R$ be orthogonal (with respect to $\langle$ ,
$\rangle$)
to all harmonic forms of bidegree $(1,0)$ \iec $R$ be orthogonal to
all holomorphic forms of degree $1$ or 
$$
\sum \int\limits_{bU_{a}}f_{a}h=0
$$
for every holomorphic form $h$ of degree $1$.

For Cousin's problem for meromorphic functions we have $g$ independent
compatibility conditions and $1$ degree of indeterminacy while for
Cousin's problem for meromorphic forms we had one
compatibility\pageoriginale condition and $g$ degress of
indeterminacy. This suggests a sort of duality between Cousin's
problems for meromorphic functions and forms. This duality will be
made precise in the theorem of Riemann-Roch.

The above results prove the existence of lots of meromorphic functions
and forms on a compact Riemann surface.


