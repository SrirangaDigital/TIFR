\chapter{Lecture 10}

\section*{Some applications}\pageoriginale

Since a closed $0$-form is a function which is constant on each
connected component, $b^{0}$ is the number of connected components of
$V$. A closed $0$-form with compact support is a function which is
constant on each compact connected component and zero on each
non-compact component. It follows that

$b^{0}_{c}=$ number of compact connected components. By Poincar\'e's
duality theorem we have

$b^{N}=b^{0}_{c}=$ number of compact connected components,

$b^{N}_{c}=b^{0}=$ number of connected components.

Let $\overset{N}{\omega}$ be an $N$-form with compact support. If
$\overset{N}{\omega}$ is to be the coboundary of an $N-1$ form with
compact support it is necessary and sufficient that
$\langle\overset{N}{\omega},f\rangle=0$ for every closed $0$-form
(orthogonality relations). If $V$ is connected, a closed $0$-form is a
constant function. So, in case $V$ is connected, for
$\overset{N}{\omega}$ to be the coboundary of an $N-1$ form with
compact support it is necessary and sufficient that
$$
\int\limits_{V}\overset{N}{\omega}=0.
$$

Similarly we can prove that a necessary and sufficient condition for
an $N$-form $\overset{N}{\omega}$ with arbitrary support to be the
coboundary of\pageoriginale an $N-1$ form is that integral of
$\overset{N}{\omega}$ on every compact connected component of the
manifold should be zero.

\section*{The third part of de Rham's Theorem}

The third part of de Rham's theorem states that if $V$ is compact all
the Betti numbers of $V$ are finite.

In the compact case there is no difference between cohomology with
compact supports and cohomology with arbitrary supports. The $p$th 
and $(N-p)$ th cohomology spaces are canonically the duals of each
other and we have the duality relation for the Betti-numbers:
$$
b^{p}=b^{N-p}
$$

\section*{Riemannian Manifolds}

An $N$-dimensional Euclidean vector space $E^{N}$ over $R$ is an $N$
dimensional vector space over $R$ with a positive definite bilinear
form (or what is the same, a positive definite quadratic form). There
is a scalar product $(x,y)$ between any two elements $x$ and $y$ of
$E^{N}$ which is bilinear and which has the properties 
$$
(x,y)=(y,x),\quad\text{and}\quad (x,x)>0\quad\text{for}\quad x\neq 0.
$$

Let $e_{1},\ldots,e_{N}$ be a basis of $E^{N}$ and
$g_{ij}=(e_{i},e_{j})$. If $x=\sum x_{i}e_{i}$ and $y=\sum y_{i}e_{i}$
are two vectors of $E^{N}$ we have
$$
(x,y)=\sum g_{ij}x_{i}y_{j}.
$$
A $C^{\infty}$ manifold $V^{N}$ is called a $C^{\infty}$ Riemannian
manifold if on each tangent space of $V^{N}$ we have a positive
definite bilinear form such\pageoriginale that the twice covariant
tensor field defined by these bilinear forms is a $C^{\infty}$ tensor
field. Thus at each tangent space of a Riemannian manifold we have a
Euclidean structure.

The condition that the tensor field defined by the bilinear forms is a
$C^{\infty}$ tensor field may be expressed in terms of local
coordinate systems as follows: for every choice of the local
coordinate system $x_{1},\ldots,x_{N}$ the functions
$$
g_{ij}(a)=\left(\left(\frac{\partial}{\partial
  x_{i}}\right)_{a},\left(\frac{\partial}{\partial
  xj}\right)_{a}\right)
$$
are $C^{\infty}$ functions in the domain of the coordinate system, the
scalar product (\;,\;) being given by the bilinear form on $T_{a}(V)$.

\section*{Riemannian structure on an arbitrary $C^{\infty}$ manifold}

We shall now show that we can introduce a $C^{\infty}$ Riemannian
structure on any $C^{\infty}$ manifold. This result is of importance
because the results on a Riemannian manifold which are of a purely
topological nature can be proved for any manifold by introducing a
Riemannian structure.

Let $V$ be a $C^{\infty}$ manifold. If $(U_{i},\varphi_{i})$ is a map
we define a positive definite quadratic form at each $T_{a}(V)$, $a\in
U_{i}$ by transporting the fundamental quadratic form
``$(dx^{2}_{1}+\cdots+dx^{2}_{N})$'' at $T_{\varphi(a)}(R^{N})$ by
means of the isomorphism between $T_{a}(V)$ and
$T_{\varphi(a)}(R^{N})$ given by the differential of the map $\varphi$
at $\ub{a}$. Let $Q_{i}(a)$ denote the positive definite quadratic
form in $T_{a}(V)$ given  by\pageoriginale the map
$(U_{i},\varphi_{i})$. Let $\{\alpha_{i}\}$ be a partition of unity
sub-ordinate to the $\{U_{i}\}$. For a $\in V$, we define a quadratic
form in $T_{a}(V)$ by: $Q(a)=\sum\alpha_{i}(a)Q_{i}(a)$ (the summation
being over all $U_{i}$ containing $\ub{a}$; only a finite number of
$\alpha_{i}(a)$ are different from zero). $Q(a)$ is a positive
definite quadratic form as the $Q_{i}(a)$s are positive definite,
$\alpha_{i}(a)\geq 0$ and at least one $\alpha_{i}(a)\neq 0$. Since
the $\alpha_{i}$ are locally finite we can find a neighbourhood $U$ of
an arbitrary point of $V$ such that
$$
Q(a)=\sum\alpha_{i}(a)Q_{i}(a)\quad\text{(finite sum) for}\quad a\in U.
$$
where $Q_{i}(a)$ are $C^{\infty}$ quadratic forms on $U$. This proves
that the quadratic forms $Q(a)$ define a $C^{\infty}$ Riemannian
structure on $V$.

In the above argument we have made essential use of the positive
definiteness of the quadratic form. The same construction would not
succeed if we want to construct a $C^{\infty}$ indefinite metric with
prescribed signature; for the sum of two quadratic forms with the same
signature may not be a quadratic form with the same signature. In
fact, we cannot put on an arbitrary $C^{\infty}$ manifold a
$C^{\infty}$ indefinite metric with arbitrarily prescribed signature.

\section*{Canonical Euclidean structures in\protect\hfil\protect\break $T^{\ast}_{a}(V)$ and
  $\overset{p}{\Lambda} T^{\ast}_{a}(V)$} 

Let $V^{N}$ be a Riemannian manifold. The positive definite quadratic
form in $T_{a}(V)$ defines a {\em canonical} isomorphism of $T_{a}(V)$
onto $T^{\ast}_{a}(V)$. For a fixed $y\in T_{a}$ and any $x\in T_{a}$,
$(x,y)$ is a linear form on $T_{a}$. We denote this linear form by
$\gamma(y)$. The\pageoriginale linear map $\gamma:y\to \gamma(y)$ is
an isomorphism of $T_{a}$ onto $T^{\ast}_{a}$. We have the relation
$$
(x,y)=\langle x,\gamma(y)\rangle
$$
connecting the Euclidean structure on $T_{a}$ and the duality between
$T_{a}$ and $T^{\ast}_{a}$.

A Euclidean structure in $T_{a}$ defines a canonical Euclidean
structure in $T^{\ast}_{a}(V)$; we simply transport the positive
definite quadratic form in $T_{a}$ to $T^{\ast}_{a}$ by means of the
canonical isomorphism $\gamma$. [Any linear map of $T^{\ast}_{a}$ to
  $T_{a}$ defines a canonical bilinear form in $T^{\ast}_{a}$; the
  canonical bilinear form in $T^{\ast}_{a}$ may also be defined as the
  bilinear form given by the map $\gamma^{-1}:T^{\ast}_{a}\to
  T_{a}$. The quadratic forms in $T_{a}$ and $T^{\ast}_{a}$ are called
  inverses of each other. If $(g_{ij})$ is the matrix of the quadratic
  form in $T_{a}$ with respect to a basis in $T_{a}$ the matrix of the
  canonical quadratic form in $T^{\ast}_{a}$ with respect to the dual
  base is the matrix $(g_{ij})^{-1}$.

A Euclidean Structure in $T_{a}$ also defines canonical Euclidean
structures in $\overset{p}{\Lambda}T_{a}$ and
$\overset{p}{\Lambda}T^{\ast}_{a}$. Let $e_{1},\ldots,e_{N}$ be an
orthonormal basis in $T_{a}$ (with respect to the quadratic form
defining the Euclidean structure). Now, a positive definite quadratic
form is uniquely determined if we specify a basis
$(x_{1},\ldots,x_{N})$ as a system of orthonormal basis; the matrix of
the quadratic form with respect to this basis is the identity
matrix. We take in $\overset{p}{\Lambda}T_{a}$ the positive definite
quadratic form for which the elements $e_{i_{1}}\wedge\ldots\wedge
e_{i_{p}}$, $i_{1}<\ldots<i_{p}$ form an orthonormal basis. This
quadratic form (\;,\;) is intrinsic; for we have
$(x_{1}\wedge\ldots\wedge x_{p},y_{1}\wedge\ldots\wedge
y_{p})$\pageoriginale
$$
\begin{vmatrix}
(x_{1},y_{1}) &\ldots & (x_{1},y_{p})\\
. & \ldots & .\\
(x_{p},y_{1}) & \ldots & (x_{p},y_{p})
\end{vmatrix}
$$
for $x_{i}$, $y_{i}\in T_{a}$ and determinant on the right is
intrinsically defined.

The canonical isomorphism $\gamma:T_{a}\to T^{\ast}_{a}$ has a
canonical extension $\gamma:\overset{p}{\Lambda}T_{a}\to
\overset{p}{\Lambda}T^{\ast}_{a}$, which is also an isomorphism. This
isomorphism defines the canonical Euclidean structure in
$\overset{p}{\Lambda}T^{\ast}_{a}$. 



