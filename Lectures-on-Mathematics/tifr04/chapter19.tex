\chapter{Lecture 19}

\section*{Some more operators}\pageoriginale

Let 
$$
\widetilde{d}=-id_{z}+id_{\ob{z}}
$$
Then
$$
\widetilde{d}=JdJ^{-1}
$$
For, if $\overset{p,q}{\omega}$ is a form of bidegree $(p,q)$,
$J\omega=i^{q-p}\omega$ and $J^{-1}\omega=i^{p-q}\omega$ so that
\begin{align*}
Jd_{z}J^{-1}\omega &= i^{q-(p+1)}i^{p-q}d_{z}\omega\\
&= -i d_{z}\omega;
\end{align*}
thus
$$
Jd_{z}J^{-1}=-id_{z}
$$
and similarly
$$
Jd_{\ob{z}}J^{-1}=id_{\ob{z}}
$$
so that
$$
\widetilde{d}=JdJ^{-1}.
$$
$\widetilde{d}$ is the transform of $d$ by the automorphism $J$. Since
$J$ and $d$ are real operators (\iec take real forms into real forms)
$\widetilde{d}$ is a real operator. Now we can decompose the operators
$d_{z}$ and $d_{\ob{z}}$ into real and imaginary parts as follows:
\begin{align*}
d_{z} &= \frac{1}{2}(d+i\widetilde{d})\\
d_{\ob{z}} &= \frac{1}{2}(d-i\widetilde{d}).
\end{align*}

We have evidently $\widetilde{d}\widetilde{d}=0$ and
$d\widetilde{d}+\widetilde{d}d=0$. 

The operators $\partial_{z}$ and $\partial \ob{z}$ are the adjoints
(with respect to the Riemannian structure) of the operators $d_{z}$
and $d_{\ob{z}}$ respectively: 
\begin{align*}
(d_{z}\alpha,\beta) &= (\alpha,\partial_{z}\beta)\\
(d_{\ob{z}}\alpha,\beta) &= (\alpha,\partial_{\ob{z}}\beta)
\end{align*}\pageoriginale
the scalar product being the global Riemannian scalar
product. $\partial_{z}$ is an operator of type $(-1,0)$ while
$\partial_{\ob{z}}$ is an operator of type $(0,-1)$. If
$\widetilde{\partial}$ is the adjoint of $\widetilde{d}$
$$
\widetilde{\partial}=i\partial_{z}-i\partial_{\ob{z}};
$$
we have
\begin{align*}
\partial_{z} &= \frac{1}{2}(\partial-i\widetilde{\partial})\\
\partial_{\ob{z}} &= \frac{1}{2}(\partial +i\widetilde{\partial}).
\end{align*}
The following relations are easily verified:
\begin{gather*}
\partial_{z}\partial_{z}=0,
\partial_{\ob{z}}\partial_{\ob{z}}=0,\partial_{z}\partial_{\ob{z}}+\partial_{\ob{z}}\partial_{z}=0\\
\partial\partial =0, \widetilde{\partial}\widetilde{\partial}=0,
\partial\widetilde{\partial}+\widetilde{\partial}\partial=0. 
\end{gather*}

We now introduce two more operators, $L$ and $\Lambda$. $L$ is simply
multiplication by $\Omega:L\omega=\Omega\wedge \omega$.

$L$ is an operator of type $(1,1)$. $\Lambda$ is the adjoint of
$L$. We have
$$
\Lambda=\ast^{-1}L\ast.
$$
For
\begin{align*}
(\Lambda\alpha,\beta) &= (\alpha,L\beta)\\
&= \int^{-1}\ast\ob{\alpha\wedge L\beta}\\
&= \int^{-1}\ast\ob{\alpha\wedge\Omega\wedge\beta}\\
&= \int\ob{\Omega\wedge\overset{-1}{\ast}\alpha\wedge \beta}\\
&=\int\ob{(L\overset{-1}{\ast}\alpha)\wedge\beta}\\
&=\int\ob{\overset{-1}{\ast}[\ast
      L\overset{-1}{\ast}\alpha]\wedge\beta}\\
&= (\ast L\overset{-1}{\ast}\alpha,\beta)\\
&= (\overset{-1}{\ast}L\ast\alpha,\beta)
\end{align*}\pageoriginale
We have used above the fact that $\Omega$ is a real form of degree
$2$.

\section*{Commutativity relations in a Kahlerian manifold}

We shall now consider the commutativity properties of the operators on
a Kahlerian manifold.

$L$ commutes with the operators $d$, $\widetilde{d}$, $d_{z}$ and
$d_{\ob{z}}$; we denote this by
$$
\sqrt[L]{d,\widetilde{d},d_{z},d_{\ob{z}}}
$$
For,
$$
d(\Omega\wedge\omega)=\Omega\wedge d\omega\text{ \  as \ } d\Omega=0;
$$
since
$$
d_{z}\Omega+d_{\ob{z}}\Omega=0
$$
and $d_{z}\Omega$ and $d_{\ob{z}}\Omega$ are forms of bidegree $(2,1)$
and $(1,2)$ we have $d_{z}\Omega=0$ and $d_{\ob{z}}\Omega=0$; it
follows that 
\begin{align*}
d_{z}(\Omega\wedge\omega) &= \Omega\wedge d_{z}\omega\\
d_{\ob{z}}(\Omega\wedge \omega) &= \Omega\wedge d_{\ob{z}}\omega
\end{align*}\pageoriginale
By taking the adjoints we find that
$$
\sqrt[\Lambda]{\partial,
  \widetilde{\partial},\partial_{z},\partial_{\ob{z}}}
$$

$L$ does not commute with $\partial$, $\widetilde{\partial}$,
$\partial_{z}$ and $\partial_{\ob{z}}$. $\Lambda$ does not commute
with $d$, $\widetilde{d}$, $d_{z}$ and $d\ob{z}$.
$$
L\partial\omega=\Omega\wedge\partial\omega,\quad \partial
L\omega=\partial(\Omega\wedge\omega)
$$
and we have no rule for the $\partial$ of a product so that it is not
possible to compare $L\partial$ and $\partial L$ directly. We have the
following formula which gives the defect of commutativity of $d$ and
$\Lambda$: writing
\begin{gather*}
[\Lambda,d]=\Lambda d-d\Lambda\text{ \  we have}\\
[\Lambda, d]=-\widetilde{\partial}
\end{gather*}
(Consequently $\Lambda$ and $d$ do not commute). This formula follows
from the following formulae:
\begin{gather*}
[\Lambda,d_{z}]=i\partial_{\ob{z}}\\
[\Lambda,d_{\ob{z}}]=-i\partial_{z}
\end{gather*}
which we shall prove in the next lecture. From these relations we have
at once 
$$
[\Lambda,\widetilde{d}]=\partial.
$$\pageoriginale
By taking the adjoints we find that
\begin{align*}
[L,\partial] &= \widetilde{d}\\ 
[L,\partial_{z}] &= \id_{\ob{z}}\\ 
[L,\partial_{\ob{z}}] &= -\id_{z}\\
[L,\widetilde{\partial}] &= -d.
\end{align*}

We shall derive some important formulae from the above formulae.
\begin{align*}
d\widetilde{\partial} &= d(d\Lambda-\Lambda d)\\
&= -d\Lambda d\\
\widetilde{\partial} d &= (d\Lambda-\Lambda d)d\\
&= d\Lambda d.
\end{align*}
Adding we find that
$$
d\widetilde{\partial}+\widetilde{\partial}d=0
$$
\iec $d$ and $\widetilde{\partial}$ anti-commute (In a Riemannian
structure $d$ and $\partial$ have no commutativity relation; in the
case of a K\"ahlerian manifold $d$ and $\widetilde{\partial}$
anti-commute). We have further the following formulae:
\begin{align*}
d\widetilde{d}+\widetilde{d}d &= 0\\
\partial \widetilde{\partial}+\widetilde{\partial}\partial &=0\\
d\widetilde{\partial} +\widetilde{\partial}d &=0\\
\partial \widetilde{d}+\widetilde{d}\partial &=0\\
d_{z}d_{\ob{z}}+d_{\ob{z}}d_{z} &=0\\
\partial_{z}\partial_{\ob{z}}+\partial_{\ob{z}}\partial_{z} &=0\\
d_{z}\partial_{\ob{z}}+\partial_{\ob{z}}d_{z} &=0\\
\partial_{z}d_{\ob{z}}+d_{\ob{z}}\partial_{z} &=0.
\end{align*}\pageoriginale
We now consider $\Delta$.
\begin{align*}
d\partial &= (d_{z}+d_{\ob{z}})(\partial_{z}+\partial_{\ob{z}})\\
&=
d_{z}\partial_{z}+d_{\ob{z}}\partial_{\ob{z}}+d_{z}\partial_{\ob{z}}+d_{\ob{z}}\partial_{z};\\
\partial d & = \partial_{z}d_{z}+\partial_{\ob{z}}d_{\ob{z}} +
\partial_{\ob{z}}d_{z}+\partial_{z}d_{\ob{z}}.  
\end{align*}
By addition,
$$
-\Delta=-\Delta_{z}-\Delta_{\ob{z}},
$$
where
\begin{align*}
& -\Delta_{z}=d_{z}\partial_{z}+\partial_{z}d_{z}\\
& -\Delta_{\ob{z}}=d_{\ob{z}}\partial_{\ob{z}}+\partial_{\ob{z}}d_{\ob{z}}.
\end{align*}
Since $\Delta_{z}$ and $\Delta_{\ob{z}}$ are pure operators (\iec
operators of type $(0,0)$), $\Delta$ is also a pure operator; in other
words, $\Delta$ does not change the bigradation. If 
$$
\widetilde{\Delta}=\widetilde{d}\widetilde{\partial}+\widetilde{\partial}\widetilde{d} 
$$
then
$$
-\widetilde{\Delta}=-\Delta_{z}-\Delta_{\ob{z}}
$$
so that
$$
\Delta=\widetilde{\Delta}.
$$
However,\pageoriginale
$$
\widetilde{\Delta}=J\Delta J^{-1}.
$$
So
$$
\Delta=J\Delta J^{-1}
$$
\iec $J$ and $\Delta$ commute.

Moreover,
\begin{align*}
d_{z}\partial_{z} &= \frac{1}{2}(d+i\widetilde{d})\cdot
\frac{1}{2}(\partial -i\widetilde{\partial})\\
&= \frac{1}{4}[d\partial
  +\widetilde{d}\widetilde{\partial}+i\widetilde{d}\partial
  -id\widetilde{\partial}]
\end{align*}
and 
$$
\partial_{z}d_{z}=\frac{1}{4}[\partial
  d+\widetilde{\partial}\widetilde{d}+i\partial
  \widetilde{d}-i\widetilde{\partial}d] 
$$
so that
$$
-\Delta_{z}=-\frac{\Delta}{4}-\frac{\widetilde{\Delta}}{4}.
$$
Similarly
$$
-\Delta_{\ob{z}}=-\frac{\Delta}{4}-\frac{\widetilde{\Delta}}{4}.
$$
Consequently
$$
\Delta_{z}=\Delta_{\ob{z}}
$$
and
$$
\boxed{\Delta=2\Delta_{z}=2\Delta_{\ob{z}}=\widetilde{\Delta}}
$$
This formula shows that $\Delta$ can be obtained in terms of $d_{z}$
and $\partial_{\ob{z}}$ alone or in terms $d_{\ob{z}}$ and
$\partial_{\ob{z}}$ alone.

We shall now see that $\Delta$ commutes with all the operators we have
introduced. Let $P^{r,s}$ be the projection of the space of forms on
the space of forms of bidegree $(r,s)$. ($P^{r,s}$ maps a form
on\pageoriginale its homogeneous component of bidegree $(r,s)$). Since
$\Delta$ is a pure operator, $\Delta$ commutes with
$P^{r,s}$. $\Delta$ commutes with $L$.
\begin{align*}
d\partial L-Ld\partial &= d\partial
L-dL\partial+dL\partial-Ld\partial\\
&= d[\partial, L]+[d,L]\partial\\
&=-d\widetilde{d}
\end{align*}
($d$ and $L$ commute as $\Omega$ is a closed form of even
degree). Similarly
$$
[\partial d,L]=\widetilde{-d}d
$$
and hence
$$
[-\Delta,L]=0.
$$
Since $\Delta$ and $L$ commute, $\Delta$ and $\Lambda$ also
commute. From the relation
$$
\Delta(\Omega\wedge\omega)=\Omega\wedge\Delta\omega
$$
we see that if a form $\omega$ is harmonic the form $\Omega\wedge
\omega$ is also harmonic. In particular the form $\Omega$ itself is
harmonic. In fact $\Omega$ is $\ast$ closed (it is already closed). To
prove this we observe that
$$
[L,\partial]=\widetilde{d}
$$
or
$$
\Omega\wedge\partial\omega-\partial(\Omega\wedge\omega)=\widetilde{d}\omega
$$
(In general we do not have a formula for the $\partial$ of a product
of two forms; however this formula gives an expression for the
$\partial$ of the product of a differential form by $\Omega$). Taking
$\omega=1$, we find that\pageoriginale $\partial\Omega=0$. Thus
$\Omega$ is closed with respect to $d$, $\partial d_{z}$,
$d_{\ob{z}}$, $\partial_{z}$ and $\partial_{\ob{z}}$.  

Since $\widetilde{d}=JdJ^{-1}$, $\Delta$ commutes with
$\widetilde{d}$; hence commutes with $\widetilde{\partial}$. Since
$\Delta$ commutes with $d$, $\widetilde{d}$, $\partial$ and
$\widetilde{\partial}$, $\Delta$ commutes with $d_{z}$, $d_{\ob{z}}$,
$\partial_{z}$ and $\partial_{\ob{z}}$. That $\Delta$ commutes with
$d_{z}$ and $d_{\ob{z}}$ is very important, as this result connects
harmonic forms with $z$ and $\ob{z}$ cohomologies.

Thus in a K\"ahlerian manifold $\Delta$ commutes with all the
operators, in particular with $d_{z}$ and $d_{\ob{z}}$.

In the case of a compact K\"ahlerian manifold we have a decomposition
of $\mathscr{D}$ (and $\mathscr{D}'$) as the direct sum of the space
of harmonic forms, the space of $z$ (or $\ob{z}$) coboundaries and the
space of $z$ (resp $\ob{z}$) star coboundaries. For we have the
decomposition
$$
I=\pi_{1}+\Delta G;
$$
replacing $\Delta$ by the new expressions we have
\begin{align*}
I &=
\pi_{1}+\widetilde{d}\widetilde{\partial}G+\widetilde{\partial}\widetilde{d}G,\\ 
I &= \pi_{1}+2d_{z}\partial_{z}G+2\partial_{z}d_{z}G,\\
I &= \pi_{1}+2d_{\ob{z}}\partial_{\ob{z}}G+2\partial_{\ob{z}}d_{\ob{z}}G.
\end{align*}
These formulae will have important consequences in connection with the
$z$ and $\ob{z}$ cohomologies.
\begin{gather*}
\Delta, G,\pi_{1}\\
\ast,d,\partial,\pi_{1}, G, \ldots,J,P^{r,s}, L,
  \Lambda,\widetilde{d}, \widetilde{\partial}, d_{z}, d_{\ob{z}},
  \partial_{z}, \partial_{\ob{z}}.  
\end{gather*}
\pageoriginale

\section*{Holomorphic forms on a K\"ahlerian manifold}

We have seen that every holomorphic form on $C^{n}$ is harmonic. In an
arbitrary Hermitian manifold it is not true that every holomorphic
form is harmonic. However in a Kahlerian manifold every holomorphic
form is harmonic. To prove this we use the fact that
$$
\Delta=2\Delta_{\ob{z}},\Delta_{\ob{z}}=d_{\ob{z}}\partial_{\ob{z}}+\partial_{\ob{z}}d_{\ob{z}}.
$$
If a form $\overset{p,0}{\omega}$ is holomorphic, then
$d_{\ob{z}}\omega=0$ and therefore
$\partial_{\ob{z}}d_{\ob{z}}\omega=0$; since $\partial_{\ob{z}}$
decreases the $\ob{z}$ degree by $1$, $\partial_{\ob{z}}\omega=0$ and
hence $d_{\ob{z}}\partial_{\ob{z}}\omega=0$. Therefore
$\Delta\omega=0$.



