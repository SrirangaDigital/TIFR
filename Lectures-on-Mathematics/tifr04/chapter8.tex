\chapter{Lecture 8}

\section*{Currents with compact support}\pageoriginale

If $T$ is a current of degree $p$ and $\varphi$ an $N-p$ form with
arbitrary support and if the supports of $T$ and $\varphi$ have
compact intersection then $\langle T,\varphi\rangle$ can be
defined. In particular if $T$ has compact support $\langle
T,\varphi\rangle$ can be defined for any $C^{\infty}N-p$ form. With
this definition of $\langle T,\varphi\rangle$ $T$ becomes a continuous
linear functional on $\overset{N-p}{\mathscr{E}}$. (\iec a current of
degree $p$ with compact support can be extended to a continuous linear
functional on $\overset{N-p}{\mathscr{E}}$). Conversely a continuous
linear functional $L$ on $\overset{N-p}{\mathscr{E}}$ defines a
current $T$ of degree $p$ by restriction to
$\overset{N-p}{\mathscr{D}}$. It can be easily shown that this current
has compact support and that
$$
\langle T,\varphi\rangle =L(\varphi)\text{ \  for every \ } \varphi\in
\mathscr{D}^{N-p} 
$$
Consequently the space of $p$-current $s$ with compact supports is
identical with the dual space $\overset{p}{\mathscr{E}'}$ of
$\overset{N-p}{\mathscr{E}}$. 

\section*{Cohomology spaces of a complex}

A complex, $E$, is a graded vector space with a differential operator
of degree $1$:
\begin{enumerate}
\renewcommand{\theenumi}{\roman{enumi}}
\renewcommand{\labelenumi}{\theenumi)}
\item $E$ is a vector space (over $R$) which is the direct sum of
  sub-spaces $E^{p}$ where $p$ runs through non-negative (sometimes,
  all) integers.

\item $E$ has a coboundary operator: there exists an endomorphism
  $d:E\to E$ such that $dE^{p}\subset E^{p+1}$ and $d^{2}=0$. 
\end{enumerate}

The\pageoriginale elements of $E^{p}$ are called elements of degree
$p$.

An element $\omega\in E$ is said to be a cocycle (or closed) if
$d\omega=0$. An element $\omega\in E$ is said to be a coboundary if
there exists an element $\widetilde{\omega}$ such that
$d\widetilde{\omega}=\omega$. Let $Z$ denote the vector space of
cocycles and $B$ the space of coboundaries. Since $d^{2}=0$, $B\subset
Z$. The space $Z/B=H$ (or $H(E)$) is defined to be the cohomology
vector space of $E$. Let $Z^{p}$ denote the space of cocycles of
degree $p$ and $B^{p}$ the space of coboundaries of degree $p$. The
space $H^{p}$ is called the $p$th cohomology vector space of $E$ and
the dimension of $H^{p}$, $b^{p}$, is called the $p$th Betti number of
the complex. We have
$$
H=\sum H^{p}
$$

If a complex $E$ is an algebra with respect to a multiplication
$(\wedge)$ satisfying the conditions.
\begin{itemize}
\item[i)] $E^{p}\wedge E^{q}\subset E^{p+q}$

\item[ii)]
  $\omega_{1}\wedge\omega_{2}=(-1)^{pq}\omega_{2}\wedge\omega_{1}$,
  $\omega_{1}\in E^{p}$, $\omega_{2}\in E^{q}$.

\item[iii)]
  $d(\omega_{1}\wedge\omega_{2})=d\omega_{1}\wedge\omega_{2}-(-1)^{p}\omega_{1}\wedge
  d\omega_{2}$, $\omega_{1}\in E^{p}$, $\omega_{2}\in E^{q}$. 
\end{itemize}
is called a differential graded algebra. (D.G.A). If $E$ is a D.G.A.,
$H(E)$ can be endowed with the structure of an algebra. For $Z$ is a
subalgebra of $E$ and $B$ is a two sided ideal of $Z$. $H(E)$ is known
as the cohomology algebra of $E$.

\section*{Cohomology on a Manifold}

Associated with a manifold $V^{N}$ we have a number of complexes and
the corresponding cohomology groups.
\begin{enumerate}
\renewcommand{\theenumi}{\roman{enumi}}
\renewcommand{\labelenumi}{\theenumi)}
\item $\mathscr{E}(V)=\sum\mathscr{E}^{p}(V)$,\pageoriginale where
  $\mathscr{E}^{p}(V)$ is the space of all $p$ forms on $V$.

\item $\widetilde{\mathscr{E}}{}^{m}(V)=\sum
  \widetilde{\mathscr{E}}^{p_{m}}$, where
  $\widetilde{\mathscr{E}}^{p_{m}}$ denotes the space of $m$-times
  differentiable $p$-forms, $\overset{p}{\omega}$, for which
  $d\overset{p}{\omega}$ is also $m$-times differentiable.

\item $\mathscr{E}'(V)=\sum\overset{p}{\mathscr{E}'}(V)$, where
  $\overset{p}{\mathscr{E}'}(V)$ is the space of currents of degree
  $p$ with compact support.

\item
  $\widetilde{\mathscr{E}}'{}^{m}(V)=\sum\overset{p}{\widetilde{\mathscr{E}'}}{}^{m}(V)$
  where $\overset{p}{\widetilde{\mathscr{E}}'}{}^{m}(V)$ is the dual of
  $\overset{N-p}{\widetilde{\mathscr{E}}}{}^{m}$ 

\item $\mathscr{D}(V)=\sum \mathscr{D}^{p}(V)$, where
  $\mathscr{D}^{p}(V)$ is the space of $p$ forms with compact support.

\item
  $\widetilde{\mathscr{D}}{}^{m}(V)=\sum\overset{p}{\widetilde{D}}{}^{m}(V)$.
  where $\overset{p}{\widetilde{D}}{}^{m'}(V)$ is the space of $m$ times
  differentiable forms, $\omega$, of degree $p$ with compact support
  for which $d\omega$ is also $m$ times differentiable.

\item $\mathscr{D}'(V)=\sum \overset{p}{\mathscr{D}'}(V)$, where
  $\overset{p}{\mathscr{D}}(V)$ is the space of currents of degree
  $p$.

\item
  $\widetilde{\mathscr{D}'}{}^{m}(V)=\overset{p}{\widetilde{D}'}{}^{m}(V)$,
  where $\overset{p}{\widetilde{\mathscr{D}'}}{}^{m}(V)$ is the dual
  space of $\overset{N-p}{\widetilde{\mathscr{D}}}{}^{m}(V)$.
\end{enumerate}

\section*{Betti-numbers of $\mathscr{E}(R^{N})$, $\mathscr{E}(S^{N})$
  and $\mathscr{E}(T^{N})$.}

We shall examine the Betti numbers of $R^{N}$, the $N$-sphere $S^{N}$,
and the $N$-Torus $T^{N}$.
\begin{enumerate}
\renewcommand{\theenumi}{\roman{enumi}}
\renewcommand{\labelenumi}{\theenumi)}
\item $R^{N}$. By Poincar\'e's theorem a closed $p$-form
  $\overset{p}{\omega}$ is a coboundary if $p\geq 1$.

So\pageoriginale
$$
b^{p}(\mathscr{E}(R^{N}))=0\text{ \  if \ } p\geq 1.
$$
If for a zero form $f$ (\iec a $C^{\infty}$ function $f$ on $R^{N}$)
$df=0$, $f$ is constant on $R^{N}$ since $R^{N}$ is connected. By
convention $B_{0}(\mathscr{E}(V))$ (the space of boundaries of degree
$0$)$= 0$.

So
$$
b^{0}(\mathscr{E}(R^{N}))=1.
$$

\item {\bf $N$-sphere $S^{N}$} (The set of points on $R^{N+1}$ defined
  by the equation $x^{2}_{1}+\cdots+x^{2}_{N+1}=1$).

Since $S^{N}$ is connected $b^{0}(\mathscr{E}(S^{N}))=1$.

It can be shown that
$$
b^{p}(\mathscr{E}(S^{N}))=0\text{ \  for \ }1\leq p\leq N-1
$$
and
$$
b^{N}(\mathscr{E}(S^{N}))=1.
$$

\item {\bf Torus $T^{N}$.} ($T^{N}=R^{N}/Z^{N}$, where $Z^{N}$ is the
  group of the integral lattice points in $R^{N}$).

It can be proved that
$$
b^{p}(\mathscr{E}(T^{N}))=\binom{N}{p}.
$$
In this case we can give the complete structure of the cohomology
ring. The differential forms $dx_{1},\ldots,dx_{N}$ on $R^{N}$ define
$N$-differential forms on $T^{N}$, which we still denote by
$dx_{1},\ldots,dx_{N}$. It turns out that the classes of the forms
$$
dx_{i_{1}}\wedge\ldots\wedge dx_{i_{p}},\quad i_{1}<\ldots<i_{p}
$$\pageoriginale
generate the $p$th cohomology group of $T^{N}$.
\end{enumerate}

We have seen that $b^{0}(\mathscr{E}(R^{N}))=1$ and
$b^{p}(\mathscr{E}(R^{N}))=0$, $p\geq 1$. The Betti numbers of
$\mathscr{D}(R^{N})$ are not the same as those of
$\mathscr{E}(R^{N})$. There is a theorem, which we may call
Poincar\'e's theorem for compact supports, which asserts that in
$R^{N}$ a closed $p$-form with compact support is the coboundary of a
$p-1$ form with compact support if $p\leq N-1$ and an $N$-form
$\overset{N}{\omega}$ with compact support is the coboundary of an
$N-1$ form with compact support if and only if
$$
\int\limits_{R^{N}}\omega=0
$$
(The integral is defined since $\overset{N}{\omega}$ has compact
support). It follows that
$$
b^{p}(\mathscr{D}(R^{N}))=0,\quad 0\leq p\leq N-1
$$
and it can be shown that
$$
b^{N}(\mathscr{D}(R^{N}))=1.
$$

This example shows that there are at least two different kinds of
cohomologies on a manifold - the cohomology with compact supports and
cohomology with arbitrary supports. One part of de Rham's theorem
asserts that, of the cohomologies given by
$$
\mathscr{E},\widetilde{\mathscr{E}}^{m},
        {\widetilde{\mathscr{D}'}^{m}}, \mathscr{D}'; \mathscr{D},
\widetilde{\mathscr{D}}^{m},\widetilde{\mathscr{E}'}^{m},\mathscr{E},
$$
only two cohomologies are distinct.


