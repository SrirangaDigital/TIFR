\chapter{Lecture 20}\label{chap20:lec20}
  
\begin{defi*}%defi 0
If\pageoriginale $\Omega$ is a directed set, a {\em direct system of double
  complexes} $\bigg\{ K_\lambda , \phi_{\mu \lambda} \bigg\}_{\lambda
  , \mu \in \Omega}$ is a system of double complexes $K_\lambda$ and
maps 
$$
\phi_{\mu \lambda} : K_\lambda \rightarrow K_\mu \quad{(\lambda <
  \mu)}, 
$$
such that 
\begin{enumerate}[(i)]
\item $\phi_{\lambda \lambda}$ is the identity, 

\item $\phi_{\nu \mu} \phi_{\mu \lambda}$ is {\em homotopic} to
  $\phi_{\nu \lambda}$ for $\lambda < \mu < \nu$. 
\end{enumerate}
  \end{defi*}  
  
\textit{If $\bigg\{ K_{\lambda}, \phi_{\mu \lambda} \bigg\}$ is a
  direct system of double complexes, there are unique\-ly determined
  direct limits:} 
  \begin{enumerate}[(i)]
\item $H^{p,q}_{12}(K ) = \text { direct limit } \{
  H^{p,q}_{12}(K_\lambda), \phi^*_{\mu \lambda} \}$, 

\item $H^{p,q}_{21}(K ) = \text {direct limit} \{
  H^{p,q}_{21}(K_\lambda), \phi^*_{\mu \lambda} \}$, 

\item $H^n (K) = \text {direct limit} \{ H^n (K_\lambda), \phi^*_{\mu
  \lambda} \}$. 
  \end{enumerate}  
  
  \begin{proof}
(i) The system $\bigg\{ H^{p,q}_{12}(K_\lambda), \phi^*_{\mu
      \lambda} \bigg\}$ is a direct system, as  
\begin{enumerate}[(a)]
\item $\phi^\ast_{\lambda \lambda}$ is the identity, since $\phi_{\lambda
  \lambda}$ is the identity, 

\item $\phi^\ast_{\nu \mu} \phi^\ast_{\mu \lambda} = (\phi_{\nu \mu}
  \phi_{\mu \lambda})^\ast = \phi^\ast_{\nu \lambda}$, since homotopic maps
  induce the same homomorphism on the cohomology groups. 
\end{enumerate}
  \end{proof}  
  
  The proofs of (ii) and (iii) are carried out in a similar
  manner. 
  
  \begin{defi*}%defi 0
We say that a system $\{ K_\lambda \}$ of double complexes is bounded
{\em on the right} if there is some integer $m$, such that
$K^{p,q}_\lambda = 0$\pageoriginale whenever $q > m$ (for all $p$ and
$\lambda$). The 
system is said {\em to be bounded on the left} (resp. {\em above,
  below}) if there is an integer $m$ such that $K^{p,q}_\lambda = 0$
whenever $q < m$ (resp. $p < m, p > m$). 
  \end{defi*}  

  \begin{proposition}\label{chap20:prop15}%prop 15
If $\bigg\{K_\lambda , \phi_{\mu \lambda} \bigg\}$  is a direct
  system of double complexes which is bounded above or on the right
  and if $H^{p,q}_{12}(K) = 0$ for all $p$ and $q$, then $H^n(K)
  = 0$ for all $n$. 
  \end{proposition}  

  \begin{proof}
	Let $\alpha \in H^n(K)$, let $\alpha_\lambda$ be its
        representative in some $H^n(K_\lambda)$ and let $a_\lambda \in
        Z^n(K_\lambda)$ represent the class $\alpha_{\lambda}$. Let $a_{\lambda}
        = a^{p,q}_{\lambda} + a^{p-1, q+1}_{\lambda} + \ldots$, where
        $p+q = n$ and the sum terminates with $a^{m,n-m}_{\lambda}$
        (resp. $a^{n-m, m}_{\lambda}$). 
  \end{proof}  
  
  Since $a_\lambda \in Z^n (K_{\lambda})$,  $d a_\lambda = (d_1 + d_2) 
  a_{\lambda} = 0$, i.e., 
  $$
  d a_{\lambda} = d_1 a^{p,q}_{\lambda} + (d_2 a^{p,q}_{\lambda} + d_1
  a^{p-1, q+1}_{\lambda}) + \cdots , 
  $$
  and the sum being direct, we have $d_1 a^{p,q}_{\lambda} = 0$ and
  $d_2 a^{p,q}_{\lambda} + d_1 a^{p-1 , q+1}_{\lambda} = 0$. Thus
  $a^{p,q}_{\lambda} \in Z^{p,q}_1(K_\lambda)$ and $d_2
  a^{p,q}_{\lambda} \in B^{p,q+1}_{1} (K_\lambda)$. Therefore,
  $a^{p,q}_{\lambda}$ represents an element of
  $Z^{p,q}_{12}(K_\lambda)$. Since $H^{p,q}_{12} = 0$, there is some
  $\mu > \lambda$ such that $a^{p,q}_{\mu} = \phi_{\mu \lambda}
  a^{p,q}_{\lambda}$ represents an element of
  $B^{p,q}_{12}(K_\mu)$. Thus, for some $b \in Z^{p,q-1}_{1}(K_\mu),
  a^{p,q}_{\mu} - d_2 b \in B^{p,q}_1 (K_\mu)$, and hence
  $a^{p,q}_{\mu} = d_2 b + d_1 c$ for some $c \in K^{p-1,q}_{\mu}$. 
  \begin{align*}
\text{ Let } a_{\mu} & = \phi_{\mu \lambda} a_{\lambda} \text{ and
  let } \\ 
e_{\mu} & = a_{\mu }- d (b + c) \\
& = a_{\mu }- d_1 b - d_2 b-d_1 c-d_2 c \\
& = a_{\mu } - a^{p,q}_{\mu} - d_2 c  \quad (\text { since } d_1 b = 0)\\
 &= (a^{p-1,q+1}_\mu - d_2 c) + a^{p-2,q+1}_\mu  + \cdots \\
 & = e^{p-1, q+1}_\mu + a^{p-2,q+2}_\mu  + \cdots
 \end{align*}  
 where\pageoriginale $e^{p-1,q+1}_\mu  = a^{p-1,q+1}_\mu - d_2 c$. Then since
 $e_\mu$ and $a_\mu$ represent the same class $\alpha_{\mu} \in H^n
 (K_\mu)$, they represent the same element $\alpha \in H^n(K)$. The
 reason for choosing this representative $e_\mu$ is the fact that its
 $(p, q)$-th component is zero. Continuing thus, after a finite
 number $k$ of steps for a suitable $\rho$, $\alpha$ is represented by
 $e_\rho = -d_2 c'$, consisting of a single term in $K^{p-k,
   q+k}_\rho$. Continuing the construction still further, since the
 system of double complexes is bounded above or to the right, after a
 finite number of steps, we obtain a representative $e_\nu = - d_2 c''
 = 0$, i.e., $\alpha$ is represented by $0 \in Z^n(K_\nu)$ for a
 suitable $\nu$. Hence $\alpha = 0$. 

\medskip
\noindent{\textbf{Proposition 15-a.}}
\textit{If $\bigg\{ K_\lambda , \phi_{\mu \lambda} \bigg\}$ is a direct
  system of double complexes which is bounded below or on the left,
  and if $H^{p,q}_{21} (K) = 0$ for all $p$ and $q$ then $H^n(K)= 0$ for
  all $n$.}
 
 \begin{proof}
This is carried out exactly as in Proposition \ref{chap20:prop15},
except that we 
eliminate the component of $a_\lambda$ with highest second degree $q$
instead of the one with highest first degree, and $-d_1 c$ plays the
role of $-d_2 c$.	 
 \end{proof} 

 \begin{proposition}\label{chap20:prop16}%prop 16
 If $\bigg\{ K_\lambda , \phi_{\mu \lambda} \bigg\}$  is a direct
  system of double complexes which is bounded above or on the right
  and if $H^{p,q}_{12} (K) = 0$ except (at most) for $p=0$,  then
  there exist isomorphisms $\theta : H^q(K) \rightarrow
  H^{o,q}_{12}(K)$ for all $q$. 
 
If another direct system $\bigg\{ K_lambda , \phi_{\mu
   \lambda} \bigg\}$ is bounded above or on the right with
 $H^{p,q}_{12}(K') = 0$ except for $p=0$ and if
   $h_{\lambda}: K'_{\lambda} \to K_{\lambda}$ are maps with
 $ h_\mu \phi'_{\mu \lambda} = \phi_{\mu \lambda} h_{\lambda}$, then
there are induced\pageoriginale homomorphisms  
 $$
 h^\ast : H^q (K') \rightarrow H^q(K) \text{ and } h^\ast : H^{o,q}_{12}(K')
 \rightarrow H^{o,q}_{12}(K) 
 $$ 
commuting with the isomorphisms $\theta$. 
\[
\xymatrix{
  H^q(K') \ar[r]^{\theta}\ar[d]_{h^{\ast}} & H^{0,q}_{12}(K')
  \ar[d]^{h^{\ast}}\\
H^q(K) \ar[r]^{\theta} & H^{0,q}_{12}(K).
}
\]
 \end{proposition}


 \begin{proof}
Let $L_\lambda$ be the subcomplex
\[
\xymatrix{
L_{\lambda} :  & \ar[d] & \ar[d] & \\
\qquad  \ldots \ar[r] & K^{-1,q-1}_{\lambda} \ar[d] \ar[r] &
K^{-1,q}_{\lambda} \ar[r] 
\ar[d] & \ldots\\
\qquad  \ldots \ar[r] & Z^{0,q-1}_1 \ar[r]\ar[d] & Z^{0,q}_1\ar[r]\ar[d] &
\ldots\\
 & 0 & 0 & 
}
\]
 of $K_{\lambda}$, with $L^{p,q}_{\lambda} = K^{p,q}_{\lambda}$ for
 $p < 0; L^{p,q}_{\lambda} = 0$ for $p >0$ and $L^{o,q}_{\lambda} =
 Z^{o,q}_1 (K_\lambda)$. Since $L_{\lambda}$ is stable under $d_1$ and
 $d_2$, it is a subcomplex of $K_{\lambda}$. 

 Let $M_\lambda$ be the subcomplex 
\[
\xymatrix{
M_{\lambda} : &  & \ar[d] & \ar[d] & \\
& \ldots \ar[r] & K^{-1,q-1}_{\lambda} \ar[d] \ar[r] &
K^{-1,q}_{\lambda} \ar[r] 
\ar[d] & \ldots\\
&  \ldots \ar[r] & B^{0,q-1}_1 \ar[r]\ar[d] & B^{0,q}_1\ar[r]\ar[d] &
\ldots\\
 & & 0 & 0 & 
}
\]
 of\pageoriginale $L_{\lambda}$, with $M^{p,q}_{\lambda} = L^{p,q}_{\lambda} =
 K^{p,q}_{\lambda}$ for $p < 0; M^{p,q}_{\lambda} = L^{p,q}_{\lambda}
 = 0$ for $p>0$ and $M^{o,q}_{\lambda} =
 B^{o,q}_{\lambda}(K_{\lambda})$. Since $M_{\lambda}$ is stable $d_1$
 and $d_2$, it is a subcomplex of $L_{\lambda}$. 
 \end{proof}

 Since $h_{\lambda} : K'_{\lambda} \rightarrow K_{\lambda}$ commutes
 with $d_1$ and $d_2$, we have $h_{\lambda} : L'_{\lambda} \rightarrow
 L_{\lambda}$ and $h_{\lambda} : M'_{\lambda} \rightarrow
 M_{\lambda}$. Thus there are induced maps $h_{\lambda} : K'_{\lambda}
 / L'_{\lambda} \rightarrow K_{\lambda}/L_{\lambda}$ and $h_{\lambda} :
 L'_{\lambda}/M'_{\lambda} \rightarrow L_{\lambda}/M_{\lambda}$ which
 commute with $i$ and $j$ in the exact sequences 
\[
\xymatrix{
0\ar[r] & L'_{\lambda} \ar[r]^i\ar[d]_{h_{\lambda}} & K'_{\lambda}
\ar[r]^j\ar[d]_{h_{\lambda}} & K'_{\lambda} / L'_{\lambda} \ar[r]
\ar[d]_{h_{\lambda}} & 0\\
0 \ar[r] & L_{\lambda} \ar[r]^i & K_{\lambda} \ar[r]^j &
K_{\lambda}/L_{\lambda} \ar[r] & 0,
}
\]
and
\[
\xymatrix{
0\ar[r] & M'_{\lambda} \ar[r]^i\ar[d]_{h_{\lambda}} & L'_{\lambda}
\ar[r]^j\ar[d]_{h_{\lambda}} & L'_{\lambda} / M'_{\lambda} \ar[r]
\ar[d]_{h_{\lambda}} & 0\\
0 \ar[r] & M_{\lambda} \ar[r]^i & M_{\lambda} \ar[r]^j &
L_{\lambda}/M_{\lambda} \ar[r] & 0.
}
\]

 Hence $h^\ast_{\lambda}$ commutes with $d^\ast$, $i^*$ and $j^*$ in the
 exact cohomology sequences: 
{\fontsize{9}{11}\selectfont
\[
\xymatrix{
\ldots \ar[r] & H^{n-1} (K'_{\lambda}/L'_{\lambda})
\ar[r]^{d^{\ast}}\ar[d]_{h^{\ast}_{\lambda}} &
H^n(L'_{\lambda})\ar[r]^{i^{\ast}}\ar[d]_{h^{\ast}_{\lambda}} &
H^{n}(K'_{\lambda}) \ar[r]^{j^{\ast}}\ar[d]_{h^{\ast}_\lambda} &
H^n(K'_{\lambda}/L'_{\lambda}) \ar[r] \ar[d]_{h^{\ast}_\lambda} &  \\
\ldots \ar[r] & H^{n-1}(K_{\lambda}/L_{\lambda}) \ar[r]^{d^{\ast}} &
H^n(L_{\lambda}) \ar[r]^{i^{\ast}} & H^n(K_{\lambda})
\ar[r]^{j^{\ast}} & H^n (K_{\lambda}/L_{\lambda})& 
}
\]
and
\[
\xymatrix{
\ldots \ar[r] & H^{n} (M'_{\lambda})
\ar[r]^{i^{\ast}}\ar[d]_{h^{\ast}_{\lambda}} &
H^n(L'_{\lambda})\ar[r]^{j^{\ast}}\ar[d]_{h^{\ast}_{\lambda}} &
H^{n}(L'_{\lambda}/M'_{\lambda}) \ar[r]^{d^{\ast}}\ar[d]_{h^{\ast}_\lambda} &
H^{n+1}(M'_{\lambda}) \ar[r] \ar[d]_{h^{\ast}_\lambda} & \ldots \\
\ldots \ar[r] & H^n(M_{\lambda}) \ar[r]^{i^{\ast}} &
H^n(L_{\lambda}) \ar[r]^{j^{\ast}} & H^n(L_{\lambda}/M_{\lambda})
\ar[r]^{d^{\ast}} & H^{n+1} (M_{\lambda}) \ar[r]& \ldots 
}
\]}\relax

  In\pageoriginale the direct limit, we have the following commutative
  diagram where each row is exact.  
{\fontsize{8}{10}\selectfont
\[
\xymatrix{
(A_1) \cdots \ar[r] & H^{n-1}(K'/L')
  \ar[r]^{d^{\ast}}\ar[d]_{h^{\ast}} &
  H^n(L')\ar[r]^{i^{\ast}}\ar[d]^{h^{\ast}} & H^n(K')
  \ar[r]^{j^{\ast}}\ar[d]_{h^{\ast}} & H^n(K'/L')
  \ar[r]\ar[d]_{h^{\ast}} & \ldots\\
(A_2) \cdots \ar[r] & H^{n-1}(K/L) \ar[r]^{d^{\ast}} & H^n(L)
  \ar[r]^{i^{\ast}} & H^n(K) \ar[r]^{j^{\ast}} & H^n(K/L) \ar[r] & \ldots
}
\]}\relax
and
{\fontsize{8}{10}\selectfont
\[
\xymatrix{
(B_1) \cdots \ar[r] & H^{n}(M')
  \ar[r]^{i^{\ast}}\ar[d]_{h^{\ast}} &
  H^n(L')\ar[r]^{j^{\ast}}\ar[d]_{h^{\ast}} & H^n(L'/M')
  \ar[r]^{d^{\ast}}\ar[d]_{h^{\ast}} & H^{n+1}(M')
  \ar[r]\ar[d]_{h^{\ast}} & \ldots\\
(B_2) \cdots \ar[r] & H^{n}(M) \ar[r]^{i^{\ast}} & H^n(L)
  \ar[r]^{j^{\ast}} & H^n(L/M) \ar[r]^{d^{\ast}} & H^{n+1}(M) \ar[r] &
  \ldots 
}
\]}\relax
  
   The quotient double complex $K_{\lambda}/L_{\lambda}$ is 
\[
\xymatrix{
K_{\lambda}/L_{\lambda}: &  & 0\ar[d] & 0 \ar[d]& \\
& \ldots \ar[r] & K^{0,q-1}_{\lambda} / Z^{0,q-1}_1 \ar[r]\ar[d] &
K^{0,q}/Z^{0,q}_1 \ar[r]\ar[d] & \ldots\\
& \ldots \ar[r] & K^{1,q-1}_{\lambda} \ar[r]\ar[d] & K^{1,q}_{\lambda}
\ar[r]\ar[d] & \ldots\\
& & & & & 
}
\]

and since the sequence $0 \rightarrow K^{o,q}_{\lambda} / Z^{o,q}_1
\rightarrow K^{1,q}_{\lambda}$ is exact, we have  $H_{1}(K_{\lambda}/L_{\lambda})$: 
%{\fontsize{7}{9}\selectfont
\[
\xymatrix{
 & 0\ar[d] & 0 \ar[d]& \\
\ldots \ar[r] & H^{1,q-1}_1(K_{\lambda})  \ar[r]\ar[d] &
H^{1,q}_1(K_{\lambda}) \ar[r]\ar[d] & \ldots\\
\ldots \ar[r] & H^{2,q-1}_1(K_{\lambda}) \ar[r]\ar[d] & H^{2,q}_1(K_{\lambda})
\ar[r]\ar[d] & \ldots\\
& & & 
}
\]\pageoriginale

Thus $H^{p,q}_{12} (K_{\lambda}/ L_{\lambda})  = 0$ for $p \leqq 0$,
and is equal to $H^{p,q}_{12}(K_{\lambda})$ for $p>0$, hence
$H^{p,q}_{12}(K/L) = $ direct limit $\bigg\{ H^{p,q}_{12}(K_{\lambda}
/ L_{\lambda}) \bigg\} = 0$ for $p \leqq 0$, and by hypothesis is also
zero for $p>0$, hence is zero for all pairs $(p,q)$. Since
$K_{\lambda}/ L_{\lambda}$ is bounded above or to the right, by
Proposition \ref{chap20:prop15}, we have $H^n (K/L) = 0$ for all
$n$. Thus, in the 
sequence $(A_2)$, we see that $i^* : H^n(L) \rightarrow H^n (K)$ is an
isomorphism.  
   
   Again, since the sequence $K^{-1,q}_\lambda \rightarrow B^{o,q}_1
   \rightarrow 0$ is exact, we have for 
\[
\xymatrix{
H_1(M_{\lambda}): &  & \ar[d] & \ar[d] & \\
& \ldots \ar[r] & H^{-1,q-1}_1 (K_{\lambda}) \ar[r] \ar[d]& H^{-1,q}_1
(K_{\lambda}) \ar[r] \ar[d]& \ldots\\
& & 0 & 0 & 
}
\]   

   Thus, $H^{p,q}_{12}(M_{\lambda}) = 0$ for $p > 0$ and is equal to
   $H^{p,q}_{12}(K_{\lambda})$ for $p <0$, hence $H^{p,q}_{12}(M) = $
   direct limit $\bigg\{ H^{p,q}_{12}(M_{\lambda}) \bigg\}$ is equal
   to zero for $p \ge 0$, and by hypothesis, is also zero for $p < 0$,
   hence is zero for all pairs $(p,q)$. As before, the conditions of
   Proposition \ref{chap20:prop15} being\pageoriginale satisfied, we
   have $H^n(M) = 0$ 
   for all $n$. Thus, in the sequence $(B_2)$, we see that $j^* :
   H^n(L) \rightarrow H^n (L/M)$ is an isomorphism.  
   
 The quotient double complex $L_{\lambda}/ M_{\lambda}$ is given by  
\[
\xymatrix{
L_{\lambda}/M_{\lambda}: &  & 0 \ar[d] & 0\ar[d] & \\
& \ldots \ar[r] & H^{0,q-1}_1 (K_{\lambda}) \ar[r] \ar[d]& H^{0,q}_1
(K_{\lambda}) \ar[r] \ar[d]& \ldots\\
& & 0 & 0 & 
}
\]   

 Thus $(L_{\lambda}/ M_{\lambda})^q = H^{o,q}_1 (K_{\lambda})$ and $d
 = d_2 :H^{o,q-1}_1 (K_{\lambda}) \rightarrow H^{o,q}_1
 (K_{\lambda})$. Hence $H^q(L_{\lambda}/M_{\lambda}) = H^{o,q}_{12}
 (K_{\lambda})$; similarly $H^q(L'_{\lambda}/ M'_{\lambda}) =
 H^{o,q}_1 (K'_{\lambda})$. Furthermore, in the limit we have
 $H^q(L/M) = H^{o,q}_{12} (K)$ and $H^q(L' /M') = H^{o,q}_1
 (K')$. 
 
 From the sequences $(A_1)$, $(A_2)$, $(B_1)$, $(B_2)$, we have the
 commutative diagram : 
\[
\xymatrix{
H^q(K')\ar[d]_{h^{\ast}} & H^q(L') \ar[l]_{i^{\ast}} \ar[r]^{j^{\ast}}
\ar[d]_{h^{\ast}} & H^q(L'/M') \ar[d]_{h^{\ast}} & = &
H^{0,q}_{12}(K')\ar[d]_{h^{\ast}} \\
H^q(K) & H^q(L) \ar[l]_{i^{\ast}} \ar[r]^{j^{\ast}} & H^q(L/M) & = &
H^{0,q}_{12} (K).
}
\]

 Then there is an isomorphism	
 $$
 \theta : H^q(K) \rightarrow H^{o,q}_{12} (K),  
 $$
 where $\theta = j^* (i^*)^{-1} : H^q (K) \rightarrow H^q(L)
 \rightarrow H^q(L/M) = H^{o,q}_{12} (K)$, $\theta $ \break
 being\pageoriginale an isomorphism since we have proved that each of
 $i^*$ and $j^*$ is an isomorphism. 
 
 Further, form $(I)$, we have obviously commutativity in the following
 diagram : 
\[
\xymatrix{
H^q(K') \ar[r]^{\theta}\ar[d]_{h^{\ast}} & H^{0,q}_{12}(K')
\ar[d]^{h^{\ast}}\\
H^q(K) \ar[r]^{\theta} & H^{0,q}_{12}(K).
}
\]

\medskip
\noindent{\textbf{Proposition 16-a.}}
\textit{If $\bigg\{K_{\lambda}, \phi_{\mu \lambda} \bigg\}$ is a direct
  system of double complexes which is bounded below or on the left and
  if  $H^{p,q}_{21} (K) = 0$  except for $q = 0$,  then there exist
  isomorphisms $\theta : H^p(K) \rightarrow H^{p,o}_{21} (K)$ for all
$p$.}

\textit{If another direct system $\bigg\{ K'_{\lambda} , \phi'_{\mu \lambda}
\bigg\}$ is bounded below or on the left with  $H^{p,q}_{21}
  (K') = 0$ except for $q=0$, and if $h_{\lambda} : K'_{\lambda}
  \rightarrow K_{\lambda}$ are maps with $h_{\mu \phi_{\mu \lambda}} =
  \phi_{\mu \lambda} h_{\lambda}$, then there are induced
  homomorphisms 
 $$
 h^\ast : H^p(K') \rightarrow H^p (K) \text{ and } h^\ast :
 H^{p,o}_{21} (K')  \rightarrow H^{p,o}_{21} (K)  
 $$
which commute with $\theta$.}
 
 \begin{remark*}%rema 0
In particular, all the propositions proved in this lecture are true
for a double complex $K = \bigg\{K^{p,q} \bigg\}$ satisfying the
conditions stated in the propositions. We have only to replace the
$\phi_{\mu \lambda}$ by the identity map $K \rightarrow K$. 
 \end{remark*} 

 \begin{exam}%exam 29
Let\pageoriginale
\begin{align*}
K^{p,q} & = Z \text{ (ring of integers) if  } q \geqq 0 \text{ and } p = - q
\text{ or  } - q-1, \\ 
& = 0 \text{ otherwise. }
\end{align*}

 For $q \geqq 0$, let $d_1 : K^{-q-1, q} \rightarrow K^{-q,q}$ 
 and 
 $$
 d^2 : K^{-q-1,q} \rightarrow K^{-q-1, q+1} 
 $$
 be the identity isomorphisms of $Z$ onto itself. The other 
 homomorphisms are all the trivial homomorphisms. Then $K = \bigg\{
 K^{p,q} \bigg\}$ is a double complex with 
 \begin{align*}
H^{p,q}_{21}(K)  = H^{p,q}_{2}(K) & = Z  \text { if } (p,q) = (0,0),\\ 
& = 0 \text{ otherwise };\\
H^{p,q}_{12}(K)  = H^{p,q}_{1}(K) &= 0 \text{ for all } (p,q),
 \end{align*} 
 and 
 \begin{align*}
H^n(K) & = Z \text { if } n= 0, \\
& = o \text { otherwise }.
 \end{align*} 
 
 This double complex is bounded below and on the left, but is
 unbounded above and on the right.	 
 \end{exam}


