\chapter{Lecture 31}\label{chap31:lec31}%Lect 31

\textit{If\pageoriginale $f : \mathscr{S} \to \mathscr{S}''$ is an epimorphism,
  $\phi : \sum^k_{i=1} \mathfrak{a}_U \to \mathscr{S}''_U$ a homomorphism, and if
  $x$ is a point of the open set $U$, then there exists an open set
  $V$ with $x \in V \subset U$ and a homomorphism} 
$$
\psi : \sum^k_{i=1} a_V \to \mathscr{S}_V \textit{ such that } f \psi
= \phi | \sum^k_{i=1} a_V. 
$$

\begin{proof}
Since $f : \mathscr{S} \to \mathscr{S}''$ is an epimorphism, each of
the sections $\theta_i = \phi h^i 1: U \to \mathscr{S}''$, $i = 1,
\ldots ,k$, is locally the image of a section in $\mathscr{S}$. Hence
there is an open set $V_i$ with $x \in V_i \subset U$ and a section
$\eta_i : V_i \to \mathscr{S}$ such that $f \eta _i \ \theta_i | V_i$.  
Let $\psi_i : \mathfrak{a}_{V_i} \to \mathscr{S}_{V_i}$ be the homomorphism
defined by $ \psi_i (a) = a \cdot \eta_i \pi (a)$ These homomorphisms
$\psi_i$ induce a homomorphism  
$$
\psi : \sum^k_{i=1} \mathfrak{a}_V \to \mathscr{S}_V 
$$
where $V = \bigcap^k_{i=1} V_i$. Then, if $\sum^k_{i=1}a_i \in
\sum^k_{i=1} \mathfrak{a}_V$, we have 
  \begin{align*}
f \psi (\sum^k_{i=1}a_i) & = f(\sum^k_{i=1}a_i \cdot \eta_i \pi
(a_i)) = \sum^k_{i=1}a_i \cdot  \theta_i  \cdot \pi (a_i)\\
& = \sum^k_{i=1}a_i \cdot \phi h^i 1 \pi
(a_i) = \sum^k_{i=1} \phi h^i a_i\\
& = \phi (\sum^k_{i=1 }a_i),
 \end{align*}  
 i.e.,  $f \psi = \phi | \sum ^k_{i=1} \mathfrak{a}_V$.
\end{proof}

 \noindent
(5) \qquad \textit{If\pageoriginale $0 \to \mathscr{S}'
   \xrightarrow{f} \mathscr{S} \xrightarrow{g} \mathscr{S}''\to 0 $
   is an exact sequence of 
   sheaves of $\mathfrak{a}$- modules such that $\mathscr{S}'$ has
   property $(a)$ and $\mathscr{S}$ has property $(b)$, then
   $\mathscr{S}''$ has property $(b)$.} 
 
 \begin{proof}
Let $U$ be an open set and let $\phi : \sum^{k}_{i=1} \mathfrak{a}_U \to
\mathscr{S}''_U$  be a homomorphism. Since $g : \mathscr{S} \to
\mathscr{S}''$ is an epimorphism, by the result proved above, if $x
\in U$, there is an open set $V$ with $x \in V \subset U$ and a
homomorphism $\psi : \sum^{k}_{i=1} \mathfrak{a}_V \to \mathscr{S}_V$
such that $g \psi =\phi | \sum^{k}_{i=1} \mathfrak{a}_V$. Since $\mathscr{S}'$
has property $(a)$, there is an open set $W$ with $x \in W \subset V$
and an epimorphism $\eta : \sum^{1}_{i=k+1} \mathfrak{a}_W \to \mathscr{S}'_W
$. Then $\psi $ and $f \eta$ induce a homomorphism $\theta :
\sum^{1}_{i=1} \mathfrak{a}_W \to \mathscr{S}_W$.  
 \end{proof} 
 
 We also have the projection homomorphisms $p : \sum^{l}_{i=1}
 \mathfrak{a}_W \to \sum^{k}_{i=1}\break \mathfrak{a}_W$ and $p' :
 \sum^1_{i=1} \mathfrak{a}_W \to \sum^{l}_{i=k+1} \mathfrak{a}_W$ such that
 $\theta = \psi p + f \eta p'$.  
\[
\xymatrix{
 & \sum^l_{i=k+1}  \mathfrak{a}_W\ar[d]_{\eta}\ar[dr]^{f\eta}&
\sum^l_{i=1} \mathfrak{a}_W \ar[r]^p\ar[d]_{\theta} \ar[l]_{p'} &
\sum^k_{i=1} \mathfrak{a}_W \ar[d]^{\phi} \ar[dl]_{\psi}\\
0 \ar[r]& \mathscr{S'}_W \ar[r]^f & \mathscr{S}_W \ar[r]^g & \mathscr{S}''_W
\ar[r] & 0
}
\]

The second square forms a commutative diagram since $g \theta = g \psi
p+ g f \eta p' = g \psi p = \phi p$ and hence $p$ maps $\ker \theta$
into $\ker \phi$. Actually, $p$ maps $\ker \theta $ \textit{onto}
$\ker \phi$, for, if $a \in \ker \phi $, then $g \psi a = \phi a =0$
and by exactness there exists $b \in \mathscr{S}'_W$ such that $f b =
\psi a$. Since $\eta $ is an epimorphism, there exists $c \in
\sum^{l}_{i=k+1} \mathfrak{a}_W$ such that $\eta c= -b$. Then $\theta
(a+c)= \psi 
a+f \eta c=0$ and $p(a+c)=a$. Thus $p$ maps $\ker \phi$. Since
$\mathscr{S}$ has $(b) \ker \theta$\pageoriginale has $(a)$, and since $p |\ker
\theta : \ker \theta \to \ker \phi$ is an epimorphism, $\ker \phi $
has $(a)$. Hence $\mathscr{S}''$ has $(b)$. 

The corresponding statement, with $(a_1)$ and $(b_1)$ in place of
$(a)$ and $(b)$, is not true as the following example shows. 

\begin{exam}\label{chap31:exam33}%Exmp 33
Let $X$ be the union of the sequence of circles $C_n = \bigg \{(x,y) :
x^2 + y^2 = x/n\bigg \}$, $n=1,2,\ldots $. let each stalk of
$\mathfrak{a}$ be the ring $Z [x_1, x_2,\ldots]$ of polynomials in
infinitely many variables, with coefficients in $Z$, and let
$\mathfrak{a}$ be constant except that, on going around the circle
$C_n, x_n$ and -$x_n$ interchange.   
\end{exam}

\noindent
More precisely, let $T_n$ be the automorphism of the ring $Z[x_1,
  x_2$,. which interchanges $x_n$ and $-x_n$. If $U$ is open in $X$
  and $U \subset C_n$, let $A_U$ be the ring of functions $f$ defined
  on $U$ and with values in $Z [x_1, x_2, \ldots]$ such that $f$ is
  constant on each component of $U$ not containing
  $\left(\dfrac{1}{n}, 0 \right)$
  and, if a component $W$ of $U$ contains $\left(\dfrac{1}{n},
  0\right)$, $f(x,y) 
  =f\left(\dfrac{1}{n}, 0 \right)$ for $(x,y) \in W$ and $y < 0$ and
  $f(x,y) =T_n 
  f\left(\dfrac{1}{n}, 0 \right)$ for $(x,y) \in W$ and $y > 0$. If $U$ is not
  contained in any $C_n$ let $A_U$ be the ring of functions, constant
  on each component of $U$, with values in 
$$
Z [x_{n_1}, x_{n_2}, \ldots] \subset Z [x_1, x_2, \ldots] 
$$
where $n_1, n_2, \ldots$, are those values of $n$ for which
$\left(\dfrac{1}{n},0 \right) \notin U$. 

If $V \subset U$ let $\rho_{_{VU}} : A_U \to A_V$ be given by
$\rho_{_{VU}} f= f|V$. Let $\mathfrak{a}$ be the sheaf of rings
determined by the presheaf $\{A_U, \rho_{_{VU}}\}$. 

Let\pageoriginale $\mathscr{I}$ be the sheaf of ideals formed by polynomials with
even coefficients, then $\mathscr{I}$ is generated by the section
given by the polynomial 2. Let 
$$
\mathscr{S}'' = \mathfrak{a}/\mathscr{I} = Z_2 [x_1, x_2, \ldots],
$$
then
$$
0 \to \mathscr{I} \to \mathfrak{a} \to \mathscr{S}'' \to 0
$$
is exact. Then, as sheaves of $\mathfrak{a}$-modules, $\mathscr{I}$ has
properties $(a)$ and $(a_1)$, $\mathfrak{a}$ has $(b)$ and $(b_1)$ and
$\mathscr{S}''$ has $(b)$ but \textit{not} $(b_1)$. 

\noindent
(6) \qquad \textit {If $0 \to \mathscr{S}' \xrightarrow{f} \mathscr{S}
  \xrightarrow{g}\mathscr{S}'' \to 0 $ is an exact sequence of sheaves
  of $\mathfrak{a}$-modules such that $\mathscr{S}'$ and
  $\mathscr{S}''$ have property $(a)$, then $\mathscr{S}$ has
  property $(a)$.}  

\begin{proof}
Since $\mathscr{S}''$ has property $(a)$, for each point $x$ there is
a neighbourhood $U$ of $x$ and an epimorphism $ \phi :
\sum^k_{i=1}\mathfrak{a}_U 
\to \mathscr{S}''_U$. There is an open set $W$ with $x \in W \subset
U$, a homomorphism $\psi : \sum ^k_{i=1} \mathfrak{a}_W \to \mathscr{S}_W$, an
epimorphism $\eta : \sum^k_{i=k+1} \mathfrak{a}_W \to \mathscr{S}'_W$ and
homomorphisms $\theta$, $p$, $p'$ as in the previous proof. If $s \in
\mathscr{S}_W$, there is some $a \in \sum^k_{i=1}\mathfrak{a}_W$ such that $\phi
a = g s$. Then $g (s- \psi a) =0$, and by exactness, for some $b \in
\mathscr{S}'_W, s-\psi a =fb$. Then for some $c \in \sum^{k}_{i=k+1}
\mathfrak{a}_W$, $b=\eta c$ and $s = \psi a + f \eta c= \theta (a+c)$. Hence
$\theta , \sum^l_{i=1} \mathfrak{a}_W \to \mathscr{S}_W$ is an \pageoriginale
epimorphism, and hence $\mathscr{S}$ has property $(a)$. 
\end{proof}

The corresponding statement, with $(a_1)$ in place of $(a)$, is not
true as the following example shows. 

\begin{exam}%exe 34
Let $X = \bigcup^{\infty}_{n=1} C_n$ as in Example
\ref{chap31:exam33}. Let $\mathfrak{a} =Z$ 
and let $\mathscr{S}_n$ be a sheaf which is locally $Z_4$, but on
going around the circle $C_n$, 1 and 3 interchange. Let
$\mathscr{S}'_n$ be the subsheaf with stalks consisting of 0 and
2; it is the constant sheaf $Z_2$. Let $\mathscr{S''}_n =
\mathscr{S}_n / \mathscr{S'}_n $; this is also $Z_2$. Then the
sequence 
$$ 
0 \rightarrow \mathscr{S'}_n \rightarrow \mathscr{S}_n \rightarrow
\mathscr{S''}_n \rightarrow 0 
$$
is exact. Let  
$$
\mathscr{S'} =  \Sigma^{\infty}_{n=1} \mathscr{S'}_n, \mathscr{S} =
\Sigma^{\infty}_{n=1} \mathscr{S}_n, \mathscr{S''} =
\Sigma^{\infty}_{n=1} \mathscr{S''}_n. 
$$

Then the sequence
$$
 0 \rightarrow \mathscr{S'} \rightarrow \mathscr{S} \rightarrow 
 \mathscr{S''} \rightarrow 0 
$$
is exact. Since $\mathscr{S'}$ and $\mathscr{S''}$ are constant
sheaves, they have property $(a_1)$ but $\mathscr{S}$ does not have
property $(a_1)$. 
\end{exam}

Statements (1), (2), (3), (4), (5), (6) (Lectures \ref{chap29:lec29},
\ref{chap30:lec30}, \ref{chap31:lec31}) 
prove the following proposition due to Serre. 

\begin{proposition}%pro 21
If $0 \rightarrow \mathscr{S'} \xrightarrow{f} \mathscr{S}
\xrightarrow{g} \mathscr{S} \rightarrow 0$ is an exact sequence
  of sheaves of $\mathfrak{a}$-modules and if two of them are
  coherent (i.e., possess property (a) and (b)), then the
  third is also coherent. 
\end{proposition}

\begin{coro*}
If\pageoriginale $\mathscr{S}_i$, $i= 1, \ldots , k$ are coherent sheaves of
$\mathfrak{a}$- modules, then $\sum^k_{i-1} \mathscr{S}_i$ is coherent  
\end{coro*}

\begin{proof}
Since the sequence of sheaves 
$$
0 \rightarrow \mathscr{S}_k \rightarrow \Sigma^k_{i-1} \mathscr{S}_i
\rightarrow \Sigma^{k-1}_{i=1} \mathscr{S}_i \rightarrow 0 
$$
is exact, the result follows by induction. 
\end{proof}

