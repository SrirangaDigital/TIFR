\chapter{Lecture 18} %lecut 18
 
Let\pageoriginale $s^p $ be a fixed $p$-simplex in Euclidean $p$-space
$ R^p $, 
with vertices $ a_0, a_1, \ldots , a_p$, i.e. $s^p$ is the convex set
spanned by points $ a_0 , \ldots , a_p $ which are in general
position. We may  assume  that $a_o $ is the origin and $ a_1 ,
\ldots , a_p $ are unit points of a coordinate axes in $R^p $, and
that $s^{p-1} $ is the face opposite  $ a_p $ in $ s^p $. 
 
\begin{defi*}%defi 0
A differentiable singular $p$-simplex in a $ C^\infty $
manifold $X$ is a $ C^\infty -$ map $t : s^p \rightarrow  X $. The
image, $\im t$, is called the support of the singular simplex $t$. The $j$-th
\textit{face}  $ \partial_j t $ is the composite map $t d_j : s^{p-1}
\rightarrow X $ where $ d_j : s^{p-1} \rightarrow s^p $ is the linear
map which maps $ a_o , \ldots , a_{p-1} $ into $ a_o, \ldots ,
\hat{a}_j, \ldots , a_p $.  
   \end{defi*}   
   
 The support of $ \partial_j t $ is contained in the support of $t$. 
 
 \begin{defi*}%defi 0
 A differentiable singular $p$-cochain in an open set  $ U
\subset X $ is a real valued function of differentiable $p$-simplexes
with supports in  $ U;  f (t ) \in R $ if $ supp t \subset U$. 
 \end{defi*} 
 
 The set $ S^p_U $ of of all differentiable $P-$ cochains in $ U $
 forms a real vector space. There is a restriction homomorphism $
 \rho_{_{VU}} : S^P_U \rightarrow  S^p_V$ for $V \subset U$ and a  
 coboundary  homomorphism $d^p : S^{p-1}_U \rightarrow S^p_U$ defined
 by  
 $$
 ( d^p f ) (t) = \sum^p_{j=0} (-1)^j f ( \partial_j t ). 
 $$
 
 The homomorphisms  $ \rho_{_{VU}} $ and $ d^p $ commute, and $ im d^o
 \subset \ker d^{p+1} $. 
 
 In particular, there is a sequence  
$$
 0 \rightarrow S^o_X \xrightarrow{d_1} S_X^1 \rightarrow \dots
 \rightarrow S^{p-1}_X \xrightarrow{d^p} S^p_X \rightarrow \dots  
 $$ 
 with\pageoriginale $ d^{p+1} d^p = 0$, i.e., $\im d^p \subset \ker
 d^{p+1} $. Let  
 $$
 H^p (S_X) = \ker d^{p+1} / im d^p .
 $$
 
 This vector space is called the $p$-th \textit{cohomology vector
   space of $X$ based on differentiable singular cochains.} 
 
 Since a singular 0-simplex may be identified with the point which
 is its support, $S^o_U $ can be identified with the vector space of
 all functions $ f : U \rightarrow R $. The vector space $
 \mathscr{E}^o (U) $ of $ C^\infty - $ functions on  $U$  is a
 subspace of $S^o_U $, and the space $R$ of constant  functions on $U$
 is a subspace of $\mathscr{E}^o (U)$, i.e., $R \subset \mathscr{E}^o 
(U) \subset S^o_U$.  
 
 The presheaf $\bigg\{S^p_U, \rho_{_{VU}} \bigg\} $ determines a sheaf
 $ \mathscr{S}^p $ and since $ \rho_{_{VU}} $ commutes with $d^p $ and
 with the inclusion homomorphism $ e : R \rightarrow S^0_U $, there
 are induced homomorphisms 
 $$
 0 \rightarrow R  \xrightarrow{e} \mathscr{S}^0 \xrightarrow{d_1}
 \dots \rightarrow \mathscr{S}^{p-1} \xrightarrow{d^p} \mathscr{S}^p
 \rightarrow \dots 
 $$
 
 Here the constant sheaf $R$ is identified with the sheaf of germs of
 constant functions. 
 
 There is a homomorphism 
 $$ 
 \big\{ g_U \big\} : \bigg\{ S^p_U, \rho_{_{VU}} \bigg\} \rightarrow
 \bar{\mathscr{S}}^p,  
 $$
where the image of an element of $ S^p_U $ is the section which it
determines. Then $d$ commutes with $ g_U $ and, in particular, with $
g_X $. Then we have the commutative diagram:  
\[
\xymatrix@=.8cm{
0 \ar[r] & S^o_X \ar[r]\ar[d]_{g_X} & \ldots \ar[r] & S^{p-1}_X
\ar[r]^{d^p} \ar[d]_{g_X} & S^p_X \ar[r]\ar[d]_{g_X} & \ldots\\
0\ar[r]& \Gamma (X,\mathscr{S}^o) \ar[r] & \ldots \ar[r] & \Gamma
(X,\mathscr{S}^{p-1}) \ar[r]^{d^p} & \Gamma(X,\mathscr{S}^p) \ar[r] &
\ldots}
\]\pageoriginale

\textit{The induced homomorphism $ g^*_X : H^p (S_X ) \rightarrow
  \Gamma ( X, \mathscr{S} ) $is an iso\-morphism for each $p$.} 

\begin{proof}
For each  covering $ \mathscr{U} $ of $X$, let $ S^p_{\mathscr{U}} $
denote the vector space of all real valued functions of
differentiable  $ p-$simplexes, which are defined for  each simplex
$t$ with supp $t$ contained in some open set of  $ \mathscr{U} $. $d^p
maps S^{p-1}_{\mathscr{U}} $ into $ S^p_{\mathscr{U}} $ and, if  $
\mathscr{W} $ is a  refinement of $ \mathscr{U} $, there is the
obvious restriction homomorphism $g_{\mathscr{W} \mathscr{U}} :
S^p_{\mathscr{U}} \rightarrow S^p_{\mathscr{W}} $ commuting with $d^p
$. Then $ \bigg\{ S^{p}_{\mathscr{U}} , g_{\mathscr{W} \mathscr{U}}
\bigg\} $ is a direct system  and it can be proved, using a method
similar to the one used in the proof of Proposition 8 (Lecture 12),
that its direct limit is $ \Gamma ( X, \mathscr{S}^p ) $.   
\end{proof} 

The induced homomorphisms $ g^\ast_{\mathscr{W} \mathscr{U}} : H^p (
S_{\mathscr{U} \mathscr{U}} ) \rightarrow H^p ( S_{\mathscr{W}}) $ are
isomorphisms. (See Cartan Seminar, 1948-49, Expos\'e 8, \S 3.).
 Hence  $g^\ast_{\mathscr{U}} : H^p ( S_{\mathscr{U}} ) \rightarrow
H^p \Gamma ( X, \mathscr{S} ) $ is an isomorphism and, in particular,
taking $ \mathscr{U} $ as the covering by one open set $X$, 
$$
g^\ast_X : H^p (S_X ) \rightarrow  H^p \Gamma ( X, \mathscr{S} )  
$$
is an isomorphism, $q.e.d$.

\textit{The sequence}
$$
0 \rightarrow R \xrightarrow{e} \mathscr{S}  \rightarrow  \dots
\rightarrow  \mathscr{S}^{p-1} \xrightarrow{d^p} \mathscr{S}^p
\rightarrow \dots  
$$
\textit{is exact.}

\begin{proof}
It\pageoriginale is  sufficient to show that  
$$
0 \rightarrow R  \xrightarrow{e} S^0_U \rightarrow \dots \rightarrow
S^{p-1}_U \xrightarrow{d^p} S^p_U \rightarrow \dots  
$$
is exact for a cofinal system of neighbourhoods $U$ of each point $ a
\in X $. For this system, take the spherical neighbourhoods $U$
contained in a coordinate neighbourhood $N$ of $ \underbar{a} $. The
result is proved using the conical homotopy operator. (See Cartan
Seminar, 1948-49, Expose 7, \S 6 ; his formula should be
replaced by  
\begin{align*}
y ( \lambda_0 , \ldots \, \ldots \lambda_{p+1} ) &= \phi ( \lambda_0 )
x \; \; ( \lambda_1 / ( 1 -\lambda_0 ) , \ldots ) ~ &\lambda_0 \neq 1, \\  
&=  0  &\lambda_0 = 1 ; 
\end{align*}
where the indefinitely differentiable function $\phi$ is chosen so
that $0 \leqq \phi ( \lambda)_0 ) \leqq 1$ for $0 \leqq \lambda_0
\leqq 1$, $\phi (0) = 1$ and $\phi ( \lambda_0 ) = 0$ for $\lambda_o $
in some neighbourhood of 1.)  
\end{proof}

\textit{The sheaf  $ \mathscr{S}^p $  is fine.} 

\begin{proof}
Let $E \subset G$  with $E$ closed and $G$ open. Define $ h_U :
S^p_U \rightarrow S^p_U$ by 
\begin{align*}
( h_U  f )  (t) &=  f (t) \quad \text{ if  supp } t \subset G \\ 
&= 0  \qquad \text{ otherwise.}
\end{align*}

Then $h_U$ is a homomorphism commuting with $\rho_{_{VU}}$ and induces a
homomorphism $h : \mathscr{S}^p \rightarrow \mathscr{S}^p$ such that
$h_x : S^p_x \rightarrow S^p_x$ is the  identity if $x \in G$, and
is zero if $x \in X - \bar{G}$. Thus $\mathscr{S}^p$ is fine, $q.e.d$.  
\end{proof}

Now\pageoriginale let  $ h_U : \mathscr{E}^p (U)  \rightarrow  S^p_U $
be the homomorphism defined by  
$$
( h_U \omega ) (t)  = \int\limits_{s^p} t^{-1} \omega ,  
$$
where $ t^{-1} \omega$  is  the inverse image of the form $\omega$ 
by $t$. Clearly $h_U $ commutes with $ \rho_{_{VU}} $, hence induces a
homomorphism $ h : \Omega^p \rightarrow  \mathscr{S}^p $ with
commutativity in  
\[
\xymatrix{
\mathscr{E}^p(U) \ar[r]^{\rho_{_{xU}}} \ar[d]_{h_U} & \Omega^p_x
\ar[d]^{h_x}\\
S^p_U \ar[r]^{\rho_{_{xU}}} & S^p_x.
}
\]

Hence there is an induced homomorphism $ h : \Gamma ( X, \Omega^p )
\rightarrow \Gamma ( X, \mathscr{S}^p ) $ with commutativity in  
\[
\xymatrix{
\mathscr{E}^p(X) \ar[r]^{f_X} \ar[d]_{h_X} & \Gamma(X,\Omega^p)
\ar[d]^h\\
S^p_X \ar[r]^{g_X} & \Gamma(X,\mathscr{S}^p).
}
\]

$h_U$ \textit{commutes with} $d^p$.

\begin{proof}
\begin{align*}
( h_U \; d^p \omega ) t &=  \int\limits_{s^p} t^{-1} ( d^p \omega )\\ 
&=  \int\limits_{s^p} d^p  ( t^{-1} \omega ) \; ( \text{ since}  d^p
  \text{ commutes with } t^{-1}) \\ 
&= \sum^p_{j=0} (-1)^j \int\limits_{d_js^p} t^{-1}  \omega (\text{ by
    Stokes' theorem for } s^p ), \\ 
&= \sum^p_{j=0} (-1)^j ( h_U \omega ) \partial_j t \\ 
&= (d^p h_U \omega ) t.
\end{align*}

Thus\pageoriginale $h_X$ induces a homomorphism $ h^*_X : H^p (
\mathscr{E} (x)) \rightarrow H^p (S_X) $. Also the homomorphisms $ h : \Omega^p
\rightarrow \mathscr{S}^P$ and hence the induced homomorphisms $h :
\Gamma (X, \Omega^p) \rightarrow \Gamma ( X, \mathscr{S}^P ) $ commute
with $d^p $, and thus there are induced homomorphisms $ h^* : H^p
\Gamma ( X, \Omega  ) \rightarrow H^p  \Gamma ( X, \mathscr{S} )
$. There is commutativity in  
\[
\xymatrix{
H^p(\mathscr{E}(X))\ar[r]^{f^{\ast}_X} \ar[d]_{h^{\ast}_X} &
H^p\Gamma(X,\Omega)\ar[d]^{h^{\ast}}\\
H^p(S_X) \ar[r]^{g^{\ast}_X} & H^p\Gamma (X,\mathscr{S}).
}
\]

\textit{The homomorphism $h^*$ is an isomorphism.}
\end{proof}

\begin{proof}
We have the two resolutions
\[
\xymatrix{
& & \Omega^0\ar[r] \ar[dd]^h & \ldots \ar[r]&
  \Omega^{p-1}\ar[r]^{d^p}\ar[dd]^h & \Omega^p \ar[r] \ar[dd]^h &
  \ldots\\
0 \ar[r] & R\ar[ur]^e \ar[dr]_e & & & & & \\
& & \mathscr{S}^0 \ar[r] & \ldots \ar[r] & \mathscr{S}^{p-1}
\ar[r]^{d^p} & \mathscr{S}^p \ar[r] & \ldots 
}
\]

\noindent
of $R$. The homomorphism $h$  commutes with $ d^p $, and commutativity
in the triangle follows from the fact that $ R \subset \Omega^o
\rightarrow \mathscr{S}^o $ and $e$, $h$ and $e$ are inclusion
homomorphisms. Hence $h^* $ is the isomorphism of the uniqueness
theorem, $q. e. d$. 
\end{proof}

\begin{thm}[de Rham]%Theorem 2.
The homomorphism 
$$
h^*_X : H^p ( \mathscr{E} (X)) \rightarrow  H^p (S_X ) 
$$
is an isomorphism.
\end{thm}

\begin{proof}
The following diagram is commutative :  
\[
\xymatrix{
H^p(\mathscr{E}(X)) \ar[r]^{f^{\ast}_X}
 \ar[d]_{h^{\ast}_X} & H^p\Gamma(X,\Omega)\ar[d]^{h^{\ast}}\\
H^p (S_X) \ar[r]^{g^{\ast}_X} & H^p\Gamma (X,\mathscr{S}).
}
\]\pageoriginale

Since $f^\ast_X$, $g^\ast_X $, and $h^\ast$ are isomorphisms, and the above
diagram is commutative, we have $h^\ast_X = g^{\ast-1}_X h^\ast
f^\ast_X$. Therefore, $h^\ast_X $ is an isomorphism.   
\end{proof}

