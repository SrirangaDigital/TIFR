\chapter{Lecture 14} % Lecture 14

\begin{defi*}
A {\em resolution}\pageoriginale of a sheaf $\mathscr{G}$ of
$A$-modules is an exact sequence of sheaves $A$-modules  
$$
0 \to \mathscr{G} \xrightarrow{e} \mathscr{S}^o \xrightarrow{d^1}
\mathscr{S}^1 \to \cdots \to \mathscr{S}^{q-1} \xrightarrow{d^q}
\mathscr{S}^q \to \cdots  
$$
such that $H^q(X,  \mathscr{S}^q)= 0$, $p \geqq 1$, $q \geqq 0$. 
\end{defi*}

There are than induced homomorphisms
{\fontsize{10}{12}\selectfont
$$
0  \xrightarrow{d^o} \Gamma(X, \mathscr{S}^o) \xrightarrow{d^1} \cdots
\to  \Gamma(X, \mathscr{S}^{q-1}) \xrightarrow{d^q} \Gamma(X,
\mathscr{S}^q) \xrightarrow{d^{q+1}} \Gamma (X, \mathscr{S}^{q+1})
\to \cdots 
$$}
for which $\im d^q \subset \ker d^{q+1}$, i.e., $d^{q+1} d^{q}=0$. The
$A$-modules $\Gamma(X, \mathscr{S}^k)\break  (k \geqq 0)$ together with the
homomorphisms $d^q$ form a formal cochain complex denoted by
$\Gamma(X, \mathscr{S})$. Let the $q$-th cohomology module of the
complex $\Gamma (X, \mathscr{S})$ be denoted by $H^q \Gamma(X,
\mathscr{S})= \ker d^{q+1}/ \im d^q$. 

\begin{exam}%exe 22
Let $X$  be the unit segment $\bigg\{ x: o \leqq x \leqq 1 \bigg\}$,
and let $\mathscr{G}$ be the subsheaf of the constant sheaf $Z_2$
formed  by omitting the points (0,1) and (1,1). A resolution   
\begin{equation*}
0 \to \mathscr{G} \xrightarrow{e} \mathscr{S}^o \xrightarrow{d^1}
\mathscr{S}^1 \to 0 \tag{1} 
\end{equation*}
of $\mathscr{G}$ is obtained by identifying $\mathscr{G}$ with the 
sheaf $\mathscr{S}'$ of Example \ref{chap12:exam16} and taking $\mathscr{S},
\mathscr{S}''$, $i$, $j$ for $\mathscr{S}^o$, $\mathscr{S}^1$, $e$,
$d^1$. (That $H^p(X, \mathscr{S}^q)=0$, $p \geqq 1$, $q \geqq 0$ can be
verified.) The induced sequence  
$$
0 \to \Gamma(X, \mathscr{S}^o) \to \Gamma (X, \mathscr{S}') \to 0  
$$ 
is\pageoriginale 
$$
0 \to 0  \to Z_2  \to  0.
$$
  \end{exam}  
  
  Another resolution
  \begin{equation*}
0 \to  \mathscr{G} \xrightarrow{e} \mathscr{S}^o \xrightarrow{d^1}
\mathscr{S}^1 \xrightarrow{d^2} \mathscr{S}^2 \to 0 \tag{2} 
  \end{equation*}  
  of $\mathscr{G}$ is obtained  by taking $e$, $\mathscr{S}^o$ as
  before, $\mathscr{S}^1 = Z_2 +Z_2$, $\mathscr{S}^2= Z_2$, $d^1= kj$
  where $k(x,1)  = (x, (1,0))$, and $d^2$ with $d^2 (x,
  (1,1))=d^2(x,(0,1))= (x,1)$. 
  
  Another resolution 
  \begin{equation*}
0 \to  \mathscr{G} \xrightarrow{e} \mathscr{S}^o \xrightarrow{d^1}
\mathscr{S}^1 \to 0 \tag{3} 
  \end{equation*}  
  of $\mathscr{G}$ is obtained by taking $\mathscr{S}^o$ to be
  subsheaf of $Z_2$  formed by omitted (0,1), with $e: \mathscr{G}
  \to \mathscr{S}^o$ as the inclusion homomorphism and with $d^1$ as
  the natural homomorphism onto the quotient sheaf $\mathscr{S}^1=
  \mathscr{S}^o /\mathscr{G}$. 
  
  Yet another  resolution 
  \begin{equation*}
0 \to  \mathscr{G} \xrightarrow{e} \mathscr{S}^o \xrightarrow{d^1}
\mathscr{S}^1 \to 0 \tag{4} 
  \end{equation*}  
  of $\mathscr{G}$ is obtained by taking $\mathscr{S}^o = Z_2$ and
  $\mathscr{S}^1 = \mathscr{S}^o/ \mathscr{G}$. 
  
  In each case $H^1 \Gamma (X, \mathscr{S}) = Z_2$, $H^p \Gamma (X,
  \mathscr{S})=0$, $p > 1$. 
  
  \begin{exam}%exe 23
Let $X$ be the sphere $x^2 +y^2+z^2=1$, and let $\mathscr{G}$ be the
constant sheaf $Z_2$. Let $ \mathscr{R}$ denote the constant sheaf
$Z_2 +Z_2$ with $i : \mathscr{G} \to \mathscr{R}$ defined by $i(1) =
(1,1)$. Let $\mathscr{R}' \subset \mathscr{R}$ consist\pageoriginale
of all zeros 
together with $((x,y,z), (0,1))$ for $z < 0$; let $\mathscr{S}^o=
\mathscr{R}/ \mathscr{R}'$ and let $j: \mathscr{R} \to \mathscr{S}^o$
be the natural  homomorphism. Let $e = ji:\mathscr{G} \to
\mathscr{S}^o$. 
  \end{exam}  
  
  Let $\mathfrak{I}_o$ be the quotient sheaf $\mathscr{S}^o /
  e(\mathscr{G})$ and let $h: \mathscr{S}^o \to \mathfrak{I}_o$ be the
  natural homomorphism. The stalks of $\mathfrak{I}_o$ are $Z_2$  on
  the equator and 0 elsewhere. 
  
  Let $\mathfrak{I}$ be the quotient  sheaf of $\mathscr{R}$
  consisting of $Z_2 +Z_2$ on the equator and 0 elsewhere. Identify
  $\mathfrak{I}_o$ with the subsheaf of $\mathfrak{I}$ consisting  of
  all zeros and all $((x,y,0), (1,1))$, and let $k :\mathfrak{I}_o \to
  \mathfrak{I}$ be the inclusion homomorphism. Let $\mathfrak{I}'$ be
  for $y > 0$ and $((x,y,0),(1,0))$ for $y < 0$. Let $\mathscr{S}'=
  \mathfrak{I}/\mathfrak{I}'$ and let $l:  \mathfrak{I} \to
  \mathscr{S}^1$ be the natural homomorphism. Let $d^1 = 1 kh:
  \mathscr{S}^o \to \mathscr{S}^1$. Let $\mathscr{S}^2=  \mathscr{S}^1
  / _{d^1} \mathscr{S}^o$ and let $d^2$ be the natural
  homomorphism. Then from the diagram: 
\[
\xymatrix{
& 0\ar[d] & & 0\ar[d] & \\
& \mathscr{R}' \ar[d] & & \mathfrak{I}'\ar[d] & \\
& \mathscr{R}\ar[d]^{j} & \mathfrak{I}_o \ar[r]^k & \mathfrak{I} \ar[d]& \\
\mathscr{G} \ar[ur]^{i} \ar[r]^e & \mathscr{S}^o \ar[ur]_h \ar[rr]^{d'}
& &  \mathscr{S}' \ar[r]^{d^2} & \mathscr{S}^2
}
\]
  we see that 
  $$
  0 \to \mathscr{G} \xrightarrow{e} \mathscr{S}^o \xrightarrow{d^1}
  \mathscr{S}^1 \xrightarrow{d^2} \mathscr{S}^2 \to 0 
  $$ 
  is a resolution of $\mathscr{G}$ and the induced sequence 
  $$
  0 \to \Gamma(X, \mathscr{S}^o) \to \Gamma(X,\mathscr{S}^1) \to
  \Gamma(X,\mathscr{S}^2) \to 0 
  $$
  is\pageoriginale 
  $$
  0 \to  Z_2 + Z_2 \to  Z_2 +Z_2 \to Z_2 + Z_2 \to 0 
  $$
  and 
  $$
  H^o \Gamma(X, \mathscr{S}) = H^2 \Gamma (X,\mathscr{S}) = Z_2, H^1
  \Gamma(X, \mathscr{S}) =0 
  $$
  
  \begin{proposition}\label{chap14:prop10}%pro 10
If $X$ is paracompact normal and if
$$
0 \to \mathscr{G} \xrightarrow{e} \mathscr{S}^o \to \cdots \to
\mathscr{S}^{q-1} \xrightarrow{d^q} \mathscr{S}^q \to \cdots 
$$
is a resolution of $\mathscr{G}$, there is a uniquely determined
  isomorphism $\eta:H^q \Gamma (X, \mathscr{S}) \to H^q (X,
  \mathscr{G})$. 

If
  $$ 
  0 \to  \mathscr{G}_1 \xrightarrow{e} \mathscr{S}^o_1
  \xrightarrow{d^1} \mathscr{S}^1_1 \to \cdots \to \mathscr{S}^{q-1}_1
  \xrightarrow{d^q} \mathscr{S}^q_1 \to \cdots 
  $$
is a resolution of another sheaf $\mathscr{G}_1$ and if
  $$
  h:(\mathscr{G}, \mathscr{S}^o,\mathscr{S}^1, \ldots)  \to
  (\mathscr{G}_1, \mathscr{S}^o_1, \mathscr{S}^1_1, \ldots) 
  $$
is a homomorphism commuting with $e,d^1,d^2, \ldots $,
then the induced homomorphism $H^*$ commutes with
  $\eta$. 
\[
\xymatrix{
H^q\Gamma(X,\mathscr{S}) \ar[r]^{\eta}\ar[d]^{h^{\ast}} &
H^q(X,\mathscr{G})\ar[d]^{h^{\ast}} \\
H^q\Gamma (X,\mathscr{S}_1) \ar[r]_{\eta} & H^q(X,\mathscr{G}_1) \quad .
}
\]
 \end{proposition}

  \begin{proof}
The\pageoriginale homomorphism $h^* : H^q(X, \mathscr{G}) \to H^q (X,
\mathscr{G}_1)$ is the usual induced homomorphism. Now, since $h$
commutes with $d^q$, $q \geqq 1$, $h$ also commutes with the
homomorphisms 
$$
d^q : \Gamma(X,\mathscr{S}^{q-1}) \to \Gamma (X, \mathscr{S}^q) \;
 (q \geqq 1), 
$$
and hence there is an induced homomorphism 
$$
h^*:H^q \Gamma (X,\mathscr{S}) \to H^q \Gamma(X, \mathscr{S}_1). 
$$
  \end{proof}  
  
  Let $\mathfrak{z}^q=\im  d^q= \ker d^{q+1} \subset \mathscr{S}^q$; then
  there are exact sequences  
  \begin{equation*}
\left.
\begin{aligned}
&0 \to  \mathscr{G} \xrightarrow{e} \mathscr{S}^o \xrightarrow{d^1_o}
  \mathfrak{z}^1 \to  0 \\ 
&0 \to \mathfrak{z}^q\xrightarrow{i^q} \mathscr{S}^q
  \xrightarrow{d^{q+1}_o} \mathfrak{z}^{g+1} \to 0 ~~ (q \geqq 1)  
\end{aligned}
\right \} \tag{1}
  \end{equation*}  
  where $i^q$ is the inclusion homomorphism and $d^q_o$  is the
  homomorphism induced  by $d^q :  i^q d^q_o =d^q$. Since $h$ commutes
  with $d$, $h$  maps $\mathfrak{z}^q$ in $\mathfrak{z}^q_1$, and
  commutes with $i$, $d_o$. 
\[
\xymatrix{
0\ar[r] & \mathscr{G} \ar[r]^e\ar[d]_h & \mathscr{S}^o \ar[r]^{d^1_o}
\ar[d]_h & \mathfrak{z}^1 \ar[r]\ar[d]_h & 0,\\
0 \ar[r] & \mathscr{G}_1 \ar[r]^e & \mathscr{S}^o_1 \ar[r]^{d^1_o} &
\mathfrak{z}^1_1 \ar[r] & 0,
}\]
\[
\xymatrix{
0\ar[r] & \mathfrak{z}^q \ar[r]^{i^q}\ar[d]_h & \mathscr{S}^q \ar[r]^{d^{q+1}_o}
\ar[d]_h & \mathfrak{z}^{q+1} \ar[r]\ar[d]_h & 0,\\
0 \ar[r] & \mathfrak{z}^q_1 \ar[r]^{i^q} & \mathscr{S}^q_1 \ar[r]^{d^{q+1}_o} &
\mathfrak{z}^{q+1}_1 \ar[r] & 0 .
}
\]  

  Hence the induced homomorphism $h^*$ of the corresponding exact
  cohomology sequences also commutes also commutes with $e^*$, $d^*_o$,
  $\delta$ and $i^*$, $d^*_o$, $\delta^*$ respectively. 

\medskip
\noindent{\textbf{Case 1.}}
$\underline{q=0}$.\pageoriginale Since $0 \to  \mathscr{G} \xrightarrow{e}
\mathscr{S}^o \xrightarrow{d^1} \mathscr{S}^1$  is exact, so is the
sequence $0 \to \Gamma(X, \mathscr{G})  \xrightarrow{e} \Gamma(X,
\mathscr{S}^o) \xrightarrow{d^1} \Gamma(X, \mathscr{S}^1)$. Then
$H^o(X, \mathscr{S})= \ker d^1= im e$, but $e$ is a monomorphism and
$\Gamma(X, \mathscr{G}) = H^o(X,\mathscr{G})$, hence $e : H^o (X,
\mathscr{G})  \to H^o \Gamma(X, \mathscr{S})$ 
is an isomorphism commuting with $h^*$. Let $\eta= e^{-1}$. 

\medskip
\noindent{\textbf{Case 2.}}
$\underline{q >0}$. The exact cohomology sequence corresponding to
that exact sequences (1) for $q-1$ (where $\mathfrak{z}^o=
\mathscr{G}$) is    
{\fontsize{10}{12}\selectfont
$$ 
0 \to \Gamma (X, \mathfrak{z}^{q-1})  \xrightarrow{i^{q-1}} \Gamma (X,
\mathscr{S}^{q-1}) \xrightarrow{d^q_o} \Gamma (X, \mathfrak{z}^q)
\xrightarrow{\delta^*} H^1(X, \mathfrak{z}^{q-1})  \to 0 \to  \cdots 
$$}\relax
since  $H^1 (X, \mathscr{S}^{q-1})=0$. Thus $\delta^*$ induces an
isomorphism 
$$
\delta^* : \Gamma (X, \mathfrak{z}^q)/  im d^q_o \to  
H^1(X,\mathfrak{z}^{q-1}) \; (q \geqq 1). 
$$
  
  Since $\im  i^q= \ker d^{q+1}_o = \ker d^{q+1}$, the monomorphism
  $i^q$ induces an isomorphism  
  \begin{align*}
  i^{q^*}: \Gamma(X, \mathfrak{z}^q)  / \im d^q_o &\to \im  i^q
 / \im d^q\\ 
  & = \ker d^{q+1} /  \im d^q\\
  & = H^q \Gamma (X,\mathscr{S}).
  \end{align*}
  
  Thus we have an isomorphism
  $$ 
  \delta^* (i^{q^*})^{-1}: H^q \Gamma (X, \mathscr{S}) \to  H^1
  (X,\mathfrak{z}^{q-1}) \;  (q \geqq 1) 
  $$
  commuting with $h^*$, since $h^*$ commutes with $\delta^*$ and
  $(i^*)^{-1}$. 
\[
\xymatrix{
& 0 \ar[d]& & \\
\Gamma(X,\mathscr{S}^{q-1}) \ar[r]^{d^q_o}\ar[dr]_{d^q} & \Gamma
(X,\mathfrak{z}^q) \ar[r]^{\delta^{\ast}} \ar[d]_{i^q} &
H^1(X,\mathfrak{z}^{q-1}) \ar[r] & 0\\ 
& \Gamma(X,\mathscr{S}^q)\ar[d]_{d^{q+1}_o} \ar[dr]^{d^{q+1}} & \\
0 \ar[r] & \Gamma (X,\mathfrak{z}^{q+1}) \ar[r]_{i^{q+1}} & \Gamma
(X,\mathscr{S}^{q+1}). 
}
\]\pageoriginale
  
  Also, the exact cohomology sequences corresponding to (1) contain  
  $$
  0 \xrightarrow{(d^{q-p+1}_o)^*}H^{p-1}(X, \mathfrak{z}^{q-p+1})
  \xrightarrow{\delta^*} H^p(X,\mathfrak{z}^{q-p})
  \xrightarrow{(i^{q-p})^*} 0  
  $$
  for $1< p  < q$ and  
  $$
   0 \xrightarrow{d^{1^*}_o}H^{q-1}(X, \mathfrak{z}^1)
   \xrightarrow{\delta^*} H^q(X,\mathscr{G}) \xrightarrow{e^*} 0 \quad 
   \text{for} \quad p = q. 
  $$
  
  Thus we have isomorphisms $(q \geqq 1)$,
  $$
  H^q \Gamma(X, \mathscr{S}) \xrightarrow{\delta^*(i^{q^*})^{-1}}
  H^1(X,\mathfrak{z}^{q-1})  \to \cdots \to H^{q-1}(X,\mathfrak{z}^1)
  \xrightarrow{\delta^*} H^q(X,\mathscr{G}) 
  $$
  commuting with $h^*$. Let $\eta$ be the composite of these
  isomorphisms,  $\eta: H^q \Gamma(X, \mathscr{S}) \to H^q(X,
  \mathscr{G})$. 
  

  \begin{thm}[Uniqueness theorem]%the 1
If $X$ is paracompact normal, and if
\begin{align*}   
&0 \to \mathscr{G} \xrightarrow{e} \mathscr{S}^o
  \xrightarrow{d^1}\mathscr{S}^1 \to \cdots \to \mathscr{S}^{q-1}
  \xrightarrow{d^q} \mathscr{S}^q \to \cdots ,\\ 
&0 \to \mathscr{G} \xrightarrow{e} \mathscr{S}^o_1
  \xrightarrow{d^1}\mathscr{S}^1_1 \to \quad \; \to \mathscr{S}^{q-1}_1
  \xrightarrow{d^q} \mathscr{S}^q_1 \to \cdots ,  
\end{align*}
are\pageoriginale two resolutions of the same sheaf $\mathscr{G}$ of
$A$-modules, there is a canonical isomorphism  
$$
 \phi:H^q \Gamma(X,\mathscr{S})  \to H^q \Gamma(X,\mathscr{S}_1).
$$
Moreover, if $h:(\mathscr{S}^o, \mathscr{S}^1,\mathscr{S}^2, \ldots)
\to (\mathscr{S}^o_1, \mathscr{S}^1_1, \mathscr{S}^2_1, \ldots)$
is a homomorphism commuting with $e, d^1, d^2, \ldots $, 
{\fontsize{9}{11}\selectfont
\[
\xymatrix{
& & \mathscr{S}^o \ar[r]^{d^1} \ar[dd]^h & \mathscr{S}^1
  \ar[r]^{d^2} \ar[dd]^h & \ldots  \ar[r] & \mathscr{S}^{q-1}
  \ar[r]^{d^q}\ar[dd]^h & \mathscr{S}^q\ar[r]\ar[dd]^h & \ldots\\
0 \ar[r] & \mathscr{G} \ar[ur]^e \ar[dr]_e & & &&&&\\
& & \mathscr{S}^o_1 \ar[r]^{d^1} & \mathscr{S}^1_1 \ar[r]^{d^2} &
\ldots \ar[r] & \mathscr{S}^{q-1}_1 \ar[r]^{d^q} & \mathscr{S}^q_1
\ar[r] & \ldots
}
\]}\relax
then the induced homomorphism
$$
h^*: H^q \Gamma (X,\mathscr{S}) \to H^q \Gamma(X,\mathscr{S}_1)
$$
is the isomorphism $\phi$.
  \end{thm}

\begin{proof}
We have the canonical isomorphisms $\eta$, $\eta_1$, 
$$
H^Q \Gamma(X,\mathscr{S}) \xrightarrow{\eta} H^q(X,\mathscr{G})
\xleftarrow{\eta_1} H^q \Gamma (X,\mathscr{S}_1); 
$$
let $\phi= \eta^{-1}_1 \eta$.
\end{proof}

There is commutativity in the diagram:
\[
\xymatrix{
H^q\Gamma(X,\mathscr{S}) \ar[r]^{\eta}\ar[d]_{h^{\ast}} &
H^q(X,\mathscr{G})\ar[d]_{h^{\ast}} \\
H^q\Gamma(X,\mathscr{S}_1) \ar[r]^{\eta_1} & H^q(X,\mathscr{G}),
}
\]
where the homomorphism $h^*$ on the right is the identity. Hence the
homomorphism $h^*$ on the left is equal to $\eta^{-1}_1 \eta =\phi$.  

