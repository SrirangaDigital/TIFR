\chapter{Lecture 2}

\begin{exam}%exe 4
Let\pageoriginale $S:\{ (x,y): x^2 +y^2 =1, x < 1 \}$ together with
the group of 
integers $Z \}$. The topology on the former set is the usual induced
topology, and the neighbourhoods for an integer $n \in Z$ are given by
$G_a(n)= \{ n,(x,y) : x^2 + y^2 =1 , a< x < 1 \}$; $X:\{ (x,y): x^2 +
y^2 =1 \}$ and let $\pi : S \to X$ be defined by  $\pi (n) = (1,0)$,
$\pi (x,y) = (x,y)$. $\pi^{-1} (1,0)$ is the group $Z$, for other
pints $(x,y) \in X$, $\pi^{-1}(x,y) = (x,y)$ is regarded as the zero
group. It is easily verified that $\mathscr{S}= (s, \pi , X)$ is a
sheaf. Here $S$ is 
locally euclidean, and has a countable base. The set of all zeros is
open (Corollary \ref{chap1:cor2}) and compact but not closed; its
closure is not compact, $S$ is $T_1$ but not Hausdorff though $X$ is
Hausdorff.  
\end{exam}

\noindent
\textbf{Exercise}. 
If $X$ is a $T_o$ or a $T_1$ space, space, so is
$S$ 

So far we have only defined sheaves of abelian groups. It is now quite
clear how we can extend the definition to the case where the stalks
are any algebraic systems. 

\textit{A sheaf of rings} is a local homeomorphism $\pi : S \to X$
such that each $\pi^{-1}(x)$ is a ring and addition and multiplication
are continuous, i.e 
\begin{align*}
& f  : S + S \to S \text{ where } f(p, q) = p + q,\\
& g : S + S \to S \text{ where } g(p, q) = p \cdot q,
\end{align*}
are continuous.

The\pageoriginale sheaf of function elements (Example
\ref{chap1:exam3}) where multiplication of 
two function elements in the same stalk is defined to be the usual
multiplication of power series is a sheaf of rings. 

In the sheaf of twisted integers (Example \ref{chap1:exam2})  each $S_x$ is
isomorphic to the ring $Z$, but this sheaf is \textit{not} a sheaf of
rings. 

\textit{A sheaf of rings with unit} is a local homeomorphism $\pi :S
\to X$ such that each $\pi^{-1}(x)$ is a ring with unit element $1_x$
and addition, multiplication and unit are continuous; i,e., 
\begin{align*}
f:S+ S \to S, & f(p,q) = p+q,\\
g:S+ S \to S, & f(p,q) = p \cdot q,\\
h:X \longrightarrow S, & h(x) = 1_x \text{ are continuous. }
\end{align*} 

\begin{exam}\label{chap2:exam5}%exe 5
Let $A$ be the ring with elements $0$, $1$, $b$, $c$; where the rules of
addition and multiplication are given by 
\begin{align*}
& 1+ 1 = b+b = c+ c=0;\\
& 1+b =c, 1+c =b, b+c =1;\\
&b^2 =b, c^2=c, bc= cb =0
\end{align*} 
\end{exam}

[The ring $A$ may be identified with the ring of functions defined on
  a set of two elements with values in the field $Z_2$.] 

Let $X: \{ x, k \leq x \leq 1 \},S:$ subspace of $X \times A$ obtained
by omitting the points $(k,1)$, $(k,b);$ and let $\pi : S \to X$, be
defined by $\pi (x,a) = x$. Addition and multiplication in a stalk are
defined by  
\begin{align*}
(x,a_1) + (x,a_2) & = (x,a_1 + a_2),\\
(x,a_1) \cdot (x,a_2) & = (x,a_1 \cdot a_2).
\end{align*}

\noindent
Then\pageoriginale $\mathscr{S} = (S, \pi, X)$ is a sheaf of rings,
each $S_x$ is a ring with unit, but $\mathscr{S}$ is \textit{not}
sheaf of rings with unit.  

\textit{A sheaf of (unitary left) $\mathfrak{a}$-modules, where
  $\mathfrak{a} = (A, \tau,  X)$ is a sheaf of rings with unit, is a
  local homeomorphism $\pi : S 
  \to X$ such that each $\pi^{-1}(x)$ is a unitary left $A_x$-module
  and addition, and multiplication by elements of $A_x$ (for each $x$)
  are continuous}; i.e.,  
\begin{align*}
f:S+ S \to S, \qquad f(p,q) & = p+q,\\
g:A+ S \to S, \qquad g(a,p) & = ap
\end{align*}
\textit{are continuous}.

$A+S$ is the subspace of $A \times S$ consisting of all pairs $(a,p)$
for which $\tau(a) = \pi(p)$. 

[If $R$ is a ring with unit element, we say that $M$ is a
  \textit{unitary, left} $R$-module if it is a left $R$-module, and
  if $1\cdot m=m$ for each $m \in M$.] 

\begin{exam}%exe 6
Let $\mathfrak{a}$ denote the sheaf of function elements on the complex sphere
$X$. Let $S_x$ consist of all $(p,q)$ of function elements at $x$. Let
$S = \bigcup\limits_{x \in X} S_x$. A neighbourhood of $(p,q)$ is defined
by analytic continuation of $p$ and $q$. In each $S_x$ addition is
defined as $(p_1,q_1)+(p_2,q_2)=(p_1+q_2,q_1+q_2)$; and if a is a
function element at $x$, define multiplication as $a(p,q)
=(a.p, a.q)$. Define $\pi : S \to X$, by $\pi(p,q)=x$ if $p$, $q$ are
function elements at $x$. Then $(S, \pi, X)$ is a sheaf of
$a$-modules. 
\end{exam}

Any\pageoriginale sheaf $\mathfrak{a}$ of rings with unit can be
regarded as a sheaf of 
$a$-modules; the product $ap$ for $a \in A_x$, $p \in A_x$ being defined
as the product $ap$ in $A_x$. 

\textit{A sheaf of $B$-modules} where $B$ is a ring with unit element is
a local homeomorphism $\pi : S \to X$ such that $\pi^{-1}(x)$ is a
unitary left $B$-module and addition, and multiplication by elements of
$B$ are continuous; i.e., 
\begin{align*}
& f:S+S \to S, \qquad f(p,q)=p+q,\\
& g_b:S \to 	S, g_b (p)=b.p \text{ for each } b \in B
\end{align*}
are continuous.

\textit{Let $\mathscr{S}=(S, \pi , X)$ be a sheaf of B-modules. This
  is equivalent to saying that $\mathscr{S}$ is a sheaf of
  $\mathbb{B}$-modules, where $\mathbb{B}$ is the constant sheaf $(X
  \times B, \tau, X)$.} 

\medskip
\noindent{\textbf{\underline{Proof:}}}

\noindent
\begin{tabular}[t]{ll}
\hline
\multicolumn{1}{c|}{$B$ is a ring with unit element.} &
\multicolumn{1}{c}{$\mathfrak{B}(X \times B, \tau, X)$ is a constant
  sheaf.} \\ 
\hline
\multicolumn{2}{c}{$\mathscr{S}=(S,\pi,X)$ is a sheaf of abelian
  groups.} \\ 
1) $\mathscr{S}$ is a sheaf of B-modules. & 1) $\mathscr{S}$ is a
sheaf of $\mathfrak{B}$-modules.\\ 
This means that $b \cdot p$ is $de$- & This means that $(\pi(p),b)
\cdot p$ is \\   
fined such that $S_x$ is a uni-& defined such that $S_x$ is a unitary \\
  left $B$-module. & left $x \times B$ module.\\
2) Addition is continuous. & 2) Addition is continuous.\\
3) Multiplication is continuous & 3) Multiplication is continuous\\
means that $h:B \times S \to S$ & means that $g: (X \times B)+ S \to S$\\
$h(b,p)=b \cdot p$ is continuous. & $g(\pi(p),b,p)=b.p$ is continuous.\\
\hline
\end{tabular}


\medskip
To\pageoriginale prove the assertion, it is enough to show that the continuity of
$h$ is equivalent to the continuity of $g$. To do this, define $\phi
: B \times S \to (X \times B)+S$ as $\phi (b,p)= (\pi(p), b,p)$, then
$g \phi =h. \phi$ is clearly 1-1. We show that $\phi$ is a
homeomorphism and the result will follow from this. $A$ base for $(X
\times B)+S$ is formed by the sets $(U \times b)+G$ where $G$ projects
homeomorphically onto $\pi (G)$. 
\begin{align*}
(U \times b)+G& = (V \times b)+G \text{ where } V = U \cap \pi(G)\\
& = (V \times b)+H \text{ where } H= (\pi |G)^{-1}V\\
& = (\pi (H) \times b) +H\\
&= \phi (b \times H).
\end{align*}

\noindent
Since the sets $b \times H$ form a base for $B \times S$, it follows
that $\phi$ is a homeomorphism. 

Thus we may identify sheaves of $(X \times B, \tau, X)$-modules with
sheaves if  $B$-modules. By abuse of language, we write $B$ for the ring as
well as for the constant  sheaf $(X \times B, \tau, X)$. 

\begin{exam}%exe 7
Let $C$ be the ring of complex numbers, $\mathscr{S}=(S, \pi , X)$ be
the sheaf of function elements on the complex sphere $X$ and for $c
\in C$, $p \in S$ define $c \cdot p$ to be the usual product of a complex
number with a power series. Then\pageoriginale $\mathscr{S}$ becomes a
sheaf of $C$- 
modules. 
\end{exam}

\begin{exam}%exe 8
Let  $C$ be the ring of complex numbers, $\mathscr{S}=(S, \pi , X)$ be
the sheaf of function elements on the complex sphere $X$. For $c \in C$
and $(p,q) \in S$ define $c(p,q)= (c \cdot p, c \cdot q)$. 
\end{exam}

\noindent
Then $\mathscr{S}$ becomes a sheaf of $C$-modules. 

\begin{exam}%exe 9
Let $\mathscr{S}=(S, \pi , X)$ be any sheaf of abelian groups and let
$Z$ be the ring of integers. For $n \in Z$ and $p \in S$ define
$n\cdot p=p+ \cdots +p$ ($n$ times) if $n>0$, $n\cdot p=-(-n)p$ if $n<0$, and
$0 \cdot p= 0_{\pi (p)}$. Thus $\mathscr{S}$ may be considered as a
sheaf of $Z$-modules.  
\end{exam}

Thus sheaves of rings with unit, sheaves of $B$-modules and sheaves of
abelian groups can be considered as special cases of sheaves of $a$-
modules. 
