\chapter{Lecture 9}\label{chap9:lec9}

\section[Cohomology groups of.....]{Cohomology groups of a space with coefficients
  in a presheaf}\pageoriginale 

\begin{defi*}%def 
A covering (an indexed covering) $\{ U_i \}_{i \in I}$ of a space $X$
is a system of open sets whose union is $X$. 
\end{defi*}

\begin{defi*}%def
If $\sum = \{ S_U, \rho_{VU} \}$ is a presheaf of $A$-module where $A$ 
is a fixed ring with unit element, a $q$-cochain $f(q = 0, 1,
\ldots)$ of a covering $\mathscr{U} = \{U_i\}_{ i \in I}$ with values
in $\sum$ is an alternating function of $q + 1$ indices with  
$$
f(i_o, i_1, \ldots , i_q) \in S_{U_{i_o}} \cap \cdots \cap U_{i_q} 
$$
or more briefly $f(\sigma) \in S_{U_\sigma}$ where $\sigma$ is the
simplex $i_o, \ldots , i_q$. 
In particular $f(i_0, i_1, \ldots , i_q) = 0$ if $U_{i_0} \cap \cdots
\cap U_{i_q} = \phi$. 
( A function $f$ is called alternating if  
\end{defi*}
\begin{enumerate}[(i)]
\item $f(i_0, i_1, \ldots , i_q) = 0$ if any two of $i_0 , \ldots ,
  i_q$ are the same, 

\item $f(j_0, j_1, \ldots , j_q) = \pm f(i_0, i_1 , \ldots , i_q)$
  according as the permutation $j_0, \ldots , j_q$  of $i_0 , \ldots ,
  i_q$ is even or odd). 
\end{enumerate}

We will often write $\rho (V, U)$ for $\rho_{VU}$.

The $q$-cochains of $\mathscr{U}$ with values in $\sum$ form an
$A$-module $C^q (\mathscr{U}, \sum)$. For $q < 0$, we define $C^q
(\mathscr{U}, \sum) = 0$. 

\begin{defi*}%def
The coboundary $\delta^{q+1} f$ (or simply $\delta f$) of $f\in
C^q(\mathscr{U}, \sum)$ is the function on $(q + 1)$-simplexes
defined by  
$$
(\delta^{q + 1} f) (\sigma) = \sum^{q + 1}_{j = 0} (-1)^j \rho 
(U_\sigma ,  U_{\partial j \sigma}) f(\partial_j \sigma), 
$$
where\pageoriginale $\partial_j \sigma = i_0 , \ldots$, $\hat{i_j} ,
\ldots$, $i_{q + 1} 
= i_0 , \ldots$, $i_{j - 1}, i_{j + 1}, \ldots , i_{q + 1}$ is the
$j$-th face of $\sigma = i_o, \ldots , i_{q + 1}$. 
\end{defi*}

\textit{If $f \in C^q (\mathscr{U}, \sum)$, then $\delta f \in
  C^{q+1} (\mathscr{U}, \sum)$.} 

\begin{proof}%proof
It is sufficient to verify that $f$ is an alternating function, e.g., 
\begin{align*}
\delta f(i_1, i_o, \ldots , i_{q + 1}) & = \rho (U_{\sigma},
U_{\partial_1 \sigma}) f(\partial_1 \sigma) - \rho (U_\sigma , U_{
  \partial_o \sigma}) f(\partial_o \sigma)\\ 
& + \sum^{q+1}_{j=2} (-1)^j \rho (U_\sigma , U_{\partial j \sigma})
f(i_1, i_o, \ldots , \hat{i_j}, \ldots , i_{q + 1})\\ 
& = -\rho (U_\sigma , U_{\partial_o j \sigma}) f(\partial_o \sigma) +
\rho (U_\sigma , U_{\partial_1 \sigma}) f(\partial_1 \sigma)\\ 
& - \sum^{q + 1}_{j = 2} (-1)^j \rho (U_\sigma , U_{\partial j
  \sigma}) f(i_o, i_1, \ldots , \hat{i_j} , \ldots , i_{q+1})\\ 
& = - \delta f(i_o , i_1, \ldots , i_{q+1}) 
\end{align*} 
and, if $i_o = i_1$,
\begin{align*}
\delta f(i_o, i_1, \ldots , i_{q+1}) &= \rho (U_\sigma , U_{\partial_o
  \sigma}) f(\partial_o \sigma) - \rho (U_\sigma , U_{\partial_1
  \sigma}) f(\partial_1 \sigma \partial)\\ 
&=0, 
\end{align*}
where $\sigma = i_o , \ldots , i_{q+1}$.
\end{proof}

It follows, since $\rho(U_\sigma , U_{\partial j \sigma})$ is a
homomorphism, that  
$$
\delta^{q+1} : C^q (\mathscr{U}, \sum) \to C^{q + 1} (\mathscr{U}, \sum)
$$
is a homomorphism. One verifies by computation that $\delta^{q+1}
\delta^q f =0$ for $f \in C^{q-1}(\mathscr{U}, \sum)$, using the fact
that for $j \le k$ 
\begin{align*}
\partial_k \partial_j \sigma = \delta_k (i_o, \ldots , \hat{i_j} , \ldots
, i_{k+1}, \ldots , i_{q+1}) & = i_o , \ldots , \hat{i_j} , \ldots ,
\hat{i_{k+1}}, \ldots , i_{q+1}\\ 
& = \partial_j \partial_{k+1} \sigma. 
\end{align*}

(The\pageoriginale computation is carried out at the end of the lecture).

Thus in $\delta^q \subset \ker \delta^{q+1}$ in the sequence
\begin{multline*}
0 \to C^o (\mathscr{U}, \sum) \xrightarrow{\delta'} C'(\mathscr{U}, 
\sum) \to \cdots \\
 \to C^{q-1}(\mathscr{U}, \sum) \xrightarrow{\delta^q}
C^q (\mathscr{U}, \sum)\xrightarrow{\delta^{q+1}} 
\end{multline*}

The quotient module $H^q(\mathscr{U}, \sum) = \ker \delta^{q+1}/ \im
\delta^q$ is called the $q$-th \textit{cohomology module of
  $\mathscr{U}$ with coefficients in the presheaf $\sum$}. 

The elements of the module $Z^q(\mathscr{U}, \sum) = \ker
\delta^{q+1}$ are called $q$-\textit{cocycles} and the elements of
the module $B^q(\mathscr{U}, \sum) = \im \delta^q$ are called
$q$-\textit{coboundaries}. Since $B^o(\mathscr{U}, \sum) = 0$, we have
$H^o (\mathscr{U}, \sum) \approx Z^o(\mathscr{U}, \sum)$. 

\begin{defi*}%def
A covering $\mathscr{W} = \{V_j\}_{j \in J}$ is said to be a
\textit{refinement} of the covering $\mathscr{U} = \{ U_i\}_{i \in I}$
if for each $j \in J$ there is some $i \in I$ with $V_j \subset U_i$. 
\end{defi*}

If $\mathscr{W}$ is a refinement of $\mathscr{U}$, choose a function
$\tau : J \to I$ with $V_j \subset U_{\tau(j)}$. Then there is a
homomorphism 
$$
\tau^+ : C^q (\mathscr{U}, \sum) \to C^q (\mathscr{W}, \sum)  
$$
defined by 
$$
\tau^+ f(\sigma) = \rho (V_\sigma , U_{\tau(\sigma)}) f(\tau \sigma)  
$$
where $\sigma = j_0, \ldots , j_q$; and $\tau \sigma = \tau(j_0),
\ldots , \tau(j_q)$. 

\textit{$\tau^+$ commutes with $\delta$ since} 
\begin{align*}
\delta^{q+1} \tau^+ f(\sigma) &= \sum^{q+1}_{k=0}(-1)^k \rho
(V_\sigma , V_{\partial k \sigma}) \tau^+ f(\partial_k \sigma)\\ 
& = \sum^{q+1}_{k=0}(-1)^k \rho (V_\sigma , V_{\partial k \sigma})
\rho(V_{\partial k \sigma} , U_{\tau(\partial_k \sigma)})
f(\tau(\partial_k \sigma)),\\ 
& = \sum^{q+1}_{k=0}(-1)^k \rho (V_\sigma , U_{\tau(\partial_k
  \sigma)})  f(\tau(\partial_k \sigma),\\ 
\tau^+ \delta^{q+1} f(\sigma) &= \rho(V_\sigma , U_{\tau \sigma})
\delta f(\tau \sigma)\\ 
& = \sum^{q+1}_{k=0}(-1)^k \rho (V_\sigma , U_{\tau \sigma})
\rho(U_{\tau \sigma}, U_{\partial_k \tau \sigma}) f(\partial_k \tau
\sigma),\\ 
& = \sum^{q+1}_{k=0}(-1)^k \rho (V_\sigma , U_{\partial_k \tau
  \sigma}) f(\partial_k \tau \sigma)
\end{align*}\pageoriginale
and $\tau \partial_k \sigma = \partial_k \tau \sigma$. Hence there is an
induced homomorphism, 
$$
\tau_{\mathscr{W} \mathscr{U}}: H^q (\mathscr{U}, \sum) \to H^q
(\mathscr{W}, \sum). 
$$

\textit{The homomorphism $\tau_{\mathscr{W} \mathscr{U}} : H^q
  (\mathscr{U}, \sum) \to H^q (\mathscr{W}, \sum) $ is independent of
  the choice of}  $\tau$. 

\begin{proof}%proof
Let $\tau : J \to I$, $\tau' : J \to I$ be two such choices. Let the
set $J$ be linearly ordered and define the function 
$$
k^{q-1} : C^q (\mathscr{U}, \sum) \to C^{q-1} (\mathscr{W}, \sum)  
$$
by
$$
(k^{q-1} f) (\sigma) = \sum^{q-1}_{h=0}(-1)^h \rho(V_\sigma, U_{\tau_h
  \sigma}) f(\tau_h \sigma)  
$$
for $\sigma = j_0 , \ldots , j_{q-1}$ with $j_0 < j_1 < \cdots <
j_{q-1}$, where 
$$
\tau_h \sigma = \tau(j_0), \ldots , \tau(j_h), \tau'(j_h), \ldots ,
\tau'(j_{q-1}), 
$$
and let $k^{q-1} f$ be alternating. Then $k^{q-1}$ is a homomorphism,
since \break $\rho (V_{\sigma}, U_{\tau_h \sigma}) : S_{U_{\tau_h} \sigma}
\to S_{V- \sigma}$ is a homomorphism. 
\end{proof}

Using\pageoriginale the facts that, for $\sigma = j_0, j_1, \ldots ,
j_q$, 
\begin{align*}
\tau_h \partial_i & = \partial_i \tau_{h+1}  \qquad  (0 \le i \le h \le q-1),\\
\tau_h \partial_i & = \partial_{i+1} \tau_h   \qquad (0 \le h < 
i \le q),\\ 
\partial_h \tau_{h-1} & = \partial_h \tau_h   \qquad (1 \le h \le q),\\
\partial_0 \tau_0 & = \tau' , \quad \partial_{q+1} \tau_q = \tau ,
\end{align*}
one finds that
$$
\delta^q k^{q-1} f + k^q \delta^{q+1} f = \tau'^+ f - \tau^+ f .  
$$

(The computation is given at the end of the lecture.)

This holds for $q=0$ with the obvious meaning of $k^{-1} : C^0
(\mathscr{U} , \sum) \to 0$. Thus if $r \in H^q (\mathscr{U}, \sum)$
is represented by a cocycle $f$, $\tau'^+ f - \tau^+ f$ is a coboundary
and $\tau_{\mathscr{W} \mathscr{U}} r$ is uniquely determined. 

\textit{If the covering $\mathscr{W}$ is a refinement of $\mathscr{W}$
  and $\mathscr{W}$ is a refinement of $\mathscr{U}$, then
  $\tau_{\mathscr{W} \mathscr{W}} \tau_{\mathscr{W} \mathscr{U}} =
  \tau_{\mathscr{W} \mathscr{U} }$ and $\tau_{\mathscr{U}
    \mathscr{U}}$ is the identity.} 

\begin{proof}%proof
If $\mathscr{W} = \big\{W_k \big\}_{k \in K}$ is a refinement of $\mathscr{W}$,
choose $\tau_1 : K \to J$ so that $W_k \subset V_{\tau_1 k}$. Then
$W_k \subset V_{\tau_1 k} \subset U_{\tau \tau_1 k}$ and $\tau_2 : K
\to I$ can be chosen to be $\tau \tau_1$. Then 
\begin{align*}
(\tau^+_1 \tau^+ f) (\sigma) & = \rho(W_\sigma , V_{\tau_1 \sigma})
  (\tau^+ f) (\tau_1 \sigma)\\ 
& = \rho(W_\sigma , V_{\tau_1 \sigma}) \rho (V_{\tau_1 \sigma},
  U_{\tau \tau_1 \sigma}) f(\tau \tau_1 \sigma)\\ 
& = \rho(W_\sigma , U_{\tau_2 \sigma}) f(\tau_2 \sigma)\\ 
& = (\tau^+_2 f)(\sigma).
\end{align*}

Thus\pageoriginale $\tau^+_1 \tau^+ = \tau^+_2$ and so for the induced
homomorphisms, 
$$
\tau_{\mathscr{W} \mathscr{W}} \tau{\mathscr{W} \mathscr{U}} = 
\tau{\mathscr{W} \mathscr{U}} : H^q (\mathscr{U} , \sum) \to H^q
(\mathscr{W} , \sum). 
$$

Similarly, for the refinement $\mathscr{U}$ of $\mathscr{U}, \tau : I
\to I$ can be chosen to be the identity, hence $\tau_{\mathscr{U}
  \mathscr{U}} : H^q (\mathscr{U}, \sum) \to H^q (\mathscr{U}, \sum)$
is the identity. 
\end{proof}
~

\phantom{a}

\vskip -1cm
\pageoriginale
\begin{align*}
& (1) \quad \delta^{q+1} \delta^q = 0. \qquad (2)\delta^q k^{q-1} +
  k^q \delta^{q+1} = \tau'^+ - \tau^+.\\ 
& (1)\quad (\delta^{q+1} \delta^q f)(\sigma) = \sum^{q+1}_{j=0}(-1)^j
  \rho(U_{\sigma}, U_{\partial_j \sigma}) (\delta^q f) (\partial_j
  \sigma)\\ 
& = \sum^{q+1}_{j=0}(-1)^j \sum^{q}_{k=0}(-1)^k \rho (U_{\sigma},
  U_{\partial_j \sigma}) \rho(U_{\partial_j \sigma} , U_{\partial_k
    \partial_j \sigma}) f(\partial_k \partial_j \sigma)\\ 
& = \sum^k_{j=0}\sum^{q}_{k=0}(-1)^{j+k} \rho (U_{\sigma},
  U_{\partial_ j \partial_{k+1} \sigma}) f(\partial_j \partial_{k+1}
  \sigma) +\\
& \qquad  \sum^{q+1}_{j=k+1} \sum^{q}_{k=0}(-1)^{j+k}
  \rho(U_{\sigma} , U_{\partial_k \partial_j \sigma}) f(\partial_k
  \partial_j \sigma)\\ 
& =0\\ 
 & (2)\quad (\delta^q k^{q-1} f)(\sigma) = \sum^{q}_{i=0} (-1)^i
  \rho(V_{\sigma}, V_{\partial_i \sigma}) (k^{q-1} f) (\partial_i
  \sigma)\\ 
 &= \sum^{q}_{i=0} \sum^{q-1}_{h=0}(-1)^{i+h} \rho (V_{\sigma},
  U_{\tau_h \partial_i  \sigma})  f(\tau_h \partial_i  \sigma)\\ 
 & = \sum^h_{i=0}\sum^{q-1}_{h=0}(-1)^{i+h} \rho (V_{\sigma},
  U_{\partial_i \tau_{h+1} \sigma}) f(\partial_i \tau_{h+1} \sigma)
  \\ 
& \qquad + 
  \sum^{q}_{i=h+1} \sum^{q-1}_{h=0}(-1)^{i+h} \rho(V_{\sigma} ,
  U_{\partial_{i+1}\tau_h \sigma}) f(\partial_{i+1} \tau_h \sigma)\\ 
 & = \sum^{h-1}_{i=0}\sum^{q}_{h=0}(-1)^{i+h-1} \rho (V_{\sigma},
  U_{\partial_i \tau_h \sigma}) f(\partial_i \tau_h \sigma) \\
& \qquad +
  \sum^{q+1}_{i=h+2} \sum^{q-1}_{h=0}(-1)^{i+h-1} \rho(V_{\sigma} ,
  U_{\partial_i \tau_h \sigma}) f(\partial_i \tau_h \sigma) (k^q
  \delta^{q+1} f) (\sigma)\\
& =\sum^{q}_{h=0}(-1)^h \rho (V_\sigma
  ,U_{\tau_h \sigma}) (\delta^{q+1} f) (\tau_h \sigma) \\ 
 & = \sum^q_{h=0}\sum^{q+1}_{i=0}(-1)^{i+h} \rho (V_{\sigma},
  U_{\partial_i \tau_h \sigma}) f(\partial_i \tau_h
  \sigma),  (\delta^q k^{q-1} f + k^q \delta^{q+1} f) (\sigma)\\
&  = \sum^{q}_{h=0} \rho (V_\sigma , U_{\partial_h \tau_h \sigma})
  f(\partial_h \tau_h \sigma) - \sum^{q}_{h=0} \rho(V_{\sigma},
  U_{\partial_{h+1} \tau_h \sigma}) f(\partial_{h+1} \tau_h \sigma)\\ 
 & = \sum^{q}_{h=0} \rho (V_\sigma , U_{\partial_h \tau_h \sigma})
  f(\partial_h \tau_h \sigma) - \sum^{q+1}_{h=1} \rho(V_{\sigma},
  U_{\partial_{h} \tau_{h-1} \sigma}) f(\partial_h \tau_{h-1}
  \sigma)\\ 
 & =  \rho (V_\sigma , U_{\partial_0 \tau_0 \sigma})  f(\partial_0
  \tau_0 \sigma) -  \rho (V_\sigma , U_{\partial_{q+1} \tau_q \sigma})
  f(\partial_{q+1} \tau_q \sigma)\\ 
 & =  \rho (V_\sigma , U_{\tau ' \sigma})  f(\tau ' \sigma) -  \rho
  (V_\sigma , U_{\tau \sigma})  f(\tau  \sigma)\\ 
 & =(\tau'^+ f) (\sigma ) - (\tau^+ f) (\sigma ). 
\end{align*}\pageoriginale

