\chapter{Lecture 12}

\begin{proposition}%%%%% 8 
 If\pageoriginale $X$ is paracompact and normal and
    if $\sum$ is a presheaf which determines the zero sheaf, the
   $H^q(X, \sum)=0$ for all $q \geq 0$. 
\end{proposition}

\begin{proof}
Let $f \in C^q (\mathscr{U},\sum)$ where $\mathscr{U} =\{ U_i \}_{i
  \in I}$ is any locally finite covering. Since $X$ is normal, we can
shrink $\mathscr{U}$ to $\mathscr{W}$, $\mathscr{W}=\{ W_i \} _{i \in I}$
with $\bar {W}_i \subset U_i$. For each $x \in X$ choose a
neighbourhood $V_x$  of $x$ such that the following conditions are
satisfied: 
\end{proof}
\begin{enumerate}[a)]
\item If $x \in U_i$, $V_x \subset U_i$,

\item If $x \in W_i$, $V_x \subset W_i$,

\item if $x \notin \bar{W}_i$, $V_x \cap W_i= \phi$,

\item if $x \in U_{i_o} \cap \cdots \cap U_{i_q}=U_\sigma$, $\rho_{_{V_x
  U_\sigma}} f(\sigma)=0$ 
\end{enumerate}

Conditions a) and b) can be satisfied, for the coverings
$\mathscr{U}$ and $\mathscr{W}$ being locally finite, each $x$ is
contained only in a finite number of sets of the coverings. To see
that condition c) can be satisfied, consider all $\bar{W}_i$ for
which $x \notin \bar{W}_i$. The union of these sets is the closed
since $\mathscr{W}$ is locally finite, and $x$ is in the open
complement of this union. Next, by condition a), $V_x \subset
U_\sigma$, and since $\sum$ determines the 0-sheaf, we can choose
$V_x$ small enough so that d) is satisfied. We can thus always
choose $V_x$ small enough so that the above conditions are fulfilled. 

If\pageoriginale the $V_x$ are chosen as above, the covering $\big\{
V_x \big\}_{x   \in X}$ is a refinement of $\mathscr{U}$. Choose the
function $\tau :X \to I$ so that $x \in W_{\tau (x)}$, then by b),
$V_x \subset W_{\tau (x)}\subset U_{\tau (x)}$. Then   
$$
\tau^+ f(\sigma)=\tau^+ f(x_o,\ldots ,x_q)= \rho(V_\sigma, U_ \sigma)f
(\tau (x_o), \ldots , \tau (x_q)). 
$$

If $V_ \sigma =\phi$, $\tau^+ f(\sigma)=0$. If $V_\sigma \neq \phi$
then $V_{x_o}$ meets each $V_{xj}$, hence meets each $W_{\tau (x_j)}$
and hence by c), $x_o \in \bar{W}_{\tau (x_j)}$. Then since $x_o \in
\bar {W}_{\tau (x_j)} \subset U_{\tau(x_i)}$, by a) $V_{x_o}\subset
U_{\tau (x_i)}$ for each $j$ and hence $V_{x_o}\subset U_{\tau
  (\sigma)}$. Hence 
\begin{align*}
\tau ^+ f(\sigma)= \rho_{V_\sigma U_{\tau(\sigma)}} f(\tau \sigma) & =
\rho_{V_\sigma V_{x_0}} \rho_{V_{x_0}} U_{\tau (\sigma)} f(\tau
\sigma)\\ 
&=0 \text{ by } d).
\end{align*}

Thus $\tau^+ f(\sigma)=0$ for all $\sigma$, hence $\tau ^+ f=0$.

If $\mathscr{U}_1$ is a proper covering and $f \in C^q (\mathscr{U}_1,
\sum)$, there is a locally finite refinement $\mathscr{U}$ of
$\mathscr{U}_1$ (since $X$ is paracompact). Then there is a refinement
$\mathscr{W}$ of $\mathscr{U}$ (found as above), a proper refinement
$\mathscr{W}_1$ of $\mathscr{W}$ (existence of $\mathscr{W}_1$ is
trivial) and a function $\tau _1:\mathscr{W}_1 \to \mathscr{U}_1$ with
$V \subset \tau_1 (V)$ for $V \in \mathscr{W}_1$ such that $\tau^+_1
f=0$. Hence every element of $\bigcup \limits_{(\mathscr{U}_1
  \text{proper})}H^q (\mathscr{U}_1 ,\sum)$ is equivalent to zero,
i.e., the direct limit $H^q(X, \sum)$ consists only of zero. 

\begin{exam}%% 15
Let $X$ be the space with four points $a$, $b$, $c$, $d$ and let a
base for the 
open sets be the sets $(a,c,d)$, $(b,c,d)$, $(c)$, $(d)$. Let $\sum$ be the
presheaf for which $S_U=z$, the group  of integers, if $U=(c,d)$; and
$S_U=0$ otherwise. The\pageoriginale homomorphisms $\rho_{VU}$ are the obvious
ones. Then $\sum$ determines the 0-sheaf, but $H^1(X,\sum)=Z$. The
space $X$ paracompact but not normal. 
\end{exam}

\textit{If $0 \to \mathscr{S}' \xrightarrow{i}
  \mathscr{S}\xrightarrow{j} \mathscr{S}''$ is an exact sequence of
  sheaves, then} 
$$
0 \to \Gamma(U,\mathscr{S}') \xrightarrow{i_U}\Gamma (U,
\mathscr{S})\xrightarrow{j_U}\Gamma (U, \mathscr{S}'') 
$$
\textit{is exact and hence} 
$$
0 \to \bar{\mathscr{S}} \to \bar{\mathscr{S}} \to \bar{\mathscr{S}}'' 
$$
\textit{is exact}

\begin{proof}
We will show that for ker $j_U\subset im i_U$ (the rest is
trivial). Let $f \in \ker j_U$. Then, $x \in U$, $jf (x)=(j_Uf)(x)=0_x$
and hence by exactness, $f(x)=ip'$ for some $p' \in S'_x$. Thus
$f(U)\subset i(S')$. But $i:S' \to i(S')$ is homeomorphism. Then $g:U
\to S'$, where $g(x)=1^{-1}f(x)$, is a section of $S'$ over $U$, and
$f=i_U g$.  
\end{proof}

One cannot in general complete the sequence

$$
0 \to \Gamma(U,\mathscr{S}')  \to \Gamma (U, \mathscr{S}) \to \Gamma 
(U, \mathscr{S}'') 
$$
by a zero on the right as the following example shows.

\begin{exam}\label{chap12:exam16}%% 16
Let $X$ be the segment $\{ x:0 \le x \le1 \}$. Let $G$ be the
4-group with elements $0$, $a$, $b$, $c$. Let $\mathscr{S}$ be the subsheaf
of the constant sheaf $G=(X \times G, \pi ,X)$ formed by omitting the
point $(0,a)$, $(0,c)$, $(1,b)$, $(1,c)$. Let $\mathscr{S}'$ be the subsheaf of
$\mathscr{S}$ formed by omitting all the points $(x,a)$,
$(x,b)$. Let\pageoriginale 
$\mathscr{S}''=(X \times Z_2 \pi,X)$ and let $j:\mathscr{S}\to
\mathscr{S}''$ be the homomorphism induced by $j: G \to Z_2$ where
$j(a)=j(b)=1$, $j(c)=j(0)=0$. The the sequence 
$$
0 \to \mathscr{S}' \xrightarrow{i} \mathscr{S}\xrightarrow{j}
\mathscr{S}'' \to 0 
$$
is exact, but the sequence
\begin{align*}
&0 \to \Gamma(X,\mathscr{S}')  \to \Gamma (X, \mathscr{S}) \to \Gamma
  (X, \mathscr{S}'') \to O, \quad i.e.,\\ 
&0 \longrightarrow O \longrightarrow O \longrightarrow Z_2
  \longrightarrow O 
\end{align*}
is not exact.
\end{exam}

\textit{If $0 \to \sum ' \xrightarrow{i} \sum \xrightarrow{j} \sum''$
  is an exact sequence of presheaves, there is an image presheaf
  $\sum''_0 \subset \sum''$ and a quotient presheaf $Q$ such that the 
  sequences} 
\begin{align*}
&O \to \sum ' \xrightarrow{i} \sum \xrightarrow{j_o} \sum''_0\to 0,\\
&O \to \sum''_0 \xrightarrow{\bar{i}}\sum'' \xrightarrow{\bar{j}} Q \to 0
\end{align*}
\textit{are exact. These sequences are `natural' in the sense that if
  $h$ is a homomorphism of exact sequences, commuting with $i$ and
  $j$:} 
\[
\xymatrix{
0\ar[r] & \sum' \ar[r]^{i}\ar[d]_h & \sum\ar[r]^{j} \ar[d]_h &
\sum''\ar[d]_h\\
0\ar[r] & \sum'_1 \ar[r]^i & \sum_1 \ar[r]^j & \sum''_1
}
\]
\textit{then there are induced homomorphisms $h^*$ of the exact
  cohomology sequences} 
{\fontsize{9}{11}\selectfont
\[
\xymatrix@C=0.45cm{
\ldots \ar[r]& H^q(X,\sum') \ar[r]^{i^{\ast}} \ar[d]_{h^{\ast}} &
H^q(X,\sum) \ar[r]^{j^{\ast}_0} \ar[d]_{h^{\ast}} & H^q(X,\sum''_0)
\ar[r]^{\delta^{\ast}_0} \ar[d]_{h^{\ast}} & H^{q+1}(X,\sum')
\ar[r]\ar[d]_{h^{\ast}} & \ldots\\
\ldots \ar[r] & H^q(X,\sum'_1) \ar[r]^{i^{\ast}} & H^q(X,\sum_1)
\ar[r]^{j^{\ast}_0} & H^q(X,\sum''_{1 \; 0}) \ar[r]^{\delta^{\ast}_0} &
H^{q+1}(X,\sum'_1) \ar[r] & \ldots
}
\]}\relax
and\pageoriginale
{\fontsize{9}{11}\selectfont
\[
\xymatrix@C=0.45cm{
\ldots \ar[r] & H^q(X,\sum''_0) \ar[r]^{\bar{i}^{\ast}}
\ar[d]_{h^{\ast}} & H^q(X,\sum'')
\ar[r]^{\bar{j}^{\ast}}\ar[d]_{h^{\ast}} &
H^q(X,Q)\ar[r]^{\bar{\delta}^{\ast}}\ar[d]_{h^{\ast}} &
H^{q+1}(X,\sum''_0) \ar[r]\ar[d]_{h^{\ast}} & \ldots\\
\ldots \ar[r] & H^q(X,\sum''_{10}) \ar[r]^{\bar{i}^{\ast}} &
H^q(X,\sum''_1) \ar[r]^{\bar{j}^{\ast}} & H^q(X,Q_1)
\ar[r]^{\bar{\delta}^{\ast}} & H^{q+1}(X,\sum''_{10}) \ar[r] & \ldots
}
\]}\relax
\textit{commuting with $i^\ast$, $j^\ast_0$, $\delta^\ast_0$ and
  $\bar{i}^\ast$, $\bar{j}^\ast$, $\bar{\delta}^\ast$ respectively.}  

\begin{proof}
If $\sum''= \{ S''_U ,\rho''_{VU} \}$, let $S''_{oU}= im j_U$. Then
since $j: \sum \to \sum''$ is a homomorphism, $\rho''_{VU}$ maps $im
j_U$ into $im j_V$. Hence, writing $S''_{oU}=im j_U$ and $Q_U=S''_U/
S''_{oU}$, there are induced homomorphisms $\rho''_{oVU}$ and $\bar
{\rho}_{VU}$ with comutativity in  
\end{proof}
\[
\xymatrix{
S_U\ar[r]^{j_{oU}} \ar[d]_{\rho_{VU}} & S''_{oU} \ar[r]^{\bar{i_U}}
\ar[d]_{\rho''_{oVU}} & S''_U \ar[r]^{\bar{j}_U}\ar[d]_{\rho''_{VU}} &
Q_U\ar[d]_{\bar{\rho}_{VU}} \\
S_V \ar[r]^{j_{oV}} & S''_{oV} \ar[r]^{\bar{i}_V} & S''_V
\ar[r]^{\bar{j}_V} & Q_V \quad .
}
\]

Clearly the systems $\sum''_o= \{ S''_{oU}, \rho''_{oVU} \}$ and $Q=\{
Q_U, \bar{\rho}_{VU} \}$ are presheaves and the sequences 
$$
O \to \Sigma' \xrightarrow{i} \Sigma\xrightarrow{j} \Sigma''_o\to O
$$
and
$$
O \to \Sigma''_o \xrightarrow{\bar{i}} \Sigma''\xrightarrow{\bar{j}} Q \to O
$$
are exact. Since $h$ commutes with $i$, $j$,
\[
\xymatrix{
0\ar[r] & \sum'\ar[r]^i\ar[d]^{h'} & \sum\ar[r]^j\ar[d]^h &
\sum''\ar[d]^{h''}\\
0 \ar[r] & \sum'_1 \ar[r]^{i_1}& \sum_1 \ar[r]^{j_1} & \sum''_1
}
\qquad 
\xymatrix{
S_U\ar[r]^{j_U}\ar[d]_{h_U} & S''_U\ar[d]^{h''_U} \\
S_{1U} \ar[r]^{j_{1U}} & S''_{1U}
}
\]\pageoriginale
$h''_U $ maps $S''_{oU} = im j_U$ into $S''_{1oU} = im j_{1U}$. Hence there
are induced homomorphisms $h''_{oU}$, $\bar{h}_U$ with comutativity in 
\[
\xymatrix{
S_U \ar[r]^{j_{oU}} \ar[d]_{h_U} & S''_{oU} \ar[r]^{\bar{i}_U}
\ar[d]^{h''_{oU}} & S''_U\ar[r]^{\bar{j}_U} \ar[d]^{h''_U}&
Q_U\ar[d]^{\bar{h}_U}\\
S_{1U} \ar[r]^{j_{oU}} & S''_{1oU} \ar[r]^{\bar{i}_U} &
S''_{1U}\ar[r]^{\bar{j}_U} & Q_{1u} \quad .
}
\]

Since $h$ is a homomorphism of presheaves, $h_U$ commutes with
$\rho_{VU}$ and $h''_U$ with $\rho''_{VU}$. Hence, since $j_{oU}$ and 
$\bar{j}_U$ are epimorphisms and $j_{oU}$, $\bar{j}_U$ commute with $\rho$
and $h$, $h''_{oU}$ and commutes with $\rho''_{oVU}$ and $\bar{h}_U$ with
$\bar{\rho}_{VU}$, i.e., the diagrams given below are commutative: 
{\fontsize{9}{11}\selectfont
\[
\xymatrix@C=0.5cm{
S''_{oU}\ar[rrr]^{\rho''_{oVU}}\ar[ddd]_{h''_{oU}} & & &
S''_{oV}\ar[ddd]^{h''_{oV}}\\
& S_U\ar[r]^{\rho_{_{VU}}}\ar[d]_{h_U} \ar[ul]_{j_{oU}} & S_V \ar[d]^{h_V}
\ar[ur]^{j_{oV}} & \\
& S_{1U} \ar[r]^{\rho_{_{VU}}} \ar[dl]_{j_o} & S_{1V} \ar[dr]^{j_o} & \\
S''_{1oU} \ar[rrr]^{\rho''_{oVU}} & & & S''_{1oV}
}
\qquad
\xymatrix@C=0.5cm{
Q_U \ar[rrr]^{\bar{\rho}_{_{VU}}}\ar[ddd]^{\bar{h}_{U}} & & &
Q_{V}\ar[ddd]^{\bar{h}_{V}}\\
& S''_U\ar[r]^{\rho''_{VU}}\ar[d]_{h''_U} \ar[ul]_{\bar{j}_{U}} &
S''_V \ar[d]^{h''_V} \ar[ur]^{\bar{j}_{V}} & \\
& S''_{1U} \ar[r]^{\rho''_{VU}} \ar[dl]_{\bar{j}} & S''_{1V}
\ar[dr]^{\bar{j}} & \\ 
Q_{1U} \ar[rrr]^{\bar{\rho}_{_{VU}}} & & & Q_{1V}
}
\]}\relax


Thus $h$, $h''_o$, $h''$, $\bar{h}$ are homomorphisms of presheaves
commuting with $j_o$, $\bar{i}$, $\bar{j}$.  
\[
\xymatrix{
\sum\ar[r]^{j_o}\ar[d]_h  & \sum''_o \ar[r]^{\bar{i}}\ar[d]_{h''_o} &
\sum''\ar[r]^{\bar{j}} \ar[d]^{h''} & Q \ar[d]^{\bar{h}}\\
\sum_1 \ar[r]^{j_o} & \sum''_{1o} \ar[r]^{\bar{i}} & \sum''_1
\ar[r]^{\bar{j}} & Q_1 \quad .
}
\]

Thus\pageoriginale $\{ h\}$ is a homomorphism of the exact sequences
\[
\xymatrix{
0\ar[r] & \sum' \ar[r]^{j} \ar[d]_{h'} & \sum \ar[r]^{j_o}\ar[d]_h &
\sum''_o \ar[r] \ar[d]_{h''_o} & 0\\
0 \ar[r] & \sum'_1 \ar[r]^i & \sum_1 \ar[r]^{j_o} & \sum''_{1o} \ar[r]
& 0
}
\]
\noindent
commuting with $i$ and $j_o$, and a homomorphism of the exact sequences
\[
\xymatrix{
0 \ar[r] & \sum''_o \ar[r]^{\bar{i}}\ar[d]_{h''_o} &
\sum''\ar[r]^{\bar{j}}\ar[d]_{h''} & Q \ar[d]_{\bar{h}} \ar[r] & 0\\
0 \ar[r] & \sum''_{1o} \ar[r]^{\bar{i}} & \sum''_1 \ar[r]^{\bar{j}} &
Q_1 \ar[r] & 0
}
\]

\noindent
commuting with $\bar{i}$ and $\bar{j}$. Therefore the induced
homomorphisms of the exact cohomology sequences commute with $i^*$, $j^*_o$,
$\delta ^*_o$ and $\bar{i}^*$, $\bar{j}^*$, $\bar{\delta}^*$ respectively. 

