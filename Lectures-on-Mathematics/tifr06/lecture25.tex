\chapter{Lecture 25}\label{chap25:lec25} % lecture 25

\begin{proposition}\label{chap25:prop18}% proposition 18
If\pageoriginale the complex of sheaves $\{\mathscr{S}^q \}$ is homotopically
    fine, there is an isomorphism $\rho : H^q C_\Phi (\mathscr{S})
    \rightarrow H^q \Gamma_{\Phi} (X,\mathscr{S})$. 

If the complex of sheaves $\{\mathscr{S}'^{q} \}$ is also
  homotopically fine, and $h : \mathscr{S}'^{q} \rightarrow
  \mathscr{S}^q$ are homomorphisms commuting with $d^q$, then $h^*$
  commutes with $\rho$. 
\[
\xymatrix{
H^qC_{\Phi} (\mathscr{S}') \ar[r]^{\rho} \ar[d]_{h^{\ast}} &
H^q\Gamma_{\Phi} (X,\mathscr{S}') \ar[d]^{d^{\ast}}\\
H^q C_{\Phi} (\mathscr{S}) \ar[r]^{\rho} &
H^q\Gamma_{\Phi}(X,\mathscr{S}). 
}
\]
\end{proposition}

\begin{proof}
The system $C_\Phi (\mathscr{U},\mathscr{S}) = \{ (C^p_\Phi
(\mathscr{U}, \mathscr{S}^q)) \}$ of double complexes is bounded above
by $p = 0$. Since (see Lecture \ref{chap24:lec24}) $H^{p,q}_{12}  C_\Phi
(\mathscr{S}) = 0$ for $p > 0$, and $H^{p,q}_{12}  C_\Phi
(\mathscr{S}) = 0$ trivially for $p < 0$, by Proposition
\ref{chap20:prop16}, there  is an isomorphism  
$$
\theta : H^q  C_{\Phi} (\mathscr{S}) \rightarrow H^{o,q}_{12} 
C_\Phi (\mathscr{S}). 
$$

Since $h : C^p_\Phi (\mathscr{U},\mathscr{S}'^q) \rightarrow C^p_\Phi
(\mathscr{U}, \mathscr{S}^q)$ commutes with $d$, $\delta$ and
$\phi_{\mathscr{W}\mathscr{U}}, h : C_\Phi (\mathscr{U},\mathscr{S}')
\rightarrow C_\Phi (\mathscr{U}, \mathscr{S})$ is a map of double
complexes which commutes with $\phi_{\mathscr{W}\mathscr{U}}$.
 Therefore $h^*$ commutes with $\theta$.  
\end{proof}

Since $\Gamma_\Phi (X, \mathscr{S}^q) = H^o_\Phi (X,\mathscr{S}^q) =$
dir $\lim H^o_\Phi (\mathscr{U},\mathscr{S}^q)$ and the homomorphism
$\tauup_{\mathscr{U}} : H^o_\Phi (\mathscr{U}, \mathscr{S}^q)
\rightarrow \Gamma_\Phi (X,\mathscr{S}^q)$ commutes with $d^q$, there
are induced homomorphisms 
\begin{align*}
& \tauup^*_{\mathscr{U}} : H^q ~ H^o_{\Phi} (\mathscr{U}, \mathscr{S})
  \rightarrow H^q (\Gamma_{\Phi} (X,\mathscr{S}),\\ 
& \tauup^* : \text{dir} \lim H^q ~ H^o_\Phi
  (\mathscr{U},\mathscr{S})\rightarrow H^q \Gamma_\Phi
  (X,\mathscr{S}). 
\end{align*}\pageoriginale 

Since the operation of forming cohomology groups commutes with the
operation of forming direct limits, (see Cartan - Eilenberg,
Homological Algebra, Proposition 9.3$^*$, p. 100), $\tauup^*$ is an
isomorphism, and since $h$ commutes with $\tauup_\mathscr{U}$, $h^*$
commutes with $\tauup^*$. 

Now,
\begin{align*}
H^{o,q}_1  C_\Phi (\mathscr{U},\mathscr{S}) & =
H^o_\Phi(\mathscr{U},\mathscr{S}^q),\\ 
H^{o,q}_{12} C_\Phi (\mathscr{U},\mathscr{S}) & = H^q
H^o_\Phi(\mathscr{U},\mathscr{S}),\\ 
H^{o,q}_{12} C_\Phi (\mathscr{S}) & = \text{ dir } \lim H^q ~
H^o_\Phi(\mathscr{U},\mathscr{S}). 
\end{align*}

Thus we have an isomorphism
$$
\tauup^* : H^{o,q}_{12}  C_\Phi (\mathscr{S}) \rightarrow H^q
\Gamma_{\Phi} (X,\mathscr{S}) 
$$
which commutes with $h^*$. Let $\rho = \tauup^* \theta$ be the
composite isomorphism then $\rho$ commutes with $h^*$. 
\[
\xymatrix{
H^qC_{\Phi}(\mathscr{S}')\ar[r]^{\theta}\ar[d]_{h^{\ast}} &
H^{o,q}_{12} C_{\Phi}(\mathscr{S}')
\ar[r]^{\tau^{\ast}}\ar[d]_{h^{\ast}} & H^q\Gamma_{\Phi}
(X,\mathscr{S}')\ar[d]_{h^{\ast}} \\
H^qC_{\Phi} (\mathscr{S}) \ar[r]^{\theta} & H^{0,q}_{12} C_{\Phi}
(\mathscr{S}) \ar[r]^{\tau^{\ast}} & H^q \Gamma_{\Phi} (X,\mathscr{S}).
}
\]

\begin{thm}[Uniqueness Theorem]% theorem 3
 Let
$$
\cdots \rightarrow \mathscr{S}^{q-1} \xrightarrow{d^q} \mathscr{S}^q
\xrightarrow{d^{q+1}} \mathscr{S}^{q+1} \rightarrow \cdots 
$$
be\pageoriginale a homotopically fine complex of sheaves of $A-$
modules with $\im 
{d^q} = \ker d^{q+1}$ for $q \neq 0$ and $\im {d^o} \subset \ker
{d^1}$ and let ${\mathscr{H}^o} = \ker {d^1} / \im {d^o} $. Let
$\{\mathscr{S}'^{q}\}$ be another such complex with $\mathscr{H}'^o$
isomorphic to $\mathscr{H}^o$. 

If $\Phi -\dim X$ is finite or the degrees of $\{\mathscr{S}'^q \}$
and $\{\mathscr{S}^q \}$ are bounded below, then any isomorphism
$\lambda : \mathscr{H}'^o \rightarrow \mathscr{H}^o$ induces an
isomorphism 
$$
\phi_\lambda : H^q \Gamma_\Phi (X,\mathscr{S}') \rightarrow H^q
\Gamma_\Phi (X,\mathscr{S}) 
$$
and if $\{\mathscr{S}''^{q} \}$ is another such complex and
  $\mu : \mathscr{H}^o \rightarrow \mathscr{H}''^o$ is an isomorphism,
  then  
$$
\phi_{\mu\lambda} =  \phi_\mu \phi_\lambda : H^q
\Gamma_\Phi(X,\mathscr{S}') \rightarrow H^q \Gamma_\Phi
(X,\mathscr{S}''). 
$$

If $h : \mathscr{S}'^q \rightarrow \mathscr{S}^q$ are homomorphisms
(for each $q$) commuting with $d^q$, and if the induced homomorphism
$h : \mathscr{H}'^o \rightarrow \mathscr{H}^o$ is an isomorphism, then
the homomorphism $h^*: H^q \Gamma_\Phi (X,\mathscr{S}') \rightarrow H^q
\Gamma_\Phi (X,\mathscr{S}$ is the isomorphism $\phi_h$. 
\end{thm}

\begin{proof}%Prf
Since the hypotheses of Propositions \ref{chap23:prop17} and
\ref{chap25:prop18}  are satisfied, there are isomorphisms 
\begin{align*}
& \eta : H^q C_\Phi (\mathscr{S}) \rightarrow H^q_\Phi
  (X,\mathscr{H}^o),\\ 
& \rho : H^q C_\Phi (\mathscr{S}) \rightarrow H^q \Gamma_\Phi
  (X,\mathscr{S}). 
\end{align*}

Since $\lambda : \mathscr{H}'^o \rightarrow \mathscr{H}^o$ is an
isomorphism, so is  
$$
\lambda^* : H^q_\Phi (X,\mathscr{H}'^o) \rightarrow H^q_\Phi
(X,\mathscr{H}^o). 
$$
 
Let\pageoriginale $\phi_\lambda$ be the isomorphism $\rho \eta^{-1}
\lambda^*  \eta\rho^{-1}$. Since $(\mu\lambda)^* = \mu^* \lambda^*$, 
$\phi_{\mu\lambda} = \phi_\mu  \phi_\lambda$. 
\[
\xymatrix{
H^q \Gamma_{\Phi} (X,\mathscr{S}') \ar[d]^{\phi_{\lambda}} &
H^qC_{\Phi}(\mathscr{S}') \ar[r]^{\eta} \ar[l]_{\rho} & H^q_{\Phi}
(X,\mathscr{H}'^0)\ar[d]^{\lambda^{\ast}} \\
H^q \Gamma_{\Phi} (X,\mathscr{S}) \ar[d]_{\phi_{\mu}} & H^qC_{\Phi}
(\mathscr{S}) \ar[r]^{\eta} \ar[l]_{\rho} & H^q_{\Phi}
(X,\mathscr{H}^0)\ar[d]^{\mu^{\ast}} \\
H^q \Gamma_{\Phi} (X,\mathscr{S}'') & H^qC_{\Phi}(\mathscr{S}'')
\ar[r]^{\eta} \ar[l]_{\rho} & H^q_{\Phi} (X,\mathscr{H}''^0).
 }
\]

If $h : \mathscr{S}'^q \rightarrow \mathscr{S}^q$ are homomorphisms
commuting with $d^q$ and inducing an isomorphism $h : \mathscr{H}'^o
\rightarrow \mathscr{H}^o$, then it follows from the commutative
diagram: 
\[
\xymatrix{
H^q\Gamma_{\Phi} (X,\mathscr{S}') \ar[d]_{\phi_h}^{h^{\ast}} & H^q
C_{\Phi} (\mathscr{S}') \ar[r]\ar[d]^{h^{\ast}} \ar[l] & H^q_{\Phi}
(X,\mathscr{H}'^0) \ar[d]^{h^{\ast}}\\
H^q \Gamma_{\Phi} (X,\mathscr{S}) & H^qC_{\Phi} (\mathscr{S})
\ar[r]\ar[l] & H^q_{\Phi} (X,\mathscr{H}^0),
 }
\]
that $\phi_h = h^* : H^q \Gamma_\Phi (X,\mathscr{S}') \rightarrow H^q
\Gamma_\Phi (X,\mathscr{S})$. 
\end{proof}

