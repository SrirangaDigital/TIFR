\chapter{Lecture 30}\label{chap30:lec30}%Chap 30

\begin{defi*}%Def
The\pageoriginale following properties of sheaves $\mathscr{S}$ of
$\mathfrak{a}$- modules over a space $X$ are called property
$(b_1)$ and property $(b)$.  
\end{defi*}

\medskip
\noindent{\textbf{Property ({\boldmath$b_1$}).}}
\textit{For each open set $U$ of $X$ and each homomorphism $f :
  \sum\limits_{i \in I} \mathfrak{a}_U \to \mathscr{S}_U$, ker $f$
  has property $(a_1)$ as a sheaf of $\mathfrak{a}_U$- modules.} 

\medskip
\noindent{\textbf{Property (b).}}
\textit{For each open set $U$ of $X$ and each homomorphism $f :
  \sum^k_{i=1} a_U \to \mathscr{S}_U$, $\ker f$ has property $(a)$ as a
  sheaf of $a_U$- modules.}

\begin{note*}
Since $(a_1)$ and $(a)$ are local properties, properties
$(b_1)$ and $(b)$ are equivalent to the following properties $(b'_1)$
and $(b')$ respectively. 
\end{note*}

\medskip
\noindent{\textbf{Property ($b'_1$).}}
\textit{For each neighbourhood $V$ of each point $x \in X$, there
  exists an open set $U$, $x \in U \subset V$, such that for each
  homomorphism $f : \sum\limits_{i \in I} \mathfrak{a}_U \to \mathscr{S}_U$,
  $\ker f$ has property $(a_1)$ as a sheaf of $\mathfrak{a}_U$- modules. }  

\medskip
\noindent{\textbf{Property ($b'$).}}
For each neighbourhood $V$ of each point $x \in X$, there exists an
open set $U$, $x \in U \subset V$ such that for each homomorphism  $f
: \sum\limits^{k}_{i=1} \mathfrak{a}_U \to \mathscr{S}_U$, $\ker f$
has property $(a)$ as a sheaf of $\mathfrak{a}_U$-modules.  
 
Thus $(b_1)$ and $(b)$ are also local properties. The\pageoriginale
sheaf $\ker f$ is called the \textit{sheaf of relations} between the
sections $f_i : U \to \mathscr{S}_U$, where $f_i =f h^i 1$,  
$$
U  \xrightarrow{1} \mathfrak{a}_U  \xrightarrow {h^i} \sum_{i \in I}
\mathfrak{a}_U \xrightarrow{f} \mathscr{S}_U. 
$$

The sheaf of relations between the sections $f_i$ is described by the
following result. 

\textit{The element of $(\ker f)_x$ for $x \in U$ are the elements 
  $\sum_i a_i$ of which $\sum_i a_i f_i (x)=0$.} 

\begin{proof}
For each element $\sum a_i$, only a finite number of the $a_i$ being
different from zero, we have 
\begin{align*}
\sum a_i & = \sum^q_{j=1} h^{i_j} a_{i_j} \text{ and }\\
f(\sum a_i) & = f (\sum^q_{j=1}h^{i_j} a_{i_j}) = \sum^q_{j=1} f
h^{i_j}(a_{i_j})\\ 
& = \sum^q_{j=1} a_{i_j} f h^{i_j}(l_x)=
\sum^q_{j=1}a_{i_j}f_{i_j}(x)\\ 
& = \sum_i a_i f_i (x),
\end{align*}
and this completes the proof.
\end{proof}

If we start with a system $\{f_i\}_{i \in I}$ of sections of
$\mathscr{S}$ over an open set $U$, then each $f_i$ defines a
homomorphism (again denoted by $f_i$) $f_i :\mathfrak{a}_U \to
\mathscr{S}_U$ where $f_i (a) = a \cdot f_i(x)$, $a \in A_x$. Then the
system $\{f_i\}$ of homomorphisms defines a homomorphism $f :
\sum\limits_{i \in I} \mathfrak{a}_U \to \mathscr{S}_U$, and the sheaf
$\ker f$ is called the \textit{sheaf of relations} between the
sections $f_i : U \to \mathscr{S}_U$.  

(2) \qquad {\textit If\pageoriginale $\mathscr{S}$ has property $(b_1)$ (resp
$(b)$), then every subsheaf of $\mathscr{S}$ has property
  $(b_1)$ (resp $(b)$)}. 

\begin{proof}
Clear.
\end{proof}

\begin{defi*}%Def
 A sheaf $\mathscr{S}$ of $\mathfrak{a}$- modules is called coherent
 if it   has properties $(a)$ and $(b)$.  
\end{defi*}

\begin{note*}
 If $\mathscr{S}$ is a coherent sheaf, then $\mathscr{S}_U$
is coherent for each open set $U$. Coherence is a local property,
i.e., if each point has a neighbourhood $U$ such that
$\mathscr{S}_U$ is coherent, then $\mathscr{S}$ is coherent. If we
define a sheaf $\mathscr{S}$ to be of \textit{finite type} if, for
each $x \in X$, there is an open set $U$, $x \in U$, such that each
stalk of $\mathscr{S}_U$ is generated by the same \textit{finite}
number of sections $f_1, \ldots,f_k$ over $U$; it is then easily
verified that conditions $(a)$ and $(b)$ for coherence are equivalent
to the following conditions: 
\begin{enumerate}[(i)]
\item The sheaf $\mathscr{S}$ is of finite type.

\item If $f_1, \ldots,f_k$ are any finite number of sections of
  $\mathscr{S}$ over any open set $U$, then the sheaf of relations
  between these finite number of sections is of finite type. 
\end{enumerate}
\end{note*}

\begin{exam}%Exmple 30
Let $\mathfrak{a}$ be the constant sheaf $Z_2$ on $0 \leq x \leq 1$, let
$\mathscr{R}$ be the subsheaf obtained by omitting (1, 1) and let
$\mathscr{S}$ be the quotient sheaf with stalks $Z_2$ at 1 and zero
elsewhere. Then the natural homomorphism $\mathfrak{a} \to \mathscr{S}$ has
kernel $\mathscr{R}$ and $\mathscr{R}$ does not have property
$(a_1)$. Hence $\mathscr{S}$ has neither $(b_1)$ nor $(b)$. 
\end{exam}

\begin{exam}%Exmple 31
Let\pageoriginale $A$ be the ring of Example \ref{chap2:exam5}, with
elements 0, 1, 
$b$, $c$, such 
that $b^2 = b$, $c^2=c$, $bc = cb = 0$. Let $\mathfrak{a}$ be the
subsheaf of the 
constant sheaf $A$ on $0 \leq x \leq 1$ obtained by omitting $(1, b)$
and $(1, c)$. Let $\mathscr{S}$ be the constant sheaf $Z_2$ on which
$1$ and $c$ of $A$ operate as the identity and $b$ operates as
zero. The sheaf $\mathscr{R}$ of relations for the section 1 of
$Z_2$ consists of 0 and $(x,b)$ for $x < 1$. If $U$ is a connected
neighbourhood of 1, the only homomorphism of $\mathfrak{a}_U$ into
$\mathscr{R}_U$ is the zero homomorphism. Thus $\mathscr{R}$ does not
have $(a_1)$ and $\mathscr{S}$ has neither $(b_1)$ nor $(b)$. 
\end{exam}

\begin{exam}%Exmple 32
Let $A$ be the ring $Z [ y_1, y_2,\ldots ]$ of polynomials in
infinitely many variables with integer coefficients. Let
$\mathfrak{a}$ be the 
constant sheaf $A$ on $0 \leq x \leq 1$ and let $\mathscr{S}$ be the
constant sheaf $Z$ on which all the indeterminates $y_1, y_2, \ldots$,
operate as zero. The sheaf of relations for the section 1 of $Z$ is
the constant sheaf formed by the ideal of all polynomials without
constant terms. This ideal is not finitely generated, hence
$\mathscr{S}$ does not have $(b)$. However, for every homomorphism $f
: \sum\limits_{i \in I} \mathfrak{a}_U \to \mathscr{S}_U$, $\ker f$ is
constant on each component of $U$ and hence has property $(a_1)$. Thus 
$\mathscr{S}$ has $(b_1)$ but not $(b)$. 
\end{exam}

\noindent
(3) \qquad \textit {If $ 0 \to \mathscr{S}' \xrightarrow{f}
  \mathscr{S}  \xrightarrow{g} \mathscr{S}'' \rightarrow 0$ is an
  exact sequence of sheaves of $\mathfrak{a}$ - modules such that
  $\mathscr{S}$ has $(a_1)$ (resp $(a)$) and $\mathscr{S}''$ has
  $(b_1)$ (resp $(b)$), then $\mathscr{S}'$ has $(a_1)$ (resp
  $(a)$).}  

\begin{proof}
If $\mathscr{S}$ has $(a_1)$, each $x \in X$ has a neighbourhood $U$
for\pageoriginale which there is an epimorphism $\phi : \sum\limits_{i
  \in I} \mathfrak{a}_U 
\to \mathscr{S}_U$. Since $g \phi : \sum \limits_{i \in I} \mathfrak{a}_U \to
\mathscr{S}''_U$ is a homomorphism and $\mathscr{S}''$ has $(b_1)$,
$\ker g \phi $ has $(a_1)$. Hence for some open set $V$ with $x \in V
\subset U$, there is a homomorphism 
$$
\psi: \sum_{j \in J} \mathfrak{a}_V \to \sum_{i \in I} \mathfrak{a}_V
$$
such that $\im \psi = (\ker g \phi )_V$. Hence, since $\phi$ is an
epimorphism $\im \phi \psi = (\ker g)_V$, and therefore $\im \phi \psi
= (\im f)_V$. Then, since $f$ is a monomorphism, there is an
epimorphism 
$$
f^{-1} \phi \psi : \sum_{j \in J} \mathfrak{a}_V \to \mathscr{S}'_V.
$$

Thus $\mathscr{S}'$ has $(a_1)$. Similarly if $\mathscr{S}$ has $(a)$
and $\mathscr{S}''$ has $(b)$, then $\mathscr{S}'$ has $(a)$. 
\end{proof}
\[
\xymatrix{
& \sum\limits_{j\in J} \mathfrak{a}_V \ar[r]^{\psi} & \sum\limits_{i \in I}
  \mathfrak{a}_V \ar[d]_{\phi} \ar[dr]^{g\phi} & & \\
0 \ar[r] & \mathscr{S}' \ar[r]^{f} & \mathscr{S} \ar[r]^g &
\mathscr{S}'' \ar[r] & 0
}
\]
\hfill{Q.e.d.}

\noindent
(4) \qquad \textit{If $0 \to \mathscr{S}' \overset{f}
  \longrightarrow \mathscr{S} \overset{g} \longrightarrow
  \mathscr{S}'' \longrightarrow 0 $ is an exact sequence of sheaves of
  $\mathfrak{a}$- modules such that $\mathscr{S}'$ and $\mathscr{S}''$ have
  $(b_1)$ (resp $(b)$) then $\mathscr{S}$ also has} $(b_1)$ (resp
$(b)$). 

\begin{proof}
Let $\mathscr{S}'$ and $\mathscr{S}''$ have $(b_1)$ and let $\phi :
\sum\limits_{i \in I} \mathfrak{a}_U \to \mathscr{S}_U$ be a given
homomorphism. Since $\mathscr{S}''$ has $(b_1)$,\pageoriginale $\ker g
\phi $ has $(a_1)$ and hence for each $x \in U$, there is an open set
$V$ with $x\in V \subset U$ and a homomorphism, 
$$
\psi : \sum_{j \in J} \mathfrak{a}_V \to \sum_{i \in I} \mathfrak{a}_V
$$
such that $\im \psi = (\ker g \phi)_V$. Then $\im \phi \psi \subset
(\ker g)_V = (\im f)_V$ and hence there is a homomorphism 
$$
\theta : \sum_{j \in J} \mathfrak{a}_V \to \mathscr{S}'_V
$$
with $f \theta = \phi \psi$. Since $\mathscr{S}'$ has $(b_1)$, $\ker
\theta $ has $(a_1)$ and there is an open set $W$ with $x \in W
\subset V$, and a homomorphism 
$$
\eta : \sum_{k \in K} \mathfrak{a}_W \to \sum_{j \in J} \mathfrak{a}_W
$$
such that $\im \eta = (\ker \theta)_W$.
\end{proof}
\[
\xymatrix{
& & & & & \sum\limits_{k\in K} \mathfrak{a}_W \ar[dll]_{\eta}
  \ar[ddl]^{\psi \eta}& \\
& & & \sum\limits_{j\in J} \mathfrak{a}_V
  \ar[dr]_{\psi}\ar[ddll]_{\theta} & & \\ 
& & & &  \sum\limits_{i\in I} \mathfrak{a}_U\ar[dl]_{\phi}\ar[dr]^{g\phi} & & \\
0 \ar[r] & \mathscr{S}'\ar[rr]^f & & \mathscr{S} \ar[rr]^g && 
\mathscr{S}'' \ar[r] & 0
}
\]

To show that $\mathscr{S}$ has $(b_1)$ it is enough to show that $\im
\psi \eta = (\ker \phi)_W$. For any element $r \in \sum\limits_{k \in K}
a_W$, 
$$
\phi \psi \eta (r) = f \phi \eta (r) = f(0) = 0. 
$$\pageoriginale

Thus $\im \psi \eta \subset (\ker \phi)_W$. Next, for any element $p
\in (\ker \phi)_W$, we have $p \in (\ker g \phi)_W = (\im \psi
)_W$. choose $q \in \sum\limits_{j \in J} a_W$ such that $\psi (q) =
p$; then 
$$
\theta (q) = f^{-1} \phi \psi (q) = f^{-1} \phi (p) = f^{-1}(0) = o. 
$$
Thus $q \in (\ker \theta)_W = \im \eta$, and hence $p = \psi (q) \in
\im \psi \eta$. Thus $\im \psi \eta = (\ker \phi)_W$ and $\mathscr{S}$
has $(b_1)$. Similarly if $\mathscr{S}'$ and $\mathscr{S}''$ have
$(b)$, it can be proved that $\mathscr{S}$ also has $(b)$. 

