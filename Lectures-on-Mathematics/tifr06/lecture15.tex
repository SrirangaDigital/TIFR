\chapter{Lecture 15}\label{chap15:lec15}%Lecture 15

We\pageoriginale now given example to show that the uniqueness theorem
fails in more general spaces.  

\begin{exam}%exe 24
Let $X$  consist of the unit segment $I=\{ x: o \leqq x \leqq 1\}$
together with two points $p$, $q$. A neighbourhood of $p$ (resp. $q$)
consists of $p$ (resp. $q$) together with all of $I$. Let
$\mathscr{G}$ be the subsheaf of the constant sheaf $Z_2$ formed by
omitting the points $(p,1)$, $(q,1)$, $(0,1)$, $(1,1)$. Let
$\mathscr{S}^o$ be the subsheaf of the constant sheaf $Z_2 +Z_2$
formed by omitting 
$(p,a)$, $(q,a)$ for all $a \neq 0$ and $(0,(1,0))$, $(0,(1,1))$, $(1,(0,1))$,
$(1,(1,1))$. Let $\mathscr{S}^1$ be the subsheaf of $Z_2$ formed by
omitting $(q,1)$. Let $\mathscr{S}^2$ have the stalk $Z_2$ at $p$ and
0 elsewhere; a neighbourhood of $(p,1)$ consists of $(p,1)$ together
with all the zeros over $I$. Then there is a resolution 
$$
0 \to \mathscr{G} \xrightarrow{e} \mathscr{S}^o \xrightarrow{d^1}
\mathscr{S}^1 \xrightarrow{d^2} \mathscr{S}^2 \xrightarrow{d^3} 0  
$$
and the corresponding sequence
$$
0 \to \Gamma (X,\mathscr{S}^o) \xrightarrow{d^1}
\Gamma(X,\mathscr{S}^1) \xrightarrow{d^2} \Gamma(X, \mathscr{S}^2)
\xrightarrow{d^3} 0 
$$
is
$$
0 \to 0 \to 0 \to Z_2 \to 0 ; 
$$
so $H^2 \Gamma (X,\mathscr{S})=  Z_2$.
\end{exam}

There\pageoriginale is also a resolution 
$$
0 \to \mathscr{G} \xrightarrow{e} \mathscr{R}^o \xrightarrow{d^1} 0 
$$
where $e$ is an isomorphism; then $H^p \Gamma (X, \mathscr{R}) = 0$
for all $p$. There is even a homomorphism $h$, commuting with $e,d^1
,\ldots$,  
\[
\xymatrix{
& & \mathscr{R}^o \ar[r]^{d^1} \ar[dd]^h & 0
  \ar[r] \ar[dd] & 0 \ar[r]\ar[dd] & 0\\
0 \ar[r] & \mathscr{G} \ar[ur]^e \ar[dr]_e & & &&\\
& & \mathscr{S}^o \ar[r]^{d^1} & \mathscr{S}^1 \ar[r] &
\mathscr{S}^2 \ar[r] &0 .
}
\]
[The space $X$ is not normal.]

\begin{defi*}%defi 
$A$ sheaf $\mathscr{S}$ of $A$-modules is called \textit{fine} if for
  every closed set $E$ in $X$ and open set $G$ in $X$ with $ E \subset
  G$, there is a homomorphism $ h : \mathscr{S} \to \mathscr{S}$ such
  that  
\begin{enumerate}[i)]
 \item $h (s) = s$ \quad if \quad $\pi (s) \in E$,

 \item $h (s) = 0_{\pi (s)}$ \quad if \quad $\pi (s) \notin \bar{G}$  
\end{enumerate}
 example of a fine sheaf.
\end{defi*}

\begin{exam}%exam 25 
For each open subset $U$ of $X$, let $S_U$ be the $A$-module of all
functions $f : U \to A$. If $V \subset U$, define $\rho_{_{VU}}$ to be
restriction homomorphism. Let $\mathscr{S}$ be the sheaf of germs of
functions determined by the presheaf $\sum = \{ S_U, \rho_{_{VU}} \}$. If
$E \subset G$ with $E$ closed and $G$ open, let $h_U : S_U \to S_U$ be
defined by 
$$
(h_U f) (x) = f (x) \; \chi_G (x)
$$
where\pageoriginale $x \in U$ and $\chi_G$ is the characteristic
function of $G$. 
$$
(\chi_G (x) =1 \in A \quad \text{if} \quad x \in G,
\chi_G (x) = 0 \in A \quad \text{if} \quad x \notin G).
$$

Then $\{ h_U\}: \sum \to \sum $ is a homomorphism. If $h : \mathscr{S}
\to \mathscr{S}$ is the induced homomorphism, $h (s) =s$ if $\pi (s) 
\in E$  and $h (s) =0 $ if $\pi (s) \notin \bar{G}$, hence the
sheaf is fine. 
\end{exam}

\medskip
\noindent{\textbf{Exercise.}}
If $M$ is any non-zero $A$-module and the space $X$
is normal, the constant sheaf $(X \, \chi M, \pi, X)$ is fine if and only
if $\dim  X \le 0$.  

\begin{note*}%note 0
 The set of all endomorphisms $h : \mathscr{S} \to \mathscr{S}$ forms
 an A-algebra, in general, non commutative, where $h_1 \cdot h_2$ is the
 \text{composite} endomorphism. The identity $1 : \mathscr{S} \to
 \mathscr{S}$ is the unit element of the algebra. 
\end{note*}

If $\mathscr{S} = (S, \pi, X)$ is a sheaf and $X_1$ is a subset of
$X$, let $X_1$ and $S_1 = \pi^{-1} (X_1)$ have the induced
topology. Then $(S_1, \pi |S_1, X_1)$ is a sheaf called the
\textit{restriction} of $\mathscr{S}$ to $X_1$.   

\textit{If $X$ is normal, the restriction of a fine sheaf
  $\mathscr{S}$ to any closed set $C$ is fine}. 

\begin{proof}%proo 
Let $E$ be any closed subset of $C$ and $G$ any open subset of $C$
with $E \subset G$. Extend $G$ to an open set $H$ of $X$, $G = H \cap
C$. Then, since $X$ is normal, $E$ closed in $X$, $H$ open in $X$ with
$E\subset H$, there is an open subset $V$ of $X$ with $E \subset V
\subset \bar{V} \subset H$. Since $\mathscr{S}$ is fine, there is a
homomorphism  $h : \mathscr{S} \to \mathscr{S}$ with 
\begin{align*}
 h (s) & = s \qquad \text{ if } \pi (s) \in E,\\
 & = 0_{\pi (s)}\qquad  \text{ if } \pi (s) \in X - \bar{V}.
 \end{align*}\pageoriginale
 
Then if $\mathscr{S}_1$ is the restriction of $\mathscr{S}$ to $C$, $h |
S_1 : S_1 \to S_1$ is a homomorphism $h_1: \mathscr{S}_1 \to
\mathscr{S}_1$ and we have 
\begin{align*}
h_1 (s) &= h(s) = s \qquad \text{ if } \pi (s) \in E\\ 
& = 0_{\pi (s)} \qquad \text{ if } \pi (s) \in C - \bar{G}
\subset C -G \subset X -H \subset X -\bar{V}. 
\end{align*}
\end{proof}

\begin{proposition}%prop 11
If X is normal, $\mathscr{U} = \{ U_i\}_{i \varepsilon I}
  \underbar{a}$ locally finite covering of $X$, and if the restriction
  of $\mathscr{S}$ to each $\bar{U}_i$ is fine (in particular, if
  $\mathscr{S}$ is fine), there is a system $\{ l_i\}_{i \in
    I}$ of homomorphisms $l_i : \mathscr{S} \to \mathscr{S}$ such
that 
\begin{enumerate}[i)]
\item for each $i \in I$ there is a closed set $E_i \subset
  U_i$ such that $ 1_i (S_x) = 0_x \text{ if } x \notin E_i$, 

\item $\sum\limits_{i \in I} 1_1 = 1$.
\end{enumerate}

(1 denotes the identity endomorphism $\mathscr{S} \to \mathscr{S}$).
\end{proposition}

\begin{proof}%proo 0
Using the normality of $X$, we shrink the locally finite covering
$\mathscr{U}= \{ U_i\}{_{i \in I}}$ to the covering
$\mathscr{U} \{ V_i\}{_{i \in I}}$ with $\bar{V}_i \subset
U_i$ and we further shrink the locally finite covering $\mathscr{U}$
to the covering $\mathscr{U} = \{ W_i\}{_{i \in I}}$ with
$\bar{W}_i \subset V_i$. Since the restriction $\mathscr{S}_i \text{
  of } \mathscr{S}$ to $\bar{U}_i$ is fine, there is a homomorphism
$g_i : \mathscr{S}_i \to \mathscr{S}_i$ with  
\begin{align*}
g_i (s) & = s \qquad \; \text{if} \quad  \pi (s) \in \bar{W}_i,\\
& = o_{\pi (s)} \quad \text{if}\quad  \pi (s) \in \bar{U}_i -
\bar{V}_i.   
\end{align*}

Let\pageoriginale the homomorphism $h_i : \mathscr{S} \to \mathscr{S}$
be defined by   
\begin{align*}
h_i (s) &= g_i (s) \quad \text{if} \quad  \pi (s) \in \bar{U}_i,\\
&= 0_{\pi (s)} \quad \text{if} \quad \pi (s) \in X - U_i.
\end{align*}

(This definition is consistent, since $g_i (s) = 0_{\pi (s)} \text {
  on } \bar{U}_i - U_i$). This $h_i: \mathscr{S} \to \mathscr{S}$ is
continuous and is a homomorphism with  
\begin{align*}
h_i (s) &= s \qquad \text{ if} \quad \pi (s) \in \bar{W}_i,\\ 
&= 0_{\pi (s)} \quad \text{if} \quad  \pi (s) \in X - \bar{V}_i.
\end{align*}

Let the set $I$ of indices be well ordered and define the homomorphisms
$l_i : \mathscr{S} \to \mathscr{S} $ by  
$$
1_i \left(\prod_{j < i} (1-h_j) \right) h_i, 
$$
where the product is taken in the same order as that of the indices. 
\end{proof}

Each point $x \in X$ has a neighbourhood $N_x$ meeting $U_i$
for only  a finite number of $i$, say $i_1 , i_2 ,\ldots, i_q$ with
$i_1 < i_2 < \cdots < i_q$. If $\pi (s) \in N_x$,  
 \begin{align*}
l_i (s) & = (1-h_{i1}) \cdots (1-h_{i{_{k-1}}}) h_{i_k} (s), i = i_k,
k=1 ,\ldots, q,\\ 
&= 0_{\pi (s)} \text{for all other $i$}.
 \end{align*} 
 
 Clearly $l_i (S_x) \subset S_x$ and $l_i | S_x : S_x \to S_x$ is a
 homomorphism. The function $l_i$ is continuous on each $\pi^{-1}
 (N_x)$ and coincides on the overlaps of two such neighbourhoods, hence
 $l_i : \mathscr{S} \to \mathscr{S}$ is continuous. Thus $l_i :
 \mathscr{S} \to \mathscr{S}$ is a homomorphism, and    
$$
 h_i (S_x)  = 0_x,  x \notin \bar{V}_i,
$$
hence $\hspace{3.6cm} l_i (S_x)  = 0_x,  x \notin \bar{V}_i$. 
  
 Take $E_i = \bar{V}_i \subset U_i$. Let $\pi (s) \in N_x $;
 then for some $i_k$, $1 \leqq k \leqq q$, $\pi (s) \in
 W_{i{_k}}$\pageoriginale and  hence $h_{i{_k}} (s) = s$. Hence  
 $$
 (1 -h_{i{_1}}) \cdots (1 -h_{i_{q}}) (s) = 0.
 $$
 
 Therefore
 \begin{align*}
\sum_{i \in I} 1_i (s) & = h_{i{_1}} (s) + (1 -h_{1{_1}})
h_{i{2}} (s) +\cdots+ (1-h_{i{_1}}) \cdots (1-h_{i{_{q-1}}}) h_{i_{q}}
(s) \\ 
& = s - (1-h_{i{_1}}) \cdots (1-h_{i{_q}}) (s)\\ 
& =s.
 \end{align*} 
 
 \begin{note*}%note 0
The homomorphisms $l_i$ are usually not uniquely determined and they
cannot therefore be expected to commute with other given
homomorphisms. 
 \end{note*}

