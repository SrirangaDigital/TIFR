 \chapter{Lecture 21}\label{chap21:lec21}%leca 21
 
 \section*{Introduction of the family $\Phi$}\pageoriginale  
 
 Let $\Phi$ be a family of paracompact normal closed subsets of a 
 topological space $X$ such that  
 \begin{enumerate}[(1)]
\item if $F \in \Phi$, then every closed subset of $F$ is in $\Phi$, 

\item if $F_1$, $F_2 \in \Phi$, then $F_1 \cup F_2 \in \Phi$,

\item if $F \in \Phi$, there is an open $U$ with $F \subset U$ and
  $\bar{U} \in \Phi$. 
 \end{enumerate} 
 
 For example, if $X$ is paracompact and normal, $\Phi$ can be taken to
 the family of all closed subsets of $X$, and, if $X$ is locally
 compact and Hausdorff, then $\Phi$ can be taken to be the family of
 all compact sets of $X$. 

\medskip
\textit{Sections with supports in the family $\Phi$}. If $\mathscr{S}$
is a sheaf of $A-$ modules, the set of all sections $f \in \Gamma (X ,
\mathscr{S})$ such that supp $f = \bigg\{ x : f(x) \neq 0_x \bigg\}$
is in $\Phi$, forms an $A-$ module (if supp $f_1 \in \Phi$ and supp $f_2
\in \Phi$ then supp $(f_1 \pm f_2) \subset$ (supp $f_1 \cup supp f_2)$
is in $\Phi)$, a submodule of $\Gamma (X, \mathscr{S})$, which we
denote by $\Gamma \Phi (X , \mathscr{S})$. 

Any homomorphism $h : \mathscr{S}^1 \rightarrow \mathscr{S}$ of two
sheaves of $A-$ modules induces a homomorphism  
$$
h : \Gamma \bar{{\Phi}}(X, \mathscr{S}') \rightarrow \Gamma \bar{\Phi}
(X, \mathscr{S}), 
$$
since a homomorphism of sheaves decreases supports (i.e., $\supp hf
\subset$ supp $f$). 

\begin{defi*}%defi 0
 A\pageoriginale $\Phi$- {\em covering of} $X$ is a locally finite
 proper covering $\mathscr{U}$ such that, if $X \notin \Phi$, there is
 a {\em special} open set $U_* \in \mathscr{U} $ with $\bigcup
 \limits_{U \in \mathscr{U}- (U_*)} \bar{U} \in \Phi$.  
\end{defi*}

\begin{remark*}%rema 0
If $X \notin \Phi$, $U_*$ is unique, and is not the empty set, for
otherwise, in each case, $X$ would belong to $\Phi$. If $X \in
\Phi$, a $\Phi$- covering is just a locally finite proper covering of
$X$. 
\end{remark*}

\textit{The $\Phi$- coverings of $X$ form a subdirected set
  $\Omega_*$ of the directed set of all locally finite proper
  coverings of the space $X$}. 
\begin{proof}

\begin{enumerate}[(i)]
\item If $X \in \Phi, \Omega_\ast$ is the directed set of all locally 
  finite proper coverings of $X$. 

\item If $X \notin \Phi $ and $\mathscr{U}$, $\mathscr{W}$ are any two 
  $\Phi$- coverings of $X$, let $\mathscr{W} =\bigg\{ W: W= U \cap V
  \bigg\}$ for some $U \in \mathscr{U}$ and some $V \in \mathscr{W}$
  with $W_\ast = U_\ast \cap V_\ast$. Then $\mathscr{W}$ is a locally finite
  proper covering of $X$ and 
$$
\bigcup \limits_{W \in \mathscr{W}-(W_\ast)} \bar{W} =(\bigcup \limits_{U
  \in \mathscr{U}-(U_\ast)} \bar{U}) \cup (\bigcup \limits_{V \in
  \mathscr{W}-(V_\ast)} \bar{V}) 
$$
is in $\Phi$, since each set contained in brackets is in $\Phi$. Thus 
$\mathscr{W}$ is a $\Phi$-covering which is a common refinement of
$\mathscr{U}$ and $\mathscr{W}$. 
\end{enumerate}
\end{proof}

\begin{remark*}%rema 0
If $\mathscr{U}$ and $\mathscr{W}$ are two $\Phi$ coverings with
special sets $U_*$ and $V_*$ then $U_* \cap V_*$ is not empty, and if
$\mathscr{W}$ is refinement of $\mathscr{U}$, then $V_* \subset U_*$
and $V_*$ is not contained in any other $U \in \mathscr{U}$. In
particular, if $\mathscr{W}$ is equivalent to $\mathscr{U}$, then
$V_*= U_*$.  
\end{remark*}

\textit{Cohomology\pageoriginale groups with supports in the family} $\Phi$. If
$\mathscr{U}$ is a $\Phi$-covering and $\sum$ a presheaf of $A$-
modules, we define  
\begin{align*}
C^p_{\Phi}\left(\mathscr{U},\sum \right) & = C^p \left(\mathscr{U},
\sum\right) \text{ for } p 
> 0, \\
\text{ and } \qquad 
C^p_{\Phi}\left(\mathscr{U},\sum \right) & \subset C^o \left(\mathscr{U},
\sum \right) \text{ for } p= 0,  
\end{align*}
where$C^p_{\Phi}\left(\mathscr{U},\sum \right)$ is the submodule of
$C^o\left(\mathscr{U},\sum \right)$ consisting of those zero cochains
which assign 
to $U_*$ the zero of $S_{U_*}$. Then we have a mapping  
$$
\delta^p : C^{p-1}_{\Phi}\left(\mathscr{U},\sum \right) \to
C^p_{\Phi}\left(\mathscr{U},\sum \right). 
$$

Let\pageoriginale
$$
H^p_{\Phi}\left(\mathscr{U},\sum\right)= \ker \delta^{p+1}/\im \delta^p. 
$$

Then $H^p_{\Phi}\left(\mathscr{U},\sum \right)$ is called the $p$-\textit{th
  cohomology module of the covering $\mathscr{U}$ with coefficients in
  the presheaf $\sum$ and supports in the family $\Phi$}. 

If a $\Phi$- covering $\mathscr{W}$ is a refinement of $\mathscr{U}$,
for each choice of the function $\tau : \mathscr{W} \to \mathscr{U},
\tau (V_*) =U_*$. We then have the mapping (Lecture \ref{chap9:lec9}) 
$$
\tau^+ : C^p_{\Phi} \left(\mathscr{U},\sum \right) \to 
C^p_{\Phi}\left(\mathscr{W},\sum \right) \qquad (p \geqq 0). 
$$
$\tau^+$ induces the homomorphism  
$$
 \tau_{\mathscr{W} \mathscr{U}}:  H^p_{\Phi}\left(\mathscr{U},\sum \right) \to
C^p_{\Phi} \left(\mathscr{W},\sum \right)  
$$
with $\tau_{\mathscr{U} \mathscr{U}}=$ identity, and
$\tau_{\mathscr{W} \mathscr{W}} \tau_{\mathscr{W} \mathscr{U}}=
\tau_{\mathscr{W} \mathscr{U}}$ if $\mathscr{U} < \mathscr{W} <
\mathscr{W}$. Thus $\{ H^p_{\Phi} \left(\mathscr{U},\sum \right),
\tau_{\mathscr{W} \mathscr{U}}\}$ is a direct system of $A$-modules. 

Let 

$H^p_{\Phi}\left(X,\sum \right)= \text{ direct limit }\bigg\{
H^p_{\Phi}\left(\mathscr{U},\sum \right), \tau_{\mathscr{W}
  \mathscr{U}}\bigg\}_{\mathscr{U}, \mathscr{W} \in \Omega_\ast}.
$ $H^p_{\Phi}(X,\sum)$ is called the $p$-\textit{th cohomology module of
  the space $X$ with coefficients in the presheaf $\sum$ and supports
  in the family $\Phi$}. 

The result analogous to Proposition \ref{chap11:prop7} (Lecture
\ref{chap11:lec11}) is 
true in this 
case. 

\medskip
\noindent{\textbf{Proposition 7-a}}
\textit{If $\mathscr{S}$ is a sheaf of $A$-modules, $H^o_{\Phi}(X,
   \mathscr{S})= \Gamma_{\Phi}(X, \mathscr{S})$.}

\begin{proof}
We can identify $H^o_{\Phi}(\mathscr{U}, \mathscr{S})=
Z^o_{\Phi}(\mathscr{U}, \mathscr{S})$(see Lecture \ref{chap11:lec11}) with the
submodule of $\Gamma_{\Phi}(X, \mathscr{S})$ consisting of the
sections $f$ with $\supp f \subset X - U_* \bigcup\limits_{U \in
  \mathscr{S}-(U_*)} \bar{U} \in \Phi$; then supp $f \in \Phi$. If
$\mathscr{W}$ is a refinement of $\mathscr{U}$, since $V_* \subset
U_*$, we have $X- U_{\ast} \subset X - V_{\ast}$, hence  
$$
\tau_{\mathscr{W}\mathscr{U}}: H^o_{\Phi}(\mathscr{U},\mathscr{S}) \to
H^o_{\Phi}(\mathscr{W},\mathscr{S}) 
$$
is the inclusion homomorphism. Thus $H^o_{\Phi}(X,\mathscr{S})$ can be
identified with a submodule of $\Gamma_{\Phi}(X,\mathscr{S})$ i.e.,
$H^o_{\Phi}(X,\mathscr{S}) \subset \Gamma_{\Phi}(X,\mathscr{S})$. 

If $f \in \Gamma_{\Phi}(X,\mathscr{S})$, let $U_*=X-\supp f$, and let
$U$ be an open set containing $\supp f$ with $\bar{U} \in \Phi$. Clearly
$\mathscr{U}=\big\{U, U_* \big\}$ is a $\Phi$-covering (with special
set $U_*$) and the cochain $g$ defined by  
$$
g(U) =f|U ; g(U_*)= 0 
$$
is the cocycle in $Z^o_{\Phi}(\mathscr{U},\mathscr{S}) =
H^o_{\Phi}(\mathscr{U},\mathscr{S})$ which is identified with $f \in
\Gamma_{\Phi}(X, \mathscr{S})$. Thus $\Gamma_{\Phi}(X, \mathscr{S})
\subset H^o_{\Phi}(X, \mathscr{S})$, hence $H^o_{\Phi}(X,
\mathscr{S})= \Gamma_{\Phi}(X, \mathscr{S})$. 
\end{proof}

\textit{Given\pageoriginale a sequence of homomorphisms of presheaves}
$$
\cdots \to \sum^{q-1} \xrightarrow{d^q} \sum^q  \xrightarrow {d^{q+1}}
\sum^{q+1} \to \cdots 
$$
\textit{with im $d^q \subset \ker d^{q+1}(i.e., d^2 =0)$ for each
  $\Phi$-covering $\mathscr{U}$ the system $\{ k^{p,q}_{\mathscr{U}}=
  C^p_{\Phi} \left(\mathscr{U}, \sum^q \right)$ with the homomorphisms  
\begin{align*}
& \delta: C^p_{\Phi}(\mathscr{U},\sum^q) \to
  C^p_{\Phi}\left(\mathscr{U},\sum^q \right)\\ 
and \qquad d : & C^p_{\Phi}\left(\mathscr{U},\sum^{q-1} \right) \to
C^p_{\Phi} \left(\mathscr{U},\sum^q \right) 
\end{align*}
forms a double complex denoted by
$K_{\mathscr{U}}=C_{\Phi} \left(\mathscr{U},\sum^q \right)$}. 

\begin{proof}
Any homomorphism $\sum' \xrightarrow{d}\sum$ of two presheaves induces
a homomorphism of $C^p_{\Phi}\left(\mathscr{U},\sum' \right)\xrightarrow{d}
C^p_{\Phi} \left(\mathscr{U},\sum \right)$ commuting with the
cobou\-ndary operator $\delta$. We have $\delta^2 =0$, and by hypothesis
$d^2 = 0$. Further $d \delta = \delta d$; so we have the commutative
case of a double complex, $q.e.d$. 
\end{proof}

For each pair $\mathscr{U}$, $\mathscr{W}$ of $\Phi$- coverings for
which $\mathscr{W}$ is a refinement of $\mathscr{U}$ choose $\tau :
\mathscr{W} \to \mathscr{U}$ with $V \subset \tau(V)$; if $\mathscr{W}
= \mathscr{U}$, let $\tau: \mathscr{U} \to \mathscr{U}$ be the
identity and let  
\begin{align*}
\phi_{\mathscr{W}\mathscr{U}} & = \tau^+ : C^p_{\Phi}(\mathscr{U},
\sum^q) \to C^p_{\Phi}\left(\mathscr{W}, \sum^q \right). \\
C_{\Phi}(\sum) & = \bigg\{ C_{\Phi} \left(\mathscr{U},
\sum \right),\phi_{\mathscr{W}\mathscr{U}} \bigg\}_{\mathscr{U},\mathscr{W}
  \in \Omega_\ast} 
\end{align*}
is a direct system of double complexes. 

(If $\mathscr{U} < \mathscr{W}$, $\phi_{\mathscr{W}\mathscr{U}}= 
\tau^+$, for an arbitrary but \textit{fixed} choice of $\tau : 
\mathscr{W} \to \mathscr{U}$ if $\mathscr{W} \neq \mathscr{U}$, and
$\tau: \mathscr{U} \to \mathscr{U}$ is\pageoriginale the identity.) 

\begin{proof}
Since $\delta$ and $d$ commute with $\phi_{\mathscr{W}\mathscr{U}},
\phi_{\mathscr{W}\mathscr{U}}: C_{\Phi}(\mathscr{W}, \sum) \to\break
C_{\Phi}(\mathscr{U},\sum)$ is a map of double complexes, and by
construction $\phi_{\mathscr{U} \mathscr{U}}$ is the identity. 
\end{proof}

If $\mathscr{W}$ is a $\Phi$- refinement of $\mathscr{W},
\phi_{\mathscr{W}\mathscr{U}} \phi_{\mathscr{W}\mathscr{U}}$
corresponds to a possible choice of $\tau : \mathscr{S} \to
\mathscr{U}$. Since for all possible choices of $\tau$, $\tau(W_\ast) =
U_\ast$, the homotopy operator $k$ (see Lecture \ref{chap9:lec9}) 
$$
k: C^p \left(\mathscr{U}, \sum^q \right) \to C^{p-1}
\left(\mathscr{W}, \sum^q\right)  
$$
maps $C^1_{\Phi} \left(\mathscr{U}, \sum^q \right) =C^1 \left(\mathscr{U},
\sum^q \right)$ into $C^o_{\Phi} \left(\mathscr{W}, \sum^q
\right)$. Thus we have two homomorphisms $k : 
 C^p_{\Phi} \left(\mathscr{U}, \sum^q \right) \to C^{p-1}_{\Phi}
 \left(\mathscr{W},\sum^q \right)$ and the trivial homomorphism
 $C^{p}_{\Phi} \left(\mathscr{U}, \sum^q \right) \to
 C^{p}_{\Phi} \left(\mathscr{W}, \sum^{q-1} \right)$ such that   
$$
\delta k + k \delta = \phi_{\mathscr{W} \mathscr{W}}
\phi_{\mathscr{W}\mathscr{U}}- \phi_{\mathscr{W} \mathscr{U}}, 
$$
and further, $d$ commutes with $k$. Hence $\phi_{\mathscr{W}
  \mathscr{U}} \phi_{\mathscr{W} \mathscr{U}}$ is homotopic to
$\phi_{\mathscr{W} \mathscr{U}}$, hence $\bigg\{ C_{\Phi}(\mathscr{U}
,\sum), \phi_{\mathscr{W} \mathscr{U}} \bigg\}$ is a direct system. 

