\chapter{Lecture 7}

With\pageoriginale the notations introduced in the last lecture, we prove

\begin{proposition}%propos 3
 If $\{A_U, S_U, \phi_{VU}\}$ is a presheaf of modules over the
  space $X$, then $\mathfrak{a} = (A, \tau, X)$ is a sheaf of rings
  with unit and $\mathscr{S}= (S, \pi, X)$ is a sheaf of
  $\mathfrak{a}$-modules.  
\end{proposition}

\begin{proof}
If $a \in A_U$, $\bar{a}: U \to A$ is continuous. For, if $a_x \in b_V$
with $b \in A_V$, there exists $c \in A_W$ with $x \in W \subset  U
\cap V $ such that $\bar{a}(W)=a_W =c_W =b_W \subset b_V$. Also
$\bar{a}: U \to A$ is an open mapping since, for open $V \subset U$,
$a_V = (\phi_{VU}a)_V$ is open by definition. 
\end{proof}

\noindent
Hence $\bar{a}: U \to a_U$ being 1-1 is a homeomorphism and the
inverse $\tau | a_U : a_U \to U$ is a homeomorphism of $a_U$ onto the
open set $U$. 

Similarly $\bar{s} : U \to S$ is continuous and $\pi | s_U: s_U \to U$
is a homeomorphism. Thus $\tau $ and $\pi$ are local homeomorphisms. 

For each $x \in X$, $\tau^{-1}(x)=A_x$ is a ring with unit element, and
$\pi^{-1}(x)= S_x$ is a unitary left $A_x$-module. 

\textit{Addition is continuous}, for if $a_x$, $b_x \in A_x$ (with $a
\in A_U$, $b \in A_{U_1}$, $x \in U \cap U_1$) and $a_x + b_x \in c_{U_2}$
with $c \in A_{U_2}$, then for some $W$ with $x \in W \subset U \cap
U_2 \cap U_2$, $\phi_{WU} a+ \phi_{WU_1}$  $b=\phi_{WU_2} c$. Thus $a_x \in
a_W$, $b_x \in b_W$ and for any $p \in a_W$, $q\in b_W$ with $\tau(p)=
\tau (q)= y$ say, we have $p+q = a_y +b_y = c_y \in c_{U_2}$. 

\textit{The unit is continuous}, for if $1_x \in A_x$ is the unit
element of $A_x$ and $1_x \in b_U$, then for some $V$ with $x \in V
\subset U$, $\phi_{VU}b$ is the unit element of $A_V$. Then $b_V \subset
b_U$ and consists of the unit elements $1_y$ for
$y \in V$. 

Similarly\pageoriginale the other operations of $\mathfrak{a}$ and
$\mathscr{S}$ are continuous.  

We remark that $a_U$, $s_U \cdots $ are \textit{sections} and that such
sections form a base for open sets of $A$, $S$. 

\begin{remark*}
The ``sheaves'' introduced originally by Leray were actually presheaves
with the indexing set $\Omega$ consisting of the family of all closed
sets instead of the family of all open sets. 
\end{remark*}

\begin{exam}%exam 11.
Let $X$ be the circle $| z |=1$; for each open set $U$ of $X$ let
$S_U$ be the abelian group of all integer valued functions in $U$ and
let $\phi_{VU}$ be the restriction homomorphism. This system is a
presheaf and the induced sheaf has Example 4 as a subsheaf. 
\end{exam}

\begin{exam}%exam 12.
Let $X$ be the real line, and $S_U$ the $R$ module ($R$ denotes the
ring of real numbers) of all real indefinitely differentiable
functions in $U$ and let $\phi_{VU}$ denote the restriction
homomorphism. This system is a presheaf and the sheaf space $S$ of the
induced sheaf is not Hausdorff. 
\end{exam}

Let $(\mathfrak{a}, \mathscr{S})$ be a sheaf, $(\bar{\mathfrak{a}},
\bar{\mathscr{S}})$ its presheaf of sections and $(\mathfrak{a}',
\mathscr{S}')$ the sheaf determined by
$(\bar{\mathfrak{a}},\bar{\mathscr{S}})$. We show that
$(\mathfrak{a}', \mathscr{S}')$ and $(\mathfrak{a}, \mathscr{S})$ are
canonically isomorphic.  
\[
\xymatrix{
& (\bar{\mathfrak{a}}, \bar{\mathscr{s}})\ar@{-->}[dr] & \\
(\mathfrak{a}, \mathscr{s})\ar@{-->}[ur] & & (\mathfrak{a}',
  \mathscr{s}')\ar[ll]_{h} 
}
\]

If $x \in U$ and $f \in \Gamma (U, \mathfrak{a})$, let $h_{xU} f
=f(x)\in A_x$. Similarly, if $s \in \Gamma (U, \mathscr{S})$, let
$h_{x U} s=s(x) \in S_x$. Then 
$$
h_{xU}: (\Gamma (U, \mathfrak{a}), \Gamma (U, \mathscr{S})) \to (A_x, S_x) 
$$\pageoriginale
is a homomorphism and, if $x \in V \subset U$, $h_{xV}
\phi_{VU}=h_{xU}$. Then there is an induced homomorphism $h_x :
(\cdots'_x, S'_x) \to (A_x, S_x)$ with $h_x \phi_{x U} = h_{xU}$. 

In fact $h_x$ is an isomorphism. For if $p \in A_x$, then there is
some section $f: U \to A$ with $f(x) = p$, then 
$$
p=f(x)=h_{xU} f =h_x \phi_{x U} f \in im h_x, 
$$
and if $p' \in A'_x$ with $h_xp'= O$, choose a representative $f \in
\Gamma (U, \mathfrak{a})$ for $p'$. Then $f(x) = h_{xU} f = h_x p' =
O$. Hence, for some $V$, with $x \in V \subset U$, $f | V =
o$. Therefore  
$$
p' = \phi_{xU} f = \phi_{xV}\phi_{VU} f = \phi_{xV}O_V = (O). 
$$
 
Thus $h_x : A'_x \to A_x$ is an isomorphism and similarly $h_x : S'_x
\to S_x$ is an isomorphism. 

Let $h : (A', S') \to (A, S)$ be given by $h | (A'_x, S'_x) = h_x$. If
$f \in \Gamma (U, \mathfrak{a})$ and $f_U$ is the induced section in
$\mathfrak{a}'$, given by $\bar{f}(x) = \phi_{xU}f$, then 
$$
h \bar{f}(x) = h_x \bar{f}(x) = h_x \phi_{xU} f = h_{xU}f = f(x) 
$$
and thus $h (\bar{f}(U)) = f(U)$. The same holds if $s \in \Gamma (U,
\mathscr{S})$. 

Thus $h$ is an isomorphism of stalks for each $x$ and, since it maps
section $f_U$ onto sections $f(U)$, $h$ is a local homeomorphism and
hence is continuous. Thus $h : (\mathfrak{a}'. \mathscr{S}) \to
(\mathfrak{a}, \mathscr{S})$ is a sheaf isomorphism. 

\textit{We\pageoriginale identify $(\mathfrak{a}', \mathscr{S}')$ with
  $(\mathfrak{a}, \mathscr{S})$ under this isomorphism.}  

If $a \in A$, $h^{-1}a$ is the class of all sections $f: U \to A$ where
$f(U)$ contains $a$, and similarly for $h^{-1}s$. 

\begin{defi*}
If $\sum'=\{A'_U, S'_U, \phi'_{VU}\}$ and $\sum = \{A_U, S_U,
\phi_{VU}\}$ are preshea\-ves over $X$, a homomorphism $f : =\sum' \to
\sum$ is a system $\{f_U\}$ of homomorphisms $f_U : (A'_U, S'_U) \to
(A_U, S_U)$ such that $f_V \phi'_{VU}= \phi_{VU}f_U$, that is, the
following diagram is commutative. 
\[
\xymatrix{
(A'_U, S'_U) \ar[r]^{\phi'_{VU}} \ar[d]_{f_U}& (A'_V,
  S'_V)\ar[d]_{f_V}\\
(A_U, S_U) \ar[r]_{\phi_{VU}} & (A_V, S_V).
}
\]
\end{defi*}

\textit{Let $(\mathfrak{a}', \mathscr{S}')$,
  $(\mathfrak{a},\mathscr{S})$ be the sheaves 
  determined by $\sum'$, $\sum$. Then the homomorphism $f : \sum' \to
  \sum$ induces a sheaf homomorphism $f :(\mathfrak{a}', \mathscr{S}') \to
  (\mathfrak{a},\mathscr{S})$.} 

\begin{proof}
For each $x$, $\{f_U\}_{x \in U}$ induces a homomorphism $f_x : (A'_x,
S'_x) \to (A_x, S_x)$, with $f_x \phi'_{xU} = \phi_{xU} f_U$ and these
homomorphism $f_x$ define a function $f : (A', S') \to (A, S)$.  
\end{proof} 

\noindent
If for $a' \in A'_U$, $f_U a' = a$, then
$$
a_x = \phi_{xU} a = \phi_{xU}f_U a' =f_x \phi'_{xU} a' = f_x (a'_x). 
$$
Thus $f(a'_U) = a_U$ and $f$ is a local homeomorphism, hence is
continuous. Hence $f$ is a sheaf homomorphism, and this completes the
proof. 

Let\pageoriginale $\sum$ be a presheaf which determines the sheaf
$(\mathfrak{a}, 
\mathscr{S}), (\bar{\mathfrak{a}}, \bar{\mathscr{S}})$ the presheaf of
sections and $(\mathfrak{a}',  
\mathscr{S}')$ the sheaf determined by it. The functions $f_U : (A_U,
S_U) \to \bigg( \Gamma ( U, \mathfrak{a}), \Gamma (U,\mathscr{S})
\bigg)$ (where $f_Ua$ is 
the section $\bar{a} : U \to A$ determined by $a$, and similarly for
$f_Us)$, determine a homomorphism, $f=\{f_U\} : \sum \to (\bar{\mathfrak{a}},
\bar{\mathscr{S}})$. In general, the homomorphism $f$ is neither an
epimorphism nor a monomorphism, hence obviously not a isomorphism. 

\textit {The induced homomorphism $f : (\mathfrak{a},\mathscr{S}) \to
  (\mathfrak{a}',\mathscr{S}')$ is the identifying isomorphism $h^{-1}$.} 

\begin{proof}
Let $a \in A$ and suppose that $a = \phi_{xU} b$, $b \in A_U$. $f(a)$ is
the class at $x$ containing $f_U b$ which is a section with $(f_U b)
(x)= \phi_{xU} b=a$. Thus $f(a)$ is the class $h^{-1}$ a of all
sections $g : U \to A$ with $g(x) = a$. 
\end{proof}
\[
\xymatrix{
\sum \ar[r]^f \ar@{-->}[d] & (\bar{a},\bar{s})\ar@{-->}[d]\\
(a,s) \ar[r]^f_{h^{-1}} \ar@{-->}[ur] & (a',s').
}
\]


