\chapter[The Polynomial of Best Approximation...]{The Polynomial of Best Approximation Chebyshev
  Polynomials}\label{chap2} 

\setcounter{section}{3}
\section{The Lagrange Polynomial}\label{chap2:sec4}

We\pageoriginale are given $n +1$ values of $x$,
 $$
 x_0, x_1, \ldots, x_n 
 $$
 and $ n + 1 $ constants $ c_0, c_1, \ldots,  c_n$.

Write $ \prod (x)  = (x - x_0)  \cdots (x - x_n) $.

The polynomial $p(x)$ of degree at most $n$ which takes the values
$c_i$ at $x_i$ is, by the partial- fraction rule for $ p (x) /
\prod(x) $, 
$$
\prod (x) \sum^{n}_{0} \frac{1}{x - x_i} \frac{c_i}{\prod (x_i)}.
$$

If the $c_i $  are the values at $x_i$  of a function $ f (x) $, we
call $ p (x) $ the Lagrange polynomial of $ f (x) $ at the $x_i
$. Here we follow the usual terminology, although Waring (1779)
used the polynomial before Lagrange (1795) and indeed it is clear
that the formula was known to  Newton. 

Suppose that the values $ x_0,  \ldots,  x_n $ are fixed. The
following lemmas follow from the definition of the Lagrange
polynomial. 

\begin{lem}\label{chap2:sec4:lem1}%lemma 1
  Given an aggregate of  polynomials $p_\alpha (x) $ of  degree at
  most $n$, where $ \alpha $ runs through an index-set $ I $, such
  that  
  $$
  | p_\alpha (x_i) | \leq A,  ~\alpha ~\text{in}~ I ; i = 0, \ldots,  n.
  $$
\end{lem}

Then, if $ a_{\alpha, r} $ is the coefficient of $ x^r $ in $ p_\alpha (x) $, 
$$
| a_{\alpha, r} | \leq A B, 
$$
where $B$ is independent of $ \alpha $.

\begin{proof}
  Write $ p_\alpha (x_i) $ for $c_i$. We have a $B$ depending only
  on the $x_i$. 
\end{proof}

\begin{lem}\label{chap2:sec4:lem2}% lemma2
  (For\pageoriginale brevity of expression, we translate de la Vallee Poussin,
  Le\c ons, 74). If, at $ n + 1 $  given points, two polynomials of
  degree at most $n$  take 'infinitely close' values, their
  corresponding  coefficients are infinitely close. 
\end{lem}

\begin{proof}
  Given $ \varepsilon $, we have two polynomials say $ p_\alpha (x) $,
  $ q_\alpha (x) $ which differ by at most $ \varepsilon $  for each
  of the   values $ x_0,  \ldots,  x_n $. By Lemma \ref{chap2:sec4:lem1},  their
  corresponding coefficients differ by at most $ B \varepsilon $. 
\end{proof}

\section{Best Approximation}\label{chap2:sec5}%\section 5.

 Let $ P_n $ be the set of polynomials $ p (x) $ of degree less than
 or equal to $n$. Then 
 $$
 P_0 \subset P_1 \subset P_2  \cdots 
 $$
 Define, for any particular $p$ in $P_n$,
 $$
 d (p, f) = \sup | f (x) - p (x)  |~  for ~ a \leq x \leq b.
 $$
 
Let $d = d_n = d_n (f) = \inf d (p,f) $ for all $p$ in $P_n $. Then $
d \ge 0 $. Our first aim is to prove that there exists  a $p$ in $P_n
$ for which the inf. is attained, i.e., that, given $ f (x) $ of $ C
(a,b) $, there is a \textit{polynomial of degree $n$  of best
  approximation.} Later we shall prove uniqueness. 

If $f$ is given,
$$
d_0 \ge d_1 \ge d_2 \ldots, 
$$
and Theorem \ref{chap1:sec1:thm1} asserts that $ \lim d_n = 0 $.

The existence of a polynomial of best approximation was known to
Chebyshev (or Tschebyscheff) (1821 - 1894) who was one of the
founders of the subject. The necessary proof was supplied by Borel (1905). 

\begin{theorem}\label{chap2:sec5:thm4} % theorem 4
  There\pageoriginale is a polynomial $ p (x) $ in $P_n $ for which
  $$
  \sup | f (x) - p(x) | = d (= d_n).
  $$
\end{theorem}

\begin{proof}
  All our polynomials being $ P_n $, we do not need the suffix $n$ to
  denote degree, and the suffixes in $ p_1, p_2, \ldots $ will be used
  to specify particular polynomials of $ P_n $. As here, we shall
  commonly omit the variable $x$ form a polynomial $ p (x) $ or a
  function  $ f (x) $.  
\end{proof}

By definition of $d$, there is a polynomial $ p_m $ with 
$$
d \leq d (p_m, f) < d + \frac{1}{m}.
$$

For all  $m$  and $ a \leq x \leq b $, 
$$
| p_m (x) | \leq d + 1 + \sup | f (x) | = A.
$$
By \S 4, Lemma \ref{chap2:sec4:lem1}, the $n + 1$ coefficients of powers  $x^0,
x^1, \ldots, x^n$ in the $p_m (x)$ all lie in a bounded region of
$ n+1 $ space. This set of points in $ n + 1 $ space has at least one
limit point, defining a polynomial $ p (x) $ for which 
\begin{align*}
  d (f, p) & \leq d (f, p_m) + d (p_m,  p) \\
  &\leq d + \frac{1}{m} + \varepsilon 
\end{align*}
where $ \varepsilon \to 0 $ as $ m \to \infty $ through a sub-sequence
for which there is convergence of the coefficients to their limits. 

Therefore $d(f, p)= d$.

\begin{theorem}\label{chap2:sec5:thm5}%theorem5.
  If $ f (x) $ is $ C (a,b) $  and $ p (x) $ satisfies Theorem
  \ref{chap2:sec5:thm4} 
    there are $n + 2$ values (or more) at which
    $$
    f (x) - p (x) = \pm d,
    $$
    with alternating sign.
\end{theorem}

\begin{proof}
  $g (x) = f (x) - p (x)$\pageoriginale is continuous. Divide $ (a, b) $  into
  sub-intervals such that $ g (x) $ does not take the value $0$ in any
  (closed)  sub-interval in which it takes the value $ \pm d
  $. Denote by $ l_1, l_2, \ldots,  l_m $ (numbered from left to
  right) those of the sub-intervals in which $ g (x) $ takes the
  value $ + d $ or $ -d $. Define $ \varepsilon_1, \varepsilon_2,
  \ldots,  \varepsilon_m $ to be  $ + 1 $ or $ -1 $ according as the
  value is $ + d $ or $ -d $. We have to prove that there are  at
  least are at least $n+1$ changes of sign in the sequence of
  $\varepsilon ' s$. Suppose there are fewer. We shall obtain a
  contradiction by constructing a polynomial of better approximation
  than $ p (x) $. 
\end{proof}

If all the $ \varepsilon 's $ have the same sign, say $+$, add a small
constant to $ p (x) $. This gives a polynomial of better
approximation.  

Generally, suppose that there are $k$  changes of sign in the sequence
of $ \varepsilon' s $, where $ k \leq n $. Let $ \varepsilon_i,
\varepsilon_{i+1} $ be different. Then $ l_i $ and $l_{i+1} $ cannot
abut (since $ g(x) $ does not vanish in either), so we can choose a
value of $x$;  lying between them. We have thus $k$ values of $x$;
call them 
$$
x_1, x_2,  \ldots, x_k.
$$

Define $ h (x) = \varepsilon_1 (x_1 -x) (x_2 -x) \cdots (x_k - x
). h (x) $ has the same sign as $ g (x) $ in each of sub-intervals
$l$. We shall prove that, if $ \eta $ is small enough, the polynomial
of $ P_n $ 
$$
p (x) + \eta h (x)
$$
has better approximation to $ f (x) $ than $ p (x) $ has.

In those intervals of the original subdivision which are not $ l's $, 
$$
\sup | g (x) | = d' ~(\text{say})~ < d.
$$

Choose\pageoriginale $\eta$ to make $|\eta h(x)|<d-d' (a \leq x \le b)$, now,
$$
|f-p-\eta h|=|g-\eta h|.
$$

In the $l's$, this is less than $d$, since $g,h$ have the same sign.

And, in the sub-intervals other than $l's$,
$$
|g-\eta h|\leq |g|+|\eta h|<d' +(d-d')=d.
$$

So $p+\eta h$ approximates better to $f$ than $p$ does.

\begin{theorem}\label{chap2:sec5:thm6} %theorem 6
  The polynomial $p(x)$ of Theorem \ref{chap2:sec5:thm4} is unique.
\end{theorem}

\begin{proof}
  Suppose that two polynomials $p,q$ satisfy Theorem
  \ref{chap2:sec5:thm4}. Let $r= \dfrac{1}{2}(p+q)$. Then
  $f-r=\frac{1}{2}(f-p)+\dfrac{1}{2}(f-q)$. Therefore $r$ satisfies
  Theorem \ref{chap2:sec5:thm4}, and so, by Theorem \ref{chap2:sec5:thm5}, 
  $$
  f-r= \pm d
  $$
  for $n+2$ values of $x$.
\end{proof}

But $f-r=d$ only if $f-p=f-q=d$. Therefore there are $n+2$ values of
$x$ for which $p(x)$ and $q(x)$, polynomials of degree at most $n$,
are equal. Therefore $p(x)\equiv q(x)$. 

In future we can (by Theorem \ref{chap2:sec5:thm6}) describe
\textit{as the best $P_n$} that polynomial $p(x)$ in $P_n$ for which 
$$
\sup |f(x)-p(x)|=d,
$$
where $d=\inf d(f,q)$ for all $q(x)$ in $P_n$. The number $d$ (or
$d_n$ if it is necessary to make then $n$ explicit) may be called
\textit{the best approximation}. 

\begin{theorem}\label{chap2:sec5:thm7}%theorem 7
  Suppose $f$ is $C(a,b)$ and $q$ is in $P_n$. Let there be $n+2$
  values of $x$ at which $f-q$ takes values alternating in sign 
  $$
  d_1,-d_2,d_3,\ldots, (-1)^{n+1}d_{n+2}.
  $$

  \textit{Then\pageoriginale the best approximation $d$ satisfies}
  $$
  d \geq \min d_i
  $$
\end{theorem}

\begin{proof}
  Suppose that $d<d_i (i=1, \ldots, n+2)$ and let $p$ be the best
  $P_n$. Then $p-q=(f-q)-(f-p)$ takes alternate signs at the $n+2$
  values in the hypothesis. Therefore $p-q$ (which is in $P_n$) has
  at least $n+1$ zeros. This is a contradiction. 
\end{proof}

\begin{coro*}
  Let $q$ be in $P_n$ and let
  $$
  \sup |f-q|=d'.
  $$
\end{coro*}

Suppose that $f-q$ takes the values $\pm d'$ alternately for $n+2$
values of $x$. Then $d'=d$ and $q$ is the best $P_n$. 

\begin{proof}
  By theorem \ref{chap2:sec5:thm6}, $d\geq d'$. But $d\leq d'$, since $d$ is the best
  approximation. 
\end{proof}

\section{Chebyshev polynomials}\label{chap2:sec6}

Theorem \ref{chap2:sec5:thm4} guarantees the existence of the best $P_n$ for a given
$f$. It is only in special cases that the explicit calculation of
this polynomial is practicable. Theorem \ref{chap2:sec5:thm7} and its corollary can
often be turned to use. 
 
\noindent
\textbf{Easy exercises:}

For $x^2$ in (0, 1), the best $P_0$ is $\dfrac{1}{2}$, the best $P_1$
is $x+ \dfrac{1}{8}$.  

For $x^4$ in $(-1, 1)$, the best $P_3$ is $x^2+ \dfrac{1}{8}$.

Consider now the general problem.
\begin{enumerate}[A.]
\item Among all $p_n(x)$ with coefficient of $x^n$ equal to $1$, find
  that which deviates least from $0$ in $(-1,1)$ in other words, that
  for which\break $\sup |p_n(x)|=d$ is least. 

  This problem can be stated in the equivalent form.

\item Find the best approximation in $P_{n-1}$ to $x^n$ in $(-1,1)$.
\end{enumerate}

From\pageoriginale Theorem \ref{chap2:sec5:thm7} (Corollary) we wish to find a $p_n(x)$ which takes
the values $\pm d$ alternately at $n+1$ points (why not $n+2$
points?). Enlightened guessing soon leads to the answer 
$$
p_n(x)=d \cos n \theta ~\text{where}~ x=\cos \theta.
$$

It is worth while to give Chebyshev's own proof of this, which does
not depend on guesswork. 

\begin{theorem}\label{chap2:sec6:thm8}%theorem 8
  Among all $p_n (x)$ with coefficient of $x^n$ equal to $1$, the polynomial
  $$
  2^{-n+1} \cos n \theta ~\text{where}~x= \cos \theta
  $$
  deviates least from $0$ in $(-1,1)$.
\end{theorem}

\begin{proof}
  let $p(x)=x^n+ \cdots$ be the required polynomial, and $d= \sup |p(x)|$.
\end{proof}

By Theorem \ref{chap2:sec5:thm5}, there are $n+1$ values of $x$ (at least) where
$p(x)=\pm d$. These may be end-points or interior points of
$(-1,1)$. At such a point which is an interior point, $p(x)$ has a
maximum or minimum and so $p'(x)=0$. Since $p'(x)$ has degree $n-1$,
the $n+1$ values must be $1,-1$ and $n-1$ others, say $x_1,x_2, \ldots
,x_{n-1}$.  

The two polynomials of degree $2n$
$$
d^2-p^2 ~\text{and}~ (1-x^2)p'^2
$$
have the same zeros, namely, $1,-1$ and each of $x_1,x_2,
\ldots,x_{n-1}$ doubly. Comparing the coefficients of $x^{2n}$ we have 
$$
n^2(d^2-p^2)=(1-x^2)p'^2.
$$

Solving this differential equation for $p$ we find, putting $x=\cos
\theta$,  
$$
p=d \cos (n \theta + C)
$$

Since $p(x)$ is a polynomial, $C=0$. But
\begin{align*}
  \cos n \theta &=2^{n-1} \cos^n \theta + \text{lower powers of}~ \cos
  \theta, \hspace{1cm}\\
  ~\text{and so}~ \hspace{2cm} &=2^{-n+1}.
\end{align*}

The\pageoriginale polynomials revealed by Theorem \ref{chap2:sec6:thm8} are named after Chebyshev and
(following the alternative spelling of his name) we define 
$$
T_n(x)=\cos (n ~\text{arc}~ \cos x).
$$

The early members of the sequence are
\begin{alignat*}{4}
  T_ 0(x) & =1,&& T_1(x)=x \qquad T_2(x)=2x^2-1\\
  T_3(x) & =4x^3 - 3x, \qquad& &T_4 (x)=8x^4 -8x^2 + 1.
\end{alignat*}

Their mode of definition is restricted to $(-1,1)$ and it is in that
interval that their utility mainly lies. But many of their properties
hold for all values of $x$. Some useful results are collected in
Theorem \ref{chap2:sec6:thm9}; the proofs can easily be supplied. 
\begin{theorem}\label{chap2:sec6:thm9} %theorem 9
  \begin{enumerate}[(1)]
  \item $y=T_n(x)$ satisfies the differential equation
    $$
    (1-x^2)y''-xy' +n^2y=0.
    $$
  \item $T_n(x)$ is the coefficient of $t^n$ in the expansion of
    the generating function 
    $$
    \frac{1-tx}{1-2tx+t^2}
    $$
  \item the recurrence relation
    $$
    T_n(x)=2xT_{n-1}(x)-T_{n-2}(x)\qquad (n \geq 2).
    $$
  \item an explicit formula for the coefficients
    $$
    T_n(x)=\sum (-1)^k \frac{n}{n-k} ~\binom{n-k}{n}~ 2^{n-2k-1} x^{n-2k}
    $$
    summed for $0 \leq k \leq \bigg [ \dfrac{n}{2} \bigg ]$
  \item orthogonality with the weight-function $1/ \sqrt{(1-x^2)}$
    $$
    \int^1_{-1} \frac{T_n (x)T_m(x)}{\sqrt{(1-x^2)}} dx= \begin{cases}0
      (m\neq n)\\ \frac{1}{2}\pi (m=n) \end{cases} 
    $$
\item for $|x| > 1$,
  $$
  2T_n(x)=\{x+ \sqrt{(x^2-1)\}^n} +\{x+ \sqrt{(x^2-1)\}^n}.
  $$
  \end{enumerate}
\end{theorem}

\medskip
\begin{center}
\large{\bf Note}
\end{center}

The\pageoriginale calculation of polynomials of best approximation is in practice
troublesome. See de la Vallee Poussin, Chapter $VI$. For a method of
calculation by a convergent sequence, see Polya, Comptes Rendus
(Paris) 157 (1913), 840. 
