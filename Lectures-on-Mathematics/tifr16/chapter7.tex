
\chapter*{APPENDIX \\ Approximation by Polynomials in the Complex
  Domain}

\setcounter{section}{0}
\section{Runge's Theorem}

The\pageoriginale problem considered till now was the approximation of a given
continuous function on a finite closed interval by polynomials in a
real variable. Even for functions of two variables, we considered only
the problem of approximation by polynomials in two independent real
variables $x,y$. in what follows, we shall consider the approximation
of a function in a domain in the plane (open connected set) by
polynomials in the complex variable $z=x+iy$ (which are analytic
functions of the variable $z$).

Let $p_n(z)$ be a sequence of polynomials and suppose that $G$ (which
we assume is not empty) is the largest open  set in which $p_n(z)$
converges, \textit{uniformly on every compact subset}. (This is the
type of approximation  we shall consider; the problem of approximation
on closed sets more difficult). By Weierstrass's theorem the limit of
the sequence $p_n(z)$ is an analytic function in $G$.  There is
moreover, a purely topological restriction on $G$, viz., every
connected component $D$ of $G$ is simply connected: for, if $C$ is a
sample closed curve contained in $D$ and $B$ is its interior, then
(maximum modulus principle)  
$$
\sup_{z \in  B \cup C} | p_n(z) - p_m(z)  | = \sup_{z \in C} | p_n(z)-
p_m(z) | \to ~\text{as}~0 n,m \to \infty 
$$
so that the sequence $p_n(z)$ converges uniformly on $B \cup C$. Hence
$B \cup C \subset G$ and since $D$ is a connected component and $C
\subset D, B \subset D$. 

The\pageoriginale main theorem, which is the analogue of Weierstrass's approximation
theorem $(Th. 1, p.2)$ and which includes a converse of the remarks
made above, runs as follows. 

\begin{theorem*}[{\bf A (Runge)}]
  Let $D$ be a domain in the plane  and $f$ an analytic
  function in $D$. Then $f$ can be approximated, uniformly on every
  compact subset of $D$, by rational functions whose poles lie outside
  $D$. If  $D$ is simply  connected, $f$ may be approximated by
  polynomials. 
\end{theorem*}

We begin by funding a sequence of open regions $G_n, n = 1,2,3,
\ldots$ bounded by polygons such that $G_n$ is relatively compact in
$G_{n+1}$,  whose limit is $D$. We may take $G_n$ as a subsequence of
the sequence $B_m$, where $B_m$ is defined to be the interior of the
union  of those squares $\dfrac{k}{2^m} \le x \le \dfrac{k+1}{2^m},
\dfrac{1}{2^m} \le y \le  \dfrac{l+1}{2^m}, k,1$ integers, $|k|, |l|
\le 2^{2^m}$, which lie in $D$. The boundary of $G_n$ can be split
into a finite number of simple closed polygons $C_{n,k}$ can be joined
by a simple are which does not meet $G_n$ to a point on the boundary
of $D$. If $C_n = \bigcup\limits_k C_{n,k}$ is the boundary of $G_n$
we have  
$$
f(z) = \frac{1}{2 \pi i} \int\limits_{C_n} \frac{f(t)}{t-z} dt,  z \in
G_n. 
$$

By the definition of the integral, we may approximate $f(z)$ uniformly
in $G_{n-1}$ by finite sums of the form 
$$
f_n(z) = \frac{1}{2 \pi  i} \sum \frac{f(t_r)}{t_r -z} (t_{r+1}- t_r)
$$
where $t_r$ are certain  points on $C_n$. Hence if  $\varepsilon_n
\downarrow 0$, we can find a sequence of rational functions $R_n$ such
that $R_n$ has poles at most on $C_n$ and  
\begin{equation}
  |f(z) - R_n(z) |  < \varepsilon_n ~\text{in}~ G_{n-1}. \tag{1}
\end{equation}

The\pageoriginale main idea in the proof of the theorem is contained in the next
step, which we state as a separate lemma. 
\begin{lemma*}
  Let $C$ be a simple are joining the points $z_o$ and $z_1$ and $K$
  a compact set not meeting $C$. Then, any  rational function whose only
  possible pole is at $z_o$ can be approximated, uniformly on $K$, by
  rational functions which no poles expect possibly at $z_1$. 
\end{lemma*}

\noindent \textbf{Proof of the lemma. }
  Let $\varepsilon  > 0$  be given and $2d$ be the distance between
  $C$ and $K$. We find points $a_o,  \ldots, a_ \mu$ on $C, a_o= z_o,
  a_\mu = z_1$ such that $|a_{k+1}- a_k | \le d$. Let $R(z)$ be the
  given rational function. There are two polynomials $p$ and $q$ so
  that  
  $$
  R(z) = p(z) +  q \left(\frac{1}{z-z_o}\right)
  $$
  and we have only to approximate $ q \left(\dfrac{1}{z-z_o}\right)=
  f(z)$. Since 
  $f(z)$ is analytic in $|z- a_1 |> d$ and is finite at $\infty$, the
  Laurent expansion of $f(z)$ about $a_1$ contains no positive powers
  of $z- a_1$ and converges uniformly on every compact subset of
  $|z-a_1| > d$. $A$ suitable partial sum then gives us a polynomial
  $p_1$ with $\bigg | f(z) - p_1 \left(\dfrac{1}{z-a_1}\right) \bigg | <
  \dfrac{\varepsilon}{\mu +1}$ for  $z \in K$. Repeating this process,
  we find successively,  polynomials $p_j, j=1, \ldots,  \mu$ with  
  $$
  \bigg| p_j \left(\frac{1}{z-a_j}\right) - p_{j+1}
  \left(\frac{1}{z-a_{j+1}}\right) \bigg | 
  < \frac{\varepsilon}{\mu +1} ~\text{for}~ z \in K. 
  $$

  Then clearly 
  $$
  \bigg | f(z) - p_\mu \left(\frac{1}{z-a_\mu}\right) \bigg | <
  \varepsilon, z \in K 
  $$
  and the lemma follows.

\begin{proofofthm*}
  Let $D$ be any domain and $f$ analytic in $D$.  Let $G_n$ be the
  sequence of regions exhausting $D$  described  above. There is a\pageoriginale
  rational function $r_n$ with poles at most on $C_{n+1}$ such
  that  
  $$
  \bigg | f(z) - r_n(z) \bigg | < \frac{1}{2n} ~\text{on}~ G_n.
  $$
\end{proofofthm*}

Every point of the boundary of $G_{n+1}$ can be joined be an arc not
meeting $\bar{G}_n$ to the boundary of $D$, so that, by  the lemma,
there is a rational function $R_n$ with poles outside $D$ such that
$\bigg | R_n(z) - r_n(z) \bigg | < \dfrac{1}{2n}$ on $G_n$ and $| f(z)
- R_n(z) | < \dfrac{1}{n}$  on $G_n$. The first part of Runge's
theorem is proved.  If $D$ is simply connected, then every connected
component of the complement is unbounded (unless $D$ is the whole
plane in which case the theorem is trivial). Hence every point of the
boundary of $G_{n+1}$ can be joined to a point $z_1(|z_1| \ge 2r)$ by
an arc which does not meet $\bar{G}_n$, $r$ being such that $G_n$ is
contained in the circle $|z|< r$. 

Now it follows as above that there is a rational function  $R_n(z)$
with all its poles lying in $|z| \ge 2r$, with  
$$
\bigg | f(z) - R_n(z) \bigg |< \frac{1}{2n} ~\text{on}~ G_n.
$$

If we expand $R_n(z)$ in a Taylor series about $z=0$ (which converges
uniformly for $|z| \le r$), then a suitable partial sum $p_n(z)$
satisfies  
$$
\displaylines{\hfill 
  \bigg | R_n(z) - p_n (z) \bigg | < \frac{1}{2n} ~\text{on}~ G_n \hfill \cr
  ~\text{so that}\hfill   
  \bigg | f(z) - p_n (z) \bigg | < \frac{1}{n} ~\text{on}~ G_n.\qquad  \hfill}
$$

This complete the proof  of Runge's theorem

The same argument proves the following theorem

\begin{Thm*}[{\bf A$^1$}] %theorem A
   Let $D$ be any plane domain. From each  connected
    component of the complement of $D$,  choose a point $z_\alpha$. Then
    any analytic function in\pageoriginale $D$  can be approximated uniformly on
    every compact set in $D$ by rational functions which have poles at
    most at the points $z_\alpha$. 
\end{Thm*} 
 
 Runge's theorem is of importance in the theory of functions.  As an
 instance of its applicability we prove the following extension to an
 arbitrary of Mittag-Leffler's theorem 

 \begin{Thm*}
   Let $D$ be a plane domain and $a_\nu ,  \nu = 1,2, \ldots$ sequence
   of points in $D$ having no limit point in $D$. Let $p_\nu$be
   polynomials (without constant term). Then there is a meromorphic
   function $f$ in $D$ with poles at most at the $a_\nu$ such that
   $f(z)-p_\nu \left(\dfrac{1}{z-a_\nu}\right)$ is analytic at $a_\nu$. 
 \end{Thm*} 

\begin{proof}
  We can construct a sequence $G_n, n=1,2, \ldots$ of regions so that
  $G_n$ is relatively compact in $G_{n+1}, \bigcup_n G_n =D$ and so
  that any point of the boundary of $G_n$ can be joined to a point not
  in $D$ by an arc not meeting $G_n$. Let  
  $$
  f_n (z) = \sum_{a_\nu  \in G_n} p_\nu \left(\frac{1}{z-a_\nu}\right)
  $$
  the sum  being over those (finitely many) $a_\nu$ which lie in
  $G_n$. Since  $f_{n+2}- f_{n+1}$ is analytic in $G_{n+1}$, we can
  find a rational function $R_{n+1}$ with no poles in $D$ such that  
  $$
  \bigg | f_{n+2} (z) - f_{n+1}(z) - R_{n+1}(z) \bigg | <
  \frac{1}{2^n}  ~\text{for}~ z ~\text{in}~ G_n. 
  $$
\end{proof} 

Since the poles of the $R_{n+1}$ lie outside $D$ and the series 
$$
\sum_{n=n_o +1}^\infty (f_{n+1} -f_n -R_n)
$$
converges\pageoriginale uniformly in $G_{n_o}$, it follows easily that
we may take  
$$
f(z)  = f_2(z) + \sum^{\infty}_{n=2} (f_{n+1}(z) - f_n(z) - R_n (z)).
$$

\section{Interpolation} % \sec 2
 
 For functions $f$ of a real variable, if $p_n$ is the (Lagrange)
 polynomial $p$ of degree $n$ which agrees with $f$ at $n+1$ equally
 spaces points on an interval, the sequence $p_n$ in general diverges
 as $n \to \infty$. The behaviour for functions of a complex variable
 is more satisfactory. We proceed to prove two of the main theorems. 
 
 Let $C$ be a simple closed rectifiable curve and $f(z)$ a function
 analytic inside and on $C$. Let $t_1,  \ldots,  t_{n+1}$ be $n+1$
 points inside $C$ (not necessarily distinct). Then the polynomial
 $p_n(z)$ of degree $n$ such that $p_n (t_i) = f(t_i), i=1, \ldots, 
 n+1$ (multiplicity being taken into account if some of the $t_i$
 coincide) is easily seen to be given by  
 $$
 f(z) - p_n(z) = \frac{1}{2 \pi i} \int\limits_{C} \frac{(z-t_1)
 \cdots (z-t_{n+1})}{(t-t_1)\cdots (t-t_{n+1})} \frac{f(t)}{t-z} dt. 
 $$
 
Our first theorem is as follows. It is also due to Runge. 

\begin{Thm*}[{\bf B}] %theorem B
  Let $f(z)$ be analytic for $|z| <  R, R> 1$  and $p_n(z)$ the
  polynomial of degree $n$ with $p_n(z_i) = f(z_i), i=0, 1, \ldots, 
  n$, where the $z_i$ are the $(n+1)^{th}$ roots of unity. 
 
  Then
  $$
  p_n(z) \to f(z) ~\text{as}~ n \to \infty
  $$
  uniformly for $|z| \le \varrho <R$.
\end{Thm*}

\begin{proof}
  Let\pageoriginale $C$ be the circle $ | z |= \rho', \rho'> \rho, \rho>1$. Then,
  for $|z| \le \rho$, 
  $$
  f(z)-p_n(z)= \frac{1}{2 \pi i} \int_c \frac{z^{n+1}-1}{t^{n+1}-1}
  \frac{f(t)}{t-z}dt, 
  $$
  so that
  \begin{align*}
    |f(z)- p_n(z) | &=  \frac{1}{2 \pi} | \int_c
    \frac{z^{n+1}-1}{t_{n+1}-1}  \frac{f(t)}{t-z} dt | \\ 
    &\le  \frac{1+ \rho^{n+1}}{(\rho'^{n+1}-1)(\rho'- \rho)} M~
    (M=\sup_{z \in C} |f (z)|) \\ 
    & \to  0 ~\text{as}~ n  \to \infty ~\text{since}~ \rho' > \rho, \rho'>1.
  \end{align*}
\end{proof}

The next theorem is due to Fejer and is considerably deeper; it
contains Theorem $B$. 

Let $C$ be a simple closed curve and suppose that $w(z)$ maps the
exterior of $C$ one-one conformally onto $| w|> 1$ in such a way that
the points at infinity correspond. Then, as is well known, $w(z)$ is
one-one continuous on $C$. Let $\alpha^{(n)}_i, i=0, 1,\ldots, n$ be the
$n+1$  points of $C$ corresponding to the $(n+1)^{th}$ roots of unity
in the  w-plane. Then we have 

\begin{Thm*}[{\bf C}]%theorem C
  Let $f(z)$ be a function analytic inside and on $C$and $p_n(z)$ the
  polynomial of degree $n$ which equals $f(z)$ at the points
  $\alpha^{(c)}_i$. Then $p_n (z) \to f(z)$, uniformly inside and on $C$. 
\end{Thm*}

We begin with a lemma.
\begin{lemma*}
  $$
  \lim_{n \to \infty} \prod^n_{i=0} \left| z-
  \alpha^{(n)}_i\right|^{\frac{1}{(n+1)}} = A | w(z) |.  
  $$
  uniformly\pageoriginale on any compact set exterior to $C, A > 0$ being a constant
  (depending on $C)$. 
\end{lemma*}

\noindent \textbf{Proof of the lemma. }
  Let $z=z(w)$ be the inverse of $w=w(z)$ and let $w_o ,  \ldots, 
  w_n$ be the $(n+1)^{th}$ roots of unity. We prove first that  
  \begin{enumerate}[1.]
  \item $\lim\limits_{n\to \infty} \prod\limits^n_{i=0} \bigg |
    \frac{z(w) - z(w_i)}{w-w_i} \bigg|^{\frac{1}{(n+1)}} = A$. 
    
    The logarithm of the term on the left is 
  \item $\lim\limits_{n \to \infty} \frac{1}{n+1} \sum_{i=o}^n  \log
    \bigg  |  \frac{z(w) - z(w_i)}{w-w_i} \bigg |  = \frac{1}{2 \pi}
    \int\limits_o^{2 \pi} \log \bigg |  \frac{z(w)-z(e^{i
        \theta})}{w-e ^{i \theta}} \bigg|d \theta $ 
  \end{enumerate}
  and the limit is uniform for $w$ in a compact set in $|w|> 1$.
 
 Now $\dfrac{z(w) - z(\zeta)}{w - \zeta}$ is an analytic function of
 $\zeta$ for $| \zeta | >1$ and fixed $w$, including $\zeta=\infty,
 \zeta= w$. Hence the integral in $(2)$ is equal to  
 $$
 \lim_{\zeta \to \infty} \log  \bigg | \frac{z(w) - z(\zeta)}{w -
   \zeta} \bigg | = \log A, ~\text{say} 
 $$
 (we have only to make the substitution $\zeta \to 1/ \zeta$ and use
 the Poison integral). 
 
From  (1), it follows that  
\begin{align*}
  \lim_{n \to \infty} \prod_{i=o}^n \frac{|z(w)
    -z(w_i)|^{\frac{1}{(n+1)}}}{|w- w_i|^{1/ (n+1)}} & = \lim_{n \to
    \infty} \frac{\prod\limits_{i=0}^n |z(w) -z
    (w_i)|^{\frac{1}{(n+1)}}}{|w^{n+1}-1 |^{1/(n+1)}} \\ 
  & = \frac{\lim\limits_{n \to \infty} \prod\limits_{i=0}^n |z(w) -
    z(w_i)^{\frac{1}{(n+1)}}}{|w|} =A 
\end{align*} 
and the lemma follows on substituting $w=w(z)$.

\begin{proofofthm*}
  Let\pageoriginale $C_R(R>1)$ denote the image under $z(w)$ of the circle $|w| =
  R$; we can choose $R >  1$ such that $f$ is analytic inside and on
  $C_R$. Let $1 < r_1  < r_2 < R$; and put  
  $$
  \pi_n(z) = \prod_{i=o}^n (z- \alpha^{(n)}_i).
  $$
\end{proofofthm*} 
 
We have 
$$
f(z)- p_n(z) = \frac{1}{2 \pi i} \int\limits_{C_{r_2}}
\frac{\pi_n(z)}{\pi_n (t)} \frac{f(t)}{t-z} dt. 
$$

If $z$ is on $C_{r_1}$ and $t$ on $C_{r_2}$, 
 $$
\displaylines{\hfill
  \lim_{n \to \infty} \bigg |\frac{\pi_n(z)}{\pi_n (t)} \bigg|^{\frac{1}{n+1}} 
  = \frac{r_1}{r_2} ~\text{(by the lemma)} \hfill \cr
  \text{so that}\hfill  
  \overline{\lim_{n \to \infty}} \bigg [ \sup_{z \in C_{r_1}} |f(z) -
    p_n (z) | \bigg]^{\frac{1}{n+1}} \le \frac{r_1}{r_2} < 1. \qquad \hfill }
$$
 
Consequently $f(z) - p_n (z) \to o$ uniformly for $z$ on $C_{r_1}$.
 
The theorem follows at once from the maximal modulus principle.
 
\section{Best Approximation}

 In this section we shall consider the problem  of best approximation. 
 
Let\pageoriginale $K$ be a  compact set containing infinitely many points and $f(z)$
a \textit{continuous} function on $K$. Our aim is to prove the
existence and uniqueness of a polynomial $p_n(z)$ of degree  $n$ such
that  
 $$
 d(f,p_n) = \sup_{z \in K} \bigg | f(z) - p_n (z) \bigg |
 $$
 is least. in general, of course, this  minimum $d(f,p_n)$ does not
 tend to zero as $n \to \infty$. 
 
 \noindent
 \textbf{Existence of a polynomial of best approximation}.
 
 Let $P_n$ be the family of all polynomials of degree  $\le n$, and
 let $f(z)$ be a continuous function on the compact set $K$. Let  
 $$
 d(f) = d = \inf_{p \in P_n} d(f,p) = \inf_{p \in P_n} (\sup_{z \in K}
 | f(z) - p(z) |). 
 $$
 
 Then we  have the 
 \begin{Thm*}[{\bf D}] %theorem D
   There exists a $p \in  P_n$ with $d(f,p) =d$.
 \end{Thm*} 

\begin{proof}
  Any polynomial $p \in P_n$ takes values  $0$ or $1$ at $n$ points at
  most. Hence $P_n$ is a quasi-normal family of order $n$, (theorem
  of Montel, see \cite{1} p. 67) i.e., given a sequence $p_\nu$ of
  polynomials in $P_n$, there is a subsequence $p_{\nu_k}$ and $n$
  points $z_i$ such that $p_{\nu_k}$ converges, uniformly on every
  compact set not containing the $z_i$, either to a finite limit
  function or to $\infty$. In the first case it is clear that
  $p_{\nu_k}$ converges uniformly on any compact set (which may
  contain some   of the $z_i$). 
\end{proof} 
 
 There is a sequence $p^{(\nu)}$ of polynomials of $P_n$ so that
 $d(f,p^{(\nu)}) \to d$. Then, clearly, if $z \in  K, |p^{(\nu)} (z) |
 \le d+1+ \sup\limits_{\zeta \in K}| f(\zeta)|$ for\pageoriginale large $\nu$ (we  may
 suppose that holds for all $\nu$). Let  $p^{(\nu_k)}$ be a
 subsequence converging outside $n$ points $z_i$, uniformly on compact
 sets. Since $K$ contains infinitely many points there are points of
 $K$ not equal to any $z_i$, and it these points  $z,|p^{(\nu_k)}(z)
 |$ is bounded. Hence the limit outside $z_i$ is finite and
 consequently, $p^{(\nu_k)}$ converges uniformly on any compact
 set. From Cauchy's inequality, it follows then that the  corresponding
 co-efficients of $ p^{(\nu_k)}$ converge, so that  $\lim\limits_{k
   \to \infty} p^{(\nu_k)} (z) = p(z) \in  P_n$. Then we have  
 $$
 d \le d(f,p) \le d(f,p^{(\nu_k)}) + d(p^{(\nu_k)}, p) \to d
 $$ 
 so that $d(f,p)=d$.
 
 [If $K$ contains a circle $|z-a| <r, r >o$, the  existence of a
   sequence $p^{(\nu_k)}$ converging uniformly on any  compact set
   follows at once, as in  the case $(Ch.II, Th.4, p. 14)$ if we use
   the Cauchy  inequalities.] 
 
 \noindent
 \textbf{Uniqueness of the polynomial of best approximation}.
 
 We shall deduce the uniqueness fro  the following theorem, as in the
 case of a real variable. 
 
 Let $p \in P_n$ satisfy $d(f,p) =d(f)$. Then $|f(z)- p(z) |$
 attains its maximum at atleast $n+2$ distinct points of $K$. 
 
 (The proof is similar in principle to the proof of  $Th.5, p.14$). 
 \begin{proof}
   Suppose that $f(z) - p(z)= g(z)$ attains its maximum modulus at
   $m$ points $(m \le n+1) z_1, \ldots,  z_m$ of $K$. Then, we can
   construct a polynomial $q(z)$ of degree $n$ such that $q(z_i)=
   g(z_i)$. Given $\varepsilon >o$, 
   we\pageoriginale can find $\delta > 0$ so that if 
   $$
   | \zeta_1 -\zeta_2 | < \delta, | g(\zeta_1)-g(\zeta_2)| <
   \varepsilon, | q(\zeta_1)- q(\zeta_2) | < \varepsilon 
   $$
 \end{proof}

Let $K^1$ be the set obtained from $K$ by removing the points of the
(open) discs $| z- z_i | < \delta$. Then 
$$
\sup_{z \in K^1} | g(z) | = d^1 < d = \sup_{z \in K} | g(z) |.
$$

Let $1 > \eta > o$ be sufficiently small. Consider $g(z)- \eta q(z)$;
then for $| z-z_i|< \delta, | g(z)- g(z_i) | < \varepsilon, | \eta
q(z)- \eta q(z_i)| < \eta \varepsilon$, so that $|g(z) - \eta q(z)| = |
\eta (g(z_i)- g(z)) + \eta g(z)- g(z) + \eta (q(z)- q(z_i))|$ (since
$q(z_i)= g(z_i)$) $< \eta \varepsilon + \eta \varepsilon + (1-\eta) d
< d$ if $2 \varepsilon < d$ 

If we choose $\eta$ so small that 
$$
\displaylines{\hfill 
  \sup_{z \in K^1} | g(z) - \eta q(z) | < d\hfill \cr
  \text{we have}\hfill  
  \sup_{z \in K} | g(z) - \eta q(z) | < d\hspace{1.3cm} \hfill}
$$
and $d(f, p+ \eta q) < d$, contradicting the definition of $d$. 

\begin{Thm*}[{\bf E}]%theorem E
  The polynomial of degree $\le n$ of best approximation is unique. 
\end{Thm*} 

\begin{proof}
  Let $d(f, p) = d = d(f, q)$; let $r(z)= \dfrac{1}{2} (p(z)+ q(z))$.
\end{proof}
Then 
$$
|f(z)- r(z) | = | \frac{1}{2} (f(z)-p(z)) + \frac{1}{2} (f(z)- q(z))| \le d
$$

Let\pageoriginale $z_1, \ldots, z_{n+2}$ be points at which $| f(z_i) - r(z_i) | =
d$. Then, unless $f(z_i)-p(z_i)=f(z_i)-q(z_i)= w_i$ with $| w_i |= d,
\dfrac{1}{2} | f(z_i)- p(z_i) + f(z_i) -q(z_i) | < d$. Hence, $p(z)$
and $q(z)$ take the same value at the $n+2$ points $z_i$ and since
they are polynomials of degree $n,p(z) = q(z)$.  

\begin{thebibliography}{99}
\bibitem{1}{Montel, P.} : Lecons sur les familles normales de
  fonctions analytiques et leurs applications, Gauthiers-Villars,
  Paris, $1927$ 
\bibitem{2}{Walsh,J.L.} : Approximation by polynomials in the
  complex domain, Mem. des Sciences Mathem., Paris, $1935$ 
\bibitem{3}{Walsh, J.L.} : Interpolation and approximation by
  rational functions in the complex domain,
  Amer. Math. Soc. Colloquium Publications, $Vol.XX$, $1935$. 
\end{thebibliography}

