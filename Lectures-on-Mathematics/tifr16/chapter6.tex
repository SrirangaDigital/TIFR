\chapter{Approximation in Terms of Differences}\label{chap6}

\setcounter{section}{14}
\section{}\label{chap6:sec15}

 This\pageoriginale is the only  chapter in the course, of which the results are not
 classical. The point of view here might lead to a re-orientation
 towards algebraic rather than trigonometric polynomials. 
 
 In Theorem \ref{chap4:sec11:thm20} (and its known extensions) the approximation
 attainable in $P_n$ or $T_n$ to a differentiable function $f(x)$ is
 expressed in terms of its first higher derivative. We shall now give
 simple examples which lead us to suppose that bounds of
 \textit{differences} of $f(x)$ rather than derivatives may be more
 directly related to the closeness of the approximation. 

 \begin{example}\label{chap6:sec15:exp1}%example 1
   If $f(x)$ attains its greatest value at $x_2$ and its least at
   $x_1$, then the best approximation in $P_0$ is  
   \begin{align*}
     &\frac{1}{2} \left\{f(x_1) +f(x_2) \right\}\\
     and  \hspace{3cm} d_o =  &\frac{1}{2} \left\{f(x_2) -f(x_1) \right\}=
     \frac{1}{2} \sup |  \Delta f |\hspace{2cm} 
   \end{align*}
 \end{example} 
 
 This depends solely on the first difference of $f(x)$; the derivative
 of $f(x)$- if it exists-has no  bearing on it. 
 
Now raise the degree by one.
\begin{example}\label{chap6:sec15:exp2} %example 2
  If $f(x)$ is $C(0,1)$, there is a linear function $p(x)$ for which 
  $$
  | f(x) - p(x) | \le \sup |f (x+2h) - 2f(x+h) +f(x) |,
  $$
  the  sup being taken over all $x,h$ such that 
  $$
  0 \le x \le x+ 2h \le 1.
  $$
\end{example} 

\begin{proof}
  Define\pageoriginale $p(x)$ to be equal to $f(x)$ at $x=0$ and $x=1$ Write
  $$
  g(x) = f(x) -p(x).
  $$
\end{proof} 
 
Then $|g(x) | $ attains its maximum,  $M$, for $0 \le x \le 1$ at $x_1$, say. 

If $0 \le x_1 \le \dfrac{1}{2}$, take $x=0, h=x_1$. Then 
$$
|g(x+ 2h) - 2g(x+h) +g(x) | = |g(2x_1)- 2g(x_1) + g(0) | \ge M.
$$

If $\dfrac{1}{2} < x_1 \le 1$ take $x=2x_1 -1$, $h=1- x_1$. Then 
$$
|g(x+ 2h) - 2g(x+h) +g(x) | = |g(1)- 2g(x_1) + g(1-x_1) | \ge M.
$$

But the second difference of $g(x)$ and $f(x)$ are equal.


By a longer argument it is possible to prove the corresponding result
for $P_2$ and the third difference. 

The general result was conjectured in $1949$ by H. Burkill.

\begin{theorem}\label{chap6:sec15:thm25} % thm 25
  There is a number $K_n$ depending only on $n$ such that given $f(x)$
  in $C(a,b)$, there is a  polynomial $p(x)$ in $P_{n-1}$ for which  
  $$
  | f(x)  - p(x) | \le K_n \sup | \Delta_n (f) |
  $$
  (where the supremum is taken for all sets of $n+1$ points $x, \ldots
 ,  x + nh$ in $(a,b) $)  
\end{theorem}

The theorem looks innocent,  but attempts at it failed until Whitney it
in  $1955$. He took for his $p(x)$ the Lagrange polynomial for the
points of division   of $(a,b)$ into $n-1$ equal parts. His  work does
not yield an estimate of $K_n$ for general $n$; in view  of Theorem
$24$, we should hardly  expect good value of $K_n$. 

Whitney's\pageoriginale elegant arguments are too long for reproduction  here, and
the reader is referred to his paper in journal de Mathematiques
36(1957), 67-95. 

It is worth observing, however, that instead of the usual $n^{\text{th}}$
difference with equal increments, we can take a more general $n^{\text{th}}$
difference depending of the values of $f(x)$ at $n+1$ arbitrary
points. The difficulty then disappears and the polynomial of best
approximation can be used instead of the Lagrange polynomial. 

\section{Definition and Properties of the \texorpdfstring{$n^{th}$}{nth}
  Difference}\label{chap6:sec16} 

If 
$$
\varphi(u) = (u- h_o) (u-h_1) \cdots (u-h_n),
$$ 
the $n^{\text{th}}$ divided difference of $f(x)$ for the values specified is
commonly defined by  
$$
D_n = D_n (f; h_o,  \ldots,  h_n) = \sum_{i=0}^n \frac{f(h_i)}{\varphi'(h_i)}.
$$

In what follows it will be convenient to suppose that 
$$
h_o >  h_1 > \cdots  > h_n
$$

To define an $n^{\text{th}}$ difference $\Delta_n$, as distinct from a divided
difference, we naturally take 
$$
\Delta_n = H_n D_n, 
$$
where $H_n$ is homogeneous of degree $n$ in the $h's$.

The most suitable definition of $H_n$ appears to be
$$
\displaylines{\hfill H_n = 2^{n}/ T_n, \hfill \cr
  ~\text{where}\hfill  
  T_n = T_n (h_o, h_1, \dots,  h_n) = \sum_{i=0}^n | \varphi'
  (h_i)|^{-1}. \hfill}
$$

In\pageoriginale the special case of equal increments  with $h_o - h_n = nh$, this
gives $H_n = n! h^n$, which is right. 

As a further check on the appropriateness of our $H_n$, we observe
that if a function is numerically less than $A$, its $n^{\text{th}}$ difference
$\Delta_n$ is numerically less than $2^n A$. 

In  working with  $D_n, \Delta_n$, etc., we shall specify the function
and the values of the variables only so far  as is necessary for
clarity. 

We are now in a position to restate and prove Theorem
\ref{chap6:sec15:thm25}, taking 
$\Delta_n (f)$ to be the  difference $H_n D_n (f)$ just defined. 

\begin{theorem*}[{$\mathbf{25'}$}] %theorem 25
  Theorem \ref{chap6:sec15:thm25} is true  with  $K_n = 2^{-n}$ and $\Delta_n(f)$ as just
  defined and the supremum taken over all values of $h_o, \ldots, 
  h_n$ in $(a,b)$. 
\end{theorem*}

\begin{proof}
  Given $f(x)$, take $p(x)$ to be its polynomial of best approximation
  of degree at most $n-1$. Then $f(x)- p(x)$ takes its greatest
  numerical value at $n+1$ points, with signs alternately  $+$ and
  $-$. These $n+1$ points  we takes as $h_o,  \ldots,  h_n$. 
\end{proof}

Since the $n^{\text{th}}$ difference of a polynomial of degree  $n-1$
is 0, we have  
$$
\displaylines{\hfill 
  \Delta_n (f; h_o,  \ldots,  h_n) = H_n \sum^{n}_{i=o} \frac{f(h_i)-
    p(h_i)}{\varphi'(h_i)}\hfill \cr 
  ~\text{So}\hfill  | \Delta_n |=  H_{n^d} \sum |  \varphi' (h_i) |^{-1}=
  2^n d,\hfill}
$$
by definition of $H_n$.

Therefore, for all $x$ in $(-1,1)$,
$$
|f(x) - p(x) |  \le d \le 2^{-n} \sup | \Delta_n (f) |.
$$

This proves Theorem $25'$.

Alternatively\pageoriginale we can prove Theorem $25'$, starting from the upper
bound of $|\Delta_n |$ instead of from  the polynomial of best
approximation. 

Suppose, then,  that $ \sup | \Delta_n| =L$ and that the bound $L$ is
assumed for the values $h_o, h_1, \ldots,  h_n$ of the independent
variable. Define points $(h_i,  y_i) $ for $i=0, 1, \ldots, $ by
taking  
$$
y_i = f(h_i) -(-1)^k \frac{L}{2^n},
$$
where $k$ is $i$ or $i+1$ according as $\Delta_n (f,h_o,  \ldots, 
h_n)$ is positive or negative. 

Construct $a \,p(x)$ of $P_{n-1}$ through the $n$ points $(h_i, y_i)$
for $i=0, 1, \dots,\break  n-1$. Write  
$$
g(x)= f(x)- p(x).
$$

Since $\Delta_n (p) \equiv 0$, $|  \Delta_n (g) | = | \Delta_n (f) |$
attains its upper bound for $h_0, h_1$, $\ldots,  h_n$. From  the
definition of $\Delta_n$, the value of $g(h_n)$ which makes $|
\Delta_n (g, h_0, h_1, \ldots,  h_n) | = L$ is $(-1)^k L/2^n$, where
$k$ is $n$ or $n+1$ according as $\Delta_n (f, h_o, \ldots,  h_n)$ is
positive or negative. 

We prove that $|g(x)| \le 2^{-n} L$ for all $x$. Suppose that $|g|$
takes values greater than $L/2^n$, say $g(h'_i) > L/2^n$ for a value
$h'_i$ between $h_{i-1}$ and $h_{i+1}$ where $g(h_i)= 2^{-n}L$. Then,
from the definition of $T_n$, 
\begin{multline*}
  D_n(g, h_o, \ldots,  h_{i-1}, h'_i, h_{i+1}, \ldots,  h_n)\\ 
  > 2^{-n} LT_n (h_o, \ldots,  h_{i-1}, h'_i, h_{i+1}, \ldots,  h_n) 
\end{multline*}
and so, by definition of $\Delta_n$,
$$
\Delta_n (g,h_o, \ldots,  h_{i-1}, h'_i, h_{i+1}, \ldots,  h_n) > L,
$$
which is a contradiction. We have therefore
$$
|f(x) - p(x) | =|g(x) | \le 2^{-n} \sup | \Delta_n (f) |, 
$$
which\pageoriginale is Theorem $25'$.

For the next result let us call the $n+1$ values
$$
-1, - \cos \frac{\pi}{n}, \ldots,  \cos \frac{\pi}{n}, 1
$$
at which the Chebyshev polynomial $\cos (n$ arc $ \cos x$) assumes the
values $\pm 1$ the Chebyshev points of the interval $(-1, 1)$. 

\begin{theorem}\label{chap6:sec16:thm26}  %theorem 26
  Suppose that $1 \ge >h_o > h_1 \ge \cdots \ge h_n \ge -1$. Then
    $$
    T_n(h_o, \ldots,  h_n) \ge 2^{n-1}
    $$
    and the sign = holds if and only if the $h_i$ are the Chebyshev points.
\end{theorem}

\begin{proof}
  The polynomial $q_n(x)$ of degree $n$ which takes the value $(-1)^i$
  at $h_i(i=0, \ldots , n)$ is  
  $$
  q_n(x) = \varphi (x) \sum_{i=0}^n
  \frac{(-1)^i}{\varphi'(h_i)\,(x-h_i)}.  
  $$
\end{proof}

Then $q_n(x) =a_n x^n + \cdots + a_o$, where 
$$
a_n = \sum_{i=o}^n \frac{1}{| \varphi' (h_i) |} = T_n (h_o, \ldots, 
h_n). 
$$

Write $t_n(x)= 2^{-n+1}a_n \cos (n ~\text{arc}~ \cos x)$. 

Then $q_n(x) - t_n (x)$ has degree $n-1$ at most.

If $a_n < 2^{n-1}$, then $|t_n (x)| < 1$ and $q_n(x) - t_n(x)$ has the
sign of $q_n(x)$ for the $n+1$ values $h_o,  \ldots ,  h_1$. If $a_n=
2^{n-1}$ the same is true on the understanding that $q_n (x)- t_n(x)$
may vanish for any of these values. So the polynomial $q_n (x)-
t_n(x)$, of degree at most $n-1$, has $n$ zeros. This is a
contradiction if $a_n < 2^{n-1}$, and is only possible for  $a_n =
2^{n-1}$  when $q_n(x) \equiv t_n (x)$. This proves the theorem. 

\begin{coro*}
  For  $1 \ge h_o  \ge \cdots  \ge h_n \ge  -1, H_n \le 2 $.
\end{coro*}

\begin{thebibliography}{99}
\bibitem{A1}{Achieser,N.I.}-\pageoriginale (English by Hyman), Theory of
  Approximation, New Yor, 1956. 
\bibitem{A2}{Bertstein,S.}- Lecons sur les properties extremales et
  la meilleure approximation des fucntions analytiques $d'une$
  variables reelle, paris, 1926 
\bibitem{A3}{Jackson,D.}- The theory of approximation, (Ameriacan
  Mathematical Society Colloquium, $XI$ New York, 1930. 
\bibitem{A4}{Natanson,I,P}. - (German by Bogel), Konstruktive
  Funktionentheorie, Berlin, 1955. 
\bibitem{A5}{de la Vallee Poussin, G.J.}- Lecons sur 1' approximation
  ses functions dune variable reelle, Paris, 1919. 
\end{thebibliography}

