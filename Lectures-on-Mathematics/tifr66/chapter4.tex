
\chapter{Lattices and the functor $\tau$}\label{sec4}

Let\pageoriginale $m$ be a reductive Lie algebra with notation as in
\S\ \ref{sec3}. Let $\mathfrak{w}$ be the Weyl group of $(m, \mathfrak{h})$ and
let $t_0$ be the unique element of $\mathfrak{w}$ with $t_0 Q = -
Q$. If $\ell(\cdot)$ is the length function on $\mathfrak{w}$, then
$\ell(t_0) = \card Q $. Put $n^{\pm} = \sum\limits_{\alpha \in Q}
m_{\pm \alpha}$. For $\alpha \in Q$, choose a subalgebra
$\mathfrak{a}^{(\alpha)}$ of $m$ with $\mathfrak{a}^{(\alpha)} \sim sl
(2, \mathbb{C})$ and having a standard basis $\bar{H}_\alpha$,
$X_\alpha$, $Y_\alpha$. Let $C_\alpha$ denote the completion functor
defined on the category $\mathscr{I}_m(a^{(\alpha)})$.

\begin{definition}\label{chap4:def4.1}
Let $\mathfrak{g}$ be a Lie algebra with $m \subseteq
\mathfrak{g}$. Let $\mathscr{I}_\mathfrak{g} (m)$ denote the category
of $\mathfrak{g}$-modules $A$ which satisfy the following conditions:

(i) A is a weight for $\mathfrak{h}$ with integral weights (ii) for
$\alpha \in Q$ and simple, $X_\alpha$ acts locally nilpotently on $A$
and  (iii) as a $U(n^-)$-module, $A$ is torsion free.
\end{definition}

\begin{lemma}\label{chap4:lem4.2}
Let $\alpha \in Q$ be simple. Then $C_\alpha$ maps
$\mathscr{I}_\mathfrak{g}(m)$ into itself.
\end{lemma}

For a proof of (\ref{chap4:lem4.2}) and other results given without
proof consult \cite[\S \ref{sec4}]{key15}.

\begin{remark*}
It is essential to assume $\alpha$ simple. Otherwise the assertion of
the lemma is false.
\end{remark*}

\begin{definition}\label{chap4:def4.3}%% 4.3
Let $A \in\mathscr{I}_\mathfrak{g}(m)$. By a lattice above $A$, we
mean a set of $\mathfrak{g}$-modules $A_s$, $s \in \mathfrak{w}$, in
$\mathscr{I}_\mathfrak{g}(m)$ such that: (i) $A = A_{t_0}$ and (ii) if
$s \in \mathfrak{w}$ and $\alpha \in Q$ is simple with $\ell (s_\alpha
s) = \ell (s)+l$, then $A_{s_\alpha s} \subset A_s$ and $A_s \simeq
C_\alpha (A_{s_\alpha s})$.
\end{definition}

The\pageoriginale fundamental result of this section has been proved
independently by Bouaziz \cite{key5} and Deodhar \cite{key7}. This
result is 

\begin{theorem}\label{chap4:thm4.4}
Let $s \in \mathfrak{w}$ and let $s = s_{\beta_1} \cdots s_{\beta_r}
= s_{\gamma_1} \cdots s_{\gamma_r}$ be two reduced expressions for
$s$. Then the two composite functors $C_{\beta_1} \circ \cdots  \circ
C_{\beta_r}$ and $C_{\gamma_1} \circ \cdots \circ C_{\gamma_r}$ are
naturally equivalent on $\mathscr{I}_\mathfrak{g}(m)$.
\end{theorem}

This theorem implies the existence of a lattice.

\begin{coro}\label{chap4:coro4.5}
Let $A \in \mathscr{I}_\mathfrak{g}(m)$. With notations as above, put
$A_{st_0} = C_{\beta_1} \circ \cdots circ C_{\beta_r}(A)$. Then $A_s$,
$s \in \mathfrak{w}$, is a lattice above $A$. Moreover if $A'_s$, $s
\in \mathfrak{w}$, is another lattice above $A$ then there exist a
unique isomorphism $\varphi: A'_1 \xrightarrow{\sim} A_1$ which
extends the identity map $A_{t_0}$ to $A'_{t_0}$. For $s \in
\mathfrak{w}$, $\varphi$ gives an isomorphism $\varphi: A_s
\xrightarrow{\sim} A'_s$.
\end{coro}

The proof of (\ref{chap4:thm4.4}) reduces easily to the case where $m$ has rank
two. In this case both proofs then relay on an ingenious use of
specific identities in the universal enveloping algebra $U(m)$.

From the corresponding result for completions we have:

\begin{prop}\label{chap4:prop4.6}
Let $A \in \mathscr{I}_\mathfrak{g}(m)$ and let $F$ be a finite
dimensional $\mathfrak{g}$-module. If $A_s$, $s \in \mathfrak{w}$, is
a lattice above $A$, then $F \otimes A_s$, $s \in \mathfrak{w}$, is a
lattice above $F\otimes A$. 
\end{prop}

\begin{definition}\label{chap4:def4.7}
For $A \in \mathscr{I}_\mathfrak{g}(m)$, let $A_s$, $s \in
\mathfrak{w}$, be a lattice above $A$. Define $\tau(A)$ to be the
quotient $A_1/ \sum\limits_{s\neq 1}A_s$.
\end{definition}

Note\pageoriginale that $\tau(A)$ may be the zero module. Let
$\mathfrak{U}(\mathfrak{g}, m)$ be the category of
$(\mathfrak{g},m)$-modules; i.e., $\mathfrak{g}$-modules which are
$U(m)$-locally finite.

\begin{theorem}\label{chap4:thm4.8}
The map $A \mapsto \tau (A)$ is a covariant additive functor from the
category $\mathscr{I}_\mathfrak{g}(m)$ into the category
$\mathfrak{U}(\mathfrak{g}, m)$. Also $\tau$ commutes with the functor
of tensoring by a finite dimensional $\mathfrak{g}$-module.
\end{theorem}


This theorem is an immediate consequence of (\ref{chap2:thm2.7}),
(\ref{chap4:coro4.5}) and (\ref{chap4:prop4.6}). 

\medskip
\noindent{\textbf{4.9 Warning:}}
In general, completion functors are left exact but nor right
exact. However, in general, the functor $\tau$ is neither right nor
left exact. 

We complete this section by describing the transfer of invariant
pairings by the functor $\tau$. Let $\sigma$ be an antiautomorphism of
$\mathfrak{g}$ with $\sigma m = m$ and $\sigma$ equal the identity on
the CSA $\mathfrak{h}$ of $m$. Invariant pairings and forms on
$\mathfrak{g}$-modules and $m$-modules are defined using this $\sigma$
(cf. (\ref{chap3:def3.1})). Let $A,B \in \mathscr{I}_\mathfrak{g}(m)$ and $\varphi
\in Inv_\mathfrak{g}(A,B)$. Let $t_0 = s_{\gamma_1} \cdots
s_{\gamma_d}$ be a reduced expression for $t_0$ and, using (\ref{chap3:thm3.14}), let
$\varphi_1 = C_{\gamma_1} \circ \cdots \circ
C_{\gamma_d}(\varphi)$. Then, if $A_s$, $B_s$, $s \in \mathfrak{w}$,
are lattices above $A$ and $B$ respectively, $\varphi_1$ is a
$\mathfrak{g}$-invariant pairing of $A_1$ and $B_1$. In \cite{key17}
it is shown that $\varphi_1$ is independent of the choice of reduced
expression; and so, $\varphi_1$ is zero on $A_1 \times \sum\limits_{s
  \neq 1} B_s$ and $\sum\limits_{s \neq 1} A_s \times B_1$. Therefore
we have:

\setcounter{prop}{9}
\begin{theorem}\label{chap4:thm4.10}
$\varphi_1$ induces a pairing of $\tau(A)$ and $\tau(B)$.
\end{theorem}

This theorem is proved in \cite{key17}.

For\pageoriginale any $\mathfrak{b}$-module $L$, put $U(L) = U(m)
\bigotimes\limits_{U(\mathfrak{b})} L$. The main result of \cite{key17}
is:

\begin{prop}\label{chap4:prop4.11}
Let $L$ and $M$ be locally finite $\mathfrak{b}$-modules which are
weight modules for $\mathfrak{h}$ with integral weights and finite
dimensional weight spaces. Assume that $U(L)$ and $U(M)$ admit
nondegenerate $M$-invariant forms. Then the map $\tau:Inv_m(U(L),
U(M)) \to Inv_m (\tau U(L), \tau U(M))$ is surjective. Further-more,
$\tau$ maps nondegenerate pairings to nondegenerate pairings. 
\end{prop}

Let notation be as in (\ref{chap3:prop3.9}) and let $\tau$ be the
functor defined on the category $\mathscr{I}_{m\times m}
(\mathfrak{t})$. 

\begin{coro}\label{chap4:coro4.12}
Let $M$ be an irreducible Verma $m$-module. For $A,B$ objects in the
category $\mathscr{O} \otimes M$, $\tau: Inv_\mathfrak{t} (A,B) \to
Inv_\mathfrak{t}(\tau A, \tau B)$ is surjective. Also, if $\varphi \in
Inv_\mathfrak{t}(A,B)$ is nondegenerate then $\tau \varphi$ is nondegenerate.
\end{coro}

The corollary follows from (\ref{chap3:lem3.10}),
(\ref{chap3:prop3.9}) and (\ref{chap4:prop4.11}). 
