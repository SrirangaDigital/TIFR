
\chapter{Invariant pairings and forms}\label{sec3}

In\pageoriginale this section invariant pairings and forms are
introduced. The main result derived is the result on splitting
discussed in the introduction, Proposition \ref{chap3:prop3.9}. For
omitted proofs consult \cite[\S \ \ref{sec6}]{key15}.

Let $m$ be a reductive Lie algebra with CSA $\mathfrak{h}$, set of
roots $\Delta$ and positive system of roots $Q$. Fix an involutive
antiautomorphism $\sigma$ of $m$ such that $\sigma$ equals the
identity on $\mathfrak{h}$. Let $\sigma$ also denote the extension of
$\sigma$ to an antiautomorphism of $U(m)$, the universal enveloping
algebra of $m$. 

\begin{definition}\label{chap3:def3.1}
For $m$-modules $A$ and $B$, a bilinear map $\varphi: A \times B \to
\mathbb{C}$ will be called an invariant pairing if $\varphi(xa, b)=
\varphi (a,x^\sigma b)$, $x \in U(m)$, $a \in A$, $b \in B$. Let
$Inv_m(A,B)$ denote the vector space of invariant pairings of $A$ and
$B$. When $A=B$, we write $Inv_m(A)$ in place of $Inv_m(A,A)$ and call
elements of $Inv_m(A)$ invariant forms on $A$. 
\end{definition}

Let $\mathscr{B}(m)$ denote the category of $m$-modules which are
weight modules for $\mathfrak{h}$ with finite dimensional weight
spaces. For $A \in \mathscr{B}(m)$, let $A^\sigma$ denote the
$U(\mathfrak{h})$-locally finite vectors in the algebraic dual to
$A$. For $x \in U (m)$, $a \in A$ and $a' \in A^\sigma$, define
$(xa') (a) = a' (x^\sigma a)$. With this action, $A^\sigma$ becomes an
$m$-module. 

\begin{lemma}\label{chap3:lem3.2}%%% 3.2
For $A \in \mathscr{B}(m)$, (i) $A^\sigma \in \mathscr{B}(m)$; in
fact, $A$ and $A^\sigma$ are isomorphic as $\mathfrak{h}$-modules,
(ii) $A \simeq (A^\sigma)^\sigma$ and (iii) $A$ is irreducible
$\Leftrightarrow A^\sigma$ is irreducible.
\end{lemma}

\begin{lemma}\label{chap3:lem3.3}
For\pageoriginale $A \in \mathscr{B} (m)$, $A$ admits a nondegenerate
invariant form if and only if $A \simeq A^\sigma$.  
\end{lemma}

\begin{prop}\label{chap3:prop3.4}
For $A \in \mathscr{B}(m)$, if $A$ has a Jordan-H\"older series of
length $r$ then $\dim Inv_m(A) \leq r^2$.
\end{prop}

Recall now our notation for Verma modules and their irreducible
quotients, i.e., $M(\lambda)$ and $L(\lambda)$, $\lambda \in
\mathfrak{h}^*$.

\begin{prop}\label{chap3:prop3.5}%%% 3.5
For $\lambda \in \mathfrak{h}^*$, $\dim Inv_m(M(\lambda)) = \dim Inv_m
(L(\lambda)) = 1$. Moreover, the forms on $M(\lambda)$ are the pull
backs of the nondegenerate forms on $L(\lambda)$. 
\end{prop}

For $\alpha \in \Delta$, let a subscript $\alpha$ denote the root
space. Now fix a Borel subalgebra $\mathfrak{b} = \mathfrak{h} \oplus
\sum\limits_{\alpha \in Q} m_\alpha$ of $m$ and put $n^- =
\sum\limits_{-\alpha \in Q} m_\alpha$. Let $\mathcal{J}=\mathcal{J}(b)$ denote the
category of finite dimensional $\mathfrak{b}$-modules which are weight
modules for $\mathfrak{h}$. For any $L \in \mathcal{J}$, put $U(L) = U(m)
\bigotimes\limits_{U(\mathfrak{b})} L$. We now can state the very
useful technical result:

\begin{prop}\label{chap3:prop3.6}%%% 3.6 
For any $L \in \mathcal{J}$, let res denote the restrict map from $U(L)$ to $1
\otimes L$. Then, for $L$, $L' \in \mathcal{J}$, res induces a natural vector
space isomorphism
$$
res: Inv_m (U(L), U(L')) \xrightarrow{\sim} Inv_\mathfrak{h} (L, L').
$$
\end{prop}

The critical role played by (\ref{chap3:prop3.6}) involves the
splitting of certain short exact sequences. Several of the following
results are examples of this type result. 

\begin{lemma}\label{chap3:lem3.7}
Let\pageoriginale $0\to A \to B \to C \to 0$ be a short exact sequence
in $\mathcal{J}$. Assume $U(A)$ and $U(C)$ admit nondegenerate
invariant forms. Then $U(B)$ also admits a nondegenerate invariant
form and the induced sequence $0 \to U(A) \to U(B) \to U(C) \to 0$ is
split exact.
\end{lemma}

\begin{proof}
Let $\varphi_A$ and $\varphi_C$ denote nondegenerate invariant forms
on $U(A)$ and $U(C)$ respectively. Every short exact sequence in
$\mathcal{J}$ splits. So we write $B = A \oplus A'$, $A'$ an
$\mathfrak{h}$-submodule of $B$. Clearly $A' \simeq C$; and so,
$Inv_\mathfrak{h} (A') \simeq Inv_\mathfrak{h}(C)$. Let $\varphi_1$ be
the element in $Inv_\mathfrak{h}(A')$ corresponding to $res
\varphi_C$. Put $\varphi\in Inv_\mathfrak{h}(B)$ equal to the
orthogonal sum of res $\varphi_A$ and $\varphi_1$; and let $\varphi_B
\in Inv_m(U(B))$ be determined by res $\varphi_B = \varphi$. Now
$\varphi_B $ restricted to $U(A)$ equals $\varphi_A$; and so, the
restriction is non-degenerate and we have the orthogonal decomposition
$U(B) = U(A) \oplus U(A)^\perp$. Then $U(A)^\perp \simeq U(C)$ and so
the induced short exact sequence splits. But then $U(B)$ admits a
nondegenerate invariant form, the orthogonal sum of $\varphi_A$ and
$\varphi_C$. This completes the proof.
\end{proof}

\begin{definition}\label{chap3:def3.8}%%% 3.8
Let $\mathfrak{L}$ be a Lie algebra with subalgebra $\mathfrak{t}$ and
let $\mathcal{C}$ be a category of $\mathcal{L}$-modules. The category
$\mathcal{C}$ is called $\mathfrak{t}$-semisimple if every short exact
sequence in $\mathcal{C}$ splits as a sequence of
$\mathfrak{t}$-modules. 
\end{definition}

\begin{note*}
If $\mathfrak{t} \neq \mathfrak{L}$ then objects in a
$\mathfrak{t}$-semisimple category of $\mathfrak{L}$-modules need not
be isomorphic to direct sums of irreducible $\mathfrak{t}$-modules.
\end{note*}

Let $\mathscr{O}$ denote the BGG category $\mathscr{O}$ for $m$. Fix
an $m$-module $M$ and let $\mathfrak{t}$
(resp. $\mathfrak{b}_\mathfrak{t}$, $n^-_\mathfrak{t}$) denote the
diagonal subalgebra in $m \times m$ (resp. $\mathfrak{b} \times
\mathfrak{b}$, $n^- \times n^-$). Let $\mathscr{O} \otimes M$ denote
the category of $m \times m$-modules with objects of the form $A
\otimes M$, $A \in \mathscr{O}$.

\begin{prop}\label{chap3:prop3.9}%%% 3.9
For\pageoriginale any irreducible Verma module $M$, $\mathscr{O}
\otimes M$ is a $\mathfrak{t}$-semisimple category of $m \times
m$-modules. Moreover, each object in $\mathscr{O} \otimes M$ admits a
nondegenerate  $\mathfrak{t}$-invariant form. 
\end{prop}

To prove (\ref{chap3:prop3.9}) we need the following lemmas.


\begin{lemma}\label{chap3:lem3.10}%%% 3.10
Let $\lambda \in \mathfrak{h}^*$ and assume $M(\lambda)$ is
irreducible. Let $A \in \mathscr{O}$. Then there exists a natural
isomorphism of $\mathfrak{t}$-modules
$$
A \otimes M(\lambda) \xrightarrow{\sim} U = U (\mathfrak{t}) 
\bigotimes\limits_{U(\mathfrak{b}_\mathfrak{t})} (A \otimes
\mathbb{C}_{\lambda -\delta}).  
$$
\end{lemma}


\begin{proof}
The injection $A \otimes \mathbb{C}_{\lambda - \delta} \hookrightarrow
 A \otimes M(\lambda)$ is a $\mathfrak{b}_\mathfrak{t}$-module map and
 so induces a unique $\mathfrak{t}$-module map $i: U \to A \otimes
 M(\lambda)$. Both $U$ and $A \otimes M(\lambda)$ are free
 $U(n^-_\mathfrak{t})$-modules with basis $A \otimes
 \mathbb{C}_{\lambda-\delta}$; and thus, $i$ is an isomorphism.
\end{proof}

\begin{lemma}\label{chap3:lem3.11}%%% 3.11
Let $A$ be an absolutely simple $m$-module. Then the map $B\mapsto B
\otimes A$ gives an isomorphism of the category $\mathscr{O}$ of
$m$-modules with the category $\mathscr{O} \times A$ of $m \times
m$-modules.
\end{lemma}

\begin{proof}
Let $C$ be a submodule of $B \otimes A$. For $c \in C$, write $c =
\sum b_i \otimes a_i$, $a_i$ linearly independent. Since $A$ is
absolutely simple choose $u_i \in U(m)$ such that $u_i a_j =
\delta_{ij} a_j$. Then $b_i \otimes a_i = (1\otimes u_i)$ $c\in
C$. This proves $C$ has the form $B'\otimes A$ for some submodule $B'$
of $B$. The lemma follows easily from this fact.
\end{proof}

\medskip
\noindent{\textbf{Proof of 3.9.}}
Let $0 \to A \to B \to C \to 0$ be a short exact sequence in
$\mathscr{O}$ and put $M= M(\lambda)$. Then
\begin{equation*}
0 \to A \otimes M \to B \otimes M \to C \otimes M \to 0 \tag{3.12}\label{eq3.12}
\end{equation*}\pageoriginale
is a short exact sequence in $\mathscr{O} \otimes M$ and all such are
of this form. We now proceed by induction on the length of a
Jordan-H\"older series for $B$. If this length is one then either $A$
or $C$ is zero and the sequence splits. Also then $B \otimes M$ is an
irreducible object in $\mathscr{O} \otimes M$; and so, by (\ref{chap3:prop3.5}) $B
\otimes M$ admits a nondegenerate $m\times m$ (hence
$\mathfrak{t}$)-invariant form. Now assume the length is greater than
one and both $A$ and $C$ are nonzero. By the induction hypothesis both
$A \otimes M$ and $C \otimes M$ admit nondegenerate
$\mathfrak{t}$-invariant forms. Using (\ref{chap3:lem3.10}) and
(\ref{chap3:lem3.7}), we conclude 
that $B\otimes M$ admits a nondegenerate $\mathfrak{t}$-invariant form
and that the sequence (\ref{eq3.12}) splits. This completes the proof.

\setcounter{prop}{12}
\begin{coro}\label{chap3:coro3.13}%%% 3.13
Let $M$ be an irreducible Verma module. Then the map $B\to B \otimes
M$ gives an isomorphism of categories $\mathscr{O}$ and $\mathscr{O}
\otimes M$.
\end{coro}

We complete this section by describing the main results on the
transfer of an invariant pairing from a pair of modules $A$, $B$ to
their completions $C(A)$, $C(B)$. Let $\mathfrak{a} \simeq sl (2,
\mathbb{C})$ be a subalgebra of $m$ and \textit{assume} the standard
basis $H,X,Y$ is given with $\sigma (H) =H$, $\sigma X = Y$, $\sigma Y
= X$. 

\begin{theorem}\label{chap3:thm3.14}
Let $A, B \in \mathscr{I}_m(\mathfrak{a})$ and $\varphi \in
Inv(A,B)$. Then there exists a unique $m$-invariant pairing
$C(\varphi)$ of $C(A)$ and $C(B)$ such that 
\begin{itemize}
\item[{\rm (i)}] $C(\varphi)$ is zero on $A \times C(B)$ and $C(A)
  \times B$.

\item[{\rm (ii)}] for $n \in \mathbb{N}$, $a \in C(A)^X_n$, $b \in
  C(B)^X_n$ we have 
\end{itemize}
$$
C(\varphi) (a,b) = \frac{1}{n! (n+1)!} \varphi(Y^{n+1} a, Y^{n+1}b).
$$\pageoriginale
\end{theorem}

Moreover, if $F$ and $F'$ are finite dimensional $m$-modules and
$\varphi_0$ is an invariant pairing of $F$ and $F'$, then $C(\varphi_0
\otimes\varphi) = \varphi_0 \otimes C(\varphi)$, both being pairings
on $C(F\otimes A) \times C(F' \otimes B)$ (cf. \ref{chap2:prop2.6}).

For a discussion of the choice of constants $\dfrac{1}{n!(n+1)!}$ in
(\ref{chap3:thm3.14}) and a result on their essential uniqueness the
reader may wish to consult \S \ref{sec8} of \cite{key15}.

\begin{remark}\label{chap3:rem3.15}
Let $m_0$ be a real form of $m$ with $m_0 \cap \mathfrak{h}$ a real
form of $\mathfrak{h}$. Let $\sigma$ denote an involutive conjugate
linear antiautomorphism of $U(m)$ which equals $(-1)\cdot$identity on
$m_0$. Using this $\sigma$, we may define invariant sesquilinear
pairings by replacing in (\ref{chap3:def3.1}) bilinear $\varphi$ by sesquilinear
$\varphi$. With the obvious modifications, (\ref{chap3:prop3.6}) and
(\ref{eq3.12}) remain 
true for sesquilinear pairings.

As usual a sesquilinear form $\varphi$ on $A \times A$ is called a
Hermitian form if $\varphi(a,a') = \overline{\varphi(a',a)}$.
\end{remark}
