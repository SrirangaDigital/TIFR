
\chapter{The character formula for $Z(\protect\underline{\lambda})$}\label{sec9}

In this\pageoriginale section we describe the isomorphisms between
images of Verma modules under $\tau$ and the principal series
modules. We retain the notation of \S\ \ref{sec6} and \S\ \ref{sec7}. 


\begin{theorem}\label{chap9:thm9.1}
Let $\u{\lambda} = (\lambda, \lambda') \in \mathfrak{L}$ and assume
$M(\lambda')$ is irreducible. Then $\tau M(\u{\lambda})$ and
$X(\u{\lambda})$ are isomorphic. 
\end{theorem}

A weaker result in a more general setting has been given in
\cite{key14}.

\begin{proof}
For $m \in \mathbb{N}$, put $\mu = \lambda' - m \delta_0$ and recall
from \S\ \ref{sec5} the functor $\psi = \psi^{(\lambda, \mu)}_{(\lambda,
 \lambda')}$. By Proposition (\ref{chap5:prop5.3}) and
(\ref{chap8:subsec8.5}), we have: 
\begin{equation*}
\psi M(\lambda, \mu) \simeq M(\lambda, \lambda'), \quad \psi
X(\lambda, \mu) \simeq X (\lambda, \lambda').\tag{$\ast$}\label{eqast1}
\end{equation*}
By (\ref{eqast1}) and the fact that $\tau$ commutes with $\psi$, it is
sufficient to prove (\ref{chap9:thm9.1}) for the parameter $(\lambda, \mu)$. Taking
$n>> 0$, and replacing  $(\lambda,\lambda')$ by $(\lambda, \mu)$ we
may assume:
\begin{equation*}
(Re \lambda')_\alpha << 0, \quad \forall \alpha \in P_0.
\tag{$\dagger$}
\end{equation*}

For any positive system $R$, set $\mathfrak{L}'(R) = \{\xi \in
\mathfrak{L'} \mid \re \xi\text{ is $R$-dominant}\}$. Let
$\mathscr{P}(\xi)$ be the proposition that $X(\xi)$ and $\tau M(\xi)$
are isomorphic. 
\end{proof} 

\setcounter{section}{9}
\setcounter{subsection}{1}
\subsection{}\label{chap9:subsec9.2}
Let $R$ be a positive system, $\xi,\mu$ elements of $\mathfrak{L}'(R)$
with $\xi - \mu$ integral. Then $\mathscr{P}(\xi)$ is true
$\Leftrightarrow \mathscr{P}(\mu)$ is true. 

\begin{proof}
Choose $R$-dominant integral $\sigma$, $\gamma \in \mathfrak{h}^*$
with $\xi - \mu = \sigma-\gamma$. Put $\zeta = \sigma + \mu = \xi +
\gamma$, $\psi_1 = \psi^{\xi + \gamma}_\xi$ and $\psi_2 =
\psi^{\sigma+ \mu}_\sigma$. Since $\tau$ commutes with $\psi_1$ and
$\psi_2$, we obtain by Proposition (\ref{chap5:prop5.3}) and
(\ref{chap8:subsec8.5}): 
$$
\mathscr{P}(\xi) \text{ is true } \Leftrightarrow \mathscr{P}(\zeta)
\text{ is true } \Leftrightarrow \mathscr{P} (\mu)\text{ is true. }
$$\pageoriginale
\end{proof}

\subsection{}\label{chap9:subsec9.3}
Let $R_0$ be a positive system for $\Delta_0$ and let $\beta$ be a
simple root in $R_0$ with $-\beta \in P_0$. Put $R'_0 = s_\beta R_0$,
$R= R_0 \times \{0\} \cup \{0\} \times (-P_0)$ and $R' = (s_\beta, l)
R$. If $\u{\xi} \in \mathfrak{L}'(R)$ and $\u{\mu} \in \mathfrak{L}'
(R')$ with $\u{\xi} - \u{\mu}$ integral, then $\mathscr{P}(\u{\xi})$
true $\Rightarrow \mathscr{P}(\u{\mu})$ is true. 


Write $\u{\xi} = (\xi, \xi')$ and $\u{\mu} = (\mu, \mu')$. By
(\ref{chap9:subsec9.2}) we may translate $\xi$ and $\mu$ and then
$\xi'$ and $\mu'$ so that we have: 

\subsection{}\label{chap9:subsec9.4}
\begin{itemize}
\item[{\rm (i)}] $\re \xi_\alpha >> 0$, \quad $\forall \alpha \in R_\circ \; 
 \backslash \;  \{ \beta\}$

\item[{\rm (ii)}] $\re \xi_{\beta} \leq 2$

\item[{\rm (iii)}] $\mu - \xi = 4 \delta_0$

\item[{\rm (iv)}] $\mu' = \xi' $

\item[{\rm (v)}] $\re \xi'_\alpha << -|\re \xi_\alpha|$, \quad
  $\forall \alpha \in P_0$. 
\end{itemize}

The proof of (\ref{chap9:subsec9.3}) is based on two lemmas which we now describe.

Let $F_0$ be the finite dimensional $\mathfrak{g}_0$-module with
highest weight $4\delta_0$ and let $F$ be the $\mathfrak{g}$-module
$F_0 \otimes \mathbb{C}$ where $\mathbb{C}$ is the trivial
$\mathfrak{g}_0$-module. Choose a $\mathfrak{b}$-module filtration of
$F$, $F = F_r \supset \ldots \supset F_0 = 0$, and let $\gamma_i \in
\mathfrak{h}^*_0$ be determined by isomorphisms $F_i/ F_{i-1} \simeq
\mathbb{C}_{(\gamma_i, 0)}$, $1 \leq i \leq r$. The $(\gamma_i , 0)$
are the weights of $F$ with multiplicity. This flag of
$\mathfrak{b}$-modules induces a $\mathfrak{g}$-module flag. 
\begin{equation*}
F\otimes M(\u{\xi}) = B = B_r \supset \ldots \supset B_0 = \{0\}
\text{ with } B_i/ B_{i-1} \simeq M (\u{\xi} + (\gamma_i , 0))
\tag{9.5}\label{eq9.5}
\end{equation*}
 
Note\pageoriginale that $B_1 \simeq M(\u{\mu})$. For any central
character $\chi$, let a subscript $\chi$ denote the generalized eigen
subspace. 

\setcounter{prop}{5}
\begin{lemma}\label{chap9:lem9.6}
Let $\chi$ denote the central character of $M(\u{\mu})$ and let
$m$ equal the dimension of the $F_0$ weight space for the weight
$s_\beta \mu - \xi$. Then there exists an exact sequence $0 \to M
(\u{\mu}) \to B_\chi \to L \to 0 $ where $L$ is isomorphic to a sum of
$m$ copies of $M(s_\beta, \mu, \mu')$. 
\end{lemma}

\begin{proof}
For $1 \leq i \leq r$, if $\gamma_i + \xi$ lies in the
$\mathfrak{w}_0$-orbit of $\mu$, then by (\ref{chap9:subsec9.4}) we
conclude that either 
$\gamma_i = \mu - \xi = 4 \delta_0$ or $\gamma_i = s_\beta \mu -
\xi$. In the first case $\gamma_i$ is the highest weight of $F_0$; and
so, $i =1$. This shows that (\ref{eq9.5}) induces a filtration on $B_\chi$ and
gives a short exact sequence as in (\ref{chap9:lem9.6}) with $L$ admitting a
filtration $L_m \supset \ldots \supset L_0 = 0$ with $L_i / L_{i-1}
\simeq M(s_\beta, \mu, \mu')$. However, $L$  is a free
$U(n^-)$-modules with cyclic highest weight space; and so, $L$ is a
sum of $m$ copies of $M(s_\beta, \mu, \mu')$. This proves (\ref{chap9:lem9.6}).
\end{proof}

We have an analogous result for principal series modules. Put $D =
F\otimes X(\u{\xi})$. 

\begin{lemma}\label{chap9:lem9.7}
Let notation be as in (\ref{chap9:lem9.6}). Then there exists a short
exact sequence, 
$0 \to X(\mu) \to D_\chi \to N \to 0$ where $N$ is isomorphic to a sum
of $m$ copies of $X(s_\beta \mu, \mu')$. Moreover, if $\nu = \u{\mu}
\mid_\mathfrak{t}$, then the $\mathfrak{t}$-module with extreme weight
$\nu$ occurs in $D_\chi$ with multiplicity one and is contained in the
image of $X(\u{\mu})$. 
\end{lemma}

\begin{proof}
Recall that $X(\u{\xi})$ is the space of $\mathfrak{t}$-finite vectors
in the algebraic dual of $\bar{M}(\u{\xi})$. Let $F^*$ be the
algebraic dual of $F$. Our filtration\pageoriginale of $F$ induces a
$\mathfrak{b}$-module dual filtration $F^* = F^*_r \supset \ldots
\supset F^*_0 = \{0\}$ with $F^*_i / F^*_{i-1} \simeq
\mathbb{C}_{(-\gamma_{r-i+1}, 0)}$. Now $0 \times \mathfrak{g}_0$ acts
by zero on $F$; and so, this filtration is also a
$\mathfrak{b}_Q$-module filtration. Inducing from $\mathfrak{b}_Q$ up
to $\mathfrak{g}$ we obtain the flag 
\begin{align*}
F^* \otimes \bar{M} (-\u{\xi}) &= C = C_r \supset \ldots \supset
C_0\\ 
&=\{0\} \text{ with } C_i / C_{i-l} \simeq \bar{M} (-\xi -\gamma_{r-i +
  1}, -\xi').  \tag{9.8}\label{eq9.8}
\end{align*}
By duality we obtain a flag $\{D_i\}_{0\leq i \leq r}$ on $D$, with
$D_i / D_{i-l} \simeq X(\xi + \gamma_i , \xi')$. Now arguing as in the
proof of (\ref{chap9:lem9.6}) but for the generalized eigen subspace
of $C$ dual to 
$\chi$ and passing by duality to $D_\chi$, we obtain the short exact
sequence in (\ref{chap9:lem9.7}). Then Frobenious reciprocity
(\ref{chap8:subsec8.3}) completes the 
proof of (\ref{chap9:lem9.7}). 
\end{proof}

\setcounter{prop}{8}
\begin{coro}\label{chap9:coro9.9}
{\bf to (\ref{chap9:lem9.6}).} The following sequence is exact.
$$
0 \to \tau M(\u{\mu}) \to \tau B_\chi \to \tau L \to 0. 
$$
\end{coro}

\begin{proof}
$B_\chi$ is an element of the $\mathfrak{t}$-semisimple category
  $\mathscr{O} \otimes M(\xi')$ (cf. (\ref{chap3:prop3.9})). 
\end{proof}

We now complete the proof of (\ref{chap9:subsec9.3}). Let $T$ be an
isomorphism $T: \tau 
M(\u{\xi}) \xrightarrow{\sim} X(\u{\xi})$ and let $S$ be the
isomorphism which is the restriction of $1 \otimes T$ to $(F\otimes
\tau M (\u{\xi}))_\chi$. Then $S: \tau B_\chi \xrightarrow{\sim}
D_\chi$. Put $\nu = \u{\mu} \mid_t$ then by (\ref{chap9:lem9.7}) and
(\ref{chap6:prop6.9}), $S$ 
injects the subspace $\tau M(\u{\mu})$ into $X(\u{\mu})$. But
$(s_\beta \mu, \mu') \in \mathfrak{L} (R)$ and by
(\ref{chap9:subsec9.2}, $\mathscr{P} 
(s_\beta \mu, \mu')$ is true. This implies that $L$ and $N$ are
isomorphic; and so, by (\ref{chap9:lem9.7}) and
(\ref{chap9:coro9.9}) the injection of $\tau M(\mu)$ 
into $X(\u{\mu})$ must be an isomorphism. This completes the proof of
(\ref{chap9:subsec9.3}). 

Next\pageoriginale we prove the theorem for one chamber.
\begin{equation*}
\text{If } \quad \u{\lambda} \in \mathfrak{L}' (-P), \quad \text{then}
\quad \mathscr{P} (\u{\lambda}) \quad \text{ is
  true}. \tag{9.10}\label{eq9.10} 
\end{equation*}

We begin with a preliminary result on minimal $\mathfrak{t}$-modules
in $(\mathfrak{g}, \mathfrak{t})$-modules. For any integral $\mu \in
\mathfrak{t}^*$, let $F_\mu$ be the irreducible finite dimensional
$\mathfrak{t}$-module with extreme weight $\mu$. 

\setcounter{prop}{10}
\begin{definition}\label{chap9:def9.11}
Let $\mu \in \mathfrak{t}^*$ be integral and let $A$ be a
$(\mathfrak{g}, \mathfrak{t})$-module. Then $F_\mu$ is called a weak
minimal $\mathfrak{t}$-type of $A$ if (i) there exists $T
\in\Hom_\mathfrak{t} (F_\mu, A)$, $T \neq 0$ (ii) for $\beta \in
P_\mathfrak{t}$, $\Hom_\mathfrak{t} (F_{\mu - \beta}, \mathfrak{p}
\cdot T (F_\mu)) = 0 $ ($\mathfrak{g} = \mathfrak{t} \oplus
\mathfrak{p}$ is the Cartan decomposition) and (iii) $\dim \Hom
(F_\mu, T(F_\mu) + \mathfrak{p} \cdot T (F_\mu)) = 1$. 
\end{definition}

\begin{theorem}\label{chap9:thm9.12}
Let $A$ be a $(\mathfrak{g}, \mathfrak{t})$-module with weak minimal
$\mathfrak{t}$-type $F_\mu$. Assume $\mu_\beta << 0$ for all $\beta
\in P_\mathfrak{t}$. Then there exists an element $\lambda \in
\mathfrak{h}^*$ with $\lambda \mid_\mathfrak{t} = \mu$ and a nonzero
$\mathfrak{g}$-module map $S$ with $S: \tau M(\lambda) \to A$. 
\end{theorem}

Both (\ref{chap9:def9.11}) and (\ref{chap9:thm9.12}) are formulations
in our notations of results from section six of \cite{key13}. 

We may assume $\lambda'$ satisfies $(\dagger)$. Then by (\ref{chap8:subsec8.3}),
$(\lambda, \lambda')\mid_\mathfrak{t}$ is a weak minimal
$\mathfrak{t}$-type of $X (\u{\lambda})$ and so by
(\ref{chap9:thm9.12}) there exists 
$\u{\nu} = (\mu, \mu')$ and a nonzero $\mathfrak{g}$-module map
$T:\tau M(\u{\gamma}) \to X(\u{\lambda})$. We claim $\u{\gamma} =
\u{\lambda}$. By (\ref{chap9:thm9.12}), $\nu + \nu' = \lambda + \lambda'$. Both
$\mathfrak{g}$-modules must have the same $Z(\mathfrak{g})$-character,
so $\u{\lambda}$ and $\u{\gamma}$ lie in the same
$\mathfrak{w}$-orbit. Write $(\nu, \nu') = (r\lambda,
s\lambda')$. Then $\lambda + \lambda' = \nu + \nu' = r \lambda + s
\lambda'$. But by $(\dagger)$, $\lambda + \lambda'$ is -$P_0$-dominant
while $r\lambda + s\lambda'$ is -$sP_0$-dominant. Therefore, $s = 1$
and $\mu'=\lambda'$. But then $\nu = \lambda$ and $\u{\lambda} =
\u{\gamma}$. Thus $T: \tau M (\u{\lambda}) \to X(\u{\lambda})$. By
assumption $M(\u{\lambda})$ is irreducible and so by (\ref{chap6:prop6.8}), $\tau
M(\u{\lambda})$\pageoriginale is irreducible. Therefore $T$ is an
injection and, by the multiplicity formulae (\ref{chap7:coro7.14}) and
(\ref{chap8:subsec8.3}), $T$ must 
be an isomorphism. This proves (\ref{eq9.10}). 

\setcounter{subsection}{12}
\subsection{}\label{chap9:subsec9.13}
$\mathscr{P}(\u{\lambda})$ is true for all $\u{\lambda} \in
\mathfrak{L}$ with $M(\lambda')$ irreducible. 

By $(\dagger)$, $\u{\lambda} \in \mathfrak{L}(S)$ where $S$ is a
positive root system of the form $S= R\times \{0\} \cup \{0\} \times
-P_0$, $R$ a positive system of $\Delta_0$. We proceed by induction on
$m = |S \cap P|$. If $m=0$ then $\mathscr{P}(\u{\lambda})$ is true by
(\ref{eq9.10}). If $m \neq 0$, then there exists $\beta \in R$ simple with
$\beta \in P_0$. Put $S' = (s_\beta, 1)S$. Then for $n \in
\mathbb{N}$, $n >> 0$, $\u{\lambda } + n \delta_{S'} \in \mathfrak{L}
(S')$ and $|S' \cap P| = |S \cap P | -1$. So by induction $\mathscr{P}
(\u{\lambda} + n \delta_{S'})$ is true. By (\ref{chap9:subsec9.3}), $\mathscr{P}
(\u{\lambda})$ is true. This proves (\ref{chap9:subsec9.13}) and
completes the proof of Theorem \ref{chap9:thm9.1}.

Using (\ref{chap9:thm9.1}) we now describe a sharper vanishing theorem than\break
Corollary \ref{chap7:coro7.13}.

\setcounter{prop}{13}
\begin{prop}\label{chap9:prop9.14}
Let $\u{\lambda} \in \mathfrak{L}$ and assume $M(\lambda')$ is
irreducible. Assume $\u{\lambda}$ does not satisfy
(\ref{chap6:subsec6.5}). Then $\tau L(\u{\lambda}) = 0$.  
\end{prop}

\begin{proof}
Let $\Delta^0$ be the sub root system of $\Delta_0$ which stabilizes
$\lambda'$. Let $\mathfrak{w}^0$ be the corresponding stabilizer of
$\lambda'$ in $\mathfrak{w}_0$ and put $P^0 = P_0 \cap \Delta^0
$. $P^0$ is a positive system for $\Delta^0$. Let $\mu$ be the unique
element in the $\mathfrak{w}^0$ orbit of $\lambda$ which is
-$P^0$-dominant. Then, putting $\u{\mu} = (\mu, \lambda')$, we have
the inclusion $M(\u{\mu}) \hookrightarrow M(\u{\lambda})$. These
modules are elements of the $\mathfrak{t}$-semisimple category
$\mathscr{O} \otimes M(\lambda')$. By (\ref{chap9:thm9.1}) and
(\ref{chap8:prop8.7}), the inclusion 
$M(\u{\mu}) \hookrightarrow M(\u{\lambda})$ induces an isomorphism
$\tau M(\u{\mu}) \xrightarrow{\sim} \tau M (\u{\lambda})$. By the exactness of $\tau$ on $\mathscr{O} 
\otimes M(\lambda')$, $\tau (M(\u{\lambda})/M(\u{\mu}))'
=0$. $L(\u{\lambda})$ occurs in $M(\u{\lambda})/ M (\u{\mu})$; and so
by exactness of $\tau$, $\tau L(\u{\lambda}) = 0$. 
\end{proof}

We\pageoriginale complete this section with a character formula for
$Z(\u{\lambda})$. For $\u{\nu} \in \mathfrak{L}$, let $E(\u{\nu})$ be
the distribution character of the principal series representation of
$G$ whose $K$-finite vectors are $\mathfrak{g}$-isomorphic to
$X(\u{\nu})$. Similarly, for $\u{\lambda}$ with $M(\lambda')$
irreducible, let $\Theta(\u{\lambda})$ denote the distribution
character of any admissible representation of $G$ with $K$-finite
vectors $\mathfrak{g}$-isomorphic to $\tau L(\u{\lambda})$. For any
module $A$ in $\mathscr{O}$, let $chA$ denote the formal character of
$A$ (cf. \cite{key24}). 

\begin{prop}\label{chap9:prop9.15}
Fix integers $m(s\lambda)$, $s \in \mathfrak{w}_\lambda$, such that
$ch \; L(\lambda) = \sum\limits_{s\lambda \in \mathfrak{w}_\lambda
  \cdot \lambda} m(s\lambda) ch M(s\lambda)$. Then 
$$
\Theta (\u{\lambda}) = \sum\limits_{s\lambda \in \mathfrak{w}_\lambda
  \cdot \lambda} m (s\lambda) \; E(s\lambda, \lambda').
$$
\end{prop}

\begin{proof}
By Proposition \ref{chap3:prop3.9}, $\tau$ is exact on $\mathscr{O} \otimes
M(\lambda')$. This exactness of $\tau$ implies that $\tau$ induces a
linear map of Grothendieck groups. So by (\ref{chap9:thm9.1}), the formula for $ch
L(\lambda)$ becomes under the map $\tau (\cdot \otimes M(\lambda'))$
the formula (\ref{chap9:prop9.15}). 
\end{proof}

An alternate and more general definition of minimal
$\mathfrak{t}$-type has been given by Vogan \cite{key34}. The reader
may wish to compare (\ref{chap9:thm9.12}) with Theorem
\ref{chap3:thm3.14} in \cite{key34}.  
