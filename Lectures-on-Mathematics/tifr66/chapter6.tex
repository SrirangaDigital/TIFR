
\chapter{Construction of irreducible admissible $(\mathfrak{g},
  \mathfrak{t})$-modules}\label{sec6} 

In\pageoriginale this section we begin the study of the
representations of a connected complex semisimple Lie group $G$. The
main result is the construction of irreducible $(\mathfrak{g},
\mathfrak{t})$-modules. If $\mathfrak{g}_0$ is the Lie algebra of $G$
then $\mathfrak{g}_0$ is a complex Lie algebra. We fix a compact real
form $\mathfrak{t}_0$ of $\mathfrak{g}_0$ and let $\mathfrak{g}_0 =
\mathfrak{t}_0 \oplus \mathfrak{p}_0$ be a Cartan decomposition. At
this point, it will be more useful to think of $\mathfrak{g}_0$ as a
real Lie algebra admitting a map $J$ defined by multiplication by
$\sqrt{-1}$. We have $J(x) = \sqrt{-1} x$, $x \in \mathfrak{g}_0$, and
$J^2 = - 1$. Now we fix a CSA $\mathfrak{t}_0$ of $\mathfrak{t}_0$ and
put $\mathfrak{a}_0 = J\mathfrak{t}_0$. Then $\mathfrak{h}_0 =
\mathfrak{t}_0 \oplus \mathfrak{a}_0$ is a CSA of $\mathfrak{g}_0$. 

Considering $\mathfrak{g}_0$ as a real Lie algebra, it is especially
useful to have the following complexification. Let $c$ denote the
conjugation of $\mathfrak{g}_0$ with respect to the real form
$\mathfrak{t}_0$. Put $\mathfrak{g} = \mathfrak{g}_0 \times
\mathfrak{g}_0$ and define $i : \mathfrak{g}_0 \to \mathfrak{g}$ by $i
(x) = (x, x^c)$ for $x \in \mathfrak{g}_0$. Clearly $i$ is an
injection and $(\mathfrak{g}, i)$ is a complexification of
$\mathfrak{g}_0$. Letting $\mathfrak{t}$ equal the diagonal in
$\mathfrak{g}$, $\mathfrak{t}$ is naturally the complexification of
$\mathfrak{t}_0$ and, abstractly, $\mathfrak{t}$ is isomorphic to the
complex Lie algebra $\mathfrak{g}_0$. Similarly, putting $\mathfrak{p}
= \{(x, -x) \mid x \in \mathfrak{g}_0 \}$, $\mathfrak{p}$ is the
complexification of $\mathfrak{p}_0$, $\mathfrak{p}_0 =
J\mathfrak{t}_0$ and $\mathfrak{g} \oplus \mathfrak{p}$. Let $\theta$
denote the Cartan involution giving this decomposition of
$\mathfrak{g}$. Next we put $\mathfrak{h} = \mathfrak{h}_0 \times
\mathfrak{h}_0$, $\mathfrak{t} = \mathfrak{t}_0 \times
\mathfrak{t}_0$, and $\mathfrak{a} = \mathfrak{a}_0 \times
\mathfrak{a}_0$. Then $\mathfrak{h}$ and $\mathfrak{t}$ are CSAs of
$\mathfrak{g}$ and $\mathfrak{t}$ respectively. The algebra
$\mathfrak{a}$ is maximal abelian in $\mathfrak{p}$ and $\mathfrak{h}
= \mathfrak{t} \oplus \mathfrak{a}$. Note also that for any real Lie
algebra denoted by a lower case German script letter, deletion of a
subscripted zero gives the complexification of the algebra in
$\mathfrak{g}$. 

\setcounter{section}{6}
\setcounter{subsection}{0}
\subsection{Conventions for roots and Weyl groups}\label{chap6:subsec6.1}

We\pageoriginale identify $\mathfrak{h}^*$ with $\mathfrak{h}^*_0 \times
\mathfrak{h}^*_0$ by the formula:
$$
(\lambda, \lambda') (H, H') = \lambda(H) + \lambda'(H'), \quad \forall
(H, H') \in \mathfrak{h}.
$$

Let $\Delta$, $\Delta_0$ and $\Delta_\mathfrak{t}$ denote the set of
roots of $(\mathfrak{g}, \mathfrak{h})$, $(\mathfrak{g}_0,
\mathfrak{h}_0)$ and $(\mathfrak{t}, \mathfrak{t})$ respectively. One
checks that $\Delta = \Delta_0 \times \{0\} \cup \{0\} \times
\Delta_0$. Let $P_0$ be a positive system for $\Delta_0$. Then $P=P_0
\times \{0\} \cup \{0\} \times P_0$ is a $\theta$-stable positive
system for $\Delta$. Also if we put $P_\mathfrak{t} = \{(\alpha, 0)
\mid_\mathfrak{t} \mid \alpha \in P_0\}$, then $P_\mathfrak{t}$ is a
positive system for $\Delta_\mathfrak{t}$. For $\nu \in
\mathfrak{h}^*_0$ let $\Delta_\nu = \{\alpha \in \Delta \mid
\nu_\alpha \in \mathbb{Z} \}$. Then $\Delta_\nu$ is a root system and
we let $P_\nu = P_0 \cap \Delta_\nu$. 


Let $\mathfrak{w}$, $\mathfrak{w}_0$ and $\mathfrak{w}_\mathfrak{t}$
be the Weyl groups of $\Delta$, $\Delta_0$ and $\Delta_\mathfrak{t}$
respectively. Then $\mathfrak{w} = \mathfrak{w}_0 \times
\mathfrak{w}_0$ and $\mathfrak{w}_\mathfrak{t}$ is the diagonal in
$\mathfrak{w}$.

\subsection{Conventions for highest weight
  modules}\label{chap6:subsec6.2}
Let $n^{\pm}$, $n^\pm_0$, $\mathfrak{b}$ and $\mathfrak{b}_0$ denote
the nilpotent and Borel subalgebras of $\mathfrak{g}$ and
$\mathfrak{g}_0$ respectively associated with the positive systems $P$
and $P_0$. We have: $n^{\pm} = n^{\pm}_0 \times n^\pm_0$,
$\mathfrak{b} = \mathfrak{b}_0 \times \mathfrak{b}_0$. Moreover, if we
put $n^\pm_\mathfrak{t} = n^\pm \cap \mathfrak{t}$ and
$\mathfrak{b}_\mathfrak{t} = \mathfrak{b} \cap \mathfrak{t}$, then
$n^\pm_\mathfrak{t}$ and $\mathfrak{b}_\mathfrak{t}$ are the nilpotent
and Borel subalgebras of $\mathfrak{t}$ associated with
$P_\mathfrak{t}$. Let $\delta$, $\delta_0$ and
$\mathfrak{\delta}_\mathfrak{t}$ denote half the sum of the elements
of $P$, $P_0$ and $P_\mathfrak{t}$ respectively. If $\mu \in
\mathfrak{h}^*$ (resp. $\mathfrak{h}^*_0$, $\mathfrak{t}^*$) then
$M(\mu)$ denotes the Verma module with $P$ (resp. $P_0$,
$P_\mathfrak{t}$)-highest weight $\mu-\delta$ (resp. $\mu-\delta_0$,
$\mu - \delta_\mathfrak{t}$) and $L(\mu)$ denotes the unique
irreducible quotient of $M(\mu)$. 

\subsection{Convention on product actions}\label{chap6:subsec6.3}
Let\pageoriginale $\mathfrak{L}_0$ be a Lie algebra, $A$ and $B$ two
$\mathfrak{L}_0$-modules. Put $\mathfrak{L} = \mathfrak{L}_0 \times
\mathfrak{L}_0$. Then $A \otimes B$ becomes an $\mathfrak{L}$-module
under the action: $(X,Y) (a \otimes b) = X a \otimes b + a \otimes Y
\cdot b$, for $X, Y \in \mathfrak{L}_0$, $a \in A$, $b \in
B$. Furthermore, $A$ and $B$ are both irreducible if and only if $A
\otimes B$ is an irreducible $\mathfrak{L}$-module.

\setcounter{prop}{3}
\begin{lemma}\label{chap6:lem6.4}
For $\lambda, \lambda' \in \mathfrak{h}^*_0$, there exist natural
isomorphisms
$$
M(\lambda, \lambda') \xrightarrow{\sim} M(\lambda) \otimes
M(\lambda'), \quad L(\lambda, \lambda') \xrightarrow{\sim} L(\lambda)
\otimes L(\lambda').
$$
\end{lemma}

\begin{proof}
The first isomorphism is induced by the natural isomorphism
$U(\mathfrak{g}) \xrightarrow{\sim} U(\mathfrak{g}_0) \otimes
U(\mathfrak{g}_0)$  while the second follows from the first and
(\ref{chap6:subsec6.3}). 
\end{proof}

We next define a set of representations of $\mathfrak{g}$ which will
be the center of our study. Put $\mathfrak{L} =\{(\lambda, \lambda')
\in \mathfrak{h}^* \mid \lambda + \lambda' \text{ is }
\Delta_0-\text{integral }\}$. Let $\mathfrak{L}'$ denote the regular
elements of $\mathfrak{L}$  is stable under the action of
$\mathfrak{w}_\mathfrak{t}$ and we let $[\lambda, \lambda']$ denote
the $\mathfrak{w}_\mathfrak{t}$-orbit of $(\lambda, \lambda') \in
\mathfrak{L}$. Let $\mathfrak{L}/\mathfrak{w}_\mathfrak{t}$  denote
the set of $\mathfrak{w}_\mathfrak{t}$-orbits. 

\setcounter{subsection}{4}
\subsection{Convention for $[\lambda,
    \lambda']$}\label{chap6:subsec6.5}
Each orbit $[\lambda, \lambda']$ in $\mathfrak{L}$ contains an element
(possibly several) $(\mu, \mu')$ satisfying the following two
conditions. Let $\alpha \in P_0$.
\begin{itemize}
\item[{\rm (i)}] $\mu'$ is -$P_\mu$-dominant

\item[{\rm (ii)}] if $\mu'_0 = 0$ then $\mu_\alpha \in - \mathbb{N}$.
\end{itemize}

\textit{We assume for convenience} that for any orbit
$[\lambda,\lambda']$, $(\lambda, \lambda')$ itself satisfies
(\ref{chap6:subsec6.5}). Note that (i) is equivalent to saying $M(\lambda')$ is
irreducible.

\setcounter{prop}{5}
\begin{definition}\label{chap6:def6.6}
For\pageoriginale $\u{\lambda} \in \mathfrak{L}$ satisfying
(\ref{chap6:subsec6.5}) and 
$\tau$ the lattice functor defined on the category $\mathscr{I}_
\mathfrak{g}( \mathfrak{t})$, define $Z(\u{\lambda}) = \tau (L(\u{\lambda}))$.
\end{definition}

Lemma \ref{chap3:lem3.10} shows, among other things, that
$L(\u{\lambda})$ is an 
object of $ \mathscr{I}_ \mathfrak{g}( \mathfrak{t})$; and so,
$Z(\u{\lambda})$ is well defined.

\begin{definition}\label{chap6:def6.7}
Let $A$ be a module and $A = A_d \supset A_{d-1} \supset \ldots
\supset A_1 \supset A_0 =0$ be a flag of submodules. The flag of
submodules $\{A_i\}$ is called a Jordan-H\"older series for $A$ if
$A_i /A_{i-1}$ is irreducible, $1\leq i \leq d$. The flag is said to
be reducible to a Jordan-H\"older series for $A$ if $A_i/A_{i-1}$ is
either zero or irreducible, $1 \leq i \leq n$. 
\end{definition}

\begin{prop}\label{chap6:prop6.8}
Let $\u{\lambda} = (\lambda, \lambda') \in  \mathfrak{L}$ and assume
$M(\lambda')$ is irreducible. 
\begin{itemize}
\item[{\rm (i)}] If $\lambda'$ is regular then $\tau$ maps a
  Jordan-H\"older series for $M(\u{\lambda})$ to a Jordan-H\"older
  series for $M(\u{\lambda})$.

\item[{\rm (ii)}] In general, $\tau$ maps a Jordan-H\"older series for
  $M(\u{\lambda})$ to a flag of submodules reducible to a
  Jordan-H\"older series for $\tau M(\u{\lambda})$.
\end{itemize}

Moreover, if $M_r \supset \ldots \supset M_0 =0$ is a Jordan-H\"older
series for $M(\lambda)$, then
\begin{equation*}
\tau (M_i / M_{i-1}) \simeq \tau M_i / \tau M_{i-1}. \tag*{$(\ast)$}\label{eqast}
\end{equation*}
\end{prop}

\begin{proof}
The category $\mathscr{O} \otimes M(\lambda')$ is
$\mathfrak{t}$-semisimple by Proposition \ref{chap3:prop3.9}; and so, $\tau$ is exact
on this subcategory of $\mathscr{I}_\mathfrak{t}(\mathfrak{t})$. This
proves {\ref{eqast}}. Using the translation functors (cf. Proposition
\ref{chap5:prop5.3}) 
it is sufficient to prove (i) assuming 
\begin{equation*}
\lambda'_\alpha << 0 \quad \text{for all} \quad \alpha \in P_0. 
 \tag{$\dagger$}\label{eqdag}
\end{equation*}\pageoriginale
\end{proof}

For $1 \leq i \leq r$, put $N_i = M_i/ M_{i-1}$. We need only show
that $\tau (N_i)$ is irreducible $1 \leq i \leq r$. We begin with a
result from \cite{key15}, regarding the image of a Verma module under
the functor $\tau$. 

\begin{prop}\label{chap6:prop6.9}
Let $\mu = (\lambda, \lambda') \mid_\mathfrak{t}$ and assume
(\ref{eqdag}) holds. Then the irreducible $\mathfrak{t}$-module with
extreme weight $\mu$ occurs in $\tau M(\lambda, \lambda')$ with
multiplicity one and is $\mathfrak{g}$ cyclic. 
\end{prop}


This result is a formulation of Proposition \ref{chap10:def10.3}
\cite{key15} in our 
notation. Note that (\ref{eqdag}) a decomposition $N_r = L \oplus B$
with $B$ an irreducible $\mathfrak{t}$-Verma module with highest
weight $\mu - 2 \delta_\mathfrak{t}$ ($\mu$ as in (\ref{chap6:prop6.9})). $\tau B$ is
an irreducible $\mathfrak{t}$-module with extreme weight $\mu$; and
so, from (\ref{chap6:prop6.9}) and the exactness of $\tau$ on $\mathscr{O} \otimes
M(\lambda')$, 

\setcounter{subsection}{9}
\subsection{}\label{chap6:subsec6.10}
The irreducible $\mathfrak{t}$-module with extreme weight $\mu$ occurs
in $\tau N_r$ with multiplicity one and is $\mathfrak{g}$-cyclic.

We now show that $\tau N_r$ is irreducible. By (\ref{chap3:prop3.5})
and (\ref{chap4:coro4.12}), $\tau 
N_r$ admits  a nondegenerate $\mathfrak{g}$-invariant form. If $L
\subset \tau N_r$ is a proper submodule then by (\ref{chap6:subsec6.10}), $L^\perp$
contains the irreducible $\mathfrak{t}$-module with extreme weight
$\mu$. This subspace is $\mathfrak{g}$-cyclic. So $L^\perp = \tau
N_r$. But the form is nondegenerate; and thus, $L=0$. This proves
$\tau N_r$ is irreducible. Now we proceed by induction on
$r-i$. Assume $l \leq i < r$. There exists $s \in \mathfrak{w}$ with
$N_i \simeq L(s\lambda, \lambda')$ and such\pageoriginale that
$M(s^\lambda, \lambda')$ has a Jordan-H\"older series of length
strictly less than $r$. By induction (\ref{chap6:prop6.8}) is true for any
Jordan-H\"older series for $M(s\lambda, \lambda')$. This implies that
$\tau N_i$ is irreducible and completes the proof of (\ref{chap6:prop6.8}). 

\setcounter{prop}{10}
\begin{coro}\label{chap6:coro6.11}
{\bf to (\ref{chap6:prop6.8}).} Let $\u{\lambda} \in \mathfrak{L}$
satisfy (\ref{chap6:subsec6.5}). Then 
$Z(\u{\lambda})$ is an irreducible $(\mathfrak{g},
\mathfrak{t})$-module if $\lambda'$ is regular and is irreducible or
zero in general.
\end{coro}
