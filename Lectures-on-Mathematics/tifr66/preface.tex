\thispagestyle{empty}

\begin{center}
{\Large\bf Lectures on}\\[5pt]
{\Large\bf Representations Of Complex Semi-Simple}\\[5pt]
{\Large\bf Lie Groups}
\vskip 1cm

\vfill


{\bf By}
\medskip

{\large\bf Thomas J. Enright}
\vfill

{\bf Tata Institute of Fundamental Research}
\medskip

{\bf Bombay}
\medskip

{\bf 1981}
\end{center}
\eject

\thispagestyle{empty}
\begin{center}
{\Large\bf Lectures on}\\[5pt]
{\Large\bf Representations Of Complex Semi-Simple}\\[5pt]
{\Large\bf Lie Groups}
\vskip 1cm

\vfill


{\bf By}
\medskip

{\large\bf Thomas J. Enright}
\vfill

{\bf Notes by}
\medskip

{\large\bf Vyjayanthi Sundar}
\vfill

Published for the 
\medskip

{\bf Tata Institute of Fundamental Research}
\medskip

{\bf Bombay}
\bigskip

{\bf Springer--Verlag}
\medskip

Berlin Heidelberg New York
\medskip


{\bf 1981}
\end{center}
\eject

\thispagestyle{empty}
\begin{center}
{\bf Author}\\
\bigskip

{\large\bf Thomas J. Enright}
\medskip

Department of Mathematics
\medskip

University of California, San Diego
\medskip

LA JOLLA, California 92093
\medskip

U.S.A.
\vfill

{\bf\copyright \quad  Tata Institute of Fundamental Research, 1981}
\vfill

\noindent\rule{\textwidth}{1pt}
\smallskip

ISBN 3-540-10829-7 Springer-Verlag, Berlin, Heidelberg. New York

ISBN 0-387-10829-7 Springer-Verlag, New York. Heidelberg. Berlin
\smallskip

\noindent\rule{\textwidth}{1pt}
\vfill

\parbox{0.7\textwidth}{No part of this book may be reproduced in any form
by print microfilm or any other means without
written permission from the Tata Institute of
Fundamental Research, Colaba, Bombay 400 005}
\vfill

\parbox{0.7\textwidth}{Printed by N. S. Ray at The Book Centre Limited,
Sion East, Bombay 400 022 and published by H. Goetze,
Springer-Verlag. Heidelberg, West Germany}
\vfill

{\bf Printed In India} 
\end{center}


\chapter{Preface}


These notes are the slightly revised lecture notes from lectures given
at the Tata Institute during Winter 1980. The purpose of the lectures
was to describe a factorial correspondence between the theory of
admissible representations for a complex semisimple Lie group and the
theory of highest weight modules for a semisimple Lie algebra. A
detailed description of the main results of this correspondence is
given in section one. 

A first draft of these notes was prepared by Vyjayanthi Sunder. I am
grateful to her for her careful work. It is also a pleasure to thank
my colleagues at the Tata Institute for their hospitality to my wife
and me during our stay at the institute. Finally, I wish to
acknowledge the support of the National Science Foundation and the
Alfred P. Sloan Foundation. 
