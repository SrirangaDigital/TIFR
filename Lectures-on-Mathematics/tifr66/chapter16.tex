
\chapter[Connections with the derived functors...]{Connections with
  the derived functors introduced by 
  Zuckerman}\label{sec16}

We\pageoriginale begin this section with a description of certain
right derived functors introduced by Zuckerman \cite{key40}. Following
this we consider connections between these derived functors and the
lattice functor $\tau$. 

Let $\mathfrak{g}$ be a finite dimensional Lie algebra over
$\mathbb{C}$ and let $\mathfrak{t}$ and $\mathfrak{t}$ be subalgebras
of $\mathfrak{g}$ with $\mathfrak{g} \supset \mathfrak{t} \supset
t$. Also assume both $\mathfrak{t}$ and $t$ are reductive in
$\mathfrak{g}$. For any Lie algebras $\mathfrak{a}$, $\mathfrak{b}$
with $\mathfrak{a} \subset \mathfrak{b}$, let $\mathcal{C}
(\mathfrak{b}, \mathfrak{a})$ denote the category of
$\mathfrak{b}$-modules which are $\mathfrak{U}(\mathfrak{a})$-locally
finite and semisimple as $\mathfrak{a}$-modules. Let $S$ be the
additive covariant functor defined on  $\mathcal{C} (\mathfrak{g},
\mathfrak{t})$ by letting $S(X)$ be the maximal subspace of
$U(\mathfrak{t})$-locally finite vectors in $X$, $X \in \mathcal{C}
(\mathfrak{g}, \mathfrak{t})$. The categories
$\mathcal{C}(\mathfrak{t},t)$ and $\mathcal{C}(\mathfrak{g},
\mathfrak{t})$ admit sufficiently many injective objects to insure
that any object has an injective resolution. We may thus define the
right derived functors of $S$. Let $X \in \mathcal{C}(\mathfrak{t},
t)$ and let $0 \to X \to X_0 \to X_1 \to \ldots$ be an injective
resolution of $X$. Then $0 \to SX_0 \to SX_1 \to \ldots$ is a complex
and we define $R^iS(X)$ to be the $i^{\rm th} $ cohomology group of
this complex. If $X \in \mathcal{C} (\mathfrak{g}, \mathfrak{t})$ then
by taking an injective resolution in $\mathcal{C}(\mathfrak{g},
\mathfrak{t})$ and applying $S$, we obtain another set of right
derived functors which we denote by $R^i_\mathfrak{g}S$. From Lemma
3.1 \cite{key19}, we find that each injective object in $\mathcal{C}
(\mathfrak{g}, \mathfrak{t})$ remains injective when considered as an
object of $\mathcal{C}(\mathfrak{t}, \mathfrak{t})$. From this we have 

\begin{lemma}\label{chap16:lem16.1}
For $X \in \mathcal{C} (\mathfrak{g}, \mathfrak{t})$, let
$X_\mathfrak{t}$ denote the $\mathfrak{t}$-module underlying
$X$. Then, for $i \in \mathbb{N}$, the $\mathfrak{t}$-module
underlying $R^i_\mathfrak{g} S(X)$ is isomorphic to $R^i
S(X_\mathfrak{t})$.  This\pageoriginale lemma shows that the functors
$R^iS$ may be employed to determine the $\mathfrak{t}$-module
structure of the $\mathfrak{g}$-modules $R^i_\mathfrak{g} S(X)$, $X
\in \mathcal{C} (\mathfrak{g}, \mathfrak{t})$. The next step is then
the determination of $R^i S(M)$ for a Verma $\mathfrak{t}$-module
$M$. Assume $t$ is a CSA for $\mathfrak{t}$ and $P_\mathfrak{t}$ is a
positive system of roots. Let $\delta_\mathfrak{t}$ equal half the sum
of positive roots and let $\mathfrak{w}_\mathfrak{t}$ denote the Weyl
group. For $\mu \in \mathfrak{t}^*$, let $M(\mu)$ be the Verma
$\mathfrak{t}$-module with highest weight $\mu - \delta_\mathfrak{t}$
and if $\mu$ is integral, let $F_\mu$ be the irreducible finite
dimensional $\mathfrak{t}$-module with extreme weight $\mu$. Put $d =
\dfrac{1}{2} \dim \mathfrak{t}/t = \card P_\mathfrak{t}$. 
\end{lemma}

\begin{prop}\label{chap16:prop16.2}
Let $\mu \in t^*$ be antidominant and let $s \in
\mathfrak{w}_\mathfrak{t}$. Then 
$$
R^i S (M(s\mu)) = 
\begin{cases}
F_{\mu + \delta_\mathfrak{t}} & \text{ if $\mu$ is regular and }\\
&  \qquad i = d + \ell (s)\\
0 & \text{ otherwise}
\end{cases}
$$
An elementary proof of (\ref{chap16:prop16.2}) is given in \cite{key19}
(cf. Proposition \ref{chap6:subsec6.3} and Theorem \ref{chap4:def4.3}
in \cite{key19}) where it is also 
shown that (\ref{chap16:prop16.2}) is equivalent to the Bott-Borel-Weil Theorem. 
\end{prop}

We now specialize to the setting of these notes. Let $\mathfrak{g}$,
$\mathfrak{t}$ and $t$ be as in \S\ \ref{sec6}. Using (\ref{chap16:lem16.1})
and (\ref{chap16:prop16.2}), the 
arguments in \cite{key15} as well as those in \S\ \ref{sec5} through \S\ \ref{sec10} in
these notes can be carried out when the lattice functor $\tau$ is
replaced by the ``middle dimension'' derived functor $R^d_\mathfrak{g}
S$, $d = \dfrac{1}{2}  \dim \mathfrak{t}/ t$. For example, recalling
the principal series modules, the precise analogue of Theorem \ref{chap9:thm9.1} is: 

\begin{theorem}\label{chap16:thm16.3}
Let\pageoriginale $\u{\lambda} = (\lambda, \lambda') \in \mathfrak{L}$
and assume $M(\lambda')$ is irreducible. Then  
$$
R^i_\mathfrak{g} S (M(\u{\lambda})) = 
\begin{cases}
X(\u{\lambda}) & \text{ if } i = d \\
0 & \text{ if } i \neq d. 
\end{cases}
$$
Assume $M(\lambda')$ is irreducible. The category $\mathscr{O} \otimes
M(\lambda')$ is $\mathfrak{t}$-semisimple by (\ref{chap3:prop3.9});
and so, by (\ref{chap16:lem16.1}) 
each $R^i_\mathfrak{g} S$ is an exact functor on $\mathscr{O} \otimes
M(\lambda')$. This fact, (\ref{chap16:thm16.3}) and the arguments of section ten
combine to prove:
\end{theorem}


\begin{theorem}\label{chap16:thm16.4}
\begin{itemize}
\item[{\rm (a)}] For $i \in \mathbb{N}$, $i \neq d$, the functor
  $R^i_\mathfrak{g} S$ is zero on $\mathscr{O} \otimes M(\lambda')$. 

\item[{\rm (b)}] $R^d_\mathfrak{g} S$ is an exact functor on
  $\mathscr{O} \otimes M(\lambda')$. 

\item[{\rm (c)}] For $\u{\lambda} \in \mathfrak{L}$ satisfying (\ref{chap6:subsec6.5}), 
\end{itemize}
$$
R^d_\mathfrak{g} S (L(\u{\lambda})) \simeq Z(\u{\lambda}) = \tau
L(\u{\lambda}). 
$$
\end{theorem}


From these results one might guess that the results of section
thirteen also remain true with $\tau$ replaced by
$R^d_\mathfrak{g}S$. However, if true a different proof must be
supplied. A related question is whether or not $\tau$ and
$R^d_\mathfrak{g}S$ are naturally equivalent on some special category
of $\mathfrak{g}$-modules. The results of these notes suggest that the
category should contain $\mathscr{O} \otimes M(\lambda')$ for
$M(\lambda')$ irreducible. 

Lastly we give a rank two example to show that the category referred
to above cannot be too large. For this example, let $\mathfrak{g}=
\mathfrak{t} = s\ell (3, \mathbb{C})$. Fix $\mu$ dominant integral and
regular and let $\alpha$ and $\beta$ be the simple roots of the
positive system $P_\mathfrak{t}$. Put $M= M(s_\alpha \mu)$, $L =
L(s_\alpha \mu)$, $M'  = M(s_\alpha s_\beta s_\alpha \mu)$ and note
that $M'$ is an irreducible Verma module. 

\begin{lemma}\label{chap16:lem16.5}
There\pageoriginale  exists a nontrivial extension of $M'$ by $M$. 
\end{lemma}

\begin{proof}
Let $\Ext^*$ denote the right derived functors of $\Hom$ on the
category $\mathscr{O}$. Consider the short exact sequence $0 \to J \to
M \to L  \to 0$ and apply $\Ext^*(M',\cdot)$. We obtain the long exact
sequence. 
$$
\cdots \to \Hom (M', L) \to \Ext^1 (M', J) \to \Ext^1 (M', M)  \to
\cdots .
$$
Clearly $\Hom (M', L) = 0$; and so, to prove the lemma it is
sufficient to check that $\Ext^1(M', J) \neq 0$. Now consider the
short exact sequence $0 \to M' \to J \to B \to 0$ where $B =
L(s_\alpha s_\beta \mu) \oplus L(s_\beta s_\alpha \mu)$. Applying
$\Ext^*(M', \cdot)$ to this short exact sequence we obtain:
$$
\cdots \to \Ext^1(M', J) \to \Ext^1 (M', B) \to \Ext^2(M', M') \to
\cdots . 
$$

However, Delorme has shown that $\Ext^j (M(s\mu), L(r\mu)) =0$ if $j >
\ell (s) - \ell (r)$. Therefore, $\Ext^2 (M', M' ) = 0$; and so, to
see that $\Ext^1 (M',\break J) \neq 0$ it is sufficient to check $\Ext^1(M',
B) \neq 0$. But the algebraic dual of $M(s_\alpha s_\beta \mu) \oplus
L(s_\beta s_\alpha \mu)$ gives a nonzero extension of $M'$ by $B$
(cf. section three). This shows $\Ext^1(M', B) \neq 0$. So, as noted
above, $\Ext^1(M', M) \neq 0$ and this completes the proof.  
 \end{proof}

\begin{prop}\label{chap16:prop16.6}
Let $E$ be ta nontrivial extension of $M'$ by $M$ as above. Then 
$$
\tau E = 0 \quad R^i S(E) = 
\begin{cases}
F_{\mu - \delta} & \text{ if } i = 3 \text{ or } 5\\
0 & \text{ otherwise}. 
\end{cases}
$$
\end{prop}

\begin{proof}
Since\pageoriginale $E$ is a nontrivial extension, the dimensions of
the spaces of $n_\mathfrak{t}$-invariants of weight $s_\alpha s_\beta
s_\alpha \mu -\delta$ and $s_\alpha \mu - \delta$ are both one. Then,
by Proposition 4.13 \cite{key15}, $\tau E = 0$. 
\end{proof}

From the short exact sequence for $E$ we obtain the long exact
sequence 
$$
\cdots  \to R^i S(M) \to R^i S(E) \to R^i S(M') \to R^{i+1} S(M) \to
\cdots . 
$$
Now (\ref{chap16:prop16.6}) follows from the formulae (\ref{chap16:prop16.2}). 

In this example $d=3$; and so, $\tau E \neq R^d S(E)$. 

\begin{thebibliography}{99}
\bibitem{key1} I. N. Bernstein, I. M. Gel'fand\pageoriginale and
  S. I. Gel'fand, Differential operators on the base affine space and
  a study of $\mathfrak{g}$-modules, Lie Groups and their
  Representations, Summer School of the Bolyai J'anos
  Math. Soc. edited by I. M. Gel'fand, Halsted Press, Division of John
  Wiley and Sons, New York 1975, 21-64. 

\bibitem{key2} I. N. Bernstein, On a category of $G$-modules,
  Funk. Anal. Appl. 10 (1976), 1-8. 

\bibitem{key3} I. N. Bernstein and S. I. Gel'fand, Tensor products of
  finite and infinite dimensional representations of semisimple Lie
  algebras, preprint.

\bibitem{key4} A. Borel and N. Wallach, Continuous Cohomology,
  Discrete Subgroups and Representations of Reductive Groups, Annals
  of Math. Studies. Princeton University Press, Princeton, 1980. 

\bibitem{key5} A. Bouaziz, Sur les representations des algebra de Lie
  semisimples construites par T, Enright, preprint. 

\bibitem{key6} N. Bourbaki, Groups et algebra de Lie, I, II-III,
  IV-VI, Hermann, Paris, 1971, 1972, 1968.

\bibitem{key7} V. Deodhar, On a construction of representations and a
  problem of Enright, Invent. Math. 57 (1980), 101-118. 

\bibitem{key8} J. Dixmier, Algebres Enveloppantes,
  Gauthier-Villars. Paris, 1974. 

\bibitem{key9} M. Duflo, Repr\'esentations irreductibles des groupes
  semi-simple complexes, Lectures Notes 497 (1975), 26-88. 

\bibitem{key10} M. Duflo, Repr\'esentations unitarires irr\'eductibles
  des groups simples complexes de rang deux, Bull. Soc. Math. Fracnce,
  107 (1979). 55-96. 

\bibitem{key11}  M. Duflo, Representations unitaires des groupes
  semi-simple complexes, preprint. 

\bibitem{key12} T. J. Enright and V. S. Varadarajan, On an
  infinitesimal characterization of the discrete series, Ann. of
  Math. 102 (1975), 1-15.

\bibitem{key13} T. J. Enright and N. R. Wallach, The fundamental
  series or representations of a real semisimple Lie algebra, Acta
  Math. 140 (1978), 1-32. 


\bibitem{key14} T. J. Enright and N. R. Wallach, The fundamental
  series of semisimple Lie algebras and semisimple Lie groups,
  manuscript. 

\bibitem{key15} T. J. Enright,\pageoriginale On the fundamental series
  of a real semisimple Lie algebra: their irreducibility, resolutions
  and multiplicity formulae, Ann. of Math., 110 (1979), 1-82. 

\bibitem{key16} T. J. Enright, On the construction and classification
  of irreducible Harish-Chandra modules, Proc, Natl. Acad. Sci. 75
  (1978), 1063-1065. 

\bibitem{key17} T. J. Enright and R. Parthasarathy, The transfer of
  invariant pairings to lattices, to appear in Pac. J. of of Math. 


\bibitem{key18} T. J. Enright, Relative Lie algebra cohomology
  andunitary representations of complex Lie groups, Duke Math. J. 46
  (1979), 513-525. 

\bibitem{key19} T. J. Enright and N. R. Wallach, Notes on homological
  algebra and representations of Lie algebras, Duke Math. J. 47
  (1980), 1-15. 

\bibitem{key20} I. M. Gel'fand and M. A. Naimark, Unitary
  Representations of the Classical Groups, Trudy
  Mat. Inst. Steklov. 36 (1950). 

\bibitem{key21} Harish-Chandra, Representations of a semisimple Lie
  group II, III, Trans. Amer. Math. Soc. 76 (1954), 26-65, 234-253. 

\bibitem{key22} Harish-Chandra, The characters of semisimple Lie
  groups, Trans. Amer. Math. Soc., 83 (1956), 98-163. 

\bibitem{key23} P. J. Hilton and U. Stammbach, A Course in Homological
  Algebra, Springer-Verlag, 1970. 

\bibitem{key24} J. C. Jantzen, Moduln mit einen h\"ochsten Gewicht,
  Lectures Notes \#750, Springer-Verlag, Berlin, 1979. 

\bibitem{key25} A. Joseph, Dixmier's problem for Verma and principal
  series submodules, preprint. 

\bibitem{key26} D. Kazhdan and G. Lustig, Representations of Coxeter
  groups and Hecke algebras, Invent. Math., 53 (1979), 165-184. 

\bibitem{key27} R. P. Langlands, On the classification of irreducible
  representations of real algebraic groups, manuscript. 

\bibitem{key28} J. Lepowsky, Algebraic results on representations of
  semisimple Lie groups, Trans. Amer. Math. Soc., 176 (1973), 1-44. 

\bibitem{key29} D. Mili\v cic, Asymptotic behaviour of matrix
  coefficients of the discrete series, Duke Math. J. 44 (1977),
  59-88. 

\bibitem{key30} K. R. Parthasarathy, R. Ranga Rao and
  V. S. Varadarajan, Representations of complex semisimple Lie groups
  and Lie algebras, Ann. of Math. 85 (1967), 383-429. 

\bibitem{key31} N. R. Wallach,\pageoriginale  Harmonic Analysis on
  Homogeneous Spaces, M. Dekker, New York, 1973. 

\bibitem{key32} G. Warner, Harmonic Analysis on Semisimple Lie Groups
  I, II, Springer-Verlag, New York, 1972. 

\bibitem{key33} V. S. Varadarajan, Infinitesimal theory of
  representations of semisimple Lie groups, Harmonic Analysis and
  Representations of Semisimple Lie Groups, editors J. Wolf, M. Cohen
  and M. DeWilde, Reidel, Boston, 1980. 

\bibitem{key34} D. Vogan, Jr., The algebraic structure of the
  representations of semisimple Lie groups I, Ann. of Math. 109
  (1979), 1-60. 

\bibitem{key35} D. Vogan and G. Zuckerman, Unitary representations
  with continuous cohomology, manuscript. 

\bibitem{key36} D. P. Zelobenko, Operational calculus on a complex
  semisimple Lie group, Math. USSR, Izvestija, 3 (1969), 881-915. 

\bibitem{key37} D. P. Zelobenko, Representations of complex semisimple
  Lie groups, Itogi Nauki i Tekhniki, 11 (1973), 51-90; English
  translation J. of Soviet Math., 4 (19750, 656-680. 

\bibitem{key38} G. Zuckerman, Tensor products of infinite-dimensional
  and finitedimensional representations of semisimple Lie groups,
  Ann. of Math. 106 (1977), 295-308. 

\bibitem{key39} G. Zuchkerman, Continuous cohomology and unitary
  representations of real reductive groups, Ann. of Math. 107 (1978),
  495-516. 

\bibitem{key40} G. Zuchkerman, Construction of some modules via
  derived functors, to appear. 

\end{thebibliography}
