
\chapter{The category of admissible $(\mathfrak{g},
  \mathfrak{t})$-modules}\label{sec13}

The\pageoriginale results of this section are an application of the
functors defined in \S\ \ref{sec12} in the setting of semisimple complex Lie
groups. Here we analyze several cases where the functor $\sigma_0$
defined in \S\ \ref{sec12} is a natural inverse to the lattice functor
$\tau$. 

We will use the notation set up in \S\ \ref{sec6} - \S\ \ref{sec10}. Fix $\u{\lambda} \in
\mathfrak{L}$, $\u{\lambda} = (\lambda, \lambda')$ and let
$\mathfrak{U}$ denote the category of all admissible $(\mathfrak{g},
\mathfrak{t})$-modules with generalized $Z(\mathfrak{g})$-character
parametrized by the $\mathfrak{w}$-orbit of $\u{\lambda}$. Assume
$\u{\lambda}$ satisfies (\ref{chap6:subsec6.5}) and let $\mathfrak{B}$ denote the
category of $\mathfrak{g}$-modules which are finitely generated, have
$L(s\lambda, \lambda')$, $s \in \mathfrak{w}_0$, as irreducible
objects, and are semisimple $\mathfrak{t}$-modules. 

\setcounter{section}{13}
\setcounter{subsection}{0}
\subsection{}\label{chap13:subsec13.1}
The reader can check that the map, $B \mapsto B/(0, n^-_0)B$, induces
a natural equivalence of $\mathfrak{B}$ onto the category of finitely
generated $\mathfrak{g}_0$-modules which are
$U(\mathfrak{b}_0)$-locally finite, have generalized
$Z(\mathfrak{g}_0)$-character with orbit $\mathfrak{w}_0 \cdot
\lambda$  and have generalized $\mathfrak{h}_0$-weights $\nu$ with
$\nu + \lambda'$ integral. 

\setcounter{prop}{1}
\begin{theorem}\label{chap13:thm13.2}
Let $\u{\lambda} = (\lambda, \lambda') \in \mathfrak{L}$ with
$M(\lambda')$ irreducible and $\lambda'$ regular. Then the functor
$\tau$ gives a natural equivalence; $\tau : \mathfrak{B}
\xrightarrow{\sim} \mathfrak{U}$. 
\end{theorem}

Using translation functors (cf. Proposition \ref{chap5:thm5.2}), we may translate the
parameter $\lambda'$ and thus assume
\begin{equation*}
\re \lambda'_\alpha <<  0 \quad \text{for all} \quad \alpha \in
P_0. \tag{13.3} \label{eq13.3}
\end{equation*}

We establish several preliminary results before proving (\ref{chap13:thm13.2}). 

Let\pageoriginale $\wedge \mathfrak{t}$ denote the exterior algebra of
$\mathfrak{t}$ and put $\psi_0 = \{(s\lambda, \lambda')\}
\mid_\mathfrak{t} - 2 \delta_\mathfrak{t} + \xi \mid s \in
\mathfrak{w}_0$,  $\xi$ a weight of $\wedge \mathfrak{t}$ \}. With $t$
the maximal element of $\mathfrak{w}_\mathfrak{t}$ and with $t'$
denoting the affine shift $t'\mu = t(\mu +\delta_\mathfrak{t}) -
\delta_\mathfrak{t}$, $\mu \in \mathfrak{t}^*$, we put $\psi =
t'\psi_0$. By (\ref{eq13.3}), the elements of $\psi$ are
$P_\mathfrak{t}$-dominant; and so, we have a set of covariant functors
$\sigma_i$, $i \in \mathbb{N}$, defined by this choice of $\psi$. Recall
the category $\mathfrak{n}$ of \S\ \ref{sec12} and note $\mathfrak{U}
\subseteq \mathfrak{n}$. 

\setcounter{prop}{3}
\begin{prop}\label{chap13:prop13.4}
Assume $\u{\mu} = (s\lambda, \lambda')$ for some $s \in
\mathfrak{w}_0$. Then 
$$
\sigma_i X(\u{\mu}) = 
\begin{cases}
M(\u{\mu})&  \text{ if } i = 0\\
0 & \text{ if } i \geq 1. 
\end{cases}
$$
\end{prop}

\begin{proof}
For $0 \leqq j \leqq d = \dim (\mathfrak{b}/
\mathfrak{b}_\mathfrak{t})$  and $\nu = \u{\mu} \mid_\mathfrak{t} - 2
\delta_\mathfrak{t}$, put $L_j = \wedge^j(\mathfrak{b}/
\mathfrak{b}_\mathfrak{t}) \otimes \mathbb{C}_\nu$ and $E_j =
U(\mathfrak{g}) \bigotimes\limits_{U(\mathfrak{b}_\mathfrak{t})}
L_j$. Proposition 5.9 of \cite{key15} gives a resolution
\begin{equation*}
0 \to E_d \to \ldots \to E_0 \to M(\u{\mu}) \to 0. 
\tag{13.5}\label{eq13.5}
\end{equation*}
Write $M = M(\u{\mu})$ and let $M_s$, $s \in
\mathfrak{w}_\mathfrak{t}$, and $E_{j,s}$, $s \in
\mathfrak{w}_\mathfrak{t}$, be lattices above $M$ and $E_j$
respectively. By applying completion functors we obtain from (\ref{eq13.5}),
for $s \in \mathfrak{w}_\mathfrak{t}$, 
\begin{equation*}
0 \to E_{d,s} \to \ldots \to E_{0,s} \to M_s \to 0. 
 \tag{13.6}\label{eq13.6}
\end{equation*}
Theorem 5.7 of \cite{key13} asserts that (\ref{eq13.6}) is exact. For $s=l$,
(\ref{eq13.6}) is a $\psi$-resolution of $M_1$. 
\end{proof}

By (\ref{eq13.3}) each $\mathfrak{g}$-module $E_i$ is isomorphic to a module
of the form $U(\mathfrak{g}) \bigotimes\limits_{U(\mathfrak{t})} D_i$
with $D_i$ a sum of irreducible Verma $\mathfrak{t}$-modules with
highest weights\pageoriginale in $\psi_0$. In turn each $E_{i,1}$ is
isomorphic to a sum of modules of type $U(\xi)$, $\xi \in \psi$
(cf. \S\ \ref{sec12}); and so, we have $\sigma E_{i,1} = E_i$, $0 \leq i \leq
d$. Since (\ref{eq13.6}) for $s=1$ is a $\psi$-resolution of $M_1$, we obtain:
\begin{equation*}
\sigma_0 M_1 = M(\u{\mu}), \sigma_i M_1 = 0, \quad i \geq 1. 
 \tag{13.7}\label{eq13.7}
\end{equation*}

We now claim:
\begin{equation*}
\sigma_i M_s = 0, \quad i \in \mathbb{N}, \quad s \neq 1.
\tag{13.8}\label{eq13.8}
\end{equation*}

From (\ref{eq13.3}) and Proposition 4.13 of
\cite{key15}, $M_s$ does not have 
any $n_\mathfrak{t}$-invariants of weight $\nu$, $\nu \in \psi$; and
so, $0 \to M_s \to 0$ is a special $\psi$-resolution of $M_s$. Clearly
(\ref{eq13.8}) follows from this. 

Using (\ref{chap7:prop7.1}), we have a resolution:
$$
0 \to \mathfrak{m}_d \to \ldots \to \mathfrak{m}_0 \to X(\u{\mu}) \to 0
$$ 
where $\mathfrak{m}_i = \sum\limits_{\ell(s)=i} M_s$. By (\ref{eq13.7}) and
(\ref{eq13.8}) we have computed $\sigma_i \mathfrak{m}_j$, $j,i\in
\mathbb{N}$. Since $\sigma_i \mathfrak{m}_j =0$ for $i  \in
\mathbb{N}$, $j \geq 1$, we obtain $\sigma_i \mathfrak{m}_0 \simeq
\sigma_i X(\u{\mu})$, $i \in \mathbb{N}$. Now formulae (\ref{eq13.7}) proves
(\ref{chap13:prop13.4}). 

For any $\mathfrak{g}$-module $A$, put $A'$ (resp. $A^\sim$) equal to
the subspace of $n_\mathfrak{t}$-invariants of weight $\nu$, $\nu \in
\psi $ (resp. $\nu \in \psi_0$). 

\setcounter{prop}{8}
\begin{lemma}\label{chap13:lem13.9}
For $A \in \mathfrak{B}$, let $A_s$ be a lattice above $A$. Then the
surjection $A_1 \to \tau (A)$ induces a bijection $A'_1
\xrightarrow{\sim} \tau (A)'$. 
\end{lemma}

\begin{proof}
By (\ref{eq13.3}), for $s \neq 1$, $A'_s = 0$. Since elements of $\psi$ are
dominant $A'_1 \to \tau (A)'$ is surjective. This implies the induced
map is a bijection. 
\end{proof}

\begin{lemma}\label{chap13:lem13.10}
Assume\pageoriginale (\ref{eq13.3}). Let $B \in \mathfrak{B}$ and let $B_s$ be
a lattice above $B$ and let $E_* \to B_1$ be a special
$\psi$-resolution of $B_1$. Then $\sigma E_* \to B$ is a complex and
$\sigma E^\sim_0 \to B^\sim$ is surjective. 
\end{lemma}

\begin{proof}
By (\ref{chap12:coro12.11}), $\sigma E_* \to B$ is a complex. By definition of
$\psi$-resolution $E'_0 \to B'_1$ is surjective. By Proposition 4.13
\cite{key15}, this implies  $\sigma E^\sim_0 \to B^\sim$ is
surjective. 
\end{proof}

\begin{lemma}\label{chap13:lem13.11}
Let $A, B \in \mathfrak{B}$ (resp. $\mathfrak{U}$) and $\varphi: A \to
B$. Then $\varphi$ is injective (resp. surjective) if and only if the
restriction $\varphi: A^\sim \to B^\sim$ (resp. $\varphi: A' \to B'$)
is injective (resp. surjective). 
\end{lemma}

\begin{proof}
By (\ref{chap3:prop3.9}), both $\mathfrak{U}$ and $\mathfrak{B}$ are
$\mathfrak{t}$-semisimple categories. So, for $C \in \mathfrak{U}$, $D
\in \mathfrak{B}$, $C \neq 0$, $D \neq 0$ implies $C' \neq 0$, $D^\sim
\neq 0$. This implies the assertions of (\ref{chap13:lem13.11}). 
\end{proof}

\begin{lemma}\label{chap13:lem13.12}
Let $A, B\in \mathfrak{U}$ and $E_* \to A$ be a special
$\psi$-resolution. If $\varphi$ is a $\mathfrak{g}$-module map
$\varphi: E_0 \to B$ with $\varphi \Iim E_1 = 0$, then $\varphi$
induces a map $\varphi'$ with the following commutative diagram:
$$
\xymatrix{
E_0  \ar[r]^\varphi \ar[d] & B\\
A \ar[ur]_{\varphi'} & 
}
$$
\end{lemma}

\begin{proof}
$E_0$ is isomorphic to a sum of modules $U(\mu)$, $\mu \in \psi$, and
  so, $E_0$ admits a unique maximal $U(\mathfrak{t})$-locally finite
  quotient. Let $D$ denote the maximal $U(\mathfrak{t})$-locally
  finite quotient of $E_0 / \Iim E_1$. Since $A \in \mathfrak{U}$, the
  map $E_0 \to A \to 0$ induces a map $D \to A$. Clearly the map gives
  an isomorphism $D' \xrightarrow{\sim} A'$; and so, by
  (\ref{chap13:lem13.11}) the
  map is an isomorphism $D \xrightarrow{\sim} A $. Since\pageoriginale
  $B \in \mathfrak{U}$, $\varphi$ induces a map of $D$ to $B$ and,
  with $D \simeq A$, we obtain a map $\varphi': A \to B$ with the
  commutative diagram (\ref{chap13:lem13.12}). 
\end{proof}


\begin{prop}\label{chap13:prop13.13}
Assume (\ref{chap13:prop13.13}). Then $\tau : \mathfrak{B} \to \mathfrak{U}$ is a
natural equivalence of categories and $\sigma_0 : \mathfrak{U} \to
\mathfrak{B}$ is the natural inverse of $\tau$. When restricted to
$\mathfrak{U}$, $\sigma_i$ is exact, for $i \in \mathbb{N}$, and
$\sigma_i = 0$ for $i \in \mathbb{N}^*$. 
\end{prop}

\begin{proof}
For $i \in \mathbb{N}$, put $\gamma_i = \sigma_i \circ \tau$ and
$\nu_i = \tau \circ \sigma_i$. We first check that:
\end{proof}

\setcounter{subsection}{13}
\subsection{}\label{chap13:subsec13.14}
$\gamma_0$ is naturally equivalent to the identity functor on
$\mathfrak{B}$. 


Let $B\in \mathfrak{B}$ and let $E_* \to \tau(B)$ be a special
$\psi$-resolution of $\tau (B)$. Using (\ref{chap13:lem13.9}), and letting $B_s$ be a
lattice above $B$, $E_*$ also gives a special $\psi$-resolution of
$B_1$. Now apply (\ref{chap12:coro12.11}) to obtain the complex $\sigma E_* \to
B$. This map induces a map $t_B: \gamma_0 B \to B$. One may easily
check that the correspondence $B \mapsto t_B$ is natural; i.e., for
$B, C \in \mathfrak{B}$, $\varphi: B \to C$ then the following diagram
is commutative: 
$$
\xymatrix@R=1.2cm@C=1.2cm{
B \ar[r]^\varphi & C \\
\gamma_0 B \ar[r]_{\gamma_0 \varphi} \ar[u]^{t_B} & \gamma_0
C\ar[u]_{t_C} 
}
$$

To prove (\ref{chap13:subsec13.14}) we must check that $t_B$ is an isomorphism for each
$B \in \mathfrak{B}$. First we check that $\gamma_i B \in
\mathfrak{B}$. Clearly $\gamma_i B$ is a semisimple $t$-module by
construction and so it is sufficient to check this for irreducible
$B$, say $B = L (s\mu, \lambda')$ with $\mu$
-$P_{\lambda'}$-dominant. We proceed by induction on $\ell(s)$. 

For\pageoriginale $s =1$, (\ref{chap13:prop13.4}) gives the result. If $\ell(s) > 0$
then put $M = M (s\lambda, \lambda')$ and let $0 \to J \to M \to B \to
0$ be the short exact sequence induced by the natural map of $M$ onto
$B$. Since $\tau$ is exact, we obtain the long exact sequence. 
$$
 \to \gamma_i J \to \gamma_i M \to \gamma_i B \to \gamma_{i-1} J \to
 \ldots \to \gamma_0 M \to \gamma_0 B \to 0. 
$$
By the induction hypothesis $\gamma_i J \in \mathfrak{B}$ while (\ref{chap13:prop13.4})
gives $\gamma_i M \in \mathfrak{B}$. Therefore $\gamma_i B \in
\mathfrak{B}$; and so, $\gamma_i $ maps $\mathfrak{B}$ into
$\mathfrak{B}$ for all $i$. 

For any $B \in \mathfrak{B}$ we have an isomorphism of
$U(\mathfrak{g})^\mathfrak{t}$-modules, $B^\sim \simeq B'_1$; and so,
we have 
\begin{equation*}
B^\sim \simeq E'_0 / \Iim E'_1 \simeq \sigma E^\sim_0 / \Iim E^\sim_1.
\tag{13.15}\label{eq13.15}
\end{equation*}

Combining (\ref{eq13.15}) and (\ref{chap13:lem13.10}), $\gamma_0 B^\sim$ and $B^\sim$ are
isomorphic by $t_B$; and then, by (\ref{chap13:lem13.11}), $t_B$ is an
isomorphism. This completes the proof of (\ref{chap13:subsec13.14}). 

Since the identity functor is exact, $\gamma_0$ is exact. This fact
and (\ref{chap13:prop13.4}) now easily imply that $\gamma_i = 0$ for $i \geq 1$. In
turn this shows that for each irreducible object $A$ in
$\mathfrak{U}$, $\sigma_i A = 0$, $i \in \mathbb{N}^*$. So $\sigma_i
=0$ on $\mathfrak{U}$, $i \in \mathbb{N}^*$, and thus, $\sigma_0$ is
exact. Next we claim: 


\setcounter{subsection}{15}
\subsection{}\label{chap13:subsec13.16}
$\nu_0$ is naturally equivalent to the identity on $\mathfrak{U}$. 

For\pageoriginale $A \in \mathfrak{U}$, let $E_* \to A$ be a special
$\psi$-resolution of $A$. Then we have the exact sequence $\sigma E_1
\to \sigma E_0 \to \sigma_0 A \to 0$. Let $C_s$, $s \in
\mathfrak{w}_\mathfrak{t}$, be the lattice above $\sigma_0 A$. By
applying completion functors to the exact sequence we obtain the
complex: $E_1 \to E_0 \to C_1$. Let $\varphi$ denote the composition
of the map $E_0 \to C_1$ with the quotient map $C_1 \to \nu_0 A \to
0$. Then by (\ref{chap13:lem13.12}), $\varphi$ induces a map $\varphi'$ of $A$ to
$\nu_0 A$. Put $s_A = \varphi'$. To prove (\ref{chap13:subsec13.16}) we must that $s_A$
is an isomorphism for all $A \in\mathfrak{U}$. For $A$ isomorphic to a
principal series $X(\u{\mu})$ with $\u{\mu} = (\mu, \lambda')$, we may
choose the special $\psi$-resolution (\ref{eq13.6}) for $s
= 1$. Since (\ref{eq13.5}) 
splits as a $\mathfrak{t}$-module sequence (cf. Theorem 6.17
\cite{key15}), we find that $s_A$ gives an isomorphism of $A'$ onto
$\nu_0 A'$. Then (\ref{chap13:lem13.11}) implies that $s_A$ is an isomorphism. Using
the exactness of $\nu_0$ and the five lemma we conclude that $s_A$ is
an isomorphism for irreducible $A$ and then for all $A$ in
$\mathfrak{U}$. This proves (\ref{chap13:subsec13.16}) and completes the proof of
(\ref{chap13:prop13.13}).


\medskip
\noindent{\textbf{Proof of Theorem (\ref{chap13:thm13.2}).~}}
The remark before (\ref{eq13.3}) and Proposition
(\ref{chap13:prop13.13}) combine to prove 
(\ref{chap13:thm13.2}). 


