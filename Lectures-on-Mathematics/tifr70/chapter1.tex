
\chapter{Preliminaries}\label{chap1}

IN\pageoriginale THIS CHAPTER, we will study some basic results about
convolution and the Fourier transform. 

\section{General Theorems About Convolutions}\label{chap1:sec1}

We will begin with a theorem about integral operators.

\begin{thm}\label{chap1:sec1:thm1.1} %theorem 1.1
  Let $K$ be a measurable function on $\mathbb{R}^n \times
  \mathbb{R}^n$ such that, for some $c > 0$, 
  
  $\int |K(x,y)|dy \leq c, \int |K(x,y)| dx \leq c$, for every
  $x,y$ in $\mathbb{R}^n$. 
\end{thm}

If $1 \leq p \leq \infty$ and $f \in L^p (\mathbb{R}^n)$, then the
function $Tf$, defined by $Tf(x)= \int K(x,y) f(y)dy$ for almost every
$x$ in $\mathbb{R}^n$, belongs to $L^p(\mathbb{R}^n)$ and further, 
$$
|| Tf  || _p \leq c || f ||_p.
$$
\begin{proof}
  If $p= \infty$, the hypothesis $\int |K(x,y)| dx \leq c$ is
  superfluous and the conclusion of the theorem is obvious. If $p <
  \infty$, let $q$ denote the conjugate exponent. Then, by H\"{o}lder 's
  inequality, 
  \begin{align*}
    |Tf (x)|&\leq \left\{ \int |K(x,y)|dy \right\} ^{1/q} \left\{ \int
    |K(x,y)|\,|f(y)|^p dy \right\}^{1/p}\\ 
    & \leq c^{1/q} \left\{ \int |K(x,y)|\,|f(y)|^p dy \right\} ^{1/p}.
  \end{align*}
\end{proof}

\noindent
From this we have,
 \begin{align*}
   \int |Tf(x)|^p dx & \leq c^{p/q} \iint |K(x,y)||f(y)|^p dy dx \\
   & \leq c^{1+p/q} \int |f(y)|^p dx= c^{1+p/q} || f || ^p_p.
 \end{align*} 

\noindent
Therefore\pageoriginale $|| Tf || _p \leq c || f || _p$.

Next, we define the convolution of two locally integrable functions.

\setcounter{defi}{1}
\begin{defi}\label{chap1:sec1:def1.2} %Definition 1.2
Let $f$ and $g$ be two locally integrable functions. The {\em
  convolution} of $f$ and $g$, denoted by $f * g$, is define by  
$$
(f * g) (x) = \int f(x - y)g (y) dy = \int f(y)g (x-y)dy = (g*f)(x),
$$
provided that the integrals in question exist.
\end{defi}

The basic theorem on convolution is the following theorem, called
\textit {Young's inequality}.
 
\setcounter{thm}{2}
\begin{thm}[Young's Inequality]\label{chap1:sec1:thm1.3}%%% thm1.3
 Let $f \in L^1 (\mathbb{R}^n)$ and $g \in L^p
  (\mathbb{R}^n)$, for $1 \leq p \leq \infty$. Then $f*g \in L^p
  (\mathbb{R}^n)$ and  
\end{thm}
$$
|| f * g || _p \leq || g || _p || f || _1.
$$
\begin{proof}
  Take $K(x, y) = f(x - y)$ in Theorem \ref{chap1:sec1:thm1.1}. Then
  $K(x, y)$ satisfies 
  all the conditions of Theorem \ref{chap1:sec1:thm1.1}. and the
  conclusion follows immediately. 
\end{proof}

The next theorem underlies one of the most important uses of
convolution. Before coming to the theorem, let us prove the following  

\setcounter{lem}{3}
\begin{lem} \label{chap1:sec1:lem1.4}% Lemma 1.4
  For a function $f$ defined on $\mathbb{R}^n$ and $x$ in
  $\mathbb{R}^n$, we define a function $f^x$ by $f^x (y)=f(y-x)$. If $f
  \in L^p, 1 \leq p < \infty$, then $\lim\limits_{x \to O} || f^x -f
  ||_p=0$. 
\end{lem} 

\begin{proof}
  If\pageoriginale $g$ is a compactly supported continuous function,
  then $g$ is uniformly 
  continuous, and so $g^x$ converges to $g$ uniformly as $x$ tends
  to $0$. 
\end{proof}

Further, for $|x| \leq 1, g^x$ and $g$ are supported in a common
compact set. Therefore, $\lim\limits_{x \to 0} || g^x-g ||
_p=0$. Given $f \in L^p$, we can find a function $g$ which is
continuous and compactly supported such that $|| f-g ||_p < \in /3$
for $\in > 0$. But then $|| g^x-f^x ||_p < \in /3$ also
holds. Therefore 
\begin{align*}
  || f^x-f || _p & \leq || f^x -g^x ||_p +|| g^x -g
  ||_p + || g-f ||_p \\ 
  & \leq 2 \in /3 + || g^x -g || _p.
\end{align*}

Since $\lim\limits{x \to 0} || g^x-g ||_p =0$, we can choose $x$ close
to $0$ so that $|| g^x-g ||_p < \in /3$. Then $|| f^x-f || < \in$ and
this proves the lemme since $\in$ is arbitrary. 

\setcounter{rem}{4}
\begin{rem} \label{chap1:sec1:rem1.5}%Remark 1.5
  \textit{The above lemma is false when} $p = \infty$. Indeed, ``$f^x
  \to f$ in $L^\infty$'' means precisely that $f$ is uniformly
  continuous. 
\end{rem}

Let us now make \textit{two important observations about convolutions}
which we shall use without comment later on. 
\begin{enumerate}[i)]
\item $\Supp (f*g) \subset \Supp f + \Supp g$, where
  $$
  A+B = \{ x+1: x \in A, y \in B \}.
  $$
\item If $f$ is of class $C^k$ and $\partial^\alpha (|\alpha| \leq
  k)$ and $g$ satisfy appropriate conditions so that differentiation
  under the integral sign is justified, then $f *g$ is of class $C^k$
  and $\partial^\alpha (f*g)=(\partial^\alpha f)*g$. 
\end{enumerate}

\setcounter{thm}{5}
\begin{thm} \label{chap1:sec1:thm1.6}%Theorem 1.6
  Let\pageoriginale $g \in L^1 (\mathbb{R}^n)$ and $\int g (x)dx
  =a$. Let $g_ \in (x)= 
  \in ^{-n}g (x/ \in)$ for $\in >0$. Then, we have the following: 
  \begin{enumerate}[\rm i)]
  \item \textit{If $f \in L^p (\mathbb{R}^n), p < \infty,f*g _ \in$
    converges to $af$ in $L^p$ as $\in$ tends to 0.} 
  \item \textit{If $f$ is bounded and continuous, then $f*g_\in$
    converge to $af$ uniformly on compact sets as $\in$ tends to 0.} 
  \end{enumerate}
\end{thm}

\begin{proof}
  By the change of variable $x \to \in x$, we see that $\int g_\in
  (x)dx=a$ for all $\in > 0$. Now 
  \begin{align*}
    (f*g_\in)(x)-af (x) &= \int f(x-y)g_ \in (y)dy - \int f(x)g_\in (y)dy \\
    &=\int [f(x-y)-f(x)] \,g_ \in (y)dy \\
    &= \int [f(x- \in y)-f(x)] g(y)dy \\
    &=\int [f^{ \in y}(x)-f(x)]g (y)dy.
  \end{align*}
\end{proof}

If $f \in L^p$ and $p < \infty$, we apply Minkoswski's inequality  for
integrals to obtain 
$$
|| f*g_ \in - af ||_p \leq \int || f^{\in y}-f || _p |g(y)|dy.
$$

The function $y \to || f^{\in y}- f ||_p$ is bounded by $2 || f || _p$
and tends to 0 as $\in$ tends to 0 for each $y$, by
lemma \ref{chap1:sec1:lem1.4}. 
Therefore, we can apply Lebesgue Dominated Convergence theorem
to get the desired result. 

On the other hand, suppose $f$ is bounded and continuous. Let $K$ be
any compact subset of $\mathbb{R}^n$. Given $\delta > 0$, choose a
compact set $G~ \mathbb{R}^n$ such that 
$$
\int\limits_{\mathbb{R}^n -G} |g(y)|dy < \delta.
$$\pageoriginale

Then
\begin{multline*}
\Sup\limits_{x \in k}( |f *g_\epsilon) (x) - af (x) | \leq 2 \delta
|| f || _ \infty \\
+ \Sup\limits_{(x, y)\epsilon K \times G}  |f(x-
\epsilon y) - f(x)| \int \limits_G | g | dy.
\end{multline*}

Since $f$ is
uniformly continuous on the compact set $K$ the second  term tends to
0 as $\in$ to 0. Since $\delta$ is arbitrary, we see that 
$$
\sup\limits {x \in K} |(f*g_ \epsilon ) (x) - af (x)| \to 0
~\text{as}~ \epsilon \to 0. 
$$

Hence the theorem is proved.
\setcounter{coro}{6}
\begin{coro} \label{chap1:sec1:coro1.7}%Corollary 1.7
  The space $C^\infty _o (\mathbb{R}^n)$ is dense in $L^p
  (\mathbb{R^n})$ for $1 \leq p < \infty$. 
\end{coro}
\begin{proof}
  Let
  \begin{alignat*}{2}
    \phi (x) & = e^{-1/(1-|x|^2)} ~&&\text{for}~ |x|< 1 \\
    &= 0  &&\text{for}~ |x| \leq 1
  \end{alignat*}

  Then $\phi \epsilon C^\infty _o (\mathbb{R}^n)$ and $\int \phi (x)
  dx = 1/a >0$. If $f \epsilon L^p$ and has compact support, then
  $a(f* \phi _ \epsilon ) \epsilon C^ \infty _o (\mathbb{R}^n)$
  and by theorem \ref{chap1:sec1:thm1.6}, $a(f* \phi_ \epsilon )$ converges to $f$ in
  $L^p$ as $\epsilon$ tends to 0. Since $L^p$ functions with
  compact support are dense in $L^p$, this completes the proof. 
\end{proof}

\setcounter{prop}{7}
\begin{prop} \label{chap1:sec1:prop1.8}%Proposition 1.8
  Suppose $K \subset \mathbb{R}^n$ is compact and $\Omega
  \supset K$ be an open subset of $\mathbb{R}^n$. Then there exists a
  $C^\infty _o$ function $\phi$ such that $\phi (x) = 1$ for $x
  \epsilon K$ and $ \Supp  \phi \subset \Omega$. 
\end{prop}

\begin{proof}
  Let\pageoriginale $V = \{ x \in  \Omega: d (x, K) \leq \dfrac{1}{2} \delta \}$ where
  $\delta = d(K, \mathbb{R}^n \backslash \Omega )$. Choose a $\phi _o \epsilon
  C^\infty_o$ such that $\Supp \phi_o \subset B \left(0, \dfrac{1}{2}
  \delta\right)$ and $\int \phi_o (x)dx=1$. Define 
  $$
  \phi (x) = \int\limits_V \phi_o (x - y) dy = (\phi_0 * X_V) (x). 
  $$
  Then $\phi (x)$ is a function with the required properties.
\end{proof}

\section{The Fourier Transform}\label{chap1:sec2} %Section 2

In this section, we will give a rapid introduction to the theory of
the Fourier transform. 

For a function $f \epsilon L^1(\mathbb{R}^n)$, the \textit {Fourier
  transform} of the function $f$, denoted by $\hat{f}$, is defined by  
$$
\hat{f}(\xi) = \int e^{-2 \pi i x. \xi}f (x) dx, \,\xi \epsilon \mathbb{R}^n.
$$
\setcounter{rem}{8}
\begin{rem}\label{chap1:sec2:rem1.9}
  Our definition of $\hat{f}$ differs from some other in the placement of
  the factor of $2 \pi$. 
\end{rem}

\noindent\textit{BASIC PROPERTIES OF THE FOURIER TRANSFORM}

For
\begin{equation*}
f \epsilon  L^1, || \hat{f} ||_\infty \leq || f
||_1.\tag{1.10}\label{chap1:sec2:eq1.10}  
\end{equation*}

The proof of this is trivial.

For
\begin{equation*}
  f,g \in L^1, (f*g)^{~\hat{}}~ (\xi)= \hat{f}(\xi)
  \hat{g}(\xi).\tag{1.11}\label{chap1:sec2:eq1.11} 
\end{equation*}

Indeed,
\begin{align*}
  (f *g)^ {\hat{}}(\xi) &= \iint e^{-2 \pi i x. \xi}f(y) g(x-y)dy dx\\
  &=\iint e^{-2 \pi i (x-y)\cdot \xi}g(x-y)e^{-2 \pi i Y. \xi} f(y)dy dx\\
  &= \int e^{-2 \pi i (x-y)\cdot \xi}g (x-y)dx \int f(y)e^{-2 \pi i y. \xi}dy\\
  &=\hat{f}(\xi)\hat{g} (\xi ).
\end{align*}

Let\pageoriginale us now consider the Fourier transform in the
Schwartz class $S=S (\mathbb{R}^n)$. 

\setcounter{prop}{11}
\begin{prop} \label{chap1:sec2:prop1.12}%Proposition1.12
  For $f \in S$, we have the following:
  \begin{enumerate}[\rm i)]
  \item $\hat {f} \epsilon C^ \infty (\mathbb{R}^n)$ \textit{and
    $\partial ^\beta \hat{f}=\hat{g}$ where $g(x) = (-2 \pi i x)^\beta
    f(x)$}. 
  \item $(\partial ^\beta f)^{\hat{}}(\xi)=(2 \pi i \xi)^\beta \hat{f}(\xi)$.
  \end{enumerate}
\end{prop}

\begin{proof}
  \begin{enumerate}[i)]
  \item Differentiation under the integral sign proves this.
  \item For this, we use integration by parts.
  \end{enumerate}
  \begin{align*}
    (\partial^\beta f)^{\hat{}} (\xi) & = \int e^{-2 \pi i
      x. \xi}(\partial ^\beta f)(x)dx \\ 
    &= (-1)^{|\beta|} \int \partial_ \beta [e^{-2 \pi i x. \xi }]f(x)dx\\
    &= (-1)^{|\beta|} (-2 \pi i \xi)^\beta \int e^{-2 \pi i x\cdot \xi}f(x)dx\\
    &= (2 \pi i \xi)^\beta \hat{f}(\xi).
  \end{align*}
\end{proof}

\setcounter{coro}{12}
\begin{coro} \label{chap1:sec2:coro1.13} %Corollary 1.13
	If $f \epsilon S$, then $\hat{f} \in S$ also.
\end{coro}

\begin{proof}
  For  multi-indices $\alpha$ and $\beta$, using proposition
  \ref{chap1:sec2:prop1.12}, we have
  \begin{align*}
    \xi ^\alpha (\partial ^\beta \hat{f})(\xi) & = \xi^\alpha ((-2 \pi i
    x)^\beta f(x))^ {\hat{~}}(\xi)\\ 
    &= (2 \pi i)^{-|\alpha|}[\partial ^\alpha ((-2 \pi i x)^\beta f(x))
      ^{\hat{}} (\xi)]\\ 
    &= (-1)^{|\beta|}(2 \pi i)^{|\beta|-|\alpha|}(\partial ^\alpha
    (x^\beta f (x)))^{\hat{~}} (\xi ) 
  \end{align*}

  Since $f  \epsilon S, \partial ^\alpha (x^ \beta f(x)) \epsilon
  L^1$ and hence $(\partial ^ \alpha (x^\beta f(x)))^{\hat{~}}
  \epsilon L^ \infty$. Thus $\xi^ \alpha$ $(\partial ^\beta \hat{f})$
  is bounded. Since $\alpha$ and $\beta$ are arbitrary, this proves that
  $\hat{f} \epsilon S$. 
\end{proof}

\setcounter{coro}{13}
\begin{coro}[RIEMANN-LEBESGUE LEMMA]\label{chap1:sec2:coro1.14} 
  If $f \epsilon L^1$, then $\hat{f}$ is
  continuous and vanishes at $\infty$. 
\end{coro}
\begin{proof}
  By\pageoriginale Corollary \ref{chap1:sec2:coro1.13}, this is true for $f \epsilon S$. Since  $S$ is
  dense in $L^1$ and $|| \hat{f} || _ \infty \leq || f ||_1$, the same
  in true for all $f \epsilon L^1$.  
\end{proof}

  Let us now compute the Fourier transform of the Gaussian.

\setcounter{thm}{14}
\begin{thm}\label{chap1:sec2:thm1.15} %Theorem 1.15
  Let $f(x)=e^{- \pi a|x|^2}$, Re $a > 0$. Then, $\hat{f}(\xi ) = a^{-n/2}$
  $e^{-a^{-1}}\pi |\xi|^2$. 
\end{thm}

\begin{proof} 
$$
\displaylines{\hfill \hat{f}(\xi)  = \int e^{-2 \pi i x. \xi} e^{-a
    \pi |x|^2} dx\hfill \cr
  \text{i.e.,}\hfill
  \hat{f}(\xi)  = \prod \limits ^{n}_{j=1} \int \limits^ {\infty}_{-
    \infty} e^{-2 \pi i x _j \xi _j}e^{- a \pi x^2_j}dx_j.\hfill} 
$$ 

Thus it suffices to consider the case $n=1$. Further, we can take
$a=1$ by making the change of variable $x \to a^{-1/2}x$. 
\end{proof}

Thus we are assuming $f(x) = e^{- \pi x^2}, x \epsilon
\mathbb{R}$. Observe that $f'(x) + 2 \pi x f(x) = 0$. Taking the
Fourier transform, we obtain  
$$
2 \pi i \xi \hat{f}(\xi) + i \hat{f'}' (\xi) = 0.
$$

Hence 
$$
\hat{f'}' (\xi)/ f (\xi)=-2 \pi \xi
$$
which, on integration, gives $f (\xi ) = c e^{-\pi \xi^2}$, $c$ being a constant.

The constant $c$ is given by 
$$
c = \hat {f}(0) = \int \limits ^\infty _{- \infty} e^{- \pi x^2} dx = 1.
$$

Therefore $\hat{f}(\xi) = e^{- \pi \xi^2}$, which completes the proof.

We now derive the \textit{Fourier inversion formula for the Schwartz
  class $S$}.

Let\pageoriginale us define $f^V (\xi) = \int e^{2 \pi i x \cdot \xi} f (x) dx =
\hat{f}(- \xi )$. 

\setcounter{thm}{15}
\begin{thm}[Fouries Inversion Theorem]\label{chap1:sec2:thm1.16} %Theorem 1.15
  For $f \epsilon
  (\hat{f})^\vee = f$. 
\end{thm}

\begin{proof}
  First, observe that for $f, g \epsilon  L^1, \int f \hat{g} = \int
  \hat{f} g$. In fact, 
  \begin{align*}
    \int \hat{f} (x) g (x) dx & = \iint e^{-2 \pi iy \cdot x} f(y)g(x) dy dx\\
    &= \int \left[\int e^{-2 \pi iy \cdot x} g(x) dx\right] f(y) dy\\
    &= \int \hat{g}(y) f(y) dy.
  \end{align*}
\end{proof}

Given $\epsilon > 0$ and $x$ in $\mathbb{R}^n$, take the function
$\phi$ defined by  
$$
\phi (\xi) = e^{-2 \pi ix\cdot \xi - \pi \epsilon^2 | \xi |^2}.
$$

Now 
\begin{align*}
\hat{\phi}(y) & = \int e^{-2 \pi i y \cdot \xi } e^{-2 \pi ix\cdot \xi - \pi
  \epsilon^2|\xi|^2}d \xi  \\ 
& = \int e^{-2 \pi i(y-x)\cdot \xi_e- \pi \epsilon^2 | \xi |^2} d \xi\\
& = \in^{-n}e^{-\pi \in^{-2} |x-y|^2}.
\end{align*}

If we take $g(x) = e^{-\pi |x|^2} $ and define $g_{\epsilon}(x) =
\epsilon^{-n} g(x/\epsilon)$, then  
$$
\hat{\phi}(y) = g_{\epsilon}(x-y).
$$

Therefore 
\begin{align*}
  \int e^{2 \pi i x\cdot \xi } \hat{f}(\xi) e^{-\pi \epsilon^2|\xi|^2} d \xi
  & = \int \hat{f}(\epsilon) \phi (\xi) d\xi\\
  & = \int f(y) \hat{\phi}(y) dy \\
  &= \int f(y) g_{\epsilon}(x-y) dy\\
  & = (f*g_{\epsilon }) (x) 
\end{align*}

But\pageoriginale as $\epsilon$ tends to 0, $(f*g_{\epsilon})$ converges to
$f$, by Theorem \ref{chap1:sec1:thm1.6} and clearly  
$$
\int e^{2 \pi i\cdot x \xi } \hat{f}(\xi) e^{-\pi
  \epsilon^2|\xi|^2} d \xi \to \int \hat{f}(\xi)e^{2 \pi i\cdot x
 .\xi_{d \xi } } 
$$
Therefore $(\hat{f})^v = f$.

\setcounter{coro}{16}
\begin{coro}\label{chap1:sec2:coro1.17}%Corollary 1.17
  The Fourier transform is an isomorphism of $S$ onto $S$.
\end{coro}

Next, we prove the \textit{ Plancherel Theorem}.

\setcounter{thm}{17}
\begin{thm}\label{chap1:sec2:thm1.18} % Theorem 1.18
  The Fourier transform uniquely extends to a unitary map of $L^2
  (\mathbb{R}^n)$ onto itself. 
\end{thm}

\begin{proof}
  For $f \epsilon S$, define $\tilde{f}(x) = \overline{f(-x)}$. Then
  it is easily checked that $\hat{\tilde{f}}=\bar{\hat{f}}$, so that  
  \begin{align*}
    || f ||^2_2 & = \int |f(x)|^2 dx \\
    & = \int f(x) \tilde{f}(-x) dx \\
    & = (f * \tilde{f}) (0)\\
    & = \int (f_* \tilde{f})^{\hat{~}} (\xi ) d \xi\\
    & = \int \hat{f}(\xi ) \hat{\tilde{f}}(\xi ) d \xi \\
    & = \int \hat{f}(\xi ) \bar{\hat{f}}(\xi) d \xi = || \hat{f}||^2_2.
  \end{align*}

  Therefore, the Fourier transform extends continuously to an isometry
  of $L^2$. It is a unitary transformation, since its image $S$ is dense
  in $ L^2$. 
\end{proof}

Let us observe how the Fourier transform interacts with translations,
rotations and dilations.  

\noindent (1.19) \qquad {\em The\pageoriginale Fourier transform and
  translation}: If $f^x(y) f(y-x)$ then  
\begin{align*}
  \hat{f}^x (\xi) & = \int e^{2 \pi i y \cdot \xi } f(y-x) dy \\
  & = \int e^{2 \pi i (z+x)\cdot \xi} f(z) dz ~(\text{ by putting }~ y-x = z ) \\
  & = \int e^{2 \pi ix\cdot \xi } \hat{f}(\xi).
\end{align*}
(1.20)\qquad  {\em The Fourier transform and rotations
  (orthogonal transformations):} 

Let $T: \varmathbb{R}^n \to \varmathbb{R}^n$ be an orthogonal
transformation. Then  
\begin{align*}
  (f \circ T)^{\hat{~}}(\xi) & = \int e^{-2 \pi ix\cdot \xi } (f o T) (x)dx \\
  & = \int e^{-2 \pi iT^{-1}y \cdot \xi} f(y)dy ~(\text{ by putting } ~y = Tx )\\
  & = \int e^{-2 \pi iy \cdot T \xi }f(y)dy \\
  & = \hat{f}(T \xi ) = (\hat{f} o T) (\xi).
\end{align*}

Thus, $(f o T)^{\hat{~}} = \hat{f} o T$ i.e. ${\hat{~}}$ commutes
with rotations. 

\noindent (1.21) \qquad 
{\em The Fourier transform and dilation}: Let $f_r (x) = r^{-n} f(x/r)$.

Then 
\begin{align*}
  \hat{f}_r (\xi) & = \int e^{-2 \pi ix \cdot \xi } r^{-n} f(x/r)dx \\
  & = \int e^{-2 \pi iy\cdot \xi } f(y)dy = \hat{f}(r \xi). 
\end{align*}

The last equation suggests, roughly: the more spread out $f$ is, the
more $\hat{f}$ will be concentrated at the origin and vice versa. This
notion can be put in a precise form as follows. 

\noindent (1.22)\qquad  {\em HEISENBERG INEQUALITY $(n=1$: For $ f
  \epsilon S (\mathbb{R})$, we have } 
$$
||xf(x)||_2 || \xi \hat{f}(\xi) ||_2 \geq (1/4 \pi) ||f||^2_2.
$$
\begin{proof}
  Observe\pageoriginale that 
  $$
  \frac{d}{dx}(xf(x)) = x \frac{df}{dx}(x) + f(x).
  $$
  Thus 
{%\fontsize{10pt}{12pt}\selectfont
  \begin{align*}
    ||f||^2_2 & = \int f(x) \overline{f(x)} dx \\
    & = \int \overline{f(x)} \left[\frac{d}{dx}(xf(x)) - x
      \frac{df}{dx}(x)\right] dx \\
    & = -\int xf(x) \frac{d\bar{f}}{dx}(x) dx - \int x \frac{df}{dx}(x)
    \bar{f}(x) dx \\ 
    & = -2 Re \int x f(x) \frac{d\bar{f}}{dx}(x) dx.\\
    & \leq 2|| xf(x) ||_2 ||\frac{d\bar{f}}{dx}||_2 ~\text{(by
      Cauchy-Schwarz) }\hspace{1.5cm}\\ 
    \text{i.e.,} \hspace{1.5cm} ||f||^2_2  &\leq 2
    ||xf(x)||_2||\frac{df}{dx}||_2. 
  \end{align*}}\relax
  
  But 
  $$
  (\frac{d\bar{f}}{dx})^{\hat{}}(\xi) = 2 \pi i \xi \hat{f}(\xi).
  $$

  Therefore,
  $$
  || f ||^2_2 \leq 2.2 \pi ||xf(x)||_2 || \xi \hat{f}(\xi) ||_2
  $$
  or 
  $$
  ||xf(x)||_2 || \xi f (\xi) ||_2 \geq (1/4 \pi) ||f||^2_2.
  $$
\end{proof}

\eject

\noindent
\textbf{A GENERALISATION TO $n$ VARIABLES }
\medskip

We replace $x$ by $x_j, \dfrac{d}{dx}$ by
$\dfrac{\partial}{\partial x_j}$. Also for any $a_j, b_j \epsilon
\varmathbb{R}$, we can replace $x_j$ and $\dfrac{\partial}{\partial
  x_j}$ by $x_j -a_j$ and $\dfrac{\partial}{\partial x_j} - b_j$
respectively. The same proof then yields:  
\begin{equation*}
  ||(x_j -a_j) f(x) ||_2 || (\xi_j -b_j)
  \hat{f}(\xi) ||_2 \geq (1/4 \pi)
  ||f||^2_2.\tag{1.23}\label{chap1:sec2:eq1.23}  
\end{equation*}\pageoriginale
  
 Let us now take $||f||_2 = 1$ and $A = (a_1, a_2, \ldots, a_n)
 \epsilon \varmathbb{R}^n$. Let $f$ be small outside a small
 nighbourhood of $A$. 
 
In this case, $||(x_j-a_j) f(x) ||_2$ will be small. Consequently, the
other factor on the left in (\ref{chap1:sec2:eq1.23}) has to be
large. That is, if the 
mass of $f$ is concentrated near one point, the mass of $\hat{f}$
cannot be concentrated near any point. 

\setcounter{rem}{23}
\begin{rem}\label{chap1:sec2:rem1.24}% Remark1.24
If we take 
$$
a_j = \int x_i |f(x)|^2 dx,b_j = \int \xi_j |\hat{f}(\xi) |^2 d \xi,
$$
then inequality (\ref{chap1:sec2:eq1.23}) is the mathematical formulation of the
\textit{ position-momentum uncertainty relation } in Quantum
Mechanics. 
 \end{rem} 

\section{Some Results From the Theory of
  Distributions}\label{chap1:sec3} %SEction 3 
 
In this section, let us recall briefly some results from the theory of
distributions. (For a more detailed treatment, see, for example \cite{7}
or \cite{8}). 
 
 In the sequel, $\mathcal{D}' (\Omega) $ will denote the \textit{
   space of distributions } on the open set $\Omega \subset
 \mathbb{R}^n$ which is the dual space of $C^{\infty}_o(\Omega)$. When
 $\Omega = \varmathbb{R}^n$, we will simply write $\mathcal{D}'$
 instead of $\mathcal{D}'(\varmathbb{R}^n)$. In the same way, $S' =
 S'(\mathbb{R}^n)$ will denote the \textit{ space of tempered
   distributions with and } $E' = E' (\varmathbb{R}^n)$ will stand for
 the \textit{space of distributions with compact support}. 
 
The value of a distribution $u \epsilon \mathcal{D}'$ at a function
$\phi \epsilon C^{\infty}_o$ will be denoted by $< u, \phi>$. If
$u$ is a locally integrable function, then $u$ defines a distribution
by $< u, \phi> = \int u(x) \phi (x) dx$. It will sometimes be\pageoriginale
convenient to write $\int u (x) \phi (x) dx$ for $< u, \phi >$, when
$u$ is an arbitrary distribution. 
  
The convergence in $\mathcal{D}'$ is the \textit{weak convergence}
defined by the following: 
 $$
 u_n,u \epsilon\mathcal{D}', u_n \to  ~\text{in means }~  < u_n, \phi
 > \to < u, \phi> ~\text{ for every }~ \phi ~\text{ in }~ C^{\infty}_o. 
 $$
 
 Let us now recall briefly \textit{ certain operations on
   distributions.}

\noindent (1.25) \qquad 
We can multiply a distribution $u \epsilon \mathcal{D}'$ by a
$C^{\infty}$ function $\phi$ to get another distribution
$\phi u$ which is defined by  
$$
< \phi u, \psi > = < u,\phi \psi > 
$$

A $C^{\infty}$ function $\psi$ is said to be \textit{tempered} if, for
every multiindex $\alpha, \partial^{\alpha} \phi $ grows at most
polynomially at $\infty$. We can multiply an $u, \epsilon S'$ by a
tempered function to get another tempered distribution. The definition
is same as in the previous case. 

\noindent (1.26) \qquad 
If $u \epsilon \mathcal{D}'$ and $f \epsilon C^{\infty}_o$, we
define the \textit{ convolution } $u * f$ by $(u *f) (x) = < u,f_x> $
where $f_x(x) = f(x-y)$. 

The function $u*f$ is $C^{\infty}$ and when $ u \epsilon E, u * f $
is in $C^{\infty}_o$. The convolution $* : \mathcal{D}' \times
C^{\infty}_o \to C^{\infty}$ can be extended to a map from
$\mathcal{D}' \times E'$ to $\mathcal{D}'$. Namely, if $u \epsilon
\mathcal{D}', v \epsilon E'$ and $\phi \epsilon C^{\infty}_o, <
u * v, \phi > = < u, \tilde{v}* \phi > $ where $\tilde{v}$ is defined
by $< \tilde{v}, \psi > = \int v(x) \psi (-x) dx$. The associative law  
$$
u * (v* w) - (u*w) * w \text{ holds for } u,v,w \epsilon \mathcal{D}'
$$
provided\pageoriginale that at most one of them does not have compact support. For
$u \epsilon S'$ and $f \epsilon S, u * f$ can also defined in
the same way and $u * f$ is a tempered $C^{\infty}$ function. 

 (1.27) \quad 
Since the Fourier transform is an isomorphism of $S$ onto $S$. and
$\int f \hat{g} = \int \hat{f}g$, the Fourier transform extends by
duality to an isomorphism of $S'$ onto $S'$. 

For $u \epsilon E' \subset S'$, we have $\hat{u}(\xi) = < u,e^{-2
  \pi i (\cdot) \cdot \xi}>$ which is an entire analytic function. 
