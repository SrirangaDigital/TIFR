\chapter{Fractional Mean Values}\label{c6}

\section{Upper and Lower Bounds for Fractional mean Values}\label{c6:sec6.1}

IN THIS CHAPTER\pageoriginale we are going to study the asymptotic
behaviour of the integral 
\begin{equation}
I_k(T,\sigma) : = \int\limits^T_0 |\zeta(\sigma + it)|^{2k} dt
\label{c6:eq6.1} 
\end{equation}
when $\frac{1}{2} \leq \sigma <1$ and $k \geq 0$ is not necessarily an
integer. If $k$ is not an integer, then it is natural to expect that
the problem becomes more difficult. In the most important case when
$\sigma =\frac{1}{2}$ we shall set $I_k (T,\frac{1}{2}) = I_k (T)$, in
accordance with the notation used in Chapter \ref{c4}. When $k$ is an
integer, we can even obtain good lower bounds for $I_k (T+H) - I_k(T)$
for a wide range of $H$, by using a method developed by
R. Balasubramanian and K. Ramachandra. This will be done in Section
\ref{c6:sec6.4}. In this section we shall use the method of
D.R. Heath-Brown, which is based on a convexity technique. Actually
this type of approach works indeed for any real $k \geq 0$, but when
$k$ is not rational it requires the Riemann hypothesis to yield sharp
results. Since $I$ prefer not to work with unproved hypotheses such as
the Riemann hypothesis is, henceforth it will be assumed that $k \geq
0$ is rational. Now we shall formulate the main result of this
section, which will be used in Section \ref{c6:sec6.2} to yield an
asymptotic formula for certain fractional mean values, and in Section
\ref{c6:sec6.3} for the study of the distribution of values of
$|\zeta(\frac{1}{2} + it)|$. This is 

\setcounter{thm}{0}
\begin{thm}\label{c6:thm6.1}
If $k \geq 0$ is a fixed rational number, then
\begin{equation}
I_k (T) = \int\limits^T_0 \left|\zeta \left(\frac{1}{2} + it \right)
\right|^{2k} dt \gg T(\log T)^{k_2}.\label{c6:eq6.2} 
\end{equation}

If $m \geq 1$\pageoriginale is an integer, then uniformly in $m$ 
\begin{equation}
T(\log T)^{1/m^2} \ll I_{1/m} (T) \ll T(\log
T)^{1/m^2}. \label{c6:eq6.3} 
\end{equation}
\end{thm}

Before we proceed to the proof of Theorem \ref{c6:thm6.1} we shall
make some preliminary remarks and prove several lemmas. 

Already in Chapter \ref{c1} we encountered the divisor function
$d_k(n)$, which represents the number of ways $n$ may be written as a
product of $k$ factors, where $k(\geq 2)$ is a fixed integer. In case
$k \geq 0$ is not an integer, we define $d_k(n)$ by 
\begin{equation}
\zeta^k (s) = \prod_p (1-p^{-s})^{-k} = \sum\limits^\infty_{n=1} d_k
(n) n^{-s} (\re s >1). \label{c6:eq6.4} 
\end{equation}
Here a branch of $\zeta^k(s)$ is defined by the relation
\begin{equation}
\zeta^k(s) = \exp (k \log \zeta(s)) = \exp \left(-k \sum\limits_p
\sum\limits^\infty_{j=1} j^{-1} p^{-js} \right) \; (\re s >
1).\label{c6:eq6.5} 
\end{equation}

Note that the above definition makes sense even if $k$ as an arbitrary
complex number. It shows that $d_k(n)$ is a multiplicative function of
$n$ for a given $k$. If $p^{\alpha}$ is an arbitrary prime power, then
from (\ref{c6:eq6.4}) we have 
\begin{equation}
d_k(p^{\alpha}) = (-1)^{\alpha} \binom{-k}{\alpha} =
\frac{k(k+1)\ldots (k+\alpha-1)}{\alpha !} = \frac{\Gamma
  (k+\alpha)}{\Gamma (k) \alpha !},\label{c6:eq6.6} 
\end{equation}
so that by multiplicativity
\begin{equation}
d_k(n) = \prod_{p^{\alpha}||n} \frac{\Gamma (k+\alpha)}{\Gamma(k)
  \alpha!}. \label{c6:eq6.7} 
\end{equation}

From (\ref{c6:eq6.6}) it follows then that always $d_k(n) \geq 0$ for
$k \geq 0$, the function $d_k(n)$ is an increasing function of $k$ for
fixed $n$, and moreover for fixed $k \geq 0$ and any $\epsilon > 0$ we
have $d_k(n) \ll_\epsilon n^{\epsilon}$. Now we state 

\setcounter{lemma}{0}
\begin{lemma}\label{c6:lem6.1}
For fixed $k \geq 0$ there exists $c_k > 0$ such that 
\begin{equation}
\left(\sigma - \frac{1}{2} \right)^{-k^2} \ll \sum\limits^N_{n=1}
d^2_k(n) n^{-2\sigma} \ll \left( \sigma -\frac{1}{2}\right)^{-k^2}
\label{c6:eq6.8} 
\end{equation}
uniformly\pageoriginale for $1/2 + c_k/\log N \leq \sigma \leq 1$ and 
\begin{equation}
\log^{k^2} N \ll \sum\limits^{N}_{n=1} d^2_k(n) n^{-1} \ll \log^{k^2}
N.\label{c6:eq6.9} 
\end{equation}

Moreover, if $k =1/m$ and $m\geq 1$ is an integer, then
(\ref{c6:eq6.9}) holds uniformly in $m$, and also (\ref{c6:eq6.8}) if
$1/2 + c / \log N \leq \sigma \leq 1$ for a suitable constant $c >0$.  
\end{lemma}

\medskip
\noindent{\textbf{PROOF OF LEMMA \ref{c6:lem6.1}.}} By taking $\sigma
=1/2 + c_k / 
\log N$ and noting that we then have $n^{-1} \ll n^{-2\sigma} \ll
n^{-1}$ for $1 \leq n \leq N$, we see that (\ref{c6:eq6.9}) follows
from (\ref{c6:eq6.8}). By induction on $\alpha$ it follows that  
{\fontsize{10}{11}\selectfont
$$
d^2_k (p^\alpha) = \left( \frac{k (k+1) \ldots (k+ \alpha -1)}{\alpha
  !}\right)^2 \leq \frac{k^2 (k^2 +1) \ldots (k^2+ \alpha -1)}{\alpha
  !} = d_{k^2} (p^\alpha), 
$$}
whence by multiplicativity $d^2_k(n) \leq d_{k^2}(n)$ for $n \geq 1$. This gives
\begin{align*}
\sum\limits^{N}_{n=1} d^2_k (n) n^{-2\sigma} & \leq
\sum\limits^N_{n=1} d_{k^2} (n) (n)^{-2\sigma} \leq
\sum\limits^{\infty}_{n=1} d_{k^2} (n) n^{-2\sigma}\\ 
&  = \zeta^{k^2} (2\sigma) \ll \left(\sigma -\frac{1}{2}
\right)^{-k^2}, 
\end{align*}
and this bound is uniform for $k \leq k_0$. To prove the lower bound
in (\ref{c6:eq6.8}) write $f(n) : = d^2_k(n) \mu^2(n)$, $\sigma =1/2 +
\delta$. Then 
\begin{align}
\sum\limits^N_{n=1} d^2_k (n) n^{-2\sigma} & \geq \sum\limits^N_{n=1}
f(n) n^{-1-2\delta} \geq \sum\limits^\infty_{n=1} f(n)n^{-1-2\delta}
\left(1-\left(\frac{n}{N}\right)^{\delta} \right)\notag\\ 
& = S (1+ 2\delta) - N^{-\delta} S (1+\delta),
\label{c6:eq6.10} 
\end{align}
where for real $s >1$
$$
S(s) : = \sum\limits^\infty_{n=1} f(n) n^{-s} = \prod_p \left(1+k^2
p^{-s}\right) = \zeta^{k^2} (s) g_k(s) 
$$
with 
$$
g_k (s) : = \exp \left( -\sum\limits_p  \left\{\log \left(1+k^2
p^{-s}\right) + k^2 \log \left(1-p^2\right) \right\}\right). 
$$

Thus $g_k(s)$ is continuous, and so it is bounded for $1 \leq s \leq
2$. If $0 < A_k \leq g_k (s) \leq B_k$\pageoriginale and $C_k \leq
(s-1) \zeta(s) \leq D_k$ for $1 \leq s \leq 2$, then we have 
$$
S(1+2\delta) - N^{-\delta} S(1+\delta) \geq A_k C^{k^2}_k (2
\delta)^{-k^2} - N^{-\delta} B_k D^{k^2}_k \delta^{-k^2}. 
$$

Since $\sigma \geq 1/2 + c_k / \log N$ we have $N^\delta \geq \exp
(c_k)$, and if $c_k$ is so large that  
$$
\exp (c_k) \geq 2 \frac{B_k}{A_k} \left(\frac{D_k}{C_k}\right)^{k^2},  
$$
then we shall have
$$
S(1+ 2 \delta) - N^{-\delta} S (1+\delta) \gg \delta^{-k^2},
$$
and consequently the lower bound in (\ref{c6:eq6.8}) will follow from
(\ref{c6:eq6.10}). For $k \leq k_0$ the argument shows that the lower
bound is also uniform, to taking $k=1/m$, $m \geq 1$ an integer, the
last part of Lemma \ref{c6:lem6.1} easily follows. 

The next two lemmas are classical convexity results from complex
function theory, in the from given by R.M. Gabriel \cite{Gabriel1},
and their proof will not be given here. 

\begin{lemma}\label{c6:lem6.2}
Let $f(z)$ be regular for $\alpha < \re z < \beta$ and continuous for
$\alpha \leq \re z \leq\beta$. suppose $f(z) \to 0$ as $|\im z| \to
\infty$ uniformly for $\alpha \leq \re z \leq \beta$. Then for $\alpha
\leq \gamma \leq \beta$ and any $q >0$ we have 
\begin{equation}
\int\limits^\infty_{-\infty} |f(\gamma + it)|^q dt \leq
\left(\int\limits^{\infty}_{-\infty} |f(\alpha + it)|^q dt
\right)^{\frac{\beta-\gamma}{\beta-\alpha}}
\left(\int\limits^\infty_{-\infty} |f(\beta + it)|^q dt
\right)^{\frac{\gamma-\alpha}{\beta-\alpha}}. 
\label{c6:eq6.11}
\end{equation}
\end{lemma}

\begin{lemma}\label{c6:lem6.3}
Let $R$ be the closed rectangle with vertices $z_0$, $\bar{z}_0$,
$-z_0$, $-\bar{z}_0$. Let $F(z)$ be continuous on $R$ and regular on
the interior of $R$. Then 
\begin{equation}
\int\limits_L |F(z)|^q |dz| \leq \left( \int\limits_{P_1} |F(z)|^q
|dz|\right)^{\frac{1}{2}} \left(\int\limits_{P_2} |F(z)|^q |dz|
\right)^{\frac{1}{2}} \label{c6:eq6.12}
\end{equation}
for any $q \geq 0$, where $L$ is the line segment from $\frac{1}{2}
(\bar{z}_0 - z_0)$ to $\frac{1}{2} (z_0 -\bar{z}_0)$, $P_1$ consists
of the three line segments connecting $\frac{1}{2} (\bar{z}_0 - z_0)$,
$\bar{z}_0$, $z_0$ and $\frac{1}{2} (z_0- \bar{z}_0)$, and $P_2$ is
the mirror image of $P_1$ in $L$. 
\end{lemma}

We shall\pageoriginale first apply Lemma \ref{c6:lem6.2} to $f(z) =
\zeta (z) \exp((z-i\tau)^2)$, $\alpha = 1 - \sigma$, $\beta = \sigma$,
$\gamma = \frac{1}{2}$, where $\frac{1}{2} \leq \sigma \leq
\frac{3}{4}$, $q = 2k > 0$ and $\tau \geq 2$. By the functional
equation $\zeta(s) = x (s) \zeta(1-s)$, $x(s) \asymp |t|^{\frac{1}{2}
  - \sigma}$ we have  
$$
\zeta(\alpha + it) \ll |\zeta (\beta + it)| (1+|t|)^{\sigma
  -\frac{1}{2}}, 
$$
whence
{\fontsize{10pt}{12pt}\selectfont
\begin{align*}
& \int\limits^\infty_{-\infty} |f(\alpha + it)|^{2k} dt \ll
  \int\limits^{\infty}_{-\infty} |\zeta(\sigma + it)|^{2k}
  (1+|t|)^{k(2\sigma-1)} e^{-2k(t-\tau)^2} dt\\ 
& \ll \left(\int\limits^{\frac{1}{2} \tau}_{-\infty} +
  \int\limits^{\infty}_{3\tau/2} \right) (1+|t|)^{2k}
  e^{-2k(t-\tau)^2} dt +\tau^{k(2\sigma -1)} \int\limits^{3\tau
    /2}_{\tau/2} |\zeta(\sigma + it)|^{2k} e^{-2k(t-\tau)^2} dt\\ 
& \ll e^{-2k\tau^2/5} + \tau^{k(2\sigma-1)}
  \int\limits^\infty_{-\infty} |\zeta(\sigma+it)|^{2k}
  e^{-2k(t-\tau)^2} dt. 
\end{align*}}

The above bound is also seen to hold uniformly for $k=1/m$, $m \leq
(\log \log T)^{\frac{1}{2}}$, which is a condition that we henceforth
assume. Since $(\beta -\gamma) / (\beta - \alpha) = (\gamma - \alpha)
/ (\beta - \alpha) = 1/2$, (\ref{c6:eq6.11}) gives then 
\begin{align*}
& \int\limits^\infty_{-\infty} \left|\zeta \left(\frac{1}{2} + it
  \right)\right|^{2k} e^{-2k(t-\tau)^2} dt\\ 
& \ll e^{-2k\tau^2/5} + \tau^{k(\sigma -\frac{1}{2})}
  \int\limits^{\infty}_{-\infty} |\zeta(\sigma + it)|^{2k}
  e^{-2k(t-\tau)^2} dt. 
\end{align*}

If we define 
$$
w(t) : = \int\limits^{2T}_T e^{-2k(t-\tau)^2} d\tau
$$
and integrate the last bound for $T \leq \tau \leq 2T$, then we obtain 

\begin{lemma}\label{c6:lem6.4}
Let $\frac{1}{2} \leq \sigma \leq \frac{3}{4}$, $k > 0$ and $T \geq
2$. Then 
\begin{equation}
J \left(\frac{1}{2} \right) \ll T^{k(\sigma -\frac{1}{2})} J(\sigma) +
e^{-k T^2/3}, \label{c6:eq6.13}
\end{equation}
where
$$
J(\sigma): = \int\limits^\infty_{-\infty} |\zeta(\sigma + it)|^{2k}
w(t) dt. 
$$

Also (\ref{c6:eq6.13}) holds uniformly for $k = 1/m$, where $m\geq 1$
is an integer such that $m \leq (\log \log T)^{\frac{1}{2}}$. 
\end{lemma}

Next we\pageoriginale take $f(z) = (z-1) \zeta(z) \exp ((z-i\tau)^2)$
(so that $f(z)$ is entire, because $z-1$ cancels the pole of
$\zeta(z)$ at $z =1$), $\gamma =\sigma$, $\alpha = 1/2$, $\beta =
3/2$, $q = 2k > 0$, $1/2 \leq \sigma \leq 3/4$, and $\tau \geq
2$. Then 
\begin{align*}
& \int\limits^\infty_{-\infty} \left| f \left(\frac{1}{2} + it
  \right)\right|^{2k} dt \ll \int\limits^{3 \tau /2}_{\tau/2} \left|f
  \left(\frac{1}{2} + it \right)\right|^{2k} dt + e^{-2k\tau^2 /5}\\ 
& \ll \tau^{2k} \int\limits^{3\tau/2}_{\tau/2} 
\left|\zeta \left(\frac{1}{2} + it \right)\right|^{2k} 
e^{-2k (t-\tau)^2} dt + e^{-2k \tau^2/5},
\end{align*}
and similarly
\begin{align*}
& \int\limits^\infty_{-\infty} \left| f \left(\frac{3}{2} + it
  \right)\right|^{2k} dt \ll \tau^{2k} \int\limits^{3\tau/2}_{\tau/2}
  \left|\zeta \left(\frac{3}{2} + it \right)\right|^{2k} e^{-2k
    (t-\tau)^2} dt + e^{-2k\tau^2/5}\\ 
& \ll \tau^{2k} \int\limits^{3\tau /2}_{\tau/2} e^{-2k (t-\tau)^2} dt
  + e^{-2k\tau^2 /5} \ll \tau^{2k}. 
\end{align*}

From (\ref{c6:eq6.11}) we conclude that
{\fontsize{10pt}{12pt}\selectfont
\begin{equation}
\int\limits^\infty_{-\infty} |f(\sigma + it)|^{2k} dt \ll \tau^{2k}
\left(\int\limits^\infty_{-\infty} \left|\zeta \left(\frac{1}{2} + it
\right)\right|^{2k}e^{-2k(t-\tau)^2} dt \right)^{\frac{3}{2} - \sigma} +
e^{-k\tau^2/3}. \label{c6:eq6.14}
\end{equation}}

But we have
\begin{align}
\int\limits^\infty_{-\infty} |\zeta(\sigma + it)|^{2k} e^{-2k
  (t-\tau)^2} dt &\ll \int\limits^{3\tau/2}_{\tau/2} |\zeta(\sigma +
it)|^{2k} e^{-2k (t - \tau)^2} dt + e^{-2k\tau^2/5}\notag \\ 
& \ll \tau^{-2k} \int\limits^{3\tau/2}_{\tau/2} |f(\sigma + it)|^{2k}
dt + e^{-2k\tau^2/5} \label{c6:eq6.15}\\
& \leq \tau^{-2k} \int\limits^\infty_{-\infty} |f(\sigma + it)|^{2k} dt
+ e^{-2k\tau^2/5}.\notag 
\end{align}

In case $k = 1/m$ we have to observe that, for $T \leq \tau \leq 2T$,  
$$
\int\limits^{3\tau/2}_{\tau/2} e^{-2(t-\tau)^2/m} dt \ll
m^{\frac{1}{2}} 
$$
instead of $\ll 1$. We combine (\ref{c6:eq6.15}) with
(\ref{c6:eq6.14}) and integrate for $T \leq \tau \leq 2T$. By using
H\"older's inequality we obtain then  

\begin{lemma}\label{c6:lem6.5}
Let\pageoriginale $\frac{1}{2} \leq \sigma \leq \frac{3}{4}$, $k>0$
and $T \geq 2$. Then 
$$
J(\sigma) \ll T^{\sigma -1/2} J \left(\frac{1}{2} \right)^{3/2 -
  \sigma} + e^{-kT^2/4}, 
$$
and if $k =1/m$, $m \geq 1$ is an integer, then uniformly for $m \leq
(\log \log T)^{\frac{1}{2}}$ we have  
$$
J(\sigma) \ll \left(m^{\frac{1}{2} }T\right)^{\sigma-\frac{1}{2}}
J\left(\frac{1}{2}\right)^{\frac{1}{2}(3-2\sigma)} + e^{-T^2 / (4m)}. 
$$
\end{lemma}

Since (\ref{c6:eq6.2}) is trivial for $k=0$, we may henceforth assume
that $k=u/v$, where $u$ and $v$ are positive coprime integers, so that
for (\ref{c6:eq6.3}) we have $k = u/v$ with $u=1$, $v = m$. In the
former case let $N=T^{\frac{1}{2}}$ and in the latter $N =
T^{\frac{3}{4}}$. We also write 
$$
S(s) : = \sum\limits_{n \leq N} d_k (n) n^{-s},\quad g(s): = \zeta^u(s) -
S^v (s). 
$$

We apply now Lemma \ref{c6:lem6.2} to the function $f(z) = g(z) \exp
(u(z-i\tau)^2)$ with $\gamma =\sigma$, $\alpha = 1/2$, $\beta = 7/8$
and $q = 2 /v$. Hence 
{\fontsize{10pt}{12pt}\selectfont
\begin{equation}
\int\limits^\infty_{-\infty} |f(\sigma + it)|^{\frac{2}{v}} dt \leq
\left(\int\limits^\infty_{-\infty} \left|f \left(\frac{1}{2} + it
\right)\right|^{\frac{2}{v}} dt \right)^{\frac{7-8\sigma}{3}}
\left(\int\limits^\infty_{-\infty} \left|f \left(\frac{7}{8} + it
\right)\right|^{\frac{2}{v}} dt \right)^{\frac{8\sigma - 4}{3}}.
\label{c6:eq6.16} 
\end{equation}}

Note that, for $\frac{1}{2} \leq \re s \leq 2$, trivial estimation gives
$$
S(s) \ll N^{\epsilon + 1 - \re s} + 1 \ll T,
$$
thus
\begin{equation}
g(s) \ll (T + |t|)^{u+v}  \quad \left(\frac{1}{2} \leq \re s \leq 2,
\; |s-1| \leq \frac{1}{10} \right).\label{c6:eq6.17} 
\end{equation}

It follows that
\begin{align}
\int\limits^\infty_{-\infty} \left|f \left(\frac{7}{8} + it
\right)\right|^{\frac{2}{v}} dt & = \int\limits^{3\tau/2}_{\tau/2} 
\left|f\left(\frac{7}{8} + it \right)\right|^{\frac{2}{v}} dt\notag\\  
& \quad +  O \left( \left(\int\limits^{\frac{1}{2} \tau}_{-\infty} +
\int\limits^\infty_{\frac{3\tau}{2}} \right) (T +|t|)^{2+2k} e^{-2k
  (t-\tau)^2} dt\right)\notag\\ 
& = \int\limits^{3\tau/2}_{\tau/2} \left|f \left(\frac{7}{8} + it
\right)\right|^{\frac{2}{v}} dt + O \left(T^{2+2k}
e^{-k\tau^2/3}\right). \label{c6:eq6.18} 
\end{align}

At this point we shall use Lemma \ref{c6:lem6.3}, which allows us to
avoid the singularity of $\zeta(s)$ at $s=1$. We take $z_0 =
\dfrac{3}{8} + \dfrac{1}{2} i\tau$, $F(z) = f\left(z - \frac{7}{8} -
i\tau \right)$\pageoriginale and $q =\dfrac{2}{v}$. For the integrals
in (\ref{c6:eq6.12}) we then have  
$$
\int\limits_L |F(z)|^q |dz| = \int\limits^{3\tau/2}_{\tau/2} \left|f 
\left(\frac{7}{8} + it \right)\right|^{\frac{2}{v}} dt 
$$
and 
\begin{align}
& \int\limits_{P_1} |F(z)|^q |dz|\label{c6:eq6.19}\\
& = \int\limits^{3\tau/2}_{\tau/2} \left|f \left(\frac{5}{4} + it
  \right)\right|^{\frac{2}{v}} dt + \int\limits^{5/4}_{7/8} \left\{
  \left|f \left(\eta + \frac{1}{2} i \tau \right)\right|^{\frac{2}{v}}
  + \left|f  \left(\eta + \frac{3 i \tau}{2}
  \right)\right|^{\frac{2}{v}}\right\} d\eta. \notag 
\end{align}

By (\ref{c6:eq6.17}) we have
$$
f\left(\eta + \frac{1}{2} i \tau \right) \ll (T + \tau)^{u+v}
e^{-\frac{1}{4} u\tau^2} 
$$
and similarly for $f\left(\eta+ \dfrac{3 i \tau}{2} \right)$. Thus the
second integral on the right of (\ref{c6:eq6.19}) is $\ll T^{2 + 2k}
\exp (-k\tau^2/3) $ and similarly 
$$
\int\limits_{P_2} |F(z)|^q |dz| = \int\limits^{3 \tau/2}_{\tau/2}
\left|f \left(\frac{1}{2} + it \right)\right|^{\frac{2}{v}} dt + O
\left(T^{2+2k} e^{-k \tau^2/3}\right).  
$$

Lemma \ref{c6:lem6.3} therefore yields
\begin{align*}
& \int\limits^{3\tau /2}_{\tau/2} \left|f \left(\frac{7}{8} + it
  \right)\right|^{\frac{2}{v}} dt\\ 
& \ll \left(\int\limits^{\infty}_{-\infty} \left|f \left(\frac{1}{2} +
  it \right)\right|^{\frac{2}{v}} dt \right)^{\frac{1}{2}}
  \left(\int\limits^{\infty}_{-\infty} \left|f \left(\frac{5}{4} + it
  \right)\right|^{\frac{2}{v}} dt \right)^{\frac{1}{2}} + T^{2+2k}
  e^{-k\tau^2/7}, 
\end{align*}
since by (\ref{c6:eq6.17})
$$
\int\limits^{3\tau /2}_{\tau/2} \left\{ \left|f \left(\frac{1}{2} + it 
\right)\right|^{\frac{2}{v}} dt + \left|f \left(\frac{5}{4} + it
\right)\right|^{\frac{2}{v}}  \right\} dt \ll (T + \tau)^{2+2k}. 
$$

From (\ref{c6:eq6.16}) and (\ref{c6:eq6.18}) we now deduce 
\begin{align*}
\int\limits^\infty_{-\infty} |f (\sigma + it)|^{\frac{2}{v}} dt &\ll
\left(\int\limits^\infty_{-\infty} \left|f\left(\frac{1}{2} + it
\right)\right|^{\frac{2}{v}} dt \right)^{\frac{5-4\sigma}{3}} \left(
\int\limits^{\frac{3\tau}{2}}_{\frac{1}{2} \tau} \left|f \left(\frac{5}{4}
+ it \right)\right|^{\frac{2}{v}} dt \right)^{\frac{4\sigma -2}{3}}\\ 
&\quad + \left(\int\limits^\infty_{-\infty} \left|f \left(\frac{1}{2}
+ it \right)\right|^{\frac{2}{v}} dt \right)^{\frac{7-8\sigma}{3}}
\left(T^{2+2k} e^{-k\tau^2/7} \right)^{\frac{8\sigma -4}{3}}. 
\end{align*}

We integrate\pageoriginale this for $T \leq \tau \leq 2T$ and write
$$
K(\sigma) : = \int\limits^\infty_{-\infty} |g(\sigma +
it)|^{\frac{2}{v}} w (t) dt, 
$$
whence
\begin{align}
K(\sigma) & \ll K \left(\frac{1}{2} \right)^{\frac{5-4\sigma}{3}}
\left\{ \int\limits^{2T}_T \int\limits^{3 \tau/2}_{\tau/2}\left|g
\left(\frac{5}{4} + it \right)\right|^{\frac{2}{v}}  e^{-2k
  (t-\tau)^2}d 
\tau dt\right\}^{\frac{4\sigma -2}{3}}\label{c6:eq6.20}\\ 
&\quad + K \left(\frac{1}{2} \right)^{\frac{7-8\sigma}{3}}
\left(e^{-kT^2/8}\right)^{\frac{8 \sigma - 4}{3}}\notag \\ 
&\ll K \left(\frac{1}{2} \right)^{\frac{5-4\sigma}{3}}
\left\{\int\limits^{3T}_{\frac{1}{2} T} \left|g \left(\frac{5}{4} + it
\right)\right|^{\frac{2}{v}} dt\right\}^{\frac{4\sigma-2}{3}} + K
\left(\frac{1}{2} \right)^{\frac{7-8\sigma}{3}} e^{-\frac{kT^2(2\sigma 
    -1)}{6}}.\notag 
\end{align}

From the definition of $g(s)$ we have
$$
g(s) = \sum\limits_{n > N} a_n n^{-s} \quad (\re s >1)
$$
with $0 \leq a_n \leq d_u(n)$, so that $a_n \leq 1$ if $u=1$, in
particular for $k=1/m$. By the mean value theorem for Dirichlet
polynomials (see (\ref{c1:eq1.15})) we find that 
\begin{align*}
& \int\limits^{3T}_{\frac{1}{2}T} \left|g \left(\frac{5}{4} + it
  \right)\right|^2 dt \ll T \sum\limits_{n > N} a^2_n n^{-5/2} +
  \sum\limits_{n > N} a^2_n n^{-3/2}\\ 
& \ll TN^{-3/2} (\log N)^{u^2-1} + N^{-\frac{1}{2}} (\log N)^{u^2-1}
  \ll TN^{-3/2} (\log N)^{u^2-1} 
\end{align*}
since $N \leq T$ and 
$$
\sum\limits_{n \leq x} d^2_u (n) \ll x (\log x)^{u^2-1}.
$$

Therefore by H\"older's inequality for integrals we obtain
$$
\int\limits^{3T}_{\frac{1}{2} T} \left|g \left(\frac{5}{4} + it
\right)\right|^{\frac{2}{v}} dt \ll T^{1-\frac{1}{v}}
\left(TN^{-\frac{3}{2}} \log^{u^2-1} T\right)^{\frac{1}{v}} =
TN^{-\frac{3}{2v}} (\log T)^{\frac{u^2-1}{v}}. 
$$

From (\ref{c6:eq6.20}) we then obtain

\begin{lemma}\label{c6:lem6.6}
Let\pageoriginale $\frac{1}{2} \leq \sigma \leq \frac{3}{4} $, $T \geq
2$. If $k=u/v$, where $u$ and $v$ are positive coprime integers, then 
$$
K(\sigma) \ll K \left(\frac{1}{2} \right)^{\frac{5-4\sigma}{3}}
\left(TN^{-\frac{3}{2v}} (\log T)^{k-\frac{1}{v}}
\right)^{\frac{4\sigma-2}{3}} + K \left(\frac{1}{3}
\right)^{\frac{7-8\sigma}{3}} e^{-\frac{kT^2 (2\sigma -1)}{6}}, 
$$
where
$$
K(\sigma) = \int\limits^\infty_{-\infty} |g(\sigma +
it)|^{\frac{2}{v}} w (t) dt. 
$$

If $k =1 / m$, $m \geq 1$ is an integer, then uniformly for $m \leq
(\log \log T)^{\frac{1}{2}}$ we have  
$$
K(\sigma) \ll K \left(\frac{1}{2} \right)^{\frac{5-4\sigma}{3}}
\left(m^{\frac{1}{2}} TN^{-\frac{3}{2m}} \right)^{\frac{4\sigma -
    2}{3}} + K \left(\frac{1}{2} \right)^{\frac{7-8\sigma}{3}}
e^{-\frac{T^2 (2\sigma-1)}{6m}}. 
$$
\end{lemma}

\medskip
\noindent{\textbf{PROOF OF \ref{c6:thm6.1}.}} As a companion to the
integrals $J(\sigma)$ and $K(\sigma)$ we define 
$$
L(\sigma) : = \int\limits^\infty_{-\infty} |S(\sigma + it)|^2 w(t) dt
\quad \left(\frac{1}{2} \leq \sigma \leq\frac{3}{4}\right).  
$$

To estimate $L(\sigma)$ note that
$$
w(t) = \int\limits^{2T}_T e^{-2k(t-\tau)^2} d\tau \ll \exp \left(-(T^2
+ t^2) k/18\right) 
$$
for $t \leq 0$ or $t \geq 3T$. Since $S(\sigma + it) \ll T$ we have 
$$
L(\sigma) = \int\limits^{3T}_0 |S(\sigma + it)|^2 w(t) dt + O(1). 
$$

Moreover $w(t) \ll 1$ for all $t$, and $w(t) \gg 1$ for $4T / 3 \leq t
\leq 5T/3$. Thus the mean value theorem for Dirichlet polynomials
yields 
$$
\int\limits^{3T}_0 |S(\sigma + it)|^2 dt = \sum\limits_{n \leq N}
d^2_k (n) n^{-2\sigma} (3T+O(n))  \ll T \sum\limits_{n \leq N}
d^{2}_k(n) n^{-2\sigma} 
$$ 
and
$$
\int\limits^{5T/3}_{4T/3} |S(\sigma + it)|^2 dt = \sum\limits_{n \leq
  N} d^2_k (n) n^{-2 \sigma} \left(\frac{T}{3} +  O(n)\right) \gg T
\sum\limits_{n\leq N} d^2_k (n) n^{-2\sigma}. 
$$

Hence\pageoriginale we may deduce from (\ref{c6:eq6.8}) of Lemma
\ref{c6:lem6.1} that 
\begin{equation}
T \left(\sigma -\frac{1}{2} \right)^{-k^2} \ll L(\sigma) \ll T
\left(\sigma -\frac{1}{2} \right)^{-k^2}\label{c6:eq6.21} 
\end{equation}
for $1/2 + c_k / \log T \leq \sigma \leq 3/4$, and from
(\ref{c6:eq6.9}) that  
\begin{equation}
T(\log T)^{k^2} \ll L \left(\frac{1}{2} \right) \ll T(\log
T)^{k^2}. \label{c6:eq6.22} 
\end{equation}

In case $k = 1/m$ we obtain, as in the previous discussion, an extra
factor $m^{\frac{1}{2}}$, and (\ref{c6:eq6.21}) and (\ref{c6:eq6.22})
become, uniformly for\,$m \leq\!(\log \log T)^{\frac{1}{2}}$,  
\begin{equation}
m^{\frac{1}{2}} T \left(\sigma -\frac{1}{2} \right)^{-1/m^2} \ll
L(\sigma) \ll m^{\frac{1}{2}} T\left(\sigma -\frac{1}{2}
\right)^{-1/m^2}\label{c6:eq6.23} 
\end{equation}
for $1/2+c/\log N \leq \sigma \leq 3/4$, and 
\begin{equation}
  m^{\frac{1}{2}} T (\log T)^{1/m^2} \ll L\left(\frac{1}{2}\right) \ll 
  m^{\frac{1}{2}} T (\log T)^{1/m^2}. \label{c6:eq6.24}
\end{equation}

We trivially have 
\begin{align*}
& |S^v (s)|^{2/v} = |\zeta^{u} (s) - g(s)|^{2/v}\\
& \leq (2\max (|\zeta^{\mu} (s)|, |g(s)|))^{2/v} \ll |\zeta(s)|^{2k} +
|g(s)|^{2/v}, 
\end{align*}
whence
\begin{equation}
L(\sigma) \ll J(\sigma) + K(\sigma).\label{c6:eq6.25}
\end{equation}

Similarly
\begin{equation}
J(\sigma) \ll L(\sigma) + K(\sigma) \label{c6:eq6.26}
\end{equation}
and 
\begin{equation}
K \left(\frac{1}{2} \right) \ll L \left(\frac{1}{2} \right) + J
\left(\frac{1}{2} \right).\label{c6:eq6.27} 
\end{equation}

In case $K(\frac{1}{2}) \leq T$ we have from (\ref{c6:eq6.25}) (with
$\sigma =\frac{1}{2}$) and (\ref{c6:eq6.22})  
\begin{equation}
T(\log T)^{k^2} \ll J \left(\frac{1}{2}
\right)\label{c6:eq6.28} 
\end{equation}

Similarly (\ref{c6:eq6.26}) and (\ref{c6:eq6.22}) show that
$K\left(\dfrac{1}{2} \right) \leq T$ implies  
\begin{equation}
J \left(\frac{1}{2} \right) \ll T (\log
T)^{k^2}. \label{c6:eq6.29} 
\end{equation}\pageoriginale

Thus we shall assume $K(\frac{1}{2}) \geq T$ and show that
(\ref{c6:eq6.28}) holds, and for $k = 1/m$ (\ref{c6:eq6.29}) also
holds (with the additional factor $m^{\frac{1}{2}}$). From these
estimates Theorem \ref{c6:thm6.1} will be deduced. From Lemma
\ref{c6:lem6.6} we have, assuming $K(\dfrac{1}{2}) \geq T$, 
{\fontsize{10}{12}\selectfont
\begin{align}
K(\delta) &\ll K \left(\frac{1}{2} \right) \left\{K \left(\frac{1}{2}
\right)^{\frac{2-4\sigma}{3}} \left(TN^{-\frac{3}{2v}} (\log
T)^{k-\frac{1}{v}} \right)^{\frac{4\sigma-2}{3}} + K \left(\frac{1}{2}
\right)^{\frac{4-8\sigma}{3} } e^{-\frac{kT^2 (4\sigma -3)}{12}}
\right\}\notag\\ 
&\leq K \left(\frac{1}{2} \right) \left\{ \left (N^{-\frac{3}{2v}}
(\log T)^{k-\frac{1}{v}} \right)^{\frac{4\sigma - 2}{3}} + \left(T^{-2}
e^{-\frac{1}{4}k T^2}\right)^{\frac{4\sigma -2}{3}} \right\} 
\label{c6:eq6.30}\\
&\ll K \left(\frac{1}{2} \right) N^{\frac{1-2\sigma}{v}} (\log
T)^{(k-\frac{1}{v})(\frac{4\sigma -2}{3})}.\notag  
\end{align}}

Now we turn  to the proof of (\ref{c6:eq6.2}), noting that the proof
of (\ref{c6:eq6.3}) is based on very similar ideas. From
(\ref{c6:eq6.25}), (\ref{c6:eq6.27}) and (\ref{c6:eq6.30}) we have,
since now $N = T^{\frac{1}{2}}$ and $k = u/v $, 
$$ 
L(\sigma) \ll J(\sigma) + \left(L \left(\frac{1}{2} \right) + J
\left(\frac{1}{2} \right) \right) \left(T \log^B
T\right)^{\frac{1-2\sigma}{4v}},  
$$
where $B = (8-8u)/3$. Thus either 
\begin{equation}
L(\sigma) \ll L \left(\frac{1}{2} \right) \left(T \log^B
T\right)^{\frac{1-2\sigma}{4v}} \label{c6:eq6.31} 
\end{equation}
or 
\begin{equation}
L(\sigma) \ll J \left(\frac{1}{2} \right) \left(T \log^B
T\right)^{\frac{1-2\sigma}{4v}} + J(\sigma), 
\label{c6:eq6.32}
\end{equation}
and by using Lemma \ref{c6:lem6.5} the latter bound gives
\setcounter{equation}{31}
\begin{equation}
L(\sigma ) \ll T^{\sigma-\frac{1}{2}} J \left(\frac{1}{2}
\right)^{\frac{3}{2}-\sigma} + (T \log^B T)^{\frac{1-2\sigma}{4v}} J
\left(\frac{1}{2} \right) + e^{-\frac{1}{4} kT^2}. \label{add-c6:eq6.32}
\end{equation}

Write $\sigma =\dfrac{1}{2} + \dfrac{\eta}{\log T}$, where $\eta >0$
is a parameter. Then (\ref{c6:eq6.21}), (\ref{c6:eq6.31}) and
(\ref{c6:eq6.22}) yield, for some constant $c(k) >0$, 
$$
T \left(\sigma -\frac{1}{2} \right)^{-k^2} \leq c(k) T (\log T)^{k^2}
e^{-\eta / (2v)}. 
$$

Hence\pageoriginale
$$
e^{-\frac{\eta}{2v}} \leq \eta^{k^2} c(k),
$$
which is false if $\eta = \eta(k)$ is sufficiently large. Thus for
this value of $\eta$ (\ref{c6:eq6.32}) holds, and using
(\ref{c6:eq6.21}) it gives 
$$ 
T \log^{k^2} T \ll L(\sigma) \ll J \left(\frac{1}{2}
\right)^{\frac{1}{2} (3-2\sigma)} + J \left(\frac{1}{2} \right) + 1, 
$$
and (\ref{c6:eq6.28}) follows. But recall that $w(t) \ll 1$ for all
$t$ and  
$$
w(t) \ll \exp (-k(t^2 + T^2)/18)
$$
for $t \leq 0$ and $t \geq 3T$. Thus
$$
J \left(\frac{1}{2} \right) \ll I_k(3T) +
\left(\int\limits^0_{-\infty} + \int\limits^{\infty}_{3T} \right)
e^{-k(t^2 + t^2)/20} dt \ll I_k(3T) + e^{-kT^2/20}, 
$$
so that (\ref{c6:eq6.2}) follows from (\ref{c6:eq6.28}). Similarly for
the case $k=1/m$, $m \geq 1$ an integer in (\ref{c6:eq6.3}) we obtain
the result from  
\begin{equation}
m^{\frac{1}{2}} T(\log T)^{1/m^2} \ll J \left(\frac{1}{2} \right) \ll
m^{\frac{1}{2}}  T  (\log T)^{1/m^2}\label{c6:eq6.33} 
\end{equation}
because in this case $w(t) \ll m^{\frac{1}{2}}$ always, and $w(t) \gg
m^{\frac{1}{2}}$ for $4 T /3 \leq t \leq 5T/3$. Thus (\ref{c6:eq6.33})
gives in the case of the upper bound 
$$ 
m^{\frac{1}{2}} \left(I_{1/m} \left(\frac{5T}{3} \right) - I_{1/m}
\left(\frac{4 T}{3} \right) \right) \ll J \left(\frac{1}{2} \right)
\ll m^{\frac{1}{2}} T (\log T)^{1/m^2}. 
$$

Replacing $T$ by $(4/5)^{n}T$ and summing over $n$ we obtain the upper
bound in (\ref{c6:eq6.3}), and the lower bound is proved as in the
previous case. One obtains (\ref{c6:eq6.33}) from the corresponding
estimates of the previous case and chooses $\sigma =\dfrac{1}{2} +
\dfrac{Dm}{\log T}$, where $D >0$ is a suitable large constant. 

In the analysis concerning the case $k = 1/m$ we have made the
assumption that $m\leq (\log \log T)^{\frac{1}{2}}$. Therefore it
remains to prove (\ref{c6:eq6.3}) when $m \geq m_0 = [(\log \log
  T)^{\frac{1}{2}}]$. In that case 
$$ 
I_{1/m} (T) = \int\limits^T_0 \left|\zeta \left(\frac{1}{2} + it
\right)\right|^{2/m} dt \geq T + \circ (T) \gg T(\log T)^{1/m^2} 
$$
for\pageoriginale $m \geq m_0$ and $T \to \infty$. This follows from
Lemma \ref{c6:lem6.7} of the next section, whose proof is independent
of Theorem \ref{c6:thm6.1}. To deal with the upper bound, denote by
$A_1$ the set of numbers $t \in [0,T]$ such that $|\zeta(\frac{1}{2} +
it)| \leq 1$, and let $A_2$ be the complement of $A_1$ in the interval
$[0,T]$. Then 
\begin{align*}
& I_{1/m} (T) = \left(\int\limits_{A_1} + \int\limits_{A_2} \right)
  \left|\zeta \left(\frac{1}{2} + it \right)\right|^{2/m} dt \leq T +
  \int\limits_{A_2} \left|\zeta \left(\frac{1}{2} + it
  \right)\right|^{2/m_\circ} dt\\ 
& \leq T + \int\limits^T_0 \left|\zeta \left(\frac{1}{2} + it
  \right)\right|^{2/m_\circ} dt \ll T + T(\log T)^{m^{-2}_0} \ll T
  (\log T)^{m^{-2}}, 
\end{align*}
since the upper bound in (\ref{c6:eq6.3}) holds for $m =m_0$.

\section{An Asymptotic formula for mean values}\label{c6:sec6.2}

In this section we shall consider the limit
\begin{equation}
\lim\limits_{T \to \infty} \frac{1}{T} \int\limits^T_0 \left|\zeta
\left(\frac{1}{2} + it \right)\right|^\lambda dt =1,
\label{c6:eq6.34} 
\end{equation}
or equivalently
\begin{equation}
\int\limits^T_0 \left|\zeta \left(\frac{1}{2} + it
\right)\right|^\lambda dt = T + \circ (T) \quad (T \to \infty)
\label{c6:eq6.35} 
\end{equation}
when $\lambda \geq 0$. If $\lambda$ is fixed, then (\ref{c6:eq6.34})
(or (\ref{c6:eq6.35})) cannot hold in view of the lower bound in
Theorem \ref{c6:thm6.1}. On the other hand, (\ref{c6:eq6.34}) is
trivial if $\lambda=0$. Therefore one feels that for certain $\lambda
= \lambda(T)$, tending sufficiently quickly to zero as $T \to \infty$,
(\ref{c6:eq6.34}) will hold. It seems interesting to determine
precisely the range of $\lambda = \lambda(T)$ for which
(\ref{c6:eq6.35}) holds. This is achieved in 

\begin{thm}\label{c6:thm6.2}
If $\psi (T)$ denotes an arbitrary positive function such that $\Psi
(T) \to \infty$ as $T \to \infty$, then for  
$$
0 \leq \lambda \leq (\Psi (T) \log \log T)^{-\frac{1}{2}}
$$
we have
\begin{equation}
\int\limits^T_0 \left|\zeta \left(\frac{1}{2} + it
\right)\right|^\lambda dt = T + \circ (T) \quad (T \to
\infty).\label{c6:eq6.36} 
\end{equation}

Moreover\pageoriginale (\ref{c6:eq6.36}) cannot hold for $\lambda \geq
C (\log \log T)^{-\frac{1}{2}}$ and a suitable $C>0$. 
\end{thm}

To prove Theorem \ref{c6:thm6.2} we shall prove the corresponding
upper and lower bound estimates. The latter is contained in the
following lemma, which was already used in the discussion of the lower
bound in Theorem \ref{c6:thm6.1} in Section \ref{c6:sec6.1}. 

\begin{lemma}\label{c6:lem6.7}
For any $\lambda \geq 0$
\begin{equation}
\int\limits^T_0 \left|\zeta \left(\frac{1}{2} +
it\right)\right|^\lambda dt \geq T + \circ (T) \quad (T \to \infty)
. \label{c6:eq6.37} 
\end{equation}
\end{lemma}

\medskip
\noindent{\textbf{PROOF OF LEMMA \ref{c6:lem6.7}.}}
Let, as usual, $N(\sigma, T)$ denote the number of complex zeros $\rho
= \beta + i\gamma$ of $\zeta(s)$ such that $\beta \geq \sigma$, $0 <
\gamma \leq T$. From J.E. Littlewood's classical lemma (see Section
9.1 of E.C. Titchmarsh \cite{Titchmarsh1}) on the zeros of an analytic
function in a rectangle, it follows that 
\begin{align*}  
2\pi \int\limits_1{\sigma_0} N (\sigma, T) d \sigma & =
\int\limits^T_0 \log |\zeta(\sigma_0 + it)| dt  - \int\limits^T_0
\log |\zeta(2+it)| dt\\ 
&\quad + \int\limits^{2}_{\sigma_0} \arg \zeta(\sigma + iT) d \sigma + 
K(\sigma_0), 
\end{align*}
where $K(\sigma_0)$ is independent of $T$, and $\frac{1}{2} \leq
\sigma_0 \leq 1$. However, uniformly for $\sigma \geq \frac{1}{2}$ we
have $\arg \zeta(\sigma + it) \ll \log t$ if $t$ is not the ordinate
of a zero of $\zeta(s)$. For $\sigma >1$ we have  
$$
\log \zeta(s) = \sum\limits_p \sum\limits^\infty_{m=1} m^{-1} p^{-ms}, 
$$
hence with $|\Lambda_1(n)| \leq 1$ we have
$$
\int\limits^T_0 \log |\zeta(2 + it)| dt = \re \left\{
\sum\limits^\infty_{n=2} \Lambda_1(n) n^{-2} \frac{n^{-iT} -1}{\log n}
\right\} = O(1). 
$$

Therefore we obtain
\begin{equation}
2 \pi \int\limits^{1}_{\sigma_0} N(\sigma, T) d \sigma = \int\limits^T_0 \log |\zeta(\sigma_0 + it)|dt + O(\log T).\label{c6:eq6.38}
\end{equation}

Since\pageoriginale $N (\sigma, T) \geq 0$ for all $\sigma$ and $T$,
(\ref{c6:eq6.38}) gives with $\sigma_0 = \frac{1}{2}$ 
$$
0 \leq 2 \pi \int\limits^1_{\frac{1}{2}} N(\sigma, T) d \delta =
\int\limits^T_0 \log \left|\zeta\left(\frac{1}{2} + it\right)\right|dt
+ O(\log T),  
$$
hence for a suitable $C > 0$
\begin{equation}
\int\limits^T_0 \log \left|\zeta \left(\frac{1}{2} + it \right)\right|
dt \geq - C\log T.\label{c6:eq6.39} 
\end{equation}

Now recall that, if $a < b$, $f(t) \geq 0$ for $a \leq t \leq b$, $f
\in C [a,b]$, then  
\begin{equation}
\frac{1}{b-a} \int\limits^b_a \log f(t) dt \leq \log
\left(\frac{1}{b-a} \int\limits^b_a f(t) dt
\right),\label{c6:eq6.40} 
\end{equation}
which is an easy consequence of the inequality between the arithmetic
and geometric means of nonnegative numbers. Hence, for $\lambda > 0$
and $f(t) = |\zeta(\frac{1}{2} + it)|$, (\ref{c6:eq6.40}) gives 
\begin{align*}
\int\limits^T_0 \log \left|\zeta \left(\frac{1}{2} + it \right)\right|
dt &= \frac{1}{\lambda} \int\limits^T_0 \log \left|\zeta
\left(\frac{1}{2} + it \right)\right|^{\lambda} dt\\
&\leq \frac{T}{\lambda} \log \left(\frac{1}{T}
\int\limits^T_0 \left|\zeta \left(\frac{1}{2} + it
\right)\right|^\lambda dt \right), 
\end{align*}
and so using (\ref{c6:eq6.39}) we obtain
$$
\frac{1}{T} \int\limits^T_0 \left|\zeta \left(\frac{1}{2} + it
\right)\right|^{\lambda} dt \geq e^{-\frac{\lambda C \log T}{T}} \geq 
e^{-\frac{2C\log T}{T}} = 1 + O \left(\frac{\log T}{T} \right) 
$$
for $0 < \lambda \leq 2$. For $\lambda = 0$ (\ref{c6:eq6.37}) is
trivial, for $\lambda \geq 2$ it follows by H\"older's inequality for
integrals and the weak bound 
$$
\int\limits^{T}_0 \left|\zeta \left(\frac{1}{2} + it \right)\right|^2
dt \gg T \log T. 
$$

\medskip
\noindent{\textbf{PROOF OF THEOREM \ref{c6:thm6.2}.}} We shall show
that for  
$$
0 \leq \lambda \leq (\psi (T) \log \log T)^{-\frac{1}{2}}
$$
we have
\begin{equation}
\int\limits^T_0 \left|\zeta \left(\frac{1}{2} + it
\right)\right|^\lambda dt \leq T + \circ (T) \quad (T \to
\infty). \label{c6:eq6.41} 
\end{equation}

Take $m = [(\log \log T)^{\frac{1}{2}}]$. Then by H\"older's
inequality and the upper bound in\pageoriginale (\ref{c6:eq6.3}) we
have, for some $C_1 > 0$, 
\begin{align*}
& \int\limits^T_0 \left|\zeta \left(\frac{1}{2} + it
  \right)\right|^\lambda dt \leq \left(\int\limits^T_0 \left|\zeta
  \left(\frac{1}{2} + it \right)\right|^{2/m} dt \right)^{\frac{1}{2}
    m \lambda} T^{1-\frac{1}{2} m\lambda}\\ 
& \leq \left(C_1 T (\log T)^{1/m^2}\right)^{\frac{1}{2} m\lambda}
  T^{1-\frac{1}{2}m\lambda} = TC^{\frac{1}{2} m\lambda}_1 (\log
  T)^{\lambda/(2m)} = T + \circ (T). 
\end{align*}

Namely 
$$
0 \leq \lambda / (2m) \leq \left(\frac{1}{2} + \circ (1) \right)
\psi^{-\frac{1}{2}} (T) (\log \log T)^{-1}, 
$$
hence, as $T \to \infty$,
$$
(\log T)^{\lambda/(2m)} = 1 + \circ (1),\quad C^{\frac{1}{2} m \lambda}_1
= 1+\circ (1). 
$$

This establishes (\ref{c6:eq6.41}) and in view of (\ref{c6:eq6.37})
proves the first part of Theorem \ref{c6:thm6.2}. 

To obtain that (\ref{c6:eq6.36}) cannot hold for $\lambda > C(\log
\log T)^{-\frac{1}{2}}$ and a suitably chosen $C$, we use the lower
bound in (\ref{c6:eq6.3}) with 
$$
m = \left[C^{-1}_3 (\log \log T)^{-\frac{1}{2}}\right] \quad (C_3 > 0 ).
$$

This gives with some $C_2 > 0$
\begin{align}
\int\limits^T_0 \left|\zeta \left(\frac{1}{2} + it
\right)\right|^{2/m} dt & \geq C_2 T (\log T)^{1/m^2} = C_2 T \exp
\left(C^2_3 + \circ (1)\right)\label{c6:eq6.42} \\ 
& \geq C_2 T \exp \left(\frac{1}{2} C^2_3 \right) \geq 2T\notag
\end{align}
for $C_2 \geq (2\log (2/C_2))^{\frac{1}{2}}$ if $C_2 <2 $, and the
bound given by (\ref{c6:eq6.42}) is trivial otherwise. Hence as $T \to
\infty$ 
\begin{align*}
\lambda_0 & = \frac{2}{m} = 2 /\left[C^{-1}_3 (\log \log
  T)^{\frac{1}{2}}\right] = (2C_3 + \circ (1)) (\log \log
T)^{-\frac{1}{2}}\\ 
&  \quad 3 C_3 (\log \log T)^{-\frac{1}{2}} = \lambda_1,
\end{align*}
say. But if for some $\lambda_0 > 0$
$$
\int\limits^T_0 \left|\zeta \left(\frac{1}{2} + it \right)
\right|^{\lambda_0} dt \geq w T,  
$$
then for $\lambda > \lambda_0$ H\"older's inequality gives 
$$
 2T \leq \int\limits^T_0 \left|\zeta \left(\frac{1}{2} + it
 \right)\right|^{\lambda_0} dt \leq \left(\int\limits^T_0 \left|\zeta
 \left(\frac{1}{2} + it \right)\right|^{\lambda} dt
 \right)^{\lambda_0/\lambda} T^{1-\lambda_0/\lambda}. 
$$\pageoriginale
and so
$$
\int\limits^T_0 \left|\zeta \left(\frac{1}{2} + it
\right)\right|^\lambda dt \geq 2^{\lambda/\lambda_0} T \geq 2T. 
$$

Therefore (\ref{c6:eq6.36}) cannot hold for $\lambda > C (\log \log
T)^{-\frac{1}{2}}$, $C = 3 C_3$, where $C_3$ is as above. This
completes the proof of Theorem \ref{c6:thm6.2}. 

\section{The values Distribution on the Critical line}\label{c6:sec6.3}

From (\ref{c6:eq6.3}) we may deduce some results on the distribution
of values of $|\zeta(\frac{1}{2} + it)|$. The order of
$|\zeta(\frac{1}{2} + it)|$ remains, of course, one of the most
important unsolved problems of analytic number theory. But the
following two theorems, due to M. Jutila, show that the corresponding
``statistical'' problem may be essentially solved. 

\begin{thm}\label{c6:thm6.3}
Let $T \geq 2$, $1 \leq V \leq \log T$, and denote by $M_T(V)$ the set
of numbers $t \in [0,T]$ such that $|\zeta(\frac{1}{2} + it)| \geq
V$. Then the measure of the set $M_T(V)$ satisfies 
\begin{equation}
\mu(M_T (V)) \ll T \exp \left\{ -\frac{\log^2 V}{\log \log T} \left(1
+ O \left( \frac{\log V}{\log \log T}\right) \right)\right\}
\label{c6:eq6.43} 
\end{equation}
and also
\begin{equation}
\mu (M_T (V)) \ll T \exp \left(-c\frac{\log^2 V}{\log \log T} \right)
\label{c6:eq6.44} 
\end{equation}
for some constant $c >0$.
 \end{thm}

\begin{coro*}
Let $\Psi (T)$ be any positive function such that $\Psi (T) \to
\infty$ as $T \to \infty$. Then 
$$
\left|\zeta \left(\frac{1}{2} + it \right)\right| \leq \exp 
\left(\Psi (T) (\log \log T)^{\frac{1}{2}}\right) 
$$
for the numbers $t \in [0,T]$ which do not belong to an exceptional
set of measure $\circ(T)$. 
\end{coro*}

\begin{thm}\label{c6:thm6.4}
There\pageoriginale exist positive constants $a_1, a_2$ and $a_3$ such
that for $T \geq 10$ we have 
$$
\exp \left(a_1 (\log \log T)^{\frac{1}{2}}\right) \leq \left|\zeta
\left(\frac{1}{2} + it \right)\right| \leq \exp \left(a_2 (\log \log
T)^{\frac{1}{2}}\right) 
$$
in a subset of measure at least $a_3T$ of the interval $[0,T]$.
\end{thm}

\medskip
\noindent{\textbf{PROOF OF THEOREM \ref{c6:thm6.3}.}} It is enough to
restrict $V$ to the range 
$$
(\log \log T)^{\frac{1}{2}} \leq \log V \leq \log \log T. 
$$

Namely for $\log V < (\log \log T)^{\frac{1}{2}}$ the $\exp (\ldots)$
terms in (\ref{c6:eq6.43}) and (\ref{c6:eq6.44}) are bounded, so the
bounds in question both reduce to $\mu(M_T (V))\break \ll T$, which is
trivial, and $\log V \leq \log \log T$ is equivalent to our assumption
that $V \leq \log T$. Now by (\ref{c6:eq6.3}), for any integer $m \geq
1$, 
$$
\mu(M_T (V)) V^{2/m} \leq \int\limits^T_0 \left|\zeta
\left(\frac{1}{2} + it \right)\right|^{2/m} dt \ll T(\log T)^{1/m^2},  
$$
whence
\begin{equation}
\mu(M_T(V)) \ll T(\log T)^{1/m^2} V^{-2/m}.\label{c6:eq6.45}
\end{equation}

As a function of $m$, the right-hand side of (\ref{c4:eq4.45}) is
minimized for $m = \log \log T/\log V$. Since $m$ must be an integer,
we take  
$$
m = [\log \log T / \log V],
$$
so that (\ref{c6:eq6.45}) yields then (\ref{c6:eq6.43}) and
(\ref{c6:eq6.44}). 

\medskip
\noindent{\textbf{PROOF OF THEOREM \ref{c6:thm6.4}.}} Let $0 < A < 1$
be a constant to be chosen sufficiently small later, and let $m = [(A
  \log \log T)^{\frac{1}{2}}] -a$, where $a$ is 0 or 1 so that $m$ is
an even integer. We suppose $T \geq \exp \left(\exp \dfrac{4}{A}
\right)$, so that $m$ is positive, and denote by $b_1, b_2, \ldots$
positive, absolute constants. 

Denote by $E$ the set of points $t \in [0,T]$ such that
\begin{equation}
\exp \left((A \log \log T)^{\frac{1}{2}}\right) \leq \left|\zeta
\left(\frac{1}{2} + it \right)\right| \leq \exp \left(A^{-1} (\log
\log T)^{\frac{1}{2}}\right),\label{c6:eq6.46}
\end{equation}
and let $F = [0,T] \backslash E$. If we can show that 
\begin{equation}
\int\limits_F \left|\zeta \left(\frac{1}{2} + it \right)\right|^{2/m}
dt \leq b_1 T,\label{c6:eq6.47} 
\end{equation}
then\pageoriginale Theorem \ref{c6:thm6.4} follows from
(\ref{c6:eq6.47}) and the lower bound in (\ref{c6:eq6.3}). Indeed the
latter gives, by our choice of $m$, 
$$
\int\limits^T_0 \left|\zeta \left(\frac{1}{2} + it
\right)\right|^{2/m} dt \geq b_2 e^{1/A} T, 
$$
whence by (\ref{c6:eq6.47}) with sufficiently small $A$ we have
\begin{align*} 
\int\limits_E \left|\zeta \left(\frac{1}{2} + it \right)\right|^{2/m} dt &=
\int\limits^T_0 \left|\zeta \left(\frac{1}{2} + it
\right)\right|^{2/m} dt\\ 
&\quad-\int\limits_F \left|\zeta \left(\frac{1}{2} + it \right)\right|^{2/m} dt \geq
\frac{1}{2} b_2 e^{\frac{1}{A}} T. 
\end{align*}

Consequently, by the defining property (\ref{c6:eq6.46}) of the set
$E$, 
$$
\mu(E) \left\{ \exp \left(A^{-1} (\log \log T)^{\frac{1}{2}}\right)
\right\}^{2/m} \geq \frac{1}{2} b_2 e^{1/A} T, 
$$
which gives
$$
\mu(E) \geq \frac{1}{2} b_2 \left\{\exp \left(A^{-1} - b_3
A^{-3/2}\right) \right\} T. 
$$

Thus Theorem \ref{c6:thm6.4} holds with $a_1 = A^{\frac{1}{2}}$, $a_2
= A^{-1}$, $a_3 = \dfrac{1}{2} b_2 \exp\break (A^{-1} - b_3 A^{-3/2})$. 

It remains to prove (\ref{c6:eq6.47}). Let $F'$ denote the subset of
$F$ in which 
$$
\left|\zeta \left(\frac{1}{2} + it \right)\right| < \exp \left((A \log
\log T)^{\frac{1}{2}}\right), 
$$
and let $F'' = F \backslash F'$. Then trivially
\begin{equation}
\int\limits_{F'} \left|\zeta \left(\frac{1}{2} + it
\right)\right|^{2/m} dt \leq b_4 T.\label{c6:eq6.48}
\end{equation}

Further, using (\ref{c6:eq6.3}) and noting that $m$ is even, we have
{\fontsize{10pt}{12pt}\selectfont
\begin{align}
\int\limits_{F''} \left|\zeta \left(\frac{1}{2} + it
\right)\right|^{2/m} dt & \leq \left\{ \exp \left(A^{-1} (\log \log
T)^{\frac{1}{2}}\right) \right\}^{-2/m} \int\limits^T_0 \left|\zeta
\left(\frac{1}{2}+it \right)\right|^{4/m} dt\notag\\
& \leq b_5 \exp \left(-b_6 A^{-3/2} + b_7 A^{-1}\right) T \leq b_8
T.\label{c6:eq6.49}
\end{align}}

Here $b_8$ may be taken to be independent of $A$, for example $b_8 =
b_5$ if $A \leq (b_6 / b_7)^2$. Combining (\ref{c6:eq6.48}) and
(\ref{c6:eq6.49}) we obtain (\ref{c6:eq6.47}), completing the proof of
Theorem \ref{c6:thm6.4}.  

One can also ask the following more general question, related to the
distribution of $\zeta(\frac{1}{2} + it)$, or equivalently $\log
\zeta(\frac{1}{2} + it)$ (if $\zeta(\frac{1}{2} + it) \neq
0$). Namely, if we are given a measurable set\pageoriginale $E
(\subseteq \mathbb{C})$ with a positive Jordan content, hos often does
$\log \zeta(\frac{1}{2} + it)$ belong to $E$? This problem was
essentially solved in an unpublished work of A. Selberg, who showed
that 
\begin{equation}
\lim\limits_{T \to \infty} \frac{1}{T} \mu \left(0 \leq t \leq T;
\frac{\log \zeta (\frac{1}{2} + it)}{\sqrt{\log \log t}} \in E \right)
= \frac{1}{\pi} \iint\limits_E e^{-x^2 -y^2} dx \; dy,\label{c6:eq6.50} 
\end{equation}
where as before $\mu(\cdot)$ is the Lebesgue measure. Roughly
speaking, this result says that $\log\zeta(\frac{1}{2} + it) /
\sqrt{\log \log t}$ is approximately normally distributed. From
(\ref{c6:eq6.50}) one can deduce the asymptotic formula 
\begin{equation}
\int\limits^T_0 \log \left|\zeta \left(\frac{1}{2} + it \right)
-a\right| dt = (2\pi)^{-\frac{1}{2}} T (\log \log T)^{\frac{1}{2}} +
O_a (T), \label{c6:eq6.51} 
\end{equation}
where $a \neq 0$ is fixed.

Independently of Selberg's unpublished work, A. Laurin\v cikas
obtained results on the value distribution of $|\zeta(\frac{1}{2} +
it)|$ and $|L(\frac{1}{2} + it, x)|$ in a series of papers. As a
special case of (\ref{c6:eq6.50}) he proved 
{\fontsize{10pt}{12pt}\selectfont
\begin{equation}
\lim\limits_{T \to \infty} \frac{1}{T} \mu  \left(0 \leq t \leq T :
\frac{\log |\zeta \left(\frac{1}{2} + it \right)|}{\sqrt{\frac{1}{2}
    \log \log T}} \leq y \right) = (2\pi)^{-\frac{1}{2}}
\int\limits^y_{-\infty} e^{-\frac{1}{2} u^2} du,  
\label{c6:eq6.52}
\end{equation}}
which is of course stronger and more precise than Theorem
\ref{c6:thm6.4}. It is interesting that Laurin\v cikas obtains
(\ref{c6:eq6.52}) by applying techniques from probabilistic number
theory to the asymptotic formula 
{\fontsize{10pt}{12pt}\selectfont
\begin{equation}
I_x (T) = \int\limits^T_0 \left|\zeta \left(\frac{1}{2} + it
\right)\right|^{2x} dt = T (\log T)^{x^2} \left(1+O\left((\log \log
T)^{-\frac{1}{4}}\right)\right), \label{c6:eq6.53} 
\end{equation}}
which holds uniformly for $k_T \leq k \leq k_0$, where $k_T = \exp
(-(\log \log T)^{\frac{1}{2}})$, $k_0$ is an arbitrary positive number
and  
\begin{equation}
x = \left(\left[k^{-1} \sqrt{2 \log \log T}\right] \pm
5\right)^{-1}.\label{c6:eq6.54} 
\end{equation}

The crucial result, which is (\ref{c6:eq6.53}), is proved by using the
method employed by D.R. Heath-Brown in proving Theorem
\ref{c6:thm6.1}. This involves the use of convexity techniques, but
since the argument requires asymptotic\pageoriginale formulas instead
of upper and lower bounds, it is technically much more involved than
the proof of Theorem \ref{c6:thm6.1}, although the underlying ideas
are essentially the same. For example, instead of the bounds
(\ref{c6:eq6.8}) and (\ref{c6:eq6.9}) furnished by Lemma
\ref{c6:lem6.1}, for the proof of (\ref{c6:eq6.53}) one requires the
following asymptotic formulas: 
\begin{align}
& \sum\limits_{n \leq N} d^2_x (n) n^{-2\sigma} = H(x) (2\sigma
  -1)^{-x^2} + O \left(N^{1-2\sigma} (\log
  N)^{x^2}\right)\label{c6:eq6.55}\\ 
& \quad + O \left((2\sigma -1)^{-x^2} \Gamma \left(x^2 + 1, (2\sigma -1)
  \log N\right)\right)\notag\\ 
& \quad + O \left((2\sigma -1) \log^{\frac{1}{2}} N\right) + O
  \left((2\sigma -1)^{-x^2} (\log \log T)^{-\frac{1}{2}}\right) \notag
\end{align}
and 
\begin{equation}
\sum\limits_{n \leq N} d^2_x(n) n^{-1} = \frac{H(x)}{\Gamma(x^2 +1)}
(\log T)^{x^2} \left(1+O \left((\log \log
T)^{-1}\right)\right). \label{c6:eq6.56} 
\end{equation}

Both (\ref{c6:eq6.55}) and (\ref{c6:eq6.56}) are uniform in $k \in
[k_T, k_0]$, in (\ref{c6:eq6.55}) one has $\sigma > 1/2$, and moreover 
\begin{align*}
& H(x) : = \prod_p \left\{(1-p^{-1})^{x^2} \sum\limits^\infty_{r=0}
  d^2_x (p^r) p^{-r} \right\},\\ 
& \Gamma (s,y) : = \int\limits^{\infty}_y x^{s-1} e^{-x} dx. 
\end{align*}

We shall not reproduce here the details of the arguments leading to
(\ref{c6:eq6.55}) - (\ref{c6:eq6.56}) and eventually to
(\ref{c6:eq6.53}) and (\ref{c6:eq6.52}). We only remark here that, in
order to get the asymptotic formulas that are needed, the weighted
analogue of $J(\sigma)$ (and correspondingly other integrals) in
Theorem \ref{c6:thm6.1} is defined somewhat differently now. Namely,
for $\sigma \geq 1/2$ one sets 
$$
J_x (\sigma) = \int\limits^\infty_{-\infty} |\zeta(\sigma + it)|^{2x}
w (t) dt,\quad w(t) = \int\limits^T_{\log^2 T} \exp \left(-2\eta(t - 2
\tau)^2\right) 
d \tau, 
$$
where $\eta = x k^{-1}$. The modified form of the weight function
$w(t)$ is found to be more expedient in the proof of (\ref{c6:eq6.53})
than the form of $w(t)$ used in the proof of Theorem \ref{c6:thm6.1}. 

\section{Mean Values Over Short Intervals}\label{c6:sec6.4}

In this\pageoriginale section we shall investigate lower bounds for 
\begin{equation}
I_k (T + H) - I_k (T) = \int\limits^{T+H}_T
\left|\zeta\left(\frac{1}{2} + it\right)\right|^{2k}
dt\label{c6:eq6.57}  
\end{equation}
when $T \geq T_0$, $k \geq 1$ is a fixed integer, and this interval
$[T, T+H]$ is ``short'' in the sense that $H = \circ (T)$ as $T \to
\infty$. The method of proof of Theorem \ref{c6:thm6.1}, which works
for all rational $k>0$, will not produce good results when $H$ is
relatively small in comparison with T. Methods for dealing
successfully with the integral in (\ref{c6:eq6.57}) have been
developed by K. Ramachandra, who alone or jointly with
R. Balasubramanian, obtained a number of significant results valid in
a wide range for $H$. Their latest and sharpest result on
(\ref{c6:eq6.57}), which follows from a general theorem, is the
following 

\begin{thm}\label{c6:thm6.5}
Let $k \geq 1$ be a fixed integer, $T \geq T_0$ and $(\log \log T)^2
\leq H \leq T$. Then uniformly in $H$ 
\begin{equation}
\int\limits^{T+H}_T \left|\zeta \left(\frac{1}{2} + it \right)
\right|^{2k} dt \geq \left(c'_k + O (H^{-1/8}) + O \left(\frac{1}{\log H} \right)
\right) H \log^{k^2} H, \label{c6:eq6.58} 
\end{equation}
where 
\begin{equation}
c'_k = \frac{1}{\Gamma (k^2 + 1)} \prod\limits_p \left\{
\left(1-p^{-1}\right)^{k^2} \sum\limits^\infty_{m=0} \left(\frac{\Gamma
  (k+m)^2}{\Gamma(k) m!}\right) p^{-m}\right\}.   
\label{c6:eq6.59}
\end{equation}
\end{thm}

\begin{proof}
Note that (\ref{c6:eq6.58}) is a remarkable sharpening of Theorem
\ref{c1:thm1.5}, and that it is also true unconditionally. Since 
\begin{equation}
\sum\limits_{n \leq x} d^2_k(n) \sim b'_k x \log^{k^2-1} x,\quad
\sum\limits_{n\leq x} d^2_k (n) n^{-1} \sim c'_k \log^{k^2} x,
\label{c6:eq6.60} 
\end{equation}
it is seen that instead of (\ref{c6:eq6.58}) it is sufficient to prove 
\begin{equation}
\int\limits^{T+H}_T \left|\zeta \left(\frac{1}{2} + it
\right)\right|^{2k} dt \geq \sum\limits_{n \leq H} \left(H+ O(H^{7/8})
+ O(n)\right) d^2_k (n) n^{-1}. \label{c6:eq6.61} 
\end{equation}
\end{proof}

To this end let\pageoriginale
\begin{align}
F(t) : & = \zeta^k \left(\frac{1}{2} + it \right) - \sum\limits_{n
  \leq H} d_k(n) n^{-\frac{1}{2}-it} - \sum\limits_{H < n \leq H +
  H^{\frac{1}{4}}} d_k(n) n^{-\frac{1}{2} -it}\label{c6:eq6.62}\\ 
& = \zeta^k \left(\frac{1}{2}+it \right) - F_1(t) - F_2(t),\notag 
\end{align}
say, and assume that (\ref{c6:eq6.61}) is false. Then it follows that 
\begin{equation}
\int\limits^{T+H}_T \left|\zeta \left(\frac{1}{2} + it
\right)\right|^{2k} dt \leq 2H \sum\limits_{n \leq H} d^2_k(n)
n^{-1}.\label{c6:eq6.63} 
\end{equation}

By using the mean value theorem for Dirichlet polynomials (see
(\ref{c1:eq1.15})) we obtain 
\begin{equation}
\int\limits^{T+H}_T |F_1(t)|^2 dt \leq 2 H \sum\limits_{n \leq H}
d^2_k (n) n^{-1},\label{c6:eq6.64} 
\end{equation}
and also
\begin{align}
& \int\limits^{T+H}_T |F_2(t)|^2  dt  \ll H\sum\limits_{H < n \leq H +
    H^{\frac{1}{4}}} d^2_k(n) n^{-1}\label{c6:eq6.65}\\[5pt] 
& \ll \sum\limits_{H < n \leq H + H^{\frac{1}{4}}} d^2_k(n) \ll
  H^{\frac{1}{2}} \ll H \sum\limits_{n \leq H} d^2_k (n) n^{-1}\notag 
\end{align}
since trivially $d_k(n) \leq n^{1/8}$ for $n \geq n_0$. Thus by
(\ref{c6:eq6.62}) 
\begin{align}
& \int\limits^{T + H}_T |F(t)|^2 dt \ll \int\limits^{T+H}_T
  \left(\left|\zeta \left(\frac{1}{2} + it \right)\right|^2 +
  |F_1(t)|^2 + |F_2(t)|^2 \right) dt\label{c6:eq6.66}\\ 
& \ll H \sum\limits_{n \leq H} d^2_k (n) n^{-1},\notag 
\end{align}
and consequently from (\ref{c6:eq6.63}) - (\ref{c6:eq6.65}) we obtain
by the Cauchy-Schwarz inequality 
\begin{align}
& \int\limits^{T+H}_T |F(t) F_2(t)| dt \ll H^{\frac{7}{8}}
  \sum\limits_{n \leq H} d^2_k (n) n^{-1},\label{c6:eq6.67}\\ 
& \quad \int\limits^{T+H}_T |F_1(t)F_2(t)| dt  \ll H^{\frac{7}{8}}
  \sum\limits_{n\leq H} d^2_k(n) n^{-1}.\notag  
\end{align}

We now\pageoriginale introduce the multiple averaging, which will have
the effect of making certain relevant integrals in the sequel
small. Let  $U=H^{7/8}$ and $r \geq 1$ a large, but fixed integer. Set 
$$
\int\limits_{(r)} \Psi (t) dt: = U^{-r} \int\limits^U_0 du_r
\int\limits^U_0 du_{r-1} \ldots \int\limits^U_0 du_1
\int\limits^{T+H-u_1 -u_2 - \ldots - u_r}_{T + U + u_1 + u_2 + \ldots
  + u_r} \Psi (t) dt.  
$$

Thus if $\Psi (t) \geq 0$ for $T \leq t \leq T + H$ and $\Psi (t)$ is
integrable, we have 
\begin{equation}
\int\limits^{T+H}_T \Psi (t) dt \geq \int\limits_{(r)} \Psi (t) dt
\geq \int\limits^{T+H-(r+1)U}_{T+(r+1)U} \Psi(t)
du.\label{c6:eq6.68} 
\end{equation}

From (\ref{c6:eq6.62}) we have 
$$
\zeta^k \left(\frac{1}{2} + it \right) = F_1(t) + F_2(t) + F(t), 
$$
hence
\begin{align*}
\left|\zeta \left(\frac{1}{2}+ it \right)\right|^{2k} & = |F_1 (t)|^2
+ |F_2 (t)|^2 + |F(t)|^2\\ 
&\quad + 2 \re \left\{ \overline{F_1(t)} F_2(t) + \overline{F_1(t)}
F(t) + \overline{F_2(t)} F(t) \right\}. 
\end{align*}

This gives
\begin{align}
&\int\limits^{T+H}_T \left|\zeta \left(\frac{1}{2} + it
  \right)\right|^{2k} dt \geq \int\limits_{(r)} \left|\zeta
  \left(\frac{1}{2} + it \right)\right|^{2k} 
  dt \int\limits_{(r)} |F_1(t)|^2 dt\geq
  \label{c6:eq6.69}\\ 
&\quad{} + 2 \re  \left\{ \int\limits_{(r)} \overline{F_1(t)} F(t)
  dt\right\} +  O \left(\int\limits_{(r)} \left\{ |F_1(t) F_2(t)| +
  |F_2(t) F(t)|\right\} dt \right).\notag 
\end{align}

The integrals in the error terms are majorized by the integrals in
(\ref{c6:eq6.67}). Thus using once again (\ref{c1:eq1.15}) we obtain 
\begin{align}
& \int\limits^{T+H}_T \left|\zeta \left(\frac{1}{2} + it
  \right)\right|^{2k} dt \geq \int\limits^{T+H - (r+1) U}_{T + (r+1)
    U} |F_1(t)|^2 dt \label{c6:eq6.70}\\ 
&\quad +  2 \re \left\{ \int\limits_{(r)} \overline{F_1(t)} F(t) dt
  \right\} + O \left(H^{7/8} \sum\limits_{n \leq H} d^2_k(n)
  n^{-1}\right)\notag\\ 
& = \sum\limits_{n \leq H} \left(H + O(H^{7/8}) + O (n)\right)
  d^2_k(n) n^{-1} + 2 \re \left\{ \int\limits_{(r)} \overline{F_1(t)}
  F(t) dt\right\}.\notag 
\end{align}\pageoriginale

We assumed that (\ref{c6:eq6.61}) is false. Therefore if we can prove that 
\begin{equation}
\int\limits_{(r)} \overline{F_1(t)} F(t) dt \ll H^{7/8} \sum\limits_{n
  \leq H} d^2_k(n) n^{-1}, \label{c6:eq6.71} 
\end{equation}
we shall obtain a contradiction, which proves (\ref{c6:eq6.61}) and
consequently (\ref{c6:eq6.58}). Let 
$$ 
A(s) : = \sum\limits_{n \leq H} d_k(n) n^{-s},\quad B(s): = \zeta^{k}
(s) - \sum\limits_{n \leq H + H^{\frac{1}{4}}} d_k(n) n^{-s}.  
$$

Then by Cauchy's theorem
$$
\int\limits_D A(1-s) B(s) ds = 0,
$$
where $D$ is the rectangle with vertices $\frac{1}{2} + i (T + U + u_1
+ \ldots + u_r)$, $2 + i (T + U + u_1 + \ldots + u_r)$, $2+ i(T+ H-u_1
- \ldots - u_r)$, $\frac{1}{2} + i (T + H - u_1 - \ldots -
u_r)$. Hence 
{\fontsize{10}{12}\selectfont
\begin{align*}
& U^{-r} \int\limits^U_0 du_r \int\limits^U_0 du_{r-1} \ldots
  \int\limits^U_0 du_1 \int\limits_D A(1-s) B (s) ds\\ 
& = U^{-r} \int\limits^U_0 du_r \int\limits^U_0 du_{r-1} \ldots
  \int\limits^U_0 du_1 \left(\int\limits_{I_1} + \int\limits_{I_2}  +
  \int\limits_{I_3} + \int\limits_{I_4} \right) A(1-s) B(s) ds = 0,  
\end{align*}}
where $I_1$ and $I_3$ are the left and right vertical sides of $D$,
and $I_2$ and $I_4$ are the lower and upper horizontal sides,
respectively. If we denote, for $1 \leq n \leq 4$, 
$$
J_4 = U^{-r} \int\limits^U_0 du_r \int\limits^U_0 du_{r-1} \ldots
\int\limits^U_0 du_1 \int\limits_{I_n} A (1-s) B(s) ds  
$$
and observe that 
$$
J_1 = i \int\limits_{(r)} \overline{F_1(t)}  F(t) dt,
$$
then (\ref{c6:eq6.71}) will follow if we can show that 
\begin{align}
& J_2 \ll H^{7/8} \sum\limits_{n \leq H} d^2_k (n) n^{-1},
  \label{c6:eq6.72}\\[4pt] 
& J_3 \ll H^{7/8} \sum\limits_{n\leq H} d^2_k (n) n^{-1},
  \label{c6:eq6.73} 
\end{align}\pageoriginale
and 
\begin{equation}
J_4 \ll H^{7/8} \sum\limits_{n \leq H} d^2_k (n)
n^{-1}. \label{c6:eq6.74} 
\end{equation}

To prove (\ref{c6:eq6.73}) note that, for $\re s > 1$,
$$
B(s) = \zeta^k (s) - \sum\limits_{m\leq H + H^{\frac{1}{4}}}
d_k(m)m^{-s} = \sum\limits_{m > H + H^{\frac{1}{4}}} d_k(m) m^{-s}. 
$$

Therefore
{\fontsize{10}{12}\selectfont
\begin{align*}
J_3 &= U^{-r} \int\limits^U_0 du_r \ldots \int\limits^U_0 du_1
\int\limits^{2+i(T+H -u_1 - \ldots - u_r)}_{2+i(T+U+u_1 + \ldots +
  u_r)} A(1-s) B(s) ds\\ 
& = i U^{-r} \sum\limits_{n \leq H} \sum\limits_{m > H +
  H^{\frac{1}{4}}} d_k(m)m^{-2} nd_k (n) \int\limits^U_0 du_r \ldots
\int\limits^U_0 du_1 \int\limits^{T+H-u_1 -\ldots -
  u_r}_{T+U+u_1+\ldots + u_r} \left(\frac{n}{m} \right)^{it} dt\\ 
& \ll_r U^{-r} \sum\limits_{n\leq H} \sum\limits_{m> H+
  H^{\frac{1}{4}}} d_k(m) m^{-2} nd_k (n) \left(\log\frac{m}{n}
\right)^{-r-1}. 
\end{align*}}

But for $m,n$ in the last sum we have
$$
\log \frac{m}{n} \geq \log \frac{H+H^{\frac{1}{4}}}{H} \geq
\frac{1}{2} H^{-\frac{3}{4}}, 
$$
and since $U =H^{7/8}$ we obtain
\begin{align*}
 J_3 &  \ll_r H^{7/8} H^{-(r+1)/8} \sum\limits_{n \leq H} nd_k(n)
 \sum\limits_{m>H} d_k(m) m^{-2}\\ 
& \ll_r H^{7/8} H^{-(r+1)/8} H \log^{2k-2} H \ll_r H^{7/8}
 \log^{k^2-1} H 
\end{align*}
for any $r \geq 7$.

It remains to prove (\ref{c6:eq6.72}) and (\ref{c6:eq6.74}). Since
both proofs are similar, only the details for the latter will be
given. Set for brevity\pageoriginale 
$$
G(s) : = A (1-s) B (s), \; T_r : = T + U + u_1 + \ldots + u_r. 
$$ 

Then we have
\begin{equation}
J_4 = U^{-r} \int\limits^{U}_0 du_r \ldots \int\limits^{U}_0 du_1
\int\limits^2_{\frac{1}{2}} G(\sigma + iT_r) d \sigma ,
\label{c6:eq6.75} 
\end{equation}
and by the theorem of residues we may write 
\begin{equation}
G(\sigma+ iT_r) =\frac{1}{2\pi ie} \int\limits_E G(w+ \sigma + iT_r)
\exp \left(-\cos \frac{w}{A} \right) \frac{dw}{w},
\label{c6:eq6.76} 
\end{equation}
where $A > 0$ is a constant, and $E$ is the rectangle with vertices
$\frac{1}{2} - \sigma + iT - iT_r$, $2+iT-iT_r$, $2+i(T+H-T_r)$,
$\frac{1}{2} - \sigma + i (T+H-T_r)$, whose vertical sides are $I_5,
I_7$ and horizontal sides are $I_6, I_8$, respectively. The kernel
function $\exp \left(-\cos \dfrac{w}{A} \right)$, which is essentially
the one used in the proof of Theorem \ref{c1:thm1.3}, is of very rapid
decay, since 
$$
\left|\exp \left(-\cos \frac{w}{A} \right)\right| = \exp \left( -\cos
\frac{u}{A} \cdot ch \frac{v}{A}\right) \quad (w=u+iv, \; A > 0). 
$$

We insert (\ref{c6:eq6.76}) in (\ref{c6:eq6.75}), obtaining
$$
J_4 = J_5 + J_6 + J_7 + J_8,
$$
say, where $J_n (n = 5,6,7,8)$ comes from the integral over $I_n$. For
$w = u+ iv$ on $I_6$ and $I_8$ we have $|v| \geq U = H^{7/8}$,
consequently for $A = 10$ 
\begin{align}
\left|\exp \left(-\cos \frac{w}{A} \right)\right| &= \exp \left(-\cos
\frac{u}{10}\cdot ch \frac{v}{10}  \right) \leq \exp
\left(-\frac{1}{2} \cos \left(\frac{1}{5} \right) e^{|v|/10}
\right)\notag\\ 
&\leq \exp \left(-\frac{1}{2} \cos \left(\frac{1}{5} \right)  e
H^{7/8}/10\right) \ll_C T^{-C}\label{c6:eq6.77} 
\end{align}
for any fixed $C>0$, since $H \geq (\log \log T)^2$. In fact, one
could increase the range for $H$ in Theorem \ref{c6:thm6.5} to $H \geq
(\log \log T)^{1+\delta}$ for any fixed $\delta > 0$ at the cost of
replacing the error term $O(H^{-1/8})$ by a weaker one. From
(\ref{c6:eq6.77}) it easily follows that the contribution of $J_6 +
J_8$ is negligible. 

Consider\pageoriginale now
\begin{align}
J_5 & = \frac{U^{-r}}{2 \pi i e} \int\limits^U_0 du_r \ldots
\int\limits^{U}_0 du_2 \int\limits^2_{\frac{1}{2}}\notag\\ 
& \qquad \left\{ \int\limits^U_0 du_1 \int\limits_{I_5} G (w+ \sigma +
i T_r) \exp \left(-\cos \frac{w}{A} \right) \frac{dw}{w} \right\}. 
\label{c6:eq6.78}
\end{align}

On $I_5$ we have  $w = \frac{1}{2} - \sigma +i (v-T_r)$, $T \leq v
\leq T + H$. Thus the integral in curly brackets in (\ref{c6:eq6.78})
becomes (with $A=10$) 
{\fontsize{10}{12}\selectfont
\begin{align*}
& i \int\limits^{T+H}_T G \left(\frac{1}{2} + iv \right) dv
  \int\limits^U_0 \exp \left\{-\cos \left(\frac{1}{2} - \sigma + i
  \left(v - T - u_1 - \ldots - u_r \right)/ 10 \right) \right\}\\[5pt] 
& \qquad \frac{du_1}{\frac{1}{2} - \sigma + i (v - T - u_1 - \ldots -
    u_r)}\\[5pt] 
& \ll \int\limits^{T+H}_T \left|G\left(\frac{1}{2} + iv\right)\right|
  dv \int\limits^U_0 \exp \left\{-\frac{1}{2} \cos \left(\frac{
    \frac{1}{2} - \sigma}{10}\right) \cdot e^{-|v-T-u_1-\ldots - u_r|}
  \right\}\\  
& \qquad \frac{du_i}{\left((\delta - 1/2)^2 + (u_1 + T + \ldots + u_r
    - v)^2\right)^{1/2}}.  
\end{align*}}

The presence of the exponential factor makes the portion of the
integral over $u_1$, for which $|u_1 + T + \ldots + u_r -v| \geq 1$,
to be $O(1)$. The remaining portion for which $|u_1 + T + \ldots + u_r
-v| < 1$ is majorized by  
{\fontsize{10}{12}\selectfont
\begin{gather*}
\int\limits^{v-T-U-\ldots - u_r +1}_{v - T - U-\ldots - u_r - 1}
\frac{du_1}{\left((\sigma -\frac{1}{2})^2 + (u_1 + T + \ldots + u_r
  -v)^2\right)^{\frac{1}{2}}} = \int\limits^1_{-1}
\frac{dx}{\left((\sigma -\frac{1}{2})^2 + x^2\right)^{\frac{1}{2}}}\\[5pt] 
= \log \frac{1+\left((\sigma -\frac{1}{2})^2 +
  1\right)^{\frac{1}{2}}}{-1 + \left((\sigma -\frac{1}{2})^2
  +1\right)^{\frac{1}{2}}} \ll \log \frac{1}{\sigma -\frac{1}{2}}.  
\end{gather*}}

Hence we obtain
{\fontsize{10pt}{12pt}\selectfont
\begin{align*}
& J_5 \ll U^{-1} \int\limits^{T + H}_T \left|G \left(\frac{1}{2} + iv
  \right)\right| dv \int\limits^2_{\frac{1}{2}} \left(\log \frac{1}{\sigma
    -\frac{1}{2}} \right)^{-1} d \sigma\\ 
& \ll U^{-1} \int\limits^{T+H}_T \left|G \left(\frac{1}{2} + iv
  \right)\right| dv = U^{-1} \int\limits^{T+H}_T |F_1(t) F(t)| dt \ll
  H^{\frac{1}{8}} \sum\limits_{n \leq H} d^2_k(n) n^{-1}, 
\end{align*}}
on using (\ref{c6:eq6.64}) and (\ref{c6:eq6.66}) coupled with the
Cauchy-Schwarz inequality. 
 
Finally we consider
$$ 
J_7 = \frac{U^{-r}}{2 \pi i e} \int\limits^U_0 du_r \ldots
\int\limits^U_0 du_1 \int\limits^2_{\frac{1}{2}} d
\sigma\int\limits_{I_7} G(w+ \sigma + i T_r) \exp \left(-\cos
\frac{w}{10} \right) \frac{dw}{w}. 
$$

On $I_7$\pageoriginale we have $w = 2 + iv$, $T - T_r \leq v \leq T +
H - T_r$. Since $|T - T_r| = |U + u_1 + \ldots + u_r| \geq H^{7/8}$,
$|T+ H - T_r| \geq H^{7/8}$, the presence of $\exp\left(-\cos
\frac{w}{10} \right)$ makes it possible to replace, in the integral
over $w$, the range for $v$ by $(-\infty, \infty)$ with a negligible
error. In the remaining integral we may interchange, in view of the
absolute convergence, the order of integration. We are left with  
\begin{align*}
& \frac{1}{2\pi e} \int\limits^{\infty}_{-\infty} \exp \left(-\cos
  \left(\frac{2+iv}{10} \right) \right) \frac{dv}{2+iv}\\ 
& \qquad \left\{ U^{-r} \int\limits^2_{\frac{1}{2}} d \sigma
  \int\limits^U_0 du_r \ldots \int\limits^U_0 du_1 G \left(2+ \sigma +
  i (v+ T_r)\right)\right\}. 
\end{align*}

But we have
\begin{align*}
& G(2 + \sigma + i (v+ T_r)) = A(-1-\sigma - iv - iT_r) B (2+ \sigma +
  iv + iT_r)\\ 
&\quad = \sum\limits_{m> H + H^{\frac{1}{4}}} \sum\limits_{n \leq H}
  d_k(m) m^{-2 - \sigma} d_k(n) n^{\sigma +1} \left(\frac{n}{m}
  \right)^{i(v+T+ U+ u_1 + \ldots + u_r)}. 
\end{align*}

If we proceed as in the case of the estimation of $J_3$ and carry out
the integrations over $u_1,u_2, \ldots u_r$, we readily seen that the
contribution of $J_7$ is $O(1)$ if $r$ is sufficiently large. This
completes the proof of the theorem. It may be noted that essentially
the multiple averaging accounts for the shape of the lower bound in
(\ref{c6:eq6.58}), while the kernel function introduced in
(\ref{c6:eq6.76}) regulates the permissible range for $H$. 

\newpage

\begin{center}
\textbf{NOTES FOR CHAPTER 6}
\end{center}

The method\pageoriginale of D.R. Heath-Brown \cite{Heath-Brown4},
where (\ref{c6:eq6.2}) was proved, is based on a convexity
technique. This is embodied in Lemma \ref{c6:lem6.2} and Lemma
\ref{c6:lem6.3}, both of which are due to R.M. Gabriel
\cite{Gabriel1}. The bounds given by (\ref{c6:eq6.3}), which have the
merit of being uniform in $m$, were proved by M. Jutila
\cite{Jutila2}, who modified the method of Heath-Brown
\cite{Heath-Brown4}. Lemmas \ref{c6:lem6.1}, \ref{c6:lem6.4},
\ref{c6:lem6.5} and \ref{c6:lem6.6} are from Heath-Brown
\cite{Heath-Brown4}. 

E.C. Titchmarsh was the first to prove that if $k$ is a fixed natural
number, then 
$$
\int\limits^{\infty}_0 \left|\zeta \left(\frac{1}{2} + it
\right)\right|^{2k} e^{-\delta t} dt \gg_k \frac{1}{\delta} \left(\log
\frac{1}{\delta} \right)^{k^2} 
$$
for $0 < \delta \leq \frac{1}{2}$ (see Theorem 7.19 of his book
\cite{Titchmarsh1}). From this it is easy to deduce that  
$$
\limsup\limits_{T \to \infty} \left\{I_k (T) T^{-1} (\log
T)^{k^2}\right\} > 0. 
$$

Improving upon this K. Ramachandra (see his series of papers
\cite{Ramachandra3}) was the first to obtain lower bounds of the type
(\ref{c6:eq6.2}). His method uses averaging techniques which do not
depend on the aforementioned results of Gabriel, and besides it is
capable of obtaining results valid over short intervals. In \cite[Part
II]{Ramachandra3}, he shows that 
\begin{equation} 
\int\limits^{T+H}_T \left. \left|\frac{d^\ell}{ds^{\ell}}
(\zeta{k}(s)) \right|_{s = \frac{1}{2} +it} \right| dt > C_{k,\ell} H
(\log H)^{\lambda},\label{c6:eq6.79} 
\end{equation}
where $100 \leq (\log T)^{1/m} \leq H \leq T$. Here $k, m,\ell$ are
any fixed natural numbers, $T \geq T_\circ (k,\ell,m)$, $\lambda =
\ell + \dfrac{1}{4} k^2$, $C_{k,\ell} > 0$ is a constant which depends
on $k,\ell$. If the kernel $\exp (z^{4a+2})$ that Ramachandra used  is
replaced by his kernel $\exp (\sin^2 z)$, then the range for $H$ can
be taken as $C_k \log \log T \leq H \leq T$ (as mentioned in Notes for
Chapter \ref{c1}, the latter type of kernel is essentially best
possible). Upper bounds for the integral in (\ref{c6:eq6.79}) are
obtained by Ramachandra in \cite[Part
  III]{Ramachandra3};\pageoriginale one of the results is that 
\begin{equation}
\int\limits^{T+H}_T \left|\zeta^{(\ell)} \left(\frac{1}{2} + it
\right)\right|dt \ll H (\log T)^{\frac{1}{4}+ \ell}
\label{c6:eq6.80} 
\end{equation}
for $\ell \geq 1$ a fixed integer, $H = T^{\lambda}$, $\frac{1}{2} <
\lambda \leq 1$. From (\ref{c6:eq6.79}) and (\ref{c6:eq6.80}) we infer
that 
$$
T(\log T)^{\frac{1}{4}} \ll \int\limits^T_0 \left|\zeta
\left(\frac{1}{2} + it \right)\right|dt\ll T(\log T)^{\frac{1}{4}}, 
$$
but it does not seem possible yet to prove (or disprove) that, for
some $C>0$, 
$$
\int\limits^T_0 \left|\zeta \left(\frac{1}{2} + it \right)\right| dt
\sim CT (\log T)^{\frac{1}{4}} \quad (T \to \infty). 
$$

The lower bound results are quite general (for example, they are
applicable to ordinary Dirichlet $L$-series, $L$-series of number
fields and so on), as was established by Ramachandra in \cite[Part
  III]{Ramachandra3}. The climax of this approach is in
Balasubramanian-Ramachandra 
\cite{Balasubramanian and Ramachandra5} (see also Ramachandra's papers
\cite{Ramachandra5}, \cite{Ramachandra7}, \cite{Ramachandra8}). Their
method is very general, and gives results like 
$$
I_k (T+H) - I_k(T) \geq C'_k H (\log H)^{k^2} \left( 1+ O \left(
\frac{\log \log T}{H} \right)  + O \left(\frac{1}{\log G} \right)
\right) 
$$
uniformly for $c_k \log \log T \leq H \leq T$, if $k \geq 1$ is a
fixed integer (and for all fixed $k>0$, if the Riemann hypothesis is
assumed). Bounds for $I_k(T+H) - I_k(T)$ for irrational $k$ were also
considered by K. Ramachandra (see \cite[Part II]{Ramachandra6}), 
and his latest result
is published in \cite[Part III]{Ramachandra6}. 

Theorem \ref{c6:thm6.2} is from A. Ivi\'c - A. Perelli \cite{Ivic and
  Perelli1}, which contains a discussion of the analogous problem for
some other zeta-functions. 

E.J. Littlewood's lemma, mentioned in the proof of Lemma
\ref{c6:lem6.7}, is proved in Section 9.9 of E.C. Titchmarsh
\cite{Titchmarsh1}. Suppose that $\Phi(s)$ is meromorphic in and upon
the boundary of the rectangle $\mathscr{D}$ bounded by the lines $t =
0$, $t = T(>0)$, $\sigma = \alpha$, $\sigma =\beta$ $B > \alpha)$, and
regular\pageoriginale and not zero on $\sigma =\beta$. Let $F(s) =
\log \Phi (s)$ in the neighborhood of $\sigma =\beta$, and in general
let 
$$
F(s) = \lim\limits_{\epsilon \to 0 +} F(\sigma + it + i \epsilon). 
$$

If $\nu(\sigma',T)$ denotes the excess of the number of zeros over the
number of poles in the part of the rectangle for which $\sigma >
\sigma'$, including zeros or poles on $t = T$, but not these on $t=0$,
then Littlewood's lemma states that 
$$ 
\int\limits_{\mathbb{D}} F(s) ds = - 2 \pi i
\int\limits^{\beta}_{\alpha} \nu(\sigma, T) d \sigma. 
$$

Theorem \ref{c6:thm6.3} and Theorem \ref{c6:thm6.4} were proved by
M. Jutila \cite{Jutila2}.  

A comprehensive account on the distribution of values of $\zeta(s)$,
including proofs of Selberg's results (\ref{c6:eq6.50}) and
(\ref{c6:eq6.51}), is to be found in the monograph of D. Joyner
\cite{Joyner1}. 

A. Laurin\v cikas' results on the distribution of $|\zeta(\frac{1}{2}
+ it)|$, some of which are mentioned at the end of Section
\ref{c6:sec6.3}, are to be found in his papers \cite{Laurincikas1},
\cite{Laurincikas2}, \cite{Laurincikas3}, \cite{Laurincikas4}. For
example, the fundamental formula (\ref{c6:eq6.52}) is proved in
\cite[Part I]{Laurincikas3} under the Riemann hypothesis, while
\cite[Part II]{Laurincikas3} contains an 
unconditional proof of this result. 

Theorem \ref{c6:thm6.5} is a special case of a general theorem due to
R. Balasubramanian and K. Ramachandra \cite{Balasubramanian and
  Ramachandra5}. I am grateful for their permission to include this
result in my text. 

\begin{thebibliography}{99}
\bibitem{Anderson1} R.J. Anderson,\pageoriginale On the function $Z(t)$ associated with the Riemann zeta-function, J. Math. Anal. Appl. 118(1986), 323-340.

\bibitem{Apostol1} T.M. Apostol, Modular functions and Dirichlet series in Number Theory, GTM 41, Springer Verlag, Berlin-Heidelberg-New York, 1976.

\bibitem{Atkinson1} F.V. Atkinson, The mean value of the zeta-function on the critical line, Proc. London Math. Soc. (2) 47(1941), 174-200.

\bibitem{Atkinson2} F.V. Atkinson, The mean value of the Riemann zeta-function, Acta Math. 81(1949), 353-376.

\bibitem{Atkinson3} F.V. Atkinson, The Riemann zeta-function, Duke Math. J. 17 (1950), 63-68.

\bibitem{Balasubramanian1} R. Balasubramanian, An improvement of a theorem of Titchmarsh on the mean square of $|\zeta(\frac{1}{2} + it)|$, Proc. London Math. Soc. (3) 36(1978), 540-576.

\bibitem{Balasubramanian2} R. Balasubramanian, On the frequency of Titchmarsh's phenomenon for $\zeta(s)$ IV, Hardy-Ramanujan J. 9 (1986), 1-10.

\bibitem{Balasubramanian and Ramachandra1} R. Balasubramanian and K. Ramachandra, On the frequency of Titchmarsh's phenomenon for $\zeta(s)$ III, Proc. Indian Acad. Sci. 86A (1977), 341-351.

\bibitem{Balasubramanian and Ramachandra2} R. Balasubramanian and K. Ramachandra, On an analytic continuation of $\zeta(s)$, Indian J. Pure Appl. Math. 18(1987), 790-793.

\bibitem{Balasubramanian and Ramachandra3} R. Balasubramanian and K. Ramachandra, Some local convexity theorems\pageoriginale for the zeta-function like analytic functions. Hardy Ramanujan J. 11 (1988), 1-12.

\bibitem{Balasubramanian and Ramachandra4} R. Balasubramanian and K. Ramachandra, A lemma in complex function theory I, Hardy-Ramanujan J. 12 (1989), 1-5 and II, ibid. 12 (1989), 6-13.

\bibitem{Balasubramanian and Ramachandra5}  R. Balasubramanian and K. Ramachandra, Proof of some conjectures on the mean-value of Titchmarsh series I, Hardy-Ramanujan J. 13 (1990), 1--20 and II ibid. 14 (1991), 1--20.

\bibitem{Balasubramanian Ivic and Ramachandra1} R. Balasubramanian, A. Ivic, K. Ramachandra, The mean square of the Riemann zeta-function on the line $\sigma=1$. Enseign. Math. (2) 38 (1992), no.~1--2, 13--25.

\bibitem{Bartz1} K. Bartz, On an effective order estimate of the Riemann zeta-function in the critical strip, Monatshefte Math. 109(1990), 267-270.

\bibitem{Bohr1} H. Bohr, Almost periodic functions, Chelsea, New York, 1947.

\bibitem{Bombieri and Iwaniec1} E. Bombieri and H. Iwaniec, On the order of $\zeta(\frac{1}{2} + it)$, Ann. Scuola Norm. Sup. Pisa Cl. Sci. (4) 13(1986), 449-472, and ibid. 13(1986), 473-486.

\bibitem{Bruggeman1} R.W. Bruggeman, Fourier coefficients of cusp forms, Invent. Math. 45(1978), 1-18.

\bibitem{Bump1} D. Bump, Automorphic forms on $GL(3,\mathbb{R})$, LNM 1083, Springer Verlag, Berlin etc., 1984.

\bibitem{Bump2} D. Bump, The Rankin-Selberg method: a survery, in ``Number theory, trace formulas and discrete groups'', Symp. in honour of A. Selberg, Oslo 1987, Academic Press, Boston etc., 1989, pp. 49-110.

\bibitem{Chandrasekharan and Narasimhan1} K. Chandrasekharan and R. Narasimhan, Hecke's functional equation and arithmetical identities, Annals of Math. 74 (1961), 1-23.

\bibitem{Chandrasekharan and Narasimhan2} K. Chandrasekharan and R. Narasimhan, Functional equations with multiple gamma factor and the average order of arithmetical functions, Annals of Math. 76(1962), 93-136.

\bibitem{Chandrasekharan and Narasimhan3} K. Chandrasekharan and R. Narasimhan, Approximate functional equations\pageoriginale for a class of zeta-functions, Math. Ann. 152(1963), 30-64.

\bibitem{Conrey and Ghosh1} J.B. Conrey and A. Ghosh, On mean values of the zeta-function, Mathematika  31(1984), 159-161.

\bibitem{Conrey and Ghosh2} J.B. Conrey and A. Ghosh, A mean value theorem for the Riemann zeta-function at its relative extrema on the critical line, J. London Math. Soc. (2) 32(1985), 193-202.

\bibitem{Conrey and Ghosh3} J.B. Conrey and A. Ghosh, On mean values of the zeta-functions II, Acta Arith. 52(1989), 367-371.

\bibitem{Conrey and Ghosh4} J.B. Conrey and A. Ghosh, Mean values of the Riemann zeta- function III, Proceedings of the Amalfi Conference on Analytic Number Theory (Maiori, 1989), 35--59, Univ. Salerno, Salerno, 1992.

\bibitem{Conrey Ghosh and Gonek1} J.B. Conrey, A. Ghosh and S.M. Gonek, Mean values of the Riemann zeta-function with applications to the distribution of zeros, in ``Number theory, trace formulas and discrete groups'', Symp. in honour of A. Selberg, Oslo 1987, Academic Press, Boston etc., 1989, pp. 185-199.

\bibitem{Corradi and Katai1} K. Corr\'adi and I. K\'atai, Egy megjegyz\'es K.S. Gangadharan, ``Two classical lattice point problems'' cim\"u dolgozat\'ahoz (in Hungarian), MTA III Ost\'aly K\"ozlem\'enyei 17(1967), 89-97.

\bibitem{Deshouillers and Iwaniec1} J.-M. Deshouillers and H. Iwaniec, Kloosterman sums and the Fourier coefficients of cusp forms, Invent. Math. 70 (1982), 219-288.

\bibitem{Deshouillers and Iwaniec2} J.-M. Deshouillers and H. Iwaniec, An additive divisor problem, J. London Math. Soc. (2) 26(1982), 1-14.

\bibitem{Deshouillers and Iwaniec3} J.-M. Deshouillers and H. Iwaniec, Power mean-values for the Riemann zeta-function, Mathematika 29(1982), 202-212.

\bibitem{Deshouillers and Iwaniec4} J.-M. Deshouillers and H. Iwaniec, Power mean-values for Dirichlet's polynomials and the  Riemann zeta-function II, Acta Arith. 43(1984), 305-312.

\bibitem{Estermann1} T. Estermann, On the representation of a number as the sum of two products, Proc. London Math. Soc. (2) 31(1930), 123-133.

\bibitem{Estermann2} T. Estermann,\pageoriginale \"Uber die Darstellung einer Zahl als Differenz von zwei Produkten, J. Reine Angew. math. 164 (1931), 173-182.

\bibitem{Estermann3} T. Estermann, On Kloosterman's sum, mathematika 8(1961), 83-86.

\bibitem{Estermann4} T. Estermann, Introduction to modern prime number theory, Cambridge University Press, Cambridge, 1969.

\bibitem{Fouvry and Tenenbaum1} E. Fouvry and G. Tenenbaum, Sur la corr\'elation des fonctions de Piltz, Rivista Mat. Iberoamericana $1 N^\circ 3$ (1985), 43-54.

\bibitem{Gabcke1} W. Gabcke, Neue Herleitung und explizite Restabsch\"atzung der Riemann-Siegel-Formel, Mathematisch-Naturwissenschaftliche Fakult\"at der Georg-August Universit\"at zu G\"ottingen, Dissertation, G\"ottingen, 1979.

\bibitem{Gabriel1} R.M. Gabriel, Some results concerning the integrals of moduli of regular functions along certain curves, J. London Math. Soc. 2 (1927), 112-117.

\bibitem{Gangadharan1} K.S. Gangadharan, Two classical lattice point problems, Proc. Cambridge Philos. Soc. 57(1961), 699-721.

\bibitem{Goldfeld1} D. Goldfeld, Analytic number theory on $GL(n,\mathbb{R})$, in ``Analytic number theory and Diophantine problems'', Progr. Math. 78, Birkha\"user, Basel-Stuttgart, 1987, pp. 159-182.

\bibitem{Goldfeld2} D. Goldfeld, Kloosterman zeta-functions for $GL(n,\mathbb{Z})$, Proc. Intern. Congress of Mathematicians, Berkeley, Calif. 1986, Berkeley, 1987, pp. 417-424. 

\bibitem{Goldfeld and Sarnak1} D. Goldfeld and P. Sarnak, Sums of Kloosterman sums, Invent. Math. 71 (1983), 243-250.

\bibitem{Good1} A. Good, Approximative Funktionalgleichungen und Mittelwertsatze, die Spitzenformen assoziiert sind, Comment. Math. Helv. 50(1975), 327-361.

\bibitem{Good2} A. Good, Ueber das quadratische Mittel der Riemannschen Zeta-function auf der kritischen Linie, Comment. Math. Helv. 52 (1977), 35-48.

\bibitem{Good3} A. Good,\pageoriginale Ein $\Omega$-Resultat f\"ur das quadratische Mittel der Riemannschen Zetafunktion auf der kritischen Linie, Invent. Math. 41(1977), 233-251. 

\bibitem{Good4} A. Good, Beitraege zur Theorie der Dirichletreihen, die Spitzen-formen zugeordnet sind, J. Number Theory 13(1981), 18-65.

\bibitem{Good5} A. Good, The square mean of Dirichlet series associated with cusp forms, Mathematika 29(1982), 278-295.

\bibitem{Good6} A. Good, On various means involving the Fourier coefficients of cusp forms, Math. Zeit. 183(1983), 95-129.

\bibitem{Graham1} S.W. Graham, An algorithm for computing optimal exponent pairs, J. London Math. Soc. (2) 33(1986), 203-218.

\bibitem{Hafner1} J.L. Hafner, New omega theorems for two classical lattice point problems, Invent. Math. 63(1981), 181-186.

\bibitem{Hafner2} J.L. Hafner, On the average order of a class of arithmetical functions, J. Number Theory  15(1982), 36-76.

\bibitem{Hafner3} J.L. Hafner, New omega results in a weighted divisor problem, J. number Theory 28(1988), 240-257.

\bibitem{Hafner and Ivic1} J. L. Hafner and A. Ivi\'c, On the mean square of the Riemann zeta-function on the critical line, J. number Theory 32(1989), 151-191.

\bibitem{Hardy1} G.H. Hardy, Ramanujan, Chelsea, New York, 1962.

\bibitem{Hardy and Littlewood1} G.H. Hardy and J.E. Littlewood, The  approximate functional equation in the theory of the zeta-function, with applications to the divisor problems of Dirichlet and Piltz, Proc. London Math. Soc.(2) 21(1922), 39-74.

\bibitem{Hardy and Littlewood2} G.H. Hardy and J.E. Littlewood, The approximate functional equation for $\zeta(s)$ and $\zeta^2(s)$, Proc. London Math. Soc. (2) 29(1929), 81-97.

\bibitem{Hecke1} E. Hecke, Ueber Modulfunktionen und die Dirichletschen Reihen mit Eulerscher Produktentwicklung I, Math. Ann. 114(1937), 1-28.

\bibitem{Heath-Brown1} D.R. Heath-Brown,\pageoriginale The twelfth power moment of the Riemann zeta-function, Quart. J. Math. (Oxford) 29(1978), 443-462.

\bibitem{Heath-Brown2} D.R. Heath-Brown, The mean value theorem for the Riemann zeta-function, Mathematika 25(1978), 177-184.

\bibitem{Heath-Brown3} D.R. Heath-Brown, the fourth power moment of the Riemann zeta-function, Proc. London Math. Soc. (3) 38(1979), 385-422.

\bibitem{Heath-Brown4} D.R. Heath-Brown, Fractional moments of the Riemann zeta-function, J. London Math. Soc. (2) 24(1981), 65-78.

\bibitem{Heath-Brown5} D.R. Heath-Brown, The divisor function $d_3(n)$ in arithmetic progressions, Acta Arith. 47(1986), 29-56.

\bibitem{Heath-Brown and Huxley1} D.R. Heath-Brown and M.N. Huxley, Exponential sums with a difference, Proc. London Math. Soc. (3) 61(1990), 227-250.

\bibitem{Hooley1} C. Hooley, On the greatest prime factor of a cubic polynomial, J. Reine Angew. Math. 303/304 (1978), 21-50.

\bibitem{Huxley1} M.N. Huxley, Introduction to Kloostermania, in ``Elementary and analytic theory of numbers'', Banach Center Publications Vol. 17, PWN - Polish Scientific Publishers, Warsaw 1985, pp. 217-305.

\bibitem{Huxley2} M.N. Huxley, Exponential sums and lattice points, Proc. London Math. Soc. (3) 60(1990), 471-502.

\bibitem{Huxley3} M.N. Huxley, Exponential sums and the Riemann zeta-function, in ``Number theory'' (eds. J.-M. De Koninck and C. Levesque), Walter de Gruyter, Berlin-New York 1989, pp. 417-423.

\bibitem{Huxley4} M.N. Huxley, Exponential sums and the Riemann zeta-function, IV,  Proc. London Math. Soc. (3) 66 (1993), no.~1, 1--40.

\bibitem{Huxley5} M.N. Huxley, Exponential sums and rounding error,  J. London Math. Soc. (2) 43 (1991), no.~2, 367--384.

\bibitem{Huxley and Watt1} M.N. Huxley and N. Watt, Exponential sums and the Riemann zeta-function, Proc. London Math. Soc. (3) 57(1988), 1-24.

\bibitem{Huxley and Watt2} M.N. Huxley and N. Watt, Exponential sums with a parameter, Proc. London Math. Soc. (3) 59(1989), 233-252.

\bibitem{Huxley and Kolesnik1} M.N. Huxley and G. Kolesnik,\pageoriginale Exponential sums and the Riemann zeta-function III, Proc. London Math. Soc. (3), 62 (1991), no.~3, 449--468.

\bibitem{Ingham1} A.E. Ingham, Mean value theorems in the theory of the Riemann zeta-function, Proc. London Math. Soc. (2) 27(1926), 273-300.

\bibitem{Ivic1} A. Ivi\'c, The Riemann zeta-function, John Wiley \& Sons, New York, 1985.

\bibitem{Ivic2} A. Ivi\'c, Sums of products of certain arithmetical functions, Publs. Inst. Math. Belgrade 41(55) (1987), 31-41.

\bibitem{Ivic3} A. Ivi\'c, On some integrals involving the mean square formula for the Riemann zeta-function, Publs. Inst. Math. Belgrade 46(60) (1989), 33-42.

\bibitem{Ivic4} A. Ivi\'c, Large values of certain number-theoretic error terms, Acta Arith. 56(1990), 135-159.

\bibitem{Ivic and Motohashi1} A. Ivi\'c and Y. Motohashi, A note on the mean value of the zeta and $L$-functions VII, Proc. Japan Acad. Ser. A 66(1990), 150-152.

\bibitem{Ivic and Motohashi2} A. Ivi\'c and Y. Motohashi,      On the fourth power moment of the Riemann zeta-function, J. Number Theory 51 (1995), no.~1, 16--45.

\bibitem{Ivic and Perelli1} A. Ivi\'c and A. Perelli, Mean values of certain zeta-functions on the critical line, Litovskij Mat. Sbornik 29(1989), 701-714.

\bibitem{Iwaniec1} H. Iwaniec, Fourier coefficients of cusp forms and the Riemann zeta-function, Expos\'e No. 18, S\'eminaire de Th\'eorie des Nombres, Universit\'e Bordeaux, 1979/80.

\bibitem{Iwaniec2} H. Iwaniec, On mean values for Dirichlet's polynomials and the Riemann zeta-function, J. London Math. Soc. (2) 22(1980), 39-45.

\bibitem{Iwaniec3} H. Iwaniec, Non-holomorphic modular forms and their applications, in ``Modular forms'' (R.A. Rankin ed.), Ellis Horwood, West Sussex, 1984, pp. 157-197.

\bibitem{Iwaniec4} H. Iwaniec, Promenade along modular forms and analytic number theory, in ``Topics in analytic number theory'' (S.W. Graham and J.D. Vaaler eds.), University of Texas Press, Austin 1985, 221-303.

\bibitem{Joyner1} D. Joyner,\pageoriginale Distribution theorems for $L$-functions, Longman Scientific \& Technical, Essex, 1986.

\bibitem{Jutila1} M. Jutila, Riemann's zeta-function and the divisor problem I, Arkiv Mat. 21(1983), 75--96 and II ibid.  31 (1993), no.~1, 61--70.

\bibitem{Jutila2} M. Jutila, On the value distribution of the zeta-function on the critical line, Bull. London Math. Soc. 15(1983), 513-518.

\bibitem{Jutila3} M. Jutila, Transformation formulae for Dirichlet polynomials, J. Number Theory 18(1984), 135-156.

\bibitem{Jutila4} M. Jutila, On the divisor problem for short intervals, Ann. Univ. Turkuensis Ser. A I 186(1984), 23-30.

\bibitem{Jutila5} M. Jutila, On a formula of Atkinson, in ``Coll. Math. Soc. J. Bolyai 34. Topics in classical number theory, Budapest 1981'', North-Holland, Amsterdam 1984, 807-823.

\bibitem{Jutila6} M. Jutila, On exponential sums involving the divisor function, J. Reine Angew. Math. 355(1985), 173-190.

\bibitem{Jutila7} M. Jutila, On the approximate functional equation for $\zeta^2(s)$ and other Dirichlet series, Quart. J. Math. Oxford (2) 37 (1986), 193-209.

\bibitem{Jutila8} M. Jutila, Remarks on the approximate functional equation for $\zeta^2(s)$, in ``The very knowledge of coding'', Univ. of Turku, Turku, 1987, 88-98.

\bibitem{Jutila9} M. Jutila, A method in the theory of exponential sums, LN 80, Tata Institute of Fundamental Research, Bombay, 1987 (publ. by Springer Verlag, Berlin-New York).

\bibitem{Jutila10} M. Jutila, The fourth power moment of the Riemann zeta-function over a short interval, in ``Coll. Math. Soc. J. Bolyai 54. Number Theory, Budapest 1987'', North-Holland, Amsterdam 1989, 221-244.

\bibitem{Jutila11} M. Jutila, Mean value estimates for exponential sums, in ``Number Theory, Ulm 1987'', LNM 1380, Springer Verlag, Berlin-New York 1989, 120-136, and II, Arch. Math. 55(1990), 267-274.

\bibitem{Jutila12} M. Jutila, Mean value estimates for exponential sums with applications to $L$-functions, Acta Arith. 57(1991), no.~2, 93--114.

\bibitem{Jutila13} M. Jutila,\pageoriginale Transformations of exponential sums, Proc. Symp. Anal. Number Theory, (Maiori, 1989), 263--270, Univ. Salerno, Salerno, 1992.

\bibitem{Katai1} I. K\'atai, The number of lattice points in a circle (Russian), Ann. Univ. Sci. Budapest Rolando E\"otv\"os, Sect, Math. 8(1965), 39-60.

\bibitem{Kober1} H. Kober, Eine Mittelwertformel der Riemannschen Zetafunktion, Compositio Math. 3(1936), 174-189.

\bibitem{Kratzel1} E. Kr\"atzel, Lattice points, VEB Deutscher Verlag der Wissenschaften, Berlin, 1988.

\bibitem{Kuznetsov1} N.V. Kuznetsov, The distribution of norms of primitive hyperbolic classes of the modular group and asymptotic formulas for the eigenvalues of the Laplace-Beltrami operator on a fundamental region of the modular group, Soviet Math. Dokl. 10 (1978), 1053-1056.

\bibitem{Kuznetsov2} N.V. Kuznetsov, Petersson's conjecture for cusp forms of weight zero and Linnik's conjecture. Sums of Kloosterman sums (Russian), Mat. Sbornik 111(1980), 335-383.

\bibitem{Kuznetsov3} N.V. Kuznetsov, Mean value of the Hecke series associated with cusp forms of weight zero (Russian), Zap. Nau\v cn. Sem. LOMI 109 (1981), 93-130.

\bibitem{Kuznetsov4} N.V. Kuznetsov, Convolution of the Fourier coefficients of the Eisenstein-Maass series (Russian), Zap. Nau\v cn. Sem. LOMI 129 (1983), 43-84.

\bibitem{Kuznetsov5} N.V. Kuznetsov, Sums of Kloostermann sums and the eighth power moment of the Riemann zeta-function, in ``Number theory and related topics'', Tata Institute of Fundamental Research, published by Oxford University Press, Oxford 1989, pp. 57-117.

\bibitem{Landau1} E. Landau, \"Uber die Gitterpundte in einem Kreise, G\"ottinger Nachrichten 1923, 58-65.

\bibitem{Laurincikas1} A. Laurin\v cikas, On the zeta-function of Riemann on the critical line (Russian), Litovskij Mat. Sbornik 25(1985), 114-118.

\bibitem{Laurincikas2} A. Laurin\v cikas,\pageoriginale On the moments of the zeta-function of Riemann on the critical line (Russian), Mat. Zametki 39(1986), 483-493.

\bibitem{Laurincikas3} A. Laurin\v cikas, The limit theorem for the Riemann zeta-function on the critical line I(Russian), Litovskij Mat. Sbornik 27 (1987), 113-132; II. ibid. 27(1987), 489-500.

\bibitem{Laurincikas4} A. Laurin\v cikas, A limit theorem for Dirichlet $L$-functions on the critical line (Russian), Litovskij Mat. Sbornik 27(1987), 699-710.

\bibitem{Linnik1} Y.V. Linnik, Additive problems and eigenvalues of the modular operators, Proc. Internat. Congress Math. Stockholm (1962), 270-284.

\bibitem{Maass1} H. Maass, \"Uber eine neue Art von nichtanalytischen automorphen Funktionen und die Bestimmung Dirichletscher Reihen durch Funktionalgleichungen, Math. Ann. 121(1949), 141-183.

\bibitem{Matsumoto1} K. Matsumoto, The mean square of the Riemann zeta-function in the critical strip, Japan. J. Math. New Ser. 15(1989), 1-13.

\bibitem{Meurman1} T. Meurman, The mean twelfth power of Dirichlet $L$-functions on the critical line, Annal. Acad. Sci. Fennicae Ser. A Math., Dissertation, Helsinki, 1984.

\bibitem{Meurman2} T. Meurman, A generalization of Atkinson's formula to $L$-functions, Acta Arith. 47(1986)m 351-370.

\bibitem{Meurman3} T. Meurman, On the mean square of the Riemann zeta-function, Quart. J. Math. (Oxford) (2) 38(1987), 337-343.

\bibitem{Meurman4} T. Meurman, On exponential sums involving Fourier coefficients of Maass wave forms, J. Reine Angew. Math. 384(1988), 192-207.

\bibitem{Meurman5} T. Meurman, Identities of the Atkinson type for $L$-functions, unpublished manuscript.

\bibitem{Meurman6} T. Meurman, On the order of the Maass $L$-function on the critical line. Coll. Math. Soc. J. Bolyai 51. Number Theory, Budapest 1987, North-Holland, Amsterdam, 1989, 325-354.

\bibitem{Montgomery and Vaughan1} H.L. Montgomery and R.C. Vaughan, Hilbert's inequality, J. London Math. Soc. (2) 8(1974), 73-82.

\bibitem{Motohashi1} Y. Motohashi, An asymptotic series for an additive divisor problem, Math. Zeit. 170(1980), 43-63.

\bibitem{Motohashi2} Y. Motohashi,\pageoriginale A note on the approximate functional equation for $\zeta^2(s)$, Proc. Japan Acad. Ser. 59 A (1983), 392-396; II ibid. 59 A (1983), 469-472; III ibid. 62 A (1986), 410-412.
 
\bibitem{Motohashi3} Y. Motohashi, A note on the mean value of the zeta and $L$-functions I, Proc. Japan Acad. Ser. 61 A (1985), 222-224; II ibid. 61 A (1985), 313-316; III ibid. 62 A (1986), 152-154; IV ibid. 62 A (1986), 311-313; V ibid. 62 A (1986), 399-401; VI 65 A (1989), 273-275.

\bibitem{Motohashi4} Y. Motohashi, Riemann-Siegel formula, Lecture Notes, University of Colorado, Boulder, 1987.

\bibitem{Motohashi5} Y. Motohashi, A spectral decomposition of the product of four zeta-values, Proc. Japan Acad. Ser. 65 A (1989), 143-146.

\bibitem{Motohashi6} Y. Motohashi, An explicit formula for the fourth power mean of the Riemann zeta-function, Acta Math. 170 (1993), no.~2, 181--220.

\bibitem{Motohashi7} Y. Motohashi, Spectral mean values of Maass waveform $L$-functions, J. Number Theory 42 (1992), no.~3, 258--284.

\bibitem{Motohashi8} Y. Motohashi, The mean square of $\zeta(s)$ off the critical line, unpublished manuscript.

\bibitem{Ogg1} P. OGG, On a convolution of $L$-series, Invent. Math. 7(1969), 297-312.

\bibitem{Oppenheim1} A. Oppenheim, Some identities in the theory of numbers, Proc. London Math. Soc. (2) 26(1927), 295-350.

\bibitem{Panteleeva1} E.I. Panteleeva, On the divisor problem of Dirichlet in number fields (Russian), Mat. Zametki 44(1988), 494-505.

\bibitem{Preissmann1} E. Preissmann, Sur une inegalit\'e de Montgomery-Vaughan, Enseignement Math. 30(1984), 95-113.

\bibitem{Preissmann2} E. Preissmann, Sur la moyenne quadratique du terme de reste du probl\`eme du cercle, C.R. Acad. Sci. Paris S\'er. I 303 (1988), 151-154.

\bibitem{Ramachandra1} K. Ramachandra,\pageoriginale Application of a theorem of Montgomery-Vaughan to the zeta-function, J. London Math. Soc. (2) 10(1975), 482-486.

\bibitem{Ramachandra2} K. Ramachandra, On the frequency of Titchmarsh's phenomenon for $\zeta(s)$ II, Acta Math. Sci. Hung. 30(1977), 7-13; VII. Ann. Acad. Sci, Fenn. Ser. A Math. 14(1989), 27-40;Hardy-R.J.13(1990), 28-33.

\bibitem{Ramachandra3} K. Ramachandra, Some remarks on the mean value of the Riemann zeta-function anf other Dirichlet series I, hardy-Ramanujan J. 1(1978), 1-15; II. ibid. 3(1980), 1-24; III. Annales Acad. Sci. Fenn. Ser. A Math. 5(1980), 145-158.

\bibitem{Ramachandra4} K. Ramachandra, Some remarks on a theorem of Montgomery and Vaughan, J. Number Theory 11(1979), 465-471.

\bibitem{Ramachandra5} K. Ramachandra, Progress towards a conjecture on the mean value of Titchmarsh series, in ``Recent progress in analytic number theory. Symposium Durham 1979 (Vol. 1)'', Academic Press, London 1981. pp. 303-318; II. Hardy-Ramanujan J. 4(1981), 1-12; III. (with R. Balasubramanian) Acta Arith. 45(1986), 309-318. 

\bibitem{Ramachandra6} K. Ramachandra, Mean values of the Riemann zeta-function and other remarks I, in ``Coll. Math. Soc. J. Bolyai 34. Topics in classical number theory Budapest 1981'', North-Holland, Amsterdam. 1984, pp. 1317-1347; II. in Proc. I.M. Vinogradov Conference (Moscow, 1981), Trudy Mat. Inst. AN SSSR 163(1984), 200-204; III. Hardy-Ramanujan J. 6(1983), 1-21.

\bibitem{Ramachandra7} K. Ramachandra, Titchmarsh series, in ``Number Theory (eds. J. M. De Koninck and C. Levesque), Qu\'ebec 1987'', Walter de Gruyter, Berlin-New York, 1989, pp. 811-814.

\bibitem{Ramachandra8} K. Ramachandra, Proof of some conjectures on the mean-value of Titchmarsh series with application to Titchmarsh's phenomenon, Hardy-Ramanujan J. 13(1990), 21-27.

\bibitem{Ramachandra and Sankaranarayanan1} K. Ramachandra and A. Sankaranarayanan, On some theorems of Littlewood and Selberg I, J. Number Theory 44(1993), no.~3, 281--291.

\bibitem{Ramanujan1} S. Ramanujan, Collected Papers, Chelsea, New York, 1962.

\bibitem{Ram Murthy1} M. Ram Murth,\pageoriginale Oscillations of Fourier coefficients of modular forms, Math. Ann. 262(1983), 431-446.

\bibitem{Rankin1} R.A. Rankin, Contributions to the theory of Ramanujan's function $\tau(n)$ and similar arithmetical functions II. The order of the Fourier coefficients of integral modular forms, Proc. Cambridge Phil. Soc. Math. 35(1939), 357-372.

\bibitem{Richert1} H.-E. Richert. Zur Absch\"atzung der Riemannschen Zetafunktion in der N\"ahe der Vertikalen $\sigma =1$, Math. Ann. 169(1967), 97-101.

\bibitem{Selberg1} A. Selberg, Bemerkungen \"uber eine Dirichletsche Reihe, die mit der Theorie der Modulformen nahe verbunden ist, Archiv Math. Naturvidenskab B 43(1940), 47-50.

\bibitem{Selberg2} A. Selberg, On the estimation of Fourier coefficients of modular forms, Proc. Symp. Pure Math. VIII, AMS, Providence, R.I, 1965, pp. 1-15.

\bibitem{Shiu1} P. Shiu, A Brun-Titchmarsh theorem for multiplicative functions, J. Reine Angew.Math. 31(1980), 161-170.

\bibitem{Siegel1} C.L. Siegel, \"Uber Riemann's Nachlass zur analytischen Zahlen-theorie, Quell. Stud. Gesch. Mat. Astr. Physik 2(1932), 45-80.

\bibitem{Srinivasan1} S. Srinivasan, A footnote to the  large sieve, J. Number Theory 9(1977), 493-498.

\bibitem{Terras1} A. Terras, Harmonic analysis on symmetric spaces and applications I, Springer Verlag, Berlin etc., 1985.

\bibitem{Titchmarsh1} E.C. Titchmarsh, The theory of the Riemann zeta-function (2nd ed. revised by D.R. Heath-Brown), Clarendon Press, Oxford, 1986.

\bibitem{Tong1} K.-C. Tong, On divisor problems III, Acta Math. Sinica 6(1956), 515-541.

\bibitem{Vinogradov1} A.I. Vinogradov, Poincar\'e series on $SL(3,\mathbb{R})$ (Russian), Zap. Nau\v cn. Sem. LOMI 160(1987), 37-40.

\bibitem{Vinogradov2} A.I. Vinogradov,\pageoriginale Analytic continuation of $\zeta_3(s,k)$ to the critical strip. Arithmetic part (Russian), Zap. nau\v cn. Sem. LOMI 162(1987), 43-76.

\bibitem{Vinogradov3} A.I. Vinogradov, The $SL_n$-technique and the density hypothesis (Russian), Zap. Nau\v cn. Sem. LOMI 168(1987), 5-10.

\bibitem{Vinogradov and Tahtadzjan1} A.I. Vinogradov and L.A. Tahtad\v zjan, Theory of Eisenstein series for the group $SL(3,\mathbb{R})$ and its application to a binary problem (Russian), Zap. Nau\v cn. Sem. LOMI 76(1978), 5-53.

\bibitem{Vinogradov and Tahtadzjan2} A.I. Vinogradov and L.A. Tahtad\v zjan, An estimate of the residue of Rankin's $L$-series, Soviet Math. Dokl. 26(1982), 528-531.

\bibitem{Voronoi1} G.F. Voronoi, Sur une fonction transcendante et ses applications \`a la sommation de quelques s\'eries, Ann. \'Ecole Normale 21(3) (1904), 207-268; 21(3) (1904), 459-534.

\bibitem{Voronoi2} G.F. Voronoi, Sur le d\'eveloppement, \`a l'\'etude des fonctions cylindriques, des sommes doubles $\sum f (pm^2 + 2 q mn + rn^2)$, o\`u $pm^2 + 2 q mn + rn^2$ est une forme quadratique \`a coefficients entiers (Verh. III Math. Kng. Heidelberg), Teubner, Leipzig, 1905, pp. 241-245.

\bibitem{Watson1} G.N. Watson, A treatise on the theory of Bessel functions, 2nd ed., Cambridge University Press, Cambridge, 1944.

\bibitem{Watt1} N. Watt, Exponential sums and the Riemann zeta-function, II, J. London Math. Soc. (2) 39(1989), 385-404.

\bibitem{Weil1} A. Weil, On some exponential sums, Proc, National Acad. Sci. USA 34(1948), 204-207.

\bibitem{Zavorotnyi1} N.I. Zavorotnyi, On the fourth moment of the Riemann zeta-function (Russian), Automorphic functions and number theory, Part I, II (Russian), 69--124a, 254, Akad. Nauk SSSR, Dalʹnevostochn. Otdel., Vladivostok, 1989.

\end{thebibliography}
