\setcounter{chapter}{-1}
\chapter{Preliminaries}\label{chap0}%chapter 0

In\pageoriginale this introductory Chapter we recall,
\begin{enumerate}[A)]
\item some basic definitions and standard results in Geometric
  Invariant Theory; 

\item the definition of a Hilbert point of a curve; 

\item the definition of a Hilbert scheme;

\item the definition and simple properties of a stable curve;

\item some basic definitions and standard results in Deformation
  theory. 
\end{enumerate}

\medskip
\noindent{\textbf{A) ~Geometric Invariant Theory}

Let $G$ be a reductive algebraic group acting on an algebraic scheme
$X$. It is natural to ask whether $X$ has a quotient by $G$, which is
reasonably good, say, in the sense of the following definition. 

\setcounter{subdefin}{-1}
\begin{subdefin}\label{chap0:subdef0.0.0} % def 0.0.0
In the above situation a {\em{good quotient}} of $X$ by $G$ is a
morphism $f : X \rightarrow Y$ of algebraic schemes, satisfying, 
\begin{enumerate}[i)]
\item $f$ is surjective, affine and $G$-invariant (i.e. $f(gx) = f(x)$
  for all $g \in G$, $x \in X$); 

\item $f_*(O_X)^G = O_Y$, $(f_*(O_X)$ is the direct image of $O_X$ and
  $f_*(O_X)^G$ is the sheaf of $G$-invariants in $f_*(O_X))$; 

\item if\pageoriginale $F$ is a $G$-invariant closed subset of $X$
  then $f(F)$ is 
  closed in $Y$ and if $F_1$ and $F_2$ are $G$-invariant closed subsets
  of $X$ such that $F_1 \cap F_2 = \phi $ then $f(F_1) \cap f(F_2)
  = \phi$. 
\end{enumerate}
\end{subdefin}

\begin{subdefin}\label{chap0:subdef0.0.1}% definition 0.0.1
With the same notations as above, a {\em{geometric quotient}} of $X$
by $G$ is a morphism $f : X~ \rightarrow ~Y$ of algebraic schemes,
satisfying, 
\begin{enumerate}[i)]
\item $f$ is a good quotient of $X$ by $G$; 

\item for every $y$ in $Y$ the fibre $f^{-1}(y)$ is exactly one
  orbit. (In particular the orbits are closed). 
\end{enumerate}
\end{subdefin}

It is easy to see that a quotient (good or geometric) is unique up to
isomorphism (if it exists). 

\setcounter{subexample}{1}
\begin{subexample}% exmaple 0.0.2
Consider the natural action of $GL(N)$ on affine $N$-space
$\mathbb{A}^N$. Clearly $\mathbb{A}^N -\{0\}$ is a single orbit in
$\mathbb{A}^N$ which is not closed. Hence a geometric quotient of
$\mathbb{A}^N$ by $GL(N)$ does not exists. 

Now suppose that $X \subset \mathbb{P}^N$ is a projective algebraic
scheme and $G$ is a reductive algebraic group acting on $X$ via a
representation $\varphi : G \rightarrow GL(N+1)$. 
\end{subexample}

\setcounter{subdefin}{2}
\begin{subdefin}\label{chap0:subdef0.0.3}% subdefin 0.0.3
In the above situation a point $x \in X$ is called
{\bf semi\-stable} if there exists a non constant $G$-invariant
homogeneous polynomial $F$ such $F(x) \neq 0$. 

Put\pageoriginale $X^{ss} = \{x \in X | x$ is semistable
$\}$. Clearly $X^{ss}$   is open in $X$.  
\end{subdefin}

\begin{subdefin}\label{chap0:subdef0.0.4}%definition 0.0.4
With the same notation as above, a point $x \in X$ is called
{\em{ stable }}, if, 
\begin{enumerate}[{\rm i)}]
\item $\dim 0(x) = \dim G$, ($0(x)$ denotes the orbit of $x$); 

\item there exists a non constant $G$-invariant homogeneous polynomial
  $F$ such that $F(x) \neq 0$ and for every $y_0$ in $X_F = \{y
  \in X | F(y) \neq 0 \}$, $0(y_0)$ is closed in $X_F$. 
\end{enumerate}

Put $X^s = \{x \in X|x$  is stable $\}$. Note that the set
$\{x \in X | \dim (0(x)) = \dim G\}$ is open $X$ because
$\dim(0(x))$ is a lower semicontinuous function of $x$. Now it is
immediate that $X^s$ is open in $X$. Both $X^{ss}$ and $X^s$ can be
empty, however in the case when they are non empty we have the
following theorem. 
\end{subdefin}

\setcounter{subtheorem}{4}
\begin{subtheorem}\label{chap0:subthm0.0.5}% Theorem 0.0.5
There exists a projective algebraic scheme $Y$ and a morphism $f_{ss}
: X^{ss} \rightarrow Y$ such that $f_{ss}$ is a good quotient of
$X^{ss}$ by $G$. Further there exists an open subset $U$ of $Y$ such
that $f^{-1}_{ss}(U) = X^s$ and $f_s : X^s \rightarrow U$ is a
geometric quotient of $X^s$ by $G$. 

There is a test for semistability using one parameter subgroups.
\end{subtheorem}

\setcounter{subdefin}{5}
\begin{subdefin}\label{chap0:subdef0.0.6}% subdefin 0.0.6
Let $G$ be an algebraic group. A {\em one parameter subgroup} $\lambda
$ {\em{(abbreviated as 1- $ps\lambda$)}} of $G$ is defined to be a
nontrivial homomorphism\pageoriginale $\lambda : G_m \rightarrow G$ of
algebraic groups.  

Let $G$ be a reductive algebraic group acting on a projective
algebraic scheme $X \subset \mathbb{P}^N$ via a representation
$\varphi : G \rightarrow GL(N+1) $. Given a $1-ps \lambda$ of $G$,
there is an induced action of $\lambda$ on the affine $(N+1)$-space
$\mathbb{A}^{N+1}$. This action can be diagonalized, i.e., there
exists a basis $e_0,e_1,\ldots,e_N$ of (the vector spacer)
$\mathbb{A}^{N+1}$ such that the action of $\lambda$ on
$\mathbb{A}^{N+1}$ is given by $\lambda(t)e_i = t^{r_i}e_i$, $t
\in K^*$, $r_i \in \mathbb{Z}$, ($0 \le i \le
N$). Let $x = \sum\limits_{i=0}^N x_i e_i$ be a point in
$\mathbb{A}^{N+1} - \{0\}$, $(x_i \in K, 0 \le i \le N)$. 
Clearly $\lambda (t)x = \sum\limits_{i=0}^N x_i e_i$. The point
$x \in \mathbb{A}^{N+1} - \{0\}$ represents a point, say
$\bar{x}$, in $\mathbb{P}^N$ 
\end{subdefin}

\begin{subdefin}\label{chap0:subdef0.0.7}% subdefin 0.0.7
With $\lambda$ and $x$ as above we define $\mu (\bar{x}, \lambda) = -
\max$. \break $\{r_i |x_i \neq 0\}$. 

It can be shown that $\mu(\bar{x}, \lambda)$ is independent of the
basis $e_0,e_1,\ldots,e_N$ and the point $x$, so that the above
definition makes sense. 
\end{subdefin}

\begin{subdefin}\label{chap0:subdef0.0.8}% subdefin 0.0.8
With the same notations as above a point $\bar{x} \in X$ is
called $\lambda$-{\em{semistable (respectively $\lambda$ -stable)}} if
$\mu(\bar{x}, \lambda) \le 0$ (respectively $\mu(\bar{x},\lambda) <
0$). 

Semistability (respectively stability) and $\lambda$-semistability
(respecti\-vely $\lambda$-stability) of a point $\bar{x}$ are related by
the following theorem. 
\end{subdefin}

\setcounter{subtheorem}{8}
\begin{subtheorem}\label{chap0:subthm0.0.9}% theorem 0.0.9
With\pageoriginale the same notations as above, $\bar{x}$ is semistable
$\Longleftrightarrow \bar{x}$ is $\lambda$-semistable for every
$1-ps\lambda$ of $G$, and $\bar{x}$ is stable $\Longleftrightarrow
\bar{x}$ is $\lambda$-stable for every $1-ps\lambda$ of $G$. 
 
It follows from the above theorem that to show that a point
$\bar{x} \in X$ is not semistable it suffices to find a single
$1-ps\lambda$ of $G$ such that $\bar{x}$ is not $\lambda$-semistable. 

The proofs of the results in this section can be found in \cite{key5}.
\end{subtheorem}

\medskip
\noindent{\textbf{B) ~Hilbert point of a curve}}

Let $X \subset \mathbb{P}^N$ be a complete curve. Let $L$ be the
restriction of $O_{\mathbb{P}^N}(1)$ to $X$. Recall that $\chi(L^m) =
h^o(X,L^m)-h^l(X,L^m)$ is a polynomial in $m$, say $P(m)$. 

By Serre's vanishing theorem there exists an integer $m'$ such that
all integers $m > m'$, $H^l(H,L^m) = 0$ and the restriction 
$$
\varphi_m : H^o (\mathbb{P}^N, O_{\mathbb{P}^N}(m)) \rightarrow
H^o(X,L^m) \text{ is surjective}. 
$$

Assume now that $m > m'$. Taking the $P(m)^{\rm th}$ exterior powers, we
get, 
$$
\overset{P(m)}{\Lambda} \varphi_m : \overset{P(m)}{\Lambda} 
H^o(\mathbb{P}^N,O_{\mathbb{P}^N}(m)) \rightarrow
\overset{P(m)}{\Lambda} H^o(X,L^m) \simeq K, 
$$
a point in the projective space $\mathbb{P}(\overset{P(m)}{\Lambda}
H^o(\mathbb{P}^N,O_{\mathbb{P}^N}(m)))$. (For a vector space $V$,
$\mathbb{P}(V)$ denotes the projective space associated to $V$, in the
sense of Grothendieck i.e. $\mathbb{P}(V)$ is the space consisting of
equivalence classes of nonzero linear forms on $V$.) 


\setcounter{subsection}{1}
\setcounter{subdefin}{-1}
\begin{subdefin}\label{chap0:subdef0.1.0}%subdefin 0.1.0
In \pageoriginale the above situation the point \\
\noindent
$\overset{P(m)}{\Lambda} \varphi_m
\in \mathbb{P} (\overset{P(m)}{\Lambda}
H^o(\mathbb{P}^N, O_{\mathbb{P}^N}(m)))$ is defined to be the
{\em{$m^{\rm th}$ Hilbert point of a curve $X$}}. 

Choose a basis $X_0,X_1,\ldots,X_N$ of
$H^o(\mathbb{P}^N, O~_{\mathbb{P}^N}(1))$. Consider the action of
$GL(N+1)$ (and hence of $SL(N+1)$) on $H^o
(\mathbb{P}^N, O_{\mathbb{P}^N}(1))$, defined by, 
$$
[a_{ij}]. X_P = \sum_{j=0}^{N}
a_{pj}X_j, [a_{ij}] \in GL(N+1),\quad (0\le p \le N). 
$$

\label{c0:l(n+1)}
We have an induced action of $SL(N+1)$ on
$\mathbb{P}(\overset{P(m)}{\Lambda} H^o(\mathbb{P}^N,
O_{\mathbb{P}^N}(m))$ described as follows.  

Recall that $H^o(\mathbb{P}^N,O_{\mathbb{P}^N} (m))$ has the basis
$B_m = \{M_1,M_2,\ldots,M_{\alpha_m} \}$ consisting of monomials of
degree $m$ in $X_0, X_1,\ldots,X_N$, $(\alpha_m = h^o (\mathbb{P}^N,\break
O_{\mathbb{P}^{N}}(m))$. $SL(N+1)$ acts on $H^o
(\mathbb{P}^N,O_{\mathbb{P}^N}(m))$, with the action given by, 
\begin{align*}
& g.X^{\gamma_o}_o X^{\gamma_1}_l \cdots X^{\gamma_N}_N =
  H^{\gamma_o}_o H^{\gamma_l}_l \cdots H^{\gamma_N}_N,\\ 
& (X^{\gamma_o}_o X^{\gamma_l}_l \cdots X^{\gamma_N}_N \in
  B_m, g \in  SL(N+1), H_p = g.X_p, \; 0\le p \le N). 
\end{align*}

Hence there is an action of $SL(N+l)$ on $\overset{P(m)}{\Lambda}
H^o(\mathbb{P}^N, O_{\mathbb{P}^N}(m))$, as follows. Recall that
$M_{i_1} \Lambda M_{i_2} \Lambda \cdots \Lambda M_{i_{P(m)}} \quad (1
\le i_1 < i_2 < \cdots < i_{P(m)} \le \alpha_m)$ is a basis of
$\overset{P(m)}{\Lambda} H^o(\mathbb{P}^N,O_{\mathbb{P}^N}(m))$. The
action of $SL(N+l)$ on this space is\pageoriginale given by, 
$$
g \cdot (M_{i_1} \Lambda M_{i_2} \Lambda \dots \Lambda M_{i_{P(m)}}) =
gM_{i_1} \Lambda  gM_{i_2} \Lambda \dots \Lambda
gM_{i_{P(m)}}, (g \in SL(N+1)). 
$$

Take the dual action $\overset{P(m)}{\Lambda}
H^o(\mathbb{P}^N,O_{\mathbb{P}^N}(m))^*$ which naturally gives an
action of $SL(N+1)$ on $\mathbb{P}(\overset{P(m)}{\Lambda}
H^o(\mathbb{P}^N,O_{\mathbb{P}^N}(m)))$. 

Let $\lambda$ be a $1-ps$ of $SL(N+1)$. When is the point $H_m(X)
\in~\mathbb{P}(\overset{P(m)}{\Lambda}\break
H^o(\mathbb{P}^N,O_{\mathbb{P}^N}(m)))\lambda$ - semistable? We try to
answer this question. 

There exists a basis $w_0,w_1,\ldots,w_N$ of
$H^o(\mathbb{P}^N,O_{\mathbb{P}^N}(1))$ and integers
$r_0,r_1,\ldots,r_N$ such that the action of $\lambda$ on
$H^o(\mathbb{P}^N,O_{\mathbb{P}^N}(1))$ is given by  
$$
 \lambda(t)w_i = t^{r_i}w_i, t \in K^*, \quad (0\le i \le N).  
$$

Let $B'_m = \{M'_1,M'_2,\ldots,M'_{\alpha_m}\}$ be a basis of 
$H^o(\mathbb{P}^N,O_{\mathbb{P}^N}(m))$ consisting of monomials of
degree $m$ in $w_0,w_1,\ldots,w_N$. In this situation we make the
following definition. 
\end{subdefin}

\begin{subdefin}\label{chap0:subdef0.1.1}%subdefin 0.1.1
For a monomial $M =
w_0^{\gamma_0}~w_1^{\gamma_1},\ldots,w_N^{\gamma_N}$, define its {\em
  $\lambda$-weight $w_\lambda(M)$}, by $w_\lambda(M) =
\sum\limits_{i=0}^{N} ~\gamma_i r_i$ and define, {\em total
  $\lambda$-weight of monomials} $M''_1,M''_2,\ldots,M''_t$ to be
$\sum\limits_{i=0}^{t}~w_\lambda (M''_i)$. 
\end{subdefin}

The\pageoriginale vector space $\overset{P(m)}{\Lambda}
H^o(\mathbb{P}^N, \, O_{\mathbb{P}^N}(m))$ has the following basic, 
$$
\bigg\{M'_{i_1}~\Lambda~M'_{i_2}~\Lambda \ldots
\Lambda~M'_{i_{P(m)}}\bigg\}_{(1 ~\le ~i_1~ < ~i_2 ~<~ \ldots
  ~i_{P(m)} ~\le~ \alpha_{m})}  
$$

\noindent
Let $\{M'_{i_{1}}~\Lambda~M'_{i_{2}}~\Lambda \ldots 
\Lambda~{M'}^*_{i_{P(m)}} \}_{(1~\le ~i_1~ < ~i_2 ~<~ \ldots
    ~i_{P(m)}  ~\le~ \alpha_{m})}$ be the basis of
  $\overset{P(m)}{\Lambda} 
H^o(\mathbb{P}^N, \; O_{\mathbb{P}^N}(m))^*$ dual to the above basis of
$\overset{P(m)}{\Lambda} H^o(\mathbb{P}^N,O_{\mathbb{P}^N}(m))$. 

\noindent
The action of $\lambda$ on $\overset{P(m)}{\Lambda}
H^o(\mathbb{P}^N, \; O_{\mathbb{P}^N}(m))^*$ is given by, 
\begin{align*}
\lambda(t) ~(M'_{i_1}~\Lambda~M'_{i_2}~\Lambda \ldots
\Lambda~{M'}^*_{i_{P(m)}}) & =
t^{-\theta}~(M'_{i_1}~\Lambda~M'_{i_2}~\Lambda \ldots
\Lambda~ {M'}^*_{i_{P(m)}}), \\
&\qquad \qquad t ~\in~ K^*, ~\theta = \sum_{j=1}^{P(m)} w_\lambda
(M'_{i_j}).  
\end{align*}

Write $H_m(X)$ as a linear combination of the vectors in the above
basis of $\overset{P(m)}{\Lambda}
H^o(\mathbb{P}^N,O_{\mathbb{P}^{N}}(m))^*$. 
$$
H_m(X) = \sum\overset{P(m)}{\Lambda}~\varphi(M'_{i_1} ~\Lambda \ldots
\Lambda ~M'_{i_{P(m)}})~(M'_{i_1}~\Lambda~M'_{i_2}~ \Lambda
~\ldots~\Lambda ~M'^*_{i_{P(m)}}) 
$$

\noindent
By definition

\noindent
$H_m(X)$  is $\lambda$-semistable (respectively $\lambda$-stable)\\ 
$\Longleftrightarrow \;\; \mu(H_m(X),\lambda) \le 0$ (respectively
   $< 0$)\\ 
$\Longleftrightarrow - \max.\left\{ - \sum_{j=1}^{P(m)}~w_\lambda
  (M'_{i_j})\right\} \le 0$ (respectively $< 0$), \\ 
where\pageoriginale the maximum is taken over all
$(i_1,i_2,\ldots,i_{P(m)})$ with 
$1 \le i_1 < i_2 < \ldots < i_{P(m)} \le \alpha_m$, such that
$\overset{P(m)}{\Lambda}~\varphi_m ~ (M'_{i_1}~\Lambda
M'_{i_2}\Lambda\ldots \Lambda 
~ M'_{i_{P(m)}})~\neq 0$. Clearly  
$$
- \max.\left\{ - \sum_{j=1}^{P(m)}~w_\lambda~(M'_{i_j}) \right\} =
\min.\left\{ \sum_{j=1}^{P(m)}~w_\lambda (M'_{i_j})\right\}. 
$$

\noindent
Thus we have the following criterion.

\label{c0:*}
\medskip
$(*)$ In the above situation $H_m(X)$ is $\lambda$-semistable 
(respectively $\lambda$-stable) $\Longleftrightarrow$ There exist
monomials $M'_{i_1}, M'_{i_2}, \ldots, M'_{i_{P(m)}}$, $(1 \le i_1 < i_2
< \ldots < i_{P(m)} \le \alpha_m)$, in $B'_m$ such that
$\varphi_m(M'_{i_1})$,
$\varphi_m(M'_{i_2}),\ldots,\varphi_m(M'_{i_{P(m)}})$ is a basis of
$H^o(X,L^m)$ and $\sum\limits_{j=1}^{P(m)}~w_\lambda~(M'_{i_j}) \le 0$
(respectively $< 0$). 

Let $\lambda$ be a $1-ps$ of $GL(N+1)$. There exists a basis
$\{w_0, w_1, \ldots, w_N\}$ of $H^o(\mathbb{P}^N, O_{\mathbb{P}^N}(1))$
and integers $r_0, r_1, \ldots, r_N$ such that the induced action of
$\lambda$ on $H^o(\mathbb{P}^N, O_{\mathbb{P}^{N}}(1))$ is given by, 
$$
\lambda(t) ~w_i = t^{r_i}w_i, \quad ~t ~\in~K^*, (0 \le i \le N).
$$

\noindent
Put $\sum\limits_{i=0}^{N}~r_i = r$. Define a $1-ps\lambda'$ of
$SL(N+1)$ so that the action of $\lambda'$ on
$H^o (\mathbb{P}^N,O_{\mathbb{P}^N}(1))$ is given by  
$$
\lambda'(t)w_i = t^{r'_i}w_i ,t~\in~K^*,r'_i = (N+1)r_i - r,\;\;(0
\le i \le N). 
$$

\begin{subdefin}\label{chap0:subdef0.1.2} % subdefin 0.1.2
In\pageoriginale the above situation the $1 - ps \lambda'$ of $SL
(N+1)$ is said to 
be the $1-ps$ of $SL (N+1)$ associated to the  $1 - ps \lambda$  of
$GL (N+1)$. We want to rewrite the condition $(*)$ for $\lambda'
-$semistability (respectively $\lambda$-stability) of $H_m (X)$ in
terms  of $\lambda-$weights of the monomials.  
\end{subdefin}

Note that for a monomial $M \in H^o (\mathbb{P}^N,
O_{\mathbb{P}^{N}} (m))$, $w_{\lambda'} (M) = (N+1) w_{\lambda} (M)
-rm$. It follows that,   
\begin{align*}
& \sum\limits^{P(m)}_{j=1} w_{\lambda'} (M'_{i_{j}}) \leq (\text{
  respectively} < 0) \\
& \hspace{3cm}\Longleftrightarrow \\
& (N+1) \sum^{P(m)}_{j=1} w_{\lambda} (M'_{i_{j}}) -
P(m) \; {\rm rm} \leq 0 \;\; (\text{ respectively} < 0 ) \\
& \hspace{3cm}\Longleftrightarrow \\
& \sum^{P(m)}_{j=1}  ~ \frac{ w_{\lambda'} (M'_{i_{j}})}{mP (m)} \leq
\frac{r}{N+1}, ~~ (\text{ respectively} <  \frac{r}{N+1} )
\end{align*}

\noindent
Thus we have the following criterion

\medskip
$(**)$ With the same notations as above,\label{page10}

$ H_m (X) \in \mathbb{P} (\overset{P (m)}{\Lambda} H^o  (
\mathbb{P}^{N}, O_{\mathbb{P}^{N}} (m))) $  is $ \lambda' $
-semistable (respectively $\lambda'$ -stable) $ \Leftrightarrow$
  there exist monomials $M'_{i_{1}}, ~ M'_{i_{2}}  ~ ,\ldots ,
  M'_{i_{P (m)}}$,  
$ (1 \le i_1 < i_2 < \cdots < i_{P (m)} \le \alpha_m )$ in $B'_m$ such that
$ \left\{ \varphi_m  (M'_{i_{1}}),  \varphi_m  (M'_{i_{2}}),\ldots,
  \varphi_m  (M'_{i_{P(m)}})  \right\}$ is a basis of $H^o (X,L^m)$ and 
$\dfrac{\sum\limits^{P(m)}_{j=1} w_{\lambda'} (M'_{i_{j}})}{mP (m)}
  \leq \dfrac{r}{N+1} ,  ~  (\text{ respectively} <  \dfrac{r}{N+1})$. 


\bigskip

\medskip
\noindent{\textbf{C) ~Hilbert Scheme}}%section c

Consider\pageoriginale the projective space $\mathbb{P}^N$ over $ \Spec
\mathbb{Z}$. Look at all closed subschemes of $\mathbb{P}^N$, flat
over $\mathbb{Z}$, with a fixed  Hilbert polynomial say $P (m)$. A
fundamental existence theorem says  that there exists a scheme,
projective over $ \Spec \mathbb{Z}$, parametrizing all these closed
subschemes of $\mathbb{P}^N$. In fact we have the following stronger
version of the  theorem.\label{page11} 

Let Sch denote the category of locally noetherian schemes. Define a
functor $ \Hilb^P_{\mathbb{P}^N}$ form  Sch to the category of sets as
follows.  

For $S$ in Sch, $\Hilb^P_{\mathbb{P}^N} (S) $ = The set of all
closed subschemes  $W$ of  $\mathbb{P}^N \times S$, flat over  $S$
such that for every $s \in S$ the induced closed  subscheme
$W_s$ of $\mathbb{P}^{N}_{k(s)}$ has Hilbert polynomial $P (m)$. 

\setcounter{subtheorem}{2}
\begin{subtheorem}\label{chap0:subthm0.1.3}% theorem 0.1.3
The functor $\Hilb^P_{\mathbb{P}^{N}}$ is representable and is
represented by a scheme projective over spec $\mathbb{Z}$. 
\end{subtheorem}

Let $H$ denote the scheme representing the functor
$\Hilb^P_{\mathbb{P}^{N}}$. Thus for all $S$ in Sch,
$\Hilb^{P}_{\mathbb{P}^{N}} (S) \simeq \Hom (S,H)$. In particular  $
\Hilb^{P}_{\mathbb{P}^{N}} (H) \simeq \Hom (H,H)$. Let $Z$ be the
closed  subscheme of $ \mathbb{P}^N ~ \times H $ which corresponds to
the  identity morphism $ i \in \Hom (H,H)$, under the  above
isomorphism. We call $Z$, the universal closed subscheme. It has the
following universal property.     

Given\pageoriginale a scheme $S$ in Sch and  a scheme  $ Y \in
\Hilb^P_{\mathbb{P}^{N}} (S) $, there exists a unique morphism $f: S
\rightarrow H$ such that  $(1 \times f)^*  ~ Z \simeq Y$. 

A proof  of the above theorem and other details  can be fond in \cite{key2},
\cite{key6}. 

\bigskip

\medskip
\noindent{\textbf{D) ~Stable Curves  (in the sense of Deligne - Mumford
  (1))}}\label{c0:st_curves} %section D 


\setcounter{subdefin}{3}
\begin{subdefin}\label{chap0:subdef0.1.4}% subdefin 0.1.4
 Let  $S$ be any scheme. {\em A stable (respectively semista\-ble) 
   curve of genus} $g \ge 2$ over $S$ s a proper flat morphism  $ \pi :
 X \rightarrow S $ such that for all  $ s \in S$  the fibre
 $X_s$ of $\pi$ over $s$, satisfies,  
\begin{enumerate}[i)]
\item $X_s$  is  a reduced, connected scheme of dim 1 with $h^1
  (X_s, O_{X_{s}}) = g$;  

\item each singular point of $X_s$ is an ordinary double point; 

\item if $E$ is an irreducible component of $X_s$ such $E \simeq
  \mathbb{P}^1$ then $E$ meets the other component of $X_s$ in at
  least 3 points, (respectively 2 points). 
\end{enumerate}
\end{subdefin}

We now quote some results on stable curves which will be needed in the
sequel. 

\setcounter{subtheorem}{4}
\begin{subtheorem}\label{chap0:subthm0.1.5} % theorem0.1.5
If $\pi: C \rightarrow \Spec K $ is a stable curve then  $H^1 (C,
\omega^n_{C/K}) = 0 $ for  $n \ge 2$ and $\omega^n_{C/K}$ is very
ample for  $n \ge 3$,  ($\omega_{C/K}$ denotes the dualizing sheaf of
$X$). 
\end{subtheorem}

\noindent
We have the following consequences of the above theorem. 

Let\pageoriginale $\pi : X \rightarrow  S $ be a stable curve of genus
$g \ge 2$. It follows from the above theorem that  for all  $s \in S
$ and for  $ n \ge 2$, $H^1 (X_s, \omega^n_{X/S} \otimes O_{X_{s}}) =
0 $. This implies that  $\pi_* (\omega^n_{X/S})$  is locally free and
there are  natural isomorphisms 
$$
\pi_*(\omega^n_{X/S} \otimes k (s) \simeq  H^o (X_s, \omega^n_{X/S}
\otimes O_{X_{s}}), \;\; (\text{cf. EGA, Chapter 3, \S 7}). 
$$

\noindent
Hence for $n \ge 3$ the relatively very ample line bundle  $
\omega^n_{X/S} $ gives an embedding of $X$ into  the projective bundle
$ \mathbb{P} (\pi_* \omega^n_{X/S}) $ over $S$, associated to the
locally sheaf   $\pi_*( \omega^n_{X/S})$ on $S$. Thus $X$ can be
realized as a family of curves $C$ in $\mathbb{P}^{n (2g-2) -g}$
with the Hilbert polynomial of $C$ given by $P (m) = n (2g-2) m-g+1$.  

Let $ p : X \rightarrow S$ and  $q : Y \rightarrow S $ be two stable
curves. Define a functor  $\Isom_S (X,Y)$ form the category of  $S$-
schemes, Sch $S$, to the category of sets, as follows. 

\noindent
$\Isom_S (X,Y) (S') $ = The set of  $S'$- isomorphisms between $ X
\underset{S}{\times} S'$ and  $ Y \underset{S}{\times} S'$ .

\begin{subtheorem}\label{chap0:subthm0.1.6}% theorem 0.1.6
The functor $\Isom_S (X, Y)$  is  represented by a scheme
$\overline{\Isom}_S (X, Y)$, quasiprojective over $S$. (cf.~\cite{key3}).  
\end{subtheorem}

Let $\pi : C \rightarrow \Spec K $ be a stable curve.\label{page12} Let $t: \Spec
\dfrac{K [\varepsilon]}{(\varepsilon^2)} ~ \rightarrow
\overline{\Isom_K} (C,C) $ be a  tangent vector at a point  $ P
\in \; \overline{\Isom_K} (C,C)$. By definition $t$ corresponds to
an automorphism of $ C \times   \Spec \dfrac{K
  [\varepsilon]}{(\varepsilon^2)} $  which is canonically identified
with a vector field  $D$ defined on the whole of $X$. Now note the
following lemma. 

\setcounter{sublemma}{6}
\begin{sublemma}\label{chap0:sublem0.1.7} % lemma 0.1.7
If \pageoriginale $\pi : C' \rightarrow \Spec K$ is  a stale curve
then a vector field defined on the whole  of $C'$ is zero.  
\end{sublemma}\label{page14}

Before we go to the  proof of the lemma we deduce the following
result. The lemma says that the tangent space to $\overline{\Isom_K}
(C,C)$ at the point $P$ is zero. Since $P$ was an arbitrary  point
of $  \overline{\Isom_K} (C,C)$ we see that $ \overline{\Isom_K} (C,C)$
is finite. Thus we have the following theorem. 

\setcounter{subtheorem}{7}
\begin{subtheorem}\label{chap0:subthm0.1.8} % lemma 0.1.8
If  $ \pi : C \rightarrow \Spec K $  is a  stable curve then the group 
of  automorphisms of $C$ is finite. 
\end{subtheorem}


\setcounter{proofofthelemma}{6}
\begin{proofofthelemma} % proof of the lemma 0.1.7
Let $D$ be a  vector field defined on the whole of $C'$. Let $C'$ be
the normalization of $C'$. Since the only singularities of $C'$ are
ordinary double points, $D$ naturally corresponds to a vector field
$\bar{D}$ on $\bar{C'}$, such that $\bar{D}$  vanishes  at all points
of $\bar{C'}$ which lie over the double points of $C'$. It follows
that if $E$ is an irreducible component of $C'$ such that $\bar{E}$,
the normalization of $E$, has genus $\ge 2$ then $\bar{D}$ vanishes on
$\bar{E}$ and hence  $D$ vanishes on $E$. 
\end{proofofthelemma}

Now consider the components  $E$ of $C'$ such that $\bar{E}$ has genus
$\leq 1$. We have the following possibilities for $E$. 
\begin{enumerate}[i)]
\item  $E$ is a  nonsingular curve of  genus $0$.

\item $E$  has one double point, $\bar{E}$ has genus $0$.

\item $E$  has at  least two double points, $\bar{E}$ has genus $0$.

\item $E$ is a nonsingular curve of genus $1$.

\item $E$  has at least one  double point, $\bar{E}$ has genus $1$.
\end{enumerate}

In\pageoriginale the cases when  $E$ has genus $0$, $\bar{D}$ has at
least 3 zeroes and when $\bar{E}$ has  genus $1, \bar{D} $ has at
least one zero. It follows that  $\bar{D}$ must be zero on $\bar{E}$
in each of the above cases. This proves the lemma.  

For the proofs of the  results in this section we refer to \cite{key1}.

\bigskip
\noindent{\textbf{E) ~Deformation Theory}}%section 

 In this section we consider complete curves  $X$ such that,
\begin{enumerate}[i)]
\item  $X$ is reduced, connected;

\item if $ P\in X$ is a singular point of $X$ then $P$ is
  necessarily ordinary double point, i.e., $\hat{O}_{X,P} \simeq
  \dfrac{K[[x,y]]}{(xy)}$, ($\hat{O}_{X,P}$ denotes the completion of
  the local ring $O_{X,P} $ of $X$ at $P$). 
\end{enumerate}
It is clear that such a curve $X$ is a local complete  intersection. 

\setcounter{subdefin}{8}
\begin{subdefin}\label{chap0:subdef0.1.9} % subdefin 0.1.9
A (flat) deformation of $X$ over a complete local $K$-
  algebra $A$  is  a flat morphism  $\varphi : \bar{X} \rightarrow
\Spec A $ such that the  special fibre of $\varphi$ (i.e., the fibre
over the closed  point of Spec $A$)  is isomorphic to $X$. 
\end{subdefin}

Recall that the  set of first order deformation of $X$
(i.e. deformations over Spec $\dfrac{K [\varphi]}{(\varepsilon^2)}$)
is canonically identified with $\Ext^1 ( \Omega^1_X, O_X)$,
($\Omega^1_X$ is the sheaf  of Kahler differentials on $X$). 

Note\pageoriginale the following lemma. 

\setcounter{sublemma}{-1}
\setcounter{subsection}{2}
\begin{sublemma} % lemma 0.2.0
 $ \Ext^1 ( \Omega^1_X, O_X)  = 0$.
\end{sublemma}

\begin{proof}
The result follows from the following observations. 
\begin{enumerate}[i)]
\item We have  the following spectral sequence.
$$
H^p (X, \underline{\Ext}^q (\Omega^1_X, O_X)) \Rightarrow \Ext^{p+q}
(\Omega^1_X, O_X). 
$$
Since $\dim X = 1$,  $H^2(X, \underline{\Ext}^o (\Omega^1_X, O_X)) = 0$.
Since $\Omega^1_X $ is locally free except at a finite number of
points, $(\underline{\Ext}^1(\Omega^1_X, O_X)$  has support at only
finitely many points and hence $ H^1 (X, \underline{\Ext}^1 (\Omega^1_X,
O_X)) = 0$. 

\item Locally $X$ can be embedded in an affine $N$-space
  $\mathbb{A}^N$. Let $I$ be be the ideal sheaf defining $X$ in
  $\mathbb{A}^N$. $\Omega^1_X$ has the following free resolution. 
$$
0 \rightarrow \frac{I}{I^2} \rightarrow \Omega^1_{\mathbb{A}^N}
\otimes O_X \rightarrow \Omega^1_X \rightarrow 0 
$$
It follows  that $ \underline{\Ext}^2 (\Omega^1_X, O_X)) = 0 $.
\end{enumerate}

\noindent
Now it is immediate that  $\underline{\Ext}^2 (\Omega^1_X, O_X)) = 0
$. Thus there are no obstructions  to lifting deformations  over Spec
$ \dfrac{A}{J} $ to deformations of over Spec $A$ ($A$ denotes an
Artin local ring with residue field  $K$, $J$ an ideal in
$A$). Equivalently the functor of deformations of $X$ over an Artin
local $K-$algebra is formally smooth. We have the following theorem. 
\end{proof}

\setcounter{subtheorem}{0}
\begin{subtheorem}\label{chap0:subthm0.2.1} % theorem 0.2.1
There\pageoriginale  exists a formal scheme  $\tilde{X}$ and a proper
flat morphism 
$\eta: \tilde{X} \rightarrow \Spec K[[ t_1,t_2,\ldots,t_r]]= T$,
$( r = \dim  \Ext^1 (\Omega^1_{X}, O_X))$, such that the special fibre
of $\eta$ is  isomorphic to $X$. Further the morphism $\eta$ has the
following properties. 
\begin{enumerate}[i)]
\item  Given a deformation $\bar{X} \rightarrow \Spec A $ of $X$  over
  an Artin local $K$-algebra $A$, there exists a morphism  $\rho :
  \Spec A \rightarrow T $ such that  $ \bar{X} \rightarrow \Spec A $
  is obtained form  $\eta : \tilde{X} \rightarrow T $ by the  base
  change $\rho : \Spec A \rightarrow T $. 

\item In the case when  $ A \simeq \dfrac{K[\omega]}{(\omega^2)}$  the
  above morphism  $\rho $ is unique so that the tangent space of $T$
  at the  closed point is canonically isomorphic to $ \Ext^1
  (\Omega^1_{X}, O_X)$. 
\end{enumerate}

$\eta : \tilde{X} \rightarrow  T $ is called a versal deformation
space for $X$. 
\end{subtheorem}

In the case when  $ \Ext^o (\Omega^1_{X}, O_X) = 0, \eta : \tilde{X}
\rightarrow T $  is universal i.e. the morphism  $\rho $  is always
unique. Thus if $X$ is a stable  curve then a versal deformation is
universal (cf.~lemma \ref{chap0:sublem0.1.7}
page~\pageref{page14}).  Further since the 
invertible sheaf  $\omega_{\tilde{X}/T}$  is  relatively ample,
$\tilde{X}$ is the  formal completion of a unique scheme, proper and
flat over $T$. We have the following theorem. 


\begin{subtheorem}\label{chap0:subthm0.2.2}% theorem 0.2.2
 If $X$  is a stable  curve then the versal deformation $ \eta :
 \tilde{X} \rightarrow T $   is universal and algebraizable. 
\end{subtheorem} 

Another fact about $ \eta : \tilde{X} \rightarrow T$ ($X$ a stable
curve) is that generic fibre of $\eta$ is nonsingular. 

\medskip
\noindent
F) \qquad 
In\pageoriginale this section we prove some results and  make a few definitions
which will be needed in the  sequel. We first prove Clifford's theorem
for  a reduced curve with ordinary double  points. The proof in this
generality is  due to Gieseker and  Morrison. 

\begin{subtheorem}[Clifford's  theorem]\label{chap0:subthm0.2.3} % theorem 0.2.3 
 Let  $X$ be a reduced curve with only nodes
and let $L$ be a a line  bundle on $X$ generated by global
sections. If $H^1 (X,L) \neq 0$, there is a curve $C \subset X$ so
that  
$$
h^o (C,L) \leq \frac{\deg_C L}{2} + 1
$$
\end{subtheorem} 

\begin{proof}
Since $ H^1 (X,L) \neq 0 $, $H^o ( X, L^{-1} \otimes \omega_X) \neq 0$,
($\omega_X$ is the dualizing sheaf  of $X$). So there is  a non-zero
$ \varphi : L \rightarrow \omega_D $. We can find a curve  $ C 
\subset X$ so that $\varphi$ is not identically zero  on each
component of $C$, but $\varphi$ vanishes at  all points    
$ C \cap \overline{X -C} = \{P_1, \ldots , P_k\}$. Since $\omega_C  =
\omega_X (-P_1 \ldots -P_k)$, we actually obtain  
$$
\varphi : L_C \rightarrow \omega_C.
$$
Choose a basis  $s_1,\ldots, s_r$ of  $\Hom (L_C,\omega_C)$ so that 
$\varphi = s_1$. We can choose a basis $t_1 \dots t_p$ of $H^o (L_C)$
so that  $t_1$  does not vanish at the zeros of $s_1$ nor any singular
point of $C$. Suppose 
$$
a_1 [s_1,t_1] + a_2 [s_1,t_2]  + \cdots = b_2 [s_2,t_1] + b_3
[s_3,t_1]  + \cdots  
$$
where\pageoriginale $[s,t]$ is in $ H^o (C, \omega_C)$. Then 
$$
[s_1,t] = [s,t_1]
$$
where  $t \in H^o (C, L_C)$ and  $s$ is a linear combination
of $s_2, \ldots, s_r$. Hence $t$ is a multiple of $t_1$, since $t$
vanishes  where  $t_1$ does. Hence $s$ is a multiple of $s_1$,
contradicting  the independence of the $s_i 's$. So  
$$
h^o (L_C)  - h^o (L^{-1}_{C} \otimes \omega_C) \leq  g+1  
$$
and 
$$
h^o (L_C)  + h^o  (L^{-1}_{C} \otimes \omega_C) \leq \deg_C (L) + 1-g. 
$$
Adding gives the desired result.
\end{proof}

\setcounter{sublemma}{3}
\begin{sublemma}\label{chap0:sublem0.2.4}% lemma 0.2.4
Fix two integers  $g\ge 2$,  $d \ge 20 (g-1)$  and put  $N =
d-g$. There exists a constant $\varepsilon > 0$  such that for all
integers not all zero, $ r_0 \leq r_1 \leq \cdots \leq r_N$, 
$\sum\limits^{N}_{i=0} r_i = 0 $ and for all integers 
$0= e_0 \leq e_1\leq\cdots \leq e_N  = d$, satisfying, 
\begin{enumerate}[i)]
\item if $e_i > 2 g-2$ then  $e_i \ge i+g$,

\item if $ e_i \leq 2 g-2$ then  $e_i \ge 2 i$,
\end{enumerate}
there exists a sequence of integers  $0 = i_1 < i_2 < 
\cdots < i_k =N$, making the following inequality true. 
\begin{equation*}
\sum^{k-1}_{t=1} (r_{i_{t+1}} - r_{i_{t}})
\frac{(e_{i_{t+1}}+e_{i_{t}})}{2} > r_N e_N + \varepsilon (r_N -r_0)
\tag{1}\label{c0:eq1} 
\end{equation*}
\end{sublemma}


\begin{proof}
We\pageoriginale use  the following combinatorial lemma proved by
Morrison. 
\end{proof}

Fix integers  $0 = e_0 \leq e_1 \leq \cdots \leq e_N$. Define a
function  
$$
 T(r_0, r_1,\ldots, r_N) = \min\limits_{0 = i_1 < \dots
  < i_k =N}   \left[\sum\limits^{k-1}_{t=1}   (r_{i_t} - r_{i_{t+1}}) 
   \dfrac{(e_{i_{t}} + e_{i_{t+1}})}{2}\right],
$$ 
 where $r_0 \ge r_1 \ge \ldots \ge r_N =0 $ are numbers with
 $\sum\limits^{N}_{i=0} r_i = 1$. Then maximum value of $T$ is
 $T_{\max}= \dfrac{1}{2}$  $\max\limits_{i \in \{1, \ldots N\} }$ 
$\dfrac{e^2_i}{ie_i - \sum\limits^{i-1}_{j=1} e_j}$  

We modify inequality \eqref{c0:eq1} as follows.

\noindent
Let $r'_i = r_i +  \mid r_0 \mid, R = \sum\limits^{N}_{i=0} r'_i =
(N+1) | r_0 |, r''_i  = \dfrac{r'_i}{R}, (0 \leq i \leq N)$. 
Inequality \eqref{c0:eq1} can be easily seen to be equivalent to the following
inequality 
$$
\sum^{k-1}_{t=1} (r''_{i_{t+1}} - r''_{i_{t}}) \frac{(e_{i_{t+1}} +
  e_{i_{t}})}{2} > e_N (r''_N - \frac{1}{N+1})  + \varepsilon r''_N. 
$$

\noindent
Here  $0 = r''_0 \leq r''_1  \leq \cdots \leq r''_N$,
$\sum\limits^{N}_{i=0} r''_i = 1$. Transferring we get,  
\begin{align*}
& e_N r''_N -  \sum\limits^{k-1}_{t=1} (r''_{i_{t+1}} - r''_{i_{t}})
\dfrac{e_{i_{t+1}}+ e_{i_{t}}}{2} <  \dfrac{e_N}{N+1}  - \varepsilon ~
r''_N, \quad \; {\rm i.e.} \\
& \sum^{k-1}_{t=1} (r''_{i_{t+1}} - r''_{i_{t}}) \frac{(e_N - e_{i_{t+1}}+
  e_N - e_{i_{t}})}{2} <  \frac{e_N}{N+1}  - \varepsilon ~ r''_N.
\end{align*}

For  $ 0 < i < N$, let $e'_i = e_N - e_{N-i}$ and  $r'''_i =
r''_{N-i}$. Thus we have $0 = e'_0 \leq e'_1 \leq \cdots \leq e'_N =
d$, \; $r'''_0 \ge r'''_1 \ge \cdots \ge r'''_N$,  $\sum
\limits^{N}_{i=0}  r'''_i = 1$. 

\noindent
Also\pageoriginale it follows from conditions  i)  and  ii) that,
 \begin{enumerate}[i)]
\item  if  $e'_i < d - (2g- 2)$ then $e'_i \leq i$,

\item if $e'_i \geq d - (2g-2) $ then $e_i \leq g+2i - N$.
\end{enumerate}

\noindent
The last inequality can be written as,
$$
\sum^{k-1}_{t=1} (r'''_{N-i_{t+1}} - r'''_{{N-i}_t})
\frac{e'_{N-i_{t+1}} + e'_{{N-i}_t}}{2} < \frac{e_N}{N+1} - t r''_N
$$

It follows from Morrison's combinatorial lemma that there exists  $ i
\in \{ 1,2,\ldots,N\}$  and a sequence of integers,  $ 0 = N-
i_k < N -i_{k-1} < \cdots < N-i_1 = N $ such that the following
inequality is true. 
$$
\sum\limits^{k-1}_{t=1} (r'''_{{N-i}_{t+1}} - r'''_{N-i_{t}})
\frac{(e'_{N-i_{t+1}} + e'_{N-i_{t}})}{2} <  \frac{1}{2} 
 \frac{e'^2_i}{ie'_i -\sum\limits^{i-1}_{j=1} e'_j} 
$$

\noindent
Thus to prove the lemma it suffices to prove that there exists an
$\varepsilon > 0 $ such that for any sequence of integers  $ 0 = e'_0
< e'_1 < \cdots < e'_N = d $ as above and for all  $1 \leq i \leq N $, 
$$
\frac{1}{2} ~ \frac{e'^2_i}{ie'_i - \sum\limits^{i-1}_{j=1} e'_j} <
\frac{d}{N+1} - ~ \varepsilon . 
$$

This can be easily checked using the bounds on  $e'_0, e'_1 , \ldots , 
e'_N $. 

Let $ X = \bigcup\limits^{p}_{i =1} X_i$ be a curve,  ($X_i$ is an
irreducible component of $X \;  1 \leq i \leq  p$). Let $ \pi_i :
\bar{X}_{\rm red} \rightarrow X _{\rm ired} $  be the normalization of  
$X_{\rm ired}$\pageoriginale and let $\bar{X}$ be the disjoint union
$\bar{X} = 
\bigcup\limits_{i=1}^p \bar{X}_{\rm ired}$. We have closed immersions
(inclusions) $\eta_i :X_{\rm ired}\to X$. Let $\pi':\bar{X} \to  X_{\rm
  ired} $ be the morphism such that  the restriction of the morphism
$\pi = \eta o \pi'$ to $\bar{X}_{\rm ired} $ is the morphism $\eta_i
o \pi_i$,  $(1 \le i \le p)$.
  
\setcounter{subdefin}{4}
\begin{subdefin}\label{chap0:subdef0.2.5} % subdefin 0.2.5
The morphism $\pi : \bar{X} \to X$ is defined to be the normalization
of $X$. 
\end{subdefin}

Let $V$ be a vector space of dimension $n$ and let 
\begin{equation*}
o \subset V_1 \subset V_2 \subset \dots \subset V_r = V,\tag{F}
\end{equation*}
be a filtration of $V$. Put $n_i = \dim V_i$, $(1 \le i \le r)$. 

\begin{subdefin}\label{chap0:subdef0.2.6}% subdefin 0.2.6
In the above situation a basis $v_1, v_2 ,\ldots, v_n$ of $V$ is said
to be a basis relative to the filtration (F) if $v_1,
v_2,\ldots,v_{n_1}$  is a basis of $V_1; v_1,v_2 ,\ldots,
v_{n_1}$,  $v_{n_{1+1}}, \ldots , v_{n_2}$ is a basis of $V_2$,
etc. 
\end{subdefin}

