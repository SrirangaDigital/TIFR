\chapter{Stability of Curves}\label{chap1} %chapter 1

Fix\pageoriginale a polynomial $P(m) = dm-g+1$ where $g$ and $d$ are
integers with 
$g \ge 2$ and $d \ge 20 (g-1)$. Put $N = d - g$. In this chapter we
prove that there exists an integer $m_o$ such that if $X$ is a
connected nonsingular (nondegenerate) curve in $\mathbb{P}^N$ with
Hilbert polynomial $P(m)$ then the $m_o^{th}$ Hilbert point of $X$,
$H_{m_o} (X) \in \mathbb{P} (\overset{P(m_o)} \Lambda)  H^o
(\mathbb{P}^N, O_{\mathbb{P}^N} (m_o))$ (cf.\ definition~\ref{chap0:subdef0.1.0}
page~\pageref{chap0:subdef0.1.0}) is stable for the natural action of
$SL(N+1)$ on 
$\mathbb{P} (\overset{P(m_o)}{\Lambda}  H^o (\mathbb{P}^N,
O_{\mathbb{P}^N} (m_o)))$ (cf.\ definition~\ref{chap0:subdef0.0.4} 
page~\pageref{chap0:subdef0.0.4}). We  
prove further that if $X$ is a connected curve in $\mathbb{P}^N$, with
Hilbert polynomial $P(m)$ such that the $m_o^{th}$ Hilbert point of
$X$, $H_{m_o} (X) \in \mathbb{P} (\overset{P(m_o)}{\Lambda}  H^o
(\mathbb{P}^N, O_{\mathbb{P}^N} (m_o)))$ is semistable, then $X$ 
is semistable in the sense of definition~\ref{chap0:subdef0.1.4} 
(page~\pageref{chap0:subdef0.1.4}).  

Recall that all curve $X$ in $\mathbb{P}^N$, such that  the Hilbert
polynomial of $X$ is $P(m)$, are parametrized by a projective
algebraic scheme, say $H$ (cf.\ Hilbert scheme, page~\pageref{page11}). Let $Z
\xrightarrow{\text { inclusion }} \mathbb{P}^N \times H$ be the
universal closed subscheme and  let $Z \xrightarrow{p_H} H$ be the
composite  
$$
Z \xrightarrow {\text { inclusion }} \mathbb{P}^N \times H
\xrightarrow {\text { projection }} H. 
$$ 
$Z \xrightarrow{p_H} H$ can be viewed as a family of curves
parametrized by $H$ such that for all geometric points $h \in H$ the
fibre $X_h$ of $Z \xrightarrow{p_H} H $ over $h$ is a curve in
$\mathbb{P}^N _{k(h)}$ and $P(m)$ is the polynomial of $X_h$.  

\medskip
\noindent{\textbf{Notation:}}

\begin{enumerate} [i)]
\item By\pageoriginale ``a curve in the family $Z_H \xrightarrow{p_H}
  H$'' we mean 
  the fibre of $Z_H \xrightarrow{p_H} H$ over a closed point of $H$,
  which is connected. 

\item $X$ denotes a curve in the family $Z_H \xrightarrow{p_H} H$.

\item $I_X$ denotes the ideal sheaf of nilpotents in $O_X$.\label{c1:nilpotents}

\item $\pi : \bar{X} \to X $ denotes the normalization of $X$.

\item $L$ denotes the restrictions of $ O_{\mathbb{P}^N} (1)$ to $X$.

\item $L'$ denotes the line bundle $\pi^* L$ on $\bar{X}$.

\item $\varphi_m$ denotes the natural restriction, 
$$
\varphi_m : H^o (\mathbb{P}^N, O_{\mathbb{P}^N} (m)) \to H^o 
(X, L^m), \; m \in \mathbb{Z}.
$$
  
\item By a nondegenerate curve in the family  $Z_H \xrightarrow{p_H}
  H$ we mean a curve in the family $Z_H \xrightarrow{p_H} H$, which is
  a nondegenerate curve in $\mathbb{P}^N$. 
\end{enumerate}

Note the following assertions. There exists positive integers $m'$, 
$m''$,  $m'''$, $q_1,q_2,q_3$, $\mu_1, \mu_2$ with $m''' > m'$, $m''>2$, $q_3
> q_1$, $\mu_1 > \mu_2$ such that for every curve $X$ in the family $Z_H
\xrightarrow{p_H} H$, the following is true:  
\begin{enumerate}[i)]
\item For all integers $ m > m'$, $H^1 (X, L^m) = 0 = H^1 (\bar{X},{L'}^m )$.

\item $I_X^{q_1} = 0$.

\item $h^o (X, I_X) \le q_2$.\label{c1:as3}\pageoriginale

\item For every complete subcurve $C$ of $X$, $h^o (C, O_C) \le q_3$.\label{c1:as4}

\item For every point $P \in X $ and for all integers  $r \ge 0$, $\dim
  \dfrac{O_{X,P}}{m^r_{X,P}} \le \mu_1 r + \mu_2$, ($O_{X,p}$ is the local
  ring $X$ at $P$ and $m_{X,P}$ is the maximal ideal in $O_{X,P}$).\label{c1:as5} 

\item For every subcurve $C$ of $X$, for every point $P \in C$\label{c1:as6}
  and for all integers $m'', m > r \ge m''$, $H^1 (C,I^{m-r} \otimes
  L_C^m) = 0$, ($I$ is the ideal subsheaf of  $O_C$ defining the point 
  $P \in C$). 

 \item For a geometric point $h \in H$ let $X_h$ denotes the fibre of
   $Z \xrightarrow{p_H} H$ over $h \in H$. For $ m>m'$ let  $\psi_m:
   H \to \mathbb{P} (\overset{P(m)} \Lambda)  H^o (\mathbb{P}^N, O
   {_{\mathbb{P}}{^N}} (m)))$  be the morphism defined by $\psi_m (h)
   = H_m (X_h)$. For all integer  $m \ge m'''$, $\psi_m$ is a closed
   immersion.\label{c1:cl.immersion}  
\end{enumerate}

We do not try prove these assertions as these can be proved by
standard arguments. 

Fix a basis $X_0, X_1 ,\ldots, X_N$ of $H^o (\mathbb{P}^N, O_{
  \mathbb{P}^N} (1))$.  Consider the action of $GL(N+1)$ (and
hence $SL(N+1)$) on $H^o (\mathbb{P}^N, O_{\mathbb{P}^N} (1))$,
defined by 
$$
[a_{ij}].X_p =\sum_{j=o}^N a_{pj} X_j, \; [a_{ij}] \in 
GL(N+1), \;\; (0 \le p \le N).  
$$
The above action of $SL(N+1)$ on $H^o (\mathbb{P}^N, O_{
  \mathbb{P}^N} (1))$ induces an action of $SL(N+1)$ on
$\mathbb{P} (\overset{P(m)} \Lambda)  H^o (\mathbb{P}^N,
O_{\mathbb{P}^N} (m_o))$, (cf. page~\pageref{c0:l(n+1)}).  

In the above situation we have the following theorem.

\setcounter{section}{1}
\setcounter{subsection}{0}
\setcounter{subtheorem}{-1}
\begin{subtheorem}\label{chap1:subthm1.0.0}% theorem 1.0.0
There\pageoriginale exists an integer $m_o > \max$. $\{m''', d \bar{q}
(3d+m''+5)\}$ 
such that for every nondegenerate nonsingular curve $X$ in the family
$Z_H \xrightarrow{p_H} H$, the $m_o^{th}$ Hilbert point of $X$, 

\noindent
$H_{m_o} (X) \in \mathbb{P} (\overset{P(m_o)} \Lambda  H^o
(\mathbb{P}^N, O_{\mathbb{P}^N} (m_o)))$  is stable 
\end{subtheorem}

\begin{remark*}
It will follow from the proof that there exist infinitely many integers
$m > \max. \{m''', d \bar{q} (3d+m''+5)\}$ such that for every
nondegenerate nonsingular curve $X$ in the family $Z_H
\xrightarrow{p_H} H$, the $m^{th}$ Hilbert point of $X$,  

\noindent
$H_m (X) \in  \mathbb{P} (\overset{P(m_o)} \Lambda  H^o
(\mathbb{P}^N, O_{\mathbb{P}^N} (m)))$  is stable. 
\end{remark*}

\begin{proof}
It suffices tn prove that there exists an integer $m_o > \max. 
\{m''',\break d \bar{q} (3d+m''+5)\}$, such that, for every nondegenerate
nonsingular curve $X$ in the family $Z_H \xrightarrow{p_H} H$ and
for every  $1-ps \lambda$ of $SL(N+1)$, the $m_o^{th}$ Hilbert point
of $X$, $H_{m_o} (X) \in \mathbb{P} (\overset{P(m_o)}\Lambda  H^o
(\mathbb{P}^N, O{_{\mathbb{P}^N}} (m_o)))$  is $\lambda$-stable,
(cf. theorem~\ref{chap0:subthm0.0.9} page~\pageref{chap0:subthm0.0.9}). 
\end{proof}

Let $X$ be a nondegenerate nonsingular curve in the family $Z_H
\xrightarrow{p_H} H$ and let $\lambda$ be a $1-ps$ of $SL(N+1)$. There
exists a basis of  $H^o (\mathbb{P}^N, O_{\mathbb{P}^N} (1))$,
say, $w_0, w_1,\ldots, w_N$, and integers $r_0 \le r_1 \le \dots
\le r_N, \sum\limits_{i=0}^N r_i = 0$, such that the action of
$\lambda$ on $H^o (\mathbb{P}^N, O_{\mathbb{P}^N} (1))$ is given
by,  
$$
\lambda (t)w_i = t^{r_i} w_i, \;\; t \in  K^*, \;\;  (0 \le i \le N)
$$

\noindent
It is easily seen that the natural restriction map $\varphi_1 : H^o
(\mathbb{P}^N, O_{\mathbb{P}^N} (1)) \to H^o (X,L)$\pageoriginale is an
isomorphism. Let $\varphi_1(w_i) = w'_i$,  $0 \le i \le N$. Let $F_{j-1}$
be the invertible subsheaf of $L$ generated by $w'_o, w'_1 ,\ldots,
w'_{j-1}$, deg $F_{j-1} = e_{j-1}$, $1 \le j \le N+1$. Note that the
integers $e_0, e_1,\ldots, e_N$ satisfy,   
\begin{enumerate}[i)]
\item if $e_j > 2g-2$ then $e_j \ge j+g$,

\item if $e_j \leq 2g-2$ then $e_j \ge 2j$.
\end{enumerate}

This is immediate by the Riemann-Roch theorem and Clifford's theorem. 


It follows from the combinatorial lemma~\ref{chap0:sublem0.2.4} 
(page~\pageref{chap0:sublem0.2.4}) that there 
exists a constant $\varepsilon > 0$ such that for all integers, $0 =
e'_0 \le e'_1 \leq \dots \le e'_N = d$, satisfying  conditions i) and
ii) and 
for all integers $r'_0 \le r'_1 \leq \dots \le r'_N$, $\sum\limits_{i=0}^N
r'_i = 0$; there exist integers  $0= i_1 < i_2 < \dots < i_k, = N$
such that the following inequality holds  
$$
\sum_{t=1}^{k'-1}  (r'_{i_{t+l}} - r'_{i_t}) 
\frac{(e'_{i_{t+1}} + e'_{i_t})}{2} >r'_N e'_N + \varepsilon (r'_N -
r'_o). 
$$

\noindent
In particular, there exist integer $ 0=i_1 < i_2 < \dots < i_k = N$,
such that,  
$$\sum\limits_{t=1}^{k-1}  (r_{i_{t+1}}  - r_{i_t}) 
\frac{e_{i_{t+l}} + e_{i_t}}{2} > r_N e_N + \varepsilon (r_N - r_o).
$$

Recall that for all positive integers $p$ and $n$,  $H^o
(\mathbb{P}^N, O_{\mathbb{P}^N} ((p+1)n))$ has a  basis $B_{(p+1)n} =
\{M_1, M_2 ,\ldots, M{_\alpha}_{_{(p+1)n}}\}$ consisting of monomials
for degree $(p+1)n$ in $w_0, w_1,\ldots,w_N$, $(\alpha{_{(p+1)n}}) = h^o
(\mathbb{P}^N, O_{\mathbb{P}^N} (p+1)n)$.\pageoriginale 

\smallskip
Let $V_{i_t}$ be the subspace of $H^o (\mathbb{P}^N, O_{\mathbb{P}^N}
(1))$, generated by \\
$S_{i_t} = \{ w_0, w_1
,\ldots, w_{i_t}\}$, \;  $(1 \le t \le  k)$.  For all integers
$t_1,t_2,s $ with $1 \le t_1 <t_2 \le k$ and $0 \le s \le p$ let
$(V^{p-s}_{i_{t_1}} \cdot V^s_{i_{t_2}} \cdot V_N)^n$ be the\break subspace of
$H^o (\mathbb{P}^N, O_{\mathbb{P}^N}((p+1)n))$ generated by
elements $w$ of the type $w = v_1 v_2 \dots v_n$, where $v_r (1 \le r
\le n)$ is as follows. 

For $ s=0$, $v_r = x_{r_1} x_{r_2} \dots x_{r_p} z_r$,
$(x_{r_j} \in S_{i_{t_1}}, 1 \leq j \leq p, z_r \in
S_{i_k})$;

\noindent
 for $ 0 \le s \le p$, $v_r= x_{r_1} x_{r_2}$, $\ldots
 x_{r(p-s)}  y_{r_1} y_{r_2} \dots y_{r_s} z_r$

\noindent
$
 (x_{r_j} \in S_{i_{t_l}}, 1 \le j < p-s, y_{r_j} \in
 S_{i_{t_2}}, 1 \le j \le S; \; z_r \in  S_{i_k});   
$
 
 \noindent
for $ s=p$, $v_r = y_{r_1} y_{r_2} \dots y_{r_p} z_r $
$(y_{r_j} \in S_{i_{t_2}}$, $1 \le j \le p$, $z_r \in
S_{i_k})$ 

\noindent
These subspaces define the following filtration of $H^o  (\mathbb{P}^N,
O_{\mathbb{P}^N} ((p+1)^n))$. 
{\fontsize{10}{12}\selectfont
\begin{align*}
& 0 \subset (V^p_{i_1} \cdot  V^o_{i_2} \cdot V_N)^n \subset
  (V^{p-1}_{i_1} \cdot  V^1_{i_2} \cdot  V_N)^n \subset \dots \subset
  (V^1_{i_1} \cdot  V^{p-1}_{i_2} \cdot  V_n)^n \\
&  \subset (V^p_{i_2} \cdot  V^o_{i_3} \cdot  V_N)^n \subset
(V^{p-1}_{i_2} \cdot  V^1_{i_3} \cdot  V_N)^n \subset .. \dots
  \quad\dots\\
&  \subset (V^p_{i_t} \cdot  V^o_{i_{t+1}}  \cdot V_N)^n \subset
 (V^{p-1}_{i_t} \cdot  V^1_{i_{t+1}}  \cdot  V_N)^n \subset \dots \subset
 (V^{p-s}_{i_{t}} \cdot  V^s_{i_{t+1}}  \cdot  V_N)^n \subset \dots \\
& \subset (V^p_{i_{k-1}} \cdot V^o_{i_k} \cdot  V_N)^n \subset
(V^{p-1}_{i_{k-1}} \cdot V^1_{i_k} \cdot  V_N)^n \subset \ldots
  \subset (V^1_{i_{k-1}} \cdot V^{p-1}_{i_k} \cdot  V_N)^n\\
& \subset (V^o_{i_{k-1}} \cdot V^p_{i_k} \cdot  V_N)^n = H^o
  (\mathbb{P}^N, O_{\mathbb{P}^N} (( p+1)n)), \tag{F}
\end{align*}}\relax

Assume now that $(p+1)n > m'$ so that the natural restriction map
$\varphi_{(p+1)n} : H^o (\mathbb{P}^N, O_{\mathbb{P}^N} ((p+1)n)) \to
H^o (X, L^{(p+1)n})$\pageoriginale is surjective. For integer $0 \le s
\le p$  and $1 \le t \le k $, let 
 $(\bar{V}^{p-s}_{i_t} \cdot \bar{V}^{s}_{i_{t+1}} \cdot \bar{V}_N )^n =
\varphi_{(p+1)n}( V^{p-s}_{i_t}  \cdot V^s_{i_{t+1}} \cdot  V_N)^n \subset
H^o (X, L^{(p+1)n})$. We have the following filtration of $H^o (X,\break
L^{(p+1)n})$.   
 \begin{align*}
&  0 \subset (\bar{V}^{p}_{i_1} \cdot  \bar{V}^o_{i_2} \cdot \bar{V}_N)^n
 \subset (\bar{V}^{p-1}_{i_1} \cdot  \bar{V}_{i_2} \cdot  V_N)^n \subset \dots
 \subset (\bar{V}^{p-s}_{i_t}\cdot  \bar{V}^s_{i_{t+1}} \cdot  \bar{V}_n)^n
 \subset \dots\\
&  (\bar{V}^{o}_{i_{k-1}} \cdot  \bar{V}^{p}_{i_k} \cdot \bar{V}_N)^n  =H^o
 (X,L ^{(p+1)}n) \tag{$\bar{F}$}
 \end{align*}
 
 \noindent
 Rewrite the basis $B_{(p+1)n} $ as

\noindent 
 $B_{(p+1)n} = \{M'_1, M'_2, \ldots,  M'_{P((p+1)n)},  M'_{P((p+1)n)+1}
,\ldots M'_{\alpha_{(p+1)n}}\}$ so that \break $M'_1, M'_2, ,\ldots
M'_{P((p+1)n)'} $ is a basis of $ H^o (X,L ^{(p+1)}n)$ relative to the
filtration $(\bar{F})$ and $M'_{P((p+1)n)+1} ,\ldots
M'_{\alpha{_{(p+1)n}}}$ are the rest of the monomials in $B_{(p+1)n}$
in some order. 
 
 Let $X'$ be a nondegenerate  nonsingular curve in the family $Z_H
 \xrightarrow{p_H} H, L'$ be the restriction of $O_{\mathbb{P}^N} (1)$
 to $X'$, $X'_0, X'_1 ,\ldots,X'_N$ be a basis of
 $H^o (\mathbb{P}^N, O_{\mathbb{P}^N} (1))$. Let $F'_{j-1}$ be the
 invertible subsheaf of $L'$ generated by the images of $X'_0, X'_1
 ,\ldots, X'_{j-1}$ under the natural restriction  
  $\varphi'_1 : H^o (\mathbb{P}^N,\break O_{\mathbb{P}^N}(1)) \to H^o
 (X', L'), (1 \le j \le N+1)$. We claim that there exists an integer
 $n'$ such that for all integers $n >n'$, $0 \le t_1 < t_2 \le N$ for
 all nonsingular curves $X'$ in the family $z_H \xrightarrow{p_H} H$ and for
 all invertible sheaves $F'_{t_1}, F'_{t_2}, L'$ as
 above,\pageoriginale   
 $$
 (\bar{V}'^{p-s}_{t_1} \cdot  \bar{V}'^s_{t_2} \cdot  \bar{V}'_N)^n = H^o (X',
 (F'^{p-s}_{t_1} \otimes F'^{s}_{t_2} \otimes L')^n),
$$

$
\bigg[(\bar{V}'^{p-s}_{t_1} \cdot  \bar{V}'^s_{t_2} \cdot  \bar{V}'_N)^n
  \text{ is defined in the same way as } (\bar{V}^{p-s}_{t_1}
  \cdot \bar{V}^s_{t_2} \cdot  \bar{V}_N)^n \bigg]. 
$

Indeed, $F'^{p-s}_{t_1}$ and $F'^{s}_{t_2}$ are generated by the
sections in $\bar{V}'^{p-s}_{t_1}$ and $\bar{V}'^{s}_{t_2}$, and the
linear system $V'_N$ is very ample. Thus the linear system $W =
\bar{V}'^{p-s}_{t_1} \cdot \bar{V}'^{s}_{t_2} \cdot  \bar{V}'_N$  is very
ample and generates $M = \bar{F}'^{p-s}_{t_1} \otimes \bar{F}'^{s}_{t_2}
\otimes L'$. Let $\psi : X' \to \mathbb{P}(W)$ be the projective
embedding derived from $W$ and let $I$ be the ideal of $\psi
(X')$. For $n >> 0$, $H^1 (\mathbb{P} (W) , I (n)) = 0 $.  For such $n$,
the map from $W^n$  to $H^o (X', M^n) $ is onto. Our claim follows
provided we can pick $n'$ independent of $X'$ and integers $t_1,
t_2$. This can be done using standard techniques. Thus for integers
$0 < t_1 < t_2 < N$, $0 \le s \le p$, $n > n'$  we have  
$$
(\bar{V}^{p-s}_{t_1} \cdot \bar{V}^{s}_{t_2} \cdot \bar{V}_N)^n = H^o 
(X,(F^{p-s}_{t_1} \otimes F^{s}_{t_2} \otimes L)^n). 
$$

\noindent
Choose integers $p_o$ and $n_o$ such that
$p_o > \max. \{d+g, \dfrac{2d+1}{\varepsilon}\}$, $n_o > \max. \{ p_o, n'\}$ and
$m_o = (p_o+1)n_o > \max.\{m''', d \bar{g} (3d+m'' + 5)\}$.
It then follows by the Riemann-Roch theorem that
\begin{align*}
 \dim (\bar{V}^{p_o-s}_{i_t} \cdot  \bar{V}^{s}_{i_{t+l}} \cdot
 \bar{V}_N)^{n_o} &= n_o ((p_o -s) e_{i_t} + se_{i_{t+l}} + e_N) -
 g+1,\\ 
& \hspace{2cm} (0\leq s \leq p, \quad 1 \leq t \leq k).
\end{align*}

We\pageoriginale now estimate, 

total $\lambda$-weight of $M'_1, M'_2,\ldots, M'_{P(m_o)}
=\sum\limits_{i=1}^{P(m_o)} w_{\lambda} (M'_i)$, 

\noindent
(cf. definition~\ref{chap0:subdef0.1.1} 
page~\pageref{chap0:subdef0.1.1}). Note that a monomial   
$M \in (V^{p_{0-s}}_{i_t} \cdot V^s_{i_{t+1}} \cdot V_N)^{n_o}
- (V^{p_o - s+1} \cdot V^{s-1}_{i_{t+1}} \cdot V_N)^{n_o}$
 has $\lambda$-weight $w_{\lambda}(M) \le n_o ((p_o - s) r_{i_{t}} +
 sr_{i_{t+1}} + r_N)$.  

\begin{landscape}
\begin{align*}
& \sum\limits_{i=1}^{P(m_o)} w_{\lambda} (M'_i) < n_o (p_o r_{i_{1}} +
r_N) (\dim \bar{V}^{p_o}_{i_{1}} \cdot  \bar{V}^o_{i_{2}}\cdot \bar{V}_N)^{n_o}\\
& + n_o ((p_o-1) r_{i_{1}} +  r_{i_{2}} + r_N)  (\dim
(\bar{V}^{p_o-1}_{i_{1}} \cdot \bar{V}^{1}_{i_{2}} \cdot  \bar{V}_N)^{n_o} - (\dim
\bar{V}^{p_o}_{i_{1}} \cdot  \bar{V}^{0}_{i_{2}} \cdot
\bar{V}_N)^{n_o}\\
& + n_o ((p_o-2) r_{i_{1}} +  2r_{i_{2}} + r_N)  (\dim
(\bar{V}^{p_o-2}_{i_{1}} \cdot \bar{V}^{2}_{i_{2}}\cdot  \bar{V}_N)^{n_o} - (\dim
\bar{V}^{p_o-1}_{i_{1}}  \cdot \bar{V}^{1}_{i_{2}} \cdot  \bar{V}_N)^{n_o})\\
& + \cdots \qquad \cdots \qquad \cdots \qquad \cdots \qquad\cdots\\
& + n_o ((p_o-s) r_{i_{t}} + sr_{i_{t+1}} + r_N) (\dim
(\bar{V}^{p_o-s}_{i_{t}} \cdot \bar{V}^s_{i_{t+l}} \cdot
\bar{V}_N)^{n_o} - \dim (\bar{V}^{p_o-s+1}_{i_{1}}
\cdot \bar{V}^{s-1}_{i_{t+1}} \cdot \bar{V}_N)^{n_o})\\
& + \cdots \qquad \cdots \qquad \cdots \qquad \cdots \qquad\cdots\\
& + n_o (p_o r_{i_{k}} + r_N) (\dim (\bar{V}^o_{i_{k-1}} \cdot
\bar{V}^{p_o}_{i_{k}} \cdot  \bar{V}_N)^{n_o}  - (\dim
\bar{V}^{1}_{i_{k-1}} \cdot  \bar{V}^{p_o-1}_{i_{k}}  \cdot \bar{V}_N)^{n_o} \\ 
& = n_o (p_o r_{i{_1}} + r_N) (n_o (p_o e_{i{_1}} + e_N)- g+1)\\
& + n_o^2 ((p_o-1) r_{i_1} + r_{i_2} + r_N) (e_{i_2} - e_{i_1}) +
n_o^2 ((p_o-2) r_{i_1} + r_{i_2} + r_N) (e_{i_2} - e_{i_1})\\
&  + \cdots +  n_o^2 ((p_o-s) r_{i_t} + sr_{i_{t+1}} + r_N)
(e_{i_{t+1}} - e_{i_{t}}) + \cdots + n_o^2 (p_o r_{i_k} + r_N) (
e_{i_k} -  e_{i_{k-1}})
\end{align*}
\end{landscape}
\pageoriginale
\begin{align*}
& = n_o^2 p_o r_{i_1} e_N +  n_o^2 r_N e_N +  n_o(p_o r_{i_1}+
r_N) (1-g)\\
& + n^2_o \sum\limits_{s=1}^{p_o} ((p_o-s) r_{i_1} + sr_{i_2} +
r_N) (e_{i_2} - e_{i_1}) + \cdots \qquad \cdots\\
& + n^2_o \sum\limits_{s=1}^{p_o} ((p_o-s)r_{i_t} + sr_{i_{t+1}} +
r_N) (e_{i_{t+1}} - e_{i_t}) + \cdots \qquad \cdots\\
& + n^2_o \sum\limits_{s=1}^{p_o} ((p_o-s)r_{i_{k-1}} + sr_{i_k} +
r_N) (e_{i_k} - e_{i_{k-1}}), \\
& \hspace{2cm}  (\because e_{i_1} = e_o = 0) < n^2_o r_N
e_N + n_op_o r_{i_1} (1-g)\\
& + n_o^2 \bigg[ \dfrac{(p_o-1) p_o r_{i_1}}{2} + \dfrac{p_o
    (p_o+1)r_{i_{2}}}{2} + p_o r_N\bigg] (e_{i_2} - e_{i_1}) + \cdots\\
& + n_o^2 \bigg[ \dfrac{(p_o-1) p_o r_{i_{t}}}{2} + \dfrac{p_o
    (p_o+1)r_{i_{t+1}}}{2} + p_o r_N \bigg] (e_{i_{t+1}} -
e_{i_t}) + \cdots\\
& + n_o^2 \bigg[ \dfrac{(p_o-1) p_o r_{i_{k-1}}}{2} + \dfrac{ p_o
    (p_o+1)r_{i_k}}{2}  + p_o r_N \bigg] (e_{i_k} - e_{i_{k-1}}) \\
& \hspace{2cm}
\because (r_{i_1} = r_o < 0, \; r_N (1-g) < 0)\\
& = n^2_o r_N e_N + n_o p_o r_{i_l} (1-g)\\
& + \frac{n^2_o p^2_o}{2} \bigg[ (r_{i_2} + r_{i_1}) (e_{i_2} - e_{i_1})
  + \cdots +  (r_{i_{t+1}} + r_{i_t}) ( e_{i_{t+1}} - e_{i_t})
  + \cdots\\
 & + (r_{i_k} + r_{i_{k-1}}) (e_{i_k} -  e_{i_{k-1}}) \bigg]
 + \frac{n^2_op_o}{2} \bigg[  (r_{i_2}- r_{i_1}) (e_{i_2} - e_{i_{1}})
   + \cdots \\
& +  (r_{i_{t-1}} -  r_{i_{t}}) (e_{i_{t+l}}) + 
 \cdots + (r_{i_k} - r_{i_{k-1}}) (e_{i_k}
  - e_{i_{k-1}})\\
& + 2r_N \sum\limits_{t=1}^{k-1} (e_{i_{t+1}} -
  e_{i_t})\bigg]
\end{align*}\pageoriginale
\begin{align*}
& = n^2_o r_N e_N +  n_o p_o r_{i_1} (1-g) n^2_o p^2_o
\sum\limits_{t=1}^{k-1} (r_{i_{t+1}} + r_{i_t})
\frac{(e_{i_{t+1}} - e_{i_t})}{2}\\
& + n^2_op_o (\sum\limits_{t=1}^{k-1} (r_{i_{t+1}} - r_{i_t}))
\dfrac{(e_{i_{t+1}} - e_{i_t})}{2} + r_N (e_{i_k} -
e_{i_1}))\\
& < n^2_o r_N e_N + n^2_o p_o n_o^{-1} r_{i_1} (1-g)\\
& + n^2_o p^2_o \bigg[ r_N e_N  -\sum\limits_{t=1}^{k-1} (r_{i_{t+1}}
  - r_{i_t})   \frac{(e_{i_{t+1}} + e_{i_t})}{2}   \bigg]\\
& + n^2_op_o \bigg[\sum\limits_{t=1}^{k-1}  \dfrac{(r_{i_{t+1}} -
    r_{i_1}) e_{i_k}}{2} + r_N e_{i_k} \bigg],\\
&\qquad (\because- (r_{i_{t+1}} - 
r_{i_t})  e_{i_k} \le 0, \;\; e_{i_1} = e_o = 0)\\ 
&  < n^2_o p^2_o \bigg[ r_N e_N -  \sum\limits_{t=1}^{k-1}
  (r_{i_{t+1}} - r_{i_t})   \dfrac{(e_{i_{t+1}} + e_{i_t})}{2}   \bigg]\\ 
& + n^2_o p_o \bigg[ \dfrac{e_{i_k} (r_{i_k} - r_{i_l})}{2}
  + r_N e_{i_k} +\bigg] + n^2_o (r_N e_N + r_{i_1} (1-g)) \\
& \hspace{2cm} (\because
p_on_o^{-1} < 1,   \; r_{i_1} (1-g) > 0)\\
& < n^2_o \bigg[ - \varepsilon (r_N -r_o) p^2_o + p_o
  (\dfrac{d(r_N-r_o)}{2} +  dr_N)  + dr_N + r_o (1-g) \bigg]  
\end{align*}

(This follows from the lemma (page 27) and the facts that
$r_{i_1} = r_o$,  $e_{i_k} = e_N =d$, $r_{i_k} = r_N)$. 
\begin{align*}
& = n^2_o (r_N - r_o) \bigg[ - \varepsilon p^2_o + p_o (\dfrac{d}{2} +
  \dfrac{dr_N} {r_N-r_o}) + \dfrac{dr_N} {r_N-r_o} + \dfrac{r_o(1-g)}
        {r_N-r_o} \bigg] \\
& < n^2_o (r_N -r_o) \bigg[ -\varepsilon p^2_o +
  \dfrac{3dp_o}{2} + d + g - 1 \bigg]\\
& \hspace{3cm} (\because \dfrac{r_N}{r_N -r_o} <1, \;\;
\dfrac{r_o (1-g)}{r_N - r_o} < g-1)\\
& = n^2_op_o(r_N-r_o) \bigg [-\varepsilon p_o + \dfrac{3d}{2} +
  \frac{d+g-1}{p_o} \bigg]\\
& < 0, (\because p_o  > \max . \{d+g, \dfrac{2d+1}{\varepsilon}\} )
\end{align*}\pageoriginale


It is immediate from the above estimate and criterion $(*)$ 
(page~\pageref{c0:*}) that the point $H_{m_o}(x) \in \mathbb{P}
(\overset{P{(m_o)}} \Lambda H^o (\mathbb{P}^N, O_{\mathbb{P}^N}
(m_o)))$ is $\lambda$-stable. Further, by our choice 
of the numbers $\varepsilon$, $p_o$ and $(p_o +1)n_o = m_o$ it is clear
from the above calculation that for every nonsingular curve $X'$ in the
family $Z_H \xrightarrow{p_H} H$, for every $1-ps \lambda'$ of
$SL(N+1)$, the point $H_{m_o}(X') \in
\mathbb{P}(\overset{P(m_o)} \Lambda H^o ( \mathbb{P}^N,
O_{\mathbb{P}^N}(m_o)))$ is $\lambda'$-stable. This proves the 
result. 


Now consider the closed immersion (cf.~page~\pageref{c1:cl.immersion}), 
$\psi_{m_o}: H \to\mathbb{P}(\overset{P(m_o)} \Lambda H^o ( \mathbb{P}^N,
 O_{\mathbb{P}^N}(m_o)))$ where $m_o$ is the integer fixed in the
 above theorem~\ref{chap1:subthm1.0.0}. Let
 $\mathbb{P}(\overset{P(m_o)} \Lambda H^o ( 
 \mathbb{P}^N, O_{\mathbb{P}^N}(m_o)))^{ss}$ be the open subset of
\break $\mathbb{P}(\overset{P{(m_o)}} \Lambda H^o(\mathbb{P}^N,
 O_{\mathbb{P}^N}(m_o)))$ consisting of semistable points and let
 $V$ 
 be the inverse image of this open set by the morphism
 $\psi_{m_o}$. Let $Z_V = p^{-1}_H(V)$. 
 
 \noindent
 By restricting the morphism $p_H$ to $Z_V$ we obtain a family $Z_V
 \xrightarrow{p_V} V$ of curves $X$, such that the $m_o^{\rm th}$ Hilbert
 point of $X$, $H_{m_o} (X)\in \mathbb{P}(\overset{P(m_o)}\Lambda$ $H^o
 ( \mathbb{P}^N, O_{\mathbb{P}^N} (m_o)))$ is 
 semistable. The above theorem~\ref{chap1:subthm1.0.0}. asserts that
 the family $Z_V 
 \xrightarrow{p_V}V$ contains all the nondegenerate nonsingular
 curves in the family $Z_H \xrightarrow{p_H}H$. 
 
 We are now ready to state the main theorem of these is lecture
 notes. 


 \begin{subtheorem}\label{chap1:subthm1.0.1} % theorem 1.0.1
Every\pageoriginale curve $X$ in the family $Z_V \xrightarrow{p_V} V$
is semistable 
in the sense of definition \ref{chap0:subdef0.1.4}. (page
\pageref{chap0:subdef0.1.4}). Further 
trace of the 
linear system $|D|$ on $X$ is complete, ($|D|$ is the complete linear
system on $\mathbb{P}^N$ corresponding to the line bundle
$O_{\mathbb{P}^N}(1)$ on $\mathbb{P}^N$).    
 \end{subtheorem} 


\medskip
\noindent{\textbf{Idea of the proof:}}
The proof of the above theorem is divided in the following 
propositions~\ref{chap1:subprop1.0.2}, \ref{chap1:subprop1.0.3}., $\ldots$,
   \ref{chap1:subprop1.0.9}. The proofs of the 
 propositions~\ref{chap1:subprop1.0.2}, \ref{chap1:subprop1.0.3}., $\ldots$,
\ref{chap1:subprop1.0.6}. are on the same lines as 
follows. Assume that the proposition is not true, i.e., let $X$ be a
curve in the family $Z_V \xrightarrow{p_V} V$ which does not have the
property stated in the proposition. Using this assumption we are able
to produce a $1-ps \lambda'$ of $SL(N+1)$ such that the $m_o^{th}$
Hilbert point of $X,H_{m_o} (X) \in \mathbb{P}(\overset{P(m_o)}
\Lambda H^o ( \mathbb{P}^N, O_{\mathbb{P}^N}(m_o)))$ is not
$\lambda'$-semistable.  

 
 \noindent
 In particular it follows that $H_{m_o}(X)$ is not semistable,
 (cf.\ theorem~\ref{chap0:subthm0.0.9} page~\pageref{chap0:subthm0.0.9}). 
This contradiction then proves the proposition. 
 
 In proposition~\ref{chap1:subprop1.0.7} we prove an important inequality (cf.
 inequality $(*')$, proposition~\ref{chap1:subprop1.0.7}, 
 page~\pageref{chap1:subprop1.0.7}) which follows from 
 the inequality in criterion $(**)$ (page~\pageref{page10}), used in the case of a
 particular $1-ps \lambda'$ of $SL(N+1)$ and the integer
 $m_o$. Propositions \ref{chap1:subprop1.0.8} and
 \ref{chap1:subprop1.0.9} are proved using the above 
 inequality. 
 
\setcounter{subprop}{1}
 \begin{subprop}\label{chap1:subprop1.0.2} % 1.0.2
Every curve $X$ in the family $Z_V \xrightarrow{p_V}V$ is a
nondegenerate curve in $\mathbb{P}^N$ i.e. $X$ is not contained in any
hyperplane in $\mathbb{P}^N$. 
\end{subprop}  
 
\begin{proof}
Suppose that the result is not true i.e. suppose that there exists a
curve $X$ in the family $Z_V \xrightarrow{p_V} V$ such that $X \subset
\mathbb{P}^N$ is a  degenerate\pageoriginale curve. We will show that
this leads to 
the contradiction that the $m_o^{th}$ Hilbert point of $X$,
$H_{m_o}(X) \in \mathbb{P}(\overset{P(m_o)} \Lambda H^o (
\mathbb{P}^N, O_{\mathbb{P}^N}(m_o)))$ is not semistable. This
contradiction will then prove the result. 
\end{proof}

That $X$ is a degenerate curve in $\mathbb{P}^N$ means that the
restriction map $\bar{\varphi}_1 : H^o ( \mathbb{P}^N ,
O_{\mathbb{P}^N}(1)) \to H^o( X_{red'} \; L_{X_{\rm red}})$ has nontrivial 
kernel, say $W_o$. Let $\dim W_o = N_o$. Choose a basis of $W_1 = H^o (
\mathbb{P}^N , O_{\mathbb{P}^N}(1))$ relative to the filtration $0
\subset W_o \subset W_1$, say $w_o, w_1, \ldots , w_{N_o -1}, \ldots
, w_N$, (cf.\ definition~\ref{chap0:subdef0.2.6} 
page~\pageref{chap0:subdef0.2.6}).   

Let $\lambda$ be a $1-ps$ of $GL(N+1)$ such that the induced action of
$\lambda$ on $W_1$ is given by,  
\begin{align*}
\lambda(t)w_i & = w_i, \quad t \in K^*, \quad (0 \le i \le N_o -1),\\
\lambda(t)w_i & = w_i, \quad t \in K^*, \quad (N_o \le i \le N).
\end{align*}

\noindent
Let $\lambda'$ be the $1-ps$ of $SL(N+1)$ associated to the $1-ps
\lambda$ of $GL(N+1)$, (cf.\ definition~\ref{chap0:subdef0.1.2}, 
page~\pageref{chap0:subdef0.1.2}). The 
rest of the proof consists of showing that $H_{m_o}(X)$ is not
$\lambda'$-semistable. 

Assume now that $m > m'$ so that $H^1 (X, L^m)=0$ and the restriction 
$$
\varphi_m : H^o ( \mathbb{P}^N, O_{\mathbb{P}^N}(m)) \to H^o (X,
L^m) \text{ surjective. } 
$$

\noindent
Let $B_m = \{M_1, M_2, \ldots, M_{\alpha_m}\}$ be a basis of $H^o
(\mathbb{P}^N, O_{\mathbb{P}^N}(m))$ consisting of monomials of
degree $m$ in $w_0, w_1, \ldots , w_N$, $(\alpha_m = h^o (
\mathbb{P}^N, O_{\mathbb{P}^N}(m)))$.\pageoriginale Recall that we have chosen
the integer $q_1$ such that $I^{q_1}_X=0$ where $I_X$ denotes the
ideal sheaf of nilpotents in $O_X$, 
(cf.\ page~\pageref{c1:nilpotents}). 

For $0 \le s \le q_1 - 1 \le m$ let $W_o^{q_1-s} \cdot W_1^{m-q_1+s}$ be
the subspace of $H^o (\mathbb{P}^N, O_{\mathbb{P}^N}(m))$
generated by elements $w$ of the type 
$$
w = x_1 x_2 \cdots x_{q_{_1} -s} y_1 y_2  \cdots y_{m-q_1+s} 
\left[\begin{aligned} & x_i ~\in ~W_o, \quad 1 \le i \le q_1-s\\  
& y_i~ \in~ W_1, \quad 1 \le i \le m-q_1 +s \end{aligned} \right.
$$

Put $W^o_o . W^m_1 = H^o( \mathbb{P}^N, O_{\mathbb{P}^N}(m))$. We
have the following filtration  of $H^o ( \mathbb{P}^N, O_{\mathbb{P}^N}(m))$,  
\begin{align*}
& 0 \subset W_o^{q_1}. W_1^{m-q_1} \subset W_o^{q_1-1}. W^{m-q_1+1}_1
\subset
W^{q_1-2}_o . W^{m-q_1+2}_1 \subset \\
& \qquad \ldots W^1_o . W^{m-1}_1 \subset
W^o_o . W^m_1 = H^o (\mathbb{P}^N, O_{\mathbb{P}^N} (m)), \tag*{(F)}
\end{align*}

\noindent
For $\le s \le q_1 < m $ let,
\begin{align*}
\bar{W}_o^{q_1-s}. \bar{W}_1^{m-q_1+s} &=
\varphi_m (W_o^{q_1-s}. W_1^{m-q_1+s})\\
&\qquad \subset H^o (X,L^m), \dim
\bar{W}^{q_1-s}_o . \bar{W}_1^{m-q_1+s} = \beta_s . 
\end{align*}

\noindent

These subspaces define the following filtration of $H^o(X,L^m)$.
\begin{align*}
0 & = \bar{W}^{q_1}_o. \bar{W}_1^{m-q_1} \subset
\bar{W}_o^{q_1-1} . \bar{W}_1^{m-q_1+1} \subset
\bar{W}_o^{q_1-2} . \bar{W}_1^{m-q_1+2} \subset\\
& \qquad \ldots \bar{W}^1_o . \bar{W}^{m-1}_1 \subset \bar{W}^o_o
. \bar{W}^m_1 = H^o (X,L^m), \tag*{$\overline{(F)}$}
\end{align*}

\noindent
Rewrite the basis $B_m$ as $B_m = \{M_1', M_2', \ldots , M'_{P(m)},
M'_{P(m)+1}, \ldots , M'_{\alpha_m}\}$ such that $\{\varphi_m (M_1'),
\varphi_m(M_2') \ldots , \varphi_m (M'_{P(m)})\}$ is a basis of
$H^o(X,L^m)$ relative to the filtration $\bar{(F)}$ and
$M'_{P(m)+1}, M'_{P(m)+2}, \ldots , M'_{\alpha_m}$ are the rest of the
monomials in $B_m$ in some order. We now estimate, total\pageoriginale
$\lambda$-weight of $M'_1, M'_2 , \ldots , M'_{P(m)} =
\sum\limits_{i=1}^{P(m)} w_\lambda (M'_i)$, (cf.\ 
definition~\ref{chap0:subdef0.1.1} page~\pageref{chap0:subdef0.1.1}). 
It follows from the definition of $\lambda$, that a monomial 
$M \in W_o^{q_1-s}. W_1^{m-q_1+s} -
W_o^{q_1-s+1}. W_1^{m-q_1+s-1}$ has $\lambda$-weight $W_\lambda
(m) = m-q_1+s$, $(1 \le s \le q_1)$. 
\begin{align*}
\sum^{P(m)}_{i=1} w_\lambda (M'_i) & = m( \beta_{q_1} - \beta_{q_1-1})
+ (m-1) (\beta_{q_1-1} - \beta_{q_1-2}) +\\ 
&\qquad\ldots + (m-q_1+1)\beta_1\\ 
& = m \beta_{q_1} - \sum_{s=1}^{q_1-1} \beta_s \ge m(dm-g+1)- (q_1-1)
(dm-g+1)\\
& \qquad (\because \beta_s \le h^o (X,L^m) = dm-g+1, 1 \le s \le q_1-1)
\end{align*}

\noindent
Thus $\sum\limits_{i=1}^{P(m)} w_ \lambda(M'_i) \ge (m-q_1+1)
(dm-g+1)$, $\quad (E_1)$ 

\noindent
total $ \lambda$-weight of  $w_o, w_1, \ldots ,w_N =
\sum\limits^{N}_{i=0} w_\lambda (w_i)$ 
\begin{align*}
& = \dim W_1 - \dim W_o, \text{ (Follows from the definition of
    $\lambda$ )}\\
& = d-g+1 - \dim W_o \le d-g,  \qquad (\because \dim W_o  \ge 1)
\end{align*}

Thus $ \sum\limits_{i=0}^{N} w_ \lambda (w_i) \le d-g$,  $\quad (E_2)$

We are now ready to get the contradiction  that the $m_o^{th}$ Hilbert
point of $X, H_{m_o}(X)$ is not semistable. 

If $H_m(X)$ is $\lambda '$ semistable then there exists monomials 
$M'_{i_1} M'_{i_2}, \ldots,\break M'_{i_{P(m)}}$ in $B_m'(1 \le i_1 < i_2 <
\cdots < i_{P(m)} \le \alpha_m)$, such that $\{\varphi_m (M'_{i_1}),
\varphi_m\break (M'_{i_2}) , \ldots ,  \varphi_m (M'_{i_{P(m)}})\}$ is a
basis of $H^o (X,L^m)$ and 
$$\dfrac{\sum\limits_{i=1}^{P(m)} w_\lambda
  (M'_{i_j})}{mP(m)} \le \dfrac{\sum\limits_{i=0}^{N} w_\lambda
  (w_i)}{d-g+1}$$ 
(cf.\ criterion $(**)$ page~\pageref{page10}). 

\noindent
Observe\pageoriginale that $\sum\limits_{i=1}^{P(m)} w_{\lambda}
(M'_i) \le 
\sum\limits_{j=1}^{P(m)} w_\lambda(M'_{i_j})$. Note the following.   
$$
H_m(X) ~ \text{ is  }~ \lambda'-\text{ semistable } \Rightarrow 
\frac{\sum\limits_{i=1}^{P(m)} w_ \lambda (M'_{i_j})}{m(dm-g+1)} \le
\frac{\sum\limits_{i=0}^{N} w_\lambda (w_i)}{d-g+1} 
$$
\begin{align*}
& \Rightarrow \frac{(m- q_1+1) (dm-g+1)}{m(dm-g+1)} \le
  \frac{d-g}{d-g+1} \text{ (Follows from $(E_1)$, $(E_2)$)}\\ 
& \Rightarrow 1 - \frac{q_1-1}{m}   \le \frac{d-g}{d-g+1} \Rightarrow
  \frac{1}{d-g+1} \le \frac{q_1-1}{m}\\ 
& \Rightarrow m \le (d-g+1) (q_1-1) \Rightarrow m < m_o \\
& \qquad \qquad (\because (d-g+1)(q_1-1) < m_o)
\end{align*}

\noindent
Thus $H_{m_o}(X)$ is not $\lambda'$-semistable. In particular it
follows that $H_{m_o}(X)$ is not semistable. (cf.\ 
theorem~\ref{chap0:subthm0.0.9} page~\pageref{chap0:subthm0.0.9}. 
This contradiction proves the result.   

The above proof can be considered as the prototype of the proofs of
the next propositions~\ref{chap1:subprop1.0.3}.,
\ref{chap1:subprop1.0.4}., \ref{chap1:subprop1.0.5}.,
\ref{chap1:subprop1.0.6}.   

\begin{subprop}\label{chap1:subprop1.0.3}%% 1.0.3.
Every curve $X$ in the family $Z_V \xrightarrow{p_V}V$ is generically
reduced i.e. the local ring of $X$, at each generic point of $X$, is
reduced. 
\end{subprop}


\begin{proof}
Assume the contrary. Let $X$ be a curve in the family $Z_V
\xrightarrow{p_V}V$ such that $X$ is not generically reduced. Write
$X= \bigcup\limits_{i=1}^{p}X_i$, ($X_i$, an irreducible component of
$X$, $1 \le i \le p$) so that the local ring of $X$ at the generic point of
$X_1$ is not reduced. We will show that this leads to the
contradiction that the $m^{\rm th}_o$\pageoriginale Hilbert point of
$X$, $H_{m_o}(X) 
\in \mathbb{P} (\overset{P(m_o)} \Lambda$ $( \mathbb{P}^N,
O_{\mathbb{P}^N}(m_o)))$ is not semistable. This contradiction will 
then prove the result.   
\end{proof}

Let $\deg_{X_{\rm ired}} L = e_i$, $1 \le i \le p$. It is easy to see that
$\deg L = d = \sum\limits^{p}_{i=1} k_i e_i$ for some positive integers
$k_1, k_2 , \ldots ,k_p$, with $k_1 \ge 2$. Let $W_o$ be the kernel of
the natural restriction map  
$$
\bar{\varphi_1} : H^o(\mathbb{P}^N, O_{\mathbb{P}^N} (1)) \to H^o
(X_{\rm lred}, L_{X_{\rm lred}}). 
$$

\medskip
\noindent{\textbf{Claim :}}
$W_o \neq 0$.

\medskip
\noindent{\textbf{Proof of the Claim:}}
Look at the exact sequence, 
$$
0 \longrightarrow 0_{X_{\rm lred}}  \longrightarrow L_{X_{\rm lred}}
\longrightarrow O_{D_1} \rightarrow 0, 
$$ 
where $D_1$ is a divisor on $X_{\rm lred}$, corresponding to the line bundle
$L_{X_{\rm lred}}$ on $X_{\rm lred}$, such that $D_1$ has support in the
smooth locus of $X_1$. It follows from the corresponding long exact
cohomology sequence that  
$$
h^o (X_{\rm lred}, L_{X_{\rm lred}}) \le h^o (X_{\rm lred}, O_{D_1}) + h^o 
(X_{\rm lred}, O_{X_{\rm lred}}) = e_1 +1.  
$$

\noindent
Now\pageoriginale note the following. 
$$
W_o = 0 \Rightarrow d-g+1 = h^o(\mathbb{P}^N, O_{\mathbb{P}^N}(1))
\le h^o (X_{\rm lred}, L_{X_{\rm lred}}) \le e_1+1 
$$ 
\begin{align*}
&\Rightarrow d-g \le e_1 \Rightarrow k_1 (d-g) + \sum_{i=2}^{p} k_i e_i
  \le k_1 e_1 + \sum_{i=2}^p k_i e_i =d\\ 
&\Rightarrow (k_1-1)d \le k_1g - \sum_{i=2}^{p} k_i e_i \le k_1g
  \Rightarrow  \frac{(k_1-1)d}{k_1} \le g\\ 
&\Rightarrow \frac{d}{2} \le g ,  (\because k_1 \ge 2 \therefore
  \frac{k_1-1}{k_1} \ge \frac{1}{2})\\[3pt]
& \qquad  d \le 2g .  
\end{align*}

It is immediate from the above contradiction that $W_o \neq 0$. Also
in view of the Proposition~\ref{chap1:subprop1.0.2} 
(page~\pageref{chap1:subprop1.0.2}) it follows that $X$ 
cannot be irreducible. 

Let $\dim W_o = N_o$. Choose a basis of $W_1 = H^o ( \mathbb{P}^N,
O_{\mathbb{P}^N}(1))$ relative to the filtration $0 \subset W_o
\subset W_1$, say $\{w_o , w_1, \ldots , w_{N_o-1}, w_{N_o}, \ldots , w_N\}$,
(cf. Definition~\ref{chap0:subdef0.2.6} page~\pageref{chap0:subdef0.2.6}).  

\noindent
Let $\lambda$ be a $1-ps$ of $GL(N+1)$ such that the action of
$\lambda$ on $W_1$ is given by,  
\begin{align*}
\lambda (t)w_i & = w_i, \quad t ~\in ~K^*, (o \le i \le N_o -1),\\
\lambda (t)w_i & = tw_i, \quad t ~\in ~K^*, (N_o \le i \le N).
\end{align*}

\noindent
Let $\lambda'$ be the $1-ps$ of $SL(N+1)$ associated to the $1-ps$
$\lambda$ of $GL(N+1)$, (cf.\ definition~\ref{chap0:subdef0.1.2} 
page~\pageref{chap0:subdef0.1.2}). The 
rest of the proof consists of showing that $H_{m_o}(X)$ is not
$\lambda'$ -semistable. 

Assume\pageoriginale now that $m > m'$ so that $H^1 (X,L^m)=0$ and the
restriction  

$\varphi_m : (H^o( \mathbb{P}^N, O_{\mathbb{P}^N}(m)) \to H^o
(X,L^m)$ is surjective. Recall that $H^o$\break $( \mathbb{P},
O_{\mathbb{P}^N}(m))$ has a basis $B_m = \{M_1, M_2 , \ldots , 
M_{\alpha_m}\}$ consisting of monomials lof degree $m$ in $w_o, w_1,
\ldots , w_N$, $(\alpha_m = h^o(\mathbb{P}^N, O_{\mathbb{P}^N}(m)))$. 

For $0 \le r  \le m$, let $W_o^{m-r}. W^r_1$ be the subspace of
$H^o(\mathbb{P}, O_{\mathbb{P}^N}(m))$ generated by elements $w$ of
the following type.  

For $r=0$,
$$
w= x_1x_2, \ldots , x_m , \quad (x_j \in W_o , \quad 1 \le j \le m),
$$
for $0 < r < m$,
\begin{align*}
&w = x_1 x_2 , \ldots , x_{m-r} y_1 y_2, \ldots , y_r,\\  
&(x_j \in
W_o , 1 \le j \le m-r; y_j \in W_1,  1 \le j \le r), 
\end{align*}
for $r = m$,
$$
w = y_1 y_2 , \dots , y_m, \quad (y_j \in W_1,  \quad 1 \le j \le m). 
$$ 
We have the following filtration of $H^o(\mathbb{P},
O_{\mathbb{P}^N}(m))$.   
\begin{align*}
& 0 \subset W^m_o. W^o_1 \subset W_o^{m-1} \cdot W_1^1 \subset
W^{m-2}_o. W^2_1 \subset \cdots \\
& \subset W_o^{q_1}. W^{m-q_1}_1 \subset
W_o^{q_1-1}. W_1^{m-q_1+1}\subset \cdots \subset  W_o^o. W^m_1 =
H^o(\mathbb{P}, O_{\mathbb{P}^N}(m)), \tag*{(F)} 
\end{align*}
Let $\bar{W}_o^{m-r}. \bar{W}^r_1 = \varphi_m (W_o^{m-r}. W^r_1)
\subset H^o (X,L^m), \dim \bar{W}_o^{m-r}. \bar{W}^r_1 = \beta_r, 0
\le r \le m$. 


\noindent
These\pageoriginale subspaces define the following filtration of 
$H^o(X, L^m)$.  
\begin{align*}
& 0 \subset \bar{W}^m_o.  \bar{W}^o_1. \bar{W}^{m-1}_0 \cdot \bar{W}^{1}_1 \subset
\bar{W}^{m-2}_0. \bar{W}^2_1 \subset \cdots\\
&  \subset \bar{W}^{q_1}_0. \bar{W}_1^{m-q_1} \subset
\bar{W}_0^{q_1-1}. \bar{W}^{m-q_1}_1 \subset \cdots \subset
\bar{W}^o_0. \bar{W}^{m}_1 = H^o (X,L^m),   \tag*{$(\bar{F})$} 
\end{align*}

\noindent
Rewrite the basis $B_m = \{M'_1, M'_2, \ldots, M'_{P(m)}, \ldots ,
M'_{\alpha_m}\}$ such that \break $\{\varphi_m (M'_1), \varphi(M'_2), \ldots
, \varphi_m (M'_{P(m)})\}$ is a basis of $H^o (X,L^m)$ relative to the
filtration  $\bar{(F)}$ and $M'_{P(m)}, M'_{P(m)+1} , \ldots ,
M'_{\alpha_m}$ are the rest of the monomials in $B_m$ in some order. 

Let $C$ be the closure of $X - X_1$ in $X$. Since $X$ is connected,
there exists a (closed) point, say $P \epsilon X_1 \cap C$.  

\begin{claim*}
$C$ can be given the structure of a closed subscheme of $X$ such that
  the kernal of the restriction map $\varphi'_m :H^o(X,L^m) \to H^o
  (C, L^m_C)$ intersects $\bar{W}^{m-r}_o. \bar{W}^r_1$ in the null
  space  i.e. $\bar{W}^{m-r}_o. \bar{W}^r_1 \cap \kernel \varphi'_m
  =0$, $0 \le r \le m -q_1$. 
 \end{claim*} 
 
 The proof of the above claim is somewhat technical. 
Hence, assuming the claim we will prove the proposition and then we
will go to the proof of the claim. 
 
 Let $I$ denote the ideal subsheaf of $O_C$ defining the point $P
 \in C$.  
 
 The exact sequence $0 \to I^{m-r} \otimes L^m_C \to L^m_C \to
 \dfrac{O_C}{I^{m-r}} \otimes L^m_C \to 0$, given the following long
 exact sequence. 
 \begin{align*}
 0 \to H^o (C,I^{m-r} \otimes L^m_C) &\to H^o (C,L^m_C) \to H^o(C,
 \frac{O_C}{I^{m-r}} \otimes L^m_C )\\ 
&\to H^1 (C, I^{m-r} \otimes  L^m_C) \to 0 
 \end{align*}
 \pageoriginale
 
 Now make the following observations.
 \begin{enumerate}[i)]
 \item $h^o  (C, L^m_C) = \chi (L^m_C) = \deg_C L^m + \chi (O_C) \le
   (d-2e_1)m + h^o(C,O_C) < (d- 2e_1) m+q_3$  \quad (\text{cf.~assertion
   iv, page~\pageref{c1:as4}}). 

 \item Since $\dfrac{O_C}{I^{m-r}} \otimes L_C^m$ has support only at
   the point $P \in C$, 

\noindent
$h^o(C, \dfrac{O_C}{I^{m-r}} \otimes
   L_C^m) = \dim \bigg[ \dfrac{O_{C,P}}{m^{m-r}_{C,P}}\bigg] \ge m-r$,

\noindent
   ($O_{C, P}$ is the local ring of $C$ at $P$ and $m_{C,P}$ is the
   maximal ideal in $O_{C,P}$).  

 \item Note that $h^o (C, \dfrac{O_C}{I^{m-r}} \otimes L_C^m ) = \dim
   \bigg[ \dfrac{O_{C,P}}{m^{m-r}_{C,P}}\bigg] \le \mu_1 (m-r) + \mu_2$
   (cf.\ assertion $v$, page~\pageref{c1:as5}). Hence it follows from the above
   long exact cohomology sequence that $h^1(C, I^{m-r} \otimes L^m_C)
   \le \mu_1 (m-r) + \mu_2$, $0 \le r \le m'' -1$.  

\item For $m'' \le r \le m-q_1$, $H^1 (C, I^{m-r} \otimes L^m_C)=0$ 
  (cf.\ assertion $vi$ page~\pageref{c1:as6}). 

\item By definition the image of $\bar{W}^{m-r}_o . \bar{W}^{r}_1
  \subset H^o (X,L^m)$ under the restriction $\varphi'_m$ is contained
  in the subspace $H^o(C, I^{m-r} \otimes L^m_C)$ of $H^o(C,L^m_C)$. 
 \end{enumerate} 
 
 It\pageoriginale follows that,
 \begin{align*}
	\beta_r & = \dim(\bar{W}^{m-r}_o. \bar{W}^r_1) \leq h^o
   (C,I^{m-r}\otimes L^m_C)\\ 
	& = h^o (C,L^m_C) - h^o(C, \frac{O_C}{I^{m-r}} \otimes L^m_C) +
   h^1 (C, I^{m-r}\otimes L^m_C)\\ 
	& \le(d- 2e_1) m + q_3 + r-m + \mu_1 (m-r) + \mu_2, \quad (0 \le r \le
   m''-1)\\ 
	&\beta_r \le (d-2e_1) m + q_3 + r-m, \quad (m'' \le r \le m-q_1)\\
	&\beta_r  \le dm-g+1, \quad (m-q_1+1 \le r \le m-1)
 \end{align*} 
 (For the last inequality note that, $\beta_r \le h^o(X,L^m)=
 dm-g+1$). 
 
 \noindent
 We now estimate total  $\lambda$-weight of $M'_1, M'_2, \ldots ,
 M'_{P(m)} = \sum\limits^{P(m)}_{i=1} w_\lambda (M'_i)$.  
 
 
 \noindent
 Note that a monomial $M \in W_o^{m-r} W^r_1 - W^{m-r+1}_o W^{r-1}_1$ has
 $\lambda$-weight\break $w_\lambda(M)=r$. 
{\fontsize{10}{12}\selectfont
 \begin{align*}
& \sum_{i=1}^{P(m)} w_\lambda (M'_i) = \sum^{m}_{r=1} r ( \beta_r-
   \beta_{r-1}) = m\beta_m - \sum_{r=0}^{m-1} \beta_r\\ 
& = m \beta_m - \sum_{r=0}^{m''-1} \beta_r - \sum_{r=m''}^{m-q_1}
   \beta_r - \sum_{r=m-q_1+1}^{m-1}\beta_r\\ 
& \ge m(dm-g+1) - \sum_{r=0}^{m-q_1} \bigg [ (d- 2e_1 )m + q_3 +r-m
     \bigg ] \\
& \qquad - \sum_{r=0}^{m''-1} \bigg[ \mu(m-r)+ \mu_2 \bigg]-
   \sum_{r=m-q_1+1}^{m-1}\bigg[ dm-g+1 \bigg ]\\ 
& = dm^2 + m(1-g)- \bigg [ (m-q_1+1)(d-2e_1)m +(m-q_1+1) q_3\\ 
&\qquad+\frac{(m-q_1)(m-q_1+1)}{2} -(m-q_1+1)m \bigg]\\ 
&\qquad-\bigg[\mu mm''-\mu_1 \frac{m''-1}{2} + \mu_2 m'' \bigg]-  (q_1-1)(dm-g+1)
   \\
& = (2e_1 + \frac{1}{2})m^2 - m \left[(1-q_1) (d-2e_1) + q_3 +
     \frac{1}{2} - q_1-(1-q_1) + \mu_1 m''\right.\\
& \qquad \left. + s(q_1-1) +g -1 \right] +
   (q_1-1)q_3 - \frac{q_1(q_1-1)}{2} + \frac{\mu_1 m'' (m''-1)}{2}\\
& \qquad  - \mu_2 m''+ (q_1-1) (g-1)\\ 
 & \ge (2e_1 + \frac{1}{2}) m^2 - m \left[(q_1 -1) (2e_1 + 1) +
   (q_3 - q_1) + \mu_1 m'' + g - \frac{1}{2}\right ]\\ 
 & = \left(2e_1 + \frac{1}{2}\right) m^2 - mS,\\ 
&\qquad\left(S = (q_1 -1)(2e_1 + 1) + (q_3 -
 q_1) + \mu_1 m'' + g - \frac{1}{2}\right)
 \end{align*}}\relax\pageoriginale
Thus,
\begin{equation*}
  \sum\limits_{i = 1}^{P(m)} w_\lambda (M'_i) \ge (2e_1 +
\frac{1}{2}) m^2 - mS,  \tag*{$(E_l)$}
\end{equation*}

\noindent
Clearly,

\noindent
$\sum\limits^{N}_{i = 0} w_\lambda (w_i) = \dim W_1 - \dim W_o$, \; 
(Follows from the definition of $\lambda$) 
\begin{equation*}
\le h^o (X_{\rm lred}, L_{X_{\rm lred}}) = e_1 + 1, \tag{$E_2$}
\end{equation*}

\noindent
We are now ready to get the contradiction that $m_o^{th}$ Hilbert
point of $X$, $H_{m_o}(X)$ is not $\lambda'-$ semistable. 

If $H_m (X)$ is $\lambda' - $ semistable $(m > m')$ then there exists
monomials $M'_{i_1}, M'_{i_2}, \ldots , M'_{i_{P(m)}} (1 \le i_1 < i_2
< \cdots < i_{P (m)} \le \alpha_m)$ such that $\{ \varphi_m
(M'_{i_1}), \varphi_m(M'_{i_2}) , \ldots , \varphi_m (M'_{i_{P(m)}})
\}  $ is a basis of $H_o (X, L^m)$ and  
$$\dfrac{\sum\limits^{P(m)}_{j
    = 1} w_\lambda (M'_{i_j})}{m (dm-g+1)} < \dfrac{\sum\limits^{N}_{i
    = 0} w_\lambda (w_i)}{d - g + 1},$$ 
(cf.\ criterion $(**)$ page
\pageref{page10}). It is easy to see that 
$$\sum\limits^{P (m)}_{i = 1}
w_\lambda (M'_i) \le \sum\limits_{j = 1}{P (m)} w_\lambda (M'_{i_j}).$$ 


\medskip
Thus,\pageoriginale

\noindent
$H_m (X)$ is $\lambda' - $semistable $(m > m')$
\begin{align*}
\Rightarrow & ~ \frac {\sum\limits_{i = 1}^{P(m)} w_\lambda (M'_i)}{m
  (dm-g+1)} \le \frac{\sum\limits_{i = 0}^{N}w_\lambda (w_i)}{d -
  g+1}\\ 
\Rightarrow & ~ \frac{(2e_1 + \frac{1}{2}) m^2 - ms}{m(dm-g+1)} \leq
\frac{e_1+1}{d-g+1},  \quad \text{ (Follows from } (E_1) \text { and } (E_2))\\ 
\Rightarrow & ~ \frac{(2e_1 + \frac{1}{2}) - \frac{S}{m}}{d} \le
\frac{e_1 +1}{d - g + 1}\\ 
\Rightarrow& (d -g +1) (2e_1 + \frac{1}{2}) -
\frac{S(d-g+1)}{m} < d(e_1 + 1)\\ 
\Rightarrow & ~ d(e_1 - \frac{1}{2}) - (g-1) (2e_1 + \frac{1}{2}) \le
\frac{S (d-g+1)}{m} \Rightarrow 1 \le \frac{S(d-g+1)}{m}, \\
& \qquad \qquad  (\because
d(e_1 - \frac{1}{2}) - (g-1) (2e_1 + \frac{1}{2}) \ge 1)\\ 
\Rightarrow & ~ m \le S (d-g+1) \Rightarrow m < m_o, \qquad (\because
m_o > S(d-g+1))  
\end{align*}

This proves that the $m_o^{\rm th}$ Hilbert point of $X, H_{m_o}(X)$ is
not $\lambda' -$\break semistable. In particular, it follows that
$H_{m_o}(X) $ is not semistable, (cf.\ theorem~\ref{chap0:subthm0.0.9}
page~\pageref{chap0:subthm0.0.9}). This contradiction proves the result. 

It remains to prove the claim. Let $P_1, P_2, \ldots , P_t$ be all the
associated (closed) points of $X$, (a point $Q \in X $ is
called an associated point of $X$ if the maximal ideal in the local
ring $0_{X, P}$ of $X$ at $P$ is associated to the zero ideal). Choose a
finite affine open cover $\{ U_i \}$ of $X$ such that any of the
points $P_1, P_2, \ldots , P_t$\pageoriginale belongs to exactly one
of the $U'_i s 
$ in $\{ U_i \}$ and further $L_{U_i} $ is trivial for every $U_i$ in
$\{ U_i \}$. 
   
Let $U_i \simeq \Spec A_i$ and let $U_i \bigcap U_k \simeq \Spec
A_{ik}$. In the ring $A_i$ let $(0) = \sum\limits^{n_i}_{j = 1}
q_{ij}$ be a primary decomposition of the zero ideal with $q_{ij}$,
$p_{ij}-$primary for some prime ideal $p_{ij}$ in $A_i$, $(1 \le j \le
n_i)$. We can assume without loss of generality that in those $U_i$
such that $U_i \bigcap X_1 \neq \emptyset$, $X_{\rm 1\red}$  is defined by the
prime ideal $q_{i1}$. 

Define an ideal subsheaf $J$ of $O_X$ as follows. If $U_i \bigcap
X_1 \neq \emptyset$, then in $U_i$, $J$ is defined by $\bigcap\limits_{j =
  2}^{n_i} q_{ij}$. If $U_i \bigcap X_1 = \emptyset$ then $J$ is defined
by $\sum\limits^{n_i}_{j = 1} q_{ij} = (0)$. In $\Spec A_{ik}= U_i
\bigcap U_k (i \neq k)$ there is no associated (closed) point of $X$
hence all the primary ideals in a primary decomposition of the zero
ideal in $A_{ik}$ are minimal and hence are uniquely determined. Thus
the above construction indeed defines an ideal sheaf. Let $C$ be the
closed subscheme of $X$ defined by the ideal $J$. Let, $\varphi'_m
: H^o (X, L^m) \to H_o (C, L^m_C)$ be the natural restriction. We now
proceed to prove that $\bar{W}^{m-r}_{0}\cdot \bar{W}^{r}_1 \bigcap $
Kernel $\varphi'_m = 0$, $(0 \le r \le m-q_1)$. 

Let $s \in \bar{W}^{m-r}_o \cdot \bar{W}^{-r}_1 \bigcap$ Kernel
$\varphi'_m$. It suffices to prove that for every open set $U_i$ in
the cover $U_i$, the restriction $s_i$ of $s$ to $U_i$ is zero. Let
$\gamma_i$ be the isomorphism $L_{U_i} \simeq
\tilde{A}_i$.\pageoriginale Let 
$\gamma_i(s_i) = q_i$. If $U_i \bigcap X_1 = \emptyset$ since $s_i
\in$ Kernel $\varphi'_m$ means that $s_i = 0$. If $U_i
\bigcap X_1 \neq \phi$ write $a_i = b_1 b_2 \cdots b_{m-r} c_1c_2
\cdots c_r$ where $b_1, b_2, \ldots , b_{m-r}$ are the images of the
sections in $W_o$ and $c_1, c_2, \ldots , c_r$ are the images of the
sections in $W_1$, under the isomorphism $I_{U_i} \simeq
\tilde{A}_i$. Since $m - r \ge q_1$, $a_i \in
p^{q_1}_{i1}$. It is easy to see that since $(\bigcap\limits^{n_i}_{j
  = 1} p_{ij})^{q_1}= 0$ and $p_{i1}$ is a minimal prime,
$p^{q_1}_{i1} \subset q_{i1}$. Thus $a_i \in q_{i1}$. Now $s_i
\in$ Kernel $\varphi'_m$, hence  $a_i \in \bigcap\limits_{n  =
  2}^{n_i} q_{ij}$. It follows that $a_i \in
\bigcap\limits^{n_i}_{j = 1}q_{ij} = 0$ i.e. $s_i = 0$. This completes
the proof of the claim. 

Now we want to prove that for every curve $X$ in the family $Z_V
\xrightarrow{p_V} V$, only singularities of $X_{\red}$ are ordinary
double points. This will follow from the next three 
propositions~\ref{chap1:subprop1.0.4}., \ref{chap1:subprop1.0.5}.,
\ref{chap1:subprop1.0.6}.  

\begin{subprop}\label{chap1:subprop1.0.4} % subprop 1.0.4 
  Let $X$ be a curve in the family $Z_V \xrightarrow {p_V}V$. Every
singular point of $X_{\red}$ is a double point. 
\end{subprop}

\begin{proof} % pro
Assume the contrary i.e. assume that there exists a point $P$ of
multiplicity $\ge 3$ on $X_{\red}$. We will show that this leads to
the contradiction that the $m_o^{\rm th}$ Hilbert point of $X$,
$H_{m_o}(X) \in \mathbb{P}  (\overset{P(m_o)}{\Lambda} H_o
(\mathbb{P}^N, O_{\mathbb{P^N}} (m_o)))$ is not semistable and then the
result will follow by the contradiction. 
\end{proof}

Let $\varphi : H^o (\mathbb{P}^N, O_{\mathbb{P}^N} (1)) \to k(P) $ be
the evaluation map, where $k(P)$ is the residue field at the point $P
\in X$. It is clear that\pageoriginale $W_o =$ kernel $\varphi$ has
dimension $N$. Choose a basis of $W_o$, say $w_0, w_1, \ldots ,
w_{N-1}$ and extend it to a basis of $W_1 =  H^o (\mathbb{P}^N,
O_{\mathbb{P}^N}(1))$ by adding a vector, say $w_N$. 

Let $\lambda$ be a $1-ps$ of $SL(N+1)$ such that the induced action of 
$\lambda$ on $W_1$ is given by, 
\begin{align*}
\lambda (t) w_i & = w_i, t \in K^*, (0 \le i \le N-1)\\
\lambda (t) w_N & = tw_N, t \in K^*.
\end{align*}

\noindent
The rest of the proof consists of showing that $H_{m_o}(X)$ is not
$\lambda$- semista\-ble. 

Let $\pi : \bar{X} \to X$ be the normalization of $X$ (cf.\ 
definition~\ref{chap0:subdef0.2.5} page~\pageref{chap0:subdef0.2.5}) 
and let $L' = \pi^* L$. Assume now that $m > m'$, 
so that $H^1 (x, L^m) = 0 = H^1 (\bar{X}, L'^{m})$ and the restriction 

$\varphi_m : H^o (\mathbb{P}^N, O_{\mathbb{P}^N}(m)) \to H^o (X, L^m)$
is surjective. Recall that $H^o (\mathbb{P}^N,\break O_{\mathbb{P}^N}(m))$
has a basis $B_m =  \{ M_1, M_2, \ldots, M_{\alpha_m}\}$ consisting of
monomials of degree $m$ in $w_0, w_1, \ldots , w_N$. 

For $0 \le r \le m$ let $W_o^{m-r}. W_1^r$ be the subspace of $H^o
(\mathbb{P}^N,\break O_{\mathbb{P}^N}(m))$ generated by elements $w$ of the
following type. 

For $r = 0$,

$$
w = x_1x_2 \cdots x_m , (x_j \in W_o, i \le j \le m); 
$$
 for $0 < r < m$, 
$$
w = x_1 x_2 \cdots x_{m-r} y_1 y_2 \cdots y_r, (x_j \in W_o, 1
\le j \le m-r ; y_j \in W_1, 1 \le j \le r) ; 
$$
for $r = m$,\pageoriginale
$$
w = y_1 y_2 \cdots y_m, \quad  (y_j \in W_1, 1 \le j \le m).
$$

\noindent
We have the following filtration of $H^o (\mathbb{P}^N,
O_{\mathbb{P}^N}(m))$. 
\begin{equation*}
0 \subset W_o^m. W_1^o \subset W_o^{m-1}. W_1^1 \subset \cdots \subset
W^o_o . W_1^m = H^o (\mathbb{P}^N, O_{\mathbb{P}^N} (m)),  \tag{F} 
\end{equation*}

\noindent
For $0 \le r \le m $ let, 
$$ 
\bar{W}_o^{m-r} . \bar{W}^{r}_1 ~ = ~ \varphi_m (W_o^{m-r}. W_1^r)
\subset H^o (X, L^m), \beta_r ~ = ~ \dim
\bar{W}^{m-r}_{0}. \bar{W}^{r}.  
$$

\noindent
These subspaces define the following filtration of $H^o (X, L^m)$.
\begin{equation*}
0 \subset \bar{W}^{m}_o  \cdot \bar{W}^o_1 \subset \bar{W}^m_o. \bar{W}^1_1
\subset ~ \cdots ~ \subset \bar{W}^o_0 \cdot \bar{W}^m_1 ~ = ~ H^o (X,
L^m), \tag{$\bar{F}$} 
\end{equation*}

\noindent
Rewrite the basis $B_m$ as $B_m = \{ M'_1, M'_2, \ldots ,
M'_{P(m)}, ~ M'_{P(m)+1}, \ldots , M'_{\alpha_m}\}$ such that $\{
\varphi_m (M'_1), ~ \varphi_m (M'_2), \ldots , \varphi_m
(M'_{P(m)})\}$ is a basis of $H^o (X, L^m)$ relative to the filtration
$\bar{F}$ (cf.\ definition~\ref{chap0:subdef0.2.6} 
page~\pageref{chap0:subdef0.2.6}) and $M'_{P(m)+1},\break  
M'_{P(m)+2}, \ldots , M'_{\alpha_m}$ are the rest of the monomials in
$B_m$ in some order. 

Since $P$ is a point of multiplicity $\ge 3$ on $X_{\red}$, we have
the following cases. 
\begin{enumerate}[i)]
\item There exists exactly one component of $X_{\red}$, say $X_{1}$,
  passing through $P$; 

\item There exist exactly two components of $X_{\red}$, say $X_1$ and
  $X_{2}$, passing through $P$; 

\item There exist at least three components of $X_{\red}$, say $X_1,
  X_2, X_3$, passing through $P$. 
\end{enumerate}

\noindent
In the\pageoriginale first case choose three points, say $P_1, P_2,
P_3$ (not necessarily distinct), from the fibre $\pi^{-1} (P)$ of
$\pi$ over 
$P$. In the second case note that at least one of the components $X_1$
and $X_2$, say $X_1$, has degree $\ge 3$ in $\mathbb{P}^N$ and $P
\in X_1$ is a singular point of $X_1$. Choose three points,
say $P_1, P_2, P_3$ from the fibre $\pi^{-1}(P)$ with $P_1, P_2
\in \bar{X}_1$ (not necessarily distinct), $P_3 \in
\bar{X}_2$, ($\bar{X}^1$ denotes the normalization of $X_1$ etc.). In
the third case choose 3 points, say, $P_1, P_2, P_3$ from the fibre
$\pi^{-1}(P)$ with $P_1 \in \bar{X}_1$, $P_2 \in
\bar{X}_2$, $P_3 \in \bar{X}_3$. In each of the above cases let
$D$ denote the divisor $P_1 + P_2 P_3$ on $\bar{X}$. 

We have homomorphisms, $\pi_{m_*} : H^o (X, L^m) \to H^o (\bar{X},
L^m)$. By definition the image of $\bar{W}^{m-r}_{0} \cdot \bar{W}^r_1
\subset H^o (X, L^m)$ under the homomorphism $\pi_{m_*}$ is contained
in the subspace $H^o (\bar{X}, L^m ((r-m)D))$ of $H^o (\bar{X},
L'^{m})$, $(0 \le r \le m-1)$. It follows that for $0 \le r \le m - 1$, 
\begin{align*}
\beta_r  = & \dim \bar{W}^{m-r}_{0}. \bar{W}^r_1 \le h^o (\bar{X},
L'^{m}((r-m)D)) + \dim (\kernel  \pi_{m*})\\ 
= & ~ dm + 3(r-m) - g_{\bar{X}} + 1 + h^1 (\bar{X}, L'^{m} (( r-m)D)) +
\dim (\kernel  \pi_{m*} )
\end{align*}
(The last equality follows from the Riemann-Roch theorem).

\begin{claim*}\label{c1:claim}
\begin{enumerate}[i)]
\item $\dim(\kernel \pi_{m*}) < q_2$,

 ($q_2$ is the integer given in assertion iii) page~\pageref{c1:as3}). 

\item $h^1 (\bar{X}, L'^m ((r-m)D)) \le 3(m-r)$, \quad $(0 \le r \le q = 2g -
  2)$.
 
\item $h^1 (\bar{X}, L'^m (( r - m)D)) = 0$, \quad $(q+1 \le r \le m-1)$.
\end{enumerate}
\end{claim*}

\medskip
\noindent{\textbf{Proof of the Claim: }\pageoriginale
\begin{enumerate}[i)]
\item Recall that the morphism $\pi : \bar{X} \to X$ has the following
factorization.
\[
\xymatrix{
\bar{X}\ar[rr]^{\pi}\ar[dr]^{\pi'} & & X\\
& X_{\red} \ar[ur]^i & 
}
\]

This gives the following commutative diagram.
\[
\xymatrix{
H^o(\bar{X}, L'^m) & & H^o(X, L^m)\ar[dl]_{i_{m*}} \ar[ll]_{\pi_{m*}}
\\
& H^o (X_{\red}, L^m_{X_{\red}}) \ar[ul]_{\pi'_{m*}}& 
}
\]

Since the homomorphism $\pi'_{m*}$ is injective, kernel $\pi_{m*} =
\kernel i_{m*}$. Let $I_X$ be the ideal sheaf of nilpotents in
$O_X$. $I_X$ has finite support, $X$ being generically reduced
(cf.\ Proposition~\ref{chap1:subprop1.0.3}. 
page~\pageref{chap1:subprop1.0.3}). Recall that $h^o (X, I_X) < 
q_2$. Consider the cohomology exact sequence, given by the following
exact sequence. 
$$
0 \to I_X \otimes L^m \to L^m \to L^m_{X_{\red}} \to 0
$$

\noindent
It follows that kernel $\pi_{m*} = \kernel i_{m*} = H^o (X, I_X
\otimes L^m)$ and hence $\dim(\kernel \pi_{m*}) = h^o (X, I_X \otimes
L^m) = h^o (X, I_X ) < q_2$.
 
\item In view of the fact that $H^1 (\bar{X}, {L'}^m) = 0$, this is
  immediate from the long exact cohomology sequence associated to the
  exact sequence  
$$
0 \to  {L'}^m ((r-m)D)) \to {L'}^m \to O_{(m-r)D} \to 0 
$$\pageoriginale

\item This follows from the following general fact. Let $C$ be an
  integral nonsingular curve of genus $g_C$ and let $M$ be a line
  bundle on $C$ of degree $\ge 2g_C -1$. Then $H^1 (C, M) = 0$. 
\end{enumerate}


It follows from the claim and the last inequality that 
{\fontsize{10}{12}\selectfont
\begin{align*}
\beta_r  & = \dim (\bar{W}^{m-r}_{0} . \bar{W}^r_1) \le dm + 3(r-m) -
g_{\bar{X}} + 1+3  (m-r) + q_2, (0 \le r \le q), \\ 
\beta_r  & = \dim (\bar{W}^{m-r}_{0}. \bar{W}^r_1) \le dm + 3 (r-m) -
g_{\bar{X}} + 1 + q_2, \quad (q+1 \le r \le m-1). 
\end{align*}}\relax

\noindent
We now estimate total $\lambda$-weight of $M'_1, M'_2, \ldots ,
M'_{P (m)} = \sum\limits_{i = 1}^{P(m)} w_\lambda (M'_i)$. 

\noindent
Note that a monomial $M \in W^{m-r}_o. W^r_1
- W_o^{m-r+1}. W^{r-1}_1$ has $\lambda$-weight $r$.  
\begin{align*}
& \sum_{i = 1}^{P(m)} w_\lambda (M'_i) = \sum_{r = 1}^{m} r(\beta_r -
  \beta_{r-1}) = m\beta_m - \sum_{r = 0}^{m - 1} \beta_r\\ 
& \ge m(dm-g+1) - \sum_{r = 0}^{m - 1} (dm + 3(r-m) - g_{\bar{X}} +
  1 + q_2) - \sum_{r = 0}^{q} 3(m-r)\\ 
& = \frac{3m^2}{2} - m(g- g_{\bar{X}} + 3q + \frac{3}{2} + q_2) +
  \frac{q(q+1)}{2}\\ 
& > \frac{3m^2}{2} - mS, \quad (S = (g - q_{\bar{X}} + 3q + \frac{3}{2} +
  q_2)).\\
& \text{Thus } \sum\limits_{i = 1}^{P (m)} w_\lambda (M'_i) >
  \frac{3m^2}{2} - mS,  \tag*{$(E_1)$}
\end{align*}

\noindent
Clearly, total $\lambda$-weight of $w_0, w_1, \ldots , w_N =
\sum\limits_{i = 0}^{N} w_\lambda (w_i) = 1, \qquad (E_2)$. 

\noindent
We are now ready to get the contradiction that the $m_o^{\rm th}$ Hilbert
point of $X$, $H_m (X)$ is not semistable. 

If\pageoriginale $H_m (X) $ is $\lambda$-semistable, $(m > m')$, then
there exist 
monomials $M'_{i_1}, M'_{i_2}, \ldots , M'_{i_{P(m)}}$ in $B_{m}, (1 \le
i_1 < i_2 < \cdots < i_{P(m)} \le  \alpha_m)$ such that $\{ \varphi_m
(M'_{i_1}), \varphi_m (M'_{i_2}), \ldots , \varphi_m (M'_{i_{P (m)}}
)\}$ is a basis of $H^o (X, L^m)$ and 
$$\dfrac{\sum\limits_{j = 1}^{P
    (m)} w_\lambda (M'_{i_j})}{mP(m)} \le \dfrac{\sum\limits_{i =
    0}^{N} w_\lambda (w_i)}{d-g+1},$$ 
(cf.\ criterion $(**)$ 
page~\pageref{page10}). It is easily seen that 
$$\sum\limits_{i = 1}^{P(m)} w_\lambda
(M'_i) \le \sum\limits_{j = 1}^{P(m)} w_\lambda (M'_{i_j}).$$ 

Now note the following.

$H_m (X)$ is $\lambda$-semistable, $(m > m')$,
\begin{align*}
\Rightarrow & ~ \frac{\sum_{i = 1}^{P (m)}w_\lambda (M'_i)}{m
  (dm-g+1)} \le \frac{\sum_{i = 0}^{N} w_\lambda (w_i)}{d-g+1}\\ 
\Rightarrow & ~  \frac{\frac{3m^2}{2}- mS}{m (dm-g+1)} \le
\frac{1}{d-g+1} \quad \text{ (Follows from $(E_1)$ and $(E_2)$)}\\ 
\Rightarrow & ~ \frac{\frac{3}{2}- \frac{S}{m}}{d} \le \frac{1}{d-g+1}
\Rightarrow \frac{3}{2}(d-g+1) - \frac{S(d-g+1)}{m}  \le  d\\ 
\Rightarrow & ~ \frac{1}{2}(d-3g+3) \le \frac{S(d-g+1)}{m} 
\Rightarrow  m \le \frac{2S(d-g+1)}{d-3g+3} \le 4S,\\
& \hspace{3cm} (\because d \ge
20(g-1) \therefore \frac{d-g+1}{d-3g+3} \le 2)\\ 
\Rightarrow & ~ m < m_o, \qquad (\because m_o > 4S).
\end{align*}

It\pageoriginale follows that $H_{m_o}(X)$ is not $\lambda$-semistable
and hence 
$H_{m_o}(X)$ is not semistable. (cf. theorem~\ref{chap0:subthm0.0.9}. 
page \pageref{chap0:subthm0.0.9}). This 
contradiction proves that the only singularities of $X_{\red} $ are
double points. 

Thus we have proved that if $X$ is a curve in the family $Z_V
\xrightarrow{p_V}V$ and $P \in X_{\red}$ is a singular point, then $P$
is necessarily a double point. The singular point $P$ is either a
cusp or a tacnode or an ordinary double point. The next two
propositions will exclude the first  two possibilities and this will
prove that if $X$ is a curve in the family $Z_V \xrightarrow{p_V}V$
then only singularities of $X_{\red}$ are ordinary double points. 

\begin{subprop}\label{chap1:subprop1.0.5} % subprop 1.0.5
If $X$ is a curve in the family $Z_V \xrightarrow{p_V}V$ then
$X_{\red}$ can not have a cusp singularity. 
\end{subprop}

\begin{proof}
If the result were not true then there exists a curve, say $X$, in the
family $Z_V \xrightarrow{p_V}V$ and a point $Q \in X_{\red}$
such that $Q$ is a cusp. Let $Y$ be the unique irreducible component
of $X$ passing through the point $Q \in X$. Let $C$ be the
closure of $X - Y $ in $X$. Let $\pi : \bar{X} \to X$ be the
normalization of $X$. By definition, $\bar{X}$ is a disjoint union of
$\bar{Y}_{\red}$ and $\bar{C}_{\red}$. Choose a point $P \in
\bar{Y}_{\red}$ such that $\pi (P) = Q$. Since the point $Q
\in X$ is a cusp. the morphism $\pi$ is ramified at the point
$P \in \bar{X}$. We will show that this leads to the
contradiction that the $m_o^{\rm th}$ Hilbert point of $X$, $H_{m_o} (X)
\in \mathbb{P} (\overset{P(m_o)}{\Lambda} H^o (\mathbb{P}^N,
O_{\mathbb{P}^N}(m_o)))$ is not semistable. The result will then
follow by the contradiction.  
\end{proof}

Since the morphism $\pi$ is ramified at the point $P \in
\bar{Y}_{\red} \subset \bar{X}$, $Y_{\red}$ is singular and hence
deg$_{Y_{\red}} L \ge 3$, (an integral curve of degree $\leq 2$ in
$\mathbb{P}^N$ is either a line or a conic and hence
nonsingular). Since the curve\pageoriginale $X$ is not contained in
any hyperplane 
in $\mathbb{P}^N$ (cf.\ proposition~\ref{chap1:subprop1.0.2}. 
page~\pageref{chap1:subprop1.0.2}), we think of 
$W_3 = H^o (\mathbb{P}^N, O_{\mathbb{P}^N} (1))$ as a subspace of $H^o
(X, L)$. Let, 
\begin{align*}
W_o & =  \{ s \in W_3 | \pi_*s \text { vanishes to order }
\ge 3 \text { at } P\}, \dim W_o = N_o,\\ 
W_1 & =  \{ s \in W_3 | \pi_* s  \text { vanishes to order }
\ge 2  \text{ at } P\}, \dim W_1 =  N_1. 
\end{align*}

\noindent
Choose a basis of $W_3$, relative to the filtration $0 \subset W_o
\subset W_1 \subset W_3$, say, $\{ w_1, w_2, \ldots , w_{N_o}, 
w_{N_o+1}, \ldots , w_{N_1},  w_{N_1+1}, \ldots, w_{N+1} \}$
(cf.\ definition~\ref{chap0:subdef0.2.6} 
page~\pageref{chap0:subdef0.2.6}). Let $\lambda$ be a $1-ps$ of 
$GL(N+1)$ such that the action of $\lambda$ on $W_3$ is given by, 
\begin{align*}
 \lambda (t) w_i  & =  w_i,  t \in K^*,  (1 \le i \le N_o),\\
 \lambda (t) w_i  & =  t w_i,  t \in K^*,  (N_o + 1 \le i \le N_1)\\
 \lambda (t) w_i  & =  t^3w_i,  t \in K^*,  (N_1 + 1 \le i \le N+1).
\end{align*}

There exists a $1-ps \lambda'$ of $SL(N+1)$, associated to the
$1-ps \lambda$ of $GL(N+1)$ (cf.\ definition~\ref{chap0:subdef0.1.2}
page~\pageref{chap0:subdef0.1.2}). The
rest of the proof consists of showing that the $m_o^{\rm th}$ Hilbert
point of $X$, $X$, 
$$H_{m_o}(X) \in \mathbb{P} ( \overset {P
  (m_o)}{\Lambda} H^o (\mathbb{P}^N, O_{\mathbb{P}^N}(m_o)))$$ 
is not $\lambda'$-semistable. 

Assume now that $m > m'$ so that $H^1 (X, L^m) = 0 = H^1 (\bar{X},
L'^m)$ and the restriction $\varphi_m : H^o (\mathbb{P}^N,
O_{\mathbb{P}^N} (m)) \to H^o (X, L^m)$ is surjective. Recall that
$H^o (\mathbb{P}^N, O_{\mathbb{P}^N}(m))$ has a basis consisting of
monomials of degree $m$ in $w_1, w_2, \ldots , w_{N+1}$, say $B_m = \{
M_1, M_2, \ldots , M_{\alpha_m}\}$, $(\alpha_m = h^o (P^N,
O_{\mathbb{P}^N} (m)))$. Let $\Omega^m_i$ be the subspace of $H^o
(\mathbb{P}^N, O_{\mathbb{P}^N}(m))$ spanned by 
$$\{ M \in B_m
| w_\lambda (M) \le i\},\quad (0 \le i \le 3m).$$ 

We\pageoriginale have the following filtration of $H^o(\mathbb{P}^{N},
O_{\mathbb{P}^{N}} (m))$.  
\begin{equation*} 
0 \subset \Omega^{m}_{o} \subset \Omega^{m}_{1}\subset \cdots \subset
\Omega^{m}_{3 m}  = H^{o}(\mathbb{P}^{N}, O_{\mathbb{P}^{N}} (m)),
\tag{F} 
\end{equation*}

\noindent
Let $\bar{\Omega}^{m}_{i} = \varphi_{m} (\Omega^{m}_{i}) \subset H^o
(X , L^{m}), \beta_{i} = \dim \bar{\Omega}^{m}_{i}, (0 \le i \le
3m)$.  

\noindent 
The above subspaces give the following filtration of $H^o(X, L^{m})$. 
\begin{equation*}
0 \subset \bar{\Omega}^{m}_{0} \subset \bar{\Omega}^{m}_{1} \subset
\ldots \subset \bar{\Omega}^{m}_{3m} = H^o (X, L^{m}),
\tag{$\bar{F}$} 
\end{equation*}

Rewrite the basis $B_{m}$ as $B_{m} = \{M'_{1}, M'_{2}, \ldots,
M'_{P(m)}, M'_{P(m)+1} , \ldots,\break M'_{\alpha_{m}}\}$ so that
$\{\varphi_{m} (M'_{1})\, \varphi_{m}(M'_{2}), \ldots,
\varphi_{m}(M'_{P(m)})\}$ is a basis of  $H^o (X, L^{m})$ relative
to the filtration $(\bar{F})$ and $M'_{P(m)+1}, M'_{P(m)+2}, \ldots ,
M'_{\alpha_{ m}} $,  are the rest of the monomials in $B_{m}$ in some
order.  

The morphism $\pi$ gives homomorphisms
$$
\pi_{m^*} : H^o (X, L^{m}) \longrightarrow H^o (\bar{X} , L'^{m}).  
$$

\begin{claim*}
The image  of $\bar{\Omega}^{m}_{i}$ under the homomorphism $\pi_{m*}$
is contained in the subspace $H^o (\bar{X}, L'^{m}((-3+i) P))$ of 
$H^o(\bar{X}, L'^{m})$, $(0 \le i \le 3m)$.  
\end{claim*}

\medskip
\noindent{\textbf{Proof of the claim: }}
First observe that for $i=0$ the claim follows from definition. Now it
suffices to prove that if $M$ is a monomial in $B_{m}$ such that $M
\in \Omega^{m}_{i} - \Omega^{m}_{i-1}$ then
$\pi_{m*}(M)\in H^o (\bar{X},L'^m((-3m+i)P)$, $(1 \leq i \leq
3m)$. 

Let $M \in \Omega_i^m - \Omega^m_{i-1}$. Suppose that $M$ has
$i_{0}$ factors from $\{w_{1}, w_{2}, \ldots,\break w_{N_{0}}\}$, $i_{1}$
factors from $\{w_{N_{0}+1} , w_{N_{0}+2}, \ldots, w_{N_{1}} \}$ and
$i_{3}$ factors from $\{w_{N_{1}+1},\break w_{N_{1}+2},\ldots , w_{N_{1}}
\}$. It follows that,  
$$
i_{0} + i_{1} + i_{3} = m ~\text{ and }~ i_{1} + 3i_{3} = i. 
$$\pageoriginale

By definition $\pi_{m*} (M) \in H^o (\bar{X}, L'^{m}
((-3i_{0} - 2i_{1}) P))$. Now note that $3m - i = 3(i_{0} + i_{1} +
i_{3}) - (i_{1} + 3i_{3}) = 3i_{0} + 2i_{1}$. This proves the claim.  

It follows from the above claim and the Riemann-Roch theorem that for
$0 \leq i \leq 3m-1$.  
\begin{align*}
\beta_{i} & = \dim \bar{\Omega}^{m}_{i} \leq h^{o} (\bar{X}, L'^{m}
((-3m+i) P)) + \dim (\kernel ~\pi_{m*})\\ 
 & = \dim - 3m + i - g_{\bar{X}} + 1 + h^{1} (\bar{X}, L'^{m}
((-3m+i)P)) + \dim (\kernel ~\pi_{m*} ) 
\end{align*}


\begin{claim*}
~\begin{enumerate}[i)]
\item $\dim (\kernel  \pi_{m*}) < q_{2}$.

\item $h^{1} (\bar{X}, L'^{m} ((-3m+i)D)) \leq 3m - i, \quad (0 \leq i
  \leq q = 2g-2)$.  

\item $h^{1}(\bar{X}, L'^{m} ((-3m+i)D)) = 0, \quad (q+1 \leq i \leq 3m-1)$. 
\end{enumerate}
($D$ denotes the divisor on $\bar{X}$, supported at $P \in
  \bar{X}$, with multiplicity one).  
\end{claim*}


\medskip
\noindent{\textbf{Proof of the claim: }}
\begin{enumerate}[i)]
\item Since the curve $X$ is generically reduced,

 $\dim (\kernel  \pi_{m*}) = h^{o} (X, I_{X}) < q_{2}$, 
(cf.\ page~\pageref{c1:claim}).  

\item In view of the fact that $H^{1} (\bar{X}, L'^{m}) = 0$, this
  follows from the long exact cohomology sequence associated to the
  following exact sequence  
$$
0 \longrightarrow L'^{m} ((-3m+i)D) \longrightarrow L'^{m}
\longrightarrow  \to O_{(3m-i) D} \to 0 
$$

\item Use\pageoriginale the following general fact. If $C$ is an
  integral smooth 
  curve of genus $g_{C}$ and if $M$ is a line bundle on $C$ with $\deg
  M \geq 2g_{C} -1$ then $H^{1} (C, M) = 0$.  
\end{enumerate}

\noindent
Hence, 
\begin{align*}
\beta_{i} & = \dim \bar{\Omega}^{m}_{i} \le dm - 3m+i - g_{\bar{X}} +
1 +  3m-i + q_{2}, (0 \le i \le q), \\ 
\beta_{i} & = \dim \bar{\Omega}^{m}_{i} \le dm - 3m+i - g_{\bar{X}} +
1 + q_{2}, \quad (q + 1\le i \le 3m-1).  
\end{align*}

We now estimate, total $\lambda$-weight of 
$$M'_{1}, M'_{2}, \ldots ,
M'_{P(m)} = \sum\limits^{P(m)}_{i=1} w_{\lambda}(M'_{i}).$$ 
Note that a monomial $M \in \Omega^{m}_{i} -  \Omega^{m}_{i-1}$, has $\lambda 
-$weight $w_{\lambda}(M) = i$.  
\begin{align*}
 & \sum^{P(m)}_{i=1}  w_{\lambda}(M'_{i})  =  \sum^{3m}_{i=1} i
  (\beta_{i} - \beta_{i-1}) = 3m\beta_{3m} - \sum^{3m-1}_{i=0}
  \beta_{i}\\ 
& \ge  3m (dm-g+1) - \sum^{3m-1}_{i=0} (dm -3m+i-g_{\bar{X}} + 1+q_{2})
- \sum^{q}_{i=0} (3m -i)\\ 
& \ge  \frac{9m^{2}}{2} - 3m(g - g_{\bar{X}} + q_{2} + q  +
\frac{1}{2})\\ 
 & = \frac{9m^{2}}{2} -mS, (S = 3(g - g_{\bar{X}} + q_{2} + q +
 \frac{1}{2}))\\
& \text{ Thus, } \sum\limits^{P(m)}_{i=0}   w_{\lambda}(M'_{i}) \ge
\frac{9m^{2}}{2} - mS, \tag*{$(E_{1})$}
\end{align*}

\noindent
Next we estimate, total $\lambda$-weight of $w_{1}, w_{2} , \ldots ,
w_{N+1} =  \sum\limits^{N+1}_{i=1}   w_{\lambda}(w_{i})$. Recall that
we have agreed to view $W_{3} = H^o (\mathbb{P}^{N},
O_{\mathbb{P}^{N}} (1) )$ as a subspace of $H^o (X, L)$. Now observe
that $\dim \dfrac{W_{3}}{W_{1}} \leq 1$  and $ \dim \dfrac{W_{1}}{W_{0}}
\leq 1$. To see  this note that the image of $W_{0}$ (respectively
$W_{1}$) under the homomorphism $\pi_{*} : H^o (X , L) \to H^o
(\bar{X} , L')$ is contained in the subspace $H^o (\bar{X}, L'
(-3p))$ (respectively\pageoriginale $H^o (\bar{X}, L' (-2P))$)  of
$H^o (\bar{X}, L')$. Now use the assumption that $\pi$ is ramified at
$P$ and use the following exact sequences   
\begin{align*}
0  & \longrightarrow L' (-P)    \longrightarrow L'    \longrightarrow
k(P) \longrightarrow 0\\ 
0  & \longrightarrow L' (-3P)  \longrightarrow L' (-2P)
\longrightarrow k(P) \longrightarrow 0 
\end{align*}

$(k(P)) = $ the residue field at the point $P \in \bar{X}$. 

\noindent
It follows that $\dim \dfrac{W_{3}}{W_{1}} \leq 1$, $\dim
\dfrac{W_{1}}{W_{0}} \leq 1$. The above considerations imply that
total $\lambda$-weight of $w_{1}, w_{2} , \ldots , w_{N+1} =
\sum\limits^{N+1}_{i=1}  w_{\lambda}(w_{i}) \leq 4, \qquad (E_{2})$.  

We are now ready to get the contradiction that the $m^{\rm th}_{0}$
Hilbert point of $X$, $H_{m_{o}} (X) \in \mathbb{P}
(\overset{P(m_{o})}\Lambda  H^o (\mathbb{P}^{N} ,
O_{\mathbb{P}^{N}}(m_{o})))$ is not $\lambda' -$ semistable.  


If $H_{m}(X)$ is $\lambda'$-semistable  $(m > m')$, then there exist
monomials $M'_{i_{1}}, M'_{i_{2}}, \ldots , M'_{i_{P(m)}}$, $(1 \le
i_{1} < i_{2} < \ldots < i_{P(m)} \le \alpha_{m})$, such that
$\{\varphi_{m} (M'_{i_{1}}), \varphi_{m}, (M'_{i_2}), \ldots,
\varphi_m (M'_{i_{P(m)}})\}$ is a
basis of $H^o (X, L^{m})$ relative to the filtration $(\bar{F})$ and  
$$
\frac{\sum^{P(m)}_{j=1} w_{\lambda}(M'_{i_{j}})}{m P(m)} \le
\frac{\sum^{N+1}_{i=1} w_{\lambda}(w'_{i})}{d-g+1}, \qquad
(\text{cf. criterion (**) page \pageref{page10}}). 
 $$

It is easily seen that $\sum\limits^{P(m)}_{i=1}  w_{\lambda}(M'_{i})
\le \sum\limits^{P(m)}_{j=1} w_{\lambda}(M'_{i_{j}})$. Thus, 

 $H_{m} (X)$ is $\lambda'$-semistable $\Rightarrow
\dfrac{\sum\limits^{P(m)}_{j=1} w_{\lambda}(M'_{i})}{m (dm-g+1)} \le
\dfrac{\sum\limits^{N+1}_{i=1} w_{\lambda}(w'_{i})}{d-g+1}$  
\begin{align*}
\Rightarrow  & \frac{\frac{9m^{2}}{2} - mS}{m (dm - g+1)} \le
\frac{4}{d-g+1},  ~\text{ (Follows from the estimates} (E_{1}),
(E_{2}))\\ 
\Rightarrow & \frac{\frac{9}{2} - \frac{S}{m}}{d} \le \frac{4}{d - g +
  1} \Rightarrow \frac{9(d - g + 1)}{2} - \frac{(d-g+1)S}{m} \le 4d\\ 
\Rightarrow & \frac{1}{2} (d-9g+9) \le \frac{(d-g+1)S}{m} \Rightarrow
m\le \frac{2 (d-g+1)S}{d-9g+9}\\ 
\Rightarrow & m \le 4S   \quad (\because d \ge 20 (g-1)  \therefore
\frac{d-g+1}{d-9g+9} \le 2)\\ 
\Rightarrow & m < m_{o} \quad (\because  m_{o} > 4S).  
\end{align*}\pageoriginale
It follows that $H_{m_{o}}(X)$ is not $\lambda'$-semistable and hence
$H_{m_{o}}(X)$ is not semistable (cf.\ theorem~\ref{chap0:subthm0.0.9}. 
page~\pageref{chap0:subthm0.0.9}). The result now follows by the contradiction.   

\begin{subprop}\label{chap1:subprop1.0.6}% subprop 1.0.6
 Let $X$ be a curve in the family $Z_{V}
 \xrightarrow{p_{V}}V$. $X_{\red}$  cannot have a tacnode singularity.  
\end{subprop}

\begin{proof}
Assume the contrary i.e. let $X$ be a curve in the family $Z_{V}
\xrightarrow{p_{V}}V$ such that $X_{\red}$ has a tacnode singularity
at a point $P$ say.  
\end{proof}

Let $X = \bigcup\limits^{p}_{i=1} X_{i}$ be the decomposition of $X$ 
in to irreducible components. Let $\pi : \bar{X} \to X$ be the
normalization of $X$ so that $\bar{X} = \bigcup\limits^{p}_{i=1}
\bar{X}_{i \red}$ (cf.\ definition~\ref{chap0:subdef0.2.5} 
page~\pageref{chap0:subdef0.2.5}) and let 
$L' = \pi^{*} L$. That $P$ is a tacnode means, 
\begin{enumerate}[i)]
\item $p > 1$ ;

\item there exist components $\bar{X}_{i}$ and $\bar{X}_{j} (i \neq
  j)$ of $X$ and points $Q_{1} \in \bar{X}_{i \red}, Q_{2}
  \in \bar{X}_{j \red}$ such that $\pi (Q_{1}) = P = \pi
  (Q_{2})$;  

\item $X_{i \red}$\pageoriginale and $ X_{j \red}$ have a common
  tangent at $P$.  
\end{enumerate}

Since $X$ is  not contained in any hyperplane in $\mathbb{P}^{N}$
(cf.\ Proposition \ref{chap1:subprop1.0.2}. 
page~\pageref{chap1:subprop1.0.2}), we can think of $W_{2} = H^o 
(\mathbb{P}^{N}, O_{\mathbb{P}^{N}} (1))$ as a subspace $H^o (X, L)$.
Define two subspaces of $W_{2}$ as follows.  
\begin{align*}
 W_{0} &  =  \{s \in W_{2} | \pi _{*} s   ~\text{ vanishes to 
   order } \geq 2  \text{ at } Q_{1} \text{ and } Q_{2}\}\\ 
 W_{1} &  =  \{s \in W_{2} | \pi _{*} s   ~\text{ vanishes to
   order }  \geq 1 \text{ at } Q_{1}  \text{ and } Q_{2}\} 
\end{align*}

Note that since $P$ is a double point of $X_{\red}$ and $X_{i \red}$
and  $X_{j \red}$ have a common tangent at $P$,  
\begin{gather*}
\pi_{*}s (s \in W_{2}) ~\text{ vanishes to order }~ \ge 2
\text{ at } Q_{1}\\
\Leftrightarrow\\
\pi_{*}s (s \in W_{2}) ~\text{ vanishes to order }~ \ge 2
\text{ at } Q_{2}.
\end{gather*}

\noindent
Let $\dim W_{0} = N_{0}$, $\dim W_{1} = N_{1}$. Choose a basis of
$W_{2}$ relative to the filtration $0 \subset W_{0} \subset W_{1}
\subset W_{2}$, say  

\noindent
$\{w_{1}, w_{2}, \ldots , w_{N_{0}}, w_{N_{0}+1} ,\ldots ,
    w_{N_{1}}, w_{N_{1}+1}  , \ldots , w_{N+1} \}$, 
(cf.\ definition~\ref{chap0:subdef0.2.6} 
page~\pageref{chap0:subdef0.2.6}).  

Let $\lambda$ be a $1-ps$ of $GL(N+1)$ such that the action of
$\lambda$ on $W_{2}$ is given by,  
\begin{align*}
\lambda (t) w_{i} & = w_{i} , t \in K^{*} ,  (1 \le i \le N_{0}), \\
\lambda (t) w_{i} & = tw_{i} , t \in K^{*} ,  (N_{0} +1 \le i \le N_{1}),\\
\lambda (t) w_{i} & = t^{2}w_{i} , t \in K^{*} ,  (N_{1}+1 \le i \le N+1). 
\end{align*}
There\pageoriginale exists a $1-ps \lambda'$ of $SL(N+1)$ associated
to the $1-ps 
\lambda$ of $GL(N+1)$, (cf.\ definition~\ref{chap0:subdef0.1.2} 
page~\pageref{chap0:subdef0.1.2}). We will 
show that the $m^{\rm th}_{0}$. Hilbert point of $X$, $H_{m_{o}} (X)$ is not
$\lambda'$-semistable. In particular it will follow that
$H_{m_{o}}(X)$ is not semistable, (cf.~theorem~\ref{chap0:subthm0.0.9}. 
page~\pageref{chap0:subthm0.0.9}), and 
the result will then follow by the contradiction.  

Assume now that $m > m'$ so that $H^{1}(X, L^{m}) = 0 = H^{1}
(\bar{X}, L'^{m})$ and the restriction $\varphi_{m} : H^o
(\mathbb{P}^{N}, O_{\mathbb{P}^{N}}(m)) \to H^o (X, L^{m})$ is
surjective. Recall that $H^o (\mathbb{P}^{N},
O_{\mathbb{P}^{N}}(m))$ has a basis $B_{m} = \{M_{1}, M_{2} , \ldots,
M_{\alpha_{m}}\}$ consisting of monomials of degree $m$ in $w_{1},
w_{2}, \ldots , w_{N+1}$. For $ 0 \leq r \leq m$ let
$W^{m-r}_{0}. W^{r}_{1}$ be the subspace of $H^o (\mathbb{P}^{N},
O_{\mathbb{P}^{N}}(m))$ generated by elements $w$ of the following
type.  

\noindent
For $r = 0$, 
$$
w = x_{1}x_{2} \dots x_{m}, (x_{s} \in W_{0} , 1 \le s \le m)
; 
$$
for $0 < r < m$,
$$
w = x_{1} x_{2} \dots x_{m-r} y_{1}y_{2} \dots y_{r},
(x_{s}\in W_{0}, 1 \le s \le m-r, y_{s} \in W_{1}, 1
\le s \le r); 
$$
for $r = m$
$$
w = y_{1}y_{2} \dots y_{m}, \quad (y_{s} \in W_{1} , 1 \le s \le
m). 
$$

Similarly, for $0 \leq r \leq m$, let $W^{m-r}_{1}. W^{r}_{2}$ be the
subspace of $H^o (\mathbb{P}^{N},\break O_{\mathbb{P}^{N}}(m))$ generated
by elements $w'$ of the following type. 

\noindent
For $r = 0$,
$$
w' = x'_{1}x'_{2} \cdots x'_{m} , (x'_{s} \in W_{1}, 1 \le s
\le m); 
$$
for $0 < r < m$,\pageoriginale 
\begin{align*}
w' & = x'_{1} x'_{2} \dots x'_{m-r} y'_{1} y'_{2} \dots y'_{r}, \\
& \qquad (x'_{s}
\in W_{1} , 1 \le s \le m-r , y'_{s} \in W_{2}, 1 \le
s \le r);  
\end{align*}
for $r = m$,
$$
w' = y'_{1}y_{2} \dots y'_{m}, \quad (y'_{s}~ \in ~W_{2} ,   1
\le s \le m). 
$$
We have the following filtration of $H^o (\mathbb{P}^{N},
O_{\mathbb{P}^{N}}(m))$;  
 \begin{align*}
&  0 \subset W^{m}_{0} . W^{o}_{1} \subset W^{m-1}_{0}. W^{1}_{1}
 \subset \cdots \subset W^{o}_{0}. W^{m}_{1} = W^{m}_{1}. W^{o}_{2}
 \subset W^{m-1}_{1}. W^{1}_{2} \subset  \cdots \\
& \qquad W^{o}_{1}. W^{m}_{2} =
 H^o (\mathbb{P}^{N}, O_{\mathbb{P}^{N}}(m)) \tag{F} 
\end{align*}
Let
{\fontsize{10}{12}\selectfont
\begin{gather*} 
\bar{W}^{m-r}_{0}. \bar{W}^{r}_1 = \varphi (W^{m-r}_{0}. W^{r}_{1})
\subset H^o (X , L^{m}),  \dim \bar{W}^{m-r}_{0}. \bar{W}^{r}_{1} =
\gamma_{r},  \\ 
\bar{W}^{m-r}_{1}. \bar{W}^{r}_2 = \varphi_m (W^{m-r}_{1}. W^{r}_{2})
\subset H^o (X , L^{m}),  \dim \bar{W}^{m-r}_{1}. \bar{W}^{r}_2 =
\beta _{r}, (0 \leq r \leq m). 
\end{gather*}}\relax
These subspace define the following filtration of $H^o (X, L^{m})$
\begin{align*}
& 0 \subset \bar{W}^{m}_{0} . \bar{W}^{o}_{1} \subset
\bar{W}^{m-1}_{0} . \bar{W}^{1}_{1} \subset \dots \subset
\bar{W}^{o}_{0}. \bar{W}^{m}_{1} = \bar{W}^{m}_{1}. \bar{W}^{o}_{2}
\subset \bar{W}^{m-1}_{1}. \bar{W}^{1}_{2} \subset  \dots\\
& \qquad \bar{W}^{o}_{1}. \bar{W}^{m}_{2}  = H^o (X, L^{m})
\tag{$\bar{F}$}  
\end{align*}
Rewrite the basis $B_{m}$ as

\noindent
$B_{m} = \{M'_{1}, M'_{2} , \ldots , M'_{P(m)} , M'_{P(m)+1} , \ldots,
M'_{\alpha_{m}}\} $ so that 

\noindent
$ \{\varphi_{m} (M'_{1}), \varphi_{m} (M'_{2}) , \ldots , \varphi_{m}
(M'_{P (m)})\}$ is a basis of $H^o (x, L^{m})$ relative to the
filtration $(F)$ (cf.\ definition~\ref{chap0:subdef0.2.6} 
page~\pageref{chap0:subdef0.2.6}) and   

\noindent
$M'_{P(m)+1}, M'_{P(m)+2},  \ldots , M'_{\alpha_{m}}$ are the rest of
the monomials in $B_{m}$ in some order.   

For\pageoriginale the rest of the proof we consider the following
cases.  
\begin{enumerate}
\item[{\rm Case 1.}] $\deg_{X_{i \red}} L \ge 2$, $\deg_{X_{j \red}} L \ge 2$.

\item[{\rm Case 2.}] $\deg_{X_{i \red}} L = 1$,  $\deg_{X_{j \red}} L \ge 2$.

\item[{\rm Case 3.}] $\deg_{X_{i \red}} L \ge 2$,  $\deg_{X_{j \red}} L = 1$.
\end{enumerate}

(Since the point $P \in X$ is a tacnode, these are the only
possibilities). We will give proofs in cases 1) and 2) and then
case 3) will follow from case 2), by interchanging the roles of
$i$ and $j$.  

\medskip
\noindent{\textbf{Case 1. }}
The morphism $\pi : \bar{X} \to X$ gives homomorphisms $\pi_{m*}:
H^o (X, L^{m}) \to H^o(\bar{X}, L'^{m})$. Let $D$ denote the
divisor $Q_{1} + Q_{2}$ on $\bar{X}$. By definition, the image of
$\bar{W}^{m-r}_{0} . \bar{W}^{r}_{1}$ (respectively $\bar{W}^{m-r}_{1}
. \bar{W}^{r}_{2}$)  $\subset H^o (X , L^{m})$ under the homomorphism
$\pi_{m*}$ is contained in the subspace  $H^o (\bar{X}, L'^{m}\break
((r-2m) D))$ (respectively $H^o (\bar{X}, L'^{m}((r-m)D))$ of
$H^o(\bar{X}, L'^{m})$, $(0 \le r \le m-1)$.   

It follows from the Riemann-Roch theorem that for $0 \leq r \leq m-1$
\begin{align*}
\gamma_{r} & = \dim ( \bar{W}^{m-r}_{0} . \bar{W}^{r}_{1}) \le h^{o}
(\bar{X}, L'^{m} ((r-2m)D)) + \dim ( \kernel  \pi_{m*}) \\ 
& = dm + 2r - 4m - g_{\bar{X}} + 1 + h^{1} ( \bar{X}, {L'}^{m}
((r-2m)D)+\dim (\kernel \pi_{m*})
\end{align*}
and
\begin{align*}
\beta_{r} & = \dim ( \bar{W}^{m-r}_{1}. \bar{W}^{r}_{2}) \le h^{o}
(\bar{X}, L'^{m} ((r-m)D) + \dim ( \kernel  \pi_{m*}) \\ 
& = dm + 2r - 2m - g_{\bar{X}} + 1 + h^{1} ( \bar{X}, L'^{m}
((r-m)D)+\dim (\kernel \pi_{m*}) 
\end{align*}

\medskip
\noindent{\textbf{Claim: }}
\begin{enumerate}[i)]
\item $\dim (\kernel  \pi_{m*}) < q_{2}$,\pageoriginale

\item $h^{1} (\bar{X}, {L'}^{m}((r-2m)D) \le 4m-2r$, $(0 \le r \le q =
  2g-2)$,  

\item $h^{1} (\bar{X}, {L'}^{m}((r-2m)D) = 0$, $(q + 1 \le r \le m-1)$,

\item $h^{1} (\bar{X}, {L'}^{m}((r-m)D) = 0$, $(0 \le r \le m-1)$.
\end{enumerate}

\medskip
\noindent{\textbf{Proof of the Claim: }}
\begin{enumerate}[i)]
\item Since the curve $X$ is generically reduced, (cf.\ 
proposition~\ref{chap1:subprop1.0.3}. page~\pageref{chap1:subprop1.0.3}), for
all integers $m$, $\dim (\kernel \pi_{m*}) =  h^{o} (X , I_{X}) < q_{2}$, 
(cf. page~\pageref{c1:claim}).  

\item In view of the fact that $H^{1} (\bar{X}, L'^{m}) = 0 $, this is
  immediate from the long exact cohomology sequence associated to the
  following exact sequence,  
$$
0 \to L'^{m}((r-2m)D) \to L'^{m} \to O_{(2m-r)D} \to 0.
$$ 

\item Recall that $\deg_{X_{ i \red}} L \ge 2 \leq \deg_{X_{j \red}}L$
  and use the following general fact. If $C$ is an integral smooth
  curve of genus $g_{C}$ and if $M$ is a line bundle of $C$ with $\deg
  M \geq  2g_{C} -1$ then $H^{1} (C, M) = 0$.   

\item This follows from the same reasoning as above. 
\end{enumerate}

It follows from these considerations that, 
\begin{align*}
\gamma_{r} & \le  dm + 2r - 4m - g_{\bar{X}} + 1 + 4m - 2r + q_{2},
\quad (0 \le r \le q = 2g - 2),\\
\gamma_{r} & \le dm + 2r - 4m - g_{\bar{X}} + 1 + q_{2}, \quad (q+1
\le r \le m-1), \\ 
\beta_{r} &  \le dm + 2r - 2m - g_{\bar{X}} + 1 + q_{2}, \quad (0 \le
r \le m-1). 
\end{align*}

\noindent 
We\pageoriginale now estimate total $\lambda$-weight of $M'_{1},
M'_{2}, \ldots , 
M'_{P(m)}  = \sum\limits^{P(m)}_{i=1} w_{\lambda}(M'_{i})$. Note that a
monomial $M \in W^{m-r}_{0}. W^{r}_{1} -
W^{m-r+1}_{0}. W^{r-1}_{1}$ has $\lambda$-weight $w_{\lambda}(M) = r$
and a monomial   
\begin{align*}
& M' \in W^{m-r}_{1} W^{r}_{2}- W^{m-r+1}_{1} W^{r-1}_{2}
  \text{ has } \lambda-\text{weight } ~ w_{\lambda} (M') = m+r. \\ 
&\sum\limits^{P(m)}_{i=1} w_{\lambda}(M'_{i}) = \sum^{m}_{r=1} (m+r)
  (\beta_{r} - \beta_{r-1}) + \sum^{m}_{r=1}  r(\gamma_{r} -
  \gamma_{r-1}) \\ 
&= 2m\beta_{m} - \sum^{m-1}_{r=0} \beta_{r} - \sum^{m-1}_{r=0} \gamma_{r}  \\
&\ge 2m(dm-g+1) - \sum^{m-1}_{r=0} (dm+2r-2m-g_{\bar{X}} + 1 +
  q_{2})\\ 
&- \sum^{m-1}_{r=0} (dm+2r-4m-g_{\bar{X}} + 1 + q_{2}) -
  \sum^{q}_{r=0} (4m-2r)\\ 
&= 2m (dm -g +1) - \sum\limits^{m-1}_{r=0} (2dm+4r-6m-2g_{\bar{X}} + 2
  +2 q_{2})\\ 
&\qquad - 4m (q+1) + q (q+1)\\ 
&> 4m^{2} - mS, (S = 2(g - g_{\bar{X}} + q_{2} + 2q + 1)). \\
& \text{Thus, } \sum\limits^{P(m)}_{i=1} w_{\lambda}(M'_{i})  > 4m^{2} -
mS,\tag{$E_1$} 
\end{align*}

\noindent
We now estimate total $\lambda$-weight of $w_{1}, w_{2}, \ldots,
w_{N+1} =  
\sum\limits^{N+1}_{i=1} w_{\lambda}(M'_{i})$. The morphism $\pi :
\bar{X} \to X$ gives a homomorphism $\pi * : H^o (X, L) \to H^o
(\bar{X} , L')$. Recall that we have agreed to view $W_{2} = H^o
(\mathbb{P}^{N} , O_{\mathbb{P}^{N}} (1))$ as a subspace of $H^o (X
, L)$. Clearly $W_{2} \bigcap \kernel \pi_{*} = 0$. By definition the
image of $W_{0}$ (respectively $W_{1}$) under the homomorphism
$\pi_{*}$ is contained in the subspace $H^o (\bar{X}, L'(-2Q
_{1}))$ (respectively $H^o (\bar{X}, L' (-Q_{1}))$) of $H^o
(\bar{X}, L')$.  

\noindent
Now\pageoriginale it is immediate from the following exact sequences
that  

\noindent
 $\dim \dfrac{W_{1}}{W_{0}} \le 1$, $\dim \dfrac{W_{2}}{W_{1}} \le 1$.
\begin{align*}
& 0 \to L'(-2Q_{1}) \to L' (-Q_{1}) \to k(Q_{1}) \to 0\\
& 0 \to L'(-Q_{1}) \to L'   \to k(Q_{1}) \to 0
\end{align*}
($k(Q_{1}) = $ the residue field the point $Q_{1} \in \bar{X}$). 

\noindent
It follows that $\sum\limits^{N+1}_{i=1} w_{\lambda}(w_{i}) \le
3, \hspace{3cm} (E_{2})$ 

We are now ready to get the contradiction that the $m^{\rm th}_{0}$
Hilbert point of $X$, $H_{m_{o}} (X)$ is not $\lambda'$-semistable. 

If $H_{m}(X)$ is semistable $(m > m')$ then there exist monomials
$M'_{i_{1}},\break M'_{i_{2}} , \ldots, M'_{i_{P(m)}}$ in $B_{m}$ such that  

\noindent
 $\{\varphi_{m} (M'_{i_{1}}), \varphi_{m} (M'_{i_{2}}), \ldots,
\varphi_{m}(M'_{i_{P(m)}})\}$ is a basis of $H^o (X , L^{m})$  
 
 \noindent
 and $\dfrac{\sum^{P(m)}_{j=1} w_{\lambda}(M'_{i_{j}})}{m P(m)} \le
 \dfrac{\sum^{N+1}_{i=1} w_{\lambda}(w_{i})}{d-g+1}$ (cf.\ 
criterion~$(**)$ page~\pageref{page10}).  

\noindent
Observe that $\sum\limits^{P(m)}_{i=1} w_{\lambda}(M'_{i}) \le
\sum\limits^{P(m)}_{j=1} w_{\lambda}(M'_{i_{j}})$. 

\noindent
Hence,
 \begin{align*}
& H_{m}(X) \text{ is } \lambda' - \text{ semistable } (m > m')
   \Rightarrow  \frac{\sum^{P(m)}_{j=1} w_{\lambda}(M'_{i})}{m
     (dm-g+1)} \le  \frac{\sum^{N+1}_{i=1}
     w_{\lambda}(w_{i})}{d-g+1}\\ 
& \Rightarrow \frac{4m^{2}-mS}{m(dm-g+1)} \le \frac{3}{d-g+1}
 \quad  \text{ (Follows from } (E_{1}), (E_{2}))\\ 
& \Rightarrow \frac{4-\frac{S}{m}}{d} \le \frac{3}{d-g+1} \Rightarrow
   d-4g+4 \le \frac{(d-g+1)S}{m}\\ 
& \Longrightarrow m \leq  \frac{(d-g+1)S}{(d-4g+4)}\leq 2S \quad (\because d
   \geq 20 (g-1) \therefore \frac{d-g+1}{d-4g+4} \le 2)\\ 
& \Longrightarrow m < m_o \quad (\because m_o > 2S)
\end{align*}\pageoriginale

Thus $H_{m_o}(X)$ is not $\lambda'-$semistable and hence $H_{m_o}(X)$
is not semista\-ble (cf.\ theorem~\ref{chap0:subthm0.0.9}. 
page~\pageref{chap0:subthm0.0.9}). This 
contradiction proves the result in case~1). 


\medskip
\noindent{\textbf{Case 2.}}
The proof in this case is on the same lines. Recall that we have
homomorphisms $\pi_m : H^o (X, L^m) \to H^o (\bar{X}, L'^m), (m > m')$,
$\pi_* : H^o (X, L) \to H^o (\bar{X}, L')$ and that we have agreed to
view $H^o (\mathbb{P}^{N},\break O_{\mathbb{P^{N}}}(1)) = W_2$ as a subspace
of $H^o(X, L)$. Let $Y$ be the closure of $X-X_i$ in $X$. Clearly a
section $\pi_* s (s \in W_o)$ vanishes on
$\bar{X}_{i\red}$. Hence the image of $\bar{W}^{m-r}_o. \bar{W}^{r}_1$,
$(0 \le r \le m-1)$, under the homomorphism $\pi_{m*}$ is contained in
the subspace

\noindent
$H^o(\bar{Y}_{\red},L'^{m}_{\bar{Y}_{\red}} (r -2m) Q_2)) \subset H^o
(\bar{X}, {L'}^{m})$, ($\bar{Y}_{\red}$ is the normalization of
$Y_{\red}$). 

\noindent
It follows that, for $0 \le r \le m-1$
\begin{align*}
\gamma_r & = \dim (\bar{W}^{m-r}_o \cdot \bar{W}^{r}_{1}) \le h^o
(\bar{Y}_{\red}, L'_{\bar{Y}_{\red}}) ((r-2m) Q_2)) +  (\kernel
\pi_{m*})\\ 
& = (d-1)m + r-2m - g_{\bar{Y}} + 1 + h^1 (\bar{Y}_{\red},
L'^{m}_{\bar{Y}_{\red}})((r-2m)Q_2))\\ 
&\qquad+ \dim (\kernel \pi_{m*}) 
\end{align*}

\noindent
Recall that, 
\begin{align*}
\beta_r&= \dim (\bar{W}^{m-r}_1 \cdot \bar{W}^{r}_{2})
\le h^o (\bar{X}, L'^{m} ((r-m)D)) + \dim (\kernel  \pi_{m*}).\\ 
&= dm +2r -2m - g_{\bar{X}} + 1 + h^{1} (\bar{X}, L'^{m} ((r-m)d)) +
\dim (\kernel \pi_{m*}). 
\end{align*}

\noindent
As before we have,
\begin{enumerate}[i)]
\item $\dim (\kernel \pi_{m*}) < q_2$ 

\item $h^1 (\bar{Y}, L'^{m}_{\bar{Y}_{\red}} ((r-2m)Q_2)) =
  0$,\pageoriginale \quad    $(0 \leq r \leq q = 2q-2)$. 

\item $h^1 (\bar{Y}, L'^{m}_{\bar{Y}_{\red}} ((r-2m)Q_2)) = 0$, \quad 
$(q +  1 \le r \le m-1)$. 

\item $h^1 (\bar{X}, L'^{m} ((r-m)D)) \le 2m - 2r$, \quad $(0 \le r \le q =
  2g - 2)$. 

\item $h^1 (\bar{X}, L'^{m} ((r-m)D)) = 0$,  \quad $(q + 1 \le r \le m-1)$. 
\end{enumerate}

\noindent
Thus,
\begin{align*}
\gamma_r & \le (d-1) m + r - 2m - g_{\bar{Y}} + 1+ 2m-r +q_2, \quad (0 \le
r \le q),\\ 
\gamma_r & \le (d-1) m + r - 2m - g_{\bar{Y}} + 1+ q_2, \quad (q+ 1 \le r
\le m-1),\\ 
\beta_r & \le dm + 2r - 2m - g_{\bar{X}} + 1 + 2m - 2r +q_2, \quad (0 \le r
\le q),\\  
\beta_r & \le dm + 2r - 2m - g_{\bar{X}} + 1 + q_2,  \quad (q + 1 \le r \le
m-1) 
\end{align*}

\noindent
We want to estimate $\sum\limits^{P(m)}_{i = 1} w_{\lambda
}(M'_{i})$. Recall that, a monomial 

$ M \in W^{m-r}_{0} \cdot W^r_1 - W^{m-r+1}_{0}. W^{r-1}_{1}$
has $\lambda-$weight $r$, a monomial 
 
 $M \in W^{m-r}_1 \cdot W^{r}_2 - W^{m-r+1}_{1} . W^{r-1}_{2}$
has $\lambda-$weight $m + r$  
{\fontsize{10}{12}\selectfont
\begin{align*}
\sum_{i = 1}^{P(m)} w_{\lambda} (M^{'}_{i}) & = \sum^{m}_{r=1} (m + r)
(\beta_r - \beta_{r-1}) + \sum^{m}_{r=1} r(\gamma_r - \gamma_{r-1})\\ 
& = 2m\beta_m - \sum^{m-1}_{r=0} \beta_r - \sum^{m-1}_{r=0}
\gamma_r.\\ 
& \ge 2m (dm - g+1) - \sum^{m-1}_{r=0} (dm + 2r-2m-g_{\bar{X}} + 1 +q_2
)\\ 
&\quad - \sum^{q}_{r=0} (2m - 2r) 
 - \sum^{m-1}_{r=0} ((d-1)m+r-2m - g_{\bar{Y}} + 1 +q_2)\\ 
&\quad - \sum^{q}_{r=0} (2m -r) 
 > \frac{7m^2}{2}-m (2g - g_{\bar{X}}-g_{\bar{Y}} + 2q_2 + 4q -
\frac{3}{2}). 
\end{align*}}\relax\pageoriginale
Put $(2g - g_{\bar{X}} - g_{\bar{Y}} + 2q_2 + 4q -
\dfrac{3}{2})=S$. Thus we have the following estimate. 
\begin{equation*}
\sum^{P(m)}_{i =1} w_{\lambda}(M^{'}_{i}) > \frac{7m^{2}}{2} - mS, \tag*{$(E'_1)$}
\end{equation*}
We have already seen that total $\lambda-$weight of 
\begin{equation*}
w_1, w_2, \ldots, w_{N+1} = \sum^{N+1}_{i=1} w_{\lambda}(w_i) \le 3, \tag*{$(E_2)$}
\end{equation*}

\noindent
As in case~1) we have, $H_m(X)$ is $\lambda'-$semistable
$\Longrightarrow \dfrac{\dfrac{7}{2}m^2 - mS}{m(dm-g+1)} \leq
\dfrac{3}{d-g+1}$ (Follows as in the previous case from 
criterion~$(**)$ (page~\pageref{page10}) and the estimates 
$(E'_{1})$ and $(E_2)$) 
\begin{align*}
\Longrightarrow & \frac{\frac{7}{2}- \frac{S}{m}}{d} \le
\frac{3}{d-g+1} \Longrightarrow \frac{1}{2}(d-7g +7) \le
\frac{(d-g+1)S}{m}\\ 
\Longrightarrow & m \le \frac{2(d-g+1)S}{d - 7g +7} \le 4S (\because d
\ge 20 (g-1) \therefore \frac{d-g+1}{d-7g-7} \le 2). 
\end{align*} 
 
Since $m_o > 4S$, the above inequality implies that $H_{m_o} (X)$ is
not semistable. This contradiction concludes the proof as in case
1. 

The next step in the proof of Theorem~\ref{chap1:subthm1.0.1}. 
is to prove an important inequality which will be needed for 
the proof of the theorem. 

\begin{subprop}\label{chap1:subprop1.0.7}%prop 1.0.7.
Let $X$ be a curve in the family $Z_V \xrightarrow{p_V} V$ such that
$X$ has at least two irreducible components. Let $C$ be a reduced,
connected, complete subcurve of $X(C \neq X)$ and let $Y$ be the
closure of $X-C$ in $X$, with the reduced structure. Recall that $\pi
: \bar{X} \to X$\pageoriginale denotes the normalization of $X$. It follows from
definition~\ref{chap0:subdef0.2.5} (page~\pageref{chap0:subdef0.2.5})
that $\bar{X}$ is a  disjoint union of 
$\bar{C}$ (normalization of $C$) and $\bar{Y}$ (normalization of
$Y$). Let $\varphi'_{1} : H^o (\mathbb{P^{N}}, O_{\mathbb{P^{N}}}(1))
\to H^o (C, I_C)$ be the natural restriction map and let $W_o =
\kernel \varphi'_{1}$. 
\end{subprop}

\noindent
If there exist points $P_1, P_2,\ldots, P_k$ on $\bar{Y}$, satisfying
\begin{enumerate}[i)]
\item $\pi (P_i) \in Y \cap C$, $(1 \leq i \leq k)$,

\item for every irreducible component $\bar{Y}_j$ of $\bar{Y}$\label{c1:cond2}
\end{enumerate}

$\deg_{\bar{Y}_j} (L'_{\bar{Y}}(-D)) \geq 0$,

($D$ denotes the divisor $\sum\limits^{k}_{i = 1} P_i$ on $\bar{Y}$, $L'
= \pi^{*} L$), then the following inequality holds, 
\begin{equation*}
\frac{h^o (C, L_C)}{d-g+1} \ge \frac{e+ \frac{k}{2}}{d}, \qquad (e=
\deg_C L) \tag{$*'$} 
\end{equation*}

\begin{proof}
Let $\dim W_o = N_o$. Choose a basis of $W_1 = H^o (\mathbb{P^N},
O_{\mathbb{P^N}}(1))$ relative to the filtration $0 \subset W_o
\subset W_1$, say 

\noindent
 $w_0,w_1, \ldots ,w_{N_o-1}, w_{N_o}, \ldots, w_N$. Let $\lambda$ be a
$1-ps$ of $GL(N+1)$ such that the action of $\lambda$ on $W_1$ is
given by,
\begin{align*}
\lambda (t) w_i & = w_i, t \in K^*, (0 \le i \le N_o -1),\\
\lambda (t) w_i & = tw_i, t \in K^*, (N_o \le i \le N).
\end{align*}

\medskip
Let\pageoriginale $\lambda'$ be the $1-ps$ of $SL(N+1)$ associated to
the $1-ps 
\lambda$ of $GL(N+1)$, (cf.\ definition~\ref{chap0:subdef0.1.2} 
page~\pageref{chap0:subdef0.1.2}). Since the 
$m^{\rm th}_o$ Hilbert point of $X$, $H_{m_o}(X) \in
\mathbb{P} (\overset{P(m_o)}{\wedge} H^o (\mathbb{P^N},
O_{\mathbb{P^N}}(m_o)))$ is $\lambda'-$semistable, (cf.\ 
theorem~\ref{chap0:subthm0.0.9}. page~\pageref{chap0:subthm0.0.9}), 
criterion~$(**)$ (page~\pageref{page10}) is 
satisfied. The required inequality will follow from the inequality in
the above mentioned criterion. 
\end{proof}


Let $B_{m_o} = \{M_1, M_2, \ldots M_{\alpha_{m_o}}\}$ be a basis of
$H^o (\mathbb{P}^N, O_{\mathbb{P}^N}(m_o))$ consisting of monomials
of degree $m_o$ in $w_0, w_1, \ldots, w_N$, 
$(\alpha_{m_o} = h^o (\mathbb{P}^N,\break O_{\mathbb{P^N}}(m_o)))$. For $0
\leq r \leq m$ let $w^{m_o - r}_o \cdot w^{r}_{1}$ be the subspace of
$H^o (\mathbb{P}^N,\break O_{\mathbb{P^N}}(m_o))$ generated by elements $w$
of the following type. 
{\fontsize{10}{12}\selectfont
$$
w= x_1 x_2 \cdots x_{m_o -r} y_1 y_2 \cdots y_r, (x_j \in W_o,
1 \le j \le m_o -r ; y_j \in W_1, 1 \le j \le r) 
$$}\relax

\noindent
As before, for $r=0$, $w$ is a product of ${x'}^{s}_{j}$ only and for
$r=m_o$, $w$ is a product of $y'^{s}_j$ only. We have the following
filtration $(F)$ of $H^o (\mathbb{P}^N, O_{\mathbb{P^N}}(m_o))$, 
\begin{equation*}
0 \subset W^{m_o}_o \cdot W^o_1 \subset W^{m_o-1}_o \cdot W^{1}_{1}
\subset \cdots \subset W^{o}_{o} \cdot W^{m_o}_1 = H^o (\mathbb{P}^N,
O_{\mathbb{P^N}}(m_o)), \tag{F} 
\end{equation*}

\noindent
Note that if $W_o = 0$, then $W^{m-r}_0 \cdot W^r_1=0 (0 \leq r \leq
m-1)$. Let 

 $\bar{W}^{m_o}_0 \cdot \bar{W}^{r}_{1} = \varphi_{m_o} (W^{m_o-r}_0
\cdot W^{r}_{1}) \subset H^o (X, L^{m_o}), \dim \bar{W}^{m_o-r}_0
\cdot \bar{W}^r_1 =\beta_r, (0 \le r \le m_o)$ 

\noindent
These subspaces define the following filtration $(\bar{F})$ of $H^o
(X, L^{m_o})$, 
\begin{equation*}
0 \subset \bar{W}^{m_o}_0 \cdot \bar{W}^{o}_1 \subset \bar{W}^{m_o
  -1}_0 \cdot \bar{W}^{1}_{1} \subset \cdots \subset \bar{W}^o_0 \cdot
\bar{W}^{m_o}_1 = H^o (X,L^{m_o}), \tag{$\bar{F}$} 
\end{equation*}\pageoriginale

\noindent
Rewrite the basis $B_{m_o}$ as $B_{m_o}= \{ M'_1, M'_2,\ldots ,
M'_{P(m_o)}, M'_{P(m_o)+1} , \ldots,\break M'_{\alpha_{m_o}}\} $ such that
$\{\varphi_{m_o} (M'_1), \varphi_{m_o} (M'_{2}), \ldots,
\varphi_{m_o}(M'_{P(m_o)})\}$ is a basis of\break $H^o (X, L^{m_o})$
relative to the filtration $(\bar{F})$ and $M'_{P(m_o)+1}$,
$M'_{P(m_o)+2}, \ldots,\break M'_{\alpha_{m_o}}$ are the rest of the
monomials in $B_{m_o}$ in some order.  

The morphism $\pi : \bar{X} \to X$ gives a homomorphism $\pi_{m_o *} :
H^o (X,\break L^{m_o}) \to H^o (\bar{X}, L^{m_o})$. Since $\bar{X}$ is a
disjoint union of $\bar{Y}$ (normalization of $Y$) and $\bar{C}$
(normalization of $C$), $H^o (\bar{X}, L^{m_o}) = H^{o}( \bar{Y},
L'^{m_o}_{\bar{Y}}) \oplus H^o (\bar{C}, L'^{m_o}_{\bar{C}})$. By definition the
section in $\pi_{m_o*} (\bar{W}^{m-r}_0 \cdot W^r_1)(0 \le r \le
m_o-1)$ vanish on $\bar{C}$ and also vanish to order $\ge m_o -r$ at
the points $P_1, P_2, \ldots, P_k$, therefore 
$$
\pi_{m_o*} (\bar{W}^{m-r}_0 \cdot \bar{W}^{r}_{1}) \subset H^o
(\bar{Y}, L'^{m_o}_{\bar{Y}} ((r-m_o)D)) \subset H^o (\bar{X},
L'^{m_o}) 
$$
($D$ denotes the divisor $\sum\limits^{k}_{i =1} P_i$ on
$\bar{Y})$. It follows that  
\begin{align*}
\beta_r & = \dim \bar{W}^{m_o-r}_0 \cdot \bar{W}^{r}_{1} \le h^o
(\bar{Y}, L'^{m_o}_{\bar{Y}} ((r-m_o)D)) + \dim (\kernel
\pi_{m_o *})\\ 
& = (d - e)m_o + k (r-m_o)-g_{\bar{Y}} + 1 +h^1 (\bar{Y},
L'^{m_o}_{\bar{Y}}((r-m_o)D))\\
&\qquad + \dim (\kernel \pi_{m_o *}) 
\end{align*}



\noindent{\textbf{Claim: }}
\begin{enumerate}[i)]
\item $\dim (\kernel ~ \pi_{m_o *}) < q_2$, 

\item $h^1 (\bar{Y}, L'^{m_o}_{\bar{Y}} ((r-m_o)D)) \le k(m_o -r)$,
  \quad $(0  \le r \le q = 2g-2)$, 

\item $h^1 (\bar{Y}, L'^{m_o}_{\bar{Y}} ((r-m_o)D))=0$, \quad $(q + 1 \le r\le
  m_o -1)$ 
\end{enumerate}

\begin{enumerate}[i)]
\item We\pageoriginale have seen this (cf.\ page~\pageref{c1:claim}).

\item The exact sequence
$$
0 \to L^{'m_o}_{\bar{Y}} ((r-m_o)D) \to L^{'m_o}_{\bar{Y}} \to O_{(m_o
  - r)D} \to 0 
$$
gives the long exact cohomology sequence
\begin{align*}
\cdots \to H^o(\bar{Y}, O_{(m_o-r)D})& \to H^1 (\bar{Y},
L'^{m_o}_{\bar{Y}}((r-m_o)D))\\ 
&\to H^{1}(\bar{Y}, L^{m_o}_{\bar{Y}})
\to \cdots 
\end{align*}
since $m_o > m', H^{1}(\bar{Y}, L'^{m_o}_{\bar{Y}})=0$. Hence
$$
h^{1}(\bar{Y}, L'^{m_o}_{\bar{Y}}((r-m_o)D)) \le h^o (\bar{Y}, O_{(m_o
  - r)D})=k(m_o-r). 
$$

\item Recall the condition ii) (page~\pageref{c1:cond2}) and use the following
  general fact. If $C'$ is an integral, smooth curve of genus $g_{C'}$
  and if $M$ is a line bundle on $C'$ with deg $M \geq 2g_{C'}-1$ then
  $H^1 (C', M)=0$.
\end{enumerate}

Hence, 
\begin{align*}
\beta_r \le & (d - e)m_o + k(r - m_o) - g_{\bar{Y}} + 1 + k(m_o -r) +
q_2, \qquad (0 \le r \le q)\\ 
\beta_r \le & (d - e)m_o + k(r - m_o) - g_{\bar{Y}} + 1 + q_2, \qquad
(q + 1 \le r \le m_o - 1). 
\end{align*}
We make of following estimate.

\noindent
total $\lambda-$ weight of 
\begin{align*}
M'_{1}, M'_{2},\ldots, M'_{(P(m_o))} &=
\sum\limits^{P(m_o)}_{i=1} w_{\lambda}(M'_{i})\\ 
& = \sum^{m_o}_{r = 1} r (\beta_r - \beta_{r - 1}),(\because 
\text{a monomial}\\ 
&\qquad M \in (W^{m-r}_0 W^r_1 - W^{m-r+1}_0
  W^{r-1}_{1})\\ 
&\qquad\text{ has } \lambda-\text{ weight } r)\\ 
& = m_o \beta_{m_o} - \sum^{m_o -1}_{r = 0} \beta_r\\
& \ge m_o (dm_o - g + 1) - \sum^{m_o -1}_{r =0} (m_o (d -e) + k (r -
  m_o)\\ 
&\qquad - g_{\bar{Y}} + 1 + q_2) - \sum\limits^{q}_{r = 0} k(m_o - r)\\ 
& =  \left(e + \frac{k}{2}\right) m^{2}_{o} - m_o \left(g - g_{\bar{Y}} + q_2 +\frac{k}{2}
  + kq\right)\\ 
&\qquad + \frac{q(q+ 1)}{2} 
\end{align*}\pageoriginale

\noindent
Put $S= (g - g_{\bar{Y}} + q_2 + \frac{k}{2} + kq)$. Thus we get, 
\begin{equation*}
\sum\limits^{P(m)}_{i = 1} w_{\lambda} (M'_{i}) > (e + \frac{k}{2})
m^2_o - m_o S, \tag{$E_1$} 
\end{equation*}

\noindent
Note that the above inequality is true even if $W_o = 0$. Clearly,
total $\lambda-$weight of $w_0, w_1, \ldots, w_N = \sum\limits^{N}_{i
  = 0} w_{\lambda} (w_i)$ 

\noindent
$= \dim W_1 - \dim W_o < h^o (C, L_C)$, (Follows from the definition
of $\lambda$) 

\noindent
\begin{equation*}
\text{ Thus we get } \sum\limits^{N}_{i = 0} w_\lambda (w_i) \le h^o
(C, L_C), \tag{$E_2$} 
\end{equation*}

If $W_o \neq 0$ then $\lambda' : G_m \to SL(N + 1)$ is a nontrivial
homomorphism i.e. a $1-ps$ of $SL(N+1)$. 

Since the $m^{\rm th}_{o}$ Hilbert point of $X$.

\noindent
$H_{m_o} (X) \in P(\overset{P(m_o)}\wedge H^o \mathbb{P}^{N},
O_{\mathbb{P}^N} (m_o))$ is $\lambda'- $semistable, there exist
monomials $M'_{i_1}, M'_{i_2}, \ldots, M'_{i_{P(m_o)}} (1 \le
i_1 < i_2 < \ldots < i_{P(m_o)} \le \alpha_{m_o})$ in $B_{m_o}$ such
that $\{\varphi_{m_o} (M'_{i_1}), \varphi_{m_o} (M'_{i_2}),
\ldots, \varphi_{m_o}(M'_{i_{P(m_o)}})\}$ is a basis of $H^o (X,
L^{m_o})$ and $\dfrac{\sum\limits^{P(m_o)} _{j=1} w_\lambda
  (M'_{i_j})}{m_o P(m_o)} \le \dfrac{\sum\limits^{N}_{i = 0}
  w_{\lambda} (w_i)}{d-g+1}$, (cf.\ criterion $(**)$, page~\pageref{page10}). 

\noindent
It is easy to see that $\sum\limits^{P(m_o)}_{i = 1} w_{\lambda}
(M'_{i}) \le \sum\limits^{P(m_o)}_{j = 1} w_{\lambda}
(M'_{i_j})$. 

\noindent
It follows\pageoriginale that, $\dfrac{\sum\limits^{P(m_o)} _{i=1} w_\lambda
  (M'_{i_j})}{m_o P(m_o)} \le \dfrac{\sum\limits^{N}_{i = 0}
  w_{\lambda} (w_i)}{d-g+1}$, 
\begin{align*}
& \Longrightarrow \frac{(e + \frac{k}{2})m^2_o - m_o S}{m_o (dm_o - g
    + 1)} < \frac{h^o (C, L_C)}{d- g+ 1}, \text{ (Follows from $(E_1)$
   and $(E_2)$)},\\ 
& \Longrightarrow \frac{e + \frac{k}{2} - \frac{S}{m_o}}{d} <
  \frac{h^o (C, L_C)}{d- g+ 1}. 
\end{align*}

Even if $W_o = 0$, we have, using $(E_1)$ and $(E_2)$
\begin{align*}
& \dfrac{(e + \frac{k}{2})m^2_o - m_o S}{m_o (dm_o - g + 1)} <
\dfrac{\sum\limits^{P(m_o)} _{j=1} w_\lambda (M'_i)}{m_o
  (P(m_o))} = 1 < \dfrac{h^o (C, L_C)}{d- g+ 1}\\
\Longrightarrow & \dfrac{e + \frac{k}{2} - \dfrac{S}{m_o}} {d} <
\dfrac{h^o (C, L_C)}{d- g+ 1}. 
\end{align*}

\noindent
We claim that the above inequality implies that $\dfrac{e +
  \dfrac{k}{2}}{d} \le \dfrac{h^o (C, L_C)}{d- g+ 1}$. In fact if the
claim were not true i.e. if $\dfrac{e + \dfrac{k}{2}}{d} \le \dfrac{h^o
  (C, L_C)}{d- g+ 1}$ then we get a contradiction as follows. 

\noindent
First note that
$$
\frac{e + \frac{k}{2}}{d} > \frac{h^o (C, L_C)}{d- g+ 1}
\Longrightarrow (d - g + 1) (e + \frac{k}{2}) - d(h^o (C, L_C)) \ge
\frac{1}{2}. 
$$

\noindent
Now,
\begin{align*}
& \dfrac{e + \dfrac{k}{2} - \dfrac{S}{m_o}}{d} \le \dfrac{h^o (C,
  L_C)}{d- g+ 1}, \quad \text{ (proved)} \\
&\Longrightarrow (d - g  +1) (e + \frac{k}{2}) -d(h^o (C, L_C)) \le
  \frac{S(d-g+1)}{m_o}\\ 
&\Longrightarrow \frac{1}{2} \le \frac{S(d-g+1)}{m_o} \Longrightarrow
  m_o \le 2S(d-g+1). 
\end{align*}

\noindent
By\pageoriginale our choice of the integer $m_o$, this is a
contradiction. This proves the required inequality, $\dfrac{h^o (C,
  L_C)}{d- g+ 1} \ge \dfrac{e + \dfrac{k}{2}}{d}$.  

\begin{subprop}\label{chap1:subprop1.0.8}%pro 1.0.8.
Every curve $X$ in the family $Z_V \xrightarrow{p_V}V$ is reduced. 
\end{subprop}

\begin{proof}
Let $X$ be a curve in the family $Z_V \xrightarrow{p_V}V$  and let
$I_X$ be the ideal sheaf of nilpotents in $O_X$. We want to show that
$I_X=0$. For the moment consider the closed subscheme $X_{\red}$ of $X$
defined by $I_X$. 
\end{proof}

\medskip
\noindent{\textbf{Claim: }}
$H^1 (X_{\red}, L_{\red})=0$, $(L_{\red} = L_{X_{\red}})$. First note
  that since the only singularities of $X_{\red}$ are ordinary double
  points (cf.\ proposition~\ref{chap1:subprop1.0.4}.,
  \ref{chap1:subprop1.0.5}., \ref{chap1:subprop1.0.6}), $X_{\red}$ has
  a   dualizing sheaf, say $\omega$. If the claim were not true then we
  have $H^o(X_{\red}, \omega \otimes L^{-1}_{\red}) \simeq H^1
  (X_{\red}, L_{\red}) \neq 0$. So there exists a nonzero section $s
  \in H^o (X, \omega \otimes L^{-1}_{\red})$ and a complete connected
  subcurve $C$ of $X_{\red}$ such that $s$ is not identically zero on
  any of the components of $C$ but $s$ vanishes at all points in $C
  \cap \overline{X-C}$. Observe that $C \neq \mathbb{P}^1$, hence
  $\deg_CL= e > 1$. 

It follows from the proof of theorem~\ref{chap0:subthm0.2.3}. 
(page~\pageref{chap0:subthm0.2.3}) that $h^o  
(C, L_C) - 1 \le \dfrac{e}{2}$. Using the inequality $(*')$
(cf.\ proposition~\ref{chap1:subprop1.0.7} 
page~\pageref{chap1:subprop1.0.7}), we get,  
$$
\frac{e + \frac{1}{2}}{d} \le \frac{h^o (C, L_C)}{d - g + 1} \le
\frac{\frac{e}{2} + 1}{ d - g + 1} 
$$
\begin{align*}
& \Longrightarrow de + \frac{d}{2} + (1 - g)(e + \frac{1}{2}) \le
  \frac{de}{2} + d \Longrightarrow \frac{d(e - 1)}{2} \le (g - 1) (e +
  \frac{1}{2})\\ 
& \Longrightarrow \frac{20 (g - 1)(e - 1)}{2} \le (g - 1) (e +
  \frac{1}{2}) \quad (\because d \ge 20 (g - 1))\\ 
& \Longrightarrow 10e - 10 < e + \frac{1}{2} \quad (\because g \ge
  2)\\ 
& \Longrightarrow 9e < 10 + \frac{1}{2} \Longrightarrow e \le 1.
\end{align*}\pageoriginale

But we have already observed that $e > 1$. This contradiction proves
the claim. 

Now consider the following exact sequence.
\begin{equation*}
0 \to I_X \otimes L \to L \to L_{\red} \to 0, \tag{1}\label{c1:eq1}
\end{equation*}
We have the long exact cohomology sequence
$$
\cdots \to H^1 (X, I_X \otimes L) \to H^1 (X, L) \to H^1 (X, L_{\red} \to 0).
$$

\noindent
Since $I_X$ has finite support (cf.\ proposition~\ref{chap1:subprop1.0.3}.,  
page~\pageref{chap1:subprop1.0.3}), 
$H^1 (X, I_X \otimes L) = 0$. We have seen that $H^1 (X_{\red},
L_{\red})=0$. Hence we conclude from the above cohomology sequence that
$H^1 (X, L)=0$. Recall that the restriction 

$\bar{\varphi} : H^o (\mathbb{P}^N, O_{\mathbb{P}^N} (1)) \to H^o
(X_{\red}, L_{\red})$ is injective, (cf.\ proposition \ref{chap1:subprop1.0.2}., 
page~\pageref{chap1:subprop1.0.2}).  

\noindent
Thus
\begin{align*}
d - g + 1 & = h^o (\mathbb{P}^N, O_{\mathbb{P}^N}(1)) \leq h^o(X, L_{\red})\\
& = h^o (X, L) - h^o (X, I_X \otimes L) \quad \text{ (Follows from (1))}\\
& = d - g +1 - h^o (X, I_X \otimes L) \quad (\because h^o (X, L) =
\chi (L) = d - g + 1)\\ 
& \Longrightarrow h^o (X, I_X \otimes L) = 0.
\end{align*}

\noindent
Since\pageoriginale $I_X \otimes L$ has finite support, it follows
that $I_X = 0$ i.e. $X$ is reduced.   

It follows from the above proof that if $X$ is a curve in the family
$Z_V \xrightarrow{p_V} V$ then $H^1 (X, L) = 0$. It is now immediate
that trace of the linear system $|D|$ on $X$ is complete. ($|D|$ is
the complete linear system of $\mathbb{P}^N$ corresponding to the line
bundle $O_{\mathbb{P}^N} (1)$ on $\mathbb{P}^N$). 

\begin{subprop}\label{chap1:subprop1.0.9}% subprop 1.0.9
Let $X$ be a curve in the family $Z_V \xrightarrow{p_V}V$ and let $Y$
be a nonsingular rational component of $X$ i.e. $Y \simeq
\mathbb{P}^{1}$, then $Y$ meets the other components of $X$ in at
least two points. 
\end{subprop}

\begin{proof}
Let $C$ be the closure of $X-Y$ in $X$ with the reduced structure and
let $g_C$ be the genus of $C$ (i.e. $g_C = h^1 (C, O_C)$). 
\end{proof}

Assume that the result is not true, i.e. assume that $Y$ meets $C$ in
exactly one point, $P$ say. Since $X$ is connected, the above
assumption implies that $C$ is connected. 

\medskip
\noindent{\textbf{Claim: }}
$g_C = g$.

\medskip
\noindent{\textbf{Proof of the Claim: }}
Since the curve $X$ is reduced and the only singularities of $X$ are
ordinary double points (cf.\ propositions~\ref{chap1:subprop1.0.8}.,
\ref{chap1:subprop1.0.4}., \ref{chap1:subprop1.0.5}., 
\ref{chap1:subprop1.0.6}.), we have the following exact sequence, 
$$
0 \to  O_X \to O_Y \oplus O_C \to  K \to 0.
$$ 

\noindent 
Take the Euler characteristics.
\begin{align*}
& \chi (O_Y) + \chi (O_C) = \chi (O_X) + 1\\ 
&\Longrightarrow 1 + 1 - h^1 (C, O_C) = 1 - h^{1} (X, O_X) +1,
  (\because Y \simeq P^1 \therefore \chi (O_Y) = 1) \\
& \Rightarrow g_C = h^1 (C,O_C) = h^1 (X, O_X)= g.
\end{align*}\pageoriginale

Now apply the inequality ($*'$) of proposition~\ref{chap1:subprop1.0.7} 
(page~\pageref{chap1:subprop1.0.7}) 
to $C$  
$$
\frac{h^o (C,L_C)}{d-g+1} \geq \frac{e + \frac{1}{2}}{d} >
\frac{e}{d} 
$$

\noindent
We have seen in the proof of the last proposition that $H^1
(X,L)=0$. Hence $H^1 (C,L_C) = 0$. Since $\chi (L_C^m) = em - g_C+ 1$,
it follows from the above inequality that 
\begin{align*}
\frac{e-g_C+ 1}{d - g+ 1} & = \frac{h^o (C,L_C)}{d-g+1} > \frac{e}{d}
\Rightarrow de+d (1 - g_C) > de + e (1-g)\\ 
&  \Rightarrow e( g-1) > d (g_C -1) \Rightarrow e>d, (\because g_C = g
\geq 2) 
\end{align*}

\noindent
This contradiction concludes the proof.
 
Theorem~\ref{chap1:subthm1.0.1}. is now completely proved. Since every
connected 
curve in the family $Z_V \xrightarrow{p_V} V$  is reduced, it can be
easily seen that there exists an open (and closed) subscheme $U$ of
$V$ parametrizing all the connected curves in the family $Z_V
\xrightarrow{p_V} V$ i.e. if $h \in V$  such that the fibre
$X_h$ of $p_V$ over $h$ is connected then $h \in U \subset
V$. By restricting the morphism $p_V$ to $p_V^{-1} (U)$ we get a
family $Z_U \xrightarrow{p_U} U$ of connected curves. 

Let $C$ be a complete, connected subcurve of a curve $X$ in the family
$Z_U \xrightarrow{p_U} U$ . Let $C'$ be the closure of $X-C$ in
$X$. Let $\pi : \bar{X} \to X$ be the normalization of $X$ and let
$P_1, P_2 , \ldots , P_k$ be all the  points on $\bar{C}'$
(normalization of $C'$) such that $\pi (P_i) \in C \cap C'$.  

\noindent
Assume\pageoriginale that the following condition is satisfied.
\begin{enumerate}[i)]
\item For every irreducible component $C_j'$ of $C'$,
\end{enumerate}
$$
\deg _{\bar{C}_j} L' \geq \# (\bar{C}_{j'} \cap \{P_1, P_2, \ldots ,P_k\}).
$$

In this situation we proved the following inequality (cf. page 57).
\begin{equation*}
\frac{h^o (C,L_C)}{d-g+1} \geq \frac{e_C + \frac{k}{2}}{d} \qquad (e_C
= \deg _C L) \tag{$*'$} 
\end{equation*}

\noindent
Note that if $C= X$ then the above inequality is trivially satisfied.

We want to prove that the above inequality ($*'$) holds for every
complete, connected subcurve $C$ of every curve $X$ in the family $Z_U
\xrightarrow{p_U} U$, even if condition i) above is not
satisfied. We prove the result by contradiction. So let $X$ be a curve
in the family $Z_U \xrightarrow{p_U} U$  and let $C$ be a complete
connected subcurve of $X$ for which the inequality ($*'$) is not
satisfied. Hence, 
\begin{equation*}
\frac{e_C - g_C +1}{d-g+1} = \frac{h^o (C, L_C)}{d-g+1} < \frac{e_C +
  \frac{k}{2}}{d} \tag{1}\label{c1:eqq1} 
\end{equation*}
($e_C = \deg _C L$, $k = \# (C\cap C')$, $C'$ is the closure of $X-C$
in $X$) 

\noindent
We may assume that $C$ is maximal in the sense that for every
complete, connected subcurve $C'$ of $X$, with $C \subsetneqq C'
\subset X$ the inequality $(*')$ holds. 

 Since the inequality $(*')$ does not hold for $C$, condition i)
 above is not satisfied i.e., for some irreducible component
 $\bar{C}'_{j}$ of  $\bar{C}', \deg_{\bar{C}_j} L < \# (C\cap C_j) =
 \ell'$.\pageoriginale Then $Y = C \cup C'_j$ is connected and the inequality
$(*')$ holds for $Y$. 
$$
\frac{h^o (Y , L_Y)}{d-g+1} \geq \frac{e_C + e'_{C_j} +
  \frac{k'}{2}}{d}, 
$$
($e_{C'_{j}} = \deg_{C'_j}L$, $k' = \# (Y \cap Y' )$, $Y'$ is the 
  closure of $X - Y$ in $X$)  
$$
 \Rightarrow \frac{e_C + e_{C'_j} - g_Y +1}{d-g+1} \geq \frac{e_C +
  e_{C'_j} + \frac{k'}{2}}{d} 
$$
($\because \quad \chi (L_Y^m) = (e_C + e_{C'_j}) m - g_Y +1$ and 
 $H^1 (Y, L_Y)=0$) 
\begin{equation*}
\Rightarrow \frac{e_C + e_{C'_j} -  g_C - g_{C'_j} - \ell' + 2 }{ d-
  g+1} \geq \frac{e_C + e_{C'_j} + \frac{k}{2} + \frac{k'' -
    \ell'}{2}}{d}\tag{2}\label{c1:eqq2}  
\end{equation*}
$$
(k'' = \# (C'_j \cap Y' ), \ell' = \# (C \cap C'_{j}))
$$

\noindent
The last inequality follows from the following formula
$$
g_Y = g_C + g_{C'_{j}} + \# (C \cap C'_j) -1
$$

\noindent
Multiply the inequality \eqref{c1:eqq1} by $-1$ and add it to the inequality
\eqref{c1:eqq2}. %%%
\begin{equation*}
\frac{e_{C'_j} - g_{C'_j} - \ell' +1}{ d-g+1} > \frac{e_{C'_j} +
  \frac{k'' - \ell'}{2}}{d} \tag{3}\label{c1:eqq3} 
\end{equation*}
\begin{align*}
& \Rightarrow e_{C'_j} - g_{C'_j}- \ell' +1 > (\frac{d-g+1}{d})
  (e_{C'_j} + \frac{k'' - \ell'}{2})\\ 
& = (1- \frac{g-1}{d}) (e_{C'_j} + \frac{k'' - \ell'}{2}) >
  \frac{19}{20} (e_{C'_j} + \frac{k'' -\ell'}{2})\\ 
&\qquad(\because d \geq 20 (g-1))\\ 
& \Rightarrow 20 (e_{C'_j} - g_{C'_j} - \ell' +1) > 19 (e_{C'_j} +
  \frac{k'' - \ell'}{2})\\ 
& \Rightarrow 20 > \frac{21 \ell'}{2} - e_{C'_j}+ 20 g_{C'_j} +
  \frac{19k''}{2} > \frac{19}{2} \ell'  + 20 g_{C'_j} + \frac{19}{2}k''\\
&\qquad(\because e_{C'_j}< \ell ')\\ 
& \Rightarrow \ell' = 2 , g_{C'_j}=0 , k'' =0, \qquad (\because \ell '
  \geq 2) 
\end{align*}\pageoriginale

Since $e_{C'_j} < \ell'$ it follows that $e_{C'_j}=1$.

\noindent
Then the inequality~\eqref{c1:eqq3} reads as $0 > 0$, a contradiction! This
proves that the inequality $(*')$ holds for $C$. It is easy to see
that the inequality $(*')$ holds for $C$ even if $C$ is not
connected. 

\noindent
Thus we have the following proposition.

\begin{subprop}\label{chap1:subprop1.0.10}% subprop 1.0.10
Let $X$ be a curve in the family $Z_U \xrightarrow{p_U} U$, $C$ be a
complete, subcurve of $X$. Let $k = \# (C \cap C')$ ($C'$ is the
closure of $X-C$ in $X$). Then the following inequality holds. 
$$
\frac{h^o (C, L_C)}{d-g+1} \geq \frac{e_C + \frac{k}{2}}{d} , \qquad
(e_C = \deg_C L) 
$$
\end{subprop}

The above result can also be stated in the following form.

\begin{subprop}\label{chap1:subprop1.0.11}% 1.0.11
Let $X$ be a curve in the family $Z_U \xrightarrow{p_U} U$. Let
$\omega_X$ be the dualizing sheaf of $X$. Put $\beta = \dfrac{d}{\deg
  \omega_X}$. For every complete subcurve $C$ of $X$, we have, 
\begin{equation*}
|\deg _C L - \beta \deg _C \omega _X | \leq \frac{k}{2} \tag{$*''$}
\end{equation*}
($k = \# (C \cap C')$, $C'$ is the closure of  $X-C$ in $X$).
\end{subprop}

\begin{proof}
Since\pageoriginale the inequality $(*')$ holds for $C$, simplifying
we get,  
\begin{align*}
\deg_C L & = e_C \geq \frac{d}{g-1} (g_C - 1 + \frac{k}{2}) -
\frac{k}{2} = \frac{d}{2g -1} (2g_C - 2 +k)- \frac{k}{2}\\ 
 & = \beta \deg_C \omega_X - \frac{k}{2} \qquad (\because \deg _C
\omega_X = 2g_C - 2+k) \\
& \Rightarrow \deg_C L - \beta \deg_C \omega_X \geq -
\frac{k}{2}. \tag{1}\label{c1:eqqq1} 
\end{align*}

Since the inequality $(*')$ holds for $C'$, we get, as above 
\begin{align*}
& \deg_C' L - \beta \deg_C' \omega _X \geq - \frac{k}{2}\\
& \Rightarrow \deg_C L - \beta \deg_C \omega_X \leq \frac{k}{2} \tag
  {2}\label{c1:eqqq2} 
\end{align*}
$(\because \deg_C' L - \beta \deg_C' \omega_X = \beta \deg_C \omega_X
- \deg_C L)$ 

\medskip
\noindent
The result follows from \eqref{c1:eqqq1} and \eqref{c1:eqqq2}.
\end{proof}

Next proposition is an easy consequence of 
proposition~\ref{chap1:subprop1.0.10}.  

\begin{subprop} % propositoin 1.0.12
Let $X$ be a curve in the family $Z_U \xrightarrow{p_U} U$ and let
$C' \subsetneqq X$ be a connected chain of smooth rational curves
meeting $C$ (closure of $X-C'$ in $X$) in two points. Then, i) $C'$
is irreducible, ii) $\deg_C' L =1$.  
\end{subprop}

\begin{proof}
In view of proposition \ref{chap1:subprop1.0.10}
(page \pageref{chap1:subprop1.0.10}) we have the following 
inequality.
\begin{align*}
& \qquad  \frac{e_C - g_C +1}{d-g+1} = \frac{h^o (C, L_C)}{d-g+1} \geq
\frac{e_C +1}{d}, \quad (e_C = \deg_C L) \\
& \Rightarrow de_C - dg_C + d \geq de_C + d + (1-g) \chi( e_C + 1 )\\ 
& \Rightarrow dg_C \leq (g-1) (e_C +1) \Rightarrow d \leq e_C +1 \quad
  (\because g-1 = g_C \geq 1)\\
& \Rightarrow e_C = d-1 \Rightarrow \deg_{C'} L = 1 \Rightarrow C'
\text{ is irreducible} 
\end{align*}\pageoriginale

\noindent
Now we make the following observation. Let $X$ be a curve in the
family $Z_U \xrightarrow{p_U} U$ and let $\omega_X$ be the dualizing
sheaf of $X$. $\omega_X$ is a line bundle ($X$ being Gorenstein) and
$\omega_X^3$ gives a morphism, $\psi_o : X \to \mathbb{P}^r$. The
above proposition implies that the image $X'$ of $X$ under $\psi_o$ is
a stable curve and the fibre $X_{x'}$ of $\psi_o$ over a point $x'
\in  X$ is either a point or $X_{x'} \simeq \mathbb{P}^1$ and
$X_{x'}$ meets the rest of the curve in two points. $X'$ is thus
obtained by replacing each smooth rational component of $X$, meeting
the rest of the curve in two points, by a node. 
\end{proof}

