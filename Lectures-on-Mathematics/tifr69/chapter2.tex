\chapter{The Moduli Space of Curves}\label{chap2}%chapter 2

In\pageoriginale this chapter we construct the Deligne-Mumford Moduli
space of stable curves. We prove that it is a reduced and irreducible
scheme projective over Spec $K$.  

We keep the same notations as in Chapter~\ref{chap1}. Thus $Z_U
\xrightarrow{p_U} U$ is a family of connected curves of genus $g \geq
2$ and degree $d \geq 20 (g-1)$ in $\mathbb{P}^N$, $(N = d-g)$, such
that the $m^{\rm th}_o$ Hilbert point of $X$, $H_{m_o} (X)
\in \mathbb{P}( \overset{P(m_o)}\wedge H^o (\mathbb{P}^N ,
O_{\mathbb{P}^{N}} (m_o)))$ is semistable for the natural action of
$SL (N+1)$ on $\mathbb{P} (\overset{P (m_o)} \wedge H^o (\mathbb{P}^N,
O_{\mathbb{P}^N} (m_o)))$. Assume now onwards that $d=n(2g-2)$ where
$n$ is an integer, $n \geq 10$. 

 For a geometric point $h \in U$ let $X_h$ be the fibre of
 $Z_U \xrightarrow{p_U} U$ over $h$, $L_h$ be the restriction of
 $O_{\mathbb{P}^N}(1)$ to $X_h$, $\omega_{X_h}$ be the dualizing sheaf
 of $X_h$. It is easily seen that the set $U_C = \{h \in U |
 L_h \simeq \omega_{X_h}^n\}$ is constructable. We want to prove that
 $U_C$ is a closed subscheme of $U$ parametrizing all curves $X_h$ in
 the family $Z_U \xrightarrow{p_U} U$ such that $L_h \simeq
 \omega_{X_h}^n$. The next proposition proves that $U_C$ is a closed
 subset of $U$. 
 
\setcounter{section}{2}
\setcounter{subprop}{-1}
 \begin{subprop}% 2.0.0
$U_C$ is a closed subset of $U$. 
   \end{subprop}   

   \begin{proof}
It suffices to prove the following. For every morphism Spec $R
\xrightarrow{\alpha}  U$, ($R$ a discrete valuation ring), if the
image of the generic point of Spec $R$ is in $U_C$ then the image of
the special  point of Spec $R$ is also in $U_C$. 
   \end{proof}   
   
   Make\pageoriginale the base change of $Z_U \xrightarrow{p_U} U$ by
   Spec $R    \xrightarrow{\alpha} U$, 
\[
\xymatrix@C=1.5cm{
Z_R\ar[r]\ar[d]^{p_R} & Z_U\ar[d]^{p_U}\\
\text{Spec } R \ar[r]^{\alpha} & U
}
\]

\noindent
The relatively very ample line bundle $O_{Z_U}(1)$ on $Z_U$ induces a
line bundle $O_{Z_R}(1)$ on $Z_R$. Let $\omega_{Z_R/R}$ be the
relative dualizing sheaf on $Z_R$. Let $h_o$ and $h_1$ be
respectively, the special point and the generic point special point
of spec $R$, and let $X_{h_o}$ and $X_{h_1}$ be the special and
generic fibres. Write $X_{h_o}= \bigcup\limits_{i=1}^{q'} C_i$ where
$C_i$ is an irreducible component of $X_{h_o}$. The restrictions of
$O_{Z_R}(1)$ and  $\omega_{Z_{R / R}}^n$ to $X_{h_1}$ are
isomorphic. This follows from the definition of $U_C$. To prove that
$\alpha (h_o) \in U_C$ is equivalent to showing that the
restrictions of the line bundles $\omega_{Z_{R / R}}^n$ and $O_{Z_R}(1)$ to
$X_{h_o}$ are isomorphic. 

Write $O_{Z_R}(1) \simeq \omega_{Z_{R / R}}^n \otimes M$, where $M$ is a
line bundle on $Z_R$ which on $Z_R - \{\text{nodes of } X_{h_o}\}$ is of
the form $O_{Z_R} (- \sum\limits_{i=1}^{q'} r_i C_i )$. Let $t$ be the
uniformizing parameter of $R$. Tensoring $O_{Z_R}(1)$ with the trivial
line bundle associated to the principal divisor $p^*_R((t^{\min
  (r_i)})$ we can assume that $r_i \geq 0$, $\min (r_i)=0$ i.e. we can
assume that $M$ is an ideal sheaf. 

Let $J= \bigcup\limits_{r_i > 0} C_i$, $J' = \bigcup\limits_{r_i = 0} C_i
$. If $g$ is the local equation of $M$ then $g \not\equiv 0$ in any
component of $J'$ and $g(x) =0$ for all $x \in J \cap
J'$. Hence $\# (J \cap J') \leq \deg _J' M$. However, we have,
$|\deg_J' M| = |\deg _J' L-n \deg_J' \omega_{X_{h_o}} | \leq \#
\dfrac{(J \cap J')}{2}$,\pageoriginale 

(cf.\ proposition~\ref{chap1:subprop1.0.11}, 
page~\pageref{chap1:subprop1.0.11}). This forces $J' = X_{h_o}$ 
i.e. $M$ is trivial. This proves the result. 

Recall that $U_C$ is precisely the set of points $h \in U$
such that the restriction of the line bundle $\omega_{Z_U/U}^n \otimes
O_{Z_U}(-1)$ to the 
fibre of $Z_U \xrightarrow{p_U} U$ over $h$ is trivial. Now using
standard arguments (cf.~\cite{key4}, page 89) $U_C$ is given the structure
of a closed subscheme of $U$, having the following properties, 
\begin{enumerate}[i)]
\item There exists a line bundle $M'$ on $U_C$ such that the
  restriction of $\omega_{Z_U/U}^n \otimes O_{Z_U}(-1)$ to $Z_{U_C}
  \times_U Z_U$ is isomorphic to $p^*_{U_C}(M')$, ($p_{U_C}$ denotes the
  projection $Z_{U_C} \to U_C$); 

\item If $f : W \to U$ is any morphism such that for some line bundle
  $M''$ on $W$ the line bundles $(1 \times f)^* (\omega_{Z_U/U}^n
  \times O_{Z_U} (-1))$ and $p^*_W (M'')$ on  $W \underset{U}\times
  Z_U$ are isomorphic then $f$ can be factored as $W \to U_C \to U$.  
\end{enumerate}

\setcounter{subtheorem}{0}
\begin{subtheorem}% theorem 2.0.1
$U_C$ is nonsingular.
\end{subtheorem}

\begin{proof}
Let $h \in U_C$ be a closed point and let $X_h$ be the fibre
of $Z_{U_C} \xrightarrow{p_{U_C}} U_C$ over $h$. Let $\tilde{X}$ be
the universal formal deformation of $X_h$over $T= \text{Spec} [[t_1,
t_2, \ldots, t_3]], (s = \dim Ext^1 (\Omega_{X_{h}/ K} , 
O_{X_h}))$, (cf.\ theorem~\ref{chap0:subthm0.2.2},\break 
page~\pageref{chap0:subthm0.2.2}).  
\end{proof}

Let $\eta : S = {\rm Spec} \hat{O}_{U_{C,h}} \to U_C$ be the natural
map. We have the following commutative diagram. 
\[
\xymatrix@R=1.5cm@C=1.5cm{
Z_{U_{C,h}} \ar[r]\ar[d] & Z_{U_C} \ar[d]\\
S \ar[r]^\eta & U_C
}
\]\pageoriginale 

\noindent
It follows from theorem~\ref{chap0:subthm0.2.2}. 
(page~\pageref{chap0:subthm0.2.2}) that there exists a 
unique morphism $f : S\to T$ such that $Z_{U_{C,h}} \simeq S
\underset{T}\times \tilde{X}$ and the isomorphism restricted to the
closed fibres is the identity morphism. 

\medskip
\noindent{\textbf{Claim: }}
 $f : S \to T$ is formally smooth , i.e.  

$\hat{O}_{U_{C,h}}\simeq [[t_1,t_2, \ldots , t_s, t_{s+1}, \ldots ,
    t_{s+s}]]$ for some nonnegative integer $s'$. 

Note that if we prove the claim, the result will follow. Choose an
isomorphism $\mathbb{P} (\pi_* (\omega_{X/T}^n)) \simeq \mathbb{P}^{n
  (2g -2)-g} \times T$, (cf.\ stable curves, page~\pageref{page12}). By the
universal property of $U_C$ we have a unique morphism $\gamma : T \to
U_C$ such that the following diagram is commutative. 
\[
\xymatrix{
\tilde{X} \ar[r]\ar[d] & Z_{U_C} \ar[d]\\
Y \ar[r]^{\gamma} & U_C
}
\]

\noindent
Clearly $\gamma$ has a factorization 

\noindent
$T \xrightarrow{\gamma} S \xrightarrow{\eta} C$\pageoriginale  and
$\gamma'$ is a section of $f$. 

Recall that $G =  PGL (N + 1)$ acts on $V_C$. Let $S_h$ be the
stabilizer of $h \in U_C$. Then $S_h$ is finite and
reduced. In fact if it were not then $S_h$ would have a nonzero
tangent vector i.e. there would be a $\dfrac{K[\varepsilon]}{
  (\varepsilon^2)}$ valued point of $G$ centered at the identity
which gives an automorphism of $ X_h \times
\dfrac{K[\varepsilon]}{\varepsilon^2} \subset \mathbb{P}^{n (2g -2)-g}
\times \dfrac{K[\varepsilon]}{\varepsilon^2}$, hence a vector field
defined on the whole of $X_h$. We have already seen that such a vector
field is necessarily zero (cf.\ lemma~\ref{chap0:sublem0.1.7}. 
page~\pageref{chap0:sublem0.1.7}). Thus the 
automorphism of $X_h \times \dfrac{K[\varepsilon]}{(\varepsilon^2)}$
must be identity and further since $X_h$ is connected and
nondegenerate in $\mathbb{P}^N$ (cf.\ 
proposition~\ref{chap1:subprop1.0.2}. 
page~\pageref{chap1:subprop1.0.2}) 
the automorphism $\mathbb{P} \times
\dfrac{K[\varepsilon]}{(\varepsilon^2)}$ must be identity. It follows
that $S_h$ is finite and reduced and hence the action of $G$ on $S$ is
formally free which amounts to saying that $S$ is formally a principal
fibre bundle over $T$ with group $G$. Therefore $S$  is formally
smooth over $T$. 

Further $U_C$ is contained in the open subset of $U$, parametrizing
stable curves. To see this let $h \in U_C$ such that the fibre
$X_h$ of $Z_U \xrightarrow{p_U} U$ over $h$ is semistable but not
stable i.e. $Y$ is a smooth rational component of $X_h$ which meets
the other components of $X_h$ in exactly two points. But then the
restriction of $\omega^n_{X_h}$ to $Y$ is very ample and $\deg_Y
\omega^n_{X_h} = n \deg_Y \omega_{X_h } = n (\deg \omega_Y + \# (Y
\bigcap Y' ) = 0$, 

\noindent
($Y'$ is the closure of $X_h - Y$ in $X_h$).
 
 \noindent
  This\pageoriginale  contradiction proves that $X_h$ is necessarily a
  stable curve.  

Recall that the morphism $U \to \mathbb{P} (\overset{P(m_o)}{\Lambda}
H^o (\mathbb{P}^N, O_{\mathbb{P}^N} (m_o)))^{ss}$ defined by
$\psi_{m_o} (h) = H_{m_o}(X_h) (h \in  U)$ is a closed
immersion and also it is a $G$-morphism. By 
Theorem~\ref{chap0:subthm0.0.5} (page~\pageref{chap0:subthm0.0.5}) 
$\mathbb{P} (\overset{P(m_o)}{\Lambda} H^o (\mathbb{P}^N,
O_{\mathbb{P}^N} (m_o)))^{ss}$ has a good quotient by $G$ which is
projective. Then it is easy to see that $U_C$ has a good quotient by
$G$ (denoted by $U_C/G$) which is projective. We are now ready to
state the main theorem of this chapter. 

\begin{subtheorem} % theorem 2.0.2
$U_C/G$ is a coarse moduli space of isomorphism classes of stable
  curves of genus $g$. Further $U_C/G$ is reduced and irreducible.  
\end{subtheorem}

\begin{proof}
We first prove that every stable curve (in the sense of 
Definition~\ref{chap0:subdef0.1.4}. page~\pageref{chap0:subdef0.1.4}) 
is represented in $U_C/G$.  
\end{proof}

Let $X$ be a stable curve of genus $g$. Let $\gamma: X' \to 
\text{Spec}R$ be a deformation of $X$ to a connected, smooth curve where 
$R$ is a discrete valuation such that, $K_1$, the residue field of $R$
is algebraically closed. Let $K_2$ be the quotient field of $R$. Note
$X' \to $ Spec $R$ is a stable curve and hence can be realized as a
family of curves in $\mathbb{P}^{n (2g-2)-g}$ (cf.\ Stable curves, 
page~\pageref{c0:st_curves}). Then by the universal property of the 
Hilbert scheme $H$ we get
a morphism $\rho :$ Spec $R \to H$. Since the generic fibre $X'_{K_2}$
of $\gamma$ is smooth, the image of Spec $K_2 \subset$ Spec $ R$ lies
in the locally closed subscheme $U_C$ of $H$ (cf.\ 
Theorem~\ref{chap1:subthm1.0.0} page~\pageref{chap1:subthm1.0.0}). Since
$U_C/G$ is complete, (cf.\ Definition~4.1. page~526 \cite{key10}) 
there exists a morphism  
$\rho': \Spec R \to U_C$\pageoriginale such that if $X'' \to \Spec R$
is the base change of $Z_{U_C} \to U_C$ by the morphism $\rho': \Spec
R \to U_C$ then the generic fibre $X''_{K_2}$ of $X'' \to \Spec R$ is
isomorphic $X'_{K_2}$. Now note the following lemma.  

\setcounter{sublemma}{2}
\begin{sublemma}\label{chap2:sublem2.0.3}% 2.0.3
Let $Y'$ and $Y''$ be two stable curves over a discrete valuation ring
$R$ with algebraically closed residue field. Let $Y$ be the generic
point of $\Spec R$ and assume that the generic fibres $Y'_Y$ and
$Y''_Y$ are smooth. Then any isomorphism between $Y'_Y$ and $Y''_Y$
extends to an isomorphism between $Y'$ and $Y''$. 
(cf.\ Lemma~1.12 \cite{key1}) 
\end{sublemma}

In view of the above lemma, it follows that the isomorphism between
$X'_{K_2}$ and $X''_{K_2}$ can be extended to an isomorphism of $X'$
and $X''$ over $\Spec R$. Observe that the isomorphism between
$X'_{K_1}$ and $X''_{K_2}$ is induced by an automorphism of
$\mathbb{P}^{n(2g-2)-g}$. Thus in the Hilbert scheme $H$, the two
points representing the curves $X'_{K_1}$ and $X''_{K_1}$ lie in the
same $G$-orbit. Now it is immediate that the morphism $\rho: \Spec R
\to H$ factors as $\Spec R \to U_C \to H$. 

Thus every stable curve is represented in $U_C$. Since $G$ acts on
$U_C$ with finite isotropy (cf.\ Theorem~\ref{chap0:subthm0.1.8}. 
page~\pageref{chap0:subthm0.1.8}) and with 
closed orbits (cf.\ Lemma~\ref{chap2:sublem2.0.3}) the good quotient
$U_C/G$ of $U_C$ by $G$ is a coarse moduli space for isomorphism 
classes of stable curves. 

It remains to prove that $U_C/G$ is reduced and irreducible. 
We know this to be true when characteristic of the ground field $K$ is
zero, \cite{key11}. So assume now that characteristic of $K$ is positive. 

\noindent
Let\pageoriginale $R$ be a discrete valuation ring such that the
quotient field of $R$ has characteristic zero and $K$ is the residue
field of $R$.  

Construct $U_C$ over $\Spec R$ and call it $U_{C,R'}$  (Note that the
method of our proof works over the base $\Spec R$, cf.\ \cite{key9}). 
Let $G_R$ be the group $PGL(N+1) (R)$. Since $U_{C,R}/G_R$ is
projective and the  generic fibre of $U_{C,R}/G_R \to \Spec R$ is
connected, Zariski's connectedness theorem\break shows that $U_{C,R}/G_R
\otimes K$ is connected. Note that $U_{C,R}/G_R \otimes K =
U_{C,R}\otimes K/G$ is just the orbit space, hence $U_{C,R}\otimes K =
U_C$ is connected. We have already seen that $U_C$ is smooth. 
Thus $U_C$ is reduced and irreducible. Recall that the structure sheaf
of $U_C/G$ is the sheaf of invariants in the structure sheaf of
$U_C$.  Hence $U_C/G$ is reduced and irreducible.

\newpage

\chapter*{Appendix}


Let\pageoriginale $X'$ be a reduced, complete, connected curve which
has at most ordinary double points. Write
$X'=\bigcup\limits^{n}_{i=1}X'_i$ ($X'_i$ an irreducible component of
$X'$). Let $L'$ be a line bundle on $X'$ and let
$\lambda_1,\lambda_2,\ldots,\lambda_n$ be positive rational numbers
with $\sum \lambda_i =1$.  Following Oda-Seshadri \cite{key8} we say that the
line bundle $L'$ is $(\lambda_i)$-semistable, if for every complete,
connected subcurve $C$ of $X'$, $\dfrac{\chi(L'_C)}{\chi(L')} \ge
_{X'_i}\underset{\subset C}\sum \lambda_i$, where the summation is
taken over all $i$ such that $X'_i \subset C$.  

Now let $X$ be a stable curve in the family $Z_U \xrightarrow{P_U} U$,
(Notation as in Chapter~\ref{chap1}, cf.\ page~\pageref{c1:eqq1}). Write $X = \bigcup
\limits^{n}_{i=1}X_i$, ($X_i$ an irreducible component of $X$). Let
$\omega_X$ be the dualizing sheaf of $X$. Let $\lambda_i=
\dfrac{\deg_{X_i}\omega_X}{\deg \omega_X} (1 \le i \le n)$. Let $L$ be
the very ample line bundle on $X$. 


\begin{prop*}
{\rm \textbf{Al.}}
The line bundle $L$ is $(\lambda_i)$-semistable.
\end{prop*}

\begin{proof}
For every complete, connected subcurve $C$ of $X$, we have
$$
\frac{e_C-g_C+1}{d-g+1} = \frac{X(L_C)}{X(L)} \ge \frac{e_C
  +\dfrac{k}{2}}{d} 
$$
($e_C=\deg_C L$, $k= \# (C \cap C')$,  $C'$ is the closure of $X-C$ in $X$),
(cf.\ Proposition~\ref{chap1:subprop1.0.10}., page~\pageref{chap1:subprop1.0.10}).  
{\fontsize{10}{12}\selectfont
\begin{align*}
\Rightarrow & \qquad de_C+d(1-g_C)\ge de_C + \frac{dk}{2}+(1-g)
(e_C+\frac{k}{2})\\ 
\Rightarrow &  \qquad(e_C+\frac{k}{2})(g-1) \ge d(g_C -1+\frac{k}{2})
=\frac{d}{2}(2g_C -2+k)=\frac{d(deg_C \omega_X)}{2} \\
\Rightarrow & \qquad \frac{e_C+\frac{k}{2}}{d} \ge \frac{deg_C
  \omega_X}{2g-2}= \sum\limits_{X_i \subset C}\lambda_i\\  
\Rightarrow &  \qquad \frac{X(L_C)}{X(L)} \ge  \sum_{X_i \subset C}
\lambda_i 
\end{align*}}\relax\pageoriginale

For a complete subcurve $Y$ of $X$, we apply the above inequality to
each connected component $C$ of $X$ and by adding we get the result. 
\end{proof}

\begin{thebibliography}{99}
\bibitem{key1} {P. Deligne, D. Mumford} \pageoriginale {The Irreducibility
  of the Space of Curves of Given Genus   Publications Mathematiques,}
  IHES 36 (1969), pages 75-110.  

\medskip

{A. Grothendieck}


\bibitem{key2} {Fondements de la geometrie algebrique}

[Extraits du Seminaire Bourbaki 1957-1962]

\bibitem{key3} {Techniques de descente et theoremes d'existence en
  geometrie algebrique}, IV, Sem. Bourbaki, 221, 1960-1961. 

\medskip

{David Mumford}


\bibitem{key4} {Abelian Varieties,} Oxford University Press. (1974)

\bibitem{key5}{Geometric Invariant Theory,} Springer-Verlag (1965).

\bibitem{key6}{Lectures on Curves on an Algebraic Surface.}

Annals of Mathematics Studies, Number 59, Princeton University Press, (1966).

\bibitem{key7}{Stability of Projective Varieties}\\
L'Enseiqnement Mathematique\\
$II^e$ Serie \\

Tome XXIII-Fascicule 1-2, 1977

pages 39-110.

\medskip
Oda-Seshadri

\bibitem{key8} {Compactifications of the Generalized Jacobian Variety,}
Transactions of the American Mathematical Society, Vol.253, 1979. pages 1-90.

\medskip

C.S. Seshadri\pageoriginale

\bibitem{key9} {Geometric Reductivity over Arbitrary Base.}

Advances in Mathematics, Vol.26, No.3, pages 225-274.

\bibitem{key10} {Quotient Spaces Modulo Reductive Algebraic Groups.}

Annals of Mathematics, Vol.95, No.3, pages 511-556.

\medskip
Andr\'e Weil

\bibitem{key11} {Modules des surfaces de Riemann}, Seminaire Bourbaki,
  expos\'e 168, 1958. 
\end{thebibliography}

