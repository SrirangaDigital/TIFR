\chapter{Introduction}

These notes are based on some lectures given at TIFR during January
and February 1980. The object of the lectures was to construct a
projective moduli space for stable curves of genus $g \ge 2$ using
Mumford's geometric invariant theory. 

The general plan of the notes is as follows: Chapter \ref{chap0} consists of
preliminaries. In particular, for $m > > 0$, we review how to attach to
each space curve $C \subset \mathbb{P}^n$ a point in some projective
space called the $m^{\rm th}$ Hilbert point of $C$. We then consider the
question of the stability of the $m^{\rm th}$ Hilbert point in the sense
of geometric invariant theory. Our first main result in Chapter~\ref{chap1} is
that if $C$ is smooth and $d \ge 20(g-1)$, then the $m^{\rm th}$ Hilbert
point of $C$ is stable. Our second main result in Chapter~\ref{chap1} is that
if the $m^{\rm th}$ Hilbert point is semi-stable, then the curve is
semi-stable as a curve. In Chapter~\ref{chap2}, we use the results of 
Chapter~\ref{chap1} to give an indirect proof that the $n$-canonical embedding of a
stable curve is stable if $n \ge 10$, and to construct the projective
moduli space for stable curves. As corollaries, we obtain proofs of
the stable reduction theorem for curves, and of the irreducibility of
the moduli space for smooth curves. 

Historically speaking, Mumford used his theory to construct a
quasi-projective moduli space for smooth curves by studying the
stability of the Chow points of spaces curves. Mumford and 
Deligne~\cite{key1} introduced the concept of stable curve in their proof of the
irreducibility of the moduli space of curves of genus $g \ge 2$, and
later F.Knudsen established the existence of a projective moduli space
for stable curves. In 1974, Mumford and I realized that the
n-canonical model of a stable curve was stable in the invariant theory
sense if $n > > 0$. Mumford then showed that the Chow point of the
n-canonical model of a stable curve is stable if $n\ge 5$, \cite{key7}. Our
treatment here parallels that of Mumford, except for technical points
arising from the difference between Chow and Hilbert points. (I
believe one could use Hilbert point methods in the case $n \ge 5$). 

I wish to thank D.R. Gokhale, who filled in many gaps in the original
lectures. I also wish to thank TIFR for inviting me for a most
enjoyable visit and my audience, especially C.S. Seshadri, for their
comments and patience. 


\newpage
\noindent
\textbf{Notation}

The following notations will be used without further comment.

\medskip
\begin{tabular}{l p{7cm}}
$K$ & a fixed algebraically closed field\\
$K^*$ & multiplicative group of non-zero elements in $K$\\
$\mathbb{A}^N$ & affine $N$-space over $K$\\
$\mathbb{P}^N$ & projective $N$-space over $K$\\
$GL(N + 1)$ & group of invertible $(N + 1) \times (N + 1)$ matrices
  over $K$.\\ 
$SL(N + 1)$ & group of elements in $GL(N + 1)$ with determinant
  $1$.\\ 
$PGL(N + 1)$ & $GL(N + 1)$/scalar multiples of the Identity matrix.\\ 
$PGL(N + 1)(R)$ & group of invertible $(N + 1) \times (N + 1)$
  matrices over a ring $R$/scalar multiples of the Identity matrix.\\ 
$1-ps \lambda$ & One parameter subgroup of an algebraic group Let $X$
  be a projective scheme and let $F$ be a coherent $0_X$ module.\\ 
$H^i(X,F)$ & $i^{th}$ cohomology of $X$ with coefficients in $F$ \\ 
$h^i(X,F)$ & $\dim H^i(X,F)$\\
$\chi(F)$ & $\sum  (-1)^i  h^i(X,F)$\\ 
$\#S$ & cardinality of a set $S$.\\
\end{tabular}







