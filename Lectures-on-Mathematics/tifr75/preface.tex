\thispagestyle{empty}
\begin{center}
{\Large\bf Lectures on}\\[5pt]
{\Large\bf Stochastic Control and Nonlinear Filtering}
\vskip 1cm

{\bf By}
\medskip

{\large\bf M.~H.~A. Davis}
\vfill

{\bf Tata Institute of Fundamental Research}

{\bf Bombay}

{\bf 1984}
\end{center}
\eject

\thispagestyle{empty}
\begin{center}
{\Large\bf Lectures on}\\[5pt]
{\Large\bf Stochastic Control and Nonlinear Filtering}
\vskip 1cm

{\bf By}
\medskip

{\large\bf M.~H.~A. Davis}
\vfill

{Lectures delivered at the}\\
{\bf Indian Institute of Science, Bangalore}\\[5pt]
{under the}\\[5pt]
{\bf T.I.F.R.--I.I.Sc. Programme in Applications of}\\[5pt]
{\bf Mathematics}
\vfill

{\bf Notes by}\\[5pt]
{\large\bf K.~M.~Ramachandran}
\vfill

{Published for the}\\[5pt]
{\bf Tata Institute of Fundamental Research}
\vfill

{\bf Springer-Verlag}\\
{Berlin Heidelberg New York Tokyo}\\
{\bf 1984}
\end{center}
\eject

\thispagestyle{empty}
\begin{center}
{\bf Author}\\[15pt]
{\large\bf M.~H.~A. Davis}\\
{Department of Electrical Engineering}\\
{Imperial College of Science and Technology}\\
{London SW 7}\\
{United Kingdom}
\vfill

{\bf\copyright Tata Institute of Fundamental Research, 1984}
\vfill

\rule{\textwidth}{.5pt}

ISBN 3-540-13343-7 Springer-Verlag, Berlin. Heidelberg.\\ New York. Tokyo

ISBN 0-387-13343-7 Springer-Verlag, New York. Heidelberg.\\ Berlin. Tokyo

\rule{\textwidth}{.5pt}

\vfill


\parbox{0.7\textwidth}{
No part of this book may be reproduced in any form by print, microfilm
or any other means without written permission from the Tata Institute
of Fundamental Research, Colaba, Bombay 400 005}
\vfill

Printed by M.~N.~Joshi at The Book Centre Limited,

Sion East, Bombay 400 022 and published by H. Goetze,

Springer-Verlag, Heidelberg, West Germany


\vskip 1cm

{\bf Printed in India}
\end{center}
\eject

\chapter*{Preface}

These notes comprise the contents of lectures I gave at the
T.I.F.R. Centre in Bangalore in April/May 1983. There are actually two
separate series of lectures, on controlled stochastic jump processes
and nonlinear filtering respectively, and the corresponding two parts
of these notes are almost disjoint. They are united however, by the
common philosophy (if that is not too grand a work for it) of treating
Markov processes by methods of stochastic calculus, and I hope the
reader will, at least, be convinced of the usefulness of this and of
the `extended generator' concept in doing calculations with Markov
precesses.

The first part is aimed at developing optimal control theory for a
class of Markov processes called piecewise-deterministic (PD)proce\-sses. These were only isolated rather recently but seen
general enough to include as special cases practically all the
non-diffusion continuous time processes of applied
probability. Optimal control for PD processes occupies a curious
position just half way between deterministic and Sto\-chastic optimal
control theory in such a way that no standard theory from either side
is adequate to deal with it. The only applicable theory that exists at
all is very recent work of D. Vermes based on the generalized dynamic
programming ideas of R.B. Vinter and R.M. Lewis, and this is what I
have attempted to describe here. Undoubtedly, further development of
control theory for PD processes will be a fruitful field of enquiry. 

Part II concentrates on the ``pathwise'' theory of filtering for
diffusion processes and on more sophisticated extensions of it due
primarily to H. Kunita. The intriguing point here is to see how
stochastic partial differential equations can be dealt with by
stochastic flow theory through what amounts to a ``doubly stochastic''
version of the FeynmanKac formula. Using this, Kunita has given an
elegant argument to show the existence of smooth conditional densities
under H\"ormander-type conditions. This is included. Ultimately, it
rests on results obtained by Bismut and others using Malliavin
calculus, since one needs a version of the H\"ormander theorem which
is valid for continuous (rather than $C^{\infty}$) $t$-dependence of
the coefficients. It was unfortunately impossible to go into such
questions in the time available.

I would like to thank Professor K.G. Ramanathan for his kind
invitation to visit Bangalore and K.M. Ramachandran for his heroic
efforts at keeping up-to-date notes on a rapidly accumulating number
of lectures, and for preparing the final version of the present
text. I would also like to thank the students and staff of the
T.I.F.R. Centre and of the I.I.Sc. Guest House for their friendly
hospitality which made my visit such a pleasant one. 
