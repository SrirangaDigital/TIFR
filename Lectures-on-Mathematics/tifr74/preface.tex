\thispagestyle{empty}
\begin{center}
{\Large\bf Lectures on}\\[5pt]
\textbf{\Large Results on Bezout's Theorem}
\vskip 1cm

{\bf By}
\medskip

{\large\bf W. Vogel}
\vfill

{\bf Tata Institute of Fundamental Research}\\[3pt]
{\bf Bombay}\\[3pt]
{\bf 1984}
\end{center}
\eject

\thispagestyle{empty}

\begin{center}
{\Large\bf Lectures on}\\[20pt]
{\Large\bf Results on Bezout's Theorem}\\
\vfill

{\bf By}
\medskip

{\large\bf W. Vogel}
\vfill

{\bf Notes by}
\medskip

{\large\bf D.P. Patil}
\vfill


{Published for the}\\[5pt]
{\bf Tata Institute of Fundamental Research}\\
{\large\bf Springer-Verlag}\\
{Berlin Heidelberg New York Tokyo}\\
{\bf 1984}
\end{center}
\eject

\thispagestyle{empty}

\begin{center}
{\large\bf Author}\\[5pt]
{\large\bf W. Vogel}\\
{Martin-Luther Universit\"at}\\
{Sektion Mathematik}\\
{DDR-401 Halle}\\
{Universit\"ats-platz 6}\\
{German Democratic Republic}

\vfill

{\bf\copyright Tata Institute of Fundamental  Research. 1984}
\vfill

\rule{\textwidth}{.5pt}

ISBN 3-540-12679-1 Springer-Verlag, Berlin. Heidelberg.\\ New York. Tokyo

ISBN 0-387-12679-1 Springer-Verlag, New York. Heidelberg.\\ Berlin. Tokyo

\rule{\textwidth}{.5pt}

\vfill

\parbox{0.7\textwidth}{No part of this book may be reproduced in any form by
print. microfilm or any other means without written permission
from the Tata Institute of Fundamental Research, Colaba, Bombay
400 005}

\vfill

{Printed by}\\[5pt]
{\large\bf  M.N. Joshi at The Book Centre Limited,}\\
{\bf  Sion East, Bombay 400 022}

\vfill

{Published by}\\[5pt] 
{\large\bf  H. Goetze,}\\
{Springer-Verlag, Heidelberg, West Germany}\\[20pt]
{\bf Printed in India}
\end{center}
\eject

\chapter*{Introduction}

 
These notes are based on a series of lectures given at the Tata
Institute in November and December, 1982. The lectures are centered
about my joint work with J\"{u}rgen St\"{u}ckrad [85] on an
algebraic approach to the intersection theory. More-over, chapter $II$
and $III$ also contain new results. 

Today, we have the remarkable theory of W.Fulton and R.~Mac\-person on
defining algebraic intersections: 

Suppose $V$ and $W$ are subvarieties of dimension $v$ and $w$ of a
nonsingular algebraic variety $X$ of dimension $n$. Then the
equivalence class $V \cdot W$ of algebraic $v + w -n$ cycles which
represents the algebraic intersection of $V$ and $W$ is defined up to
rational equivalence in $X$. This intersection theory produces
subvarieties $Y_i$ of $V \cap W$, cycle classes $\alpha_i$ on $Y_i$,
positive integers $m_i$, with $\sum m_i \alpha_i$ representing $V
\cdot W$,
and $\deg \alpha_i \ge \deg Y_i$ even in the case $\dim (V \cap W)
\neq v + w - n$.  

Our object here is to give an algebraic approach to the intersection
theory by studying a formula for $\deg (V)$. $\deg (W)$ in terms of
algebraic data, if $V$ and $W$ are Gubvarieties of $X =
\mathbb{P}^n_K$. 

The basis of our formula is a method for expressing the intersection
multiplicity of two properly intersecting varieties as the length of a
certain primary ideal associated to them in a canonical way. Using the
geometry of the join construction in $\mathbb{P}^{2n+1}$ over a field
extension of $K$, we may apply this method even if $\dim (V \cap W) >
\dim V + \dim W - n$. More precisely, we will prove the following
statement in Chapter II:  

Let $X, Y $ be pure dimensional projective subvarieties of
$\mathbb{P}^n_K$. There is a collection $\left\{ C_i \right\}$ of
subvarieties of  $ X \cap Y$ (one of which may be $\phi$),  including
all irreducible components of $X \cap Y$, and intersection numbers,
say $j(X, Y; C_i) \ge 1$, of $X$ and $Y$ along $C_i$ given by the
length of primary ideals, such that  
$$
\deg (X)  \cdot \deg (Y)= \sum\limits_{C_i} j(X, Y; C_i) \cdot \deg
(C_i), 
$$
where we put $\deg (\phi) = 1$.

The key is that our approach does provide an explicit description of
the subvarieties $C_i \subset \mathbb{P}^n$ counted with
multiplicities $j(X, Y; C_i)$, which are canonically determined over
a field extension of $K$. 

In case $\dim (X \cap Y) = \dim X + \dim Y - n$, then our collection
$\left\{ C_i \right\}$ only consists of the irreducible components $C$
of $X \cap Y$ and the multiplicities $j(X, Y; C)$ coincide with Weil's
intersection numbers; that is, our statement also provides the
classical theorem of Bezout. Furthermore, by combining our approach
with the properties of reduced system of parameters, we open the way
to a deeper study of Serre's observations on ``multiplicity'' and
``length'' (see:  J.-P.Serre [72], p.V-20). 

In 1982, W. Fulton asked me how imbedded components contribute to
intersection theory. Using our approach, we are able to study some
pathologies in chapter III. (One construction is due to
R. Achilles). Of course, it would be very interesting to say something
about how imbedded components contribute to intersection
multiplicities. Also, it appears hard to give reasonably sharp
estimates on the error term between deg $(X). \deg (Y)$ and $\sum j
(X, Y; C_j). \deg (C_j)$ or even $\sum \deg (C_j)$ where $C_j$ runs
through all irreducible components of $X \cap Y$. Therefore, we will
discuss some examples, applications and problems in chapter III. 

I wish to express my gratitude to the Tata Institute of Fundamental
Research of Bombay, in particular to Balwant Singh, for the kind
invitation to visit the School of Mathematics. Dilip P. Patil has
written these notes and it is a pleasure for me to thank him for his
efficiency, his remarks and for the time- consuming and relatively
thankless task of writing up these lecture notes. I am also grateful
for the many insightful comments and suggestions made by persons
attending the lectures, including R.C. Cowsik, N. Mohan Kumar,
M.P. Murthy, Dilip P. Patil, Balwant singh, Uwe Storch and
J.-L. Verdier. The typists of the School of Mathematics have typed
these manuscripts with care and I thank them very much. 

Finally I am deeply grateful to R.Sridharan for showing me collected
poems and plays of Rabindranath Tagore.

Let me finish with an example from ``Stray birds'': 

The bird wishes it were a cloud.

The cloud wishes it were a bird.

However, all errors which now appear are due to myself.

\hfill {Wolfgang Vogel}

\begin{center}
\textbf{NOTATION }
\end{center}

The following notation will be used in the sequel.

We denote the set of natural numbers (respectively, non-negative
integers, integers, rational numbers) by $\mathbb{N} (\resp
. \mathbb{Z}^+, \mathbb{Z}, \mathcal{Q}$. For $n \in
\mathbb{N}$, we write $``n >> 1''$ for ``all sufficiently large
integers $n''$. By a ring, we shall always mean a commutative ring
with identity. All ring homomorphisms considered are supposed to be
unitary and, in particular, all modules considered are unitary. If $A$
is a ring, Spec $(A)$ denotes the set of all prime ideals of $A$. For
any ideal $I \subset A$ and any A-module $M$, if $N \subset M $ is an
A-submodule then $(N \underset{M}{:} I): = \{ m \in M | I\cdot m
\subset N \}$. 

For any field $K, \bar{K}$ denotes the algebraic closure of $K$ and
$\mathds{P}^n_K$ denotes the projective n-space over $K$. 
