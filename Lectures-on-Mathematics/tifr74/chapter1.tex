\setcounter{chapter}{-1}
\chapter{Historical Introduction}\label{chap0}

\section{The Classical Case}\label{chap0:sec1} %sec A

The\pageoriginale simplest case of Bezout's theorem over an algebraically closed
field is the following very simple theorem. 

\subsection{Fundamental Principle}\label{chap0:sec1:subsec0.1} %%% 0.1 

The number of roots of a polynomial $f (x)$ in one
  variable, counted with their multiplicities, equals the degree of
$f(x)$. 

This so-called fundamental theorem of algebra was conjectured by
Girard (from the Netherlands) in 1629. In 1799, C.F. Gauss
provided the first proof of this statement. M.Kneser \cite{40} produced a
very simple proof of this fundamental principle in 1981. This proof
also yields a constructive aspect of the fundamental theorem of
algebra. 

The definition of this multiplicity is well-known and clear. Nowadays,
the problem of determining the multiplicity of polynomial root by
machine computation is also considered (see e.g. \cite{101}). 

The second simple case to consider is that of plane curves. The
problem of intersection of two algebraic plane curves is already
tackled by Newton; he and Leibnitz had a clear idea of 'elimination'
process expressing the fact that two algebraic equations in one
variable have a common root, and using such a  process,\pageoriginale Newton
observed in \cite{53} that the abscissas (for instance) of the
intersection points of two curves of respective degrees $m,n$ are
given by an equation of degree $\le m.n$. This result was gradually
improved during the $18^{th}$ century, until Bezout, using a refined
elimination process, was able to prove that, in general, the equation
giving the intersections had exactly the degree $m$. $n$; however, no
general attempt was made during that period to attach an integer
measuring the `multiplicity' of the intersection to each intersection
point in such a way that the sum of multiplicities should always be
$m.n$ (see also \cite{14}). Therefore the classical theorem of Bezout
states that two plane curves of degree $m$  and $n$, intersect in
atmost $m.n$ different points, unless they have infinitely many points
in common. In this form, however, the theorem was also stated by
Maclaurin in his 'Geometrica Organica', published in 1720 (see [48,
  p. 67/68]); nevertheless the first correct proof was given by
Bezout. An interesting fact, usually not mentioned in the literature,
is that:  In 1764, Bezout not only proved the above mentioned
theorem, but also the following n-dimensional version:

\subsection{}\label{chap0:sec1:subsec0.2}%% 0.2
Let $X$ be an algebraic projective
  sub-variety of a projective $n$-space. If $X$ is a complete
  intersection of dimension zero the degree of $X$ is equal to the
  product of the degrees of the polynomials defining $X$. 
 
The\pageoriginale proof can be found in the paper \cite{4}, \cite{5}
and \cite{6}. In his 
book ``on algebraic equations'', published in 1770, \cite{7}, a
statement of this theorem can be found already in the foreword. We
quote from page $XII$: 

`Le degr$\acute{e}$ de $\ell'$ $\acute{e}$quation finale
r$\acute{e}$sultante d'unnombre quelcoque d'$\acute{e}$quations
compl$\acute{e}$tes renfermant un pareil nombre d'inconnues, and de
degr$\acute{e}$s quelconques, est $\acute{e}$gal au produit des
exposants des degr$\acute{e}$s de ces
$\acute{e}$quations. Th$\acute{e}$or$\acute{e}$me dont $\ell$a
verit$\acute{e}$ n'etait connue et d$\acute{e}$montr$\acute{e}$e que
pout deux $\acute{e}$quations seulement.' 

The theorem appears again on page 32 as theorem 47. The special
cases $n = 2,3$ are interpreted geometrically on page 33 in section
$3^0$ and it is mentioned there, that these results are already known
from Geometry. (For these historical remarks, see also (\cite{61}). 

Let us look at projective plane curves $C$ defined by the equation $F
(X_0,X_1,X_2) = 0$ and $D$  defined by the equation $G (X_0,X_1,X_2) =
0$ of degree $n$ and $m$, respectively, without common
components. Then we get 

\subsection{BEZOUT'S Theorem.}\label{chap0:sec1:subsec0.3}%\subsection{0.3}
 $m.n = \sum\limits_P i (C, D; P)$
 where the sum is over all common points $P$ of $C$ and $D$
  and where the positive integer $i(C, D;P)$ is the intersection
  multiplicity\pageoriginale of $C$ and $D$ at $P$. 

We wish to show that this multiplicity is defined, for instance, in
terms of a resultant: 

Given $P$, we may choose our coordinates so that at $P$ we have $X_2 =
1$ and $X_O = X_1 = 0$. By the Preparation Theorem of Weierstrass
\cite{102} (after a suitable change of coordinates ) we can write  
$$
\displaylines{\hfill
  F(X_0,X_1,1) = f' (X_0, X_1)\cdot \bar{f} (X_0,X_1)\hfill\cr  
  \text{and} \hfill
  G(X_0,X_1,1)  = g' (X_0, X_1)\cdot \bar{g} (X_0,X_1)\quad \hfill}
$$
where $f' (X_0, X_1)$ and $g' (X_0, X_1)$ are power series in $X_0$
and $X_1$ such that $f'(0,0) \neq 0 \neq g' (0,0)$ and where 
$$
\displaylines{\hfill
  \bar {f} (X_0, X_1)= X^e_1 + V_1(X_0) X^{e-1}_1 +\cdots + V_e
  (X_0)\hfill \cr
  \text{and}\hfill 
  \bar {g} (X_0, X_1)=  X^\ell_1 + W_1(X_0) X^{\ell-1}_1 +\cdots +
  W_\ell (X_0)\hfill} 
$$
where $V_i(X_0)$ and $W_j(X_0)$ are power series with $V_i(0) = W_j(0)
= 0$. Following Sylvester \cite{86}, we define the $X_1$ resultant of
$\bar{f}$ and $\bar{g}$, denoted by $Res_{X_1} (\bar{f}, \bar{g})$, to
the $(e + \ell) \times (e + \ell)$ determinant  
$$
\left|
\begin{aligned}
1  \quad & V_1 \cdots V_e\\
	 & 1 V_1 \cdots V_e\\
         & \cdots \cdots \cdots \cdots  \\
         & & 1 V_1 \cdots V_e \\
	1  \quad & W_1 \cdots W_\ell \\
	& 1 W_1 \cdots W_\ell\\
	& \cdots \cdots \cdots \cdots  \\
	& & 1  W_1 \cdots W_\ell 
\end{aligned}
\right |
$$
with\pageoriginale zero in all blank spaces.

The application of the Preparation Theorem of Weirstrass, enables us
to get, from the Resultant Theorem (see e.g. \cite{92}), that $Res_X
(\bar{f}, \bar{g}) = 0$. Now $Res_{X_1} (\bar{f}, \bar{g})$ is a power
series in $X_0$ and we define 
$$
i(C,D;P) = X_0-\text{ order of } ~\text{Res}_{X_1} (\bar{f}, \bar{g}).
$$

It is also possible to define the above multiplicity by using the
theory of infinitely near singularities (see, for instance, \cite{1},
ch. $VI$). 

However, Poncelet, as a consequence of his general vaque `Principle of
continuity' given in $1822$, had already proposed to defined the
intersection multiplicity at one point of two subvarieties $U,V$ of
complementary dimensions (see definition below) by having $V$ (for
instance) vary continuously in such a way that for some position $V'$
of $V$ all the intersection points with $U$ should be simple, and
counting the number of these points which collapses to the given point
when $V'$ tended to $V$, in such a way the total number of
intersections (counted with multiplicities) would remain constant
\textit {(`principle of conservation of number')}; and it is thus that
Poncelet proved Bezout's Theorem, by observing that a curve $C$ in a
plane belongs to the continuous family of all curves of the same
degree $m$, and that in that family there exist curves which
degenerate into a system\pageoriginale of straight lines, each meeting a fixed curve
$\Gamma$ of degree $n$ in $n$ distinct points. Many mathematicians in
the 19th century had extensively used such arguments, and in 1912,
Severi had convincingly argued for their essential correctness, see
\cite{73}. 

In view of our exposition below, we wish to mention that the starting
point of $C$. Chevalley's considerations \cite{11}, \cite{12} has been the
observation that the intersection multiplicity at the origin 0 of
two affine curves $f(X,Y) = 0, g(X,Y) = 0$, may be defined to be the
degree of the field extension $K ((X,Y)) \mid K ((f,g))$, where
$K((x,y))$ is the field of quotients of the ring of power series in
$X,Y$ with coefficients in the base field $K$, and where $K((f,g))$ is
the field of quotients of the ring of those power series in $X,Y$
which can be expressed as power series in $f$ and $g$. From there
$C$. Chevalley was led to the definition of multiplicity of a local
ring with respect to a system of parameters, and then to the general
notion of intersection multiplicity. 

The ideal generalization of these observations would be the well-known
theorem of Bezout. First we note that the degree of an algebraic
projective subvariety $V$ of a projective n-space $\mathbb{P}^n_k$
($K$ algebraically closed field), denoted by deg $(V)$, is the number
of points in which almost all linear subspaces $L \subset
\mathbb{P}^n_k$ of dimension $n - d $ meet $X$, where $d$ is the
dimension of $V$. Let $V_1, V_2$\pageoriginale be unmixed varieties of dimensions
$r,s$ and degrees $d,e$ in $\mathbb{P}^n_k$, respectively. Assume
that all irreducible components $V_1 \cap V_2$ have dimension = $r + s
- n$, and suppose that $r + s - n\ge 0$. For each irreducible component
$C$ of $V_1 \cap V_2$, define intersection multiplicity $i (V_1, V_2 ;
C)$ of $V_1$ and $V_2$ along $C$. Then we should have  
$$
\sum_C i (V_1, V_2 ; C).deg (C) = d.e,
$$
where the sum is taken over all irreducible components of $V_1 \cap
V_2$. The hardest part of this generalization is the correct
definition of the intersection multiplicity and, by way, historically it
took may attempts before a satisfactory treatment was given by
$A$. Weil \cite{103} in $1946$. Therefore the proof of Bezout's Theorem
has taken three centuries and a lot of work to master it. 

To get equality in the above equation, one may follow different
approaches to arrive at several different multiplicity theories. At
the beginning of this century, one investigated the notion of the
length of a primary ideal in order to define intersection
multiplicities. This multiplicity is defined as follows: 

Let  $V_1 = V(I_1), V_2 = V(I_2) = V(I_2) \subset \mathbb{P}^n_k $ be
projective varieties defined by homogeneous ideals $I_1, I_2 \subset K
[X_0, \ldots,  X_n]$. Let $C$ be an irreducible component of $V_1 \cap
V_2$. Denote by $A(V_i;C)$ the local ring of $V_i$ at $C$. Then we set 
 $$
 \ell(V_1, V_2 ; C) = \text{ the length of } A(V_1 ; C) /I_2.  A(V_1 ; C).
 $$
 
 For\pageoriginale instance, this multiplicity yields the intersection multiplicity
 as set forth in the beginning for projective plane curves. Furthermore,
 this length provides the ``right'' intersection number for unmixed
 subvarieties $V_1, V_2 \subset \mathbb{P}^n_k$ with $n \le 3$ and
 $\dim (V_1 \cap V_2) = \dim V_1 + \dim V_2 - n$ (see, e.g.,
 W. Gr\"obner \cite{26}). Therefore prior to 1928 most mathematicians
 hoped that this multiplicity yield for Bezout's Theorem the correct
 intersection multiplicity for the irreducible components of two
 projective varieties of arbitrary dimensions (see, e.g. Lasker
 \cite{44}, Macaulay \cite{49}). And, by the way, we want to mention that
 Grobner's papers \cite{26}, \cite{29} are a plea for adoptions of the notion
 of intersection multiplicity which is based on this length of primary
 ideals. He also posed the following problem: 

\setcounter{problem}{3}
\begin{problem}\label{chap0:sec1:prob0.4}
  What are some of the deeper lying reasons that the so-
  called generalized Bezout's Theorem 
  $$
  \deg (V_2 \cap V_2) = \deg (V_1) \cdot \deg (V_2)
  $$
  is not true under certain circumstances ?
\end{problem}

In 1928, B.L.Van der Waerden \cite{90} studied the space curve given
parametrically by $\left\{ s^4, s^3 t, s^3, t^4 \right\}$ to show that
the length 
does not yield the correct multiplicity, in order for Bezout's Theorem
to be valid in projective space $\mathbb{P}^n_k$ with $n \ge
4$\pageoriginale and he has written in [89, p. 770]: 

``In these cases we must reject the notion of length and try to find 
another definition of multiplicity'' (see also [64, p. 100].  

We will study this example (see also \cite{26}, \cite{50}
or \cite{32}). The 
leading coefficient of the Hilbert polynomial of a homogeneous ideal $I
\subset K [X_0, \ldots$,  $X_n]$ will be denoted by $h_0(I)$. Let $V =
V(I)$ be a projective variety defined by a homogeneous ideal $I
\subset K [X_0, \ldots,  X_n]$. Then we have $\deg (V) = h_0(I)$. 

\setcounter{example}{4}
\begin{example}\label{chap0:sec1:exp0.5}
  Let $V_1, V_2$ be the subvarieties of projective space
  $\mathbb{P}^4_k$ with defining prime ideals: 
  \begin{align*}
    \mathscr{Y}_1 & = (X_0 X_3 -X_1 X_2, X^3_1 - X^2_0 X_2, X_0 X^2_2  -
    X^2_1 X\, X_3, X_1 X^2_3 - X^3_2) \\ 
    \mathscr{Y}_2 & = (X_0, X_3)
  \end{align*}
\end{example}

Then $V_1 \cap V_2 = C$ with the defining prime ideal $\mathscr{Y}:
I(C) = (X_0, X_1, X_2, X_3)$. It is easy to see that (see,
e.g. \ref{chap1:sec3:subsec1.42}, (iii) $h_0 (\mathscr{Y}_1) = 4,
h_0(\mathscr{Y}_2) = 1, 
h_0(\mathscr{Y}) = 1$ and therefore $i (V_1, V_2;C) = 4$. Since
$\mathscr{Y}_1 + \mathscr{Y}_2 = (X_0,X_3,X_1 X_2,X^3_1,X^3_2) \subset
(X_0,X_3,X_1X_2,X^2_1,X^3_2) \subset (X_0,X_3,X_1$, $X\,  X^3_2,) \subset
(X_0,X_3,X_1,X^2_2,) \subset(X_0, X_1, X_2, X_3)$, we have $\ell (V_1,
V_2; C) = 5$. Therefore we obtain 

$\deg (V_1). \deg (V_2) = i(V_1, V_2; C). \deg (C) \neq \ell (V_1,
V_2;C) \deg (C)$ 

Nowadays it is well-known that
$$
\ell (V_1, V_2;C) = i(V_1, V_2;C)
$$
if\pageoriginale and only if the local rings $A(V_1, C)$ of $V_1$ at $C$ and $A(V_2,
C)$ of $V_2$ at $C$ are Cohen - Macaulay rings for all irreducible
components $C$ of $V_1 \cap V_2$ where $\dim (V_1 \cap V_2) = \dim V_1
+ \dim V_2 - n$ (see \cite{72}, p. V- 20; see also (3.25)). We assume
again that $\dim(V_1 \cap V_2)=\dim V_1 + \dim V_2-n$. Without loss of
generality, we may suppose, by applying our observations of \S
 \ref{chap2} 
of chapter $I$, that one of the two intersecting varieties $V_1$ and
$V_2$ is complete intersection, say $V_1$. 

Having this assumption, we get that
$$
\ell(V_1, V_2; C) \ge i(V_1,V_2; C)
$$
for every irreducible component $C$ (see also
(\ref{chap3:sec2:lem3.18})). Let $V_2$ be a 
complete intersection. Then there arises another problem posed by
D.A. Buchsbaum \cite{9} in 1965. 

\setcounter{problem}{5}
\begin{problem}\label{chap0:sec1:prob0.6}
  Is it true that $\ell(V_1, V_2 ; C) - i(V_1, V_2 ; C) $ is
  independent of $V_2$, that is, does there exist an invariant $I(A)$,
  of the local ring $A:  = A(V_1 ;C)$ of $V_1$ at $C$ such that  
  $$
  \ell (V_1, V_2 ; C) - i (V_1, V_2 ; C) = I(A) ?
  $$
\end{problem}

This is not the case, however. The first counter - example is given in
\cite{95}. The theory of local Buchsbaum rings started from this negative
answer to the problem of D.A. Buchsbaum. The concept of Buchsbaum
rings was introduced in \cite{82} and \cite{83}, and the theory is now
developing rapidly. The basic underlying idea of\pageoriginale a Buchsbaum ring
continues the well-known concept Cohen-Macaulay ring, its necessity
being created by open questions in Commutative algebra and Algebraic
geometry. For instance, such a necessity to investigate generalized
Cohen Macaulay structure arose while classifying algebraic curves in
$\mathbb{P}^3_k$ or while studying singularities of algebraic
varieties. Furthermore, it was shown by Shiro Goto (Nihon University,
Tokyo) and his colleagues that interesting and extensive classes of
Buchsbaum rings do exist (see, e.g. \cite{23}).  

However, our observations from the Chapter $II$ yield the intersection
multiplicities by the length of well - defined primary ideals. Hence
these considerations again provide the connection between the
different view points which are treated in the work Lasker - Macaulay
- Gr\"obner and Severi - van der Waerden - Weil concerning the
multiplicity theory in the classical case, that is, in case $\dim (V_!
\cap V_2) = \dim V_1 + \dim V_2 - n$. We want to end this
section with some remarks on Buchsbaum's problem. First we give the
following definition: 

\setcounter{definition}{6}
\begin{definition}\label{chap0:sec1:def0.7}
  Let $A$ be a local ring with maximal ideal $\mathfrak{M}$. A
  sequence $\{ a_1, \ldots,  a_r \}$ of elements of is a
  $\mathfrak{M}$ is a \textit { weak A- 
    sequence } if for each $i = 1, \ldots, r$ 
  $$
  \mathfrak{M} \cdot [(a_1, \ldots,  a_{i-1}):  a_i] \subseteq (a_1,
  \ldots,  a_{i-1}) 
  $$
  for $i=1$ we set $(a_1, \ldots,  a_{i-1}) = (0)$ in A).
\end{definition}

 If\pageoriginale every system of parameters of $A$ is a weak A- sequence, we say
 that $A$ is a \textit{Buchsbaum ring}. 

Note that Buchsbaum rings yields a generalization if Cohen Macau\-lay rings.

In connection with Buchsbaum's problem and with our observations
concerning the theory of multiplicities in the paper \cite{82}, we get an
important theorem (see \cite{82}). 

\setcounter{theorem}{7}
\begin{theorem}\label{chap0:sec1:thm8}
  $A$ local ring $A$ is a Buchsbaum ring if and only if the
  difference between the length and the multiplicity of any ideal $q$
  generated by a system of parameters is independent of $q$. 
  
  In order to construct simple Buchsbaum rings and examples which show
  that the above problem is not true in general, we have to state the
  following lemma (see \cite{82}, \cite{87}, or \cite{97}). 
\end{theorem}

\setcounter{lemma}{8}
\begin{lemma}\label{chap0:sec1:lem0.9}
  Let $A$ be a local ring. First we assume that $\dim (A)
  =1$. The following statements are equivalent: 
    \begin{enumerate}[\rm (i)]
    \item A \textit {is a Buchsbaum ring}.
    \item $\mathfrak{M} U ((0)) = (0), \textit { where } U((0))$ \textit{is  the
      intersection of all minimal primary zero ideals belonging to the ideal}
      $(0)$ in $A$. Now, suppose that $\dim (A) > \text{depth}\, (A) \ge 1$ then
      the following statements are equivalent: 
    \item A \textit {is a Buchsbaum ring.}
    \item \textit{There exists a non-zero - divisor}\pageoriginale $x \in \mathfrak{M}^2$ such
      that $A/(x)$ is a Buchsbaum ring. 
    \item \textit{For every non-zero - divisor} $x \in \mathfrak{M}^2$, the ring
      $A/(x)$ is a Buchsbaum ring.
    \end{enumerate}
\end{lemma}
Applying the statements (i), (ii) of the lemma, we get the following
simple examples. 

\setcounter{example}{9}
\begin{example}\label{chap0:sec1:exp0.10}
 Let $K$ be any field
\begin{enumerate}[(1)]
\item We set $A:  = K [[X, Y]]/(X) \cap (X^2, Y)$ then it is not
  difficult to show that $A$ is Buchsbaum non - Cohen - Macaulay
  ring. 
\item We set $A:  = K [[X, Y]]/(X) \cap (X^3, Y)$ then $A$ is not a
  Buchsbaum ring. 

  For the view point of the theory of intersection multiplicities, we
  can construct the following examples by using the statements (iii),
  (iv) of the lemma. 
\item Take the curve $V \subset \mathbb{P}^3_k$ given parametrically
  by $\{ S^5, S^4t, St^4, t^5 \}$. Let $A$ be the local ring of the
  affine cone over $V$ at the vertex, that is, $A = K [X_0, X_1,X_2,
    X_3]_{(X_0, X_1,X_2, X_3)/ \mathscr{Y}_V}$ where $\mathscr{Y}_V =
  (X_0 X_3 -X_1X_2, X^3_0 X_2 - X^4_1, X^2_0 X^2_2 -X^3_1
  X_3,  X_0 X^3_2 - X_1^2X_3^2, X^4_2 - X_1 X^3_3)$. Then $A$ is not a Buchsbaum
  ring (see \cite{62}). We get again this statement from the following
  explicit calculations: 

  Consider the cone $C(V) \subset \mathbb{P}^4_k$ with defining ideal
  $\mathscr{Y}_V$ and the surfaces\pageoriginale $W$ and $W'$ defined by the
  equations $X_0 = X_3 = 0$ and $X_1 = X^2_0 + X^2_3= 0$,
  respectively. It is easy to see that $C(V) \cap W = C (V) \cap W' =
  C$, where $C$ is given by $X_0 = X_2 = X_3 = X_4 = 0$. Some simple
  calculations yield: 

  $\ell(C(V), W;C) = 7, i(C(V), W; C) = 5 $ and $\ell(C(V), W';C) = 13$,

  $i(C(V), W'; C) = 10 $ and hence

  $\ell(C(V), W;C) - i(C(V), W'; C) \neq \ell(C(V), W';C) - i(C(V), W'; C)$.

  Therefore this example shows that the answer to the above problem of
  D.A. Buchsbaum is negative.   
\item Take the curve $V \subset \mathbb{P}^3_K$ given parametrically by
  $\{ s^4, s^3t, st^3, t^4 \}$. Let $A$ be the local ring of the
  affine cone over $V$ at the vertex, that is $A = K [X_0, X_1, X_2,
    X_3]_{(X_0, X_1, X_2, X_3)} \mathscr{Y}_V$, where $\mathscr{Y}_V =
  (X_0 X_3 -X_1X_2, X^2_0X_2 - X^3_1, X_0 X^2_2 - X^2_1 X_3, X_1 X^2_3
  - X^3_2)$. Then $A$ is a Buchsbaum ring (see e.g \cite{83}).  
\end{enumerate} 
\end{example}

\setcounter{remark}{10}
\begin{remark}\label{chap0:sec1:rem0.11}
 This last example has an interesting history. This curve was discovered
 by G. Salmon ([\cite{67}, p. 40]) already in 1849 and a little later in
 1857 by J.Steiner ([\cite{79}, p. 138]) by using the theory of residual
 intersections. This curve was used by F.S.Macaulay ([\cite{49}, p. 98]) in
 1916. His purpose was to show that not every prime ideal in a
 polynomial ring is perfect. In 1928, B.L. Van der Waerden \cite{90}
 studied this example to show that the length of a primary ideal does
 not yield the correct local intersection 
multiplicity\pageoriginale in order Bezout's theorem to be valid in projective space
$ \mathds{P}^n_k$ with $n \geq 4$, and he has written (as cited
already); ``In these cases we must reject the notion of length and try
to find another definition of multiplicity''. As a result, the notion
of intersection multiplicity of two algebraic varieties was put on a
solid base by Van der Waerden for the first time, (see
e.g. \cite{88}, \cite{91}, \cite{92}). We know now that this prime ideal of
$F.S$. Macaulay is not a Cohen-Macaulay ideal, but a Buchsbaum ideal
(i.e., the local ring of $A$ of example (4) is not a Cohen-Macaulay
ring, but is a Buchsbaum ring). This fact motivated us to create a
foundation for the theory of Buchsbaum rings. (For more specific
information on Buchsbaum rings, see also the forthcoming book by
$W$. Vogel with J. St\"uckrad.) 
\end{remark}

\section{The Non-classical Case}\label{chap0:sec2} 

Let $V_1, V_2 \subset  \mathbb{P}^n_k$ be algebraic projective
varieties. The projective dimension theorem states that every
irreducible component of $V_1\cap V_2$ has dimension $\geq \dim
V_1+\dim v_2-n$. Knowing the dimensions of the irreducible components
of $V_1\cap V_2$, we can ask for more precise information about the
geometry of $V_1\cap V_2$. The classical case in the first section
works in case of $\dim V_1\cap V_2 = \dim V_1 +\dim V_2-n$. The
purpose of this section is to study the non-classical case, that is
$\dim V_1 \cap V_2>\dim V_1+\dim V_2-n$ If $V_1, V_2$ are irreducible
varieties,\pageoriginale what can one say about the geometry of $V_1\cap V_2 ?$. A
typical question in this direction was asked by 5. Kleiman:  Is the
number of irreducible components of $V_1\cap V_2$ bounded by the
Bezout's number $\deg (V_1)\cdot\deg  (V_2) ?$ A special case of this
question was studied by C.G.J. Jacobi \cite{36} already in 1836. But we
want to mention that Jacobi's observations relies on a modification of
an idea of Euler \cite{16} from 1748. We would like to describe
Jacobi's observation. 

\setcounter{subsection}{11}
\subsection{JACOBI'S Example}\label{chap:sec2:subsec0.11}

Let $F_1,F_2,F_3$ be three hypersurfaces in $ \mathbb{P}^3_K$. Assume
that the intersection $F_1\cap F_2\cap F_3$ is given by one
irreducible curve, say $C$ and a finite set of isolated points, say
$P_1,\ldots, P_r$. Then $\prod\limits^3_{i=1} \deg (F_i)- \deg (C)
\geq $ number of isolated points of $F_1\cap F_2\cap F_3$. The first
section of this example was given by Salmon and Fielder [68] in
their book on geometry, published in $1874$, by studying the
intersection of $r$ hypersurfaces in $ \mathbb{P}^n_k$. The assumption
is again that this intersection is given by one irreducible curve and
a finite set of isolated points. In 1891, M, Pieri \cite{59} studied
the intersection of two subvarieties, say M. Pieri \cite{59} studied the
intersection of two subvarieties, say $V_1,V_2$ of $ \mathbb{P}^n_K$
assuming that $V_1\cap V_2$ is given by one irreducible component of
dimension $\dim V_1\cap V_2$ and a finite set of isolated points. Also,
it seems that a starting point of an intersection theory in the
non-classical case was discovered by M.Pieri.\pageoriginale In 1947, $56$ years
after M. Pieri, F. Severi \cite{78} suggested a beautiful solution to the
decomposition of Bezout's number $\deg(V_1)\cdot \deg (V_2)$ for any
irreducible subvarieties $V_1,V_2$ of $\mathbb{P}^n_k$. Unfortunately,
Sever i's solution is not true. The first counter- example was given by
$R$. Lazarfeld \cite{45} in 1981. But Lazarfeld also shows how Severi's
procedure can be modified so that it does yield a solution to the
stated problem.

Nowadays, we have a remarkable theory of W. Fulton and R. Mac\-Pherson on
defining algebraic intersection (see, e.g. \cite{18}, \cite{19}). Suppose
$V_1$ and $V_2$ are subvarieties of dimension $r$ and $s$ of a
non-singular algebraic variety $X$ of dimension $n$. Then the
equivalence class $V_1\cdot V_2$ of algebraic $r+s-n$ cycles which
represents the algebraic intersection of $V_1$ and $V_2$ is defined
upto rational equivalence in $X$. This intersection theory produces
subvarieties $W_i$ of $V_1\cap V_2$, cycle classes $\alpha_i$ on $W_i$
positive integers $m_i$ with $\sum m_i \alpha_i $ representing
$V_1\cdot V_2$ and $\deg \alpha_i \ge\deg W_i$ even in the case $\dim
V_1\cap V_2 \neq r+s-n$.

Our object here is to describe the algebraic approach of \cite{85} (see
also \cite{56}) to the intersection theory by studying a formula for
$\deg (V_1)\cdot \deg (V_2)$ in terms of algebraic data, if $V_1$ and
$V_2$ are pure dimensional subvarieties of $\mathbb{P}^n_k$. The basis
of this formula is a method (see \cite{8}, \cite{98}) for expressing the
intersection\pageoriginale multiplicity of two properly intersecting varieties as the length
of a certain primary ideal associated to them in a canonical
way. Using the geometry of the join construction in
$\mathbb{P}^{2n+1}_{\bar{K}}$ over a field extension $\bar {K}$ of $K$
we may apply this method even if $\dim (V_1\cap V_2)> \dim V_1 + \dim
V_2 -n $. The 
key is that algebraic approach provides an explicit description of the
subvarieties $C_i$ and the intersection numbers $j(V_1,V_2; C_i)$
which are canonically determined over a field extension of $K$.
