
\chapter[Applications to Second-Order Problems...]{Applications to
  Second-Order Problems Over Polygonal Domains}\label{chap7}

WE\pageoriginale APPLY THE results of the preceding section in
studying the convergence of the finite element method, i.e. the
convergence of the solutions $u_{h}$ of $(P_{h})$ to the solution $u$
of a problem $(P)$ which corresponds to the choice $V=H^{1}(\Omega)$
or $H^{1}_{0}(\Omega)$, which we saw in Sec.~\ref{chap2} led to
second-order problems.


Let $\Omega$ be a polygonal domain throughout.

\begin{definition}\label{chap7-defi7.1}
A family $(\mathfrak{t}_{h})$ of triangulations of $\Omega$ is regular
is
\begin{itemize}
\item[(i)] for all $\mathfrak{t}_{h}$ and for each
  $K\in\mathfrak{t}_{h}$, the finite elements $(K,\Sigma,P)$ are all
  affine equivalent to a single finite element,
  $(\hat{K},\hat{\Sigma},\hat{P})$ called the reference finite element
  of the family;

\item[(ii)] there exists a constant $\sigma$ such that for all
  $\mathfrak{t}_{h}$ and for each $K\in\mathfrak{t}_{h}$ we have
\begin{equation*}
\frac{h_{K}}{\rho_{K}}\leq \sigma\tag{7.1}\label{chap7-eq7.1}
\end{equation*}
where $h_{K}$, $\rho_{K}$ are as in Theorem~\ref{chap6-thm6.6};

\item[(iii)] for a given triangulation $\mathfrak{t}_{h}$ if
\begin{equation*}
h=\max\limits_{K\in\mathfrak{t}_{h}}h_{K},\tag{7.2}\label{chap7-eq7.2}
\end{equation*}
then $h\to 0$.
\end{itemize}
\end{definition}

\begin{remark}\label{chap7-rem7.1}
The condition (ii) in definition \ref{chap7-defi7.1} assures us that
as $h\to 0$ the triangles do not become ``flat''; cf.\@
Exercise~\ref{chap7-exer7.1}. 
\end{remark}

\begin{exercise}\label{chap7-exer7.1}
If $n=2$ and the sets $K$ are triangles, show that condition (ii) of
definition~\ref{chap7-defi7.1} is valid if and only if there exists
$\theta_{0}>0$ such that for all\pageoriginale $\mathfrak{t}_{h}$ and
$K\in\mathfrak{t}_{h}$, $\theta_{K}\geq \theta_{0}>0$, $\theta_{K}$
being the smallest angle in $K$.
\end{exercise}

\begin{exercise}\label{chap7-exer7.2}
Consider the space $V_{h}$ associate with $\mathfrak{t}_{h}$. Since
$V_{h}$ is finite dimensional all norms are equivalent and hence
$$
|v_{h}|_{0,\infty,\Omega}\leq C_{h}|v_{h}|_{0,\Omega}\quad\text{for
  all}\quad v_{h}\in V_{h},
$$
for some constant $C_{h}$, a priori dependent upon $h$, which we may
evaluate as follows: If $(\mathfrak{t}_{h})$ is a {\em regular family}
of triangulations, show that there exists a constant $C$, independent
of $h$, such that
\begin{equation*}
|v_{h}|_{0,\infty,\Omega}\leq
\frac{C} {h^{n/2}}|v_{h}|_{0,\Omega}\quad\text{for}\quad v_{h}\in
V_{h}.\tag{7.3}\label{chap7-eq7.3} 
\end{equation*}

Also show that there exists a constant $C$ such that
\begin{equation*}
|v_{h}|_{1,\Omega}\leq \frac{C}{h}|v_{h}|_{0,\Omega}\quad\text{for
  all}\quad v_{h}\in V_{h}.\tag{7.4}\label{7.4}
\end{equation*}
\end{exercise}

We now obtain an estimate for the error $||u-u_{h}||_{1,\Omega}$ when
the family of triangulations is regular, which also gives convergence.

\begin{theorem}\label{chap7-thm7.1}
Let $(\mathfrak{t}_{h})$ be a regular family of triangulations on
$\Omega$ of class $C^{0}$ (i.e.\@ $V_{h}\subset
C^{0}(\overline{\Omega})$) with reference finite element
$(\hat{K},\hat{\Sigma},\hat{P})$. We assume that there exists an
integer $k\geq 1$ such that
\begin{itemize}
\item[\rm(i)] $P_{K}\subset\hat{P}\subset H^{1}(\hat{K})$

\item[\rm(ii)] $H^{k+1}(\hat{K})\hookrightarrow C^{s}(\hat{K})$ where
  $s(=0,1,\text{~ or~ }2)$ is the maximal order of derivatives in
  $\hat{\sum}$. 

\item[\rm(iii)] $u\in H^{k+1}(\Omega)$ (Regularity assumption).

Then there exists a constant $C$ (independent of $V_{h}$) such that
\begin{equation*}
||u-u_{h}||_{1,\Omega}\leq
Ch^{k}|u|_{k+1,\Omega}.\tag{7.5}\label{chap7-eq7.5} 
\end{equation*}
\end{itemize}
\end{theorem}

\begin{proof}
Since\pageoriginale $V_{h}\subset C^{0}(\overline{\Omega})$,
$P_{K}\subset H^{1}(K)$, we have $V_{h}\subset V$. By (ii) and (iii)
of the hypothesis we have that the $V_{h}$-interpolate of $u$, viz.\@
$\pi_{h}u$ is well-defined. Since $\pi_{h}u\in V_{h}$, by our
fundamental result (see Theorem~\ref{chap3-thm3.1} or relation
\eqref{chap6-eq6.1}), it suffices to estimate
$||u-\pi_{h}u||_{1,\Omega}$. Now,
\begin{align*}
& H^{k+1}(\hat{K})\hookrightarrow C^{s}(\hat{K}),\\
& H^{k+1}(\hat{K})\hookrightarrow H^{1}(\hat{K})\quad (k\geq 1),\\
& P_{k}\subset P\subset H^{1}(\hat{K}),
\end{align*}
and we may apply Theorem~\ref{chap6-thm6.6} with $p=q=2$, $m=1$ to get
\begin{equation*}
\begin{split}
|u-\pi_{K}u|_{1,K} &\leq C|u|_{k+1,K}\frac{h^{k+1}_{K}}{\rho_{K}}\\
&\leq C|u|_{k+1,K}h^{k}_{K}\quad(\text{Since~ }
\frac{h_{K}}{\rho_{K}}\leq \sigma).
\end{split}\tag{7.6}\label{chap7-eq7.6}
\end{equation*}

Similarly with $m=0$ we get
\begin{equation*}
|u-\pi_{K}u|_{0,K}\leq
C|u|_{k+1,K}h^{k+1}_{K}.\tag{7.7}\label{chap7-eq7.7} 
\end{equation*}

These together give
\begin{equation*}
||u-\pi_{K}u||_{1,K}\leq C
h^{k}_{K}|u|_{k+1,K}.\tag{7.8}\label{chap7-eq7.8} 
\end{equation*}

Now since $h_{K}\leq h$,
\begin{align*}
||u-\pi_{h}u||_{1,\Omega} &=
\left(\sum_{K\in\mathfrak{t}_{h}}||u-\pi_{K}u||^{2}_{1,K}\right)^{\frac{1}{2}}\\
&\leq
C\ h^{k}\left(\sum_{K\in\mathfrak{t}_{h}}|u|^{2}_{k+1,K}\right)^{\frac{1}{2}}\\
&= C\ h^{K}|u|_{k+1,\Omega}.
\end{align*}

This completes the proof.
\end{proof}

\begin{example}\label{chap7-exam7.1}
Consider\pageoriginale triangulations by triangles of type (1). Then $k=1$,
$\hat{P}=P_{1}$ and if $n=2$ or $3$, $H^{2}(\hat{K})\hookrightarrow
C^{0}(\hat{K})$. If $u\in H^{2}(\Omega)$, Theorem \ref{chap7-thm7.1}
says that
$$
||u-u_{h}||_{1,\Omega}\leq C\ h|u|_{2,\Omega}.
$$
\end{example}

We conclude the analysis of convergence in the norm
$||\cdot||_{1,\Omega}$ with the following result.

\begin{theorem}\label{chap7-thm7.2}
Let $(\mathfrak{t}_{h})$ be a regular family of triangulations of
$\Omega$, of class $C^{0}$. Let $s=0$ or $1$ and let $P_{1}\subset
\hat{P}\subset H^{1}(\hat{K})$. Then (with the assumption that $u\in
V=H^{1}(\Omega)$ or $H^{1}_{0}(\Omega)$) we have
\begin{equation*}
\lim\limits_{h\to
  0}||u-u_{h}||_{1,\Omega}=0\tag{7.9}\label{chap7-eq7.9} 
\end{equation*}
\end{theorem}

\begin{proof}
Let $\mathscr{V}=V\cap W^{2,\infty}(\Omega)$. Since $s\leq 1$,
$W^{2,\infty}(\cdot)\hookrightarrow C^{s}(\cdot)$ and
$W^{2,\infty}(\cdot)\hookrightarrow H^{1}(\cdot)$. The second
inclusion follows `a fortiori' from the first with $s=1$. Also,
$P_{1}\subset \hat{P}\subset H^{1}(\hat{K})$. Thus we may apply
Theorem \ref{chap6-thm6.6} with $k=1$, $p=\infty$, $m=1$, $q=2$. Then
for all $v\in\mathscr{V}$,
\begin{align*}
||v-\pi_{K}v||_{1,K} &\leq C(\text{meas~
}K)^{\frac{1}{2}}h|v|_{2,\infty,K}\\
&\leq C(\text{meas~ }K)^{\frac{1}{2}}h|v|_{2,\infty,\Omega}.
\end{align*}

Summing over $K$, we get
\begin{align*}
||v-\pi_{h}v||_{1,\Omega} &\leq
C\ h|v|_{2,\infty,\Omega}\left(\sum_{K\in\mathfrak{t}_{h}}\text{meas~
}K\right)^{\frac{1}{2}}\\
&= C\ h|v|_{2,\infty,\Omega}, 
\end{align*}
since $\sum_{K\in\mathfrak{t}_{h}}$ meas $K=\text{meas~ }\Omega$, a
constant. Thus, for all $v\in\mathscr{V}$, 
\begin{equation*}
\lim\limits_{h\to 0}||v-\pi_{h}v||_{1,\Omega}=0\tag{7.10}\label{chap7-eq7.10}
\end{equation*}\pageoriginale

Notice that $\overline{\mathscr{V}}=V$. Hence choose
$v_{0}\in\mathscr{V}$ such that $||u-v_{0}||_{1,\Omega}\leq
\epsilon/2$ where $\epsilon>0$ is any preassigned quantity. Then
once $v_{0}$ is chosen, by \eqref{chap7-eq7.10} choose $h_{0}$ such
that for all $h\leq h_{0}$, $||v_{0}-\pi_{h}v_{0}||_{1,\Omega}\leq
\epsilon/2$. Now, by \eqref{chap6-eq6.1}
\begin{align*}
||u-u_{h}||_{1,\Omega} &\leq C||u-\pi_{h}v_{0}||_{1,\Omega}\\
&\leq
C\left(||u-v_{0}||_{1,\Omega}+||v_{0}-\pi_{h}v_{0}||_{1,\Omega}\right)\\
&\leq C\epsilon,\text{~ for~ }h\leq h_{0}.
\end{align*}

This gives \eqref{chap7-eq7.9} and completes the proof.
\end{proof}

We now have, by Theorem~\ref{chap7-thm7.1}, $|u-u_{h}|_{0,\Omega}\leq
||u-u_{h}||_{1,\Omega}=0(h^{k})$. We now show, by another argument
that $|u-u_{h}|_{0,\Omega}=0(h^{k+1})$, (at least in some cases) there
by giving a more rapid convergence than expected. This is done by the
{\em Aubin-Nitsche argument} (also known as the {\em duality
  argument}). We describe this in an abstract setting.

Let $V$ be a normed space with norm denoted by $||\cdot||$. Let $H$ be
a Hilbert space with norm $|\cdot|$ and inner product $(\cdot,\cdot)$
such that
\begin{equation*}
\begin{cases}
{\rm(i)}~ V\hookrightarrow H,\quad\text{and}\\
{\rm(ii)}~ \overline{V}=H.
\end{cases}\tag{7.11}\label{chap7-eq7.11}
\end{equation*}

For second-order problems: $V=H^{1}(\Omega)$ or $H^{1}_{0}(\Omega)$
and $H=L^{2}(\Omega)$.

Since $H$ is a Hilbert space, we may identify it with its
dual. Further since $V$ is dense in $H$, we have that $H$ may be
identified with a subspace of $V'$, the dual of $V$. For, if $g\in H$,
define $\tilde{g}\in V'$ by $\tilde{g}(v)=(g,v)\cdot \tilde{g}\in V'$
since $|\tilde{g}(v)|\leq C|g|~||v||$. If $\tilde{g}(v)=0$ for all
$v\in V$, then $(g,v)=0$ for all $v\in H$ as well since
$\overline{V}=H$. Thus $g=0$. This proves the identification. In the
sequel we will\pageoriginale set $g=\tilde{g}$.

Recall that $u$ and $u_{h}$ are the solutions of the problems:
\begin{align*}
& a(u,v)=f(v)\quad\text{for all}\quad v\in V,\tag{$P$}\\
& a(u_{h},v_{h})=f(v_{h})\quad\text{for all}\quad v_{h}\in
  V_{h}\subset V,\tag{$P_{h}$}
\end{align*}
and that the assumptions on $(P)$ are as in the Lax-Milgram
lemma. Then we have the following theorem.

\begin{theorem}\label{chap7-thm7.3}
Let the spaces $H$ and $V$ satisfy \eqref{chap7-eq7.11}. Then with our
above mentioned notations,
\begin{equation*}
|u-u_{h}|\leq M||u-u_{h}||\left[\sup\limits_{g\in
    H}\left\{\frac{1}{|g|}\inf\limits_{\varphi_{h}\in
    V_{h}}||\varphi-\varphi_{h}||\right\}\right],\tag{7.12}\label{chap7-eq7.12} 
\end{equation*}
where for each $g\in H$, $\varphi\in V$ is the corresponding unique
solution of the problem 
\begin{equation*}
a(v,\varphi)=(g,v)\quad\text{for all}\quad v\in V,\tag{$P^{*}$}
\end{equation*}
and $M$ the constant occurring in the inequality giving continuity of
$a(\cdot,\cdot)$. 
\end{theorem}

\begin{remark}\label{chap7-rem7.2}
Note that unlike in $(P)$, we solve for the {\em second} argument of
$a(\cdot,\cdot)$ in $(P^{*})$. This is called the {\em adjoint
  problem} of $(P)$. The existence and uniqueness of the solution of
$(P^{*})$ are proved in an identical manner. Note that if
$a(\cdot,\cdot)$ is symmetric, then $(P)$ is {\em self-adjoint} in the
sense that $(P)=(P^{*})$.
\end{remark}

\begin{proof}
From the elementary theory of Hilbert spaces, we have
\begin{equation*}
|u-u_{h}|=\sup\limits_{\substack{g\in H\\ g\neq
    0}}\frac{|(g,u-u_{h})|}{|g|}.\tag{7.13}\label{chap7-eq7.13} 
\end{equation*}

For a given $g\in H$,
\begin{equation*}
(g,u-u_{\underline{h}})=a(u-u_{\underline{h}},\varphi)\tag{7.14}\label{chap7-eq7.14} 
\end{equation*}

Also\pageoriginale if $\varphi_{h}\in V_{h}$ we have,
\begin{equation*}
a(u-u_{h},\varphi_{h})=0.\tag{7.15}\label{chap7-eq7.15}
\end{equation*}

Thus \eqref{chap7-eq7.14} and \eqref{chap7-eq7.15} give
\begin{equation*}
(g,u-u_{h})=a(u-u_{h},\varphi-\varphi_{h}),\tag{7.16}\label{chap7-eq7.16} 
\end{equation*}
which gives us
$$
|(g,u-u_{h})|\leq M||u-u_{h}||~||\varphi-\varphi_{h}||,
$$
and hence
$$
|u-u_{h}|\leq M||u-u_{h}|\sup\limits_{\substack{g\in H\\ g\neq
    0}}\left(\frac{||\varphi-\varphi_{h}||}{|g|}\right). 
$$
by \eqref{chap7-eq7.13}. Since this is true for any $\varphi_{h}\in
V_{h}$ we may take infimum over $V_{h}$ to get \eqref{chap7-eq7.12},
which completes the proof.
\end{proof}

For dimensions $\leq 3$ and Lagrange finite elements we now show that
$|u-u_{h}|_{0,\Omega}=0(h^{k+1})$. For this we need one more
definition.

\begin{definition}\label{chap7-defi7.2}
Let $V=H^{1}(\Omega)$ or $H^{1}_{0}(\Omega)$, $H=L^{2}(\Omega)$. The
adjoint problem is said to be regular if the following hold:
\begin{itemize}
\item[(i)] for all $g\in L^{2}(\Omega)$, the solution $\varphi$ of the
  adjoint problem for $g$ belongs to $H^{2}(\Omega)\cap V$;

\item[(ii)] there exists a constant $C$ such that for all $g\in
  L^{2}(\Omega)$ 
\begin{equation*}
||\varphi||_{2,\Omega}\leq
C|g|_{0,\Omega},\tag{7.17}\label{chap7-eq7.17} 
\end{equation*}
where $\varphi$ is the solution of the adjoint problem for $g$.
\end{itemize}
\end{definition}

\begin{theorem}\label{chap7-thm7.4}
Let $(\mathfrak{t}_{h})$ be a regular family of triangulations on
$\Omega$ with reference finite element
$(\hat{K},\hat{\Sigma},\hat{P})$. Let $s=0$ and $n\leq 3$. Suppose
there exists an\pageoriginale integer $k\geq 1$ such that $u\in
H^{k+1}(\Omega)$, $P_{k}\subset \hat{P}\subset H^{1}(\hat{K})$. Assume
further that the adjoint problem is regular in the sense of
Definition~\ref{chap7-defi7.2}. Then there exists a constant $C$
independent of $h$ such that
\begin{equation*}
|u-u_{h}|_{0,\Omega}\leq
C\ h^{k+1}|u|_{k+1,\Omega}.\tag{7.18}\label{chap7-eq7.18} 
\end{equation*}
\end{theorem}

\begin{proof}
Since $n\leq 3$, $H^{2}(\cdot)\hookrightarrow C^{0}(\cdot)$. Also,
$H^{2}(\cdot)\hookrightarrow H^{1}(\cdot)$ and $P_{1}\subset
\hat{P}\subset H^{1}(\hat{H})$. Thus for $\varphi\in H^{2}(\Omega)$,
by Theorem~\ref{chap7-thm7.1},
$$
||\varphi-\pi_{h}\varphi||_{1,\Omega}\leq C\ h|\varphi|_{2,\Omega}.
$$

Hence
\begin{equation*}
\inf\limits_{\varphi_{h}\in
  V_{h}}||\varphi-\varphi_{h}||_{1,\Omega}\leq
C\ h|\varphi|_{2,\Omega}.\tag{7.19}\label{chap7-eq7.19} 
\end{equation*}

By \eqref{chap7-eq7.12} and \eqref{chap7-eq7.19}.
$$
|u-u_{h}|_{0,\Omega}\leq M||u-u_{h}||_{1,\Omega}\sup\limits_{g\in
  L^{2}(\Omega)}\left(\frac{1}{|g|_{0,\Omega}}Ch|\varphi|_{2,\Omega}\right). 
$$

By the regularity of $(P^{*})$,
\begin{equation*}
\frac{|\varphi|_{2,\Omega}}{|g|_{0,\Omega}}\leq
\frac{||\varphi||_{2,\Omega}}{|g|_{0,\Omega}}\leq \text{
  constant}.\tag{7.20}\label{chap7-eq7.20} 
\end{equation*}

Thus, $|u-u_{h}|_{0,\Omega}\leq C\ h||u-u_{h}||_{1,\Omega}$
$$
\leq C\ h(h^{k}|u|_{k+1,\Omega})\quad\text{(by theorem~\ref{chap7-thm7.1})}.
$$

This gives \eqref{chap7-eq7.18} and completes the proof.
\end{proof}

We finally give an estimate for the error in the $L^{\infty}$-norm.

\begin{theorem}\label{chap7-thm7.5}
Let $(\mathfrak{t}_{h})$ be a regular family of triangulations on
$\Omega\subset \mathbb{R}^{n}$, where $n\leq 3$. Assume further that for
all $\mathfrak{t}_{h}$ and $K\in \mathfrak{t}_{h}$.
\begin{equation*}
0<\tau\leq \frac{h_{K}}{h}\leq ,\1,\quad \tau~\text{ being a
  constant.}\tag{7.21}\label{chap7-eq7.21} 
\end{equation*}

Let\pageoriginale $u\in H^{2}(\Omega)$ and $P_{1}\subset
\hat{P}\subset H^{1}(\hat{K})\cap L^{\infty}(\hat{K})$. If
$(P^{\ast})$ is regular, then there exists a constant $C$ independent
of $h$ such that
\begin{equation*}
\begin{cases}
|u-u_{h}|_{0,\infty,\Omega}\leq C\ h|u|_{2,\Omega};\text{~ if~ }n=2\\
|u-u_{h}|_{0,\infty,\Omega}\leq C\sqrt{h}|u|_{2,\Omega}\text{~ if~ } n=3.
\end{cases}\tag{7.22}\label{chap7-eq7.22}
\end{equation*}
\end{theorem}

\begin{proof}
Assume $n=2$. Now
\begin{equation*}
|u-u_{h}|_{0,\infty,\Omega}\leq
|u-\pi_{h}u|_{0,\infty,\Omega}+|\pi_{h}u-u_{h}|_{0,\infty,\Omega}.\tag{7.23}\label{chap7-eq7.23} 
\end{equation*}

Note that since $(u_{h}-\pi_{h}u)\in V_{h}$, we may apply
Exercise~\ref{chap7-exer7.1} to get
\begin{equation*}
|u_{h}-\pi_{h}u|_{0,\infty,\Omega}\leq
\frac{C}{h}|u_{h}-\pi_{h}u|_{0,\Omega}.\tag{7.24}\label{chap7-eq7.24} 
\end{equation*}

Thus,
\begin{align*}
|u_{h}-\pi_{h}u|_{0,\infty,\Omega} &\leq
\frac{C}{h}\left[|u_{h}-u|_{0,\Omega}+|u-\pi_{h}u|_{0,\Omega}\right]\\ 
&\leq
\frac{C}{h}\left[C_{1}h^{2}|u|_{2,\Omega}+C_{2}h^{2}|u|_{2,\Omega}\right]\\
&\leq C\ h|u|_{2,\Omega}\quad\text{(by Theorem~\ref{chap7-thm7.4} and
  Theorem~\ref{chap6-thm6.6}).} 
\end{align*}

Also $H^{2}(\cdot)\hookrightarrow C^{0}(\cdot)$;
$H^{2}(\cdot)\hookrightarrow L^{\infty}(\cdot)$ and $P_{1}\subset
\hat{P}\subset L^{\infty}(\hat{K})$. Thus, by
Theorem~\ref{chap6-thm6.6} with $k=1$, $p=2$, $m=0$, $q=\infty$,
$$
|u-\pi_{K}u|_{0,\infty,K}\leq C(\meas K)^{-\frac{1}{2}}h^{2}|u|_{2,K}.
$$

Since $n=2$,
$$
\meas K\geq C\rho^{2}_{K}\geq \frac{C}{\sigma^{2}}h^{2}_{K}\geq
\frac{C\tau^{2}}{\sigma^{2}}h^{2} 
$$
by \eqref{chap7-eq7.1}, so that $(\meas K)^{-\frac{1}{2}}\leq
C\ h^{-1}$ and therefore,
$$
|u-\pi_{K}u|_{0,\infty,K}\leq C\ h|u|_{2,K}.
$$

Hence we obtain \eqref{chap7-eq7.22} for $n=2$ since 
$$
|u-\pi_{h}u|_{0,\infty,\Omega}=\max\limits_{K\in
  \mathfrak{t}_{h}}|u-\pi_{K}u|_{0,\infty,K}\leq C\ h|u|_{2,\Omega}.
$$\pageoriginale

For $n=3$, the only variation in the proof occurs in the fact that
$$
|u_{h}-\pi_{h}u|_{0,\infty,\Omega}\leq
\frac{C}{h^{3/2}}|u_{h}-\pi_{h}u|_{0,\Omega} 
$$
as in Exercise~\ref{chap7-exer7.1} and that now
$$
\meas K\geq C\rho^{3}_{K}\geq \frac{C}{\sigma^{3}}\cdot h^{3}_{K}\geq
\frac{C\tau^{3}}{\sigma^{3}}\cdot h^{3}.
$$

This completes the proof.
\end{proof}

\noindent
{\bf References:}~ One may refer to Ciarlet and Raviart \cite{key6}
for $0(h)$ convergence in the norm $|\cdot|_{0,\infty,\Omega}$ for any
$n$. See also Bramble and Thom\'ee \cite{key1}.

