
\chapter{Interpolation Theory in Sobolev Spaces}\label{chap6}

WE\pageoriginale OUTLINED THE internal approximation method in
Sec.~\ref{chap3}. We are naturally interested in the convergence of
the solutions $u_{h}\in V_{h}$ to the global solution $u\in V$. As a
key step in this analysis we obtained the error estimate (cf.\@
Theorem \ref{chap3-thm3.1}):
\begin{equation*}
||u-u_{h}||\leq C\inf_{v_{h}\in
  V_{h}}||u-v_{h}||.\tag{6.1}\label{chap6-eq6.1} 
\end{equation*}

To be more specific let us consider an example. Given $\Omega\subset
\mathbb{R}^{2}$ a polygon, consider the solution of the following
problem, which is therefore posed in the space $V=H^{1}_{0}(\Omega)$:
\begin{equation*}
\begin{cases}
-\Delta u+au=f\text{~ in~ }\Omega,\\
u=0\text{~ on~ }\Gamma.
\end{cases}\tag{6.2}\label{chap6-eq6.2}
\end{equation*}

Let $\mathfrak{t}_{h}$ be a triangulation of $\Omega$ by triangles of
type (1), (2) or (3). Then $u_{h}\in V_{h}\subset H^{1}_{0}(\Omega)$
and \eqref{chap6-eq6.1} reads as
\begin{equation*}
||u-u_{h}||_{1,\Omega}\leq C\inf\limits_{v_{h}\in
  V_{h}}||u-v_{h}||_{1,\Omega}.\tag{6.3}\label{chap6-eq6.3} 
\end{equation*}

We know `a priori' that $u\in H^{1}_{0}(\Omega)$. Let us assume for
the moment that $u\in C^{0}(\overline{\Omega})$. (Such assumptions are
made possible by the various regularity theorems. For instance, $u\in
H^{2}(\Omega)\subset C^{0}(\overline{\Omega})$ if $f\in L^{2}(\Omega)$
and $\Omega$ is a convex polygon). If $u\in C^{0}(\overline{\Omega})$,
then we may define the $V_{h}$-interpolate of $u$, i.e., $\pi_{h}u$ by
$\pi_{h}u(b_{j})=u(b_{j})$ for the nodes $b_{j}$ of the
triangulation. Note also that $\pi_{h}u|_{K}=\pi_{K}u$ (cf.\@
\eqref{chap5-eq5.5}). Now from \eqref{chap6-eq6.3} we get, 
\begin{align*}
& ||u-u_{h}||_{1,\Omega}\leq C||u-\pi_{h}u||_{1,\Omega}\\
&\qquad
  = C\left[\sum_{K\in\mathfrak{t}_{h}}||u-\pi_{h}u||^{2}_{1,K}\right]^{\frac{1}{2}}\\
&\qquad
  =C\left[\sum_{K\in\mathfrak{t}_{h}}||u-\pi_{K}u||^{2}_{1,K}\right]^{\frac{1}{2}} 
\end{align*}\pageoriginale

{\em Thus the problem of estimating $||u-u_{h}||_{1,\Omega}$ is
  reduced to the problem of estimating $||u-\pi_{K}u||_{1,K}$.} This
is one central problem in the finite element method and motivates the
study of interpolation theory in Sobolev spaces.

We consider more general types of Sobolev spaces for they are no more
complicated for this purpose than those defined in Sec.~\ref{chap2}.

\begin{definition}\label{chap6-defi6.1}
Let $m\geq 0$ be an integer, and $1\leq p\leq +\infty$. Then the
Sobolev space $W^{m,p}(\Omega)$ for $\Omega\subset \mathbb{R}^{n}$,
open, is defined by
$$ 
W^{m,p}(\Omega)=\{v\in L^{p}(\Omega);\ \p^{\alpha}v\in
L^{p}(\Omega)\quad\text{for all}\quad |\alpha|\leq m\}.
$$
\end{definition}

\begin{remark}\label{chap6-rem6.1}
$H^{m}(\Omega)=W^{m,2}(\Omega)$.
\end{remark}

On the space $W^{m,p}(\Omega)$ we define a norm
$||\cdot||_{m,p,\Omega}$ by
\begin{equation*}
||v||_{m,p,\Omega}=\left(\int_{\Omega}\sum_{|\alpha|\leq
  m}|\p^{\alpha}v|^{p}dx\right)^{1/p}\tag{6.4}\label{chap6-eq6.4} 
\end{equation*}
and the semi-norm $|\cdot|_{m,p,\Omega}$ by
\begin{equation*}
|v|_{m,p,\Omega}=\left(\int_{\Omega}\sum_{|\alpha|=m|}|\p^{\alpha}v|^{p}dx\right)^{1/p}\tag{6.5}\label{chap6-eq6.5} 
\end{equation*}

If $k\geq 1$ is an integer, consider the space
$W^{k+1,p}(\Omega)/P_{k}$. If $\dot{v}$ stands for the equivalence
class of $v\in W^{k+1,p}(\Omega)$ we may define the analogues of
\eqref{chap6-eq6.4} and \eqref{chap6-eq6.5} respectively by
\begin{equation*}
||\dot{v}||_{k+1,p,\Omega}=\inf\limits_{p\in
  P_{k}}||v+p||_{k+1,p,\Omega}\tag{6.6}\label{chap6-eq6.6} 
\end{equation*}
and\pageoriginale
\begin{equation*}
|\dot{v}|_{k+1,p,\Omega}=|v|_{k+1,p,\Omega}.\tag{6.7}\label{chap6-eq6.7}
\end{equation*}

These are obviously well-defined and $||\cdot||_{k+1,p,\Omega}$
defines the quotient norm on the quotient space above. We then have
the following key result, whose proof may be found in Ne\v{c}as
\cite{key20} for instance.

\begin{theorem}\label{chap6-thm6.1}
In $W^{k+1,p}(\Omega)/P_{k}$, the semi-norm $|\dot{v}|_{k+1,p,\Omega}$
is a norm \break equivalent to the quotient norm
$||\dot{v}||_{k+1,p,\Omega}$, i.e., there exists a constant
$C=C(\Omega)$ such that for all $\dot{v}\in W^{k+1,p}(\Omega)/P_{k}$ 
\begin{equation*}
|\dot{v}|_{k+1,p,\Omega}\leq ||\dot{v}||_{k+1,p,\Omega}\leq
C|\dot{v}|_{k+1,p,\Omega}.\tag{6.8}\label{chap6-eq6.8} 
\end{equation*}
\end{theorem}

Equivalently, we may state

\begin{theorem}\label{chap6-thm6.2}
There exists a constant $C=C(\Omega)$ such that for each $v\in
W^{k+1,p}(\Omega)$ 
\begin{equation*}
{\displaystyle{\mathop{\text{inf.}}_{p\in
      P_{k}}}}||v+p||_{k+1,p,\Omega}\leq
C|v|_{k+1,p,\Omega}.\tag{6.9}\label{chap6-eq6.9} 
\end{equation*}
\end{theorem}
\noindent
(Note: This result holds if $\Omega$ has a continuous boundary and if
it is bounded so that $P_{k}\subset W^{k+1,p}(\Omega)$.)

We now prove the following

\begin{theorem}\label{chap6-thm6.3}
Let $W^{k+1,p}(\Omega)$ and $W^{m,q}(\Omega)$ be such that
$W^{k+1,p}(\Omega)\hookrightarrow W^{m,q}(\Omega)$ (continuous
injection). Let
$\pi\in\mathscr{L}(W^{k+1},\rho(\Omega),W^{m,q}(\Omega))$, i.e. a
continuous linear map, such that for each $p\in P_{k}$, $\pi
p=p$. Then there exists $C=C(\Omega)$ such that for each $v\in
W^{k+1,p}(\Omega)$ 
\begin{equation*}
|v-\pi v|_{m,q,\Omega}\leq
C||I-\pi||_{\mathscr{L}(W^{k+1},P(\Omega),W^{m,q}(\Omega))}|v|_{k+1,p,\Omega} 
\end{equation*}\pageoriginale
\end{theorem}

\begin{proof}
For each $v\in W^{k+1,p}(\Omega)$ and for each $p\in P_{k}$, we can
write 
$$
v-\pi v=(I-\pi)(v+p).
$$

Thus,
\begin{align*}
|v-\pi v|_{m,q,\Omega} &\leq ||v-\pi v||_{m,q,\Omega}\\
&=
  ||I-\pi||_{\mathscr{L}(W^{k+1,p}(\Omega),W^{m,q}(\Omega))}||v+p||_{k+1,p,\Omega}, 
\end{align*}
for all $p\in P_{K}$. Hence,
\begin{align*}
|v-\pi v|_{m,q,\Omega} &\leq
||I-\pi||_{\mathscr{L}(W^{k+1,p}(\Omega),W^{m,q}(\Omega))}\inf\limits_{p\in
P_{k}}||v+p||_{k+1,p,\Omega}\\ 
&\leq
|C||I-\pi||_{\mathscr{L}(W^{k+1,p}(\Omega),W^{m,q}(\Omega))}|v|_{k+1,p,\Omega} 
\end{align*}

By theorem \ref{chap6-thm6.2}, this completes the proof.
\end{proof}

\begin{definition}\label{chap6-defi6.2}
Two open subsets $\Omega$, $\hat{\Omega}$ of $\mathbb{R}^{n}$ are said
to be affine equivalent if there exists an invertible affine map $F$
mapping $\hat{x}$ to $B\hat{x}+b$, $B$ an invertible $(n\times n)$
matrix and $b\in \mathbb{R}^{n}$, such that $F(\hat{\Omega})=\Omega$. 
\end{definition}

If $\Omega$, $\hat{\Omega}$ are affine equivalent, then we have a
bijection between their points given by $\hat{x}\leftrightarrow
x=F(\hat{x})$. Also we have bijections between smooth functions on
$\Omega$ and $\hat{\Omega}$ defined by $(v:\Omega\to
\mathbb{R})\leftrightarrow (\hat{v}:\hat{\Omega}\to \mathbb{R})$ where
$v(x)=\hat{v}(\hat{x})$. 

The following theorem gives estimates of $|v|_{m,p,\Omega}$ and
$|\hat{v}|_{m,p,\hat{\Omega}}$ each in terms of the other.

\begin{theorem}\label{chap6-thm6.4}
Let $\Omega$, $\hat{\Omega}\subset \mathbb{R}^{n}$ be affine
equivalent. Then there exist\pageoriginale constants $C$, $\hat{C}$
such that for all $v\in W^{m,p}(\Omega)$
\begin{equation*}
|\hat{v}|_{m,p,\hat{\Omega}}\leq C||B||^{m}||\det
B|^{-1/p}|v|_{m,p,\Omega}\tag{6.11}\label{chap6-eq6.11} 
\end{equation*}
and for all $\hat{v}\in W^{m,p}(\hat{\Omega})$
\begin{equation*}
|v|_{m,p,\Omega}\leq C||B^{-1}||^{m}|\det
B|^{1/p}|\hat{v}|_{m,p,\hat{\Omega}}.\tag{6.12}\label{chap6-eq6.12} 
\end{equation*}
\end{theorem}

\noindent
{\bf Note:}
\begin{itemize}
\item[(i)] It suffices to prove either \eqref{chap6-eq6.11} or
  \eqref{chap6-eq6.12}. We get the other by merely interchanging the
  roles of $\Omega$ and $\hat{\Omega}$. We will prove the former.

\item[(ii)] $||B||$ is the usual norm of the linear transformation
  defined by $B$, viz.\@
  $||B||=\sup\limits_{\substack{x\in\mathbb{R}^{n}\\ x\neq
      0}}\dfrac{||Bx||}{||x||}$. (Recall that
  $F(\hat{\Omega})=\Omega$, $F(\hat{x})=B\hat{x}+b$). 
\end{itemize}

\begin{proof}
Let $\{e_{1},\ldots,e_{n}\}$ be the standard basis for
$\mathbb{R}^{n}$. Let $\alpha$ be a multi-index with $|\alpha|=m$. By
choosing a suitable collection $\{e_{1\alpha},\ldots,e_{m\alpha}\}$
with appropriate number of repetitions from the basis, we may write,
$$
(\p^{\alpha}\hat{v})(\hat{x})=(D^{m}\hat{v})(\hat{x})(e_{1\alpha},\ldots,e_{m\alpha}), 
$$
where $D^{m}\hat{v}$ is the $m^{\text{th}}$ order Fr\'echet derivative
of $\hat{v}$ and $D^{m}\hat{v}(\hat{x})$ is consequently an $m$-linear
form on $\mathbb{R}^{n}$. Thus,
$$
|\p^{\alpha}\hat{v}(\hat{x})|\leq
||D^{m}\hat{v}(\hat{x})||=\sup\limits_{\substack{||\xi_{i}||=1\\ 1\leq
    i\leq m}}|D^{m}\hat{v}(\hat{x})(\xi_{1},\ldots,\xi_{m})|. 
$$

Since this is true for all $|\alpha|=m$, we get
\begin{equation*}
|\hat{v}|_{m,p,\hat{\Omega}}\leq
C_{1}\left(\int_{\hat{\Omega}}||D^{m}\hat{v}(\hat{x})||^{p}d\hat{x}\right)^{1/p}\leq
C_{2}|\hat{v}|_{m,p,\hat{\Omega}}.\tag{6.13}\label{chap6-eq6.13}
\end{equation*}

The first inequality is a consequence of our preceding argument. The
second follows by a straightforward argument. By composition of
functions in differentiation:
\begin{equation*}
D^{m}\hat{v}(\hat{x})(\xi_{1},\ldots,\xi_{m})=D^{m}v(x)(B\xi_{1},\ldots,B\xi_{m}); \tag{6.14}\label{chap6-eq6.14}
\end{equation*}

This\pageoriginale gives
\begin{equation*}
||D^{m}\hat{v}(\hat{x})||\leq
||D^{m}v(x)||~||B||^{m}.\tag{6.15}\label{chap6-eq6.15} 
\end{equation*}

Hence the first inequality in \eqref{chap6-eq6.13} may be rewritten as
\begin{align*}
|\hat{v}|^{p}_{m,p,\hat{\Omega}} &\leq
C^{p}_{1}||B||^{mp}\int_{\hat{\Omega}}||D^{m}v(F(\hat{x}))||^{p}d\hat{x}\\ 
&= C^{p}_{1}||B||^{mp}|\det B|^{-1}\int_{\Omega}||D^{m}v(x)||^{p}dx\\
&\leq C^{p}||B||^{mp}|\det B|^{-1}|v|_{m,p,\Omega}.
\end{align*}
by an inequality similar to the second inequality of
\eqref{chap6-eq6.13}. Raising to power $1/p$ on either side we get
\eqref{chap6-eq6.11}. This completes the proof.
\end{proof}

We now estimate the norms $||B||$ and $||B^{-1}||$ in terms of the
`sizes' of $\Omega$ and $\hat{\Omega}$. More precisely, if $h$,
(resp.\@ $\hat{h}$) the supremum of the diameters of all balls that
can be inscribed in $\Omega$, (resp.\@ $\hat{\Omega}$), we have the
following: 

\begin{theorem}\label{chap6-thm6.5}
$||B||\leq h/\hat{\rho},\quad\text{and}\quad ||B^{-1}||\leq \hat{h}/\rho$.
\end{theorem}

\begin{proof}
Again it suffices to establish one of these. Now,
$$
||B||=\sup\limits_{||\xi||=\hat{\rho}}\left(\frac{1}{\hat{\rho}}||B\xi||\right). 
$$

Let $\xi\in\mathbb{R}^{n}$ with $||\xi||=\hat{\rho}$. Choose
$\hat{y}$, $\hat{z}\in\overline{\hat{\Omega}}$ such that
$\xi=\hat{y}-\hat{z}$. Then $B\xi=B\hat{y}-B\hat{z}=y-z$, where
$F(\hat{y})=y$, $F(\hat{z})=z$. But $y$, $z\in\overline{\Omega}$ and
hence $||y-z||\leq h$. Thus $||B\xi||\leq h$. Hence $||B||\leq
h/\hat{\rho}$, which completes the proof.
\end{proof}

We conclude this section with an {\em important, often used, result.}

\begin{theorem}\label{chap6-thm6.6}
Let $(\hat{K},\hat{\Sigma},\hat{P})$ be a finite element. Let
$s(=0,1\text{~ or~ }2)$ be the maximal order of derivatives occurring
in $\hat{\Sigma}$. Assume that: 
\begin{itemize}
\item[(i)] $W^{k+1,p}(\hat{K})\hookrightarrow
  C^{s}(\hat{K})$\pageoriginale

\item[(ii)] $W^{k+1,p}(\hat{K})\hookrightarrow W^{m,q}(\hat{K})$

\item[(iii)] $P_{k}\subset \hat{P}\subset W^{m,q}(\hat{K})$
\end{itemize}

Then there exists a constant $C=C(\hat{K},\hat{\Sigma},\hat{P})$ such
that for all affine equivalent finite elements $(K,\Sigma,P)$ we have
\begin{equation*}
|v-\pi_{K}v|_{m,q,K}\leq C(\text{meas~}K)^{\frac{1}{q}-\frac{1}{p}}\frac{h^{k+1}_{K}}{\rho^{m}_{K}}|v|_{k+1,p,K}\tag{6.16}\label{chap6-eq6.16}
\end{equation*}
for all $v\in W^{k+1,p}(K)$, where $h_{K}$ is the diameter of $K$ and
$\rho_{K}$ is the supremum of diameters of all balls inscribed in $K$.
\end{theorem}

\begin{proof}
Since $P_{k}\subset \hat{P}$, for any polynomial $p\in P_{k}$ we have
$\hat{\pi}p=p$. We may write
\begin{equation*}
\begin{split}
\hat{\pi}\hat{v} &=
\sum_{i}\hat{v}(\hat{a}^{0}_{i})\hat{p}^{0}_{i}+\sum_{i,k}(D\hat{v}(\hat{a}^{1}_{i})(\hat{\xi}^{1}_{i,k}))\hat{p}^{1}_{i,k}\\
&= +\sum_{i,k,l}(D^{2}\hat{v}(\hat{a}^{2}_{i})(\hat{\xi}^{2}_{i,k},\hat{\xi}^{2}_{i,l}))\hat{p}^{2}_{i,k,l},
\end{split}.\tag{6.17}\label{chap6-eq6.17} 
\end{equation*}
all these sums being finite (the second and third may or may not be
present).
We claim that $\hat{\pi}\in\mathscr{L}(W^{k+1},
p(\hat{K}),W^{m,q}(\hat{K}))$. Since $\hat{P}\subset W^{m,q}(\hat{K})$
all the basis functions in \eqref{chap6-eq6.17} are in
$W^{m,q}(\hat{K})$. Thus,
\begin{equation*}
\begin{split}
||\hat{\pi}\hat{v}||_{m,q,\hat{K}} &\leq
\sum_{i}|\hat{v}(\hat{a}^{0}_{i})|~||\hat{p}^{0}_{i}||_{m,q,\hat{K}}\\
&\quad
+\sum_{i,k}|D\hat{v}(\hat{a}^{1}_{i})(\xi^{1}_{i,k})|~||\hat{p}^{1}_{ik}||_{m,q,\hat{K}}\\
&\quad +\sum_{i,k,l}|D^{2}\hat{v}(\hat{a}^{2}_{i})(\hat{\xi}^{2}_{i,k},\hat{\xi}^{2}_{i,l})|~||\hat{p}^{2}_{ikl}||_{m,q,\hat{K}}
\end{split}\tag{6.18}\label{chap6-eq6.18}
\end{equation*}

Since $W^{k+l,p}(\hat{K})\hookrightarrow C^{s}(\hat{K})$ and all the
numbers $\hat{v}(\hat{a}^{0}_{i})$, etc\ldots, are bounded by their
essential supremum over $\hat{K}$, 
$$
||\hat{\pi}\hat{v}||_{m,q,\hat{K}}\leq C||\hat{v}||_{k+1,p,\hat{K}}.
$$\pageoriginale

Hence the claim is valid. Now by virtue of (ii) and also our
observation on preservation of polynomials, we may apply theorem
\ref{chap6-thm6.3} to $\hat{\pi}$. Hence there exists
$C=C(\hat{K},\hat{\Sigma},\hat{P})$ such that
$$
|\hat{v}-\hat{\pi}\hat{v}|_{m,q,\hat{K}}\leq
C|\hat{v}|_{k+1,p,\hat{K}}\quad\text{for}\quad \hat{v}\in
W^{k+1,p}(\hat{K}). 
$$

Notice that $\hat{\pi}\hat{v}=\widehat{\pi_{K}v}$ by
Theorem~\ref{chap5-thm5.1}. Thus
$\hat{v}-\hat{\pi}\hat{v}=\widehat{v-\pi_{K}v}$. Thus if
$F_{K}(\hat{K})=K$ where $F_{K}(\hat{x})=B_{K}\hat{x}+b_{K}$, we get
\begin{equation*}
|v-\pi_{K}v|_{m,q,K}\leq C_{1}||B^{-1}_{K}||^{m}|\det
B_{K}|^{1/q}|\hat{v}-\hat{\pi}\hat{v}|_{m,q,\hat{K}},\tag{6.19}\label{chap6-eq6.19} 
\end{equation*}
by Theorem~\ref{chap6-thm6.4}. Also by the same theorem
\begin{equation*}
|\hat{v}|_{k+1,p,\hat{K}}\leq C_{2}||B_{K}||^{k+1}|\det
B_{K}|^{-1/p}|v|_{k+l,p,K}.\tag{6.20}\label{chap6-eq6.20} 
\end{equation*}

Further $|\det B_{K}|$ being the Jacobian of the transformation, we
have $|\det B_{K}|=\dfrac{\text{meas~ }K}{\text{meas~}\hat{K}}$ and
$||B_{K}||\leq \dfrac{h_{K}}{\hat{\rho}}$, $||B^{-1}_{K}||\leq
\hat{h}/\rho_{K}$ by theorem \ref{chap6-thm6.5}. Since $\hat{h}$,
$\hat{\rho}$, meas $\hat{K}$ are constants, combining
\eqref{chap6-eq6.19}, \eqref{chap6-eq6.20} and the preceding
observation we complete the proof of the theorem.
\end{proof}

\noindent
{\bf References:}~See Bramble and Hilbert \cite{key28}, Bramble and
Zl\'qmal \cite{key2}, Ciarlet and Raviart \cite{key7}, Ciarlet and
Wagschal \cite{key8}, Strang \cite{key21}, \v{Z}eni\v{s}ek
\cite{key23,key24} and Zl\'amal \cite{key25,key32}. 

