\chapter{Linear Diffusion}%chap 6

We\pageoriginale recall the definition of a diffusion. A strong Markov
process whose 
path functions are continuous before the killing time is called a
diffision. In this section we develop the theory (due to Feller) of
linear diffusion. 

\section{Generalities}\label{chap6-sec1}%sec 1.
\begin{defi*}
  A diffusion whose state space $S$ is a linear connected set is
  called  a {\em linear diffusion}. 
\end{defi*}

$S$ is therefore one of the following sets, upto isomorphism
i.e., order preserving homeomorphism of linear connected sets:   
$$
(1)~ [0, 1],\quad (2)~[0, 1),\quad  (3)~(0, 1],\quad (4)~(0, 1), \quad
(5)~\{ 0 \}. 
$$

Let $\sigma_b$ denote the first passage time for $b$, i.e. $\sigma_b
= \inf \{t : x_t = b \}$. If $P_a (\sigma_b < \infty ) > 0 $, we write
$a \rightarrow b;$ if $a \rightarrow b$ for some $b > a$, we write $a
\in C_+ $; if $a \rightarrow b$ for some $b < a$, we write $a \in
C_-$. If $a \nrightarrow b $ for any $b > a$, i.e. if $a \notin C_+$
we write $a \in K_-$; similarly if $a \nrightarrow b$ for any $b<a$,
i.e. if $a \not\in C_{-}$, we wrire $a \in K_+$. Thus if $a \in K_+$ and
$b<a$  then $P_a (\sigma_b = \infty) =1$, i.e. $P_a(x_t \geq a$ for
all $t< \sigma_\infty ) =1$. 

Every point of the state space $S$ belongs to one of the following sets:
\begin{enumerate}
\item $C_+ \cap C_- = K^c_+ \cap K^c_-$.\pageoriginale These points are called
  \textit{regular points} or \textit{second order points}. 

\item $C_+ - C_-= K_+-K_-$. A point of this set is called a
  \textit{pure right shunt}. 

\item $C_- -C_+ = K= K_+$. A point of this set is called a
  \textit{pure left Shunt}. Both left and right shunts are sometimes
  called \textit{points  of first order}, i.e. a point of first order
  is an element of $C_+ - C_{-}) \cup (C_{-}- C_+)$. 

\item $C^c_- \cap C^c_+ = K_- \cap K_+$. These points are called
  \textit{trap points} or \textit{points of order zero}. 
\end{enumerate}

The intuitive meanings of the above should be clear; for instance of
particle starting at a pure right shunt travels to the right with
probability~1.

\setcounter{thm}{0} 
\begin{thm}\label{chap6-sec1-thm1}%them 1.
If $a \in C_+ (\in c_-)$, then $P_a(\sigma_{a_+}=0)=1 (P_a
  (\sigma_{a_-} = 0) = 1)$, where $\sigma_{a_+} = \inf \{t : x_t > a
  \} (\sigma_{a_-} = \inf \{t : x_t < a \})$. 
\end{thm}

\begin{proof}
Let $\sigma= \sigma_{a_+}$ and $a \in C_+$. There exists $b>a$, and
  $t$ such that $P_a (\sigma_b < t) > 0$. Now $E_a (d^{- \sigma_b})
  \geq e^{-t} P_a (\sigma_b < t) > 0$. Since $\sigma_b(w) = \sigma(w)
  + \sigma_b (w^+_\sigma)$ we have, by the strong Markov property, 
  \begin{align*}
    0 < E_a (e^{- \sigma_b}) &= E_a (e^{- \sigma_b}: \sigma_b < \infty )
    = E_a (e^{- \sigma_b}: \sigma < \infty , \sigma_b < \infty \\ 
    &= E_a (e^{- \sigma - \sigma_b (w^+_\sigma)}: \sigma < \infty ,
    \sigma_b (w^+_\sigma ) < \infty )\\ 
    & =  E_a [e^{- \sigma}
      E_{x_\sigma} (e^{- \sigma_b}: \sigma_b < \infty ): \sigma <
      \infty ] \\ 
    &= E_a (e^{-\sigma} E_a (e^{- \sigma_b}): \sigma < \infty ) = E_a (e^{-
      \sigma_b}) E_a(e^{-\sigma}). \tag*{Q.E.D} 
  \end{align*}
\end{proof}

\begin{remark*}
  If $\mathbb{M}$ is not strong Markov the theorem is not true
  (e.g.\pageoriginale exponential holding time process). 
\end{remark*}

\begin{thm}\label{chap6-sec1-thm2}%2
If $a \rightarrow b > a (< a)$ then $[a, b)  \subset C_+ ((b, a] \subset C_-$.
\end{thm}

\begin{proof}
Let $a \leq \xi < b$. Then $P_a(\sigma_\xi < \sigma_b) =1$, because
the paths are continuous. We have 
\begin{gather*}
    0 < P_a (\sigma_b < \infty ) = P_a (\sigma_\xi < \infty ,\sigma_b
    < \infty) = P_a(\sigma_\xi < \infty , \sigma_b (w^+_{\sigma_\xi})
    < \infty) \\ 
    = E_a [P_{x_{\sigma_\xi}}	(\sigma_b < \infty ): \sigma_\xi <
      \infty = P_\xi [\sigma_b < \infty ] P_a (\sigma_\xi < \infty ), 
  \end{gather*}
 since by continuity $x (\sigma_\xi) = \xi$. Therefore $P_\xi
 (\sigma_b < \infty ) > 0$. 
Q.E.D.
\end{proof}

\begin{coro*}
If $a \in C_+$ then some right neighbourhood (i.e. a set which
contains an interval [a, b )) of  $a, U_+ (a) \subset C_+$. 
\end{coro*}

\begin{thm}\label{chap6-sec1-thm3}% them 3.
The set of regular points is open.
\end{thm}

\begin{proof}
 Since $a \in C_+$, there exists $b > a $ with $P_a (\sigma_b <
  \infty) > 0$ and since $a \in C_-$, by Theorem
  \ref{chap6-sec1-thm1}, $P_a (\sigma_{a_-} 
  = 0) = 1$. Hence $P_a (\sigma_{a_-} < \sigma_b < \infty)> 0$; this
  implies that there exists $c < a$ such that $P_a (\sigma_c <
  \sigma_b < \infty)> 0$. Noting that $a \in C_+$ and using Theorem
  \ref{chap6-sec1-thm1}, there exists $d, a < d < b$, with $P_a
  (\sigma_d < \sigma_c < \sigma_d < \infty) > 0$. Using the strong
  Markov property  
  $$
  P_a (\sigma_d < \infty ) P_d (\sigma_c < \infty) P_c (\sigma_d < \infty ) > 0,
  $$
  so that $P_d (\sigma_c < \infty) > 0, P_c(\sigma_b < \infty )>
  0$. Hence $(c, d] \subset C_-$ and $[c, b) \in C_+$. \qquad Q.E.D.  
\end{proof}

\begin{thm}\label{chap6-sec1-thm4}%them 4
  $K_+$ is right closed, i.e. $a_n \in K_+, a_n \uparrow a $ imply
  $a \in K_+ (k_-$ is left closed). 
\end{thm}

\begin{proof}
If $a \notin K_+$ then $a \in c_-$. There exists $b < a$ and $a
\rightarrow b$.  

Then\pageoriginale $(b, a] \subset c_- $ so that $ (b, a] \cap K_+ =
    \phi$. \qquad Q.E.D. 
\end{proof}

\section{Generator in the restricted sense}\label{chap6-sec2}%sec 2.

In the section of strong Markov processes we introduced a generator in
the restricted sense; we modify this to suit our special
requirements. Let $\mathscr{D}(S) = \{ f : f \in \ \mathbb{B} (S)$ and
$f$ is right continuous at every point of $C_+$ and left continuous at
every point of $C_-$. $\mathscr{D}(S)$ is smaller than the classes
$D(S)$ introduced before in the section of strong Markov
processes. Clearly $f \in \mathscr{D}(S)$ is continuous at every
regular point and $\mathscr{D}(S) \supset C (S)$. 

\setcounter{thm}{0}
\begin{thm}\label{chap6-sec2-thm1}%them 1.
  $\mathscr{D}(S) \supset G_\alpha \beta (S)$; a fortiori $G_\alpha
  \mathscr{D} (S) \subset \mathscr{D}(S)$. 
\end{thm}

\begin{proof}
Let $a \in C_+$. Then 
\begin{align*}
G_\alpha f(a) &= E_a \left( \int ^\infty_0 e^{- \alpha t} f (x_t)dt\right)\\
&= E_a \left(\int^{\sigma _b}_0 e^{- \alpha t} f(x_t) dt\right) + E_a
\left(\int^{\infty}_{\sigma_b}e^{- \alpha t} f (x_t)dt\right)\\ 
& = E_a
    \left(\int^{\sigma_b}_0 e^{- \alpha t} f (x_t)dt \right) + E_a (e^{- \alpha
      \sigma_b} G_\alpha f(x_{\sigma_b}))\\ 
    &= E_a \left(\int^{\sigma_b}_0 e^{-\alpha t} f(x_t) dt\right) + E_a (e^{-
    \alpha \sigma_b}) G_\alpha f(b). 
  \end{align*}

  Now 
  $$
  | E_a \left(\int^{\sigma_b}_0 e^{\alpha t}
    f(x_t)dt\right)| \leq || f || 
    \frac{1-E(e^{- \alpha \sigma_b})}{\alpha} \xrightarrow[b \rightarrow
      a]{} ||f||  \frac{1-E(e^{- \alpha
        \sigma_{a_+}})}{\alpha}=0,
$$
since
$$
P_a(\sigma_{a_+}=0 )=1.
$$
Q.E.D


We can prove that $G_\alpha \mathscr{D}(S)$ is independent of $\alpha$
and the other results easily. 
\end{proof}

\begin{thm}\label{chap6-sec2-thm2}%2
$G_\alpha f=0$,\pageoriginale $f \in \mathscr{D}(S)$ imply $f \equiv 0$.
\end{thm}

\begin{proof}
It is enough to show that $P_a(f(x_t) \rightarrow f(a)$ as $t - 0) = 1$.

If a is regular, $f$ is continuous at a and there is nothing to
prove. If $a \in C_+ - C_- , f$ is right continuous at a and $P_a (x_t
\geq a$ for 0 $\leq t < \sigma_\infty ) = 1$ and again the result is
immediate. If $a \in C_- - C_+$ the same is true. If a is a trap $P_a
(x_t=a$ for $0 \le t < \sigma_\infty ) = 1$ and since
$P_a(\sigma_\infty > 0)=1$ (because $P_a (x_0 =a )=1)$ the result
follows again. 
\end{proof}

\begin{defi*}
We define the generator in the restricted sense as $\mathscr{G}u =
\alpha u - f$ where $u = G_\alpha f$ with $f \in
\mathscr{D}(S)$. One easily verifies that $\mathscr{G} u$ is 
independent of $\alpha$. 
\end{defi*}

\begin{thm}\label{chap6-sec2-thm3}%3
If a is a trap, then $P_a (\sigma_\infty > t) \equiv P_a (\tau_a >
t)= e^{-kt}$ and $\mathscr{G}u(a)=-k u(a)$ where $k \geq 0$ and
$\tau_a =$ first leaving time from $a=\inf \{t : x_t \neq a \}$. 
\end{thm}

\begin{proof}
Proceeding as in the case of a Morkov process with discrete space
  (Section 2, \S\ 8) we show that $P_a (\tau_a > t)= e^{- kt}$ and
  $\dfrac{1}{k}= E_a (\tau_a)$ if $\infty > k >0$. If $k=0$,
  $P_a(\tau_a > t)=1$ for all $t$, giving $P_a(\tau_a = \infty)= 1$
  i.e. $P_a (x_t=a$ for all $t) =1$ (such a point is called a
  \textit{conservative trap}). We have $ \mathscr{G} u(a)= \alpha u(a)-
  f(a) = \alpha \int^\infty_o e^{- \alpha t} E_a (f(x_t ))dt-
  f(a)=f(a) -f(a)=0 $. Let now $\infty > k > 0$. Since
  $\dfrac{1}{k}=E_a (\tau_a)$, by Dynkins formula, 
  $$
  E_a \left(\int^{\tau_a}_o \mathscr{G} u(x_t) dt \right)= E_a (u(x_{\tau_a})) -
  u(a),
$$
i.e.,
$$
E_a (\tau_a \mathscr{G} u(a)) = - u(a), \quad
  \text{since}\quad 
  u(x_{\tau_a}) = u(\infty)=0.
  $$
Q.E.D.
\end{proof}

\begin{thm}[Dynkin]\label{chap6-sec2-thm4}%4
  If\pageoriginale a is not a trap then $E_a (\tau_U) < \infty $ for a sufficiently
  small open neighbourhood $U$ of a and  
  $$
  \mathscr{G} u(a) = \lim_{U \rightarrow a} \frac{E_a( u(x_{\tau_U}))
    - u(a)}{ E_a (\tau_U)}, 
  $$
  where $\tau_U = $ first leaving time from $U$.
\end{thm}


\begin{proof}
  We prove that if a is not a trap, there exists $u_0 \in \mathscr{D}
  (\mathscr{G})$ such that $ u_0 (a) > 0$. Let $\mathscr{G} u(a) = 0$
  for every $u \in \mathscr{D}(\mathscr{G})$. Then for all $f \in C
  (S), \alpha. G_\alpha f(a) - f(a)=0 $ i.e. $\int^\infty_o H_t f(a)
  e^{- \alpha t } dt = \dfrac{1}{\alpha} f(a) = \int^\infty_0 e^{-
    \alpha t} f(a)dt$. Since for $f \in C(S), H_t f$ is right
  continuous in $t$,\break $H_t f(a)= f(a)$ i.e. $\int f(b) P(t, a, db)= f(a)
  $ for all $f \in C(S)$. It follows that $P(t, a, db) = \delta_s (db)
  $ i.e. 
  $P_a(x_t = a)= 1$ for all $t$. By right continuity $P_a (x_t = a)$
  for all $t) =1$, i.e. $a$ is a trap. Thus there exists $u_0$ such
  that $\mathscr{G} u_0 (a) \neq 0$. 

From the definition of $\mathscr{D} (S)$ we see that there exists
$\epsilon_0 > 0$ and a neighbourhood $U_0 (a)$ such that  
\begin{equation*}
  \mathscr{G}u_0 (b) > \in_0 
  \begin{cases}
    \text{ for } b \in U_0 (a) \text{ if } a \text{ is regular, } \\
    \text{ for } b \in U_0 (a) \text{ and } b \geq a \text{ if $a$ is
      a pure right shunt,}\\ 
    \text{ for } b \in U_0 (a) \text{ and } b \leq a \text{ if $a$ is
      a pure left shunt.} 
  \end{cases}
\end{equation*}

Therefore $P_a (\mathscr{G} u_0 (x_t) > \in_0$ for $0 \leq t <
\tau_{U_0}) = 1$. Now put $\tau_n = n \Lambda \tau_{U_0}$. Then 
$$
E_a \left(\int^{\tau_n}_{0} \mathscr{G} u_0 (x_t)dt)= E_a(u_0
  (x_{\tau_n}))-u_0 (z)\right)
$$
so that 
$$ 
\epsilon_0 E_a (\tau_n) \leq 2 || u_0 ||.
$$

Letting\pageoriginale $n \to \infty$, $E_a (\tau_{U_0}) \leq 2
\dfrac{||u_0||}{\in_0} < \infty$. Therefore for $U \subset U_0 (a)$,
$E_a (\tau_U) < \infty$. 

Now let $u \in \mathscr{D} (\mathscr{G})$. For every $\in > 0$, there
exists a open neighbourhood $U (a) \subset U_0 (a)$ such that 
{\fontsize{10pt}{12pt}\selectfont
\begin{equation*}
  | \mathscr{G}u (b) - \mathscr{G} u (a) | < \in 
  \begin{cases}
    \text{ for } b \in U(a) \text{ if } a \text{ is regular, } \\
    \text{ for } b \in U(a) \text{ and } b \geq a \text{ if $a$ is a
      pure right shunt,}\\ 
    \text{ for } b \in U(a) \text{ and } b \leq a \text{ if $a$ is a
      pure left shunt.} 
  \end{cases}
\end{equation*}}\relax

Therefore $P_a (|\mathscr{G} u(x_t) - \mathscr{G} u(a)| < \in $ for $0
\leq t < \tau_U) = 1$.~Using Dynkin's formula the proof can be easily
completed. 

\section{Local generator}\label{chap6-sec3} %sec 3.

Let $\mathbb{M} = (S, W, P_a)$ denote a linear diffusion, and $S'$ a
closed interval in $S$. Put $W' = W_c (S')$, $P'_a (B') = P_a
[w^{-}_{\tau}\in B']$,
where $\tau \equiv \tau_{(S')^0} (w)$ is the first leaving time from
the interior $(S')^0$ of $(S')$. We prove that $\mathbb{M}' = (S', W',
P'_a)$ is also a linear diffusion. We shall verify the strong Markov
property for $\mathbb{M}'$. First we show that, if $\sigma '(w')$ is a
Markov time in $W'$, then $\sigma (w) = \sigma^{'}
(w^-_{\tau_{(w)}})$ is a Markov time in $W$ Now 
{\fontsize{10pt}{12pt}\selectfont
\begin{align*}
(w: \sigma (w) \geq t) &= [w : \sigma'(w^{-}_{\tau(w)}) \geq t)] = (w :
  w^-_{\tau (w)}) \in B'_t), B'_t \in \mathbb{B}'_t\\ 
  &= (w : (w^-_{\tau (w)})^-_t \in B'), B' \in \mathbb{B}' \subset \mathbb{B} \\
  &= (w : t \leq \tau (w), w^-_t \in B' ) \cup (w : \tau (w) < t,
  (w^-_t)^-_{\tau (w)} \in B')\\ 
  &= (w : t \leq \tau (w), w^-_t B') \cup (w: \tau (w) < t,
  (w^-_t)_{\tau (w^-_t)} \in B') \in \mathbb{B}_t 
\end{align*}}\relax
since\pageoriginale $w \to w^-_t$ is $\mathbb{B}_t$-measurable and $w \to
w^-_{\sigma _1}$ is $\mathbb{B}$-measurable for any Markov time
$\sigma_1$ we have $(w: (w^-_t)^-_{\sigma_1 (w^-_t)}\in B) \in
\mathbb{B}_t$ for any $B \in \mathbb{B}$. 

Thus $\sigma$ is a Markov time in $W$. Let $f'_1 \in
\mathscr{B}'_{\sigma' +}$ and $f'_2 \in \mathscr{B}'$. Then by
definition of $P'_a$ we have 
$$
E'_a\left[ f'_1 (w') f'_2 (w^{'+}_{\sigma'(w')})\right] = E_a \left[
  f'_1 (w^{-}_{\tau(w)})f'_2 ((w^-_{\tau(w)})^+_{\sigma(w)})\right] 
$$

Put $f_1(w)= f'_1 (w^-_{\tau(w)})$ and $f_2(w)= f^1_2
(w^-_{\tau(w)})$. Let $\sigma_2= \sigma \wedge \tau$. We show that
$f_1 \in \mathscr{B}_{\sigma_2 +}$. Now 
\begin{align*}
f_1 (w^-_{\sigma 2(w)+\delta} & = f'_1 ((w^-_{\sigma_2
    (w)+\delta})^-_{\tau (w^-_{\sigma_2 (w)+\delta)}}) = f'_1\left
  [(w^-_{\tau (w)})^-_{\sigma (w)+ \delta }\right]\\  
  & = f'_1 \left[ (w^-_{\tau (w)})^-_{\sigma' (w^-_{\tau(w)})+\delta}] +
    \delta \right] = f'_1 (w^-_{\tau (w)}),\text{ since }f'_1 \in
  \mathscr{B}'_{\sigma ' +}  
\end{align*}

This proves that $f_1 \in \mathscr{B}_{\sigma _2 +}$. From the
definition of $\tau$ and $\sigma_2$ 
we can see without difficulty that for any $t \geq 0$,
$$
\sigma_2 (w)+ (t \wedge \tau (w^+_{\sigma_2 (w)})) = \tau (w) \wedge
(\sigma (w) + t ) 
$$

Hence $(w^+_{\sigma_2 (w)}) ^-_{\tau ( w ^+_{ \sigma_2 (w)}}) = (
w^-_{\tau (w) })^+_{\sigma (w)} $, so that  
$$
f_2 [w^+_{\sigma_2(w)}] = f^1 _2 \left[ (w ^+ _{\sigma_2 (w)})
  ^-_{\tau (w^+_{\sigma_2} (w))}\right] = f'_2 \left [ (w^-_{\tau(w)})
  ^+_{\sigma (w)}\right].  
$$

Thus
\begin{align*}
E'_a \left [ f'_1 (w') f'_2 (w^{'+}_{\sigma'(w')}) \right] & = E_a
  \left[ f'_1 (w^- _{\tau (w)}) f'_2 (( w^- _{\tau (w)}) ^+_{\sigma
      (w)})\right]\\ 
  &= E_a \left[ f_1 (w) f_2 (w^+_{\sigma_2}) \right] = E_a \left[f_{1} (w)
    E_{x_{\sigma 2}}( f_2 (w))\right]\\ 
  &= E_a \left[f'_1 (w^-_{\tau (w)}) E_{x_\sigma (w)} ( w^-_{\tau
      (w)}) (f'_2 (w ^-_{\tau(w)}))\right]\\
  & = E'_a \left[ f'_1
    (w')E_{x_{\sigma'}} (f'_2 (w'))\right]  
\end{align*}\pageoriginale
which proves that $\mathbb{M}'$ is a linear diffusion.$\mathbb{M}'$ is
called the \textit{stopped process} at the boundary $\partial S'$ of
$S'$. We also denote $\mathbb{M}'$ by $\mathbb{M}_{S'}$, its generator
by $\mathscr{G}'$ or $\mathscr{G}_{s'}$ etc. 

A point $a \in S$ is called a \textit{conservative point} if there
exists a neighbourhood $U$ such that $\mathbb{M}_{\bar{U}}$ is
conservative. The set of all conservative points is evidently open. Let
a be a conservative regular point and $S'$a closed interval containing
a such that $\mathbb{M}_{S'}$ is conservative. We shall prove that if
$u \in \mathscr{D}(\mathscr{G})$, then $u'= u| S' \in \mathscr{D}
(\mathscr{G}')$ and $\mathscr{G}' u' = \mathscr{G} u$ in $(S')^0$;
more generally if $S' \supset S''$, if $u' \in
\mathscr{D}(\mathscr{G}_{S'})$ then $u'' = u'/ S''$ (restriction to  
$S''$) is in $\mathscr{D}(\mathscr{G}_{S''})$ and $\mathscr{G}' u'=
\mathscr{G}'' u''$ in $(S'')^0$. Then we can define $\mathscr{G}_a$
the local generator as the inductive limit of $\mathscr{G}_{S'}$ as $S'
\downarrow a$ in the following way. Consider the set $\mathscr{D}_a$
of all functions defined in a neighbourhood (which may depend on the
function) right (left) continuous at points of $C_+(C\_)$. Introduce
an equivalence relation in $\mathscr{D}_a$ by putting $f \sim g$ if
only if there exists a neighbourhood $U$ of a such that $f =g$ in
$U$. Let $\bar{\mathscr{D}}_a(S) = \mathscr{D}_a(S)/ \sim$ (the
equivalence classes). Define $\mathscr{D}(\mathscr{G}_a)=
\Big\{\bar{u}:\bar{u}\in \mathscr{D}_a(S) $ and there exist $U=U(a)$
with $u|U \in \mathscr{D}(\mathscr{G}_{\bar{U}})$. Define\pageoriginale
$\mathscr{D}\mathscr{G}_a \bar{u}= (\mathscr{G}_{\bar{U}}u)/ \sim$ where
$\bar{u} = u | \sim , u | U \in \mathscr{D} (\mathscr{G}_{\bar{U}})$. 
From above it follows that this is independent of the
choice of $u$. We now prove that if $u \epsilon \mathscr{D}
(\mathscr{G})$ then $u' = u| S' \in \mathscr{D}(\mathscr{G}')$ and
$\mathscr{G}_u=\mathscr{G}' u'$ in $(S')^0$. Note that if $[b,c]= S',
\tau = \tau_U = \sigma _b \wedge \sigma_c, U = (S')^o$. 
We have
\begin{align*}
  u(\xi) &= G_\alpha f (\xi) = E_\xi \left(\int \limits^{\infty}_{0}
  e^{-\alpha t} f(x_t) dt\right)\\
  &= E_\xi \left(\int^{\tau}_{0}e^{-\alpha t} f(x_t) dt\right) + E_\xi \left(
  \int\limits^{\infty}_{\sigma_b} e^{-\alpha t} f(x_t) dt: \sigma_b <
  \sigma_c \right)\\ 
  & \hspace{4cm} + E_\xi \left( \int\limits^{\infty}_{\sigma_c} e^{-\alpha t}
  f(x_t)dt \sigma_c < \sigma _b \right)\\ 
  &= E_\xi \left(\int\limits^{\tau}_{0} e^{- \alpha t} f(x_t(w^-_\tau)) dt\right)
  + G_\alpha f (b) E_\xi (e^{-\alpha \sigma_b}: \sigma_b <
  \sigma_c)\\ 
  & \hspace{2cm}+  G_\alpha f (c) E_{\xi} (e^{-\alpha \sigma _c}:
  \sigma_c < \sigma_b )   
\end{align*}
by strong Markov property. Put $f' = f$ in $U$, $f'(b) = \alpha
G_\alpha f (b)$ and $f'(c) = \alpha G_\alpha f (c)$. Then it is easy
to show that $u' = u| s' = G'_\alpha f' $ and $\mathscr{G}' u' =
\mathscr{G} u $ in $U$. 

\begin{defi*}% definition 0
$\mathscr{G}_a$ is called the {\em local generator } at $a$.
\end{defi*}

\section{Feller's form of generators (1) Scale}\label{chap6-sec4}% section 4

We shall derive Feller's cannocial form of generators by purely
probabilistic methods following Dynkin in the following articles.  

Let a be a conservative regular point. There exists $U = U (a) = (b,
c)$ such that $\bar{U}$ has only conservative regular point and $E_\xi
(\tau_U ) = E_\xi (\delta _{\partial U})< \infty$ for $\xi \in U$. Put
$s(\xi) = P_\xi (\sigma _c < \sigma _b)$. 
\begin{itemize}
\item[$(1^\circ)$] $s\in \mathscr{D}
(\mathscr{G}_{\bar{U}})$ and $\mathscr{G}_{\bar{U}} s = 0 $ in
$\bar{U}$. 

Let $f(c)= 1$ and $f(\xi) = 0$, $\xi \in [b,c)$. Then $f \in
\mathscr{D}(\bar{U)}= \mathscr{D}'$ and 
\begin{gather*}
G'_\epsilon f(\xi) = E'_\xi
\left(\int\limits^{\infty}_{0} e^{-\epsilon t} f (x_t) dt\right) =
E_\xi \left(
\int^{\infty}_{0} e^{-\epsilon t} f(x_t (w^-_{\tau_U})) dt\right)=\\
E_\xi \left(
\int\limits^{\infty}_{\sigma_c} e^{-\epsilon t} dt: \sigma_c <
\sigma_b\right).
\end{gather*}\pageoriginale

Hence $\lim\limits_{\epsilon \downarrow 0} \in G'_\epsilon f(\xi) = P_\xi
(\sigma_c < \sigma_b) = s(\xi)$. The resolvent equation gives 
\begin{gather*}
(G'_\alpha - G'_\epsilon) f + (\alpha - \epsilon) G'_\alpha
  G'_\epsilon f =0\quad \text{or}\\ 
  \epsilon (G'_\alpha - G'_\epsilon) f + (\alpha - \epsilon) G'_\alpha
  G'_\epsilon f = 0. 
\end{gather*}

Letting $\epsilon \to 0$ we get $-s(\xi) + \alpha G'_\alpha s(\xi) =
0$. Therefore firstly $s \in \mathscr{D}'$ and again since $s = \alpha 
G'_\alpha s, s\in\mathscr{D(\mathscr{G'})}$ and  
$$
\mathscr{G}' s=\alpha s- (G'_\alpha)^{-1} s = \alpha s-\alpha s = 0\ \
\text{in\ }\bar{U}. 
$$

\item[$(2^{\circ})$]
is continuous in $\bar{U}$.

Since $s\in\mathscr{D'}$ and all points of $U$ are regular for $s', s$
is continuous in $U$.  

It remains to prove that $s$ is continuous at $b$ and $c$. We prove
the continuity at $c$; continuity at $b$ is proved in the same way. To
prove this we shall first prove that $e= \lim\limits_{\xi \uparrow c}
E (e^{-\sigma_{c-}})=1$ or $0$, $\sigma_{c-} = \lim\limits _{\eta
  \uparrow c} \sigma_\eta$. Let $\xi < \eta < \xi < c$. Then $E_\xi
(e^{-\sigma_\eta}) = E_\xi (e^{- \sigma_\eta}) E_\eta (e^{-
  \sigma_{c-}})$. 
Letting $\eta \uparrow c$, now and $\xi \uparrow c$ finally
we get $e=e^2$ so that $e=1$ or $0$. Since $c$ is regular, there exists
$\xi < c$ such that $P_\xi (\sigma_c < \infty) > 0$ and then $E_\xi
(e^{- \sigma_c})> 0$.\pageoriginale Also $\sigma _c \geq \sigma_{c-}$. It follows
that $E_\xi (e^{-\sigma_{c-}})>0$. 
Hence $e=1$. Since $\xi$ is conservative
$$
P_\xi (x_{\sigma_{c-}} = \infty , \sigma_{c-} < \infty ) \leq P_\xi
(\sigma_\infty < \infty) = 0. 
$$

Therefore since $\sigma_c \geq \sigma_{c-}$ and the paths are
continuous before the killing time, $P_\xi (\sigma _c =
\sigma_{c-})=1$. We have proved that $\lim\limits_{\xi \uparrow c}
E_\xi (e^{- \sigma_c})\break =1$. For every $\epsilon > 0$, therefore,
$\lim\limits_{\xi \to c} P_\xi (\sigma_c < \epsilon)=1$. Also 
{\fontsize{10pt}{12pt}\selectfont
$$
S(\xi) = P_\xi (\sigma _c < \sigma _b) \geq P_\xi (\sigma _c < \epsilon,
\sigma _b \geq\epsilon) \geq P_\xi (\sigma _c < \epsilon) - P_\xi (\sigma_b <
\epsilon) 
$$}\relax


If $b< \xi_0 < \xi < c$, then
\begin{align*}
& P_\xi (\sigma_b < \epsilon) \leq P_\xi (\sigma_{\xi_{0}} < \infty, \sigma_b
(w^+_{\sigma_{\xi_0}})< \epsilon)\\
& = P_{\xi} (\sigma_{\xi_{0}} < \infty)P_{\xi_0}
(\sigma_b < \epsilon)\leq P_{\xi_{0}}(\sigma_{b}<\epsilon) 
\end{align*}

Therefore $s(\xi)\geq P_\xi (\sigma_c < \epsilon) - P_{\xi_0} (\sigma_b <
\epsilon)$. Letting $\xi \uparrow c$ first and $\epsilon\downarrow 0$ 
next, we get $\lim\limits_{\xi \to c} s (\xi) \geq 1$ i.e. $s(\xi)$ is
continuous at $\xi = c$. 

\item[$(3^{\circ})$]
$s(\xi)$ is strictly increasing.

The set of points $\xi$, $b< \xi \leq c$ such that $s (\xi) =0$ is
closed in $(b,c]$. If $P_{\xi_0} (\sigma_c < \sigma_b) = 0$, the same
    is evidently true for any $b < \xi < \xi_0$. Since $\xi_0$ is
    regular $\lim\limits_{\eta\downarrow\xi_0} P_{\xi_0} (\sigma_\eta
    < \epsilon) = 1$ for any $\epsilon > 0$. Also $P_{xi_0} (\sigma_b >
    0)=1$. It easily follows that $\lim\limits_{\eta \downarrow \xi
      _0} P_{\xi_0} (\sigma_\eta < \sigma _b) = 1$. Choose $\eta _0 >
    \xi _0$ with $P_{\xi_0} (\sigma_{\eta_0}< \sigma_b)> 0$. Then
    $P_{\xi_0} (\sigma_\eta < \sigma _b)> 0$ for any $\xi_0 < \eta <
    \eta_0$. Now that if $a < \xi$ then\pageoriginale 
    $(\sigma _a < \sigma_\xi) =
    (w:\sigma_\xi(w^-_{\sigma a})=\infty)$, and hence is in
    $\mathbb{B}_{\sigma_a}$. 
    We have $0=P_{\xi_0} (\sigma_a <
    \sigma_b) = P_{\xi_0} (\sigma_\eta < \sigma_b ) P_\eta (\sigma_c
    < \sigma_b)$. Thus $P_\eta (\sigma_c< \sigma_b) =0$. The
    connectedness of $(b,c]$ shows that $s(\xi)\neq 0$ in
    $(b,c]$. Exactly similar argument also shows that $s(\xi) < 1$ in
    $[b,c)$. Now if $\xi < \eta$, we replace $c$ by $\eta$ and repeat the
  argument to get $P_\xi (\sigma_\eta < \sigma_b)<1$. Thus if $\sigma
  < \eta$ 
$$
s(\xi) = P_\xi (\sigma_c < \sigma_b)= P_\xi (\sigma_\eta < \sigma_b)
P_\eta (\sigma_c < \sigma_b) < P_\eta (\sigma_c < \sigma_b) 
$$

\item[$(4^{\circ})$] $\alpha s+ \beta$ is the general solution of
  $\mathscr{G}'u =0$. 

Let $f(\xi) =1$ for $b\leq \xi \leq c$. Then $f(\xi)=1= \alpha E'_\xi
(\int^{\infty}_{0} e^{-\alpha t} f(x_t) dt)\break = \alpha 0'_\alpha
f(\xi)$. This firstly shows that $f \in\mathscr{D}' (s')$ and then the
same equation shows that $f \in\mathscr{D (\mathscr{G}')}$. Thus
$\mathscr{G}'1 \alpha \cdot 1-(G'_\alpha)^{-1} 1 = \alpha - \alpha =
0$. Hence since $\mathscr{G}' s= 0$ $\mathscr{G}' (\alpha s +
\beta)=0$. Now let $\mathscr{G}'u=0$. Then 
\begin{align*}
  0 & =E'_\xi
  \left(\int\limits^{\tau _{U}}_0 \mathscr{G}' u (x_t) dt\right)= E'_\xi (u
  (x_{\tau_U})) - u (\xi) = E_\xi (u(x_{\tau_U})) - u (\xi)\\
  & = u (b) P_\xi (\sigma_b< \sigma_c) + u (c) P_\xi (\sigma_c
  < \sigma_b) - u (\xi). 
\end{align*}

Therefore $u$ is linear in $s$.

\item[$(5^\circ)$] If $b < b' \leq \xi \leq c' < c$ then $P_\xi (\sigma_{c'} <
\sigma _{b'}) = \dfrac{s (\xi) - s (b')}{s(c') -s (b')}$  

Let $x = P_{\xi} (\sigma_{c'} < \sigma_{b'})$, $y= P_\xi (\sigma_{b'} <
\sigma_{c'})$; then $x+y=1$ and  
$$
P_{\xi} (\sigma_c < \sigma_b) = P_\xi (\sigma_{c'} < \sigma _b) P_{c'}
(\sigma_c < \sigma_b) 
$$ 

Also 
$$
(\sigma_{c'} < \sigma_b) = (\sigma_{c'} < \sigma_{b'}) \cup
(\sigma_{c'} > \sigma_{b'}, \sigma_b (w^+_{\sigma_{b'}}) >
\sigma_{c'} (w^+_{\sigma_{b'}}))
$$ 

Therefore\pageoriginale 
\begin{align*}
P_\xi (\sigma_c < \sigma_b) & = P_\xi (\sigma_{c'} <
  \sigma_{b'}) P_{c'} (\sigma_c < \sigma_b)+ P_\xi (\sigma_{c'} > \sigma
  _{b'} , \sigma_b (w^+_{\sigma_{b'}})\\ 
  & \hspace{2cm} > \sigma_{c'} (w^+
  _{\sigma_{b'}})) P_{c'} (\sigma_c < \sigma _b) \\
  &= x s(c') + P_\xi (\sigma_{c'} > \sigma_{b'}) P_{b'} (\sigma_b >
  \sigma_{c'}) P_{c'} (\sigma_c < \sigma_b)\\ 
  &= x s(c') + P_\xi (\sigma_{b'} < \sigma_{c'}) P_{b'} (\sigma_c < \sigma_b)
\end{align*}
i.e.,\qquad $s(\xi) = x ~ s(c') + y ~ s(b')$. 
Solving for $x$ we get the result.
\end{itemize}
\end{proof}

\begin{defi*}
 $s$ is called the {\em canonical sacle} in $b$, $c$.
\end{defi*}

\section{Feller's form of generator (2) Speed
  measure}\label{chap6-sec5} % section 5

Let $p(\xi) = E'_\xi (\tau_U) = E_\xi (\tau_U)$, $U = (b, c)$. Put $f=1$
for $x \in U$, $f (b) = f(c) =0$. Then $G'_\epsilon f (\xi)=E'_{\xi}
\left(\int\limits^{\infty}_{0} e^{- \epsilon t} f(x_t)dt\right) = E_\xi
\left(\int\limits^{\tau_U}_{0} e^{- \epsilon t} dt\right)$, so that  
$$
\lim\limits_{\epsilon \downarrow 0} G'_\epsilon f(\xi) = p(\xi)
$$

We have $G'_\alpha f - G'_\epsilon f + (\alpha - \epsilon) G'_\alpha
G'_{\epsilon} ~ f = 0$. Letting $\epsilon \to 0$  
$$
G'_\alpha f - p + \alpha G'_\alpha p = 0.
$$

This shows that $p \in \mathscr{D} (\mathscr{G}')$, because, $f$
being indentically 1 in $U$, is continuous at every regular-point
and $b$, $c$ are traps for $\mathbb{M}'$. We have, 
\begin{itemize}
\item[$(1^{\circ})$] $\mathscr{G}' p= -f$ i.e. $\mathscr{G}' p= -1$ in $U$,
  $\mathscr{G}' p(b) = \mathscr{G}' p(c) = 0$ 

\item $p$ is continuous in $\bar{U}$ and $p(b) = p(c) = 0$. 


We\pageoriginale prove that $p(c-) = p(c) =0$. 
Let $b < \xi < c$ and $\tau_\xi =
  \tau_{(b,\xi)}$. Then if $b< \xi _0 < \xi$  
\begin{multline*}
E_{\xi_0} (\tau_U ) = E_{\xi_0} (\tau_\xi ) + E_{\xi_0} (\tau_U
(w^+_{\tau _\xi})) = E_{\xi_0} (\tau_\xi)\\ 
+ E_{\xi_0} (E_{x _{\tau_{\xi}}}(\tau_U):\sigma_\xi < \sigma _b)= E_{\xi_0} (\tau_\xi)
  + E_\xi (\tau_U) P_{\xi_0} (\sigma_ \xi < \sigma_b).  
  \end{multline*}

  Now as $\xi \to c, E_{\xi_0} (\tau_\xi) \to E_{\xi_0} (\tau _U)$ and
  $\sigma_\xi \to \sigma_c$. Therefore $\lim\limits_{\xi \to c} E_\xi
  (\tau_U) P_{\xi_0} (\sigma_c < \sigma_b) = 0 $ i.e. $p(c-) = 0$. 
  

\item[$(3^{\circ})$] 
$p$ is the only solution of $\mathscr{G}' u = -1$ in $U$, $u(b) =
  u(c)=0$.  

  For $-p(\xi) = -E_\xi (\tau_U) = E_\xi (\int\limits^{\tau_{U}}_{0}
  \mathscr{G}' u(x_t) dt) = E_\xi (u (x_{\tau_{U}})) - u (\xi)$. Since
  $x _{\tau_{U}} = b $ or $c, ~ u(x_{\tau_{U}}) =  0$.\hfill Q.E.D. 

\item[$(4^{\circ})$] We have proved that $s: [b, c] \to [0, 1]$ is $1-1$
  continuous. We define a mapping $p'$ on $[0, 1]$ by $p'(s (\xi)) =
  p(\xi)$. To prove that $p'$ is strictly concave in $[0, 1]$. We
  express this by ``$p$ is concave in $s$''. We have to prove that, if
  $b \leq \eta < \xi < \zeta \leq c$ 
  $$
  p (\xi) > \frac{s (\xi) - s (\eta)} {s (\zeta) -s (\eta)} p(\zeta) +
  \frac{s (\zeta) -s (\xi)} {s (\zeta) - s (\eta)} p(\eta). 
  $$


  Now $p (\xi) = E_\xi (\tau_U) = E_\xi (\sigma= \tau_U (w^+_\sigma))$,
  $\sigma= \tau_{\eta,\zeta} > E_ \xi (\tau_U (w^+_\sigma )) =$ right
  side of the above inequlity 

\item[$(5^\circ)$] $m (\xi) = \dfrac{d^+p}{ds}$ is
  strictly increasing and bounded if there exists an interval $V
  \supset \tau$ such that $E_\xi (\tau_V) \infty$ and $\mathbb{M}_V$
  is conservative. (The measure $dm$ is called the \textit{speed
    measure} for $\bar{U}$).  
\end{itemize}
 
From $(4^\circ)$ the right derivative $\dfrac{d^+ p}{ds}$ exists and
strictly increases.\pageoriginale 
We prove that it is bounded. Let $V = (b_1, c_1)
\supset [b, c]$. Put $p_1(\xi) = E_\xi (\tau_V)$, $s_1 (\xi) = P_\xi
(\sigma_{c_1} < \sigma_{b_1})$. We have  
$$
P_1(\xi) = E_\xi (\tau_U) + E_\xi (\tau_{V}(w^+_{\tau _{U}})) = p(\xi) + s (\xi)
P_1 (c) + (1-s(\xi)) P_1 (b).  
$$  
 
From this one easily sees that 
$$
m_1 (\xi) =- {d+p_1(\xi)}{ds_1} = [m(\xi) - (P_1(c)-p_1 (b))]
\frac{1}{s_1(c)-s_1 (b)} 
$$
\hfill{Q.E.D.}

\section{Feller's form of generators (3)}\label{chap6-sec6} % section 6

 \begin{theorem*}[Feller]
 $u \in \mathscr{D} (\mathscr{G}')$ if and only if
   \begin{enumerate}[\rm (1)]
   \item $u$ is of bounded variation in $U$

   \item $du < ds$ i.e. $du$ is absolutely continuous with respect to $ds$.

   \item $\dfrac{du}{ds}$ (Radon Nikodym derivative) is of bounded
     variation in $U$. 

   \item $d\dfrac{du}{ds}<dm $ in $U$.

   \item $(d\dfrac{du}{ds})/dm $ (which we shall write
     $\dfrac{d}{dm}\dfrac{du}{ds}$ has a continuous version in $U$. 

   \item $u$ is continuous at $b$ and $c$ i.e. $u$ is continuous in
     $\bar{U}$ and $\mathscr{G}' u = \dfrac{d}{dm} \dfrac{du}{ds}$ in
     $U, \mathscr{G}' u =0$ at $b$ and $c$.  
\end{enumerate} 
 \end{theorem*}
  
\begin{proof}(Dynkin)
  Let $u \in \mathscr{d} (\mathscr{G}')$. Then for some $f \in \mathscr{D}'$
  \begin{multline*}
  u(\xi) = G'_\alpha f (\xi) = E _\xi \left(\int\limits^{\tau_U}_{0}
  e^{-\alpha t} f(x_t) dt\right)\\ 
  + E_\xi (e^{-\alpha \sigma_b}: \sigma_b <
  \sigma_c ) \frac{f(b)}{\alpha} + \frac{f (c)} {\alpha} E_\xi
  (e^{-\alpha \tau_c}: \sigma_c < \sigma_b). 
  \end{multline*}
  
  Thus $\lim\limits_{\xi \to b} u(\xi) = \dfrac{f (b)}{\alpha} = u(b)
  $ and $\lim\limits_{\xi \to c} u (\xi) = \dfrac{f (c)}{\alpha} =
  u(c)$. $u$ is therefore\pageoriginale continuous in $\bar{U}$. 
  
  Let $[\alpha , \beta] \subset U$. If $\mathscr{G}' V \geq 0$ in
  $(\alpha , \beta)$ then Dynkin's formula shows 
  $$
  0 \leq E \left(\int\limits^{\sigma_\alpha \wedge \sigma_\beta }_{0}
  \mathscr{G}' v (x_t) dt\right) = v (\alpha) \frac{s (\beta) -s (\xi)}{s
    (\beta) - s (\alpha)} + v (\beta) \frac{s (\xi) - s (\alpha)} {s
    (\beta) - s(\alpha)} - v (\xi), 
  $$
  so that $v$ is convex in $s$ and hence is of bounded variation in
  [$\alpha, \beta$]. Also $\dfrac{d ^+v}{ds}$ exists and increases in
  [$\alpha, \beta$] and $dv$ is absolutely continuous with respect to
  $ds$. If $\mathscr{G}' v \geq \lambda$ in $(\alpha, \beta)$, then
  $\mathscr{G}' (v +\lambda p) \geq 0$ in $(\alpha, \beta)$. Therefore
  $d\dfrac{d^+v}{ds} \geq \lambda dm$. Similarly if $\mathscr{G'}v
  \leq \mu$ in $(\alpha, \beta)$ then $d \dfrac{d^+v}{ds} \leq \mu  dm$.

Consider a division $\Delta = (b = \alpha_0 < \alpha_1 < \cdots <
\alpha_n =c)$ of [$b, c$]. Put  
$$
\lambda_i = \inf\limits_{\xi \in (\alpha_i, \alpha_{i + 1})}
\mathscr{G}' u(\xi), \mu_i = \sup\limits_{\xi \in (\alpha_{i}, \alpha_{i+1})}
\mathscr{G}' u (\xi). 
$$

Then $\mu_i dm \geq d \dfrac{d^+u}{ds} \geq \lambda_i dm $ in
$(\alpha_i, \alpha _{i+1})$ and $\mu_i dm \geq \mathscr{G}' udm \geq
\lambda_i dm$ in $(\alpha_i, \alpha_{i+1})$. Putting $\lambda (\xi) =
\lambda_i$ and $\mu (\xi) = \mu_i$ for $\alpha_i \leq \xi <
\alpha_{i+1}$ we have $ 
\mu (\xi) dm \geq d \dfrac{d^+u}{ds} \geq \lambda (\xi) dm$, and $\mu
(\xi) dm \geq \mathscr{G}' u (\xi) dm \geq \lambda (\xi)
dm$. Therefore $(\mu (\xi) - \lambda (\xi)) dm \geq d \dfrac{d ^+
  u}{ds} - \mathscr{G}' u (\xi) dm \geq - (\mu (\xi)- \lambda (\xi))
dm$. As $\delta (\Delta) = \max\limits_i [\alpha_{i+1}-\alpha_i]$
tends to zero, $\mu (\xi) - \lambda (\xi) \to 0$. We have  
$$
d \frac{d ^+ u}{ds} = \mathscr{G}' u ~ dm ~\text{ \ in \ } ~ U.
$$\pageoriginale

Conversely suppose that $u$ satisfies all the above six
conditions. Define $f = \alpha u - \dfrac{d}{dm}\dfrac{d}{ds} u$ in $U$
and $f(b)=\alpha u(b), f(c)=\alpha u(c)$. Then since $f$ is continuous
in $U, f \in \mathscr{D}'$. Let $v=G'_\alpha f$. From what we have
already proved $\alpha v - \dfrac{d}{dm} \dfrac{d}{ds} v =f$ in $U$, $v(b)=
\dfrac{b}{\alpha}f(b)$, $v(c) = \dfrac{1}{\alpha} f (c)$. If $\theta = u-v$ then
$\theta$ is continuous in $\bar{U}, \theta (b)= \theta (c) =0$, and
$\alpha \theta - \dfrac{d} {dm} \dfrac{d}{ds}\theta = 0$. There exists
a point $\xi_0$ such that $\theta (\xi_0) $is a maximum. Now $\theta
(\xi_0) > \to \theta (\xi) > 0$ near $\xi_0 \Rightarrow
\dfrac{d}{dm}\dfrac{d}{ds} \theta > 0$ near $\xi_0 \Rightarrow
\dfrac{d}{ds} \theta$ strictly increases near $\xi_0$. Then if $\xi >
\xi _o > \eta $ are near $\xi_o$ we have $\theta (\xi_o) - \theta
(\eta) = \int\limits^{\xi_o}_{\eta} \dfrac{d \theta}{ds} ds <
\dfrac{d\theta}{ds} (\xi_0) [ s (\xi_0)- s (\eta)]$. Hence $\dfrac{d
  \theta}{ds}(\xi_{0}) > 0$. On the other hand $\theta (\xi) - \theta (\xi_o) =
\int\limits ^{\xi}_{\xi_o} \dfrac{d\theta}{ds} ds >
\dfrac{d\theta}{ds} (\xi_o) [s (\xi) - s (\xi_o)]$, a
contradiction. Therefore $\theta(\xi) \leq 0$. Smiliarly we prove
$\theta (\xi) \geq 0$.\quad Q.E.D. 
\end{proof}

\markright{7. Feller's form of generators (4)\ldots}

\section[Feller's form of generators...]{Feller's form of generators (4) Conservative compact
  interval}\label{chap6-sec7}% section 7 

\markright{7. Feller's form of generators (4)\ldots}

Let $I=[b, c]$ be a conservative compact regular interval i.e., a
compact interval consisting only of conservative regular points. We
shall prove the following  

\begin{theorem*}[Feller]
  All the results of the three articles hold for $\mathbb{M}_I$.
\end{theorem*}

\begin{proof}
Since every $a \in I$ is conservative regular, we can associate with
  any $a \in I$ an open interval $U(a)$ such that $E_\xi (\tau_{U(a)})
  < \infty$ for $\xi \in U(a)$ and then the results of the last three
  articles are true\pageoriginale for $\mathbb{M}_{\overline{U(a)}}$. Denote the
  quantities $s, p, m$ etc. for $\bar{U}$ by $s_U, p_U, m_U$, etc. Let
  $s = P_\xi (\sigma_c < \sigma_b)$. Then from $(5^0)$ of $\xi 4$ we
  get  
  $$
  s (\xi ) = s(b') + [s(c') - s(b')] P_\xi (\sigma_{c'} < \sigma_{b'}),
  $$
  where $\xi \theta(b', c')$ is an interval such that the results of
  the last three articles are true for $\mathbb{M}_{[b', c']}$. This
  equation shows that $s(\xi)$ is strictly increasing and continuous in
  some neighbourhood of the point $\xi$. Therefore $s(\xi)$ is
  strictly increasing and continuous in $I$, and $s$ is linear in
  $s_U$ in $U$, for every interval $U$ such that the results of the last
  three articles are true for $\mathbb{M}_{\bar{U}}$. Let $dm$ be a
  measure defined on $\mathbb{B}(I)$ as follows. $dm =
  \dfrac{1}{\alpha_U} dm_U$ if in $U$, $s=\alpha_U s_U + \beta_U$. Let
  $V\cap U =W \neq \phi$. Since $p_U = p_W + p_U (b') + s_W (\xi)
        [p_U (c') - p_U (b')]$ where $U= (b', c')$ we have
        $\dfrac{1}{\alpha_U} dm_U = dm_W = \dfrac{1}{\alpha_V}
        dm_V$. Therefore the measure $dm$ is uniquely defined on
        $\mathbb{B}(I)$ and $\dfrac{d}{dm}\dfrac{d}{ds} =
        \dfrac{d}{ds_U} \dfrac{d}{ds_U}$ in $U$. $dm$ is defined by a
        strictly increasing function $m$ (say) in $I$. Consider now
        the following ``differential equation'' 
  $$
  \frac{d}{dm}\frac{d}{ds} u=- 1 \text{ \ in \ } (b, c) \text{ \ and
    \ } u(b+) = u(c-) = 0. 
  $$

Then $p(\xi) = - \int\limits^{\xi}_{b+} m (\eta) ds (\eta) +
\int\limits^{c-}_{b+} m(\eta) ds (\eta) [s (\xi) - S (b)]
\dfrac{1}{s (c)- s (b)}$ is a solution and $(\alpha - \dfrac{d}{dm}
\dfrac{d}{ds}) p = \alpha p+1$ in $(b, c)$ and $p(b+) = p(c-) =
0$. Let $f = \alpha p+1$ in $(b, c)$ and $f(b) = f(c) = 0$. Since
$f$ is continuous in $(b, c)$, $f \in \mathscr{D}_I$. Let $v = G^I
f$. Then $v \in \mathscr{D} (\mathscr{G}^I)$ and\pageoriginale 
$(\alpha - \mathscr{G}^I)v = \alpha  p + 1$. $v \in (\mathscr {G}^I)
\Rightarrow v \in \mathscr{D} (\mathscr{G}^{\bar{U}})$ so that
$(\alpha - \dfrac{d}{dm_U} \dfrac{d}{ds_U}) v = \alpha p+1$ in
$U$. Therefore $(\alpha - \dfrac{d}{dm}\dfrac{d}{dS}) v = \alpha p+1$ 
in $(b, c)$. Since $v \in \mathscr{D}(\mathscr{G}^I)$, it is
continuous in $I$. Let $\theta = p - v$. $\theta$ is continuous in $I$
and $(\alpha - \dfrac{d}{dm} \dfrac{d}{dS})\theta =0$. We prove as
in \S\ \ref{chap6-sec6} that $\theta = 0$. Thus $p(\xi) = v (\xi) \in \mathscr{D}
(\mathscr{G}^I)$. Using Dynkin's formula we have, if $\tau_n =
\tau_{(b, c)} \wedge n = \tau \wedge n$ (say)  
$$
E_\xi \left(\int\limits^{\tau_n}_{0} \mathscr{G}^I (p (x_t) dt) = E_\xi (p
  (x_{\tau_n})) - p(\xi)\right)
$$
i.e.,
$$ 
E_\xi (\tau_n) \leq 2 || p || < \infty. \text{ \ We get \ } E_\xi (\tau)
  < \infty.
$$
 
Again using Dynkin's formula 
$$
E_{\xi} \left(\int\limits^{\tau}_{0} \mathscr{G}^I p(x_t)dt\right) = E_\xi
(p(x_\tau)) - p (\xi) \text{ \ i.e. \ }p(\xi) = E_\xi (\tau_{(b, c)}). 
$$

The proof of the theorem can be completed as in \S\ \ref{chap6-sec6}.
\end{proof}


\begin{thebibliography}{99}\pageoriginale
%Section 0

\bibitem{key1} {DOOB, J.L}. -- \textit{Stochastic Processes}, New York (1952)

\bibitem{key2} {HALMOS, P.R.} -- \textit{Measure Theory}, New York
  (1950)

\bibitem{key3} {KOLMOGROV, A. N.} -- \textit{Grundbegriffe der
  Wahrscheinlickeitsrechnung} Ergeb. der Math. \underline{2}, 3 (1993)
  [English translation: Foundations of the theory of Probability
    (1956)] 

Sections 1 and 2

\bibitem{key4} {BLUMENTHAL, R.M.} -An extended Markov property,
  \textit{Trans}. \textit{Amer Math. Soc}., 85(1957), pp. 52-72 

\bibitem{key5} {CHUNG, K.L} -On a basic property of Markov chains,
  \textit{Ann. Math}. 68(1958), pp.126-149 

\bibitem{key6} {DYNKIN}, E. B. Infinitesimal operators of Markov
  processes, \textit{Theory of Probility and its applications } 1; 1
  (1956) (in Russian),pp. 38-60 

\bibitem{key7} {KAC, M.} -On some connections between Probability Theory
  and differential and integral equations. \textit{Proc. of the Second
    Berkely Symposium on Mathematical Statistics and Probability},
  1951,pp. 199-215 

\bibitem{key8} {RAY, D. B.} -Resolvents, Translation functions and
  strongly Markovian Process, \textit{Ann. Math}. 70, 1(1959),
  pp. 43-72 

\bibitem{key9} {RAY, D. B.} -Stationary Markov process with continuous
  paths, \textit{Trans. Amer. Math. Soc}., 49 (1951), pp. 137-172 

Sections 3

\bibitem{key10} {DOOB, J. L.}\pageoriginale -A probability approach to
  the heat equation, \textit{Trans. Amer. Math. Soc.}80, 1(1955), pp. 216-280 

\bibitem{key11} {DOOB, J. L.} -Conditional Brownian mation and the
  boundary limits of harmonic functions, \textit{Bull. Soc
    Math. France,} 85(1957), pp. 431-458 

\bibitem{key12} {DOOB, J. L.} -Probability methods applied to the first
  boundary value problem, \textit{Proc. of Thrid Berkeley Symposium on
    Math. stat. and Probability, 2, pp. 49-80} 

\bibitem{key13} {DOOB, J. L.} -Probability theory and the first boundary
  value problem, \textit{IIi. Jour. Math.,} 2 -1 (1958), pp. 19-36 

\bibitem{key14} {DOOB, J. L.} -Semi-martingales and subharmonic functions,
  \textit{Trans Amer. Math. Soc}, 77-1 (1954), pp. 86-121 

\bibitem{key15} {HUNT, G. A.} -Markov processes and
  potentials,\textit{III. Jour. Math.,} 1(1957), pp. 44-93;
  pp. 316-369 and 2(1958), pp. 151-213 

\bibitem{key16} {ITO, K \& McKEAN, H. P} -Potentials and the random walk,
  \textit{Ill. Jour. Math.}4-1(1960), pp. 119-132 

\bibitem{key17} { LEVY, P.} -\textit{Processes Stochastique et Movement
  Brownien}, Paris, 1948 

Section 4

\bibitem{key18} {DOOB, J. L.} - loc.cit.

\bibitem{key19} {FROSTMAN, O.} - Potential d'equilibre et capacite des
  ensembles avecquelques applications a la theorie des fonctions,
  \textit{These pour le doctract}, Luad, 1935, pp. 1-118 

\bibitem{key20} {ITO, K.} - Stochastic Processes,
  \textit{jap. Jour. Math}., 18(1942), pp. 261-301 

\bibitem{key21} {LEVY, P.} - \textit{Theorie de l'addition des variables
  aleatoires}, 2nd ed., Paris, (1954) [Chap. 7]  

\bibitem{key22} {RIESZ, M.} - Integrale de Riemann-Lioville et Potentiels
  \textit{Acta Sci. Szeged}. 9(1938); pp.1 - 42 

Section 5

\bibitem{key23} {DOOB, J. L.} -loc.cit [Chap. 10]

\bibitem{key24} {ITO, K.} -Stochastic differential equations,
  \textit{Memoir. Amer. Math. Soc.,} 4, 1951 

Section 6\pageoriginale

\bibitem{key25} {DYNKIN, E. B.} -loc.cit.

\bibitem{key26} {FELLER, W.} -The parabolic differential equations and the
  associated semi groups of transformations,
  \textit{Ann. Math.}55(1952), pp. 468-519 

\bibitem{key27} {FELLER, W.} -On second order differntial operatorts,
  \textit{Ann. Math} 61-1(1955), pp. 90-108 

\bibitem{key28} {FELLER, W.} -On the intrinsic form for second order
  differential operators, \textit{Ill. Jour. Math}. 2-1(1958),
  pp. 1-18 

\bibitem{key29} {RAY, D. B} loc.cit.
 \end{thebibliography} 

