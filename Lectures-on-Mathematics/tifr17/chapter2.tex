\chapter{Nevanlinna's Second Fundamental Theorem}

\setcounter{section}{2}
\setcounter{subsection}{0}
\subsection{}\label{part2-sec2.1}\pageoriginale

As the name indicates this theorem is most fundamental in the study of
meromorphic functions. It is an extension of Picard's Theorem, but
goes much farther. We develop the theorem in theorems \ref{part2-thm8}
and \ref{part2-thm9} of this chapter, and then proceed systematically
to explore some of its consequences.

\begin{thm}\label{part2-thm8}
Suppose that $f(z)$ is a non-constant meromorphic function in $|z|\leq
r$. Let $a_{1},a_{2},\ldots,a_{q}$ be distinct finite complex numbers,
$\delta>0$ such that $|a_{\mu}-a_{\nu}|\geq\delta$ for $1\leq \mu\leq
\nu\leq q$. Then,
$$
m(r,f)+\sum^{\nu=q}_{\nu=1}m(r,a_{\nu})\leq 2T(r,f)-N_{1}(r)+S(r)
$$
where $N_{1}(r)$ is positive and is given by
$$
N_{1}(r)=N(r,1/f')+2N(r,f)-N(r,f')
$$
and
\begin{align*}
S(r)&=m(r,f'/f)+m\left(r,\sum^{\nu
  q}_{\nu=1}f'/(f-a_{\nu})\right)\\
&\qquad\qquad{}+q\log^{+}\left(\frac{3q}{\delta}\right)+\log
2+\log^{+}1/|f'(0)| 
\end{align*}
with modifications if $f(0)=0$ or $\infty$, $f'(0)=0$.
\end{thm}

\begin{proof}
Let a new function $F$ be defined as follows
$$
F(z)=\sum^{\nu=q}_{\nu=1}1/[f(z)-a_{\nu}]
$$
and suppose that for some $\nu$, $|f(z)-a_{\nu}|<\delta/3q$. Then for
$\mu\neq \nu$,
\begin{align*}
|f(z)-a_{\mu}| &\geq |a_{\mu}-a_{\nu}|-|f(z)-a_{\nu}|\\
&\geq \delta-\delta/3q\\
&> 2\delta/3.\qquad (q\geq 1)
\end{align*}
Therefore,
\begin{align*}
1/|f(z)-a_{\mu}| &< 3/2\delta\quad\text{for}\quad \mu\neq \nu\\
&< (1/2q)[1/|f(z)-a_{\nu}|] 
\end{align*}
Again,\pageoriginale
\begin{align*}
|F(z)| &\geq 1/|f(z)-a_{\nu}|-\sum_{\mu\neq \nu}1/f(z)-a_{\mu}|\\
&\geq [1/|f(z)-a_{\nu}|][1-(q-1)/2q]\\
&\geq 1/2|f(z)-a_{\nu}|
\end{align*}
Hence
$$
\log^{+}|F(z)|\geq \log^{+} 1/|f(z)-a_{\nu}|-\log  2.
$$
In this case,
\begin{align*}
\log^{+}|F(z)| &\geq
\sum^{q}_{\mu=1}\log^{+}1/|f(z)-a_{\mu}|-q\log^{+}2/\delta-\log 2\\
&\geq
\sum^{q}_{\mu=1}\log^{+}\frac{1}{|f(z)-a_{\mu}|}-q\log^{+}3q-\log 2.
\end{align*}
since all the term for $\mu\neq \nu$ are at most
$\log^{+}2/\delta$. This is true if $|f(z)-a_{\nu}|<\delta /3q$ for
some $\nu\leq q$. This inequality is true evidently for at most one
$\nu$ (with the condition $|a_{\mu}-a_{\nu}|\geq \delta$). If it is
not true for any value then we have trivially,
$$
\log^{+}|F(z)|\geq
\sum^{q}_{\nu=1}\log^{+}1/f(z)-a_{\nu}|-q\log^{+}3q/\delta-\log 2
$$
(because, $\log^{+}|F(z)|\geq 0$). So the last relationship holds good
in all cases. 
\end{proof}

Taking integrals we deduce,
\begin{equation*}
m(r,F)\geq
\sum^{q}_{\nu=1}m(r,1/f-a_{\nu})-q\log^{+}\frac{3q}{\delta}-\log 2
\tag{2.1}\label{part2-eq2.1} 
\end{equation*}

Again to get an inequality in the other direction,
$$
m(r,F)=m\left(r,\frac{1}{f}\dfrac{f}{f'}f'F\right)
$$
By equation $(1.2')$ of 1.2 \ie,
\begin{align*}
T(r,f) & =T(r,1/f)+\log |f(0)|\\
m(r,f/f') &= m(r,f'/f)+N(r,f'/f)-N(r,f/f')+\log[|f(0)|/|f'(0)|]\\
m(r,1/f) &= T(r,f)-N(r,1/f)+\log 1/|f(0)|
\end{align*}\pageoriginale
Hence we get finally,
\begin{align*}
m(r,F) &\leq T(r,f) - N(r,1/f)+\log 1/|f(0)|+m(r,f'/f)\\
&+N(r,f'/f)- N(r,f/f')+\log |f(0)|/|f'(0)|+m(r,f'F)
\end{align*}
The above inequality, combined with (\ref{part2-sec2.1}) gives
$\sum\limits^{q}_{\nu=1}m(r,a_{\nu})-q\log^{+}\unskip\break 3q/\delta-\log 2\leq$
Right hand side of the above inequality. Add to both the sides
$m(r,f)$, we get the inequality,
\begin{align*}
m(r,f)+\sum^{q}_{\nu=1}m(r,a_{\nu}) &\leq
T(r,f)+[m(r,f)+N(r,f)]-N(r,f)\\
&{}+m(r,f'/f)+\log 1/|f'(0)|+\log
  2\\
&{}+m\left[r,\sum^{q}_{\nu=1}f'/(f-a_{\nu})\right]
+q\log^{+}3q/\delta-N(r,1/f)\\
&{}+N(r,f'/f)-N(r,f/f').\\
&= 2T(r,f)-N_{1}(r)+m(r,f'/f)+\log
  \frac{2}{|f'(0)|}\\
&{}+q\log^{+}\frac{3q}{\delta}
+m\left[r,\sum^{q}_{\nu=1}\frac{f'}{(f,a_{\nu})}\right]
\end{align*}
where, $N_{1}(r)=N(r,f)+N(r,1/f)+N(r,f/f')-N(r,f'/f)$. Now for any two
functions $f(z)$ and $g(z)$ Jensen's formula gives
\begin{align*}
N(r,f/g)-N(r,g/f)
&=\frac{1}{2\pi}\int\limits^{2\pi}_{0}\log\left|\frac{g(re^{i\theta})}{f(re^{i\theta})}\right|d\theta-\log|g(0)/f(0)|\\
&=
\frac{1}{2\pi}\int\limits^{2\pi}_{0}\log|g(re^{i\theta})|d\theta-\log|g(0)|\\
&\quad
+\frac{1}{2\pi}\int\limits^{2\pi}_{0}\log\frac{1}{|f(re^{i\theta})|}d\theta+\log
|f(0)|\\
&= N(r,1/g)-N(r,g)+N(r,f)-N(r,1/f)
\end{align*}
Thus\pageoriginale
\begin{align*}
N_{1}(r) &= N(r,f)+N(r,1/f)+N(r,1/f')\\
&{\;}+N(r,f)-N(r,f')-N(r,1/f)\\
&= 2N(r,f)+N(r,1/f')-N(r,f')
\end{align*}
as required.

\subsection{Estimation of the error term}\label{part2-sec2.2}

We shall firstly prove some lemmas.

\setcounter{lem}{0}
\begin{lem}\label{part2-lem1}
Suppose that $f(z)$ is meromorphic in the $|z|\leq R$ and that
$0<r<R$. Let $\rho=\frac{1}{2}(r+R)$ and $\delta(z)$ the distance of
$z$ from the nearest pole or zero of $f(z)$ in $|z|<\rho$. Then,
\begin{align*}
m(r,f'/f)<\log^{+}T(R,f) &+ 2\log^{+}[1/(R-r)]+2\log^{+}R\\
&+ \frac{1}{2\pi}
\int\limits^{2\pi}_{0}\log^{+}\frac{1}{\delta(re^{i\theta})}d\theta+0(1) 
\end{align*}
$0(1)$ depending only on the behaviour of $f(z)$ at $z=0$.
\end{lem}

%page 049.JPG
\begin{proof}
We have by differentiation of Poisson formula as in theorem
\ref{part1-thm5},
\begin{align*}
f'(z)/f(z) &= \frac{1}{2\pi}\int\limits^{2\pi}_{0}\log |f(\rho
e^{i\phi})|\frac{2\rho e^{i\phi}}{(\rho e^{i\phi}-z)^{2}}d\phi\\
&{\,} +\sum_{\mu}\left[\frac{\ob{a}_{\mu}}{(\rho^{2}-\ob{a}_{\mu}z)}-\frac{1}{(a_{\mu}-z)}\right]+\sum_{\nu}\left[\frac{1}{(b_{\nu}-z)}-\frac{\ob{b}_{\nu}}{(\rho^{2}-\ob{b}_{\nu}z)}\right]
\end{align*}
where the sums as usual run over the zeros $a_{\mu}$ and poles
$b_{\nu}$ in $|z|<\rho$. From $|\rho^{2}-\ob{a}_{\mu}z|\geq
\rho^{2}-\rho|z|=\rho^{2}-r\rho=(\rho-r)\rho$ for $|z|=r$ we get
$\dfrac{|\ob{a}_{\mu}|}{|\rho^{2}-\ob{a}_{\mu}z|}\leq
\dfrac{\rho}{\rho^{2}-\rho r}=\dfrac{1}{(\rho-r)}$ and by definition
of $\delta(z)$
$$
\left|\frac{1}{(a_{\mu}-z)}\right|\leq \frac{1}{\delta(z)},
\left|\frac{1}{b_{\nu}-z}\right|\leq \frac{1}{\delta(z)}
$$
Hence\pageoriginale
\begin{gather*}
\left|\sum\left[\frac{\ob{a}_{\mu}}{(\rho^{2}-\ob{a}_{\mu}z)}-\frac{1}{(a_{\mu}-z)}\right]+\sum\left[\frac{1}{(b_{\nu}-z)}-\frac{\ob{b}_{\nu}}{\rho^{2}-b_{\nu}z}\right]\right|\\
\leq
    [n(\rho,f)+n(\rho,1/f)]\left[\frac{1}{\delta(z)}+\frac{1}{(\rho-r)}\right] 
\end{gather*}
We now estimate $n(\rho,f)$ and $n(\rho,1/f)$. For this we have,
$$
\int\limits^{R}_{\rho}n(t,f)dt/t\leq N(R,f)\leq T(R,f)
$$
and since $n(t,f)/t\geq n(\rho,f)/R$ for $t$ greater than $\rho$,
$\int\limits^{R}_{\rho}n(t,f)dt/t\geq n(\rho,f)(R-f)/R$. Therefore
$[n(\rho,f)](R-\rho)/R\leq T(R,f)$ or
$n(\rho,f)\leq RT(R,f)/(R-\rho)=2RT(R,f)/(R-r)$ since $2\rho=(R+r)$. 

Similarly $n(\rho,1/f)\leq
2RT(R,1/f)/(R-r)=2R[T(R,f)+0(1)]/(R-r)$. $0(1)$ depends only on the
behaviour of $f(z)$ at $z=0$. Thus
$$
n(\rho,f)+n(\rho,1/f)\leq \frac{4R}{(R-r)}[T(R,f)+0(1)]
$$
and so
\begin{gather*}
\left|\sum\left[\frac{\ob{a}_{\mu}}{\rho^{2}-\ob{a}_{\mu}z}-\frac{1}{(a_{\mu}-z)}\right]+\sum\left[\frac{1}{(b_{\nu}-z)}-\frac{\ob{b}_{\nu}}{(\rho^{2}-b_{\nu}z)}\right]\right|\\
\leq
\frac{4R}{(R-r)}[T(R,f)+0(1)]\left[\frac{1}{\delta(z)}+\frac{2}{(R-r)}\right] 
\end{gather*}
Further,
\begin{gather*}
\frac{1}{2\pi}\left|\int\limits^{2\pi}_{0}\log|f(\rho
e^{i\phi})|\frac{2\rho e^{i\phi}}{(\rho e^{i\phi}-z)^{2}}d\phi\right|\\
\leq \frac{1}{2\pi}\frac{2\rho}{(\rho-r)^{2}}\int\limits^{2\pi}_{0}\log |f(\rho e^{i\phi})|d\phi 
\end{gather*}\pageoriginale
for,
\begin{align*}
|\rho e^{i\phi}-z| &\geq ||\rho e^{i\phi}|-|z||=(\rho-r)\\
&= \frac{1}{\pi}\frac{4\rho}{(R-r)^{2}}\left[\int\limits^{2\pi}_{0}\log^{+}|f(\rho e^{i\phi})d\phi+\int\limits^{2\pi}_{0}\log^{+}\frac{1}{|f(e^{i\phi}\rho)|}d\phi\right]\\
&= \frac{8\rho}{(R-r)^{2}}[m(\rho,f)+m(\rho,1/f)]\leq \frac{8\rho}{(R-r)^{2}}[2T(\rho,f)+0(1)]\\
&= \frac{16\rho}{(R-r)^{2}}[T(\rho,f)+0(1)]\leq \frac{16R}{(R-r)^{2}}[T(R,f)+0(1)] 
\end{align*}
$0(1)$ depending only on the behaviour of $f(2)$ at $z=0$. Thus from
the above inequalities and the equation for $f'(z)/f(z)$ we get
finally, 
\begin{align*}
|f'(z)/f(z)| &\leq
\frac{4R}{(R-r)}[T(R,f)+0(1)]\left[\frac{2}{(R-r)}+\frac{1}{\delta(z)}\right]\\
&\quad +\frac{16R}{(R-r)^{2}}[T(R,f)+0(1)]\\
&=\frac{4R}{(R-r)^{2}}\left[6T(R,f)+T(R,
  f)\frac{(R-r)}{\delta(z)}+0(1)+0(1)\frac{R-r}{\delta(z)}\right]\\
&\leq
\frac{4R}{(R-r)^{2}}\left[6+\frac{R-r}{\delta(z)}\right][T(R, f)+0(1)]
\end{align*}
Hence,
\begin{align*}
& \log^{+}|f'(z)/f(z)|\leq
\log^{+}\frac{4R}{(R-r)^{2}}+\log^{+}\left(6+\frac{R-r}{\delta(z)}\right)+\log^{+}T(R,f)\\
&{}+0(1)\leq
\log^{+}R \,2\log^{+}\frac{1}{(R-r)}+\log^{+}T(R,f)\log^{+}\frac{(R-r)}{\delta(z)}+0(1)\\ 
&\leq
2\log^{+}R+\log^{+}\frac{1}{\delta(z)}+2\log^{+}\frac{1}{(R-r)}+\log^{+}T(R,f)+0(1) 
\end{align*}\pageoriginale
$0(1)$ depending only on the behaviour of $f(z)$ at $z=0$. Integrating
the above inequality on the circle $|z|=r$,
\begin{gather*}
\frac{1}{2\pi}\int\limits^{2\pi}_{0}\log|f'(re^{i\theta}) /
f(re^{i\theta})|d\theta\leq 2\log^{+}R+\frac{1}{2\pi}
\int\limits^{2\pi}_{0}\log^{+}\frac{1}{\delta(re^{i}\theta)}d\theta\\ 
+2\log^{+}\frac{1}{(R-r)}+\log^{+}T(R,f)+0(1)
\end{gather*}
which gives the lemma.
\end{proof}

\begin{lem}\label{part2-lem2}
Let $z$ be any complex number and $0<r<\infty$. Let $E_{k}$ be the set
of all $\theta$ such that $|z-re^{i\theta}|<kr$ where $0<k<1$. Then
$$
\int\limits_{E_{k}}\log[r/|(z-re^{i\theta})|]d\theta <\pi
\left[\log\left(\frac{1}{k}\right)+1\right] 
$$
\end{lem}

\begin{proof}
We may after rotation assume $z$ real and positive. For if $z=0$
$E_{k}$ is obviously void and there is nothing to prove. So let
$z>0$. Then for $\theta$ in $E_{k}\, |z-re^{i\theta}|\geq
|\Iim(z-re^{i\theta})|=r\sin\theta$ and $E_{k}$ is contained in an
interval of the form $|\theta|<\theta_{0}$ where $r\sin\theta_{0}\leq
kr$, that is, $\sin\theta_{0}\leq k$. So,
$$
\int\limits_{E_{k}}\log [r/|re^{i\theta}-z|]d\theta\leq
2\int\limits^{\theta_{0}}_{\theta}\log^{+}\frac{1}{\Sin\theta}d\theta
$$
Now, $\theta<\dfrac{\pi}{2}$ for when $\dfrac{\pi}{2}\leq |\theta|\leq
\pi$, $|z-re^{i\theta}|>|re^{i\theta}|=r$. Thus\pageoriginale
\begin{multline*}
\frac{\sin \theta}{\theta}\geq \dfrac{2}{\pi}|\theta|<\theta_{0}
\text{ \ we get \ } \int\limits_{E_{k}}\log
     [r/|re^{i\theta}-z|]d\theta\leq
     2\int\limits^{\theta_{0}}_{0}\log\frac{\pi}{2\theta}d\theta\\
     =2\int\limits^{\theta_{0}}_{0}\log\frac{\pi}{2\theta}d\theta
     \qquad 
\end{multline*}
because $\theta_{0}<\dfrac{\pi}{2}$ or $\dfrac{\pi}{2\theta}>1$.
\begin{align*}
\int\limits_{E_{k}}\log [r/|(re^{i\theta}-z)|]d\theta &\leq
2\int\limits^{\theta_{0}}_{0}\log\frac{\pi}{2}d\theta-2\int\limits^{\theta_{0}}_{0}\log\theta
d\theta\\
&=2\theta_{0}\log\frac{\pi}{2}-2\theta_{0}\log\theta_{0}+2\theta_{0}
\end{align*}
\end{proof}

\begin{lem}\label{part2-lem3}
With the hypothesis of lemma \ref{part2-lem1},
$$
m(r,f'/f)\leq
3\log^{+}T(R,f)+4\log^{+}R+4\log^{+}[1/(R-r)]+\log^{+}(1/r)+0(1)
$$
\end{lem}

\begin{proof}
Notations being the same as in the proof of lemma \ref{part2-lem1},
write, $[1/\delta(z)]=[r/\delta(z)]\cdot (1/r)$. Then
$\log\dfrac{1}{\delta(z)}\leq
\log\dfrac{1}{r}-\log\dfrac{r}{\delta(z)}$. So
$$
\frac{1}{2\pi}\int\limits^{2\pi}_{0}\log^{+}\frac{1}{\delta(re^{i\theta})}d\theta\leq
\log^{+}\frac{1}{r}+\frac{1}{2\pi}\int\limits^{2\pi}_{0}\log^{+}\frac{r}{\delta(re^{i\theta})}d\theta 
$$
Let $E$ denote the set of $\theta$ in $(0,2\pi)$ for which
$\delta(re^{i\theta})<\dfrac{r}{n^{2}}$ where
$n=n(\rho,f)+n(\rho,1/f)$. (If $n=0$ so that there are no zeros and
poles we can put $\delta(z)=+\infty$ and there is nothing to prove. So
assume $n\geq 1$). For each point $\theta$ in $E$ there is a zero or
pole $z_{\nu}$ such that
$\delta(re^{i\theta})=|z_{\nu}-re^{i\theta}|<r/n^{2}$\pageoriginale
and then
$$
\log^{+}[r/\delta(z)]=\log^{+}[r/|re^{i\theta}-z_{\nu}|]
$$
Now write $\log_{0}x=\log x$ if $x\geq n^{2}$ and $\log_{0}x=0$
otherwise. Then since $n\geq 1$, $\log_{0}x\geq 0$ always.

Also for $\theta$ in $E$
$\log^{+}[r/\delta(z)]=\log^{+}[r/|(re^{i\theta}-z_{\nu})|]$ for some
$\nu$ Since, $1<n^{2}<[r/|re^{i\theta}-z_{\nu}|]$,
$$
\log^+\frac{r}{|re^{i\theta}-z_{\nu}|}=\log_{0}\frac{r}{|re^{i\theta}-z_{\nu}|}\leq
\sum_{\mu}\log_{0}\frac{r}{|re^{i\theta}-z|} 
$$
where the sum is taken over all zeros and poles $z_{\mu}$ in
$|z|<\rho$. Thus 
\begin{align*}
\int\limits_{E}\log^{+}\frac{r}{\delta(re^{i\theta})}d\theta &\leq
\sum_{\mu}\int\limits_{E}\log_{0}\frac{r}{|re^{i\theta}-z_{\mu}|}d\theta\\
&\leq \sum_{\mu}\frac{\pi}{n^{2}}[\log n^{2}+1]=\frac{\pi}{n}[2\log
  n+1]\leq A.
\end{align*}
by lemma \ref{part2-lem2} with $k=1/n^{2}$, and where $A$ is an
absolute constant.

Also on the complement of $E$, $\delta(re^{i\theta})\geq r/n^{2}$ and, 
$$
\int\limits_{\text{compl. of }
  E}\log^{+}\frac{r}{\delta(re^{i\theta})}d\theta\leq
  \int\limits^{2\pi}_{0}\log n^{2}d\theta=2\pi\log n^{2}
$$
Adding we obtain
\begin{gather*}
\frac{1}{2\pi}\int\limits^{2\pi}_{0}\log^{+}\frac{r}{\delta(re^{i\theta})}d\theta\leq
     [2\log n+A]\\
\frac{1}{2\pi}\int\limits^{2\pi}_{0}\log^{+}\frac{1}{|\delta(re^{i\theta})|}d\theta=\frac{1}{2\pi}\int\limits^{2\pi}_{0}\log^{+}\frac{1}{r}\frac{r}{|\delta(re^{i\theta})|}d\theta\\
\leq
\frac{1}{2\pi}\int\limits^{2\pi}_{0}\log\frac{r}{|\delta(re^{i\theta})|}d\theta+\log^{+}\frac{1}{r}\\
\leq 2\log^{+}n+\log^{+}1/r+A.
\end{gather*}\pageoriginale
where $n=n(\rho,f)+n(\rho,1/f)$.

We have from $(2.2)$
$$
n=n(\rho,f)+n(\rho,1/f)\leq 4R[T(R,f)+0(1)]/(R-r).
$$
Hence,
$$
\log^{+}n\leq \log^{+}R+\log^{+}1/(R-r)+\log^{+}T(R,f)+0(1),
$$
giving
\begin{align*}
\frac{1}{2\pi}\int\limits^{2\pi}_{0}\log
\frac{1}{|\delta(re^{i\theta})|}d\theta &\leq
\log^{+}\frac{1}{r}+2\log^{+}\frac{1}{R-r}+2\log^{+}R\\
&+2\log^{+}T(R,f)+0(1)
\end{align*}
From this and lemma \ref{part2-lem1}, lemma (\ref{part2-lem3})
follows. That is,
$$
m(r,f'/f)<3\log^{+}T(R,f)+4\log^{+}1/(R-r)+4\log^{+}R+\log^{+}1/r \,0(1).
$$
\end{proof}

\begin{lem}[Borel]\label{part2-lem4}
\begin{enumerate}
\renewcommand{\theenumi}{\roman{enumi}}
\renewcommand{\labelenumi}{\rm(\theenumi)}
\item Suppose $T(r)$ is continuous, increasing and $T(r)\geq 1$ for
  $r_{0}\leq r<\infty$. Then we have 
\begin{equation*}
T[r+1/T(r)]<2T(r)\tag{2.3}\label{part2-eq2.3}
\end{equation*}
outside a set $E$ of $r$ which has length (that is) linear measure at
most $2$.

\item If\pageoriginale $T(r)$ is continuous and increasing for
  $r_{0}\leq r<1$ and 
  $T(r)\geq 1$ then we have
\begin{equation*}
T[r+(1-r)/(eT(r))]<2T(r)\tag{2.4}\label{part2-eq2.4}
\end{equation*}
outside a set $E$ of $r$ such that $\int\limits_{E}dr/(1-r)\leq 2$. In
particular $T[r+(1-r)/eT(r)]<2T(r)$ for some $r$ such that
$\rho<r<\rho'$ if $r_{0}<\rho<1$ and $1-\rho'<(1-\rho)/e^{2}$.
\end{enumerate}
\end{lem}

\begin{proof}
We prove first (i) [that is in the plane]. Let $r_{1}$ be the lower
bound of all $r$ for which \eqref{part2-eq2.3} is false. If there are
no such $r$ there is nothing to prove. We now define by induction a
sequence of numbers $r_{n}$. Suppose that $r_{n}$ has been defined and
write $r'_{n}=r_{n}+1/T(r_{n})$. Define then $r_{n+1}$ as the lower
bound of all $r\geq r'_{n}$ for which \eqref{part2-eq2.3} is false. We
have already defined $r_{1}$ and so we obtain the sequence
$(r_{n})$. Note that by continuity of $T(r)$ \eqref{part2-eq2.3} is
false for $r=r_{n}$, for $n=1,2,3,\ldots$ that is $r_{n}$ belongs to
$E_{0}$, $E_{0}$ being the exceptional set. From the definition of
$r_{n+1}$ is follows that there are no points of $E$ in
$(r'_{n},r_{n+1})$ and that the set of closed intervals
$[r_{n},r'_{n}]$ contains $E_{0}$. If there are an infinity of
$r_{n}$, $(r_{n})$ cannot tend to a finite limit $r$. For then since
$r_{n}<r'_{n}\leq r_{n+1}$, $r'_{n}$ tends to $r$ also. But
$r'_{n}-r_{n}=1/T(r_{n})$ which is greater or equal to $1/T(r)>0$,
since $T(r)$ is increasing, for all $m$ which is a contradiction.
\end{proof}

It remains to be shown that $\sum(r'_{n}-r_{n})\leq 2$. Now
$T(r'_{n})=T[r_{n}+1/T(r_{n})]\geq 2T(r_{n})$ since $r_{n}$ belongs to
$E$. And so $T(r_{n+1})\geq T(r'_{n})\geq 2T(r_{n})$. Therefore,
$T(r_{n+1})\geq 2T(r_{n})\geq \ldots \geq 2^{n}T(r_{1})\geq
2^{n}$\pageoriginale since $T(r)\geq 1$. Thus $T(r_{n})\geq
2^{n-1}$. Now $\sum(r'_{n}-r_{n})=\sum 1/T(r_{n})\leq \sum 2^{1-n}\leq
2$.

To prove part (ii) of the theorem, set $R$ $\log [1/(1-r)]$ getting
$r=1-e^{-R}$ and put $T(r)=\varphi(R)\cdot \varphi(R)$ then is
continuous and increasing for $R_{0}=\log[1/(1-r_{0})]\leq R<\infty$
and $\varphi(R)\geq 1$. Apply then the first part to
$\varphi(R)$. Then we have $\varphi[R+1/\varphi(R)]<2\varphi(R)$ for
$R>\log [1/(1-r_{0})]$ outside a set $E$ of $R$ such that $2\geq \sum
(R'_{n}-R_{n})=\int\limits_{E}dR=\int dr/(1-r)$. Translating back to
$r$, $R'=R+1/\varphi(R)$ becomes
$\log[1/(1-r')]=\log[1/(1-r)]+1/T(r)$. That is
$(1-r')=(1-r)e^{-\frac{1}{T(r)}}$ and $T(r')<2T(r)$. By the first mean
value theorem $f(b)=f(a)+(b-a)f'(x)$, where $a\leq x\leq b$. Since
$T(r)$ increases with $r$, $T[r+(1-r)/eT(r)]<2T(r)$ outside the
exceptional set $E$ of $r$ for which $\int\limits_{E}dr/(1-r)\leq
2$. If $E$ contains the whole of the interval $\rho<r<\rho'$ then
$\int^{\rho'}_{\rho}dr/(1-r)\leq \int\limits_{E}dr/(1-r)\leq 2$, that
is $\log(1-\rho)/(1-\rho')\leq 2$, and so $(1-\rho)/(1-\rho')\leq
e^{2}$ as required.

\subsection{}\label{part2-sec2.3}

\begin{thm}\label{part2-thm9}
If $f(z)$ is meromorphic and non-constant in the plane and $S(r,f)$
denotes the error term in Theorem \ref{part2-thm8} then we have
\begin{equation*}
S(r,f)=O[\log T(r,f)]+O(\log r)\tag{2.5}\label{part2-eq2.5}
\end{equation*}
as $r$ tends to infinity through all values if $f(z)$ has finite
order, and through all values outside a set of finite linear measure
otherwise. 

{\rm(ii)}~If $f(z)$ is meromorphic (non-constant) in $|z|<1$ and the\break
$\mathop{\ob{\lim}}\limits_{r\to 1}\{T(r,f)/\log[1/(1-r)]\}=\infty$,
then we have $S(r,f)=O[T(r,f)]$ as $r$ tends to one on a set $E$ such
that $\int\limits_{E}dr/(1-r)=\infty$. 
\end{thm}

\begin{proof}
If $\varphi(z)=\prod\limits^{q}_{\nu=1}[f(z)-a_{\nu}]$ then
$$
S(r,f)=m(r,f'/f)+m[r,\varphi'/\varphi]+O(1)
$$
because\pageoriginale
$S(r,f)=m(r,f'/f)+m(r,\sum\limits^{q}_{\nu=1}f'/(f-a_{\nu}))+O(1)$ and
$\sum\limits^{q}_{\nu=1}f'/(f-a_{\nu})=\dfrac{\varphi'}{\varphi}$ by
logarithmic differentiation. Therefore from the lemma \ref{part2-lem3},
for any $R>r$ we have,
\begin{align*}
S(r,f) &\leq 3\log^{+}T(R,f)+4\log^{+}R+4\log^{+}[1/(R-r)]\\
&\quad +\log^{+}(1/r)+3\log^{+}T(R,\varphi)\\
&\quad 4\log^{+}R+4\log^{+}1/(R-r)+\log^{+}(1/r)+0(1)
\end{align*}
Also
\begin{equation*}
T(r,\varphi)\leq \sum^{q}_{\nu=1}T(r,f-a_{\nu})\leq
T(r,f)+O(1),\ldots\tag{2.6}\label{part2-eq2.6} 
\end{equation*}
Thus 
\begin{gather*}
S(r,f)\leq 3(1+q)\log^{+} T(R,f)+8\log^{+}R+8\log^{+}[1/(R-r)]\\
+2\log^{+}(1/r)+O(1)
\end{gather*}
If $r$ is greater than $1$ (we are considering $S(r,f)$ for large $r$)
$\log^{+}(1/r)=0$. Now suppose that $f(z)$ is meromorphic of finite
order so that $T(r,f)<r^{K}$ for $r$ greater than $r_{0}$. Also choose
$R=r^{2}$ and $r\geq 2$ so that $R-r>1$.

Then $\log^{+}\left(\dfrac{1}{R-r}\right)=0$. We get
\begin{gather*}
S(r,f)\leq 3(1+q)\log^{+}T(R,f)+8\log^{+}R+ O(1),\\
\log^{+}T(R,f)<k\log^{+}R<2k\log^{+}r=2k\log r
\end{gather*}
since $R=r^{2}$ and $r>1$. We thus have finally, $S(r,f)\leq
6(1+q)K\log r+16\log r+O(1)$ showing that $S(r,f)=0(\log r)$ which
gives \eqref{part2-eq2.5} since $\log^{+}T(r,f)=O(\log r)$.
\end{proof}

Note\pageoriginale that by our examples $T(r,f)/\log r$ tends to
infinity unless $f(z)$ is a rational function in which case $S(r,f)$
is bounded because $f'/f\to 0$ as $z\to \infty$ for any polynomial and
hence for any rational function. Thus if $f(z)$ has finite order
$S(r,f)/T(r,f)\to 0$ as $r\to \infty$. 

If $f(z)$ has infinite order take $R=[r+1/T(r)]$, then $\log^{+}R\sim
\log r[\text{since } T(r)\to \infty]\cdot
\log^{+}1/(R-r)=\log^{+}T(r)$ and $\log 1/r=0$ finally. Outside the
exceptional set of lemma \ref{part2-lem4}, $T(R,f)<2T(r,f)$ and so,
$\log^{+}T(R,f)\leq \log^{+}T(r,f)+\log 2$. Hence again we have
\eqref{part2-eq2.5} outside the exceptional set. This completely
proves i) In order to prove (ii) denote by $r_{n}$ a sequence such
that $T(r_{n},f)/\log\left(\dfrac{1}{1-r_{n}}\right)\to \infty$ as
$n\to \infty$ and by taking a sub sequence if necessary we can assume
that $1-r_{n+1}<\dfrac{1-r_{n}}{10}$. Then let $r_{n}$ be defined by
$1-r'_{n}=(1-r_{n})/10$ so that $r_{n}<r'_{n}<r_{n+1}<1$. Then since
$\int\limits^{r'_{n}}_{r_{n}}\dfrac{dr}{1-r}=\log
(1-r_{n})/(1-r_{n})=\log 10>2$ the union $E_{1}$ of all the intervals
$(r_{n},r'_{n})$ is such that
$\int\limits_{E_{1}}\dfrac{dr}{1-r}=+\infty$.

{\bf Further each such interval contains a point not in the
  exceptional} set $E$, for $T(r,f)$ because by lemma \ref{part2-lem4}
$\int\limits_{E}dr/(1-r)\leq 2$, provided only that $T(r_{1})>1$. For
a not exceptional point $r$ of $E_{1}$ take $R=r+(1-r)/eT(r)$, then 
$$
\log^{+}\frac{1}{R-r}=\log^{+}\frac{eT(r)}{1-r}<\log^{+}T(r)+\log^{+}\frac{1}{1-r}+1 
$$
and\pageoriginale\ $\log^{+}T(R)<\log^{+}T(r)+\log 2$, Thus by
\eqref{part2-eq2.6} 
\begin{align*}
S(r,f)&<3(1+q)\log^{+}T(R,f)+8\log^{+}R+8\log^{+}\dfrac{1}{(R-r)}+O(1)\\
&<(3+3q+8)\log^{+}T(r)+8\log^{+}\frac{1}{(1-r)}+O(1)
\end{align*}
Also,
$\log^{+}[1/(1-r)]<\log^{+}1/(1-r'_{n})+\log^{+}
\frac{10}{(1-r_{n})}<\log^{+}\frac{1}{1-r_{n}}+0(1)
=O[T(r_{n})]+O(1)=O[T(r)]$.  

So since $\log T(r,f)=O$ $T(r)$ we get $S(r,f)=O$ $T(r)$. This proves
(ii) for a set $E_{1}$ of $r$ such that
$\int\limits_{E_{1}}\dfrac{dr}{1-r}=+\infty$ and containing at least
one point in each interval $r_{n}<r<r'_{n}$. In face $E$ comprises all
the $r$ in the sequence of intervals $[r_{n},r'_{n}]$, $n>1$ except
possibly a set $E_{0}$ such that
$\int\limits_{E_{0}}\dfrac{dr}{1-r}\leq 2$.

\subsection{Applications}\label{part2-sec2.4}\pageoriginale

\begin{defi*}
Let $n(t,a)$ denote the number of roots of $f(z)=a$ in $|z|\leq t$,
the multiple roots being counted according to their multiplicity and
$\ob{n}(t,a)$ the number of roots of $f(z)=a$ in $|z|<t$ with the
multiple roots counted simply. Further let
$\ob{N}(t,a)=\int\limits^{r}_{0}\ob{n}(t,a)-\ob{n}(0,a)dt/t$. [\,$\ob{n}(0,a)$
  is equal to one if $f(0)=a$ and zero otherwise]; and $N(r,a)$
as before with $n(t,a)$ instead of $\ob{n}(t,a)$. Let now the function
$f(z)$ be meromorphic, and non-constant in $|z|<R$, $0<R\leq \infty$
and suppose that $f(z)$ satisfies the hypotheses of theorem
\ref{part2-thm9}, so that $\mathop{\ub{\lim}}\limits_{r\to
  R}[S(r,f)/T(r)]=0$ and $T(r)$ tends to infinity as $r$ tends to $R$.
\end{defi*}

Now write $\delta(a)=\mathop{\ub{\lim}}\limits_{r\to
  R}[m(r,a)/T(r)]=1-\mathop{\ob{\lim}}\limits_{r\to R}[N(r,a)/T(r)]$
because
$$
[m(r,a)+N(r,a)]/T(r)=[T(r)+0(1)]/T(r)\to 1;
$$
thus
\begin{align*}
\mathop{\ub{\lim}}\limits_{r\to
  R}[m(r,a)/T(r)]=\mathop{\ub{\lim}}\limits_{r\to
  R}1+\frac{-N(r,a)}{T(r)} &= 1+\mathop{\ub{\lim}}\limits_{r\to
  R}-N(r,a)/T(r)\\
&= 1-\mathop{\ob{\lim}}\limits_{r\to R}N(r,a)/T(r)
\end{align*}
Again write,
$$
\theta(a)=\mathop{\ub{\lim}}\limits_{r\to R}[N(r,a)-\ob{N}(r,a)]/T(r)
$$
and
$$
\Theta(a)=1-\mathop{\ob{\lim}}\limits_{r\to
  R}\ob{N}(r,a)/T(r)=\mathop{\ub{\lim}}\limits_{r\to
  R}[1-\ob{N}(r,a)/T(r)].
$$
Clearly, $\delta(a)$, $\theta(a)$ and $\Theta(a)$ lie in the closed
interval $[0,1]$. Also
\begin{gather*}
1-\frac{\ob{N}}{T}=1-\frac{N}{T}+\frac{N-\ob{N}}{T}\\
\mathop{\ub{\lim}}\limits_{r\to R} 1-\frac{\ob{N}}{T}\geq
\mathop{\ub{\lim}}\limits_{r\to
  R}\left(1-\frac{N}{T}\right)+\mathop{\ub{\lim}}\limits_{r\to R}\left(\frac{N-\ob{N}}{T}\right)
\end{gather*}\pageoriginale
\ie $\Theta(a)\geq \delta(a)+\theta(a)$.

The quantity $\delta(a)$ is called the {\em defect of $a$} and
$\theta(a)$ the {\em Branching index} (Verzweigungsindex) of $a$. Now
we have the defect relation of Nevanlinna. This is the second
fundamental theorem in its most effective form and is very important
in the theory.

\begin{thm}\label{part2-thm10}
If $f(z)$ satisfies the hypotheses of the Theorem \ref{part2-thm9},
then $\Theta(a)=0$ except possibly for a finite or countable sequence
$a_{\nu}$ of values of $a$ and for these $\sum \Theta(a_{\nu})\leq 2$.
\end{thm}

\begin{proof}
From theorem \ref{part2-thm8}, for any $q$ distinct values $a_{\nu}$, of a
including $a_{1}=\infty$
$$
\sum^{q}_{\nu=1}m(r,a_{\nu})<2T(r,f)-N_{1}(r)+S(r)
$$
and adding $\sum\limits^{q}_{\nu=1}N(r,a_{\nu})$ to both sides and
using first fundamental theorem $T(r,a)=T(r,f)+O(1)$, we get
\begin{align*}
& qT(r,f)<2T(r,f)-N_{1}(r)+\sum\limits^q_{\nu=1}N(r,a_{\nu})+S(r,f)+0(1)\\
& (q-2)T(r,f)<\sum^{q}_{\nu=1}N(r,a_{\nu}) - N_{1}(r)+S(r)+0(1).
\end{align*}
Now, $N_{1}(r)=2N(r,f)-N(r,f')+N(r,1/f')$, and since by definition
$N(r,f)=N(r,\infty)$,
$N(r,\infty)-N_{1}(r)=N(r,a_{1})-N_{1}(r)=N(r,f')-N(r,f)-N(r,1/f')$.

So,\pageoriginale
$$
(q-2)T(r,f)<\sum^{q}_{\nu=1}N(r,a_{\nu})+N(r,f')-N(r,f)-N(r,1/f')+S(r)+0(1)
$$
If $f(z)$ has a pole of order $p$ at $z_{0}$, $f'(z)$ has a pole of
order $p+1$ at $z_{0}$ so that $n(t,f')-n(t,f)=\ob{n}(t,f)$ and so
$N(r,f')-N(r,f)=\ob{N}(r,f)$,. Similarly if $a$ is anyone of $a_{2}$,
$a_{3},\ldots , a_{q}$ (finite) and $f(z)=a$ has a root of multiplicity
$p$, $f'(z)$ has there a zero of order $p-1$. Thus
$$
\sum^{q}_{\nu=2}N(r,a_{\nu})-N(r,1/f')=\sum^{q}_{\nu=2}\ob{N}(r,a_{\nu})-N_{0}(r,1/f') 
$$
where $N_{0}(r,1/f')$ refers to zeros of $f'(z)$ at points other than
the roots of $f(z)=a$.

Hence we get,
\begin{align*}
(q-2)T(r,f) &<
  \sum^{q}_{\nu=2}\ob{N}(r,a_{\nu})-N_{0}(r,1/f')+S(r)+\ob{N}(r,\infty)+0(1)\\
\text{\ie \ } (q-2)T(r,f) &<
\sum^{q}_{\nu=1}\ob{N}(r,a_{\nu})-N_{0}(r,1/f')+S(r,f)+0(1)\\
\text{\ie \ } \sum^{q}_{\nu=1}\frac{N(r,a_{\nu})}{T(r)} &\geq
(q-2)-\frac{S(r,f)+0(1)}{T(r)}\text{ \ since \ } N_{0}(r,1/f')\geq 0
\end{align*}
since $\mathop{\ub{\lim}}\limits_{r\to R}S(r,f)/T(r)=0$ and $T(r,f)\to
\infty$ as $r\to R$ 
$$
\mathop{\ob{\lim}}\limits_{r\to
  R}\left[-\frac{O(1)+S(r)}{T(r)}\right]=0. \text{ \ Thus \ }
\mathop{\ob{\lim}}\limits_{r\to
  R}\sum^{q}_{\nu=1}\frac{N(r,a_{\nu})}{T(r)}\geq (q-2)
$$
and afortiori 
$$
\sum^{q}_{\nu=1}\mathop{\ob{\lim}}\limits_{r\to R}\frac{N(r,a_{\nu})}{T(r)}\geq q-2 
$$\pageoriginale
that is,
$$
\sum^{q}_{\nu=1}[1-\Theta(a_{\nu})]\geq (q-2)\quad\text{as required.}
$$
This shows that $\Theta(a)>(1/n)$ at most for $2n$ values of $a$ and
so $\Theta(a)>0$ for at most a countable number of $a$. If these $a$'s
are arranged in a sequence $\sum\limits^{q}_{r=1}\Theta(a_{r})\leq 2$
for any finite $q$, and so to infinity. 
\end{proof}

\subsubsection{Consequences}\label{part2-subsubsec2.4.1}

\begin{enumerate}
\renewcommand{\labelenumi}{(\theenumi)}
\item Since $\delta(a)\leq \Theta(a)$ we have $\sum
  \delta(a_{\nu})\leq 2$ and thus there exists at most two values of $a$
  for which $\delta(a)=1$, or more generally
  $\delta(a)>\frac{2}{3}$. Thus if the equation $f(z)=a$ has only a
  finite number of roots in the plane, $N(r,a)=0(\log r)$ as $r$ tends
  to infinity, and we should have
$$
\mathop{\ob{\lim}}\limits_{r\to \infty}\log[r/T(r)]>0,\text{ \ \ie \ }
\mathop{\ub{\lim}}\limits_{r\to \infty}\frac{T(r)}{\log r}<\infty.
$$
\iec $f(z)$ is rational. Thus if $f(z)$ is transcendental in the plane
$\delta(a) <1$ the equation $f(z)=a$ has infinitely many roots. The
same is true in all cases if $R$ is finite, since otherwise
$N(r,a)=0(1)$ and so $\mathop{\ub{\lim}}\limits_{r\to R}T(r)<+\infty$
that is $T(r)=0(1)$ as $r$ tends to $R$, since $T(r)$ is
increasing. This would contradict 
$$\mathop{\ob{\lim}}\limits_{r\to
  R}T(r)/\log\left(\dfrac{1}{1-r}\right)=+\infty.$$ 
This result thus contains Picard's theorem as a special case. 

\item $\Theta$\pageoriginale {\em in relation to $N$ and $\ob{N}$.}
  Suppose that the 
  function $f(z)=a$ has only multiple roots of multiplicity $k\geq
  2$. Then 
$$
\ob{N}(r,a)\leq (1/k)N(r,a)\leq (1/k)[T(r,f)+0(1)]
$$
Hence in this case,
\begin{gather*}
\mathop{\ob{\lim}}\limits_{r\to R}\ob{N}(r,a)/T(r)\leq (1/k)\\
\Theta(a)=1-\mathop{\ob{\lim}}\limits_{r\to R}\ob{N}(r,a)/T(r)\geq
1-(1/k)\geq \frac{1}{2}
\end{gather*}
If $a_{1}$, $a_{2},\ldots,a_{q}$ are $q$ such values with $k=k_{\nu}$
for $a_{\nu}$, we have since $\sum\limits_{\nu}\Theta(a_{\nu})\leq 2$,
$\sum\limits_{\nu}[1-(1/k_{\nu})]\leq 2$.
\end{enumerate}

In particular there can be at the most four such values $a_{\nu}$ of $a$
for $[1-(1/k_{\nu})]\geq \frac{1}{2}$.

If $f(z)$ is regular $m(r,f)=T(r,f)$ and
$\delta(\infty)=\ub{\lim}m(r,f)/T(r,f)=1$. and so $\Theta(\infty)=1$
because $\Theta(\infty)\leq \delta(\infty)=1$. So,
$\sum\limits_{\nu}\Theta(a_{\nu})\leq 1$ for any finite number of
finite $a_{\nu}$'s. So that there can be only two such values
$a_{\nu}$ which are taken multiply. Such values are called fully
branched (Vollst\"andig Verzweight). These results are best possible,
for $\sin z$ and $\cos z$, have the values $\pm  1$ ``fully
branched''. Again for the Wierstrassian elliptic function $P(z)$ which
satisfies the differential equation
$$
[P'(z)]^{2}=(P(z)-e_{1})(P(z)-e_{2})(P(z)-e_{3})
$$
where $e_{1}$, $e_{2}$, $e_{3}$ are distinct finite numbers the values
$e_{1}$ $e_{2}$ and $e_{3}$ are evidently fully branched and so is
infinity. If $P(z)$ has a pole of order $k$, $P(z)\sim A(z-\zeta)^{-k}$
and $[P'(z)]^{2}\sim P(z)^{3}\sim A^{3}(z-\zeta)^{-3k}$ 
from\pageoriginale the differential equation.

But 
$$
[\mathscr{P}'(z)]^{2}\sim A'z^{-2k-2}\quad\text{\iec}\quad -2k-2=-3k
$$
Hence we have $k=2$ so all the poles are double and infinity is fully
branched.

We also note that the equation $w^{2}=\prod^{q}_{\nu=1}(z-a_{\nu})$
can have no parametric solution $z=\rho(t)$, $w=\psi(t)$ which are
integral functions of $t$ if $q\geq 3$ or which are meromorphic if
$q\geq 5$. Because if $\varphi(t_{0})=a_{1}$ for instance then
$w^{2}=\psi^{2}(t)$ has a zero at $t_{0}$ and so $\psi(t)$ has a zero
and $\psi^{2}(t)$ has at least a double zero at $t_{0}$. Hence also
$\varphi(t)^{-a_{1}}$ has at least a double zero at $t_{0}$ and all
roots of $\varphi(t)=a_{\nu}$ therefore will be multiple and
$\sum\limits^{q}_{\nu=1}\Theta(a_{\nu})\geq 2$ for $\varphi(t)$, a
contradiction.

We just remark that this result extends to a general equation\break
$g(z,w)=0$ of genus greater than $1$.

\subsubsection{}\label{part2-subsubsec2.4.2}\pageoriginale

\begin{thm}\label{part2-thm11}
Suppose $f(z)$ is meromorphic of finite order in the plane and
$\Theta(a_{1})=\Theta(a_{2})=1$, $a_{1}\neq a_{2}$. Then if $a$ is not
equal to $a_{1}$, $a_{2}$, $\ob{N}(r,a)\sim T(r)$ as $r$ tends to infinity.
\end{thm}

\begin{proof}
In fact we have if $f(z)$ is not rational
$$
T(r,f)<\ob{N}(r,a_{1})+\ob{N}(r,a_{2})+\ob{N}(r,a)+S(r,f)+O(1)
$$
and in this case $S(r,f)=O[T(r)]$, (even $S(r)=0(\log r)$). By
hypothesis since $\Theta(a_{i})=1-\ob{\lim}[\ob{N}(r,a)/T(r)]$,
$N(r,a_{i})=0[T(r)]\,i=1,2$. Therefore $[1+0(1)]T(r,f)<\ob{N}(r,a)$
as $r\to \infty$.

That is $\mathop{\ub{\lim}}\limits_{r\to \infty}[\ob{N}(r,a)/T(r)]\geq
1$, and evidently $\mathop{\ob{\lim}}\limits_{r\to
  \infty}\dfrac{\ob{N}(r,a)}{T(r)}\leq 1$ that is,
$\ob{N}(r,a)/T(r)\to 1$, as $r$ tends to infinity. Similarly since
$N(r,a)\geq \ob{N}(r,a)$, $\mathop{\ub{\lim}}\limits_{r\to
  \infty}N(r,a)/T(r)\geq 1$ and again $N(r,a)\sim T(r)$. If $f(z)$ is
rational these results follow by elementary methods; in fact in this
case there is at most one $\ub{a}$ namely $a=f(\infty)$ for which
$N(r,a)=OT(r)$ unless $f$ is a constant.
\end{proof}

\begin{thm}\label{part2-thm12}
If $a_{\nu}(z)$ for $\nu=1,2,3$ are three functions satisfying\break
$T(r,a_{\nu}) =O[T(r,f)]$ as $r\to R$, and $f(z)$ is as in theorem
\ref{part2-thm10}, then we have
$$
[1+O(1)]T(r,f)<\sum^{3}_{i=1}N\left(r,\frac{1}{f-a_{i}(z)}\right)
$$
as $r$ tends to $R$ through a suitable sequence of values.
\end{thm}

\begin{proof}
Set\pageoriginale
$$
\varphi(z)=\frac{f(z)-a_{1}(z)}{f(z)-a_{3}(z)} \;\;
\frac{a_{2}(z)-a_{3}(z)}{a_{2}(z)-a_{1}(z)}  
$$
It is easy to see using $T(r,a_{i})=O[T(r,f)]$ that $T(r, \ )=T(r,
f)=[1+O(1)]T(r,f)$, also $\varphi=0$, $1$, $\infty$, only if
$f-a_{1}(z)$, $f-a_{2}(z)$, $f-a_{3}(z)$ are zero or if two suitable
$a$'s are equal.
\begin{align*}
N\left[r,\frac{1}{a_{2}-a_{3}}\right] &\leq
T[r,1/(a_{2}-a_{3})]=T[r,a_{2}-a_{3}]+O(1)\\
&\leq T(r,a_{2})+T(r,a_{3})+O(1)=O[T(r)]
\end{align*}
So, $N(r,1/\varphi)\leq N\left(r,\dfrac{1}{f-a_{1}}\right) +
O[T(r,f)]$ etc.\@ by Jensen's formula and hypothesis. 
\end{proof}

In the course of the proof of theorem \ref{part2-thm10} we obtained,
\begin{align*}
(q-2)T(r,f) &< \sum^{q}_{i=1}N(r,a_{i})-N_{0}(r,1/f')+S(r,f)+O(1)\\
&< \sum^{q}_{i=1}N(r,a_{i})+S(r,f)+O(1).
\end{align*}
Take $f=\varphi$, and $a_{1}=\infty$, $a_{2}=0$, $a_{3}=1$ to get
$$
T(r,\varphi)<N(r,\varphi) + N\left(r,\frac{1}{\varphi}\right) +
N\left(r,\dfrac{1}{\varphi-1}\right)+O[T(r,\varphi)]  
$$
That is,
$$ 
[1+O(1)] T(r,\varphi)<N(r,\varphi) + N\left(r,\frac{1}{\varphi}\right)
+ N\left(r,\frac{1}{\varphi-1}\right) 
$$
for a suitable sequence of $r$ tending to $R$. \; $\varphi=0$, only if
either $f=a_{1}$, or $a_{2}=a_{3}$. So
$N\left(r,\dfrac{1}{\varphi}\right)\leq
N\left(r,\dfrac{1}{f-a_{1}}\right)+N\left(r,\dfrac{1}{a_{2}-a_{3}}\right)$
which\pageoriginale is equal to
$N\left(r,\dfrac{1}{f-a_{1}}\right)+O[T(r)]$. Similar reasoning gives
$N\left(r,\dfrac{1}{\varphi}\right)+N(r,\varphi)+N\left(r,\dfrac{1}{\varphi-1}\right)<\sum\limits^{3}_{i=1}N\left(r,\dfrac{1}{f-a_{i}}\right)+O[T(r)]$. Thus
since $T(r,\varphi)=[1+O(1)]T(r)$ the result follows.

\begin{remark*}
Note that the same reasoning cannot be applied to more than three
functions. In fact it is not known whether the analogous result is
still true if we take more than three functions. 
\end{remark*}

\subsection[Picard values of meromorphic...]{Picard values of meromorphic functions and their
  derivatives:}\label{part2-sec2.5}\pageoriginale 

A function $f(z)$ will be called admissible if it satisfies the
hypothesis of theorem \ref{part2-thm9} in a circle $|z|<R$ and also
$f(z)$ is transcendental if $R=\infty$.

Note that if $f^{(l)}(z)$ $l$th derivative of $f(z)$, all the poles of
$f^{(1)}(z)$ have multiplicity at least $l+1$. Therefore,
$\ob{N}[r,f^{(l)}]\leq N(r,f^{(l)})/(l+1)$ and 
$$
\Theta(\infty)=\ub{\lim}\left(1-\frac{N(r,f^{(l)})}{T(r,f^{(l)})}\geq
\ub{\lim} \right)
1-\frac{1}{(l+1)}\frac{N(r,f^{(l)})}{T(r,f^{(l)})}\geq \frac{l}{(l+1)} 
$$
We obtain for $f^{(l)}(z)$, since
$\sum\limits^{q}_{0}\Theta(a_{\nu})\leq 2$, if $a_{1},\ldots,a_{q}$,
are distinct and finite, $\sum\limits^{q}_{1}\Theta(a_{\nu})\leq
1+\dfrac{1}{l+1}$. Thus there can be at most one finite value which is
taken only a finite number of times or more generally for which
$\Theta(a)>\dfrac{3}{4}$. 

Write now $\psi(z)=a_{0}(z)+f(z)+\cdots+a_{l}(z)f^{l}(z)a_{i}(z)$
being functions satisfying $T[r,a_{i}(z)]=OT(r,f)$ and we assume
$\psi(z)$ is not identically constant. Then we have the following
sharpened form of a theorem of Milloux.

\begin{thm}\label{part2-thm13}
If $f(z)$ is admissible in $|z|<R$ then,
$$
T(r,f)<\ob{N}(r,f)+N(r,1/f)+\ob{N}\left[r,\frac{1}{\psi-1}\right]-N_{0}(r,1/\psi')+S_{1}(r,f) 
$$
where $N_{0}(r,1/\psi')$ indicates that zeros of $\psi'$ corresponding
to the repeated roots of $\psi=1$ are to be omitted, and
$\mathop{\ub{\lim}}\limits_{r\to R}S_{1}(r,f)/T(r,f)=0$. Note that this
reduces to the second fundamental theorem for $q=3$ when $\psi=f$.
\end{thm}

\subsubsection{}\label{part2-subsubsec2.5.1}\pageoriginale

Firstly let us prove some lemmas.

\begin{lem}\label{part2-lem5}
If $l$ is a positive integer and $f(z)$ admissible in $|z|<R$, such
$0<r<\rho<R$ then,
{\fontsize{9}{11}\selectfont
$$
m\left[r,\frac{f^{(l)}}{f}\right]<A(l)\left[\log^{+}T(\rho,f)+\log^{+}\rho+\log^{+}\frac{1}{\rho-r}+\log^{+}(1/r)+O(1)\right] 
$$}
$A(l)$ being a constant depending only on $l$.
\end{lem}

\begin{proof}
The proof is by induction. The result is true for $l=1$, by Lemma
\ref{part2-lem3} of section \ref{part2-sec2.1}. Take
$\rho_{1}=\frac{1}{2}(\rho+r)$ and assume that result for $l$. Now,
\begin{align*}
T[\rho_{l},f^{1}(z)] &= m[\rho_{1},f^{l}(z)]+N[\rho_{1},f^{l}(z)]\\
&\leq m[\rho_{1},f]+m(\rho_{1},f^{(l)}/f)+(l+1)N(\rho_{1},f) 
\end{align*}
since at a pole of $f$ of order $k$, $f^{(l)}$ has a pole of order
$k+l\leq k(l+1)$. By our induction hypothesis the right hand side is
less than
\begin{align*}
&(l+1)T(\rho_{1},f)+A(l)\left[\log^{+}T(\rho,f)+\log^{+}\rho\phantom{\frac{1}{\rho-\rho_{1}}}\right.\\
&\qquad\qquad\qquad\left.{\,}+\log^{+}\frac{1}{\rho-\rho_{1}}+\log^{+}\frac{1}{\rho_{1}}
    + O(1)\right]\\
&< [l+1+A(l)]T(\rho,f)+A(l)\left[\log^{+}\rho+\log^{+}\frac{1}{\rho-\rho_{1}}+\log^{+}\frac{1}{\rho_{1}}+O(1)\right]
\end{align*}
So,
\begin{align*}
\log^{+}T[\rho_{1},f^{(l)}] &<
\log^{+}T(\rho,f)+\log^{+}(\log^{+}\rho)+\log^{+}\left(\log^{+}\frac{1}{\rho-\rho_{1}}\right)\\
&\quad +\log^{+}\left(\log^{+}\frac{1}{\rho_{1}}\right)+O(1)\\
&<
\log^{+}T(\rho,f)+\log^{+}\rho+\log^{+}\frac{1}{\rho-\rho_{1}}+\log^{+}\frac{1}{\rho_{1}}+O(1) 
\end{align*}
Also\pageoriginale\ by Lemma \ref{part2-lem3} applied to $f^{(l)}$,
\begin{align*}
m\left(r,\frac{f^{l+1}}{f^{l}}\right) &<
3\log^{+}T\left[\rho_{1},f^{(l)}\right]+4\log^{+}\rho_{1}+\log^{+}\frac{1}{r}\\
&\qquad\qquad+4\log^{+}\frac{1}{\rho_{1}-r}+O(1)\\
&<
3\log^{+}T(\rho,f)+7\log^{+}\rho+7\log^{+}\frac{1}{\rho-r}+2\log^{+}\frac{1}{r}+O(1)\\  
&<
A\left[\log^{+}T(\rho,f)+\log^{+}\rho+\log^{+}\frac{1}{\rho-r}+\log^{+}\frac{1}{r}+O(1)\right] 
\end{align*}
because,
$$
\rho_{1}-r=\rho-\rho_{1}=\frac{1}{2}(\rho-r) (\rho>\rho_{1}>r)
$$
Therefore, since
$$
m\left[r,\frac{f^{l+1}}{f}\right]<m\left[r,\frac{f^{l+1}}{f^{(l)}}\right]+m\left[r,f^{(l)}/f\right]. 
$$
$m\left(r,\dfrac{f^{l+1}}{f^{l}}\right)$ is less than
$$
[A+A(l)]\left[\log^{+}T(\rho,f)+\log^{+}\frac{1}{\rho-r} +
  \log^{+}(1/r)+O(1)+\log^{+}\rho\right]  
$$
completing the inductive proof.
\end{proof}

\begin{lem}\label{part2-lem6}
If $\psi(z)$ is defined as in theorem \ref{part2-thm13}, and
$0<r<\rho<R$, then
$m(r,\psi/f)<O[T(r,f)]+A(l)[\log^{+}T(\rho,f)+\log^{+}\dfrac{1}{\rho-r}+\log^{+}\rho]+O(1)$
as $r\to R$ in any manner.
\end{lem}

\begin{proof}
\begin{align*}
m(r,\psi/f) &\leq
\sum^{l}_{i=0}m\left[r,a_{i}\frac{(z)^{f^{(i)}}}{f}\right]+\log(l+1).\\ 
&\leq
\sum^{l}_{i=0}m[r,a_{i}(z)]+\sum^{l}_{i=0}m[r,f^{(i)}/f]+\log(l+1)+(l+1)\log
2.\\
&= O[T(r)]+\sum^{l}_{i=0}m[r,f^{(i)}/f]+O(1)\\
&<
O[T(r)]+A(l)\left[\log^{+}T(\rho,f)+\log^{+}\frac{1}{\rho-r}+\log^{+}\rho\right]+O(1) 
\end{align*}
from\pageoriginale the lemma \ref{part2-lem5} and because
$m[r,a_{i}(z)]\leq T[r,a_{i}(z)]=O[T(r)]$, and $\log^{+}(1/r)$ remains
bounded as $r$ tends to $R$.
\end{proof}

\begin{lem}\label{part2-lem7}
If $\psi(z)$ is defined as above then as $r\to R$ in any manner while
$0<r<\rho<R$,
\begin{align*}
T(r,\psi)&<[l+1+O(1)]T(r,f)+A(l)\\
&\qquad\left[\log^{+}T(\rho,f)+\log^{+}\left(\frac{1}{\rho-r}\right)+\log^{+}\rho+O(1)\right]. 
\end{align*}
\end{lem}

\begin{proof}
$m(r,\psi)\leq m(r,\psi/f)+m(r,f)$. If $f$ has a pole of order $K$ at
  a point and $a_{\nu}(z)$ a pole of order $K_{\nu}$ then
  $a_{\nu}(z)f^{(\nu)}$ has a pole of order $\nu+K+K_{\nu}$, and so
  $\psi(z)$ has a pole or of order $\Max.\, (\nu+K+K_{\nu})\leq
  (l+1)K+\sum K_{\nu}$. This gives
\begin{align*}
N(r,\psi) &\leq (l+1)N(r,f)+\sum_{\nu}N(r,a_{\nu})\\
&\leq (l+1)N(r,f)+O[T(r,f)]
\end{align*}
Adding the above two inequalities,
$$
T(r,\psi)\leq [l+1+O(1)]T(r,f)+m(r,\psi/f),
$$
and now the lemma follows from the previous lemma.
\end{proof}

\begin{lem}\label{part2-lem8}
If $S(r,\psi)$ is defined as in Theorem \ref{part2-thm8} with $\psi$
instead of $f$ then if $0<r<\rho<R$ and $r$ tends to $R$,
$$
S(r,\psi)<A\left[\log^{+}T(\rho,f)+\log^{+}\left(\frac{1}{\rho-r}\right)+\log^{+}\rho+O(1)\right]. 
$$
\end{lem}

\begin{proof}
Let $\rho_{1}=\frac{1}{2}(\rho+r)$. Lemma \ref{part2-lem3} gives then
$$
S(r,\psi)<A\left[\log^{+}T(\rho_{1},\psi)+\log^{+}\left(\frac{1}{\rho_{1}-r}\right)+\log^{+}\rho_{1}+O(1)\right]. 
$$
By lemma \ref{part2-lem7} since $\log^{+}x\leq x$, 
$$
T(\rho_{1},\psi)<A(l)\left[T(\rho,f)+\left(\frac{1}{\rho-\rho_{1}}\right)+\rho+O(1)\right]
$$\pageoriginale
that is,
$\log^{+}T(\rho_{1},\psi)<\log^{+}T(\rho,f)+\log^{+}\dfrac{1}{\rho-\rho_{1}}+\log^{+}\rho+O(1)$. 

Substituting this and remembering that $r<\rho_{1}<\rho$,
$\rho-\rho_{1}=\rho_{1}-r=\frac{1}{2}(\rho-r)$ we get the result.
\end{proof}

Now we are ready to prove theorem \ref{part2-thm13}.

\subsubsection{}\label{part2-subsubsec2.5.2}

\setcounter{proofofthm}{12}
\begin{proofofthm}\label{part2-proofofthm13}
We have from theorem \ref{part2-thm8},
$$
m(r,\psi)+m\left(r,\frac{1}{\psi}\right)+m\left(r,\frac{1}{\psi-1}\right)\leq
2T(r,\psi)-N_{1}(r,\psi)+S(r,\psi)
$$
\ie
\begin{align*}
T\left(r,\frac{1}{\psi}\right)+T\left(r,\frac{1}{\psi-1}\right) &\leq
T(r,\psi)-N_{1}(r,\psi)+S(r,\psi)\\
&\quad
+N(r,\psi)+N\left(r,\frac{1}{\psi}\right)+N\left(r,\frac{1}{\psi-1}\right) 
\end{align*}
\ie
$$
T(r,\psi)\leq
N(r,\psi)-N_{1}(r,\psi)+N\left(r,\frac{1}{\psi}\right)+N\left(r,\frac{1}{\psi-1}\right)+S(r,\psi)+O(1) 
$$
also
\begin{align*}
&
  N(r,\psi)-N_{1}(r,\psi)=\ob{N}(r,\psi)-N\left(r,\frac{1}{\psi'}\right)\quad\text{and}\\
&
  N\left(r,\frac{1}{\psi-1}\right)-N\left(r,\frac{1}{\psi'}\right)=\ob{N}\left(r,\frac{1}{\psi-1}\right)-N_{0}\left(r,\frac{1}{\psi'}\right). \text{
    \;\ Hence}
\end{align*}
Thus,
$$
T(r,\psi)\ob{N}(r, \ )+N(r,1/
\ )+\ob{N}\left(r,\frac{1}{-1}\right)-N_{0}(r,1/1)+S(r, \ )+O(1)
$$
where $N_{0}(r,1/\psi')$ denotes the fact that zeros of $\psi'$ at
multiple roots of $\psi-1$ are omitted. Thus, since
$T(r,\psi)=m(r,1/\psi)+N(r,1/\psi)+O(1)m(r,1/\psi)\leq
\ob{N}(r,\psi)+\ob{N}\left(r,\dfrac{1}{\psi-1}\right)-N_{0}(r,1/\psi')+S(r,\psi)+O(1)$. Note
again\pageoriginale that poles of $\psi$ occur only at poles of $f$ or
of $a_{\nu}(z)$, and in $\ob{N}$ each pole is counted only once. Then
\begin{align*}
\ob{N}(r,\psi) &\leq \ob{N}(r,f)+\sum_{\nu}N(r,a_{\nu})\\
&\leq \ob{N}(r,f)+OT(r,f).
\end{align*}
Again,
\begin{align*}
T(r,f) &= m(r,1/f)+N(r,1/f)+O(1)\\
&\leq m(r,\psi/f)+m(r,1/\psi)+N(r,1/f)+O(1).
\end{align*}
Substituting we obtain,
\begin{gather*}
[O(1)+1] T(r,f)\leq \ob{N}(r,f) +
\ob{N}\left(r,\frac{1}{\psi-1}\right)+N(r,1/f)-N_{0}(r,1/\psi')\\
+S(r,\psi)+ m(r,\psi/f),
\end{gather*}
since $f(z)$ being admissible $O(1)=O[T(r)]$. Now if $0<r<\rho<R$
lemmas \ref{part2-lem6} and \ref{part2-lem8} give
$$
m(r,\psi/f)+S(r,\psi)<A(l)\left[\log^{+}T(\rho,f)+\log^{+}\frac{1}{\rho-r}+\log^{+}\rho+O(1)\right]. 
$$
and now the result is completed just as in theorem \ref{part2-thm9} by
means of lemma \ref{part2-lem4}. Hence theorem \ref{part2-thm13}.
\end{proofofthm}

\subsubsection{Consequences.}\label{part2-subsubsec2.5.3}

\begin{thm}[Milloux]\label{part2-thm14}
If $f(z)$ is admissible in $|z|<R$ and is regular there then either
$f(z)$ assumes every finite value infinitely often or every derivative
of $f(z)$ assumes every finite value except possibly zero, infinitely often.
\end{thm}

\begin{proof}
Suppose $f(z)=a$, $f^{(l)}(z)=b$ have only a finite number of roots
where $b\neq 0$. Choose $g(z)=f(z)-a$,
$\psi(z)=\dfrac{g^{(l)}(z)}{b}=\dfrac{f^{(l)}(z)}{b}$ in\pageoriginale
theorem \ref{part2-thm13}. Then since $\ob{N}(r,g)=0$, $g$ being
regular
$$
[1+0(1)]T(r,g)<\ob{N}\left[r,\frac{1}{g^{(l)}(z)-b}\right]+N(r,1/g)+S_{1}(r)
$$
where $\mathop{\ub{\lim}}\limits_{r\to R}S_{1}(r)/T(r,g)=0$.

If $R=\infty$ this gives $\mathop{\ub{\lim}}\limits_{r\to
  \infty}T(r,g)/\log r<+\infty$ by assumptions that
$n\left[t,\dfrac{1}{f^{(l)}(z)-b}\right]$ and $n(t,1/g)$ are finite as
$r$ tends to infinity so that $g(z)$ is rational and so is
$f(z)$. That is $f(z)$ is a polynomial.
\end{proof}

If $R$ is less than infinity we obtain,
$\mathop{\ub{\lim}}\limits_{r\to R}T(r,g)<\infty$, and so since
$T(r,g)$ increases with $r\lim T(r,g)<\infty$ and the same result
applies to $f(z)$ giving a contradiction to admissibility.

\begin{thm}[Saxer]\label{part2-thm15}
If $f(z)$ is meromorphic in the plane and $f$, $f'$, $f''$ have only a
finite number of zeros and poles then
$f(z)=P_{1}(z)/P_{2}(z)\break e^{P_{3}(z)}$  $P_{1}$, $P_{2}$, $P_{3}$ being
polynomials. If $f$, $f'$, $f''$ have no zeros and poles then
$f(z)=e^{a+bz}$ where $a$, $b$ are constants.
\end{thm}

\begin{proof}
Set $g(z)=f(z)/f'(z)$. Then
$g'(z)=1-[f(z)f''(z)/{f'}^{2}(z)]$. Suppose that $g(z)$ is
transcendental and so admissible in the plane. Now $g=0$, $\infty$
only when $f'=0$, on $f=0$, $\infty$ that is a finite number of times
and so, $N(r,g)+N(r,1/g)=O(\log r)$. Next $g'(z)=1$ only for $f=0$ or
$f''=0$ that is $N\left(r,\dfrac{1}{g'-1}\right)=O(\log r)$ by
hypothesis. Now theorem \ref{part2-thm13} applied to $g(z)$ gives for
a sequence of $r$ tending to infinity, taking $\psi=g'$,
$$
[1+O(1)]T(r,g)=O(\log r).
$$
This\pageoriginale implies $g(z)$ is rational, \ie a
contradiction. Hence $g(z)$ is rational so that $f'/f=g$ is
rational. Now $f'/f$ has simple poles with integer residues at the
poles and zeros of $f(z)$. Since $f'/f$ is rational by expanding it in
partial fractions we get,
$$
f'/f=\sum_{r}k_{r}/(z-z_{r})+P(z)
$$
with $k_{\nu}$ integers and $P(z)$ a polynomial. Integrating the
above,
$$
f(z)=\prod_{r}(z-z_{r})^{k}\int\limits_{e}P(z)dz
$$
This proves the first part and if $f(z)$ has no zeros or poles the
product term disappears and $f=e^{P(z)}$, $f'(z)=P'(z)e^{P(z)}$ and
$f'(z)\neq 0$ implies $P'(z)\neq 0$, which gives $P(z)=a+bz$.
\end{proof}

\begin{remark*}
Note that we cannot do the same thing with $f(z)$ and $f'(z)$. For if
$f(z)=e^{g(z)}$, $f'(z)=g'(z)e^{g(z)}$ and if we put $g'(z)=e^{h(z)}$
where $h(z)$ is an arbitrary integral function, so that $g(z)=\int
e^{h(z)}dz$ and $f(z)=\exp\cdot \left[\int e^{h(z)}dz\right]$. Then
$f$ is an integral function with $f\neq 0$, and $f'\neq 0$. So in
theorem \ref{part2-thm15} we cannot leave out the restriction on
$f''$. Further $F(z)=\int f(z)dz$ is a function for which $F'\neq 0$,
$F''\neq 0$ and so we cannot leave the restriction on $f$. 
\end{remark*}

But we will show that we can leave out the restriction on $f'$. This
is precisely Theorem \ref{part2-thm19}.

In\pageoriginale connection with theorem \ref{part2-thm15} we also
quote the following extension by Csillag \cite{1}.

\begin{thm}\label{part2-thm16}
If $l$ and $m$ are different positive integers and $f(z)$ an integral
function such that $f(z)\neq 0$, $f^{(l)}(z)\neq 0$ and
$f^{(m)}(z)\neq 0$, then $f(z)$ is equal to $e^{az+b}$.
\end{thm}

\subsection{Elimination of \texorpdfstring{$N(r,f)$}{Nrf}}\label{part2-sec2.6}

We shall prove the following theorem

\begin{thm}\label{part2-thm17}
If $f(z)$ is admissible in $|z|<R$, and $l\geq 1$ then,
$$
T(r,f)<[2+(1/l)]N(r,1/f)+2[1+(1/l)]\ob{N}\left(r,\frac{1}{f^{(l)}-1}\right)+S_{2}(r,f), 
$$
where $\mathop{\ub{\lim}}\limits_{r\to R}S_{2}(r,f)/T(r)=0$.
\end{thm}

Let us set $\psi(z)=f^{(l)}(z)$ in theorem \ref{part2-thm13} (Th.\@ of
Milloux), to get
\begin{equation*}
T(r,f)<\ob{N}(r,f)+N(r,1/f)+\ob{N}\left(r,\frac{1}{f^{(l)}-1}\right)-N_{0}\left(r,\frac{1}{f^{(l+1)}}\right)+S_{1}(r,f)\tag{2.7}\label{part2-eq2.7}
\end{equation*}
where in $N_{0}\left(r,\dfrac{1}{f^{(l+1)}}\right)$ zeros of
$f^{(l+1)}$ at multiple roots of $f^{(l)}(z)=1$ are to be omitted.

Further we need,

\begin{lem}\label{part2-lem9}
If $g(z)=\left[f^{(l+1)}(z)\right]^{l+1}/[1-f^{(l)}(z)]^{l+2}$ then 
\begin{align*}
lN_{1}(r,f)\leq
\ob{N}_{2}(r,f)+\ob{N}\left(r,\frac{1}{f^{(l)}-1}\right) &+
N_{0}\left(r,\frac{1}{f^{(l+1)}}\right)+m(r,g'/g)\\
&+ \log|g(0)/g'(0)|\tag{2.8}\label{part2-eq2.8}
\end{align*}
where\pageoriginale\ $N_{1}(r,f)$ stands for the $N$-sum over the
simple poles of $f(z)$;\break $\ob{N}_{2}(r,f)$ for the $N$ sum over
multiple poles of $f(z)$, with each pole counted only once. Thus
$\ob{N}(r,f)=N_{1}(r,f)+N_{2}(r,f)$. 
\end{lem}

Now at a simple pole $z_{0}$ of $f(z)$, $f(z)=O(1)+[a/(z-z_{0})]$
where $a$ is not equal to zero.

Differentiating $l$ times,
$$
1-f^{(l)}(z)=\frac{al!(-1)^{l+1}}{(z-z_{0})^{l+1}}+O(1)
$$
This can be written as
$$
1-f^{(l)}(z)=\frac{al!(-1)^{l+1}}{(z-z_{0})^{l+1}}[1+O\{(z-z_{0})^{l+1}\}]
$$
The differentiation of both the sides again gives,
$$
f^{(l+1)}(z)=\left[1+O\{(z-z_{0})^{l+2}\}\right]\frac{a(l+1)!(-1)^{l+1}}{(z-z_{0})l+2}
$$
Hence,
$$
g(z)=\frac{(-1)^{l+1}(l+1)^{l+1}}{a(l)!}[1+O\{(z-z_{0})^{l+1}\}]
$$
So, $g(z_{0})\neq 0$, $\infty$ but $g'(z)$ has a zero of order at
least $\ub{l}$ at $z=z_{0}$. Now we have
$$
N(r,g/g')-N(r,g'/g)=m(r,g'/g)-m(r,g/g')+\log|g(0)/g'(0)|
$$\pageoriginale
from Jensen's formula. As we saw at the end of section $2$ the left
hand side is
\begin{align*}
N(r,g)&+N(r,1/g')-N(r,g')-N(r,1/g)\\ 
&= N(r,1/g')-N(r,1/g)-\ob{N}(r,g)\\
&= N_{0}(r,1/g')-\ob{N}(r,1/g)-\ob{N}(r,g)
\end{align*}
where in $N_{0}(r,1/g')$ only zeros of $g'$ which are not zeros of $g$
are to be considered. By our above analysis, 
\begin{align*}
lN_{1}(r,f) &\leq N_{0}(r,1/g')\\
&\quad \ob{N}(r,1/g)+\ob{N}(r,g)+m(r,g'/g)+\log|g(0)/g'(0)|
\end{align*}
Then note that $g=0$, $\infty$ only at poles of $f(z)$ which must be
multiple, at zeros of $f^{(l)}(z)-1$, and zeros of $f^{(l+1)}(z)$
which are not zeros of $f^{(l)}(z)-1$. This gives lemma
\ref{part2-lem9}.

Now \eqref{part2-eq2.7} gives on writing $T(r,f)=m(r,f)+N(r,f)$
\begin{equation*}
N(r,f)-\ob{N}(r,f)<N(r,1/f)+\ob{N}\left(r,\frac{1}{f^{(l+1)}}\right)-N_{0}\left(r,\frac{1}{f^{(l+1)}}\right)+S_{1}(r,f)\tag{2.9}\label{part2-eq2.9}
\end{equation*}
On the left the contribution of each multiple pole to the sum being
counted once for $\ob{N}$, but at least twice for $N$. So,
\begin{equation*}
N_{2}(r,f)\leq N(r,f)-\ob{N}(r,f)\tag{2.10}\label{part2-eq2.10}
\end{equation*}
Also $\ob{N}(r,f)=\ob{N}_{2}(r,f)+N_{1}(r,f)$. Hence it follows from
lemma \ref{part2-lem9} that,
\begin{align*}
N(r,f)&\leq [1+(1/l)]N_{2}(r,f)\\
&+\frac{1}{l}\left[\ob{N}\left(r,\frac{1}{f^{(l)}-1}\right)
   + N_{0}\left(r,\frac{1}{f^{(l+1)}}\right)+m(r,g'/g)\right]
+ \frac{1}{l}\log\frac{|g(0)|}{|g'(0)|}
\end{align*}
By\pageoriginale\ the inequalities \eqref{part2-eq2.9}
and \eqref{part2-eq2.10} it is at the most,
\begin{align*}
&\left(1+\frac{1}{l}\right)\left[N(r,1/f)+\ob{N}(r,1/(f^{(l)}-1))-N_{0}\left(r,\frac{1}{f^{(l+1)}}\right)+S_{1}(r,f)\right]\\
&
  +(1/l)\left[\ob{N}\left(r,\frac{1}{f^{(l)}-1}\right)+N_{0}\left(r,\frac{1}{f^{(l+1)}}\right)+m(r,g'/g)+\log|g(0)/g'(0)|\right]\\
&=(1+1/l)N(r,1/f)+(1+2/l)\ob{N}(r,1/(f^{(l)}-1))\\
&{}-N_{0}\left(r,\frac{1}{f^{(l+1)}}\right)+S_{3}(r) 
\end{align*}
where
$$
S_{3}(r)=[1+(1/l)]S_{1}(r,f)+(1/l)[m(r,g'/g)+\log|g(0)/g'(0)|].
$$
Substituting for $\ob{N}(r,f)$ in \eqref{part2-eq2.7} we obtain,
\begin{align*}
T(r,f)&<[2+(1/l)]N(r,1/f)+2[1+(1/l)]\\
&\ob{N}\left(r,\frac{1}{f^{(l)}-1}\right) -2N_{0}
\left(r,\frac{1}{f^{(l+1)}} \right)
+S_{1}(r,f)+S_{3}(r,f)
\end{align*}
We observe that the above gives the result of theorem
\ref{part2-thm17}, provided we pose
\begin{align*}
S_{2}(r,f) &= S_{1}(r,f)+S_{3}(r,f)\\
&= [2+(1/l)]S_{1}(r,f)+(1/l)[m(r,g'/g)+\log|g(0)/g'(0)|]
\end{align*}
and in order to complete the proof it is sufficient to prove that if
$r<\rho<R$, and $r$ tends to $R$,
$$
m(r,g'/g)<A[\log^{+}T(\rho,f)+\log^{+}1/(\rho-r)+\log^{+}\rho+0(1)]
$$
The above inequality is true for, 
\begin{gather*}
\log g(z)=(l+1)\log f^{(l+1)}(z)-(l+2)\log [1-f^{(l)}(z)]\\
g'/g=(l+1)\left[f^{(l+2)}/f^{(l+1)}\right]+(l+2)f^{(l+1)}/(1-f^{(l)})\\
m(r,g'/g)\leq m(r,f^{(l+2)}/f^{(l+1)})+m\,
\left(r,f^{(l+1)}/(f^{(l)}-1)\right)+O(1) 
\end{gather*}\pageoriginale
Now the result follows from Lemma \ref{part2-lem8}, with
$\psi=f^{(l)}-1$ or $f^{(l+1)}$. Hence the proof of
theorem \ref{part2-thm17} is complete.

\subsection{Consequences}\label{part2-sec2.7}

\begin{thm}\label{part2-thm18}
The result of theorem \ref{part2-thm14} extends to meromorphic
functions without any further hypothesis. To be precise, if $f(z)$ is
admissible in $|z|<R$ and is meromorphic there, then either $f(z)$
assumes every finite value infinitely often or every derivative of
$f(z)$ assumes every finite value except possibly zero infinitely often.
\end{thm}

The proof is as before by the consideration of $g(z)=[f(z)-a]/b$
instead of $f(z)$; if the equations $f(z)=a$ and $f^{(l)}=b$ have only
a finite number of roots. For then $g=0$ and $g^{(l)}=1$ have only
finite number of roots and we can use Theorem \ref{part2-thm17}.

\begin{thm}\label{part2-thm19}
Theorem \ref{part2-thm15} (Saxer's theorem) still holds if the
hypothesis of $f'(z)$ is omitted. That is, if $f(z)$ is meromorphic in
the plane and $f$, and $f''$ have only a finite number of zeros and
poles then,
$$
f(z)=P_{1}(z)/P_{2}(z)e^{P_{3}(z)}
$$
$P_{1}$, $P_{2}$, $P_{3}$ being polynomials. If $f''$, $f$ have no
zeros and poles then $f(z)=e^{az+b}$, $a$ and $b$ being constants.
\end{thm}

\begin{proof}
If\pageoriginale $f(z)$ and $f''(z)$ have only a finite number of
zeros and poles, put $g(z)=f(z)/f'(z)$. Then
$g'(z)=1-\{f(z)f''/[f'(z)]^{2}\}g(z)$ has only finite number of zeros
and so does $g'(z)-1=ff''/{f'}^{2}$ namely at the poles of $f$ and at
the zeros of $f$ and $f''$. Hence by Theorem \ref{part2-thm18}, $g(z)$
is rational and now the proof is completed as in Saxer's theorem. If
$f(z)$ and $f''(z)$ have no zeros or poles, $f(z)=e^{P(z)}$ where
$P(z)$ is a polynomial. Therefore, $f''(z)$ equals
$[P''(z)+P'(z)^{2}]e^{P(z)}$, and if $P'(z)$ has degree $n$ greater
than or equal to $1$, then $P'(z)^{2}$, $P''(z)$ have degree $2n$,
$n-1$, respectively, so that $f''(z)$ has $2n$ zeros. Thus
$P'(z)=a=\text{ const. } P(z)=az+b$.  
\end{proof}

A slightly more delicate analysis shows that if $f(z)$, $f''(z)$ have
no zeros but $f(z)$ may have a finite number of poles, then
$f(z)=e^{az+b}$ or $(az+b)^{-n}$, where $n$ is a positive integer.
