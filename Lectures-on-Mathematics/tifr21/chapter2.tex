
\chapter{Generators and Relations}\label{chap2} % chapter 2

\section{}\label{chap2:sec1} % sec 1.

In\pageoriginale this chapter we shall show how to construct the smallest subgroup
containing a given set of elements of group. The concept of relation
will also introduced.  

As an immediate consequence of the theorem of the last
chapter, we have 

\setcounter{theorem}{0}
\begin{theorem}\label{chap2:sec1:thm1} % then 1
  The intersection of an arbitrary family of subgroups of a groups is
  a subgroup. 
\end{theorem} 
 
 Let $G$ be a group and $E$ a subset of $G$. The subgroup
 $$
 gp (E) = \bigcap_{E \subseteq S \subseteq G}S
 $$
 is the subgroup \textit{generated by $E$}. We call $E$ a set of
 \textit {generators} of $gp (E)$. Since a subgroup by definition is
 non-empty, it follows that 
 $$
 gp (\phi) = \{1 \}.
 $$ 
 
 We call $\{1 \}$ the \textit{trivial subgroup} of $G$.
 
 If $X$ is any set, we denote by $|X|$ its cardinal.
 
 If $E \subseteq G$ and $|E| < \mathfrak{X}_\circ$, then $gp(E)$ is
 \textit{finitely generated}. Similarly, $gp(E)$ is countably
 generated if $|E| \leq \mathfrak{X}_\circ$. 
 
 If $|E| = 1$, then $gp(E)$ is a \textit{cyclic group}.
 
 We shall now construct $gp(E)$, given $E \subseteq G$. We construct a\pageoriginale
 non-decreasing sequence of sets inductively. Put $E_1 = E \cup
 \{1\}$. 

 Having defined $E_1, \ldots, E_n$ define $E_{n+1} = E_n E^{-1}_{n}$
 write $S = \bigcup\limits_{n=1}^{\infty} E_n$. It is immediately seen
 that 
 $$
 E_1 \subseteq S. 
 $$
 
 Also, if $T$ is an arbitrary subgroup of $G$ containing $E$, then 
 $$
 E_1 \subseteq T. 
 $$
 
 If $E_{n}  \subseteq T$, then also $E_{n+1} \subseteq T$.
 
 It follows that
\begin{equation}
  S \subseteq T  \tag{1}\label{chap2:sec1:eq1}
\end{equation} 
$$
\displaylines{\text{and thus}\hfill 
  S \subseteq gp(E) = \bigcap_{E \subseteq T \leq G}T.\hfill}
$$
 
We now prove that $S$ is a subgroup. $S$ is non-empty and all the $E_n
's$ contain $1$. If $f \in  E_n$, then 
$$
f. 1^{-1} = f \in  E_{n+1}.
$$
 
Therefore $E_n \subseteq E_{n+1}$. Thus $\{E_n \}$ is a non-decreasing
sequence of sets. Let $x$, $y$, $\in  S$; so that $x
\in  E_m$, $y \in  E_n$ for some $m,n$. Put $p = \max
(m,n)$. Then, $x \in  E_p, y \in  E_p$ and hence
$xy^{-1} \in  E_{p+1} \subseteq S$. This proves that $S$ is
closed under right division. Therefore $S$ is a subgroup containing
$E$. Thus, $gp(E) \subseteq S$, and combining this with\pageoriginale
(\ref{chap2:sec1:eq1}), we get  
$$
S= gp (E).
$$

\section{}\label{chap2:sec2}% sec 2

The above construction shows that any element of $E$ has a
`representation' in terms of elements of $E$ as  
$$
w(e_1,\ldots, e_n) = e^{m_1}_1 e^{m_2}_2 \cdots e^{m_n}_n, m_i = \pm
1, e_i \in E, i = 1, \ldots,  n.  
$$

Such expressions are called \textit{words}. It is not assumed that
differently indexed $e_i$ are different. Different words may represent
the same element. For example, $abb^{-1} c^{-1}$ and $ac^{-1}$ are
different words representing the same element $ac^{-1}$. The word
containing no $e$ at all is the \textit{empty word}. This represents
the unit element, and we therefore denote it (somewhat ambiguously) by
$1$. It is easy to see that any element of $G$ that can be represented
by a word in the elements of $E$ is in $gp (E)$. Thus we have,
\begin{theorem}\label{chap2:sec2:thm2}%them 2
  The subgroup $gp(E)$ consists of all the elements of $G$ represented
  by the `words' formed by the element of $E$. 
\end{theorem}

\section{}\label{chap2:sec3}% sec 3
Cyclic groups are the simplest types of groups which one comes
across. The theorem below gives the structure of subgroups of a cyclic
group. 

\begin{theorem}\label{chap2:sec3:thm3}%them 3.
  If $G$ is cyclic, all subgroups of $G$ are cyclic.
\end{theorem}

\begin{proof}
  Let $\{ a \}$ be a generator of $G$. Then,
  $$
  gp(\{ a \}) = gp(a) = C.
  $$
\end{proof}

Let\pageoriginale $S$ be a subgroup of $G$. If $= \{ 1 \}$, it is cyclic as
claimed. If $\{ 1 \} < S \leq G$, then there is an element $a^k
\in  S$, $a^k \neq 1$; also, $a^{-k} \in  S$. Let $N$ be
the set of positive integers defined by  
$$
N= \left\{ n \Big| a^n \in  S \right\}
$$

Since $k \in  N$ or $-k \in  N$, $N \neq \phi$. Denote
by $m$ the least positive integer in $N$. We claim that $a_m$ is a
generator of $S$. 

Trivially, 
$$
gp(a_m) \leq S.
$$
If $c = a^\ell \in  S$, then $|\ell |\geq m$. Write
$$
\ell = mq + r, 0 \leq r<m
$$
Then $a^r = a^\ell a^{-mq} \in  S$, and therefore $r=0$. Hence
$c=a^{mq} \in  gp(a^m)$. Thus, 
$$
S \leq gp(a^m)
$$

Combining this with the above inequality, we have 
$$
S= gp(a^m)
$$
and the theorem is proved.

\section{}\label{chap2:sec4}% sec 4.

In this context, we ask the following question
\begin{prob*}
  What\pageoriginale groups can be subgroups of two-generator groups? 
\end{prob*}
 
A partial answer to this problem will be given now. It will be
completely soled in the subsequent chapters.

\begin{theorem}\label{chap2:sec4:thm4}%them 4
  Countably generated groups are countable.
\end{theorem}

Let $G = gp(E)$, where $E$ is countable. Let $E = \big\{ e_1,
e_2,\ldots \big\}$. We have seen that $gp(E)$ consists of all the
elements represented by `words' in $e_1, e_2, \ldots$. 

If $g$ is an element of $G$ which can be represented by a word $w$ in
$e_1, e_2, \ldots$, then $g$ can be written in the form 
$$
g = w = e^{m_1}_{i_1} e^{m_2}_{i_2} \cdot e^{m_\ell}_{i_\ell},
$$
where the $i_j$ are positive integers and $m_i$ are integers,
positive, zero, for negative. Note that different $w' s$ can represent
the same element. Corresponding to each $m_i$, we define 
\begin{align*}
  m^+_i &= \max (m_i, 0),\\
  m^-_i &= \max (-m_i, 0).
\end{align*}

If $m_i \geq 0$, then $m^+_i = m_i$, $m^{-}_i = 0$. If $m_i < 0$, then
$m^+ _i = 0$, $m^{-}_i = -m_i$. Thus at most one of $m^+_i$, $m^-_i$
is non-zero. 

We now construct $a ~ 1-1$ mapping $\gamma$ of the set of all $w' s$
into the set of positive integers. This will prove that $g p(E)$ is
countable. 

We\pageoriginale write $\gamma (1) = 1$. If $e^{m_1}_{i_1} ~ e^{m_2}_{i_2} \cdot
e^{m_\ell}_{i_\ell}$, then define $\gamma (w) = 2^{i_1} 3^{m^+_1}
5^{m^-_1} 7^{i_2}$ $11^{m^{-}_2} 13^{m^-}_2 \ldots p^{m^-_\ell}_{3\ell}$
where $p_n$ denotes the $n^{th}$ prime when the set of all positive
primes is arranged in increasing order. 

The numbers $\gamma (w)$ are called \textit{G\"{o}del numbers}. 

Since every positive integer can be written as a product of prime
powers uniquely, it follows that every positive integer is the
G\"{o}del number of \textit{at most one} word; hence $\gamma$ is
$1-1$. Therefore the set of words in $E$, and also $gp(E)$, is
countable. 

\begin{remark*}
  This theorem can also be proved by making use of the construction we
  have given for $gp(E)$. That is to say, 
  $$
  gp(E) = \bigcup_{n=1}^\infty E_n,
  $$
  where $E_1 = E \cup \{ 1 \}$, $E_{n+1} = E_n E_{n}^{-1}$. If $E$ is
  countable, so is $E_1$. If $E_n$ is countable, so is $E_{n+1}$
  because $| E_{n+1}| \leq | E_n^{-1} | ~ | E_n^{-1}| = | E_n
  |^2$. Therefore all the $E_n's$ are countable, and so is their union
  $gp(E)$. 
\end{remark*}

\begin{coro*}
  Necessary for a group to be embeddable in a two-generator group is
  that it be countable 
\end{coro*}

\section{}\label{chap2:sec5}% sec 5.

Let $G = gp(E)$ be a group with $E$ as the set of generators. Then
every element of $G$ can be represented by a `word' formed of some
finite number of elements of $E$. We denote by $w(e_1, \ldots,  e_n)$
a\pageoriginale word consisting of the `letters' $e_1, \ldots,  e_n$ only (not
necessarily all). Let $w(e_1. \ldots, e_n)$, $v(e_1^1, \ldots, 
e_m^1)$ be two words in $E$. We say that 
$$
w(e_1, \ldots, e_n) = v(e_1^1, \ldots,  e_m^1)
$$
is a \textit{relation} in $G$, if this equation holds when $w(e_1,
\ldots, e_n)$ and $v(e_1^{'}, \ldots$, $e_n')$ are considered as
elements of $G$. Without loss of generality we can write the above
relation in the form 
$$
w(e_1, \ldots, e_n) = v(e_1, \ldots, e_n).
$$
In the subsequent pages $\underbar{e}$ will stand for $(e_1, \ldots
,e_n)$. We say that  
$$
w(\underbar{e}) = u(\underbar{e})
$$
is a \textit{trivial relation} if it follows from the group axioms and
does not depend upon the particular group under consideration. For
example, 
$$
e_1 e_2 e_2^{-1} e_3 e_4 e_4^{-1} = e_1 e_3 e_5 e_5^{-1}
$$
is a trivial relation. A relation of the the type 
$$
e_1 e_2 = e_2 e_1,
$$
if valid, is a non-trivial relation.  

Let 
$$
\displaylines{\hfill~~ 
  w(e_1, \ldots, e_n) = e^{m_1}_1 e^{m_2}_2 \cdot e^{m_n}_n, m_i = \pm
  1, e_i \in E, i=1, \ldots, n\hfill\cr 
  \text{and} \hfill  
  v(f_1, \ldots, f_r) = f^{\ell_1}_1 f^{\ell_2}_2 \cdots f^{\ell_r}_r,
  f_i \in  E, \ell_i = \pm 1, i=1,\ldots,  2 \hfill}
$$
be\pageoriginale two words. By the \textit{product} of the words $w$ and $v$ (taken
in this order), we mean the word 
$$
v = w(e_1,\ldots, e_r) v (f_1,\ldots, f_r) = e_1^{m_1} e_2^{m_2}
\cdots e_n^{m_n} f_1^{l_1},\ldots, f_1^{l_r} 
$$
similarly the \textit{inverse} of $w$ is defined to be the word
$$
w^{-1} = e_n^{-m_n} \cdots e_2^{-m_2} e_1^{-m_1}.
$$
We now state certain elementary properties of relations which are
immediate from the definitions given above. 
\begin{enumerate}[(1)]
\item $v(\underbar{e}) = v(\underbar{e})$ is a trivial relation.
\item If $u(\underbar{e}) = v(\underbar{e})$ is a relation, then so is
  $v(\underbar{e}) = u(\underbar{e})$. 
\item If $u(\underbar{e}) = v(\underbar{e})$ and $v(\underbar{e}) =
  w(\underbar{e})$ are relations, then so is $u(\underbar{e}) =
  w(\underbar{e})$. 
\item If $u(\underbar{e}) = v(\underbar{e})$ is a relation, then so is
  $u^{-1}(\underbar{e}) = v^{-1}(\underbar{e})$ 
\item If $u(\underbar{e}) = v(\underbar{e})$ and $u'(\underbar{e}) =
  v'(\underbar{e})$ are relations then so is
  $u(\underbar{e})u'(\underbar{e}) = v(\underbar{e})v'(\underbar{e})$ 
\item For any word $u(\underbar{e})$,
  $$
  u(\underbar{e}) u^{-1}(\underbar{e}) = 1
  $$
  is a trivial relation.
\end{enumerate}

\section{}\label{chap2:sec6}% sec 6.

In\pageoriginale what follows, we shall abbreviate $v(\underbar{e})$ as $v$ for
convenience, when confusion is not possible. 

we say that a relation
$$
u^* = v^*
$$
follows from (or is a consequence of) relations $u_1= v_1,\ldots, u_r
= v_r$, if it can be derived from these by a finite chain of
applications of (1) - (6). We say that two relations $u = v$ and $u'
= v'$ are equivalent if each follows from the other in the above
sense. 

\begin{example*}
  Every relation $u = v$ is equivalent to a relation of the form
  $$
  w = 1.
  $$
  We can in fact prove this with 
  $$
  w = u v^{-1}.
  $$
  Suppose that $u = v$ is true. Then by $(4)$, we have
  $$
  u^{-1} = v^{-1}
  $$
  An application of $(2)$ gives
  $$
  v^{-1} = u^{-1}
  $$
  Also by $(1)$,
  $$
  u = u
  $$
  is\pageoriginale a relation.
\end{example*}

Multiplying these two relations using $(5)$, we get 
$$
\displaylines{\hfill u v^{-1} = u u^{-1} \hfill \cr
  \text{By (6),} \hfill u u^{-1} = 1 \hspace{1.8cm}\hfill }
$$
is a relation. By the transitivity of relations $((3))$, we have 
$$
u v^{-1}= 1.
$$

Similarly we can prove that $u v^{-1} = 1$ implies that $u = v$. 

Let $G = gp(E)$ be a group. Consider the set of relations valid in
$G$. Let $R$ be a set relations in the elements of $E$ such that all
relations in the elements of $E$ follow from $R$. 

We then say that $R$ is a set of \textit{defining relations} of the
group with respect to the system to generators $E$. The group $G$ is
completely determined by $E$ and $R$. We write 
$$
G = gp(E ; R)
$$
We call $(E, R)$ a \textit{presentation} of $G$.

$G$ is \textit{finitely presented} if there is some presentation $(E;
R)$ of $G$ with $| E | < \mathcal{N}_0$ and $| R | <
\mathcal{N}_0$. Similarly a countably \textit{presented} group is
defined. 

It is easy to see that all finite groups are finitely presented. An
infinite cyclic group is also finitely presented. 

There\pageoriginale exist groups which are finitely generated but not finitely
presented. Examples will be given later. 

The following problem about finitely presented groups is unsolved.

\begin{prob*}
  What groups can be embedded in finitely presented groups? 
  \footnote{\textit{$^*$ note added November 1959.} A very
    significant advance towards a solution of this problem has
    recently been made by GRAHAM HIGMAN (unpublished). He has
    determined all finitely generated subgroups, and a large class of
    not finitely generated subgroups, of finitely presented groups.}  
\end{prob*}

\noindent Not all countable groups can be embedded in finitely presented
groups. But I know of no example of a countable group of which it can
be proved that it cannot be embedded in a finitely presented group. 

\section{The Word Problem for groups}\label{chap2:sec7}%sec 7

In this section we give a brief account of what is known as the ``Word
Problem''. This problem arose with the development of mathematical
logic. A precise statement of the problem entirely depends upon a
precise definition of the concept of a ``procedure'' (also,
``algorithm'', ``rule'', ``effective procedure'', ``recurrive procedure'',
``computational proce-\break dure'' of ``process''), which was given by Church,
Turing, Kleene and Post (see Kleene (1952)). 

\section*{The Word Problem for groups}

Given a group presentation $(E ; R)$ of a group $G$, to give a
``procedure'' to decide, for any two words $u^*, v^*$ in the elements
of $E$, whether
$$
u^*(e) = v^*(e)
$$
is\pageoriginale a consequence of the relations $R$. Here, roughtly speaking a
``procedure'' is a set of rules or instructions that could be so
formulated as to be programmed for an automatic computer with the data
$E$, $R$ and $u^*$, $v^*$ (suitably coded) and the computer so
programmed as to answer by a somehow cased ``follows" or ``does not
follow". 

A similar problem can be formulated for other algebraic
systems. Markov and Post have proved the insolubility of the word
problem in associative systems. Turing (1950) proved the insolubility
of the word problem for semi-groups with cancellations. (A semi-group
with cancellation is an algebraic system with an associative binary
operator, with the property) 
\begin{align*}
  ax &= bx \text{ implies } a = b, \quad \text{ and }\\
  ya &= yb \text{ implies } a = b, \quad \text{ for all } x, y, a,
  b,\in  S ). 
\end{align*}

The solubility of the word problem for groups has been proved in
special cases. Magnus (1932) has constructed a procedure to solve the
word problem for an arbitrary group with a single defining
relation. Similarly V. A. Tartakovski (1949) - (1952), H. Sheik (1956)
and J. L. Britton (1956), (1957) have given solutions for the word
problem in special classes of groups. However, the question of the
existence of a procedure for the solution of the word problem for
groups\pageoriginale in general remained open until Novikov (1952), (1955), (1958)
proved that in general the word problem for groups is not
soluble. Later, Boone (1954) (1955) (1957) (1958) (1959) and Britton
(1958) gave different proofs for the insolubility of the word problem
for groups. 

We conclude this chapter with a precise statement of the insolubility
of the word problem for groups. 

\begin{theorem*}[(Novikov, Boone, Britton)] 
  There is a finite presentation\break $(E ; R)$ such that the word problem
  is insoluble in the group 
  $$
  G = gp(E ; R),
  $$
  in the sense that to every effective procedure $M$ that purports to
  solve the word problem for $G$, there is a word $w_M (\underbar{e})$
  such that the equation 
  $$
  w_M (\underbar{e}) =1
  $$
  defeats the procedure.
\end{theorem*}
