
\chapter{Identical Relations and Varieties of Groups}\label{chap5} % chapter 5.

\section{}\label{chap5:sec1} % sec 1.

In\pageoriginale the preceding chapter we have seen that an arbitrary
mapping from a 
set generators of a free group into any other group can be extended to
a homomorphism. In fact this property completely characterises the
free groups. In order to generalise this notion of being ``free", we
introduce certain classes of groups called \textit{varieties} of groups  

While proving that the free groups of different ranks are not
isomorphic we have come across an example of a group $G$ in which the
equation $x^2 =1$ holds for all $x$ in $G$. Such equations are called
\textit{identical relations} or \textit{laws}. 

\begin{defi*}
  A law or identical relation is a relation of the form
  $$
  u (\underbar{X}) = v(\underbar{X})
  $$
  where $u$ and $v$ are words in $\underbar{X} = (X_1, \ldots, 
  X_n)$. We say that the law $u (\underbar{X}) = v(\underbar{X})$
  holds in a group $G$ if the equation $u (\mathfrak{f}) =
  v(\mathfrak{f})$ holds when we substitute arbitrary elements $g_1,
  \ldots,  g_n$ of $G$ for the ``variables" $X_1, \ldots,  X_n$. For
  instance if $u(\underbar{X}) = X_1 X_2$ and $v (\underbar{X})= X_2
  X_1$, then in an abelian group the law $u (\underbar{X}) =
  v(\underbar{X})$ holds.  
  
  The following fundamental relations can be easily verified 
  \begin{enumerate}[\rm (1)]
  \item If $u \equiv v$, then $u = v$ is a law.
  \item If $u = v$, is a law then so is $v= u$.
  \item If\pageoriginale $u = v$ and $v = w$ are laws, then so is $u = w$.
    \item If $u=v$ is a law, then so is $u^{-1}= v^{-1}$.
  \item If $u = v$ and $u' = v'$ are laws, then $uu' = vv'$ is a law.
  \item $XX^{-1} = 1$ and $X^{-1}X =1$ are laws.
  \item If $u (\underbar{X}) \equiv u (X_1, \ldots,  X_n) = v(X_1,
    \ldots,  X_n) \equiv v(\underbar{X}) $ is a law and $Y_1
    (\underbar{Z})$, $\ldots Y_n (\underbar{Z})$ are words in variables
    $Z_1, \ldots,  Z_p$ then $u (Y_1(\underbar{Z}), \ldots ,Y_n
    (\underbar{Z}))$ $= v (Y_1(\underbar{Z}), \ldots,  Y_n
    (\underbar{Z}) )$ is a law. 
  \end{enumerate}
\end{defi*}

The rule (7) is a called the \textit{substitution rule}.

[If we assume $XX^{-1} =1$ and $(7)$ we can derive the law $X^{-1}X =
  1$. For put $Y = X^{-1}$. Then $YY^{-1} =1$ is a law. i.e. $X^{-1}
  (X^{-1})^{-1}= X^{-1} X=1$ is a law.] 

These rules can be used can be used to derive from given laws that are
valid in a group further laws that ``follow" from the given laws. 

\begin{example*}
  If $X^2 = 1$ is a law in a group, then so is
  $$
  XY = YX
  $$
\end{example*}

\begin{proof}
  The law $XY = YX$ is equivalent to $X^{-1} Y^{-1} XY = 1$. Now 
  \begin{align*}
    X^{-1} Y^{-1} XY & = X^{-1}Y^{-1} XX^{-1} Y^{-1}XX^{-1}X^{-1} XYXY\\
    & = (X^{-1} Y^{-1}X)^2 (X^{-1})^2 (XY)^2
  \end{align*}
  Applying $(5)$ and $(7)$ we have $X^{-1}Y^{-1}XY = 1$, i.e. $XY =
  YX$ is a law.  
\end{proof}

It is easily seen (as for relations) that every law $u(\underbar{X}) =
v(\underbar{X})$ is equivalent to a law $w (\underbar{X}) = 1$; and it
is often convenient to write all laws in this from. 

\section{Varieties}\label{chap5:sec2} % sec 2.

Throughout this chapter we shall assume that the set of variables 
$\bigg\{ X_1, X_2$, $\ldots \bigg\}$ is countable. This is just for
convenience and not a real restriction. 

Let $L$ be a set of laws invariables $\bigg\{ X_1, X_2, \ldots
\bigg\}$. The class of all groups satisfying the laws of $L$ is called
\textit{variety}. We call this the variety \textit{defined} by $L$ and
denote it by $V_{=L}$ as it clearly depends on $L$. For example if $L$
consists of the single law $X^{-1}_1 X^{-1}_2  X_1 X_2 =1$, then
$V_{=L}$ is the class of all abelian groups. 

A variety may be defined by different sets of laws. For instance if $L
= \bigg\{ X^2_1 = 1\bigg\}, L' = \bigg\{ X^2_1 = 1, X^{-1}_1X^{-1}_2
X_1 X_2 = 1\bigg\}$, then $V_{=L}= V_{=L'}$. 

It is easily seen that if $L \subseteq L'$, then $V_{=L'} \subseteq
V_{=L}$. We say that a variety $\underset{=}V$ is \textit{finitely
  based} if there exists a finite set of laws defining
$\underset{=}V$.  

In this context there are still some undecided questions. 

\begin{prob*}
  Are all varieties of groups finitely based?
\end{prob*}
 
Let $\underset{=}C$ be a class of groups, and consider the ``least
variety'' to which all groups of  $\underset{=}C$ belong: this is the
variety defined by all those laws that are (simultaneously ) valid in
all groups in $\underset{=}C$. We can take, as the simplest case
$\underset{=}C$ to consist of just a single group $G$. 

\begin{prob*}
  If $\underset{=}V$ is the least variety to which the finite group
  $G$ belongs, is $\underset{=}V$ necessarily finitely based?   
\end{prob*}
 
Even\pageoriginale this problem is not solved in general; only if $G$ is further
assumed to be nilpotent is the answer known to be positive
[R.C. Lyndon $1952$]; of. also $p.163$. 

Let $V_{= L}$ be a variety, without loss of generality we can assume
that all laws in $L$ are of the form $w=1$, where $w$ is a normal word
in the variables $X_1,X_2,\ldots $. Let $E$ be any set with
$|E|=n$. Let $R$ be the set of all relations of the form
$w(e_1,\ldots,e_m)=1$ with $e_i$ arbitrary elements of $E$ and $w(X_1,
\ldots, X_m)=1$ a law in $L$ and $m \leq n$. Consider the group $F_L =
gp(E;R)$. Now, if $G$ is a group in the variety $V_{=L}$, then any
mapping $\varphi$ of $E$ into $G$ is extendable to a homomorphism
$\varphi^*$ of $F_L$ into $G$. For if $w(e_1,\ldots,e_n)=1$ is a
relation in $R$, then $w(X_1,\ldots,X_n)=1$ is a law in $L$; and
therefore, since $G \in  V_{=L}, w(e^\varphi_1,
\ldots,e^\varphi_n)=1$ is a relation in $G$. Thus the set $R$ of
defining relations in $F_L$ go over to relations in $G$ upon applying
$\varphi$ 

Therefore, by von Dyck's theorem the mapping $\varphi$ is extendable
to a homomorphism $\varphi^*$ of $F_L$ into $G$. Thus in a way, this
is a generalisation of free groups, We call $F_L$ a \textit{free group
  of} $V_{=L}$ (reduced free or relatively free) of rank $n$. It is
easy to see that $F_L$ itself is a member of $V_{=L}$ and it depends
upon $n$. In particular if $L$ is the empty set we get the free group
in the ordinary sense (in this context called \textit{absolutely free
  groups)}. 

\section{Burnside conjectures}\label{chap5:sec3}%Sec 3.

Let\pageoriginale $L$ be the set consisting of the single law $X^n=1$. We denote the
corresponding $V_{=L}$ by $B_{=n}$. Group of $B_{=n}$ are called
groups of \textit{exponent} $n$. We call $B_n$ the \textit{Burnside
  variety} after $W$.  
Burnside (1852-1927). There is a problem connected with this known as
the Burnside conjectures (Burnside $W.1902$). We first state the
original conjecture, now known as the Full Burnside Conjecture; and
afterwards a weaker form, the so-called Restricted Burnside
Conjecture. \textit{Full Burnside Conjecture}. Every finitely
generated group in $B_{=n}$, that is of finite exponent $n$, is
finite. Let $B_{d, n}$. denote a $d$ generator free group of
$B_{=n}$. The Full Burnside Conjecture is equivalent to saying that $|
B_{d, n}| < \mathscr{N}_\circ$, for every positive integer $d$ for
every group with $d$ generators and exponent $n$ is an epimorphic
image of $B_{d, n}$. 

The following problem is weaker than the above conjecture.

\heading{Restricted Burnside Conjecture.}

There is a bound $\beta(d, n)$ such that every finite $d$ generator
group of exponent $n$ has order $\leq \beta(d, n)$. This conjecture is
an easy consequence of the full Burnside conjecture. For if the full
conjecture is true, then $B_{d, n}$ is finite and we can take
$\beta(d, n) = |B_{d, n}|$. The present state of knowledge of the
Burnside conjecture is for from complete. The following are the
results so far obtained in this direction. In the following $d$
denotes the number of generators, $n$ the exponent. We abbreviate the
Restricted Burnside  

Conjecture\pageoriginale and the full Burnside conjecture $RBC$ and $FBC$
respectively. 

\noindent 
\begin{tabular}{cccccp{4.4cm}}
\hline
\hline
d & n & RBC & FBC & &\multicolumn{1}{c}{REMARKS} \\
\hline
all & $2$ & & true & &Trivial. In fact $|B_{d,2}|=2^d$.\\
all & $3$ & & true & & Burnside (1902). The order of $B_{d,3}$ was given
by Levi and van der Waerden (1933).
 $|B_{d,3}|=3^{\binom{d}{1}+\binom{d}{2}+\binom{d}{3}}$. \\
all & 4 & & true & &Sanov(1940). The order of $B_{d,4}$ is not known.\\
$2$ & $5$ & true & unsolved & &Kostrikin (1955). \\
all & $5$ & true & unsolved & &G.Higman (1956). \\
all & $6$ & true & &  &P. Hall and G. Higman (1956). \\
all & $6$ &  & true &   &M. Hall, Jr. (1959).\\
all & $12$ & true & unsolved & &P. Hall and G. Higman (1956). \\
all & all prime & p true & - & &Kostrikin(1959).\\
all & \multicolumn{1}{p{1.3cm}}{$pq$  ($p, q$ different primes)} &
true & unsolved  &   {\Bpara{-8}{-15}{180}{30}}&
      \raisebox{-.2cm}{\multirow{3}{4.4cm}{follows form a combination of Kostrikind (1959),
      Hall and Higman(1956)}}\\
all &  \multicolumn{1}{p{1.3cm}}{4p (p, a prime)}&  true & unsolved&&   \\
$2$ & $\geq $ & $72$ &  not true & &Novikov(1959).\\
\hline
\end{tabular}

\section[A consequences of the result of Novikov...]{A consequences of the result of Novikov (1959) and Kostrikin
  (1959)}\label{chap5:sec4}%4 

Using\pageoriginale the result of Novikov and Kostrikin, we shall derive an
interesting consequence. As the Burnside conjecture is not true for
$d=2, n \geq 73$, it follows that $B_{2,73}$, the $2$ generator free
group of $B_{=73}$, is infinite. But $73$ is a prime, and therefore by
Kostrikin's result, there exists a maximal finite 2 generator group
of exponent 73. Let us denote this group by $B^*_{2,73}$. We know that
$B^*_{2,73}$ is an epimorphic image of $B_{2,73}$ and therefore is
isomorphic to a quotient group of $B_{2,73}$. Thus $B^*_{2,73} \cong
B_{2,73}/N, N \Delta B_{2,73}$. Therefore $N$ is an infinite normal
subgroup of finite index in $B_{2,73}$. We now state the following
theorem without proof. 

\begin{theorem*}[(0, Schreier (1972); see Kurosh (1956) pp.36-37)] 
  A subgroup
  of finite index of a finitely generated group is finitely
  generated. 
\end{theorem*}

By this theorem $N$ is finitely generated. Now, it is known that a
finitely generated group contains a maximal normal
subgroup. (B.H. Neumann $1937^b$). Let $M$ be a maximal normal
subgroup in $N$. 
Then it is easily seen that $N/M$ is simple; that is to say, $N/M$
does not contain any proper non-trivial normal subgroup. We assert
that the group $N/M$ is infinite. To prove this we quote another
theorem with out proof. 

\begin{theorem*}[(R. Baer $1953$)]
  If a finitely generated group contains a prop\-er subgroup of finite
  index it also contains a characteristic (for definition see section
  6 of this chapter) proper subgroup of finite index. 
\end{theorem*}

If\pageoriginale $N/M$ is finite, by the above theorem, there exists a
characteristic proper subgroup $K$ of finite index in $N$. It follows
that $K$ is a normal subgroup of $B_{2,73}$ and is of finite index
in $B_{2,73}$. Therefore $B_{2,73/K}$ is a finite group of exponent
$73$, whose order exceeds that of $B^*_{2,73}$. This is
impossible. Therefore $N/M$ is infinite. Thus we arrive at in infinite
group $N/M$ which is simple, finitely generated and of exponent 73. 

\section{}\label{chap5:sec5}% 5

We return to the considerations of section \ref{chap5:sec2}. Let $V_{=L}$ be a
variety determined by a set of laws $L$. Without loss of generality we
can assume that every law of $L$ is of the form $w(X_1,\ldots,X_n)=1$
where $w(X_1,\ldots,X_n)$ is a normal word in the variables $X_1,
X_2,\ldots$. We denote by $F_n$ the free group generated by the
variables $X_1, X_2, \ldots,X_n$ and by $F$ the free group generated
by all the variables $X_1, X_2,\ldots$. That is to say, 
$$
F_n =gp\left(\bigg \{X_1,\ldots, X_n \bigg \}, \phi\right),~F_\omega = gp
\left(\bigg\{X_1, X_2,\ldots \bigg\}, \phi\right) 
$$

By $F$ we shall mean either $F_n$ or $F_\omega$. With every $V_{=L}$
we associate a subgroup $W$ of $F$ in the following way. Define 
\begin{equation*}
  \left\{
  W=
  w(X_1,\ldots,X_m) \Bigg|
  \begin{aligned}
    & w(X_1,\ldots,X_m)=1 \\
    & w(X_1,\ldots,X_m) \in ^F
  \end{aligned}
  \text{ valid in all group } of ~\overset{V}{=}L
  \right\}
\end{equation*}
That $W$ is a group is easy to verify.

Now\pageoriginale let $F_L$ be a free group of $V_{=L}$ with $E$ as the set of
generators and of the same rank as $F$. Consider some one-one mapping
$\varphi$ of $X$ onto $E$, where $X$ denotes the set of generators of
$F$. We can extend $\varphi$ to an epimorphism $\varphi^*$ of $F$ onto
$F_L$. The kernel of $\varphi^*$, by the definition of $F_L$, is
precisely the group $W$ we have defined above. Therefore $W$ is a
normal subgroup of $F$ and $F_L$ is isomorphic to $F/W$. The
substitution rule which we have for laws in a group gives some more
information about $W$. If $w(x_1, X_2, \ldots, X_m)\in  W$, and
$Y_1(\underbar{X}),\ldots,Y_m(\underbar{X}) \in  F$, then also
$w(Y_1(\underbar{X}),\ldots,Y_m(\underbar{X})) \in  W$.  
 
We make the following definition.

\begin{defi*}
  Let $E$ be any set, $S \subseteq E$ and $\eta$ a mapping of $E \to
  E$. We say that the subset $S$ admits the mapping $\eta$ if $S^\eta
  \subseteq S$. 
\end{defi*}
 
\setcounter{theorem}{0}
\begin{theorem}\label{chap5:sec5:thm1}%Thm 1
  The subgroup $W \leq F$ admits all endomorphisms of $F$.
\end{theorem} 
 
\begin{proof}
  Let $\eta$ be any endomorphism of $F$ and $X^\eta_i =
  Y_i(\underbar{X})$. If $w(X_1,\ldots$, $X_m)$ is in $W$, then 
  \begin{align*}
    \bigg ( w(X_1,X_2,\ldots, X_m) \bigg) ^\eta & = w (X^\eta_1,\ldots
   , X^\eta_m)\\
    & = w(Y_1(X),\ldots, Y_m(X)) \in  W.
  \end{align*}
  Therefore $W^\eta \subseteq W$. This proved the theorem.
 \end{proof} 
 
Let $G$ be any group. For every $t \in  G$, we define the
mapping $\varphi_t$ of $G$ onto itself such that 
$$
x^{\varphi_t} = t^{-1} xt \text{ for all } x \in  G.
$$
now $(xy)^{\varphi_t}=t^{-1} x y t = (t^{-1}x t) (t^{-1} y
t)=(x)^{\varphi_t} y^{\varphi_t}$ for all $x$ and $y$ in\pageoriginale
$G$. Therefore $\varphi_t$ is an endomorphism of $G$. But 
$$
x^{\varphi_t \varphi_{t-1}} = (t^{-1}x t)^{\varphi_t} = t(t^{-1} x
t)t^{-1} = x = x^{\varphi_t -1 \varphi_t}. 
$$
Thus $\varphi_t \varphi_{t^{-1}} = \ell = \varphi_{t^{-1}} \varphi_t$; in
other every $\varphi_t$ has a two sided inverse. Thus $\varphi_t$ is
an automorphism of $G$. We call $\varphi_t$ an \textit{inner
  automorphism} of $G$. An automorphism which is not an inner
automorphism is called an \textit {outer automorphism}. 
  
 Let us denote by $A_I$ the set of all inner automorphisms of $G$. It
 is easy to see that $A_I$ is a group. There is a natural mapping
 $\varphi$ of $G$ onto $A_I$ defined by $s^\varphi = \varphi_s$ for
 all $s$ in $G$. This mapping $\varphi$ is easily seen to be an
 epimorphism. Then kernel $Z$ of $\varphi$ consists precisely of those
 elements of $G$ which commute with every element of $G$. [For proofs
   see Kurosh (1955), Ch. 4, \S 12]. We call $Z$ the center of
 $G$. By the definition of inner automorphisms it follows that $N
 \Delta G$ if and only if $N$ admits all inner automorphisms of $G$. 
 
A subgroup $H \leq G$ is \textit{characteristic} in $G$ if it admits
all automorphisms of $G$. Similarly a subgroup $H \leq G$ is
\textit{fully invariant} in $G$ if it admits all endomorphisms of
$G$. By the definition of full invariance it follows that the subgroup
$W$ in Theorem \ref{chap5:sec5:thm1} is fully invariant in $F$. Every fully invariant
subgroup of $G$ is trivially characteristic in $G$ and every
characteristic subgroup of $G$ is normal in $G$. We remark that the
centre $z$ of a group $G$ is a characteristic subgroup.\pageoriginale For if $a
\in  Z$, then $ax = xa$ for every $x$ in $G$. Therefore 
$$
a^\top x^\top = (ax)^\top = (xn)^\top = x^\top a^\top
$$
for every automorphism $\top$ of $G$. Now since $x^\top$ runs through
all the elements of $G$ it follows that $a^\top$ is in $Z$ and
therefore $Z$ is a characteristic subgroup of $G$. In general the
centre of a group is not a fully invariant subgroup. [See Kurosh
(1955), ch. 4 ~15]. 
 
One can easily verify that the intersection of an arbitrary family of
characteristic (fully invariant) subgroups of a group is a
characteristic (fully invariant) subgroup. Thus we can talk of
characteristic (fully invariant subgroup generated by a set of
elements and also of the lattice of characteristic (fully invariant)
subgroups of a group. 
 
In general a characteristic subgroup is not a fully invariant
subgroup. [See Neumann and Neumann (1951)]. The following is an
unsolved problem in this direction. 
 
\noindent \textbf{Unsolved problem.} 
  Is there a characteristic subgroup of a free group $F$ of infinite
  rank which is NOT fully invariant in $F$? 
 
\begin{theorem}\label{chap5:sec5:thm2}
  The relation ``characteristic'' and ``fully invariant'' are transitive;
  that is to say, if $K \leq H \leq G$ with $K$ characteristic (fully
  invariant) in $H$ and $H$ characteristic (fully invariant) in $G$
  then $K$ is characteristic (fully invariant) in $G$. 
\end{theorem} 
 
\begin{proof}
  Let\pageoriginale $\alpha$ be an automorphism of $G$; $\alpha'$ the restriction
  of $\alpha$ to $H$. Then, because $H$ is characteristic in $G, H^\alpha \leq
  H$. Applying the automorphism $\alpha^{-1}$ to $H$, we have
  $H^{\alpha-1} \leq H$. Therefore $H=(H^{\alpha-1})^\alpha \leq
  H^\alpha$. Hence we have $H^\alpha = H$. i.e. $H^{\alpha'}=H$. Therefore
  $\alpha'$ is an automorphism of $H$. Now since $K$ is characteristic
  in $H$, we have $K^\alpha = K^{\alpha'} \leq K$. hence $K$ is
  characteristic in $G$. 
\end{proof} 
 
The proof in the case of full invariance is similar and actually even
easier and we omit it. 
 
The transitivity is not true for the relation ``normal''. In other words
if $K \Delta H \Delta g$, in general it is not true that $K \Delta
G$. For example take for $G$ the symmetric group $S_4$ of permutations
on four letters or the alternating group $A_4$. Let 
\begin{align*}
  H & = V_4 = \bigg\{ 1, (12)(34), (13)(24), (14)(23)\bigg \} ~~\text{
    and } \\ 
  K & = \bigg \{1, (12) (34)\bigg \}.
\end{align*} 
 We know that $H \Delta G$, and $K \Delta H$. Now $(123) \in 
 A_4$. $(123)^{-1} = (132)$ and $(123)^{-1}K(123) = \bigg \{1, (14)
 (23)\bigg \} \neq K$. Therefore $K$ is not normal in $G$.
 
We say that $H \leq G$ is \textit{accessible} (or \textit{subnormal})
in $G$ (notation $H \Delta \Delta G)$ if there exists subgroups $H_0 =
H, H_1,\ldots, H_n = G$, such that $H_0 \Delta H_1 \Delta H_2 \cdots
\Delta H_n $. 
 
The accessible subgroups of finite group were introduced by
H. Wielandt (1939) and further studied by H. Wielandt and recently
by\pageoriginale B. Huppert. It is easy to verify that the intersection of two and
hence the intersection of a finite number of accessible subgroups is
an accessible subgroup. The intersection of an infinite number of
accessible subgroups need not be an accessible subgroup. 
 
If a group $G$ has a composition series [Kurosh (1955), $CH. 5, \S
  16$] then the join of any two accessible groups is again an
accessible group (Wielandt (1939)). The following is an unsolved
problem. 
 
\noindent \textbf{Unsolved problem. }
  Is the join of two accessible subgroup of an infinite group (without
  composition series) accessible? 
 
\setcounter{section}{6}
\section{Verbal Subgroups}\label{chap5:sec7}%Sec 7
 
Let $L$ be any set of words \footnote{This is a slight change of
  notation - earlier $L$ stood for a set of laws $=1$, now only for
  the set of their left-hand sides.} in the variables $X_1, X_2,\ldots
$ and $G$ a group. Consider the set, 
 $$
 S= \Bigg \{w(g_1,\ldots,g_n) \bigg | w(x_1,\ldots,x_n)\in  L, 
 g_i \in  G i=1,2,\ldots n \Bigg\} 
 $$
 This is not in general a subgroup of $G$. We call $H=gp(S) \leq G$,
 the word \textit{subgroup} or a \textit{verbal subgroup} defined by
 $L$. 
 
\begin{theorem}\label{chap5:sec7:thm3}%Thm 3
  Every verbal subgroup $H$ of a group $G$ is fully invariant.
\end{theorem}
 
\begin{proof}
  Let $\eta$ be an endomorphism of $G$ and the verbal subgroup $H$ be
  defined by $L$. It is enough to prove that $S^\eta \subseteq S$, for
  every endomorphism $\eta $ of $G$, where $S$ is the set of
  generators of $H$ as defined above. Now if $w(g_1,\ldots 
 ,  g_n) \in  S, w(X_1,\ldots,X_n)\in  L_\eta
  $, then $\bigg\{w(g_1,\ldots,g_n)\bigg \}^\eta = w(g^\eta
  _1,\ldots$, $g^\eta_n) \in  S$. Therefore $S^\eta \subseteq S$;
  this\pageoriginale is true of every endomorphism of $G$. Hence $H$ is fully
  invariant in $G$. The converse of this theorem is not true in
  general; but happens to be in the case of free group. 
\end{proof}  
 
\begin{theorem}\label{chap5:sec7:thm4}%Thm 4
  Every fully invariant subgroup of a free group is verbal.
\end{theorem} 
 
\begin{proof}
  Let $W$ Be a fully invariant subgroup of a free group $F$. Let $L$
  be the set of all normal words that occur in $W$. If
  $Y_1(\underbar{X}),  \ldots,Y_n (\underbar{X}) \in  F$, where
  $\underbar{X}=(X_1,\ldots,X_n)$ and $X_i \in  X$, and where
  $X$ denotes the set of variables as well as the set of generators of
  $F$, then the mapping defined by 
  $$
  X^\eta_i = Y_i (\underbar{X}), ~ i = 1,\ldots,n,
  $$
  can be extended to an endomorphism of $F$ which also we denote by
  $\eta$. 
  Now if $w(X_1,\ldots,X_n) \in  L$, then
  $w(X_1,\ldots,X_n)^\eta =
  w(Y_1(\underbar{X}),\ldots,Y_n(\underbar{X}))\in  W$ as $W$
  is fully invariant in $F$. Therefore  
  \begin{multline*}
  S= \Bigg \{w(Y_1(Y_1(\underbar{X}),\ldots,Y_n (\underbar{X})) \bigg |\\
  w(X_1,\ldots,X_n)\in  L,   
  Y_i (\underbar{X}) \in  F,  i=1,2,\ldots n \Bigg\}
  \end{multline*}
  is contained in $W$. But clearly also $W \subseteq S$. Thus $S=W$, and
  also $gp(S)=W$. Hence the theorem. 
\end{proof}

It follows that the intersection of any arbitrary family of verbal
subgroup of a free group is a verbal subgroup. In general in an
arbitrary group this is not true [B.M. Neumann $(1937^a)$]. It is easy
to verify that the join of two verbal subgroups of a group is a verbal
subgroup. 

\section{}\label{chap5:sec8}%8.

We\pageoriginale shall now give an important example of a verbal subgroup. Let $G$ be
any group. Let $L$ consist of the single word $X^{-1}_1 X^{-1}_2 X_1
X_2 = [X_1, X_2]$. The verbal subgroup $G'$ of $G$ defined by $L$ is
called the \textit{commutator subgroup} or \textit{derived subgroup}
of $G$. 
  
Evidently the commutator subgroup of an abelian group is the trivial
group. For any group it is easily seen that the quotient group $G/G'$
is abelian [Kurosh (1955)]. 
  
\begin{theorem}\label{chap5:sec8:thm5}%Thm 5
  Let $W$ be a verbal subgroup, defined by a set $L$ of words, of the
  free group $F$. Then the quotient group $F/W$ is the free group of
  the variety $V_{=L}$ defined by the laws $w(\underbar{X})=1$, for
  all $w(\underbar{X}) \in  L$ and it has the same rank as
  $F$. 
\end{theorem}  
  
\begin{proof}
  Now $F_L$, the corresponding free group of the variety $V_{=L}$, is
  isomorphic to $F/W^*$, where $W^*$ consists of all $w(X_1, \ldots,
  X_n)$ such that $w(X_1 , \ldots , X_n)=1$ is a law in all the groups
  of $V_L$. We also know that $W^*$ is fully invariant in $F$. If $w(X_1,\ldots,
  X_n)$ is in $L$, then $w(Y_1(\underbar{X}), \ldots,Y_n(\underbar{X}))$
  $\in  W^*$ for arbitrary $Y_i (\underbar{X}) \in 
  F$. Therefore $W \leq W^*$. Now $F/W \in  V_{=L}$. Therefore,
  if $w(X_1, \ldots, X_n) \in  W^*$ then the law $w(X_1,\ldots
 ,  X_n)=1$ holds in $F/W$. In other words $w(X_1,\ldots,X_n)
  \in  W$. Therefore $W^* \leq W$; we get $W = W^*$. Hence the
  theorem. 
\end{proof}  
  
\begin{theorem}\label{chap5:sec8:thm6}%Thm 6
  Every verbal subgroup $W$ of a free group $F$ is the fully invariant
  closure of the set $L$ (i.e. the fully invariant subgroup generated
  by $L$) of words consisting of either one or no word of the from
  $X^k_1$ and apart from that ``commutator words'' i.e. words contained
  in\pageoriginale the derived group $F'$. 
\end{theorem}  
  
\begin{proof}
  We have already remarked that the quotient group $F/F'$ is
  abe\-lian. Therefore every $w \in  W$ can be written as
  $w=X^{k_n}_1 \cdot X^{k_n}_n w'$ with $w' \in  F'$. Let
  $\eta$ be the endomorphism of $F$ defined by $X^\eta_1 = X_1,
  X^\eta_i = 1$ for $i \neq 1$. Since $W$ is fully invariant in $F$ it
  follows that $w^\eta = X^{k_1}_1 w'^{\eta}=X^{k_1}_1$. Similarly
  $\eta_i$ defined by $X^{\eta_i}_i = X_1$ and $X^{\eta_i}_j = 1$ for
  $j \neq i$, generates an endomorphism of $F$ and therefore
  $w^{\eta_i}=X_1^{k_i} w'^{\eta_i}= X^{k_i}_1$, since $w'^{\eta_i}=1$. Let
  $gp(X^k_1) = gp(X_1) \cap W$. Then $k/k_i$ for $i=1,2,\ldots n$. If
  $\Pi_i$ is the endomorphism defined by $X^{\Pi_i}_1 = X_i,
  X^{\Pi_i}_j = 1$ for $i \neq j$, then $(X^k)^{\Pi_i} = X^k_i \varphi
  W$. Let $L$ be the set consisting of $X^k_1$ and all the $w's$ that
  occur when each $w \in  W$ is written as $w = X^{k_1}_1
  \cdots X^{k_n}_n w'$. It is easily seen that any invariant subgroup
  of $F$ that contains $L$ also contains $W$. But $W$ itself is fully
  invariant in $F$. Hence $W$ is the fully invariant closure of
  $L$. When $k=0$, $L$ is a subset of $W'$. 
\end{proof}  
  
\begin{coro*}[B.M. Neumann, $1937^1$] 
  If $k \neq 1$, then the reduced free \break groups of the variety are
  non-isomorphic for different ranks. [If $k=1$, the free groups of
    the variety are all the trivial groups.] 
\end{coro*}  
