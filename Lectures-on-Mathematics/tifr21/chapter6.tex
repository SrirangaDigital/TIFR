
\chapter{Group-theoretical Constructions}\label{chap6}%Chap VI

\section[The Cartesian product and the...]{The Cartesian product and the direct product of a family of
  groups}\label{chap6:sec1}%Sec 1 

Let\pageoriginale $\bigg \{G_i\bigg\}_{i \in  I}$ be a family of group
indexed by a non-empty set $I$. Let $T$ denote the set of all
functions on $I$ with values in $G_i$. Consider the set $P$ defined by  
$$
P = \Bigg \{ f \in  T \bigg | f(i) \in  G_i \text{ for
  all } i \in  I \Bigg \}. 
$$
We turn $P$ into a group by introducing the following multiplication:
If $f, g \in  P$. Then 
$$
fg = h, \text{ where } h(i) = f(i) g(i), \text{ for all } i \in  I.
$$
It is easy to see that $h \in  P$. We take the function $e
\in  P$, defined by  
$$
e(i) = 1_i \text{ for every } i \in  I
$$
(where $1_i$ is the unit element of $G_i$) as the unit element. For, 
$$
ef = fe = f, \text{ for all } f \in  P.
$$
For every $f \in  P$, we take the function $f^{-1}$ defined by 
$$
f^{-1}(i) = (f(i))^{-1}, \text{ for every } i \in  I,
$$
as the inverse of $f$. It is easy to verify that $f^{-1} \in 
P$ and $ff^{-1} = f^{-1} f = e$, for every $f \in  P$. We have
only to verify the associative\pageoriginale law. Let $f, g, h \in  P$. We
have 
\begin{gather*}
  ((fg)h)(i) = (fg)(i)h(i) = (f(i)g(i))h(i) = f(i)(g(i)h(i)) \\
  = f(i)(gh)(i) = (f(gh))i,
\end{gather*}
for every $i \in  I$. Therefore for all $f, g, h, \in P$,
$$
(fg)h = f(gh).
$$
This proves that $P$ is a group. We call $P$ the \textit{Cartesian
  product} (unrestricted, full, or strong direct product) of
$\bigg\{G_i\bigg \}_{i \in  I}$. 

Consider now the set $P^*$ defined by
$$
P^* = \Bigg \{ f \Bigg | f \in  P \text{ and }\bigg|\bigg\{ i
\bigg| f(i) \neq 1_i \bigg\} \bigg | < \chi_0\Bigg\}. 
$$

That is to say, $P^*$ consists precisely of all $f \in  P$ with
$f(i)=1_i$ except for a finite number of indices $i$. It is easy to
see that $P^*$ is a subgroup of $P$. The subgroup $P^*$ is known as
the \textit{direct product} (restricted or weak direct product) of
$\bigg\{G_i\bigg\}_{i \in  I}$. If $| I | < \chi_0$, then $P =
P^*$; that is to say, the concepts of the Cartesian product and the
direct product coincide when the index set is finite. The two products
we have just defined are important, and they occur frequently in the
group theory. 

Hereafter, we shall denote all the unit elements that occur by $1$;
unless is a possibility of confusion. 

Consider now, for every $i \in  I$, the set
$$
H_i = \left\{ f \bigg | f  \in  P \text{ and } f(j) =1 \text{
  for all } j \neq i \right\}. 
$$
We\pageoriginale claim that $H_i \Delta P$ and that $H_i \cong G_i$. Let $f, g
\in  H_i$. Then $f(j) =1, g(j) =1$, for $j \neq i$. Therefore,
$f^{-1}g(j)= f^{-1}(j) g(j)=(f(j))^{-1} g(j) = 1^{-1}1=1$, for all $j
\neq i$. Hence $f^{-1}g \in  H_i$, and therefore $H_i \le
P$. In fact $H_i \le P^* \le P$. Now, let $f \in  P, h
\in  H_i$. Then  
$$
(f^{-1}hf)(j) = (f(j))^{-1} h(j) f(j) =(f(j))^{-1} 1f(j) =1, \text{
  for } j \neq i. 
$$ 

Therefore $H_i \Delta P$. Consider now the mapping $\prod_i$ of $P$
onto $G_i$ defined by  
$$
f^{\prod_i}= f(i), \text{ for every} f \in  P.
$$

We have, for arbitrary $f, g \in  P$,
$$
(fg)^{\prod i}= (fg)(i) = f(i) = f(i) g(i) =  f^{\prod_i}g^{\prod_i}.
$$

Therefore $\prod_i$ is a homomorphism and in fact, clearly, an
epimorphism. We call $\prod_i$ the projection of $P$ onto $G_i$. Let
us now restrict $\prod_i$ to the subgroup $H_i$. We shall denote this
restricted mapping also by $\prod_i$. We claim that $\prod_i$ is an
isomorphism of $H_i$ onto $G_i$.  To check that this mapping is
`onto', we have only to observe that for every $a \in G_i$, the
function $h_a \in  H_i$ defined by  
$$
h_a (j) =1 \text{ for } j \neq i,  \text{ and } h_a (i) =a
$$
is mapped on a by $\prod_i$. Obviously, the kernel of $\prod_i$ in $H_i$ is
trivial,\pageoriginale and therefore 
$$
H_i  \cong G_i,  \text{ for all } i \in  I. 
$$

Thus we have in $P$ isomorphic copies of the groups $G_i$. The group
$P$ is something called the \textit{internal Cartesian product} of $\{
H_i \}_{i \in  I}$, and the \textit{external Cartesian product}
of $\{ G_i\}_{ i \in  I}$. 

It is easy to see that for $i \neq j$, every element of $H_i$ commutes
with every element of $H_j$. 

We have already seen that $H_i \Delta P^*$, for all $i \in 
I$. We assert now that $P^*$ is the subgroup generated by $\{ H_i
\}_{i \in  I}$ in $P$. Trivially 
$$
gp( \{ H_i \}_{i \in  I}) \le P^*.
$$

Let now $f^* \in  P^*$ with $f^*(i_j)= a_j \neq 1, j=1,  \ldots
, n$ and $f^*(i)=1$ for $i \neq i_1,  \ldots,  i_n$. Define $h_{i_j}
\in  H_{i_j}. j=1, \ldots,  n$ as follows: 
$$
h_{i_j} (i_j) = a_j, h_{i_j}(i)=1 \text{ for } i \neq i_j. 
$$

Then
$$
f^* = h_{i_1} h_{i_2} \cdots h_{i_n} \in  ~gp ( \{ H_i \}_{i \in  I}).
$$

Therefore, $P^*= gp ( \{ H_i\}_{i \in  I})$.

The following, theorem and the example we give show that certain
properties of the $G_i$ are retained in the direct product, but not in
the Cartesian product. 

We call a group \textit{ periodic } if all of its elements are of
finite order.\pageoriginale
 
\setcounter{theorem}{0}
\begin{theorem}\label{chap6:sec1:thm1} % Thm 1
  The direct product of periodic groups is periodic.
\end{theorem}

\begin{proof}
  Let $f \in  P^*$. Let 
  $$
  \left\{ i \bigg| i \in  I, f(i) \neq 1 \right\}= \bigg\{ i_1,
  \ldots,  i_n \bigg\}. 
  $$

  If $m$ is the least common multiple of the orders of $f(i_1), \ldots, 
  f(i_n)$, then $f^m=1$. This proves the theorem. 
\end{proof}

In general this is not true for Cartesian products $P$. For example,
let $G_i= gp(a_i : a^{i+1}_i =1), i=1, 2, 3, \ldots $; that is to say,
$G_i$ is a cyclic of order $i+1$, generated by $a_i$. Consider $f_0
\in  P$ defined by   
$$
f_o(i) = a_i, i = 1,2,3, \ldots
$$

For any positive integer $m$, we have 
$$
f^m_0(m)= a_m^m \neq 1,
$$
therefore $f_o$ is of infinite order.

Let $\{ G_i \}_{i \in  I}$ be a countable family of countable
groups. Then $P^* = gp( \{ H_i\}_{i \in  I})$ is countably
generated, since each $H_i$, being isomorphic to $G_i$, is
countable. On the other hand, the Cartesian product of a countably
infinite family of non-trivial countable groups has the cardinal of the
continuum. For it is easily seen that  
$$
2^ {\mathscr{N}_o} \le |P| \le  \mathscr{N}_o^{\mathscr{N}_o}=
2{\mathscr{N}_o}.  
$$

We\pageoriginale have already remarked that the Cartesian product and the direct
product of a family of groups are equal if the index set $I$ is
finite. (The converse is also true if there are no trivial groups in
the family.) If $I= \{ 1,2, \ldots,  n\}$, we denote this product by   
$$
P = P^* = G_1 \times G_2 \times \cdots \times G_n
$$

(Note that the same notation is used for the set product of the
$G_i$; but there is little danger of confusion.) 

The following theorems are easy to prove. We shall state them here
without proof. 

\begin{theorem}\label{chap6:sec1:thm2} % thm 2
  If $\{ G_i\}_{i \in  I}$ and $\{G'_i\}_{i \in  I}$ are
  two families of groups indexed by the same set $I$, and  
  $$
  G_i \cong G'_i \text{ for every } i \in  I,
  $$
  then $P \cong P'$ and $P^* \cong P'^*$ where $P, P'$ denote the
  Cartesian products of $\{G_i\}_{i \in  I}$ and $\{ G'_i\}_{i
    \in  I}$ respectively, and $P^*, P'^*$ the corresponding
  direct products. 
\end{theorem}

\begin{theorem}\label{chap6:sec1:thm3} %%thm 3
  If $\{ I_j \}_{j \in  J}$ is a partition of the index set
  $I$, and $P_j, P^*_j$ are the Cartesian product and direct product
  of the family  $\{G_i \}_{i \in  I_j}$, then the Cartesian
  product (direct product) of  $\{ P_j\}_{j \in  J} (\{
  P^*_j\}_{j \in  J})$ is isomorphic to the Cartesian product
  (direct product of  $\{ G_i\}_{i \in  I}$.  
\end{theorem}

In particular, if $I= \{ 1, 2, 3 \}$, we have
$$
G_1 \times (G_2 \times G_3) \cong (G_1 \times G_2) \times G_3. 
$$

If\pageoriginale the $G_i$ are all isomorphic to  a group $G$, then we call $P$ the
\textit{ Cartesian power} of $G$, and $P^*$ the \textit{direct power}
of $G$. By Theorem \ref{chap6:sec1:thm2}, we may replace all the $G_i$ by $G$. Then $P$
will be the set of all functions on $I$ with values in $G$. We denote
this set by $G^I$. If $f, g \in  G^I$, then $fg(i)= f(i)
g(i)$. The unit element is the function $e \in  G^I$ such that
$e(i) =1$ for all $i \in  I$. The inverse of $f \in 
G^I$ is the function $f^{-1}$ such that $f^{-1}(i) =(f(i))^{-1}$ for
all $i \in  I$. 

When $I$ is a finite set, say $I= 1, 2, \ldots,  n$, we write $G^n$ for $G^I$.

The Cartesian or direct power of a group $G$ does not depend on the
index set $I$, but only on the cardinal of $I$ (See Kuroshm  $1955$,
$\S 17$). 

\section{The splitting extension}\label{chap6:sec2}% \sec 2

In this section we shall give a group-theoretical construction which
is more general then the direct product. This construction will be
later used in proving certain embedding theorems. 

Let $G$ be any group, and $A \Delta G$, with $G/A \cong B$. We call
$G$ an \textit{extension} of $A$ by $B$. We now pose the following
question. Given two groups $A$ and $B$, does there exist an extension
of $A$ by $B$?  We assert that the direct product of $A$ and $B$ is
one such extension. For, let $G= A \times B$ be the direct product of
$A$ and $B$. According  to our definition of the direct product an
element of $G$ is a function $f$ on the set  $\{ 1,2\}$ with values in
$A \cup B$, such that $f(1) \in  A$, and\pageoriginale $f(2)\in 
B$. We shall denote this function by the pair $(f(1), f(2));$ in other
words $(a, b) \in  A \times B$ is the function on $\{ 1, 2 \}$
such that $f(1) = a, f(2) =b$. Further, if $(a, b), (a', b')
\in  A \times B$, then  
$$
(a, b)(a', b') = (aa', bb') ;
$$ 
the unit element of $A \times B$ is $(1, 1)$ and $(a^{-1}, b^{-1})$ is the
inverse of $(a, b)$ in our new notation. We have seen in the last
section that the projection $\prod_2$ of $G$ onto $B$ is an
epimorphism with the set $\left\{ (a,1) \bigg| a \in  A
\right\}$ as the kernel. Clearly, the kernel is isomorphic to $A$
in a natural way. If we identify this set with $A$, we have  
$$
G/A \cong B.
$$

Thus $G$ is an extension of $A$ by $B$. But in general this is not the
only extension of $A$ by $B$.

We shall now give another method of constructing an extension of $A$
by $B$. Let $\alpha$ be a homomorphism of $B$ into the group of
automorphisms of $A$; this is to say $\alpha (b)$ for any $b
\in  B$ is an automorphism of $A$, and further $\alpha(bb') =
\alpha (b) \alpha(b')$ for all $ b, b' \in  B$: this is the
homomorphism property of $\alpha$. We take the product \textit{set}  
$$
G=B \times A= \left\{ (b,a) \bigg| b \in  B, a \in  A \right\},
$$
and make it a group by introducing the following multiplication: 
$$
(b, a)(b', a')= (bb', a^{\alpha(b')}a'), \text{ for } b, b'
\in  B, \text{ and } a, a' \in  A. 
$$

We\pageoriginale take (1, 1) as the unit elements of $G$. (The unit elements of
both $A$ and $P$ are denoted by $1$. ) For 
$$
(1, 1)(b, a) = (1b,1^{\alpha(b)}a) = (b,a)
$$
as $\alpha (b)$, being an automorphism of $A$, must map $1$ on $1$; and 
$$
(b, a)(1,1) =(b1, a^{\alpha (1)}1) = (b, a), 
$$ 
since $\alpha$ is a homomorphism and thus $\alpha (1)$ must be the
unit be the unit element of the group of automorphisms of $A$, that is
the identity automorphism. The inverse of $(b, a)$ we take as  
$$
(b, a)^{-1}= (b^{-1},(a^{\alpha(b^{-1})})^{-1})
$$
For,  
$$
(b,a)(b^{-1},(a^{\alpha(b^{-1})})^{-1}) = (bb^{-1},
a^{\alpha(b^{-1})}((a^{\alpha(b^{-1})})^{-1}) =(1,1).
$$ 
Similarly,  
$$
(b^{-1},(a^{\alpha(b^{-1})})^{-1})(b,a)
=(b^{-1}b,((a^{\alpha(b^{-1})})^{-1})^{\alpha (b)}a).
$$ 
But 
\begin{multline*}
  ((a^{\alpha(b^{-1})})^{-1})^{\alpha (b)} =
  ((a^{\alpha(b^{-1})})^{\alpha (b)})^{-1}\\ 
  =(a^{\alpha(b^{-1}) \alpha
    (b)})^{-1}= ((a^{\alpha(b^{-1}b)})^{-1} = (a^{\alpha (1)})^{-1}=
  a^{-1}.
\end{multline*}

Therefore, $(b^{-1}, (a^{\alpha(b^{-1})})^{-1})(b, a) =(1, 1)$. We have
now only to verify the associative law. Let $(b, a), (b', a')$ and
$(b'', a'') \in  B \times A$. Then  
\begin{align*}
((b, a)(b', a'))(b'', a'') &=(ba, a ^{\alpha(b')}a')(b'', a'') \\
  &=((bb')b'', (a^{\alpha (b')}a')^{\alpha (b'')}a '') \\
  &= (b(b'b''), (a^{\alpha (b') \alpha (b'') }a'^{\alpha (b'')}) a'') \\
  &=(b(b'b''),  a^{\alpha (b' b'')}a'^{(b'')} a'') \\
  &=(b,a)(b'b'', a'^{(b'')} a'') \\
  &= (b,a) ((b',a')(b'',a'')). 
\end{align*}

Thus\pageoriginale $B \times A$ is a group with the multiplication we have defined.

To show that $G$ is an extension of  $A$ by $B$, we have first to
identify $A$ with some subgroup of $G$. In other words we have to find
a suitable monomorphic image of $A$ in $G$. Consider the mapping
$\prod_1$ of $A$ into $G$ defined by 
$$
a^{\prod_1} = (a,a) \text{ for all } a \in  A.
$$

Now,
$$
(aa')^{\prod_1}= (1,aa') = (11,a^{\alpha(1)} a') = (1,a)(1,a') =
a^{\prod_1} {a'}^{\prod_1} 
$$
and $a^{\prod_1}= (1, 1)$ if and only if $a=1$. Therefore $\prod_1$ is
a monomorphism of $A$ into $G$, the monomorphic image the subgroup
$\left\{ (1, a) \bigg | a \in  A \right\} \le G$. We identify
$A$ with this monomorphic image; in\pageoriginale other words we write a for $(1,
a)$, for all $a \in  A$. 

Similarly, consider the mapping the mapping $\prod_2$ of $B$ into $G$
defined by 
$$
b^{\prod _2} = (b,a), \text{ for all } b \in  B.
$$

We have 
$$
(bb')^{\prod_2} = (bb',1) = (bb', 1^{\alpha (b')}1) = (b,1)(b',1)=
b^{\prod_2} {b'}^{\prod_2} 
$$

Further $b^{\prod_2}= (1, 1)$ if and only if $b=1$. Therefore $\prod_2$
is a monomorphism of $B$ into $G$, and  
$$
B^{\prod_2}= \left\{ (b,a) \bigg| b^* \in  B \right\} \le G.
$$

We identity $B$ with $B^{\prod_2}$ and write $b$ for $(b, a)$, for all
$b \in  B$. 

Now, 
$$
ba= (b, a) (1, a)= (b1,1^{\alpha (1)}a) = (b, a).
$$ 

Therefore every element $(b,a)$ of $G$ can be written as 
$$
(b, a)= ba, \text{ with } b \in  B, a  \in  A.
$$

By the identification we have made, it is easily seen that $A \cap B=
\{ 1 \}$. We claim that the representation of a pair $(b, a)$ as a
product ba is unique. For if  
$$
ba= b' a', \text{ with } b,b' \in  \in  B, a,a' \in  A,
$$
then\pageoriginale ${b'^{-1}}b=a' a^{-1}$. But $A \cap B = \{ 1\}$. Hence,
$$
{b'}^{-1}b =a' a^{-1}=1, \text{ i.e }, a= a', b=b'. 
$$

Consider now the mapping $\prod$ of $G$ onto $B$ defined by
$$
(ba)^{\prod}=b.
$$

(Note that the uniqueness of the representation ba ensure that
$\prod$ is a mapping. ) We assert that $\prod$ is an epimorphism of
$G$ onto $B$ with $A$ as kernel, For, 
$$
((ba)(b'a'))^{\prod} = (bb'a^{\alpha (b')}a')^{\prod} = bb'
=(ba)^{\prod} (b'a')^{\prod} 
$$
for all $b, b' \in  B, a, a' \in  A$.It is easy to see
that the kernel of $\prod$ is $A$ and therefore  
$$
A \Delta G,  G/A \cong B.
$$

Hence $G$ is an extension of $A$ by $B$. We call  $G$ a
\text{splitting extension} (split extension or semi-direct product) of
$A$ by $B$. 

By the above construction it follows that $G$ depends on the
homomorphism $\alpha$ also. In particular, if we take for $\alpha$ the
trivial homomorphism, that is, the mapping which maps every element of
$B$ onto the identity automorphism of $A$, it is easy to see that the
corresponding splitting extension is the direct product of $A$ and
$B$. 

If $\alpha$ is an isomorphism of $B$ onto the group of automorphisms
of $A$, then corresponding splitting extension is known as\pageoriginale the
\textit{holomorph} of $A$. 

We note that in a splitting extension of $A$ by $B$, 
$$
b^{-1} ab=a^{\alpha (b)} \text{ for all } a \in  A ;
$$
that is to say, all the automorphism $\alpha (b)$ of $A$ are induced
by inner automorphisms of the splitting extension.  In particular when
$G$ is the holomorph of $A$, all the automorphisms of $A$ are induced
by the inner automorphisms of $G$. 

Not all extensions of $A$ by $B$ are necessarily splitting
extensions. Consider the group $Q$ generated by two elements $i, j$
with the defining relations  
$$
i^{-1}ji = j^{-1}, j^{-1}ij =i^{-1}.
$$

This group is known as the \textit{quaternion} group (see Coxeter and
Mo\-ser, 1957). It is not difficult to prove the order of $Q$ is $8$,
the element $i$ is four, and the only subgroup of order $2$ in $Q$ is
$\{ 1, i^2 \}$. Let now $A= gp(i)$. Then the subgroup $A$ being of
index $2$ in $Q$ is a normal subgroup of $Q$. Thus $Q$ is an extension
of $A$ by a cyclic group of order $2$. But the only subgroup of order
$2$ of $Q$ is $gp(i^2)$, which  is contained in $A$. Therefore $Q$ is
not a splitting extension of $A$. The subgroup $gp(i^2)$ is also
normal in $Q$, as it is the only subgroup of order $2$ of $Q$. But  
$$
Q/gp(i^2) \cong V_4 = gp(a,b: a^2= b^2 =1).
$$

However,\pageoriginale $Q$ contains only one subgroup of order $2$, hence cannot
contain any subgroup isomorphic to $V_4$. Therefore $Q$ is not a
splitting extension of $gp(i^2)$. 

\section{}\label{chap6:sec3} % Sec 3

The quaternion group $Q$ is a finite group which is presented by two
generators and two relations.  Let $G$ be a group generated by a
minimal set of generators consisting of $d$ elements, and let  the
number of defining relations in these generators be $e$. It is not
difficult to prove that if $e < d$, then the group $G$ is infinite. Thus
for finite groups, one necessarily has $e \ge d$. Obviously the finite
cyclic group are examples of finite  groups with $e = d=1$. Some
examples of finite groups with $e = d =2$ can be found in B.H.Neumann
$(1956)$. 

H.Mennicke (Kiel, Germany now Glasgow ) has shown that the following
group is finite: 
$$
G= gp \left(a, b, c: a^b, b^3, b^c=b^3, c^a= c^3\right).
$$ 

It is not difficult to verify  that $G$ cannot be generated by
generated by fewer than three elements; thus $G$ is an example of a
finite group with $e= d=3$. Later Mennicke and $I. P$. Macdonald
(Manchester) independently have given an infinite sequence if finite
groups  with $e= d =3$. (The results of Mennick and Macdonald are to
be published in Arch. Math. and Canod J.math., respectively. This
suggests the following 

\noindent \textbf{Unsolved problem. }
  Are there finite groups with $e = d =4$ that cannot be\pageoriginale generated by
  fewer than $4$ elements? 

\section{}\label{chap6:sec4}%Sec 4

Let $G$ be a group, and $A, B$ subgroups of $G$ satisfying the
following conditions: 

(i)~ $G= AB$,  \qquad (ii)~ $A \cap B = \{ 1\}$

We call $G$ the \textit{general product} of the subgroups  $A$ and
$B$\footnote{Note:- In the recent literature it is also often
  called the Zappa-Szep-Redei product}. 

If $G$ is the general product of its subgroups $A$ and $B$, then it
can also be written as $G=AB$. For, 
$$
G=G^{-1} = (AB)^{-1} = B^{-1}A^{-1} = BA.
$$

Every $g \in  G$ can be represented as the product of an
element of $A$ and an element of $B$. Moreover, this representation is
unique. For, if $g= ab= a'b'$ with $a, a' \in  A, b,b'
\in  B$, then  
$$
{a'}^{-1} = b' b^{-1} \in  A \cap B = \{ 1 \}
$$ 

Hence ${a'}^{-1} a=1= b' b^{-1}$, i.e, $a=a', b=b'$.

We have seen (section $2$ of this chapter) that if $G$ is a splitting
extension of its subgroup $A$ by a subgroup $B$, then  

(i)~ $G= BA$, \quad (ii)~ $B \cap A= \{ 1\}$~  and ~ (iii)~  $A \Delta
G$. 

We claim that conditions (i), (ii) and (iii) are sufficient in
order that $G$ be a splitting extension of $A$ by $B$. To prove this,
we define\pageoriginale a mapping $\alpha$ of $B$ into the group of automorphisms of $A$ as
follows: for every $b \in  B$, 
$$
a^{\alpha (b)}= b^{-1} ab \text{ for all } a \in  A.
$$ 

Since $A \Delta G$, it admits all inner automorphisms of $G$, and
hence $\alpha (b)$ is an automorphism of $A$. We assert that $\alpha$
is a homomorphism of $B$ into the group of automorphisms of $A$. For, 
\begin{align*}
  a^{\alpha(bb')} & =(bb')^{-1} a(bb') = b'^{-1}(b^{-1} ab)b' =
  b'^{-1}a^{\alpha (b)}b' \\ 
  &= (a^{\alpha(b)})^{\alpha (b')} = a^{\alpha (b) \alpha (b')}
\end{align*}
for every $a \in  A$ and all $b, b' \in  B$. Hence 
$$
\alpha(bb') = \alpha (b) \alpha(b') \text{ for all } b,b' \in  B;
$$
that is, $\alpha$ is a homomorphism.

The condition (ii) immediately gives $ba=b' a'$ is and only if $b=b',
a=a'$. Now, $(ba)(b' a')= bb' b'^{-1} ab' a' = bb' a^{\alpha
  (b')}a'$. This proves that $G$ is a splitting extension of $A$ by
$B$. 

If, decides conditions $(i)$, $(ii)$ and $(iii)$, $G$ also satisfies
$(iv)$ $B \Delta G$, then $G$ is the direct product of $A$ and $B$. For,
$$
a^{\alpha (b)}= b^{-1} ab= a a^{-1} b^{-1} ab =a [a,b]
$$
for all $a \in  A, b \in  B$. And since $A \Delta G, B
\Delta G$, we have  
$$
\displaylines{\hfill 
  [a, b] =  (a^{-1}b^{-1 }a) b = a^{-1} (b^{-1}ab) \,A \cap B= \{ 1\},
  \hfill \cr
  \text{i.e.,} \hfill [a,b] = 1,  \text{ for all } a \in  A, b
  \in  B.\hspace{2.8cm}\hfill } 
$$

That\pageoriginale is $a^{\alpha (b)}=a$ for all $a \in  A$; thus $\alpha
(b)$ is the  identity automorphism of $A$ for every $b \in 
B$. Therefore, $\alpha$ is the trivial homomorphism, and $G$ is the
direct product of $A$ and $B$. 

Conversely if $G$ is the (internal) direct product of its subgroup $A$
and $B$, then $G$ satisfies  (i), (ii), (iii) and (iv). 

Then we have 
\begin{theorem}\label{chap6:sec4:thm4} %Thm 4
  \begin{enumerate}
    \renewcommand{\labelenumi}{\rm \theenumi.}
  \item $G$ is a splitting extension  of $A$ by $B$ if and only if it
    satisfies conditions (i), (ii) and (iii). 
  \item $G$ is the direct product of $A$ and $B$ if and only if it
    satisfies conditions (i), (ii), (iii) and (iv). 
  \end{enumerate}
\end{theorem} 
 
\section[Regular permutation representations...]{Regular permutation representations of a group by right
  multiplications}\label{chap6:sec5} % \sec 5 
 
Let $G$ be a group. We know that the set of  all one-one mapping of
$G$ onto $G$, or \textit{permutations} of $G$ forms a group (called
the symmetric group) with the composition of mapping as
multiplication. We shall embed $G$ in this permutation  group; in
other words, we shall find a monomorphic image  of $G$ in this group. 
 
For every $g \in  G$, we define a permutation $\rho (g)$ of $G$ by 
$$
x^{\rho (g)}= xg,  \text{ for all } x \in  G.
$$
 
It is easy to verify that $\rho (g)$ is a permutation of $G$; but this
also\pageoriginale follows from the homomorphism property to be moved now. Consider the
mapping $\rho$ of $G$ into the group of permutations of $G$, defined
by  
$$
g^\rho = \rho(g) \text{ for all } g \in  G.
$$
 
 We claim that $\rho$ is a monomorphism. Let $g,h \in  G$. Then 
 \begin{align*}
   x^{\rho (gh)} & = x(gh) = (xg) h =x^{\rho(g)}h= (x^{\rho (g)})^{\rho_{(h)}} \\
   &= x^{\rho_{(g)}}, \text{ for all } x \in  G.
 \end{align*} 
 
 Therefore,
 $$
 \rho (gh) =\rho (g) \rho(h), \text{ for all } g,h \in  G.
 $$
 
 Further, $\rho(g)=1$ means
 $$
 x^{\rho(g)}=xg=x, \text{  for all  } x \in G.
 $$
 
 In particular if we take $x=1$, we get $g =1$. Hence $\rho$ is a
 homomorphism with trivial  kernel, that is, a homomorphism. Thus $G
 \cong \rho(G)$. 
 
 We call $\rho(g)$a \textit{right multiplication}, and $\rho (G)$ the
 regular permutation representation by right multiplications. 
 
In this context, we can realise the holomorph of $G$ as a subgroup of
the symmetric group $S_G$ of all permutation of $G$, namely as the
normaliser of $\rho(G)$ in $S_G$. 
 
\section{Wreath Product}\label{chap6:sec6}  %Sec 6
 
Let\pageoriginale $A$ be an abstract group, and $B$ a permutation group of a set
$Y$. Consider $A^Y$, the Cartesian power of $A$; this consists of all
functions  on $Y$ with values in $A$. If $f,g \in  A^Y$, then  
$$
fg(y) = f(y) g(y), \text{ for all } y \in Y. 
$$
 
We want to represent $B$ as an automorphism group of $A^Y$. In other
words we want to find a homomorphism of $B$ into the group of
automorphisms of $A^Y$. For every $b \in  B$, we define a
mapping $\alpha (b)$ of $A^Y$ into $A^Y$ by  
$$
f^{\alpha (b)} (y) = f(y^{b^{-1}}) \text{ for all } y \in  Y.
$$
 
We first prove that $\alpha (b)$ is an endomorphism of $A^Y$. We have 
\begin{align*}
  (fg)^{\alpha (b)} (y) & = (fg)(y^{b^{-1}}) = f(y^{b-1}) g (y^{b^{-1}}) \\
  & =f^{\alpha (b)} (y) g^{\alpha (b)} (y) = (f^{\alpha (b)}g^{\alpha (b)}) (y), 
\end{align*}
for all $y \in  Y$. Therefore 
$$
(fg)^{\alpha (b)}= f^{\alpha (b)} g^{\alpha (b)}, \text{ for all } f,g
\in  A^Y. 
$$

Further,
\begin{align*}
  f^{\alpha (bb')}(y) & =f(y^{(bb')^{-1}}) = f(y^{b'^{-1} b^{-1}}) \\
  &= f((y^{b'^{-1}})^{b^{-1}}) = f^{\alpha (b) (y^{b'^{-1}})}  =
  (f^{\alpha (b)})^{\alpha (b')}(y) \\ 
  &= f^{\alpha (b) \alpha (b')}(y),~ \text{ for all } ~y  \in  Y.
\end{align*}
Hence\pageoriginale \qquad $ \alpha (bb') = \alpha (b) \alpha (b')$

Again, this is true for all $b, b' \in  B$, hence the mapping
$\alpha$ of $B$ into the semigroup of endomorphisms of $A^Y$ is a
homomorphism. It follows that $\alpha(B)$ is a group, and also that
every $\alpha(b)$ is an automorphism of $A^Y$. (Incidentally, one
easily verifies that $\alpha$ is a monomorphism, provided that $A$ is
non - trivial). 

We now form the splitting extension $P$ of $A^Y$ by $B$ in terms of
$\alpha$. Every element $p$ of $P$ can be written uniquely as  
$$
p = bf,  b \in  B, f \in  A^Y.
$$
if $p' = b' f'$ with $b' \in  B, f' \in  A^Y$ is any
other element of $P$, then 
$$
pp' = (bf)(b'f') = bb' f^{\alpha(b')}f'
$$

We call $P$ the (\textit{Cartesian, full, or unrestricted) wreath
  product of } $A $ and $B$ write 
$$
P = A Wr B
$$
(P.\pageoriginale Hall uses the notation $A \bar{\wr}B$, see $P$. Hall
($1954^b$).) 

Instead of taking the Cartesian power $A^Y$, we could start with the
corresponding direct power of $A$; we then arrive at a group $P^*$ the
\textit{direct (or restricted) wreath product} of $A$ and $B$, and we
write 
$$
P ^*= A wr B.
$$
($P$. Hall uses the notation $A \bar{\wr} B$. If $Y$ is a finite set,
the two wreath products are equal: 
$$
A Wr B = A wr B.
$$

Next we shall consider the case when both $A$ and $B$ are abstract
groups. We represent $B$ as a permutation group of $Y = B$ by right
multiplications and form the wreath product $P$ of $A$ and the
permutation group of $Y$ which represents $B$. We call $P$ the wreath
product of the abstract groups $A$ and $B$. We shall identify	every
element $b$ of $B$ with the corresponding right multiplication
$\rho(b)$ and write $b$ for $\rho(b)$; that is, 
$$
y^{\rho(b)} = y^b,   \text{   for all   } y \in  B.
$$
As before $\alpha$ is the homomorphism of $B$ into the group of
automorphism of $A^B$ defined by 
$$
f^{\alpha(b)}(y) = f(y^{b^{-1}}) = f(yb^{-1}), \text{ for all }y
\in  B. 
$$

This is a slight simplification of the notation, and we further
simplify it by writing $b$ for $\alpha(b)$. Thus we write 
$$
f^b(y) = f(yb^{-1}), \text{  for all  }y \in  B, f \in  A^B.
$$
(This accords with our usual notation, by which $b^{-1}f b = f^b$).

Every element $p$ of $P$ can be written uniquely as $p = bf$ with $b
\in  B,  f \in  A^B$; and  
$$
(bf) (b' f') = bb' f^{b'} f', \text{ for all } b,b' \in  B,
f,f' \in  A^B. 
$$

Thus\pageoriginale by this convention of identifying the abstract group $B$ with the
group of all right multiplications of $B$, we form the wreath product
of any two abstract groups. 

Now suppose both $A$ and $B$ are permutation groups, say of sets $X$
and $Y$ respectively. In this case we can give a particularly simple
permutation representation on the product set $X~ Y$ for the wreath
product of $A$ and $B$. To this end, we reverse the order of the
factors in the splitting extension $P$ of $A^Y$ by $B$, that is, we
now write the element of $P$ in the form 
$$
p = fb, f \in  A^Y,  b \in  B.
$$
Then multiplication of such products takes the form
\begin{align*}
  (fb) (f'b') & = f bf' - b^{-1} bb' = ff'^{b-1} bb' \\
  & = f^* b^*~~ \text{ say },
\end{align*}
where $f^* = ff'^{b^{-1}} \in  A^Y$ and $b^* = bb' \in 
B$. For every $fb$ of $P$, we define a mapping $(f, b)$ of the set $X
\times Y$ into itself as follows: 
$$
(x, y)^{(f, b)} = (x^{f(y)}, y), \text{ for all } (x,y) \in  X \times Y.
$$

We shall now show that the mapping $\varphi$ of $P$ into the set of
all mapping of $X \times Y$ into itself, defined by  
$$
(fb)^{\varphi} = (f, b)
$$
is\pageoriginale a monomorphism. Let $fb, f' b' \in  P$, with $f, f'
\in  A^Y, b, b' \in B$.

Then
$$
(fb) (f' b') = ff^{b-1} bb' = f^* b^*.
$$

Now,
\begin{align*}
  (x,y)^{(f,b)(f', b')}& = \left(x^{f(y)}, y^b\right)^{(f',  b')}\\
  & =  \left(\left(x^{f(y)}\right)^{f' (y^b)},  (y^b)^{b'}\right) \\
  & = \left(x^{f(y)^{f'^{b-1}}}(y), y^{bb'}\right) \\
  & = \left(x^{ff'^{b-1}}(y), y^{bb'}\right) = \left(x^{f^*}(y), y^{b^*}\right);
\end{align*}
and as this is true for all $(x,y) \in  X \times Y$ it follows
that  
$$
\displaylines{\hfill 
  (b,b)(f',  b') = (f^*,  b^*),\hfill \cr
  \text{that is,}\hfill ((fb) (f' b')) = (fb) (f' b')\hspace{.6cm} \hfill }
$$
This proves that $\varphi$ is a homomorphism.

It follows that every $(f, b)$ is a permutation of $X \times Y$. We
claim that $\varphi$ is a monomorphism of $P$ into the symmetric group
of permutations of $X \times Y$. For if $(f, b) = (f', b')$, then  
$$
(x,y)^{(f, b)} = (x^{f(y)}, y^b) = (x^{f' (y)}, y^{b'}) = (x, y)^{(f', b')}
$$
for\pageoriginale all $(x, y) \in X \times Y$. Hence
$$
x^{f(y)} = x^{f'(y)} ~~\text{for all }~~ x \in  X
$$
Therefore $f(y) = f'(y)$.

Again this holds for all $y \in  Y$; thus $f = f'$. Similarly,
$y^b = y^{b'}$ for all $y \in  Y$; hence $b = b'$. This show that
$\varphi$ is a monomorphism. Thus we have represented $P$ as a group
of permutations of $X \times Y$. 

In the following, we shall identify the wreath product of the
permutation groups $A$ and $B$ (of the sets $X$ and $Y$ respectively),
with its representation as a permutation group of $X \times Y$. 

The above permutation representation of the wreath product of two
permutation groups makes the wreath product associative. In other\break
words, if $A,  B$ and $C$ are permutation groups of sets $X, Y$, and
$Z$ respectively, then 
$$
(A Wr B) W r C \cong A Wr (B \,Wr\, C).
$$
In fact, if we make the natural identification of $((x, y),z)
\in  (X \times Y) \times Z$ and  
$$
(x,(y,z)) \in  X \times (Y \times Z)
$$
with the triplet $(x, y, z ) \in  X \times Y \times Z$ then $(A
Wr B) \,Wr\, C$ and $A Wr (B Wr C)$ become the same permutation group of
$X \times Y \times Z$. This will consist of the mapping $(F, g,c)$
where $F \in  A^{Y \times Z}, g \in  B^z,  c \in 
c$ and\pageoriginale  
$$
(x, y, z)^{(F, g, c)} = \left(x^{F(y, z)}, y^{g(z)}, z^c\right).
$$

Write $P = A Wr B, Q = B Wr C$. Then
$$
\displaylines{\hfill 
  P = \bigg\{ (f,b) \bigg | f \in  A^Y,  b \in  B
  \bigg\},\hfill \cr
  \text{ and } \hfill (A Wr B) Wr C = P Wr C = \bigg\{ (\varphi,  c)\bigg|
  \varphi \in  P^z,  c \in  C \bigg\}\hfill } 
$$

Now, if $\varphi \in  P^z, \varphi (z)$ is of the form
$$
\varphi(z) = (f_z, b_z),  f_z \in  A^Y,  b_z \in  B.
$$

Write 
$$
f_z(y) = F(y,z), b_z = g(z).
$$
We have 
\begin{align*}
   ((x,y),z)^{(\varphi, c)} & = \left((x,y)^{\varphi(z)}, z^c\right)\\
  & = \left((x, y)^{(f_z,  b_z)},  z^c\right) =  \left((x^{f_z (y)},
  y^{b_z}), z^c\right) \\ 
  & = \left((x^{F(y,z)},  y^{g(z)}), z^c\right) \\
  & = \left(x^{F(y,z)}, y^{g(z)}, z^c\right) \quad \text{ (by our
    identification)}\\ 
  & = (x, y, z)^{(F, g, c)} \quad \text{  (say)  }.
\end{align*}

Conversely, by retracing the above steps, one can easily see that any
triplet of the form $(F,  g, c)$ with $F \in  A^{Y \times Z}, 
g \in  B^Z$ and\pageoriginale $c \in  C$ is ( by our identification)
an element of $(AWr B)Wr C$. Thus the group $(A Wr B)Wr C$ consists
of all permutations of $X \times Y \times Z$ of the form $(F,  g, c)$
with $F \in  A^{Y \times Z},  g \in  B^Z,  c \in 
C$, and  
$$
(x,  y, z)^{(F, g, c)} = \left(x^{F(y,z)}, y^{g(z)}, z^c\right)
$$
for all $(x, y, z) \in  X \times Y \times Z$.

Similarly, we have	
$$
\displaylines{\hfill 
  Q = \bigg\{ (g,c) \bigg | f \in  B^z,  c \in  C \bigg\},
  \hfill \cr
  \text{and} \hfill A Wr (B Wr C) = A Wr Q = \bigg\{ (F,q)\bigg | F \in 
  A^{Y \times Z}, q \in Q  \bigg\}\hfill }
$$

Let $q = (g,c) \in  Q$.Then	
\begin{align*}
   (x,(y,  z))^{(F, q)} & = (x^{F(y,z)}, (y,  z)^q)\\
  & = \left(x^{F(y,z)},  (y ^{g(z)}, z^c)\right) \\
  & = \left(x^{F(y,z)},  y ^{g(z)}, z^c\right), \quad \text{again by
    our identification} \\ 
  & = (x, y, z)^{(F,g,c)}.
\end{align*}

Conversely, we can prove that any $(F, g,c)$ is an element of $A Wr (B$
 $Wr C)$. Thus we have proved that 
$$
(A Wr B) Wr C = A W (B Wr C).
$$
Let us now compute the cardinality of the group $(A Wr B) Wr C$. It is
easy to see that  
\begin{align*}
  | A Wr B| &= |B| |A|^{|Y|} \\
  \text{and}\hspace{1.6cm}   
  |(A Wr B) Wr C| & = |A Wr B|^{|Z|} |C|\\
  & = (|B| |A|^{|Y|})^{|Z|} |C| = |A| ^{|Y| |Z|} |B|^{|Z|} |C| \\
  & = |A Wr (B Wr C)| ~~ \text{because of associativity}.
\end{align*}

In\pageoriginale general the wreath product of two abstract groups as we have
defined it is not associative. Let $A$ and $B$ be two abstract
groups. Then by definition $A Wr B$ is a group with the set $B \times
A^B$ as carrier and therefore  
$$
|A Wr B| = |A|^{|B|} |B|. 
$$
Let now $A, B, C$ be three abstract groups of orders say $2,  3, 5$
respectively 
$$
|A| = 2, |B| = 3,  |C| = 5.
$$
Then we have
\begin{align*}
  |A Wr B| &= |A|^{|B|} |B| = 2^3 3,  ~~\text{ and  }\\
  |(A Wr B) Wr C| & = |A Wr B|^{|C|} |C| = (2^3 3)^5 5 = 2^{15} 3^5 5
\end{align*}
on the other hand $|B Wr C| = |B ^{|C|}|C| = 3^5 5 $ and 
$$
|A Wr (B Wr C)| = |A|^{|B Wr C|}  |B Wr C| = 2^{3^5.5} 3^5 5.
$$

Hence\pageoriginale
$$
A Wr (B Wr C) \neq (A Wr B) Wr C.
$$

Thus in general the wreath product of abstract groups is not
associative and the wreath products of two groups $A$ and $B$ depends
upon the permutation representation we choose for $B$. 

\section{}\label{chap6:sec7}%Sec 7

We shall later have occasion to use the wreath product of group while
certain embedding theorems. As a first illustration of wreath products
and their usefulness, we ally them to find the sylow subgroups of
finite symmetric groups. 

Let $A$ and $B$ be cyclic groups of order $3$, say
$$
A = gp(a_o : a^3_0 = 1), B = gp(b_0 : b^3_0 = 1).
$$
The groups $A$ and $B$ can be regarded as permutation groups on the
set $X = \{1, 2, 3 \} = Y$ by identifying $a_0$ and $b_0$ with the
cycle $(123)$; thus 
$$
1^{a_0} = 2, 2^{a_0} = 3, 3^{a_0} = 1,
$$
and similarly for $b_0$. Write $P = A Wr B$. The group $P$ has
permutation representation on the set $X \times Y$, since the groups
$A$ and $B$ are permutation groups on the set $X = Y \{1, 2, 3 \}$. 

Now,
$$
X \times Y = \left\{(1,1), (1,2), (1,3), (2,1)(2,2), (2,3), (3,1), (3,2),
(3,3)  \right\} 
$$
For\pageoriginale convenience, we rename these pairs $1,2,3,4,5,6,7,8,9$ in the same
order; i.e., 
\begin{alignat*}{4}
  (1, i) &= i, & \qquad \qquad &~~(i=1, 2, 3) \\
  (2, j) &= 3+j, && ~~(j=1, 2, 3)\\
  (3, k) &= 6+k, & &~~(k=1, 2, 3)
\end{alignat*}

The group $A^Y = A \times A \times A$ consists of all functions on the
set $\{ 1, 2, 3 \}$ with values in $A$. In our usual notation, 
$$
P = \{(f, b) | f \in  A \times A \times A, b \in  B \},
$$
where $(f, b)$ is the permutation of $X \times Y$ such that
$$
(x, y)^{(f, b)} = (x^{f(y)}, y^b),  x  \in  X, y \in  Y.
$$

Define $f_i \in  A^Y,  i = 1, 2, 3$ by 
$$
f_i(j) = 1 \text{ for } i \neq j, f_i(i) = a_0 (j = 1,2,3).
$$

Then it is easy to verify that 
$$
A^Y = gp (f_1,  f_2, f_3).
$$

Since $(f, b)= (f, 1) (1, b)$ for all $f \in  A^Y,  b
\in  B, $we have 
$$
P = gp ((f_1,  1), (f_2, 1), (f_3,  1), (1, b_0)).
$$

Now we can easily write down the permutations $(f_1,  1), (f_2,  1)
(f_3, 1)$ and $(1, b_0)$.\pageoriginale We have 
\begin{align*}
  (1, 1)^{(f_1, 1)} = \left(1^{f_1(1)}, 1^1\right) = (2,1) \\
  (2, 1)^{(f_1, 1)} = \left(2^{f_1(1)}, 1^1\right) = (3,1) \\
  (3, 1)^{(f_1, 1)} = \left(3^{f_1(1)}, 1^1\right) = (1,1) 
\end{align*}
$$
\displaylines{\text{and}\hfill
(i,j)^{(f_1, 1)} = (i^{f_1(j)}, j^1) = (i,j), \text{ for } i=1,2,3, j
  = 2,3.\hfill }
$$

Thus, in the alternative notation
$$
(f_1,  1) = (147).
$$
Similarly, $(f_2, 1) = (258), (f_3, 1) = (369)$. Further
\begin{alignat*}{4}
  (1, 1)^{(1,b_0)} & = &(1^{1(1)}, 1^{b_0}) = ~&(1,2) \\
  (1,2)^{(1,b_0)}  & = && (1,3) \\
  (1,3)^{(1,b_0)}  & = & & (1,1),
\end{alignat*}
and so on.  Therefore, 
$$
(1, b_0) = (123) (256) (789)
$$
But\pageoriginale
\begin{align*}
  & (1, b_0)^{-1} (f_1,  1) (1, b_0)\\
  & = (321) (654) (987) (123) (456) (789) \\
  & = (258) = (f_2,  1)
\end{align*}
Similarly, $(1, b_0)^{-1} (f_2,  1) (1, b_0) = (f_3, 1)$.

Hence the group $P$ is generated by the two permutations $(f_1, 1)$
and $(1, b_0)$; that is, by $(147)$ and $(123) (456) (789)$. We also
note that $P$ is here represented as a group of permutations of degree
$9$, that is, as a subgroup of the symmetric group $S_9$. The order of
the group $P$ is  
$$
|P| = |A|^{|Y|} |B| = 3^3 3 = 3^4.
$$

It is easy to see that $3^4 T 9!$; that is $3^4$ is the highest power
of $3$ dividing the order $9 ! of S_9$. Thus $P$ is a "sylow subgroup"
of $S_9$. 

Let $G$ be a finite group, and $p$ a prime. If $p^k T G$, the $G$ has
subgroups of order $p^k$. Such subgroups are called \textit{sylow
  subgroups.} There are a number of important theorems (known as sylow
theorems) about these subgroups. See e.g Kurosh $(1956,  \S 54)$ and
Zassenhaus (1958, Ch. IV, p. 135). 

The example considered above is a particular case of the following theorem.

\begin{theorem}[Kaloujnine, 1948]\label{chap6:sec7:thm5}%Thm 5
  The\pageoriginale sylow $p$-subgroup of $S_{p^n}$ is the wreath product
  $$
  P_n = C_p Wr C_p Wr \cdots Wr C_p (n \text{ times })
  $$
  were $C_p = gp (c_0 : c^p_0 = 1)$ is the cyclic group of order
  $p$. The group $C_p$ can be regarded as the subgroup generated by
  the cycle $(12 \cdots p)$ in $S_p$. 
\end{theorem}

Let $X = \{ 1, 2,  \ldots,  p\} = X_1 = \cdots = X_n $, and
$$
Z = \bigg\{(x_1, x_2, \ldots,  x_n) \bigg| x_i \in X_i, i = 1,
\ldots, n  \bigg\}; 
$$
that is to say, 
$$
Z = X_1 \times X_2 \cdots \times X_n = X^n.
$$

We rename the elements $(x_1, \ldots, x_n)$ of $Z$, and write $1 +
\sum\limits^n_{i = 1} (x_i - 1) p^{i-1}$ for $(x_1, \ldots, x_n)$. We
note that $|Z| = p^n$. In the new notation, we have $P_n \le
S_{p^n}$. 

Since the wreath product is associative (note that we are using
permutation groups), we get 
$$
P_n = P_{n-1} W r C_p.
$$
Therefore $|P_n| = | P_{n-1} | ^p | C_p | = | P_{n-1} | ^p p$

\quad \qquad \qquad $= p^{k(n)}$ (say).

Here $k(n)$ is defined by the recurrence relation
$$
k(1) = 1, k(n) = p k (n-1) +1.
$$
We\pageoriginale shall prove by induction that 
$$
p^{k(n)} \top p^n;
$$
For $n=1$, this is obvious.  Assume that 
$$
p^{k(n-1)} \top p^{n-1};
$$
Now,
$$
p^n = \left( \prod^{p^{n-1}}_{r=1} r \right) \left(
\prod^{p^{n-1}}_{r=1}(p^{n-1}+r) \right) \left( \prod^{p^{n-1}}_{r=1}(2
p^{n-1} + r) \right) \cdots \left( \prod^{p^{n-1}}_{r=1}((p-1) p+r)
\right) 
$$
We have $p^s \top (mp^{n-1} + r)$ if and only if $p^s \top r$, $m <
p-1, 1 \le r \le p^{n-1}$. Therefore,  
$$
p^{k(n-1)} \top  \prod^{p^{n-1}}_{r=1} (m p^{n-1} + r) ~\text{for}~ m < p - 1.
$$

But $p^{k(n-1)+1} \top  \prod^{p^{n-1}}_{r=1} ((p-1) p^{n-1} + r$),
since the last term of this product is $p^n$. Hence $(p^{k(n-1)})^p p
= p^{k(n)} \top p^n$; for all $n$, Thus $P_n$ is a sylow subgroup of
$S_{p^n}$. It is not difficult to use this result to compute the sylow
subgroups of any symmetric group $S_m$.  
