
\chapter{Homomorphisms of Groups}\label{chap3} % chapter 3.

\section{}\label{chap3:sec1}%sec 1.

We\pageoriginale shall, in the this chapter introduce the concepts of
homomorphism, 
isomorphism and other important mappings of a group into another
group. 

Let $G$ and $H$ be any two groups. A mapping $\varphi$ if $G$ into $H$
is a \textit{homomorphism} if it preserves the group operations. On
the face of it $\varphi$ has to satisfy 
\begin{enumerate}[(i)]
\item $(\in \{ ~ \})^{\varphi} = \in \{ ~ \}$ 
\item $(l (g))^{\varphi} = l (g)^{\varphi}$, for every $g \in  G$
\item $(\pi (g, g'))^{\varphi} = \pi (g^\varphi,  g{'}^\varphi)$,
  for all $g, g' \in  G$. 
\end{enumerate}

To make the notation less clumsy, we have used the same symbols $\in, 
l,  \pi$ for the operators of the groups $G$ and $H$. These three
conditions written in the multiplicative notation read as follows.  
\begin{enumerate}[(a)]
\item $1^{\varphi} = 1$ (we use the same symbol `1' for the neutral
  elements of both $G$ and $H$) 
\item $(g^{-1})^{\varphi} = (g^\varphi)^{-1}$
\item $(gg')^\varphi = g^\varphi g{'}^\varphi$
\end{enumerate}

The definition of homomorphism given here is capable of
generalisation, and thus we can speak of a homomorphism of an
algebraic system into another. But, here we shall confine our
attention to groups. In the case of groups conditions $(a)$ and $(b)$
are contained in $(c)$. Thus we have 

\setcounter{theorem}{0}
\begin{theorem}\label{chap3:sec1:thm1}%them 1.
  A\pageoriginale mapping $\varphi$ of a group $G$ into a group $H$ is a
  homomorphism if and only if it satisfies $(c)$. 
\end{theorem}

\begin{proof}
  If $\varphi$ is a homomorphism of $G$ into $H$, then trivially
  $\varphi$ satisfies ($c$). 
\end{proof}

Now, let $\varphi$ be a mapping of $G$ into $H$ satisfying $(c)$. We
first observe that in a group the natural element is the only
idempotent element. (An element $x$ is \textit{idempotent} if it
satisfies the equation $x^2 = x$.) For if $x$ is any idempotent
element of a group, then 
$$
\displaylines{\hfill xx = x = x1;\hfill \cr
\text{therefore,} \hfill x = 1.\hfill }
$$

Now,
$$
1^\varphi 1^\varphi = (11)^\varphi = 1^\varphi. 
$$
Therefore $1^\varphi$ is idempotent and hence the neutral element of
$H$. Similarly 
$$
\displaylines{\hfill 
  g^\varphi (g^{-1})^\varphi = (gg^{-1})^\varphi = 1^\varphi;\hfill \cr
  \text{hence}\hfill (g^{-1})^\varphi = (g^\varphi)^{-1}.\hfill }
$$

Thus $\varphi$ is a homomorphism of $G$ into $H$.

Let\pageoriginale $X, Y$ be any two sets and $\theta$ a mapping of $X$ into $Y$;
further let $E \subseteq X$, $F \subseteq Y$. We define 
\begin{align*}
  E^\theta & = \Big\{e^\theta \Big| e \in  E \Big\}\\
  F^{\theta^{-1}} &= \Big\{e \big| e \in  X, e^\theta \in  F \Big\}
\end{align*}

The following two propositions are easy to verify.

Let $\varphi$ be a homomorphism of a group $G$ into another group $H$.
\begin{enumerate}[(A)]
\item If $S \leq G$, then $S^\varphi \leq H$
\item If $T \leq H$, then $T^{\varphi^{-1}} \leq G$.
\end{enumerate}

In particular,
$$
\big\{ 1 \big\}^{\varphi^{-1}} = N \leq G.
$$

The subgroup $N \leq G$ is uniquely de terminal by $\varphi$ and is
called the \textit{kernel} of the homomorphism. 

A homomorphism $\varphi$ or $G$ into $H$ is an \textit{epimorphism} if
it maps $G$ \textit{onto} $H$; in other words, if  
$$
G^\varphi = H.
$$

A homomorphism $\varphi$ of $G$ onto $H$ is a \textit{monomorphism} if
it is one-to-one (briefly $1-1$), i. e. $x^\varphi = y^\varphi$
implies $x = y$, for all $x, y \in  G$. A homomorphism which is
both an epimorphism and monomorphism is an \textit{isomorphism}. 
\begin{enumerate}
\item[(C)] If $\varphi$ is an isomorphism of $G$ onto$H$, then the inverse
  mapping\pageoriginale $\varphi^{-1}$ of $\varphi$ exists and is an isomorphism of
  $H$ onto $G$.  
\end{enumerate}

\begin{proof}
  The equation
  $$
  g^\varphi = h, \text{ with } g \in  G, h \in  H,
  $$
  has one and only one solution in $G$. We define
  $$
  h^{\varphi^{-1}} = g, \text{ if } g^\varphi = h.
  $$
  The mapping $\varphi^{-1}$ is `onto', because for any $g \in 
  G$, we have  
  $$
  (g^\varphi)^{\varphi^{-1}} = g.
  $$

  Also, if
  \begin{align*}
    h^{\varphi^{-1}} & = h'^{\varphi^{-1}}, \text{ with } h, h'
    \in  H, \text{ and }\\ 
    g^\varphi & = h, ~ g'^\varphi = h', \text{ then } \\
    g & = g', ~\text{ and therefore } \\
    h & = g^\varphi = g'^\varphi = h'.
  \end{align*}
  Hence $\varphi^{-1}$ is one-to-one.
\end{proof}

Now, let $h, h' \in  H$, with $h = g^\varphi$, $h' =
g{'}^{\varphi}$, then $(hh')^{\varphi^{-1}} = (g^\varphi
g{'}^\varphi)^{\varphi^{-1}}$ $= gg' = h^{\varphi^{-1}}
h{'}^{\varphi^{-1}}$. Hence $\varphi^{-1}$ is an isomorphism of $H$
onto $G$. It is easy to see that $\varphi^{-1}$ is the two-sided
inverse of $\varphi$, in other words; the composite mappings $\varphi
~ \varphi^{-1}$ and $\varphi^{-1} ~ \varphi$ are the identity mappings
of $G$ and $H$ respectively. 

We\pageoriginale say that two groups $G$ and $H$ are \textit{isomorphic} if there is
an isomorphism $\varphi$ of $G$ onto $H$. We then write 
$$
G \cong H. 
$$

Let $G, H$ and $K$ be any three groups and $\varphi$ and $\psi$ be
homomorphism of $G$ into $H$ and $H$ into $K$ respectively. Then we
have  
\begin{enumerate}
\item[(D)] The composite mapping $\varphi \psi$ of $G$ into $K$ is a
  homomorphism.  
\end{enumerate}

For let $g ~ g' \in  G$; then
$$
(gg')^{\varphi \psi} = ((gg')^\varphi)^\psi = (g^\varphi g{'}^\varphi
)^\psi = (g^\varphi)^\psi = (g{'}^\varphi )^\psi = g^{\varphi \psi} =
g^{\varphi \psi} 
$$

In $(D)$, if $\varphi$ and $\psi$ are isomorphisms then so is $\varphi
\psi$. This is easy to verify. 

It follows from the above considerations that isomorphism is an
equivalence relation on the class of all groups. Thus, we have  
\begin{enumerate}
\item[(R)] $G \cong G$;
\item[(S)] $G \cong H$ implies $H \cong G$;
\item[(T)] $G \cong G ~ \& ~ H \cong K$ implies $G \cong K$.
\end{enumerate}

Let $G$ be a group. A homomorphism of $G$ into itself is an
\textit{endomorphism}. An isomorphism of $G$ onto itself is an
\textit{automorphism}.  

The product of two homomorphisms, or more generally, the product of two
mapping is defined only under certain restrictions, viz. that the
range of the first mapping is contained in the domain of the\pageoriginale
second. This is, however, always the case for mapping of a set into
itself. 

By mere computation one can verify the associativity of the
multiplication of mappings whenever the multiplication is defined. 

Thus the set of all endomorphisms of a group $G$ is closed under an
associative binary operation and therefore forms an algebraic system
called \textit{semi-group}. 

Now consider the set of all automorphisms of a group $G$. Trivially
the identity mapping $L$ belongs to this set and under the
multiplication of automorphisms it acts as an unit element. By what we
have already proved automorphism possesses a right inverse (in fact it
is the two sided inverse) and the multiplication is associative. 

Thus we have,
\begin{theorem}\label{chap3:sec1:thm2}%them 2.
  The set of all automorphisms of a group $G$ forms a group. 
\end{theorem}

Let $\varphi$ be an endomorphism of $G$ possessing a left inverse
$\theta$ and a right inverse $\psi$. Then $\varphi$ is an automorphism
and $\theta = \psi$. For if, 
$$
\displaylines{\hfill 
  x^\varphi_1 = x^\varphi_2, \text{ with } x_1, x_2 \in  G\hfill \cr
  \text{then,} \hfill  (x_1^\varphi)^\theta = (x_\alpha^\varphi)^\theta\hfill\cr
  \text{i.e.,}\hfill 
  x_1 = x_1^{\varphi \theta} = x_2^{\varphi \theta} = x_2.\hfill }
$$

Therefore\pageoriginale $\varphi$ is 1-1. Further for any $x \in  G$, we
have  
$$
(x^\psi)^\varphi = x^{\psi \varphi} = x.
$$

Therefore, $\varphi$ is `onto' and hence an automorphism. This, in
turn, proves that $\theta = \psi$. 

Thus we have proved that the automorphisms of $G$ are precisely the
endomorphisms having a left inverse and right inverse. But an
endomorphism which is not an epimorphism may possess a left inverse
which is not a right inverse. Similarly an endomorphism which is not a
monomorphism can have a right inverse which not a left inverse. 

Let $X$ be any set. A mapping $\pi$ of $X$ into itself is a
\textit{permutation} if it is 1-1 and `onto'. Thus every
automorphism of a group $G$ is a permutation of $G$. 

With the usual techniques, we can verify the following:
\begin{theorem}\label{chap3:sec1:thm3}%them 3.
  The set of all permutations on $X$ form a group with the composite
  of permutations as the binary operation. 
\end{theorem}

\section{Equivalence relations and congruences}\label{chap3:sec2} % sec 2.

Let $G$ and $H$ be any two sets, not necessarily groups, $\varphi$, a
mapping of $G$ into $H$. We introduce an equivalence relation `$\sim$'
as follows: 
$$
g \sim g' \text{ if and only if } g^\varphi = g{'}^\varphi.
$$

It is immediate that `$\sim$' satisfies the following conditions:
\begin{enumerate}
\item[(R)]  $g \sim g$;\pageoriginale
\item[(S)]  $g \sim g'$ implies $g' \sim g$, for all $g, g' \in  G$;
\item[(T)]  $g \sim g'$, $g' \sim g''$ implies $g \sim g''$, for all $g,
  g', g'' \in  G$. 
\end{enumerate}

Hence `$ \sim $' is an equivalence relation.

Now let $G$ and $H$ be groups and $\varphi$ a homomorphism of $G$ into
$H$. Then '$\sim$' in addition to $R, S, T$ also satisfies the
following condition: 
$$
\displaylines{\hfill 
  g \sim g_1, g' \sim g'_1 \text{ implies } gg' \sim g_1 g'_1.\hfill}
$$
For
$$  
g^\varphi = g^\varphi_1, g'^\varphi = g'^\varphi_1. 
$$

Therefore,
$$
(gg')^\varphi = g^\varphi ~ g'^\varphi = g^\varphi_1 ~ g^\varphi_1 =
(g_1 ~ g_1')^\varphi. 
$$

Further 
$$
\displaylines{\hfill 
  g \sim g_1 ~\text{ implies }~ g^{-1} \sim g^{-1}_1 \hfill \cr
  \text{For}\hfill 
  g^\varphi = g^\varphi_1 \text{ implies } (g^{-1}) = (g^{-1}_1)^\varphi
  \hfill} 
$$

Such an equivalence relation is called a \textit{congruence}. 

\begin{defi*}
  Let $G$ be a group and `$\sim$' an equivalence relation satisfying
  the condition. 
  $$
  g \sim g',  g_1 \sim g'_1 \text{ implies } gg_1 \sim g' g'_1
  $$

  Then\pageoriginale we call `$\sim$' a congruence. Strictly speaking we should
  also demand that  
  $$
  g \sim g' \text{ implies } g^{-1} \sim g'^{-1}. 
  $$

  But in the case of groups our definition implies this. For if, 
  \begin{align*}
    & g \sim g', \text{ then }\\
    & g' \sim g
  \end{align*}

  Now, 
  $$
  \displaylines{\hfill g^{-1} \sim g^{-1}\hfill \cr 
    \text{and therefore} \hfill 
    g' ~ g^{-1} \sim gg^{-1} = 1. \hfill }
  $$
  
  Again, 
  $$
  g'^{-1} \sim g'^{-1}.
  $$
  
  Therefore
  $$
  g^{-1} = g'^{-1} (g' g^{-1}) \sim g'^{-1} 1 = g'^{-1}.
  $$
\end{defi*}

Let $X$ be any set, $\varphi$ a mapping of $X$ into another set
$Y$. We have seen that $\varphi$ induces an equivalence relation
`$\sim$' in $X$. Every equivalence relation splits $X$ into disjoint
\textit{blocks}. Let `$\sim$' be any equivalence relation in
$X$. Define, for $g \in  X$ 
$$
[ g ] = \Big\{x \in X E \Big| x \in  g \Big\}. 
$$

Then clearly either
$$
\displaylines{\hfill 
  [ g ] \cap [ h ] = \phi, \hfill \cr
  \text{or} \hfill [ g ] = [ h ].\hfill }
$$

Conversely\pageoriginale every partition of $X$ into blocks gives rise to an
equivalence relation. To see this we have only to define ``$x \sim y$
if and only if $x, y$ belong to the same block''. 

Now let $G$ be a group and `$\sim$' a congruence relation in $G$. That
is to say, 
$$
g \sim g_1, h \sim h_1 \text{ implies } gh \sim g_1 ~ h_1, \text{ for
} g, h, g_1, h_1 \in  G. 
$$

In this case the block [$gh$] depends only on [$g$] and [$h$] and not
on the particular element $g$ and $h$. For if 
$$
\displaylines{\hfill 
  [ g ] = [ g' ] \hfill \cr
  \text{and}\hfill [ h ] = [ h' ] \hfill \cr
  \text{then } \hfill [ gh ] = [ g' h' ].\hfill }
$$

This follows easily from the definition of a congruence in a group. 

Now we shall prove that the product of [$g$] and [$h$] is again a
block. In other words, 
$$
[ g ] [ h ] = [ gh ] 
$$

Let, $p \in  [ gh ]$, then $p \sim gh$.
But 
$$
g^{-1} \sim g^{-1}.
$$

Therefore 
$$
g^{-1} p \sim g^{-1} (gh) = h
$$

Thus\pageoriginale
$$
p =g (g^{-1} p), \text{ with } g \in  [g], g^{-1} p \in 
[h]  
$$

Hence
$$
  p \in  [g] [h]
$$
thus
\begin{equation*}
  [gh] \subseteq [g] [h]. \tag{1}\label{chap3:sec2:eq1}
\end{equation*}

Conversely  if $ x =\in  [g] [h]$, then
$$
x = g' h', \text { with } g' \in  [g], h' \in  [h]
$$

Hence
$$
\displaylines{\hfill 
  g' h' \sim gh, ~\text{ and } \hfill \cr
  \text{therefore}\hfill 
  x = g' h' \in  [gh]. \hfill}
$$

This gives
\begin{equation}
  [g] [h] \subseteq gh. \tag{2}\label{chap3:sec2:eq2}
\end{equation}

Combining this with (\ref{chap3:sec2:eq1}) we have
$$
[g] [h] = [gh]
$$

This multiplication of blocks turns the set of blocks into a group. We
have 
$$
[1] [g] = [1g] =[g]
$$

Similarly\pageoriginale
\begin{align*}
  [g] [1] & = [g1]  = [g],\\
  [g][g^{-1}] & = [gg^{-1}]  = [1],\\
  [g^{-1}] [g] & = [g^{-1} g] =[1]
\end{align*}

The above equations prove the following theorem.
\begin{theorem}\label{chap3:sec2:thm4} % them 4
  The books associated with a congruence in a group $G$ from a group
\end{theorem}

We denote this group by $G/\sim$. The block $[l]$ is the neutral element
of this group and [$g^{-1}$] is the inverse of $[g]$. We call $G/
\sim$ the \textit{quotient group} (also the \textit{ factor group})
with respect of the congruence `$\sim$' 

The notion of congruence, as well as the notion of the quotient
algebra with respect to a congruence, can be defined much more
generally than for groups, namely for arbitrary algebraic systems.   

In the case of groups, the book $[l]$ plays an important part. In fact
we shall see that it completely determines the congruence associated
with it. 

We one define an epimorphism $\theta$ of $G$ onto $G \sim$. Write
$$
g^\theta = [g]
$$

The equation
$$
[gh] = [g] [h]
$$
demonstrates\pageoriginale that $\theta$ is a homomorphism. Obviously $\theta$ is onto
$G / \sim$ and therefore $\theta$ is an epimorphism. 

Consider now
$$
\{[1]\}^{\theta{^{-1}}} = \bigg\{ x \in  G\bigg| x^\theta =
  [1]\bigg\} = \bigg\{ x \in  G \bigg| [x] = [1]\bigg\}. 
$$

We see from this that
$$
\{ [1]\}^\theta{^{-1}} = [1] ~\text{(considered as a set)}.
$$

Thus $[l]$ is the kernel of $\theta$ and we denote it by $N$.

\begin{defi*}
  Let $S \le G$; then the set $S_g$ is called a {\em right coset} of
  $S$. Similarly left cost coset is defined 
\end{defi*}

We shall now prove that every block, with respect to a certain
congruence is a right coset of the kernel of the epimorphism induced
by the congruence under consideration. 

Let `$\sim$' be a congruence in $G$ and $\theta$ the corresponding
epimorphism of $G$ onto $G/ \sim$, and again 
$$
N = [1] = [1]^{\theta^{-1}}.
$$
Then 
$$
[g] = Ng; 
$$

\noindent \text{for let} \hspace{1cm}  $x \in  ~Ng$;  then 

$$
x = ng ~~\text { with }~~ n \in  N.
$$

Now,\pageoriginale
$$
\displaylines{\hfill 
  n \sim 1, g \sim g,  ~\text{ give } \hfill \cr
  \hfill ng \sim 1g = g \hfill \cr
  \text{i.e.,} \hfill x = ng \in  [g]\hfill }
$$

Therefore 
$$
Ng \subseteq [g]
$$

Conversely if $x \in  [g]$, then
\begin{gather*}
  x \sim g, g^{-1} \sim g^{-1} \text{ imply }\\
  xg^{-1} \sim gg^{-1} = 1
\end{gather*}
Therefore \qquad $x = (xg^{-1})g \in  Ng$.

Thus
\begin{align*}
  [g] & \subseteq Ng, \text { and it follows that  } \\
  [g] &= Ng, \text{ as claimed}
\end{align*}

Similarly
$$
[g] = gN
$$

Hence 
$$
Ng = [g] = gN.
$$

Thus\pageoriginale we have proved the following theorem.

\begin{theorem}\label{chap3:sec2:thm5} % them 5.
  Let `$\sim$' be a congruence in $G$. The mapping $\theta$ of $G$
  into $G/ \sim$ defined by 
  $$
  g^\theta = [g] \text{ with } g \in G
  $$
  is an epimorphism with kernel $N=[1]$. Further every element of $G /
  \sim$ is a right coset (left coset) of $N$. Also $N$ commutes with
  every elements of $G$.  
\end{theorem}

Let $\mathscr{R}$ be the set of all congruences in $G$. Every $\sim
\in  \mathscr{R}$ in a $1-1$ manner determines the associated
natural epimorphism. Let $\mathscr{M}$ denoted the set of all such
associated natural epimorphisms. Also every $\theta \in 
\mathscr{M}$ determines uniquely a kernel $N$. Let $\mathscr{N}$ be
the of all such kernels. By the above theorem every $N \in 
\mathscr{N}$ determines completely all the blocks and therefore
uniquely determines the associated congruences which in turn
determines the natural epimorphism. The consideration above prove the
following  theorem.  

\begin{theorem}\label{chap3:sec2:thm6} % them 6.
  There is a `natural' 1-1 correspondence between $\mathbb{R},
  \mathscr{M}$ and $\mathscr{N}$. 

  Because of the above theorem we shall write $G/N$ for $G / \sim$
  where $N$ is the kernel determined by $\sim$. 
\end{theorem}

\section{Factorisation of a homomorphism}\label{chap3:sec3} % sec 3.

We shall now show that every homomorphism of a group onto another can
be factorised  ``canonically''. 

\begin{theorem}\label{chap3:sec3:thm7} % them 7.
  Let $G$ and $H$ be any two groups, $\varphi$ a homomorphism of $G$
  into\pageoriginale $H$. Then $\varphi$ can be factorised in the from
  $\varphi = 
  \theta \psi$, where $\theta$ is the {\em canonical epimorphism} (or
  natural epimorphism) of $G$onto $G/ \sim (= G/N)$, $\sim$ being the
  congruence determined by $\varphi$, and where $\psi$ is a
  monomorphism of $G/ \sim$ into $H$.  
\end{theorem}

\begin{proof}
  Define $\psi$ on $G/ \sim$ with values in $H$ by 
  $$
  [g]^\psi = g ^\varphi
  $$

  Let 
  $$
  \displaylines{\hfill [g] = [g']\hspace{1.7cm}\hfill \cr
    \text{then} \hfill g \sim g' \hspace{2.3cm}\hfill \cr
    \text{and therefore}\hfill g^\varphi = g'^\varphi.\hspace{3.5cm} \hfill }
  $$
  
  This proves that $\psi$ is a defined mapping. Further, 
  \begin{align*}
    ([g] [h])^\psi &= ([gh])^\psi = (gh)^\varphi\\
    &=g^\varphi h^\varphi = [g]^\psi [h]^\psi,  \text { for all } g, h
    \in  G. 
  \end{align*}
  
  Also $\psi$ is 1-1. For if
  \begin{align*}
    [g]^\psi & = [h]^\psi,  \text{ with } g,h \in  G, \text { then }\\
    g^\varphi & = h^\varphi ; \text { that is }\\
    g & \sim h \text{ and }
  \end{align*}
  therefore\pageoriginale
  $$
  [g] = [h]. 
  $$

  Thus $\psi$ is a monomorphism of $G/\sim$ into $H$. Now,
  $$
  g^{\theta \psi} = [g]^\psi = g^\varphi, \text{ for all } g \in 
  G. 
  $$

  Therefore
  $$
  \varphi = \theta \psi
  $$
  
  Hence the theorem.
\end{proof}

\section{Normal subgroups}\label{chap3:sec4}%sec 4

We now proceed to characterise the kernels of the homomorphisms of a
group $G$. We have already seen that the kernel determined by a
congruence in $G$ commutes with all the elements of $G$. We shall
prove that the kernel of any homomorphism of $G$ has this
property. The following establishes this.    

\begin{theorem}\label{chap3:sec4:thm8} % them 8
  Let $G$ and $H$ be any two groups, $\varphi$ a homomorphism of $G$
  into $H$. Then the kernel of $\varphi$ is also kernel $N$ associated
  with the congruence `$\sim$' determined by $\varphi$. 
\end{theorem}

\begin{proof}
  \begin{align*}
    \{ 1 \}^{\varphi^{-1}} & = \bigg\{ x \big| x^\varphi = 1\bigg\}\\
    & = \bigg\{ x \big| x^\varphi = 1^\varphi \bigg\} = \bigg\{ x
    \big| x \sim 1\bigg\}\\ 
    & = [1] = \bigg\{ 1\bigg\}^{\theta {^{-1}}} = N.
  \end{align*}
  Thus\pageoriginale the kernel of any homomorphism of the group $G$ into $H$ commutes
  with all the elements of $G$. 
\end{proof}

We now make the following definition.

\begin{defi*}
  Let $N \le G$. Then is a {\em normal subgroup} (also self-conjugate
  or invariant) of $G$ (notation $N \triangle G$) if 
  $$
  Ng = gN \text { for all } g \in G.
  $$

  Thus the kernel of a homomorphism of $G$ into $H$ is a normal subgroup 
\end{defi*}

Let $N \triangle G$. Define $x \sim y$ if and only if $xy^{-1}
\in  N$. A straight forward verification shows that `$\sim$' is
a congruence relation in $G$. Further, 
$$
[1] = \bigg\{ x \big| x \sim 1\bigg\} =\bigg\{ x \big| x \in  N
\bigg\} = N 
$$

Hence we have
\begin{theorem}\label{chap3:sec4:thm9} % them 9.
  Every normal subgroup $N \triangle G$ determines a congruence `$\sim$'
  in $G$ with 
  $$
  [1] = N.
  $$
\end{theorem}

\begin{coro*}
  If $N \triangle G$, then $N$ is the kernel of some homomorphism of $G$. 
\end{coro*}

\begin{proof}
  We have only to consider the natural epimorphism $\theta$ of $G$
  onto $G/ \sim = G/N$. Theorem \ref{chap3:sec2:thm6} and Theorem
  \ref{chap3:sec4:thm9} together imply 
\end{proof}

\begin{theorem}\label{chap3:sec4:thm10} % them 10
  Let $\mathscr{C}$ be the set of all congruences in $G, \mathscr{N}$
  the set of all normal subgroup of $G$. Then there is a $1-1$ mapping
  $\alpha$ of\pageoriginale $\mathscr{C}$ {\em onto} $\mathscr{N}$, in
  a natural way.  
\end{theorem}

\begin{proof}
  Define $\alpha$ as
  $$
  \alpha (\sim) = \bigg\{ x \big| x \sim 1\bigg\} = [1] =N, \text{ for
    all } \sim \in  \mathscr{C} 
  $$
  $\alpha$ serves our purpose.
\end{proof}

\section{The graph of a binary relation}\label{chap3:sec5}%sec 5

Let $E$ be any set and $`*'$ a binary relation in $E$. With every such
relation there is associated a set $R \subseteq E \times E$, namely 
$$
R = \bigg\{ (x,y) \big| x*y, x \in  E, y \in  E\bigg\}. 
$$

The subset $R$ is called the \text{ graph } of the binary
relation. Conversely to every $R \subseteq E \times E$ there is a
binary whose graph is $R$; and this correspondence is $1-1$. We shall
usually identify the binary relation `*' with its graph and refer to
$R$ itself as the binary relation. In particular, with this
identification, an equivalence relation in $E$ will be subset of $E
\times E$. We shall now interpret the reflexive, symmetric and
transitive laws in terms of the subset of the product set $E \times
E$. We call the subset $\triangle \subseteq E \times E$, defined by     
$$
\triangle = \bigg\{ (x,x)\big | x \in E\bigg\}
$$
the \textit{ diagonal } of $E \times E$.

Let $R \subseteq E \times E$, $S \subseteq E \times E$ be two binary
relations in $E$. Then 
$$
R^{-1} = \bigg\{ (x,y) \bigg| (y,x) \in  R\bigg\}
$$
is\pageoriginale the inverse of the relation $R$. By the \textit{product} of the
relations $R$ and $S$ we mean the relation 
$$
RS = \bigg\{ (x,z) \big| \exists y,y \in  E, (x,y) \in 
R, (y,z) \in  S\bigg\}. 
$$

Let $R$ be a binary relations in $E$. Then $R$ is reflexive if and
only in $\triangle \subseteq R$; also $R$ is symmetric if and if
$R^{-1} \subseteq R$; finally $R$ is transitive if and only if $R^2 (=
RR) \subseteq R$.Thus $R$ is equivalence relation if and only if it
has all three properties: 
\begin{enumerate}
\item [(R)] $\triangle \subseteq R$,
\item [(S)] $R^{-1} \subseteq R$,
\item [(T)] $R^2 \subseteq R$.
\end{enumerate}

It is immediate from the above definitions that
$$
R \triangle = \triangle R =R, \text { for all } R \subseteq E \times E.
$$

Further the symmetric and transitive laws are in the equivalent to $R
-R^{-1}$ and $R^2 = R$, respectively. The following fact is formally
analogous to Theorem \ref{chap1:sec6:thm1} of Chapter 1; we omit the proof. 

\begin{theorem}\label{chap3:sec5:thm11} % them 11.
  $R \subseteq E \times E$ is an equivalence relation if and only if
  \begin{enumerate}
  \item [\quad (1)]  $R \neq \phi$
  \item [\quad (2)]  $RR^{-1} \subseteq R$.
  \end{enumerate}
\end{theorem}

\section{The graph of a congruence in a group}\label{chap3:sec6}%sec 6

Let $G$ be a group. Before considering the congruences in a group, we
shall introduce a group structure on the product set $G \times
G$\pageoriginale in a 
natural way. Define the unit element of $G \times G$ to be (1,1) with
$1 \in  G$, the inverse of $(g,h)$ to be $(g^{-1}, h^{-1})$ and
the product of $(g,h)$ and $(g',h')$ to be  
$$
(g,h) (g',h') = (gg',hh'), \text { with } g,g', h,h' \in  G.  
$$

It is easily seen that this turns $G \times G$ into a group. We call
this group \textit {the direct square} of $G$. In fact we can define
the direct product of any family of groups. We shall have occasion to
return to this topic later (See Chapter \ref{chap6}). 

Let $R \subseteq G \times G$ be a congruence in $G$. Since
$$
\displaylines{\hfill \triangle \subseteq R, \hfill \cr
  \text{it follows that} \hfill  
  (1,1) \in  R.\hspace{2cm}\hfill \cr
  \text{If}\hfill (g,h) \in  R \text{ and } (g', h')
  \in  R \text{then}\hfill \cr  
  \hfill g \sim h \text{ and } g' \sim h'. \hfill }
$$

Therefore 
\begin{gather*}
  gg' \sim hh' ; \text { that is}\\
  (gg', hh') \in  R.
\end{gather*}

Thus 
$$
(g,h) (g', h') = (gg', hh') \in  R.
$$

Further\pageoriginale if $(g,h) \in  R$, then
\begin{gather*}
  g \sim h \text{ and therefore}\\
  g^{-1} \sim h^{-1} ; \text { that is}\\
  (g^{-1}, h^{-1}) \in  R.
\end{gather*}

Thus we have proved that $R$ is a subgroup of $G \times G$, that is in
symbols 
$$
R \le G \times G.
$$

Conversely reversing the above arguments we can prove that if $R$ is
an relation and $R \le G \times G$, then $R$ is a congruence in
$G$. Thus we have following theorem.  

\begin{theorem}\label{chap3:sec6:thm12} % them 12
  The equivalence relation $R \subseteq G \times G$ is a congruence in
  $G$ if and only if $ R \le G \times G$ 
\end{theorem}

\section{The lattice of congruences and normal subgroups}\label{chap3:sec7} % sec 7.

Let $R,S$ be two binary relations in $E$. By the \textit{intersection}
of relations $R$ and $S$, we mean the relation whose graph is $R
\bigcap S$. The following theorem is an immediate consequence of the
definition of an equivalence relation. 

\begin{theorem}\label{chap3:sec7:thm13} % them 13.
  The intersection of any family of equivalence relations in $E$ is an
  equivalence relation.  
\end{theorem}

Let $S$ be any binary relation in $E$. Then

$R = R (S) \bigcap\limits_{S \subseteq R_i} R_i$ is the equivalence relation
\textit{ generated by}\pageoriginale $S$ where $R_i$ runs all the equivalence
relations containing $S$. In particular, 
$$
R (\phi) = \triangle
$$
is the \textit {identity relation}.

In general, the union of two equivalence relations need not be an
equivalence relation. We make the following definition. 

\begin{defi*}
  Let $\{R_i\}_{i \in  I}$ be a family of equivalence
  relations. The \textit {join} of $\{R_i\}_{i \in  I}$ is the
  equivalence relation generated by $\bigcup _{R_i}$. 
\end{defi*} 

The discussion above, leads to the following theorem. 

\begin{theorem}\label{chap3:sec7:thm14}% them 14.
  The set of all equivalence relation in $E$ is a lattice on $E\times
  E $ with ``$\subseteq \text{ in } E X E$'' as the partial order. 
\end{theorem} 
 
In this case, the `cap' and `cup' operations are the set intersection
and the `join' as we have defined above. 
 
Let us turn to groups. Let $G$ be a group. Similar to
Theorem \ref{chap3:sec7:thm13}, we
have for congruences the following theorem. 

\begin{theorem}\label{chap3:sec7:thm15} % them 15.
  The intersection of any family of congruences in $G$ is again a
  congruence. 
\end{theorem}

Thus we can now speak of \textit{the congruence generated by a binary
  relation} in $G$. As in the case of equivalence relations we can
similarly define the join  of a family of congruences in $G$. Note,
however, that ``join'' means different things according as we deal
with the lattices of equivalence or of congruences.  

Analogous\pageoriginale to Theorem $14$ is the following theorem.
\begin{theorem}\label{chap3:sec7:thm16} % them 16.
  The set $R$ of all congruences in a group $G$ is a lattice on $G
  \times G$ with ``$ \subseteq \text { in } G \times G$'' as the
  partial order.  
\end{theorem}  
 
Of course, the `cap' and the `cup' operations again are the set
intersection and the join. The lattice of congruences of a group have
important properties. But we shall not discuss them here. We only
mention that the lattice of congruences of a group $G$ is not a
sub-lattice of the lattice of equivalence relations.   

Let us now consider the set $\mathscr{N}$ of all normal subgroups of
$G$. We shall show that $\mathscr{N}$ is a lattice with set inclusion
as the partial order. For this we need to following theorem, the proof
of which is straight-forward; and we omit it.   

\begin{theorem}\label{chap3:sec7:thm17} % them 17.
  The intersection of a family $\{ N_i\}_{i \in  I}$ of normal
  subgroups is a normal subgroup. 
\end{theorem} 

We can now speak of the normal subgroup generated by a subset of
$G$. An an immediate consequence of this theorem we have 
 
\begin{theorem}\label{chap3:sec7:thm18} % them 18.
  The set $\mathscr{N}$ of all normal subgroups of $G$ is a lattice
  with inclusion as the partial order. 
\end{theorem}

Here again the `cap' operations is the intersection and the `cup'
operation is the ``join'', where the join of a family of normal
subgroups is the normal subgroup generated by the union of the groups
of this family. 

We have already need (Theorem \ref{chap3:sec4:thm10}) that there is
natural 1-1 mapping 
$\lambda$ of the set of all congruences in $G$ onto $\mathscr{N}$. In\pageoriginale
fact this mapping is a lattice isomorphism of $\mathscr{R}$ onto
$\mathscr{N}$. To prove this we have only to show that this mapping
$\lambda$ preserves the partial order. 
 
 Let $R \subseteq R'$ with $R, R' \in  \mathscr{R}$, and
 $R^\lambda =N$, $R^\lambda = N'$. Then,  
 $$
 N= \bigg\{ x \big | (x,1) \in  R \bigg\} \subseteq \bigg\{ x
 \big| (x,1) \in  R'\bigg\} = N' 
 $$
 
Thus we have,
\begin{theorem}\label{chap3:sec7:thm19} % them 19.
  The lattice $\mathscr{R}$ and $\mathscr{N}$ are isomorphic.
\end{theorem} 
 
 \section[Extension of a mapping of a set of...]{Extension of a mapping of a set of generators of a group to
   a homomorphism}\label{chap3:sec8} % sec 8. 
 
 Let $G=gp(E)$ and $H = gp (F)$ be groups and $\varphi$ an arbitrary
 mapping of $E$ into $F$. Under what conditions can $\varphi$ be
 extended to a homomorphism of $G$ into $H$? In other words, when can
 a homomorphism $\psi$ of $G$ into $H$ exists, with   
 $$
 e^\psi = e^\varphi,  \text{ for all } e \in  E ?
 $$
 
 Further if such a mapping $\psi$ exists, is it unique? The mapping
 $\varphi$ induces, in a natural way, a mapping $\varphi^*$ on the set
 of all words in $E$ with values in $H$; namely 
\begin{align*}
   (w (\underbar{e})) \varphi^* &= w (\underbar{e}^\varphi), \text{
     where }\\ 
   \underbar{e}^\varphi &= (e_1, \ldots, e_n)^\varphi = (e_1^\varphi, 
   \ldots, e_n^\varphi), e_i \in  E, i= 1, \ldots,  n. 
\end{align*} 

In\pageoriginale general $\varphi^*$ need not be a well define mapping of $G$. For
an element of $G$ may have more than one word representation it and
it is bot always true that the images by $\varphi^*$ of all these
words are the same elements of $H$. Whenever $\varphi^*$ induces a
mapping on $G$, we shall denote the induced mapping also by
$\varphi^*$. 

Suppose now, that $\varphi$ can be extended to a homomorphism $\psi$
of $G$ into $H$. Let $g=w(\underline{e})\in  G$. Then 
$$
g^\phi=(w(\underline{e}))^\psi =w(\underline{e}^\psi)
=w(\underline{e}^\varphi). 
$$

This shows that $\varphi^*$ induces a mapping on $G$ and that $\psi$
coincides with $\varphi^*$. Conversely let $\varphi^*$ induce a
mapping on $G$. If $g=w(\underbar{e})$, $h=u(\underbar{e})$, then $g^{\varphi^*}=w
(\underbar{e}^\varphi)$, $h^{\varphi^*} =u(\underbar{e}^\varphi)$; and
$(gh)^{\varphi^*}=(w(\underbar{e})u(\underbar{e}))^{\varphi^*} =
w(\underbar{e}^\varphi)u(\underbar{e}^\varphi)=g^{\varphi^*} h^{\varphi^*}$. Hence
$\varphi^*$ is a homomorphism of $G$ into $H$. Thus we have proved the
following theorem.
 
\begin{theorem}\label{chap3:sec8:thm20}%Them 20.
  The mapping $\varphi$ can be extended to a homomorphism of $G$ into
  $H$ if and only if $\varphi^*$ induces a mapping on $G$. Further,
  there can be only one such extension and this then coincides with
  $\varphi^*$. 
\end{theorem}

Let $\varphi$ be a mapping of $E$ into $H$ and $\varphi^*$ the mapping
induced by $\varphi$ on the set of all words in $E$. Suppose
$\varphi^*$ induces a mapping on $G$. Let 
$$
u(\underbar{e})=v(\underbar{e})
$$
be a relation in $G$. Suppose $\varphi^*$ induces a mapping on $G$, we have
$$
\displaylines{\hfill 
  (u(\underbar{e}))^{\varphi^*}= (v(\underbar{e}))^{\varphi^*};\hfill \cr 
  \text{ that is} \hfill 
  u(\underbar{e}^\varphi)=v(\underbar{e}^\varphi)\hspace{1.1cm}\hfill }
$$
is\pageoriginale a relation valid in $H$.

Conversely suppose every relation
$$
u(\underbar{e})=v(\underbar{e})
$$
in $G$ leads to a valid relation
$$
u(\underbar{e}^\varphi)=v(\underbar{e}^\varphi) \text{ in } H.
$$

Now if $x$ is any element in $G$, say
$$
x=u(\underbar{e}).
$$

Then 
$$
x^{\varphi^*}=(u(\underbar{e}))^{\varphi^*}=u(\underbar{e}^\varphi).
$$

If also  
$$
\displaylines{\hfill 
  x=v(\underbar{e})\hspace{.8cm}\hfill \cr
  \text{then} \hfill  
  x^{\varphi^*}=(v(\underbar{e}))^{\varphi^*}=v(\underbar{e}^\varphi) \hfill }
$$

But 
$$
u(\underbar{e})=v(\underbar{e}) \quad(=x)
$$
is\pageoriginale a relation in $G$. Therefore
$$
u(\underbar{e}^\varphi)=v(\underbar{e}^\varphi) 
$$
is valid relation in $H$; that is $\varphi^*$ induces a well-defined
mapping on $G$. Hence by Theorem \ref{chap3:sec8:thm20}, we have 

\begin{theorem}\label{chap3:sec8:thm21} % Them 21
  A mapping $\varphi$ of of $E$ into $H$ can be extended to a
  homomorphism of $G=gp(E)$ into $H$ if and only if every relation 
  $$
  u(\underbar{e})=v(\underbar{e}) \text{ in } G
  $$
  leads to a relation
  $$
  u(\underbar{e}^\varphi)=v(\underbar{e}^\varphi) \text{ in } H.
  $$

  Since every relation can be derived from the defining relations, we
  have the following corollary. 
\end{theorem}

\begin{coro*}[ven Dyck (1882)] 
  Let $G=gp(E)$ and $H=gp(F)$, and let $\varphi$ be a mapping of $E$
  into $F$. Then $\varphi$ can be extended to a homomorphism of $G$
  into $K$ if and only if every defining relation of the form 
  $$
  u(\underbar{e})=v(\underbar{e})
  $$
  turns into a valid relation
  $$
  u(\underbar{e}^\varphi)= v(\underbar{e}^\varphi)
  $$
  between the elements of $F$ upon applying.
\end{coro*}
