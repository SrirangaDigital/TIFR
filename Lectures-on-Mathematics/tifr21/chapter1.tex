
\chapter{Introduction, Definitions and Notations}\label{chap1}

\section{Abstract Algebras}\label{chap1:sec1} % sec 1 

In\pageoriginale this chapter we shall derive certain properties of
groups and fix certain notations.

Let $E$ be any set and $\Omega$ a set of functions defined on the
\textit{Cartesian products} 
$$
E^o = \{\phi\}, E, E^2,  \ldots E^n,  \ldots
$$
with values in $E$, where $\phi$ denotes the empty set and 
$$
E^n = \left\{ (x_1,  \ldots,  x_n) \big| x_i \in  E, i=1, 
\ldots n \right\}. 
$$

The pair ($E$, $\Omega$) is called an \textit{algebraic system } or an
\textit{abstract algebra}. $E$ is called the \textit{carrier} of the
algebra ($E$, $\Omega$) and the elements of $\Omega$ are called
\textit{operators}. 

If $\omega \in  \Omega$ is a function on $E^n$ with values in
$E$, we say that $\omega$ is in \textit{n-ary operator }. Thus if
$\omega$ is an n-ary operator, then 
$$
\omega (x_1, \ldots, x_n) \in  E, ~\text{for all}~ (x_1,  \ldots
, x_n) \in  E^n. 
$$

A \textit{nullary operator } is a function on the set $\{ \phi \}$
with value in $E$. Thus if $\omega$ is a nullary operator then it is a
function with the argument  $\phi$ and with value in $E$. We denote
this value by $\in  \{~ \}$. 

We\pageoriginale shall use the terms \textit{unary} and \textit{binary} operators
for 1-ary and 2-ary operators respectively. 

\section{Groups}\label{chap1:sec2} % sec 2

We are here interested in a particular class of algebraic system
called groups. 
\begin{defi*}
  A group ($G$, $\Omega$) is an algebraic system with $G$ as its
  carrier and $\Omega$ consisting of a nullary operator $\in $,
  an unary operator $L$ and a binary operator $\pi$, related by the
  following laws: 
  \begin{enumerate}[(1)]
  \item $\pi$ ($x$, $\pi$ ($y$, $z$)) $= \pi$ ($\pi$ ($x$, $y$), $z$),
    for every $x$, $y$, $z \in  G$ (Associative Law); 
  \item $\pi$ ($x$, $L (x)$) $= \in  \{~ \}$, for every $x
    \in  G$; 
  \item $\pi$ ($x$, $\in  \{~\}$) $= x$, for every $x \in  G$. 
  \end{enumerate}
\end{defi*}

We shall for convenience write, 
\begin{align*}
  \pi (x, y) & = xy,\\
  L (x) &= x^{-1}, \\
  \in  \{ \} & = 1.
\end{align*} 

In this notation, it is customary to call \textit{xy} the product
elements $x$ and $y$. The above three laws read as follows when
written multiplicatively. 
\begin{enumerate}[(1$'$)]
\item $x (yz) = (xy) z$, \quad (Associative law)
\item $xx^{-1} = 1$,
\item $x1 = x$.
\end{enumerate}

Because\pageoriginale of $(3')$ we say that $1$ is a \textit{right neutral }
element. Similarly as suggested by $(2') x^{-1}$ is a \textit{right
  inverse} of $x$. For the sake of brevity we shall identify the group
($G$, $\Omega$) with its carrier $G$ and refer to $G$ as a group
through this chapter. 

If a group $G$ in addition to the above three laws satisfies 
\begin{enumerate}
\item[(4)] $\pi$ ($x$, $y$) $= \pi$ ($y$, $x$), for all $x$, $y \in 
  G$ (Commutative law)  

  or
  
\item[(4$'$)] $xy= yx$ (in the multiplicative notation)
\end{enumerate}  
then $G$ is an \textit{abelian group} (or a \textit{Commutative group}).

For the abelian group it is sometimes convenient to use the following
additive notation 
\begin{align*}
  \pi (x,y) &= x+ y\\
  L (x) & = -x\\
  \in  \{ ~\} & = 0.
\end{align*}

\section{Some elementary properties of groups}\label{chap1:sec3} % sec 3.

\begin{enumerate}[(1)]
\item In the definition of a group the associative law is formulated
  for products of three elements of $G$. One can prove by induction on
  the number of factors that the corresponding law holds for products
  of any finite number of factors; in other words, the product will be
  independent of the way in which the brackets are inserted. The
  brackets are, therefore, irrelevant and will later on usually be
  omitted. The proof of the general associative law is straight
  forward and we omit it. 
\item The\pageoriginale right neutral element $'t'$ is also a left neutral element;
  in other words,  
  $$
  1x = x, ~\text{for all}~ x \in G.
  $$
  
  \begin{proof}
    From law $(3')$, it follows that
    $$
    11 = 1, 
    $$
    then
    \begin{gather*}
      1 (xx^{-1}) = xx^{-1} \quad \text { from }(2'),\\
      (1x)x^{-1} = xx^{-1} \quad \text{from} (1').
    \end{gather*}
  \end{proof}
  
  Therefore,
  
  $((1x)x^{-1}) (x^{-1})^{-1} = (xx^{-1}) (x^{-1})^{-1}$, where
  $(x^{-1})^{-1}$ is the right inverse of $x^{-1}$.  

  An application of the associative law gives, 
  $$
  \displaylines{\hfill 
  (1x) (x^{-1} (x^{-1})^{-1}) = x(x^{-1} (x^{-1})^{-1}),\hfill \cr
    \text{and therefore}\hfill 
    (1x)1 = x1 =x.\hfill}
  $$

  Finally, by another application of the associative law, 
  $$
  1(x1)= 1x = x 
  $$

  This proves (2).
  
\item The right inverse $x^{-1}$ is also a left- inverse of $x$; in
  other\pageoriginale words,  
  $$
  x^{-1}x=1,  \text{ for all } x \in  G.
  $$
  \begin{proof}
    $(x^{-1} x) x^{-1} = x^{-1} (xx^{-1}) = x^{-1} 1 = x^{-1}$.
    
    Therefore,
    \begin{gather*}
      ((x^{-1} x) x^{-1})(x^{-1})^{-1}= x^{-1} (x^{-1})^{-1}= 1,\\
      (x^{-1}x) (x^{-1} (x^{-1})^{-1}) = x^{-1} (x^{-1})^{-1} =1.
    \end{gather*}

    Hence, $(x^{-1} x) 1=1$, 
    $$
    x^{-1} x = (x^{-1} x) 1=1.
    $$

    This proves (3).
  \end{proof}
  
  We say that 1 is a (two-sided) neutral element or unit element, now
  that it is both left neutral and right neutral. Similarly $x^{-1}$ is
  an inverse of $x$. 
  
  
\item There is only one right neutral element in $G$. For let $n$ be
  any right neutral element. An application of $(2)$ immediately gives 
  $$
  n = 1n = 1.
  $$
  
  This, in particular proves that $1$ is the only neutral element of $G$.
  
\item The equation $ax = b$, with $a, b \in G$, has the unique
  solution $x = a^{-1}b$, in $G$. It is easy to verify that $a^{-1}b$
  is a solution of the above equation. Now if $x$ and $y$ are two
  solutions of the equation, we have 
  $$
  x = 1x = (a^{-1}a) x = a^{-1} (ax) = a^{-1} (ay) = (a^{-1} a)y = 1y = y.
  $$

  This\pageoriginale proves the uniqueness. 
  
  Thus in a group the left cancellation law holds. Dually it follows
  that the right cancellation law also holds. As a consequence of
  $(5)$, $x^{-1}$ is the only inverse of $x$ and also $x$ is the
  inverse of $x^{-1}$ 
\end{enumerate}

\section{}\label{chap1:sec4} % sec 4

We note that we have defined groups by postulates of the from ``for
all $x$, $y$, $z,\ldots, a$ certain equation is true''. This does not
mean that we have made no existential assumptions; but all existential
assumptions have gone into the general algebraic frame work; that is
to say, they are of the form  ``there is a nullary operator
$\in , $ a unary operator $L$'', and so on. A class of algebraic
systems that is singled out, like that of groups, by postulates of the
form ``for all $x,\ldots$,  the equation $\cdots$ holds'' is said to be
\textit{equation-ally defined}, or a \textit{variety} of algebraic
systems. Thus groups, as we have defined them, form a variety. Not all
important and interesting classes of algebraic systems form varieties;
thus e.g. the class of fields is not a variety. This will be shown
later. 

\section[The multiplication of subsets of groups...]{The multiplication of subsets of groups and its relation to
  the lattice operations}\label{chap1:sec5} % sec 5

Let ($E$, $\Omega$) be an algebraic system, $\omega \in 
\Omega$, an $n$-ary operator, $X_1, \ldots,  X_n$, $n$ subsets of the
carrier $E$. We define the set 
\begin{gather*}
  \omega (X_1, \ldots,  X_n) \subseteq E ~~\text{by}\\
  \omega (X_1, \ldots,  X_n) = \left\{ \omega (X_1,  \ldots,  X_n)
  \big| x_i \in  X_i, i=1, \ldots,  n \right\}. 
\end{gather*}

Let\pageoriginale $G$ be a group, $D$, $E$ and $F$ subsets of $G$. Correspondingly we have
\begin{align*}
  EF & = \bigg\{ ef \big| e \in  E, f \in  F \bigg\}, \\
  E^{-1} & = \bigg\{e^{-1} \big | e \in  E  \bigg\}.
\end{align*}

We denote the set $E \{ f \}$ by $Ef$ and similarly $\{ e \} F$ by
$eF$. Also we identify $\{ e \} \{ f \}$ with element $ef$. 

Using the associativity of the multiplication of the elements of $G$
it is easy to verify that the same holds for the multiplication of
sets. In other words, 
$$
D (EF) = (DE)F,  ~\text{for }~ D, E, F ~\text{ subsets  of }~ G.
$$

Let $F \subseteq G$, $\{ D_i \}_{i \in  I}$ be a family of
subsets of $G$; then  
\begin{enumerate}[(1)]
\item $(\cup D_i )F = \cup D_i F$.
\end{enumerate}

\begin{proof}
  Let $g \in  (\cup D_i ) F$; then
  $$
  g= df  \text{ with } d \in  \cup D_i,  f \in  F; \text{ now }
  $$
  $$
  d \in  D_j  \text{ for some } j \in  I; \text { hence }
  $$
  $$
  g = df \in  D_j F \subseteq \cup D_i F,
  $$
  and thus, as $g$ was arbitrary,
  $$
  (\cup D_i)F \subseteq \cup D_i F.
  $$
  
  Conversely,\pageoriginale if $g \in  \cup D_i F$, then
  $$
  \displaylines{\hfill 
  g \in  D_j F ~\text{for some}~ j \in  I;\hfill \cr
  \text{thus}\hfill 
  g = df ~~\text{with}~~ d \in  D_j,  f \in  F,\hfill }
  $$
  
  Hence $d \in  \cup D_i$, and therefore
  $$
  g = df \in  (\cup D_i) F, ~\text{and again as }~ g ~\text{was
    arbitrary},
  $$
  $$
  \cup D_i F \subseteq (\cup D_i ) F.
  $$

  Combining this with the above inclusion we have the required equality.
  
  In particular, we have
  $$
  (D \cup E) F = DF \cup EF, ~\text{for}~ D, E, F \subseteq G. 
  $$
\end{proof}

A similar straightforward verification shows that
\begin{enumerate}
\item[(2)] $(\cup D_i)F \subseteq \cap D_i F$.
\end{enumerate}

In particular, we have
$$
(D \cap E)F \subseteq DF \cap EF.
$$

The following example demonstrates that in general inclusion cannot be
replaced by equality in (2). 

Take $G$ to be the additive group of integers, and 
$$
\displaylines{\hfill 
  E = \{1\}, D = \{ -1 \}, F=G,  \text{then}\hfill \cr
  \hfill (D \cap E)F = \phi,   \quad DF \cap EF = G.\hfill }
$$\pageoriginale

\section{Subgroups}\label{chap1:sec6} % sec 6

Let $S \subseteq G$, and
\begin{enumerate}[(1)]
\item $\in  \{~ \} \in  S$,
\item $L (s) S$, for every $s \in  S$,
\item $\pi (s, t) \in S $, for every $s, t \in S$.
\end{enumerate}

It is obvious that $S$ is a group with the set of operators $\Omega =
\bigg\{ \in ,  L,  \pi \bigg\}$. We call $S$, a
\textit{subgroup} of $G$. It should be noted that by definition a
subgroup is non-empty. If a subgroup $S$ of $G$ is a proper subset of
$G$, we call it a \textit{proper subgroup}. 

Hereafter the notation ``$S \leq G$''  will be used for ``$S$ is a
subgroup of $G$''. When $S$ is a proper subgroup of $G$, we shall
write ``$S < G$''. 

The definition of a subgroup is immediately seen to be equivalent to
the following conditions. 
\begin{enumerate}[(1$'$)]
\item $1 \in  S$, 
\item $S^{-1} \subseteq S$,
\item $SS \subseteq S$.
\end{enumerate}

These three conditions can be replaced by the apparently weaker
condition given in the following simple theorem. 

\begin{theorem}\label{chap1:sec6:thm1} % them 1
  The subset $S$ of the group $G$ is a subgroup if, and only if 
  \begin{enumerate}[\rm (i)]
  \item $S \neq \phi$, 
  \item $SS^{-1} \subseteq S$.
  \end{enumerate}
\end{theorem}

 Condition\pageoriginale (ii) means that for any $s$, $t \in  S$, the
 ``right quotient'' $st^{-1} \in  S $: we then say that $S$ is
 \textit{closed} under right  division. Similarly closure under left
 division can be defined. 

\begin{proof}
  The `only if' part of the theorem is trivial. We proceed to prove the
  `if' part. Since $S \neq \phi $, there is an element $x \in 
  S$, and therefore by hypothesis 
  $$
  xx^{-1} = 1 \in  S.
  $$
  Now, for any $x \in  S$, 
  $$
  1x^{-1} = x^{-1}\in  S.
  $$
  
  Further, if $x$, $y$, $\in  S$, then by what we have just proved
  $$
  \displaylines{\hfill y^{-1} \in S\hfill \cr
    \text{and therefore,} \hfill 
    xy = x(y^{-1})^{-1} \in  S.\hfill }
  $$
  This proves that $S$ is a subgroup. 
\end{proof}

Also, by symmetry it follows that a non-empty subset of $G$, closed
under left division is a subgroup.  

In the above theorem instead of right or left division, we can also
take their transposes. In other words, a non-empty subset $S$ of $G$
is a subgroup if and only if it is closed under one of the following
four binary operations. 

\medskip
\begin{tabular}{llll}
  (1) & $\varphi$ ($x$, $y$) $= xy^{-1}$, &  (2) & $\varphi^{*}
  (x,y) = y^{-1}x$,\\ 
  (3) & $\psi (x,y) = x^{-1}y$, & (4) & $\psi^{*} (x,y) = yx^{-1}$.
\end{tabular}
\medskip

Graham\pageoriginale Higman (Higman and Neumann, 1952) has suggested a more general
problem which stands unsolved in the case of non-abelian groups.  
\begin{prob*}
  Let $\varphi$ be a binary operator (expressible in terms of $\varepsilon,  L
 ,  \pi$ and two variables) with the property that $S \subseteq G$ is
  a subgroup if only if 
  \begin{enumerate}[(1)]
  \item $S \neq \phi$
  \item $\varphi (x,y) \in  S$, for all $x,y \in  S$.
  \end{enumerate}
\end{prob*}

What forms can $\varphi$ take?

In the case of abelian groups it is proved that the only possibilities
are the above four functions (which in this case reduce to only two
functions, right and left division). 

Nothing is known in the case of non-abelian groups.
