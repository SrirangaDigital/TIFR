
\chapter{An Embedding Theorem}\label{chap8}%theo VIII

\section{}\label{chap8:sec1}%sec 1

The\pageoriginale group theoretical constructions which we have
discussed in Chapter \ref{chap6} will be used to prove the following
embedding theorem.  

\setcounter{theorem}{0}
\begin{theorem}[Higman,Neumann and Neumann]\label{chap8:sec1:thm1}%theo 1
  Every countable group $G$ can be embedding in a $2$-generator group $H$.
\end{theorem}

\begin{proof}
  Let
  $$
  G = gp (a_1,a_2, \ldots)
  $$
  be a group generated by $\{ a_i\}_{i \varepsilon I}$ where $I$ is
  countable; and let $C$ be an infinite cyclic group generated by an
  element $c$,thus 
  $$
  C= gp (c).
  $$
  
  [Later we shall modify this by choosing $C$ is a finite cyclic group
    provided that certain conditions are satisfied; cf $p. 146$.] 
\end{proof}

We form the wreath product of the groups $G$ and $C$, 
$$
P= G Wr C.
$$

Every element of $P$ is of the form $c^s f$ where $f$ is a function on
$C$ with values in $G$ and the product of any two elements $c^sf$ and
$c^tg$\pageoriginale of $P$ is given by 
\begin{multline*}
  (c^s f) (c^s g) = c^{s+t} f^c{^{^t}}g, ~\text{ where }~ f, g \in
  G^C, ~\text{ and }\\
  f^c{^{^t}} (c^n) = f (c^{n-t}), ~\text{ for }~ n =0,
  \pm 1, \pm 2,\ldots 
\end{multline*}

In the group $P$ (and in fact in $G^C$) we single out certain elements
$g_i, i \in I$, defined by 
$$
g_i (c^j) = g^{-j}_i, i \in I, j=0, \pm 1, \pm 2, \ldots
$$

We now compute the elements $k_i, i \in I$, where
$$
k_i = [g_i,c].
$$

As
$$
k_i = g^{-1}_i c^{-1} g_i c = g_i^{-1} g^c_i,
$$
we see that $k_i \in G^C$; and
\begin{align*}
  k_i (c^j) &= g^{-1}_i g^c_i (c^j) = g_i^{-1} (c^j) g^c_i (c^j)\\
  & = g^{-1}_i (c^j) g_i (c^{j-1}) = a_i^j a^{-(j-1)}_i  = a_i.
\end{align*}

Thus $k_i$ are constant functions taking the value $a_i$ for all
$c^j_s$. The constant functions clearly form a group $G^\triangle$ and
this is isomorphic to $G$. We call $G^\triangle$ the diagonal subgroup
of $G^C$. [The diagonal can be defined in arbitrary cartesian powers,
  not\pageoriginale only of groups.] It is not difficult to see that all constant
functions are generated by those among them whose values are the
generators $a_i$ of $G$, thus 
$$
gp (\{k_i\}_{i \in I}) = G. 
$$

Note also that we have embedded $G$ in the commutator subgroup of $P$.

Now let $B$ be a cyclic group generated by an element $b$ and let $B$
be ``big enough'' to contain $b_i \in B, i \in I$,
satisfying the following conditions 
$$
b_i \neq 1, b_i \neq b_j \text { for } i \neq j 
$$
and $1 \neq b_i b_j, b_i b_j \neq b_k$, for all $i,j,k \in
I$. This we can achieve by taking $B$ to be infinite cyclic group 
 $$
 B = gp (b), \text { if } |I| = \mathscr{N}_0 :
 $$
  and if $|I| = g < \mathscr{N}_o$, we can either take $B$ to be the
  infinite cyclic group or 
 $$
 B = gp (b; b^m = 1), m =3d \text { or } m \ge 4d-1.
 $$
 
 When $B$ is the infinite cyclic group or $m = 3d$, we choose for instance
 $$
 b_i = b^{3i-1}, i \in I.
 $$
 
 If $m  \ge 4d-1$, we choose
  $$
 b_i= b^{2i-1}, i \in I.
$$

It\pageoriginale is easy to verify that $b_i, i \in I$ satisfy the above
conditions. We now form the wreath product of $P$ and $B$. Let 
$$
Q = P Wr B.
$$

Define $q \in Q$ (in fact $q \in P^B$) by
\begin{align*}
  q (1) &= c\\
  g (b_i^{-1}) & = g_i, i \in I,
\end{align*}
and $q (y) = 1$, for $y \neq 1, b_i^{-1}, i \in I$. Define further
$$
h_i [q^{b_i}, q] \in P^B \le  Q, \text { for } i \in I.
$$

We now compute $h_i$
\begin{align*}
  h_i (1) &= [q^{b_i}, q] (1) = [q^{b_i} (1), q(1)]\\
  &= [q (b^{-1}_i), q(1)] = [g_i, c]= k_i;\\
  \text{ next } \hspace{1.5cm} h_i (b^{-1}_j) & = [q^{b_i}, q] (b_j^{-1})
  = [q^{b_i} (b^{-1}_j), q (b^{-1}_j)]\hspace{2cm}\\ 
  & = [g (b_j^{-1} b_i^{-1}), q(b_j^{-1})].
\end{align*}

Now, we have chosen $b_i$, such that 
$$
1 \neq b_i b_j, b_i b_j \neq b_k, \text{ for } i,j,k \in I.
$$

Therefore\pageoriginale we have
$$
h_i (b_j^{-1}) = [q (b_j^{-1} b_i^{-1}), q(b_j^{-1})] = [1,g_j^{-1}] =
1, j \in I. 
$$

Finally
$$
h_i (y) = [q^{b_i} (g), q (y)] = [q^{b_i} (y), 1] = 1, \text{ for } y
\neq 1, b_j ^{-1}, j \in I. 
$$

Thus
\begin{align*}
  h_i (1) & = k_i,\\
  h_i (y) & = 1, y \neq 1.
\end{align*}
 
 We denote the group generated by the $h_i$ by $G^*$; it is then obvious that
 $$
 G^* = gp (\{ h_i\}_{i \in I}) \cong gp \{k_i\}_{i \in I} \cong G.
 $$
 
 Further
 $$
 G^* \le gp (q,b) = H
 $$
 
 This proves the theorem.
 
 This theorem was first proved (Higman, Neumann, Neumann, $1949$)
 using quite different methods. The proof (Neumann and Neumann,
 $1959$) which we have given here provides answer to a number of
 interesting questions of the form: if $G$ has the property $P$, can
 $H$ be chosen to have the property $P$ or some property closely
 related\pageoriginale to $P ?$  
 
 \section{Corollaries}\label{chap8:sec2}%sec 2

\subsection{}\label{chap8:sec2:subsec1} 
If $G \in \underset{=}{V}$, a variety, then $H
   \in \underset{=}{V} \underset{=}{A}^2$. For $G^C
   \in \underset{=}{V}$ and $P = G \,Wr\, C \in
   \underset{=}{V} \underset{=}{A}$, and therefore, 
   $$
   Q = P Wr B \in \underset{=}{V} \underset{=}{A}^2
   $$

   Since $H \le Q$, we have
   $$
   H \in \underset{=}{V} \underset{=}{A}^2.
   $$
   
   In particular if $G \in \underset{=}{A}^\ell$, we get
   $$
   H \in \underset{=}{A}^{\ell + 2}; \text{ thus we have }
   $$
 
\subsection{}\label{chap8:sec2:subsec2} 
A countable group which is soluble of length
$\le \ell$ can be    embedded in a 2-generator group, soluble of
length $\ell + 2$.  
 
 
 This is the best possible result in the sense that we can make
 examples of groups that are countable and soluble of length $\le
 \ell$ and which cannot be embedded in any finitely generated soluble
 group of length $\ell +1$. 

 We shall have give an example with $\ell = 1$; that is, we shall give
 an example of a countable abelian group which cannot be embedded in
 any finitely generated metablian group. In this context, we need a
 theorem which we state here without proof Theorem ($P$. Hall,
 $1954^b$). A finitely generated metabelian group satisfies the
 maximal condition for normal subgroups. 

 Consider\pageoriginale the group $G$ with a
 countable set of generators $a_1,a_2,\ldots$ presented by 
 $$
 G = gp (a_1, a_2, \ldots ; a^p_1 = 1, a^p_2 = a_1, \ldots, a^0_{i+1}
 = a_i, \ldots), 
 $$
 where $p$ is a prime. It is easy to verify that $G$ is isomorphic to
 the group of all $p^n$ th roots of unity for $n = 1,2, \ldots$. The
 group $G$ is known as ``Prufer $p^\infty$ - group'' or quasi-cyclic
 group. This group has many interesting properties. For instance all
 proper subgroups of $G$ are finite cyclic groups. For if $H \neq 1$
 is a proper subgroup of $G$, then 
 $$
 H = gp (a_n),
 $$
 where $n$ is the least positive integer such that $a_{n+1} \notin
 H$. It is easy to verify that 
 $$
 G/H \cong G.
 $$
 
 Thus all the factor groups of $G$ are either isomorphic to $G$ or the
 trivial group.  

 We shall now show that $G$ cannot be embedded in a finitely generated
 metabelian group.  

 Assume $G$ to be embedded in a metabelian group $K$. We shall
 identify the isomorphic of $G$ in $K$ with $G$ and take $G \le
 K$.\pageoriginale Consider the canonical mapping $\varphi$ of $K$ onto $K/K'$ where
 $K'$ is the derived subgroup of $K$. Since all the factor groups of
 $G$ are either isomorphic to $G$ or the trivial group, the image
 $G_1$ of $G$ under $\varphi$ is either group or is isomorphic to
 $G$. 

 Now if $G_1 \cong G$, then $G_1$ is not finitely generated. But we
 know that every subgroup of a finitely generated abelian group is
 finitely generated. (See Kurosh, $1955 \S 20, p.149$) Therefore,
 $K/K'$ and hence $K$ is not finitely generated. 

 On the other hand, if $G$ is the trivial group, then
 $$
 G \le K'.
 $$

 Let $G^K$ be the normal closure of $G$ in $K$. Then
 $$
 G^K \le K'.
 $$

 Define $A_n \le G^K$ by
 $$
 A_n \bigg\{ g \bigg| g \in G^K,  g^{p^{n}} = 1\bigg\}.
 $$

 Since $K$ is merabelian, $K'$ is abelian. Therefore $A_n$ is a group,
 for every $n$. We claim that $A_n$ are invariant in $G^K$. For, let
 $\eta$ be an endomorphism of $G^K$, then $(g^\eta)^{p^n} =
 (g^{p{^n}})^\eta = 1^\eta = 1$, for all $g \in A_n$; that   
 $$
 A_n^\eta \le A_n.
 $$

 Therefore $A_n$ are fully invariant and thus, a fortiori,
 characteristic\pageoriginale in $G^K$. But 
 $$
 G^K \triangle K (\text { trivially }).
 $$

 Hence,
 $$
 A_n \triangle K, \text { for all } n.
 $$

 Now we assert that
 $$
 A_1 \le A_2 \le A_3 \cdots
 $$
 is an infinite strictly ascending chain. For,
 $$
 a_{n+1} \in A_{n+1} \text { and } a_{n+1} \notin A_n.
 $$

 Therefore by P. Hall's theorem $K$ cannot be finitely generated. Thus
 $G$ cannot embedded in any finitely generated metabelian group. 

\subsection{}\label{chap8:sec2:subsec3} 
If $G$ is abelian, in the proof of Theorem \ref{chap8:sec1:thm1}
we can take the group 
$C$ to be of order $2$. But in this case we define $g_i \in
G^C, i \in I$, by 
\begin{align*}
  g_i (1) &= a_i\\
  g_i (c) &= 1, \text{ where } C = gp (c; c^2 = 1).
\end{align*}

Then,\pageoriginale
\begin{align*}
  k_i & = [g_i,c]= g^{-1}_i g^c_i \in G^c, i \in I \text{ and }\\
  k_i(1) & = g^{-1}_i (1) g^c_i (1) = a^{-1}_i,\\
  k_i(c) & = g^{-1}_i(c) g^c_i(c) = a_i.  
\end{align*}

It is easy to verify that when $G$ is abelian the mapping $\varphi$ of
$\big\{ a_i \big\}_{i \in I}$ into $P= G \,Wr \,C$ defined by   
$$
a^{\varphi}_i = k_i, i \in I
$$
can be extended a monomorphism of $G$ into $P$. Now one can proceed as
in the proof of the Theorem \ref{chap8:sec1:thm1}. 

If further, $G$ is finitely generated, we have seen that $B$ could be
taken to be a finite group. We have 


\subsection{}\label{chap8:sec2:subsec4} 
If $G \in \underset{=}{A}$ and finitely generated, then $H
\in \underset{=}{A}^3$ can be chosen as an abelian-by-finite
group. 

Now,
$$ 
\displaylines{\hfill  
  G^C \Delta P; ~~\text{ and } \hfill \cr 
  \text{hence}\hfill  (G^C)^B \Delta P^B.\phantom{hencehen}\hfill } 
$$

It is easy to verify that  
$$
P^B / (G^C)^B \cong (P/ G^C)^B.
$$

Now,\pageoriginale since $|B|< \infty$ and $|P/ G^C| < \infty$, we have 
$$
|P^B / (G^C)^B| < \infty.
$$

Now for any $f \in (G^C)^B$, we have 
$$
b^{-1} f b=f^b (G^C)^B.
$$

Therefore  
$$
(G^C)^B \Delta gp(P^B,b) =Q.
$$

Further, 
$$
Q/P^B = \frac{Q / (G^C)^B}{P^B/ (G^C)^B}
$$
Since, $|Q/P^B| < \infty, |P^B / (G^C)^B| < \infty$, follows that 
$$
|Q/ (G^C)^B| < \infty.
$$
 
As $G$ is abelian, the group $Q$ is abelian- by - finite. It is not
difficult to prove that the property of being abelian-by-finite is
inherited by subgroups (and also by factor groups). 

\subsection{}\label{chap8:sec2:subsec5} 
If $a^n_i =1$, for all $i \in I$, in the embedding procedure
of Theorem \ref{chap8:sec1:thm1} we can take $C$ be to be a cyclic group of order
$n$. It is easy\pageoriginale to verify that in this case the functions $g_i$ are un
ambiguously defined. 

\subsection{}\label{chap8:sec2:subsec6} 
It $G$ has finite exponent $n$ and is finitely generated, say by $d$
elements, then $H$ can be chosen of finite exponent $n^{2+r}$, where
$r$ is an integer such that $m = n^r$ is a possible choice for the
order of the group $B$ occurring in the proof of the theorem; that is
$m = 3d$ or $m \geq 4d-1$. 

Now,
$$
P= G \,Wr \,C,
$$
where $C$ is a cyclic group of order $n$. Since $G$ is a group of
exponent $n$, so is $G^C$. Further  
$$
P/G^C \cong C.
$$

If $x \in P$, then $x^n \in G^C$, and 
$$
(x^n)^n = x^{n^2} = 1;
$$
that is $P$ and therefore $P^E$ is of exponent $n^2$. Again
$$
Q/P^B \cong B,
$$
where $B$ has been chosen to be a finite cyclic group of order $n^r$. 

Now if $y \in Q$, then $y^{n^r} \in P^B$, and therefore
$$
(y^n{^r})^{n^2} = y^{n^{2+r}}= 1,
$$
that\pageoriginale is $Q$, and hence $H \leq Q$ is of exponent $n^{2+r}$.

From Corollary $6$, we immediately get the following reduction
theorems for the Burnside conjectures. 

\subsection{Reduction Theorem for the Full Burnside
  Conjecture}\label{chap8:sec2:subsec7}  %sec 7

All finitely generated groups of exponent $n$, are finite if all
2-generator groups of exponent $n^s$, for all $s$, are finite. 

This ``reduction theorem'' was first proved by Sanov $(1945)$. It has
lost interest in view of Novikov's recent results. 

\subsection{Reduction Theorem for the Restricted Burnside
  Conjecture}\label{chap8:sec2:subsec8} %sec 8

It there is a number $\beta (n^k, 2)$ such that every finite
$2-$generator group of exponent $n^k$ has order $\leq \beta (n^k,2)$,
then there is a number $\beta (n,d)$ such that every finite
d-generator group with $d \leq  \dfrac{1}{4} (n^{k-2} +1)$ and of
exponent $n$ has order $\leq \beta (n,d)$. In fact, 
$$
\beta (n,d) \leq \beta (n^k, 2).
$$
