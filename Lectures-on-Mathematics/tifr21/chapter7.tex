
\chapter{Varieties of Groups (Contd.)}\label{chap7} % chp VII

\section{}\label{chap7:sec1}%Sec 1

Let\pageoriginale $\underset{=}V$ be a variety defined by a set of laws $L$, that is
$\underset{=}V$ consists of all groups in which the laws of $L$
hold. If $G \in \underset{=}V$ and $H \le G$, then $H
\in \underset{=}V$. Let $G'$ be any epimorphic image of $G$;
that is, there is an epimorphism $\varphi$ of $G$ onto $G'$. Now if  
$$
w(X_1,  \ldots,  X_n) = 1
$$
is a law in $\underset{=}V$, then it is also a law in $G'$. For, let
$g'_1, \ldots,  g'_n$ be arbitrary elements of $G'$. Because $\varphi$ is
an epimorphism, there exist elements $g_1, \ldots,  g_n \in G$
such that  
$$
g^\varphi_i = g'_i,  1, \ldots, n.
$$

Now,
\begin{align*}
  (w(g_1, \ldots, g_n))^\varphi = 1 &= w(g^\varphi_1,  \ldots, 
  g^\varphi_n)\\ 
  \text{i.e.,}\qquad
  w(g'_1, \ldots,  g'_n) & = 1 ; ~\text{thus} \\
  w(X_1, \ldots,  X_n) &= 1
\end{align*}
is a law in $G$.  Therefore
$$
G' \in \underset{=}V
$$

Let\pageoriginale $ \{ G_i  \}_{i \in I}$ be an arbitrary family of groups of
$ \underset{=}{V}$. We assert that the cartesian product $P$ of  $  \{
G_i  \}_{i \in I} $ is in the variety $ \underset{=}{V}
$. Consider 
$$
f^* = w ( f_1, \ldots, f_n ) \in P, 
$$
where $ f_1, \ldots,  f_n $ are arbitrary elements of $P$ and 
$$
w ( X_1, \ldots,  X_n ) = 1,
$$
is a law in $L$. Then
\begin{gather*}
  f^* (i) = w ( f_1 (i), \ldots,  f_n (i)) = 1, \text{ for all } i
  \in I, \text{ since }\\ 
  f_1 (i), \ldots,  f_n (i)  \in  G_i  \text{ and } G_i
  \in \underset{=}{V}. \text{ Therefore }\\ 
  f^* = w ( f_1, \ldots, f_n ) = 1 p \text{ that is } \\
  w ( X_1, \ldots, X_n ) = 1, \\
\end{gather*}
is a law in $P$. That is 
$$
P \in \underset{=}{V}.
$$

Hence we have prove 
\setcounter{theorem}{0}
\begin{theorem}\label{chap7:sec1:thm1} %Thm 1
  Every variety is closed under the operations of forming  subgroups
  $(S)$, epimorphic maps $(Q)$ and cartesian products  $(R)$.  
\end{theorem}

Theorem $1$, enables us to make new groups of a variety $
\underset{=}{V} $ by using there which we already know. A variety in
general is not closed under the operation of  ``wreathing''. 

The\pageoriginale converse of the above theorem is also true. Before proceeding to
prove the converse we wish to remark that many of the concepts which
we have introduced for groups can be generalised to abstract algebraic
system in a natural way. For example, we can speak of a subalgebraic
system of an algebraic system, a homomorphism of an algebraic system
in to another, the cartesian product of a family of algebraic
systems. Note that the concept of direct product cannot in general be
introduced in the theory of algebraic system, as we may not have an
analogue of the neutral element of a group. Thus proofs of
Theorem \ref{chap7:sec1:thm1} and Theorem \ref{chap7:sec1:thm2}
can easily be  carried over to abstract algebraic systems. 

\begin{theorem}\label{chap7:sec1:thm2} %theorem 2
  A class of groups  closed under the operations $ Q,R,S $ is a variety.
\end{theorem}

We first prove two lemmas.

Let $ \underset{=}{G} $ be a class of groups. We form the closure $
\underset{=}{C} $ of  $ \underset{=}{G} $ under the operations $
Q,R,S $. Let $ \underset{=}{V} $ be the least variety containing $
\underset{=}{G} $. By Theorem \ref{chap7:sec1:thm1} $ \underset{=}{V}
$ is closed under the operations $ Q,R,S $. Therefore 
$$
\underset{=}{C} ~~\underline{\subset} ~~\underset{=}{V}.
$$
\setcounter{Lemma}{0}
\begin{Lemma}\label{chap7:sec1:lem1}%Lem 1
  There is a group  $G^*$ with the following properties:
  \begin{enumerate}[(i)]
  \item  $G^* \in \underset{=}{C}$
  \item Every law $w (\underbar{X}) = 1$
  \end{enumerate}
  valid in $G^*$, is valid in every group of $\underset{=}{G}$ ( and
  is hence a law of $\underset{=}{V} )$.  
\end{Lemma}

\begin{proof}
  Consider\pageoriginale the class $\underset{=}{F}$ of all finitely generated
  groups of $\underset{=}{C}$. We split $\underset{=}{F}$ into
  disjoint classes $\underset{=}{H}_\alpha $ of mutually isomorphic
  groups, that is any two groups of $\underset{=}{F}$ are isomorphic
  if and only if they belong to the same $\underset{=}{H}_\alpha
  $. From each $\underset{=}{G}_\alpha$ we choose a group $H_\alpha$
  and form the cartesian product $G^*$ of  $H_\alpha s $. Since each
  $H_\alpha \in \underset{=}{C} $, and $\underset{=}{C}$ is
  closed under the operations $Q,R,S$, we have  
  $$
  G^* ~\in~ \underset{=}{C}. 
  $$
\end{proof}

Let
$$
 w(X_1, \ldots, X_n ) = 1,
$$
be a law in $G^*$ and $G \in \underset{=}{C} $. For any $g_1,
\ldots,g_n \in G $, let  
$$
H = gp (g_1, \ldots, g_n) ~ ( \leq G ). 
$$

Now, $G \in \underset{=}{G} $ and $\underset{=}{C}$ is closed
under the operations of taking subgroups. 

Therefore
\begin{align*}
  &H \in \underset{=}{C}; \text{ infact } \\
  &H \in \underset{=}{F}.
\end{align*}

Hence
$$
H \simeq H_\alpha \text{ for some } \alpha.
$$

Denote this isomorphism by $\theta$. Let $\varphi_\alpha$ be the
projection of $G^*$ onto $H_\alpha$. Then $\varphi_\alpha \theta^{-1}
$ is an epimorphism of $G^*$ onto $H$. 

Therefore,\pageoriginale
$$
\displaylines{\text{ as } \hfill w (X_1, \ldots, X_n) = 1 \hfill} 
$$
is  a law in $G^*$, it is also a law in $H$, and thus in particular
$$
w (g_1, \ldots, g_n) =1 
$$
is a relation in $H$ and in $G$. But $g_1, \ldots,g_n$ were
arbitrarily chosen in $G$. Hence 
$$
w(X_1, \ldots,X_n) = 1 
$$
is a law in $G$. Thus every law valid in $G^*$ is also valid in
$\underset{=}{G}$ and hence in $\underset{=}{V}$. 

We have to verify from an axiomatic set-theoretic point of view that
the construction of the cartesian product of the $H_\alpha$ is
legitimate, that is to say we have to verify that the $H_\alpha$ form
a family ( or that they can be indexed by a set ). Note that we have
made a distinction between ``class'' and  ``set'', though no emphasis has
been placed on this  distinction, as being outside  group theory
proper.  

Now, every $H_\alpha$ is isomorphic to a quotient group of  a free
group of  finite rank, say 
$$
H~ \simeq ~F_n /R,
$$
where $F_n$ is the free group of rank $n$ and $R$ a suitable normal
subgroup of $F_n$. Clearly $F_n$ is countable for every $n$ and
therefore the\pageoriginale cardinality of the set of all such Rs cannot exceed
$2^{\mathcal{N} 0}$. Hence there cannot be more $H_\alpha$ $s$ than
$\mathcal{N}_{0} 2^{\mathcal{N}0} = 2^{\mathcal{N}0} $; and thus they
form a family. Hence we have 

\begin{coro*}
  The group $G^*$ of the lemma can be chosen to have order 
  $$
  |G^*| \leq 2^{2^{\mathcal{N}_0}}.
  $$
\end{coro*} 

\begin{Lemma}\label{chap7:sec1:lem2}%Lem 2
  Let $G^*$ be a group with the property that every law valid in $G^*$
  is also valid in $\underset{=}{G}$ ( and hence in $\underset{=}{V} )
  $, and let $I$ be a set. Then  there is a subgroup $F^* \leq
  G^{*^{G^{*^{I}}}}$ such that $F^*$ is generated by a set of cardinal
  $|I|$, say 
  $$
  F^* = gp ( \big \{ f_i \big \}_{i \in I} ) 
  $$
  and if $ G \in \underset{=}{V}$ is also generated by a set
  of cardinal $|I|$, say  
  $$
  G = gp ( \big \{ e_i \big \}_{i \in I} ),
  $$
  then there is an epimorphism $\varphi$ of $F^*$ onto $G$ with 
  $$
  f^\varphi_i = e_i,  ~ i \in I. 
  $$
\end{Lemma} 

\begin{proof}
  Every element of $G^{*^{G^{*^{I}}}} $ is a function on $G^{*^{I}}$
  with values in $G^*$. To every $i \in I$, we define $f_i
  \in G^{*^{g^{*^PI}}} $, by  
  $$
  f_i (g) = g (i), \text{ for all } g \in G^{*^{I}}.
  $$

  Let\pageoriginale 
  $$
  F^* = gp ( \big \{ f_i\big \}_{i \in I}).
  $$

  We define the mapping $\varphi$ of $ \big \{ f_i \big \}_{i \varphi
  I}$ onto $ \big \{ e_i \big \}_{i \in I} $ by 
  $$
  f_i = e_i, \text{ for all }  i \in I.
  $$

  We claim that $\varphi$ can be extended to an epimorphism of $F^*$
  onto $G$. To prove this we have only to show that all the relations of
  $F^*$ go  over to the relations of $G$ upon applying $\varphi$. 
\end{proof}

Let
$$
u (f_{i_{1}}, \ldots, f_{i_{n}}) = 1, 
$$
be a relation in $F^*$, with  $f_{i_{1}}, \ldots, f_{i_{n}}
\in F^* $. Then  
$$
\displaylines{\hfill 
  u(f_{i_{n}}, \ldots, f_{i_{n}} ) (g) = 1, \text{ for all } g
  \in G^{*^{I}}\hfill \cr 
  \text{i.e.,}\hfill u(f_{i_{n}}, \ldots, f_{i_{n}} (g)) = 1$, for all $g
  \in G^{*^{I}} \hspace{.55cm}\hfill \cr
  \text{i.e.,} \hfill u (g (i_1), \ldots,g (i_n)) = 1$, for all $ g
  \in G^{*^{I}}.\hspace{.8cm}\hfill } 
$$

Let $g^*_1,  \ldots,g^*_n$ be arbitrary elements of $G^*$. There is an
element of $G^{*^{I}}$, that is a function on $I$ to $G^*$, which
takes the values $g^*_1, \ldots, g^*_n$, at $i_1, \ldots, i_n$
respectively. We only have to define $h \in G^{*^{I}}$ by  
$$
h (i_1) = g^*_1, \ldots, h(i_n) = g^*_n 
$$
and\pageoriginale $h(i)$ arbitrary otherwise, say
$$
h(i) = 1 \text{ when } i \neq i_1, \ldots, i_n.
$$

Then
$$
u(g^*_1, \ldots, g^*_n)  = u(h(i_1), \ldots, h (i_n )) = 1,  
$$
thus, as $ g^*_1, \ldots, g^*_n $ where arbitrary elements of $G^*$,
$$
u (X_1, \ldots,X_n ) = 1
$$
is a law in $G^*$ and therefore a law in $ \underset{=}{V} $; that is,
$$
u( X_1, \ldots, X_n) = 1
$$
is a law in $G$ in particular
$$
u(e_{i_{1}}, \ldots, e_{i_{n}} ) = 1.
$$
This proves $\varphi$ can be extended to  an epimorphism of $F^*$ onto
$G$. Hence the lemma. 

\begin{proofofthm*}[2]%Prf of thm 2
  We shall now prove that
  $$
  \underset{=}{C} = \underset{=}{V}. 
  $$
\end{proofofthm*}

Let $G$ be any group of $\underset{=}{V}$ and $E$ be a set of
generators of $E$, 
$$
G = gp(E).
$$

By Lemma \ref{chap7:sec1:thm1}, there is a group $G^* \in
\underset{=}{C} $ such 
that every law in $G^*$ is  a\pageoriginale law in $\underset{=}{V}$. We choose  an
index set $I$ with $|I| = |E|$. Then by Lemma \ref{chap7:sec1:thm2}, there is a subgroup
$F^* \leq G^{*^{G^{*^{I}}}}$ such that $G$ is an epimorphic images of
$F^*$. Now, since $\underset{=}{C}$ is closed under the operations
$Q,R,S$, we have 
$$
G^* ~\in ~\underset{=}{C};
$$
and therefore $G$, being an epimorphic image of $F^*$, is in
$\underset{=}{C}$ that is 
$$
\underset{=}{V}~ \subseteq ~\underset{=}{C}.
$$
combining this with the reversed inclusion which we have already\break
proved, we get 
$$
\underset{=}{C}~ =~ \underset{=}{V}.
$$

\setcounter{corollary}{0}
\begin{corollary}\label{chap7:sec1:coro1}%Corlry 1
  The group $F^*$ is a reduced free group, of rank $|I|$, of the
  variety $\underset{=}{V}$. 
\end{corollary}

\begin{corollary}\label{chap7:sec1:coro2}%Corlry 2
  Let the class $\underset{=}{G}$ consist of a single group $G_0$
  only, and let $G_0$ be finite. Then every reduced free group $F^*$
  of  finite rank  $d$ of the  least variety $\underset{=}{V}$
  containing  $G_0$ is finite,  and its order is bounded by 
  $$
  |F^*|\leq|G_0|^{|G_0|^{d}}
  $$
\end{corollary}

\begin{proof}
  Take $G_0$ as the $G^*$ of Lemma \ref{chap7:sec1:thm2} and $|I| = d
  $. By Corollary\pageoriginale 
  \ref{chap7:sec1:coro1}, the group $F^*$ is a reduced free group of
  rank $d$.   

  Further
  $$
  |F^*| \leq  |F_0|^{|G_0^{d}}
  $$
  
  Now, since $F^*$  is a finite group it has finite number of defining
  relations, say 
  $$
  u_i(\underbar{f}) = 1, ~ i=1, \ldots,n.
  $$
  We have already proved that 
  $$
  u_i(\underbar{X}) = 1, ~ i=1, \ldots,n
  $$
  are laws in $\underset{=}{V}$. Therefore every law of
  $\underset{=}{V}$ not involving more than $d$ variables is a
  consequence of these $n$ laws. In other words, the set of laws, not
  involving more than $d$ variables, where $d$ is an arbitrary
  positive integer, is ``finitely based''. Notice that  this does not
  prove that $\underset{=}{V}$ is finitely based. (See section 2,
  ch.5, p.67).  
\end{proof}

\begin{theorem}[P. Hall (unpublished)]\label{chap7:sec1:thm3} 
  Let $F = gp ( \big\{ f_i \big \}_{i
    \in I})$ be a group with the property that every mapping
  $\eta$ of $ \big \{ f_i \big \}_{i \in I} $ into $F$ can be
  extended to an endomorphism of $F$. Then $F$ is a reduced free group
  of rank $|I|$ of the least variety containing $F$.  
\end{theorem}

\begin{proof}
  Let
  $$
  u(f_{i_{1}}, \ldots, f_{i_{n}}) = 1,
  $$
  be\pageoriginale a relation in $F$. We assert that 
  $$
  u(X_1, \ldots,X_n) = 1
  $$
  is a law in $F$. Let $b_1, \ldots,b_n$ be arbitrary elements of
  $F$. Consider the mapping $\eta$ of $\big\{ f_i \big\}_{i
    \in I}$ into $F$ defined by 
  $$
  f^\eta_{i_{k}} = b_k, ~ k = 1,\ldots,n
  $$
  and arbitrarily otherwise, say
  $$
  f_i = f_i, ~ i \neq i_1, \ldots,i_n. 
  $$
\end{proof}

By the hypothesis  of the theorem, $\eta$ can be extended to an
endomorphism of $F$. which we also denote by  $\eta$. Now 
$$
u(b_1,\ldots,b_n) = u(f^\eta_{i_{1}},\ldots,f^\eta_{i_{n}}) = (u
(f_{i_{1}},\ldots,f_{i_{n}} ))^\eta = 1^\eta = 1. 
$$

As $b_1,\ldots,b_n$ were chosen arbitrarily in $F$,
$$
u(X_1,\ldots,X_n) = 1
$$
is a law in $F$.

It follows that $F$ is written as the factor group of a free group
with respect to a normal ("relation") subgroup $R$, then $R$ is verbal
in the free group $(cf$. Chapter $5)$. By Theorem $5,p.79 $ $F$ then
is a reduced free group of rank $|I|$ as claimed. 

\section{}\label{chap7:sec2} 

In\pageoriginale this section we shall construct new varieties out of
given varieties. 

Let $\underset{=}{C}, \underset{=}{D}$ be any two classes of
groups. We say that a group $G$ is a $\underset{=}{C}-$ by
$-\underset{=}{D}$ group if $G$ is an extension of a group $A
\in \underset{=}{C}$ by a group $B \in
\underset{=}{D}$. We define the class $\underset{=}{C}-$ by
$-\underset{=}{D}$ as the class of all such groups $G$. [ Thus e.g. a
  finite-by-abelian group  is one with a finite normal subgroup whose
  factor group is abelian.]  

Let $\underset{=}{U},\underset{=}{V}$ be two varieties defined by the
set of laws $M$ and $N$ respectively, where 
\begin{align*}
  M &= \big\{ u_i (\underbar{X}) = 1 \big \}_{i \in I} ~  \text{and} \\
  N &=  \big\{ v_j (\underbar{X}) = 1 \big\}_{j \in J}.
 \end{align*} 
 
 Without loss of generality we can assume that the set $X$ of variables
 is countable, say 
 $$
 X = \big\{ X_1, X_2,\ldots \big\}.
 $$

 We denote by $F$ the  free group on these variables, 
 $$
 F = gp(X, \phi ).
 $$

Our objective is to prove that the class $\underset{=}{U}-$ by
$-\underset{=}{V}$ is a variety. Let 
$$
\displaylines{\hfill 
  U = \big \{ u(\underbar{X})| u(\underbar{X}) = 1  \text{ a law in }
  \underset{=}{U} \big \}\hfill \cr  
  \text{and}\hfill  
  V = \big\{ v(\underbar{X})| v(\underbar{X}) = 1\text{ a law in
  }\underset{=}{V} \big\}.\hfill } 
$$

We\pageoriginale know that the groups $U, V$ are verbal subgroups of $F$; in fact
$U, V$ are the verbal subgroups generated by the left-hand sides of
$M$ and $N$ respectively. We shall also denote these left-hand sides
by $M$ and $N$ respectively. 

Let
$$
u (X_1,\ldots,X_m) = 1,
$$
be a law in $\underset{=}{U}$ and 
$$
v_i (X_1,\ldots,X_{n_{i}} ) = 1, i = 1,\ldots,m,
$$
be laws in $\underset{=}{V}$. Write 
\begin{multline*}
  w(\underbar{X})=u(\underbar{v}(\underbar{X}))=u(v_1
  (X_1,\ldots,X_{n_{1}} ), v_2 (X_{n_{1}+1},\cdots,X_{n_{1}+n_2}
  ),\\
  \ldots, v_m (X_{n_1+,\cdots + n_{m-1}+ 1}, \ldots, X_{n_{1}+\ldots
    +n_m} )). 
\end{multline*}

Let $L$ denote the set of all laws of the form $w(\underbar{X}) =
u(\underbar{v}(\underbar{X}) = 1$, with $u(\bar{X}) \in U, ~
v( \underbar{X}) \in V$. We also denote the set of all
left-hand sides of $L$ by $L$. Let $W$ be the verbal subgroup
generated by $L$ in $F$ and $\underbar{W}$ be the variety defined by
$L$. We shall use the notation 
$$
\displaylines{\hfill 
  W = U_\circ V\hfill \cr
  \text{and} \hfill 
  \underset{=}{W} = \underset{=}{U} \underset{=}{V}.\phantom{and}\hfill }
$$

We\pageoriginale shall now prove that 
$$
\underset{=}{U}- \text{ by } -\underset{=}{V} = ~ \underset{=}{U}
\underset{=}{V}. 
$$
If $H$ is any set of words in the variables $X_1, X_2,\ldots$ and $G$
any group, then we denote the verbal subgroup defined by $H$ in $G$ by
$G_H$. In particular 
$$
W = F_L = F_W, ~ U = F_U = F_{\big\{ u_i \big \} ~ i \in I}, V
= F_V = F_{\big\{ v_j \big\} ~ j \in J}. 
$$

Let $G$ be any group in the class $\underset{=}{U}-$ by
$-\underset{=}{V}$. Then there exist groups $A,B$, such that 
$$
A \Delta G, ~ G/A \simeq B, ~  \text{ with } ~ A \in
\underset{=}{U}, ~ B \in \underset{=}{V}; 
$$
that is, there is an epimorphism $\beta$ of $G$ onto  $B$ with $A$ as
its kernel. We assert that the verbal subgroup defined by $V$ in $G$,
namely $G_V$, is a subgroup of $A$. For consider 
$$
v (g_1,\ldots,g_n ) \text{ with } v(\underbar{X}) \in V,
g_1,\ldots,g_n \in G; 
$$
we have  $(v(g_1,\ldots,g_n ))^\beta = v(g^\beta_1,\ldots,g^\beta_n) = 1$,
since $B \in \underset{=}{V}$. 

Hence\pageoriginale 
$$
G_V ~\leq ~A. 
$$
Now if 
\begin{gather*}
  w(\underbar{X}) = u(\underbar{v}(\underbar{X})) ~ \in W, 
  \text{ where } \\ 
  \underbar{v}(\underbar{X}) = (v_1 (\underbar{X}), \ldots v_m
  (\underbar{X})), \text{ then }
\end{gather*}

$v_i(g) \in G_V \leq A $, with $g s$ belonging to $G$ and for
$i = 1,\ldots,m$. Since 
$$
u(X_1,\ldots,X_m) = 1 
$$
is a law in $\underset{=}{U}$ and hence in $A$, we have
\begin{align*}
  u(\underbar{v}(g)) &= 1 ; \text{ that is } \\
  u(\underbar{v}(\underbar{X})) &= 1
\end{align*}
is a law in $G$ in other words,
$$
G \in \underset{=}{W} = \underset{=}{U} ~ \underset{=}{V}.
$$

Hence 
$$
\underset{=}{U} - \text{ by } -\underset{=}{V} \subseteq \underset{=}{U}
\underset{=}{V}.  
$$

Conversely let $G$ be any group of the variety $\underset{=}{U} ~
\underset{=}{V}$. The verbal subgroup\pageoriginale $G_V$ is fully invariant and
hence  trivially normal in $G$. It is easy to verify that $G/G_V \in
\underset{=}{V}$.  
(This is in fact true for any group $G$.) We claim that  
$$
G_V~ \in ~\underset{=}{U};
$$
for let 
$$
u(X_1,\ldots,X_m ) = 1
$$
be any law in $\underset{=}{U}$ and $v_1(\underbar{g}),\ldots,v_m
(\underbar{g}) ~ \in G_V $; then 
$$
\displaylines{\hfill 
  u(\underbar{v}(\underbar{g})) =  u (v_1 (\underbar{g}), \ldots,v_m
  (\underbar{g})) = 1; \hfill \cr 
  \text{that is,}\hfill 
  u(X_1,\ldots,X_m ) = 1 \hfill }
$$
is a law in $G_V$.Hence
$$
\displaylines{\hfill G_V \in \underset{=}{U},\hfill \cr
\text{i.e.,} \hfill G \in \underset{=}{U}-~\text{ by }~
-\underset{=}{V}. \hfill }
$$

Therefore,
$$
\underset{=}{U} \underset{=}{V} \underline{\subset} \underset{=}{U} -
\text{ by } - \underset{=}{V}. 
$$

Combining this with the above  reversed inclusion we get
$$
\underset{=}{U}\underset{=}{V} = \underset{=}{U} - \text{ by } -
\underset{=}{V}. 
$$

This\pageoriginale proves that $\underset{=} U- by-\underset{=}V$ is a variety. In
the case of varieties we shall use the simpler notation and write
$\underset{=}U \underset{=}V$ instead of $\underset{=}U$ -by
$-\underset{=}V$. 

\begin{theorem}[Hanna Neumann, 1956]\label{chap7:sec2:thm4} %theo 4
  The multiplication of varieties is associative. 
\end{theorem}

\begin{proof}
  Let $\underset{=}{T}, \underset{=}{U}, \underset{=}{V}$ be three
  varieties defined by the set of laws $L, M$ and $N$
  respectively. The variety $\underset{=}{T} \underset{=}{U}$ is
  defined by all laws of the form 
  \begin{align*}
    w (\underbar{X}) & = t (\underbar{u} (\underbar{X})) = 1, \text{ where}\\
    t (\underbar{X}) & = 1 \text{ and } u (\underbar{X}) = 1
  \end{align*}
  are laws in $\underset{=}{T}$ and $\underset{=}{U}$
  respectively. Therefore the variety $(\underset{=}{T}
  \underset{=}{U}) \underset{=}{V}$ is defined by all laws of the form 
  $$
  w (\underbar{v} (\underbar{X})) = t (\underbar{u}(\underbar{v}
  (\underbar{X}))) = 1, \text{ where } 
  $$
  $\underbar{v} (\underbar{X}) = 1$ are laws in
  $\underset{=}{v}$. Similarly one can see that the variety
  $\underset{=}{T} (\underset{=}{U} \underset{=}{V})$ is also defined
  by all laws of the form 
  $$
  t (\underbar{u} (\underbar{v}(\underbar{X}))) = 1.
  $$

  This proves that 
  $$
  (\underset{=}{T} \underset{=}{U}) \underset{=}{v} = \underset{=}{T}
  (\underset{=}{U} \underset{=}{V}). 
  $$
\end{proof}

The above theorem can also be proved in the following way. We first
observe that if $\underset{=}{U}$ is any variety defined by a set of
laws $M$, then\pageoriginale 
$$
G \in \underset{=}{U}
$$ 
if and only if 
$$
G_M = \{1\}.
$$

Further if $\underset{=}{V}$ is any other variety defined by a set laws
$N$, then 
$$ 
G \in \underset{=}{U} \underset{=}{V},
$$
if and only if  
$$
(G_N)_M = 1.
$$

For if, $G \in \underset{=}{U} \underset{=}{V}$, then $G_N
\in \underset{=}{U}$. Hence 
$$
G \in (\underset{=}{T} \underset{=}{U}) \underset{=}{V}
$$
if and only if 
$$
((G_N)_M)_L = 1.
$$

Let the variety $\underset{=}{U} \underset{=}{V}$ be defined by
$P$. Then $G \in T(\underset{=}{U} \underset{=}{V})$ if and
only if 
$$
(G_P)_L =1.
$$

To prove the theorem we have only to prove that
$$
G_P = (G_N)_M.
$$

It\pageoriginale is easy to verify that $G_P$ is the least normal subgroup of $G$
such that $G/G_P \in \underset{=}{U} \underset{=}{V}$. Now, 
$$
((G/(G_N)_M)_N)_M = (G_N)_M /(G_N)_M =\{1\};
$$
that is to say 
$$
G/(G_N)_M \in \underset{=}{U} \underset{=}{V}.
$$

Further if $S$ is any normal subgroup of $G$ such that
\begin{gather*}
  G/S \in \underset{=}{U} \underset{=}{V}, \text{ then} \\
  ((G/S)_N)_M = \{ 1 \} ; \text {that is}\\
  (G_N)_M \le S.
\end{gather*}

Thus $(G_N)_M$ is the unique minimal normal subgroup of $G$ such that
$$
(G_N)_M \in \underset{=}{U} \underset{=}{V}.
$$

Therefore
$$
G_P = (G_N)_M.
$$

This proves the theorem.

The associative law does not hold for arbitrary classes of groups; in
other words if $\underset{=}{C}, \underset{=}{D}, \underset{=}{E}$ are
three classes of groups, then\pageoriginale in general 
$$
(\underset{=}{C} - \text{ by } -\underset{=}{D}) - \text{ by }
\underset{=}{E} \neq \underset{=}{C} -  \text{ by } (\underset{=}{D} -
\text{ by }-\underset{=}{E}). 
$$

Consider the following example. Let $\underset{=}{C}$ be the class of
all cyclic groups. Consider the normal series of $A_4$, 
\begin{align*}
  & \{1\} \triangle C \triangle B \triangle A_4,  \text{ where}\\
  & B =\{ 1, (12) (34), (13) (24), (14) (23)\},\\
  & C =  \{ 1,(12) (34)\}.
\end{align*}

The groups $A_4 /B, B/C$ and $C$ are cyclic groups; that is $A_4/B
\in \underset{=}{C}$ and 
\begin{gather*}
  B \in \underset{=}{C} - by -\underset{=}{C} ; \text{ thus}\\
  A_4 \in (\underset{=}{C} -by- \underset{=}{C}) -by - \underset{=}{C}.
\end{gather*}

But, $A_4 \notin \underset{=}{C} - \text{ by } - (\underset{=}{C} -by
-\underset{=}{C})$, as $A_4$ does not contain any cyclic normal
subgroup. 

Let $\underset{=}{U}$ be variety. We define $\underset{=}{U}^1 =
\underset{=}{U}, \underset{=}{U}^{n+1} = \underset{=}{U}^n
\underset{=}{U} = \underset{=}{U} U^n$. As the multiplication of
varieties is associative, $\underset{=}{U}^n$ is uniquely determined. 

Let $\underset{=}{A}$ be the variety of all abelian groups. We call
the variety $\underset{=}{A}^n$ the variety of \textit{soluble groups}
of length $n$. A group $G$ is \textit{soluble} if it is in
$\underset{=}{A}^n$ for some $n$. It is immediate that 
$$
\underset{=}{A}^1 \subseteq \underset{=}{A}^2 \subseteq
\underset{=}{A}^3 \subseteq \ldots, 
$$

It\pageoriginale is easy to verify that this definition is equivalent to the
following more usual definition which can be found in most text books
on group theory. 

A group $G$ is soluble if there exits a ``normal series''. 
$$
\{1\} = H_0 \triangle H_1 \triangle H_2 \triangle \cdots \triangle H_n = G
$$
with $H_{i+1}/ H_i$ abelian, for $n=0, 1,\ldots, n-1$. When $G$ is
finite and soluble, then $G$ has a series with the corresponding
factor groups cyclic. A (not necessarily finite) group $G$ is said to
be \textit{polycyclic} if it has a normal series with the
corresponding factor groups cyclic. Thus every finite soluble group is
a polycyclic group. Polycyclic groups were first studied by Hirsch who
called them S-groups. The term ``polycyclic '' is due to P. Hall who
introduced it as a part of a systematic terminology.  

The class of all soluble groups do not form a variety. One can prove
that for every integer $n$, there is a $G_n \in
\underset{=}{A}^n, G_n  \notin
\underset{=}{A}^{n-1}$. Consider the cartesian product $P$ of $G_n,
n=1,2 \ldots$. If $P$ were soluble, then $P \in
\underset{=}{A}^m$, for some $m$. Therefore every $G_n$, being an
epimorphic image of a subgroup of $P$ is in $\underset{=}{A}^m$. This
is absurd. Thus, the class of all soluble groups is not closed under
the operation of taking cartesian products and therefore does not from
a variety. 

We have already remarked in Chapter $1$ that the class of all fields
does not form a variety. To see this, it suffices to observe\pageoriginale that the
class of fields is not closed under the operation of forming cartesian
products: in fact one easily sees that the direct product of two
fields contains proper zero-divisions and thus cannot be a field.   

