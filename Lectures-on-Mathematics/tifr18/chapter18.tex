\chapter{Lecture 18}\label{chap18}

\setcounter{section}{31}
\section{Extensions of Algebraic Function Fields}\label{chap18:sec32}%Sec 32

In\pageoriginale the next five Lectures, we shall investigate the relations between
an algebraic function field and an extension of it (see below for
definition). In particular, we consider in the end the connection
between the $\zeta$-function of a function field and the
$\zeta$-function of a constant field extension. This gives in
particular the interesting result that for algebraic function fields
over a finite constant field, the smallest positive degree of a
divisor is 1. 

We start with the definition of an extension.

\begin{defi*}
  Let $K$ be an algebraic function field with constant field $k$. An
  \textit{extension} of $K$ an algebraic function field $L$ with
  constant field $l$ such that $L \supset K$ and $l \cap K=k$. 
\end{defi*}

Now let $L/l$ be an extension of $K/k$ and $\mathscr{K}$ a prime
divisor of $L$. If $v_\mathscr{K}(X)=0$ for all $X \in \mathscr{K}$
is said to be \textit{ variable over $K$} or \textit{trivial} on
$K$. If this were not true, the restriction of $v_\mathscr{K}$ to $K$
defines a valuation of $K$ trivial on $k$ and should therefore
correspond to a prime divisor $\mathscr{Y}$ of $K$. In this case,
$\mathscr{K}$ is said to be \textit{ fixed on $K$} and is said to lie
over the prime divisor $\mathscr{Y}$ of $K$. Also, since the
restriction of $v_\mathscr{K}$ to $K$ and $v_\mathscr{Y}$ are equivalent
valuations with values in $Z$, the latter having the whole of $Z$ for its
value group, there exist a positive integer $e_{L/K}(\mathscr{K})$
such that 
$$
v_\mathscr{K}(X)=e_{L/K}v_\mathscr{Y}(X) \text{ for all } X \in K
$$
$e_{L/K}(\mathscr{K})$\pageoriginale is called \textit{ ramification index} of
$\mathscr{K}$ over $K$. 

The relation between the residue fields of $\mathscr{K}$ and
$\mathscr{Y}$ is given by the following 

\begin{lemma*}%Lem
  If $L/l$ is an extension of $K/k$ and a prime divisor $\mathscr{K}$
  of $L$ lies over a prime divisor $\mathscr{Y}$ of $K$, the residue
  field $K_\mathscr{Y}$ of $\mathscr{Y}$ can be canonically imbedded
  in the residue field $L_\mathscr{K}$ of $\mathscr{Y}$ 
\end{lemma*}

\begin{proof}
  Let $\mathscr{O},\mathscr{K}$ denote respectively the valuation ring
  and maximal ideal of $\mathscr{K}$ and $\mathscr{O}$ and
  $\mathscr{Y}$ those of $\mathscr{Y}$. Since $v_\mathscr{K}$ and
  $v_\mathscr{Y}$ are equivalent on $K$, we clearly have
  $\mathscr{O}\supset \mathscr{Y}$ and $\mathscr{Y}=\mathscr{O} \cap
  \mathscr{K}$. Hence we have monomorphism
  $\mathscr{Y}=\mathscr{O}/\mathscr{Y} \to
  \mathscr{O}/\mathscr{K}=L_\mathscr{K}$ and $K_\mathscr{Y}$ can be
  considered as imbedded as a subfield of $L_\mathscr{K}$. 
\end{proof}

We now give condition for $L/K$ to be a finite or an arbitrary
algebraic extension. 

\begin{lemma*}%Lem
  Let $L/l$ be an extension of $K/k$. Then among the following
  statements, (1), (2) and (3) are equivalent and so are
  $(1')$, $(2')$ and $(3')$. 
\end{lemma*}
\noindent 
\begin{tabular}{lp{4cm}lp{4cm}}
  (1)& $[l:k]<\infty$ & $(1')$& $l$ is algebraic over $k$.\\
  (2)& $[L:K]< \infty$ & $(2')$ & $L$ is algebraic over $K$\\
  (3)& If $\mathscr{K}$ is any prime divisor of $L$ lying over the
  prime divisor\hfill\break   $\mathscr{Y}$ of $K[L_\mathscr{K}:K_\mathscr{Y}]<
  \infty$ &  (3') & If $\mathscr{K}$ is any prime of $L$ 
  lying over the prime divisor $\mathscr{Y}$ 
  of  $K,L _\mathscr{K}$ is algebraic over $K_\mathscr{Y}$ 
\end{tabular}

\begin{proof}
  We shall show that $(1)\Leftrightarrow (2)\Leftrightarrow(3)$. The
  proof that $(1)'\Leftrightarrow (2)'\Leftrightarrow (3)'$ is
  similar, and even simpler 
\end{proof}

The equation (valid even when either side is infinite)
$$
[L_\mathscr{K}:k]=[L_\mathscr{K}:K_\mathscr{Y}][K_\mathscr{Y}:k] =
[L_\mathscr{K}:l][l:k]  
$$
show\pageoriginale that $(1)\Leftrightarrow (3)$ since $[K_\mathscr{Y}:k]$ and
  $[L_\mathscr{Y}:l]$ are both finite. 

We shall now prove that $(1)\Leftrightarrow (2)$. Let $X$ be any
transcendental elements of $K$ over $k$. Since $X \notin k$ and $L
\cap k=l$, $X\notin l$ and is therefore transcendental over $1$. It
follows that $[K: k(X)]< \infty$ and $[L:l (X)]< \infty$, and from the
equalities 
$$
[L: k(X)] = [L:K][K:k(X)]=[L:l(X)][l(X):k(X)],
$$
it follows that $[L:K]< \infty \Leftrightarrow [l(X):k(X)]< \infty$. 

We shall now prove that $[l(X) : k(X)]=[l:k]$, which would finish the
proof of the theorem. 

Suppose $\alpha_1, \ldots, \alpha_n$ are $n$ linearly independent
element of $l$ over $K$. We assert that they are also linearly
independent over $k(X)$. For it there were a linear relation among
these with coefficients in $k(X)$ with at least one non-vanishing
coefficient, we may clearly assume that it is of the form 
$$
\sum^n_{i=1}\alpha_i g_i(X)=0
$$
at least one $g_i(X)$ with a non-zero constant term. Since $X$ is
transcendental over $l$, we may put $X=0$ in the above equation to
obtain a linear relation among the $\alpha_1$ over $k$ with at least
one non-zero co-efficient. But this is impossible since the $\alpha_i$
are linearly independent by a assumption. Hence, [$l(X); k(X)$] $\geq$
[$l : k$]. 

To\pageoriginale prove the reverse inequality, we may assume that $[l:k]<
\infty$. Hence, there exists a finite set $\beta_1, \ldots \beta_r$ of
element of $l$ such that $l=k(\beta_1 ,\ldots \beta_r)$. Then 
\begin{align*}
[l(X):k(X)]& =[k(X, \beta_1 ,\ldots, \beta_r):k(X)]\\
  &=[k(X), \beta_1, \ldots \beta_r]:k(X,\beta_1 ,\ldots,
  \beta_{r-1}).[k(X, \beta_1, \ldots, \beta_{r-1})]:\\ 
  & \qquad k[X,\beta_1
    ,\ldots,\beta_{r-2}] \ldots [k(X,\beta_1):k(X)]\\ 
  &\leq [k, \beta_1, \ldots \beta_r]:k(,\beta_1 ,\ldots,
  \beta_{r-1}).[k(, \beta_1, \ldots, \beta_{r-2})]:\\ 
  & \qquad k[,\beta_1 ,\ldots, \beta_{r-2}] \ldots [k(\beta_1):k(X)]
\end{align*}
since the degree of $\beta_1$ over $k(X, \beta_1 \cdots \beta_{i-1})$
is less than or equal to its degree over the smaller field $k(\beta_1
\cdots \beta_{i-1}), [k(\beta_1 \cdots \beta_r):k]=[l:k]$. 

The proof of the lemma is completed.

We shall call an extension $L/l$ over $K/k$ satisfying any of the
conditions (1), (2), (3) of lemma a finite extension. If $\mathscr{K}$
is a prime divisor of $L$ lying over a prime divisor $\mathscr{Y}$ of
$K$, the positive integer
$[L_\mathscr{K}:K_\mathscr{Y}]=d_{L/K}(\mathscr{K})$ is called the
relative degree of $\mathscr{K}$ over $K$. It follows from the proof
of the above lemma that 
\begin{align*}
  d_{L/K}(\mathscr{K})& =\frac{[L_\mathscr{K}:l][l:k]}{[K_\mathscr{Y}:k]}
  = \frac{d_L(\mathscr{K})}{d_K(\mathscr{Y})}[l:k],\\ 
  d_L(\mathscr{K})[l:k]& =d_{L/K}(\mathscr{K})d_K(\mathscr{Y})
\end{align*}
(The\pageoriginale suffix to $d$ indicated the field in which the degree is taken)

If $L/l$ is an algebraic extension of $K/k$, there does not exist any
prime divisor of $L$ which is variable over $K$. For, suppose $v$ is a
valuation on $L$ which is trivial on $K$. Any elements $\alpha \in L$
satisfies an irreducible equation 
$$
\displaylines{\hfill 
  \alpha^n+a_1 \alpha^{n-1}+ \cdots +a_n =0, a_i \in K,\hfill \cr
  \text{so that}  \hfill 0=v(a_n)=v(\alpha)+v(\alpha^{n-1}+ \quad
  +a_{n-1})\qquad \hfill }
$$
If $v(\alpha)>0$, we have
$$
v(\alpha^{n-1}+ \cdots + a_{n-1})= \min ((n-1)v(\alpha),(n-2)v(\alpha),
\quad 0)=0, 
$$
and we obtain a contradiction by substituting in the previous
equation. Thus, $v$ cannot be positive for any element of $L$, which
is impossible. 

Now, let $L/l$ be any extension of $K/k$. We shall prove that there
are at most a finite number of prime divisors of $L$ lying over a given
prime divisor $\mathscr{Y}$ of $K$, and that there is at least one. 

Let $g$ be the genus of $K$ and let $C$ be the class of the divisor
$\mathscr{Y}^{g+1}$ . Since $d(C)=d(\mathscr{Y}^{g+1})$, $\geq g+1$, we have 
$$
N(C)\geq d(C)-g+1 \geq 2,
$$
and there exists at least one more integral divisor $\mathscr{U}$ in
$C$. Then, $\mathscr{Y}^{g+1}$ $\mathscr{U}^{-1}=(X)_K$ where $X \in K$
and $X$ transcendental over $k$. Hence $X$ is also transcendental over
$l$ and the divisor $(X)_L$ has a decomposition
$(X)_L=\dfrac{\mathscr{K}_1^{a_1}\mathscr{K}_h^{a_h}}{\mathscr{N}_X}$,
where $h \geq 1$, and $a_i>0$. We\pageoriginale assert that
$\mathscr{K}_1,\ldots\mathscr{K}_n$ are precisely the divisors of $L$
lying over $\mathscr{Y}$. For, if $\mathscr{K}$ lies over
$\mathscr{Y},v_\mathscr{K}(X)>0$ and hence should be one of the
$\mathscr{K}_i$; and since $v_{\mathscr{K}_i}(X)>0$, the restriction
of $v_{\mathscr{K}_i}$ to $K$ should be a prime divisor occurring in
the numerator of $(X)_K$, and the only such prime divisor is
$\mathscr{Y}$. Our contention is proved. 

We now have the
\begin{theorem*}%Thm
  Suppose $L/l$ is an algebraic extension of $K/k$. Let $\mathscr{Y}$
  be a prime divisor of $K$ and $\mathscr{K}_1,\ldots \mathscr{K}_h$
  all the prime divisors of $L$ lying over $\mathscr{Y}$. Then, 
  $$
  [L:K]=\sum^h_{\nu =1}d_{L/K}(\mathscr{K}_\nu)e_{L/K}(\mathscr{K}_\nu)
  $$
\end{theorem*}

\begin{proof}
  Choose an element $X \in K$ as above. We have
  $$
  \displaylines{\hfill 
    (X)_k=\frac{\mathscr{Y}^t}{(\mathscr{N}_X)_K}\hfill \cr
    \text{and} \hfill (X)_L =
    \dfrac{(\mathfrak{z}_X)_L}{(\mathscr{N}_X)_L} =
    \dfrac{\mathscr{K}_1^{v_{\mathscr{K}_1}(X)}\mathscr{K}_2^{
    v_{\mathscr{K}_2} (X)} \ldots ~\mathscr{K}_h^{v_{\mathscr{K}h}(X)}}
          {(\mathscr{N}_ X)_L}\hfill }
$$ 
  
  Therefore we have 
  $$
  [L:l(X)]=d((\mathfrak{z}_X)_L)=\sum^h_{\nu=1}v_{\mathscr{K}_\nu}(X)d_L
  (\mathscr{K}_\nu)=v_\mathscr{Y} (X)\sum^h_{\nu=1}e_{L/K}
  (\mathscr{K}_\nu) d_L(\mathscr{K}_\nu) 
  $$
  and on the other hand  
  $$
  [K:k(X)]=d_K(\mathscr{Y}^t)=td_K (\mathscr{Y})=v_\mathscr{Y}(X)d_K
  (\mathscr{Y}). 
  $$
\end{proof} 

Hence,\pageoriginale\
\begin{align*}
  [L:K]& = \frac{[L:l(X)][l(X):k(X)]}{[K:k(X)]}=
  \frac{[l:k]}{d_K(\mathscr{Y})}\sum^h_{\nu=1}
  e_{L/K}(\mathscr{K}_\nu)d_L (\mathscr{K}_\nu)\\ 
  & =\sum_{\nu=1}^h e_{L/K}(\mathscr{K}_\nu)d_{L/k} (\mathscr{K}_\nu),
\end{align*}
which is the formula we want.

As corollaries, we deduce the inequalities
\begin{align*}
  h & \leq [L:K],\\
  d_{L/K}(\mathscr{K}_\nu)& \leq [L:K]\\
  \text{ and }\hspace{3cm} e_{L/K}(\mathscr{K}_\nu)& \leq
       [L:K].\hspace{3cm} 
\end{align*}
