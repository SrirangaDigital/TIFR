\chapter{Lecture 21}\label{chap21}

\setcounter{section}{35}
\section{Constant Field Extensions}\label{chap21:sec36}%Sec 36

An\pageoriginale extension $L/l$ of an algebraic function field $K/k$ is said to be
a \textit{constant field extension} If $L$ is the composite extension
$Kl$ of $K$ and $l$. 

The following question arises. Given an algebraic function field $K/k$
and an extension $l$ of $k$, is it possible to find a constant field
extension $L'/l'$ of $K/k$ such that $l$ is $k$-isomorphic to $l'$ ?
This is not possible in general, since the constant field of $L'=Kl'$
will in general be larger than the isomorphic image $l'$ of
$l^.$. More precisely, we have the following 

\begin{theorem*}%Thm
  Let $K/k$ be an algebraic function field and $l'_0$ an extension of
  $k$. Then there exists an algebraic function field $L/l$ which is an
  extension of $K/k$ with the following properties: 
  \begin{enumerate}[\rm (1)]
  \item there exists a subfield $l_0$ of $l$ containing $k$ and a
    $k$-isomorphism $\lambda : l_0 \to l'_0$ 
  \item $L = Kl_0$. 
  \end{enumerate}
\end{theorem*}

If $L^*/l^*$ is another extension of $K/k$ with a subfield $l^*_0$
of $l^*$ and a $k$-isomorphism $\lambda^* : l^*_0 \to l'_0$ having
the properties (1) and (2), there exists a $K$-isomorphism $\rho :
L^* \to L$ such that the restriction of $\rho$ to $l^*_0$ coincides
with the map $\lambda^{-1}_0 \lambda^*$ of $l^*_0$ onto $l_0$. 

$l$\pageoriginale is a purely inseparable finite extension of $l_0$.

\begin{proof}
  Construction of a composite field $L = Kl_0$.
\end{proof}

Let $\{u'_i\}$ be a transcendence basis of $l'_0$ over $k$. Take a $\{
u_i\}$ of independent transcendental elements $u_i$ over $K$ in
one-one correspondence $u_i \leftrightarrow u'_i$ with the set
$\{u'_i\}$ and let $\Omega$ be the algebraic closure of
$K(\{u_i\})$. Then we have an isomorphism $\lambda$ of $k(u'_i)$ onto
$k(u_i)$ trivial on $k$, defined by $\lambda u'_i = u_i. \lambda$ can
be extended to an isomorphism $\lambda$ of $l'_0$ onto subfield $l'_0$
of $\Omega$, and $\Omega$ contains the composite field $Kl_0 = L$. Let
$l$ be the algebraic closure of $l_0$ in $L$. 

If $X$ is a transcendental element of $K$ over $k, X$ is also
transcendental over $l_0$. For if it were not, there exists a finite
subset $(u_1, \ldots u_n)$ of the $u_i$ such that $X$ is algebraic
over $k(u_1, \ldots u_n)$. Hence there exists a relation of the form 
$$
f_o (u_1, \ldots u_n) X^r + \cdots \cdots + f_r (u_1 , \ldots u_n) =
0, f_i \in k[u_1 , \ldots u_n], 
$$
with at least one non-constant polynomial $f_i$. But this would imply
that the set $(u_1, \ldots u_n)$ is algebraically related over $K$, a
contradiction. 

Also, since $L = Kl_0$, and $l_0 \supset k$,
$$
[L : l_0 (X)] \le [ K : k(X) ] < \infty,
$$
and therefore $L/l$ is an algebraic function field with constant
field $l \supset l_0$. The conditions $(1)$ and $(2)$ are evidently
fulfilled. Moreover, since,\pageoriginale $X$ is transcendental over $l$, we have 
$$
[l : l_0] = [l(X) : l_0 (X)] \le [L : l_0 (X)] < \infty
$$

Only the second part of the theorem asserting uniqueness upto
isomorphism and the last part asserting that $l$ is purely
in-separable over $l_0$ remain to be proved. 

Suppose $L^*/l^*$ is another extension of $K/k$ satisfying the
conditions of the theorem. We have to set up an isomorphism $\rho :
L^* \to L$ such that $\rho$ fixes the elements of $K$ and $\rho$
restricted to $l^*_0$ is the isomorphism $\lambda_1 = \lambda^{-1}_0 \lambda^*$
of $l^*_0$ onto $l_0$. Since any element of $L^* = Kl_0^*$ can be
written in the form 
$$
\frac{\sum k_i l^*_i}{\sum k'_j l'^{*}_j} k_i, k'_j \in K, l^*_i,
l'^{*}_j \in l^*_0, 
$$
we should necessarily have
$$
\rho 
\begin{pmatrix} 
  \frac{\sum k_i l^*_i}{\sum k'_j l^{' *}_j}
\end{pmatrix}
= \frac{\sum k_i \lambda_1 (l^*_i)}{\sum k'_j \lambda_1 (l'^{*}_j)}
$$

We make this the definition of $\rho$. In order to prove that it is
well defined, we have to verify that if $0$ has the representation
$\sum k_i l^*_i$, then $\sum k_i \lambda_1 (l^*_i) = 0$. To prove that
the map is an isomorphism (and also to prove that the denominator of
the right side does not vanish when $\sum k'_j l^{' *}_j \neq 0)$ we
have to prove that $\sum k_i \lambda (l_i^*) = 0 \Rightarrow \sum k_i
l^*_i = 0$. 

Thus,\pageoriginale we have set up the required isomorphism provided we can prove that
$$
\sum k_i l_i^* = 0 \Leftrightarrow \sum k_i \lambda_1 (l^*_i) = 0. 
$$

But since these expressions involve only a finite number of elements
of $l_0$ and $l^*_0$, we may assume that $l_0$ (and consequently
$l_0^*)$ is finitely generated over $k$. 

A simple argument shows that to prove the pure inseparability of $l$
over $l_0$ we may also assume that $l_0$ is finitely generated over
$k$. 

First assume that $l_0$ is a purely transcendental  extension $k(u_1,
\ldots u_n)$ of $k$. Then $l_0^* = k(u_1^*, \ldots u_n^*)$, where
$u^*_i = \lambda_1^{-1} (u_i)$. Then $u_1, \ldots u_n$ are
algebraically independent over $K$, and so are $u_1^* , \ldots ,
u_n^*$. Hence there exists an isomorphism $\rho : L^* = K(u_1^* ,
\ldots, u^*_n) \to K(u_1 , \ldots u_n) = L$. In this case, the constant
field $l$ coincides with $l_0 = k(u_1 , \ldots, u_n)$. This follows
from the following more general 
\begin{lemma*}
  If $A$ is a field which is algebraically closed in another field
  $B$, and if $X_1, \ldots , X_n$ is a set of algebraically
  independent elements over $B, A(X_1 , \ldots X_n)$ is algebraically
  closed in $B(X_1, \ldots , X_n)$. 
\end{lemma*}

\begin{proof}
  We may clearly assume that $n = 1, X_1 =  X$.
\end{proof}

Let $\alpha = \alpha_0 \dfrac{f(X)}{g(X)} $ be any element of $B(X)$,
where $\alpha_0 \neq 0$ is an element of $B$ and $f(X)$ and $g(X)$ are
coprime polynomials over\pageoriginale $B$ with leading coefficients $1$. If
$\alpha$ is algebraic over $A(X)$, we have 
$$
\displaylines{
  \varphi_r (X) \alpha^r + \cdots \cdots \cdots + \varphi_o (X) = 0,
  \varphi _i (X) \in A[X],\cr 
  \varphi_o , \ldots, \varphi_r \quad \text{ coprime}\cr
  \text{i.e.,}\hfill\varphi_r (X) \alpha_o^r f^r (X) + \cdots \cdots + \varphi_o
  (X) g^r (X) = 0.\hfill } 
$$
 
 Let $\xi$ be any root of $f(X)$. Substituting $X = \xi$ in the above
 equation (which we may do since $X$ is transcendental over $B$) we
 obtain 
 $$
 \varphi_o (\xi) g^r (\xi) = 0
 $$
 and since $g(\xi ) \neq 0, f$ and $g$ being coprime,
 $$
 \varphi_0 (\xi) = 0
 $$
 and so $\xi $ is algebraic over $A$. Since every root of $f$ is
 algebraic over $A$ and $f(X)$ has leading coefficient $1$, the
 coefficients of $f$ are algebraic over $A$ and hence lie in
 $A$. Similarly, $g$ is also a polynomial over $A$. Substituting for
 $X$ a root $\delta$ of $f(X) - g(X)$, we get, since $f(\delta) =
 g(\delta) \neq 0, \varphi_r (\delta) \alpha_0^r + \cdots + \varphi_0
 (\delta) = 0$, hence $\alpha_0$ is algebraic over $A$ and therefore
 in $A$, since not all $\varphi_i (\delta) = 0 (\varphi_0 , \ldots
 \varphi_r$ being coprime). Our lemma is proved. 
  
 We are therefore left with the case when $l_0$ is a finite algebraic
 extension of $k$. Then we have $l_0 = k(\alpha_1 , \ldots ,
 \alpha_m)$. We\pageoriginale use induction on $m$. The result is trivial when $m =
 0$. 
 
 Suppose the result holds for $m - 1$ in the place of $m$. Put $k_1 =
 k(\alpha_1, \ldots ,\alpha_{m - 1})$ and $K_1 = K k_1 =  K(\alpha_1 ,
 \ldots , \alpha_{m-1})$. Let $\alpha^*_i = \lambda^{-1}_1 (\alpha_i)$,\break
$k_1^* = k^* (\alpha^*_1 \cdots \alpha^*_{m-1})$ and $K_1^* = K k_1^* =
 K (\alpha^*_1, \ldots \alpha^*_{m-1})$. Let $l_1$ and $l_1^*$ be the
 algebraic closures of $k_1$ and $k^*_1$ in $K_1$ and $K^*_1$
 respectively. By our induction hypothesis, we have 
 \begin{enumerate} [(1)]
\item an isomorphism $\rho_1: K^*_1 \to K_1$ such that $\rho_1$ when
  restricted to $k^*_1$ coincides with the restriction of $\lambda_1$
  to $k^*_1$, and  
\item $l_1$ and $l^*_1$ are purely inseparable extensions of $k_1$ and
  $k^*_1$ respectively. 
 \end{enumerate} 
 
 Put $\alpha_m = \alpha$ and $\alpha^*_m = \lambda^{-1}_1 (\alpha_m) =
 \alpha^*$. Then, $L = K_1 (\alpha)$ and $L^* = K^*_1 (\alpha^*)$. We
 would be through if we can extend the isomorphism $\rho_1$ to an
 isomorphism $\rho : L^* \to L$ such that $\rho (\alpha^*)= \lambda_1
 (\alpha^*)$ and if we prove that the constant field $l$ of $L$ is
 purely inseparable over $l_0$. 
 
 To prove that we can extend the isomorphism $\rho_1$ to $\rho$, it is
 necessary and sufficient to show that if $F^* (X)$ is the irreducible
 polynomial of $\alpha^*$ over $K^*_1, \rho_1 F^*(X)$ is the
 irreducible polynomial of $\alpha$ over $K_1$. 
 
 Assume that $F^* (X)$ has leading coefficient $1$. Since one of its
 roots is algebraic over $k^*_1$, all its roots are algebraic over
 $k^*_1$ and $F^* (X)$ is therefore a polynomial with coefficients in
 the algebraic closure, $l^*_1$ of $k^*_1$ in $K^*_1$. Also, since
 $l^*_1$ is purely inseparable\pageoriginale over $k^*_1$, the irreducible
 polynomial of $\alpha^*$ over $k^*_1$ is a certain power of $F^*(X)$
 of the form $(F^* (X))^{p^t}$, $t \ge 0$. Since $\lambda_1$ is an
 isomorphism of $l^*_o = k^*_1 (\alpha^*)$ onto $l_o = k_1(\alpha)$
 with $\lambda_1 (\alpha^*) = \alpha$, we deduce that $\lambda_1 (F^*
 (X))^{p^t}= \rho_1(F^*(X))^{p^t}$ is the irreducible polynomial of
 $\alpha$ over $k_1$. 
 
 Again, since $\rho_1$ maps $k^*_1$ onto $k_1$, it maps the algebraic
 closure $l^*_1$ of $k^*_1$ in $K^*_1$ onto the algebraic closure
 $l_1$ of $k_1$ in $K_1$. This proves that the irreducible equation of
 $\alpha$ over $K_1$ (or what is the same, $l_1$) with leading
 coefficient $1$ is equal to $\rho_1 F^* (X)$ since $\rho_1 F^{*} (X)$
 is obviously the only irreducible factor of $\rho_1 (F^*(X))^{p^t}$
 over $l_1$. 
 
 Hence $\rho_1$ can be extended to an isomorphism $\rho$ having the
 requisite properties. 
 
 To prove that $l$ is purely inseparable over $l_0$, notice that since
 $l_1$ is purely inseparable over $k_1, l_1 (\alpha)$ is purely
 in-separable over $k_1(\alpha) = l_0$. It is therefore sufficient to
 prove that $l$ is purely inseparable over $l_1 (\alpha)$. 
 
 Now since $l_1$ is algebraically closed in $K_1$, the irreducible
 polynomial of $\alpha$ over $K_1$ with leading coefficient $l$
 coincides with its irreducible polynomial over $l_1$. Therefore we
 have 
 
\medskip
\noindent 
$\begin{aligned}
   \big [K_1 (\alpha) : K_1 \big ] & = \big [ l_1 (\alpha) : l_1 \big]\\
   \text{and similarly} \qquad  
   \big [ l_1 (\alpha, X) : l_1 (X) \big ] & = \big [ l_1 (\alpha) : l_1
     \big ].
\end{aligned}
 $\medskip
 
 From\pageoriginale these two equalities and the following one
 \begin{multline*}
 \big [ K_1 (\alpha) : l_1 (X) \big] = \big [ K_1 (\alpha) : K_1 \big
 ] \big [ K_1 : l_1 (X) \big ]\\ 
 = \big [K_1 (\alpha) : l_1 (\alpha, X)
   \big ] \big [l_1 (\alpha, X) : l_1 (X) \big]. 
 \end{multline*}
we deduce that
$$
\big [K_1 : l_1 (X) \big ] = \big [K_1 (\alpha) : l_1 (\alpha, X) \big ].
$$

Now, let $\beta$ be a constant of $K_1 (\alpha)$. Then there exists an
integer $t \ge 1$ such that $\beta^{p^t}$ is separably algebraic over
$l_1 (\alpha)$. By a well-known theorem, the extension $l_1 (\alpha,
\beta^{p^t})$ is a simple extension $l_1 (\gamma)$ of $l_1$. We have 
$$
\big [K_1 (\alpha) : l_1 (\alpha, \beta^{p^t}, X) \big ] = \big [K_1
  (\gamma) : l_1 (\gamma, X) \big] = \big [ K_1 : l_1 (X) \big] 
$$
by an argument which is familiar to us, and using our previous 
equality, we get   
$$
\big [ K_1 (\alpha) : l_1 (\alpha, \beta^{p^t}, X) \big ] = \big [K_1
  (\alpha) : l_1 (\alpha, X) \big] 
$$
and hence $l_1 (\alpha, \beta^{p^t}, X) = l_1 (\alpha, X)$ and
$\beta^{p^t} \in l_1 (\alpha, X)$. Since $\beta^{p^t} $ is algebraic
over $l_1 (\alpha)$ which is algebraically closed in $l_1 (\alpha, X),
\beta^{p^t} \in l_1 (\alpha)$ and $\beta$ is purely inseparable over
$l_1 (\alpha)$. 

Our theorem is completely proved. We shall give an example where $l
\neq l_0$. Let $k_0$ be a field of characteristic $p > 0$ and $u$ and
$v$ two algebraically independent elements over $k_0$. Let $k = k_0
(u, v)$ and $X$ a variable over $k$. Put $K = k (X, Y)$ where $Y$
satisfies\pageoriginale the equation $Y^p = u X^p + v$. We shall show that the
constant field is $k$. If it were not, let $k^1$ be the constant
field. Since $K = k(X, Y)$ is of degree $l$ or $p$ over $k(X)$ and since
$\big [k^1 (X) : k(X) \big]= \big [k^1 : k \big] > 1$, we deduce that
$K = k^1 (X)$. Hence $Y = u^{1/p} X + v^{1/p} \in k^1 (X)$. But since
$X$ is transcendental over $k $, (and hence also over
$k (u^{1/p}, v^{1/p})$, we deduce that $u^{1/p}$ and $v^{1/p} $ are
both in $k^1$. Hence  
$$
\big [ k^1 : k \big] \ge \big [k(u^{1/p}, v ^{1/p}) : k \big] = \big
     [k(u^{1/p}, v^{1/p}) : k (u^{1/p}) \big] \big [ k(u^{1/p}) : k
       \big] = p^2, 
$$
while on the other hand
$$
\big [ k^1 : k \big] = \big [k^1 (X) : k(X) \big ] \le \big [ K : k(X)
  \big] \le 
$$
which is a contradiction.

Now, take $l_0 = k(v^{1/p})$. Then $K l_0$ clearly contains the element
$\dfrac{Y - v^{1/p}}{X} = u^{1/p}$ and hence $l =  k(u^{1/p},
v^{1/p}), l \neq l_o$ and $\big [l : l_o \big] = p$. 
