
\chapter{Lecture 4}\label{chap4}

\setcounter{section}{5}
\section{Valuations of Algebraic Function Fields}\label{chap4:sec6}

It\pageoriginale is our purpose in this paragraph to prove that all valuations of an
algebraic function field $K$ which are trivial on the constant field
$k$ are discrete. Henceforward, when we talk of valuations or places
of an algebraic function field $K$, we shall only mean those which are
trivial on $k$. Valuations will always be written additively. We
require some lemmas. 

\begin{Lemma}\label{chap4:sec6:lem1}%lem 1
  Let $K/L$ be a finite algebraic extension of degree $[K : L]=n$ and
  let $v$ be a valuation on $K$ with valuation group $V$. If $V_\circ$
  denotes the subgroup of $V$ which is the image of $L^*$ under $v$ and
  $m$ the index of $V_\circ$ in $V$, we have $m \leq n$. 
\end{Lemma}

\begin{proof}
  It is enough to prove that of any $n+1$ elements $\alpha_1, \ldots
  \alpha_{n+1}$ of $V$, at least two lie in the same coset modulo
  $V_\circ$. 
\end{proof}

Choose $a_i \in K$ such that $v(a_i) = \alpha_i \,(i=1, \ldots
n+1)$. Since there can be at most $n$ linearly independent elements of
$K$ over $L$, we should have 
$$
\sum^{n+1}_{i=1} l_i a_i = 0, l_i \in L, \text{ not all } l_i \text{
  being zero }. 
$$

This implies that $v(l_i a_i) = y(l_j a_j)$ for some $i$ and $j, i
\neq j $ (see Lecture \ref{chap1}, \S\ \ref{chap3}). Hence, we deduce that 
\begin{align*}
  v(l_i) + v(a_i) & = v(l_i a_i) = v(l_j a_j) = v(l_j) + v(a_j),\\
  \alpha_i - \alpha_j & = v(a_i) -v(a_j) = v(l_j) -v(l_i) \in
  V_\circ 
\end{align*}
and\pageoriginale $\alpha_i$ and $\alpha_j$ are in the same coset modulo
$V_\circ$. Our lemma is proved. 

Let $V$ be an ordered abelian group. We shall say that $V$ is
\textit{archime\-dean} if for any pair of elements $\alpha, \beta$ in
$V$ with $\alpha > 0$, there corresponds an integer $n$ such that $n
\alpha > \beta$. We shall call any valuation with value group
archimedean an \textit{archimedean valuation}. 

\begin{Lemma}\label{chap4:sec6:lem2}%lem 2
  An ordered group $V$ is isomorphic to $Z$ if and only if $(i)$ it is
  archimedean and $(ii)$ there exists an element $\xi > 0$ in $V$ such
  that it is the least positive element; i.e., $\alpha > 0 \Rightarrow
  \alpha \geq \xi$. 
\end{Lemma}

\begin{proof}
  The necessity is evident. Now, let $\alpha$ be any element of $V$.
\end{proof}

Then by assumption, there exists a smallest integer $n$ such that $n
\xi \leq \alpha < (n+1) \xi$. Thus, 
$$
0 \leq \alpha -n \xi < (n+1) \xi -n \xi = \xi,
$$
and since $\xi$ is the least positive element, we have $\alpha
=n\xi$. The mapping $\alpha \in V \to n \in Z$ is clearly an order
preserving isomorphism, and the lemma is proved. 

\begin{Lemma}\label{chap4:sec6:lem3}%lem 3
  If a subgroup $V_\circ$ of finite index of an ordered group $V$ is
  isomorphic to $Z, V$ is itself isomorphic to $Z$. 
\end{Lemma}

\begin{proof}
  Let the index be $\bigg [ V : V_\circ \bigg] =n$. Let $\alpha, \beta$
  be any two elements of $V$, with $\alpha > 0$. Then $n \alpha$ and
  $n \beta$ are in $V_\circ, n \,\alpha > 0$. 
\end{proof}

Since $V_\circ$ is archimedean, there exists an integer $m$ such that
$m n \alpha > n \beta$, from which it follows that $m \alpha >
\beta$. 

Again,\pageoriginale consider the set of positive elements $\alpha$ in $V$. Then $n
\alpha$ are in $V_\circ$ and are positive, and hence contain a least
element $n \xi$ (since there is an order preserving isomorphism
between $V_\circ$ and $Z$). Clearly, $\xi$ is then the least positive
element of $V$. 

$V$ is therefore isomorphic to $Z$, by Lemma~\ref{chap4:sec6:lem2}.

We finally have the
\begin{theorem*}
  All valuations of an algebraic function field $K$ are discrete.
\end{theorem*}

\begin{proof}
  If $X$ is any transcendental element of $K/_k$, the degree $\bigg[ K
    : k(X)\bigg]\break < \infty$. Since we know that all valuations of
  $k(X)$ are discrete, our result by applying Lemma~\ref{chap4:sec6:lem1} and
  Lemma~\ref{chap4:sec6:lem3}. 
\end{proof}

\section{The Degree of a Place} \label{chap4:sec7}%sce 7

Let $\mathscr{Y}$ be a place of an algebraic function field $K$ with
constant field $k$ onto the field $k_{\mathscr{Y}} \cup \infty$. Since
$\mathscr{Y}$ is an isomorphism when restricted to $k$, we may assume
that $k_\mathscr{Y}$ is an extension of $k$. We shall moreover assume
that $k_\mathscr{Y}$ is the quotient
$\mathscr{O}_\mathscr{Y}/\mathscr{M}_\mathscr{Y}$ where
$\mathscr{O}_\mathscr{Y}$ is the ring of the place $\mathscr{Y}$ and
$\mathscr{M}_\mathscr{Y}$ the maximal ideal. We now prove the 
\begin{theorem*}
  Let $\mathscr{Y}$ be a place of an algebraic function field. Then
  $k_{\mathscr{Y}/k}$ is an algebraic extension of finite degree. 
\end{theorem*}

\begin{proof}
  Choose an element $X \neq 0$ in $K$ such that $\mathscr{Y} (X) =
  0$. Then $X$ should be transcendental, since $\mathscr{Y}$ is
  trivial on the field of constants. Let $\bigg[K : k(X)\bigg] =n <
  \infty$. Let $\alpha_1,\ldots \alpha_{n+1}$ be any\pageoriginale $(n+1)$ elements
  of $k_\mathscr{Y}$. Then, we should have $\alpha_i = \mathscr{Y}
  (a_i)$ for some $a_i \in K(i =1, \ldots n+1)$. There therefore exist
  polynomials $f_i(X)$ in $k[X]$ such that 

  $\sum \limits^{n+1}_{i=1}f_i(X) a_i = 0$, not all $f_i(X)$ having
  constant term zero. 
\end{proof}

Writing $f_i(X)=l_i + Xg_i (X)$, we have
$$
\sum^{n+1}_{i=1} l_i a_i = -X \sum^{n+1}_{i=1} a_i g_i(X), \text{ and taking }
$$
the $\mathscr{Y}$-image, $\sum \limits^{n+1}_{i=1} l_i \alpha_i =
-\mathscr{Y} (X) \sum \limits^{n+1}_{i=1} a_i g_i (\mathscr{Y}X) = 0,
l_i \in k$, not all $l_i$ being zero. 

Thus, we deduce that the degree of $ k_\mathscr{Y}/_k$ is at most $n$.

The degree $f_\mathscr{Y}$ of $k_\mathscr{Y}$ over $k$ is called the
\textit{degree of the place} $\mathscr{Y}$. Note that $f_\mathscr{Y}$
is always $\geq 1$. If the constant field is algebraically closed
($e.g$. in the case of the complex number field), $f_\mathscr{Y} = 1$,
since $k_\mathscr{Y}$, being an algebraic extension of $k$, should
coincide with $k$. 

Finally, we shall make a few remarks concerning notation.

If $\mathscr{Y}$ is a place of an algebraic function field, we shall
denote the corresponding valuation with values in $Z$ by
$v_\mathscr{Y}$ ($v_\mathscr{Y}$ is said to be a normed valuation at
the place $\mathscr{Y}$). The ring of the place shall be denoted by
$\mathscr{O}_\mathscr{Y}$ and its unique maximal ideal by
$\mathscr{Y}$. (This is not likely to cause any confusion). 

\section{Independence of Valuations}\label{chap4:sec8}%sec 8

In\pageoriginale this section, we shall prove certain extremely useful result on
valuations of an arbitrary field $K$. 

%%% \text \textcolonmonetary

\begin{theorem*}
  Let $K$ be an arbitrary field and $v_i (i=1,\ldots n)$ a set of
  valuations on $K$ with valuation rings $\mathscr{O}_i$ such that
  $\mathscr{O}_i \nsubset  \mathscr{O}_j$ if $i
  \neq j$. There is then an element $X \in K$ such that $v_1 (X) \geq
  0$, $v_i (X) < 0\, (i=1,\ldots n)$. 
\end{theorem*}

\begin{proof}
  We shall use induction. If $n=2$, since $\mathscr{O}_1 
  \nsubset \mathscr{O}_2$, there is an $X \in \mathscr{O}_1,
  X \notin \mathscr{O}_2$, and this $X$ satisfies the required
  conditions. Suppose now that the theorem is true for $n-1$ instead
  of $n$. Then there exists a $Y \in K$ such that 
  $$
  v_1 (Y) \geq 0, v_i(Y) < 0 \,(i=2, \ldots n-1).
  $$
\end{proof}

Since $\mathscr{O}_1 nsubset \mathscr{O}_n$, we can
find $a\, Z \in K$ such that 
$$
v_1 (Z) \geq 0, v_n (Z) < 0
$$

Let $m$ be a positive integer. Put $X = Y + Z^m$. Then 
$$
v_1(Y + Z^m) \geq \min (v_1 (Y), mv_1 (Z)) \geq 0
$$

Now suppose $r$ is one of the integers $2, 3, \ldots n$. If $v_r (Z)
\geq 0$, $r$ cannot be $n$, and since $v_r(Y)< 0$, we have 
$$
v_r (Y+Z^m) = v_r (Y) < 0.
$$

If $v_r (Z) < 0$ and $v_r(Y + Z^{m_r}) \geq 0$ for some $m_r$, for $m
> m_r$ we\pageoriginale have  
$$
v_r(Y+Z^m) = v_r(Y+Z^{mr} + Z^m - Z^{m_r}) = \min (v_r (Y + Z^{m_r}),
v_r (Z^m - Z^{m_r})), 
$$
and $v_r (Z^m-Z^{m_r}) = v_r (Z^{m_r}) + v_r(1-Z^{m-m_r}) = m_r v_r
(Z) < 0$, 

Since $v_r (1 - Z^{m-mr}) = v_r (1) =0$

Thus, $v_r (Y+Z^m) < 0$ for large enough $m$ in any case. Hence $X$
satisfies the required conditions. 

If we assume that the valuations $v_i$ of the theorem are archimedean,
then the hypothesis that $\mathscr{O}_i \nsubset
\mathscr{O}_j$ for $i \neq j$ can be replaced by the weaker one that
the valuations are inequivalent (which simply states that $\mathscr{O}_i \neq
\mathscr{O}_j$ for $i \neq j$). 

To prove this, we have only to show that if $v$ and $v^1$ are two
archime\-dean valuations such that the corresponding valuation rings
$\mathscr{O}$ and $\mathscr{O}^1$ satisfy $\mathscr{O} \supset
\mathscr{O}^1$, then $v$ and $v^1$ are equivalent. For, consider an
element $a \in K^*$ such that $v(a) > 0$. Then, $v\left(\dfrac{1}{a}\right) < 0$,
and consequently $\dfrac{1}{a}$ is not in $\mathscr{O}$, and hence not
in $\mathscr{O}^1$. Thus, $v^1\left(\dfrac{1}{a}\right) < 0 , v^1 (a) >
0$. Conversely, suppose $a \in K^*$ and $v^1 (a) > 0$. Then by
assumption, $v(a) \geq 0$. Suppose now that $v(a)=0$. Find $b \in K^*$
such that $v(b) < 0$. If $n$ is any positive integer, we have $v(a^nb)
= nv(a) + v(b) < 0$, $a_n b \notin \mathscr{O}$. 

But since $v^1$ is archimedean and $v^1(a) > 0$, for large enough $n$
we have 
$$
v^1 (a^nb) = nv^1(a) + v^1(b) > 0~,a^nb \in \mathscr{O}^1.
$$\pageoriginale\

This contradicts our assumption that $\mathscr{O}^1 \subset
\mathscr{O}$, and thus, $v(a) > 0$. Hence $v$ and $v^1$ are
equivalent. Under the assumption that the $v_i$ archimedean, we can
replace in the theorem above the first inequality $V_1 (X) \geq 0$
even by the strict inequality $v_1 (X) > 0$. To prove this let $X_1
\in K^*$ satisfy $v_1(X_1) \geq 0$, $v_i (X_1) < 0 , i > 1$. Let $Y$
be an element in $K^*$ with $v_1 (Y) > 0$. Then, if $X = X_1 {^mY}$,
where $m$ is a sufficiently large positive integer, we have 
\begin{align*}
  v_1 (X) & = v_1(X_1^mY) = mv_1(X_1) + v_1(Y) > 0,\\
  v_i (X) & = v_i(X_1^mY) = mv_i(X_1) + v_i(Y) < 0,~i=2, \ldots , n.
\end{align*}

We shall hence forward assume that all valuations considered are
archimedean. To get the strongest form of our theorem, we need two
lemmas. 

\setcounter{Lemma}{0}
\begin{Lemma}\label{chap4:sec8:lem1}%lem  1
  If $v_i \,(i=1, \ldots n)$ are inequivalent archimedean valuations,
  and $\rho_i$ are elements of the corresponding valuations group, we
  can find $X_i (i=1, \ldots n)$ in $K$ such that $v_i(X_i -1) > \rho
  _i, v_j(X_i) > \rho_j , i \neq j $ 
\end{Lemma}

\begin{proof}
  Choose $Y_i \in K$ such that
  $$
  v_i(Y_i) > 0, v_j(Y_i) < 0 \text{ for } j \neq i.
  $$
  Put $X_i = \dfrac{1}{1+Y^m_i}$. Then, if $m$ is chosen large enough,
  we have (since the valuation are archimedean) 
  $$
  v_j(X_i) = -v_j(1+Y^m_i) = -mv_j(Y_i) > \rho _j, \quad i \neq j
  $$\pageoriginale\
  and $v_i (X_i -1) = v_i (\dfrac{-Y^m_i}{1+Y^m_i}) = mv_i (Y_i) - v_i
  (1+Y^m_i) = mv_i (Y_i) > \rho_i$. since $v-i(1+Y^m_i)=0$. 
\end{proof}

A set of valuations $v_i (i=1, \ldots n)$ are said to be independent
if given any set of elements $a_i \in K$ and any set of elements
$\rho_i$ in the respective valuation groups of $v_i$, we can find an
$X \in K$ such that 
$$
v_i (X - a_i) > \rho_i.
$$

We then have the following
\begin{Lemma}\label{chap4:sec8:lem2}
  Any finite set of inequivalent archimedean valuations are independent.
\end{Lemma}

\begin{proof}
  Suppose $v_i(i=1, \ldots n)$ is a given set of inequivalent
  archimedean valuations. If $a_i \in K$ and $\rho_i$ are elements of
  the valuation group of the $v_i$, put $\sigma_i = \rho_i - \min
  \limits^n_{j=1} v_i(a_j)$. Choose $X_i$ as in Lemma~\ref{chap4:sec8:lem1} for the
  $v_i$ and $\sigma_i$. Put $X = \sum \limits^{n}_{1} a_i X_i$. Then 
  $$
  v_i (X-a_i) = v_i \left(\sum_{j \neq i} a_j X_j + a_i (X_i -1)\right) >
  \sigma_i + \min^n_{j=1}v_i (a_j) = \rho_i, 
  $$
  and our lemma is proved.
\end{proof}

Finally, we have the following theorem, which we shall refer to in
future as the theorem of independence of valuations. 

\begin{theorem*}
  If $v_i (i =1, \ldots n)$ are inequivalent archimedean valuations,
  $\rho_i$ is an element of the value group of $v_i$ for every $i$,
  and $a_i$ are given\pageoriginale elements of the field, there exists an element
  $X$ of the field such that 
  $$
  v_i (X - a_i) = \rho _i
  $$
\end{theorem*}

\begin{proof}
  Choose $Y$ by lemma \ref{chap4:sec8:lem2} such  that $v_i (Y-a_i) > \rho_i $. Find
  $b_i \in K $ such that $v_i(b_i) = \rho_i$ and an $Z \in K$ such
  that $v_i (Z-b_i) > \rho_i$. Then it follows that $v_i (Z)=\min (v_i
  (Z-b_i), v_i(b_i)) = \rho_i$. 
\end{proof}

Put $X = Y+Z$. Then,
$$
v_i (X-a_i) = v_i(Z+Y-a_i) = v_i (Z) = \rho_i,
$$
and $X$ satisfies the conditions of the theorem.

\begin{coro*}
  There are an infinity of places of any algebraic function field.
\end{coro*}

\begin{proof}
  Suppose there are only a finite number of places $\mathscr{Y}_1,
  \ldots, \mathscr{Y}_n$. 
\end{proof}

Choose an $X$ such that $v_{\mathscr{Y}_i} (X) > 0\, (i=1, \ldots
n)$. Then, $v_{\mathscr{Y}_i}(X+1) = v_{\mathscr{Y}_i}(1) = 0$ for
all the places $, \mathscr{Y}_i$, which is impossible since $X$ an
consequently $X+1$ is a transcendental element over $k$. 
