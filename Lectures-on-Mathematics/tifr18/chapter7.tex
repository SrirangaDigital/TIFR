\chapter{Lecture 7}\label{chap7}

\setcounter{section}{11}
\section{The Riemann Theorem}\label{chap7:sec1}%sec 12

In\pageoriginale the last lecture, we saw that if $X$ is any element of $K$, the
integer $l(\mathscr{N}^{-m} )+ d(\mathscr{N}_X^{-m})$ remains bounded
below as $m$ runs through all integral values. We now prove the
following stronger result, known as Riemann's theorem. 
\begin{theorem*}%thm
  Let $X$ be any transcendental element of $K$ and $(1-g)$ the lower
  bound of $l(\mathscr{N}_X^{-m}) + d(\mathscr{N}_X^{-m})$. Then, for
  any divisor $\mathscr{U}$, 
  $$
  l(\mathscr{U}) + d(\mathscr{U}) \ge 1 - g.
  $$
\end{theorem*}

\begin{proof}
  Let $\mathscr{U} = \mathscr{U}_1 \mathscr{U}_2^{-1}$, where
  $\mathscr{U}_1$ and $\mathscr{U}_2$ are integral divisors. Then
  clearly $\mathscr{U}_2^{-1}$ divides $\mathscr{U}$, and we have 
  $$
  l(\mathscr{U}) + D(\mathscr{U}) \ge l(\mathscr{U}_2^{-1}) +
  d(\mathscr{U}_2^{-1}) . 
  $$

  It is therefore enough to prove the inequality with
  $\mathscr{U}_2^{-1}$ in the place of $\mathscr{U}$. 
\end{proof}

The key to the proof lies in the statement that $l(\delta) +
d(\delta)$ is unaltered when we replace $\delta$ by $\delta(Z)$, where
$(Z)$ is a principal divisor. To prove this, consider the map defined
on $L(\delta)$ by  
$$
Y \in L (\delta) \to YZ
$$

This is clearly a $k$-isomorphism of the vector space $L(\delta)$ onto
the vector space $L(\delta(Z))$. This proves that $l(\delta) =
l(\delta(Z))$, and since we already know that $d(\delta) =
d(\delta(Z))$, our statement follows. 

Observe\pageoriginale now that for any non-negative integer $m$, we have
$l(\mathscr{N}_X^{-m}$ $\mathscr{U}_2) + d(\mathscr{N}_X^{-m}
\mathscr{U}_2) \ge l(\mathscr{N}_X^{-m}) +d(\mathscr{N}_X^{-m}) \ge
l-g$. Since $X$ is transcendental, $d(\mathscr{N}_X) > 0$ and it
follows that for large enough $m$, 
$$
l(\mathscr{N}_X^{-m} \mathscr{U}_2) \geq m d(\mathscr{N}_X  -
d(\mathscr{U}_2) +  l-g > 0 
$$

For such an $m$ therefore, there exists a non-zero element $Z$ in
$L(\mathscr{N}_X^{-m}$ $\mathscr{U})$. This clearly means that the
divisor $(Z) \mathscr{N}_X^m$ $\mathscr{U}_2^{-1}$ is integral, or that
$\mathscr{N}_X^{-m}$ divides $(Z) \mathscr{U}_2^{-1}$. Hence, we
deduce that  
\begin{multline*}
l(\mathscr{U}_2^{-1}) + d(\mathscr{U}_2^{-1}) =
l((Z)\mathscr{U}_2^{-1}) + d((Z)\mathscr{U}_2^{-1})\\ 
\ge 1(\mathscr{N}_X^{-m}) + d(\mathscr{N}_X^{-m}) \ge l -g 
\end{multline*}
which proves our theorem.

The integer $g$ is called the \textit{genus} of the field. Since
$$
1 + 0 = l(\mathscr{N}) + d(\mathscr{N}) \ge 1 - g ,
$$
it follows that $g$ is always non-negative. The integer $\delta
(\mathscr{U}^{-1}) = l(\mathscr{U}) + d(\mathscr{U}) + g-1$, which is
non-negative by the above theorem, is called the \textit {degree of
  speciality} of the divisor $\mathscr{U}$. We say that $\mathscr{U}$
is a \textit{non-special} or \textit {special} divisor according as
$\delta (\mathscr{U}^{-1})$ is or is not equal to zero. We shall
interpret $\delta (\mathscr{U}^{-1})$ later. Incidentally, we have
proved that if $\mathscr{U}$ is any divisor and $X \in K^*$, the
dimensions of the spaces $L(\mathscr{U} (X))$ and $L(\mathscr{U})$ are
the same. 

This  enables us to define the \textit{dimension of a divisor
  class}$C$. Choose any element $\mathscr{U}^{-1}$ in $C$ and define
the dimension $N(C)$ of $C$ to be $l(\mathscr{U})$. By the remark,
this is independent of the choice of $\mathscr{U}^{-1}$ in $C$. 

\section{Repartitions}\label{chap7:sce13}%sec 13

We\pageoriginale now consider the following question, to which we are led naturally
by the theorem of \S \ref{chap4:sec7}. If for every place $\mathscr{Y}$ of $K$, we
are given an element $X_\mathscr{Y}$ of $K$, can we find an $X$ in $K$
such that $v_\mathscr{Y}(X - X_\mathscr{Y}) \ge 0$ holds for every
$\mathscr{Y}$? A necessary condition for such an $X$ to exist is
that $v_\mathscr{Y} (X_\mathscr{Y}) \ge 0$ for all but a finite number
of $\mathscr{Y}$. For, suppose $v_\mathscr{Y}(X_\mathscr{Y})< 0$ for
some $\mathscr{Y}$. Then, since $v_\mathscr{Y} (X - X_\mathscr{Y}) \ge
0$, 
{\fontsize{10pt}{12pt}\selectfont
$$
v_\mathscr{Y}(X) = v_\mathscr{Y} (X-X_\mathscr{Y} + X_\mathscr{Y}) =
\min (v_\mathscr{Y}(X-X_\mathscr{Y}), v_\mathscr{Y}(X_\mathscr{Y})) =
v_\mathscr{Y} (X_\mathscr{Y}) < O , 
$$}\relax
and this can hold for at most a finite number of $\mathscr{Y}$. We now
make the following 

\begin{defi*}%
  A {\em repartition} $\mathscr{C}$ is a mapping $\mathscr{Y} \to
  \mathscr{G}_\mathscr{Y}$ of the set of prime divisors $\mathscr{Y}$
  of $K$ into the field $K$ such that $v_\mathscr{Y}
  (\mathscr{C}_\mathscr{Y}) \ge o$ for all but a finite number of
  $\mathscr{Y}$. 
\end{defi*}  
  
 We can define the operations of addition and multiplication in the
 space $\mathfrak{X}$ of repartitions in an obvious manner. If
 $\mathscr{C}$ and $\mathscr{G}$ are two repartitions, and $a$ an
 element of the constant field $k$, 
 $$
 (\mathscr{C}  + \mathscr{G})_\mathscr{Y} = \mathscr{C}_\mathscr{Y} +
 \mathscr{G}_\mathscr{Y} , (\mathscr{C} \mathscr{G})_\mathscr{Y} =
 \mathscr{C}_\mathscr{Y} \mathscr{G}_\mathscr{Y} , (a \mathscr{C}
 )_\mathscr{Y} = a \mathscr{C}_\mathscr{Y} . 
 $$
 
  The newly defined mappings are immediately verified to be
  repartitions. Thus, $\mathfrak{X}$ becomes an algebra over the field
  $k$. We can imbed the field $K$ in $\mathfrak{X}$ by defining for
  every $X \in K$ the repartition $\mathscr{C}_X$ by the equations
  $(\mathscr{C}_X)_\mathscr{Y} = X$ for every $\mathscr{Y}$. The
  condition for this to be a repartition clearly holds, and one can
  easily verify that this is an isomorphic\pageoriginale imbedding of $K$ in the
  algebra $\mathfrak{X}$. 
  
 We can now extend to repartitions the valuations of the field $K$ by
 defining for every place $\mathscr{Y}$, 
 $$
 v_\mathscr{Y}(\mathscr{C}) = v_\mathscr{Y} (\mathscr{C}_\mathscr{Y}).
 $$
 
Clearly, we have the following relations
\begin{align*}
  v_\mathscr{Y} (\mathscr{C} \mathscr{G}) 
  & = v_\mathscr{Y} (\mathscr{C}) + v_\mathscr{Y} (\mathscr{G} )\\ 
  v_\mathscr{Y}(\mathscr{C} + \mathscr{G}) 
  & \ge \min (v_\mathscr{Y}  (\mathscr{C}), v_\mathscr{Y} (\mathscr{G} )) , 
\end{align*} 
 and $v_{\mathscr{Y}} (\mathscr{C}_X) = v_\mathscr{Y} (X)$
 
 This leads to the notion of the divisibility of a repartition
 $\mathscr{C}$ by a divisor $\mathscr{U}$. We shall say that
 $\mathscr{C}$ is \textit{divisible} by $\mathscr{U}$ if $v_\mathscr{Y}
 (\mathscr{C}) \ge v_\mathscr{Y} (\mathscr{U})$ for every
 $\mathscr{Y}$, and that $\mathscr{C}$ and $\mathscr{G}$ are
 congruent modulo $\mathscr{U} (\mathscr{C} \equiv \mathscr{G}
 (\mathscr{U}) $ in symbols) if $\mathscr{C} - \mathscr{G}$ is
 divisible by $\mathscr{U}$. 
 
 The problem posed at the beginning of this article may be restated in
 the following generalised form. Given a repartition $\mathscr{C}$ and
 a divisor $\mathscr{U}$, to find an element $X$ of the field such that
 $X \equiv \mathscr{C} (\mathscr{U})$. (The original problem is the case
 $\mathscr{U} = \mathscr{N}$). 
 
 If $\mathscr{U}$ is a divisor, let us denote by $\wedge
 (\mathscr{U})$ the vector space (over $k$) of all repartitions
 divisible by $\mathscr{U}$. Then we have the following 
\begin{theorem*}
  If $\mathscr{U}$ and $\delta$ are two divisors such that
  $\mathscr{U}$ divides $\delta$, then $\wedge (\mathscr{U}) \supset
  \wedge (\delta)$ and 
  $$
  \dim_k \frac{\wedge (\mathscr{U})} {\wedge (\delta)} = d (\delta) -
  d(\mathscr{U}). 
  $$
\end{theorem*} 

\begin{proof}
  Let\pageoriginale $S$ denote the set of prime divisors occurring in either
  $\mathscr{U}$ or $\delta$ with a non-zero exponent. Since $\dim_k
  \dfrac{\Gamma(\mathscr{U}/S)} {\Gamma (\delta /S)} = d(\delta) -
  d(\mathscr{U})$, it is enough to set up an isomorphism of
  $\dfrac{\Gamma (\mathscr{U}/S)} {\Gamma (\delta/S)}$ onto the space
  $ \dfrac{\wedge (\mathscr{U})} {\wedge(\delta)}$. 
\end{proof} 
 
 If $x \in \Gamma (\mathscr{U},/S)$, define a repartition
 $\mathscr{G}_x$ as follows: 
 $$
 (\mathscr{G}_X)_\mathscr{Y} = 
 \begin{cases} 
   X &\text{if}~ \mathscr{Y} \in S \\ 
   0 & \text{if}~ \mathscr{Y} \notin S 
 \end{cases} 
 $$
 
Clearly, $\mathscr{G}_X \in \wedge (\mathscr{U})$, and $X \to
\mathscr{G}_X$ is a $k$-homomorphism of $\Gamma (\mathscr{U}/S)$ into
$\wedge (\mathscr{U})$. The image of an element $X \in \Gamma
(\mathscr{U}/S)$ lies in $\wedge(\delta)$ if and only if
$v_\mathscr{Y}(X) \ge v_\mathscr{Y}(\delta)$ for every $\mathscr{Y}
\in S$, i.e., if and only if $X \in \Gamma(\delta/S)$. Thus, we have
an isomorphism of $\dfrac{\Gamma(\mathscr{U}/S)} {\Gamma(\delta/S)}$
into $\dfrac{\wedge(\mathscr{U})}{\wedge (\delta)}$. We shall show
that this is onto. Given any repartition $\mathscr{C} \in \wedge
(\mathscr{U})$, find $X \in K$ such that 
$$
 v_\mathscr{Y}(X- \mathscr{C}) \ge v_\mathscr{Y} (\delta) ~\text{for
   every}~ \mathscr{Y} \in S. 
 $$
 
 This means that the repartition $\mathscr{G}_X - \mathscr{C}$ is an
 element of $\wedge(\delta)$. Also, the above condition implies that
 for $\mathscr{Y} \in S, ~ v_\mathscr{Y}(X)\ge \min
 (v_\mathscr{Y}(\mathscr{C})$,\break $v_\mathscr{Y}(\delta)) \ge v_\mathscr{Y}
 (\mathscr{U})$. Thus, $X$ is an element of $\Gamma (\mathscr{U}/S)$
 and its image in $\dfrac{\wedge(\mathscr{U})} {\wedge(\delta)}$ is
 the coset $\mathscr{C} + \wedge (\delta)$. Our theorem is thus
 proved. 
