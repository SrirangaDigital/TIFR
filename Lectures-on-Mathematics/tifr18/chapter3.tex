\chapter{Lecture 3}\label{chap3}

\setcounter{section}{3}
\section{The Valuations of Rational Function Field}\label{chap3:sec4}

Let\pageoriginale $K = k(X)$ be a rational function field over $k$ ; i.e., $K$ is
got by adjoining to $k$ a single transcendental element $X$ over
$k$. We seek for all the valuations $v$ of $K$ which are
\textit{trivial on $k$}, that is, $v(a)=1$ for every $a \in K^*$. It
is easily seen that these are the valuations which correspond to
places whose restrictions to $k$ are monomorphic. 

We shall henceforward write all our ordered groups additively.

Let $\varphi$ be a place of $K = k(X)$ onto $\sum \cup (\infty)$. We
consider two cases 
\begin{Case}\label{chap3:sec4:case1} % case 1
  Let $\varphi (X) = \xi \neq \infty$. Then, the polynomial ring
  $k[X]$ is contained in $\mathscr{O}_{\varphi}$, and $\mathscr{Y} \cap
  k[X]$ is a prime ideal in $k [X]$. Hence, it should be of the form
  $(p(X))$, where $p(X)$ is an irreducible polynomial in $X$. Now, if
  $r(X) \in K$, it can be written in the form $r(X) = (p(X))^\rho
  \dfrac{g(X)}{h(X)}$, where $g(X)$ and $h(X)$ are coprime and prime
  to $p(X)$. Let us agree to denote the image in $\sum$ of an element
  $c$ in $k$ by $\bar{c}$, and that of a polynomial $f$ over $k$ by
  $\bar{f}$. Then,we clearly have  
  \begin{equation*}
    \varphi (r(x)) = 
    \begin{cases}
      0 & \text{ if  } \rho > o \\ 
      \frac{g(\xi)}{h(\xi)} & \text{ if } \rho = 0 \\ 
      \infty & \text{ if }\rho < 0 
    \end{cases}
  \end{equation*}
\end{Case}

Conversely, suppose $p(X)$ is an irreducible polynomial in $k[X]$\pageoriginale and
$\xi$ a root of $p(X)$. The above equations then define a mapping of
$k(X)$ onto $k(\xi) \cup \{ \infty\}$, which is a place, as is verified
easily. We have thus determined all places of $k(X)$ under case
\ref{chap3:sec4:case1} 
(upto equivalence). 

If $Z$ is the additive group of integers with the natural order, the
valuation $v$ associated with the place $\varphi$ above is given by
$v(r(X)) = \rho$. 
\begin{Case} % case 
  Suppose now that $\varphi (X) = \infty$. Then $\varphi
  (\dfrac{1}{X}) = 0$. Then since $K = k(X) = k(\dfrac{1}{X})$, we see
  that $\varphi$ is determined by an irreducible polynomial $p(\dfrac{1}{X})$,
  and since $\varphi \dfrac{1}{X}) = 0, p(Y)$ should divide $Y$. Thus,
  $p(Y)$ must be $Y$ (except for a constant in $k$), and if 
  \begin{align*}
    r(X) & = \frac{a_0 + a_1 x+ - + a_n x^n}{b_0 + b_1 x+ - + b_m x^m},
    a_n , b_m \neq 0, \\ 
    \varphi (r(X)) & = \varphi \left( ^{(\frac{1}{X})^{m-n}}
    \frac{\frac{a_0}{x^n} + \frac{a_1}{x^{n-1}} + - +
      a_n}{\frac{b_0}{x^m} + \frac{b_1}{x^{m-1}} + - + b_m} \right)
    = \begin{cases} 
      o & \text{if}~ m > n \\ 
      \frac{a_n}{b_m} & \text{if}~ m = n \\ 
      \infty & \text{if}~  m < n 
    \end{cases} 
  \end{align*}
\end{Case}

The corresponding valuation with values in $Z$ is given by $v(r(X)) =
m-n = -\deg r (X)$, where the degree of a rational function is defined
in the degree of the numerator-the degree of the denominator.  

We shall say that a valuation is \textit{discrete} if the valuation
group may be taken to be $Z$. We have in particular proved that all
valuations of a rational function field trivial over the constant
field are discrete. We shall extend this result later to all algebraic
function fields of one variable. 

\section{Extensions of Places}\label{chap3:sec5}

Given\pageoriginale a field $K$, a subfield $L$ and a place $\varphi_L$ of $L$ into
$\sum$, we wish to prove in this section that there exists a place
$\varphi_K$ of $K$ into $\sum^1$, where $\sum^1$ is a field containing
$\sum$ and the restriction of $\varphi_K$ to $L$ is $\varphi_L$. Such
a $\varphi_K$ is
called an extension of the place $\varphi_L$ to $K$. For the proof of
this theorem, we require the following 
\begin{lemma*}[({\em Chevalley}) ]
  Let $K$ be a field, $\mathscr{O}$ a subring and $\varphi$ a
  homomorphism of $\mathscr{O}$ into a field $\Delta$ which we assume
  to be algebraically closed. Let $q$ be any element of $K^*$, and
  $\mathscr{O}[q]$ the ring generated by $\mathscr{O}$ and $q$ in
  $K$. Then $\varphi$ can be extended into a homomorphism $\Phi$ of at
  least one of the rings $\mathscr{O}[q], \mathscr{O}[\dfrac{1}{q}]$,
  such that $\Phi$ restricted to $\mathscr{O}$ coincides with
  $\varphi$. 
\end{lemma*}

\begin{proof}
  We may assume that $\varphi$ is not identically zero. Since the
  image of $\mathscr{O}$ is contained in a field, the kernel of
  $\varphi$ is a prime ideal $\mathscr{Y}$ which is not the whole ring
  $\mathscr{O}$. Let $\mathscr{O}^1 = \mathscr{O} \cup\left\{ \dfrac{a}{b}\, a, b
  \in \mathscr{O} , b \notin \mathscr{Y} \right\}$. Clearly, $\mathscr{O}^1$ is
  a ring with unit, and $\varphi$ has a unique extension
  $\tilde{\varphi}$ to $\mathscr{O}^1$ as a homomorphism, give by
  $\tilde{\varphi}\left(\dfrac{a}{b}\right) = \dfrac{\varphi
    (a)}{\varphi (b)}$. The image 
  by $\tilde{\varphi}$ is then the quotient field $\sum$ of $\varphi
  (\mathscr{O})$. We shall denote $\tilde{\varphi}(a)$ by $\bar{a}$
  for $a \in \mathscr{O}^1$. 
\end{proof}

Let $X$ and $\bar{X}$ be indeterminates over $\mathscr{O}^1$ and
$\sum$ respectively. $\tilde{\varphi}$ can be extended uniquely to a
homomorphism $\bar{\varphi}$ of $\mathscr{O}^1 [X]$ onto $\sum
[\bar{X}]$ which takes $X$ to $\bar{X}$ by defining	 
$$
\bar{\varphi} (a_0 + a_1 X + - + a_n X^n) = \bar {a}_0 + \bar{a}_1
\bar{X} + - + \bar{a}_n \bar{X}^n. 
$$

Let\pageoriginale $\mathscr{U}$ be the ideal $\mathscr{O}^1[X]$ consisting of all
polynomials which vanish for $X=q$, and let $\bar{\mathscr{U}}$ be the
ideal $\bar{\varphi}(\mathscr{U})$ in $\sum[\bar{X}]$. We consider
three cases. 

\setcounter{Case}{0}
\begin{Case} % case
  Let $\bar{\mathscr{U}} = (0)$. In this case, we define $\Phi(q)$ to
  be any fixed element of $\Delta$. $\Phi$ is uniquely determined on
  all other elements of $\mathscr{O}^1[q]$ by the requirement that it
  be a ring homomorphism which is an extension of $\tilde{\varphi}$. In
  order that it be well defined, it is enough to verify that if any
  polynomial over $\mathscr{O}^1$ vanishes for $q$, its image by
  $\bar{\varphi}$ vanishes for $\Phi(q)$. But this is implied by our
  assumption. 
\end{Case}

\begin{Case} %  case 2
  Let $\bar{\mathscr{U}} \neq (0), \neq \sum [\bar{X}]$. Then
  $\bar{\mathscr{U}} = (f(\bar{X})) $, where $f$ is a non-constant
  polynomial over $\sum$. Let $\alpha$ be any root $f(\bar{X})$ in
  $\Delta$ (there is a root in $\Delta$ since $\Delta$ is
  algebraically closed). Define $\Phi(q) = \alpha$. This can be
  extended uniquely to a homomorphism of $\mathscr{O}^1[q]$, since the
  image by $\bar{\varphi}$ of any polynomial vanishing for $q$ is of
  the form $f (\bar{X})f(\bar{X})$, and therefore vanishes for
  $\bar{X} = \alpha$. 
\end{Case}

\begin{Case} % case 3
  Suppose $\bar{\mathscr{U}} = \sum [\bar{X}]$. Then the homomorphism
  clearly cannot be extended to $\mathscr{O}^1 [q]$. Suppose now that
  it cannot be extended to $\mathscr{O}^1 [\dfrac{1}{q}]$ either. Then
  if $\delta$ denotes the ideal of all polynomials in $\mathscr{O}^1
  [X]$ which vanish for $\dfrac{1}{q}$, and if $\bar{\delta}$ is the ideal
  $\bar{\varphi}(\delta) $in $\sum [\bar{X}]$, we should have
  $\bar{\delta} = \sum [\bar{X}]$. Hence, there exist polynomials
  $f(X)=a_0 + 	a_1 X + - + a_n X^n$ and $b_0 + 	b_1 X + - +
  b_m X^m$ such that $\bar{\varphi} (f(X)) = \bar{\varphi}(g(X)) = 1$, $f(q)
  = g \left(\dfrac{1}{q}\right) = 0$. We may assume that $f$ and $g$ are of
  minimal degree $n$ and $m$ satisfying the required conditions.\pageoriginale Let
  us assume that $m \le n$. Then, we have $\bar{a}_0 = \bar{b}_0 = 1,
  \bar{a}_i = \bar{b}_j = 1$ for $i$, $j > 0$. Let $g_0 (X) = b_0 X^m
  + \cdots + b_m$. Applying the division algorithm to the polynomials
  $b_0 ^n f(X)$ and $g_0 (X)$, we obtain  
  $$
  b_0 ^n f(X) = g_0(X) Q(X) + R(X), Q(X), R(X) \in \mathscr{O}^1 [X],
  \deg R < m. 
  $$
\end{Case}

Substituting $X = q$, we obtain $R(q) = 0$. Also, acting with
$\bar{\varphi}$, we have  
$$
1 = \bar{b}_0 ^n \bar{f}(\bar{X}) = \bar{g}_0(\bar{X})
\bar{Q}(\bar{X}) + \bar{R}(\bar{X}) =  \bar{Q}(\bar{X}) \bar{X}^m +
\bar{R}(\bar{X}), 
$$
and hence, we deduce that $\bar{Q}(\bar{X}) = 0$, $\bar{R} (\bar{X}) =
1$. Thus, $R(X)$ is a polynomial with $R(q) = 0, \bar{R}(\bar{X}) =
1$, and deg $F(X) < m \le n$, which contradicts our assumption on the
minimality of the degree of $f(X)$. Our lemma is thus prove. 

We can now prove the 
\begin{theorem*}
  Let $K$ be a field and $\mathscr{O}$ a subring of $K$. Let $\varphi$
  be a homomorphism of $\mathscr{O}$ in an algebraically closed field
  $\Delta$. Then it can be extended either to a homomorphism of $K$ in
  $\Delta$ or to a place of $K$ in $\Delta \cup (\infty)$. In
  particular, any place of a subfield of $K$ can be extended to a
  place of $K$. 
\end{theorem*}

\begin{proof}
  Consider the family of pairs $\{ \varphi_{\alpha},
  \mathscr{O}_{\alpha}\}$, where $\mathscr{O}_{\alpha}$ is a subring
  if $K$ containing $\mathscr{O}$ and $\varphi_{\alpha} $ a homomorphism
  of $\mathscr{O}_{\alpha}$ in $\Delta$ extending $\varphi$ on
  $\mathscr{O}$. The family is non-empty, since it contains $(\varphi
  , \mathscr{O})$. We introduce a partial order in this family by
  defining $(\varphi_{\alpha}, \mathscr{O}_{\alpha}) >
  (\varphi_{\beta}, \mathscr{O}_{\beta})$ if $\mathscr{O}_{\alpha}\supset
  \mathscr{O}_{\beta}$ and $\varphi_{\alpha}$ is an extension of
  $\varphi_{\beta}$. 
\end{proof}

The\pageoriginale family clearly being inductive, it has a maximal element by Zorn's
lemma. Let us denote it by $(\Phi, \mathscr{O})$. $\mathscr{O}$ is
either the whole of $K$ or a valuation ring of $K$. For if not, there
exists a $q \in K$ such that neither $q$ nor $\dfrac{1}{q}$ belongs to
$\mathscr{O}$. we may then extend $\Phi$ to a homomorphism of at least
one of $\mathscr{O}[q]$ or $\mathscr{O}\left[\dfrac{1}{q}\right]$ in
$\Delta$. Since both these rings contain $\mathscr{O}$ strictly, this
contradicts the maximality of $(\Phi , \mathscr{O})$. 

If $\mathscr{O}$ were not the whole of $K, \Phi$ must vanish on every
non - unit of $\mathscr{O}$; for if $q$ were a non-unit and $\Phi(q)
\neq 0$, we may define $\Phi \left(\dfrac{1}{q}\right) = \dfrac{1}{\Phi (q)}$ and
extend this to a homomorphism of $\mathscr{O} \left[\dfrac{1}{q}\right]$, which
again contradicts the maximality of $\mathscr{O}$. This proves that
$\Phi	$ can be extended to a place of $K$ by defining it to be
$\infty$ outside $\mathscr{O}$. 

In particular, a place $\varphi$ of a subfield $L$ of $K$ , when
considered as a homomorphism of its valuation ring
$\mathscr{O}_{\varphi}$ and extended to $K$, gives a place on $K$; for
if $\varphi$ where a homomorphism of the whole of $K$ in $\Delta$, it
should be an isomorphism (since the kernel, being a proper ideal in
$K$, should be the zero ideal). But $\Phi$ being an extension of
$\varphi$, the kernel contains at least one non-zero element. 

\begin{coro*}
  If $K/k$ is an algebraic function field and $X$ any element of $K$
  transcendental over $k$, there exists at least one valuation $v$ for
  which $v(X) > 0$. 
\end{coro*}

\begin{proof}
  We have already shown in the previous section that there exists a
  place $\mathscr{Y}_1$ in $k(X)$ such that $v_{\mathscr{Y}_1}(X) > 0$. If we extend
  this place $\mathscr{Y}_1$ to a place $\mathscr{Y}$ of $K$, we
  clearly have $v_{\mathscr{Y}}(X) > 0$. 
\end{proof}
