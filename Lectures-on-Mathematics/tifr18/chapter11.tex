\chapter{Lecture 11}\label{chap11} %cha 11

\setcounter{section}{21}
\section{The Infinite Product for \texorpdfstring{$\zeta (s,K)$}{zetasK}}\label{chap11:sec:22} %sec 22

Let\pageoriginale $K$ be an algebraic function field over a finite constant field
$k$. Then, the $\zeta$ function of $K$ can be expressed as an infinite
product taken over all prime divisors of $K$. 
\begin{theorem*}
  For $s = \sigma + i t$, and $\sigma > 1$, we have
  $$
  \zeta (s,K) = \prod_\mathscr{Y} \frac{1}{1-(N \mathscr{Y})^{-s}},
  $$
  where the product is taken over all prime divisors $\mathscr{Y}$ of
  $K$. The product is absolutely convergent, and hence does not depend
  on the order of the factors. 
\end{theorem*}

\begin{proof}
  Since $\sigma > 1$, we have for any integer $m>o$,
  $$
  \prod_{N \mathscr{Y} \le m} \frac{1}{1-N \mathscr{Y}^{-s}} =\prod_{ N
    \mathscr{Y} \le m} \left(1+ \frac{1}{(N \mathscr{Y})^{s}} +
  \frac{1}{(N \mathscr{Y})^{2s}} + \cdots\right), 
  $$
  and since there are only a finite number of factors in the product, 
  each factor being an absolutely convergent series, we may multiply these 
  to obtain  
  $$
  \prod_{N \mathscr{Y} \le m} \frac{1}{1-(N \mathscr{Y})^{-s}} =
  \sum_{n \mathscr{U} \le m} \frac{1}{(N \mathscr{U})^s} + \sum_{N
    \mathscr{U}> m} \frac{1}{(N \mathscr{U})^s},
  $$
  where the first 
  summation is over all integral divisors $\mathscr{U}$ with $N
  \mathscr{U} \le m$, and the second over all integral divisors
  $\mathscr{U}$ which do not contain any prime divisor $\mathscr{U}$
  with $N \mathscr{U} >m$ and which satisfy the inequality $N
  \mathscr{U} > m$. 
\end{proof}

Hence\pageoriginale\
\begin{align*}
  \bigg|  \prod_{N \mathscr{Y} \le m} \frac{1}{1-(N \mathscr{Y})} -s -
  \sum _{N \mathscr{U} \le m} \frac{1}{N \mathscr{U}} s \bigg| & = \bigg|
  \sum^1_{N \mathscr{U} > m} \frac{1}{(N \mathscr{U})^s} \bigg|\\ 
  & \le \sum_{ N \mathscr{U} > m} \frac{1}{(N \mathscr{U})^\sigma} 
\end{align*}
and letting $m \to \infty$, we obtain the asserted equality, since
$\sum\limits_{N \mathscr{U} > m} (\dfrac{1}{ N \mathscr{U}})^\sigma$,
being the remainder of the convergent series for $\zeta (\sigma, K)$,
tends to zero as $m \to \infty$. 

The absolute convergence of the product is deduced from the inequality 
$$
\bigg|  \frac{1}{ (N \mathscr{Y})^s} + \frac{1}{ (N \mathscr{Y})^{2s}}
+  - \bigg| \le \frac{1}{ (N \mathscr{Y})^\sigma} + \frac{1}{ (N
  \mathscr{Y})^{2 \sigma}} 
$$

As a corollary to this theorem, we see that $\zeta (s, K)$ has no zero
for $\sigma > 1$.  

\section{The Functional Equation}\label{chap11:sec:23} %sec 24

In the last lecture, we obtained the following formula:
$$
\displaylines{
  (q-1)^\xi (s,K) = F (U) + R (U) , \quad \text{ where}\quad  U = q^{-s},\phantom{Wi}\cr
  F(U) = \sum _{o \le d (c) \le 2g -2} q^{N (C)} U^{d (C)}\phantom{WWWW}\cr
  \text{and}\hfill  R(U) = hq^{1-g} \frac{(Uq)^{\max (o,2g-2+\rho)}}{1-(Uq)\rho} -
  \frac{h}{1-U\rho}.\hfill }
$$

Substituting for $N (C)$ in $F (U)$ from the
theorem of Riemann-Roch, we\pageoriginale obtain 
\begin{align*}
  F (U) & =\sum_{o \le d (C) \le 2g-2} q^{d (C) - g + 1 + N (WC^{-1})}
  \mathscr{V}^{d(C)}\\
  & = q^{1-g} \sum_{o \le d (C) \le 2g-2} \left(\frac{1}{q^U}\right){^{ d
      (WC^{-1}) -2g+2}} q^{N (WC^{-1})} 
\end{align*}
	
Writing $WC^{-1} = C^1$, and noticing that $C^1$ runs through the same
set of classes as $C$ in the summation, we have 
$$
F(U) = q^{q-1} U^{2g-2} \sum_{o \le d (C^1) \le 2g-2}
\left(\frac{1}{qU}\right)^{d (C^1)} q^{N (C^1)} =q^{g-1} U^{2g-2}
F\left(\frac{1}{qU}\right). 
$$ 

We shall prove that a similar functional equation holds for $R
(U)$. First suppose that $g > o$. Then, 
\begin{align*}
  R (U) & = hg ^{1-g} \frac{(qU)^{2g-2 +\rho}}{1-(qU)^\rho} - \frac{h}{1-U^\rho}\\
  & =  - h g^{1-g}
  \frac{\left(\frac{1}{qU}\right)^{2g+2}}{1-\left(\frac{1}{qU}\right)\rho}
  + \frac{h. (q. \frac{1}{Uq})^\rho}{1- \left(q. \frac{1}{Uq}\right)^\rho}\\ 
  & = U^{2 g-2} q^{g-1} \left[ h q ^{1-g}\frac{\left(q \frac{1}{qU}\right)^{2g-2
        +\rho}}{1- \left(g. \frac{1}Uq\right)^\rho} - \frac{h}{1-
      \left(\frac{1}{Uq}\right)^\rho}\right]\\
& = q^{g-1} U^{2g-2} R
  \left(\frac{1}{qU}\right). 
\end{align*}


If $g=o$, the only divisor class of degree zero is $E$ and hence $h
=1$. Also, since $\rho$ divides $2g -2 =-2, \rho =1 \text{ or } \rho
=2$. If $\rho = 1$, we obtain 
\begin{align*}
   R (U) & = \frac{q}{1-qU} - \frac{1}{1-U}= \frac{q-1}{(1-U) (1-qU)}\\
  & = q^{-1} U^{-2} \frac{q-1}{\left(1- \frac{1}{qU}\right) \left(1-q
     .\frac{1}{qU}\right)}\\ 
  & = q^{-1} U^{-2} R \left(\frac{1}{qU}\right).
\end{align*}

If\pageoriginale $\rho =2$, we get
\begin{align*}
  R(U) & = \frac{q}{1-(qU)^2} - \frac{1}{1-U^2} =
  \frac{-q. \left(\frac{1}{qU}\right)^2}{1- \left(\frac{1}{qU}\right)^2} +
  \frac{U^{-2}}{1-\left(q.\frac{1}{qU}\right)^2}\\ 
  & = q^{-1} U^{-2} R \left(\frac{1}{qU}\right).
\end{align*}

Thus, in any case, we have the functional equation
$$
R (U) = q^{g-1} U^{2g-2} R \left(\frac{1}{qU}\right).
$$

Adding the equation for $F (U)$ and $R(U)$, we see that $\zeta (s, K)$
satisfies the functional equation 
\begin{align*}
  (q-1) \zeta  (s,K) & = q^{g-1} q^{s (2-2g)} (q-1) \zeta (1-s,K), \text{ or }\\
  \zeta(s,K) & =q^{g-1} q^{s (2-2g)} \zeta (1-s.K)
\end{align*}

We may also rewrite this in the from
$$
q^{s (q-1)} \zeta (s,K) = q ^{(1-s) (g-1)} \zeta (1-s,K).
$$

Thus, the function on the left is unaltered by the transformation $s
\to 1-s$. 

\section{\texorpdfstring{$L$}{L}-series}\label{chap11:sec:24} %sce 24

We wish to study the $L$-series associated to characters of the class
group of an algebraic function with a finite constant field.  

\begin{defi*}
  A\pageoriginale {\em character $\mathcal{X}$  of finite order} on the class group
  $\mathfrak{R}$ is a homomorphism of $\mathfrak{R}$ into the
  multiplicative group $C^*$ of non zero complex numbers such that
  there exists in integer $N$ with $\mathcal{X}^N (C) =1$ for all $C$
  in $\mathfrak{R}$.  

  $\mathcal{X} (C)$ is therefore an $N^{th}$ root of unity for all
  $C$.  We may define $\mathcal{X}$ on the group $v$ of divisors
  by comparing with the natural homomorphism $v \to \mathfrak{R},
  i.e$., by putting $\mathcal{X} (\mathscr{U}) = \mathcal{X}
  (\mathscr{U} E)$ for any divisor $\mathscr{U}$. 
\end{defi*}

The $L$-function $L (s ,\mathcal{X} , K)$ associated to a character
$\mathcal{X}$ (which we shall always assume to be of finite order) is
then defined for $s = \sigma + it$, $\sigma > 1$ by the series 
$$
L (s, \mathfrak{X}, K) = \sum_\mathscr{U} \mathcal{X} (\mathscr{U}) (N
\mathscr{U})^{-s} 
$$
where the summation is over all integral divisors $\mathscr{U}$ of the
field. The absolute value of the terms of this series is majorised by
the corresponding term of the series $\sum \dfrac{1}{(N
  \mathscr{U})^\sigma} = \zeta (\sigma, K)$ and is therefore
absolutely convergent for $\sigma > 1$. We may prove along exactly the
same lines as in the case of  the $\zeta$ -function the following
product formula:   
$$
L (s, \mathfrak{X}, K) = \prod_{\mathscr{Y}} \frac{1}{ 1- \mathcal{X}
  (\mathscr{Y}) (N \mathscr{Y})^{-s}} 
$$
where $\mathscr{Y}$ runs through all prime divisors and $\sigma > 1$.
