\chapter{Lecture 8}\label{chap8}
 
\setcounter{section}{13}
\section{Differentials}\label{chap8:sec14}%sec 14
 
In\pageoriginale this article, we wish to introduce the important notion of a
differential of an algebraic function field. As a preparation, we
prove the  
 \begin{theorem*}
   If $\mathscr{U}$ and $\delta$ are two divisors and $\mathscr{U}$
   divides $\delta$, then  
   $$
   \displaylines{\hfill 
   \dim_k \frac{\wedge(\mathscr{U}) + K} {\wedge (\delta) + K} =
   (l(\delta) + d(\delta)) - (l(\mathscr{U}) + d(\mathscr{U})) \hfill \cr
   \text{and}\hfill 
   \dim_k \frac{\mathfrak{X}}{\wedge(\mathscr{U})+K} =
   \delta(\mathscr{U}^{-1}) = l (\mathscr{U}) + d(\mathscr{U}) + g
   -1.\hfill } 
   $$
 \end{theorem*} 
  
\begin{proof} 
  We have
  $$
  \frac{\wedge(\mathscr{U}) + K} {\wedge (\delta) + K} =
  \frac{\wedge(\mathscr{U}) + (\wedge(\delta) + K)} {\wedge (\delta) +
    K} \simeq \frac{\wedge(\mathscr{U})}{(\wedge(\delta) + K)\cap
    \wedge(\mathscr{U})} 
  $$

  But it is easily verified that $(\wedge(\delta)+ K) \cap \wedge
  (\mathscr{U}) = \wedge (\delta) + L(\mathscr{U})$. Hence, we obtain 
  \begin{multline*}
    \frac{\wedge(\mathscr{U}) + K} {\wedge(\delta) + K } \simeq
    \frac{\wedge(\mathscr{U})} {\wedge(\delta) + L(\mathscr{U})} ~
    \frac{\wedge(\mathscr{U})/\wedge (\delta)} {\wedge (\delta) + L
      (\mathscr{U})/ \wedge (\delta)}\\ 
    \frac{\wedge(\mathscr{U})/\wedge(\delta)} {L(\mathscr{U}) /L
      (\mathscr{U}) \cap \wedge (\delta )} 
     = \frac{\wedge(\mathscr{U})/\wedge(\delta)} {L(\mathscr{U})/L(\delta)}
  \end{multline*}

  Thus,
  \begin{multline*}
    \dim_k \frac{\wedge(\mathscr{U}) +K} {\wedge (\delta) + K} =
    \dim_k \frac{\wedge(\mathscr{U})} {\wedge (\delta)} - \dim_k
    \frac{L(\mathscr{U})} {L(\delta)}\\ 
     = (d(\delta) - d(\mathscr{U})) - (l(\mathscr{U}) - l(\delta)) =
    (l(\delta) + d(\delta)) - (l(\mathscr{U}) + d(\mathscr{U})), 
  \end{multline*}
  which\pageoriginale is the first  part of the theorem.
\end{proof} 

 Now, choose a divisor $\mathfrak{L}$ such that
 $$
 l(\mathfrak{L}) + d(\mathfrak{L}) = 1 -g .
 $$
 
 Putting $\mu_ \mathscr{Y} = \min (v_\mathscr{Y} (\delta),
 v_\mathscr{Y})$, and $\mathscr{U} = \prod\limits_{\mathscr{Y}}
 \mathscr{Y}^{\mu_{\mathscr{Y}}}$, we see that $\mathscr{U}$ divides
 both $\delta$ and $\mathfrak{L}$. Hence 
$$
\displaylines{
  1 - g \le l (\mathscr{U}) + d (\mathscr{U}) \le l (\mathfrak{L})  +
  d(\mathfrak{L}) = 1 -g,\cr 
  \text{and hence}\hfill  l(\mathscr{U}) + d(\mathscr{U}) = 1-g.\hfill}
$$

 Moreover, we have
 $$
 \dim_k \frac{\mathfrak{X}} {\wedge(\delta) +K} \ge \dim_k
 \frac{\wedge(\mathscr{U}) + K} {\wedge (\delta) + K} = l(\delta) +
 d(\delta) - 1 + g = \delta(\delta^{-1}). 
 $$

To prove the opposite inequality, suppose $\mathscr{C}_1 , \ldots
\mathscr{C}_m$ are $m$ linearly independent elements of $\mathfrak{X}$
over $k$ module $\wedge (\delta) + K$. If we put $\nu_\mathscr{Y}
=\min\limits_i (v_\mathscr{Y}(\mathscr{C}_i), v_\mathscr{Y} (\delta))$
and $\mathscr{U}= \prod\limits_\mathscr{Y}
\mathscr{Y}^{\gamma_{\mathscr{Y}}}$, clearly all the $\mathscr{C}_i$ lie in
$\wedge(\mathscr{U})$. We deduce that 
$$
m \le \dim_k \frac{\wedge(\mathscr{U}) + K} {\wedge (\delta) + K}  =
(l(\delta) + d(\delta)) - (l(\mathscr{U}) + d(\mathscr{U})) \le
l(\delta) + d(\delta)-1 +g,  
$$
which proves that $\dim_k \dfrac{\mathfrak{X}} {\wedge(\delta) + K}$
is finite and $\le \delta (\delta^{-1})$. The second part of the
theorem is therefore proved. 

\begin{defi*}
  A {\em differential} $\omega$ is a linear mapping of $\mathfrak{X}$
  into $k$ which vanishes on some sub space of the form $\wedge
  (\mathscr{U} + K)$. 
\end{defi*}

In this case, $\omega$ is said to be \textit{divisible} by
$\mathscr{U}^{-1}. \omega$ is said to be of the \textit{first kind} if
it is divisible by $\mathscr{N}$ 

If\pageoriginale $\delta$ divides $\mathscr{U}$, clearly $\wedge (\delta^{-1}) + K
\subset \wedge (\mathscr{U}^{-1}) + K$, and therefore every
differential divisible by $\mathscr{U}$ is also divisible by
$\delta$. 

Consider now the set $D(\mathscr{U})$ of differentials $\omega$ of $K$
which are divisible by $\mathscr{U}$. This is the dual of the finite
dimensional vector space $\mathfrak{X}/_{\wedge (\mathscr{U}^{-1})+
  K}$, and therefore becomes a vector space over $k$ of dimension 
$$
\dim_k D(\mathscr{U}) = \dim_k \frac{\mathfrak{X}} {\wedge
  (\mathscr{U}^{-1}) + K} = \delta (\mathscr{U}). 
$$

If $\omega_1$ and $\omega_2$ are two differentials divisible by
$\mathscr{U}_1$ and $\mathscr{U}_2$ respectively, their sum is a
linear function on $\mathfrak{X}$ which clearly vanishes on
$\wedge(\mathscr{U}^{-1}) + K$, where $\mathscr{U}$ is
$(\mathscr{U}_1, \mathscr{U}_2)$.\,(The \textit{greatest common divisor}~(g.c.d.) of two divisors $\mathscr{U}_1$ and $\mathscr{U}_2$ is the
divisor $\mathscr{U} = \prod\limits_ \mathscr{Y} \mathscr{Y}^{\min
  (v_\mathscr{Y} (\mathscr{U}_1), v_\mathscr{Y} (\mathscr{U}_2))}$.)
Thus, $\omega_1 + \omega_2$ is a differential divisible by
$\mathscr{U}$. Similarly, for a differential $\omega$ divisible by
$\mathscr{U}$ and an element $X \in K$, we define the differential $X
\omega$ by 
$$
X \omega (\mathscr{C}) = \omega (X \mathscr{C}).
$$

$X \omega $ is seen to be divisible by $(X) \mathscr{U}$. We then obtain
\begin{align*}
  (XY) & \omega = X(Y \omega)\\
  (X+Y) \omega &  = X \omega + Y \omega\\
  X(\omega_1 + & \omega_2) = X \omega_1 + X \omega_2
\end{align*}

It follows that the differentials form a vector space over $K$. We now prove the
\begin{theorem*}
  If\pageoriginale $\omega_0$ is a non-zero differential, every differential can be
  written uniquely in the form $\omega =X \omega_0$ for some $X \in
  K$. In other words, the dimension over $K$ of the space of
  differentials of $K$ is one. 
\end{theorem*}

\begin{proof}
  Let $\omega_0$ be divisible by $\delta_0^{-1}$ and $\omega$ by
  $\delta^{-1}$. Let $\mathscr{U}$ be an integral divisor, to be
  chosen suitably later. The two mappings  
$$
\displaylines{\hfill 
    X_0 \in L (\mathscr{U}^{-1} \delta_0)   \to X_0 \omega_0 \in
    D(\mathscr{U}^{-1})\hfill \cr 
    \text{and} \hfill  X \in L (\mathscr{U}^{-1} \delta)  \to X \omega
    \in D(\mathscr{U}^{-1})\hfill } 
$$
  are clearly k-isomorphisms of $L(\mathscr{U}^{-1} \delta_0 )$ and
  $L(\mathscr{U}^{-1} \delta)$ respectively into
  $D(\mathscr{U}^{-1})$. Hence, the sum of the dimensions of the images
  in $D(\mathscr{U}^{-1})$.  is  
  $$
  l(\mathscr{U}^{-1} \delta_0) + l(\mathscr{U}^{-1} \delta) \ge
  2d(\mathscr{U}) - d(\delta) - d(\delta_0) + 2 - 2g, 
  $$
  and is therefore $> \dim D(\mathscr{U}^{-1}) = \delta
  (\mathscr{U}^{-1}) = d(\mathscr{U}) + g-1$ if $\mathscr{U}$ is
  chosen so that $d(\mathscr{U})$  is sufficiently large. With such a
  choice of $\mathscr{U}$, therefore, we see that the images must have
  a non-zero intersection in $D(\mathscr{U}^{-1})$. Hence, for some
  $X_0$ and $X$ different from zero, we must have 
  $$
  X_0 \omega_0 = X \omega , \omega = X_0 X^{-1} \omega_0 = Y \omega_0 , Y\in K .
  $$

  The uniqueness is trivial.
\end{proof}

We shall now associate with every differential $\omega$ a divisor. We
require a preliminary 
\begin{lemma*}
  If a differential $\omega$ is divisible by two divisors
  $\mathscr{U}$ and $\delta$, it is also divisible by the least common
  multiple $\mathscr{L}$ of $\mathscr{U}$ and $\delta$ (Definition\pageoriginale of
  l.c.m. obvious). 
\end{lemma*}

\begin{proof}
  Suppose $\mathscr{C} \in \wedge (\mathfrak{L}^{-1})$. Then,
  $$
  v_\mathscr{Y} (\mathscr{C}) \ge -v_{\mathscr{Y}} (\mathfrak{L}) =
  -\max (v_\mathscr{Y}(\mathscr{U}), v_\mathscr{Y} (\mathfrak{z})) . 
  $$

  Define two repartitions $\mathscr{C}^1$ and $\mathscr{C}''$ by the equations
  \begin{align*}
    \mathscr{C}^1_{\mathscr{Y}_1}  & = \mathscr{C}_\mathscr{Y} ,
    \mathscr{C}''_\mathscr{Y} = 0 ~ \text{for all} ~ \mathscr{C} ~
    \text{such that}~ v_\mathscr{Y}(\mathscr{U}) \ge v_\mathscr{Y}
    (\delta)\\ 
    \mathscr{C}^1_\mathscr{Y} & = 0, \mathscr{C}''_\mathscr{Y} =
    \mathscr{C}_\mathscr{Y}~\text{for all}~ \mathscr{Y} ~\text{such
      that}~ v_\mathscr{Y}(\mathscr{U}) < v_\mathscr{Y} (\delta). 
  \end{align*}
  $\mathscr{C}^1$ and $\mathscr{C}''$ are by the above definition
  divisible by $\mathscr{U}^{-1}$ and $\mathscr{Y}^{-1}$ respectively,
  and $\mathscr{C} = \mathscr{C}^1 + \mathscr{C}''$. Hence, 
  $$
  \omega (\mathscr{C}) = \omega(\mathscr{C}^1) + \omega(\mathscr{C}'') = 0 .
  $$

  Since $\omega$ must vanish on $K, \omega$ must vanish on
  $\wedge(\mathfrak{L}^{-1}) + K$. Our lemma follows 
\end{proof}

\begin{theorem*}
  To any differential $\omega \neq 0$, there corresponds a unique
  divisor $(\omega)$ such that $\omega$ is divisible by $\mathscr{U}$
  if and only if $(\omega)$ is divisible by $\mathscr{U}$. 
\end{theorem*}

\begin{proof}
  Suppose $\omega$ is divisible by a divisor $\mathscr{U}$. Then, the
  mapping $X \in L(\mathscr{U}^{-1}) \to X\omega \in D (\mathscr{N})$ is
  clearly a k-isomorphism of $L(\mathscr{U}^{-1})$ into
  $D(\mathscr{N})$. Hence, we deduce that 
  $$
  l(\mathscr{U}^{-1}) \le \dim D (\mathscr{N}) = \delta (\mathscr{N})
  = l(\mathscr{N}) + d(\mathscr{N}) + g -1 = g . 
  $$

  On other hand, we have
  $$
  l(\mathscr{U}^{-1}) + d(\mathscr{U}^{-1}) = 1 - g + \delta
  (\mathscr{U}) \ge 2 - g, 
  $$
  since $\delta(\mathscr{U}) \ge 1$. Combining these two inequalities,
  we deduce that 
  $$
  d(\mathscr{U}) \le 2 g - 2 .
  $$\pageoriginale\

  This proves that the degrees of all divisors dividing a certain
  differential $\omega$ are bounded by $2g - 2$. 
\end{proof}

Now, choose a divisor $(\omega)$ dividing $\omega$ and of maximal
degree. If $\mathscr{U}$ were any another divisor of $\omega$, the
least common multiple $\delta$ of $\mathscr{U}$ and $(\omega)$ would
have degree at least that of $(\omega)$, and would divide $\omega$ by
the above lemma. Hence, we deduce that  $\delta = (\omega)$ or that
$\mathscr{U}$ divides $(\omega)$. 

The uniqueness of $(\omega)$ also follows from this. Our theorem is proved.

\setcounter{corollary}{0}
\begin{corollary}\label{chap8:sec14:coro1}%corollary 1
  If $X \in K^*, (X \omega) = (X) (\omega)$. This follows from the
  easily verified fact that $\mathscr{U}$ divides $\omega$ if and only
  if $(X) \mathscr{U}$ divides $X \omega$. 
\end{corollary}

This corollary, together with the theorem that the space of
differentials is one dimensional over $K$, proves that the divisors of
all differentials form a class $W$. This class is called the
\textit{canonical class} 

\section{The Riemann-Roch theorem}\label{chap8:sec15}%sec 15 

Let $C$ be a class and $\mathscr{U}$ any divisor of $C$. If
$\mathscr{U}_i (i=1, \ldots , n)$ are elements of $C, \mathscr{U}_i
\mathscr{U}^{-1} = (X_i)$ are principal divisors. We shall say that
the divisors $\mathscr{U}_i$, are linearly independent if $X_i(i=1,
\ldots , n)$ are linearly independent over $k$. This does not depend
on the choice of $\mathscr{U}$ or of the respectively $X_i$ of
$\mathscr{U}_i \mathscr{U}^{-1}$, as is easy to verify. We now prove
the  
\begin{lemma*}
  The dimension $N(C)$ of a class $C$ is the maximum number of
  linearly\pageoriginale independent integral divisors of $C$. 
\end{lemma*}

\begin{proof}
  Let $\mathscr{U}$ be any divisor of $C$. Then, the divisors $(X_1)
  \mathscr{U}, (X_2) \mathscr{U}, \ldots$, $(X_n) \mathscr{U}$ are
  linearly independent integral divisors of $C$ if and only if $X_1,
  \ldots ,X_n$ are linearly independent elements of $L
  (\mathscr{U}^{-1})$. Our lemma follows. 
\end{proof}

We now prove the celebrated theorem of Riemann-Roch.
\begin{theorem*}%the 0
  If $C$ is any divisor class,
  $$
  N(C) = d (C) - g + 1 + N (WC^{-1}).
  $$
\end{theorem*}

\begin{proof}
  Let $\mathscr{U} \in C$. Then,
  $$
  N (C) = l(\mathscr{U}^{-1}) = d (\mathscr{U}) - g + 1 + \delta
  (\mathscr{U}) = d(C) - g + 1 + \delta (\mathscr{U}). 
  $$
\end{proof}

But $\delta (\mathscr{U})$ being the dimension of $D(\mathscr{U})$ is
the maximum number of linearly independent differentials divisible by
$\mathscr{U}$. Hence, it is the maximum number of linearly independent
differentials $\omega_1 , \ldots, \omega_n$ such that $(\omega_1)
\mathscr{U}^{-1}, (\omega_2) \mathscr{U}^{-1}, \ldots, (\omega_n)
\mathscr{U}^{-1}$ are integral. By the above lemma, we conclude that
$\delta (\mathscr{U}) = N (WC ^{-1})$, and our theorem is proved. 

\noindent \textbf{Corollaries}
$E$ and $W$ shall denote principal and canonical classes respectively.

\begin{enumerate}[(a)]
\item $N(E) = l (n) = 1, d(E) = d (N) = 0$.

  $1 = N(E) = d (E) - g + 1 + N(W) \Longrightarrow N(W) = g$.
  
  $g = N(W) = d(W) - g +  1 + N(E) \Longrightarrow d(W) = 2g - 2$.

\item If $d (C) < 0$, or if $d (C) = 0$ and $C \neq E, N(C) = 0$.

  If\pageoriginale $d(C) > 2g - 2$ or if $d(C) = 2g - 2$ and $C \neq W$,
  $$
  N(C) = d (C) - g + 1
  $$

  \begin{proof}
    Suppose $N(C) > 0$. Then there exists an integral divisor
    $\mathscr{U}$ in $C$, and hence $d(C) = d (\mathscr{U}) \ge 0$,
    equality holding if and only if $\mathscr{U} = N$ or $C = E$. 
  \end{proof}

  The second part follows immediately on applying that first part to
  the divisor $WC^{-1}$. 
\item If $W^1$ is a class and $g^1$ an integer such that 
  $$
  N(C) = d (C) - g^1 + 1 + N(W^1 C^{-1}),
  $$

  We must have $W = W^1$ and $g=g^1$.
\end{enumerate}

\begin{proof}
  Exactly as in $(a)$, we deduce that $N(W^1) = g^1, d(W^1) =
  2g^1-2$. Again as in the second part of $(b)$, we deduce that if
  $d(C) > 2g^1 - 2, N(C) = d(C) - g^1 + 1$. Hence, for $d(C) > \max
  (2g - 2, 2g^1 - 2)$, 
  $$
  N(C) = d(C) - g + 1 = d(C) - g^1 + 1, g = g^1.
  $$

  Hence $N(W^1) = g$ and $d(W^1) = 2g - 2$, and it follows from the
  second part of $(b)$ that $W = W^1$. 
\end{proof}

This shows that the class $W$ and integer $g$ are uniquely determined
by the Riemann-Roch theorem. 

Let us give another application of the Riemann-Roch theorem. We shall
say that a \textit{ divisor $\mathscr{U}$ divides a class $C$ } if it
divides every integral divisor of $C$. We then have the following 
\begin{theorem*}
  If $C$ is any class and $\mathscr{U}$ an integral divisor,
  $$
  N (C) \ge N (C\mathscr{U}) \le N (C) + d (\mathscr{U})
  $$

  The first inequality becomes an equality if and only if
  $\mathscr{U}$ divides the class $C\mathscr{U}$, and the second if and only if
  $\mathscr{U}$ divides the class $WC^{-1}$. 
\end{theorem*}

\begin{proof}
  Since\pageoriginale the maximum number of linearly independent integral divisors
  in $C \mathscr{U}$ is clearly greater than or equal to the number
  of such divisors
  in $C$, the first part of the inequality follows. Suppose now that
  equality prevails. Then there exists a maximal set $\delta_1, \ldots
  \delta_n$ of linearly independent integral divisors in $C$, such
  that $\mathscr{U} \delta_1, \ldots \mathscr{U} \delta_n$ forms such
  a set in $C \mathscr{U}$. But since every integral divisor in $C
  \mathscr{U}$ is `linear combination of divisors of such a set' with
  coefficients in $k$ (in an obvious sense), every integral divisor of
  $C \mathscr{U}$ is divisible by $\mathscr{U}$. 
\end{proof}

Now, by the theorem of Riemann-Roch,
$$
N(C \mathscr{U}) = d (C) + d(\mathscr{U}) + 1 - g + N
(WC^{-1}\mathscr{U} ^{-1}), 
$$
and the second inequality together with the condition of equality
follows by applying the first to $WC^{-1} \mathscr{U}^{-1} $ instead
of $C$. 

\begin{coro*}%coro 0
  For any class $C, N(C) \ge \max (0, d(C) + 1)$.
\end{coro*}

\begin{proof}
  If N(C) = 0, there is nothing to prove, if $N(C) > 0$, there exists
  an integral divisor $\mathscr{U}$ in  $C$. Hence we obtain 
  $$
  N(C) = N (E \mathscr{U}) \le N(E) + d (\mathscr{U}) = d(\mathscr{U})
  + 1 = d(C) + 1, 
  $$
  and our corollay is proved.
\end{proof}
