\chapter{Lecture 17}\label{chap17}%chap17

\setcounter{section}{29}
\section{\texorpdfstring{$L$}{L}-functions Modulo \texorpdfstring{$\mathcal{F}$}{F}}\label{chap17:sec30}%Sec 30

Throughout\pageoriginale this lecture, we shall assume that $K$ is an algebraic
function field over a finite constant field $k$ with $q = p^f$
elements. Let $\mathcal{F}$ be an integral divisor of $K$ and
$\mathcal{X}$ a character modulo $\mathcal{F}$. For a complex variable
$s = \sigma + $ it $\sigma > 1$, we define the L-function w.r.t. the
character $\mathcal{F}$ by the infinite series  
$$
L(s, \mathcal{X} ) = \sum_{\mathscr{U}} \frac{\mathcal{X}
  (\mathscr{U})}{(N \mathscr{U})^s}, 
$$
the summation being over all integral divisors. The absolute
convergence, etc. of the series do not offer any difficulty to prove,
and we may also get the following product formula: 
$$
L (s, \mathcal{X}, K) = \prod_{\mathscr{Y}} \left(1- \frac{\mathcal{X}
  (\mathscr{Y})}{(N \mathscr{Y})^s}\right)^{-1} \text{ for } \sigma > 1. 
$$

\section{The Functional Equations of the \texorpdfstring{$L$}-functions.}%Sec 31

Before proceeding to the proof of the functional equation of the
$L$-functions, we shall prove some essential lemmas. Since it is
possible to prove these in a general setting, we shall do so. 

Let $A$ be a commutative algebra with unit element $1$ of finite rank
$f$ over a finite $k$ with $q$ elements. We shall assume that a
mapping $\mathcal{X}$ of $A$ into the complex number is given with the
following properties: 
\begin{enumerate}[1)]
\item $\chi (X)=0$\pageoriginale if and only if $X$ is a zero divisor is $A$
\item $\chi (XY)=\chi (X)\chi(Y)$ for $X,Y\in A$
\item $\chi (\alpha X)=\chi (X)$ for $X \in A$ and $\alpha \in k^*$
\item If $X$ is a zero divisor of $A$, there exists a $Y$ in $A$ such
  that $XY=X$ and $\chi (Y)\neq 0, \chi(Y)\neq 1$. 
\end{enumerate}

We shall also assume that a $k$-linear mapping $S$ of $A$ in $k$ is
given which is such that if $X \in A, X\neq 0$, There exists a $Y \in
A$ such that $S(XY)=1$. Lastly, we assume that a complex valued
function $\psi$ is given on $k$ such that $(a)\, \psi (0)=1$ and
$(b)\sum\limits_{\alpha \in k} \psi (\alpha)=0$. $(\text{e.g,}\, \psi
(0))=1,\psi (\alpha)=-\dfrac{1}{q-1}$ for $\alpha \neq 0$ satisfies
the required conditions). We then have the following lemmas. 

\setcounter{Lemma}{0}
\begin{Lemma}\label{chap17:sec31:lem1}%Lem 1
  Let $V$ be a vector subspace of $A$. Then,
\end{Lemma}
$$
\sum_{X \in V}\psi (S(XY))= 
\begin{cases}
  q^{\dim V} \text{ if }Y \in V_{comp}.\\ 
  0 \qquad \text{ otherwise }.
\end{cases}
$$

\begin{proof}
  If $Y \in V_{comp},S(X, Y)=0$ for every $X \in V$ and the first
  equality follows from condition (a) for $\psi$. 
\end{proof}

If $Y \notin V_{comp}$, we can find $X_1 \in V$ such that
$S(X_1Y)=1$. Complete $X_1$ to a basis $X_1, \ldots X_d$ of $V$ over
$k$. Then the sum on the left is 
$$ 
\sum_{\alpha_1 ,\ldots,\alpha_d \in K} \psi \left(S\left(\sum^d_{i=1}\alpha_i
X_i Y\right)\right) =\sum_{\alpha_1 ,\ldots,\alpha_d \in
  K}\sum_{\alpha_1 \in K}\psi
\left(\alpha_1+S \left(\sum^d_{i=2} \alpha_1 X_i Y\right)\right)=0, 
$$
since for fixed $\alpha_2,\ldots \alpha_d$, the sum
$\alpha_1+S(\sum^d_{i=2}\alpha_i X_i Y)$ runs through all elements of
$k$ when $\alpha_1$ does. Our lemma is proved. 

\begin{Lemma}\label{chap17:sec31:lem2}%Lem 2
  Define\pageoriginale the generalised Gaussian sum $G(X, \chi)$ for $X \in A$ by
  $$
  G(X,\chi)=\sum_{Y\in A}\chi (Y)\psi (S(XY)).
  $$
  
  Then we have
  $$
  G(X,\chi)=\overline{\chi (X)} G(1,\chi)
  $$
\end{Lemma}

\begin{proof}
  Since $A$ is a finite algebra every non-zero divisor is a unit. (In
  fact, if $u$ is a non-zero divisor, $ua \neq ub$ if $a \neq b$, and
  since $A$ is a finite set, $uA=A$. In particular, there exists and
  element $u^1 \in A$ with $uu^1=1$). 
\end{proof}

Now, if $X$ is a zero divisor, there exists an element $u \in A$ by
$4)$ with $Xu=X$ and $\chi (u)\neq 0$ or $1$; thus, $u$, is a non-zero
divisor. Hence we obtain 
\begin{align*}
  G(X,\chi) &= \sum_{Y \in A} \chi (Y) \psi (S(XuY))= \sum_{Y \in A}
  \chi (Y u^{-1}) \psi (S(XY))\\ 
  &= \chi (u^{-1})G(X,\chi),
\end{align*}
and since $\chi (u^{-1})=\chi^{-1}(u)\neq 1,G(X, \chi)=0 =
\overline{\chi (X)}G(1, \chi)$. If $X$ is not a zero divisor, 
$$
G(X,\chi)=\sum_{Y \in A} \chi(Y) \psi (S(XY))= \sum_{Y \in A} \chi
(X^{-1}Y)\psi (S(Y))=\chi^{-1}(X)G(1,\chi). 
$$

Now, the units of $A$ form a finite group and therefore there
exists an integer $n \neq 0$ with $X^n=1$. Hence $\chi^n(X)=\chi
(X^n)=\chi(1)=1, |\chi(X)|=1$ and
$\chi^{-1}(X)=\overline{\chi(X)}$. The lemma is proved. 

\begin{Lemma}\label{chap17:sec31:lem3}%Lem 3
  Let\pageoriginale $V$ be a subspace of $A$ and
  $$
  M(V,\chi)=\sum_{X \in V}\chi (X).
  $$
  Then,
  $$
  M(V_{comp}, \bar{\chi})=q^{-\dim V}\Delta (\chi)M(V, \chi),
  $$
  where $\Delta (\chi)$ depends only on $\chi$ and $|\Delta (\chi)|=q^{f/2}$
\end{Lemma}

\begin{proof}
  By the first two lemmas, we have
{\fontsize{10pt}{12pt}\selectfont
  \begin{gather*}
    q^{-\dim V}M(V_{comp}, \bar{\chi})=\sum_{X \in
      V_{comp}}\bar{\chi}(X)q^{\dim V}=\sum_{X \in
      A}\bar{\chi}(X)\sum_{X \in V}\psi (S(XY))  \\
    =\sum_{Y \in V}\sum_{X \in A}\bar{\chi}(X)\psi (S(XY))\sum_{Y \in
      V}G(Y, \bar{\chi})=\sum_{Y \in V}\chi(Y)G(1, \bar{\chi})= G(1,
    \bar{X}) M(V, \chi) 
  \end{gather*}}\relax
\end{proof}

Put $\Delta (\chi)=G(1, \bar{\chi})$. It only remains to prove that
$|G(1, \bar{\chi})|=q^{f/2}$. 

Now, $k$ can be imbedded in $A$ by the mapping $\alpha \in k \to
\alpha 1\cdot \in A$. Also, 
$$
M(k,\chi)\sum_{\alpha \in k} \chi (\alpha)= \sum_{\alpha \in k^*}
\chi(1)=q-1 \neq 0. 
$$

Choose $\psi$ to be real. Taking $V=k$ in the above formula, we obtain 
\begin{align*}
  G(1,\chi)G(1,\bar{\chi})M(k,\chi) & = G (1,\chi)q M
  (k_{comp}\bar{\chi}) = q^{\dim k_{comp}+1} M(k, \chi), \\
  G(1,\chi)G(1, \bar{\chi})& =q^f
\end{align*}

But since $\psi$ is real, \quad $G(1,
\bar{\chi})=\overline{G(1,\chi)}$, and therefore 
$$
|\Delta(\chi)|=|G(1, \bar{\chi})|=q^{f/2}
$$
Our\pageoriginale lemma is proved.

Since we had $\Delta (\bar{\chi})=G(1, \chi)$, we see that $G(1,
\chi)$ does not depends on the function $\psi$. By lemma \ref{chap17:sec31:lem2}, it
follows that $G(X, \chi)$ is independent of $\psi$. (This may also be
proved directly.) 

We now proceed to the functional equation

\begin{theorem*}
  Let $\mathcal{F}$ be an integral divisor which is not the unit
  divisor and $\chi$ a proper character modulo $\mathcal{F}$. Then
  $L(s,\chi ,K)$ is a polynomial in $U=q^{-s}$ of degree
  $2g-2+d(\mathcal{F})$ and satisfies the functional equation 
  $$
  q^{s(g-1 +\frac{1}{2}d(\mathcal{F}))}L(s, \chi,K)=\epsilon
  (\chi)q^{(1-s)(g-1 +\frac{1}{2}d(\mathcal{F}))} L(1-s,\bar{\chi},K) 
  $$
  where $\epsilon (\chi)$ is a constant depending on $\chi$, with
  $|\epsilon (\chi)|=1$. 
\end{theorem*}

\begin{proof}
  Let $R$ be the algebra $\Gamma(\mathscr{N}/\mathscr{U}_1, \ldots
  \mathscr{U}_r)$, $i$ the ideal
  $\Gamma(\mathcal{F}/\mathscr{U}_1,\break \ldots \mathscr{U}_r)$ and
  $\bar{R}$ the quotient algebra $R/i$. 
\end{proof}

Now, if $X \in R, X \neq 0$, $\mathscr{N}_X$ is prime of
$\mathcal{F}$ and $\chi((X))$ is meaningful. We shall show that
$\chi((X))$ is constant on the cosets modulo the ideal $i$. 

Let $X,Y \in R,X-Y \in i$. If $(X)$ is not coprime to
$\mathcal{F},(Y)$ cannot be coprime to $\mathcal{F}$ and therefore
$\chi((X))=\chi ((Y))=0$. If $X$ is coprime to $\mathcal{F}$, we have 
$$
v_{\mathscr{U}i}\left(\frac{Y}{X}-1\right)=v_{\mathscr{U}i}(Y-X)\geq
v_{\mathscr{U}i} (\mathcal{F}), 
$$
and hence $\dfrac{Y}{X}\equiv 1(\mod^+ \mathcal{F}),\chi(X)=\chi(Y)$.

Thus, we may define for $\bar{X}\in\bar{R}$ a mapping which again we
shall denote by $\chi$ by the equation 
$$
\chi (\bar{X})=\chi ((X)).
$$\pageoriginale\

Clearly we have $\chi (\bar{X})\chi(\bar{Y})=\chi (\overline{XY})$ and
$\chi(\alpha \bar{X})=\chi (\bar{X})$ if $\alpha \in k^*$. Also, $\chi
(\bar{X})=0$ if and only if $v_{\mathscr{U}_i}(X)>0$ for some
$i$. Find a $Y \in K$ such that 
\begin{gather*}
  v_{\mathscr{U}_i}(Y)=v_{\mathscr{U}_i}(\mathcal{F})-v_{\mathscr{U}_i}(X)\\
  v_{\mathscr{U}_i}(Y)=v_{\mathscr{U}_i}(\mathcal{F}) \text{ for } j \neq i.
\end{gather*}

Then, $\bar{Y} \in \bar{R}, \bar{Y}\neq 0$ and
$\bar{X}\bar{Y}=0$. Thus $\chi (\bar{X})=0$ implies that $\bar{X}$  is
a zero divisor. Conversely, if $\bar{X}$ is a zero divisor, we should
have $v_{\mathscr{U}_i}(X)>0$ for some $i$ and therefore $\chi
(\bar{X})=0$. 

Now, suppose $\bar{Z}$ is a zero divisor in $\bar{R}$. Then
$v_{\mathscr{U}_i}(Z)>0$ for some $i$. Since $\chi$ is proper
character modulo $\mathcal{F},\mathcal{F}_{\mathscr{U}_i}^{-1}$ is not
a modulus of definition of $\chi$. Hence there exists an element $X
\in K^*$, with $(X)$ coprime to $\mathcal{F}, X \equiv 1 (\mod^+
\mathcal{F} \mathscr{U}_i^{-1})$, and $\chi((X))\neq 1$. Then we
have $(X-1)Z \equiv 0(\mod^+ \mathcal{F}), \bar{X}\bar{Z}=\bar{Z}$,
and $\chi(\bar{X})\neq 0$ or $1$. 

Hence, the map $\chi$ on $\bar{R}$ satisfies the conditions 1), 2), 3)
and 4) stipulated at the beginning of this lecture. We have already
seen (Lecture \ref{chap14}) that if $\omega$ is a differential such that
$\mathcal{F}(\omega)$ is coprime to $\mathcal{F}, S(\bar{X})=S(X)=\sum
\limits^r_{i=1}\omega ^{\mathscr{U}_i}(X)$ has all the requisite
properties. Now, 
$$
L(s, \chi ,K)=\sum_C q^{-sd(c)} \sum_{\mathscr{U} \in C}\chi(\mathscr{U}),
$$
where the first summation is over all classes $C$ and the second over
all integral divisors in the class $C$. Choose a divisor
$\mathscr{U}_C$ in the class\pageoriginale $C$ coprime to $\mathcal{F}$. We have 
$$
\sum_{\mathscr{U} \in C}\chi (\mathscr{U})= \frac{\chi
  (\mathscr{U}_C)}{q-1}\sum_{X \in L^* (\mathscr{U}_C^{-1})}\chi
((X)), 
$$
since every integral divisor $\mathscr{U}\in C$ can be written in
precisely $(q-1)$ ways in the form $(X)\mathscr{U}_C$, where $X$ is a
non-zero element of $L(\mathscr{U}_C^{-1})$.  

Again, since $\mathscr{U}_C$ is coprime to $\mathcal{F},
L(\mathscr{U}_C^{-1})\subset R$ and two elements of
$L(\mathscr{U}_C^{-1})$ go to the same coset modulo $i$ if and only
if their difference in $L(\mathscr{U}_C^{-1})\mathcal{F}$. We
therefore have 
$$
\sum_{\mathscr{U} \in
  C}\chi(\mathscr{U})=\frac{\chi(\mathscr{U}_C)}{q-1}
q^{l(\mathscr{U}_C^{-1}\mathcal{F})}\frac{\sum \chi
  (\bar{X})}{\bar{X}\in L(\mathscr{U}_C^{-1})} 
$$

Applying lemma \ref{chap17:sec31:lem3} with $V=\overline{L(\mathscr{U}_C^{-1})}$, we have
$$
\sum_{\bar{X}\in L(\mathscr{U}_C^{-1})}\chi
(\bar{X})=M(V,\chi)=\Delta^{-1}(\chi)q^{\dim V}M(V_{comp},
\bar{\chi}). 
$$

But by the theorem of lecture \ref{chap14}, we have
$$
\overline{L(\mathscr{U}_C^{-1})}_{compl.} =\overline{L(\mathscr{U}_C
  \mathcal{F}(\omega))} 
$$

Now, if $d(C)<0, N(C)=l(\mathscr{U}_C^{-1})=0$ and hence
$M(V,\chi)=0$. Also, if $d(C)>2g-2+d(\mathcal{F}),
d(C^{-1}\mathcal{F}W)<0$, and $N(C^*)=l(\mathscr{U}_C
\mathcal{F}(\omega))=0$; hence again $M(V, \chi)=0$. Substituting in
the expression for $L(s, \chi, K)$, we see that it is a polynomial in
$U=q^{-s}$ of degree at most $2g-2+d(\mathcal{F})$. 

We\pageoriginale have
\begin{align*}
  & (q-1)L(s, x, K)=\sum_C q^{-sd(C)}\chi
  (\mathscr{U}_C) q^{l(\mathscr{U}_C^{-1}\mathcal{F})}
  M(\overline{L(\mathscr{U}_C^{-1})}, \chi) \\
  &= \frac{1}{\Delta (\chi)}\sum_C q^{-sd (C)+N(C)}\chi
  (\mathscr{U}_C)M\overline{(L(\mathscr{U}_c^{-1})}_{comp}\bar{\chi})
  \qquad \text{ by lemma \ref{chap17:sec31:lem3}},\\ 
  &=\frac{1}{\Delta (\chi)}\sum_C q^{-sd (C)+N(C)}\chi
  (\mathscr{U}_C)M (L((\mathscr{U}_c^{-1})^*)\bar{\chi})\\ 
  & \hspace{3cm}\text{by the theorem of lecture \ref{chap14},} \\
  & = \frac{1}{\Delta (\chi)} \sum q^{(1-s)d(C)-g+1+N(WC^{-1})} \bar{\chi}
  (\mathscr{U}_C^{-1})M(L(\mathscr{U}_C^{-1})^*),\bar{\chi})\\  
  &=\frac{\chi(\omega)\mathcal{F}}{\Delta(\chi)}q^{(2s-1)(1-g)+
    (1-s)d(\mathcal{F})}\\
  & \quad \times 
  \sum_C q^{(1-s)d(C^*)} \bar{\chi}((\mathscr{U}_C^{-1})^{*-1})
  q^{l(\mathscr{U}_C^{-1})\mathcal{F}^*)} M(L((\mathscr{U}_C^{-1})^*),\chi)\\ 
  &=(q-1) \frac{\chi((\omega)\mathcal{F})}{\Delta(\chi)}
  q^{(2s-1)(1-g) + (1-s)d(\mathcal{F})} L(1-s, \bar{\chi}, K), ~\text{ since}~C^*
\end{align*}
runs through all classes as $C$ does. This gives
$$
\displaylines{\hfill 
  q^{s(g-1+\frac{1}{2}d(\mathcal{F}))} L(s,\chi,K)= \epsilon
  (\chi)q^{(1-s)(g-1+ \frac{1}{2}d(\mathcal{F}))}L(1-s,
  \bar{\chi},K),\hfill \cr
  \text{where}\hfill 
  \epsilon(\chi)=\dfrac{\chi((\omega)\mathcal{F}) q^{\dfrac{1}{2}
    d(\mathcal{F})}}{\Delta(\chi)}, |\epsilon (\chi)|=1.\hfill } 
$$
That the degree in $U$ of $L(s, \chi, K)$ is exactly
$2g-2+d(\mathcal{F})$ follows by comparing the coefficients of highest
powers on both sides of the equation. Our theorem is proved. 
