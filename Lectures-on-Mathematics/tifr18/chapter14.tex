\chapter{Lecture 14}\label{chap14}%cha14

\setcounter{section}{26}
\section{A Consequence of the Riemann-Roch Theorem}\label{chap14:sec27}%sec 27

In\pageoriginale this lecture, we  shall prove a theorem which will be crucial in
the proof of the functional equation of the $L$-functions modulo an
integral divisor. 

Let $ \mathcal{F} $ be an integral divisor and $\mathscr{U}_1, \ldots,
\mathscr{U}_r$ 
the distinct prime divisors occurring in it. We prove a series of
lemmas leading upto the proof of the theorem we mentioned above.  

\setcounter{Lemma}{0}
\begin{Lemma}\label{chap14:sec27:lem1} %lem 1
  In  any class $C$, there exists a divisor $ \mathscr{U} $ such that
  $ v_{\mathscr{U}_{\nu}}(\mathscr{U}) = 0 ( \nu = 1 , \cdots r ) $. 
\end{Lemma}

\begin{proof}
  Let $ \mathscr{U}_\circ $ be any divisor of $C$. By the independence
  theorem for valuations, we may find $ X \in K $ with  
  $$
  v_{\mathscr{U}_{\nu}} (X) = - v_{\mathscr{U}_{\nu}} ( \mathscr{U}_\circ )
  $$
  Then $ \mathscr{U} = (X) \mathscr{U}_\circ \in C $ satisfies the
  required conditions. 
\end{proof}

Let us denote by $R$ the vector space $ \Gamma ( n / \mathscr{U}_1 $,
$ \mathscr{U}_r ) $ and by $i$ the subspace $ \Gamma ( \mathcal{F} /
\mathscr{U}_1, \ldots \mathscr{U}_r ) $. $R$ is not only a vector
space over $k$, but also an algebra. In fact, if $ X,Y \in R $, 

$v_{\mathscr{U}_{i}} (XY) = v_{\mathscr{U}_{i}} (X) +
v_{\mathscr{U}_{i}} (Y)  \ge 0 $, $ XY \in R $. $ ( i = 1, \dots r )~$
$i$ is an ideal of $R$, since $X \in R$,  $ Y \in i $ implies that  
$$
v_{\mathscr{U}_{i}} (XY) = v_{\mathscr{U}_{i}} (X) +
v_{\mathscr{U}_{i}} (Y)  \ge v_{\mathscr{U}_{i}} ( \mathcal{F}), XY
\in i. 
$$

Thus,\pageoriginale the quotient $ \bar{R} = R/i$ is an algebra of 
$$
\rank  \dim_k =
\dfrac{\Gamma (n / \mathscr{U}r) ~ \mathscr{U}r ) }{\Gamma (
  \mathcal{F} / \mathscr{U}_1 ~ \mathscr{U}_r)} = d ( \mathcal{F} ) 
$$
over the  field $k$. 

Now, choose a differential $ \omega $ such that the divisor $
\mathcal{F} ( \omega ) $ is coprime to $ \mathcal{F} $. This is
possible by Lemma \ref{chap14:sec27:lem1}. We shall assume this  $\omega$ to be chosen and
fixed throughout the discussion. We define a linear function $S$ on
the algebra $R$ by  
$$
S (X) = \sum^{r}_{\nu = 1} \omega^{\mathscr{U}_{\nu}} (X), ~ X \in R .
$$

$S$ vanishes on the ideal $i$, for if $ X \in i
$, $ v_{\mathscr{U}_{\nu}} (X) \ge v_{\mathscr{U}_{\nu}} ( \mathcal{F}
) $, and since $ \mathcal{F} ( \omega) $ is  corprime to each $
\mathscr{U}_\nu $, $ v_{\mathscr{U}_{\nu}} ( \mathcal{F}) +
v_{\mathscr{U}_{\nu}} (( \omega ))  = 0 $, $ v_{\mathscr{U}_{\nu}} (
\mathcal{F}) = -v_{\mathscr{U}_{\nu}} (( \omega )) $. By the last
lemma of the previous lecture, we deduce that   
$$
S (X) = \sum^{r}_{\nu=1} \omega^{\mathscr{U}_\nu} (X) = 0 .
$$

Thus, $S$ induces a linear map from the quotient $ \bar{R} $ to the
field $k$. This in turn gives rise to a bilinear form on $ \bar{R} $
defined by $ ( \bar{X}, \bar{Y} ) \rightarrow S ( \bar{XY} ) $. Our
next lemma states that this is non - degenerate.  

\begin{Lemma}\label{chap14:sec27:lem2}
  If $ \bar{X} \in \bar{R} $, $ \bar{X} \neq 0 $, there exist a $
  \bar{Y} \in \bar{R} $ such that $ S ( \bar{X}. \bar{Y} ) = 1 $. 
\end{Lemma}

\begin{proof}
  Let $ X $  be any element of the coset $ \bar{X} $. Since $ \bar{X}
  \neq 0 $, we have $ X \not \in i $ and $
  v_{\mathscr{U}_{\nu}} (X)  < v_{\mathscr{U}_{\nu}} ( \mathcal{F} ) =
  -v_{\mathscr{U}_{\nu}} (( \omega )) $ for some $ \nu $. Thus, $ X
  \omega $ is a differential with $ v_{\mathscr{U}_{\nu}} (X \omega )
  < 0 $. By the last lemma of the previous lecture, we deduce that
  there  exists an element $ Y_1 \in K $ such that  $
  v_{\mathscr{U}_{\nu}} (Y_1) \ge 0 $ and $ ( X \omega
  )^{\mathscr{U}_\nu} ( Y_1 ) \neq 0 $.  
\end{proof}

Find\pageoriginale $ Y_2 \in K $ such that 
\begin{gather*}
  v_{\mathscr{U}_{\nu}} ( Y_2 -Y_1 ) \ge \max ( 0, -
  v_{\mathscr{U}_{\nu}} (( X \omega ))), \\ 
  v_{\mathscr{U}_{\nu}} (Y_2 ) \ge \max ( 0, -v_{\mathscr{U}_{\mu}} ~
  (( X \omega ))). \text{ for } \mu \neq \nu .  
\end{gather*}

Since we also have
$$
 v_{\mathscr{U}_{\nu}} (Y_2) \ge \min ( v_{\mathscr{U}_{\nu}} ( Y_2
 -Y_1 ) ,  v_{\mathscr{U}_{\nu}} (Y_1)) \ge 0 , 
 $$
 it follows that $ Y_2 \in \bar{R} $. Also, for any $ \mu \neq \nu $,
 we have by the second condition, $ (X \omega )^{\mathscr{U}_\mu}
 (Y_2) = 0 $. Again, it follows from the first condition that  
 $$
(X \omega ) ^{\mathscr{U}_\nu} ( Y_2 - Y_1 ) = 0, ( X \omega )
 ^{\mathscr{U}_\nu} (Y_2) = (X \omega) ^{\mathscr{U}_\nu} (Y_1) \neq 0  
$$ 
Thus, 
$$
S ( \bar{X} \bar{Y}_2 ) = \sum^{r}_{\lambda=1}
\omega^{\mathscr{U}_\lambda} (XY_2 ) = \sum^{r}_{\lambda=1} (X \omega
)^{\mathscr{U}_\lambda } (Y_2) = (X \omega )^{\mathscr{U}_\nu} (Y_2) ~
=  \varrho \neq 0, 
$$
and $ \bar{Y} = \bar{Y}_{2_{\bar{\varrho}}} $ satisfies the conditions
of our lemma. 

Now, let $ \mathscr{U} $ be any divisor coprime to $ \mathcal{F}
$. Then, we assert that $ L (\mathscr{U}) $ is a subspace of $R$. In
fact, we have for  $ X \in L (\mathscr{U}) $, $v_{\mathscr{U}_{\nu}}
(X) \ge v_{\mathscr{U}_{\nu}}  (\mathscr{U}) = 0 $. Also, we assert
that  $ L ( \mathscr{U} ) \cap i = L ( \mathscr{U}
\mathcal{F} ) $. This follows from the following argument $X \in L (
\mathscr{U} \mathcal{F}) \Longleftrightarrow v_{\mathscr{U}} (X) \ge ~
v_\mathscr{U} (\mathscr{U}) + v_\mathscr{U} ( \mathcal{F}) $  for
every $\mathscr{U} \Longleftrightarrow v_{\mathscr{U}} (X) \geq
v_{\mathscr{U}}(\mathscr{U})$ for every $\mathscr{U}$ and
$v_{\mathscr{U}_v} (X) \ge v_{\mathscr{U}_{\nu}}
( \mathcal{F} ) $ for $ ( \nu = 1, ~~ , r ) $. Hence, if we denote by
$ \overline{ L ( \mathscr{U} )} $ the image of $ L ( \mathscr{U} ) $
under the natural homomorphism form $ R $ to  $ \bar{R} $, we have  
\begin{multline*}
  \dim_k (\overline{L (\mathscr{U})}) = \dim_k  \left(
  \frac{L(\mathscr{U})+i}{i} \right) = \dim
  \frac{L(\mathscr{U})}{L(\mathscr{U})\cap i}\\ 
  = \dim_k
  \frac{L(\mathscr{U})}{L( \mathscr{U} \mathcal{F})} = l (\mathscr{U}) -
  l ( \mathscr{U} \mathcal{F}). 
\end{multline*}\pageoriginale\

This proves that the  dimension of $ \overline{L (\mathscr{U})} $ depends
only on the class of $ \mathscr{U} $. Since there are divisors in any
class prime to $ \mathcal{F} $, we may define $N_0 (C) $ for a class
$C$ to be $ \dim_k  \overline{L (\mathscr{U})} $ for any $ \mathscr{U}
\in C^{-1} $ coprime to  $ \mathcal{F} $. 

For any class $C$, we shall call the class $ C^* = WC^{-1} \mathcal{F}
$ the \textit{ complementary class } of $C$ modulo $ \mathcal{F} $. We
then prove the following  

\begin{Lemma}\label{chap14:sec27:lem3}%lem 3
  For  any class $C$, we have 
  $$
  N_0 (C) + N_0 (C^*) = d ( \mathcal{F} ).
  $$
\end{Lemma}

\begin{proof}
  Let $ \mathscr{U} \in C^{-1} $ and $ \delta \in C^{* -1} $ be prime
  to $ \mathcal{F} $. Then,  
  \begin{align*}
    N_0 (C) + N_0 (C^*) &=  \dim \overline{L ( \mathscr{U} )}+ \dim
    \overline{L ( \delta )} \\ 
    &= (l (\mathscr{U}) - l ( \mathscr{U} \mathcal{F} )) + (l (\delta)
    - l ( \delta \mathcal{F}) \\ 
    &= N (C) - N ( C \mathcal{F}^{-1} ) + N (WC^{-1} \mathcal{F} ) - N
    ( WC^{-1} ) \\ 
    &= (d (C) - g +1 ) - (d( C \mathcal{F}^{-1} ) - g+1 ) = d
    (\mathcal{F} ) 
  \end{align*}
  by the Riemann-Roch theorem. Our lemma is proved.
\end{proof}

Now, let $V$ be a vector space on a field $k$ and $ B : V x V
\rightarrow K $ a non-degenerate bilinear form on $V$. ( A bilinear
form is said to be non-degenerate if for every $ X \neq 0 $, there
exist a  $ X_1 \in V $ such that $ B (X,X_1 ) \neq 0 $ and for every $
Y \neq 0 $ there exists a $ Y_1 \in V $ such that $ B ( Y_1,Y ) \neq 0
$. Let $V_1$ be a subspace of $V$. Then  we define \textit{the
  complementary subspace of } $ V_1 $ with respect to the bilinear form
$B$ to the space $ V_{1_{comp}} $ of all elements $ Y \in V $ such
that\pageoriginale $ B ( X, Y ) = 0 $ for every $X \in V_1 $. We then have the   

\begin{Lemma}\label{chap14:sec27:lem4}%lem 4
  Let $V$ be a finite dimensional vector space and $B$ $a$;
  non-degenerate bilinear form on $V$. Then, if $V_1$ is a subspace of
  $V$, we have  
  $$
  \dim_k V_1 + \dim_k V_{1_{comp.}}  = \dim V .
  $$
\end{Lemma}

\begin{proof}
  Let $V^*$ be the dual of $V$. We can define a homomorphism $ \varphi
  : V \rightarrow V^* $ which takes an element $ X \in V $ ot the
  linear map $\varphi (X) \in V^* $ defined by  
  $$
  \varphi (X) (Y) = B( Y,X )
  $$
  Since $B$ is non-degenerate, $ \varphi (X) \neq 0 $ if $ X \neq 0 $,
  and $ \varphi $ is a monomorphism. Since $V$ is finite dimensional,
  $ \dim V^* = \dim V = \dim \varphi (V) $. Thus, $ \varphi $ is also
  an epimorphism.  
\end{proof}

Now, let $ V_2 $ be the complementary subspace of $ V_1 $. Then, every
element of  $ \varphi (V_2) $ is  a linear map of $ V $ into $k$ which
vanishes on the subspace $V_1 $, and  conversely every element of $V^*$
which vanishes on $ V_1 $ should be of the form $ \varphi (X) $, where
$ X \in V_2 $. Hence, we deduce that $ \varphi (V_2 ) $ is isomorphic
to the dual of the quotient space $ V / V_1 $. Therefore, 
\begin{gather*}
  \dim_k V_2 = \dim_k \varphi  (V_2) = \dim_k ( V/V_{1} )^*  = \dim_k V
  - \dim_k V_1 ,\\ 
  \dim_k V_1 + \dim_k V_2 = \dim_k V.
\end{gather*}

The lemma is proved.

Now,\pageoriginale let $\mathscr{U}$ be a divisor prime to $ \mathcal{F} $ and $
\omega $ the ( already chosen and fixed ) differential such that $
(\omega) \mathcal{F} $ is prime to $ \mathcal{F} $. We define \textit{
  the complementary divisor} $ \mathscr{U}^* $ of $\mathscr{U}$ as the
divisor $ (\omega)^{-1} \mathcal{F}^{-1} \mathscr{U}^{-1} $. It is
clear that $ \mathscr{U}^* $ is also prime to $ \mathcal{F} $  and
that if $ \mathscr{U}^{-1} \in C $, $ \mathscr{U}^{* -1} \in ~ C^*
$. We may then state our theorem as follows. 

\begin{theorem*}
  The complementary space of $ \overline{L (\mathscr{U})} $ in $
  \bar{R} $ with respect to the  bilinear form $ B ( \bar{X}, \bar{Y}
  ) = S (XY) $ defined on $ \bar{R} $ is $ \overline{L(\mathscr{U}^*)}
  $.  
\end{theorem*}

\begin{proof}
  Suppose $ \bar{X} \in  \overline{L (\mathscr{U})} $ and $ \bar{Y}
  \in \overline{L (\mathscr{U}^* )} $. Then, for any prime divisor $
  \mathscr{Y} \neq $ any of the  $ \mathscr{U}_\nu $, we have  
  $$
  v_\mathscr{Y} (XY) \ge v_\mathscr{Y} ( \mathscr{U} \mathscr{U}^* ) =
  v_\mathscr{Y} (( \omega) ^{-1} \mathcal{F}^{-1} ) = - v_\mathscr{Y}
  (( \omega)), 
  $$
  and therefore by the lemma of the previous lecture,
  $$
  \omega^\mathscr{Y} ( XY )= 0 
  $$
  Hence, we obtain, 
  $$
  S ( \bar{X} \bar{Y} ) = S ( X Y ) = \sum^{n}_{i=1}
  \omega^{\mathscr{U}} (XY) = \sum_{\mathscr{Y}} \omega^\mathscr{Y}
  (XY) = \omega (XY) = 0, 
  $$
  since $ XY \in K $. Thus, we deduce that 
  $$
  \overline{L(\mathscr{U}^*)} \subset \overline{(( L (\mathscr{U})}_{compl}. 
  $$
\end{proof}

Now, let $C$ and $C^*$ be the classes of $ \mathscr{U}^{-1} $ and $
\mathscr{U}^{* -1} $. We then have 
\begin{align*}
  \dim_k \overline{L( \mathscr{U}^* )} &= N_0 (C^* ) = d ( \mathcal{F}
  ) - N_0 (C) \\ 
  &= \dim_k \bar{R} - \dim_k  \overline{L ( \mathscr{U} )} \\
  &= \dim_k \overline{( L (\mathscr{U} ))}_{compl.}
\end{align*}
since\pageoriginale $ B ( \bar{X}, \bar{Y} ) = S ( \bar{X} \bar{Y} )$ is
non-degenerate by lemma \ref{chap14:sec27:lem2}. Hence, it follows that $ \overline{( L
  (\mathscr{U})}_{comp.}  = \overline{L (\mathscr{U}^*)} $. 
