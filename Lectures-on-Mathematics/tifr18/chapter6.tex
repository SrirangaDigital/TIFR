\chapter{Lecture 6}\label{chap6}

\setcounter{section}{9}
\section{The Space \texorpdfstring{$L (\mathscr{U})$}{LU}}\label{chap6:sec10}%sec 10

Let\pageoriginale $\mathscr{U}$ be any divisor of an algebraic function field $K$.

We shall denote by $L(\mathscr{U})$ the set of all elements of $K$
which are divisible by $\mathscr{U}$. Clearly $L(\mathscr{U}) =
\Gamma(\mathscr{U}/S)$ if $S$ is the set of all prime divisors of $K$,
and thus we deduce that $L(\mathscr{U})$ is a vector space over $k$
and if $\mathscr{U}$ divides $\delta, L (\mathscr{U}) \supset
L(\delta)$. We now prove the following important 
\begin{theorem*}
  For any divisor $\mathscr{U}$, the vector space $L(\mathscr{U})$ is
  finite dimensional over $k$. If we denote its dimension by
  $l(\mathscr{U})$, and if $\mathscr{U}$ divides $\delta$, we have 
  $$
  l(\mathscr{U}) + d(\mathscr{U}) \leq l (\delta) + d(\delta).
  $$
\end{theorem*}

\begin{proof}
  Let $S$ be the set of prime divisors occurring in $\mathscr{U}$ or
  $\delta$.  

  Then, it easy to see that
  $$
  L(\delta) = L(\mathscr{U}) \cap \Gamma(\delta/S)
  $$
  Hence, by Noether's isomorphism theorem,
  $$
  \displaylines{
  \frac{L(\mathscr{U})}{L(\delta)}=\frac{L(\mathscr{U})}{L(\mathscr{U})
    \cap \Gamma(\delta/S)} \simeq \frac{L(\mathscr{U})+ \Gamma
    (\delta/S)}{L(\delta/S)} \subset
  \frac{L(\mathscr{U}/S)}{\Gamma(\delta/S)}, \cr
  \text{and therefore,}\quad  \dim_k \dfrac{L(\mathscr{U})}{L(\delta)} \leq \dim_k
  \dfrac{\Gamma(\mathscr{U}/S)}{\Gamma(\delta/S)}=d(\delta)
  -d(\mathscr{U})\hfill}
  $$

  Now, choose for $\delta$ any integral divisor which is a multiple of
  $\mathscr{U}$ and is not the unit divisor $\mathfrak{n}$. Then, $L(\delta) =
  (0)$; for if $X$ were a non-zero element of $L(~~)$, it cannot be a
  constant since $v_\mathscr{Y}(X) \geq v_\mathscr{Y}(\delta) > 0$ for
  at least one $\mathscr{Y}$, and it cannot be transcendental\pageoriginale over $k$
  since $v_\mathscr{Y}(X) \geq v_\mathscr{Y}(\delta) \geq 0$ for all
  $\mathscr{Y}$. 
\end{proof}

This is not possible. This together with the above inequality proves
that $\dim_k L (\mathscr{U}) = l(\mathscr{U}) < \infty$, and our
theorem is completely proved. 

Since $L (\mathscr{N})$ clearly contains only the constants, $l
(\mathscr{N}) = 1$. 

\section{The Principal Divisors}\label{chap6:sce11}

We shall now associate to every non-zero element of $K$ a divisor. For
this, we need the  
\begin{theorem*}
  Let $X \in K^*$. Then there are only a finite number of prime
  divisors $\mathscr{Y}$ with $v_{\mathscr{Y}}(X) \neq 0$.  
\end{theorem*}

\begin{proof}
  If $X \in k, v_{\mathscr{Y}}(X) =0$ for all $\mathscr{Y}$ and the
  theorem is valid. 

  Hence assume that $X$ is transcendental over $k$. Let $[K:
    k(X)]=N$. Suppose $\mathscr{Y}_1, \ldots \mathscr{Y}_n$ are prime
  divisors for which $v_{\mathscr{Y}_i}(X) > 0$. Let $\delta = \prod
  \limits^n_{i=1} \mathscr{Y}_i v_{\mathscr{Y}_i}(X)$, and $S= \{
  \mathscr{Y}_1, \ldots \mathscr{Y}_n\}$. 
\end{proof}

Then, $\dim_k \frac{\Gamma(\mathscr{N}/S)}{\Gamma(\delta /S)} =
(\delta) = \sum \limits^n_{i =1} f_{\mathscr{Y}_i}
v_{\mathscr{Y}_i}(X)$. We shall show that this is at most equal to
$N$. 

Let in fact $Y_1, \ldots ,Y_{N+1}$ be any $(N+1)$ elements of $\Gamma
(\mathscr{N}/S)$. Since $[K: k(X)] =N$, we should have $\sum
\limits^{N+1}_{j =1} f_j (X) Y_j = 0, f_j (X) \in k [X]$, with at
least one $f_j$ having a non-zero constant term. Writing $f_j(X) =a_j
+ Xg_j (X)$, the above  relation may be rewritten as $\sum
\limits^{N+1}_1 a_j Y_j = -X \sum \limits^{N+1}_1 g_j$ $(X) Y_j$, not
all $aj$ being zero, and hence 
$$
v_{\mathscr{Y}_\nu} \left(\sum^{N+1}_{1} a_j Y_j\right) = v_{\mathscr{Y}_\nu} (X)
+ v_{\mathscr{Y}_\nu} \left(\sum^{N+1}_1 g_j (X) Y_j\right) \ge
v_{\mathscr{Y}_\nu} (X) 
$$\pageoriginale\

This proves that $\sum\limits _{1}^{N+1} a_j Y_j \in \Gamma
(\delta/S)$, and therefore 
$$
n \le \sum^n_{i=1} f_{\mathscr{Y}_i} v_{\mathscr{Y}_I} (X) = d
(\delta) = \dim_k \frac{\Gamma (\mathscr{N}/S)} {\Gamma(\delta/S)} \le
N , 
$$

By considering $\dfrac{1}{X}$ instead of $X$, we deduce that the
number of prime divisors $\mathscr{Y}$ for which $v_{\mathscr{Y}} (X)
< 0$ is also finite, and our theorem follows. 

The method of defining \textit{the divisor} $(X)$ corresponding to an
element $X \in K^*$ is now dear. We define the \textit{numerator}
$\mathfrak{z}_X$ of $X$ to be the divisor $\prod\limits_{v_\mathscr{Y}
  (X) > 0} \mathscr{Y} v_{\mathscr{Y}} (X)$ (the product being taken
over all $\mathscr{Y}$ for which $v_\mathscr{Y} (X) >0$), the
\textit{denominator} $\mathscr{N}_X$ of $X$ to be the divisor
$\prod\limits_{v_\mathscr{Y} (X) < 0} \mathscr{Y}^{-v_{\mathscr{Y} (X)}}$,
and the \textit{principal divisor $(X)$ of $X$} to be
$\prod\limits_{v_y (X) \neq 0} \mathscr{Y}^{v_\mathscr{Y} (X)}
=\dfrac{\mathfrak{z}_X}{\mathscr{N}_X}$. 

If $X, Y \in K^*$, clearly $(XY) = (X)(Y)$ and $(X^{-1}) =
(X)^{-1}$. Thus, the principal divisors form a subgroup $\mathscr{Z}$
of the group of divisors $\vartheta$. The quotient group $\mathfrak{R}
= \dfrac{\vartheta}{\mathscr{Z}}$ is called \textit{the group of
  divisor classes}. The following sequence of homomorphisms is easily
seen to be exact (i.e., the image of a homomorphism is equal to the
kernel of the next). 
$$
1 \to k^* \to K^* \to \vartheta \to \mathfrak{R} \to 1 .
$$

In the course of the proof of the above theorem, we proved the
inequalities $d(\mathscr{N}_x) \le N, d(\mathfrak{z}_x) \le N$, for a
transcendental $X$, where $[K : k(X)] = N$. 

We\pageoriginale will now show that equality holds
\begin{theorem*}%thm
  Let $X$ be a transcendental element of $K$, and put $N = [K :
    k(X)]$. Then, 
  $$
  d(\mathfrak{z}_X) = d(\mathscr{N}_X) = N .
  $$
\end{theorem*}

In order to prove the theorem, we shall first prove a lemma. We shall
say that $Y \in K$ is an \textit{integral algebraic function} of $X$,
if it satisfies a relation 
$$
Y^m + f_{m-1} (X)Y^{m-1} + - + f_o (X) = o, f_i(X) \in k[X] .
$$

We then have the 
\begin{lemma*}%lemma
  If $Y$ is an integral algebraic function of $X$, and a prime
  divisor $\mathscr{Y}$ does not divide $\mathscr{N}_X$, it does not
  divide $\mathscr{N}_Y$. 
\end{lemma*}

\begin{proof}
  Since $\mathscr{Y}$ does not divide $\mathscr{N}_X,
  v_{\mathscr{Y}}(X) \ge 0$, and hence 
  \begin{multline*}
    mv_\mathscr{Y}(Y) = v_\mathscr{Y} (Y^m) = v_\mathscr{Y} (f_{m-1}
    (X) Y^{m-1} +\\ 
    \cdots + f_0 (X)) \ge \min^{m-1}_{\nu = 0} (\gamma
    v_\mathscr{Y} (Y)) = \nu_0 v_\mathscr{Y} (Y),
  \end{multline*}
  for some $\nu_0$ such that $0 \le \nu_0 \le m-1$. This proves that $
  (m-\nu_0) v_\mathscr{Y} (Y) \ge 0 , v_\mathscr{Y} (Y)\ge 0$, and
  therefore $\mathscr{Y}$ does not divide $\mathscr{N}_Y$. Now, let $Y$
  be any element of $K$ satisfying the equation 
  $$
  f_m(X) Y^m + \cdots + f_0 (X) = 0
  $$
  
  Then, the element $Z = f_m(X)Y$ satisfying the equation
  $$
  Z^m + g_{m-1} (X) Z^{m-1} + \cdots + g_o (X) = 0,
  $$
  where $g_k (X) = f_k (X) f^{m-k-1}_m (x)$, and hence $Z$ is an
  integral function of\pageoriginale $X$ 
\end{proof}

Suppose then that $Y_1 , \ldots Y_N$ is a basis of $K/k(X)$. By the
above remark, we any assume that the $Y_i$ are integral functions of
$X$. The elements $X^i Y_j (i=0, \ldots t; j=1 , \ldots N)$ are then
linearly independent over $k$, for any non-negative integer $t$. By
the above lemma, we can find integer $s$ such that $\mathscr{N}^s_X
(Y_j)$ are integral divisors. Hence, $\mathscr{N}_X^{s+t} (X^i) (Y_j)$
is an integral divisor for $(i=0, \ldots t, j=1, \ldots N)$, which
implies that $X^i Y_j$ are elements of
$L(\mathscr{N}_X^{-s-t})$. Since these are linearly independent and
$N(t+1)$ in number, we obtain 
$$
N(t+1) \le 1(\mathscr{N}_X^{-s-t}) \le l(\mathscr{N}) +d(\mathscr{N}) -
d(\mathscr{N}_X^{-s-t}) = 1+ (s+t) d (\mathscr{N}_x), 
$$
the latter inequality holding because $\mathscr{N}^{-s-t}_X$ divides
$\mathscr{N}$. 

Thus, $d(\mathscr{N}_X) \ge \dfrac{Nt + N -1}{s+t} \to N$ as $t \to
\infty$, which taken together with the opposite inequality we proved
earlier shows that $d(\mathscr{N}_X) = N$. 

It is clearly sufficient to show that $d(\mathscr{N}_X) = N$, since
the other follows on replacing $X$ by $\dfrac{1}{X}$ and observing
that $k(X) = k(1/X)$. 

\begin{corollary}\label{chap6:sec11:coro1}%corollary 1
  If $X \in K^*, d((X)) = 0$. This is clear when $X$ is a constant. If
  $X$ be a variable, $d((X)) = d(\mathfrak{z}_X) -d(\mathscr{N}_X) =
  N-N=0$. Hence we get the exact sequence 
  $$
  1 \to k^* \to K^* \to \vartheta \to \mathfrak{R}_0 \to 1 ,
  $$
  where $\vartheta_0$ is the group of divisor of degree zero and
  $\mathfrak{R}_0 = \dfrac{\vartheta_0}{\mathscr{Z}}$ is the group of
  divisor classes of degree zero. 
\end{corollary}

\begin{corollary}\label{chap6:sec11:coro2}%corollary'2
  Suppose $C \in \mathfrak{R}$ is a class of divisors. IF
  $\mathscr{U}, \delta$ are two divisors of this class, there exists
  an $X \in K^*$ such that $\mathscr{U} = (X) \delta$. Hence,\pageoriginale\
  $d(\mathscr{U}) = d((X)) + d(\delta) = d(\delta)$, and therefore we
  may define the degree $d(C)$ of the class $C$ to be the degree of
  any one of its divisors. 
\end{corollary}	

\begin{corollary}\label{chap6:sec11:coro3}%corollary 3
  If $X$ is any transcendental element, there exists an integer $Q$
  dependent only on $X$ such that for all integral $m$, we have 
  $$
  l(\mathscr{N}^{-m}_X) + d(\mathscr{N}^{-m}_X) \ge -Q .
  $$
\end{corollary}

\begin{proof}
  We saw in the course of the proof of the theorem that for $t \ge 0$,
  $$
  l(\mathscr{N}^{-s-t}_X) \ge N(t+1) = d(\mathscr{N}_X) (t+1) ,
  $$
  and writing $m = s + t$, we obtain for $m \ge s$,
  $$
  l(\mathscr{N}_X^{-m}) + d(\mathscr{N}^{-m}_X) \ge (1-s) d(\mathscr{N}_X) = -Q .
  $$
  
  For $m < s$, since $\mathscr{N}^{-s}_X$ divides $\mathscr{N}^{-m}_X$, we have 
  $$
  l(\mathscr{N}_X^{-m}) + d(\mathscr{N}^{-m}_X) \ge
  l(\mathscr{N}_X^{-s}) + d(\mathscr{N}_X^{-s}) \ge -Q . 
  $$
\end{proof}
