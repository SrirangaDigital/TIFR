\chapter{Lecture 12}\label{chap12}%ch 12

\setcounter{section}{24}
\section{The Functional Equation for the \texorpdfstring{$L$}{L}-functions}\label{chap12:sec25}%sec 25

Let\pageoriginale $\mathcal{X}$ be a character of finite order on the class group of
an algebraic field $K$ over a constant field $k$ with $q=p^f$
elements. We consider two cases.

\setcounter{Case}{0}
\begin{Case}% case 1
  $\mathcal{X}$ when restricted to the subgroup $\mathfrak{R}_o$ of
  $\mathfrak{R}$ of all classes of degree zero, is trivial; i.e.,
  $\mathcal{X} (C_o)=1$ for $C_o \in \mathscr{R}_o$.   
\end{Case}

Let $\rho$ be as before the smallest positive integer which is the
degree of a class, and let $C ^{(\rho)}$ be a class of degree
$\rho$. Then $\mathfrak{R}$ is clearly the direct product of
$\mathfrak{R}_o$ and the cyclic group generated by $C ^{(\rho)}$. Set
$\mathcal{X} (C^{(\rho)}) = e^{2 \pi i \xi}$. We then have 
\begin{align*}
   L (s, \mathcal{X}, K) & = \sum_\mathscr{U} \mathcal{X} (\mathscr{U})
  (N \mathscr{U})^{-s} = \sum_{ C_o \in \mathfrak{R}_o}
  \sum\limits_{\substack{n = - \infty\\ \mathscr{U} \in C^o
      C^{(\rho)^n}}}^\infty e^{2 \pi i \xi n} q^{-n \rho s}\\ 
  &=  \sum_\mathscr{U} (N \mathscr{U})^{- \left(s -\frac{2 \pi i \xi}{\rho
    \log q}\right)} = \zeta \left(s- \frac{2 \pi i \xi}{\rho \log q}\right). 
\end{align*}

Thus, the $L$-function reduces to a $\zeta$- function in this case. We
can therefore deduce a functional equation for $L (s, \mathcal{X}, K)$
from the functional equation for $\zeta (s,K)$. Define the character
$\bar{\mathcal{X}}$ conjugate to $\mathcal{X}$ by putting
$\bar{\mathcal{X}} (C) = \bar{\mathcal{X} (C)}$. Clearly,
$\bar{\mathcal{X}}$ is a character of finite order on the class group
and is trivial on $\mathfrak{R}_o$, also $\bar{\mathcal{X}} C^{(\rho)}
= e^{-2 \pi i \xi}$. Then, we obtain the following relation for $L (s,
\mathcal{X}, K)$ by substituting from the functional equation for the
$\zeta$ function. 
\begin{align*}
  q^{s(g-1)} L (s, \mathcal{X} , K) & = q^{s(g-1)} \zeta \left(s-\frac{2
    \pi i \xi}{\rho \log q}\right)\\ 
  & = q^{2 (q-1)} \frac{2 \pi i \xi}{\rho \log q} \zeta \left(1-s + \frac{2
    \pi i \xi}{\rho \log q}\right) q^{(1-s)(g-1)}\\ 
  & = \mathcal{X} (W)_q{^{(1-s) (g-1)}} L (1-s, \bar{\mathcal{X}}, K) 
\end{align*}
since\pageoriginale $\mathcal{X} (W) = (\mathcal{X} (C ^{(\rho)}))
\dfrac{2g-2}{\rho} = e \dfrac{2 \pi i \xi}{\rho} (2g-2)$ 

\begin{Case} % case 2
  Suppose now that $\mathcal{X}$ when restricted to $\mathfrak{R}_o$
  is not identically 1. Let $C^1_o$ be a fixed class with
  $\mathcal{X} (C^1_o) \neq 1$. Then, 
  $$
  \mathcal{X} (C^1_o) \sum_{C_o \in \mathfrak{R}_o} \mathcal{X}
  (C_o)  = \sum_{C_o \in \mathfrak{R}_o}  \mathcal{X} (C_o^1
  C_o) = \sum_{C_o \in \mathfrak{R}_o} \mathcal{X} (C_o), 
  $$
  and therefore $\sum\limits_{C_o \in \mathfrak{R}} \mathcal{X} (C_o) = $  
\end{Case}

Again, using the fact that $N (C)=0$ if $d (C) < 0$, we obtain 
\begin{align*}
  (q-1)L (s, \mathcal{X}, K) & = \sum_{C_o \in \mathfrak{R}_o}
  \mathcal{X} (C_o) \sum^\infty_{n=0} \mathcal{X}^n (C ^{(\rho)}) (q^{N
    (C_o C{^{(\rho)}})^n} -1) q^{-n \rho s}\\
  & =\sum_{C_o \in \mathfrak{R}_o} \mathcal{X} (C_o)
  \sum_{n=o}^{\frac{2g-2}{\rho}} \mathcal{X}^n (C ^{(\rho)}) (q^{N (C_o
  C^{(\rho)^n})}-1)\\ 
  & \quad + \sum_{C_o \in \mathfrak{R}_o} \mathcal{X} (C_o) \sum_{n >
    \frac{2g-2}{\rho}} \mathcal{X}^n (C^{(\rho)}) (q^{n \rho - g+1} -1)
  q^{-n \rho s} 
\end{align*}

The second sum vanishes, since $\sum\limits_{C_o \in \mathfrak{R}_o}
\mathcal{X} (C_o) = 0$. 

Thus, we obtain
$$
(q-1) L (s,\mathcal{X}, K) = \sum_{ o \le d (C) \le 2g-2} \mathcal{X}
(C) q^{N (C)} U^{d (C)} 
$$

The\pageoriginale coefficient of $U^{2g-2}$ is given by
\begin{multline*}
  \sum_{d (C) = 2g-2} \mathcal{X} (C) q^{N (C)} = \mathcal{X} (W)
  \sum_{C_o \in \mathfrak{R}_o} \mathcal{X} (C_o) q^{N (C_o
    W)}\\ 
  = \mathcal{X} (W) \left\{ \sum_{C_o \neq E} \mathcal{X} (C_o) q^{g-1} +
  q^g\right\}, 
\end{multline*}
$
\begin{aligned}
  \text{since}\quad  N (C_o W) & = d (C_o W) - g+1 = g-1~\text{ if}~ C_o \neq
  E ~\text{and}~ N (W) = g,\\ 
  & =\mathcal{X} (W) \left\{ \sum_{C_o \in \mathfrak{R}_o}
  \mathcal{X} (C_o) q^{g-1} + (q^g-q^{g-1})\right\}\\ 
  &= (q-1) \mathcal{X}
  (W) q^{g-1} \neq 0.
\end{aligned}
$
\medskip

Thus, $L (s, \mathcal{X},K)$ is a polynomial in $U =q^s$ of degree
$2g-2$ and leading coefficient $q^{g-1} \mathcal{X} (W)$ 

Again, by substituting from the Riemann-Roch theorem, we have
\begin{align*}
   (q-1) L(s, \mathcal{X},K) & = \sum_{o \le d (C) \le 2g-2}
  \mathcal{X} (C) q^{N(C)} U^{d (C)}\\ 
  & = \sum_{o \le d (C) \le 2g-2}
  \mathcal{X} (C) q^{d (C) - g + 1 + N (WC^{-1})} U^{d (C)}\\ 
  & = q^{g-1} U^{2g-2} \mathcal{X} (W) \sum_{ o \le d (C) \le 2g-2}
  \overline{\mathcal{X} (WC^{-1})} q^{N (WC-1)}\\ 
  & \hspace{5cm}\left(\frac{1}{qU}\right)^{d
    (WC^{-1})} 
\end{align*}

Writing $C^1$ for $WC^{-1}$ and noting that $C^1$ runs through exactly
the same range of summation as does $C$, we obtain 
$$
(q-1) L(s, \mathcal{X}, K) = (q-1) q^{g-1} U^{2g-2} \mathcal{X} (W) L
(1-s, \bar{\mathcal{X}}, K), 
$$
which is again the same functional equation that we got in Case
$1$. We have therefore proved the  
\begin{theorem*}
  For any character $\mathcal{X}$ of finite order,
  $$
  q^{s (g-1)} L (s,\mathcal{X}, K) = \mathcal{X} (W) q^{(1-s) (g-1)} L
  (1-s \bar{\mathcal{X}},K). 
  $$
\end{theorem*}
