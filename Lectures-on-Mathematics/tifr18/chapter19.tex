\chapter{Lecture 19}\label{chap19}

\setcounter{section}{32}
\section{Application of Galois Theory}\label{chap19:sec33}%Sec 33

We\pageoriginale shall now recall some basic facts of Galois theory which we shall
use in the sequel. 

Let $A$ be a field and $B$ a finite algebraic extension of $A. B_s$
shall denote the subfields of all separable elements over $A$ of
$B$. The \textit{ separable degree } $[B : A]_s$ is then define to be
the degree of the field $B_s$ over $A.B$ is a purely inseparable
extension over $B_s$ and its degree is called the \textit{degree of
  inseparability} of $B$ over $A$, and is denoted by $[B:A]_i$. We
obtained the relation 
$$
[B:A]=[B:B_s].[B_s:A]=[B:A]_s.[B:A]_i
$$

If $B/A$ is a normal extension, we define the \textit{Galois group}
$G(B/A)$ to be the group of all automorphisms of $B$ which
leave all the elements of $A$ fixed. The set $A_1$ of all elements
left fixed by every automorphism belonging to $G(B/A)$ is clearly a
field containing $A$. It is called the \textit{fixed field} of
$G(B/A)$. $A_1$ is a purely inseparable extension of $A$ and $B$ a
separable extension of $A_1$, and  we have 
$$
[B:A_1]=[B:B]_s, [A_1:A]=[B:A]_i.
$$

We shall now apply these facts to the theory of algebraic
functions. Let $L/l$ ba a extension of $K/k$. We define the
\textit{relative separable degree $d_{L/K}(\mathscr{K})_s$ and the
  relative inseparable}  
\textit{degree}\pageoriginale $d_{L /K}(\mathscr{K})_i$ of a prime divisor
$\mathscr{K}$ of $L$ lying over the prime divisor $\mathscr{Y}$ of $K$
to be respectively $[L _\mathscr{K} : K_\mathscr{Y}]_s$ and
$[L_\mathscr{K} : K_\mathscr{Y}]_i$. $\mathscr{K}$ is said to be
\textit{separable, inseparable or purely inseparable } accordingly as
$d_{L/ _K}(\mathscr{K})_i$ is equal to one, greater than one or equal
to $d_{L/_K}(\mathscr{K})$. 

Suppose now that $L^{1} /_{l'}  \supset {L /l} \supset K / _k$ is a
tower of algebraic function fields. Let $\mathscr{K}^1$ be a prime
divisor of $L^1$ lying over a prime divisor $\mathscr{K}$ of $L$,
which again lies  over a prime divisor $\mathscr{Y}$ of $K$. It is
easy to see that the following relation  between the ramification
indices holds 
$$
e_{L'/K} (\mathscr{K}^1) = e_{L'} (\mathscr{K}^1). e_{L /K}(\mathscr{K})
$$ 

If moreover we assume that [$L^1: K$]$ < \infty$, we have 
$$
d_{L^1 /_K}(\mathscr{K}^1) = d_{L^1 / _L}(\mathscr{K}^1) d_{L /_ K}(\mathscr{K}) ,
$$
and similar relations for the separable  and inseparable degrees.

Now, suppose $L/_l$ and $L^1/ _{l^1}$ are two extensions  of $K /_k$
and  $\sigma$ is an isomorphism of $L$ onto $L^1$ which maps $l$  on
$l^1$ and fixes every element of $K$. If  $\mathscr{K}$ is any  prime
divisor of $L$, we define the prime  divisor $\sigma \mathscr{K}$ in
$L^1$ by the equation 
$$
v_{\sigma \mathscr{K}} (Y) = v_{\mathscr{K}}(\sigma^{-1}Y) \text{ for } Y \in L^1.
$$

Clearly,  $\mathscr{K} \to \sigma \mathscr{K}$ is a one-one and onto
mapping of the set of prime divisors of $L$ onto the prime divisors of
$L^1$. It is also immediate that  the\pageoriginale isomorphism $\sigma$ maps the
valuation ring and the maximal ideal of $\mathscr{K}$ onto those of
$\sigma \mathscr{K}$. Hence we have an induced isomorphism
$\bar{\sigma}$ : $L_\mathscr{K} \to  L^1_{\sigma \mathscr{K}}$ of the
residue class fields.  If $\mathscr{K}$ lies over a prime divisor
$\mathscr{Y}$ of $K$, $\sigma \mathscr{K}$ also lies over
$\mathscr{Y}$ and $\bar{\sigma}$ fixes every element of
$K_\mathscr{Y}$. 

From  these facts, it follows that 
$$
e_{L^1/K} (\sigma \mathscr{K})= e_{L / _K}(\mathscr{K})
$$

and if $[L : K] <\infty$
$$
d_{L^1 / _ K} (\sigma \mathscr{K}) = d_{L / _K} (\mathscr{K}).
$$

We have the following theorem for finite  normal extensions.

\begin{theorem*}%Thm
  Let $L_l$ be a finite normal extension  of $K/k$, and
  $\mathscr{K}$ a prime  divisor of $L$ lying over a prime divisor
  $\mathscr{Y}$ of $K$. Then  every  prime divisor  of $L$ lying over
  $\mathscr{Y}$ is of the form $\sigma \mathscr{K}$, where $\sigma$ is
  an element of $G(L/K)$ 
\end{theorem*}

\begin{proof}
  We have already seen that $\sigma \mathscr{K}$ lies over
  $\mathscr{Y}$ for every\break $\sigma \in  G(L /K)$. To prove the converse
  statement, let $\mathscr{K} = \mathscr{K}_1,\ldots ,\mathscr{K}_h$ be all
  the prime divisors of $L$ lying over $\mathscr{Y}$. Find $a Y \in L$
  such that  
  \begin{align*}
    v_{\mathscr{K}} (Y) > 0 \\
    v_{\mathscr{K}_j}(Y) = 0 & \text{ for } j=2, \ldots h .
  \end{align*}
\end{proof}

Then ,
{\fontsize{10pt}{12pt}\selectfont
$$
v_\mathscr{K}(N_{L / _K}Y) = [L:K]_i \sum_{\sigma \in G(L/K)} v_K
(\sigma Y) = [L:K]_i \sum_{\sigma \in G (L
  /K)} v_{\sigma^{-1}\mathscr{K}} (Y) > 0 , 
$$}\relax
since\pageoriginale each  $\sigma^{-1} \mathscr{K}$ for $\sigma \in  G(L /K)$ is a
certain $\mathscr{K}_i$. Because $\mathscr{K}$ lies over
$\mathscr{Y}$, we deduce that  
$$
v_\mathscr{Y}(N_{L /_K}Y) > 0,
$$
and consequently for $j=2,\ldots , h$, we have 
$$
v_{\mathscr{K}_j}(N_{L/ _K} Y) = [L : K]_i \sum_{\sigma \in G(L/
  K)} v_{\sigma^{-1} \mathscr{K}_j} (Y) > 0. 
$$

Since at least one term of the sum on the right must be positive, and
since the only prime divisor lying over $\mathscr{Y}$ whose valuation
on $Y$ is positive is $\mathscr{K}$, there exists an automorphism
$\sigma_j$ such that $\mathscr{K}= \sigma_j^{-1} \mathscr{K}_j$,
$\mathscr{K}_j = \sigma_j \mathscr{K}$ for every $j$. Our theorem is
proved. 

For the rest of the lecture, we shall assume that $L /l$ is a finite
normal extension  of $K/k$ and $G(L /K)$ the Galois group. 

If $\mathscr{K}$ is a prime divisor of $L$, we define the
\textit{decomposition group} (Zerlegungs gruppe) $Z(\mathscr{K})$ of
$\mathscr{K}$ to be the subgroup of $G(L /K)$ of all elements $\sigma
\in G(L /K)$ such that $\sigma \mathscr{K} = \mathscr{K}$. It follows
that if $\sigma, \sigma^1 \in G(L /K)$, $\sigma \mathscr{K} =
\sigma^{1} \mathscr{K}$ if and only if $\sigma$ and $\sigma^1$  belong
to the same left coset of $G(L /K)$ modulo $Z (\mathscr{K})$. Because
of the above theorem, we are able to deduce that the number of prime
divisors of $L$ lying over a fixed prime divisor of $K$ is equal to
the index in $G(L /K)$ of the decomposition  group of any one of
them. 

It is also easy to obtain the relation  between the decomposition
groups of two  prime divisors $\mathscr{K}$ and $\sigma \mathscr{K}$
lying over the same prime divisor of $K$. In fact, we have  
$$
\tau  \in Z(\sigma \mathscr{K}) \Leftrightarrow \tau  \sigma
\mathscr{K} = \sigma \mathscr{K} \Leftrightarrow \sigma^\gamma
\tau  \sigma \mathscr{K} = \mathscr{K} \Leftrightarrow \sigma^{-1}
\tau  \sigma \in Z (\mathscr{K}) 
$$
and\pageoriginale therefore $Z(\sigma \mathscr{K}) = \sigma Z (\mathscr{K}) \sigma^{-1}$

\begin{theorem*}%Thm
  Let $\mathscr{K}$ be a prime divisor of $L$ lying over the prime
  divisor $\mathscr{Y}$ of  $K$. Then $L_{\mathscr{K}} /
  K_{\mathscr{Y}}$ is a normal extension. Every element $\sigma$ of
  $Z(\mathscr{K})$ induces an automorphism $\bar{\sigma}$ of
  $L_{\mathscr{K}}$ over $K_{\mathscr{Y}}$, and every  automorphism of
  $L_{\mathscr{K}}$ over $K_{\mathscr{Y}}$ is got in this way. 
\end{theorem*}

\begin{proof}
  Let $\mathscr{K} = \mathscr{K}_1, \mathscr{K}_2, \ldots ,\mathscr{K}_h$ be
  all the prime divisors of $L$ lying over $\mathscr{Y}$. If $\bar{y}
  \in  L_\mathscr{K}$, we can find a representative of $\bar{y}$ in
  the valuation ring $\mathscr{O}_\mathscr{K}$ of $\mathscr{K}$ such
  that  
  $$
  v_{\mathscr{K}_j} (y) > 0 \text{ for } j=2, \ldots , h.
  $$
\end{proof}

In  fact, if $y^1$ is any representative of $\bar{y}$, choose a $y \in
L$ such that  
\begin{align*}
  &v_\mathscr{K} (y-y^1) > 0 \\
  &v_{\mathscr{K} j}(y) > 0 \text{ for } j=2 ,\ldots,  h.
\end{align*}
$y$ satisfies the required condition.

The field polynomial of $y$ over $K$ is given by 
$$
f(X) = \left\{\prod_{\sigma \in G}  (X- \sigma y) \right\}^{[L : K]_i}
$$

Now, if $\sigma \notin Z(\mathscr{K}) , \sigma^{-1}\mathscr{K} \neq
\mathscr{K}$ and therefore $v_\mathscr{K}(\sigma y) = v_{\sigma^{-1}
  \mathscr{K}}\break (y) > 0$. Passing to the quotient modulo $\mathscr{K}$
in the above equation, we get  
$$
\overline{f(X)}=  \left \{  \prod_{\sigma \in  Z(\mathscr{K}) }(X-
\overline{\sigma y})\right \}^{[L:K]_i} X^M, 
$$
where\pageoriginale $M$ is a non-negative integer. But this is a polynomial over
$K_\mathscr{Y}$ which is satisfied  by  $\bar{y}$, and which has all
its roots lying in $L_\mathscr{K}$. $L_\mathscr{K}$ is thus a normal
extension of $K_\mathscr{Y}$. 

Now if  $\sigma \in Z(\mathscr{K}), \sigma \mathscr{K}=\mathscr{K}$and
by what we have already seen, $\sigma$ induces a $K_\mathscr{Y}$
isomorphism $\bar{\sigma}$ of $L_\mathscr{K}$ onto  $L_\mathscr{K}$,
i.e. an automorphism of $L_\mathscr{K}$. To prove the final part of
the theorem, notice that $L_\mathscr{K}$ is a separable extension of
the fixed field $(K_\mathscr{Y})$ of the Galois group
$G(L_\mathscr{K} / K_\mathscr{Y})$, and is therefore simple. Hence any
automorphism of $L_\mathscr{K} /  K_\mathscr{Y}$ is uniquely
determined  by its effect on a single element $\bar{y}$. But the
working  above proves that every conjugate of $\bar{y}$ io of the form
$\overline{\sigma y}= \bar{\sigma} (\bar{y})$ for some $\sigma \in
G(L/K)$. Hence every automorphism of $L_{\mathscr{K}}$ over
$K_{\mathscr{Y}}$ is of the form  $\bar{\sigma}$ with $\sigma \in
G(L/K)$. Our theorem is proved. 

We define the \textit{inertia group} $T(\mathscr{K})$ of a prime
divisor $\mathscr{K}$ of $L$ to be the subgroup of all elements
$\sigma$ of $Z(\mathscr{K})$ for which  $\bar{\sigma}$ is the identity
automorphism of $L_\mathscr{K}$. It is clearly a normal subgroup of
$Z(\mathscr{K})$. The theorem proved above then establishes an
isomorphism $G(L_\mathscr{K} / K_\mathscr{Y}) \simeq
\dfrac{Z(\mathscr{K})}{T(\mathscr{K})}$. 

We now given some consequences of the theorems of this lecture.
\begin{enumerate}
\item $[G: (e)]= [L:K]_s = h[Z(\mathscr{K}):(e)]$. 
\item $[Z(\mathscr{K}): T(\mathscr{K})]= [ L_\mathscr{K} :
  K_\mathscr{Y}]_s =d_{L /K}{(\mathscr{K})}_s$ 
\item $[L;K] = \sum\limits_{\nu = 1}^{h} e_{L /_K}(\mathscr{K}_\nu)
  d_{L /_K}(\mathscr{K}_\nu) =h e_{L / _K}(\mathscr{K}) d_{L /_K}
  (\mathscr{K})$ 

  \begin{multline*}
    \therefore ~e_{L / _K}(\mathscr{K}) d_{L /K}(\mathscr{K}) =
    \dfrac{[L:K]}{[L:K]_s} [Z(\mathscr{K}):(e)]\\
    = {[L:K]_i [Z(\mathscr{K}): (e)]} 
  \end{multline*} 

 Hence,\pageoriginale 
\item $[Z(\mathscr{K}) : (e) ] = \dfrac{e_{L/K}{(\mathscr{K})}
  d_{L/K} {(\mathscr{K})}}{[L:K]_i}$ 
\item $[T(\mathscr{K}) :(e) ] = \dfrac{{e_{L /K} {(\mathscr{K})}
    d_{L/K} {(\mathscr{K})}}}{[L : K]_i d_{L/K}{(\mathscr{K})}_s} =e_{L/
  K} (\mathscr{K}) \frac{d_{L/K}(\mathscr{K})_i}{[L : K]_i}$ 
\end{enumerate}

It follows from $(5)$ that if $L$ is separable over $K$,
$T(\mathscr{K}) \neq 1$ if and only if at least one of 
$e_{L/K}(\mathscr{K})$ or $d_{L/K}(\mathscr{K})_i$ is greater than one. 
