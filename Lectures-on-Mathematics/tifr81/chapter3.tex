\chapter[Semilinear Elliptic Equations I]{Semilinear Elliptic\hfill\break Equations I}\label{chap3}

We\pageoriginale consider the Dirichlet problem
\begin{equation*}
-\Delta u=f(x,u)\q\text{in}\q \Omega,\q u=0\q\text{on}\q \partial
\Omega,\tag{3.1}\label{chap3-eq3.1} 
\end{equation*}
where $\Omega$ is a bounded smooth domain in $\mathbb{R}^{N}$, $N\geq
2$ and $\p \Omega$ denotes its boundary. We assume all along that
$f:\Omega\times \mathbb{R}\to \mathbb{R}$ is a Carath\'eodory
function. By a {\em classical solution} of \eqref{chap3-eq3.1} we mean
a function $u\in C^{2}(\Omega)\cap C^{0}(\overline{\Omega})$ which
  satisfies the equation at every point $x\in \Omega$ and which
  vanishes on $\p \Omega$. By a {\em generalized solution} of
  \eqref{chap3-eq3.1} we mean a function $u\in H^{1}_{0}(\Omega)$
  which satisfies \eqref{chap3-eq3.1} in the weak sense, i.e.
\begin{equation*}
\int_{\Omega}\nabla u\cdot \nabla v=\int_{\Omega}f(x,u)v,\q \forall
v\in C^{\infty}_{c}(\Omega).\tag{3.2}\label{chap3-eq3.2}
\end{equation*}

We see that in order to have things well defined in
\eqref{chap3-eq3.2}, the function $f(x,s)$ has to obey some growth
conditions on the real variable $s$. We will not say which they are,
since a stronger assumption will be assumed shortly, when we look for
generalized solution as critical points of a functional. Namely let us
consider 
\begin{equation*}
\Phi(u)=\frac{1}{2}\int_{\Omega}|\nabla
u|^{2}-\int_{\Omega}F(x,u)\tag{3.3}\label{chap3-eq3.3} 
\end{equation*}
where $F(x,s)=\int^{s}_{0}f(x,\tau)d\tau$. In order to have
$\Phi:H'_{0}(\Omega)\to \mathbb{R}$ well defined we should require
that $F(x,u)\in L^{1}(\Omega)$ for $u\in H^{1}_{0}(\Omega)$. In view
of the Sobolev imbedding theorem $H^{1}_{0}\hookrightarrow L^{p}$
(continuous imbedding) if $1\leq p\leq 2N/(N-2)$\pageoriginale if
$N\geq 3$ and $1\leq p<\infty$ if $N=2$. So using Theorem
\ref{chap2-thm2.8} we should require that $f$ satisfies the following
condition 
\begin{equation*}
|f(x,s)|\leq c|s|^{p-1}+b(x)\tag{3.4}\label{chap3-eq3.4}
\end{equation*}
where $p$ satisfies the conditions of the Sobolev imbedding and
$$
b(x)\in L^{p'},\q \frac{1}{p}+\frac{1}{p'}=1.
$$

Using Theorem \ref{chap2-thm2.8} we conclude that:

{\em if $f$ satisfies \eqref{chap3-eq3.4} the functional $\Phi$
  defined in \eqref{chap3-eq3.3} is continuous Fr\'echet
  differentiable}, i.e., $C^{1}$, {\em and}
\begin{equation*}
\langle \Phi'(u),v\rangle =\int_{\Omega}\nabla u\cdot \nabla
v-\int_{\Omega}f(x,u)v,\q \forall v\in
H^{1}_{0}\tag{3.5}\label{chap3-eq3.5} 
\end{equation*}
{\em where $\langle\,, \rangle$ denotes the inner product in
  $H^{1}_{0}(\Omega)$.}

It follows readily that the critical points of $\Phi$ are precisely
the generalized solutions of \eqref{chap3-eq3.1}. So the search for
solutions of \eqref{chap3-eq3.1} is transformed in the investigation
of critical points of $\Phi$. In this chapter we study conditions
under which $\Phi$ has a minimum.

$\Phi$ is {\em bounded below} if the following condition is satisfied:
\begin{equation*}
F(x,s)\leq \frac{1}{2}\mu s^{2}+a(x)\tag{3.6}\label{chap3-eq3.6}
\end{equation*}
where $a(x)\in L^{1}(\Omega)$ and $\mu$ is a constant $0<\mu\leq
\lambda_{1}$. [Here $\lambda_{1}$ denotes the first eigenvalue of the
  Laplacian subject to Dirichlet boundary conditions]. Indeed we can
estimate
$$
\Phi(u)\geq \frac{1}{2}\int_{\Omega}|\nabla u|^{2}-\frac{1}{2}\mu
\int_{\Omega}u^{2}-\int_{\Omega}a(x)\geq -\int_{\Omega}a(x)
$$
where we have used the variational characterization of the first
eigenvalue. 

$\Phi$ is {\em weakly lower semicontinuous in} $H^{1}_{0}$ if
condition \eqref{chap3-eq3.4} is satisfied with $1\leq p<2N/(N-2)$ if
$N\geq 3$ and $1\leq p<\infty$ if $N=2$. Indeed
$$
\Phi(u)=\frac{1}{2}||u||^{2}_{H^{1}}-\Psi(u)
$$
where\pageoriginale $\Psi(u)=\int_{\Omega}F(x,u)$ has been studied in
Section 1.2, and the claim follows using the fact that
the norm is weakly lower semicontinuous and under the hypothesis
$\Psi$ is weakly continuous from $H^{1}_{0}$ into $\mathbb{R}$. Let us
prove this last statement. Let $u_{n}\rightharpoonup u$ in
$H^{1}_{0}$. Going to a subsequence if necessary we have $u_{n}\to u$
in $L^{p}$ with $p$ restricted as above to insure the compact
imbedding $H^{1}_{0}\hookrightarrow L^{p}$. Now use the continuity of
the functional $\Psi$ to conclude.

Now we can state the following result

\begin{theorem}\label{chap3-thm3.1}
Assume \eqref{chap3-eq3.6} and \eqref{chap3-eq3.4} with $1\leq
p<2N/(N-2)$ if $N\geq 3$ and $1\leq p<\infty$ if $N=2$. Then for each
$r>0$ there exist $\lambda_{r}\leq 0$ and $u_{r}\in H^{1}_{0}$ with
$||u_{r}||_{H^{1}}\leq r$ such that $\Phi'(u_{r})=\lambda_{r}u_{r}$,
and $\Phi$ restricted to the ball of radius $r$ around $\mathcal{O}$
assumes its infimum at $u_{r}$. 
\end{theorem}

\begin{proof}
The ball $\overline{B}_{r}(0)=\{u\in H^{1}_{0}:||u||_{H^{1}}\leq r\}$
is weakly compact. So applying Theorem \ref{chap1-thm1.1} to the
functional $\Phi$ restricted to $\overline{B}_{r}(0)$ we obtain a
point $u_{r}\in \overline{B}_{r}(0)$ such that
$$
\Phi(u_{r})=\Inf\{\Phi(u):u\in \overline{B}_{r}(0)\}.
$$
Now let $v\in \overline{B}_{r}(0)$ be arbitrary then
$$
\Phi(u_{r})\leq \Phi(tv+(1-t)u_{r})=\Phi(u_{r})+t\langle
\Phi'(u_{r}),v-u_{r}\rangle +o(t)
$$
which implies
\begin{equation*}
\langle \Phi'(u_{r}),v-u_{r}\rangle \geq 0.\tag{3.7}\label{chap3-eq3.7}
\end{equation*}

If $u_{r}$ is an interior point of $\overline{B}_{r}(0)$ then
$v-u_{r}$ covers a ball about the origin. Consequently
$\Phi'(u_{r})=0$. If $u_{r}\in \p \overline{B}_{r}(0)$ we proceed as
follows. In the case when $\Phi'(u_{r})=0$ we have the thesis with
$\lambda_{r}=0$. Otherwise when $\Phi'(u_{r})\neq 0$ we assume by
contradiction that $\Phi'(u_{r})/||\Phi'(u_{r})||\neq
-u_{r}/||u_{r}||$. Then $v=-r\Phi'(u_{r})/||\Phi'(u_{r})||$ is in $\p
\overline{B}_{r}(0)$ and $v\neq u_{r}$. So $\langle
v,u_{r}\rangle<r^{2}$. On the other hand with such a $v$ in
\eqref{chap3-eq3.7} we obtain $0\leq \langle
-v,v-u_{r}\rangle\Rightarrow r^{2}\leq \langle v,u_{r}\rangle$,
contradiction. 
\end{proof}

\begin{corollary}\label{chap3-coro3.2}
In addition to the hypothesis of Theorem \ref{chap3-thm3.1} assume
that there exists $r>0$ such that
\begin{equation*}
\Phi(u)\geq a>0\q\text{for}\q u\in \p
\overline{B}_{r}(0)\tag{3.8}\label{chap3-eq3.8} 
\end{equation*}
where $a$ is some given constant. Then $\Phi$ has a critical point.
\end{corollary}

\begin{proof}
Since $\Phi(0)=0$, we conclude from \eqref{chap3-eq3.8} that the
infimum of $\Phi$ in $\overline{B}_{r}(0)$\pageoriginale is achieved
at an interior point of that ball.
\end{proof}

\begin{remarks*}[(Sufficient conditions that insure \eqref{chap3-eq3.8})]
~
\begin{enumerate}
\renewcommand{\labelenumi}{(\theenumi)}
\item Assume $\mu<\lambda_{1}$ in condition \eqref{chap3-eq3.6}. Then 
\begin{equation*}
\Phi(u)\geq \frac{1}{2}\int |\nabla u|^{2}-\frac{\mu}{2}\int
u^{2}-C|\Omega|\geq
\frac{1}{2}\left(1-\frac{\mu}{\lambda_{1}}\right)\int |\nabla
u|^{2}-C|\Omega|\tag{3.9}\label{chap3-eq3.9} 
\end{equation*}
where we have used the variational characterization of the first
eigenvalue. It follows from \eqref{chap3-eq3.9} that $\Phi(u)\to
+\infty$ as $||u||\to \infty$, that is, $\Phi$ is {\em coercive}. So
\eqref{chap3-eq3.8} is satisfied.

\item In particular, if there exists $\overline{\mu}<\lambda_{1}$ such
  that
$$
\mathop{\lim\sup}\limits_{|s|\to\infty}\frac{f(x,s)}{s}\leq \overline{\mu}
$$
then one has \eqref{chap3-eq3.6} with a $\mu<\lambda_{1}$, and $\Phi$
is coercive as proved above.

\item (A result of Mawhin-Ward-Willem \cite{key60}). {\em Assume that}
\begin{equation*}
\mathop{\lim\sup}\limits_{|s|\to\infty}\frac{2F(x,s)}{s^{2}}\leq
\alpha (x)\leq \lambda_{1}\tag{3.10}\label{chap3-eq3.10}
\end{equation*}
{\em where $\alpha(x)\in L^{\infty}(\Omega)$ and $\alpha(x)<\lambda_{1}$ on
a set of positive measure. Then under hypotheses} \eqref{chap3-eq3.4}
{\em and} \eqref{chap3-eq3.10}, {\em the Dirichlet problem has a
  generalized solution $u\in H^{1}_{0}(\Omega)$.} To prove this
statement all it remains to do is to prove that condition
\eqref{chap3-eq3.8} is satisfied. First we claim that there exists
$\epsilon_{0}>0$ such that
\begin{equation*}
\Theta(u)\equiv \int_{\Omega}|\nabla
u|^{2}-\int_{\Omega}\alpha(x)u^{2}\geq \epsilon_{0},\q \forall
||u||_{H^{1}_{0}}=1.\tag{3.11}\label{chap3-eq3.11} 
\end{equation*}
Assume by contradiction that there exists a sequence $(u_{n})$
in\break 
$H^{1}_{0}(\Omega)$ with $||u_{n}||_{H^{1}}=1$ and $\Theta(u_{n})\to
0$. We may assume without loss of generality that
$u_{n}\rightharpoonup u_{0}$ (weakly) in $H^{1}_{0}$ and $u_{n}\to u$
in $L^{2}$. As a consequence of the fact that $\alpha(x)\leq
\lambda_{1}$ in $\Omega$, we have $\Theta(u_{n})\geq 0$ and then 
\begin{equation*}
0\leq \int |\nabla u_{0}|^{2}-\int\alpha(x)u^{2}_{0}\leq
0.\tag{3.12}\label{chap3-eq3.12} 
\end{equation*}
On the other hand, $\Theta(u_{n})=1-\int\alpha(x)u^{2}_{n}$ gives
$\int \alpha (x)u^{2}_{0}=1$. From \eqref{chap3-eq3.12} we get
$||u_{0}||_{H^{1}}=1$, which implies that $u_{n}\to u_{0}$ (strongly)
in $H^{1}_{0}$. This implies that\pageoriginale $u_{0}\nequiv 0$. Now
observe that $\Theta:H^{1}_{0}\to R$ is weakly lower semicontinuous,
that $\Theta(u)\geq 0$ for all $u\in H^{1}_{0}$ and
$\Theta(u_{0})=0$. So $u_{0}$ is a critical point of $\Theta$, which
implies that $u_{0}\in H^{1}_{0}(\Omega)$ is a generalized solution of
$-\Delta u_{0}=\alpha(x)u_{0}$. Thus $u_{0}\in W^{2,2}(\Omega)$ and it
is a strong solution of an elliptic equation. By the Aleksandrov
maximum principle (see for instance, Gilbarg-Trudinger
\cite[p. 246]{key46}) we see that $u_{0}\neq 0$ a.e. in
$\Omega$. Using \eqref{chap3-eq3.12} again we have
$$
\lambda_{1}\int_{\Omega}u^{2}_{0}\leq \int_{\Omega}|\nabla
u_{0}|^{2}\leq \int \alpha (x)u^{2}_{0}<\lambda_{1}\int u^{2}_{0},
$$
which is impossible. So \eqref{chap3-eq3.11} is proved.

Next it follows from \eqref{chap3-eq3.10} that given
$\epsilon<\lambda_{1}\epsilon_{0}$ (the $\epsilon_{0}$ of
\eqref{chap3-eq3.11}) there exists a constant $c_{\epsilon}>0$ such
that
$$
F(x,s)\leq \frac{\alpha(x)+\epsilon}{2}s^{2}+c_{\epsilon},\q \forall
x\in \Omega,\q \forall s\in \mathbb{R}.
$$
Then we estimate $\Phi$ as follows
$$
\Phi(u)\geq \frac{1}{2}\int |\nabla u|^{2}-\frac{1}{2}\int
\alpha(x)u^{2}-\frac{\epsilon}{2}\int u^{2}-c_{\epsilon}|\Omega|.
$$
Using \eqref{chap3-eq3.11} we get
$$
\Phi(u)\geq \frac{1}{2}\epsilon_{0}\int |\nabla
u|^{2}-\frac{1}{2}\frac{\epsilon}{\lambda_{1}}\int |\nabla
u|^{2}-c_{\epsilon}|\Omega| 
$$
which implies that $\Phi$ is coercive, and in particular
\eqref{chap3-eq3.8} is satisfied.
\end{enumerate}
\end{remarks*}

\begin{remark*}
We observe that in all cases considered above we in fact pro\-ved that
$\Phi$ were coercive. We remark that condition \eqref{chap3-eq3.8}
could be attained without coerciveness. It would be interesting to
find some other reasonable condition on $F$ to insure
\eqref{chap3-eq3.8}. On this line, see the work of de
Figueiredo-Gossez \cite{key42}.
\end{remark*}

\noindent
{\bf Final Remark.}~ ({\em Existence of a minimum without the growth
  condition} \eqref{chap3-eq3.4}). Let us look at the functional
$\Phi$ assuming the following condition: for some constant $b>0$ and
$a(x)\in L^{1}(\Omega)$ one has
\begin{equation*}
F(x,s)\leq b|s|^{p}+a(x)\tag{3.13}\label{chap3-eq3.13}
\end{equation*}
where $1\leq p<2N/(N-2)$ if $N\geq 3$ and $1\leq p<\infty$ if
$N=2$. For $u\in H^{1}_{0}$ we have
$$
\int F(x,u(x))dx\leq b\int_{\Omega}|u(x)|^{p}dx+\int a(x)dx
$$
where\pageoriginale the integral on the left side could be
$-\infty$. In view of the Sobolev imbedding it is $<+\infty$. So the
functional $\Phi$ could assume the value $+\infty$. Let us now check
its weakly lower semicontinuity at a point $u_{0}\in
H^{1}_{0}(\Omega)$ where $\Phi(u_{0})<+\infty$. So $F(x,u_{0}(x))\in
L^{1}$. Now take a sequence $u_{n}\rightharpoonup u_{0}$ in
$H^{1}_{0}$. Passing to subsequence if necessary we may suppose that
$u_{n}\to u_{0}$ in $L^{p}$, $u_{n}(x)\to u_{0}(x)$ a.e.\@ in $\Omega$
and $|u_{n}(x)|\leq h(x)$ for some $h\in L^{p}$.

It follows then from \eqref{chap3-eq3.13} that
$$
F(x,u_{n}(x))\leq bh(x)^{p}+a(x).
$$

Since the right side of the above inequality is in $L^{1}$ we can
apply Fatou's lemma and conclude that
$$
\lim\sup \int_{\Omega}F(x,u_{n}(x))dx\leq \int F(x,u_{0}(x))dx
$$

Consequently we have
\begin{align*}
\lim\inf \Phi(u_{n}) &\geq \lim\inf \frac{1}{2}\int |\nabla
u_{n}|^{2}-\lim\sup \int F(x,u_{n})\\
&\geq \frac{1}{2}\int|\nabla u_{0}|^{2}-\int F(x,u_{0}).
\end{align*}

So $\Phi:H^{1}_{0}(\Omega)\to \mathbb{R}\cup \{+\infty\}$ is defined
and weakly lower semicontinuous. By Theorem \ref{chap1-thm1.1} $\Phi$
has a minimum in any ball $B_{r}(0)$ contained in $H^{1}_{0}$. If $F$
satisfies condition \eqref{chap3-eq3.10} (which by the way implies
\eqref{chap3-eq3.13}) we see by Remark 3 above that $\Phi$ is
coercive. Thus $\Phi$ has a global minimum in $H^{1}_{0}$. Without
further conditions (namely \eqref{chap3-eq3.4}) one cannot prove that
such a minimum is a critical point of $\Phi$. 


