\chapter{Minimization of Lower Semicontinuous
  Functionals}\label{chap1}

Let\pageoriginale $X$ be a Hausdorff topological space. A functional
$\Phi:X\to \mathbb{R}\cup\{+\infty\}$ is said to be {\em lower
  semicontinuous} if for every $a\in \mathbb{R}$ the set $\{x\in
X:\Phi(x)>a\}$ is open. We use the terminology functional to designate
a real valued function. A Hausdorff topological space $X$ is {\em
  compact} if every covering of $X$ by open sets contains a finite
subcovering. The following basic theorem implies most of the results
used in the minimization of functionals.

\begin{theorem}\label{chap1-thm1.1}
Let $X$ be a compact topological space and $\Phi:X\to \mathbb{R}\cup
\{+\infty\}$ a lower semicontinuous functional. Then {\rm(a)} $\Phi$
is bounded below, and {\rm(b)} the infimum of $\Phi$ is achieved at a
point $x_{0}\in X$.
\end{theorem}

\begin{proof}
The open sets $A_{n}=\{x\in X:\Phi(x)>-n\}$, for $n\in \mathbb{N}$,
constitute an open covering of $X$. By compactness there exists a
$n_{0}\in \mathbb{N}$ such that
$$
\bigcup\limits^{n_{0}}_{j=1}A_{j}=X.
$$
So $\Phi(x)>n_{0}$ for all $x\in X$.

(b) Now let $\ell=\Inf \Phi$, $\ell>-\infty$. Assume by contradiction
that $\ell$ is not achieved. This means that
$$
\bigcup\limits^{\infty}_{n=1}\left\{x\in
X:\Phi(x)>\ell+\frac{1}{n}\right\}=X. 
$$

By\pageoriginale compactness again it follows that there exist a
$n_{1}\in \mathbb{N}$ such that
$$
\bigcup\limits^{n_{1}}_{n=1}\left\{x\in
X:\Phi(x)>\ell+\frac{1}{n}\right\}=X. 
$$

But this implies that $\Phi(x)>\ell +\frac{1}{n_{1}}$ for all $x\in
X$, which contradicts the fact that $\ell$ is the infimum of $\Phi$.
\end{proof}

In many cases it is simpler to work with a notion of lower
semicontinuity given in terms of sequences. A function $\Phi:X\to
\mathbb{R}\cup\{+\infty\}$ is said to be {\em sequentially lower
  semicontinuous} if for every sequence $(x_{n})$ with $\lim
x_{n}=x_{0}$, it follows that $\Phi(x_{0})\leq \lim \inf
\Phi(x_{n})$. The relationship between the two notions of lower
semicontinuity is expounded in the following proposition. 

\begin{proposition}\label{chap1-prop1.2}
{\rm(a)}~ Every lower semicontinuous function $\Phi:X\to
\mathbb{R}\cup \{+\infty\}$ is sequentially lower
semicontinuous. {\rm(b)}~ If $X$ satisfies the First Axiom of
Countability, then every sequentially lower semicontinuous function is
lower semicontinuous.
\end{proposition}

\begin{proof}
\begin{itemize}
\item[(a)] Let $x_{n}\to x_{0}$ in $X$. Suppose first that
  $\Phi(x_{0})<\infty$. For each $\epsilon>0$ consider the open set
  $A=\{x\in X:\Phi(x)>\Phi(x_{0})-\epsilon\}$. Since $x_{0}\in A$, it
  follows that there exists $n_{0}=n_{0}(\epsilon)$ such that
  $x_{n}\in A$ for all $n\geq n_{0}$. For such $n$'s,
  $\Phi(x_{n})>\Phi(x_{0})-\epsilon$, which implies that
  $\lim\inf\Phi(x_{n})\geq \Phi(x_{0})-\epsilon$. Since $\epsilon>0$
  is arbitrary it follows that $\lim\inf\Phi(x_{n})\geq
  \Phi(x_{0})$. If $\Phi(x_{0})=+\infty$ take $A=\{x\in X:\Phi(x)>M\}$
  for arbitrary $M>0$ and proceed in similar way.

\item[(b)] Conversely we claim that for each real number $a$ the set
  $F=\{x\in \Omega:\Phi(x)\leq a\}$ is closed. Suppose by
  contradiction that this is not the case, that is, there exists
  $x_{0}\in \overline{F}\backslash F$, and so $\Phi(x_{0})>a$. On the
  other hand, let $\mathcal{O}_{n}$ be a countable basis of open
  neighborhoods of $x_{0}$. For each $n\in \mathbb{N}$ there exists
  $x_{n}\in F\cap \mathcal{O}_{n}$. Thus $x_{n}\to x_{0}$. Using the
  fact that $\Phi$ is sequentially lower semicontinuous and
  $\Phi(x_{n})\leq a$ we obtain that $\Phi(x_{0})\leq a$, which is
  impossible. 
\end{itemize}
\end{proof}

\begin{corollary}\label{chap1-coro1.3}
If $X$ is a metric space, then the notions of lower semicontinuity and
sequentially lower semicontinuity coincide.
\end{corollary}

\noindent
{\bf Semicontinuity at a Point.}~The notion of lower semicontinuity
can be localized as follows. Let $\Phi:X\to \mathbb{R}\cup\{+\infty\}$
be a functional and $x_{0}\in X$. We say that $\Phi$ is {\em lower
  semicontinuous} at $x_{0}$ if for all $a<\Phi(x_{0})$ there exists
an\pageoriginale open neighborhood $V$ of $x_{0}$ such that
$a<\Phi(x)$ for all $x\in V$. It is easy to see that a lower
semicontinuous functional is lower semicontinuous at all points $x\in
X$. And conversely a functional which is lower semicontinuous at all
points is lower semicontinuous. The reader can provide similar
definitions and statements for sequential lower semicontinuity.

\medskip
\noindent
{\bf Some Examples When {\boldmath$X=\mathbb{R}$}.}~Let
$\Phi:\mathbb{R}\to \mathbb{R}\cup \{+\infty\}$. It is clear that
$\Phi$ is lower semicontinuous at all points of continuity. If $x_{0}$
is a point where there is a jump discontinuity and $\Phi$ is lower
semicontinuous there, then $\Phi(x_{0})=\min
\{\Phi(x_{0}-0),\Phi(x_{0}+0)\}$. If $\lim \Phi(x)=+\infty$ as $x\to
x_{0}$ then $\Phi(x_{0})=+\infty$ if $\Phi$ is to be lower
semicontinuous there. If $\Phi$ is lower semicontinuous the set
$\{x\in \mathbb{R}:\Phi(x)=+\infty\}$ is not necessarity
closed. Example:~$\Phi(x)=0$ if $0\leq x\leq 1$ and $\Phi(x)=+\infty$
elsewhere. 

\medskip
\noindent
{\bf Functionals Defined in Banach Spaces.}~ In the case when $X$ is a
Banach space there are two topologies which are very useful. Namely
the norm topology $\tau$ (also called the strong topology) which is a
metric topology and the weak topology $\tau^{\omega}$ which is not
metric in general. We recall that the weak topology is defined by
giving a basis of open sets as follows. For each $\epsilon>0$ and each
finite set of bounded linear functionals $\ell_{1},\ldots,\ell_{n}\in
X^{*}$, $X^{*}$ is the dual space of $X$, we define the (weak) open
set $\{x\in
X:|\ell_{1}(x)|<\epsilon,\ldots,|\ell_{n}(x)|<\epsilon\}$. It follows
easily that $\tau$ is a finer topology than $\tau^{\omega}$, i.e.\@
given a weak open set there exists a strong open set contained in
it. The converse is not true in general. [We remark that finite
  dimensionality of $X$ implies that these two topologies are the
  same]. It follows then that a weakly lower semicontinuous functional
$\Phi:X\to \mathbb{R}\cup \{+\infty\}$, $X$ a Banach space, is
(strongly) lower semicontinuous. A similar statement holds for the
sequential lower semicontinuity, since every strongly convergent
sequence is weakly convergent. In general, a (strongly) lower
semicontinuous functional is not weakly lower semicontinuous. However
the following result holds.

\begin{theorem}\label{chap1-thm1.4}
Let $X$ be a Banach space, and $\Phi:X\to \mathbb{R}\cup \{+\infty\}$
a convex function. Then the notions of (strong) lower semicontinuity
and weak lower semicontinuity coincide.
\end{theorem}

\begin{proof}
\begin{enumerate}
\renewcommand{\theenumi}{\roman{enumi}}
\renewcommand{\labelenumi}{(\theenumi)}
\item Case of sequential lower semicontinuity. Suppose
  $x_{n}\rightharpoonup x_{0}$ (the half arrow $\rightharpoonup$
  denotes weak convergence). We claim that the hypothesis of
  $\Phi$\pageoriginale being (strong) lower semicontinuous implies
  that
$$
\Phi(x_{0})\leq \lim\inf\Phi(x_{n}).
$$

Let $\ell=\lim\inf\Phi(x_{n})$, and passing to a subsequence (call it
$x_{n}$ again) we may assume that $\ell=\lim\Phi(x_{n})$. If
$\ell=+\infty$ there is nothing to prove. If $-\infty<\ell<\infty$, we
proceed as folows. Given $\epsilon>0$ there is $n_{0}=n_{0}(\epsilon)$
such that $\Phi(x_{n})\leq \ell+\epsilon$ for all $n\geq
n_{0}(\epsilon)$. Renaming the sequence we may assume that
$\Phi(x_{n})\leq \ell+\epsilon$ for all $n$. Since $x$ is the weak
limit of $(x_{n})$ it follows from Mazur's theorem [which is
  essentially the fact that the convex hull $co(x_{n})$ of the
  sequence $(x_{n})$ has weak closure coinciding with its strong
  closure] that there exists a sequence
$$
yN=\sum\limits^{k_{N}}_{j=1}\alpha^{N}_{j}x_{j},\quad
\sum\limits^{k_{N}}_{j=1}\alpha^{N}_{j}=1,\quad \alpha^{N}_{j}\geq 0,
$$
such that $yN\to x_{0}$ as $N\to \infty$. By convexity
$$
\Phi(yN)\leq \sum\limits^{k_{N}}_{j=1}\alpha^{N}_{j}\Phi(x_{j})\leq
\ell +\epsilon
$$
and by the (strong) lower semicontinuity $\Phi(x_{0})\leq
\ell+\epsilon$. Since $\epsilon>0$ is arbitrary we get
$\Phi(x_{0})\leq \ell$. If $\ell=-\infty$, we proceed in a similar
way, just replacing the statement $\Phi(x_{n})\leq \ell+\epsilon$ by
$\Phi(x_{n})\leq -M$ for all $n\geq n(M)$, where $M>0$ is arbitrary.

\item Case of lower semicontinuity (nonsequential). Given $a\in
  \mathbb{R}$ we claim that the set $\{x\in X:\Phi(x)\leq a\}$ is
  weakly closed. Since such a set is convex, the result follows from
  the fact that for a convex set being weakly closed is the same as
  strongly closed.
\end{enumerate}
\end{proof}

Now we discuss the relationship between sequential weak lower
semicontinuity and weak lower semicontinuity, in the case of
functionals $\Phi:A\to \mathbb{R}\cup \{+\infty\}$ defined in a subset
$A$ of a Banach space $X$. As in the case of a general topological
space, every weak lower semicontinuous functional is also sequentially
weak lower semicontinuous. The converse has to do with the fact that
the topology in $A$ ought to satisfy the First Axiom of
Countability. For that matter one restricts to the case when $A$ is
bounded. The reason is: infinite dimensional Banach spaces $X$ (even
separable Hilbert spaces) do not satisfy the First Axiom of
Countability under the weak topology. The same statement is true for
the weak topology induced in unbounded subsets\pageoriginale of
$X$. See the example below

\begin{example*}[(von Neumann)]
Let $X$ be the Hilbert space $\ell^{2}$, and let $A\subset \ell^{2}$
be the set of points $x_{mn}$, $m$, $n=1,2,\ldots$, whose coordinates
are
$$
x_{mn}(i)=
\begin{cases}
1, & \text{if~ } i=m\\
m, & \text{if~ } i=n\\
0, & \text{otherwise}
\end{cases}
$$

Then $0$ belongs to weak closure of $A$, but there is no sequence of
points in $A$ which converge weakly to $0$. [Indeed, if there is a
  sequence $x_{m_{j}n_{j}}\rightharpoonup 0$, then
  $(y,x_{m_{j}n_{j}})_{\ell^{2}}\to 0$, for all $y\in \ell^{2}$. Take
  $y=(1,1/2,1/3,\ldots)$ and see that this is not possible. On the
  other hand given any basic (weak) open neighborhood of $0$, $\{x\in
  \ell^{2}:(y,x)_{\ell^{2}}<\epsilon\}$ for arbitrary $y\in \ell^{2}$
  and $\epsilon>0$, we see that $x_{mn}$ belongs to this neighborhood
  if we take $m$ such that $|y_{m}|<\epsilon/2$ and then $n$ such that
  $|y_{n}|<\epsilon/2m$]. 
\end{example*}

However, if the dual $X^{*}$ of $X$ is {\em separable}, then the
induced topology in a {\em bounded} subset $A$ of $X$ by the weak
topology of $X$ is first countable. In particular this is the case if
$X$ is {\em reflexive} and {\em separable}, since this implies $X^{*}$
separable. It is noticeable that in the case when $X$ is {\em
  reflexive} (with no separability assumption made) the following
result holds.

\begin{theorem}[Browder \cite{key19}]\label{chap1-thm1.5}
Let $X$ be a reflexive Banach space, A $a$ bounded subset of $X$,
$x_{0}$ a point in the weak closure of $A$. Then there exists an
infinite sequence $(x_{k})$ in $A$ converging weakly to $x_{0}$ in $X$.
\end{theorem}

\begin{proof}
It suffices to construct a closed separable subspace $X_{0}$ of $X$
such that $x_{0}$ lies in the weak closure of $C$ in $X_{0}$, where
$C=A\cap X_{0}$. Since $X_{0}$ is reflexive and separable, it is first
countable and then there exists a sequence $(x_{k})$ in $C$ which
converges to $x_{0}$ in the weak topology of $X_{0}$. So $(x_{k})$
lies in $A$ and converges to $x_{0}$ in the weak topology of $X$. The
construction of $X_{0}$ goes as follows. Let $B$ be the unit closed
ball in $X^{*}$. For each positive integer $n$, $B^{n}$ is compact in
the product of weak topologies. Now for each fixed integer $m>0$, each
$[\overline{\omega}_{1},\ldots,\overline{\omega}_{n}]\in B^{n}$ has a
(weak) neighborhood $V$ in $B^{n}$ such that 
$$
\bigcap\limits_{[\omega_{1},\ldots,\omega_{n}]\in
  V}\bigcap\limits^{n}_{j=1}\left\{x\in A:|\langle
\omega_{j},x-x_{0}\rangle |<\frac{1}{m}\right\}=\emptyset. 
$$

By compactness we construct a finite set $F_{nm}\subset A$ with the
property that given any $[\omega_{1},\ldots,\omega_{n}]\in B^{n}$
there is $x\in A$ such that $|\langle \omega_{j},x-x_{0}\rangle
|<\frac{1}{m}$ for all\pageoriginale $j=1,\ldots,n$. Now let
$$
A_{0}=\bigcup\limits^{\infty}_{n,m=1}F_{nm}.
$$

Then $A_{0}$ is countable and $x_{0}$ is in weak closure of
$A_{0}$. Let $X_{0}$ be the closed subspace generated by $A_{0}$. So
$X_{0}$ is separable, and denoting by $C=X_{0}\cap A$ it follows that
$x_{0}$ is in the closure of $C$ in the weak topology of $X$. Using
the Hahn Banach theorem it follows that $x_{0}$ is the closure of $C$
in the weak topology of $X_{0}$. 
\end{proof}

\begin{remark*}
The Erberlein-Smulian theorem states: ``Let $X$ be a Banach space and
$A$ a subset of $X$. Let $\overline{A}$ denote its weak closure. Then
$\overline{A}$ is weakly compact if and only if $A$ is weakly
sequentially precompact, i.e., any sequence in $A$ contains a
subsequence which converges weakly''. See Dunford-Schwartz
\cite[p. 430]{key35}. Compare this statement with Theorem
\ref{chap1-thm1.5} and appreciate the difference!
\end{remark*}

\begin{coro*}
In any reflexive Banach space $X$ a weakly lower semicontinuous
functional $\Phi:A\to \mathbb{R}$, where $A$ is a bounded subset of
$X$, is sequentially weakly lower semicontinuous, and conversely.
\end{coro*}
