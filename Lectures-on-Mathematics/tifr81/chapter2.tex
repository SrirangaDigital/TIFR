\chapter{Nemytskii Mappings}\label{chap2}

Let\pageoriginale $\Omega$ be an open subset of $\mathbb{R}^{N}$, $N\geq 1$. A
function $f:\Omega\times \mathbb{R}\to \mathbb{R}$ is said to be a
{\em Carath\'eodary function} if (a) for each fixed $s\in \mathbb{R}$
the function $x\mapsto f(x,s)$ is (Lebesgue) measurable in $\Omega$,
(b) for fixed $x\in \Omega(a.e.)$ the function $s\mapsto f(x,s)$ is
continuous in $\mathbb{R}$. Let $\mathcal{M}$ be the set of all
measurable functions $u:\Omega\to \mathbb{R}$.

\begin{theorem}\label{chap2-thm2.1}
If $f:\Omega\times \mathbb{R}\to \mathbb{R}$ is Carath\'eodory then
the function $x\mapsto f(x,u(x))$ is measurable for all $u\in \mathcal{M}$.
\end{theorem}

\begin{proof}
Let $u_{n}(x)$ be a sequence of simple functions converging a.e.\@ to
$u(x)$. Each function $f(x,u_{n}(x))$ is measurable in view of (a)
above. Now (b) implies that $f(x,u_{n}(x))$ converges a.e.\@ to
$f(x,u(x))$, which gives its measurability.
\end{proof}

Thus a Carath\'eodory function $f$ defines a mapping
$N_{f}:\mathcal{M}\to \mathcal{M}$, which is called a {\em Nemytskii
  mapping}. The mapping $N_{f}$ has a certain type of continuity as
expresed by the following result.

\begin{theorem}\label{chap2-thm2.2}
Assume that $\Omega$ has finite measure. Let $(u_{n})$ be a sequence
in $\mathcal{M}$ which converges in measure to $u\in
\mathcal{M}$. Then $N_{f}u_{n}$ converges in measure to $N_{f}u$.
\end{theorem}

\begin{proof}
By replacing $f(x,s)$ by $g(x,s)=f(x,s+u(x))-f(x,u(x))$ we may assume
that $f(x,0)=0$. And moreover our claim becomes to prove that if
$(u_{n})$ converges in measure to $0$ then $f(x,u_{n}(x))$ also
converges in measure to $0$. So we want to show that given
$\epsilon>0$ there exists $n_{0}=n_{0}(\epsilon)$ such that
$$
|\{x\in \Omega:|f(x,u_{n}(x))|\geq \epsilon\}|<\epsilon\quad \forall
n\geq n_{0},
$$
where\pageoriginale $|A|$ denotes the Lebesgue measure of a set
$A$. Let
$$
\Omega_{k}=\{x\in \Omega:|s|<1/k\Rightarrow |f(x,s)|<\epsilon\}.
$$

Clearly $\Omega_{1}\subset \Omega_{2}\subset\ldots$ and
$\Omega=\bigcup\limits^{\infty}_{k=1}\Omega_{k}(a.e.)$. Thus
$|\Omega_{k}|\to |\Omega|$. So there exists $\overline{k}$ such that
$|\Omega|-|\Omega_{\overline{k}}|<\epsilon/2$. Now let
$$
A_{n}=\{x\in \Omega:|u_{n}(x)|<1/\overline{k}\}.
$$

Since $u_{n}$ converges in measure to $0$, it follows that there
exists $n_{0}=n_{0}(\epsilon)$ such that for all $n\geq n_{0}$ one has
$|\Omega|-|A_{n}|<\epsilon/2$. Now let
$$
D_{n}=\{x\in \Omega:|f(x,u_{n}(x))|<\epsilon\}.
$$

Clearly $A_{n}\cap \Omega_{\overline{k}}\subset D_{n}$. So
$$
|\Omega|-|D_{n}|\leq
(|\Omega|-|A_{n}|)+(|\Omega|-|\Omega_{\overline{k}}|)<\epsilon 
$$
and the claim is proved.
\end{proof}

\begin{remark*}
The above proof is essentially the one in Ambrosetti-Prodi
\cite{key2}. The proof in Vainberg \cite{key78} is due to Nemytskii
and relies heavily in the following result (see references in
Vainberg's book; see also Scorza-Dragoni \cite{key74} and J.-P. Gossez
\cite{key47} for still another proof). ``Let $f:\Omega\times I\to
\mathbb{R}$ be a Carath\'eodory function, where $I$ is some bounded
closed interval in $\mathbb{R}$. Then given $\epsilon>0$ there exists
a closed set $F\subset \Omega$ with $|\Omega\backslash F|<\epsilon$
such that the restriction of $f$ to $F\times I$ is continuous''. This
is a sort of uniform (with respect to $s\in I$) Lusin's Theorem.
\end{remark*}

Now we are interested in knowing when $N_{f}$ maps an $L^{p}$ space in
some other $L^{p}$ space.

\begin{theorem}\label{chap2-thm2.3}
Suppose that there is a constant $c>0$, a function $b(x)\in
L^{q}(\Omega)$, $1\leq q\leq \infty$, and $r>0$ such that
\begin{equation*}
|f(x,s)|\leq c|s|^{r}+b(x),\quad \forall x\in \Omega,\quad \forall
s\in \mathbb{R}.\tag{2.1}\label{chap2-eq2.1} 
\end{equation*}

Then {\rm(a)} $N_{f}$ maps $L^{qr}$ into $L^{q}$, {\rm(b)} $N_{f}$ is
continuous and bounded (that is, it maps bounded sets into bounded sets).
\end{theorem}

\begin{proof}
It follows from \eqref{chap2-eq2.1} using Minkowski inequality
\begin{equation*}
||N_{f^{u}}||_{L^{q}}\leq
c|||u|^{r}||_{L^{q}}+||b||_{L^{q}}=c||u||^{r}_{L^{q^{r}}}+||b||_{L^{q}}\tag{2.2}\label{chap2-eq2.2} 
\end{equation*}
which\pageoriginale gives (a) and the fact that $N_{f}$ is
bounded. Now suppose that $u_{n}\to u$ in $L^{qr}$, and we claim
$N_{f}u_{n}\to N_{f}u$ in $L^{q}$. Given any subsequence of $(u_{n})$
there is a further subsequence (call it again $u_{n}$) such that
$|u_{n}(x)|\leq h(x)$ for some $h\in L^{q^{r}}(\Omega)$. It follows
from \eqref{chap2-eq2.1} that
$$
|f(x,u_{n}(x))|\leq c|h(x)|^{r}+b(x)\in L^{q}(\Omega).
$$

Since $f(x,u_{n}(x))$ converges a.e.\@ to $f(x,u(x))$, the result
follows from the Lebesgue Dominated Convergence Theorem and a standard
result on metric spaces.
\end{proof}

It is remarkable that the sufficient condition \eqref{chap2-eq2.1} is
indeed necessary for a Carath\'eodory function $f$ defining a
Nemytskii map between $L^{p}$ spaces. Indeed

\begin{theorem}\label{chap2-thm2.4}
Suppose $N_{f}$ maps $L^{p}(\Omega)$ into $L^{q}(\Omega)$ for $1\leq
p<\infty$, $1\leq q<\infty$. Then there is a constant $c>0$ and
$b(x)\in L^{q}(\Omega)$ such that
\begin{equation*}
|f(x,s)|\leq c|s|^{p/q}+b(x)\tag{2.3}\label{chap2-eq2.3}
\end{equation*}
\end{theorem}

\begin{remark*}
We shall prove the above theorem for the case when $\Omega$ is
bounded, although the result is true for unbounded domains. It is also
true that if $N_{f}$ maps $L^{p}(\Omega)$, $1\leq p<\infty$ into
$L^{\infty}(\Omega)$ then there exists a function $b(x)\in
L^{\infty}(\Omega)$ such that $|f(x,s)|\leq b(x)$. See Vainberg
\cite{key78}. 
\end{remark*}

Before proving Theorem \ref{chap2-thm2.4} we prove the following
result.

\begin{theorem}\label{chap2-thm2.5}
Let $\Omega$ be a bounded domain. Suppose $N_{f}$ maps $L^{p}(\Omega)$
into $L^{q}(\Omega)$ for $1\leq p<\infty$, $1\leq q<\infty$. Then
$N_{f}$ is continuous and bounded. 
\end{theorem}

\begin{proof}
(a)~ Continuity of $N_{f}$. By proceeding as in the proof of Theorem
  \ref{chap2-thm2.2} we may suppose that $f(x,0)=0$, as well as to
  reduce to the question of continuity at $0$. Suppose by
  contradiction that $u_{n}\to 0$ in $L^{p}$ and $N_{f}u\nrightarrow 0$ in
  $L^{q}$. So by passing to subsequences if necessary we may assume
  that there is a positive constant $a$ such that 
\begin{equation*}
\sum\limits^{\infty}_{n=1}||u_{n}||^{p}_{L^{p}}<\infty\quad\text{and}\quad
\int_{\Omega}|f(x,u_{n}(x))|^{q}\geq a,\quad \forall
n.\tag{2.4}\label{chap2-eq2.4} 
\end{equation*}

Let us denote by
$$
B_{n}=\left\{x\in\Omega:|f(x,u_{n}(x))|>\left(\frac{a}{3|\Omega|}\right)^{1/q}\right\}
$$

In\pageoriginale view of Theorem \ref{chap2-thm2.2} it follows that
$|B_{n}|\to 0$. Now we construct a decreasing sequence of positive
numbers $\epsilon_{j}$, and select a subsequence $(u_{n_{j}})$ of
$(u_{n})$ as follows.

\medskip
\noindent
{\bf 1st step:~} $\epsilon_{1}=|\Omega|~u_{n_{1}}=u_{1}$.
\smallskip

\noindent
{\bf 2nd step:~} choose $\epsilon_{2}<\epsilon_{1}/2$ and such that
$$
\int_{D}|f(x,u_{n_{1}}(x))|^{q}<\frac{a}{3}\ \forall
D\subset\Omega,\quad |D|\leq 2\epsilon_{2},
$$
then choose $n_{2}$ such $|B_{n_{2}}|<\epsilon_{2}$. 
\smallskip

\noindent
{\bf 3rd step:~} choose $\epsilon_{3}<\epsilon_{2}/2$ and such that
$$
\int_{D}|f(x,u_{n_{2}}(x))|^{q}<\frac{a}{3}~\forall D\subset
\Omega,\quad |D|\leq 2\epsilon_{3}.
$$ 
then choose $n_{3}$ such that $|B_{n_{3}}|<\epsilon_{3}$.
\smallskip

And so on. Let $D_{n_{j}}=B_{n_{j}}\backslash
\bigcup\limits^{\infty}_{i=j+1}B_{n_{i}}$. Observe that the $D'_{j}s$ are
pairwise disjoint. Define
$$
u(x)=
\begin{cases}
u_{n_{j}}(x) & \text{if~} x\in D_{n_{j}},\quad j=1,2,\ldots\\
0 & \text{otherwise}
\end{cases}
$$

The function $u$ is in $L^{p}$ in view of \eqref{chap2-eq2.4}. So by
the hypothesis of the theorem $f(x,u(x))$ should be in
$L^{q}(\Omega)$. We now show that this is not the case, so arriving to
contradiction. Let
$$
K_{j}\equiv
\int_{D_{n_{j}}}|f(x,u(x))|^{q}=\int_{D_{n_{j}}}|f(x,u_{n_{j}}(x))|^{q}=\int_{B_{n_{j}}}-\int_{B_{n_{j}}\backslash
  D_{n_{j}}}\equiv I_{j}-J_{j}. 
$$

Next we estimate the integrals in the right side as follows:
\begin{align*}
I_{j} &=
\int_{B_{n_{j}}}|f(x,u_{n_{j}}(x))|^{q}=\int_{\Omega}|f(x,u_{n_{j}}(x))|^{q}-\int_{\Omega\backslash
  B_{n_{j}}}|f(x,u_{n_{j}}(x))|^{q}\\
&\geq a-\frac{a}{3}=\frac{2a}{3}
\end{align*}
and to estimate $J_{j}$ we observe that $B_{n_{j}}\backslash
D_{n_{j}}\subset \bigcup\limits^{\infty}_{i=j+1}B_{n_{i}}$. We see that
$|B_{n_{j}}\backslash D_{n_{j}}|\leq
\sum\limits^{\infty}_{i=j+1}\epsilon_{i}\leq 2\epsilon_{j+1}$. Consequently
$J_{j}<a/3$. Thus $K_{j}\geq a/3$. And so
$$
\int_{\Omega}|f(x,u(x))|^{q}=\sum\limits^{\infty}_{j=1}K_{j}=\infty. 
$$

(b)~ Now\pageoriginale we prove that $N_{f}$ is bounded. As in part
(a) we assume that $f(x,0)=0$. By the continuity of $N_{f}$ at $0$ we
see that there exists $r>0$ such that for all $u\in L^{p}$ with
$||u||_{L^{p}}\leq r$ one has $||N_{f}u||_{L^{p}}\leq 1$. Now given
any $u$ in $L^{p}$ let $n$ (integer) be such that $nr^{p}\leq
||u||^{p}_{L^{p}}\leq (n+1)r^{p}$. Then $\Omega$ can be decomposed
into $n+1$ pairwise disjoint sets $\Omega_{j}$ such that
$\int_{\Omega_{j}}|u|^{p}\leq r^{p}$. So
$$
\int_{\Omega}|f(x,u(x))|^{q}=\sum\limits^{n+1}_{j=1}\int_{\Omega_{j}}|f(x,u(x))|^{q}\leq
n+1\leq \left(\frac{||u||_{L^{p}}}{r}\right)^{p}+1
$$
\end{proof}

\noindent
{\bf Proof of Theorem \ref{chap2-thm2.4}.}~ Using the fact that
$N_{f}$ is bounded we get a constant $c>0$ such that
\begin{equation*}
\int_{\Omega}|f(x,u(x))|^{q}dx\leq c^{q}\quad\text{if}\quad
\int_{\Omega}|u(x)|^{p}\leq 1.\tag{2.5}\label{chap2-eq2.5}
\end{equation*}

Now define the function $H:\Omega\times \mathbb{R}\to \mathbb{R}$ by
$$
H(x,s)=\max\{|f(x,s)|-c|s|^{p/q};0\}.
$$

Using the inequality $\alpha^{q}+(1-\alpha)^{q}\leq 1$ for $0\leq
\alpha\leq 1$ we get
\begin{equation*}
H(x,s)^{q}\leq |f(x,s)|^{q}+c^{q}|s|^{p}\quad\text{for}\quad
H(x,s)>0.\tag{2.6}\label{chap2-eq2.6}
\end{equation*}

Let $u\in L^{p}$ and $D=\{x\in \Omega:H(x,u(x))>0\}$. There exist
$n\geq 0$ integer and $0\leq \epsilon<1$ such that
$$
\int_{D}|u(x)|^{p}dx=n+\epsilon.
$$

So there are $n+1$ disjoint sets $D_{i}$ such that
$$
D=\bigcup\limits^{n+1}_{i=1}D_{i}\q\text{and}\q
\int_{D_{i}}|u(x)|^{p}dx\leq 1.
$$

From \eqref{chap2-eq2.5} we get
$$
\int_{D}|f(x,u(x))|^{q}dx=\sum\limits^{n+1}_{i=1}\int_{D_{i}}|f(x,u(x)))^{q}dx\leq
(n+1)c^{q}. 
$$

Then using this estimate in \eqref{chap2-eq2.6} we have
\begin{equation*}
\int_{\Omega}H(x,u(x))^{q}\leq (n+1)c^{q}-(n+\epsilon)c^{q}\leq
c^{q}\tag{2.7}\label{chap2-eq2.7} 
\end{equation*}
which\pageoriginale then holds for all $u\in L^{p}$.

Now using the Lemma below we see that for each positive integer $k$
there exists $u_{k}\in \mathcal{M}$ with $|u_{k}(x)|\leq k$ such that
$$
b_{k}(x)=\sup\limits_{|s|\leq k}H(x,s)=H(x,u_{k}(x)).
$$

It follows from \eqref{chap2-eq2.7} that $b_{k}(x)\in L^{q}(\Omega)$
and $||b_{k}||_{L^{q}}\leq c$. Now let us define the function $b(x)$
by
\begin{equation*}
b(x)\equiv \sup\limits_{-\infty<s<\infty}H(x,s)=\lim\limits_{k\to
  \infty}b_{k}(x).\tag{2.8}\label{chap2-eq2.8} 
\end{equation*}

It follows from Fatou's lemma that $b(x)\in L^{q}$ and
$||b||_{L^{q}}\leq c$. From \eqref{chap2-eq2.8} we finally obtain
\eqref{chap2-eq2.3}.\hfill$\Box$

\begin{lemma*}
Let $f:\Omega\times I\to \mathbb{R}$ be a Carath\'edory function,
where $I$ is some fixed bounded closed interval. Let us define the
function
$$
c(x)=\max\limits_{s\in I}f(x,s).
$$

Then $c\in \mathcal{M}$ and there exists $\overline{u}\in \mathcal{M}$
such that
\begin{equation*}
c(x)=f(x,\overline{u}(x)).\tag{2.9}\label{chap2-eq2.9} 
\end{equation*}
\end{lemma*}

\begin{proof}
\begin{enumerate}
\renewcommand{\theenumi}{\roman{enumi}}
\renewcommand{\labelenumi}{(\theenumi)}
\item For each fixed $s$ the function $x\mapsto f_{s}(x)$ is
  measurable. We claim that
$$
c(x)=\sup \{f_{s}(x):s\in I,\q s-\text{rational}\}
$$
showing then that $c$ is measurable. To prove the claim let $x_{0}\in
\Omega(a.e.)$ and choose $s_{0}\in I$ such that
$c(x_{0})=f(x_{0},s_{0})$. Since $s_{0}$ is a limit point of rational
numbers and $f(x_{0},s)$ is a continuous function the claim is proved.

\item For each $x\in\Omega(a.e.)$ let $F_{x}=\{s\in I:f(x,s)=c(x)\}$
  which is a closed set. Let us define a function
  $\overline{u}:\Omega\to \mathbb{R}$ by
  $\overline{u}(x)=\min_{s}F_{x}$. Clearly the function $\overline{u}$
  satisfies the relation in \eqref{chap2-eq2.9}. It remains to show
  that $\overline{u}\in \mathcal{M}$. To do that it suffices to prove
  that the sets
$$
B_{\alpha}=\{x\in \Omega:\overline{u}(x)>\alpha\}\ \forall \alpha \in
I
$$
are measurable. [Recall that $\overline{u}(x)\in I$ for $x\in
  \Omega$]. Let $\beta$ be the lower end of $I$. Now fixed $\alpha\in
I$ we define the function $c_{\alpha}:\Omega\to \mathbb{R}$ by
$$
c_{\alpha}(x)=\max\limits_{\beta\leq s\leq \alpha}f(x,s)
$$
which\pageoriginale is measurable by part (i) proved above. The proof
is completed by observing that
$$
B_{\alpha}=\{x\in \Omega:c(x)>c_{\alpha}(x)\}.
$$
\end{enumerate}
\end{proof}

\begin{remark*}
The Nemytskii mapping $N_{f}$ defined from $L^{p}$ into $L^{q}$ with
$1\leq p<\infty$, $1\leq q<\infty$ is not compact in general. In fact,
the requirement that $N_{f}$ is compact implies that there exists
a $b(x)\in L^{q}(\Omega)$ such that $f(x,s)=b(x)$ for all $s\in
\mathbb{R}$. See Krasnoselskii \cite{key53}.
\end{remark*}

\noindent
{\bf The Differentiability of Nemytskii Mappings.}~ Suppose that a
Cara\-th\'eodory function $f(x,s)$ satisfies condition
\eqref{chap2-eq2.3}. Then it defines a mapping from $L^{p}$ into
$L^{q}$. It is natural to ask: if $f(x,s)$ has a partial derivative
$f'_{s}(x,s)$ with respect to $s$, which is also a Carath\'eodory
function, does $f'_{s}(x,s)$ define with respect to $s$, which is also
a Carath\'eodory function, does $f'_{s}(x,s)$ define a Nemytskii map
between some $L^{p}$ spaces? In view of Theorem \ref{chap2-thm2.4} we
see that the answer to this question is no in general. The reason is
that \eqref{chap2-eq2.3} poses no restriction on the growth of the
derivative. Viewing the differentiability of a Nemytskii-mapping
$N_{f}$ associated with a Carath\'eodory function $f(x,s)$ we {\em
  start assuming} that $f'_{s}(x,s)$ is Carath\'eodory and
\begin{equation*}
|f'_{s}(x,s)|\leq c|s|^{m}+b(x),\q \forall s\in \mathbb{R}\q \forall
x\in \Omega.\tag{2.10}\label{chap2-eq2.10}
\end{equation*}
where $b(x)\in L^{n}(\Omega)$, $1\leq n\leq \infty$,
$m>0$. Integrating \eqref{chap2-eq2.10} with respect to $s$ we obtain
\begin{equation*}
|f(x,s)|\leq
\frac{c}{m+1}|s|^{m+1}+b(x)|s|+a(x),\tag{2.11}\label{chap2-eq2.11} 
\end{equation*}
where $a(x)$ is an arbitrary function. Shortly we impose a condition
on $a(x)$ so as to having a Nemytskii map defined between adequate
$L^{p}$ spaces. Using Young's inequality in \eqref{chap2-eq2.11} we
have
$$
|f(x,s)|\leq
\frac{c+1}{m+1}|s|^{m+1}+\frac{m}{m+1}b(x)^{(m+1)/m}+a(x). 
$$

Observe that the function $b(x)^{(m+1)/m}$ is in $L^{q}(\Omega)$,
where $q=mn/(m+1)$. So if we pick $a\in L^{q}$ it follows from Theorem
\ref{chap2-thm2.3} that (assuming \eqref{chap2-eq2.10}): 
\begin{gather*}
N_{f}:L^{p}\to L^{q}\q p=mn\q \text{and}\q
q=mn/(m+1)\tag{2.12}\label{chap2-eq2.12}\\ 
N_{f'_{*}}:L^{p}\to L^{n}.\tag{2.13}\label{chap2-eq2.13} 
\end{gather*}

Now we are ready to study the differentiability of the mapping
$N_{f}$. 

\begin{theorem}\label{chap2-thm2.6}
Assume\pageoriginale \eqref{chap2-eq2.10} and the notation in
\eqref{chap2-eq2.12} and \eqref{chap2-eq2.13}. Then $N_{f}$ is
continuously Fr\'echet differentiable with 
$$
N'_{f}:L^{p}\to \mathcal{L}(L^{p},L^{q})
$$ 
defined by
\begin{equation*}
N'_{f}(u)[v]=N_{f'_{*}}(u)v(=f'_{s}(x,u(x))v(x)),\q \forall u, v\in
L^{p}.\tag{2.14}\label{chap2-eq2.14} 
\end{equation*}
\end{theorem}

\begin{proof}
We first observe that under our hypotheses the function $x\mapsto
f'_{s}(x,u(x))v(x)$ is in $L^{q}(\Omega)$. Indeed by H\"older's
inequality
$$
\int_{\Omega}|f'_{s}(x,u(x))v(x)|^{q}\leq
\left(\int_{\Omega}|f'_{s}(x,v(x)|^{pq/(p-q)/p})^{(p-q)/p}\right)\left(\int_{\Omega}|v(x)|^{p}\right)^{q/p}.
$$

Observe that $pq/(p-q)=n$ and use \eqref{chap2-eq2.13} above. Now we
claim that for fixed $u\in L^{p}$
$$
\omega(v)\equiv N_{f}(u+v)-N_{f}(u)-f'_{s}(x,u)v
$$
is $o(v)$ for $v\in L^{p}$, that is
$||\omega(v)||_{L^{q}}/||v||_{L^{p}}\to 0$ as $||v||_{L^{p}}\to
0$. Since
\begin{align*}
f(u(x)+v(x))-f(u(x)) &= \int^{1}_{0}\frac{d}{dt}f(x,u(x)+tv(x))dt\\
&= \int^{1}_{0}f'_{s}(x,u(x)+tv(x))v(x)dt
\end{align*}
we have
$$
\int_{\Omega}|\omega(v)|^{q}dx=\int_{\Omega}|\int^{1}_{0}[f'_{s}(x,u(x)+tv(x))-f'_{s}(x,u(x))]v(x)dt|^{q}dx. 
$$

Using H\"older's inequality and Fubini, we obtain
\begin{align*}
& \int_{\Omega}|\omega(v)|^{q}dx\leq\\
& \leq
  \left(\int^{1}_{0}\int_{\Omega}|f'_{s}(x,u(x)+tv(x))-f'_{s}(x,u(x))|^{n}dx\ dt\right)^{q/n}||v||^{q}_{L^{p}}. 
\end{align*}

Using \eqref{chap2-eq2.13} and the fact that $N_{f'_{*}}$ is a
continuous operator we have the claim proved. The continuity of
$N'_{f}$ follows readily \eqref{chap2-eq2.14} and
\eqref{chap2-eq2.13}. 
\end{proof}

\begin{remark*}
We observe that in the previous theorem $p>q$, since we have assumed
$m>0$. What happens if $m=0$, that is
$$
|f'_{s}(x,s)|\leq b(x)
$$
where $b(x)\in L^{n}(\Omega)$? First of all we observe that
$$
N_{f'_{*}}:L^{p}\to L^{n}\q \forall p\geq 1
$$
and\pageoriginale proceeding as above (supposing $1\leq n<\infty$)
$$
N_{f}:L^{p}\to L^{q}\q\forall p\geq 1\q\text{and}\q q=np/(n+p)
$$
and we are precisely in the same situation as in \eqref{chap2-eq2.12},
\eqref{chap2-eq2.13}. Now assume $n=+\infty$, i.e., there exists $M>0$
\begin{equation*}
|f'_{s}(x,s)|\leq M\q\forall x\in \Omega,\q \forall s\in
\mathbb{R}.\tag{2.15}\label{chap2-eq2.15} 
\end{equation*}

Integrating we obtain
\begin{equation*}
|f(x,s)|\leq M|s|+b(x)\tag{2.16}\label{chap2-eq2.16}
\end{equation*}

It follows under \eqref{chap2-eq2.15} and \eqref{chap2-eq2.16} that
\begin{gather*}
N_{f'_{*}}:L^{p}\to L^{\infty}\q \forall 1\leq p\leq \infty\\
N_{f}:L^{p}\to L^{p}\q (\text{taking}~ b\in L^{p}).
\end{gather*}
\end{remark*}

It is interesting to observe that such an $N_{f}$ cannot be Fr\'echet
differentiable in general. Indeed:

\begin{theorem}\label{chap2-thm2.7}
Assume \eqref{chap2-eq2.15}. If $N_{f}:L^{p}\to L^{p}$ is Fr\'echet
differentiable then there exist functions $a(x)\in L^{\infty}$ and
$b(x)\in L^{p}$ such that $f(x,s)=a(x)s+b(x)$.
\end{theorem}

\begin{proof}
\begin{enumerate}
\renewcommand{\theenumi}{\alph{enumi}}
\renewcommand{\labelenumi}{(\theenumi)}
\item Let us prove that the G\^ateaux derivative of $N_{f}$ at $u$ in
  the direction $v$ is given by
$$
\frac{d}{dv}N_{f}(u)=f'_{s}(x,u(x))v(x).
$$

First we observe that $f'_{s}(x,u(x))v(x)\in L^{q}$. So we have to
prove that
$$
\omega_{t}(x)\equiv
t^{-1}[f(x,u(x)+tv(x))-f(x,u(x))]-f'_{s}(x,u(x))v(x)
$$
goes to $0$ in $L^{p}$ as $t\to 0$. As in the proof of Theorem
\ref{chap2-thm2.6} we write
$$
\omega_{t}(x)=\int^{1}_{0}[f'_{s}(x,u(x)+t\tau
  v(x))-f'_{s}(x,u(x))]v(x)d\tau. 
$$

So
{\fontsize{10pt}{12pt}\selectfont
$$
\int_{\Omega}|\omega_{t}(x)|^{p}dx\leq
\int^{1}_{0}\int_{\Omega}|f'_{s}(x,u(x)+t\tau
v(x))-f'_{s}(x,u(x))|^{p}|v(x)|^{p}dx\ d\tau. 
$$}

Now\pageoriginale for each $\tau\in [0,1]$ and each $x\in
\Omega(a.e.)$ the integrand of the double integral goes to zero. On
the other hand this integrand is bounded by $(2M)^{p}|v(x)|^{p}$. So
the result follows by the Lebesgue Dominated convergence Theorem.

\item Now suppose $N_{f}$ is Fr\'echet differentiable. Then its
  Fr\'echet derivative is equal to the G\^ateaux derivative, and
  assuming that $f(x,0)=0$ we have that
\begin{equation*}
||u||^{-1}_{L^{p}}||f(x,u)-f'_{s}(x,0)u||_{L^{p}}\to 0\q\text{as}\q
||u||_{L^{1}}\to 0.\tag{2.17}\label{chap2-eq2.17} 
\end{equation*}
Now for each fixed $\ell\in \mathbb{R}$ and $x_{0}\in \Omega$ consider
a sequence $u_{\delta}(x)=\ell_{\chi B_{\delta}(x_{0})}$, i.e., a
multiple of the characteristic function of the ball
$B_{\delta}(x_{0})$. For such functions the expression in
\eqref{chap2-eq2.17} raised to the power $p$ can be written as 
$$
\frac{1}{\ell^{p}vol\ B_{\delta}(x_{0})}\int_{B_{\delta}(x_{0})}|f(x,\ell)-f'_{s}(x,0)\ell|^{p}dx. 
$$
So taking the limit as $\delta\to 0$ we obtain
$$
\frac{1}{\ell^{p}}|f(x_{0},\ell)-f'_{s}(x_{0},0)\ell|=0,\q x_{0}\in
\Omega(a.e.) 
$$
which shows that $f(x_{0},\ell)=f'_{s}(x_{0},0)\ell$. Since the
previous arguments can be done for all $x_{0}\in \Omega(a.e.)$ and all
$\ell\in \mathbb{R}$, we obtain that $f(x,s)=a(x)s$ where
$a(x)=f'_{s}(x,0)$ is an $L^{\infty}$ function.
\end{enumerate}
\end{proof}

\noindent
{\bf The Potential of a Nemytskii Mapping.}~ Let $f(x,s)$ be a
Carath\'eodory function for which there are constants $0<m$, $1\leq
p\leq \infty$ and a function $b(x)\in L^{p/m}(\Omega)$ such that
$$
|f(x,s)|\leq c|s|^{m}+b(x).
$$

Denoting by $F(x,s)=\int^{s}_{0}f(x,\tau)d\tau$ we obtain that
$$
|F(x,s)|\leq c_{1}|s|^{m+1}+c(x)
$$
where $c(x)\in L^{p/(m+1)}(\Omega)$. (See the paragraphs before
Theorem \ref{chap2-thm2.7}). Then $N_{f}:L^{p}\to L^{p/m}$ and
$N_{F}:L^{p}\to L^{p/(m+1)}$.

In particular, if $p=m+1$, $(\Rightarrow p>1)$ the inequalities above
become 
\begin{gather*}
|f(x,s)|\leq c|s|^{p-1}+b(x),\q b(x)\in
L^{p'},\q\frac{1}{p}+\frac{1}{p'}=1\tag{2.18}\label{chap2-eq2.18}\\
|F(x,s)|\leq c_{1}|s|^{p}+c(x),\q c(x)\in L^{1} 
\end{gather*}
and\pageoriginale we have that $N_{f}:L^{p}\to L^{p'}$ and
$N_{F}:L^{p}\to L^{1}$.

\begin{theorem}\label{chap2-thm2.8}
Assume \eqref{chap2-eq2.18}. Then
$$
\Psi(u)=\int_{\Omega}F(x,u(x))dx
$$
defines a continuous functional $\Psi:L^{p}(\Omega)\to \mathbb{R}$,
which is continuously Fr\'echet differentiable.
\end{theorem}

\begin{proof}
The continuity of $N_{F}$ implies that $\Psi$ is continuous. We claim
that $\Psi'=N_{f}$. So all we have to do is proving that
$$
\omega(v)\equiv
\int_{\Omega}F(x,u+v)-\int_{\Omega}F(x,u)-\int_{\Omega}f(x,u)v=o(v) 
$$
as $v\to 0$ in $L^{p}$. As in the calculations done in the proof of
Theorem \ref{chap2-thm2.6} we obtain
$$
\omega(v)=\int_{\Omega}\int^{1}_{0}[f(x,u+tv)-f(x,u)]v\ dt\ dx.
$$

Using Fubini's theorem and H\"older's inequality
$$
|\omega(v)|\leq
\int^{1}_{0}||N_{f}(u+tv)-N_{f}(u)||_{L^{p'}}dt||v||_{L^{p}}. 
$$

The integral in the above expression goes to zero as $||v||_{L^{p}}\to
0$ by the Lebesgue Dominated Convergence Theorem with an application
of Theorem \ref{chap2-thm2.3}. So
$$
||v||^{-1}_{L^{p}}\omega(v)\to 0\q\text{as}\q ||v||_{L^{p}}\to 0.
$$
\end{proof}


