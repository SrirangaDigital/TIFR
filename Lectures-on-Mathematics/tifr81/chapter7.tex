\chapter{Support Points and Suport Functionals}\label{chap7}

{\bf Introduction.}~ Let\pageoriginale $X$ be a Banach space and $C$ a
closed convex subset of $X$. We {\em always assume} that $C\neq X$ and
$C\neq \emptyset$. A point $x_{0}\in C$ is said to be a {\em support
  point} if there exists a bounded linear functional $f\in X^{*}$ such
that $f(x_{0})=\Sup_{C}f$. A given functional $f\in X^{*}$ is said to
be a {\em support functional} if there exists $x_{0}\in C$ such that
$f(x_{0})=\Sup_{C}f$. We {\em always assume} that $f\neq 0$. The
terminology ``support'' comes from the geometric fact that the
hyperplane $H$ determined by $f$, where $H=\{x\in X:f(x)=f(x_{0})\}$,
touches $C$ at $x_{0}$ and leaves $C$ in one of half spaces determined
by $H$. Two basic questions will be studied in this chapter. 

\begin{problem}\label{chap7-prob1}
Given $C$ a closed convex subset of $X$. Are all points in the
boundary $\p C$ of $x$ support points? If not, how large is the set of
support points?
\end{problem}

\begin{problem}\label{chap7-prob2}
Given $C$ a closed convex subset of $X$. Are all functionals $f\in
X^{*}$ support functionals? If not, how large is the set of support
functionals of a given $C$?
\end{problem}

\noindent
{\bf Six Remarks and Examples.}
\begin{enumerate}
\renewcommand{\labelenumi}{(\theenumi)}
\item Of course the above questions make sense if $f$ is bounded on
  $C$, i.e., there exists $M\in \mathbb{R}$ such that $f(x)\leq M$ for
  all $x\in C$. This will be achieved in particular if $C$ is
  bounded. In many cases studied here we assume that $C$ is a closed
  bounded convex subset of $X$. 

\item Given\pageoriginale $f$ and $C$, it is not true in general that
  $f$ supports $C$ at some point. For example:
$$
C=\{(x,y)\in \mathbb{R}^{2}:x>0, y>0\;\;\text{and}\;\; xy\geq
1\}\;\;\text{and}\;\; f(x,y)=-y.
$$

However if $C$ is a closed convex bounded subset of $\mathbb{R}^{N}$
(or more generally of a reflexive Banach space $X$) then any
continuous linear functional $f$ supports $C$. This follows readily
from Theorem \ref{chap1-thm1.1}: $C$ is weakly compact and $-f$ is
weakly continuous. 

\item The previous result is false in general if $X$ is not
  reflexive. Example: Let $X$ be the Banach space of all continuous
  functions $x:[0,1]\to \mathbb{R}$ with $x(0)=x(1)=0$ and the norm
  $||x||_{\infty}=\Max\{|x(t)|:t\in [0,1]\}$. Consider the continuous
  linear functional $f(x)=\int^{1}_{0}x(t)dt$ and let $C$ be unit
  closed ball in $X:\{x\in X:||x||_{\infty}\leq 1\}$. Clearly $f$ does
  not support $C$. However see Theorem \ref{chap7-thm7.2} below. 

\item Let $C$ be a closed bounded convex subset of a Banach space and
  let $x_{0}\in \p C$. It is not true in general that there exists a
  functional $f$ supporting $C$ at $x_{0}$. Example: let
$$
C=\left\{\xi\in \ell^{2}:\xi_{j}\geq 0,\q
||\xi||^{2}=\sum\limits^{\infty}_{j=1}\xi^{2}_{j}\leq 1\right\}.
$$

We claim first that $C=\p C$. Indeed, given $\xi\in C$ and
$\epsilon>0$, let $n_{0}$ be chosen such that
$|\xi_{n_{0}}|<\epsilon/2$. The point
$$
\widetilde{\xi}=(\xi_{1},\ldots,\xi_{n_{0}-1},-\epsilon/2,\xi_{n_{0}+1},\ldots)\not\in C
$$
and
$$
||\xi-\widetilde{\xi}||=|\xi_{n_{0}}+\frac{\epsilon}{2}|<\epsilon. 
$$

Next we show that the points $\hat{\xi}\in C$, with
$\hat{\xi}_{j}>0$ and $||\hat{\xi}||<1$, are not support
points. Indeed, fix one such $\hat{\xi}$ and suppose that there
exists a functional $f$ such that $\Sup_{C}f=f(\hat{\xi})$. Since
$0\in C$ it follows that $0\leq f(\hat{\xi})$. Also there exists a
$t>1$ such that $t\hat{\xi}\in C$. So $f(t\hat{\xi})\leq
f(\hat{\xi})$, which then implies that $f(\hat{\xi})=0$. By
the Riesz representation theorem let $f=(\eta_{1},\eta_{2},\ldots)\in
\ell^{2}$. So we have
$$
\sum^{\infty}_{j+1}\eta_{j}\hat{\xi}_{j}=0,
$$
which\pageoriginale implies that there exists a $j_{0}$ such that
$\eta_{j_{0}}>0$. Since the point
$e_{j_{0}}=(0,\ldots,0,1,0,\ldots)\in C$, [here $1$ is in the
  $j^{th}_{0}$ component and $0$ in the remaining ones] and
$f(e_{j_{0}})=\eta_{j_{0}}>0$  contradicting the fact proved above
that $\Sup_{C}f=0$.

\item However if the closed bounded convex set $C$ has an interior,
  then all points on the boundary $\p C$ are support points. This is
  an immediate consequence of the Hahn Banach theorem: given any
  $x_{0}\in \p C$ there exists a functional $f\in X^{*}$ which
  separates $x_{0}$ and $\Int C$. As we saw in the example in 4 above,
  if the interior of $C$ is empty then there are points in $C(=\p C)$
  which are not support points. However, see Theorem
  \ref{chap7-thm7.1}, which then provides a satisfactory answer to
  Problem \ref{chap7-prob1}.

\item For Problem \ref{chap7-prob2}, the example in (3) above provides
  a negative answer to the first question. The second question in
  answered in Theorem \ref{chap7-thm7.2}.
\end{enumerate}


\begin{theorem}[Bishop-Phelps \cite{key12}]\label{chap7-thm7.1}
Let $C$ be a closed convex subset of a Banach space. Then the set of
support points of $C$ are dense in $\p C$.
\end{theorem}

\begin{theorem}[Bishop-Phelps \cite{key12}]\label{chap7-thm7.2}
Let $C$ be a closed bounded convex subset of a Banach space $X$. Then
the set of continuous linear functionals which support $C$ is dense in
$X^{*}$. 
\end{theorem}

The proof of Theorem \ref{chap7-thm7.1} relies on Theorem
\ref{chap7-thm7.4} which will follow from the result below, whose
proof uses the Ekeland Variational Principle.

\begin{theorem}[The Drop Theorem, Dane\u{s} \cite{key30}]\label{chap7-thm7.3}
Let $S$ be a closed subset of Banach space $X$. Let $y\in X\backslash
S$ and $R=\dist (y,S)$. Let $r$ and $\rho$ be prositive real numbers
such that $0<r<R<\rho$. Then there exists $x_{0}\in S$ such that
\begin{equation*}
||y-x_{0}||\leq \rho\q\text{and}\q D(y,r;x_{0})\cap
S=\{x_{0}\}\tag{7.1}\label{chap7-eq7.1} 
\end{equation*}
where $D(y,r;x_{0})=co(\overline{B}_{r}(y)\cup \{x_{0}\})$. [This set
  is called a ``drop'', in view of its evocative geometry].
\end{theorem}

\begin{remark*}
By definition $\dist(y,S)=\Inf\{||y-x||:x\in S\}$. If $X$ is reflexive
this infimum is achieved, but in general this is not so. The notation
``co'' above means the convex hull. And $\overline{B}_{r}(y)=\{x\in
X:||x-y||\leq r\}$.
\end{remark*}

\noindent
{\bf Proof of Theorem \ref{chap7-thm7.3}.}~ By\pageoriginale a
translation we may assume that $y=0$. Let
$F=\overline{B}_{\rho}(0)\cap S$ which is a closed subset of $X$, and
consequently a complete metrix space with a distance induced naturally
by the norm of $X$. Define the following functional $\Phi:F\to
\mathbb{R}$ by
$$
\Phi(x)=\frac{\rho+r}{R-r}||x||.
$$

By the Ekeland Variational Principle, given $\epsilon=1$ there exists
$x_{0}\in F$ such that
\begin{equation*}
\Phi(x_{0})<\Phi(x)+||x-x_{0}||.\tag{7.2}\label{chap7-eq7.2}
\end{equation*}

Such an $x_{0}$ satisfies the first requirement of \eqref{chap7-eq7.1}
and now we claim that $\{x_{0}\}=D(0,r;x_{0})\cap S$. Suppose by
contradiction that there is another point $x\neq x_{0}$ in this
intersection. So
\begin{equation*}
x\in S\q\text{and}\q x=(1-t)x_{0}+tv\tag{7.3}\label{chap7-eq7.3}
\end{equation*}
for some $v\in \overline{B}_{r}(0)$ and $0\leq t\leq 1$.

Clearly $0<t<1$. From \eqref{chap7-eq7.3}: $||x||\leq
(1-t)||x_{0}||+t||v||$, which gives 
\begin{equation*}
t(R-r)\leq t(||x_{0}||-||v||)\leq
||x_{0}||-||x||.\tag{7.4}\label{chap7-eq7.4} 
\end{equation*}

If follows from \eqref{chap7-eq7.2} and \eqref{chap7-eq7.3} that
$$
\frac{\rho+r}{R-r}||x_{0}||<\frac{\rho+r}{R-r}||x||+||x-x_{0}||=\frac{\rho+r}{R-r}||x||+t||x_{0}-v||. 
$$

Using \eqref{chap7-eq7.4} to estimate $t$ in the above inequality and
estimating $||x_{0}-v||\leq \rho+r$, we obtain
$||x_{0}||<||x||+(||x_{0}||-||x||)$, which is impossible!\hfill$\Box$

\begin{remark*}
The above theorem is due to Danes, who gave in \cite{key30} a proof,
different from the above one, using the following result of
Krasnoselskii and Zabreiko \cite{key55}: ``Let $X$ be a Banach space
and let $x$ and $y$ be given points in $X$ such that
$0<r<\rho<||x-y||$. Then
$$
\diam[D(x,r;y)\backslash B_{\rho}(x)]\leq
\frac{2[||x-y||+r]}{||x-y||-r}(||x-y||-\rho)''. 
$$

The above proof is essentially the one in Br{\o}ndsted
\cite{key17}. Relations between the Drop Theorem and the Ekeland
Variational Principle have been pointed out by several people, Brezis
and Browder \cite{key16}, Dane\u{s} \cite{key31}, Penot \cite{key61}. 
\end{remark*}

\begin{theorem}[Browder \cite{key20}]\label{chap7-thm7.4}
Let\pageoriginale $S$ be a closed subset of a Banach space $X$. Let
$\epsilon>0$ and 
$z\in \p S$. Then there exist $\delta>0$, a convex closed cone $K$
with non-empty interior and $x_{0}\in S$ such that
\begin{equation*}
||x_{0}-z||<\epsilon\q\text{and}\q S\cap (x_{0}+K)\cap
B_{\delta}(x_{0})=\{x_{0}\}.\tag{7.5}\label{chap7-eq7.5} 
\end{equation*}
\end{theorem}

\begin{remark*}
For the sake of geometric images, the above theorem means that: ``a
closed set $S$ satisfies a local (exterior) cone condition on a dense
set of $\p S$''.
\end{remark*}

\noindent
{\bf Proof of Theorem \ref{chap7-thm7.4}.}~ Chose $y\not\in S$ such
that $||z-y||\leq \epsilon/3$. Then $R\equiv \dist (y,S)\leq
\epsilon/3$. Take $\rho=\epsilon/2$ and choose $r<R$. By the Drop
Theorem, there exists $x_{0}\in S$ such that
\begin{equation*}
||x_{0}-y||\leq \epsilon/2\q\text{and}\q D(y,r;x_{0})\cap
S=\{x_{0}\}\tag{7.6}\label{chap7-eq7.6} 
\end{equation*}

Since $||x_{0}-z||\leq ||x_{0}-y||+||y-z||\leq \epsilon/2+\epsilon/3$,
the first assertion in \eqref{chap7-eq7.5} follows. For the second one
take $\delta<||x_{0}-y||-r$. It suffices to prove that the points
\begin{equation*}
x=x_{0}+t(v-x_{0})\q\text{with}\q t\geq 0,\q v\in
\overline{B}_{r}(y),\q ||x-x_{0}||<\delta\tag{7.7}\label{chap7-eq7.7} 
\end{equation*}
are in the drop $D(y,r;x_{0})$. Then we would take the cone $K$ as the
set of halflines with end point at $0$ and passing through the points
of the ball $\overline{B}_{r}(y-x_{0})$, i.e.
$$
K=\{u\in X:u=t(v-x_{0}),\q t\geq 0\q v\in \overline{B}_{r}(y)\}
$$

So $K+x_{0}=\{x_{0}+t(v-x_{0}):t\geq 0, v\in \overline{B}_{r}(y)\}$ as
in \eqref{chap7-eq7.7}. To prove the above claim, all we have to do is
to show that the $t$'s in \eqref{chap7-eq7.7} have to be $\leq 1$, and
so the $x$ in \eqref{chap7-eq7.7} is indeed a point in the drop
$D(y,r;x_{0})$. We rewrite $x$
$$
x=x_{0}+t(y-x_{0})+t(v-y)\Rightarrow t(y-x_{0})+t(v-y)=x-x_{0}
$$

Estimating we obtain $t||y-x_{0}||-t||v-y||\leq ||x-x_{0}||<\delta$ or
$$
\displaylines{
\hfill t(||y-x_{0}||-r)<\delta<||x_{0}-y||-r\Rightarrow t<1.\hfill\llap{$\Box$}
}
$$

\noindent
{\bf Proof of Theorem \ref{chap7-thm7.1}.}~ Let $z\in \p C$ and
$\epsilon>0$ be given. By Theorem \ref{chap7-thm7.4} there exists
$x_{0}\in \p C$, $K$ and $\delta>0$ such that
\begin{equation*}
C\cap (x_{0}+K)\cap B_{\delta}(x_{0})=\{x_{0}\},\q
||x_{0}-z||<\epsilon.\tag{7.8}\label{chap7-eq7.8} 
\end{equation*}

Now\pageoriginale we claim that in fact $C\cap
(x_{0}+K)=\{x_{0}\}$. Otherwise let $x\neq x_{0}$ with $x\in C\cap
(x_{0}+K)$; then $\overline{x}=x_{0}+t(x-x_{0})$ for small to $t>0$ is
$\neq 0$ and belongs to $C\cap (x_{0}+x)\cap B_{\delta}(x_{0})$,
contradicting \eqref{chap7-eq7.8}. Next, based in the assertion just
proved we see that $C$ and $U=\Int(K+x_{0})$ are disjoint. By the Hahn
Banach theorem there exists a continuous linear functional $f\in
X^{*}$ such that $\Sup_{C}f\leq \Inf_{u\in U}f(u)$. So $\Sup_{C}f\leq
f(x_{0})$, and indeed there is equality because $x_{0}\in
C$.\hfill$\Box$

To prove Theorem \ref{chap7-thm7.2} we will use two lemmas due to
Phelps \cite{key62}.

\begin{lemma}\label{chap7-lem7.6}
Let $S$ be a closed subset of a Banach space $X$. Let $f\in X^{*}$ be
such that $||f||_{X^{*}}=1$ and $\Sup_{S}f<\infty$, and let
$0<k<1$. Then the set $K$ defined below is a closed convex cone
\begin{equation*}
K=\{x\in X:k||x||\leq f(x)\}.\tag{7.9}\label{chap7-eq7.9}
\end{equation*}

Moreover, for all $z\in S$ there exists $x_{0}\in S$ such that
\begin{equation*}
x_{0}\in z+K\q S\cap (x_{0}+K)=\{x_{0}\}.\tag{7.10}\label{chap7-eq7.10}
\end{equation*}
\end{lemma}

\begin{proof}
The verification that $K$ is a non-empty closed convex cone is
straightforward. To prove the second assertion let $F=(z+K)\cap S$ and
consider the functional $\Phi:F\to \mathbb{R}$ defined as
$\Phi=-f|_{F}$. Take $\epsilon<k$ and use the Ekeland Variational
Principle: there exists $x_{0}\in F$ such that
\begin{equation*}
-f(x_{0})<-f(x)+\epsilon||x_{0}-x||,\q \forall x\in F,\q x\neq
x_{0}.\tag{7.11}\label{chap7-eq7.11}
\end{equation*}

Let $y\in S\cap (x_{0}+K)$. First we claim that $y\in F$; indeed,
since $y-x_{0}\in K$ and $x_{0}-z\in K$ it follows that $y-z\in
K$. Next we show that $y=x_{0}$. Otherwise, from \eqref{chap7-eq7.11}
\begin{equation*}
-f(x_{0})<-f(y)+\epsilon||y-x_{0}||\tag{7.12}\label{chap7-eq7.12}
\end{equation*}

Since $y-x_{0}\in K$ we have $k||y-x_{0}||\leq f(y-x_{0})$. This
together with \eqref{chap7-eq7.12} gives a contradiction.
\end{proof}

\begin{lemma}\label{chap7-lem7.7}
Let $C$ be a closed convex subset of a Banach space $X$. Let $f\in
X^{*}$, $||f||_{X^{*}}=1$ and $0<k<1$. Let $K$ be as in
\eqref{chap7-eq7.9}. Suppose that $x_{0}\in C$ is such that
\begin{equation*}
C\cap (x_{0}+K)=\{x_{0}\}.\tag{7.13}\label{chap7-eq7.13} 
\end{equation*}

Then\pageoriginale there exists $0\neq g\in X^{*}$ such that
\begin{equation*}
\Sup_{C}g=g(x_{0})\q  ||g-f||_{X^{*}}\leq k.\tag{7.14}\label{chap7-eq7.14}
\end{equation*}
\end{lemma}

\begin{proof}
Consider the functional $\Phi:X\to \mathbb{R}$ defined by
$$
\Phi(x)=k||x||-f(x)
$$

Clearly $\Phi$ is continuous and convex. Now apply the Hahn-Banach
theorem to separate the sets:
\begin{gather*}
C_{1}=\{(x,r)\in X\times \mathbb{R}:\Phi(x)<r\}\\
C_{2}=\{(x,r)\in X\times \mathbb{R}:x\in C-x_{0},r=0\}.
\end{gather*}
$C_{1}$ is the interior of the epigraph, epi$\Phi$, which is open and
convex. $C_{2}$ is convex and closed. $C_{1}\cap C_{2}=\emptyset$;
otherwise there exists $x\in C-x_{0}$ such that $\Phi(x)<0$, i.e.\@
$x\in K$. So $x+x_{0}\in C$ and $x+x_{0}\in x_{0}+K$. Using
\eqref{chap7-eq7.13} we obtain $x=0$. But this is impossible since
$(0,0)\not\in C_{1}$. Then by the Hahn Banach theorem there exists
$F\in (X\times \mathbb{R})^{*}$ such that
\begin{equation*}
\Sup_{C_{2}}F\leq \Inf_{C_{1}}F\tag{7.15}\label{chap7-eq7.15}
\end{equation*}

Observe that corresponding to $F$ there are (unique) $g\in X^{*}$ and
$t\in \mathbb{R}$ such that $F(x,r)=g(x)+tr$ for all $(x,r)\in X\times
\mathbb{R}$. Since $(0,0)\in C_{2}\cap \overline{C}_{1}$ it follows
that the Sup and the Inf in \eqref{chap7-eq7.15} are both equal to
$0$. Since for all $x\in X$, $(x,\Phi(x))\in \overline{C}_{1}$ we see
that
\begin{equation*}
g(x)+t\Phi(x)\geq 0\tag{7.16}\label{chap7-eq7.16}
\end{equation*}

This shows that $t$ cannot be equal to $0$, [otherwise $g=0$ and so
  $F=0$]. The point $(0,1)\in C_{1}$. So $0\leq F(0,1)=t$. Thus $t>0$,
and without loss of generality we may assume $t=1$. If follows from
\eqref{chap7-eq7.16} that $g(x)+k||x||-f(x)\geq 0$ for all $x\in
\mathbb{R}$, and this gives the second assertion in
\eqref{chap7-eq7.14}. Now for $x\in C$, it follows that
$(x-x_{0},0)\in C_{2}$ and so $g(x-x_{0})\leq 0$ which gives the first
assertion in \eqref{chap7-eq7.14}. 
\end{proof}

\noindent
{\bf Proof of Theorem \ref{chap7-thm7.2}.}~ Given $0\neq
\hat{f}\in X^{*}$ and $\epsilon>0$, let $f\equiv
\hat{f}/||\hat{f}||_{X^{*}}$ and $K$ as defined in
\eqref{chap7-eq7.9} with $k=\epsilon/||\hat{f}||_{X^{*}}$. Choose
a $z\in C$. Then by Lemma \ref{chap7-lem7.6} there exists $x_{0}\in C$
such that $S\cap (x_{0}+K)=\{x_{0}\}$. Now by Lemma \ref{chap7-lem7.7}
there exists $0\neq g\in X^{*}$ such that $||g-f||_{X^{*}}\leq k$ and
$g$ supports $C$. Denote by
$\hat{g}=||\hat{f}||_{X^{*}g}$. Then
$||\hat{g}-\hat{f}||\leq \epsilon$ and $\hat{g}$ supports
$C$. Since the previous inequality is true for any
$0<\epsilon<||\hat{f}||_{X^{*}}$ the density follows.\hfill$\Box$ 

