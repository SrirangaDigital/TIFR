\thispagestyle{empty}
\begin{center}
{\Large\bf Preface}
\end{center}
\medskip

Since its appearance in 1972 the variational principle of Ekeland has
found many applications in different fields in Analysis. The best
references for those are by Ekeland himself: his survey article
\cite{key23} and his book with J.-P. Aubin \cite{key2}. Not all
material presented here appears in those places. Some are scattered
around and there lies my motivation in writing these notes. Since they
are intended to students I included a lot of related material. Those
are the detours. A chapter on Nemytskii mappings may sound
strange. However I believe it is useful, since their properties so
often used are seldom proved. We always say to the students: go and
look in Krasnoselskii or Vainberg! I think some of the proofs
presented here are more straightforward. There are two chapters on
applications to $PDE$. However I limited myself to semilinear
elliptic. The central chapter is on Br\'ezis proof of the minimax
theorems of Ambrosetti and Rabinowitz. To be self contained I had to
develop some convex analysis, which was later used to give a complete
treatment of the duality mapping so popular in my childhood days! I
wrote these notes as a tourist on vacations. Although the main road is
smooth, the scenery is so beautiful that one cannot resist to go into
the side roads. That is why I discussed some of the geometry of Banach
spaces. Some of the material presented here was part of a course
delivered at the Tata Institute of Fundamental Research in Bangalore,
India during the months of January and February 1987. Some preliminary
drafts were written by Subhasree Gadam, to whom I express may
gratitude. I would like to thank my colleagues at UNICAMP for their
hospitality and Elda Mortari for her patience and cheerful willingness
in texing these notes.

\bigskip

\hfill Campinas, October 1987

