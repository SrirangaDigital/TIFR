\chapter{Ekeland Variational Principle}\label{chap4}

{\bf Introduction.}~ We\pageoriginale saw in Chapter \ref{chap1} that
a functional bounded below assumes its infimum if it has some type of
continuity in a topology that renders (local) compactness to the
domain of said functional. However in many situations of interest in
applications this is not the case. For example, functionals defined in
(infinite dimensional) Hilbert spaces which are continuous in the norm
topology but not in the weak topology. Problems with this set up can
be handled efficiently by Ekeland Variational Principle. This
principle discovered in 1972 has found a multitude of applications in
different fields of Analysis. It has also served to provide simple and
elegant proofs of known results. And as we see it is a tool that
unifies many results where the underlining idea is some sort of
approximation. Our motivation to write these notes is to make an
attempt to exhibit all these features, which we find mathematically
quite interesting.

\begin{theorem}[Ekeland Principle - weak form]\label{chap4-thm4.1}
Let $(X,d)$ be a complete metric space. Let $\Phi:X\to \mathbb{R}\cup
\{+\infty\}$ be lower semicontinuous and bounded below. Then given any
$\epsilon>0$ there exists $u_{\epsilon}\in X$ such that
\begin{equation*}
\Phi(u_{\epsilon})\leq \Inf_{X}\Phi+\epsilon,\tag{4.1}\label{chap4-eq4.1}
\end{equation*}
and
\begin{equation*}
\Phi(u_{\epsilon})<\Phi(u)+\epsilon d(u,u_{\epsilon}),\q \forall u\in
X\q \text{with}\q u\neq u_{\epsilon}.\tag{4.2}\label{chap4-eq4.2}
\end{equation*}
\end{theorem}

For future applications one needs a stronger version of Theorem
\ref{chap4-thm4.1}. Observe that \eqref{chap4-eq4.5} below gives
information on the whereabouts of the point
$u_{\lambda}$.\pageoriginale As we shall see in Theorem
\ref{chap4-thm4.3} the point $u_{\lambda}$ in Theorem
\ref{chap4-thm4.2} [or $u_{\epsilon}$ in Theorem \ref{chap4-thm4.1}]
is a sort of ``almost'' critical point. Hence its importance.

\begin{theorem}[Ekeland Principle - strong form]\label{chap4-thm4.2}
Let $X$ be a complete metric space and $\Phi:X\to \mathbb{R}\cup
\{+\infty\}$ a lower semicontinuous function which is bounded
below. Let $\epsilon>0$ and $\overline{u}\in X$ be given such that
\begin{equation*}
\Phi(\overline{u})\leq
\Inf_{X}\Phi+\frac{\epsilon}{2}.\tag{4.3}\label{chap4-eq4.3} 
\end{equation*}

Then given $\lambda>0$ there exists $u_{\lambda}\in X$ such that
\begin{gather*}
\Phi(u_{\lambda})\leq \Phi(\overline{u})\tag{4.4}\label{chap4-eq4.4}\\
d(u_{\lambda},\overline{u})\leq \lambda\tag{4.5}\label{chap4-eq4.5}\\
\Phi(u_{\lambda}),\Phi(u)+\frac{\epsilon}{\lambda}d(u,u_{\lambda})\q
\forall u\neq u_{\lambda}.\tag{4.6}\label{chap4-eq4.6} 
\end{gather*}
\end{theorem}

\begin{proof}
For notational simplification let us put
$d_{\lambda}(x,y)=(1/\lambda)d(x,y)$. Let us define a partial order in
$X$ by
$$
u\leq v\Longleftrightarrow \Phi(u)\leq \Phi(v)-\epsilon d_{\lambda}(u,v).
$$

It is straightforward that: (i) (reflexivity) $u\leq u$; (ii)
(antisymmetry) $u\leq v$ and $v\leq u$ imply $u=v$; (iii)
(transitivity) $u\leq v$ and $v\leq w$ imply $u\leq w$; all these
three properties for all $u$, $v$, $\omega$ in $X$. Now we define a
sequence $(S_{n})$ of subsets of $X$ as follows. Start with
$u_{1}=\overline{u}$ and define
$$
S_{1}=\{u\in X:u\leq u_{1}\};\q u_{2}\in S_{1}\q s.t.\q
\Phi(u_{2})\leq \Inf_{S_{1}}\Phi+\frac{\epsilon}{2^{2}}
$$ 
and inductively
$$
S_{n}=\{u\in X:u\leq u_{n}\};\q u_{n+1}\in S_{n}\q s.t.\q
\Phi(u_{n+1})\leq \Inf_{S_{n}}\Phi+\frac{\epsilon}{2^{n+1}}. 
$$

Clearly $S_{1}\supset S_{2}\supset S_{3}\supset\cdots$ Each $S_{n}$ is
closed: let $x_{j}\in S_{n}$ with $x_{j}\to x\in X$. We have
$\Phi(x_{j})\leq \Phi(u_{n})-\epsilon
d_{\lambda}(x_{j},u_{n})$. Taking limits using the lower
semicontinuity of $\Phi$ and the continuity of $d$ we conclude that
$x\in S_{n}$. Now we prove that the diameters of these sets go to
zero: diam $S_{n}\to 0$. Indeed, take an arbitrary point $x\in
S_{n}$. On one hand, $x\leq u_{n}$ implies
\begin{equation*}
\Phi(x)\leq \Phi(u_{n})-\epsilon
d_{\lambda}(x,u_{n}).\tag{4.7}\label{chap4-eq4.7} 
\end{equation*}

On\pageoriginale the other hand, we observe that $x$ belongs also to
$S_{n-1}$. So 
it is one of the points which entered in the competition when we
picked $u_{n}$. So 
\begin{equation*}
\Phi(u_{n})\leq
\Phi(x)+\frac{\epsilon}{2^{n}}.\tag{4.8}\label{chap4-eq4.8} 
\end{equation*}

From \eqref{chap4-eq4.7} and \eqref{chap4-eq4.8} we get
$$
d_{\lambda}(x,u_{n})\leq 2^{-n}\q \forall x\in S_{n}
$$
which gives $\diam S_{n}\leq 2^{-n+1}$. Now we claim that the unique
point in the intersection of the $S_{n}$'s satisfies conditions
\eqref{chap4-eq4.4} -- \eqref{chap4-eq4.5} -- \eqref{chap4-eq4.6}. Let
then $\bigcap\limits^{\infty}_{n=1}S_{n}=\{u_{\lambda}\}$. Since $u_{\lambda}\in
S_{1}$, \eqref{chap4-eq4.4} is clear. Now let $u\neq u_{\lambda}$. We
cannot have $u\leq u_{\lambda}$, because otherwise $u$ would belong to
the intersection of the $S_{n}$'s. So $u\nleq u_{\lambda}$, which
means that 
$$
\Phi(u)>\Phi(u_{\lambda})-\epsilon d_{\lambda}(u,u_{\lambda})
$$
thus proving \eqref{chap4-eq4.6}. Finally to prove \eqref{chap4-eq4.5}
we write
$$
d_{\lambda}(\overline{u},u_{n})\leq
\sum\limits^{n-1}_{j=1}d_{\lambda}(u_{j},u_{j+1})\leq
\sum\limits^{n-1}_{j=1}2^{-j} 
$$
and take limits as $n\to \infty$. 
\end{proof}

\begin{remark*}
The above results and further theorems in this chapter are due to
Ekeland. See \cite{key37}, \cite{key38}, and his survey paper
\cite{key39}.
\end{remark*}

\noindent
{\bf Connections With Fixed Point Theory.}~ Now we show that Ekeland's
Principle implies a Fixed Point Theorem due to Caristi
\cite{key22}. See also \cite{key23}. As a matter of fact, the two
results are equivalent in the sense that Ekeland's Principle can also
be proved from Caristi's theorem.

\begin{theorem}[Caristi Fixed Point Theorem]\label{chap4-thm4.3}
Let $X$ be a complete metric space, and $\Phi:X\to \mathbb{R}\cup
\{+\infty\}$ a lower semicontinuous functional which is bounded
below. Let $T:X\to 2^{X}$ be a multivalued mapping such that
\begin{equation*}
\Phi(y)\leq \Phi(x)-d(x,y),\q \forall x\in X,\q \forall y\in
Tx.\tag{4.9}\label{chap4-eq4.9} 
\end{equation*}

Then\pageoriginale there exists $x_{0}\in X$ such that $x_{0}\in Tx_{0}$.
\end{theorem}

\begin{proof}
Using Theorem \ref{chap4-thm4.1} with $\epsilon=1$ we find $x_{0}\in
X$ such that
\begin{equation*}
\Phi(x_{0})<\Phi(x)+d(x,x_{0})\q \forall x\neq
x_{0}.\tag{4.10}\label{chap4-eq4.10} 
\end{equation*}

Now we claim that $x_{0}\in Tx_{0}$. Otherwise all $y\in Tx_{0}$ are
such that $y\neq x_{0}$. So we have from \eqref{chap4-eq4.9} and
\eqref{chap4-eq4.10} that
$$
\Phi(y)\leq \Phi(x_{0})-d(x_{0},y)\q\text{and}\q \Phi(x_{0})<\Phi(y)+d(x_{0},y)
$$
which cannot hold simultaneously.
\end{proof}

\noindent
{\bf Proof of Theorem \ref{chap4-thm4.1} from Theorem
  \ref{chap4-thm4.3}.}~ Let us use the notation $d_{1}=\epsilon d$,
which is an equivalent distance in $X$. Suppose by contradiction that
there is no $u_{\epsilon}$ satisfying \eqref{chap4-eq4.2}. So for each
$x\in X$ the set $\{y\in X:\Phi(x)\geq \Phi(y)+d_{1}(x,y);y\neq x\}$
is not empty. Let us denote this set by $T_{x}$. In this way we have
produced a multivalued mapping $T$ in $(X,d_{1})$ which satisfies
condition \eqref{chap4-eq4.9}. By Theorem \ref{chap4-thm4.3} it should
exist $x_{0}\in X$ such that $x_{0}\in Tx_{0}$. But this is
impossible: from the very definition of $Tx$, we have that $x\not\in
Tx$.\hfill$\Box$

\begin{remark*}
If $T$ is a contraction in a complete metric space, that is, if there
exists a constant $k$, $0\leq k<1$, such that
$$
d(Tx,Ty)\leq kd(x,y),\q \forall x, y\in X,
$$
then $T$ satisfies condition \eqref{chap4-eq4.9} with
$\Phi(x)=\frac{1}{1-k}d(x,Tx)$. So that part of the Contraction
Mapping Principle which says about the existence of a fixed point can
be obtained from Theorem \ref{chap4-thm4.3}. Of course the Contraction
Mapping Principle is much more than this. Its well known proof uses an
iteration procedure (the method of successive approximations) which
gives an effective computation of the fixed point, with an error
estimate, etc\ldots
\end{remark*}

\medskip
\noindent
{\bf Application of Theorem \ref{chap4-thm4.1} to Functionals Defined
  in Banach Spa\-ces.}~ Now we put more structure on the space $X$ where
the functionals are defined. In fact we suppose that $X$ is a Banach
space. This will allow us to use a Differential Calculus, and then we
could appreciate better the meaning of the relation
\eqref{chap4-eq4.2}. Loosely speaking \eqref{chap4-eq4.2} has to do
with some Newton quocient being small.

\begin{theorem}\label{chap4-thm4.4}
Let $X$ be a Banach space and $\Phi:X\to \mathbb{R}$ a lower
semicontinuous functional which is bounded below. In addition, suppose
that $\Phi$ is Gteaux\pageoriginale differentiable at every point
$x\in X$. Then for each $\epsilon>0$ there exists $u_{\epsilon}\in X$
such that
\begin{align*}
&\Phi(u_{\epsilon})\leq
  \Inf_{X}\Phi+\epsilon\tag{4.11}\label{chap4-eq4.11}\\
&||D\Phi(u_{\epsilon})||_{X}\cdot \leq
  \epsilon.\tag{4.12}\label{chap4-eq4.12} 
\end{align*}
\end{theorem}

\begin{proof}
It follows from Theorem \ref{chap4-thm4.1} that there exists
$u_{\epsilon}\in X$ such that \eqref{chap4-eq4.11} holds and
\begin{equation*}
\Phi(u_{\epsilon})\leq \Phi(u)+\epsilon||u-u_{\epsilon}||\q \forall
u\in X.\tag{4.13}\label{chap4-eq4.13} 
\end{equation*}

Let $v\in X$ and $t>0$ be arbitrary. Putting $u=u_{\epsilon}+tv$ in
\eqref{chap4-eq4.13} we obtain
$$
t^{-1}[\Phi(u_{\epsilon})-\Phi(u_{\epsilon}+tv)]\leq \epsilon||v||.
$$

Passing to the limit as $t\to 0$ we get $-\langle D\Phi(u),v\rangle
\leq \epsilon||v||$ for each given $v\in X$. Since this inequality is
true for $v$ and $-v$ we obtain $|\langle D\Phi(u),v\rangle|\leq
\epsilon||v||$, for all $v\in X$. But then
$$
||D\Phi(u)||_{X}\cdot =\sup\limits_{v\in V\ v\neq 0}\frac{\langle
  D\Phi(u),v\rangle}{||v||}\leq \epsilon.
$$
\end{proof}

\begin{remark}\label{chap4-rem1}
The fact that $\Phi$ is Gteaux differentiable does not imply that
$\Phi$ is lower semicontinuous. One has simple examples, even for
$X=\mathbb{R}^{2}$. 
\end{remark}

\begin{remark}\label{chap4-rem2}
In terms of functional equations Theorem \ref{chap4-thm4.4} means the
following. Suppose that $T:X\to X^{*}$ is an operator which is a {\em
  gradient}, i.e., there exists a functional $\Phi:X\to \mathbb{R}$
such that $T=D\Phi$. The functional $\Phi$ is called the {\em
  potential} of $T$. If $\Phi$ satisfies the conditions of Theorem
\ref{chap4-thm4.4}, then that theorem says that the equation
$Tx=x^{*}$ has a solution $x$ for {\em some} $x^{*}$ in a ball of
radius $\epsilon$ around $0$ in $X^{*}$. And this for all
$\epsilon>0$. As a matter of fact one could say more if additional
conditions are set on $\Phi$. Namely
\end{remark}

\begin{theorem}\label{chap4-thm4.5}
In addition to the hypotheses of Theorem \ref{chap4-thm4.4} assume
that there are constants $k>0$ and $C$ such that
$$
\Phi(u)\geq k||u||-C.
$$

Let $B^{*}$ denote the unit ball about the origin in $X^{*}$. Then
$D\Phi(X)$ is dense in $kB^{*}$. 
\end{theorem}

\begin{proof}
We shoule prove that given $\epsilon>0$ and $u^{*}\in kB^{*}$ there
exists $u_{\epsilon}\in X$ such that
$||D\Phi(u_{\epsilon})-u^{*}||_{X^{*}}\leq \epsilon$. So consider the
functional $\Psi(u)=\Phi(u)-\langle u^{*},u\rangle$.\pageoriginale 
It is easy to see that $\Psi$ is lower semicontinuous and Gteaux
differentiable. Boundedness below follows from
\eqref{chap4-eq4.14}. So by Theorem \ref{chap4-thm4.4} we obtain
$u_{\epsilon}$ such that $||D\Psi(u_{\epsilon})||_{X^{*}}\leq
\epsilon$. Since $D\Psi(u)=D\Psi(u)-u^{*}$, the result follows.
\end{proof}

\begin{corollary}\label{chap4-coro4.6}
In addition to the hypotheses of Theorem \ref{chap4-thm4.4} assume
that there exists a continuous function $\varphi:[0,\infty)\to
  \mathbb{R}$ such that $\varphi(t)/t\to +\infty$ as $t\to +\infty$
  and $\Phi(u)\geq \varphi(||u||)$ for all $u\in X$. Then $D\Phi(X)$
  is dense in $X^{*}$. 
\end{corollary}

\begin{proof}
Let $k>0$. Choose $t_{0}>0$ such that $\varphi(t)/t\geq k$ for
$t>t_{0}$. So $\Phi(u)\geq k||u||$ if $||u||>t_{0}$. If $||u||\leq
t_{0}$, $\Phi(u)\geq C$ where $C=\min \{\varphi(t):0\leq t\leq
t_{0}\}$. Applying Theorem \ref{chap4-thm4.5} we see that $D\Phi(X)$
is dense in $kB^{*}$. Since $k$ is arbitrary the result follows.
\end{proof}

For the next result one needs a very useful concept, a sort of
compactness condition for a functional $\Phi$. We say that a $C^{1}$
functional satisfies the {\em Palais - Smale condition} [or $(PS)$
  condition, for short] if every sequence $(u_{n})$ in $X$ which
satisfies 
$$
|\Phi(u_{n})|\leq \text{~const.~ and~ } \Phi'(u_{n})\to 0\q\text{in}\q
X^{*} 
$$
possesses a convergent (in the norm) subsequence.

\begin{theorem}\label{chap4-thm4.7}
Let $X$ be a Banach space and $\Phi:X\to \mathbb{R}$ a $C^{1}$
functional which satisfies the $(PS)$ condition. Suppose in addition
that $\Phi$ is bounded below. Then the infimum of $\Phi$ is achieved
at a point $u_{0}\in X$ and $u_{0}$ is a critical point of $\Phi$,
i.e., $\Phi'(u_{0})=0$.
\end{theorem}

\begin{proof}
Using Theorem \ref{chap4-thm4.4} we see that for each positive integer
$n$ there is $u_{n}\in X$ such that
\begin{equation*}
\Phi(u_{n})\leq \Inf_{X}\Phi+\frac{1}{n}\q ||\Phi'(u_{n})||\leq
\frac{1}{n}.\tag{4.15}\label{chap4-eq4.15}
\end{equation*}

Using (PS) we have a subsequence $(u_{n_{j}})$ and an element
$u_{0}\in X$ such that $u_{n_{j}}\to u_{0}$. Finally from the
continuity of both $\Phi$ and $\Phi'$ we get \eqref{chap4-eq4.15}. 
\begin{equation*}
\Phi(u_{0})=\Inf_{X}\Phi\q \Phi'(u_{0})=0\tag{4.16}\label{chap4-eq4.16}
\end{equation*}
\end{proof}


\begin{remarks*}
\begin{enumerate}
\renewcommand{\labelenumi}{(\theenumi)}
\item As a matter of fact the result is true without the continuity of
  $\Phi'$. The mere existence of the Fr\'echet differential at each
  point suffices. Indeed, we have only to show that the first
  statement in \eqref{chap4-eq4.16} implies the second.\pageoriginale
  This is a standard fact in the Calculus of Variations. Here it goes
  its simple proof: take $v\in X$, $||v||=1$, arbitrary and $t>0$. So
$$
\Phi(u_{0})\leq \Phi(u_{0}+tv)=\Phi(u_{0})+t\langle
\Phi'(u_{0}),v\rangle +o(t)
$$
from which follows that
$||\Phi'(u_{0})||_{X^{*}}=\sup\limits_{||v||=1}\langle
\Phi'(u_{0}),v\rangle\leq \frac{o(t)}{t}$ for all $t>0$. Making $t\to
0$ we get the result. 

\item The boundedness below of $\Phi$ it could be obtained by a
  condition like the one in Corollary \ref{chap4-coro4.6}. Observe
  that a condition like $\Phi(u)\to +\infty$ ad $||u||\to \infty$
  (usually called coerciveness) [or even the stronger one
    $\Phi(u)/||u||\to +\infty$ as $||u||\to +\infty$] does not
  guarantee that $\Phi$ is bounded below. See Chapter \ref{chap1}.

\item Theorem \ref{chap4-thm4.7} appears in Chang \cite{key24} with a
  different proof and restricted to Hilbert spaces. Possibly that
  proof could be extended to the case of general Banach using a the
  flow given by subgradient, like in \cite{key68}, instead of the
  gradient flow.
\end{enumerate}
\end{remarks*}
