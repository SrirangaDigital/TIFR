\thispagestyle{empty}
\begin{center}
{\Large\bf Lectures on}\\[5pt]
{\Large\bf Numerical Methods For Non-Linear}\\[5pt] 
{\Large\bf Variational Problems}
\vskip 1cm


{\bf By}        
\medskip

{\large\bf R. Glowinski}
\vfill

{\bf Tata Institute of Fundamental Research}

{\bf Bombay}

{\bf 1980}
\end{center}

\eject

\thispagestyle{empty}
\begin{center}
{\Large\bf Lectures on}\\[5pt]
{\Large\bf Numerical Methods For Non-Linear}\\[5pt]
{\Large\bf  Variational Problems}
\vskip 1cm


{\bf By}
\medskip

{\large\bf R. Glowinski}
\vfill

{\bf Notes by}
\medskip

{\large\bf G. Vijayasundaram}\\[4pt]
{\large\bf Adimurthi}
\vfill

Published for the 

{\bf Tata Institute of Fundamental Research, Bombay}

{\bf Springer-Verlag}

Berlin Heidelberg New York

{\bf 1980}
\end{center}

\eject

\thispagestyle{empty}
\begin{center}
{\bf Author}
\medskip

{\large\bf R. Glowinski}

Universit\'e Pierre et Marie Curie 

Laboratoire d'Analyse Num\'erique 189

Tour 55--65 5\'em \'etage 

4, Place Jussieu

75230 PARIS CEDEX 05 

FRANCE
\vfill 

{\bf\copyright\ Tata Institute of Fundamental Research, 1980}
\vfill

\hrule 
\medskip

ISBN 3-540-08774-5 Springer-Verlag, Berlin Heidelberg, New York

ISBN 0-387-08774-5 Springer-Verlag,  New York. Heidelberg,. Berlin 
\medskip

\hrule
\vfill

\parbox{0.7\textwidth}{No part of this book may be reproduced in any form 
by print, microfilm or any other means without written permission from 
the Tata Institute of Fundamental Research, Colaba, Bombay 400 005}
\vfill

Printed by N.S. Ray at The Book Centre Limited Sion East, Bombay 400
022 and Published by H. Goetze Springer-Verlag, Heidelberg, West
Germany

Printed in India
\end{center}
\eject

~
\thispagestyle{empty}
\vfill
\begin{center}

\raggedleft \textsl{To the memory of {\large\bfseries G. Stampacchia}}

\end{center}
\vfill



\chapter{Preface}
%\markboth{Preface}{}

%\addcontentsline{toc}{chapter}{Preface}

These notes correspond to a course of about fifteen lectures given
at the Tata Institute of Fundamental Research Centre, Indian Institute
of Science, Bangalore in January and February 1977.

The main goal of this course and of the corresponding notes is to
provide an introduction to the study of Nonlinear Variational
Problems; they do not have pretention to cover all the aspects of this
very important subject, since for example the Navier--Stokes equations
for newtonian incompressible viscous flows have not been considered
here (we refer for this last problem to, e.g., TEMAM [1] and
GIRAULT-RAVIART [1]).

Some questions pertinent to the main subject of these notes have not
been treated here since they have been considered in the
T.I.F.R. Lecture Notes of P.G. CIARLET [1] and J.CEA [2].

Chapters~\ref{chap1} and \ref{chap2} are concerned with
\textit{Elliptic Variational Inequalitites} (E.V.I.) more precisely
with their approximation (mostly by finite element methods) and also
their iterative solution. Several examples, coming from Mechanics
illustrate the methods which are described in these two chapters.

The following Chapter~\ref{chap3} is only an introduction to the
approximation of Parabolic Variational Inequalities (P.V.I.); we have
however studied with some details a particular P.V.I. related to the
unsteady flow of some viscous plastic media (Bingham fluids) in a
cylindrical pipe.

In Chapter~\ref{chap4} we show how Variational Inequalities concepts
and methods may be useful to study some Nonlinear Variational
equations.

In Chapter~\ref{chap5} we discuss the iterative solution of some
Variational Problems with a very specific structure allowing their
solution by decomposition - coordination methods via augmented
lagrangians; several iterative methods are described and illustrated
by examples, mostly from Mechanics.

In Chapter~\ref{chap6}, which unlike the previous chapters is largely
heuristical, we show how some of the tools of the Chapters I--IV may
be used to solve numerically a difficult and important nonlinear
problem of Fluid Dynamics: namely the steady transonic potential flow
of an inviscid compressible fluid. This last chapter is obviously just
an introduction to this very important and difficult subject.

I would like to thank all the people who make my stay in India a most
enjoyable experience and more particularly Professors K.G. RAMANATHAN,
K. BALAGANGADHARAN and M.K.V. MURTHY.

These Notes were taken by M. ADIMURTHI and M.G. VIJAYASUNDARAM; I
would like to thank them for their devoted efforts.

I would like to thank also S. KESAVAN and L. REINHART for their
careful reading of the proofs and the various improvements they have
suggested. Eventually I would like to express all my acknowledgements
to Mrs. F. WEBER for her beautiful typing of these Notes and to
Mr. M. Bazot who did all the artwork.

\begin{flushright}
 {\large\bf R. Glowinski}\\
 Rocquencourt, France\\
 November, 1979
\end{flushright}

