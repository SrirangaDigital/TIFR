
\chapter[Generalities On Elliptic Variational...]{Generalities On Elliptic Variational 
Inequalities And On Their Approximation}\label{chap1}

\section{Introduction}\label{c1:s1}

 An\pageoriginale important and very useful class of non-linear problems arising
 from mechanics, physics etc. consists of the so-called Variational
 Inequalities. We mainly consider the following two types of
 variational inequalities, namely 
\begin{enumerate}
\item Elliptic Variational Inequalities (EVI),
\item Parabolic Variational Inequalities (PVI).
\end{enumerate}

In this chapter (following LIONS-STAMPACCHIA \ref{k69:e1}) we shall restrict
our attention to the study of the existence, uniqueness and
approximation of the solutions of EVI. 

\section{Functional Context}\label{c1:s2}

 In this section we consider two classes of EVI, namely EVI of the
 first kind and EVI of the second kind. 

\subsection{Notations}\label{c1:ss2.1}
\begin{itemize}
\item $V$ : real Hilbert space with scalar product $(\cdot , \cdot)$ and
  associated norm $\parallel \cdot \parallel$. 
\item $V^*$ : the dual space of $V$.
\item $a(\cdot , \cdot) : V \times V \to \mathbb{R}$ is a bilinear, continuous
  and \textit{$V$ -elliptic} form on $V \times V$.  
\end{itemize}

A bilinear form $a (\cdot , \cdot)$ is said to be \textit{$V$ -elliptic} if
there exists a positive constant $\alpha$ such that $a(v, v) \geq
\alpha \parallel v \parallel^2 \forall v \in V$.  

In general we do not assume $a(\cdot , \cdot)$ to be symmetric, since in some
applications non-symmetric bilinear forms may occur naturally (see for
instance COMINCIOLI [\ref{k37:e1}]).  
\begin{itemize}
\item $L : V \to \mathbb{R}$ continuous, linear functional. 
\item $K$ is a closed, convex, non-empty subset of $V$.
\item $j(\cdot) : V \to \bar{\mathbb{R}} = \mathbb{R} \cup \{\infty\}$
  is a convex, lower semi -continuous (l.s.c. ) and proper functional  
\end{itemize}
$(j(\cdot)$ is proper if $j(v) > -\infty\,  \forall v \in V$ and $j \not \equiv \infty)$.

\subsection{EVI of first kind}\label{c1:ss2.2}

\textit{To find\pageoriginale $u \in V$ such that $u$ is a solution of the
  problem} 
\begin{equation*}
(P_1)\begin{cases}
a(u, v - u) \geq L(v - u), \forall v \in K,\\
u \in K.
\end{cases}
\end{equation*}

\subsection{EVI of second kind}\label{c1:ss2.3}

\textit{To find $u \in V$ such that $u$ is a solution of the problem}
\begin{equation*}
(P_2)
\begin{cases}
  a(u, v - u) + j(v) - j(u) \geq L(v - u)  \forall v \in V,\\
  u \in V.
\end{cases}
\end{equation*}

\subsection{Remarks:}\label{c1:ss2.4}
 
\begin{remark}\label{c1:rem2.1}
The cases above considered are the simplest and most important. LIONS
and BENSOUSSAN [1] considered some generalization of problem $(P_1)$
called {\em Quasi Variational Inequalities} (QVI) which arises for
instance from Decision Sciences. A typical problem of QVI is : 

{\em To find $u \in V$ such that}
\begin{equation*}
\begin{cases}
a(u, v - u) \geq L(v - u)\, \forall v \in K(u),\\
u \in K(u)
\end{cases}
\end{equation*}
where $v \to K(v)$ is a family of closed, convex non-empty subsets of $V$.
\end{remark}

\begin{remark}\label{c1:rem2.2}
If $K = V$ and $j \equiv 0$ then the problems $(P_1)$ and $(P_2)$
reduce to the classical variational equation 
\begin{equation*}
\begin{cases}
a(u, v) = L(v)~\forall v \in V,\\
&\\
u \in V.
\end{cases}
\end{equation*}
\end{remark}

\begin{remark}\label{c1:rem2.3}
The distinction\pageoriginale between $(P_1)$ and $(P_2)$ is artificial, for $(P_1)$
can be considered as a particular case of $(P_2)$ by replacing
$j(\cdot)$ in $(P_2)$ by the indicator function $I_K$ of $K$ defined
by  
\begin{equation*}
I_K(v) =
\begin{cases}
0 \text{ if } v \in K\\
+\infty \text{ if } v \not \in K.
\end{cases}
\end{equation*}
\end{remark}

Even though $(P_1)$ is a particular case of $(P_2)$ it is worthwhile
considering $(P_1)$ separately because it arises in a natural way and
we will get geometrical insight into the problem. 

\begin{exercise}\label{c1:exer2.1}%Exercise 2.1 :
Prove that $I_K$ is a convex, l.s.c. and proper functional. 
\end{exercise}

\begin{exercise}\label{c1:exer2.2}%Exercise 2.2 :
Show that $(P_1)$ is equivalent to the problem of finding $u \in
V$ such that $a(u, v - u) + I_K(v) - I_K(u) \geq L(v - u)\, \forall
v \in V$. 
\end{exercise}

\section[Existence And Uniqueness Results 
For EVI...]{Existence And Uniqueness Results 
For EVI of\hfil\break First Kind}\label{c1:s3}%Sec 3

\subsection{A Theorem of existence and
  uniqueness}\label{c1:ss3.1}%Subsec 3.1 

\begin{theorem}\label{c1:thm3.1}%Thm 3.1.
(LIONS-STAMPACCHIA \ref{k69:e1}). {\em The problem $(P_1)$ has one and
 only one solution} 
\end{theorem}

\begin{proof}
(I)Uniqueness:

Let $u_1$ and $u_2$ be solutions of $(P_1)$. We have then 
\begin{gather}
a(u_1, v - u_1) \geq L(v - u_1) \ \forall v \in K, u_1
\in K, \tag{3.1}\label{c1:eq3.1}\\
a(u_2,v-u_2) \geq L(v-u_2) \ \forall v \in K, u_2 \in
K. \tag{3.2}\label{c1:eq3.2}
\end{gather}
\end{proof}

Putting $u_2$ for $v$ in \eqref{c1:eq3.1} and $u_1$ for $v$ in \eqref{c1:eq3.2}
and adding we get, by using the $V$ -ellipticity of $a(\cdot , \cdot)$, 
$$
\alpha \parallel u_2 - u_1 \parallel^2 \leq a(u_2 - u_1, u_2 - u_1)\leq 0
$$
which proves $u_1 = u_2$ since $\alpha > 0$.

\textbf{(2) Existence}

We\pageoriginale use a generalization of the proof used by CLARLET [1] for proving
the Lax-Milgram Lemma, i. e. we will reduce the problem $(P_1)$ to a
{\em fixed point} problem. 

By the Riesz representation theorem for Hilbert space there exist $A
\in \mathscr{L}(V, V) (A = A^t$ if $a(\cdot , \cdot)$ is symmetric) and
$\ell \in V$ such that  
\begin{equation}
(Au, v) = a(u, v) ~~\forall u, v \in V \text{ and }L(v) = (\ell,
  v) ~~\forall v \in V. \tag{3.3}\label{C1:eq3.3} 
\end{equation}

Then the problem $(P_1)$ is equivalent to finding $u \in V$ such that 
\begin{equation}
\begin{cases}
(u - \rho (Au - \ell) - u, v - u) \leq 0 ~~ \forall v \in K, \\
&\\
u \in K, \rho > 0.
\end{cases}\tag{3.4}\label{c1:eq3.4}
\end{equation}

This is equivalent to finding $u$ such that 
\begin{equation}
u = P_K(u - \rho (Au - \ell )), \text{ for some } \rho > 0,
\tag{3.5}\label{c1:eq3.5}
\end{equation}
where $P_K$ denotes the projection operator from $V$ to $K$ in the
$\parallel \cdot \parallel$ norm. Consider the map $W_\rho : V \to V$
defined by  
\begin{equation}
W_\rho (v) = P_K(v - \rho(Av - \ell)). \tag{3.6}\label{c1:eq3.6}
\end{equation}

Let $v_1$, $v_2 \in V$. Then since $P_K$ is a contraction we have 
\begin{equation*}
\begin{cases}
\parallel W_\rho (v_1) -W_\rho (v_2) \parallel ^2 \leq 
\parallel v_2 -v_1 \parallel ^2 + \rho^2 \parallel A(v_2- v_1 \parallel ^2 \\
&\\
-2\rho a(v_2-v_1, v_2-v_1).
\end{cases}
\end{equation*}
Hence we have 
\begin{equation}
\parallel W_\rho (v_1) - W_\rho (v_2) \parallel ^2 \leq (1 - 2 \rho 
\alpha +\rho^2 \parallel A \parallel ^2 ) \parallel v_2 - v_1
\parallel^2. \tag{3.7}\label{c1:eq3.7}
\end{equation}
Thus $W_\rho$ is a strict contraction mapping if $0 < \rho < \dfrac{2
  \alpha}{\parallel A \parallel^2}$. By taking $\rho$ in this range we
have a unique solution for the fixed point problem which implies the
existence of a solution for $(P_1)$.  

\subsection{Remarks}\label{c1:ss3.2}%Sec 3.2

\begin{remark}\label{c1:rem3.1}%Remk 3.1
If\pageoriginale $K = V$, Theorem~\ref{c1:thm3.1} reduces to Lax-Milgram Lemma 
(see CIARLET [\ref{k32:e1}]). 
\end{remark}

\begin{remark}\label{c1:rem3.2}%Remk 3.2
If $a(\cdot , \cdot)$ is symmetric then Theorem~\ref{c1:thm3.1} can be proved using
optimization methods (see CEA [1]). 
\end{remark}
Let $J : V \to \mathbb{R}$ be defined by 
\begin{equation}
J(v) = \frac{1}{2} a(v, v) - L(v). \tag{3.8}\label{c1:eq3.8}
\end{equation}
Then 
\begin{enumerate}[(i)]
\item $\lim \limits_{\parallel v \parallel \to +\infty} J(v) =
  +\infty$


since $J(v) = \dfrac{1}{2}a(v, v) - L(v) \geq \dfrac{\alpha}{2}
\parallel v \parallel^2 - \parallel L \parallel \parallel v
\parallel$. 

\item $J$ \textit{is strictly convex.}

Since $L$ is linear, to prove the strict convexity of $J$ it suffices
to prove that  
$$
v \to a(v, v)
$$
is strictly convex. Let $0 < t <1$ and $u$, $v \in V$ with $u
\neq v$; $0 < a(v - u, v - u) = a(u, u)+ a(v, v)-2a(u, v)$. Hence we
have  
\begin{equation}
2a (u, v) <a(u, u) +a(v, v). \tag{3.9}\label{c1:eq3.9}
\end{equation}
Using \eqref{c1:eq3.9} we have 
\begin{equation}
\begin{cases}
  a(tu + (1 - t)v, tu + (1 - t)v) =\\ 
  \hspace{2cm}t^2 a(u, u) +2t(1 - t) a (u, v) + (1
- t)^2 a(v, v) <\\ 
\\
< ta(u, u) + (1 - t) a (v, v).
\end{cases}\tag{3.10}\label{c1:eq3.10}
\end{equation}
Therefore $a(v, v)$ is strictly convex.

\item Since\pageoriginale $a(\cdot , \cdot)$ and $L$ are continuous, $J$ is continuous.
\end{enumerate}

From these properties of $J$ and standard results of Optimization
Theory (cf. CEA [\ref{k27:e1}]) it follows that the minimization problem of finding
$u$ such that  
\begin{equation*}
(\pi) 
\begin{cases}
J(u) \leq J(v) ~~\forall v \in K, \\
&\\
u \in K
\end{cases}
\end{equation*}
has one and only one solution. Therefore $(\pi)$ is equivalent to the
problem of finding $u$ such that  
\begin{equation}
\begin{cases}
(J'(u), v - u) \geq 0 ~~\forall v \in K,\\
&\tag{3.11}\label{c1:eq3.11}\\
u \in K,
\end{cases}
\end{equation}
where $J'(u)$ is the \textit{Gateaux derivative} of $J$ at $u$. Since
$(J'(u),v) = a(u, v) - L(v)$ we see that $(P_1)$ and $(\pi)$ are
equivalent if $a(\cdot , \cdot)$ is symmetric. 

\begin{exercise}\label{c1:exer3.1}%Exercise 3.1
Prove that $(J'(u), v) =a (u, v) - L(v) ~~\forall u$, $v \in V$
and hence deduce that $J'(u) = Au \sim \ell ~~\forall u \in V$. 
\end{exercise}

\begin{remark}\label{c1:rem3.3}%Remk 3.3
The proof of Theorem~\ref{c1:thm3.1} given a natural a natural algorithm for
solving $(P_1)$ since $v \to P_K(v - \rho (Av - \ell))$ is a
{\em contraction} mapping for $0 < \rho < \dfrac{2 \alpha}{\parallel A
  \parallel^2}$. 
Hence we can use the following algorithm to find $u$:
\begin{equation}
\text{ Let }u^0 \in V, \tag{3.12}\label{c1:eq3.12}
\end{equation}
\begin{equation}
u^{n+1} = P_K(u^n - \rho (Au^n - \ell)). \tag{3.13}\label{c1:eq3.13}
\end{equation}
\end{remark}

Then $u^n \to u$ strongly in $V$ where $u$ is the solution of
$(P_1)$. In practice it is not easy to calculate $\ell$ and $A$ unless
$V = V^*$. To project over $K$ may be as difficult as solving
$(P_1)$. In general this method cannot be used for computing the
solution of $(P_1)$ if $K \neq V$ (at least not so directly). 

We\pageoriginale observe that if $a(\cdot , \cdot)$ is symmetric then
$J'(u) = Au - \ell$ 
and hence \eqref{c1:eq3.13} becomes  
\begin{equation}
u^{n+1} = P_K(u^n - \rho(J'(u^n)). \tag{3.13'}\label{c1:eq3.13'}
\end{equation}

This method is know as the \textit{Gradient -Projection} method.

\section[Existence And Uniqueness Results...]{Existence And Uniqueness Results 
For EVI of\hfil\break Second Kind}\label{c1:s4}%Sec 1.4 

\begin{theorem}\label{c1:thm4.1}%4.1
(LIONS-STAMPACCHIA [\ref{k69:e1}]) Problem $(P_2)$ has one only one
  solution. 
\end{theorem}

\begin{proof}
As in Theorem~\ref{c1:thm3.1} we shall first prove uniqueness and then
existence  

\textbf{(1) Uniqueness.} Let $u_1$ and $u_2$ be two solutions of $(P_2)$. Then we have 
\begin{align}
a(u_1, v - u_1) + j(v) - j(u_1) \geq L(v - u_1) ~\forall v \in
V, u_1 \in V, \tag{4.1}\label{c1:eq4.1}\\ 
a(u_2, v - u_2) + j(v) - j(u_2) \geq L(v - u_2) ~\forall v \in
V, u_2 \in V, \tag{4.2}\label{c1:eq4.2} 
\end{align}
\end{proof}

Since $j( \cdot)$ is a proper map there exists $v_0 \in V$ such
that $-\infty < j(v_0) < \infty$. Hence for $i = 1$, 2 
\begin{equation}
-\infty < j(u_i) \leq j(v_0) - L(v_0 - u_i) + a (u_i, v_0 -
u_i). \tag{4.3}\label{c1:eq4.3} 
\end{equation}

This shows that $j(u_i)$ is finite for $i=1$, $2$. Hence by
substituting $u_2$ for $v$ in \eqref{c1:eq4.1} and $u_1$ for $v$ in \eqref{c1:eq4.2} and
adding we obtain  
\begin{equation}
\alpha \parallel u_1 - u_2 \parallel^2  \leq a (_1 - u_2, u_1 - u_2)
\leq 0. \tag{4.4}\label{c1:eq4.4} 
\end{equation}
Hence $u_1 = u_2$. 

\textbf{(2) Existence.} For each $u \in V$ and $\rho > 0$ we
associate a problem $(\pi^u_\rho)$ of type $(P_2)$ defined as below : 

\textit{To find $w \in V$ such that}
\begin{equation}
(\pi^u_\rho)
\begin{cases}
(w,v-w) + \rho j(v) - \rho j(w) \\
 \hspace{2cm}\geq (u, v-w) +\rho L(v -w) - \rho
  a(u, v-w) ~~ \forall v \in V, \\ 
&\\
w \in V.
\end{cases}\tag{4.5}\label{c1:eq4.5}
\end{equation}

The advantage\pageoriginale of considering this problem over the problem $(P_2)$ is
that the bilinear form associated with $(\pi^u_\rho)$ is the inner
product of $V$ which is symmetric.  

Let us first assume that $(\pi^u_\rho)$ has a unique solution for all
$u \in V$ and $\rho > 0$. For each $\rho$ define the map $f_\rho
: V \to V$ by $f_\rho(u) = w$ where $w$ is the unique solution of
$(\pi^u_\rho)$. 

We shall show that $f_\rho$ is a uniformly strict contraction mapping
for suitable chosen $\rho$.  

Let $u_1$, $u_2 \in V$ and $w_i = f_\rho(u_i)$, $i=1$, $2$. Since
$j(\cdot)$ is proper we have $j(u_i)$ finite which can be proved as in
\eqref{c1:eq4.3}. Therefore we have  
\begin{align*}
(w_1,w_2 - w_1) & + \rho j(w_2)-\rho j(w_1)\\ 
 &\geq (u_1, w_2 - w_1)+\rho
  L(w_2-w_1 ) - \rho a(u_1, w_2 - w_1), \tag{4.6}\label{c1:eq4.6}\\ 
(w_2, w_1 - w_2) & + \rho j(w_1) - \rho j(w_2)\\ 
  & \geq (u_2, w_1-w_2)+\rho
  L(w_1 - w_2 ) -\rho a(u_2,w_1-w_2). \tag{4.7}\label{c1:eq4.7} 
\end{align*}
Adding these inequalities we obtain 
\begin{equation*}
\left\{\begin{aligned}
\parallel f_\rho (u_1) -f_\rho (u_2) \parallel^2 & = \parallel w_2-w_1
\parallel^2 \\ 
&\\
& = \leq ((I -\rho A) (u_2-u_1), w_2 - w_1)\\ 
&\\
& = \leq \parallel I- \rho A \parallel ~\parallel u_2-u_1 \parallel
~\parallel w_2-w_1 \parallel. 
\end{aligned}\right.\tag{4.8}\label{c1:eq4.8}
\end{equation*}
Hence 
$$
\parallel f_\rho (u_1) -f_\rho (u_2) \parallel \leq \parallel I- \rho
A\parallel ~\parallel u_2-u_1\parallel 
$$

It is easy to show that $\parallel I - \rho A \parallel < 1$ when $0 <
\rho < \dfrac{2 \alpha}{\parallel A \parallel^2}$. This proves that
$f_\rho$ is uniformly a strict contracting mapping and hence has a
unique fixed point $u$. This $u$ turns out to be the solution of
$(P_2)$ since $f_\rho(u) = u$ implies $(u, v - u) + \rho j(v) - \rho
j(u) \geq (u, v - u) + \rho L(v - u) - \rho a(u, v - u) ~ \forall v
\in V$. Therefore  
\begin{equation}
a(u, v - u) + j(v) - j(u) \geq L(v - u) ~~ \forall v \in
V. \tag{4.9}\label{c1:eq4.9} 
\end{equation}
Hence $(P_2)$ has a unique solution.

The existence and uniqueness of the problem $(\pi^u_\rho)$ follows
from the following  

\begin{lemma}\label{c1:lem4.1}%Lem 4.1
Let\pageoriginale $b : V \times V \to \mathbb{R}$ be a symmetric continuous,
bilinear, $V$ -elliptic form with $V$ -elliptic constant $\beta$. Let
$L \in V^*$ and $j : V \to \bar{\mathbb{R}}$ be a convex,
l.s.c. proper functional. Let $J(v) = \dfrac{1}{2}b (v, v) + j(v)-
L(v)$. Then the minimization problem $(\pi)$:  

{\em To find $u$ such that}
\begin{equation*}
(\pi) 
\begin{cases}
J(u) \leq J(v)\ \forall v \in V,\\
&\\
u \in V
\end{cases}
\end{equation*}
{\em has a unique solution which is characterised by}
\begin{equation}
\begin{cases}
b(u, v - u) +j(v) -j(u) \geq L(v - u) ~~\forall v \in V,\\
&\\
u \in V.
\end{cases}\tag{4.10}\label{c1:eq4.10}
\end{equation}
\end{lemma}

\begin{proof}
{\em (i)~Existence and uniqueness of $u$}

Since $b(\cdot , \cdot)$ is strictly convex, $j$ is convex and $L$ is
linear, 
we have $J$ strictly convex. $J$ is l.s.c. because $b(\cdot , \cdot)$
and $L$ 
are continuous and $j$ is l.s.c. 
\end{proof}

Since $j$ is convex, l.s.c. and proper, there exists $\lambda \in
V^*$ and $\mu \in \mathbb{R}$ such that  
$$
j(v)\geq \lambda (v) +  \mu \text{ (cf. EKLAND - TEMAM [1])},
$$
therefore
\begin{equation}
\begin{cases}
J(v) & \geq \frac{\beta}{2} \parallel v \parallel^2 - \parallel
\lambda \parallel \parallel v\parallel -\parallel L \parallel
~~\parallel v\parallel + \mu\\ 
&\\
& =\left(\sqrt{\frac{\beta}{2}} \parallel v \parallel - \frac{(\parallel
  \lambda \parallel + \parallel L \parallel )}{2}
\sqrt{\frac{2}{\beta}}\right)^2 + \mu - \frac{(\parallel \lambda \parallel +
  \parallel L \parallel )^2}{2\beta}. 
\end{cases}\tag{4.11}\label{c1:eq4.11}
\end{equation}
Hence
\begin{equation}
J(v) \to + \infty \text{ as } \parallel  v \parallel  \to +
\infty. \tag{4.12}\label{c1:eq4.12} 
\end{equation}

Hence\pageoriginale (cf. CEA [\ref{k27:e1}] ) there exists a unique solution for the optimization problem $(\pi)$. 

\textbf{Characterisation of} $u$ : We show that the problem $(\pi)$ is
equivalent to (4.10) and thus get a characterisation of $u$. 

\textbf{(2) Necessity of} \eqref{c1:eq4.10} : Let $0 < t \leq 1$. Let $u$ be the
solution of $(\pi)$. Then for all $v \in V$ we have  
\begin{equation}
J(u) \leq J(u + t(v - u)). \tag{4.13}\label{c1:eq4.13}
\end{equation}
Set $J_0 (V) = \dfrac{1}{2} b(v, v) - L(v)$, then \eqref{c1:eq4.13} becomes
\begin{equation}
\begin{cases}
0 & \leq J_0 (u + t(v - u)) - J_0(u) + j(u + t(v - u)) - j(u)\\
&\\
& \leq J_0 (u + t(v - u)) - J_0(u) + t[j(v) - j(u)]\quad\forall v \in V
\end{cases}\tag{4.14}\label{c1:eq4.14}
\end{equation}
got by using convexity of $j$. Dividing by $t$ in \eqref{c1:eq4.14} and taking
the limit as $t \to 0$ we get  
\begin{equation}
0 \leq (J'_0(u), v - u) + j(v) - j(u) ~~\forall v \in
V. \tag{4.15}\label{c1:eq4.15} 
\end{equation}
Since $b(\cdot , \cdot)$ is symmetric we have 
\begin{equation}
(J'_0(v), w) = b(v, w) - L(w) ~~\forall v, w \in
  V. \tag{4.16}\label{c1:eq4.16}
\end{equation}
From \eqref{c1:eq4.15} and \ref{c1:eq4.16} we obtain 
$$
b(u, v - u) + j(v) - j(u ) \geq L(v - u) ~~\forall v \in V.
$$
This proves the necessity. 

\textbf{(3) Sufficiency of} \eqref{c1:eq4.10}. Let $u$ be a solution
of \eqref{c1:eq4.10} ; for $v \in V$ 
\begin{equation}
J(v) - J(u) = \frac{1}{2} [b(v, v) -b (u, u)] + j(v) - j(u) -L(v -
u). \tag{4.17}\label{c1:eq4.17}
\end{equation}
But 
\begin{align*}
b(v,v) & = b(u+v-u, u+v-u)\\
&\\
& = b(u,u)+2b(u,v-u) +b(u-v,u-v).
\end{align*}
Therefore\pageoriginale
\begin{equation*}
J(v) - J(u) = b(u, v - u)+ j (v) - j(u) - L(v - u)+ \frac{1}{2}b(v -
u, v - u). \tag{4.18}\label{c1:eq4.18} 
\end{equation*}

Since $u$ is a solution of \eqref{c1:eq4.10} and $b(v - u, v - u) \geq 0$ we get 
\begin{equation*}
J(v) - J(u) \geq 0. \tag{4.19}\label{c1:eq4.19}
\end{equation*}

Hence $u$ is a solution of $(\pi)$. 

By taking $b(\cdot , \cdot)$ to be the inner product in $V$ and replacing
$j(v)$ and $L(v)$ in Lemma~\ref{c1:lem4.1} by $ \rho j(v)$ and $(u, v) + \rho L(v)
-\rho a(u,v)$, respectively, we get the solution for $(\pi^u_\rho)$. 

\begin{remark}\label{c1:rem4.1}%remk  4.1
From the proof of Theorem~\ref{c1:thm4.1} we get an algorithm for solving
$(P_2)$. This algorithm is given by  
\begin{equation*}
\begin{cases}
 (1)&u^0 \in V, 0 < \rho < \frac{2 \alpha}{\parallel A \parallel}^2,\\
(2) &(u^{n+1}, v - u^{n+1}) + \rho j(v) - \rho j(u^{n+1}) \geq (u^n, v
  - u^{n+1})\\ 
 & \hspace{2cm}+ \rho L(v - u^{n+1}) -\rho a(u^n, v-u^{n+1}) ~~\forall v
  \in V,\\ 
(3)& u^{n+1} \in V.
\end{cases}\tag{4.20}\label{c1:eq4.20} 
\end{equation*}
\end{remark}

Then one can easily see that $u_n \to u$ \textit{strongly} in $V$ and
$u$ will be the solution of $(P_2)$. Difficulties may arise in using
this scheme when $j( \cdot)$ is not assumed to be differentiable. At
each iteration the problem we have to solve is also a problem of the
same order of difficulty as that of the original problem (actually
conditioning can be better provided $\rho$ has been conveniently
chosen). If $a(\cdot , \cdot)$ is not symmetric the fact that $(\cdot , \cdot)$ is
symmetric can also give some simplification.  

\section{-Internal Approximation of EVI of First Kind}\label{c1:s5}%Sec 5

\subsection{Introduction}\label{c1:ss5.1}%Subsec 5.1

In this chapter we shall study the approximation of EVI of the first
kind from an abstract, axiomatic point of view. 

\subsection{The continuous problem}\label{c1:ss5.2}%Sec 5.2

The\pageoriginale assumptions on $V$, $K$, $L$ and $a(\cdot , \cdot)$ are as in 
section~\ref{c1:s2}. We are interested in the approximation of
\begin{equation*}
(P_1) 
\begin{cases}
a(u, v - u) \geq L(v - u) ~~\forall v \in K,\\
&\\
u \in K,
\end{cases}
\end{equation*}
which has one and only solution by Theorem~\ref{c1:thm3.1}.

\subsection{The approximate problem}\label{c1:ss5.3}%subSec 5.3
 
\subsubsection{The approximation of $V$ and $K$}\label{c1:sss5.3.1}
%subsubsec 5.3.1 

We are given a parameter $h$ converging to $0$ and a family $(V_h)_h$
of closed subspaces of $V$. (In practice $V_h$ are finite dimensional
and the parameter $h$ varies over a sequence). We are also given a
family $(K_h)_h$ of closed, convex, non-empty subsets of $V$ with $K_h
\subset V_h$ ~~$\forall h$ (in general we do not assume $K_h \subset
K)$ such that $(K_h)_h$ satisfies the following two conditions : 

\begin{enumerate}[(i)]
\item If $(v_h)_h$ is such that $v_h \in K_h$ ~$\forall h$ and
  $(v_h)_H$ is bounded in $V$ then the \textit{weak} cluster points of
  $(v_h)_h$ belong to $K$. 

\item Assume there exist $\chi \subset V$, $\bar{\chi} = K$ and $r_h :
  \chi \to K_h$ such that $\lim \limits_{h \to 0} r_h v=v$
  \textit{strongly in}$V$,~~$\forall v \in \chi$. 
\end{enumerate}

\begin{remark}\label{c1:rem5.1}%rem 5.1
If $K_h \subset K$ $\forall h$ then (i) is trivially satisfied because
$K$ is weakly closed. 
\end{remark}

\begin{remark}\label{c1:rem5.2}%5.2
$\displaystyle{\mathop{\cap}_h} K_h \subset K$.
\end{remark}

\begin{remark}\label{c1:rem5.3}%remk 5.3
A useful variant of condition (ii) for $r_h$ is (ii)' Assume there
exists a subset $\chi \subset V$ such that $\bar{\chi} = K$ and $r_h :
\chi \to V_h$ with the property that for each $v \in \chi$, there
exists $h_0 = h_0 (v)$ with $r_h v \in K_h$ for all $h \leq h_0
(v)$ and $\lim \limits_{h \to 0} r_h v = v$ {\em strongly} in $V$. 
\end{remark}

\subsubsection{Approximation of $(P_1)$:}\label{c1:sss5.3.2}%Subsec

 The problem $(P_1)$ is approximated by 
\begin{equation*}
(P_{1h})
\begin{cases}
a(u_h, v_h - u_h) \geq L(v_h - u_h) ~~\forall v_h \in K_h,\\
&\\
u_h \in K_h.
\end{cases}
\end{equation*}

\begin{theorem}\label{c1:thm5.1}%the 5.1
$(P_{1h})$\pageoriginale {\em has a unique solution.}
\end{theorem}

\begin{proof}
In Theorem~\ref{c1:thm3.1} taking $V$ to be $V_h$ and $K$ to be $K_h$
we have the result. 
\end{proof}

\begin{remark}\label{c1:rem5.4}%rem 5.4
In most of the cases it will be necessary to replace $a(\cdot , \cdot)$ and
$L$ by $a_h( . , .)$ and $L_h$ (usually defined - in practical cases -
from $a(\cdot , \cdot)$ and $L$ by a Numerical Integration procedure). Since
there is nothing very new on that matter compared to the classical
linear case, we shall say nothing about this problem for which we
refer to CIARLET [1, Chap. 8]. 
\end{remark}

\subsection{Convergence results}\label{c1:ss5.4}

\begin{theorem}\label{c1:thm5.2}%the 5.2
With the above assumptions on $K$ and $(K_h)_h$ we have $\lim
\limits_{h \to 0} \parallel u_h - u \parallel_V = 0$ with $u_h$ the
solution of $(P_{1h})$ and $u$ the solution of $(P_1)$. 
\end{theorem}

\begin{proof}
In this kind of convergence we usually divide the proof into three
parts. First we obtain a priori estimates for $(u_h)_h$, then weak
convergence of $(u_h)_h$ and finally with the help of weak
convergence, we will prove strong convergence. 
\end{proof}

\noindent\textbf{(1) Estimation for $u_h$.}

We will now show that there exist constants $C_1$ and $C_2$
independent of $h$ such that  
\begin{equation}
\parallel u_h \parallel^2 \leq C_1 \parallel u_h \parallel + C_2,
\forall h. \tag{5.1}\label{c1:eq5.1} 
\end{equation}
Since $u_h$ is the solution of $(P_{1h})$ we have
\begin{equation}
a(u_h, v_h - u_h) \geq L(v_h - u_h) 
\forall v_h \in K_h \tag{5.2}\label{c1:eq5.2} 
\end{equation}
i.e.
$$
a(u_h, u_h) \leq a(u_h, v_h) - L(v_h - u_h).
$$
By $V$ -ellipticity we get 
\begin{equation}
\alpha \parallel u_h \parallel^2 \leq \parallel A \parallel \cdot
\parallel u_h \parallel \cdot \parallel v_h \parallel + \parallel L
\parallel (\parallel v_h \parallel + \parallel u-h \parallel )~~
\forall v_h \in K_h. \tag{5.3}\label{c1:eq5.3} 
\end{equation}

Let $v_0 \in \chi$ and $v_h = r_h v_0 \in K_h$. By condition
(ii) on $K_h$ we have $r_h v_0 \to v_0$ strongly in $V$ and hence
$\parallel v_h \parallel$ is uniformly bounded by a constant
$m$. Hence \eqref{c1:eq5.3} can be written as  
$$
\parallel u_h \parallel^2 \leq \frac{1}{\alpha} \{(m \parallel A
\parallel + \parallel L\parallel ) \parallel u_h \parallel + \parallel
L \parallel m\} = C_1 \parallel u_h \parallel + C_2, 
$$\pageoriginale
where $C_1 = \dfrac{1}{\alpha}(m \parallel A \parallel + \parallel L
\parallel)$ and $C_2 = \dfrac{m}{\alpha} \parallel L \parallel$; then
\eqref{c1:eq5.1} implies $\parallel u_h \parallel \leq C ~~ \forall h$.  

\medskip
\noindent
\textbf{(2) Weak convergence of $(u_h)_h$ :} Relation \eqref{c1:eq5.1} gives
$u_h$ is uniformly bounded. Hence there exists a subsequence say
$\{u_{h_{i}}\}$ such that $u_{h_{i}}$ converges to $u^*$ weakly in
$V$. By condition (i) on $(K_h)_h$ we have $u^* \in K$. We will
prove that $u^*$ is a solution for $(P_1)$. We have  
\begin{equation}
a(u_{h_{i}}, u_{h_{i}}) \leq a(u_{h_{i}}, v_{h_{i}})-L(v_{h_{i}},
u_{h_{i}}) ~~\forall v_{h_{i}} \in K_{h_{i}}. \tag{5.4}\label{c1:eq5.4} 
\end{equation}
Let $v \in \chi$ and $v_{h_{i}} = r_{h_{i}} v$. Then \eqref{c1:eq5.4} becomes 
\begin{equation}
a(u_{h_{i}}, u_{h_{i}}) \leq a(u_{h_{i}}, r_{h_{i}} v)-L(r_{h_{i}}
v-u_{h_{i}}). \tag{5.5}\label{c1:eq5.5} 
\end{equation}
Since $r_{h_{i}} v$ converges strongly to $v$ and $u_{h_{i}}$
converges to $u^*$ weakly as $h_i \to 0$ taking the limit in
\eqref{c1:eq5.5} we get  
\begin{equation}
\mathop{\lim\inf}_{h_{i} \to 0} a(u_{h_{i}}, u_{h_{i}} )
\leq a(u^*, v) -L(v-u^* ) ~~\forall v \in \chi.\tag{5.6}\label{c1:eq5.6} 
\end{equation}
Also we have 
\begin{align*}
 & 0 \leq a(u_{h_{i}}-u^*, u_{h_{i}} -u^*)  \leq a(u_{h_{i}},
  u_{h_{i}} )-a(u_{h_{i}},u^*)-a (u^*,u_{h_{i}}) +a(u^*,u^*)
\end{align*}
i. e.
\begin{equation*}
 a(u_{h_{i}},u^*) +a(u^*,u_{h_{i}} )-a(u^*,u^* )
\leq a(u_{h_{i}},u_{h_{i}}). 
\end{equation*}

By taking the limit we obtain
\begin{equation}
a(u^*, u^*) \leq \mathop{\lim\inf}_{h_i \to 0} a(u_{h_{i}},
u_{h_{i}}). \tag{5.7}\label{c1:eq5.7} 
\end{equation}
From \eqref{c1:eq5.6} and \eqref{c1:eq5.7} we get
$$
a(u^*, u^*) \leq \mathop{\lim\inf}_{h_i \to 0} a(u_{h_{i}},
u_{h_{i}}) \leq a(u^*, v)-L(v - u^*) ~\forall v \in \chi. 
$$

Therefore\pageoriginale we have, 
\begin{equation}
\tag{5.8}\label{c1:eq5.8}
\begin{cases}
a(u^*, v-u^*) \geq L(v-u^*) ~~\forall v \in \chi, \\
&\\
u^* \in K.
\end{cases}
\end{equation}
Since $\chi$ is dense in $K$ and $a(\cdot , \cdot)$, $L$ are continuous, we
get from \eqref{c1:eq5.8}
\begin{equation}
\begin{cases}
a(u^*, v-u^*) \geq L(v-u^*) ~~\forall v \in K, \\
u^* \in K.
\end{cases}\tag{5.9}\label{c1:eq5.9}
\end{equation}

Hence $u^*$ is a solution of $(P_1)$. By Theorem~\ref{c1:thm3.1}, the solution for
$(P_1)$ is unique and hence $u^* = u$ is the unique solution. Hence
$u$ is the only cluster point of $\{u_h\}_h$ in the weak topology of
$V$. Hence the whole $\{u_h\}_h$ converges to $u$ weakly. 

\medskip
\noindent\textbf{(3) Strong convergence}: We have by $V-$ellipticity
of $a(\cdot , \cdot)$ 
\begin{equation}
0 \leq \alpha \parallel u_h - u \parallel^2 \leq a(u_h - u, u_h - u) =
a(u_h, u_h) - a(u_h, u)-a(u, u_h) + a(u, u) \tag{5.10}\label{c1:eq5.10} 
\end{equation}
where $u_h$ is the solution of $(P_{1h})$ and $u$ is the solution of
$(P_1)$. Since $u_h$ is the solution of $(P_{1h})$ and $r_h v \in
K_h$ for any $v \in \chi$, we get by $(P_{1h})$ 
\begin{equation}
a(u_h, u_h) \leq a(u_h, r_h v) - L(r_h v - u_h) ~~~\forall v \in
\chi. \tag{5.11}\label{c1:eq5.11} 
\end{equation}
Since $\lim \limits_{h \to 0} u_h = u$ \textit{weakly} in $V$ and
$\lim \limits_{h \to 0} r_h v= v$ \textit{strongly} in $V$ (by
condition (ii)) we obtain, from \eqref{c1:eq5.10}, \eqref{c1:eq5.11} and after taking the
lim, that $\forall v \in \chi$ we have: 
\begin{equation}
0 \leq \alpha \lim \inf \parallel u_h - u \parallel^2 \leq \alpha \lim
\sup \parallel u_h - u \parallel^2 \leq a (u, v - u) - L(v -
u). \tag{5.12}\label{c1:eq5.12} 
\end{equation}
By \textit{density} and \textit{continuity}, \eqref{c1:eq5.12} also holds
$\forall v \in K$; then taking $v =u$ in \eqref{c1:eq5.12} we obtain that  
$$
\lim \limits_{h \to 0} \parallel u_h - u \parallel^2 = 0
$$
i.e. the strong convergence. 

\begin{remark}%remk5.5
Error\pageoriginale estimates for the EVI of the first kind can be
found in FALK 
[\ref{k44:e1}], [\ref{k44:e2}], [\ref{k44:e3}], STRANG-MOSCO
[\ref{k86:e1}], STRANG [\ref{k84:e1}], 
GLOWIN\-SKI-LIONS-TREMOLIERES (G.L.T.) [\ref{k53:e1}], [\ref{k53:e2}],
CIARLET [\ref{k33:e1}], BREZZI 
[\ref{k18:e1}], FALK-MERCIER [\ref{k45:e1}], GLOWINSKI
[\ref{k51:e1}]. But like in many nonlinear 
problems the methods used to obtain these estimates are specific to
the particular problem under consideration (as we shall see in the
following sections). 

This remark still holds for the approximation of EVI of the second
kind which is the subject of Section~6. 
\end{remark}

\begin{remark}\label{c1:rem5.6}%remk 5.6
If for a given problem, several approximations are\break available, and if
computations are needed, the choice of the approximations to be used
is not obvious. We have to take into account not only the convergence
properties of the method, but also the computation involved in that
method. Some iterative methods are best suited only for some
problems. Some methods are easier to program than others. 
\end{remark}

\section{Internal Approximation of EVI of Second Kind}\label{c1:s6}%Sec 6

\subsection{The Continuous Problem}\label{c1:ss6.1}%Subsec 6.1

The assumptions on $V$, $a(\cdot , \cdot)$, $L$ $j(\cdot)$ being as in 
Section~\ref{c1:ss2.1}, we shall consider the approximation of  
\begin{equation*}
(P_2)
\begin{cases}
a(u, v - u) + j(v) - j(u) \geq L(v - u) ~~~\forall v \in V,\\
&\\
u \in V
\end{cases}
\end{equation*}
which has one and only one solution by Theorem~\ref{c1:thm4.1}.

\subsection{Definition of the approximate
  problem}\label{c1:ss6.2}%Subsec 6.2 

\textbf{Preliminary remark}: We assume in the sequel that $j : V \to
\mathbb{R}$ is continuous. We can prove the same sort of results as in
this section under less restrictions (see Chapter~\ref{chap4},
Section~\ref{c4:s2}).  

\subsubsection{Approximation of $V$}\label{c1:sss6.2.1}%Subsubsec 6.2.1
 
Given\pageoriginale a real parameter $h$ converging to 0 and a family
$(V_h)_h$ of 
closed subspaces of $V$ (in practice we will take $V_h$ to be finite
dimensional and $h$ to vary over a sequence), we assume that $(V_h)_h$
satisfies 
\begin{enumerate}[(i)]
\item there exists $U \subset V$ such that $\bar{U} = V$ and for each
  $h$, a map $r_h : U \to V_h$ such that $\lim \limits_{h \to 0}r_h v
  = v$ strongly in $V$, $\forall v \in U$. 

\subsubsection{Approximation of
  $j(\cdot)$}\label{c1:sss6.2.2}%Subsubsec 6.2.2   

We approximate the functional $j(\cdot)$ by $(j_h)_h$ where for each
$h$, $j_h$ satisfies 
\begin{equation}
\begin{cases}
j_h : V_h \to \bar{\mathbb{R}}\\
& \tag{6.1}\\
j_h \text{\em is convex, l.s.c. and uniformly proper in }h.
\end{cases}
\end{equation}


The family $(j_h)_h$ is said to be \textit{uniformly proper in} $h$ if
there exist $\lambda \in V^*$ and $\mu \in \mathbb{R}$ such
that  
\begin{equation*}
h(v_h) \geq \lambda (v_h) + \mu ~~ \forall v_h \in V_h, \forall
h. \tag{6.2}\label{c1:eq6.2}
\end{equation*}
Furthermore we assume that $(j_h)_h$ satisfies
\item if $v_h \to v$ \textit{weakly} in $V$ then 
$$
\mathop{\lim\inf}_{h \to 0} j_h(v_h) \geq j(v)
$$
\item $\lim \limits_{h \to 0} j_h(r_h v) = j(v) ~~ \forall v \in U$.
\end{enumerate}

\begin{remark}\label{c1:rem6.1}%remk 6.1
In all the applications we know, if $j(\cdot)$ is a continuous
functional then it is always possible to construct continuous $j_h$
satisfying (ii) and (iii). 
\end{remark}

\begin{remark}\label{c1:rem6.2}%remk 6.2
In some cases we are fortunate enough to have $j_h(v_h) = j(v_h)
\forall v_h$, $\forall h$, and then (ii) and (iii) are trivially
satisfied. 
\end{remark}

\subsubsection{Approximation of $(P_2)$}\label{c1:sss6.2.3}%Subsubsec
                                %6.2.3  

We\pageoriginale approximate $(P_2)$ by 
\begin{equation*}
(P_{2h})
\begin{cases}
a(u_h, v_h - u_h) + j_h(v_h) - j_h(u_h) \geq L(v_h - u_h) ~~~\forall
v_h \in V_h, \\ 
&\\
u_h \in V_h.
\end{cases}
\end{equation*}

\begin{theorem}\label{c1:thm6.1}%them 6.1
{\em Problem $(P_{2h})$ has one only one solution.}
\end{theorem}

\begin{proof}
In Theorem~\ref{c1:thm4.1} taking $V$ to be $V_h$, $j(\cdot)$ to be
$j_h(\cdot)$ we get the result. 
\end{proof}

\begin{remark}%remk 6.3
  Remark~\ref{c1:rem5.4} of Section~\ref{c1:s5} still holds for $(P_2)$
  and $(P_{2h})$. 
\end{remark}

\subsection{Convergence results}\label{c1:ss6.3}%Subsec 6.3

\begin{theorem}\label{c1:thm6.2}%thm 6.2
Under the above assumptions on $(V_h)_h$ and $(j_h)_h$ we have 
\begin{equation}
\begin{cases}
\lim \limits_{h \to 0} \parallel u_h -u \parallel = 0,\\
\lim \limits_{h \to 0} j_h(u_h) = j(u).
\end{cases}\tag{6.3}\label{c1:eq6.3}
\end{equation}
\end{theorem}
\begin{proof}
As in the proof of Theorem 5.2 we divide the proof into three parts.
\end{proof}

\textbf{(1) Estimation for }$u_h$. We will show that there exist
positive constants $C_1$ and $C_2$ independent of $h$ such that  
\begin{equation}
\parallel u_h \parallel^2 \leq C_1 \parallel u_h \parallel + C_2 ~~~
\forall h. \tag{6.4}\label{c1:eq6.4} 
\end{equation}

Since $u_h$ is the solution of $(P_{2h})$ we have 
\begin{equation}
a(u_h, u_h) +j_h(u_h) \leq a(u_h, v_h)+j_h(v_h) - L(v_h - u_h) ~~
\forall v_h \in V_h. \tag{6.5}\label{c1:eq6.5} 
\end{equation}
By using relation \eqref{c1:eq6.2} we get 
\begin{align*}
  \alpha \parallel u_h \parallel^2 \leq \parallel \lambda \parallel
  ~\parallel u_h \parallel & + | \mu | + \parallel A \parallel ~\parallel
  u_h \parallel ~\parallel  v_h \parallel\\ 
  & + | j_h (v_h) | + \parallel L
  \parallel ( \parallel v_h \parallel + \parallel u_h \parallel
  ). \tag{6.6}\label{c1:eq6.6} 
\end{align*}

Let $v_0 \in U$ and $v_h = r_h v_0$. By using condition (i) and
(iii) there exists a constant $m$, independent of $h$ such that
$\parallel v_h \parallel \leq m$ and $|j_h(v_h) | \leq m$. Therefore
\eqref{c1:eq6.6} can be written as  

\begin{equation*}
\begin{cases}
\parallel u_h \parallel^2 & \leq \frac{1}{\alpha} (\parallel \lambda
\parallel + \parallel A \parallel \cdot m + \parallel L \parallel )
\parallel u_h \parallel + \frac{m}{\alpha} (1+ \parallel L \parallel )
+ \frac{| \mu |}{\alpha}\\ 
&\\
& = C_1 \parallel u_h \parallel + C_2
\end{cases}
\end{equation*}
where\pageoriginale 
$$
C_1 = \frac{1}{\alpha}(\parallel \lambda \parallel + \parallel A
\parallel \cdot m + \parallel L \parallel ) ~~\text{\em  and } C_2 =
\frac{m}{\alpha}(1 + \parallel L \parallel ) + \frac{| \mu |}{\alpha} 
$$
and \eqref{c1:eq6.4} implies 
$$
\parallel u_h \parallel \leq C ~\forall h \text{ where } C \text{ is a
  constant}. 
$$

\textbf{(2) Weak convergence of $u_h$:} Relation \eqref{c1:eq6.4} gives that
$u_h$ is uniformly bounded. Therefore there exists a subsequence
$(u_{h_{i}})_{h_{i}}$ such that $u_{h_{i}} \to u_h$ weakly in $V$. 

Since $u_h$ is the solution of $(P_{1h})$ and $r_h v \in V_h ~~
\forall h$ and $\forall v \in U$ we get  
\begin{equation}
a(u_{h_{i}}, u_{h_{i}})+j_{h_{i}} (u_{h_{i}}) \leq a(u_{h_{i}},
r_{h_{i}} v) + j_{h_{i}}(r_h v_i) -L(r_{h_{i}} v-u_{h_{i}}). 
\tag{6.7}\label{c1:eq6.7} 
\end{equation}
By condition (iii) and weak convergence of $\{u_{h_{i}}\}$ we get  
\begin{equation}
\mathop{\lim\inf}_{h \to 0}
[a(u_{h_{i}},u_{h_{i}})+j_{h_{i}}(u_{h_{i}})] \leq a(u^*, v)
+j(v)-L(v-u^*) ~ \forall v \in U. \tag{6.8}\label{c1:eq6.8}  
\end{equation}
As in \eqref{c1:eq5.7} and using condition (ii), we get
\begin{equation}
a(u^*, u^*) +j(u^*) \leq \mathop{\lim\inf}_{h \to 0}
[a(u_{h_{i}},u_{h_{i}})+j_{h_{i}}(u_{h_{i}})]. \tag{6.9}\label{c1:eq6.9} 
\end{equation}
From \eqref{c1:eq6.8}, \eqref{c1:eq6.9} and using the density of $U$
we have  
\begin{equation*}
\begin{cases}
a(u^*, v - u^*) + j(v) - j(u^*) \geq L(v - u^*) ~~\forall v \in V,\\
&\\
u^* \in V.
\end{cases}
\end{equation*}
This implies $u^*$ is a solution of $(P_2)$. Hence $u^* = u$ is the
unique solution of $(P_2)$ and this shows that $(u_h)$ converges to u
\textit{weakly}. 

\medskip
\textbf{(3) Strong convergence of $(u_h)_h$:} We\pageoriginale have by $V$
-ellipticity of $a(\cdot , \cdot)$ and by $(P_{2h})$ 
\begin{equation}
\begin{cases}
\alpha \parallel u_h -u \parallel^2 & +j_h(u_h) \leq a(u_h -u,
u_h-u)+j_h(u_h)=\\ 
&\\
& = a(u_h,u_h)-a(u,u_h)-a(u_h,u)+a(u,u)+j_h(u_h) \leq \\
& \leq a(u_h,r-h v)+j_h(r_hv) -L(r_hv-u_h)-a(u,u_h)\\
&\\
& -a(u_h,u)+a(u,u) ~\forall v \in U.
\end{cases}\tag{6.10}\label{c1:eq6.10}
\end{equation}

The right hand side of inequality \eqref{c1:eq6.10} tends to $a(u,
v-u)+j(v) - L(v-u)$ as $h \to 0 \forall v \in U$. Therefore we have  
\begin{equation}
\begin{cases}
\mathop{\lim\inf}_{h \to 0} j_h(u_h) & \leq \mathop{\lim\inf}_{h
\to 0} [\alpha \parallel u_h-u \parallel^2 + j_h(u_h)]
\leq \\ 
&\\
& \leq \mathop{\lim\sup}_{h \to 0} [\alpha \parallel u_h-u
  \parallel^2 + j_h(u_h)] \leq \\ 
&\\ 
& \leq a(u,v-u) +j(v)-L(v-u)\, \forall v \in U.
\end{cases}\tag{6.11}\label{c1:eq6.11}
\end{equation}
By density of $U$, \eqref{c1:eq6.11} holds $\forall v \in V$. Replacing $v$
by $u$ in \eqref{c1:eq6.11} and using condition (ii) we obtain 
$$
j(u) \leq \mathop{\lim\inf}_{h \to 0} j_h(u_h) \leq 
\mathop{\lim\sup}_{h \to 0} [\alpha \parallel u-u_h \parallel^2 +
  j_h(u_h)] \leq j(u). 
$$
This implies that
\begin{align*}
&\lim _{h \to 0} j_h(u_h) = j(u)~~ \text{\em and}\\
&\lim_{h \to 0} \parallel u_h-u \parallel = 0.
\end{align*}
This proves the theorem.

\section{References}\label{c1:s7}%Sec 7

For\pageoriginale generalities on variational inequalities from a
theoretical point 
of view see Lions-Stampacchia [\ref{k69:e1}], Lions [\ref{k67:e1}],
Ekeland-Temam [\ref{k87:e1}].  

For generalities on the approximation of variational inequalities
{\small from}
the numerical point of view see Falk [\ref{k44:e1}],
Glowinski-Lions-Tremolieres 
[\ref{k53:e1}], [\ref{k53:e2}], Strang [\ref{k84:e1}],
Brezzi-Hager-Raviart [\ref{k18:e1}].  
