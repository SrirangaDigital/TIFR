\thispagestyle{empty}
\begin{center}
  {\Large\bf Lectures on}\\[5pt]
  {\Large\bf Stochastic Flows And Applications}
  \vskip 1cm

  {\bf By}
  \medskip

  {\large\bf H.Kunita}\\[10pt]
  Lectures delivered at the \\
  {\bf Indian Institute Of Science Bangalore}\\
  under the \\
  {\bf T.I.F.R. -- I.I.Sc. Programme}\\ 
  {\bf In Applications Of  Mathematics}
\vfill


{\bf   Notes by}
\medskip

{\large\bf M.K. Ghosh}
\vfill


  Published for the\\
  \textbf{Tata Institute Of Fundamental Research}\\
  {\bf Springer-Verlag}\\
  Berlin Heidelberg New York Tokyo\\
  {\bf 1986}
\end{center}
\eject

\thispagestyle{empty}
\begin{center}
  {\bf Author}\\
  {\large\bf H.Kunita}\\
  Faculty of Engineering 36\\
  Department of Applied Sciences\\
  Kyushu University\\
  Hakozaki, Fukuoka 812\\
  JAPAN  
\end{center}
\vfill

\begin{center}
  \copyright \textbf{Tata Institute Of Fundamental Research, 1986}\\[20pt]
  ISBN 3-540-12878-6 Springer-verlag, Berlin, Heidelberg,\\ New
  York. Tokyo\\
  ISBN 0-387-12878-6 Springer-verlag, New  York. Heidelberg.\\ Berlin. Tokyo
\end{center}
\vfill

\begin{center}
\parbox{0.9\textwidth}{%
  No part of this book may be reproduced in any form by print,
  microfilm or any other means without written permission from the
  Tata Institute of Fundamental Research, Colaba, Bombay-400 005.}
\vfill

\parbox{0.9\textwidth}{Printed by K.M.Aarif at paper print India, Worli, 
  Bombay 400 018 and published by H.Goetze,
  Springer-Verlag, Heidelberg, West Germany
  Printed In India}
\end{center}

\eject

\thispagestyle{empty}

\chapter*{Preface}


These notes are based on a series of lectures given at
T.I.F.R. Centre, Bangalore during November and December 1985. The
lectures consisted of two parts. In the first part, I presented basic
properties of stochastic flows, specially of Brownian flows. Their
relations with local characteristics and with stochastic differential
equations were central problems. I intended to show the homeomorphism
property of the flows without using stochastic differential
equations. In the second part, as an application of the first part, I
presented various limit theorems for stochastic flows. These include
the following: 
\begin{enumerate}[(a)]
\item Approximation theorems of stochastic differential equations and\break
  stochastic flows due to Bismut, Ikeda - Watanabe, Malliavin, Dowell
  et al. 
\item Limit theorems for driven processes due to Papanicolaou- Stroock-
  Varadhan. 
\item Limit theorems for stochastic ordinary differential equations
  due to Khasminskii, Papanicolau - Kohler, Kesten - Papanicolaou et
  al. 
\end{enumerate} 	

I intended to treat these limit theorems in a unified method.

I would like to thank M.K. Ghosh for his efforts in completing these
notes. Also I wish to express my gratitude to Professor
M.S. Raghunathan and $T.I.F.R$, for giving me this opportunity to
visit India. Finally I would like to thank Ms. Shantha for her
typing. 
\vskip 1cm

\hfill{\large\bf H.Kunita}

