\chapter{Smooth Rigidity Problem}\label{c17}

Recall\pageoriginale we remarked in Lecture \ref{c1} that the obvious
smooth analogue of Borel's Conjecture \ref{c1:conj1.3} is false. Namely, Browder
had shown in \cite{13} that it is false even in the basic case where
$M$ is an $n$-torus. In fact, it was shown in Lecture \ref{c16} that
$T^{n}$ and $T^{n}\# \Sigma^{n}(n\geq 5)$ are not diffeomorphic when
$\Sigma^{n}$ is an exotic sphere; although they are clearly
homeomorphic. 

But when it is assumed that both $M$ and $N$ in Conjecture
\ref{c1:conj1.3} are non-positively curved Riemannian manifolds, then
smooth rigidity frequently happens. The most fundamental instance of
this is an immediate consequence of Mostow's Strong Rigidity Theorem;
cf.\@ \cite{75}. Namely, Mostow showed that any isomorphism between
fundamental groups is, in fact, induced by an isometry if $M$ and $N$
satisfy some more geometric constraints and provided we are allowed to
change the metric on $M$ by scaling it on each irreducible metric
factor of its universal cover. Adequate extra constraints are that
both manifolds be locally symmetric spaces and that the universal
cover of $M$ does not have a 1 or 2 dimensional metric
factor. Mostow's result led Lawson and Yau \cite[p. 673, Problem
  12]{99} to pose the problem of whether smooth rigidity always hods
when both $M$ and $N$ are negatively curved; in particular, does
$\pi_{1}M\simeq \pi_{1}N$ imply that $M$ and $N$ are diffeomorphic?
Farrell and Jones showed in \cite{39} that this is {\em not} always
true even when $M$ is a real hyperbolic manifold. This lecture is
devoted to constructing such a counterexample to smooth rigidity. It
is loosely motivated by the construction used to prove Theorem
\ref{c15:thm15.4}. 

The manifold $N^{m}$ in this counterexample will be $M^{m}\#
\Sigma^{m}$ where $\Sigma^{m}$ is an exotic sphere and $M^{m}$ is a
stably parallelizable, real hyperbolic manifold (with $m\geq 5$) and
having sufficiently large injectivity radius; i.e., every closed
geodesic in $M^{m}$ must be sufficiently long. To implement this
program, we need to know that such manifolds $M^{m}$ exist. To show
this, we will use the following three results: the first due to
Sullivan \cite{94}, the second to Borel \cite{9} and the third to
Malcev \cite{71}; cf.\@ \cite{68}.

\begin{thm}[Sullivan]\label{c17:thm17.1}
Every\pageoriginale closed real hyperbolic manifold has a stably
parallelizable finite sheeted cover.
\end{thm}

\begin{thm}[Borel]\label{c17:thm17.2}
There exists closed real hyperbolic manifolds in every dimension
$m\geq 2$, as well as closed complex hyperbolic manifolds in every
even (real) dimension.
\end{thm}

\begin{thm}[Malcev]\label{c17:thm17.3}
Let $M$ be a closed Riemannian manifold which is either real or
complex hyperbolic, then $\pi_{1}M$ is residually finite; i.e., the
intersection of all its subgroups of finite index contains only the
trivial element.
\end{thm}

The combination of Theorem \ref{c17:thm17.1} with Theorem
\ref{c17:thm17.2} clearly yields the existence, in every dimension
$m\geq 2$, of a $m$-dimensional, closed and stably parallelizable real
hyperbolic manifold $\mathcal{M}^{m}$. Furthermore, any finite sheeted
over of $\mathcal{M}^{m}$ will have these same properties. And Theorem
\ref{c17:thm17.3} can be used, as follows, to show that there exist
finite sheeted covers of $\mathcal{M}^{m}$ of arbitrarily large
injectivity radius $r$. Since there are only a finite number of closed
geodesics in $\mathcal{M}^{m}$ with length less than $2r+1$, we can
use \ref{c17:thm17.3} to find a normal subgroup $\Gamma$ with finite
index in $\pi_{1}(\mathcal{M}^{m})$ such that every closed geodesic in
$\mathcal{M}^{m}$ which represents the free homotopy class of an
element on $\Gamma-\{e\}$ must have length at least $2r+1$. Then the
injectivity radius of the finite sheeted cover of $\mathcal{M}^{m}$
which corresponds to $\Gamma$ is at least $r$. This demonstrates the
existence of the manifolds $M^{m}$ needed for our counterexample.

It remains to put a negatively curved Riemannian metric on $M^{m}\#
\Sigma^{m}$. We use the following result to do this.

\begin{lemma}\label{c17:lem17.4}
Given $m\geq 5$ and $\epsilon>0$, there exists a real number
$\alpha>0$ such that the following is true. Let $M^{m}$ be a
$m$-dimensional, oriented, closed, real hyperbolic manifold whose
injectivity radius is bigger than $\alpha$, and let $\Sigma^{m}$ be an
exotic sphere. Then $M^{m}\# \Sigma^{m}$ supports a Riemannian metric
whose sectional curvatures all lie in the open interval
$(-1-\epsilon,-1+\epsilon)$. 
\end{lemma}

Before\pageoriginale proving \ref{c17:lem17.4}, let us precisely
describe the counterexamples from \cite{39} that it immediately
yields.

\begin{thm}\label{c17:thm17.5}
The following statement is true in each dimension $m\geq 5$ such that
there exists an exotic $m$-dimensional sphere (e.g., $m=4k-1$ where
$k\geq 2$ and $k\in \mathbb{Z}$). Given $\epsilon>0$, there exist two
$m$-dimensional, closed, negatively curved Riemannian manifolds
$M^{m}$ and $N^{m}$ such that
\begin{enumerate}
\item $M^{m}$ is real hyperbolic;

\item the sectional curvatures of $N^{n}$ are all contained in the
  interval $(-1-\epsilon,-1+\epsilon)$;

\item $M^{m}$ is not diffeomorphic to $N^{m}$;

\item $M^{m}$ is homeomorphic to $N^{m}$.
\end{enumerate}
\end{thm}

The remainder of this lecture is devoted to proving
\ref{c17:lem17.4}. Each exotic sphere $\Sigma^{m}$ arises by taking 2
disjoint copies of the closed $m$-ball $\mathbb{D}^{m}$ and
identifying their boundaries $S^{m-1}$ by a self-diffeomorphism
$f:S^{m-1}\to S^{m-1}$. (Note that this construction yields $S^{m}$ when
  $f=\id_{S^{m-1}}$.) Since there are only a finite number of exotic
  spheres in each dimension $m\geq 5$, it suffices to consider a
  single $\Sigma^{m}$ and thus fix a single diffeomorphism
  $f:S^{m-1}\to S^{m-1}$.

The connected sum $M^{m}\# \Sigma^{m}$ is likewise constructed from
the disjoint union
$$
(M^{m}-\Int(\mathbb{D}^{m}))\amalg \mathbb{D}^{m}
$$
by identifying the boundaries of its two components using $f$. Let us
make this construction more explicit. Fix a point $x\in M^{m}$ and
look at the exponential map 
$$
\exp :T_{x}M^{m}\to M^{m}.
$$

The closed ball in $T_{x}M^{m}$ of radius $\alpha$ and center $0$,
denoted by $\alpha\mathbb{D}^{m}$, is smoothly embedded via exp into
$M^{m}$ since the injectivity radius of $M^{m}$ is greater than
$\alpha$. Dilate $f$ by $\alpha$ to define a self-diffeomorphism
$f_{\alpha}$ of $\partial(\alpha\mathbb{D}^{m})$; i.e., set
$$
f_{\alpha}(\alpha x)=\alpha f(x)\quad\text{for all}\quad x\in
  S^{m-1}=\partial \mathbb{D}^{m}
$$

Then\pageoriginale $M^{m}\# \Sigma^{m}$ is
\begin{equation*}
(M^{m}-\Int(\alpha\mathbb{D}^{m}))\amalg_{f_{\alpha}}(\alpha\mathbb{D}^{m})\tag{0} 
\end{equation*}
where $\amalg_{f_{\alpha}}$ means to glue together the boundaries of
the two components in the disjoint union $\amalg$ using $f_{\alpha}$.

We put a Riemannian metric $B_{\alpha}(,)$ on $M^{m}$ in terms of this
decomposition (0) as follows. Restricted to both
$M^{m}-\Int(\alpha\mathbb{D}^{m})$ and
$\frac{\alpha}{2}\mathbb{D}^{m}$, $B_{\alpha}(,)$ is the real
hyperbolic metric. Then we interpolate to define $B_{\alpha}(,)$ on
the rest of $M^{m}$; namely, on
$\alpha\mathbb{D}^{m}-\Int(\frac{\alpha}{2}\mathbb{D}^{m})$ which we
denote by $A_{\alpha}$. To do this interpolation, put a Riemannian
metric $\langle ,\rangle$ on $S^{m-1}\times [\frac{1}{2},1]$ with the
following properties (1-4).
\begin{enumerate}
\renewcommand{\labelenumi}{(\theenumi)}
\item The manifolds $S^{m-1}\times t$ and $x\times [\frac{1}{2},1]$
  intersect perpendicularly at $(x,t)$ for each $(x,t)\in
  S^{m-1}\times[\frac{1}{2},1]$.

\item The map $t\mapsto (x,t)$ is an isometry between
  $[\frac{1}{2},1]$ and $[\frac{1}{2},1]\times x$ for each $x\in
  S^{m-1}$ 

\item The map $x\mapsto (x,\frac{1}{2})$ is an isomtry from 
$S^{m-1}$ to $S^{m-1}\times \frac{1}{2}$.

\item The map $x\mapsto (f(x),1)$ is an isometry from $S^{m-1}$ to
  $S^{m-1}\times 1$.
\end{enumerate}

It is easy to construct such a metric $\langle,\rangle$. Then we
``warp'' $\langle,\rangle$ by $\sinh^{2}(\alpha t)$ to do the
interpolation. That is, let $\xi$ and $\eta$ be the distributions
tangent, respectively, to the first and second factors in the product
structure $S^{m-1}\times [\frac{1}{2},1]$. And define $B_{\alpha}(,)$
on $A_{\alpha}$ by
\begin{equation*}
B_{\alpha}(u,v)=
\begin{cases}
\sinh^{2}(\alpha t)\langle,v\rangle & \text{if~ }u,v\in \xi\\
\alpha^{2}\langle u,v\rangle & \text{if~ } u,v\in \eta\\
0 & \text{if~ } u\in \xi\text{~ and~ } v\in \eta.
\end{cases}\tag{5}\label{c17:eq5}
\end{equation*}

\begin{remark*}
In this formula \eqref{c17:eq5}, $S^{m-1}\times [\frac{1}{2},1]$ is
identified with $A_{\alpha}$ via multiplication by $\alpha$ on the
$[\frac{1}{2},1]$ factor; i.e., we used multiplication by $\alpha$ to
shift $\langle,\rangle$ to a Riemannian metric
$\langle,\rangle_{\alpha}$ on $A_{\alpha}$ and then ``warped'' this
metric by $\sinh^{2}(t)$.
\end{remark*}

It is easily seen that these definitions fit together to give a
well-defined Riemannian metric on all of $M^{m}\# \Sigma^{m}$. Note
also that the Riemannian metric $B_{\alpha}(,)$ defined on
$S^{m-1}\times [\frac{1}{2},1]$\pageoriginale by \eqref{c17:eq5} is
independent of $M^{m}$. Hence to complete the proof of Lemma
\ref{c17:lem17.4}, it suffices to verify the following statement.

\begin{lemma}\label{c17:lem17.6}
The sectional curvatures of the Riemannian metric $B_{\alpha}(,)$
defined by formula \eqref{c17:eq5}, converge uniformly to $-1$ as
$\alpha\to \infty$.
\end{lemma}

The proof of \ref{c17:lem17.6} used that sectional curvatures are
computable in terms of the first and second order partial derivatives
of the first fundamental form together with the form itself. The
following is a precise statement of what is used.

\begin{thm}\label{c17:thm17.7}
Given a positive integer $m$, there exists a polynomial $p()$ such
that the following is true. Let $g_{ij}(y)$, where $y\in
\mathbb{R}^{m}$ and $1\leq i$, $j\leq m$, be any smooth Riemannian
metric on $\mathbb{R}^{m}$ satisfying
$$
g_{ij}(0) =
\begin{cases}
1 & \text{if~ } i=j\\
0 & \text{if~ } i\neq j
\end{cases}
$$
and let $u=(u_{1},u_{2},\ldots,u_{m})$ and
$v=(v_{1},v_{2},\ldots,v_{m})$ be any pair of vectors in
$\mathbb{R}^{m}$ satisfying
$$
\Sigma^{m}_{i=1}(u_{i})^{2}=1,\quad
\Sigma^{m}_{i=1}(v_{i})^{2}=1,\quad \Sigma^{m}_{i=1}(u_{i}v_{i})=0.
$$

Then the sectional curvature of this Riemannian metric at $0$ in the
direction of the tangent plane spanned by $u$ and $v$ is the
polynomial $p()$ evaluated at
$$
\{u_{i}\}^{m}_{i=1},\quad \{v_{i}\}^{m}_{i=1},\quad
\left\{\frac{\partial g_{ij}}{\partial
  x_{k}}\right\}^{m}_{i,j,k=1},\quad
\left\{\frac{\partial^{2}g_{ij}}{\partial x_{k}\partial
  x_{1}}\right\}^{m}_{i,j,k,l=1} 
$$
\end{thm}

\begin{remark*}
This fundamental result is a direct consequence of Cartan's second
structural equation combined with the Koszul formula description of
the Levi-Civita connection; cf.\@ \cite[\S\S 5.3 and 6.2]{59}.
\end{remark*}

We proceed to sketch the proof of \ref{c17:lem17.6}. (See \cite{39}
for details.) Given $(x,t)\in S^{m}\times [\frac{1}{2},1]$, one first
constructs local co-ordinates
$y_{1},y_{2},\ldots,\break y_{m-1},\overline{t}$ about $(x,t)$ sending
$(x,t)$ to $0\in \mathbb{R}^{m}$ and such that the matrix entries
$g_{ij}(y_{1},y_{2},\ldots,y_{m-1},\overline{t})$ of the first
fundamental form\pageoriginale satisfy
$$
g_{ij}(0)=
\begin{cases}
1 & \text{if~ } i=j\\
0 & \text{if~ } i\neq j
\end{cases}
$$
and their partial derivatives $D(g_{ij})$ evaluated at $0$ have the
following limiting values $V$ (uniformly in $(x,t)$) as $\alpha\to
+\infty$:
\begin{equation*}
V=
\begin{cases}
0 & \text{if~ } D=\dfrac{\partial}{\partial
  y_{k}},\dfrac{\partial^{2}}{\partial_{y_{k}}\partial\overline{t}}\text{~
  or~ } \dfrac{\partial^{2}}{\partial y_{k}\partial y_{l}}\\[7pt]
2 & \text{if~ } D=\dfrac{\partial}{\partial \overline{t}}\\[7pt]
4 & \text{if~ }
D=\dfrac{\partial^{2}}{\partial\overline{t}\partial\overline{t}}. 
\end{cases}\tag{6}\label{c17:eq6}
\end{equation*}

Recall next the ``cusp description'' of real hyperbolic
$m$-dimensional space $\mathbb{H}^{m}$ is given by the warped product
$\mathbb{R}^{m-1}\times_{e^{t}}\mathbb{R}$ of the Euclidean spaces
$\mathbb{R}^{m-1}$ and $\mathbb{R}$ in terms of the warping function
$e^{t}$ on $\mathbb{R}$. (See \cite[pp. 204-211]{82} for the
definition and basic properties of warped products. Note that we've
reversed above the normal order of base and fibre.) Let $(h_{ij})$ be
the first fundamental form of $\mathbb{H}^{m}$ in terms of the
canonical co-ordinates $(y_{1},y_{2},\ldots,y_{m-1},\overline{t})$ on
$\mathbb{R}^{m-1}\times_{e^{\overline{t}}}\mathbb{R}$. Then it is
easily seen that
$$
h_{ij}(0)=
\begin{cases}
1 & \text{if~ } i=j\\
0 & \text{if~ } i\neq j.
\end{cases}
$$
and that the values $V$ of the partial derivatives $D(h_{ij})$
evaluated at $0$ are given by formula \eqref{c17:eq6}. But the
sectional curvatures of $H^{m}$ are identically $-1$. Hence an
elementary continuity argument based on Theorem \ref{c17:thm17.7}
shows that the sectional curvatures of $B_{\alpha}(,)$ approach the
value $-1$ uniformly as $\alpha\to +\infty$.
