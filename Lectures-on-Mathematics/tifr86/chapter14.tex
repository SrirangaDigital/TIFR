\chapter{Non-positively Curved Manifolds}

This\pageoriginale lecture is devoted to motivating the proof of the
following topological rigidity result due to Farrell and Jones
\cite{43}.

\begin{thm}\label{c14:thm14.1}
  Let $M^m$ be a closed non-positively curved Riemannian
  manifold. Then $|\mathcal{S} (M^m \times \mathbb{D}^n, \partial)|=1$
  when $m + n \geq 5$.
\end{thm}

\begin{remark*}
  This is a partial verification of Borel's Conjecture
  \ref{c1:conj1.3} and its generalization, Conjecture
  \ref{c2:sec2.3}. The special cases of \ref{c14:thm14.1} when $M$ is
  Riemannian flat or real hyperbolic were proven earlier by Farrell
  and Hsiang in \cite{32} and by Farrell and Jones in \cite{38},
  respectively. The assertion of \ref{c14:thm14.1} is also true when
  $M^m$ is a closed infrasolvmanifold. This was proven in \cite{37} by
  Farrell and Jones extending the result for infranilmanifolds proven
  by Farrell and Hsiang in \cite{32}. Yau showed in \cite{98} that a
  closed infrasolvmanifold $M^m$ supports a non-positively curved
  Riemannian metric only when $\pi_1 (M)$ is viratually abelian;
  hence, neither class of manifolds contains the other.
\end{remark*}

Theorem \ref{c14:thm14.1} is proven by the surgical method for
analyzing 
$$
\mathcal{S}(M^m \times \mathbb{D}^n, \partial)
$$ 
described
in Lecture \ref{c4}. Recall this is a three step method. Step
\ref{c4:step1} is an immediate consequence of Theroem \ref{c7:thm7.1}
in Lecture \ref{c7} since closed (connected) non-positively curved
Riemannian manifolds satisfy conditin $(*)$. (See the Remark at the
end of Lecture \ref{c6}.) And Step \ref{c4:step3} is a direct
consequence of the following result proven by Farrell and Jones in
\cite{42}. 

\begin{thm}\label{c14:thm14.2}
  Let $M$ be a closed (connected) non-positively curved Riemannian
  manifold. Then $Wh (\pi_1 M)=0$.
\end{thm}

\begin{remark*}
  The special cases of \ref{c14:thm14.2} when $M$ is Riemannian flat
  or real hyperbolic were proven earlier by Farrell and Hsiang in
  \cite{29} and by Farrell and Jones in \cite{36},
  respectively. Farrell and Hsiang also showed in \cite{31} that $Wh
  (\pi_1 M)=0$ when $M$ is a closed infrasolv-manifold.
\end{remark*}

Theorem \ref{c14:thm14.2} was discussed for the case when $M$ is real
hyperbolic in a separate course of lectures given by Professor
Raghunathan. We also refer the reader to the expositions
of\pageoriginale the Riemannian flat and real hyperbolic cases of
Theorem \ref{c14:thm14.2} given in chapters 2 and 3 of the book
\cite{40}. Lectures \cite{9} to \cite{13} above give the background material
prerequisite for the foliated control theorem used in proving Theorem
\ref{c14:thm14.2}. 

We hence only discuss Step 2 in this lecture. This is the most
complicated step and the last to be solved. We make simplifying
assumptions in order to make the discussion as transparent as
possible; e.g., we assume throughout that $M^m$ is orientable and
$n=0$. Refer now back to Lecture \ref{c5} and note that
\begin{align*}
  L^s_{m+n} (\pi_1 M^m) & = L_{m+n} (\pi_1 M^m) ~\text{and}\\
  \ob{\mathcal{S}} (M \times \mathbb{D}^n, \partial) & = \mathcal{S} (M
    \times \mathbb{D}^n, \partial)
\end{align*}
since $Wh (\pi_1 M)=0$. These facts together with the periodicity of
the surgery exact sequence (and Theorem \ref{c7:thm7.1}) yield the
following short exact sequence of pointed sets 
$$
0 \to [M^m \times I, \partial; G/\ttop] \xrightarrow{\sigma} L_{m+1}
(\pi_1 M^m) \to \mathcal{S}(M^m) \to 0.
$$

Hence it remains to show that $\sigma$ is an epimorphism. The argument
accomplishing this is modeled after the one used to solve Step 3 in
\cite{36}, \cite{42}; i.e., the method used to show that the only
$h$-cobordism with base $M$ is the cylinder.

The $s$-cobordism theorem was used in that argument. It's surgery
analogue is the algebraic classification of normal cobordisms over $M$
due to Wall \cite{96}. We refer back to Lecture \ref{c4} and expand on
the discussion given there. Given a group $\Gamma$, Wall algebraically
defined a sequence of abelian groups $L_n (\Gamma)$ with $L_{n+4}
(\Gamma)= L_n (\Gamma)$ for all $n \in \mathbb{Z}$. He then showed
that there is a natural bijection between the equivalence classes of
normal cobordisms $W$ over $M^m \times I^{n-1}$ and $L_{m+n} (\pi_1
M^m)$ with the trivial normal cobordism corresponding to 0. Denote
this correspondence by
$$
W \mapsto \omega (W) \in L_{m+n} (\pi_1 M^m).
$$

He also proved the following product formula. Let $N^{4k}$ be a simply
connected closed oriented manifold and $\mathcal{W}= (W, f, \simeq)$
be a normal cobordism over $M^m \times I^{n-1}$. Form a
new\pageoriginale normal cobordism $\mathcal{W} \times N$ over $M^m
\times I^{n-1}\times N^{4k}$ by producting $W$ with $N$; i.e., 
$$
\mathcal{W} \times N = (W \times N, f \times id, \simeq \times id).
$$

Then,
$$
\omega (\mathcal{W} \times N)= \text{Index}~ (N) \omega (\mathcal{W})
$$
under the (algebraic) identification
$$
L_{m+n+ 4k} (\pi_1 (M \times N)) = L_{m+n} (\pi_1 M).
$$

This product formula has following geometric consequence.

\begin{prop}\label{c14:prop14.3}
  Let $K^{4k}$ be a closed oriented simply connected manifold with
  Index $(K)=1$. Let $f : N \to M$ be a homotopy equivalence where $N$
  is also a closed manifold. If
  $$
  f \times id: N \times K \to M \times K
  $$
  is homotopic to a homeomorphism, then $f$ is also homotopic to a
  homeomorphism. 
\end{prop}

The complex projective plane $\mathbb{C}P^2$ is the natural candidate
for $K$ when applying \ref{c14:prop14.3}. It is important for this
purpose to have the following alternate description of
$\mathbb{C}P^2$. Let $C_2$ denote the cyclic group of order 2. It was
a natural action on $S^n \times S^n$ determined by
the involution
$$
(x, y) \mapsto (y, x)
$$
where $x, y \in {S}^n$. Denote the orbit space of this action
by $F_n$; i.e.,
$$
F_n = {S}^n \times {S}^n /C_2.
$$

\begin{lemma}\label{c14:lem14.4}
  $\mathbb{C} P^2 = F_2$.
\end{lemma}

\begin{proof}
  Let $\mathfrak{s}l_2 (\mathbb{C})$ be the set of all $2 \times 2$
  matrices with 
  complex number entries and trace zero. Since $\mathfrak{s}l_2\mathbb{C}$ is a
  3-dimensional $\mathbb{C}$-vector space, $\mathbb{C}P^2$ can be
  identified as the set of\pageoriginale all equivalence classes $[A]$
  of non-zero matrices $A \in \mathfrak{s}l_2 (\mathbb{C})$ where $A$
  is equivalent to $B$ if and only if $A= z B$ for some $z \in
  \mathbb{C}$. The characteristic polynomial of $A \in \mathfrak{s}
  l_2 (\mathbb{C})$ is $\lambda^2+ \det A$. Consequently, $A$ has
  two distinct 1-dimensional eigenspaces if $\det A \neq 0$, and a
  single 1-dimensional eigenspace if $\det A = 0$ and $A \neq
  0$. Also, $A$ and $z A$ have the same eigenspaces provided $z \neq
  0$. These eigenspaces correspond to points in $S^2$ under the
  identification $S^2 = \mathbb{C}P^1$. The assignment
  $$
  [A] \mapsto ~\text{the eigenspaces of}~ A
  $$
  determines a homeomorphism of $\mathbb{C}P^2$ to $F_2$.
\end{proof}

\begin{remark*}
  Theorem \ref{c14:thm14.1} was first proved in the case where $M^m$
  is a hyperbolic 3-dimensional manifold by making use of Lemma
  14.4. It was then realized that the general result for $m$ odd could
  be proven using $F_{m-1}$ once one could handle the technical
  complications arising from the fact that $F_k$ is not a manifold
  when $k > 2$. The following result is used in overcoming these
  complications. It shows that $F_k$ is ``very close'' to being a
  manifold of index equal to 1 when $k$ is even.
\end{remark*}

\begin{lemma}\label{c14:lem14.5}
  Let $n$ be an even positive integer. Then $F_n$ has the following
  properties. 
  \begin{enumerate}
 \renewcommand{\labelenumi}{\rm\theenumi.}
    \item $F_n$ is an orientable $2n$-dimensional
      $\mathbb{Z}[\frac{1}{2}]$-homology manifold.
      \item $F_n$ is simply connected.
        \item 
          $$
          H_i (F_n)=
          \begin{cases}
            \mathbb{Z} & \text{if} ~i =0, n, 2n\\
            \mathbb{Z}_2 & \text{if}~ n < i < 2n~ \text{and}~ i
            ~\text{is even}\\
            0 & \text{otherwise}
          \end{cases}
          $$

        \item 
          $$
          H^i (F_n)=
          \begin{cases}
            \mathbb{Z} & \text{if} ~i =0, n, 2n\\
            \mathbb{Z}_2 & \text{if}~ n +2 < i < 2n~ \text{and}~ i
            ~\text{is odd}\\
            0 & \text{otherwise}
          \end{cases}
          $$
       \item The\pageoriginale cup product pairing
         $$
         H^n (F_n) \otimes H^n (F_n) \to H^{2n} (F_n)
         $$
         is unimodular and its signature is either 1 or $-1$.
  \end{enumerate}
\end{lemma}

\begin{proof}
  There is a natural stratification of $F_n$ consisting two strata $B$
  and $T$. The bottom stratum $B$ consists of all (agreeing) unordered
  pairs $\langle u, v \rangle$ where $u= v$; while the top stratum $T$
  consists of all (disagreeing) pairs $\langle u, v \rangle$ where $u
  \neq v$. Note that $B$ can be idenitified with $S^n$. also real
  projective $n$-space $\mathbb{R}P^n$ can be identified with the set
  of all unorderd pairs $\langle u, -u\rangle$ in $F_n$. It is seen
  that $F_n$ is the union of ``tubular neighborhoods'' of $S^n$ and
  $\mathbb{R}P^n$ intersecting in their boundaries. The first tubular
  neighborhood is a bundle over $S^n$ with fiber the cone on
  $\mathbb{R}P^{n-1}$. The second tubular neighborhood is a bundle
  over $\mathbb{R}P^n$ with fiber $\mathbb{D}^n$. Furthermore, they
  intersect in the total space of the $\mathbb{R} P^{n-1}$-bundle
  associated to the tangent bundle of $S^n$. This description of $F_n$
  can be used to verify \ref{c14:lem14.5}.
\end{proof}

\noindent\textbf{Caveat.} The fundamental class of $B$ represents
\textit{twice} a generator of $H_n (F_n)$. On the other hand, if we
fix a point $y_0 \in S^n$, then the map $x \mapsto \langle x,
y_0\rangle$ is an embedding of $S^n$ in $F_n$ which represents a
generator of $H_n (F_n)$.

Let $f: N \to M$ represent an element in $\mathcal{S}(M)$. Then $f
\times id: N \times S^1 \to M \times S^1$ represents an element in
$\mathcal{S}(M \times S^1)$. This defines a map $\mathcal{S}(M) \mapsto
\mathcal{S}(M \times S^1)$. It is seen that this map is monic by using
Theorem \ref{c14:thm14.2}. Hence it suffices to show that $f \times
id$ is homotopic to a homeomorphism in order to prove Theorem
\ref{c14:thm14.1}. Note that $M \times S^1$ is also non-positively
curved. One consequence of this discussion is that we may assume that
$m = \dim M$ is odd when proving \ref{c14:thm14.1}.

We now formulate a variant of Proposition \ref{c14:prop14.3} which is
used in showing that $f \times id: N \times S^1 \to M \times S^1$ is
homotopic to a homeomorphism. There is a bundle $p : \mathcal{F} M \to
M \times S^1$ whose fiber over a point $(x, \theta) \in M \times S^1$
consists of all unordered pairs\pageoriginale of unit length vectors
$\langle u, v\rangle$ tangent to $M \times S^1$ at $(x, \theta)$
satisfying the following two constraints.
\begin{enumerate}
\item If $u \neq v$, then both $u$ and $v$ are tangent to the level
  surface $M \times \theta$.
  \item If $u = v$, then the projection $\ob{u}$ of $u$ onto $T_\theta
    S^1$ points in the counterclockwise direction (or is 0).
\end{enumerate}

The total space $\mathcal{F}M$ is stratified with three strata:
\begin{align*}
  \mathbb{B} & = \{\langle u, u\rangle \mid \ob{u} =0 \},\\
  \mathbb{A} & = \{\langle u, u\rangle \mid \ob{u} \neq 0 \},\\
  \mathbb{T} & = \{\langle u, u\rangle \mid {u} \neq v \}.
\end{align*}

Note that $\mathbb{B}$ is the bottom stratum and $\mathcal{F} M -
\mathbb{B}$ is the union of the two open sets $A$ and
$\mathbb{T}$. The restriction of $p$ to each stratum is a
sub-bundle. Let $\mathcal{F}_x$, $B_x$, $A_x$ and $T_x$ denote the
fibers of these bundles over $x \in M \times S^1$; i.e., 
\begin{align*}
  \mathcal{F}_x & = p^{-1} (x),\\
  B_x & = p^{-1} (x) \cap \mathbb{B},\\
  A_x & = p^{-1} (x) \cap \mathbb{A},\\
  T_x & = p^{-1} (x) \cap \mathbb{T}.
\end{align*}

Note that $B_x = S^{m-1}$, $A_x = \mathbb{D}^m$, $T_x \cup B_x=
F_{m-1}$ and the bundle $p : \mathbb{B} \to M \times S^1$ is the
pullback of the tangent unit sphere bundle of $M$ under the projection
$M \times S^1 \to M$ onto the first factor.

The space $F_{m-1}$ will play the role of the index one manifold $K$
in our variant of Proposition \ref{c14:prop14.3}. Since it is
unfortunately not a manifold when $m > 3$, we need to introduce the
auxiliary fibers $A_x$. Hence the total fiber is homeomorphic to
$F_{m-1} \cup \mathbb{D}^m$ where the subspace $B$ in $F_{m-1}$ is
identified with $S^{m-1}= \partial \mathbb{D}^m$. Let $\mathcal{F}_f
\to N \times S^1$ denote the pullback of $\mathcal{F} M \to M \times
S^1$ along $f \times id: N \times S^1 \to M \times S^1$ and let
$\hat{f} : \mathcal{F}_f \to \mathcal{F}M$ be the\pageoriginale
induced bundle map. Note that the stratification of $\mathcal{F}M$
induces one on $\mathcal{F}_f$ and that $\hat{f}$ preserves strata. We
say that $\hat{f}$ is \textit{admissibly homotopic to a split map}
provided there exists a homotopy $h_t$, $t \in [0, 1]$, with $h_0=
\hat{f}$ and satisfying the following four conditions.
\begin{enumerate}
\item Each $h_t$ is strata preserving.
\item Over some closed ``tubular neighborhood'' $\mathcal{N}_0$ of
  $\mathbb{B}$ in $\mathbb{B} \cup \mathbb{T}$, each $h_t$ is a
  bundle map; in particular, $h_t$ maps fibers homeomorphically to
  fibers.
  \item There is a large closed ``tubular neighborhood''
    $\mathcal{N}_1$ of $\mathbb{B}$ in $\mathbb{B} \cup \mathbb{T}$
    such that $h_1$ is a homeomorphism over $\mathbb{B} \cup
    \mathbb{T}- \Int \mathcal{N}_1$ and over $\mathbb{B} \cup
    \mathbb{A}$.
    \item Let $\rho : \mathcal{N}_1 \to M \times S^1$ denote the
      composition of the two bundle projections $\mathcal{N}_1 \to
      \mathbb{B}$ and $\mathbb{B} \to M \times S^1$. Then there is a
      triangularion $K$ for $M \times S^1$ such that $h_1$ is
      transverse to $\rho^{-1} (\sigma)$ for each simplex $\sigma$ of
      $K$. Furthermore, $h_1: h_1^{-1} (\rho^{-1} (\sigma)) \to
      \rho^{-1} (\sigma)$ is a homotopy equivalence.
\end{enumerate}

The variant of  Proposition \ref{c14:prop14.3} needed to prove Theorem
\ref{c14:thm14.1} is the following result.

\begin{prop}\label{c14:prop14.6}
  The map $f : N \to M$ is homotopic to a homeomorphism provided
  $\hat{f} : \mathcal{F}_1 \to \mathcal{F} M$ is admissibly homotopic
  to a split map.
\end{prop}

Proposition \ref{c14:prop14.6} is the surgery theory part of the proof
of Theorem \ref{c14:thm14.1} and is proven in [38, Theorem 4.3]. The
geometry of $M$ (in particular, its non-positive curvature) is used to
show that the hypothesis of \ref{c14:prop14.6} is satisfied; i.e.,
that $\hat{f}$ is admissibly homotopic to a split map. This is done in
[43, Proposition 0.4]. The key step there is the construction of a
``focal transfer'' which improves on the ``asymptotic transfer'' used
in \cite{38} to prove \ref{c14:thm14.1} in the special case when $M$
is real hyperbolic. The focal transfer is used to gain control; after
this, applications of both foliated and ordinary control theorems are
made to complete the argument. The reader is referred to \cite{43} and
\cite{38} for details. Also, [40 Chapter 5] contains a detailed sketch
of the proof of Theorem \ref{c14:thm14.1} for the case when $M$ is
real hyperbolic.
