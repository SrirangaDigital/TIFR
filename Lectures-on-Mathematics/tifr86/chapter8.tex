\chapter{Novikov's Conjecture}\label{c8}

The\pageoriginale proof of Theorem \ref{c7:thm7.1} is completed in this
lecture. After that, I formulate a conjecture due to Novikov which is
related to the splitting of the assembly map.

Let $(W, F , \simeq)$ be a simple normal cobordism over $M \times I$
where $\dim\break M \geq 5$. Recall from lecture 7 that we have identified
$\partial _+ W$ with $M \times I$ so that $h|_{M \times 0}= id_{M}$
and $h|_{M \times 1}$ is a homeomorphism, where $h$ is an abbreviation
for $F_+$. Let $\tilde{h}: \tilde{M} \times I \to \tilde{M} \times I$
denote the unique lift satisfying $\tilde{h}|_{\tilde{M} \times
  0}=\text{id}_{\tilde{M}\times 0}$. And let $\ob{h}: \mathbb{D}^{m+1} \to \mathbb{D}^{m+1}$ be the
unique extension of $\tilde{h}$ which exists because $M$ satisfies
condition $(*)$ and hence condition $(**)$ is satisfied. (See Lecture
\ref{c7}.) Note the following properties of $\ob{h}$.

\begin{enumerate}
\setcounter{enumi}{-1}
\item $\ob{h}$ is $\Gamma$-equivariant. (Recall $\Gamma = \pi_1 M$.)
  \item $\ob{h}|_{\partial \mathbb{D}^{m+1}}$ is a self homromorphism
    of $\partial \mathbb{D}^{m+1}$
    \item $\ob{h}|_{\partial _- \mathbb{D}^{m+1}}= id_{\partial_-
      \mathbb{D}^{m+1}}$.
      \item $\ob{h}^{-1} (\partial \mathbb{D}^m)= \partial \mathbb{D}^m$.
\end{enumerate}

Recall, also from Lecture \ref{c7}, that $q: \ob{E} \to M$ is the
$\mathbb{D}^{m+1}$-bundle associated to the universal cover $\tilde{M}
\to M$. Let $E = \ob{E}- \partial_0 \ob{E}$; it is identified with the
total space of the $\tilde{M} \times [0, 1]$-bundle associated to
$\tilde{M} \to M$. (Remember the $\Gamma$-spaces $\tilde{M} \times [0,
1]$ and $\mathbb{D}^{m+1}- \partial \mathbb{D}^m$ were identified in
Lecture \ref{c7}.) So property 0 yields that $id_{\tilde{M}} \times
\ob{h}$ induces a self bundle map of $\ob{E} \to M$ covering
$id_M$. Denote this map by $k: \ob{E} \to \ob{E}$. Fix a triangulation
of $M$. The follwing properties $1', 2', 3'$ of $k$ are consequences
of the corresponding properties 1, 2, 3 of $\ob{h}$. And property $0'$
is obvious since $k$ is a bundle map covering $id_M$.

\begin{itemize}
\item[$0'$.] $k (q^{-1} (\Delta)) \subset q^{-1} (\Delta)$\pageoriginale for each
  closed simplex $\Delta$ of the fixed triangulation of $M$.
\item[$1'$.] $k|_{\partial \ob{E}}$ is a self homeomorphism of
  $\partial \ob{E}$.
\item[$2'$.] $k|_{\partial_- \ob{E}}= id_{\partial_- \ob{E}}$.
\item[$3'$.] $k^{-1} (\partial_0 \ob{E})= \partial_0 \ob{E}$.
\end{itemize}

\begin{lemma}\label{c8:lem8.1}
  There is a homotopy $k_t (0 \leq t \leq 1)$ of $k$ such that $k_0=
  k$, $k_1$ is a homeomorphism, and each map $k_t$ satisfies
  properties $0', 1', 2', 3'$. 
\end{lemma}

\begin{proof}
  The construction of $k_t$ proceeds by induction over the skeleta of
  $M$ via a standard obstruction theory argument. Note the
  following. If $\Delta$ is an $n$-simplex in $M$, then
  $q^{-1}(\Delta)$ is homeomorphic to $\mathbb{D}^{n + m+1}$. Hence,
  the obstructions encountered in extending the homotopy from over the
  $(n-1)$-skeleton to over the  $n$-skeleton lie in $\mathcal{S}
  (\mathbb{D}^{n+ m+1}, \partial)$. But these all vanish because of
  Theorem \ref{c6:thm6.2}; i.e., $|\mathcal{S} (\mathbb{D}^s,
  \partial)|=1$ when $s \geq 5$.
\end{proof}

Let us now continue with the proof of Theorem
\ref{c7:thm7.1}. Consider the universal cover $\tilde{W}$ of $W$ and
let $\tilde{F}: \tilde{W} \to (\tilde{M} \times I) \times [0, 1]$ be
the lift of $F$ such that $\tilde{F}_- = id_{\tilde{M} \times
  I}$. (Recall $F_- = id_{M \times I}$). Since $\tilde{W}$ is a
$\Gamma$-space, we can let $V^{2m+2}\to M$ be the $\tilde{W}$-bundle
associated to the universal cover $\tilde{M} \to M$. Note that $E
\times I \to M$ is the $(\tilde{M} \times I)\times I$-bundle
associated to $\tilde{M} \to M$. Hence $id_{\tilde{M}} \times
\tilde{F}$ induces a bundle map $G : V \to E \times I$ covering
$id_M$. Let $\partial_+ \tilde{W}$, $\partial_- \tilde{W}$,
$\partial_0 \tilde{W}$ be the parts of $\tilde{W}$ lying over
$\partial_+ W$, $\partial_- W$, $\partial_0 W$, respectively. These
are $\Gamma$-invariant subspaces of $\partial \tilde{W}$ and $\partial
\tilde{W}= \partial _+ \tilde{W} \cup \partial_0 \tilde{W} \cup
\partial _- \tilde{W}$. Denote the total spaces of associated
  subbundles by $\partial_+ V$, $\partial_0 V$, $\partial_- V$ 
  respectively. We can identify $G|_{\partial_- V}:\partial_{-}V\to
  E\times 0$
  with $id_E$, and $G|_{\partial_+ V} : \partial_+ V \to E \times 1$
  with $k|_{E}$. Also observe that $G|_{\partial_0 V}: \partial_0 V
  \to \partial E \times I$ is a homeomorphism. We may assume, after a
  special homoropy using Lemma \ref{c8:lem8.1}, that $G|_{\partial_+
    V}$ is a homeomorphism and hence
\begin{equation*}
  G|_{\partial V}: \partial V \to \partial (E \times I) ~\text{is a
    homeomorphism}\tag{1}\label{c8:eq1} 
\end{equation*}
 
Identify\pageoriginale $M$ with the submanifold of $\tilde{M} \times_\Gamma
\tilde{M}$ which is the image of diagonal in $\tilde{M} \times
\tilde{M}$ under the quotient map. (We think of this submanifold as
the ``zero-section'' to the bundle $\tilde{M} \times_\Gamma \tilde{M}
\to M$.) We also, in this way, identify $M \times I^2$ with a
submanifold of $E \times I$ using that $E \times I = (\tilde{M} \times_\Gamma \tilde{M}) \times I^2$. Note that
\begin{equation*}
  \partial (M \times I^2) \subset \partial (E \times
  I). \tag{2}\label{c8:eq2} 
\end{equation*}

Bacause of facts \eqref{c8:eq1} and \eqref{c8:eq2}, we can make $G$
transverse to $M \times I^2$ rel $\partial$. Then, Set
$\mathcal{W}^{m+2}= G^{-1} (M \times I^2)$ and let $\mathcal{F}:
\mathcal{W}^{m+2}\to M \times I^2$ be $G|_{\mathcal{W}^{m+2}}$. Note
that $F|_{\partial \mathcal{W}}: \partial \mathcal{W} \to \partial (M
\times I^2)$ is a homeomorphism which over $(M \times I) \times 0$ is
$id_{M \times I}$. The isomorphism $\simeq$ also naturally induces an
isomorphism $\cong$ of the stable normal bundle of $\mathcal{W}$ to a
bundle over $M \times I^2$. The triple $(\mathcal{W}, \mathcal{F},
\cong)$ is therefore a special normal cobordism over $M \times I$. One
checks that the correspondence
$$
(W, F, \simeq) \mapsto (\mathcal{W}, \mathcal{F}, \cong)
$$ 
sends equivalent simple normal cobordisms to equivalent special normal
cobordisms and hence defines a function
$$
d: L^s_{m+2} (\pi_1 M) \to [M \times I^2, \partial; G/\ttop].
$$
(This checking uses a relative version of Lemma \ref{c8:lem8.1}.)

Now suppose that $(W, F, \cong)$ is a special normal cobordism. Then,
in constructing $(\mathcal{W}, \mathcal{F}, \cong)$, it is unnecessary
to use Lemma \ref{c8:lem8.1}. Hence $\mathcal{W}$ is the ``graph of
$F$''. To be more precise, let $\tilde{f} : \tilde{W} \to \tilde{M}$
denote the composition of $\tilde{F}: \tilde{W} \to \tilde{M} \times
I^2$ with projection onto the first factor of $\tilde{M} \times
I^2$. Then $\mathcal{W}=G^{-1} (M \times I^2)$ is the image of
\textit{graph} $(\tilde{f}) \subset \tilde{M} \times \tilde{W}$ under
the quotient map to $V= \tilde{M} \times_\Gamma \tilde{w}$. Since the
map $x \to (\tilde{f} (x), x)$ is $\Gamma$-equivariant, it determines
a cross-section $\hat{f}$ to the bundle $V \to W$ and $\hat{f}: W \to
\tilde{W}$ is a homeomorphism such that $\mathcal{F}\circ \hat{f}=
F$. In this way, we see that $(W, F, \simeq)$ and $(\mathcal{W},
\mathcal{F}, \cong)$ are isomorphic, hence equivalent, special normal
cobordisms. Therefore, $d \circ \sigma = id$. \hfill Q.E.D.

\setcounter{remark}{0}
\begin{remark}\label{c8:rem1}
  Lemma\pageoriginale \ref{c8:lem8.1} is the proof Theorem
  \ref{c7:thm7.1}. Note we only needed a proper homotopy rel
  $\partial$ of $k|_{E}$ to a homeomorphism. But the obstructions
  encountered in extending such a homotopy from the $(n-1)$-skeleton
  of $M$ to the $n$-skeleton lie in $\pi_{n+1}(G/\ttop)$ and
  these groups are $\mathbb{Z}$, $\mathbb{Z}_2$ or 0 depending on the
  congruence class of $n \mod 4$. But condition $(*)$ enabled us to
  convert the problem to one over $\ob{E}$ where the obstructions were
  automatically 0, since $\mathcal{S} (\mathbb{D}^n, \partial)|= 1$
  for all $n \geq 5$. 
\end{remark}

\begin{remark}\label{c8:rem2} %%% 2
  The simple assembly map $\ob{\sigma} : [M^m \times \mathbb{D}^n,
    \partial ; G/\ttop]\to L_{n+m}^s(\pi_1 M^m)$ is also a split
  injection when $M^m= \Gamma \backslash G/K$, where $G$ is a
  virtually connected Lie group, $K$ is a maximal compact subgroup of
  $G$ and $\Gamma$ is a co-compact, discrete, torsion-free subgroup of
  $G$. This is proven by Farrell and Hsiang in \cite{33},
  \cite{34}. Here it is not known, in general, that $M^m$ satisfies
  condition $(*)$. (It does when $G$ is a semi-simple linear Lie
  group, since $M^m$ then supports a non-positively curved Riemannian
  metric.) A weaker condition than condition $(*)$ does hold, and
  hence a stronger fact is needed than that
  $|\mathcal{S}(\mathbb{D}^n, \partial)|=1$ for $n \geq 5$. The needed
  fact is that $|\mathcal{S} (N^n \times \mathbb{D}^i, \partial)|=1$
  for every closed infranilmanifold $N^n$, provided $n + i\geq
  5$. This fact was proven by Farrell and Hsiang in \cite{32}. 
\end{remark}

Theorem \ref{c7:thm7.1} has the following geometric consequence which
is also proven in \cite{30}.

\begin{coro}\label{c8:coro8.2}
  Let $f: N \to M$ be a homotopy equivalence between closed manifolds
  such that $M$ supports a non-positively curved Riemannian
  metric. Then, $f \times id: N \times \mathbb{R}^3 \to  M \times
  \mathbb{R}^3$ is properly homotopic to a homeomorphism.
\end{coro}

This corollary is a consequence of the surgery exact sequence together
with another fundamental result in surgery theory; namely, the $\pi -
\pi$ theorem, due to C.T.C. Wall (See \cite{30} for details.)

\begin{remark}\label{c8:rem3}%%% 3
  It was conceivable, when \cite{30} was written, that every closed
  aspherical manifold $M^m$ satisfies conditin $(*)$. But Davis
  \cite{21} showed this is not so. His examples where the universal
  cover $\tilde{M}^m$ is not homeomorphic to $\mathbb{R}^m$ contradict
  property 1 of condition $(*)$. (See Remark 1 in Lecture \ref{c3}.)
\end{remark}

On\pageoriginale the other hand, $M^m \times \mathcal{S}^1$ satisfies
property 1 of 
condition $(*)$ whenever $\tilde{M}^m$ is homeomorphic to
$\mathbb{R}^m$. This is seen as follows. Let $\mathbb{Z}$ denote the
additive group of integers. Its natural action by translations on
$\mathbb{R}$ extends to an action on $[- \infty, + \infty)$ where each
  group element fixes $- \infty$. We hence have a product action of
  $\pi_1 (M \times \mathcal{S}^1)= \pi_1 (M) \times \mathbb{Z}$ on
  $\tilde{M}\times [-\infty, + \infty)= \mathbb{R}^m \times [0, +
      \infty)$ which extends to its one point compactification
      $\mathbb{D}^{m+1}$. If we let this be the action posited in
      Definition \ref{c6:defi6.3}, then it satisfies property 1 of
      condition $(*)$; but, not property 2.

Note that the universal cover $X$ of $M^m \times \mathcal{S}^1$ is
homeomorphic to $\mathbb{R}^{m+1}$ when $m \geq 5$. This fact is due
to Newman \cite{51}, since $X$ is contractible and simply connected at
infinity. Consequently, $M^m \times \mathcal{S}^1 \times \mathcal{S}^1
\times \mathcal{S}^1$ satisfies property 1 of condition $(*)$ whenever
$M^m$ is a closed aspherical manifold with $m \geq 4$. Also, if one
examines the proof in \cite{30} of Corollary \ref{c8:coro8.2}, it is
seen that this result remains true when the condition ``M is
non-positively curved''  is replaced by the weaker condition ``$M
\times S^1 \times S^1\times S^1$
satisfies condition $(*)$.''

There is the following question apropos Remark \ref{c8:rem3}

\begin{qus}\label{c8:qus8.3}
  Let $M^m$ be a closed manifold such that $\pi_1 M^m$ is virtually
  solvable. Suppose $M^m$ satisfies condition $(*)$. Does this imply
  that $\pi_1M$ is virtually abelian?
\end{qus}

this question is motivated by Yau's result \cite{98} that such an $M$
cannot support a non-positively curved Riemannian metric which is not
flat.

We end this lecture with a description of a conjecture due to Novikov
and its relation to splitting the assembly map. There are two sets of
rational characteristic classes associated to be a manifold $M$;
namely, its rational Pontryagin classes $p_i (M)$ and its $L$-genera
$L_i (M)$; both of which are elements of $H^{4i} (M\mathbb{Q})$ and
are defined for all integers $i \geq 0$. They contain essentially the
same information about $M$ since the $L$-genera are polynomials in the
Pontryagin classes and vice versa. Novikov \cite{78} proved the
fundamental fact that the rational Pontryagin classes are topological
invariants; i.e.,  if $f: M \to N$ is a homeomorphism between
manifolds, then $f^* (p_i (N))= p_i (M)$. This is, of course,
equivalent to the analogous statement for $L$-genera.

Associate\pageoriginale to any map $f: M \to N$ elements $L_i (f) \in
H^{4i} (M, \mathbb{Q})$ defined by 
$$
L_i (f) = L_i (M) - f^* (L_i (N)).
$$

Then Novikov's theorem is equivalent to saying $L_i (f)=0$, for all $i
\geq 0$, when $f$ is a homeomorphism. Now it is easy to construct
examples where this vanishing fails when $f$ is merely a homotopy
equivalence. But Novikov conjectured a partial vanishing result which
we proceed to formulate.

\begin{defi}\label{c8:defi8.4}
  Given a group $\Gamma$, let $H_\Gamma^i (M, \mathbb{Q})$ denote the
  subset of\break $H^i (M , \mathbb{Q})$ consisting of all elements of the
  form 
$\ob{\phi}^{*}(x)$ where $x\in H^{i}(\Gamma,\mathbb{Q})$ and
  $\phi:\pi_{1}M\to \Gamma$ is a group homolorphism. Here,
$\ob{\phi}: M \to K (\Gamma, 1)$ is the continuous map induced
  by $\phi$. 
\end{defi}

\begin{conj}[Novikov \cite{79}]\label{c8:conj8.5}
   Let $f: M^m \to N^m$ be a homotopy equivalence
  between closed (connected) orientable manifolds. Then the cup
  products $L_i (f) \cup x$ vanish for all $i$ and every $x \in
  H_\Gamma^{m- 4i} (M, \mathbb{Q})$, where $\Gamma$ is an arbitrary
  group.
\end{conj}

If we fix a group $\Gamma$ but allow $f$, $M$ and $N$ to vary, then
the above assertion is called Novikov's conjecture for the group
$\Gamma$. 

\begin{remark}\label{c8:rem4}
  Suppose $H^* (M, \mathbb{Q})= H_{\Gamma}^* (M \mathbb{Q})$. This
  happens, for example, when $M$ is aspherical and $\pi_1 M
  =\Gamma$. With this assumption, the rational Pontryagin classes of
  $M$ are invariants of homotopy equivalence provided Novikov's
  conjecture for $\Gamma$ is true. This assection follows from
  Poincare duality.
\end{remark}

Wall [97, pp. 263-267], expanding on ideas of Novikov \cite{79}, gives
the following relation ship between Novikov's conjecture and the
assembly map.

\begin{thm}\label{c8:thm8.6}
  Let $M^m$ be a compact, orientable, aspherical manifold with $\pi_1
  M^m =\Gamma$. Then, Novikov's conjecture for $\Gamma$ is true if and
  only if the (rational) assembly maps
  $$
  \ob{\sigma}_n : [M^m \times \mathbb{D}^n, \partial ; G/ \ttop]
  \otimes \mathbb{Q} \to L_{n+m}^s (\pi_1 M^m) \otimes \mathbb{Q}
  $$
  are monomorphisms for all integers $n$ satisfying both $n \geq 2$
  and $n+ m \geq 7$.
\end{thm}

\begin{remark}\label{c8:rem5}
  Hence, Theorem \ref{c7:thm7.1} implies that Novikov's conjecture for
  $\Gamma$ is true when $\Gamma = \pi_1 M$ and $M$ is a closed
  (connected) non-positively curved Riemannian manifold.
\end{remark}

However,\pageoriginale this result was proven much earlier and via a different
technique in Mishchenko's seminal paper \cite{73}.

\begin{remark}\label{c8:rem6}
  Although much work has been done verifying Novikov's conjecture for
  a very large class of groups $\Gamma$, it remains open and is still
  an active area of research. See Kasparov's paper \cite{64} for a
  description of the state of the conjecture as of 1988. Additional
  important work on it has been done since that date.
\end{remark}
