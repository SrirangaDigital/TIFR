\chapter{Surgery Exact Sequence}\label{c5}

In\pageoriginale the last lecture, we mentioned an $H$-space
$G/$Top. We now note down some informations about $G/$Top without
actually defining it:
\begin{align*}
  \pi_n (G/\ttop) & = 
  \begin{cases}
    0, & \text{if}~ n ~\text{is odd}\\
    \mathbb{Z}, & \text{if}~ n \equiv 0 \mod 4\\
    \mathbb{Z}_2, & \text{if}~ n \equiv 2 \mod 4
  \end{cases}\\
  G/\ttop~ \oplus \mathbb{Q} & = \prod_{n=0}^\infty K
  (\mathbb{Q}, 4n)\\
  G/\ttop~ \oplus \mathbb{Z}_{(2)} & = \prod_{n=0}^\infty
  K(\mathbb{Z}_{(2)}) \times \prod_{n=0}^\infty K(\mathbb{Z}_2, 4 n
  +2)\\
  G/\ttop~ \oplus \mathbb{Z}_{\text{odd}} & = BO \oplus
  \mathbb{Z}_{\text{odd}}.
\end{align*}

(In these formulas, $\mathbb{Z}_{(2)}= \mathbb{Z} \left[\frac{1}{3},
  \frac{1}{5}, \cdots \right]$ and  $\mathbb{Z}_{\text{odd}} =
\mathbb{Z} \left[\frac{1}{2} \right]$.) It is a consequence of the
second formula above that
$$
[X, A; G/Top] \oplus \mathbb{Q}= \bigoplus_{n=0}^{\infty} H^{4n} (X, A
; \mathbb{Q})
$$
for any pair of topological spaces $(X, A)$.


\noindent \textbf{Historical Remarks.} D. Sullivan \cite{93}
determined tha properties of $G/PL$ in his works on the
Hauptvermutung. His results combined with those of Kirby-Siebenmann
\cite{67} yield the results on $G/$Top mentioned above. The
formulation of the surgery exact sequecne given below is also due to
work of Sullivan and Wall refining the earlier work of Browder
\cite{12} and Novikov \cite{77}. The Kervaire-Milnor paper \cite{65}
on $\mathcal{S}^S(S^n)$ was the prototype for this method of studying
$\mathcal{S}(M)$.  

\begin{defi}\label{c5:defi5.1}
  We next define a variant of $\mathcal{S}(M)$ denoted
  $\ob{\mathcal{S}}(M)$. The 
  underlying set of $\ob{\mathcal{S}}(M)$ is same as that of
  $\mathcal{S}(M)$. But now 
  $[N_1, f_1]$ is said to be equivalent to $[N_2, f_2]$ if there exist
  an $h$-cobordism $W$ between $N_1$ and $N_2$ and a map $F:W \to M
  \times I$ such that $F|_{\partial -W}= f_1$ and $F_{\partial +W} =
  f_2$, where $\partial^- W = N_1$ and $\partial^+ W = N_2$.
\end{defi}

Note\pageoriginale that $\mathcal{S}(M) = \ob{\mathcal{S}}(M)$ when
$Wh(\pi_1 (M))=0$ and 
$\dim M\geq 5$. A set $\ob{\mathcal{S}}(M, \partial M)$ can be defined
similarly 
when $M$ is a compact manifold with boundary. (The notation
$\ob{\mathcal{S}}(M,\partial M)$ is sometimes abbreviated to
$\ob{\mathcal{S}}(M, \partial)$ and 
likewise $[M, \partial M; G/\ttop]$ to $[M, \partial;
  G/\ttop]$.) 

\noindent \textbf{Surgery Exact Sequence.} Let $M^m$ be a compact
connected manifold with non-empty boundary. For any non-negarive
integer $n$, we can form a new manifold $M^m \times \mathbb{D}^n$,
where $\mathbb{D}^n$ is the closed $n$-ball. Then there is long exact
sequence of pointed sets:
\begin{align*}
  \cdots \mathop{\longrightarrow}^{\pi} & \ob{\mathcal{S}} (M \times
    \mathbb{D}^n, \partial) \mathop{\longrightarrow}^{\omega} [M \times
      \mathbb{D}^n, \partial; G/\ttop]
    \mathop{\longrightarrow}^{\sigma} L_{m+n}(\pi_1 M) \longrightarrow \\
  \cdots \longrightarrow & \ob{\mathcal{S}} (m \times \mathbb{D}^1, \partial)
    \mathop{\longrightarrow}^{\omega} [M \times \mathbb{D}^1,
      \partial; G/\ttop] \mathop{\longrightarrow}^{\sigma}
    L_{m+1} (\pi_1 M) \mathop{\longrightarrow}^{\tau} \\
    & \ob{\mathcal{S}}(M,
    \partial)\mathop{\longrightarrow}^{\omega} [M, \partial; G/\ttop]
    \mathop{\longrightarrow}^{\sigma} L_m (\pi_1 M).  
\end{align*}

The maps $\sigma$ (when $n \geq 1$) and $\tau$ can be defined using
the identifications mentioned earlier. Recall that $L_{m+n} (\pi_1 M)=
L_{m+ n} (\pi_1 (M \times \mathbb{D}^{n-1}))$ is the set of
equivalence classes of normal cobordisms on$M \times \mathbb{D}^{n-1}$
and that 
$$
[M \times \mathbb{D}^n, \partial; G/\ttop] = [M \times
  \mathbb{D}^{n-1} \times [0, 1], \partial; G/\ttop]
$$
consists of the equivalence classes of special normal cobordisms on $M
\times \mathbb{D}^{n-1}$. Then, $\sigma$ is the map which forgets the
special structure; while, $\tau$ sends a normal cobordism $W$ to its
$top\, \partial^+ W$.

The maps $\omega$, when $n \geq 1$, similarly have a natural geometric
description. We illustrate this when $n=1$. Let $(W, F)$ represent an
element $x$ in $\ob{\mathcal{S}}(M \times [0, 1], \partial)$. Then, $W$ is an
$h$-cobordism between $\partial^- W$ and $\partial^+ W$. Furthermore,
the restricutions $F|_{\partial^- W}: \partial^- W \to M \times 0$ and
$F|_{\partial^+ W}: \partial^+ W \to M \times 1$ are both
homeomorphisms. If the first of these two homeomorphisms is $id_M$,
then $(W, F, \simeq)$ is also a special normal cobordism and,
considered as such, is $\omega (x)$. A bundle isomorphism $\simeq$
with domain $N (W)$ is determined since $F$ is a homotopy
equivalence. But it is easy to see that $(W, F)$ is equivalent in
$\ob{\mathcal{S}}(M \times [0, 1], \partial)$ to an object $(W', F')$ such that
$\partial^- W'=M$ and $F'|_{\partial^- W'} = id_M$. In this way,
$\omega(x)$ is defined.

With\pageoriginale this above description of the maps $\sigma$, $\tau$ and $\omega$,
it is not difficult to show that the surgery sequence (above the last
few terms; i.e., above those near $L_m(\pi_1 M)$) is exact. The last
few terms pose extra difficult. But the following important
periodicity result, due to Kirby-Siebenmann \cite{67}, allow us to
avoid worrying about these difficulties when considering Borel's
Conjecture.

\begin{thm}[Kirby-Siebenmann]\label{c5:thm5.2}
  Let $M$ be a compact, connected manifold with $\dim M \geq 5$. If
  $\partial M \neq \phi$, then 
  $$
  |S (M , \partial|= |S(M \times \mathbb{D}^4, \partial)|.
  $$
  If $\partial M = \phi$, then
  $$
  |S (M)| \leq |S(M \times \mathbb{D}^4, \partial)|.
  $$
\end{thm}

\begin{defi}\label{c5:defi5.3}
  The map $\sigma$ in the surgery seqences is called either the
  surgery map or the assembly map.
\end{defi}

\setcounter{remark}{0}
\begin{remark}
  When $n \geq 1$, $\sigma$ is a group homomorphism.
\end{remark}

\begin{remark}
  If $\partial M= \phi$, then the surgery sequence still exists and
  remains exact, provided we omit its last term.
\end{remark}
