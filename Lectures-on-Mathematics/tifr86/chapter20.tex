\chapter{Final Remarks}\label{c20}

This\pageoriginale final lecture contains a potpourri of some of the
more recent results related to the topics discussed in my previous
lectures.

I start by mentioning a result due to Ontaneda \cite{83} giving
counterexamples to ``$PL$-rigidity'' for closed negatively curved
manifolds.

\begin{thm}[Ontaneda]\label{c20:thm20.1}
Given $\epsilon>0$, there exist a pair of $6$-dimen\-sional, closed,
negatively curved Riemannian manifolds $M$ and $N$ with the following
four properties.
\begin{enumerate}
\item $M$ is real hyperbolic.

\item The sectional curvatures of $N$ are all contained in the
  interval $(-1-\epsilon,-1+\epsilon)$.

\item $M$ and $N$ are homeomorphic.

\item But, $M$ and $N$ are not piecewise linearly homeomorphic.
\end{enumerate}
\end{thm}

\begin{remark*}
Since $M$ and $N$ are smooth manifolds, they can be piecewise smoothly
triangulated. Property 4 of Theorem \ref{c20:thm20.1} means that the
underlying simplicial complex of any such triangulation of $M$ must be
different from that of any such triangulation of $N$. In particular,
$M$ and $N$ are {\em not} diffeomrophic. On the other hand, the
counterexamples to smooth-rigidity for negatively curved manifolds
given by Theorems 17.4 and \ref{c18:thm18.1} are
piecewise linearly homeomorphic.
\end{remark*}

Ontaneda's construction builds on the ideas used in making the\break
counterexamples given by Theorem 17.4; but employs the
Kirby-Sie\-benmann obstruction to $PL$-equivalence, which lies in
$H^{3}(M,\mathbb{Z}_{2})$, instead of exotic spheres. The paper of
Millson and Raghunathan \cite{71} provides the real hyperbolic
manifolds $M$ with rich enough cohomology structure to carry out the
argument.

There are also counterexamples to smooth rigidity for finite volume
but non-compact negatively curved Riemannian manifolds. These were
constructed by Farrell and Jones in \cite{44}. Here is employed a
different technique to change the differential structure than that
used to prove Theorems 17.4 and \ref{c18:thm18.1}. A new
technique is necessary since connected summing\pageoriginale with an
exotic sphere {\em never} changes the smooth structure on a
non-compact, connected manifold $M^{m}$, where $m\geq 5$. The
technique used in \cite{44} is a type of ``Dehn surgery'' along a
closed geodesic in $M^{m}$ using an $m-1$ dimensional exotic sphere.

There are also examples constructed by Farrell and Jones in \cite{47}
of complete real hyperbolic manifolds $M$ with finite volume where
some exotic smoothing of $M$ {\em cannot} support a complete, finite
volume, pin\-ched negatively curved Riemannian metric. However, it is an
open question whether any such examples exist where $M$ is closed. See
\cite{3} for a related result.

These counterexamples taken together with the discussions in Lectures
\ref{c17}-\ref{c19} motivate a search for extra geometric conditions
which will yield smooth or $PL$-rigidity when both $M$ and $N$ are
non-positively curved. This problem was addressed by Farrell and Jones
in \cite{49}. I now describe the result obtained in \cite{49}; but,
make the added assumption that both $M$ and $N$ are negatively curved
to sharpen the discussion. We also assume that $M$ and $N$ are closed
with $\dim M\geq 5$ and $\alpha:\pi_{1}(M)\to \pi_{1}(N)$ is an
isomorphism. Let $\tilde{M}(\infty)$, $\tilde{N}(\infty)$ be the
Eberlein-O'Neill visibility spheres of the universal covers
$\tilde{M}$, $\tilde{N}$ of $M$ and $N$, respectively; cf.\@
\cite{26}. The fundamental groups $\pi_{1}(M)$ and $\pi_{1}(N)$ act
naturally on $\tilde{M}(\infty)$ and $\tilde{N}(\infty)$,
respectively. And Mostow showed (implicitly; cf.\@ \cite{75}) there is
a unique $\alpha$-equivariant homeomorphism
$\alpha_{\infty}:\tilde{M}(\infty)\to \tilde{N}(\infty)$. There is
also a natural map
$$
F:S\tilde{M}\to \tilde{M}(\infty)
$$
defined by $F(v)=\gamma_{v}(+\infty)$. Here, $S\tilde{M}\to \tilde{M}$
denotes the tangent sphere bundle of $\tilde{M}$;  $\gamma_{v}$ is the
unique geodesic in $\tilde{M}$ satisfying $\gamma_{v}(0)=v$ and
$\gamma_{v}(+\infty)$ is the asymptoty class containing the geodesic
ray $\{\gamma_{v}(t)\mid t\geq 0\}$. (Recall from \cite{26} that the
asymptoty classes of geodesic rays in $\tilde{M}$ are the points of
$\tilde{M}(\infty)$.) The map $F$ restricted to any fiber of
$S\tilde{M}\to \tilde{M}$ is a homeomorphism onto
$\tilde{M}(\infty)$. And the visibility sphere $\tilde{M}(\infty)$ is
said to be {\em naturally} $C^{1}$ provided it has a $C^{1}$-manifold
structure such that $F$ is a $C^{1}$-map and when restricted to each
fiber of $S\tilde{M}\to \tilde{M}$ is a $C^{1}$-diffeomorphism.

\begin{remarks*}
If\pageoriginale $\tilde{M}(\infty)$ is naturally $C^{1}$, then this
$C^{1}$ structure is unique and the action of $\pi_{1}(M)$ on
$\tilde{M}(\infty)$ is via $C^{1}$-diffeomorphisms. Furthermore,
$\tilde{M}(\infty)$ is naturally $C^{1}$ when $M$ is {\em strictly
  $\frac{1}{4}$-pinched}; i.e., when there exists a positive real
number $b$ such that all the sectional curvatures of $M$ lie in the
open interval $(-b,-b/4)$. This second comment is a consequence of the
fundamental result of Hirsch and Pugh \cite{61}.
\end{remarks*}

The rigidity result in \cite{49} can now be stated as follows.

\begin{thm}\label{c20:thm20.2}
Assume, in addition to the above assumption, that both
$\tilde{M}(\infty)$ and $\tilde{N}(\infty)$ are naturally $C^{1}$ and
that $\alpha_{\infty}$ is a $C^{1}$-diffeomrophism. Then, $\alpha$ is
induced by a piecewise linear homeomorphism. In fact, there is a
smooth diffeomrophism
$$
f:M\# s\Sigma\to N
$$
inducing $\alpha$. Here, $\Sigma$ is a homotopy sphere and $s$ denotes
the Euler characteristic of $M$.
\end{thm}

\begin{remark*}
If $\dim M$ is odd, then $s=0$. And hence $f:M\to N$ is a
diffeomorphism; i.e., smooth-rigidity holds in this case.
\end{remark*}

Unfortunately, the condition in \ref{c20:thm20.2} that
$\alpha_{\infty}$ is a $C^{1}$-diffeomorphism is quite strong. In
particular, it is {\em not} a necessary condition for smooth
rigidity. Mostow showed in his original work on strong rigidity
\cite{74} that if $\alpha_{\infty}$ is a $C^{1}$-diffeomorphism and
$M$ and $N$ are both real hyperbolic manifolds, then $\alpha$ is
induced by an isometry; even when $\dim M=2!$ In fact, we do not know
an example where the conclusions of \ref{c20:thm20.2} cannot be
replaced by the stronger statement ``$\alpha$ is induced by an
isometry after multiplying the metric on $M$ by a suitable constant.''
Hopefully, weaker conditions ``at $\infty$'' will be found which imply
smooth (or $PL$)-rigidity.

Let $M$ be a closed and connected Riemannian manifold. Then there is a
natural sequence of groups and homomorphisms
$$
\Iso(M)\to \Diff(M)\to \tTop(M)\to \Out(\pi_{1}M);
$$
consisting\pageoriginale of all self-isometries, diffeomorphisms,
homeomorphisms of $M$ and outer auto-morphisms of $\pi_{1}(M)$,
respectively. An immediate consequence of Mostow's Strong Rigidity
Theorem; cf.\@ \cite{75}, is that the composition of these
homomorphisms maps $\Iso(M)$ onto $\Out(\pi_{1}M)$ when $M$ is a
non-positively curved locally symmetric space whose universal cover
has no 1 or 2 dimensional metric factor. Likewise, it is an immediate
consequence of Theorem \ref{c14:thm14.1} that $\tTop(M)\to
\Out(\pi_{1}M)$ is an epimorphism when $M$ is non-positively curved
and $\dim M\neq 3,4$. On the other hand, the examples of Theorem
\ref{c17:thm17.5} were used by Farrell and Jones in \cite{41} to show
that the homomorphism $\Diff(M)\to \Out(\pi_{1}M)$ is {\em not}, in
general, an epimorphism under these same assumptions. However, it had
been hoped that this map was always epimorphic; cf.\@ \cite{86}. One
reason for this optimism was the following fundamental result due to
Eells and Sampson \cite{28}.

\begin{thm}\label{c20:thm20.3}
Let $f:M\to N$ be a homotopy equivalence where $N$ is a closed
non-positively curved Riemannian manifold. Then $f$ is homotopic to a
harmonic map.
\end{thm}

\begin{remark*}
A harmonic map is a smooth map which is a critical point of the energy
functional. And the energy of $f:M\to N$ is essentially the integral
over all $v\in SM$ of $\frac{1}{2}|dr(v)|^{2}$.
\end{remark*}

It was hoped that every harmonic homotopy equivalence between closed
non-positively curved Riemannian manifolds was diffeomor\-phi\-sm, or at
least a homeomorphism; cf.\@ \cite[Problems 5.4 and 5.5]{27}. Theorem
\ref{c17:thm17.5} showed that the diffeomorphism conclusion is false,
in general. But the homeomorphism conclusion is still an open
problem. The paper \cite{50} of Farrell and Jones is an attempt to
address this problem. Among other things, there are constructed in
\cite{50} examples of harmonic homotopy equivalences $f:M\to N$ which
are {\em not} homeomorphisms even though $N$ is negatively curved. But
in these examples it is unknown if $M$ can also be non-positively
curved. The examples are based on a topological result due to Hatcher
and Igusa, cf.\@ \cite[\S 4]{57}.

I finish these lectures with some additional comments about Whitehead
groups. Recall Theorem \ref{c14:thm14.2} showed that $\Wh(\Gamma)=0$
for a large class of torsion-free groups $\Gamma$; namely, for
$\Gamma=\pi_{1}(M)$\pageoriginale where $M$ is a closed (connected)
non-positively curved Riemannian manifold. Much is also known about
$\Wh(F)$ where $F$ is a finite group, cf.\@ \cite{4}; in particular,
it is finitely generated and its rank is $r-q$, where $r$ is the
number of irreducible real representations of $F$ and $q$ is the
number of irreducible rational representations. On the other hand,
there are many examples due to M.P.\@ Murthy (cf.\@ \cite{4}) of
finitely generated abelian groups $\Gamma$ such that $\Wh(\Gamma)$ is
{\em not} finitely generated. Recently, Farrell and Jones in \cite{45}
have given a method for ``computing'' $\Wh(\Gamma)$ in terms of
$\Wh(S)$, $\tilde{K}_{0}(\mathbb{Z}S)$ and $K_{-n}(\mathbb{Z}S)(n\geq
1)$, where $S$ varies over the class of all virtually cyclic subgroups
of $\Gamma$. This method is valid for any subgroup $\Gamma$ of a
uniform lattice in a Lie group $G$ (where $G$ has only finitely many
connected components). Their method is additionally conjectured in
\cite{45} to be valid for {\em all} groups $\Gamma$. A similar
conjecture is also made in \cite{45} for a method of calculating the
surgery $L$-groups of integral group rings.

\begin{remarks*}
A group is {\em virtually cyclic} if it contains a cyclic subgroup
with finite index; e.g., finite groups, the infinite cyclic group, and
the infinite dihedral group are virtually cyclic. The {\em lower
  $K$-groups} of a ring $R$, denoted by $K_{-n}(R)$ where $n$ is any
integer $\geq 1$, were defined by Bass in \cite{4}.
\end{remarks*}

W.-C.\@ Hsiang conjectured in \cite{62} that
$K_{-n}(\mathbb{Z}\Gamma)=0$ for every group $\Gamma$ and every
integer $n\geq 2$. Farrell and Jones in \cite{48} verified Hsiang's
conjecture for any subgroup $\Gamma$ of a uniform lattice in a Lie
group with finitely many connected components. They did this by
directly verifying Hsiang's conjecture for all infinite virtually
cyclic groups and then applying the main result of \cite{45}. Carter
in \cite{17} had previously verified Hsiang's conjecture for all
finite groups.


