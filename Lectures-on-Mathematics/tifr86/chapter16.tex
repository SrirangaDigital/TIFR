\chapter{Exotic Smoothings}\label{c16}

\setcounter{pageoriginal}{67}
This\pageoriginale lecture is concerned with the problem of detecing
non-diffeomor\-phic smooth structures on the same topological manifold
$M$. Let $(N,f)$ be a pair consisting of a smooth manifold $N$
together with a homeomorphism $f:N\to M$. Two such pairs
$(N_{1},f_{1})$ and $(N_{2},f_{2})$ are equivalent provided there
exists a diffeomorphism $g:N_{1}\to N_{2}$ such that the composition
$f_{2}\circ g$ is {\em topologically concordant} (a.k.a.\@
topologically pseudoisotopic) to $f_{1}$; i.e., there exists a
homeomorphism $F:N_{1}\times [0,1]\to M$ such that
$$
F|_{N_{1}\times 0}=f_{1}\quad\text{and}\quad F|_{N_{2}\times 0}=f_{2}.
$$

Note that $F$ is {\em not} required to be level preserving; i.e.,
$F(N_{1}\times t)$ needn't be contained in $M\times t$. (If $F$ is
additionally level preserving, then it is a {\em topological
  isotopy}.) The set of all such equivalence classes is denoted
$\mathcal{C}(M)$ and an equivalence class is called a smooth structure
on $M$.

The key to analyzing $\mathcal{C}(M)$ is the following result due to
Kirby and Siebenmann \cite[p. 194]{67}.

\begin{thm}\label{c16:thm16.1}
There exists a connected $H$-space $\tTop/O$ such that there is a
bijection between $\mathcal{C}(M)$ and $[M;\tTop/O]$ for any smooth
manifold $M$ with $\dim M\geq 5$. Furthermore, the equivalence class
of $(M,\id_{M})$ corresponds to the homotopy class of the constant map
under this bijection.
\end{thm}

Assume $n\geq 5$ and recall that $\theta_{n}$ is
$\mathcal{S}^{s}(S^{n})$. Consider the obvious forgetting information map
$$
\mathcal{C}(S^{n})\to \mathcal{S}^{s}(S^{n}).
$$

This map is a bijection which can be seen by using the fact that both
$|\mathcal{S}(S^{n})|=1$ and
$|\mathcal{S}(S^{n}\times[0,1],\partial)|=1$. We can hence identify
$\mathcal{C}(S^{n})$ with the set of all equivalence classes of
oriented $n$-dimensional homotopy spheres $\Sigma$. Here two oriented
homotopy spheres $\Sigma_{1}$ and $\Sigma_{2}$ are equivalent provided
they are orientation preservingly diffeomorphic. Also
the\pageoriginale abelian group structure on $\theta_{n}$ given by
connected sum agrees with the one given by Theorem \ref{c16:thm16.1}
via the identification $\mathcal{C}(S^{n})=\pi_{n}(\tTop/O)$. 

We can more generally show that the natural map $\mathcal{C}(M)\to
\mathcal{S}^{s}(M)$ is a bijection for any closed smooth manifold $M$
such that both $\mathcal{S}(M)$ and $\mathcal{S}(M\times
[0,1],\partial)$ have cardinality 1 (and $\dim M\geq 5$). To do this,
one notices that $|\mathcal{S}(M\times [0,1],\partial)|=1$ implies
that any self-diffeomorphism of $M$ which is homotopic to $\id_{M}$
is, in fact, topologically pseudoisotopic to $\id_{M}$. Combining this
observation with Theorem \ref{c14:thm14.1}, we see, in particular,
that the natural map $\mathcal{C}(M^{m})\to \mathcal{S}^{s}(M^{m})$ is
a bijection for every closed non-positively curved manifold $M^{m}$
(with $m\geq 5$).

Recall now from Lecture \ref{c2} how $\pi_{0}\mathcal{E}(M^{m})$ acts
on $\mathcal{S}^{s}(M^{m})$. One sees immediately from this
description how Mostow's Rigidity Theorem, cf.\@ \cite{75}, implies
that the concordance class of $(M^{m},\id_{M^{m}})$ is a fixed point
of this action whenever $M^{m}$ is a non-positively curved locally
symmetric space such that its universal cover has neither a one nor a
two dimensional metric factor. This is, in particular, the case when
$M^{m}$ is negatively curved. And Bieberbach's Rigidity Theorem, cf.\@
\cite{6}, shows that this is also the case when $M^{m}$ is a flat
Riemannian manifold. Stringing the above remarks together yields the
following consequence of \ref{c16:thm16.1}.

\begin{coro}\label{c16:coro16.2}
Let $M^{m}$ be a closed Riemannian manifold (with $m\geq 5$) which is
a locally symmetric space whose sectional curvatures are either
identically zero or all negative. Let $(N^{m},f)$ be a smoothing of
$M^{m}$. If $N^{m}$ is diffeomorphic to $M^{m}$, then $(N^{m},f)$ and
$(M^{m},\id_{M^{m}})$ represent the same element in
$\mathcal{C}(M^{m})$; i.e., they are topologically concordant.
\end{coro}

We will next apply \ref{c16:coro16.2} to the problem of determining
when connected sum with a homotopy sphere $\Sigma$ changes the
differential structure on a smooth oriented (connected) manifold
$M$. Start by noting that the homeomorphism $M\# \Sigma$ to $M$ which
is the inclusion map outside of $\Sigma$ is well defined up to
topological concordance. We will denote the class in $\mathcal{C}(M)$
of $M\# \Sigma$ equipped with this homeomorphism by $[M\#
  \Sigma]$. (Note that $[M^{m}\# S^{m}]$ is\pageoriginale the class of
$(M^{m},\id_{M^{m}})$.) Let $f_{M}:M^{m}\to S^{m}$ be a degree-one map
and note that $f_{M}$ is well-defined up to homotopy. Composition with
$f_{M}$ defines a homeomorphism.
$$
f^{*}_{M}:[S^{m},\tTop/O]\to [M^{m},\tTop/O].
$$

And in terms of the identifications
$$
\theta_{m}=[S^{m},\tTop/O]\quad\text{and}\quad
\mathcal{C}(M^{m})=[M^{m},\tTop/O] 
$$
given by \ref{c16:thm16.1}, $f^{*}_{M}$ becomes $[\Sigma^{m}]\mapsto
[M^{m}\# \Sigma^{m}]$. 

Recall that a smooth manifold is {\em stably parallelizable} (a.k.a.\@
a $\pi$-{\em manifold}) if its tangent bundle is stably trivial. We
need the following result due to Browder \cite{13} and Brumfiel
\cite{14}.

\begin{lemma}\label{c16:lem16.3}
Assume that $M^{m}$ is an oriented closed (connected) smooth manifold
which is stably parallelizable and that $m\geq 5$. Then
$f^{*}_{M}:\theta_{m}\to \mathcal{C}(M^{m})$ is monic.
\end{lemma}

\begin{proof}
Since $X\mapsto [X,\tTop/O]$ is a homotopy functor on the category of
topological spaces, \ref{c16:lem16.3} would follow immediately if
$f_{M}:M^{m}\to S^{m}$ is homotopically split. That is, if there
exists a map $g:S^{m}\to M^{m}$ such that $f_{M}\circ g$ is homotopic
to $\id_{S^{m}}$. Unfortunately, $f_{M}$ is only homotopically split
when $M$ is a homotopy sphere. But we can use the fact that $M$ is
stably parallelizable to always stably split $f_{M}$ up to homotopy;
i.e., to show that the $(m+1)$-fold suspension
$$
\Sigma^{m+1}(f_{M}):\Sigma^{m+1}M^{m}\to S^{2m+1}
$$
of $f_{M}$ is homotopically split. This is done as follows. Note first
that $M^{m}\times \mathbb{D}^{m+1}$ can be identified with a
codimension-0 smooth submanifold of $S^{2m+1}$ by using the Whitney
embedding theorem together with the fact that $M$ is stably
parallelizable. Let $\ast$ be a base point in $M$. Then dual to the
inclusion. 
$$
\mathbb{M}^{m}\times \mathbb{D}^{m+1}\subseteq S^{2m+1}
$$
is\pageoriginale a quotient map $\phi:S^{2m+1}\to \Sigma^{m+1}M^{m}$
realizing the $(m+1)$-fold reduced suspension $\Sigma^{m+1}M^{m}$ of
$M^{m}$ as a quotient space of $S^{2m+1}$. Namely, $\phi$ collapses
everything outside of $M^{m}\times \Int(\mathbb{D}^{m+1})$ together
with $\ast\times \mathbb{D}^{m+1}$ to the base point of
$\Sigma^{m+1}M^{m}$, and is a bijection between the remaining
points. And it is easy to see that the composition
$\Sigma^{m+1}(f_{M})\circ \phi$ is homotopic to $\id_{S^{m}}$; i.e.,
$\Sigma^{m+1}(f_{M})$ is homotopically split.

But this is enough to show that $f^{*}_{M}$ is monic since $\tTop/O$
is an $\infty$-loop space \cite{8}; in particular, there exists a
topological space $Y$ such that $\Omega^{m+1}(Y)=\tTop/O$. This fact
is used to identify the functor
$$
X\mapsto [X,\tTop/O]=[X,\Omega^{m+1}(Y)]
$$
with the functor $X\mapsto [\Sigma^{m+1}X,Y]$. Consequently,
$$
f^{*}_{M}:[S^{m},\tTop/O]\to [M^{m},\tTop/O]
$$
is identified with 
$$
(\Sigma^{m+1}(f_{M}))^{*}:[S^{2m+1},Y]\to [\Sigma^{m+1}M^{m},Y].
$$

But, this last homomorphism is monic since $\Sigma^{m+1}(f_{M})$ is
homotopically split.
\end{proof}

A homotopy $m$-sphere is called {\em exotic} if it is not
diffeomorphic to $S^{m}$. The following result is an immediate
consequence of \ref{c16:coro16.2} and \ref{c16:lem16.3}; it will be
used to construct exotic smoothings of some symmetric spaces.

\begin{coro}\label{c16:coro16.4}
Let $M^{m}$ be a closed, oriented (connected) stably parallelizable
Riemannian locally symmetric space (with $m\geq 5$) whose sectional
curvatures are either identically zero or all negative (e.g., $M^{m}$
could be an $m$-torus). Let $\Sigma^{m}$ be an exotic homotopy sphere,
then $M^{m}\# \Sigma^{m}$ is not diffeomorphic to $M^{m}$.
\end{coro}

\begin{remark*}
Recall we asserted in Lecture \ref{c15} that $T^{n}\# \Sigma^{n}$ is
{\em not} diffeomorphic to any infranilmanifold when $\Sigma^{n}$ is
exotic (and $n\geq 5$). We now give a more complete argument for this
fact. Corollary \ref{c16:coro16.4} shows that $T^{n}\# \Sigma^{n}$ is
{\em not} diffeomrophic to $T^{n}$. Also Malcev's
Rigidity\pageoriginale Theorem, cf.\@ \cite{70}, shows that any closed
infranilmanifold with abelian fundamental group must be Riemannian
flat. And finally Bieberbach's Rigidity Theorem shows that any such
manifold is diffeomorphic to a torus.
\end{remark*}


