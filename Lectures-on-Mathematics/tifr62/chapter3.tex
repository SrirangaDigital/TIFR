
\chapter{Castelnuovo's Theorems}\label{chap3}

\section{}\label{chap3:sec1} IN\pageoriginale THIS CHAPTER we will be proving 
two theorems of Castelnuovo. In the proof we will use a theorem of Bertini
which we will not prove (\cf for example Hartshorne's book).

\begin{THM*}
{\bf (Bertini)} Let $C$ be a smooth curve embedded in
$\mathbb{P}^r,r\geq 3$. Also assume that $C$ is not contained in any
hyperplane in $\mathbb{P}^r$. Then for generic hyperplane $H$ of
$\mathbb{P}^r,H\cap C$ is reduced and any $r$ points of $H\cap C$
generate $H$ as a linear space.
\end{THM*}

\begin{REM*}
If $C$ is contained in $\mathbb{P}^3$, where $C$ is a smooth integral
curve, Bertini's theorem implies that, there exists a point in
$\mathbb{P}^3$, such that the projection of $C$ from this point onto a
curve $C'$ in $\mathbb{P}^2$ is a birational isomorphism and $C'$ has
only ordinary double points as singularities.
\end{REM*}

We will now prove a lemma, which will be the lemma for our proofs of
Castelnuovo's theorems. 
\begin{lem}\label{chap3:lem1}
Let $S$ be a set of $(r-1)k+1$ points in $\mathbb{P}^{r-1}$ such that
any $r$ points of $S$ generate $\mathbb{P}^{r-1}$. Then, the canonical
map $H^\circ(\mathbb{P}^{r-1},\break O_{\mathbb{P}^{r-1}}(k))\longrightarrow
H^\circ(S,O_S(k)$ is surjective.
\end{lem}

\begin{proof}
The statement of the lemma is equivalent to proving that for any $P$
in $S$, there exists a homogeneous polynomial $F$ of degree $k$ (in
the $r$ variables corresponding to the homogeneous co-ordinates of
$\mathbb{P}^{r-1}$) such that $F$ vanishes at every point of $S$ other
than $P$ and $F(P)$ is not equal to zero. By the hypothesis, the
subspace generated by any $(r-1)$ points in $S$ is a
hyperplane. We\pageoriginale partition $S-P$ into $k$ parts consisting
of $(r-1)$ points each. So there exists homogeneous linear polynomials
$(F_i)1\leq i\leq k$, which define these $k$ hyperplanes. So
$F_i(P)\neq 0$ for every $i$ and for any $Q\in S-\{P\}$, there exists
some $F_j$ such that $F_j(Q)=0$. Taking $F=\prod\limits_{i=1}^kF_i$, we
see that our requirements are met.
\end{proof}

\begin{Cor}\label{chap3:cor1}
If $S'\subset S,S$ as above, then the canonical map\break $H^\circ
(\mathbb{P}^{r-1},O_{\mathbb{P}^{r-1}}(k))\longrightarrow H^\circ
(S',O_{S'})$ is surjective.
\end{Cor}

\begin{Cor}\label{chap3:cor2}
Let $C$ be a smooth curve of degree $d$ embedded in $\mathbb{P}^r$ and
$H$ a hyperplane as in Bertini's theorem (\ie $H\cap C$ is a finite
set of reduced points and any $r$ points of $H\cap C$ generate
$H$). Define 
$$
t_n=\rank (H^\circ(\mathbb{P}^r, O_{\mathbb{P}^r}(n)) \longrightarrow
H^\circ(C\cap H,O_{C\cap H}(n)).
$$

Then $t_n\geq\inf(d,(r-1)n+1)$. 
\end{Cor}

\begin{proof}
If $d\leq(r-1)n+1$, we take the $d$ points of $H\cap C$ and choose
$((r-1)n+1-d)$ points on $H$ such that any $r$ points from these
$((r-1)n+1)$ points generate $H$. (This is always possible, since the
$d$ points of $H\cap C$ have this property.) Denote this set by S.S
satisfies the conditions of the lemma. Since $C\cap H\subset S$, by
Corollary \ref{chap3:lem1}, we get $t_n\geq d$.

If $d>(r-1)n+1$ and if $S$ is a subset of $C\cap H$, consisting of
$((r-1)n+1)$ points, the image of
$H^\circ(\mathbb{P}^r,O_{\mathbb{P}^r}(n))$ in $H^\circ(C\cap H,
O_{C\cap H}(n))$ contains the subspace $H^\circ(S,O_S(n))$ and hence
$t_n\geq(r-1)n+1$. 
\end{proof}

\section{}\label{chap3:sec2}\pageoriginale
\begin{itemize}
\item [a)] {\bf First Theorem of Castelnuovo}
\begin{THM}\label{chap3:thm1}
Let $C$ be a smooth curve embedded in $\mathbb{P}^r,r\geq 3$, and let
degree of $C$ be $d$. Assume that $C$ is not contained in any
hyperplane. Define $\chi$ as the largest integer such that, 
$$
(r-1)\chi +1<d.
$$

Then $H^1(C,O_C(n))=0$ for every $n\geq\chi$.
\end{THM}

\begin{proof}
Choose $H$, a hyperplane of $\mathbb{P}^r$, as in Bertini's
theorem. So $H\cap C$ is reduced and any $r$ points in $H\cap C$
generate $H$. So by lemma $H^\circ(H,O_H(m))\longrightarrow H^\circ
(C\cap H,O_{C\cap H}(m))\longrightarrow 0$ is exact for every
$m>\chi$. 

We have a commutative diagram of exact sequences:
\[
\xymatrix{
H^\circ(\mathbb{P}^r, O_{\mathbb{P}^r}(m)) \ar[d] \ar[r] &
H^\circ(H,O_H(m))\ar[d] \ar[r] & 0\\
H^\circ(C,O_C(m)) \ar[r]^\varphi & H^\circ(C\cap H,O_{C\cap H}(m))
\ar[d]\\ 
& 0 &    
}
\]
for $m>\chi$. So we see that $\varphi$ is surjective for
$m>\chi$. Since $H$ is a hyperplane, we have an exact sequence, $0
\longrightarrow(-1)\longrightarrow\longrightarrow_H \longrightarrow
0$, which tensoring by $O_C$ gives an exact sequence,
$$
0\longrightarrow O_C(-1)\longrightarrow O_C\longrightarrow O_{C\cap H}
\longrightarrow 0.
$$

(The sequence is exact at the left because $C$ is integral and $C$ is
not contained in $H$). So for any $m$ in $\mathbb{Z}$ we get an exact
sequence
$$
0\longrightarrow O_C(m-1)\longrightarrow O_C(m)\longrightarrow
O_{C\cap H}(m)\longrightarrow 0.
$$

The corresponding exact sequence of cohomologies is,
\begin{gather*}
0\longrightarrow H^\circ(O_C(m-1))\longrightarrow H^\circ(O_C(m))
\xrightarrow{\varphi}H^\circ(O_{C\cap H}(m))\\
\longrightarrow H^1 (O_C(m-1))
H^1(O_C(m))\longrightarrow 0.
\end{gather*}\pageoriginale

Since $\varphi$ is surjective for $m>\chi$, we see that $H^1(O_C(m-1)
\xrightarrow{\sim}H^1(O_C(m))$ for $m>\chi$. Since $H^1(O_C(n))=0$ for
large $n$, we see that, $H^1O_C(m))=0$ for every
$m>\chi$. \hfill{Q.E.D}

Define $d_n=\dim H^1(\mathbb{P}^r,J(n))$ where $J$ is the sheaf of
ideals of the curve $C$ (as above) in $\mathbb{P}^r$.
\end{proof}

\begin{lem}\label{chap3:lem2}
If $n\geq\chi$, then $d_n\geq d_{n+1}$.
\end{lem}

\begin{proof}
Let $H$ be a hyperplane as before. We have an exact sequence,
$$
0\longrightarrow O(-1)\longrightarrow O\longrightarrow O_H
\longrightarrow 0.
$$

Tensoring by $J$ which is a torsion free $0$-module, we get an exact
sequence,
$$
0\longrightarrow J(-1)\longrightarrow J\longrightarrow J_H
\longrightarrow 0.
$$

So for any integer $n$ we have the exact sequence,
$$
0\longrightarrow J(n)\longrightarrow J(n+1)\longrightarrow J_H(n+1)
\longrightarrow 0,
$$
which in turn gives an exact sequence of cohomologies,
\begin{equation*}
H^1(\mathbb{P}^r,J(n))\xrightarrow{\varphi_n}H^1(\mathbb{P}^r,J(n+1))
\longrightarrow H^1(\mathbb{P}^r,J_H(n+1))\tag{*}\label{chap3:eq*}
\end{equation*}

So cokernel of $\varphi_n$ is contained in
$H^1(\mathbb{P}^r,J_H(n+1))$ for every $n$.

Tensoring the exact sequence $0\longrightarrow J\longrightarrow
O\longrightarrow O_C\longrightarrow 0$ by ${}_H$, we get the sequence,
$0\longrightarrow J_H\longrightarrow O_H\longrightarrow O_{C\cap H}
\longrightarrow 0$, to be exact because $J$ is a prime ideal. so we
get an exact sequence of cohomologies, 
\begin{multline*}
H^\circ(\mathbb{P}^r,O_H(n+1))\longrightarrow
H^\circ(\mathbb{P}^r,O_{C\cap H}(n+1))\longrightarrow
H^1(\mathbb{P}^r, J_H(n+1)\\
H^1(\mathbb{P}^r, O_H(n+1))\longrightarrow 0,\quad\text{for every}
\quad n\in\mathbb{Z}.
\end{multline*}\pageoriginale
$H^1(\mathbb{P}^r,O_H(n+1)=0$ for $n\geq 0$. Also by lemma
\ref{chap3:lem1},
$$
H^\circ(\mathbb{P}^r,O_H(n+1))\longrightarrow
H^\circ(\mathbb{P}^r,O_{C\cap H}(n+1))
$$
is surjective for $n\geq\chi$. Thus $H^1(\mathbb{P}^r,J_H(n+1))=0$ for
$n\geq\chi$. So from \eqref{chap3:eq*}, we see that $\varphi_n$ is
surjective for $n\geq\chi.\,\ie\, d_n\geq d_{n+1}$ for $n\geq\chi$.
\end{proof}
\item [b)] {\bf COROLLARY (M. Noether).} Let $C$ be a smooth curve of
  genus greater than or equal to 3 and non-hyperelliptic (\ie
  $\omega_C$ is very ample.) Then the canonical embedding is
  arithmeticaly normal. 
\end{itemize}

\begin{proof}
Since $\dim H^\circ(C,\omega_C)=g=\quad\text{genus of}\quad C$, the
canonical embedding is $C\longrightarrow\mathbb{P}^{g-1}$ and degree
of $C=2g-2$. If $g=3$, then $C\hookrightarrow\mathbb{P}^2$ and hence
$C$ is a complete intersection and in particular arithmetically
normal. So we can assume $g$ is greater than $3$. Arithmetic normality
is equivalent to $O_C(n)$ being complete for every $n\geq 0. O_C(1)$
is complete by hypothesis. Now we will calculate $\chi$ for this curve
\begin{align*}
&(g-2)\chi +1<2g-2\\
\text{\ie}\qquad &\chi<\frac{2g-3}{g-2}=2+\frac{1}{g-2}
\end{align*}
and so $\chi=2$. By lemma \ref{chap3:lem2}, we know that $d_n\geq
d_{n+1}$ for $n\geq 2$. So if we show that $d_2=0$, then $O_C(n)$ is
complete for every $n\geq 0$. 

Choosing $H$ as before, we have an exact sequence, $0\longrightarrow
O(-1)\longrightarrow O\longrightarrow O_H\longrightarrow 0$ Therefore,
$0\longrightarrow O_C(1)\longrightarrow O_C(2)\longrightarrow O_{C\cap
  H}(2)\longrightarrow 0$ is exact. Thus we get a commutative diagram
of exact sequences: 
{\fontsize{9}{11}\selectfont
\[
\xymatrix{
& 0 & & & \\
0 \ar[r] & H^\circ(C,O_C(1)) \ar[r]^i \ar[u]& H^\circ(C,O_C(2)) \ar[r]^>>>>>>\pi &
H^\circ(C\cap H,O_{C\cap H}(2))\\
0 \ar[r] & H^\circ(\mathbb{P}^r,O_{\mathbb{P}^r}(1)) \ar[r] \ar[u]^f
& H^\circ(\mathbb{P}^r,O_{\mathbb{P}^r}(2)) \ar[u]^{\bar{g}} \ar @{-->}[ur]_h 
}
\]}\pageoriginale
rank of $h=t_2$. By Corollary \ref{chap3:cor2} of lemma
\ref{chap3:lem1}, we have, $t_2\geq\inf(2g-2,(g-2).2+1)=2g-3$. Since
$f$ is surjective, Image of $i$ is contained in Image of
$\bar{g}$. Dimension of Image of $i=g.\dim
H^\circ(C,O_C(2))=4g-4-g+1=3g-3$ by Riemann-Roch theorem. Thus
$$
\rank\;\text{of} \;h=\rank \;\text{of}\;\bar{g}-\dim (\Iim i)=(\rank
\;\text{of}\;\bar{g})-g.
$$

Hence $(\rank \;\text{of}\; \bar{g})=t_2+g\geq 2g-3+g=3g-3$. Thus
$\bar{g}$ is surjective and $d_2=0$.
\end{proof}

\begin{lem}\label{chap3:lem3}
Let $\chi$ and $d_n$ be as before. If $n\geq\chi +1$ then,
$d_n>d_{n+1}$ or $d_n=0$.
\end{lem}

\begin{proof}
\begin{enumerate}
\item Choose $H$ as before. Now choose another hyperplane $H_1$ such
  that $H\cap H_1\cap C=\phi\emptyset$. Let $f=0$ and $g=0$ be their
  respective equations. Let $J$ be the sheaf of ideals of $C$ in
  $\mathbb{P}^r$ and consider the Koszul complex with respect to
  $(f,g)$: 
{\fontsize{10}{12}\selectfont
\[
\xymatrix{
& 0 \ar[d] & 0 \ar[d] & 0 \ar[d] &\\
0 \ar[r] & J(n-1) \ar[d] \ar[r] & J(n)\oplus J(n) \ar[d] \ar[r] &
J(n+1) \ar[d] &\\
0 \ar[r] & O_{\mathbb{P}}(n-1) \ar[d] \ar[r] & O_{\mathbb{P}}(n)\oplus
O_{\mathbb{P}}(n) \ar[d] \ar[r] & O_{\mathbb{P}}(n+1) \ar[d] &\\
0 \ar[r] & O_C(n-1) \ar[d] \ar[r] & O_C(n)\oplus O_C(n) \ar[d] \ar[r]
& O_C(n+1) \ar[d] \ar[r] & 0\\
& 0 & 0 & 0 & 
}
\]}

The bottom sequence is exact because, $f$ and $g$ have no common zeros
on $C$. So we see that the mapping cone,
\newpage

{\fontsize{9}{11}\selectfont
\[
\rotatebox{90}{
\xymatrix{
0 \ar[r] & J(n-1) \ar[r] & O_{\mathbb{P}}(n-1)\oplus J(n)\oplus J(n)
\ar[rr] \ar[dr] && O_{\mathbb{P}}(n)\oplus O_{\mathbb{P}}(n)\oplus
J(n+1) \ar[r] & O_{\mathbb{P}}(n+1) \ar[r] & 0\\
& & & M(n+1) \ar[ur] \ar[dr] & & & \\
& & 0 \ar[ur] & & 0 & &  
}}
\]}\pageoriginale
is exact. From this we get a complex,
\begin{align*}
H^1(J(n-1))&\longrightarrow H^1(O_{\mathbb{P}}(n-1)\oplus J(n)\oplus
J(n))\\
&\longrightarrow H^1(O_{\mathbb{P}}(n)\oplus O_{\mathbb{P}}(n)
\oplus J(n+1)).
\end{align*}

Denote the homology at the middle by $W_{n+1}$. 

\item {\bf CLAIM}.
$W_{n+1}= Coker\,(H^\circ(J(n+1)\oplus O_{\mathbb{P}}(n)
\xrightarrow{\alpha}H^\circ(O_{\mathbb{P}}(n+1))$. We have exact
sequences as follows for $n\geq\chi +1$. 
\[
\xymatrix{
& & 0 & &\\
0 \ar[r] & W_{n+1} \ar[r] & H^1(M(n+1)) \ar[r] \ar[u] & H^1(J(n+1))
\ar[r] & 0\\
& & H^1(J(n)+J(n)) \ar[u]\\
& & H^1(J(n-1) \ar[u] & & 
}
\]
$\beta$ is surjective because Coker $\beta$ is contained in,
$H^2(J(n-1))=H^1(C,O_C(n-1))=0$ by Castelnuovo's theorem, since
$n\geq\chi +1$. 
\end{enumerate}

We obtain the result by using the following lemma:
\end{proof}

\setcounter{dashlem}{2}
\begin{dashlem}\label{chap3:lem3'}
If $n>\chi,d_n=d_{n+1}$ implies $W_{n+1}=0$ and $d_{n+1}=d_{n+2}$. It
is clear that lemma \ref{chap3:lem3'}$'$ implies lemma \ref{chap3:lem3}.
\end{dashlem}

\begin{proof}
It is clear from the commutative diagrams we have considered that,
$H^1(J(n)\oplus J(n))\longrightarrow H^1(J(n+1))$ is surjective and
(image $H^1(J(n-1))\longrightarrow H^1(J(n)\oplus J(n)))$ is at least
$d_n$-dimensional for $n>\chi$. We have the exact sequence,
$H^1(J(n-1))\xrightarrow{\varphi}H^1(J(n)\oplus J(n)) \longrightarrow
H^1(J(n+1))\longrightarrow 0$ if\pageoriginale $n>\chi. \dim(\Iim
\varphi)\geq d_n, \dim (\ker\Psi)=2d_n-d_{n+1}$. If $d_n=d_{n+1}$,
then $\dim \ker\Psi=d_n\geq\dim(\Iim\varphi)\geq d_n$. Hence $\ker\Psi
= \Iim\varphi$. Therefore $W_{n+1}=0$. So we get,
$H^\circ(J(n+1))\oplus O_{\mathbb{P}}^2(n))\longrightarrow H^\circ
(O_{\mathbb{P}}(n+1))$ is surjective, and hence a commutative diagram
of exact sequence:
\[
\rotatebox{90}{
\xymatrix{
H^\circ(J(n+1)\oplus O_{\mathbb{P}}(n))\otimes H^\circ (\mathbb{P},
O_{\mathbb{P}} (1)) \ar[d] \ar[r]^{\alpha\otimes Id} & H^\circ
(O_{\mathbb{P}}(n+1))\otimes H^\circ(\mathbb{P}, O_{\mathbb{P}}(1))
\ar[r] \ar[d] & 0\\
H^\circ(J(n+2)\oplus O_{\mathbb{P}}^2(n+1)) \ar[r]^{\alpha'} & H^\circ
(O_{\mathbb{P}}(n+2)) \ar[d]&\\
& 0 & 
}}
\]
Therefore $\alpha'$ is surjective and hence $W_{n+2}=0$. As before we
get a complex, 
$$
H^1(J(n))\xrightarrow{\varphi}H^1(J(n+1)\oplus J(n+1))\longrightarrow
H^1(J(n+2))\longrightarrow 0
$$
and since $W_{n+2}=0$, this is exact. $\dim(\Iim\varphi)=d_{n+1}$. But
$\dim H^1(J(n))=d_n=d_{n+1}$. Therefore $\dim \Iim\varphi=d_{n+1}$ and
hence $\varphi$ is injective. So $d_{n+1}=2d_{n+1}-d_{n+2} \,\ie\,
d_{n+1}= d_{n+2}$.
\end{proof}

\section{}\label{chap3:sec3}
\begin{THM}\label{chap3:thm2}
{\bf (Castelnuovo).} Let $C$ be a smooth irreducible curve embedded in
$\mathbb{P}^r, r\geq 3$ and let $J$ be the ideal sheaf of $C$. Then
$H^1(\mathbb{P}^r,\break J(n))=0$ for $n\geq d-2$ where $d$ is the degree of
$C$ [\ie for such $n,O_C(n)$ is a complete linear system on $C$.]
\end{THM}

\begin{proof}
First we will show that it is sufficient to do this when $r=3$. Any
smooth curve in $\mathbb{P}^r$ can be projected from a point to
$\mathbb{P}^{r-1}$ isomorphically if $r>3$. Let $\pi :\mathbb{P}^r-(P)
\longrightarrow\mathbb{P}^{r-1}$ be the projection map, so that $C
\longrightarrow \pi(C)$ is an isomorphism. Because this is a
projection from a point and since $P$ can be chosen such that $d^\circ
C= d^\circ\pi(C)[d^\circ$ denotes the degree of the curve]. This
can\pageoriginale be easily checked if we go through the construction
of $\pi$ and using Bertini. Since $C\simeq\pi(C)$, we see that $O_C
\xrightarrow[\sim]{\pi}O_{\pi(C)}$ and for these different embeddings
of $C$ and $\pi(C)$, we still have $O_C(n)\xrightarrow[\sim]{\pi}
O_{\pi(C)}(n)$. So we have the following commutative diagram, $r>3$, 
\begin{align*}
H^\circ(\mathbb{P}^r,O_{\mathbb{P}^r}(n))&=H^\circ(\mathbb{P}^r-(P),O_{\mathbb{P}^r}
(n))\\
&\longrightarrow H^\circ(C,O_C(n))\longrightarrow
H^1(\mathbb{P}^r,J_C(n))\longrightarrow 0\\
H^\circ(\mathbb{P}^{r-1},O_{\mathbb{P}^{r-1}}(n))&\longrightarrow
H^\circ (\pi(C),O_{\pi(C)}(n))\\
&\longrightarrow H^1 (\mathbb{P}^{r-1},
J_{\pi(C)}(n))\longrightarrow 0
\end{align*}
If we have proved the result in $\mathbb{P}^{r-1}$, then $H^\circ
(\mathbb{P}^{r-1}, O_{\mathbb{P}^{r-1}}(n))\longrightarrow\break
H^\circ(\pi(C),O_{\pi(C)}(n))$ is surjective and by commutativity,
$H^\circ (\mathbb{P}^r-(P),\break O_{\mathbb{P}^r}(n)) \longrightarrow
H^\circ (C,O_C(n))$ is surjective. So $H^1(\mathbb{P}^r, J_C
(n))=0$. Thus it is sufficient to prove the result in $\mathbb{P}^3$ 

So assume $C\longrightarrow\mathbb{P}^3$. If $C$ is already a plane
curve we have nothing to prove, since then $C$ is complete
intersection. If that is not the case, since $C$ is not contained in
any hyperplane, we can project from a ponit in $\mathbb{P}^3$, so that
the image curve is
birational to $C$ and has only ordinary double points as
singularities. (Remark after Bertini's theorem.) Let $C'$ denote the
image of $C$ in $\mathbb{P}^2$. Consider cones over $C$ and $C'$. They
are graded rings over $k$. Cone over $C=A=k[x_0,x_1,x_2,x_3],x_i$'s so
chosen that, cone over $C'=B=k[x_0,x_1,x_2]$. Since $C'$ has only
ordinary double points we see that $x_3$ satisfies a degree 2 equation
over $B$.

We have maps, $B_n+x_3 B_{n-1}\longrightarrow A_n$, which gives a
graded $B$ - homomorphism, $B+x_3B\longrightarrow A$. 
$$
\Proj B=C'\quad\text{and}\quad\Proj A=C.
$$

This in turn gives a morphism of $O_{C'}$-sheaves $O_{C'}\oplus O_{C'}
(-1)\longrightarrow O_C$. 

If\pageoriginale $Q\in C'$ is a non-singular point, then $O_{C',Q}
\longrightarrow O_{C,\pi}-1_{(Q)}$ is an isomorphism and hence the
above map is surjective. If $Q$ is a singular point, we have a
sequence,
$$
0\longrightarrow O_{C',Q}\longrightarrow O_{C,\pi}-1_{(Q)}
\longrightarrow O_{C,\pi}-1_{(Q)/O_{C',Q}}\longrightarrow 0
$$

But since $\ell(O_{C,\pi}-1_{(Q)/O_{C',Q}})=1(C'$ has only ordinary
double points as singularities) and image of $O_{C',Q}(-1)$ is not
contained in the image of $O_{C',Q}$ under the given map, we see that
the above map is surjective. 
\end{proof}

\begin{claim*}
We have an exact sequence of $O_{C'}$-sheaves,
\begin{equation*}
0\longrightarrow\underline{f}(-1)\longrightarrow O_{C'}\oplus
O_{C'}(-1)\longrightarrow O_C\longrightarrow 0\tag{*}\label{chap3:eq1*}
\end{equation*}
where $\underline{f}\subset O_{C'}\subset O_C$ is the conductor sheaf.
\end{claim*}

This map at graded ring level is as follows: If $F$ denotes the graded
ideal corresponding to $\underline{f}$,
$$
F_{n-1}\xrightarrow{\varphi}B_n+B_{n-1}\longrightarrow A_n
$$
$\alpha\longrightarrow(-x_3\alpha,\alpha)[x_3\alpha\in B_n$ because
$\alpha$ is in the conductor.] and $\varphi(\alpha,\beta)=\alpha
+x_3\beta$. So it is clear that the sequence \eqref{chap3:eq1*} is a
complex. Also it is clear that $\varphi$ is an injection. So we have
only to check exactness at the middle. We have the complex locally,
$$
0\longrightarrow\underline{f}(-1)_Q\longrightarrow O_{C',Q}\oplus
O_{C',Q}(-1)\longrightarrow O_{C,\pi^{-1}(Q)}\longrightarrow 0.
$$

Let $(\alpha,\beta)\longrightarrow 0$ \ie
$$
\alpha +x_3\beta =0\Longrightarrow\alpha = -x_3\beta\Longrightarrow
x_3 \beta\in \; C',Q.
$$

Since $x_3$ generates $O_{C,\pi^{-1}(Q)}$ over $O_{C',Q}$ and $x_3$ is
integral of degree 2 over\pageoriginale $O_{C',Q}$, we see that this
implies, $\beta\in\underline{f}(-1)_{Q^.}\,\ie\,(\alpha,\beta)=
(-x_3\beta,\beta)$ with $\beta\in\underline{f}(-1)_Q$. This proves the
claim.
\begin{align*}
\omega_{C'} &= O_{C'}(d-3)\\
\omega_C &= \Hom_{O_{C'}}(0_C,\omega_{C'})\\
&= \Hom_{O_{C'}}(O_{C'},O_{C'})\otimes\omega_{C'}=\underline{f}
\otimes_{O_{C'}}\omega_{C'}=\underline{f}\otimes_{O_{C'}}(d-3).
\end{align*}

Therefore $\underline{f}(-1)=\omega_C\otimes_{O_{C'}}O_{C'}(2-d)$.

\noindent Tensoring \eqref{chap3:eq*} by $O_{C'}(d-2)$ and noting that
$\pi^*(O_{C'}(r))=O_C(r)$, we see that,
$$
0\longrightarrow\omega_C\longrightarrow O_{C'}(d-2)\oplus O_{C'}(d-3)
\longrightarrow O_C(d-2)\longrightarrow 0
$$
is exact, as $O_{C'}$-modules.

\noindent So we have an exact sequence,
\begin{gather*}
0\longrightarrow H^\circ(C,\omega_C)\longrightarrow
H^\circ(C,O_{C'}(d-2))\oplus H^\circ(C',O_{C'}(d-3))\\
\longrightarrow H^\circ (C,O_C(d-2))\\
\xrightarrow{\alpha}H^1(C,\omega_C)\xrightarrow{\beta}H^1(C',O_{C'}
(d-2)) \oplus H^1(C',O_{C'}(d-3))\longrightarrow 0
\end{gather*}
$H^1(C,O_C(d-2))=0$ by Castelnuovo's first theorem: since $d\geq 3$,
because $C$ is not a complete intersection, and
$O_{C'}(d-3)=\omega_{C'}$, we see that $H^1(C',O_{C'}(d-3))\neq 0$ and
$H^1(C,\omega_C)\simeq k$, because $C$ is smooth and $\omega_C$ is the
canonical line bundle. So $\beta$ surjective implies that $\beta$ is
an isomorphism and hence $\alpha$ is the zero map. So we have a
commutative diagram of exact sequences:

\newpage



\[
\rotatebox{90}{
\xymatrix{
& & 0 & &\\
0 \ar[r] & H^\circ(C,\omega_C) \ar[r] &
H^\circ(C',O_{C'}(d-2))\oplus H^\circ(C',O_{C'}(d-3))\ar[u] \ar[r] &
H^\circ(C,O_C(d-2) \ar[r] & 0\\
& & H^\circ(\mathbb{P}^2,O_{\mathbb{P}^2}(d-2))\oplus\ar[u]H^\circ
(\mathbb{P}^2,O_{\mathbb{P}^2}(d-3)\ar[r] & H^\circ(\mathbb{P}^3,
O_{\mathbb{P}^3} (d-2)) \ar[u]&
}}
\]

\newpage


So it follows that, $H^\circ(\mathbb{P}^3,O_{\mathbb{P}^3}(d-2))
\longrightarrow H^\circ(C,O_C(d-2))$\pageoriginale is surjective. \ie
$H^1(\mathbb{P}^3,J(d-2))=0$. 

\noindent Thus by lemma \ref{chap3:lem2}, we see that 
$$
H^1(\mathbb{P}^3,J(n))=0\quad\text{for}\quad n\geq d-2.
$$

