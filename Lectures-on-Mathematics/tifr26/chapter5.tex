\chapter{Linear mapping in cones}

If $A$\pageoriginale is a linear operator in $R^{(n)}$ with a matrix
$(a_{ij})$ with 
non-negative elements, $a_{ij}\geq 0$, then, by a famous theorem of
Perron and Frobenius (see for example Gantmacher, The theory of
matrices), there exists an eigen vector of $A$ with non-negative
coordinates and with eigenvalue $\rho$, such that all other
eigenvalues satisfy $|\lambda| \leq \rho$. 

If we take $E=R^{(n)}$ and $C$ to be the set of all vectors in $E$
with non-negative coordinates, then $C$ is a positive cone in $E$,
and $n\times n$ matrices with non-negative elements correspond to
linear operators in $E$ that map $C$ into itself. Then theorems of the
present chapter may be regarded as generalizations of the
Perron-Frobenius Theorem. $A$ great many such generalizations with
various methods of proof have been published during recent years, and
our list of references is far from complete.  

The idea of the method of proof adopted here is the use of the simple
concept of 'topological divisor of zero'. 

Let $\mathcal{U}$ be a Banach algebra with a unit element $e$, and
let $a$ be a frontier point of the set of invertible elements. Then a is a
topological divisor of zero, i.e., there exists a sequence $(x_n)$
with $||x_n|| = 1 (n = 1,2,\ldots)$ such that  
$$
\lim_{n \to \infty} ax_n = 0
$$\pageoriginale

\begin{proof}
  There exists a sequence $(a_n)$ of invertible elements such that 
  $$
  \lim_{n \to \infty} a_n = a
  $$
  Then the sequence $(|| a^{-1}_n ||)$ is unbounded. For otherwise
  $$
  \displaylines{\hfill 
  \lim_{n \to \infty}(a_n - a) a_n^{-1} = 0\hfill \cr
  \text{i.e.,} \hfill \lim \limits_{n \to \infty} (e - a a_n^{-1}) =0
  \hfill }
  $$
  But this implies that  $a a_n^{-1}$ is invertible for some $n$, and
  therefore $a$ has a right inverse. Similarly $a$ has a left inverse, and
  $a$ is invertible, which is absurd since the set of invertible
  elements is open. 
\end{proof}

We may therefore suppose that $|| a_n^{-1} || \to \infty$, and take
$x_n = || a_n^{-1} ||^{-1}\break a_n^{-1}$. Then $|| x_n || = 1$, and 
$$
\lim_{n \to \infty} ax_n = \lim\limits_{n \to \infty} (a -a_n) x_n +
a_n x_n = 0. 
$$

In particular, if $\lambda$ is a frontier point of the spectrum of an
element $b$, then $\lambda e - b$ is a frontier point of the set of
invertible elements of $\mathcal{U}$ and so there exists $(x_n)$, with
$|| x_n || = 1$ and  
$$
\lim_{n \to \infty} (\lambda e - b) x_n = 0
$$

If further\pageoriginale we know that $\lambda \neq 0$, and $b x_{n_k}
\to u$ for some subsequence $(x_{n_k})$. Then  
\begin{align*}
  \lambda x_{n_k} & \to u ,\\
  \lambda b x_{n_k} & \to bu ,
\end{align*}
so that $bu = \lambda u$ and $u \neq 0$, since $|| u || = \lambda$.

Actually, our method is not quite so simple as this, for our Banach
algebra is a Banach algebra of operation on a Banach space $X$, and we
have to replace the sequence $(x_n)$ of operators by a sequence of
elements of $X$. 

Until we reach the statement of our main theorem (Theorem
\ref{chap5:thm5.1}) we 
shall use the following notation. 

$E$ will denote a normed and partially ordered vector space with norm
$|| x ||$ and positive cone $C$. We suppose that $C$ is complete with
respect to $|| x ||$, and that 
$$
E = C - C
$$
We do not suppose that $E$ is complete with respect to $|| x ||$. We
denote by $B$ the intersection of $C$ and the closed unit of $E$,
i.e., 
$$
B = \{ x : x \in C ~ \text{ and } ~ || x || \leq 1\}
$$
We denote by $B^0$ the convex symmetric hull of $B$, i..e.
$$
B^0 = \{ \alpha x - \beta y : x, y \in B, \alpha \geq 0, \beta \geq 0,
\alpha + \beta = 1 \}, 
$$\pageoriginale
and by $|| x ||_c$ the Mindowski functional of $B^0$, i.e.,
$$
|| x ||_c = \inf \{ \lambda : \lambda > 0, x \in \lambda B^0 \}
$$

\setcounter{section}{5}
\setcounter{lemma}{0}
\begin{lemma}% lemma 5.1
  \begin{enumerate}
  \item[$(\alpha)$] $ || x ||_c$ is a norm on $E$ and satisfies 
    $$
    || x ||_c = || x || \; (x \in C) , || x ||_c \geq || x || \qquad (x \in E)
    $$

  \item[$(\beta)$] $E$  is complete and $C$ is a closed subset of $E$
    with respect to $|| x ||_c$. 
  \end{enumerate}
\end{lemma}

\begin{proof}
  $(\alpha)$~ since $E = C - C, B^0$ is an absorbing set for
    $E$, and so the Mindowski functional $|| x ||_c$ is defined on
    $E$. Since $B^0$ is convex and symmetric, $|| x ||_c$ is a
    seminorm on $E$. If $z \in E$ and $|| z ||_c < 1$, then $z \in
    B^0$ i.e. $z = \alpha x - \beta y$ with $x, y \in B$ and $\alpha
    \geq 0$, $\beta \geq 0$, $\alpha + \beta = 1$. Therefore 
    $$
    || z || = || \alpha x - \beta y || \leq \alpha || x || + \beta ||
    y || \leq \alpha + \beta =  1.
    $$
    This proves that
    $$
    || x || \leq || x ||_c \qquad (x \in E)
    $$
    and completes the proof that $|| x ||_c$ is a norm. Since $B
    \subset B^0$, we have  
    $$
    || x ||_c \leq ~ 1  \qquad (x \in B),
    $$
    and therefore
    $$
    || x ||_c \leq || x || \qquad (x \in C)
    $$\pageoriginale
    This completes the proof of $(\alpha)$.

$(\beta)$~ Let $(z_n)$ be a Cauchy sequence in $E$ with
    respect to $x_c$. Then there exists a strictly increasing sequence
    $(n_k)$ of positive integers such that 
    $$
    p, q \geq n_k \Rightarrow || z_p - z_q ||_c < 2^{-k}
    $$

  Let $w_k = z_{n_k} \; (k = 1, 2, \ldots)$. Then, in particular,
  $$
  || w_{k+1} - w_k ||_c < 2^{-k} \; (k = 1, 2, \ldots)
  $$
  Therefore
  $$
  w_{k + 1} - w_k \in 2^{-k} B^0,
  $$
  and so $w_{k+1} - w_k = \alpha_k x_k - \beta_k y_k$, with
  $$
  \alpha_k \geq 0 , \beta \geq 0, \alpha_k + \beta_k = 1 , x_k , y_k
  \in 2^{-k} B 
  $$
  Let
  $$
  s_n = \sum^n_{k=1} \alpha_k x_k, \; t_n = \sum^n_{k=1} \beta_k y_k .
  $$
\end{proof}

Then $p > q$ gives
$$
|| s_p - s_q || \leq \sum^p_{q+1} \alpha_k || x_k || \le \sum^p_{k = q
  + 1} 2^{-k} < 2^{-q}, 
$$
and similarly
$$
|| t_p - t_q || < 2^{-q}.
$$
Since\pageoriginale $C$ is complete, there exist $s$, $t$ in $C$ such
that 
$$
\lim_{n \to \infty} || s - s_n || = 0, \quad \lim\limits_{n \to
  \infty} || t-t_n || = 0 
$$
Also, $\lim\limits_{p \to \infty}  s_p - s_q = s - s_q$, and $s_p -
s_q \in C$ whenever $p > q$. Hence $s - s_n \in C ~ (n=1, 2,
\ldots)$. Therefore 
$$
\displaylines{\hfill 
  || s - s_n ||_c = || s-s_n || \quad (n = 1, 2, \ldots), \hfill \cr
  \text{and so}\hfill 
  \lim_{n \to \infty} || s - s_n ||_c = 0.\hfill }
$$

Similarly, $\lim\limits_{n \to \infty} || t - t_n ||_c = 0$, and so
$(w_n)$ converges with respect to $|| x ||_c$ to $w_1 + s - t$. It is
now easily seen that $(z_n)$ converges with respect to $|| x ||_c$,
and so $E$ is complete with respect to $|| x ||_c$. That $C$ is a
closed subset of $E$ with respect to $|| x ||_c$ is a simple
consequence of the inequality 
$$
|| x || \leq || x ||_c \qquad (x \in E)
$$
and the closeness of $C$ with respect to $|| x ||$, (in fact a larger
norm gives a stronger topology). 


\begin{Definition}%definition 5.1
  A linear operator in $E$ is said to be {\em positive } if it maps
  $C$ into $C$ and to be {\em partially bounded } if it maps $B$ into
  a bounded set. The {\em partial bound } $p(T)$ of a partially
  bounded linear operator $T$ is defined by 
  $$
  p(T) = \sup \{ || T || : x \in B\}
  $$
  Given\pageoriginale partially bounded positive linear operators $S$
  and $T$, we have 
  $$
  p(ST) \leq p(S) p(T),
  $$
  and therefore the limit
  $$
  \lim _{n \to \infty} \left\{ p(T)^n \right\}^{\frac{1}{n}}
  $$
  exists. It is called the \textit{ partial spectral radius } of $T$.
\end{Definition}

\begin{lemma} %lemma 5.2
  A positive linear operator $T$ is partially bounded if and only if
  it is a bounded linear operator in the Banach space $(E, ||
  x\ ||_c)$. For such an operator $T$, 
  $$
  p(T) = || T ||_c = \sup \left\{ || Tx ||_c : x \in E ~ \text{ and }
  ~ || x ||_c \leq ~ 1 \right\}, 
  $$
  and the partial spectral radius $\mu$ of $T$ is equal to its
  spectral radius as an operator in $(E, || x ||_c)$, i.e. 
  $$
  \mu = \lim_{n \to \infty} \{|| T^n ||_c\}^{\frac{1}{n}}
  $$
  Finally, if $\lambda > \mu$, there exists a partially bounded
  positive linear operator $R_\lambda$, such that 
  $$
  (\lambda I - T) R_ \lambda = R_ \lambda (\lambda I - T) = I,
  $$
  where $I$ is the identity operator in $E$.
\end{lemma}

\begin{proof}
  Let $T$ be a partially bounded positive linear operator in $E	$ and
  let $x \in E$ with $|| x ||_c < 1$. Then $x \in B^0$, and so  
  $$
 x = \alpha y - \beta z
$$\pageoriginale
with $\alpha \geq 0$, $\beta \geq 0$, $\alpha + \beta = 1$,
    $y$, $z \in B$. Then 
  $$
  Tx = \alpha Ty - \beta Tz,
  $$
  and so 
  \begin{align*}
    || Tx ||_c & \leq \alpha || Ty ||_c + \beta ||Tz||_c\\
    & = \alpha ||Ty || + \beta ||Tz || \\
    & \leq (\alpha+ \beta) p(T) = p(T)
  \end{align*}
  Thus $T$ is a bounded linear operator in $(E, ||x ||_c)$ and
  $$
  || T ||_c \leq p(T)
  $$
  For the converse and the reversed inequality it is enough to note that 
  \begin{align*}
    ||T||_c \geq  & \sup \left\{ ||Tx ||_c : x \in C ~ \text{ and } ~
    ||x ||_c \leq 1 \right\}\\ 
    & \sup \left\{ || Tx || : x \in B \right\} = p(T).
  \end{align*}
\end{proof}

That $\mu = \lim\limits_{n \to \infty}\left\{ || T^n
||_c\right\}^{\dfrac{1}{n}}$ is an obvious consequence of the fact
that $|| T ||_c = p(T)$ for each partially bounded linear operator. 

If $\lambda > \mu$, the series
$$
\frac{1}{\lambda} I + \frac{1}{\lambda 2} T + \frac{1}{\lambda^3} T^2
+ \cdots 
$$
converges\pageoriginale with respect to the operator norm for bounded
linear operators in the Banach space $(E, ||x||_c)$ to a bounded
linear operator $R_ \lambda$, and   
$$
(\lambda I - T) R_ \lambda = R_ \lambda (\lambda I - T) = I.
$$

Since $C$ is closed with respect to $|| x ||_c$, and the partial sums
of the series are obviously positive operators, it follows that $R_
\lambda $ is a positive operator. 

\begin{Definition}%definition 5.2
  A positive linear operator $T$ is said to be a {\em normalising
    operator} if it satisfies the following condition: 
  $$
  \alpha_n \geq 0, y_n \in B, \alpha_n x \geq Ty_n, \; \lim_{n \to
    \infty} \alpha_n = 0 \qquad \lim_{n \to \infty} || Ty_n || = 0 
  $$
\end{Definition}

If $C$ is a normal cone, then every positive linear operator is
obviously a normalising operator. We shall see later that a certain
compactness condition on $T$ suffices to make $T$ a normalising
operator (without restriction on $C$). 

\begin{lemma}\label{chap5:lem5.3}%lemma 5.3
  Let $T$ be a normalizing partially bounded positive linear operator
  with partial spectral radius $\mu$. Then 
  $$
  \lim_{\lambda \to \mu + 0} p(R_ \lambda) = \infty
  $$
\end{lemma}

\begin{coro*}%coro 0
  For each such operator $T$, the partial spectral radius $\mu$ is in
  the spectrum of $T$ regarded as an operator in the Banach space $(E,
  || x ||_c)$. 
\end{coro*}

Suppose that the conditions of the lemma are satisfied but that $p (R_
\lambda)$ does not tend to infinity as $\lambda$ decreases to
$\mu$. Then\pageoriginale there exists a positive constant $M$ such
that $p(R_\nu) \leq M$ for some $\nu$ gather than and arbitrarily
close to $\mu$.  

The case $\mu = 0$ is easily settled. For if $\mu = 0$, then, from the
formula $(\lambda I - T) R_ \lambda = I$, it follows that 
\begin{equation*}
  \lambda R- \lambda x \geq x \qquad (\lambda > 0, x \in C)
  \tag{1}\label{chap5:eq1} 
\end{equation*}

If we let $\lambda$ tend to zero through values for which $p(R_
\lambda) \leq M$, the left hand side of (\ref{chap5:eq1}) tends to
zero, and, since $C$ is closed, we obtain  
$$
- x \in C \qquad (x \in C).
$$
But this implies that $C=(0)$ which was excluded by our axioms on
$C$. 

Suppose now that $\mu > 0$. Then we may choose $\lambda, \nu$ with 
$$
0 < \lambda < \mu < \nu < \lambda + M^{-1},
$$
and with $p(R_\nu) \leq M$. Since $|| R ||_c = p(R_\nu)$, it follows
that the series 
$$
R_\nu + (\nu - \lambda) R^2_\nu (\nu - \lambda)^2 R^3_\nu + \cdots
$$
converges with respect to the operator norm $||~ ||_c$ to a partially
bounded positive linear operator $S$ which is easily seen to satisfy 
$$
S(\lambda I - T)  = (\lambda I - T) S = I
$$\pageoriginale
This gives
$$
Sx = \lambda^{-1} x + \lambda^{-1} T Sx \qquad (x \in C),
$$
from which it follows by induction that
\begin{equation}
  Sx \geq \lambda^{-(n+1)} T^n x \qquad (x \in C , n = 0, 1, 2,
  \ldots) \tag{2}\label{chap5:eq2} 
\end{equation}
since $\lambda < \mu$ and $ \lim \limits_{n \to \infty} || T^n ||_c^{\dfrac{1}{n}} = \mu $, we have 
$$
\lim_{n \to \infty} || \lambda^{-(n+1)} T^n ||_c ~ = ~ \infty
$$
By the principle of uniform boundedness, there exists a point $x$ in
$E$ for which the sequence $|| \lambda^{-(n+1)} T^n x ||_c$ is
unbounded. Since $E = C - C$, it follows that there exists a point $w$
in $C$ for which the sequence $|| \lambda^{-(n+1)} T^n w ||$ is
unbounded. But given any unbounded sequence $\{ a_n \}$ of
non-negative real numbers, there exists a subsequence $(a_{n_k})$ such
that 
\begin{align*}
  a_{n_k} > k   & \qquad (k = 1, 2, \ldots) \tag{3}\label{chap5:eq3}\\
  a_{n_k} > a_j & \qquad (j < n_k , k = 1, 2, \ldots)
  \tag{4}\label{chap5:eq4} 
\end{align*} 

This\pageoriginale is easily proved by induction. For if $n_1, \ldots , n_{r-1}$
have been chosen so that (\ref{chap5:eq3}) and (\ref{chap5:eq4}) are
satisfied for $k = 1, 2, \ldots , r-1$, we take $n_r$ to be the
smallest positive integer $s$ for which 
$$
a_s > a_{n_{r-1}} + r
$$
Hence we see that there exists a strictly increasing sequence $(n_k)$
of positive integers for which 
\begin{equation*}
  \lim_{k \to \infty} \lambda^{-(n_k + 1)} T^{n_k} = \infty
  \tag{5}\label{chap5:eq5} 
\end{equation*} 
and
\begin{equation*}
  ||\lambda^{-(n_k + 1)} T^{n_k} || \geq ~ || \lambda^{-n_k} T^{n_k -
    1} w|| \tag{6}\label{chap5:eq6}
\end{equation*} 
since
$$
|| T^{n_k} w || \leq p(T) || T^{n_k -1} w ||,
$$
we also have
\begin{equation*}
  \lim_{k \to \infty} || \lambda^{-n_k} T^{n_k - 1} w || = \infty
  \tag{7}\label{chap5:eq7}
\end{equation*} 

By (\ref{chap5:eq7}), there is no loss of generality in supposing that
$T^{n_k - 1} 
\neq 0$ for all $k$, and we may take 
\begin{equation}
  y_k = || T^{n_k - 1} w || ^{-1} T^{n_k - 1} w \tag{8}\label{chap5:eq8}
\end{equation} 
 
 Then, by (\ref{chap5:eq2}),
 \begin{equation}
   \lambda^{n_k} || T^{n_k - 1} w ||^{-1} S w \geq \lambda^{-1} Ty_k
   \tag{9}\label{chap5:eq9} 
 \end{equation} 
 
 Since\pageoriginale $y_k \in B$, and $T$ is a normalizing operator,
 it follows from (\ref{chap5:eq7}) and (\ref{chap5:eq9}) that  
 \begin{equation}
   \lim_{k \to \infty} \lambda^{-1} ||Ty_k || ~ = ~ 0
   \tag{10}\label{chap5:eq10} 
 \end{equation} 
 
 But, by (\ref{chap5:eq6}), 
 $$
 \lambda^{-1} || T^{n_k} w || \geq || T^{n_k -1} w || ,
 $$
 which obviously contradicts (\ref{chap5:eq10}). This contradiction
 proves the lemma. 
 
 To deduce the corollary, it is enough to appeal to the continuity of
 the resolvent operator on the resolvent set. 
 
\begin{Definition}%definition 5.3
  Let $\tau_N$ denote the given norm topology in $E, \tau$ a second
  linear topology in $E$, and $A$ a subset of the positive cone
  $C$. We say that $\tau$ is sequentially {\em stronger than $\tau_N$
    at $0$ relative to} $A$ if $0$ is a $(\tau_N)$-cluster point of
  each sequence of point of $A$ of which $0$ is a $\tau$-cluster
  point. 
\end{Definition} 
 
We recall that to say that $0$ is a $\tau$-cluster point of a sequence
$(a_n)$ means that each $\tau$-neighbourhood of $0$ contains points
$a_n$ which \textit{arbitrarily large} $n$. 
 
\begin{thmm}\label{chap5:thm5.1} % theorem 5.1
  Let $E$ be a normed and partially ordered vector space with norm
  topology $\tau_N$, positive cone $C$ complete with respect to the
  norm, and let $B = \{ x : x \in C, || x || \leq 1 \}$. Let $T$ be a
  partially bounded positive linear operator in $E$ with partial
  spectral radius $\mu$, and let $\tau$ be a linear topology in $E$
  with respect to which\pageoriginale $C$ is closed and $T$ is
  continuous.  
\end{thmm} 
 
Let $A$ be a subset of $C$ that is contained in a countably
$\tau$-compact subset of $C$, and let $\tau$ be sequentially stronger
than $\tau_N$ at $0$ relative to $A$. If either 
\begin{enumerate}[(i)]
\item $A = TB$ ~ and ~ $\mu > 0$,
  
  or 
\item $A = B$,
 
  then there exists a non-zero vector $u$ in $C$ with $Tu = \mu u$.
  \begin{proof}
    Since we can restrict our consideration to $C - C$, we shall
    suppose that in fact $E = C - C$. 
  \end{proof}
  
  Since $TB \subset p(T) B$, both (i) and (ii) imply

\item $TB$ is contained in a subset of $C$ that is countably compact
  with respect to $\tau$, and $\tau$ is sequentially stronger than
  $\tau_N$ at 0 relative to $TB$. 
\end{enumerate} 

We prove that under condition (iii), $T$ is a normalizing operator. Let
$$
\alpha_n x = Ty_n + z_n
$$
with $\alpha_n \geq 0, \lim \limits_{n \to \infty} \alpha_n = 0$, $y_n
\in B$ and $z_n \in C$. If $|| Ty_n ||$ does not converge to zero, we
may select a subsequence $(Ty_{n_k})$ such that 
\begin{equation}
  || Ty_{n_k} || \geq \varepsilon > 0 \qquad (k = 1, 2, \ldots)
  \tag{1}\label{chap5:eqq1} 
\end{equation}

We have
$$
Ty_{n_k} + z_{n_k} \to 0 \qquad (\tau),
$$
and\pageoriginale $(Ty_{n_k})$ has a $\tau$-cluster point, $v$ say, in
$C$. It follows that $-v$ is a $\tau$-cluster point of $(z_{n_k})$,
and, since $C$ is $\tau$-closed, $-v \in C, ~  ~ v = 0 $. This now
implies that $0$ is a $\tau$-cluster point of $(Ty_{n_k})$ and
therefore is a $\tau_N$-cluster point of $(Ty_{n_k})$, which
contradicts (\ref{chap5:eqq1}).   

Thus, by Lemma \ref{chap5:lem5.3}, we have
$$
\displaylines{\hfill 
  \lim_{\lambda \to \mu + 0} p(R_\lambda)   =  \infty ,\hfill \cr
  \text{i.e.,} \hfill  \lim_{\lambda \to \mu + 0} || R_ \lambda ||_c
  =  \infty.\hfill \phantom{i.e.,}}
$$ 

Applying the principle of uniform boundedness, we see that there
exists a sequence $(\lambda_n)$ converging decreasingly to $\mu$ and a
vector $w$ in $C$ with $|| w || = 1$ such that 
$$
\lim_{n \to \infty} R_{\lambda_n} w = \infty,
$$
and we may suppose that $|| R_{\lambda_n} w || \neq 0 ~ (n=1, 2,
\ldots)$. Let $\alpha_n = || R_{\lambda_n} w ||^{-1}$ and $u_n =
\alpha_n R_{\lambda_n} w$. Then $u_n \in B, || u_n || = 1 , \lim
\limits_{n \to \infty} \alpha_n = 0$, and $\mu u_n - Tu_n = (\mu -
\lambda_n) u_n + (\lambda_n I - T) u_n = (\mu - \lambda_n) u_n +
\alpha_n w$. 

Suppose that condition (ii) in the statement of the theorem  is
satisfied. Since $B$ is $\tau$-countably compact and $u_n \in B$, it
follows from (2) that 
\begin{equation}
  \lim_{n \to \infty} \mu u_n - Tu_n = 0 \qquad (\tau)
  \tag{3}\label{chap5:eqq3}
\end{equation}\pageoriginale

Also $(u_n)$ has a $\tau$-cluster point $u$ in $C$, and
(\ref{chap5:eqq3}) shows that $\mu u - Tu = 0$. 

We have $u \neq 0$, for otherwise $a$ is 0 $\tau_N$-cluster point of
$(u_n)$, which contradicts $|| u_n || = 1$. 

Finally suppose that the condition (i) is satisfied. Then by (2).
$$
(\mu - I - T) Tu_n = T(\mu I - T) u_n = (\mu - \lambda_n) Tu_n + \alpha_n Tw
$$ 
Since $TB$ is contained in a $\tau$-countably-compact subset of $C$,
this shows that 
$$
\lim_{n \to \infty} (\mu I - T) Tu_n = 0 \qquad (\tau)
$$
and that $(Tu_n)$ has a $\tau$-cluster point $v$ in $K$. By the
$\tau$-continuity of $(\mu I - T)$, we have therefore 
$$
(\mu I - T)v = 0
$$
If $v = 0$, then $0$ is a $\tau_N$-cluster point of $(Tu_n)$. But, by (2),
$$
\lim_{n \to \infty} \mu u_n - Tu_n = 0 \qquad (\tau_N)
$$
and therefore $0$ is a $\tau_N$-cluster point of $(\mu u_n)$. But
since $\mu > 0$, and $|| u_n || = 1$, this is absurd. 

The statement of Theorem \ref{chap5:thm5.1} is somewhat complicated in
that it seeks to combine generally and precision. A number of less
complicated\pageoriginale but also less general theorems are easily
deduced from it.   

\begin{thmm}\label{chap5:thm5.2}%theorem 5.2
  Let $C$ be a complete positive cone in a normed space $E$, let $B =
  \{ x : x \in C ~ \text{ and } ~ || x || \le 1\}$, and let $T$ be a
  positive linear operator which is continuous in $C$, and maps $B$
  into a compact set and has a positive partial spectral radius
  $\mu$. Then there exists a non-zero vector $u$ in $C$ with 
  $$
  Tu = \mu u.
  $$
\end{thmm}

\begin{proof}
In Theorem \ref{chap5:thm5.1}, take $\tau = \tau_N$.
\end{proof}

\begin{exam} %example 1
  Let $E = C_R [0, 1]$ with the uniform norm, and let $C$ be the
  positive cone in $E$ consisting of those functions $f$ belonging to
  $E$ that are increasing, convex in $[0, 1]$ and satisfy $f(0) = 0
  $. 
\end{exam} 
 
 Let $0 < k < 1$, and let $T$ denote the operator in $E$ defined by 
 $$
 (Tf) (x) = f(kx) \qquad (f \in E, 0 \leq x \leq 1)
 $$
 Plainly $T$ is a bounded linear operator in $E$ and maps $C$ into itself.
 
 For $f$ in $C$ we have $|| f || = f(1)$, and also since $x = (1-x) 0 + x 1$,
 $$
 f(x) \leq (1-x) f(0) + x f(1) = x f (1) \qquad (0 \leq x \leq 1)
 $$
 since $T^n f (x) = f(K^n x)$, it follows that
 $$
 || T^n f|| = f(k^n) \leq k^n f(1) = k^n || f || \qquad (f \in C)
 $$
 and\pageoriginale equality is attained with the function $f(x) =
 x$. Thus 
 $$
 p(T^n) = k^n,
 $$
 and the partial spectral radius of $T$ is $k > 0$. Also $T$ maps $B$
 into a compact set, for given a convex function $f$ and  $0 < x_1 <
 x_2 \leq y_1 < y_2 \leq 1$, we have 
 $$
 \frac{f(x_2) - f(x_1)} {x_2 - x_1} \leq \frac{f(y_2) - f(y_1)} {y_2 - y_1}
 $$
 Given $f \in B$, and $0 \leq s < t \leq 1$, we therefore have
 $$
 \displaylines{\hfill 
 0 \leq \frac{f(kt) - f(ks)}{kt - ks} \leq \frac {f(1) - f(k)} {1-k}
 \leq \frac{1}{1-k}, \hfill \cr  
 \text{i.e.,} \hfill 0 \leq (Tf) (t) - (Tf) (s) \leq \frac{k}{1-k}
 (1-s) \hfill }
 $$
  
 This proves that $TB$ is an equicontinuous set of functions, and,
 since $TB$ is also bounded, it is contained in a compact set. 
  
  For this particular operator $T$ the conclusion of Theorem
  \ref{chap5:thm5.2} is 
  of course trivial since the function $u$ given by $u(x) = x (0 \leq
  x \leq 1)$ plainly satisfies $Tu = ku$. 
  
The example is however of interest in that it provides a simple
example of a bounded linear operator completely continuous in a cone
that is not a compact linear operator in any subspace of $E$ that
contains $C$. In fact let $g_n, h_n, f_n$ be defined by
\begin{align*}
    g_n (x) &= k^{-n} x  \quad (0 \leq x \leq 1) ,\\
    h_n(x)  &= 
    \begin{cases}  
      0 & (0 < x < k^n)\\ 
      k^{-n}(x-k^n) & (k^n \leq x \leq 1) 
    \end{cases}\hspace{2.5cm} \\
    \text{and}\quad  f_n & = g_n - h_n.
\end{align*}\pageoriginale
 Then $g_n, h_n \in C, f_n \in C - C$ and since
 $$
 \displaylines{\hfill 
 f_n(x) = 
 \begin{cases}  
   k^{-n} x & (0 < x < k^n) \\
   1 & (k^n \leq x \leq 1) 
 \end{cases} \hfill \cr
 \text{we have}\hfill  
 || f_n || = 1  \qquad (n = 1, 2, \ldots).\phantom{we have} \hfill }
 $$
 
 Also,
 $$
 Tf_n = f_{n-1} \qquad (n = 1, 2, \ldots), 
 $$
 and for $r > s$,
 $$
 || f_r - f_s || \geq f_r (k^r) - f_s (k^r) = 1 - k^{r-s} > 1 -k .
 $$
 It follows that no subsequence of $(Tf_n)$ converges.
 
 \begin{exam}%example 2
   A slight modification of the last example yields a less trivial
   application of Theorem \ref{chap5:thm5.2}. Let $C$ denote the class
   of continuous, increasing, convex functions $f$ on $[0, 1]$ with
   $f(0) = 0$, and let $\phi$ be an element of $C$ that satisfies 
   $$
   \phi (1) < 1, ~  ~ \phi^1_+ (0) > 0.
   $$
 \end{exam} 
 
 Then there exists an element $g$ of $C$ such that $g \circ \phi =
 \phi^1_+ (0) g$. ($f \circ g$ denotes the composition $(f \circ g) (x) =
 f(g(x))$). As before we take\pageoriginale $E = C_R [0 , 1]$, and
 consider the linear operator $T$ given by   
 $$
 Tf = f \circ \phi \qquad (f \in E)
 $$
 That Theorem \ref{chap5:thm5.2} is applicable is proved as in the
 last example, except for showing that the partial spectral radius
 $\mu$ is given by   
 $$
 \mu = \phi'_+ (0)
 $$
 To prove this we consider the sequence $(\phi_n)$ of functions
 defined by  
 $$
 \phi_n = T^n \phi \qquad (n = 1, 2, \ldots).
 $$
 Let $k = \phi$ (1). Then, for $f$ in $C$,
 $$
 || Tf (x)|| = (Tf) (1) = f (\phi (1)) \leq \phi (1) f(1), 
 $$
 and so $p(T) \leq k < 1$.
 $$
 \lim_{n \to \infty} \phi_n (1) = 0.
 $$
 Hence
 $$
 \lim_{n \to \infty} \frac{|| t^n \phi ||} {|| T^{n-1} \phi ||}=
 \lim_{n \to \infty} \frac{\phi_n(1)}{\phi_{n-1} (1)} = \lim_{n \to
   \infty} \frac{\phi(\phi_{n-1}(1))}{\phi_{n-1} (1)} = \phi'_+ (0),  
 $$
 and therefore $\lim\limits_{n \to \infty} || T^n \phi ||
 ^{\dfrac{1}{n}} = \phi'_+ (0)$, 
 $$
 \mu \geq \phi'_+ (0).
 $$
 Finally $(T^nf) (x) = f (\phi_n (x))$, so that
 $$
 p(T^n) \leq \phi_n (1),
 $$
 and\pageoriginale so $\mu \leq \phi'_+ (0)$. This completes the proof
 that Theorem \ref{chap5:thm5.2} is applicable. 
 
 In this particular example, we can calculate an eigenvector $g$ by an
 iterative process. In fact, if we take $g_n$ defined by 
 $$
 g_n (x) = \frac{\phi_n(x)}{\phi_{n} (1)} \qquad (0 \leq x \leq 1 ,
 \qquad n =1, 2, \ldots) 
 $$
 Then the sequence $(g_n)$ converges decreasingly to a function $g$
 with the required properties. 
 
\begin{exam}\label{chap5:exam3} %example 3
  Let $E$, $C$ and $\phi$ be defined as in the last example, and let
  $K(x, y)$ be a function continuous on the square 
  $$
  [0 , 1] \times [0 , 1]
  $$
  which belongs to $C$ as a function of $x$ for each fixed $y$ in $[0
    , 1]$. Let $T$ be the operator defined on $E$ by 
  $$
  (Tf) (x) = f(\phi (x)) + \int^1_0 K(x, y) f(y) dy \qquad (0 \leq x \leq 1).
  $$
\end{exam} 

Then $T$ is a bounded linear operator mapping $C$ into itself. $T$
maps $B$ into a compact set, and its spectral radius $\mu$ satisfies  
$$
\mu \geq \phi'_+ (0) > 0 .
$$
Thus Theorem \ref{chap5:thm5.2} is again applicable. 

\begin{exam} %example 4
  A variant on Example \ref{chap5:exam3} is given by
  $$
  (Tf) (x) = f(\phi (x)) + \int^x_0 k(y) f(y) dy
  $$
  where\pageoriginale $k$ is increasing non-negative and continuous in
  $[0,1]$. Again Theorem \ref{chap5:thm5.2} is applicable and so there
  exists a non-zero function $f$ in $C$ with 
  $$
  \mu f (x) -f (\phi (x)) = \int^x_0 k (y) f (y) dy \qquad (0 \le x \le 1),
  $$
  where $\mu$ is the partial spectral radius of $T$. From this we see
  that $\mu f (x) - f (\phi (x))$ is differentiable and we have a
  solution of the functional equation 
  $$
  \frac{d}{dx} \left\{ \mu f (x) - f (\phi (x))\right\} = k (x) f (x)
  $$
\end{exam} 

\begin{exam}%exam 5
  That the conclusion of Theorem \ref{chap5:thm5.2} need not hold if
  $\mu =0$ is 
  seen by taking the following example. Let $K (x,y)$ be continuous
  and non-negative in the square $[0,1] \times [0,1]$, and suppose
  that   
  $$
  K (1,y) > 0 \quad (0 \le y \le 1).
  $$
\end{exam}

Let $E = C_R [0,1]$, let $C$ consist of all $f \in E$ with
$$
f (x) \ge 0 \quad (0 \le x \le 1),
$$
and let $T$ be the Volterra operator defined by
$$
(Tf) (x) = \int^x_0 K (x,y) f (y) dy.
$$

Then $T$ satisfies the condition of Theorem \ref{chap5:thm5.2} except that its
spectral radius is zero (and hence its partial spectral radius is
zero). 

Also,\pageoriginale if $f \in C$ and $Tf =0$, we have 
$$
\int^1_0 K (1,y) f (y) dy =0,
$$
and so $f=0$.

\begin{Definition}%def 5.4
  Given a normed and partially ordered vector space $X$ with positive
  cone $K$, a non zero linear functional $f$ such that $f (x) \ge 0\;  (x
  \in K)$ is called a {\em positive continuous linear
    functional.} 
\end{Definition}

The following theorem is quite easily deduced from Theorem
\ref{chap5:thm5.1}.

\begin{thmm}\label{chap5:thm5.3}%theo 5.3
  Let $X$ be a normed and partially ordered vector space with a closed
  positive cone $K$, and suppose that there exists a subset $H$ of $K$
  with the properties: 
  \begin{enumerate}[(i)]
  \item Given $x \varepsilon X$ with $|| x || \le 1$, there exists $h
    H$ with $-h x h$, 
  \item $H$ is contained in a compact set.
  \end{enumerate}

    Let $T$ be a partially bounded positive linear operator in $X$
    with partial spectral radius $\mu$. 
 \end{thmm} 

Then there exists a positive continuous linear functional $f$ and a
real number $\mu^*$ with $0 \le \mu^* \le \mu$ such that 
$$
f (Tx) = \mu^* f (x) \quad (x \in X).
$$

If also $K$ is a normal cone, then $\mu = \mu^*$

\begin{proof}
  Let $X^*$ be the dual space of $X$, and let $C$ be the dual cone
  $K^*$ consisting of all $f$ in $X^*$ that satisfy 
  $$
  f (x) \ge 0 \quad (x \in K)
  $$\pageoriginale
  We have seen in Lemma \ref{chap4:lem4.1} (chapter \ref{chap4})
  that since $K$ is closed, 
  $K^* \neq 0$. By condition (i) in the theorem, $X =K-K$, and
  therefore $K^* \cap (-K^*) = (0)$. This proves that $C=K^*$ is
  indeed a positive cone. It is clearly a closed, and therefore
  complete, subset of the Banach space $X^*$. We take $E =C-C$. 
\end{proof}

Let $M=\sup \left\{ ||h|| : h \in H\right\}$, and as usual,
let $B= \{ f : f \in C $ and $ ||f|| \le 1\}$.  Given $f
\in B$ and $x \in X$ with $||x|| \le 1$, there exists
$h \in H$ with  
$$
-h \le x \le h,
$$
and therefore $-Th \le Tx \le Th$,
\begin{gather*}
  - f (Th) \le f (Tx) \le f (Th),\\
  |f (Tx)| \le f (Th) \le ||Th|| \le  p(T) ||h|| \le p (T). M,
\end{gather*}
where $p(T)$ denotes the partial bound of $T$. Thus for each $f
\in E$ we have an element $T^* f$ of $E$ given by 
$$
(T^* f) (x) = f (Tx) \quad (x \in X),
$$
and we obtain a partially bounded positive linear operator $T^*$ in
$E$ with partial bound $p(T^*)$ satisfying 
$$
p (T^*) \le M p(T).
$$
similarly\pageoriginale 
$$
p(T^{* n}) \le M p(T^n),
$$
and therefore 
$$
\mu^* \le \mu
$$
where $\mu^*$ denotes the partial spectral radius of $T^*$. We take
$\tau$ to be the weak topology in $E$. Plainly $C$ is $\tau$-closed,
$B$ is $\tau$-compact and $T^*$ is $\tau$-continuous. In order to
apply Theorem \ref{chap5:thm5.1} it only remains to prove that $\tau$ is
sequentially stronger than $\tau_N$ at 0 relative to $B$. To prove
this, suppose that $f_n \in B (n=1,2\ldots)$ and that 0 is
a $\tau$-cluster point of the sequence $(f_n)$.  

Since $H$ is contained in a (norm) compact set, given $\varepsilon >
0$. there exists $h_1 ,\ldots,h_r$ in $H$ such that for each $h
\in H$ there is some $k (1 \le k \le r)$ with  
\begin{equation}
  ||h-h_k|| <\frac{\varepsilon}{2} \tag{1}\label{chap5:equa1}
\end{equation} 
 
Since 0 is a weak $*$-cluster point of $(f_n)$, there exists an
infinite set $\Lambda$ of positive integers such that 
 \begin{equation*}
   |f_n (h_k)| < \frac{\varepsilon}{2} \quad (k=1,2,\ldots, r; n
   \in \Lambda) \tag{2}\label{chap5:equa2} 
 \end{equation*} 
 
 Therefore by (\ref{chap5:equa1}) and (\ref{chap5:equa2}) and the fact
 that $||f_n|| \le 1$, 
 \begin{equation*}
   |f_n (h)|< \varepsilon \quad (h \in H, n \in \Lambda)
   \tag{3}\label{chap5:equa3} 
 \end{equation*} 
 
 Given $x \in X$ with $||x|| < 1$, there exists $h \in
 H$ with $-h \le x \le h$, and so,\pageoriginale by (\ref{chap5:equa3}), 
 $$
 |f_n (x)| \le |f_n(h)| < \varepsilon \quad (n \in \Lambda)
 $$
 
 Therefore
 $$
 ||f_n|| \le \varepsilon \quad (n \in \lambda),
 $$
 and 0 is a $\tau_N$-cluster point of the sequence $(f_n)$.
 
 Suppose now that $K$ is a normal cone, i.e. for some $\gamma > 0$, 
 $$
 ||x + y|| \ge \gamma ||x|| \quad (x, \; y \in K).
 $$
 
 Then, for each point $x$ in $K$,
 $$
 d (x,-K) = \inf_{y \in K} ||x + y|| > \gamma ||x||
 $$
 
 Therefore, for each $x$ in $K$, there exists $f \in K^*$ with
 $||f|| \le 1$ and 
 $$
 f (x) \ge \gamma ||x||
 $$
 
 In particular, given $\varepsilon > 0$, there exists $x_0 \in
 K$ with $||x_0|| \le 1$, and 
 $$
 ||Tx_0|| > p (T)-
 $$
 
 Then there exits $f_0 \in B$ with
 $$
 f_0 (Tx_0) \ge \gamma ||Tx_0|| > \gamma (p(T)- \varepsilon).
 $$
 
 Therefore
 $$
 \displaylines{\hfill 
   ||T^* f_0|| > \gamma (p(T) - \varepsilon), \hfill \cr
   \text{and}\hfill p(T^*) \ge \gamma p(T). \hfill }
 $$\pageoriginale
 
 Since, similarly,
 $$
 \displaylines{\hfill p(T^{*{n}}) \ge \gamma p (T^n), \hfill \cr
   \text{we have} \hfill  \mu^* \ge \mu, \hfill }
 $$
 and the proof is complete.
 
 As a special case of Theorem \ref{chap5:thm5.3}, we have

 \begin{thmm}[Krein and Rutman ]%theo 5.4
   Let $X$ be a normed and partially ordered over space with a closed
   normal positive cone $K$ with non-empty interior. Let $T$ be a
   positive linear operator in $X$. Then 
   \begin{enumerate}[\rm(i)]
   \item $T$ is a bounded linear operator in $X$. 
   \item There exists a positive continuous linear functional $f$ such that
   \end{enumerate}
   $$
   f (Tx) = \rho f(x) \quad (x \in X),
   $$
   where $\rho$ is the spectral radius of $T$.
 \end{thmm} 

\begin{proof}
  Since $K$ has non-empty interior, there exists a point $e$ of $K$
  such that  
$$
||x|| \le 1 \Rightarrow e + x \in K
$$
since\pageoriginale $||-x|| \leq ||x||$, we have
\begin{equation*}
||x|| \le 1 \quad e \pm x \in K \Rightarrow -e \le x \le e
\tag{1}
\end{equation*}
Thus conditions (i) and (ii) of Theorem \ref{chap5:thm5.3} are
satisfied with \break 
$H=(e)$. 
\end{proof}

Give a positive linear operator $T$ and $||x|| \le 1$, we have $-e \le
x \le e$, and so $-Te \le Tx \le Te$. Since $K$ is a normal cone,
there exists a positive constant $\gamma$ with 
 \begin{equation*}
   ||y + z|| \ge \gamma ||y|| \quad (y,z \in K )\tag{2}
 \end{equation*} 
 
 We have $Te \pm Tx \varepsilon K$, and so
 $$
 \displaylines{\hfill 
   2 ||Te|| = ||(Te + Tx) + (Te -Tx)|| \ge \gamma ||Te -Tx|| \ge
   \left\{||Tx||-||Te||\right\},\hfill \cr 
   \text{and so} \hfill 
   ||Tx|| \leq \frac{2+\gamma} {\gamma} ||Te||,\hfill }
 $$
 which proves that $T$ is bounded, and also gives
 \begin{equation}
   ||T|| \le \frac{2+\gamma} {\gamma} ||Te|| \tag{3}
 \end{equation} 
 
 It follows from (3) that
 $$
 ||T|| \le \frac{2+\gamma}{\gamma} ||e|| \; p(T),
 $$
 and similarly, for any positive integer $n$,
 $$
 ||T^n|| \le \frac{2+\gamma}{\gamma} ||e|| \; p (T^n),
 $$
 
 Therefore\pageoriginale
 $$
 \rho = \lim_{n \to \infty} ||T^n||^{1/n} \leq \lim_{n \to \infty}
 p(T^n)^{1/n} = \mu 
 $$
 On the other hand it is obvious that $\mu \le \rho$, and so the
 theorem now follows from Theorem \ref{chap5:thm5.3}. 
 
 As a special case of Theorem \ref{chap5:thm5.2} we have the following
 theorem. 
 
 \begin{thmm}[Krein and Rutman theorem
     \ref{chap6:thm6.1}]\label{chap5:thm5.5}%theo 5.5 
   Let $X$ be a partially ordered Banach space with a positive cone
   $K$ such that $X$ is the closed linear hull of $K$. Let $T$ be a
   compact linear operator in $X$ that maps $K$ into itself and has a
   positive spectral radius $\rho$. Then there exists a non-zero
   vector $u$ in $K$ and a positive continuous linear functional $f$
   such that 
   $$
   Tu = \rho u, T^* f = \rho f.
   $$
 \end{thmm} 
 
 The proof depends on the following lemma concerning  compact liner
 operators. 
 
 \begin{lemma}\label{chap5:lem5.4}%lemm 5.4
   Let $T$ be compact linear operator in a normed space $X$, and let
   $T$ have a positive spectral radius. Then there exists a vector $x$
   in $X$ such that 
   $$
   \underset{n \to \infty}{\text{\rm limsup} } \; ||T^n||^{-1} ||T^n x|| > 0.
   $$
 \end{lemma} 

 \begin{proof}
   Let $\varepsilon > 0$, and suppose that the lemma is
   false. Then,for each $x$ in $X$ there exists a positive integer
   $N_x$ such that  
   $$
   n \ge N_x \Rightarrow ||T^n x|| <\frac{\varepsilon}{2} ||T^n||
   $$\pageoriginale

   Also
   $$
   \displaylines{\hfill 
     ||x'-x|| < \frac{\varepsilon}{2} \Rightarrow ||T^n x' -T^n x||
     <\frac{\varepsilon}{2} ||T^n||, \hfill \cr 
     \text{and so} \hfill 
     ||T^n x'|| <\varepsilon ||T^n|| \; (n \ge N_x,|| x' -x|| <
     \frac{\varepsilon}{2})\hfill } 
   $$
Let $S$ denote the closed unit ball in $X$.Then $\overline{TS}$ is
compact and so has a finite covering by open balls of radius
$\frac{\varepsilon}{2}$ and centers $x_1,\ldots x_m$ say. Let 
$$
N \max (N_{x{_1}},\ldots,N_{x{_m}}).
$$

Then
\begin{gather*}
  ||T^n x|| < \varepsilon ||T^n|| \; (n \ge N, x \in TS),\\
  ||T^{n+1} x|| \le \varepsilon ||T^n|| \; (n \ge N, x \in S),
\end{gather*}
and so
$$
||T^{n+1}|| \le \varepsilon ||T^n|| \; (n \ge N),
$$
from which it follows that $\lim\limits_{n \to \infty} ||T^n||^{1/n}
\leq \varepsilon$. 
 \end{proof} 

\setcounter{proofofthm}{4}
 \begin{proofofthm}%%% 5.5
By Lemma \ref{chap5:lem5.4} and the fact that $X=\overline{K-K}$,
there exists a vector $x$ in $C$ with 
$$
\lim_{n \to \infty} \sup ||T^n||^{-1} ||T^n x|| > 0.
$$
\end{proofofthm} 

It easily\pageoriginale follows from this that
$$
\mu \lim_{n \to \infty} p(T^n)^{1/n} = \lim_{n \to \infty} ||T^n||^{1/n} = \rho
$$
Applying Theorem \ref{chap5:thm5.2} with $E=X$ and $C =X$ we see that
there exists a non-zero vector $u$ in $K$ with  
 $$
 Tu = \rho u
 $$
 As in the beginning of the proof of Theorem \ref{chap5:thm5.3}, the
 set $K^*$ of  continuous linear functions $f$ with 
 $$
 f (x) \ge 0 \quad (x \in K)
 $$
is a positive cone in the dual space $X^*$. Also, $T^*$ is a compact
linear operator in $X^*$ and maps $K^*$ into itself. Applying Theorem
\ref{chap5:thm5.2} with $E =X^*$ and $C=K^*$, we conclude that there
exists a positive continuous linear functional $f$ with 
 $$
 T^* f = \mu^* f,
 $$ 
 where $\mu^*$ is the partial spectral radius of $T^*$. Since the
 spectral radius of $T^*$ is equal to that of $T$, we have $\mu^* \le
 \rho$. It only remains to prove that $\mu^* \ge \rho$. There exists
 $u \in K$ with $||u|| =1$ and $Tu =\rho u$. We have 
 $$
 T^n_u = \rho^n u \quad (n=1,2,\ldots).
 $$
 
 Since $\delta = d (u,-K) >0$, there exists $\phi$ in $K^*$ with
 $||\phi|| \le |$ and\pageoriginale $\phi (u) = \delta$. Then 
 $$
 \phi (T^n u) = \phi (\rho^n u) = \rho^n \delta,
 $$
 and so we have in turn $(T^{*^{n}} \phi) (u) = \rho^n \delta$,
 \begin{gather*}
   ||T^{*^{n}} \phi|| \ge \rho^n \delta,\\
   p(T^{*^{n}}) \ge \rho^n \delta, \qquad (n =1,2,\ldots)\\
   \mu^* \ge \rho
 \end{gather*}
 
\begin{remark*}
  It is in fact enough in Theorem \ref{chap5:thm5.5} that $K$ be a
  complete cone in a normed space (rather than a closed cone in a
  complete space).  
\end{remark*}
  
In our next theorem we give a formula for the calculation of positive
eigenvectors corresponding to $\rho$ valid under the conditions of
Theorem \ref{chap5:thm5.5}. For its proof we shall need some result from the
classical Riesz Schauder theory of compact operators in Banach
spaces. We shall state these results without proof (see Dunford and
Schwartz \cite{key14}). 
 
 Let $T$ be a compact linear operator in a complex Banach space
 $X$. The spectrum $\sigma (T)$ is the set of all complex numbers
 $\lambda$ for which $\lambda I -T$ is not a one-to-one mapping of $X$
 onto itself. Then $\sigma (T)$ is a countable set contained in the
 disc $| \zeta|| \le \rho$ where $\rho = \lim\limits_{n \to \infty}
 ||T^n||^{1/n}$. Also each element of $\sigma (T)$ other than $0$ is
 an eigenvalue, and is an isolated point of $\sigma (T), i.e$. has
 a\pageoriginale  neighbourhood containing no other point of $\sigma
 (T)$.  
 
 Let $\lambda$ be a non-zero point of $\sigma (T)$. Then there is a
 positive integer $\gamma = \gamma (\lambda)$ called the \textit{index} of
 $\lambda$ which is the smallest integer $n$ with the property that 
 $$
 (T -\lambda I)^{n+1} x=0 \Rightarrow (T -\lambda I)^n x =0.
 $$
 [The null space of $(T -\lambda I)^n$ increases with  $n$, but
   eventually we come to an integer after which it remains constant.]
 Let 
$$N=\left\{ x: (T-\lambda I)^\gamma x=0\right\},$$ and let 
 $$
 M=(T-\lambda I)^\gamma X.
 $$
 
 Then $N$ and $M$ are closed subspaces of $X$ and
 \begin{equation}
   X=N \oplus M \tag{1}
 \end{equation} 
 (i.e., each vector is $x$ has a unique expression $x=y+z$ with $y
 \in N$ and $z \in M$.) 
 
 Also $M$ and $N$ are invariant subspaces for $T$,
 $$
 T (M) \subset M, \quad TN \subset N.
$$

There exist bounded linear projections $P$ and $Q$ orthogonal to each
other and with ranges $M$ and $N$ respectively. 
\begin{align*}
  & I = P + Q , P Q = Q P = 0, P^2 = P, Q^2 = Q \tag{2}\\
  & P X = M , Q X =N.
\end{align*}

Let\pageoriginale $T_M$ denote the restriction of $T$ to $M$. Then
$T_M$ is a compact linear operator in $M$ and  
\begin{equation}
  \alpha \neq \lambda, \alpha \in \sigma (T)
  \Longleftrightarrow  \alpha \in \sigma (T_M)\tag{3} 
\end{equation}

\begin{thmm}\label{chap5:thm5.6}%the 5.6
  Let $X, K,  T, \rho$ satisfy the conditions of Theorem
  \ref{chap5:thm5.5} and let $\gamma$ be the index of $\rho$. Let $Q$
  be the projection onto the null space of $(T - \rho I)^\gamma$ which
  is orthogonal to the range of $(T -\rho I)$. Then  
  \begin{enumerate}[\rm(i)]
  \item $\lim\limits_{n \to \infty} (1+ \rho)^{-n} \begin{pmatrix} n
    \\\gamma-1\end{pmatrix}^{-1} (I+T)^n = (1+\rho)^{1-\nu} (T-\rho I)^{-1}
    Q$, 
  \item $ \lim\limits_{n \to \infty} ||(I + T)^n||^{-1} (I+T)^n =
    ||(T-\rho I)^{\nu-1} Q ||^{-1} (T-\rho I)^{-1} Q$, 
  \end{enumerate}
  the convergence being with respect to the operator norm.
\end{thmm}

\begin{proof}
  Since there is a natural isometry between bounded linear operator in
  a real Banach space and their complexifications, and since this
  isometry preserves compactness of operators, there is no loss of
  generality in supposing that $X$ is a complex Banach space. 
\end{proof}

Taking $\lambda = \rho$ in the above considerations, we have
continuous linear projections $P,Q$ on to the range $M$ and null space
$N$ of $A^\gamma, \gamma$ being the index of $\rho$, where $A = T -\rho I$,
and (1), (2), (3) hold. 

Suppose $\alpha$ is a non-zero point in the spectrum of $P+TP$. We
prove that $\alpha -1$ is in the spectrum of $T_M$ (the restriction of
$T$ to the subspace $M$.) In fact 
$$
S = \alpha I - (P + T P)
$$\pageoriginale
is not a $(1-1)$ mapping of $X$ onto itself. Either
\begin{enumerate}[i)]
\item the mapping $S$ is not $(1-1)$, 

or
\item the range of the mapping $S$ in not the whole of $X$.
\end{enumerate}

In case (i) there exists a non-zero vector $x$ in $X$ with 
$$
(P + TP)x = \alpha x.
$$

It follows that $x \in PX = M$, $x = Px$, and so
$$
(I+T)x = \alpha x, \; Tx = (\alpha -1)x, \; \alpha-1 \in \sigma 
(T_M). 
$$

In case (ii), since $SM \subset M,SN \subset N$ and $X = M+N$,
either $SM \neq M$ or $SN \neq N$. But as $P$ is zero on $N$, and
$\alpha \neq 0$, we have $SN = N$. Hence $SM \neq M$. But 
$$
Sx = \{(\alpha-1) I -T \}x, \quad (x \in M)
$$
and so $\alpha-1 \in \sigma (T_M)$. It follows from this and
(3) that if $\alpha$ is a non-zero point of the spectrum of $P+TP$,
then 
$$
\alpha -1 \in \sigma (T) \ (\rho).
$$

Therefore
$$
\lim_{n \to \infty} || (P+TP)^n ||^{1/n} \le \sup \left\{ |1+\zeta| |
\zeta \in \sigma (T), \zeta \neq \rho\right\} 
$$
Since all points $\zeta$ of $\sigma(T)$ satisfy $|\zeta| \le \rho$,
and since $\rho$ is as isolated\pageoriginale point of $\sigma (T)$,
it follows that   
$$
\lim_{n \to \infty} ||(P+TP)^n||^{1/n} = k < 1 + \rho
$$
since $TM \subset M$, we have $PTP = TP$, and so 
$$
\lim_{n \to \infty} ||(I+T)^n P ||^{1/n} = k.
$$

We choose $\eta$ with $k < \eta < 1 + \rho$. Then there exists $n_0$ with
\begin{equation}
||(I + T)^n P || < \eta^n \quad (n \ge n_0) \tag{1}
\end{equation}

We have
\begin{align*}
  I+T & = (1+ \rho) I+A \hspace{3cm}\\
  \text{and} \hspace{3cm} A^n Q &  = 0 \; (n \ge \nu).
\end{align*}

Hence
{\fontsize{10}{12}\selectfont
$$
(I + T)^n Q = \left\{ (1 +\rho)^n I + 
\begin{pmatrix} n\\  1 \end{pmatrix} 
(1+\rho)^{n-1} A + \cdots + 
\begin{pmatrix} n\\-1 \end{pmatrix} 
(1+\rho)^{n - \nu +1} A^{\nu-1}\right\} Q 
$$}\relax

It follows that
\begin{equation*}
  \lim_{n \to \infty} (1+\rho)^{-n} 
  \begin{pmatrix} n \\ \nu-1\end{pmatrix}^{-1} 
    (I + T)^n Q = (1+ \rho)^{1-\nu} A ^{\nu-1} Q \tag{2}
\end{equation*}

Also (1) gives,
\begin{equation*}
  \lim_{n \to \infty} (1+\rho)^{-n} 
  \begin{pmatrix} n \\ \nu-1\end{pmatrix}^{-1} 
    (I + T)^n P = 0\tag{3}
\end{equation*}
and,\pageoriginale since $(I + T)^n = (I+T)^n P + (I +T)^n Q$, (2) and
(3) give 
\begin{equation*}
  \lim_{n \to \infty} (1+\rho)^{-n} 
  \begin{pmatrix} n \\ \nu-1\end{pmatrix}^{-1} 
    (I + T)^n  = (1+ \rho)^{1-\nu} A^{\nu-1} Q \tag{4}
\end{equation*}

Taking norms, we have
\begin{equation*}
  \lim_{n \to \infty} (1+\rho)^{-n} 
  \begin{pmatrix}
 n \\ 
\nu-1\end{pmatrix}^{-1} 
    ||(I +T)^n || = (1+\rho)^{1-\nu} ||A^{\nu-1} Q || \tag{5}
\end{equation*}

By definition of $\nu$, the null space of $A^{\nu-1}$ is not the whole
of $N$, and so $A^{\gamma-1} Q \neq 0$. Thus (4) and (5) combine to
give (ii). 

Further results connected with Theorem \ref{chap5:thm5.5} are given by
Krein and Rutman \cite{key20}. In particular very precise results are
proved (Theorem \ref{chap6:thm6.3} \cite{key20}) for an operator $T$
which satisfies 
the conditions of Theorem \ref{chap5:thm5.5} and also maps each
non-zero point of $K$ into the  interior of $K$. In this case $\nu=1$,
and the result of Theorem \ref{chap5:thm5.6} takes the specially simple from   
$$
\lim_{n \to \infty} (1+\rho)^{-n} (I+T)^n = Q.
$$

Here $Q$ is a dimensional operator
$$
Qx = f (x) u ,
$$
where $u$ is a positive eigenvector of $T, f$ a positive eigenvector
of $T^*$ normalized by taking $f(u) = 1$. 
