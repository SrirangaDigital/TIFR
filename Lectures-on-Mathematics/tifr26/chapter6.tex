\chapter{Self-adjoint linear operator in a Hilbert space}

It would\pageoriginale be foolish of me to attempt to give in these
lectures an 
account of the many methods that have been developed for the study of
the spectral resolution of a self adjoint operator. I shall limit
myself to giving an account of a certain explicit formula for  the
projections belonging to the spectral resolution. In general this is a
theorem about projections; in the case of a compact operator it
becomes a theorem about eigenvectors. 

\begin{Definition}\label{chap6:def6.1}%def 6.1
  A complex (real) Hilbert space is a vector space over $C(R)$ with a mapping
  $$
  H \times H \to C (R)
  $$
  called the scalar product and denoted by $(x,y)$ which satisfies the
  following axioms 
  \begin{enumerate}[i)]
  \item $(x,y) = \overline{(y,x)}$

  \item $(x_1 + x_2,y) = (x_1,y) + (x_2,y) \; (x_1,x_2, y \in H)$

  \item $(\alpha x,y) = \alpha (x,y)\; (x,y \in H, \alpha
    \in C ( R))$

  \item $(x,y) > 0 $ for $x \neq 0$; $(x,x) =0$ for $x=0$ \; $(x \in H)$ 

  \item $H$ is a Banach space with the norm $||x|| = (x,x)^{\frac{1}{2}}$.
  \end{enumerate}
\end{Definition}

Let $H$ be a real or complex Hilbert space, and let $\mathscr{S}$
denote the class of all \textit{ bounded symmetric} operators in
$H$, i.e., bounded linear mappings of $T$ into itself such that  
$$
(Tx,y) = (x,Ty) \quad (x,y \in H)
$$ 
$\mathscr{S}$ is\pageoriginale the class bounded self adjoint
operators, $T =T^*$. A 
relation $\le$ is introduced into $\mathscr{S}$ by writing $A \le B$
or $B \ge A$ to denote that  
$$
(Ax,x) \le (Bx,x) \quad (x \in H).
$$

In particular, the operators $T$ belonging to $\mathscr{S}$ that satisfy
$$
T \ge 0
$$
are called \textit{ positive operators }.

Note that this definition of `positive' has nothing to do with the
property of mapping a cone into itself that we have considered in
earlier chapters. 

We establish a few elementary properties of the relation. First we
recall a few obvious properties of commutants. 

Let $B$ denote the class of all bounded linear operators in $H$, and
given $E \subset B$, let 
$$
E' = \{ T : T \in B \text{ and } AT=TA (A \in E)\}
$$
$E'$ is called the \textit{commutant} of $E$.
\begin{enumerate}[(i)]
\item $E_1 \subset E_2 \Rightarrow E'_2 \subset E'_1$ \qquad (obvious)
\item $E'$ is strongly closed.

Let $T_n \in E'$ converge strongly towards $T \in B$. Then $T_n A = A
T_n \quad (n=1,2, \ldots)$ 
$$
TA_x=\lim_{n \to \infty} T_n Ax= \lim_{n \to \infty} A T_n x = AT_x
\quad (x \in H) 
$$\pageoriginale

\item $E$ self-adjoint $\Rightarrow E'$ self-adjoint.
\begin{align*}
  T \subset E' \Rightarrow & AT = TA \quad (A \in E)\\
  & A^* T = TA^* \quad(A \in E)\\
  & T^* A = AT^* \quad (A \in E)
\end{align*}

\item $E'$ is a complex linear algebra $\qquad$(obvious)

\item $E = (T) \Rightarrow E''$ is a strongly closed commutative
  algebra containing $T$. 
  \begin{gather*}
    (T) \subset (T)' \Rightarrow (T)'' \subset (T)'\\
    A_1, A_2 \in (T)' \Rightarrow A_1 \in (T)'', \quad A_2 \in (T)'
    \quad A_1 A_2 = A_2 A_1. 
  \end{gather*}
\end{enumerate}

We next establish some well known elementary propositions concerning
positive operators. 
\begin{enumerate}[(a)]
\item For any positive operator $T$, the \textit{generalized Schwartz
  inequality} holds i.e., 
  $$
  |(Tx,y)|^2 \leq (Tx,x) (Ty,y)
  $$

  \begin{proof}
    $B(x,y)= (Tx,y)$ is a positive semi-definite symmetric bi-linear
    form and so the generalized Schwarz inequality for this form. 
  \end{proof}

\item If $T$ is a positive operator, then
  $$
  || T || = \sup \{(Tx,x): || x|| \leq 1 \}
  $$

  \begin{proof}
    Let\pageoriginale $T$ be a positive operator and let
    $$
    M =\sup \{(Tx,x) : || x || \leq 1 \}
    $$
    By the Schwarz inequality
    $$
    | (Tx,x)| \leq || Tx|| ~ ||x||,
    $$
    and so $M \leq || T||$. On the other hand putting $y=Tx$ in the
    generalized Schwarz inequality, we have 
    $$
    ||Tx ||^4 =(Tx,Tx)^2 = |(Tx,y)|^2 \leq (Tx,x)(Ty,y) \leq M^2 ||
    x||^2 ||Tx||^2, 
    $$
    and so $|| T|| \leq M$.
  \end{proof}

\item The set of positive operators is a positive cone in
  $\mathscr{S}$.

  \begin{proof}
    If $T$ and $-T$ are both positive, we have
    $$
    (Tx,n) = 0 \quad (x \in H)
    $$
    
    By $(b)$, this gives $T =0$. The other properties of the cone are obvious. 
  \end{proof}

\item Let $(T_n)$ be a bounded increasing sequence of elements of
  $\mathscr{S}$, i.e., 
  $$
  T_n \leq T_{n+1} \leq M.I \quad (n=1,2,\ldots)
  $$
  Then $(T_n)$ converges strongly to an element $T$ of $s$ i.e.
  $$
  \lim_{n \to \infty} T_n x = Tx \quad (x \in H)
  $$
\end{enumerate}

\begin{proof}
  For\pageoriginale $m < n$, let $A_{m,n} = T_n -T_m$. By the
  generalized Schwartz 
  inequality $(a)$ with $T= A_{m,n}$ and $y= A_{m,n} x$, we have $||
  A_{m,n} x ||^4 =  ( A_{m,n} x, ~ A_{m,n} x)^2 ~ = | (A_{m,n} x,y)
  |^2 \leq (A_{m,n} x,x) (A_{m,n} y,y)$. Since $0 \leq A_{m,n} \leq M
  I$, we have $(A_{m,n} y,y) \leq M^3 || x ||^2$. Hence $|| T_n x -
  T_m x ||^4 \leq M^3 || x ||^2\break \big \{ T_n x,x ) - T_m x,x ) \big
  \}$. Since the sequence $\big \{ (T_n x,x ) \big \}$ is a bounded
  increasing sequence of real numbers, it follows that $(T_n x)$ is a
  Cauchy sequence which converges to an element $Tx \in H$ in view of
  the axiom $(v)$ in definition \ref{chap6:def6.1}. 
  \begin{enumerate}[(e)]
  \item $T \ge 0 \Rightarrow T^n \ge 0 ~ ( n=1,2,\ldots )$.
  \end{enumerate}
\end{proof}

\begin{proof}
 $(T^{2K} x,x ) = ( T^K x, T^K x ) \ge 0$
 $$
 ( T^{2k+1} x,x )  = ~ ( T. T^x, T^k x ) \ge 0. 
 $$
  \begin{enumerate}[(f)]
\item Each positive operator $T$ is the square of a positive operator
  $T^{\frac{1}{2}}$, and $T^{\frac{1}{2}}$ belongs to the second
  commutant $(T)''$ of $T$. 
  \end{enumerate}
\end{proof}

\begin{proof}
  Suppose that $0 \leq A \leq I$, and let $B = I -A$, so that also 
  $$
  0 \leq B \leq I.
  $$
\end{proof}

Let $Y_n$ be the sequence defined inductively by 
$$
Y_0 = 0, ~ Y_{n+1} = \frac{1}{2} (B + Y_n^2) \; (n = 0,1,2, \ldots ).
$$
By induction we have $|| Y_n || \leq 1$, and so 
$$
0 \leq Y_n \leq I.
$$

Also,\pageoriginale since
$$
Y_{n+1} -Y_n = \frac{1}{2} (Y^2_n - Y^2_{n-1}) = \frac{1}{2} (Y_n +
Y_{n-1}) (Y_n - Y_{n-1}), 
$$
we see by induction that $Y_{n+1} -Y_n$ is a polynomial in $B$ with
non-negative real coefficients. Since $B^n \ge 0$, for every $n$, it
follows that $(Y_n)$ is an increasing sequence. Hence $(Y_n)$
converges strongly to a positive operator $Y$, and we have 
$$
0 \leq Y \leq I,~ Y = \frac{1}{2} (B + Y^2)
$$
Let $X= I- Y$. Then $X$ is a positive operator and 
$$
X^2 = A.
$$

If a bounded linear operator commutes with $A$, it commutes with each
polynomial in $A$, hence it commutes with $Y_n$, and therefore with
$X$. If $0 \leq T \leq M I$, $A = \dfrac{1}{M}.T$ satisfies $0 \leq A
\leq I$ and so the proposition holds for $T$. 

The positive square root $T^{\frac{1}{2}}$ is in fact unique but we
do not need this fact. 

\begin{enumerate}[(g)]
\item $A \ge 0, ~ B\ge 0, ~ AB = BA \Rightarrow AB \ge 0$.
\end{enumerate}

\begin{proof}
  Since $A \in (B')$, we have $B^{\frac{1}{2}} \in (A)'$ and so 
  $$
  AB = AB^{\frac{1}{2}} B^{\frac{1}{2}} = B^{\frac{1}{2}} AB^{\frac{1}{2}}.
  $$
\end{proof}

Finally, $( B^{\frac{1}{2}}  AB^{\frac{1}{2}} x,x ) = ~ (
AB^{\frac{1}{2}} x, B^{\frac{1}{2}} x) \ge 0$. 

$(h) ~ T \ge 0 \; ~I + T$ is invertible, $(I + T^{-1}) \ge 0$, and 
$$
(I + T)^{-1} \in  (T)''.
$$\pageoriginale

\begin{proof}
  We have  
  \begin{align*}
    & I  \leq I + T \leq (1 + M)I, \\
    & \frac{1}{1+M} \leq A \leq I, \\
  \end{align*}
  where $A = \dfrac{1}{1+M} (I + T)$. Therefore
  $$
  || I - A || \leq || \left(1- \frac{1}{1+M} \right) I || =
  \frac{M}{1+M} < 1. 
  $$
\end{proof}

Therefore the Neumann series
$$
I +  (I- A) + (I - A)^2 + \cdots
$$
converges in operator norm and since $(I- A)^k \ge 0$ its sum is a
positive operator $B$. We have  
$$
I = B (I -A) = I + (I -A) B = B,
$$
and so $AB = BA = I$.

Finally $( 1 + M)^{-1} B$ is the required positive inverse of $I +
T$. By a \textit{ projection } we mean an operator $P$ belonging to
$\mathscr{S}$ with $P^2 = P$. 
\begin{enumerate}[(i)]
\item Each projection $P$ satisfies $0 \leq P \leq I$.
\end{enumerate}

\begin{proof}
  Since $P \in \mathscr{S}$ and $P = P^2$, we have $P \ge 0$.
\end{proof}

Since $I - P \in \mathscr{S}$ and $(I - P)^2 = I -P$ we have $I - P \ge 0$.

\begin{enumerate}[(j)]
\item For projections $P_1, P_2$
  $$
  P_1 \ge P_2 \Longleftrightarrow P_2 = P_2 P_1 \Longleftrightarrow
  P_2 = P_1 P_2  
  $$\pageoriginale
\end{enumerate}

\begin{proof}
  $P_1 P_2 = P_2 \Rightarrow P_2 = P^*_2 = ( P_2 P_1)^*  = P^*_1 P^*_2
  = P_1 P_2$ 
  $$
  P_1 P_2 = P_2 ~ P_2 = P^*_2 = (P_2 P_1)^* = P^*_2 P^*_1 = P_2 P_1 .
  $$
\end{proof}

Thus, if $P_2 = P_2 P_1$, we also have  $P_1 P_2 = P_2$, and so 
$$
(P_1-P_2)^2 = P^2_1 -P_1 P_2 -P_2 P_1 + P^2_2 = P_1 -P_2,
$$
and therefore \qquad $P_1 \ge P_2$.

Finally suppose that $P_1 \ge P_2$. If $P_1 x = 0$, then $P_2 x = 0$,
for  
$$
(P_2 x, ~ P_2 x) = (P^2_2 x,x) = (P_2 x,x ) ~ \leq (P_1 x,x) = 0.
$$
Since $P_1 (I - P_1) = 0$,
$$
\displaylines{\hfill 
  P_2 (I -P_1) = 0 ,\hfill \cr
  \text{i.e.,} \hfill  P_2 = P_2 P_1.\hfill }
$$

\setcounter{section}{6}
\setcounter{lemma}{0}
\begin{lemma}%lemma 6.1
  Let $A \ge 0$, and let $B = 2 A^2 (I + A^2)^{-1}$. Then
  \begin{enumerate}[\rm i)]
  \item $B \in A''$,

  \item $0 \leq B \leq A$,

  \item $I -B = (I -A) (I + A) (I + A^2)^{-1}$,

  \item  if $P$ is a projective permutable with $A$ and $P \leq A$, 
    then $P \leq B$. 
  \end{enumerate}
\end{lemma}

\begin{proof}
  Proposition $(h)$ implies $(i)$.
\end{proof}

That $B \ge 0$ is clear since $A^2$ and $(I + A^2)^{-1}$ are
permutable.\pageoriginale Also 
$$
(I + A^2) (A-B) = A+A^3 -2A^2 = A (I -A)^2 \ge 0,
$$
and so, using the permutability of the operators,
$$
A-B = (I+A^2)^{-1} (I+A^2) (A-B) \ge 0 
$$
This proves (ii), and (iii) is straight forward.

Let $P$ be a projection such that $P \in A'$ and $P \leq A$. We have
$$
P = P^2 \leq PA ~ \leq A^2,
$$
and therefore
$$
P = P^2 \leq A^2 P.
$$

Therefore 
\begin{align*}
  (I+A^2) (B-P) & = 2A^2 - ~ (I+A^2) P \ge 2A^2 ~ - 2A^2 P\\ 
  & = 2A^2 (I - P) \ge 0
\end{align*}
Finally, since $(I + A^2)^{-1}$ is permutable with all the operators
concerned, 
$$
B - P \ge (I + A^2)^{-1} 2A^2 (I - P) \ge 0.
$$

\begin{thmm}\label{chap6:thm6.1} %theorem 6.1
  Let $A$ be a positive operator, and let the sequence $(A_n)$ be
  defined inductively by  
  $$
  A_1 = A, ~  A_{n+1} = 2A^2_n (I + A^2_n)^{-1} ~ \; (n=1, 2, \ldots)
  $$
  Then\pageoriginale
  \begin{enumerate}[\rm i)]
  \item $0 \leq A_{n+1} \leq A_n$  \; $ (n=1,2, \ldots)$,

  \item the sequence $(A_n)$ converges strongly to a projection $Q$
    belonging to $A''$. 

  \item $Q \leq A$,

  \item $(I-A) (I-Q) \ge 0$,

  \item $Q$ is maximal in the sense if $P$ is a projection permutable
    with $A$ and satisfying $P \leq A$, then $P \leq Q$. 
  \end{enumerate}
\end{thmm}

\begin{proof}
  (i) This follows at once from Lemma 1. (ii) and (iii). It
  follows from (i) and Proposition $(d)$ that $(A_n)$ converges
  strongly to a positive operator $Q$ with $Q \le A$, and that $Q \in
  (A)''$. It remains to prove that $Q$ is a projection.  
\end{proof}

Since $0 \leq A_n \leq A$, we have
$$
|| A_n ||\leq || A || \;  (n = 1,2,\ldots);
$$
and therefore, since
$$
\lim_{n \to \infty} ~ A_n x = Qx \;  (x \in H),
$$
we have in turn, 
\begin{gather*}
  \lim_{n \to \infty} A^2_n x = Q^2 x \;  (x \in H),\\
  \lim_{n \to \infty} A_{n+1} \big \{ (I + Q^2) x - (I + A^2_n) x \big
  \} ~ = 0  \;  ~ ~(x \in H),\\ 
  \lim_{n \to \infty} A_{n+1} (I + Q^2) x = \lim_{n \to \infty} 2A^2_n
  x ~ = 2Q^2 x  \; ~ ~ (x \in H). 
\end{gather*}

But\pageoriginale 
$$
\displaylines{\hfill 
  \lim _{n \to \infty} A_{n+1} (I + Q^2) x = Q (I + Q^2)x ~ \; ~ (x \in
  H),\hfill \cr
  \text{and so} \hfill  Q (I + Q^2)= 2Q^2. \hfill }
$$
Therefore
$$
(Q -Q^2)^2 = 0.
$$
But since $Q - Q^2$  is symmetric, this gives
$$
(Q -Q)^2 = 0,
$$
i.e.,  $Q$ is a projection.

(iv)~ By Lemma 1 (iii),
\begin{align*}
  (I - A_n) & = (I - A_{n-1}) (I + A_{n-1}) (I + A^2_{n-1})^{-1}, \\
  & = (I -A) \prod^{n-1}_{k = 1} (I + A_k) (I + A^2_k)^{-1};
\end{align*}
$$
\displaylines{\text{and so}\hfill 
  (I - A) (I- A_n) \ge 0 ~ \; ~ (n = 1,2,\ldots), \hfill \cr
  \text{which gives}\hfill 
  (I - A) (I - Q) \ge 0. \hfill }
$$

(v)~ Let $P$ be a projection permutable with $A$ such that $P \leq A$.

By\pageoriginale repeated application of Lemma 1  (iv),  $P$ is
permutable with $A_n$ and $P \leq A_n$. In the limit we have $P \leq
Q$.   

\begin{thmm} % theorem 6.2.
  Let $T$ be a bounded symmetric operator with 
  $$
  m I \leq T \leq M I.
  $$
\end{thmm}

Let $E_M = I$, and for $\lambda < M$ let $E_\lambda $ be the
projection $Q$ of Theorem $1$ corresponding to $A$ given by  
$$
A = \frac{1}{M- \lambda} (MI - T)
$$
Then
\begin{enumerate}[i)]
\item the projections $E$ belongs to $(T)''$;
\item $E_\lambda = 0 ~ ( \lambda < m)$, $E_M = I$; 
\item $E_\lambda \leq ~ E_\mu ~ ( \lambda \leq \mu)$;
\item $\lambda (E_\mu ~ -E_\lambda ) \leq T (E_\mu -E_\lambda)  \leq
  \mu (E_\mu - E_\lambda) ~ (\lambda \leq \mu)$; 
\item the maily $(E_\lambda)$ is strongly continuous on the right.
\end{enumerate}

\begin{proof}
  \begin{enumerate}[i)]
  \item follows from Theorem 1 ~ (ii).
  \item If $E_\lambda \neq 0$, there exists a non-zero $x_0$ with
    $E_\lambda x_0 = x_0$. 
  \end{enumerate} 
\end{proof}

By Theorem 1 (iii), we have 
$$
E_\lambda \leq \frac{1}{M- \lambda} (MI - T),
$$
and so 
$$
(x_0,x_0) = (E_\lambda x_0, x_0) \leq \frac{1}{M - \lambda} (( MI -T )
x_0,x_0). 
$$

Since $MI -T ~ \leq (M- m)I$, this gives
$$
(x_0,x_0) \leq \frac{M- m}{M- \lambda} (x_0, x_0),
$$\pageoriginale
and so $\lambda \ge m$. This proves (ii).
\begin{enumerate}[(iii)]
\item This is obvious except when $\lambda < \mu < M$. In this case,
  since $E_\lambda$ is a projection permutable with $\dfrac{1}{M-\mu}
  (MI-T)$, and  
\end{enumerate}
$$
E_\lambda \leq \frac{1}{M -\lambda} (MI - T) \leq \frac{1}{M - \mu}
(MI - T), 
$$
Then 1 (v) shows that 
$$
E_\lambda ~ \leq ~ E_\mu
$$
\begin{enumerate}[(iv)]
\item Suppose that $\lambda < \mu \leq M$. By Theorem 1 ~ (iv),
  $$
  \displaylines{\hfill 
  \left(I - \frac{1}{M - \lambda} (MI - T)\right) (I - E_\lambda) \ge
  0,\hfill \cr 
  \text{i.e.,} \hfill 
    (I - \lambda I) (I - E_\lambda) \ge 0. \hfill }
  $$
\end{enumerate}

Since $E_\lambda \leq E_\mu$, we have $E_\lambda E_\mu = E_{\lambda}$,
and so this gives  
$$
(T - \lambda I) (E_\mu - E_\lambda) \ge 0,
$$
which is the left hand inequality in (iv). The right hand inequality
is obvious if $\mu = M$ (since $T \leq MI$); and if $\mu < M$ we
have  
$$
\displaylines{\hfill 
  E_\mu \leq \frac{1}{M - \mu} (MI - T), \hfill \cr
  \text{i.e.,} \hfill T \leq MI - (M - \mu) E_\mu. \hfill }
$$

Since $E_\mu (E_\mu -E_\lambda) = E_\mu - E_\lambda$, this gives.
$$
T (E_\mu - E_\lambda) ~ \leq \mu (E_\mu - E_\lambda),
$$\pageoriginale
and (iv) is proved.

\begin{enumerate}[(v)]
\item Suppose $\mu < M$. If $(E_\lambda)$ is not strongly continuous
  on the right at $\mu$, there exists a sequence $( \lambda_n )$
  convergent decreasing to $\mu$ but such that $E_{\lambda_{n}}$ does
  not converge strongly to $E_\mu$. 
\end{enumerate}

Since $(E_{\lambda_{n}})$ is a decreasing sequence of operators, with
$E_{\lambda_{n}} \ge E_\mu$, there exists a positive operator $J$ such
that $J \ge E_\mu$ and $(E_{\lambda_{n}})$ converges strongly to
$J$. Then $J$ is a projection, permutable with $T$, and 
$$
E_\mu \leq J \leq E_{\lambda_{n}} ~ \leq \frac{1}{M - \lambda_n} (MI-
T) ~ \; (n=1,2,\ldots) 
$$
It follows that 
$$
E_\mu \leq J \leq \frac{1}{M - \mu} (MI- T),
$$
and so by the maximal property of $E_\mu$,
$$
J \leq E_\mu.
$$
This completes the proof of the theorem.

\begin{coro*}{\em(The spectral theorem).}
  $$
  T= \int \limits^{M}_{M- \varepsilon} \lambda d E_\lambda \; ~ (
  \varepsilon > 0) 
  $$ 
  The integral being the uniform limit of its Riemann-Stieltjes
  sums. In fact let 
  $$
  m - \varepsilon = \lambda_0 < \lambda_1 < \cdots ~ < \lambda_n = M.
  $$\pageoriginale
\end{coro*}

Then
$$
\lambda_{k-1} ( E_{\lambda_{k}} -E_{\lambda_{k-1}} ) \leq T	 (
E_{\lambda_{k}} - E_{\lambda_{k-1}} ) \leq \lambda_k (E_{\lambda_{k}}
- E_{\lambda_{k-1}}) 
$$
and so since $E_{\lambda_{n}} = I$ and $E_{\lambda_{0}} = 0$,
\begin{gather*}
  \sum^{n}_{k=1} \lambda_{k-1} ( E_{\lambda_{k}} - E_{\lambda_{k-1}} )
  \leq T  \leq \sum^{n}_{k=1} \lambda_k (E_{\lambda_{k}}
  -E_{\lambda_{k-1}}), \\ 
  0 \leq T - \sum^{n}_{k=1} \lambda_{k-1} (E_{\lambda_{k}}
  -E_{\lambda_{k-1}}) \leq \sum^{n}_{k=1} (\lambda_k - \lambda_{k-1})
  E_{\lambda_{k}} -E_{\lambda_{k-1}} ) \\ 
  \leq \max (\lambda_k -\lambda_{k-1}) I 
 \end{gather*} 
Hence 
$$
|| T - \sum^{n}_{k=1} \lambda_{k-1} (E_{\lambda_{k}} -
E_{\lambda_{k-1}}) || \to 0 ~\text{ as }~ \max (\lambda_k
-\lambda_{k-1}) \to 0  
$$
Moreover 
$$
T^r = \int \limits^{M}_{m- \varepsilon} \lambda^r d E_\lambda ~ \; (r=0,1,2,
\ldots) 
$$
To see this we rewrite (iv) in the form
$$
(M-\mu) (E_\mu - E_\lambda ) \leq (MI- T) (E_\mu - E_\lambda ) \leq
(M-\lambda) (E_\mu - E_\lambda) 
$$
since $MI - T \ge 0, ~ M - \lambda \ge 0$, and $M- \mu \ge 0$, it
follows that  
$$
(M-\mu)^r(E_\mu -E_\lambda) \leq (MI -T)^r (E_\mu - E_\lambda)  
 \leq (M-\lambda)^r  (E_\mu -E_\lambda) 
$$\pageoriginale
Therefore, as in the preceeding argument,
$$
(MI -T)^r = \int \limits^{M}_{m- \varepsilon} ~ (M- \lambda)^r
~dE_\lambda ~ ~ \; (r=0,1,2,\ldots) 
$$
and the required result follows by induction.

In the next theorem we $\infty$nsider the special simplification which
occurs when the operator is also compact. We need a simple lemma. 

\begin{lemma} %lemma 6.2
  Let $A$ be a positive operator, and let $(A_n)$ be the sequence
  constructed as in Theorem 7. Then $A_n = A^2 B_n \; (n=2,3,\ldots)$,
  where each $B_n$ belongs to $(A)''$ and  
  $$
  0 \leq B_{n+1} \leq B_n ~ ~ (n=2,3,\ldots)
  $$
\end{lemma}

\begin{proof}
  We have 
  $$
  A_2 = 2A^2 (I + A^2)^{-1} = A^2 B_2,
  $$
  with $B_2 = 2 (I + A^2)^{-1}$. If $A_n = A^2 B_n$ with $B_n \ge 0$
  and $B_n \in (A)''$, then 
  \begin{align*}
    A_{n+1} &= 2A^4 B^2_n (I + A^4 B^2_n)^{-1}\\
    &= A^2 B_{n+1},
  \end{align*}
  with $B_{n+1} = 2A^2 B^2_n (I + A^4 B^2_n)^{-1}$. Then
  \begin{align*}
    B_n - B_{n+1} &= (I + A^4 B^2_n)^{-1} \big \{ B_n (I + A^4 B^2_n)
    -2A^2 B^2_n \big \} \\ 
    &= (I + A^4 B^2_n)^{-1} ~ B_n ~ (I - A^2 B_n)^2,
  \end{align*}\pageoriginale
  so that \qquad $0 \leq B_{n+1} \leq B_n$.
\end{proof}

\begin{thmm}\label{chap6:thm6.3} % theorem 6.3
  Let $A$ be a compact positive operator, and let $(A_n)$ and  $Q$ be
  the corresponding and projection defined as in Theorem 1. Then  
  \begin{enumerate}[(i)]
  \item $(A_n)$ converges uniformly to $Q$;

  \item $Q$ has finite rank;

  \item the range of $Q$ is spanned by eigenvectors of $A$
    corresponding to eigenvalues $\lambda$ with $\lambda \ge 1$, and
    all such eigenvectors lie in the range of $Q$. 
  \end{enumerate}
\end{thmm}

\begin{proof}
  Let $B_n$ be the sequence defined in Lemma 2. Then $(B_n)$
  converges strongly to an operator $B$ in $(A)''$. Since $A_n = A^2
  B_n$ and $A_n$ converges strongly to $Q$, we have 
  $$
  Q = A^2 B.
  $$
\end{proof}

Let $K$ be the unit ball in the Hilbert space $H$, and let $E = ~
\overline{(AK)}$. Then $E$ is  a norm compact set. Since $\lim
\limits_{n \to \infty} ~ B_n x = Bx \; (x \in H)$, we have  
$$
\lim_{n \to \infty} (B_n x, x) = (Bx,x) \; ~ (x \in E).
$$

With respect to the norm topology in $E$, the functions $(B_n x ,x)$
are continuous real functions converging decreasingly to the
continuous\pageoriginale real function $(Bx,x)$. Therefore, by Dini's
theorem, the convergence is uniform on $E$. Since 
$$
(A_n x,x) = (A^2 B_n x,x) = (B_n Ax, Ax)
$$
It follows that  $(A_n x,x)$ converges uniformly on $K$ to $(BAx, Ax)
=$ \break  $(A^2 ~ Bx,x) = (Qx,x)$. Therefore $A_n \to Q$ uniformly
(i.e., with respect to the operator norm). 

Since $Q = A^2 B$ and $A$ is compact, we know that $Q$ is compact, and
therefore has finite rank, i.e., its range $H_Q$ is finite  dimensional;
for $Q$ is the identity operator in the Banach space $H_Q$ and a
Banach space in which a ball is compact is finite -dimensional.  

The rang $H_Q$ of $Q$ is a finite dimensional Hilbert space and $A$
maps $Q$ into itself (since $QA = AQ$). By the elementary theory of
symmetric matrices, $H_Q$ is spanned by eigenvectors $u_1, \ldots,u_r$
with real eigenvalues $\lambda_1, \ldots,\lambda_r$, 
$$
A_i ~ u_i = \lambda_i ~ u_i
$$

Since $Qu_i = u_i$, and  $A \ge Q$, we have
$$
\lambda_i (u_i,u_i) = (Au_i , ~ u_i) \ge (Qu_i , ~u_i) = (u_i,~ u_i),
$$
and so  $\lambda_i \ge 1$.

Conversely, let $u$ be an eigenvector of $A$ with $Au = \lambda u,
\lambda \ge 1$. We may suppose that $|| u || = 1$ and then define a
projection $P$ by\pageoriginale taking  
$$
Px = (x,u)u.
$$
$APx = (x,u)Au  = \lambda (x,u)u = (x, \lambda u)u = (x, Au)u = (Ax,u)
= PAx$. Also $P \leq A$, for given  $x \in H$, we have $x = H$, we
have $x = \xi u+v$ with  $(u,v) = 0$. Then $Ax = ~ \lambda \xi u +
Av$, and  
$$
(u, Av) = (Au,v) = \lambda (u,v) = 0. \text{ So }
$$
$(Ax,x) = ~ \lambda | \xi |^2 ~ (u,u) + (Av,v)^2 $
$$
\ge \lambda |\xi|^2 (u,u) = \lambda (Px,x) \ge  (Px,x).
$$
By the maximal property of $Q$, $P \leq Q$. Hence $P = QP$, $u \in H_Q$. 

The iterative method, given here can also be applied to construct the
projections belonging to the spectral family of an unbounded
self-adjoint operator, and details of this may be found in my paper. 
