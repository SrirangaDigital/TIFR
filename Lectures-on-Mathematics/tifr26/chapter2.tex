\chapter{Fixed point theorems in normed linear spaces}\label{chap2}

In Chapter\pageoriginale \ref{chap1}, we proved fixed  point
theorems in 
metric spaces 
without any algebraic structure. We now consider spaces with a linear
structure but non-linear mappings in them. In this chapter we restrict
our attention to normed spaces, but our main result will be extended to
general locally convex spaces in Chapter \ref{chap3}. 

\begin{Definition}%def 2.1
  Let $E$ be a vector space over. A mapping of $E$ into $R$ is called
  a norm on $E$ if it satisfies the following axioms. 
\end{Definition}

\begin{enumerate}[i)]
\item $p(x) \geq 0 \; (x \in E)$

\item $p(x) = 0$ if and only if $x=0$

\item $p(x+y) \leq p(x) + p(y) \; (x,y \in E)$.
\end{enumerate}

A vector space $E$ with a specified norm on it called a normed
space. The norm of an element $x \in E$ will usually be denoted by
$|| x||$. A normed space is a metric space with the metric $d(x,y)=
x-y \; (x,y \varepsilon E)$ and the corresponding metric topology is
called the 
normed topology. A normed linear space complete in the metric defined
by the norm is called a Banach space. We now recall some definitions
and well known properties of linear spaces. Two norms $p_1$ and $p_2$
on a vector  space $E$ are said to be equivalent if there exist
positive constants $k$, $k'$ such that 
$$
p_1 (x) \leq k p_2 (x) , \; p_2 (x) \leq k' p_1 (x)\; (x \in E)
$$
Two\pageoriginale norms are equivalent if and only if they define the
same topology. 

\begin{Definition}%def 2.2
  A mapping $f$ of a vector space $E$ into $R$ is called a linear
  functional on $E$ if it satisfies 
\begin{enumerate}[i)]
\item $f(x+y) = f(x) + f(y)\; (x,y \in E)$

\item $f (\alpha x)  = \alpha f (x) \; (x \in E, \alpha\in R )$. 
\end{enumerate}

\noindent
A mapping $p : E \to R$ is called a sublinear functional if 

\begin{enumerate}[i)$'$]
\item $p (x + y ) \leq p(x) + p(y) \; (x,y \in E)$

\item $p(\alpha x ) = \alpha p (x) \; (x \in E, \alpha \geq 0)$.
\end{enumerate}
\end{Definition}

\noindent
\textbf{Hahn-Banach Theorem}. Let $E_0$ be a subspace of a vector
space $E$ over $R$; let $p$ be a sublinear functional on $E$ and let 
$f_0$ be a linear functional on $E_0$ that satisfies 
$$
f_0 (x) \leq p (x)\; (x \in E_0).
$$

Then there exists a linear functional $f$ on $E$ that satisfies
\begin{enumerate}[i)]
\item $f(x) \leq p(x) \; (x \in E)$,

\item $f(x) = f_0 (x) \; (x \in E_0)$.
\end{enumerate}

\noindent
[For the proof refer to Dunford and Schwartz  (\cite{key14}, p. 62) or  Day
  \cite[p.9]{key13}]. 

\begin{coro*}%coro 0
  Given a sublinear functional on $E$ and $x_0 \in E$, there exists a
  linear functional $f$ such that 
  $$
  f (x_0) = p(x_0), \; f(x) \leq p(x) \; (x \in E).
  $$
\end{coro*}

In particular, a norm being a sublinear functional, given a point
$x_0$ of a\pageoriginale normed space $E$, there exists a linear
functional $f$ on $E$ such that  
$$
|f(x)| \leq ||x || \; (x \in E) \text{~ and~ } f(x_0) = x_0
$$

\begin{defi*}%def 0
  A norm $p$ on a vector space $E$ said to be strictly convex if $p
  (x+y) = p(x) + p(y)$ only when $x$ and $y$ are linearly dependent. 
\end{defi*}

\setcounter{thmm}{0}
\begin{thmm}[Clarkson]\label{chap2:thm2.1}%the 2.1
  If a normed space $E$ has a countable everywhere dense subset, then
  there exists a strictly  convex norm on $E$ equivalent to the given
  norm. 
\end{thmm}

\noindent \textit{Proof.}
  Let $S$ denote the surface of the unit ball in $E$,
  $$
  S = \{ x : ||x|| = 1 \}
  $$
  Then there exists a countable set $(x_n)$ of points of $S$ that is
  dense in $S$. For each $n$, there exists a linear functional $f_n$
  on $E$ such  that
  \begin{equation*}
  f_n (x_n) = || x_n|| =1 \text{~ and~ } |f_n (x)| \leq || x|| \; (x \in
  E).\tag*{$\Box$} 
  \end{equation*}

If $x\neq 0$, then $f_n (x) \neq 0$ for some $n$. For, by
homogeneity, it is enough to consider $x$ with $||x|| =1$, and for such
$x$ there exists $n$ with $|| x - x_n || < \dfrac{1}{2}$. But then 
\begin{gather*}
  f_n (x) = f_n (x_n) + f_n (x-x_n) \geq 1-f_n (x-x_n)\\
  \geq 1- || x-x_n|| > \frac{1}{2}
\end{gather*}

We now\pageoriginale take $p(x) = || x || + \left\{
\sum\limits_{n=1}^{\infty} 
2^{-n} (f_n (x))^2 \right\}^{\frac{1}{2}}$. It is easily verified
that $p$ is a norm on $E$ and that 
$$
||x|| \leq p (x) \leq 2 || x||.
$$

Finally $p$ is strictly convex. To see this, suppose that 
$$
p (x + y) = p(x) + p(y),
$$
and write $\xi _n = f_n (x), y_n = f_n (y)$. Then
$$
\left\{\sum_{n=1}^{\infty} 2^{-n} (\xi_n + \eta_n)^2 \right\}
^{\frac{1}{2}}= \left\{\sum_{n=1}^{\infty} s^{-n} \xi ^2_n \right\}
^{\frac{1}{2}}+\left\{\sum_{n=1}^{\infty} 2^{-n} \eta^2_n \right\}^{\frac{1}{2}} 
$$
and we have the case of equality in Minkowsiki's inequality. It
follows that the sequence $(\xi_n)$ and $(\eta_n)$ are linearly
dependent. Thus there exist $\lambda, \nu$, not both zero, such that 
$$
\lambda \xi _{n}+ \mu \eta _n = 0 \; (n=1, 2, \ldots)  
$$
But this implies that 
$$
f_n (\lambda x + \mu y) = 0 \; (n=1, 2, \ldots),
$$
and so $\lambda x + \mu y = 0$. This completes the proof.

\setcounter{section}{2}
\begin{lemma}\label{chap2:lem2.1}%lem 2.1x
  Let $K$ be a compact convex subset of a normed space $E$ with a
  strictly convex norm. Then to each point $x$ of $E$ corresponds a
  unique point $Px$ of $K$ at $K$ at minimum distance from $x$,
  i.e., with 
  $$
  || x - Px || = \inf \{ || x-y || : y \varepsilon K \}
  $$ 
  and the\pageoriginale mapping $x \to Px$ is continuous in $E$.
\end{lemma}

\begin{proof}
  Let $x \in E$, and let the function $f$ be defined on $K$ by $f (y)
  = || x-y||$. Then $f$ is a continuous mapping of the compact set $K$
  into $\mathbb{R}$ and therefore attains its minimum at a point $z$
  say of $K$ 
  \begin{equation*}
  || x - z || = \inf \{ || x-y || : y \in  K \}.\tag*{$\Box$}
  \end{equation*}

Evidently for $x \in K$, $z = x$ is uniquely determined. If $x \not\in
K$, suppose that $z'$ is such that 
\begin{equation*}
  0 \neq || x-z|| = || x-z'|| \tag{1}
\end{equation*}
since $K$ is convex, $y = \dfrac{1}{2} (z+z') \in K$ and therefore 
$$
|| x-y|| \geq || x-z|| = || x-z'|| = \frac{1}{2} || x-z|| +
\frac{1}{2} || x-z'|| 
$$
But 
$$
\displaylines{\hfill 
  (x - y) = \dfrac{1}{2} (x-z) + \dfrac{1}{2} (x-z'),\hfill \cr 
  \text{so that}\hfill 
  || x-y|| \leq \frac{1}{2}|| x-z'|| + \frac{1}{2} || x-z'||\hfill }
$$
Hence $|| x-y|| = || \dfrac{1}{2} (x-z) + \dfrac{1}{2} (x-z') || =
||\dfrac{1}{2} (x-z)|| + || \dfrac{1}{2} (x-z')||$ 

As the norm is strictly convex,
$$
\lambda (x-z) + \mu (x-z') = 0
$$
for $\lambda, \mu$ not both zero. By $(1), | \lambda | = | \mu|$ and
so $x-z = \pm (x-z')$. If $x-z = - (x-z')$, then $x= \dfrac{z+z'}{2}
\in K$, which is not true. Hence $x-z = x-z'$ or $z = z'$. This proves
that the mapping $x \to Px = z$ is uniquely on $E$. Given $x,x'\in E$, 
$$
|| x - Px || \le || x - Px' || \le || x - x' || + || x' - Px'||,
$$\pageoriginale
and similarly $|| x'- Px' | | \le || x -x' || + || x - Px ||$. So
\begin{equation*}
  |~~ || x - Px || - ||  x' - Px' ||~~ | \le  ||x - x'|| \tag{2}
\end{equation*}

Let $x_n \in E (n = 1,2, \ldots)$ converge to $x \in E$. Then the
sequence $Px_n$ in the compact metric space $K$ has a subsequence
$Px_{n_k}$ converging to $y \in K$. Then 
\begin{equation*}
  \lim\limits_{k \to \infty} || x_{n_k} - Px_{n_k} || = || x - y || \tag{3}
\end{equation*}

By (2) $|~~ || x_n - Px_n || - || x - Px ||~~| \le || x_n - x || \to
0$ as $n \to \infty$, and so $||x -y || = || x - Px ||$. Hence $Px = y
i.e. \lim\limits_{k \to \infty}  Px_{n_k} = Px$. Thus if $(x_n)$
converges to $x, (Px_n)$ has a subsequence converging to $Px$ ans so
every subsequence of $(Px_n)$ has a subsequence converging to
$Px$. Therefore $(Px_n)$ converges to $Px$ and $P$ is continuous. 
\end{proof}

\begin{Definition}%def 2.3
  The mapping $P$ of Lemma \ref{chap2:lem2.1} is called the metric
  projection onto $K$. 
\end{Definition}

\begin{Definition}%defi 2.4
  A subset $A$ of a normed space is said to be bounded if there exists
  a constant $M$ such that $||x|| \le M\; (x \in A)$. 
\end{Definition}
 
 We now state without proof three properties of finite dimensional
 normed spaces. 
 
 \begin{lemma}%2.2
 Every finite dimensional normed space is complete.
 \end{lemma} 

 \begin{lemma}\label{chap2:lem2.3}%lem 2.3
 Every bounded closed subset of a finite dimensional norm\-ed space is compact.
 \end{lemma} 

\begin{lemma}[Brouwer fixed point theorem]%lemm 2.4
  Let $K$ be a non-empty compact\pageoriginale convex subset of a
  finite dimensional normed space, and let $T$ be a continuous mapping
  of $K$ into itself. Then $T$ has a fixed point in $K$.  
\end{lemma} 

The proofs of the first two of these Lemmas are elementary. (Refer to
Dunford and Schwartz \cite[p. 244-245]{key14}.) The Brouwer fixed point
theorem on the other hand is far from trivial. For a proof using some
elements of algebraic topology refer to P. Alexandroff and H. Hopf
(\cite{key1}, p.376-378). A proof of a more analytical kind is given by
Dunford ans Schwartz (\cite{key14}, p.467). 

\begin{thmm}[Schauder]\label{chap2:thm2.2}%the 2.2
  Let $K$ be a non-empty closed convex subset of a normed space. Let
  $T$ be a continuous mapping of $K$ into a cumpact subset of
  $K$. Then $T$ has fixed point in $K$. 
\end{thmm} 

\begin{proof}
  Let $E$ denote the normed space and let $T K \subset A$, a compact
  subset of K. A is contained in a closed convex bounded subset of
  $E$. 
 $$
 T(B \cap K) \subset T (K) \subset A \subset B
 $$
 so $T(B \cap K)$ is contained in a compact subset of $B$, $K$ and there
 is no loss of generality in supposing that $K$ is bounded. If $A_0$
 is a countable dense subset of the compact metric space $A$, then the
 set of all rational linear combinations of elements of $A_0$ is a
 countable dense subset of the closed linear subspace $E_0$ spanned by
 $A_0$ and $A \subset E_0$ . Then $T (K \cap E_0) \subset T (K)
 \subset A$, a compact subset of $E_0$, and $K \cap E_0$ is closed and
 convex. Hence without loss of generality we may assume that $K$ is a
 bounded closed convex subset of a separable normed space $E$ with a
 strictly convex norm (Theorem \ref{chap2:thm2.1}). 
\end{proof}

 Given\pageoriginale a positive integer $n$, there exists a $\dfrac{1}{n}$-net
 $Tx_1, \ldots, Tx_m$ say in $T K $, so that 
\begin{equation}
  \min\limits_{1 \le k \le n} || Tx - Tx_k || < \frac{1}{n} \; (x \in K)
  \label{chap2:tag1} 
\end{equation} 
 
Let $E_n$ denote the linear hull of $Tx_1,\ldots, Tx_m$. $K_n = K \cap
E_n$ is a closed bounded subset of $E_n$ and therefore compact (Lemma
\ref{chap2:lem2.3}). Since the norm is strictly convex, the metric
projection $Pn$ of 
$E$ onto the convex compact subset $Kn$ exists. $T_n = P_n T$ is a
continuous mapping of the non-empty convex compact subset $Kn$ into
itself, and therefore by the Brouwer fixed point theorem, it has a
fixed point $u_n \varepsilon  K_n$, 
\begin{equation}
  T_n u_n = u_n \label{chap2:tag2}
\end{equation}
 
By (\ref{chap2:tag1}), since $Tx _k \in K_n \; (k = 1, 2, \ldots, m)$, we have
\begin{equation*}
  || Tx - T_n x || < \frac{1}{n} \tag{3}\label{chap2:tag3}
\end{equation*} 
 
The sequence $\{Tu_n\}$ of $T K $ has a subsequence $Tu_{n_k}$
converging to a point $v \in K$. By (\ref{chap2:tag2}) and
(\ref{chap2:tag3}), $|| u_{n_k} - v || 
= || T_{n_k} u_{n_k} - v|| \le || T_{n_k} u_{n_k} - Tu_{n_k} || + ||
Tu _{n_k} - v || < \dfrac{1}{n} + || Tu_{n_k} - v|| $. Therefore,
$\lim\limits_{k \to \infty} u_{n_k} = u$, and by continuity of $T,
\lim\limits_{k \to \infty} Tu_{n_k} = Tv$ or $Tv = v$. 

\begin{example*}%exam 0
  Suppose that a function $f(x, y)$ of two real variables is
  continuous on a neighbourhood of $(x_0, y_0)$. Then we can choose
  $\varepsilon > 0$ such that $f$ is continuous in the rectangle 
 $$
 |x -x_0| \le \varepsilon, |y - y_0| \le m \varepsilon
 $$\pageoriginale
 and satisfies there the inequality 
 $$
 |f (x, y)| \le m.
 $$
\end{example*}

 Let $E$ denote the Banach space $C_R [x_0 - \varepsilon$, $x_0 +
   \varepsilon]$, which is a Banach space with the 
 uniform norm 
 $$
 || \varphi || = \sup \bigg\{ | \Phi (t) | : |t - x_0| \le \varepsilon
 \bigg\} 
 $$ 
 
 Let $K$ be the subset of $E$ consisting of all continuous mappings of
 $[x_0 - \varepsilon, x_0 + \varepsilon]$ into $[y_0- m \varepsilon,
   y_0 + m \varepsilon]$. Then $K$ is a bounded closed convex subset
 of $E$. Let $T$ be the mapping defined on $K$ by 
 $$
 (T \phi) (x) = y_0  + \int^x_{xo} f (t, \phi (t)) dt \qquad (|x -x_0|
 \le \varepsilon) 
 $$
 
 Then $T K \subset K$. Also since
 $$
 \bigg| (T \phi) (x) - (T \phi) (x') \bigg| \le \bigg| \int^x_{x'}
 f(t, (t)) dt \bigg| \le m | x - x' |\; (\phi \in K), 
 $$
$T K$ is an equicontinuous set. Since also $T K $ is bounded, $T K$
 is contained in a compact set by the Ascoli - Arzela
 theorem. Therefore, by Theorem \ref{chap2:thm2.2}, $T$ has a fixed
 point $\phi$ in $K$ i.e., 
$$
\phi (x) = y_0 + \int^x_{x_0} f (t, \phi (t)) \, dt \; (|x - x_0| \le
\varepsilon). 
$$

Then $\phi$ is differentiable in $[x_0 - \varepsilon, x_0  +
  \varepsilon]$ and provides a solution $y = \phi (x)$ there of the
differential equation 
$$
\frac{dy}{dx} = f(x, y)
$$\pageoriginale
with $\phi (x_0) = y_0$. This is Peano's theorem. As a particular case
of Schauder's theorem, we have 

\begin{thmm}\label{chap2:thm2.3}%theo 2.3
  Let $K$ be a non-empty compact convex subset of a normed space, and
  let $T$ be a continuous mapping of $K$ into itself. Then $T$ has a
  fixed point in $K$. 
\end{thmm}

\begin{remark*}%rema 0
  Theorem \ref{chap2:thm2.2} and \ref{chap2:thm2.3} are almost
  equivalent, in the sense that Theorem 
  \ref{chap2:thm2.2}, with the additional hypothesis that $K$ be
  complete, follows 
  from Theorem \ref{chap2:thm2.3}. For, if $K$ is a complete convex
  set and $T K $ is 
  contained in a compact subset $A$ of $K$, then the closed convex
  hull of $A$ is a compact convex subset $K_0$ of $K$, and $T K_0
  \subset K_0$. 
\end{remark*}

\begin{Definition}%def 2.5
  A mapping $T$ which is continuous and maps each boun\-ded set into a
  compact set is said to be completely continuous. 
\end{Definition}

\begin{thmm}\label{chap2:thm2.4}%the 2.4
  Let $T$ be a completely continuous mapping of a normed space $E$
  into itself and let $T E$ be bounded. Then $T$ has a fixed point. 
\end{thmm}

\begin{proof}
  Let $K$ be the closed convex hull of $T E$. Then $K$ is bounded and
  so $T K$ is contained in a compact subset of $K$. By Theorem
  \ref{chap2:thm2.2}, 
  $T$ has a fixed point in $K$. 
\end{proof}

The Theorem \ref{chap2:thm2.4} implies Theorem \ref{chap2:thm2.3} is
seen as follows. Let $K$ be a 
compact convex set and let $T$ be continuous mapping of $K$ into
itself. There is no loss of generality is supposing that the norm in
$E$ is strictly convex. Let $P$ be the metric projection of $E$
onto\pageoriginale $K$, and let $\tilde{T} = TP$. Then $\tilde{T}$
satisfies the 
conditions of theorem 24, and so there exists $u$ in $E$ with $T u =
u$. Since $T$ maps $E$ into $K$, we have $u \in K$ and so $Pu = u Tu =
TRu = u$. 

\begin{lemma}\label{chap2:lem2.5}%lemm 2.5
  Let $K$ be a non-empty complete convex subset of a normed space $E$,
  let $A$ be a continuous mapping of $K$ into a compact subset of $E$,
  and let $F$ be a mapping of $K  \times K$ into $K$ such that 
\begin{enumerate}[(i)]
\item $|| F (x, y) - F (x, y) || \le k || y - y' || \; (x, y, y' \in
  K)$, where $k$ is a constant with $0 < k < 1$, 

\item $||F (x, y) - F(x', y)|| \le ||Ax - Ax'|| \; (x, x', y \in
  K)$. Then there exists a point $u$ in $K$ with 
  $$
  F(u, u) = u.
  $$
\end{enumerate}
\end{lemma}

\begin{proof}
  For each fixed $x$, the mapping $y \to F (x, y)$ is a contraction
  mapping of the complete metric space $K$ into itself, and it
  therefore has a unique fixed point in $K$ which we denote by $Tx$, 
  \begin{align*}
    Tx & = F(x, Tx) \qquad (x \in K).\\
    \text{We have}\, || Tx - Tx' || &= || F (x, Tx) - F(x', Tx')||\\
    & \le || F(x, Tx) - F(x', Tx)||\\
    &\qquad{}+ || F(x', Tx) - F(x', Tx') ||\\
    & \le || Ax - Ax' || + k || Tx - Tx'|| \tag*{$\Box$}
\end{align*}

Therefore $||Tx - Tx'|| \le \dfrac{1}{1 - k} || Ax - Ax' ||$, 
(1) which shows that $Tk$ is continuous and that $TK \subset K$ is
precompact since $AK$ is compact, since $K$ is complete,
$\overline{TK} \subset K$ is compact. By the Schander theorem, $T$ has
fixed point $u$ in $K$, 
$$
Tu = u.
$$\pageoriginale
But then
$$
F(u, u) = F(u, Tu) = Tu = u
$$
\end{proof}

\begin{thmm}[Kranoselsk\u{u}]%theo 2.5
  Let $K$ be a non-empty complete convex subset of  a normed space
  $E$, let $A$ be a continuous mapping of $K$ into a compact subset of
  $E$, let $B$ map $K$ and satisfy a Lipschitz condition 
  $$
  || Bx - Bx' || \le k || x - x' || \qquad (x, x' \in k)
  $$
  with $0 < k < 1$ and let $Ax + By \in K$ for all $x, y$ in $K$. Then
  there is a point $u \in K$ with 
  $$
  Au + Bu = u
  $$
\end{thmm}

\begin{proof}
  Take $F(x, y) = Ax + By$ and apply Lemma \ref{chap2:lem2.5}.
\end{proof}

\begin{coro*}%coro
  Let $K$ be a non-empty complete convex subset of a normed space, let
  $A$ be a continuous of $K$ into a compact subset of $K$, let $B$ map
  $K$ into itself ans satisfy the Lipschitz condition 
  $$
  || Bx - Bx' || \le || x - x' || \quad (x, x' \varepsilon K),
  $$
  and let $0 < \alpha < 1$. Then there exists a point $u \in K$ with
  $$
  \alpha Au + (1 - \alpha) Bu = u
  $$ 
\end{coro*}

In general, under the condition of Schauder's theorem, we have no
method for the calculation of a fixed point of a mapping. However
there is a special case in which this can be done using a method due
to Krasnoselsk\u{u}. 

\begin{Definition}%defi 2.5
  A norm $p$ is uniformly convex if it satisfies
  \begin{align*}
  & p (x_n) = p(y_n) = 1\;  (n = 1,2, \ldots), \lim\limits_{n \to \infty} 
  p(x_n + y_n)= 2\\ 
   \Longrightarrow & \lim\limits_{n \to \infty} p (x_n-y_n) =0. 
  \end{align*}\pageoriginale
\end{Definition}

\begin{lemma*}
  Let $p$ be a uniformly convex norm, and let $\varepsilon M$ be positive
  constants. Then there exists a constant $\delta$ with $0 < \delta <
  1$ such that 
  $$
  p(x) \le M, p(y) \le M, p(x - y) \ge \varepsilon \Rightarrow p (x + y) \le 2
  \delta \max (p(x), p (y)). 
  $$
\end{lemma*}

\begin{proof}
  For all $x$, $y$, we have
  \begin{equation*}
    p \left(\frac{1}{2} (x + y)\right) \le \frac{1}{2} p (x) +
    \frac{1}{2} p(y) \le \max (p(x), p(y)). \tag{1}\label{chap2:equ1} 
  \end{equation*}
  
  If there is no constant $\delta$ with the stated properties, there
  exist sequences $(x_n), (y_n)$ with $p(x_n) \le M$, $p(y_n) \le M$, 
\begin{equation*}
    p(x_n - y_n) \ge \varepsilon, \tag{2}\label{chap2:equ2} 
\end{equation*}
and
\begin{equation*}
 p \left(\frac{1}{2} (x_n + y_n)\right) > \left(1 - \frac{1}{n}\right)
 \max (p(x_n), p(y_n)).  \tag{3}\label{chap2:equ3}  
\end{equation*}

Let $\alpha_n = p (x_n)$, $\beta_n = p (y_n)$, $\gamma_n = \max (\alpha_n,
\beta_n)$. By (\ref{chap2:equ1}) and (\ref{chap2:equ2}), 
\begin{equation*}
  \gamma_n \ge \frac{1}{2}, \tag{4}\label{chap2:equ4}
\end{equation*}
and so, by (\ref{chap2:equ1}) and (\ref{chap2:equ3})
\begin{equation*}
  \lim_{n \to \infty} \frac{1}{\gamma_n} p \left(\frac{1}{2} (x_n +
  y_n)\right) = 1.  \tag{5}\label{chap2:equ5} 
\end{equation*}

It follows from (\ref{chap2:equ1}) and (\ref{chap2:equ5}), that
\begin{equation*}
\lim_{n \to \infty} \frac{\alpha_n + \beta_n}{2 \gamma_n} =
1. \tag{6}\label{chap2:equ6} 
\end{equation*}

Since $(\gamma_n)$ is bounded, there exists a convergent sequence
$(\gamma_{n_k})$, and by (\ref{chap2:equ4}) 
\begin{equation*}
  \lim_{k \to \infty} \gamma_{n_k} = \gamma \ge \varepsilon_{\gamma_2}
  \tag{7}\label{chap2:equ7} 
\end{equation*}
$$
\lim\limits_{k \to \infty} (\gamma_{n_k} - \alpha_{n_k}) +
(\gamma_{n_k} - \beta_{n_k}) = 0, 
$$\pageoriginale
and, since each bracket is non-negative, each tends to zero. Therefore
\begin{equation*}
  \lim_{k \to \infty} \alpha_{n_k} = \lim_{k \to \infty}
  \beta_{n_\alpha} = \gamma \tag{8}\label{chap2:equ8} 
\end{equation*}

By discarding some terms of the subsequence if necessary, we may
suppose that $\alpha_{n_k} \ge 0$ and $\beta_{n_k} \ge 0$ for all
$k$. Since 
\begin{align*}
\bigg| p \left(\frac{1}{\alpha_{n_k}} x_{n_k} + \frac{1}{\beta_{n_k}} y_{n
  _k} \right) & - p \left(\frac{1}{\gamma_{n_k}} x_{n_k} + \frac{1}{\gamma_{n_k}}
y_{n_k} \right) \bigg| \\
  & \le p \left(\frac{1}{\alpha_{n_k}} - \frac{1}{\gamma_{n_k}} \right) x_{n_k} +
  \left(\frac{1}{\beta_{n_k}} - \frac{1}{\gamma_{ n_k}}\right) y_{n_k}\\ 
  & \le M \left\{ \frac{1}{\alpha_{n_k}} - \frac{1}{\gamma_{n_k}} +
  \frac{1}{\beta_{n_k}} - \frac{1}{\gamma_{n_k}} \right\}, 
\end{align*}
it follows from (\ref{chap2:equ5}), (\ref{chap2:equ7}),
(\ref{chap2:equ8}), that 
$$
\lim_{k \to \infty} p \left(\frac{1}{\alpha_{n_k}} x_{n_k} +
\frac{1}{\beta_{n_k}} y_{n_k} \right) = 2. 
$$

Therefore,
$$
\lim_{k \to \infty} p \left(\frac{1}{\alpha_{n_k}} x_{n_k} -
\frac{1}{\beta_{n_k}} y_{n_k} \right) = 0.
$$
and so
$$
\lim_{k \to \infty} \frac{1}{\gamma_{n_k}} p(x_{n_k} - y_{n_k}) = 
0, 
$$
which contradicts (\ref{chap2:equ2}).
\end{proof}

\begin{thmm}[Krashoselsk\u{u}]\label{chap2:thm2.6}%theo 2.6
  Let\pageoriginale $K$ be a bounded closed convex set in a Banach
  space $E$ with a 
  uniformly convex norm. Let $T$ be a mapping of $K$ into a compact
  subset of $K$ that satisfies a Lipschitz condition with Lipschitz
  constant 1, and let $x_0$ be an arbitrary point of $K$. Then the
  sequence defined by  
  $$
  x_{n + 1} = \frac{1}{2} (x_n + Tx_n) \qquad (n =0, 1,2, \ldots )
  $$
  converges to a fixed point of $T$ in $K$.
\end{thmm}

\begin{proof}
  By Schauder's theorem, there is a nonempty set $F$ of fixed points
  of $T$ in $K$. We prove first that 
  $$
  || x_{n + 1} - y || \le || x_n - y || \quad (y \in F, n = 0, 1,2, \ldots)
  $$
  In fact if $y = T_y$, then
  \begin{align*}
  || x_{n + 1} - y || & = || \frac{1}{2} (x_n + Tx_n) - \frac{1}{2} (y
  + Ty) ||\\ 
    & = || \frac{1}{2} (x_n - y) + \frac{1}{2} (Tx_n - Ty) ||\\
    & \le \frac{1}{2} || x_n - y || + \frac{1}{2} || Tx_n - Ty ||\\
    & \le || x_n - y ||
  \end{align*}
  which is (1).
\end{proof}

Suppose that there exist an $\varepsilon > 0$ and $N$, such that
\begin{equation}
|| x_n - Tx_n || \ge \varepsilon \text{ for all } n \ge N \tag{2}
\end{equation}

Then $|| x_n - y - (Tx_n - T_y) || \ge \varepsilon$ for all $n \ge N, y \in F$.

Also $|| Tx_n - T_y || \le || x_n - y || \le || x_0 - y ||$, by \qquad (1).

Since the norm is uniformly convex, this implies that there exists a
constant $\delta, 0 < \delta < 1$, such that 
\begin{align*}
  || x_{n + 1} - y || & = || \frac{1}{2} (x_n + Tx_n) - \frac{1}{2} (y
  + Ty) ||\\ 
  & = || \frac{1}{2} (x_n - y) + \frac{1}{2} (Tx_n - Ty)||\\
  & \le \max \left\{ ||x_n - y ||, || Tx_n - Ty || \right\}\\
  & \le || x_n - y || \text{ for } n \ge N.
\end{align*}
Therefore\pageoriginale $\lim\limits_{n \to \infty} x_n = y$ where
$T_y = y$. 

If there does not exist an $\varepsilon > 0$ for which (2) holds, there
exists a sequence $n_k$ of integers such that $\lim\limits_{k \to
  \infty} (x_{n_k} - Tx_{n_k}) = 0$, and such that $(Tx_{n_k})$
converges. But this implies that $\lim\limits_{k \to \infty} x_{n_k} =
u = \lim\limits_{k \to \infty} Tx_{n_k}$ and so $Tu = u$. 

Hence $|| x_{n + 1} - u || \le || x_n - u||$, by (1). Since
$\lim\limits_{k} || x_{n_k} - u || = 0$, we have $\lim\limits_{n \to \infty}
|| x_n - u || = 0$ and the theorem is proved. 

The following theorem was proved by Altam by means of the concept of
`degree of a mapping', but we can easily deduce it from schander's
theorem. 

\begin{thmm}[Altman]%theo 2.7
  Let $E$ be a normed space, let $Q$ be the closed ball of radius $r > 0$,
  $$
  Q = \left\{ x : || x || \le r \right\}
  $$
  and let $T$ be a continuous mapping of $Q$ into a compact subset of
  $E$ such that 
  $$
  || Tx - x||^2 \ge || Tx ||^2 - || x ||^2 \quad ( || x || = r) 
  $$
Then $T$ has fixed point in $Q$.
\end{thmm}

\begin{proof}
  Suppose $T$ has no fixed point in $Q$, then
  \begin{equation*}
    || Tx - x || + || x || > || Tx || \qquad ( || x || = r) \tag{1}
  \end{equation*}
  For
  \begin{gather*}
    (|| Tx - x || + || x ||)^2 - || Tx ||^2 = || Tx - x ||^2 + || x ||^2
    - || Tx ||^2\\ 
    + 2 || x || \; || Tx - x || \ge 2r || Tx - x || > 0
  \end{gather*}
  
  Let\pageoriginale $P$ be the mapping defined by 
  $$
  Px = 
  \begin{cases}
    x  \qquad (x \in Q)\\
    \frac{r}{ || x ||} x \quad (x \notin Q)
  \end{cases}
  $$  
Plainly $P$ is a continuous projection of $E$ onto $Q$.
\end{proof}

Let $\tilde{T} = PT$

Then $T$ maps $Q$ continuously into a compact subset of $Q$. Hence, by
the Schauder theorem, $\tilde{T}$ has a fixed point $u$ in $Q$, 
$$
PTu = u
$$
If $Tu \in Q$, then $PTu = Tu$, and 
$$
Tu = u 
$$
If $Tu \not\in Q$, then $||Tu || > r$ and 
$$
u = PTu = \frac{r}{ || Tu ||} Tu
$$
If follows that $|| u || = r$, and we have
\begin{gather*}
  || Tu - u || + || u || = || \frac{|| Tu ||}{r} u - u || + || u ||\\
  = \left( \frac{|| Tu ||}{r} - 1 + 1 \right) || u || = || Tu ||
\end{gather*}
which contradicts (1)

\medskip
\begin{center}
\textbf{supplementary results and exercises}
\end{center}

\begin{enumerate}[(1)]
\item For further results connected with Theorem \ref{chap2:thm2.6},
  see \cite{key19} 

\item Let $A$ be a continuous mapping of a normed space $E$ into
  itself which maps bounded sets into compact sets and satisfies 
  $$
  \lim_{|| x || \to \infty} \frac{|| Ax ||}{|| x ||} = 0
  $$
  Then given arbitrary real $\lambda > 0$ and $y$ in $E$, the equation
  $$
  x = \lambda Ax + y
  $$
  has a\pageoriginale solution $x$ in $E$
  
  Consider the mapping $Tx = \lambda Ax + y$
  
  Clearly $T$ has all the properties of $A$
  
  Let $S_n = \{x : x \in E, || x || \le n \} \quad (n = 1, 2, \ldots)$ 

Then
\begin{equation*}
T S_n \subset S_n\text{~~ for some $n$ }\tag{1}\label{eq1}
\end{equation*}

Otherwise $||Tx_n|| > n$, for some 
\begin{equation*}
x_n \in S_n ~~ n = 1,2, \ldots\tag{2}\label{eq2}
\end{equation*}

 If $\{x_n\}$ were bounded, then $\{Tx_n\}$ will be contained in a
 compact set and therefore $||T x_n ||$ will be bounded which
 contradicts (2). Hence $|| x_n || \to \infty$ as $n \to \infty$ But  
 $$
 \frac{|| Tx_n ||}{|| x_n ||} > 1 \text { so } \lim\limits_{|| x_n ||
   \to \infty} \frac{|| Tx_n ||}{|| x _n ||} \ge  1 
 $$
 
 As this is not true, $T$ maps some $S_n$ into its compact subset;
 Schau\-der's theorem then $a$ gives a fixed point $x$ which is the
 required solution. 

 \item Let $\sum\limits^{\infty}_{k = 1} a_k$ be a convergent series
   of non-negative real numbers and let $(f_k)$ be a sequence of
   continuous mappings of the real line $R$ into itself such that 
   $$
   |f_k (t) | \le a_k \qquad (t \in R, k = 1,2, \ldots)
   $$
   Given $\alpha \in R$, there exists a convergent real sequence
   $(\xi_k)$ such that 
   \begin{enumerate}[(i)]
   \item $\xi = \alpha$
   \item $\xi_{k + 1} - \xi_k = f_k (\xi_k) \quad (k = 1,2, \ldots)$
   \end{enumerate} 
   consider the mapping $T$ of $(c)$ into itself given by
   $$
   (Tx)_1 = \alpha
   $$
   $$
   (Tx)_{n + 1} = \alpha + \sum^{n}_{k = 1} f_k (\xi_k) \quad (n = 1,2, \ldots)
   $$\pageoriginale
   where $x = (\xi_k)$.

 \item Let $E$ be a Banach space with a uniformly convex norm, and let
   $K$ be a bounded closed convex subset of $E$. Then the metric
   projection $E \xrightarrow{\text{ onto }}K$ exists and is
   uniformly continuous on each bounded subset of $E$. 

 \item Brodsku and Milman \cite{key10}, give conditions under which a convex
   set in a Banach space has a point invariant under all isometric
   self mappings. In this connection see also Dunford and Schwartz
   (\cite{key14}, p.459).  

 \item Browder (11) gives some generalization of the Schauder theorem
   which appear to lie rather deep. Perhaps the most striking of these
   results is the following generalization of theorem
   \ref{chap2:thm2.4}. Let $T$ be 
   a continuous mapping of a Banach space $E$ into itself that maps
   bounded sets into compact sets. If, for some positive integer $m,
   T^m E$ is bounded, then $T$ has a fixed point. For a generalization
   of the Schauder theorem of a different kind see Stepaneek (32). 

 \item Aronszajn \cite{key2} gives general regularity condition on $T$
   sufficient to establish that the set of its fixed points is an
   $R_\delta$ i.e. is a homeomorphic image of the intersection of
   decreasing sequence of absolute retracts. 
\end{enumerate}
