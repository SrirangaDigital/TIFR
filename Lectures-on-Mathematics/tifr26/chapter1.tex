\chapter{The contraction mapping theorem}\label{chap1}

 Given\pageoriginale a mapping $T$ of a set $E$ into itself, an
 element $u$ of $E$ 
 is called a fixed point of the mapping $T$ if $Tu= u$. Our problem is
 to find conditions on $T$ and $E$ sufficient to ensure the existence
 of a fixed point of $T$ in $E$. We shall also be interested in
 uniqueness and in procedures for the calculation of fixed points. 
 
 \begin{Definition}%def 1.1
  Let $E$ be a nonempty set. A real valued function $d$ defined on $E
  \times E$ is called a distance function or metric in $E$ if it
  satisfies the following conditions 
 \begin{enumerate}[i)]
 \item $d(x,y)\geq 0$, $x$, $y \varepsilon E$

 \item $d(x,y ) = 0 \Longleftrightarrow x=y$

 \item $d(x,y) = d(y,x)$

 \item $d(x,z) \leq d(x,y)+ d(y,z)$
 \end{enumerate} 
 A nonempty set with a specified distance function is called a metric
 space.
 \end{Definition}

\begin{example*}%exa
 Let $X$ be a set and $E$ denote a set of bounded real valued
 functions defined on $X$. Let $d$ be defined on $E \times E$ by  
 $$
 d (f,g) = \sup \bigg\{ |f(t) - g(t)| : t \varepsilon X \bigg \}, f,g \varepsilon E.
 $$
 \end{example*}
 
 Then $d$ is a metric on $E$ called the uniform metric or uniform
 distance function. 
 
 \begin{Definition}%def 1.2
 A sequence\pageoriginale $\{x_n \}$ in a metric space $(E,d)$ is said
 to converge 
 to an element $x$ of $E$ if  
 $$
 \lim_{n \to \infty} d (x_n, x) =0
 $$
 A sequence $x_n$ of elements of a metric space $(E,d)$ is called a
 Cauchy sequence if given $\epsilon > 0$, there exists $N$ such that for
 $p,q \geq N$, $d(x_p , x_q)< \epsilon$. 
 \end{Definition} 
 
 A metric space $(E,d)$ is said to be complete if every Cauchy
 sequence of its elements converges to an element of $E$. It is easily
 verified that each sequence in a metric space converges to at most
 one on point, and that every convergent sequence is a Cauchy
 sequence. 
 
\begin{example*}%exa 0
 The space $C_R [0,1]$ of all continuous real valued functions on the
 closed interval $[0,1]$ with the uniform distance is a complete
 metric space. It is not complete in the metric $d'$ defined by  
 $$
 d'(f,g) = \int_{0}^{1} |f (x) -g (x) |dx f, \;\; g \varepsilon C_R [0,1].
 $$
 \end{example*}
 
\begin{Definition}%def 1.3
  A mapping $T$ of a metric space $E$ into itself is said to satisfy a
  Lipschitz condition with Lipschitz constant $K$ if  
  $$
  d (Tx , Ty) \leq K d(x,y)\;  (x,y \varepsilon E )
  $$
  If this conditions is satisfied with a Lipschitz constant $K$ such
  that $0 \leq K < 1$ then $T$ is called a contraction mapping. 
\end{Definition} 

\begin{thmm}[The contraction mapping theorem]\label{chap1:thm1.1}%the 1.1
  Let $T$ be a contraction mapping
  of a complete metric space $E$ into itself. Then 
  \begin{enumerate}[i)]
  \item $T$ has\pageoriginale a unique fixed point $u$ in $E$

  \item If $x_o$ is an arbitrary point of $E$, and $(x_n)$ is defined
    inductively by 
  $$
  x_{n+1} = Tx_n \; (n=0, 1,2,\ldots),
  $$
  then $\lim\limits_{n \to \infty} x_n = u$ and
  $$
  d(x_n, u) \leq \frac{K^n }{1-K} d (x_1, x_o)
  $$
  where $K$ is a Lipschitz constant for $T$.
  \end{enumerate}
\end{thmm} 

\begin{proof}
  Let $K$ be a Lipschitz constant for $T$ with $0 \leq K < 1$. Let
  $x_o \varepsilon E$ and let $x_n$ be the sequence defined by 
  $$
  x_{n+1} = Tx_n \; (n=0 , 1,2,\ldots)
  $$
  We have 
  \begin{equation*}
    d(x_{r+1}, x_{s+1}) = d(Tx_r , Tx_s) \leq Kd (x_r, x_s)
    \tag{1}\label{chap1:eq1} 
  \end{equation*} 
  and so
  \begin{equation*}
    d(x_{r+1} , x_r) \leq K^r (x_1, x_o) \tag{2}\label{chap1:eq2}
  \end{equation*} 
  Given $p ~~q$, we have by (\ref{chap1:eq1}) and (\ref{chap1:eq2}),
  \begin{align*}
    d(x_p , x_q ) & \leq K^q d (x_{p-q}, x_o)\\
    &\leq K^q \bigg \{ d (x_{p-q}, x_{p-q-1}) + d(x_{p-q-1},x_{p-q-2}
    ) + \cdots + d(x_1, x_o)\bigg \}\\ 
    & \leq K^q \bigg \{ K^{p-q-1} + K^{p-q-2} + \cdots + K +  1 \bigg
    \} d(x_1, x_o)\\ 
    & \leq \frac{K^q}{1-K} d (x_1, x_o) \tag{3}\label{chap1:eq3} 
  \end{align*} 
  since\pageoriginale the right hand side tends to zero as $q \to \infty$, it
  follows that $(x_n)$ is a Cauchy sequence, and since $E$ is
  complete, $(x_n)$ converges to an element $u$ of $E$. Since
  $d(x_{n+1}, Tu) \leq K d(x_n , u) \to 0 $ as $n \to \infty$, 
  $$
  Tu = \lim_{n \to \infty} x_{n+1} = u.
  $$
  $d(u,x_n ) \leq d (u, x_p) + d(x_p, x_n) \leq d(u,x_p) +
  \dfrac{K^n}{1-K} d(x_1, x_o)$ for $n< p$, by
  (\ref{chap1:eq3}). Letting $p \to 
  \infty$, we obtain   
  $$
  d(u, x_n) \leq \frac{K^n}{1-K} d (x_1, x_o)
  $$
\end{proof}

\begin{example*}%exam 0
  As an example of the applications of the contraction mapping
  theorem, we prove Picard's theorem on the existence of solution of
  ordinary differential equation. 
\end{example*}

Let $D$ denote an open set in $R^2, (x_o, y_o) \varepsilon D$. Let $f(x,y)$ be
a real valued function defined and continuous in $D$, and let it
satisfy  the Lipschitz condition: 
$$
|f(x,y_1) - f (x, y_2)| \leq M |y_1 - y_2| \; ((x,y_1), (x, y_2) \varepsilon D) 
$$
Then there exists a $t > 0$, and a function $\phi (x)$ continuous and
differentiable in $[x_o - t, x_o + t]$ such that $i) \phi (x_o) = y_o$,
(ii) $y = \phi (x)$ satisfies the differential equation  
$$
\frac{dy}{dx} = f (x,y) \text{~ for~ } x \varepsilon [x_o - t, x_o +t]
$$

We show first that there exists an $\epsilon > 0$ and a function $\phi (x)$
continuous\pageoriginale in $[x_o - t, x_o + t]$ such that iii) $\phi
(x) = y_o + 
\int_{x_o}^{x} f (t, \phi (t)) dt \; (x_o - t \leq x \leq x_o +t)$, and
$\epsilon (x, \phi (x)) \varepsilon D (x_o - \leq x \leq x_o + t)$. Then it
follows from the continuity of $f(t, \phi (t))$ that $\phi(x)$ is
in fact  differentiable in $[x_o - t, x_o + t]$ and satisfies (i)
and (ii). 

Let $\mathcal{U}$ denote a closed disc of centre $(x_o, y_o)$ with
positive radius and contained in the open set $D$, and let $m$ denote
the least upper bound of the continuous function $|f|$ on the compact
set $\mathcal{U}$. We now choose $t$, $\delta$ such that $0 < t <
M^{-1}$, the rectangle $|x-x_o| \leq \delta$ is contained  in
$\mathcal{U}$, and $m t < \delta$. Let $E$ denote the set of all
continuous functions mapping $[x_o - t, x_o + t]$ into $[y_o- \delta,
  y_o + \delta]$. With respect to the uniform distance function $E$ is
a  closed subset of the complete metric  space $C_R [x_o - t , x_o+t]$
and is therefore complete. We define a mapping $T \phi = \psi $ for
$\phi \in E$ by 
$$
\psi (x) = y_o + \int_{x_{o}}^{x} f(t,\phi (t))dt
$$ 

Clearly $\psi (x)$ is continuous in $[x_o -t, x_o + t]$. Also $|\psi
(x) - y_o| \leq m |x-x_o|\leq m t < \delta$ whenever $|x-x_o |\leq
t$. Thus $T$ maps $E$ into it self. Finally, $T$ is a contraction
mapping, for if $\phi _i \in E, \psi _i = T \phi)i (i=1,2)$, then 
\begin{align*}
  |\psi _1 (x)- \psi_2 (x)| & = |\int_{x_{o}}^{x} \bigg\{ f (t', \phi_1
  (t')) - f (t', \phi_2 (t'))\bigg \} \; dt' |\\
  & \leq |x- x_o| M \sup|\phi_1
  (t')| - \phi_2(t') | \; (x_o - t \leq t \leq x_o + t)|\\ 
  & \leq t M d (\phi_1, \phi_2)
\end{align*}
Hence\pageoriginale 
$$
d (\psi_1, \psi_2) \leq t M d (\phi_1, \phi_2)
$$
As $t M < 1$, this proves that $T$ is a contraction mapping. By the
contraction mapping theorem, there exists $\phi \varepsilon E$ with $T \phi =
\phi$ i.e., with 
$$
\phi (x) = y_o + \int_{x_{o}}^{x} f (t, \phi(t)) dt
$$
This complete the proof of Picard's theorem.  

A similar method may be applied to prove the existence of solutions of
systems of ordinary differential equations of the form 
$$
\frac{dy_i}{dx}= f_i (x,y_1, \ldots , y_n) (i=1, 2, \ldots, n)
$$
with given initial conditions. Instead of considering real valued
functions defined on $[x_o - t , x_o +t]$, one considers vector valued
functions mapping $[x_o - t, x_o +t]$ into $R^n$. 

In the following theorem we are concerned with the continuity of the
fixed point. 

\begin{thmm}%the 1.2
  Let $E$ be a complete metric space, and let $T$ and $T_n (n=1,
  2,\ldots)$ be contraction mappings of $E$ into itself with the same
  Lipschitz constant $K<1$, and with fixed points $u$ and $u_n$
  respectively. Suppose that $\lim\limits_{n \to \infty} T_nx = Tx$
  for every $x \varepsilon E$. Then $\lim\limits_{n \to \infty} u_n = u$. By
  the inequality in Theorem \ref{chap1:thm1.1}, we have for each $r=1,
  2,\ldots$,  
  $$
  d (u_r , T_r^n x_o) \leq \frac{K^n}{1-K} d (T_r x_o, x_o), \;\; x_o
  \varepsilon E 
  $$
  setting\pageoriginale $n=0$ and $x_o = u$, we have
  $$
  d (u_r, u) \leq \frac{1}{1-K} d(T_r u, u) = \frac{1}{-1} d (T_r u, Tu)
  $$
But $d(T_r u, Tu) \to 0$ as $r \to \infty$. Hence
$$
\lim_{r \to \infty} d(u_r, u) =0
$$
\end{thmm}

\begin{example*}
  In the notation of the last example, suppose that $y_n$ is a real
  sequence converging to $y_o$ and let $T_n$ be the mapping defined on
  $E$ by 
  $$
  (T_n \phi)(x) = y_n + \int_{x_{o}}^{x}f (t, \phi (t)) dt
  $$
\end{example*}

Then $|(T_n \phi) (x) - y_o| \leq |y_n - y_o| + m \varepsilon < \delta$ for
$n$ sufficiently large i.e. $T_n$ map $E$ into itself for $n$
sufficiently large. Also the mapping $T_n, T$ have the same Lipschitz
constant $\varepsilon M < 1$. Obviously for each $\phi \varepsilon E, \lim\limits_{n
  \to \infty} T_n \phi = T \phi $. Hence if $\phi_n$ is the unique
fixed point of $T_n (n =1, 2, \ldots)$ then $\lim\limits_{n \to
  \infty} \phi_n = \phi$. In other words, if $\phi_n$ is the solution
of the differential equation 
$$
\frac{dy}{dx} = f(x,y)
$$
in $[x_o -t, x_o +t]$ with the initial condition $\phi_n (x_o) = y_n$,
then $\phi_n$ converges uniformly to the solution $\phi$ with $\phi
(x_o) = y_o$.  

\begin{remark*}
  The contraction mapping theorem is the simplest of the fixed point
  theorems that we shall consider. It is concerned with mappings of a
  complete metric space into itself and in this respect is very
  general.\pageoriginale The theorem is also satisfactory in that the
  fixed point is 
  always  unique and is obtained by an explicit calculation. Its
  disadvantage is that the condition that the mapping be a contraction
  is a somewhat severe restriction. In the rest of this chapter we
  shall obtain certain extension of the contraction mapping theorem in
  which the conclusion is obtained under modified conditions.  
\end{remark*}

\begin{Definition}%def 1.4
  A mapping $T$ of a metric space $E$ into a metric space $E'$ is said
  to be continuous if for every convergent sequence $(x_n) $ of $E$,  
  $$
  \lim_{n \to \infty} T_{x_{n}} = T (\lim_{n \to \infty} x_n).
  $$
\end{Definition}

\begin{thmm}%the 1.3
  Let $T$ be a continuous mapping of a complete metric space $E$ into
  itself such that $T^{k}$ is a contraction mapping of $E$ for some
  positive integer $k$. Then $T$ has a unique fixed point in $E$. 
\end{thmm}

\begin{proof}
  $T^k$ has a unique fixed point $u$ in $E$ and $u = \lim\limits_{n
    \to \infty} (T^k)^n x_o$, $x_o \varepsilon E$ arbitrary. 
\end{proof}

Also $\lim\limits_{n \to \infty} (T^k)^n Tx_o = u$. Hence 
\begin{align*}
  u & = \lim_{n \to \infty}(T^k)^n Tx_o = \lim_{n \to \infty} T (T^k)^n x_o\\
  & = T \lim_{n \to \infty} (T^k)^n x_o  \text{ (by the continuity of
    ) T}\\ 
  & = Tu.
\end{align*}

The uniqueness of the fixed point of $T$ is obvious, since each fixed
point of $T$ is also a fixed point of $T^k$. 

\begin{example*}
  We\pageoriginale consider the non-linear integral equation
  \begin{equation*}
    f(x) = \lambda \int^{x}_{a} K (x,y, f(y))dy + g(x)
    \tag{1}\label{chap1:exameq1} 
  \end{equation*}
where $g$ is continuous in $[a,b]$ and $K (x,y,z)$ is continuous in
the region $[a,b]\times [a,b] \times R$ and satisfies the Lipschitz
condition. 
$$
|K (x,y,z_1) - K (x,y,z_2)| \leq M |z_1 - z_2|.
$$
\end{example*}
(The classical Volterra equation is obtained by taking $K(x,y,z)= H
(x,y).z$, with $H$ continuous in $[a,b] \times [a,b]$). Let $E= C_K
[a,b]$ and $T$ be the mapping of $E$ into itself given by  
$$
(Tf) (x) = \lambda \int^{x}_{a} K (x,y,f(u)) dy + g(x)\;  (f \varepsilon E, a
\leq x \leq b). 
$$

Given $f_1$, $f_2$ in $E$ it is easy to prove by induction on $n$ that
$$
|(T^n f_1) (x) - (T^n f_2) (x)| \leq \frac{1}{n !} |\lambda|^n M^n d
(f_1, f_2) (x-a)^n , \; (a \leq x \leq b) 
$$

Then 
$$
d(T^n f_1 , T^n f_2 ) \leq \frac{1}{n!} |\lambda|^n M^n (b-a)^n d
(f_1, f_2) 
$$

This proves that all $T^n$ and in particular $T$, are continuous and,
for $n$ sufficiently large $\dfrac{1}{n!} |\lambda|^n M^n (b-a)^n <
1$, so that $T^n$ is a contraction mapping for $n$ large. Applying the
theorem, we have a unique $f \varepsilon E$ with $Tf = f$ which is the
required unique solution of the equation (\ref{chap1:exameq1}).  

\begin{Definition}%def 1.5
  Let\pageoriginale $(E, d)$ be a metric space and $\varepsilon >
  0$. A finite sequence 
  $x_o, x_1, \ldots, x_n$ of points of $E$ is called an $\varepsilon$ - chain
  joining $x_0$ and $x_n$ if  
  $$
  d (x_{i-1}, x_i) < \varepsilon \; (i=1, 2, \ldots, n)
  $$
\end{Definition}

The metric space $(E, d)$ is said to be $\varepsilon$-\textit{chainable} if
for each pair $(x,y)$ of its points there exists an $\varepsilon$- chain
joining $x$ and $y$. 

\begin{thmm}[Edelstein]\label{chap1:thm1.4}%the 1.4
  Let $T$ be a mapping of a complete $\varepsilon$- chainable
  metric space $(E, d)$ into itself, and suppose that there is a real
  number $K$ with $0 \leq K < 1$ such that 
  $$
  d(x,y) < \varepsilon \Rightarrow d (Tx , Ty) \leq K d(x,y)
  $$
\end{thmm}

Then $T$ has a unique fixed point $u$ in $E$, and $u =  \lim\limits_{n
  \to \infty} T^n x_o$ where $x_o $ is an arbitrary element of $E$. 

\begin{proof}
  $(E,d)$ being $\varepsilon$ chainable we define for $x,y \varepsilon E$,
  $$
  d_{\varepsilon} (x,y) = \inf \sum_{i=1}^{n} d (x_{i-1}, x_i)
  $$
  where the infimum is taken over all $\varepsilon$-chains $x_0, \ldots,
  x_n$ joining $x_0 = x$ and $x_n = y$. Then $d$ is a distance
  function on $E$ satisfying  
  \begin{enumerate}[i)]
  \item $d (x,y) \leq d_{\varepsilon} (x,y)$
  \item $d (x,y) = d_{\varepsilon} (x,y) \text{ for } d (x,y) < \varepsilon$
  \end{enumerate} 
  From (ii) it follows that a sequence $(x_n)$, $x_n \varepsilon E$ is a Cauchy
  sequence with respect to $d_{\varepsilon}$ if and only if it is a Cauchy
  sequence with\pageoriginale respect to $d$ and is convergent with respect to
  $d_{\varepsilon}$ if and only if it converges with respect with respect to
  $d$. Hence $(E,d)$ being complete, $(E, d_{\varepsilon})$ is also a complete
  metric space. Moreover $T$ is a contraction mapping with respect to
  $d_{\varepsilon}$. Given $x$, $y \varepsilon E$, and any
  $\varepsilon$-chain $x_o, \ldots, 
  x_n$ with $x_o = x,x_n = y$, we have  
  $$
  d (x_{i-1}, x_i) < \varepsilon \; (i=1, 2, \ldots, n),
  $$
  so that
  $$
  d (Tx_{i-1}, Tx_i)\leq K d(x_{i-1}, x_i) < \varepsilon \; (i=1, 2, \ldots, n)
  $$
  Hence $Tx_o, \ldots, Tx_n$ is an $\varepsilon$- chain joining $T_x$
  and $T_y$ and  
  $$
  d_{\varepsilon} (Tx, Ty) \leq \sum_{i=1}^{n} d (Tc_{i-1}, Tx_i) \leq K
  \sum_{i=1}^{n} d(x_{i-1}, x_i) 
  $$
  $x_o, \ldots, x_n$ being an arbitrary $\varepsilon$- chain, we have
  $$
  d_{\varepsilon}(Tx, Ty) \leq K d_{\varepsilon} (x,y)
  $$
  and $T$ has a unique fixed point $u \varepsilon E$ given by 
  \begin{equation}
    \lim_{n \to \infty} d_\varepsilon (T^n x_0 , u)=0 \text{ for } x_o
    \varepsilon E \text{ arbitrary } \tag{1}\label{chap1:proofeq1} 
  \end{equation}
  But in view of the observations made in the beginning of this proof,
  (\ref{chap1:proofeq1}) implies that 
  $$
  \lim_{n \to \infty} d (T^n x_o, u)=0
  $$
\end{proof}

\begin{example*}
  Let $E$ be a connected compact subset of a domain $D$ in the complex
  plane. Let $f$ be a complex holomorphic function in $D$
  which\pageoriginale maps 
  $E$ into itself and satisfies $|f' (z)| < 1 \;  (z, \varepsilon E)$. Then
  there is a 
  unique point $z$ in $E$ with $f(z) =z$. Since $f'$ is continuous in
  the compact set $E$, there is a construct $K$ with $0 < K < 1$ such
  that $|f'(z)| < K \; \varepsilon E)$. For each point $\omega
  \varepsilon E$ there 
  exists $\rho_{\omega} > 0$ such that $f(x)$ is holomorphic in the
  disc $S (\omega, 2 \rho_{\omega})$ of center $\omega$ and radius $ 2
  \rho_{\omega}$ and satisfies $|f' (z)|< K$ there. $E$ being compact,
  we can choose $\omega_1, \ldots, \omega_n \varepsilon E$ such that $E$ is
  covered by 
  $$
  S(\omega_1, 2 \rho_{\omega_{1}}), \ldots, S (\omega_n, 2 \rho_{\omega_{n}})
  $$  
\end{example*}

Let $\varepsilon = \min \{ \rho \omega_i, i=1, 2, \ldots, n \}$. If 
$z,z'\varepsilon E$ and $|z-z'|< \varepsilon$ then $z,z' \varepsilon
S(\omega_i, 2 \rho_{\omega_{1}})$ for some $i$ and so  
$$
|f(z)- f(z')| = |\int_{z'}^{z} f' (\omega) d \omega |\leq K |z-z'|.
$$

This proves that Theorem \ref{chap1:thm1.4} is applicable to the
mapping $z \to f(z)$ and we have a unique fixed point.  

\begin{Definition}% def 1.6
  A mapping $T$ of a metric space $E$ into itself is said to be {\em
    contractive} if  
  $$
  d(Tx, Ty)< d(x,y) \; (x \neq y, x,y \varepsilon E)
  $$
  and is said to be $\varepsilon$-contractive if
  $$
  0< d (x,y)< \varepsilon \; \Rightarrow d(Tx, Ty)< d(x,y)
  $$
\end{Definition}

  \begin{remark*}
    A contractive mapping of a complete metric space into itself need
    not have a fixed point. e.g. let $E= \{x/ x \geq 1 \}$ with the
    usual\pageoriginale distance $d (x,y) = |x-y|$,let $T :E \to E$ be
    given by $Tx  = x + \dfrac{1}{x}$. 
  \end{remark*}

  \begin{thmm}[Edelsten]\label{chap1:thm1.5}%the 1.5
    Let $T$ be an $\varepsilon$-contractive mapping of a metric space $E$
    into itself, and let $x_o$ be a point of $E$ such that the
    sequence $(T^n x_o)$ has a subsequence convergent  to a point $u$
    of $E$. Then $u$ is a periodic point of $T$, i.e. there is a
    positive integer $k$ such that 
    $$
    T^k u = u
    $$
  \end{thmm}

\begin{proof}
  Let $(n_i)$ be a strictly increasing sequence of positive integers
  such that $\lim\limits_{i \to \infty} T^{n_{i}} x_o = u$ and let
  $x_i = T^{n_{i}}x_o$. There exists $N$ such that $d(x_i, u) < \; \varepsilon/4$
  for $i \geq N$. Choose any $i \geq N$ and let $k = n_{i+1} -
  n_i$. Then 
  $$
  d(x_{i+1}, T^k u) = d(T^k x_i , T^k u) \leq d(x_i, u) <
  \varepsilon/4 
  $$
  and 
  $$
  d (T^k u, u) \leq d(T^k u, x_{i+1}) + d(x_{i+1}, u) < \varepsilon/2
  $$
\end{proof}

Suppose that $v = T^k u \neq u$. Then $T$ being $\varepsilon$-contractive, 
$$
d(Tu, Tv)< d(u,v) \text{ or } \frac{d(Tu, Tv)}{d(u,v)}< 1.
$$

The function $(x,y) \to \dfrac{Tx, Ty}{d(u,v)}$ is continuous at
$(u,v)$. So there exist $\delta, K> 0$ with $0 < K < 1$ such that
$d(x, u)< \delta, d(y,v) < \delta$ implies that $d (Tx, Ty) < K
d(x,y)$. As $\lim\limits_{r \to \infty} T^K x_r = T^K u = v$, there
exists $N' \geq N$ such that $d (x_r, u) < \delta , d (T x_r , v)<
\delta$ for $r \geq N'$\pageoriginale and so 
\begin{gather*}
  d(Tx_r, T T^K x_r) < K d (x_r, T^K x_r) \tag{1}\label{chap1:eqq1}\\
  d (x_r, T^K x_r) \leq d(x_r, u) + d(u, T^K u) + d(T^K u, T^K x_r)\\
  <\frac{\varepsilon}{4} + \frac{\varepsilon}{2}+
  \frac{\varepsilon}{4} = \varepsilon \text{ for } r\geq 
  N' > N \tag{2}\label{chap1:eqq2} 
\end{gather*}
From (\ref{chap1:eqq1}) and (\ref{chap1:eqq2})
$$
d(Tx_r , T T^K x_r) < K d(x_r , T^K x_r)< \varepsilon \text{ for } r \geq N'
$$
and so $T$ being $\varepsilon$-contractive,
\begin{equation*}
  d(T^p x_r, T^p T^k x_r) < K d(x_r , T^K x_r) \text{ for } n \geq N' ,
  p> 0 \tag{3}\label{chap1:eqq3} 
\end{equation*}
Setting $p = n_{r+1} - n_r $ in (\ref{chap1:eqq3})
$$
d(x_{r+1}, T^k x_{r+1}) < Kd(x_r , T^k x_r) \text{ for any } r \geq N' 
$$
Hence $d (x_s, T^k x_s) < K^{s-r} d(x_r , T^k x_r) < K^{s-r}
\varepsilon$ and $d(u,v) < d (u,x_s)+ d(x_s, T^k x_s) +d (T^k x_s , v)
\to 0$ as $s \to \infty$ This contradicts the assumption that $d(u,v)
> 0$. Thus $u = v=T^k u$. 

\begin{thmm}[Edelstein ]%the 1.6
  Let $T$ be a contractive mapping of a metric space $E$ into itself,
  and let $x_o$ be a point of $E$ such that the sequence $T^n x_o$ has
  a subsequence convergent to a point $u$ of $E$. Then $u$ is a fixed
  point of $T$ and is unique. 
\end{thmm}

\begin{proof}
  By\pageoriginale Theorem \ref{chap1:thm1.5}, there exists an integer
  $k > 0$ such that $T^k u = 
  u$. Suppose that $v = Tu \neq, u$. Then $T^k u = u$, $T^k v=v$ and
  $d(u,v) = d(T^k u, T^k v) < d(u,v)$, since $T$ is contractive. As
  this is impossible, $u=v$ is a fixed point. The uniqueness is also
  immediate. 
\end{proof}

\begin{coro*}%coro 0
  If $T$ is a contractive mapping of a metric space $E$ into a compact
  subset of $E$, then $T$ has a unique fixed point $u$ in $E$ and $u = 
  \lim\limits_{n \to \infty} T^n x_o$ where $x_o$ is an arbitrary
  point of $E$. 
\end{coro*}
