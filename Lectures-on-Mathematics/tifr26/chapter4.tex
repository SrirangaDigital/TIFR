\chapter{Nonlinear mappings in cones}\label{chap4}
 
The theorems\pageoriginale in this chapter are mainly due to Krein and
Rutman \cite{key20} 
and to Schaefer \cite{key28}. They may be regarded as a further step in the
transition from nonlinear to linear problems. We will be content with
considering normed spaces only, through theorems of the kind studied
here have been proved for general locally convex spaces by
H. Schaeffer. 
 
 \begin{Definition}%4.1
   A subset $C$ of a vector space $E$ over $R$ is called a {\em
     positive cone} if it satisfies   
   \begin{enumerate}[(i)]
   \item $x, y \in C \Rightarrow x + y \in C$
   \item $x \in C, \alpha \geq 0 \Rightarrow \alpha \in C$
   \item $x, -x \in C \Rightarrow x=0 $
   \item $C$ contains non-zero vectors.
   \end{enumerate}
 \end{Definition} 
 
 A vector space $E$ over $R$ with a specified positive cone is called
 a {\em partially ordered vector space}, and we write $x \leq y$
(or $y \geq x)$ to denote that $y-x \in C$. It is easily
 verified that this relation $\leq$ is a relation of partial order in
 the usual sense i.e., 
\begin{enumerate}
\item[(v)] $x \leq x \; (x \in E)$,
\item [(vi)] $x \leq y, y \leq z \Rightarrow x \leq z$,
\item [(vii)] $x \leq y, y \leq x \Rightarrow x = y $.
\end{enumerate}
Also the partial ordering and the linear structure are related by the 
properties:\pageoriginale
\begin{itemize}
\item[(viii)]
 $x_i \leq y_i \; (i=1,2) \Rightarrow x_1 + x_2 \leq y_1 + y_2$,

\item[(ix)] $x \leq y, 0 \leq \alpha \leq \beta \Rightarrow \alpha x \leq
\beta y$. 
\end{itemize}

Conversely, given a non-trivial relation $\leq$ in $E$ satisfying
(v),\ldots,(xi), the set $\{x : 0 \leq x \}$ is a positive cone in
$E$ to which the given relation corresponds in the above manner. 

\begin{Definition}%defi 4.2
  Let $C$ be a positive cone in a normed space $E$. A mapping $T$ of a
  subset $D$ of $C$ into $C$ is said to be {\em strictly positive} in
  $D$ if  
  $$
  x_n \in D, \lim_{n \to \infty} Tx_n =0\Rightarrow \lim_{n \to \infty} x_n =0
  $$ 
  A mapping $T$ defined on $C$ is said to be {\em completely}
  continuous in $C$ if it is continuous in $C$ and maps each bounded
  subset of $C$ into a compact set. 
\end{Definition}

\begin{thmm}[Morgenstern \cite{key23}]%theo 4.1
  Let $C$ be a closed positive cone in a normed vector space $E$ such
  that the norm is additive on $C$ i.e. 
  $$
  || x + y || = || x || + || y || \quad (x,y \in C)
  $$
  
  Let $c > 0$, let $K= \{ x : x \in C, || x || = c \}$, and let $T$ be
  a continuous and positive mapping on $K$ and strictly positive on
  $K$ and map $K$ into a compact subset of $C$. Then there exists $u$
  in $K$ and $\lambda > 0$ such that $Tu = \lambda_u$.  
\end{thmm}

\begin{proof}
  Since the norm is additive on $C, K$ is a convex set. Since $T$ is
  strictly positive on $K$,  
  $$
  \inf \{ || Tx ||, x \in K\} > 0
  $$
  and therefore the mapping $A$ defined on $K$ by
  $$
  Ax = c || Tx ||^{-1} Tx
  $$\pageoriginale
  is continuous and maps $K$ into a compact subset $A$ itself. By
  the Schauder theorem, there exists $u \in K$ with $Au =u$. 
\end{proof}

\begin{coro*}%coro 0
  Let $C$ be a closed positive cone in a normed vector space $E$ such
  that the norm is additive on $C$. Let $T$ be a strictly  
  positive and completely continuous mapping of $C$ into itself. Then
  for each $c > 0$, there exists $u_c$ in $C$ and $\lambda_c > 0$ such
  that $Tu_c = \lambda_c u_c $ and $|| u_c || =c$. 
\end{coro*}

\begin{Definition}%defe 4.3
  A positive cone $C$ in a normed vector space is said be normal if
  there exists a positive constant $\gamma$ such that  
  $$
  || x + y || \geq \gamma || x ||\; (x,y\in C).
  $$
\end{Definition}

\begin{thmm}[Schaefer]%theo 4.2
  Let $C$ be a closed normal positive cone in a normed space. Let $c >
  0$ and  let $K = \left\{ x| {}^{x \in C}_{|| x || \leq c }\right\}$. 
  Let $T$ be a 
  continuous and strictly positive on $K$ and map $K$ into a compact
  set. Then there exists $u \in C$, and $\lambda > 0$, such that $Tu =
  \lambda_u $ and $|| u || = c$.  
\end{thmm}

\begin{proof}
  Since $TK$ is contained in a compact set we can choose $\mu  > 0$,
  such that $\mu Tk \subset K$. Let $A = \mu T$, let $y$ be a point of
  $K$ with $y = c$, and let $B$ be the mapping defined on $K$ by $Bx=
  c^{-1} x Ax + c^{-1} (c -x) y \; (x \in k)$, since $K$ is convex, we have
  $BK \subset K$. Also $B$ is continuous in $K$, and maps $K$ into a compact
  set. Since $T$ is strictly positive on $K$, there exists $c > 0$ such
  that  
  $$
  x \in K, || x || \geq \frac{1}{2} c \Rightarrow || Ax || \geq \varepsilon  
  $$\pageoriginale
  since $C$ is a normal cone, it follows that 
  $$
  x \in K, || x || \geq \frac{1}{2} c \Rightarrow ||Bx || \geq \gamma
  c^{-1} \frac{1}{2} c \varepsilon = \frac{1}{2} \gamma \varepsilon  
  $$
  On the other hand,
  $$
  x \in K, || x || \leq \frac{1}{2} c \Rightarrow c^{-1} (c - || x ||)
  || y || \geq \frac{1}{2} c \Rightarrow || Bx || \frac{1}{2} \gamma c  
  $$
  Therefore $|| Bx || \geq \dfrac{1}{2} \gamma \min (\varepsilon, c) >
  0 \; (x \in K)$. 
  
  It follows that the mapping 
  $$
  x \to c || Bx ||^{-1} Bx
  $$ 
  is a continuous mapping of $K$ into a compact subset of
  itself. Therefore, there exists $ u \in K$ with 
  $$
  u = c || Bu ||^{-1} Bu
  $$
  Plainly $|| u || = c$, and so $Bu = Au$, and we have 
  $$
  Au = \lambda, \text{ with } \lambda = c^{-1} || Au || > 0.
  $$%raghu
  The following theorem due to Krein and Rutman \cite[Theorem 9.1]{key20}
  marks a further transition towards a linear problem. 
\end{proof}

\begin{Definition}%defi 4.4
  Let $E$ be a partially ordered vector space with positive cone $C$,
  let $T$ be a mapping of $C$ into itself and let $c$ be be positive
  real number. $T$ is said to be  
  \begin{enumerate}[(i)]
  \item positive-homogeneous of\pageoriginale 
    $$
    T(\alpha x) = \alpha Tx \;  (\alpha \geq 0, x \in C)
    $$

  \item monotonic increasing if 
    $$
    x,y \in C, x \leq y \Rightarrow Tx \leq Ty
    $$
  \item c-dominant if there exists a nonzero vector $u$ in $C$ with
    $Tu \geq cu$. 
  \end{enumerate}
\end{Definition}

\setcounter{section}{4}
\setcounter{lemma}{0}
\begin{lemma}\label{chap4:lem4.1}%lemma 4.1
  Let $C$ be a closed positive cone in a normed space $E$, and let $u$
  be a point that does not belong to $-C$. Then there exists a
  continuous linear functional $f$ on $E$ such that  
  \begin{enumerate}[(i)]
  \item $f (u) = d (u, -C) >0$,
  \item $ f (x) \geq 0 \; (x \in C)$,
  \item $|| f || \leq 1$
  \end{enumerate}
\end{lemma}

\begin{proof}
  Let
  $$
  p(x) = d (x,-C) = \inf \{|| x+y||: y \in C\} 
  $$
  Then $p$ is a sublinear functional on $E$, with the properties 
  \begin{enumerate}[(a)]
  \item $ p(u) = d (u,-C)>0$,
  \item $ p (x) = 0 (x \in -c)$,
  \item $p(x) \leq || x ||\; (x \in E)$.
  \end{enumerate}
\end{proof}

By the Hahn-Banach theorem there exists a linear functional $f$ on $E$
with $f(u) = p(u)$ and with  
$$
f (x) \leq  p(x) \quad  (x \in E).
$$
Plainly\pageoriginale $f$ has the required properties. 

\begin{thmm}[Krein and Rutman]%theo 4.3
  Let $E$ be a partially ordered normed vector space with a closed
  positive cone $C$. Let $T$ be a completely continuous mapping of $G$
  in to itself which is positive-homogeneous, monotonic increasing,
  and $c$-dominant for some $c > 0$. Then there exists a nonzero vector
  $v$ in $C$ and a real number $\lambda \geq c$ such that $Tv =
  \lambda v$. 
\end{thmm}

\begin{proof}
  Since $T$ is positive-homogeneous and $c$-dominant, there exists a
  vector $u$ in $C$ with $|| u || = 1$ and  
  \begin{equation}
    Tu \geq cu \tag{1}\label{chap4:eq1}
  \end{equation}
  since $u \notin -C$, Lemma \ref{chap4:lem4.1} establishes the
  existence of a 
  continuous linear functional $f$ on $E$ with 
  \begin{gather*} 
    f (u) > 0, f(x) \geq 0 \; (x \in C) \tag{2}\label{chap4:eq2}\\
 \text{and} \hspace{4cm}
|| f || = 1 \hspace{4cm} \tag{3}\label{chap4:eq3}
  \end{gather*}
  We now prove that
\begin{equation*}
x \in C, \alpha > 0, \beta > 0, Tx = \alpha x -
  \beta u \Rightarrow \alpha > c \tag{4}\label{chap4:eq4}.
\end{equation*}
 Let $\Gamma$ denote the set of 
  positive real numbers $t$ with $x \geq tu$. Since $x =
  \dfrac{\beta}{\alpha} u + \dfrac{1}{\alpha} Tx \geq
  \dfrac{\beta}{\alpha} u$, we have $\dfrac{\beta}{\alpha} \in
  \Gamma$. Also $\Gamma$ is bounded above, for otherwise  
  $$
  \frac{1}{n} x \geq u \; (n=1,2,\ldots),
  $$
  and since $C$ is closed, this gives $0 \geq u, u=0$ with is not true.

Let\pageoriginale $m$ denote the least upper bound of $\Gamma$. Using
again the fact that $C$ is closed, we have   
$$
x \geq mu 
$$
and therefore  
$$
Tx \geq T(mu) = mTu \geq mcu 
$$
Since $Tx = \alpha x - \beta u$, this gives 
$$
x\geq \frac{\beta + mc }{\alpha}u, 
$$
and therefore 
\begin{gather*}
  \frac{\beta+mc }{\alpha }\leq m,\\
  m(\alpha-c) \geq \beta > 0, \\
  \alpha > c 
\end{gather*}
In the rest of the proof $\varepsilon$ will denote a real number with 
\begin{equation}
  0<  \varepsilon < \frac{1}{2} \tag{5}\label{chap4:eq5}
\end{equation}

Let $ K_\varepsilon = \{ x : x \in E, \mid\mid  x \mid\mid \leq 1, x
geq \varepsilon \mid \mid x \mid \mid u, f(x) \geq \varepsilon f (u)$ 
Clearly $K$  is  a closed, convex, bounded subset of $E$. Next we note
that, for some $\delta > 0$ ,  
\begin{gather*}
  \mid \mid Tx \mid \mid \ge \delta \in f (u) 0 (x \in  K_\in)
  \tag{6}\label{chap4:eq6}    
\end{gather*}
For $x \in K_\varepsilon, x \geq \varepsilon \mid \mid x \mid \mid u$
gives  
$$
Tx \geq \varepsilon \mid \mid x \mid \mid Tu
$$\pageoriginale
since $T$ is positive homogeneous and monotonic increasing .
$$
\displaylines{\text{ By  (\ref{chap4:eq3})},\hfill   \mid\mid Tx  \mid\mid \geq
  f(Tx) \hfill \cr
  \text{and by (\ref{chap4:eq2})} \hfill f (Tx) \geq f (\varepsilon c
  \mid\mid x    
  \mid\mid u) = \varepsilon c \mid\mid x   \mid\mid f (u) \; (x \epsilon
  K_\in)\hfill \cr
  \text{i.e.,}\hfill  
  \mid\mid Tx   \mid\mid \geq \varepsilon c   \mid\mid x   \mid\mid f
  (u) \geq 
  \varepsilon e f (x) f (u) \hfill \cr  
  \hfill \geq \delta \varepsilon f (u) \quad {(x \in K_\varepsilon)}
  \hfill } 
$$
with \quad $\delta = \varepsilon c f(u) > 0 $.

Let $V_\epsilon $ be the mapping defined by taking 
\begin{gather*}
  V_\epsilon (0) = 0,\\
  V_\varepsilon (x) =   \mid\mid x   \mid\mid.   \mid\mid x+2
  \varepsilon   \mid\mid x 
  \mid\mid u   \mid\mid^{-1} (x + 2 \varepsilon   \mid\mid x
  \mid\mid u ), x 
  \neq 0 
\end{gather*}
 $V$ is well defined since 
 \begin{gather*}
  \mid\mid x + 2 \epsilon   \mid\mid x   \mid\mid u   \mid\mid \geq
  \mid\mid x   \mid\mid - 2 \varepsilon   \mid\mid x   \mid\mid ~  \mid\mid  u
  \mid\mid \\ 
  =   \mid\mid x   \mid\mid ( 1 - 2\varepsilon ) > 0 \text { if }   \mid\mid x
  \mid\mid \neq 0. 
  \end{gather*}  
  Plainly $V_\epsilon $ is continuous in $E$ and 
  \begin{equation}
    \mid\mid V_\epsilon x   \mid\mid =   \mid\mid x   \mid\mid
    \tag{7}\label{chap4:eq7} 
  \end{equation}
  Also 
  \begin{equation}
    x \in C, \mid\mid x   \mid\mid = 1 \; V_\varepsilon x
    \in K_\varepsilon \tag{8}\label{chap4:eq8}  
  \end{equation}
For $f(V_\varepsilon x ) = || x || || x + 2 \varepsilon || x || u ||
^{-1} \{f(x) + 2 \varepsilon || x || f (u ) \}$ 
$$
\leq \frac{2\epsilon}{1+2\epsilon}f (u) \geq  \varepsilon f (u) 
$$
  Let\pageoriginale $A_\varepsilon $ be the mapping defined on
  $K_\epsilon $ by  
  $$
  \displaylines{\hfill 
  A_\varepsilon = V_\varepsilon L T \hfill \cr 
  \text{ when}\hfill 
  Lx = \frac{x}{||x|| } \quad x \neq 0 \hfill } 
  $$
  Then by (\ref{chap4:eq6}) and (\ref{chap4:eq8})  
  $$
  A_\varepsilon K_\varepsilon \subset K_\varepsilon 
  $$
  By (\ref{chap4:eq6}), $V_\varepsilon  L$ is continuous in
  $\overline{TK_\varepsilon}$, 
  and $A_\varepsilon $  continuously into a compact subset of $
  K_\varepsilon $. Applying the Schauder theorem, we see that there
  exists a point $x_\varepsilon $ in $K_\varepsilon$ such that  
  $$
  A_\varepsilon x_\varepsilon = x_\varepsilon, 
  $$
  \begin{equation*}
    i.e., \quad V_\varepsilon \Bigg\{\frac {Tx}{||Tx_\varepsilon ||}\Bigg\} =
    x_\varepsilon \tag{9}\label{chap4:eq9}  
\end{equation*}
  i.e. $ ||Tx_\varepsilon ||^{-1} Tx_\varepsilon + 2 \varepsilon U = ||~ ||
  Tx_\varepsilon ||^{-1} Tx_\varepsilon + 2 \varepsilon u || x_\varepsilon$ 
  This can be written in the form 
  \begin{equation*}
    Tx_\varepsilon = \alpha_\varepsilon x_\varepsilon -
    \beta_\varepsilon u, \tag{10}\label{chap4:eq10} 
  \end{equation*}
  where \quad $c < \alpha_\varepsilon < (1+2\varepsilon ) ||
  Tx_\varepsilon ||$. 

  We now choose a sequence $(\varepsilon_n) $ such that $\lim\limits_{n
    \rightarrow \infty} \varepsilon_n = 0$, and such that the sequences
  $(Tx_{\varepsilon_n}) $ and $(\alpha\varepsilon_n)$ converges. Let $v=
  \lim \limits_{n \rightarrow\infty} Tx_{\varepsilon _n}$ and $ \lambda=
  \lim\limits_{n\rightarrow \infty} \alpha_{\epsilon_n} $. Then $
  \lambda\geq c $, and since   
  $\lim \limits{n \rightarrow\infty} \beta_{\varepsilon_n}=
  0$,\pageoriginale  (\ref{chap4:eq10})  gives
  $$
  \lim\limits_{n \rightarrow \infty} x_{\varepsilon_n}= \frac{1}{\lambda}v
  $$
By continuity and positive homogeneity of $T$,
$$
\frac{1}{\lambda}Tv = T\left(\frac{1}{\lambda}v\right) =  \lim\limits_{n
  \rightarrow \infty} Tx_{\epsilon_n} = v  
$$
Finally by (\ref{chap4:eq7}) and (\ref{chap4:eq9}),  $|| x_\epsilon ||
= 1$ and so $v \neq 0$. 
\end{proof}

\begin{remark*}%rem 
  The theorems in this chapter are unsatisfactory in that each of them
  involves an adhoc condition (strict positivity and $c$-domi-nance). 
  It turns out that for linear mappings such an ad hoc
  condition can be avoided, and $I$ think that there is still scope
  for proving a better theorem on non-linear mappings also.  
\end{remark*}
