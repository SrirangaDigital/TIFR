\chapter{The Schauder - Tychonoff theorem}\label{chap3}

It this\pageoriginale chapter we are concerned with non-linear
operators in general locally convex spaces. 
 
\begin{Definition}%def 3.1
  A vector space $E$ over $R$ which is also a topological space is
  called a {\em linear topological space  (l.t.s)} if the mappings 
  \begin{align*}
    (x, y)  & \to x + y\\
    (\alpha, x) & \to \alpha x
  \end{align*}
  from $E \times E$ and $R \times E$ respectively into $E$ are
  continuous. If also every open set in $E$ is a union of convex open
  sets, then $F$ is said to be {\em locally convex}. 
\end{Definition} 
 
We establish the elementary properties of a l.t.s $E$. Since the
mapping $(\alpha, x) \to \alpha x$ is continuous, the mapping $ x  \to
\alpha x$, with fixed $\alpha$, is continuous. Therefore, if $\alpha$
is a non-zero constant then the mapping $x \to \alpha x$ is a
homeomorphism, and so 
\begin{enumerate}[a)]
\item $G$ open, $\alpha \neq 0 \Longrightarrow \alpha G$ open.

In particular
\item $G$ open implies that - $G$ is open.
\item Similarly, $G$ open $\Longleftrightarrow y + G$ open, and so
\item $V$ is neighbourhood of $0$ if and only if $y + V$ is
  neighbourhood of $y$. 

  Let $V$\pageoriginale be a neighbourhood of $0$, and let $x \in
  E$. Since the 
  mapping $\alpha \to \alpha x$ is continuous, and $0x =0$, we have
  $\dfrac{1}{\lambda}x \in V$ for all sufficiently large $\lambda$,
  i.e., 

\item $x \in \lambda V$ for all sufficiently large $\lambda$.

  We prove next that

\item The closure of a convex set is convex.
  
  For $0 \leq \alpha \leq 1$, the mapping $f : E \times E \to E$  given by 
  $$
  (x,y) \to \alpha x + (1- \alpha)y
  $$
  is continuous and $f(K \times K) \subset K$. Therefore $f
  (\overline{K \times K}) \subset \bar{K}$, where $\bar{K}$ denotes
  the closure of $K$. But $\overline{K \times K}= \bar{K}\times
  \bar{K}$ and so $f(\bar{K} \times \bar{K}) \subset \bar{K}$
  i.e., $\alpha a+ (1-\alpha)b \in \bar{K}$ for $a, b \in \bar{K}$. 

\item The interior of a convex set is convex.

  Let $K_0$ be the interior of a convex set $K$, let $a,b \in K_0$ and
  $0 < \alpha < 1$. By (a), $\alpha K_0, (1-\alpha)K_0$ are open
  sets. By (c) $ \alpha K_0 + (1-\alpha)K_0$ is a union of open sets
  and is therefore open. Since 
  $$
  \alpha a + (1- \alpha) b \in K_0 + (1-\alpha)  K_0 \subset K,
  $$
  it follows that $\alpha a+(1-\alpha)b \in K_0$. $A$ subset $A$ of a
  vector space $E$ over $R$ is said to be symmetric if $-A = A$. 

\item Let $U$ be a neighbourhood of 0 in a locally convex
  $l.t.s$. Then there exists a closed convex symmetric neighbourhood
  $V$ of 0 with $V \subset U$. 

  Since 0 is an interior point of $U$ and the space is locally convex,
  there exists a convex open set $G$ with $0 \in G \subset
  U$. Let\pageoriginale $H 
  = \dfrac{1}{2} (G \cap - G)$, and $V = \bar{H}$. By $(b)$ and $(f)
  \; V  $ is a closed convex symmetric neighbourhood of 0. Finally $V
  \subset U$; for if $v \in V$, then $v +H$ is an open set containing
  and therefore has nonempty intersection with $H$, ie there exists
  $h, h'$ in $H$ with $v+h =h'$. Since $H$ is convex and symmetric, 
  $$
  v=h'-h \in 2H ~ \subset ~G. \text{ Thus } V \subset G \subset U.
  $$
\end{enumerate}

\begin{defi*}
  Given an l.t.s. $E$ over $k$, a subset $A$ of $E$ is said
  \textit{ absorb points } if for every $x$ in $E$, 
  $$
  x ~ \in \lambda A
  $$
  for all sufficiently large $\lambda$.
\end{defi*}

\begin{defi*}
  Given a convex set $K$ that absorbs points, the 
  
  \noindent
  \textit{Minkowski functional } $p_K$ is defined by 
  $$
  p_K(x) = \inf \{\lambda ; \lambda > 0, \text{ and } x \in \lambda K\}
  $$
\end{defi*}

\begin{defi*}
  A mapping $p$ of $E$ into $R$ is called a \textit{seminorm} on $E$
  if it satisfies the axioms. 
\begin{enumerate}[i)]
\item $p(x) \geq 0 \qquad (x \in E)$
\item $p (\alpha x) = | \alpha | p(x) \qquad (x \in E, \alpha \in R)$
\item $p(x+y) ~\leq p(x)+ p(y)$
\end{enumerate}
\end{defi*}

Given a seminorm $p$, the \textit{seminorm topology} determined by
$p$ is the class of unions of open balls  
$$
S(x, \in) = \{y : p(y-x) < \varepsilon\} ~ (\varepsilon >0)
$$
With this topology $E$ is a locally convex $l.t.s$. which is not in
general a Hausdorff space. 

The Minkowski\pageoriginale functional of a convex set $K$ that
absorbs points is 
sublinear, and if $K$ is also symmetric, then it is a seminorm. Also
if $x \in \lambda K$ and $\mu > \lambda$, then $x \in \mu K$, for $0
\in K$ since $K$ absorbs points and  
$$
\frac{1}{\mu} x = \frac{\lambda}{\mu} \left(\frac{1}{\lambda}x\right)
+ \left(1-\frac{\lambda}{\mu}\right) \cdot 0 \in K 
$$
\begin{enumerate}[i)]
\item If $K$ is a closed convex symmetric neighbourhood of $0$ in a
  $l.t.s$, the $p$ Minkowski functional $p_K$ is a continuous
  semi-norm in $E$, and  
  $$
  K= \{x; p_K (x) \leq 1\}
  $$
  Conversely, if $p$ is a continuous seminorm in $E$, then $\{x : p(x)
  \leq 1\}$ is a closed convex symmetric neighbourhood $K$ of $0$, and
  $p_K = p$. 
\end{enumerate}

\begin{proof}
  Let $K$ be a closed convex symmetric neighbourhood of 0.

Then $p_K$ is a seminorm on $E$, and so 
$$
|p_K (x')-p_K (x)| \leq p_K (x'-x) \qquad (x', x \in E)
$$
Given $\varepsilon > 0$,
\begin{align*}
  x' \varepsilon,  x + \varepsilon K & \Rightarrow x' -x  \varepsilon K\\
  & \Rightarrow p_K (x'-x) \leq \varepsilon\\
  & \Rightarrow |p_K (x')-p_K(x) | \leq \varepsilon
\end{align*}
since $x + \varepsilon K$ is a neighbourhood of $x$, this shows that $p_K$ is
continuous. If $x \varepsilon K$, then $p_K (x) \leq 1$, by the
definition of $p_k$. On the other hand, if $p_K (x) \leq 1$, then $x
~\varepsilon ~\lambda K (\lambda > 1)$, 
$$
\frac{1}{\lambda}x \in K (\lambda > 1),
$$
and, since $K$ is closed, $x ~ \in K$.
\end{proof}

Thus\pageoriginale
$$
K = \{ x : p_k (x) \leq 1\}.
$$

Conversely, let $p$ be a continuous semi-norm on $E$, and let $K= \{x
: p(x) \leq 1\}$. That $K$ is a closed convex symmetric neighbourhood
of the origin is evident. We have 
$$
p(x) \leq 1 \Leftrightarrow x \in K \Leftrightarrow p_k (x) \leq 1,
$$
and, since $p$ and $p_k$ are both positive-homogeneous, it follows
that $p=p_k$.  
\begin{enumerate}[i)]
\item[(j)]  Let $x$ be a nonzero point of a Hausdorff locally convex
  $l.t.s.E$. Then there exists a continuous semi-norm $p$ on $E$ with
  $p(x)>0$.

  \begin{proof}
    Since $x \neq 0$  and $E$ is a Hausdorff space, there exists a
    neighbourhood $U$ of 0 such that $x \notin U$. By $(h)$ there
    exists a closed convex symmetric neighbourhood $U$ of 0 with $V
    \subset U$. By $(i)$, there exists a continuous semi-norm $p$ on $E$
    such that 
    $$
    V=\{ y: p(y) \leq 1 \}
    $$
    Hence $p(x) > 1$.
  \end{proof}
  
\item[(k)] Let $E$ be a vector space over $K$. Let $p$ be a semi-norm on $E$,
  and let $N=\{x : p(x)= 0\}$. Then $N$ is a subspace of $E$, and the
  functional $q$ defined on the quotient space $\dfrac{E}{N}$ by 
  $$
  q(\tilde{x})= p(x) \qquad (x \in \tilde{x}, \tilde{x} ~ \frac{E}{N}) 
  $$
  is a norm on $E/N$
  
  \begin{proof}
    If $x,y \in N$,\pageoriginale then 
    $$
 0\leq p(x+y) \leq p(x)+ p(y)=0,
    $$
    and so $x+y \in N$. Also $p(x)=0$ implies $p(\lambda x)=0$, and so
    $N$ is a linear subspace of $E$. The definition of $q(\tilde{x})$ is
    in fact free from ambiguity, for if $x, x' \in \tilde{x}$, then
    $x-x' \in N$, and so  
    \begin{gather*}
      |p{x}-p(x')| \leq p(x-x') =0,\\
      p(x)=p(x').
    \end{gather*}
  Finally that $q$ satisfies the axioms of a norm is entirely
  straight-forward. 
  \end{proof}
\end{enumerate} %%till here

Lastly, among these preliminary results, we need a proposition\break which
is a special case of a general theorem on uniform spaces. However, it
is more convenient for our purposes to prove the special case than to
invoke the general theory. 

(b)~ Let $E, F$ be linear topological spaces, let $K$ be a compact
subset of $E$ and let $T$ be a continuous mapping of $K$ into
$F$. Given a neighbourhood $U$ of $0$ in $F$, there exists a
neighbourhood $V$ of $0$ in $E$ such that 
$$
x,x' \in K, ~ x-x' \in V \Rightarrow Tx-Tx' \in U.
$$

\begin{proof}
  Let $H$ be an open set containing $0$ such that
  $$
  H-H \subset U
  $$
  Given\pageoriginale $x  \epsilon K$, there exists a neighbourhood
  $G(x)$ of 0 such that $ 
  x' \in K \cap (X + G(x))\Rightarrow Tx' \in Tx + H$.
  
  Let $V (x)$ be an open neighbourhood of  0 in $E$ such that 
  $$
  V(x) + V(x)\subset G(x),
  $$
  since $K$ is compact and is covered by open sets $x + V(x)$, it has
  a finite covering 
  $$
  x_1+V(x_1),\ldots, x_n+V(x_n).
  $$
  
  Let $V=  \bigcap\limits_{i=1}^n V(x_i)$.
  
  Then $V$ is a neighbourhood of $0$ in $E$. Suppose $x, x' \in K$ and
  $x-x' \in V$. Then there exists $j$ with 
$$
  \displaylines{
    \hfill x' \in x_j+V(x_j)\subset x_j +G (x_j)\hfill \cr
    \hfill x-x_j=x-x'+x'-x_j \in V + V(X_j)\subset
    V(x_j)+V(x_j)\subset G(x_j)\hfill \cr
    \text{since}\hfill  x,x' \in x_j + G(x_j),\hfill \cr 
    \text{we have}\hfill 
    Tx \in Tx_j + H, Tx' \in Tx_j + H, \hfill \cr
    \text{and so}\hfill Tx-Tx' \in H-H \subset U. \hfill }
$$  

  We are now ready to prove the main theorem by which we are able to
  deduce properties of operators in a locally convex linear
  topological space from the corresponding properties of operators in
  normed spaces. The main idea of this theorem was derived from the
  proof\pageoriginale of the Schauder-Tychnoff theorem in Dunford and
  Schwartz \cite{key14} p.454.  
\end{proof}

\begin{thmm}\label{chap3:thm3.1}%theo 3.1
  Let $K$ be a compact subset of a locally convex l.t.s $E,T$ a
  continuous mapping of $K$ into itself, $p_0$ a continuous
  semi-norm on $E$. 
\end{thmm}

Then there exists a semi-norm $q$ on the linear $L(K)$ of $K$ such that
\begin{enumerate}[i)]
\item  $q(x)\geq p_0(x)\quad (x \in L(K))$;

\item $q$ is continuous on $K-K$;

\item $K$ is compact with respect to the semi-norm topology given by $q$;

\item $T$ is uniformly continuous in $K$ with respect to $q$
  i.e., given $\epsilon > 0$, there exists $\delta > 0$, such that 
  $$
  x, x' \in K, \; q(x-x')< \delta \Rightarrow q(Tx-Tx')< \epsilon
  $$
\end{enumerate}

\begin{remark*}%rem 0
  It would be better if one could prove the existence of a continuous
  semi-norm $q$  on $E$ satisfying (i) and (iv). 
\end{remark*}

\begin{proof}
  Since $p_0$ is bounded on $K$  there is no real loss of
  generality in supposing that 
  $$
  p_0(x)\leq 1 \quad    (x \in K).
  $$
  It is convenient to introduce the following definition. We say that a
  set $\Gamma$ of continuous semi-norm \textit{dominates} a set
  $\Gamma$ of continuous semi-norms if the following two conditions are
  satisfied.
\end{proof}

\begin{enumerate}[a)]
\item $p'(x)\leq 1$\pageoriginale \quad $(x \in K, p' \in \Gamma')$ 

\item given $p \in \Gamma$ and $\epsilon > 0$, there exists $p' \in
  \Gamma'$ and $\delta>0$ such that $x,x'\in K,p'(x-x')< \delta
  \Rightarrow p(Tx-Tx')< \epsilon$. 
\end{enumerate}

We construct a countable self-dominating set containing
$p_0$. Given a continuous semi-norm $p$, and a positive integer
$n$, the set 
$$
\left\{x:p(x)<\frac{1}{n}\right\}
$$
is a neighbourhood of 0. Therefore, by proposition (2), there
exists a neighbourhood $V$ of 0 in $E$ such that 
$$
x, x' \in K, x-x' \in V \Rightarrow p(Tx-Tx')<\frac{1}{n}.
$$

By (h), we may suppose that $V$ is a closed convex symmetric
neighbourhood of 0, and then by (i), $p_V$ is a continuous
semi-norm and 
$$
V=\{x:p_V(x)\leq 1\}.
$$

Multiplying $p$ by an appropriate positive constant $\delta_n$, we
obtain a continuous semi-norm $q_n$ such that 
$$
\displaylines{\hfill q_n(x)\leq 1 \quad (x \in K),\hfill \cr
  \text{and such that}\hfill 
  x,x' \in K,q_n(x-x')<\delta_n \Rightarrow p(Tx-Tx')<\frac{1}{n}.\hfill }
$$
Plainly\pageoriginale the set of semi-norms $q_n$ is a countable set
dominating the set (p).  

It follows that given a countable set $\Gamma$ of continuous
semi-norms, there exists a countable set $\Gamma'$ that dominates
$\Gamma$. Now the set $(p_0)$ is dominated by a countable set
$\Gamma_1, \Gamma_1$ is dominated by a countable set $\Gamma_2$, and
so on. Finally, we take  
$$
\Gamma=(p_0)\cup  \bigcup \limits^\infty_{n=1} \Gamma_n.
$$

Then $\Gamma$ is a countable self-dominating set. Let
$(p_n)^\infty_0$ be an enumeration of $\Gamma$ and take 
\begin{equation*}
 q(x)=\sum^\infty_{n=0}2^{-n} p_n (x) \tag{1}\label{chap3:eq1}
\end{equation*}
since
$$
p_n(x)\leq 2 \quad(x \in K-K),
$$
the series (\ref{chap3:eq1}) converges uniformly on $K-K$, and so $q$ is
continuous on $K-K$. Also the series converges on $L(K)$ (linear hull
of $K$) and $q$ is a semi-norm there satisfying (i). Given $x \epsilon K$,
let 
$$
S(x,\rho)=\{x'; x' \in K \text { and }q(x-x')<\rho\}
$$
since $q$ is continuous on $K-K$, $S(x,\rho)$ is an open subset of $K$ in
the topology $\tau$ on $L(K)$ induced from the initial topology on
$E$. Hence each open subset of $K$ in the topology induced by $\tau_q$
(topology\pageoriginale on $L(K)$ defined by $q$) is also open in the
topology induced by $\tau$. (iii) is now an immediate consequence of
the $\tau$-compact-ness of $K$.  

Given $\epsilon > 0$, we choose $N$ with $2^{-N}<\dfrac{\epsilon}{4}$ since
we have  
$$
\sum^\infty_{n=N+1}\frac{1}{2^n}p_n(x-x')\leq
\sum^\infty_{n=N+1}\frac{1}{2^{n+1}}<\frac{\epsilon}{2}\quad(x,x' \in K),
$$
and so
\begin{equation*}
q(x-x') \subset \sum^N_{n=0}\frac{1}{2^n}p_n(x-x')+\frac{\epsilon}{2} 
\quad (x,x' \in K) \tag{2}\label{chap3:eq2}
 \end{equation*}
since $T$ maps $K$ into itself, (\ref{chap3:eq2}) gives
\begin{equation}
  q(Tx-Tx')<\sum^N_{n=0} \frac{1}{2^n} p_n(Tx-Tx') + \frac{\epsilon}{2}\quad
  (x,x'\in K) \tag{3}\label{chap3:eq3} 
\end{equation}
since $\Gamma$ is self-dominated, for each $n$, there exists $k_n$ and
$\delta_n > 0$ such that 
\begin{equation*}
  p_k{_{n}}(x-x')<\delta_n \Rightarrow p_n(Tx-Tx') <
  \frac{\epsilon}{4}(x,x' 
  \in K)\tag{4}\label{chap3:eq4} 
\end{equation*}

Let $N' = \max (k_0, \ldots, k_N)$, and
$$
=2^{-N'} \min (\delta_0,\ldots, \delta_N).
$$
Then since $ p_n \leq 2^{N'}q$ for $n \leq N'$, we have
$$ 
q(x-x')< \delta \Rightarrow_{p_{k_n}}(x-x')<\delta_n \quad (n\leq N) 
$$
and so, by (\ref{chap3:eq4})
$$
x, x' \epsilon k, q(x-x'), < \delta \Rightarrow p_n (Tx-Tx') <
\frac{\epsilon}{4} \; (n=0, 1, \ldots , N). 
$$\pageoriginale
Therefore, by (\ref{chap3:eq3}),
$$
x, x' \in K, q(x-x') < \delta \Rightarrow q(Tx-Tx') < \varepsilon.
$$

\begin{thmm}[Schauder-Tychonoff]\label{chap3:thm3.2}%3.2
  Let $K$ be a non- empty compact convex subset of a locally convex
  Hausdorff $l.t.s E$, and let $T$ be a continuous mapping of $K$ into
  itself. Then $T$ has a fixed point in $K$. 
\end{thmm}

\begin{proof}
  There is no loss of generality in supposing that $L(K)= E$. Suppose
  that $T$ has no fixed point in $K$. Then $Tx-Tx \neq 0 \, (x \in K)$. 
\end{proof}

It follows by proposition (j) that for each point $x$ of $K$ there
exists a continuous semi-norm $p_x$ such that  
$$
p_x (Tx-x)>0
$$
By continuity of $T$ and $p_x$, there exists a neighbourhood $U_x$ of
$x$ such that  
$$
p_x (T_y - y ) > 0 \quad (y \in U_x)
$$

Since $K$ is compact, there is a finite covering of $K$ by such
neighbourhood say 
\begin{gather*}
  U_{x_1}, \ldots , U_{x_m}.\\
  \text{Let}  \qquad p = p_{x_1} + p_{x_2} + \cdots + p_{x_m}.
\end{gather*}
Then\pageoriginale $p$ is a continuous semi-norm and 
\begin{equation}
  p(Tx-x) > 0 \qquad (x \in K) \tag{1}\label{chap3:eqq1}
\end{equation}

Let $q$ be the semi-norm constructed as in Theorem \ref{chap3:thm3.1}
with $p_0 = 
p$. Then $q$ is defined on $L(K)=E, q \geq p, K$ is compact in the
semi-norm topology $\tau_q$, and given $\varepsilon > 0$, there exists
$\delta > 0$ such that  
\begin{equation}
  x, x' \epsilon K, q(x-x') < \delta \Rightarrow q(Tx- Tx') < \varepsilon
  \tag{2}\label{chap3:eqq2} 
\end{equation}

Let $ N= \left\{ x: q(x)=0 \right\}$. By lemma 3.4,
$E/N$ is a normed space with the norm given by   
$$
||\tilde{x}|| = q(x)
$$ 
where $\tilde{x} $ is the coset of $x$. Let $\tilde{K}= \left\{ x : x
\in K \right\}$. Then since mapping $x \rightarrow \tilde{x}$ is
a continuous homomorphism from $E$ with the topology $\tau_q$ to $E/N$
with the norm topology, $K$ is a compact convex set in $E/N$. Also by
(2), 
\begin{equation}
  x, x' \in K, q(x-x')=0 \Rightarrow q(Tx-Tx')=0
  \tag{3}\label{chap3:eqq3}  
\end{equation}
For each $\tilde{x}$ in $\tilde{K}$ there exists a point $x$ in $
\tilde{x} \cap K$, and we define $\tilde{T} \tilde{x}$ by taking  
 $$
 \tilde{T}\tilde{x}= \widetilde{T x}
 $$

 By (\ref{chap3:eqq3}), this definition is unambiguous, and $\tilde{T}$ maps
 $\tilde{K}$ into itself. Also, by (\ref{chap3:eqq2}), given
 $\varepsilon > 0 $, there 
 exists $\delta > 0$ such that $\tilde{x}, \tilde{x'} \in K, ||
 \tilde{x}-\tilde{x}' || < \delta \Rightarrow || \tilde{T} \tilde{x}-
 \tilde{T}\tilde{x}'|| < \epsilon$. For given $\tilde{x}$, $\tilde{x}'
 \epsilon K$, there\pageoriginale exist $x \in K \cap \tilde{x}$ and $x'
 \epsilon K \cap x'$ and $q(x-x')= || \tilde{x}- \tilde{x}'||$. Hence
 $\tilde{T}$ is a continuous mapping of the compact convex subset
 $\tilde{K}$ of the normed space $E/N$. Applying the Schauder fixed
 point theorem, $\tilde{T}$ has a fixed point $\tilde{u}$ say 
 $$
 \tilde{T}\tilde{u}= \tilde{u}.
 $$
 Since $\tilde{u} \epsilon \tilde{K}$, there exists $u \in K \cap
 \tilde{u}$, and we have $\tilde{T} \tilde{u}= \widetilde{T u}$. 
 Thus 
 \begin{gather*}
   Tu- u \in N,\\
   \text{i.e.,}\qquad q(Tu-u)=0
 \end{gather*}
 It follows that $p(Tu-u)=0$, which contradicts (\ref{chap3:eqq1})
 since $u  \in K$.  
  
\begin{problem}
  It will be noticed that Theorem \ref{chap3:thm3.2} generalizes
  theorem \ref{chap2:thm2.3} 
  rather than the full force of the Schauder theorem
  (\ref{chap2:thm2.2}). It is 
  not known whether the following proposition is true. 
\end{problem}
 
$Q$. Let $K$ be a closed convex subset of a locally convex Hausdorff
l.t.s. $E$, and let $T$ be a continuous mapping of $K$ into a compact
subset of $K$. Then $T$ has a fixed point in $K$. 
 
It is obvious that if $T$ maps $K$ into a \textit{compact convex}
subset $H$ of $K$, then $T$ has a fixed point. For  
$$
T H \subset T K \subset H
$$
and we can apply theorem \ref{chap3:thm3.2} to $H$ instead of $K$. In
particular, 
$Q$ will\pageoriginale hold if every compact subset of $K$ is
contained in compact 
convex subset of $K$. By an elementary theorem of Bourbaki (Espaces
Vectoriels Topologiques, $Ch. II, p. 80$) the convex hull of a
precompact subset of a locally convex Hausdorff $l.t.s$ is
precompact. Thus we can obtain a true theorem from $Q$ by supposing
that $K$ be complete instead of closed or $E$ quasi-complete. However,
this is certainly unnecessarily restrictive. By the Krein-Smulian
Theorem \cite{key21}, if $E$ is a branch space with the weak topology as the
specified topology, then the closed convex hull of each compact subset
of $E$ is compact, and so the proposition $Q$ holds, even though $K$
need not be completes (in the weak topology). 

\begin{example*}%exam 0
  Let $E$ be a reflexive Banach space, $K$ a closed convex subset of
  $E$, $T$ a weakly continuous mapping of $K$ into a bounded subset of
  $K$. Then $T$ has a fixed point in $K$. 
\end{example*} 
 
For since $K$ is norm closed and convex it is also weakly closed. Also
since $E$ is reflexive, each bounded weakly closed subset of $E$ is
weakly compact. Hence the weakly closed convex hull of $T K$ is weakly
compact. 
