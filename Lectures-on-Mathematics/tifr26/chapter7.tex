\chapter{Simultaneous fixed points}

 In this\pageoriginale brief chapter we are concerned with the
 existence of a simultaneous fixed point of a family $\mathscr{F}$ of
 mappings   
 $$
 Tu = u ~ (T \in \mathscr{F}).
 $$
We first state without proof two well known theorems on this question
proofs of which will be found in Dunford and Schwartz \cite{key14}
pp.~456-457. We then prove a theorem on families of mappings of a
$\infty ne$ into  itself. Some further results in the present context
will appear in  the next chapter in the theory of a special class of
semialgebras.  

\begin{thmm}[Markov- Kankutani]%theorem 7.1
  Let $K$ be a compact convex subset of a Hausdorff linear topological
  space. Let $\mathscr{F}$ be a commuting family of continuous affine
  mappings of $K$ into itself. Then\pageoriginale there exists a point
  $u$ in $K$ with  
  $$
  Tu = u ~ (T \in \mathscr{F})
  $$
  A mapping $T$ is said to be \textit{affine} if
  $$
  T (\alpha x + (1 -\alpha)y) = \alpha Tx + (1-\alpha) Ty ~\;  (x,y \in
  K, ~ 0 \leq \alpha \leq 1 ) 
  $$
\end{thmm}

\begin{thmm}[Kakutani] %theorem 7.2
  Let $K$ be a compact convex subset of a  Hausdorff locally convex
  space and let $G$ be a group of linear mappings which is
  equi-continuous on $K$ and satisfies $GK \subset K$. Then there
  exists $u$ in $K$ with 
  $$
  Tu = u ~ \; (T \in G)
  $$
  $G$ is said to be equi-continuous on $K$ if given a neighbourhood
  $V$ of 0, there exists a neighbourhood $U$ of 0 such that  
  $k_1, k_2 \in K$, $k_1 - k_2 \in U$ implies that $Tk_1- Tk_2 \in V (T
  \in G)$. 
\end{thmm}

We have chosen to include a theorem on mappings of cones into
themselves in this chapter because of the opportunity it gives to
employ a method of proof quite different from the methods used in the
rest of these lectures. The method depends on the concept of
\textit{ideal} in a partially ordered vector space. 

\begin{Definition} % definition 7.1
  Let $E$ be a partially ordered vector space with positive cone
  $C$. A subset $J$ of $E$ is called an \text{ideal} if  
  \begin{enumerate}[(i)]
  \item $J$ is a linear subspace 
  \item $j \in J \Rightarrow [0, j] \subset J$,
  \end{enumerate} 
  Here [$a,b$] $= \big \{ x : a \leq x \leq b \big \}$; and conditions
  (i) and (ii) are equivalent to (i) and  
  
  (ii)$'$  $j \in J \Rightarrow [ -j,j ] \subset J$

  For let $j \in J$ and $-j \leq x \leq j$. Then $0 \leq x + j \leq
  2j$. By (i), $2j \in J$ and by (ii), $x + j \in J$. Again by (i), 
  $x \in J$. 
\end{Definition}

\begin{example*}%examl 0
  If $E_\circ$ is a linear subspace of $E$ with $E_\circ ~ C =
  (0)$. Then $E_\circ$ is an ideal. (For then [$0,j$] for $j \in J$
  is empty unless $j = 0 )$. 
\end{example*}

\begin{Definition}% definition 7.2
  An element $e$ of $C$ is called an {\em order unit} if
  [$-e,e$]\pageoriginale absorbs all points of $E$,  i.e., if for
  every $x$ in $E$  
  $$
  -\lambda e \leq x \leq  \lambda e
  $$
  for all sufficiently large $\lambda$.
\end{Definition}

By definition of a positive cone, $C$ contains non-zero elements, and
so certainly 
$$
e > 0.
$$

We establish a few elementary properties of ideals. An ideal $J$ is
\textit{proper} if $0 \neq J \neq E$. 

\begin{enumerate}[a)]
\item An element of $C$ is an order unit of $E$ if and only if it is
  not contained in any proper ideal of $E$. 
\end{enumerate}

\begin{proof}
  Let $J$ be an ideal containing an order unit $e$. Then $\lambda e
  \in J$ and  
  $$
  E \subset \bigcup\limits_{\lambda > 0}[ -\lambda e, \lambda e ]
  \subset J  
  $$
  Hence $E = J$ is not a proper ideal.
\end{proof}

Conversely, if $e \in C$ is not contained in any proper ideal, then
the ideal $\bigcup\limits_{\lambda > 0} [ -\lambda e, \lambda e ]$ which
contains $e$ is the whole of $E$ i.e., $e$ is an order unit. 

\begin{enumerate}[(b)]
\item If $E$ has an order unit, then each proper ideal of $E$ is
  contained in a maximal proper ideal of $E$. 
\end{enumerate}

Proof is immediate using (a) and Zorn's lemma.

\begin{enumerate}[(c)]
\item If $E$  has an order unit $e$, and $J$ is a proper ideal of $E$,
  then\pageoriginale $E /J$ is a partially ordered vector space with
  order unit.  
\end{enumerate}

\begin{proof}
  Since $J$ is a  linear subspace, $E /J$ is a vector space whose
  elements are the cosets $\tilde{x} = x + J, ~ x \in E$. The set
  $\tilde{C} = \big \{ \tilde{x} : x \in C \big \} $ is a positive
  cone in $E / J$. It contains the non-zero element $\tilde{e}$, and
  $\tilde{x}, ~ - \tilde{x} \in \tilde{C}$ implies that there exist
  $j, ~ j' \in J$ such that  
  $$
  x + j, ~  - x + j' \in C.
  $$

  This gives 
  $$
  \displaylines{\hfill  
  0 \leq x + j \leq ~ j' + j\hfill \cr
  \text{and so}\hfill  
  x + j \in [ 0, j + j ] ~ \subset ~ J , \hfill \cr
  \hfill x \in J ,~  \tilde{x}= 0.\hfill }
  $$
  
  Finally $\tilde{e}$ is an order unit. For
  $$
  \displaylines{\hfill 
    - \lambda e \leq x \leq \lambda e \hfill \cr
    \text{implies that}\hfill  
    -\lambda \tilde{e} \leq \tilde{x} \leq \lambda
    \tilde{e}\hfill\phantom{implies that} }
  $$
  which proves that $\tilde{e}$ is an order unit since $x \to \tilde{x}$
  is an `onto' mapping. 
\end{proof}

\begin{enumerate}[(d)]
\item If $E$\pageoriginale has an order unit, and $M$ is a maximal
  proper ideal of $E$, then $E/M$ has no proper ideals.  
\end{enumerate}

\begin{proof}
  If $\tilde{J}$ is an ideal of $E/M$, then
  $$
  J = \big \{ x : \tilde{x} \in \tilde{J} \big \}
  $$
  is an ideal of $E$ containing $M$. Hence $J = M$ or $J = E ~
  i.e$. $\tilde{J} = (0)$ or $\tilde{J} =  E/M$. 
\end{proof}

\begin{enumerate}[(e)]
 \item If $E$ has no proper ideals then $E \simeq R$.
 \end{enumerate}
 
\begin{proof}
  $C$ contains a non-zero element $e$. Since there are no proper
  ideals, $e$ is an  order unit. 
\end{proof} 

Let 
\begin{equation*}
  p (x) = \inf [ \xi : x \leq \xi e ] ~ \; (x \in E) \tag{1}
\end{equation*}
Let \qquad $y = p (x)e - x $.

If $x \in E$, then either $x \in C$ or $-x \in C$, for otherwise $(x)$
is an ideal. Hence $y \in C$ or $-y \in C$ ~ i.e. ~$y \ge 0$ or $y <
0$. If $y < 0$ for some $x$, then $-y \ge \varepsilon e$ for some
$\varepsilon > 0$, and  $x \ge (p (x) + \varepsilon)e$ which contradicts
(1). If $y > 0$, for some $x \in E$, then $y$ is an order unit and so   
$$
y \ge ~ \varepsilon e ~ \text{ for some } ~ \varepsilon > 0
$$
But then $(p (x) - \varepsilon) e \ge x$ which also contradicts
(1). Hence  
$$
y = p (x)e - x = 0 ~ \; x \in E)
$$
i.e.\pageoriginale $e \neq 0$ spans the whole space $E$ and so $E$ is
isomorphic to $R$.  
\begin{enumerate}
\item[(f)] Let $E$ have an order unit $e$, and let $M$ be a maximal
  proper ideal. Then there exists a linear functional $f$ on $E$ with  
  \begin{enumerate}[(i)]
  \item $f(x) \geq 0 \qquad (x \in C)$,

  \item $f(e) = 1$,

  \item $f(x) = 0, \qquad (x \in M)$.
  \end{enumerate}
  (i.e., $M$ is the null space of a normalised positive linear functional).
\end{enumerate}

\begin{proof}
  $E/M$ has no proper ideals, and $\tilde{e} > 0$ where $x \to
  \tilde{x}$ is the canonical mapping $E \to E/M$. By (e), for
  each $\tilde{x}$,  there exists a real number $\xi$ with  
  $$
  \tilde{x} = \xi \tilde{e}
  $$
  
  Let $f(x)= \xi$. Then it is easily verified that $f$ is a linear
  functional with the required properties. 
\end{proof}
\begin{enumerate}
\item[(g)] Let $E$ have an order unit and have dimension greater than
  one, and let $T$ be a positive linear mapping of $E$ into itself. 

  Then there exists a proper $T$-invariant ideal, i.e. a proper ideal
  $J$ with $T J \subset J$. 
\end{enumerate}

\begin{proof}
  Let $e$ be the order unit, let $p$ be the Minkowski functional of
  $[-e, e]$, and let  
  $$
  N = \{ x : p(x) = 0 \}.
  $$
  
  The\pageoriginale set $N$ is an ideal in $E$. For if $j \in N$ then 
  $$
  - \varepsilon \leq j \leq \varepsilon e \text{~ for all~ } \varepsilon
  > 0 
  $$
  Hence, if $0 \leq x \leq j$, then 
  $$
  - \varepsilon e \leq x \leq \varepsilon e \quad ( \varepsilon > 0)
  $$
  and so $p(x) = 0$ and $x \in N$. $N \neq E$ since $p(e) = 1$. Also
  $TN \subset N$. For $x \in N$, we have  
  $$
  \displaylines{\hfill 
    - \varepsilon e \leq x \leq \varepsilon e \qquad (\varepsilon >
    0)\hfill \cr 
    \text{ and so }\hfill - \varepsilon Te \leq Tx \leq \varepsilon Te
    \quad (\varepsilon  >     0)\hfill }
$$
But  
$$
\displaylines{\hfill 
  - \alpha e \leq Te \leq \alpha e \quad \text{ for some }\quad  \alpha
  > 0 \hfill \cr 
  \text{so that}\hfill -\varepsilon  \alpha e \leq Tx \leq
  \varepsilon \alpha e ~\text{ for some }~ \alpha > 0 ~\text{ and all
  }~ \varepsilon > 0\hfill } 
$$ 
  
  Hence
  $$
  \alpha p (Tx) = 0 \text{ i.e. } Tx \in N.
  $$ 
\end{proof}

Thus if $N \neq (0)$, it is a $T$-invariant proper ideal. Suppose
$N=(0)$, so that $p$ is a norm. Let $H$ denote the set of all positive
linear functionals $f$ with $f(e) =1$. As $E$ is of dimension greater
than one, by (e), (b) and (f), $H$ is nonempty. If for some $f \in
H$, we have $f(Te) = 0$, then $f(Tx) =0$ for all $x \in E$ and so the
null-space of $f$ is a proper T-invariant ideal. Suppose then  that
$f(Te) \neq 0$ for all $f$ in $H$. Clearly $H$ is a convex weak *
closed subset of the dual space $E^*$ of the normed space $(E,p)$. Also
$H$ is contained in the unit ball of $E^*$. 

For\pageoriginale we have 
$$
\displaylines{\hfill 
  -p (x) e \leq x \leq p (x) e \qquad (x \in E)\hfill \cr
  \text{so that} \hfill 
  -p(x) f (e) \leq f(x) \leq p(x) f(e) \qquad (x \in E, f \in H)\hfill \cr
  \text{i.e.,}\hfill 
  -p(x) \leq f (x) \leq p (x) \qquad (x \in E, f \in H)\hfill \cr
  \text{or} \hfill  |f(x) | \leq p(x) \hfill }
$$
Hence $H$ is weakly compact.

As
$$
-p(x) Te \leq T(x) \leq p(x) T(e) \qquad (x \in E),
$$
$T$ is a bounded linear transformation of $(E,p)$ into itself with
$||T|| \leq T(e)$. Therefore its transpose $T^*$ is a weak* continuous
mapping of $E^*$ into itself. Thus the mapping $S$ defined by  
$$
Sf = \frac{1}{f(Te)} T^*f
$$
is a weak* continuous mapping of the convex, weak* compact subset $H$
of $E^*$ into itself. By the Schander-Tychnoff fixed point theorem $S$
has a fixed point $f_0$ in $H$, and the null space of $f_0$ is a
proper $T$-invariant ideal. 
\begin{enumerate}
\item[(h)] Under the condition of (g) there exists a maximal proper
  ideal $M$ and a non-negative real number such that  
  $$
  Tx - \mu x \in M \qquad (x \in E).
  $$
\end{enumerate}
\begin{proof}
  By (g) and Zorn's lemma, there exists a maximal proper $T$-invariant
  ideal $M$. In fact $M$ is a maximal proper ideal, for
  otherwise\pageoriginale $E / 
  M$ has dimension greater than 1 and so there is a proper
  $\tilde{T}$-invariant ideal $\tilde{M}_1$ in $E/ M$, where
  $\tilde{T}$ is the mapping on $E/M$ given by  
  $$
  \tilde{T} \tilde{x} = \tilde{T} x,
  $$ 
  $x \to \tilde{x}$ being the canonical mapping $E \to E/ M$. Then the
  inverse image $M_1$ of $\tilde{M}_1$ by this mapping is a proper
  $T$-invariant ideal containing $M$ strictly which contradicts the
  definition of $M$. 
  
  The maximal proper ideal $M$ is the null space of a normalised
  positive linear functional $f$.  
  
  Since $f(e) = 1$, we have
  $$
  \displaylines{\hfill 
    x - f(x) e \in M \qquad (x \in E), \hfill \cr
    \text{and so} \hfill 
    Tx -= f(x) Te \in M \qquad (x \in E) \hfill \cr
    \hfill f\{ Tx- f (x) Te \} = 0 \qquad (x \in E)\hfill}
  $$
  Writing $\mu = f(Te)$, we have 
  $$
  f(Tx - \mu x) = 0 \qquad (x \in E),
  $$
  and so $Tx - \mu x \in M \; (x \in E)$.
\end{proof}
   
\begin{thmm}%theo 7.3
  Let $E$ be a partially ordered vector space with an order unit $e$
  and with dimension greater than one. $\mathscr{F}$, be a commuting
  family of positive linear mappings of $E$. Then there
  exists\pageoriginale a maximal proper ideal which is $T$-invariant
  for all $T$ in.   
\end{thmm}
 
\begin{proof}
  We prove that there is a proper ideal that is $\mathscr{F}$
  invariant (i.e. $T$-invariant for every $T \in \mathscr{F}$), and
  then the proof is completed by applying Zorn's lemma as in (h). 
\end{proof}

If every $T \in \mathscr{F}$ is a constant multiple of the identity
mapping, then this assertion is obvious, for every proper ideal is
then $\mathscr{F}$-invariant. Suppose then that $T_0 \in \mathscr{F}$
is not a constant multiple of the identity. Then by (h), there
exists a maximal proper ideal $M$ and a constant real number such that  
$$
T_0 x - \mu x \in M \qquad (x \in E).
$$

Let $E_0 = \{ T_0 x - \mu x : x \in E \}$. Then $E_0$ is a proper
subspace of $E$ and since  
$$
T(T_0x- \mu x) = T_0 (Tx) - \mu (Tx) \qquad (T \in \mathscr{F}),
$$
$E_0$ is $\mathscr{F}$-invariant. If $E_0 \cap C = (0)$ then $E_0$ is
the required proper $\mathscr{F}$-invariant ideal. 

Otherwise, let
$$
J = \bigcup\limits_{y \in E_0} [-y, y]
$$
Then $(0) \neq J \subset M$, and $J$ is an $\mathscr{F}$ invariant
ideal.

\begin{coro*}
  There exists a normalized positive linear functional $f$ such that 
  $$
  f(Tx) = f(Te) f(x) \qquad (T \in \mathscr{F}, x \in E).
  $$
\end{coro*}
