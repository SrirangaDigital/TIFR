

\chapter{Blowing up a closed point of a two dimensional scheme}\label{chap2}%chap 2 

\markright{\thechapter. Blowing up a closed point of a two dimensional scheme}

Let\pageoriginale $X$ be an equidimensional, noetherian, two dimensional regular
pre\-scheme, and $x$  closed point of $X$. We shall then define another
prescheme $X'$ together with a morphism $\sigma : X' \rightarrow X$
such that (i) $\sigma$ is proper, (ii) the fibre $\sigma^{-1} (x)$,
considered as a scheme over the residue field $k(x) = k$ at $x$,
isomorphic to the projective line $\mathbb{P}'(k)$ over $k$, and (iii)
$\sigma$ induces an isomorphism of $X' - \sigma^{-1}(x)$ onto $X - \{
x \}$. The prescheme $X'$ is said to be the \textit{dilatation} of $X$
at $x$, or is said to be obtained from $X$ by blowing up the closed
point $x$.  

First assume that $X = \Spec A$, where $A$ is a noetherian,
equidimensional, regular domain of dimension 2, and suppose also
that the maximal ideal $\mathcal{M}_x$ defining $x$ is generated by
two elements $u,~ v~ \in A$. Let $R$ be the quotient field of $A$, $w' =
\dfrac{v}{u} \in R$, and $B' = A [w']$ the subring of $R$ generated by
$w'$ over $A$. We define $Y'$ to be the  
scheme $\Spec B'$, and $\sigma ' : Y' \rightarrow X$ to be the
morphism induced by the inclusion $A \hookrightarrow B'$ of rings. To
study $Y'$ and $\sigma '$, we need the following  

\begin{lemma*}
  If $f(T)$ is a polynomial in $A [T]$ such that $f(w') = 0, f(T) =
  (u T- v ). g (T)$ with $g(T)$ $ \in A [T]$. 
\end{lemma*}

\begin{proof}
  This being trivial when deg. $f = 0$, it is sufficient to prove it
  for $f$ of positive degree, assuming it to hold for lower degree
  polynomials. Let $f(T) = a_o T^n + a_1 T^{n-1} + \cdots
  a_n$. Multiplying the equation\pageoriginale $f(w') = f
  \left(\dfrac{v}{u}\right) = 0$ by 
  $u^n$, we see tht $u$ divides $a_o v^n$. In the local ring
  $A_{\mathcal{M}}, u$ is a prime element and $v$ is not a multiple of
  $u$, since $A_{\mathcal{M}}$ is a regular local ring and $(u,v)$ form
  a regular system of parameters. Hence $\dfrac{a_o}{u}$ belong to
  $A_{\mathcal{M}}$. On the other hand, if $\mathscr{Y}$ is a prime ideal of
  $A$ distinct from $\mathcal{M}$ either $u$ or $v$ is invertible in $
  A_{\mathscr{Y}}$ since $(u,v) = \mathcal{M}$, so that $\dfrac{a}{u}
  = - \left(a_1. \dfrac{1}{v} + a_2 . \dfrac{u}{v^2} + \cdots + a_n \dfrac
  {u^{n-1}}{v^n}\right)$ belongs to $A_{\mathscr{Y}}$ also. Since
  $\bigcap\limits_{\mathscr{Y} \in ~ \Spec ~ A} A_{\mathscr{Y}} = A, b
  = \dfrac{a_o}{u} \in A$. Thus, $f(t) = b T^{n-1} (uT-v) + g(T)$
  where $g(T) \in A [T], \deg g < $ deg $f$ and $g(w') = 0$. The
  assertion follows from the induction hypothesis. 
\end{proof}

It follows from the lemma that we can identify $B'$ which the residue
ring $A[W']/_{(u.W'-v)}'$, where $W'$ is an indeterminate, such that
$W'$ corresponds to $w'$. 

We have moreover,
$$
\mathcal{M}B' = uB' + v B' = u.B' + u. w' B' = u B',
$$
which shows that the inverse image in $Y'$ of the closed set $V(u)$ of
$X$ defined by $u$ coincides with the inverse image of the point $x =
V(\mathcal{M})$. Further, the fibre $\sigma '^{-1} (X)$, considered as
a scheme over the residue field $k(x) = k$, is given by 
$$
\sigma '^{-1} (X) = \Spec ~ k ~ \underset{X}\times Y' = \Spec
\left(\frac{A}{\mathcal{M}}\otimes_A B'\right), 
$$
and we have the isomorphisms
$$
\frac{A}{\mathcal{M}}\otimes_A B' = \frac{B'} {\mathcal{M}B'} \simeq
\frac{B'}{u B'} = \frac{A[W']} {(u W' - v, u)} = \frac{a[W']}{(u,v)}
\simeq k[W'], 
$$\pageoriginale
where $W'$ denotes an indeterminate over $k$, and $W'$ is the image if
$w' \in B'$ under the above isomorphism. In particular, $\sigma '^{-1}
(x)$ is isomorphic to the affine line $\mathbb{A'} (k) = \Spec ~
k[W']$ over $k$. Now, let $X_u$ and $Y'_u$ denote the open subsets of
$X$ and $Y'$ respectively where $u \neq 0$. Then $Y'_u = \sigma '^{-1}
(X_u)$, and since  
$$
Y'_u \simeq \Spec B' \left[ \frac {1}{u} \right] \simeq ~ \Spec ~ A \left[
  \frac{v}{u} , \frac{1}{u}\right] = \Spec ~ A \left[ \frac{1}{u} \right]
\simeq X_u, 
$$
$\sigma'$ induces an isomorphism of $Y'_u$ onto $X_u$.

We have thus proved that $\sigma'^{-1}(V(u)) = \sigma'^{-1} (x) =
\Spec k [W']\simeq \mathbb{A}' (k)$, where $W'$ is the `restriction'
of $w'$ to the fibre $\sigma'^{-1}(x)$, and $W'$ is transcendental over
$k$; and that $\sigma'$ induces an isomorphism of $\sigma'^{-1}(X_u)
= Y'_u$ onto $X_u$.  

Interchanging $u$ and $v$, we see that if we put $w'' = \dfrac {u}{v}
\in R, B'' = A[w''] \subset R, Y'' = \Spec B''$ and $\sigma'' : 
Y'' \rightarrow X$ is the morphism induced by the inclusion $A
\hookrightarrow B''$ of rings then $\sigma''^{-1} (V(v)) =
\sigma''^{-1}(x) = \Spec ~ k[W''] \simeq \mathbb{A'}(k)$, where $W''$,
the restriction of $w''$ to $\sigma''^{-1}(x)$, is transcendental over
$k$, and $\sigma''$ induces an isomorphism of $\sigma''^{-1} (X_v) =
Y''_v$ onto $X_v$. 

We shall now obtain the required scheme $X'$ and the morphism $\sigma
: X' \rightarrow X$ by 'patching up' $Y'$ and $Y''$. Since $Y'_{w'}
\simeq \Spec B'\left[ \dfrac{1}{w'} \right] \simeq  \Spec  A \left[ w',
  \dfrac{1}{w'} \right] = \Spec A [w' , w'']$, and similarly $Y''_{w''}
\simeq \Spec A [w' , w'']$, we have an $X$-isomorphism $\chi$ of $Y'_{w'}$ onto
$Y''_{w''}$. Therefore there exists a prescheme $X'$ covered by two
affine open subsets $Z'$ and $Z''$ and isomorphisms\pageoriginale $\tau ' : Z'
\rightarrow Y'$ and $\tau '' : Z'' \rightarrow Y''$ such that $\tau ' (Z'
\cap Z'') = Y'_{w'} \subset Y', \tau '' (Z' \cap Z'') = Y''_{w''} \subset
Y''$ and $\tau'' \circ {\tau'}^{-1}$ defined on $Y'_{w'}$ is the
isomorphism $\chi$ of $Y'_{w'}$ onto $Y''_{w''}$. Since $\chi$ is a morphism
of $X$-schemes, we obtain a morphism $\sigma$  of $X'$ onto $X$ such
that $\sigma \circ {\sigma'}^{-1} =  \sigma'$, $\sigma o \sigma''^{-1} =
\sigma''; X'$ is by definition of finite type over $X$, and $X'$ is a
scheme, since $Z' \cap Z'' \simeq \Spec A[w', w'']$ is affine and its
co-ordinate ring $A [w', w'']$ generated by the restrictions to $Z'
\cap Z''$ of the co-ordinate rings $A[w']$ and $A[w'']$ of  $Z'$ and
$Z''$ respectively. ($\EGA$, I, 5.5.6). 

Now, $\sigma'^{-1} (x) \cap Y'_{w'}$ is isomorphic to $\Spec k \left[W',
\dfrac{1}{W'}\right]$ since it is the set of points of $\sigma'^{-1} (x)
\simeq  \Spec k [W']$ where $W'$ (which is the restriction to
$\sigma'^{-1}(x)$ of $w'$) is different from zero, and similarly
$\sigma''^{-1} (x) \cap Y{''}_{w''} \simeq \Spec ~ k \left[ W'' ,
\dfrac{1}{W''} \right]$. The restriction $\chi_1$ of $\chi$ to
$\sigma'^{-1} (x)\cap Y'_{w'}$ is an isomorphism of $\sim'^{-1}(x)
\cap Y'_{w'}$ onto ${\sigma''}^{-1} (x) \cap Y''_{w''}$, which is clearly
induced by the isomorphism of $k$-algebras $k \left[W' ,
\dfrac{1}{W'}\right]\simeq k \left[W'' ,\dfrac{1}{W''}\right]$ determined by $W'
\rightarrow \frac{1}{W''}$. It follows that $\sigma^{-1} (x)$, which is
obtained by patching up $\sigma'^{-1} (x)$ and $\sigma''^{-1}(x)$ by
the isomorphism $\chi_1$, is isomorphic to the projective line
$\mathbb{P}' (k)$ over $k$. Further, $\sigma$ is an isomorphism of
$X' - \sigma^{-1} (x)$ onto $X-\{ x \}$, since $\sigma'$ and ${\sigma ''}$
are isomorphisms of $Y'-\sigma'^{-1} (x)$ and $Y '' - \sigma''^{-1}
(x)$ respectively onto $X_u$ and $X_v$ respectively. 
 
 We assert that $X'$ is a regular scheme. It is clearly sufficient to
 show that $Y'$ and $Y''$ are regular. If $y'$ is a
 point\pageoriginale of $Y'$ and 
 $y = \sigma'(y') \neq x$, $\mathscr{O}_{y}$ and $\mathscr{O}_{y'}$ are
 isomorphic, and $y'$ is a simple point. Hence suppose $y'$ is a closed point
 of the fibre $\sigma'^{-1} (x) = \Spec ~ k[W']$. Then there is an
 irreducible polynomial $\bar{f}(W') \in k [W']$ which generates the
 ideal of $y'$ on $\sigma'^{-1} (x)$. Let $f(w')$ be any lift of
 $\bar{f} (W')$ in $B' = A[w']$. Since $u$ generates the ideal
 $\mathcal{M} B'$ defining the fibre $\sigma'^{-1}(x)$ in $B'$, the
 maximal ideal of $B'$ which defines the point $y'$ on $Y'$ is
 generated by $u$ and $f(w')$. Since $\mathscr{O}_y$ is a two
 dimensional local ring it follows that it is regular. This proves our
 assertion.  
 
To prove that $X'$ is uniquely determined upto an $X$-isomorphism by
$X$ and the closed point $x$ of $X$ and does not depend on the choice
of the parameters $u$, $v$, we adopt a more intrinsic definition of
`blowing up' due to Grothendieck. Let $S$ be the graded subring 
$$
S = A + \mathcal{M} T + \mathcal{M}^2 T^2 + \cdots
$$
of the polynomial ring $A[T]$, and let Proj $S$ be the projective $X-$
scheme determined by S($\EGA$, II, 2.3). We shall show that $X'$ is
isomorphic to Proj $S$ as $X$- schemes. If $U$ and $V$ are independent
variables over $A$, we have a surjective homomorphism of graded
algebras $A[U,V] \rightarrow S $ determined by $ U \rightarrow u T, V
\rightarrow v.T$. It follows from the lemma proved above that the
kernel of this homomorphism is the ideal generated in $A[U,V]$ by the
element $uV - vU$, so that $\dfrac{A[U, V]}{(uV - vU)} \simeq S$. Proj
$S$ is therefore covered by two affine open sets isomorphic
respectively\pageoriginale to Spec $\dfrac{A[U]}{(vU - u)} \simeq
\Spec B'' = Y''$ and $\Spec \dfrac{A[V]}{(uV -v)} \simeq \Spec B' =
Y'$. The intersection of these affine open sets corresponds precisely
to the open subsets $Y'_{w'}$ and $Y ''_{w''}$ of $Y'$ and $Y''$ 
respectively, and the isomorphism of $Y '_{w'}$ and $Y''_{w''}$ so
obtained is precisely $\chi$. This shows that $X'$ and Proj $S$ are
isomorphic over $X$. Since Proj $S$ is uniquely determined by $X$ and
$x$, it follows that $X'$ is uniquely determined upto an
$X$-isomorphism by $X$ and $x$, and does not depend on the regular
system of parameters used. We also see that
$X'\overset{\sigma}{\rightarrow} X$ is a proper morphism ($\EGA$, II,
5.5.3).  

Now consider the general case, when $X$ is any noetherian (or even
locally noetherian) everywhere two dimensional regular prescheme and
$x$ any closed point of $X$. Since the maximal ideal of $x$ in the
local ring $\mathscr{O}_x$ 
is generated by two elements, we can choose an affine open
neighbourhood $X_o$ of $x$ in $X$ such that $X_o = \Spec A_o$ where
$A_o$ is a noetherian regular domain of dimension 2 and the maximal
ideal $\mathcal{M}_o$ defining $x$ in $A_o$ is generated by two
elements. Let $X_1$ be the open set $X - \{ x \}$ of $X$. Let $\{
X'_o, \sigma_o \}$ be the dilatation of $x_o$ with respect to $x$ as 
defined above, and put $X '_1 = X_1$ and $\sigma_1 : X '_1 \rightarrow
X_1$ the identity morphism. By identifying the open subset $X'_o -
\sigma^{-1}_o (x)$ and $\sigma^{-1}_1(X_o - \{ x \})$ of $X'_o$ and
$X'_1$ respectively by means of the composite isomorphism 
$$
X'_o - \sigma^{-1}_o (x) \xrightarrow{\sigma} X_o - \{ x \} 
\xrightarrow{\sigma^{-1}_1} \sigma^{-1}_1 (X_o - \{ x \}), 
$$
we obtain\pageoriginale a prescheme $X'$ and a morphism $\sigma : X'
\rightarrow X$. Then $X'$ is again (locally) noetherian, regular and
two dimensional, $\sigma $ is proper (since being proper is a local
property on the second space), and in particular also separated;
$\sigma^{-1} (x)$ is isomorphic to the projective line $\mathbb{P'}
(k)$ on $k$; and $\sigma$ induces an isomorphism of $X' - \sigma^{-1}
(x)$ on $X - \{x\}$. $X'$ is called the \textit{dilatation} of $X$ at
$x$, and is said to be obtained by blowing up the point $x$ of $X$. 

We shall now study the inverse images of divisors on $X$ by the
morphism $\sigma: X' \rightarrow X$. Since both $X$ and $X'$ are
regular noetherian preschemes, it makes no difference whether we
define a divisor as an element of the free abelian group generated by
irreducible reduced closed subschemes of codimension one, or as a
coherent sheaf of inversible fractionary ideals and we shall use the
latter definition. Let $D$ be a positive divisor on $X$ whose support
contains $x$. Then there exists an irreducible affine open
neighbourhood $X_o = \Spec A_o$ of $x$ such that the maximal ideal
$\mathcal{M}_x$ is generated by two elements $(u,v)$ of $A_o$ and $D$
is defined by a non-zero element $f \in A_o$. The unique integer $l$
such that $f \in \mathcal{M}^l_x$, but $f \notin \mathcal{M}^{l+1}_x$,
is called the \textit{multiplicity} of $D$ at $x$, and is clearly independent
of the $f$ chosen to represent $D$ at $x$. (Since $A_o$ is a regular
ring, it is well known that the map $f \rightarrow l$ extends uniquely
to a discrete valuation on the quotient field $R$ of $A_o$ so that we
can define the multiplicity at $x$ of an arbitrary, not necessarily
positive, divisor.) We can then write 
$$ 
f = \sum^{l}_{i= 0} a_i u^i v^{l-i} + g, a_i \in A_o, g \in
\mathcal{M}^{l+1}_x 
$$\pageoriginale
and not all $a_i$ belonging to $\mathcal{M}_x$ Let $Y '_o =  \Spec A_o
[w]$ be the affine open subset of $\sigma^{-1} (X_o)$ considered
earlier, where $w =\dfrac{v}{u}$. In the ring $A_o [w]$, we have 
$$
f = u^l \sum^l_{i=0} a_i \left(\frac{v}{u}\right)^{l-i} + u^{l+1} h = u^l
\left(\sum^l_{i=0} a_i w^{l-i} + uh\right), h \in A_o [w]. 
$$

Since $\dfrac{A_o [w]}{\mathcal{M}_x A_o [w]} = \dfrac{A_o [w]}{u A_o
  [w]} = k[W]$ where $W$, the image of $w$ in $A_o [w]/$ $\mathcal{M}_x
A_o [w]$, is transcendental over $k$, we see that
$\sum\limits^{l}_{i=0} a_i w^{l-i} + uh \notin u.A_o [w]$ since the
image $\bar{a}_i$ in $A_o / \mathcal{M}_x$ is different from $0$. Thus
if we put $L = \sigma^{-1} (x)$, the divisor $\sigma^* (D)$, which is
defined in $\sigma^{-1} (X_o)$ by the element $f$, contains $L$ with
multiplicity exactly $l$. 

Let $\tau $ be the restriction of $\sigma$ to the open subset $X' -
\sigma^{-1} (x)$ of $X'$, so that $\tau$ is an isomorphism of this
open subset onto $X - \{ x \}$. Since $X'$ is regular, there is a
unique divisor on $X'$ which shall denote by $\sigma '(D)$, which does
not contain $L$ as a component and which restricted to $X' -
\sigma^{-1} (x)$ coincides with $\tau^* (D)$. One may describe $\sigma
'(D)$ as the closure in $X'$ of $\tau^*(D); \sigma'(D)$ is called the
\textit{proper transform} of $D$ by $\sigma$. We then have  
$$
\sigma^*(D) = \sigma '(D) + l. L
$$
where\pageoriginale $l$ is the multiplicity of $D$ at $x$.

Now, $\sigma '(D)$ is defined in $\sigma^{-1}(X_o)$ by the element
$\sum\limits^l_{i=0} a_i w^i + u. h$.  Reading this element modulo
$\mathcal{M}_x A_o [W] = u A_o [w]$, we see that the points of
intersection of $\sigma'(D)$ with the fibre $\sigma^{-1}(x) \cap Y '_o
= \Spec ~ k[W]$ are precisely the closed points of this fibre
corresponding to the irreducible factors of the polynomial
$\sum\limits^{l}_{o} \bar{a}_i W^{l-i}$ in $k[W]$. In particular, if $D$
is a `curve' which is regular at $x, l=1$ and $D$ is defined by an $f$
of the form $a_o v + a_1 u + g$, where $a_o, a_1 \in A_o, \bar{a}_o
\neq 0$ in $k$, and $g \in \mathcal{M}^{2}_x $; and in this case,
$\sigma '(D) = \overline{\tau^{-1}(D)}$ passes through the
$k$-rational point $W = - \dfrac {\bar{a}_1}{\bar{a}_o}$ of
$\sigma^{-1} (x) \cap Y '_o$ This point depends only on the image
$\bar{f} = \bar{a}_o \bar{v} + \bar{a}_1 \bar{u}$ of $f$ in the
$k$-vector space $\mathcal{M}_x/_{ \mathcal{M}_x^2} $ of differentials
at $x$. 

Let us say that two irreducible closed one-dimensional regular
subschemes $D_1$ and $D_2$ (or, as we shall say more shortly, regular
curves) containing $x$ have \textit{contact of order} $\ge l$ at $x$
if there exits elements $f_1, f_2$ of $A_\circ$ defining $D_1$ and
$D_2$ respectively in a possibly smaller neighbourhood of $x$, such
that in the local ring $\mathscr{O}_x$ of $x$ on $X$, we have $f_1 =
f_2 (\mod \mathcal{M}^{l+1}_x)$. We have then proved that $\sigma
'(D_1)$ and $\sigma'(D_2)$ pass through the same point $y$ of
$\sigma^{-1}(x)$ if and only if they have contact of order $\geq 1$.
Suppose now that two curves $D_1$ and $D_2$ are regular at $x$ and
have order of contact $l > 0$ at $x$, and let their
defining\pageoriginale elements 
be chosen that $f_1 \equiv f_2 (\mathcal{M}^{l+1}_x)$. We may then
write $f_1 = a_o v + a_1 u, a_i \in A_o$ with say $\bar{a}_o \neq 0$
in $k$, so that we have $f_2 = a_o v + a_1 u g, g \in
\mathcal{M}^{l+1} _x$. In the ring $A_o [w]$ of $Y '_o$, the defining
elements of the proper transforms $\sigma '(D_1)$ and $\sigma'(D_2)$
are respectively $a_ow + a_1$ and $a_o w + u^l h$, where $h \in A_o
[w]$. It follows that $\sigma '(D_1)$ and $\sigma'(D_2)$ have order of
contact $l-1$ at the point $W = \dfrac{-\bar{a}_1}{\bar{a}_o}$ of the
fibre $\sigma^{-1}(x) \cap Y'_o \simeq \Spec ~ k [W]$. Thus, the
order of contact at a point of the fibre $\sigma^{-1} (x)$ of the
proper transforms of two regular curves through $x$ is one less than
the order of contact of these curves at $x$.
