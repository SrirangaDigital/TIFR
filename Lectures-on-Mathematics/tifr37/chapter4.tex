\chapter[The elimination of indeterminacies of a rational...]{The elimination of indeterminacies of a rational map by
  dilatations}\label{chap4}%chap 4  

\markright{\thechapter. The elimination of indeterminacies of a rational...}
 
Let\pageoriginale $X$ and $Y$ be locally noetheiran  two dimensional regular
preschemes and $f: X \to Y$ a morphism. We say that $f$ is a
\textit{composite} of \textit{dilatations} if there exist a finite set
of preschemes $Y_k(0 \le k \le n)$ $X=Y_n, Y=Y_o$, and morphisms
$f_k:Y_{k+1} \to Y_k(0 \le k \le n-1)$ such that $f=f_o o f_1 o \ldots
o f_{n-1}$ and each $(Y_{k+1},f_k)$ is $Y_k$-isomorphic to the
dilatation of $Y_k$ \textit{at some closed point of $Y_k$}. 
 
 We shall prove the following:

 \begin{theorem*}{\em (Elimination of indeterminacies).}
   Let $X$ be a noetherian, two dimensional, regular prescheme over an
   arbitrary base prescheme $B, Y$ a proper $B$-scheme and $\varphi$ a
   $B$-rational map of $X$ into $Y$. Then there exists a $B$-prescheme
   $Z$ and $B$-\textit{morphisms} $f:Z \to X$ and $\psi : Z \to Y$
   such that $f$ is a composite of dilatations and $\varphi \circ f$
   is represented by $\psi$. 
 \end{theorem*}

   Before we set out to prove this theorem, we shall deduce an
   immediate corollary. For any reduced, irreducible and $B$-proper
   scheme $X$, we denote by $\mathcal{B}(X/B)$ the class of all
   couples $(Y/B,\varphi)$ consisting of a reduced, irreducible and
   $B$-proper scheme $Y$ and a birational map $\varphi$ of $Y/B$ into
   $X/B$. We introduce a partial ordering in this class by saying that
   $(Y/B,\varphi)$ \textit{dominates} $(Y'/B, \varphi')(Y \ge Y'$ in
   symbols, when $B$, $\varphi$ and $\varphi'$ are understood) if the
   $B$-rational map $\varphi'^{-1}\circ \varphi$ is represented by a
   $B$-morphism $\psi$ of $Y$ onto $Y'$. $\psi$ is then automatically
   proper, since the composite $Y \overset{\psi} \to Y' \to B$ is
   proper and $Y' \to B$ is separated. We shall show\pageoriginale that
   $\mathcal{B}(X/B)$ is directed for this partial ordering. For
   $(Y/B,\varphi)$ and $(Y'/B,\varphi')$ belonging to this class, let
   $U$ be the maximal open set of definition of $\varphi'^{-1} \circ
   \varphi =\chi$, and $i_U$ the inclusion of $U$ in $Y$.  
   Let $\Gamma$ be the unique reduced closed subscheme of $Y
   \times_{B} Y'$ having for support the closure in $Y \times_B Y'$ of
   the image of $U$ by means of the immersion $U
   \xrightarrow{(i_U,\chi)} Y \times _B Y'$. If we denote by $f$
   and $f'$ the restrictions to $\Gamma$ of the projections of $Y \times_B
   Y'$ onto the first and second factors respectively, and if we put
   $\psi = \varphi \circ f = \varphi \circ f', \psi$ is birational,
   $\Gamma$ is an irreducible, reduced $B$-proper scheme and $(\Gamma,
   \psi)$ dominates both $(Y, \varphi)$ and $(Y', \varphi')$. This
   proves our assertion. We now have the  

\begin{coro*}
  Let $X$ be an irreducible, noetherian, two dimensional,
  regular and $B$-proper scheme. The members $(X',f)$ of $\mathfrak{B}(X/B)$,
  where $f:X' \to X$ is a composite of dilatations, form a cofinal
  system in  $\mathfrak{B} (X/B)$, for the partial ordering introduced
  above. In other words, given a reduced and $B$-proper scheme $Y$ and a
  $B$-birational map $\varphi$ of $Y$ into $X$, there is an $(X',f)$ as
  above and a proper birational $B$-morphism $\psi : X' \to Y$ such that
  $\varphi \circ \psi = f$. 
\end{coro*}

\begin{proof}
  Apply the theorem with $\varphi$ replaced by $\varphi^{-1}$.
\end{proof} 
 
It is convenient to prove a slightly more general version of the
theorem, and for this purpose, we give a definition. Let $X$ be a
reduced $B$-prescheme and $Y$ a $B$-scheme, and let $\varphi$ be a
rational map of $X$ into $Y$ whose maximal open set of definition is
$U$. Denote by $i_U$ the inclusion of $U$ in $X$, and let $\Gamma$ be
the reduced closed subscheme\pageoriginale of $X \times_B Y$ whose
support is the 
closure in  $X \times_B Y$ of the image of $U$ by the immersion $U
\xrightarrow{(i_U, \varphi)} X \times_B Y$. Let $\pi$ be the
restriction to $\Gamma$ of the first projection  $X \times_B Y \to
X$. We say that a point $x \in X$ is a \textit{point of
  indeterminacy} of $\varphi$ if (i) $x \in X-U$, and
(ii) $\pi^{-1}(x) \neq \phi$. If we suppose further that $Y$ is of
finite type over $B$, a point $x \in X$  such that $\mathscr{O}_x$ is
a discrete valuation ring cannot be an indeterminacy point of
$\varphi$. Indeed, this is what we have proved at the beginning of
lecture \ref{chap3}. We note further that when $Y$ is $B$-proper, $\pi (\Gamma)
= X$ so that the points of indeterminacy are precisely the points of
the set $X - U$. In view of this, the above theorem is an immediate
consequence of the following  

\begin{theorem*}
  Let $X$ be a noetherian two dimensional regular $B$-prescheme, and $Y$
  a $B$-scheme of finite type. Let $\varphi$ be a rational map of $X/B$
  into $Y/B$. Then there is a prescheme $Z$ and a morphism $f : Z \to
  X$ which is a composite of dilatations, such that $\varphi \circ f$ has no
  points of indeterminacy on $Z$.  
\end{theorem*} 

\begin{proof}
  We shall denote by $U$ the maximal open set of definition of
  $\varphi$, and by $\Gamma$ the reduced closed subscheme of $X \times_B Y$
  having for support the closure in $X \times_B Y$  of the image of the
  immersion $U \xrightarrow{(i_U,\varphi)} X \times_B Y$. Let $\pi_1$
  be the restriction to $\Gamma$ of $ X \times_B Y \to X$.  
 
Since $X$ is regular, it is the disjoint union of its components which
are finite in number, so that we may assume that $X$ is further
irreducible. 
\end{proof}

We\pageoriginale shall first make some simplifications. 
\begin{enumerate}
\renewcommand{\labelenumi}{\theenumi)} 
\item We may assume $B$ and $Y$ to be affine. In fact, since $X$ is
  quasi-compact, its image in $B$ is 
covered by a finite number of affine open sets $B_i(i = 1, \ldots
,k)$, and the inverse image of each $B_i$ in $Y$ is a-finite union of
affine open sets $Y_{ij}(j=1, \ldots ,n_i)$ of $Y$. Let $X_i$ be the
inverse image of $B_i$ in $X$, and $X_{ij} = \varphi^{-1}(Y_{ij})
\subset X_i$. If a certain $X_{ij}$ is void, $\Gamma \cap X \times_B  
Y_{ij}$ is also void, and we leave out this $Y_{ij}$ and renumber the
rest. Then we define $\varphi_{ij}$ to be the rational map of
$X_i/B_i$ into $Y_{ij}/B_i$ defined by the $B_i$ morphism $\varphi$ of
$X_{ij}$. 
 
Suppose the theorem is true for $B$ and $Y$ affine. Then by blowing up
$X$ a finite number of times at points in or lying over $X_1$, we can
eliminate all points of indeterminacy of the $B$-rational map
$\varphi_{11}$ of $X_1$ into $Y_{11}$. Suppose $X'$ is the prescheme
obtained from $X$ in this way, $X'_i$ the inverse images in $X'$ of
the $X_i$ and $\varphi'_{ij}$ the rational maps from $X'_i$ into
$Y_{ij}$. Now if $n_1 > 1$, again by blowing up $X'$ a finite number
of times at points over $X_1$ we can eliminate all indeterminacies of
$\varphi'_{12}$ for $X'_1$ to $Y_{12}$. If $X''$ is the new prescheme
we get over $X$ and $X''_{11}$ the inverse image of $X'_1$ the rational
map $X''_{11} \to X'_1 \overset{\varphi'_{11}} \longrightarrow Y_{11}$
does not have any indeterminacies either, since $\varphi'_{11}$ does
not and $X''_{11} \to X'_1$ is a morphism. Repeating this for all
$(i,j)$, we obtain an $Z/X$ such that if $Z_i$ is the inverse image of
$B_i$, the rational maps of $Z_i$ into the $Y_{ij}$ do not have
indeterminacies. If $\psi$ is the morphism of $Z$ onto $X$ and
$\Gamma_Z$ the closed\pageoriginale subset of $Z \times_B Y$ defined
with respect to $ 
Z ; Y$ and $\varphi \circ \psi $ in the same way as $\Gamma$ was
defined with respect to $X, Y$ and $\varphi$, then, we have evidently
$\Gamma_{Z} \subset \bigcup\limits_{(i,j)} (Z_{i}X_{B_{i}} Y_{ij})$,
and this shows that $\varphi \circ \psi$ has no indeterminacies.  

This complete the proof of 1).

\item We may assume $B = \Spec \mathbb{Z}$ and $Y = \Spec \mathbb{Z}[T] =
\mathbb{A}' (\mathbb{Z})$ 

Because of 1), we may assume that $B = \Spec A$, and $Y = \Spec ~ C$,
where $C$ is an $A$-algebra of finite type. By expressing $C$ as the
quotient of a polynomial ring in a finite number of variables, we see
that we can realise $Y$ as a closed subscheme of a scheme $B
\times_{\mathbb{Z}} \mathbb{A}^{n} (\mathbb{Z})= \mathbb{A}^{n} (A) = \Spec
~ A [T_{1}, T_{2}, \ldots , T_{n}]$. Let $\varphi'$ be the composite
of $\varphi$ with the inclusion $Y \hookrightarrow \mathbb{A}^{n}(A)$,
and suppose the theorem to be true for $\varphi'$. We then get a
morphism $f:Z \to X$ which is a composite of dilatations such that
$\varphi '\circ f$ has no indeterminacies. Let $V$ be the set of
definition of $\varphi' \circ f$, so that the morphism $V
\xrightarrow{(i_{V}, \varphi' \circ f)} Z \times_A \mathbb{A}^{n}(A)$ is
a closed immersion. The image of an open (hence dense) subset of $V$
is contained in the closed subscheme $Z \times_A Y$, so that the image of
$V$ is itself contained in this closed subscheme. Since $V$  is
reduced, the above morphism factors as $V \to Z \times_A Y \to Z \times_A
\mathbb{A}^{n}(A)$. Composing the first of these morphisms with the
projection of $Z \times_A Y$ onto $Y$, we get a morphism of $V$ into $Y$
which considered as a rational map of $Z$ into $Y$ has no
indeterminacies, since $V \to Z \times_A Y$ is a closed immersion. 

Thus,\pageoriginale we are reduced to proving the theorem in the case
when $B = 
\Spec A$ and $Y = \mathbb{A}^{n} (A) \simeq A
\times_{\mathbb{Z}}\mathbb{A}^{n} (\mathbb{Z})$. But now, for any
$B$-prescheme $Z$, $B$- morphisms (\resp $B$-rational maps) of $Z$ into
$\mathbb{A}^{n}(A)$ are in canonical one-one correspondence with
$Z$-sections (\resp $Z$-rational sections) of $Z \times_A \mathbb{A}^{n}(A)
\simeq Z \times_A  (A \underset{\mathbb{Z}}{\times} \mathbb{A}^{n}
(\mathbb{Z}))$ and the sets of definition of a rational map and the
corresponding rational section coincide. This is by the very
definition of the product. $A$ $B$-rational map has no indeterminacies
if and only if the corresponding $Z$-section is a closed immersion. But
now, we have a canonical isomorphism of $Z$-schemes $Z \times_{A} (A
\times_{\mathbb{Z}} \mathbb{A}^{n}(\mathbb{Z})) \simeq Z
\underset{\mathbb{Z}}{\times} \mathbb{A}^{n} (\mathbb{Z})$ and
applying the remarks made above with $A$ replaced
by $\mathbb{Z}$ we see that it is sufficient to prove the theorem in
the case when $B =\Spec \mathbb{Z}, Y = \mathbb{A}^{n} (\mathbb{Z})$. 

Again, let $Z$ be an $X$-prescheme and $\chi$ a $Z$-rational section
of $Z \underset{\mathbb{Z}} \times \mathbb{A}^{n} (\mathbb{Z}) \simeq
(Z \underset{\mathbb{Z}}{\times} \mathbb{A'} (\mathbb{Z})) \times_{Z}
(Z \underset{\mathbb{Z}}{\times}
\mathbb{A'}(\mathbb{Z})\times_{Z}..$ ($n$ times) with maximal open set of
definition $V$. Let $p_{i}$ be the $i^{th}$ projection of $Z
\underset{\mathbb{Z}}{\times} \mathbb{A'}(\mathbb{Z}))\times_{Z}
\ldots \times_{Z} (Z \underset{\mathbb{Z}}{\times}
\mathbb{A'}(\mathbb{Z}))$ and let $p_{i}\circ \chi$ have the maximal
open set of definition $V_{i}$. We have then clearly
$\bigcap\limits_{i} V_{i} \subset V$. If each $p_{i} \circ \chi$, as a
$Z$-morphism of $V_{i}$ into $Z \underset{\mathbb{Z}}{\times}
\mathbb{A'}(\mathbb{Z})$ is a closed immersion, so is the morphism 
$$
\cap V_{i} \simeq V_{1} \times_{Z} V_{2} \times_{Z} \times \cdots
\cdots \times_{Z} V_{n} \to (Z \underset{\mathbb{Z}}{\times}
\mathbb{A'}(\mathbb{Z})) \times_{Z}\cdots \times_{Z}(Z
\underset{\mathbb{Z}}{\times} \mathbb{A'}(\mathbb{Z})) 
$$       

This proves that if each $p_{i} \circ \chi$ were a closed immersion,
$V = \cap V_{i}$ and $\chi$ is a closed immersion of $V$. 

Hence,\pageoriginale it is sufficient to prove the theorem when $B = \Spec
\mathbb{Z}$ and $Y = \mathbb{A}^{1} (\mathbb{Z})$. 

\item We are thus reduced to the case when $\varphi$ is a rational
function $f \in R (X)$. When $f = 0, f$ is a morphism of $X$ into
$\mathbb{A'}(\mathbb{Z})$ and there is nothing to prove. Thus we may
assume that $f \in R(X), f \neq 0$. 

\noindent
We shall make use of the following
\end{enumerate}

\begin{lemma*}%lemma 
  Write the divisor $D$ of $f$ on $X$, considered as a cycle of
  codimension one, as $D_{1}-D_{2}$ where $D_{1}$ are positive
  integral linear combinations of irreducible closed subsets of
  dimension one, such that $D_{1}$ and $D_{2}$ have no common
  components. Denote by $|D_{i}|$ the support of $D_{i}$ on $X$. Then
  the maximal open set of definition of $f$ is $X-|D_{2}|$, and the
  points of indeterminacy are precisely the points of $|D_{1}|~ \cap
  ~|D_{2}|$.   
\end{lemma*}

\begin{proof}
  Identifying the local ring $\mathscr{O}_{x}$ of a point $x \in X$
  with a subring of $R(X)$, we see from the definitions that $f$ is
  defined at $x$ if and only if $f \in \mathscr{O}_{x}$. Writing $A$
  for $\mathscr{O}_{x},~ A$ is an integrally closed noetherian domain,
  and hence a well-known result, $A =\bigcap\limits_{\mathscr{G}}
  A_{\mathscr{G}}$ where $\mathscr{G}$ runs through the prime ideals
  in $A$ of high one. It follows from this and the definition of
  $D_{2}$ that $f$ is regular at $x$ if and only if no component of
  $D_{2}$ contains $x$, that is, if and only if $x \notin~
  |D_{2}|$. This proves the first assertion.   
\end{proof}

Let $\underbar{0}$ be the closed subset of $\mathbb{A'}(\mathbb{Z}) =
\Spec \mathbb{Z}[T]$ defined by the ideal $(T)$, so that $\underbar{0}
\simeq \Spec \dfrac{\mathbb{Z}[T]}{(T)} = \Spec \mathbb{Z}$. By the
very definition of $D_{1}, ~|D_{1}| - (|D_{1}| \cap |D_{2}|)$ is the
inverse image of $\underbar{0}$\pageoriginale for the morphism $X- ~|D_{2}|
\xrightarrow{f} \mathbb{A'}(\mathbb{Z})$. Since $(|D_{1}| - (|D_{1}|
~\cap ~ |D_{2}|))~ \underset{\mathbb{Z}}{\times} \underbar{0} \simeq
(|D_{1}| -(|D_{1}| ~\cap ~ |D_{2}|)) \underset{\mathbb{Z}}{\times}
\Spec \mathbb{Z} \simeq |D_{1}| - (|D_{1}|~ \cap~ |D_{2}|)$, the image
of $|D_{1}| - (|D_{1}| \cap ~ |D_{2}|)$ for the immersion $X - |D_{2}|
\xrightarrow{(1, f)}X  \underset{\mathbb{Z}}{\times} \mathbb{A'}
(\mathbb{Z})$ is the subscheme $(|D_{1}| - (|D_{1}| ~\cap ~ |D_{2}|))
\underset{\mathbb{Z}}{\times} \underbar{0}$ of $ X
\underset{\mathbb{Z}}{\times} \mathbb{A'}(\mathbb{Z})$. Since $D_{1}$
and $D_{2}$ have no common components, $|D_{1}| - (|D_{1}| ~\cap ~
|D_{2}|)$ is dense in $|D_{1}|$, and it follows that $\Gamma$
(defined at the beginning of the proof of the theorem), being a closed
set, must contain $|D_{1}| \underset{\mathbb{Z}}{\times} \underbar{0}
(\simeq |D_{1}|)$. Hence, every point of $|D_{1}| ~\cap ~ |D_{2}|$ is
a point of indeterminacy.  

It only-remains to be shown that no point of $|D_{2}|-(|D_{1}| ~\cap ~
|D_{2}|)$ is a point of indeterminacy of $f$. Applying what we have
proved to $\dfrac{1}{f}$ instead of $f$, we see that $\dfrac{1}{f}$ is
defined as a morphism from $X - |D_{1}|$ into $\mathbb{A}^{1}
(\mathbb{Z})$, and the inverse image of $\underbar{0}$ is precisely
$|D_{2}|-(|D_{1}| ~\cap ~ |D_{2}|)$. Now, it is easy to deduce from
definition that the image of the open dense subset  
$$ 
\Gamma \cap \{ (X - |D_{1}| - |D_{2}|) \underset{\mathbb{Z}}{\times}
\mathbb{A}^1 (\mathbb{Z})\} \text{  of } \Gamma \cap \{(X - |D_{1}|) 
\underset{\mathbb{Z}}{\times} \mathbb{A}^1 (\mathbb{Z}) \} 
$$
under the morphism
$$
(X - |D_{1}|) \underset{\mathbb{Z}}{\times} \mathbb{A'}
(\mathbb{Z})\xrightarrow{(\frac{1}{f}, 1)} \mathbb{A'}(\mathbb{Z})
\underset{\mathbb{Z}}{\times} \mathbb{A'}(\mathbb{Z}) = \Spec ~
\mathbb{Z}[T,T'] 
$$
is contained in the closed subset $H$ of $\mathbb{A'}(\mathbb{Z})
\times_{\mathbb{Z}} \mathbb{A'}(\mathbb{Z})$ defined by $TT'
-1$. Hence, the image of $\Gamma \cap \{(X - |D_{1}|)
\underset{\mathbb{Z}}{\times} \mathbb{A'} (\mathbb{Z})\}$ is itself
contained in this closed subset. But the projection of $H$ onto the
first factor $\mathbb{A'} (\mathbb{Z})$ of $\mathbb{A'} (\mathbb{Z})
\underset{\mathbb{Z}}{\times} \mathbb{A'} (\mathbb{Z})$ does not
$\underbar{meet}$ the set $\underbar{0}$,\pageoriginale since the
image of $T$ under 
the homomorphism $\mathbb{Z}[T] \to \mathbb{Z}[T,T']/(TT' - 1)$
generates (as an ideal) the second ring. Thus, the projection of
$\Gamma \cap \{(X - |D_{1}|) \underset{\mathbb{Z}}{\times} \mathbb{A'}
(\mathbb{Z})$ into $X - |D_{1}|$ does not meet $|D_{2}| - (|D_{1}|
~\cap~ |D_{2}|)$.  

This completes the proof of the lemma.

We now return to the proof of the theorem under the assumptions
(3). We give the proof here only under the assumption that $X$ is
Japanese. The proof in the general case will be given in Lecture
\ref{chap6}. Because of the lemma and the assumption that $X$ is noetherian,
$f$ has only a finite number of indeterminacy points. Because of the
theorem of resolution of singularities of a one-dimensional closed
subscheme at a finite number of points by dilatations, we can find a
morphism $\tau : X' \to X$ which is a composite of dilatations, such
that with the notations of lemma, the components of the proper
transforms $\tau' (D_{1})$ and $\tau' (D_{2})$ are all regular at all
points of $\tau^{-1} (x)$, where $x$ is any indeterminacy of $f$ on
$X$. But now, if we write div $(f \circ \tau) = D'_{1} - D'_{2}$ where
$D'_{i}$ are divisors $\ge 0$ on $X'$ with no common components, we
have   
$$
D'_{1} - D'_{2} = div ( f \circ \tau) = \tau^{*} (div ~ f) = \tau^{*}
(D_{1}) - \tau^{*}(D_{2}) 
$$    
and $\tau^{*}(D_{i})$ differ from $\tau'(D_{i})$ only by a linear
combination of components of fibres $\tau^{-1}(x)$ of $\tau$. But the
components of the fibres $\tau^{-1}(x)$ are isomorphic to
$\mathbb{P'}(k')$, where $k'$ is an extension of the field $k$, so
that these are again regular. 

Thus,\pageoriginale we may assume that all components of $D_{1}$ and $D_{2}$ are
regular at the points of indeterminacy. Further, let $C_{1}, C_{2}$ be
two components of either $D_{1}$ or $D_{2}$ which meet at a point $x$
of $X$, and suppose they are regular at $x$ and have order of contact
$l$ at $x$. Thus we have seen that the order of contact of their
proper transforms $C'_{1}$ and $C'_{2}$ at any point in the dilatation
of $X$ at $x$ is $l-1$ (with the interpretation that when $l = 0,
C'_{1}$ and $C'_{2}$ do not intersect at any point of the fibre over
$x$). Moreover, $C'_{1}$ and the fibre over $x$ have order of contact
0 (that is, intersect transversally). It follows from these
observations, that we may actually assume that at any point of
indeterminacy $x$ of $f$ on $X$, there is exactly one component of
$D_{1}$ and one component of $D_{2}$, that these are both regular at
$x$ and that they have order of contact $0$ at $x$.  

Under this assumption, for any point of indeterminacy $x$ of $f$ on
$X$, let $C_{i} (i = 1, 2)$ be the unique component of $D_{i}(i =
1,2)$ which contains $x$,and let $r_{i} > 0$ be the multiplicities
with which $C_{i}$ occur in $D_{i}$. Put $n(x, f) = \max (r_{1},
r_{2})$. Let $\sigma : X' \to X$ be the dilatation of $X$ at $x$, and
$L = \sigma^{-1}(x)$. Then the rational function $f \circ \sigma$ on
$X'$ also satisfies these assumptions. Further, if the only components
of div $(f \circ \sigma)$ which pass through any point of
$\sigma^{-1}(x)$ are the proper transforms $C'_{1}, C'_{2}$ of $C_{1},
C_{2}$ respectively and $L$, and these occur with multiplicities
$r_{1}, -r_{2}$ and $r_{1} - r_{2}$ respectively. $C'_{1}$ and $L$
intersect at a unique point $x_{1}$ of $L, C'_{2}$ and $L$ intersect
at another distinct unique point $x_{2}$ of $L$.\pageoriginale If
$r_{1} > r_{2}, f \circ \sigma$ is regular at $x_{1}$ and
 $n(x_{2}, f \circ \sigma) =
  \max (r_{2}, r_{1}- r_{2}) < r_{1} = n(x,f)$, and similarly when
$r_{2} > r_{1}$ also, $n(x_{1}, f \circ \sigma) < n(x, f)$, and
$x_{2}$ is a polar point of $f$. If $r_{1} = r_{2}, f \circ \sigma$
has no indeterminacies at any point of $\sigma^{-1}(x)$. It trivially
follows from these there is a morphism $\tau : Y \to X$ which is a
composite of dilatations, such that $\tau$ is an isomorphism of $Y -
\tau^{-1}(x)$ onto $X-\{x\}$ and $f \circ \tau$ has no indeterminacies
on $\tau^{-1}(x)$. By repeating this procedure for each of the finite
number of points of indeterminacy, we arrive at an $Z$ and a morphism
$f : Z \to X$ having the properties stated in the theorem. 

This completes the proof of the theorem.

\begin{remark*}%rem 
  We shall later give a much simpler proof of case $3)$ (which is the
  really difficult case) of the theorem. But the proof given here is
  more straightforward. 
\end{remark*}        

Our next theorem gives the structure of any proper birational morphism
of two dimensional noetherian regular preschemes. 

\begin{theorem*}{\em (of decomposition).} 
  Let $X$ and $Y$ be two dimensional noetherian regular preschemes,
  and $f:X \to Y$ a proper birational morphism. Then $f$ is a
  composite of dilatations.  
\end{theorem*}
 
\begin{proof}
  We may clearly assume $X$ and $Y$ to be irreducible.  
\end{proof} 
 
 Since $f$ is proper, the indeterminacy set of $f^{-1}$ is precisely
 the complement of the set of definition of $f^{-1}$, and is therefore
 a closed set. Since no points of $Y$ such that $\mathscr{O}_{y}$ is a
 discrete valuation ring can belong to this set, and since $Y$ is
 regular and\pageoriginale noetherian, the indeterminacy set consists of a finite
 number of closed points of $Y$, and if $Y'$ denotes the complement,
 $f | f^{-1}(Y')\to Y'$ is an isomorphism. 
 
 Thus, to prove	the theorem, we may assume further that $X,Y$
 irreducible, and there is a single closed point $y$ of $Y$ such that
 $f:X - f^{-1}(y) \to Y - \{y\}$ is an isomorphism 
 
 Let $\sigma : Y' \to Y$ be the dilatation of $Y$ at $y$, and $g =
 \sigma^{-1} \circ f$. We shall show that $g$ is a morphism. Suppose we
 have done this. All components of the fibre $f^{-1}(y)$ cannot be
 mapped onto single points of $Y'$ by $g$, since $g$ is proper and
 hence surjective, and the fibre $L = \sigma^{-1}(y)\simeq
 \mathbb{P'}(k(y))$ contains an infinity of points. Thus, for any
 point $y' \in Y'$, the number of components of dimension $1$ of
 $g^{-1}(y')$ is strictly less than the number of components of
 $f^{-1}(y)$. The theorem would then follow by induction, using
 $Z.M.T$.  
 
 Thus, it is sufficient to show that $g$ is a morphism. Since $X$ and
 $Y'$ are birationally equivalent over $Y$, there is a unique closed
 reduced component $Z$ of $X \times_{Y} Y'$ which dominates both $X$
 and $Y'$. Let $p_{1}$ and $p_{2}$ be the restrictions to $Z$ of the
 projections of $X \times_{Y}Y'$ onto its first and second factors, so
 that $p_{1}$ and $p_{2}$ are birational proper morphisms. It is
 sufficient to show that $p_{1}$ is an isomorphism, since it would
 then follow that $g = \sigma^{-1} \circ f = p_{2}~ o~ p^{-1}_{1}$ is
 a morphism, Suppose then that $p_{1}$ is not an isomorphism. It then
 follows by $Z.M.T$ that there is a closed point $x$ of $X$ such that
 the fibre $p^{-1}_{1}(x)$ contains an irreducible one dimensional
 component $C$. Let\pageoriginale $z$ be the generic point  of $C$, and $y' =
 p_{2}(z)$. Then $y'$ cannot be a closed point of $Y'$, since if it
 were, $(x) \times_{Y} (y')$ would be a finite set of closed points of
 $X \times_{Y} Y'$ and $z$ would be contained in this set. Hence,
 $\mathscr{O}_{y', Y'}$ is a discrete valuation ring with is dominated
 by the local domain $\mathscr{O}_{z, Z}$ and their quotient fields
 (which are $R(Y')$ and $R(Z)$ respectively) are isomorphic. It
 follows that the homomorphism $\mathscr{O}_{y',Y'} \to
 \mathscr{O}_{z, Z}$ is an isomorphism. Further, since $Y'-L$ and $X
 - f^{-1}(y)$ are $Y$-isomorphic, both are isomorphic to $Z -
 p^{-1}_{1}(f^{-1}(y)) = Z - p^{-1}_{2}(L)$. It follows that $y'$ must
 be a point of $L$, and since $\mathscr{O}_{y',Y'}$ is a discrete
 valuation ring, $y'$ must be the generic point of $L$. 
 
 Let $u$ and $v$ be a regular system of parameters of
 $\mathscr{O}_{y}$. Then we have seen that either one of $u \circ
 \sigma$ or $v \circ \sigma$ generates the maximal ideal of
 $\mathscr{O}_{y', Y'}$. Hence either one of $u \circ \sigma \circ
 p_{2} = u \circ f ~\circ~p_{1}$ and $v	 \circ \sigma \circ p_{2} = v
 \circ f \circ p_{1}$ generates the maximal ideal of
 $\mathscr{O}_{z,Z}$. Since we have a local homomorphism
 $\mathscr{O}_{x,X}\to \mathscr{O}_{z,Z}$, it follows that none of the
 elements $u \circ f$, $v \circ f$ of $\mathscr{O}_{x,X}$ can be in
 $\mathcal{M}^{2}_{x,X}$. Now the fibre $f^{-1}(y)$ cannot contain any
 isolated closed points by $Z.M.T$, so that there is a component $D$
 of $f^{-1}(y)$ of dimension 1 which contains $x$. Let $t$ be the
 element of $\mathscr{O}_{x,X}$ which defines $D$ at $x$. Since $u
 \circ f$ and $v \circ f$ vanish on $D$, $u \circ f = t u'$, $v \circ f =
 t v'$ for some $u'$, $v' \in \mathscr{O}_{x,X}$. Since $u \circ f$ and
 $v \circ f$ do not belong to $\mathcal{M}^{2}_{x}$, $u'$ and $v'$ must
 be invertible in $\mathscr{O}_{x,X}$, so that $\dfrac{u \circ f}{v
   \circ f} = \dfrac{u'}{v'}$ (and similarly\pageoriginale $\dfrac{v \circ f}{u
   \circ f}$) is defined at $x$. Let $A$ be the co-ordinate ring of an
 affine neighbourhood of $y$ in $Y$ such that $Au + Av$ is the maximal
 ideal of $A$ which defines $y$. Identifying all the above local rings
 with subrings of the field $R(Y)$, we have the diagram  

\[
\xymatrix@M=6pt{
 & \mathscr{O}_{y', Y'} \ar[rd]& \\
  B=A \left[\dfrac{u}{v}\right] \ar@{^{(}->}[ru] \ar@{^{(}->}[rd] &  &
  \mathscr{O}_{z, Z}\\ 
  & \mathscr{O}_{x, X} \ar[ru]& 
}
\] 

From the very definition of the dilatation, $\mathscr{O}_{y',Y'} =
\mathscr{O}_{z,Z}$ is the localisation of $B$ with respect to the
prime ideal $\mathcal{M}_{y',Y'} \cap B = \mathcal{M}_{z,Z}\cap B =
\mathcal{M}_{x,X} \cap B$. Hence, $\mathscr{O}_{x,X}$ contains
$\mathscr{O}_{z,Z}$, so that $\mathscr{O}_{x,X}$ and
$\mathscr{O}_{z,Z}$ coincide and $\mathscr{O}_{x,X}$ is a discrete
valuation ring. This contradicts the fact that $x$ is a closed points,
since $X$ is two dimensional at $x$. 
 
Thus, the theorem is proved.      
