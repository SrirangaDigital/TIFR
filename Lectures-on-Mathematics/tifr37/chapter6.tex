\chapter{Intersection theory on two dimensional regular
  preschemes}\label{chap6}%chap 6 

\markright{\thechapter. Intersection theory on two dimensional regular
  preschemes}
 
Let\pageoriginale $X$ be a noetherian two dimensional regular
prescheme. We shall 
say that two divisors $D_1$ and $D_2$ on $X$ \textit{intersect
  properly at a closed point } $x \in X$ if $x$ is an isolated point
of the intersection $|D_1| \cap | D_2|$ of their supports. When $D_1$
and $D_2$ intersect properly at a closed point $x$ of $X$, we shall
associate to $D_1, D_2$ and $x$ an integer $(D_1, D_2)_x \in
\mathbb{Z}$ called the \textit{intersection multiplicity} of $D_1$
and $D_2$ at $x$. First assume that $D_1$ and $D_2$ are effective,
that is, that they are linear combinations of irreducible divisors
with non-negative integral coefficients, or equivalently, that they
are defined at any point of $X$ by functions which are regular at
$x$. Suppose $D_i$ is defined at $x$ by an element $f_i$ of the local
ring $\mathscr{O}_x$ of $x$ at $X$. Since $x$ is an isolated point of
$|D_1 | \cap | D_2|$ by assumption, the ideal $(f_1, f_2)$ in
$\mathscr{O}_x$ is primary for the maximal ideal $\mathfrak{M}_x$ of
$\mathscr{O}_x$. We then define 
\begin{equation*}
  (D_1. D_2)_x = l_{\mathscr{O}x}(\mathscr{O}_x / (f_1, f_2))
  \tag{1}\label{chap6:eq1} 
\end{equation*}
where $l_{\mathscr{O}x}(M)$ denotes the length over $\mathscr{O}_x$ of
an Artinian module $M$. This definition is obviously independent of
the choice of the $f_i$ defining $D_i$, and we have evidently
$(D_1. D_2)_x = (D_2.D_1)_x$. Suppose $D'_1$ is another effective
divisor intersecting $D_2$ properly at $x$. and let $f'_1$ be a
defining element of $D'_1$ at $x$. Let $B$ be the local\pageoriginale  ring
$\mathscr{O}_x / f_2. \mathscr{O}_x$; then $f_1$ and $f'_1$ are
not-zero-divisors in $B$, since $f_1$ and $f'_1$ do not belong to the
prime ideals associated to $f_2$ in $\mathscr{O}_x$. We have therefore 
\begin{align*}
  (D_1 + D'_1, D_2) & = l_{\mathscr{O}_x} (\mathscr{O}_x / (f_1
  f'_1,f_2))= l_B (B/f_1 f'_1 B)\\ 
  &= l_B(B/f_1 B) + l_B(f_1 B/f_1 f'_1 B)\\
  &= l_B (B/f_1 B)+ l_B (B/f'_1 B)\\
  &=(D_1. D_2)_x + (D'_1, D_2)_x \tag{2}\label{chap6:eq2}
\end{align*}

If now $D_1$ and $D_2$ are arbitrary (not necessarily effective)
divisors intersecting properly at $x$, we can write $D_i = D'_i -
D''_i$ where $D'_i, D''_i$ are effective and either one of $D'_1,
D''_1$ intersects either one of $D'_2, D''_2$ properly at $x$, and we
define 
$$
(D_1.D_2)_x = (D'_1. D'_2)_x + (D''_1. D''_2)_x -
(D'_1.D''_2)_x - (D''_1.D'_2)_x. 
$$

It follows from what we have shown above that this integer is
independent of the representation of $D_i$ as
$D'_i - D''_i$. Commutativity and biadditivity of intersection
multiplicity continue to hold for arbitrary divisors (which intersect
properly). 

Now suppose $X$ is a \textit{ B-scheme of finite type } fulfilling the
condition stated at the beginning  of the last paragraph, and let $\varphi
: X \to B$ be the structural morphism. Let $b$ be a point of $B$ and
$D$ a divisor on $X$ such that $|D|\subset \varphi^{-1}(b)$  and $|D|$
considered as a reduced\pageoriginale  scheme is \textit{proper over
  the residue 
  residue field $k(b)$ at $b$}. Since any closed point $x$ of $X$
belonging to $\varphi^{-1}(b)$ is also a closed point of
$\varphi^{-1}(b)$ is a scheme of finite type over $k(b)$, it follows
from Hilbert's Nulletellensatz that $k(x)$ is a finite algebraic
extension of $k(b)$. Let $D'$ be any divisor on $X$ which does not
have any common component with $D$, so that $D$ and $D'$ intersect
properly at any point of $|D|\cap|D'|$ and $|D|\cap|D'|$ is a finite
set of closed points of $X$. We define \textit{the intersection
  number} $(D. D')$ of $D$ and $D'$ (over $b$) as 
\begin{equation*}
  (D. D')= \sum_{x \in |D|\cap|D'|}[k(x) : k(b)]. (D. D')_x
  \tag{3}\label{chap6:eq3} 
\end{equation*}

We shall show that if $D''$ is another divisor which is linearly
equivalent to $D'$ and has no common component with $D$, we have 
\begin{equation*}
(D. D')=(D. D'')\tag{4}\label{chap6:eq4}
\end{equation*}

We may clearly assume $D$ to be an irreducible divisor, and $X$ to be
irreducible. We have to show that if $f \in R(X), f \neq 0$ and div
$(f)$ does not contain $D$ as a component, $(D. div(f)) = 0$. Let $C$
be the reduced subscheme of $X$ with $|D|$ for its underlying set, so
that $C$ is an integral scheme, proper over $k(b)$ and of dimension
one (a \textit{ curve } over $k(b)$, for short). By assumption, the
restriction $g$ of $f$ to $C$ is defined as a rational function on
$C$, and $g$ is not identically $0$. 

We\pageoriginale recall the definition of the \textit{order} of a
non-zero rational 
function $g$ on a curve a field $k$ at a closed point $x$ of $C$. If
$g$ belongs to the local ring $\mathscr{O}_{x,C} $ of $C$ at $x$, we
define the order $o_x(g)$ of $g$ at $x$ to be the integer
$l_k(\mathscr{O}_{x/g\mathscr{O}_x})$. Since for
$g,g'\in\mathscr{O}_x$, both different from $0$, we have  
{\fontsize{10}{12}\selectfont
$$ 
O_x(gg')=l_k(\mathscr{O}_{x/gg'\mathscr{O}_x})= l_k
\left(\dfrac{\mathscr{O}_x}{g\mathscr{O}_x}\right) + l_k
\left(\dfrac{g\mathscr{O}_x}{gg'\mathscr{O}_x}\right) + l_k
\left(\dfrac{\mathscr{O}_x}{g'\mathscr{O}_x}\right) =
\mathscr{O}_x(g)+\mathscr{O}_x(g'),
$$}\relax    
we can extend the definition of $O_x$ uniquely to the multiplicative
group $R(C)^*$ of non-zero elements of $R(C)$ such that $O_x:R(C)^*\to
\mathbb{Z}$ is a homomorphism. With this definition, it is clear that
we have, for $D$ and $f$, and $g$ as in the previous paragraph,  
$$
(D. {\rm div} (f))=\sum_{x\in C} O_x(g)
$$

Thus, if we can show that the sum of the orders of a non-zero element
$g$ of $R(C)$ over all closed points of a complete curve $C$ over a
field $k$ is $0$, the proof of (\ref{chap6:eq4}) would be complete. We shall now
prove this statement. Let $C'$ be the normalisation of $C$ in its
field $R(C)$ of rational functions, and $\pi:C'\to C$ the associated
morphism. We identify $R(C)$ and $R(C')$  by means of $\pi$. Let $x$
be a closed point of $C$ and $x'_1,\ldots,x'_n$ be the points of the
fibre $\pi^{-1}(x)$. We shall show that  
$$
O_x(g)=\sum^n_{i=1} 0_{x'_i}(g)
$$

We may\pageoriginale  assume that $g\in O_x$, because of the additivity of the
order. Let $\overline{\mathscr{O}_x}$ be the integral closure of
$\mathscr{O}_x$, so that $\overline{\mathscr{O}_x}$ is a semilocal ring
and the local rings $\mathscr{O}_{x'_1,C'}$ are the localisations of
$\overline{\mathscr{O}_x}$  at the distinct maximal ideals of
$\overline{\mathscr{O}_x}$. Further $\overline{\mathscr{O}_x}$ is an
$\mathscr{O}_x$ -module of finite type. We have therefore
\begin{multline*}
  \sum\limits^n_{i=1} o_{x_l(g)}=l_k
  \left(\frac{\overline{\mathscr{O}_x}}{g{\overline{\mathscr{O}_x}}}\right)
  = l_k \left(\frac{\overline{\mathscr{O}_x}}{g{\mathscr{O}_x}}\right) -
  l_k \left(\frac{g\overline{\mathscr{O}_x}}{g{\mathscr{O}_x}}\right)\\
  =l_k \left(\frac{\overline{\mathscr{O}_x}}{g{\mathscr{O}_x}}\right)
  = l_k \left(\frac{\overline{\mathscr{O}_x}}{{\mathscr{O}_x}}\right)
  = l_k \left(\frac{\overline{\mathscr{O}_x}}{g{\mathscr{O}_x}}\right)
  = O_x(g). 
\end{multline*}

Since we have $\sum\limits_{x\in C} O_x(g) = \sum\limits_{x\in C}
\sum\limits_{\pi (x'_i)=x} O_{x'_l}(g) =  \sum\limits_{x'\in
  C'} O_{x'}(C')$, it is sufficient to prove the equation
$\sum\limits_{x\in C}O_x(g)=0$ for a \textit{ regular } complete curve
a field $k$. This is of course well-known, and we shall indeed deduce
this form a more general lemma, which we postpone for the moment in
order not to break the continuity of the argument.  

It follows from (\ref{chap6:eq4}) ($X$ assumed noetherian, regular, two dimensional,
and separated, and of finite type over a base prescheme $B, D$ a
divisor with $|D|\subset \varphi^{-1}(b)$ where $b$ is a point of $B$
and $|D|$ proper over $k(b))$, that $(D.D')$ depends only on the
divisor class of $D'$ (that is, on the image of $D'$ in $\vartheta
(X)/\vartheta_l(X)$). We shall show that in every divisor class,
there is a divisor $D'$ which has no common components with $D$. For
this, one may clearly assume $X$ to be irreducible. Let $D''$ be any
divisor on $X$, so that we can write $D''=D''_1+D''_2$,\pageoriginale
where all the 
components of $D''_1$ are components of $D$ and $D''_2$ has no common
components with $D$. Let $C_1,\ldots,C_r$ be the irreducible component
of $D$, and $V_1,\ldots,V_r$ the corresponding discrete valuations of
the functions field $R(X)$ of $X$, Since $X$ is by assumption
separated over $B$ and $C_i$ are contained in the same fibre
$\varphi^{-1}(b)$, the valuations $V_i$ are all distinct. If we write
$D''_1 = \sum\limits^r_1 n_i C_i$, it follows from the theorem of
independence of  valuations (\cite{key14}) that we can find an $f\in
  R(X)^*$, with $V_i(f)=-n_i$. Thus, if we put $D'=D'' + div(f)$, $D'$ and
  $D''$ are linearly equivalent and $D'$ has no common component with
  $D$. Thus, we have a unique homomorphism of
  $\dfrac{\vartheta(X)}{\vartheta_l(X)}$ into $\mathbb{Z}$ which for a
  divisor class $(D')$ takes the value $(D,D')$  if $D'$ takes the
  value $(D.D')$ if $D'$ has no common components with $D$. We shall
  denote the image of a divisor class $\xi$ under this homomorphism
  again by the same symbol $(D.\xi)$.  

If the base $B$ is Spec $K$ where $K$ is a field and $X$ is proper
over $B$, it follows that we have a symmetric from
$\dfrac{\vartheta(X)}{\vartheta_l(X)} \times
\dfrac{\vartheta(X)}{\vartheta_l (X)}\to \mathbb{Z}$ 
which is defined by the condition that if $D$ and $D'$ are
divisors having no common components, the value $((D).(D'))$ of this
bilinear form on the pair $((D),(D'))$ is the intersection number
$(D.D')$. We shall again call this integer $((D).(D'))$ the
intersection number of the divisor classes $(D)$ and $(D')$. Assume
further that $K$ is algebraically closed. It can then be shown that if
$D\in \vartheta_a(X)$ (that is, if $D$ is algebraically equivalent to
$0$), $(D.D')=0$ for any $D'$. Thus, the above pairing `goes down' to
a pairing\pageoriginale  of the Neron-Severi group
$S(X)=\dfrac{\vartheta(X)}{\vartheta_a(X)}$. It has been shown that
for any element $\xi \in S(X), (\xi .\eta)=0$ for all $\eta \in S(X)$
if and only if $\xi$ is a torsion element of $S(X)$ (\cite{key22}). Thus, if we
put $S_{Q}(X)=Q\otimes_{\mathbb{Z}}S(X)$, we have a non-degenerate
rational symmetric bilinear form on the finite dimensional vector space
$S_Q(X)$. It is also known that the signature of this form is
$(+,\underbrace{-,\ldots,-}_{r-1})$ where $r$ is the dimension of
$S_Q(X)$. We will neither prove nor use any of these results
mentioned. 

Next, suppose the base prescheme $B$ is noetherian, regular and of
dimension one and $\pi : X\to B$ is surjective. Any closed point $b\in B$
determines a divisors of $B$, which is defined by a generator of the
maximal ideal $\mathfrak{M}_b$ of the local ring $\mathscr{O}_{b, B}$ at
$b$ and by $1$ at all other points of $B$. We shall denote the inverse
image of this divisor under the structural morphism $\pi:X\to B$ also
by $\pi^{-1}(b)$, and it will always to be clear from the contest
whether we mean by $\pi^{-1}(b)$ this divisor or the closed subscheme
of $X$ which is the fibre at $b$. We now have the following lemma,
which when $X$ is proper over an algebraically closed field, is a
special case of the invariance of intersection number under algebraic
equivalence stated int the earlier paragraph.  

\begin{lemma*}%\lemm
  Let $B$ noetherian irreducible one-dimensional regular presch\-eme and
  let $\pi: X\to B$ be a surjective proper morphism, where $X$ is two
  dimensional and regular. If $D$ is a divisor on $X$ and $b_1,b_2$ are
  two closed points of $B$ such that $D$ intersects $\pi^{-1}(b_i)$
  properly, we have\pageoriginale   
  \begin{equation*}
    (D.\pi^{-1}(b_1))=(D,\pi^{-1}(b_2)).\tag{5}\label{chap6:eq5}
  \end{equation*}
\end{lemma*}

\begin{proof}
  We may clearly suppose that $B-Spec$ $A$ Where $A$ is a Dedekind
  domain, and also that $D$ is irreducible. Let us denote the reduced
  subscheme of $X$ having $|D|$ for support by the same symbol $D$,
  and let $\pi_{1}$ denote the restriction of $\pi$ to $D$. Since
  $\pi_{1}(D)$ is a closed irreducible subset of $B$, it must either
  be a single closed point $b$ of $B$ or it must be the whole of
  $B$. In the former case, we have necessarily $b\neq b_i$, so that
  both sides of the above equation (\ref{chap6:eq5}) become. In the latter case,
  it is easily seen that (Since both $D$ and $B$ are one dimensional
  and $\pi_1$ of finite type) $\pi^{-1}_{1}(b)$ is finite for any
  point $b  B$. Since $\pi_1$ is also proper, it follows from a
  theorem of Chevalley [$\EGA$ III, 4.4.2] that $D$ is isomorphic as a
  $B-scheme$ to Spec $C$, where $C$  is an $A$-alegbra which is an
  $A$-module of finite type. Since $D$ is irreducible and reduced and of
  dimension one, $C$ is a domain and $C$ has no. $A$-torsion. A being a
  Dedekind domain, this implies that $C$ is a projective $A$-module of
  rank $r$(=degree of quotient field of $C$ over that of $A$) say. It
  follows that for any maximal ideal $\mathfrak{M}$ of $A$, we have 
  $$
  l_A\left(\frac{C}{M C}\right)=r.
  $$
  But
  $l_A \left(\dfrac{C}{\mathfrak{M}C}\right) = r = l_{A/_\mathfrak{M}}
  \left(\dfrac{C}{\mathfrak{M}C} \right)=r$
  is clearly the intersection number $(\pi^{-1}(b)$. $(D))$, where $b$ is
  the point of $B$ defined by $\mathfrak{M}$. This proves the lemma.  
\end{proof}

We\pageoriginale shall deduce that for a regular curve $C$ over a
field $k$ and for 
an element $f\in R(C)^{*}$, we have $\sum\limits_{X\in C} O_x(f)=0$,
as promised earlier. We take $X = ~C ~X_k \mathbb{A}'(k)=\Spec
\mathscr{O}_C[T]$ and $B=\mathbb{A'}(k)=\Spec k[T]$ in the above
lemma. Let $D$ be the principal divisor on $X$ defined by the rational
function $g=T f+(1-T)$ on $X$. Let $0$ and $1$ be the points of
$\mathbb{A'}(k)$ defined by the maximal ideals $(T)$. The fibres
$\pi^{-1}(0)$ and $\pi^{-1}(1)$ are canonically isomorphic with $C$
over $k$, and the restrictions of $g$ to $\pi^{-1}(0)$ and
$\pi^{-1}(1)$ go over into the rational functions $1$ and $f$ on $C$
respectively by means of these isomorphisms. It follows that 
$$
\sum\limits _{X\in
  C}O_x(f)=(\pi^{-1}(1).D)=(\pi^{-1}(0.)D)=\sum\limits_{x \in C}
O_x(1)=0, 
$$
which was the assertion to be proved.  

Let us return ot the case of a regular two dimensional prescheme $X$
proper over a regular one dimensional noetherian prescheme $B$. By the
lemma, for any divisor  $D$ on $X$, the intersection number
$(D.\pi^{-1}(b))$ is independent of the closed point $b$ of $B$. We
shall therefore call this integer the intersection number of $D$ with
a fibre. Suppose now $D$ itself has support $|D|$ contained in the
fibre $\pi^{-1}(b)$. Then we assert that $(D.\pi^{-1}(b))=0$. Infact,
let $f$ be a rational function on $B$ which is regular at $b$ and
which generates the maximal ideal $\mathfrak{M}_b$ of
$\mathscr{O}_{b,B}$. It is then clear for the rational function
$f\circ\pi$ on $X$, we have  
$$
\displaylines{\hfill 
  \text{div}~ (f ~ \circ \bar{x}~)=\pi^{-1}(b)+C,\hfill \cr 
  \text{where}\hfill   |C|\cap \pi^{-1}=\phi. \phantom{wherewwwwi}\hfill } 
$$\pageoriginale 
We have therefore 
\begin{align*}
(D.\pi^{-1}(b)) & = (D.\pi^{-1}(b))+(D.C)\\
  & = (D. ~\text{div} ~(f~ \circ~\pi))=0 ,
\end{align*}
which proves our assertion. Now, let $\vartheta_b$ be the subgroup of
$\vartheta(X)$ consisting of those divisors with support contained in
$\pi^{-1}(b)$. Then $\vartheta_b$ is a free abelian group on the
irreducible components of $\pi^{-1}(b)$, and $\vartheta_b$ decomposes
as a direct sum $\vartheta_b=\vartheta_1\oplus\cdots\oplus \vartheta_r$, where
each $\vartheta_i$ consists of those divisors with support contained
in a connected component $F_i$ of $\pi^{-1}(b)$. It is then cleat that
$\vartheta_i$ and $\vartheta_j$ are orthogonal for the symmetric
bilinear form $(,)$ for $i\neq j$. Let us write $\pi^{b}=D_1+\cdots
+D_r$ where $D_i$  is an effective divisor in $\vartheta_i$. Since
$\pi^{-1}(b)$ is orthogonal to the whole of $\vartheta_b$, it follows
that each $D_i$ is orthogonal to $\vartheta_i$. We shall show that for
each $C\in \vartheta_i, C \notin \mathbb{Z} D_i, (C^2)<0$.  

This follows immediately from a purely formal result:

\begin{lemma*}%\lemm
  Let $V$ be a Q-vector space and $C_1,\ldots,C_n$ a basis of
  $V$. Suppose that in $V$ a bilinear symmetric form $( , )$ is given
  such that: 
  \begin{enumerate}[a)]
  \item $(C_i,C_j) \geq 0$ for $i \neq j$.

  \item There exists $m_1,\ldots,m_n, m_i>0$, such that 
    $(C_i, \vartheta_0)= 0$, $i=1,2,\ldots,\break n,\vartheta_o=\sum m_i C_i$.

  \item The set $\big\{C_1 ,\ldots,C_n \big\}$ is connected, that is,
    it cannot be divided into two parts in such a way that
    $(C_i,C_j)=0$ if $C_i$ and $C_j$ are in different parts.  
  \end{enumerate}
\end{lemma*}

Then\pageoriginale $(\vartheta,\vartheta)\leq 0$, for any $\vartheta \in
V$. Moreover, if $(\vartheta, \vartheta)=0$ then $\vartheta=\alpha
\vartheta_\circ, \alpha\in Q$.  

\begin{proof}
  First, we prove by induction on $n$ that $(\vartheta,\vartheta)\leq
  0$. The cases $n=1$ and $n=2$ are easy to verify. 

Suppose the statement true for ${n-1} \geq 2$
\begin{enumerate}[1)]
\item  If $(A,A)>0$ where $A=\sum\limits^{n}_{i=1}\alpha_i C_i$,
  $\alpha_i \in \mathbb{Q}$, then the vector $A$ cannot have two
  coordinates equal to zero.  

  If, for example, $\alpha_{n-1}=\alpha_n=0$, then we take $C^*_1=
  C_1,C^*_2= C_2,\ldots,C^*_{n-2}$, $C^*_{n-2}=m_{n-1}C_{n-1}+m_n
  C_n$. Let $V^*$ be the subspace of $V$ spanned by $C^{*}_{1}, \ldots
  , C^{*}_{n-1}$ and $(,)^{*}$ be the restriction of the form $(,)$ to
  $v^*$. Then $a)$, $b)$, $c)$ are satisfied for $V^*$ and
  $C^*,\ldots, C^*_{n-1}$. Hence $(,)^*$ is semi-negative. But $A\in
  V^*$. Thus $(A,A)\leq 0$. Contradiction.  

\item If $A=\alpha_1 C_1+\cdots+\alpha_nC_n$, $(A,A) >0$ and
  $\alpha_i=0$ for some $i$ then either all the remaining coordinates
  are positive or are all negative. 
\end{enumerate}

  Assume that $\alpha_n=0$, $\alpha_1>0,\ldots,\alpha_i >0$,
  $\alpha_{i+1}< 0,\ldots,\alpha_{n-1}<0$ where $1<i<n-1$. Then
  $A=A_1-A_2$ for $A_1=\alpha_1C_1+\cdots+\alpha_iC_i$ and $
  A_2=(-\alpha_{i+1})C_{i+1}+\cdots+(-\alpha_{n-1})C_{n-1}$. It follows
  form $a)$ that $(A_1,A_2)\geq 0$. But
  $o<(A,A)=(A_1,A_1)+(A_2,A_2)-2(A_1,A_2)$. Thus either $(A_1,A_1)>0$
  or $(A_2,A_2)>0$. Since $A_1,A_2$ have at least two coordinates
  equal to zero (for $A_1$ the $n^{th}$ and $(n-1)^{st}$; for $A_2$,
  the $n^{th}$ and $1^{st}$; this contradicts $1)$ 

Let\pageoriginale  $(A,A)>0;$ we may assume that $\alpha_n=0$;
otherwise we replace 
$A$ by $A-\dfrac{\alpha_{n}}{m_n}\vartheta_\circ$. Hence, it follows
from 2) that we may assume that
$\alpha_1>0,\ldots,\alpha_{n-1}>0$. Since, $n-1\geq 2$, we may choose
$i\neq j$, $1\leq i < j \leq n-1$ such that
$\dfrac{\alpha_i}{m_i}-\dfrac{\alpha_j}{m_j}\geq 0$. Let
$B=A-\dfrac{\alpha_j}{m_j} \vartheta_o$. Then   
$$
(B,B)>0
$$
The $i^{\rm th}$ coordinate of $B=\alpha_i-\dfrac{\alpha_j}{m_j}m_i\geq
0$, the $j^{\rm th}$ coordinate of $B=\alpha_j - \alpha_{j/m_j} m_j=0$, the
$n^{\rm th}$ coordinate of $B = 0 - \dfrac{\alpha_j}{m_j} m_n<0$. As
$i\neq j$, this contradicts $2)$. Now we can prove the second
statement.  

Since $(\vartheta,\vartheta_\circ)=0$ for all $\nu \in V$, the rank of
the bilinear form $(,)$ is $\leq n-1$. We have to prove that it equals
$n-1$. Let  it be $\leq n-2$ and let 
$$
W=\bigg\{ \omega \in V: (\omega,\nu)=0 ~\text{ for all }~  \vartheta
\in V\bigg\}. 
$$ 

Then on $V/W$ we have negative definite form. Let $C_1,\ldots,C_e$ be
such that they give a basis in 
$V/W$. Then $(\alpha_1C_1+\cdots+ \alpha_e C_e)^2$ is $< 0$ except when all
$\alpha_i = 0$. Put $C_{e+1} = L + D$ where $D \in W$ and $L$ is a
linear combination of $C_1,\ldots,C_e$. Let $L=L_1-L_2$ where $L_1$
(\resp -$L_2$) contains all $C_i$ that enter in $L$ with positive
(resp. negative) coefficients. Then $(C_{e+1},L_i)\geq 0$ because of
$a)$. On the other hand, $(C_{e+1},L_1)=(L_1,L_1)-(L_2,L_1)\leq 0$. This
is possible only when $(L_1,L_1)=0$ and $(L_1,L_2)=0$. From this it
follows that $L_1=0$ and so $C_{e+1}+L_2\in W$. In particular,
$(C_{e+1}+L_2,c_i)$ is zero for all $C_i$.\pageoriginale Since we have
assume that 
dim $V/W\leq n-2$, there have to be $C_i$ that do not enter in
$C_{e+1} + L_2$ and because of $a)$ this gives a division of the set
$C_1,\ldots, C_n$, into two parts with properties contradicting $c)$. 
\end{proof}

This proves that the restriction of the symmetric bilinear form
$\vartheta_i$ has the signature $(0, \underbrace{,-,-,\ldots-}_{s_i-1})$
where $s_i$ is the number of irreducible components of the connected
component $F_i$. On the group $\vartheta_b$ itself this form is
negative semidefinite, and has null space of dimension equal to the
number $r$ of connected components of the fibre $\pi^{-1}(b)$. 

We now consider the case when the base is also of dimension two. Let
$X,Y$ be noetherian integral schemes of dimension two with $Y$
regular, and let $f:Y \to X$ be a proper surjective morphism. Let $x$
be a closed point of $X$, and let $C_1,\ldots, C_s$ be the irreducible
divisors contained in $f^{-1}(x)$. We shall show that the restriction
of $( , )$ to the free group generated  by the  $C_i$ is negative
definite. For this purpose, we fix a non-zero rational function $g$ on
$X$  which is regular at $x$ and vanishes at $x$ such that on $Y ,~if
~div(g\circ f)=\sum\limits^s_i m_iC_i+P$  where  $P$ does not contain
any $C_i$ as a component, we have $m_i > 0$, $P$. is effective in a
neighbourhood of $f^{-1}(x)$ and $P$ has non-empty proper intersection
with each $C_i$.  (This is possible, since we can find\pageoriginale
for each $i$ an 
irreducible divisor $D_i$ distinct from all the $C_j$ and intersecting
$C_i$, and we have only to take $g$ to be a function regular at $x$
and vanishing on $f(\bigcup\limits^{s}_{i = 1} D_i)$. Then $(C_i,C_j)$
and $(C_i,P)$ are defined. Let $V$ be a Q-vector space with basis
$C_1,\ldots,C_s$, $P$. We define the function $(, )$ on $V$ taking for
$(C_i,C_j), i\neq j$, and $(C_i, P)$ the values already defined. We
define $(P,P)$ and $(C_i,C_i)$  by means of the relation $(C_i,P+\sum
m_jC_j)=0$. Then $P+\sum m_j C_j$ is orthogonal to all the basis
elements and we can apply the preceding lemma. Conditions $a),b),c)$
are satisfied. Hence $( , )$ is negative definite on the space spanned
by $C_1,\ldots C_s$. This shows that the restriction of $( , )$ to the
group $\vartheta_x$ of divisors on $Y$ having support contained in
$f^{-1}(x)$ is negative definite. This is a theorem due to
Mumford. Suppose now that $X$ is a surface over an algebraically
closed field $K$ and $x$ a normal singular point. In this case, we can
find an $f:Y\to X$ such that $f$ is birational and proper $Y$ regular
by the Theorem of reduction of singularities of a surface
(\cite{key1}). Using the negative definiteness of $( , )$ on the fibre,
Mumford shows (\cite{key17}) that there is a unique inverse image operation
$f^*:\vartheta (X) \otimes_{\mathbb{Z}}Q\to
\vartheta(Y)\otimes_{\mathbb{Z}}Q $ and a unique intersection theory
of (properly intersecting) divisorial cycle on $X$, where the
intersection multiplicities are in\pageoriginale  general fractions,
such that the 
`projection formula' is valid. This intersection theory is independent
of the choice of $Y$, and enjoys all the `good' properties of
intersection theory on a non-singular variety. The denominator of the
intersection multiplicities is det $((C_i,C_j))$. 

Let $X,Y$ and $f$ be as in the beginning of the last paragraph, and
suppose further that $x$ is also regular. We define the `projection'
homomorphism $f_* : \vartheta (Y) \to \vartheta (X)$ by putting for an
irreducible divisor $C$ on $Y$,  
\begin{equation*} 
  f_* (C) = 
  \begin{cases}
    0 ~~\text{ if } \dim f(C) = 0 \\
    [k(z): k (f(z))] f (C) , \text { if } \dim f(C) = 1, \text{ where }\\
    z ~~\text{ is a generic point of } C.
  \end{cases}
\end{equation*}

Let $x$ be a closed point of $X$ and $C$ a divisor on $Y$ with $|C|
\subset f^{-1} (x)$. Then for any divisor $D$ on $X$, we have  
\begin{equation*}
(C.f^* (D)) = 0, \text { for } | C |  \subset f^{-1}
  (x)\tag{6}\label{chap6:eq6} 
\end{equation*}
since $D$ is defined in a neighbourhood of $x$ by a rational
function. Next, suppose that $C$ is any divisor on $Y$ such that $C$
and $f^* (D)$ intersect properly at all points of their intersection
on $f^{-1}(x)$. Then we have the projection formula  
\begin{equation*}
  (f_*(C).D)_x = \sum_{z \in f^{-1}(x)} [k(z):k(x)] (C.f^*
  (D))_z\tag{7}\label{chap6:eq7} 
\end{equation*}

To prove\pageoriginale  this, we may assume $X$ affine, $D =  div
(g)$ where $g$ is 
regular on $X,C$ irreducible and not a component of a fibre over
$X$. By the theorem of Chevalley mentioned earlier, $C$ and $C' =
f(C)$ are affine one dimensional schemes and $A
= \Gamma(C,\mathscr{O}_C)$ is a $\Gamma(C',\mathscr{O}_C') =
A'$-module of finite type. If $g_1$ denotes the restriction of $g$ to 
$C',\mathfrak{M}'$ the maximal ideal of $x$ in $A'$ and $\mathfrak{M}_i$
the maximal ideals of $A$ lying over $\mathfrak{M}'$, we have 
\begin{align*}
  (f_*(C).D)_x = n(C'.D)_x & = n
  l_{A'}
  \left(\frac{A'_{\mathfrak{M}'}}{g_1.A'_{\mathfrak{M}'}}\right),n=[R(C)
    : R(C')],\\  
  \text{and}\;  
  \sum_{z\in f^{-1}(x)} [k(z):k(x)] (C.f^*(D))_z & =
  \sum_i[k(z_i):k(x)] l_{A_{\mathfrak{M}_i}}
  \left(\frac{A_{\mathfrak{M}_i}}{g.A_{\mathfrak{M}_i}}\right)\\ 
  & =  l_{A'_{\mathfrak{M}'}} \left(\frac{A\otimes_{{A}'}A'
    _{\mathfrak{M}'}}{g A\otimes_{{A}'}A'_{\mathfrak{M}'}}\right). 
\end{align*}

If we put $B = A \otimes_{A'}A'_{\mathfrak{M}'}$, and if $\theta\in A $
generates $R(C)$ over $R(C'),\break B/A'_{\mathfrak{M}'}[\theta]$ is of finite
length over $A'_{\mathfrak{M}'}$ and $A'_{\mathfrak{M}'}[\theta]$ if
free of rank  $n$ over $A'_{\mathfrak{M}'}$. We have therefore 
\begin{align*}
  l_{A'_{\mathfrak{M}'}}(B/gB) & =
  l_{A'_{\mathfrak{M}'}}(B/gA'_{\mathfrak{M}'}[\theta])
  -l_{A'_{\mathfrak{M}'}} \left(\frac{gB}{gA'_{\mathfrak{M}'}[\theta]}\right)\\ 
  & = l_{A'_{\mathfrak{M}'}}
  \left(\frac{B}{A'_{\mathfrak{M}'}[\theta]}\right) +
  l_{A'_{\mathfrak{M}'}} \left(\frac{A'_{\mathfrak{M}'}[\theta
  ]}{gA'_{\mathfrak{M}'[\theta]}}\right) -l_{A'_{\mathfrak{M}'}}
  \left(\frac{B}{A'_{\mathfrak{M}'}[\theta]} \right)\\
  & = l_{A'_{\mathfrak{M}'}}  \left(\frac{A'_{\mathfrak{M}'}}{gA'_{\mathfrak{M}'}} 
  \otimes_{A'_{\mathfrak{M}'}}A'_{\mathfrak{M}'}[\theta] \right) =
  n l_{A'_{\mathfrak{M}'}} \left(\frac{A'_{\mathfrak{M}'}}{gA'_{\mathfrak{M}'}}\right)
\end{align*}\pageoriginale 
which prove (\ref{chap6:eq7}). One deduces immediately from (\ref{chap6:eq6}),
and (\ref{chap6:eq7}) that when $X$ (and hence also $Y$) is \textit{proper
  over a field} $K$, we have  
\begin{gather*}
  (f_*(C).D) = (C.f^*(D)),\tag{8}\label{chap6:eq8}\\
  (f^*(D).f^*(D')) = n(D.D')
\end{gather*}
for $C,C'\in \vartheta(Y)/\vartheta_l(Y)$ and $D,D'\in
\vartheta(X)/\vartheta_l(X)$, where $n$ is the degree of $f$, that is, $n=
       [k(y):k(f(y))]$, $y$ being a generic point of $Y$. 

Let us apply (\ref{chap6:eq6}) and (\ref{chap6:eq7}) to the case when $Y = X'$ is the
dilatation of $X$ at a closed point $x$ and $f = \sigma$ is the
associated morphism. We put $L = \sigma^{-1}(x)\simeq \mathbb{P}' (k)$,
where $k$ denotes the residue field at $x$ as usual. If $(u,v)$ is a
system of uniformising parameters at $x$, so that $A w =
A\left[\dfrac{v}{u}\right]$ is the\pageoriginale  ring of an affine
open set $Y_o$ on 
$X'$ ($\Spec A$ being a suitable affine neighbourhood of $x$ on $X$)
and if $D$ = div $(v)$ on $X$, we have $\sigma^*(D) = \sigma'(D) +
L,\sigma'(D)$ being defined in $Y'_o$ by the function $w =
\dfrac{v}{u}\in A[w]$. But now,
$$
(\sigma'(D).L) = l_A
\left(\frac{A[w]}{(w,u)}\right) = l_A \left(\frac{A[w]}{(u,v,w)}\right) =
l \left(\frac{A}{\mathfrak{M}}\right) = 1.
$$ 

Thus we obtain (\ref{chap6:eq6}), 
\begin{align*}
  0 = & (\sigma^*(D).L) = 1 + (L^2), \tag{9}\label{chap6:eq9}\\ 
  & (L^2) = 1 
\end{align*}

Further, for any divisor $D$ on $X$ of multiplicity $l$ at $x$, we
have proved in lecture \ref{chap2} that $\sigma^*(D) = \sigma'(D) + l.L$. We
thus obtain from (\ref{chap6:eq6}), 
  \begin{align*}
    0 = \; & (\sigma^*(D).L) = (\sigma'(D).L) + l(L^2) =
    (\sigma'(D).L) -l,\\ 
    \; & (\sigma'(D).L) = l = \text{ Multiplicity of $D$ at $x$}
  \tag{10}\label{chap6:eq10}
  \end{align*}

Suppose $D_i(i = 1,2)$ are two divisors on $X$ of respective
multiplicities $l_i(i = 1,2)$, and suppose $D_i$ intersect properly at
$x$. Since $\sigma_*(\sigma'(D_1)) = D_1$, we have by (\ref{chap6:eq7}), 
$$
(D_1.D_2)_x = l_2(\sigma'(D_1).L) + \sum_{z \in L} [k(z):k(x)]
(\sigma'(D_1) . \sigma' (D_2))_z, 
$$
so that by (\ref{chap6:eq10}),
\begin{equation*}
  \sum_{z\in L}[k(z):k(x)] (\sigma'(D_1) ). \sigma'(D_2))z
  =(D_1.D_2)_x -l_1 l_2 \tag{11}\label{chap6:eq11}
\end{equation*}

Using this formula (\ref{chap6:eq11}) we can very easily give a proof of the
elimination of indeterminacies of a rational function f on $X$,
without the hypothesis\pageoriginale that $X$ be a Japanese
scheme. (This was promised in 
Lecture \ref{chap4}). Suppose in fact that div $(f) = D_1-D_2$ where $D_i$ are
effective divisors without common component. We have shown in Lecture
\ref{chap4} that a closed point $x$ of $X$ is an indeterminacy point of $f$ if
and only if $x \in |D_1|\cap|D_2|$. Let us put in this case $\nu
(x,f)= (D_1.D_2)_x$. 
On the blown up scheme $X'$, we have div $(f \circ \sigma)=
\sigma^*(D_1)-\sigma^*(D_2)=\sigma'(D_1)-\sigma'(D_2)+(l_1-l_2)$. $L$,
where $l_i$ is the multiplicity of $D_i$ at $x$. Suppose for instance
that $l_1 \geq l_2$, so that div $(f \circ
\sigma)=(\sigma'(D_1)+(l_1-l_2)L)-\sigma'(D_2)$, and $\sigma'
(D_1)+(l_1-l_2)L$ and $\sigma' (D_2)$ are effective and have no common
components. We have therefore 
\begin{align*}
  \sum_{z \in \bar{\sigma}^1 (x)} \nu (z,f \circ \sigma)[k(z): k(x)]& =
  \sum_{z \in \bar{\sigma}^1(x)} [k(z):k(x)](\sigma'(D_1). \sigma'(D_2))_z\\ 
  & \hspace{2cm}+(l_1 -l_2)(L. \sigma'(D_2))\\
  & =(D_1.D_2){_x}- l_1l_2 + (l_1-l_2).l_2\\
  & = \nu (x, f)-l^2_2,
\end{align*}
so that in particular $\nu (z, f \circ \sigma)< \nu (x,f)$ for any $z
\in \sigma^{-1}(x)$. Since a decreasing sequence of non-negative
integers must terminate, it follows that after a finite number of
blowings up at points lying over $x$, we must have $\nu (*,f)=o$ at
all points lying over $x$, or what is the same $f$ has no
indeterminates at point lying over $x$. This\pageoriginale completes
the proof of the theorem on elimination of indeterminates in the
general case. This proof is due to Averbouh.  

We have proved that when $X$ is a locally noetherian regular, two
dimensional prescheme, $X'$ the prescheme obtained by blowing up a
closed point $x$ of $X$ and $L$ the fibre over $x$, then $L \simeq
\mathbb{P'}(k(x))$ and $(L^2)=-[k(x):k(b)]$. The converse of this
theorem also holds under suitable assumptions of projectivity, and
this is a celebrated theorem due to Castelnuovo. 

\begin{theorem*}[(Castelnuovo)]%the
  Let $B$ be a locality noetherian prescheme and $X'$ a projective
  regular $B$-scheme with structural morphism $\pi' : X' \rightarrow
  B$. Let $b$ be a closed point of $B$ and $\mathscr{I}$ an invertible
  sheaf of ideals of $\mathscr{O}_{X'}$, such that 
  \begin{enumerate}[\rm (i)]
  \item the closed subscheme $L$ defined by $\mathscr{I}$ (with
    structure sheaf as the restriction of
    $\mathscr{O}_{X'}/\mathscr{I}$) is contained in the fibre
    $\pi'^{-1}(b)$, and is isomorphic to a projective line
    $\mathbb{P'}(K)$ over an extension $K$ of $k(b)$. ($K$ is
    necessarily a finite algebraic extension of $k(b)$, by Hilbert's
    Nullstellensatz); 

  \item the restriction $\mathscr{I}
    \otimes_{X'}\mathscr{O}_L=\mathscr{I}/_{\mathscr{I}^2} |L$ of
    $\mathscr{I}$ to $L$ is the unique inversible sheaf on
    $\mathbb{P}'(K)$ of degree $ + 1$. 
  \end{enumerate}
\end{theorem*}

Then there is a unique (upto isomorphism) projective $B$-scheme $X$ and
a projective. $B$-morphism $\sigma : X' \rightarrow X$ such that
$\sigma (L)$ is a closed point $x$ of $X$ with the local ring
$\mathscr{O}_{x,X}$ regular, two dimensional, and residue field
$k(x)=K$, and $\sigma$ induces an isomorphism of $X'-L$ onto $X-
\{x\}$. 

\begin{proof}
  We\pageoriginale may assume that $X'$ = Proj $(\mathscr{S}')$ where
  $\mathscr{S}' 
  = \sum\limits_{n \geq o } \mathscr{S}'_n$ is a quasi - coherent
  sheaf of graded $\mathscr{O}_B$-algebras of positive degrees such
  that $\mathscr{S}'_1$ is $\mathscr{O}_B$-coherent and generates
  $\mathscr{S}'$ over $\mathscr{O}_B$. Let $\mathscr{O}_{X'}(1)$ be the
  fundamental sheaf on $X$ ($\EGA$, II, (3.2.5.1)). Because of the theorem
  of Serre ($\EGA$, III, 2.2.1), we may assume, be replacing
  $\mathscr{S}'$ by  $\underset{n \geq o}{\oplus} \mathscr{S}nd$ for a
  suitably large $d$, if necessary, that $R'
  \pi'_*(\mathscr{O}_{X'}(1))$ is zero in a neighbourhood of
  $b$. Since the restriction $\mathscr{O}_L \otimes_{\mathscr{O}_{X'}}
  \mathscr{O}_{X'}(1)$ of $\mathscr{O}_{X'} (1)$ to $L$ is very ample
  for the morphism $L  \rightarrow B$ ($\EGA$, II, (4.4.10)),
  $\mathscr{O}_{X'}(1) 
  \otimes_{\mathscr{O}_{X'}}\mathscr{O}_L$ is an inversible sheaf of
  \textit{ positive } degree $k$ on $L$. We denote by $\mathscr{L}$
  inversible sheaf  
  $$
  \mathscr{L}= \mathscr{O}_{X'}(1) \otimes_{\mathscr{O}_{X'}}
  \mathscr{I}^{\otimes (-k)}. 
  $$

  The direct image $\pi'_*(\underset{n \geq o} {\oplus}
  \mathscr{L}^{\otimes n}) $ is a quasi- coherent sheaf of graded
  $\mathscr{O}_B$-algebras and each $\pi'_*(\mathscr{L}^{\otimes n})$
  is $\mathscr{O}_B$-coherent ($\EGA$, III, 2.2.1). Let $\mathscr{S}$ be
  the subsheaf of graded algebras of $\pi'_*(\underset{n \geq o}
  \oplus \mathscr{L}^{\otimes n})$ generated over $\mathscr{O}_B$ by
  $\pi'_*(\mathscr{L})$, so that $\mathscr{S}$ is again quasi-coherent
  and is generated over $\mathscr{O}_B$ by $\mathscr{S}_{1'}$, which
  is coherent. We put $X$ = Proj $(\mathscr{S})$ and denote by $\pi$
  the structural morphism $X \rightarrow B$, so that $X$ is a
  projective $B$-scheme. 
\end{proof}

The inclusion $\mathscr{S}\hookrightarrow \pi'_*(\underset{n \geq o}
\oplus \mathscr{L}^{\otimes n})$ induces a homomorphism 
$$
 \psi : \pi^{'*}(\mathscr{S})\rightarrow \underset{n \geq o} \oplus
 \mathscr{L}^{\otimes n} 
 $$
of graded\pageoriginale $\mathscr{O}_{X'}$-algebras ($\EGA$, 0,
4.4.3), and this in turn induces an $X'$-morphism 
$$ 
\sigma_1 : G(\psi)\rightarrow Proj (\pi^{'^*}(\mathscr{S}))
$$
of an open subset $G(\psi)$ of Proj $(\underset{n \geq o} \oplus
\mathscr{L}^{\otimes n})$, which is canonically determined by
$\psi$. Now, $\mathscr{L}$ being an inversible sheaf on $X'$, the
structural morphism Proj $(\underset{n \geq o} \oplus
\mathscr{L}^{\otimes n})\rightarrow X'$ is an isomorphism, ($\EGA$ II,
(3.1.7) and (3.1.8), (iii)), and we may identify $G(\psi)$ with
an open subset of $X'$. Further, we have a canonical isomorphism of
Proj $(\pi^{'*}(\mathscr{S}))$ with $X' X_B$ Proj $(\mathscr{S})=X'
\times_B X $ as $X'$- schemes, and composing $\sigma_1$ with the projection
Proj $(\pi^{'*}(\mathscr{S}))\rightarrow X$, we get a $B$-morphism 
$$
\gamma_{\mathscr{L}, \psi} = \sigma : G(\psi) \rightarrow X.
$$

We shall show successively that $(a)~ G(\psi)$ contains $X' - L$ and
$\sigma$ restricted to $X' - L$ is an open immersion; $(b)~ G(\psi)$
contains $L$, so that $G(\psi) = X'$, and $\sigma(L)$ is a closed
point of $X$ not belonging to $\sigma(X' - L); \sigma(X' -
L)\cup\sigma(L) = X;$ and $(c) ~\mathscr{O}_{x,X}$ is a regular
(noetherian) local ring of dimension two with residue field $k(x) =
K$. 

Since the definitions of $\mathscr{S}$, $X,G(\psi)$ and $\sigma$ are
compatible with restriction to an open subset of $B$, we can restrict
ourselves to the case when $B = \Spec A$ is affine with $A$ a
noetherian ring. We shall make no further assumptions in the proof of
$(a)$, but for the proof of $(b)$, and $(c)$, we shall assume
$H'(X',\mathscr{O}_{X'}(1)) = 0$. It is clearly\pageoriginale
sufficient to give the proofs in these cases. 

Hence, suppose that $B = \Spec A$ where $A$ is a noetherian ring so
that $X' = \text { Proj } (S')$, where $S'$ is a graded $A$-algebra of
positive degrees such that $S'_1$ is an A-module of finite type which
generates $S'$ over $A$. With $\mathscr{L}$ as above, let $S$ be the
$A$-subalgebra of $\underset{n\geq
  0}{\oplus}\Gamma(X',\mathscr{L}^{\otimes^{n}})$ generated over $A$ by
$\Gamma(X',\mathscr{L})$ so that $X = \text { Proj } S$. Let
$i:S\rightarrow  \underset{n\geq
  0}{\otimes}\Gamma(X',\mathscr{L}^{\otimes^{n}})$ be the inclusion
homomorphism of $A$-algebras. For an element $f \in S'_{d}(d>0)$, we
denote as usual by $D_+(f)$ the affine open set of $X'$ consisting of
those prime ideals in Proj $(S)$ which do not contain $f$. If $S'_f$
is the ring of quotients of $S'$ with respect to the multiplicatively
closed set $1,f,f^2,\ldots,S'_f$ is a graded $A$-algebra, and
$S'_{(f)}$ shall denote the $A$-algebra of elements of degree $\circ$
in $S'_f$. We 
then have a canonical isomorphism of affine schemes $D_+(f)\simeq
\Spec S'_{(f)}$, by the very definition of Proj $(S')$. Let us denote
by $\circ$ the restrictions $\varrho$ the restriction homomorphism
$\underset{n\geq 0}{\oplus}\Gamma(X',\mathscr{L}^{\otimes
  n})\rightarrow \underset{n\geq
  0}{\oplus}\Gamma(D_+(f),\mathscr{L}^{\otimes n})$. If $f$ is such
that the inversible sheaf $\mathscr{L}$ is trivial on $D_+(f)$, then
$$
\underset{n\geq 0}{\oplus}\Gamma(D_+(f),\mathscr{L}^{\otimes
  n}) \overset{\sim}{\rightarrow} \Gamma (D_+(f), \mathscr{O}_{X'})[T]\simeq
S'_{(f)}[T] 
$$
 as graded $A$-algebras, so that Proj $(\underset{n\geq
  0}{\oplus}\Gamma(D_+ (f),\mathscr{L}^{\otimes n}))$ is $\simeq$ Proj
$(S'_{(f)}\break[T]) \simeq  \Spec  S'_{(f)}\simeq D_+(f)$
($\EGA$, II, (3.1.7)). The homomorphism $\varrho \circ i:S \rightarrow
\underset{n\geq 0} {\oplus} \Gamma(D_+(f),\mathscr{L}^{\otimes n})$ of
graded algebras induces\pageoriginale a morphism of an open set of
$D_+(f)$ into $X =$ Proj  $S$, and by definition, this open set is precisely
$G(\psi)\cap D_+(f)$ and the morphism is the restriction of $\sigma $
to $G(\psi)\cap D_+(f)$.  

Suppose $x,y$ are two distinct points of $X' - L$. One can then find a
$g \in S'_1$ such that $x, y \in D_+ (g)$ and a $h \in S'_d$ such that
$D_+(h)\cap L = \phi$ and $x,y \in
D_+(h)$. Since $\mathscr{L}$ is isomorphic to $\mathscr{O}_{X'}(1)$ on
$X' - L$ and $\mathscr{O}_{X'}(1)$ is trivial on $D_+(g)$, it follows
that $\mathscr{L}$ is trivial on $D_+ (gh)$ and $x,y \in
D_+(gh)$. Thus, in order to prove $(a)$, it is sufficient to show that
for any $f \in S'_d$ such that $D_+(f) \cap L = \phi$ and
$\mathscr{L}$ is trivial on $D_+(f),G(\psi) > D_+(f)$ and $\sigma$ is
an open immersion restricted to $D_+(f)$. 

Let $\tau$ be the section of $\mathscr{I}^{-1} = $ Hom
$_{\mathscr{O}_{X'}}(\mathscr{I},\mathscr{O}_{X'})$ corresponding to
the canonical inclusion of $\mathscr{I}$ in $\mathscr{O}_{X'}$, so
that we have $\tau(x) \neq 0 $ if $ x\notin L$. Let $\theta :
\underset{n\geq 0}{\oplus}\mathscr{O}_{X'}(1) \rightarrow \underset{n
  \geq 0}{\oplus}\mathscr{L}^{\otimes n}$ be the homomorphism of
sheaves of graded algebras defined by $\theta_n(a_1 \otimes \cdots
\otimes a_n) = (a_1 \otimes \tau^k)\otimes (a_2 \otimes \tau^{k})
\otimes \cdots \otimes(a_n \otimes \tau^{k})$. Then the restriction of
$\theta$ to $X' - L$ is an isomorphism. 

Let $\alpha : S' \rightarrow \underset{n\neq o}{\oplus}\Gamma
(X',\mathscr{O}_{X'}(n))$ be the canonical homomorphism
($\EGA$, II, (2.6.2.3)). Since $S'$ is generated over $A$ by $S'_1$ and
$S$ is the $A$-subalgebra of $\underset{n\geq 0}{\oplus}\Gamma
(X',\mathscr{L}^{\otimes n})$ generated by $\Gamma(X',\mathscr{L})$,
it follows that $\Gamma (\theta) o \alpha (S') \subset S$. We thus
have a commutative diagram\pageoriginale  
\[
\xymatrix{S'\ar[r]^>>>>>\alpha\ar[d]^{\gamma = \Gamma(\theta).\alpha} &
  \ar[d]^{\Gamma(\theta)} \sum\limits_{n \geq
    0}\Gamma(X',\mathscr{O}_{X'}(n))  
  \ar[r]^{\varrho'} &  \sum\limits_{n \geq 0}\Gamma
  (D_+(f),\mathscr{O}_{X'}(n)) \ar[d]^{\Gamma(D_+(f),\theta)}\\ 
  S \ar[r]^>>>>>i &  \sum\limits_{n\geq o} \Gamma
  (X',\mathscr{L}^{\otimes^n}) \ar[r]^>>>>>\varrho & \sum\limits_{n \geq
    o} \Gamma (D_+(f), \mathscr{L}^{\otimes n}) = C
}
\]
of $A$-algebras (\, $\varrho' $ and $\varrho $ are restriction
maps). Suppose $f$ satisfies the conditions mentioned at the end of
the previous paragraph. Since $\theta $ is an isomorphism restricted
to $D_+(f),\Gamma (D^+(f),\theta $) is an isomorphism. If $f$ is of
degree $d,\Gamma(D_+(f),\mathscr{O}_{X'}(d))$ is generated over
$\Gamma(D_+(f),\mathscr{O}_{X'}) = S'_{(f)}$ by the element $\varrho'~
o \alpha (f)$. Hence $\varrho ~\circ ~i(\gamma(f))$ generates
$\Gamma(D_+(f),\mathscr{L}^{\otimes d})$ over
$\Gamma(D_+(f),\mathscr{O}_{X'})$. But any homogeneous prime ideal of
$C$ which contains $C_d = \Gamma(D_+(f),\mathscr{L}^{\otimes d})$
necessarily contains $C^+ = \underset{n\geq 1}{\oplus} C_n$. It
follows that the morphism $\sigma|D_+(f) \cap G(\psi)$ maps
$D_+(f)\cap G(\psi)$ into the open subset $D_+(\gamma(f)) = \Spec
S_{(\gamma(f))}$ of $X$. This induces a commutative diagram 
\[
\xymatrix{S'_{(f)} \ar[d] \ar[r] & \Gamma (D_+ (f),  \mathscr{O}_{X'})
  \ar@{=}[d]\\ 
S_{(\gamma (f))} \ar[r] & \Gamma (D_+ (f), \mathscr{O}_{X'}).
}
\]

The hupper\pageoriginale  homomorphism is an isomorphism, by
definition of the 
structure sheaf $\mathscr{O}_{X'}$ on Proj $(S')$ and the definition
of $\alpha$. The lower homomorphism is injective because of ($\EGA$,
I, (9.3.1)). Hence
$S_{(\gamma(f))}\overset{\sim}{\rightarrow}\Gamma(D_+(f),\mathscr{O}_{X'})$
is an isomorphism, which means precisely that $\sigma$ when restricted
to $D_+(f)$ is an isomorphism onto $D_+(\gamma(f))\subset X$. Thus,
$\sigma$ restricted to $X' - L$ is an open immersion, and $(a)$ is
proved. 

We now prove $(b)$ under the assumption that
$H'(X',\mathscr{O}_{X'}(1)) = 0$ (and of course $B = \Spec A$,
affine). Because of ($\EGA$, II, (3.7.4)), it is sufficient to show
that there is an $s \in S_1 = \Gamma(X',\mathscr{L})$ such that $s(x)
\neq 0 $ for any $x \in L$. Now, the inversible sheaf
$\mathscr{L}\otimes_{\mathscr{O}_{X'}}\mathscr{O}_L =
\mathscr{O}_{X'}(1)\otimes_{\mathscr{O}_{X'}}\mathscr{I}^{\otimes
  (-k)}\otimes_{\mathscr{O}_L}\mathscr{O}_L$ is of degree $0$ on $L
\simeq \mathbb{P}'(K)$, and hence is isomorphic to the `trivial'
inversible sheaf $\mathscr{O}_L$ on $L$. Thus,
$H^o(L,\mathscr{L}_{\otimes_{\mathscr{O}_{X'}}} \mathscr{O}_L )\neq 0$
and any non-zero section of this sheaf on $L$ does not vanish at any
point of $L$. On the other hand, denoting by $\tau$ the `canonical'
section of $\mathscr{I}^{-1}$ as above we have the exact sequences 
$$
0 \rightarrow \mathscr{O}_{X'}(1)
\otimes_{\mathscr{O}_{X'}}\mathscr{I}^{\otimes(-\mu)}\xrightarrow{\otimes\tau}
\mathscr{O}_{X'}(1)\otimes_{\mathscr{O}_{X'}}\mathscr{I}^{\otimes}\xrightarrow{(-\mu
  -1)} \mathscr{D}_{k-\mu -1} \rightarrow 0 
$$
where $\mathscr{D}_r$ denotes the unique inversible on $L$ an degree $\gamma$. 

This leads to the cohomology exact sequence
\begin{align*}
  H^o(X',\mathscr{O}_{X'}(1)\otimes  \mathscr{I}^{\otimes (-\mu -1)})
  & \rightarrow H^o(L,\mathscr{D}_{k-\mu -1})\\ 
  &\rightarrow
  H'(X',\mathscr{O}_{X'}(1)\otimes\mathscr{I}^{\otimes(-\mu}))\\ 
  &\rightarrow H^o(X',\mathscr{O}_{X'}(1)\otimes
  \mathscr{I}^{\otimes(-\mu -1)})\\
 &\rightarrow H'(L,\mathscr{D}_{k-\mu    -1}) 
\end{align*}

Now,\pageoriginale  for $0\leq\mu\leq k-1$ we have
$H'(L,\mathscr{D}_{k-\mu -1}) = 0$ 
($EGA$, III, (2.1.12))  and since $H'(X',\mathscr{O}_{X'}(1)) = 0$ by
assumption, it follows that
$H'(X',\mathscr{O}_{X'}(1)\otimes\mathscr{I}^{\otimes(-r)}) = 0$ for
$0\leq r \leq k$ and consequently
$H^o(X',\mathscr{O}_{X'},\break (1)\otimes\mathscr{I}^{\otimes(-r)})
\rightarrow H^o(L,\mathscr{D}_{k-r})$ us surjective for $1 \leq r \leq
k+1$. In particular, $H^o(X',\mathscr{L})\rightarrow
H^o(X',\mathscr{D}_o)$ is surjective, which proves that $\sigma$ is
defined on $L$ in view of our earlier remark. To show that $\sigma(L)$
is a single point, we have to show that any section on $X'$ of
$\mathscr{L}^{\otimes n} (n\geq 1)$ vanishing at any point of $L$
vanishes everywhere on $L$. But this follows from the fact that
$\mathscr{L}$, and hence also $\mathscr{L}^{\otimes n}$ induces the
`trivial' invertible sheaf on $L$. Now by ($\EGA$ II (3.7.5)),
$\sigma$ is dominant, and since $X'$ is proper over $B,\sigma(X')$ is
closed so that $\sigma(X') = X$. Since $\pi \circ \sigma = \pi'$ is
projective and $\pi$ itself being projective and hence separated,
$\sigma$ is projective. Let $x$ be any point of $X' - L$, and let $s
\in \Gamma(X',\mathscr{O}_{X'}(1))$ such that $s(x) \neq 0$. Then the
section $\theta (s) = s \otimes \tau^{k}$ of $\mathscr{L}$ vanishes on
$L$ but not at $x$, which shows that $\sigma(x) \neq \sigma(L)$. Hence
the closed point $\sigma(L)$ does not belong to $\sigma(X' -L)$. This
proves $(b)$. 

We shall now prove $(c)$. The canonical homomorphism $\gamma:\break
\sigma^* (\mathscr{O}_X(1)) \rightarrow\mathscr{L}$ ($\EGA$ II
(3.7.9.1) is in our case an isomorphism since $\mathscr{L}$ is
generated by its global sections as we have proved above. Further, if
$\alpha : S_1 \rightarrow \Gamma (X,\mathscr{O}_X (1))$ is the natural
map ($\EGA$ II (2.6.2.2.)) the composite $\Gamma(\gamma^{b}) o \alpha
: \Gamma(X',\mathscr{L}) = S_1 \rightarrow (X',\mathscr{L})$ is the
identity, so that $\Gamma(\gamma^b):\Gamma(X,\mathscr{O}_x(1))
\rightarrow \Gamma(X',\mathscr{L})$ is surjective.\pageoriginale 

By means of the given isomorphism $L \simeq \mathbb{P'}(K)$, we can
identity $\Gamma (L, \mathscr{O}_L)$ with $K$, and we therefore have a
monomorphism of fields\break
$\frac{\mathscr{O}_{x,X}}{\mathfrak{M}_{x,X}}=k(x)\hookrightarrow
\Gamma (L, \mathscr{O}_L)=K. $ Let $i: \mathfrak{M}_x \mathscr{O}_X(1)
\hookrightarrow \mathscr{O}_X(1)$ be the canonical injection. Since
$L$ is contained  in the fibre $\sigma^{-1}(x)$, the composite map
$\sigma^{*}(\mathfrak{M}_x \mathscr{O}_X(1)) \xrightarrow{\sigma^*(i)}
\sigma^*(\mathscr{O}_X(1)) \xrightarrow{\Gamma(\gamma^b)} \mathscr{L}\to
\mathscr{D}_o$ is zero, so that we obtain a homomorphism $\gamma_1 :
\sigma^{*} (\mathfrak{M}_x \mathscr{O}_X(1)) \rightarrow
\mathscr{I}\mathscr{L}$ 
We have the commutative diagram
{\fontsize{10}{12}\selectfont
$$
\xymatrix@R=1cm@C=.75cm{O \ar[r] &  \Gamma (X, \mathfrak{M}_x
  \mathscr{O}_X (1)) 
  \ar[d]^{\Gamma(\gamma_1^b)} \ar[r] & \ar[r] \Gamma (X, \mathscr{O}_X(1))
  \ar[d]^{\Gamma (\gamma^b)} & \Gamma (X, k(x)\otimes
  \mathscr{O}_X(1)) \ar[d]^\varphi& \\
O \ar[r] &  \Gamma (X', \mathscr{I}\mathscr{L})  \ar[r]& \Gamma (X',
\mathscr{L})  \ar[r] & \Gamma(L, \mathscr{D}_\circ) \ar[r] & O  }
$$}\relax
where the two rows are exact and $\Gamma(\gamma^b)$ is
surjective. Hence $\varphi$ is surjective. But $\varphi$ is a
homomorphism of a one-dimensional $k(x)$ vector space onto a
one-dimensional $K$-vector space compatible with the inclusion
$k(X)\hookrightarrow K$. It follows that $k(x)=K$ and $\varphi$ is an
isomorphism. Thus $\Gamma (\gamma^b_1)$ is also surjective. 
 
 Now, choose  a section $s_o \in \Gamma (X, \mathscr{O}_X(1))$ such
 that $s_o(x) \neq 0$, so that $s_o(y) \neq 0$ for $y$ belonging to an
 affine open neighbourhood $U$ of $x$. Then $t_o=\Gamma (
 \gamma^b_1)(s_o)$ is a section of $\mathscr{L}$ over $X'$ not vanishing
 at any point of $\sigma^{-1}(U)$. We then have isomorphisms
 $\mathscr{O}_X \Big| U \overset{\sim}\rightarrow \mathscr{O}_X(1)
 \Big|  U $ and  $\mathscr{O}_X' \Big| \sigma^{-1} (U)
 \overset{\sim}\rightarrow \mathscr{L} \Big|\sigma^{-1}(U)$ which
 carry the identity\pageoriginale sections of $\mathscr{O}_X \Big| U $ and
 $\mathscr{O}_{X'} \big|\sigma^{-1}(U)$ into the sections $s_o$ and
 $t_o$ of $\mathscr{O}_X(1) \Big| U \text { and } \mathscr{L} 
 \bigg | \sigma^{-1}(U)$ respectively. By means of these isomorphisms,
 $\gamma^b$ transforms into the natural isomorphism $\sigma^{*}_U(
 \mathscr{O}_X \Big| U) \overset{\sim}\rightarrow \mathscr{O}_{X'}
 \Big| \sigma^{-1}(U)$, where  $\sigma_U: \sigma^{-1}(U) \rightarrow
 U$ is the restriction of $\sigma$, and $\gamma^b_1$ transforms into
 the natural homomorphism $\sigma^{*}_U( \mathfrak{M}_x \mathscr{O}_X
 \Big| U) \rightarrow \mathscr{I} \mathscr{O}_{X'}
 \Big|\sigma^{-1} (U)$. Now, since $\Gamma (\gamma^b_1) : \Gamma (X,
 \mathfrak{M}_x \mathscr{O}_X (1)) \to \Gamma (X' , \mathscr{I}
 \mathscr{L})$ and $j: \Gamma (X', \mathscr{I}
 \mathscr{L}) \rightarrow \Gamma \left(L ,
 \dfrac{\mathscr{I}\mathscr{L}}{\mathscr{I}^2 \mathscr{L}}\right)$ are both
 surjective, so is the composite $j o \Gamma (\gamma^b_1):\Gamma \left(X,
 \mathfrak{M}_x \mathscr{O}_X (1)\right) \rightarrow \Gamma \left(L,
 \dfrac{\mathscr{I} \mathscr{L}}{\mathscr{I}^2 \mathscr{L}}\right)$
 surjective. A fortiori,the corresponding map when we replace $\Gamma
 (X, \mathfrak{M}_x \mathscr{O}_X (1))$ by the `bigger' $\Gamma \left(U ,
 \mathfrak{M}_x \mathscr{O}_X (1)\right)$ is also surjective. Since the image
 of $\Gamma \left(U, \mathfrak{M}^2_x \mathscr{O}^{(1)}_X\right)$ in $\Gamma (L,
 \mathscr{I}/ \mathscr{I}^2 )$ is zero, we deduce that the
 `characteristic map' 
 $$ 
 \psi_1: \frac{\mathfrak{M}_x}{\mathfrak{M}^2_x} \rightarrow \Gamma(L,
 \mathscr{I} / \mathscr{I}^2) 
 $$   
 is surjective. Now, $\dim \mathscr{O}_{x,X}=\dim_x X \ge 2$, since the
 projection $\sigma (C)$ of an irreducible closed subscheme $C$
 through a closed point $p$ of $L$, such that  $\dim_p C=1$ and $p$
 is an isolated point  of $C \cap L$, satisfies the conditions that $x
 \in \sigma (C)$, $\sigma (C)$ is closed irreducible but $\sigma(C)$ is not an
 irreducible component of $X$. Further, $\Gamma (L, \mathscr{I}/
 \mathscr{I}^2) \simeq \Gamma (L , \mathscr{D}_1)$ is a two
 dimensional vector space over $K$. If we can show that $\psi_1$ is
 also\pageoriginale injective, it would follow that
 $\mathscr{O}_{x,X}$ is a regular  local ring of dimension 2.  
 
 Now, as a sheaf of $\mathscr{O}_L-$ algebras, $\underset{n \ge
   0}\oplus  \mathscr{I}^n / \mathscr{I}^{n+1} \simeq \underset{n \ge
   0} \oplus \mathscr{D}^{\otimes n}_1 $ since
 $\dfrac{\mathscr{I}}{\mathscr{I}^{n+1}} \simeq \mathscr{I}^n \otimes
 _{\mathscr{O}_X} \mathscr{O}_L \simeq(
 \mathscr{I}_{\otimes_{\mathscr{O}_X}} \mathscr{O}_L )^{\otimes n}
 \simeq \mathscr{D}_1^{\otimes n}$, Hence the $K$-algebra $\underset{n
   \ge 0}\oplus \Gamma (L, \mathscr{I}^n / _{\mathscr{I}^{n+1}}$
 isomorphic to $\underset{n \ge  0} \oplus \Gamma (L,
 \mathscr{D}_1^{\otimes n})$, and hence to a polynomial ring in two
 variables over  $K$. Now, for every $k \ge 0$, we have a
 $K$-homomorphism 
$$
\psi_k:
 \dfrac{\mathfrak{M}^k_x}{\mathfrak{M}^{k+1}_x} \to \Gamma (L, 
 \mathscr{I}^k /_{\mathscr{I}^{k+1}}),
$$
 and hence a homomorphism $K$-algebras 
 $$
 \underset{n \ge  0} \oplus  
 \dfrac{\mathfrak{M}^n_x}{\mathfrak{M}^{n+1}_x} \to \underset{n \ge  0}
 \oplus \Gamma \left(L, \dfrac{\mathscr{I}^n}{\mathscr{I}^{n+1}}\right).
 $$ 
 
 Since the second algebra is generated by its first degree elements
 and $\psi_1$ is surjective, $\psi$ is surjective. 
 
 Now, put $C= \Gamma (U, \mathscr{O}_X), C'= \Gamma (U,
 \sigma_{U^*}(\mathscr{O}_{X'}))= \Gamma (\sigma^{-1}(U),
 \mathscr{O}_{X'})$, and let $\mathfrak{M}$ be the maximal ideal in $C$
 defining $x$. Since $\sigma_{U*}(\mathscr{O}_{X'})$ is coherent by
 Serre's theorem, $C'$ is a C-module of finite type and
 $\sigma_{U*}(\mathscr{O}_{X'})$ isomorphic to the sheaf
 $\overset{\sim}{C'}$ associated to $C'$. Further, since $\sigma_U$ is
 an isomorphism $\sigma^{-1}(U)- \sigma^{-1}(x)$ onto $U- \{x\}$, the
 natural homomorphism $\mathfrak{M}^2_x \mathscr{O}_X \to \sigma_*(
 \mathscr{O}_{X'})$ is an isomorphism when restricted to $X -
 \{x\}$. It clearly follows that the $C$-module  $C'/Im(\mathfrak{M}^2)$
 is annihilated by $\mathfrak{M}^k$ for some suitably large $k$. On the
 other hand, if $\mathscr{O}_l$, is the ideal $\Gamma(\sigma^{-1}(U),\break
 \mathscr{I}^l)$ of $C'$, we have by ($\EGA$, III,
 (4.1.7))\pageoriginale  that for 
 $l$ sufficiently large, $\mathscr{O}_l \subset \mathfrak{M}^k C'
 \subset \iim (\mathfrak{M}^2)$. Assuming for the moment that $C
 \rightarrow C'$ is injective, we shall show that $\psi_1:
 \dfrac{\mathfrak{M}}{\mathfrak{M}^2} \to \Gamma \left(L,
 \dfrac{\mathscr{I}}{\mathscr{I}^2}\right) \simeq
 \dfrac{\mathscr{O}_1}{\mathscr{O}_2}$ is injective. We shall show by
 descending induction on $k$ that the kernel of $\mathfrak{M} \to
 \dfrac{\mathscr{O}_1}{\mathscr{O}_k}$ is contained in $\mathfrak{M}^2$
 for $k \ge 2$. Since $\mathscr{O}_l \subset \iim \mathfrak{M}^2$ for $l$
 sufficiently large, it is sufficient to show that if this assertion
 holds for $l > 2$, then it holds for $l-1$. Now, we shown above that
 $\mathfrak{M}^{l-1} \to \dfrac{\mathscr{O}_{l-1}}{\mathscr{O}_l}=
 \Gamma (L, \mathscr{I}^{l-1}/_{\mathscr{I}^l})$ is surjective. If $f
 \in \mathfrak{M}$ is mapped into $\mathscr{O}_{l-1}$, we can thus find
 $g \in \mathfrak{M}^{l-1}$ such that $(f-g)$ goes into
 $\mathscr{O}_l$. But then, we have $f=(f-g)+ g \in \mathfrak{M}^2 +
 \mathfrak{M}^{l-1}= \mathfrak{M}^2$. Thus shows that $\ker (
 \mathfrak{M}\to \mathscr{O}_l / _{\mathscr{O}_{l-1}}) \subset
 \mathfrak{M}^2$, completing the induction. 
 
 It only remains to show that $C \to C'$ is injective. Since
 $\mathscr{O}_X \rightarrow \sigma_*(\mathscr{O}_{X'})$ is an
 isomorphism on $U - \{x\}$, the kernel of $C \to C'$ is an ideal
 $\mathscr{O}$ annihilated by $\mathfrak{M}^k$ ($k$ large). Let $f$ by
 any element of $\mathscr{O}$. By definition of proj $(S)$, $f$ can be
 written as $\dfrac{\alpha}{\beta}$ with $\alpha,\beta \in \; S_d,
 \beta(x) \neq 0 $ ($\beta$ being considered as a section of $\Gamma
 (X, \mathscr{O}_X (d)))$ is zero in a neighbourhood of $x$. This
 means that when $\alpha$ is considered as an element of $\Gamma (X',
 \mathscr{L})$ it vanishes on $V-L$, where $V$ is a neighbourhood of
 $L$. Since the local rings $\mathscr{O}_{y, X'}$ are regular and
 hence  reduced, $\alpha$ vanishes on the whole of $V$. Now, $\sigma
 (X' -V)$ is a closed\pageoriginale subset of $X$ not containing $x$,
 so that find a 
 $\gamma \in S_{d'}$ such that $\gamma(x) \neq 0$ and $\gamma (y)=0$
 for $y \in \sigma (X-V)$. It then follows that
 $\alpha. \gamma^{d''}=0$ in $S_{d+d' d''}$ for $d''$ sufficiently
 large ($\EGA$, I, (9.3.1)). But this means that $f=
 \dfrac{\alpha}{\beta}=0$ in $\mathscr{O}_{x,X}$, so that
 $\mathscr{O}= (0)$. 
 
 The proof of the theorem is complete.
 
\begin{remark*}%rema
\begin{enumerate}
  \item 
   When $X'$ is a projective variety over an algebraically closed
  field, the proof  may be summarised as follows. Choose a projective
  imbedding such that $H' (X' , \mathscr{O}_{X'}(1))=0$. If $H$ is the
  linear system of hyperplane sections, the complete linear system $|
  H + kL|$, where $k=(H.L)$, has no base points, and thus defines a
  morphism of $X'$ into a projective space. This morphism is an
  immersion restricted to  $X' - L$ and contracts $L$ to a simple
  point of the image variety. 
 
  It is only because we have not developed the machinery of projective
  imbedding by linear systems in sufficient generality in Lecture \ref{chap3}
  that we have had to borrow freely the necessary apparatus from $\EGA$,
  II. 
 
  \item The regularity of $\mathscr{O}_{X,x}$ can be proved as follows.
    By applying the fundamental theorem on proper morphisms ($\EGA$, III,
    (4.1.5)), one shows first that $\sigma_*( \mathscr{O}_{X'})_x$ is a
    regular local ring of dimension $2$. Since
    $\dfrac{\mathfrak{M}_x}{\mathfrak{M}^2_x} \overset{\psi}\rightarrow
    \Gamma (L, \mathscr{I}/ _{\mathscr{I}^2})$ is surjective, we see that
    the image  of $\mathfrak{M}_x$ in $\sigma_* (\mathscr{O}_{X'})_x$
    generates the maximal ideal of this local ring. By Nakayama's lemma,
    $\mathscr{O}_x \to \sigma _*(\mathscr{O}_{X'})_x$ is surjective. But we
    have also shown\pageoriginale that it is injective, and this
    concludes the proof.  
    
  \item The full strength of our assumption $(ii)$, that  $\mathscr{I}/
    _{\mathscr{I}^2}$ is the line bundle of degree $1$ on $\mathbb{P}'
    (K)$ was used only in the proof of the regularity of
    $\mathscr{O}_{X,x'}$. Suppose  we only assume that  $\mathscr{I}/
    _{\mathscr{I}^2}$ is of  {\textit{positive}} degree $l$. If we assume
    that $R' \pi'_*( \mathscr{O}_{X'}(l))=0$ in a neighbourhood of the point
    $b$ and if we work with the inversible sheaf $\mathscr{L}=
    \mathscr{O}_{X'}(l) \otimes \mathscr{I}^{\otimes(-k)}$ where $k$ is
    the degree of $\mathscr{O}_{X'}(1)\otimes_{\mathscr{O}_{X'}}
    \mathscr{O}_L$ on $L$, we get exactly as above a B-projective scheme
    $X$ and a projective  morphism $\sigma : X' \to X$ such that $\sigma
    (L)$ is a point $x$ of $X$ and the restriction of $\sigma $ to $X'
    -L$ is an isomorphism onto $X- \{x\}$. We can only say that $x$ is a
    {\textit{ normal }} point of $X$. 
    
  \item Grauert (\cite{key5}) has proves that if $X'$
    is a complex manifold of dimension  2 and  $C_1, \cdots ,C_n$ a
    system of compact, irreducible, one dimension analytic sets on $X'$,
    such that the intersection matrix $(C_i. C_j)$ is negative definite,
    and $\bigcup \limits_{i=1}^n C_i$ is connected, there is a normal
    complex space $X$ and a proper holomorphic  map  $\sigma : X' \to X$
    such that $\sigma( \bigcup \limits_1^n C_i)$ is a point $x$ and $X$
    and $\sigma$ restricted to $X'- \bigcup \limits_1^n C_i$ is an
    isomorphism onto $X- \{x\}$. Thus, for the contractibility of a
    system of curves on a complex 2-manifold, the condition of Mumford on
    the negative definiteness of the intersection matrix is necessary and
    sufficient. The image point $x$ of the set  $\bigcup \limits_1^n C_i$
    will in general be a singular point. 
    
    However,\pageoriginale ever if we assume that $X'$ is a non-singular
    projective surface and $C$ a non-singular curve on $X'$ with
    $(C^2)<0$, if the genus of $C$ is $\ge 1$, the analytic space $X$
    need not be algebraic. It is thus that Grauert constructs an example
    of a normal compact complex space of dimension $2$ with an isolated
    singular point, which is not an algebraic surface but whose field of
    meromorphic function has transcendence degree $2$ over
    $\mathbb{C}$. We shall give another due to Hironaka which is perhaps
    simpler. 
\end{enumerate}
\end{remark*}

 Let $C$ be a non-singular cubic curve on $\mathbb{P}^2$, and  $P_i(1
 \le i \le 10)$ a set of $10$ distinct points on $C$. Let $X'$ be the
 non-singular surface obtained by blowing  up the $P_i$, and let $C'$
 be the proper transform of $C$ on $X'$. Since $(C^2)=9$, we deduce by
 (12) that $(C'^{2})=9-10=-1$ (blowing up a simple point of a
 curve  diminishes self intersection by $l$ as follows by
 (12). Hence, by Grauert's theorem, there is a proper holomorphic
 map $\sigma $ of $X'$ onto a normal complex space $X$ such that
 $\sigma (C')$ is a point of $X$ and $\sigma$ is an isomorphism of $X'
 - C'$ onto  $X- \{x\}$. If $X$ were an algebraic variety, we can find
 a curve $D$ on $X$ not passing through $x$. The inverse image
 $\sigma^{-1}(D)$ would then be disjoint with $C'$.  This means that
 the projection of  $\sigma^{-1}(D)$ on $\mathbb{P}^2$ can intersect
 $C$ at most at the points  $P_i$. Let the intersection multiplicity
 of the projection with be $C$ at $P_i$ be $n_i$. Then not all $n_i$
 are zero, and if we regard  $C$ as an abelian group in the usual way
 with an inflexion point for $0$, we must have the relation  
 $$
 \sum_1^{10} n_i P_i =0.
 $$   

 If the\pageoriginale $P_i$ were chosen `in general position', such a
 relation cannot hold, and  thus $X$ is not an algebraic surface.  

It may be mentioned in this connection that if $X$ is a compact
complex manifold such that there are two algebraically independent
meromorphic functions on $X$ over $\mathbb{C}$, then $X$ is projective
algebraic (Theorem of Chow-Kodaira). 

We now deduce some corollaries from the theorem.

\setcounter{corollary}{0}
\begin{corollary}%\coro 1
  Let $X'$ be a locally noetherian two dimensional regular prescheme,
  $L$ a closed subscheme and $U$ an open neighbourhood of $L$, such
  that  
  \begin{enumerate}[\rm (i)]
  \item $U$ is quasi-projective over a noetherian ring $A$.

  \item there is a closed point $b$ of $\Spec A$ such that $L$ is
    contained in the fibre over $b$, and is isomorphic as
    $k(b)$-scheme to $\mathbb{P}'(K)$, where $K$ is an extension of
    $k(b)$. 

  \item $(L^2)=- [ K : k(b)]$.
  \end{enumerate}
\end{corollary}  
   
Then there is a locally noetherian two dimensional regular pre\-sch\-eme
$X$, a closed point $x$ of $X$ and a morphism $\sigma :X' \rightarrow
X$ such that  $X'$ is $X$-isomorphic to the dilatation of $X$ at $x$.  
\begin{proof}
  This results immediately from the theorem, in view of the theorem of
  decomposition of lecture \ref{chap4}, since  a quasi-projective scheme over a
  quasi-compact base admits an open immersion in a projective scheme
  over same base ($\EGA$, II, (5.3.3)), and since the `contractibility'
  of $L$ depends clearly only on an open neighbourhood of $L$.    
\end{proof}
 
\begin{corollary}[Zariski]%coro 2
  Let\pageoriginale $X$ be an algebraic scheme over a field $K$ with isolated
  singular ($i.e.$, non-singular) points. Then $X$ is quasi-projective
  over $K$ if and only if the set of singular points of $X$ is
  contained in an affine open subset of $X$. 
\end{corollary}

\begin{proof}
  Since a finite set of points of a quasi -projective $K$-scheme is
  contained in an affine open  subset, the condition is clearly
  necessary.  
\end{proof}
 
 Suppose conversely that the condition is fulfilled. Since the set of
 singular points of $X_{red}$ is contained  in the singular set of
 $X$, and since $X$ is quasi-projective over $K$ if $X_{red}$ is so
 ($\EGA$, II, $(4.5.14)$), we may assume that $X$ is reduced. Again, by
 ($\EGA$, II, $(5.3.6)$), we may assume $X$ connected. By assumption,
 there is a projective  $K$-scheme $Y$, a dense open subset $U$ of $Y$,
 a non-void open neighbourhood $V$ of the singular set $S$ of $X$, and
 an isomorphism $\varphi :U \rightarrow V$. By normalising $Y'$
 outside $\varphi^{-1}(S)'$, we may assume that $Y- \varphi^{-1}(S)$
 is normal ($\EGA$, II, (6.1.11)). In particular the singular set of
 $Y$ is again finite. By the theorem of resolution of singularities
 of a surface (\cite{key1}), we may actually
 assume that $Y-\varphi^{-1}(S)$ is regular. Since $U$ is dense in $Y,
 \varphi$ defines a rational map $\psi$ of $Y- \varphi^{-1}(S)$ into
 $X$. This rational map has no indeterminacies in $U$, since $\varphi$
 is defined on $U$. By the theorem of elimination of indeterminacies
 of Lecture \ref{chap4}, we may further assume that $\psi$ has no
 indeterminacies on $Y-\varphi^{-1}(S)$. (Blowing up is a projective
 morphism. One needs to blow up points outside of $U$, so that we
 still have an open subset of the blown up variety isomorphic to
 $V$). But this means that there is an open subset\pageoriginale $U'$
 of $Y$ and a 
 morphism $\chi :U' \rightarrow X$ such that (i) $\chi$ is proper,
 (ii) there is an open subset $U$ of $U'$ and an isomorphism $\chi
 |U:U \rightarrow V$ onto an open neighbourhood $V$ of $S$, and
 (iii) $U' - (\chi^{-1}(S) \cap U)$ is regular. Since $X$ is connected,
 any  irreducible component intersects $S$ if $S \neq \phi$, and there
 is only one irreducible component if $S =\phi$, so that in any case
 $V$ is dense in $X$. Hence $\chi$ is also surjective. Let $\xi : V
 \rightarrow U$ be the inverse of $\chi / U$. The  morphism $V \to U'
\times_K V$ defined by $x \mapsto (\xi (x), x)$ is a closed immersion,
 being the graph of the composite morphism $V \xrightarrow {\xi} U
 \hookrightarrow U'$. Let $\Gamma'$ be the closed subscheme of $U'
 \times_{K}V$ which is the image.  By definition of $\chi$, the graph of
 $\chi$ in $U' \underset{K}X_k X$ is the closure of $\Gamma'$ in $U' X_k
 X$. Since  $\Gamma$ is already closed in $U' X_K V$, it follows that
 $\chi^{-1}(V)=U$, and hence $\chi^{-1}(S) \subset U$. Thus, $\chi| U'
 -\chi^{-1}(S)$ is a proper surjective morphism of $U'- \chi^{-1}(S)$
 onto $X-S$. By the theorem of decomposition of Lecture \ref{chap4}, it
 follows that $\chi$ is a composite of dilatations at regular points
 of $X$ and points lying over regular points of $X$. But now, by the
 theorem of  Castelnuovo, if  $\sigma : X' \rightarrow X$ is a
 dilatation at a regular point of $X$ and $X'$ is quasi-projective,
 $X$ is also  quasi -projective. This proves the corollary. 
 
The first example of a variety which is not quasi -projective is due
to Nagata, who exhibits a complete, normal surface with exactly two
singular points which is not projective. We give a construction. The
first remark is the following. If $C$ is an elliptic curve (over an
alg. closed field $K$), $x$ a point of $C, X'$ the variety obtained
from $X= \mathbb{P}' \times C$\pageoriginale by blowing up $(0,x)$ and
$C'$ the 
proper transform of $\{0\} \times C$ in $X', C'$ can be contracted to
a point  $y$ of a normal  projective variety $Y$, in the sense that
there is a morphism $\tau : X' \to Y$, such that $\tau^{-1}(y)=C'$ and
$\tau | X'- C'$ is an isomorphism onto $Y- \{y\}$. To prove this,
observe first that  for $n$ large; $(n,x)$ is very ample for $C$, so
that $(0)\times C+n (\mathbb{P}' \times (x))=D$ is very  ample for
$X$. (Segre imbedding). By what we have said in lecture \ref{chap3}, for $m$
sufficiently large, the divisors of the complete linear system. $| mD
|$ which contain the point $(0,x)$ give a projective imbedding of
$X'$, whose  hyperplane sections are of the form $\sigma^* (D_1)-L$,
where  $\sigma : X' \rightarrow X$ is the canonical morphism, $L=
\sigma^{-1}((0,x))$ and $D_1 \in | m D|, x \in Supp D_1$. Thus, if $H$
is a hyperplane section of $X'$, we have the  \textit{linear} (and not
merely numerical) equivalence $H.C' \sim (mn-1)x'$ where $x'$ is the
point of $C'$ lying over $x$. Replacing $H$ by a sufficiently high
multiple, we may assume that $H.C'=kx',k >0$ and $H'(X',
\mathscr{O}_{X'}(1))=0$. As in the proof of Castelnuovo theorem, we
shall show that the complete  linear  system $| H + k C'|$ has no base
points, and  gives a morphism $\varphi $ of $X'$ into a projective
space which  is an immersion  restricted to $X'-C'$ and which
contracts $C'$ to a point of the image. Now, we have the linear
equivalence $C'^{2}=-x'$ on $C'$. Hence, if $\mathscr{L}$ is the
inversible sheaf $\mathscr{O}_{X'}(1) \otimes \mathscr{I}^{\otimes
  -(k)}$, where $\mathscr{I}$ is the defining  sheaf of ideals of
$C'$, $\mathscr{L} \otimes_{\mathscr{O}_{X'}} \mathscr{O}_{C'}$ is
trivial on $C'$. This means that any section  of $\mathscr{L}
\otimes_{\mathscr{O}_{X'}} \mathscr{O}_{C'}$ on $C'$ is either
everywhere $0$ on $C'$ or does not vanish\pageoriginale anywhere on
$C'$. If we show that  
$H^o (X', \mathscr{L}) \rightarrow H^o (C',
\mathscr{L}\otimes_{\mathscr{O}_{X'}} \mathscr{O}_{C'})$ is
surjective, it would follow that  $\varphi$ is defined on $C'$ and
contracts $C'$ to a point. Now, for any line bundle $\mathfrak{K}$ of
\textit{positive}  degree on an elliptic curve $C'$, we have $H' (C',
\mathfrak{K})=0(\Omega' \simeq'  \mathscr{O}_{C'}, H' (C' ,
\mathfrak{K}) \simeq H^o(C',\mathfrak{K}^*)=0$ by Serre duality). From
the exact sequences $H'(X', \mathscr{O_{X'}}(1)\otimes \mathscr{I}^{-
  \nu}) \to H^o (X', \mathscr{O}_{X'}(1) \otimes \mathscr{I}^{- \mu
  -1}) \to H' (C', \mathscr{O}_{X'}(1) \otimes \dfrac{\mathscr{I}-
  \mu-1}{\mathscr{I}-\mu})$ we deduce by induction  on $\mu$ that
$H'(C', \mathscr{L}\otimes \mathscr{O}_{C'}=0$, so that  $H^o (X',
\mathscr{L}) \rightarrow H^o (C', \mathscr{L} \otimes
\mathscr{O}_{C'})$ is surjective. Now, $\varphi$ is clearly  defined
and is an immersion  on $X'-C'$ (See proof of theorem). By
normalising  the image  variety if necessary, we obtain  $\tau: X'
\rightarrow Y$ having the required properties. 
 
Now, let $x,y$ be two points of $C$ such that $nx$ and $ny$ are not
linearly  equivalent for any $n \in  \mathbb{Z}$, $n \neq 0$. Let $X''
\xrightarrow {\sigma}X$ be the morphism obtained  by blowing  upto
$(0,x)$ and $(\infty, y)$, and let $C', C''$ be the proper transforms
of $\{0\} \times C$ and $\{\infty \}\times C$ on $X''$.
 
Since the contractibility of a curve to a normal point of an
\textit{abstract} variety depends only on an open neighbourhood of the
curve, we see from above  that there is an abstract normal  variety
$Y''$ and a proper  morphism $\xi : X'' \rightarrow Y''$, such that
$\xi (C')= y'$ and $\xi (C'')=y''$ are  points of $Y''$,  and $\xi$
restricted to  $X''-C'-C''$ is an isomorphism onto $Y''- \bigg\{y',
y'' \bigg\}$. We assert that $Y''$ is not projective.  

For,\pageoriginale if it were, there would be a curve $D$ on $Y''$
which intersects x
the curves $\xiup(\sigma^{-1} (0,x))$ and $\xiup(\sigma^{-1}
(\infty,y))$ but such that neither of the points $y',y''$ belongs to
$D$. (One can take for $D$ a hyperplane section not containing $y' $
or $y''$). But then, $\xiup ^{-1} (D)$ is a  curve on $X''$
intersecting both $\sigma ^{-1} (0,x)$ and $\sigma ^{-1} (\infty,y)$,
but not intersecting $C' $ or  $C''$. Hence the curve $\sigma
(\xiup^{-1} (D)) = D'$ on $X$ intersects $0 \times C$ only at the
point $(0,x)$ and $\infty \times C$ only at the point $(\infty,y)$. It
follows that if $D'$. $(0 \times C) = m(0,x)$ and  $D' (\infty \times C)
= n (\infty, y), m,n > 0$  the divisors $mx$ and $ny$ are linearly
equivalent on $C$. This is impossible by assumption. Thus, $Y''$ is
not projective.  
