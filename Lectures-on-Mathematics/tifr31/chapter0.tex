\setcounter{chapter}{-1}
\chapter{}\label{chap0}

In 1900,\pageoriginale at the international Congress of
Mathematicians in Paris,\break 
Hilbert posed twenty-three problems. His complete address was
published in Archiv. f. Math. U. Phys. (3), 1, (1901) 44-63, 213-237 (one can
also find it in  Hilbert's Gesammelte Werke). 

The fourteenth problem may be formulated as follows:
\noindent
\textit{The Fourteenth Problems. Let $K$ be a field
and  $x_{1}, \ldots , x_{n}$ algebraically independents
elements over $K$. Let  $L$ be  a subfield of 
$K(x_1, \ldots , x_n)$ containing $K$. Is the ring
$K[x_1,\ldots,x_n ]\bigcap L$ finitely generated
over $K$?} 

The motivation for this problem  is the following special case,
connected  with theory of invariants. 

\noindent
\textit{The  Original fourteenth  problem. Let
$K$ be a field and $G$ a subgroup of the  full linear
group  $GL(n,K)$. Then $G$ acts as a group of
automorphisms of $K[x_1,\ldots,x_n]$. Let  $I_G$ be
the  ring  of elements of $K[x_1, \ldots,x_n]$. invariant
under $G$. Is $I_G$ finitely generated over $K$?} 

Contributions  to the original fourteenth  problem were made  in
particular cases. In fact it was  proved that $I_G$ is finitely
generated in the following cases. 
\begin{enumerate}
\item $K$ is the complex number  field  and $G=SL(n,K)$ acting  by
means  of its tensor representations. (D. Hilbert Math. Ann. 36(1890)
473 - 534). 

\item $K$ is the complex number  field and $G$  satisfies the
following condition: there  exists  a conjugate $G^{\ast}$ of $G$ such
that  $A \in G^{\ast}  \Rightarrow {}^t\bar{A}\in G^\ast$, where
`$-$' and `$t$' indicate complex conjugation and transpose respectively
(E. Fischer\pageoriginale Crelle Journal 140(1911) 48 - 81). 

\item $K$ is an arbitrary field and $G$ a finite  group ($E$. Noether,
for characteristic $K=0$, Math. Ann. 77 (1916) 89 - 92, for
characteristic $K \neq 0$ G\"ottinger Nachr. (1926 28-35).  

\item $K$ is the complex number  field and $G$,  a one parameter group
(R. Weitzerb\"ock Acta Math. 58 (1932) 231 - 293). 

\item $K$ is  the  complex number  field  and $G$,  a connected
semi-simple  Lie  group (H. Weyl  Classical groups (1936) Princeton
Univ. Press). 
\end{enumerate}

The  next significant contribution was made  by 0.Zariski in 1953. He
generalized (Bull. Sci., Math.78(1954) 155-168) the fourteenth problem
in the following way. 

\textit{Problem of Zariski. Let  $K$ be  a field
and $K[a_1, \ldots , a_n]$ an affine  normal  domain
(i.e.a finitely generated integrally closed domain over
$K$). Let  $L$ be  a subfield  of $K(a_1 , \ldots
,a_n)$ containing $K$. Is the  ring  
$K[a_1, \ldots , a_n] \bigcap L$ finitely  generated over
$K$?}

He answered the question in the  affirmative  when  trans. $\deg_K L\leq
2$. Later, in  1957, $D$. Rees (Illinois J. of Math. 2(1958) 145 - 149)
gave  a counter  example  to the  problem  of Zariski when
trans. $\deg_K L=3$. 

Finally, in 1958, Nagata (Proceedings of the  International  Congress
of Mathematicians, Edinburgh (1958) 459 - 462) gave a counter example  to
the original  fourteenth problem  itself. This  counter example was in
the case  when trans. $\deg_K L=13$. Then, in  1959,  Nagata
(Amer. J. Math. 91, 3(1959) 766 - 772) gave  another counter example  in
the   case when  trans. $\deg_K L=4$.  

The  groups\pageoriginale  occurring  in these examples  are
commutative. So in view 
of Weyl's  result  we seek the answer to the  original  fourteenth
problem  in the   case  when $G$ is a  non- comutative,
non-semi-simple Lie group. The  examples  mentioned  above  May be
made  to yield  one  with  $G$ non-commutative  by considering  what
is essentially  the direct  product by a non-commutative group. More
interesting is the  case  when $G$ is  a connected Lie group  such
that $[G,G]=G$. Even  in this  case the  answer is in the  negative
as we shall  see later  in this course  of lectures.

