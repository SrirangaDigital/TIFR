\chapter{A generalisation of the original fourteenth problem} %%% chap 1 


{\bf 1.}~ We first\pageoriginale generalise the original fourteenth problem
in  the following  way: \textit{Generalised fourteenth  problem. Let  $K$ be
a field. Let $R=K[a_1, \ldots , a_n]$ be a finitely generated
  ring over $K$ ($R$ need  not be an integral domain). 
Let  $G$ be a group  of automorphism of $R$ over
$K$. Assume that  for every $f \in R$, $\sum\limits_{g \in G}
f^gK$ is a finite  dimensional vector  space  over
$K$. Let  $I_G$ be the  ring  of invariants of $R$
under $G$. Is  $I_G$ finitely  generated
  over $K$?} 

\begin{rem}%%% 1
 When $a_1, \ldots , a_n$ are  algebraically independent  and $G$  is
 a subgroup of $GL(n,k)$ acting  on $R$  in the `usual way',
 elements  of $R$ satisfy the  finiteness condition  we have  imposed
 in the  problem. 
\end{rem}

\begin{rem}%% 2
 The omission of the  assumption that $a_1, \ldots , a_n$ are
 algebraically independent  is helpful. For  instance  let $N$ be
 normal  subgroup  of $G$  such that $I_N$  is finitely
 generated. Then  $G$ acts  on $I_N$, as  $N$ is  normal  in
 $G$. Hence  $G/N$ acts on $I_N$ and $(I_N)_{G/N}=I_G$. So, for
 instance if we know that our generalized problem is true  for (i)
 finite group (ii) connected semi-simple Lie groups, (iii)
 diagonalizable  groups, then  we can  get  immediately  that: if $G$
 is an  algebraic  group  whose radical $N$ is  diagonalizable then
 the answer to generalized problem  is in the affirmative. Note  that
 $I_N$ need not be polynomial ring  even if $a_1, \ldots , a_n$ are
 algebraically independent.   
\end{rem}

\begin{rem}%% 3
We do\pageoriginale not assume  $R$ is  an integral domain, as  the
constant  field 
extension (of \S\ 3 below) of a finitely generated integral domain
over a field  $K$ need  not be  an integral  domain. 
\end{rem}

{\bf 2.}~ {\textit{Algebraic linear groups.}}
 In this  section  by an affine
  variety we mean the set of rational  points over $K$ of a certain
  affine variety with points in an universal domain. We shall give
  Zariski topology to an algebraic variety. The  group $GL(n,k)$,
  being  the complement of the hypersurface  defined by
  $\det(x_{ij})=0$ in $K^{n^2}$, is an affine variety (in fact it is
  isomorphic to the hypersurface $z \det(x_{ij})=1$ in
  $K^{n^2+1}$). The group  operation  of $GL(n,K)$ are  regular
  i.e. $GL(n,K)$ is an algebraic group. In this  course  of lectures
  by an algebraic  group we always mean  closed  subgroups of
  $GL(n,k)$. We define connectedness and irreducibility under Zariski
  topology. In the case of algebraic groups  these two concepts
  coincide. If $G_o$ is the  connected  component of an  algebraic
  group $G$, then $G_o$ is a normal subgroup of finite index in $G$. An
  algebraic group contains the largest connected  normal  solvable
  subgroup of $G$, which we call the  \textit{radical} of $G$. For
  details  about algebraic  groups  we refer to seminaire
  $C$. Chevalley Vol.1 (1956-1958) and $A$. Borol Groups Lin\'eaires
  Alg\'ebriques, Annals of Mathematics, 64 (1956). 
We remark that in the generalized problem we may assume that  $G$ is
a subgroup of $GL(m,K)$ for some $m$. Because of the finiteness
condition on the elements of $R=K[a_1 , \ldots , a_n],
V=\sum\limits^{n}_{i=1}\sum\limits_{g \in G} a_i^gK$ is
finite dimensional vector space. We  choose  a basis  $b_1, \ldots ,
b_m$ for  $V$. Then  $R=K[b_1, \ldots , b_m]$ and  $G$ is a group of
automorphisms  of  $V$. Therefore in the\pageoriginale generalised
problem  we may 
assume $G\subseteq GL(n,k)$. From  now  on, whenever we say
that  a subgroup  $G$ of  $GL(n,K)$ acts on $R=K [a_1, \ldots, a_n]$
we mean that  it acts linearly on the vector space
$\sum\limits^{n}_{i=1} Ka_i$. The following theorem shows that in the
problem we may assume that  $G$ is an algebraic group 

\begin{thm}%thm 1
Let  $R=K[a_1, \ldots , a_n]$ and let $G \subseteq  G(n,k)$ act as a group
of automorphisms of $R$ over $K$. Let  $G^*$ be the closure  of $G$ in
$GL(n,K)$. Then  $G^*$ acts  as a group  of automorphisms of $R$ over
$K$ and  $I_G=I_{G^*}$. 
\end{thm}

This theorem is a consequence of the following  lemma (put $S=I_G$). 

\begin{lemma*}
Let  $S \subseteq R$ and let  $H$ be  the set of elements  of
$GL(n,k)$ which  induce  automorphisms of $R$ over $K$ and leave
every  $s \in S$ invariant. Then $H$ is  algebraic. 
\end{lemma*}

\begin{proof}
 Let $H_1$ denote  the set of elements  of $GL (n,K)$ which  induce
 automorphisms of $R$ over $K$. Let $\mathcal{U}$ be the  ideal of the
 ring  of  polynomials $K[x_1, \ldots , x_n]$ such that $K[a_1, \ldots
   , a_n] \thickapprox  K [x_1, \ldots , x_n]/ \mathcal{U}$. Now $g
 \in  H_1$ if and only  if $\mathcal{U}^g= \mathcal{U}$. Let $f_1,
 \ldots , f_m$ generate $\mathcal{U}$. We may assume that $f_1, \ldots
 , f_m $ are linearly independent  over $K$. Extend $f_1, \ldots ,
 f_m$  to a linearly independent basis $f_1, \ldots , f_m, f_{m+1}, \ldots ,
 f_l$ of  the  vector  space 
$$
V=  \sum\limits_{i=1}^{n}  \sum\limits_{g \in GL(n,k)}  f_i^g
K. \quad \text{Let} f_i^t =  \sum\limits_{j=1}^{l} \lambda_{ij}f_j, i=1,
\ldots ,m, 
$$
where $t=(t_{rs}),t_{rs}$ are  indeterminates. Then  $\lambda_{ij}$
are polynomials in $t_{rs}$. The  condition
$\mathcal{U}^g=\mathcal{U}$ is equivalent to  
$$
\lambda_{ij}(g)=0,i=1, \ldots , m, j=m+1, \ldots, l.
$$\pageoriginale

Hence $H_1$ is algebraic.


Let  now $H$ $=\bigg\{g \in H_1 \mid s^g=s$, $s \in S \bigg\}$.

It is enough  to prove  the  lemma  when  $S$ consists  of a single
element $s$. Let  $s,s_1, \ldots , s_k$ be a linearly independent
basis  of the  vector  space  $\sum\limits_{h \in H_1} s^h  K$. Let
$y=(y_{rs})$ be a generic  points  or may  component  of  $H_1$ over
$K$. We have  $s^y= \lambda_0s + \sum\limits_{i=1}^{k}  \lambda_i
s_i$, where $\lambda_i$ are  polynomials in $y_{rs}$. The  condition
$s^y=s$ is  equivalent  to $\lambda_\circ(s)=1$, $\lambda_i(g)=0$ for
$i > 0$. Hence  $H$ is algebraic. 
\end{proof}

{\bf 3.}~ \textit{Constant field  extension.}
 We shall  now  study  the
  behaviour  of  invariants under  the  constant  field
  extensions. Let  $R'$ and  $R''$ be two commutative rings containing
  a field  $K$. Let  $G'$ (respectively $G''$) act on $R'$
  (respectively $R''$) as a  group  of automorphisms over $K$. Then
  the direct  product $G'\times G''$ acts on $R \otimes_K R''$ as a
  group  of automorphisms; in  fact  we have  only  to define $( a'
  \otimes a'') (g',g'')=a'^{g'}\otimes a''^{g''}$, $a' \in R'$, $a'' \in
  R''$, $(g',g'')\in G' \times G''$. 

\begin{lem}\label{chap1:lem1}%%% 1
 Let $I_{G'\times G''}$ denote  the  ring of elements of  $ R'{\otimes}_K
 R''$  invariant under $G' \times G''$ Let  $ I_G'$ (respectively
 $I''_{G''}$) denote  the  ring  of elements  of $R'$ (respectively
 $R''$) invariant under  $G'$ (respectively $G''$). Then  $I_{G'\times
 G''}=I_{G'} \otimes_K I''_{G''}$. 
\end{lem}

\begin{proof}
We have only to prove  that  $I_{G'\times G''} \subseteq  I_{G'}'
\bigotimes\limits_K I''_{G''}$. Choose a linearly independent  basis 
$(i'_{\lambda'})_{\lambda ' \in  \Lambda'}$ and
$(i''_{\lambda''})_{\lambda''\in \Lambda''}$ for  the vector  spaces
$I'_{G'}$ and    $I''_{g''}$ respectively over  $K$. Extend  these
bases to a\pageoriginale linearly independent bases of $R'$ and $R''$
respectively, say  
$$ 
R' = \sum_{\mu '\in M'} i'_{\mu'} K,R'' = \sum_{\mu''\in 
  N''} i''_{\mu''} K \text{ with } \Lambda' \subseteq  M' \text{ and }
\Lambda'' \subseteq  M''. 
$$
Let $f= \sum _{(\mu', \mu'')}  \underset{\in M' \times
  M''}{k_{\mu' \mu''}}   (i_\mu \otimes i_{\mu''}) \in  I_{G'\times G''}
, k_{\mu' \mu''} \in   K$.

For every $g \in G$, we  have
$$
f^{g'} = \sum_{(\mu' , \mu'')} k_{\mu' \mu''} (i^{g'}_\mu \otimes
i''_{\mu '})= \sum\limits_{\mu''} (\sum\limits_{\mu'} k_{\mu '\mu''}
i_\mu^{g'}) \otimes i''_{\mu''} =f. 
$$

Hence  
$$
\sum\limits_{\mu '}k_{\mu ' \mu''} i'^{ g'}_{\mu '} =
\sum\limits_{\mu '} k_{\mu ' \mu''} i'_{\mu'}, g' \in G', \mu'' \in
M''. 
$$

Hence  $k_{\mu'\mu''}=0$, for $=\mu'\notin \Lambda'$. Similarly 
$k_{\mu' \mu''}=0$, for ${\mu'' \notin \Lambda''}$. i.e. $f\in I'_{G'}
\otimes_K L''_{G''}$ and  the  lemma is proved.  
\end{proof}

\begin{lem}\label{chap1:lem2}%%% 2
With  the above notation, $R' \otimes_K R''$ is finitely generated
over $K$  if and   only  if $R'$  and  $R''$  are  finitely
generated over $K$. 
\end{lem}

\begin{proof}
It is  clear  that  if  $R'$ and $R''$ are finitely
  generated  over  $K$,  so is  $R' \otimes_K R$. Now  let  $R'
  \otimes_K R$. be finitely  generated over $K$,  say $f_i=
  \sum\limits_{j} \gamma'_{ij} \otimes \gamma''_{ij}$, $i=1, \ldots ,l$
  are the  generators. Then  $R' =K\bigg[\gamma_{ij}'\bigg]$ and $R'
  =K \bigg[\gamma''_{ij}\bigg]$. 
\end{proof}


Let  $R=K[a_1, \ldots ,a_n]$ and let $G$ act  on $R$ as as group of
automorphisms of $R$ over $K$. Let $K'$ be  a filed containing
$K$. Then $G$ acts on $K'\otimes_{K} R$ which we denote by $K'[a_1,
  \ldots , a_n]$. Let  $I'_{G'}$ be  the invariant elements  of $K'
\otimes_K R$ under $G$. Then  in Lemma \ref{chap1:lem1} if we put $R'=K'$,
$R''=R,G'=\{ 1 \}$, $G'' =G$, we get $I'_G =K' \otimes_K I_G$. As
in\pageoriginale Lemma \ref{chap1:lem2} it follows that:  

\begin{proposition}\label{chap1:prop1}%%% 1
 $I'_G$ is finitely generated over $K'$ if and  only  if $I_G$ is
  finitely generated  over $K$. 
\end{proposition}

The above proposition helps us to confine ourselves to a smaller
field  when  $K$ is `too big'. Let $G'$ be a subgroup of $G$ such
that  $G'$  is dense in the  closure  $\bar{G}$ of $G$. For
instance  when  $K$ is  of characteristic zero  or $\bar{G}$ is a
torus  group  we can  take $G'$ to be  finitely generated. When  $K$ 
is  of characteristic  $p \neq 0$ we may  take $G'$ to be  countably
generated. Let  $K [x_1, \ldots , x_n] /  \mathcal{U}\thickapprox
K [a_1, \ldots , a_n]$ with the $x_1, \ldots , x_n$ algebraically
independent over $K$. Let  $K'$ be a subfield of $K$ such that
elements  of $G'$ are $K'$-rational and  $\mathcal{U}$ is defined
over $K'$ For instance  if we can choose  $G'$  finitely  generated,
$K'$ can be chosen to be finitely generated  over  the  prime
field. If  $G'$ can be  chosen  countably  generated, then  $K'$ can
be  choosen to be countably generated  over  the  prime  field. Now
as the  ideal $\mathcal{U}$ is  defined  over  $K', K \otimes_{K'}$, $K'
[a_1, \ldots , a_n]=K[a_1 , \ldots ,a_n]$. Hence by proposition
\ref{chap1:prop1}, 
$I_G$ is  finitely generated over  $K$ is and only  if $G'-$
invariants in $K' [a_1, \ldots, a_n]$ are  finitely  generated. 


Of course, as we noted  in \S\ 2, we can  enlarge  $G'$  to an
algebraic group. 

\medskip
\noindent\textit{3. Invariants of a finite group.}
 We shall  now consider  the  original 14th problems (in fact  the
 generalised problem) when   $G$ is  finite.  

\begin{thm}[E. Noether]%%% 2
 Let  $R=K [a_1, \ldots,a_n]$ and $G$ a finite group
  acting  on $R$ as a group  of automorphisms  of $R$  over  $K$. Then
  $I_G$\pageoriginale is finitely generated over $K$. 
 \end{thm}

\begin{proof}
Let $G= \left\{1=g_1, \ldots ,g_h \right\}$. Let  $a \in R$. Set 
$$
S_1=\sum_{i=1}^{h} a^{g_i}, S_2 = \sum_{i<j} a^{g_i}a ^{g_j}, \ldots ,
S_h = a^{g_1}\ldots a^{g_n}. 
$$
Then  $S_i \in I_G$, $i=1, \ldots, h$ and $a^h-S_1 a^{h-1}+ \cdots
+ (-1)^h S_h=0$. 
\end{proof}

Hence $R$ is integral over $I_G$. Now  the  theorem   follows  from
the  following  Lemma: 

\begin{lemma*}
  Let $R=K[a_1, \ldots , a_n]$ and let  $S$ be  subring of $R$
  containing  $K$ such  that  $R$  is integral  over $S$. Then $S$ is
  finitely generated over $K$. 
\end{lemma*}

\begin{proof}
 There exist  $G_{ij} \in  S$, $1 \leq  i \leq  n$,  $0 \leq  j \leq
 m_i  -1$ such that 
$$
a_i^{m_1} + C_{i m_i - 1} a_i^{m_i-1}+ \ldots + C_{i o} =0.
$$
Set  $S' =K\bigg[C_{ij}\bigg]_{\substack{1 \leq i \leq n \\ 0 \le j \le
    n_i - 1}}$. Then $R$ is finite $S'$ module. As $S'$ is noetherian,
$s$ is also  finite  $S'$-module. Hence  $S$  is finitely generated over
$K$. 
\end{proof}

\begin{coro*}
 Let $G_0$ be a normal subgroup  of $G$ ($G$ not  necessarily finite)
 of finite  index. Let  $G$ act  on $R$ as a group  of automorphisms
 over $K$.  If  $I_{G_0}$ is  finitely generated over $K$, then, so
 is $I_G$. 
\end{coro*}

\begin{proof}
 The group $G/ G_0$ acts on $I_{G_0}$  and  $(I_{G_0})_{G/ G_0} = I_G$ 
\end{proof}

\begin{remark*}
 Suppose that $R$ (of the above corollary) is an integral domain  Then
 the  converse of the above  Corollary  is also  true. Let  $K_0 L$ be
 the\pageoriginale quotient of $I_{G_0}$ and $I_G$ respectively. Then
 $K_0$ is a finite separable algebraic extension of $L$. Let  $I_G$ be
 finitely generated. As $G/ G_0$ is finite, $I_{G_0}$ is integral over  
 $I_G$. Hence  $I_{G_0}$ is a finite $I_G$-module  and therefore
 finitely generated over $K$. 
\end{remark*}

