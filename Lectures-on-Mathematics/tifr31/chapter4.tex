\chapter{Theorem of Weitzenb\"ock}%%% chapter 4

\setcounter{thm}{0}
\begin{thm}[Weitzenb\"ock]\label{chap4:thm1}%%% 1
    Let\pageoriginale $K$ be the complex number field and $G$ a 
    complex one parameter Lie subgroup of $GL(n,K)$ acting on the ring of
    polynomials $K[x_1,...,x_n]$. Then the ring of invariants $I_G$ of
    $K[x_1,...,x_n]$ is finitely generated over $K$. (see 
        also C.S. Seshadri; On a theorem of Weitzenb\"ock in invariant
        Theory, J. Math. Kyoto Univ. 1-3 (1962), 403-409). 
  \end{thm} 
 
 Let $G=\{e^{tA}|t \in K\}$ be a one parameter Lie group, $A$ being a
 constant matrix. Let $A=N+S$, where$ N$ and $S$ are nilpotent and
 semi-simple matrices respectively with $NS=SN$. Let
 $G_1=\{e^{tN}|t \in K\}$ and $G_2=\{e^{tS}|t \in K\}$. If $N \neq
 O$, then the mapping $t \to e^{tN}$ of the additive group $K$ onto
 $G_1$ is an algebraic isomorphism, since $N$ is nilpotent. Since
 every element of $G_2$ is semi-simple, the closure
 $\overset{-}{G_2}$ of $G$ is a torus group. Further since the
 elements of $G_2$ and $\overset{-}{G_2}$ commute, the mapping $(g_1,
 g_2)\rightarrow g_1 g_2$ of $G_1 \times \overset{-}{G_2}$ onto 
 is an algebraic homomorphism. Hence $G_1 \bar{G}_2$ is a closed subgroup
 of $ GL (n,k)$. Hence the closure $\overset{-}{G}$ of $G$ is
 contained in $G_1  \overset{-}{G_2}$. Since $G_1 \subseteq G$
 and $G_2 \subseteq G$, it follows that $\overset{-}{G} =
 G_1  \overset{-}{G_2}$. Thus $\overset{-}{G}$ is a torus group or
 a direct product of $a$ torus group and the additive group $K$. Hence
 Theorem \ref{chap4:thm1} is equivalent to the following, by virtue of
 Chapter \ref{chap2}, Section 2, Corollary to Theorem \ref{chap2:thm1}. 

 \begin{thm}%% 2
   Let $G$ be a unipotent algebraic group of dimension one of $GL(n, k)$
   acting on $K[x_1 , \ldots, x_n]$, where $K$ is a field of
   characteristic\pageoriginale zero. Then $I_G$ is finitely generated
   over K.   
 \end{thm} 
 
 Our proof is mostly due to C. S. Seshadri.
 
 We first prove two lemmas. Unless otherwise stated $K$ denotes a
 field of characteristic zero. 

\setcounter{lem}{0}
 \begin{lem}\label{chap4:lem1}%%% 1
   Let $V$ be an affine variety and $G$ $a$ connected algebraic group
   acting on $V$. Let $W$ be a subvariety of $V$ and $H$ a
   subgroup of  $G$. Let $W$ be stable under $H$ and suppose $\bigg\{
   x^h \mid h \in H \bigg\} = \bigg\{ x^g \mid g \in G \bigg\} \cap
   W$, for every $x \in W$. Let $f$ be a $H$-invariant regular
   function on $W$. Then there exists a $G$-invariant rational
   function $f^\ast$ on $ \overline  {W^G}$ such that $f^\ast$ is
   integral over the local ring $\mho_x$ in $\overset{-}{W_G}$ for
   every $x \in W^G ; f^\ast$ takes unique value at $x$, for $x \in
   W^G$, and such that $f^\ast$ induces $f$ on W. (Lemma of Seshadri). 
 \end{lem} 
 
\begin{proof}
  Assume first $\overset{-}{W^G}$ is normal. The function $f$ defines
  a regular function $F$ on $W \times G$ defined by $F(x,g) = f(x)$,
  for $x \in W, g \in G$. 

  Let $Q$ be the regular mapping of $W \times G$ into
  $\overset{-}{W^G}$ defined by $\varphi(x,y)=x^g$. If $x^g =
  x^{\prime g^\prime}$, for $x, x^\prime \in W, g, g^\prime \in G$
  then $x^{gg'^{- 1}} = x' = x^h$ for some $h \in H$, by
  hypothesis. Thus $f(x^\prime) = f(x^h) = f(x)$. Thus F is constant
  on each fibre $\varphi^{-1}(x^g)$, for $x \in W, g \in G$. Let A be
  the coordinate ring of $\overset{-}{W^G}$. Let $U$ be the affine
  variety defined by the affine ring A[F]. The generic fibre of the
  projection of $U$ onto $W^G$ is reduced to $a$ single point. As the
  ground field $K$ is of characteristic zero, $U$ and $W^G$ are
  birationally equivalent. Thus $F$ induces $a$ rational function
  $f^\ast$ on $\overset{-}{W^G}$. By definition of $F$, $f^\ast$ is
  $G$-invariant. Further $f^\ast$ assumes the unique finite value $f(x)$
  at $x^g$, for $x \in W, g \in G$. As $\overset{-}{W^G}$ is normal
  $f^\ast$ is regular value at $x^g$ and the lemma is proved in the
  case $\overset{-}{W^G}$ is normal. 
\end{proof}
 
 Suppose\pageoriginale $\overset{-}{W^G}$ is not normal. Without loss
 of generality 
 we may assume $\overset{-}{W^G} = V$. Let $\overset{\sim}{V}$ be the
 derived normal model of $V$. Let $\overset{\sim}{W_1}, \ldots ,
 \overset{\sim}{W_s}$ correspond to $W$, in $\tilde{V}$. How $G$ operates on
 $\overset{\sim}{V}$. As $W^G$ is dense in $V$, so are
 $\overset{\sim}{W_{i}^{G}}$ in $\overset{\sim}{V}$. The functions $f$
 induces $H$-invariant regular functions $f_i $ on
 $\overset{\sim}{W_i} 1
 \leq i \leq s$. By what we have just proved, we get $G$-invariant
 rational function $f_{i}^{\ast}$, $1 \leq i \leq s$ of
 $\overset{\sim}{V}$ which is regular on $\overset{\sim}{W_{i}^{G}}$,
 and induces $f_i$. Now $f_{i}^{\ast}$, and $f_{j}^{\ast}$ take the
 same value on $\overset{\sim}{W_{i}^{G}} \cap
 \overset{\sim}{W_{j}^{G}}$. As $\overset{\sim}{W_{i}^{G}}$ contains a
 non-empty open-set of $\overset{\sim}{V}, f_{1}^{\ast} = \ldots =
 f_{s}^{\ast} = f^\ast $. Thus $f^*$ is regular on
 $\overset{\sim}{W_{i}^{G}}$, $1 \leq i \leq s$. Hence $f^*$ is integral
 over the local ring of $x^g$ in $W^G$, for $x \in W, g \in G$. Lemma
 \ref{chap4:lem1} is completely proved. 

\setcounter{rem}{0}
\begin{rem}%%% 1
  If the ground field is of characteristic $p \ne 0$, then Lemma
  \ref{chap4:lem1} 
  is true under the following modification of the rationality
  condition for $f^* : f^*$ is in a purely inseparable extension of
  the function field of $\overset{\sim}{W^G}$.  
\end{rem}

\begin{rem}%%% 2
  It is interesting to note that $W^G$ is open. For, each fibre 
  $\varphi^{-1}(x^g)$ is irreducible and of dimension equal to
  dimension of H, because of the condition on the orbits we have
  imposed. Thus $\varphi$ has no fundamental points and $W^G$ is
  open. 
\end{rem}

\begin{coro*}
  If furthermore $G$ is semi-simple and codimension of ${(W^G)}^C$ in
  $\overset{-}{W^G}$ is at least 2, then the ring $I_H$ of
  $H$-invariants in the coordinate ring of $W$ is finitely generated.  
\end{coro*}

\begin{proof}
  As codim ${(W^G)}^C \ge 2$, any function $f$ on $\overset{-}{W^G}$
  which is integral over the local rings of points of $W^G$ in
  $\overset{-}{W^G}$ is integral over the coordinate ring $A$ of
  $\overset{-}{W^G}$. Let $B$ be the ring got by adjoining all
  rational\pageoriginale functions $f^*$ on $\overset{-}{W^G}$, as in
  Lemma  \ref{chap4:lem1}. Since $B$ is contained in the derived
  normal ring of $A$, 
  $B$ is an affine ring and $G$ acts on $B$. As $G$ is semi-simple,
  the ring of invariants $I_G$ of $B$ is finitely generated. By Lemma
   \ref{chap4:lem1}, the ring $I_H$ of $H$-invariants in the
   coordinate ring of $W$ is the homomorphic image of $I_G$ and hence
   is finitely generated.     
\end{proof}

\begin{lem}\label{chap4:lem2}%%% 2
  Let $K$ be of characteristic zero and let $\rho$ be a rational
  representation of the additive group $H = \bigg\{ \begin{pmatrix} 1
  & \lambda\\0 & 1 \end{pmatrix} \mid \lambda \in K \bigg\}$ in
  $GL(n, k)$. Then there exists a rational representation
  $\rho^*$ of $SL(2, k)$ in $GL(n,\lambda)$ such that
  $\rho^* = \rho$ on $H$. 
 \end{lem}

\begin{proof}
  We shall denote the element $\begin{pmatrix} 1 & \lambda \\ 0 &
    1 \end{pmatrix}$ of H by $\lambda$. Choose $\lambda$ such  
  that $\rho (\lambda)\ne 1 (\lambda \ne 0$ is enough). Let the
  Jordon normal form of $\rho(\lambda)$ be 
  $$
  A = \begin{pmatrix}
    A_1 & & 0\\ 
& \ddots & \\ 
0 & &  A_r
 \end{pmatrix}, \quad
  \text{where}\quad  A_i
  = \begin{pmatrix} 
    110 & . &    0\\
    011 & & .. 0 \\
    .. &..\\ 
    & &  1\\
    0.. & . & 1
 \end{pmatrix}.
  $$ 
  Let $A_i$ have $n_i$ rows. Let $ \rho_{i}^{*}$ be the
  representation of $SL(2, K)$ given by 
  homogeneous forms of degree $n_i -1$ in two variables. It is easy to
  check that the Jordan normal form of $\rho_{i}^{*}(\lambda)$
  is $A_i$. Hence we may assume that
  $\rho_{i}^{*}(\lambda)=A_i$. We take
  $\rho^{*}=\begin{pmatrix} \rho_{i}^{*} &    0\\
    0 & \rho_{r}^{*} \end{pmatrix}$. 
 \end{proof}
 
\setcounter{rem}{1}
\begin{rem}%%% 2
  Lemma  \ref{chap4:lem2} is not true in the case where $K$ is of
  characteristic  
  $p \ne 0$, because the group $H$ has non-faithful rational
  representations which are not trivial. 
\end{rem}

\setcounter{proofoftheorem}{1}
 \begin{proofoftheorem}%%% 2
   Let\pageoriginale $G$ be given by a rational representation $\rho$
   of $H = \bigg\{ \begin{pmatrix} 1 & \lambda\\ 0 & 1 \end{pmatrix}
   \mid \lambda \in K \bigg\}$ in GL(n,K). By Lemma  \ref{chap4:lem2}
   we extend this 
   representation to $a$ rational representation $\rho^*$ of
   SL(2, K). Let 
   \begin{align*}
     G^{\prime} & = \left\{ 
     \begin{pmatrix} \rho^* (g) & 0\\ 0 & g \end{pmatrix}    
     \mid g \in SL(2,k) \right\}\hspace{3cm}\\ 
     \text{and}\qquad H^{\prime} & = \left\{ 
     \begin{pmatrix} \rho (\lambda) & 0\\ 0 & 
       \begin{pmatrix} 
         1 & \lambda\\0 & 1 
       \end{pmatrix} 
     \end{pmatrix}    \mid \lambda \in K \right\}
   \end{align*}
 
   The group $G^{\prime}$ acts on $K= [x_1, \ldots , x_n , x_{n+1},
     x_{n+2}]$. Let $V = K^n \times K^2$ and $W$ the subvariety defined
   by the equations $x_{n+1}-1 = 0 = x_{n+2}$. The group $G^{\prime}$
   acts on $V$ and $W$ is stable under $H^{\prime}$. If for a point 
   $$
   \displaylines{\hfill  
   Q =   (a_1, \ldots , a_n , 1, 0) \in W, \quad  Q^{g'} \in W, \hfill \cr 
   \text{where}\hfill g^{\prime} = 
\begin{pmatrix} 
\rho^*(g) & 0\\
 0 & g 
\end{pmatrix} \in G^{\prime}, 
g \in SL(2, K).\hfill }
 $$ 
   
   Then $(1,0)^g = (1,0)$. Hence $g \in
   H$. That is $g^{\prime} \in H^{\prime}$. Hence the orbit condition
   of Lemma  \ref{chap4:lem1} is satisfied. As $W^{G \prime}$ contains
   the complement 
   of the hyperplane $x_{n+1}=0, W^{G \prime}$ is dense in $V$. Further
   $(W^{G \prime})^C$ is contained in the variety defined by the
   equations $x_{n+1} = 0 = x_{n+2}$. The conditions of the corollary
   to Lemma  \ref{chap4:lem1} are satisfied. Hence the ring of
   $H^{\prime}$-invariants 
   $I_{H \prime}$ in the coordinate ring of $W$ is finitely
   generated. That is $I_G$ is finitely generated. 
 \end{proofoftheorem}
