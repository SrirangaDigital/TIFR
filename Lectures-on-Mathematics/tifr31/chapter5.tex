\chapter{Zarisk's Theorem and Rees' counter example}%%% chapter 5

{\bf 1. $\MG$-transform}. Let\pageoriginale $R$ be an integral domain with
 quotient field $L$. Let $\MG$ be an ideal of $R$. The set 
 $$
 S(\MG ;  R) = \bigg\{ f \mid f \in L , f \MG^n \subseteq R 
 ~\text{ for some }~ n   \bigg\} 
 $$
 is defined to be the $\MG$- transform of $R$. 

\setcounter{rem}{0}
 \begin{rem}%%% 1
   $S(\MG ; R)$ is an integral domain containing $R$.
 \end{rem}
 
\begin{rem}\label{chap5:rem2} %% 2
  It is clear that if $\MG$ and $\mathfrak{b}$ are two ideals with finite
  basis such that $\sqrt{\MG}= \sqrt{\MG}$, then $
  S(\MG ; R) = s(\mathfrak{b} ; R) $ (we recall that for an ideal
  $\MG$ in $R$,    $\sqrt{\MG}= \bigg\{ x \mid x \in R, x^n
  \in \MG,$ for some $n$ $\bigg\}$).  
\end{rem}

\begin{rem}\label{chap5:rem3}%% 3
  If $R$ is a noetherian normal ring and height $\MG \ge 2$, then
  $S(\MG, R) = R $.
\end{rem}

\begin{proof}
  Let $f \in S(\MG , R)$. The $f \MG^n \subseteq R$, for
  some $n$. As $\MG $ is not contained in any prime ideal of height 1,
  $f \in R_{\mathscr{Y}}$, for every prime ideal $\mathscr{Y}$ of
  height 1. Hence $f \in R$, $R$ being normal. 
\end{proof}

\begin{coro*}
  Let $R$ be as in Remark \ref{chap5:rem3} and $\MG = \MG_1 \cap \MG_2$ with 
  height $\MG_2 \ge 2$. Then $S(\MG , R) = s(\MG_1 , R)$.
\end{coro*}

\begin{rem}%%% 4
  If $R$ is an affine ring and height $\MG \ge 2$, then $S(\MG ; R)$ 
 is integral over $R$.
\end{rem}

\begin{proof}
  By passing to the derived normal ring we may assume that $R$ is 
  normal any apply Remark \ref{chap5:rem3}. 

For any integer $ n \ge 0$, set $\MG^{-n}= \bigg\{ f \mid f \in
  L , f \MG^{n} \subseteq R\bigg\}$, Then $S(\MG ; R) =
  \underset{n}{\bigcup} \MG^{-n}$. We shall abbreviate
  $S(\MG;  R)$ to $S$ when there is no confusion. 
\end{proof}

\begin{rem}%%% 5
  If $R$\pageoriginale is noetherian, $\MG^{-n}$ is a finite
  $R$-module.  
\end{rem}

\begin{proof}
  $\MG^{-n}\subseteq R \frac{1}{a}, a \ne 0, a \in \MG^{-n}$. 
\end{proof} 

\setcounter{proposition}{0}
\begin{proposition}\label{chap5:prop1}%%% 1
  Let $\MG$ be an ideal in $R$ with a finite basis  
  $a_1 , \ldots , a_n$, $(a_i \ne 0, i = 1 , \ldots , n)$. Let $t_1,
  \ldots, t_{n-1}$ be algebraically independent elements over
  $R$. Set $t_n = (1 -  \sum\limits_{i=1}^{n-1} a_i t_i ) \mid a_n$;
  this is an element of $R[t_1, \ldots, t_{n-1},
    \frac{1}{a_n}]$. Then  
 \begin{itemize}
\item[{\rm (i)}] $S = R [t]\cap L$, where $t$ stands for $(t_1 ,
  \ldots , t_n)$. 

\item[{\rm (ii)}] Further if $R$ is normal, then $S = R^* \cap L$,
  where $R^*$ is the derived normal ring $R[t]$.   
 \end{itemize}
\end{proposition}

\begin{proof}
 \begin{itemize}
\item[{\rm (i)}] Let $c = f(t) \in R [t]\cap L $. Choose $r$ greater
  than the degree of $f$. Then since $\sum\limits_{i=1}^n a_i t_i =1$,
  we have  
$$
a_{i}^{r} c \in R [t_1 , \ldots , t_{i-1} , \hat{t_i},
    t_{i+1}, \ldots , t_n],
$$ 
where `$\widehat{\empty}$' on $t_i$ indicates
  that $t_i$ has been omitted. As $t_i , \ldots , t_{i-1}, \hat{t_i},\break
  t_{i+1}, \ldots , t_n$ are algebraically independent over $R,
  a_{i}^{r} c \in R, i = 1 ,\break \ldots , n.$ Hence$\MG^{nr} c
  \subseteq R, i.e. c \in S$. Thus $R[t]\cap L \subseteq S$. On the
  other hand let $c \in S$. Then $c \MG^r \subseteq R$, for some
  $r$. Let $m_1 , \ldots , m_l$ be all the monomials in the $a_i$ of
  degree $r$. Raising the equation $\underset{i=1}{\overset{n}{\sum}}
  a_i t_i =1$ to the rth power, we have
  $\underset{i=1}{\overset{l}{\sum}} m_i f_i =1$, where $f_i \in
  R[t]$. Since $cm_i \in R , i =1, \ldots , t$, we have $c =
  \underset{i=1}{\overset{l}{\sum}} (cm_i) f_i \in R[t]$. Hence $S
  \subseteq R [t]\cap L$ and proof of (i) is complete. 

\item[{\rm (ii)}] To prove that $S = R^* \cap L$, we have only to show the
  inclusion $S \supseteq R^* \cap L$. Let $c \in R^* \cap L$. Then $c$
  is integral over $R[t]$ and we have a monic equation $c^s + f_1
  (t)c^{s-1}+ \ldots + f_s (t) = 0$, $f_i (t) \in R [t] , i=1, \ldots ,
  s$. 

  Let\pageoriginale $r \ge \underset{1\le i \le s}{\max}$ (degree of
  $f_i$). Then 
  $a_{i}^{r}c$ is integral over $R[t_1,\ldots , t_{i-1}$, $\hat{t_i},
    t_{i+1}, \ldots , t_n]$. As $R$ is normal and $t_1 ,\ldots , t_{i-1},
  \hat{t_i}, t_{i+1}, \ldots , t_n$ are algebraically independent,
  $R[t_1 ,\ldots , t_{i-1}, \hat{t_i}, t_{i+1}, \ldots , t_n]$ is
  normal. Hence $a_{i}^{r}c \in R$. Therefore  $c \MG^{nr}\subseteq R$,
  i.e., $c \in S$. Hence $R^* \cap L \subseteq S$ and (ii is proved). 
 \end{itemize}
\end{proof}

Let $R$ be an affine ring over a ground field $K$. Let $L^{\prime}$ be
a field with $K \subseteq L^{\prime} \subseteq L$, where $L$ is the
quotient field of $R$. Then, is $R \cap L^{\prime}$ an affine ring?
Let us call this \textit{Generalized Zariski's Problem.} We recall that
this is just the restatement of Zarisk's Problem (see Chapter \ref{chap0})
without the hyper thesis of normality on $R$. We shall later in this
chapter give a counter example to the above problem (with trans $deg_k
L^{\prime} = 2$).  

Let $R$ be an affine ring. By proposition \ref{chap5:prop1} it follows that : 
\begin{itemize}
\item[{\rm (i)}] If there exists an ideal $\MG$ in $R$ such that
  $S(\MG ;R)$ is not finitely generated, then $S(\MG ; R)$ is a
  counter example to the Generalized Zariski's Problem. 

\item[{\rm (ii)}] If further $R$ is normal and if there exists an
  ideal $\MG$ in $R$ such that $S(\MG ; R)$ is not finitely generated,
  then $S(\MG ; R)$ is a counter example to Zariski's Problem.  
\end{itemize}


\medskip
\noindent{\bf 2. Krull Rings and $\MG$ - transforms}

We shall now proceed to prove the converse of proposition \ref{chap5:prop1} (see
proposition \ref{chap5:prop4}) in the case when $R$ is normal. For
that we need some 
generalities on Krull rings. We say that an integral domain $R$ is
\textit{Krull ring} if there exists a set I of discrete valuations of
the quotient\pageoriginale field $L$ of $R$, such that (i) $R =
\bigcap\limits_{v \in I }  R_{v}$, where  $R_{v}$ denotes the
valuation  ring of $v$. \break  (ii) For  $a \in R$, $a \neq 0$, $v (a) = 0$,
for all but a finite number of  $v \in I$.   

\begin{proposition}\label{chap5:prop2}%Proposition 2
  An integral domain $R$  is a Krull ring if and only if the following
  two conditions are satisfied: 
  \begin{enumerate}[{\rm (i)}]
  \item  For every prime  ideal $\mathscr{Y}$ of height 1, 
  $R_{\mathscr{Y}}$  is a discrete valuation ring.

  \item Every  principal  ideal of $R$ is  the intersection of a
    finite number of primary ideals of height one. 
  \end{enumerate}
\end{proposition}

  For the proof  of Proposition \ref{chap5:prop2}, we refer to:
  $M$. Nagata ``Local 
  Rings''  Interscience  Publishers, Now York, 1962). 

\setcounter{rem}{0}
\begin{rem}%%% 1
  It easily follows that $R$ = $\bigcap R_{\mathscr{Y}}$, where
  $\mathscr{Y} $ runs through all prime ideals of height 1. 
\end{rem}

\begin{rem}%% 2
  Let $R = \bigcap\limits_{v \in I} R_{v}$  be a Krull ring defined
  by  a family of  discrete  valuations  $\{R_{v}\}_{v \in I}$  
  of  the  quotient field of $R$. Then for a   prime ideal
  $\mathscr{Y}$  of $R$ of height 1, we have $R_{\mathscr{Y}}  =
  R_{v}$, for some  $v \in I$. (For  prof  see $M$. Nagata ``Local
  rings'', Interscience Publishers, New York, 1962). 
\end{rem}
  
\begin{proposition}\label{chap5:prop3} %proposition 3
  If  $R$ is a Krull ring then the $\MG$-transform $S$ of
  $R$  is  also  a Krull ring. 
 \end{proposition} 

\begin{proof}
  Let  $J$ be  the  set of  those prime  ideals  $\mathscr{Y}$ of
  height  $1$ which  do not  contain  $\MG$. We  shall prove
  that $ S = \bigcap\limits_{\mathscr{Y}\in J} R_{\mathscr{Y}}$. Let
  $x \in S$. Then $x \MG^{n} \subseteq  R$, 
  for  some  $n$. As  $ \MG^{n} \not\subset  \mathscr{Y}$,
  for  $\mathscr{Y} \in  J$, $X \in \bigcap\limits_{\mathscr{Y} \in J}
  R_{\mathscr{Y}}$.  
 \end{proof}

 \noindent Conversely let $y \in \bigcap\limits_{\mathscr{Y} \in J}
 R_{\mathscr{Y}}$. Now as $R$ is a Krull ring, $\MG$ is
 contained\pageoriginale in a finite number  of prime ideals  of
 height 1. Therefore there exists an $n \ge 0 $  such  that $y
 \MG^{n} \subseteq R_{\mg}$ where $\mg$ is any
 prime ideal of height 1 containing  $\MG$.  Hence $y \MG^{n} \subseteq
 \bigcap\limits_{\mathscr{Y}} R_{\mathscr{Y}}, \mathscr{Y}$ running
 through all prime ideals  of height 1. By the remark  after
 Proposition \ref{chap5:prop2}, $y \MG^n \subseteq R$. Hence  $y \in S$  and
 the proposition is proved. 
   
\begin{proposition}\label{chap5:prop4}%proposition 4
  Let  $R$ be an  affine  normal  ring  over  a ground  field
  $K$. Let $L'$ be a field such that  $K \subseteq  L' \subseteq L$,
  where  $L$ is  the quotient  field of $R$. Set $R' = R \cap
  L'$. Then there  exists an affine normal ring  $\mathscr{O}$  and an
  ideal  $\MG$ of $\mathscr{O}$  such that  $R' = S
  (\MG ; \mathscr{O})$. 
\end{proposition}

We  shall prove  the  assertion in several  steps.
 \begin{enumerate}[(i)]
 \item $R'$ is a Krull ring.
   \begin{proof}
     We have $R = \bigcap \limits_{\mathscr{Y}}  R_{\mathscr{Y}}$,
     where $y$ runs through  all prime  ideals  of height 1.  
     Now $R' = \bigcap \limits _{\mathscr{Y}}  (R_{\mathscr{Y}}
     \bigcap L')$. Hence $R'$ is a Krull ring. 
   \end{proof}

 \item For  every prime ideal  $\mg$  of  height 1 in $R'$
   there exists a prime ideal $\mg$ of height  1 in  $R$
   lying above $\mg$ i.e. $\mg  \bigcap R' =
   \mg'$. 

   \begin{proof}
     We may assume  that  $L'$ is the  quotient field  of $R'$. We
     have  $R' = \bigcap\limits_{\mathscr{Y}} (R_{\mathscr{Y}} \bigcap
     L')$, as in  (i) By Remark \ref{chap5:rem2}  following  Proposition
     \ref{chap5:prop2}, we have 
     $R'_{\mg'} = R_{\mg} \bigcap L'$  for  some
     prime  ideal $\mg$  of  height  1  of $R$. 
      This proves  (ii).
   \end{proof}

 \item  There  exists a normal affine  ring  $\mathscr{O}' \subseteq
   R'$  such that  for every prime  ideal $\rho'$ of height 1
   of $R'$, we have  height  $(\mg' \bigcap  \mathscr{O}') = 1$.  
 
\begin{proof}
  We may  assume  that $L'$ is the  quotient  field of  $R'$. Take  a
  normal affine ring   $R'' \subseteq R' $  such that $L'$ is the
  quotient field of $R''$. 
    Let $Q'$ be  the set  of prime  ideals $\mg$ of  height  1
  in $R'$ with height \break $(\mg' \bigcap  R'')
  \big>1$.\pageoriginale  Let $T$ be 
  set of prime ideals $\mathscr{Y}$ of height 1 of $R$ such that
  heights $(\mathscr{Y} \bigcap R'') > 1$. Let $V,V''$ be the
  affine  varieties defined by  $R$ and  $R''$ respectively. For  a
  $\mathscr{Y} \in T , \mathscr{Y} \cap R''$ 
  defines isolated fundamental subvariety of $V''$ with respect to $V$
  under the morphism  $f:V \to V''$  defined by the inclusion $R''
  \subseteq R$.  
  Hence $T$ is finite. Therefore by  (ii), $Q'$  is finite. Let
  $R_{1}$  be an affine normal ring  such that  
  $R'' \subseteq R_{1}  \subseteq  R'$. Let  $Q'_{1}$  be  the set of
  prime  ideals  $\mg'_{1}$  of height 1 of $R'$ such  that
  height $(\mg'_{1} \bigcap R_{1})> 1$. Then $Q'_{1} \subseteq Q'$. We
  next prove that for a $\mg' \in Q'$, there exists an 
  affine normal ring  $R_{1}$ with $R''\subseteq R_{1} \subseteq R'$
  such that $\mg' \not\in Q'_{1}$, where  $Q'_{1}$  is as
  above. We claim that $R'_{\mg'}$ is a discrete valuation
  ring such that $trans.\underset{K}{deg}  R'_{\mg' /
    \mg' R' \mg'} = trans. \underset{K}{deg}$. $L'
  -1$. Let $\mg$ be a prime ideal of height 1 of $R$ lying
  above $\mg'$ (see (ii)). Let $x_{1}, \ldots x_{r} \in R$
  be such that (i) $x_{1},\ldots , x_{r}$ are algebraically
  independent over $R'$ (ii) $x_{1},\ldots ,x_{r} {\rm mod} \mg$
  from a transcendence base for $R/\mg$  over $R' /
  \mg'$. Such a choice is possible since $R_{\mg}$
  and $R'_{\mg'}$ are valuation rings. Since $R$ is  affine
  and $\mathcal{G}$ is of height  $1$, we have
  $\underset{K}{\transdeg}.R/ \mathcal{G}$ =
  $\underset{K}{\transdeg}. L-1 $. Also $\underset{K}{\transdeg}. R
  / \mathcal{G} =\underset{K}{\transdeg}. R'/ \mathcal{G}' +
  r$. Hence $\underset{K}{\transdeg}. R'/ \mathcal{G}' =$
  $\underset{K}{\transdeg}. L- 1 - r \ge $
  $\underset{K}{\transdeg}. L' -1$. Hence
  $\underset{K}{\transdeg}. R' / \mathcal{G}' = $
  $\underset{K}{\transdeg} L'-1$.  Since $R''$  is  affine and height
  $(\mathcal{G} \bigcap R'') >1$, we have trans. $^{R''}/ _{(R''
    \bigcap \mathcal{G}) \leq }\underset{K}{\transdeg}. L'- 2 $
  Let $y_{1}, \ldots , y_{l} \in R'$ be such that $y_{1}, \ldots ,
  y_{l}$ $\mod$ $\mathcal{G}'$ from a transcendence  base of  $R'/
  \mathcal{G}'$ over $^{R''}/ _{\mathcal{G} \bigcap R''}$. Let $R'''$
  be the derived  normal ring of $R''[y_{1},\dots , y_{l}]$. Then
  height  $(\mathcal{G}' \bigcap R'')= 1$. Since $Q'$ is\pageoriginale
  finite, in a  finite number of steps we arrive  at a normal affine ring
  $\mathscr{O}' \subseteq R'$, with the same quotient field as $R'$
  such that for every prime  ideal  $\mg'$ of height 1 of $R'$
  we have height $(\mg' \bigcap \mathscr{O}') \le 1$. But
  $\mathscr{O}'$ and $R'$ have the same quotient field. Therefore
  height $(\mathscr{O}' \bigcap \mg') =1$.  
   This proves (iii).
\end{proof}

\item Let $\mathscr{O}$ be as in (iii). Let $P$ be the set of prime
 ideals $\mathscr{G}$ of height 1 in $\mathscr{G}'$ for which there
 does not exist any prime ideal of height 1 in $R$, lying over
 $\mathcal{G} $. Let $V, V'$ be the  affine varieties defined by $R$
 and $\mathscr{G}'$ and let $f:V \to V'$ be the morphism induced by
 the inclusion  
 $\mathscr{G}' \subseteq R $. There are only a finite number of
 subvarieties of codimension 1 in $V'$ to which there do not
 correspond any subvariety of codimension 1 in $V$. Hence the set
 $P$  is finite. To prove Proposition \ref{chap5:prop4}, we take $\mathscr{O} =
 \mathscr{O}', \MG = \bigcap\limits_{\mathscr{Y} \in P}
 \mathscr{Y}$. Let $F$ be the set of prime  ideals of height 1 which
 do not contain $\MG$. Then $S(\MG;\mathscr{O}) =
 \bigcap\limits_{\mg \in  F }  \mathscr{O}_{\mg}$
 (See  Proposition \ref{chap5:prop3}). We have, $R' = \bigcap\limits_{\mg'}
 R' _{\mg'}$, where $\mg'$ runs through prime ideals
 of  $R'$  of height 1. Now  it  follows easily by our construction
 of $\mathscr{O}$ that $R' =  S (\MG;\mathscr{O})$. 
\end{enumerate}
 
Let $R$ be an  integral domain and let  $\MG$ be an ideal of
 $R$. We  say that the  $\MG$-transform of $R$ is 
 \textit{finite} if $S(\MG; R ) = R [\MG^{-n}]$ for some 
 $n \ge 0 $.  
 
\setcounter{thm}{0}
\begin{thm}\label{chap5:thm1}%%% 1
  Let $R$ be a normal  affine ring and $\MG$ be  an  ideal of
  $R$. Then the $\MG$-transform $S$ of $R$  is finite if  and  only
  if the  $\MG$ $P$- transform of $P$ is finite for  any $P =
  R_{\mathscr{Y}}$, where $\mathscr{Y}$, is a prime ideal of $R$.  
 \end{thm} 
 
\begin{proof}
  Clearly\pageoriginale  $\MG^{-n} P = (\MG P)^{-n}$, for 
  every $n$. Hence if $\MG'$ 
  transform  of $R$  finite, then  so  is  $\MG P$-
  transform of $P$, for  every  $P$. Conversely assume  $S$  is not
  finite. We  then define an  increasing sequence of  normal rings  by
  induction as follows: 
 
  Set  $R_{0} = R$. Having  defined  $R_{j},  0 \leq j \leq i$, we define
  $R_{i+1}$ as  the  derived normal ring of $R_i
  \bigg[(\MG(R_i))^{-1} \bigg ]$. Then we have $ S\supset R_i
  \supset R_{i-1} \bigg [(\MG_{i-1})^{-1}\bigg ]$, where such
  $R_i$ is an affine normal ring. By the definition of
  $\MG$-transform, $S=\underset{i}\cup R_i $. Further 
          $\MG R_i$- transform of $R_i$ is also $S$. We claim
          that height $\MG R_i=1$ for every $i$. For if height
          $\MG R_i >  1$, then the $\MG
          R_i-$transform of $R_i$ is integral over $R_i$ and therefore
          $S=R_i$. This contradicts the assumption that $S$ is not
          finite. Let $\MG_i$ be the intersection of those
          prime ideals of height 1 of $R_i$ which contains
          $\MG R_i$. Then we have $\MG_i \subseteq
          \MG_{i+1}$. For let $\mathscr{R}$ be a prime ideals
          of height 1 in $R_{i+1}$ containing
          $\MG R_{i+1}$. If height $(y \cap R_i=1)$, then by definition
          of ``$\MG_i$, we have $\MG_i \subseteq \mathscr{Y} \cap R_i$''.   
\end{proof}

 Otherwise let height $\mathscr{R}\cap R_i  \rangle 1$. Since the
 $\MG$-transform of  $R_i$ is  $S$ (see remark after the
 definition of $\MG$-transform), we have $(\MG R_i)^{-1} \MG_i^{m}
 \subseteq R_i$, for some $m$. Now if 
 $\MG_i \not\subset \mathscr{R} \cap R_i$, then
 $(\MG R_i)^{-1} \subseteq (R_i)_{\mathscr{R}\cap R_i}$. Hence
 $R \bigg [(\MG R_i)^{-1}\bigg ]\subseteq
 (R_i)_{\mathscr{R}\cap R_i}$. Therefore, $R_{i+1}\subseteq
 (R_i)_{\mathscr{R}\cap R_i}$, since $R_i$ is normal. Hence
 $(R_{i+1})_{\mathscr{R}} =(R_i)_{\mathscr{R}\cap R_i}$. This
 contradicts the assumption that height $\mathscr{R} \cap R_i >
 1$. Hence $\MG_i \subseteq \mathscr{R} \cap R_i$. Therefore
 $\MG_i \subseteq \MG_{i+1}$, for $i \ge 0$. Set
 $\MG^*=\underset {i}\cup \MG_i$. Then $\MG^*$
 is a proper ideals of $S=\underset{i}\cup R_i$. Let $\mathscr{Y}^*$
 be a\pageoriginale prime ideal containing $\MG^*$, got
 $\mathscr{Y}=\mathscr{Y}^* R_i 1$  and $P \MG  R_\mathscr{Y}$. We now
 consider the $\MG P$-transfer of 
 $P$. We can $(R_i)_i=(R_\mathscr{Y})_i $ Where $T=R-\mathscr{Y}$ and
 $(R_\mathscr{Y})_i$ are defined by $n$ infection as follows:  
 
 We set $(R_\mathscr{Y})_o = R_\mathscr{Y}$ and having defined
 $(R_\mathscr{Y})_o ,\ldots,(R \mathscr{Y})_i$ we define
 $(R_\mathscr{Y})_{i+1}$ as the derived normal ring of
 $(R_\mathscr{Y})_i  \bigg [(\MG(R_\mathscr{Y})_i)^{-1}\bigg ]$.  
 Since height $\MG(R_i)_T =1$ by our construction, $(R_i)_T$
 cannot be the $\MG P$-transform of $P$. Hence
 $\MG P$-transform is not finite. 
 
\medskip
\noindent{\bf 3. Geometric meaning of the $\MG$-transform.} 
   
 Let $V$ be a variety. We call an affine variety $V'$, an
 \textit{associated affine variety}  of $V$ if i) $V' \supseteq V$.
 ii) The set of divisors of $V'$ coincide with that $V i.e.$ the set of
 local rings of rank 1 of $V$ and $V'$ are the same. 
    
\begin{thm}\label{chap5:thm2} %%% 2
  Let $F$ be a proper closed set of an affine variety $V$ and 
  let $\MG$ be an ideal which defines $F$ in the affine ring
  $R$ of $V$. Then $V-F$ has an associated affine variety if and only
  if the $\MG$-transform $S$ of $R$ is finite; in this case $S$
  defines an associated affine variety and $S$ contains and is integral
  over the affine ring of any associates affine variety of $V-F$. 
 \end{thm}
 
\setcounter{lem}{0}
\begin{lem}\label{chap5:lem1}%%% 1
  Let $R$ be an integral domain and let $\MG$ be an ideal of
  $R$. Set $R'=R \bigg [\MG^{-n}\bigg ]$ or $S(\MG;R)$. Then
  the correspondence $\mathscr{Y'}\rightsquigarrow \mathscr{Y'}\cap R$
  establishes a $1-1$ corresponding between the set of prime ideals of
  $R'$ not containing $\MG$ and the set of prime ideals of $R$
  not containing $\MG$. Further, for a prime ideals
  $\mathscr{Y'}$ of $R'$ with $\mathscr{Y'}\not\supset \MG$, we
  have $ R'_\mathscr{Y'} = R_{\mathscr{Y'} \cap R}$. 
\end{lem} 

\begin{proof}
  Let\pageoriginale $\mathscr{Y}$  be a prime ideals of $R$ which does
  not contains 
  $\MG$. Let $a \in \mathscr{U}, a \notin \mathscr{Y}$. Then
  $R' \subseteq R \bigg[\frac{1}{a} \bigg ]$. Since $\mathscr{Y} R
  \bigg[\frac{1}{a} \bigg ]$ is a prime ideal, so is $\mathscr{Y'}=
  \mathscr{Y} R \bigg[\frac{1}{a} \bigg ] \bigcap R'$. Further
  $\mathscr{Y'} \bigcap R = \mathscr{Y}$ and $R'_\mathscr{Y'}=
  R_\mathscr{Y} $. 

  Conversely if $\mathscr{Y'}$ is a prime ideal of $R'$  which does
  not contain $\MG$, then $\mathscr{Y}=\mathscr{Y'} \bigcap
  R$ does not contain $\MG$ and $R_\mathscr{Y}=R'_\mathscr{Y'}$. 
\end{proof}

\begin{lem}\label{chap5:lem2}%%% 2
  Let $\MG$ be an ideal of a noetherian domain $R$. Let $S$
  be the $\MG$-transform of $R$ and $R'$ a subring of $S$
  containing $R$. Then the $\MG R'$-transfer of $R'$ is $S$. 
\end{lem}

\begin{proof}
  Let $S'$ be the $\MG R'$-transform of $R'$. We have only to
  prove that $S' \subseteq S$. Let $f \in S'$. Then there exists an
  $n$ such that $f \MG^n \subseteq R'$. Let $a_1,\ldots,a_l
  \in \MG^n$ generate $\MG^n$. There exists an $m$
  such that $fa_i \MG^m \subseteq R$. Hence  $f \MG^{n+m} \subseteq R$
  and the lemma is proved.  
\end{proof}

\begin{coro*}
  With the above notation, if $a \in S$, then $aS: \MG S = aS$
  and consequently $\MG S$ is not of height 1.
 \end{coro*}
    
\begin{proof}
  Let $f \in S, f \MG \subseteq S$ a i.e. $\frac{f}{a}
  \MG \subseteq S$. $\frac{f}{a} \in S $.     

  We now prove Theorem \ref{chap5:thm2}. Suppose $S$ is finite. Then $S$ is an
  affine ring and defines an associated affine variety by lemma
  \ref{chap5:lem1} 
  and Corollary. to Lemma \ref{chap5:lem2}. Conversely assume that $V'$ is an
  associated affine variety of $V-F$. Let $R'$ be the affine ring of
  $V'$. Let $x' \in R'$. Set $\MG_{x'} = \bigg \{ y \mid y
  \in R, yx' \in R \bigg \}$. Since $x' \in R_\mathscr{Y}$, for
  $\mathscr{Y} \not\supset \MG$ (by the hypothesis) we
  have $\MG_{x'} \not\subset \mathscr{Y}$, for
  $\mathscr{Y} \not\supset \MG$. Hence $\MG_{x'}$ contains
  a power of $\MG$. Therefore $x' \in S$ i.e., $R' \subseteq S
  $. Since the divisors of $V'$ and $V-F$ are the same, height
  $(\MG R')\ge 2$. Hence $S$ is integral over $R'$. Hence $S$
  is an affine ring and therefore finite. 
      
  Let\pageoriginale $V$ be an affine variety defined by an affine ring
  $R$ and let $F$ be a closed set defined by an ideal $\MG$.   
\end{proof}   
     
\setcounter{dashthm}{2}
\begin{dashthm}\label{chap5:thm3'}%%% 3
  The variety $V-F$ is affine if and only if $1 \in \MG S$,   
  where $S$ is the $\MG$-transform of $R$. In this case $F$ is
  pure of codimension 1 and  $S$ is the affine ring of $V-F$. 
 \end{dashthm} 
     
\begin{proof}
  Suppose $V-F$ is affine. Then $V-F$ is an associated affine variety
  of $V-f$. Let $R'$ be the coordinated ring of $V-F$. Then $R'
  \subseteq S$ (by Theorem \ref{chap5:thm2}). Now $1 \in \MG R'$. Hence $1
  \in \MG S$. Conversely suppose that $1 \in
  \MG S$. Then $1 \in \MG  \MG^{-n}$ for some
  $n$. Set $R' = R  \bigg[\MG^{-n} \bigg ] $. Since
  $\MG R' \ni 1$, Lemma \ref{chap5:lem1} of Theorem \ref{chap5:thm2},
  proves that the 
  affine variety defined by $R'$ is $V-F$. $R' = S$ because
  $\MG R' \ni 1$ (by virtue of Lemma \ref{chap5:lem2}). 
\end{proof}  
   
 It now remains to prove that $F$ is pure of codimension 1
 if $V-F$ is affine. Suppose the contrary. Let $F_{1}$ be an
 irreducible component of $F$ with codimension $F_{1} >  1$. Let
 $f \in R $, with $f$ not vanishing on $F_1$ and vanishing on all tho
 other irreducible components. Then considering $R \bigg [\frac{1}{f}
   \bigg ]$ we may suppose that $F=F_1$. But this would mean that $S$
 is integral over $R$. Hence $1 \not\in \MG S$. Contradiction.  
     
 \begin{thm}\label{chap5:thm3}%%% 3
   $V-F$ is an affine variety if and only if $\MG(\MG(P)^{{-n(P)}}\ni
   1 $  for every local ring $P$ of $F$ (the  integer $n(P)$ depending
   on $P$).  
\end{thm}     
     
\begin{proof}
  If $V-F$ is affine, then by Theorem \ref{chap5:thm3}, $1 \in \MG S$ i.e $1
  \in \MG \MG^{-n}$  
  for some $n$. Hence $1 \in \MG (\MG P)^{-n}$ for
  every $P$ of $F$. Conversely assume that $1 \in \MG (\MG P)^{-n
    (P)}$ for every $P$ of $F$. Then by lemma \ref{chap5:lem2} of
  Theorem \ref{chap5:thm2}, the
  $\MG P$-transform of $P$ is finite for every $P$ of $F$. Hence
  by Theorem \ref{chap5:thm2}, $S$ is finite. Then $1\in \MG S$ and $V-F$ is an
  affine\pageoriginale variety. 
\end{proof} 

\setcounter{corollary}{0}
\begin{corollary}%% 1
  If $V$ is an affine variety and $F$ a divisorial closed subset of
  $F$ such that some multiple of $F$ is locally principle, then $V-F$
  is an affine variety. 
\end{corollary}
     
\begin{corollary}%%% 2
  If $V$  is a non-singular affine variety and $F$ a divisorial
  closed subset of $V$, then $V-F$ is an affine variety. 
 \end{corollary}     

\begin{corollary}\label{chap5:coro3}%%% 3
  If $V$ is an affine curve and $F$ a closed subset of $V$, then $V-F$
  is again affine. 
\end{corollary}    


\begin{proof}
  It is sufficient to prove the Corollary when $F$ consists of a
  single point $P$. If $P$ is normal then by theorem
  \ref{chap5:thm3'}$' $, $V-F$ is
  affine. If $P$ is not normal we consider the derived normal ring
  $P'$ of $P$. Let $C$ be the conductor of $P'$ with respect to
  $P$. Then $\MG^n \subseteq C$ for some $n,
  \MG$ being an ideal which defines $F$. By considering $R
  \bigg [\MG^{-n} \bigg ]$ we are reduced to the case when $F$
  consists of normal points. This proves the Corollary. 
\end{proof}     

\medskip
\noindent{\bf 4. Zariski's theorem and some related results.}

\begin{thm}[Zariski's]\label{chap5:thm4}%%% 4
  Let $R$ be an affine ring over a ground filed $K$ and let $\Omega$
  be the quotient field of $R$. Let $L$ be a subfield of $\Omega$
  containing  $K$. 
\begin{itemize}
\item[{\rm (1)}] If $\transdeg. \underset{K}{L}=1$, then $R \bigcap L$
  is an affine ring 

\item[{\rm (2)}] If $R$ is normal and $\underset{K}\transdeg L=2 $,
  then $R \bigcap L$ is an affine ring.  
\end{itemize}
\end{thm}


\medskip
\noindent{\textbf{Proof of (1)}} is a consequence of Proposition
\ref{chap5:prop4}, 
Theorem \ref{chap5:thm2} and Corollary \ref{chap5:coro3} to theorem
\ref{chap5:thm3'}$'$.  
 
\medskip 
\noindent{\textbf{Proof of (2)}} By virtue of Proposition
\ref{chap5:prop4}, (2) 
is contained in the following theorem: 
    
\begin{dashthm}\label{chap5:thm4'}%%% 4
  Let\pageoriginale $R$ be an affine normal ring of dimension 2 over a
  ground field $K$. Then  for any ideals $\MG$ if $R$ the
  $\MG$-transform $S (\MG, R)$ is finite.      
\end{dashthm}
     
We require the following lemma

\begin{lemma*}
  Let $S$ be a Krull ring, $\mg'$ a prime ideal of height 1. Let
  $\MG$ be an ideal of $S$ such that $\MG R_\mg =\mg R_\mg$. Then $\MG
  :\mg \not\subset \mg$.      
\end{lemma*} 
     
\begin{proof}
  Since $R$ is a Krull ring  $R_\mg$ is a discrete valuation
  ring.Therefore there is an $a \in \MG $ such that
  $aR_\mg=\mg R_\mg$. Since $R$ is a Krull
  ring, $a R=\mg \cap \mg_1 \cap\ldots \cap
  \mg_n ,\mg_i$ being primary ideals of  height 1
  different from $\mg$. Then $\mg \not\supset
  \mg_1 \cap \ldots \cap \mg_n =aR:\mg
  \subseteq \MG :\mg$.     
\end{proof} 

\medskip
\noindent{\textbf{Proof of Theorem $4'$.}}
   We may assume that the ideal $\MG$ is pure of height 1. Choose an
   element $b \in \MG$ such that $\mathscr{V}_{\mathscr{Y}_i} 
   (b)=n_i,1 \le i \le r$ where $\mathscr{U}_{\mathscr{Y}_i}$ is the
   normal valuation corresponding to $\mathscr{Y}_i$. Then
   $\mathscr{U}\cap \mathscr{U}^1 =bR$  where $\mathscr{U}'$ is an
   ideal pure of height 1 such that $\mathscr{U}$ and $\mathscr{U}'$
   do not have common prime divisors of height 1. We have $\MG S = bS$
   (cf. Cor to Lemma \ref{chap5:lem2}, p.50). Choose an element a
   $a \in \MG$ such that a is not contained in any of the prime
   divisors of $\MG$ and
   $\mathscr{V}_{\mathscr{Y}_i}(a) >\mathscr{V}_{\mathscr{Y}_i}
   (b)=n_i ,1 \le i \le r$. Let $x$ be a transcendental element over
   $R$. Extend the ground field $K$ to $K(x)$. We remark that the 
   $\MG K(x)[R]$-transform of $R'=K(x)[R]$ is finite
   if and only if the $\MG$-transfer of $R$ is finite. Now
   $\MG R' = \mathscr{K}^{(n_1)}_1 \bigcap \ldots \bigcap
   \mathscr{K}^{(n_r)}_r$, where $\mathscr {K}$ is the centre on $R'$
   of the valuation $\mathscr{V}_{\mathscr{K}_i}$ on $\Omega (x)$
   define by $\mathscr{V} _{\mathscr{K}_i}(\sum a_1 x^i)=\min\limits_j
   \mathscr{V}_{\mathscr{Y}_i}(a_j)$. The element $a$ and $b$ do not have a common
   prime divisor in $S$. The choice\pageoriginale of the element $a$
   and the following any lemma show that we may assume that $bS$ is a
   prime ideal.
     
\begin{lemma*}
  Let $S$ be a Krull ring and let $a,b \in c$ with $a,b$ not having a
  common prime divisor. Then $ax-b$ prime in $S[x]$. 
\end{lemma*}
       
    Now assume that the $\MG$-transform $S$ of $R$ is not
    finite. Then as in the proof of Theorem \ref{chap5:thm1} (of this chapter),
    there exist normal local rings $(P_i,\mathscr{W})$, $0 \le i <
    \infty $ such that $P_0 =R_\mathscr{Y}$ for suitable prime ideal
    $\mathscr{Y} \supset \MG$ and $(P_{i},\mathscr{W}_i)$ dominates
    $(P_{i-1},\mathscr{W}_{i-1})$. Furthermore $S^* =\underset{i}\bigcup
    P_i=S_M$, where $M$ is a maximal ideals of $S$. Set
    $\mathscr{W}^\ast = \cup \mathscr{W}_i$. Consider $\mathscr{W} =
    \mathscr{W}^* \cap R$. Since $bS \cap R = \MG'$, the canonical
    mapping $\varphi: R/\mathscr{V}' \to S / bS$ is an injection. For
    $s \in S$, we have $s \MG ^m \subseteq R$ for some $m$. Since
    $\MG^m \not\subset \MG'$ is of dimension 1, by the theorem of
    Krull-Akizuki (see M. Nagata ``Local rings'', Theorem 33.2, p.115)
    $S/bS$ is noetherian and $S/M$ is of finite length over $R
    /\mathscr{W}$. In particular $\mathscr{W}^*$ is finitely
    generated. Hence exists all such that $\mathscr{W}_1 S^* =
    \mathscr{W}^*$ and that $S^* / \mathscr{W}^*$ is of finite length
    over $P_1/ \mathscr{W}_1$.

We now proceed to prove that $S^*$ is noetherian. Since height
$(\mathscr{W}^*) = 2$, by virtue of a theorem of Cohen we need only
prove that every prime ideal of height 1 of $S^*$ is finitely
generated. (see M. Nagata, ``Local rings'' Theorem 3.4, p.8).

Let $\mg^*$ be a prime ideal of height 1. Set $\mg = \mg^* \cap
R$. Then $S^* \mg^* = R_\mg$. Hence by the lemma proved we have
$\mg''= S^* : \mg \not\subset \mg^*$. But $\mg S^* \subseteq
\mg''$. We claim that height $\mg'' \geq 2$. For  
 if $\mg'' \subseteq \mathcal{R}$,\pageoriginale a prime ideal
        of height 1 of $S^*$, then 1. $\mg=S^*_\mathcal{K}=S^*_{\mg^*}$. Hence
        $\mathscr{K}=\mg^*$ a contradiction to the fact that
        $\mg'' 4'' \mid $.Hence either $\mg''=S^*$ or
        $\mg'' ~ is~\mathscr{NW}$-primary. Hence
        $\mathscr{NW}^{*t}\subseteq \mg''$ for some $t$. Since
        $S/\mathscr{NW}^{*r}$ is artinian and $\mathscr{NW}^{*r}$ is
        finitely generated we conclude that $\mg''$ is finitely
        generated. Now $\mg''/_{\mg \mg''}$ is a
        finitely generated over $S^*/\mg^*$. But in
        $S^*/\mg^*$ the only prime ideals are $\boxplus ^* /
        \mg^*$ and $(0)$ and $S^*/ \mg^*$ is
        Noetherian. Hence $\mg^* \cap \mg''
        /\mg^* \mg''$ is finitely generated. Hence
        $\mg'' \cap \mg^* \diagup \mg S^*$,
        being residue class module of $\mg'' \cap \mg^*
        \diagup \mg^* \mg''$, is finitely
        generated. Hence $\mg" \cap \mg^*$ is finitely
        generated. Since $S^*/\mg''$ and $S^*/\mg^*$
        are noetherian, we have $S^* \diagup \mg''\cap
        \mg^*$ is noetherian. Hence $\mg^*
        /\mg'' \cap  \mg^*$ is finitely
        generated. Hence $\mg^*$ is finitely generated. Hence
        $S^*$ is noetherian. 
    
    The ring, $P_{1}$ being a geometric normal local ring, is
    analytically normal. Further $S^*$ and $P_1$ have the some
    quotient field, $\mathscr {NW}_{1} S^* =\mathscr {NW}^*$ and $S^*
    /_{\mathscr {NW}^*}$ is of finite length over $R/_{\mathscr
      {NW}^*}$ and therefore over $P_1/ \mathscr {NW}_1 P_1$. Hence by
    Zariski's main Theorem (see $M$. Megata, ``Local ring'' Theorem
    37.4, P.137) $P_1 =S^*$. This is contradiction to the
    construction of the $P_i$. 
    
\begin{thm}%%% 5
  Let $V$ be a normal  affine variety  of dimension 2 and lot $F$ be
  a divisorial closed subset of $V$. Then $V-F$ is affine. 
\end{thm}    
    
\begin{proof}
  Let $\mathscr{U}$ be the ideal defining $F$. By Theorem \ref{chap5:thm4} the
  $\mathscr{U}$-transfer $S$ of $R$ is finite. Let $V'$ be the affine
  variety  defined by $S$. Since height $\mathscr{U}\le 2$ and $V$ is
  of dimension 2, $V-F$ is isomorphic to an open subset $V''$ of $V'$
  such that $V'-V''$ consists of atmost a a finite number\pageoriginale
  of points. Let 
  $\times ' \in V'-V''$. Consider the morphism $V' \to V$ induced by
  the inclusion $R \subseteq S$. Since $f_{-1}f(\times ')$ is discrete,
  by Zariski's Main Theorem $f$ is bio-holomorphic at $x'$. Let $W$ be
  an irreducible component of $F$ passing through $f(x')$. Since $x'$
  and $f(X')$ are bio-holomorphic there is subvariety of codimension
  $1$ of $V'$ passing through $x'$ and laying over $W$. This
  contradicts the fact that $V'$ is the associated affine variety of
  $V-F$. Hence $V-V''$ is empty and $V-F$ is affine.  		   
\end{proof}     

We now proceed to give an example to show that Theorem
\ref{chap5:thm4}(2) is false 
if we do not assume that $R$ is normal. Take $R=K[X,Y,Z]/(f)$. Where 
\begin{enumerate}[(1)]
\item $ f(X,Y,Z)=Y(Z+YT)+X(u_1 YZ+ U_2 Z^2)$ 
\item $f$  is irreducible   
\item $T,U_1, \in U_2 \in K [X,Y]$.
\end{enumerate}  

 Let the image $X,Y,Z,T,U_1,U_2$ be denoted by $x,y,z,t,u_1~and~u_2$
 respectively so that $R=K[x,y,z]=0$. Set $\mathscr{Y}=(x,y)$. Since
 $R/_{\mathscr {Y}}=K[X,Y,Z]/_{(f,X,Y)} \approx K [Z]$, the ideal
 $\mathscr{Y}$ is prime. Similarly $\mg=(y,z)$ is prime. We
 shall show that the $\mathscr{Y}$-transfer of $R$ is not finite. We
 first prove that 
 
 \noindent
 (*) \quad $\mathscr{Y}^{-1}=R+z_1 R, z_1 =(z+yt)/X $.
    
 \begin{proof}
   We have $(x)=\mathscr{Y}\cap (x,2+yt)$. For let $\lambda x +\mu
   y=ax+ \beta (z+yt)$.  
   Then $\mu y^2 \in (x)$ and therefore $\mu y^2 t \in (x)$ i.e. $\mu yz
   \in (x)$. But $z$ is not a zero divisor module $(x)$. Hence $\mid (y
   \in (x)$. Therefore $(x)=\mathscr{Y} \cap (x,z+yt)$. Lot now $g \in
   \mathscr{Y}^{-1}$. Then $g=\frac {\gamma}{x}=\frac
           {\gamma}{y}~i.u. \gamma \in (x):(y)$. Since
           $(x)=\mathscr{Y} \bigcap (x,Z+yt)$,\pageoriginale we have
           $(x):(y)=(x,Z+yt):(y)$. But $\ell \in K [x,y]$ and therefore
           y is not a zero divisor module $(x,z+yt)$. Hence
           $(x,Z+yt):(y)=(x,Z+yt)$. Hence $\gamma \in R+z_1 F$. This
           proves (*). Set $R_1=R[\mathscr{Y}^{-1}]=R
           [z_1]=K[x,y,z_1]~~K[X,Y,Z_1]/_{\mathscr{U}}$ (say) where
           $Z_1=(Z+YT)/_X$. We have $f_1(X,YZ)=Xf_1 (X,Y,Z_1)$ where
           $f_1(X,YZ)=Xf_1 (X,Y,Z_1)=Y(Z_1+Yt_1)+X(U'_1 Y Z_1 +U'_2
           Z_1^2),T_1=U_2 T^2-U_1 T, U'_! =U_1-2U_2 T, U'_2=U'_2
           X$. Now $f(x,y,z)=X f_1 (x,y,z_1)=0$.  Hence $f_1
           (x,y,z_1)=0 ~i.e.~~f_1 (x,y,z_1)=0 \in \mathscr{U}$. We
           claim next that the element $f_1 (x,y,z_1)=0$ is prime in
           $K[X,Y,Z_1]$. Suppose $g_2 (x,y,z_1)$. Then one of the
           $g_1$ say $g_2$ is a unit in $K \bigg [X,Y,Z,
             \frac{1}{X}\bigg ]$. Thus $f_1 (x,y,z_1)=X^r g_1
           (x,y,z_1)=0$ for some $r \ge 0$. But $f_1 (x,Y,Z_1)$ is not
           divisible by in $K[X,Y,Z_1]$. hence $f_1 (X,Y,Z_1)$ is
           irreducible. Hence it follow that $R_1 =
           R[\mathscr{Y}^{-1}]=K[X,Y,Z_1]/_{(f_1)}$. Further $f_1$
           satisfied the same condition as $f$. Proceeding in the same
           way we see that the $\mathscr{Y}$-transform $S$ of $R$ is
           obtained by the successive adjunction of elements $
           Z_1,Z_2,Z_3,\ldots,$ where $Z_{n+1}\frac{z_n +yt_n}{X}, tn
           \in K[x,y]$. This shows that $S$ is not finite. 
 \end{proof}    
    

\medskip
\noindent{\textbf{5. Rees' counter example.}}

 Let $K$ be a field of an arbitrary characteristic and lot $C$ be a
 non-singular plane cubic curve defined over $K$. For a natural
 number $n$ and a fixed point $Q$ of $C, T_n =\bigg \{ P \mid np ~\text{or}~
 nQ \bigg \}$ is a finite act,  because $C$ is of positive genus. We
 choose here $Q$ to be a point of inflexion (it is well known that a
 non-singular plan cubic has 9 points of inflexion). Then $P \in
 T_{3d}$ if and only if there is a plane curve $C_d$ of
 degree\pageoriginale $d$ 
 such that $C_d.C=3dP$. Thus we are that the set of points $P$ and
 $C$, such that $C_d.C$ is a multiple of $P$ on suitable curve $C_d$
 of positive degree, is a countable set. Therefore there is a point
 $P$ of $C$ such that no np $(0.>0)$ is linearly equivalent to any
 $C.C_d$ (on $C$). (Note that the system of $C_d.C$ is complete linear
 this follows from the arithmetic normality of $C$.) We fix such a
 point $P$ also and we enlarge $K$, if necessary, so that $P$ and $Q$
 are rational over $K$. 
     
 Let $H=K [x,y,z]$ be the homogeneous coordinate ring of $C,(x,y,z)$
 being a generic point of $C$ over $K$. Let $\mathscr{Y}$ be the prime
 ideals of $H$ which defines $P$. Let $t$ be a transcendental element
 over $H$ and consider the ring $S$ generated by all of $at^{-n}$ with
 $a \in \mathscr{Y}^{(n)}$ ($n$ runs through all natural numbers) over
 $H[t]$. 
     
 We want to show that:
 \begin{enumerate}[(1)]
 \item $S$ is not finitely generated over $K$.

 \item There is a normal affine ring $R$ such that $S=R \cap L$,where
   $L$ is the field of quotient of $S$. 
 \end{enumerate}      

 This $S$ is the Roes' counter example to Zariski problem in the case
 of transcendence degree 3. 
     


\medskip
\noindent{\textbf{Proof of (1).}} Consider $\deg \mathscr{Y}^{(n)}$
(=minimum if degree 
 of elements of $\mathscr{Y}^{(n)})$. if a homogeneous element $h$ of
 degree $d$ is in $\mathscr{Y}^{(n)}$, then $h$ defines $C_d.C$ with a
 suitable plane curve $C_d$ of degree $d$. Since $h \in
 \mathscr{Y}^{(n)}, C_d.C$ contains $nP$. By the choice of $P, C_d.C
 \neq nP$, whence $3d=deg C_d. C >n ~\text{and}~d > n/3$. Let $\sigma$ be one
 of 1, 2, 3 \pageoriginale and such that $m' \sigma$ in divisible by
3. Set $d = (n + \sigma)/3$. Since $C$ is an abelian variety, there is
a point $R$ of $C$ such that $ n(P-Q) + (R-Q) \sim 0$.  Then  
 $$
 nP + R+ (\sigma-1) \sim 3 U \sim C_d. C, 
 $$
where $C_d$ is a curve of degree $d$.  Since $C$ is a non-singular
plane curve, the system of all $C_d.C$ (with fixed $d$) is a complete
linear system,  whence there is a $C_d$ such that $nP+R+(\sigma-1) Q
=C_d. C.$  Let $h$ be the homogeneous form of degree $d$ in $H$ defined
by $C_d$.  Then $h~~ \mathscr{Y}(n)$. Thus we see that  
$$
(\ast) \qquad \qquad  \deg \mathscr{Y}^{(n)}= (n+\sigma) /3, 1\leq
\sigma \leq 3. 
$$

This $(\ast)$ being shown, we see that $S$ is not finitely generated
over $K$ by the same way an in the construction of the fourteenth
problem in Chapter III. 

\medskip
\noindent{\textbf{Proof of (2).}}  We first show that
$$
(\ast \ast) \qquad \qquad S = H [t,t^{-1}] \cap V,
$$
where $V$ is the valuation ring obtained as follows:


Let $p$ be a prime element of the valuation ring
$H_{\mathscr{Y}}$.  Then $t/p$ is transcendental over
$H_{\mathscr{Y}}$, whence we have a valuation ring
$H_{\mathscr{Y}}(t/p)( =
H_{\mathscr{Y}}[t/p]_{pH_{\mathscr{Y}}[t/p]}).H_{\mathscr{Y}}(t/p)$ 
is independent of the particular choice of $p$ and this valuation ring
is denoted by $V$. 

It is obvious that $S \leq H [t,t^{-1}] \cap V$.  Let $f$ be an
arbitrary element of  $H [t,t^{-1}] \cap V$.  Then $f$ is of the form
$\sum a_i t^i$\pageoriginale (finite sum) with $a_i \epsilon H$ ($i$
may be negative). Let $v$ be a valuation defined by $V$. Then by the
construction of 
$V, v(\sum a_i t^i) = \min V(a_i t^i) \geq 0$. Therefore  $a_i
\epsilon \mathscr{Y} ^{(-i)}$ if $i< U$, which implies that $f
\epsilon S$. Thus $(\ast \ast)$ is proved. 
	
This being settled, it remains only to prove the following lemma by
virtue of Proposition \ref{chap5:prop1} (\S\ 1). 

\begin{lemma*}
  Let $R$ be a normal affine ring of a function field $L$ over a field
  $K$ and let $V_1, \ldots, V_n$ be divisional valuation rings of $L$
  (i.e., $V_i$ are discrete valuation rings of $L$ over $K$ such that
  trans.  $\deg_{k} L-1$ = trans.  $\deg$. of the residue  class field
  of $V_i$).  Then there is a normal affine ring $\mathcal{V}$ with an
  ideal $\MG$ such that $R \cap V_1 \cap \cdots \cap V_n = S
  (\MG ; \mathcal{V})$. 
\end{lemma*}

The proof is substantially the same as that of Proposition
\ref{chap5:prop4} (\S\ 2) and we omit the detail.  
