

\chapter{A generalization of the original  fourteenth
  problem(contd.)}\label{chap2} %%% chap 2  

{\bf 1.} In this\pageoriginale chapter we shall consider the  generalized problems
  with certain assumption on the representation of $G$. In fact we
  shall prove the following: 

\setcounter{thm}{0}
\begin{thm}\label{chap2:thm1}%%% 1
 Let $R=K [a_1, \ldots , a_n]$ and let $G \subseteq GL(n,K)$ act on
 $R$ as a group of automorphisms of $R$ over $K$. Suppose $G$
 satisfies the following condition: ($\ast$) If $\lambda$ is a
 representation  of $G$ by a finite dimensional $K(G)$- module $M
 \subset R$ and  $\lambda =\left(\begin{smallmatrix} 1 & \ast_ {\ldots} \ast
 \\ 0 &  \\ \vdots &  \lambda' \\ 0 & \end{smallmatrix}\right)$, then  $\lambda$
 is  equivalent  to  $\left(\begin{smallmatrix} 1, & 0_ {\ldots} 0 \\ 0 &  \\ :
   & \lambda' \\ 0 & \end{smallmatrix}\right)$. Then $I_G$ is finitely generated.  
We shall prove theorem in several steps.
\end{thm}

\setcounter{lem}{0}
\begin{lem}\label{chap2:lem1}%%% 1
 Let $G$ satisfy the condition $(\ast)$ and let  $f \in R$. Then there
 exists  an $f^\ast \in I_G \bigcap (\sum\limits_{g} f^g K)$ such that
 $ f-f^* \in  \sum\limits_{g,g'} (f^g - f^{g'}) K$. 
 \end{lem}
 
 \begin{proof}
Let $M=\sum\limits_{g}f^g K$ and $N= \sum\limits_{g,g'}
   (f^g - f^{g'}) K$. The  vector spaces  $M$ and $N$  are
   $G$-modules. Let  the  dimension of $M$ be $m$. We  shall prove the
   lemma  by induction on $m$. Suppose  that  for all  $h \in R$ such
   that  dimension of  $\sum\limits_{g} h^g$ $K$ is  $< m$, there
   exist  $h^\ast \in I_G \bigcap (\sum\limits_{g} h^g K)$  with
   $h-h^\ast \in \sum\limits_{g,g'} (h^g-h^{g'}) K$. The  assertion  is
   trivial when  $m=0$. Further if  $f \in N$, we may take  $f^\ast
   =0$. Suppose  $f' \notin N$. Then $M = Kf+N$ and  $f$ is  $G$-invariant
   module $N$. Hence  by the  condition $(\ast)$ there  exists  a
   $G$-Module  $N^\ast$ of dimension 1 in $M$ and an $f' \in I_G$
   \pageoriginale with $M = K f^\prime + N^ \ast$. If  $f^\prime
   \notin  N$, then 
   $M=Kf^\prime +N$ and the  lemma  is proved. Suppose  $f'
   \in  N$. Set  $f= \lambda f^\prime + h$ with  $\lambda \in K$,
   $h  \in N^\ast$. Since  $ M_{1} = \sum\limits_{g} h^g K \subset
   N^{\ast}$,  we  have  $\dim_K M_1\leqslant \dim
   N^{\ast}=m-1$. Hence  by  induction  hypothesis  there  exists  an
   $ h^{\ast}\in  I_G \bigcap (\sum\limits_{g} h^g K)$ such  that
   $h- h^{\ast} \in N_{1} =\sum\limits_{g,g_1}(h^g-h^{g1}) K$. Since
   $N_{1}\subseteq N $, the  proof of Lemma \ref{chap2:lem1}  is complete with
   $f^{\ast}= h^{\ast}$.  
 \end{proof}

\begin{remark*}
 If $R$  is a graded  ring  and   $f\in R$ is homogeneous, the
 representations  of $G$  which  occur in the  above  proof  are all
 given  by  $G$-modules generated by  homogeneous  elements of the
 same  degree. 
\end{remark*}

\setcounter{proposition}{0}
\begin{proposition}\label{chap2:prop1}%%% 1
 Let  $R$ and  $G$ be as in Theorem \ref{chap2:thm1} and  let  $R'$ be a ring
 containing $K$. Let $G$ act  on $R'$ as a group of automorphisms of
 $R'$ over $K$. Assume  that there  is a surjective  homomorphism
 $\varphi$ of  $R$ onto  $R'$ such  that  $\varphi (a^g)=\varphi
 (a)^g$ for all $a \in R$. Let  $I'_G$  denote  the $G$-invariant
 elements of $R'$. Then  $ \varphi (I_G)=I'_G$. 
\end{proposition}

\begin{proof}
 We  have only to show  that $I'_G \subseteq \varphi (I_G)$. Let  $f'
 \in I'_G$. Let  $f \in R$  be such  that $\varphi(f)=f'$. By  Lemma
 \ref{chap2:lem1} there  exists  an $f^\ast \in  I_G$ such  that  $f-
 f^\ast  \in  N= 
 \sum\limits_{g,g'} (f^g -f^{g'})K$. But  as  $f'$  is $G$-invariant,
 $N$ is  contained  in the  kernel  of $\varphi$. Hence $\varphi
 (f^\ast)= f'$. Hence $\varphi(I_G)=I'_G$. 
\end{proof}


We shall  now prove  Theorem \ref{chap2:thm1} in the  case  when  $R$
is a graded 
ring i.e. $R =K [a_1, \ldots , a_n] \thickapprox  K[x_1, \ldots
  ,x_n] / \mathcal{U}$  with $x_1,\ldots,x_n $ algebraically
independent  over   $K$ and $\mathcal{U}$, a homogeneous ideal of  $K
[x_1, \ldots x_n]$.  

\begin{lem}\label{chap2:lem2}%%% 2
 Let\pageoriginale  $S =   \sum\limits^\infty_{1=0}   S_i $ be a graded
  ring. Assume that the ideal $\sum\limits_{i>0}  S_i$ have a finite
  basis. Them $S$ is finitely generated over $S_0$. 
\end{lem}

\begin{proof}
  Let $ h_i$, $i = 1,\ldots, r $ be a set of generators for $I$. We may
  assume that $h_i$ are all homogeneous, say $h_i \in S_{j(i)}$,
  $j(i)>0$. Then we assert that $ S = S_0[h_1, \ldots ,h_r]$. It is
  enough to prove that every homogeneous element $f$ of $S$ is in $
  S_0 [h_1, \ldots ,h_r]$. The proof is by induction on the degree $i$
  of $f$. For $i = 0$, there is nothing to prove.   
Assume  $i > 0$.  Suppose that for all $ t < i$, $S_t \subseteq S_0
[h_1, \ldots ,h_r]$. We have $ f =\sum\limits_{k=1}^r g_k  h_k$,  $g_k
\in  S_{i-j}(i)$. By induction hypothesis $g_k \in S_0 [h_1, \ldots
  ,h_r]$, $1 \leq k \leq r$. Hence $  f \in S_0 [h_1, \ldots ,h_r]$. 
\end{proof}

\begin{lem}\label{chap2:lem3}%%% 3
 Let $R$ and $G$ be as in Theorem \ref{chap2:thm1}. Let $f0_1, \ldots ,
  f_r \in I_G$.  Then  $(\sum\limits^r_{i=1} f_i R )\bigcap I_G =
  \sum\limits_{i=1}^r I_G f_i$.  
\end{lem}

\begin{proof}
 The proof is by induction on $r$. For $r = 0$ there is nothing  to
 prove. Assume $(\sum\limits_{i=1}^s  f_iR)  \bigcap I_G =
 \sum\limits_{i=1}^s f_i I_G$ for $ s<r$. Let $f \in (
 \sum\limits^r_{i=1} f_i R )\bigcap I_G$. Then $  f=
 \sum\limits_{i-1}^i h_i f_i$, $h_i \in R$. By Lemma  \ref{chap2:lem1}
 there exists an 
 $h' \in N = \sum\limits_{g, g'}(h^{g}_r - h^{g'}_r) X$ such that $
 h_r +h' \in I_G$. As, for  $g, g' \in G, \sum\limits_{i=i}^r\sum
 (h^{g}_i - h^{g'}_i) f_i =0$, there exist $ h'_i \in R$, $1 \leq i
 \leq r-1$ with $\sum\limits_{i=i}^{r-1} h'_i f_i + h' f_r = 0$. Hence
 $ f - (h_r + h') f_r =  \sum\limits_{i=1}^{r-1} (h_i + h'_i) f_i$. 
 
 But $ f - (h_r + h') f_r \in I_G $ and therefore by induction
 hypothesis, there exist $ h''_i \in I_G$, $1 \leq i \leq r-1 $  such
 that $ \sum\limits_{i=1}^{r-1} (h_i +h'_i) f_i =
 \sum\limits_{i=1}^{r-1} h''_1 f_i$. 
 
 The\pageoriginale proof of Lemma  \ref{chap2:lem3} is complete. 
 \end{proof}
 
 \begin{remark*}
 If $R$ is graded and  $f_i $ are homogeneous, it is enough to assume
 the condition (*) for representations of $G$ given by $G$-modules
 generated by homogeneous elements of $R$ of the same degree. 
  \end{remark*} 

 \begin{proposition}\label{chap2:prop2}%%% 2
 Let $R = K [a_1, \ldots , a_n]$ be a graded ring.  Assume that
 every representation $ \lambda$ of $G$ given by a $G$-module
 generated by homogeneous elements of $R$ of the same degree and of
 the form  $\lambda = \begin{pmatrix} 1 & *\dots *\\ 0 &  \\ \vdots &
   \lambda^{\prime}\\ 0 \end{pmatrix} $ is equivalent to
 $ \begin{pmatrix} 1 & 0\dots 0\\ 0 &  \\ \vdots &
   \lambda^{\prime}\\ 0 \end{pmatrix}$. Then $ I_G $ is finitely
 generated over $K$. 
    \end{proposition} 

\begin{proof}
 As R is graded, so is $I_G$. Let I be ideal of $I_G$ of all
 elements of positive degree. 
\end{proof}  

As $R$ is noetherian, the ideal $R$, $I$ of $R$ is finitely
generated. Let $f_1, \ldots , f_r$, $\in$, $I$ generate $R$, $I$. We
may assume the $ f_i $ to be homogeneous. By Lemma  \ref{chap2:lem3}
and the remark 
following it we have $ (\sum\limits^r_{i=1}, R \, f_i) \bigcap I_G =
\sum\limits_{i=i}^r I_G \, f_i = I$. Hence by Lemma  \ref{chap2:lem2}, $I_G $ is
finitely generated over $K$.   

We shall now consider the non-graded case. Let R = K $ [a_1, \ldots ,
  a_n]$. Let be a transcendental element over $R$. Consider the
homogeneous ring $ R^{*} =K  [a_1 t , \ldots , a_n t, t]$. Then $G$
acts on $ R^{*} $ if we set $t^{g} =t$, for every $ g\in G$. Further
the $G$-module homomorphism $ \varphi : R^{*} \, \rightarrow \,R$
defined by $\varphi\, (f(a_1 t, \,\ldots \,a_n \, t, t) = f (a_1 \,
\ldots , a_n , 1) $ induces an isomorphism of a finite $G$-submodule
of $ R^{*} $ generated by homogeneous elements of the same degree of $
R^{*} $ onto a finite $G$-submodule of $R$. 
Hence the condition (*) is satisfied by all representation 
$\lambda^{*}$ of $G$ given\pageoriginale by $G$-submodules of $ R^{*}$
generated by homogeneous elements  
of the same degree. Hence by proposition \ref{chap2:prop2}, $L^{*}_G$ the ring of
G-invariant elements of $R^{*}$ is finitely generated over
$K$. Further as every  $ f \in R $ is the image of a homogeneous in $
R^{*}$, it follows from Proposition \ref{chap2:prop1} and the remark after Lemma
\ref{chap2:lem1} that $\varphi(I^{*}_G) = I_G$. Hence $I_G$ is
finitely generated 
over $K$. 

\medskip
\noindent
{\bf 2.} \quad We now give examples where the condition (*) is satisfied. It is
obvious that if every rational representation $\lambda$ of $G$ is
completely reducible then it satisfies the condition (*). We shall
later give some criteria of complete reducibility of rational
representation of an algebraic group.  
\begin{enumerate}[(1)]
\item Consider a torus group $G$ (i.e. a connected algebraic linear
  group which is diagonalizable), acting on $R = K \, [a_1, \ldots ,
    a_n]$. In this case if $G$ is diagonalized each $K \, a^{i_{1}}_1
  \ldots a^{i_{n}}_n $ is a $G$-module. Thus $R$ is the direct sum of
  simple $G$-modules. Hence every representation $ \lambda $ of $G$
  given by a $G$-sub-module of $R$ is completely reducible. Hence
  $L_G$ is finitely generated.  

\item Let $K$ be a field of characteristic zero and let $G \subseteq
  GL (n, K)$ be semi-simple. Then it is well known that (see,
  Chevalley, Theorie des groupes de Lie, III) every rational
  representation of $G$ is completely reducible. Hence in this case
  again $I_G$ is finitely generated.  

\item Combining (1) and (2) we have : If $G$ is an algebraic group
  whose radical is a torus group, then $I_G$ is finitely generated.           
\end{enumerate}

We add here a Corollary to Theorem \ref{chap2:thm1}:  

\begin{coro*}
 Assume\pageoriginale that $K$ is of characteristic zero and that an
 algebraic group 
 $G \subseteq GL (n, K) $ acts on $R = K   [a_1 ,\ldots , a_n]$. Let
 $N$ be a normal subgroup of $G$ which contains the unipotent part of
 the radical of $G$. If $ I_N = $ the set of $N$-invariants of $R$ is
 finitely generated over $K$, then so is $I_G$  
\end{coro*}

\begin{proof}
 $G$ acts on $I_N$ and $(I_N)_G  = I_G$. The closure of $N$ being
  denoted by $\bar{N}$, the action of $G$ on $I_N$ is really an action
  of $ G/ \bar{N}$, whose radical is a torus group. Therefore by (3)
  above, $I_G$ is finitely generated. 
\end{proof}

As a further application of Theorem \ref{chap2:thm1} we prove the following:

\begin{thm}[Generalization of Fischer's theorem]%%%% 2
  Let $K$ denote the complex
  number field and let $G \subseteq GL(n, K)$ act on $R = K [a_1,
    \ldots ,a_n]$ as a group of automorphisms of $R$ over $K$. Assume
  that for every  $A \in G$, $\bar{t-}{A}$ (complex conjugate of $A$)
  is in $G$. Then $I_G$ is finitely generated over $K$.   
 \end{thm}

\begin{proof}
 Let $\bar{G}$  be the closure of $G$. Then $\bar{G}$ also satisfies
 the hypothesis of the above theorem. Hence we may assume $G$ is
 algebraic. Let $H$ be the radical of $G$. Let $H_u$ be the set of
 unipotent elements of $H$ (we recall that $A \in GL(n, K)$ said to be
 unipotent if all the eigen values of $A$ are 1). It is well known
 that $H_u$ is an algebraic group. It is enough to prove that $H_u$
 consists of identity element only. For then $H$ is a torus group ($A$
 connected linear algebraic group which has no unipotent elements
 other identity is a torus group if either the field is algebraically
 closed or the group is solvable). It follows that\pageoriginale $I_G$
 is finitely generated by (3) above.  
\end{proof}

It remains to prove that $H_u$ consists of identity element only.  
Let $A \subseteq H_u$. Now $H$ is characteristic in $G$ (i.e. admists all
automorphisms of $G$). Considering the automorphism $g \rightarrow
({}^t \bar{g})^{-1}$,  we see that $ (\overset{t-}{A})^{-1}  \in
H$ and hence $  (\overset{t-}{A})^{-1} \in H_u $ since $
(\overset{t-}{A})^{-1} $ is unipotent. The element $  \overset{t-}{A}
A$ is unipotent and hermitian. Hence $\overset{t-}{A} A = E$, the
identity matrix. Hence $A$ is unitary. But the only unipotent unitary
matrix is the identity matrix. Thus $A = E$ and the theorem is
proved. 

\medskip
\noindent{\bf 3.}
\quad Let $R = K [a_1, \ldots , a_n]$ be an integral domain and let $G
\subseteq GL (n, K)$ act on $R$ and satisfy the condition (*). Let $V$  be
the affine $K$-variety defined by $R$ (the points of $V$ lie say, in
the algebraic closure $  \overline{K}$ of $K$). Let $L$ be the
function field of $R$ and let $L_G $ be the field of $G$-invariant
elements in $L$. The group $G$ acts on $V$ in a natural way. A subset
$F$ of $V$ is said to be $G$-admissible if for every $P \in F$, $P^{g}
\in F$, for every $g \in G$.  

\begin{thm}\label{chap2:thm3}% theorem 3
 Let $F$ be a $G$-admissible closed subset of $V$. Let $T = \bigg\{f
 \bigg| f \in L_G$, $f$ regular at every point of $F \bigg\}$. Then
 $T$ is a ring of quotients of $I_G$. 
 \end{thm}

\begin{proof}
 Let $S$ denote the multiplicatively closed set of all $s \in I_G$
 such that $s$ does not vanish at any point of $F$. We shall show that
 $T= (I_G)_s$. We have only to show that $T = (I_G)_s$. Let $f
 \in T$. Consider the ideals $\mathcal{U} = \bigg\{ g \bigg | g \in R$,
 $gf \in R \bigg\}$ and $ \mathcal{U} (F) = \bigg\{ h \bigg|h \in R$,
 $h(F) = 0 \bigg\}$. 
 As the closed set defined by $\mathcal{U}$ and $F$ are disjoint, we
 have, $\mathcal{U} + \mathcal{U} (F)= R$. Hence there exists a $g \in
 \mathcal{U}$, $g' \in \mathcal{U}(F)$ with $g + g' = 1$.  
 
 
 It is\pageoriginale clear that the ideals $\mathcal{U}$ and 
 $\mathcal{U} (F) $ are  
 $G$-admissible. Hence $\sum\limits_{\sigma, \sigma' \in G}
 (g^{\sigma} - g^{\sigma'}) K \subseteq \mathcal{U}(F) \bigcap
 (\sum\limits_{\sigma} g^{\sigma} K) \subset
 \sum\limits_{\sigma} g^{\sigma} K \underline 
 {\subseteq} \mathcal{U}$. By Lemma \ref{chap2:lem1} of Theorem
 \ref{chap2:thm1} there exists a $ 
 g^{*} \in  I_G \bigcap \sum\limits_\sigma 
 g^{\sigma}K$ such that $g - g^{*} \in \sum\limits_{\sigma, \sigma'} 
 (g^{\sigma} - g^{\sigma'}) K$. Hence $g^{*}(P) = 1$ for every $P \in
 F$ and $g^{*} f \in R$. The theorem is proved.  
\end{proof}

Consider  the relation $ \sim $ in $V$ defined by  $P \sim Q$ if
$\big\{\overline{P^{G}}\big\} \bigcap  \big\{\overline{Q^{G}}\big\}
\neq \phi$ for $P, Q \in V$ (where $P^{G}$ is the orbit of $P$, namely
the set of all $ P^{\sigma}$ with $\sigma \in G$ and $\bar{\infty}$
denotes the closure in $V$).  

\begin{thm} % theorem 4
 {\rm (1)} $\sim$ is an equivalence relation. {\rm (2)} The quotient
 set $V/\sim$ 
  i.e. the set of all equivalence classes by $\sim$ acquires the 
  structure of an affine $K$-variety with coordinate ring $I_G$ such
  that the natural mapping $V \rightarrow V/\sim$ is regular.  
 \end{thm} 

\medskip
\noindent{\textbf{Proof of (1):}} 
We shall in fact give the following characterization which 
 proves that $\sim$ is an equivalence relation: $P \sim Q$ if and
 only if $f(P) = f(Q)$, for every $f \in I_G$. Let $P \sim Q$ and let
 $P' \in \big\{\overline{P^{G}}\big\} \cap
 \big\{\overline{Q^{G}}\big\}$. We have $f(P) = f(P^{G}) = f(P') =
 f(Q^{G}) = f(Q)$, for every $ f \in I_G$. To prove the converse we
 remark that if $F_{1}$ and $F_{2}$ are two $G$-admissible disjoint
 closed sets, then as in the proof of Theorem \ref{chap2:thm3} we can
 find an $f \in 
 I_G$ such that $f(F_{1}) = 1$, $f(F_{2}) = 0$. The closed set $
 \big\{\overline{P^{G}}\big\}$, for $P \in V$ is $G$-admissible. If
 $P, Q$ are not related by $\sim$ we can separate them by an $f \in
 I_G$. Hence (1) is proved.  

\medskip
\noindent{\textbf{Proof of (2):}}
By Theorem \ref{chap2:thm1}, $I_G$ is finitely generated over $K$. Let
 $f_{1}(a),\ldots , f_{l}(a)$ generate $I_{G}$. Let $W$ be the
 $K$-affine variety defined by $I_{G} = K[f_{1}, \ldots
   ,f_{l}]$. Consider the regular mapping $\varphi : V \rightarrow W$
 defined\pageoriginale by $\varphi (P) = \overline{P} = (f_1(P),
 \ldots f_{l}(P)) 
 \in W \subseteq \overline{K}^{l}$. By (1), $\varphi (P) = \varphi(Q)$
 if and only if $P \sim Q$. Hence $ \overline{\varphi}^{1}
 (\overline{P}) = \bigg\{Q \big|P \sim Q \bigg\} = \underline{P}$,
 say. Let $  \overline{\mathcal{M}}$ be the maximal ideal
 corresponding to a point $\overline{P}$ of $W$. Then by Lemma
 \ref{chap2:lem3} of 
 Theorem \ref{chap2:thm1} $\overline{\mathcal{M}} R \bigcap I_G =
 \overline{\mathcal{M}}$. Hence $\varphi$ surjective and we identify
 the equivalence class $\underline{P}$ with the point $\overline{P}$
 of $W$. By Theorem \ref{chap2:thm3} the local ring at $\overline{P}$
 of $W$ is $(I_G)_{\overline{P}} = \underset{x \in \underline{P}}{(\bigcap} \mho
 _{x}) \bigcap L_G$, where $\mho_{y}$ denotes the local ring at the
 point $y$ of $V$.  

\begin{corollary}\label{chap2:coro1}%%% 1
 Let $Q \in \big\{\overline{P^{G}}\big\}$. Then      
$(I_G)_{\overline{P}} = \begin{pmatrix} \underset{Q^{*} \in
     \big\{\overline{Q^{G}}\big\}}{\bigcap}
   \mho_{Q^{*}} \end{pmatrix} \bigcap L_G$. 
\end{corollary}

\begin{proof}
 We have only to prove  
$$
\bigcap\limits_{Q^{*} \in \big\{\overline{Q^{G}}\big\}  \mho_{Q^{*}}}
 \subseteq (I_G)_{\overline{P}}. \quad \text{ Let} \quad f \in
 \bigcap\limits_{Q^{*} \in    \big\{\overline{Q^{G}}\big\}}
 \mho_{Q^{*}}.  
$$
Then by Theorem \ref{chap2:thm3}, $f = h/g$, $h$, $g \in I_G$, $g
(Q^{*}) \neq 0$, for every $Q^{*} \in Q^{G}$. But as $Q \in 
\big\{\overline{P^{G}}\big\}$, $g(Q) = g(P')$ for every $P' \in
\underline{P}$.  Hence $f \in (I_G)_{\overline{P}}$.  
\end{proof}

\begin{corollary}%% 2
Under the hypothesis of Corollary \ref{chap2:coro1}, if $Q^{G}$ is closed,  then
$\mho_{Q} \bigcap L_G = \left(\begin{smallmatrix} \underset{P^{*} \in
    \big\{\overline{P^{G}}\big\}}{\bigcap}
  \mho_{p^{*}} \end{smallmatrix}\right) \bigcap L_G$.  
\end{corollary}

\begin{proof}
 For any $Q \in V$, $(\underset{Q^{*}\in Q^{G}}{\bigcap} \mho_{Q^{*}})
 \bigcap L_G = \mho_Q \cap L_G$. As $ Q^{G}$ is closed, by Corollary
 \ref{chap2:coro1}, $\mho_Q 
 \bigcap L_G = (I_G)_{\overline{P}} = (\bigcap\limits_{P^{*} \in
   {\{P^{\overline{G}}\}}}\mho _{P^{*}}) \bigcap L_G$.   
\end{proof}

As we noted before, our Theorem \ref{chap2:thm1} implies the following 

\begin{thmstar}\label{chap2:thmstar1}
 Let\pageoriginale  $R = K   [a_l, \ldots, a_n]$  and let $G \subseteq
 GL(n, K)$  act on $R$ as a group of automorphisms of $R$. over $K$.
 If every  rational representation of $G$ is completely
 reducible, then $I_G$  is finitely generated.    
\end{thmstar}

What we like to remark here is that when we want to prove this Theorem
\ref{chap2:thmstar1}{${}^*$}, we need not use the technique at the end
of \S 1, and we can prove as follows: $R$ is a homomorphic image of a
polynomial ring $K[x_1, \ldots , x_n]$ on which $G$ acts
naturally. The polynomial ring is graded, hence the result on the
graded case and Proposition \ref{chap2:prop1} prove Theorem
\ref{chap2:thmstar1}{${}^*$}.     
