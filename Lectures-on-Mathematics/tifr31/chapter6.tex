\chapter[Complete reducibility of rational representation....]{Complete reducibility of rational representation of a matrix
  group}%Chapter VI 

This\pageoriginale chapter is mostly a representation of M. Nagata:
Complete reducibility of rational representation of a matric group,
J. Math. Kyoto Univ. 1-1 (1961), 87-99. 

It is well known in the classical case that every rational
representation of a semi-simple algebraic linear group is completely
reducible.  But the same argument becomes false in the case where the
universal domain is of characteristic $p\neq 0$.  For instance, when
$K$ is a universal domain of characteristic 2, the simple group
$SL(2, K)$ has the following rational representation $\rho$  which is
not completely reducible: 
$$
\rho 
\begin{pmatrix} 
a & b\\
c & d 
\end{pmatrix} = 
\begin{pmatrix} 
1 & ac & bd\\ 
0 & a^2 & b^2\\ 
0 & c^2 & d^2
\end{pmatrix}.
$$

(This $\rho$ is not completely reducible because $ac$ and $bd$ are not
linear polynomials in $a^2, b^2, c^2, d^2$.) Therefore it is an
interesting question to ask conditions for an algebraic linear group
$G$ so that every rational representation of $G$ is completely
reducible. 

Now, our answer of the above question, can be stated as follows: 
\begin{enumerate}
\renewcommand{\labelenumi}{\bf(\theenumi)}
\item When $p \neq 0:$ Every rational representation of $G$ is
completely reducible if and only if there is a normal subgroup
$G_{\circ}$ of finite index such that (i)  $G_{\circ}$ is a subgroup
of a torus group (i.e., diagonalizable) and (ii) the index of
$G_{\circ}$ in $G$ is prime to $p$.  If $G$ is
connected,\pageoriginale then the 
above condition is equivalent to the condition that the representation
of $G$ by homogeneous forms of degree $p$ is completely reducible.  On
the other hand, if $G$ is an algebraic group (which may not be
connected), then the complete reducibility of all rational
representations of $G$ is equivalent to the condition that every
element of $G$ is semi-simple (i.e., diagonalizable). 

\item When $p=0:$ Each of the following two conditions is equivalent to
the complete reducibility of all rational representations of $G$.   
\end{enumerate}

\begin{enumerate}[(I)]
\item The closure of $G$ has a faithful rational representation which
  is completely reducible. 
\item The radical of the closure of $G$ is a torus group.
\end{enumerate}

We shall prove also the following interesting theorem concerning the
complete reducibility of rational representations of a connected
algebraic linear group: 

If $G$ is a connected algebraic linear group, then every rational
representation of $G$ is completely reducible if (and only if) the
following is true: 

If $\rho'= \begin{pmatrix} 1 & \tau \\ 0 & \rho \end{pmatrix}$ is a
rational representation of $G$, then $\rho'$ is equivalent to the
representation $\begin{pmatrix} 1 & 0 \\ 0 & \rho \end{pmatrix}$. 

\medskip
\noindent{\textbf{1. Preliminaries on connected algebraic linear
    groups.}}

Throughout\pageoriginale this chapter, $K$ denotes a universal domain
of an arbitrary characteristic, unless the contrary is explicitly
stated.  Let $G$ be a connected algebraic linear group contained in
$GL(n, K)$. A Borel subgroup $B$ of $G$ is defined to be a maximal
connected solvable subgroup of $G$.  Then as was proved by Borel, the
following is true:    

\setcounter{lem}{0}
\begin{lem}\label{chap6:lem1}%% 1
  The homogeneous variety $G/B$ is a projective variety.  On the other
  hand, every element of $G$ is in some conjugate of $B$. 
\end{lem}
 
Now we have:	

\begin{lem}\label{chap6:lem2}%%% 2
  If $u \in G$ is unipotent, $G$ being a connected algebraic
  linear group, then there is a closed connected unipotent subgroup of
  $G$ which contains $u$. 
\end{lem}

\begin{proof}
  By the last half of Lemma \ref{chap6:lem1}, we see that $u$ is in a Borel
  subgroup $B$ of $G$.  Since $B$ is solvable, the set $U$ of all
  unipotent elements of $B$ is a closed connected subgroup, which
  proves the assertion. 
\end{proof}

On the other hand, the following was proved by Borel:
	
\begin{lem}\label{chap6:lem3}%%% 3
  If a connected algebraic linear group $G$ consists merely of
  semi-simple elements, then $G$ is commutative, hence is a torus
  group. 
\end{lem}

Next we shall concern with an algebraic group which is not connected: 
	
\begin{lem}\label{chap6:lem4}%%% 4
  Let $G$ be an algebraic linear group and let $G_{\circ}$ be the
  connected component of the identity of $G$.  Then each coset
  $G_{\circ}g(g \in G)$ contains an element of finite order. 
\end{lem}
 
\begin{proof}
  Let  $A$ be the smallest algebraic group containing $g$ and let
  $A_{\circ}$ be the connected component of the identity of $A$.
  Since $A$ is commutative and\pageoriginale since $A_{\circ}$ is
  infinitely  divisible (in the additive formulation), $A_{\circ} g$
  contains an  element of finite order, which proves the assertion.  
\end{proof}
 
\medskip
\noindent{\textbf{2. Preliminaries on group representations.}} 

Let $G$ be an abstract group, let $G_{\circ}$ be a normal subgroup of
$G$ and let $K$ be a field of characteristic $p$ which may be zero,
throughout this section, except for in Lemmas \ref{chap6:lem8} and
\ref{chap6:lem9}.  
	
The following lemma is well known :

\begin{lem}\label{chap6:lem5}%%% 5
  If a finite $K$-module $M$ is a simple $K-G$-module, then $M$ is the
  direct sum of a finite number of $K-G_{\circ}$-modules which are
  simple. 
\end{lem}
 
\begin{coro*}
  If a representation $\rho$ of $G$ in $GL (n, K)$ is completely
  reducible, then the restriction of $\rho$ on $G_{\circ}$ is
  completely reducible.   
\end{coro*} 

The converse of the above Corollary is not true in general if $p \neq
0$, but we have: 
	
\begin{lem}\label{chap6:lem6} %%% 6
  Let $\rho$ be a representation of $G$ in $GL(n, K)$.  If the
  restriction $\rho_{\circ}$ of $\rho$ on $G_{\circ}$ is completely
  reducible and if the index $t=[G:G_{\circ}]$ is finite and not
  divisible by $p$, then $\rho$ itself is completely reducible. 
\end{lem}

\begin{proof}
  If $\rho$ is not completely reducible, then $\rho$ contains a
  representation of the form $\begin{pmatrix} \rho_1 & \tau \\ 0 &
    \rho_2 \end{pmatrix}$ which is not completely reducible and such
  that $\rho_1, \rho_2$ are irreducible.  Hence we may resume that
  $\rho = \begin{pmatrix}\rho_1 & \tau \\ 0 & \rho_2 \end{pmatrix}$ and
  that $\rho_1$, $\rho_2$ are irreducible.  Let the representation
  module of $\rho$ be $M^\ast. M^\ast$ contains the representation
  module $M$ of $\rho_2$ and $M^\ast/ M$ is the representation module
  of $\rho_1$.  Since $\rho_0$ is completely reducible, we see
  that\pageoriginale $M$ 
  is a direct summand of $M^\ast$ are a  $G_{\circ}-K$-module. Hence
  $M^\ast = M \oplus N_1 \oplus \cdots \oplus N_r$, where $N_i$  are
  simple $G_{\circ}-K$-modules. For each  $N_i$ we fix a linearly
  independent basis $a,\ldots,a_is$ over $K$; we note here that the
  number $s$ is independent of $i$ because $M^\ast /M$ is a simple
  $G-K$-module (remember the well know proof of Lemma \ref{chap6:lem5}).  For each
  $(r,s)$-matrix $b = (b_{ij})$ over the module $M$, we define $N(b)
  = \sum_{ij}(a_{ij}+b_{ij}) K$.  Thus we have a one-one
  correspondence between all of $b$ and all of submodules $N$ such
  that $M^\ast = M \oplus N$  as a $K$-module.  We may assume, on the
  other hand, that $\rho_1$ is given by the linearly independent basis
  $a_{11}, \ldots , a_{rs}$  modulo $M$ of $M^\ast/M$.  Each $g
  \in G$ defines a linear transformation $f(g)$ on the module of
  $(r,s)$-matrices over $M^\ast$ as follows: If $(x_{11},
  \ldots,x_{rs})  \rho_1 (g) = (y_{11}, \ldots, x_{rs})$, then
  $(x_{ij}). f(g) = (x_{ij})$.  We define also an $(r,s)$ - matrix
  $c(g)$ over $M$ by the relation $N(c(g))= N(0)^g$.  If $b$ and  $b'$
  are such that $N(b)^g=N(b')$, then we have $(a_{ij}+b_{ij})^g =
  (a_{ij}+b'_{ij}). f(g)$. Since  $(a_{ij})^g = (a_{ij}+c(g)).f(g)$,
  we see that $b'=c(g) + b^g. f(g)^{-1}$. Thus: 
  $$
(1) \qquad \qquad  N(b)^g= N(c (g) + b^g.f(g)^{-1}).
  $$
  If we apply this formula to the case where $b=c(h)$  with $h
  \in G$, then we have 
$$
 (2) \qquad \qquad  c(hg)= c(g) + c(h)^g. f(g)^{-1}.
  $$

Now, let $g_1 \ldots, g_t$  be such that $G=\sum G_{\circ} g_i$ and we
set   $d=t^{-1}\break (\sum c(g_i))$.  We want to show that $N(d)^g = N(d)$
for every $g\in G$.  Indeed, $c(g) + d^g. f(g)^{-1} = c(g) +
t^{-1}(\sum c(g_i)^g. f(g)^{-1}) = c(g) + t^{-1} (\sum(c(g_i
g)-c(g))=t^{-1}(\sum c(g_i g))=d$.\pageoriginale  Therefore $M^\ast =
M \oplus N (d)$ is a representation module of $G$, which completes the
proof.  
\end{proof}

\begin{coro*}
  If $p = 0$, then the coverer, of the Corollary to Lemma
  \ref{chap6:lem5} is true,   provided that the index $[G:G_{\circ}]$
  is finite.  
\end{coro*}

\begin{lem}\label{chap6:lem7}%%% 7
  Let $H$ be a subgroup of finite index of $G$.  If a representation
  $\rho$ of  $H$ in $GL (n, K)$ is not completely reducible,  then the
  representation $\rho^\ast$ of  $G$ included by  $\rho$ is not
  completely reducible. 
\end{lem}

\begin{proof}
  Let $M$ be the representation module of $\rho$.  $M$ contains an
  $K-H$-module $N$ which is not a direct summand of $M$.  Let $M^\ast$
  be the representation module of $\rho^\ast$.  Then $M^\ast$ is of
  the form $M \oplus \sum M_{g_i}$ where $g_i$ are such that  $G = H +
  \sum H_{g_i}(g_i \notin H)$. It is obvious that $\sum M_{g_i}$ is
  $H$-admissible.  $M^\ast$ contains $N^\ast = N \oplus \sum N_{g_i}$.
  If $N^\ast$ is a direct summand of $M^\ast$ as a  $G-K$-module,
  then we have $M \oplus \sum M_{g_i}= N \oplus \sum N_{g_i} \oplus
  N'$ as an $H-K-\text { Module }$.  Then we see that $ M = N \oplus
  (M \cap ( \sum N_{g_i}+ N'))$ as an $H-K$-module, which is a
  contradiction. Hence $N^\ast$ is not a direct summand of $M^\ast$
  and $\rho^\ast$ is not completely reducible.   
\end{proof}

\begin{coro*}
  If a finite group $G^\ast$ has order which is divisible by  $p$,
  then  $G^\ast$ has a representation which is not completely
  reducible.
\end{coro*}

\begin{proof} 
  $G^\ast$ has an element a whose order is $p$.  Then the sub-group
  $\{a^i\}$ is represented by $\bigg\{\begin{pmatrix} 1 & i \\ 0 &
  1 \end{pmatrix}\bigg\}$, and we see the assertion by Lemma \ref{chap6:lem7}. 
\end{proof}

Next we observe relationship between rational representations of a
matric group $G$ and those of the closure of $G$. 

\begin{lem}\label{chap6:lem8}%%% 8
  Let\pageoriginale $G$ be a matric group and let $G^\ast$ be the
  closure of $G$.  Let 
  $\rho^\ast$ be a rational representation of $G^\ast$ and let $\rho$ be
  the restriction of $\rho^\ast$ on $G$.  Then $\rho$ is irreducible
  if and only if $\rho^\ast$ is irreducible.  $\rho$ is completely
  reducible if and only if $\rho^\ast$ is completely reducible.   
\end{lem}

\begin{proof}
  $\rho(G)$  is dense in $\rho^\ast (G^\ast)$ and we see the assertions easily. 
\end{proof}

\begin{lem}\label{chap6:lem9}%%% 9
  Let $N$ be a normal subgroup of a matric group $G$ and let $\rho$ be
  an irreducible rational representation of $G$ into $GL(n, K), K$
  being an universal domain.  If $N$ consists only of unipotent
  elements, then $N$ is contained in the kernel of the irreducible
  representation $\rho$. 
\end{lem}

\begin{proof}
  Since the set of all unipotent matrices in $GL(n, K)$ is closed, and
  since the image of unipotent element under a rational representation
  is again unipotent, the closure $N^\ast$ of $\rho(N)$ consists only
  of unipotent elements.  Therefore $N^\ast$ is nilpotent, hence is
  solvable.  Therefore we may assume that every element $(a_{ij})$ of
  $\rho(N)$ is such that $a_{ij}=0$ if $i > j$, whence $a_{ii}=1$
  for every $i$.  On the other hand, the Corollary to Lemma \ref{chap6:lem5} says
  that the restriction of $\rho$ on $N$ is completely reducible,
  whence $\rho (N)$ must consist only of the identity, which completes
  the proof. 
\end{proof}


\medskip
\noindent{\textbf{3. The main result in the case where $G$ is
  connected and $p \neq 0$.}}  

\setcounter{thm}{0}
\begin{thm}\label{chap6:thm1}%%% 1 
  Let $K$ be a universal domain of characteristic $p \neq 0$ and let
  $G$ be a connected matric group contained in $GL(n,K)$.  Then the
  following three conditions are equivalent to each other: 
  \begin{enumerate}[\rm (I)]
  \item Every rational representation of $G$ is completely reducible.

  \item $G$ is contained in a term group, i.e., there is an element a
    of $GL(n, K)$ such that $a^{-1} Ga$ is a subgroup of the diagonal
    group. 

  \item The\pageoriginale representation of $G$ by homogeneous forms
    of degree $p$ is completely reducible. 
  \end{enumerate}
\end{thm}

\begin{proof}
  It is obvious by virtue of Lemma \ref{chap6:lem8} that such of the above
  conditions for $G$ is equivalent to that for the closure of $G$.
  Therefore we may assume that $G$ is a connected algebraic linear group.
  It is well known that (II) implies (I) and it is obvious that (I)
  implies (III).  Thus we have only to show that (III) implies (II).
  Assume that (III) is true and that (II) is not true and we shall
  lead to a contradiction.  Lemma \ref{chap6:lem3} shows that $G$ contains an
  element $g$ which is not semi-simple.  Then the unipotent part $g_u$
  of $g$ is different from the identity and is contained in $G$
  (cf. Borel's paper ``Groupes leneaires algebriques, Ann.  of Math
  64, No.1 (1956) 20-82), hence $G$ contains a connected closed
  unipotent subgroup $U \neq 1$ by Lemma \ref{chap6:lem2}.  The representation
  module $F_p$ of the representation $_p$ of $G$ by homogeneous forms
  of degree $p$ is nothing but the module of homogeneous forms of
  degree $p$ in $n$ variables $X_1,\ldots, X_n$ on which element $g$
  of $G$ operates by the rule $h(X_{1}, \ldots, X_n)^g = h ((X_{1},
  \ldots, X_n)g)$.  $F_p$ contains $M=\sum X^{p}_{i} K$, which is also
  a representation module of $G$.  Hence (III) implies that $M$ is a
  direct summand of $F_p$.  Thus $F_p = N \oplus M$. For each monomial
  $n_{i_1\ldots i_{n}} = X^{i_1}_1 \ldots X^{i_n}_n$ with $i_j$ such
  that $i_j < p$ and $\sum i_j = p$, there is a uniquely determined
  element $m_{i_1\ldots i_n}$ of $M$ such that $f_{i_1\ldots i_n} =
  n_{i_1\ldots i_n} + m_{i_1\ldots i_n}$ form linearly independent
  basis for $N$.  We note that $N$ and $M$ are representation modules
  of $U$.  Hence we have only to show that: 
\end{proof}

The decomposition $F_p= N \oplus M$ as a
representation module of the connected closed unipotent group $U$lead
us to a contradiction. 

Let\pageoriginale $u=(u_{ij})$ be a  generic point of $U$ over the
universal domain $K$. We may 
replace $U$ with conjugate of $U$. Hence we may assume first that
$u_{ij}=0$ if $i>j,$ where $u_{ij}=1$ for $i$. set
$K^\ast=K(\bigg\{u^{p}_{ij}\bigg\})$, and we choose $(K, 1)$ that
$u_{k1}\notin 1^\ast , u_{ijj \in k^\ast , u_{ij} \in K^\ast }$ if
$i>K$ and such that $u_{kj}\in K^\ast$ if $j>1$. for each $A^{-1 }U A
$ (A being a triangular unipotent matrix), we can associate such $a
(k, 1)$ and we may assume that the pair $(k,1)$ for ${U}$ is
lexicographically smallest among those $(k, 1)$ for $A^{-1} U
A$. assume for a moment that is a linear relation $\sum_i \alpha_i
u_{ki}\in K^\ast$ with $\alpha_1 \in k$ and $\alpha_1\neq 0 $. We may
assume that $\alpha_1=1\text{ and that }\alpha_i=0$ if $u_{ki}\in
k^\ast $. Hence, in particular, $\alpha_1= \ldots = \alpha _k=
\alpha_{1+1}=\ldots  \alpha_n =0 $. Consider the unit matrix $1$ and
the matrix $c^{1}= (c^{1}_{ij})$ such that (i) $c^{1}_{ij}=0$ if $j
\neq 1$, (ii) $c'_{i1}=\alpha_i$ if $i \neq 1$ and (iii)
$c^{1}_{11}=0$. Set $c=1+c'$. Then obviously $c^{-1}= 1-c^1$. Since $
c^{-1}u \equiv u$ modulo $k^\ast$, We see easily that such a $(k,1)$
defined for $c^{-1}U c$ has the same $K$ and a smaller 1 than our
$(k,1)$, which is a contradiction. Therefore: 
$$
(1) \hspace{2cm} \text{If $  \alpha _i \in  K$ and if $\alpha_1 \neq
    0$, then $\sum i  \alpha_i u_{ki} \notin K^\ast$.}
$$


  Now, let $a = {(a_{ij})}$ be an arbitrary element of $U$. Then ua is
  also a generic point of $U$ over  $K$. Since  $u_{kj}(j>l)$ is in
  $K^\ast$, the $(k, j)$. component of ua must be in $K^\ast$. This
  shows by virtue of (1) above that $ a_{lj}=0$ if $j>1$. since a is
  arbitrary, we see that $u_{lj}=0$ for every $j \neq l$.Thus $X_1$ is
  $U$-invariant. Now we consider the elements $ f_{i_{1}} \ldots i_n
  (i_j < p, \sum i_j =p)$. We denote by ${g_j}$ the element $
  {f_{i_{l}} \ldots i_n}$ such that ${i_j=1, i_1=p^{-1}}$ for each $
  {j=k, k+1, \ldots , 1-1, 1+1, \ldots , n}$.  Since we have 
  $$ 
(2) \hspace{2cm}  {(X_k X^{p-l}_{l})^u= \sum u_{kj}Z_j X^{p-l}_{l}},
  $$\pageoriginale
  we see that
  $$ 
(3) \hspace{2cm}  {g^u_k = \sum_{j \neq l} u_{kj}g_j}.
  $$

Consider the coefficient of ${Z^p_1}$ in ${g^u_k};$ let it be $d$. (3)
shows that d is a linear combination of ${u_kj(j \neq l)}$ with
coefficients in $K$. On the other hand, (2) shows that ${d-u_{kl}}$
must be in $K^\ast$. Thus we have a contradiction to (1) above, which
completes the proof of Theorem \ref{chap6:thm1}.  


\medskip
\noindent{\textbf{4. The main result in the case where $p \neq 0$.}}


\begin{thm}\label{chap6:thm2}%%% 2
  Let $K$ be a universal domain of characteristic $p \neq 0$ and let
  $G$ be a matric group contained in $GL(n ,k)$. Then the following
  conditions are equivalent to each other: 
  \begin{enumerate}[(I)]
  \item Every rational representation of $G$ is completely reducible.

  \item  There is a normal subgroup $G_o$ of finite index such that
    (i) $G_o$ is a subgroup of a torus group and (ii) the index of
    $G_o$ in  $G$ is not divisible by $p$. 

  \item The connected component $G_o$ of the identity of $G$ is a
    subgroup of a torus group and $[G: G_o]$ is not divisible by $p$. 
    
    If $G$ is an algebraic linear group, then the above conditions
    are equivalent to the following condition: 

  \item Every element of $G$ is semi-simple.
  \end{enumerate}
\end{thm}

\begin{proof}
  It is obvious that (III) implies (II) and that (II) implies
  (I) by virtue of Lemma \ref{chap6:lem6}. Therefore, by Lemma
  \ref{chap6:lem8}, we have only 
  to prove the equivalence of (I), (III), (IV) in the case  where $G$
  is an algebraic linear\pageoriginale group. Thus we assume that $G$
  is algebraic 
  let $G_o$ be the connected component of the identity of $G$. Assume
first  that (IV) is true. Then $G_o$ consists merely of semi simple elements,
hence $G_o$ is a torus group by Lemma \ref{chap6:lem3}. If a semi-simple a has
  a finite order, then the order is prime to $p$. Therefore Lemma
  \ref{chap6:lem4} 
  implies that$[G: G_o]$ is not divisible by $p$.  Thus (IV) implies
  (III). As we have remarked above, (III) implies (I). Assume
  now that (IV) is not true. Then, as we have seen in the proof of
  Theorem \ref{chap6:thm1}, there is a unipotent element $u$ of $G$ which is
  different from the  identity, If $u \in G_o$, then $G_o$ has a
  rational representation which is not completely reducible, hence $G$
  itself has such one by Lemma \ref{chap6:lem7} or by the corollary to Lemma
  \ref{chap6:lem8}. If $u \notin G_o$, then the finite group $G/G_0$ has a
  representation which is not completely reducible, which is a
  rational representation of $G$. Thus we see that (I) is not
  true. Therefore (I) implies (IV), which completes the proof of
  Theorem \ref{chap6:thm2}.   
\end{proof}

\medskip
\noindent{\textbf{5. The main result in the case where $p=0$.}}

\begin{thm}\label{chap6:thm3}%%% 3
  Let $K$ be a universal domain of characteristic $p=0$ and let $G$
  be a matric group contained in $GL(n,k)$. Then the following
  conditions are equivalent to each other: 
\begin{enumerate}[(I)]
\item Every rational representation of $G$ is completely reducible.

\item The closure of $G$ has a faithful rational representation which
  is completely reducible. 

\item  The radical of the closure of $G$ is a torus group.
\end{enumerate}
\end{thm}

\begin{proof}
  It is obvious that (I) implies (II) by virtue of Lemma
  \ref{chap6:lem8}. Lemma \ref{chap6:lem9} shows that (II) implies
  (III). In order to show 
  that (III) implies (I),  we shall prove the following lemma: 
\end{proof}

\begin{lem}\label{chap6:lem10}%% 10
  Let\pageoriginale $G$ be a connected algebraic linear group and let
  $R$ be the radical 
  of $G$. If $R$ is a torus group, then there is a closed connected normal
  subgroup $S$ such that (i) $G= RS$ and (ii) $R \cap S$ is a finite
  group. Furthermore, $R$ is contained in the center of $G$ (hence $R$
  is the connected component of the identity of the center of $G$). 
\end{lem}

\begin{proof}
  For the fact that $R$ is contained in the center of $G$, see Borel's
  paper. Let $S$ be the subgroup generated by all unipotent elements
  of $G$. Then $S$ is obviously a normal subgroup. Each unipotent
  element is in a closed connected unipotent subgroup of $G$, hence
  $S$ is generated by closed connected subgroups, and therefore $S$ is
  a closed connected subgroup of $G$. Now, we may assume that $R$ is a
  diagonal group and that each $g \in G$ is given by  
  $$
  g=
  \begin{pmatrix}
    \rho_1(g) & \tau_{12} (g)\cdots & \tau_{1r}(g)\\
    0 & \rho_2(g) \cdots & \tau_{2r} (g)\\
    &...................................&\\
    0 &....................&\rho_r(g)
  \end{pmatrix}
  $$
with irreducible representations ${\rho_1, \ldots, \rho_r}$. If ${u
    \in G}$ is unipotent, then $\rho_i(u)$ is unipotent, whence the
  determinant of $ {\rho_i(u)}$ is 1. Therefore we see that if $s
  \in S$, then the determinant of $\rho_i (s)$ is $1$. On the other
  hand, since $R$ is in the center of $G, \rho_i (R)$ is in the center
  of $\rho_i (G)$, hence by the famous lemma of Schur every element of
  $\rho_i(R)$ is of the form $k.\rho_i(1)$ with $k \in K$. Therefore
  we see that $R \cap S$ is a finite group. Since $S$ a is closed
  normal subgroup. $RS$ is a closed normal subgroup. Since $G/R$ is
  semi- simple, we see that $G/RS$\pageoriginale is semi-simple,
  unless $G= RS$. If 
  $G \neq RS$, then $G/RS$ contains a non-trivial unipotent element,  whence
  there must be a unipotent element of $G$ outside of $RS$, which is a
  contradiction to our construction of $S$. Therefore $G=RS $, which
  completes  the proof. 
  
  Now we proceed with the proof of Theorem \ref{chap6:thm3}. By the Corollary to
  Lemma \ref{chap6:lem6},  we may assume that $G$ is connected.  Lemma
  \ref{chap6:lem8} allows 
  us to assume that $G$ is an algebraic linear group. Let $R$ be the
  radical of $G$ and let $S$ be the normal subgroup given in Lemma
  \ref{chap6:lem10}. Since $R \cap S$ is a finite group and since $G=RS$, we see
  that $S$ is semi-simple, whence every rational representation of
  $S$ is completely reducible. Let $\rho$ be an arbitrary rational
  representation of $G$. We may assume  that $\rho(R)$ is a diagonal
  group, whence the completes reducibility of the restriction of $\rho$
  on $S$ implies the complete reducibility of $\rho$, which completes
  the proof. 
\end{proof}

\eject

\medskip
\noindent{\textbf{6. Another result.}}

Let $G$  be a connected algebraic linear group with  universal domain
$K$, throughout this section. 

\begin{thm}\label{chap6:thm4}%%% 4
  Every rational representation of $G$ is completely reducible if (and
  only if) the following is true: 
  
  If $ \rho''=\begin{pmatrix}
  1 & 0 \\0 & \rho'
  \end{pmatrix} $
  is a rational representation of $G$, then $\rho''$ is equivalent to
  the representation  
  $\begin{pmatrix}
    1 & 0 \\0 & \rho'
  \end{pmatrix}$.
\end{thm}

\begin{proof}
  Let$ \rho=
  \begin{pmatrix}
    \rho _1 & \tau \\0 & \rho_2
  \end{pmatrix}$
  be a rational representation of $G$. We have only to show that
  $\rho$ is equivalent to the representation  
  $\begin{pmatrix}
    \rho_1 & 0\\ 0 & \rho_2
  \end{pmatrix}$.
\end{proof}
    
Since\pageoriginale quad $\rho (ab) = \rho(a) \rho (b)$, we have
\begin{enumerate}[(1)]
\item $$ 
\tau (ab)= \rho_1 (a) \tau(b) + \tau(a) \rho_2 (b) \text{ for
  any } a, b-G.
 $$
  Let $x$ be a generic point of $G$ over $K$ consider $f(x)= T(x)$ 
$\rho _2(x)^{-1}$. $f(a)$ is then well defined for any $a \in G$. The
  relation (1) implies that $f(ab) =\rho_1(a) \tau (1) \rho_2 (b) ^{-1}
  \rho _2 (a) ^{-1} +  \tau(a) \rho_2 (a)^{-1}=\rho_1 (a) f(b) \rho_2 (a)^{-1}
  +f(a)$ for any $a, b \in G$, whence 
  
\item $$ f(xa) =\rho_1 (x) f(a) f_a (x) ^{-1} +f(x)   \text{ for any }
  a\in G.$$ 

  Let $m,n$ be such that $T$ is an(m,n) matrix and consider the module
  $L$ of all $(m,n)$-matrices over $K(x)$. Each element $C$ of
  $G$ defines an $K$-linear map $ \phi_{g} $ or. $L$ as follows: 
  $$ 
  \phi_g (w_{ij}(x)) = (w_{ij}(xg)).
  $$
  
  Thus $L$ becomes  $K-G$-module. Let  $M$ be the set of
  all  $\rho_1(x)c \rho_2\break (x)^{-1}$ with $(m, n)$-matrices
  $c$ over $K$. Then $M$ is a finite $K$-module contained in $L$.
  Since $\rho_1 (xa) c \rho_2 (xa)^{-1}= \rho_1 (x) (\rho_1
  (a) c \rho_2 (a)^{-1}$, $x \rho_2\break (x)^{-1} (a \in G)$, $M$ is
  $G$-admissible. Set $N=f(x) K+M$. Then the relation (2) shows 
  that $N$ is also a finite $K-G$-module. We consider a
  representation $\rho^* $ of $G$ by the module $N$.  The relation
  (2) shows  that $f(x)$ is $G$-invariant modulo $M$, hence either
  $f(x) \in $ $M$ or $\rho^* $ is equivalent to a representation of
  the form  $\begin{pmatrix} 1 & \lambda \\ 0 &
    \rho' \end{pmatrix}$. The former case implies that $f(x) + \rho_1
  (x) c \rho (x) ^{-1}=0$ with some $(m, n)$-matrix $c$  over
  $K$. By our assumption, the latter case implies that there is an
  element $ \rho _1(x)c \rho _2 (x)^{-1}$  of $M$ such that
  $f(x)+\rho_1(x)_c \rho_2(x)^{-1}$ is
  $G$-invariant. Hence,\pageoriginale in any 
  case, there is an $(m,n)$-matrix $c$ over $K$ such that $f(x)+
  \rho _1 (x) c \rho_2 (x) ^{-1}$ is $G$-invariant. Set $\tau^* = \tau
  -c  \rho_2 +\rho_1 c$.
  Then, transforming $\rho $ by the matrix $\begin{pmatrix} \rho_1(1)
    & c \\0 & \rho_2(1)\end{pmatrix}$,  we see that $\rho $ is
  equivalent to the representation $\begin{pmatrix} \rho_1 & \tau^*\\ 0 &
    \rho_2 \end{pmatrix}$. Set $f^* (x) = \tau^*(x) \rho_2 (x)^{-1}$.  
  Then $f^*(x) =f(x) -c + \rho _1 (x) c \rho_2 (x)^{-1}$, which is
  $G$-invariant by our choice of $c$. Therefore $f^*(xa) =f^*(x) $ for
  any $a \in G$, whence $f^*(x)=f^*(xx^{-1})=0$. This shows that
  $\tau^*=O$, which completes the proof of Theorem \ref{chap6:thm4}. 
\end{enumerate}

  We note by the way that the matrix $f(x)$ has an interesting
  property as follows: 

\begin{prop*}
  Assume that $ \rho = \begin{pmatrix} \rho_1 & \tau\\ 0 &
    \rho_2 \end{pmatrix}$ is a rational representation of $G$. Set $H
  =\{ h|h \in G, \tau (h)=0\}$. Then the homogeneous variety $G/H$ = $ \{ 
  gH\}$ is a quasi-affine variety, on which the coordinates of a
  point $gH$ are given by $f(g)$. 
\end{prop*}

\begin{proof}
  Since $\tau(1) =1$, the formula (1) in the above proof shows that $ \tau
  (a^{-1}) =-\rho _1 (a)^{-1} \tau(a) \rho _2(a)^{-1}$, hence $\tau (a^{-1}b)
  = \rho_1(a)^{-1} \bigg[\tau(b)\rho_2 (b) ^{-1}-\tau(a) \rho
    _2(a)^{-1}\bigg] \rho_2 (b)$. There fore $f(a)=f(b)$ if and only
  if $aH=bH$, which prove the assertion. 
\end{proof}

\begin{remark*}
  Note that the above preposition only proves that $G/H$ is a
  quasi-affine and not affine as stated in M. Nagata: Complete
  reducibility of rational representations, J. Math. hyeto Univ.,
  1-1 (1961), 87-99. 
\end{remark*}
