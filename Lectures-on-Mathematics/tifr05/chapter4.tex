\chapter{Norm and Trace}\label{chap4}%cha 4

\section{Norm and trace}\label{c4:s1}\pageoriginale%\section 1.
 Let $ K/k $ be a finite extension and let $ \omega_1 , \ldots ,
 \omega_n $ be a base of $ K/k $ so that every $ \omega \in K $ may be
 written  
 $$
 \omega = \sum_{i} a_i \omega_i
 $$
 $a_i \in k $. By means of the regular representation 
 $$
 \omega \rightarrow A_\omega  
 $$
 where $ A_\omega = ( a_{ij} ) $ is an $n-$ rowed square  matrix with  
 $$
 \omega \omega_i = \sum_{j} a_{ij} \omega_j \qquad i = 1, \ldots , n  
 $$
 the field $ K $  becomes isomorphic to the subalgebra formed by $
 A_\omega $ in the algebra $ \mathfrak{m}_n (k) $ of $n$ rowed
 matrices over $k$. We denote by $ N_{K/k} \omega$, $S_{K/K}\omega $
 the  \textit{norm} and \textit{trace} respectively of  $ \omega \in K
 $ over $k$ and  they are defined by  
 \begin{align*}
N_{K/k} \omega &=\left| A_\omega \right|\\
S_{K/k} \omega &= \text{trace }   A_\omega.\\
 \end{align*} 
 
 Defined as such, it follows that 
 \begin{align*}
& N_{K/k} \omega \omega'  = N_{K/k} \omega \; . \;  N_{K/k} \omega' \\ 
& S_{K/k} ( \omega + \omega' ) = S_{K/k}  \omega +  S_{K/k}  \omega' \\
 \end{align*}
  for $\omega$, $\omega' \in K $. Obviously $ \omega \rightarrow
 N_{K/k} \omega $ is a homomorphism of $ K^* $ into $K^*$  and
 similarly $ \omega \rightarrow S_{K/k} \omega $ is a homomorphism of $
 K^+ $, the additive group, into $k^+ $. 
 
 Let $ \omega'_1 , \ldots , \omega'_n $ be any other basis of  $ K/ k
 $. Then  
 $$
 \begin{pmatrix} \omega'_1 \\ \vdots \\ \omega'_n\end{pmatrix} =
   P \begin{pmatrix} \omega_1 \\ \vdots \\ \omega_n\end{pmatrix} 
 $$ 
 where\pageoriginale  $P$ is a non-singular matrix in $ \mathfrak{m}_n
 (k) $. Since   
 $$
 \omega \begin{pmatrix} \omega_1 \\ \vdots \\ \omega_n\end{pmatrix} =
   A_\omega \begin{pmatrix} \omega_1 \\ \vdots
     \\ \omega_n\end{pmatrix} 
 $$ 
 it follows that  
 $$
  \omega  \begin{pmatrix} \omega'_1 \\ \vdots
    \\ \omega'_n\end{pmatrix} =  P A_\omega' P^{-1}  \begin{pmatrix}
      \omega'_1 \\ \vdots \\ \omega'_n \end{pmatrix} 
 $$ 
  which shows  that by means of the new basis the matrix associated to
  $ \omega $ is $ B_\omega $ where 
  $$
  B_\omega =  P A_\omega P-1 \\
  $$  
  and then we have 
  \begin{align*}
  \mid B_\omega \mid &= \left| A_\omega \right| \\
  \text{ Trace } B_\omega &= \text{ Trace } A_\omega .
  \end{align*}
  
  This shows that $ N_{K/k} \omega $  and $ S_{K/k} \omega $ are
  invariantly defined and do not depend on a basis of $ K/k $. 
  
  We write 
  $$
  f_{K/k} (x)  = \mid x E - A_\omega \mid 
  $$
  and call it the \textit{characteristic polynomial} of $ \omega 
  $. Obviously  $ f_{K/k} (0) = (-1)^n \mid A_\omega \mid $ so that  
  \begin{equation*}
N_{K/k}  \omega = (-1)^n  f_{K/k}  (0) = (-1)^n a_n . \tag{1}\label{c4:eq1}  
  \end{equation*}  
  
  We also see easily that 
  \begin{equation*}
S_{K/k} \omega = -a_1 \tag{2}\label{c4:eq2}  
  \end{equation*}  
  where
  $$
  f_{K/k} (x)  = x^n + a_1 x^{n-1}+ \cdots + a_n  
  $$
\noindent
$a_1 , \ldots , a_n  \in k$. 

Let $k \subset L \subset K$  be a tower of finite extensions. Let
$(K:L) = m$ and let $\Omega_1, \ldots, \Omega_m$ be a basis of $K/L
$. Similarly let\pageoriginale $(L:k)=n$ and let $ \omega_1 , \ldots , 
\omega_n$ be a base of $ L/k $. Then $(\omega_1 \Omega_1 , \ldots
, \omega_n \Omega_m) $ is  a base of $ K/k $. Let $ \omega \in L $
and consider the  matrix of $ \omega $ by the regular representation
of $ K/k $  in terms of the base  $ ( \omega_1 \Omega_1 , \ldots ,
\omega_n \Omega_m ) $. Call it $ \bar{A}_\omega $. 

Then it is  trivial to see that 
$$
\bar{A}_\omega   = \begin{pmatrix} A_\omega  \quad  0 \\  \cdots
  \cdots \cdots  \\  \cdots \cdots \cdots \\  0  \quad
  A_\omega\end{pmatrix} 
$$ 
a matrix of  $mn$ rows and columns. Therefore 
\begin{align*}
N_{K/k} \omega &= \mid \bar{A}_\omega \mid ( N_{K/k} \omega ) ^{ (K:L) } \\
S_{K/k} \omega  &= ( K : L )  S_{K/k} \omega\\
\end{align*}

Also the characteristic polynomials of $ \omega $ as belonging to $L$
and to $K$ respectively are $ f_{L/k} (x) $ and $ f_{K/k}   (x) $ and
they are  related by  
\begin{equation*}
f_{K/k}   (x) = (f_{L/k} (x))^{(K : L)} , \tag{3}\label{c4:eq3}
\end{equation*}

In particular let $ L = k(\omega)$. Then $f_{L/k} (x)$ is the
minimum polynomial of $ \omega $. We, therefore, have the

\setcounter{thm}{0}
\begin{thm}\label{c4:thm1}%\theorem1
If $ K/k $ is a finite extension and $ \omega \in K $, $ \varphi
   (x) $ its minimum polynomial over $k$ and $f (x) $ its
   characteristic polynomial, then 
 $$
 f (x) = ( \varphi (x)  )^r
 $$
where $ r =  (K:k (\omega) ) $.
 \end{thm} 
 
 From our formulae above, it follows that we can compute the norm and
 trace of $\omega$ in $K$ from a knowledge of its minimum
 polynomial. 
 
 Let\pageoriginale now $ K/k $ be a finite extension and $ \omega$ an
 element of $K$. Let $[K: k (\omega)] = m'$, $[k(\omega):k] = n'$ be
 the degrees of separability of $K$ over $k(\omega)$ and  $ k (\omega)
 $ over $k$ respectively. Then $K/k$ has $m' n' $ distinct
 $k$-isomorphisms in an algebraic  closure $ \Omega $ of $k$. Let $ \left\{
 \sigma_{ij} \right\}$, $\begin{smallmatrix} i = 1, \ldots , n'\\ j = 1,
   \ldots , m' \end{smallmatrix} $, be these isomorphisms and let notation
 be  so chosen  that  $ \sigma_{i 1} , \ldots , \sigma_{i m} $, have
 the same the same effect on $ k (\omega) $. Then we may take $
 \sigma_{1 1} , \sigma_{1 2} , \ldots , \sigma_{n' 1} $ as  a
 complete  system of distinct isomorphisms of $ k (\omega) / k $ in $
 \Omega $.  
 
 By our considerations above $ f_{k (\omega) /k} (x) $ is the
 polynomial of $ \omega $ as  well  as the characteristic polynomial
 of $ \omega $ in $k (\omega) / k$. If $f_{K/k} (x)$ is the
 characteristic polynomial of $\omega$ in $k$, then 
  \begin{equation*}
f_{K/k} (x) = ( f_{k (\omega)/k }(x) ) ^{( K : k (\omega ) )} \tag{4}\label{c4:eq4}
 \end{equation*} 
 
 Now, because of the properties of the isomorphisms $\sigma_{ij}$
 \begin{equation*}
\prod^{n'}_{i=1} \prod^{m'}_{j=1} (x-\sigma_{ij} \omega ) =
\prod^{n'}_{i=1} (x-\sigma_{i1} \omega )^{\mathfrak{m'}}.
\tag{5}\label{c4:eq5} 
 \end{equation*} 
 
 But since  $ f_{k (\omega)/k } (x) $ is the minimum polynomial of $
 \omega $, we have    
 $$ 
 f_{k (\omega)/k} (x)   = \left\{ \prod^{n'}_{i =1} (x- \sigma_{i1} 
 \omega ) \right\}^{\{ k (\omega) : k\}}  
 $$
 where $ \{k(\omega):k\} $, as usual, denotes the degree of
 inseparability of $k(\omega)$ over $k$. Using~\eqref{c4:eq5} we get  
 $$ 
\left \{ \prod_{i,j} (x-\sigma_{ij}(\omega) \right\}^{\{ k:k\}}  = \left\{
 \prod^{n'}_{i =1} (x-\sigma_{i1}\omega)\right\}^{\mathfrak{m}' \{K
   : k\}} .
 $$
 
 But $ \{ K : k\} =  \{ K : k (\omega) \} \cdot \{ k(\omega) : k \} $
 so that 
 $$
 \left\{ \prod_{i,j} ( x - \sigma_{ij}\omega) \right\}^{\{
   K:k \}} = \left\{f_{ k(\omega)/k}(x) \right\}^{m' \{ K : k(\omega)\}}   
 $$\pageoriginale
which proves that 
\begin{equation*}
\begin{fbox}{$f_{K/k} (x) = \left\{ \prod \limits_{\sigma} ( ( x - \sigma
    \omega ) \right\} ^{\{ K : k\}}$}\end{fbox} \tag{6}\label{c4:eq6}    
\end{equation*}
 where $ \sigma $ runs through all the distinct isomorphisms of
 $k(\omega)$  in $ \Omega $. Using \eqref{c4:eq1} and \eqref{c4:eq2} we get   
 \begin{equation*}
 N_{K/k} \omega = \left\{ \prod_{\sigma} \omega^\sigma \right\}^{\{ K :
   k\}} \tag{7}\label{c4:eq7}   
 \end{equation*}
 $ \omega^\sigma $ is a conjugate of $\omega$. Similarly  
 \begin{equation*}
S_{K/k} \omega = \left\{ K:k \right\} \sum_{\sigma} \omega^{\sigma}
. \tag{8}\label{c4:eq8}    
 \end{equation*} 
 
 If $ K/k $ is inseparable, then $ \{ K: k\} = p^t$, $t \ge 1 $ so that
 for every $ \omega \in K $ 
 $$
 S_{K/k} \omega = o .
 $$

On the other hand, suppose $ K/k $ is finite and separable. Let $
\sigma_1 ,\break \ldots , \sigma_n $ be all the distinct isomorphisms of $
K/k $ in  $ \Omega $, an algebraic closure of $K$. Then $ n = ( K : k )
$ and since $ \sigma_1 , \ldots , \sigma_n $ are independent  $k$-linear
functions of $ K/k $ in $ \Omega $, it follows that  
$$
\sigma = \sigma_1 + \cdots + \sigma_n 
$$
is a non - trivial $k-$linear function of $ K/k$ in $ \Omega
$. Therefore there exists  a  $ \omega \in K $ such that $ \sigma
\omega \neq 0 $. But by formula \eqref{c4:eq8}, 
$$
\sigma \omega = \omega^{\sigma_1} + \ldots + \omega^{\sigma_n} =
S_{K/k} \omega 
$$
so that we have the 

\begin{thm}\label{c4:thm2}%theorem 2
 A finite extension $ K/k $ is separable, if and only if there exist
 in $K$ an element whose trace over $k$ in not zero.  
\end{thm}

In case\pageoriginale $k$ has characteristic zero, or $k$ has
characteristic $ p \not\mid n = ( K: k ) $, the unit element 1 in $k$
has trace $ \neq o $. In order  to obtain an element $ \omega $ in $ K
$ with $ S_{K/k} \omega \neq 0 $, in every case we proceed thus:   

Let $ K/k $ be separable and $ K= k (\alpha) $ for an element
$\alpha $. Let $ \varphi (x) $ be its irreducible polynomial over $k$
and  
$$
\varphi (x) = ( x - \alpha_1 ) \cdots ( x - \alpha_n ).
$$

It follows then that 
$$
\frac{x^{n-1}}{\varphi (x)} = \sum_{i}
\frac{\alpha^{n-1}_{i}}{\varphi' (\alpha_i)}   \frac{1}{x
  -\alpha_i} . 
$$

Comparing coefficients of $ x^{n-1} $ on both sides we get  
$$
\sum_{i} \frac{\alpha^{n-1}_{i}}{\varphi' (\alpha_i)} = 1 .
$$

If we put $ \omega = \dfrac{\alpha^{n-1}}{\varphi' (\alpha)} $ and
observe that $ \varphi' (\alpha) \in K $, we get  
$$
S_{K/K} \omega = 1 .
$$

Using formula \eqref{c4:eq6}, it follows that if $ k\subset L\subset K $ is a
tower of finite extensions and $ \omega \in K $, then 
\begin{align*}
N_{K/k} \omega &= N_{L/k}( N_{K/L}\omega ) \\
S_{K/k} \omega &= S_{L/k} ( S_{K/L} \omega ) \\
\end{align*}

We now give a simple application of formula \eqref{c4:eq7} to finite fields.

Let $k$ be a finite field of $ q = p^a $ elements so that $p$ is the
characteristic of $k$. Let $K$ be a finite extension of  $k$ so that $
(K:k) = n $. Then $K$ has $q^n$  elements. The galois group of $ K/ k$
is cyclic of order n. Let $\sigma$ be  the Frobenius
automorphism. Then for $ \omega \in K $, 
$$
\omega^\sigma = \omega^q .
$$\pageoriginale

The norm of $ \omega $  is 
\begin{align*}
N_{K/k} \omega &=  \omega \cdot \omega^\sigma \cdot \omega^{\sigma^{2}} \cdots
\omega^{\sigma^{n-1}} \\ 
&= \omega^{\dfrac{q^n-1}{q-1}}.
\end{align*}

Since $K^*$  has $ q^n -1 $ elements,
$$
\alpha^{q^{n}-1} = 1
$$
for all $ \alpha \in K^* $. Since $ K^* $  is a cyclic group, the
number of elements in $ K^* $ with 
$$
\alpha^{\dfrac{q^n-1}{q-1}} = 1
$$
is precisely $ q^n -1 /q-1 $, since $q - 1$ divides $ q^n-1 $. 

Now $ \omega \rightarrow N_{K/k} \omega $ is a homomorphism of $ K^* $
into  $k^* $ and the kernel of the homomorphism is the set of $ \omega $
in  $ K^* $ with 
$$
1= N_{K/k} \omega = \omega^{\dfrac{q^n-1}{q-1}}  . 
$$

By the first homomorphism theorem we have, since $k^*$ has only $ q-1 $
elements, the  


\begin{thm}\label{c4:thm3}%theorem 3
 If $ K/k $ is finite and $k$ is a finite field, then every
  non-zero element of $k$ is the norm of exactly $ ( K^* : k^* ) $
  elements of $K$.  
\end{thm}

It is clear that this theorem is not in general true if $k$ is an
infinite field. 


\section{Discriminant}\label{c4:s2}%sec 2

Let\pageoriginale $K/ k$ be a finite extension and $\omega_1, \ldots ,
\omega_n$ a basis of $K/k$. Suppose  $\sigma$ is a $k$-linear map of $K$
into $k$, that is   
\begin{align*}
 & \sigma (\omega ) \in k\\
 & \sigma (\omega +  \omega ')  = \sigma \omega + \sigma \omega ' \\
 & \sigma (\lambda) = \lambda \sigma (\omega) 
\end{align*}
where $\omega$, $\omega ' \in K$, $\lambda \in K$. Let $M^\sigma$ denote
the matrix  
$$
M^\sigma = (\sigma (\omega_i \omega j)) 
$$
of $n$ rows and columns. We denote by $D_{K/k}\sigma (\omega_1 ,
\ldots , \omega _n)$ its determinant and call it the $\sigma$-
\textit{discriminant of the basis} $\omega _1, \ldots , \omega _n $ 
of $K/k$. If $\omega'_1, \ldots , \omega'_n$ is another basis, then   
$$
\begin{pmatrix}
\omega'_1 \\
\vdots \\
\omega'_n 
\end{pmatrix}
= P 
\begin{pmatrix}
\omega_1 \\
\vdots \\
\omega_n 
\end{pmatrix}
$$
where $P$ is an $n$ rowed non-singular matrix with elements in $k$. 

If $P = (p_{ij})$ then 
$$
\omega' _i = \sum_j p_{ij} \; \omega_j 
$$
so that 
$$
\sigma (\omega'_a \omega'_b) = \sum_{ i , j} p_{ ai }p_{bj} \sigma
(\omega _i \omega_j ) 
$$
which proves that 
$$
D^\sigma_{K/k} (\omega' _1 , \ldots , \omega'_ n ) = |P| ^2 D^\sigma
_{K / k} ( \omega _1, \ldots , \omega_n )  .
$$

Therefore $D_{ K /k}^\sigma = 0$ if it is zero for some basis.

We now prove 

\begin{thm}\label{c4:thm4}%theorem 4
 If $\omega _1 , \ldots , \omega_n$  is a basis of $K/k$ and
   $\sigma $ a $k$-linear map of $K$ into $k$, then  
 $$
 D^\sigma_{ K/k} (\omega_1 , \ldots, \omega_n ) = 0 
 $$
if and\pageoriginale only if $\sigma$ is the zero linear mapping.  
\end{thm}

\begin{proof}
If $\sigma$ is the zero linear map, that is, one that assigns to
every element $\omega$ in $K$, the zero element, then $D_{K/K}^\sigma
= 0 $. Now let $D_{K/k}^\sigma = 0$. This means that the matrix
$M^{\sigma}$ with elements in $k$ has determinant zero. Therefore there
exist $a_1, \ldots, a_n$ in $k$, not all zero, such that  
$$
M^\sigma \begin{pmatrix}  a_ 1 \\ \vdots \\ a_n \end{pmatrix}
=  \begin{pmatrix}  0 \\ \vdots \\ 0 \end{pmatrix} .  
$$

This means that 
$$
\sum^n_{ j = 1} \sigma (\omega_i \omega_j )a_j = 0 ; \;\; 1=1, \ldots , n .  
$$

If we put $z = \sum\limits_j a_j \omega_j $. then we have  
$$
\sigma (\omega _i z ) =  0 , i = 1 , \ldots , n . 
$$
\end{proof}

Let $\omega$ be any element in $K$. Since $a_1 , \ldots , a_n$ are not
all zero, $z \neq 0$ and so put  
$$
\frac{\omega}{z} = b_1 \omega_1 + \cdots + b_n \omega_n,  b_i \in k. 
$$

Then $\sigma (\omega) = \sigma \left(\dfrac{\omega}{z} \cdot z \right) 
= \sum\limits_i b_i \sigma (\omega _i z) = o $. This proves that
$\sigma$ is the trivial 
map.  

The mapping $\omega \to S_{K/K} \omega $ is also a $k$-linear map of $K$ 
into $k$. For a basis $\omega_1 , \ldots, \omega_n$ of $K/k$ we call  
$$
D_{K/k}(\omega_1 , \ldots \omega_n) = | (S_{k /K} (\omega _i \omega_j)) | 
$$
the \textit{discriminant of the basis} $\omega _1 , \ldots
\omega_n$. Using theorem~\ref{c4:thm2} and theorem~\ref{c4:thm4}, we get   

\begin{thm}\label{c4:thm5}%theorem 5
Discriminant of a base of $K/k$ is not zero if and only if $K/k$ is
separable. 
\end{thm}

Let\pageoriginale $K/k$ be finite separable. Let $\sigma_1 , \ldots ,
\sigma _n$ be the distinct $k$-isomor\-phisms of $K$ over $k$ in
$\Omega$. Then   
$$
S_{K/k} \omega = \sum_i \omega^{\sigma_i}.
$$

Therefore 
\begin{equation*}
D_{ K /k} (\omega _1 , \ldots , \omega _n) = 
\begin{vmatrix} 
\omega_1 ^{\sigma _1}, & \omega_1 ^{\sigma _2}, & \ldots, & \omega_1
^{\sigma  _n} \\ 
& \vdots\\
\omega_1 ^{\sigma _1},&   \ldots &  \ldots, & \omega_1 ^{\sigma _n}  
\end{vmatrix}^2. \tag{9}\label{c4:eq9}  
\end{equation*}

Since $K/k$ is finite separable, $ K = k (\omega)$ for some
$\omega$. Also $1, \omega, \omega^2,\break \ldots , \omega^{ n -1}$ form a
base of $K/k$ and we have  
$$
D(1, \omega , \ldots , \omega^{ n - 1}) = 
\begin{vmatrix}
1, & \ldots , & 1 \\
\omega^{ \sigma_1}, & \ldots , & \omega^{ \sigma _1} \\
\hdotsfor{3} \\
(\omega^{ n - 1} )^{ \sigma_1}, &  \ldots, &  (\omega ^{ n -
  1})^ {\sigma_n} 
\end{vmatrix}.
$$

Also, $D( 1, \omega , \ldots , \omega^{ n -1 })\big / D(\omega_1,
\ldots , \omega_n)$ is the square of an element of $k$. The
determinant  
$$
\begin{vmatrix}
1, &\ldots , & 1\\
\omega^{\sigma_1} ,& \ldots, &\omega^{\sigma_n}\\
\hdotsfor{3}\\
(\omega^{n-1})^{\sigma_n} ,& \ldots ,  &(\omega^{ n-1})^{\sigma_n}
\end{vmatrix}
$$
is the called Van-der-Monde determinant. We call $D(1, \omega ,
\ldots , \omega^{ n -1})$ is the \textit{discriminant} of $\omega$
and denote it by $D_{ K/k} (\omega)$.  

Let $f(x)$ be the minimum polynomial of $\omega$. Then $f(x) = (x -
\omega^{\sigma_1}) \ldots (x - \omega^{\sigma_n})$. Since
$f'(\omega) \neq 0$ , we have  
$$
f' (\omega) = (\omega^{\sigma_1} - \omega^{\sigma_2}) (\omega^{\sigma_1}
- \omega^{\sigma_3} ). \ldots . (\omega^{\sigma_1} - \omega^{\sigma_n} ) 
$$
and is an element of $K$. We call it the \textit{different} of
$\omega$ and denote it $d_{K/k}(\omega)$. Also the Vander - monde
determinant shows that 
\begin{center}
\begin{fbox}
{$D_{K/k} (\omega)= (-1)^{n(n - 1)2} N_{K/k}
    (d_{K/k}\omega)$} 
\end{fbox}
\end{center}

Suppose\pageoriginale $K/k$ is a finite galois extension. Let
$\sigma_1, \ldots , \sigma_n$ be the distinct automorphisms of
$K/k$. We shall now prove   

\begin{thm}\label{c4:thm6}%the 6
If $k$ contains sufficiently many elements, then there exists in
  $K$ an element $\omega$ such that $\omega^{\sigma_1}, \ldots ,
  \omega^{\sigma_n}$ form a basis of $K$ over $k$. 
\end{thm}

\begin{proof}
It $\omega \in K$ such that 
$$
D(\omega^{\sigma_1}, \ldots , \omega^{\sigma_n}) \neq o 
$$
then $\omega^{\sigma_1}, \ldots , \omega^{\sigma_n}$ form a base of
$K/k$. For,if  
$$
\sum_i a_i \omega^{\sigma_i} = o, \quad  a_1, \ldots, a_n \in k
$$
not all zero, then since $\sigma_1, \ldots, \sigma_n$ form a group 
$$
\sum_{i}a_i \omega^{\sigma_j \sigma_i} = o, \quad j=1, \ldots,n. 
$$

Therefore from the expression \eqref{c4:eq9} for $D_{K/k}(\omega^{\sigma_1},
\ldots , \omega^{\sigma_n})$, it follows that  
$$
\left(S_{K/k}(\omega^{\sigma_i}\omega^{\sigma_j})  \right)
\begin{pmatrix}
a_1\\ \vdots \\ a_n \\ 
\end{pmatrix}
=
\begin{pmatrix}
o\\
 \vdots \\ 
 o
\end{pmatrix}
$$
or that $D(\omega^{\sigma_ 1},\ldots,\omega^{\sigma_n})=o$ which is a
contradiction. We have therefore to find an $\omega$ with this
property. Put  
$$
\omega^{\sigma_i}= x_1\omega^{\sigma_i}_1+ \cdots + x_n
\omega^{\sigma_i}_n, \quad i= 1,2, \ldots  
$$
where $x_1, \ldots ,x_n$ are indeterminates and $\omega_1, \ldots ,
\omega_n$ is a basis of $K/k$ 
\end{proof}

Then 
\begin{align*}
S_{K/k}(\omega^{\sigma_i}\omega^{\sigma_j}) &= \omega^{\sigma_1
  \sigma_i}\omega^{\sigma_1\sigma_j}, + \cdots + \omega^{\sigma_n
  \sigma_i}\omega^{\sigma_n \sigma_j}\\ 
&= \sum_{a,b} S_{K/k}(\omega^{\sigma_i}_a \omega^{\sigma_j}_b) x_a
x_b. 
\end{align*}

Then\pageoriginale $D_{K/k}(\omega^{\sigma_1}, \ldots,
\omega^{\sigma_n})$ defined as the determinant of the matrix
$(S_{K/k}(\omega^{\sigma_i} \omega^{\sigma_j}))$ is a polynomial in
$x_1, \ldots, x_n$ with coefficients in $k$.  

In order to prove that $D_{K/k}(\omega^{\sigma_1}, \ldots,
\omega^{\sigma_n})$ is non-zero polynomial, notice that by definition, 
$$
\begin{pmatrix}
\omega^{\sigma_1}\\
\vdots\\
\omega^{\sigma_n}
\end{pmatrix}
=
\begin{pmatrix}
\omega^{\sigma_1}_1, & \omega^{\sigma_1}_2, & \ldots, & \omega^{\sigma_1}_n\\
\hdotsfor{4}\\
\omega^{\sigma_n}_1, &  \omega^{\sigma_n}_2, &  \ldots, & \omega^{\sigma_n}_n\\
\end{pmatrix}
\begin{pmatrix}
x_1\\
\vdots\\
x_n
\end{pmatrix}
$$
so that if $\omega^{\sigma_1}=1$ and $\omega^{\sigma_i}=o$ for $i>1$,
then $x_1, \ldots, x_n$ are not all zero and for this set of values of
$x_1, \ldots, x_n$, the polynomial $D_{K/k}(\omega^{\sigma_1},\break \ldots,
\omega^{\sigma_n})$ has a value $\neq o$, as can be see from the fact
that  
$$
D_{K/k}(\omega^{\sigma_1}, \ldots, \omega^{\sigma_n})=
\begin{vmatrix}
\omega^{\sigma_1}, & \ldots, & \omega^{\sigma_n}\\
\hdotsfor{3}\\
\omega^{\sigma_n\sigma_i}, & \ldots, & \omega^{\sigma_n \sigma_n}
\end{vmatrix}^2
$$

Therefore if $k$ has sufficiently many elements, there exist values in
$k$ of $x_1, \ldots, x_n$, not all zero, such that
$D_{K/k}(\omega^{\sigma_1}, \ldots, \omega^{\sigma_n})\neq o$.  

This proves the theorem

In particular, if $k$ is an infinite field, there exists a base of
$K/k$ consisting of an element and its conjugates. Such a base is said
to be a \textit{normal base}. 

The theorem is also true if $k$ is a finite field.

