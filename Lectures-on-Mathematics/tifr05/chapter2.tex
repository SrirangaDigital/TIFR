 \chapter{Algebraic extension fields}\label{chap2}

\section{Conjugate elements}\label{c2:s1}%\section1
 Let\pageoriginale $ \Omega$ be an algebraic closure of $k$ and $K$ an
 intermediary 
 field. Let $\Omega '$  be an algebraic  closure of $K$  and so of
 $k$. Then there is an isomorphism $\tau$ of  $ \Omega'$ on $ \Omega$
 which is trivial on $k$. The restriction of this isomorphism to $K$
 gives a field $ \tau K $ in $ \Omega$ which is $k$-isomorphic to
 $K$. Conversely suppose $K$ and $K'$  are two subfields of $ \Omega$
 which are $k$-isomorphic. Since $\Omega$ is a common algebraic closure
 of  $K$  and $K'$, there  exists an automorphism of  $ \Omega$ which
 extends  the $k$-isomorphism of $K$  and $K'$. Thus 
  \begin{enumerate}[1)]
\item \textit{Two subfields $K, K' $  of $ \Omega / k $  are
  $k$-isomorphic if and only if there exists a $k$-automorphism $ \sigma
  $ of $ \Omega$ such that  $ \sigma K = K'$.} 

We call two such fields $K$ and $K'$ \textit{conjugate fields over} $k$.

We  define two elements $ \omega, \omega' $  of  $ \Omega / k $, to be
\textit{conjugate} over $k$ if there  exists a $k$-automorphism
$\sigma$ of $ \Omega $ such that  
$$
\sigma \omega = \omega'
$$

 The automorphisms of $\Omega$ which are  trivial on $k$ form a group
 and  so the above relation of conjugacy is an equivalence
 relation. We can therefore put elements of $ \Omega$  into classes of
 conjugate elements over $k$. We then have  

 \item \textit{Each class of conjugate elements over $k$  contains
   only a finite number of elements.} 
\end{enumerate}  

\begin{proof}
Let $C$\pageoriginale be a class of  conjugate elements and  $ \omega
\in C $. Let 
$ f(x) $ be the minimum polynomial of $\omega $ in  $ k [ x ] $. Let $
\sigma $ be an automorphism of  $ \Omega / k $. Then $ \sigma \omega
\in C $. But $ \sigma \omega $ is a root of  $ f^\sigma (x) = f (x)
$. Also if  $ \omega' \in C $  then $ \omega' = \sigma \omega $ for
some  automorphism $ \sigma $ of  $ \Omega /k $. In  that  case $
\sigma \omega  = \omega' $ is  again a root of  $f (x)$. Thus the
elements in $C$ are all roots of the irreducible polynomial $f(x)
$. Our contention follows. 
  \end{proof}  
  
  Notice that  if $ \alpha$, $\beta$ are any two roots, lying in
  $\Omega$,  of the irreducible polynomial $ f (x) $, then $ k (
  \alpha) $ and  $k (\beta) $ are $k$-isomorphic. This isomorphism can
  be extended into  an automorphism of $\Omega$. Thus 

\setcounter{thm}{0}
\begin{thm}\label{c2:thm1}%them 1
To each class of conjugate elements of $ \Omega$  there is
   associated an irreducible polynomial in $ k [ x ] $ whose distinct
   roots are all the elements of this class. 
\end{thm}  
  
If $ \alpha \in  \Omega$ we shall denote by $C_\alpha$ the class of
$\alpha $. $ C_\alpha$ is  a finite set. 
  

\section{Normal extensions}\label{c2:s2}%\section2

Suppose $K$ is a  subfield of $ \Omega / k $ and $ \sigma $ an
automorphism of $ \Omega / k $. Let $ \sigma K \subset K$. We assert
that $\sigma K = K $. For let  $\alpha \in K$ and denote by
$\bar{C}_\alpha$ the set  
$$
C_\alpha \cap K 
$$

Since $ \sigma K \subset K $ we have $ \sigma \alpha \in K$ so
$\sigma \alpha \in \bar{C}_\alpha$. Thus  
$$
\sigma \bar{C}_\alpha \subset_\alpha
$$
$ \bar{C}_\alpha$ is  a finite set and $ \sigma $ is an isomorphism
of $ K $ into itself.  

Thus
$$
\sigma \bar{C}_\alpha = \bar{C}_\alpha
$$ 
which\pageoriginale means $ \alpha \in  \sigma K $. Thus $ K = \sigma K$.

We shall now study a class of fields $ K \subset  \Omega / k $ which
have the property  
$$
 \sigma K \subset K,
$$
for all automorphisms $ \sigma$ of $ \Omega /k $. We shall call such
fields, \textit{normal extensions} of  $ k $  in $\Omega $. 

 Let $ K/k $  be a normal extension of $k$ and  $ \Omega$ algebraic
 closure of $k$ containing $K$. Let $ \alpha \in K $  and $ C_\alpha
 $ the class of $ \alpha $. We assert that $ C_\alpha \subset K
 $. For if  $ \beta $ is an element of $ C_\alpha $, there is an
 automorphism $ \sigma $ of  $\Omega / k $ for which $ \beta  =
 \sigma \alpha $. Since $ \sigma K \subset K $, it follows that $
 \beta  \in K $. Now any element $ \alpha $  in $K$  is a root of an
 irreducible polynomial in $ k [ x ] $. Since all the elements of $
 C_\alpha $ are roots  of this polynomial, it follows that if $ f(x)$
 is an  irreducible polynomial with one root in $K$, then all roots 
 of $ f (x) $  lie in $K$. 
 
 Conversely let $K$ be a subfield of $ \Omega / k $ with this
 property. Let $ \sigma $ be an automorphism of $ \Omega /k $ and $
 \alpha / K $. Let $\sigma$ be an automorphism of $\Omega /k$ and
 $\alpha \in K$. Let $C_\alpha$ be the class of $\alpha$. Since
 $C_\alpha \subset K, \sigma \alpha \in K $. But $\alpha$ is
 arbitrary in  $ K$. Therefore 
 $$
 \sigma K \subset K
 $$
and $ K $ is normal. Thus the 

\begin{thm}\label{c2:thm2}%them 2
Let $ k \subset K \subset  \Omega $. Then $ \sigma K = K $ for all
  automorphisms $ \sigma $ of $ \Omega / k \Longleftrightarrow  $
  every irreducible polynomial $ f (x) \in k [ x ] $ which has one
  root in $K$ has all roots in $K$. 
\end{thm}

Let $f (x) $ be a polynomial in $ k [ x ] $ and $K$ its splitting
field. Let $\Omega$ be an algebraic  closure of $K$. Let $ \alpha_1
, \ldots , \alpha_n $\pageoriginale be the distinct roots of $f (x) $
in $ \Omega $. Then   
$$
K= k ( \alpha_1 , \ldots , \alpha_n ) 
$$

Let $\sigma$ be an automorphism of $ \Omega /k$. $ \sigma \alpha_j =
\alpha_j $ for some $ j$. Thus $ \sigma $ takes the set $ \alpha_1 ,
\ldots , \alpha_n $ onto itself. Since every element of $K$ is a
rational function of $\alpha_1 , \ldots , \alpha_n $, it follows
that $\sigma K \subset K $. Thus 
\begin{enumerate}[i)]
\item \textit{The splitting field of a polynomial in $ k [ x ] $ is
  a normal extension of} k. 

Let $ \{ K_\alpha \} $ be a  family of normal subfields of $ \Omega /
k $. Then $\bigcap\limits_{\alpha} K_\alpha $ is trivially
normal. Consider $k (\bigcup \limits_{\alpha} K_\alpha ) $. This
again is normal since for any automorphism $\sigma$ of $ \Omega /k $. 
$$
\sigma k \left( \bigcup_{\alpha} K_{\alpha} \right)  \subset k \left(
\bigcup_{\alpha} \sigma K_{\alpha} \right) \subset k \left(
\bigcup_{\alpha} K_\alpha \right)   
$$

Let now $ \{ f_\alpha (x) \} $ be a set of polynomials in $ k [ x ] $
and $ K_\alpha $ their splitting fields, then $ K (
\bigcup\limits_{\alpha} K_{\alpha} ) $ is normal. Also it is easy to
see that   
$$
L = k  \left( \bigcup_{\alpha} K_\alpha \right) 
$$
is the intersection of all subfields of $\Omega /k $  in which every
one of the polynomials $f_\alpha (x) $ splits completely. Thus 

 \item  \textit{if $ \{ f _\alpha (x) \} $  is a set of polynomials in
   $ k [ x ] $, the subfield of $ \Omega $  generated by all the roots
   of  $ \{ f_\alpha (x) \} $ is normal}. 
 
 We also have 

 \item \textit{ If  $K/k$  is normal and $k \subset L \subset K$ then
   $K/ L$  is also normal}. 

For\pageoriginale if $\sigma$ is an $L$- automorphism of $\Omega$,
then $\sigma$ is also a $k$-automor\-phism of $\Omega$ and so $\sigma K
\subset K$.   

The $k-$automorphisms of $\Omega $ form a group $G ( \Omega /k )
$. From what we have seen above, it follows that a subfield $K$  of
$\Omega /k $  is normal if and only if $ \sigma K =K $ for every $
\sigma \in G ( \Omega /k ) $. Now a $k-$ automorphism of $ K $ can be
be  extended into an automorphism of $ \Omega /k $, because every such
automorphism is an isomorphism of $K$ in $ \Omega $. It therefore
follows 

\item \textit{$K/k $ is normal if and only if every isomorphism of $K$
  in $\Omega /k $  is an automorphism of $K$ over k.} 
\end{enumerate}

As an example, let $ \Gamma $ be the field of rational numbers  and $
f (x) = x^3 - 2$. Then $ f (x) $ is irreducible in $ \Gamma [ x ]
$. Let $ \alpha = \sqrt[3]{2}$ be  one of its roots. $ \Gamma (
\alpha ) $ is of degree 3 over $ \Gamma $ and is not normal since
it does not contain $ \rho$ where $ \rho = \frac{ -1 +
  \sqrt{-3}}{2} $. However the field $ \Gamma ( \alpha, \rho ) $ of
degree 6 over $ \Gamma $ is normal and is the splitting field of  $
x^3 - 2 $. 

If $K$ is the field of  complex numbers, consider $ K(z)$ the field of
rational function of 2 over $K$. Consider the polynomial $ x^3 -z $
in $ K (z) [ x ] $. This is irreducible. Let  $ \omega =
z^{\frac{1}{3}} $  be a root of this polynomial. Then $K(z)  (\omega)$
is of degree 3  $K(z)$ and is the splitting field of the polynomial
$ x^3 - z$. 

\section{Isomorphisms of fields}\label{c2:s3}%section 3. 

 Let $ K/k $  be an algebraic extension of $ k$ and $ W $
 \textit{any} extension of $ K $ and so of $k$. A mapping $ \sigma $
 of $ K $ into $ W $ is  said\pageoriginale to  be \textit{$k-$
   linear} if for $  \alpha, \beta \in K$  
$$
\sigma ( \alpha + \beta ) = \sigma \alpha + \sigma \beta 
$$

$\sigma \alpha \in W $ and if $ \lambda \in k, \sigma ( \lambda \alpha
) = \lambda \sigma \alpha $. If $ \sigma $ is a $k-$linear map of $ K
$ into $ W $ we define $\alpha \sigma $ for $\alpha$ in $ W $ by 
 $$
 ( \alpha \sigma ) \beta = \alpha \sigma \beta
 $$
 for $ \beta \in K $. This again is a $k-$linear map and so the
 $k-$linear maps of $K$ into $W$ form a vector space $V$ over $W$. 

A $k-$isomorphism $\sigma$ of $K$ into $W$ is obviously a $k- $linear
map and so $\sigma \in V $. We shall say, two isomorphisms $
\sigma, \tau $ of $K$  into $W$ (trivial on $k$) are
\textit{distinct} if there exists at least one $ \omega \in K, 
\omega \neq 0 $ such that  
$$
\sigma \omega \neq \tau \omega
$$

Let $S$ be the set of mutually distinct isomorphisms of $K$  into
$W$. We then have  

\begin{thm}\label{c2:thm3}%them 3
$S$ is a set of linearly independent elements of $V$ over $W$.
\end{thm}

\begin{proof}
We have naturally to show that every finite subset of $S$ is linearly
independent over $W$. Let on the contrary $\sigma_1 , \ldots ,
\sigma_n $ be a finite subset of $S$ satisfying a non trivial linear
relation  
$$
\sum_{i} \alpha_i \sigma_i = 0
$$
$ \alpha_i \in W $. We may clearly assume that no proper subset of $
\sigma_1 , \ldots , \sigma_n $ is linearly dependent. Then in the
above expression all $ \alpha_i $ are different from zero. Let
$\omega$ be any element of $K$. Then  
$$
\sum_{i}\alpha_i \sigma_i \omega = 0 
$$
\end{proof}

If\pageoriginale we replace $\omega$ by $\omega \omega{'}$ we get,
since $\sigma _i{'} s$ are isomorphisms,   
$$
\sum_{i} \alpha_i \sigma_i \omega{'}. \sigma_i \omega = o
$$
for every $\omega \in K$. This means that $\sigma_i ,\ldots ,
\sigma_n$ satisfy another linear relation 
$$
\sum_{i} \alpha_i \sigma_i \omega'. \sigma_i \omega = o 
$$

Since the isomorphisms are mutually distinct, we can choose
$\omega{'}$ in $K$ in such a way that  
$$
\sigma_1 \omega{'} \neq \sigma_n \omega{'}
$$

We then get from the two linear relations, the expression 
$$
\sum_{i = 1}^{n - 1} \bigg( \frac{\alpha_i}{\alpha n} - \frac{\sigma_i
  \omega{'} \alpha_i}{\sigma_n \omega{'} \alpha_i} \bigg) \sigma_i =
o. 
$$

This relation is non trivial since the coefficient of $\sigma_1$ is
different from zero. This leads to a contradiction and our theorem is
proved. 

Suppose $\dim V < \infty$ then it would mean that $S$ is a finite
set. But the converse is false. We have however the  

\begin{thm}\label{c2:thm4}%them 4
If $(K : k) < \infty$, then dim $V = (K : k)$
\end{thm}

\begin{proof}
Let $(K : k) = n$ and $\omega_1, \ldots , \omega_n$ a basis of
$K/k$. 
\end{proof}

Consider the $k$-linear mappings $\sigma_1, \ldots , \sigma_n$ defined by 
\begin{equation*}
\begin{cases}
\sigma_i (\omega_j ) & = o \qquad i \neq j \\
 &=  1  \qquad  i \neq j
\end{cases}
\end{equation*}

Then $\sigma_1 , \ldots , \sigma_n$ are linearly independent elements
of $V$ over $W$. For, let $\sum\limits_{i} \alpha_i \sigma_i = o$,
$\alpha_i \in W$. Then 
$$
\bigg( \sum_{i} \alpha_i \sigma_i \bigg) \omega_i = o
$$
for $j = 1, \ldots , n$. This proves that $\alpha_j = o$. Now let
$\sigma$ be any\pageoriginale $k$-linear mapping. It is uniquely
determined by its effects on $\omega_1 , \ldots , \omega_n$. Put
$\alpha _i = \sigma \omega_i$ and let $\tau$ be given by   
$$
\tau = \sigma - \sum_i \alpha_i \sigma_i
$$

Then $\tau (\omega_j ) = \sigma \omega_j \sum_i \alpha_i \sigma_i
(\omega_j ) = o$ so that $\tau = o$. Our contention is established. 

From this we obtain the very important 

\begin{coro*}%corollary 0
If $(K : k) < \infty$ then $K$ has in $\Omega/k$ at most $(K :
  k)$ distinct $k$-isomorphisms. 
\end{coro*}

Let $\alpha \in \Omega$. Consider the field $k (\alpha )/ k$. Let
$\alpha^{(1)} (= \alpha ), \ldots, \alpha^{(n)}$ be the distinct
conjugates of $\alpha$ over $k$. An isomorphism $\sigma$ of $k
(\alpha)/k$ is determined completely by its effect on $\alpha$. Since
every isomorphism comes from an automorphism of $\Omega/k$, it follows
that $k (\alpha^{(i)})$ are all the distinct isomorphic images of $k
(\alpha)$. Thus  

$1)$ \textit{Number of distinct $k$-isomorphisms of $k (\alpha)$ in
  $\Omega$ is equal to the number of distinct roots in $\Omega$ of the
  minimum polynomial of $\alpha$}. 

Let $K/k$ be an algebraic extension and $\Omega$ an algebraic closure
of $k$ containing $K$. Let $K$ have the property that $K/k$ has only
finitely many distinct $k$-isomorphisms in $\Omega$. Let $K^{(1)} (= K),
K^{(2)} , \ldots ,K^{(n)}$ be the distinct isomorphic fields. Let
$\alpha \in \Omega $ and let $\alpha$ have over $K$ exactly $m$
distinct conjugates $\alpha^{(1)} (= \alpha), \ldots
,\alpha^{(m)}$. This means that if $f (x)$ is the minimum polynomial
of $\alpha$ over $K$, then $f(x)$ has in $\Omega$, $m$ distinct
roots. We claim that $K(\alpha)$ has\pageoriginale over $k$ exactly
$mn$ distinct isomorphisms in $\Omega$.   

For, let $\sigma_i (i = 1, \ldots , n)$ be the $k$-isomorphisms defined
by  
$$
\sigma_i K^{(1)} = K^{(i)}
$$

Let $f^{\sigma_i} (x)$ be the image of the polynomial $f(x)$ in $K[x]$
by means of the above isomorphism. Let the roots of $f^{\sigma_i}(x)$
by $\alpha^{(i_1)} , \ldots , \alpha^{(i_n)}$ these being the
distinct ones. There exists then an isomorphism $\sigma_{ij} (j = 1,
\ldots , m)$ extending $\sigma_i$ of $K^{(1)} ( \alpha^{(1)})$ on $K
^{(i)} ( \alpha^{(ij)})$. Since $i$ has $n$ values, it follows that
there are at least $mn$ distinct isomorphisms of $K(\alpha)$ over $k$.  

Let now $\sigma$ be any automorphism of $\Omega /k$. Let $\sigma K =
K^{(i)}$. Then it takes $\alpha^{(1)}$ into a root $\alpha^{(ij)}$ of
$f^\sigma (x) = f^{\sigma_i} (x)$ where $\sigma_i$ is the isomorphism
which coincides with $\sigma$ on $K$. Thus since every isomorphism of
$K(\alpha)$ over $k$ comes from an automorphism of $\Omega/k$, our
contention is established.  


Let now $K= k(\omega_1 , \ldots , \omega_n)$ and $K_i = k (\omega_1 ,
\ldots , \omega_i)$ so that $K_o= k$ and $K_n= K$. Let $K_i$ have over
$K_{i-1}$ exactly $P_i$ distinct $K_{i-1}$-isomor\-phisms. Then $K/k$
has exactly $p_1 \cdots p_n$ distinct $k$-isomorphisms in
$\Omega$. Hence  

2) ~\textit{ If $K \supset L \supset k$ be a tower of finite
  extensions and $K$ has ever $L$, $n$ distinct $L$-isomorphisms in
  $\Omega$ and $L$ has over $k$, $m$ distinct $k$-isomor\-phisms then $K$
  has over $k$ precisely $mn$ distinct $k$-isomorphisms.} 

In particular let $(K : k) < \infty$ and let $K$ have in $\Omega$
exactly\pageoriginale $(K : k)$ distinct isomorphisms. Let $L$ be any
intermediary field. Let $\underline{a}$ be the number of distinct
$L$-isomorphisms of $K$ and $\underline{b}$ the number of distinct
$k$-isomorphisms of $L$.   

Then
$$
(K : k) = ab \leq (K : L) (L : k) = (K : k)
$$
But $a \leq (K : L), b \leq (L : k)$. Thus $a = (K : L)$ and $b = (L
:k)$. 


\section{Separability}\label{c2:s4} %sec 4.

Let $\Omega$ be an algebraic closure of $k$ and $ \omega \in
\Omega$. Let $\phi (x)$ be the minimum polynomial of $\omega$ in
$k$. Suppose $k (\omega)/k$ has exactly $(k (\omega ) : k)$ distinct
$k$-isomorphisms in $\Omega$. Then from the last article it follows that
all the roots of $\phi (x)$ are distinct. Conversely let the
irreducible polynomial $\phi (x)$ be of degree $n$ and all its $n$
roots distinct. Then $k(\omega)/k$ has $n$ distinct $k$-isomorphisms
$\omega$ being a root of $\phi (x)$. But it can have no more. 

Let us therefore make the 

\begin{defi*}%defin 0
An element $\omega \in \Omega$ is said to be separably algebraic or
separable over $k$ if its minimum polynomial has all roots
distinct. Otherwise it is said to be inseparable. 
\end{defi*}
\begin{enumerate}[1)]
\item \textit{Let $W/k$ be any extension field and $\omega \in W$
  separable over $k$. Let $L$ be an intermediary field. Then $\omega$
  is separable over $L$.} 

For, the minimum polynomial of $\omega$ over $L$ divides that over
$k$. 

\item $\omega \in \Omega $\pageoriginale \textit{separable over $k
  \Leftrightarrow k   (\omega)/k$ has in $\Omega (k (\omega) :k)$
  distinct   $k$-isomorphisms.}  

Let now $K = k (\omega_1, \ldots ,\omega_n)$ and let $\omega_1 ,
\ldots \omega_n$ be all separable over $k$. Put $K_i = k(\omega_1
,\ldots , \omega)$ so that $K_o = k$ and $K_n = K$. Now
$K_{i-1}(\omega_i)$ and $\omega_i$ is separable over $K_{i-1}$ so that
$K_i$ over $K_{i-1}$ has exactly $(K_i : K_{i-1})$ distinct $K_{i-1}$-
isomorphisms. This proves that $K$ has over $k$ 
$$
(K_n : K_{n-1}) \ldots (K_1 : K_o) = (K_n : K_o) = (K : k)
$$
distinct $k$-isomorphisms. If therefore $\omega \in K$, Then by previous
article $k(\omega)$ has exactly $(k (\omega) : k)$ distinct
isomorphisms and hence $\omega$ is separable over $k$. Conversely if
$K/k$ is finite and every element of $K$  is separable over $k$, then
$K/k$ has exactly $(K : k)$ distinct $k$-isomorphisms. Hence 

\item $(K  : k) < \infty$, $K/k$  \textit{has $( K : k)$ distinct
  $k$-isomorphisms $\Leftrightarrow$ every element of $K$ is separable
  over $k$}. 

Let us now make the 

\begin{defi*}%defn 0 
A subfield $K$ of $\Omega /k$ \textit{is said to be separable over $k$
  if every element of $K$ is separable over $k$.} 
\end{defi*}

From 3) and the definition, it follows that 

\item $K/k$ \textit{is separable $\Leftrightarrow$ for every subfield
  $L$ of $K$ with $(L : K) < \infty , L$ has exactly $(L : K)$
  distinct isomorphisms over $k$.} 

\item $K/L, L/k$ separably algebraic $\Leftrightarrow K/k$ separable. 

For, let $\omega \in K$. Then $\omega$ is separable over $L$. Let
$\omega_1 ,\ldots , \omega_n$ be the coefficients in the irreducible
polynomial\pageoriginale satisfied by $\omega$ over $L$. Then $\omega$
has over $K_1 = k (\omega_1 ,\ldots ,\omega_n)$ exactly $(K_1 (\omega)
: K_1)$ distinct $K_1$-isomorphisms. Also $K_1/k$ is finite
separable. Thus $K_1 (\omega)$ has over $k$ exactly $(K_1(\omega) :
k)$ distinct isomorphisms which proves that $\omega$ is separable over
$k$. The converse follows from 2).  

\item \textit{If $\{ K_\alpha \}$ is a family of separable subfields
  of $\Omega$ then} 

$2)$ $\bigcap\limits _{\alpha} K_{\alpha}$ \textit{ and } $b)
  k(\bigcup\limits_{\alpha} K_\alpha )$ \textit{are separable.} 

$a)$ follows easily because every element of $K_\alpha$ is separable
  over $k$. $b)$ follows since every element of $k
  (\bigcup\limits_\alpha K_\alpha)$ is a rational function of a finite
  number of elements and as each of these is separable the result
  follows from 3). 

\item \textit{Let} $K/k$ be any extension-not necessarily
  algebraic.\textit{ The set $L$ of elements of $k$ separably
    algebraic over $K$ is a field}. 

This is evident. We call $L$ the \textit{separable closure} of $k$ in
$K$.  

We had already defined an algebraic element $\omega$ to be inseparable
if its minimum polynomial has repeated roots. Let us study the nature
of irreducible polynomials. 

Let $f (x) = a_o + a_1 x + \cdots + a_n x^n$ be an irreducible
polynomial in $k[ x ]$. If it has a root $\omega \in \Omega$ which is
repeated, then $\omega$ is a root of  
$$
f^1 (x) = a_1 + 2a_2 x + \cdots + n a_n x^{n-1}.
$$

Thus $f (x) | f^1 (x)$ which can happen only if
$$
i a_i = o, i = 1, \ldots , n.
$$

Let\pageoriginale $k$ have characteristic zero. Then $i a_i = o
\Rightarrow a_i = 0$ that is $f(x)$ is a constant polynomial. Thus   

\item \textit{Over a field of characteristic zero, every non constant
  irreducible polynomial has all roots distinct.} 

Let now $k$ have characteristic $p \neq o$. if $p \chi i$ then $i a_i
= o \Rightarrow a_i = o$. Thus for $f^1 (x)$ to be identically zero we
must have $a_i = o$ for $p \chi i$. In this case 
$$
f(x) = a_o + a_p x^p + \cdots
$$
or that $f(x) \in k[x^p]$. Let $e$ be the largest integer such that $f
(x) \in k[ x^p ]$ but not in $k [x^{p^{e + 1}}]$. Consider the
polynomial $\phi (y)$ with $\phi (x^{p^e}) = f(x)$. Then $\phi (y)$ is
irreducible in $k[y]$ and $\phi (y)$ has \textit{no repeated
  roots}. Let $\beta_1, \ldots , \beta_t$ be the roots of $\phi (y)$
in $\Omega$. Then 
$$
f(x) = ( x^{p^e} - \beta_1 ) \cdots (x^{p^e} - \beta_t).
$$

Thus $n = t \cdot p^e$. The polynomial $x^{P^e} - \beta_i$ has in $\Omega$
all roots identical to one of them say  $\alpha_i$. Then 
$$
x^{P^e} - \beta_i = x^{P^e} - \Delta \alpha^{p^e}_i = (x - \alpha_i)^{p^e}
$$
Thus 
$$
f(x) = \{ (x - \alpha_1) \cdots (x - \alpha_t ) \}^{p^e}
$$


Moreover since $\beta_1 \ldots \beta_t$ are distinct, $\alpha_1 ,
\ldots ,\alpha_t$ are also distinct.\break Hence  

\item \textit{Over a field of characteristic $p \neq o$, the roots of
  an irreducible polynomial are repeated equally often, the
  multiplicity of a} root being $p^e$, $e \geq o$. 

\textit{It is important to note} that $(x - \alpha_1) \ldots (x  -
\alpha_t)$ is \textit{not} a polynomial\pageoriginale in $k[x]$ and
$t$ is \textit{not} necessarily prime to $p$.   

We call $t$ the \textit{reduced degree} of $f(x)$ (or of any of its
roots ) and $p^e$, \textit{its degree of inseparability}. Thus

\fbox{Degree of: Reduced degree $X$-degree of inseparability}

If $\omega \in \Omega$ then we had seen earlier that $k(\omega) / k$
has as many distinct isomorphisms in $\Omega$ as there are distinct
roots in $\Omega$ of the minimum polynomial of $\omega$ over $k$. If
we call the reduced degree of $\dfrac{k(\omega)}{k}$ as the \textit{reduced 
  degree of} $\omega$ we have  

\item \textit{Reduced degree of $\omega = $ Number of distinct roots
  of the minimum polynomial of $\omega$ over $k$.} 

We may now call a polynomial \textit{separable} if and only if every
root of it in $\Omega$ is separable. In particular if $f(x) \in k[x]$
is irreducible then $f(x)$ is separable if one root of it is
separable. 

Let $\omega \in \Omega$ and $f(x)$ the minimum polynomial of $\omega$
in $k[x]$. If $t$ = reduced degree of $\omega$, then  
$$
f(x) = \{ (x - \omega_1)\ldots (x - \omega_t) \}^{p^e}
$$
$n = t - p^e$. Let $\omega_1 = \omega$. Consider $\omega_1^{p^e} =
\beta_1$. Then 
$$
f(x) = (x^{p^e} - \beta_1) \cdots (x^{p^e} - \beta_t)
$$
and $\beta_1, \ldots \beta_t$ are separable over $k$. Consider the
field $k (\beta_1)$ which is a subfield of $k(\omega). \beta$ being of
degree $t$ over $k, (k (\beta_1) : k) = t$. This means that  
$$
(k (\omega ) : k (\beta ) ) = p^{e}.
$$

But the interesting fact to note is that $k(\omega)$ has over
$k(\beta)$ only the identity isomorphism or that $k(\omega)$ is fixed
by every $k (\beta)$-automor\-phism\pageoriginale of $\Omega/k (\beta)$.  

Also since every element of $k(\omega)$ is a rational function of
$\omega$ over $k(\beta)$, it follows that  
$$
\lambda^{p^e} \in k(\beta)
$$
for every $\lambda \in k(\omega)$. Thus the integer $e$ has the
property that for every $\lambda \in k(\omega), \lambda^{p^e} \in k
(\beta)$ and there is at least one $\lambda$ (for instance $\omega$)
for which $\lambda^{p^e} \notin k(\beta)$. $\underbar{e}$ is called the
\textit{exponent} of $\omega$, equivalently of $k(\omega)$. We define
the exponent of an algebraic element $\alpha$ over $k$ to be the
integer $e \geq o$ such that $\alpha^{p^e} $ is separable but not
$\alpha^{p^{e-1}}$. Hence

\item \textit{Exponent of} $\alpha$ is zero $\Leftrightarrow \alpha$
  \textit{is separable over} $k$. 
\end{enumerate}

We shall now extend this notion of exponent and reduced degree to any
finite extension. 

Let $K/k$ be finite so that $ K = k (\omega_1 , \ldots ,
\omega_n)$. Put as before $K_o = k$, $K_i = k (\omega_1 ,\ldots
,\omega_i)$ so that $K_n = K$. Let $\omega_i$ have reduced degree $d_i$
and exponent $e_i$ over $K_{i-1}$. Then 
$$
(K_i :  K_{i-1}) = d_i  p^{e_i}
$$

From the definition of $d_i$, it follows that the number of distinct
$k$-isomorphisms of $K/k$ is $d_1 \ldots d_n$. We put  
$$
d = d_1 \cdots d_n
$$
and call it the \textit{reduced degree} of $K/k$. Then  
$$
(K : k) = d. \ldots p^f
$$
where $f = e_1 + \cdots + e_n$. We call $p^f$ the degree of
\textit{inseparability} of $K/k$. 

In order\pageoriginale to be able to give another interpretation to
the integer $\underline{d}$ we make the following considerations.  

Let $\Omega \supset K \supset k$ and let $K/k$ have the property that
every $k$ automorphism of $\Omega/k$ acts like identity on $K$. Thus
if $\sigma \in G(\Omega/k)$ and $\omega \in K$, then  
$$
\sigma \omega = \omega
$$

All elements of $k$ have this property. Let $\omega \in K$, $\omega
\notin k$. Then by definition, $\omega$ has in $\Omega$ only one
conjugate. The irreducible polynomial of $\omega$ has all roots
equal. Thus the minimum polynomial of $\omega$ is  
$$
x^{p^m} - a
$$
where $\underbar{a} \in k$. i.e., $\omega^{p^m} \in k$. On the other
hand let $K$ be an extension of $k$ in $\Omega$ with the property that
for every $\omega \in K$ 
$$
\omega^{p^m} \in k
$$
for some integer $m \geq o$. Let $\sigma$ be an automorphism of
$\Omega/k$. Then $\sigma \omega^{p^m} = \omega^{p^m}$. But  
$$
o = \sigma \omega^{p^m} - \omega^{p^m} = (\sigma \omega - \omega )^{p^m}
$$
which shows that $\sigma \omega = \omega$. \, $\omega$ being arbitrary in
$K$, it follows that every element of $G(\Omega /k )$ is identity on
$K$.  

Hence for $\omega \in \Omega$ the following three statements are
equivalent  
\begin{enumerate}[1)]
\item $\sigma \omega = \omega$ for all $\sigma \in G(\Omega /k)$
\item $\omega^{p^m} \in k$ for some $m \geq o$ depending on $\omega$ 
\item The irreducible polynomial of $\omega$ over $k$ is of the form
  $x^{p^m} - a, a \in k$. 
\end{enumerate}

We call an element $\omega \in \Omega$ which satisfies any one of the
above\pageoriginale conditions, a \textit{purely inseparable algebraic
  element over   $k$.}  
 
Let us make the 

\begin{defi*}%defi 0
 A subfield $ K /k$ of $\Omega /k$  is said to be purely
  inseparable if element $\omega$ of $K$ is purely inseparable.
\end{defi*}

From what we have seen above, it follows that $K /k$ is purely
inseparable is equivalent to the fact that every $k$-automorphism of 
$\Omega$ is identity on $K$.  

Let $K/k$ be an algebraic extension and $L$ the maximal separable
subfield of $K/k$. For every $\omega \in K, \omega ^{p^e}$ is
separable for some $e \geq o $ which means that $\omega ^{p^e} \in
L$. Thus $K/ L$ is a purely inseparable extension field. This means
that every $k$-isomorphism of $L/k$ can be extended uniquely to a
$k$-isomorphism of $K/k$.  

Let, in particular, $K/k$ be a finite extension and $L$ the maximal
separable subfield of $K$. Then $K/L$ is purely inseparable and $K/L$
has no $L$-isomorphism other than the identity. Thus the number of
distinct isomorphism other than the identity. Thus the number of
distinct isomorphisms of $K/k$ equals $(L : k)$. But from what has
gone before  
$$
(K : L ) = p^f, (L : k) = d.
$$

For this reason we shall call $\underline{d}$ also the \textit{degree
  of separability} of $K / k$ and denote it by $[ K : k]$. We shall
denote the degree of inseparability, $p^f$, by $\big\{ K :
k\big\}$. Then  
$$
(K : k) = [K : k] \big\{ K : k\big\}. 
$$

\begin{note*}%note 0
 In case $k$ has characteristic zero, every algebraic element over $k$
 is separable.  
 \end{note*}
 
If $\Omega$ is the algebraic closure of $k$ and $L$ the maximal
separable\pageoriginale subfield of $\Omega$ then $\Omega / L$  is purely
inseparable. $\Omega$ coincides with $L$ in case $k$ has characteristic
zero. But it can happen that $L$ is a proper subfield of $\Omega$.  

Let $K /k$ be an algebraic extension and $L$ the maximal separable
subfield. Consider the exponents of all elements in $K$. Let $e$ be
the maximum of these if it exists. we call $\underline{e}$ the
\textit{exponent} of the extension $K/k$. It can happen that
$\underline{e}$ is finite but $K/L$ is infinite. 

If $K/k$ is a finite extension then $K /L$ has degree $p^f$ so that
the maximum $e$ of the exponents of elements of $K$ exists. If
$\underline{e}$ is the exponent of $K/k$ then  
$$
e \leq f
$$

It can happen that $e < f$. For instance let $k$ have characteristic
$p \neq 0$ and let $ \alpha \in k$ be not a $p^{\rm th}$ power in $k$. Then $k
(\alpha^{ 1 /p})$ is of degree $p$ over $k$. Let $\beta$ in $k$ be
not a $p^{\rm th}$ power in $k$. Then $k(\alpha^{ 1/p} , \beta^{ 1/p})$ is of
degree $p^2$ over $k, \beta \not\in k(\alpha^{ 1/p})$ and for every
$\lambda \in k(\alpha^{ 1/p}$, $\beta^{ 1/p})$, $\lambda ^p \in k$. 

We may for instance take $k(x, y)$ to be the field of rational
functions of two variables and $K = k (x^{ 1/p} , y^{ 1/p})$. Then $(K
: k(x , y)) = p^2$ and $\lambda^p \in k (x, y)$ for every $\lambda \in
K$.  

\section{Perfect fields}\label{c2:s5} % section 5

Let $k$ be a field of characteristic $p > 0$. Let $\Omega$ be its
algebraic closure. Let $\omega \in k$. Then there is only one element
$\omega' \in \Omega$ such that $\omega'^{p} = \omega$. We can therefore
write $\omega^{ 1/p}$ without any ambiguity. Let $k^{p^{-1}}$ be the
field generated in $\Omega /k$ by the $p^{\rm th}$ roots of all elements of
$k$. Similarly from\pageoriginale $k^{p^{-2}}, \ldots $ Let  
$$
K = \bigcup_{n \geq 0 }k^{ p^{-n}}
$$

Obviously $K$ is a field; for if $\alpha$, $\beta \in K$, $\alpha$,
$\beta \in k^{ p^{-n}}$ for some large $n$. We denote $K$ by $k^{
  p^{-\infty}}$   

We shall study $k^{ p^{-\infty}}$ in relation to $k$ and $\Omega$ . $k^{
  p^{-\infty}}$ is called the \textit{ root field of }$k$.  

Let $\omega \in k^{ p^{-\infty}}$. Then $\omega \in k^{ p^{-\infty}}$ for some
$n$ so that $\omega^{ p^\infty} \in k $ or $\omega$ is purely
inseparable. On the other hand let $\omega \in \Omega$ be purely
inseparable. Then $\omega^{ p^n} \in k$  for some $n$ i.e., $\omega
\in k^{ p^{-\infty}} \subset k^{ p^{-\infty}}$. Thus  
\begin{enumerate}[1)]
\item $k^{ p^{-\infty}}$ \textit{ is the largest purely inseparable
  subfield of } $\Omega / k$. 

Therefore every automorphism of  $\Omega / k$ is identity on $k^{
  p^{-\infty}}$. The set of elements of $\Omega$ which are fixed under all
the $k-$ automorphisms of $\Omega /k$ form a field called the fixed of
$G( \Omega / k)$. Since every such element $\omega$, fixed under
$G(\Omega / k)$ is purely inseparable, $\Omega \in k^{
  p^{-\infty}}$. Hence  

\item $k^{ p^{-\infty}}$ \textit{is the fixed field of the group of $k-$
  automorphism of $\Omega /k$.}  

Let $f(x)$ be an irreducible polynomial in $k^{ p^{-\infty }}[x]$. We
assert that this is separable. For if not $f(x) \in k^{
  p^{-\infty}}[x^p]$. Thus $f(x) = a_o + a_1 x^p + \cdots + a_n
x^{np}$. Since $a_i \in k^{ p^{-\infty }} [X^p]$, it is in some $k^{
  p^{-t }}$ and so $a_i = b_i^p$ for $ b_i \in k^{ p^{-\infty
}}$. Hence  
$$
f(x) = (b_o + b_1 x + \cdots + b_n x^n)^p
$$
which is contradicts the fact that $f(x)$ is irreducible. Hence 

\item $\Omega /k^{ p^{-\infty }}$\pageoriginale \textit{is a separable extension.}

We now make the 

\begin{defi*}%Defn 0
A field $k$ is said to be perfect if every algebraic extension
  of $k$ is separable.  
\end{defi*}

It follows from the definition that
\begin{enumerate} [1)]
\item \textit{ An  algebraically closed field is perfect }
\item \textit{ A field of characteristic zero is perfect. }

We shall now prove

\item \textit{ A field $k$ of characteristic $p > 0$ is perfect if and
  only if $k = k^{ p^{-\infty }}$.} 

Let $k$ have no inseparable extension. Then for $a \in k$, $a^{ 1/ p}
\in k$ also; for, otherwise $k(a^{1/p})$ is inseparable over
$k$. Thus $ k =  k^{ p^{-1}} =  \cdots = k^{ p^{-\infty }}$. The
converse has already been proved.  

We deduce immediately 

\item \textit{ A finite field is perfect. } 

For if $k$ is a finite field of characteristic $p > r$ then $a \to
a^p$ is an automorphism of $k$.  

\item \textit{ Any algebraic extension of a perfect field is perfect.} 

For let $K/k$ be algebraic and $k$ be perfect. If $\alpha$ is
inseparable over $K$, then it is already so over $k$.  

An example of an imperfect field is the field of rational functions of
one variable $x$ over a finite field $k$. For if $k$ has characteristic
$p$, then $x^{ 1/p} \not\in k(x)$ and $k(x^{1/p})$ is a purely inseparable
extension over $k(x)$.  
\end{enumerate}
\end{enumerate}

\medskip
\noindent{\textbf{Note 1.}}
If $\alpha \in \Omega$ is inseparable over $k$, it is not true that it
is inseparable over every intermediary field, whereas this is
true\pageoriginale if $\alpha$ is separable.    

\medskip
\noindent{\textbf{Note 2.}}
If $K/k$ is algebraic and $K \cap k^{ p^{-\infty }}$ contains $k$
properly then $K$ is an inseparable extension. But the converse of
this is not true, that is, if $K/k$ is an inseparable extension, it
can happen that there are no elements in $K$ which are purely
inseparable over $k$. We give to this end the following example due to
Bourbaki.  

Let $k$ be a field of characteristic $p > 2$ and let $f(x)$ by in
irreducible polynomial  
$$
f(x) = x^n + a_1 x^{ n -1} + \cdots + a_n
$$
in $k[x]$. If $\alpha_1, \ldots , \alpha_t$ are the distinct roots of
$f(x)$ in $\Omega$ then  
$$
f (x) = \Big\{ (x - \alpha_1 ) \ldots  (x - \alpha_t )\Big\}^{p^e}
e\geq 1. 
$$
where $n = t. \cdot p^e$. Put $\phi (x) = f(x^p)$. Then
$$
\phi (x) = \big\{ (x^p - \alpha_1) \ldots (x^p - \alpha_t)\}^{p^e} 
$$
If $\beta_i = \alpha_i^{ 1/p}$ then 
$$
\phi (x) =\big\{ (x - \beta_1) \ldots (x - \beta_t)\big\}^{ p^{e+1}}  
$$
and $\beta, \ldots , \beta_t$ are distinct since $\alpha_1 \ldots
\alpha_t$ are distinct. Suppose $\phi (x)$ is reducible in $k[x]$ and
let $\psi (x)$ be an irreducible factor of $\phi (x)$  in $k[x]$. Then  
$$
\psi (x) = \Big\{  (x - \beta_1) \cdots (x - \beta _t)\Big\}^{p^\mu}
$$
for $\mu \geq 0$ and $\ell \leq t$. (This is because roots of $\psi
(x)$ occur with the same multiplicity). We can write  
$$
\psi (x) = \big\{ (x^p - \alpha_1)\ldots (x^p - \alpha_\ell)\big\}^{p^{ \mu -1}}
$$
$(\mu$ has to be $\geq 1)$ since otherwise, it will mean that $(x -
\alpha_1) \cdots (x - \alpha_t) \in k [x])$. Now this will mean that
$\phi (x) = \psi (x)$. $W(x)$ in $ k [ x^p]$ so that $\ell = t$. Hence  
$$
\psi (x) = \{ (x - \beta_1) \cdots (x - \beta_t) \} ^{p^\mu}, \mu \geq 1.
$$\pageoriginale

Since $\psi (x)$ is irreducible and
$$
\psi (x) = \{ (x^p - \alpha_1) \ldots (x ^p - \alpha_t) \} ^{p^{\mu-1}} 
$$
we see that $\mu -1 =e $ or $\mu = e + 1$. Thus
$$
\phi (x) = f(x^p) = \{ \psi (x) \}^p
$$

Thus if $f(x^p)$ is reducible, it is the $p^{th}$ power of an
irreducible polynomial. In this case $a_i = b^p_i, b_i \in k, i = 1,
\ldots n$. 

Conversely if $a_i = b^p_i , b_i \in k, i = 1, \ldots n$. then
$f(x^p)$ is reducible. Hence $f(x^p)$ is reducible $ f(x) \in k^p
[x]$. 

Let $k$ now be a field of characteristic $p > 2$ given by $k = \Gamma
(x , y)$, the field of rational functions in two variables $x$, $y$ over
the prime field $\Gamma$ if $p$ elements. Consider the polynomial  
$$
f(z) = z^{2p} + x z^{p} + y,
$$
in $k[z]$. Since $x^{1/p} , y^{1/p}$ do not lie in $k,f(z)$ is
irreducible in $k$. Let $\vartheta$ be a root of $f(z)$. Then
$(k(\vartheta):k) = 2p$. Let $\beta$ be in $k(\vartheta)$ and not in
$k$ such that $\beta^p \in k$. Then $k(\vartheta) \supset k(\beta)
\supset k$. Also $(k(\beta) : k) = p$. In $k(\beta)[x]$ the polynomial
$f(z)$ cannot be irreducible, since $\big( k (\vartheta) : k( \beta
)\big ) = 2$. It is reducible and so $k(\beta)$ will contain $x^{1/p}$
and $y^{1/p}$. But then $k(x^{1/p} , y^{1/p}) \supset k(\beta)$.  

Thus 
$$
p^2 = \big( k(x^{1/p} , y^{1/p}) : k \big) \le \big( k(\beta) : k \big
) \le \big(\vartheta) : k \big) = 2 p 
$$
but this is impossible. Thus there is no element $\beta$ in $k
(\vartheta)$ with $\beta^p \in k$. All the same $k(\vartheta)$ is
inseparable. 
\begin{enumerate} 
\renewcommand{\labelenumi}{\theenumi)}
\setcounter{enumi}{5}
\item If $k$ \textit {is not perfect then} $k^{p^{-\infty}}$
  \textit{is an infinite extension of} $k$. 
\end{enumerate}

For, if $k^{p^{-\infty}} / k$ is finite then, since $k^{p^{-\infty}} =
\bigcup \limits_{n}k^{p^{- n}}$ we have $k^{p^{- n }} = k^{p - (n +
  1)}$ for some $n$, 

 Applying\pageoriginale the mapping $a \rightarrow a^{a^{p^n}}$
we find $k = k^{p^{-1}}$. But this is false. Thus 
$$
\bigg( k^{p^{-(n+1)}} : k^{p^{-n}} \bigg) > 1
$$
which proves our contention.

\section{Simple extensions}\label{c2:s6}%[scc 6]

An algebraic extension $K/k$ is said to be \textit{simple} if there is
an $\omega \in K$ such that $K = k(\omega)$. Obviously $(K : k)$ is
finite. We call $\omega$ a \textit{primitive element} of $K$. The
primitive element is not unique for, $\omega + \lambda$, $\lambda \in
k$ also is primitive. We now wish to find conditions when an algebraic
extension would be simple. We first prove	 

\begin{lemma*}
Let $k$ {\em be an infinite field and} $\alpha , \beta$ {\em elements
  in an algebraic closure} $\Omega$ of $k$ {\em such that} $\alpha$
{\em is separable over} $k$. Then $k(\alpha , \beta)$ {\em is a simple
  extension of} $k$. 
\end{lemma*}

\begin{proof}
Let $f(x)$ and $\phi (x)$ be the irreducible polynomials of $\alpha$
and $\beta$ respectively in $k[x]$, so that  
\begin{align*}
f(x) & = (x - \alpha_1) \cdots (x - \alpha_n) \\
\phi(x) & = (x - \beta_1) \cdots (x - \beta_m).
\end{align*}
\end{proof}

Since $\alpha$ is separable, $\alpha_1 , \ldots \alpha_n$ are all
distinct. We shall put $\alpha = \alpha_1$, and $\beta =
\beta_1$. Construct the linear polynomials  
$$
\beta_i + X \alpha_j (i = 1, \ldots , m; j=1, \ldots , n)
$$

These $mn$ polynomials are in $\Omega$, the algebraic closure of $k$
and since $k$ is infinite, there exists an element $\lambda \in k$
such that  
$$
\beta_i + \lambda \alpha_j \neq \beta_i, + \lambda \alpha_j, \alpha \neq J' 
$$
 
Put
$$
\gamma = \beta_1 + \lambda \alpha_1 = \beta + \lambda \alpha
$$
Obviously\pageoriginale $\lambda$ can be chosen so that $\gamma \neq 0$

Now $k(\gamma) \subset k (\alpha , \beta)$. We shall will now show
that $\alpha \in k(\gamma)$. From definition of $\gamma$, $\beta$ also
will be in $k(\gamma)$ and that will mean  
$$
k(\gamma) \subset k(\alpha , \beta) \subset k(\gamma).
$$

In order to do this consider the polynomial $\phi (\gamma - \lambda
. x)$in  $k (\gamma)[x]$. Also it vanishes for $x =
\alpha$. Furthermore by our choice of $\lambda , \phi(\gamma - \lambda
\alpha_i) \neq 0$ for $i > 1$. In the algebraic closure $\Omega$,
therefore, $f(x)$ and $\phi (\gamma - \lambda x)$ have just $x -
\alpha$ as a factor. But $f(x)$ and $\phi (\gamma - \lambda x)$ are
both polynomials in $k(\gamma)[x]$. So $x - \alpha$ is the greatest
common divisor of $\phi (\gamma - \lambda x)$ and $f(x)$ in
$k(\gamma)[x]$. Thus $\alpha \in k (\gamma)$. Our lemma is
demonstrated. 

We have therefore

\begin{coro*}
If $\alpha_1 , \ldots , \alpha_n$ are separably algebraic
  and $\beta \in \Omega $ then\break $k(\beta , \alpha , \ldots
\alpha_n)$ is a simple extension. 
\end{coro*}


We deduce immediately

\begin{coro*}
 A finite separable extension is simple.
\end{coro*}

Let $\Gamma$ be the field of rational numbers and $\Gamma (\omega ,
\rho)$ the splitting field of the polynomial $x^3 - 2 $ in $\Gamma
       [x]$. Then $\Gamma (\omega , \rho)$ is simple. A primitive
       element $\gamma$ is given by $\omega + \rho = \gamma$.  Then
       $\Gamma(\gamma)$ is of degree $6$ over $\Gamma$. It is easy to
       see that $\gamma$ has over $\Gamma$ the minimum polynomial  
$$
(x^3 - 3x - 3)^2 + 3x (x+1) (x^3 - 3x - 3) + 9x^2 (x + 1)^2
$$

Let now $K/k$ be a finite extension and $L$ the maximal separable
subfield of $K/k$. Then ($K : L) = p^f$ the degree of inseparability
and $(L : k) = d$ the reduced degree. If we consider the exponents of
elements of $K$, these have a maximum $e$ and  
$$
e \le f.
$$

We had\pageoriginale given an example of $e < f$. We shall now prove
the  

\begin{thm}\label{c2:thm5}%%%% 5
If $e$ is the exponent and $p^f$ the degree of
  inseparability of finite extension $K$ of $k$, then $e = f
\Longleftrightarrow K/k$ is simple. 
\end{thm}

\begin{proof}
Let $K = k(\omega)$. Let $K_o$ be the maximal separable subfield of
$K/k$. Now $(K : K_o) = p^f$. But $p^f$ is degree of $\omega$. Thus $e
= f$ 
\end{proof}

Let now $K /k$ be a finite extension and $e = f$. There exists then a
$\omega$ in $K$ such that $\omega^{p^e}$ is separable and in $K_o$ but
for no, $t < e$ \, $\omega^{p^t}$ is in $K_o$. Thus $K = K_o (\omega)$. $K_o$
being a finite separable extension by our lemma, $K_o = k(\beta)$ for
$\beta$ separable. Thus  
$$
K = k (\omega, \beta ).
$$

Using the lemma again, our contention follows.

We now investigate the number of intermediary fields between $K$ and
$k$ where $K$ is an algebraic extension of $k$. Let us first consider
a simple extension $K = k(\omega)$. Let $\phi (x)$ be the minimum
polynomial of $\omega$ in $k[x]$. Let $L$ be any intermediary
field. Let $f(x)$ be the minimum polynomial of $\omega$ over $L$. Then
$f(x)$ divides $\phi(x)$. Let $f(x)$ have coefficients $a_0 , \ldots
a_n$ in $L$. Put $L_f$ the field $k(a_0 , \ldots , a_n)$. Then $f(x)$
is minimum polynomial of $\omega$ over $L_f$. Thus  
$$
(K : L) = (K : L_f)
$$

But $L_f \subset L$. This proves that $L = L_f$, and so for every
intermediary field there is a unique divisor of $\phi (x)$. Since
$\phi(x)$ has in $\Omega$ only finitely many factors, $K/k$ has only a
finite number of intermediary fields. 

We will now prove that the converse is also true. We shall
assume\pageoriginale $k$ is infinite.  

Let now $K/k$ be an algebraic extension having only a finite number of
intermediary fields. Let $\alpha$, $\beta \in K$. Consider the elements
$\alpha + \lambda \beta$ for $\lambda \in k$. Since $k$ is infinite,
the fields $k(\alpha + \lambda \beta)$ are infinite in number and
cannot be all distinct. Let for $\lambda = \lambda_1$, $\lambda_2 \quad
\lambda_1 \neq \lambda_2$ 
$$
k(\alpha + \lambda_1 \beta) = k(\alpha + \lambda_2 \beta) =
k(\gamma). 
$$

Then $\alpha + \lambda_1 \beta , \alpha + \lambda_2 \beta$ are in
$k(\gamma)$. Thus $(\lambda_1 - \lambda_2 ) \in k (\gamma)$. Hence
$\beta \in k(\gamma)$ because $\lambda_1 - \lambda_2 \in k$. This
means that $\alpha \in k(\gamma)$. 

Therefore
$$
k(\alpha , \beta) \subset k(\gamma) \subset k (\alpha , \beta).
$$

Hence every subfield of $K$, generated by 2 and hence by a finite
number of elements is simple. Let $K$ be a maximal subfield of $K/k$
which is simple. (This exists since $K/k$ has only finitely many
intermediary fields).  Let $K_0 = k(\omega)$. Let $\beta \in K$ and
$\beta \notin K_0$. Then $k(\gamma) = k(\omega , \beta) \subset K$ and
$K_0 \subset k(\gamma)$ contradicting maximality of $K_0$. Thus
$\beta(\gamma) = K$. We have proved 


\begin{thm}\label{c2:thm6}%the 6
$K/k$ is simple $\Longleftrightarrow K/k$ has only
    finitely many intermediary fields. 
\end{thm}

We deduce
\begin{coro*}
If $K/k$ is simple, then every intermediary field is simple.
\end{coro*}

\begin{Note}%%% 1
We have the fact that if $K/k$ is infinite there exist infinitely many
intermediary fields. 
\end{Note}

\begin{Note}%%% 2
Theorem~\ref{c2:thm6} has been proved on the assumption that $k$ is
an infinite field. If $k$ is finite the theorem is still true and  we
give\pageoriginale a proof later.  
\end{Note}

\section{Galois extensions}\label{c2:s7}%[sec 7]

Let $K/k$ be an algebraic extension and $G$ the group of automorphism
of $K$ which are trivial on $k$. Let $L$ be the subset of all elements
of $K$ which are fixed by $G$. $L$ is then a subfield of $K$ and is
called the \textit{fixed field} of $G$. We shall now consider the
class of algebraic extensions $K/k$ which are such that the group
$G(K/k)$ of automorphisms of $K$ which are trivial on $k$, has $k$ as
the fixed field. We call such extensions \textit{galois extensions},
the group $G (K/k)$ itself being called the \textit{galois group} of
$K/k$. 

We now prove the 

\begin{thm}\label{c2:thm7}%the 7
$k$  is the fixed field of the group of $k$ automorphisms of
   $K \Longleftrightarrow K/k$ is a normal and separable
    extension. 
\end{thm}

\begin{proof} 
Let $k$ be the fixed field of the group $G(K/k)$ of $k$-automorphisms of
$K$. Let $\omega \in K$. Let $\omega_1 (= \omega), \ldots \omega_n$ be
all the distinct conjugates of $\omega$ that lie in $K$. Consider the
polynomial  
$$
f(x) = (x - \omega_1) \cdots (x - \omega_n)
$$

If $\sigma$ is an element of $G(K/k), \sigma$ permutes $\omega_1 ,
\ldots , \omega_n$ so that $\sigma$ leaves the polynomial $f(x)$
unaltered. The coefficients of this polynomial are fixed under all
elements of $G$ and hence since $k$ is the fixed field $G, f(x) \in
k[x]$. Hence the minimum polynomial roots of the minimum polynomial of
$\omega$, since $\omega_1 , \ldots \omega_n$ are conjugates. Thus
$f(x)/\phi(x)$. Therefore $K$ splitting field of $\phi(x)$, $\phi(x)$
has all roots distinct. Thus $K/k$ is normal and separable. 
\end{proof}

Suppose\pageoriginale now $K/k$ is normal and separable. Consider the
group $G(K / k)$ of $k$-automorphisms of $K$. Let $\alpha \in
K$. Since $K/k$ is 
separable, all conjugates of $\alpha$ are distinct. Also since $K/k$
is normal $K$ contains all the conjugates. If $\alpha$ is fixed under
all $\sigma \in G(K/k)$, then $\alpha$ is a purely inseparable element
of $K$ and hence is in $k$. 

Our theorem is thus proved.

We thus see that galois extensions are identical with extension fields
which are both normal and separable. 

Examples of Galois extensions are the splitting fields of polynomials
over perfect fields. 

Let $k$ be a field of characteristic $\neq 2$ and let $K =
k(\sqrt{\alpha})$ for $\alpha \in k$ and $\sqrt{\alpha} \notin
k. (\sqrt{\alpha})^2 = \alpha \in k$. Every element of $K$ is uniquely
of the form $a+\sqrt{\alpha} \cdot b$, $a, b \in k$. If $\sigma$ is an
automorphism of $K$ which is trivial on $k$, then its effect on $K$ is
determined by its effect on $\sqrt{\alpha}$. Now 
$$
\alpha = \sigma \left\{ (\sqrt{\alpha})^2 \right\} = \sigma (\sqrt{\alpha})
. \sigma (\sqrt{\alpha}) 
$$
or that $\sigma (\sqrt{\alpha})/ \sqrt{\alpha} = \lambda$ is such that
$\lambda^2 = 1$. Since $\lambda \in K$, $\lambda = \pm 1$. Thus $\sigma$
is either th identity or the automorphism 
$$
\sigma (\sqrt{\alpha}) = - \sqrt{\alpha}
$$

Thus $G(K/k)$ is a group of order 2. $K/k$ is normal and separable.

We shall obtain some important properties of galois extensions.	
\begin{enumerate}[1)]
\item \textit{If $K/k$ is a galois extension and $k \subset L
  \subset K$, then $K/L$ is a galois extension also}. 

For, $K/L$ is clearly separable. We had already seen that it
is\pageoriginale normal. 

If we denote by $G(K /L)$ the galois group of $K$ over $L$, than
$G(K/L)$ is a subgroup of $G(K/k)$.  

\item \textit{If} $k \subset L_1 \subset L_2 \subset K$ then
  $G(K/L_2)$ \textit{is a subgroup of } $G(K / L_1)$. This is
  trivial. 

\item If $\{ K_\alpha \}$ \textit{is a family of galois extensions
  of} $k$ \textit{contained in} $\Omega$ then $\bigcap \limits_\alpha
  K_\alpha$ and $k (\bigcup \limits_{\alpha} K_\alpha)$ \textit{are
    galois}. 

This follows from the fact that this is already true for normal and
also for separable extensions. 

\item If $K/k$ \textit{is galois and} $L$ and $L'$ \textit{are two
  intermediary fields of} $K/k$ \textit{which are conjugate over} $k$,
  then $G(G/L)$ and $G(K/L')$ \textit{are conjugate subgroups of
  } $G(K/k)$ \textit{and conversely}. 
\end{enumerate}

\begin{proof}
Since $L$ and $L'$ are conjugate over $k$ let $\sigma$ be an
automorphism of $K/k$ so that $\sigma L = L'$. Let $\tau \in
G(K/\sigma L)$. Then for every $\omega \in \sigma L$ 
$$
\tau \omega = \omega
$$

But $\omega = \sigma \omega'$ for $\omega' \in L$. Thus
$$
\sigma^{-1} \tau \sigma \omega' = \omega'
$$

Since this is true for every $\omega' \in L$, it follows that  
$$
\sigma^{-1} G(K/\sigma L )\sigma \subset G(K/L)
$$

In a similar manner one proves that $\sigma G(K/L)\sigma^{-1} \subset
G(K/ \sigma L)$ which proves our contention. 
\end{proof}


Conversely suppose that $L$ and $L'$ are two subfields such that
$G(K/L)$ and $G(K/L')$ are conjugate subgroups of $G(K/k)$. 
Let $G(K/L') = \sigma^{-1}\break G(K/L) \sigma$. Let $\omega \in L'$ and $\tau
\in G(K/L)$. Then $\sigma^{-1} \tau \sigma \in G(K/L')$\pageoriginale and so  
$$
\sigma^{-1} \tau \sigma \omega = \omega
$$
or $\tau(\sigma \omega) = \sigma \omega$. This being true for all
$\tau$, it follows that $\sigma \omega \in L$ for all $\omega$ in
$L'$. Thus $\sigma L' \subset L'$. We can similarly prove that
$\sigma^{-1} L \subset L'$ which proves our statement. 

In particular let $L/k$ be a normal extension of $k$ which is
contained in $K$. Then $\sigma L = L$ for all $\sigma \in
G(K/k)$. This means that $G(K/L)$ is a normal subgroup of $G(K/k)$. On
the other hand if $L$ is any subfield such that $G(K/L)$ is a normal
subgroup of $G(K/k)$ then by above $\sigma L = L$ for all $\sigma \in
G(K/k)$ which proves that $L/k$ is normal. Thus 
\begin{enumerate}
\item [5)]\textit{Let} $k \subset L \subset K$. Then $L/k$ \textit{is
  normal} $\Longleftrightarrow G(K/L)$ \textit{is a normal subgroup
  of} $G(K/k)$ 

\item [6)] If $L/k$ is normal, then $G(K/k) \simeq G(L/k) / G(K/L)$.
\end{enumerate}

Let $\sigma \in G(K/k)$ and $\bar{\sigma}$ the restriction of $\sigma$
to $L$. Then $\bar{\sigma}$ is an automorphism of $L/k$ so that
$\bar{\sigma} \in G(L/k)$. Now $\sigma \rightarrow \bar{\sigma}$ is a
homomorphism of $G(K/k)$ into $G(L/k)$. For 

$\overline{\sigma \tau} \omega = \sigma \tau \omega = \sigma (\tau
\omega) = \bar{\sigma} \bar{\tau} \omega$ for all $\omega \in L$. Thus
$\bar{\sigma}\bar{\tau} = \overline{\sigma \tau}$ 

The homomorphism\pageoriginale is \textit{onto} since every
automorphism of $L/k$ can be extended into an automorphism of
$K/k$. Now $\bar{\sigma}$ is identity if and only if  
$$
\bar{\sigma} \omega = \omega
$$
for all $\omega \in L$. Thus $\sigma \in G(K/L)$. Also every $\sigma
\in G(K/L)$ has this property so that the kernel of the homomorphism is
$G(K/L)$.  

Let $K/k $ be a finite galois extension. Every isomorphism of $K/k$ in
$\Omega$ is an an automorphism. Also $K/k$ being separable, $K/k$ has
exactly ($K : k$) district isomorphisms. This shows that  
$$
(K : k) = \text{order of } G.
$$

We shall now prove the converse	
\begin{enumerate}
\item [7)] \textit{If $G$ is a finite group of automorphisms
  of $K/k$ having $k$ as the fixed field then ($K :
  k$) = order of $G$}.  
\end{enumerate}

\begin{proof}
Let $\sigma_1 , \ldots , \sigma_n$ be the $n$ elements of $G$ and
$\omega_1, \ldots \omega_{n+1}$ any $n+1$ elements of $K$. Denote by
$V_K$ the vector space over $K$ of $n$ dimensions formed by $n$- tuples
$(\alpha_1 , \ldots , \alpha_n)$. Define $n+1$ vectors $\Omega_1,
\ldots , \Omega_{n+1}$ by  
$$
\Omega_i = (\sigma_i (\omega_i), \ldots , \sigma_n (\omega_i)) i = 1, \ldots 
, n+1 
$$

Among these vectors there exists $m \le n$ vectors linearly
independent over $K$. Let $\Omega_1 , \ldots , \Omega_m$ be
independent. Then 
$$
\Omega_{m+1} = \sum^m_{n=1} a_i \Omega_i  \quad a_i \in K
$$

This equation gives, for the components of the $\Omega's$,
$$
\sigma_\ell (\omega_{m+1}) = \sum^m_{1=i} a_i \sigma_\ell (\omega_i)
\alpha = 1, \ldots ,n. 
$$
\end{proof}

Since $\sigma_1 , \ldots , \sigma_n$ form a group then $\sigma_n
\sigma_1, \ldots , \sigma_n \sigma_n$ are again the
elements\pageoriginale $\sigma_1 , \ldots , \sigma_n$ in some
order. Thus  
$$
\sigma_h \sigma_{\ell} (\omega_{m+1}) = \sum^m_{i=1} \sigma_h (a_i | \sigma_h 
\sigma_\ell(\omega_i) 
$$
which means 
$$
\sigma_\ell(\omega_{m+1}) = \sum^m_{i=1} \sigma_h (a_i)\sigma_\ell
(\omega_i)\quad i= 1, \ldots , n 
$$
subtracting we have
$$
\sum^m_{i=1} (\sigma_h (a_i)-a_i \sigma_\ell (\omega_i) = 0 
$$
which means that
$$
\sum^m_{i=1} \big( \sigma_h (a_i) - a_i \big) \Omega_i = 0  
$$

From linear independence, it follows that $\sigma_h(a_i) = a_i$ for
all $i$. But $h$ is arbitrary. Thus $a_i \in k$. We therefore have by
taking $\sigma_1$ to be the identity element of $G$ 
$$
\omega_{m+1} = \sum^m_{i=1} a_i \omega_i
$$
$a_i$ are in $k$, not all zero. Hence
$$
(K : k) \le n.
$$

But every element of $G$ is an isomorphism of $K/k$. Thus 
$$
(K : k) \ge \text{ order of } G = n.
$$

Our assertion is established.

Suppose $K/k $ is a galois extension. For every subfield $L$ of $K/k$,
the extension $K/L$ is galois. We denote its galois group by $G(L)$
and this is a subgroup of $G(K /k)$. Suppose $g$ is any subgroup of
$G(K/k)$ and let $F(g)$ be its fixed field. Then $F(g)$ is a subfield
of $K$. The galois group $G(F(g))$ of $K/F(g)$ contains\pageoriginale
g. \textit{In   general one has only} 
$$
g \subset G(F(g))
$$

Let now $K/k$ be a \textit{finite} galois extension. Let $g$ be a
subgroup of $G(K/k) = G$ and $F(g)$, the fixed field of $g$. Then by
above  
$$
\bigg( K : F(g) \bigg) = \text{ order of } g.
$$
and so
$$
g = G\big( K/F(g) \big)
$$

If $g_1$ and $g_2$ are two subgroups of $G$ with $g_1 \subset g_2$
then $F(g_1)$ and $F(g_2)$ are distinct. For if $F(g_1)$ and $F(g_2)$
are identical, then by above $g_1 = G(K/F (g_2)) = g_2$. We thus have
the  

\medskip
\noindent{\textbf{Main Theorem}}(of finite galois theory).
 \textit{Let $K/k$ be a finite galois extension
    with galois group $G$. Let $M$ denote the class of all
    subgroups of $G$ and $N$ the class all subfield of
  $K/k$. Let $\phi$  be the mapping which assigns to every
    subgroup $g \in M$, the fixed field $F(g)$ of $g$ in
  $N$. Then $\phi $ is a mapping of $M$ onto $N$  which is
    biunivocal.} 

In order to restore this property even for infinite extensions, we
develop a method due originally to Krull. 

Let $K/k$ be a galois extension with galois group $G(K/k )$. For every
$\omega \in K$, we denote by $G_{\omega}$ the galois group
$G(K/k(\omega))$. This then is a subgroup of $G(K/k)$. We make
$G(K/k)$ into a topological group by prescribing the $\{G_\omega \}$
as a fundamental system of neighbourhoods of the identity element
$\underline{e} \in G(K/k)$. Obviously 
$\bigcap \limits_\alpha G_\alpha = (e)$. For $ \sigma \in
\bigcap_{\alpha} G_{\alpha} \, \Rightarrow \, \sigma
\alpha = \alpha$ for all $\alpha \in K$ so that $\sigma = e$. It is
easy to verify that $\{ G_\alpha \}$ satisfy the axioms\pageoriginale
for a fundamental system of open sets containing the identity elements 
$\underline{e}$.  

Any open set in $G$ is therefore a union of sets of type $\{ \sigma
G_\alpha \}$ or a finite intersection of such. Also since $G_{\alpha}$
are open subgroups they are closed; for, $\sigma G_\alpha$ is open for
all $\sigma $ and hence  
$$
\bigcup_{\sigma \neq e} \sigma G_{\alpha}
$$
is also open. Therefore $G_{\alpha} $ is closed. This proves that the
topology on $G$ makes it \textit{totally disconnected}. We call the
topology on $G$ the \textit{Krull topology}. 

If $g$ is a subgroup of $G$, $\bar{g}$ the closure of $g$ is also a
subgroup. We now prove the  

\begin{lemma*}
Let $g$ be a subgroup of $G$ a and $L$ its fixed field. Then
$$
G(K/L) = \bar{g} \quad \text{ (the closure of g)}.
$$
 \end{lemma*} 

 \begin{proof}
Let $\omega$ be an element in $K$ and $f(x)$ its minimum polynomial in
$k$. Consider $f(x)$ as a polynomial over $L$ and let $L'$ be its
splitting field over $L$. Then $L' /L$ is  a galois extension. The
restriction of elements of $g$ to $L'$ are automorphisms of $L'$ with
$L$ as fixed field (by definition of $L$). By finite galois theory
these are all the elements of the galois group of $L' /L$. This means
that every automorphism of $L'/L$ comes from an elements of $g$.  
 \end{proof} 
 
 Let $\sigma \in G(K/L)$. Let $\omega$ be any element in $K$ and $G$
 the group $G(K/k (\omega ))$. The restriction of $\sigma$ to $L'$ is
 an automorphism of $L'$ with $L$ as fixed field. There is thus an
 element $\tau \in g$, which has on $L'$ the same effect as
 $\sigma$. Hence $\tau^{-1} \sigma$ is identity on $L'$ and since
 $\omega \in L'$ we get 
 $$
 \tau^{-1} \sigma \omega = \omega
 $$
  This means that $\tau^{-1 } \sigma \in G_\omega$ by definition of
 $G_{\omega}$. Hence $\sigma^{-1} \tau \in G_\omega$\pageoriginale or
  $\tau \in 
 \sigma G_{\omega}$. But $\sigma G_{\omega}$ is an open set containing
 $\sigma$. Therefore since $\{ \sigma G_\omega \}$ for all $G_\omega$
 form a fundamental system of neighbourhoods of $\sigma$ we conclude
 that 
 $$
 \sigma \in \bar{g}.
 $$
  Thus $G(K/L) \subset \bar{g}$
 
 Let now $\sigma \in \bar{g}$. Then $\sigma G_\omega$ is a
 neighbourhood of $\sigma$ and so intersects $g$ in a non empty
 set. Let $\omega \in L$. Let $\tau \in \sigma G_\omega \cap g$. Let
 $\sigma'$ in $G_{\omega}$ such that  
 $$
 \sigma \sigma' = \tau
 $$
  By definition of $\tau , \tau \omega = \omega$. But $\tau \omega =
  \sigma \sigma ' \omega = \omega$. Therefore $\sigma \sigma ' \in
  G_{\omega}$. But $\sigma '$ is already is in $G_{\omega}$. Therefore
  $\sigma \omega = \omega$. Also $\omega$ being arbitrary 
 $$
 \sigma L = L
 $$
 which means that $\sigma \in G(K / L)$. Thus
 $$
 \bar{g} \subset G(K/L)
 $$
 and our contention is established.
 
 From the lemma, it follows that if $L$ is an intermediary field, the
 galois group $G(K/L)$ is a closed subgroup of $G(K/k)$. On the other
 hand if $g$ is a closed subgroup of $G$ and $F(g)$ its fixed field
 then  
 $$
 G \big( K/F(g) \big) = \bar{g} = g .
 $$
 we have hence the fundamental

 \begin{thm}\label{c2:thm8}%the 8
Let $K/k$ be a galois extension and $G(K/k)$ the
  galois group with the Krull - topology. Let $M$ denote the
  set of closed subgroups of $G$ and $N$ the set of
  intermediary fields of $K/k$. Let $\phi$  be the mapping which
  assigns to every $g \in M$, the fixed field $F(g)$ of\pageoriginale
  $g$ in $N$. Then $\phi$ is a biunivocal mapping of $M$ on $N$.  
 \end{thm} 
 
 Suppose now that $K/k$ is a galois extension and $L$ is an
 intermediary field. Let $G(K/k)$ and $G(K/L)$ be the galois
 groups. On $G(K/L)$ there are two topologies, one that is induced by
 the topology on $G(K/k)$ and the other the topology that $G(K/L)$
 possesses as a galois extension.\pageoriginale If $L'$ is a subfield
 of $K/L$ so  that $L' / L$ is finite, then $G(K/L')$ is an open set
 in the  inherent topology on $G(K/L)$. On the other hand $L' =
 L(\omega)$  since $L'/L$ is a separable extension.   
 
 Thus
 $$
 G(K/L') \subset G_{\omega} \cap G(K/L)
 $$
 which proves that the two topologies are equivalent. Here we have
 used the fact that a finite separable extension of $L$ is simple. We
 gave already seen the truth of this statement if $L$ is infinite. In
 case $L$ is finite it is proved in the next section. 
 `
 In a similar manner if $K/k$ is galois and $L$ is a normal extension
 of $k$ in $K$, then on $G(L/k)$ there are two topologies, one the
 inherent one and the other the topology of the quotient group $G(K/k)
 / G(K/L)$. One can prove that the two topologies are equivalent.  
 
 We call an extension $K/k$ \textit{abelian} or \textit{solvable}
 according as $G(K/k)$ is abelian or a solvable group. If $K/k$ is a
 galois 	extension and $G(K/k)$ its galois group with the Krull
 topology let $H$ denote the closure of the algebraic commutator
 subgroup of $G(K/k)$. $H$ is called the topological commutator
 subgroup. If $L$ is its fixed field, then since $H$ is normal, $L/k$
 is a galois extension. Its galois group is isomorphic to $G/H$ which
 is abelian. From the property of the commutator subgroup, it follows
 that $L$ is the  {\textit maximal abelian subfield} of $K/k$. 

\section{Finite fields}\label{c2:s8}%sec 8

Let $K$ be a finite field of $q$ elements, $q=p^n$ where $p$ is the
characteristic of $K$. Let $\Gamma$ be the prime field of $p$
elements. Then 
$$
(K:\Gamma )=n
$$

$K^*$ the group of non-zero elements of $K$ is an abelian group of
order $q-1$. For $\alpha \in K^*$ we have 
$$
\alpha^{q-1}=1
$$ 
1 being the unit element of $K$. The $q-1$ elements of $K^*$ are
roots of $x^{q-1}-1$. Also 
$$
x^{q-1}-1=\prod_{\alpha \in K^*}(x-\alpha).
$$

Let $\alpha \in K^*$. Let $\underbar{d}$ be its order as an element of
the finite group $K^*$. Then $\alpha ^d=1$. Consider the polynomial
$x^d - 1$. It has in $K^*$ at most $\underbar{d}$ roots. Also
$d/q-1$. But  
$$
x^{q-1}-1=(x^d -1)(x^{q-1-d}+\cdots)
$$

Since $x^d-1$ and $x^{q-1-d}+\cdots$ both have respectively at most
$d$ and $q-1-d$ roots in $K^*$ and they together $q-1$ roots in $K^*$
it follows that for every divisor $d$ of $q-1$, $x^d-1$ has exactly $d$
roots in $K^*$. These roots from a group of order $\underbar{d}$. If
it is cyclic then there is an element of order $\underbar{d}$ and
there are exactly $\phi (d)$ elements of order $\underbar{d}$. Also 
$$
\sum_{d/q-1}\phi (d)=q-1
$$
which proves that for every divisor $d$ of $q-1$ there are $\phi
(d)\geq 1$\pageoriginale elements of order $\underbar{d}$. Thus  
\begin{enumerate}[1)]
\item {\textit The multiplicative group of a finite field is cyclic.} 

Let $k$ be a finite of $q$ elements and $K$ a finite extension of $k$
of degree $n$. Then $K$ has $q^n$ elements. Since $K^*$ is cyclic, let
$\rho $ be a generator of $K^*$. Then 
$$
K=k(\rho)
$$
which proves

\item{\textit Every finite extension of a finite field is simple.} 

For a finite field of characteristic $p$, $a \to a^{p^e}$ is an
automorphism of $K$. Since $k$ has $q$ elements we have  
$$
a^q=a
$$
for every $a \in k$.

Now $a \to a^q$ is an automorphism of $K/k$ which fixes elements of
$k$. Call this automorphism $\sigma$. $\sigma$ is determined uniquely
by its effect on a generator $\rho$ of $K^*$. Consider the
automorphisms 
$$
1,\sigma , \sigma^2, \ldots , \sigma^{n-1}
$$ 
These are distinct. For,
$$
\sigma^i \rho = \sigma^{i -1}(\sigma \rho) =\sigma^{i-1}
(\rho^q)=\rho^{q^i} 
$$
Hence $\rho^q=1 \Longleftrightarrow q^i \equiv o(\mod q^n)$ or
$i=o$ (Since $i < n$).  
But $K/k$ being of degree $n$ cannot have more than $n$
automorphisms. We have 

\item {\textit The galois group of a finite extension of a finite
  field is cyclic.} 

The generator $\sigma$ of this cyclic group, defined by
$$
\sigma a=a^q
$$
is called the {\textit Frobenius automorphism.} It is defined without
any reference\pageoriginale to a generator of $K^*$. 

Let $L$ be an intermediary field of $K/k$. Then $(L:k)$ is a divisor
of $(K:k)=n$. If $d=(L:k)$ then $L$ has $q^d$ elements. Also since
$K/k$ has a cyclic galois group, there is one and only one subgroup of
a given order $\underbar{d}$. Hence 

\item {\textit The number of intermediary fields of $K/k$ is equal to
  the number of divisors of $n$.} 
\end{enumerate}


