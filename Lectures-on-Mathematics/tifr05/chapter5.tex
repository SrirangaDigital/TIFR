\chapter{Composite extensions}\label{chap5}%che 5

\section{Kronecker product of Vector spaces}\label{c5:s1}\pageoriginale%sec 1

Let $V_1$ and $V_2$ be two vector spaces over a field $k$ and $V_1 \times
V_2$ their cartesian product. If $W$ is any vector space over $k$, a
bilinear function  $f(x,y)$ on $V_1 \times  V_2$ is, by definition, a
function on  $V_1 \times  V_2$  into $W$ such that for every $x \in V_1$ the
mapping $\lambda_x: y \to f(x,y)$ is a linear function on $V_2$ into
$W$ and for every $y \in V_2$ the function $\mu_y: x \to f(x,y)$ is a
linear function on $V_1$ to $W$. 

A Vector space $T$ over $k$ is said to be a \textit{Kronecker product}
or tensor product of $V_1$ and $V_2$ over $k$, if there exists a bilinear
function $\theta$ on  $V_1 \times  V_2$  into $T$ such that  
\begin{enumerate}[1)]
\item $T$ is generated by $\theta (V_1 \times  V_2)$ 

\item if $V_3$ is any vector space over $k$, then for every bilinear
  function $\varphi$ on  $V_1\times  V_2$  into $V_3$ there exists a
  linear function $\sigma$ on $T$ into $V_3$ such that $\sigma \theta =
  \varphi$. 
\end{enumerate}

This is shown by the commutative diagram
\[
\xymatrix@R=0.5cm{
V_1 \times V_2 \ar[rr]^{\varphi} \ar[dr]_{\theta} & & V_3\\
& T \ar[ur]_{\sigma}&
}
\]

We shall now prove the

\setcounter{thm}{0}
\begin{thm}\label{c5:thm1} %%% 1
For any two vector spaces $V_1$ and $V_2$ over $k$ there exists one
and upto $k$-isomorphism only one Kronecker product $T$ of $V_1$ and
$V_2$ over $k$. 
\end{thm}

\begin{proof}% proof 
The\pageoriginale uniqueness is easy to establish. For, let $T_1$ and
$T_2$ be two 
vector spaces satisfying the  conditions 1) and 2). Let $T_1$ be
generated by $\theta_1(V_1 \times V_2)$ and $T_2$ by $\theta_2(V_1 \times
V_2)$. Let $\sigma$ be the linear map of $T_1$ into $T_2$ such that
$\sigma \cdot \theta_1 = \theta_2$ and $\tau$ the linear map of $T_2$ into
$T_1$ such that $\tau\cdot\theta_2 = \theta_1$. Since $\theta_1(V_1 \times
 V_2)$ generates $T_1$ we see that $\tau\cdot\sigma $ is identity on
$T_1$. Similarly $\sigma \cdot\tau$ is identity on $T_2$. Thus $\sigma$
and $\tau$ are isomorphisms onto. 
\end{proof}

We now prove the existence of the space $T$.

Let $V$ be the vector space formed by finite linear combinations
$$
\sum a_{xy}  (x, y)
$$
$a_{xy}\in  k$ and $(x, y)  \in  V_1 \times  V_2$. Every bilinear
function $f$ on $V_1 \times V_2$ into $V_3$ can be extended into a
linear function $\bar{f}$ of $V$ into $V_3$ by the prescription 
$$
\bar{f} \Big(\sum  a_{xy}  (x,y)\Big) = \sum a_{x,y}  f(x,y)
$$

Let $W$ be the subspace of $V$ generated by elements of the type 
\begin{gather*} 
(x + x^1,y) - (x,y) - (x^1, y)\\
(x, y + y^1) - (x,y) - (x,y^1)\\
(a  x,y) - a(x, y)\\
(x, b  y) - b(x, y)
\end{gather*}
where $x$, $x^1 \in V_1$; $y$, $y^1\in  V_2$ and $a$, $b \in
k$. $W$ is independent of $V_3$. Also if $f$ is a bilinear function on
$V_1 \times V_2$, its extension $\bar{f}$ on $V$ vanishes on
$W$. Furthermore if $f$ is any linear function on $V$ vanishing on
$W$, its restriction to $V_1 \times V_2$ is a bilinear
function.\pageoriginale It is 
then clear that the space  of bilinear functions on $V_1\times V_2$ is
isomorphic to the space of linear functions on $V$ which vanish on
$W$. If $\sigma$ is any linear function on $V/W$ into $V_3$ and
$\theta$ the natural homomorphism of $V$ on $V/W$ then $\sigma \cdot \theta
$ is a linear function on $V$ vanishing on $W$. Also $\sigma
\rightarrow \sigma \cdot \theta$ is an isomorphism of the space of linear
functions on $V/W$ into $V_3$ on the space of linear functions on $V$
vanishing on $W$. $\theta$ is clearly a bilinear function on $V_1\times
V_2$ into $T = V/W$ and furthermore, by definition of $V$, $V/W$ is
generated by $\theta (V_1 \times V_2)$. Thus $T$ is the required
space. 

By taking $k$ itself as a vector space over $k$, we have  

\begin{thm}\label{c5:thm2} %theorem 2
The space of bilinear functions on $V_1\times V_2$ into $k$ is
  isomorphic to the dual of the Kronecker product $T$. 
\end{thm}

We denote by $V_1  \text{\textcircled{$x$}}_k  V_2$ the kronecker product
space. When there is no doubt about the field over which the kronecker
product is taken we will simply write $V_1 \text{\textcircled{$x$}} V_2$. 

If $T = V_1  \text{\textcircled{$x$}} V_2$ we denote by $x \text{\textcircled{$x$}}
y$ the element in $T$ which corresponds to $(x,y)$ by the bilinear
function $\theta$ on $V_1 \times V_2$ into $T$. Since $\theta ( V_1 \times
 V_2)$ generates $T$, every element of $T$ is of the form 
$$
\sum a_{x,y}  (x  \text{\textcircled{$x$}}  y) \quad a_{xy}  \in  k.
$$

Clearly
\begin{equation*}
\left \{
\begin{aligned}
& x  \text{\textcircled{$x$}}  o = o  \text{\textcircled{$x$}}  y = o 
  \text{\textcircled{$x$}}  o\\ 
& (x + x^1)  \text{\textcircled{$x$}}  y = x  \text{\textcircled{$x$}}  y + x^1 
  \text{\textcircled{$x$}}  y \\ 
& x  \text{\textcircled{$x$}} (y + y^1) = x  \text{\textcircled{$x$}} y + x
  \text{\textcircled{$x$}}  y^1 \\ 
& ax \text{\textcircled{$x$}}  y = a (x  \text{\textcircled{$x$}}  y)\\ 
& x  \text{\textcircled{$x$}}  \text{ by } = b (x  \text{\textcircled{$x$}}  y) 
\end{aligned}
\right.
\end{equation*}
with\pageoriginale obvious notations.

From our considerations it follows that in order to define a linear
function on $V_1 \text{\textcircled{$x$}}  V_2$ it is enough to define it on
elements of the type $x  \text{\textcircled{$x$}}  y$. Also for every such
linear function, there is a bilinear function on $V_1 \times V_2$. 

It is easy to see that the mapping $x  \text{\textcircled{$x$}}  y
\rightarrow y \text{\textcircled{$x$}}  x$ of $V_1  \text{\textcircled{$x$}}  V_2$
to $V_2 \text{\textcircled{$x$}}  V_1$ is an isomorphism onto. 

Suppose now that $V^*_1$ and $V^*_2$ are duals of $V_1$ and $V_2$
respectively over $k$. If $\sigma  \in  V^*_{1}$ and $\tau \in 
V_2^*$ and we define for $(x,y)  \in  V_{1}  \times  V_{2}$ the
function  
$$
\sigma.\tau(x,y) = \sigma x . \tau y, 
$$
then $\sigma.\tau$ is a bilinear function on $V_1 \times  V_2$ into
$k$. We shall now prove the  

\begin{thm}\label{c5:thm3}% theorem 3
If $\{x_\alpha \}$ is a base of $V_1$ over $k$ and $\{y_\beta \}$
  a base of $V_2$ over $k$, then $\{x_\alpha \text{\textcircled{$x$}} 
  y_\beta \}$ is a base of $V_1 \text{\textcircled{$x$}} V_2$ over $k$. 
\end{thm}

\begin{proof}% proof
We first prove that $\{ x_\alpha \text{\textcircled{$x$}}  y_\beta \}$ are
linearly independent over $k$. For all $\sum a_{\alpha\beta}
(x_\alpha  \text{\textcircled{$x$}}  y_\beta ) = o$ for $a_{\alpha\beta} \in
 k$, $a_{\alpha\beta} = o$ for all but a finite number of $\alpha$,
$\beta$. By the method of constructing the tensor product, it follows
that $\sum a_{\alpha\beta} (x_\alpha, y_\beta )$ is an element of
$W$. Let $\sigma$ and $\tau$ be elements of $V^*_1$ and $V^*_2$
defined respectively by  
\begin{align*}
\sigma (x_\alpha ) & = 1 \quad \text{ if } \alpha = \gamma \\
& = o  \quad \text{otherwise}
\end{align*}
\begin{align*}
\tauup (y_\beta ) & = 1 \quad \text{ if } \quad \beta = \delta\\
& = o \quad \text{ otherwise }
\end{align*}\pageoriginale
\end{proof}

Then $\sigma \cdot \tau$ is a bilinear function on $V_1 \times V_2$ and
hence vanishes on $W$. Thus 
$$
\sigma.\tau \Big(\sum a_{\alpha \beta} (x_\alpha  \text{\textcircled{$x$}} 
y_\beta)\Big) = o. 
$$

But the left side equals $a_{\gamma\delta}$. Thus all the coefficients
$a_{\alpha\beta}$ vanish. 

Next any element of $V_1  \text{\textcircled{$x$}}  V_2$ is a linear
combination of elements of the type $x  \text{\textcircled{$x$}}  y$, $x \in
V_1$, $y \in V_2$. But then $x = \sum a_\alpha x_\alpha$ and $y = \sum
b_\beta  y_\beta$ so that  
$$
x  \text{\textcircled{$x$}}  y = \Big(\sum a_\alpha  x_\alpha\Big) 
\text{\textcircled{$x$}}  \Big(\sum b_\beta  y_\beta \Big) 
$$
which equals $\sum a_\alpha b_\beta  (x_\alpha \text{\textcircled{$x$}} 
y_\beta )$. Our theorem is proved. 

We have incidentally the 

\begin{coro*} %corollary 0
If $V_1$ and $V_2$ are finite dimensional over $k$ then 
$$
\dim V_1  \text{\textcircled{$x$}}  V_2 = \dim V_1 \cdot \dim  V_2.
$$
\end{coro*}

Also since the dual of $V_1^*$ is isomorphic in a natural manner with
$V_1$ when $\dim \cdot V_1$ is finite, we get  

\begin{coro*}% corollary 0 
If $V_1$ and $V_2$ are finite dimensional over $k$ then $V_1 
    \text{\textcircled{$x$}}_k  V_2$ is isomorphic to the dual of the space of
    bilinear functions on $V_1 \times V_2$ into $k$. 
\end{coro*}

Let now $A_1$ and $A_2$ be two associative algebras over a field
$k$. We can form the Kronecker product $A = A_1 \text{\textcircled{$x$}}_k 
A_2$ of the vector spaces $A_1$ and $A_2$ over $k$. We shall now
introduce a multiplication into $A$ so as to make it into an
associative algebra. 

In order\pageoriginale to do this, observe that the multiplication
defined has to be 
a bilinear function on $A \times A$ into $A$. Since $A$ is generated by
elements of the type $x  \text{\textcircled{$x$}}  y$, it is enough to define
this bilinear function on elements of the type $(z,z^1)$ in $A \times
A$ where $z = x  \text{\textcircled{$x$}} y$ and $z^1 = x^1 \text{\textcircled{$x$}}
 y^1$. Put 
$$
f(z,z^1) = z \cdot z^1 = x \cdot x^1  \text{\textcircled{$x$}}  y \cdot y^1.
$$


Now since $(x,y) \rightarrow x  x^1\text{\textcircled{$x$}}  y y^1$ is a
bilinear function on $A_1 \times A_2$ into $A$, by our previous
considerations $z \rightarrow z\cdot z^1$ is a linear function on $A$ into
$A$. Similarly $z^1 \rightarrow z \cdot z^1$ is a linear function on
$A$. This proves that $f$ is bilinear and that the multiplication so
defined distributes addition. That the multiplication is associative
is trivial to see. 

$A$ is called the \textit{Kronecker product algebra}.

We obtain some very simple consequences from the definition.

$\alpha$) If $e_1$ and $e_2$ are respectively the unit elements of
the algebras $A_1$ and $A_2$ then $e_1 \text{\textcircled{$x$}}  e_2$  is the
unit element of $A_1 \text{\textcircled{$x$}}  A_2$. 

For, since $A = A_1 \text{\textcircled{$x$}}  A_2$ is generated by elements
$x \text{\textcircled{$x$}}  y$, it is enough to verify $(x  \text{\textcircled{$x$}}
 y)  (e_1  \text{\textcircled{$x$}}  e_2) = (e_1  \text{\textcircled{$x$}}  e_2 )
 (x  \text{\textcircled{$x$}}  y)$. But this is trivial. 

$\beta$) If $A_1$ has $x_1, \ldots, x_m$ as a base over $k$ and $A_2$
 has $y_2,\ldots , y_n$ as a base over $k$ then $(x_i
 \text{\textcircled{$x$}} y_j)$ is a base of $A_1 \text{\textcircled{$x$}} A_2$
 over $k$. Furthermore if the multiplication tables for the bases are  
\begin{align*}
x_i  x_j & = \sum_t a_{ij}^{(t)}  x_t\\
y_i  y_j & = \sum_{t'} b_{ij}^{(t')}  y_{t'}
\end{align*}
then\pageoriginale 
$$
(x_p  \text{\textcircled{$x$}}  y_q)  (x_r  \text{\textcircled{$x$}}  y_s) =
\sum_{\lambda,\mu} a_{pr}^{(\lambda)}   b^{(\mu)}_{qs} (x_\lambda 
\text{\textcircled{$x$}}  y_\mu ) 
$$

$\gamma $) If $A_1$ and $A_2$ have unit elements $e_1$ and $e_2$
respectively then the mappings  
\begin{align*}
x & \rightarrow x  \text{\textcircled{$x$}}  e_2\\
y & \rightarrow  e_1  \text{\textcircled{$x$}}  y
\end{align*}
are isomorphisms of $A_1$ and $A_2$ into $A$. Thus $A$ contains
subalgebras isomorphic to $A_1$ and $A_2$. 

An important special case is the one where one of the algebras is a
field. Let $A$ be an algebra over $k$ and $K$ an extension field of
$k$. Let $A$ have unit element $e_1$ and $K$ the unit element
$e_2$. Form the Kronecker product $A \text{\textcircled{$x$}} K$ over
$k$. Then $A \text{\textcircled{$x$}}  K$ contains subalgebras $A_1$ and
$K_1$ isomorphic to $A$ and $K$ respectively. For any $x
\text{\textcircled{$x$}}  t$ in $A  \text{\textcircled{$x$}}  K$ we have  
$$
x  \text{\textcircled{$x$}}  t = x  \text{\textcircled{$x$}}  e_2. e_1 
\text{\textcircled{$x$}}  t = e_1  \text{\textcircled{$x$}}  t.  x  \text{\textcircled{$x$}}
 e_2 
$$
 so that $A_1$ and $K_1$ commute. Also every element of $A
\text{\textcircled{$x$}}  K$ is of the form $\sum a_{\alpha\beta} (x_\alpha
\text{\textcircled{$x$}}  t_\beta)$ where $\{ x_\alpha\}$ is a base of $A$ over
$k$ and $\{t_\beta \}$ a base of $K$ over $k$. But this expression can
be written 
$$
\sum_\alpha \Big(\sum_\beta a_{\alpha \beta} (e_1  \text{\textcircled{$x$}}
t_\beta)\Big) (x_\alpha  \text{\textcircled{$x$}}  e_2). 
$$

This shows that $A  \text{\textcircled{$x$}}  K$ is an algebra over $K_1$ with
the base $\{x_\alpha  \text{\textcircled{$x$}}  e_2 \}$. If we identify $A_1$
with $A$ and $K_1$ with $K$ then $A  \text{\textcircled{$x$}}  K$ can be
considered as an algebra over $K$, a basis of $A$ over $k$ serving as
a base of $A  \text{\textcircled{$x$}}  K$ over $K$. $A  \text{\textcircled{$x$}}  K$ is
then called the \textit{algebra got from} $A$ \textit{by extending
  the ground field} $k$ \textit{to} $K$.\pageoriginale We shall denote
it by $A_K$.  

It is clear that $A_K$ is commutative if and only if $A$ is a
commutative algebra. 

\medskip
\text{\underline{Note}}. Even if $A$ is a field over $k, A_K$ need
\textit{not} be a field over $K$.  

Let $\Gamma$ be the rational number field and $\Gamma o =
\Gamma(\sqrt{d})$ the quadratic field over $\Gamma$. Let
$\Gamma$ be the real number field and consider the Kronecker
product $A = \Gamma_o  \text{\textcircled{$x$}}  \bar{\Gamma}$ over
$\Gamma$. Let $e_1$, $e_2$ be a basis of $\Gamma_o$ over
$\Gamma$ with the multiplication table, 
$$
e^2_1 = e_1, \quad e_1  e_2 = e_2  e_1 = e_2, \quad e^2_2 = d e_1.
$$

The elements of $\Gamma_o$ are of the form $ae_1 + be_2$, $a$, $b$ in
$\Gamma$. The elements of $A$ are of the form $ae_1 + be_2$ with
$a$, $b  \in \bar{\Gamma}$. 

Let first $d > o$. Then $\Gamma_o  \text{\textcircled{$x$}}
\bar{\Gamma}$ is not an integrity domain. For,  
$$
( \sqrt{d}  e_1 + e_2 )  (\sqrt{d}  e_1 - e_2) = o.
$$

It can however be seen that $A$ is then the direct sum of the two
fields $\lambda \bar{\Gamma}$ and $\mu \bar{\Gamma}$ where 
$$
\lambda = \frac{1}{2} (e_1 + \frac{e_2}{\sqrt{d}}),   \quad \mu = 
\frac{1}{2} (e_1 - \frac{e_2}{\sqrt{d}}). 
$$

Let $d < o$. Then $A$ is a field. For if $a  e_1 + b  e_2 \neq o$,
then $a^2 - d  b^2 \neq o$. Put $f = \dfrac{a}{a^2 - db^2}$, $g =
\dfrac{-b}{a^2 - db^2}$. Then 
$$
(a  e_1 + b   e_2)  (f  e_1 + g  e_2) = e_1.
$$


\section{Composite fields}\label{c5:s2} % section 2

Let $K_1$ and $K_2$ be two extension fields of $k$. Suppose $K_1$ and
$K_2$ are both contained in an extension field $\Omega$ of $k$. Then
the composite of $K_1$ and $K_2$ is the field generated over $k$ by
$K_1$ and\pageoriginale $K_2$. In general, given two fields $K_1$ and
$K_2$ which 
are extensions of $k$, there does not exist an extension field
$\Omega$ containing both. Suppose, however, there is a field
$\Omega/k$ which contains k-isomorphic images $K^\sigma_1$, $K^\tauup_2$
of $K_1$ and $K_2$, then a composite of $K_1$ and $K_2$ is defined to
be the fields $k(K^\sigma_1  \cup  K^\tauup_2)$. A \textit{composite
  extension} of $K_1$ and $K_2$ is therefore given by a triplet
$(\Omega, \sigma, \tauup )$ consisting of 1) and extension field
$\Omega$ of $k$ and 2) isomorphisms $\sigma$, $\tauup$ of $K_1$ and
$K_2$ respectively in $\Omega$ which are identity on $k$. The
composite extension is then $k(K^\sigma_1  \cup K^\tauup_2)$. We wish
to study these various composites of $K_1$ and $K_2$. 

If $\Omega'$ is another extension of $k$ and $\sigma', \tauup'$  two
k-isomorphisms of $K_1$ and $K_2$ respectively in $\Omega'$ then
$k(K^{\sigma'}_1  \cup  K^{\tauup'}_2)$ is another composite
extension. We say that these two composite extensions are
\textit{equivalent} if there exists a k-isomorphism $\mu$ of
$k(K^\sigma_1  \cup  K^{\tauup}_2)$ on $k(K^{\sigma'}_1  \cup
K^{\tauup'}_2)$ such that  
$$
\mu  \sigma = \sigma' , \mu  \tauup = \tauup'.
$$

Obviously this is an equivalence relation and we can talk of a class
of composite extensions. 

If $k(K^\sigma_1  \cup  K^\tau_2)$ is a composite extension we
denote by $R(K^\sigma_1  \cup  K^\tau_2)$ the ring generated over
$k$ by $K^\sigma_1$ and $K^{\tau}_2$ in $\Omega$. This ring is, in
general, different from $k(K^\sigma_1  \cup  K^\tau_2)$. 

Let now $\bar{K} = K_1  \text{\textcircled{$x$}}  K_2$ be the Kronecker product
of $K_1$ and $K_2$. $\bar{K}$ contains subfields isomorphic to $K_1$ and
$K_2$. We shall identify these subfields with $K_1$ and $K_2$
respectively. Let now $k(K^\sigma_1  \cup  K^\tau_2)$ be a composite
extension and $R(K^\sigma_1  \cup  K^\tau_2)$ the ring\pageoriginale
of the composite extension. Define the mapping $\varphi$ of $\bar{K}$ into
$R(K^\sigma_1  \cup  K^\tau_2)$ by  
$$
\varphi \Big(\sum a_{xy}  (x  \text{\textcircled{$x$}} y)\Big) = \sum a_{xy}
x^\sigma  y^\tau, 
$$
where $a_{xy}  \in  k$. Then $\varphi$ coincides with $\sigma$ on
$K_1$ and with $\tau$ on $K_2$. Since $\sigma$ and $\tau$ are
isomorphisms, it follows that $\varphi$ is a $k$-homomorphism of
$\bar{K}$ on $R(K^\sigma_1  \cup  K^\tau_2)$. Since $\Omega$ is a
field, it follows that kernel of the homomorphism is a prime ideal
$\mathscr{Y}$ of $\bar{K}$. Thus 
$$
\bar{K}/\mathscr{G} \simeq R(K_1^\sigma  \cup  K^\tau_2).
$$

If $k(K^{\sigma'}_1  \cup K^{\tau'}_2)$ is another composite
extension, then $\mu$ defined earlier, is an isomorphism of
$R(K^\sigma_1  \cup  K^\tau_2)$ on $R(K^{\sigma'}_1  \cup
K^{\tau'}_2)$. Consider the homomorphism $\varphi$ defined
above. Define $\bar{\varphi}$ on $\bar{K}$ by $\bar{\varphi} =
\mu\cdot\varphi$. We have 
$$
\bar{\varphi}(\sum a_{xy} \Big(x  \text{\textcircled{$x$}}  y)\Big) = \mu
\Big(\sum a_{xy}  x^\sigma  y^\tau \Big) = \sum a_{xy}  x^{\sigma'}
y^{\tau'}. 
$$

Then $\bar{\varphi}$ is a homomorphism of $\bar{K}$ on
$R(K^{\sigma'}_1 \cup K^{\tau'}_2)$. But, since $\mu$ is an
isomorphism, it follows that $\bar{\varphi}$ has $\mathscr{G}$ as the
kernel. Thus the prime ideal $\mathscr{G}$ is the same for a class of
composite extensions. 

Conversely, if two composite extensions correspond to the same prime
ideal of $\bar{K}$, it can be seen that they are equivalent. 

We have, now, only to prove the existence of a composite extension
associated with a prime ideal of $\bar{K}$. Let $\mathscr{G}$ be a
prime ideal of $\bar{K}$ and $\mathscr{G} \neq \bar{K}$. Since
$\bar{K}$ has a unit element, a prime ideal $\mathscr{G} \neq \bar{K}$
always exists. Let $A$ be the integrity domain 
$$
A = \bar{K}/\mathscr{G}
$$\pageoriginale

Let $\varphi$ be the natural homomorphism of $\bar{K}$ on $A$. Then
$K^{\varphi}_1$ and $K^{\varphi}_2$ are subfields of $A$. Since
$\bar{K}$ is generated by elements of the type $x  \text{\textcircled{$x$}}  y$,
$K_1^{\varphi}$ and $K^{\varphi}_2$, are different from zero. Clearly
$A = R (K^{\varphi}_1  \cup  K^\varphi_2)$. Hence the quotient field
of $A$ is a composite extension. Hence for every prime ideal
$\mathscr{G} \neq \bar{K}$, there exists a composite extension. We
have hence proved the  

\begin{thm}\label{c5:thm4}% theorem 4
The classes of composite extensions of $K_1$ and $K_2$ stand in (1,1)
  correspondence with the prime ideal $\mathscr{G} \neq \bar{K}$ of
  the Kronecker product $\bar{K}$ of $K_1$ and $K_2$ over $k$.  
\end{thm}

Consider now the special case where $K_2/k$ is \textit{algebraic}. Let
$k(K^{\sigma}_1  \cup  K^{\tau}_2)$ be a composite extension. Then  
$$
k(K^\sigma_1  \cup  K^\tau_2) \supset  R (K^\sigma_1  \cup
K^\tau_2) \supset  K^\sigma_1. 
$$

Since every element of $K_2$ is algebraic over $k$, $k(K^\sigma_1  \cup
K^\tauup_2)$ is algebraic over $K^\sigma_1$. This means that
$R(K^\sigma_1  \cup  K^\tau_2)$ is a field and so coincides with
$k(K^\sigma_1  \cup  K^\tau_2)$. Thus 

\begin{thm}\label{c5:thm5}% theorem 5 
If $K/k$ is algebraic, then every prime ideal $\mathscr{G} \neq
  \bar{K}$ of $K$ is a maximal ideal. 
\end{thm}

Let $K/k$ be an algebraic extension and $L/k$ any extension. The
Kronecker product $\bar{K} = K  \text{\textcircled{$x$}}_k  L$ is the extended
algebra $(K)_L$ of $K$ by extending $k$ to $L$. $\bar{K}$ is  thus an
algebra (commutative) over $L$. If $\mathscr{G}$ is a prime of
$\bar{K}$, then by above, it is a maximal ideal and
$\bar{K}/\mathscr{G}$ gives a composite extension. Since $\bar{K}$ is
an algebra over $L$ we may regard $\bar{K}/\mathscr{G}$ as an
extension field of $L$. 

Let now\pageoriginale $\mathscr{G}_1, \ldots, \mathscr{G}_m$ be $m$
distinct maximal ideals of $\bar{K}$, none of them equal to
$\bar{K}$. Let $L_i = \bar{K}/\mathscr{G}_i$ be a composite
extension. Form the direct sum algebra  
$$
\sum_i  L_i
$$
as a commutative algebra over $L$. We shall now prove 

\begin{thm}\label{c5:thm6}% theorem 6
$$
\fbox{$\sum_i  \bar{K}/\mathscr{G}_i  \simeq  \bar{K}/
  \bigcap\limits_i  \mathscr{G}_i$} 
$$
\end{thm}

\begin{proof}% proof
We shall construct a homomorphism $\varphi$ of $\bar{K}$ on $\sum_i
L_i$ and show that the kernel is $\bigcap\limits_i  \mathscr{G}_i$. 
\end{proof}

In order to do this let us denote by $\sigma_i$ the natural
homomorphism of $\bar{K}$ on $L_i$ (this is identity on $L$), $i = 1,
\ldots, m$. If $x  \in  \bar{K}$, then $\sigma_i  x  \in  L_i$. Define
$\varphi$ on $\bar{K}$ by 
$$
\varphi(x) = \sum_i  \sigma_i  x.
$$
 
That this is a homomorphism on $\bar{K}$ is easily seen; for, if $x$, $y
\in  \bar{K}$ 
\begin{gather*}
\varphi(x+y) = \sum_i  \sigma_i(x+y) = \sum_i  \sigma_i  x + \sum_i
\sigma_i  y = \varphi x + \varphi y.\\ 
\varphi(x  y) = \sum_i  \sigma_i (x  y) = \sum_i  (\sigma_i  x)
(\sigma_i y) = (\sum_i  \sigma_i  x) (\sum_i \sigma_i  y) = \varphi
x  \varphi  y. 
\end{gather*} 
 
The kernel of the homomorphism is set of $x$ such that $\varphi  x =
o$. Thus $\sigma_i  x = o$ so that $x \in  \mathscr{G}_i$ for all
$i$. Hence $x  \in  \cap_i \mathscr{G}_i$. But every  $y  \in  \cap_i
\mathscr{G}_i$ has the property $\varphi  y = o$. Thus the kernel is
precisely $\bigcap\limits_i  \mathscr{G}_i$. 

We have only to prove that the homomorphism is \textit{onto}.

To this\pageoriginale end, notice that for each $i,i = 1,\ldots, m$,
there is a $b_i \in  \bar{K}$ with   
\begin{equation*}
b_i
\begin{cases}
\in  \mathscr{G}_j, \quad j \neq i\\
\notin  \mathscr{G}_i
\end{cases}
\end{equation*}

For, since $\mathscr{G}_i$ and $\mathscr{G}_j$, $j \neq i$ are distinct,
there is $a_j  \in  \mathscr{G}_j$ which is not in
$\mathscr{G}_i$. Put $b_i = \prod\limits_{i  \neq  j}  a_j$. Then
$b_i$ satisfies above conditions since $\mathscr{G}_i$ is maximal. 

Let now $\sum\limits_i  c_i$ be an element in the direct sum, $c_i
\in  L_i$. By definition of $b_i$ 
\begin{equation*}
\sigma_j  b_i
\begin{cases}
= o \quad \text{ if }  j \neq i\\
\neq o \quad \text{ if }  j = i
\end{cases}
\end{equation*}

Since $L_i$ is a field, there exists $x_i \neq o$ in $L_i$ such that 
$$
x_i  \sigma_i  b_i = o_i.
$$
$\sigma_i$ being a homomorphism of $\bar{K}$ on $L_i$, let $y_i  \in
\bar{K}$ with $\sigma_i y_i = x_i$. Put 
$$
c = \sum_i  b_i  y_i
$$

Then
$$
\varphi (c) = \sum_i \sum_j  \sigma_j (b_i  y_i) = \sum_i  c_i
$$
which proves the theorem completely.

Suppose in particular $K/k$ is finite. Then $K_L$ has over $L$ the
degree $(K : k)$. Since, for a maximal ideal, $\mathscr{G} \neq
\bar{K}$, $\bar{K}/\mathscr{G}$ has over $L$ at most the degree $(K :
k)$, we get 
$$
1 \le (\bar{K}/\mathscr{G}_i : L) \le (K : k) \quad i = 1, \ldots, m.
$$

This means that $\bar{K}$ has only finitely many maximal ideals and  
$$
\bar{K}/_{\bigcap\limits_{\mathscr{G}} \mathscr{G}}  \simeq
\sum_{\mathscr{G}}   \bar{K}/\mathscr{G} 
$$
the\pageoriginale summations running through all maximal ideals of
$\bar{K}$.   

Thus $K$ and $L$ have over $k$ only finitely many inequivalent
composite extensions. 


\section{Applications}\label{c5:s3} % section 3

Throughout this section $\Omega$ will denote an algebraically closed
extension of $k$ and $K$ and $L$ will be two intermediary fields
between $\Omega$ and $k$. A composite of $K$ and $L$ in $\Omega$ will
be the field generated over $k$ by $K$ and $L$. It will be denoted by
$KL$. 

Let $\bar{K}$ be the Kronecker product over $k$ of $K$ and $L$. There
is, then, a homomorphism $\varphi$ of $\bar{K}$ on $R(K  \cup  L)$
given by  
$$
\varphi\Big(\sum a_{xy}  (x  \text{\textcircled{$x$}}  y)\Big) = \sum  a_{xy}
x\cdot y. 
$$

Suppose that this homomorphism is an isomorphism of $\bar{K}$ onto
$R(K  \cup  L)$. This means that  
$$
\sum a_{xy}  (x  \text{\textcircled{$x$}}  y) = o \Longleftrightarrow \sum
a_{xy}  x\cdot y = o 
$$
or that every set of elements of $K$ which are linearly independent
over $k$ are also so over $L$. Incidentally, this gives 
$$
K  \cap  L = k.
$$


Conversely, suppose $K$ and $L$ have the property that every set of
elements of $K$ which are linearly independent over $k$ are also so
over $L$. Then the mapping $\varphi$ is an isomorphism of $\bar{K}$ on
$R(K  \cup  L)$. For, if $\sum a_{xy}  x  y = o $ we express $x$ and
$y$ in terms of a base of $L/k$ and a base of $K/k$ giving 
$$
\sum b_{\alpha \beta}  x_\alpha  y_\beta = o
$$

But\pageoriginale this means all $b_{\alpha\beta}$ are zero. Therefore
$\varphi$ is an isomorphism.  

It shows that every set of elements of $L$ which are linearly
independent over $k$ are also so over $K$. 

We call two such fields $L$ and $K$ \textit{linearly disjoint over}
$k$. Note that $L  \cap  K = k$. We deduce immediately 

$1)$ \textit{If $L$ and $K$ are are linearly disjoint over $k$ then
  any intermediary field of $K/k$ and any intermediary field of $L/k$
  are also linearly disjoint}. 


Suppose now that $K/k$ is algebraic. Then every prime ideals of
$\bar{K}$ is maximal. Let, in addition, $K$ and $L$ be linearly
disjoint over $k$. Since $\bar{K}$ is isomorphic to $R(K  \cup  L)$,
it follows that $(o)$ is a maximal ideal. Hence $\bar{K}$ is a
field. Thus 

2) \textit{If $K/k$ is algebraic and $K$ and $L$ are linearly disjoint
  over $k$, there exists but one class of composite extensions of $K$
  and $L$ over $k$}. 

Let $K/k$  be a finite extension. Then, for some maximal ideal
$\mathscr{G}$, $\bar{K}/\mathscr{G}$ is isomorphic to $KL$. Since
$\bar{K}/\mathscr{G}$ may be considered as an extension field of $L$,
we get $(\bar{K}/\mathscr{G} : L) \le (K : k)$, that is  
$$
(KL : L) \le (K : k)
$$

Clearly if $\mathscr{G} = (o)$, equality exists, and then $K$ and $L$
are linearly disjoint over $k$. The converse is true, by above
considerations. Hence, in particular, 

3) \textit{If $(K : k) = m$ and $(L : k) = n$, then
$$
(KL : k) \le mn;
$$
equality\pageoriginale occurs if and only if $K$ and $L$ are linearly
disjoint over $k$}.  

We now consider the important case, $K/k$ \textit{galois}. By the
considerations above, it follows that $KL/k$ is algebraic over
$L$. Since $\Omega$ is a algebraically closed, it contains the
algebraic closure of $KL$. Let $\sigma$ be an isomorphism of $KL$ in
$\Omega$, which is identity on $L$. Its restriction to $K$ is an
isomorphism of $K$ in $\Omega$. But $\Omega$ contains the algebraic
closure of $K$ and hence $\sigma K = K$. Since $KL$ is generated by
$K$ and $L$, it follows that $\sigma  KL  \subset  KL$. Hence $KL/L$
is a galois extension. ($KL/L$ is already separable since elements of
$K$ are separable over $k$). 
\[
\xymatrix@R=0.5cm{
& \Omega \ar@{-}[d] & \\
& KL\ar@{-}[dl] \ar@{-}[dr] & \\
K \ar@{-}[dr] & & L \ar@{-}[dl]\\
& K\cap L \ar@{-}[d]&  \\
& k & 
}
\]

We now prove the 

\begin{thm}\label{c5:thm7}% theorem 7
 If $K/k$ is a galois extension and $L$, any extension of $k$ then
$$
G(KL/L) \simeq G(K/K  \cap  L)
$$
\end{thm}

\begin{proof}% proof
In the first place, we shall prove that $K$ and $L$ are linearly
disjoint over $K  \cap  L$. For this, it is therefore enough to prove
that every finite set $y_1, \ldots, y_m$ of elements of $L$ which are
linearly independent over $K  \cap  L$, are also so over $K$.  
\end{proof}

If possible, let $y_1, \ldots , y_m$ be dependent over $K$ so that
$\sum x_i  y_i = o$, $x_i  \in  K$. Let us assume that $y_1, \ldots ,
y_m$ are dependent over $K$ but no proper subset of them is linearly
dependent over $K$. Therefore, all the $x_i$ are different from zero. 

We\pageoriginale may assume $x_1 = 1$. Let $\sigma$ be an element of
$G(KL /L)$. Then 
$$ 
0 = \sigma (\sum_i x_i y_i) = \sum_i \sigma x_i\cdot \sigma y_i
$$

By subtraction we get, since $\sigma 1 = 1$, 
$$
0 = \sum_{i \neq 1} (x_i - \sigma x_i) y_i
$$

This means that $x_i = \sigma x_i$, $i=2 ,\ldots, m$. But, since
$\sigma$ is arbitrary in $G (KL/L)$, it follows that $x_i \in
L$. But $x_i \in K$. Thus $y_1 ,\ldots, y_m$ are linearly
dependent on $K \cap L$ which is a contradiction. Hence $K$ and $L$
are linearly disjoint over $K \cap L$. 

Therefore, $KL$ is isomorphic to the Kronecker product of $K$ and $L$
over $K \cap L $. 

Suppose $\sigma$ is any $L$-automorphism of $KL$. Its restriction to
$K$ is an automorphism of $K$ and leaves $K \cap L$ fixed. Consider
the mapping $\sigma \to \bar{\sigma}$ of $G(KL/L)$ into $G(K/K \cap
L)$. This is clearly a homomorphism. If $\bar{\sigma}$ is identity
element of $G(K /K \cap L)$, then $\sigma$ is identity on $K$. Since
it is already identity on $L$, it is identity on $KL$. Thus, $G (KL/L)$
is isomorphic to a subgroup of $G (K /K \cap L)$. To see that this
isomorphism is onto $G (K/K \cap L)$, let  $\tau \varepsilon G (K /K
\cap L)$. Any element of $KL$ may be written (since $K$ and $L$ are
linearly disjoint) in the from,   
$$
\sum_\alpha x_\alpha y_\alpha
$$
where $x_\alpha$ are linearly independent elements of $K$ over $K
\cap L$ and $y_\alpha \in L$. This expression is
unique. Extend $\tau$ to $\bar{\tau}$ in $G(KL/L)$ by defining  
$$
\bar{\tau} \left(\sum_\alpha x_\alpha y_\alpha \right) =
\sum_\alpha y_\alpha \tau (x_\alpha). 
$$

This\pageoriginale is well defined; for, if $\sum_\alpha y_\alpha
\tau(x_\alpha) 
= 0$, then, since $\{ x_\alpha\}$ are linearly independent over $K
\cap L$, $\{ \tau (x_\alpha)\}$ are also linearly independent over $K
\cap L$ and since $K$ and $L$ are linearly disjoint over $K \cap L$,
all $y_\alpha$ are zero. Thus $\bar{\tau}$ is an automorphism of
$KL/L$.  

Our theorem is thus proved.

In particular, if $K$ and $L$ are both galois extensions of $k$ and $K
\cap L= k$ then, by above, 
\begin{align*}
G (KL/L) & \simeq G (K/k)\\
G (KL/K) & \simeq G (L/k)
\end{align*}

Also, since $KL/k$ is algebraic, let $\sigma$ be an isomorphism
(trivial on $k$) of $KL$ in $\Omega$. Since its restrictions on $K$
and $L$ are auto morphisms, it follows that $KL/k$ is a galois
extension. We now prove 

\begin{thm}\label{c5:thm8}%theo 8
$G(KL/k)$ is isomorphic to the direct product of $G(K/k)$ and 
  $G(L/k)$. 
\end{thm} 

\begin{proof}
$G(K/k)$ and $G (KL/L)$ are isomorphic and for every element $\sigma$
  in $G(K/k)$, by the previous theorem, we have the extension
  $\bar{\sigma}$, an element of $G(KL/L)$ determined uniquely by
  $\sigma$. Similarly, if $\tau \in G (L /k)$, $\bar{\tau}$
  denotes its unique extension into an element of $G(KL/k)$. 

\[
\xymatrix@R=0.5cm{
& KL\ar@{-}[dl] \ar@{-}[dd]\\
K \ar@{-}[dd] &\\
& L \\
k \ar@{-}[ur]&  
}
\]
\end{proof}

We now consider the mapping 
$$
(\sigma, \tau) \to \bar{\sigma} \bar{\tau}
$$
of the direct product $G(K/k)\cdot G(L/k)$ into $G (KL/k)$. 

Suppose\pageoriginale $\lambda$ is an element of $G(KL/k)$. Its
restriction 
$\lambda$ to $K$ is an element of $G(K/k)$. Consider
$\bar{\lambda_1}$. Now $\bar{\lambda}^{-1}_1 \lambda$ is identity on
$K$. By the isomorphism of $G(KL/k)$ and $G(L/k)$, this defines a
unique element $\mu_1$ of $G(L/k)$. Hence 
$$
\lambda = \bar{\lambda}_1 \bar{\mu}_1.
$$

Thus the mapping above is a mapping of the direct product $G(K/k) \cdot
G(L/k)$ onto $G(KL/k)$. We have only to prove that is a homomorphism
to obtain the theorem. It is clearly seen that $\lambda$ is identity
if and only if $\lambda_1$ and $\mu_1$ are identity.  

Let $\sigma$, $\sigma'$ be two elements of $G(K/k)$ and $\tau$, $\tau'$ in
$G(L/k)$. Let $\lambda$ and $\mu$ be the unique elements in $G(KL/k)$
defined by  
$$
\lambda = \bar{\sigma} \bar{\tau}, \quad \mu = \bar{\sigma'}, \bar{\tau'}. 
$$ 

Consider to element $\lambda \mu$. Its restriction to $K$ is $\sigma
\sigma'$ and its restriction to $L$ is $\tau \tau'$. Thus 
$$
\lambda \mu = \overline{\sigma \sigma'} \cdot \overline{\tau \tau}'
$$
which proves that the mapping is a homomorphism. 

The theorem is now completely proved.



