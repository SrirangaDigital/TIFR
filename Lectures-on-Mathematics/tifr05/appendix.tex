\backmatter

\chapter{Appendix}

\centerline{\fontsize{12}{14}\selectfont{\textbf{Abelian groups}}}

\setcounter{section}{0}
\section{Decomposition theorem}\pageoriginale %sec 1.

All the groups that we deal with here are abelian. Before proving the
main decomposition theorem for finite abelian groups we shall prove
some lemmas.  

\setcounter{lem}{0}
\begin{lem}\label{app:lem1}%lemma 1
If $a$, $b$ are elements in $G$ and have orders $m$ and $n$
   respectively and $(m, n) = 1$, then $ab$ has order $mn$. 
\end{lem}

\begin{proof}
Clearly if $t$ is the order of $ab$, $t | mn$, since   
$$
(ab)^{mn} = (a^m)^n (b^n)^m = e, 
$$
$e$ being unit element of $G$. Also $a^t = b^{-t}$. Thus 
$$
e = a^{tm} = b^{-tm}
$$
so that $n /t$. Similarly $m | t$. Hence $t = mn$.
\end{proof}

\begin{lem}\label{app:lem2}%lemma 2
Let $p$ be a prime number dividing the order $n$ of the group
$G$. Then there is, in $G$, an element of order $p$. 
\end{lem}

\begin{proof}
We use induction on $n$. Let $\underbar{a}$ be an element in $G$ of
order $m$. If $p | m$, then $a^{m/p}$ has order $p$ and we are
through. Suppose $p \nmid m$. Let $H$ be the cyclic group generated by
$\underbar{a} \cdot  G/H$ has then order $n/m$ which is divisible by
$p$. Since $\dfrac{n}{m} | n$, induction hypothesis applies, so that
there is a coset $Hb$ of order $p$. If $b$ has order $t$ then $b^t =
e$ and so $(Hb)^t = H$ which means that $p | t$ and so $b^{t/p}$ has
order $p$.  
\end{proof}

\begin{lem}\label{app:lem3}%lemma 3
  Let $G$ be a finite group and $\lambda$ the maximum of the orders
  of elements of $G$. Then
$$
a^\lambda = e
$$
for\pageoriginale all $a \in G$.
\end{lem}  

\begin{proof}
Let $b$ be an element of order $\lambda$. Let $\underbar{a}$ be any
element in $G$ and let $\mu$ be its order. To prove the lemma, it is
enough to prove that $\mu | \lambda$. If not, there is a prime $p$
which divides $\mu$ to a higher power than it does $\lambda$. Let $p^r$
be the highest power of $p$ dividing $\mu$ and $p^s$ the highest power
dividing $\lambda$. Then $r > s$. We will see that this leads to a 
contradiction.       
\end{proof}

Since $\dfrac{\mu}{p^r}$ and $p^r$ are coprime,
$a^{\frac{\mu}{p^r}}$ has the order $p^r$. Similarly $b^{p^s}$
has order $\dfrac{\lambda}{p^s}$. By lemma \ref{app:lem1}, 
$$
c=a^{\frac{\mu}{p^r}} \cdot b^{p^s}
$$
has order $p^r$. $\dfrac{\lambda}{p^s} > \lambda$, which contradicts
the definition of $\lambda$. Hence $\mu |\lambda$.   

Lemmas \ref{app:lem2} and \ref{app:lem3} show that $\lambda | n$ where
$n$ is the order of the group and that $\lambda$ and $n$ have the same
prime factors. $\lambda$ is called the \textit{exponent} of the finite
group $G$.  

A set $a_1 , \ldots,a_n$ of elements of a finite group $G$ are said to
be \textit{independent} if  
$$
a_1^{x_1} \cdots a_n^{x_n} = e 
$$
implies $a^{x_i}_i = e$, $i=1 ,\ldots, n$.


If $G$ is a finite group and is a direct product of cyclic groups $G_1
,\break \ldots, G_n$ and if $a_i, i= 1 ,\ldots, n$ is a generator of $G_i$,
then $a_1 ,\ldots, a_n$ are independent elements of $G$. They are said
to form a \textit{base} of $G$.

 We\pageoriginale shall now prove  

\setcounter{thm}{0}
\begin{thm}\label{app:thm1}%them 1
Let $G$ be a finite group of order $n$. Then $G$ is the direct
  product of cyclic groups $G_1 ,\ldots,G_l$ of orders  $\lambda_1
  ,\ldots,\lambda_l$  such that $\lambda_i | \lambda_{i-1}, i = 2 ,
  \ldots, l, \lambda_\ell >1 $. 
\end{thm}

\begin{proof}
We prove the theorem by induction on the order $n$ of the group
$G$. Let us therefore assume theorem proved for groups of order $ <
n$. Let $G$ have order $n$. Suppose $\lambda_1$ is the exponent of
$G$. If $\lambda_1 = n$, then $G$ is cyclic and there and there is
nothing to prove. Let therefore $\lambda_1 < n$. There is an element
$a_1$ of order $\lambda_1$. Let $G_1$ be the cyclic group generated by
$a_1$. $G/G_1$ has order $\dfrac{n}{\lambda_1} < n$. Hence
induction hypothesis works on $G/G_1$. 
\end{proof}

$G/G_1$ is thus the direct product of cyclic groups $W_2 ,\ldots, W_l$
of order $\lambda_2 , \ldots, \lambda_l$ respectively and $\lambda_l
|\lambda_{l-1}| \ldots |\lambda_2$. Let $H_i$ be a generator of
$W_i$. Then $H_i = G_l b_i$ for some $b_i$ in the coset $H_i$. Let the
element $b_i $ in $G$ have order $t_i$. Then $t_i | \lambda_1$ by
lemma \ref{app:lem3}. But   
$$
H_i^{t_i} = b_i^{t_i}  G_1^{t_i} = G_1
$$ 
which proves that $\lambda_i | \lambda_1$. Thus $\lambda_l |
\lambda_{l-1}| \ldots| \lambda_1$.  

Let now $b_i^{\lambda_i} = a_1^{x_i}$. Put $x_i = y_i \cdot z_i$ where
$(y_i, \lambda_1) = 1$ and all the prime factors of $z_i$ divide
$\lambda_1$. Choose $u_i$ prime to $\lambda_1$ such that  
$$
u_i y_i \equiv 1 ({\rm mod} \, \lambda_1).
$$

Then\pageoriginale $b_i^{u_i \lambda_i} = a_1^{z_i}$. Since
$\lambda_1$ is the exponent of $G$,   
$$
e = b_i^{u_i \lambda_1} = (a_1^{z_1})^{\lambda_1 / \lambda_i}.
$$

Since $a_1$ has order $\lambda_1$ this means that $\lambda_i | z_i$,
$i = 2, 3,\ldots, l$.   

Put now 
$$
a_i = b_i^{u_i} \quad a_1^{-z_i / \lambda_i}. 
$$

Since $u_i$ is prime to $\lambda_1$ and so to $\lambda_i$, the coset
$F_i = G_l a_i$ is also a generator of $W_i$. Thus $G_1 a_2 ,\ldots,
G_1 a_l$ is a base of $G/G_1$. 

Let $a_i$ have order $f_i$. Then
$$
e = a_i^{f_i} = b_i ^{u_i f_i} a_i ^{-f_i z_i / \lambda_i}. 
$$

This means that 
$$
b_i ^{u_i f_i} = a_i ^{f_i z_i / \lambda_i}.
$$

Therefore by definition of $b_i , \lambda_i | u_i f_i$. But $(u_i
\lambda_i) =1$. Hence $\lambda_i | f_i$. 

On the other hand
$$
a_i^{\lambda_i} = b_i^{u_i \; \lambda_i} a_i ^{-z_i} = e.  
$$

Hence $f_i = \lambda_i$. We have thus elements $a_1 ,\ldots, a_l$ in $G$
which have orders $\lambda_1 ,\ldots, \lambda_l$ satisfying $\lambda_l
| \lambda_{l-1}|\ldots|\lambda_1$. 

We maintain that $a_1 ,\ldots, a_1$ are independent elements of
$G$. For, if 
$$
a_1^{v_1} \ldots a_l^{v_l} = e, 
$$
then $a_2^{v_2} \cdot \cdots  \cdot a_l^{v_l} = a_1^{-v_1}$ which means that
$F_2^{v_2 }\ldots F_l^{v_l} =G_1$. But\pageoriginale since $F_2
,\ldots, F_l$ are  independent, $\lambda_i | v_i$, $i = 2
,\ldots,l$. But this will mean that $a_1^{v_1} = e$ or $\lambda_1 |
v_1$.   

Since $\lambda_1 \ldots \lambda_l = n $, if follows that $a_1
,\ldots,a_l$ form a base of $G$ and the theorem is proved.  

Let $G$ be a group of group of order $n$ and let $G$ be direct product
of cyclic groups $G_1 , G_2 ,\ldots, G_l$ of orders $\lambda_1 ,\ldots,
\lambda_l$. We now prove 

\begin{lem}%lemma 4
Let $\mu$ be a divisor of $n$. The number $N(\mu)$ of elements $a
  \in G$ with 
$$
a^\mu = e
$$
is given by
$$
N(\mu) =\prod^\ell_{i=1} (\mu , \lambda_i).
$$
\end{lem}

\begin{proof}
Let $a_1 ,\ldots, a_l$ be a base of $G$ so that $a_i$ is of order
$\lambda_i$. Any $a \in G$ has the form 
$$
a = a_1^{x_1} \cdots a_l^{x_l}. 
$$
If $a^\mu = e$, then $e =a_1^{x_1 \mu} \cdots a_l^{x_l \mu}$. Since
$a_1 ,\ldots, a_l$ is a base, this means that $x_i \mu \equiv 0 ({\rm
  mod} \,\lambda_i)$, $i = 1 ,\lambda,l$. Hence $x_i$ has precisely $(\mu,
\lambda_i)$ possibilities and our lemma is proved.  

We can now prove the 
\end{proof}

\begin{thm}\label{app:thm2}%them 2
If $G$ is the direct product of cyclic groups $G_1 ,\ldots G_l,$ of
orders $\lambda_1 ,\ldots, \lambda_l$ respectively with $\lambda_l |
\lambda_{l-1}|\cdots|\lambda_1$ and $G$ is also the direct product of
cyclic groups $H_1 ,\ldots,H_m$ of orders $\mu_1 ,\ldots, \mu _m$
respectively with $\mu_m | \mu_{m-1} |\cdots| \mu_1$,\pageoriginale
then $m=1$ and  
$$
\lambda_i = \mu_i, i =1 ,\ldots,l.
$$
\end{thm}

\begin{proof}
Without loss in generality let $l \ge m$. Let $a_1 ,\ldots,a_l$ be a
base of $G$ in the decomposition $G_1 \times  G_2 \times \cdots
\times  G_1$. Since the number of elements $\underbar{a}$ with $a^\mu
=e$ is independent of the decomposition  
$$
N(\mu) = \prod_{1=1}^\ell (\mu , \lambda_i) = \prod_{j=1}^m (\mu,
\mu_j) 
$$

Put now $\mu = \lambda_1$. Then $ N(\mu) = \lambda^1_1$. But since
$(\mu, \mu_j) \le \mu$, it follows that $\lambda_\ell^\ell \le
\lambda^m_\ell$, so that $l \le m$. This proves 
$$
l=m.
$$
\end{proof}

Also it follows that each factor $(\lambda_1, \mu_j) = \lambda_\ell$ 
 or $\lambda_\ell | \mu_\ell$. Inverting the roles of $\lambda$ and
 $\mu$ we get  
$$
\lambda_\ell = \mu_\ell .
$$

Suppose now it is proved that $\lambda_{q+1} = \mu_{q+1} ,\ldots,
\lambda_1 = \mu_1$. Then by putting $\mu= \lambda_q$, we have  
$$
N(\mu) = \lambda^q_q. \lambda_{q+1} \cdots \lambda_1 = \prod^q_{j=1}
(\lambda_q , \mu_j) \lambda_{q+1} \cdots \lambda_\ell . 
$$

By the same reasoning as before, it follows that $\lambda_q = \mu_q$
and we are, therefore, through. 

For this reason, the integers $\lambda_1 ,\ldots , \lambda_l$ are
called the \textit{canonical invariants} of $G$. From theorems
\ref{app:thm1} and \ref{app:thm2} we have the \textit{Corollary Two
  finite groups $G$ and $G'$ are 
  isomorphic if and only if they have the same canonical invariants. } 

\section{Characters and duality}\pageoriginale %sec 2.

Let $G$ be a group, not necessarily abelian and $Z$ a cyclic group. A
homomorphism $\chi$ of $G$ into $Z$ is called a  \textit{character} of
$G$. Thus 
$$
\chi (a) \chi(b) = \chi (ab).
$$

If we denote the unit element of $G$ by $e$ and that of $Z$ by 1, then
$$
\chi (e) = 1.
$$

The character $\chi_o$ defined by $\chi_o (a) =1$ for all $a \in G$ is
called the \textit{Principal character.} 

If $\chi_1$ and $\chi_2$ are two characters, we define their product
$\chi = \chi_1 \chi_2$ by 
$$
\chi(a) = \chi_1 (a) \chi_2 (a)
$$
and the inverse of $\chi_1$  by 
$$
\chi^{-1}_1 (a) = (\chi_1 (a) )^{-1}.
$$

Under this definition, the characters form  a multiplicative abelian
group $G^*$ called the \textit{character group} of $G$. 

Since $\chi$ is a homomorphism, denote by $G_{\chi}$ the kernel of the
homomorphism $\chi$ of $G$ into $Z$. Then $G/G_{\chi}$ is
abelian. Denote by $H$ the subgroup of $G$ given by 
$$
H = \bigcap_\chi G_\chi, \qquad \chi \in G^*.
$$

Then clearly $G/H$ is abelian. 

We call $Z$ an \textit{admissible group} for $G$ if $H$ consists only
of the identity element. This means first that $G$ is abelian and
\pageoriginale furthermore that given any two elements $a$, $b$ in $G$
there exists a character $\chi$ of $G$ such that, if $a \neq b$,  
$$
\chi (a) \neq \chi (b).
$$

If $\chi$ is a character of $G$, then $\chi$ can be considered as a
character of $G/G_o$ where $\chi$ is trivial on $G_o$, by defining
$\chi (G_o a) = \chi (a)$, $G_o a$ being a coset of $G$ modulo $G_o$. In
particular, the elements of $G^*$ can be considered as characters of
$G/H$. Moreover $Z$ is now an admissible group for $G/H$. 

We now prove the 

\begin{thm}\label{app:thm3}%them 3
If $G$ is a finite abelian group and $Z$ is an admissible group for
$G$, then $G^*$ is finite and $G$ is isomorphic to $G^*$. 
\end{thm} 

\begin{proof}
In the first place $Z$ is a finite group. For, if $a \in G$, $a \neq e$,
there is a character $\chi$ such that  
$$
\chi (a) \neq 1.
$$

If $G$ is of order $n$, then $\chi (a^n) = (\chi (a))^n = 1$ so that
$\chi (a)$ in an element of $Z$ of finite order. Since $Z$ is cyclic
it follows that $Z$ is finite. 
\end{proof}

From this it follows that $G^*$ is finite.

Since $(\chi (a))^n = 1$ for every $\chi$ and every $\underbar{a}$,
there is no loss in generality if we assume that $Z$ is a cyclic group
of order $n$.  

In order to prove the theorem let us first assume that $G$ is a cyclic
group of order $n$. Let $\underbar{a}$ be a generator of $G$ and
$\underbar{b}$ a generator of the cyclic group $Z$ of order
$\underbar{n}$. Define the\pageoriginale character $\chi_1$ of $G$ by  
$$
\chi_1 (a) = b.
$$

Since $\underbar{a}$ generates $G$ any character is determined uniquely by its
effect on $\underbar{a}$. $\chi_1$ is an element of order $n$ in
$G^*$. Let $\chi$ be any character of $G$. Let 
$$
\chi (a) = b^\mu
$$
for some integer $\mu$. Consider the character $\tilde{\chi} =\chi
\cdot \chi_1^{- \mu}$. 
$$
\tilde{\chi}(a) = (\chi_1(a))^{- \mu} \chi (a) = b^{-\mu} \; b^{\mu} = 1  
$$
which shows that $\tilde{\chi} = \chi_o$ is the principal
character. Hence $G^*$ is a cyclic group of order $n$ and the mapping 
$$
a \to \chi_1 
$$
establishes an isomorphism of $G$ on $G^*$.

Let now $G$ be finite non-cyclic abelian of order $n$. $G$ is then a
direct product of cyclic groups $G_1 , \ldots , G_l$ of orders
$\lambda_1 , \ldots, \lambda_l$ respectively. Let $a_i$ be a generator
of $G_i$ so that $a_1 ,\ldots , a_l$ is a base of $G$. Since
$\lambda_1 ,\ldots , \lambda_l$ divide $n$, we define l characters
$\chi_1, \chi_2 ,\ldots, \chi_l$ of $G$ by    
\begin{align*}
\chi_i (a_j) & = 1 \qquad j \neq i \\
\chi_i (a_i) & = b_i  \qquad i = 1 ,\ldots , l,
\end{align*}
where $b_i$ is an element in $Z$ of order $\lambda_i$. These
characters are then independent elements of the abelian group
$G^*$. For, if $\chi^{t_1}_1 \cdots \chi_l^{t_l} = \chi_0$, then, for
any $\underbar{a} \in G$,  
$$
\chi_1 ^{t_1}(a) \ldots \chi_\ell^{t_l} (a) =1.
$$

Taking\pageoriginale for $\underbar{a}$ successively $a_1 ,\ldots,a_l$
we see that $\lambda_i|t_i$ and so $\chi_1 ,\ldots, \chi_l$ are
independent.   

Let $\chi$ be any character of $G$. Then $\chi$ is determined by its
effect on $a_1 , \ldots , a_l$. Let $\chi (a_i) = s_i$. Since
$a_i^{\lambda_i} = e$, $(\chi(a_i))^{\lambda_i} = 1$. But  $(\chi(a_i)
)^{\lambda_i} = s_i^{\lambda_i}$. Thus $s_i^{\lambda_i} = 1$. $Z$
being cyclic, there exists only one subgroup of order
$\lambda_i$. Thus 
$$
\chi (a_i) = s_i = b_i^{\mu_i}, 
$$
for some integer $\mu_i ({\rm mod}\; \lambda_i)$. Consider the character
$\tilde{\chi} = \chi  \chi_1^{-\mu_1} \cdots \chi$. It us clear that
$\tilde{\chi} (a_i) = 1$ for all $\underbar{i}$ so that $\tilde{\chi}=
\chi_o$ or  
$$
\chi = \chi^{^\mu_1}_1 \cdots \chi_\ell^{^\mu_\ell} .
$$

Thus $G^*$ is the product of cyclic group generated by  $\chi_1
,\ldots, \chi_1$. By Corollary to theorem \ref{app:thm2}, it follows
that $G$ and $G^*$ are isomorphic.  

\begin{coro*} 
If $G$ is a finite group and $G^*$ its character group, then whatever
may be $Z$, 
$$
\text{Order} G^* \le \text{Order} G.
$$
\end{coro*}

\begin{proof}
For, if $H$ is the subgroup of $G$ defined earlier, then $G/H$ is
abelian and finite, since $G$ is finite. Also $Z$ is admissible for
$G/H$. Furthermore every character of $G$ can be considered as a
character of $G/H$, by definition of $H$. Hence by theorem \ref{app:thm3}  
\end{proof}

Order $G^* \leq $ Order $G/H \le$ Order $G$. 

Let us now go back to the situation where $G$ is finite abelian and
$Z$ is admissible for $G$. Then $G \simeq G^*$. Let us
define\pageoriginale on $G \times G^*$ the function  
$$
(a, \chi) = \chi (a).
$$ 

For a fixed $\chi$, the mapping $a \to (a, \chi)$ is a character of
$G$ and so an element of $G^*$. By definition of product of character,
it follows that, for fixed $\underbar{a}$, the mapping   
$$
\bar{a} : \chi \to (a, \chi)
$$
is a homomorphism of $G^*$ into $Z$ and hence a character of $G^*$,
Let $G^{**}$ denote the character group of $G^*$. By Corollary above, 
$$
\text{Order} G^{**} \leq \text{Order} G^* = \text{Order} G.
$$
 
Consider now the mapping
$$
\sigma : a \to \bar{a}
$$
of $G$ into $G^{**}$. This is clearly a homomorphism. If $\bar{a}$ is
identity, then $(a, \chi)=1$ for all $\chi$. But since $Z$ is
admissible for $G$, it follows that $a = e$. Hence $\sigma$ is an
isomorphism of $G$ into $G^{**}$. Therefore we have   

\begin{thm}\label{app:thm4}%them 4
The mapping $a \to \bar{a}$ is a natural isomorphism of $G^*$ on $G^{**}$
\end{thm}

Note that the isomorphism of $G$ on $G^*$ is \textit{not natural}.

Under the conditions of theorem \ref{app:thm3}, we call $G^*$ the dual of
$G$. Then $G^{**}$ is the dual of $G^*$ and theorem \ref{app:thm4}
shows that the dual of $G^*$ is naturally isomorphic to $G$. Theorem
\ref{app:thm4} is called the \textit{duality theorem} for finite
abelian groups.   

\section{Pairing of two groups}%sec 3

Let $G$ and $G'$ be two groups, $\sigma, \sigma' ,\ldots$, elements of
$G'$ and $\tau, \tau' ,\ldots,$ elements of $G'$. Let $Z$ be a cyclic
group. Suppose\pageoriginale there is a function $(\sigma, \tau)$ on
$G \times  G'$ into $Z$ such that for every $\sigma$, the mapping  
$$
\lambda_\sigma : \tau \to (\sigma, \tau)
$$
is a homomorphism of $G'$ into $Z$ and for every $\tau$, the mapping  
$$
\mu_\tau : \sigma \to (\sigma, \tau) 
$$
is a homomorphism of $G$ into $Z$, Thus $\lambda_\sigma$ and
$\mu_\tau$ are characters of $G'$ and $G$ respectively. We then say
that $G$   and $G'$ are \textit{paired} to $Z$ and that $(\sigma, 
\tau)$ is a pairing. 

For every $\sigma$, let $G'_\sigma$ denote the kernel in $G'$ of the
homomorphism $\lambda_\sigma$. Put 
$$
H'=  \bigcap_\sigma G'_\sigma
$$
for all $\sigma \in G$. Then $G' /H'$ is abelian. Also, $H'$ is the
set of $\tau$ in $G'$ such that  
$$
(\sigma, \tau)=1
$$
for all $\sigma \in G$. Define  in a similar way the subgroup  $H$ of
$G$.  

We are going to prove

\begin{thm}%them 5
If $G/H$ is finite, then so is $G'/H'$ and both are then isomorphic to
each other. 
  \end{thm}  

  \begin{proof}
From fixed $\sigma$, consider the function 
$$
\chi_\sigma(\tau) = (\sigma, \tau).
$$
  \end{proof}  
  
  This is clearly a character of $G'$. By definition of $H'$, it
  follows  that for each $\sigma$, $\chi_\sigma$ is a character of
  $G'/H'$.  
  
  Consider now the mapping
  $$
 \sigma \to  \chi_\sigma 
  $$ 
  of $G$ into $(G'/ H')^*$. This is again a homomorphism of $G$
  into $(G'/ H')^*$. The  kernel of the homomorphism is the set of
  $\sigma$ for which $\chi_{\sigma}$ is the principal character of $G'/
  H'$. By definition of $H$, it follows that  $H$ is the
  kernel. Hence  
  \begin{equation*}
G/H \simeq \text{ a subgroup of } (G'/ H')^*. \tag{1}\label{app:eq1} 
   \end{equation*}\pageoriginale   
  
 In  a similar way
    \begin{equation*}
(G'/H') \simeq \text{ a subgroup of } (G/ H)^*. \tag{2}\label{app:eq2} 
   \end{equation*}  
   
   Let now $G/H$ be finite. Then by Corollary to theorem \ref{app:thm3},  
$$
\text{ord} (G/H)^* \le \text{ord} G/H.
$$ 

By (\ref{app:eq2}), this means that
$$
\text{order} (G'/H') \leq \text{ order } G/H.  
$$
   
Reversing the roles of $G$ and $G'$  we see that $G/H$ and $G'/H'$
have the same order. (\ref{app:eq1}) and (\ref{app:eq2}) together with theorem
\ref{app:thm3} prove the theorem.  
   
If, in particular, we take $G^*$ for $G'$, then $H'=(\chi_o)$ and so
we have  

\begin{coro*}
If $G$ is any group, $G^*$ its character group and if $G/H$ is
  finite then 
$$
G/H \simeq G^*.
$$
\end{coro*}   


\begin{thebibliography}{99}
\bibitem{key1}{A. A Albert}\pageoriginale Modern Higher Algebra,
  Chicago, (1937)  

\bibitem{key2}{E.Artin} Galois Theory, Notre Dame, (1944)

\bibitem{key3}{E.Artin} Algebraic numbers and Algebraic functions,
  Princeton (1951)

\bibitem{key4}{E. Artin and 0.Schreier}  Eine Kennzeichnung der reell
  abgeschlossene Korper Hamb, Abhand, Bd 5 (1927) p. 225-231 

\bibitem{key5}{N. Bourbaki} Algebra, Chapters 3-6, paris, 1949

\bibitem{key6}{C. Chevalley} Introduction to the theory of Algebraic
  functions, New York, 1951 

\bibitem{key7}{C. Chevalley} Sur la theorie du corps de classes dans les
  corps finis et les corps locaux Jour. Faculty of Science, Tokyo,
  Vol, 2, 1933,P. 365-476  

\bibitem{key8}{M. Deuring} Algebren, Ergebn der Math, Vol.4, 1935

\bibitem{key9}{S. Eilenberg and S. Maclane} Cohomology theory in abstract
  groups, Annals of Mathematics, Vol.48 (1947), P. 51-78  

\bibitem{key10}{H. Hasse} Zahlentheorie, Berlin, 1949

\bibitem{key11}{W. Krull} Galoissche theorie der unendlicher
  erweiterungen Math. Annalen Bd 100 (1928), P.678-698  

\bibitem{key12}{N. H McCoy} Rings and ideals,Carus Mathematical
  Monographs, No.8 (1948) 

\bibitem{key13}{A. Ostrowski}\pageoriginale Untersuchugen zur
  arithmetischen  Theorie   der Korper Math. Zeit, Bd 39 (1935),
  P.269-404  

\bibitem{key14}{G. Pickert} Einfuhrung in die Hohere Algebra,Gottingen
  1951  

\bibitem{key15}{B.Steinite} Algebraische Theorie der Korper Crelles
  Jour. Bd 137 (1910), P. 167-309 

\bibitem{key16}{B.L Van-der-Waerden} ModerneAlgebra, Leipzing, 1931

\bibitem{key17}{A. Weil} Foundations of Algebraic gemetry New York, 1946

\bibitem{key18}{E.Witt} Uber die commutativitat endlicher Schiefkorper
  Hamb. Abhand, Vol 8, (1931), P.413 

\bibitem{key19}{E. Witt} Der existenzsatz fur Abelsche Funktionekorper
  Crelles jour. Bd 173 (1935), P 43-51 
\end{thebibliography}


