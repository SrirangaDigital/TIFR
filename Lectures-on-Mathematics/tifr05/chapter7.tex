\chapter{Formally real fields}\label{chap7} % chapter VII

\section{Ordered rings}\label{c7:s1}\pageoriginale % section 1

A commutative ring $R$ is said to be \textit{ordered} if there is an
ordering relation  $>$ (greater than) such that  
\begin{enumerate}
\renewcommand{\labelenumi}{(1)}
\item for every $a \in R$, $a >o$, $a=o$ or $-a >o$.

\item $a$, $b \in R$, $a> o$, $b >o \Rightarrow a+b >o$, $ab >o$.
\end{enumerate}

We may then define $a > b$ by $a-b>o$. If $a>b$, then, for any $c$ in
$R$, $a+c> b+c$ and if $c >o$, $ac> bc$. 

If $- a> o$, we shall say a is \textit{negative} and if $a > o$, a
said to be \textit{positive}. We denote ``$\underbar{a}$ is negative''
 by $a < o$. 
 
Let us denote, by $P$, the set of elements $a \in R$ with $a \ge
o$. Then, from the definition or ordered ring, we have 
\begin{align*}
A_1) & P+ P \subset P\\
A_2) & P \cdot   P \subset P\\
A_3) & P \cap (-P) = (o)\\
A_4) & P \cup (-P) = R,
\end{align*}
where $P+P$ denotes the set of elements of $R$ of the form $a+b$, $a$,
$b \in P$; $-P$ denotes the set of elements $-a$, $a \in P \cdot  P$
shall be called the set of \textit{non-negative elements of} $R$. The
only element which is both positive and negative  is zero. Clearly, if
$R$ is a any subset $P$ of $R$ satisfying the four conditions above
again determine an order  on $R$.  

Two elements $a$, $b \in R$, $a \neq o \neq b$ are said to have the
\textit{same} or \textit{opposite signs} according as $ab >o$ or $ab <
o$. 

Let\pageoriginale now $R$ be an ordered ring  with unit element 1. Let
$a \in R$, 
$a \neq o$. Then  $a^2= a  \cdot a=  (-a) \cdot (-a)$, so that $a^2>
o$ for $a \neq o$ in $R$. More generally, every finite sum of squares
of elements of $R$ is positive. These elements will be contained in
the set of non-negative elements in every order of $R$.  

Since $R$ has a unit element 1 and $1^2=1$, we have $1 >o$. Also,
$n \cdot  1=1+1 \cdots + 1$, $n$ times so $n >  o$. This  proves 
\begin{enumerate}
\renewcommand{\labelenumi}{\theenumi)}
\item \textit{An ordered ring with unit element has characteristic}
  $o$. 

Let $a \neq o$, $b \neq o$ be elements of $R$. Then  $a$ or $-a$ is
positive. Similarly $b$ or $-b$ is positive. Hence $ab$ or $-ab$ is
positive which proves that $ab \neq o$. Therefore  

\item \textit{An order ring is an integrity domain}.
 
We  define an ordered field to be an ordered ring  whose non-zero
elements form a multiplicative commutative group. We have  

\item \textit{If  $k$ is an ordered field, its positive elements form
  a multiplicative group}. 

For, if $x \in  k$, $x >  o$; then  $xx^{-1}>o$. If $-x^{-1} >o$, then
$-xx^{-1}> o$ which contradicts $x^{-1} x >o$. Thus $x^{-1} >o$ and
$(3)$ is proved. 
Suppose  $R$ and $R'$ are two rings, $R \subset R'$. If $R'$ is
ordered, clearly $R$ is ordered by means of the induced order. If
however, $R$ is ordered, it may not be possible, in all cases, to
extend this order to $R'$. However, in case, it is always possible,
namely 

\item \textit{If $K$ is the quotient field of an ordered ring $R$ then
  the order in $R$ can be uniquely extended to $K$}. 
\end{enumerate}

\begin{proof} %pro
Let\pageoriginale $R$ have an order. It is an integrity domain. Any
elements $x \in 
K$ is of the form $x = \dfrac{a}{b}$, $a$, $b \in R$, $b \neq o$. Let $x
\neq o$. Then $a \neq o$. define $x >  o$ by $ab >o$. Then this
defines an order in $K$. In the first place, the definition does not
depend on the way $x$ is expressed in the form $\dfrac{a}{b}$. Suppose
$x= \dfrac{a'}{b'}$. Then $ab' =a'b$. Since $b' \neq o$, multiplying
both sides of this equality by $bb'$, we have 
$$
ab \cdot b'^2= a'b' \cdot b^2.
$$
\end{proof}

Since $ab >o$, $b'^{2}>o$, $b^{2} >o$, it follows that $a'b' >o$, that
is $\dfrac{a'}{b'} >o$. 

In order to prove that $A_1 , \ldots, A_4$ are satisfies, let $x=
\dfrac{a}{b}> o$ and $y=  \dfrac{a'}{b'} >o$. Then  $ab >o$ and $a'b'
>o$.  
$$
x+y = \frac{ab' +a'b}{bb'}
$$  

Now $(ab' +a'b)bb'= ab \cdot b'^2 + a' b'  \cdot b^2$. Since $ab >o$, $a'b'
>o$, $b^2 >o$, $b'^2 >o$, it follows that $(ab' +a'b) bb' > o$ or $x+y
>o$. In similar manner, $xy >o$. 

Suppose  $x \ge o$ and $y \ge o$ and $x+y =o$;  then $x=o$,
$y=o$. For, if $x= \dfrac{a}{b}$, $y= \dfrac{c}{d}$, then $ab \ge o$,
$cd \ge o$ and $x+y = \dfrac{ad+bc}{bd}=o$, so that $ad+bc=o$. Thus
$ab d^2 +cd b^2 =o$, which means that since all elements are in $R$,
$ab=o$, $cd=o$, i.e. $a=o=c$. 

If $x \in R$, than $x= \dfrac{ab}{a}$. If $x=o$, then $ba^2=o$ or
$b=o$, so that the order coincides on $R$  with the given order in
$R$. 

That the extension is unique can be, trivially, seen. 

Since\pageoriginale every ordered field has characteristic zero, it
contains a subfield isomorphic to $\Gamma$, the rational number field
$\Gamma$ has thus induced order. We shall now prove 
\begin{enumerate}
\renewcommand{\labelenumi}{(\theenumi)}
\setcounter{enumi}{4}
\item $\Gamma$  \textit{can be ordered in one way only}. 
\end{enumerate}

For, if $Z$ denotes the set of integers, then $\Gamma$ is the
quotient field  of $Z$. On $Z$, there is only one order since $1 > o$
and hence $n=1+1 +\cdots + 1 >o$. Thus, all natural integers have to
be positive. 

\section{Extensions of orders}\label{c7:s2}%%%% 2 

Hereafter, we consider ordered fields $k$. Our main  task will be the
study of extensions of orders in $k$ to extension fields $K$ of
$k$. For this purpose, we introduce  the notion of a \textit{positive
  form} on $k$. 

Let $k$ be an ordered field. A polynomial $\sum\limits_{i=1}^m a_i 
x_i^2$, $a_i \in  k$, is said to be an  \textit{$m$-ary form} over
$k$. It is said to be \textit{positive} if $a_i > o, i=1, \ldots
m$. An $m$-ary form is said to represent $\underbar{a} \in  k$, if there
exist $\alpha_1 , \ldots , \alpha_m$ in $k$ such that   
$$
\sum_i a_i  \alpha^2_i =a.
$$

Clearly a positive form represents zero, only if $\alpha_1 , \ldots ,
\alpha_m =o$. Let $k$ be an ordered field and $K/k$, an extension
field. We shall prove 

\setcounter{thm}{0}
\begin{thm}\label{c7:thm1} % the 1
 $K$ has an order extending the order in $k$, if and only if, every
    positive form over  $k$ is still positive in $K$. 
\end{thm}

\begin{proof} % pro
We\pageoriginale have only to prove the sufficient. To this end,
consider the family $M$ of subsets $\{ M_\alpha \}$ of $K$ having the
following properties. Denote, by  $S$, the set of elements in $K$ of
the form  
$$
\sum_{i} a_i \alpha^2_i,
$$
$a_i \in k$ and $a_i >o$ and $\alpha_i \in K$. ($\alpha_i$ can be
all zero also). 
\end{proof}

Then 
$$
\begin{aligned}[7]
 1) \quad & M_\alpha \supset S &  & \quad  2) & &  M_\alpha
 +M_\alpha \subset   M_\alpha&  \\  
 3) \quad & M_\alpha M_\alpha \subset M_\alpha & & \quad  4)
 &&M_\alpha \cap (-  M_\alpha )= (o).& 
\end{aligned}
$$

This family is not empty, since $S$ satisfies this condition. In the
usual way,  we make $M$ a partially ordered  set and obtain a maximal
set $P$. We have now to show that  
$$
P \cup(-P)=K
$$
and then $P$ will determine an order. Let $x \neq o$ be an elements of
$K$ which is not in $P$. Define 
$$
Q= P- xP
$$
as the set of elements of the form $a-x  b$, $a$, $b \in  P$. Obviously
$Q$ satisfies (\ref{chap7:eq1}). To see that $Q$ satisfies (2),
observe that  if 
$a-x b$, $c-x d$ are in $Q$ then  
$$
(a-xb)+(c-xd)=(a+c) - x(b+d);
$$
but $a$, $b$, $c$, $d$ being in $P$ which is an element of $M$, $a+c$, $b+d$
are in $P$. In a similar way $Q$ satisfies (3). That $Q$ satisfies
(4) can be seen as follows: Let $a-xb$ and $c-xd$ be in $Q$ with
$a$, $b$, $c$, $d$, in  $P$. Let $(a+c)-x(b+d) =o$. Then $b+d=o$. For,
if $b+d \neq o$, then  
$$
x= \frac{a+c}{b+d} = (a+c) (b+d) \frac{1}{(b+d)^2}
$$\pageoriginale
and so is an element of $P$, which is a contradiction. Hence $b+d=0$
and, therefore, $a+c=0$. Since $a$, $b$, $c$, $d$ are all in $P$, it
follows that $a=b=c=d=0$. Thus $Q \in M$. But $Q \supset P$ so that,
by maximality of $P$, $Q=P$. This means that $-x\in P$ or $x \in
-P$. The theorem is therefore proved.

If $\Gamma$ is the field of rational numbers and $n$ is a positive
integer, then $n=1+1+\cdots +1$. If $r=\dfrac{a}{b}$ is a positive
rational number, then
$$
\frac{a}{b} = \frac{ab}{b^2} = \frac{1+1\cdots + 1}{b^2},
$$
so that every positive rational number is a sum of squares. This shows
that every positive form over $\Gamma$ can be put in the form $x^2_1 +
\cdots + x^2_n$.

If $k$ is an ordered field then $\sum\limits_i \alpha^2_i = 0$,
$\alpha_i \in k$ implies that $\alpha_i =0$, $i=1,\ldots$. On the
other hand, if $k$ has the property that $\sum \alpha^2_i = 0$,
$\alpha_i \in k$ implies $\alpha_i=0$, then $n=1+1+\cdots + 1 \neq 0$
so that k has characteristic zero. Since every positive form over
$\Gamma$ is essentially of the type $x^2_1+\cdots + x^2_n$, we have
the 

\begin{thm}\label{c7:thm2}%% 2
$k$ is ordered if and only if $\sum\limits_i \alpha^2_i =0$, $\alpha_i
  = k$ implies $\alpha_i =0$, $i=1, 2, \ldots$.
\end{thm}

It is obvious that, if, in a field $k$, $\sum\limits_i \alpha^2_i=0$,
with $\alpha_1, \alpha_2, \ldots$ not all zero, then
$$
-1 = \sum_i \beta_i 2, 
$$
$\beta_i \in k$.\pageoriginale A field in which $-1$ is not a sum of
squares is called a \textit{formally real field}. From theorem
\ref{c7:thm2}, it follows that formally real fields are identical
with ordered fields.  

We shall, now, prove the following application of theorem \ref{c7:thm1}.

\begin{thm}\label{c7:thm3}%%% 3
Let $k$ be a formally real field with a given order and $f(x)$, an
irreducible polynomial in $k[x]$. Let $a$, $b$ be two elements in $k$
such that $f(a) f(b) <0$. Suppose $\alpha$ is a root of $f(x)$ in
$\Omega$, an algebraic closure of $k$. Then $K=k(\alpha)$ is ordered
with an order which is an extension of the given order in $k$.
\end{thm}

\begin{proof}
Let $f(x)$ be of degree $n$. Then every element in $k(\alpha)$ is a
polynomial in $\alpha$ of degree $n-1$ with coefficients in $k$. If
$k(\alpha)$ is not ordered with an order extending that in $k$, then
there is a positive form in $k$ which represents $-1$. That is, 
$$
-1 = \sum_i a_i \{\varphi_i (\alpha)\}^2,
$$
$a_i>0$ in $k$. This means that, in $k[x]$,
$$
1+\sum_i a_i \{\varphi_i(x)\}^2 = f(x) \psi(x).
$$

Since $f(x)$ has degree $n$ and left side has degree $\leq 2n-2$,
$\psi (x)$ has, at most, the degree $n-2$.
\end{proof}

We now use induction on $n$. If $n=1$, these is nothing to
prove. Assume theorem proved for $n-1$ instead of $n$. Let $g(x)$ be
in irreducible factor of $\psi(x)$. Then $g(x)$ has degree $\leq
n-2$. Now
\begin{align*}
0<1 + \sum_i a_i \{\varphi_i(a)\}^2 & = f(a) \psi (a)\\
0<1 + \sum_i a_i \{\varphi_i (b)\}^2 & = f(b) \psi (b).
\end{align*}

Since\pageoriginale $f(a) f(b) <0$, it follows that $\psi(a)
\psi(b)<0$. Therefore, 
at least one irreducible factor, say $g(x)$, of $\psi(x)$ has the
property $g(a)g(b)<0$. If $\beta$ is a root of $g(x)$ in $\Omega$,
then $g(\beta) = f(\beta) =0$. But by induction hypothesis, $k(\beta)$
has an order extending the given order in $k$. Hence
$$
0=1+\sum_i a_i \{\varphi_i (\beta)\}^2 >0,
$$
which is a contradiction. Thus our theorem is completely proved.

Suppose $k$ is an ordered field, let us denote, by $|a|$, the
\textit{absolute value} of $a\in k$ by 
$$
|a|=
\begin{cases}
0 & \text{ if } a=0\\
a & \text{ if } a >0\\
-a & \text{ if } a<0.
\end{cases}
$$

It is then easy to prove that 
\begin{gather*}
|ab| = |a| \; |b|,\\
|a+b| \leq |a| + |b|.
\end{gather*}

Let $f(x) = x^n +a_1 x^{n-1} + \cdots + a_n$ be a polynomial in
$k[x]$. Put $M=\max(1, |a_1| + \cdots + |a_n|)$. If $t\neq 0 \in k$
and $|t|>M$, then 
$$
t^{-n} f(t) = 1+ a_1 t^{-1} + \cdots + a_n t^{-n} >0
$$
which shows that $t^n$ and $f(t)$ have the same sign.

Suppose now $f(x)$ is irreducible and of odd degree. Then, if $M$ is
defined as above,
$$
f(M) f(-M) < 0. 
$$
Therefore, by theorem \ref{c7:thm3}, if $\beta$ is a root of $f(x)$ in an
algebraic closure of $k$, then $k(\beta)$ has an order extending that
in $k$.

If\pageoriginale $a \in k$ and $a>0$, then the polynomial $x^2-a$
changes sign in $k$. For, $(a+1)^2>a$ and $-a<0$. Thus 
$$
((a+1)^2 -a) (0-a) <0.
$$
Therefore, $k(\sqrt{a})$ has an order extending the given order in
$k$.


\section{Real closed fields}\label{c7:s3}%%% section 3

We had seen above that, under certain circumstances, an order in a
field $k$ can be extended to a finite algebraic extension of $k$. We
shall consider, now, a class of fields called \textit{real closed}
fields defined as follows: $k$ is said to be real closed, if 
\begin{enumerate}[1)]
\item $k$ is ordered

\item $k$ has no proper algebraic extension $K$ with an order
  extending that in $k$.
\end{enumerate}

Before we establish the existence of such fields, we shall obtain some
of their properties. We first prove

\begin{thm}\label{c7:thm4}%%% 4
For a formally real field $k$, the following  properties are
equivalent:
\begin{enumerate}[1)]
\item $k(i)$ is algebraically closed, $i$ being a root of $x^2+1$.

\item $k$ is real closed.

\item Every polynomial of odd degree over $k$ has a root in $k$ and
  every positive element of $k$ is a square in $k$.
\end{enumerate}
\end{thm}

\begin{proof}
$1\Rightarrow 2$. $k(i)$ being of degree 2 over $k$, there are no
  intermediary fields, so that, $k$ being ordered, and $k(i)$ being
  algebraically closed, $k$ has no ordered algebraic extension.

$2 \Rightarrow 3$.\pageoriginale Suppose $f(x)$ is a polynomial of odd
  degree. Then 
  it changes sign in $k$. Hence an irreducible factor of $f(x)$, also
  of odd degree, changes sign in $k$. If $\alpha$ is a root of this
  irreducible factor, then $k(\alpha)$ is ordered with an order
  extending that in $k$. Hence $\alpha \in k$. If $a>0$ in $k$, then
  $x^2-a$ changes sign in $k$. Thus $\sqrt{a} \in k$.

$3 \Rightarrow 1$. The poof of this part consists of three
  steps. Firstly, every element of $K=k(i)$ is a square. For, let
  $a+bi$ be an element of $K$, $a$, $b\in k$. Put
$$
a+ib = (c+id)^2
$$
where $c$ and $d$ have to be determined in $k$. Since 1, $i$ form a
base of $K/k$ we get
$$
c^2-d^2 = a, \quad 2cd =b.
$$
Therefore, $(c^2+d^2)^2 = a^2+b^2$. But $a^2+b^2>0$ in $k$ and, since
every positive element is a square, there is a $\lambda >0$ in $k$
such that 
$$
c^2+d^2=\lambda.
$$

Solving for $c^2$ and $d^2$, we have
$$
c^2 = \frac{\lambda +a}{2},  \quad  d^2=\frac{\lambda-a}{2}.
$$
But, since $\lambda^2=a^2+b^2$, it follows that $\lambda \geq |a|$,
$\lambda \geq |b|$. Therefore $\dfrac{\lambda+a}{2} \geq 0$,
$\dfrac{\lambda-a}{2} \geq 0$. Therefore
$$
c = \pm \sqrt{\frac{\lambda+a}{2}}, \quad d = \pm
\sqrt{\frac{\lambda-a}{2}} 
$$ 
exist in $k$. The arbitrariness in the signs of $c$ and $d$ can be
fixed from the fact that 2 $cd =b$.
\end{proof}

This\pageoriginale proves that every quadratic polynomial over $k$ has
a root in $K$. For, if $ax^2 + bx + c \in k [x]$, then, in an
algebraic closure of $K$, $\dfrac{-b\pm \sqrt{b^2-4ac}}{2a}$ are its
roots $(a\neq 0)$. But $\sqrt{b^2-4ac} \in K$. 

The second step consists in showing that every polynomial in $k[x]$
has a root in $K$. Let $f(x)$ be in $k[x]$ and let $N$ be its degree
$N=2^n \cdot q$, $q$ odd. We shall use induction on $n$. If $n=0$,
then $N=q$, and so whatever odd number $q$ be, $f(x)$ has a root
already in $k$. Let us, therefore, assume proved that every polynomial
of degree $2^{n-1} q'$, where $q'$ is odd, with coefficients in $k$
has a root in $K$. Let $f(x)$ be of degree $N = 2^n \cdot q$, $q$
odd. Let $\alpha_1, \ldots, \alpha_t$ be the distinct roots of $f(x)$
in an algebraic closure of $K$. Let $\mu \in k$ be an element to be
suitably chosen later. Put 
$$
\lambda_{ij}(\mu) = \lambda_{ij} = \alpha_i  + \alpha_j + \mu \alpha_i
\alpha_j i, j = 1, \ldots, t, i \neq j.
$$

Consider now the polynomial
$$
\varphi_{\mu} (x) = \prod_{i \neq j} (x-\lambda_{ij}).
$$

This has degree $\dfrac{N(N-1)}{2} = 2^{n-1} \cdot q', q'$ odd. Also,
by every permutation of the symbols $1, \ldots, t$, the polynomial
goes over into itself. Thus $\varphi_{\mu}(x) \in k[x]$. Since its
degree satisfies the induction hypothesis, for every $\mu \in k$,
there is an $i$ and $aj$ such that $\lambda_{ij}(\mu) \in K$. Since
$k$ is an infinite field, there exist $\mu$, $\mu'$, $\mu \neq \mu'$
and both in $k$ such that $\lambda_{ij}(\mu)$ and $\lambda_{ij}(\mu')$
for two integers $i$ and $j$ are in $K$. This means that
$\alpha_i\alpha_j$ and, hence, $\alpha_i + \alpha_j$ are in $K$. The
polynomial $x^2-x(\alpha_i + \alpha_j) +\alpha_i \alpha_j$ is
a\pageoriginale polynomial in $K[x]$. By what we proved above, both
its roots are in $K$. Thus our contention is proved. 

The third step consists in proving that every polynomial in $K[x]$,
has a root in $K$. For, let $f(x)$ be a polynomial in $K[x]$. Let
$\sigma$ be the generating automorphism of $K/k$. It is of order
2. Denote by $f^{\sigma}(x)$ the polynomial obtained from $f$ by
applying $\sigma$ on the coefficients of $f$. Then $\varphi = f(x)
f^{\sigma}(x)$ is a polynomial in $k[x]$. The second step shows that
$\varphi$ has a root $\alpha$ in $K$. Furthermore if $\alpha$ is a
root of $\varphi$, $\alpha^{\sigma}$ is also a root of $\varphi$, so
that either $\alpha$ is a root of $f(x)$ or $\alpha^{\sigma}$ is a
root of $f(x)$.

We have thus proved theorem \ref{c7:thm4} completely.

We deduce from this an important corollary due to \textit{Artin} and\break
\textit{Schreier}. 



\setcounter{corollary}{0}
\begin{corollary}\label{c7:cor1}%%%  1
If $\Omega$ is an algebraically closed field and $K$ a subfield such
that $1<(\Omega:K)< \infty$, then $K$ is real closed.
\end{corollary}

\begin{proof}
We had already proved that $K(i) = \Omega$ and that $K$ has
characteristic zero. By virtue of theorem \ref{c7:thm4}, it is
enough to prove that $K$ is formally real. Every element of $\Omega$
is of the form $a+ib$, $a$, $b\in K$. Also 
$$
a + ib = (c+id)^2,
$$
for $c$, $d$ in $K$, since $\Omega$ is algebraically closed. Thus
$$
a^2+ b^2  = (c^2+d^2)^2.
$$

Hence every sum of two squares and, hence, of any number of squares is
a square. Therefore
$$
-1 = \sum_{i} a^2_i
$$
is\pageoriginale impossible in $K$. By theorem \ref{c7:thm2},
therefore, $K$ is formally real. 
\end{proof}

This proves that the real closed fields are those and only those which
are such that their algebraic closures are finite over them.

We have again

\begin{corollary}\label{c7:cor2}%% 2
A real closed field has only one order.
\end{corollary}

For, the set of positive elements coincides with the set of squares of
the elements of the field.

This shows that, if on ordered field has two distinct orders, it has
algebraic extensions which are ordered. It must be remembered that if
a field has only one order, it is not necessarily real closed. The
rational number field, for example, has only one order.

Suppose $k$ is a real closed field. Then every irreducible polynomial
in $k[x]$ is of degree one or two. Suppose $f(x)$ is a polynomial in
$k[x]$ and $a$, $b$ in $k$ such that 
$$
f(a) f(b) <0.
$$
Then one of the irreducible factors $\varphi(x)$ of $f(x)$ must have
the property that $\varphi(a) \varphi(b) < 0$. If $\alpha$ is a root
of $\varphi(x)$, then $k(\alpha)$ is ordered. But, $k$ being real
closed, $\alpha \in k$. $\varphi(x)$ must be a linear polynomial. Thus
$\varphi(x) = x-\alpha$ and 
$$
(a-\alpha) (b-\alpha) < 0
$$
which means that $\alpha$ lies between $a$ and $b$. Hence the 

\begin{thm}\label{chap7:thm5}%% 5
If $k$ is a real closed field, $f(x)$ a polynomial in $k[x]$, $a$, $b$
in $k$ such that $f(a) f(b) <0$, then there is a root $\alpha$ of
$f(x)$ in $k$ between $a$ and $b$.
\end{thm}

Furthermore,\pageoriginale we had seen that there is an $M$ in $k$
depending only on the coefficients of $f(x)$ such that $f(a)$ has the
same sign as $a^n$, $n = \deg f(x)$, if $|a|>M$. This shows 

\textit{All the roots of $f(x)$ that lie in $k$ lie between $\pm M$}.

Let $k$ be a real closed field and $f(x)$ a polynomial in $k[x]$. Let
$f'(x)$ be its derivative. Put $\varphi_0 = f$, $\varphi_1 =
f'$. Since $k[x]$ is a Euclidean ring, define, by the Euclidean
algorithm, the polynomials
\begin{align*}
& \varphi_0 = A_1 \quad \varphi_1 -- \varphi_2\\
& \varphi_1 = A_2 \quad \varphi_2 -- \varphi_3\\
& \quad \ldots \quad \ldots \\ 
& \varphi_{r-1} = A_r \quad \varphi_r.
\end{align*}

It is then well-known that $\varphi_r (x)$ is the greatest common
divisor of $\varphi_0$ and $\varphi_1$. The sequence $\varphi_0,
\varphi_1, \ldots, \varphi_r$ of polynomials in $k[x]$, is known as
the \textit{Sturmian polynomial sequence}.

Let $a\in k$ be such that $\varphi_0 (a) \neq 0$. Then $\phi_r (a)
\neq 0$. Consider the set of elements $\varphi_0 (a)$, $\varphi_1 (a),
\ldots, \varphi_r(a)$ in $k$. The non-zero ones among them have a
sign. Denote, by $\omega(a)$, the number of changes of sign in the
sequence of elements,
$$
\varphi_0 (a), \varphi_1(a), \ldots, \varphi_r(a),
$$
in this order, taking only the non-zero elements. A very important
theorem due to \textit{Sturm} is

\begin{thm}%% 6
Let $b$ and $c$ be two elements of $k$, $b<c$ and $\varphi_0 (b) \neq
0$, $\varphi_0 (c) \neq 0$. Let $\omega(b)$ and $\omega (c)$ denote
the number of changes\pageoriginale of sign in the Sturmian sequence
for the values $b$ and $c$. Then $f(x)$ has precisely $\omega(b) -
\omega(c)$ distinct roots in $k$ between $b$ and $c$. 
\end{thm}

\begin{proof}
Since $\varphi_r(b) \neq 0$, $\varphi_r(c) \neq 0$, we may divide all
the Sturmian polynomials by $\varphi_r(x)$ and obtain the sequence
$\bar{\varphi}_0 (x)$, $\bar{\varphi}_1(x), \ldots \bar{\varphi}_{r-1}
(x)$, $\bar{\varphi}_r (x) (=1)$. Now $\bar{\varphi}_0(x)$ has no
multiple roots. For, if $\alpha$ is a root of $f(x)$ of multiplicity
$t$, then 
\begin{align*}
\varphi_0 (x) & = (x-\alpha)^t \psi_1(x), \quad \psi_1(\alpha) \neq
0\\
\varphi_1 (x) & = t(x-\alpha)^{t-1}\psi_1(x) + (x-\alpha)^t \psi'_1(x)
\end{align*}
so that $(x-\alpha)^{t-1}$ is the highest power of of $x-\alpha$, that
divides $\varphi_1(x)$. Hence
\begin{align*}
\bar{\varphi}_0 (x) & = (x-\alpha) \psi_2(x)\\
\bar{\varphi}_1(x) & = t \psi_2(x) + (x-\alpha) \psi_3 (x),
\end{align*}
$\psi_2(x)$ and $\psi_3(x)$ being polynomials over $k$. Note that
$\bar{\varphi_1}(x)$  is \textit{not} the derivative of
$\bar{\varphi}_0(x)$. We shall drop the `bars' on the $\varphi's$ and
write them as $\varphi_0, \varphi_1, \ldots,\varphi_{r-1},
\varphi_r=1$ and $\varphi_0$ having no multiple roots. Note that
$\omega(b)$ or $\omega(c)$ is not altered by doing the above.
\end{proof}

The finite number of polynomials $\varphi_0, \varphi_1, \ldots,
\varphi_{r-1}$ have only finitely many roots between $b$ and $c$. By
means of these roots, we shall split the interval $(b,c)$ into
finitely many subintervals, the end points of which are these
roots. We shall study how the function $\omega(a)$ changes as
$\underbar{a}$ runs from $b$ to $c$.
\begin{enumerate}[1)]
\item No two\pageoriginale consecutive functions of the Sturmian series
  $\varphi_0(x)$,\break $\varphi_1(x), \ldots,\varphi_{r-1}(x)$ can vanish at
  one and the same point, inside the interval $(b,c)$. For, suppose
  $b<a<c$ and $\varphi_i(a) = 0 = \varphi_{i+1}(a)$, $0< i + 1
  <r$. Then
$$
\varphi_i (x) = A_i \varphi_{i+1} (x) - \varphi_{i+2}(x)
$$
so that $\varphi_{i+2}(a)=0$, and, so on, finally
$\varphi_r(a)=0$. But $\varphi_r (a)=1$.

\item Inside any one of the intervals, each function keeps a constant
  sign; for, if any function changed sign then, by theorem
  \ref{chap7:thm5}, there   would be a zero inside this interval. 

Let $\underbar{d}$ denote an end point of an interval and $L$ and $R$
the intervals to the left of $d$ and to the right of $d$, having $d$
as a common end point.

\item Suppose $\underbar{d}$ is a zero of $\varphi_1$ for $0<
  1<r$. Then
$$
\varphi_{l-1} = A_l \varphi_l - \varphi_{l+1},
$$
so that $\varphi_{l-1}(d) = - \varphi_{l+1}(d)$ and none of them is
zero by (\ref{chap7:eq1}). Because of (2), $\varphi_{l-1}$ has in
$L$ the same sign 
as at $d$. Similarly in $R$. The same is true of
$\varphi_{l+1}$. Thus, whatever sign $\varphi_l$ might have in $L$ and
$R$, the function $\omega(a)$  remains constant when $\underbar{a}$
goes from $L$ to $R$ crossing a zero of $\varphi_l$, $0<1<r$.

\item Let now $d$ be a zero of $\varphi_0$. Then $d$ is not a zero of
  $\varphi_1$. We have
\begin{align*}
\varphi_0 (x) & = (x-d) \psi(x)\\
\varphi_1(x) & = m \psi (x) + (x-d) \psi_1(x),
\end{align*}
where $m$ is an integer $>0$, $\psi(x)$ and $\psi_1(x)$ are
polynomials over $k$, and $\psi(d) \neq 0$. At $\underbar{d}$,
$\varphi_1(d)$ has the same sign as $\psi(d)$.
\end{enumerate}

In\pageoriginale $L$, $\varphi_0$ has the sign of $\varphi_0(a) = (a-d) \psi
(a)$. But $a-d<0$, so that $\varphi_0$ has the sign of $-\psi(a)$. In
$L$, $\varphi_1(a)$ has the same sign as $\psi(d)$. Also $\psi(x)$ has
no zero in $L$. Hence $\varphi_1(a)$ has the same sign as
$\psi(a)$. Therefore in $L$, $\varphi_0$ and $\varphi_1$ have opposite
signs. In $R$, exactly the opposite happens, namely $\varphi_0(a) =
(a-d)$. $\psi(d)$, $a-d>0$. Hence $\varphi_0$ and $\varphi_1$ have the
same sign in $R$. Thus $\omega(a)$ is lessened by 1, whenever
$\underbar{a}$ crosses a zero of $\varphi_0(x)$ and remains constant
in all other cases.

Our theorem is, thus, completely proved.

We make the following remark.

\begin{remark*}
Suppose $k$ is a formally real field and 
$$
f(x) = x^n + a_1 x^{n-1} + \cdots + a_n,
$$
a polynomial in $k[x]$. Let, as before, $M = \max (1, |a_1| + \cdots +
|a_n|)$. Suppose there exists a real closed algebraic extension $K$ of
$k$ with an order which is an extension of the given order in
$k$. $f(x)$ can, then, be considered as a polynomial in $K[x]$ and, as
seen earlier, $f(x)$ has no roots in $K$ outside the interval $(-M,
M)$. The number of these distinct roots is thus independent of $K$.

We now prove
\end{remark*}

\begin{thm}\label{chap7:thm7}%% 7
Let $k$ be an ordered field, $\Omega$ its algebraic closure. Suppose
there exist two real closed subfields $K$, $K^{\ast}$ of $\Omega/k$
with orders extending the given order in $k$. Then $K$ and $K^{\ast}$
are $k$ isomorphic.
\end{thm}

\begin{proof}
\begin{enumerate}[1)]
\item Let $f(x)$ be a polynomial in $k[x]$ and $\alpha_1, \ldots,
  \alpha_t$ the distinct roots of $f(x)$ in $K$. Let $\alpha^{\ast}_1,
  \ldots, \alpha^{\ast}_t$ be the distinct\pageoriginale roots of $f(x)$ in
  $K^{\ast}$. Put $L = k (\alpha_1, \ldots, \alpha_t)$ and $L^{\ast} =
  k (\alpha^{\ast}_1, \ldots, \alpha^{\ast}_t)$. Since $L/k$ is
  finite, $L=k(\xi)$ for some $\xi$ in $K$. Suppose $\varphi(x)$ is
  the minimum polynomial of $\xi$ in $k[x]$. Let $\xi^{\ast}$ be a
  root of $\varphi(x)$ in $K^{\ast}$. Then $k(\xi)$ and
  $k(\xi^{\ast})$ are $k$-isomorphic. If $\rho$ is this isomorphism,
  then $\rho \xi = \xi^{\ast}$. But $k(\xi) = L = k( \alpha_1, \ldots,
  \alpha_t)$. Hence $\rho \alpha_1, \ldots, \rho \alpha_t$ will be
  distinct roots of $f(x)$ in $K^{\ast}$. Thus $L^{\ast} = k
  (\xi^{\ast})$. Hence
$$
L \simeq L^{\ast}.
$$

\item Suppose $\varphi(x)$ is any polynomial in $k[x]$, $\beta_1,
  \ldots, \beta_s$ its distinct roots in $K$. Let $\beta^{\ast}_1,
  \ldots, \beta^{\ast}_s$ be the corresponding roots in
  $K^{\ast}$. Consider all the positive quantities among $\beta_i -
  \beta_j$, $i \neq j$. Their square roots exist in $K$. Let $\psi(x)$
  be a polynomial in $k[x]$, among whose roots are these square roots
  and let $\delta_1,\ldots, \delta_g$ be the roots of $\psi(x)$ in
  $K$, $\delta^{\ast}_1, \ldots, \delta^{\ast}_g$ the corresponding
  roots in $K^{\ast}$. Then, from above,
\begin{align*}
 F & = k (\beta_1, \ldots, \beta_s, \quad \delta_1, \ldots, \delta_g)
\simeq\\
& \simeq k (\beta^{\ast}_1, \ldots, \beta^{\ast}_s, \quad
\delta^{\ast}_1, \ldots, \delta^{\ast}_g) = F^{\ast}.
\end{align*}

Let $\tau$ be this isomorphism. Let notation be such that 
\begin{align*}
\tau(\beta_i) & = \beta^{\ast}_i\\
\tau(\delta_i) & = \delta^{\ast}_i.
\end{align*}

Suppose $\beta_i > \beta_j$. Then $\beta_i - \beta_j >0$ so that
$\beta_i - \beta_j = \delta^2_t$, for some $t$. Also, $\tau(\beta_i -
\beta_j) =\beta^{\ast}_i - \beta^{\ast}_j$. But
$$
\tau(\beta_i - \beta_j) = \tau (\delta^2_t) = \delta^{\ast 2}_t.
$$ 

Hence $\beta^{\ast}_i - \beta^{\ast}_j = \delta^{\ast^2}_t >0$, which
proves that 
$$
\beta^{\ast}_i > \beta^{\ast}_j.
$$\pageoriginale
The isomorphism $\tau$ between $F$ and $F^{\ast}$ preserves order
between the roots of $\varphi(x)$ in $K$.

\item In order, now, to construct the isomorphism between $K$ and
  $K^{\ast}$, we remark that any such isomorphism has to preserve
  order. For, if $\sigma$ is this isomorphism and $\alpha > \beta$ in
  $K$, then $\alpha - \beta = \delta^2$ so that 
$$
\sigma (\alpha - \beta) = \sigma(\delta^2) = (\sigma(\delta))^2 > 0 
$$
so that $\sigma \alpha > \sigma \beta$. We shall, therefore, construct
an order preserving map which we shall show to be an isomorphism.

\item Let $\alpha$ be an element in $K$, $h(x)$ its minimum polynomial
  over $k$. Let $\alpha_1, \ldots, \alpha_t$ be the distinct roots of
  $h(x)$ in $K$ and let notation be so chosen that $\alpha, < \alpha_2
  < \ldots < \alpha_t$. Let $\alpha = \alpha_i$. Let
  $\alpha^{\ast}_1, \ldots, \alpha^{\ast}_t$ be the distinct roots of
  $h(x)$ in $K^{\ast}$ and, again, let the notation be such that
$$
\alpha^{\ast}_1 < \alpha^{\ast}_2 < \ldots < \alpha^{\ast}_t.
$$

Define, now, $\sigma$ on $K$ by 
$$
\sigma \alpha = \alpha^{\ast}_i.
$$

Let $\alpha$ and $\beta$ be two elements of $K$ and let $f(x)$ be a
polynomial in $k[x]$n whose roots in $K$ are $\alpha$, $\beta$,
$\alpha+\beta$, $\alpha \beta, \ldots$. Construct the fields $F$ and
$F^{\ast}$ and the $k$-isomorphism $\tau$ of $F$ on $F^{\ast}$. Since
$\tau$ preserves order of roots of $f(x)$, it preserves order of roots
of the factor of $f(x)$ which has $\alpha$ as its root. Similarly of
the factor having $\beta$ as a root. Hence
\begin{align*}
\tau(\tau) & = \sigma (\alpha), \quad \tau (\beta) = \sigma (\beta),
\tau(\alpha + \beta) & = \sigma (\alpha + \beta), \tau (\alpha \beta)
= \sigma (\alpha \beta).
\end{align*}\pageoriginale

Hence $\sigma(\alpha + \beta) = \sigma \alpha + \sigma \beta$, $\sigma
(\alpha \beta) = \sigma \alpha \cdot \sigma \beta$. Thus $\sigma$ is
an isomorphism of $K$ into $K^{\ast}$. We have similarly an
isomorphism $\sigma^{\ast}$ of $K^{\ast}$ into $K$. Thus $\sigma \cdot
\sigma^{\ast}$ is identity on $K^{\ast}$. Hence $K$ and $K^{\ast}$ are
$k$-isomorphic.
\end{enumerate}
\end{proof}

\begin{coro*}
The only automorphism of $K$ over $k$ is the identity.
\end{coro*}

We shall now prove the theorem regarding the existence of real closed
fields, namely

\begin{thm}\label{chap7:thm8}%%% 8
If $k$ is an ordered field with a given order and $\Omega$, its
algebraic closure, there exists in $\Omega$, upto $k$-isomorphism,
only one real closed field $K$ with an order extending the given order
in $k$.
\end{thm}

\begin{proof}
Let $V$ be the family of formally real subfields of $\Omega/k$ which
have an order extending that in $k$. $V$ is not empty, since $k\in
V$. We partially order $V$ by inclusion. Let $\{k_{\alpha}\}$ be a
totally ordered subfamily of $V$. Let
$K_0=\bigcup\limits_{\alpha} k_{\alpha}$. Then $K_0$ is a field which is
contained in $V$. This can be easily seen. By Zorn's lemma, there
exists a maximal element $K$ in $V$. $K$ has an order extending the
order in $k$. To prove that $K$ is real closed, let $f(x)$ be a
polynomial of odd degree over $K$. It changes sign. Therefore there is
an irreducible factor, also of odd degree, which changes sign in
$K$. This factor must have a root in $K$. Else, by theorem
\ref{c7:thm3} there 
exists an algebraic extension with an order extending that in
$k$. This will contradict maximality of $K$. In a similar way, every
positive element of $K$ is a square. By theorem \ref{c7:thm4}, $K$ is real
closed. If $K$ and $K^{\ast}$ are two real closed subfields with
orders extending\pageoriginale that in $k$, then, by theorem
\ref{chap7:thm7}, they are $k$-isomorphic. 
\end{proof}

 Suppose now that $k$ is a perfect field and $ \Omega $, its algebraic
 closure. Let $G$ be the galois group of $ \Omega /K $. If $G$ has
 elements of finite  order (not equal to identity), let $K$ be the
 fixed field of the cyclic group generated by one of them. Then $(
 \Omega : K )$ is finite and by Artin-Schreier theorem this order has
 to be two. Thus any element of finite order in $ G $ has to have the
 order two. Furthermore, in this case, $k$ is an ordered field. 
 
 On the other hand, if $k$ is an ordered field and $ \Omega $, its
 algebraic closure, there exists, then by theorem \ref{chap7:thm8}, a
 real closed 
 subfield of $ \Omega /k $, say $K$. This means that $G$ has an
 element of order 2. Moreover no two elements of order 2
 commute. For if $ \alpha$, $\beta $ are of order 2 and  commute, then
 $1$, $\alpha$, $\beta$, $\alpha \beta $ is a group order  4 which must
 have a fixed field $L$ such that $ ( \Omega : L ) = 4 $. This is
 impossible, by Artin-Schreier theorem. Hence the  

\begin{thm}%\theorem9
If $k$ is a perfect field, $ \Omega $ its algebraic closure and
   $G$ the galois group of $ \Omega/k $ then $G$ has elements of
   finite order if and only if  $k$ is formally real. Also, then, all
   these elements have order  2 and no two of them commute. 
\end{thm}

\section{Completion under an order}\label{c7:s4}%\section 4.

 Let  $k$ be a formally real field with a given  order. We had defined
 a function $ \mid \, \mid $ on  $k$ with values in  $k$ such that  
 \begin{gather*}
| ab | =  | a | \cdot | b |, \\
| a + b | \leq | a | + | b | .\\
 \end{gather*} 
 
 This implies that 
 $$
 |a | - |b |\leq  | a - b | .
 $$
 
 Also\pageoriginale $a \rightarrow |a|$  is a homomorphism of $k^*$ into
 the set of positive elements of  $k$. The function $ |\, | $
 defines a metric on the field  $k$. We define a \textit{Cauchy
   sequence} in  $k$ to be a sequence   $ ( a_1, \ldots , a_n , .) $
 of elements of  $k$ such that for \textit{every} $ \varepsilon > 0 $
 in  $k$, there exists $n_0$, an integer  such that  
 $$
 \mid a_n -a_m \mid < \varepsilon , n, m > n_0 .
 $$
 
Obviously, if $m>n_0+1$,
$$
|a_m|=|a_m-a_{no}-a_{no}|<\varepsilon +|a_{n_0}|,
$$
so that all elements of the sequence from $n_0$ onwards have a value
less than a certain positive element of  $k$.  

A Cauchy sequence is  said to be a \textit{null sequence} if, for
every $ \varepsilon > 0 $ in  $k$, there is an integer $ n_0 = n_0
(\varepsilon) $ such that  
$$
\mid a_n \mid <  \varepsilon ,  \quad  n > n_0 .
$$

The sum and product of two Cauchy sequence is defined as follows:-
\begin{align*}
( a_1, a_2 , \ldots ) + ( b_1, b_2 , \ldots )  &= ( a_1 + b_1, a_2
  +b_2 , \ldots ) \\ 
( a_1, a_2 , \ldots ) - ( b_1, b_2 , \ldots )  &= ( a_1 + b_1, a_2
  +b_2 , \ldots )\\ 
\end{align*}
\noindent
and it is easy to verify that the Cauchy sequences in $k$  form a ring
$R$  and the null sequences, an ideal $\mathscr{Y}$ of $R$. We
assert that  $ \mathscr{Y} $ is a maximal ideal. For, let $ ( a_1 ,
\ldots ) $ be a Cauchy sequence in  $k$ which is not a null
sequence. Then there exists  a $ \lambda > 0 $  in  $k$ and an
integer $n$  such that  
\begin{equation*}
\mid a_m  \mid >  \lambda ,  m > n .  \tag{1}\label{chap7:eq1}
\end{equation*}

For,\pageoriginale if not, for every $ \varepsilon > 0 $ and integer
$n$, there exist an infinity of $ m > n $ for which $ \mid a_m \mid <
\varepsilon $. Since $ ( a_1, \ldots ) $ is a Cauchy sequence, there
exists  $ n_0 = n_0  (\varepsilon) $ such that    
$$
| a_{m_{1}} - a_{m_{2}}| < \varepsilon , \quad m_1 ,m_2 > n_0 .
$$

Let $ m_0 > n_0 $ such that $ | a_{m_{0}}| < \varepsilon
$. Then, for all $ m > m_0 $, 
$$
| a_m | \leq | a_m - a_{m_0} \mid +    \mid a_{m_{0}}  < 2 
\varepsilon  
$$
which proves that $ ( a_1 , \ldots ) $ is a null sequence,
contradicting our assumption. 


Let 1 be the unit element of $k$. Then the  sequence  $ ( 1, 1,
\ldots ) $ is a Cauchy sequence  and is the unit element in $R$. Let $
c= ( a_1 , \ldots ) $ be in $R$ but not in $ \mathscr{Y} $. Let $m$ be
defined as in  (\ref{chap7:eq1}). Then $ c_1 =  ( 0, 0, \ldots 0, a^{-1}_{m},\break
a^{-1}_{m+1} , \ldots ) $ is also a  Cauchy sequence. For, let $
\varepsilon > 0 $ and $n$, an integer so that  
$$
\mid a_{n_{1}}  - a_{n_{2}} \mid < \varepsilon  ,\qquad  n_1, n_2 > n_0 .
$$ 

Then
$$
\left| \frac{1}{a_{n_{1}}} -\frac{1}{a_{n_{2}}} \right| =
\frac{1}{| a_{n_1} | \, | a_{n_2}|} \left| a_{n_{2}} - a_{n_2} \right|
< \frac{\varepsilon}{\lambda^2},
$$
 if $ n_1$, $n_2 > \max ( m, n_0 ) $. Also $ c c_1 = ( 0, 0, 0, \ldots ,
 1,  1 ,\ldots ) $  which proves that  $ c c_1 \equiv ( 1, 1, 1,
 \ldots $  
 $ ( mod \mathscr{Y} ) $. This proves that $ \mathscr{Y} $ is a
 maximal ideal and, therefore, 
\begin{enumerate}[1)]
\item $R/ \mathscr{Y} $ \textit{is a field} $ \bar{k} $.
\end{enumerate}

$ \bar{k} $ is called the \textit{completion} of $k$ under the given
order. For every  $ a \in k $, consider the Cauchy sequence $
\bar{a} = ( a, a, \ldots ) $. Then $ a \rightarrow \bar{a} $ is a
non-trivial homomorphism of  $k$  into $ \bar{k} $. 

Since\pageoriginale $\bar{k}$ is a field, this is an isomorphism. $
\bar{k} $ thus contains a subfield isomorphic to $k$. We shall
identify it with $k$ itself.  

\begin{enumerate}
\renewcommand{\labelenumi}{(\theenumi)}
\setcounter{enumi}{1}
\item \textit{$ \bar{k} $ is an extension field of $k$}.
\end{enumerate}

We shall now make $ \bar{k} $ an ordered field with an order which is
the extension of the order in $k$. To this end, define a sequence $ c
= ( a_1, \ldots , ) $  of $R$ to be  \textit{positive} if there is an
$ \varepsilon > 0 $  and an $ n_0 = n_0 ( \varepsilon ) $ such that  
$$
a_n > \varepsilon ,  \quad n > n_0 .
$$

If  $ b = ( b_1 , \ldots ) $ is a null sequence, then  
$$
|b_n| < \varepsilon/_2  ,  \quad n > m = m ( \varepsilon ). 
$$
\noindent
If $ p > max ( m, n_0 ) $, then 
$$
a_p + b_p > \varepsilon / 2
$$
which shows that $ a + b $ is also a positive sequence. The definition
of positive sequence, therefore, depends only on the residue class mod
$ \mathscr{Y} $. 

Let $P$  denote  the set of residue classes  containing the  positive
sequences and the null sequence. We shall show that $P$ determines an
order in $ \bar{k} $ 

 First, let $c_1$ and  $c_2$ be two positive sequences. There exist $ 
 \varepsilon_1 > 0$,  $\varepsilon_2 > 0 $  and two integers  $n_1$ and
 $n_2$ such that  
  \begin{align*}
 a_p >  E_1 ,  & \qquad  p > n_1 , \\
 b_p >  E_2 ,  & \qquad  p > n_2 ;\\
 \end{align*} 
  if  $ p > \max ( n_1,n_2) $, then
 $$
 a_p + b_p > \varepsilon_1 + \varepsilon_2 ,
 $$\pageoriginale
so that $ c_1 + c_2 $ is a positive sequence or $ P + P \subset P
$. In a similar manner, $ P P \subset  P $. 

Let now   $ c = ( a_1 , \ldots ) $ be not a null sequence. Then $c$
and $-c$ cannot both be positive. For then, there exist $ \varepsilon,
\varepsilon' $ both positive and integers $n$, $n'$ such that  
\begin{equation*}
 \begin{cases}
a_p > \varepsilon ,&p > n \\
-a_p > \varepsilon' ,&p > n .\\
\end{cases}
\end{equation*}

If $ p > \max (n, n^{1}) $, then 
$$
0 = a_p -a_p > \varepsilon + \varepsilon' > 0,
$$
which is absurd. Thus $ P \cap  ( -P) = (0) $.

Suppose now that  $ c =( a_1 , \ldots ) $ is not a null
sequence. Suppose it is not positive. Then $ -c = ( -a_1, -a_2 ,
\ldots ) $ is a positive  sequence. For, otherwise, for every
$\varepsilon $ and every $n$, 
$$
-a_p  < \varepsilon , \quad  p>n .
$$

But, $c$ being not positive, we have 
$$
a_p  < \varepsilon ,  \quad p > n_1 .
$$

$c$ being a Cauchy sequence, there exists $n_0$ with 
$$
| a_p -a_q | < \varepsilon , \quad  p, q > n_0 .
$$

Hence, if $ p > \max ( n_0, n_1 ) $,
$$
| a_p| \leq | a_p- a_q | +| a_q | < 2 \varepsilon  
$$
which means that $c$ is a null-sequence. This contradiction proves
that  
$$
P \cup ( - P ) = \bar{k}.
$$

We\pageoriginale therefore see that 
\begin{enumerate}
\renewcommand{\labelenumi}{(\theenumi)}
\setcounter{enumi}{2}
\item \textit{$ \bar{k} $ is an ordered field with an order extending
  the order in  $k$}.
\end{enumerate}


We shall denote the element $ ( a_1 ,\ldots ) $ in $ \bar{k} $ by $
\bar{a} $ and write  
$$
\bar{a} = \lim_{n \rightarrow \infty } a_n .
$$ 


Then, clearly, given any $ \varepsilon > 0 $, there exists $ n_0 $
such that the element $ \bar{a} - ( a_{n_{0}})$ in $k$  has all its
elements, from some index on, less than $ \varepsilon $ in absolute
value. This justifies our notation. One clearly has  
\begin{align*}
\lim a_n + \lim b_n &= \lim a_n + b_n .\\
\lim a_n . \lim b_n &= \lim a_n + b_n .\\
\frac{\lim a_n}{\lim b_n} &=  \lim \frac{a_n}{b_n},
 \end{align*} 
 if $ ( b_1 ,b_2, \ldots) $  is not a null sequence.
 
 If $ \Gamma $  is the rational number field, it is ordered and the
 completion $ \bar{\Gamma} $ under this unique order is called the
 \textit{real number field}.  
 
 This method of construction of the real number field goes back to
 \textit{Cantor}. 
 

 \section{Archimedian ordered fields}\label{c7:s5}%section 5.
 
 Ordered fields can be put into two classes. 
 
 A field $k$ is said to be \textit{archimedian ordered} if, for every
 two elements $ a$, $b $ in $ k$, $ a > 0 $, $ b > 0 $ there exists
 an integer $n$ such that  
 $$
 n b > a
 $$\pageoriginale
and, similarly, there is an integer  $m$ with  $ma > b$.

We may state equivalently that, for every $a > 0$, there is an
integer $n$ with  
$$
n > a .
$$

A field $k$ is said to be  \textit{non-archimedian} ordered if  there
exists  $ a > 0 $ such that  
$$
a > n 
$$
for every integer $n$.

$\Gamma$, the rational number field is archimedian
ordered. Consider, now, the ring $ \Gamma $ [$ x $] of
polynomials. For  
$$
f (x) = a_0 x^n + a_1 x^{n-1} + \dots + a_n ,   a_0 \neq  0 
$$
define  $f(x)>0$, if  $a_0 >0$ as a rational number. That this
is an order can be verified easily. Also the order is non-archimedian
because  
$$
x^2 + 1 > n ,
$$
for every integer $n$. This order can be extended to $ \Gamma(x)$.  

This examples shows that an archimedian order in $k$ can be extended
into a non-archimedian order in an extension field which is
transcendental over $k$. That this is not possible in algebraic
extensions is shown by  

\begin{thm}%theorem 10
 If $k$  is  archimedian ordered and  $ K/k $  is algebraic with
  an order extending the order in $k$, the extended order in $K$ is
  again archimedian. 
\end{thm}

\begin{proof}
Let $ \alpha $ be  $ > 0 $ in the extended order in $K$. Let  
$ f (x) = x^n + a_1 x^{n-1} + \cdots a_n  $ be the minimum polynomial
of $ \alpha $ in  $k [x]$.\pageoriginale Consider the quantities $ 1-
a_i $, $ i = 1, \ldots , n $. They are in $k$ and since $k$ and is
archimedian ordered, there exists an integer  $ t > 0 $ such that   
$$
t > 1 - a_i,   i = 1, \ldots , n . 
$$

We now assert that $ \alpha  < t $. For if $ \alpha \ge t $, then 
\begin{align*}
0 = f (\alpha)  = \alpha^n &+  a_1 \alpha^{n-1} + \ldots + a_n \\
\ge  \alpha^n &+  ( \alpha^{n-1} + \ldots + ) ( 1-t ) \\
\ge \alpha^n  &+ ( \alpha^{n-1} + \ldots  + 1 ) ( 1 - \alpha ) = 1
,\\  
\end{align*}
which is absurd. Our theorem is proved.

We had already introduced the real number field. It is clearly
archimedian ordered. In fact, the completion of an archimedian ordered
field is archimedian. We can prove even more, as shown by  
\end{proof}

\begin{thm}%theorem 11
Every complete archimedian ordered  field is  isomorphic to the
  field of real numbers. 
\end{thm}

\begin{proof}
Let $K$ be an archimedian ordered field. It has a subfield isomorphic
to $ \Gamma $, the field of rational numbers. We shall identify it
with $ \Gamma $ it self. Let $\bar{k}$ be the completion of
$k$. Then  
$$
\bar{k} \supset \bar{\Gamma}.
$$
\end{proof}

In order to prove the inequality the  other way round, let $R$ be the
ring of Cauchy sequences in $k$ and $ \mathscr{Y} $, the  maximal
ideal formed by null sequences. We shall show that every residue class
of  $ R / \mathscr{Y} $ can be represented by a Cauchy sequence of
rational numbers.  Therefore, let $ c = ( a_1,a_2 , \ldots ) $  be a
positive Cauchy sequence in $k$. Since $k$ is archimedian ordered,
there exists, for every $n$ greater\pageoriginale than  a certain $m$,
an integer $ \lambda_n $ such that   
$$
\lambda_n < n  a_n <  1 + \lambda_n 
$$
which means that 
$$
\left| a_n -   \frac{\lambda_n}{n} \right|   <  \frac{1}{n}.
$$

Let $ d = ( 0 , 0, 0,  \ldots \dfrac{\lambda_m}{m},
\dfrac{\lambda_{m+1}}{m+1}, \ldots )$ be a sequence of rational
numbers. Let $ \varepsilon > 0 $ be any positive quantity in
$k$. There exists, then, an integer $t$ such that $ \varepsilon t > 1
$, since $k$ is archimedian ordered. Then, for  $ n > t $ and $m$, 
$$
\big | a_n  - \frac{\lambda_n}{n}  \big |< \varepsilon
$$ 
which shows that $ c - d \in \mathscr{Y} $, which proves that $
\bar{k} \subset \bar{\Gamma} $.  

We shall now prove the important 

\begin{thm}%theorem 12
 The real number field $\bar{\Gamma} $ is real closed.
\end{thm}

\begin{proof}
We shall show that every polynomial which changes sign in $
\bar{\Gamma} $ has  a root in $ \bar\Gamma $.  
\end{proof}

Let $ b < c $ be two elements of  $ \bar{\Gamma} $ and $ f (x) $ a
polynomial in $ \bar{\Gamma} [ x ]$  with $ f (b) > 0 $ and  $ f (c) <
0 $. We define two sequences of rational numbers  $ b_0$, $b_1$, $b_2$,
$\ldots $ and $ c_0, c_1, c_2, \ldots $ in the following way. Define
the integers $ \lambda_0, \lambda_1, \lambda_2, \ldots $ inductively
in the following manner. $ \lambda_0 = 0 $, and having defined  $
\lambda_0, \lambda_1 , \ldots , \lambda_{n-1}, \lambda_n $ is  defined
as the largest integer such that  
$$
f ( b + \frac{( c -b ) \lambda_n}{2^n} ) \ge 0 .
$$

We shall put $ b_n = b + \dfrac{( c -b ) \lambda_n}{2^n} $ and  since
we want  $b_n$ to be an increasing sequence, we shall, in addition,
require that  
$$
\frac{\lambda_n}{2}  \leq \lambda_{n-1} + 1 .
$$\pageoriginale

That such a sequence can be found is easily seen. Put now 
$$
c_n = b_n + \frac{c-b}{2^n} .
$$

Then $ c_0, c_1,\ldots $ is  a decreasing sequence 
$$
b \leq b_n \leq b_{n+1} \leq c_{n+1} \leq c_n \leq c. 
$$

Also $ (b_n) $ and $ (c_n) $ are Cauchy sequences; for,
$$
\mid b_{n+1} - b_n  \mid  < \frac{c-b}{2^{n+1}}.
$$

Furthermore, the sequence $ ( c_n -b_n ) $ is a null sequence. For,
$$
c_n = b_n =  \frac{c-b}{2^n}
$$

There is, therefore, an $ \alpha $ between $b$ and $c$ such that  
$$
\alpha = \lim b_n = \lim c_n .
$$
By definition of $ \lambda_n$, $f(c_n) < 0 $ and  $ f (b_n) \ge o $.

Therefore
$$
f (\alpha) = \lim f (b_n) \ge  0,  f (\alpha) = \lim f (c_n) < 0. 
$$

This shows that $ f (\alpha) = 0  $. Hence $ \bar{\Gamma} $ is real
closed.  

We have the 

 \begin{coro*}%%% 
$\bar{\Gamma} ( i )$ is algebraically closed.
 \end{coro*} 

 $ \bar{\Gamma} ( i ) $ is the complex number field. We have thus
 proved the `fundamental theorem of algebra'. 
 
 The  algebraic closure of $ \Gamma $ in $ \bar{\Gamma} $ is called the
 field of \textit{real algebraic numbers  and  is real closed}. 

