\chapter{Lecture}\label{lec2}

\setcounter{section}{1}

\subsection{}\label{lec2:sec1:subsec6}%%% 1.6

Let\pageoriginale $H'_o (A ; \Omega)$ be the dual of $H_o(A; \Omega)$. 

\setcounter{theorem}{2}
\begin{theorem}\label{lec2:sec1:subsec6:thm1.3}%theo 1.3
   \begin{enumerate}[a)]
   \item $H'_o (A; \Omega)$ is space of distributions.
   \item If $T \in H'_o (A ; \Omega)$, then there exists a
     unique $g \in H_o(A; \Omega)$ such that $T =(1+ \sum A^*_i
     A_j)g$. 
   \item The correspondence $T \to g$ is a topological isomorphism of
     $H_o (A; \Omega)$ onto $H'_o (A ; \Omega)$. 
   \end{enumerate}
\end{theorem}
\begin{proof}
Let $u \to L(u)$ be a continuous linear form on $H_o (A; \Omega)$. The
restrictions of $L$ to $\mathscr{D} (\Omega)$ is continuous on
$\mathscr{D}(\Omega)$ with its usual topology. Hence it define a
distribution $T_{L}$ so that $L(\varphi)= \langle T_L,
\varphi \rangle$ for all $\varphi \in \mathscr{D}(\Omega)$. If
$T_L=0$, then $\langle T_L, \varphi \rangle =L(\varphi)=0$ for all
$\varphi \in \mathscr{D}(\Omega)$. Since $\mathscr{D}$ is
dense in $H_O(A ; \Omega), L = 0$. This proves $H'_o (A ; \Omega)
\subset \mathscr{D} (\Omega)$. Now, if $L \in H'_o(A; \Omega)$
by Riesz's theorem, there exists $g_L \in H_o(A; \Omega)$ such
that $L (\bar{u})=(g_L,u)_{(H_o (A; \Omega))}$. Hence for every
$\varphi \in \mathscr{D}(\Omega)$, 
\end{proof}
\begin{align*}
L(\varphi)= \langle T_L, \varphi \rangle (g_L, \bar{\varphi})  &=(g_L,
\bar{\varphi})_o + \sum ^\gamma _{i=1} (A_i g,A_i)_o\\ 
&= \langle (1+ \sum A^*_i A_i)g, \varphi \rangle.
\end{align*}

Hence $T=(1+ \sum A^*_i A_i)g$ and $T \to 0$ in $H'_o(A; \Omega)$ if
and only if $g \to 0$ in $H_o(A; \Omega)$. 
\begin{remark*}
  As we shall see later on this theorem constitutes the solution of
  certain (weak) Dirichlet's problem. 
\end{remark*}

\begin{proposition}\label{lec2:sec1:subsec6:prop1.5}%proposi 1.5 
  Every distribution $T \in H'_o(A; \Omega)$ can be written in
  the form $T =g+ \sum A^* _i f_i$ with $f_i \in L^2 (\Omega)$
  and $g \in H_o(A; \Omega)$ and conversely any distribution of
  the above form is in $H'_o (A; \Omega)$. 
\end{proposition}

Since\pageoriginale by theorem \ref{lec2:sec1:subsec6:thm1.3}, any $T \in H'_o(A; \Omega)$ is of the
form $T =g+\sum A^*_i A_i g$, putting $A_i g=f_i$ we obtain the first
part. Conversely if $s=g+\sum A^*_i f_i$, we have for any $\varphi$ in
$\mathscr{D}_\nu (\Omega)$, 
$$
\langle s, \bar{\varphi} \rangle =\langle g, \bar{\varphi} \rangle
+\sum \langle A^*_if_i \bar{\varphi} \rangle =(g, \varphi)_o +\sum^\nu
_{i=1} (f_i,A_i \varphi)_o. 
$$

Hence $\varphi \to (s, \varphi)$ is a continuous semi-linear
functional on $\mathscr{D}(\Omega)$ with the topology induced by $H_o
(A, \Omega)$, for, if $\varphi \to 0$ in $L^2$ and $A_i \varphi \to 0$
in $L^2$, then $\langle S, \bar{\varphi} \rangle \to 0$. Hence $S
\in H'_o (A; \Omega)$.  

Notice that the above representation $S=g+ \sum A^*_if_i$ is \textit{
  not unique}. 
\begin{coro*}
$A^*_i$ is a continuous mapping of $L^2$ into $H'_o (A; \Omega)$.
\end{coro*}

\subsection{Regularization}\label{lec2:sec1:subsec7}

When $\Omega=R^n$, we write simply $H(A),\mathscr{D}$ instead of $H(A;
\Omega) \mathscr{D}(\Omega)$, etc. 

Let $\rho_k$ be a sequence in $\mathscr{D}$ such that
\begin{enumerate}[1)]
\item $\rho _k \ge 0$,
\item $\int \limits _{R^n} \rho _k (x)dx=1$
\item Support of $\rho_k \subset B_{r_k},r_k \to 0$, $B_{r_k}$ is the
  ball of radius $r_k$. 
\end{enumerate}

Such a sequence exists; for let $\rho \in \mathscr{D}$ be such
that $\rho \ge 0, \int \rho dx=1$ and the support of $\rho$ is
contained in the ball $|x|<1$. We obtain such a function by
considering 
$\begin{cases}
ae^{-\frac{1}{1-|x|^2}} &|x|<1 \\ 
0  &|x|\ge 1
\end{cases}$
with suitable $a$ to make the integral equal to 1. Let $\rho '_k= \rho
(kx). \rho_k$ have their support in the balls $|x|< \dfrac{1}{k}$. Let
$\int \limits _ \sigma \rho'_k dx=\alpha_k$. Then $\rho _k (x)= \alpha
_k .\rho' (k x)$ is a sequence of the required type. 

Such\pageoriginale a sequence is called a regularization sequence.

\begin{theorem}\label{lec2:sec1:subsec7:thm1.4}%the 1.4
  \begin{enumerate}[1)]
  \item If $u \in H(A)$, then $u^* \varphi \in H(A)$,
    for $\varphi \in \mathscr{D}$, where $*$ denotes the
    convolution product. 
  \item $u^* \rho_k \to u$ in $H(A)$, where $\rho_k$ is a regularization
    sequence. 
  \end{enumerate}
\end{theorem}
\begin{proof}
  \begin{enumerate}[1)]
  \item Since $u \in L^2$ and $A_i u \in L^2$ for
    $\varphi \in \mathscr{D},u ^* \varphi \in L^2$ and
    $A_i (u * \varphi)=u* A_i \varphi \in L^2$. Hence $u *
    \varphi H(A)$. 
  \item $u* \rho _k \to u$ and $A_i (u * \rho_k)= A_i (u)* \rho_k \to
    A_i u$ in $L^2$. Hence $u * \rho _K$ tends to $u$ in $H(A)$. 
  \end{enumerate}
\end{proof}

\subsection{Problem of local type}\label{lec2:sec1:subsec8}
    
\textit{In general if $u H (A; \Omega)$ and $\varphi \in
  \mathscr{D}(\Omega)$, it is not true} that $\varphi u$ is in $H(A;
\Omega)$. The problem of determining sufficient conditions in order that
$\varphi u$ should be in $H(A; \Omega)$ is the problem of local type. 

\subsection{Some generalizations}\label{lec2:sec1:subsec9}

Beside considering operators $A_i$ with constant coefficients, we
could consider the case of operators with variable coefficients $A=
\sum a_p (x).D^p,a_p (x) \in \xi (\Omega)$. (It is also
possible, of course, to consider operator with not ``smooth''
coefficients). We could define as above $H(A; \Omega)$ to be the space
of $u \in L^2 (\Omega)$ such that $A_i u \in L^2
(\Omega)$. Similarly as before we can prove that $H(A; \Omega)$ is a
Hilbert space. We can consider also then the problem of determining
$H'_o (A; \Omega)$. However, if $A_i$ are of variable
coefficients $A_i (\rho _k * u)\neq (A_i u)* \rho _k$ so that theorem
\ref{lec2:sec1:subsec7:thm1.4} is no longer true. 

We\pageoriginale could replace $L^2 (\Omega)$ by any normal space of distributions
$E$, i.e., a space $E$ such that $\mathscr{D}(\Omega),\subset E \subset
\mathscr{D'} (\Omega)$ the inclusion being continuous, and
$\mathscr{D}$ being everywhere dense in $E$. $H(A,E, \Omega)$ will be
the space of $u \in E$ for which $A_i u \in E$. We
topologize $H(A, E, \Omega)$ in such a way that the mapping $\underset
{u \to u}{u \to A_i u}$ are continuous from $H(A, E, \Omega)$ to
$E$. If, for instant, $E$ is a Frechet space, then $H(A, E, \Omega)$
also is a Frechet space. 

