
\chapter{Lecture}\label{lec25}%%% 25

\setcounter{section}{13}

Now\pageoriginale we consider the regularity at the boundary of solutions so
determined. This means we want to determine whether if $f \in H^k
(\Omega)$ implies $u \in H^k (\Omega)$. The solution of this
problem in full generality is not known though it would be desirable
to know it, for in that case, for large $k$ weak solutions would be
usual ones. We shall show that this is the case with certain kind of
operators in $\Omega = \{x_n > 0\}$ with constant coefficients. Let 
$$
\Omega = \{x_n > 0\} \text { and } Bu = D^m_y u + D^{m-1}_y
\wedge_1 u + D^{m-2}_y \wedge_2 u + \cdots + \wedge_m u, 
$$
where $\wedge_m$ are partial differential in $x_1, \ldots, x_{n-1}$
operators with constant coefficients of order $\le k$. Let $V = H(B,
\Omega)$ and $a(u, v) = (Bu, Bv)_o + (u, v)_o$ be a sesquilinear form
on $V, a(u, v)$ is V-elliptic. Let $Q = L^2 (\Omega)$. If $f
\in L^2 (\Omega)$, by \S\ \ref{lec5:sec3}, there exists $u \in N$
such that $a(u, v) = (f, v)_o$ for all $v \in V$. Further
$(B^* B + 1) u = f$. To consider the regularity of $\underline{u}$ we
consider first its tangential derivatives and next the normal ones. 

\begin{proposition}\label{lec25:sec13:subsec1:prop13.1}%prop 13.1
  Let $D^p_T f \in L^2$ for all $|p| \le \mu$ for any positive
  integer $\mu$. Then $D^p_\tau u$ and $B D^p_\tau u$ are in $L^2$ for $|p|
  \le \mu$. Let $v^h (x) = \dfrac{1}{h}(v(x+h) - v(x))$ where $h =
  (0,\ldots, h, \ldots, 0), h_n = 0$. Since $B$ is with constant
  coefficients, $v^h \in V$ if $v \in V$. Hence $a(u,
  v^h) = (f, v^h)_o, i.e., (Bu, Bv^h)_o + (u, v^h)_o = (f, v^h)_o$ for
  all $v \in V$. Since $B$ is with constant coefficients 
  \begin{equation}
    (Bh^h, Bv) + (u^{-h}, v)_o = (f^{-h}, v)_o. \tag{1}\label{lec25:sec13:subsec1:eq1}
  \end{equation}
  Putting $v = u^{-h}$,
  $$
  |Bu^{-h}|^{2}_{o} + |u^{-h}|^2_o \le C |f^{-h}|_o|u^{-h}|_o.
  $$
\end{proposition}

If\pageoriginale $D_\tau f$ in $L^2, Bu^{-h}$ and $u^{-h}$  are bounded in $L^2$
which means $B D_\tau u$ and $D_\tau u \in L^2$. Letting $h
\to 0$ in (\ref{lec25:sec13:subsec1:eq1}), 
$$
(B(D_\tau u), Bv)_o + (D_\tau u, v)_o = (D_\tau f, v)_o.
$$

If now $D^{(\mu)}_{t} f \in L^2$ we can repeat the process
proving that if $D^{\mu}_{\tau} f \in L^2$, then
$D^{\mu}_{\tau} u$ and $B D^{\mu}_{\tau} u \in L^2$. Now we
consider normal derivatives. 

\begin{theorem}\label{lec25:sec13:subsec1:thm13.1}%theo 13.1
  Let $\mu= m$. Under the hypothesis of proposition \ref{lec25:sec13:subsec1:prop13.1} $u \in
  H^m (\Omega)$. 
\end{theorem}

We use the
\begin{lemma}\label{lec25:sec13:subsec1:lem13.1}%lemm 13.1
Let $\Omega = \{y > 0\}$. Consider $\mu$ such that

\medskip
(1) \quad 
$\begin{cases}
  D^p_\tau u \in L^2 & {\rm for }~ | p | \le k\\
  D_y D^p_\tau u \in L^2 & {\rm for }~ | p | \le k -1\\
  \qquad \vdots & \qquad \vdots\\
  D^{k-1}_y D^p_\tau u \in L^2 & {\rm for }~  | p | \leq 1 
\end{cases}$ \qquad  and 

(2)\quad $D^m_y u \in H^{-m + k}$.
\end{lemma}

Then $D^k_y \in L^2$.

If we denote by $E(\Omega)$ the space defined by all the conditions
above, the lemma means $E(\Omega) = H^k (\Omega)$. In general, i.e., for
arbitrary $\Omega, H^k (\Omega) \subset E (\Omega)$. This lemma should
hold for $\Omega$ with smooth boundary, though as yet it is not
proved. 

Assuming the lemma for a moment, we complete the proof of the theorem.

We have $Bu = D^m_y u + D^{m-1}_y + D^{m-1}_y \wedge_1 u + \cdots + \wedge_m u$.

From proposition $13.|, D^p_\tau u \in L^2$ and $D^{p}_{\tau}
Bu \in L^2$ for $|p| \le m$. Hence $\wedge_k u \in H^o
(\Omega)$ and $D^{m-k} \wedge_k u \in H^{-m+1}$. Hence by
lemma \ref{lec25:sec13:subsec1:lem13.1} 
$$
D_y u \in L^2.
$$

Next\pageoriginale we consider $D_\tau B u$ which is in $L^2$.
$$
D_\tau B u = D^m_y D_\tau u  + D^{m-1} \wedge_1 D_\tau u + \cdots
$$

This gives $D^m_y D_\tau u \in H^{-m + 1}$. But $D^{p}(D_\tau
u) L^2, |p| \le 1$. By lemma \ref{lec25:sec13:subsec1:lem13.1} again $D_y D_\tau u \in
L^2$. 

Proceeding similarly we obtain $D^k_y u \in L^2$. Hence $u
\in H^k (\Omega)$. Now we prove if $\Omega = \{ x_n > 0 \}$,
then $E(\Omega) = H^k (\Omega)$. 

\begin{lemma}\label{lec25:sec13:subsec1:lem13.2}%\lemm 13.2
  If $\Omega = R^n, H^k = E(R^n)$.
\end{lemma}

Let by Fourier transformation $x_1, \ldots, x_{n-1}$ go into $\xi_1,
\ldots, \xi_{n-1}$ and $x_n$ into $\xi_n$. Actually we need use only
$D^p_\tau u \in L^2 |p| \le k$ and $D^m_y u \in H^{-m + k}(\Omega)$; 

i.e.,
\begin{equation*}
  (1 + | \xi |^k ) \hat{u} \in L^2 \quad \text{ and } \quad \frac{| 
    \eta |^m \hat{u}}{1 + | \xi |^{m - k} + | \eta |^{m - k}} \in L^2 \tag{1}\label{lec25:sec13:subsec1:lem13.2:eq1} 
\end{equation*}

We may also assume $m > k$. We have to conclude that $|\eta|^k \hat{u}
\in L^2$. Now we use the following inequality 
$$
|\eta|^k \leq c_1 (1 + |\xi|^k) + c_2 \frac{|\eta|^m}{1 + |\xi
  |^{m -k} + |\eta|^{m-k}}. 
$$

For then $|\eta|^k \hat{u} \in L^2$ by (\ref{lec25:sec13:subsec1:eq1}). To prove the inequality we have to prove that
$$
|\eta|^k + |\xi|^{m-k} |\xi|^k + |\eta|^m \le c_1 (1 + |\xi|^k) (1
+ |\xi|^{m-k}+ |\eta|^{m-k}) + c_2 |\eta|^m.
$$ 

Since $m > k, |\eta|^{k } \le c_3 |\eta|^m 1+c$. Hence we need prove
$$
|\eta|^m + |\xi|^{m-k} |\eta|^k \le c_1 (1+|\xi|^k)(1+ |\xi|^{m-k} +
|\eta|^{m-k}) + c_2 | \eta|^m. 
$$

But $|\xi|^{m-k} |\eta|^k \le \dfrac{|\eta|^{kp}}{p} +
\dfrac{|\eta|^{( m -k )q}}{q} \left(ab \le \dfrac{a^p}{p} + \dfrac{b^q}{q},
\dfrac{1}{p} + \dfrac{1}{q} = 1\right)$. 

Hence $|\eta|^m + |\xi|^{m - k} |\eta|^k \le |\eta|^m + |\xi|^m$
. This is trivially less than right hand side of the inequality. 

\begin{lemma}\label{lec25:sec13:subsec1:lem13.3}%lemm 13.3
  If $\rho \in D_{^L\infty}(\Omega)$ and $u \in E (\Omega)$,
  then $\rho u \in E (\Omega)$. This follows from the definition
  of $E (\Omega) $ itself. 
\end{lemma}

\begin{lemma}\label{lec25:sec13:subsec1:lem13.4}%lemm 13.4
  $E(R^n) \Omega$,\pageoriginale i.e., restrictions of $E(R^n)$ to $\Omega$ is dense
  in $E(\Omega)$. 
  
  Let $u_t (x) = u (x', y + t)$ for $t > 0$. Let $v_t = u_t (x) |
  \Omega$. $v_\tau \to u$ in $E(\Omega)$. Let 
  \begin{equation*}
    \rho (y) = 
    \begin{cases}
      0 & {\rm for }~ y < -t\\
      1 & {\rm for }~ y > 1 \qquad \rho(y) \in \mathscr{D}_{L
        \infty} (\Omega) \\ 
      0 < y < 1 & {\rm elsewhere. }
    \end{cases}
  \end{equation*}
  $\rho u_\tau \in V$ by lemma \ref{lec25:sec13:subsec1:lem13.3}, and $(\rho u_\tau) \Omega =
  v_\tau$. The extension $\widetilde{\rho u}_c$ of $\delta u_t$ are in
  $E(R^n)$ and their restrictions $v_\tau$ are dense in $E(\Omega)$. 
\end{lemma}

\begin{lemma}\label{lec25:sec13:subsec1:lem13.5}%lemm 13.5
  $E(\Omega) \cap \mathscr{D} (\bar{\Omega})$ is dense in $E(\Omega)$.
\end{lemma}

From lemma \ref{lec25:sec13:subsec1:lem13.4}, we need prove that if $u = v_{\Omega}$ with $u
\in E(R^n)$ then $u$ can be approached by function from
$E(\Omega) \cap \mathscr{D} (\bar{\Omega})$. For $\cup= \lim \cup
\times \rho_n$ with $\rho_n \to \delta$. The restrictions of $\cup *
\rho_n \to u$.  

To complete the proof of the lemma \ref{lec25:sec13:subsec1:lem13.1}, then, we prove
\begin{lemma}\label{lec25:sec13:subsec1:lem13.6}%lemm 13.6
  Let $u \in E (\Omega) \cap \mathscr{D} (\bar{\Omega})$. Then
{\fontsize{9}{11}\selectfont
  \begin{equation*}
    U(x) =
    \begin{cases}
      u(x), & x_n \geq 0\\
      \lambda_1 u(x',\ldots,y) + \lambda_2 u(x',\ldots, -\frac{y}{2}) + \cdots   +  \lambda_n (x', \ldots, -\frac{y}{n}) x_n< 0 
    \end{cases}
  \end{equation*}}\relax
  is in $E(R^n)$ for suitable $\lambda's$.
\end{lemma}

If we prove this since $E(\Omega) \cap \mathscr{D} (\bar{\Omega})$ is
dense in $E(\Omega)$, we have a continuous mapping $\pi : E (\Omega)
\to E (R^n) = H^k (R^n)$. Hence $E(\Omega) \subset H^k (\Omega)$,
which proves that $H^k (\Omega) = E^k (\Omega)$. $\lambda's$ are
determined so that on $y= 0, \dfrac{\partial^R \cup}{\partial y}$
should be equal from above and below. A simple argument shows that
$\cup \in \mathscr{E} (R^n)$. 

\section{Systems}\label{lec25:sec14}%sec 14

\subsection{}\label{lec25:sec14:subsec1} %%%% 14.1

We\pageoriginale shall consider briefly systems. We shall
denote in this article $H^m (\Omega)$ by $H^m$. Let $H^{(m)} = H^{m_1}
\times \cdots \times  H^{m_\nu}$ with the usual product Hilbert
structures. In $H^{(m)}$ the closure of $(\mathscr{D}(\Omega))^{\nu}$
is $H^{m_1}_o \times \cdots \times H^m_o$. An element of $H^{(m)}(\Omega)$ we denote
by $\overset{\to}u = (u_1, \ldots, u_\nu)$ with $u_i \in
H^{mi}$. Let $V$ be such that $H^{(m)}_o \subset V \subset H^{(m)}$. Let  
$$
a(u, v) = \sum \int a_{pq, \lambda u} (x) D^{q} u_{\mu}  D^p
v_{\lambda} dx ~\underset{\mu = 1, \ldots, \nu}{\lambda = 1, \ldots,
  \nu} 
$$
be a sesquilinear form with $a_{pq, \lambda \mu} \in L^\infty(\Omega)$ and 
$$
a_{pq, \lambda u} = 0, \text { if } |p|\ge m_\lambda \text{ or } |q| > m_\mu.
$$

This last condition assures that $a(u, v)$ is continuous on $V \times V$.

If $a(\overset{\to}u, \overset{\to}u ) \ge \alpha || u||^2$ for all $u
\in V$ and for $\alpha > 0$ and if $Q = L^2$ then from the
general theory of \S\ \ref{lec5:sec3}, there exists a space $N$ and an operator $A$
which establishes an isomorphism of $N$ onto $Q'$ so that  
$$
\langle A \overset{\to}u, \overset{\overline{\to}}\varphi \rangle = a
(\overset{\to}u, \overset{\to}\varphi) \text{ for }
\overset{\to}\varphi \in (\mathscr{D}(\Omega))^\nu,  
$$ 
i.e., \quad $\langle A_1 \overset{\to}u \overset{\overline{\to}}\varphi_1
\rangle + \cdots + \langle A_\nu \overset{\to}u
\overset{\overline{\to}}\varphi_1 \rangle = \sum \int a_{pq \lambda
  \mu} D^q u_\mu D^p_\lambda dx$. 

Hence 
$$
A_\lambda (\overset{\to}u) = \sum\limits_{p, q, \mu} (-1)^p D^p (a_{pq
  \lambda \mu } D^q u_\mu) 
$$

Hence the theory solves the differential systems
$$
A_\lambda \overset{\to}u = \overset{\to}f.
$$

The variety of boundary value problems solved is much larger; e.g., if
$\nu = 2, m_1 = m_2$ and $V$ may be defined as consisting of $(u_1,
u_2)$ such that $\gamma u_1 = \gamma u_2$. 

\subsection{}\label{lec25:sec14:subsec2} %%% subsec 14.2 

We\pageoriginale now give, following Nirenberg \cite{k13} an
example which presents a little strange  behaviour. We take $n = 2, \nu
=2, m_1 =1, m_2 = 3$. We write $x_1 =x$ and $x_2 = y$, so that $V =
H^1 \times H^3$. Let $L_1, M_2, L_3, M_3, N_z$ be the differential
operators the order of which is equal to the index. Let 
\begin{align*}
  a(u, v) & = (D_x u_1, D_x v_1) + (D_y u_1, D_y v_1) + (u_1, L^{*}_{1} v_1) +\\
  & + (-D^3_y u_2, D_y v_1) + (L_3 u_2, v_1) + (D_y u_1, D^3_y v_2) + \\
  & + (u_1, M_3^* v_2) + (D^3_x u_2, D^3_{x} v_2) + (D^{3}_y u_2, D^{3}_{y} v_2) +\\
  & + 3 (D^2_x D_y u_2, D^2_x D_y v_2) + 3 (D_x D^2_y u_2, D_x D^2_y v_2)\\
  & + (M_2 u_2, N_3 v_2).
\end{align*}

\begin{lemma}\label{lec25:sec14:subsec2:lem14.1}%lemm 14.1
  If $\Omega$ is three strongly regular,
  $$
  a ( \overset{\to}u , \overset{\to}v) + \lambda (\overset{\to}u ,
  \overset{\to}v)  
  $$
  elliptic for $\lambda$ large enough. The system $A$ associated with $a(u,v)$  is 
  \begin{align*}
    A_1 (\overset{\to}u) & = - (D^2_x + D^2_y) u_1 + L_1 u + D^4_y u_3 + L_3 u_2\\
    A_2 (\overset{\to}u) & = - \underline{D^4_y u_1} + M_3 u_1 - (D^6_x +
    D^6_y + 3D^2_x D^4_y) u_2 + N_3 M_2 u_1.   
  \end{align*}
\end{lemma} 

From the underlined term in the operator it would look like as if we
have to assume $v_1 \in H^2$ and $u_2 \in H^2$. While
existence and uniqueness in ensured in $H^1 \times H^3$ itself, i.e.,
we require four conditions on boundary while from the differential
equation it looks as if we require five conditions. Further
$a(\overset{\to}u, \overset{\to}v)$ is not elliptic on $H^2 \times
H^3$. This happens because in computation of the real part of
$a(\overset{\to}u, \overset{\to}v)$ the terms involving $D^{2}_{y}
u_1, e.g., (-D^3_y u_2, D_y v_1) + (D_y u_1, D^3_y u_2) = e$, give
zero real part as they are of the form $z - \bar{z}$. To see that the
form is $H^1 \times H^3$ is straight forward by using the definition
of strong regularity and the above remarks. 

%%%%%%%%%%%%%%%%%%%%%%%%%%%%%%%%%%%%%

\begin{thebibliography}{99}
\bibitem{k1}{S. Agmon}: Comm. Pure Applied Maths., X (1957), p.179-239.
\bibitem{k2}{N. Aronszajn} 
  \begin{enumerate}
    \item Theory of Coerciveness, Laurence Tech. Report,
      (1954).\label{k2:e1}   

      \item Theory of Reproducing Kernels, Trans. Amer. Math. Soc.,
        (1950).\label{k2:e2}   
  \end{enumerate}

\bibitem {k3}{N. Aronszajn and K.T.Smith} : Regularity at the
  boundary (forthcoming). 
\bibitem{k4}{Bicadze} : On the uniqueness of the solutions of the
  Dirichlet problem for elliptic partial differential equations, Us
  tekhi Math. Nauk, 3 (1948), p. 211-212. 
\bibitem {k5}{F.E.Browder} : On the regularity properties of
  solutions of elliptic differential equations, Comm. Pure
  Appl. Math., IX(1956), p. 351-361. 
\bibitem {k6}{Campanato} : Forthcoming paper on the regularity at
  the boundary for Picone problems. 
\bibitem {k7}{J. Deny and J.L. Lions} : Surless espaces de
  Beppo-Levi, Amales Inst. Fourier, 1955. 
\bibitem {k8}{L.Garding} : Dirichlet's problem for linear
  elliptic partial differential equations, Math. Scand. Vol. 1
  (1953), p. 55-72. 
\bibitem {k9}{L.Hormander and J.L.Lions} : Completion par la xxxxx
  Dirichlet, Math, Scand. (1956). 
\bibitem {k10}{J.L.Lions:}
  \begin{enumerate}
    \item  Problems aux limits en theoie des
      distributions, Acta Math. 94 (1955), p. 13-153.\label{k10:e1}

    \item  Sur certain problemes aux limits, S.M.F. (1955).\label{k10:e2}

    \item O verts m-reguliers, Revista de la Union Mat. Argentina,
      XVII (1955), p. 103-116.\label{k10:e3} 

    \item Sur les problems aux limits du type dervees oblique, Annals
      of Math. 64(1956), p, 207 - 239.\label{k10:e4}

    \item  Conditions aux limits de Visik-Soboleff et
      problemes mixtes. C.R.Acad. Sci. T 244 (1957)
      p. 1126-1128.\label{k10:e5}  
      
    \item Contribution a un problem de M.M.Picone-Annale di Mat. Pura
      ad applicata (1956).\label{k10:e6} 

    \item Problemes mixtes pour operateurs paraboliues,
      C.R.Acad. Sc. T 244, (1957).\label{k10:e7} 
  \end{enumerate}
\bibitem {k11}{E. Magenes} : Scuola Normale Sup.di Pisa, 10,
  1956, p 75-84. 
\bibitem {k12}{B.Malgrange} : Existence et approximation $\cdots$
  Annales Institut Fourier (1956) 
\bibitem {k13}{Morrey and Nirenberg} : Comm.Pure Applied
  Maths. X(1957) p. 271-290. 
\bibitem {k14}{L.Nirenberg}
  \begin{enumerate}
    \item Remarks on strongly elliptic partial
      differential equations, Comm. Pure Appl. Math. (1955), VIII,
      p.648-674.\label{k14:e1} 

    \item Estimates and existence of solutions of elliptic equations,
      Comm. Pure Appl. Math. IX (1956), p. 509-529.\label{k14:e2}
  \end{enumerate}
\bibitem {k15}{M. Schechter}  On estimating, $\ldots$, Amer. J. of Maths. LXXIX
  (1957), p. 431-443. 
\bibitem {k16}{L.Schwartz:} 
  \begin{enumerate}
    \item Theoric des distributions, Paris (1950) et (1951).\label{k16:e1}  

    \item Seminaire, Paris (1954-1955).\label{k16:e2}

    \item Espaces de fonctions differentiables a valuus
      vectorielles.\label{k16:e3} 
  \end{enumerate}
\bibitem {k17}{Stampacchia:} Ricerche di Matematica, 5 (1956), p. 3-24.
\bibitem {k18}{Visik} Mat. Sborink, 1953.
\bibitem {k19}{Visik-Ladyzenskaya:} Uspecki Mat. Nauk, 1956.
\bibitem {k20}{Visik-Soboleff:} Doklady, 111 (1956), p. 521-523.
\bibitem {k21}{Yosida:} Lectures on Semi-group theory, Tata
  Institute of Fundamental Research, Bombay, (1957). 
\end{thebibliography}

