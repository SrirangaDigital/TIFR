\chapter{Lecture}\label{lec18}%% 18
\setcounter{section}{9}

\subsection[Regularity at the boundary in the case of...]{Regularity at the boundary in the case of problem of\hfil\break
  \textit{oblique type}}\label{lec18:sec9:subsec8}   

Let\pageoriginale $\Omega = \{ x_n > 0 \}$. In \S\ \ref{lec4:sec2:subsec4}, we have defined map
$\gamma$ of $H^1 (\Omega )$ onto $H^{\frac{1}{2}}( \Gamma )$. Let
$\Lambda u = \Sigma \alpha_i \dfrac{\partial u}{\partial x_i}$ where
$\alpha_i$ are \textit{real} constants, and let $a(u, v) = (u, v)_1 +
\lambda (u, v)_o + < \Lambda \gamma u, \gamma \bar{v} >$ with $\lambda
> 0$ be a sesquilinear form on $H^1 (\Omega )$. In \S\  \ref{lec11:sec6:subsec7}, we have
proved that the form $a(u, v)$ is $H^1 (\Omega )$ elliptic and that
the operator $A$ associated with it is $-\triangle + \lambda$. We gave
there a formal interpretation of the space $N$. Now we prove some
regularity theorems justifying the formal interpretation in regular
cases.
  
\begin{theorem}\label{lec18:sec9:subsec8:thm9.4}%theorem 9. 4. 
  If $f \in L^2$ and $u \in N$ is such that $Au = f$,
  then $u \in H^1 (\Omega )$.  
\end{theorem}

As it is by now usual we consider the difference quotients $u^h (x)
=$\break  
$\dfrac{u(x + h) - u (x)}{h}$ and prove that they are bounded in $H^1
(\Omega )$. This will imply that $\dfrac{\partial_u}{\partial x_i} \in
H^1$
for $i = 1, \ldots , n-1$. Next we consider $\dfrac{\partial
  u}{\partial x_n}$. We know $a(u, v) = (f, v)_o$ for all $v
\in H^1 (\Omega )$. Hence $a(u, v^h) = (f, v^h)_o$ for all $v
\in H^1 (\Omega )$. Since $a (u, v)$ has constant coefficients
we have $a(u, v^h) = -a (u^{-h}, v)$. Hence $a(u^{-h}, v) = (f,
v^h)_o$ and so $\big| a (u^{-h}, v) \big| \leq c | v^h |_o \leq c || v
||_1$. Taking $v = u^{-h}$ we have $\alpha || u^{-h}||^2 \leq
\big| a(u^{-h}, u^{-h}) \big| \leq c || u^{-h} ||_1$ . Hence $||
u^{-h} ||_1 \leq c$. Next $- \Delta u + \lambda u = f $ and $\Delta u
= \dfrac{\partial^2 u}{\partial x^2_n} + $ tangential
derivatives. Since $\Delta u \in L^2$, $u \in  L^2 , f \in
L^2$ and as has been proved the tangential derivative are in $L^2$ we
have $\dfrac{\partial^2 u}{\partial x^2_n} \in L^2$, this
complete the proof that $u \in H^1 (\Omega )$.  

The\pageoriginale same proof can be adopted to prove the 

\begin{coro*}%% corollary
  If $f \in H^k$, then $u \in H^{k + 2}$. 
\end{coro*}

If $k$ is large enough, say $2k > n$, then we have proved in that $H^k
(\Omega ) \subset \mathscr{E}^o (\Omega )$. Hence for $2k > n+2, u
\in \mathscr{E}^2 (\bar{\Omega})$. Hence the formal
interpretation given \S\ \ref{lec11:sec6:subsec7} for $u \in N $ is a genuine one
and we have if $2k > n+2$ and if $f \in H^k$ and $u
\in N$ is such that $Au = f$, then $u$ satisfies $\Lambda
\gamma u = \gamma \dfrac{\partial u}{\partial x_n}$.  

\subsection{Regularity at the boundary for some more problems}\label{lec18:sec9:subsec9}

In the \S\ \ref{lec12:sec6:subsec8}6.8 we have considered the case where $V$ consists of $u
\in H^1 (\Omega )$ such that $\gamma u \in H^1 (\Gamma
)$, the topology on $V$ being given by the norm $|| u ||_1 + || \gamma
u ||_1$. If $a (u, v) = (u, v)_1 + \lambda (u, v)_o + (\gamma u,
\gamma v)$ with $\lambda > 0$, then we have proved that $a (u, v )$ is
$V$-elliptic, that the operator defined by $a(u, v)$ is $-\Delta +
\lambda$ and that the boundary value problem solved formally was
$\dfrac{\partial u}{\partial x_n} (x' , 0) = \Delta _\Gamma u$. We
prove now the  

\begin{theorem}\label{lec18:sec9:subsec9:thm9.5}%theorem 9. 5 
  If $f \in L^2$ and $u \in N$ is such that $Au = f$,
  then $u \in H^1 (\Omega )$ and $\gamma u \in H^1
  (\Gamma )$.  
\end{theorem}

\begin{proof}
  First of all we observe that $V$ is closed for translations, i.e.,
  $u \in V \Rightarrow v^h \in V$ for sufficiently small
  $h$. Now we know $a(u, v) = (f, v)_o$ for all $v \in V$ and
  hence $a(u, v^h) = (f, v^h)_o$. Since $a(u, v)$ is with constant
  coefficients $-a (u^{-h}, v) = + a(u, v^h) = (f, v^h)_o$. Hence $\big|
  a (u^{-h}, v)\big| \leq c || v ||_V$. Putting $v = u^{-h}$ we obtain
  $|| u^{-h} ||^2_1 + || \gamma u^{-h} ||^2_1 \leq c || u^{-h}
  ||_1$. Hence $u^h$ and $\gamma u^h$ are bounded so that as usual,
  $D_\tau u \in H^1 (\Omega )$ and $D_\tau \gamma u \in
  H^1 (\Gamma )$. Further since $-\Delta u \in L^2$ and $D_\tau
  u \in H^1 (\Omega)$, we have $\dfrac{\partial^2 u}{\partial
    x_n} \in L^2$. Hence $u \in H^2 (\Omega)$.  
\end{proof}

\begin{coro*}%corollary
  If\pageoriginale $f \in H^k$, then $u \in H^{k +}$. 
\end{coro*}

For sufficiently large $k, e. g. , 2k > n$, we have $H^k \subset
\mathscr{E}^o$. Hence for $k > \dfrac{n}{2} + 1$, the formal boundary
condition becomes a genuine one, and we have  
$$
\frac{\partial u}{\partial x_n} (x', 0) - \Delta \Gamma u = 0. 
$$

\section{Visik-Soboleff Problems}\label{lec18:sec10} %sec 10

\subsection{}\label{lec18:sec10:subsec1}

In a sense these problems generalize non-homogeneous boundary value
problems, e.g., such ones in which solutions of $Au = f$ are sought
which would attain in some sense boundary values given a
priori. However, since not until late this aspect of the problem will
be evident from the way we shall formulate the problem, and since the
hypothesis we shall have to assume in order to ensure the existence
and the uniqueness of solutions will not be obvious, in this lecture
we prefer to discuss the development of the problem and deduce
theorems as consequences thereof.  

Let $\Omega$ be an open set in $R^n$ and $V$ be such that $H^m_o
(\Omega ) \subset V \subset H^m ( ~ )$. Let $Q = L^2 (\Omega )$ and
$a(u, v) = \sum\limits_{|p| , |q| \leq m} \int \Omega a_{pq} D^q_u \overline{D^p
  v} dx + $ some surface integrals for $u, v \in  V$. (However in the
sequel we shall drop surface integrals as their inclusion only
complicates the technical details. ). As in theorem \ref{lec5:sec3:subsec2:thm3.1}, we define
the spaces $N$ and the operator $\Lambda = \sum_{|p| , |q| \leq m}
(-1)^p D^p (a_{pq} D^q)$.  

We shall assume $a(u, v)$ to be V-elliptic, i. e. , $| a(u, u) | \geq
\alpha || u ||^2_m$ . for some $\alpha > 0$ and all $u \subset V$. In
this case it is known that $A$ is an isomorphism of $N$ onto
$L^2$. Let $a^* (u, v) = \bar{a(v, u)}$. Then $\big| a^* (u, u) \big|
\geq \alpha || u ||^2_m$\pageoriginale for all $u \in V$ and the operator
$A^* = \sum (-1)^{|p|} (D^p \overline{a_{qp}(x)}D^q)$ it defines is an
isomorphism of $N^*$ onto $Q = L^2$.  

Suppose now there exists $\mathscr{A}_{qp} \in
\mathscr{D}_{L^\infty}(R^n)$ such that $\mathscr{A}_{pq} = a_{pq}$ on
$\Omega$ and let $\mathscr{A} = \sum (-1)^{|p|} D^p
(\mathscr{A}_{pq}(x) D^q)$. We remark that though $A$ is elliptic,
$\mathscr{A}$ need not be elliptic. Let for $f \in L^2 (\Omega
), u \in N$ be such that $Au = f$. Let $\tilde{f}$ and
$\tilde{u}$ be the extensions of $f$ and $u$ respectively obtained by
defining them to be zero outside $\Omega$. Of course, we do not have
$\mathscr{A}\tilde{u} = \tilde{f}$. The difference $\mathscr{A}u-f$ is
given by the  
\begin{proposition}\label{lec18:sec10:subsec1:prop10.1}%propos 10. 1
  If $u \in N$ be such that $Au = f$, then for every $v
  \in H^{2m} (R^n)$ such that $v_\Omega \in N^*$ we have
  $< \mathscr{A} u - f, \bar{v} > = 0$, where $v_\Omega$ is the
  restriction of $v$ to $\Omega$.  
\end{proposition}

\begin{proof}
  $< \mathscr{A} u -f, \bar{v}> = < \tilde{u}, \overline{\mathscr{A}^*
    v}  > - <
  \tilde{f}, \bar{v}>$ for $v \in H^{2m} (R^n)$.  
\end{proof}

Now since $u$ vanishes outside $\Omega$, we have
$$
< \tilde{u}, \mathscr{A}^* v > = (u, A^* v_\Omega )_o = (\overline{A^*
  v_\Omega , u}_o).  
$$

Since $v_{\Omega} \in N^*$ we have $\overline{(A^* v_{\Omega},
  u)_o = \phi a^*(v_{\Omega}, u)} = a(u, v_{\Omega}) = (Au,
v_{\Omega})_o$. Further $< \tilde{f}, \bar{v} > = (f, v_{\Omega})$ as
$f$ vanishes outside $\Omega$. Hence 
$$
< \mathscr{A} \tilde{u} - \tilde{f}, \bar{v} > = < Au - f, \bar{v} > =
0, \text{ for } v \in H^{2m} (R^n) 
$$
such that $v_{\Omega} \in N^*$. 

Now arises the converse problem. Let $w \in L^2 (R^n)$ be such
that the support of $w$ is contained in $\bar{\Omega}$ and let there
exist $f \in L^2 (\Omega)$ such that $< \mathscr{A} w - f,
\bar{v} > = 0$ for all $v \in H^m (R^n)$ such that $v_{\Omega}
\in N^*$. Does there exist $u \in N$ such that $w =
\tilde{u}$ and $Au = f$. Let $u_o \in N$ be the solution of
$Au_o = f$. By proposition \ref{lec18:sec10:subsec1:prop10.1}, $< \mathscr{A} \tilde{u}_o -
\tilde{f}, \bar{v} > = 0$ for\pageoriginale $v \in H^m (R^n)$ such that
$v_{\Omega} \in N^*$. Hence $ < \mathscr{A} - w- \tilde{u}_o,
\bar{u} > = 0$, i.e., $ < - (w -\tilde{u}_o), \mathscr{A}^* v > =
0$. Since and $\tilde{u}_o$ have their support in $\Omega$, the above
means $(w-\tilde{u}_o, \mathscr{A}* v_{\Omega}) = 0$.  

In order to have $w-\tilde{u}_o = 0$, we have to secure that $\Lambda
^* v_{\Omega}$ be dense in $L^2 (\Omega )$. $A^*$ being an isomorphism
of $N^*$ onto $L^2 (\Omega)$ we must consider when the solution $x
\in N^*$ of $Ax = g$ is restriction of a $v \in H^{2m}
(R^n)$. This would follow $(a)$ if we should apply the theory of \S\
\ref{lec15:sec9}. Then it would follow that $x \in H^{2m} (\Omega )$, and
$(b)$ if $\Omega$ had $2m$-extension property, then there would exist
$x \in H^{2m} (R^n)$, such that $(\pi x)_{\Omega} = x$.  

In other words, for every $g \in L^2 (\Omega )$ there exists
$v_\Omega$ such that $A^* v_\Omega = g$ if the above two conditions are
satisfied. We have proved then the  
\begin{proposition}\label{lec18:sec10:subsec1:prop10.2}%proposition10. 2
  Besides the hypothesis of Proposition \ref{lec18:sec10:subsec1:prop10.1}. , assume 
  \begin{enumerate}[\rm 1)]
  \item $A^* u = g$ with $u \in N^*$ and $g \in H^o$
    implies $u \in H^{2m} (\Omega)$,  
  \item $\Omega$ has $2m$-extension property, and
  \item there is given $\in H^o (R^n)$ such that the support of
    $\bar{\Omega}$ 
    contained in $\bar{\Omega}$ and $ < \mathscr{A}w - f, \bar{v} > = 0$
    for all $v \in H^{2m} (R^n)$ such that $v_{\Omega} \in
    N^*$. Then $w u_o, u_o \in N$ being the solution of $Au_o = f$.  
  \end{enumerate}
\end{proposition}

\begin{remarks*}
  Since $\mathscr{D}$ is dense in $L^2 (\Omega )$ instead of assuming
  the theory of \S\ \ref{lec15:sec9}, it would be enough to assume that $A^* x = g$,
  $g \in \mathscr{D} (\Omega )$ implies $x \in H^{2m}
  (\Omega)$.  
\end{remarks*}

(2) It is not known whether (1) and (2) in proposition 10.2
  are independent or not, or whether (2) is a consequence of (1)
  . The condition (3) can be put more succinctly by making the
  following 
  \begin{definition}\label{lec18:sec10:subsec1:def10.1}% defini 10.1
    $M^o$\pageoriginale is the subspace of $H^{-2m} (R^n)$ consisting of distribution
  $T$ such that $< T, \bar{v} > = 0$ for all $v \in H^{2m}
  (R^n)$ such that $v_{\Omega} \in N^*$.  
\end{definition}

It is easily seen that $M^o$ is a closed subspace of $H^{-2m} (R^n)$
and that the support of $T \in M^o$ is contained in
$\lceil$. We may summarize the proposition \ref{lec18:sec10:subsec1:prop10.1} and \ref{lec18:sec10:subsec1:prop10.2} in the following.  

\begin{theorem}\label{lec18:sec10:subsec1:thm10.1}%theorem10.1
  Under the hypothesis of Proposition \ref{lec18:sec10:subsec1:prop10.1} and \ref{lec18:sec10:subsec1:prop10.2} the boundary value
  problem ``Given $f \in L^2 (\Omega )$, find $u \in N$
  such that $Au = f''$ is equivalent to ``Given $f \in L^2
  (\Omega )$, find $w ~ \in H^o (R^n)$ such that $\mathscr{A} w
  - \tilde{f} \in M^o$''.  
\end{theorem}

\subsection{}\label{lec18:sec10:subsec2}

Now the second formulation has an advantage over the first one that it
can be generalized. In the first instance we notice that instead of
$f$ we could take any $T \in H^{-2m}_{\bar{\Omega}}$\footnote{$H^{-m}
  (\Omega)$ consists of $u \in H^{-m}(R^n)$ such that the support of $u \subset
  \bar{\Omega}$} and raise the problem 
\begin{problem}\label{lec18:sec10:subsec2:prob10.1}%problem 10. 1
  Given $T \in H^{-2m}_{\bar{\Omega}}$ does there exist $w
  \in L^2 (R^n)$ with the support in $\Omega$ such that
  $\mathscr{A} w - T \in M^o$.  
\end{problem}

Similarly a much general problem could be formulated by defining new
spaces $M^k$.  
\begin{definition}\label{lec18:sec10:subsec2:def10.2}%definition 10. 2
  $M^k$ is the subspace of $H^{-(k+2m)}(R^n), k$ being a non-negative
  integer, such that $< T, \bar{v} > = 0$ for all $v \in
  H^{k+2m} (\Omega )$ with $v_{\Omega} \in N^*$.  
\end{definition}

\begin{lemma}\label{lec18:sec10:subsec2:lem10.1}%lemma 10. 1
  $M^k$ is a closed subspace of $H^{-k + 2m}$. 
\end{lemma}

\begin{proof}
  We prove only that the support of $T \in M^k$ is contained in
  the other assertion being then obvious. If $\varphi \in
  \mathscr{D} (\sigma \bar{\Omega})$, take $v = \tilde{\varphi}$. Then
  $v_{\Omega} =0$, and hence is in $N^*$. Then $< T, \bar{\varphi} > = <
  T, v_{\Omega} > = 
  0$. Hence the support of $T$ is contained in $\bar{\Omega}$. If now
  $\varphi \in \mathscr{D} (\Omega)$,  
  then\pageoriginale again let $v = \tilde{\varphi}$. Now $v \in H^{k+2m}
  (R^n)$ and $v_{\Omega} = \varphi \in  N^*$. Hence $< T, \varphi
  > = 0$. This proves that the support of $T$ is contained in $\Gamma$.  
\end{proof}

We have now the 
\begin{problem}\label{lec18:sec10:subsec2:prob10.2}%probl 10. 2
  Given $T \in H^{-(k + 2m)}$ does there exist $U \in
  H^{-k} (R^n)$ with support in $\bar{\Omega}$ such that $\mathscr{A} U
  - T \in M^k$. For $K = 0$ we get the problem \ref{lec18:sec10:subsec2:prob10.1} 
\end{problem}
