
\chapter{Lecture}\label{lec10}%%% 10

\setcounter{section}{5}
\section{Operators of order 2}\label{lec10:sec6}%sec 6

\subsection{}\label{lec10:sec6:subsec1}

Hitherto\pageoriginale we considered the problems in which the $V$-elliptic from $a
(u,v)$ was given a priori and then we solved boundary value problems
for the operator $A$ associated with the form $a (u,v)$. Now we with
to consider the natural converse  

\begin{prob*} 
Given a differential operator $A$, determine the spaces $V$ and
$V$-elliptic forms $a(u, v)$ on $V$ such that 
\begin{enumerate}[\rm 1)]
\item $\langle Au, \bar{\varphi} \rangle = a (u, \varphi)$ for all $u
  \in V$ and $\varphi \in \mathscr{D} (\Omega)$ 
\item $a(u,v)$ is V-elliptic.
\end{enumerate}
\end{prob*}

Stated in this general from the problem has not been completely
solved, even in the case of differential operator of order $2$;
however, several results, depending on the domain $\Omega$,
coefficients of $A$, $V$ and $a(u,v)$ are know and we give some of
these. 

We shall always consider the case when $V \subset H' (\Omega)$. We
take a second order differential operator $A$ in the form 
$$
A=  \sum_{i,j=1}^{n} \frac{\partial}{\partial x_i} \left(g_{ij}(x)
\frac{\partial}{\partial x_i}\right) + \sum g_i (x) \frac{\partial}{\partial
  x_i} + g_o (x), g_{ij}, g_i , g_o \text{ in }~ L^{\infty} (\Omega). 
$$

A more general form would be $\sum\limits_{|p| \leq 2} a_p (x)
D^p$ which reduces to the above if $a_p (x)$ are regular enough. 

We associate with $A$ the form
$$
a(u,v) = \sum_{i, j=1}^{n} \int\limits_{\Omega} g_{ij} \frac{\partial
  u}{\partial x_j} \frac{\partial \bar{v}}{\partial x_i} dx + \sum
\int\limits_{\Omega} g_i \frac{\partial u}{\partial x_i} \bar{v} +
\int g_o u \bar{v} 
$$
and consider the ellipticity of this form. Another kind of 
sesquilinear from will be considered later. We observe that with the 
same operator several\pageoriginale forms can be associated in the above fashion, 
merely by rearranging the operator. For instance, let   
$$
A =- \frac{\partial ^2}{\partial x^2 _1}- \frac{\partial^2}{\partial
  x_1 \partial x_2} - \frac{\partial ^2}{\partial x^2_2}. 
$$

We may write
$$
A=- \frac{\partial^2}{\partial x^2_1} - \frac{\partial^2}{\partial
  x^2_2}- \frac{\partial}{\partial x_2} \left(\frac{1}{2} + i\right)
\frac{\partial}{\partial x_2} - \frac{\partial}{\partial x_1}
\left(\frac{1}{2} - i\right) \frac{\partial}{\partial x_2}.  
$$

The associated forms are
\begin{align*}
  a(u,v) & = \left(\frac{\partial u}{\partial x_1}, \frac{\partial
    v}{\partial x_1}\right)_o + \left(\frac{\partial u}{\partial x_2},
  \frac{\partial v}{\partial x_2}\right)_o + \left(\frac{\partial
    u}{\partial x_2}, \frac{\partial v}{\partial x_1}\right)_o, \text{and}\\ 
  a(u,v) & = \left(\frac{\partial u}{\partial x_1} , \frac{\partial
    v}{\partial x_1}\right)_o + \left(\frac{\partial u}{\partial x_2},
  \frac{\partial v}{\partial x_2}\right)_o + \left(\frac{1}{2} +
  i\right) \left(\frac{\partial 
    u}{\partial x_1}, \frac{\partial v}{\partial x_2}\right)_o\\ 
  & \hspace{5cm}+
  \left(\frac{1}{2} -i\right) \left(\frac{\partial u}{\partial x_1} ,
  \frac{\partial v}{\partial x_2}\right) 
\end{align*}
which are different.

Let $(u, v)_g$ be the leading part of $a(u,v)$,
$$
(u, v)_g = \sum_{i, j=1}^{n} \int g_{ij} \frac{\partial u}{\partial
  x_j} \frac{\partial \bar{v}}{\partial x_i} dx. 
$$

To determine when $a(u, v)$ is elliptic, we have to investigate when

$\re  (u,u)_g \geq \alpha |u|^2_1$ for all $u \in V$ and for
some $\alpha > 0$. 

\subsection{}\label{lec10:sec6:subsec2}

\begin{theorem}\label{lec10:sec6:subsec2:thm6.1}%the 6.1
  Let $\Omega$ be a bounded open set in $R^n, g_{ij}$ be constants and
  $V = H^1 (\Omega)$. A necessary and sufficient condition that 
  \begin{equation}
    \re  (u,u)_g \geq \alpha |u|^2_1 \text{ for all } u \in H^1
    (\Omega) \tag{1}\label{lec10:sec6:subsec2:eq1} 
  \end{equation} 
 is that 
\begin{equation*}
  \sum (g_{ij} + \bar{g}_{ij})p_i \bar{p}_i~ \text{for all complex}~
  (p_i).\tag{2}\label{lec10:sec6:subsec2:eq2}
\end{equation*}
\end{theorem}

\noindent \textit{Proof.}
\begin{enumerate}[(a)]
\item {\em Necessity}. Let $u(x) = \sum\limits_{i=1}^{n} p_i
  x_i$. Because $\Omega$ is bounded $u (x) \in H^1
  (\Omega)$. Hence by (\ref{lec10:sec6:subsec2:eq1}) 
\begin{gather*}
  \re \left(\sum \int\limits_{\Omega} g_{ij} p_i \bar{p}_i dx\right) \geq \alpha
  \sum |p_i|^2 \int\limits_{\Omega}dx,
\end{gather*}
  i.e., \hspace{2.5cm} $\re \left(\sum g_{ij} p_j \bar{p}_i \right)
  \geq \alpha \sum |p_i|^2$ \hfil which is  (\ref{lec10:sec6:subsec2:eq2}). 

\item \textit{Sufficiency}.\pageoriginale From (\ref{lec10:sec6:subsec2:eq2}) we have
$$
\sum (g_{ij} + \bar{g}_{ji}) \frac{\partial u}{\partial x_j} (x)
\frac{\partial \bar{v}}{\partial x_i} (x) \geq \alpha \sum
|\frac{\partial u}{\partial x_i} (x)|^2 ~\text{a.e.} 
$$

Integrating over   
$$
\sum \int\limits_{\Omega} (g_{ij} + \bar{g}_{ji}) \frac{\partial
  u}{\partial x_j} \frac{\partial \bar{u}}{\partial x_i} dx \geq
\alpha |u|^2_1 
$$
i.e., \hspace{4cm} $\re (u, u)_g \geq \alpha |u|^2_1$. \hfill $\Box$
\end{enumerate}

\begin{theorem}\label{lec10:sec6:subsec2:thm6.2}%the 6.2
  Let $\Omega = R^n$, and $g_{ij}$ be constant. Then a necessary and
  sufficient condition in order that (\ref{lec10:sec6:subsec2:eq1}) holds is that  
  \begin{equation*}
    \re  \left(\sum g_{ij} \xi_i \xi_j\right) \geq \alpha \sum \xi_i ^2 \text{ for
      real }\xi_i \text{ and for some } \alpha > 0 \tag{3}\label{lec10:sec6:subsec2:eq3} 
  \end{equation*}
\end{theorem}

 (We observe (\ref{lec10:sec6:subsec2:eq2}) $\Longrightarrow$ (\ref{lec10:sec6:subsec2:eq3}),  but converse is not true, e.g., the example quoted above). 

\begin{proof}
  By Fourier transform
  \begin{align*}
    (u, u)_g & = \sum g_{ij} \int 2 \pi i \xi_i \hat{u}. \overline{2 \pi i
      \xi_j \hat{u} d \xi}\\ 
    & = 4 \pi^2 \int \sum g_{ij} \xi_i \xi_j |\hat{u}|^2 d \xi.
  \end{align*} 
\end{proof}
 
 Hence (\ref{lec10:sec6:subsec2:eq1}) is equivalent to
 \begin{equation*}
   \re \left(\int \sum _{ij} \xi_i \xi_j |\hat{u} (\xi)|^2 d
   \xi\right) \geq \alpha 
   |\xi|^2 |\hat{u} (\xi)|^2 d \xi \text{ for all } u \in H^1
   \tag{4}\label{lec10:sec6:subsec2:eq4} 
 \end{equation*} 
 
Let \qquad $p (\xi) = \re  (\sum g_{ij} \xi_i \xi_j) - \alpha |\xi|^2$.

Form (\ref{lec10:sec6:subsec2:eq4}), (\ref{lec10:sec6:subsec2:eq1}) is equivalent to
\begin{equation}
  \int P(\xi ) |\hat{u}(\xi)|^2 d \xi \geq 0 \tag{5}\label{lec10:sec6:subsec2:eq5}
\end{equation}

We have to prove (\ref{lec10:sec6:subsec2:eq5}) holds if and only if $P (\xi) \geq 0$.

Sufficiency is trivial. To see the necessity if $P (\xi_o)< 0, P (\xi)
< 0$ in a certain neighbourhood and then to obtain a contradiction we\pageoriginale
need take $u$ the Fourier transform of which has support in this
neighbourhood. 

The following problem however is not answered: If $x_i \in H^1
(\Omega)$, $(\Omega)$ of capacity $> 0$ is (\ref{lec10:sec6:subsec2:eq2}) necessary in order
that (\ref{lec10:sec6:subsec2:eq1}) holds for $u \in H^1 (\Omega)$. 
\medskip

\subsection{\texorpdfstring{$V = H^1_o (\Omega), g_{ij}$}{VH1} constant}\label{lec10:sec6:subsec3}

\begin{theorem}\label{lec10:sec6:subsec3:thm6.3}%the 6.3
  Let $V = H^1 _o (\Omega)$ and $g_{ij}$ be constant. A necessary and
  sufficient condition in order that 
\end{theorem}

\begin{equation*}
  \re  (u,u)_g \geq \alpha |u|^2_1~\text{for all}~ u \in H^1_o
  (\Omega) ~\text{for some}~ \alpha > 0 \tag{6}\label{lec10:sec6:subsec3:eq6} 
\end{equation*}
is that
\begin{equation*}
\re  \left(\sum g_{ij} \xi_i \xi_j\right)\geq \alpha \sum |\xi|^2 ~\text{for all
}~  \xi_i \in \mathbb{Z}^n. \tag{7}\label{lec10:sec6:subsec3:eq7} 
\end{equation*}

\begin{proof}%proof
  In order to apply theorem \ref{lec10:sec6:subsec2:thm6.2} we prove that (\ref{lec10:sec6:subsec3:eq6}) implies that (\ref{lec10:sec6:subsec2:eq1})
  holds for $u \in H^1_o (R^n) = H^1 (R^n)$. We require a
  lemma. We may assume without loss of generality that the origin is in
  $\Omega$. Further, we observe $\cup \lambda \Omega = R^n$. 
\end{proof}

\begin{lemma}\label{lec10:sec6:subsec3:lem6.1}
  (\ref{lec10:sec6:subsec3:eq6}) holds if and only if $\re  (u, u)_g \geq \partial |u|^2_1$ for
  all $ u \in H^1 _0 (\Omega) $ for all $\lambda$.  
\end{lemma}

\begin{proof}
  Let $u \in H^1_0(\Omega)$. Define $u_\lambda (x) = u(\lambda
  x)$ for $x \in \Omega$.  
\end{proof}

It is easily seen that $u_\lambda \in H^1_o$. From (\ref{lec10:sec6:subsec3:eq6}) we get 
\begin{equation*}
\re  \left( \sum\int\limits_\Omega g _{ij} \frac{\partial
  u_\lambda}{\partial x_j}\frac{\bar{\partial u_\lambda}}{\partial
  x_i}  dx\right) \geq \alpha \sum\int\limits_\Omega \bigg|  \frac{\partial
  u_\lambda}{\partial x_i}\bigg|^2 dx \tag{8}\label{lec10:sec6:subsec3:eq8} 
\end{equation*}
Since $\dfrac{\partial u_\lambda }{\partial x_i}(x) = \lambda
\dfrac{\partial u(\lambda x)}{\partial x_i}$, from (\ref{lec10:sec6:subsec3:eq8}) we get  
$$
\re  \left(\sum \int\limits_\Omega g_{ij} \frac{\partial u ( \lambda
  x)}{\partial x_i} ~ \overline{\frac{\partial u (\lambda x)}{\partial x_i}}dx\right)\geq \alpha \sum \int \bigg| \frac{\partial u
  (\lambda x)}{\partial x_i}\bigg|^2 dx. 
$$

Putting\pageoriginale $\lambda x = y$, we get the required inequality and lemma
(\ref{lec10:sec6:subsec3:lem6.1}) is proved. 

Returning to the proof of theorem, let $\varphi \in
\mathscr{D}(R^n)$. There exists $\lambda $ such that $K \subset
\lambda \Omega$. Then $\varphi \in \mathscr{D}(R^n)$. and
hence $\varphi \in H^1_o(\lambda \Omega)$. This means  
$$
\re  ~(u, u)_g \geq \alpha |u|^2, \text{ for all } \varphi \in
\mathscr{D} (R^n). 
$$

Since $\mathscr{D}(R^n)$ is dense in $H^1_o (R^n)$, we have proved 
$$
\re ~ (u, u)_g \geq \alpha |u|^2_1 \text{ for all } u \in H^1_o (R^n).
$$

Theorem \ref{lec10:sec6:subsec2:thm6.2} then gives (\ref{lec10:sec6:subsec3:eq7}). 

\subsection{}\label{lec10:sec6:subsec4} 

Some problems with variable coefficients : $V =
H^1_0(\Omega)$. 

\begin{theorem}\label{lec10:sec6:subsec4:thm6.4} %Theorem 6.4
  Let $\Omega$ be any open set in $R^n$ and $g_{ij}$ be continuous. 
  
  If
  \begin{equation*}
    \re  (u , u)_g \geq \alpha |u|^2_1 ~\text{ for all }~ u \in
    H^1_0 (\Omega), \tag{9}\label{lec10:sec6:subsec4:eq9}
  \end{equation*}
  then
  $$
  \re  \sum g_{ij}(x_o) \xi _i \xi _j \geq \alpha \sum_{i=1}^n \bigg|
  \xi _i\bigg| ~\text{for all }~ (\xi_i) \in R^n .
  $$
\end{theorem}

\begin{proof}
  Given any $\in > 0$, let $B$ be a neighbourhood of $x_o$ such that
  $$
  |(u, u)_{g(x_o)} - (u, u)_g | \le \in | u |^2_1 \text{ for all
  } u \in H^1_o (B). 
  $$
  
  We need choose $B_\epsilon$ such that $\Big| g_{ij}(x) - g_{ij}(x_o)\Big|$
  are sufficiently small. (\ref{lec10:sec6:subsec4:eq9}) gives then  
  $$
  \re  (u , u)_g \geq (\alpha - \in |u|^2_1) \text{ for all } u
  \in H_0^1 (\beta _\in).  
  $$
From theorem \ref{lec10:sec6:subsec3:thm6.3}, it follows that  
$$
\re  \sum g_{ij}(x_o) \xi_i \xi_j \geq (\alpha - \xi ) \sum |\xi|^2_i. 
$$
Since this is true for arbitrarily small $\in $, we have 
$$
\re  \sum g_{ij}(x_o) \xi_i \xi_j \geq (\alpha - \xi ) \sum |\xi _i |^2. 
$$

Regarding the sufficiency of the above condition, we have 
\end{proof}

\begin{theorem}\label{lec10:sec6:subsec4:thm6.5} %Theorem 6.5
  Garding's\pageoriginale inequality. If $\re  \Sigma g_{ij} \xi_i \xi_j \geq
  \alpha \Sigma |\xi_i|^2$ for some $\alpha > 0$ for all $x \in
  \bar{\Omega}$ and $\Omega$ is bounded then there exists $\lambda
  > 0$ such that  
  $$
  \re  (u, u)_g + \lambda |u|^2_0 \geq \alpha |u|^2_1 \text{ for all }u
  \in H^1_o (\Omega).   
  $$
  
  We do not prove this. For a proof, see Yosida \cite{k21}.
\end{theorem}

We have a general sufficient condition
\begin{theorem}\label{lec10:sec6:subsec4:thm6.6} %Theorem 6.6
  If $\Sigma (g_{ij} + \bar{g}_{ji}) p_j \bar{p}_i \geq \alpha
  \Sigma|p_i|^2)$ for some $\alpha > 0$ and $p_i $complex $a.e$. in
  $\Omega$, then  
  $$
  \re  (u, u)_g \geq \alpha |u|^2_1\text{ for all } u \in H^1 (\Omega). 
  $$
\end{theorem}

Having seen some cases when $\re  (a(u, u)_g) \geq \alpha |u |^2_1$ we
see now some examples when different forms $a (u, v)$ giving rise to
the same operator $A$ are $V$-elliptic.  

\begin{enumerate}[1)]
\item Let $a(u, v)=  (u, v)_g + (g_o u, v)_o$ with $g_o (x)\geq \beta > 0$. 

  Then 
  $$
  \re  (a(u, u)) \geq \alpha |u|^2_1 + \beta|u_0|^2 \geq min (\alpha,
  \beta) || u||^2_1. 
  $$

  Hence $a(u, v)$ is $V$-elliptic for any $V$  such that $H^1_o \subset
  V \subset H^1$. 
\item Let $a(u, v) = (u, v)_g + (g_o u, v)_o + \Sigma \left(g_i
  \dfrac{\partial u }{\partial x_i}, v\right)_o, g_i$ real constants, $
  g_o (x)\geq \beta > 0$. Let $V = H^1_0 (\Omega)$. Let $V =
  H^1_0(\Omega)$. We first observe that for $u 
  \in H^1_0(\Omega) \re  \left(u, \dfrac{\partial u}{\partial
    x_i}\right)= 0$. For, if $\varphi \in \mathscr{D}(\Omega)$, by
  integration by parts $\left(\dfrac{\partial \varphi}{\partial
    x_i},\varphi \right)_o = \left( \varphi \dfrac{\partial \varphi }{\partial
    x_i}\right)_o$ and since  
  $$
  \re  \left(\frac{\partial \varphi }{\partial x_i}, \varphi\right)_o =
  \left(\frac{\partial \varphi }{\partial x_i}, \varphi\right)_o +
  \left(\varphi, \frac{\partial \varphi }{\partial x_i}\right)_o 
  $$
  we have $\re  (\varphi , \dfrac{\partial \varphi}{\partial x_i}) = 0$
  for all $\varphi \in \mathscr{D}(\Omega)$. Since
  $\mathscr{D}(\Omega)$ is dense in $H^1_o(\Omega) $we have the result
  for $u \in H^1_o(\Omega)$. Hence $\re  (a (u, u)) = \re  (u,
  u)_g+ \Re (g_o u, u)$. Hence $a(u, v)$ is $H^1_o (\Omega)$ elliptic.  
\end{enumerate}
