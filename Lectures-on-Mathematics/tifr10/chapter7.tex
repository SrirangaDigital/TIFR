
\chapter{Lecture}\label{lec7}%%% 7

\setcounter{section}{3}
\subsection{}\label{lec7:sec3:subsec4}  

Hitherto\pageoriginale we considered the particular
case where $H^1_0 \subset V \subset  H^1$. Now we shall consider a
more general case in which $\mathscr{D} \subset V \subset Q \subset
\mathscr{D}', \mathscr{D}$ being \textit{dense} in $Q$, but not
necessarily in $V$. Involution in $Q$ is as before, viz. $f
\rightarrow \bar{f}$. 

Let $a(u, v)$ be a continuous sesquilinear form on $V$. In this
situation the operator $A$ and the space $N$ associated with $a(u, v)$
can be characterized in another way as follows. For a fixed $u
\in V$, the mapping $\varphi \rightarrow a(u, \varphi)$ for
$\varphi \in \mathscr{D}$ is a continuous semi-linear form on
$\mathscr{D}(\Omega)$ and hence defines an element $\mathscr{A} u
\in \mathscr{D}' (\Omega)$ so that $<\mathscr{A}u,
\bar{\varphi}> = a(u, \varphi)$. 

This defines a mapping $\mathscr{A} : V \rightarrow \mathscr{D}'
(\Omega)$. Let $\eta$ be the space of $u \in V$ such that $(a)
\mathscr{A} u \in Q'$ and $(b) <\mathscr{A}u, \bar{\varphi}> =
a(u, v) $ for all $v \in V$. On $\eta$ we introduce the
topology so as to make both the injection $\eta \rightarrow V$ and the
mapping $\mathscr{A} : \eta \rightarrow Q'$ continuous. 

\begin{theorem}\label{lec7:sec3:subsec4:thm3.5}%thm 3.5
  $\eta = N$ and for $u \in' N, \mathscr{A}u = Au$.
\end{theorem}

\begin{proof}
\begin{enumerate}[1)]
\item Let $u \in N$. Then $v \rightarrow a (u, v)$ is a
  continuous semilinear form on $V$ with the topology induced by $Q$
  and $a(u, v) = <Au, \bar{v}>$ with $Au \in Q'$. This holds
  in particular if $v= \varphi \in \mathscr{D}
  (\Omega)$. Hence $a(u, \varphi) = <Au, \bar{\varphi}> =
  <\mathscr{A}u, \bar{\varphi}>$ for all $\varphi \in
  \mathscr{D}(\Omega)$. This means $\mathscr{A}u =Au$ and that
  $\mathscr{A} u \in Q'$. Hence $<\mathscr{A}u, \bar{v}> =
  <Au, \bar{v}> = a(u, v)$ for all $v \in V$ and so $u
  \in \mathcal{M}$. 
\item Conversely, let $u \in \mathcal{M}$. Then $a(u, v) =
  <\mathscr{A}u, \bar{v}>$ for all $v \in V$ and $\mathscr{A}u
  = f \in Q'$. Hence $a(u, v) = <f, \bar{v}>$ so that the
  mapping\pageoriginale $v \rightarrow a(u, v)$ is continuous on $V$ with the
  topology induced by $Q$. Hence $u \in N$ and $\mathscr{A}u =
  f = Au$. 
\end{enumerate}
\end{proof}

\begin{remark*}%remark
  In practice it is the operator $\mathscr{A} $ that is known a priori
  and $A$ is the restriction of  $\mathscr{A} $to $N$. We agree however
  to denote $\mathscr{A} $ by $A$ itself. 
\end{remark*}

\heading{Generalizations.}

Let $\nu$ be an integer. If $E$ is a topological vector space, let
$E^\nu $ be $E \times \ldots \times E$, the topology on $E^\nu$ being the
product topology. Let $V, Q$ be such that $\mathscr{D}(\Omega)^\nu
\subset V \subset Q \mathscr{D}'(\Omega)^\nu$. Let $a(u, v)$ be a
continuous sesquilinear form on $V$. As in before, we can define the
operator $\mathscr{A} \in \mathscr{L}(V,
\mathscr{D}'^\nu)$. The operator $\mathscr{A} $ on
$\mathscr{D}(\Omega)$ may be considered to be a generalisation of
differential systems. 

\heading{General examples:}
\begin{enumerate}[a)]
\item An interesting example of the above kind would be where $V$ is
  the set of functions continuous on a given discrete set in
  $R^n$. The solutions of this problem may be considered to be finite
  difference approximation to boundary value problems. 
\item Let $\Omega$ be an open set in $R^n$ and $A_1, \ldots , A_\nu$
  be differential operators with constant coefficients. Let $V$ be such
  that $H^0 (A, \Omega) \subset V \subset H^1 (A, \Omega)$. Let $Q =
  L^2(\Omega)$. Then $\mathscr{D}(\Omega)\subset V \subset Q \subset
  \mathscr{D}'(\Omega)$ and $\mathscr{D}(\Omega)$ is dense in $Q$. Let 
$$
a(u, v) = \sum^{\nu}_{j, i=1} \int_{\Omega} g_{ij}(x)A_j
(u)\overline{A_i(v)} dx + \int_{\Omega} g_0 (x) u \bar{v} dx 
$$
with $ g_o, g_{ij} \in L^{\infty}(\Omega)$. 
$a(u,v)$\pageoriginale is a continuous sesquilinear form on $V$. The corresponding
operator $\mathscr{A} = \sum A^*_i (g_{ij}A_j) + g_0 $. 
\end{enumerate}

\subsection{Green's kernel}\label{lec7:sec3:subsec5}

We have proved that in the case $a(u, v)$ is $V$-elliptic, the operator
$A$ is an isomorphism of $N$ onto $Q'$. Let $G$ be the \textit{inverse
  operator} of $A$. $G$ is then an isomorphism of $Q'$ onto $N$. The
restriction of $G$ to $\mathscr{D}(\Omega)$ is then a continuous
mapping of  $\mathscr{D}(\Omega)$ into $\mathscr{D}'(\Omega)$ and
conversely the restriction of $G$ to $\mathscr{D}(\Omega)$ defines $G$
uniquely $\mathscr{D}$ is dense in $Q'$. 

Now, $L$. Schwartz's kernel Theorem [\ref{k16:e3}] states that any continuous
mapping of $\mathscr{D}$ into $\mathscr{D}'$ is defined by an element
of $\mathscr{D}'(\Omega_x \times \Omega_y)$, the space of distributions on
$\Omega_x \times \Omega_y$. 

Thus $G$ defines an element $G_{x,y} \in \mathscr{D}'(\Omega_x
\times \Omega_y)$. 

\begin{definition}\label{lec7:sec3:subsec5:def3.3}%definition 3.3
  $G_{x,y}$ defined above is called the Green's kernel of the form $a(u,
  v)$ on $V$. 
\end{definition}

\subsection{Relations with unbounded operators}\label{lec7:sec3:subsec6}

Let $\Omega$ be an open set in $R^n . V, Q$ be two vector spaces not
necessarily of distributions, $Q$ being a Hilbert space and $V \subset
Q$. Let $a(u, v)$ be a continuous sesquilinear form on $V$. As we
have seen already in (\S\ \ref{lec5:sec3:subsec1}), this defines a space $N$ and an
operator $A : N \rightarrow Q$ by identifying $Q'$ and $Q$. This
operator \textit{in the topology induced on $N$ by $Q$} is an
unbounded operator. 

Let $a^*(u, v) = \overline{a(v, u)}$. On $V, a^*(u,v)$ is a continuous
sesquilinear form. Let the spaces $N$ and operator $A$ associated with\pageoriginale
$a^*(u, v) $ be denoted by $N^*$ and $A^*$, 
$$
\displaylines{
\text{i.e.,}\hfill u \in N^* \Leftrightarrow v \rightarrow A^*(u,
v)\hfill}
$$
is continuous on $V$ with the topology induced by $Q$ and 
$$
a^*(u, v) = <A^*u, \bar{v}> = (A^*u, v)_Q.
$$

We shall give a theorem establishing relationships
between usual concepts associated with the unbounded operators and $N,
A$ and $A^*$. 

\begin{theorem}\label{lec7:sec3:subsec5:thm3.6}%thm 3.6
  Suppose there exists $\lambda > 0$ such that  
  $$
  \re  a(u, v) + \lambda |v|^2_Q \ge \alpha |u|^2_v \text{ for all }
  \quad u \in V. 
  $$
\end{theorem} 
Then 

 \begin{tabular}{lll}
(1) &  $N$ is dense in $Q$. &\\
(2) &  \multicolumn{1}{p{3cm}|}{$A$ is closed.} &
   \multirow{2}{4cm}{(definitions will be recalled in the course of
     proof)}\\ 
(3) &  \multicolumn{1}{p{3cm}|}{$A^*$ is the adjoint of $A$.}  
 \end{tabular}

\begin{proof}
  We first prove that $A$ is closed. We have to prove that if $u_n
  \in D_A$ (the domain of definition of $A$) and if $u_n
  \rightarrow u$ in $Q$ and $Au_n \rightarrow f$ in $Q$, then $u
  \in D_A$ and $Au = f$. 
\end{proof}

$a(u, v) + \lambda (u, v)$ is a continuous sesquilinear form on $V$
and the space and the operator associated with it are $N$ and $A +
\lambda$ respectively. By assumption this form is V-elliptic and hence
by theorem \ref{lec5:sec3:subsec2:thm3.1}, $A + \lambda$\pageoriginale is an isomorphism of $N$ onto $Q' =
Q$. 

Now, $(A + \lambda) u_n \rightarrow f + \lambda~ u~ in~ Q$ and hence
\begin{gather*}
u_n =  (A + \lambda)^{-1}  (A + \lambda) u_n \rightarrow  (A +
\lambda)^{-1} (f + \lambda_u)~ in~ N. 
\end{gather*}

Hence $u_n \rightarrow  (A + \lambda)^{-1}(f + \lambda u)$ in $Q$ also
and so $u =  (A + \lambda)^{-1} (f + \lambda u)$, and $u \in
N$. Further $Au_n \rightarrow Au$ in $Q$ and so $Au = f$. Hence $A$ is
closed. 

Now we prove that $N$ is dense in $Q$. We need prove if $f \in
Q$ and $(u, f)_Q = 0$ for all $u \in N$. Then $f = 0$. Since
$(A + \lambda)$ is an isomorphism of $N$ onto $Q$, there exists $w
\in N$ such that $(A + \lambda)w = f$. Hence $((A + \lambda)w,
u)_Q = 0$ for all $u \in N$. But  
$$
((A +\lambda)w, u)_Q = (Aw, u)Q + \lambda (w, u)Q = a(w, u) +\lambda (w, u)_Q.
$$

Taking $u = w$ in particular, we get
$$
0 = \re  ~a (w, w) + \lambda |w|^2_Q \ge \alpha |w|^2 Q.
$$

Hence $w = 0$ and so $f = 0$.

Now we prove that the adjoint of $A$ is $A^*$. The domain of the
adjoint $\tilde{A}$  of $A$ consists of $ u \in Q$ such that
the mapping $v \rightarrow (Av, u)_Q$ is continuous on $N$ with the
topology induced by $Q$. Since $N$ is dense in $Q$, this mapping can
be extended to a linear form on $Q$ and hence by Riesz'z theorem we
have $\tilde{A} u \in Q$ such that 
$$
(Av, u)_Q = (v, \tilde{A}u)_Q \quad \text{ for } \quad v \in
D_A \quad \text{ and } \quad u \in D_{\tilde{A}}. 
$$

This defines $\tilde{A}$ on $D_{\tilde{A}}$.

Since
\begin{align*}
  (Av, u)_Q = a(v, u)& = \overline{a^*(u, v)} = \overline{(A^*u, v)_Q}\\ 
  & = (v, A^*u)_Q  \quad \text{for} \quad v \in N \quad
  \text{and} \quad u \in N^*,  \tag{1}\label{lec7:sec3:subsec5:eq1} 
\end{align*}
we have $N^* \in D_{\tilde{A}}$, and $\tilde{A} = A^*$ on
$N^*$. We need only prove now $D_{\tilde{A}} \subset N^*$. 

Let $u \in D_{\tilde{A}}$, then there exists $u_0 \in
N^*$ such that $(A^* + \lambda)u_o = (\tilde{A} + \lambda)u$, since
$A^* + \lambda$ is an isomorphism of $N^*$ onto $Q$ on account of
V-ellipticity of $a^*(u, v)$. Now, for all $v \in N$ 
\begin{align*}
((A + \lambda)v, u)_Q & = (v, (\tilde{A} + \lambda)u)_Q = (v, (A* +
  \lambda)u_0)_Q\\ 
& = a(v, u_o) + \lambda(v, u_o)_Q (by (1))\\
& = (Av, u_o) + \lambda(v, u_o)_Q \text{ since $v \in N$ and
    by definition of $A$.} 
\end{align*}

Hence for all $v \in N, ((A + \lambda)v, u-u_o)_Q = 0$. Since
$(A + \lambda)$ is an isomorphism of $N$ onto $Q'$, this means $u-u_0
= 0$, i.e., $u \in N^*$, which completes the proof. 
