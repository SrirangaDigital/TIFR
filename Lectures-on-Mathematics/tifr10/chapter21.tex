\chapter{Lecture}\label{lec21}%% 21

\setcounter{section}{11}

\subsection{Aronszajn-Smith Problems}\label{lec21:sec11:subsec2} 

We\pageoriginale prove a lemma which will be required often. 
\begin{lemma}\label{lec21:sec11:subsec2:lem11.1} %lemma 11. 1
  Let $\Omega = ]0, 1 L^n$ and $\Gamma = \{ \Omega \cap \{x_n =
    0\}\}$. Let $F = H^{m- \frac{1}{2}}(\Gamma) \times \ldots \times $ $
    H^{\frac{1}{2}} (\Gamma)$. Let $f_\alpha$ be a bounded set in $F$ such
    that all $f_\alpha$ have their support in a fixed compact set $\sum$
    in $\Gamma$. Then there exists a bounded set $v_\alpha$ in $H^m
    (\Omega)$ such that $\overset{\to}{\gamma} v_\alpha = f_\alpha ,
    \overset{\to}{\gamma} $ being as defined in 11.1 above, and
    $v_\alpha \equiv 0$ near $\partial \Omega - \sum$.  
\end{lemma}

\begin{proof}%proof
  By Theorem \ref{lec20:sec11:subsec1:thm11.1}, $\overset{\to}{\gamma} : H^m (\Omega) \to F$ is onto
  with kernel $H^m_\circ$. Hence $\overset{\to}{\gamma}$ induces an
  isomorphism $\overset{\to}{\gamma}_1$ of $H^m / H^m_o$ onto $F$. Since
  $f_\alpha$ is a bounded set in $F$, $\overset{\to}{\gamma}^{-1} (f_\alpha)$ is
  bounded in $H^m / H^m_\circ$. Therefore we can choose a bounded set
  $\omega_\alpha \in H^m$ such that $\gamma \omega_\alpha =
  f$. Next let $\varphi \in \mathscr{D}(\bar{\Omega})$ be zero
  near $\partial \Omega - \sum $ and $\varphi = 1$ on $\sum$. Then
  $v_\alpha = \varphi v_\alpha$ are bounded, vanish near $\partial
  \Omega - \sum$ and $\overset{\to}{\gamma}v_\alpha =
  \overset{\to}{\gamma} (\varphi) \overset{\to}{\gamma} (\omega_\alpha)
  = \overset{\to}{f}_{\alpha} $.  
\end{proof}

In \S\ \ref{lec15:sec9}, we considered regularity at the boundary of some
problems related to the operator $A$ defined by a sesquilinear form $a
(u, v)$ on $V$ such that $H^m_o \subset V \subset H^m$. Now we take up
a particular example of a different space $V$. In this case the
technique used in \S\ \ref{lec15:sec9} is not at once applicable. Since the
preliminary step of using local maps is at any rate permissible we
assume $\Omega = ] 0, 1 [^n$. Further to avoid technical details, we
  assume that $m = 2$. Let $\vartheta$ be the subspace of $H^2
  (\Omega)$ consisting of functions 
\begin{enumerate}[a)]
\item vanish near $\partial \Omega - \sum$, 
\item $Bu = 0$ on where
  $$
  Bu = \frac{\partial u}{\partial x_n} (x' , 0) + \sum_{i = 1}^{n}
  \alpha_i (x') \frac{\partial u}{\partial x_i} (x' , 0) + \alpha_o (x')
  u (x', 0) 
  $$
  with\pageoriginale $\alpha_o, \ldots , \alpha_{n-1} \in \mathscr{D} ( \xi )$. 
\end{enumerate}

Let $a(u, v) = \sum\limits_{|p|, |q| \leq} \int_\Omega a_{pq} (x) D^q
u \overline{D^p v} dx$ with $a_{pq} \in \mathscr{E}
(\bar{\Omega})$.  

Let $\re a(u, u) \geq \alpha || u ||^2_2$ for all $u \in
\vartheta$. In this case according to the theory of \S\ \ref{lec5:sec3}, as
transformed by local maps as in \S, for a given $f \in L^2
(\Omega)$, there exists $u \in N$ such that $a (u, v) = (f,
v)_o $ for all $v \in \vartheta$. We prove now 
\begin{theorem}\label{lec21:sec11:subsec2:thm11.2} % theorem 11. 2
  Let $u \in H^2 (\Omega)$ with $Bu = 0$ and $a(u, v) = (f,
  v)_o$ for all $v \in \vartheta, f \in L^2$. Then $u
  \in H^4 (\Omega^{\epsilon})$ for every $\in > 0$.  
\end{theorem}

\begin{proof} %proof
Though a shorter proof by induction is possible in order to bring out
the significance of the method we give a direct proof. Since after
having proved that $D^p_T u \in H^2$ for $| p | \leq 2$, to
prove $D^m _y u \in H^2$ no use of boundary conditions need be
made as in $\S 9$, to prove the theorem, we have to prove $D^P_\tau u
\in H^2$ for $| p | \leq 2$. Further if $\phi \in
\mathscr{D} \bar{\Omega}), \varphi \equiv 0$, near $\partial^- -
\sum$, to prove $D^p_\tau u \in H^2 $ for $|p | \leq 2$, it is
enough to prove $D^p_\tau (\phi u ) \in H^2 $ for $| p | \leq
2$. We break this in two steps.  
\end{proof}

\begin{step}\label{lec21:sec11:subsec2:step1} %step 1
  $D^1_\tau (\phi u) \in H^2$. 
\end{step}

As usual we need prove $(\phi u ^{-h}) $ is bounded in $H^2 (\Omega)$
by $c || u ||_2$, and for this we consider $a ((\phi u)^{-h}, v)$.  

\begin{lemma}\label{lec21:sec11:subsec2:lem11.2}% lemma 11. 2
  $\big| a (\phi u)^{-h}, v) \big | \leq c || v ||_2$. 
\end{lemma}

We write
$$
a (( \phi u)^{-h}, v) = [a((\phi u )^{-h}, v) + a (\phi u, v^h)] -
b(u, v^h) - a(u, \phi v^h) 
$$
where $b(u, v) = a (\phi u, v) - a(u, \phi v)$. 

As in \S\ \ref{lec15:sec9}, we can estimate $a((\phi u)^{-h}, v) + a(\phi u,
v^{-h})$ and $b(u, v)$ by almost the same methods. It remains to be
proved that\pageoriginale $\big | a (u, \phi v^h \big | \leq c || v ||_2$. We cannot
put $a(u, \phi v^h) = (f, \phi v^h)_o$ as $\vartheta$ is not
necessarily closed for translations. However by ``correcting" $\phi
v^h$ with a ``compensating" function $\omega^h$ we prove that $\big |
a (u, \phi v^h) \big | < c || v ||_2$. More precisely we prove the  
\begin{lemma}\label{lec21:sec11:subsec2:lem11.3} % lemma 11. 3
  There exists $w_h$ in $H^2 (\Omega)$ such that
  \begin{enumerate} [(a)]
  \item $\phi v^h - w_h \in \vartheta$
  \item $|| w_h ||_2 \leq c || v ||_2$. 
  \end{enumerate}
\end{lemma}

Assuming for a moment the lemma \ref{lec21:sec11:subsec2:lem11.3}, we prove lemma \ref{lec21:sec11:subsec2:lem11.2}. We have 
$$
a(u, \phi v^h) = a(u, \phi v^h - w_h) + a(u, w_h). 
$$

Since $\phi v^h - w_h \in \vartheta, a (u, \phi v^h - w_h) =
(f, \phi v^h - w_h)_o$.  

Hence $\Big | a(u, \phi v^h - w_h ) \Big | \leq | f |_o | \phi v^h -
w_h |_o \leq c( || v||_1 + |w_h|_o ) \leq c || v ||_2$.  

Further $\Big | a (u, w_h) \Big | \leq c || w_h ||_2 \leq c|| v||_2$,
whence the lemma \ref{lec21:sec11:subsec2:lem11.2}.  

Now we prove lemma \ref{lec21:sec11:subsec2:lem11.3}. We have to find $w_h$ such that
$$
\phi v^j - w_h \in \vartheta, \text { i. e. , } B (\phi v^h - w_h) = 0, 
$$

i.e., $\dfrac{\partial w_h}{\partial x_n} (x', 0) + \sum \alpha_i
(x') \dfrac{\partial w_h}{\partial x_i} (x' , 0) + \alpha_o (x') w_h
(x' , 0) = B(\phi v^h)$.

This holds if $\dfrac{\partial wh}{\partial x_n} (x' , 0) = B (\phi
v^h)$, and $w_h (x' , 0) = 0$.  

If we prove that $B (\phi v^h)$ is bounded in $H^{\frac{1}{2}}$ by $c'
|| v ||_2$, by using lemma \ref{lec21:sec11:subsec2:lem11.1}, we can find $w_h$ bounded by $c ||
v ||_2$, such that $\gamma w_h = 0$ and $\gamma_1 w_h = B(\phi v^h)$
which will prove the lemma. Now 
$$
B(\phi v^h) = g_h + k_h
$$
where \quad $g_h (x' , 0) = \phi (x' , 0) B v^h$, 

and \quad $k_h (x' , 0) = \dfrac{\partial \varphi}{\partial x_n} (x' ,
0) v^h (x' , 0) + \sum \dfrac{\partial \varphi}{\partial x_i} (x' , 0)
\alpha_i v^h (x' , 0)$.  

Since $\varphi$ has compact support all $k_h$ have support in a fixed compact. 

Further since $v \in H^2$, we have $v^h (x' , 0) \in
H^{3/2} (\Gamma \Gamma )$ and since $\dfrac{\partial \varphi}{\partial
  x_h}$ are\pageoriginale smooth, we have $k_h (x) \in H^{3/2}
(\Gamma)$. Now as $h \to 0, v^h (x' , 0) \to D_\tau v(x' , 0)$, hence
$k_h$ is bounded by $c || v ||_2$ in $H^{\frac{1}{2}} (\Gamma)$.  

It remains to see that $g_h$ is bounded in $H^{\frac{1}{2}} (\Gamma)$
by $c || v ||_2$. Since $Bv = 0$, $(Bv)^h = 0$, and since 
$$
(Bv)^h = Bv^h + \sum \alpha^h_i \frac{\partial v}{\partial x_i}(x' +
h, 0) + \alpha^h_o v (x' + h, 0) 
$$
we have $g_h = \varphi Bv^h = - \phi (x' , 0) (\sum \alpha^h_i
\dfrac{\partial v}{\partial x_i} (x' + h, 0) + \alpha^h_o v(x' + h,
0))$.  

As $h \to 0, \alpha^h_i$ are uniformly bounded and $\dfrac{\partial
  v}{\partial x_i} $ are bounded in $H^{\frac{1}{2}} (\Gamma)$ as
translations are continuous.  

This proves then that $B (\phi v^h)$ is bounded in
$H^{\frac{1}{2}}(\Gamma)$ and the proof of lemma \ref{lec21:sec11:subsec2:lem11.3} and hence
that of lemma \ref{lec21:sec11:subsec2:lem11.2} is complete.  

Now we are in a position to prove the
\begin{lemma}\label{lec21:sec11:subsec2:lem11.4} %lemma11. 4
$|| (\phi u)^{-h} ||_2 \leq c$. 
\end{lemma}

We cannot prove this as in \S\ \ref{lec15:sec9} by taking $v = (\phi u)^{-h}$ in
lemma \ref{lec21:sec11:subsec2:lem11.2}, and using ellipticity for $(\phi u)^{-h}$ does not
necessarily belong to $\vartheta$. We again correct this by 
\begin{lemma}\label{lec21:sec11:subsec2:lem11.5} %lemma 11. 5
There exists $w_h \in H^2 (\Omega)$ such that
$$
a) (\phi u)^{-h} - w_h \in \vartheta \text { and } b) || w_h
||_2 \leq c || u ||_2 = c' 
$$
(since $u$ is fixed). 
\end{lemma}

To prove this we note $(\phi u )^{-h} = \phi u^{-h} + \phi u^{-h}
(x-h)$ and from lemma \ref{lec21:sec11:subsec2:lem11.3}, there exists $w'_h$ such that $\phi
u^{-h} - w_h \in \vartheta$, and $|| w_h ||_2 \leq c || u
||_2$. We have only to look then for $w^{(2)}_h$ such that 
\begin{gather*}
\phi^{-h } u(x-h) - w^{(2)}_h \in \vartheta, \text { and }\\
|| w_h^{(2)} ||_2 \leq c || u ||_2. 
\end{gather*}

We\pageoriginale have to find $w^{(2)}_h$ bounded in $H^2 (\Omega)$ by $c || u ||_2$
and such that 
\begin{align*}
  w^{(2)}_h (x' , 0) & = \phi^{-h} u(x' - h, 0)\\
  \frac{\partial w_h}{\partial x_n}(x' , 0) & = \frac{\partial}{\partial
    x_n} (\phi^{-h} u(x' - h, x_n)_{x_n} = 0.  
\end{align*}

Hence, by lemma \ref{lec21:sec11:subsec2:lem11.1} such $w^{(2)}_h$ as required above exist and
the lemma \ref{lec21:sec11:subsec2:lem11.5} is proved.  

To prove lemma \ref{lec21:sec11:subsec2:lem11.4} consider now
\begin{align*}
  a((\phi u)^{-h} - w'_h, (\phi u)^{-h} - w' _h) & = a((\varphi u)^{-h},
  (\phi u)^{-h} - w'_h). a(w'_h, (\phi u)^{-h} - w'_h), \\ 
  & = X_h - Y_h. 
\end{align*}

By lemma \ref{lec21:sec11:subsec2:lem11.2}, we have
\begin{align*}
  & | X_h | \leq c || (\phi u)^{-h} - w'_h ||_2\\
  & | Y_h | \leq c || w'_h ||_2 || (\phi u)^{-h} - w'_h ||_2 \leq c' ||
  (\phi u^{-h} - w'_h ||^2_2.  
\end{align*}

On account of ellipticity, $|a((\phi u)^{-h} - w'_h, (\phi u )^{-h } -
w'_h)| \geq \alpha || (\phi u)^{-h} - w'_h ||^2_2$.  

Hence $|| (\phi u)^{-h} -w'_h ||_2 \leq c$ and since $|| w'_h ||_2
\leq c$ we get the lemma. This completes the first step of the proof,
viz. $D^p_\tau u \in H^2 (\Omega), |p| = 1$.  
