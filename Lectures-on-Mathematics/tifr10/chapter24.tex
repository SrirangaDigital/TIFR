
\chapter{Lecture}\label{lec24}%% 24

\setcounter{section}{12}
\subsection{Study at the boundary.}\label{lec24:sec12:subsec3}

\begin{definition}\label{lec24:sec12:subsec3:def12.6}%def 12.6
  We\pageoriginale say that $a(u, v)$ is regular at the boundary if $u \in N$
  is such that $a(u, v) = 0$ for every $v \in V$ vanishing
  outside a neighbourhood of some compact $K \subset \Omega$ then $u$ is
  $C^\infty$ in a neighbourhood of $\Gamma$. 
\end{definition}

If $V = H^m_o (\Omega)$ or $H^m (\Omega)$ and $a(u, v) = \sum a_{pq}
D^q \overline{D^p}v$ then the results on regularity at the boundary of
\S\ \ref{lec15:sec9} state that under the conditions specified in theorem \ref{lec15:sec9:subsec1:thm9.1}, $a(u, v)$ is regular at the boundary. 

\begin{theorem}\label{lec24:sec12:subsec3:thm12.6}%thoe 12.6
  Under the hypothesis of theorem \ref{lec23:sec12:subsec2:thm12.3}, if further $a(u, v)$ is regular
  at the boundary, then for fixed $y, G_x(y)$ is $C^\infty$ in a
  neighbourhood of $\Gamma$. 
\end{theorem}

This means in this case $G(x, y)$ for fixed $y$ is a usual function in
a neighbourhood of $\Gamma$ satisfying usual boundary conditions. 

By theorem \ref{lec23:sec12:subsec2:thm12.5}, $G_x (y) = G(\delta_x (y)) = S + u$ with $S
\in \mathscr{E'}$ and $u \in N$. Hence $Au + AS =
\delta_x (y)$, i.e., $Au = \delta_x (y) - AS = T$ for $T \in
\mathscr{E'}$. Let $K$ be the support of $T$, which on account of the
splitting proved in theorem \ref{lec23:sec12:subsec1:thm12.1} can be taken in any arbitrary
neighbourhood of $y$.  

Let $\in V$ such that $v = 0$ in neighbourhood of $K$. Now by
regularization we can find $\varphi_n$ vanishing on the support of $v$
such that $T = \lim \varphi_n $ in $\mathscr{E'}$. Then $\langle Au,
\bar{v} \rangle = \lim \langle \varphi_n, \bar{v} \rangle$. Hence $ a(u, v) = 0$ for all  $v \in V$    
vanishing in a neighbourhood of $K$. By regularity at the boundary of
$K, u$ is $C^\infty$ in a neighbourhood of $\Gamma$. This completes
the proof of the theorem. 

\section[Regularity at the Boundary Problems for...]{Regularity at the Boundary Problems for General Decompositions.}\label{lec24:sec13}%sec 13 

\subsection{}\label{lec24:sec13:subsec1} 

Hitherto\pageoriginale we considered boundary value problems for differential 
operators in the space $H^m (\Omega)$. For this we obtained $A$ as the
operator associated with a form $a (u, v)$ on $H^m(\Omega)$ is the
space of type $H(\{A\}, \Omega)$ where $\{A\}$ stands for the system
$D^{(p)}$. More generally we consider now what problems are solved by
considering $A$ as the operator associated with sesquilinear forms
$a(u, v)$ on spaces $H(\{A_i\}, \Omega)$. That this solves now
problems can be seen from the following example. Let $A = \Delta^2 +
1$. Consider on $H(\Delta, \Omega)$ the sesquilinear form $a(u, v) =
(\Delta u, \Delta v)_o + (u, v)_o$. The operator $A$ associated with
$a(u, v)$ is $\Delta^2 + 1. a(u, v)$ is $H(\Delta, \Omega)$ elliptic
and hence for $f \in Q'$ where $Q$ is such that $H (\Delta,
\Omega)$ is dense in $Q$, we have $u \in N$ such that $Au =
f$. 

Firstly we observe that $H^2 (\Omega)$ may be contained in $H(\Delta,
\Omega)$ strictly. For example, if $\Omega$ is a domain such that for
a given $T \in H^{-2}_{\bar{\Omega}}$, there exists
$\in H^o$ such that $- \Delta U- T \in M^o$, i.e., for
which Visik-Soboleff problem is soluble, then there exists $u
\in H(\Delta, \Omega)$ such that $u \notin H^1 (\Omega)$. For,
let $T \in H^{-2}_{\bar{\Omega}}$ be defined by $\langle T,
\bar{\varphi} \rangle = \int _r f (\gamma \bar{\varphi}) d \sigma $
for $f \in L^2 (\Gamma)$ and such that $f \notin
H^{\frac{1}{2}} (\Gamma) = \gamma (H^1 (\Omega))$. 

Now if $\cup$ is the corresponding solution, let $\underline{u}$ be
its restriction to $\Omega$. We have $u \in L^2 (\Omega)$ and
$- \Delta u = u $ by \S\ 10. Hence $u \in H(\Delta, \Omega):$
If $u$ were in $H^1 (\Omega)$, then $\gamma u = f$ would be in
$H^{\frac{1}{2}} (\Gamma)$ contrary to the assumption. Another more
elementary example can be given for a circle. It is easy to construct
examples such that $u \in L^2$ and $\Delta u = 0$, but $\gamma
u \notin H^1$. Thus Hadamard's classical example with $u = \sum a_n
r^n e^{in \theta}$ with suitable $a_n$ is of this type. 

However\pageoriginale it is true that $H^2_o (\Omega) = H(\Delta, \Omega)$. For, by
Plancherel's formula, the two norms are equivalent on
$\mathscr{D}(\Omega)$. This raises in fact the question : To determine
the conditions on $A_i$ and $\Omega$ so that $H(A; \Omega) = H^m
(\Omega)$ where $m=$ highest of orders of the operators $A_i$. 

Now we interpret formally the boundary value problems that are solved
on $V \in H(\Delta, \Omega)$. We write first of all Green's
formula 
\begin{equation}
  \int \Delta^2 u\cdot \bar{v} dx = \int_\Gamma \frac{\partial \Delta
    u}{\Delta n}. \bar{v} d \sigma - \int_\Gamma \Delta u
  . \frac{\partial \bar{v}}{\partial n} d \sigma + \int \Delta
  u. \overline{\Delta v} dx \tag{1}\label{lec24:sec13:subsec1:eq1} 
\end{equation}
\begin{enumerate}[a)]
\item Let $V = H_o (\Delta, \Omega)$. Since $H_2 (\Delta, \Omega) =
  H^2_o (\Omega)$ no new problem is solved. 
\item $V= H (\Delta, \Omega)$. Given $f \in Q'$ there exists
  $u \in N$ such that $a(u, v) = (f, v)_o$ for all $v
  \in V$. Further 
\begin{equation}
(\Delta^2 + 1)u = f \tag{2}\label{lec24:sec13:subsec1:eq2}
\end{equation}

Hence $\int_\Omega (\Delta^2 u - \bar{\Delta}v| dx + \int_\Omega u,
\bar{v} dx = \int (f- \bar{v})dx$ 

Using (\ref{lec24:sec13:subsec1:eq1}) and (\ref{lec24:sec13:subsec1:eq2}),
$$
\int_\Gamma \frac{\partial \Delta u}{\partial n}. \bar{v} d \sigma +
\rho \in  \Delta u. \frac{\partial v}{\partial n} d = 0 \text { for
  all } v \in V. 
$$

Formally this means $\Delta u = 0$, and $\dfrac{\partial \Delta
  u}{\partial n} = 0$. 
\item $V=$ Closure in $H(\Delta, \Omega)$ of continuous function with
  $u = 0$. Then $u \in N$ implies $u_\Gamma =0$ and $\Delta
  u_{\Gamma}=0$. 
\item $V=$ Closure in $H(\Delta, \Omega)$ of continuous function with
  $\dfrac{\partial u}{\partial n}_\Gamma = 0$ Then $u \in N$
  implies $\dfrac{\partial u}{\partial n}_\Gamma = 0$ and
  $\dfrac{\partial \Delta u}{\partial n}_\Gamma =0$. 
\end{enumerate}

However, problems in which $\dfrac{\partial u}{\partial n} = 0$ and
$\Delta u = 0$ or $u = 0$ and $\dfrac{\partial \Delta u}{\partial n} =
0$ are not solved by this method. 
