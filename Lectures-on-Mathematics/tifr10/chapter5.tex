
\chapter{Lecture}\label{lec5}%chap 5

\setcounter{section}{2}
\section{General Elliptic Boundary Value Problems}\label{lec5:sec3}%sec 3

\subsection{General theory}\label{lec5:sec3:subsec1}

We\pageoriginale formulate at the beginning certain problems on topological vector
spaces and solve them. Later on we shall show how these answers will
help us in solving many of the classical boundary value problems for
elliptic differential equations. 

As a matter of notation, we shall write $A \subset B$, where $A$ and
$B$ are two topological vector spaces to mean the injection $i : A
\rightarrow B$ is continuous or that the topology $A$ is finer than
the topology induced by $B$. 

Let $V$ be a Hilbert space over complex numbers. We shall denote by
$|u|_V$ the norm in $V$. Let $Q$ be a locally convex topological
vector space such that  
\begin{enumerate}[1)]
\item $V \subset Q$ and $V$ is dense in $Q$;
\item On $Q$ an involution (i.e., an anti-linear isomorphism of order
  two)  $f \rightarrow \bar{f}$ is given which leaves $V$ invariant; 
\item Let $V$ be given a continuous sesquilinear form $a (u,v)$(i.e.,
  $a (\lambda u, v) = \lambda a$ $(u,v)$, and $a(u, \lambda v) =
  \bar{\lambda} a (u, v)$. Let $Q'$ be the dual space of $Q$. On
  $Q'$ an involution is induced by the given one in $Q$ by
  the following formula $ < \bar{f}, g > = < f, \bar{g}>$. 

  We raise now the 
  \begin{problem}\label{lec5:sec3:subsec1:prob3.1}%prop 3.1
    Give $f \in Q'$ does there exist a $u \in V$ such that
  \end{problem}
\item a $(u, v)= < f , \bar{v}>$  for all $v \in V$.
  
  We shall show later that large classes of elliptic problems can be put
  in this form. 
  \begin{definition}\label{lec5:sec3:subsec1:def3.1}  %def 3.1
    The\pageoriginale space  $\underline{N}$ will consists of all $u \in V$ such
    that the mapping $v \rightarrow a (u, v)$ is continuous on $V$ with
    the {\em topology of $Q$}. 
    
    Since $V$ is dense in $Q$ we can extend this mapping to $Q$. Hence for
    every $u \in N$ we have an $Au \in Q'$ such that 
  \end{definition}
\item \quad $a(u, v) = <Au, \overline{v}>$.

  The mapping $A : N \rightarrow Q'$ is linear. On $N$ we introduce the
  upper bound topology to make the mapping $N \rightarrow V$ and $A : N
  \rightarrow Q'$ continuous. We ask now the 
\end{enumerate}

\begin{problem}\label{lec5:sec3:subsec1:prob3.2}%prob 3.2
  Is the mapping $A$ onto $Q'?$
\end{problem}

\begin{lemma}\label{lec5:sec3:subsec1:lem3.1}%lem 3.1
  Problem 1 is equivalent to problem 2.
\end{lemma}

\begin{proof}%proof
Let $f \in Q'$ and let $u$ be a solution of problem $1$, i.e.,
$a(u, v) = <f, \overline{v}>$. Hence the mapping $v \rightarrow a(u,
v) =   <f, \overline{v}>$ is continuous on $V$ with the topology of
$Q$. Hence $u \in N$. Further $<A u, \overline{v}> = a (u,v) =
<f, \overline{v}>$ for all $v \in V$, and since $V$ is dense
in $Q, Au = f$. Conversely, let $f \in Q'$ be given and $u
\in N$ be such that $Au = f$. Then $a(u, v) = <Au,
\overline{v}> =  <f, \overline{v}>$, for all $v \in V$, i.e.,
$u$ is a solution of problem 1. 
\end{proof}

\subsection{}\label{lec5:sec3:subsec2} 

We now consider certain sufficient handy
condition so that $A$ should be an isomorphism of $N$ onto $Q'$ 
\begin{definition}\label{lec5:sec3:subsec2:def3.2}%def 3.2
  We shall say that the sesquilinear $a(u, v)$ is elliptic on $V$, or is
  $V$-elliptic, if there exists an $\alpha > 0$ such that 
  $$
  \re  (a(u, u)) \ge \alpha | u|^2_V \text{ for all } u \in V.
  $$
\end{definition}

\begin{theorem}\label{lec5:sec3:subsec2:thm3.1}%thm 3.1
  Let $V, Q, a(u, v)$ be as given in \S\ \ref{lec5:sec3:subsec2:thm3.1}. If $a(u, v)$ is {\em  
  V-elliptic}, then $A$ is an {\em isomorphism} of $N$ onto $Q'$. 
\end{theorem}

\begin{proof}%proof
Let\pageoriginale
\begin{align*}a_1(u, v) & = \frac{1}{2}\big[  a(u, v) + i~
  \overline{a(v,u)}  \big ]\\ 
and \quad a_2(u, v) & = \frac{1}{2} i \big[  a(u, v) -
  \overline{a(v,u)}  \big ]. 
\end{align*}

Then $a_1(u, v) $ and $a_2(u, v) $ are hermitian and
\begin{align*}
a(u, v)& = a_1(u, v) + i a_2(u, v).\\
Put \quad [u, v] &= a_1(u, v).
\end{align*}

Since $|a(u, v)| \le C |u|_V |v|_V$, it follows that $[u,u]\le C
|u|^2_V$.  On account of the V-ellipticity, $[u, u]= \re  a (u, u)\ge
\alpha |u|^2_V$. Hence the form $[u, v]$ defines on $V$ an Hilbertian
structure equivalent to the one defined by $(u, v)_V$. 
\end{proof}

Now, any $f \in Q'$ defines a continuous semi-linear function
on $V$ and hence there exists $Kf$ such that 
$$
<f, \overline{v}> = [Kf, v], \quad K \in \mathscr{L} (Q', V).
$$

For a fixed $u \in V$, the mapping $v \rightarrow a_2(u, v)$
is a semi-linear continuous mapping on $V$, hence 
$$
a_2(u, v) = [Hu, v].
$$

 Further $H$ is hermitian for the scalar product defined by $[u,
   v]$. For $[Hu, v] = a_2 (u, v)= \overline{a_2 (v, u)} =
 [\overline{Hv, u}] = [u, Hv]$. 
 $$
 \displaylines{\text{Hence } \hfill a(u, v)  = [u, v] + i[Hu,
     v],\hfill \cr
   \text{and we have to solve} \qquad\quad  a(u, v)  = <f, \overline{v}> =
        [ Kf, v],\hfill \cr 
        \text{i.e.,} \hfill (1 + i H)u  = Kf.\hfill }
 $$
 
 From Hilbert space theory, we know that if $H$ is hermitian $(1 +
 iH)$ is non-singular. Hence 
 $$
 u = (1 + iH)^{-1}Kf,
 $$
which proves that $A$ is an isomorphism.
