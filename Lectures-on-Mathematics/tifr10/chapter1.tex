\chapter{Lecture}\label{lec1}%% 1

\section{Spaces \texorpdfstring{$H(A; \Omega)$}{H(AOmega)}}\label{lec1:sec1}%sec 1

\subsection{General notations}\label{lec1:sec1:subsec1} 
We\pageoriginale shall recall some standard definition and fix
some usual 
  notation. $R^n(x=(x_1, \ldots , x_m))$ will denote the
  $n$-dimensional Euclidean space, $\Omega$ an open set of $R^n,
  \mathscr{D}(\Omega)$ will be the space of all indefinitely
  differentiable functions (written sometimes $C^ \infty$ functions)
  with compact support in $\Omega$ with the usual topology of
  Schwartz. $\mathscr{D'}(\Omega)$ will be the space of distributions
  over $\Omega$. $L^2(\Omega)$ will be the space of all square
  integrable functions on $\Omega$. The norm of a function
  $\mathscr{F}\in L^2 (\Omega)$ will be denoted by $||
  \mathscr{F} ||_o$. \textit{Derivatives of functions on $\Omega$ will
  always be taken in sense of distributions}; more precisely $D^p$
  will stand for $\dfrac{\partial ^{|p|}}{\partial x^p_1, \ldots
  \partial x^{p_n}_{n}}$ where $p=(p_1, \ldots ,p_n)$ with
  $p_i$ non-negative integers and $|p|=p_1 + \cdots + p_n$ is the
  order of $D^p$. If $T \in \mathscr{D}' (\Omega), \langle
  D^pT, \varphi \rangle = (-1)^{|p|} \langle T,D^p \varphi
  \rangle$. 

\subsection{Spaces \texorpdfstring{${H(A;\Omega)}$}{H(AOmega)}}\label{lec1:sec1:subsec2} 

A defferential operator with constant coefficients $\underbar{A}$ is
an expression of the for $A= \sum\limits _{|p| \le m} a_p D^p$ where
$a_p$ are all constants. The highest integer $m$ for which there
exists an $a_p \neq 0$ for $|p|=m$ will be called the order of the
operator $A$. 
   \begin{definition}\label{lec1:sec1:subsec2:def1.1} %defi 1.1
     Let $A= \{ A_1 , \ldots , A_\nu \}$ be a system of differential
     operator with constants coefficients. By $H(A; \Omega)$ we shall
     denote the space of $u \in L^2(\Omega)$ for which $A_i u
     \in L^2 (\Omega)$. 
   \end{definition}

Evidently $\mathscr{D}(\Omega)\subset H(A; \Omega)$. On $H(A;\Omega)$
we define a sesquilinear form by 
   \begin{equation}
     (u,v)_{H(A; \Omega)}= (u,v)_o + \sum ^\gamma _{i=1}(A_i u,A_i v)_o. \tag{1} 
     \label{lec1:sec1:subsec2:eq1}     
   \end{equation}

\begin{theorem}\label{lec1:sec1:subsec2:thm1.1} %the 1.1
  With\pageoriginale the norm defined by (\ref{lec1:sec1:subsec2:eq1}), $H(A ; \Omega)$ is a Hilbert space.
\end{theorem}

\begin{proof}
  It is evident from the expression (\ref{lec1:sec1:subsec2:eq1}) that $(u,v)= \overline{(v, u)}   
  (u, u)\geq 0$, and that $(u,u) = 0$ if and only $u = 0$. So it remains
  to verify that under the topology defined by the norm, $H(A;\Omega)$
  is complete. If $\{ u_n \}$ is any Cauchy sequences in $H(A; \Omega)$,
  from (1) it follows that $\{ u_n \}$ and $\{A_i u_n \}$ are Cauchy
  sequences in $L^2(\Omega)$. Hence $\{ u_n \}$ and $\{A_i u_n \}$
  converge to $u$ and $v_i$ respectively, say, in $L^2(\Omega)$. Since
  the convergence in $L^2(\Omega)$ implies the convergence in
  $\mathscr{D}'(\Omega),\{ u_n \}$ and $\{A_i u_n \}$ converge to $u$
  and $v_i$ in $\mathscr{D}' (\Omega)$ respectively. Since $A_i$ are
  continuous on $\mathscr{D}'(\Omega), A_i (u_u) \to A_i(u)$ in
  $\mathscr{D}'(\Omega)$. Hence $A_i(u)=v_i$ which proves that $u
  \in H(A; \Omega)$. 
\end{proof}

\begin{proposition}\label{lec1:sec1:subsec2:prop1.1}%prop 1.1
If $W \subset \Omega$ and for $u \in H(A; \Omega),u_w$ denotes
the restriction of $u$ to $W$, then $(a)u_w \in H(A; W)$, and
$(b)$ the mapping $u \to u_w$ is continuous mapping of $H(A; \Omega)
\to H (A; W)$. 
\end{proposition}

\subsection{The space \texorpdfstring{$H_0(A; \Omega)$}{H0(AOmega)}}\label{lec1:sec1:subsec3}

\begin{definition}\label{lec1:sec1:subsec3:def1.2}%defi 1.2
$H_0(A ; \Omega)$ will be the closure of $\mathscr{D}(\Omega)$ in
  $H(A;\Omega).H(A;\Omega)$ will be the ``orthogonal complement'' of $H_0(A;
  \Omega)$ in $H_0(A; \Omega)$. 
\end{definition}

The following question then naturally arises:
\begin{problem}\label{lec1:sec1:subsec3:prob1.1}%pro 1.1
  When is $H_0(A; \Omega)=H(A; \Omega)?$
\end{problem}

If A differential operator, let $A^*$ denote the differential operator
defined by $\langle A^* T, \varphi \rangle = \langle T, \overline {A
  \varphi} \rangle$. If $A = \sum a_p D^p$, then it is easily verified
that $A^*=\sum (-1)^p \bar{a}_p D^p$. 

\begin{proposition}\label{lec1:sec1:subsec3:prop1.2}%prop 1.2
  $H_0^\perp(A; \Omega)$\pageoriginale is the space of solution in $H(A; \Omega)$ of $(1+ \sum \limits ^ \gamma _{i=1} A^*_i A_i)$ $T=0$. 
\end{proposition}

\begin{proof}
  $T \in H_0^1(A; \Omega)$ if and only $T$ is orthogonal to
  every $\varphi \in \mathscr{D}(\Omega)$, i.e., if and only
  if 
\end{proof}
$$
(T, \varphi)_o + \sum ^\gamma _ {i=1}(A_i T,A_i \varphi)=0
$$
for all $\varphi \in \mathscr{D}(\Omega)$, which is equivalent
to say that  
$$
\langle T + \sum A^*_i A_i T, \varphi \rangle =0 \text{ for all }
\varphi \in \mathscr{D}(\Omega), 
$$
or that
$$
(1 + \sum A^*_i A_i)T=0.
$$

\heading{Some examples.}

   \begin{enumerate}[1)]
   \item If there is no differential operators, $H(A ; \Omega)=L^2
     (\Omega)=H_o (\Lambda ; \Omega)$. 
   \item Let $\Omega =]0,1[,x=x_1, A= \dfrac {d}{dx}; H(A ; \Omega)= \{
    u/u \in L^2$ and only $u' \in L^2$. Then $T \in
    H ^\perp _o ( A ; \Omega)$ if and if $T-T''=0$, i.e.,$T= \lambda
    e^x+ \mu e^{-x}$. Hence $H^\perp _o (A ; \Omega)$ is space of
    dimension $2$. 
  \item Let $\Omega =]0,+ \infty [ ,x = x_1, A =
    \dfrac{d}{dx}.T=\lambda e^x + \mu e^{-x}$ is in $L^2 (\Omega)$ is
    $\lambda =0$. Hence $H^ \perp _o (A ; \Omega)$ is of dimension $1$. 
  \item Let $\Omega =] 0, + \infty [ ,x=x_1, A = \dfrac{d}{dx}H^ \perp
    _o (A ; \Omega)= \{ \underbar {0} \}$, i.e., 
   \end{enumerate}
$$
H_o (A ; \Omega)=H (A ; \Omega).
$$

In general it can be proved that if $\Omega =]0, 1[ A=
       \dfrac{d^m}{dx^m}, H^\perp_o (A, \Omega)$ is
       $2$m-dimensional. 

\subsection{}\label{lec1:sec1:subsec4} 

We recall some properties of Fourier transformations of
  distributions. Let $\mathscr{S}$ be the space of fastly decreasing
  functions in $R^n, \mathscr{S}'$ be the dual of $\mathscr{S}$
  consisting of tempered distributions. For $T \in \mathscr
  {S}'$ we\pageoriginale shall denote the Fourier Transform of $T$ by $\underline
  {\mathscr{F} T=\hat {T}}$. We know that $L^2 (R^n) \subset \mathscr
  {S}'$ and that $|\hat {T}|_o=|T|_o$ if $T \in L^2 (R^n)$
  (Plancherel's formula). Also $\mathscr {F}(D^p T)= (2 \pi i
  \xi)^p \hat {T}$, where $\xi =(\xi_1 , \ldots , \xi _n)$ and
  $\xi ^p =\xi^{p_1}_n \ldots \xi_n^{p_n}$. Since $\mathscr {F}$ is linear, it follows
  that if $A= \sum\limits_{|p|\le m} D^p$ is any differential operator
  with consists coefficients 
\begin{align*}
\mathscr {F} (A T)&= A(2 \pi i \xi) \hat{T} \text{ where }\\
  A_j(2 \pi i \xi)&= \sum _{|p|\le m} a_p \xi^{p_1}_1 \cdots \cdots
\xi^{p_n}_n (2 \pi i)^{|p|}\\ 
& = \sum a_p (2 \pi i \xi)^p.
\end{align*}

\begin{proposition}\label{lec1:sec1:subsec4:prop1.3}%prop 1.3
  $u \in H (A,R^n)$ if and only if $\hat{u} \in L^2
  (R^n)$ and $A_j (2 \pi i \xi ) \hat{u} \in L^2
  (R^n), j=1, \ldots ,N$. 
\end{proposition}

This is immediate as $u \in L^2 \Leftrightarrow \hat{u}
\in L^2(R^n)$ and $A_j u \in L^2 \Leftrightarrow A_j
(2 \pi i \xi) \hat{u} \in L^2 (R^n)$. 

\begin{proposition}\label{lec1:sec1:subsec4:prop1.4}%prop 1.4
  $H_o (A, R^n)=H(A,R^n)$, for any $A= \{ A_1 ,\ldots ,A_\nu \}$ with
  constant coefficients. 
\end{proposition}

From Proposition \ref{lec1:sec1:subsec3:prop1.2} we have $T \in H^ \perp _o (A, R^n)$ if
and only if $T \in L^2, A_j T \in L^2$ and $(1+ \sum
\limits_{j=1}^\nu A^*_j A_j)T=0$. By taking Fourier transforms, it follows
that $T \in H^ \perp _o (A,R^n)$ if and only if $\hat{T}
\in L^2 A_j \hat{T}\in L^2$ and $(1+\sum | A_j (2 \pi
i \xi)|^2) \hat{T}=0$. 


But since $(1 + \sum_j | A_j (2 \pi i \xi)|^2) \neq 0, \hat{T} = 0$
a.e., and hence $T=0$ which proves the proposition. 

\subsection{Extension of functions in \texorpdfstring{$H_0(A; \Omega)$}{H0(A;Omega)} to \texorpdfstring{$R^n$}{Rn}}\label{lec1:sec1:subsec5} 

\begin{theorem}\label{lec1:sec1:subsec5:thm1.2}%the 1.2
  There exists one and only one continuous linear mapping $u \to
  \tilde{u}$ of $H_0(A; \Omega)$ into $H(A ,R^n)$ such that if $u
  \in \mathscr{D} (\Omega), \tilde{u}=u$. a.e. in $\Omega$. 
\end{theorem}

For\pageoriginale $\varphi \in \mathscr{D}(\Omega)$, define
$\tilde{\varphi}= \begin{cases} \varphi(x)  & \text{if}~ x \in
  \Omega \\  0 & \text{if}~ x \not\in \Omega.\end{cases}$ 


Then $\tilde{\varphi} \in \mathscr{D}(R^n)$ and $|\tilde{
  \varphi}|_{H(A ,R^n)}= |\varphi|_{H(A, \Omega)}$. Hence $\varphi \to
\tilde{\varphi}$ is a continuous mapping of $\mathscr{D} (\Omega)$
with the topology induced by $H(A; \Omega)$ into $H(A,R^n)$. This
proves the theorem 
\begin{definition}\label{lec1:sec1:subsec5:def1.3}%defio 1.3
  If $u \in H_o(A; \Omega), \tilde{u}$ is called the extension of $u$ to $R^n$.
\end{definition}

\begin{remark*}
  If $u \in H (A; \Omega)$ and we put $\tilde{u}(x)=$
  $
  \begin{cases}
    u(x), & x \in \Omega \\ 0, & x \notin \Omega
  \end{cases}$
  it is not true that $\tilde{u}\in H(A,R^n)$. What that theorem
\ref{lec1:sec1:subsec5:thm1.2} says is that if $u \in H_o(A; \Omega)$, then $\tilde{u}
\in H_o (A,R^n)$. Thus if $A \dfrac{d}{dx}, \Omega =]0, 1[$
       then for $u=1, \dfrac{d}{dx} \tilde{u}$ is not in $L^2(R)$. 
\end{remark*}
