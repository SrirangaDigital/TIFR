
\chapter{Lecture}\label{lec15} %%% 15

\setcounter{section}{8}

\section{Regularity at the boundary}\label{lec15:sec9} %%%

In\pageoriginale the last lecture we dealt with the regularity in the interior or
local regularity of the solutions of the elliptic differential
equations. Now we wish to consider the regularity of the solutions in
$\bar{\Omega}$. In a sense such solutions can be extended to the
boundary. These should not be confused with problems in which boundary
values to be attained are given. These will be considered in a general
set up under the name of Visik-Sobolev problems. 

\subsection{}\label{lec15:sec9:subsec1}  %%% 9.1

\begin{theorem}\label{lec15:sec9:subsec1:thm9.1} %theorem 9.1
Let $\Omega$ be a bounded open set in $R^n$ with a boundary which is
an $n-1$ dimensional $C^\infty$ manifold. Let 
$$
a(u,v) = \sum_{|p|,|q|\le m} ~\int  a_{pq} ~ D^q ~ u ~ \overline{D^p ~ v} ~ dx
$$
with $a_{pq} ~ \in ~ \mathscr{E} ~ (\bar{\Omega})$ be given
such that $\re  ~ (a(u,u)) \ge ~ \alpha ~ || u ||^2_m$ for some
$\alpha ~ > ~ 0$ and for all $u ~ \in ~ H^m (\Omega)$. Let $V
= H^m (\Omega)$ and $Q = L^2 (\Omega)$ and let $A$ and $N$ be as
determined in theorem $3.1$. If $f ~ \in ~ L^2(\Omega)$ and $u
~ \in ~ N$ is such that $Au = f$, then $u ~ \in ~
H^{2m}(\Omega)$.  
\end{theorem}

\begin{remark*}% remark 
  If we do not take any condition on the boundary (eg., $u ~ \in
  ~ N$) then we can assert only that $u ~ \in ~
  \mathscr{L}^{2m}(\Omega)$ and cannot assert in general that $u ~
  \in ~ H^{2m}(\Omega)$. 
\end{remark*}

The proof of this theorem is fairly complicated and will be broken in
several steps.

\setcounter{step}{0} 
\begin{step}\label{lec15:sec9:subsec1:step1} % step 1
  First we reduce the problem to one in a cube in the following way: Let
  $O_i$ be a finite covering by relatively compact open sets of the
  boundary $\Gamma$ such that there exists $C^\infty$ homeomorphisms $\psi_i$ of
  $O_i$ to\pageoriginale $ \bar{W} = 
  \begin{cases} 
    0 < \epsilon_i < 1,\\
    -1 < \epsilon_n < n
  \end{cases}$  $i = 1, \ldots n-1$ such that $\psi_i$ maps $0_i ~
  \cap ~ \Omega$ onto $W_+ = 
  \begin{cases} 
    0 < \epsilon_i < 1,\\ 
    0 < \epsilon_n < 1 
  \end{cases}$  $i= 1,\ldots, n-1$ and $\Gamma  \cap  O_i$ onto
  $W_0 = \big\{ W \cap \{ \xi_n = O\}\big\}$. Since the regularity is
  the interior of $u ~ \in ~ H^m(\Omega)$ has been already
  proved to prove that $u ~ \in ~ H^{2m} (\Omega)$, it remains
  only to prove the restrictions of $u$ to $O_i$, i.e., $u_{O_i} ~
  \in ~ H^{2m}(O_i)$. The homeomorphisms $\psi_i$ define
  isomorphisms of $H^m(O_i ~ \cap ~ \Omega)$ onto $H^m(W_+)$. Let
  $u_1,v_1 ~ \in ~ H^m(W_+)$. Define $a_0(u,v) = a
  (\psi^{-1}(u_1), \psi^{-1}(v_1))$. (We drop $i$ from the
  suffix). This definition is possible as $A$ is an operator of local
  type, more precisely 
  $$
  a(\psi^{-1}(u_1)),(\psi^{-1}(v_1)) = \int\limits_{O}  a_{pq}(x) ~
  D^q(\psi^{-1}(u_1) \overline{D^p(\psi^{-1}(v_1)}dx. 
  $$
  $a_0(u,v)$ is a continuous sesquilinear form on $H^m(W_+)$. Now by
  theorem \ref{lec5:sec3:subsec2:thm3.1}, $a(u,v) = (f,v)_o$ for all $v ~ \in ~
  H^m(\Omega)$. Let, in particular, $v$ vanish near the boundary of $O -
  \Gamma ~ \cap ~ O$, and have its support in $O$. Then $a(u,v) -
  a_o(u,v) = (f,v)_0$. Hence if $v_1$ is in $H^m(W_+)$, and vanishes,
  near the boundary of $W_+ -  \Gamma$, then  
  $$
  a(\psi(u),v) = (\psi(f),v)_0, \text{ when } \psi(f) ~ \in ~ L^2 (W_+).
  $$
\end{step}

If we prove now that $\psi(u)~ \in ~ H^{2m}(W^\epsilon)$
for every $\in > 0$, where $W^\epsilon = \begin{cases} 1-
  \in  < \epsilon < \in \\ 0 <  \epsilon_n  <
  1 \end{cases}$, then by an obvious shrinking argument, we will have
proved the theorem. 

\subsection{}\label{lec15:sec9:subsec2} %% 9.2

\begin{step}\label{lec15:sec9:subsec2:step2}% step 2
  Thus our problem is reduced to the following one. Let ~ $\Omega =
  \big\{  0  <  x_i  <  1\big\}$, $i = 1, \ldots, n$, be $n$-dimensional
  cube in $R^n$. Let $a(u,v) = \sum \int\limits_{m} a_{pq} (x)  D^q u$ 
  $\overline{D^p u} dx$ with $a_{pq} ~ \in ~ \mathscr{E}
  (\bar{\Omega})$ be an elliptic form on $H^m(W)$. Let $f ~
  \in ~ L^2 (\Omega)$ and $u ~ \in ~ H^m(\omega)$ be
  such\pageoriginale that for every $v ~ \in ~ H^m(\Omega)$ which is zero near
  $\partial ~ \Omega - \sum$, we have  
  \begin{equation}
    a(u,v) = (f,v)_0.\tag{1}\label{lec15:sec9:subsec2:eq1}
  \end{equation}
\end{step}

Then we have to prove that $u ~ \in ~
H^{2m}(\Omega^\epsilon)$ for every $\in > 0$, where  
\begin{equation*}
\Omega^\epsilon = \left\{
   \begin{aligned}
     \in ~ & < ~ x_i ~ < ~ 1 - \in , ~ i = 1, \ldots , n-1\\
     0 ~ & < ~ x_n ~ < ~ 1
   \end{aligned}
\right \}.
\end{equation*}

We shall prove this in two steps.  First we consider the derivatives
of $u$ in the direction parallel to $x_n$ axis, which we call
tangential derivatives and denote them by $D^{p}_\tau(u)$ with $p =
(p_1,\ldots,p_{n-1},0)$. By an induction argument and considering
difference quotients as in the previous lecture, we shall prove that
$\underset{|p|=m}{D^p} ~ u ~ \in ~ H^m(\Omega)$. In the next
section we shall consider $D^m_{x_n} ~ u$. 

\begin{proposition}\label{lec15:sec9:subsec2:prop9.1}% proposition 9.1
  Under the hypothesis of the reduced problem $\underset{|p|=m}{D_\tau^p} u  \in\break 
  H^m(\Omega^\epsilon)$. 
\end{proposition}

\begin{proof}% proof
  If $u ~ \in ~ H^m(\Omega)$ is such that $v = 0$ near
  $\partial \Omega - \sum$, then we denote by $v^h(x) =
  \dfrac{1}{h}[v(x+h)- v(x)]$ which is defined for sufficiently small
  $h$, where $h = (h,0,\ldots,0)$. We note two simple identities
  relating to $v^h$. 
  \begin{enumerate}
  \item $\int_\Omega ~ u^h ~ v ~ dx + \int_\Omega ~ u v^h ~ dx = 0$
    where $u$ and $v$ both vanish near $\partial \Omega - \sum$. 
  \item $(au)^{-h} = a ~ u^{-h} + a^{-h} ~ u(x-h)$.
  \end{enumerate}

  Let $\phi$ be a function in $\mathscr{D}(\bar{\Omega})$ vanishing near
  $\partial\Omega - \sum$. $u$ is in $H^m(\Omega)$ and vanishes near the
  boundary. Using Leibnitz's formula, it is seen at once that to prove
  $u  \in  H^m(\Omega)$, it is enough to show that $D_\tau (\phi  u) 
  \in  H^m (\Omega)$. We shall prove first that $(\phi ~
  u)^{-h}$ is bounded. 
\end{proof}

Let\pageoriginale
\begin{align*}
  b(u,v) &  = a(\phi ~ u,v) - a(v,\phi u)\\
  & = \sum_{|p| \le  m, |q|  \le  m,  |p|+|q|  \le 2m-1} \int
  b_{pq} (x) ~ d^q u ~ \overline{D^p v} ~ dx \tag{2}\label{lec15:sec9:subsec2:eq2} 
\end{align*}
where $b_{pq}(x)$ are products of derivatives of $\phi$ with
$a_{pq}'$s, and so vanish near $\partial \Omega - \sum$ and are in
$\mathscr{D}(\bar{\Omega})$. Using (\ref{lec15:sec9:subsec2:eq1}) and (\ref{lec15:sec9:subsec2:eq2}), we have  
\begin{equation}
  a(\phi ~ u)^{-h},v) = [a((\phi ~ u)^{-h}, v) + a(\phi u , v^h)] -
  b(u,v^h) - (f,\phi v^h)_o. \tag{3}\label{lec15:sec9:subsec2:eq3} 
\end{equation} 

Now we prove three lemmas.
\begin{lemma}\label{lec15:sec9:subsec2:lem9.1}% lemma 9.1
  $| ~ a((\phi ~ u)^{-h}, v) + a(\phi ~ u,v^h) | ~ \le ~ c_1 ~ || u ||_m$.
\end{lemma}

\begin{lemma}\label{lec15:sec9:subsec2:lem9.2} % lemma 9.2
  $| ~ b(u,v^h) ~ | ~ \le ~ c_2 ~ || v ||_m$.
\end{lemma}

\begin{lemma}\label{lec15:sec9:subsec2:lem9.3}% lemma 9.3
  $| ~ (f, \phi ~ v^h)_0 ~ |\leq ~ c_3 ~ || v ||_m$.
\end{lemma} 

 Using these in (\ref{lec15:sec9:subsec2:eq3}) with  $v = ( \phi ~ u)^{-h}$ and using
 ellipticity condition, we have 
 $$
 \alpha ~ || ~ (\phi ~ u)^{-h} ~ ||^2_m ~ \le c_4 ~ || ~ (\phi ~ u)^{-h} ~ ||_m.
 $$
 
Hence $(\phi ~ u)^{-h}$  is bounded in $H^m$. Since bounded sets in
$H^m$ are weakly compact, there exists a sequence $h_i$ such that
$(\phi ~ u)^{-h_i}$ converges weakly to a function $g ~ \in ~
H^m$. However since $(\phi ~ u)^{-h_i} \rightarrow \dfrac{\partial ~
  u}{ \partial ~ x_i}$ in $\mathscr{D}$, we have $\dfrac{\partial ~
  u}{\partial ~ x_1} ~ \in  ~ H^m$. This proves the
proposition \ref{lec15:sec9:subsec2:prop9.1}, that $D_T ~ u ~ \in ~ H^m$. It remains to
prove the above lemma \ref{lec15:sec9:subsec2:lem9.1}, \ref{lec15:sec9:subsec2:lem9.2} 
and \ref{lec15:sec9:subsec2:lem9.3}.   

\begin{description}
  \item [Proof of Lemma 9.1.]% lemma 9.1
  $a((\phi ~ u)^{-h}, v)) + a(\phi u , v^h)$ consists of sums of terms like
  \begin{align*}
    X & = \int\limits_{|p|=m,|q|=m}a(x) ~ D^q(\phi ~ u)^{-h} ~
    \overline{D^p v} ~ dx + \int\limits_{|p| = m, |q| = m} a(x)D^q(\phi ~
    u) ~ \overline{D^p v^h} ~  dx\\ 
    & = \int a(x) ~ D^q(\phi ~ u)^{-h} ~ \overline{D^p v} ~ dx -
    \int((a(x)D^q (\phi ~ u)) ~ \overline{D^p ~ v} ~ dx\\  
    & = \int ~ a(x)D^q(\phi ~ u)^{-h} ~\overline{D^p v} ~ dx -
    \int[a(x)D^q(\phi ~ u)^{-h}\\ 
      & \hspace{4cm}+ a^{-h} (x) D^q (\phi ~ u)
      (x-h)]~\overline{D^p v} ~ dx\\ 
    & = - \int a^{-h} ~ D^q(\phi ~ u) (x-h) D^p v ~ dx.
  \end{align*}
  
  Since\pageoriginale $a^{-h}$ are bounded and translations are continuous in $H^m$
  and $|q| \leq m$, we have, by using Schwartz's lemma.  
  $$
  |X| \leq c|D^p v|_o \leq c_1||v||_m. 
  $$
\item [Proof of Lemma 9.2.]%lemma 9.2
  By definition, $b(u, v^h) = {\underset{ |p| \leq m, |q| \leq m,
      |p|+|q| \leq 2m-1}{\sum \int b_{pq}(x) D^q u \overline{D^p v^n}
      dx}}$. If $|p| \leq m-1$, then as $v \in H^m,
  \overline{D^p v^h}$ is bounded in $L^2$. If $|p|=m$ we have $|q| \leq
  m-1$, and $\int b_{pq}(x) D^q u \overline{D^p v^h} dx = - \int (b_{pq}
    D^q u)^{-h} \overline{D^p v}dx$, and since $b_{pq} \in
  \mathscr{D}(\bar{\Omega})$ and $u \in H^m(\Omega)$, we have
  $(b_{pq} D^qu)^{-h}$ bounded in $L^2$; so that at any rate $|b(u,
  v^h)| \leq c||v||_m$.  

\item [Proof of Lemma 9.3.]%lemma 9.3
  This follows easily, for as $h \to 0, v^h \to D_{\tau} v $ in $L^2$
  and hence $(f, \phi v^h)_o \leq c||D_{\tau} v||_{L^2} \leq c||v||_m$.  
\end{description}
