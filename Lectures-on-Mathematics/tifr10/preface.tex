\thispagestyle{empty}

\begin{center}
{\Large\bf Lectures on Elliptic}\\[5pt]
{\Large\bf Partial Differential Equations}
\vskip 1cm

{\bf  By}
\medskip

{\large\bf  J.L. Lions}
\vfill

{\bf  Tata Institute of Fundamental Research, Bombay}

{\bf  1957}
\end{center}

\eject

\thispagestyle{empty}
\begin{center}
{\Large\bf Lectures on} \\[5pt]
{\Large\bf Elliptic Partial Differential Equations}
\vskip 1cm

{\bf  By}
\medskip

{\large\bf  J. L. Lions}
\vfill

{\bf  Notes by}
\medskip

{\large\bf  B. V. Singbal}
\vfill

{\bf  Tata Institute of Fundamental Research,}

{\bf  Bombay}

{\bf  1957}
\end{center}

\chapter*{Introduction}

\addcontentsline{toc}{chapter}{Introduction}

In these lectures we study the boundary value problems associated with
elliptic equation by using essentially $L^2$ \textit{estimates} (or
abstract analogues of such estimates. We consider only linear problem,
and we do not study the Schauder estimates. 

We give first a general theory of ``weak'' boundary value problems for
elliptic operators. (We do not study the \textit{non-continuous}
sesquilinear forms; of. Visik \cite{k17}, Lions [\ref{k10:e7}], Visik-Ladyzeuskaya
\cite{k19}). 

We study then the \textit{problems of regularity}-firstly regularity
in the interior, and secondly the more difficult question of
regularity at the boundary. We use the Nirenberg method for Dirichlet
and Neumann problems and for more general cases we use an additional
idea of Aronsazajn-Smith. 

These results are applied to the study of new boundary problems:
\textit{the problems of Visik-Soboleff}. These problems are related
and generalize the problems of the Italian School (cf. Magenes
\cite{k11}). 
 
We conclude with the study of the Green's kernels, some indications on
unsolved problems and short study of systems. Due to lack of time we
have not studied the work of Schechter \cite{k15} nor the work of
Morrey-Niren-berg \cite{k13} which rots essentially on $L^2$ estimates. 
