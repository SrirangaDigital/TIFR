\chapter{Non-Analytic Modular Forms}\label{chap4}

\section{The Invariant Differential Equations}\label{chap4:sec1}\pageoriginale

The theory of non-analytic modular forms has a close connection with
Siegel's researches in the theory of indefinite quadratic forms and is
decisively influenced by this theory. The Eisenstein series 
$$
G(\tau, \bar{\tau}); \alpha, \beta) = \sum_{(c,d) \neq (0,0)}
(c\tau+d)^{-\alpha} (c\bar{\tau}+d)^{-\beta}, (\re(\alpha+\beta)>2)
$$
where $\alpha-\beta$ is an even integer and the sum runs over all
pairs of integers $(c,d)\neq (0,0)$ is the prototype of a non-analytic
modular form. In what follows, the functions $(c\tau+d)^{-\alpha}$ and
$(c\bar{\tau} +d)^{-\beta}$ for real numbers $(c,d) \neq (0,0)$ and
$\tau \in \mathscr{G}$ will be defined by
\begin{align*}
(c\tau+d)^{-\alpha} & = e^{-\alpha\log(c\tau+d)},
(c\bar{\tau}+d)^{-\beta} = e^{-\beta\log(c\bar{\tau}+d)} \\
\text{with } \log(c\tau+d) & = \log |c\tau+d| + i \arg(c\tau+d), -\pi
<\arg(c\tau + d)\leq \pi\\
\text{and } \log(c\bar{\tau}+d) & = \log |c\bar{\tau}+d| + i
\arg(c\bar{\tau}+d), -\pi \leq \arg(c\bar{\tau}+d) < \pi;
\end{align*}
here the branches of the logarithm are so chosen that always 
$$
\log (c\tau+d) + \log (c\bar{\tau}+d) \text{ is real}.
$$

Let $S = \left(\begin{smallmatrix}
a&b\\c&d\end{smallmatrix}\right)$ be a real matrix with $|S|=ad-bc>0$
and let $\alpha,\beta$ be two complex numbers. Then for any $f(\tau,
\bar{\tau})$, we define $\mathop{f|S}\limits_{\alpha, \beta}$ by
$$
(\mathop{f|S}\limits_{\alpha, \beta})(\tau,\bar{\tau}) =
(c\tau+d)^{-\alpha} (c\bar{\tau}+d)^{-\beta} f(S<\tau>, S<\bar{\tau}>).
$$ 
It is obvious that the series defined above satisfies the
transformation formula \pageoriginale 
$$
(G(\quad,;\alpha,\beta)|_{\alpha, \beta}S) (\tau,\bar{\tau}) = G
(\tau, \bar{\tau};\alpha, \beta)
$$
for $S$ belonging to the modular group. In this section, our aim is to
find a differential equation, which has the given series $G(\tau,
\bar{\tau}; \alpha, \beta)$ as a solution and which is left invariant
by the transformations $f\to \mathop{f|S}\limits_{\alpha,\beta}$.

Following a method of Selberg, we shall show here that this problem
can be reduced in a natural way to the eigen-value problem for
Laplace's differential operator in a certain 3-dimensional Riemannian
space to be defined. Let $\mathscr{R}$ denote the direct product of
the upper half-plane and the real $t$-axis i.e.
$$
\mathscr{R} = \{(\tau,t)| \tau = x+ iy, y >0 \text{ and } t \text{
  real}\}. 
$$
To any real matrix $R = \left(\begin{smallmatrix}
\ast & \ast \\ c & d\end{smallmatrix}\right)$ with $|R|=1$ and any
real number a, we associate a transformation $R_a$ of $\mathscr{R}$
defined by 
$$
R_a(\tau, t) = (R<\tau>, t + \arg(c\tau+d) + 2\pi a).
$$
If $R$ and $S$ are two elements of the group $\Omega$ of all real
matrices of determinant 1, and $a,b$ are any two real numbers, then we
define the composite of two transformations $R_a$ and $S_b$ by
$$
(R_a\cdot S_b) (\tau, t) =R_a (S_b(\tau, t)).
$$
We shall now show that the composite of two transformations $R_a$ and
$S_b$ is again a transformation of the same type associated to
$RS$. Let $(m_1, m_2)$, $(c,d)$ and $(m^{\ast}_1, m^{\ast}_2)$ be the
second rows of the matrices $R,S$ and $RS$
respectively. \pageoriginale Then, by definition, we have 
{\fontsize{9}{11}\selectfont
\begin{align*}
(R_a\cdot S_b) (\tau,t) & = R_a(S<\tau>, t + \arg(c\tau+d)+2\pi a)\\
& = (RS<\tau>, t + \arg(c\tau+d) + 2\pi a+ \arg(m_1S<\tau>+m_2)+2\pi
  b)\\
& =(RS<\tau>, t + \arg(m^{\ast}_1\tau + m^{\ast}_2) + 2\pi
  (a+b+w(R,S)))\\
& = (RS)_{a+b+w(R,S)} (\tau,t),
\end{align*}}\relax
showing that
\begin{equation*}
R_a \cdot S_b = (RS)_{a+b+w(R,S)}. \tag{1}\label{c4:eq1:1}
\end{equation*}
It is obvious from the definition of the mapping $R_a$ that $R_a$ is
equal to the identity mapping on the space $\mathscr{R}$ when $R=E$
and $a=0$ or $R=-E$ and $a=-\dfrac{1}{2}$. From \eqref{c4:eq1:1}, it follows that
the mapping $(S^{-1})_{-b-w(S,S^{-1})}$ is the inverse $(S_b)^{-1}$ of
the mapping $S_b$. Thus the set $\hat{\Omega}$ of all transformations
$R_a$ constitutes a group. If we substitute $R=-E$ and
$a=-\dfrac{1}{2}$ in~\eqref{c4:eq1:1}, we see that 
$$
S_b= (-S)_{b-\frac{1}{2}+w(-E,S)},
$$
which shows that every element $R_a$ of $\hat{\Omega}$ has two
representations namely, one associated to the matrix $R$ and the other
to the matrix $-R$. Let $Z$ denote the subgroup of $\Omega$ consisting
of the two elements $\pm E$ and $\hat{Z}$ denote the subgroup of
$\hat{\Omega}$ consisting of the elements $(\pm E)_a$ for every real
number a. It can be verified that the kernel of the homomorphism
$$
R_a \to RZ
$$
from $\hat{\Omega}$ to $\Omega /Z$ is $\hat{Z}$ i.e. $\hat{\Omega}/Z$
and $\Omega/Z$ are isomorphic. Given a discrete \pageoriginale
subgroup $\Gamma$ of $\Omega$, we are interested in finding a
\textit{relation preserving representation} of $\Gamma/Z$ in the group
$\hat{\Omega}$ i.e. we want to find a special set of representatives
for the cosets of a subgroup $\hat{\Gamma}$ of $\hat{\Omega}$ modulo
$\hat{Z}$, where $\hat{\Gamma}=\{S_a|S\in \Gamma,$ a arbitrary
real\}, such that these representatives themselves form a group
isomorphic with $\Gamma/Z$. We shall obtain one such representation,
if we define a real-valued function $w(R)$ for every $R\in
\Gamma$ satisfying the equations
\begin{align*}
R_{w(R)} & = (-R)_{w(-R)} \tag{2}\label{c4:eq1:2}\\
R_{w(R)} S_{w(S)} & = (RS)_{w(RS)}\tag{3}\label{c4:eq1:3}
\end{align*}
for $R$ and $S$ belonging to $\Gamma$. For the existence of such a
function $w(R)$, it is obviously necessary and sufficient that the
following two equations are satisfied:
\begin{align*}
w(-R) & = w(R) -\frac{1}{2} + w(-E,R)\tag{4}\label{c4:eq1:4}\\
w(RS) & = w(R) + w(S) + w(R,S). \tag{5}\label{c4:eq1:5}
\end{align*}
It can be shown that equations \eqref{c4:eq1:4} 
and \eqref{c4:eq1:5} are simultaneously solvable
for a horocyclic group for which the system of generators and defining
relations are well-known. We shall give here a solution for the
modular group, which we denote by $\Gamma$. Since $V^3=T^2=-E$, we
must have 
$$
(V_{w(V)})^3 = (T_{w(T)})^2 = (-E)_{w(-E)} = E_0,
$$
where $E_0$ is the unit elements of $\hat{\Omega}$. This implies that 
$$
3w(V) + w(V,V) + w(V,V^2) = 2w (T) + w(T,T) = w (-E) = - \frac{1}{2}. 
$$
It now follows that 
$$
w(V) = \frac{1}{6} , \quad w(T) = \frac{1}{4}.
$$\pageoriginale 
Let $\hat{\Gamma}_0$ be the subgroup of $\hat{\Omega}$ generated by
$V_{\dfrac{1}{6}}$ and $T_{\dfrac{1}{4}}$. Then the correspondence
$V_{\dfrac{1}{6}} \to VZ$ and $T_{\dfrac{1}{4}} \to TZ$ can be
extended to a homomorphism of $\hat{\Gamma}_0$ onto
$\Gamma/Z$. Therefore all the relations, which are satisfied by
$v_{\dfrac{1}{6}}$ and $T_{\dfrac{1}{4}}$, will hold between $VZ$ and
$TZ$. But the converse is also true, because the defining relations
$V^3Z=Z$ and $T^2Z=Z$ for the group $\Gamma/Z$ are trivially satisfied
by $V_{\dfrac{1}{6}}$ and $T_{\dfrac{1}{4}}$. Therefore the groups
$\hat{\Gamma}_0$ and $\Gamma/Z$ are isomorphic. This shows that if
$S_a$ and $S_b$ are two elements of $\hat{\Gamma}_0$, then $S_a=S_b$
and therefore $a=b$. Thus we can write $a=w(S)$ and it is obvious that
the function $w(S)$ satisfies the equations \eqref{c4:eq1:4} 
and \eqref{c4:eq1:5}. From $T=UV$,
it follows that 
$$
w(T) =w(U) + w(V), w(U) =\frac{1}{12}.
$$

The group $\Omega/Z$ cannot have a relation-preserving representation
in $\hat{\Omega}$; for, if it were true, then 
$$
\begin{pmatrix}
\lambda & 0\\
0 & \lambda^{-1}
\end{pmatrix} \begin{pmatrix}
1&1\\0&1
\end{pmatrix} \begin{pmatrix}
\lambda^{-1} & 0\\
0 & \lambda
\end{pmatrix} = \begin{pmatrix}
1 & \lambda^2\\
0 & 1
\end{pmatrix} \text{ for real } \lambda > 0
$$
would imply that
\begin{align*}
w(\begin{pmatrix}
1&\lambda^2\\0&1
\end{pmatrix}) & = w(\begin{pmatrix}
\lambda & 0\\0 & \lambda^{-1}
\end{pmatrix}) + w (\begin{pmatrix}
1&1\\0&1
\end{pmatrix}) + w (\begin{pmatrix}
\lambda^{-1} & 0\\
0 & \lambda
\end{pmatrix})\\
& = w(\begin{pmatrix}
1&1\\0&1
\end{pmatrix}) = \frac{1}{12}, 
\end{align*}
whereas \pageoriginale $w(U^2)=2w(U)=\dfrac{1}{6}$, leading to a
contradiction for $\lambda=\surd 2$.

We define a positive definite metric form on $\mathscr{R}$ by 
\begin{equation*}
ds^2 =\frac{dx^2+dy^2}{y^2} + 
(dt -\frac{dx}{dy})^2. \tag{6}\label{c4:eq1:6}
\end{equation*}
This metric form on $\mathscr{R}$ is invariant under the
transformation of $\hat{\Omega}$. In order to prove this, it is
sufficient to prove that $dt-\dfrac{dx}{2y}$ is invariant under the
transformations of $\hat{\Gamma}$, because we have already shown that
$\dfrac{dx^2+dy^2}{y^2}$ is invariant under $\Gamma$ and therefore
under $\hat{\Gamma}$. Let $S =
\left(\begin{smallmatrix} \alpha & \beta\\ \gamma &
  \delta\end{smallmatrix}\right)$ and $(\tau^{\ast},
  t^{\ast})=S_a(\tau,t)$ for some real number $a$. Then
\begin{align*}
dt^{\ast} -dt & = d(\arg\gamma \tau + \delta) = \frac{1}{2i} \{d\log
(\gamma \tau+\delta) - d\log (\gamma \bar{\tau}+\delta)\}\\
& = \frac{\gamma}{2i} \{\frac{d\tau}{\gamma\tau+\delta} -
\frac{d\bar{\tau}}{\gamma \bar{\gamma}+\delta}\}
\end{align*}
and
\begin{align*}
\frac{dx^{\ast}}{2y^{\ast}} - \frac{dx}{2y} & =
\frac{d\tau^{\ast}+d\bar{\tau}^{\ast}}{4y^{\ast}}
-\frac{d\tau+d\bar{\tau}}{4y} \\
& = \frac{1}{4y}
\{(\frac{\gamma\bar{\tau}+\delta}{\gamma\tau+\delta}-1) d\tau +
(\frac{\gamma \tau + \delta}{\gamma \bar{\tau}+\delta}
-1)d\bar{\tau}\}\\
& = \frac{\gamma}{2i} \{\frac{d\tau}{\gamma \tau + \delta} -
\frac{d\bar{\tau}}{\gamma \bar{\tau} + \delta}\}
\end{align*}
implying what we wanted to show. Hence $\mathscr{R}$ is a Riemannian
space with the group $\hat{\Gamma}$ acting on it. The Laplacian
$\Delta$ on the space $\mathscr{R}$ is given by
\begin{equation*}
\Delta = y^2(\frac{\partial}{\partial x^2} +
\frac{\partial^2}{\partial y^2}) + y \frac{\partial^2}{\partial x
  \partial t} + \frac{5}{4} 
\frac{\partial^2}{\partial t^2}. \tag{7}\label{c4:eq1:7}
\end{equation*}
We shall adopt the notation
$$
(\varphi|S_a)(\tau, t) = \varphi(S_a(\tau,t)),
$$\pageoriginale
where $\varphi(\tau,t)$ is any function defined on $\mathscr{R}$ and
$S_a$ is an element of $\hat{\Gamma}$. We now formulate the indicated
eigen-value problem. Let $\Gamma_0$ be a subgroup of finite index in
$\Gamma$ and $v_1$ an even abelian character of $\Gamma_0$. Moreover,
let $\hat{\Gamma}_0$ denote the relation-preserving representation of
$\Gamma_0$ in $\hat{\Omega}$. Then $\hat{\Gamma}_0$ consists of the
transformations $S_{w(S)} (S\in \Gamma_0)$. We look for a real
analytic function $\varphi(\tau, t)$ which satisfies the conditions:
\begin{itemize}
\item[1)] $(\Delta+\lambda)\varphi(\tau, t)=0$ for some real $\lambda
  \geq 0$,

\item[2)] $(\varphi|S_{w(S)}) (\tau,t) = v_1(S)
  \varphi(\tau,t) \;\; (S\in \Gamma_0)$,

\item[3)] $\lim\limits_{T\to\infty} \dfrac{1}{2T} \int\limits^T_{-T}
  \int\int\limits_{\mathfrak{F}_0} \varphi (\tau, t)
  \overline{\varphi(\tau, t)} y^{-2} dx dy dt < \infty$, 
\end{itemize}
where $\mathfrak{F}_0$ is a fundamental domain for $\Gamma_0$ in
$\mathscr{G}$.

It can be seen that the functions
$$
\varphi(\tau, t) = g(\tau)e^{-irt}
$$
satisfy conditions 1), 2) and 3), where $r$ is a given real number and
$g(\tau)$ satisfies the conditions:
\begin{itemize}
\item[1')] $\{y^2(\dfrac{\partial^2}{\partial x^2} +
  \dfrac{\partial^2}{\partial y^2}) - iry \dfrac{\partial}{\partial x} +
  \lambda - \dfrac{5}{4} r^2\} g(\tau)=0$,

\item[2')] $g(S<\tau>) e^{-ir \arg(c\tau+d)} = v_0(S)v_1(S)g(\tau)$,
  with $v_0(S)=e^{2\pi iw (S)}$ and $S =
  \left(\begin{smallmatrix} a&b\\c&d\end{smallmatrix}\right)
    \in \Gamma_0$,

\item[3')] $\int\int\limits_{\mathfrak{F}_0}g(\tau) \overline{g(\tau)}
  y^{-2} dx dy < \infty$.  
\end{itemize}
By \pageoriginale \eqref{c4:eq1:4} and \eqref{c4:eq1:5}, we have
$$
v_0(RS) = \sigma^{(r)} (R,S) v_0(R) v_0(S), v_0(-E) = e^{\pi ir}
$$
showing that $v_0$ is a multiplier system for the group $\Gamma_0$ and
weight $r$ and therefore $v(S)=v_0(S)v_1(S)$ runs over all the
multiplier systems for the group $\Gamma_0$ and weight $r$ when $v_1$
runs over all the even abelian characters of $\Gamma_0$. As a matter
of fact, $v_0(S)$ is the multiplier system for $\Delta^{r/12}(\tau)$
mentioned in theorem~\ref{chap3:thm19}. Finally, we have
$$
e^{-ir\arg(c\tau+d)} =
e^{\frac{r}{2}\{\log(c\bar{\tau}+d)-\log(c\tau+d)\}} =
(c\bar{\tau}+d)^{\frac{r}{2}} / (c\tau+d)^{\frac{r}{2}}.
$$
Let $\alpha=\dfrac{q+r}{2}$, $\beta=\dfrac{q-r}{2}$ and $q$ an
arbitrary number. We introduce $f(\tau)$ by 
$$
g(\tau) = y^{\frac{q}{2}} f(\tau).
$$
Then the conditions 1'), 2') and 3') lead to the following conditions
for $f$:
\begin{itemize}
\item[1'')] $\{y^2 (\dfrac{\partial^2}{\partial x^2} +
  \dfrac{\partial^2}{\partial y^2}) -iry \dfrac{\partial}{\partial} + qy
  \dfrac{\partial}{\partial y} + \lambda - \dfrac{5}{4} r^2 + \dfrac{q}{2}
  (\dfrac{q}{2}-1)\} f(\tau) =0$,

\item[2'')] $(\mathop{f|S}_{\alpha, \beta}) (\tau)= v(S) f(\tau)$ for
  $S\in \Gamma_0$, 

\item[3'')] $\int\int\limits_{\mathfrak{F}_0}f(\tau)
  \overline{f(\tau)} y^{\re(\alpha+\beta)^{-2}} dx dy < \infty$. 
\end{itemize}
Let us choose $q$ in such a manner that the sum of the constant terms
in the differential equation 1'') vanishes. Then 1'') reduces to 
\begin{equation*}
\Omega_{\alpha\beta}f(\tau) = 0 \tag{8}\label{c4:eq1:8}
\end{equation*}
with $\Omega_{\alpha\beta}= -y^2(\dfrac{\partial^2}{\partial x^2} +
\dfrac{\partial^2}{\partial y^2}) +
i(\alpha-\beta)y\dfrac{\partial}{\partial x} - (\alpha+\beta)
y\dfrac{\partial}{\partial y}$, where \pageoriginale $\alpha$ and
$\beta$ are now given by
$$ 
\alpha + \beta = 1 + \sqrt{5r^2 - 4\lambda + 1}, \alpha - \beta = r.
$$

We shall show that the differential equation \eqref{c4:eq1:8} has the invariant
property mentioned above. Since we are interested in the functions
$f(\tau)$ which are non-analytic in $\tau$ but analytic in both the
independent variables $\tau=x+iy$ and $\bar{\tau}=x-iy$, we shall
write $f(\tau,\bar{\tau})$ instead of $f(\tau)$ for the solutions of
\eqref{c4:eq1:8}, as it seems to be a more suitable notation. Changing the
variables from $x,y$ to $\tau$ and $\bar{\tau}$ in
$\Omega_{\alpha\beta}$ we obtain 
\begin{equation*}
\Omega_{\alpha\beta} = (\tau-\bar{\tau})^2 \frac{\partial^2}{\partial
  \tau \partial \bar{\tau}} - \beta(\tau-\bar{\tau})
\frac{\partial}{\partial \tau} + \alpha (\tau-\bar{\tau})
\frac{\partial}{\partial \bar{\tau}}, \tag{9}\label{c4:eq1:9}
\end{equation*}
with 
\begin{equation*}
\frac{\partial}{\partial \tau} = \frac{1}{2} (\frac{\partial}{\partial
x} - i \frac{\partial}{\partial y}), \frac{\partial}{\partial
  \bar{\tau}}  =\frac{1}{2} (\frac{\partial}{\partial
  x}+i\frac{\partial}{\partial y}). \tag{9a}\label{c4:eq1:9a}
\end{equation*}
We look upon \eqref{c4:eq1:9a} as a definition of $\dfrac{\partial}{\partial \tau}$
and $\dfrac{\partial}{\partial \bar{\tau}}$. The invariance property
of the Laplacian $\Delta$ expressed by
$$
\Delta((\varphi|S_a)(\tau,t)) = (\Delta\varphi|S_a)(\tau,t) \text{ for
} S_a \in \hat{\Omega}
$$
implies for $\Omega_{\alpha,\beta}$ the invariance property
\begin{equation*}
\Omega_{\alpha, \beta} ((\mathop{f|S}_{\alpha, \beta})(\tau,
\bar{\tau})) = (\Omega_{\alpha, \beta} \mathop{f|S}_{\alpha,\beta})
(\tau, \bar{\tau}), \text{ for } S\in \Omega. \tag{10}\label{c4:eq1:10}
\end{equation*}
This is the invariance property of the differential equation mentioned
in the beginning. The two parameter domains defined by
\begin{equation*}
1) \quad r,\lambda \text{ real and } \lambda \geq 0, \quad 2) \quad
\alpha, \beta \text{ both real } \tag{11}\label{c4:eq1:11}
\end{equation*}
will be of particular interest. In the theory of indefinite quadratic
forms, $(2\alpha, 2\beta)$ will \pageoriginale occur as the signature of an
indefinite quadratic form.

We shall now define certain linear differential operators, which\break
transform the series $G(\tau,\bar{\tau};\alpha,\beta)$ into one of the
series $G(\tau,\bar{\tau}; \alpha\pm 1,\beta \pm 1)$ and which are
connected in a natural way to the operator $\Omega_{\alpha\beta}$. We
set
\begin{align*}
K_{\alpha} & =\alpha + (\tau-\bar{\tau}) \frac{\partial}{\partial
  \tau} = \alpha + y (i\frac{\partial}{\partial
  x}+\frac{\partial}{\partial y}). \tag{12}\label{c4:eq1:12}\\
\Lambda_{\beta} & = -\beta + (\tau-\bar{\tau})
\frac{\partial}{\partial \bar{\tau}} = -\beta + y (i
\frac{\partial}{\partial x} - 
\frac{\partial}{\partial y}). \tag{13}\label{c4:eq1:13}
\end{align*}
It is an immediate consequence of the definition that
\begin{align*}
K_{\alpha} G(\tau,\bar{\tau};\alpha,\beta) & = \alpha G
(\tau,\bar{\tau};\alpha+1,\beta-1)\\
\Lambda_{\beta} G(\tau, \bar{\tau};\alpha, \beta) &= -\beta G (\tau,
\bar{\tau}; \alpha -1, \beta+1), 
\end{align*}
where the differentiation is formally carried out term by term, which
is quite justified under the assumption $\re(\alpha+\beta)>2$. It can
now be deduced that $G(\tau, \bar{\tau}; \alpha, \beta)$ satisfies the
two differential equations:
$$
\{\Lambda_{\beta-1} K_{\alpha} + \alpha (\beta-1)\} G(\tau,\bar{\tau};
\alpha, \beta) = \{K_{\alpha-1} \Lambda_{\beta} + \beta(\alpha-1)\}
G(\tau, \bar{\tau};\alpha, \beta) = 0.
$$
But by the definition of $\Omega_{\alpha,\beta}$, $K_{\alpha}$ and
$\Lambda_{\beta}$, we have 
\begin{equation*}
\Lambda_{\beta-1} K_{\alpha} + \alpha(\beta-1) = K_{\alpha-1}
\Lambda_{\beta} + \beta(\alpha-1) = 
\Omega_{\alpha \beta}; \tag{14}\label{c4:eq1:14}
\end{equation*}
therefore the two differential equations are identical. With the help
of \eqref{c4:eq1:14}, we obtain the following commutation relations:
\begin{align*}
\Omega_{\alpha\beta} \Lambda_{\beta-1} & = \Lambda_{\beta-1}
\Omega_{\alpha +1, \beta-1}, \\
K_{\alpha} \Omega_{\alpha \beta} & = \Omega_{\alpha+1,\beta-1}
K_{\alpha} \tag{15}\label{c4:eq1:15}
\end{align*}

Let \pageoriginale $\{\alpha, \beta\}$ denote the space of functions
$f(\tau,\bar{\tau})$, which are real analytic in $\mathscr{G}$ and are
solutions of the differential equation $\Omega_{\alpha\beta}f=0$. We
define the operators $K$ and $\Lambda$ on the space $\{\alpha, \beta\}$
by
\begin{equation*}
K f = K_{\alpha} f, \Lambda f = \Lambda_{\beta} f \text{ for any }
f\in \{\alpha, \beta\}. \tag{16}\label{c4:eq1:16}
\end{equation*}
With the help of \eqref{c4:eq1:15}, we see immediately that 
$$
K\{\alpha, \beta\} \subset \{\alpha + 1, \beta-1\}, \Lambda\{\alpha,
\beta\} \subset \{\alpha-1, \beta+1\}.
$$
This shows that, for integral $n \geq 0$, the $n-th$ iterate $K^n$
(respectively $\Lambda^n$) of $K$ (respectively $\Lambda$), is
well-defined and is identical with the operator $K_{\alpha+n-1} \ldots
K_{\alpha+1}K_{\alpha}$ (respectively $\Lambda_{\beta+n-1} \ldots
\Lambda_{\beta+1} \Lambda_{\beta}$) on the space $\{\alpha,
\beta\}$. For any real matrix $S$ with positive determinant, we also
define
$$
f|S =\mathop{f|S}_{\alpha,\beta} \text{ for } f\in \{\alpha,
\beta\}. 
$$
We shall now prove that the operators $K$ and $\Lambda$ commute with
this operator corresponding to
$S=\left(\begin{smallmatrix} a&b\\c&d\end{smallmatrix}\right)$ with
  $|S|>0$: namely,
\begin{equation*}
(K|f)|S = K(f|S), \quad (\Lambda f)|S
= \Lambda(f|S). \tag{17}\label{c4:eq1:17}
\end{equation*}
For the proof, we can assume without loss of generality that
$|S|=1$. Let us set $\tau_1=S<\tau>$, $\bar{\tau}_1 =
S<\bar{\tau}>$. Then indeed
\begin{align*}
(Kf) |S (\tau, \bar{\tau}) &= (c\tau
  +d)^{-\alpha-1}(c\bar{\tau}+d)^{-\beta+1} \{\alpha +
  (\tau_1-\bar{\tau}_1) \frac{\partial}{\partial \tau_1}\} f
  (\tau,\bar{\tau}_1)\\
& = (c\tau+d)^{-\alpha-1} (c\bar{\tau}+d)^{-\beta+1} \{\alpha +
  (\tau-\bar{\tau}) \frac{c\tau+d}{c\bar{\tau}+d}
  \frac{\partial}{\partial \tau}\} f(\tau_1,\bar{\tau}_1)\\
& = \{\alpha + (\tau-\bar{\tau}) \frac{\partial}{\partial \tau}\}
  (c\tau+d)^{-\alpha} (c\bar{\tau}+d)^{-\beta} f(\tau_1,
  \bar{\tau}_1)\\
& = K(f|S)(\tau,\bar{\tau}).
\end{align*}
The second \pageoriginale statement follows in a similar way.

For any function $f(\tau,\bar{\tau})$ we define the operator $\bX$ by
\begin{equation*}
\bX f(\tau,\bar{\tau}) = 
f(-\bar{\tau},-\tau), \tag{18}\label{c4:eq1:18}
\end{equation*}
which is equivalent to the substitution $x\to -x$ leaving $y$
fixed. It is an easy consequence of the definition that
$$
\bX \{\alpha, \beta\} = \{\beta, \alpha\}.
$$
If $S = \left(\begin{smallmatrix} a&b\\c&d\end{smallmatrix}\right)$
  belongs to $\Omega$ and $S^{\ast} =
  \left(\begin{smallmatrix}1&0\\0&-1 \end{smallmatrix}\right)
  S\left(\begin{smallmatrix} 1&0\\0&-1 \end{smallmatrix}\right)^{-1} =
  \left(\begin{smallmatrix} a&-b\\-c&d\end{smallmatrix}\right)$, then,
    for any $f\in \{\alpha,\beta\}$, we have 
\begin{equation*}
\mathop{(\bX f)|S^{\ast}}_{\beta,\alpha}
 = u(S) \bX \mathop{(f|S)_{\alpha,\beta}}, \tag{19}\label{c4:eq1:19}
\end{equation*}
where $u(S)$ is the factor system given by
$$
u(S) = u_{\alpha,\beta} (S) = 
\begin{cases}
1 & \text{ for } c\neq0\\
e^{(\alpha-\beta)i\pi(1-\sgn d)} & \text{ for } c=0.
\end{cases}
$$
For, with $g=\bX f$, we have $g (\tau,\bar{\tau}) = f(-\bar{\tau},
-\tau)$ and
{\fontsize{10}{12}\selectfont
\begin{align*}
\mathop{((\bX)|S^{\ast})}_{\beta, \alpha} (\tau,\bar{\tau}) & =
\mathop{(g|S^{\ast})}_{\beta,\alpha} (\tau,\bar{\tau}) =
(-c\tau+d)^{-\beta} (-c\bar{\tau}+d)^{-\alpha}
g(S^{\ast}<\tau>,S^{\ast}<\bar{\tau}>)\\
& = u(S)(c(-\bar{\tau})+d)^{-\alpha} (c(-\tau)+d)^{-\beta}
f(S<-\bar{\tau}>, S<-\tau>) \\
& = u(S) \bX \mathop{(f|S)}_{\alpha,\beta} (\tau,\bar{\tau}).
\end{align*}}\relax
The factor system $u(S)$ appears on the right hand side of~\ref{c4:eq1:19},
because the definitions of the powers of $(c\tau+d)$ and
$(c\bar{\tau}+d)$ are different in the upper and lower half-planes. It
can be verified easily that 
$$
\bX^2 = 1, K \bX = - \bX \Lambda, \bX K = -\Lambda \bX.
$$

We introduce \pageoriginale here an operator $\Theta$, which will be
used in the theory of Hecke operators in the next chapter and which
acts on the space $\{\alpha, \beta\}$ under the assumptions
$$
r=\alpha-\beta \text{ integral } \geq 0, \Gamma(\beta) \neq \infty. 
$$
It is defined by
\begin{equation*}
\Theta = \frac{\Gamma(\beta)}
{\Gamma(\alpha)} \bX \Lambda^r. \tag{20}\label{c4:eq1:20}
\end{equation*}
We shall show that 
$$
\Theta \{\alpha,\beta\} = \{\alpha, \beta\}, \quad \Theta^2 =1.
$$
By definition, we have
$$
\Theta\{\alpha, \beta\} = \bX \Lambda^r \{\alpha, \beta\} \subset \bX
\{\alpha, r, \beta+r\} = \bX \{\beta, \alpha\} = \{\alpha, \beta\}.
$$
Moreover $K\bX=-\bX \Lambda$ implies that the operator 
\begin{align*}
\Theta^2 & = (\frac{\Gamma(\beta)}{\Gamma(\alpha)})^2 \bX \Lambda^r
\bX \Lambda^r = (-1)^r (\frac{\Gamma(\beta)}{\Gamma(\alpha)})^2 K^r
\bX^2 \Lambda^r\\
& = (-1)^r (\frac{\Gamma(\beta)}{\Gamma(\alpha)})^2 K^r \Lambda^r
=(-1)^r (\frac{\Gamma(\beta)}{\Gamma(\alpha)})^2 K_{\alpha-1}
K_{\alpha-2} \ldots K_{\beta} \Lambda_{\alpha-1} \ldots \Lambda_{\beta}
\end{align*}
on the space $\{\alpha, \beta\}$.
But on the space $\{\beta+n-1, \alpha-1-n\}$,
$K_{\beta+n}\Lambda_{\alpha-1-n}=-(\beta + n)(\alpha-1-n)$; therefore,
it follows that 
$$
\Theta^2 = (\frac{\Gamma(\beta)}{\Gamma(\alpha)})^2 \prod^{r-1}_{n=0}
(\beta+n) (\alpha-1-n) = (\frac{\Gamma(\beta)}{\Gamma(\alpha)})^2
\prod^{r-1}_{n=0} (\beta+n)^2=1.
$$
Consequently $\{\alpha, \beta\} \subset \Theta \{\alpha, \beta\}$,
which proves that
$$
\Theta \{\alpha, \beta\} = \{\alpha,\beta\}.
$$

\section{Non-Analytic Forms}\label{chap4:sec2}\pageoriginale

For our later use, we determine all the periodic functions which are
contained in the space $\{\alpha, \beta\}$ and increase at most as a
power of $y$ uniformly in $x$, when $y$ tends to infinity. We define,
for $y>0$ and $\in = \pm 1$, the function $W(\in y ;
\alpha, \beta)$ by
\begin{equation*}
W (\in y; \alpha, \beta) = y^{-\frac{1}{2}q}
W_{\frac{1}{2}r\in, \frac{1}{2} (q-1)} (2y) \tag{1}\label{c4:eq2:1}
\end{equation*}
where $q=\alpha+\beta$, $r=\alpha-\beta$ and $W_{\ell,m}(y)$ is
Whittaker's function which is a solution of Whittaker's differential
equation
\begin{equation*}
\{4y^2 \frac{d^2}{dy^2} + 1 - 4m^2 + 
4 \ell y - y^2\} W(y) =0 \tag{2}\label{c4:eq2:2}
\end{equation*}
We set, for $y>0$,
$$
u(y,q) = \frac{y^{1-q}-1}{1-q} = 
\begin{cases}
\sum^{\infty}_{n=1} \frac{(\log y)^n}{n!} (1-q)^{n-1} & \text{ for } q
\neq 1\\
\log y &\text{ for } q=1.
\end{cases}
$$

\setcounter{lem}{5}

\begin{lem}\label{chap4:lem6}
Let $g_{\in}(y)e^{i\in x} \in \{\alpha,
\beta\}$ and let $g_{\in}(y)=o(y^k)$ for $y=\to \infty$ with a
certain constant $K$ if $\in \neq 0$. Then
$$
g_0(y) = a \;\;  u (y,q) + b, g_{\in} (y) = a W (\in y,
\alpha, \beta) \text{ for } \in^2=1. 
$$
\end{lem}

\begin{proof}
Since $g_{\in}(y)e^{i\in x}$ belongs to $\{\alpha,
\beta\}$, it satisfies the differential equation $\Omega_{\alpha
  \beta} g_{\in}(y)e^{i\in x}=0$, which shows that
\begin{equation*}
\{y\frac{d^2}{dy^2} + q \frac{d}{dy} + \in r - \in^2 y
\} g_{\in} (y) =0. \tag{3}\label{c4:eq2:3}
\end{equation*}
If $\in =0$, then 1 and the function $u(y,q)$ form a system of
independent solutions and therefore $g_0(y)=au(y,q)+b$ for some
constants $a$ and $b$. Let \pageoriginale $\in^2 = 1$. Substituting
$g_{\in}(y) = y^{-\dfrac{1}{2}q}W_0(\dfrac{1}{2}r\in,
\dfrac{1}{2}(q-1), 2y)$ in \eqref{c4:eq2:3} we see that $W(y)=W_0(\ell,m,y)$ is a
solution of differential equation \eqref{c4:eq2:2}. But Whittaker's differential
equation has two independent solutions, one of which tends to $\infty$
exponentially when $y\to\infty$ and therefore cannot occur in
$g_{\in}(y)$; the other solution of \eqref{c4:eq2:2} is $W_{\ell,m}(y)$
with the asymptotic behaviour given by
\begin{equation*}
W_{\ell,m}(y) \approx e^{-\frac{1}{2}y} y^{\ell}
\{1+\sum^{\infty}_{n=1} \frac{1}{n!y^n} \prod^n_{q=1} [m^2-(\ell
  +\frac{1}{2}-q)^2]\}. \tag{5}\label{c4:eq2:5}
\end{equation*}
Therefore we see by \eqref{c4:eq2:1} that
$$
g_{\in}(y) = aW(\in y ; \alpha , \beta)
$$
and the lemma is proved.

The behaviour of the functions mentioned in 
lemma~\ref{chap4:lem6} under the action of the 
operators $K_{\alpha},\Lambda_{\beta}$ and $\bX$ is given by
\end{proof}

\begin{lem}\label{chap4:lem7}
\begin{enumerate}
\item $\bX 1 = 1, \bX u (y,q) = u (y,q)$,
$$
\bX W (\in y, \alpha, \beta) e^{i\in x} = W
(-\in y, \beta, \alpha) e^{-i\in x} (\in^2=1).
$$


\item $K_{\alpha}1=\alpha, K_{\alpha} u (y,q)=(1-\beta) u(y,q)+1$
{\fontsize{9}{11}\selectfont
$$ 
K_{\alpha} W (\in y ; \alpha, \beta) e^{i\in x} = 
\begin{cases}
-W(\in y; \alpha+1, \beta-1) e^{i\in x} & \text{ for }
\in =1\\
-\alpha(\beta-1) W (\in y; \alpha +1, \beta-1)e^{i\in
  x} & \text{ for } \in = -1.
\end{cases} 
$$}\relax

\item $\Lambda_{\beta}1 = -\beta, \Lambda_{\beta} u(y,q) = (\alpha-1)
  u(y,q)-1$.
{\fontsize{9}{11}\selectfont
$$
\Lambda_{\beta} W (\in y; \alpha, \beta) e^{i\in x} = 
\begin{cases}
(\alpha-1) \beta W (\in y ; \alpha -1, \beta+1)
  e^{i\in x} & \text{ for } \in =1\\
W(\in y ; \alpha-1, \beta +1)e^{i\in x} & \text{ for }
\in = -1.
\end{cases}
$$}\relax
\end{enumerate}
\end{lem}

\begin{proof}
The \pageoriginale assertion about the action of the operators $\bX$,
$K_{\alpha}$ and $\Lambda_{\beta}$ on the functions 1 and $u(y,q)$ is
trivial. Since $W(\in y ; \alpha, \beta)=W(-\in y;
\beta, \alpha)$, it follows that $\bX(W(\in y ; \alpha,
\beta)e^{i\in x}) = W(-\in y; \beta,
\alpha)e^{-i\in x}$. In order to prove the remaining
statements in the lemma, we shall make use of certain identities
between the solutions $W_0(\ell,m,y)$ of Whittaker's differential
equation for different values of the parameters $\ell$ and $m$. It can
be verified that
$$
yW'_0 (\ell, m, y) \pm (\ell-\frac{1}{2}y) W_0 (\ell, m, y)
$$
is again a solution of the type $W_0(\ell \pm 1, m,y)$. Let us assume
that $W_0(\ell,m,y)=W_{\ell,m}(y)$. Then the asymptotic behaviour of
this function shows that, but for a constant factor, $W_0(\ell \pm 1,
m,y)$ is identical with $W_{\ell\pm 1, m}(y)$. This constant factor
can be determined by considering the asymptotic expansion 
\eqref{c4:eq2:5} of $W_{\ell,m}(y)$. Thus we obtain the identities
\begin{align*}
yW'_{\ell,m}(y) - (\ell - \frac{1}{2}y) W_{\ell,m}(y) & = -(m^2-(\ell-\frac{1}{2})^2)
W_{\ell-1,m}(y),\\
yW'_{\ell,m}(y) + (\ell -\frac{1}{2}y) W_{\ell,m}(y) & =
-W_{\ell+1,m}(y).
\end{align*}
By the definition of the operator $K_{\alpha}$, we have
{\fontsize{9}{11}\selectfont
\begin{align*}
K_{\alpha}W(\in y; \alpha, \beta) e^{i\in x} & = 
\{\alpha + y (i\frac{\partial}{\partial x} + \frac{\partial}{\partial
  y})\} y^{-\frac{1}{2}q} W_{\frac{1}{2}r\in,
  \frac{1}{2}(q-1)} (2y)^{ei\in x}\\
& = y^{-\frac{1}{2}q} e^{i\in x} \{\alpha -\frac{1}{2} q -
\in y + y \frac{\partial}{\partial y}\}
W_{\frac{1}{2}r\in, \frac{1}{2}(q-1)} (2y)\\
& = y^{-\frac{1}{2}q} e^{i\in x} (\alpha
-\frac{1}{2}q-\in y) W_{\frac{1}{2}r\in,
  \frac{1}{2}(q-1)} (2y) + 2y W'_{\frac{1}{2}r\in,
  \frac{1}{2}(q-1)} (2y)\\
& = \begin{cases}
-y^{-\frac{1}{2}q} W_{\frac{1}{2}r+1, \frac{1}{2}(q-1)} (2y) e^{ix} &
\text{ for } \in =1\\
-\frac{q+r}{2} \frac{q-r-2}{2} y^{-\frac{1}{2}q} W_{-\frac{1}{2}r-1,
  \frac{1}{2}(q-1)} (2y) e^{-ix} & \text{ for } \in =-1.
\end{cases}\\
& = \begin{cases}
-W(y;\alpha +1, \beta -1)e^{ix} & \text{ for } \in=1\\
-\alpha (\beta-1) W(-y;\alpha + 1, \beta-1) e^{ix} \text{ for }
\in =-1.
\end{cases}
\end{align*}}\relax\pageoriginale

The corresponding result for $\Lambda_{\beta}W(\in y ; \alpha,
\beta)e^{i\in x}$ can be proved in a similar way or could be
derived from above with the help of the identity
$\Lambda_{\beta}=-\bX K_{\beta} \bX$.

We shall now find the asymptotic behaviour of the function
$W(\in y; \alpha, \beta)$ as $y\to 0$ and $y\to \infty$. 
\end{proof}

\begin{lem}\label{chap4:lem8}
For $y>0$ and $\in =\pm 1$, the following asymptotic formulae
hold:
\begin{align*}
& W(\in y ; \alpha, \beta) \sim 2^{\frac{1}{2}r\in}
  y^{\frac{1}{2}(q-\in r)} e^{-y} (y\to \infty), \\
& W(\in y; \alpha, \beta) \sim 2^{\frac{1}{2}(2-q)}
  \frac{\Gamma(q-1)}{\Gamma(\frac{q-\in r}{2})} y^{1-q} (y\to
  0), 
\end{align*}
\textit{provided that in the latter case} $\re(q-1)>0$, $\re(q\pm
\in r)>0$.
\end{lem}

\begin{proof}
The first formula follows from the asymptotic expansion \eqref{c4:eq2:5} for
$W_{\ell,m}(y)$ and the second from the integral representation
$$
W_{\ell,m}(y) = \frac{y^{\frac{1}{2}-m}
  e^{-\frac{1}{2}y}}{\Gamma(m+\frac{1}{2}-\ell)}
\int\limits^{\infty}_0 e^{-u} u^{m-\ell-\frac{1}{2}}
(u+y)^{m+\ell-\frac{1}{2}} du,
$$
for $\re(m+\dfrac{1}{2}\pm \ell)>0$ and $\re m >0$. It is now clear
that
$$
W_{\ell,m}(y) \sim \frac{\Gamma(2m)}{\Gamma(m+\frac{1}{2}-\ell)}
y^{\frac{1}{2}-m} (y\to 0).
$$
\end{proof}

Let \pageoriginale $\alpha$ and $\beta$ be two arbitrary complex numbers with
$r=\alpha-\beta$ real. Let $\Gamma$ denote a horocyclic group and $v$
a multiplier system for the group $\Gamma$ and weight $r$. We assume
that $-E$ belongs to $\Gamma$. By an \textit{automorphic form of the
  type} $\{\Gamma, \alpha, \beta, v\}$, we mean a function
$f(\tau,\bar{\tau})$ with the following properties:
\begin{itemize}
\item[1)] $f(\tau, \bar{\tau})$ is real analytic in $\mathscr{G}$ and
  is a solution of the differential equation $\Omega_{\alpha
    \beta}f=0$,

\item[2)] $(f|S)(\tau,\bar{\tau}) = v(S) f(\tau, \bar{\tau})$ for $S
  \in \Gamma$ and

\item[3)] If $A^{-1}<\infty>$ is a parabolic cusp of $\Gamma$, then
  with some constant $K>0$, $(f|A^{-1})(\tau, \bar{\tau}) = o(y^K)$
  for $y\to \infty$, uniformly in $x$.
\end{itemize}
We shall denote the space of these automorphic forms by $[\Gamma,
  \alpha, \beta, v]$. Let $N>0$ be the least number so determined that
$H=A^{-1} U^N A$ belongs to $\Gamma$ and let 
$$
\sigma(A,H) v(H) = \sigma(H,A^{-1}) v(H) = v^{A^{-1}}(U^N) = e^{2\pi i
\kappa}, 0 \leq \kappa < 1.
$$
Using the characterising properties 1), 2), 3) above for $f(\tau,
\bar{\tau}) \in [\Gamma, \alpha, \beta,\break v]$, we shall show
that, at the parabolic cusp $A^{-1}<\infty>$, there exists for $f$ a
Fourier expansion of the type 
\begin{equation*}
(f|A^{-1}) (\tau,\bar{\tau})  = a_0u(y,q) + b_0 +
\sum_{n\kappa\neq0} a_{n+\kappa} W (\frac{2\pi(n+\kappa)}{N} y;\alpha,
\beta) e^{2\pi i (n+\kappa)x/N} \tag{6}\label{c4:eq2:6}
\end{equation*}
where \pageoriginale $a_0$ and $b_0$ are equal to 0 in case $\kappa
>0$. Since the substitution $\tau\to \tau + N$ transforms
$(f|A^{-1})(\tau,\bar{\tau})$ to $e^{2\pi
  i\kappa}(f|A^{-1})(\tau,\bar{\tau})$, we have, in any case, the
Fourier expansion
\begin{align*}
(f|A^{-1}) (\tau,\bar{\tau}) & = \sum^{\infty}_{n=-\infty}
  \alpha_{n+\kappa} (y) e^{2\pi i (n+\kappa)x/N}\\
\text{with } \qquad \alpha_{n+\kappa}(y) & = \frac{1}{N}
\int\limits^N_0 (f|A^{-1}) (\tau,\bar{\tau})e^{-2\pi i(n+\kappa)x/N} dx.
\end{align*}
Because of property 3) above, we conclude that, for $y\to \infty$,
$\alpha_{n+\kappa}(y)$ increases atmost as a power of $y$. But
$\Omega_{\alpha|\beta}(f|A^{-1}) = (\Omega_{\alpha\beta} f)|A^{-1}=0$,
which implies that $\alpha_{n+\kappa}(y)e^{2\pi i(n+\kappa)x/N}$
satisfies the requirements of lemma~\ref{chap4:lem6}. Therefore, the Fourier
expansion of $(f|A^{-1})(\tau , \bar{\tau})$ may be seen to be of the
type \eqref{c4:eq2:6}. Conversely, if $f(\tau,\bar{\tau})$ is such that for a real
matrix $A$ with $|A|=1$, $(f|A^{-1})(\tau,\bar{\tau})$ has a series
expansion as in \eqref{c4:eq2:6}, then lemma 8 shows that 
$$
(f|A^{-1})(\tau,\bar{\tau}) = o(y^K) \text{ for } y \to \infty,
$$
with some positive constant $K$.

Applying $K_{\alpha},\Lambda_{\beta}$ to \eqref{c4:eq2:6} and making 
use of lemma~\ref{chap4:lem7} and \eqref{c4:eq1:17} of \S~\ref{chap4:sec1}, we get
\begin{align*}
&((K_{\alpha}f)|A^{-1}) (\tau,\bar{\tau}) = (1-\beta) a_0 u(y,q) +
  a_0 + \alpha b_0 -\\
&\qquad - \sum_{n+\kappa >0} a_{n+\kappa} W (\frac{2\pi(n+\kappa)}{N}) y;
  \alpha+1, \beta-1) e^{2\pi i(n+\kappa)x/N} - \tag{7}\label{c4:eq2:7}\\
&\qquad - \alpha(\beta-1) \sum_{n+\kappa<0} a_{n+\kappa} W
  (\frac{2\pi(n+\kappa)}{N}y; \alpha +1, \beta-1) e^{2\pi
    i(n+\kappa)x/N},\\
&((\Lambda_{\beta}f)|A^{-1})(\tau, \bar{\tau})  = (\alpha-1)
 a_0 u(y,q) -a_0 -\beta b_0 + \\
&\qquad + (\alpha-1) \beta \sum_{n+\kappa>0} a_{n+\kappa} W
 (\frac{2\pi(n+\kappa)}{N} y;\alpha-1, \beta+1) e^{2\pi
   i(n+\kappa)x/N} + \tag{8}\label{c4:eq2:8}\\
&\qquad + \sum_{n+\kappa < 0} a_{n+\kappa} W (\frac{2\pi(n+\kappa)}{N} y;
 \alpha-1, \beta+1) e^{2\pi i(n+\kappa)x/N}.
\end{align*}\pageoriginale

With the help of \eqref{c4:eq2:7}, \eqref{c4:eq2:8} and \S~\ref{chap3:sec2} 
of chapter~\ref{chap3}, in which the
transformation properties of a multiplier system have been described,
the following relations can be established:
\begin{align*}
[\Gamma, \alpha, \beta, v] |S & = [S^{-1}\Gamma S, \alpha, \beta, v^S]
\text{ for } S \in \Omega, \tag{9}\label{c4:eq2:9}\\
\bX [\Gamma, \alpha, \beta, v] & = [\Gamma^{\ast}, \beta, \alpha,
  v^{\ast}] \tag{10}\label{c4:eq2:10}
\end{align*}
with $v^{\ast}(S^{\ast}) = u(S)v(S)$, $S^{\ast} =
\left(\begin{smallmatrix} 1&0\\0&-1\end{smallmatrix}\right)^{-1} S
  \left(\begin{smallmatrix} 1&0\\0&-1 \end{smallmatrix}\right)$,
  $\Gamma^{\ast} = \left(\begin{smallmatrix}
    1&0\\0&-1\end{smallmatrix}\right)^{-1}\Gamma
    \left(\begin{smallmatrix} 1&0\\0&-1\end{smallmatrix}\right)$, 
\begin{gather*}
K_{\alpha} [\Gamma, \alpha,\beta,v] \subset [\Gamma, \alpha+1,
  \beta-1, v], \tag{11}\label{c4:eq2:11}\\
\Lambda_{\beta}[\Gamma,\alpha,\beta,v] \subset [\Gamma, \alpha-1,
  \beta+1, v]. \tag{12}\label{c4:eq2:12}
\end{gather*}
If, further $\alpha(\beta-1)\neq 0$, then $K_{\alpha}$ maps the space
$[\Gamma, \alpha, \beta,v]$ onto the space $[\Gamma, \alpha+1,
  \beta-1,v]$, because the operator
$K_{\alpha}\dfrac{1}{\alpha(1-\beta)} \Lambda_{\beta-1}$ acts as the
identity on the space $[\Gamma, \alpha+1, \beta-1, v]$ and therefore
any $f\in[\Gamma,\alpha+1, \beta-1,v]$ is the image of
$\dfrac{1}{\alpha(1-\beta)}\Lambda_{\beta-1} f\in [\Gamma,
  \alpha,\beta,v]$ under $K_{\alpha}$. Similarly, it can be seen that
$\Lambda_{\beta}$ maps the space $[\Gamma, \alpha, \beta, v]$ onto the
space $[\Gamma, \alpha-1, \beta+1, v]$, in case $\beta(\alpha-1)\neq
0$. Thus from \eqref{c4:eq2:11} and \eqref{c4:eq2:12}, we see that
\begin{align*}
K_{\alpha}[\Gamma, \alpha, \beta, v] & = [\Gamma, \alpha+1, \beta-1,
  v], \text{ in case } \alpha(\beta-1) \neq 0 \tag{11'}\label{c4:eq2:11'}\\
\Lambda_{\beta} [\Gamma, \alpha, \beta, v] & = [\Gamma, \alpha-1,
  \beta+1,v], \text{ in case } \beta (\alpha-1) \neq 0 \tag{12'}\label{c4:eq2:12'}
\end{align*}
For $r=\alpha-\beta$ integral $\geq 0$ and $\Gamma(\beta)\neq \infty$,
we have 
\begin{equation*}
\Theta [\Gamma, \alpha, \beta, v] = [\Gamma^{\ast}, \alpha, \beta
  ,v^{\ast}], \tag{13}\label{c4:eq2:13}
\end{equation*}\pageoriginale
because of the relations
\begin{align*}
\Theta[\Gamma, \alpha, \beta, v] & = \bX \Lambda^r[\Gamma, \alpha,
  \beta, v] \subset \bX [\Gamma, \alpha-r, \beta+r, v]\\
& = \bX [\Gamma, \beta, \alpha, v] = [\Gamma^{\ast}, \alpha, \beta, v^{\ast}]
\end{align*}
and $\Theta^2=1$.

By complete induction on $h$, it follows from \eqref{c4:eq2:8} that 
{\fontsize{10}{12}\selectfont
\begin{align*}
& ((\Lambda^hf)|A^{-1})(\tau,\bar{\tau})  = \\
& \frac{\Gamma(\alpha)}{\Gamma(\alpha-h)} a_0 u (y,q) + a_0
\sum^{h-1}_{\ell=0} (-1)^{h-1} \frac{\Gamma(\alpha)
  \Gamma(\beta+h)}{\Gamma(\alpha-1)\Gamma(\beta+\ell+1)} + (-1)^h
\frac{\Gamma(\beta+h)}{\Gamma(\beta)} b_0 + \\
& + \frac{\Gamma(\beta+h)\Gamma(\alpha)}{\Gamma(\beta)\Gamma(\alpha-h)}
\sum_{n+\kappa>0} a_{n+\kappa} W(\frac{2\pi(n+\kappa)}{N} y;
\alpha-h, \beta+h) e^{2\pi i(n+\kappa)x/N}\\
& + \sum_{n+\kappa<0} a_{n+\kappa} W
(\frac{2\pi(n+\kappa)}{N}y;\alpha-h, \beta+h) e^{2\pi i(n+\kappa)x/N}.
\end{align*}}\relax
where the $\Gamma$-functions $\dfrac{\Gamma(\beta+h)}{\Gamma(\beta)}$
and $\dfrac{\Gamma(\beta+h)}{\Gamma(\beta+\ell+1)}$ represent the
value of the analytic functions $\dfrac{\Gamma(z+h)}{\Gamma(z)}$ and
$\dfrac{\Gamma(z+h)}{\Gamma(z+\ell+1)}$ at the point $z=\beta$. Using
lemma~\ref{chap4:lem7} and \eqref{c4:eq1:19} of \S~\ref{chap4:sec1} 
for $r=\alpha-\beta$ integral $\geq 0$ and
$\Gamma(\beta) \neq \infty$, we now see with $u(A^{-1}) = u_{\alpha,
\beta}(A^{-1})$ that 
\begin{align*}
&u(A^{-1}) (\Theta f|A^{\ast-1}) (\tau,\bar{\tau}) = u(A^{-1})
\frac{\Gamma(\beta)}{\Gamma(\alpha)} ((\bX \Lambda^r f)|A^{\ast-1})
(\tau,\bar{\tau}) =\\
&\qquad = \frac{\Gamma(\beta)}{\Gamma(\alpha)} (\bX
\{(\Lambda^rf)|A^{-1}\})(\tau,\bar{\tau}) \\
&\qquad = a_0 u(y,q) + \sum^{r-1}_{\ell=0} (-1)^{r-\ell}
\frac{\Gamma(\beta)\Gamma(\alpha)}{\Gamma
  (\alpha-\ell)\Gamma(\beta+\ell+l)} \cdot a_0 + (-1)^rb_0 + \\
&\qquad + \frac{\Gamma(\alpha)}{\Gamma(\beta)} \sum_{n+\kappa > 0}
a_{n+\kappa} W (\frac{-2\pi i(n+\kappa)}{N} y; \alpha, \beta)
e^{-2\pi(n+\kappa)x/N} + \\
&\qquad + \frac{\Gamma(\beta)}{\Gamma(\alpha)} \sum_{n+\kappa<0}
a_{n+\kappa} W(\frac{-2\pi(n+\kappa)}{N} y; \alpha, \beta) e^{-2\pi
  i(n+\kappa)x/N}. 
\end{align*}
Let \pageoriginale $\kappa^{\ast}$ be so chosen that $\kappa \equiv
-\kappa^{\ast}(\mod 1)$. Then $n+\kappa=-(n^{\ast}+\kappa^{\ast})$ for
some integer $n^{\ast}$. Therefore, if, finally, we replace
$n^{\ast}$ again by $n$, we obtain
\begin{align*}
& u(A^{-1}) (\Theta f|A^{\ast -1})(\tau,\bar{\tau})\\
& = a_0 u(y,q) + \sum^{r-1}_{\ell=0} (-1)^{r-\ell}
  \frac{\Gamma(\beta)\Gamma(\alpha)}{\Gamma(\alpha-\ell)\Gamma(\beta+\ell+1)}
  a_0 + (-1)^r b_0 + \\
& + \frac{\Gamma(\beta)}{\Gamma(\alpha} \sum_{n+\kappa^{\ast}>0}
a_{-n-\kappa^{\ast}} W (\frac{2\pi(n+\kappa^{\ast})}{N}
y;\alpha,\beta) e^{2\pi i(n+\kappa^{\ast})x/N} +\\
& + \frac{\Gamma(\alpha)}{\Gamma(\beta)} \sum_{n+\kappa^{\ast}>0}
a_{-n-\kappa^{\ast}} W (\frac{2\pi(n+\kappa^{\ast})}{N}y;\alpha,
\beta) e^{2\pi i(n+\kappa^{\ast})x/N}. \tag{14}\label{c4:eq2:14}
\end{align*}
Here $A^{\ast} = \left(\begin{smallmatrix}
  1&0\\0&-1 \end{smallmatrix}\right) A
\left(\begin{smallmatrix}
  1&0\\0&-1\end{smallmatrix}\right)^{-1}$. Since
  $\left(\begin{smallmatrix}
    1&N\\0&1 \end{smallmatrix}\right)^\ast =
  \left(\begin{smallmatrix} 1&-N\\0&1\end{smallmatrix}\right)$, it can
    be proved, using just the properties of multiplier systems, that 
$$
v^{\ast^{A^{\ast^{-1}}}} (U^N) v^{A^{-1}} (U^{N}) =1
$$
corresponding to $\kappa+\kappa^{\ast}\equiv 0(\mod 1)$. If $r\equiv
0(\mod 2)$, then applying $\Theta$ on both sides of \eqref{c4:eq2:14} and using the
fact that $\Theta^2=1$, we see immediately that 
\begin{equation*}
\sum^{r-1}_{\ell=0} (-1)^{r-\ell}
\frac{\Gamma(\alpha)\Gamma(\beta)}{\Gamma(\alpha-\ell)\Gamma(\beta+\ell+1)}
=0 \;\; (\text{for } r\equiv 0(\mod 2)), \tag{15}\label{c4:eq2:15}
\end{equation*}
which can also be proved directly.

We remark here that the space $[\Gamma, \alpha,v]$ of analytic modular
forms of real weight $\alpha$ and multiplier system $v$ is contained
in the space $[\Gamma, \alpha, 0,v]$ and it is determined in this
space by the condition $\dfrac{\partial f}{\partial \bar{\tau}}=0$.

Following a method of Siegel, we shall now prove that, under some
assumptions, the space $[\Gamma, \alpha, \beta, v]$ is a vector space
of finite dimension over the complex number field.

\setcounter{thm}{27}

\begin{thm}\label{chap4:thm28}
Let \pageoriginale $\Gamma_0$ be a subgroup of finite index in the
modular group $\Gamma$. Let $r=\alpha-\beta$ be real and
$p=\re(\alpha+\beta)\geq 0$. Then the space $[\Gamma_0, \alpha,
  \beta,v]$ is of finite dimension over the complex number field.
\end{thm}

\begin{proof}
We shall prove that theorem in a number of steps.
\begin{enumerate}
\renewcommand{\labelenumi}{\theenumi)}
\item Since $r=\alpha-\beta$ is real, we have 
$$
|(c\tau+d)^{\alpha} (c\bar{\tau}+d)^{\beta}| = |c\tau+d|^p \text{ for
  real } (c,d) \neq (0,0).
$$
For $L\in \Gamma_0$ and $A\in \Gamma$ we set 
$$
\tau^{\ast} = L<\tau>, \hat{\tau} = A<\tau>.
$$
We shall denote by $y,y^{\ast}$ and $\hat{y}$ the imaginary part of
$\tau, \tau^{\ast}$ and $\hat{\tau}$ respectively. Let $f$ be a
function in $[\Gamma_0, \alpha, \beta, v]$. Then obviously
$$
y^{\frac{p}{2}} |f(\tau,\bar{\tau})| = y^{\ast\frac{p}{2}}
|f(\tau^{\ast}, \bar{\tau}^{\ast})| = \hat{y}^{\frac{p}{2}}
|(f|A^{-1})(\hat{\tau}, \bar{\hat{\tau}})|.
$$
Let $A^{-1}_1<\infty>, \ldots, A^{-1}_{\sigma<\infty>}$ with $A_{\ell}
\in \Gamma$ for $1 \leq \ell \leq \sigma$ be a complete system
of inequivalent parabolic cusps of $\Gamma_0$ and let $\mathscr{L}_1,
\ldots, \mathscr{L}_{\sigma}$ be cusp sectors at these cusps. Then
$$
\mathfrak{F}_0 = \bigcup^{\sigma}_{n=1} \mathscr{L}_n
$$
is a fundamental domain for $\Gamma_0$. Let $N_{\ell}$ denote the
width of the cusp sector $\mathscr{L}_{\ell}$ and let $N=\max\limits_{1\leq
  \ell \leq \sigma}N_{\ell}$. Net
\begin{align*}
(f|A^{-1}_{\ell})(\tau,\bar{\tau}) &= a^{(\ell)}_0 u(y,q) +
b^{(\ell)}_0\\
&\qquad + \sum_{n+\kappa_{\ell} \neq 0}
a^{(\ell)}_{n+\kappa_{\ell}} W
(\frac{2\pi(n+\kappa_{\ell})y}{N_{\ell}}; \alpha, \beta)e^{2\pi
  i(n+\kappa_{\ell})x/N_{\ell}} 
\end{align*}
be the Fourier expansion at the cusp $A^{-1}_{\ell}<\infty>$ for any
$f$ in $[\Gamma_0,\alpha, \beta,v]$.
Let \pageoriginale us suppose that
$$
a^{(\ell)}_0 = b^{(\ell)}_0 = a^{(\ell)}_{n+\kappa_{\ell}} = 0 \text{
  for } |n+\kappa_{\ell}| \leq m, \ell = 1, 2, \ldots, \sigma.
$$
Then we shall show that $f$ vanishes identically for $m$ chosen
sufficiently large; this will establish the theorem. In the following,
$C_1,C_2,\ldots$ will denote constants which depend only on $\Gamma_0,
\alpha, \beta, v$ and not on $f$.

\item We estimate $(f|A^{-1}_{\ell})(\tau,\bar{\tau})$ in
  $A_{\ell}<\mathscr{L}_{\ell}>$. Since the Fourier expansion of
  $(f|A^{-1}_{\ell})(\tau,\bar{\tau})$ converges in the whole of
  $\mathscr{G}$, we see that 
$$
|a^{(\ell)}_{n+\kappa_{\ell}} W(\frac{2\pi
  (n+\kappa_{\ell})}{N_{\ell}} \eta; \alpha, \beta)| \leq C(\eta).
$$
for all $n,\ell$ and $\eta>0$. We normalise the modular form $f$ in
the beginning itself, so that $C(\eta)=1$ for
$\eta=\dfrac{1}{2\sqrt{3}}$. Moreover, with this $\eta$, we have
$y\geq 3\eta$ for $\tau\in A_{\ell}
<\mathscr{L}_{\ell}>$. Using the asymtotic formula for the function
$W(\in y; \alpha, \beta)$ given in lemma~\ref{chap4:lem8}, we obtain
$$
\left|
\frac{W(\frac{2\pi(n+\kappa_{\ell})}{N_{\ell}}y;\alpha,\beta)}{W
  (\frac{2\pi(n+\kappa_{\ell})}{N_{\ell}}\eta;\alpha,\beta)} \right|
\sim (\frac{\eta}{y})^{\frac{1}{2}(p-\in r)} \;\; 
e^{-2\pi|n+\kappa_{\ell}|(y-\eta)/N_{\ell}}
$$
for $|n+\kappa_{\ell}|\to \infty$, where $\in
=\sgn(n+\kappa_{\ell})$. With the special choice of
$\eta=\dfrac{1}{2\sqrt{3}}$ and for $\tau \in A_{\ell}
<\mathscr{L}_{\ell}>$, it follows now that
$$
\left| 
\frac{W(\frac{2\pi(n+\kappa_{\ell})}{N_{\ell}}y;\alpha,\beta)}{W
(\frac{2\pi(n+\kappa_{\ell})}{N_{\ell}}\eta;\alpha,\beta)} \right|
\leq C_1 e^{-2\pi|n+\kappa_{\ell}|y/(3N_{\ell})}
$$
for $n+\kappa_{\ell} \neq 0$. This implies that
\begin{align*}
|(f|A^{-1}_{\ell})(\tau,\bar{\tau})| & \leq \sum_{|n+\kappa_{\ell}|>m} \left| 
\frac{W(\frac{2\pi(n+\kappa_{\ell})}{N_{\ell}}y;\alpha,\beta)}{W
  (\frac{2\pi(n+\kappa_{\ell})}{N_{\ell}}\eta;\alpha,\beta)} \right|\\
& \leq C_1 \sum_{|n+\kappa_{\ell}|>m} e^{-2\pi
  |n+\kappa_{\ell}|y/(3N_{\ell})}\\
& \leq C_2 e^{-2\pi m y /(3N_{\ell})}.
\end{align*}\pageoriginale
In particular, we see that the function
$$
|(f|A^{-1}_{\ell})(\tau,\bar{\tau})| e^{-2\pi my/(3N_{\ell})} \to 0
\text{ as } y \to \infty,
$$
uniformly in $x$. Therefore it has in the domain $A_{\ell}
<\mathscr{L}_{\ell}>$ a non-negative maximum $M_{\ell}$ which is
attained at a finite point $\tau_{\ell}$ of this domain. Thus
$$
|(f|A^{-1}_{\ell}(\tau,\bar{\tau}))| \leq M_{\ell} e^{-2\pi
my/(3N_{\ell})} \text{ for } \tau \in A_{\ell}
<\mathscr{L}_{\ell}> 
$$
and euqality holds when $\tau =\tau_{\ell}$. Let 
$$
M_{\ell} \leq M_h \text{ for } \ell=1,2, \ldots, \sigma.
$$

\item We shall now estimate $(f|A^{-1}_h)(\tau,\bar{\tau})$ in
  $\mathscr{G}$. For a given point $\tau \in \mathscr{G}$, we
  set $\tau'=A^{-1}_h<\tau>$, $\tau''=L<\tau'>$, where $L$ belongs to
  $\Gamma_0$ and is so chosen that $\tau''$ belongs to
  $\mathfrak{F}_0$. Since $\mathfrak{F}_0 =
  \bigcup^{\sigma}_{n=1}\mathscr{L}_n$, the point $\tau''$ belongs to
  at least one of the cusp sectors $\mathscr{L}_{\ell}$, $1\leq\ell
  \leq \sigma$. Let $\tau''\in \mathscr{L}_{\ell}$; then we
  set $\tau^{\ast}=A_{\ell}<\tau''>$. We shall denote by $y,y',y''$
  and $y^{\ast}$ the imaginary parts of $\tau,\tau',\tau''$ and
  $\tau^{\ast}$ respectively. By 1), we have 

\begin{align*}
|y^{\frac{p}{2}} (f|A^{-1}_h)(\tau,\bar{\tau})| = y'^{\frac{p}{2}}
|f(\tau',\bar{\tau'})| & = y''^{\frac{p}{2}}
||f(\tau'',\bar{\tau''})|\\
& = y^{\ast\frac{p}{2}} |(f|A^{-1}_{\ell})(\tau^{\ast},\bar{\tau^{\ast}})|.
\end{align*}

Using \pageoriginale the estimates of 2), we get 
\begin{align*}
|(f|A^{-1}_h)(\tau,\bar{\tau})| & \leq y^{-\frac{p}{2}}
y^{\ast\frac{p}{2}} M_{\ell} e^{-2\pi my^{\ast}/(3N_{\ell})}\\
& \leq M_h(\frac{3pN_{\ell}}{4\pi e my})^{\frac{p}{2}} \leq M_h
(\frac{3pN}{4\pi emy})^{\frac{p}{2}},
\end{align*}
because the function $y^{p/2}e^{-2\pi my/(3N_{\ell})}$ has the
maximum $(\dfrac{3pN_{\ell}}{4\pi em})^{p/2}$ at the point
$y=\dfrac{3pN_{\ell}}{4\pi m}$. This estimate holds even when $p=0$,
in which case we assume that $p^p=1$.

\item We proceed to find a bound for
  $|(f|A^{-1}_h)(\tau_h,\bar{\tau}_h)|$ where $\tau_h$ is the point
  mentioned in 2) i.e. the point belonging to $A_h<\mathscr{L}_h>$
  such that 
$$
M_h e^{-2\pi my_h/(3N_h)} = |(f|A^{-1}_h)(\tau_h,\bar{\tau_h})| \text{
  with } \tau_h = x_h + iy_h.
$$
Let $\tau=x+\dfrac{i}{3}y_h$. Then
\begin{align*}
a^{(h)}_{n+\kappa_h} & W(\frac{2\pi(n+\kappa_h)}{N_h}
\frac{y_h}{3};\alpha,\beta)\\
& = \frac{1}{N_h} \int\limits^N_0 (f|A^{-1}_h)
(\tau,\bar{\tau})e^{-2\pi i(n+\kappa_h)x/N_{h}}dx.
\end{align*}
With the help of the preceding step 3), we see now that
$$
|a^{(h)}_{n+\kappa_h} W (\frac{2\pi
  (n+\kappa_h)}{N_h}\frac{y_h}{3};\alpha, \beta)| \leq M_h
(\frac{9pN}{4\pi emy_h})^{p/2},
$$
so that
\begin{align*}
M_h e^{-2\pi my_h/(3N_h)} & = |(f|A^{-1}_h)(\tau_h,\bar{\tau}_h)|\\
& \leq M_h(\frac{9pN}{4\pi emy_h})^{p/2} \sum_{|n+\kappa_h|>m}
\frac{W(\frac{2\pi(n+\kappa_h)}{N_h} y_h; \alpha, \beta)}{W
(\frac{2\pi(n+\kappa_h)}{3N_h}y_h;\alpha, \beta)}\\
& \leq C_3 M_h (\frac{9pN}{4\pi e my_h})^{p/2} \sum_{|n+\kappa_h|>m}
e^{-\pi|n+\kappa_h|y_h/N_h}\\
& \leq C_4 M_h (\frac{9pN}{4\pi emy_h})^{p/2} e^{-\pi my_h/N_h}
\end{align*}\pageoriginale
showing that 
$$
M_h e^{\pi my_h/(3N_h)} \leq C_4 M_h (\frac{9pN}{4\pi emy_h})^{p/2}. 
$$
Thus
$$
M_h e^{\pi m \eta/N} \leq C_5 M_h (\frac{3pN}{4\pi em\eta})^{p/2}
$$
for $p\geq 2$. Now it is obvious that $M_h=0$, if $m$ is sufficiently
large, which implies that $f=0$. Hence the theorem is proved.
\end{enumerate}
\end{proof}

\section{Eisenstein Series}\label{chap4:sec3}
We associate to two functions $f(\tau,\bar{\tau})\in \{\alpha,
\beta\}$ and $g(\tau,\bar{\tau})\in \{\alpha',\beta'\}$ a
differential form $\omega(f,g)$ defined by
$$
\omega(f,g) = y^{\gamma-1}\{f\Lambda_{\beta'} g d\bar{\tau} + g
K_{\alpha} f d\tau\},
$$
where $\alpha'$, $\beta'$, $\gamma$ are complex numbers which
independently of $\alpha, \beta$, will be so determined that
$d\omega(f,g)=0$. With the help of the differential equations
satisfied by $f$ and $g$, we see by simple calculation that 
\begin{align*}
d\omega(f,g) & = \{\frac{\partial}{\partial
  \tau}(y^{\gamma-1}f\Lambda_{\beta},g) - \frac{\partial}{\partial
  \bar{\tau}} (y^{\gamma-1} gK_{\alpha}f)\} d\tau \Lambda d
\bar{\tau}\\
& = \left\{\frac{i}{2}(\gamma-1)(\beta'-\alpha) y^{\gamma-2} f\cdot g +
(\gamma-\alpha-\alpha') y^{\gamma-1} f\frac{\partial g}{\partial
  \tau} \right. +\\
&\quad + (\gamma-\beta-\beta') y^{\gamma-1} g\frac{\partial
  f}{\partial \tau} \} d\tau \Lambda d\bar{\tau}.
\end{align*}
This \pageoriginale differential vanishes trivially, if any one of the
following two conditions is satisfied:
\begin{align*}
\alpha' & = \beta, \quad \beta' = \alpha, \quad \gamma = \alpha+
\beta, \tag{1}\label{c4:eq3:1}\\ 
\alpha' & = 1-\alpha, \quad \beta'=
1-\beta, \quad \gamma=1. \tag{2}\label{c4:eq3:2}
\end{align*}
We shall denote in the following by $\omega(f,g)|S(S\in
\Omega)$ the differential form which is obtained by substituting
$S<\tau>$ for $\tau$ in $\omega(f,g)$. We shall show that if the
numbers $\alpha, \beta$ and $\alpha', \beta'$ satisfy any one of the
conditions \eqref{c4:eq3:1} and \eqref{c4:eq3:2}, then
$$
\omega(f,g)|S = \omega(f|S, g|S).
$$
Let $S = \left(\begin{smallmatrix} a&b\\c&d \end{smallmatrix}\right)$
with $|S|=1$. We set
$$
\hat{\tau} = S<\tau> \text{ and } \hat{M} = M (S<\tau>, S <\bar{\tau}>),
$$
where $M=M(\tau,\bar{\tau})$ denotes an arbitrary function or
operator. By definition
\begin{align*}
\omega(f|S,g|S) & = y^{\gamma-1} \{(f|S)\Lambda_{\beta}, (g|S)
d\bar{\tau} + (g|S) K_{\alpha} (f|S)d \tau\}\\
& = y^{\gamma-1} \{(f|S) (\Lambda_{\beta},g)|S d\bar{\tau} +
(g|S)(K_{\alpha}f)|S d\tau\}\\
& = y^{\gamma-1}
\left\{(c\tau+d)^{-\alpha-\alpha'+1}(c\bar{\tau}+d)^{-\beta-\beta'-1}
\hat{f} \hat{\Lambda}_{\beta}, \hat{g} d\bar{\tau}+ \right.\\
& \quad \left. + (c\tau+d)^{-\alpha-\alpha'-1}
(c\bar{\tau}+d)^{-\beta-\beta'+1} \hat{g} \hat{K}_{\alpha} \hat{f}
d\tau \right\}\\
& = \hat{y}^{\gamma-1} \{\hat{f}\hat{\Lambda}_{\beta}, \hat{g}
d\bar{\hat{\tau}} + \hat{g} \hat{K}_{\alpha} \hat{f} d\hat{\tau}\}\\
& = \omega(f,g)|S.
\end{align*}
We shall say that two spaces $[\Gamma,\alpha, \beta, v]$ and
$[\Gamma,\alpha', \beta', v']$ are \textit{adjoint of the first or the
second kind} according as $\alpha,\beta$ and $\alpha',\beta'$ satisfy
condition \eqref{c4:eq3:1} or \eqref{c4:eq3:2} 
above and $v\cdot v'=1$. If $f$ and $g$ belong to
the adjoint spaces $[\Gamma, \alpha, \beta,v]$ and $[\Gamma, \alpha',
\beta', v']$ respectively, then it follows 
from \pageoriginale the transformation formula for $\omega(f,g)$ that 
$$
\omega(f,g) |S=\omega(f,g) \text{ for } S \in \Gamma.
$$

We shall now examine what the existence of an invariant differential
for  two adjoint spaces of the first kind for a subgroup $\Gamma_0$ of
finite index in the modular group means. Let $f\in [\Gamma_0,
  \alpha, \beta, v]$, $g\in[\Gamma_0, \alpha',\beta',v']$ and
let $\alpha'=\beta$, $\beta'=\alpha$, $\gamma=q$ and $v\cdot v'=1$. As
before, let $\mathscr{L}_1, \mathscr{L}_2, \ldots,
\mathscr{L}_{\sigma}$ be cusp sectors at the cusps $A^{-1}_1<\infty>$,
$A^{-1}_2<\infty>, \ldots, A^{-1}_{\sigma}$ of $\Gamma_0$, which
constitute a complete system of inequivalent parabolic cusps of
$\Gamma_0$. Then
$$
\mathfrak{F}_0 = \bigcup^{\sigma}_{\ell=1} \mathscr{L}_{\ell}
$$
is a fundamental domain for $\Gamma_0$. Let
$N_{\ell},\ell=1,2,\ldots,\sigma$ be the width of the cusp sector
$\mathscr{L}_{\ell}$. We remove the parabolic cusp from the domain
$\mathscr{L}_{\ell}$ with the help of a circular arc $c_{\ell}$, which
is mapped by the transformation $\tau\to A_{\ell}<\tau>$ onto a
segment $s_{\ell}$ of a line $y=y_{\ell}>1$, and denote the remaining
compact part by $\mathscr{L}^{\ast}_{\ell}$. Since $d\omega(f,g)=0$,
it follows that 
$$
\int\limits_{\partial \mathscr{L}^{\ast}_{\ell}} \omega(f,g) = 0,
$$
where $\partial \mathscr{L}^{\ast}_{\ell}$ is the boundary of
$\mathscr{L}^{\ast}_{\ell}$ oriented in the positive
direction. Consequently, we obtain
$$
\sum^{\sigma}_{\ell=1} \int\limits_{\partial
  \mathscr{L}^{\ast}_{\ell}} \omega(f,g)=0
$$
and here the sum of the integrals along those edges, which are
equivalent but oriented in the opposite direction, vanishes, because
$\omega(f,g)$ is invariant \pageoriginale under the transformations of
$\Gamma_0$. Using again the transformation formula for
$\omega(f,g)$, we see that
\begin{align*}
\sum^{\sigma}_{\ell=1} \int\limits_{\partial
  \mathscr{L}^{\ast}_{\ell}} = \sum^{\sigma}_{\ell=1}
\int\limits_{c_{\ell}} \omega(f,g) & = \sum^{\sigma}_{\ell=1}
\int\limits_{s_{\ell}} \omega(f,g)|A^{-1}_{\ell}\\
& = \sum^{\sigma}_{\ell=1} \int\limits_{s_{\ell}}
\omega(f|A^{-1}_{\ell}, g|A^{-1}_{\ell}) =0 \tag{3}\label{c4:eq3:3}
\end{align*}
Let 
$$
(f|A^{-1}_{\ell}) (\tau,\bar{\tau}) = \varphi_{\ell}(y) +
\sum_{n+\kappa_{\ell}\neq 0} a_{n+\kappa_{\ell}} W
(\frac{2\pi(n+\kappa_{\ell})}{N_{\ell}} y; \alpha, \beta) e^{2\pi i
  (n+\kappa_{\ell})x/N_{\ell}} 
$$
and
$$
(g|A^{-1}_{\ell})(\tau,\bar{\tau}) = \psi_{\ell}(y) +
\sum_{n+\kappa'_{\ell} \neq 0} a_{n+\kappa'_{\ell}}
W(\frac{2\pi(n+\kappa'_{\ell})}{N_{\ell}}y;\alpha, \beta) e^{2\pi
  i(n+\kappa'_{\ell})x/N_{\ell}} 
$$
be the Fourier expansions for $f(\tau,\bar{\tau})$ and
$g(\tau,\bar{\tau})$ at the cups $A^{-1}_{\ell}<\infty>$ of
$\Gamma_0$. Here
\begin{align*}
\varphi_{\ell}(y) & = a'_{\ell} u(y,q) + a''_{\ell} \text{ and}\\
\psi_{\ell}(y) = b'_{\ell} u (y,q) + b''_{\ell}.
\end{align*}
These functions vanish when $v$ and consequently $v'$ too, is ramified
at the cusp $A^{-1}_{\ell}<\infty>$. Moreover we have
$\kappa_{\ell}+\kappa'_{\ell} \equiv 0(\mod 1)$, because $v\cdot
v'=1$. By \S~\ref{chap4:sec2}, lemma~\ref{chap4:lem7}, we get 
\begin{align*}
& K_{\alpha}(f|A^{-1}_{\ell}) = (K_{\alpha}f)|A^{-1}_{\ell}\\
& = K_{\alpha}\varphi_{\ell}(y) - \sum_{n+\kappa_{\ell}>0}
  a_{n+\kappa_{\ell}} W (\frac{2\pi(n+\kappa_{\ell})}{N_{\ell}} y;
  \alpha+1, \beta-1) e^{2\pi i (n+\kappa_{\ell})x/N_{\ell}}\\
& - \alpha(\beta-1) \sum_{n+\kappa_{\ell}<0} a_{n+\kappa_{\ell}}
  W(\frac{2\pi (n+\kappa_{\ell})}{N_{\ell}} y; \alpha+1, \beta-1)
  e^{2\pi i (n+\kappa_{\ell})x/N_{\ell}}
  \end{align*}
and \pageoriginale
\begin{align*}
& (\Lambda_{\beta'} g)|A^{-1}_{\ell}= \Lambda_{\beta'}
  (g|A^{-1}_{\ell})\\
& = \Lambda_{\beta'} \psi_{\ell}(y) + (\alpha'-1) \beta'
  \sum_{n+\kappa'_{\ell}>0} b_{n+\kappa'_{\ell}} W
  (\frac{2\pi(n+\kappa'_{\ell})}{N_{\ell}}y;\\
&\qquad\alpha'-1,
  \beta'+1)e^{2\pi i(n+\kappa'_{\ell}x/N_{\ell})} +\\
& + \sum_{n+\kappa'_{\ell}<0} b_{n+\kappa'_{\ell}}
  W(\frac{2\pi(n+\kappa'_{\ell})}{N_{\ell}}y;\alpha'-1,\beta'+1)
  e^{2\pi i(n+\kappa'_{\ell})x/N_{\ell}}.
\end{align*}
Since dy vanishes on $S_{\ell}$, it is sufficient to calculate the
value of\break $\omega(f|A^{-1}_{\ell}, g|A^{-1}_{\ell})$ modulo dy. By the
definition of $\omega(f,g)$, we have 
\begin{equation*}
\omega(f|A^{-1}_{\ell}, g|A^{-1}_{\ell}) = y^{q-1} \{f|A^{-1}_{\ell}
(\Lambda_{\beta},\,g) |A^{-1}_{\ell}(K_{\alpha}f)A^{-1}_{\alpha} \} dx
(\mod dy). \tag{4}\label{c4:eq3:4}
\end{equation*}
Let $X_{\ell}(y)dx$ denote the terms independent of $x$ on the right
hand side of equation \eqref{c4:eq3:4}. Then 
{\fontsize{9}{11}\selectfont
\begin{align*}
& y^{1-q} X_{\ell}(y)\\
 & = \varphi_{\ell}(y)\Lambda_{\beta'}
\psi_{\ell}(y) K_{\alpha} \varphi_{\ell} (y) + \\
& + (\alpha'-1) \beta' \sum_{n+\kappa_{\ell}<0} a_{n+\kappa_{\ell}}
b_{-n-\kappa_{\ell}} W(\frac{2\pi(n+\kappa)_{\ell}}{N_{\ell}}
y;\alpha,\beta)
W(-\frac{2\pi(n+\kappa_{\ell})}{N_{\ell}}y;\alpha'-1,\beta'+1) +\\
& + \sum_{n+\kappa_{\ell}>0} a_{n+\kappa_{\ell}} b_{-n-\kappa_{\ell}}
W(\frac{2\pi(n+\kappa_{\ell})}{N_{\ell}}y;\alpha,\beta)
W(-\frac{2\pi(n+\kappa_{\ell})}{N_{\ell}}y;\alpha'-1, \beta'+1)-\\
& - \sum_{n+\kappa_{\ell}>0}  b_{-n-\kappa_{\ell}}
W(-\frac{2\pi(n+\kappa_{\ell})}{N_{\ell}}y; \alpha',\beta')
W(\frac{2\pi(n+\kappa_{\ell})}{N_{\ell}}y;\alpha+1, \beta-1)-\\
& -\alpha(\beta-1) \sum_{n+\kappa_{\ell}<0}
b_{-n-\kappa_{\ell}}a_{n+\kappa_{\ell}}
W(-\frac{2\pi(n+\kappa_{\ell})}{N_{\ell}}y;\alpha', \beta') W
(\frac{2\pi(n+\kappa_{\ell})}{N_{\ell}}; \alpha+1, \beta-1).
\end{align*}}\relax
Using the relations $W(-y;\alpha, \beta) = W(y;\beta,\alpha)$,
$\alpha'=\beta$ and $\beta'=\alpha$, we obtain 
$$
y^{1-q} \chi_{\ell} (y) = \varphi_{\ell}(y) \Lambda_{\beta'}
\psi_{\ell}(y) + \psi_{\ell}(y) K_{\alpha} \varphi_{\ell} (y).
$$
Since \pageoriginale the value of the integral of the terms containing
$x$ explicitly on the right hand side of \eqref{c4:eq3:3} vanishes, in view of the
integrand being then a periodic function of period $N_{\ell}$, equal
also to the length of $s_{\ell}$, it follows that 
\begin{equation*}
\sum^{\sigma}_{\ell=1} \int\limits_{s_{\ell}} \omega(f|A^{-1}_{\ell},
g|A^{-1}_{\ell}) = \sum^{\sigma}_{\ell=1} \int\limits_{s_{\ell}}
\chi_{\ell} (y) dx =0 \tag{5}\label{c4:eq3:5}
\end{equation*}
But a simple calculation shows that
$$
\chi_{\ell}(y) = b''_{\ell} a'_{\ell} - a''_{\ell} b'_{\ell}.
$$
Therefore \eqref{c4:eq3:5} implies that 
\begin{equation*}
\sum^{\sigma}_{\ell=1} N_{\ell} (b''_{\ell} a'_{\ell} - a''_{\ell}
b'_{\ell}) =0. \tag{6}\label{c4:eq3:6}
\end{equation*}
Let the notation be so chosen that the multiplier system $v$ and
therefore $v'$ be unramified at the parabolic cusps
$A^{-1}_{\ell}<\infty>(1\leq \ell \leq \sigma_0)$. Then \eqref{c4:eq3:6} leads to
the bilinear relation
\begin{gather*}
\sum^{\sigma_0}_{\ell=1} N_{\ell} (b''_{\ell}a'_{\ell} -
a''_{\ell}b'_{\ell}) =0 \\
\text{i.e. } \mathcal{G}' \mathfrak{n} \mathfrak{b}
=0 
\end{gather*}
where $\mathfrak{n}$ respectively $\mathfrak{b}$ denote the column
vector with components $a'_1, a'_2,\break\ldots, a'_{\sigma_0}$, $a''_1,
a''_2, \ldots, a''_{\sigma_0}$ respectively $b'_1, b'_2, \ldots,
b'_{\sigma_0}$, $b''_1, b''_2, \ldots, b''_{\sigma_0}$ and
$$
\mathfrak{n} = \begin{pmatrix}
0&D\\-D&0
\end{pmatrix}
$$
is a $2\sigma_0$-rowed square matrix and $D$, the $\sigma_0$-rowed
diagonal matrix with \pageoriginale $N_1, N_2, \ldots, N_{\sigma_0}$
on the diagonal. If there exist $\mu$ forms $f$ with linearly
independent vectors $\mathscr{G}$ and $\nu$ forms $g$ with linearly
independent vectors $\mathscr{b}$, then a consideration of the rank
shows that 
$$
\mu+\nu\geq 2\sigma_0.
$$
We formulate the results proved above in

\begin{thm}\label{chap4:thm29}
Let $\Gamma_0$ be a subgroup of finite index in the modular group and
let $A^{-1}_{\ell}<\infty>(\ell=1,2,\ldots,\sigma_0)$ be a complete
system of inequivalent parabolic cusps of $\Gamma_0$ at which a
multiplier system $v$ for the group $\Gamma_0$ and real weight
$\alpha-\beta$ is unramified. Let $\mathscr{R}$ (respectively
$\Gamma$) be the linear space of the vectors $\{\varphi_1(y), \ldots,
\varphi_{\sigma_0}(y)\}$ (respectively $\{\psi_1(y), \ldots,
\psi_{\sigma_0}(y)\}$), where $\varphi_{\ell}(y)$ (respectively
$\psi_{\ell}(y)$) is the term independent of $x$ in the Fourier
expansion for $f\in[\Gamma_0, \alpha, \beta, v]$ (respectively
$g\in [\Gamma_0, \beta, \alpha, v^{1}]$), at the parabolic cusp
$A^{-1}_{\ell}<\infty>$, $\ell=1,2,\ldots, \sigma_0$. Then
$$
\text{dimension } \mathscr{R} + \text{ dimension } \gamma \leq 2 \sigma_0.
$$
\end{thm}

Under the additional assumption $\re(\alpha+\beta)>2$, which enables
us to give Eisentein series as explicit examples of modular forms, we
shall prove that the spaces $\mathscr{R}$ and $\gamma$ have dimension
at least equal to $\sigma_0$, so that we have indeed the relation
$$
\text{dimension } \mathscr{R} = \text{ dimension } \gamma =\sigma_0.
$$
Let $v$ be unramified at the cusp $A^{-1}<\infty>$. Then we define the
Eisenstein series
$$
G(\tau,\bar{\tau}; \alpha, \beta, v, A, \Gamma_0) = \sum_{M\in
\gamma(A,\Gamma_0)} \{\sigma(A,L) v(L)
(m_1\tau+m_2)^{\alpha}(m_1\bar{\tau}+m_2)^{\beta}\}^{-1}
$$
where \pageoriginale $M=AL=\left(\begin{smallmatrix}
\ast&\ast\\m_1&m_2 \end{smallmatrix}\right)$ and
$\gamma(A,\Gamma_0)$ has the same meaning as in chapter~\ref{chap3}, 
\S~\ref{chap3:sec2}. As in the analytic case, the following transformation 
formulae can be proved:
\begin{align*}
&(G(\quad,; \alpha, \beta, v, A, \Gamma_0)\mathop{|}_{\alpha,\beta}S)
  (\tau,\bar{\tau})\\
&\qquad = \frac{1}{\sigma(A,S)} G(\tau,\bar{\tau};
  \alpha, \beta, v^S, A S, S^{-1}\Gamma_0S) \text{ for } S
  \in \Gamma,\\
&(G(\quad, ; \alpha, \beta, v, A, \Gamma_0) \mathop{|}_{\alpha,\beta}L)
  (\tau,\bar{\tau})\\ 
&\qquad = v(L)
  G(\tau,\bar{\tau};\alpha,\beta,v,A,\Gamma_0) \text{ for } L
  \in \Gamma_0.
\end{align*}
These transformation formulae show that the Eisenstein series is a
modular form in $[\Gamma_0,\alpha,\beta,v]$. Moreover, it is not
difficult to prove that the form $G(\tau,\bar{\tau};\alpha,
\beta,v,A,\Gamma_0)$ does not vanish at a cusp $B^{-1}<\infty>$ if and
only if $B^{-1}<\infty>$ is equivalent to $A^{-1}<\infty>$ under
$\Gamma_0$. Thus there exist as many linearly independent Eisentein
series as the number of inequivalent parabolic cusps of $\Gamma_0$ at
which $v$ is unramified. Hence our assertion about the dimensions of
$\mathscr{R}$ and $\gamma$ is proved. We call a form
$f\in[\Gamma_0, \alpha, \beta, v]$ a cusp form, when the
functions $\varphi_1(y),\ldots, \varphi_{\sigma_{0}}(y)$ mentioned in
theorem~\ref{chap4:thm29} vanish. Thus we have 

\begin{thm}\label{chap4:thm30}
Let $\Gamma_0$ be a subgroup of finite index in the modular group. If
$\Gamma=\alpha-\beta$ is real and $\re(\alpha+\beta)>2$, then for
every form $f$ belonging to $[\Gamma_0, \alpha, \beta,v]$, there
exists a linear combination $G(\tau,\bar{\tau})$ of Eisenstein series,
so that $f(\tau,\bar{\tau})-G(\tau,\bar{\tau})$ is a cusp form.
\end{thm}

Regarding the existence of cusp forms in the space $[\Gamma_0,
  \alpha,\beta, v]$, we prove the following 

\begin{thm}\label{chap4:thm31}
If, in addition to the assumptions of theorem~\ref{chap4:thm30}, we assume that
$\re\alpha>0$, $\re \beta>0$ then the space $[\Gamma_0, \alpha,
  \beta,v]$ is generated by Eisenstein series and its dimension is
equal to the number of inequivalent parabolic cusps of $\Gamma_0$ at
which the multiplier system $v$ is unramified.
\end{thm}

\begin{proof}
For \pageoriginale the proof of the theorem, it is sufficient to prove
that if $f\in [\Gamma_0, \alpha, \beta, v]$ is a cusp form,
then $f=0$. Following the proof of lemma~\ref{chap3:lem5} 
(chapter~\ref{chap3}, \S~\ref{chap3:sec3}) it can
be shown that $y^{p/2}|f(\tau,\bar{\tau})|$ is bounded in
$\mathscr{G}$, where $p=\re(\alpha+\beta)$. Let 
$$
f(\tau,\bar{\tau}) = \sum_{n+\kappa\neq 0} a_{n+\kappa} W
(\frac{2\pi(n+\kappa)}{N} y ; \alpha, \beta) e^{2\pi i(n+\kappa)x/N}
$$
be the Fourier series of $f(\tau,\bar{\tau})$ at the parabolic cusp
$\infty$ of $\Gamma_0$. Then
$$
|a_{n+\kappa} y^{p/2} W(\frac{2\pi(n+\kappa)}{N}y;\alpha,\beta)| =
|\frac{1}{N} \int\limits^N_0 y^{p/2} f(\tau,\bar{\tau}) e^{-2\pi
  i(n+\kappa)x/N} dx | \leq C
$$
with a suitable positive constant $C$. But by \S~\ref{chap4:sec2}, 
lemma~\ref{chap4:lem8},
$$y^{p/2}W(\dfrac{2\pi(n+\kappa)}{N}y;\alpha, \beta)$$ 
is unbounded as
$y\to 0$, because $p=\re q>2$; therefore, the above inequality can
hold only if $a_{n+\kappa}=0$ for $n+\kappa \neq 0$. Hence the theorem
is proved.

In some special cases, using the method adopted in the proof of
theorem~\ref{chap4:thm28}, it can be proved that the cusp forms 
identically vanish even when the assumptions of theorem~\ref{chap4:thm31} are 
not satisfied. The  following theorem is an example in this regard.
\end{proof}

\begin{thm}\label{chap4:thm32}
Let $v$ be an even abelian character of the theta group
$\Gamma_{\vartheta}$ with $v^2=1$ and let $\alpha \geq 0$. Then the
space $[\Gamma_{\vartheta},\alpha, \alpha, v]$ contains no cusp form
which does not vanish identically.
\end{thm}

First of all, we remark that there exist exactly 4 characters of
$\Gamma_{\vartheta}$ of the type mentioned in theorem~\ref{chap4:thm32}. Since
$\Gamma[2]$ is a subgroup of index 2 and
$\Gamma_{\vartheta}=\Gamma[2]\cup\Gamma[2]T$ is a coset decomposition
of $\Gamma_{\vartheta}$ modulo $\Gamma[2]$, it follows that
$v_1(T)=-1$, $v_1(S)=1$ for $S\in\Gamma[2]$ defines an
\pageoriginale even abelian character of $\Gamma_{\vartheta}$ and
$v^2_1=1$. But we have proved already that $\Gamma_{\vartheta}$ is
generated by $T$ and $U^2$; therefore, $v_1$ is uniquely defined by
$v_1(T)=-1$ and $v_1(U^2)=1$. Let $A$ denote the matrix
$\dfrac{1}{\sqrt{2}} \left(\begin{smallmatrix} -1 &3\\
-1& 1\end{smallmatrix}\right)$. Then it can be seen that the mapping
$S\to ASA^{-1}$ is an automorphism of the group $\Gamma_{\vartheta}$,
which maps $T$ to $U^2TU^{-2}$ and $U^2$ to $TU^{-2}$. This shows that
$v_2(S)=v_1(ASA^{-1})$ is another even abelian character of
$\Gamma_{\vartheta}$ such that $v^2_2=1;v_2$ is different from $v_1$
because $v_2(U^2)=-1$. A third character $v_3$ of $\Gamma_{\vartheta}$
of the same type as above is defined by $v_1v_2$ so that $v_3(T)=1$
and $v_3(U^2)=-1$. Thus we have obtained four characters namely
$1,v_1,v_2$ and $v_3=v_1v_2$ of the desired type. Since the group
$\Gamma_{\vartheta}$ is generated by $T$ and $U^2$, these are all the
even abelian characters $v$ of $\Gamma_{\vartheta}$, with $v^2=1$.

\setcounter{proofofthm}{31}
\begin{proofofthm}\label{chap4:prfthm32}
The case $\alpha>1$ follows from theorem~\ref{chap4:thm31}. If $\alpha=0$, then any
function $f\in [\Gamma_{\vartheta},0,0,v]$ is a harmonic
function. If $f\in [\Gamma_{\vartheta},0,0,v]$ is a cusp form
and does not vanish identically, then $f(\tau,\bar{\tau})$ attains its
maximum at a finite point $\tau_0$ of a fundamental domain of
$\Gamma_{\vartheta}$. But this contradicts the maximum modulus
principle for harmonic functions unless $f$ is constant in which case
it vanishes identically; therefore the theorem is proved for
$\alpha=0$.
\end{proofofthm}

In what follows, we confine ourselves to the case $0<\alpha\leq
1$. Let $\mathfrak{F}_0$ be the fundamental domain of
$\Gamma_{\vartheta}$ given by
$$
\mathfrak{F}_0 = \{\tau|\tau=x+iy, |x-1| \leq 1, |\tau| \geq 1,
|\tau-2| \geq 1, y >0\}.
$$
The domain $\mathfrak{F}_0$ is decomposed by the circle
$|\tau-1|=\sqrt{2}$ into two parts, one of which, say $\mathscr{L}_1$,
is unbounded and the other, $\mathscr{L}_2$ is bounded. Moreover, the
above-defined \pageoriginale elliptic transformation $A$ maps
$\mathscr{L}_2$ onto $\mathscr{L}_1$. We set $A_1=E$ and
$A_2=A$. Obviously
$A_1<\mathscr{L}_1>=A_2<\mathscr{L}_2>=\mathscr{L}_1$, which implies
that $y\geq 1$ for $\tau\in A_{\ell} <\mathscr{L}_{\ell}>$. 

Let $f(\tau,\bar{\tau})$ be a cusp form belonging to
$[\Gamma_{\vartheta}, \alpha, \alpha, v]$. Since the width of the cusp
sector $A_{\ell}<\mathscr{L}_{\ell}>$ is 2, we have a
Fourier expansion of the type
$$
(f|A^{-1})(\tau,\bar{\tau}) = \sum_{n+\kappa_{\ell}\neq 0}
a^{(\ell)}_{n+\kappa_{\ell}} W(\pi
(n+\kappa_{\ell})y;\alpha,\alpha)e^{\pi i(n+\kappa_{\ell})x} (\ell=1,2).
$$
Here $\kappa_{\ell}=0$ or $\dfrac{1}{2}(\ell=1,2)$, because
$v^2=1$. Now $g=y^{\alpha}|f(\tau,\bar{\tau})|$ is invariant under the
transformations of $\Gamma_{\vartheta}$ and for
$\tau'=x'+iy'=A_2<\tau>$, we obviously have 
$$
y^{\alpha} |f(\tau,\bar{\tau})| = y'^{\alpha}
|(f|A^{-1}_2)(\tau',\bar{\tau}')|. 
$$
It follows that $g$ attains its maximum $M$ at a point $\tau^{\ast}
\neq \infty,1$ of $\mathfrak{F}_0$. Let $\tau^{\ast}$ belong to
$\mathscr{L}_{\ell}$. Then we set $\tau_0=x_0+iy_0 =
A_{\ell}<\tau^{\ast}>$. It is obvious that $y_0\geq 1$. Moreover 
$$
M=y^{\alpha_0} |(f|A^{-1}_{\ell})(\tau_0,\bar{\tau})|.
$$
We shall now prove that $M=0$; this will imply that $f=0$ and prove
the theorem when $0< \alpha \leq 1$. From the integral representation
$$
W(\pm y;\alpha,\alpha) = \frac{y^{-\alpha_e-y}}{\Gamma(\alpha)}
\int\limits^{\infty}_{0} e^{-u} u^{\alpha-1}
(1+\frac{u}{2y})^{\alpha-1} du,
$$
it is evident that 
$$
y^{\alpha}W(y;\alpha,\alpha)e^y
$$
increases monotonically to 1 for $y\to \infty$, so that 
\begin{equation*}
(\frac{a}{y})^{\alpha} W (a;\alpha,\alpha) e^{a-y}\leq
  W(y;\alpha,\alpha) \leq y^{-\alpha} e^{-y} \text{ for } 0 < a \leq
  y. \tag{7}\label{c4:eq3:7} 
\end{equation*}
Since \pageoriginale $0\leq 1-\alpha<1$, it follows that
{\fontsize{10}{12}\selectfont
\begin{align*}
& (1+\frac{u}{2y})^{1-\alpha} \leq 1 + (1-\alpha) \frac{u}{2y}\\
\Longrightarrow & (1+\frac{u}{2y})^{\alpha-1} \geq
(1+(1-\alpha)\frac{u}{2y})^{-1} \geq 1 -(1-\alpha) \frac{u}{2y}\\
\Longrightarrow & \frac{1}{\Gamma(\alpha)} \int\limits^{\infty}_0
e^{-u} u^{\alpha-1} (1+\frac{u}{2y})^{\alpha-1} du \geq
\frac{1}{\Gamma(\alpha)} \int\limits^{\infty}_0 e^{-u} u^{\alpha-1}
\{1-(1-\alpha)\frac{u}{2y}\} du\\
& = \frac{1}{\Gamma(\alpha)} \{\Gamma(\alpha)-\frac{(1-\alpha)}{2y}
\Gamma(\alpha+1)\} = 1-\frac{\alpha(1-\alpha)}{2y} \geq 1 -\frac{1}{8y}.
\end{align*}}\relax
This implies that
\begin{equation*}
y^{\alpha} W(y;\alpha, \alpha) e^y \geq 1-1/(8y) \text{ for } 0 <
\alpha \leq 1 \text{ and } y >0. \tag{8}\label{c4:eq3:8}
\end{equation*}
We shall now estimate $a^{(\ell)}_{n+\kappa_{\ell}}$ with the help of 
$$
a^{(\ell)}_{n+\kappa_{\ell}} W(\pi(n+\kappa_{\ell})\rho y_0; \alpha,
\alpha) =\frac{1}{2} \int\limits^2_0
(f|A^{-1}_{\ell})(\tau,\bar{\tau}) e^{-\pi i(n+\kappa_{\ell})x}dx,
$$
where $\tau=x+iy_0\rho$ with $\rho$ an arbitrary constant belonging to
the interval $0<\rho<1$. We obtain from above that
$$
|a^{(\ell)}_{n+\kappa_{\ell}}| W(\pi(n+\kappa_{\ell})\rho
y_0;\alpha,\alpha)  \leq M(\rho y_0)^{-\alpha} (n+\kappa_{\ell} \neq
0). 
$$
Using~\eqref{c4:eq3:7}, we see that
\begin{align*}
M & = y^{\alpha}_0(f|A^{-1}_{\ell}) (\tau_0\bar{\tau}_0)|\\
& \leq \rho^{-\alpha}M\sum_{n+\kappa_{\ell}\neq 0}
\frac{W(\pi|n+\kappa_{\ell}|y_0;\alpha,
  \alpha)}{W(\pi|n+\kappa_{\ell}|\rho y_0;\alpha,\alpha)}\\
& \leq \rho^{-\alpha} M \sum_{n+\kappa_{\ell} \neq 0}
\frac{W(\pi|n+\kappa_{\ell}|y_0;\alpha,\alpha)}{W(\pi(1-\kappa_{\ell})
  \rho;\alpha,\alpha)} \frac{(\pi|n+\kappa_{\ell}|\rho y_0)^{\alpha}
  e^{\pi|n+\kappa_{\ell}|\rho
    y_0}}{(\pi(1-\kappa_{\ell})\rho)^{\alpha}e^{\pi(1-\kappa_{\ell})\rho}}   
\end{align*}
since \pageoriginale $\kappa_{\ell}=0$ or $\dfrac{1}{2}$ implies that
$|n+\kappa_{\ell}|\geq 1-\kappa_{\ell}$. On using~\eqref{c4:eq3:8}, this estimate
gives the inequality
$$
M\leq \frac{2M}{1-\frac{1}{8\pi(1-\kappa_{\ell})\rho}}
\sum_{n+\kappa_{\ell}>0} e^{-\pi(n+\kappa_{\ell})(1-\rho)}.
$$
Replacing $n+\kappa_{\ell}$ by $n+1-\kappa_{\ell}$ and summing up the
right hand side of this inequality, we see that
\begin{equation*}
M\leq \frac{2M}{1-\frac{1}{8\pi(1-\kappa_{\ell})\rho}}
\frac{e^{-\pi(1-\kappa_{\ell})(1-\rho)}}
{1-e^{-\pi(1-\rho)}}  \tag{9}\label{c4:eq3:9}
\end{equation*}
It is now obvious that $M=0$, if there exists a real number $\rho$
with $0<\rho<1$ such that
\begin{align*}
& 2<(1-\frac{1}{8\pi(1-\kappa_{\ell})\rho})
(e^{\pi(1-\kappa_{\ell})(1-\rho)} -e^{-\pi\kappa_{\ell}(1-\rho)}) \\
\text{i.e. }  & 2 < (1-\frac{1}{8\pi \rho})
(e^{\pi(1-\rho)}-1) \text{ when } \kappa_{\ell}=0 \\
\text{and } & 1 < (1-\frac{1}{4\pi\rho}) \sinh
\frac{\pi}{2}(1-\rho) \text{ when } \kappa_{\ell} =\frac{1}{2}. 
\end{align*}
For the first case, it is sufficient to take $\rho=\dfrac{1}{2}$,
because
$$
1-\frac{1}{4\pi} >\frac{11}{12} \text{ and } e^{\frac{\pi}{2}}-1>3.
$$
For the second case, let the real number $\rho$ be so determined that
$\dfrac{\pi}{2}(1-\rho)=1.185$. Then $0<\rho<1$ and
\begin{align*}
\sinh \frac{\pi}{2}(1-\rho) & = 1.482470\ldots > 1.4824\\
\frac{4\pi \rho}{4\pi \rho-1} & = \frac{4\pi-9.48}{4\pi-10.48} < 1.4794,
\end{align*}
which together imply that the second inequality is possible. Hence the
proof of theorem~\ref{chap4:thm32} is complete.

A direct \pageoriginale consequence of theorem~\ref{chap4:thm31} is 

\begin{thm}\label{chap4:thm33}
Let $\Gamma$ be the modular group and let $\alpha, \beta$ ge complex
numbers such that $r=\alpha-\beta$ is real and $\re\alpha$,
$\re\beta$, $\re(\alpha+\beta-2)>0$. Then
$$
\text{dimension } [\Gamma, \alpha, \beta,v]=\begin{cases}
1 & \text{ if } r \equiv 0 (\mod 2), v=1\\
0 & \text{ otherwise}.
\end{cases}
$$
\end{thm}

\begin{proof}
By theorem~\ref{chap4:thm31}, the dimension of $[\Gamma, \alpha, \beta, v]$ is equal
to 1 if and only if the multiplier system $v$ is unramified at
$\infty$ i.e. $\kappa=0$. Then it follows from \eqref{c3:eq1:6} of 
chapter~\ref{chap3}, \S~\ref{chap3:sec1} that $r\equiv 0(\mod 2)$ 
and $v=1$. Hence theorem~\ref{chap4:thm33} is proved.
\end{proof}

We shall now determine the Fourier coefficients of the Eisenstein
series for the modular group $\Gamma$ under the assumptions
$r=\alpha-\beta \equiv 0(\mod 2)$ and $v=1$. Instead of considering
the series $G(\tau,\bar{\tau};\alpha, \beta, 1, E, \Gamma)$, we
consider the series
$$
G(\tau,\bar{\tau}, \alpha, \beta) = \sum_{(m,n)\neq(0,0)}
(m\tau+n)^{-\alpha} (m\bar{\tau}+n)^{-\beta},
$$
because the Fourier coefficients turn out to be simple in this
case. It is obvious that $G(\tau,\bar{\tau};\alpha,\beta)$ defines a
modular form in $[\Gamma,\alpha,\beta,1]$ in case
$p=\re(\alpha+\beta)>2$. First of all, we find the Fourier series of
the periodic function
$$
f(\tau,\bar{\tau};\alpha,\beta) = \sum^{\infty}_{n=-\infty}
(\tau+n)^{-\alpha} (\bar{\tau}+n)^{-\beta} (\rho>2)
$$
defined for $\tau\in \mathscr{G}$. Let
$$
f(\tau,\bar{\tau};\alpha,\beta) =e^{\pi i(\beta-\alpha)/2}
\sum^{\infty}_{n=-\infty} h_n(y;\alpha,\beta) e^{2\pi inx}
$$
with \pageoriginale
\begin{align*}
h_n(y;\alpha,\beta) & = e^{\pi i(\alpha-\beta)/2}
\int\limits^{\infty}_{-\infty} \tau^{-\alpha} {\bar{\tau}}^{-\beta}
e^{-2\pi inx}dx\\
& = \int\limits^{\infty}_{-\infty} (-i\tau)^{-\alpha}
(i\bar{\tau})^{-\beta} e^{-2\pi i nx}dx\\
& = y^{1-\alpha-\beta} \int\limits^{\infty}_{-\infty} (1-ix)^{-\alpha}
(1+ix)^{-\beta} e^{-2\pi i n yx}dx.
\end{align*}
Here the integrand is defined by
$$
(1-ix)^{-\alpha} = e^{-\alpha\log(1-ix)}, (1+ix)^{-\beta} =
e^{-\beta\log(1+ix)}, 
$$
where the branch of the logarithm is chosen in such a way that log
$(1\pm ix)$ is real for $x=0$. In order to express the function
$$
h(t;\alpha,\beta) = \int\limits^{\infty}_{-\infty} (1-ix)^{-\alpha}
(1+ix)^{-\beta} e^{-itx} dx \;\; (\re (\alpha+\beta)>1)
$$
in terms of Whittaker's function, we consider the gamma integrals 
\begin{align*}
(1-ix)^{-\alpha}\Gamma(\alpha) & = \int\limits^{\infty}_0
  e^{-(1-ix)\xi} \xi^{\alpha-1} d\xi \;\; (\re \alpha >0)\\
(1+ix)^{-\beta} \Gamma(\beta) & = \int\limits^{\infty}_0
  e^{-(1+ix)\eta} \eta^{\beta-1} d\eta \;\; (\re \beta >0)
\end{align*}
which imply that
$$
(1-ix)^{-\alpha} (1+ix)^{-\beta} \Gamma(\alpha)\Gamma(\beta) = 
\int\limits^{\infty}_{0} \int\limits^{\infty}_{0}
e^{-(1-ix)\xi-(1+ix)\eta} \xi^{\alpha-1} \eta^{\beta-1} d\xi d\eta.
$$
The substitution $\xi+\eta=u$, $\xi-\eta=t$ leads us to 
\begin{align*} 
&2^{q-1} \Gamma(\alpha) \Gamma(\beta) (1-ix)^{-\alpha} (1+ix)^{-\beta}\\
&\qquad = \int\limits^{\infty}_{-\infty} e^{itx} \left\{\int\limits_{u>|t|} e^{-u}
(u+t)^{\alpha-1} (u-t)^{\alpha-1} (u-t)^{\beta-1}du \right\} dt,
\end{align*}
with $q=\alpha + \beta$. This shows that the function
$(1-ix)^{-\alpha}(1+ix)^{-\beta}$ is the Fourier \pageoriginale
transform of the $u$-integral. Let $\re(\alpha+\beta)>1$. Then the
inverse of the above Fourier transform exists and we get
$$
\frac{\Gamma(\alpha)\Gamma(\beta)2^{q-1}}{2\pi}
\int\limits^{\infty}_{-\infty} (1-ix)^{-\alpha} (1+ix)^{-\beta} 
e^{-itx} dx = \int\limits_{u>|t|} e^{-u} (u+t)^{\alpha-1}
(u-t)^{\beta-1} du.
$$
Consequently, we obtain
$$
h(t;\alpha,\beta) =\frac{2\pi}{\Gamma(\alpha)\Gamma(\beta)} 2^{1-q}
\int\limits_{u>|t|} e^{-u} (u+t)^{\alpha-1}(u-t)^{\beta-1} du.
$$
Making use of the integral representation of Whittaker's function
given in the proof of lemma~\ref{chap4:lem8}, we see that 
$$
h(t;\alpha,\beta) = \frac{2\pi}{\Gamma(\frac{q+\in
    \Gamma}{2})} 2^{-\frac{1}{2}q} |t|^{q-1} W(t;\alpha,\beta) \;\; 
(\in = \sgn t)
$$
and in particular,
$$
h(0;\alpha,\beta) = \frac{2\pi
  \Gamma(q-1)}{\Gamma(\alpha)\Gamma(\beta)} 2^{1-q}.
$$
By the definition of the function $h(t;\alpha,\beta)$, the Fourier
coefficients of $f(\tau,\bar{\tau};\alpha,\beta)$ are given by
\begin{equation*}
h_n(y;\alpha,\beta) = y^{1-q} h(2\pi n y; 
\alpha, \beta). \tag{10}\label{c4:eq3:10} 
\end{equation*}
We shall now determine the Fourier expansion of the Eisenstein series
$G(\tau,\bar{\tau}; \alpha,\beta)$. Here we shall make use of the
assumption $r=\alpha-\beta\equiv 0(\mod 2)$. Let us set $r=2k$ ($k$
integral). Then
\begin{align*}
G(\tau,\bar{\tau};\alpha,\beta) & = 2 \sum^{\infty}_{n=1}
n^{-\alpha-\beta} + 2 \sum^{\infty}_{m=1} \sum^{\infty}_{n=-\infty}
(m\tau+n)^{-\alpha} (m\bar{\tau}+n)^{-\beta}\\
& = 2\zeta(q) + 2 \sum^{\infty}_{m=1} f(m\tau,m\bar{\tau};\alpha,\beta),
\end{align*}
because, \pageoriginale for $m>0$, $m\tau$ belongs to $\mathscr{G}$
with $\tau$. Thus
$$
G(\tau,\bar{\tau};\alpha,\beta) = 2\zeta (q) + 2\sum^{\infty}_{m=1}
\sum^{\infty}_{n=-\infty} h_n(my;\alpha,\beta) e^{2\pi imnx}.
$$
But from \eqref{c4:eq3:10}, it is obvious that 
\begin{align*}
h_n(my;\alpha,\beta) & = (my)^{1-q} h(2\pi n my;\alpha,\beta)\\
& = m^{1-q} y^{1-q} h(2\pi n my;\alpha,\beta)\\
& = m^{1-q} h_{mn} (y;\alpha,\beta);
\end{align*}
therefore
$$
G(\tau,\bar{\tau};\alpha,\beta) = 2 \zeta(q) + 2(-1)^k
\sum^{\infty}_{m=1} m^{1-q} \sum^{\infty}_{n=-\infty} h_{mn}
(y;\alpha,\beta) e^{2\pi i mnx}.
$$
Collecting the terms for which $mn=\ell$, we get 
\begin{align*}
G(\tau,\bar{\tau};\alpha,\beta) & = 2\zeta(q) + 2(-1)^k
\sum^{\infty}_{\ell=-\infty} \{\sum_{\substack{d|\ell\\d>0}} d^{1-q}\}
h_{\ell}(y;\alpha,\beta) e^{2\pi i \ell x}\\
& = \varphi_{k}(y,q) + 2(-1)^k (\sqrt{2}\pi)^q\\ 
&\sum_{n\neq 0}
\frac{d_{q-1}(n)}{\Gamma(\frac{q}{2}+\in k)} W(2\pi
ny;\alpha,\beta) e^{2\pi i nx}\\
\text{with } \in & = \sgn n, d_{q-1}(n) =
\sum_{\substack{d|n\\d>0}} d^{q-1} \text{ and } \\
\varphi_k(y,q) & = 2\zeta (q) + 2(-1)^k \zeta(q-1) h_0
(y;\alpha,\beta)\\
& = 2\zeta(q) + (-1)^k2^{3-q} \pi
\frac{\Gamma(q-1)\zeta(q-1)}{\Gamma(\frac{q}{2}+k)\Gamma(\frac{q}{2}-k)}
\{(1-q) u (y,q)+1\}\\
& = 2\zeta (q) + (-1)^k 2^{3-q}
\pi\frac{\Gamma(q-1)\zeta(q-1)}{\Gamma(\frac{q}{2}+k)
  \Gamma(\frac{q}{2}-k)} +\\
& \qquad (-1)^k 2^{3-q} \pi
\frac{\Gamma(q-1)\zeta(q-1)}{\Gamma(\frac{q}{2}+k)
  \Gamma(\frac{q}{2}-k)} (1-q)u(y,q).
\end{align*}
The \pageoriginale analytic continuation of the function
$G(\tau,\bar{\tau};\alpha,\beta)-\varphi_k (y,q)$, which is so far
defined for $\re q>2$, in the whole of the $q$-plane (for a fixed
integer $k$) is obvious from the series as well as the estimate 
$$
W(\pm y;\alpha,\beta) \leq C e^{-(1-\in)y} \text{ for } y \geq
y_0 >0, |\alpha| \leq m, |\beta| \leq m
$$
with a positive constant $C=C(y_0,\in, m)$, where
$\in, y_0$ and $m$ are given positive numbers. In order to
obtain this estimate for the function\break $W(y;\alpha,\beta)$, we consider
the well-known integral representation for\break $W_{k,m}(y)$ (see Whittaker
and Watson: A Course on Modern Analysis) of which the following two
integrals are an immediate consequence:
\begin{align*}
W(y;\alpha,\beta) & = \frac{2^{r/2}y^{-\beta} e^{-y}}{\Gamma(\beta)}
\int\limits^{\infty}_0 u^{\beta-1} (1+\frac{u}{2y})^{\alpha-1} e^{-u}
 du \text{ for } \re \beta >0,\\
W(y;\alpha,\beta) & =
\frac{-2^{r/2}\Gamma(1-\beta)e^{-y}y^{\beta}}{2\pi i}
\int\limits^{(0+)}_{+\infty} (-u)^{\beta-1}
(1+\frac{u}{2y})^{\alpha-1} e^{-u} du\\
&\qquad \text{ for } \Gamma(1-\beta)
\neq \infty.
\end{align*}
In the second integral, the path of integration is a loop, which
starts from $\infty$, circles round the point 0 in the positive
direction so that the points 0 and $-2y$ are separated and then goes
over again to $\infty$. The integrand is uniquely defined by the
requirements $|\arg(-u)|\leq \pi$, in case $u$ is not $\geq 0$ and
$|\arg(1+\dfrac{u}{2y})|< \pi$ in case $(1+\dfrac{u}{2y})$ is not
$\leq 0$. The estimate for the function $W(-y;\alpha, \beta)$ follows
from that of $W(y;\alpha,\beta)$ because of the identity
$$
W(-y;\alpha,\beta) =W(y;\alpha,\alpha).
$$

For the analytic continuation of the function $\varphi_{k}(y,q)$ in
the complex $q$-plane, we shall make use of the well-known facts that
the functions \pageoriginale
$\xi(s)=s(l-s)\pi^{-s/2}\Gamma(\dfrac{s}{2}) \zeta(s)$ is an entire function
of $s$, satisfies the functional equation $\xi(1-s)=\xi(s)$ and for
$0\leq \re s\leq 1$ has the same zeros as the zeta function $\zeta(s)$
and does not vanish outside this strip. Further, we need the identity
$$
\Gamma(s) = \frac{1}{\surd\pi} 2^{s-1} \Gamma(\frac{s}{2})
\Gamma(\frac{s+1}{2}) 
$$
for the $\Gamma$-function.

By the definition of the function $\varphi_k(y,q)$, we have
\begin{align*}
\varphi_k(y,q) & = \frac{\pi^{q/2}}{(1-q)\Gamma(\frac{q}{2}+1)}
\xi(1-q) + \frac{(-1)^k
  \pi^{q/2}\Gamma(\frac{q}{2}-1)}{(1-q)\Gamma(\frac{q}{2}+k)
  \Gamma(\frac{q}{2}-k)} \xi(q-1) +\\
& \qquad +\frac{(-1)^k \pi^{q/2}
  \Gamma(\frac{q}{2}-1)}{\Gamma(\frac{q}{2}-k) \Gamma(\frac{q}{2}+k)}
\xi (q-1) u(y,q).
\end{align*}
But obviously
$$
\frac{\Gamma(\frac{2-q}{2}+|k|)}{\Gamma(2-\frac{q}{2})} = (-1)^{k+1}
\frac{\Gamma(\frac{q}{2}-1)}{\Gamma(\frac{q}{2}-|k|)}; 
$$
therefore
{\fontsize{10}{12}\selectfont
\begin{align*}
\varphi_k(y,q) & = \frac{\pi^{\frac{q}{2}}}{\Gamma(\frac{q}{2}+|k|)}
\frac{1}{(1-q)}
\left\{\frac{\Gamma(\frac{q}{2}+|k|)}{\Gamma(\frac{q}{2}+1)} \xi (1-q)
-\frac{\Gamma(\frac{2-q}{2}+|k|)}{\Gamma(2-\frac{q}{2})}
\xi(q-1)\right\} +\\
& \qquad + \frac{(-1)^k \pi^{q/2}
  \Gamma(\frac{q}{2}-1)}{\Gamma(\frac{q}{2}+k) \Gamma(\frac{q}{2}-k)}
\xi(q-1)u(y,q).
\end{align*}}\relax
The expression in the brackets is an odd function of $q-1$ and
therefore has at $q=1$ a zero of order at least equal to 1.

We introduce \pageoriginale
\begin{align*}
G^{\ast} (\tau,\bar{\tau};\alpha,\beta) & = \frac{q}{2}
(1-\frac{q}{2}) \pi^{-q/2} \Gamma(\frac{q}{2}+|k|)
G(\tau,\bar{\tau};\alpha,\beta),\\
\varphi^{\ast}_k(y,q) & = \frac{q}{2} (1-\frac{q}{2}) \pi^{-q/2}
\Gamma(\frac{q}{2}+|k|) \varphi_k(y,q)\\
& = \frac{1}{1-q} \left\{(1-\frac{q}{2})
\frac{\Gamma(\frac{q}{2}+|k|)}{\Gamma(\frac{q}{2})} \xi(1-q)\right.\\
&\qquad\qquad\left.-\frac{q}{2} \frac{\Gamma(1-\frac{q}{2}+|k|)}{\Gamma(1-\frac{q}{2})}
\xi (q-1)\right\}+ \\
& \qquad + (-1)^{k+1} \frac{q}{2}
\frac{\Gamma(\frac{q}{2})}{\Gamma(\frac{q}{2} -|k|)} \xi (q-1) u(y,q).
\end{align*}
It is obvious that $\varphi^{\ast}_k(y,q)$ as well as 
\begin{align*}
& G^{\ast} (\tau, \bar{\tau}; \alpha, \beta) -\varphi^{\ast}_k(y,q)\\
& = 2(-1)^k (2\pi)^{q/2} \frac{q}{2} (1-\frac{q}{2}) \sum_{n\neq 0}
  \frac{\Gamma(\frac{q}{2}+|k|)}{\Gamma(\frac{q}{2}+\in k)}
  d_{q-1} (n) W(2\pi n y;\alpha, \beta)e^{2\pi i n x}
\end{align*}
with $\in=\sgn n$, for the given integer
$k=\dfrac{1}{2}(\alpha-\beta)$, is an entire function of $q$. The
transformation formula of $G(\tau,\bar{\tau}; \alpha,\beta)$ and
therefore of the function $G^{\ast}(\tau,\bar{\tau};\alpha,\beta)$ was
proved for $\re q>2$ but remains valid by analytic continuation; thus 
$$
G^{\ast}(\tau\bar{\tau};\alpha,\beta) \in [\Gamma,\alpha,\beta,1].
$$
With the help of 
$$
\zeta(0) = -\frac{1}{2}, \zeta'(0) = -\frac{1}{2} \log 2\pi, \Gamma'(1)
=-\gamma (\gamma=\text{Euler's constant})
$$
we obtain, by simple calculation, that
$$
\xi(0) = -1, \quad \xi'(0) = 1 + \frac{1}{2} (\gamma-\log 4\pi)
$$
and finally, for some special values of $k$ and $q$,
$$
\varphi^{\ast}_k(y,q) = \begin{cases}
-1 & \text{ for } k=0, q=0\\
-\frac{1}{y} & \text{ for } k=0, q=2\\
\frac{\Gamma(k+\frac{1}{2})}{\Gamma(\frac{1}{2})} \{\sum^{k}_{h=1}
\frac{1}{2h-1} + \frac{1}{2} (\gamma+\log\frac{y}{4\pi})\} & \text{ for
} k \geqq 0, q=1.
\end{cases}
$$\pageoriginale
Corresponding to $k=0$, $q=0,2,1$ we get the following Eisenstein
series
\begin{gather*}
G^{\ast} (\tau,\bar{\tau};0,0) =-1, G^{\ast} (\tau,\bar{\tau};1,1)
=-\frac{1}{y},\\
G^{\ast} (\tau,\bar{\tau};\frac{1}{2},\frac{1}{2}) =\frac{1}{2}(\gamma
+ \log \frac{y}{4m}) + \sum_{n\neq 0} d_0(n) K_0(2\pi |n|y)e^{2\pi inx}
\end{gather*}
where 
$$
K_0(y) =\sqrt{\frac{\pi}{2}} W(\pm y;\frac{1}{2},\frac{1}{2})
$$
is the well-known Bessel function of pure imaginary argument. Since
$\xi(q)$ vanishes only in the critical strip $0<\re q<1$ of Riemann's
zeta function $\zeta(q)$, it is easy to see that
$\varphi^{\ast}_k(y,q)$ for given $q$ and $k$ (integral) does not
represent a cusp form for any choice of $\alpha,\beta$ provided
$k=\dfrac{1}{2}(\alpha-\beta)$ is integral.

Since $y^{\dfrac{1}{2}q}W(\in y;\alpha,\beta)$ with
$y>0,\in^2=1$, according to the definition of this function,
depends only on $y,r\in$ and $(q-1)^2$, it is not hard to
prove that $G^{\ast}$ satisfies the following functional equations 
\begin{align*}
G^{\ast} (\tau,\bar{\tau;\beta,\alpha}) & = G^{\ast} (-\bar{\tau},
-\tau;\alpha, \beta),\\
G^{\ast} (\tau,\bar{\tau}; 1-\alpha, 1-\beta) & = y^{q-1} G^{\ast}
(\tau,\bar{\tau};\beta, \alpha).
\end{align*}

The following is an immediate consequence of the results proved above.

\begin{thm}\label{chap4:thm34}
The linear \pageoriginale space $[\Gamma, \alpha,\alpha,1]$, where
$\Gamma$ is the modular\break group, has dimension 1 over the complex number
field in case $\alpha\geq 0$ and in that case it is generated by
$G^{\ast}(\tau,\bar{\tau};\alpha,\alpha)$.
\end{thm}

\begin{proof}
By theorem~\ref{chap4:thm32}, there exist no cusp forms in $[\Gamma,\alpha,\alpha,1]$
which do not vanish identically. Thus by theorem~\ref{chap4:thm29}, the space
$[\Gamma,\alpha,\alpha,1]$ is isomorphic with the space
$\mathscr{R}=\gamma$ and therefore dimension
$[\Gamma,\alpha,\alpha,1]=$ dimension $\mathscr{R}\leq 1$. But, for
$\alpha \geq 0$, there exists a form in $[\Gamma, \alpha, \alpha, 1]$,
which does not vanish identically, namely, the Eisenstein
series. Hence the theorem is proved.
\end{proof}

\begin{thebibliography}{99}
\bibitem{c4:key1} H. Maass'' \"Uber eine neue Art von nichtanalytischen
  automorphen Funktionen und die Bestimmung Dirichletscher Reihen
  durch Funktionalgleichungen, Math. Ann., 121 (1949), 141-183.

\bibitem{c4:key2} H. Maass: Automorphe Funktionen und indefinite
  quadratische Formen, Sitzungsber. Heidelberger
  Akad. Wiss. Math.-naturwiss. K1., 1949, 1-42.

\bibitem{c4:key3} H. Maass: Die Differentialgleichungen in der Theorie
  der elliptischen Modulfunktionen, Math. Ann., 125 (1953), 235-263.

\bibitem{c4:key4} W. Roelcke: \"Uber die Wellengleichung bei
  Grenzkreisgruppen erster Art, Sitzungsber. Heidelberger
  Akad. Wiss. Math-naturwiss. Kl., 1953/1955, 4 Abh., 161-267.

\bibitem{c4:key5} A. Selberg: Harmonic analysis and discontinuous groups
  in weakly symmetric Riemannian spaces with applications to Dirichlet
  series, Report of an International Colloquium on Zeta Functions,
  Bombay. 1956.
\end{thebibliography}


