\chapter[Pierce Decompositions. Standard Quadratic...]{Pierce
  Decompositions. Standard Quadratic Jordan Matrix 
  Algebras}\label{c2}

In\pageoriginale this chapter we shall develop two of the main tools for the
structure theory: Peirce decomposition and the strong coordinatization
Theorem. The corresponding discussion in the linear case is given in
the author's book [2], Chapter III.

\section{Idempotents. Pierce decompositions}\label{c2:sec1}

An element $e$ of $\mathscr{J}=(\mathscr{J},U,1)$ is {\em idempotent}
if $e^{2}=e$. Then $e^{3}=eU_e=e^{2}U_e =(e^{2})^{2}(QJ
23)=e^{2}=e$. Then $e^{n}=e^{n-2}U_e=e$ for $n\geqq 1$, by
induction. Also $U^{n}_e=U_{e^{n}}=U_e$. The idempotents $e$ and $f$
are said to be {\em orthogonal} $(e\bot f)$ if $e\circ
f=fU_e=eU_f=0$. If we apply $QJ12$ with $a=e$, $c=f$ to $1$ we see
that $e\circ f=0$ and $fU_e=0$ imply $eU_f=0$. Hence $e$ and $f$ are
orthogonal if $e\circ f=0$ and either $eU_f=0$ ar $fU_e=0$. If $e$ and
$f$ are orthogonal then $e+f$ is idempotent since
$(e+f)^{2}=e^{2}+e\circ f+f^{2}=e+f$. If $e,f,g$ are orthogonal
idempotents (that is, $e\bot f,g$ and $f\bot g$) then $(e+f)\circ
g=(e+f)U_g=0$ so $(e+f)\bot g$. It follows that if $e,f,g,h$ are
orthogonal idempotents then $e+f$ and $g+h$ are orthogonal
idempotents. If $e$ is idempotent then $f=1-e$ is idempotent since
$f^{2}=(1-e)^{2}=1=-e\circ 1+e=1-e$. Also $e\circ f=e\circ 1-e\circ
e=2e-2e=0$ and $fU_{\circ}=(1-e)U_e=e-e=0$ so $e$ and $f$ are
orthogonal.

We\pageoriginale recall that an endomorphism $E$ of a module is called
a projection if $E$ is idempotent: $E^{2}=E$, and the projections $E$
and $F$ are orthogonal if $EF=0=FE$. If $E$ is a projection and $X$ is
an endomorphism satisfying the Jordan conditions: $EXE=EX+XE=0$ then
it clear that $EX=0=XE$. We now prove

\begin{lemma}\label{c2:lem1}
  If $e$ and $f$ are orthogonal idempotents in $\mathscr{J}$ then $U_e$,
  $U_f$ and $U_{e,f}$ are orthogonal projections. If $e,f,g$ are
  orthogonal idempotents then the projections $U_e$, $U_{e,f}$ and
  $U_{f,g}$ are orthogonal. If $e,f,g,h$ are orthogonal idempotents then
  $U_{e,f}$ and $U_{g,h}$ are orthogonal.
\end{lemma}

\begin{proof}
We have seen that $U^{2}_e=U_{e^{2}}$, so $U^{2}_e=U_e$ is a
projection. We have $U_eU_{e,f}U_e=U_{e^{2}.eof}(QJ11)=0$ and
$U_eU_{e,f}U_e=U_{eU_{e},fU_{e}}(QJ3)=0$. Hence
\begin{equation*}
  U_eU_{e,f}=0=U_{e,f}U_{e}\tag{1}\label{c2:eq1}
\end{equation*}

Also
$U_eU_f+U_fU_e=U^{2}_eU_f+U_fU_{e^{2}}=-U_{e,f}U_eU_{e,f}+U_{eU_{e},eU_{f}}+U_{eU_{e,f}}(QJ7)=0$
by $(1), eU_f=0$ and $eU_{e,f}=eef=e^{2}\circ f=0$. Since
$U_eU_fU_e=U_{fU_{e}}=0$ we have
\begin{equation*}
  U_eU_f=0=U_fU_e.\tag{2}\label{c2:eq2}
\end{equation*}

Now $e+f$ is idempotent so $U_{e+f}=U_c+U_{e,f}+U_f$ is idempotent. By
\eqref{c2:eq1} and \eqref{c2:eq2} this gives
$U_e+U^{2}_{e,f}+U_f=U_e+U_{e,f}+U_f$. Hence 
\begin{equation*}
  U^{2}_{e,f}=U_{e,f}\tag{3}\label{c2:eq3}
\end{equation*}

Since\pageoriginale $e\bot f,g, e\bot f+g$, so $U_e
U_{f+g}=0=U_{f+g}U_e$. By \eqref{c2:eq2} this gives
\begin{equation*}
  U_eU_{f,g}=0=U_{f,g}U_e.\tag{4}\label{c2:eq4}
\end{equation*}
we have $U_{e+f}U_{f,g}U_{e+f}=U_{fU_{e+f}},gU_{e+f}=0$ since $g$ and
$e+f$ are orthogonal. Since $U_{e+f}=U_e+U_f+U_{e,f}$ this, \eqref{c2:eq1} and
\eqref{c2:eq4} gives $U_{e,f}U_{f,g}$ $U_{e,f}=0$. Also
$U_{e,f}U_{f,g}+U_{f,g}U_{e,f}+U_fU_{e,g}+U_{e,g}U_f=U_{e\circ f
  fog}+U_{f^{2},eog}$. (taking $b=1, a=f, c=e, d=g$ in
$QJ8$)$=0$. Hence
\begin{equation*}
  U_{e,f}U_{f,g}=0=U_{f,g}U_{e,f}.\tag{5}\label{c2:eq5}
\end{equation*}

Finally, $e\bot g,h$ and $f\bot g,h$ so $e+f\bot g,h$. Hence, by
\eqref{c2:eq4},
$U_{e,f}U_{g,h}=(U_{e+f}-U_e-U_f)U_{g,h}=0$ and $U_{g,h}U_{e,f}=0$.

A set of orthogonal idempotents $\{e_i|i=1,2,\ldots,n\}$ will be
called {\em supplementery} if $\sum e_i=1$. Then this gives
\begin{equation*}
  1=U_1=\sum\limits_{1}^{n}U_{e_{i}} +
  \sum\limits_{1<j}U_{e_{i},e_{j}}. \tag{6}\label{c2:eq6} 
\end{equation*}

The foregoing lemma shown that the $n(n+1)/2$ operators
$U_{e_{i}},U_{e_{i},e_{j}}$ with distinct subscripts are orthogonal
projections. Since they are supplementary in End $\mathscr{J}$ in the
sense that their sum is $1$ we have
\begin{equation*}
  \mathscr{J}=\sum\limits_{i\le j}\bigoplus
  \mathscr{J}_{ij}, \mathscr{J}_{ii} = \mathscr{J} U_{e_{i}},
  \mathscr{J}_{ij} = \mathscr{J} U_{e_{i},e_{j}},i<j\tag{7}\label{c2:eq7}
\end{equation*}
which we call the {\em pierce decomposition} of $\mathscr{J}$ {\em
  relative to the} $e_i$. We shall call $\mathscr{J}_{ij}$ the {\em
  pierce} $(i,j)$-{\em component} of $\mathscr{J}$ relative to the
$e_i$. $\mathscr{J}_{ii}=U_{e_{i}}$\pageoriginale is an inner ideal
called the {\em 
  Pierce inner ideal determined by the idempotent} $e_i$.

We shall now derive a list of formulas for the products
$a_{ij}U_{b_{kl}}$ where $a_{ij}\epsilon
\mathscr{J}_{ij},b_{kl}\epsilon \mathscr{J}_{kl}$. For this purpose we require
\end{proof}

\begin{lemma}\label{c2:lem2}
If $a_{ij}\epsilon \mathscr{J}_{ij}$ then
\begin{gather*}
  U_{e_{i}}U_{a_{ii}}U_{e_{i}}=U_{a_{ii}}\tag{8}\label{c2:eq8}\\
  U_{a_{ij}}=U_{e_{i}}U_{a_{ij}}U_{e_{j}}+U_{e_{j}} U_{a_{ij}}
  U_{e_{i}} + U_{e_{i}e_{j}},U_{a_{ij}}U_{e_{i},e_{j}},i\neq
  j\tag{9}\label{c2:eq9}\\
  V_{a_{ii}}=U_{e_{i}}U_{a_{ii},e_{i}}U_{e_{i}}+\sum\limits_{j\neq
    i}(U_{e_{i}}V_{aii}U_{e_{j}}+U_{e_{j}}V_{a_{ii}}U_{e_{i}}+\\
  U_{e_{i},e_{j}}V_{a_{ii}}U_{e_{i},e_{j}})\tag{10}\label{c2:eq10}\\
  U_{a_{ii},c_{ij}}=U_{e_{i}}U_{a_{ii},c_{ij}}U_{e_{i},e_{j}} +
  U_{e_{i},e_{j}} U_{a_{ii}c_{ij}}U_{e_{i}}i\neq j\tag{11}\label{c2:eq11}
\end{gather*}
\end{lemma}

\begin{proof}
The first is clear from $QJ3$ and $a_{ij}U_{e_{i}}=a_{ii}$. The second
follows by taking $a=e_{i},b=a_{ij},c=e_j$ in $QJ7$. For (1) we have
$V_{a_{ii}}=U_{1,a_{ii}}=U_{e_{i},a_{ii}}+\sum\limits_{j\neq
  i}U_{e_{j}a_{ii}}$. Then
$U_{e_{i},a_{ii}}=U_{e_{i}}U_{e_{i},a_{ii}}U_{e_{i}}$ by $QJ3$ and
  $U_{e_{j},a_{ii}}=U_{e_{i}}V_{a_{ii}}U_{e_{j}}+U_{e_{j}}V_{a_{ii}}U_{e_{i}}+U_{e_{i},e_{j}}V_{a_{ii}}U_{e_{i},e_{j}}$
  follows by putting $a=e_i, b=a_{ii},c=e_{j}$ in $QJ15$. Hence (10)
  holds. To obtain (11) we belinearrize $QJ6$ relative to $b$ to
  obtain $U_a
  U_{b.d}U_{a,c}+U_{a,c}U_{b,d}U_a=U_{bU_{a},du_{a,c}}+U_{dU_{a},b}U_{a,c}$
  and put $a=e_i, b=a_{ij},c=e_{j}, d=c_{ij}$ in this.

To\pageoriginale formulate the results on the products
$a_{ij}U_{b_{kl}}$ of elements in Pierce components in a compact from
we consider triples of unordered pairs of induced taken from
$\{1,2,\ldots,n\}:(pq,rs,uv)$. In any pair $pq$ we allow $p=q$ and we
assume $pq=qp$. Also we identify $(pq, rs, uv)=(uv, rs, pq)$. We shall
call such a triple {\em  connected} if it can be written as 
$$
(pq, qr,rs)
$$
It is easily seen that the only triples which are not connected are
those of one of the following two forms:
\begin{align*}
&(pq,rs, -)\quad\text{with}\quad\{p,q\}\cap\{r,s\}= \\
&(pq,qr,qs)\quad\text{with}\quad r\neq p,q,s.
\end{align*}
We can now state the important 
\end{proof}

\noindent
{\textbf{Pierce decomposition theorm}}. Let $\{e_i|i=1,2,\ldots,n\}$
be a supplementary set of orthogonal idempotents,
$\mathscr{J}=\sum\mathscr{J}_{ij}$ the corresponding Pierce
decomposition of $\mathscr{J}$. Let $a_{pq}\epsilon \mathscr{J}_{pq}$
etc. Then for any connected triple $(pq, qr, rs)$ we have
\begin{align*}
&\{a_{pq}b_{qr}c_{rs}\}\epsilon \mathscr{J}_{ps}\tag*{PD 1}\\
b_{qr}&U_{a_{pq}}=a_{pq}b_{qr}a_{pq}\epsilon
\mathscr{J}_{PS}\quad\text{if}\quad pq=rs\tag*{PD 2}
\end{align*}
If $(pq, rs, uv)$ is not connected then
\begin{align*}
\{a_{pq}b_{rs}c_{uv}\}&=0\quad\text{and}\quad
b_{rs}U_{a_{pq}}=a_{pq}b_{rs}a_{pq}\tag*{PD 3}\\
&=0\quad\text{for}\quad pq=uv
\end{align*}\pageoriginale

Also
\begin{align*}
\{a_{pq}&b_{qr}c_{rs}\}=(a_{pq}\circ b_{qr})\circ c_{rs}\\
&\text{if}(qr, pq, rs)\quad\text{is not connected}\tag*{PD 4}\\
\{a_{pq}&b_{qr}c_{rp}\}=((a_{pq}\circ b_{qr})\circ
c_{rp})U_{e_{p}}\quad \text{if}p\neq r.
\end{align*}
If $p\neq q$ then
\begin{equation*}
  a_{pq}V_{e_{p}} = a_{pq},a_{pq}V_{a_{pp}}U_{b_{pp}}= a_{pq}
  V_{b_{pp}} V_{a_{pp}}V_{b_{pp}}.\tag*{PD 5}
\end{equation*}

(In other words, if $\ob{V}_{a_{pp}}$ denotes the restriction of
$V_{a_{pp}}$ to $\mathscr{J}_{pq}$ then $a_{pp}\to \ob{V}_{a_{pp}}$ is
a homomorphism of the quadratic Jordan algebra \break $(\mathscr{J}_{pp},U,e_p)$
into (End $\mathscr{J}_{pq}$)$^{(q)}$. Finally, we have
\begin{equation*}
e_qU_{a_{pq}}=a^{2}_{pq}U_{e_p},p\neq q\tag*{PD 6}.
\end{equation*}

\begin{proof}
We prove first $PD 1-3$. The formulas in this set of $\{a_{pq}
b_{rs}c_{uv}\}$ with $pq=ur$ are obtained by bilinearization of
$b_{rs}U_{a_{pq}}$. Hence we may drop $\{a_{pq}b_{rs}c_{uv}\}$ for
$pq=uv$. Then the only formula in $PD 1-3$ involving just one index
that we have to prove is $b_{ii}U_{a_{ii}}\epsilon \mathscr{J}$. This
is clear from \eqref{c2:eq8}. Next we consider the formulas in PD $1-3$ which
involve two distinct induces $i,j$. These are:\pageoriginale
\begin{align*}
&b_{ii}U_{a_{ij}}\epsilon \mathscr{J}_{jj}\tag{12}\label{c2:eq12}\\
&b_{ij}U_{a_{ij}}\epsilon \mathscr{J}_{ij}\tag{13}\label{c2:eq13}\\
&\{a_{ij}b_{ji}c_{ii}\}\epsilon \mathscr{J}_{ii}\tag{14}\label{c2:eq14}\\
&\{a_{ii}b_{ii}c_{ij}\}\epsilon \mathscr{J}_{ij}\tag{15}\label{c2:eq15}\\
&\{a_{ii}b_{ij}c_{jj}\}\epsilon\mathscr{J}_{ij}\tag{16}\label{c2:eq16}\\
b_{ji}U_{a_{ii}}&=0,b_{ij}U_{a_{ii}}=0, \{a_{ii}b_{jj}c_{ij}\}=0,\{a_{ii}b_{jj}c_{jj}\}=0\tag{17}\label{c2:eq17}
\end{align*}
\eqref{c2:eq12} and \eqref{c2:eq13} follow from \eqref{c2:eq9}, and
the first two equations in 
\eqref{c2:eq17} follow from \eqref{c2:eq8}. \eqref{c2:eq14} and
\eqref{c2:eq15} and the third part of
\eqref{c2:eq17} follow from \eqref{c2:eq11}. To prove \eqref{c2:eq16}
and the last part of \eqref{c2:eq17} 
we note first that $\mathscr{J}_{ii}\circ \mathscr{J}_{ij}\subseteq
\mathscr{J}_{ij}$ and $\mathscr{J}_{ii}\circ \mathscr{J}_{jj}=0$ if
$i\neq j$. The first of these is an immediate consequence of
\eqref{c2:eq10}. Also \eqref{c2:eq10} implies that $\mathscr{J}_{ii}\circ
\mathscr{J}_{jj}\subseteq \mathscr{J}_{ii}$. By symmetry,
$\mathscr{J}_{ii}\circ \mathscr{J}_{jj}\subseteq \mathscr{J}_{jj}$ and
since $\mathscr{J}_{ii}\cap \mathscr{J}_{jj}=0$ we have
$\mathscr{J}_{ii}\circ \mathscr{J}_{jj}=0$. By $QJ 27$, we have
$\{a_{ii}b_{ij}c_{jj}\}=-\{b_{ij}a_{ii}c_{jj}\}+(a_{ii}\circ
b_{ij})\circ c_{jj}=(a_{ii}\circ b_{ij})\circ c_{jj}$ (third of \eqref{c2:eq17})
$\epsilon \mathscr{J}_{ij}$. Also
$\{a_{ii}b_{jj}c_{jj}\}=-\{b_{jj}a_{ii}c_{jj}\}+(a_{ii}\circ
b_{jj})\circ c_{jj}=0$, by the first of \eqref{c2:eq17} and
$\mathscr{J}_{ii}\circ \mathscr{J}_{jj}=0$. Next we consider $PD 1-2$
for three distinct indices $i,j,k$. The formulas we have to establish
are
\begin{align*}
&\{a_{ij}b_{ij}c_{jk}\}\epsilon\mathscr{J}_{ik}\tag{18}\label{c2:eq18}\\
&\{a_{ij}b_{jj}c_{jk}\}\epsilon\mathscr{J}_{ik}\tag{19}\label{c2:eq19}\\
&\{a_{ij}b_{ji}c_{ik}\}\epsilon \mathscr{J}_{ik}\tag{20}\label{c2:eq20}\\
&\{a_{ij}b_{jk}c_{ki}\}\epsilon \mathscr{J}_{ii}\tag{21}\label{c2:eq21}
\end{align*}

To\pageoriginale prove these we make the following observation. Let $S$ and $T$ be
non-vacaous disjoint subsets of the index set $\{1,2,\ldots,n\}$ and
put $e_s=\sum\limits_{i\epsilon S}e_i,e_T=\sum\limits_{j\epsilon
  T}e_j$. It follows easily, as before, that $e_S$ and $e_T$ are
orthogonal idempotents. Also $\mathscr{J}U_{e_{S}}=\mathscr{J}U_{\sum
  e_{i}}\subseteq \sum\limits_{i\epsilon
  S}U_{e_{i}}+\sum\limits_{i,i''\epsilon S}\mathscr{J}U_{e_{i}}e_{i}$
and since the $U_{e_{i}}$ and $U_{e_{i},e_{i'}}$ are orthogonal
projections with sum
$U_{e_{S}},\mathscr{J}\break U_{e_{i}}=\mathscr{J}U_{e_{i}}U_{e_{S}}$ and
$\mathscr{J}U_{e_{i},e_{i}'}=\mathscr{J}U_{e_{i},e_{i}'}U_{e_{S}}$. Hence
$\sum\limits_{i\epsilon
  S}\mathscr{J}U_{e_{i}}+\sum\limits_{i,i'\epsilon S}\break\mathscr{J}
U_{e_{i},e_{i}}\subseteq \mathscr{J} U_{e_{S}}$ and we have the
equality $\mathscr{J}U_{e_{S}}=\sum\limits_{i\epsilon
  S}\mathscr{J}U_{e_{i}}+\sum\limits_{i,i'\epsilon
  S}\mathscr{J}U_{e_{i},e_{i}'}=\sum\limits_{i,i'\epsilon
  S}\mathscr{J}_{ii'}$. Similary, we have
$U_{e_{S},e_{T}}=\sum\limits_{\substack {i\epsilon S\\j\epsilon
    T}}\mathscr{J}_{ij}$. We now consider the supplementary set of
orthogonal idempotents $\{e_j+e_k,e_l,l \neq j,k\}$. Since
$a_{ii}\epsilon \mathscr{J} U_{e_{i}},a_{ij},b_{ij},
c_{ik},c_{ki}\epsilon \mathscr{J} U_{e_{i},e_{s}},e_s=e_j+e_k$ and
$b_{jj},c_{jk},b_{jk}\epsilon \mathscr{J} U_{e_{S}}$
\eqref{c2:eq14}-\eqref{c2:eq16} imply
that the left hand sides of \eqref{c2:eq18}-\eqref{c2:eq21} are contained in
$\mathscr{J}_{ij}+\mathscr{J}_{ik},\mathscr{J}_{ij}+\mathscr{J}_{ik},
\mathscr{J}_{ij}+\mathscr{J}_{ik},\mathscr{J}_{ii}$
respectively. Similarly, if we use the set of orthogonal idempotents
$\{e_i+e_j,e_l,l\neq i,j\}$ we see that the left hand sides of
\eqref{c2:eq18}-\eqref{c2:eq20} are contained in
$\mathscr{J}_{ik}+\mathscr{J}_{ik}$. Since 
$(\mathscr{J}_{ij}+\mathscr{J}_{ik})\cap(\mathscr{J}_{ik}+\mathscr{J}_{jk})=\mathscr{J}_{ik}$ 
we obtain \eqref{c2:eq18}-\eqref{c2:eq20}. Next we consider the case of four distinct
induces\pageoriginale $i,j,k,l$. The only connected triple here is
$(ij,jk,kl)$. If we use the set of orthogonal idempotents
$\{e_i+e_j+e_k,e_m,m\neq i,j,k\}$ as just indicated we obtain that
$\{a_{ij}b_{jk}c_{kl}\}\epsilon
\mathscr{J}_{il}+\mathscr{J}_{jl}+\mathscr{J}_{kl}$. Similarly, using
$\{e_j+e_k+e_l,e_m,m\neq j,k,l\}$ we get that
$\{a_{ij}b_{jk}c_{kl}\}\epsilon
\mathscr{J}_{ij}+\mathscr{J}_{kj}+\mathscr{J}_{il}$. Taking the
intersection of the right hand sides gives
$\{a_{ij}b_{jk}c_{kl}\}\in \mathscr{J}_{il}$. Since a connected
triple cannot have more than four distinct induces this concludes the
proof of $PD 1$ and $PD 2$. We consider next the triples which are not
connected. The first possibility is $(pq, rs,-)$ with
$\{p,q\}\cap\{r,s\}=\phi$. Choose a subset $S$ of the index set so
that $p,q\epsilon S, r, s\notin S$ and put $e_S=\sum\limits_{i\epsilon
  S}e_i,e_T=\sum\limits_{j\notin S}e_j$. Then we can conclude $\{a_{pq}
b_{rs}-\}=0$ and $b_{rs} U_{a_{Pq}}=0$ from \eqref{c2:eq17} applied to the set of
orthogonal idempotents $\{e_S,e_T\}$ since $a_{pq}\epsilon \mathscr{J}
U_{e_{S}}$ and $b_{rs}\epsilon \mathscr{J} U_{e_{T}}$. Finally suppose
we have $(pq, qr, qs)$ where $r\neq p,q,s$. In this case we obtain
$\{a_{pq}b_{qr}c_{qr}\}=0$ by applying the second part of \eqref{c2:eq17} to the
two orthogonal idempotents $e_r$ and $e'_r=1-e_r$. This proves $PD
3$. To prove $PD 4$ we note that the hypothesis that $(qr, pq, rs)$ is
not connected and $PD 3$ imply that $\{b_{qr}a_{pq}c_{rs}\}=0$. The
first part of $PD 4$ follows from this and $QJ 27$. For the second
part of $PD 4$ we note that $\{a_{pq}b_{qr}c_{rp}\}\epsilon
\mathscr{J}_{pp}$ by $PD 1$ so
$\{a_{pq}b_{qr}c_{rp}\}=\{a_{pq}b_{qr}c_{rp}\}$. By $QJ 27$,       
$\{a_{pq}b_{qr}c_{rp}\}=-\{b_{qr}a_{pq}c_{rp}\}+(a_{pq}\circ
  b_{qr})\circ c_{rp}$. Since $\{b_{qr}a_{pq}c_{rp}\}\epsilon
  \mathscr{J}_{rr}$ and $r\neq p$, applying $U_{e_{p}}$ to the two
  sides of the foregoing equations gives $PD 4$. If $p\neq q$ we have
  $a_{pq}V_{e_{p}}=a_{pq}U_{1,e_{p}}=a_{pq}U_{e_{q}},e_p
  +\sum\limits_{l\neq
    q}a_{pq}Ue_{l},e_{p}=a_{pq}U_{e_{p}},e_q+\sum\limits_{m\neq q,
    p}a_{pq}U_{e_{m},e_{p}}+2a_{pq}U_{e_{p}}=a_{pq}$ since $a_{pq}$
  since $a_{pq}\epsilon \mathscr{J}_{pq}$ and the\pageoriginale
  $U_{e_{i}},U_{e_{i},e_{j}}$ are orthogonal projections. This is the
  first part of $PD 5$. The second part follows directly from $QJ 21$
  and $a_{pq} U_{a_{pp}}=0=a_{pq} U_{a_{pp},b_{pp}}$. To obtian $PD
  6$ we use
  $a^{2}_{pq}U_{e_{p}}=1U_{a_{pq}}U_{e_{p}}=e_{p}U_{a_{pq}}U_{e_{p}}+e_q
  U_{a_{pq}} U_{e_{p}}+e_qU_{a_{pq}}U_{e_{p}}$ (by PD 3). Since
  $e_p U_{a_{pq}}\in \mathscr{J}_{pp}$ and $q_q U_{a_{pq}} \in
  \mathscr{J}_{pp}$ this reduced to $e_qU_{a_{pq}}$, which proves $PD 6$.

The formaulas $PD 4$ 4 imply some uesful associatively formulas for
$\circ$. Suppose we have a connected triple $(pq, qr,rs)$ such that
$(qr, pr, rs)$ and $(pq, rs, qr)$ are note connected. Then we can
apply $PD $ also to $\{c_{ps}b_{qr}a_{pq}\}$ to obtain.

\begin{equation*}
(a_{pq}\circ b_{qr})\circ c_{rs}=a_{pq}\circ (b_{qr}\circ
  c_{rs})\tag{22}\label{c2:eq22}
\end{equation*}

If $(qr. pr. rs)$ and $(pq, rs, qr)$ are not connected. Special
cases of this are
\begin{align*}
&(a_{ii}\circ a_{ij})\circ a_{jj}=a_{ii}\circ(a_{ij}\circ a_{jj}),i\neq
  j\tag{23}\label{c2:eq23}\\
&(a_{ij}\circ a_{jj})\circ a_{jk}=a_{ij}\circ (a_{jj}\circ a_{jk}),
   i,j,k\neq\tag{24}\label{c2:eq24}\\
&(a_{ij}\circ a_{jk})\circ a_{kl} =a_{ij}\circ (a_{jk}\circ a_{kl},
    i,j,k, l\neq.\tag{25}\label{c2:eq25}
\end{align*}
Similarly we have the following consequece of the second part of
$PD4$:
\begin{equation*}
((a_{pq}\circ b_{qr})\circ
  c_{rp})U_{e_{p}}=(a_{pq}\circ(b_{qr^{\circ}}c_{rp}))U_{e_{p}},p\neq
  q,r\tag{26}\label{c2:eq26} 
\end{equation*}

We\pageoriginale note also that the $PD$ theorem permits us to deduce
the following formulas for the squaring composition and its
bilinearization:
\begin{align*}
a^{2}_{pq}\epsilon &\mathscr{J}_{pp}+\mathscr{J}_{qq},a_{pq}\circ
a_{qr}\epsilon
\mathscr{J}_{pr}\text{if}\{p,q\}\Phi\{qr\}\tag{27}\label{c2:eq27} \\
&a_{pq}\circ a_{rs}=0\quad\text{if}\quad\{p,q\}\cap\{r,s\}=\emptyset.
\end{align*}

We leave this to the reader to check. It is easily verified also that 
\begin{equation*}
\mathscr{J}_{ij}\{x|xV_{e_{i}}=x=xV_{e_{j}}\},i\neq j\tag{28}\label{c2:eq28}
\end{equation*}
 \end{proof}

\section{Standard quadratic Jordan matrix algebras}\label{c2:sec2}

Let $(\mathscr{O},j)$ be a (unital) non-associative algebra with
involution, $\mathscr{O}_o$ a $\Phi$-submodule of
$\mathscr{H}(\mathscr{O},j)$ containing all the {\em norms} $a
\ob{a}(\ob{a}=a^{j}).a\epsilon \mathscr{O}$. Then $\mathscr{O}_o$
contains every $a\ob{b}+b\ob{a}, a,b \epsilon \mathscr{O}$ hence all
the {\em tracis} $a+\ob{a}$. It follows that, if $\Phi$ contains
$\frac{1}{2}$ then $\mathscr{O}_o=\mathscr{H}(\mathscr{O},j)$.On the
other hand, as the example of an octonian algebra with standard
involution of a field of characteristic two shows, we may have
$\mathscr{O}_o\subset\mathscr{H}(\mathscr{O},j)$. We consider the
algebra $\mathscr{O}_n$ of $n\times n$ matrices with entries in
$\mathscr{O}$ and the standard involution $J_1:A\to \ob{A}^t$ in
$\mathscr{O}_n$. Let $\mathscr{H}(\mathscr{O}_n,\mathscr{O}_o)$ be
the set of matrices with entries in $\mathscr{O}$ satisfying
$\ob{A}^{t}=A$ and having diagonal entries in $\mathscr{O}_o$. We use
the notation we introduced in considering
$\mathscr{H}(\mathscr{O}_3)\equiv\mathscr{H}(\mathscr{O}_3\Phi)$ and
write
\begin{align*}
  \alpha[ii]&=\alpha e_{ii},\alpha \epsilon
  \mathscr{O}_o\tag{29}\label{c2:eq29} \\
  a[ij]&=ae_{ij}+\ob{a}e_{ij},a\epsilon \mathscr{O},i\neq j.
\end{align*}

Then\pageoriginale it is clear that
$\mathscr{H}(\mathscr{O}_n,\mathscr{O}_o)$ is the set of sums of the
matrices $\alpha[ii]$ and $a[ii]$. Let
$\mathscr{H}_{ii}(\mathscr{O}_o)=\{\alpha[ii]|\alpha\epsilon
\mathscr{O}_o\}$, $\mathscr{H}_{ij}=\{a[ij]|a\epsilon \mathscr{O}\}$
for $i\neq j$. Then we have
\begin{equation*}
  \mathscr{H}(\mathscr{O}_n,o)=\sum\limits_{i} \mathscr{H}_{ii}
  (\mathscr{O}_o) +\sum\limits_{i<j}\mathscr{H}_{ij}\tag{30}\label{c2:eq30}
\end{equation*}
and the sum is direct. Let $A^{2}$ denote the usual square of the
matrix $A\epsilon \mathscr{O}_n$ and put $A\circ B=AB+BA$. Then we
have the following formulas:
\begin{align*}
\alpha[ii]^{2}&=\alpha^{2}[ii]\tag*{M1}\\
a[ij]^{2}&=a\ob{a}[ii]+\ob{a}a[jj],i\neq j\tag*{M2}\\
\alpha[ii]\circ a[ij]&=\alpha a[ij],i\neq j\tag*{M3}\\
a[ij]\circ b[jk]&=ab[ik],i,j,k\neq\tag*{M4}\\
\alpha[ii]\circ\beta[jj]&=0, \alpha[ii]\circ a[jk]=0\\
a[ij]\circ &b[kl]=0,i,j,k,l\neq\tag*{M5}
\end{align*}

It is clear that these formulas together with $a[ij]=\ob{a}[ji]$
determine $A^{2}$ for $A\epsilon
\mathscr{H}(\mathscr{O}_n,\mathscr{O}_o)$ and they show that
$A^{2}\epsilon \mathscr{H}(\mathscr{O}_n, \mathscr{O}_o)$ if
$A\epsilon \mathscr{H}(\mathscr{O}_n,\break \mathscr{O}_o)$. Hence also
$A\circ B\epsilon \mathscr{H}(\mathscr{O}_n, \mathscr{O}_o)$ if
$A,B\epsilon \mathscr{H}(\mathscr{O}_n, \mathscr{O}_o)$.

Let $V_A$ in $\mathscr{H}(\mathscr{O}_n, \mathscr{O}_o)$ denote the
endomorphism $X\to X\circ A$, and suppose from now on that $n\ge
3$. Suppose we have the following identity in
$\mathscr{H}(\mathscr{O}_n, \mathscr{O}_o)$:
\begin{equation*}
  [V_A, V_{B\circ C}]+[V_BV_{A\circ C}]+[V_CV_{A\circ B}]=0\tag{31}\label{c2:eq31}
\end{equation*}

If\pageoriginale we write $[A,B,C]_o$ for the associator $(A\circ
B)\circ C-A\circ (B\circ C)$ then \eqref{c2:eq31} is the same as 
\begin{equation*}
[A,D,B\circ C]_o+[B,D,A\circ C]_o+[C,D A\circ B]_o=0\tag{$31'$}
\end{equation*}
Assume first $n\geq 4$ and take $A=a[ij]$, $B=b[jk]$, $C=c[kl]$,
$D=1[ll]$ where $i,j,k,l\neq$. This gives $[a,b,c][il]=0$ where
$[a,b,c]=(ab)c-a(bc)$ the associator in $\mathscr{O}$. Hence
$[a,b,c]=0$, $a,b,c\epsilon \mathscr{O}$, so $\mathscr{O}$ must be
associative if $n\geq 4$ and \eqref{c2:eq31} holds. Next let $n=3$. Let
$\alpha\epsilon \mathscr{O}_o$ and take $A=\alpha[ii]$, $B=b[jk]$,
$C=l[kk]$, $D=d[ij]$ in (25') where $i,j,k\neq$. This gives $[\alpha,
  d,b][ik]=0$ so
\begin{equation*}
[\alpha, a,b]=0,\quad \alpha\epsilon \mathscr{O}_o,a,b\epsilon
\mathscr{O}\tag{32}\label{c2:eq32}
\end{equation*}
Assume next that in addition to \eqref{c2:eq25} we have the identity
\begin{equation*}
  [V_AV_{A\circ B}]+[V_BV_{A^{2}}]=0\tag{33}
\end{equation*}
or
\begin{equation*}
  [A,D.A\circ B]_0+[B,D,A^{2}]_o=0\tag{33'}
\end{equation*}

Taking $A=a[ij], D=1[kk], B=b[jk], i,j, k\neq$, we obtain $[\ob{a},
  a,b]$ $[jk]=0$ so $[\ob{a},a,b]=0 ,a,b\epsilon \mathscr{O}$. Since
$\mathscr{O}_o$ contians all the traces $a+\ob{a}$ we have by (33),
$[a+\ob{a},a,b]=0$ and, by the result just proved,
$[a,a,b]=0$. Applying the involution we obtain $[b,a,a]=0$. Thus
$\mathscr{O}$ must be alternative. Then (33) implies that
$\mathscr{O}_o\subseteq N(\mathscr{O})$ the nucleus of
$\mathscr{O}$. In particular, we see that we have the\pageoriginale
conditions on $\mathscr{O}$ we noted in \S $1.8$, namely,
$\mathscr{O}$ is alternative and all norms $x\ob{x}\epsilon
N(\mathscr{O})$. We saw also that if these conditions hold then the
$\Phi$ -module $N_o$ spanned by the norms and
$N'=N(\mathscr{O})\cap\mathscr{H}(\mathscr{O},j)$ have the property
that $xN_o\ob{x}\subseteq N_o$, $xN'\ob{x}\equiv N_o,x\epsilon
\mathscr{O}$. We note also that the two conditions (31) and (33) are
consequences of $[V_AV_{A^{2}}]=0$ and the hypothesis that this
carries over $\mathscr{O}_{\ub{\rho}}$ where $\ub{\rho}$ is
commutative associative algebra over the base ring $\Phi$. Our results
give the following
\begin{lemma*}
Let $(\mathscr{O},j)$ be an algebra with involution over $\Phi,
\mathscr{O}_o$ a $\Phi$-submodule of $\mathscr{O}$ containing all
$x\ob{x},x\epsilon \mathscr{O}$. Let
$\mathscr{H}(\mathscr{O}_n,\mathscr{O}_o)$ be the $\Phi$-module of
matrices $A\epsilon \mathscr{O}_n$ such that $\ob{A}^{t}=A$ and the
diagonal elements of $A$ are in $\mathscr{O}_o$. If $A\epsilon
\mathscr{H}(\mathscr{O}_n,\mathscr{O}_o)$ let $A^{2}$ be usual square
of $A, A\circ B=AB+BA$. Assume $n\geqq 3$.Then the identities (31) and
(33) in $\mathscr{H}(\mathscr{O}_n,\mathscr{O}_o)$ imply that
$\mathscr{O}$ is associative if $n\geqq 4$ and $\mathscr{O}$ is
alternative and $\mathscr{O}_o\subseteq N(\mathscr{O})$ if $n=3$.

We can now prove the following
\end{lemma*}

\begin{thm}\label{c2:thm1}
Let $(\mathscr{O},j)$ be an algebra with involution, $\mathscr{O}_o$ a
$\Phi$ -sub module of $\mathscr{H}(\mathscr{O}.j)$ containing all
the norms $x\ob{x},x\epsilon \mathscr{O}$. Let
$\mathscr{H}(\mathscr{O}_n,\mathscr{O}_o)$ be the set of $n\times n$
matrices with entries in $\mathscr{O}$ such that $\ob{A}^{t}=A$ and
the diagonal elements are in $\mathscr{O}_o$. Assume $n\ge 3$. Then
there exists at most one quadratic Jordan structure on
$\mathscr{H}(\mathscr{O}_n,\mathscr{O}_o)$ satisfying the following
conditions: $1U_A=A^{2}$ the usual matrix square, the elements
$e_i=1[ii]=e_{ii},i=1,2,\ldots,n$ are a supplementary set a orthogonal
idempotents in $\mathscr{H}(\mathscr{O}_n,\mathscr{O}_o)$, the
submodule\pageoriginale $\mathscr{H}_{ij} =\{a[ij]|a\epsilon
\mathscr{O}\}$, $i\neq j$ is the pierce $(i,j)$ -module and
$\mathscr{H}_{ii}(\mathscr{O}_o)=\{\alpha[i,i]|\alpha\epsilon
\mathscr{O}_o\}$ is the pierce $(i,i)$-module relative to the set
$\{e_i\}$. Necessary condition for the existence of such a structure
are: $\mathscr{O}$ associative if $n>3$, $\mathscr{O}$ alternative
with $\mathscr{O}_o\subseteq N\mathscr{O}$ if $n=3$ and
$x\mathscr{O}_o\ob{x}\subseteq\mathscr{O}$, $x\epsilon \mathscr{O}$.
\end{thm}

\begin{proof}
  Suppose we have a quadratic mapping $U$ so that $(\mathscr{J},U,1)$ is
  quadratic Jordan and the given conditions hold. Then we shall
  establish the following formulas for the $U$ operator, in which
  $i,j,k,l\neq, \alpha, \beta\epsilon \mathscr{O}_o,\break
  a,b,\epsilon \mathscr{O}$: 
  
  \begin{description}
  \item[QM 1] $\beta[ii]U_{\alpha[ii]}=(\alpha(\beta)\alpha[ii]$
    
  \item[QM 2] $\alpha[ii] U_{[ij]}=\ob{a}(\alpha a)[jj]$
    
  \item[QM 3] $b[ij]U_{a[ij]}=a(\ob{b}a)[ij]$
    
  \item[QM 4] $\{\alpha[ii] a[ij]b[ji]\}=((\alpha a)b+\ob{(\alpha
    a)b})[ii]$
    
  \item[QM 5] $\{\alpha[ii]\beta[ii]a[ii]\}=\alpha(\beta a)[ij]$
    
  \item[QM 6] $\{\alpha[ii] a[ij]\beta[jj]\}=\alpha(a\beta)[ij]$
    
  \item[QM 7] $\{\alpha[ii]a[ij]b[jk]\}=\alpha(ab)[ik]$
    
  \item[QM 8] $\{a[ij]\alpha[jj]b[jk]\}=a(\alpha b)[ik]$
    
  \item[QM 9] $\{a[ij] b[ji] c[ik]\}=a(bc)[ik]$
    
  \item[QM 10] $\{a[ij]b[jk]c[ki]\}=(a(bc)+\ob{a(bc)})[ii]$
    
  \item[QM 11] $\{a[ij] b[jk]c[kl]=a(bc)[il]$.
  \end{description}
\end{proof}

The\pageoriginale formulas $QM 4- QM 11$ are immediate consequences of
$PD 4$ and $M1-M4$. To prove $QM 1$ we note that
$\beta[ii]U_{\alpha[ii]}\epsilon \mathscr{H}(\mathscr{O}_o)$ so this
has the form $\gamma[ii], \gamma\epsilon \mathscr{O}_o$. Then
$\gamma[ij]=\beta[ii]U_{\alpha[ii]}o1[ij] =
1[ij]V_{\beta[ii]U_{\alpha[ii]}} = 1[ij]V_{\alpha[ii]}V_{\beta[ii]}(PD 
6)=(\alpha \beta)\alpha[ij](M3)$ Hence $QM 1$ holds. For $QM 2$ we
recall the identity
\begin{equation*}
  U_bV_a=V_{b,a}V_b+V_aV_b-V_bV_{a,b}\tag{QJ 18}
\end{equation*}

Let $k\neq i,j$. Then $\alpha[ii]U_{a[ij]}\circ
1[jk]=\alpha[ii] U_{a[ij]}V_{1[jk]}=\alpha[ii]$ $V_{a[ij],1[jk]},V_{a[ij]}+\alpha[ii]V_{1[jk]}U_{a[ij]}-\alpha[ii]V_{a[ij]}V_{1[jk],a[ij]}=\ob{a}(\alpha
a)[jk]$ by $PD 4$ and $M1-4$. Since $\alpha[ii]U_{a[ij]}\epsilon
\mathscr{J}_{jj}$ this proves $QM 2$. To prove $QM 3$ we again use $QJ
18$ to write $b[ij]U_{a[ij]}\circ
1[jk]=b[ij]U_{a[ij]}V_{1[jk]}=b[ij]V_{a[ij],1[jk]}V_{a[ij]}$ since the
other two terms given by  $QJ 18$ are $0$ be $PD 3$. Also the first
term is $\{1[jk]b[ij]a[ij]\}\circ a[ij]=((1[jk]\circ b[ij])\circ
a[ij])\circ a[ij]=a(\ob{b} a)[ik]$. Hence $QM 3$ holds. The $PD$
theorem and the argument just used shows that $U$ is unique. Since the
identity $(V_A V_{A^{2}})=0$ holds in
$\mathscr{H}(\mathscr{O}_n,\mathscr{O}_o)$ and in extensions obtained
by extending  the ring $\Phi$, the  Lemma implies that is associative
if $n>3$ and alternative with $\mathscr{O}_0\subseteq N(\mathscr{O})$
if $n=3$. It is clear from $QM 2$ that $x\mathscr{O}_o\ob{x}\subseteq
\mathscr{O}_o$. This completes the proof.

The conditions for $n>3$ given in this theorem are clearly sufficient
since in this case $\mathscr{O}_n$ is associative and
$\mathscr{H}(\mathscr{O}_n,\mathscr{O}_o)$ is a subalgebra of
$\mathscr{O}^{(q)}_n$. Moreover, it is easy to see that the square
$1U_A$ in $\mathscr{H}(\mathscr{O}_n, \mathscr{O}_o)$ coincides with
the usual $A^{2}$ (cf. the proof of the\pageoriginale Corollary to
Theorem 1.5 (\S1.8)) and the conditions on the $e_{i}1[ii]$ hold. The
unique quadratic Jordan structure given in this case is that defined
by $BU_A=ABA$. We now consider the case $n=3$. Suppose $\mathscr{O}$
is an alternative algebra such that all norms $x\ob{x}\epsilon
N(\mathscr{O})$. Let $N_o$ be the submodule generated by all norms,
$N'=\mathscr{H}(\mathscr{O}, j)\cap N(\mathscr{O})$. Then we have
shown in \S $1.8$ that $xN_o\ob{x}\subseteq N_o$ and $xN'\ob{x} \subseteq
N'$. Hence we can take these as choices for the submodule
$\mathscr{O}_o$. It is clear also that any $\mathscr{O}_o$ satisfying
the conditions of the theorem satisfies
$N'\supseteq\mathscr{O}_o\supseteq N_o$. It can be verified by a
rather lengthy fairly direct calculation that
$\mathscr{H}(\mathscr{O}_3,N')$ with the usual $1$ and the $U$
operator defined by $QM 1-11$ is a quadratic Jordan algebra. We omit
the proof of this (due to McMrimmon). In the associative case
$N'=\mathscr{H}(\mathscr{O},j)\cap
N(\mathscr{O})=\mathscr{H}(\mathscr{O},j)$ and
$\mathscr{H}(\mathscr{O}_n,N')=\mathscr{H}(\mathscr{O}_n)$ the
complete set of hermitian matrices with entries in
$\mathscr{O}$. Accordingly, we shall now define a {\em standard
quadratic Jordan matrix algebra} to be any algebra of the form
$\mathscr{H}(\mathscr{O}_n),n=1,2,3,\ldots,$ to be any algebra of the
form $\mathscr{H}(\mathscr{O}_n), n=1,2,3,\ldots$, where
$(\mathscr{O},j)$ is an associative algebra with involution {\em or}
an algebra $\mathscr{H}(\mathscr{O}_3,N')$ where $(\mathscr{O}, j)$ is
alternative with involution such that all norms $x\ob{x}\epsilon
N(\mathscr{O})$ and $N'=\mathscr{H}(\mathscr{O},j)\cap
N(\mathscr{O})$. For the sake of uniformity we abbreviate
$\mathscr{H}(\mathscr{O}_3, N')=\mathscr{H}(\mathscr{O}_3)$. It is
easily seen that if $n\geqq 3$ and $N_o$ is the submodule of
$\mathscr{H}(\mathscr{O},j)$ spanned by the norms then
$\mathscr{H}(\mathscr{O}_n, N_o)$ is the come of $\mathscr{H}(\mathscr{O}_n)$.

\begin{thm}\label{c2:thm2}
Let \pageoriginale $\mathscr{H}(\mathscr{O}_n)$, $n\geqq 3$, be a
standard quadratic Jordan matrix algebra. A subset $\mathscr{J}$ of
$\mathscr{H}(\mathscr{O}_n)$ is an outer ideal containing $1$ if and
only if $\mathscr{J}=\mathscr{H}(\mathscr{O}_n, \mathscr{O}_o)$ where
$\mathscr{O}_o$ is a $\Phi$-submodule of $N'=\mathscr{H}(\mathscr{O},
j)\cap N(\mathscr{O})$ such that $1\epsilon \mathscr{O}_o$ and $x
\mathscr{O}_o\ob{x}\subseteq \mathscr{O}_o$, $x\epsilon
\mathscr{O}$. $\mathscr{J}=\mathscr{H}(\mathscr{O}_n, \mathscr{O}_o)$
is simple if and only if $(\mathscr{O},j)$ is simple.
\end{thm}

\begin{proof}
Let $\mathscr{O}_o$ be a submodule of $N'$ containing $1$ and every
$x\alpha \ob{x}$,\break $x\epsilon \mathscr{O}$, $\alpha \epsilon
\mathscr{O}_o$. Then $\mathscr{O}_o$ contains all the norms $x\ob{x}$
and all the traces $x+\ob{x}$. It is therefore clear from $QM 1-11$
(especially $QM 1, 2,4, 10)$ that
$\mathscr{H}(\mathscr{O}_n,\mathscr{O}_o)$ is an outer ideal. Since
$1\epsilon \mathscr{O}_o$, $\mathscr{H}(\mathscr{O}_n, \mathscr{O}_o)$
contains $1=\sum\limits_{1}^{n}1[ii]$. Conversely, let $\mathscr{J}$ be
an outer ideal of $\mathscr{H}(\mathscr{O}_n)$ containing $1$. Then
$\mathscr{J}$ contains $e_i=1 U_{e_{i}} ,e_{i}=1[ii]$ and every
$a[ij]=\{e_ie_ia[ij]\}$, $a\epsilon \mathscr{O}$, $i\neq j$. Also, if
$\beta[ii]\epsilon\mathscr{J}$ then
$\beta[jj]=\beta[ii]U_{1[ij]}\epsilon \mathscr{J}$ for $j\neq i$. If
$b\epsilon\mathscr{J}$ and $b=\sum\limits_{i\leqq j}b_{ij}$,
$b_{ij}\epsilon \mathscr{H}_{ij}$, then $b_{ii}=bU_{e_{i}}\epsilon
\mathscr{J}$. It is clear from these results that
$\mathscr{J}=\mathscr{H}(\mathscr{O}_n, \mathscr{O}_o)$ where
$\mathscr{O}_o$ is a submodule of $N'$ containing $1$. Since
$\alpha[ii]U_{a[ij]}=\ob{a}\alpha a[jj]\epsilon \mathscr{J}$ if
$\alpha \epsilon \mathscr{O}_o$, $a\epsilon \mathscr{O}, i\neq j$, it
is clear that $a\mathscr{O}_o\ob{a}\subseteq \mathscr{O}_o$,
$a\epsilon \mathscr{O}$. Let $\mathscr{Z}$ be an ideal in
$(\mathscr{O},j)$ and let $k$ be the subset of
$\mathscr{J}=\mathscr{H}(\mathscr{O}_n,\mathscr{O}_o)$ of matrices all
of whose entries are in $\mathscr{Z}$. Then inspection of $QM 1-QM 11$
shows that $k$ is an ideal of $\mathscr{J}$. Hence simplicity of
$\mathscr{J}$ implies simplicity of $(\mathscr{O},j)$. Conversely,
suppose $(\mathscr{O}, j)$ is simple and let $k$ be an ideal $\neq 0$
in $\mathscr{J}$. If $b\epsilon k$ and $b=\sum\limits_{i\leqq j}
b_{ij}, b_{ij}\epsilon \mathscr{H}_{ij}$, then operating on $b$ with
$U_{e_{i}}$ or $U_{e_{i},e_{j}}$ shows\pageoriginale that every $b_{ij}\epsilon
k$. Let $\mathscr{Z}=\{b\epsilon \mathscr{O}|b[12]\epsilon k\}$. We
now use the formulas $M1-M5$ for the squaring operator, which are
consequences of $QM1-QM11$. Since $k$ is an ideal it follows from $M4$
that $\mathscr{Z}=\{b|b[ij]\epsilon k, i\neq j\}$ and $\mathscr{Z}$ is
an ideal of $(\mathscr{O}, j)$. Also, by $M3$, if $\beta[ii]\epsilon
k$ then $\beta \epsilon \mathscr{Z}$. It is clear from these results
that $\mathscr{Z}\neq 0$ so $\mathscr{Z}=\mathscr{O}$, hence
$a[ij]\epsilon k$ for all $a\epsilon \mathscr{O}$, $i\neq j$. Now let
$\alpha \epsilon \mathscr{O}_o$. Then, by $QM 2$,
$\alpha[jj]=\alpha[ii]U_{1[ij]}\epsilon k$ if $i\neq j$. Hence
$k=\mathscr{J}$ and $\mathscr{J}$ is simple.

It is clear from the first part of Theorem \ref{c2:thm2} that
$\mathscr{H}(\mathscr{O}_n.N_o)$,  $N_o$ the submodule generated by the norms is
the core of $\mathscr{H}(\mathscr{O}_n)$ ($=$outer ideal generated
by:1).
\end{proof}

\section[Connectedness and strong connectedness of...]{Connectedness
  and strong connectedness of orthogonal 
  idem-potents}\label{c2:sec3}

In this  section we shall give some lemmas on orthogonal idem-potents
which will be used to prove the Strong Coordinatization Theorem (in
the next section) and will play a role in the structure theory of
chapter III.

\setcounter{defn}{0}
\begin{defn}\label{c2:defn1}
If $e_1$ and $e_2$ are orhtogonal idempotents in $\mathscr{J}$ then
$e_1$ and $e_2$ are {\em connected} (strongly {\em connected}) in
$\mathscr{J}$ if the Pierce submodule
$\mathscr{J}_{12}=\mathscr{J}U_{e_{1},e_{2}}$ contains an element
$u_{12}$ which is invertible in $\mathscr{J}U_e,e=e_{1}+e_2$
(satisfies $u^{2}_{12}=e_1+e_2)$. Then we say also that $e_1$ and
$e_2$ are {\em connected}({\em strongly connected}) by $U_{12}$.

Note that $u^{2}=1$ implies $U^{2}_u=1$ so $u$ is invertible. Thus
strong connectedness implies connectedness. Moreover, if $e_1$ and
$e_2$ are strongly connected by $u_{12}$ then $u_{12}$ is its own
  inverse in $\mathscr{J}U_e$. For,\pageoriginale
  $u_{12}U_{u_{12}}=u_{12}U_{e_{1},e_{2}}U_{u_{12}}=e_1V_{u_{12},e_{2}}U_{u_{12}}=e_1U_{u_{12}}V_{e_{2},U_{12}}(Q
  J4) =u^{2}_{12} U_{e_{2}}V_{e_{2},u_{12}}$ $(PD
  6)=e_2V_{e_{2},u_{12}}=\{e_2e_2u_{12}\}=e^{2}_2\circ u_{12}=u_{12}$.

If $U_{12}$ connects $e_1$ and $e_2$ then
$\mathscr{J}_{11}U_{u_{12}}\subseteq
\mathscr{J}_{22},\mathscr{J}_{22}U_{u_{12}}\subseteq
\mathscr{J}_{11}$, $\mathscr{J}_{12}U_{u_{12}}\subseteq\mathscr{J}_{12}$
by $PD 2$. Hence if $U'$ is the inverse of the restriction of
$U_{u_{12}}$ to $U_e$ then
$\mathscr{J}_{12}U'=\mathscr{J}_{12}$. Hence the inverse
$u_{21}=u_{12}U'$ of $u_{12}$ in $\mathscr{J}U_e$ is contained in
$\mathscr{J}_{12}$. If $f=1-e$ then $f$ and $e$ are orthogonal
idempotents and since $e_i\epsilon \mathscr{J}U_e$, $\{e_1, e_2, f\}$
is a supplementary set of orthogonal idempotents in $\mathscr{J}$. It
is clear also from the $PD$ formulas that $\mathscr{J}U_e+\mathscr{J}U_f$
is a subalgebra of $\mathscr{J}$, that $U_e$ and $U_1$ are ideals in
this subalgebra and
$\mathscr{J}U_e+\mathscr{J}U_f=\mathscr{J}U_e\oplus
\mathscr{J}U_f$. If follows that if $x\epsilon \mathscr{J} U_e$,
$y\epsilon \mathscr{J} U_f$ then $x+y$ is invertible in
$\mathscr{J}U_e+\mathscr{J}U_f$ if and only if $x$ is invertible in
$\mathscr{J}U_e$ and $y$ is invertible in $\mathscr{J}U_f$. In this
case, the definition of invertibility shows that $x+y$ is invertible
in $\mathscr{J}$. Also if $x'$ and $y'$ respectively are the inverse
of $x$ and $y$ in $\mathscr{J}U_e$ and $\mathscr{J}U_f$ then $x'+y'$
is the invese of $x+y$ in $\mathscr{J}U_e +\mathscr{J}U_f$ and so in
$\mathscr{J}$. In particular, if  $u_12\epsilon
\mathscr{J}_{12}=\mathscr{J}U_{e_{1},e_{2}}$ is invertible in
$\mathscr{J}U_e$ with inverse $u_{21}\epsilon \mathscr{J}_{12}$ then
$c_{12}=f+u_{12}$ is invertible in $\mathscr{J}$ with inverse
$c_{21}=f+u_{21}$. We have the Pierce decomposition
$\mathscr{J}=\mathscr{J}U_e\oplus \mathscr{J}U_{e,f}\oplus
\mathscr{J}U_f$ relative to $\{e,f\}$ and we have the following
usefull formulas for the action of
$U_{c_{12}}=U_{u_{12}}+u_f+u_{f,u_{12}}$ on these submodules:
$$
xU_{c12}=xU_{u12}, x\epsilon\mathscr{J}U_e(PD 3)
$$\pageoriginale
\begin{align*}
yU_{C_{12}}&=y\circ u_{12},y\epsilon \mathscr{J} U_{e,f}(PD
3,4,5)\tag{34}\label{c2:eq34}\\ 
zU_{c_{12}}&=z,z\epsilon \mathscr{J} U_f (PD 3)
\end{align*}
If $U^{2}_{12}=e$ so $u_{21}=u_{12}$ then $c^{2}_{12}=1$ and
$U_{c_{12}}$ is an automorphism of $\mathscr{J}$ such that
$U_{c^{2}_{12}}=1$ (see \S 1.11).
\end{defn}

\setcounter{lemma}{0}
\begin{lemma}\label{c2:sec3:lem1}
  Let $e_1,e_2,e_3$ be pairwise orthogonal idempotents in
  $\mathscr{J}$ such that $e_1$ and $e_2$ are connected (strongly
  connected) by $u_{12}$ and $e_2$ and $e_3$ are connected (strongly
  connected) by $u_{23}$. Then $e_1$ and $e_3$ are connected (strongly
  connected) by $u_{13}=u_{12}\circ u_{23}$ In the strongly connected
  case $c_{12}=u_{12}+1-e_1-e_2,c_{23}=u_{23}+1-e_{2}-e_3$,
  $c_{13}=u_{13}+1-e_1-e_3$ satisfy $c^{2}_{ij}=1$, $U_{c_{ij}}$ is an
  automorphism such that $U^{2}_{c_{ij}}=1$ and we have
\begin{align*}
&c_{13}=c_{12}U_{c_{23}}=c_{23}U_{c_{12}}\tag{35}\label{c2:eq35}\\
U_{c_{13}}=U_{c_{23}}&U_{c_{12}}U_{c_{23}}=U_{c_{12}}U_{c_{23}}U_{c_{12}}\tag*{\text{so}}
\end{align*}
\end{lemma}

\begin{proof}
Put $e_4=1-e_1-e_2-e_3$ so $\{e_i|i=1,2,3,4\}$ is a supplementary set
of orthogonal idempotents. Let $\mathscr{J}=\sum\mathscr{J}_{ij}$ be
the corresponding Pierce decomposition. Then
$c_{12}=u_{12}+e_3+e_4,c_{23}=u_{23}+e_1+e_4$. Since $u_{12}$ is
invertible in $\mathscr{J}_{11}+\mathscr{J}_{12}+\mathscr{J}_{22}$
with inverse $u_{21},c_{12}$ is invertible in $\mathscr{J}$ with
inverse $c_{21}=u_{21}+e_3+e_4$. Similarly, $c_{23}$ is invertible
with inverse $c_{32}=u_{32}+e_1+e_4$. By the\pageoriginale Theorem on
inverse $c_{13}=c_{23}U_{c_{12}}$ is invertible with inverse
$c_{31}=c_{32}U_{c_{21}}$. We have

\begin{align*}
c_{13}&=c_{23}U_{c_{12}}= u_{23} u_{c_{12}} + e_1 U_{c_{12}} + e_4U_{c_{12}}\\
&=u_{13}+u^{2}_{12}+e_1U_{u_{12}}+e_{4}\quad (\text{by} (34))\\
&=u_{13}+u^{2}_{12}U_{e_{2}}+e_4 (PD 6).
\end{align*}
Similarly, $c_{31}=u_{31}+e_1U_{u_{21}}+e_4$. Since
$u^{2}_{12}U_{e_{2}}$ and $u^{2}_{21}U_{e_{2}}\epsilon
\mathscr{J}_{22}$ it follows that $u_{13}$ is invertible in
$\mathscr{J}_{11}+\mathscr{J}_{13}+\mathscr{J}_{31}$ with inverse
$u_{31}$. Hence $e_{1}$ and $e_{3}$ are connected by
$u_{13}=u_{12}\circ u_{23}$. If $u^{2}_{12}=e_1+e_2$ and
$u^{2}_{23}=e_2+e_3$ then $c^{2}_{12}=1$ and $c^{2}_{23}=1$. Then
$U_{c_{12}}, U_{c_{23}}$ are automorphisms with square $1$. Also
$c_{13}=c_{23}U_{c_{12}}$ satisfies $c^{2}_{13}=1$ so $U_{c_{13}}$ is
an automorphism and $U^{2}_{c_{13}}=1$. The formula for $c_{13}$ now
becomes $c_{13}=u_{23}\circ u_{12}+e_{2}+e_{4}$. A similar calculation
gives $c_{12}U_{c_{23}}=u_{12}\circ u_{23}+c_2+c_4$. Hence
$c_{12}U_{c_{23}}=c_{23}U_{c_{12}}$ so $(35)$ and its consequence
$U_{c_{23}} U_{c_{12}}U_{c_{23}}=U_{c_{12}}U_{c_{23}}U_{c_{12}}$ hold.

The following lemma is of technical importance since it permits the
reduction of considerations on connected idempotents to strongly
connected idempotents.
\end{proof}

\begin{lemma}\label{c2:sec3:lem2}
Let $\{e_i i=1,\ldots,n\}$ be a supplementary set of orthogonal
idempotents in $\mathscr{J}, \mathscr{J}=\sum \mathscr{J}_{ij}$ the
corresponding Pierce decompostion. Assume $e_1$ and $e_j,j>1$, are
connected by $u_{1j}$ with inverse $u_{ji}$ in
$\mathscr{J}_{11}+\mathscr{J}_{1j}+\mathscr{J}_{jj}$. Then
\begin{equation*}
  u_j=u^{2}_{ij}U_{e_{j}}\quad\text{and}\quad
  v_j=u^{2}_{j1}U_{e_{j}},j>1\tag{36}\label{c2:eq36} 
\end{equation*}\pageoriginale
are inverses in $\mathscr{J}_{jj}$ and if we put $u_1=e_1=v_1$ then
$u=\sum\limits_{1}^{n}u_i$, $v=\sum\limits_{1}^{n}v_i$ are inverses in
$\mathscr{J}$. The set $\{u_i\}$ is a supplementry set of orthogonal
idempotents in the $v$-isotope $\mathscr{J}=\mathscr{J}^{(v)}$ and
$u_j$ is strongly connected by $u_{1j}$ to $u_1$ in $\mathscr{J}$. The
Pierce submodule $\widetilde{\mathscr{J}}_{ij}$ of
$\widetilde{\mathscr{J}}$ relative to the $u_i$ coincides with
$\mathscr{J}_{ij}$. Moreover,
$\mathscr{J}_{11}=\widetilde{\mathscr{J}}_{11}$ are algebras and
$\mathscr{J}_{jj}$ and $\widetilde{\mathscr{J}}_{jj},j>1$, are
isotpic. Also, if $j>1$,  $x_{11}\epsilon \mathscr{J}_{11}$,
$x_{1j}\epsilon
\mathscr{J}_{1j},x_{1j}V_{x_{11}}=x_{1j}\widetilde{V}_{x_{11}}$ where
$\widetilde{V}$ is the  $V$-operator in $\widetilde{\mathscr{J}}$. 
\end{lemma}

\begin{proof}
It is clear that $u^{2}_{1j},u^{2}_{j1}\epsilon
\mathscr{J}_{11}+\mathscr{J}_{jj}$ and these are inverses in
$\mathscr{J}_{11}+\mathscr{J}_{jj}$. It follows that
$u_j=u^{2}_{1j}U_{ej}$ and $v_{1}=u^{2}_{jl}U_{e_{j}}$ are inverses
in $\mathscr{J}_{jj}$ and $u=\sum\limits_{1}^{n}u_i$,
$v=\sum\limits_{1}^{n}v_i(u_1=e_1=v_1)$ are inverses in
$\mathscr{J}$. Now consider the isotope
$\widetilde{\mathscr{J}}=\mathscr{J}^{(v)}$ with unit element $u$. We
have $U_{u_{i}}^{(v)}=U_vU_{ui}=\sum_j
U_{v_{i}}U_{u_{i}}+\sum\limits_{j<k}U_{v_{j},v_{k}}U_{u_{i}}$. It is
clear from the $PD$. Theorem $(PD 1-3)$ that $\mathscr{J}
U_{v_{j}}\subseteq
\mathscr{J}_{jj},\mathscr{J}U_{v_{j},v_{k}}\subseteq\mathscr{J}_{jk}$
  and $\mathscr{J}_{jk}$ and $\mathscr{J}_{pq}U_{u_{i}}=0$ unless
  $p=q=i$. Hence $U_{u_{i}^{(v)}}=U_{v_{i}}U_{u_{i}}$ and
    $\mathscr{J}U_{v_{i}}U_{u_{i}}=\mathscr{J}_{ii}U_{v_{i}}U_{u_{i}}$. Also
    since the restrictions of $U_{v_{i}}$ and $U_{u_{i}}$ to
    $\mathscr{J}_{ii}$ are inverses we have
\begin{equation*}
U^{(v)}_{u_{i}}=U_{e_{i}}.\tag{37} \label{c2:eq37}
\end{equation*}

Similarly, replacing $e_i$ by $e_i+e_j, u_i$ by $u_i+u_j,v_i$ by
$v_i+v_j$, $i\neq 1$. We obtain
$U^{(v)}_{u_{i}+u_{j}}=u_{e_{i}+e_{j}}$. This and (37) imply
\begin{equation*}
  U^{(v)}_{u_{i},u_{j}}=U_{e_{i},e_{j}},i\neq j\tag{38}\label{c2:eq38}
\end{equation*}

Now\pageoriginale $uU_{ui}^{(v)}\left(\sum\limits_{1}^{n}u_k\right)
U_{e_{i}}=u_i,uU_{u_{i},u_{j}}^{(v)}=\sum\limits_{k}u_kU_{e_{i},e_{j}}=0$
and $u_i U^{(v)}_{U_{j}}=u_i U_{e_{j}}=0$ if $i\neq j$. These shows
that the $u_i$ are orthogonal idempotents in the isotope
$\widetilde{\mathscr{J}}=\mathscr{J}^{(v)}$. Since their sum is $u$
  they are supplementary. Then \eqref{c2:eq37} and \eqref{c2:eq38} show that
  $\widetilde{\mathscr{J}}_{ii}= \mathscr{J}_{ii},
  \widetilde{\mathscr{J}}_{ij} =\mathscr{J}_{ij}$ 
    for the corresponding Pierce submodules. Since $u_{1j}\epsilon 
    \mathscr{J}_{ij}, u_j\epsilon \mathscr{J}_{1j}$. More over,
    $uU^{(v)}_{u_{1j}}=uU_vU_{u_{lj}}=vU_{u_{lj}}=(e_1+ v_j)
    U_{u_{1j}}=e_1U_{u_{1j}} + v_1U_{u_{1j}}=u^{2}_{1j}U_{e_{j}}+u^{2}
    _ {j1}U_{e_{j}}U_{u_{1j}}$
    (PD 6 and (\eqref{c2:eq36}))$=u_j+e_1 U_{u_{jl}}U_{u_{lj}}=u_j+e_1$. Hence
    $u_{1j}$ strongly connects $e_1$ and $u_i$ in
    $\widetilde{\mathscr{J}}$. Let $x_i,y_i\epsilon
    \mathscr{J}_{ii}=\widetilde{\mathscr{J}}_{ii}$, $x\epsilon
    \mathscr{J}_{lj}=\widetilde{\mathscr{J}}_{lj}, j>1$. Then
    $x_iU^{(v)}_{y_i}=x_iU_vU_{y_{i}}=x_iU_{v_{i}}U_{y_{i}}$. If $i=1$,
    $v_1=e_1$ so $x_1U_{y_{1}}^{(v)} = x_i U_{y1}$. Thus
    $\mathscr{J}_{ii}$ and $\widetilde{\mathscr{J}}_{ii}$ are isotopic
    $\mathscr{J}_{11}$ and $\widetilde{\mathscr{J}}_{11}$ are
    identical as algebras. Finally, $x\widetilde{V}_{x_1}=u
    U_{x,x_{1}}^{(v)}=uU_v U_{x,x_1}=
    vU_{x,x_1}=\{xvx_1\}=\{xe_1x_1\}=\{x\sum\limits_1^{n}e_i x_1\}=x
    V_{x_1}$. This completes the proof.
\end{proof}

\begin{lemma*}
Let $\{e_i|=i=1,2,\ldots,n\}$ be a supplementary set os orthogonal
idempotents in $\mathscr{J}$ such that $e_1$ is strongly connected to
$e_j, j>1$, by $u_{1j}$. Put $c_{1j}= u_{1j}+1-e_1-e_j$ as above and
$U_{(1j)}=U_{c_{1j}}$.Then there exists a unique isomorphism $\pi\to
U_{\pi}$ of the symmetric group $S_n$ into $\Aut \mathscr{J}$ such that
$(1j)\to U_{(ij)}$. Moreover, $e_iU_{\pi}=e_{i\pi}$ and if
$i\pi=i\pi'$, $j\pi=j\pi'$ then $U_{\pi}=U_{\pi'}$ on
$\mathscr{J}_{ij}$ ($i=j$ allowed).
\end{lemma*}


\begin{proof}
We have seen that $U_{(1j)}$ is an automorphism of period two. Now it
is known that the symmetric group $S_n$ is generated by the
transpositions $(1j)$ and that the defining realations for there is
$(1j)^{2}=1, ((1j)$ $(1k)^{3}=1$,
$((1j)(1k)(1j)1l))^{2}=1,j,k,l\neq$. By lemma \ref{c2:lem1}, we have
$U_{(1j)}U_{(1k)}U_{(1j)}=U_{(1k)}U_{(1j)}U_{(1k)}$. Hence\pageoriginale
$(U_{(1j)}U_{(1k)})^{3}=U_{(1j)} U_{(1k)}$
$U_{(1j)}U_{(1k)}U_{(1j)}U_{(1k)}$ $=U_{(1j)}U_{(1k)}U_{(lj)}U_{(1j)}U_{(1k)}U_{(1j)}=1$. Also,
if $j,k,\break l\neq$, then
$U_{(1j)}U_{(1k)}U_{(1j)}$ $U_{(1l)}U_{(1j)}$ $U_{(1j)}U_{(1k)}U_{(1j)}
U_{(1l)}=U_{c_{1_{l}}U_{c_{1_{j}}}U_{c_{1_{j}}}}\break U_{c_{1_{1}}}= U_{c_{1_{1}}}U_{c_{jk}}U_{c_{1l}}$
where $c_{jk}=c_{lk}U_{c_{1j}}$. The form of $c_{jk}$ derived in Lemma
\ref{c2:lem1} and \eqref{c2:eq34} imply that $c_{1l}U_{c_{jk}}=c_{1l}$. Hence
$U_{c_{1l}}U_{c_{jk}}U_{c_{1l}}=1$. These relations imply that we have
a unique monpmorphism of $S_n$ into Aut $\mathscr{J}$ such that
$(1j)\to U_{(1j)}=U_{c_{1j}}$. By \eqref{c2:eq34} $e_1U_{(1j)}=e_1U_{c_{1j}}=
e_1U_{U_{1j}}=u^{2}_{1j}U_{e_{j}} (PD 6)=e_j$. Hence
$e_{i}U_{\pi} = e_{i\pi}$ for $\pi \epsilon S_n$. Now suppose $\pi$
and $\pi'$ satisfy $e_{i} \pi =e_i \pi'$, $e_j\pi'=e_j \pi'$. Put
$\pi''=\pi'\pi^{-1}$. Then $e_i\pi''=e_i,e_j\pi''=e_j$ so $\pi''$ is a
product of transpositions which fix $i$ and $j$. If $(kl)$ is such a
transposition then $U_{(kl)}=U_{(1l)}U_{(1k)}U_{(1l)}$. By \eqref{c2:eq34} this
acts as identity on $\mathscr{J}_{ij}$. Hence $\pi''$ is $1$ on
$\mathscr{J}_{ij}$ and $\pi=\pi''$ on $\mathscr{J}_{ij}$.
\end{proof}


\section{Strong coordinatization theorem}\label{c2:sec4}

We shall now obtain an important characterization of the quadratic
Jordan algebras $\mathscr{H}(\mathscr{O}_n,\mathscr{O}_o),n\geqq 3$,
or equivalently, in view of Theorem \ref{c2:thm2}, of the outer ideals containing
$1$ in standard quadratic Jordan algebras $\mathscr{H}(\mathscr{O}_n),
n\geqq 3$. We shall call a triple $(\mathscr{O},j,\mathscr{O}_o)$ a
{\em coordinate algebra}, if $(\mathscr{O}, j)$ is an alternative
algebra with involution and $\mathscr{O}_o$ is a $\Phi$ -submodule of
$N'(\mathscr{O})=\mathscr{H}(\mathscr{O},j)\cap N(\mathscr{O})$
suchthat $1\epsilon \mathscr{O}_o$ and $x\mathscr{O}_o\ob{x}\subseteq
\mathscr{O}_o,x\epsilon \mathscr{O}$. We call
$(\mathscr{O},j,\mathscr{O}_o)$ {\em associative} if $\mathscr{O}$ is
associative. We shall show that
$\mathscr{J}=\mathscr{H}(\mathscr{O}_n,\mathscr{O}_o), n\geqq 3$, is
characterised by the following two conditions: (1) $\mathscr{J}$
contains $n\geqq 3$ supplementary strongly connected orthogonal
idempotents, $(2)$ $\mathscr{J}$ is non-degenerate\pageoriginale in
the sense that $\ker U=0$. Consider an
$\mathscr{H}(\mathscr{O}_n,\mathscr{O}_o),n\geqq 3$. Let $e_i=1[ii]$,
$u_{1j}=1[1j],j>1$ (notation as in \S 2). Then the $e_i$ are
orthogonal idempotents and $\sum\limits_{1}^{n}e_i=1$. Also $u_{1j}$
is in the Pierce $(i,j)$ component of
$\mathscr{H}(\mathscr{O}_n,\mathscr{O}_o)$ relative to the $e_i$ and
$u^{2}_{1j}=e_1+e_j$ so $e_1$ and $e_j$ are strongly connected by
$u_{1j}$. Hence (1) holds. To prove (2) we recall that $\ker U$ is an
ideal (\S 1.5) so if $z=\sum\limits_{1\leqq j} z_{ij}[ij]\epsilon \ker
U$ then $z_{ii}[ii]=zU_{e_{i}}\epsilon \ker U$ and
$z_{ij}[ij]=zU_{e_{i},e_{j}},i\neq j$, $\ker U$. We recall also that
all products involving an element of $\ker U$ are $0$ except those of
the form $zU_a,z \in  \ker U$. Hence $z_{ij}[ij]=0$ and
$z_{ii}[ij]=z_{ii}[ii]\circ 1[ij]=0$. Then $z_{ii}=0$ and $z=0$. Thus
$\mathscr{H}(\mathscr{O}_n,\mathscr{O}_o)$ is non-degenerate.

We shall now prove that the conditions (1) and (2) are sufficient for
a quadratic Jordan algebra to be isomorphic to an algebra
$\mathscr{H}(\mathscr{O}_n.\mathscr{O}_o)$, $n\geqq 3$. We have the
following

\noindent
{\textbf{Strong coordinatization Theorm}}. Let $\mathscr{J}$ be a
quadratic Jordan algebra satisfying:(1)$\mathscr{J}$ is
non-degenerate, (2) $\mathscr{J}$ contains $n\geqq 3$ supplementary
strongly connected orthogonal idempotents. Then there exists a
coordinate algebra $(\mathscr{O},j\mathscr{O}_o)$ which is associative
if $n\geqq 4$ such that $\mathscr{J}$ is isomorphic to
$\mathscr{H}(\mathscr{O}_n,\mathscr{O}_o)$. More precisely, let
$\{e_i|i=1,\ldots,n\}$ be a supplementary set of strongly connected
orthogonal idempotents and let $e_1$ be strongly connected to $e_j,j>1$,
by $u_{1j}$. Then there an isomorphism $\eta$ of $\mathscr{J}$ onto
$\mathscr{H}(\mathscr{O}_n,\mathscr{O}_o)$ such that
$e^{\eta}_i=1[ii],i=1,\ldots,n,$, $u_{1j}^{\eta}=1[1j],j=2,\ldots,n$.

\begin{proof}
Put\pageoriginale $c_{ij}=u_{ij}+1-e_1-e_j$. By lemma $3$ of \S 3 we have an
isomorphism $\pi \to U_{\pi}$ of $S_n$ into Aut $\mathscr{J}$ such
that $U_{(ij)}=U_{c_{1j}},e_iU_{\pi}= e_{i \pi}$. Then for the Pierce module
$\mathscr{J}_{pq}$ (relative to the $e_i$) we have
$\mathscr{J}_{pq}U_{\pi}=\mathscr{J}_{pq}U_{p\pi,q\pi}$. Also if $\pi,
\pi' \epsilon S_n$ satisfy $p{\pi}=p{\pi'}, q{\pi}= q \pi'$ then $U_{\pi}$
and $U_{\pi'}$ have the same restrictions to $\mathscr{J}_{pq}$. This
implies that if $p\pi=p$ and $q\pi=q$ then $U_{\pi}$ is the identity
on $\mathscr{J}_{pq}$. By Lemma $1$ of \S $3$, $c_{jk}=c_{1j}
U_{c_{1k}}$, for $i,j,k\neq$ satisfies
$c_{jk}=c_{kj}=c_{1k}U_{c_{1j}}=u_{jk}+1-e_j-e_k$ where
$u_{jk}=u_{1j}\circ u_{1k}$ also $u^{2}_{jk}=e_j+ e_k,
U_{(jk)}=U_{(1k)}U_{(1j)}U_{(1k)}=U_{c_{jk}}$. By \eqref{c2:eq34}, if $i,j,k\neq$
then $x_{ij}U_{(jk)}=x_{ij}\circ u_{jk}$. Hence $(x_{ij}\circ
u_{jk})\circ u_{jk}= x_{ij}U^{2}_{(jk)}=x_{ij}$. In particular, if
$1,j,k\neq$ then $u_{jk}\circ u_{1k}=(u_{1j}\circ u_{lk})\circ
u_{lk}=u_{1j}$. We note also that
\begin{equation*}
  U^{-1}_{\pi}U_{e_{i}}U_{\pi}=U_{e_{i}\pi} \tag{39}\label{c2:eq39}
\end{equation*}
since for $\pi=(1j)$ we have
$U_{(1j)}U_{e_{i}}U_{(1j)}=U_{c_{1j}}U_{e_{i}}U_{e_{1j}}=U_{e_{i}}U_{c_{1j}}=
U_{e_{i}}U_{1j}=U_{e_{i(1j)}}$ and $U^{-1}_{\pi\pi'}
U_{e_{i}}U_{\pi\pi'}=U^{-1}_{\pi'}U^{-1}_{\pi}U_{e_{i}}U_{\pi}U_{\pi'}$.
Let $\mathscr{O}=\mathscr{J}_{12}$ and define for $x,y$
\begin{equation*}
xy=xU_{(23)}\circ yU_{(13)}\tag{40}\label{c2:eq40}
\end{equation*}

Since $xU_{(23)}\epsilon \mathscr{J}_{13}$ and $yU_{(13)}\epsilon
\mathscr{J}_{23},xy\epsilon \mathscr{O}=\mathscr{J}_{12}$. Also the
product is $\Phi$ -bilinear. Define for $x\epsilon \mathscr{O}$.
\begin{equation*}
  j:x\to \ob{x}=xU_{(12)}\tag{41}\label{c2:eq41}
\end{equation*}

Since\pageoriginale $U_{(12)}$ maps $\mathscr{J}_{12}$ into itself and
$U^{2}_{(12)}=1$, $j$ is a $\Phi$-isomorphism of $\mathscr{J}_{12}$ such
that $j^{2}=1$. Also if $x,y\epsilon \mathscr{O}$,
\begin{align*}
  \ob{xy}&=(xU_{(23)}\circ y U_{(13)})U_{(12)}\\
  &=x U_{(23)}U_{(12)}\circ yU_{(13)}U_{(12)}\\
  &=y U_{(12)}U_{(23)}\circ x U_{(12)}U_{(13)}\\
  &=\ob{y}\ob{x}\\
  xu_{12}&=xU_{(23)}\circ 12^{U}(13)\\
  &=xU_{(23)}\circ U_{23}\\
  &=xU^{2}_{23}\\
  &=x\\
  \ob{u}_{12}&=u^{3}_{12}=u_{12}\quad (\text{see} \S 3)
\end{align*}

These relations imply that $u_{12}$ acts as the unit element of
$\mathscr{O}$ relative to its product and $j$ is an involution in
$\mathscr{O}$.

We now define $n^{2}$ ``coordinate mappings''
$\eta_{pq},p,q=1,2,\ldots, n$ of $\mathscr{J}$ into $\mathscr{O}$ as
follows:
\begin{align*}
  \eta_{ij}&=U_{e_{i},e_{j}}U_{\pi}\quad \text{if}\quad i\neq j, i\pi=1,
  j\pi=2\tag{42}\label{c2:eq42}\\
  \eta_{ii}&=U_{e_{i}}U_{\pi}V_{c_{12}}\quad\text{if}\quad
  i\pi=1\tag{43}\label{c2:eq43} 
\end{align*}

It\pageoriginale is clear from the preliminary remark that this is
independent of the choice of $\pi$. Also $\eta_{pq}=1$ on all the
Pierce submodules except $\mathscr{J}_{pq}$. If $i\neq j$, $U_{\pi}$
is a $\Phi$ isomorphism of $\mathscr{J}_{ij}$ onto
$\mathscr{O}=\mathscr{J}_{12}$. Hence $\eta_{ij}$ is a
$\Phi$-isomorphism of $\mathscr{J}_{ij}$ onto $\mathscr{O}$. Since
$\mathscr{J}_{ii}U_{e_{i}}U_{\pi}=\mathscr{J}_{11}$ it is clear also
that $\eta_{ii}$ is a $\Phi$ homomorphism of $\mathscr{J}_{ii}$ onto
$\mathscr{O}_o\equiv\mathscr{J}_{11}^{\eta_{11}}=\mathscr{J}_{11}V_{c_{12}}$. We
now prove for $x\epsilon \mathscr{J}$
\begin{equation*}
  \ob{x^{\eta_{pq}}}=x^{\eta_{pq}}.\tag{44}\label{c2:eq44}
\end{equation*}

If $i\neq j$, we have
$\ob{x^{\eta_{ij}}}=xU_{e_{i},e_{j}}U_{\pi}U_{(12)}=xU_{e_j,e_i}U_{\pi'}$,
where $i\pi'=2, j\pi'=1$. Hence $\ob{x^{\eta_{ij}}}=x^{\eta_{ji}}$. To
prove $\ob{x^{\eta_{ii}}}=x^{\eta_{ii}}$ we require
\begin{equation*}
  V_{c_{12}}U_{c_{12}}=V_{c_{12}}=U_{c_{12}}V_{c_{12}}\tag{45}\label{c2:eq45}
\end{equation*}

We have
$V_{c_{12}}U_{c_{12}}U_{c^{2}_{12},c_{12}}=U_{C_{12}}V_{c_{12}}$ by
$QJ 24$. Since $c^{2}_{12}=1$ and $U_{1,c_{12}}=V_{c_{12}}$ we have
\eqref{c2:eq45}. Now $\ob{x^{\eta_{ii}}}=xU_{e_{i}}UV_{c_{12}}=xU_{e_{i}}U
V_{c_{12}}=X^{\eta_{ii}}$. 

Define the mapping $\eta$ of $\mathscr{J}$ onto
$\mathscr{H}(\mathscr{O}_{n},\mathscr{O}_{o})$ by
\begin{equation*}
  x=\sum\limits_{p\leqq q}x^{\eta_{pq}}[pq].\tag{46}\label{c2:eq46}
\end{equation*}

This is a $\Phi$-homomorphism of $\mathscr{J}$ onto
$\mathscr{H}(\mathscr{O}_n,\mathscr{O}_o)$ since, if $x\epsilon
\mathscr{J}_{pq}$, $x=x^{\eta_{pq}}[pq]$. It is clear that
$\mathscr{J}^{\eta}_{pq}=\mathscr{H}_{pq}$ where these are defined as
usual. We have
$e^{\eta}_i=e_iU_{e_{i}}U_{\pi}V_{c_{12}}[ii]=e_1V_{c_{12}}[ii]=e_1V_{u_{12}}[ii]=
U_{12}[ii]=1[ii](U_{12}=1$ in $\mathscr{O})$. Hence $1^{\eta}=1$
also. Also $u^{\eta}_{12}=n_{12}[12]$ and\pageoriginale if $1,2,j\neq$
then $u^{\eta}_{ij}=u_{1j}U_{(2j)}[ij]=(U_{1j}=(u_{1j}\circ
u_{2j})[1j]=u_{12}[1j]=1[1j]$ (since $u_{1j}\circ u_{2j}=u_{12}$ was
  shown in the first paragraph).

We now consider $\mathscr{J}$ relative to the squaring operation
$(a^{2}=1U_a)$ and we shall prove for $i,j,k\neq, x_{ij}\epsilon
\mathscr{J}_{ij}$ etc:
\begin{align*}
  (x_{ij}\circ y_{jk})^{\eta}& =x^{\eta}_{ij}\circ
  y^{\eta}_{jk}\tag{47}\label{c2:eq47}\\
  (x_{ii}\circ y_{ij})^{\eta}& =x_{ii}^{\eta}\circ
  y^{\eta}_{ij}\tag{48}\label{c2:eq48}\\
  (x^{2}_{ii})^{\eta}& =(x^{\eta}_{ii})^{2}\tag{49}\label{c2:eq49}\\
  (x^{2}_{ij})^{\eta}& =(x^{\eta}_{ij})^{2}\tag{50}\label{c2:eq50}
\end{align*}

For \eqref{c2:eq47} let $\pi$ be a permutation such that $i\pi=1$, $j\pi=3$,
$k {\pi}=2$ and put $x_{ij}U_{\pi}=x\epsilon
\mathscr{J}_{13},y_{jk}U_{\pi}=y\epsilon \mathscr{J}_{23}$. Then
\begin{align*}
  x^{\eta}_{ij}\circ y^{\eta}_{jk}&=xU_{(23)}[ij]\circ y U_{(13)}[jk]\\
  &(xU_{(23)})(yU_{(13)})[ik]\\
  &=(xU^{2}_{(23)}\circ y U^{2}_{(13)})[ik]\\
  &=(x\circ y)[ik]\\
  &=(x_{ij}\circ y_{jk})U_{\pi}[ik]\\
  &=(x_{ij}\circ y_{jk})^{\eta}.
\end{align*}

For\pageoriginale \eqref{c2:eq48} let $\pi$ be a permutation such that
$i\pi=1$, $j\pi=2$. Put $x_{ii}U_{\pi}=x\epsilon
\mathscr{J}_{11},y_{ij} U_{\pi}= y\epsilon \mathscr{J}_{12}$. Then
\begin{align*}
  x_{ii}\circ y_{ij}&=xV_{c_{12}}[ii]\circ y[ij]\\
  &=((x V_{c_{12}})y)[ij]\\
  &=(xV_{c_{12}}U_{(23)}\circ yU_{(13)})[ij].
\end{align*}

Since $x\epsilon \mathscr{J}_{11}, xU_{(23)}=x$ (see first paragraph)
and $xV_{c_{12}}=x\circ u_{12}$. Hence $xV_{c_{12}}U_{(23)}=(x\circ
u_{12})U_{(23)}=x\circ u_{12}U_{(23)}=x\circ (u_{12}\circ
u_{23})=x\circ
u_{13}=xV_{c_{13}}=xV_{c_{13}}U_{c_{13}}$(cf. $(45)$)$=(x\circ
u_{13})U_{c_{13}}$. Then $xV_{c_{12}}U_{23}\circ yU_{(13)}=((x\circ
u_{13})\circ y)U_{(13)}=((x\circ y)\circ u_{13})U_{13}= x\circ
y=(x_{ii}\circ y_{ii})U_{\pi}$. Hence $x_{ii}\circ
y_{iij}=(x_{ii}\circ y_{ij})U_{\pi}[ij]=(x_{ii}\circ y_{ij})^{\eta}$.

For \eqref{c2:eq49} choose $\pi$ so that $i\pi=1$ and put
$x=x_{ii}U_{\pi}\epsilon \mathscr{J}_{11}$. Then
\begin{align*}
  (x_{ij})^{2}&=(xV_{c_{12}}[ii])^{2}=(x V_{c_{12}})^{2}[ii]\\
  &=(x V_{c_{12}}U_{(23)}\circ V_{c_{12}}U_{(13)}[ii]\\
  &=(xV_{c_{13}}\circ xV_{c_{12}})U_{(13)}[ii]\quad (\text{proof of}
  (48))\\
  &=((x\circ u_{13})\circ (x\circ u_{12}))U_{(13)}[ii]\\
  &=((x \circ(x\circ U_{12}))\circ u_{13})U_{(13)}[ii]\\
  &=(x^{2}\circ u_{12})U^{2}_{(13)}[ii]\\
  &=(x^{2}\circ u_{12})[ii]\\
  &=x^{2}_{ii}U_{\pi}V_{c_{12}}[ii]\\
  &=(x^{2}_{ii})^{\eta}.
\end{align*}

For\pageoriginale \eqref{c2:eq50} let $\pi$ satisfy $i\pi=1$, $j\pi=2$ and put
$x=x_{ij}U_{\pi}\epsilon \mathscr{J}_{12}$. Then
\begin{align*}
  (x_{ij})^{2}&=x_{ij}U_{\pi}[ij]^{2}=(x[ij])^{2}\\
  &= x\ob{x}[ii]+\ob{x} x[jj].
\end{align*}

Now $x\ob{x}=xU_{(23)}\circ xU_{(12)}U_{(13)}=xU_{(23)}\circ
xU_{(13)}U_{(23)}= (x\circ xU_{(13)})U_{(23)}=(x\circ (x\circ
u_{13}))U_{(23)}=(x^{2}\circ u_{13})U_{(23)}=(x^{2}U_{e_1}\circ
u_{13})U_{(23)}=x^{2}U_{e_{1}}\circ
u_{12}=x^{2}U_{e_{1}}V_{c_{12}}=x^{2}_{ij}U_{\pi}U_{e_{1}}V_{c_{12}}$. Similary,
$\ob{x}x=x^{2}_{ij}U_{\pi}U_{e_{2}}V_{c_{12}}$. Hence
$$
(x_{ij})^{2}=(x^{2}_{ij}U_{\pi}U_{e_{1}}U_{e_{12}})[ii]+(x^{2}_{ij}UU_{e_{2}}V_{c_{12}})[jj].
$$

On the other hand,
$x^{2}_{ij}=x^{2}_{ij}U_{e_{i}}+x^{2}_{ij}U_{e_{j}}$ so
$(x^{2}_{ij})^{\eta}=(x^{2}_{ij}U_{e_{i}}U_{\pi}V_{c_{12}})\break [ii] +(x^{2}_{ij}U_{e_{j}}U_{\pi}U_{(12)}V_{c_{12}}V_{c_{12}})[jj]$ 
and
$x^{2}_1U_{e_{i}}U_{\pi}V_{c_{12}}=x^{2}_{ij}$ $U_{\pi}$ $U_{e_{1}}V_{c_{12}}$,
$x^{2}_{ij}U_{e_{j}}U_{\pi}$ $U_{(12)}V_{c_{12}}=x^{2}_{ij}U_{\pi}U_{e_{2}}U_{c_{12}}V_{c_{12}}=x^{2}_{ij}U_{\pi}U_{e_{2}}V_{c_{12}}$. Hence
\eqref{c2:eq50} holds.

It is clear from these formulas that $(x^{2})^{\eta}=(x^{\eta})^{2}$
for $x\epsilon \mathscr{J}$ and
$\mathscr{H}(\mathscr{O}_n,\mathscr{O}_o)$ is closed under
squaring. We now introduce a $0$-operator in
$\mathscr{H}(\mathscr{O}_n,\mathscr{O}_o)$ by $QM1-11$ and the
formulas giving $0$. Clearly $AU_{B}\epsilon \mathscr{O}_n$ but it is
not immediately clear that $AU_g\epsilon
\mathscr{H}(\mathscr{O}_n,\mathscr{O}_o)$. We claim that
$xU^{\eta})=x^{\eta}U_{y^{\eta}},x,y,\epsilon \mathscr{J}$. It is
sufficient to prove this and\pageoriginale
$\{xyz\}^{\eta}=\{x^{\eta}y^{\eta}z^{\eta}\}$ for $x,y,z$ in Pierce
components. We note first that since $\eta$ maps $\mathscr{J}_{pq}$
into $\mathscr{H}_{pq}$ we have $(xU_{e_i})^{\eta}=x^{\eta}U_{1[ii]}$,
$(x U_{e_i,e_{j}})^{\eta}=x^{\eta}U_{1[ii],1[jj]}$. From this it follows
that we can carry over the proof of theorem \ref{c2:thm1}. For the relation
corresponding to $QM 4$ we have $\{x_{ii}yz\}^{\eta}=(((x_{ii}\circ
y_{ij})\circ z_{ij})U_{e_{i}})^{\eta}=((x_{ii}\circ z_{ij}) \circ
z_{ij}) U_{1[ii]}$
and $\{x^{\eta}_{ii}y^{\eta}_{ij}z^{\eta}_{ij}\}=((x_{ii}\circ
y_{ij})\circ z_{ij})U_{1[ii]}$ by the formulas in
$\mathscr{H}(\mathscr{O}_n,\mathscr{O}_o)$. Hence
$\{x_{ii}y_{ij}z_{ij}\}^{\eta}=\{x^{\eta}_{ii}y^{\eta}_{ij}z^{\eta}_{ij}\}$
holds. The formulas corresponding to $QM 5-11$ are obtained in a
similar manner. For $QM 2$ we note that $u^{\eta}_{jk}=1[jk], j\neq k$
and $V_{1[jk]}$ is injective on $\mathscr{H}_{ij}$ if
$i,j,k\neq$. Hence it suffices to prove
$(x_{ii}U_{y_{ij}})^{\eta}\circ
u^{\eta}_{jk}=x^{\eta}_{ii}U_{y_{ij}}^{\eta}\circ u^{\eta}_{jk}$. Now
$(x_{ii}U_{y_{ij}})^{\eta}\circ
u^{\eta}_{jk}=(x_{ii}U_{y_{ij}}V_{u_{jk}})^{\eta}$
$=(x_{ii}V_{y_{ij},u_{jk}}V_{y_{ij}})^{\eta}$ 
as in the proof of $QM 2$ in Theorem \ref{c2:thm1}) $=$
$(\{x_{ii}y_{ij}u_{jk}\}\circ
y_{ij})^{\eta}=\{x^{\eta}_{ii}y^{\eta}_{ij}u^{\eta}_{jk}\}\circ
y_{ij}^{\eta}=x^{\eta}_{ii}U_{y_{ij}}^{\eta}\circ u^{\eta}_{jk}$ by
the formulas in $\mathscr{H}(\mathscr{O}_n,\mathscr{O}_o)$. A similar
argument applies to $QM 1,3$. Hence $(x
U_y)^{\eta}=x^{\eta}U_{y^{\eta}}$ which with
$\mathscr{J}^{\eta}=\mathscr{H}(\mathscr{O}_n,\mathscr{O}_o)$ implies
that $AU_B\epsilon \mathscr{H}(\mathscr{O}_n.\mathscr{O}_o)$ for
$A,B\epsilon \mathscr{H}(\mathscr{O}_n.\mathscr{O}_o)$. It follows
that $(\mathscr{H}(\mathscr{O}_n, \mathscr{O}_o),U,1)$ is Jordan and
$\eta$ is a homomorphism. It now follows from Theorem \ref{c2:thm1} that
$\mathscr{O}$ is associative if $n\geqq 4$ and alternative with
$\mathscr{O}_o\subseteq N(\mathscr{O})$ if $n=3$. Also
$x\mathscr{O}_o\ob{x}\subseteq \mathscr{O}_o$ if $x\epsilon
\mathscr{O}$.

It remains to prove that $\eta$ is an isomorphism. Let
$\mathfrak{K}=\ker^{\eta}$. Since $\mathfrak{K}=$ is an ideal,
$\mathfrak{K}=\sum \mathfrak{K}_{ij}\mathfrak{K}=\mathfrak{K}\cap
\mathscr{J}_{ij}$ also since $\eta_{ij}$ is injective if $i\neq j$ we
have $\mathfrak{K}_{ij}=0$, if $i\neq j$. Hence
$\mathfrak{K}=\sum\mathfrak{K}_{ii}$ and it suffices to show
$\mathfrak{K}_{ii}=0, i=1,\ldots,n$. Let $z\epsilon\mathfrak{K}_{ii}$
and consider the\pageoriginale products $\{xyz\},x,y$ in Pierce
modules $\mathscr{J}_{pq}$. Such a product is $0$ since
$\mathfrak{K}_{pq}=\mathfrak{K}\cap \mathscr{J}_{pq}=0$ Unless $x,y
\mathscr{J}_{ij},i\neq j$ or $x,y,\epsilon \mathscr{J}_{ii}$. In the
first case we note that $x\circ z=0$ since $\mathfrak{K}_{ij}=0$ so
$\{xyz\}=0$ by $PD 4$. In the second case we write $x=wU_{u_{ij}}$
($u_{ij}$ as above), $w\epsilon \mathscr{J}_{jj}$ where $j\neq
i$. This can be done since
$\mathscr{J}_{jj}U_{u_{ij}}=\mathscr{J}_{jj}U_{(ij)}=\mathscr{J}_{ii}$. Then
$\{xyz\}=wU_{u_{ij}}V_{y,z}=0$ by $QJ 9$, the PD relations and
$\mathfrak{K}_{ij}=0$. Our result implies that $\{xyz\}=0$ for all
$x,y\epsilon \mathscr{J}$ so $U_{x,z}=0$ for all $x$. We show that
$U_{z}=0$. For this it suffices to show that $xU_z=0$ if $x\epsilon
\mathscr{J}_{ii}$ or $wU_{u_{ij}}u_z=0$ for $w\epsilon
\mathscr{J}_{jj},i\neq j$. This follows from $QJ 17$ and the PD
relations. We have now shown that $\mathfrak{K}_{ii}\subseteq\ker
U$. Hence $\mathfrak{K}_{ii}=0$ and the proof is complete.
\end{proof}

\begin{remarks*}
The hypothesis that $\mathscr{J}$ is non-degenerate is used only at
the  last stage of the proof. If this is dropped the argument shows
that $\mathfrak{K}=\ker \eta\subseteq \ker U$. The converse
inequality holds since $(\ker U)^{\eta}$ is contained in the radical
of $U$ in
$\mathscr{J}^{\eta}=\mathscr{H}(\mathscr{O}_n,\mathscr{O}_o)$. Since
$\mathscr{H}(\mathscr{O}_n,\mathscr{O}_o)$ is non-degenerate we have
$(rad U)^{\eta}=0$ so $\ker U\equiv\mathfrak{K}$ and $\mathfrak{K}=\ker
U$ in any case. This gives a characterization of the algebras
satisfying the first hypothesis of the S.C.T. IF $\mathscr{J}$ has no
two torsion $\ker u=0$ so in this case we can drop the second
hypothesis and obtain the conclusion of S.C.T.

The S.C.T can be strenghtened to give a coordinatization Theorem in
which the second hypothesis is replaced by the weaker one that the
$e_i$ are connected. In this case we get an isomorphism onto a
``canonical matrix algebra'' (cf. Jacobson \cite{Jacobson2}, p.137).
\end{remarks*}
