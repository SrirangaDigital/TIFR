\chapter{Basic Concepts}\label{c1}

In\pageoriginale this chapter we give the basic definitions and
general results for\break  quadratic Jordan algebras. These algebras are
$\Phi$-modules for a commutative ring$\Phi$ equipped with a
multiplicative composition which is linear in one of the variables and
quadratic in the other. If the base ring $\Phi$ contains
$\frac{1}{\mathscr{L}}$ then the notion of  aquadratic Jordan algebra
is equivalent to the usual notion of a (linear) Jordan algebra (see \S
$4$). The results of this chapter parallel those of Chapter I and a
part of Chapter II of the author's book \cite{JacobsonMcCrimmon1}. 

\section{Special Jordan and quadratic Jordan algebras}\label{c1:sec1}

It will be convenient from now on to deal with algebras over a
(unital) commutative ring $\Phi$. An associative algebra
$\mathfrak{a}$ over $\Phi$ is a left (unital) $\Phi$-module together
with a product $xy$ which is $\Phi$-bilinear and associative. The
results of chapter $0$ carry over without change to algebras. We
remark that rings are just algebras over $\Phi=\mathbb{Z}$ the ring of
integers. 

Let $(\mathfrak{a}, J)$ be an associative algebra with involution and
let $\mathscr{H}(\mathfrak{a}, J)$ denote the subset of symmetric
elements $(a^{J}=a)$ of. It is clear that
$\mathscr{H}(\mathfrak{a},J)$ is a $\Phi$-submodule. What other
closure proportions does $\mathscr{H}(\mathfrak{a},\break J)$ have? Clearly
if $a\in \mathscr{H}(\mathfrak{a}, J)$ and $n=1,2,3,\ldots$ then
$a^{n}\in \mathscr{H}=\mathscr{H}(\mathfrak{a},J)$. In particular,
$a^{2}\in \mathscr{H}$ and hence $ab+ba=(a+b)^{2}-a^{2}-b^{2}\in
\mathscr{H}$ if $a,b\in \mathscr{H}$. We note also that if, $a,b\in H$
then $aba\in \mathscr{H}$. We now observe that this last fact implies
all the others since $\mathscr{H} is a \Phi$-module and contains $1\in
\mathfrak{a}$. For, let $\mathscr{J}$ be any $\Phi$-submodule of
$\mathfrak{a}$ containing $1$ and $aba$ for every $a,b\in
\mathscr{J}$. Then $\mathscr{J}$ contains\pageoriginale
$abc+cba=(a+c)-aba-cba, a,b,c$ in $\mathscr{J}$. Hence contains
$ab+ba=ab1+1ba$. Also $\mathscr{J}$ contains $a^{2}=a1a, a^{3}=aaa$
and $a^{n}=aa^{n-2}a,n\ge 4$. In view of this it is natural to
consider $aba$ as the primary composition in $\mathscr{H}$ besides the
module composition and the property that $\mathscr{H}$ contains $1$. 

There is one serious drawback in using the composition $aba$, namely,
this is quadratic in $a$. It is considerably easier to deal with
bilinear compositions. We now note that if $\Phi$ contains an element
$\frac{1}{2}$ such that $\frac{1}{2}+\frac{1}{2}=1$ (necessarily
unique)then we can replace $aba$ by the bilinear product $a\cdot
b=\frac{1}{2}(ab+ba)$. More precisely, let $\mathscr{J}$ be a
$\Phi$-submodule of the associative algebra $\mathfrak{a}$ such that
$1\in \mathscr{J}$. Then $\mathscr{J}$ is closed under $a.b$ if and
only if it closed under $aba$. We have see that if $\mathscr{J}$ is
closed under $aba$ then it is closed under $ab+ba$, hence, under
$a\cdot b=\frac{1}{2}(ab+ba)$. Conversely, if $\mathscr{J}$ is closed
under $a\cdot b$ then it is closed under $aba$ since 
$$
2(a,b),a=\frac{1}{2}(b a^{2}+a^{2}b)+aba
$$
so
\begin{equation*}
  2(b,a)\cdot a-b,a^{2}=aba.\tag{1}\label{c1:eq1}
\end{equation*}
These observations lead us to define (tentatively) a {\em special quadratic Jordan algebra} $\mathscr{J}$ as a $\Phi$-submodule of an associative algebra $\mathfrak{a}/\Phi$, $\Phi$ a commutative associative ring (with $1$) containing $1$ and $aba$ for $a,b\in \mathscr{J}$. We call $\mathscr{J}$ a {\em special (linear) Jordan algebra} if $\Phi$ contains $\frac{1}{2}$. In this case the closure conditions are equivalent to : $1\in \mathscr{J}$ and $a\cdot b=\frac{1}{2}(ab+ba)\in \mathscr{J}$ if $a,b\in \mathscr{J}$. We have seen that if $\mathfrak{a}, J)$ is an associative algebra with involution then $\mathscr{H}(\mathfrak{a},J)$ the set of $J$-symmetric elements\pageoriginale is a special qudratic Jordan algebra. Of course, $\mathfrak{a}$ itself is a special quadratic Jordan algebra. We now give another important example as follows.

Let $V$ be a vector space over a field $\Phi$, $Q$ a quadratic form on
$V$ and $C(V,Q)$ the corresponding Clifford algebra. Thus if $T(V)$ is
the tensor algebra $\Phi\oplus V\oplus(V\otimes V)\oplus\ldots\oplus
V^{(i)}\ldots, V^{(i)}=V\otimes V\otimes\ldots \otimes V$ ($i$ times)
with the usual multiplication then $C(V,Q)=T(V)/\overline{k}$ where
$\overline{k}$ is the ideal in $T=T(V)$ generated by the elements
$x\otimes x-Q(x),x\in V$. It is known that the mapping $\alpha + x\to
\alpha +x +\overline{k}$ of $\Phi \oplus V$ into $C=C(V,Q)$ is
injective. Hence we may identify $\Phi\oplus V$ with the corresponding
subspace of $C$. Then $C$ is generated by $\Phi \oplus V$ and we have
the relation $x^{2}=Q(x),x\in V$, in $C$. We claim that
$\mathscr{J}\equiv\Phi+V$ is a special quadratic Jordan algebra in
$C$. Let $a=\alpha+x, b=\beta+y, \alpha, \beta \in V$. Then
$aba=(\alpha+x)(\beta+y)(\alpha+x)=\alpha^{2}\beta+2\alpha\beta
x+\alpha^{2}y+\alpha(xy+yx)+\beta x^{2}+xyx$. Now $x^{2}=Q(x)$ gives
$xy+yx=(x+y)^{2}-x^{2}y^{2}=Q(x,y)$ where $A(x,y)=Q(x+y)-Q(x)-Q(y)$ is
the symmetric bilinear form associated with $Q$. Hence
$xyx=-yx^{2}+Q(x,y)x$ and  
\begin{align*}
  aba=(\alpha^{2}\beta&+\alpha Q(x,y)+\beta Q(x))+(2\alpha\beta +Q(x,y))x\\
  &+(\alpha^{2}-Q(x))y\tag{2}\label{c1:eq2}
\end{align*}
$\in \mathscr{J}=\Phi+V$. Since $\mathscr{J}\oplus 1$ and is a
subspace of $C/\Phi$ it is clear that $\mathscr{J}$ is a special
quadratic Jordan algebra. 

\section{Definition of Jordan and quadratic Jordan algebras}\label{c1:sec2}

These\pageoriginale notions arise in studying the properties of the compositions
$a,b=\frac{1}{2}(ab+ba)$ and $aba$ is an associative algebra, over
$\Phi$ where in the first case $\Phi\ni\frac{1}{2}$. We note that
$a\cdot b=b\cdot a$ and if $a^{2}=a.a$ then $(a^{2},b)\cdot
a=\frac{1}{4}[(a^{2}b+ba^{2})a+a(a^{2}b+ba^{2})]
=\frac{1}{4}(a^{2}ba+aba^{2}+ba^{3}+a6^{3}b), 
a^{2}\cdot(b\cdot a)=\frac{1}{4}
a^{2}(ab+ba)+(ab+ba)a^{2}=\frac{1}{4}(a^{3}b+ba^{3}+a^{2}ba+aba^{2})$. These
observations and the fac, which can be verified by
experimentation, that other simple identities on $a\cdot b$ are
consequences of $a\cdot b=b\cdot a$ and $(a^{2}\cdot b)\cdot
a=a^{2}\cdot(b\cdot a)$ lead to the following 


\begin{dashdefi}
  An algebra $\mathscr{J}$ over $\Phi$ is called a {\em (unital
    linear) Jordan algebra} if $1) \Phi$ contains $\frac{1}{2}, 2)
  \mathscr{J}$ contains an element 1 such that $a\cdot 1=a=1\cdot a,\, a\in \mathscr{J}$,
  $3)$ the product $a\cdot b$, satisfies $a\cdot b=b\cdot a$,
  $(a^{2}\cdot b)\cdot a=a^{2}\cdot (b\cdot a)$ where $a^{2}=a\cdot
  a$. 

  It is clear that if $\mathscr{J}$ is a special Jordan algebra then
  $\mathscr{J}$ is  a Jordan algebra with $a\cdot
  b=\frac{1}{2}(ab+ba)$. If $\mathfrak{a}$ is an associative algebra
  over $\Phi \ni\frac{1}{2}$ then $\mathfrak{a}$ defines the Jordan
  algebra $\mathfrak{a}^{+}$ whose underlying $\Phi$-module is
  $\mathfrak{a}$ and whose multiplication composition is $a\cdot
  b=\frac{1}{2}(ab+ba)$. 

  If $\mathscr{J}$ is Jordan we denote the mapping $x\to x\cdot a$ by
  $R_{a}$. This is a $\Phi$-endomorphism of $\mathscr{J}$. We can
  formulate the Jordan conditions on $a\cdot b$ in terms of $R_a$ and
  this will give our preferred definition of a Jordan algebra as
  follows: 
\end{dashdefi}

\begin{defn}\label{c1:defn1}
  A{\em (unital linear)Jordan algebra} over a commutative ring $\Phi$
  (with 1)containing $\frac{1}{2}$ is a triple $(\mathscr{J},R,1)$
  such that $\mathscr{J}$ is a (unital) left $\Phi$-module, $R$ is a
  mapping of $\mathscr{J}$ into End $\mathscr{J}$ (the associative
  $\Phi$-algebra of endomorphisms of $\mathscr{J}$) such that 
  \begin{enumerate}[\rm J1]
  \item $R:a\to R_a$ is a $\Phi$-homomorphism.\pageoriginale
    
  \item $R_1=1$
    
  \item $R_aR_{aR_{a}}=R_{aR_{a}} R_a$
    
  \item If $L_a$ is defined by $xL_a=aR_x$ then $L_a=R_a$.

    Definitions $1$ and $1'$ are equivalent: If $\mathscr{J}$ is
    Jordan in the sense of Defintion $1'$ then we define $R_a$ as
    $x\to x\cdot a$ and obtain $J_l-J_4$ of definition
    $1$. Conversely, if $\mathscr{J}$ is Jordan in the sense of
    definition $1$ then we define $a\cdot b=aR_b$. Then the conditions
    of Definition $1'$ hold. Moreover, the passage from the bilinear
    composition $a\cdot b$ to the mapping $R$ is the inverse of that
    from $R$ to $a\cdot b$. 

    Let $\mathscr{J}$ be Jordan in the sense of the second definition
    and write $a\cdot b=a R_b=bR_a, a^{2}=a\cdot a$. Then
    $[R_aR_{a^{2}}]=0$ where $[AB]=AB-BA$ for the $\Phi$-endomorphisms
    $A, B$. Let $(a)=[R_a, R_{a^{2}}]$ and consider the identity 
    $$
    0=f (a+b+c)-f (a+b)-f (b+c)-f (a+c)+f (a)+f (b)+f (c). 
$$
This gives
\begin{align*}
[R_a R_{b\cdot c}]&+[R_aR_{c\cdot b}]+[R_bR_{a\cdot c}]+[R_b R_{c\cdot a}]\\
&+[R_c R_{a\cdot b}]+[R_c R_{b\cdot a}]=0,
\end{align*}
as is readily checked. Since $a\cdot b=b\cdot a$ we get
$2[R_a R_{b\cdot c}]+2[R_b R_{a\cdot c}]+2[R_c R_{a\cdot b}]=0$. Since $\Phi$ contains $\frac{1}{2}$\pageoriginale we obtain
\item $[R_a R_{b\cdot c}]+[R_b R_{a\cdot c}]+[R_c R_{a\cdot b}]=0$.
\end{enumerate}
\end{defn}

Let $\underline{\rho}$ be a commutative associative algebra over
$\Phi$ ($=$ commutative associative ring extension of $\Phi$). If $m$
is a $\Phi$-module we write
$m_{\underline{\rho}}=\underline{\rho}\otimes_{\Phi} m$ regarded as
(left unital)$\underline{\rho}$-module in the usual way. We have the
$\Phi$ -homomorphism $\nu : x\to 1 \otimes x$ of $m$ into
$m_{\underline{\rho}}$ as $\Phi$-module. In the cases in which this is
injective we shall identify $x$ and $1\otimes x$ and $m$ and its image
$1\otimes m(=m^{\nu})$. In any case $1\otimes m$ generates
$m_{\underline{\rho}}$ as $\underline{\rho}$-module. If $n$ is a
second $\Phi$-module and $\eta$ is a homorphism of $m$ into $n$ then
there exists a unique homorphism $\eta_{\underline{\rho}}$ of
$m_{\underline{\rho}}$ into $n_{\underline{\rho}}$ such that 
\begin{equation*}
\vcenter{
  \xymatrix{
    m\ar[d]_{\nu}\ar[r]^{\eta}& n\ar[d]^{\nu}\\
    m_{\underline{\rho}}\ar[r]_{\eta_{\underline{\rho}}}& n_{\underline{\rho}}
  }\tag{3}\label{c1:eq3}}
\end{equation*}
is commutative. It follows that if  $\gamma$ End $m$ and
$\tilde{\gamma}$ denotes the 
resultant of $\nu$: End $m\to$(End $m$)$_{\underline{\rho}}$ and the
canonical mapping of (End $m$)$_{\underline{\rho}}$ into End
$m_{\underline{\rho}}$. Then we have a unique homomorphism
$\widetilde{\eta}$ of $m_{\underline{\rho}}$ into End
$m_{\underline{\rho}}$ such that 
\[
\vcenter{
\xymatrix{
m\ar[d]_{\nu}\ar[r]^{\eta}& \text{End}~ m\ar[d]^{\widetilde{\nu}}\\
m_{\underline{\rho}}\ar[r]_{\widetilde{\eta}}& \text{End}~ m_{\underline{\rho}}
}}\tag{4}\label{c1:eq4}
\]\pageoriginale
is commutative.

Now supppose $\Phi\ni\frac{1}{2}$ and $(\mathscr{J},R,1)$ is a Jordan
algebra over $\Phi$. Let $\widetilde{R}$ be homomorphism of
$\mathscr{J}_{\underline{\rho}}$ into End
$\mathscr{J}_{\underline{\rho}}$ determined as in (4) by $R$ and put
$\widetilde{1}=1\otimes1$. If we use the definition of $\mathscr{J}
J5$, and the fact that $1\otimes\mathscr{J}$ generates
$\mathscr{J}_{\underline{\rho}}$ it is straight forward to check that
$(\mathscr{J}_{\underline{\rho}},\widetilde{R}, \widetilde{1})$ is a
Jordan algebra. 

We formulate next the notion of a (unital) quadratic Jordan
algebra. This is arrived at by considering the properties of the
product $aba$ in an associative algebra or, equivalently, the mapping
$U_a:x\to a\times a$. Note that $U_a\in$ End $\mathfrak{a}$ where
$\mathfrak{a}$ the given associate algebra. Also $U: a\to U_a$ is
quadratic is in the sense of the following  

\begin{defn}\label{c1:defn2}
Let $\mathfrak{m}$  and $\mathfrak{n}$ be left (unital)
$\Phi$-modules, $\Phi$-modules, $\Phi$ an arbitrary (unital)
commutative associative ring. Then a mapping $Q:a\to Q(a)$ (or $Q_a$)
of $\mathfrak{n}$ into $\mathfrak{m}$ is called {\em quadratic} if $1)
Q(\alpha a)=\alpha^{2}Q(a), \alpha \in \Phi, a\in \mathfrak{m}, 2)
Q(a,b)\equiv Q(a+b)-Q(a)-Q(b)$ is $\Phi$\pageoriginale-bilinear from
$\mathfrak{m}$ to $\mathfrak{n}$. The {\em kernel} of $Q$ is the set
of $z$ such that $Q(z)=0=Q(a,z),a\in \mathfrak{m}$. 

The associated $\Phi$-bilinear mapping $Q(a,b)$ is symmetric:
$Q(a,b)=Q(b,a)$. The kernel ker $Q$ is a submodule. If $Q$ and $Q'$
are quadratic mappings of $\mathfrak{m}$ into $\mathfrak{n}$ then so
is $Q+Q'$ and $\beta Q, \beta \in \Phi$. Hence the set of these
mappings is a $\Phi$-module. The resultant of a quadratic mapping and
a $\Phi$ -homomorphism and of a $\Phi$ -homomorphism and a quadratic
mapping is a quadratic mapping. If $Q$ is a quadratic mapping of
$\mathfrak{m}$ into $\mathfrak{n}$ and $R$ is contained in ker $Q$
then $Q(a+R)=Q(a)$ defines a quadratic mapping of
$\overline{\mathfrak{m}}=\mathfrak{m}/R$ into $\mathfrak{n}$. If $Q$
and $Q'$ are quadratic mappings and
$Q(a_i)=Q'(a_i),Q(a_i,a_j)=Q'(a_i,a_j)$ for all $a_i,a_j$ on a set of
generators $\{a_i\}$ then $Q=Q'$. In particular, if $Q(a_i)=0$, $A(a_i
, a_j)=0$ then
$Q=0$. Let $\mathscr{F}$ be a free left module with base $\{x_i | i\in
I\}$ and let $i\to b_i,\{i,j\}\to b_{ij}$ be mappings of the index set
$I$ and of the set $I_2$ of distinct unordered paris of elements
$i_{1j}, i_{1j} I$ into $n$. 

If $x\in\mathscr{F}$ and $x=\sum\xi_{i} x_i$ (finite sum) then we
define $Q(x)=\sum \xi_i^{2} b_i+\displaystyle{\sum_{i<j}}\xi_i\xi_jb_{ij}$. Then it
is easy to check that $Q$ is a quadratic mapping of $\mathscr{F}$ into
$n$. It $\mathscr{F}$ is free with base $\{x_i\}$ and
$\underline{\rho}$ is a commutative associative algebra over $\Phi$
then $\mathscr{F}_{\ub{\rho}}$ is free with base $\{1\otimes x_i\}$. It follows the remark
just made that the following lemma holds for $m=\mathscr{F}$ free: 
\end{defn}

\begin{lemma*}
Let\pageoriginale $Q$ be a quadratic mapping of $\mathfrak{m}$ into $\mathfrak{n}$ where these are left modules over $\Phi$ and let $\underline{\rho}$ be an associative commutative ring extension of $\Phi$. Then exists a unique quadratic mappings $Q_{\underline{\rho}}$ of $\mathfrak{m}_{\underline{\rho}}$ into $\mathfrak{n}_{\underline{\rho}}$ such that the following diagram is commutative
\end{lemma*}
\[
\vcenter{
\xymatrix{
\mathfrak{m}\ar[d]_{\nu}\ar[r]^{Q}& \mathfrak{n}\ar[d]^{\nu}\\
\mathfrak{m}_{\underline{\rho}}\ar[r]_{Q_{\underline{\rho}}}& \mathfrak{n}
}}\tag{5}\label{c1:eq5}
\]

\begin{proof}
Let $\mathscr{F}\xrightarrow{\eta} \mathfrak{m} \to 0$ be an exact
sequence of modules where $\mathscr{F}$ is free and put
$\mathfrak{K}= \ker \eta$ the kernel of $\eta$. Then we have the
corresponding homomorphism $\eta_{\underline{\rho}}$ of
$\mathscr{F}_{\underline{\rho}}$ onto $m_{\underline{\rho}}$ (as in
(3)) and, as is well-known, $\ker
\eta_{\underline{\rho}}=\underline{\rho}(1\otimes \mathfrak{K})$ the
${\underline{\rho}}$ -submodule generated by $1\otimes
\mathfrak{K}=\{1\otimes k| k\in \mathfrak{K}\}$. We have the
isomorphism $\widetilde{x}+{\underline{\rho}}(1\otimes
\mathfrak{K})\to \widetilde{x}^{\eta_{\underline{\rho}}}$ of
$\mathscr{F}/{\underline{\rho}}(1\otimes \overline{k})$ onto
$m_{\underline{\rho}}$. We define $Q^{\eta}$ of $\mathscr{F}$ to $\eta$
by $Q(x)=Q(x^{\eta}), x\in \mathscr{F}$. Since this is the resultant
of $\eta$ and $Q$ it is aquadratic mapping. Also $\ker
Q^{\eta}\supseteq \mathfrak{K}$. Since
$\mathscr{F}_{\underline{\rho}}$ is $\underline{\rho}$ -free
$Q^{\eta}$ determines the quadratic mapping
$Q^{\eta}_{\underline{\rho}}$ of $\mathscr{F}$ into $\eta$ so that $(5)$
is commutative for $m=\mathscr{F}$. We have ker
$Q^{\eta}_{\underline{\rho}}\supseteq \underline{\rho}(1\otimes
\mathfrak{K})$. Hence $\widetilde{x}+\underline{\rho}(1\otimes
\mathfrak{K})\to Q^{\eta}_{\underline{\rho}}(\widetilde{x})$ is a
quadratic mapping of
$\mathscr{F}_{\underline{\rho}}/\underline{\rho}(1\otimes
\mathfrak{K})$ into $n_{\underline{\rho}}$. Using the isomorphism of
$\mathscr{F}_{\underline{\rho}}/\underline{\rho}(1\otimes
\mathfrak{K}$) and $m_{\underline{\rho}}$ this can be transferred to
the quadratic mapping
$Q_{\underline{\rho}}:\overline{x}^{\eta_{\underline{\rho}}}\to
Q_{\underline{\rho}}(\overline{x})$ of $m_{\underline{\rho}}$ into
$n$. If $x\in \mathscr{F}$ then $1\otimes x^{\eta}=(1\otimes
x)^{\eta_{\underline{\rho}}}\to Q^{\eta}_{\underline{\rho}}(1\otimes
x)=1\otimes Q^{\eta}(x)=1\otimes Q(x^{\eta})$. Hence
$Q_{\underline{\rho}}$ satisfies the commutativity in\pageoriginale
(5). The uniqueness of $Q_{\ub{\rho}}$ is clear since $1\otimes
\mathfrak{m}$ generates $\mathfrak{m}_{\underline{\rho}}$.  

Let $\mathfrak{n}=$ End $\mathfrak{m}$. Then it is immediate from the
lemma that if $Q$ is a quadratic mapping of $\mathfrak{m}$ into End
$\mathfrak{m}$ then there exists a unique quadratic mapping
$\widetilde{Q}$ of $\mathfrak{m}_{\underline{\rho}}$ into End
$\mathfrak{m}_{\underline{\rho}}$ such that commutativiy holds in: 
\begin{equation*}
\vcenter{\xymatrix{
\mathfrak{m}\ar[d]_{\nu}\ar[r]^{Q}&
\text{End}\mathfrak{m}\ar[d]^{\widetilde{\nu}}\\ 
\mathfrak{m}_{\underline{\rho}}\ar[r]_{\widetilde{Q}}&
\text{End}\mathfrak{m}_{\underline{\rho}} }}\tag{6}\label{c1:eq6}
\end{equation*}
where $\tilde{nu}$ is $A\to 1\otimes A$ and $(\rho \times A)=\rho
\otimes x A$.

We are now ready to define a quadratic Jordan algebra. We have two
objectives in mind: first, to give simple axioms which will be
adequate for studying the composition $ax a=xU_a$ in associative
algebras and second, to characterize the mapping $U_a\equiv
2R^{2}_a-R_{a^{2}}$ in Jordan algebras. We recall that in an
associative algebra $xU_a=x(2R^{2}_a-R_{a^{2}})$. The following
definiion is due to McCrimmon \cite{McCrimmon1}. 
\end{proof}

\begin{defn}\label{c1:defn3}
  A {\em (unital) quadratic Jordan algebra} over a commutative
  associative ring $\Phi$ (with $1$) is a triple $(\mathscr{J},U,1)$
  where $\mathscr{J}$ is a (unital) left $\Phi$ -module, $1$ a
  distinguished element of $\mathscr{J}$ and $U$ is a mapping of
  $\mathscr{J}$ into End $\mathscr{J}$ such that 
\end{defn}
\pageoriginale
\begin{enumerate}[\rm QJ1]
\item $U$ is quadratic

\item $U_1=1$

\item $U_a U_b U_a= U_b U_a$

\item If $U_{a,b}=U_{a+b}-U_a-U_b$ and $V_{a,b}$ is defined by
  $xV_{a,b}=a U_{x,b}$ then $U_b V_{a,b}=V_{b,a} U_b.$

\item If $\underline{\rho}$ is any commuatative associative algebra
  over $\Phi$ and $\widetilde{U}$ is the quadratic mapping of
  $\mathscr{J}_{\underline{\rho}}$ into End
  $\mathscr{J}_{\underline{\rho}}$ as in $(6)$, then $\widetilde{U}$
  satisfies $QJ3$ and $5$. 
\end{enumerate}

It is clear from $QJ_5$ that
$(\mathscr{J}_{ub{\rho}},\widetilde{U},\widetilde{1}), 
\widetilde{1}=1\otimes 1$ is a quadratic Jordan algebra over
$\underline{\rho}$. We remark also that $QJ_4$ states that $b U_{a,x}
U_{b}=a U_{b, x U_b}\cdot$. Since the left side is symmetric in $a$ and
so is the right. Hence $a U_{b, x U_b}=xU_{b,aU_b}\cdot$. This gives
the following addendum to $QJ_4$: 
\begin{equation*}
  U_b V_{a,b}=V_{b,a} U_b=U_{aU_{b},b}\cdot\tag*{$QJ4'$}
\end{equation*}

Let $\mathfrak{a}$ be an associative algebra over $\Phi$ and define
$U_a$ to be $x\to a x a$. Then $U_a\in$ End $\mathfrak{a}$ and
$QJ1$ and $QJ2$ are evidently satisfied since $U_{a,b}$ is the mapping
$x\to a x b+b x a$. We have $xU_a U_b U_a=a(b(a x a)b)a$
and $xU_{bU_a}=xU_{aba}=(aba)x(aba)$ so $QJ3$ holds by the associative
law. Now $xV_{a.b}=a U_{x,b} = xab+bax$. Hence $xU_b
V_{a,b}=b(xba+abx)b=bxbab+babxb$ and\pageoriginale
$V_{b,a}U_b=b(xba+abx)b=bx bab+bab x b$. Thus $QJ4$ holds. Now $QJ5$
is clear since $\mathfrak{a}_{\underline\rho}$ is an associative
algebra and the mapping $\widetilde{U}$ of
$\mathfrak{a}_{\underline{\rho}}$ is $\widetilde{a}\to
\widetilde{U}_a$ where
$xU_{\widetilde{a}}=\tilde{a}\,\tilde{x}\,\tilde{a}$. We
denote $(\mathfrak{a}, U, 1)$ by $\mathfrak{a}$. 

If $(\mathscr{J},U,1)$ is a quadratic Jordan algebra, a {\em
  subalgebra} $\mathscr{B}$ of $\mathscr{J}$ is a $\Phi$-submodule
containing $1$ and every $a U_b,a,b\in\mathscr{B},a$ {\em
  homomorphism} $\eta$ of $\mathscr{J}$ into a second quadratic Jordan
algebra is a module homomorphism such that $1^{\eta}=1,
(aU_b)^{\eta}=a^{\eta}U_b^{\eta}$. Monomorphism, isomorphism,
automorphism are defined in the obvious way. A quadratic Jordan
algebra $\mathscr{J}$ will be called {\em special} if there exists a
monomorphism of $\mathscr{J}$ into algebra
$\mathfrak{a}(q),\mathfrak{a}$ associative. It is immediate that this
is essentially the same definition we gave before. 

If $\mathfrak{a}$ is an associative algebra over $\Phi\ni\frac{1}{2}$
then we can form the Jordan algebra $\mathfrak{a}^{+}$ and the
quadratic Jordan algebra $\mathfrak{a}(q)$. For $\mathfrak{a}^{+},
xR_a=\frac{1}{2}(a+a)$ and for $a^{(q)}$, $xU_a=axa$. The relation
\eqref{c1:eq1} :
$2(x,a),a-xa^{2}=axa$ shows that 
\begin{equation*}
U_a=2R^{2}_a-R_{a^{2}}\tag{7}\label{c1:eq7}
\end{equation*}
is the formula expressing $U$ in terms of $R$. Conversely, we can
express $R$ in terms of $U$ by noting that $U_{a,b}=U_{a+b}- U_a-U_b$ is $x\to
a x b + b x a$ so $V_a\equiv U_{a,1} = U_{1,a}$ is $x\to
ax+xa$. Hencec 
\begin{equation*}
R_a=\frac{1}{2}V_a,V_a=U_{a,1}=U_{1,a}.\tag{8}
\end{equation*}

Now\pageoriginale let $(\mathscr{J},R,1)$ be any Jordan algebra (over
$\Phi\ni\frac{1}{2}$) and define $U$ by \eqref{c1:eq7}. Then we claim that
$(\mathscr{J},U,1)$ is a quadratic Jordan algebra. It is clear that
$a\to U_a=2R^{2}_a-R_{a^{2}}$ is quadratic in $a$ and $R_1=1$ gives
$U_1=1$. Also, since $\mathscr{J}_{\underline\rho}$ is Jordan for any
commutative associative algebra $\underline{\rho}$ over $\Phi$, it is
enough to prove that $QJ3$ and $QJ4$ hold. We need to recall some
basic identities, namely,  
\begin{itemize}
\item[J6] $R_aR_bR_c+R_cR_bR_a+R_{(a\cdot c)\cdot b}=R_{a\cdot
  b}R_c+R_{b\cdot c}R_a+R_{c\cdot a}R_c$ 

\item[J7] $[R_c[R_a R_b]]=R_{c[R_aR_b]}$.
\end{itemize}

The first of these is obtained by writing $J5$ in element form:
$(d\cdot a), (b\cdot c)-(d\cdot(b\cdot c))a+$etc., interchanging $b$
and $d$ and re-interpreting this as operation identity. This gives
$R_aR_bR_c+R_cR_bR_a+R_{(a\cdot c)\cdot b}=R_aR_{b\cdot c}+R_b
R_{a\cdot c}+ R_cR_{a\cdot b}$. This and $J5$ give $J6$. To obtain
$J7$ we interchange $a$ and  $b$ in $J6$ and subtract the resulting
relations frm $J6$. Special cases of $J5$ and $J6$ are 
\begin{itemize}
\item[$J5'$] $[R_{a^{2}}R_b]+2[R_{a,b}R_a]=0$

\item[$J6'$] $R_a^{2}R_b+R_b R^{2}_a+R_{(a,b),b}=R_{a^{2}}R_b+2R_{a,b}R_a$.
\end{itemize}
using \eqref{c1:eq7} we obtain $U_{a,b}=2(R_aR_b+R_b
R_a)-2R_{a,b}$. Then $xV_{a,b}=a U_{x,b}$ gives 
\begin{equation*}
  V_{a,b}=2(R_aR_b-R_bR_a+R_{a\cdot b})\tag{9}\label{c1:eq9}
\end{equation*}\pageoriginale
We shall now prove $QJ4$, which is equivalent to:
\begin{align*}
(2R^{2}_a-&R_{a^{2}})(R_{a\cdot b}+R_b R_a - R_a R_b)-\\
&(R_{a\cdot b}+R_a R_b -R_b R_a)(2R^{2}_a-R_{a^{2}})=0
\end{align*}
The left hand side of this after a little juggling becomes
\begin{align*}
  2[R^{2}_a R_b&+R_b R_a^{2},R_a]+2R_a[R_a+R_{a\cdot b}]+2[R_a R_{a\cdot b}]R_a\\
  -[R_{a^{2}}&R_b]R_a+R_a[R_b R_{a^{2}}]+[R_a R_{a^{2}}+R_{a^{2}}R_a, R_b]\\
  &+[R_{a\cdot b a^{2}}]\\
  =2[&R^{2}_aR_b+R_b R^{2}_a, R_a]+[R_aR_{a^{2}}+R_{a^{2}}R_aR_b]\\
  &+[R_{a\cdot b}R_{a^{2}}]\cdot (\text{by} J5')\\
  =2[&R_{a^{2}} R_b+2R_{a\cdot b} R_a-R_{(a\cdot b)\cdot a'}
    R_a]+2[R_aR_{a^{2}},R_b]\\ 
  &+2[R_{(a\cdot b)\cdot a}, R_a] (J6' \,\text{and}\, J5' \,\text{with}\, b\to
  a\cdot b)\\ 
  =4[&R_{a\cdot b} R_a]R_a+2[R_{a^{2}}R_b]R_a=0(J5').
\end{align*}
Hence $QJ4$ holds.

For\pageoriginale the proof of $QJ3$ we begin with the following identity
\begin{itemize}
\item[J8.] $[V_{a,b}V_{c,d}]V_{a,bV_{c,d}}-V_{aV_{d,c}},b$
\end{itemize}
(cf. the author's book [2], (5) on p.325). To derive
this we note that $J7$ shows that $[R_aR_b]$ is a derivation in
$\mathscr{J}$. For any derivation $D$ we have directly
$:[V_{a,b}D]=V_{aD,b}+V_{a,bD}$. Also $[V_{a,b} R_c]=V_{a,b R_c}-V_{a
  R_c, b}$ follows directly from $J5$ and $J7$. Then $J9$, is a
consequence of these two relations. We note next that the left hand
side of $J8$ is skew in the pairs $(a,b),(c,d)$. Hence we have the
consequence 
\begin{itemize}
\item[$J9$] $V_{a,bV_{c,d}}-V_{aV_{d,c},b}=V_{c V_{b,a}},d-V_{c,dV_{a,b}}$.
\end{itemize}
We now use the formula $xV_{a,b}=aU_{x,b}$ defining $V_{a,b}$ to write
$J8$ and $J9$ in the following equivalent forms: 
\begin{itemize}
\item[$J8'$] $U_{aU_{c,b},d}-U_{c,d}V_{a,b}=U_{b,d}V_{a,c}-V_{d,a}U-{c,b}$

\item[$J9'$] $V_{c,d}V_{a,b}-V_{d U_{a,c},b}=V_{a,c} U_{b,d} -U_{a,d} V_{c,d}$.
\end{itemize}
Taking $d=aU_b,c=b$ in $J8'$ gives
\begin{equation*}
2U_{aU_{b}}=U_{b,aU_{b}}V_{a,b}-V_{aU_{b},a}U_b\tag{10}\label{c1:eq10}
\end{equation*}

Replacing\pageoriginale $a\to b$, $c\to b$,$ b\to a$, $d\to a$ in $J9'$ gives
\begin{equation*}
V_{aU_{b,a}}=V^{2}_{b,a}-2U_bU_b\tag{11}\label{c1:eq11}
\end{equation*}

If we substitute this in the last term of \eqref{c1:eq10} we get
$2U_{aU_{b}}=U_{b,a} U_{b}$ $V_{a,b} - V^2_{b,a} U_b +2U_bU_aU_b$. Since
$QJ4$ has the consequence 
$QJ4':U_b V_{a,b}=V_{b,a}U_b=U_{aU_{b}},b$ (as above) the
foregoing reduces to $QJ3$\footnote{The proof we have given of $QJ4$
  was communicated to us by McCrommon, that of $QJ3$ by Meyberg. The
  first direct proof of $QJ_3$ was given by Macdonald. Subsequently he
  gave a general theorem on identities from which $QJ3$ and $QJ4$ are
  immediate consequences. See Macdonald [1] and the author's book
  [2] pp.40-48).} 

\section{Basic identities}\label{c1:sec3}

In this section we shall derive a long but of identities which will be
adequate for the subsequent considerations. No attempt has been made
to reduce the set to a minimal one. On the contrary we have tried to
list almost every identity which will occur in the sequel. 

Let $(\mathscr{J},U,1)$ be a quadratic Jordan algebra over $\Phi$. We
write $aba=bU_a, abc=bU_{a,c}$ so $b\to aba$ is the $\Phi$
-endomorphism $U_a$ for fixed $a$ and $a\to aba$ is a quadratic
mapping of $\mathscr{J}$ into itself for fixed $b$. We put
$a^{2}=1U_a, a\circ b=(a+b)^{2}-a^{2}-b^{2}=1U_{a,b}=aV_{1,b}$ and
$V_a=U_{a,1}=U_{1,a}$. We have $a\circ b=b \circ a, a\circ a=2a^{2}$,
$U_{a,a}=2U_a,V_1=V_{1,1}=2$. Taking $b=1$ in $QJ4$ gives
$V_{a,1}=V_{1,a}$ so $1U_{a,} =aU_{1,x}$. Then $a\circ x=aV_x$. Since $a\circ
x=x\circ a$ we\pageoriginale have $xV_a=aV_x:$ Also
$xV_{1,a}=1U_{a,x}=a\circ x =xV_a$ so $V_a=V_{1,a}=V_{a,1}$. We shall
now apply a process of linearization to deduce consequences of $QJ3$
and $QJ4$. This method consists of applying $QJ3$ and $QJ4$ to
$\mathscr{J}_{\underline{\rho}}=\Phi[\lambda]$  the polynomial algebra
over $\Phi$ in the indeterminate $\lambda$. Since $\underline{\rho}$
is $\Phi$ -free the canonical mapping of $\mathscr{J}$ into
$\mathscr{J}_{\underline{\rho}}$ is injective so we may identify
$\mathscr{J}$ with its image $1\otimes \mathscr{J}$ in
$\mathscr{J}_{\underline{\rho}}$ and regard $\widetilde{U}$ as the
unique extension of $U$ to a quadratic mapping of
$\mathscr{J}_{\underline{\rho}}$ into End
$\mathscr{J}_{\underline{\rho}}$. We write $U$ for
$\widetilde{U}$. The elements of $\mathscr{J}_{\underline{\rho}}$ can
be written in one and only one way in the form $a_o+\lambda
a_1+\lambda^{2}a_2+\cdots+\lambda^{n} a_n, a_i\in \mathscr{J}$, and
the endomorphism of $\mathscr{J}_{\underline{\rho}}$ can be written in
one and only way as $A_o+\lambda A_1+\cdots$. where $A_i$ is the
endomorphism in $\mathscr{J}_{\underline{\rho}}$ which extends the
endomorphism $A_i$ of $\mathscr{J}$. Now let $a,b,c\in \mathscr{J}$
and consider the identity $U_{a+\lambda c}U_bU_{a+\lambda
  c}=U_{bU_{a+\lambda_{c}}}$ which holds in
$\mathscr{J}_{\underline{\rho}}$ by $QJ5$. Comparing coefficients of
$\lambda$ and $\lambda^{2}$ we obtain 
\begin{itemize}
\item[QJ6] $U_aU_bU_{a,c}+U_{a,c}U_b U_a=U_{bU_a}=U_{bU_{a},bU_{a,c}}$

\item[QJ7] $U_a U_b U_c+U_c U_b U_a+ U_{a,c} U_b U_{a,c}=U_{b
  U_a,bU_c}+U_{b, U_{a,c}}$. 
\end{itemize}
The same method applied to the variable $a$ in $QJ6$ and $b$ in $QJ4$ gives
$$
U_a U_b U_{c,d}+U_{a,c} U_b U_{a,d}+U_{c,d}U_b U_a + U_{a,d} U_b U_{a,c}
$$
\begin{itemize}
\item[QJ8] $=U_{b U_{a,c},b U_{a,d}}+U_{bU_{a},b U_{c,d}}$

\item[QJ9] $V_{b,a} U_{b,c}+V_{c,a}U_b=U_{b,c}V_{a,b}+U_b V_{a,c}$
\end{itemize}\pageoriginale

We remark that comparison of the coefficients of the other powers of
$\lambda$ in the foregoing identities yields identitites which we have
already displayed. Also, if the method is applied to a variable in
which the identity is quadratic, say $Q(a)=0$ (e.g. $QJ3$ and the
variable $a$) then  we obtain in this way the bilinerizaation
$Q(a,b)=Q(a+b)-Q(a)-Q(b)=0$. We shall usually not display these
bilinearizations.\footnote{An exception to this rule in $QJ8$ which is
  the bilinerization of $QJ7$ with respecet to $c$}. 

We shall now show that if $(\mathscr{J},U,1$) satisifies $QJ1-4$,
$6-9$ then $QJ5$ holds, so $(\mathscr{J}, U, 1)$ is a quadratic Jordan
algebra. Hence $QJ\, 1-4, 6-9$ constitute an intrinsic set of conditions
defining quadratic Jordan algebras. Let $\underline{\rho}$ be any
commutative associative algebra over $\Phi$ and consider
$\mathscr{J}_{\underline{\rho}}$ where it is assumed that
$(\mathscr{J},U,1)$ satisfies $QJ1-4,6-9$. If $a\in \mathscr{J}$ we
put $a'=1\otimes a$. Then $QJ1-4, 6-9$ hold in
$\mathscr{J}\underline{\rho}$ for all choices of the arguments in
$\mathscr{J}'=1\otimes \mathscr{J}$. Also the bilinearizations of
these conditions hold for all values of the argument in
$\mathscr{J}'$. Since $QJ7-QJ9$ are either $\Phi$-linear
($=\Phi$. endomorphisms) or $\Phi$ -quadratic in their arguments these
hold for all choices of the arguments in
$\mathscr{J}_{\underline{\rho}}$. Similarly, $QJ6$ holds for all $a\in
\mathscr{J}'$ and all $b,c$ in $\mathscr{J}_{\underline{\rho}}$. The
validity of $QJ8$ in $\mathscr{J}_{\underline{\rho}}$ now implies that
if $QJ6$ holds for the arguments $(a=a_1, b, c)$ and $(a_2, b, c)$ in
$\mathscr{J}_{\underline{\rho}}$  then it holds for $(a=a_1+\rho
a_2,b,c)$ for any $\rho \in \underline{\rho}$. 

It\pageoriginale follows from this that $QJ6$ holds in
$\mathscr{J}_{\underline{\rho}}$. A similar argument using $QJ9$ shows
that $QJ4$ holds in $\mathscr{J}_{\underline{\rho}}$. Similarly, using
$QJ6$ and $QJ7$ in $\mathscr{J}_{\underline{\rho}}$ one sees that if
$QJ3$ holds for $b$ in $\mathscr{J}_{\underline{\rho}}$ and $a=a_1,a=a_2$
in $\mathscr{J}_{\underline{\rho}}$ then it holds for $b$ and $a=a_1+\rho
a_2$. It follows that $QJ3$ holds in $\mathscr{J}_{\underline{\rho}}$. We
have therefore proved. 


\begin{thm}\label{c1:thm1}
  Let $\mathscr{J}$ be a left $\Phi$ -module, $U$ a mapping of
  $\mathscr{J}$ into End $\mathscr{J}$ satisfying $QJ1-QJ4,
  QJ6-QJ9$. Then $QJ5$ holds so $\mathscr{J}$ is a quadratic Jordan
  algebra. 
\end{thm}

The same argument implies the following result

\begin{thm}\label{c1:thm2}
  Let $\mathscr{J}$ be a left $\Phi$ -module, $U$ a quadratic mapping
  of $\mathscr{J}$ into End $\mathscr{J}$ such that $U_1=1$ and $QJ3,
  4, 6-9$ and all their bilinearizations hold for all choices of the
  arguments in a set of generators of the $\Phi$ -module
  $\mathscr{J}$. Then $\mathscr{J}$ is a quadratic Jordan algebra. 
\end{thm}

It is easy to prove by a Vandermonde determinant argument that if
$\Phi$ is a field of cardinality $|\Phi|\ge 4$ then $QJ6-9$ follows from
$QJ3, 4$ without the intervention of $QJ5$. Hence in this case $QJ1-4$
are a defining set of conditions for a quadratic Jordan algebra over
$\Phi$.  

If we put $b=1$ in $QJ3,6$ and $7$ we obtain respectively
\begin{align*}
  &U^{2}_a=U_{a^{2}}\tag*{QJ 10}\\
  U_aU_{a,c}&+U_{a,c}U_a=U_{a^{2},aoc}\tag*{QJ 11}\\
  U_{a^{2},c^{2}}&+U_{aoc}=U_a U_c+U_cU_a+U^{2}_{a,c}\tag*{QJ 12}
\end{align*}
If\pageoriginale we replace $b$ by $b+1$ in $QJ 3,6,7$ and use 
the foregoing we obtain
\begin{align*}
  &U_aV_bU_a=U_{bU_{a},a^{2}}\tag*{QJ13}\\
  &U_aV_bU_{a,c}+U_{a,c}V_bU_a=U_{bU_{a},aoc}+U_{a^{2},bU_{a,c}}\tag*{QJ14}\\
  &U_{a^{2},bU_c}+U_{bU_{a},c^{2}}+U_{bU_{a,c},aoc}\tag*{QJ15}\\
  &=U_aV_b U_c + U_c V_b U_a + U_{a,c} V_b U_{a,c}
\end{align*}

Putting $c=1$ in $QJ6, QJ7$ and $QJ9$ gives

\begin{align*}
&U_aU_bV_a+V_aU_bU_a=U_{bU_{a},boa}\tag*{QJ16}\\
&U_{bU_{a},b}+U_{boa}=U_aU_b+U_bU_a+V_aU_bV_a\tag{QJ17}\\
&U_bV_a+V_bV_{a,b}=V_{b,a}V_b+V_aU_b\tag*{QJ18}.
\end{align*}
If we put $a=1$ in $QJ6$ we get
\begin{equation*}
U_bV_c+V_cU_b=U_{b,boc}\tag*{QJ19}.
\end{equation*}
Putting $c=1$ in $QJ12$ and replacing $b$ by $b+1$ in $QJ17$ give respectively:
\begin{align*}
&2U_a=V^{2}_a-V_{a^{2}}\tag*{QJ20}\\
&V_{bU_{a}}+U_{a^{2},b}+2U_{a,aob}=U_aV_b=V_bU_a+V_aV_bV_a.\tag*{QJ21}
\end{align*}
If we apply the two sides of $QJ3$ to $1$ we obtain
\begin{equation*}
a^{2}U_bU_a=(b U_a)^{2}\tag*{QJ22}
\end{equation*}\pageoriginale
which for $b=1$ is
\begin{equation*}
a^{2}U_a=(a^{2})^{2}\tag*{QJ23}
\end{equation*}
Next we put $a=1$ in $QJ4'$ to obtain
\begin{equation*}
V_bU_b=U_bV_b=U_{b,b^{2}}\tag*{QJ24}
\end{equation*}
Putting $b=1$ in $QJ9$ gives
\begin{equation*}
V_aV_c+V_{c,a}=V_cV_a+V_{a,c}\tag*{QJ25}
\end{equation*}
or
\begin{equation*}
(x\circ a)\circ c+c U_{a,}=(x\circ c)\circ a + aU_c,\tag*{QJ 25'}
\end{equation*}
putting $x=c$ we obtain
\begin{align*}
(c\circ a)\circ c+c U_{a,c}&=(c\circ c)\circ a+ a U_{c,c}\\
&=2c^{2}\circ a + 2a U_c
\end{align*}
which can be simplified by $QJ 20$ to give
\begin{equation*}
\{acc\}=c^{2}\circ a.\tag*{QJ26}
\end{equation*}
It is useful to list also the bilinearization of this:
\begin{equation*}
\{abc\}+\{bac\}=(a \circ b)\circ c\tag*{QJ 27}
\end{equation*}
which\pageoriginale has the operator form
\begin{equation*}
V_{b,c}=V_aV_c-U_{b,c}\tag*{QJ $27'$}
\end{equation*}
If we operate with the two sides of $QJ17$ on $1$ we obtain
$$
(a \circ b)^{2}=a^{2}U_b+b^{2}U_a+2aU_b\circ a-b U_a\circ b.
$$

Also, if we replace a by $a+\lambda b$ in $QJ24$ and compare
coefficients of $\lambda$ we obtain 
\begin{align*}
U_{a,b}V_a&+U_bV_b=U_{a,a\circ b}+U_{b,a^{2}}\\\tag*{QJ28}
&V_aU_{a,b}+V_bU_a.
\end{align*}
Applying the first and last of these to $b$ gives
\begin{align*}
bU_a\circ b&=-\{abb\}\circ a+\{ab\circ ab\}+2b^{2}U_a\\
&=-b^{2}V_a^{2}+2b^{2}U_a+\{aa\circ bb\}\tag*{QJ 26}\\
&=-b^{2}\circ a^{2}+\{aa\circ bb\}\tag*{QJ 20}
\end{align*}
which is symmetric in $a$ and $b$. Hence we have
\begin{equation*}
bU_a\circ b=aU_b\circ a\tag*{QJ29}
\end{equation*}
Using this and the foregoing formula for $(a\circ b)^{2}$ we obtain
\begin{align*}
(a\circ b)^{2}&=a^{2}U_b+b^{2}U_a+aU_b\circ a\tag*{QJ 30}\\
&=a^{2}U_b+b^{2}U_a+bU_a\circ b
\end{align*}\pageoriginale
We wish to prove next
\begin{equation*}
V_{aU_{b},a}=V_{b,bU_{a}}\tag*{QJ 31}
\end{equation*}

In element form this is $\{caU_b a\}=\{cbbU_a\}$. Using $QJ27$ this is
equivalent to $(c \circ a U_b)\circ a-\{a U_b c a\}=(c\circ b)\circ b
U_a-\{b c bU_a\}$ which is equivalent to  
\begin{equation*}
V_{aU_{b}}V_a-U_{aU_{b},a}=V_b V_{bU_{a}}-U_{b,bU_{a}}\tag*{QJ 32}
\end{equation*}

If we interchange $a$ and $b$ in $QJ17$ and subtract we obtain $V_a
U_b V_a-V_b U_aV_b=U_{bU_{a},b}-U_{aU_{b},a}$ which implies that $QJ
32$ is equivalent to $V_a U_b V_a-V_b U_a V_b=V_b
V_{bU_{a}}-V_{aU_{b}} V_a$. Hence it suffics to prove 
\begin{equation*}
(V_{aU_{b}}+V_aU_b)V_a=V_b(V_{bU_{a}}+U_aV_b)\tag*{QJ33}
\end{equation*}

We note next that bilinearization of $QJ 29$ relative to $b$ gives $b
U_a\circ c+c U_a\circ b = aU_{b,c}\circ a=c V_{a,b}\circ a$. hence 
\begin{equation*}
  V_{bU_{a}}+U_aV_b=V_{a,b}V_a\tag*{QJ 34}
\end{equation*}

Using this on the right hand side of $QJ33$ gives $V_b V_{a,b}V_a$
also, by $QJ 34$, $V_{aU_{b}}=V_{b,a}V_b-U_bV_a$ so $V_{aU_{b}}+V_a
U_b=V_{b,a}V_b-U_bV_a+V_aU_b=V_bV_{a,b}(QJ18)$. Hence the left hand
side of $QJ33$ reduces to $V_b, V_{a,b} V_a$ also\pageoriginale . This
proves $QJ33$ and with it $QJ32$ and $31$. 

We shall now define the powers of $a$ by $a^{\circ}=1$, $a^{1}=a,a^{2}=1
U_a$, as before, and $a^{n}=a^{n-2}U_{a},n\, 2$. Then $QJ3$ implies
that $U_{a^{n}}=U^{n}_a$. Also by induction on $n$ we have 
\begin{equation*}
(a^{m})^{n}=a^{m^{n}}\tag{QJ 35}
\end{equation*}
We shall now prove
\begin{equation*}
a^{m}\circ a^{n}=2a^{m+n}\tag{QJ 36}
\end{equation*}
by induction on $m+n$. This clear if $n+m\leqq 2$. Moreover, we may
assume $m\leqq n$. We now note that $QJ 36$ will follow if we can show
that $U_aV_{a^{n}}=V_{a^{n}}U_a$. for then $a^{m}\circ
a^{n}=b^{m}V_{a^{n}}=a^{m-2}U_aV_{a^{n}}$ (since $m\geqq
2$)$=a^{n-2}V_{a^{n}}U_a=2a^{n+n-2}U_a=2a^{m+n}$. To prove the
required operation commutativity we shall show that $V_{a^{n}}$ is in
the subalgebra $\mathfrak{a}$ of End $\mathscr{J}$ generated by the
commuting operations $U_a, V_a(QJ 24)$. More genrally, we shall prove
that $U_{a^{m},a^{n}}$ and $V_{a^{m},a^{n}}\in \mathfrak{a}$. Since
$V_{a^{m},a^{n}}= V_{a^{m}} V_{a^{n}}-U_{a^{m},a^{n}}(QJ
27')=U_{a^{m},1}U_{a^{n},1}-U_{a^{m},a^{n}}$ it suffices to show this
for $U_{a^{m},a^{n}}$. We use induction on $m+n$. The result is clear
for $m+n\leqq 2$ by $QJ 20$ and $U_{a,a}=2U_a$ so we assume $m\geqq n,
m\geqq 2$. If $n\geqq 2, U_{a^{n},a^{n}}=U_aU_{a^{n-2},a^{n-2}}U_a$ by
$QJ3$, so the result holds by induction in this case if $n=1,
U_{a^{m},a}=U_{a^{m-2}U_{a,a}}=V_{a,a^{m-2}}U_a$ by $QJ4'$, so the
result is valid in this case. Finally, if $n=0,
U_{a^{m},1}=V_{a^{m}}=V_{a^{m-2}U_a}=V_{a,a^{m-2}}V_a-U_a
V_{a^{m-2}}(QJ 34)$. Hence the result holds in this case also. 
This\pageoriginale completes the proof that
$U_{a^{m},a^{n}},V_{a^{m},a^{n}}\in \mathfrak{a}$ and consequently of
$QJ 36$. 

We shall now prove a general theorem on operator identities involving
the operators $U_{a^{m},a^{n}},V_{a^{m},a^{n}}$ 

\begin{thm}\label{c1:thm3}
  If  $f(\lambda_1, \lambda_2,\ldots)$ is a polynomial in
  indeterminates $\lambda_1,\lambda_2,\dots$ with coefficients in
  $\Phi$ such that
  $f(U_{a^{n_{1}}}$,
  $U_{a^{n_{2}}},\ldots,U_{a^{n_{n},a^{m_{n}}}},\ldots$,
  $V_{a^{n_{1}},a^{m_{1}}}$, $\ldots)=0$ 
  is an identity for all special quadratic Jordan algebras then this
  is an identity for all quadratic Jordan algebras. 
\end{thm}

\begin{proof}
  If $X$ is an arbitrary non-vacuous set then there exists a
  free\break 
  quadratic Jordan algebra $F(X)$ over $\Phi$ (freely) generated by
  $X$ whose characteristic property is that $F(X)$ contains $X$ and
  every mapping $X\to \mathscr{J}$ of $X$ into a quadratic Jordan
  algebra $(\mathscr{J},U,1)$ has a unique extension to a homomorphism
  of $F(X)$ into $(\mathscr{J},U,1)$.\footnote{This is a special case
    of a general result proved in Cohn, {\em Universal Algebra},
    pp.116-121 and p.170. A simple construction of free Jordan
    algebras and more generally of (linear) algebras defined by
    identities is given in Jacobson \cite{Jacobson1}, pp.23-31. It is
    not difficult 
    to modify this so that it applies to quadratic Jordan algebras.}
  Let $X$ contain more than one element one of which is denoted as
  $x$. It is clear from the universal property of $F(X)$ that of
  $f(U_{n1},U_{n2},\ldots,\ldots)=0$ holds in $F(X)$ then
  $f(U_{a^{n1}},U_{a^{n2}},\ldots)=0$ holds in every quadratic Jordan
  algebra. Hence it suffices to prove
  $f(U_{n1},U_{n2},\ldots,\ldots)=0$. Let $Y$ be a set of the same
  cardinality as $X$ and suppose $x \to y$ is a bijective mapping
  of $X$ onto $Y$. Let $\Phi\{Y\}$ be the free associative
  algebra\pageoriginale (with $1$) generated by $Y$ and let $F_s(Y)$
  be the subalgebra of $\phi\{y\}{^{(q)}}$ generated by $Y$. We have a
  homomorphism of $F(X)$ onto $F_s(Y)$ such that $x\to y$. If
  $\mathscr{J}$ is a quadratic Jordan algebra then we denote the
  subalgebra of End $\mathscr{J}$ generated by the $U_a,a\in
  \mathscr{J}$, as Env $U(\mathscr{J})$. It is clear that Env
  $U(\mathscr{J})$ contains all $U_{a,b}$ and all $V_{a,b}$. Moreover
  it is easily seen that if $a\to a^{\eta}$ is a homomorphism of
  $(\mathscr{J},U,1)$ onto a second quadratic Jordan algebra
  $(\mathscr{J}',U',1')$ then there exists a (unique) homomorphism of
  Env $U(\mathscr{J})$ onto Env $U(\mathscr{J}')$ such that $U_a\to
  U'_{a^{\eta}},a\in \mathscr{J}$. Then also $U_{a,b}\to
  U'_{a^{\eta},b^{\eta}}$ and $V_{a,b}\to V'_{a^{\eta},b^{\eta}}$. In
  particular we have such a homoorphism of Env $U(F(X))$ onto Env
  $U(F_s(Y))$. Let $\mathscr{X}$ and $\mathscr{Y}$ respectively denote
  the subalgebra of Env $U(F(X))$ and Env $U(F_s(Y))$ generated by all
  $U_{x^{n}},U_{x^{n},x^{m}},V_{x^{n},x^{m}}$ and
  $U_{y^{n}},U_{y^{n},y^{m}},V_{y^{n},y^{m}}$. Then the restriction of
  our homomorphism of Env $U(F(X))$ onto Env $U(F(Y))$ to
  $\mathscr{X}$ is a homomorphism of $\mathscr{X}$ onto $\mathscr{Y}$
  such that $U_{x^{n}}\to U_{y^{n}},U_{x^{n},x^{m}}\to
  U_{y^{n},y^{m}},V_{x^{n},x^{m}}\to V_{y^{n},y^{m}}$. Since $F(y)$
  is special, $f(U_{y^{n1}},U_{y^{n_{2}}},\ldots,\ldots)=0$ holds. It
  will follow that $f(U_{x^{n1}},U_{x^{n2}},\ldots)=0$ holds in $F(X)$
  if we can show that the homomorphism of $\mathscr{X}$ onto
  $\mathscr{Y}$ is an isomorphism. We have seen that $\mathscr{X}$ is
  generated by $U_x$ and $V_x$ and $\mathscr{Y}$ is generated by $U_y$
  and $V_y$. Since $U_x\to U_y$ and $V_x\to V_y$ the isomorphism will
  follow by showing that $U_y$ and $V_y$ are algebraically independent
  over $\Phi$. Now in $\Phi\{Y\}^{(q)}$ we have
  $U_y=y_Ry_L,V_y=y_R=y_L$ where $a_R$ is $b\to ba$ and $a_L$ is a
  $b\to ab$ and $y_L$ and $y_R$ commute\pageoriginale and are
  algebraically independent over $\Phi$ since if $z\in Y,z\neq y$, then
  $zk^{k}_Ry^{l}_L=y^{l}zy^{k}$ and the elements $y^{l} zy^{k},
  l,k=0,1,2,\ldots$ are $\Phi$ -independent. Now $V_y=y_R+y_L$ and
  $U_y=y_R y_L$ are the ``elementary symmetric'' functions of $y_R$
  and $y_L$. The usual proof of the algebraic independence of the
  elementary symmetric function (e.g. Jacobson, {\em Lectures in
    Abstract Algebra}, p.108) carries over to show that $U_y$ and
  $V_y$ are algebraically independent operators in
  $\Phi\{Y\}^{(q)}$. It follows that they are algebraically
  independent operators also in $F_s(Y)$. This completes the proof of
  the theorem. 

We now give two important instances of Theorem \ref{c1:thm3} which we shall
need. Let $f(\lambda)\in \Phi[\lambda]$. an indeterminate and let
$f(a)$ be defined in the obvious way if $a\in \mathscr{J}$ a quadratic
Jordan algebra. Suppose $\mathscr{J}$ is a subalgebra of
$\mathfrak{a}(q), \mathfrak{a}$ associative. Then we claim that
$U_{f(a)}U_{g(a)}=U_{(fg)(a)}$ holds in $\mathfrak{a}^{(q)}$, hence in
$\mathscr{J}$. For,
$xu_{(fg)(a)}=f(a)g(a)xf(a)g(a)=g(a)f(a)xf(a)g(a)=xU_{f(a)}U_{g(a)}$. It
follows from Theorem \ref{c1:thm3} that 
\begin{equation*}
U_{f(a)}U_{g(a)}=U_{(fg)(a)}\tag*{QJ 37}
\end{equation*}
in any quadratic Jordan algebra. Another application of the theorem is
the proof of 
\begin{equation*}
  V_{a^{n},a^{n}}=V_{a^{m+n}}\tag*{QJ 38}
\end{equation*}

This follows since in $\mathfrak{a}(q), \mathfrak{a}$ associative, we have $x
V_{a^{m},a^{n}}=xa^{m}a^{n}+a^{n}a^{m}x=xa^{m+n}+a^{m+n}x=xV_{a^{m+n}}$. Similarly,
one proves 
\begin{equation*}
V_{a^{n}}U_{a^{n}}=U_{a^{n},a^{m+n}}\tag{QJ 39}
\end{equation*}\pageoriginale
and
\begin{equation*}
V_{a^{n}}=V_{a^{n-1}}V_a-V_{a^{n-2}}U_a, n\ge 2\tag*{QJ 40}
\end{equation*}
The list of identities we have given will be adequate for the results
which will be developed in this monograph. Other aspects of the theory
require additional identities. Nearly all of these are consequences of
the analogues for quadratic Jordan algebras of Macdomalli
theorem. This result, which states that the extension of Theorem \ref{c1:thm3}
to subalgebras with two generators is valid, has been proved by
McCrimmon in \cite{McCrimmon3}. 
\end{proof}


\section{Category isomorphism for $\Phi\ni\frac{1}{2}$. Characteristic
  two case.}\label{c1:sec4} 

We shall show first that if $\Phi \ni\frac{1}{2}$ then the two notions
of Jordan algebra and quadratic Jordan algebra are equivalent. Let
$CJ(CQJ)$ denote the category whose objects are Jordan algebras
(quadratic Jordan algebras) over $\Phi$ with morphisms as
homomorphisms. We have the following {\em Category Isomorphism
  Theorem}. Let $(\mathscr{J},R,1)$ be a Jordan algebra over a
commutative ring $\Phi$ containing $\frac{1}{2}$. Define $U$ by
$U_a=2R_a^{2}-R_{a^{2}}$. Then $(\mathscr{J},U,1)$ is a quadratic
Jordan algebra. Let $(\mathscr{J},U,1)$ be a quadratic Jordan algebra
over $\Phi$ and define $R$ by $R_a=\frac{1}{2}V_a,V_a=U_{a,1}$. Then
$(\mathscr{J}, R,1)$ is a Jordan algebra. The two constructions are
inverses. Moreover, a mapping $\eta$ of $\mathscr{J}$ is a
homomorphism of $(\mathscr{J},R,1)$ if and only if it is a
homomorphism of $(\mathscr{J},U,1)$. Hence
$(\mathscr{J},R,1)\to(\mathscr{J},U,1)$, $\eta\to\eta$ is an isomorphism
of the category $CJ$\pageoriginale onto $CQJ$. 

\begin{proof}
Let $(\mathscr{J},R,1)$ be unital jordan over $\Phi\ni\frac{1}{2}$ and
$U_a=2R_a^{2}-R_{a^{2}}$. Then we have shown in \S $2$ that
$(\mathscr{J},U,1)$ is a quadratic Jordan algebra. We have
$U_{a,b}=2(R_aR_b+R_bR_a-R_{a,b})$ so $V_a=U_{a,1}=2R_a$ and
$\frac{1}{2}V_a=R_a$. Next let $(\mathscr{J},U,1)$ be a quadratic
Jordan algebra over $\Phi\ni\frac{1}{2}$ and let
$R_a=\frac{1}{2}V_a$. By $QJ 24, 20, [V_a,V_{a^{2}}]=0$ so
$[R_aR_{a^{2}}]=0$. Moreover, $a^{2}=1U_a=\frac{1}{2}(a\circ
a)=aR_a$. Also $aV_b=bV_a$ gives $aR_b=bR_a$ and we have $R_1=1$ and
$a\to R_a$ is a $\Phi$ -homomorphism of $\mathscr{J}$ into End
$\mathscr{J}$. Hence $(\mathscr{J},R,1)$ is a linear Jordan
algebra. By $QJ 20$,
$U_A=\frac{1}{2}V^{2}_a-\frac{1}{2}V_{a^{2}}=2R_a^{2}-R_{a^{2}}$. This
proves the assertions on the passage from $(\mathscr{J},R,1)$ to
$(\mathscr{J},U,1)$ and back. The rest is clear. 

We consider next the opposite extreme of the foregoing, namely, that
in which $2\Phi =0$ or, equivalently, $2(1)=1+1=0$. Let
$(\mathscr{J},U,1)$ be a quadratic Jordan algebra over $\Phi$. We
claim that $\mathscr{J}$ is a $2$ -Lie algebra ($=$ restricted Lie
algebra of characteristic two) if we define $[ab]=a\circ b$ and
$a^{[2]}=a^{2}$. We have $[aa]=a\circ a=2a^{2}=0$ and
$[[ab]c]+[[bc]a]+[[ca]b]=(a\circ b)\circ c =(b \circ c)\circ a+(c\circ
a)\circ b=\{abc\}+\{bac\}+\{bca\}+\{cba\}+\{cab\}+\{acb\} (QJ
27)=2\{abc\}+2\{bca\}+2\{cab\}=0$. Also $(a+b)^{2}=a^{2}+b^{2}+[a,b]$
and $ba^{2}=[[ba]a]$ since $V_{a^{2}}=V^{2}_a$ by $QJ 20$. Hence the
axioms for a $2$-Lie algebra hold (Jacobson, Lie Algebras
P. $6$). This proves 
\end{proof}

\begin{thm}\label{c1:thm4}
  Let $(\mathscr{J},U,1)$ be a quadratic Jordan algebra over $\Phi$
  such that $2\Phi =0$. Then $\mathscr{J}$ is a $2$ lie algebra
  relative to $[ab]=a\circ b$ and $a^{[2]}=a^{2}$. 
\end{thm}

\section{Inner and outer ideals. Difference algebras.}\label{c1:sec5}

\begin{defn}
  Let\pageoriginale $(\mathscr{J},U,1)$ be a quadratic Jordan
  algebra. A subset $\mathscr{L}$ of $\mathscr{J}$ is called an {\em
    inner (outer) ideals} if $\mathscr{L}$ is a sub-module and
  $bab=aU_b(aba=bU_a)\in \mathscr{L}$ for all $a\in \mathscr{J}, b\in
  \mathscr{L}\mathscr{L}$ is an {\em ideal} if it is both an inner and
  an outer ideal. 
\end{defn}

The condition can be written symbolically as
$\mathscr{J}U_{\mathscr{L}}\subseteq \mathscr{L}$ for an ideal,
$\mathscr{L}U_{\mathscr{J}}\subseteq \mathscr{L}$ for an outer ideal. If
$\mathscr{L}$ is an inner ideal and $a\in\mathscr{J}$ then
$\mathscr{L}_{U_{a}}$ is an inner ideal since for $c=bU_a,
\mathscr{J}U_c=\mathscr{J}U_{bU_{a}}=\mathscr{J} U_aU_bU_a\subseteq
\mathscr{J}U_bU_a\subseteq \mathscr{L} U_a$. In particular,
$\mathscr{J}U_a$ is an inner ideal called the {\em principal inner
  ideal determined by $a$}. This need not contain $a$. The inner ideal
generated by a is $\Phi a +\mathscr{J}U_a$. For this contains a, is
contained in every inner ideal containing $a$ and is an inner ideal,
since a typical element of $\Phi a+\mathscr{J} U_a$ is $\alpha a+b
U_a, \alpha \in \Phi,b\in \mathscr{J}$, and $U_{\alpha
  a+bU_a}=\alpha^{2}U_a+\alpha U_{a,bU_a}+U_aU_bU_a$. Since
$U_{a,bU_{a}}=V_{a,b}U_a$ by $QJ4'$ we see that $\mathscr{J}U_{\alpha
  a+bu_{a}}\subseteq \mathscr{J} U_a$, so $\Phi
a+\mathscr{J}U_a$ is an inner ideal. The outer ideal generated by a is
the smallest submodule of $\mathscr{J}$ contianing a and stable under
all $U_b,b\in \mathscr{J}$. The principal inner ideal detrermined by
$1$ is $\mathscr{J}U_1=\mathscr{J}$. On the other hand, as we shall
see, the outer ideal generated by $1$  need not be $\mathscr{J}$. We
shall call this the {\rm cone} of $\mathscr{J}$. 

If $\mathscr{L}$ is an outer ideal then
$\{a_1ba_2\}=bU_{a_{1},a_{2}}= bU_{a_1+a_2}-bU_{a_1}-bU_{a_2}\in
\mathscr{L}$ for $b\in \mathscr{L}, a_i\in \mathscr{J}$. In
particular, $b\circ a=bu_{a,1}\in \mathscr{L}$, $b\in \mathscr{L},
a\in \mathscr{L}$. By $QJ27$ it follows that $\{ba_1a_2\}\in
\mathscr{L},b\in \mathscr{L}, a_i\in \mathscr{J}$. If $\Phi$ contains
$\frac{1}{2}$ then $\mathscr{L}$ is an outer ideal if and only if it
is an ideal and if and only if $\mathscr{L}$ is an ideal in
$(\mathscr{J},R,1)$ where $(\mathscr{J},R,1)$ is the
Jordan\pageoriginale algebra corresponding to $(\mathscr{J},U,1)$ in
the usual way. For, if $\mathscr{L}$ is an outer ideal then $b\cdot
a=\frac{1}{2} b\circ a\in \mathscr{L}, a\in \mathscr{J}, b\in
\mathscr{L}$. On the other hand, if $\mathscr{L}$ is an ideal of
$(\mathscr{J},R,1)$ then $bR_a=aR_b\in \mathscr{L}$ and this implies
that $bU_a$ and $aU_b\in \mathscr{L}$. 

It is clear that the intersection of inner (outer) ideals is an inner
(outer) ideal and the sum of outer ideals is an outer ideal. It is
easily checked that the sum of an inner ideal and an ideal is an inner
ideal. 

Let $\mathscr{L}$ be an ideal in $\mathscr{J},U,1), b_i\in \mathscr{L}, a_i\in \mathscr{J}$. Then we have seen that $a_1U_{a_{2},b_{2}}=\{a_2a_1b_2\}\in \mathscr{L}$. Hence

\begin{align*}
(a_1+b_1)U_{a_{2}+b_{2}}&=(a_1+b_1)(U_{a_{2}}+U_{a_{2},b_{2}}+U_{b_{2}})\\
&\equiv a_1U_{a_{2}}(\mod \mathscr{L})
\end{align*}

It follows that if we define in
$\overline{\mathscr{J}}=\mathscr{J}/\mathscr{L}=\{\overline{a}+\mathscr{L}|a\in\mathscr{J}\},
\overline{a_1}\break U_{\overline{a_{2}}}=\overline{a_1U_{a_{2}}}$    
then this is single valued. It is immediate that
$(\ob{\mathscr{J}}, \ob{U}, \ob{1})$ is a quadratic Jordan algebra and we
have the canonical homomorphism $a\to \overline{a}$ of
$(\mathscr{J},U,1)$ onto $(\overline{\mathscr{J},U}1)$. Conversely, if
$\eta$ is a homomorphism of $(\mathscr{J},U,1)$ then
$\mathscr{L}=\ker \eta$ is an ideal and we have the isomorphism
$\overline{\eta}:\overline{a}\to a^{\eta}$ of
$(\overline{\mathscr{J}}=\mathscr{J}/\mathscr{L}, \overline{U, 1})$
onto $(\mathscr{J}^{\eta},U,1)$. This fundamental theorem has its
well-known consequences.
 
\begin{examples*}
\begin{itemize}
\item[(1)] Let $\mathscr{J}=\mathscr{H}(\mathbb{Z}_n)$ the quadratic
  Jordan algebra (over the ring of integers $\mathbb{Z}$) of $n\times
  n$ integral symmetric matrices. Let
  $E=\text{diag}~ \{1,\ldots,1,0,\ldots,0\}$. Then $E$ is an idempotent
  $(E^{2}=E)$ and the principal inner ideals
  $\mathscr{J}U_E=E\mathscr{H} E$ is the set of matrices of the
  form\pageoriginale 
  $$
  \begin{pmatrix}
    A & 0\\
    0 & 0
  \end{pmatrix},
  $$
  $A\in \mathbb{Z}_r$. Next let $B=(b_{ij})\in
  \mathscr{H}(\mathbb{Z}_n)$ have even diagonal elements and let $A\in
  \mathscr{H}(\mathbb{Z}_n)$. Then $(i,i)$ -entry of $C=ABA$ is  
  \begin{align*}
    c_{ii}&=\sum_{j,k}a_{ij}b_{jk} a_{ki}=\sum_{j,k} a_{ij}b_{jk}a_{ik}\\
    &=\sum_{j}a^{2}_{ij}b_{jj}+2\sum_{j<k}a_{ij}b_{jk}a_{ik}
  \end{align*}
  Hence $C$ has even diagonal elements. Thus the set $\mathscr{L}$ of
  integral symmetric matrices with even diagonal elements is an outer
  ideal in $\mathscr{H}(\mathbb{Z}_n)$. If $m\in \mathbb{Z}$ the set
  $m\mathscr{H}(\mathbb{Z}_n)$ of integral symmetric matrices whose
  entries are divisible by $m$ is an ideal in
  $\mathscr{H}(\mathbb{Z}_n)$. 

\item[(2)]  Let $\underline{\rho}=\Phi(\lambda)$ the filed of rational
  expressions in an indeterminate $\lambda$ over a field $\Phi$ of
  characteristic two. let $\mathscr{H} \underline{(\rho_{n})}$ be the
  set of $n\times n$ symmetric matrices with entries in
  $\ub{\rho}$. This is a quadratic Jordan algebra over $\Phi$
  (with $ABA$ as usual). Let $\mathscr{L}$ be the
  subset of matrices with diagonal entries in
  $\Phi(\lambda^{2})$. Then $\mathscr{L}$ is an outer ideal containing
  $1$. It is easy that this is the cone of
  $\mathscr{H}(\underline{\rho_{n}})$. 

\item[(3)] Let $\Phi$ be a field of characteristic two,
  $\mathfrak{a}=\Phi[\lambda],\lambda$ an indeterminate. Consider
  $\mathfrak{a}^{(q)}$  and the subspace
  $\mathscr{L}=\Phi\lambda^{2}+\sum\limits_{i\geqq 4}\Phi
  \lambda^{i}$. 

  It\pageoriginale is readily checked that $\mathscr{L}$ is an ideal in
  $\mathfrak{a}^{(q)}$. Let $\mathscr{J}=\mathfrak{a}^{(q)}/\mathscr{L}$
  and put $\overline{\lambda}=\lambda+\mathscr{L}$. Then
  $\overline{\lambda}=\lambda+\mathscr{L}$. Then
  $\overline{\lambda}^{2}=0$ but $\overline{\lambda}^{3}\neq 0$ in
  $\mathscr{J}$. If $\mathscr{J}$ is a special Jordan algebra, say,
  $\mathscr{J}$ is a subalgebra of $\mathfrak{L}^{(q)},\mathfrak{L}$,
  associative, then the Jordan power $X^{n}$ of $X\in \mathscr{J}$
  coincides with the associative power $X^{n}$ in $\mathfrak{L}$ since
  $X^{n}=X^{n-2}U_X=XX^{n-2}X$. Hence it is clear that in a special
  Jordan algebra $X^{n}=0$ implies $X^{n+1}=0$. It follows that
  $\mathscr{J}=\mathfrak{a}^{(q)}/\mathscr{L}$ is not special.
  
  We have defined ker $U=\{z|U_z=0=U_{z,a},a\in
  \mathscr{J}\}$. This is an ideal since it is a submodule and if
  $a\in \mathscr{J}, z\in \ker U$, then $aU_z=0$, and
  $U_{zU_a}=U_aU_zU_a=0$. Also we can show that for $b\in
  \mathscr{J}$, $U_{zU_a,b}=0$. To see this we note that $U_{z,a}=0$,
  $a\in \mathscr{J}$ implies $V_z=U_{z,1}=0$. Then
  $V_{z,a}=0=V_{a,z}$ by $QJ 27'$. By $QJ9$, we have
  $\{b\{zba\}c\}+b\{zca\}b=\{\{bzc\}ab\}+\{(bzb)ac\}$ which gives (using
  $a$ as operand): $V_{b,z}U_{b,c}+V_c, U_b=U_{b,b}
  V_{z,c}+U_{c,zU_{b}}$. This implies $U_{c, zU_b}=0$ or
  $U_{b,zU_a}=0$. Hence $zU_a\in$ker $U$. The argument we have
  used show that every $\{\quad\}$ and $-U_{-}$ with one of the arguments
  $z\in$ ker $U$ is $0$, with the exception of $zU_a$. In
  particular $2z=z\cdot 1=0$ which show that ker $U=0$ if
  $\mathscr{J}$ has no two torsion. We call $\mathscr{J}$ {\em
    nondegenerate} if ker $U=0$.
\end{itemize}
\end{examples*}

\section{Special universal envelopes}\label{c1:sec6}

A homomorphism of
$(\mathscr{J},U,1)$ into $\mathfrak{a}(q)$ where $\mathfrak{a}$ is
associative is called an {\em associative specialization of}
$\mathscr{J}$ {\em into} $\mathfrak{a}$. A {\em special universal
  envelops} for $\mathscr{J}$ is a pair $(S(\mathscr{J}),\sigma_u)$
where $S(\mathscr{J})$ is an associative algebra and $\sigma_u$ is an
associative specilalization of into $S(\mathscr{J})$ such that if
$\sigma$ is an-associative specialisation of
$\mathscr{J}$\pageoriginale into an associative algebra $\mathfrak{a}$
then there exists a unique homomorphism $\eta$ of $s(\mathscr{J})$
into $\mathfrak{a}$ such that
\begin{equation*}
\vcenter{\xymatrix{
\mathscr{J}\ar[d]\ar[r]^{\sigma_u}& S(\mathscr{J})\ar[dl]^{\eta}\\
\mathfrak{a}
}}\tag{12}
\end{equation*}
is commutative. To construct an $(S(\mathscr{J}),\sigma_u)$ let
$T(\mathscr{J})$ be the tensor algebra defined by the $\Phi$ -module
$\mathscr{J}:T(\mathscr{J})=\Phi\oplus(\mathscr{J}\oplus(\mathscr{J}\otimes\mathscr{J})\oplus\ldots\ldots$ 
where all these tensor products are taken over $\Phi$. Multiplication
in $T(\mathscr{J})$ is defined by $(x_1\otimes\ldots\otimes
x_r)(x_{r+1}\otimes\ldots \otimes x_s)=x_1\otimes\cdots\otimes
x_s,x_i\in \mathscr{J}$, and the rule that the unit element
$\Phi$ of $1_\Phi$ is unit for $T(\mathscr{J})$. Then $T(\mathscr{J})$
is an associative algebra over $\Phi$. Let $k$ be the ideal in
$T(\mathscr{J})$ generated by the elements $1-1_{\Phi}(1\in
\mathscr{J}), aba-a\otimes b\otimes a, a,b\in \mathscr{J}$. Put
$S(\mathscr{J})=T(\mathscr{J})/\mathfrak{K}$ and $a^{\sigma
  u}=a+\mathfrak{K},a\in \mathscr{J}$. Then it is readily seen
that $((\mathscr{J}),\sigma_u)$ is a special universal envelope for
$(\mathscr{J},U,1)$.\footnote{cf. Jacobson [2],pp. 65-72, for the
  corresponding discussion for Jordan algebras}. It is clear that we
have an involution 
$\pi'$ of $T(\mathscr{J})$ such that $(x_1\otimes x_2\ldots \otimes
x_r)^{\pi'}=x_r\otimes x_{r-1}\otimes\ldots \otimes x_1,x_i\in
\mathscr{J}$. Since $(aba-a\otimes b\otimes a)^{\pi'}=aba-a\otimes
b\times a$ it is clear that $\mathfrak{K}^{\pi'}\subseteq
\mathfrak{K}$. Hence $\pi'$ induces an involution $\pi$ in
$S(\mathscr{J})/\mathfrak{K}$. We have
$\displaystyle{a^{\mathop{\sigma}_u \pi} =a^{\sigma_
  u}, a\in \mathscr{J}}$, and since the $a^{\sigma_u}$ generate
$S(\mathscr{J})$ it is clear that $\pi$ is the only involution
satisfying $a^{\sigma_u\pi}=a^{\sigma u \pi}$. We shall call $\pi$ the
{\em main involution} of $S(\mathscr{J})$. If $\xi$ is a homomorphism
of $(\mathscr{J},U,1)$ into $(\mathscr{J}',U',1')$ then we have a
unique homomorphism $\xi_u$ of $S(\mathscr{J})$ into $S(\mathscr{J}')$
such that
\begin{equation*}
\vcenter{\xymatrix{
\mathscr{J}\ar[d]_{\sigma_u}\ar[r]^{\xi}& \mathscr{J}'\ar[d]^{\sigma_u}\\
S(\mathscr{J})\ar[r]_{\xi}& S(\mathscr{J}')
}}\tag{13}
\end{equation*}\pageoriginale
It is immediate that $(\mathscr{J},U,1)$ in special if and only if the
mapping $\sigma_{u}$ of $\mathscr{J}$ into $S(\mathscr{J})$ is
injective. In this case it is convenient to identify $\mathscr{J}$
with its image in $S(\mathscr{J})$ and so regard $\mathscr{J}$ as a
subset of $S(\mathscr{J}),\sigma_u$ as the injection mapping. Then
$\mathscr{J}$ is a sub-algebra of the quadratic Jordan algebra
$S(\mathscr{J})^{(q)}$ and the universal property of $\sigma_{u}$
states that any homomorphism of $\mathscr{J}$ into an
$\mathfrak{a}^{(q)},\mathfrak{a}$ associative, has a unique extension
to a homomorphism of $S(\mathscr{J})$ into $\mathfrak{a}$.

\section[Quadratic Jordan algebras of quadratic forms...]{Quadratic
  Jordan algebras of quadratic forms\hfil\break 
with base points.}\label{c1:sec7} 

We consider a class of quadratic Jordan algebras $(\mathscr{J},U,1)$
over a field $\Phi$ satisfying the following conditions:
\begin{itemize}
\item[1.] There exists a linear function $T$ and a quadratic form $Q$
  on $\mathscr{J}$ (to $\Phi$) such that
\footnotetext[1]{This condition is superfluous if $\Phi$ contains
  $\frac{1}{2}$ since in this case $X^{K}\cdot X=X^{k+1}$ so \eqref{c1:eq14}
  implies \eqref{c1:eq15}.}
\begin{align*}
  &X^{2}-T(X)X+Q(X)=0\tag{14}\label{c1:eq14}\\
  &X^{3}-T(X)X^{2}+Q(X)X=0\tag{15}\label{c1:eq15}\footnotemark[1]
\end{align*}
\item[2.] The same conditions hold for
  $\mathscr{J}_{\underline{\rho}}$ where $\underline{\rho}$ is any
  extension field of $\Phi$ and $T$ and $Q$ for
  $\mathscr{J}_{\underline{\rho}}$ are the extenstions of these
  functions\pageoriginale on $\mathscr{J}$ to a linear function and a
  quadratic form on $\mathscr{J}_{\underline{\rho}}$ respectively. (We
  assume $\mathscr{J}$ imbedded in $\mathscr{J}_{\underline{\rho}}$
  and write, $U, 1$ for the U-operator and unit in
  $\mathscr{J}_{\underline{\rho}}$).
\item[3.] $\mathscr{J}\neq \Phi$
\end{itemize}
Taking $\underline{\rho}=\Phi(\lambda)$, an indeterminate and
replacing $x$ by $x+y$ in \eqref{c1:eq14} we obtain
\begin{equation*}
x\circ y =T(x)y+T(y)x-Q(x,y)\tag{16}\label{c1:eq16}
\end{equation*}
where $Q(x,y)$ is the symmeteric bilinear form associated with the
qua\-dra\-tic form $Q(Q(x,y)=Q(x+y)-Q(x)-Q(y))$. Similarly, \eqref{c1:eq15} and
$X^{3}=xU_x$ give
\begin{equation*}
yU_x=T(x)x\circ y + T(y) x^{2}-Q(x)y-Q(x,y)x-x^{2}\circ y\tag{17}\label{c1:eq17}
\end{equation*}
Putting $y=1$ in \eqref{c1:eq16} gives $2x=T(x)1+T(1)x-Q(x,1)1$. If we take
$x\notin \Phi 1$
\begin{equation*}
  T(1)=2\tag{18}\label{c1:eq18}
\end{equation*}
Then $x=1$ in \eqref{c1:eq14} gives
\begin{equation*}
  Q(1)=1\tag{19}\label{c1:eq19}
\end{equation*}
Also using the formulas for $X^{2}$ and $x\circ y$, \eqref{c1:eq17} becomes
\begin{equation*}
  yU_x=Q(x)y+T(y)T(x)x-Q(x,y)x-T(y)Q(x)1\tag{20}\label{c1:eq20}
\end{equation*}

We can write this in a somewhat more compact form by introducing
$\overline{x}=T(x)1-x$.\pageoriginale Then \eqref{c1:eq20} becomes
\begin{equation*}
  yU_x=Q(y,\overline{x})x-Q(x)\overline{y}\tag*{20'}
\end{equation*}

Conversely, suppose we are given a quadratic form $Q$ on a vector
space $\mathscr{J}$ with a {\em base point} $1$ such that
$Q(1)=1$. Define $T(x)=Q(x, 1),\ob{x}=T(x)1-x$ and $U_x$ by $(20')$ (or
($20$)). Then one can verify by direct calculation that
$(\mathscr{J},U,1)$ is a quadratic Jordan algebra satisfying condition
$1$ and $2$. We proceed to prove more that this by showing that
$(\mathscr{J},U,1)$ is a special quadratic Jordan algebra satisfying
$1$ and $2$. For this purpose we introduce the Clifford algebra of $Q$
with base point $1$ which is defined as follows. Let $T(\mathscr{J})$
be the tensor algebra over $\mathscr{J}$ and let $\mathscr{L}$  be the
ideal in $T(\mathscr{J})$ generated by $1\Phi -1$ and all $x \otimes
x-T(x)x+Q(x)1$ where $1$, $x\in \mathscr{J}$ and
$T(x)=Q(x,1)$. Then we define {\em the clifford algebra}
$C(\mathscr{J},Q,1)$ of {\em the quadratic form} $Q$ {\em will base
  point} (such that $Q(1)=1$) to be $T(\mathscr{J})/\mathscr{L}$. If
$x\in \mathscr{J}$ we put $x^{\sigma_u}= x+\mathscr{L}$. Then we
have $(x^{\sigma_u})^{2}-T(x)x^{\sigma_u}+Q(x)1=0$. This implies that
$x^{\sigma_u} y^{\sigma_y}+y^{\sigma_u} x^{\sigma_u}=T(x)y^{\sigma
  _u}+T(y)x^{\sigma_u}-Q(x,y)1=0$. Since $y^{\sigma_u}U_{x^{\sigma
    _u}}=x^{\sigma_u} y^{\sigma_u} x^{\sigma_u}$ we obtain
\begin{equation*}
y^{\sigma_u}u_{x^{\sigma_u}}=Q(x)y^{\sigma_u}+T(y)T(x)x^{\sigma
  _u}-Q(x,y)x^{\sigma_u}-T(y)Q(x)^{1}.\tag{21}\label{c1:eq21}
\end{equation*}
Also we have $1^{\sigma_u}=1$. This and (21) show that
$\mathscr{J}^{\sigma_u}=\{x^{\sigma_u}|x\in \mathscr{J}\}$ is a
subalgebra of $C(\mathscr{J}, Q, 1)^{(q)}$. We shall that $\sigma_u$
is injective. Then it will follow from (20) and (21) that
($\mathscr{J}, U,1)$ is a\pageoriginale quadratic Jordan algebra and
$\sigma_u$ is an associative specialization of $\mathscr{J}$ in
$C=C(\mathscr{J},Q,1)$. It is clear from the definition of
$C(\mathscr{J},Q,1)$  that if $x\to x^{\sigma}$ is a linear mapping of
$\mathscr{J}$ into an associative algebra $\mathfrak{a}$ suchthat
$1^{\sigma}=1$ and $(x^{\sigma})^{2}-T(x)x+Q(x)1=0$ then there exists a
  unique homomorphism of $C(\mathscr{J}, Q, 1)$ into $\mathfrak{a}$
  such that $x^{\sigma_u}\to x^{\sigma}, x\in \mathscr{J}$.

We consider first the case in which $(Q, 1)$ is {\em pure} in the
sense that $\mathscr{J}=\Phi 1+V$ where $V$ is a subspace such that
$T(v)=0,v\in V$. If the characteristic is $\neq 2$ then
$T(1)=Q(1,1)=2\neq 0$ and $\mathscr{J}=\Phi 1\oplus(\Phi
1)^{\bot}(1\Phi 1)^{\bot}$ the orthogonal complement of $\Phi 1$
relative to $Q(x,y))$. Then $T(v)=Q(1,v)=0$ for $v\in (\Phi
1)^{\bot}$ and $(Q,1)$ is pure. If the characteristic is two then
$T(1)=0$ so $(Q,1)$ is pure if and only if $T\equiv 0$. In this case
$V$ can be taken to be any subspace such that $\mathscr{J}=\Phi1\oplus
V$. Now let $C(V,-Q)$ be the Clifford algebra of $V$ relative to the
restriction of $-Q$ to $V$. The canonical mapping of $\Phi 1+ V$ into
$C(V, -Q)$ is injective so we can identify $\mathscr{J}=\Phi 1\oplus
V$ with the corresponding subset of $C(V,-Q)$. Let $x=\alpha
1+v,\alpha \in \Phi,v \in V$. Then $T(x)=2\alpha,
Q(x)=\alpha^{2}+Q(v)$ and in $C(V,-Q), x^{2}=\alpha^{2}1+2\alpha
v+v^{2}=(\alpha^{2}-Q(v))1+2\alpha v$. Hence $x^{2}-T(x)+Q(x)=0$. It
follows from the universal property of $C(\mathscr{J},Q,q)$ that we
have a homomorphism of $C(\mathscr{J},Q,1)$ into $C(V,-Q)$ such that
$x^{\sigma_v}\to x$. Clearly this implies that $\sigma_u$ is
injective.

Suppose next that $(Q,1)$ isnot pure so the characteristic is two and
$T\neq 0$. We can choose $d$ so that $T(d)=1$ and write
$\mathscr{J}=\Phi d\oplus W$\pageoriginale where $W$ is the hyperplane
in $\mathscr{J}$ defined by $T(x)=0$. Then $1\in W$ and $W=\Phi
1\oplus V,V$ a subspace. We again consider $C(V,Q)(Q=-Q)$ since
char$=2$) and we identify $W=\Phi 1+V$ with the corresponding subset
of $C(V,Q)$. Let $D'$ be the derivation in the tensor algebra
$T(V)$ such that $vD'=v+Q(v,d)1$. Since this maps $v\otimes v+Q(V)$
into $0$ it maps the ideal $\mathfrak{K}$ defining $C(V,Q)$ into
itself. Hence this induces a derivartion $D$ in $C(V,Q)$ such that
$vD=v+Q(v,d)1$. Now put $\mathscr{L}=C(V,Q)$ and let $\mathscr{L}[t,
  D]$ be the algebra of differential polynomials in an indeterminate
$t$ with coefficients in $\mathscr{L}$ such that
\begin{equation*}
ct+tc=cD,\quad c\in \mathscr{L}\tag{22}\label{c1:eq22}
\end{equation*}
since the characteristic is two, $D^{2}$ is a derivation. Since
$vD^{2}=vD, v\in V$, and $V$ generates $\mathscr{L},
D^{2}=D$. Also $ct^{2}+t^{2}c=cD^{2}$ so $c(t^{2}+t)=(t^{2}+t)c,
c\in \mathscr{L}$. Since $t^{2}+t$ commutes with $t$ also, it is
clear that this polynomial is in the center of $\mathscr{L}[t,
  D]$. Hence also $g(t)=t^{2}+t+Q(d)1$ is in the center. Let $(g(t))$
be the ideal in $\mathscr{L}[t, D]$ generated by $g(t)$ and put
$\mathscr{O}=\mathscr{L}[t,D]/(g(t))$. It is clear from the division
algorithm (which is applicable to $g(t)$ since its leading coefficient
is $1$) that every element of $\mathscr{L}[t, D]$ is congrument modulo
$(g(t))$ to an element of the form $c_o+c_1t,c_i\in
\mathscr{L}$. Also $c_o+c_1t\equiv 0 (\mod g(t))$ implies
$c_o=c_1=0$. Hence we can identify $\mathscr{O}$ with the set of
elements of the form $c_o+c_1t, c_i\in \mathscr{L}$, and we have
the realtions $vt+tv=v+Q(v,d),t^{2}+t+Q(d)=0$. We have the injective
linear mapping\pageoriginale $x=\alpha 1 +v+\beta d\to y=\alpha
1+v+\beta t,\quad, \alpha \beta \in \Phi, v\in V$, of
$\mathscr{J}=\Phi 1+V+\Phi d $ into $\mathscr{O}$. Moreover,
$T(x)=\beta$, $Q(x)=\alpha^{2}+Q(v)+\beta^{2}Q(d)+\beta Q(v,d)+\alpha
\beta$ and 
\begin{align*}
  y^{2}&=\alpha^{2}+Q(v)+\beta^{2}t^{2}+\beta (vt+tv)\\
  &=\alpha^{2}+Q(v)+\beta^{2}(t+Q(d)1)+\beta(v+Q(v,d))\\
  &=T(x)y+Q(x)1=T(x)y-Q(x)'.
\end{align*}
Hence by the universal property of $C(\mathscr{J},Q,1)$ we have a
homomorphism of $C(\mathscr{J}, Q,1)$ into $\mathscr{O}$ such that
$(\alpha 1+v+\beta d)^{\sigma_u}\to y=\alpha 1+v+\beta t$. Clearly
this implies that $\sigma_u$ is injective.

We have now proved that $(\mathscr{J},U,1)$ is a special quadratic
Joradan algebra and $\sigma_u$ is an associative specialization of
$\mathscr{J}$ into, $C(\mathscr{J},Q,1)$. We now take $y=1$ in $(20)$
to obtain $x^{2}=Q(x)1+2T(x)x-T(x)x-2Q(x)1$ (Since
$T(x)=Q(x,1),T(1)=Q(1,1)=2)=T(x)x-Q(x)$. Since $\mathscr{J}$ is
special we have $x^{3}-T(x)x^{2}+Q(x)x=0$ in $\mathscr{J}$. If
$\underline{\rho}$ is an extension field of $\Phi$ then it is clear
that the extension of $U$ to a quadratic mapping of
$\mathscr{J}_{\underline{\rho}}$ into End
$\mathscr{J}_{\underline{\rho}}$ is given by \eqref{c1:eq20} where $Q$ and $T$
are the extensions of $Q$ and $T$ to a quadratic form a linear
function on $\mathscr{J}_{\underline{\rho}}$. It follows as in
$\mathscr{J}$ that we have $x^{2}-T(x)x+Q(x)1=0=x^{3}-T(x) x^{2}+Q(x)x$
also $\mathscr{J}_{\underline{\rho}}$. Thus conditions $1$ and $2$
hold.

Now let $\sigma$ be an associative specialization of $\mathscr{J}$
into $\mathfrak{a}$. Since $\sigma$ is a homomorphism of $\mathscr{J}$
into $\mathfrak{a}^{(q)}$ we have
$(x^{k})^{\sigma}=(x^{\sigma})^{k},k=0,1,2,\ldots$ Since
$x^{2}-T(x)x+Q(x)1=0$ in $\mathscr{J}$ we have
$(x^{\sigma})^{2}-T(x)x^{\sigma}+Q(x)1=0$.\pageoriginale By the
universal property 
of $C(\mathscr{J},Q,1)$ we have a unique homomorphism of
$C(\mathscr{J},Q, 1)$ into $\mathfrak{a}$ such that $x^{\sigma_u}\to
x^{\sigma}$. It follows that $(C(\mathscr{J},Q,1),\sigma_u)$ is a
special universal envelope for $(\mathscr{J},U,1)$.

We shall call $(\mathscr{J},U,1)$ the {\em (quadratic Jordan) algebra
  of the form Q with base point} $1$. If we to indicate $Q$ and $1$
then we use the notation Jord $(Q,1)$ for this $(\mathscr{J},U,1)$.

\section{The exceptional quadratic Jordan algebra}\label{c1:sec8}

$\mathscr{H}(\mathscr{O}_3)$, $\mathscr{O}$ {\em an Octonion
  algebra}. A quadratic Jordan algebra which is not special will be
called {\em exceptional}. We have already given one example of this
sort, example (3) of \S $5$. We shall now give the most important
examples of exceptional quadratic Jordan algebra. These are based on
Octonion algebras. We proceed to define these for an arbitrary basic
field.

Let $\Phi$ be a field and let $\rho=\Phi[u]$ be the algebra over
$\Phi$ with base $(1,u)$ over $\Phi$ where $1$ is unit and
\begin{equation*}
u^{2}-y+\rho\quad (\rho=\rho 1),\tag{23}\label{c1:eq23}
\end{equation*}
$\rho \in \Phi$, $4\rho \neq -1$. This is a commutative associative
algebra which has the involution 
\begin{equation*}
x=\alpha+\beta u\to \overline{x}=\alpha+\beta
(1-u),\alpha,\beta\in \Phi\tag{24}\label{c1:eq24}
\end{equation*}

Next we define a {\em quaternion algebra} over $\Phi$ which as $\Phi$
-module is a direct sum of two copies of $\Phi[u]$, so its elements
are pairs $(a,b),a,b\in \Phi [u]$ with the usual vector space
structure. We define a\pageoriginale product in $\mathcal{O}=\Phi[u]\oplus
\Phi[u]$ by 
\begin{equation*}
(a,b)(c,d)=(ac+\sigma \overline{d} b, da
  +b\overline{c})\tag{25}\label{c1:eq25} 
\end{equation*}
where $a,b,c,d\in\Phi [u]$ and $\sigma$ is a fixed non-zero
element of $\Phi$. Then $\mathcal{O}$ is an associative algebra with
$1=(1,0)$ and $\sigma/$ has the standard involution.
\begin{equation*}
x(a,b)\to \overline{x}=(\overline{a},-b).\tag{26}\label{c1:eq26} 
\end{equation*}

Finally, let $\mathcal{O}=\sigma/\oplus\sigma/$ as vector space over
$\Phi$ and define a product in $\mathcal{O}$ by \eqref{c1:eq25} where $\sigma$
replaced by $\tau \neq 0$ in $\Phi$ and the elements are now
in $\mathcal{O}$. The resulting algebra $\mathcal{O}$ is called an
{\em Octonion algebra} over $\Phi$. It has the standard involution
\eqref{c1:eq26}. These algebras are not associative but are {\em alternative}
in the sense that they satisfy the following weakening of the
associative law called the {\em alternative laws:}
\begin{equation*}
x^{2}y=x(xy),\quad yx^{2}=(yx)x\tag{27}\label{c1:eq27}
\end{equation*}

In $\mathcal{O}$ we have $x+\overline{x}=t(x)$ where $t$ is a linear
funtion and $x\overline{x}=n(x)=\overline{x} x$ where $n(x)$ is
quadratic form on $\mathcal{O}$ (values in $\Phi$). $t$ and  $n$ are
called respectively the {\em trace} and {\em norm}.

We write $v=(0,1)$ in $\mathscr{O}$. Then $u$ and $v$ generate $\mathscr{O}$
and we have the basic rules: $vu=\overline{u}v=(1-u)v,u^{2}=u+\rho,
v^{2}=\sigma$. Similarly, we put $w=(0,1)$ in $\mathcal{O}$ and we
have $wu=uw$, $wv=\overline{v}w=-vw$, $w^{2}=\tau u,v,w$ generate
$\mathcal{O}$ and every element of $\mathcal{O}$ can be written
in\pageoriginale one and only one was as $a+bw, a,b\in
\mathscr{O}$. Suppose the base field $\Phi$ is algebraically closed. Then
$\Phi[u]$ is a direct sum of two copies of $\Phi$ since the polynomial
$\lambda^{2}-\lambda -\rho$ is a product of distinct linear
factors. Then $\Phi[u]=\Phi[e]$ where $e^{2}=e,\overline{e}=1-e$. Thus
in this case we may take $\rho=0$. Also replacing $v$ and $w$ by
mutliples of these elements we may suppose $v^{2}=1$, $w^{2}=1$. Then
$(1,u,v,uv,w,uw,vw, (uv)w)$ is a base for $\mathcal{O}$ whose
multiplication table has coefficients which are $0,\pm 1$. For
arbitrary $\Phi$ we shall say that $\mathcal{O}$ is a {\em split}
Octonion algebra if $\rho=0$, $\sigma=0=\tau=1$, or equivalently the
base $(1,u, v,\ldots)$ has the multiplication table just indicated.

Now suppose $\Phi$ is of characteristic $\neq 2$, $\mathcal{O}$ an
octonion algebra over $\Phi$. Let $\mathcal{O}_3$ be the set of
$3\times 3$ matrices with entries in $\mathcal{O}$. This is an algebra
over $\mathcal{O}$ with the usual vector space compositions and matrix
multiplication. We have the standard involution in this algebra: $A\to
\overline{A}^{t}$ where $\overline{A}=(\overline{a}_{ij})$ for
$A=(a_{ij})$. Let $\mathscr{H}(\mathcal{O}_{3})$ be the $\Phi$
-subspace of matrices satisfying $\overline{A}^{t}=A$. This is closed
under the bilinear product $A\cdot B=\frac{1}{2}(AB+BA)$ and it is
well-known that $(\mathscr{H}(\mathcal{O}_3),R,1)$ is a Jordan algebra
if $X R_A=X\cdot A$ (See Jacobson's book [2], p.21). We now consider
the quadratic Jordan algebra $(\mathscr{H}(\mathcal{O}_3),U,1)$ where
$U_a=2R^{2}_a-R_{a^{2}}$ and we wish to analize the $U$ operator in
this algebra. For this we introduce the following notation.
\begin{align*}
\alpha[ii]&=\alpha e_{ii},\quad \alpha \in \Phi\\
a[ij]&=ae_{ij}+\overline{a} e_{ij},a\in \mathcal{O}, i\neq j\tag{$29'$}
\end{align*}

Here\pageoriginale the $e_{ij}$ are the usual matrix units $e_{ij}$
has $1$ in the $(i,j)$ -position $0$'s elsewhere. We have
\begin{equation*}
a[ji]=\overline{a}[ij]\tag{$27'$}
\end{equation*}
and if $\mathscr{H}_{ii}=\{\alpha[ii],\alpha\in \Phi\},
\mathscr{H}_{ij}=\{a[ij]|i \neq j, a\in \mathcal{O}\}$, then 

\begin{equation*}
  \mathscr{H}=(\mathcal{O}_3)=\mathscr{H}_{11}\oplus
  \mathscr{H}_{22}\oplus \mathscr{H}_{33}\oplus \mathscr{H}_{12}\oplus
  \mathscr{H}_{23}\oplus \mathscr{H}_{13}\tag{28}\label{c1:eq28}
\end{equation*}

The $\mathscr{H}_{ii}$ are one dimensional and the
$\mathscr{H}_{ij},i\neq j$, are eight dimensional so dim
$\mathscr{H}=27$. Any $AU_B, A,B\in \mathscr{H}$ is a sum of
elements $xU_y$ where $x,y$ are in the spaces $\mathscr{H}_{ij}$ and
$xU_{y,z}$ where $x,y,z$ are in the $\mathscr{H}_{ij}$ and $y$ and $z$
are not in the same subspace. It is easily checked that the non-zero
$xU_y$, $xU_{y,z}$ of the type just indicated are the following:
\begin{itemize}
\item[(i)] $\beta [ii]U_{\alpha[ii]}=\alpha\beta\alpha[ii]$

\item[(ii)] $\alpha[ii]U_{a[ij]}=\overline{a}\alpha a[jj]$

\item[(iii)] $b[ij]U_{a[ij]}=a\overline{b}a[ij]$
\end{itemize}
(It is easily seen that $(ax)a=a(xa)$ in any alternative
algebra. Hence this is abbreviated to $axa$.)
\begin{itemize}
\item[(iv)] $\{\alpha[ii]a[ij]b[ji]\}=(\alpha
  ab+\overline{\alpha(ab)})[ii]$

\item[(v)] $\{\alpha[ii] \beta[ii]a[ij]\}=\alpha \beta a[ij]$

\item[(vi)] $\{\alpha[ii] a[ij] \beta[jj]\}=\alpha \beta a[ij]$

\item[(vii)] $\{\alpha[ii]a[ij]b[jk]\}=\alpha ab[ik]$

\item[(viii)] $\{a[ij]\alpha[jj]b[jk]\}=\alpha ab [ik]$\pageoriginale

\item[(ix)] $\{a[ij]b[ji]c[ik]\}=a(bc)[ik]$

\item[(x)] $\{a[ij] b[jk] c[ki]\}=(a(bc)+\overline{a(bc)})[ii]$.
\end{itemize}

Now let $\Phi_o$ be a subring of $\phi$ containing $1$. Then
$(\mathscr{H},U,1)$ can be regarded as a quadratic Jordan algebra over
$\Phi_0$. The foregoing formulas show that if $\mathcal{O}_o$ is a
subalgebra of $(\mathcal{O}/\Phi_o,j)$, that is, a subalgebra of
$\mathcal{O}/\Phi_o$ stable under $j$, then the subset $\mathscr{H}_o$
of $\mathscr{H}$ of matrices having entries in $\mathcal{O}_o$ is a
subalgebra of $(\mathscr{H},U,1)$. This is clear since $\mathscr{H}_o$
is the set of sums of $\alpha[ii]$, $a[ij]$ where $\alpha, a\in
\mathcal{O}_o$. It is clear also that if $\mathfrak{K}$ is an ideal in
$(\mathcal{O}_o,j)$ then the set $\mathscr{Z}$ of matrices with
entries in $\mathfrak{K}$ is an ideal in $(\mathscr{H}_o,U,1)$. Hence
we have the quadratic Jordan algebra
$(\mathscr{H}_o/\mathscr{Z},\overline{U}, \overline{1})$.

If $\Phi$ has characteristic $\neq 2$ then $\overline{a}=a$ in
$\mathcal{O}$ if and only if $a=\alpha \in \Phi$. This is not the
case for characteristic two accordingly, in this case we let
$\mathscr{H}(\mathcal{O}_3)$ denote the set of $3\times 3$ matrices
with entries in $\mathcal{O}$ such that $\overline{A}^{t}=A$ and {\em
  the diagonal entries are in} $\Phi$. (For $\Phi$ of characteristic
$\neq 2$ the latter condition is implied by the former.) Then
$\mathscr{H}=\mathscr{H}(\mathcal{O}_3)=\sum\limits_{i\le j=1}^{3}
\mathscr{H}_{ij}$ where $\mathscr{H}_{ij}$ is as before. Then it is
easily seen that there is a unique quadratic mapping of $\mathscr{H}$
into End $\mathscr{H}$ such that the formulas (i)-(x) hold and all
other $xU_y, xU_{y,z}$ are $0$ where $x,y,z$ are in the subspaces
$\mathscr{H}_{ij}, y$ and $z$ not in the same subspace. For any
characteristic we have
\begin{thm}\label{c1:thm5}
  $(\mathscr{H},U,1)$ is a quadratic Jordan algebra.
\end{thm}\pageoriginale

\begin{proof}
The case in which the characteristic $\neq 2$ has been settled before,
so we assume the characteristic is $2$. Assume first that
$\Phi=\mathbb{Z}_2$ the field of two elements and $\mathcal{O}$ is the
split octonion algebra over $\Phi$. Let $\mathcal{O}'$ be the split
octonion algebra over the rationals $Q,\mathcal{O}_o$ the $\mathbb{Z}$
-subalgebra of $(\mathcal{O}'j)$ of integral linear combinations of
the base $(1,u,v,\ldots)$ Then we have the $\mathbb{Z}$ -quadratic
Jordan algebra $(\mathscr{H}(\mathcal{O}_{o3}),
U,1)/2\mathscr{H}(\mathcal{O}_{0_3})$ which can be regarded as a
$\mathbb{Z}_2$ -quadratic Jordan algebra. Moreover, it is clear that
$(\mathscr{H}(\mathcal{O}_3),U,1)$ (over $\mathbb{Z}_2$) is isomorphic
to this. Hence $\mathscr{H}(\mathcal{O}_3,U,1)$ is a quadratic Jordan
algebra over $\mathbb{Z}_2$. Now let $\Phi$ be arbitrary of
characteristic two, $\mathcal{O}$ an arbitrary octonion algebra. To
prove $(\mathscr{H}(\mathcal{O}_3),U,1)$ is Jordan it is enough to
show that the conditions $QJ 3,4, 6 - 9$ hold for the $U$ -operator
(Theorem \ref{c1:thm1}). These hold if and only if they hold for
$(\mathscr{H}(\mathcal{O}_3)_{\Omega},U,1)$ where $\Omega$ is the
algebraic closure of $\Phi$. Also we may identify
$\mathscr{H}(\mathcal{O}_3)_{\Omega}$ with
$\mathscr{H}(\mathcal{O}_{\Omega})_3$. Hence it suffices to assume
$\Phi$ algebraically closed. Then $\mathcal{O}$ is split. Now it is
clear from the definition of  a split algebra that if $\mathcal{O}_o$
is the split octonion algebra over $\mathbb{Z}_2$ then
$\mathcal{O}=\mathcal{O}_{o\Phi}=\Phi\otimes_{\mathbb{Z}2}\mathcal{O}_0$
so
$(\mathscr{H}(\mathcal{O}_3),U1)=(\mathscr{H}(\mathcal{O}_{o3})_{\Phi},U,1)$. Since 
we have just seen that the latter is a quadratic Jordan algebra it
follows that $(\mathscr{H}(\mathcal{O}_3),U,1)$ is a quadratic Jordan
algebra. 

We have seen in Theorem \ref{c1:thm3} that a quadratic Jordan algebra
over $\Phi$ with $2\Phi=0$, is a $2$-Lie algebra relative to
$[a,b]=a\circ b= (a+b)^{2}-a^{2}-b^{2}$ and $a^{[2]}=a^{2}$. In
particular, this holds for
$(\mathscr{H}(\mathcal{O}_3),U,1)$\pageoriginale where $\mathcal{O}$
is an octonion algebra over a field of characteristic two. We now note
that in this case $A^{2}=1U_A$ is the same as the square of the matrix
$A\in\mathscr{H}$ as defined in $\mathcal{O}_3$. To see this it
is sufficient to show that $1U_{a[ij]}=(ae_{ij}+\overline{a}\cdot
e_{ji})(ae_{ij}+\overline{a}
e_{ji})=n(a)(e_{ii}+e_{jj})=n(a)[ii]+n(a)[jj],1U_{\alpha[ii]}=(\alpha
e_{ii})(\alpha e_{ii})=\alpha^{2}[ii]$, $\{x1y\}=xy+yx$ if $x,y$ are
in different spaces $\mathscr{H}_{ij}$. By the defining formulas
$1U_{a[ij]}=1[ii]U_{a[ij]}+1[ji]U_{a[ij]}=n(a)[jj]+n(a)[ii]$ (by
$(ii)$), $1U_{\alpha[ii]}=1[ii]U_{\alpha[ii]}=1[ii]
U_{\alpha[ii]}=\alpha^{2}[ii]$ (by $(i)$). By
$(v)$,$\{\alpha[ii]1c[ij]\}=\{\alpha[ii]1[ii]c [ij]\}=\alpha
c[ij]=\alpha c e_{ij}+\alpha \overline{c} e_{ji}$. On the other hand,
$\alpha e_{ii}(c e_{ii}+\overline{c} e_{ji})+(c e_{ij}+\overline{c}
e_{ji})(\alpha e_{ii})=\alpha c e_{ij}+ \alpha \overline{c}
e_{ji}$. By ($Viii$). $\{a[ij]1c[jk]\}=ac[ik]$. Also $a[ij]c[jk]+c[jk]
a[ij]=(ae_{ij}+\overline{a} e_{ji})(c e_{jk}+\overline{c} e_{kj})+(c
e_{jk}+\overline{c} e_{kj}) (a e_{ij}+\overline{a} e_{ji})=ac e_{ik}
+\overline{c}\overline{a} e_{ki}=ac[ik]$. The remaining $\{x1y\}$ and
$xy+yx$ are $0$. Hence we have proved our assertion and we have the
following consequence of Theorems \ref{c1:thm4} and \ref{c1:thm5}:
\end{proof}

\begin{coro*}
Let $\mathcal{O}$ be an octonion algebra over a field of
characteristic two, $\mathscr{H}(\mathcal{O}_3)$ the set of $3\times
3$ hermitian matrices in $\mathcal{O}_3$ with diagonal entries in
$\Phi$. Then $\mathscr{H}(\mathcal{O}_3)$ is a $2$-Lie algebra
relative to $AB=AB+BA$ and $A^{[2]}=A^{2}$.

Theorem \ref{c1:thm5} has an important generalization in which the octonion
algebra is replaced by an alternative algebra with involution
$(\mathcal{O}, j)$ such that all norms $d\overline{d}=dd^{j},d\in
\mathcal{O}$, are in the nucleus. We recall that the nucleus of  a
non-associative algebra is the set of elements $\alpha$ such that
$[\alpha, x,y]=(\alpha x)y-\alpha(xy)=0,[x,\alpha,y]=0,[x,y,\alpha]=0$
for all $x,y$\pageoriginale in the algebra. In an alternative algebra
the associator $[x,y,z]\equiv (xy)z-x(yz)$ is an alternating function
in the sense that $[x,y,z]$ is unchanged under even permutation of the
arguments and changes sign under odd permutation. Hence $\alpha
\in N(\mathcal{O})$ the nucleus of the alternative algebra
$\mathcal{O}$ if and only if $[\alpha,x,y]=0,x,y\in
\mathcal{O}$. Now suppose $(\mathcal{O}, j)$ is an alternative algebra
satisfying the condition that norms are in the nucleous
$N(\mathcal{O})$. Let $N_o$ be the $\Phi$ -submodule of
$N(\mathcal{O})$ generated by the norms. If $x,y\in \mathcal{O}$
then
$(x+y)(\overline{x}+\overline{y})-x\overline{x}-y\overline{y}=x\overline{y}+y\overline{x}\in
N_o$. In particular, $t(x) = x+ \ob{x} \in N_o$.
It follows that if $\Phi$ contains $\frac{1}{2}$ then
$\mathscr{H}(\mathcal{O}, j)\subseteq N(\mathcal{O})$ so the
condition in this case is that the symmetric elements of $\mathcal{O}$
are contained in the nucleus. Again suppose $\mathcal{O}$ arbitrary
and $(\mathcal{O},j)$ satisfies the norm condition. Then we have the
following results (McCrimmon):
\begin{itemize}
\item[1)] $xN_o\overline{x}\subseteq N_o,\quad x\in \mathcal{O}$

\item[2)] $xN\overline{x} \subseteq N$

\item[3)] If $N'=N\cap\mathscr{H}(\mathcal{O},j)$then
  $xN'\overline{x}\subseteq N'$
\end{itemize}
\end{coro*}

\begin{proof}
  \begin{enumerate}
    \item  We shall use $(xa)\overline{x}=x(a\overline{x})$ which we write
  as $x a\overline{x}$. Also we shall need Moufang's identity:
  $(ax)(ya)=a(xy)a$ which holds in any alternative algebra. It is
  enough to prove $x(y \overline{y})\overline{x}\in
  N_o,x,y\in \mathcal{O}$. We have
  $x(y\overline{y})=x(y(t(y)-y))=x(yt(y))-xy^{2}=(xy)t(y)-(xy)t(y)-(xy)y=(xy)\overline{y}$. Hence
  $x(y\overline{y})\overline{x}=(x(y\overline{y}))(t(x)-x)=(x(y\overline{y}))t(x)-(x(y\overline{y}))x=((xy)\overline{y})t(x)-(xy)(\overline{y}x)$
  (by Moufang)$=(xy)(\overline{y} t(x)-\overline{y}
  x)=(xy)(\overline{y}\overline{x})=(xy)(\ob{x}\ob{y})\in
  N_o$. 
  \item (2) We use $\alpha[x,y,z]=[\alpha x, y, z]=[x\alpha, y,
    z]=[x,y,z]\alpha$ for\pageoriginale $x,y,z\in \mathcal{O},
  \alpha \in N$, and $(xyx)z=x(y(xz))$ (see the author's book
         [2], pp. 18-19). We have to show that $[x\alpha
           \overline{x},y,z]=0$ if $\alpha \in N, x, y, z\in
         \mathcal{O}$. Since $x\overline{x}\in N$ this will
         follow by showing that $[x\alpha
           \overline{x},y,z]=[x\overline{x},y,z]\alpha$. For this we
         have the following calculation:
         \begin{align*}
           [x\alpha \overline{x},y,z]&=[x \alpha t(x),y,z]-[x\alpha x,y,z]\\
           &=t(x)[x,y,z]\alpha + (x(\alpha(x(yz)))-(x(\alpha (xy)))z\\
           &=t(x)[x,y,z]\alpha - (x\alpha) [x,y,z]+(x\alpha)((xy)z)\\
           &-[x\alpha, xy,z]=(x\alpha)((xy)z)\\
           &=g(x,y,z)\alpha
         \end{align*}
         where $g(x,y,z)=t(x)[x,y,z]-x[x,y,z]-[x,x y,z]$. Taking $\alpha=1$ we
         have $g(x,y,z)=[x\overline{x},y,z]$. Hence $[x\alpha \overline{x}, yz,
           z]=[x\overline{x},y,z]\alpha=0$. 
       \item  is an immediate consequence of this. 
  \end{enumerate}

Now consider the algebra $\mathcal{O}_3$ of $3\times 3$ matrices with
entries in $\mathcal{O}$. By $\mathscr{H}(\mathcal{O}_3)$ we shall now
understand the set of hermitian matrices of
$\mathcal{O}_3(\overline{A}^{t}=A)${\em with} diagonal entries in
$N'=N\cap\mathscr{H}(\quad, j)$. If $\mathcal{O}$ is associative then
$\mathscr{H}(\mathcal{O}_3)$ is just the set of hermitian matrices. In
any case the elements of $\mathscr{H}(\mathcal{O}_3)$ are sums of
elements $\alpha[ii]$, $\alpha\in N'$, and $a[ij], a\in
\mathcal{O}, i\neq j$. Since $N_o\subseteq N'$ and all traces are in
$N_o$ it is clear from $(1)-(3)$ that the right hand sides of
$(i)-(x)$ are contained in $\mathscr{H}(\mathcal{O}_3)$. Hence we can
define a unique quadratic mapping of $\mathscr{H}(\mathcal{O}_3)$ into
End $\mathscr{H}(\mathcal{O}_3)$ such that $(i)-(x)$ hold and the
remaining $x U_y, xU_{y,z}=0$ for $x,y$ of the form $\alpha[ii]$ or
$a[ij]$. It has been proved\pageoriginale by McCrimmon that
$\mathscr{H}(\mathcal{O}_3,U,1)$ is a quadratic Jordan algebra.

The algebras $\mathscr{H}(\mathcal{O}_3)$ with $\mathcal{O}$ not
associative are exceptional. In fact, we have the following stronger result:
\end{proof}

\begin{thm}\label{c1:thm6}
If $(\mathscr{H}(\mathcal{O}_3),U,1)$ is a homomorphic image of a
special quadratic Jordan algebra then $\mathcal{O}$ is associative.
\end{thm}

\begin{proof}
The proof we sketch is due to Glennie and is given in detail on p.49
of the author's book [2]. One can show that the following identity
holds in every $\mathfrak{a}^{(q)},\mathfrak{a}$ associative:
\begin{align*}
  xzx &\circ \{y(zy^{2} z)x\}-yzy \circ \{x(zy^{2} z)y\}\\
  &=x(z\{x(yzy)y\}z)x-y(z\{y(xzx)x\}z)y.\tag{29}\label{c1:eq29}
\end{align*}

On the other hand, if one takes
$$
x=1[12],\, y=1[23],\, z=a[21]+b[13]+c[32]
$$
then one can see that the $(1,3)$ entry in the matrix on the left side
of \eqref{c1:eq29} is $a(bc)-(ab)c$ while the $(1,3)$ -entry on the right hand
side is $0$. Hence if \eqref{c1:eq29} is to hold in
$\mathscr{H}(\mathcal{O}_3)$ then $a(b c)=(ab)c$ for  all
$a,b,c\in \mathcal{O}$ so $\mathcal{O}$ is associative. Clearly
this identity holds if $(\mathscr{H}(\mathcal{O}_3),U,1)$ is a
homomorphic image of a special quadratic Jordan algebra.
\end{proof}

\section{Quadratic Jordan algebras defined by certian cubic
  forms.}\label{c1:sec9} 

In this section we assume the base ring $\Phi$ is an infinite
field. We shall give another definition of the quadratic Jordan
structure on $\mathscr{H}(\mathcal{O}_3)$, $\mathcal{O}$
an\pageoriginale octonion algebra over $\Phi$. As before,
$\mathscr{H}(\mathcal{O}_3)$ denotes the set of $3\times 3$ hermitian
matrices with entries in $\mathcal{O}$, diagonal entries in $\Phi$. If
$a=\sum\limits_{1}^{3}\alpha_i[ii]+\sum\limits_{i=1}^{3} a_i[jk]$, where
$(i,j,k)$ is a cyclic permutation of $(1,2,3)$ and the notations are
as in \S $8$, then we define a ``determinant'' by
\begin{equation*}
  N(a)=\det  a=\alpha_1\alpha_2\alpha_3-\sum\limits_{1}^{3}
  \alpha_in(a_i)+t((a_1a_2)a_3)\tag{30}\label{c1:eq30}  
\end{equation*}

Here $n(a)=a\overline{a}, t(a)=a+\overline{a}$ in $\mathcal{O}$. It is
known that $t((a_1a_2)a_3)=t(a_1(a_2 a_3))$ so we write this as
$t(a_1a_2a_3)$. Also it is known that $t(a_1a_2a_3)$ is unchanged
under cyclic permutation of the arguments. If $f$ is a rational
mapping of $\mathscr{H}$ into a second finite dimensional space then
we let $\Delta^{b}_af$ denote the directional derivative of $f$ at $a$
in the direction $b$ (see the author's book, pp, 215-221). In
particular, if $f$ is a polynomial function then we have $f(a+\lambda
b)\equiv f(a)+(\Delta_a^{b}f)\lambda (mod \lambda^{2})$ and
$\Delta_{a}^{b}f$ is determined by this condition. Since $N$ is
polynomial mapping which is homogeneous of degree three we have
\begin{equation*}
N(a+\lambda b)=N(a)+(\Delta_a^{b}N)\lambda + (\Delta_b^{a} N)
\lambda^{2}+N(b)\lambda^{3}\tag{31}\label{c1:eq31} 
\end{equation*}

By \eqref{c1:eq30} we have for $a=\sum \alpha_i [ii]+\sum a_i [jk], b=\sum
\beta_i[ii]\sum b_i [jk]$ that
\begin{equation*}
  \Delta_a^{b}N=\sum\limits_i\beta_i\alpha_j\alpha_k-\sum\limits_{i}\beta_i
  n(a_i)-\sum\limits_i\alpha_i t(\ob{a}_{i},b_i)+\sum\limits_{i}
  t(b_i a_j, a_i)\tag{32}\label{c1:eq32}
\end{equation*}

We define $T(a, b)=-\Delta_{1}^{a}\Delta^{b}log
N=(\Delta_{1}^{a}N) (\Delta_{1}^{b} N)-\Delta_{1}^{a}(\Delta^{b} N)$ so
$T$\pageoriginale is a symmetric bilinear form in $a$ and $b$. By
\eqref{c1:eq32}, we have
\begin{align*}
  T(a,b)&=\left(\sum\alpha_i\right) \left(\sum\beta_i\right) -
  \sum\limits_i\beta_i(\alpha_j+\alpha_k)-\sum 
  t(\ob{a}_i,b_i)\\
&=\sum \alpha_i\beta_i-\sum t(\ob{a}_i,b_i)\tag{33}\label{c1:eq33}
\end{align*}

Since $t(a,b)$ is non-degenerate on $\mathcal{O}$, $T(a,b)$ is
non-degenerate on $\mathscr{H}$. If we define the ``adjoint matrix'',
$a^{\sharp}$ by
\begin{equation*}
a^{\sharp}\sum\limits_{i}(\alpha_j\alpha_k-n(a_i)[ii] + \sum\limits_i
(\ob{a_ja_k}-\alpha_i a_i)[jk]\tag{34}\label{c1:eq34} 
\end{equation*}
it is easy to check that
\begin{equation*}
  \Delta^{b}_aN=T(a^{\sharp},b)\tag{35}\label{c1:eq35}
\end{equation*}

A straight forward verifiction using Moufang's identity shows that we
have
\begin{equation*}
  a^{\sharp\sharp}=N(a)a.\tag{36}\label{c1:eq36}
\end{equation*}

It is clear from the definition of $N$ that $N(1)=1$. We now define
$T(a)=T(a, 1)=T(a,1^{\sharp})$ since $1^{\sharp}=1$ by \eqref{c1:eq34}. Then
\eqref{c1:eq33} gives $T(a)=\sum\alpha_i$. We define $a\times
b=(a+b)^{\sharp}-a^{\sharp}-b^{\sharp}$. We have $T(a,b)=(\Delta_1^{a}
N)(\Delta_1^{b} N)-\Delta^{a}_1(\Delta^{b}
N)=T(a)T(b)-\Delta_1^{a}(\Delta^{b} N)$ (by \eqref{c1:eq35}). Since $N$ is
cubic form ($=$homogeneous polynomial function of degree three we have
$\Delta_{i}^{a}(\Delta^{b}N)=\Delta_x^{c}(\Delta^{a}
(\Delta^{b}(^{b}N)))$ is independent of $x$ and is symmetric in $a, b,
c$. Hence $\Delta^{a}_{1}(\Delta^{b}
N)=\Delta_b^{a}(\Delta^{a}N)=\Delta_b^{a}T(a^{\sharp})$ (by
\eqref{c1:eq35})$=T(a\times b)$. Hence we have
\begin{equation*}
  T(a\times b)=T(a)T(b)-T(a,b)\tag{37}\label{c1:eq37}
\end{equation*}\pageoriginale

 We define the {\em characteristic polynomial} $f_a(\lambda)N(\lambda
 1-a)$. By (31) and the definition of $T(a)$ we have
\begin{equation*}
  f_a(\lambda)=N(\lambda 1-a) = \lambda^{3}-T(a) \lambda^{2} +
  S(a)\lambda -N(a)\tag{38}\label{c1:eq38} 
\end{equation*}
where $S(a)=T(a^{\sharp})$. Direct verification, using \eqref{c1:eq34} and the
foregoing definitions shows that
\begin{equation*}
a^{\sharp}=a^{2}-T(a)a+S(a)1\tag{39}\label{c1:eq39}
\end{equation*}
where $a^{2}$ is the usual matrix square. Since $S(a)=T(a^{\sharp})$
and $T$ is linear and satisfies $T(1)=3$ this gives
\begin{equation*}
  2T(a^{\sharp})=T(a)^{2}-T(a^{2})\tag{40}\label{c1:eq40}
\end{equation*}
We now suppose that $\mathscr{H}=\mathscr{H}(\mathcal{O}_3)$ is
endowed with the quadratic Jordan structure given in \S $8$. Then
$a^{2}=1U_a$ is the usual square of a and $a^{3}=aU_a$. We shall now
establish the following formula for the $U$-operator in $\mathscr{H}$:
\begin{equation*}
  bU_a=T(a,b)a-a^{\sharp}\times b\tag{41}\label{c1:eq41} 
\end{equation*}

We shall establish this using the foregoing formulas and the
Hamil\-ton-Cayley type theorem that
\begin{equation*}
  f_a(a)=a^{3}-T(a)a^{2}+S(a)a-N(a)1=0,\tag{42}\label{c1:eq42}
\end{equation*}
which we prove first. Suppose first that the characteristic is $\neq
2$. Then\pageoriginale $a^{3}aU_a=\frac{1}{2}(aa^{2}+a^{2}a)$. Then
one can verify \eqref{c1:eq42} by direct calculation (see the author's book [2],
p.232). Next assume char. $=2$. Then we shall establish \eqref{c1:eq42} by a
reduction $\mod$ $2$ argument similar to that used in \S $8$. We note
first that we may assume the base field is algebraically closed. Then
$\mathcal{O}$ is split and has a canonical base $(u_1,u_2,\ldots,u_8)$
with multiplication table in $\mathbb{Z}_2$ as in\S $8$. We obtain a
corresponding canonial base $(v_1,\ldots,v_{27})$ for
$\mathcal{O}/\Phi$ where $v_i=i[ii]$, $i=1,2,3$ and $v_j$, for $j>3$,
has the form $u_k[12]$, $u_k[13]$ or $u_k[23]$, $k=1,2,\ldots,8$. Now
let $\xi_1, \xi_2,\ldots,\xi_{27}$ be indeterminates and consider the
``generic'' element $x=\sum\limits_1^{27}\xi_jv_j$ in
$\mathscr{H}_{\ub{\rho}},\ub{\rho}=\Phi(\xi),\xi=(\xi_1,\ldots,\xi_{27})$. By
specialization it suffices to prove $(42)$ for $a=x$. Let
$\ub{\rho}_o=\mathbb{Z}_2(\xi),\mathscr{H}_o=\sum \ub{\rho}_o v_j$, so
$\mathscr{H}_o$ is quadratic Jordan algebra over $\ub{\rho}_o$ and
$x\in \mathscr{H}_o$. The functions $S,T,N$ on $\mathscr{H}_o$
are the restriction of the corresponding ones on $\mathscr{H}_o$ are
the restriction of the corresponding ones on $\mathscr{H}$. Hence it
suffices to prove the result for $x$ in $\mathscr{H}_o$. This follows
by applying a $\mathbb{Z}$ -homomorphism of an algebra
$\mathscr{H}'=\sum\ub{\rho}'v'_j$ where $\ub{\rho}'=\mathbb{Z}(\xi)$
and the $v'_j$ are obtained from a canonical base of the split
octonion algebra $\mathscr{O}'/Q$ as in \S $8$.

We now begin with \eqref{c1:eq42}. A linearization of this relation by 
replacing a by $a+\lambda b$ and taking the coefficient of $\lambda$ gives
\begin{align*}
  bU_a&=-a^{2}\circ b+T(a)(a\circ b)+T(b) a^{2}-T(a\times
  b)a-T(a^{\sharp})b+T(a^{\sharp},b)1\\
  &=-(a^{\sharp}+T(a)a-T(a)a-T(a^{\sharp})1)\circ b+T(b)(a^{\sharp}+\\
  &T(a)a-T(a^{\sharp})1)+T(a)a\times b -T(a^{\sharp})b\\
  &-(T(a)T(b)-T(a,b))a+T(a^{\sharp},b)\quad ((39), (37))\\
  &=-a^{\sharp}\circ b
  +T(a^{\sharp})b+T(b)a^{\sharp}-(T(b)T(a^{\sharp})-T(a^{\sharp},b))\\
  &+T(a,b)a\\
  &=T(a,b)a-a^{\sharp}\times b
\end{align*}\pageoriginale
using \eqref{c1:eq37} and $a\times b=a\circ b T(a)b-T(b)a+T(a\times
b)1$ which is the linearization of \eqref{c1:eq39}. Hence \eqref{c1:eq41} holds.

We now assume we have a finite dimensional vector space $\mathscr{J}$
over an infinite field $\Phi$ equipped with a cubic form $N$, a point
$1$ satisfying $N(1)=1$, such that:
\begin{itemize}
\item[(i)] $T(a,b)=-\Delta_{1}^{a}\Delta^{b}\log N = (\Delta_1^{a} N)
  (\Delta_1^{b}N)-\Delta_1^{a}(\Delta^{b}N)$ 
\end{itemize}
is a non-degenerate symmetric bilinear form in $a$ and $b$.
\begin{itemize}
\item[(ii)] If $a^{\sharp}$ is defined by
  $T(a^{\sharp},b)=\Delta_a^{b}N$ then $a^{\sharp\sharp}=N(a)a$. 
\end{itemize}
We define
\begin{itemize}
\item[(iii)] $bU_a=T(a,b)a-a^{\sharp}\times b$
\end{itemize}
where $a\times b=(a+b)^{\sharp}-a^{\sharp}-b^{\sharp}$. Then  we have 

\begin{thm}\label{c1:thm7}
  $(\mathscr{J}, U, 1)$ is a quadratic Jordan algebra.
\end{thm}

\begin{proof}
  We can linearize (ii) to obtain
  \begin{align*}
    &a\times (a\times b)=N(a)b+T(a^{\sharp},b)a\tag{43}\label{c1:eq43}\\
    & a^{\sharp}\times b^{\sharp}+(a\times b)^{\sharp} =
    T(a^{\sharp},b) b+T(b^{\sharp},a)a\tag{44}\label{c1:eq44}
  \end{align*}
  we have $T(1,b)=(\Delta'_1 N)(\Delta^b_1
  N)-\Delta'_1(\Delta^{b}N)=3N(1)\Delta_1^{b}N-2\Delta^{b}_1 N$ (by
  Euler's theorem on homogeneous polynomial
  functions)$=\Delta_1^{b}N=T(1^{\sharp},b)$.
  
  Hence\pageoriginale
  \begin{equation*}
    1^{\sharp}=1\tag{45}\label{c1:eq45}
  \end{equation*}
  by the non-degeneracy of $T$. Since $N$ is a cubic form
  $\Delta_x^{c}(\Delta^{a}\Delta^{b}N)$ is independent of $x$ and
  symmetric in $a,b,c$. By $(ii)$ we have $T(a\times
  c,b)=\Delta^{a}_c\Delta_N^{b}=\Delta_x^{c}(\Delta^{a}\Delta^{b}N)$. Hence
  $T(a\times c,b)$ is symmetric in $a,b,c$ and in particular we have
  $T(a,b\times 1)=T(a\times b,
  1)=\Delta^{a}_1\Delta^{b}N=(\Delta_{1}^{a}N)(\Delta_1^{b}N)-T(a,b)=T(a)T(b)-T(a,b)$
  where $T(a)=T(a, 1^{\sharp})=T(a,1)$. This and the non-degeneracy of
  $T$ imply
  \begin{equation*}
    b\times 1 =T(b)1-b\tag{46}\label{c1:eq46}
  \end{equation*}
  By \eqref{c1:eq43} and the symmetry of $T(a\times c,b)$ we have $T(b,(c\times
  a^{\sharp})\times a)=T(b\times a, c\times a^{\sharp})=T(b\times
  a)\times
  a^{\sharp},C=T(N(a)b,c)+T(T(a^{\sharp},b)a,c)=T(b,N(a)c)+T(b,T(a,c)a^{\sharp})$. Hence
  \begin{equation*}
    (c\times a^{\sharp})\times a=N(a)c+T(a,c)a^{\sharp}\tag{47}\label{c1:eq47}
  \end{equation*}
  Since $T(bU_a,c)=T(a,b)T(a,c)-T(a^{\sharp}\times b, c)$ is symmetric
  in $b$ and $c$ we have
  \begin{equation*}
    T(bU_a,c)=T(b,cU_a)\tag{48}\label{c1:eq48}
  \end{equation*}

  Next we note that $T(bU_a)^{\sharp},c)=T((T(a,b)a-a^{\sharp}\times
  b)^{\sharp},c)=T(T(a,\break b)^{2}$ $a^{\sharp}+(a^{\sharp}\times
  b)^{\sharp}-T(a,b)(a^{\sharp}\times b)\times a,
  c)=T9(a,b)^{2}T(a^{\sharp},c)-T(N)(a)a\times
  b^{\sharp}-N(a)T(a,b)b-T(a^{\sharp},b^{\sharp})a^{\sharp},c-T(a,b)T(N(a)b+
T(a,b)a^{\sharp},c)$ (by \eqref{c1:eq44}, \eqref{c1:eq47} and (ii))$=T(a^{\sharp},
b^{\sharp})T(a^{\sharp},c)-N(a)T(a\times b^{\sharp},c)=T(b^{\sharp}
U_{a\sharp},c)$. Hence
\begin{equation*}
  (bU_a)^{\sharp}=b^{\sharp}U_{a\sharp}\tag{49}\label{c1:eq49} 
\end{equation*}\pageoriginale

Linearization of this relative to $b$ gives
\begin{equation*}
  bU_a\times cU_a=(b\times c)U_{a^{\sharp}}\tag{50}\label{c1:eq50}
\end{equation*}

We can now provce $QJ3$. For this we consider
$T(xU_{bU_a,y}) = T(T$ $(bU_a,x)bU_a-(bU_a)^{\sharp}\times x,y)$. Since
$T(bU_{a,x})=T(b, xU_{a})$ and $T((bU_a)^{\sharp}\times x,
y)=T(b^{\sharp}U_{a^{\sharp}}\times x,
y)=T(b^{\sharp}U_{a^{\sharp}},x\times y)=T(b^{\sharp}, (x\times y)
U_{a^{\sharp}})=T(b^{\sharp},xU_a\times yU_a)$ (by
$(50)$)$=T(b^{\sharp}\times xU_a, yU_a)=T((b^{\sharp}\times
xU_{a})U_a,y)$ the foregoing relation becomes
$T(xU_{bU_{a}},y)=T(T(b,xU_a)bU_a-(b^{\sharp}\times
xU_a)U_a,y)=T(xU_aU_bU_a,y)$. Hence $QJ3$ holds. To prove $QJ4$ we
note that the definition of $U_a$ and $V_{a,b}$ give
$xV_{a,b}=T(x,a)b+T(a,b)x-(x\times b)\times a$. Hence
\begin{align*}
xV_{a,b}U_a&-xU_aV_{b,a}=T(x,a)bU_a-((x\times b)\times a)U_a\\
&-T(xU_a,b)a+(xU_a\times a)\times b
\end{align*}

Using the symmetry of $T(x\times y,z)$ and \eqref{c1:eq48} we obtain
\begin{gather*}
T(T(x,a)bU_a-((x\times b)\times a)U_a-T(xU_a,b)a+\\
(xU_a\times a)\times b,y)=T(b,T(x,a)yU_a-\\
T(b,((yU_a\times a)\times x)-T(b,T(a,y)x U_a+T(b,y\times (xU_a\times a)).
\end{gather*}

It suffices to show this is $0$ and this will follows by showing that
\pageoriginale$T(x,a)yU_a-(yU_a\times a)\times x$ is symmetric in $x$ and $y$. We
have $T(x,a)yU_a-(yU_a\times a)\times
x=T(x,a)T(y,a)a-T(x,a)a^{\sharp}\times y-T(a,y)(a\times a)\times
x+((a^{\sharp}\times y)\times x=T(x,a)T(y,a)a-T(x,a)a^{\sharp}\times
y-2T(a,y)a^{\sharp}\times x +N(a)(y\times x)+T(a,y)a^{\sharp}\times
x=T(x,a)T(y,a)a-T(x,a)a^{\sharp}\times y-T(a,y)a^{\sharp}\times
x+N(a)(y\times x)$. Since this is symmetric in $x$ and $y$ we have
$QJ4$. Also we have $xU_1=T(x)1-1\times x=x$ by $(46)$. To prove $QJ5$
we observe that if $\ub{\rho}$ is an extension field of $\Phi$ then
$QJ3$ and $QJ4$ are valid for $\mathscr{J}_{\ub{\rho}}$ since the
hypothesis made on $N$ carry over to the polynomial extension of $N$
to $\mathscr{J}_{\ub{\rho}}$. In particular, there hold if
$\ub{\rho}=\Phi(\lambda)$. Then the argument in \S $3$ shows that
$QJ6-9$ hold in $\mathscr{J}$. Hence $\mathscr{J}$ is a quadratic
Jordan algebra by Theorem \ref{c1:thm1}. 

A cubic form $N$ and element $1$ with $N91)=1$ satisfying (i)-(iii)
will be called {\em admissible}. We shall now give another imporatant
example of an admissible $(N,1)$ which is due to Tits (see the
author's book [2], pp.412-422). Let $\mathfrak{a}$ be a central simple
associative algebra of degree three (so $\dim\mathfrak{a}=q$) and let
$n$ be the generic ($=$ reduced) norm on $\mathfrak{a}$, $t$ the
generic trace. Let
$\mathscr{J}=\mathfrak{a}\oplus\mathfrak{a}\oplus\mathfrak{a}$ a
direct sum of three copies of $\mathfrak{a}$. We write the elements of
$\mathscr{J}$ as triples $x=(a_0,a_1,a_2)$, $a_i\in
\mathfrak{a}$. Let $\mu$ be a non-zero element of $\Phi$ and define
\begin{equation*}
  N(x)=n(a_o)+\mu n(a_1)+\mu^{-1}n(a_2)-t(a_oa_1a_2)\tag{51}\label{c1:eq51} 
\end{equation*}
If we put $1=(1,0,0)$ we have $N(1)=1$. It is not difficult to verify
that $(N,1)$ is admissible. Hence $\mathscr{J}$ with the $U$-operator
defined by (iii) is a quadratic Jordan algebra. It can be shown
that\pageoriginale these algebras are of the form 
$\mathscr{H}(\mathcal{O}_3)$ and hence are exceptional also. 
\end{proof}

\section{Inverses}\label{c1:sec10}

If $a,b$ are elements of an associative algebra such that $aba=a,
ab^{2}a=1$ then $a$ is invertiable by the second condition, so
$ab=1=ba$ by the first. Thus $a$ is invertible with inverse
$b=a^{-1}$. Conversely, if $a$ is invertible then $aa^{-1}a=a$ and
$aa^{-2}a=1$. This motivates the following:

\begin{defn}\label{c1:defn5}
An element $a$ of a quadratic Jordan algebra $(\mathscr{J},U,1)$ is
{\em invertible} if there exists $a b$ in $\mathscr{J}$ such that
$aba=a, ab^{2}a=1$. Then $b$ is called an {\em inverse} of $a$.

The foregoing remark shows that if $\mathscr{J}=\mathfrak{a}^{(q)}$,
$\mathfrak{a}$ associative then $a$ is invertible in $\mathscr{J}$
with inverse $b$ if and only if $ab=1=ba$ in $\mathfrak{a}$. If
$\sigma$ is a homomorphism of $\mathscr{J}$ into $\mathscr{J}'$ and
$a$ is invertible with inverse $b$ then $a^{\sigma}$ is invertible in
$\mathscr{J}'$ with inverse $b^{\sigma}$. In particular if
$\mathscr{J}'=\mathfrak{a}^{(q)}$ then
$a^{\sigma}b^{\sigma}=1=b^{\sigma}a^{\sigma}$. 
\end{defn}
We have the following


\begin{thminv}
  The following conditions are equivalent: (i) $a$ is invertible, (ii)
  $U_a$ is invertible in End $\mathscr{J}$, (iii) $1\in
  \mathscr{J}U_a(2)$ If $a$ is invertible the inverse $b$ is unique and
  $b=aU_a^{-1}$. Also $U_b=U_a^{-1}$. \qquad and if we put $b=a^{-1}$
  then $a^{-1}$ is invertible and $(a^{-1})^{-1}=a$. (3) We have $a\circ
  a^{-1}=2$, $a^{2}\circ a^{-1}=2a, V=V_aU^{-1}_a=U_{a}^{-1}V_a$. (4)
  $aba$ is invertibel if and only if $a$ and $b$ are invertible, in
  which case $(aba)^{-1}=a^{-1}b^{-1}a^{-1}$.
\end{thminv}

\begin{proof}\pageoriginale
\begin{itemize}
\item[(1)] If $b^{2}U_a=1$ then $1=U_1=U_{b^{2}U_a}=U_aU_{b^{2}}U_a$
 so $U_a$ is invertible. Then
  (i)$\Rightarrow$.(ii). Evidently(ii) $\Rightarrow$ (iii). Now assume
  (iii). Then there exists a $c$ such that $1=cU_{a}$. Then
  $1=U_{a}U_{c}U_{a}$ so $U_{a}$ is invertible. Then there exists $ab$
  such that $bU_a=a$. Hence $U_bU_bU_a=U_a$ and since $U_a$ is
  invertible, $U_aU_b=1=U_bU_a$. Then $b^{2}U_a=1U_bU_a=1$ so $a$ is
  invertible with $b$ as inverse. Thus (iii) $\Rightarrow$ (i).

\item[(2)] If $a$ is invertible with $b$ as inverse then $bU_a=a$ and
  since $U_a^{-1}$ exists, $b=aU_a^{-1}$ is unique. Also
  $U_aU_bU_a=U_a$ gives $U_aU_b=1U_bU_a$ so
  $U_{a^{-1}}=U^{-1}_a$. Also $U_b$ -invertible implies that $b$ is
  invertible and its inverse is $bU_b^{-1}=bU_a=a$. This completes the
  proof of $(2)$.

\item[(3)] By $QJ24$ we have $U_aV_a=U_{a,a^{2}}=V_aU_a$. Hence
 $(a^{-1}\circ
  a)U_a=a^{-1}V_aU_a=a^{-1}U_aV_a=aV_a=2a^{2}=2U_a$. Since $U^{-1}_a$  
  exists this gives $a^{-1}\circ a=2$. By $QJ20$ we have $a^{-1}\circ
  a^{2}=a^{-1}V_{a^{2}}=a^{-1}(V^{2}_a-2U_a)=4a-2a=2a$. Also, by
  $QJ13$ and $24$,
  $U_aV_{a^{-1}}U_a=U_{a^{-1}U_{a,a^{2}}}=U_{a,a^{2}}=U_aV_a=V_aU_a$. Hence
    $V_{a^{-1}}=U_a^{-1}V_{a}=V_aU^{-1}_a$ (4). The first assertion is
      clear since $U_{aba}=U_aU_bU_a$. Also if $a$ and $b$ are
      invertible then
      $(aba)^{-1}=(aba)U^{-1}_{aba}=bU_aU^{-1}_a$ $U^{-1}_bU^{-1}_a =
      b^{-1}U^{-1}_a = b^{-1}U_{a^{-1}}=a^{-1}b^{-1}a^{-1}$. 
\end{itemize}

\noindent
{\textbf{Remark by M.B. Rege.}}
There are two other conditions for $a$ to be invertible with $b$ as
inverse which can  be added to those given in 

(1): (iv) $aba=a$ and $ba^{2}b=1$, (v) $aba=a$ and $b$ is the only
element of $\mathscr{J}$ satisfying this condition. These are
well-known for associative\pageoriginale algebras. The associative
case of (iv) applied to $U_a$ and $U_b$ given (iv) in the Jordan
case. (v) is an immediate consequence.

If $n$ is a positive integer then we define
$a^{-n}=(a^{-1})^{n}$. Then it is easy to extend $QJ 32, 33$ to all
integral powers. It is easy to see also that for arbitrary integral
$m,n, U_{a^{m},a^{n}},V_{a^{m},a^{n}}$ are contained in the
(commutative) subalgebra of End $\mathscr{J}$ generated by $U_a,V_a$
and $U^{-1}_a$.

A quadratic Jordan algebra $\mathscr{J}$ is called a {\em division
  algebra} if $1\neq 0$ in $\mathscr{J}$ and every non-zero element of
$\mathscr{J}$ is invertible. If is an associative division algebra
then $\mathfrak{a}^{(q)}$ is a quadratic Jordan division algebra. Also
if $(\mathfrak{a},J)$ is an associative division algebra with
involution then $\mathscr{H}(\mathfrak{a},J)$ is a quadratic Jordan
division algebra since if $0\neq h \in \mathscr{H}$ then
$(h^{-1})^{J}=(h^{J})^{-1}=h^{-1}\in \mathscr{H}$. If $Q$ is a
quadratic form with basic point $1$ on a vector space $\mathscr{J}$
then we have seen that the quadratic Jordan algebra $\mathscr{J}=$Jord
$(Q,1)$ is special and can be identified with a subalgebra of
$C(\mathscr{J},Q,1)^{(q)}$ where $C(\mathscr{J},Q,1)$ is the Clifford
algebra of $Q$ with base point $1$. In $C$ we have the equation
$x^{2}-T(x)xQ(x)=0, x\in \mathscr{J}$. Hence
$x\ob{x}=Q(x)1=\ob{x}x$ for $\ob{x}=T(x)1-x$. This shows that $x$ is
invertible in $C$ if and only if $Q(x)\neq 0$ in which case
$x^{-1}=Q(x)^{-1}\ob{x}$. It follows that $x$ is invertible in
$\mathscr{J}$ if and only if $Q(x)\neq 0$. Hence
$\mathscr{J}=$Jord$(Q,1)$ is a division algebra if and onky if $Q$ is
unisotropic in the sense that $Q(x)\neq 0$ if $x\neq 0$ in
$\mathscr{J}$. 

The\pageoriginale existence of exceptional Jordan
division algebras was first established by Albert. Examples of these
can be obtained by using Tits construction of algebras defined by
cubic forms as in \S $9$. In fact, it can be seen that if the algebra
$\mathfrak{a}$ used in Tits construction is a division algebra and
$\mu$ is not a generic norm in $\mathfrak{a}$ then the Tits' algebra
defined by $\mathfrak{a}$ and $\mu$ is a division algebra.

An element $a\in \mathscr{J}$ is called a {\em zero divisor} if
$U_a$ is not injective, equivalently, there exists $ab\neq 0$ in
$\mathscr{J}$ such that $bU_a=0$. Clearly, if $a$ is invertible then
it is not a zero divisor. An element $z$ is called an {\em absolute
  zero divisor} if $U_z=0$ and $\mathscr{J}$ is called {\em strongly
  non-degenerate} if $\mathscr{J}$ contains no absolute zero divisors
$\neq 0$. This condition is stronger than the condition that
$\mathscr{J}$ is non-degenerate which was defined by: ker $U=0$, where
ker $U=\{z|U_z=0=u_{z,a},a\in \mathscr{J}\}$.

Let $\Phi$ be a field. An element $a\in (\mathscr{J}/\Phi, U,1)$
is called {\em algebraic} if the subalgebra $\Phi[a]$ generated by a
is finite dimensional. Clearly $\Phi[a]$ is the $\Phi$-subspace
spanned by the powers $a^{m},m=0,1,2,\ldots$, and we have the
homomorphism of $\Phi[\lambda]^{(q)}$, $\lambda$ an indeterminate,
onto $\Phi[a]$ such that $\lambda \to a $. Let $k_a$ be the kernal of
this homomorphism. If the characteristic is $\neq 2$ then the ideals
of $\Phi[\lambda]^{(q)}$ are the same as those of $\Phi[\lambda]^{+}$,
which is the Jordan algebra associated with $\Phi[\lambda]^{(q)}$ by
the category isomorphism. Since $ab=\frac{1}{2}(ab+ba)=a\cdot b$ in
$\Phi[\lambda]$ we have $\Phi[\lambda]^{+}=\Phi[\lambda]$ as
algebras. Hence the ideals of $\Phi[\lambda]^{(q)}$ are ideals of the
assoicative algebra $\Phi[\lambda]$ if char $\Phi\neq 2$. If
char\pageoriginale $\Phi=2$, Example $(3)$ of \S $5$ shows that there
exist ideals of $\Phi[\lambda]^{(q)}$ which are not ideals of
$\Phi[\lambda]$. Let $\mathfrak{K}$ be an ideal $\neq 0$ in
$\Phi[\lambda]^{(q)}$, $f(\lambda)\neq 0$ an element of
$\mathfrak{K}$. Then
$g(\lambda)f()^{2}=g(\lambda)U_{f(\lambda)}\in
\mathfrak{K}$. Hence $\mathfrak{K}$ contains the ideal
$(f(\lambda)^{2})$ of $\Phi[\lambda]$. The sum of all such ideals is
an ideal $(m(\lambda))$ of $\Phi[\lambda]$. We may assume $m(\lambda)$
monic. In particular, if $a$ is an algebraic element of $\mathscr{J}$
then $\mathfrak{K}_a$ contains a unique ideal $(m_a(\lambda))$ of
$\Phi[\lambda]$ maximal in $\mathfrak{K}_a$ where $m_a(\lambda)$ is
monic. We shall call $m_a(\lambda)$ the {\em minimim polynomial} of the
algebraic element $a$. If $m_a(0)=0$ so $m_a(\lambda)=\lambda
h(\lambda)$ then $h(\lambda)\notin(m_a(\lambda))$ so these exists $a
g(\lambda) \in \Phi [\lambda]$ such that
$h(\lambda)g(\lambda)\in \mathfrak{K}_a$. Hence $h(a)g(a)\neq 0$
and $h(a)g(a)U_a=0$. Thus $m_a(0)=0$ implies that $a$ is a zero
divisor. On the other hand, suppose there exists a polynomial
$f(\lambda)$ such that $f(a)=0$ and $f(0)\neq 0$. Then we have a
relation $1=g(a)$ where $g(0)=0$. Then $1=g(a)^{2}$ and
$g(\lambda)^{2}=\lambda^{2}h(\lambda)$. Then $1=h(a)U_a$ and $a$ is
invertible by the Theorem on Inverses. Hence an algebraic element $a$
is either a zero divisior or is invertible according as $m_a(0)=0$ or
$m_a(0)\neq 0$. It is easily seen also that if $a$ is algebraic then
$\Phi[a]$ is a quadratic Jordan division algebra if and only if
$\mathfrak{K}_a=(m_a(\lambda))$ where $m_a(\lambda)$ is
irreducible. We have it to the reader to prove this.

If $\mathscr{J}$ is strongly non-degenerate then
$\mathfrak{K}_a=(m_a(\lambda))$ for every algebraic element $a$. For,
if $g(\lambda)\in \mathfrak{K}_a$ and $f(\lambda)\in
\Phi[\lambda]$ then $U_{(fg)(a)}U_{g(a)}=0(QJ 34)$. Hence $(f,g)(a)=0$
and $f(\lambda)g(\lambda)\in_a$. Thus $\mathfrak{K}_a$ is an
ideal of $\Phi[\lambda]$ and $\mathfrak{K}_a=(m_a(\lambda))$ by
definition of $(m_a(\lambda))$.
\end{proof}

\section{Isotopes}\label{c1:sec11}

This\pageoriginale is an important notion in the Jordan theory which,
like inverses, has an associative back ground.

Let $\mathfrak{a}$ be an associative algebra, $c$ an invertible
element $\mathfrak{a}$. Then we can define a new algebra
$\mathfrak{a}(c)$ which is the same $\Phi$-module as $\mathfrak{a}$
and which has the product $x_c y=xcy$. We have $(x_cy)_c=xcycz$ and
$x_c(y_cz)=xcy cz$ so $\mathfrak{a}^{(c)}$ is associative. Also
$x_cc^{-1}=xcc^{-1}=x$ and $c^{-1}_cx=c^{-1}c x=x$ so $c^{-1}$ is unit
for $\mathfrak{a}(c)$. The mapping $c_R:x\to xc$ is an isomorphism of
$\mathfrak{a}^{(c)}$ onto $\mathfrak{a}$ since
$(x_cy)c_R=xcyc=(xc_R)(yc_R)$. An element $u$ is invertible in
$\mathfrak{a}$ if and only if, it is invertiable in
$\mathfrak{a}^{(c)}$ since $uv=1=vu$ is equivalent to
$u_c(c^{-1}vc^{-1})=c^{-1}=(c^{-1}vc^{-1})_cu$. If $d$ is invertible
in $\mathfrak{a}$ (or $\mathfrak{a}^{(c)})$ then we can form the
algebra $(\mathfrak{a}^{(c)})^{(d)}$. The product here is
$x_{c,d}y=x_cd_cy=xcdcy$. Hence
$(\mathfrak{a}^{(c)})^{(d)}=\mathfrak{a}^{(cdc)}$. In particular, if
we taked $d=c^{-2}$ then we see that
$(\mathfrak{a}^{(c)})^{c^{-2}}=\mathfrak{a}$. Finally, we consider the
quadratic Jordan algebras $\mathfrak{a}^{(q)}$ and
$\mathfrak{a}^{(c)(q)}$. The $U$-operator in the first is $U_a:x\to
ax a$ and in the second it is $U_a^{(c)}:x\to a_cx_ca=ac x
ca$. Hence we have $U_a^{(c)}=U_cU_a$.

The considerations lead to the definition and basic properties of
isotopy for quadratic Jordan algebras. Let $(\mathscr{J},U,1)$ be a
quadratic Jordan algebra $c$ an invertible element of
$\mathscr{J}$. We define a mapping $U^{(c)}$ of $\mathscr{J}$ into End
$\mathscr{J}$ by
\begin{equation*}
  U_a^{(c)}=U_cU_a\tag{52}\label{c1:eq52}
\end{equation*}
and\pageoriginale we put $1^{(c)}=c^{-1}$. Evedently $U^{(c)}$ is a
  quadratic mapping we have $U_{1(c)}^{(c)}=U_cU_{c^{-1}}=1$. Hence
  the axioms $QJ1$ and $QJ2$ for quadratic Jordan algebras hold. Also
\begin{align*}
  U^{(c)}_{aU^{(c)}_b}&=U_cU_{aU_cU_b}=U_cU_bU_cU_aU_cU_b\\
  &=U_b^{(b)}U_a^{(c)}U_b^{(c)}
\end{align*}

Hence $QJ3$ holds. Next we define $V_{a,b}^{(c)}$ by
$xV_{a,b}^{(c)}=aU_{x,b}^{(c)}$ where
$U_{a,b}^{(c)}=U_{a+b}^{(c)}-U_a^{(c)}-U_b^{(c)}=U_cU_{a+b}-U_cU_a=U_cU_b=U_c
U_{a,b}$. Then $xV_{a,b}^{(c)}=aU_c U_{x,b}=xV_{aU_c,b}$. Thus
$V_{a,b}^{(c)}=V_{aU_{c}b}$. Now we have
\begin{align*}
  xV_{b,a}^{(c)}U_b^{(c)}&=bU_{x,a}^{(c)}U_b^{(c)}=bU_cU_{x,a}U_cU_b\\
  &=b U_{xU_c,aU_c}U_b(\text{bilinearization of} QJ 3)\\
  &=xU_c U_{b,aU_c}U_b\\
  &=xU_cU_b V_{aU_c,b}\quad (QJ 4)\\
  &=xU_b^{(c)}V_{a,b}^{(c)}
\end{align*}

Hence $QJ4$ holds for $(\mathscr{J}, U^{(c)},1^{(c)})$. It is clear
also that these properties carry over to $\mathscr{J}_{\ub{\rho}}$ for
$\ub{\rho}$ any commutative associative algebra over $\Phi$. Hence
$(\mathscr{J}, U^{(c)}, 1^{(c)})$ is a quadratic Jordan algebra.

\begin{defn}\label{c1:defn6}
  If\pageoriginale $c$ is an invertible element of $(\mathscr{J},U,1)$ then the
  quadratic Jordan algebra
  $\mathscr{J}^{(c)}=(\mathscr{J},U^{(c)},1^{(c)})$ where
  $U_a^{(c)}=U_cU_a,1^{(c)}=c^{-1}$ is called the $c$-{\em isotope} of
  $(\mathscr{J},U,1)$. 
\end{defn}

It is clear from the formula $U_a^{(c)}=U_cU_a$ and the fact that $a$
is invertible in $\mathscr{J}$ if and only if $U_a$ is invertible in
End $\mathscr{J}$ that $a$ is invertible in $\mathscr{J}$ if and only
if it is invertible in the isotope $\mathscr{J}^{(c)}$. If $d$ is
another invertible element then we can form the $d$-isotope
$(\mathscr{J}^{(c)})^{(d)}$ of $\mathscr{J}^{(c)}$. Its $U$ operator
is $U^{(c)(d)}$ where 
\begin{align*}
U_a^{(c)(d)}&=U_d^{(c)}U_a^{(c)}=U_cU_dU_cU_a=U_{cdc}U_a\\
&=U_a^{(cdc)}
\end{align*}

Also we recall that $cdc$ is invertible and $1^{(c)(d)}=(cdc)^{-1}$
since
$d(U_d^{(c)})^{-1}=d(U_cU_d)^{-1}=dU_{d}^{-1}U_{c^{-1}}=c^{-1}d^{-1}c^{-1}
= (cdc)^{-1}$. Hence
$(\mathscr{J}^{(c)})^{(d)}=\mathscr{J}^{(cdc)}$. In this sense we have
transitivity of the construction of isotopes. Also since
$\mathscr{J}^{(1)}=\mathscr{J},\mathscr{J}$ is its own
isotope. Finally, we have
$(\mathscr{J}^{(c)})^{(c^{-2})}=\mathscr{J}^{(cc^{-2}c)} =
\mathscr{J}^{(1)} = \mathscr{J}$
so $\mathscr{J}$ is  the $c^{-2}$ -isotope of the $c$ -isotope
$(\mathscr{J},c)$. In this sence the construction is symmetric.

Unlike the situation in the associative case isotopic quadratic Jordan
algebras need not be isomorphic. An important instance of isotopy
whihc gives examples of isotopic, non-isomorphic algebras is obtained
as follows. Let $(\mathfrak{a},J)$ be an associative algebra
with\pageoriginale involution, $h$ an invertible element of
$\mathscr{H}(\mathfrak{a},J)$. Then the mapping $K:x\to h^{-1}x^{J}h$
is also an involution in $\mathfrak{a}$. We claim that the quadratic
Jordan algebra $\mathscr{H}(\mathfrak{a}, K)$ is isomorphic to the
$h$-isotope of $\mathscr{H}(\mathfrak{a},J)$. Let $x\in
\mathscr{H}(\mathfrak{a},J)$ then $xh_R=xh\in
\mathscr{H}(\mathfrak{a},K)$ since
$(xh)^{K}=h^{-1}(xh)^{J}h=h^{-1}(hx)h=xh$. It follows that $h_R$ is a
$\Phi$ -isomorphism of $\mathscr{H}(\mathfrak{a},J)$ onto
$\mathscr{H}(\mathfrak{a},K)$. Moreover, if $x,y\in
\mathscr{H}(\mathfrak{a},J)$ then $x
U_y^{(h)}h_R=(xU_hU_y)h_R=yh x hyh=(xh_R)U_{yh_R}$. Hence $h_R$ is
an isomorphism of the quadratic Jordan algebra
$\mathscr{H}(\mathfrak{a},J)^{(h)}$ onto
$\mathscr{H}(\mathfrak{a},K)$.

It is easy to give examples such that $\mathscr{H}(\mathfrak{a},J)$
and $\mathscr{H}(\mathfrak{a},K)$ are not isomorphic. This gives
examples of $\mathscr{H}(\mathfrak{a},J)$ which is not isomorphic to
the isotope to the isotope $\mathscr{H}(\mathfrak{a},J)^{(h)}$ For
  example, let $\mathfrak{a}=\mathbb{R}_2$ the algebra of $2\times 2$
  matrices over the reals $\mathbb{R}$, $J$ the standard involution in
  $\mathbb{R}_2$. If $a\in \mathscr{H}(\mathbb{R}_2)$ say
  $a=(a_{ij})$ then tr $a^{2}=\sum a^{2}_{ij}\neq 0$. Hence $a$ is not
  nilpotent. Let $h=$diag$\{1,-1\}$ and consider the involution
  $K:x\to h^{-1}xh$ in $\mathbb{R}_2$. Then $\begin{pmatrix}
    1 & -1\\
    1 &-1\\
  \end{pmatrix}=\begin{pmatrix}
  1 & 1\\
  1 & 1
  \end{pmatrix}\begin{pmatrix}
  1 & 0\\
  0 & -1
  \end{pmatrix}\in \mathscr{H}(\mathbb{R}_2,K)$. Also
  $\begin{pmatrix}
    1 & -1\\
    1 &-1
  \end{pmatrix}^{2}=0$. Hence $\mathscr{H}(\mathbb{R}_2,K)$ contains
  non-zero nilpotent elements. Thus $\mathscr{H}(\mathbb{R}_2)$ is not
  isomorphic to $\mathscr{H}(\mathbb{R}_2,K)$ and the latter is
  isomorphic to the isotope $\mathscr{H}(\mathbb{R}_2)^{(h)}$.

It is convenient (as in the foregoing discussion) to extend the notion
of isotopy to apply to different algebras and to define isotopic
mappings. Accordingly we give
\begin{defn}\pageoriginale
Let $(\mathscr{J},U,1)$, $(\mathscr{J}',U',1)$ be quadratic Jordan
algebras. A mapping $\eta$ of $\mathscr{J}$ into $\mathscr{J}'$ is
called an {\em isotopy} if $\eta$ is an isomorphism of
$(\mathscr{J},U,1)$ onto an isotope $\mathscr{J}'^{(c')}$ of
$\mathscr{J}'$. If such a mapping exists then $(\mathscr{J},U,1)$ and
$(\mathscr{J}',U',1')$ are called {\em isotopic} (or {\em isotopes})

Using $\eta=1$ we see that the isotope $\mathscr{J}^{(c)}$ and
$\mathscr{J}$ are isotopic in the sense of the present
definition. Also it is clear that isomorphic algebras are
isotopic. The definition gives $1^{\eta}=(c')^{-1}$ and
$(xU_a)^{\eta}=x^{\eta}U'_{a^{\eta}}(c')=x^{\eta}U'_{c'}U'_{a^{\eta}}$
or $U_a\eta=\eta U'_{c'}U'_{a^{\eta}}$. Hence we have
\begin{equation*}
U'_{a^{\eta}}=\eta^{\ast}U_a\eta\tag{53}\label{c1:eq53}
\end{equation*}
where $\eta^{\ast}=U'^{-1}_{c'}\eta^{-1}U'_{(c')^{-1}}\eta^{-1}$ is a
module isomorphism of $\mathscr{J}'$ onto $\mathscr{J}$. Conversely,
let $\eta$ be a module isomorphism of $\mathscr{J}$ onto
$\mathscr{J}'$ such that there exists a module isomorphism
$\eta^{\ast}$ of $\mathscr{J}'$ onto $\mathscr{J}$ satisfying
    \eqref{c1:eq53}. Then \eqref{c1:eq53} implies that $a$ is
    invertible in $\mathscr{J}$ if 
    and only if $a^{\eta}$ is invertible on $\mathscr{J}'$. Also
    $U'_{1}\eta=\eta^{\ast}\eta$ so $\eta^{\ast}=U'_{1
    ^\eta}\eta^{-1}=U'^{-1}_{c'}\eta^{-1}$ where
  $c'=(1^{\eta})^{-1}$. Then $U_a^{\eta}=\eta U'_{c'}U'_{a^{\eta}}=\eta
  U'_{a^{(c')}}$ and $\eta$ is an isomorphism of $(\mathscr{J},U,1)$
  onto $(\mathscr{J}',U'^{(c')},1'^{(c')},1'^{(c')})$ so $\eta$ is an
  isotopy of $\mathscr{J}$ onto $\mathscr{J}'$. Hence we have shown
  that a module isomorphism $\eta$ of $\mathscr{J}$ onto
  $\mathscr{J}'$ is an isotopy if and only if there exists a module
  isomorphism $\eta^{\ast}$ of $\mathscr{J}'$ onto $\mathscr{J}$
  satisfying \eqref{c1:eq53}. It is cleat r that the isotopy $\eta$ is an
  isomorphism of $\mathscr{J}$ onto $\mathscr{J}'$ if and only if
  $\eta^{\ast}=\eta^{-1}$ and $1^{\eta}=1'$. The latter condition
  implies the former since we have\pageoriginale
  $\eta^{\ast}=U'_1\eta^{-1}$. Hence an isotopy $\eta$ is an
  isomorphism if and only if $1^{\eta}=1'$.

  If $\eta$ is an isotopy of $\mathscr{J}$ onto $\mathscr{J}'$ and
  \eqref{c1:eq53} 
  holds then $U_{a'}\eta^{-1}=(\eta^{\ast})^{-1}U'_{a'}\eta-1$ which
  shows that $\eta^{-1}$ is an isotopy of $\mathscr{J}'$ onto
  $\mathscr{J}$. If $\mathscr{J}$ is an isotopy of $\mathscr{J}'$ onto
  $\mathscr{J}''$ and $U''_{a'\zeta}=\zeta U_{a'}\zeta,a'\in
  \mathscr{J}'$, then
  $U''_{a^{\eta\zeta}}=\zeta^{\ast}\eta^{\ast}U_a\eta\zeta$. Hence
  $\eta\zeta$ is an isotopy of $\mathscr{J}$ onto $\mathscr{J}''$ and
  $(\eta\zeta)^{\ast}=\zeta^{\ast}\eta^{\ast}$. It is clear from this
  that isotopy is an equivalence relation. Since $\eta^{-1}$ is an
  isotopy it is an isomorphism of $\mathscr{J}'$ onto an isotope of
  $\mathscr{J}$. Hence $\eta$ is also an isomorphism of an isotope of
  $\mathscr{J}$ onto $\mathscr{J}'$ (as well as of $\mathscr{J}$ on an
  isotope of $\mathscr{J}'$).

The set of isotopies of $\mathscr{J}$ onto $\mathscr{J}$ is a group of
transformations of $\mathscr{J}$. Following Koecher, we call this the
{\em structure group} of $\mathscr{J}$ and we denote it as Str
$\mathscr{J}$. Clearly Str $\mathscr{J}$ contians the group of
automorphisms Aut $\mathscr{J}$ as a subgroup. Moreover, Aut
$\mathscr{J}$ is the subgroup of Str$\mathscr{J}$ of such that
$1^{\eta}=1$. If $c$ is invertible then $U_c$ is a module isomorphism
of $\mathscr{J}$ onto $\mathscr{J}$ and $U_{aU_c}=U_cU_aU_c$. Hence
(53) holds for $\eta=U_c,\eta^{\ast}=U_c$ so $U_c\in$ Str
$\mathscr{J}$. It is clear from the foregoing discussion that $U_c$ is
an isomorphism of $(\mathscr{J},U,1)$ onto the $c^{-2}=(1U_c)^{-1}$
isotope $(\mathscr{J}, U^{(c^{-2})},c^{2})$. In particular, if
$c^{2}=1$ then $U_c$ is an automorphism of $(\mathscr{J},U,1)$. The
subgroup of Str $\mathscr{J}$ generated by the $U_c,c$ invertible is
called the {\em inner structure group}. We denote this as
Instr. $\mathscr{J}$. If $\eta \in$ Str $\mathscr{J}$,
$U_{a^{\eta}}=U_{1\eta}\eta^{-1}U_a\eta$ so
$\eta^{-1}U_a\eta=(U_{1\eta})^{-1}U_{a^{\eta}}$ which implies that
Instr $\mathscr{J}$ is a normal subgroup of Str $\mathscr{J}$. It
follows also that Aut $\mathscr{J}\cap$ Instr $\mathscr{J}$ is
normal\pageoriginale in Str $\mathscr{J}$. We call this the group of
{\em inner automorphisms}.

We have seen that if $\eta$ is an isotopy of $\mathscr{J}$ on
$\mathscr{J}'$ then $\eta^{\ast}=(U_{1^\eta})^{-1}$ $\eta^{-1}$. since
$(U_{1^\eta})^{-1}\in$ Str $\mathscr{J}'$ and $\eta^{-1}$ is an
isotopy of $\mathscr{J}'$ onto $\mathscr{J}$. We see that
$\eta^{\ast}$ is an isotopy of $\mathscr{J}'$ onto $\mathscr{J}$. In
particular, if $\eta\in$Str $\mathscr{J}$ then
$\eta^{\ast}\in$ Str $\mathscr{J}$. We have
$(\eta\zeta)^{\ast}=\zeta^{\ast}\eta^{\ast}$ and $U^{\ast}_c=U_c$ for
invertible $c$. Hence
$\eta^{\ast}=(\eta^{\ast})^{-1}U^{-1}_{1^{\eta}}=\eta
U_{1^{\eta}}U^{-1}_{1^{n}}=\eta$ Thus $\eta \to \eta^{\ast}$ is an
anti-automorphism of Str $\mathscr{J}$ such that
$\eta^{\ast\ast}=\eta$ and this stabilzes Instr $\mathscr{J}$.If
$\eta$ is an automorphism then $U_{a^{\eta}}=\eta^{-1}U_a\eta$ so
$\eta^{\ast}=\eta^{-1}$.

Let $c$ be invertible and consider the isotope $(\|mathscr{J},c)$. Let
$\eta \in$Str $\mathscr{J}$, so
$U_{a^{\eta}}=\eta^{\ast}U_{a}^{\eta}$, $a\in \mathscr{J}$. Then
$U_{a^{\eta}}^{(c)}=U_cU_{a^{\eta}}=U_c\eta^{\ast}U_a\eta
=(U_c\eta^{\ast}U_c^{-1})U_cU_a\eta=(U_c\eta^{\ast}U_c^{-1})U_a^{(c)}\eta$. Hence
$\eta \in$Str $\mathscr{J}^{(c)}$. By symmetry, Str
$\mathscr{J}^{(c)}=$Str$\mathscr{J}$. Similarly, one sees that Instr
$\mathscr{J}^{(c)}=$Instr $\mathscr{J}$.

If $\mathscr{Z}$ is an inner(outer) ideal in $\mathscr{J}$ then
$\mathscr{Z}$ is an inner (outer) ideal of the isotope
$\mathscr{J}^{(c)}$. For, if $\mathscr{Z}$ is inner and $b\in
\mathscr{Z}$ then $\mathscr{J} U^{(c)}_b=\mathscr{J}U_cU_b=\mathscr{J}
U_b\subseteq \mathscr{Z}$ and if $\mathscr{Z}$ is outer and $a\in
\mathscr{J}$ then $\mathscr{Z}U_{a}^{(c)}=\mathscr{Z}U_cU_a\subseteq
\mathscr{Z}$. Since an isotopy of $\mathscr{J}$ into $\mathscr{J}'$
maps an isotope of $\mathscr{J}$ onto $\mathscr{J}'$ it is clear that
if $\mathscr{Z}$ is an inner (outer) ideal in $\mathscr{J}$ then
$\mathscr{Z}^{\eta}$ is an inner (outer) ideal of $\mathscr{J}'$.
\end{defn}
