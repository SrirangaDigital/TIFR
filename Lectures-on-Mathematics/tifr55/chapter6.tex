

\chapter{Applications of the Large Sieve}\label{chap6}%chap 6

THE\pageoriginale SIGNIFICANCE of the large sieve is due to its
usefulness towards 
the solution of important problems of number theory. For this purpose
the large sieve is employed in two ways; namely, in proving results
which have number-theoretic consequences of depth and on the other
side, for direct applications to number theory. We shall defer the
discussion of the (latter) arithmetical version of the large sieve to
the next chapter and confine ourselves here to a brief survey of the
(aforementioned) indirect applications. 


\medskip
\noindent
{\bf 1. Moments of the Dirichlet's {\boldmath$L$}-series}
\smallskip 

In this section we mention the applications to the moments of the
Dirichlet's $L$-series. For the historical introduction to this topic we
refer to Montgomery \cite{key5} (Chapter \ref{chap10}). 

Gallagher \cite{key1} has shown that the large sieve (in its version of
Chapter \ref{chap3}) can be used to prove 
\begin{equation*}
\sum_{q \leq Q} ~ \sum_{\chi\mod  q}^* | L(\frac{1}{2}+it,\chi) |^4
\ll Q^2 T^2 \log^4 (QT). |t|\leq T, T \geq 2. \tag{6.1}\label{eq6.1} 
\end{equation*}

Montgomery (\cite{key5} (Lemma 10.5)) reduced the problem of
estimating the mean fourth power of $L(s,\chi^*)$, using the work of
Lavrik on approximate functional equations for the Dirichlet's
$L$-series, to an application of the result \eqref{eq3.36} and
\eqref{eq3.37}, and obtained (\cite{key5} (Theorem 10.1))
{\fontsize{10pt}{12pt}\selectfont
\begin{equation*}
\sum_{\chi \mod q}^* \int\limits_{-T}^{T} | L(\sigma + it, \chi
|^4 dt \ll\varphi (q) T \log^4 (qT) \text{ for } | \sigma -
\frac{1}{2} | \ll \log^{-1} (qT), T\geq 2, \tag{6.2}\label{eq6.2} 
\end{equation*}}\relax
which may be considered as the average-version of the generalized
Lindelof hypothesis: 
\begin{equation*}
L(\frac{1}{2} + it,\chi ) \underset{\varepsilon}{\ll}
(q|t|)^\varepsilon , \quad |t|\geq 1. \tag{6.3}\label{eq6.3} 
\end{equation*}

Then\pageoriginale he proceeds to derive easily (\cite{key5}
(Corollary 10.2)) 
\begin{equation*}
\sum_{\chi \mod q}^* \int\limits_{-T}^{T} | L(\frac{1}{2} + it,\chi )
L' (\frac{1}{2} + it, \chi ) |^2 dt \ll \varphi (q) T \log^6 (qT), T
\geq 2. \tag{6.4}\label{eq6.4} 
\end{equation*}


\medskip
\noindent
{\bf 2. Density theorems.}
\smallskip

The next important applications of the large sieve (employed in its
hybrid version of Chapter \ref{chap5} along with many other ingenious ideas)
concern to the `statistical density theorems' for the zeros of the
Dirichlet's $L$-series and, in particular, of the Riemann zeta
function. For the history of this subject, see Montgomery \cite{key5}
(Chapter \ref{chap12}). 

We recall the following standard notation required for the description
of the results of this section. As usual, 
\begin{equation*}
N(\sigma ,T,\chi ) \quad (\sigma \leq 1) \tag{6.5}\label{eq6.5}
\end{equation*}
denotes the number of zeros
\begin{equation*}
\rho = \beta + i \gamma \tag{6.6}\label{eq6.6}
\end{equation*}
of the function
\begin{equation*}
L(S,\chi ) \tag{6.7}\label{eq6.7}
\end{equation*}
in the rectangle
\begin{equation*}
\sigma \leq \beta \leq 1, | \gamma | \leq T. \tag{6.8}\label{eq6.8}
\end{equation*}

Particularly for the Riemann zeta function, i.e., $q = 1$, we use
\begin{equation*}
N(\sigma , T) \tag{6.9}\label{eq6.9}
\end{equation*}
instead of \eqref{eq6.5}.

Regarding the average over $\chi$ of \eqref{eq6.5}, for a fixed $q$, the
best known results at present are the following. We have 
{\fontsize{10pt}{12pt}\selectfont
\begin{equation*}
\sum_{\chi \mod q} N(\sigma ,T,\chi ) \ll 
\begin{cases}
(qT)^{\frac{3}{2-\sigma}(1-\sigma} \log^9 (qT), \frac{1}{2} \leq
  \sigma \leq \frac{3}{4}\\
\qquad (\text{ Montgomery } [5] (\text{ Theorem
  }12.1)),\\ 
(qT)^{\frac{3}{3\sigma -1}(1-\sigma ) + \varepsilon},\frac{3}{4}\leq
  \sigma \leq \frac{4}{5}(\text{ Huxley } [12]),\quad (\varepsilon >
  0).\\ 
(qT)^{(2+\varepsilon )(1-\sigma)},\frac{4}{5} \leq \sigma \leq 1
  (\text{ Jutila } [12]) 
\end{cases} \tag{6.10}\label{eq6.10}
\end{equation*}}\relax\pageoriginale
(Here and in what follows, the $\ll$ - constant is understood to
depend on $\epsilon$ whenever the bound contains $\epsilon$.) It
is easily deduced from \eqref{eq6.10} that 
\begin{equation*}
\sum_{\chi \mod q} N(\sigma , T,\chi ) \ll
(qT)^{(\frac{12}{5}+\varepsilon)(1-\sigma)},\quad \frac{1}{2}\leq \sigma
\leq 1,\quad (\varepsilon > 0), \tag{6.11}\label{eq6.11} 
\end{equation*}
holds uniformly in $\sigma$, $q\geq 1$ and $T \geq 2$. And, for the
average over $q$ we have 
{\fontsize{9pt}{11pt}\selectfont
\begin{equation*}
\sum_{q \leq Q}\sum_{\chi \text{ mod }q}^* N(\sigma ,T,\chi ) \ll 
\begin{cases}
(Q^3T)^{\frac{3}{2-\sigma}(1-\sigma )}\log^9(qT),\frac{1}{2}\leq
  \sigma \leq \frac{3}{4}\\
\qquad (\text{ Montgomery }[5] (\text{Theorem $12.2$}))\\
(Q^2T)^{\frac{3}{3\sigma-1}(1-\sigma)+\varepsilon},\frac{3}{4}\leq \sigma \leq \frac{4}{5}(\text{ Huxley }[12]),(\varepsilon > 0).\\
(Q^2T)^{2+\varepsilon)(1-\sigma)},\frac{4}{5}\leq \sigma \leq 1 (\text{ Jutila }[12]),
\end{cases} \tag{6.12}\label{eq6.12}
\end{equation*}}\relax
from which one gets the estimate analogous to \eqref{eq6.11}
\begin{equation*}
\sum_{q \leq Q}\sum_{\chi \text{ mod }q}^* N(\sigma ,T,\chi) \ll
(Q^2T)^{(\frac{12}{5} + \varepsilon )(1-\sigma )}, \frac{1}{2} \leq
\sigma \leq 1, (\varepsilon > 0) \tag{6.13}\label{eq6.13} 
\end{equation*}
valid uniformly in $\sigma$, $Q\geq 1$ and $T \geq 2$.

The function $N(\sigma ,T)$ has been investigated more
extensively. Estimates of the form 
\begin{equation*}
N(\sigma,T )\ll T^{2(1-\sigma)+\epsilon}\quad (\epsilon >
0). \tag{6.14}\label{eq6.14} 
\end{equation*}
valid in $\sigma > \alpha$ (for some $\alpha$) uniformly, are called
`density hypothesis'. It is known that the Lindel\"of hypothesis: 
\begin{equation*}
\zeta (\frac{1}{2}+it) \ll (1+|t|)^{\epsilon} \quad (\epsilon >
0) \tag{6.15} \label{eq6.15}
\end{equation*}
implies \eqref{eq6.14}, with $\alpha = \dfrac{1}{2}$. For some applications
proved under the assumption of\pageoriginale 
\eqref{eq6.15} it suffices instead to have a
result of the type \eqref{eq6.14}. Now the density hypothesis is known for  
\begin{equation*}
\sigma \geq \frac{11}{14} = 0.78571 \ldots \tag{6.16}\label{eq6.16} 
\end{equation*}
and this result is due to Jutila \cite{key9}, \cite{key10}
(cf. \cite{key12}). In $\sigma \geq \frac{61}{74}$ still better
estimates are available; namely,  
\begin{equation*}
N(\sigma ,T) \leq T^{\lambda (\sigma)(1-\sigma )+\epsilon}\quad
(\epsilon > 0) \tag{6.17}\label{eq6.17} 
\end{equation*}
with
\begin{equation*}
\lambda(\sigma) =
\begin{cases}
\frac{48}{37(2\sigma -1},\frac{61}{74}\leq \sigma \leq \frac{37}{42}
(\text{ Huxley }[11]),\\ 
\frac{3}{2\sigma},\frac{37}{42}\leq \sigma \leq
\frac{37+\sqrt{73}}{48} (\text{ Huxley }[11]),\\ 
\frac{4(3\sigma -2)}{3(4\sigma - 3)(2\sigma
  -1)},\frac{37+\sqrt{73}}{48}\leq \sigma \leq 1\\
\qquad (\text{Montgomery }[5] (\text{ Corollary }12.4)). 
\end{cases}\tag{6.18}\label{eq6.18}
\end{equation*}

Close to the line $\sigma = 1$, we have (Hal\'asz and Tur\'an
\cite{key1}) even 
\begin{equation*}
N(\sigma ,T)\underset{\delta}{\ll} T^{(1-\sigma)^{3/2}} \log^3
(\frac{1}{1-\sigma}),\sigma \geq 1 - \delta \quad (\delta > 0)
\tag{6.19}\label{eq6.19} 
\end{equation*}
and (Montgomery \cite{key5} (Corollary 12.5))
\begin{equation*}
N(\sigma ,T)\ll T^{167(1-\sigma)^{3/2}} \log^{17}T,\sigma \geq
\frac{1}{2},T \geq 2. \tag{6.20}\label{eq6.20} 
\end{equation*}

On the other side, in the vicinity of $\sigma = \frac{1}{2}$ and for
$(\frac{1}{2}\leq)\sigma \leq \frac{3}{4}$ still the best known
density estimates are due to A. Selberg and Ingham (cf. Montgomery
\cite{key5} (Chapter \ref{chap12})). For $\sigma$ between $\dfrac{3}{4}$ and
Jutila's bound \eqref{eq6.16}, the best known estimate \eqref{eq6.17} with 
{\fontsize{9pt}{11pt}\selectfont
\begin{equation*}
\lambda(\sigma) = 3 (\underset{k \in \mathbb{N}}{\min} \max
\frac{1}{(3\sigma -1) + \frac{2}{k}(1-\sigma)} \; , \; \frac{1}{2k(4\sigma
  -3)+3(1-\sigma )}),\frac{3}{4} < \sigma \leq \frac{11}{14}
\tag{6.21}\label{eq6.21}  
\end{equation*}}\relax
is also due to Jutila \cite{key10}.

The connection between the order of $\zeta(s)$ and the density
estimates has already been indicated (cf. \eqref{eq6.14} and
\eqref{eq6.15}). Indeed, some of the aforementioned
bounds\pageoriginale for $N(\sigma,T)$ 
can be improved slightly by using better estimates for $\zeta(s)$. For
general results in this context we refer to Bombieri \cite{key3} (and for
$L(s,\chi)$ to Forti and Viola \cite{key1}). 

\medskip
\noindent
{\bf 3. Mean-value Theorems of the Bombieri type.}
\smallskip

In this section and the next we mention the applications, involving
number-theoretic functions, which have important consequences in\break
(proper) number theory. As regards the notation almost all are
standard and so we repeat only one of these, namely, the Hurwitz's
zeta function,defined through 
\begin{equation*}
\zeta (s,w) = \sum_{n=0}^{\infty} (n+w)^{-s}, s = \sigma + it \quad
(\sigma > 1),0 < w \leq 1, \tag{6.22}\label{eq6.22}  
\end{equation*}
and analytic continuation.

Now, one of the most important applications of the large sieve has
been to what we shall call as Bombieri's prime number theorem\break 
(Bombieri \cite{key1}): For any given number $U(>0)$ there exists a value $C
= C(U)$ such that 
\begin{equation*}
\sum_{q \leq x^{\frac{1}{2}}\log^{-C_x}}\underset{2\leq y \leq
  x}{\max} \quad \underset{(\ell ,q) = 1} {\max} \Big| \chi (y; q,\ell
) - \frac{li \;  y}{\varphi (q)} \Big| \ll_U \frac{x}{(\log x)^{U}},
\tag{6.23}\label{eq6.23}   
\end{equation*}
or equivalently
\begin{equation*}
\sum_{q \leq x^{\frac{1}{2}}\log^{-C_x}}\underset{2 \leq y \leq
  x}{\max}\quad \underset{(\ell,q)=1}{\max} | \psi(y;q,\ell) -
\frac{y}{\varphi} | \ll_{U} \frac{x}{(\log
  x)^U}. \tag{6.24}\label{eq6.24}   
\end{equation*}

A result of this kind can be derived, either via estimates of the type
\eqref{eq6.12} or directly, from the large sieve. Bombieri proved
\eqref{eq6.24} (via \eqref{eq6.12}-type result) with the value $C = 3U
+ 23$, and the best known result now is with 
\begin{equation*}
C = U  + \frac{7}{2}, \tag{6.25}\label{eq6.25} 
\end{equation*}
due to Vaughan \cite{key6}, who obtained this by a refinement of Gallagher's
\cite{key2} method (-a direct application of the large sieve-) for a proof
of \eqref{eq6.24}. 
Jutila\pageoriginale \cite{key1} has proved a corresponding result for short
intervals which states that 
\begin{equation*}
\sum_{q\leq x^\beta} ~ \underset{z\leq x^{\theta}}{\max}\quad
\underset{(\ell,q) =1}{\max} | \psi (x+z;q,\ell ) - \psi(x;q,\ell) -
\frac{z}{\varphi (q)} | \underset{U,\varepsilon}{\ll}
\frac{x^{\theta}}{(\log x)^{U}},0 < \theta <
1. \tag{6.26}\label{eq6.26}   
\end{equation*}
where 
\begin{equation*}
\beta = \beta (\theta ,\epsilon) = \frac{4c\theta + 2\theta -1
  -4c}{6+4c} - \epsilon \tag{6.27}\label{eq6.27}  
\end{equation*}
in which $c$ denotes a constant satisfying, for the functionl in
\eqref{eq6.22}, 
\begin{equation*}
\zeta(\frac{1}{2}+ it,w) \ll_{\delta}(1+|t|)^{c+\delta}\text{ for
  every }\delta > 0. \tag{6.28}\label{eq6.28}  
\end{equation*}

For the various (more or less equivalent) forms of \eqref{eq6.23} and
\eqref{eq6.24} we refer the reader to Elliott and Halberstam \cite{key2} and
Montgomery \cite{key5} (Chapter 15). As an example, we mention an
interesting remark of Montgomery and Vaughan \cite{key2} in connection with
the following result derived from \eqref{eq3.4} 
\begin{equation*}
\sum_{q \leq \frac{N^{\frac{1}{2}}}{200}} \log
(\frac{N^{\frac{1}{2}}}{q})\sum_{\chi \mod q}^* | \psi (N,\chi ) |^2 <
N^2 \log N, \quad N > N_0, \tag{6.29}\label{eq6.29}  
\end{equation*}
where (cf. \eqref{eq1.75})
\begin{equation*}
\psi (N,\chi ) : = \sum_{n\leq N} \Lambda (n) \chi
(n). \tag{6.30}\label{eq6.30}   
\end{equation*}

Now the term corresponding to $q = 1$ in \eqref{eq6.29} contributes already
$\dfrac{1}{2}N^2 \log N + o (N^2)$ to the sum. Further, if
$L(s,\chi_1)$ has a Siegel - zero, $q_1 = N^{\delta}$, then
$|\psi(N,\chi_1)| > (1 - \delta) N $ and consequently the contribution
from the term for $\chi_1$ in \eqref{eq6.29} is atleast $(\frac{1}{2} - 2
\delta ) N^2 \log N$ so that 
\begin{equation*}
\sum_{N^{\delta}< q \leq \frac{N \frac{1}{2}}{200}} \sum_{\chi \mod q}^*
| \psi (N, \chi ) |^2 \ll \delta N^2 \log N, \tag{6.31}\label{eq6.31}  
\end{equation*}
which seems rather unlikely to be true for sufficiently small
$\delta$. 

There are results, analogous to \eqref{eq6.23}, concerning the average order
of remainder terms with respect to other number-theoretic
functions. From these results\pageoriginale also one has been able to
obtain results 
which could only be proved earlier either by the use of the
complicated Linnik's dispersion method or only under the assumption of
the generalized Riemann hypothesis. 

Such analogues are now available for the functions $d_k(n)$
(A.I. Vinogradov \cite{key1}, Motohashi \cite{key1}, \cite{key7}),
$r(n)$ (Motohashi \cite{key2}, Siebert and Wolke \cite{key1}) and for
certain powers of these 
functions. Certain other special functions have also been investigated
(Siebert and Wolke \cite{key1}, see \eqref{eq6.33} below, and Wolke
\cite{key7}) and
interestingly we have now general results of the type \eqref{eq6.23}, based
on the large sieve, due to Wolke (cf. Wolke \cite{key5}, \cite{key6},
Siebert and 
Wolke \cite{key1}): Under certain conditions (stemming from the work of
Wirsing) for a multiplicative number-theoretic function $f(n)$, one
has 
{\fontsize{9pt}{11pt}\selectfont
\begin{equation*}
  \sum_{q\leq x^{\frac{1}{2}}\log^{-C_x}} \underset{y \leq x}{\max}
\underset{(\ell ,q)=1}{\max} | \sum_{\substack{n\leq y\\ n \equiv \ell
    \mod q}} f(n) - \frac{1}{\varphi (n)}\sum_{\substack{n\leq y
    \\ (n,q) =1}} f(n) | \underset{U}{\ll} \frac{x}{(\log x)^U}, U >
0, C = C(U). \tag{6.32}\label{eq6.32}  
\end{equation*}}\relax

As an example, we mention the following consequence of \eqref{eq6.32}
(Siebert and Wolke \cite{key1}): 
\begin{equation*}
\sum_{q\leq x^{\frac{1}{2}}\log^{-C_x}} \underset{y\leq x}{\max} \quad
\underset{\ell}{\max} \; \Big| \sum_{\substack{n\leq y\\n \equiv\ell \text{
      mod } q}} \; \mu (n) \Big| {\ell}_{{}U} \frac{x}{(\log
  x)^U}, U > 0, C=C(U). \tag{6.33}\label{eq6.33} 
\end{equation*}

Orr \cite{key1} (cf. \cite{key2}) has derived such a result for the
number of squarefree integers (i.e., $\mu^2(n)$) in an arithmetic
progression in an elementary way. For another elementary derivation of
such results from a different type of mean-value theorems we refer to
the next section. 

\medskip
\noindent
{\bf 4. Mean-value Theorems of the Barban-Danport-Halberstam type.}
\smallskip


There is another type of mean-value theorem corresponding to \eqref{eq6.24}
which deals with the mean-square instead and has a considerably much
wider range of\pageoriginale validity (for $q$): 
\begin{equation*}
\sum_{q\leq x \log} -(U + 1)_x \mathop{\sum{}'}\limits_{\ell = 1}^{q}
(\psi (x;q,\ell) 
- \frac{x}{\varphi (q)})^2 \underset{U}{\ll} \frac{x^2}{(\log x)^U}
\text{~ for~ } U > 0. \tag{6.34}\label{eq6.34} 
\end{equation*}

Such a result was found (in a slightly weaker form) first by Barban
(\cite{key6} (Theorem 1), cf. \cite{key10} (Theorem 3.2)) and was later
rediscovered by Davenport and Halberstam \cite{key2}. The improved form
\eqref{eq6.34} is due to Gallagher \cite{key1}. The proof is based on the
Siegel-Walfisz theorem and the identity (Davenport and Halberstam
\cite{key2}) \eqref{eq1.76}, 
\begin{equation*}
\mathop{\sum{}'}\limits_{\ell = 1}^{q} (\psi (x;q,\ell ) -
\frac{x}{\varphi (q)})^2 = 
\frac{1}{\varphi (q)} \sum_{\chi \neq \chi_0} | \psi (x,\chi ) |^2,
\tag{6.35}\label{eq6.35}  
\end{equation*}
and an application of the large sieve in its form of Chapter \ref{chap3}.

Montgomery (\cite{key4}, cf. \cite{key5} (Chapter 17)) discovered a
proof of \eqref{eq6.34} independent of the large sieve and also
succeeded in obtaining the\break asymptotic formulae: 
{\fontsize{10}{12}\selectfont
\begin{equation*} 
\sum_{q \leq Q} \mathop{\sum{}'}\limits_{\ell = 1}^{q}(\psi (x;q,\ell
) - \frac{x}{\varphi (q)})^2 =  
\begin{cases}
Qx \log Q + o (Qx+x^2 \log^{-U}x),\\
\qquad \text{ for } Q \leq x,(\text{ for any fixed } U \leq 0)\\
Qx \log x - \frac{\zeta (2)\zeta(3)}{\zeta (6)}x^2 \log \frac{Q}{x}-\\
Qx + Ax^2 + o(Qx \log^{-U}x),\\
\text{ for } Q > x.
\end{cases} \tag{6.36}\label{eq6.36} 
\end{equation*}}

The part $Q\leq x$ of \eqref{eq6.36} is an improved version of Montgomery's
first result (Croft \cite{key1}). 

Montgomery's method of proof of \eqref{eq6.36} is based on a deep theorem of
Lavrik \cite{key1}, \cite{key2}, about the average order of the
error-term in the generalized twin-prime problem, which may be stated as 
{\fontsize{10}{12}\selectfont
\begin{equation*}
\sum_{q < \frac{x}{2}} (\sum_{2q < n \leq x} \Lambda (n)\Lambda (n-2q)
-2 (x -2q) \prod_{p>2} (1-\frac{1}{(p-1)^2})\prod_{2<
  p|q}\frac{p-1}{p-2})^2 \leq_{{}U} \frac{x^3}{\log^{U}_x}
\tag{6.37}\label{eq6.37}  
\end{equation*}}
and which in turn depends on the method of I.M. Vinogradov for the
estimation of exponential sums. 

Regarding\pageoriginale the first part ($Q \leq x$) of \eqref{eq6.36},
Hooley \cite{key5} has shown that one can replace the error term by  
\begin{equation*}
AQx + O(Q^{5/4}x^{3/4}+x^2 \log^{-U}x) \tag{6.38}\label{eq6.38} 
\end{equation*}
as an application of the (simpler) large sieve method only.

Hooley \cite{key8} has also proved, on the basis of the large sieve method
(of Chapter \ref{chap3}),  the general result of the
Barban-Davenport-Halberstam type: Let $\gamma$ be a set of positive integers
and suppose that there holds, for all $U > 0$ and all integers $q$, $\ell$,
(for some function $g(q,b)$)  
\begin{equation*}
S(x;q,l): = \sum_{\substack{n \in \gamma \\{n \leq x}\\ {n \equiv \ell
      \mod q}}}  1  = g(q,(l,q)) x + O_U(x \log^{-U}x)\text{~ as~ } x \to
\infty .\tag{6.39}\label{eq6.39}  
\end{equation*}

Then
\begin{gather*}
\sum_{q \leq Q} \sum^{q}_{l=1}(S(x;q,l))-g(q , (l,q)x)^2  = O(Qx) + O(x^2
\log^{-U}x)\\
 \text{for~ } 1 \leq Q \leq x \text{~~(for any fixed
  U)}. \tag{6.40}\label{eq6.40}  
\end{gather*} 

Combining this with the method of his paper \cite{key5} he derived that
{\fontsize{10pt}{12pt}\selectfont
\begin{equation*}
\sum_{q \leq Q} \sum^{q}_{l=1} (\sum _{\substack {n \leq x\\ {n=l \mod
      q}}}) \mu (n))^2 = \frac{6}{\pi^2}Qx + O (x^2 \log^{-U }x)
\text{ for } 1\leq x, (\text{ for any fixed
}U). \tag{6.41}\label{eq6.41}  
\end{equation*}}\relax  
  
An interesting connection between the mean-value theorems of the
squared expression and those of the preceding section was noted by
Barban \cite{key6}. He observed that from a mean-value theorem of the type
\eqref{eq6.40} it is possible to derive in an elementary manner a
Bombieri-type result for a related function (depending on the
parameter $x$). Here we shall present a proof in the simplest case:
Let us put 
\begin{equation*}
\psi _2 (x;q,l): = \mathop{\sum _{n_1 \leq x}\sum_{n_2 \leq
    x}}_{n_{1}n_{2}\equiv \ell \mod q}\Lambda (n_1)\Lambda (n_2),
\tag{6.42}\label{eq6.42}  
\end{equation*}      
\begin{equation*}
E_2(x;q,l): = \psi_2 (x;q,l)- \frac{x}{\phi (q)}, \tag{6.43}\label{eq6.43} 
\end{equation*} 
\begin{equation*}
E_1 (x;q,l): = \phi (x;q,l)-\frac{x}{\phi (q)}, \tag{6.44}\label{eq6.44} 
\end{equation*}\pageoriginale
and assume that
\begin{equation*}
(\ell,q)=1. \tag{6.45}\label{eq6.45} 
\end{equation*}  

We have
\begin{gather*}
\psi_2 (x; q, l) = \mathop{\sum{}'}\limits_{h=1}^{q} \sum_{\substack {n_1 \leq
    \sqrt{x}\\{n_1 \equiv h \mod q}}} \Lambda (n_1)
\sum_{\substack {n_2  \leq \sqrt{x}\\{n_1 \equiv h^{-1} l \mod q}}}
\Lambda (n_2)\\ 
= \mathop{\sum{}'}\limits^{q}_{h=1} \psi (\sqrt{x}; q, h)\psi
(\sqrt{x};q,h^{-1} l)  \tag{6.46}\label{eq6.46}  
\end{gather*} 
(here $h^{-1}$ represent the residue class $\mod q$ for which
$hh^{-1}\equiv 1 \mod q$) and  
\begin{equation*}
\Big|\mathop{\sum{}'}\limits_{b=1}^{q}(\psi(\sqrt{x};q,b) - \frac
            {\sqrt{x}}{\phi(q)}  \Big|\leq \Big|\psi
            (\sqrt{x})-\sqrt{x}|+ \log ~
            q. \tag{6.47}\label{eq6.47}   
\end{equation*} 
Subtracting $\dfrac{x}{\phi(q)}$ from both the the sides of \eqref{eq6.46}
and using \eqref{eq6.47} we obtain, by Cauchy's inequality, 
\begin{equation*}
\begin{cases}
E_2 (x;q,l) & \leq \mathop{\sum{}'}\limits^{q}_{h=1} E_1 (\sqrt{x},q,h) E_1
(\sqrt{x},q,h^{-1}l)\\
& \qquad +2 \frac{\sqrt{x}}{\phi(q)}\left \{ |\psi
(\sqrt{x})-\sqrt{x}|+\log q \right\} \leq \\  
& \leq \mathop{\sum{}'}\limits^{q}_{b=1} E^2_1 (\sqrt{x};q,b) + 2
\frac{\sqrt{x}}{\phi (q)}  
\left\{ |\psi (\sqrt{x})-\sqrt{x}| + \log ~ q \right\}  
\end{cases}\tag{6.48}\label{eq6.48} 
\end{equation*}  
uniformly in $\ell$ subject to \eqref{eq6.45}. Now summation over $q$
and the use of \eqref{eq6.34} along with an application of the prime number
theorem in the form 
\begin{equation*}
(\psi (\sqrt{x})-\sqrt{x})<< \frac{\sqrt{x}}{(\log x)^{U+1}},
  \tag{6.49}\label{eq6.49}  
\end{equation*} 
give
\begin{equation*}
\sum_{q \leq \sqrt{x}\log^{-(U+1)}x} \underset{(l,q)=1} \max \Big|\sum
_{\substack {n_1 \leq \sqrt{x}\\ {n_1 n_2 \equiv }}}\sum_{\substack {n_2
    \leq \sqrt{x}\\ {l \mod q}}} \Lambda (n_1)\Lambda (n_2) -
\frac{x}{\phi (q)}\Big| \ll_U \frac{x}{(\log x
  )^U}. \tag{6.50}\label{eq6.50}  
\end{equation*}  

Thus we have shown that this Bombieri-type result is an elementary
consequences of \eqref{eq6.34}.   

\medskip
\noindent
{\bf 5. Some number-theoretic applications.} 
\smallskip

In this last section we merely record a few applications of the
preceding result to pure number theory. 

First\pageoriginale of all we have the important consequences about
the difference between consecutive primes that  
\begin{equation*}
p_{n+1} - p_n < p_n^{\delta + \xi} \text{ for } n\geq n_0 (\epsilon),
\; \epsilon >
0, \tag{6.51}\label{eq6.51} 
\end{equation*} 
with
\begin{equation*}
\delta = 1-\frac{1}{\lambda} \tag{6.52}\label{eq6.52} 
\end{equation*}
whenever there is a result \eqref{eq6.17} with uniform $\lambda$ valid in
$\dfrac{1}{2} \leq \sigma \leq 1$ (cf., for the history of this
question, Montgomery \cite{key5} (Chapter 14)). Therefore the density
hypothesis \eqref{eq6.14} with $\alpha = \dfrac{1}{2}$ would give
\eqref{eq6.51} 
with $\delta =\dfrac{1}{2}$. Montgomery \cite{key3} proved \eqref{eq6.51} with
$\delta =3/5$ and Huxley \cite{key8} (cf. \cite{key7} (p.~119))
succeeded in getting the best known value 
\begin{equation*}
\delta = \frac{7}{12}. \tag{6.53}\label{eq6.53}
\end{equation*}
(Now, in view of our initial statement, we can get \eqref{eq6.53} from
either of \eqref{eq6.11} or \eqref{eq6.13} also.) 

An analogous application of \eqref{eq6.10} is known with respect to the
least prime $p_1(q,\ell)$ in the arithmetic progression $\{\ell,
\ell+q, \ell+2q, \ldots \}, (\ell,q) = 1$, $0 \leq\ell < q$. This
stated (Iwaniec \cite{key4})   
\begin{equation*}
p_1(q,\ell) \ll_{\epsilon , q(q)} q^{\lambda +\epsilon}, \epsilon > 0,
\tag{6.54}\label{eq6.54} 
\end{equation*}         
so that one has from Jutila's result \eqref{eq6.11} that
\begin{equation*}
P_1(q,l) \ll_{\epsilon ,q(q)}  q^{\frac{12}{5} + \epsilon},\epsilon >
0. \tag{6.55}\label{eq6.55} 
\end{equation*}
(However, observe that the $\ll$-constant depend on the kernel of $q$
(cf. \eqref{eq1.19}). Here, with respect to Linnik's famous theorem,
the best known exponent in \eqref{eq6.54} with $\ll$-constant
independent of $q(q)$ is 550 proved unconditionally by Jutila.) 

Regarding\pageoriginale the analogues, for other arithmetical functions, of the
Bombieri theorem as well as of Barban-Davenport-Halberstam theorem
mentioned earlier, we have their applications in the proof of various
delicate asymptotic formulae, involving such function as the divisor
function $d_k(n)$, $r(n)$, $\mu (n)$ etc., as also general function
fulfilling certain conditions, their powers and some mixed forms of
these functions, and further, with the argument running through
certain polynomial sequences. For these problems we refer to Barban
\cite{key6}, \cite{key10}, Elliot and Halbertam \cite{key1}, Hooley
\cite{key1}, \cite{key6}, Huxley and Iwaniec \cite{key1}. Indlekofer
\cite{key1}, \cite{key2}, Iwaniec \cite{key2}, \cite{key5}, Katai
\cite{key1}, Linnik \cite{key4}, Motohashi \cite{key1}, \cite{key2},
\cite{key4}, \cite{key7}, Orr
\cite{key1}, \cite{key2}, Proter \cite{key2}. \cite{key3}, Rodriquez
\cite{key1}, Siebert and Wolke \cite{key1}, Vaughan \cite{key4},
\cite{key5}, A.I. Vinograndov \cite{key1}, and Wolke \cite{key3}, 
\cite{key5}, \cite{key6}. 

\medskip

\begin{center}
{\bf NOTES}
\end{center}   

\begin{enumerate}[{\bf 1.:}]
\item Montgomery and Vaugham \cite{key1} have shown that the proof of the
  classical formula 
\begin{equation*}
\int\limits^ T_0 |\zeta (\frac{1}{2} + it)|^2 dt = T \log T + O(T)
\tag{6.56}\label{eq6.56} 
\end{equation*}
can be greatly simplified by an application of Theorem
\ref{chap4-thm4.3}. Ramachandra \cite{key6}, \cite{key7} has extended
this to various other 
moments with reports to $\zeta (s)$ and $L$-functions. Such results are
important for the results mentioned in the second section of this
chapter (as has already been indicated in the context of \eqref{eq6.15}).  

\eqref{eq6.1}, \eqref{eq6.2}: For similar results involving
$|L(\dfrac{1}{2} + it, \chi)|^2$ see Gallagher
\cite{key8}. More general forms, with an averaging over certain
well-spaced $t$-sets of points, can be obtained by an additional use
of \eqref{eq6.2} (see Montgomery \cite{key5}) (Theorem\pageoriginale
10.3 and Corollary 10.4), and Huxley \cite{key7} (pp.~97,108)). Such
results can be found in Ramachandra \cite{key3} for $\zeta (s)$ in
$\dfrac{1}{2} \leq \sigma < 1$, for $L(s, \chi)$ in Ramachandra
\cite{key7} and Jutila \cite{key10}, and with $\chi \mod p$ for
$L(s, \chi)$ in Elliott \cite{key9}. 

\eqref{eq6.2}:~ A simple proof of \eqref{eq6.2} has been given by
Ramachandra \cite{key7}. For a result of this type with real
characters only see Jutila \cite{key7}. 

\eqref{eq6.4}:~ For a mean-value theorem for $|L' (\dfrac{1}{2} + it,
\chi)|^2$ which can be derived in the same way, see Vaughan
\cite{key6}. 

\item
\eqref{eq6.5}:~ Gallagher \cite{key8} has proved that one has 
\begin{equation*}
N(\sigma , T, \chi) \ll T^{3(1- \sigma)} \log ^C T, \text{ for
} q\leq T \tag{6.57}\label{eq6.57} 
\end{equation*}   
and 
\begin{equation*}
\sum_{\chi \mod q}(N(\sigma , T + 1, \chi)-N(\sigma , T,
\chi)) \ll q^{3(1- \sigma)} \log ^C T, \text{ for } T\leq q
\tag{6.58}\label{eq6.58} 
\end{equation*}   
with some constant $C$.
   
For results about the size of $L(1, \chi), \chi \mod p$,
we refer to the papers of Bareman, Chowla and Erd\"os \cite{key1}, Barban
\cite{key8}, \cite{key10}. Elliott \cite{key1}. Joshi \cite{key1} and
for any $L(s,\chi)$ to Elliott \cite{key8}.  

\eqref{eq6.10}, \eqref{eq6.12}:~ For estimates of these averages,
under certain restrictions on $q$ and $Q$, respectively, see
Ramachandra \cite{key4}. For earlier results, in particular, regarding the
`generalized density hypothesis' (namely, the estimates
\eqref{eq6.10}, \eqref{eq6.12} with the exponent $2(1-\sigma) + \epsilon$
 instead), see 
Balasubramanian and Ramachandra \cite{key1}, Huxley \cite{key9},
Huxley and Jutila \cite{key1}, and Jutila \cite{key5}, \cite{key9},
\cite{key10}, \cite{key11}.   

\eqref{eq6.12}:~ For a result, with the summation restriction to real
characters only, which has better estimates in some cases, see Jutila
\cite{key4}, \cite{key6}. A. Selberg \cite{key6},
cf. Montgomery\pageoriginale 
\cite{key8}) had proved (using his results \eqref{eq3.14} and
\eqref{eq3.33}) that  
\begin{equation*}
\sum_{q\leq Q} \sum_{\mathcal{X} \mod q}^* N(\sigma , T, \mathcal{X})
\ll_{\epsilon} (Q^5 T^3)^{(1 + \epsilon)(1 -\sigma)}, \frac{1}{2} \leq \sigma
\leq 1 (\epsilon > 0). \tag{6.59}\label{eq6.59} 
\end{equation*}

In this connection we have also the following deep result due to
Bombieri (\cite{key6} (Th\'eor\`em 14)): If there is a `Siegel-zero' $\beta_1$
(relative to $T \geq 2$ and a certain constant $c_0 > 0$) then holds 
\begin{equation*}
\sum _{q \leq T} \sum _{\chi \mod q}^* N(\sigma, T \chi)
\ll ((1-\beta_1 )\log T)T^{c(1-\sigma )} \tag{6.60}\label{eq6.60} 
\end{equation*}
with some absolute and effective $c$ and $\ll$-constants, where on the
left-hand side the exceptional zeros are not included. As an
application of this Bombieri \cite{key6} derives the well-known theorem of
Siegel: One has 
\begin{equation*}
1-\beta _1 \geq c (\epsilon )T^{-\epsilon}, T \geq 2 (\epsilon > o),
\tag{6.61}\label{eq6.61} 
\end{equation*}
with (as in all other known proofs) an ineffective $c(\epsilon) > o$. We
briefly sketch this deduction. Introducing $\theta$ as the supremum of
the real parts of the zeros of all the Dirichlet's $L$-functions, we see
easily that one can suppose (for the purpose of \eqref{eq6.61}) $\theta
=1$. Now, taking $T$ to satisfy 
\begin{equation*}
T \geq \max (q_0, |\gamma _0|, \exp (\frac{c_0}{1-\beta_0})), \beta_0
>1-\epsilon . \tag{6.62}\label{eq6.62} 
 \end{equation*}
 where $q_0$ is the (least) modules of the $L$-function which has a zero
 $\rho _\circ = \beta_\circ + i \gamma _\circ$ (with $\beta_\circ > 1-
 \epsilon$), we 
 see that $\rho_\circ$ is not an exceptional zero (relative to $T$
 and $c_0$) and also that the left-left-hand side \eqref{eq6.60} with
 $\sigma = 1 - \epsilon $ is $\geq 1$. Hence we have  
 \begin{equation*}
(1-\beta_1) \gg ((\log T)T^{c \epsilon})^{-1} \tag{6.63}\label{eq6.63}
 \end{equation*}
which gives \eqref{eq6.61} for $T$ subject to \eqref{eq6.62} and so
for all $T \geq 2$. 

\eqref{eq6.64}:~ Hal\'asz and Tur\'an \cite{key1} have shown that
\eqref{eq6.15} gives (even)
\begin{equation*}
N(\sigma ,T) \underset{\xi \delta} \ll T^\epsilon, \text{ for }\sigma \geq
\frac{3}{4} + \delta , (\text{ with } \epsilon >
o). \tag{6.64}\label{eq6.64} 
\end{equation*}       

Also they have proved in \cite{key2} that the generalized Lindel of
hypothesis, 
\begin{equation*}
L(s,\chi) \underset{\epsilon} \ll _T q^{\epsilon}, \text{ uniformly for }
\sigma \geq \frac{1}{2}, |t|\leq T, \tag{6.65}\label{eq6.65} 
\end{equation*}
implies the uniformly estimates, for \eqref{eq6.12}.
{\fontsize{10pt}{12pt}\selectfont
\begin{equation*}
\underset{\epsilon} \ll _T Q^{4(1- \sigma )+\epsilon} \text{ for } \frac{1}{2}
\leq \sigma \leq 1 \text{ and } \underset{\epsilon} \ll _{T,\delta} Q^\epsilon
\text{ for } \sigma \geq \frac{3}{4} + \delta (\epsilon > o, \delta >
o). \tag{6.66}\label{eq6.66} 
\end{equation*}}\relax\pageoriginale

\eqref{eq6.16}:~ In this connection, for earlier results, see
Montgomery \cite{key3}, Huxley \cite{key6}, Ramachandra \cite{key8},
Forti and Viola \cite{key2}, Huxley \cite{key9}, \cite{key11},
\cite{key12}, and Jutila \cite{key8}.  

\eqref{eq6.17}:~ Ramachandra \cite{key9} has proved for $\sigma \geq
\dfrac{1}{2}$ that 
\begin{equation*}
N(\sigma, T + T ^{5/12}) - N(\sigma ,T) \underset{\epsilon} \ll
T^{\frac{5}{9-6 \sigma}(1-\sigma + \epsilon )}, (\epsilon > 0)
\tag{6.67}\label{eq6.67} 
\end{equation*} 
and more generally, Balasubramanian \cite{key1} has shown that,
uniformly in $\sigma \geq \dfrac{1}{2}$ 
\begin{equation*}
N(\sigma ,T + H)-N (\sigma ,T) \ll H^{\frac{4}{3-2 \sigma}(1- \sigma
  )} \log^{100}H \text{~ for~ } T^{\frac{27}{82}}\geq H\geq
T. \tag{6.68} \label{eq6.68}
\end{equation*}     

\eqref{eq6.21}:~ This type of exponents in \eqref{eq6.10} and
\eqref{eq6.12} occurs also in Ramachandra \cite{key8}, and Balasubramanian
and Ramachandran \cite{key1}.  


\item
\eqref{eq6.23}, \eqref{eq6.24}:~ The first result of this type is due
to R\'enyi \cite{key2}. Replacing the sum on the left by 
\begin{equation*}
\sum_{q \leq x^{\alpha}}. \tag{6.69}\label{eq6.69}
\end{equation*}  

Barban \cite{key2} improved R\'enyi's result (i,e., with a certain small
$\alpha > 0$) to $\alpha =\dfrac{1}{6}-\epsilon$ (for every $\epsilon > 0$) and
succeeded then (\cite{key4}, \cite{key5}, \cite{key6}) in extending it
to all $\alpha < 3/8$. Since the value of the result increases considerably with
$\alpha$, Bombieri's theorem has led to most important
applications. A.I. Vinogradov \cite{key1} proved the same theorem (through
in a slightly weaker form, namely, with $\alpha =\dfrac{1}{2}-\epsilon$ for
every $\epsilon >0$) simultaneously. A result of this type with $\alpha
>\dfrac{1}{2}$ would have important consequences (cf. Buchstad \cite{key2},
Helberstam, Jurkat and Richert \cite{key1}) and it has been conjectured by
Elliott and Halberstam \cite{key2} (cf. Elliott \cite{key6}) that
every value of $\alpha <1$ is admissible.  

\eqref{eq6.25}:~ Gallagher \cite{key2} uses in his proof of \eqref{eq6.24}
the decomposition $\dfrac{L'}{L} = (1-LG)^2 \dfrac{L'}{L} + 2 L' G-LL'
G^2$\pageoriginale where $G$ is a partial sum of the Dirichlet's series for
$\dfrac{1}{L}$, instead of which Vaughan \cite{key6}, to obtain
\eqref{eq6.25}, uses the more efficient splitting $\dfrac{L'}{L} =
(\dfrac{L'}{L} + F)(1 - LG) + (L' + LF)G-F$ with $F$ being the partial sum
corresponding to $\dfrac{L'}{L}$. 

\eqref{eq6.26}:~ For a different proof of (a slightly weaker form
of) \eqref{eq6.26} see Motohashi \cite{key5}; more recent results
which in certain 
ranges (for $\theta$) improves \eqref{eq6.26} are due to Huxley and Iwaniec
\cite{key1} (according to Motgomery (oral communication), S. Ricci, a student
of his, has obtained a similar result).  

\eqref{eq6.28}:~ The best exponents known here,
\begin{equation*}
c \leq \frac{173}{1067} < \frac{1}{6}, \tag{6.70}\label{eq6.70}
\end{equation*} 
is due to Kolesnik \cite{key1}.

\eqref{eq6.29}:~ For similar results, see Bombieri and Davenport
\cite{key2}, Montgomery \cite{key5} (Chapter 15), and Vaughan
\cite{key6}.   

\eqref{eq6.32}:~ In this context, we mentioned that the function $r(n)$
which appears in this and the last section of this Chapter and for
which also \eqref{eq6.32} holds, denotes the general  
\begin{equation*}
r(n):= \sum_{d|n} \chi(d), \chi \mod q, n \in N :
\tag{6.71}\label{eq6.71} 
\end{equation*} 
in particular, when $\chi$ is the non-principle character to
the modulus 4, $r(n)$ denotes the number of representation of $n$ as a
sum of two squares (apart from a factor of $4$).  

\item
\eqref{eq6.36}:~ For stronger and more general results, and also for
those which can be obtained on the assumption of the generalized
Riemann hypothesis, see Hooley \cite{key5}, \cite{key7},
\cite{key9}. The corresponding problem for  
\begin{equation*}
\sum _{\substack {n \geq x \\ {n \equiv l \mod q}}} \mu^2 (n)
\tag{6.72}\label{eq6.72} 
\end{equation*}\pageoriginale
has been investigated extensively, both with the help of the large
sieve as well as by elementary methods only (Orr \cite{key1} (cf. Orr
\cite{key2}). Warlimont \cite{key1}, \cite{key2}, and Croft
\cite{key1}). Motohashi \cite{key6}  has 
used the idea of Montgomery's proof, namely without using the large
sieve, to obtain an asymptotic formula with respect to  
\begin{equation*}
\sum_{\substack{n \leq x \\ {n \equiv l \mod q}}} d(n)
\tag{6.73}\label{eq6.73} 
\end{equation*}

\eqref{eq6.50}:~ Barban (\cite{key10} (Theorem 3.3)) has extended his
result also to the more general sums 
\begin{equation*}
\sum_{\substack {p_i \leq x\\ {i = 1, \ldots ,k}\\ {p_1 \ldots p_k
      \equiv l \mod q}}} \tag{6.74}\label{eq6.74} 
\end{equation*}
where
\begin{equation*}
0 < \alpha_1 \leq \ldots \leq  \alpha_k, \alpha_1 + \cdots +\alpha_k = 1,
k\geq 2, \tag{6.75} \label{eq6.75} 
\end{equation*}
of which \eqref{eq6.50} corresponds to the case 
\begin{equation*}
k = 2; \alpha_1 = \alpha_2 = \frac{1}{2}. \tag{6.76}\label{eq6.76}
\end{equation*}

Orr \cite{key1}, \cite{key2} has obtained results, with sums of the
type \eqref{eq6.74}, 
where primes are replaced by squarefree numbers, and also  with mixed
products instead. 

For results of others types concerning short interval we merely
mention two recent papers: Gallagher \cite{key5}, and Ramachandra
\cite{key10}.  

\item
\eqref{eq6.51}:~ The history of the more general question of the
number of primes in short interval\pageoriginale
$$
[x, x+x^\delta]
$$
starts with Hohesisel, who was the first to prove a result with a
$\delta < 1$, actually with any $\delta > 1-(33000)^{-1}$. Another
famous question, concerning small differences of primes, is to find
an estimate of the form 
\begin{equation*}
\lim_{n \to \infty} ~ \frac{p_{n+1}-p_n}{\log p_n}\leq \theta , \qquad
(p_n:n^{\text{th}} \text{ prime}). \tag{6.77}\label{eq6.77} 
\end{equation*} 
\end{enumerate} 

It is conjectured that this $\varliminf$ is zero, an obvious
consequences of the twin-prime conjecture and it has been known that,
under the assumption of the generalized Riemann hypothesis, one would
have \eqref{eq6.77} with $\theta =\dfrac{1}{2}$. However, Bombieri and
Debenport \cite{key1} obtained \eqref{eq6.77} unconditionally with a $\theta <
\dfrac{1}{2}$, in fact, with $\theta =\dfrac{1}{8}(2 + \sqrt{3}) =
0.46650 \ldots$, where they used among other ideas the large
sieve. Now we have \eqref{eq6.77}, due to Huxley \cite{key10}, with 
\begin{equation*}
\theta = \frac{4 + \pi}{16} = 0.44634 \ldots \tag{6.78}\label{eq6.78} 
\end{equation*} 

For earlier results generalizations and allied questions see, for
instance, Huxley \cite{key2}, Uchiyama \cite{key1}, Moreno
\cite{key1}, and Wolke \cite{key11}. 

\eqref{eq6.54}:~ For $q = p^r$, see Gallagher \cite{key5}: for previous
result and further references see Jutila \cite{key5}. 

\eqref{eq6.55}:~ The corresponding problem for general sequences, in
particular for squarefree numbers, was dealt with by Wolke \cite{key1},
\cite{key2} as an application of (2.103) (see also Warlimont \cite{key2}). 
 
We have more application for the result of Section $\underbar{2}$.,
as also of those in the notes to that section. First of all there is
the question of the least character non-residues for which we refer to
Montgomery \cite{key5} (Chapter \ref{chap13}). and Elliott
\cite{key4}. For estimations of 
character sums, see Montgomery \cite{key5} (Chapter
\ref{chap13}),\pageoriginale Gallagher 
\cite{key4} \cite{key8}, and Jutila \cite{key6} (as also Jutila
\cite{key2}). Regarding 
primitive  roots  and Artin's conjecture we mention the papers of
Gallagher \cite{key1}. Burgess and Elliott \cite{key1},  Goldfeld
\cite{key1}, Elliott \cite{key1}, \cite{key2}, Vaughan \cite{key1}, and 
for some related questions, Gallagher \cite{key1}, \cite{key3}, and Goldfeld
\cite{key2}. Finally, for the applications to the problem of the
largest prime factor of numbers we refer to the papers of Goldfeld
\cite{key2}, Hooley \cite{key2}, \cite{key4},
Jutila \cite{key3}, Motohashi \cite{key3}, \cite{key11}, and
Ramachandra \cite{key1}, \cite{key2}.  
