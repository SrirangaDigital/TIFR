
\chapter{The `Sieve Form' of the Large Sieve}\label{chap7}%chap 7
 
AS\pageoriginale HAS been mentioned in the beginning of the preceding
chapter, we consider now the (direct) general application of the large
sieve in its arithmetical form, in the sense of $B$. of Chapter
\ref{chap0}. Let us recall the situation described there. Consider the set of
integers in an  interval $(M, M+N]$. For each prime $p$ (be-longing
  to a certain 
  set $\mathfrak{p}$) we drop from our set all such numbers as which
  fall in any one of certain $\omega (p)$ of the residue classes mod
  $p$, and denote the set of remaining integers by $\gamma$. Our
  object is to obtain an upper bound for $S: = |\gamma |$.   
 
If $\omega (p)$ is (absolutely) bounded the sieve methods of Brun and
Selberg yield satisfactory result. On the other hand, these method
fail if, for instance, $\omega (p)$ is an increasing function of
$p$. This was the reason, as has already been started in our remark
preceding \eqref{eq0.26}, that Linnik called his method the large sieve. In
this context one might ask whether it is not possible to have a
version of the large sieve which shall include the Selberg sieve, say
when $\omega(p)$ is bounded. 
 
This problem has not yet been solved. However as we shall see, it is
possible to adapt the large sieve for this purpose provided we
confine ourselves to the `linear sieve' and aim merely for the
simplest Selberg upper bound. 
 
 In connection with his problem, we recall that in the original form
 (of Linnik's method) there was the defect, of having the summation
 restricted to primes only (cf. \eqref{eq0.54}), which existed upto
 the first  paper of Gallagher \cite{key1}. 
 Bombieri was the first to notice (cf,
 Bombieri and Davenport \cite{key2}) that one can solve this problem if the
 dimension $k$ (cf. \eqref{eq11.3}) of the sieve problem in question equals
 one (i.e., $\omega (p) = 1$ on the average). In the general
 case. Montgomery \cite{key1} obtained\pageoriginale this result first, by
 discovering the identity (in our notation of Chapter \ref{chap0}) 
\begin{equation}
q \sum^q_{h=1}\Big|\sum _{d|q} \frac{\mu (d)}{d}
S(\frac{q}{d},h)\Big|^2 = \mathop{\sum{}'}\limits^{q}_{l=1}\Big|
T(\frac{l}{q})\Big|^2 
\; \forall q \in \mathbb{N}, \tag{7.1}\label{eq7.1} 
\end{equation}
and thereby extending \eqref{eq0.53} to composite numbers.

However, there is a simpler way of dealing with this problem which
does not make use of \eqref{eq7.1} but starts instead with the well-known
formula \eqref{eq1.34}. This was first found for $\kappa = 1$ only
(cf. Richert \cite{key2}). Later Huxley \cite{key5} 
(see also Montgomery \cite{key5} (Chapter \ref{chap3}),
and Huxley \cite{key7} (Chapter \ref{chap8}) 
succeeded  in extending this method to
the general case (cf. \eqref{eq7.24}). In both the cases the question is
reduced to an application of the large sieve in its version of
Chapter \ref{chap2}. Therefore, the best known solution so far, for our sieve
problem, is due to Montgomery and Vaughan \cite{key2} (regarding
\eqref{eq7.6}, see also Gallagher \cite{key6}), 
who derived this from their strong result given in Chapter \ref{chap2}. 

Before stating the main result of this chapter we sketch now the
aforementioned simpler approach in the case $\kappa=1$: Let
$\mathscr{Q}$ denote the
set of all natural numbers composed of only primes $p$ in our set
$\mathfrak{p}$ and let us drop (as in Selberg sieve) from the set of integers in
($M,M+N$] those divisible by some $p \in \mathfrak{p}$ to obtain our
  $\gamma(\omega(p)=1)$. (All other notation are as usual.) By
  \eqref{eq1.34}. we have (since ($n,q)=1$) 
\begin{equation}
\mu(q) = \mathop{\sum{}'}\limits^{q}_{\ell=1} e(n \frac{\ell}{q})\quad
\forall n \in \gamma, \; \forall q \in \mathscr{Q}. \tag{7.2}\label{eq7.2} 
\end{equation}

Summing this over all $n \in \gamma$ and interchanging summation
we obtain (on squaring both sides) 
\begin{equation*}
\mu^2 (q) S^2 =| \mathop{\sum{}'}\limits^{q}_{l=1}T(\frac{l}{q})|^2,
\tag{7.3}\label{eq7.3} 
\end{equation*}
which gives, by Cauchy's inequality and summing (after a division bu
$\varphi(q))$ overall\pageoriginale  $q \in \mathscr{Q}$, $q\leq Q$, 
\begin{equation}
(\sum_ {\substack{q \leq Q \\ {q \in \mathscr{Q}}}}
  \frac{\mu^2(q)}{\varphi(q)})S^2 \le \sum_ {\substack{q \le Q \\ {q
        \in \mathscr{Q}}}}
       \mathop{\sum{}'}\limits^{q}_{\ell=1}|T(\frac{\ell}{q})|^2.
 \tag{7.4}\label{eq7.4}   
\end{equation}

Now an application of \eqref{eq2.90} gives an upper bound of the
desired from for $S$: 
\begin{equation}
S \leq \frac{N+Q^2}{(\sum _{\substack {q \leq Q \\ q \in \mathscr{Q}}}
  \frac{\mu^2(q)}{\varphi(q)})}. \tag{7.5}\label{eq7.5} 
\end{equation}

Now we come to the main result of this chapter. We state it as

\setcounter{section}{7}
\setcounter{theorem}{0}
\begin{theorem}\label{chap7-thm7.1}%them 7.1
Let $ \gamma \subset (M, M+N]$ be a set of $S$ integers. For each
  prime $p$ let us denote by $\omega(p)$ the number of residue classes
  which do not have any number from $\gamma$. Then, for any $z > 0$, we
  have 
\begin{equation}
S \leq \frac{N+z^2}{L(z)}, \tag{7.6}\label{eq7.6}
\end{equation}
where
\begin{equation}
L(z)=\sum_{q \leq z} \mu^2 (q) \prod_{p|q}
\frac{\omega(p)}{p-\omega(p)}, \tag{7.7}\label{eq7.7} 
\end{equation}
and also
\begin{equation}
S \leq \frac{1}{L^*(z)}, \tag{7.8}\label{eq7.8}
\end{equation}
with
\begin{equation}
L^* (z) = \sum_{q \leq z}(N + \frac{3}{2}qz)^{-1} \mu^2
(q)\prod_{p|q}\frac{\omega (p)}{p-\omega(p)}. \tag{7.9}\label{eq7.9} 
\end{equation}
\end{theorem}

\begin{remarks*}
Observe here that the inequalities \eqref{eq7.6}, \eqref{eq7.8}, and
also those of 
Theorem \ref{chap7-thm7.2}, do not deteriorate if $z$ is replaced by
its integral 
part (with an obvious interpretation if $0 < z < 1$. Therefore, we
set $\underbar{Q=[z]}$ and it suffices now to prove our results for
$Q \geq 1$. Further, these results remain true if $\omega(p) =p$ for
prime $p$ (since then $S = 0$). Hence we assume throughout
that\pageoriginale  
\begin{equation}
\omega (p) < p \text{~ for all~ } p \tag{7.10}\label{eq7.10} 
\end{equation}
holds.
\end{remarks*}

\begin{proof}
Our first objective is to prove \eqref{eq7.24}, an inequality which
trivially if $\omega(p) = 0$ for some $p|q$ or if $\mu (q) = 0$ and
also for $q=1$. So we can impose the conditions that 
\begin{equation}
1<q(\leq Q) \text{~ is a squarefree number,} \tag{7.11}\label{eq7.11}
\end{equation}
as well as (cf. \eqref{eq7.10})
\begin{equation}
0 < \omega (p) < p\quad \forall p|q.\tag{7.12}\label{eq7.12}
\end{equation}

Now, for each $p|q$ we have (for certain $h=h(p)$)
\begin{equation}
n \not\equiv h_i \mod p,i =1, \ldots ,\omega (p),\quad \forall n
\in \gamma, \tag{7.13}\label{eq7.13} 
\end{equation}
which restrictions are equivalent to, by Chinese remainder theorem and
\eqref{eq7.11}, (with certain $f=f(q)$) 
\begin{equation}
(n-f_j,q)=1,  j=1, \ldots , \omega (q), \forall n \in \gamma
  \tag{7.14}\label{eq7.14} 
\end{equation}
with
\begin{equation}
\omega (q)=\prod_{p|q}\omega (p), \tag{7.15}\label{eq7.15}
\end{equation}
so that $\omega(p)$ become a multiplicative function ($\nequiv 0$,
because of \eqref{eq7.12}) on setting 
\begin{equation}
\omega (1) = 1 ; \tag{7.16}\label{eq7.16} 
\end{equation}
i.e., we also have
\begin{equation}
\omega \in m. \tag{7.17}\label{eq7.17}
\end{equation}

Next, in view of \eqref{eq7.14}, we have, by \eqref{eq1.34},
\begin{equation}
\mu (q) = \mathop{\sum{}'}\limits^{q}_{l = 1} e((n-f_j)
\frac{\ell}{q})\quad \forall n\in 
\gamma \text{~ and~ } j=1, \ldots \omega (q). \tag{7.18}\label{eq7.18} 
\end{equation}

Here summing over all $n \in \gamma$ and also over all $j$ followed by
squaring both the sides\pageoriginale results in, by Cauchy's
inequality, 
\begin{equation}
\begin{cases}
S^2 \mu^2 (q) \omega^2 (q) &= (\mathop{\sum{}'}\limits^{q}_{\ell=1}
\sum\limits_{n \in \gamma} e (n \frac{\ell}{q}) \sum \limits^{\omega
  (q)}_{j=1} e(-f_j \frac{\ell}{q}))^2 \leq \\  
&\leq (\mathop{\sum{}'}\limits^{q}_{\ell =1}|T(\frac{\ell}{q})|^2)
(\mathop{\sum{}'}\limits^{q}_{\ell =1} \sum \limits^{\omega (q)}_{j,
  j'=1} e((f_j,-f_j)\frac{\ell}{q})), 
\end{cases} \tag{7.19}\label{eq7.19} 
\end{equation}
where we have employed as before (cf. \eqref{eq7.3}) our usual
definition (cf. \eqref{eq2.89}) 
\begin{equation}
T(x): = \sum_{n \in \gamma} e(nx). \tag{7.20} \label{eq7.20}
\end{equation}

Denoting the second factor in the last expression of \eqref{eq7.19} by
$\sum_\circ(q)$ and taking the  summation inside, we see that one has on
using \eqref{eq1.29} and \eqref{eq1.32} 
\begin{equation}
\begin{cases}
\sum_\circ (q) &= \sum \limits^{\omega (q)}_{j,j'=1} c_q(f_{j'} - f_j) =
\sum \limits^{\omega (q)}_{j,j'= 1} \sum\limits_{\substack{d|q
    \\ {d|f_{j'} - f_j}}} d \mu (\frac{q}{d})= \\ 
 &=\sum \limits_{d|q} d \mu (\frac{q}{d})\sum \limits^{d}_{b=1}
(\sum\limits^{\omega (q)}_{\substack{j=1\\ {f_j \equiv b \mod d}}}
1)^2. \tag{7.21}\label{eq7.21} 
\end{cases}
\end{equation}

Note that here the $b's$ for which the corresponding inner sum is not
empty are precisely those $\omega(d)$ forbidden residue classes $\mod
d$, because of \eqref{eq7.14}. Further, for each such $b$ the inner sum
counts the same number of $f_j$'s, namely $\omega(q/d)$. Hence, from
\eqref{eq7.21}. 
\begin{equation}
\sum_\circ (q) =\sum_{d|q} d \mu (\frac{q}{d}) \omega (d) \omega^2
(\frac{q}{d}). \tag{7.22}\label{eq7.22} 
\end{equation}

Now taking $f_1(n)=n \omega(n)$, $f_2(n)= \mu (n) \omega^2(n)$ in
\eqref{eq1.18} it follows, because of \eqref{eq7.17}, \eqref{eq1.11}
and \eqref{eq1.12}, that  
\begin{equation}
\sum_\circ (q)= \prod_{p|q}(p \omega(p) - \omega^2 (p)) = \prod_{p|q}\{
\omega (p)(p- \omega (p)) \}. \tag{7.23}\label{eq7.23} 
\end{equation}

Using \eqref{eq7.23} in \eqref{eq7.19} and noting \eqref{eq7.15} we
obtain 
\begin{equation}
S^2 \mu^2 (q)\prod_{p|q}\frac{\omega (p)}{p-\omega(p)} \leq
\mathop{\sum{}'}\limits^{q}_{l=1}|T(\frac{l}{q})|^2. \tag{7.24}\label{eq7.24} 
\end{equation}

This\pageoriginale is the basic inequality providing the connection
between our sieve problem for $\gamma$ and large sieve method. In view of the
remarks at the beginning of the proof, \eqref{eq7.24} is valid for all $q
\epsilon \mathbb{N}$. 

Finally, summing \eqref{eq7.24} over all $q \leq Q$ it follows, by
\eqref{eq7.7} and \eqref{eq2.90}. that 
\begin{equation}
S^2 L(Q)\leq \sum_{q \leq Q} \mathop{\sum{}'}\limits^{q}_{\ell
  =1}\Big|T(\frac{l}{q})\Big|^2 \leq (N + Q^2) \sum_{n = \in \gamma}1
=(N + Q^2)S. \tag{7.25}\label{eq7.25} 
\end{equation}

This proves \eqref{eq7.6}. on recalling our earlier remark preceding
\eqref{eq7.10}. Further, \eqref{eq7.8} is proved in the same manner by
multiplying \eqref{eq7.24} by ($N + \dfrac{3}{2}q Q)^{-1}$ before
summation and then using \eqref{eq2.91}. Thus Theorem
\ref{chap7-thm7.1} is completely proved. 
\end{proof}

The following seemingly more general result is easily derived from
Theorem \ref{chap7-thm7.1}: 

\begin{theorem}\label{chap7-thm7.2}%the 7.2
Under the assumptions of Theorem \ref{chap7-thm7.1}, let $a_n$ be
arbitrary complex numbers satisfying 
\begin{equation}
a_n=0 \quad \forall n \notin \gamma. \tag{7.26}\label{eq7.26}
\end{equation}
\end{theorem}

Then, for any $z >0$, we have 
\begin{equation}
\Big| \sum_{M < n \leq M+N} a_n \Big|^2 \leq \frac{N+z^2}{L(z)}
\sum_{M < n \leq M+N} |a_n|^2 \tag{7.27}\label{eq7.27} 
\end{equation}
and 
\begin{equation}
\Big| \sum_{M < n \leq M+N} a_n \Big|^2 \leq \frac{1}{L^*(z)}\sum_{M <
  n \leq M+N} |a_n|^2. \tag{7.28}\label{eq7.28} 
\end{equation}

\begin{proof}
It follows from \eqref{eq7.26}, that by Cauchy's inequality the
left-hand side of \eqref{eq7.27} (and so also of \eqref{eq7.28}) is  
\begin{equation}
\leq (\sum_{n \in \gamma} 1)(\sum_{M < n \leq M+N} |a_n|^2 ) = S
\sum_{M < n \leq M+N} |a_n|^2 , \tag{7.29}\label{eq7.29} 
\end{equation}
from which our results are readily obtained from Theorem \ref{chap7-thm7.1}.  
\end{proof}

\begin{remark*}
One\pageoriginale might like to consider Theorem \ref{chap7-thm7.2} as
a weighted from of Theorem \ref{chap7-thm7.1}; but the restriction
\eqref{eq7.26} and the relation 
\eqref{eq7.29} show that Theorem \ref{chap7-thm7.1} is never weaker
than Theorem \ref{chap7-thm7.2}.  
\end{remark*}

\medskip
\begin{center}
{\bf NOTES} 
\end{center}

\eqref{eq7.1}:~ Montgomery's \cite{key1} proof of \eqref{eq7.1}
proceeds in the following way: Set  
\begin{equation}
\tilde T (q,h): = \sum^{q}_{\ell =1} T(\frac{\ell}{q})
e(-h\frac{\ell}{q}) \tag{7.30}\label{eq7.30} 
\end{equation}
so that, by \eqref{eq1.27}, one has
\begin{equation}
\sum^{q}_{h=1}|\tilde T (q,h)|^2 = q \mathop{\sum{}'}\limits^{q}_{\ell
  =1}|T(\frac{\ell}{q})|^2. \tag{7.31}\label{eq7.31} 
\end{equation}

Now, using \eqref{eq1.29}, \eqref{eq1.32} and \eqref{eq0.2},
\begin{equation}
\tilde T (q,h) = \sum_{n \epsilon \gamma}\sum^{q}_{h=1}
e((n-h)\frac{\ell}{q})=\sum_{n \epsilon \gamma} \sum_
{\substack{d|q \\ { d|n-h}}} d \mu (\frac{q}{d})= \sum_{d|q} d \mu
(\frac{q}{d}) S(d,h), \tag{7.32}\label{eq7.32} 
\end{equation}
from which \eqref{eq7.1} is derived by means of \eqref{eq1.30} and
\eqref{eq7.31}. 

For identities of this type see \eqref{eq1.27}, Montgomery
\cite{key1}, Huxley \cite{key7} (Chapter 18), and Sokolovskij
\cite{key1}. 


Theorem \ref{chap7-thm7.1}:~
This result contains the ($B$)-version of the large sieve, as well as,
in the cases mentioned in the introduction of this chapter, the
`small' sieves. On the other hand, as per an observation made by
Kobayashi \cite{key1}, one can also derive Theorem \ref{chap7-thm7.1},
and consequently also Theorem \ref{chap7-thm7.2} and Theorem
\ref{chap8-thm8.1}, in these cases from the Selberg sieve with the
additional tool of Theorem \ref{chap2-thm2.6} (see Halberstam and
Richert \cite{key1} (pp.~125--126)). 

In the case of $\omega(p)$ being close to $p$ (at least on the
average). the `larger sieve' of Gallagher \cite{key3} is more
effective. This sieve also includes prime-power moduli. 

\eqref{eq7.6}:\pageoriginale Johnsen \cite{key1} has generalized Montgomery's
  \cite{key1} first result of this kind to include non-squarefree
  numbers also by 
  reducing the question to an inequality of the type of \eqref{eq7.24}, so
  that the improved version \eqref{eq2.90} leads to the following:

`For each $p$ remove all but $g(p)$ residue classes $\mod p$. In each
  of the remaining classes, remove all but $g(p^2)$ different residue
  classes $\mod p^2$. and so on. Then the number of $n \leq N$ which
  remain is at most $(N+z^2)/\tilde{L} (z)$, for every $z > 0$, where  
\begin{equation}
\tilde{L}(z): = \sum_{q \leq z} \prod_{p^\nu \parallel q}(
\frac{p^\nu}{h(p^\nu)}- \frac{p^{\nu-1}}{h(p^{\nu-1})})
\tag{7.33}\label{eq7.33}  
\end{equation}
with $h(p^\nu) =g(p)g(p^2) \ldots g(p^\nu)$ being the number of
residue classes $\mod p^\nu$ remaining at the $\nu^{\text{th}}$ stage'. 

A simpler proof of this result has been given by Gallagher \cite{key7}
(cf. Gallagher \cite{key6}). 

\eqref{eq7.22}: For the remark that precedes \eqref{eq7.22} note that
otherwise, by \eqref{eq7.15} and \eqref{eq7.11}, there will be one
forbidden $b \mod d$ with more than $\dfrac{\omega(q)}{\omega(d)} =
\omega(\dfrac{q}{d})$ of distinct $f_j 's \equiv b \mod d $ and hence
for any $n(\in \gamma )$ and one such $f_j$ we would have $(n-f_{j'}
\dfrac{q}{d})>1$ contrary to \eqref{eq7.14}. 

\eqref{eq7.24}: For a variant of the proof of \eqref{eq7.24} see
Montgomery \cite{key5} (Chapter \ref{chap3}), and Bombieri \cite{key6}
(p.~21). 

