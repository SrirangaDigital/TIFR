
\chapter{On Goldbach's Conjecture and Prime-Twins}\label{chap13}%\sec 13

NOW\pageoriginale WE have prepared the ground for providing the main
results of (the second part of) these lectures. These results
represent the best approximation (in case sense) to the two most
prominent problems in the additive theory of prime numbers (namely,
those mentioned in the title of this chapter). The first proof of
these results is due to Chen \cite{key1} (for simplifications see Ross
\cite{key1}. cf. Malberstan and Richert \cite{key1} (Chapter
11)). Here we shall give a 
proof with further simplifications, in respect of (numerical)
calculations and more specifically, involving only elementary
functions. Actually this is accomplished at the expense of the quality
of the constants $(c_\circ)$ occurring in these results, which do not
however affect the qualitative statement of these results. For the
proof we use the special form (as given in Theorem
\ref{chap12-thm12.2}) of Kuhn's 
sieve and, instead of the earlier practices of translating this
theorem into the language of additive number theory, follow the idea
of Chen by subtracting an additional term. Then, as one would expect,
reformulate the problem for an application of Theorem
\ref{chap10-thm10.3}, thereby completing the proof. 
 
We start with (cf. \eqref{eq12.37}, \eqref{eq10.16})
\begin{equation*}
G(N):=\sum_{\substack{p <N-1 \\ (N-p,
    \prod\limits_{p'<N^{1/8}} p') =1\\ \mu (N-p) \neq 0}} \big
\{1-\frac{1}{2}\sum_{\substack{N^{1/8}\leq p' < N^{1/3}\\p' |N-p}}
-\frac{1}{2} \sum_{\substack{p_1p_2p_3 \leq N \\N^{1/8} \leq p_1
    <N^{1/3} \leq p_2 p_3 \\ N-p=p_1p_2p_3}}. \tag{13.1}\label{eq13.1} 
 \end{equation*} 
 
As we shall see later, $G(N)>0$ implies (in the direction of Goldbach's
conjecture) that
 \begin{equation*}
N = p + P_2 \tag{13.2}\label{eq13.2}
\end{equation*} 
is\pageoriginale soluble. One can even obtain a lower bound for
$G(N)$ (when $N$ is even and large) in terms of $N$, by means of
Theorems \ref{chap12-thm12.2} and \ref{chap10-thm10.3}. 
Indeed, with suitable choices for the
parameters in these theorems, namely 
\begin{equation*}
X=N,2 | N \in \mathbb{N}, v=8 \tag{13.3}\label{eq13.3}
\end{equation*} 
in both of them and further
\begin{equation*}
h=-N, u=3, \lambda=2, \tag{13.4}\label{eq13.4}
\end{equation*} 
in Theorem \ref{chap12-thm12.2}, while in Theorem \ref{chap10-thm10.3}
\begin{equation*}
h=N, \tag{13.5}\label{eq13.5}
\end{equation*} 
 one obtains an estimate for an expression similar to
 $G(N)$. Actually, then one would have 
\begin{equation*}
\begin{cases}
\sum\limits_{\substack{p \leq N \\ (N-p,
    \prod\limits_{\substack{p'<N^{1/8}\\p' \nmid  N }
        N}}p')=1}\big \{1-\frac{1}{2} \sum\limits_{\substack{N^{1/8}
    \leq p' < N^{1/3}\\p' |N-p \\p' \nmid N}}1-
\frac{1}{2}\sum\limits_{\substack{p_1p_2p_3 \leq N \\ N^{1/8} \leq
    p_1<N^{1/3}\leq p_2<p_3\\N-p=p_1p_2p_3}} \geq \\ \geq
4\mathfrak{S}(N)\frac{N}{\log^2 N}\{\log 3-\frac{1}{2} \log
6-\frac{1}{2}c(8)+0(\log N)^{-\frac{1}{15}}\}.  
\end{cases}\tag{13.6}\label{eq13.6}
 \end{equation*}
 
Now a comparison of the left-hand side here with $G(N)$ shows that we
need the estimate 
 \begin{equation*}
1+\sum_{p'|N}1+ \sum _{N^{1/8} \leq p' \leq N^{\frac{1}{2}}}
\sum_{\substack{p \leq N \\ N-p=0 \mod p'^2}} 1 \ll \log N+
\sum_{N^{1/8}\leq p}\frac{N}{p'^2} \ll N^{7/8}. \tag{13.7}\label{eq13.7} 
\end{equation*}  
which provides a bound for the (difference in) contributions to these
expressions arising from (the possible $p=N -1$,) the numbers $N-p$
satisfying either of the conditions 
\begin{equation*}
(N-p, N)>1 \text{~ and~ } \mu (N-p)=0 \tag{13.8}\label{eq13.8}
\end{equation*}  

Hence\pageoriginale we have $G(N)$ also the same lower estimate given by the
right-hand side of \eqref{eq13.6}. Since, by \eqref{eq10.79} with $v=8$, 
\begin{gather*}
4 \bigg\{ \log \; 3- \frac{1}{2} \log  6-\frac{1}{2} c(8) \bigg\}=2
\log  \frac{3}{2}- 2c (8) \ge 2 \log\\
 \frac{3}{2}- \frac{3}{2} \log
\frac{64}{39} (> \frac{1}{2} \log \frac{3^7}{2.10^3}=;
c_0)>0,\tag{13.9}\label{eq13.9} 
\end{gather*}
we see that one has (for instance) the lower bound
\begin{equation*}
G(N) \ge  c_0 \mathfrak{S} (N) \frac{N}{\log^2 N} \quad\text{for}\quad N \ge
N_0,\tag{13.10}\label{eq13.10} 
\end{equation*}
where $N_0$ is some absolute constant.

Next, we elaborate on the remark above pertaining to \eqref{eq13.12}. To
being with observe that $G(N)$ does not exceed the part,
\textit{$(G^*(N)$ say,} of its defining sum comprising  only of all
the \textit{positive} (i,e., $> 0$) terms. And also note that any term
of \eqref{eq13.1} with its second inner sum at least 1 makes the first
sum, accompanying it, $\ge 1$ so that such a term is $ \le 0$. Thus
we see that the terms occurring in $G^*(N)$ have their second inner sum
empty and the first one is atmost $1$. In otherwords, one (since $\mu
(N-p)\neq 0$)   
\begin{equation*}
\begin{cases}
G(N)\le G^* (N) \le \frac{1}{2} | \{ p \le N:N-p=p',\\ 
N^{\frac{1}{8}}
\le p' < N^{\frac{1}{3}}\} \cup \{p \le N:N-p= p' p_2,\\
N^{1/8}\le p' <
N^{\frac{1}{3}} \le p_2 \}+ 
| \{ p \le N:N-p=p_1,\\
p_1 \ge N^{\frac{1}{3}}\} \cup \{ p \le N:N-p=
p_1 p_2, N^{\frac{1}{3}} \le  p_1 < p_2\} | 
\end{cases}\tag{13.11}\label{eq13.11}
\end{equation*}
and also
\begin{equation*}
G(N) \le | \{ p \le N: N-p =P_2 \} |.\tag{13.12}\label{eq13.12}
\end{equation*}

From \eqref{eq13.10}, \eqref{eq13.12} we obtain the desired

\setcounter{section}{13}
\setcounter{theorem}{0}
\begin{theorem}\label{chap13-thm13.1}
There is an absolute constant $N_0$ such that for all even numbers $N
\ge N_0$, we have   
\begin{equation*}
|\{ p \le N:N-p=P_2 \} \le c_0  \mathbb{C} (N) \frac{N}{\log^2 N},\quad
(c_0 >0),\tag{13.13}\label{eq13.13} 
\end{equation*}
where $c_0$ may be taken constant defined in \eqref{eq13.9}; in particular,
there is always\pageoriginale a solution of the equation  
$$
N= p+P_2  \quad\text{if}\quad 2|N\quad \text{and}\quad N \ge N_0.
$$
\end{theorem}

The proof of the corresponding result regarding the generalized\break
prime-twins proceeds analogously. Now  we choose again
(cf. \eqref{eq13.3})  
\begin{equation*}
x, 2 |h, 0 < |h| \le x^{\frac{1}{3}}, v=8\tag{13.14}\label{eq13.14}
\end{equation*}
in both the Theorems \ref{chap12-thm12.2}, \ref{chap10-thm10.3} and
further 
\begin{equation*}
u=3,v=8, \lambda = 2\tag{13.15}\label{eq13.15}
\end{equation*}
in Theorem \ref{chap12-thm12.2}. Also we note that the sum
$\sum\limits_{p \le x}$ on the left-hand side of \eqref{eq12.37} can
be replaced by $\sum\limits_{p+h \le x}$ (cf. \eqref{eq13.14}) apart
from  a negligible error of the order
$O(x^{\frac{1}{3}})$. Lastly, by \eqref{eq13.14}, we observe
that the $p$'s counted in \eqref{eq10.16} satisfy (irrespective of the
sign of $h$) $p+h \le x$ and also $p+h=p_1 p_2 p_3$. 

Now, by the choices \eqref{eq13.14}, \eqref{eq13.15} for the
parameters, it follows from Theorems \ref{chap12-thm12.2} and
\ref{chap10-thm10.3} (corresponding to \eqref{eq13.6})   
\begin{equation*}
\begin{cases}
\sum\limits_{\substack{p+h \le x \\(p+h,{\prod\limits_{\substack{p' <
          x^{1/8}\\p' \nmid  h}}} p')}=1} \bigg\{ 1 -
\frac{1}{2} \sum\limits_{\substack{x^{1/8} \le p' <x^{1/3}\\p'
    |p+h\\p'\not\mid h}} 1- \frac{1}{2} \sum\limits_{\substack{p_1 p_2
    p_3 \le x\\x^{1/8}\le p_1 < x^{1/3} \le p_2 < p_3 \\ p+h = p_1
    p_2p_3 }} 1 \bigg\} \\  
 \ge 4 \mathfrak{S}(h) \frac{x}{\log^2 x} \bigg\{ \frac{1}{2} \log
 \frac{3}{2}- \frac{1}{2}c(8) +O(( \log
 x)^{-\frac{1}{15}}\bigg\}.
\end{cases}\tag{13.16} \label{eq13.16}
\end{equation*}

From here, on using the argument of \eqref{eq13.7} and \eqref{eq13.9},
one obtains (for the analogue of $G(N)$ the lower bound 
\begin{equation*}
\begin{cases}
T_h(x) =\sum\limits_{\substack{p+h \le x
    \\(p+h,{\prod\limits_{\substack{p' < x^{1/8}\\ \mu(p+h) \neq 0}}}
    p')}=1} \bigg\{ 1 - \frac{1}{2} \sum\limits_{\substack{x^{1/8} \le
    p' <x^{1/3}\\p' |p+ h}} 1- \frac{1}{2} \sum\limits_{\substack{p_1
    p_2 p_3 \le x\\x^{1/8}\le p_1 < x^{1/3} \le p_2 < p_3 \\ p+h = p_1
    p_2p_3 }} 1 \bigg\} \\ 
\ge c_0 \mathfrak{S}(h) \frac{x}{\log^2 x} \text{ for } x \ge x_0. 
 \end{cases}\tag{13.17}\label{eq13.17}
\end{equation*}

Again\pageoriginale the reasoning leading to \eqref{eq13.11}, and so
also to \eqref{eq13.12}, is applicable with $p+h$ in place of
$N-p$. Thus one arrives at the final 
\begin{theorem}\label{chap13-thm13.2}%the 13.2
There is an absolute constant $x_0$ such that for any even number
$h$ (determined with respect to $x$) satisfying 
\begin{equation*}
0 < |h| \le x^{\frac{1}{3}},\tag{13.18}\label{eq13.18}
\end{equation*}
we have 
\begin{equation*}
|\{p+h \le x: p+h =P_2 \} |c_0 \mathfrak{S}(h) \frac{x}{\log^2x}
\text{for} x \ge x_0, (c_0 >0).\tag{13.19}\label{eq13.19} 
\end{equation*}
where $c_0$ (again) may be taken as the constant defined in \eqref{eq13.9};
is particular, for any non-zero even number $h$, there are infinitely
many primes $p$ such that  
\begin{equation*}
p+h=p_2.\tag{13.20}\label{eq13.20}
\end{equation*} 
\end{theorem}

\medskip
\begin{center}
{\bf NOTES}
\end{center} 
 
Chen's theorem affords a beautiful instance of the effective use of
various  powerful tools of numbers theorey. As an inspection of its
proof (and  those of Theorems \ref{chap12-thm12.2} and
\ref{chap10-thm10.3}) discloses we have 
employed Kuhn's sieve, Selberg's sieve (several times), Bombieri's
prime number theorem, Siegel-Walfisz theorem, contour integration,
the hybrid form of the large sieve and Chen's new idea described
below. 
 
Chen's idea, which has already been briefly indicated at the beginning
of this Chapter, can be described as follows. (We shall confine
ourselves, for this purpose, to Theorem \ref{chap13-thm13.1}) 
One first sifts the sequence $N-p$ so that the remaining numbers
satisfy.   
\begin{equation*}
N-p = P_3\tag{13.21}\label{eq13.21}
\end{equation*} 
(and then the numbers of these remaining ones is estimated from
below. cf. Theorem \ref{chap12-thm12.2}). 
Now to remove from the rest those which are
of the form $p_1p_2p_3$,\pageoriginale 
one subtracts (from the preceding lower bound) 
another sieve estimate (from above) for the numbers of solutions of  
\begin{equation*}
N-p_1p_2p_3=p\tag{13.22}\label{eq13.22}
\end{equation*} 
(cf. Theorem \ref{chap10-thm10.3}). If, as has been shown to be the
case for the problems under consideration, the lower bound exceeds the
upper estimate (for \eqref{eq13.22}), it follows then that (since the
surviving members of $\{N-p\}$ are all now $P_2's$) there must be
solutions of the equation 
\begin{equation*}
N=p+P_2.\tag{13.23}\label{eq13.23}
\end{equation*}

This procedure of inverting the equation \eqref{eq13.21} to
\eqref{eq13.22} has\break turned out to be much more fruitful than any
further known improvement 
of the sieve method one has started with. Other successful
applications of this last step have been given by Indlekofer \cite{key2},
Huxley and Iwaniec \cite{key1}. 

In this context we recall the remarks preceding \eqref{eq12.40}. To briefly
expand on that statement, we mention that to attempt an improvement
upon \eqref{eq12.46} (or \eqref{eq12.47}) by Chen's method one would require
(cf. \eqref{eq13.21} and \eqref{eq13.22}), limiting ourselves now to
the simplest case, for 
\begin{equation*}
n^2+1= P_2 \tag{13.24}\label{eq13.24}
\end{equation*}  
a satisfactory upper bound for the number of solutions of 
\begin{equation*}
p_1p_2p_3-1=n^2\tag{13.25}\label{eq13.25}
\end{equation*}  
with $p_i$'s restricted by some conditions (like those in Theorem
\ref{chap10-thm10.3}). Surprisingly  we do not any method of obtaining a
satisfactory estimate for the number of square in a sequence under
appropriate conditions, and specifically not even in this case. An
explanation would  clearly be that the sequence of squares, even
though\pageoriginale more regularly distributed, is much thinner than
the sequence 
of primes. This also indicated that the corresponding problem with
respect to \eqref{eq12.46} (or \eqref{eq12.47} is  much complicated. 
  
Theorems \ref{chap13-thm13.1} and \ref{chap13-thm13.2}: It is possible
to state Theorem \ref{chap13-thm13.1} in 
more precise form with respect to the prime factors of $P_2$, as can
be seen from a combination of \eqref{eq13.10} and \eqref{eq13.11}. A similar
remarks applies to Theorem \ref{chap13-thm13.2} also. 
  
Regarding the constant $c_0$ (cf. \eqref{eq13.13}, \eqref{eq13.19}) in
these theorems we note that our value of  
\begin{equation*}
c_0 \ge 0.0446\tag{13.26}\label{eq13.26}
\end{equation*} 
(cf. \eqref{eq13.9}) is rather small. One of the reasons for this is
the choice $v=8$, which was made to enable us to deal
(cf. \eqref{eq11.73}, \eqref{eq11.74}) with elementary functions only
(and also to 
simplify numerical calculations), as has been mentioned in the
introductions to this chapter, a convenience not available under the
better choice (of all earlier proofs) of $v=10$. Of course, we can
replace \eqref{eq13.26} by (cf. \eqref{eq13.9}) 
\begin{equation*}
c_0 \ge 2 \log \frac{3}{2}- \frac{3}{2} \log \frac{64}{39} \ge
0.0679,\tag{13.27}\label{eq13.27} 
\end{equation*}     
 but a better constant can be obtained by taking $v=10$. With this
 later choice, Chen \cite{key1} obtained  
 \begin{equation*}
c_0 \ge 0.3354\tag{13.28}\label{eq13.28}
\end{equation*}   
and any numerical integration one can even get
\begin{equation*}
c_0 \ge 0.3445\tag{13.29}\label{eq13.29}
\end{equation*}
(see Halberstam and Richert \cite{key1} (p. 338)). It returns out that
$v=11$ is close to the optimal choice, by considering the
non-elementary functions $F$, $f$ in a  wider range, and then one is
led to  
\begin{equation*}
c_0 \ge 0.3716.\tag{13.30}\label{eq13.30}
\end{equation*}\pageoriginale

In all these cases $u$ is kept fixed to be 3, a value convenient in
the arithmetical interpretation of the estimates of the weighted
sieve. The constant $c_0$ can further be slightly  improved by taking
$u>3$ but then Chen's procedure becomes more complicated. As to the
constant $c_0$ under consideration it is worthwhile to compare it
with 1, in view of the conjecture of Hardy and Littlewood mentioned
(under Theorem \ref{chap10-thm10.1}) in the Notes for Chapter
\ref{chap10}. This suggests 
that our present constant $c_0$ should be capable of much further
improvement. In this context, we add that one has theorems
(corresponding to Theorems \ref{chap13-thm13.1} and
\ref{chap13-thm13.2}) with $P_2$'s in 
\eqref{eq13.13} and \eqref{eq13.19} replaced by $P_3$ for a better
(corresponding) constant  
\begin{equation*}
c_0 \ge \frac{13}{6}\tag{13.31}\label{eq13.31}
\end{equation*}
(cf. Helberstam and Richert \cite{key1} (Theorem
\ref{chap9-thm9.2})). Comparing the 
methods of proof it is considered to be likely  that the constant
$c_0$ in Chen's theorem can be further improved by using the
logarithmic weights instead of Kuhn's weights. 

Continuing on with related questions we mention now a few results
concerning Goldbach numbers (namely, those even numbers $N$ which can
be written as sums of two primes) 
\begin{equation*}
N=p+p' \; (2|N).\tag{13.32}\label{eq13.32}
\end{equation*}
  
Ramachandra \cite{key4} has derived from  an estimate of
\eqref{eq6.11}-type, actually from (the uniform)  
\begin{equation*}
\sum_\chi N(\sigma, T,\chi ) \ll (q^2 T)^{g(1-\sigma )} \log^{14} (qT)
\text{ for } \frac{1}{2} \le \sigma \le 1,\tag{13.33}\label{eq13.33} 
\end{equation*}
that the numbers of Goldbach numbers  in the interval 
\begin{equation*}
x \le N \le x+x^{\lambda}\tag{13.34}\label{eq13.34}
\end{equation*}\pageoriginale
has the asymptotic formula 
\begin{equation*}
\frac{1}{2}x^{\lambda}+ O_{\lambda, A} (x^\lambda \log
^{-A}x)\text{~ as~ } x \to \infty, {\textit{~ if~ }} (1 \ge) \lambda > 1 -
\frac{1}{g},\tag{13.35}\label{eq13.35}  
\end{equation*}
and has also deduced from \eqref{eq13.35}, by combining with a result of
Montgomery \cite{key3}, that  (for $x \ge x_0$) there is always a prime $p$
in the interval  \eqref{eq13.34} such that both $p+1$ and $p-1$ are Goldbach
numbers. 

Many results have been proved (in various forms) in order to show that
`almost all' (with respect to the error-term  corresponding to that of
\eqref{eq13.15} even numbers are Goldbach numbers. The best result known
here, upto this time, is due to Montgomery an Vaughan \cite{key3} and states
that the number of even integers $N \le x$, which are not Goldbach
numbers is 
\begin{equation*}
\ll x^{1- \delta}, \text{ for some } \delta >
0.\tag{13.36}\label{eq13.36} 
\end{equation*}  
(For a previous result, see Vaughan \cite{key2}.)
 
Further, turning to problems allied to Theorem \ref{chap13-thm13.2}, we have
(cf. \eqref{eq12.40}) the observation  of Vaughan \cite{key3} which
yields, when combined with Chen's method the following result: Either
the equation  
\begin{equation*}
2p+1=p'\tag{13.37}\label{eq13.37}
\end{equation*}  
has infinitely many solutions or 
\begin{equation*}
2p+1 =p_1p_2\tag{13.38}\label{eq13.38}
\end{equation*}  
has infinitely many solutions, in which  extend event one , in
particular, infinitely many solutions of the equation 
\begin{equation*}
d(n+1)= d(n).\tag{13.39}\label{eq13.39}
\end{equation*}
  
This statement concerning \eqref{eq13.39} is a conjecture of Erd\"{o}s and\break
Mirsky \cite{key1} (cf.\pageoriginale Helberstam and Richert
\cite{key1} (p. 338)). 

Lastly we mention an application, due to Jutila \cite{key1}, to a question
allied to Theorem \ref{chap13-thm13.2}. He deduced from his result
\eqref{eq6.26}, by 
combining it with a theorem of Levin \cite{key2} (reference to Richert
\cite{key1} would permit to replace `8' by `7' in the following statement),
that  \textit{for every integer $r \ge 8$} there exists a numbers
$\theta (r)$ satisfying (with $c$ an in \eqref{eq6.28})  
\begin{equation*}
(c^* : = \frac{1+4c}{2+4c} \le) \; \theta(r) <
  1\tag{13.40}\label{eq13.40} 
\end{equation*}
(and $\theta(r)$ decreasing to $C^*$ for increasing $r$), such that 
\begin{equation*}
x < p < x+x^{\theta (r)}, p+2=P_r, \text{ for all } x \ge
x_0,\tag{13.41}\label{eq13.41} 
\end{equation*} 
is soluble. Also he started that a similar result can be derived for
(an almost-) Goldbach problems, which may be interpreted as that the
equation 
\begin{equation*}
N=p+P_r, N \ge N_0, 2|N, r \ge 8,\tag{13.42}\label{eq13.42}
\end{equation*}  
has a solution in two `almost equal' (-in a sense which is stronger
for larger $r$-) numbers $p$, $P_r$. 


\begin{thebibliography}{99}\pageoriginale  
\bibitem{key1}  ANDRUHAEV, H.M. 
\begin{enumerate}
\item The addition problem for prime and
  near-prime numbers in algebraic number fields. (Russian) \textit{
    DOkal. Akad. Nauk SSSR 159} (1964), 1207-1209 = {\em Soviet
    Math. Dokl.} {\em 5} (1964), 1666-1668, MR {\em 30}, 1116.  
\end{enumerate}

\bibitem{key2} ANKENY, N.C. and ONISHI, H. 
\begin{enumerate}
\item The general
  sieve. \textit{Acta Arith}. {\em 10} (1964/65), 31-62. MR {\em 29}, 4740.
\end{enumerate}

\bibitem{key3} BALASUBRAMANIAN, R. 
\begin{enumerate}
\item
An improvement of a theorem
  of Titchmarsh on the mean square of $|\zeta (\dfrac{1}{2} +
  it)|$. \textit{Proc. London Math. Soc}. (to appear).  
\end{enumerate}

\bibitem{key4}
BALASUBRAMANIAN, R. and RAMACHANDRA, K.
\begin{enumerate}
\item Two remarks on a result of Ramachandra.  \textit{J. Indian
  Math. soc}. (N.S.) {\em 38} (1974), 395-397.
\end{enumerate}

\bibitem{key5} BARBAN, M.B.
\begin{enumerate}
\item Arithmetic functions on `rare' set. (Russian) \textit{Doki. Akad.
  Nauk} Uz SSR {\em 1961}, no. 8, 10-12. 

\item New applications of the `great sieve' of
  Yu. V. Linnik. (Russian) \textit{Akad. Nauk} Uzbek. SSR
  \textit{Trudy Inst. Mat}. No. {\em 22} (1961), 1-20. MR {\em 30},
  1990.  

\item Linnik's ``great sieve'' and a limit theorem for the class
  number of ideals of an imaginary quadratic field. (Russian)
  \textit{Izv. Akad. Nauk} SSSR \textit{Ser. Mat}. {\em 26} (1962),
  573-580. MR {\em 27}, 1426. 

\item The density of zeros of Dirichlet $L$-series and the problems
  of the addition of primes and almost primes. (Russian)
  \textit{Dokl. Akad. Nauk} Uz SSR {\em 1963}, no.1, 9-10.  

\item The ``density'' of the zeros of Dirichlet $L$-series and the
  problems of the sum of primes and ``near primes'' (Russian)
  \textit{Mat. Sbornik} (N.S.) {\em 61 (103)} (1963), 418-425. MR {\em
    30}, 1992.  

\item Analogues of the divisor problem of
  Titchmarsh. (Russian. English summary) \textit{Vestnik
    Leningrad. Univ. Ser. Mat. Meh. Astronom}. {\em 18} (1963), no. 4,
  5-13. MR {\em 28}, 57. 

\item Multiplicative functions of $\sum_R$ - equidistributed
  sequences. (Russian. Uzbek summary) \textit{Izv, Akad. Nauk} Uz
  SSR Ser. Fiz. - Mat. Nauk {\em 1964} no. 6, 13-19. MR {\em 31}, 1239. 

\item On a theorem of P. Bateman, S. Chowla and
  P. Erd\"{o}s. (Russian. English summary) \textit{Magyar
  Tud. Akad. Mat, Kutat\'o Int . K\"ozl}. {\em 9} (1964), 429-435. MR
  {\em 33}, 109.   

\item A\pageoriginale remark on the author's paper  ``New applications of the
  `large sieve' of Yu. V. Linnik''. (Russian) \textit{Theory
  Probability Math. Static}. (Russian),
  pp. 130-133. \textit{Izdat. ``Nauka'' Uzbek. SSR, Tashkent, 1964},
  MR {\em 33}. 4037. 

\item The  ``large sieve'' method and its application to number
  theorey.  (Russian) \textit{Uspehi Mat. Nauk 21} (1966), no.
  1 (127), 51-102 =  \textit{Russian Math, Surveys 21} (1966),
  no. 1, 49-103. MR {\em 33}, 7320.   
\end{enumerate}

\bibitem{key6} BARBAN, M.B. and VEHOV, P.P.
\begin{enumerate}
\item An external problem. (Russian) \textit{Trudy
  Moskov. Mat. Ob\v{s}\v{c}}. {\em 18} (1968), 83-90 =
  \textit{Trans. Moscow Math. Soc}. {\em 18} (1968), 91-99. MR {\em
    38}, 4410. 
\end{enumerate}

\bibitem{key7} BATEMAN, P.T., CHOWLA, S. and ERD\"{O}S, P.
\begin{enumerate}
\item Remarks on the size of
  $L(1,\chi)$. \textit{Publ. Math. Debrecen} {\em 1} (1950),
  165-182. MR {\em 12}, 244; {\em 13}, 1138.
 \end{enumerate}
 
\bibitem{key8} BELLMAN, R.
 \begin{enumerate}
\item Almost orthogonal series. \textit{Bull. Amer. Math. Soc}. (2)
  {\em 50} (1944), 517-519. MR {\em 6}, p. 48.
 \end{enumerate}
 
\bibitem{key9} BOAS, R.P.
 \begin{enumerate}
\item A general moment problem. \textit{Amer. J. Math}. {\em 63}
  (1941), 361-370. MR {\em 2}, p. 281.
 \end{enumerate} 
 
\bibitem{key10} BOMBIERI, E.
 \begin{enumerate}
\item On the large sieve. \textit{Mathematika} {\em 12} (1965),
  201-225. MR {\em 33}, 5590. 

\item On a theorem of van Lint and Richert. \textit{Symposia
  Mathematica} Vol. {\em IV} (INDAM, Rome, 1968/69), 175-180. MR {\em
  43}, 4791. 

\item Density theorems for the zeta
  function. \textit{Proc. Sympos. Pure Math.} {\em 20}
  (1969). 352-358. MR {\em 47}, 4945.

\item A Note on the large sieve. \textit{Acta Arith}. {\em 18} (1971),
  401-404. MR  {\em 44}, 3982. 

\item On large sieve inequalities and their
  applications. \textit{Trudy Mat. Inst. Steklov}. {\em 132} (1973),
  251-256.   

\item Le grand crible dans la theorie analytique des
  nombres. \textit{Soc. Math. France, Asterisque} No. {\em 18}. 1974.  

\item On twin almost primes. \textit{Acta Arith}. {\em 28} (1975),
  177-193. 

\item Corrigendum to my paper ``On twin almost primes'' and an
  addendum on Selberg's sieve. \textit{Acta Arith}. {\em 28} (1976),
  457-461.  
 \end{enumerate}
 
\bibitem{key11} BOMBIERJ, E. and DAVENPORT, H.
\begin{enumerate}
\item Small differences between prime
  numbers. \textit{Proc. Roy. Soc}. Ser. A {\em 293} (1966), 1-18. MR
  {\em 33}, 7314. 

\item \textit{On\pageoriginale the large sieve method}. Number theory
  and Analysis (Papers in Honor of Edmund Landau) pp. 9-22. New York,
  1969. MR ~ {\em 41}, 5327.

\item Some inequalities involving trigonometrical
  polynomials. \textit{Ann. Scuola Norm. Sup. Pisa} (3) {\em 23} (1969),
  223-241. MR  {\em 40}, 2636. 
 \end{enumerate}
 
\bibitem{key12} BUCHSTAB, A.A.
\begin{enumerate}
\item New results in the investigation of the Goldbach-Euler problem
  and the problem of prime pairs. (Russian) \textit{Dokl. Akod. Nauk}
    SSSR {\em 162} (1965).735-738 = \textit{Soviet Math. Dokl}. {\em 6}
  (1965), 729-732. MR {\em 31}, 2226. 

\item Combinatorial strengthening of the sieve method of
  Eratosthenes. (Russian) \textit{Uspehi Mat. Nauk 22} (1967),
  no. 3 (135), 199-226 = \textit{Russain Math. Surveys 22} (1967),
  no. 3, 205-233, MR {\em 36}, 1413.  

\item A simplefied modification of the combinatorial sieve. (Russian)
  \textit{Moskov. Gos. Ped. Inst, U\v{c}en Zap.} 375 (1971), 187-194.
  MR {\em 48}, 11019.  
\end{enumerate} 

\bibitem{key13} BURGESS, D.A.
\begin{enumerate}
\item The average of the least primitive root modulo
  $p^2$. \textit{Acta Arith. 18} (1971), 263-271. MR {\em 45}, 212.
 \end{enumerate}
 
\bibitem{key14} BURGESS, D.A and ELLIOTT, P.D.T.A.
 \begin{enumerate}
\item The average of the least primitive root. \textit{Mathematika
  15} (1968), 39-50. MR {\em 38}, 5736.  
 \end{enumerate} 
 
\bibitem{key15} CHEN, J.
 \begin{enumerate}
\item  On the representation of a large even integer as the sum of a
  prime and the product of atmost two primes. \textit{Sci. Sinica
    16} (1973), 157-176.  
 \end{enumerate}

\bibitem{key16} CHOI, S.L.G. 
 \begin{enumerate}
\item On a theorem of Roth. \textit{Math. Ann. 179} (1969),
  319-328. MR {\em 39}, 1421.
\end{enumerate}  
 
\bibitem{key17} CROFT, M.J.
\begin{enumerate}
\item Square-free numbers in a arithmetic
  progressions. \textit{Proc. London Math. Soc}. (3) {\em 30}
  (1975). 143-159.   
\end{enumerate}  
  
\bibitem{key18} DAVENPORT, H.
\begin{enumerate}
\item \textit{Multiplicative numbers theorey}. Chicago , III., 1967,
  vii + 189 pp. MR {\em 36}, 117.
\end{enumerate}

\bibitem{key19} DAVENPORT, H. and HALBERSTAM, H.
\begin{enumerate}
\item The values of a trigionometrical polynomial at well spaced
  points.  \textit{Mathematika 13} (1966), 91-96. MR {\em 33}, 5592;
  \textit{Corrigendum and Addendum: ibid. 14} (1967), 229-232. MR {\em
    36}, 2569. 

\item Primes in arithmetic progressions. \textit{Michigan Math. J.
  13} (1966), 485-489, MR {\em 34}, 156; {\em Corrigendum: ibid.
  15} (1968), 505. MR {\em 38}, 2099. 
\end{enumerate}

\bibitem{key20} ELLIOTT, P.D.T.A.\pageoriginale
\begin{enumerate}
\item On the size of $L(1, \chi)$. \textit{J. Reine Angew. Math. 236}
  (1969), 26-36. MR {\em 40}, 2619.

\item The distribution of primitive roots. {\em Canad. J. Math. 21}
  (1969), 822-841. MR {\em 40}, 104.

\item A restricted mean value theorem. \textit{J. London Math. Soc.}
  (2) {\em 1} (1969), 447-460 MR {\em 40}, 111.

\item On mean value of $f(p)$. \textit{Proc. London Math. Soc.} (3)
  {\em 21} (1970), 28-96. MR {\em 42}, 1783.

\item The Turan-Kubilius inequality, and a limitation theorem for
  the large sieve. \textit{Amet. J. Math. 92} (1970), 293-300. MR {\em
    41}, 8360.    

\item Some applications of a theorem of Raikov to number
  theory. \textit{J. Number Theory 2} (1970), 22-55. MR {\em 41}, 3422. 

\item One inequalities of large sieve type. {\em Acta Arith}. {\em 18}
  (1971), 405-422. MR {\em 44}, 3983.

\item On the distribution of arg $L (s, \chi)$ in the half-plane
  $\sigma > \frac{1}{2}$. {\em Acta Arith. 20} (1972), 155-169. MR
  {\em 48}, 2090. 

\item On the distribution of the values of Dirichlet $L$-series in the
  half-plane $\sigma > \frac{1}{2}$. {\em Indag. Math. 33} (1971),
  222-234. MR {\em 45}, 194. 
\end{enumerate}

\bibitem{key21}
ELLIOTT, P.D.T.A. and HALBERSTAM, H.
\begin{enumerate}
\item Some applications of Bombieri's theorem. \textit{Mathematika
  13} (1966), 196-203. MR {\em 34}, 5788. 

\item A conjecture in prime number theory. \textit{Symposia
  Mathematica} Vol. {\em IV} (INDAM, Rome, 1968/69), 59-72. MR {\em
  43}, 1943.
\end{enumerate}

\bibitem{key22} ERD\"{O}S, P.
\begin{enumerate}
\item Remarks on number theorey. I. (Hungarian. Russian and English
  summaries) \textit{Mat. Lapok 12} (1961), 10-17. MR {\em 26}, 2410. 

\item Some remarks on number theory. III. (Hungarian. Russian and
  English summaries) \textit{Mat. Lapok 13} (1962), 28-38. MR {\em 26},
  2412. 

\item Remarks on number theorey, V. (Hungarian. English summary)
  \textit{Mat. Lapok 17} (1966), 135-155. MR {\em 36}, 133. 
\end{enumerate}

\bibitem{key23} ERD\"{O}S, P. and MIRSKY, L.
\begin{enumerate}
\item The distribution of values of the divisor function
  $d(n)$. \textit{Proc. London Math. Soc}. (3) {\em 2} (1952),
  257-271. MR {\em 14}, p. 249.
\end{enumerate}

\bibitem{key24} ERD\"{O}S, P. R\'ENYI, A.
\begin{enumerate}
\item Some remarks on the large sieve of
  Yu. V. Linnik. \textit{Ann. Univ. Sci. Budapest. E\"otv\"os
    Sect. Math. 11} (1968). 3-13. MR 39, 2718.
\end{enumerate}

\bibitem{key25} ESTERMANN, T.\pageoriginale
\begin{enumerate}
\item On an asymptotic formula due to Titchmarsh. \textit{J. London
  Math. Soc. 6} (1931), 250-251. 
\end{enumerate}

\bibitem{key26} FOGELS, E.K.
\begin{enumerate}
\item Analogue of Brun-Titchmarsh theorem. (Russian. Latvian summary)
  \textit{Latvijas PSR Zin\=atnu Akad. Fiz. Mat. Inst. Raksti 2}
  (1950), 46-58. MR {\em 13}, p. 725.
\end{enumerate}

\bibitem{key27} FORTI. M. and VIOLA, C.
\begin{enumerate}
\item On large sieve type estimated for the Direchlet series
  operator. \textit{Proc. Sympos. Pure Math. 24} (1973), 31-49.

\item Density estimates for the zeros of $L$-functions. \textit{Acta
  Arith. 23} (1973), 379-391. MR {\em 51}, 3081.
\end{enumerate}

\bibitem{key28} GALLAGHER, P.X.
\begin{enumerate}
\item The large sieve. \textit{Mathematika 14} (1967), 14-20. MR {\em
  35}, 5411.

\item Bombieri's mean value theorem. \textit{Mathematika 15} (1968),
  1-6. MR {\em 38}, 5724.

\item A larger sieve. \textit{Acta Arith. 18} (1971), 77-81. MR {\em
  45}, 214.

\item A large sieve density estimate near $\sigma =
  1$. \textit{Invent. Math. 11} (1970), 329-339. MR {\em 43}, 4775.

\item Primes in progressions to prime-power
  modulus. \textit{Invent. Math. 16} (1972), 191-201. MR {\em 46}, 3462. 

\item Sieving by prime powers. \textit{Proc. of the Number Theory
  Conference} (Univ. Colorado, Boulder, Colo., 1972) 95-99. MR
  {\em 50}, 4457.

\item Sieving by prime powers. \textit{Acta Arith. 24}
  (1973/74), 491-497. MR {\em 49}, 2613.

\item Local mean value and density estimates for Dirichlet
  $L$-functions. \textit{Indag. Math. 37} (1975), 259-264. MR {\em
  51}, 5517.
\end{enumerate}

\bibitem{key29} GELFOND, A.O. and LINNIK, Yu. V.
\begin{enumerate}
\item {\em Elementary methods in analytic number theory}. Moscow,
  1962, 272 pp. = Chicago, 1965, xii + 242 pp. MR {\em 32}, 5575 =
  Oxford - New York - Toronto, Ont. 1966,  xi + 232 pp.  MR {\em 34},
  1252. 
\end{enumerate}

\bibitem{key30} GOLDFELD, M.
\begin{enumerate}
\item Artin's conjecture on the average. \textit{Mathematika 15}
  (1968), 223-226. MR {\em 39}, 2711.

\item On the number of primes $p$ for which $p+a$ has a large
  prime factor. \textit{Mathematika 16} (1969), 23-27. MR {\em 39},
  5493.  

\item A\pageoriginale large sieve for a class of non-abelian
  $L$-functions. \textit{Israel J. Math. 14} (1973), 39-49. MR
  {\em 47}, 8489. 

\item A further improvement of the Brun-Titchmarsh
  theorem. \textit{J. London Math. Soc}. (2) {\em 11} (1975), 434-444. 
\end{enumerate}

\bibitem{key31} GREAVES, G.
\begin{enumerate}
\item Large prime factors of binary forms. \textit{J. Number Theory 3}
  (1971), 35-59. MR {\em 42}, 5909.

\item An application of a theorem of Barban-Davenport and
  Halberstam. \textit{Bull. London Math. Soc. 6} (1974), 1-9.
\end{enumerate}

\bibitem{key32} HAL\'ASZ, G.
\begin{enumerate}
\item \"Uber die Mittelwerte multiplilkativer zahlentheoretischer
  Funktionen. \textit{Acta
    Math. Acad. Sci. Hungar. 19} (1968), 365-403. MR {\em 37}, 6254. 
\end{enumerate}

\bibitem{key33} HAL\'ASZ, G. and TUR\'AN, P.
\begin{enumerate}
\item On the distribution of roots of Riemann zeta and allied
  functions. I. \textit{J. Number Theory 1} (1969). 121-137. MR
  {\em 38}, 4422. 

\item On the distribution of roots of Riemann zeta and allied
  functions. II. \textit{Acta
    Math. Acad. Sci. Hungar. 21} (1970), 403-419. MR {\em 42}, 3035. 
\end{enumerate}

\bibitem{key34} HALBERSTAM, H.
\begin{enumerate}
\item Footnote to the Titchmarsh-Linnik divisor
  problem. \textit{Proc. Amer. Math. Soc. 18} (1967), 187-188. MR {\em
    34}, 4221.
\end{enumerate}

\bibitem{key35} HALBERSTAM, H., JURKAT, W. and RICHERT, H.-E.
\begin{enumerate}
\item Un nouveau r\'esultat de la m\'ethode du
  crible. \textit{C.R. Acad. Sci. Paris} S\'er. A-B {\em 264} (1967),
  A920-A923. MR {\em 36}, 6374. 
\end{enumerate}

\bibitem{key36} HALBERSTAM, H. and RICHERT, H.-E.
\begin{enumerate}
\item \textit{Sieve methods}. London-New York - San Francisco, 1974,
  xiv + 364 pp.
\end{enumerate}

\bibitem{key37} HALBERSTAM, H. and ROTH, K.F.
\begin{enumerate}
\item \textit{Sequences}. Vol. I. Oxford, 1966.  xx + 291 pp. MR {\em
  35}, 1565.
\end{enumerate}

\bibitem{key38} HALL, R.R.
\begin{enumerate}
\item Halving an estimate obtained from Selberg's upper bound
  method. \textit{Acta Arith. 25} (1973/74), 347-351. MR {\em 49}, 4949.
\end{enumerate}

\bibitem{key39} HLAWKA, E.
\begin{enumerate}
\item Bemerkungen zum grossen Sieb von
  Linnik. \textit{\"Oterreich. Akad. Wiss. Math. -
    Natur. Kl}. S.-B. II. {\em 178}(1970), 13-18. MR {\em 42}, 224.

\item Zum grossen Sieb von Linnik. \textit{Acta
  Arith. 27}, (1975), 89-100. MR {\em 51}, 3091. 
\end{enumerate}

\bibitem{key40} HOOLEY, C.\pageoriginale
\begin{enumerate}
\item An asymptotic formula in the theory of
  numbers. \textit{Proc. London Math. Soc}. (3) {\em 7}(1957),
  396-413. MR {\em 19}, p. 839. 

\item On the Brun-Titchmarsh theorem. \textit{J. Reine
  Angew. Math. 255} (1972), 60-79. MR {\em 46}, 3463. 

\item On the intervals between numbers that are sums of two
  squares. III. \textit{J. Reine Angew. Math. 267} (1974), 207-218. 

\item On the greatest prime factor of $p+a$. \textit{Mathematika 20}
  (1973), 135-143. MR {\em 50}, 7057. 

\item On the Barban-Davenport-Halberstam theorem. I. {\em J. Reine
  Angew. Math. 274/275} (1975), 206-223. 

\item On the Brun-Titchmarsh theorem. II. \textit{Proc. London
  Math. Soc.} (3) {\em 30} (1975), 114-128. MR {\em 51}, 5531. 

\item On the Barban-Davenpoet-Halberstam
  theorem. II. \textit{J. London Math. Soc.} (2) {\em 9} (1975), 625-636. 

\item On the Barban-Davenport-Halberstam
  theorem. III. \textit{J. London Math. Soc}. (2) {\em 10} (1975),
  249-256.  

\item On the Barban-Davenport-Halberstam
  theorem. IV. \textit{J. London Math. Soc}. (2) {\em 11} (1975),
  399-407. 

\item \textit{Applications of sieve methods to the theory of
  numbers}. Cambridge, 1976, 122 pp.
\end{enumerate}

\bibitem{key41} HUA, L.-K.
\begin{enumerate}
\item \textit{Die Abschatzung von Exponentialsummen und ihre
  Anwendung in der Zahlentheorie}. Enzykl. math. Wiss., I. 2, Heft 13,
  Teil I, Leipzig, 1959. 123 pp. MR {\em 24}, A94. 
\end{enumerate}

\bibitem{key42} HUXLEY, M.N.
\begin{enumerate}
\item The large sieve inequality for algebraic number
  fields. \textit{Mathematika 15} (1968), 178-187. MR {\em 38}. 5737.

\item On the differences of primes in arithmetical
  progressions. \textit{Acta
    Arith. 15} (1968/69), 367-392. MR {\em 39}, 5494. 

\item The large sieve inequality for algebraic number
  fields. II. Means of moments of Hecke
  zeta-functions. \textit{Proc. London
    Math. Soc}. (3) {\em 21} (1970), 108-128, MR {\em 42}, 5944. 

\item The large sieve inequality for algebraic number
  fields. III. Zero-density results. \textit{J. London
    Math. Soc}. (2) {\em 3} (1971), 233-240. MR {\em 43}, 1944. 

\item Irregularity\pageoriginale in sifted
  sequences. \textit{J. Number Theory 4} (1972), 437-454. MR {\em 47},
  180. 

\item On the difference between consecutive
  primes. \textit{Inven. Math. 15} (1972), 164-170. MR {\em 45}, 1856.

\item \textit{The distribution of prime numbers}. Oxford, 1972, x
  + 128 pp.

\item The difference between consecutive
  primes. \textit{Proc. Sympos. Pure
    Math. 24} (1973), 141-145. MR {\em 50}, 9816.

\item Large values of Dirichlet polynomials. \textit{Acta
  Arith. 24} (1973), 329-346. MR {\em 48}, 8406.

\item Small differences between consecutive
  primes. \textit{Mathematika 20} (1973), 229-223. MR {\em 50}, 4509.

\item Large values of Dirichlet polynomials. II. \textit{Acta
  Arith. 27} (1975), 159-169. MR {\em 50}, 12934.

\item Large values of Dirichlet polynomials. III. \textit{Acta
  Arith. 26} (1975), 435-444.
\end{enumerate}

\bibitem{key43} HUXLEY, M.N. and IWANIEC, H.
\begin{enumerate}
\item Bombieri's theorem in short intervals. \textit{Mathematika 22}
  (1975), 188-194.
\end{enumerate}

\bibitem{key44} HUXLEY, M.N. AND JUTILA, M.
\begin{enumerate}
\item Large values of Dirichlet polynomials. IV. \textit{Acta Arith.}
  (to appear). 
\end{enumerate}

\bibitem{key45} INDLEKFER, K.-H.
\begin{enumerate}
\item Eine asymptotische Formel in der
  Zahlentheoric. \textit{Arch. Math. 23} (1972), 619-624. MR {\em 47},
  6629.

\item Scharfe untere Absch\=atzung f\=ur die Anzahlfunktion der
  $B$-Zwillinge. \textit{Acta Arith. 26} (1974/75), 207-212. MR {\em
  50}, 7064.
\end{enumerate}

\bibitem{key46} IWANIEC, H.
\begin{enumerate}
\item On the error term in the linear sieve. \textit{Acta
  Arith. 19} (1971), 1-30. MR {\em 45}, 5104. 

\item Primes of the type $\varphi(x,y)+A$ where $\varphi$ is a
  quadratic form. \textit{Acta Arith. 21} (1972), 203-234. MR {\em
    46}, 3466. 

\item Primes represented by quadratic polynomials in two
  variables. \textit{Acta Arith. 24} (1973/74), 435-459. MR {\em 49},
  7210.  

\item On zeraos of Dirichlet's $L$
  series. \textit{Invent. Math. 23} (1974), 97-104. MR {\em 49}, 8947. 

\item The generalized Hardy-Littlewood's problem involving a quadratic
  polynomial with coprime discriminants. \textit{Acta
    Arith. 27} (1975), 421-46. MR {\em 51}, 3. 

\item The half dimensional sieve. \textit{Acta
  Arith. 29} (1976), 69-95. 

\item The sieve of Eratosthenes-Legendre. (to appear).
\end{enumerate}

\bibitem{key47} JOHNSEN, J.\pageoriginale
\begin{enumerate}
\item On the large sieve method in $GF[q,x]$. \textit{Mathematika 18}
  (1971), 172-184. MR {\em 46}, 1761.
\end{enumerate}

\bibitem{key48} JOSHI, P.T.
\begin{enumerate}
\item The size of $L(1,\chi)$ for real nonprincipal residue characters
  $\chi$ with prime modulus. \textit{J. Number Theory 2} (1970),
  58-73. MR {\em 40}, 4215.
\end{enumerate}

\bibitem{key49} JURKAT, W.B. and RICHERT, H.-E.
\begin{enumerate}
\item An improvement of Selberg's sieve method. I. \textit{Acta
  Arith. 11} (1965), 217-240. MR {\em 34}, 2540.
\end{enumerate}

\bibitem{key50} JUTILA, M.
\begin{enumerate}
\item A statistical density theorem for $L$-functions with
  applications. {\em Acta Arith. 16} (1969/70), 207-216. MR {\em 40},
  5557. 

\item On character sums and class numbers. \textit{J. Number Theory
  5} (1973), 203-214. MR {\em 49}, 230. 

\item On numbers with a large prime factor. \textit{J. Indian
  Math. Soc}. (N.S.) {\em 37} (1973), 43-53. MR {\em 50}, 12936. 

\item A density estimate for $L$-functions with a real
  character. \textit{Ann. Acad. Sci. Fenn}. Ser. AI
  No. 508 (1972), 10 pp. MR {\em 46}, 1725.
 
\item On a density theorem of H.L. Montgomery for
  $L$-functions. \textit{Ann. Acad. Sci. Fenn}, Ser. AI
  No. 520 (1972), 13 pp. MR {\em 48}, 6023. 

\item On mean values of Dirichlet polynomials with real
  characters. \textit{Acta Arith. 27} (1975), 191-198. 

\item On mean values of $L$-functions and short character sums with
  real characters. \textit{Acta Arith. 26} (1975), 405-410. 

\item \textit{On the large values of Dirichlet
  polynomials}. Proceedings of the Colloquium on Number Theory
  (Debrecen, 1974). 

\item \textit{Zero-density estimates for $L$-functions}. Zahlentheorie
  (Tagung, Forachungzinst., Oberwolfach, 1975) (to appear). 

\item Zero-density estimates for $L$-functions. \textit{Acta
  Arith.} (to appear). 

\item Zero-density estimates for $L$-functions. II \textit{Acta
  Arith.} (to appear). 

\item On Linnik's density theorem. (to appear). 
\end{enumerate}

\bibitem{key51} KATAI, I.
\begin{enumerate}
\item On an application of the large sieve: Shifted prime
  numbers, which have no prime divisors from a given arithmetical
  progression. \textit{Acta
    Math. Acad. Sci. Hungar. 21} (1970), 151-173. MR {\em 41}, 8373. 
\end{enumerate}

\bibitem{key52} KLIMOV, N.I.\pageoriginale
\begin{enumerate}
\item Upper estimates of some number theoretical functions. (Russian)
  \textit{Dokl. Akad. Nauk} SSSR {\em 111} (1956), 16-18. MR {\em 19},
  p.~251.

\item Combination of elementary and analytic methods in the theory of
  numbers. (Russian) \textit{Uspehi Mat. Nauk 13} (1958),  no. 3(81),
  145-164. 
\end{enumerate}

\bibitem{key53} KOBAYASHI, I.
\begin{enumerate}
\item A note on the Selberg sieve and the large
  sieve. \textit{Proc. Japan Acad. 49} (1973), 1-5. MR {\em 48}, 3903.
\end{enumerate}

\bibitem{key54} KOLESNIK, G.A.
\begin{enumerate}
\item An estimate for certain trigonometric sums. (Russian)
  \textit{Acta Arith. 25} (1973/74), 7-30. MR {\em 48}, 11002. 
\end{enumerate}

\bibitem{key55} KUBILIUS, I.P.
\begin{enumerate}
\item Probabilistic methods in the theory of numbers. (Russian)
  \textit{Uspehi Mat. Nauk} (N.S.) {\em 11} (1956), no. 2(68), 31-66. MR
  {\em 18}, p.~17 = \textit{Amer. Math. Soc. Transl}. (2) {\em 19}
  (1962), 47-85. MR {\em 24}, A 1266. 

\item \textit{Probabilistic methods in the theory of numbers}. Vilna,
  1959. 164 pp. MR {\em 23}, A 134; 2nd ed. Vilna, 1962, 221 pp. MR
  {\em 26}, 3691 = Providence, R.I., 1964, xvii + 182 pp. MR
  {\em 28}, 3956. 
\end{enumerate}

\bibitem{key56} KUHN, P.
\begin{enumerate}
\item Zur Viggo Brun'schen Siebmethode. I. \textit{Norske
  Vid. Selsk. Forh. Trondhjem 14} (1941). no. 39, 145-148. MR
  {\em 8}, p. 503. 

\item \textit{Neue Abschatzungen auf Grund der Viggo Brunschen
  Siebmethode}. 12. Skand. Mat. Kongr., Lund, 1953, 160-168. MR
  {\em 16}, p. 676. 

\item \textit{\"Uber die Primteiler eines
  Polynoms}. Proc. Internat. Congress Math., Amsterdam,
  1954, {\em 2}, 35-37. 
\end{enumerate}

\bibitem{key57} LAVRIK, A.F.
\begin{enumerate}
\item On the twin prime hypothesis of the theory of primes by the
  method of I.M. Vinogradov. (Russian) \textit{Dokl. Akad. Nauk} SSSR
  {\em 132} (1960), 1013-1015 = \textit{Soviet Math. Dokl. 1} 
  (1960), 700-702. MR {\em 28}, 1183. 

\item An approximate functional equation for the Dirichlet
  $L$-function. (Russian) \textit{Trudy
  Moscov. Mat. Ob\v{s}\v{c}. 18} (1968), 91-104 = \textit{Trans. Moscow
  Math. Soc. 18} (1968), 101-115. MR {\em 38}, 4424. 
\end{enumerate}

\bibitem{key58} LEVIN, B.V.\pageoriginale
\begin{enumerate}
\item Distribution of ``near primes" in polynomial sequence. (Russian)
  \textit{Mat. Sbornik} (N.S.) {\em 61(103)} (1963), 389-407. MR {\em
    30}, 1991. 

\item A one-dimensional sieve. (Russian) \textit{Acta
  Arith. 10} (1964/65), 387-397. MR {\em 31}, 4774. 
\end{enumerate}

\bibitem{key59} LINNIK, Yu. V.
\begin{enumerate}
\item ``The large sieve''. \textit{C.R.(Doklady) Acad. Sci. URSS}
  (N.S.) {\em 30} (1941), 292-294 (290-292). MR {\em 2}, p.~349.

\item On Erd\"os's theorem on the addition of numerical
  sequences. (English. Russian summary)
  \textit{Rec. Math. (Mat.Sbornik)} N.S. {\em 10 (52)} (1942). 67-78. MR
  {\em 4}, p.~131. 

\item A remark on the least quadratic
  non-residue. \textit{C.R. (Doklady) Acad. Sci. URSS} (N.S.)
  {\em 36} (1942), 119-120. (131-132). MR {\em 4}, p.~189. 

\item \textit{The dispersion method in binary additive
  problems}. Leningrad, 1961, 208 pp. MR {\em 25},3920 = Providence,
  R.I., 1963, x + 186 pp. MR {\em 29}, 5804. 
\end{enumerate}

\bibitem{key60} VAN LINT, J.H. AND RICHERT, H.-E.
\begin{enumerate}
\item On primes in arithmetic progressions. \textit{Acta Arith. 11}
  (1965), 209-216. MR {\em 32}, 5613.
\end{enumerate}

\bibitem{key61} LIU, M.-C.
\begin{enumerate}
\item On a result of Davenport and Halberstam. \textit{J. Number
  Theory 1} (1969), 385-389. MR {\em 40}, 2637.
\end{enumerate}

\bibitem{key62} MATHEWS, K.R.
\begin{enumerate}
\item On an inequality of Davenport and Halberstam. \textit{J. London
  Math. Soc}. (2) {\em 4} (1972), 638-642. MR {\em 46}, 1726.

\item On a bilinear form associated with the large
  sieve. \textit{J. London Math. Soc}. (2) {\em 5} (1972), 567-570. MR
  {\em 47}, 6631.

\item Hermitian forms and the large and small sieves. \textit{J. Number
  Theory} {\em 5} (1973), 16-23.
\end{enumerate}

\bibitem{key63} MEIJER, H.G.
\begin{enumerate}
\item Sets of primes with intermediate
  density. \textit{Math. Scand. 34} (1974), 37-43. MR {\em 50}. 7082. 

\item A\pageoriginale problem on prime numbers. \textit{Nieuw
  Arch. Wisk.} (3) 
  {\em 22} (1974), 125-127. MR {\em 50}, 12937. 
\end{enumerate}

\bibitem{key64} MIECH,R.J
\begin{enumerate}
\item Almost primes generated by a polynomial. \textit{Acta
  Arith. 10} (1964/65), 9-30. MR {\em 29}, 1174.

\item A uniform result on almost primes. \textit{Acta
  Arith. 11} (1966), 371-391. MR {\em 34}, 1287. 
\end{enumerate}

\bibitem{key65} MONTGOMERY, H.L.
\begin{enumerate}
\item A note on the large sieve. \textit{J. London
  Math. Soc. 43} (1968), 93-98. MR {\em 37}, 184. 

\item Mean and large values of Dirichlet
  polynomials. \textit{Invent. Math 8} (1969), 334-345. MR {\em 42},
  3029.  

\item Zeros of $L$-functions. \textit{Invent. Math.
  8} (1969), 346-354. MR {\em 40}, 2620. 

\item Primes in arithmetic progressions. \textit{Michigan Math. J.
  17} (1970), 33-39. MR {\em 41}, 1660. 

\item \textit{Topics in multiplicative number theory}. Lecture Notes
  in Math. {\em 227} (1971), ix + 178 pp., Berlin-New York. MR {\em
    49}, 2616. 

\item Hilbert's inequality and the large sieve. \textit{Proc. of the
  Number Theory Conference} (Univ. Colorado, Boulder, Colo., 1972),
  156-161. MR {\em 50}, 4457. 

\item The pair correlation of zeros of the zeta
  function. \textit{Proc. Sympos. Pure Math.} {\em 24} (1973), 181-193. MR
  {\em 49}, 2590. 

\item Linnik's theorem. (unpublished).
\end{enumerate}

\bibitem{key66} MONTGOMERY, H.L. and VAUGHAN, R.C.
\begin{enumerate}
\item Hilbert's inequality. \textit{J. London Math. Soc}. (2)
  {\em 8} (1974), 73-82. MR {\em 49}, 2544.
 
\item The large sieve. \textit{Mathematika 20} (1973), 119-134. 

\item The exceptional set in Goldbach's problem. \textit{Acta
  Arith. 27} (1975), 353-370. 
\end{enumerate}

\bibitem{key67} MORENO, C.J.
\begin{enumerate}
\item The average size of gaps between primes. \textit{Mathematika
  21} (1974), 96-100. 
\end{enumerate}

\bibitem{key68} MOTOHASHI, Y.\pageoriginale
\begin{enumerate}
\item An asymptotic formula in the theory of numbers. \textit{Acta
  Arith. 16} (1969/70), 255-264. MR {\em 42}, 1786. 

\item On the distribution of prime numbers which are of the form
  $x^2+y^2+1$. \textit{Acta Arith. 16} (1969/70), 351-363. MR
  {\em 44}, 5284. 

\item A note on the least prime in an arithmetic progression with a
  prime difference. \textit{Acta Arith. 17} (1970), 283-285. MR
  {\em 42}, 3030. 

\item On the distribution of prime numbers which are of the form
  ``$x^2+y^2+1$''. II. \textit{Acta
  Math. Acad. Sci. Hungar. 22} (1971), 207-210. MR {\em 44}, 5285. 

\item On the mean value theorem for the remainder term in the prime
  number theorem for short arithmetic progressions. \textit{Proc. Japan
    Acad. 47} (1971), 653-657. MR {\em 46}, 3464. 

\item On the distribution of the divisor function in arithmetic
  progressions. \textit{Acta Arith. 22} (1973), 175-199. MR
  {\em 49}, 4952. 

\item On the representations of an integer as a sum of two squares and
  a product of four factors. \textit{J. Math. Soc. Japan
  25} (1973), 475-505. MR {\em 48}, 3899. 

\item On some improvements of the Brun-Titchmarsh
  theorem. \textit{J. Math. Soc. Japan 26} (1974), 306-323. MR
  {\em 49}, 2617. 

\item On some improvements of the Brun-Titchmarsh theorem. II.
  (Japanese) \textit{S\=urikaiseki Kenky\=ujo K\=oky\=uroku 193} 
   (1973), 97-109. 

\item A study of the paper of M.B. Barban and P.P. Vehov entitled ``On
  an extremal problem''. (Japanese) \textit{S\=urikaiseki Kenky\=ujo
    K\=olky\=uroku 222} (1974), 9-50. 

\item On some improvements of the Brun-Titchmarsh
  theorem. III. \textit{J. Math. Soc. Japan 27} (1975), 444-453. 

\item On almost primes in arithmetic
  progressions. \textit{J. Math. Soc. Japan}. (to appear). 

\item On almost primes in arithmetic
  progressions. II. \textit{Proc. Japan Acad. 51} (1975), 545-547.
 
\item On the zero-density theorem of Linnik. (to appear).

\item \textit{On almost primes in arithmetic
  progressions}. Zahlentheorie (Tagung, Math. Forschungzinst.,
  Oberwolfach, 1975). (to appear). 
\end{enumerate}

\bibitem{key69} ORR, R.C.
\begin{enumerate}
\item \textit{Remainder estimates for squarefree integers in
  arithmetic progressions}. Dissertation, Syracuse University, 1969.

\item Remainder\pageoriginale estimates for squarefree integers in
  arithmetic progression. \textit{J. Number Theory 3} (1971), 474-497. MR
  {\em 44}, 3973.
\end{enumerate}

\bibitem{key70} PAN, C.
\begin{enumerate}
\item On the representations of an even integer as the sum of a prime
  and an almost prime. (Chinese) \textit{Acta Math. Sinica 12} (1962),
  95-106= \textit{Chinese Math. -Acta 3} (1963), 101-112. MR {\em 29},
  4727. = \textit{Sci. Sinica 11} (1962), 873-888. (Russian) MR {\em
    27}, 1427.

\item On the representation of even numbers as the sum of the prime
  and a product of not more than 4 primes. (Chinese) \textit{Acta
    Sci. Natur. Univ. Shangtung 1962}, no. 2, 40-62 =
  \textit{Sci. Sinica 12} (1963), 455-473. (Russian) MR {\em 28}, 73.

\item A note on the large sieve method and its applications. (Chinese)
  \textit{Acta Math. Sinica 13} (1963), 262-268 = \textit{Chinese
    Math. - Acta. 4} (1963), 283-290. MR {\em 29}, 82. 

\item A new application of the Yu. V. Linnik large sieve
  method. (Chinese) \textit{Acta Math. Sinica 14} (1964), 597-608 =
  \textit{Chinese Math. - Acta 5} (1964), 642-652. MR
  {\em 30}, 3871 = \textit{Sci. Sinica 13} (1964), 1045-1053. (Russian) MR
  {\em 30}, 3872. 
\end{enumerate}

\bibitem{key71} PORTER, J.W.
\begin{enumerate}
\item \textit{Some divisor problems in the theory of
  numbers}. Ph.D. thesis, Nottingham 1973. (to appear in Acta Arith.).

\item The generalized Titchmarsh-Linnik divisor problem. 
\textit{Proc. London Math. Soc}. (3) {\em 24} (1972), 15-26. MR {\em
  45}, 5098.

\item The mean value of multiplicative functions on the set of numbers
  of the form $p+a$. \textit{J. London Math. Soc}. (2) {\em 10}
  (1975), 447-456. 
\end{enumerate}

\bibitem{key72} RAMACHANDRA, K.
\begin{enumerate}
\item A note on numbers with a large prime factor. \textit{J. London
  Math. Soc}. (2) {\em 1} (1969), 303-306. MR {\em 40}, 118.

\item A note on numbers with a large prime
  factor. II. \textit{J. Indian Math. Soc}. (N.S.) {\em 34}
  (1970), 39-48 (1971). MR {\em 45}, 8616. 

\item On a discrete mean value theorem for
  $\zeta(s)$. \textit{J. Indian Math. Soc}. (N.S.) {\em 36} (1972),
  307-316 (1973). MR {\em 50}, 2095. 

\item On the number of Goldbach numbers in small
  intervals. \textit{J. Indian Math. Soc}. (N.S) {\em 37} (1973), 157-170. MR
  {\em 50}, 12941. 

\item Largest prime factor of the product of $k$-consecutive
  integers. (Russian summary) \textit{Trudy
    Mat. Inst. Steklov}. {\em 132} (1973), 77-81, 264. MR {\em 49},
  2600.  

\item Application\pageoriginale of a theorem of Montgomery and Vaughan
  to the zeta-function. \textit{J. London Math. Soc}. (2) {\em 10}
  (1975), 482-486. 

\item A simple proof of the mean fourth power estimate for $\zeta
  (\frac{1}{2} + it)$ and $ L(\frac{1}{2} + it \chi)$. 
\textit{Ann. Scuola Norm. Sup. Pisa} (4) {\em 1} (1974), 81-97. 

\item Some new density estimates for the zeros of the Riemann
  zeta-function. \textit{Ann. Acad. Sci. Fenn.} Ser. AI
  {\em 1} (1975), 177-182. 

\item On the frequency of Titchmarsh's phenomenon for
  $\zeta(s)$. II. \textit{Acta Math. Acad. Sci. Hungar}. (to appear). 

\item Some problems of analytic number theory. \textit{Acta
  Arith}. (to appear). 
\end{enumerate}

\bibitem{key73} RANKIN, R.A.
\begin{enumerate}
\item The difference between consecutive prime
  numbers. V. \textit{Proc. Edinburgh Math. Soc.} (2) {\em 13}
  (1962/63). 331-332. MR {\em 28}, 3978.
\end{enumerate}

\bibitem{key74} R\'ENTI,A.
\begin{enumerate}
\item On the representation of an even number as the sum of a single
  prime and single almost-prime number. (Russian)
  \textit{Dokl. Akak. Nauk} SSSR (N.S) {\em 56} (1947), 455-458. MR
         {\em 9}, p.~136.

\item On the representation of an even number as the sum of a single
  almost-prime number. (Russian) \textit{Izv. Akad. Nauk}
  SSSR. Ser. Mat. {\em 12} (1948), 57-78. MR {\em 9}, p.~413 =
  \textit{Amer. Math. Soc. Transl}. (2) {\em 19} (1962), 299-231. MR
         {\em 24}, A 1264. 

\item Generalization of the ``large sieve" of
  Yu. V. Linnik. \textit{Math. Centrum Amsterdam 1948}, 5 pp. 

\item Un noveau th\'eor\`eme concernant les functions ind\'ependantes
  et ses applications \`a la th\'eorie des numbers. \textit{J. Math. Pures
    Appl}. (9) {\em 28} (1949), 137-149, MR {\em 11}, p.~161. 

\item On a theorem of the theory of probability and its application in
  number theory. (Russian. Czech summary) \textit{\v{C}asopis
    P\v{e}st. Mat. Eys}. {\em 74} (1949), 167-175 (1950). MR {\em 12},
  p.~590. 

\item \textit{Probability methods in number
  theory}. Publ. Math. Collectae Budapest {\em 1} (1949), no. 21, 9
  pp. MR {\em 12}, p.~161. 

\item On the large sieve of Yu. V. Linnik. \textit{Compositio
  Math}. {\em 8} (1951), 68-75. MR {\em 11} (1950). p.~581. 

\item Sur un th\'eor\`em g\'en\'eral de
  probabilit\'e. \textit{Ann. Inst. Fourier 1} (1949), 43-52 (1950),
  MR {\em 14}, p.~886. 

\item Probability methods in number theory. (Chinese)
  \textit{Advancement in Math. 4} (1958), 465-510. MR {\em 20}, 4535. 

\item On\pageoriginale the probabilistic generalization of the large sieve of
  Linnik. (Hungarian and Russian summaries) \textit{Magyar
    Tud. Akad. Mat. Kutat\'o Int. K\"ozl}. {\em 3} (1958), 199-206. MR
  {\em 22}, B 1937. 

\item Probabilistic methods in number
  theory. \textit{Proc. Int. Congr. Math. 1958}, 529-539. MR {\em 22},
  9478. 

\item New version of the probabilistic generalization of the large
  sieve. (Russian summary) \textit{Acta
    Math. Acad. Sci. Hungar}. {\em 10} (1959), 217-226. MR {\em 22},
  1938.  
\end{enumerate}

\bibitem{key75} RICHERT, H.-E.
\begin{enumerate}
\item Selberg's sieve with weights. \textit{Mathematika} {\em 16}
  (1969), 1-22. MR {\em 40}, 119.

\item Selberg's sieve with weights. \textit{Proc. Sympos. Pure
  Math. 20} (1971), 287-310. MR {\em 47}, 3286, 6632. 
\end{enumerate}

\bibitem{key76} RIEGER, G.J.
\begin{enumerate}
\item Zum Sieb von Linnik. \textit{Arch. Math. 11} (1960), 14-22. MR
  {\em 27}, 3595. 

\item Das grosse Sieb von Linnik f\"ur algebraische
  Zahlen. \textit{Arch. Math. 12} (1961), 184-187. MR {\em 24}, A 3152. 

\item \"Uber die Folge der Zahlen der Gestalt
  $p_1+p_2$. \textit{Arch. Math}. {\em 15} (1964), 33-41. MR {\em 28},
  3023. 
\end{enumerate}

\bibitem{key77} RODRIQUEZ, G.
\begin{enumerate}
\item Sul problema dei divisori di Titchmarsh. (English summary)
  \textit{Boll. Un. Mat. Ital}. (3) {\em 20} (1965), 358-366. MR {\em
    33}, 5574. 
\end{enumerate}

\bibitem{key78} ROSS, P.M.
\begin{enumerate}
\item On Chen's theorem that each large even number has the form
  $p_1+p_2$ or $p_1+p_2p_3$. \textit{J. London Math. Soc}. (2) {\em
  10} (1975), 500-506.
\end{enumerate}

\bibitem{key79} ROSSER, J.B. and SCHOENFELD, L.
\begin{enumerate}
\item Approximate formulas for some functions of prime
  numbers. \textit{Illinois J. Math}. {\em 6} (1962), 64-94. MR {\em
    25}, 1139.
\end{enumerate}

\bibitem{key80} ROTH. K.F
\begin{enumerate}
\item Remark concerining integer sequences. \textit{Acta Arith. 9}
  (1964). 257-260. MR {\em 29}, 5806. 

\item On the  large sieves of Linnik and R\'{e}nyi
  \textit{Mathematika 12} (1965), 1-9. MR {\em 33}, 5589. 

\item Irregularities\pageoriginale of sequences relative to arithmetic
  progressions 
  \textit{Math. Ann.} {\em 169} (1967), 1-25. MR  {\em 34}, 4235.  

\item \textit{The large sieve}. Inaugural Lecture, 23 january
  1968, Imp. Coll. London {\em 1968}, 9 pp. MR {\em 38}, 5740. 

\item Irregulararities of sequences relative to arithmatic
  progressions. III. \textit{J. Number Theory 2} (1970),
  125-142. MR {\em 41}, 6180. 
\end{enumerate}

\bibitem{key81} SAMANDAROV, A.G. 
\begin{enumerate}
\item The large sieve the algebraic fields. (Russain)
  \textit{Mat. Zametki 2} (1967), 673-680. MR {\em 36}, 6379. 
\end{enumerate}

\bibitem{key82} SCHAAL, W. 
\begin{enumerate}
\item On the large sieve method in algebraic number
  fields. \textit{J. Number Theory 2} (1970), 249-270. MR {\em
    42}. 7826. 
\end{enumerate}

\bibitem{key83} SCOURFIELD, E.J.
\begin{enumerate}
\item An asymoptomic formula for the property $(n, f(n))=1$ for a
  class of multiplicative funcitons. \textit{Acta Arith. 29},
  no. 4. (to appear). 
\end{enumerate}

\bibitem{key84} SELBERG, A,.
\begin{enumerate}
\item On an elementary method in the theory of primes. \textit{Norske
  Vid. Selsk. Forh., Trondhjem 19} (1947), no. 18, 64-67. MR {\em 9},
  p.~271. 

\item \textit{Twin prime problem}.  Manuscript 16 pp. + Appendix 7
  pp. (unpublished)
 
\item On elementary method in prime number-theory and their
  limitations {\em 11. Skand. Mat. Kongr., Trondheim 1949}, 13-22.  MR
  {\em 14}, p.~726.
 
\item The general sieve method and its place in  prime number
  theory. \textit{Proc. Internat. Congress Math}., Cambridge, Mass
  {\em 1} (1950), 286-292. MR  {\em 13}, p.~438. 

\item Sieve methods. \textit{Proc, Sympos, Pure Math. 20} (1971),
  311-351. MR {\em 47}, 3286.  

\item Remarks on sieves. \textit{Proc. of the Number Theory
  Conference} (Univ. Colorado, Boulder, Colo., 1972) 205-216. MR
  {\em 50}. 4457.  
\end{enumerate}

\bibitem{key85} SIEBERT, H. and WOLKE, D. 
\begin{enumerate}
\item \"{U}ber einige Analoga zum Bombierischen
  Primazahlsatz. \textit{Math. Z. 122} (1971), 327-341.
\end{enumerate}

\bibitem{key86} SOKOLOVSKIJ, A.V.\pageoriginale
\begin{enumerate}
\item On the large sieve. (Russian) \textit{Acta Arith.} {\em 25}
  (1974), 301-306. 
\end{enumerate}

\bibitem{key87} STEPANOV, B.V. 
\begin{enumerate}
\item On the mean value of the $k^{\text{th}}$ power of the number of classes
  for an imaginary quadratic field. (Russain) \textit{Dokl. Akad.
    Nauk} SSSR {\em 124} (1959). 984-986. MR {\em 21}, 4948.  
\end{enumerate}

\bibitem{key88} TUR\'AN, P.
\begin{enumerate}
\item \"{U}ber die Primzahlen der arithmetischen
  Progression. \textit{Acta Sci. Math.} (Szaged) {\em 8} (1937), 226-235,
\end{enumerate}

\bibitem{key89} UCHIYAMA, S. 
\begin{enumerate} 
\item On the difference between consecutive prime
  numbers. \textit{Acta Arith. 27} (1975), 153-157. MR {\em 51}, 3085.
\end{enumerate}

\bibitem{key90} UCHIYAMA, M. AND UCHIYAMA, S.
\begin{enumerate}
\item On the representation of large even integers as sums of a prime
  and an almost prime. \textit{Proc. Japan Acad. 40} (1964),
  150-154. MR {\em 29}, 2234. 
\end{enumerate}

\bibitem{key91} VAUGHAN, R.C.
\begin{enumerate}
\item Some applications of Montogomery's sieve, \textit{J. Number
  Theory 5} (1973), 64-69. MR {\em 49}, 7222. 

\item On Goldbach's problem. \textit{Acta Arith}. {\em 22} (1972),
  21-48. MR {\em 48}, 6045. 

\item A remark on the divisor function $d(n)$. \textit{Glasgow
  Math. J.} {\em 14} (1973), 54-55. MR {\em 47}, 169.  

\item On the number of solutions of the equation $p=a+n_1\ldots n_k$
  with $a<p\leq x$ \textit{J. London Math. Soc}. (2) {\em 6} (1972),
  43-45. MR {\em 46}, 8994 a.  

\item On the number of solutions of the equation $p+ n_{1}\ldots
  n_k=N$ {\em J. London Math. Soc.} (2) {\em 6} (1973), 326-328. MR
  {\em 46}, 8994 b.  

\item Mean value theorems in prime number theory. \textit{J. London
  Math. Soc}. (2) {\em 10} (1975), 153-162. 
\end{enumerate}

\bibitem{key92} VINOGRADOV, A.I.
\begin{enumerate}
\item The density hypothesis for Dirichelet $L$-series. (Russain)
  {\em lzv. Akad. Nauk} SSSR Ser. Mat. {\em 29} (1965), 903-934. MR
  {\em 33},
  5579; \textit{Corrigendum: ibid 30} (1966), 719-720. MR {\em 33}, 2607
  = \textit{Amer. Math. Soc. Transl.} (2) {\em 82} (1969) 9-46. 
\end{enumerate}

\bibitem{key93} WANG, Y.\pageoriginale 
\begin{enumerate}
\item A note on some properties of the arithmatical functions
  $\phi(n)$, $\sigma(n)$ and $d(n)$. (Chinese. English summary)
  \textit{Acta Math. Sinica 8} (1958), 
  $1-11$ = {\em Chinese Math. - Acta 8} (1966), 585-598 = {\em
    Amer. Math. Soc. Transl.} (2) {\em 37} (1964),
  143-156. MR {\em 20}, 4533. 

\item On the representation of large integer as a sum of a prime and an
  almost prime. {\em Sci. Sinica 11} (1962), 1033-1054. MR {\em 27}, 1424. 
\end{enumerate}

\bibitem{key94} WANG, Y., HSIEH , S. and Yu. K. 
\begin{enumerate}
\item Remarks on the difference of consecutive
  primes. \textit{Sci. Sinica 14} (1965), 786-788. MR {\em 32}, 5617.
\end{enumerate}

\bibitem{key95} WARD, D.R.
\begin{enumerate}
\item Some series involving Euler's function \textit{J. London
  Math. Soc. 2} (1927), 210-214.
\end{enumerate}

\bibitem{key96} WARLIMONT, R.
\begin{enumerate}
\item On squarefree members in arithmatic
  progressions. \textit{Monatsh. Math. 73} (1969), 433-448. MR
  {\em 41}, 3430. 

\item \"{U}ber die kleinsten quadratfreien Zahlen in arithmetischen
  Progression mit primen Differenzen. \textit{J. Reine Angew. Math.
  253} (1972), 19-23. MR {\em 46}, 1728.	 
\end{enumerate}

\bibitem{key97} WILSON, R.J.
\begin{enumerate}
\item The large sieve in algebraic number fields. \textit{Mathematika
  16} (1969), 189-204. MR {\em 41}, 8374.
\end{enumerate}

\bibitem{key98} WOLKE, D.
\begin{enumerate}
\item Farrey-Br\={u}che mit primem Nenner und das grosse
  Sieb. \textit{Math. Z. 114} (1970), 145-158. MR {\em 41}, 5228.

\item On the large sieve with primes. \textit{Acta
  Math. Ascad. Sci. Hungar. 22} (1971), 239-247. MR {\em 45}, 215.

\item Multiplikative Funktionen auf schnell wachsenden
  Folgen. \textit{J. Reine Angew. Math. 251} (1971), 54-67. MR {\em 44},
  6629.

\item Polynominal values with small prime divisors. \textit{Acta Arith.
19} (1971), 327-333. MR {\em 45}, 1871.

\item \"{U}ber das summatorische Verhalten zalentheoretischer
  Funktionen. \textit{Math. Ann. 194} (1971), 147-166. MR {\em 46}, 3465.

\item \"{U}ber\pageoriginale die mittlere verteilung der Werte
  zahlentheoretischer 
  Funktionen auf Restklassen. I.{\em  Math. Ann. 202} (1973), 1-25. MR
  {\em 48}, 6030.

\item \"{U}ber die mittlere Verteilung der Werte zahlentheoretischer
  Funktionen auf Restklassem. II. \textit{Math. Ann. 204} (1973),
  145-153. MR {\em 50}, 12940.

\item Eine Bemerkung zum Sieb des
  Erastothnes. \textit{Monatsh. Math. 78} (1974), 264-266.

\item Das Selbegasche Sieb f\"{u}r zahletheoretische Funktionen,
  I. \textit{Arch. Math. 24} (1973), 632-639. MR {\em 49}, 2618. 

\item A lower bound for the large sieve
  inequality. \textit{Bull. London Math. Soc. 6} (1974), 315-318. MR
  {\em 50}, 7065.

\item Grosse Differenzen zwischen aufeinanderfolgenden Primizahlen,
  \textit{Math. Ann. 218} (1975), 269-271. 
\end{enumerate}
\end{thebibliography}
