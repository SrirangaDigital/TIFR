
\chapter{Selberg's Sieve}\label{chap9}

NOW\pageoriginale WE turn to the small sieves. The most elegant
version of a small sieve is due to Selberg. In this chapter we present
its simplest version with a view to clarifying the main ideas
involved. Also, later in the next chapter we importance in the proof
of the remarkable theorem of Chen. 

This sieve method can be considered as concerning the question of
finding bounding for  
\begin{equation*}
S (\mathscr{A}, \mathfrak{f}, z).\tag{9.1}\label{eq9.1}
\end{equation*}
the number of elements in a (finite) sequence, depending on several
parameters, 
\begin{equation*}
\mathscr{A}: = \{a: \cdots \}, a \in \mathbb{Z}.\tag{9.2}\label{eq9.2}
\end{equation*}
of (not necessarily distinct and not necessarily positive) integers,
which are not divisible by any prime number $< z$, 
\begin{equation*}
z \geq  (z \in \mathbb{R}),\tag{9.3}\label{eq9.3}
\end{equation*}
belonging to a set of primes 
\begin{equation*}
\mathfrak{f}: = \{ p : \ldots \}.\tag{9.4}\label{eq9.4}
\end{equation*}

Introducing 
\begin{equation*}
P(z): =\prod_{\substack{p<z\\ p \in
    \mathfrak{p}}}p,\tag{9.5}\label{eq9.5} 
\end{equation*}
we can restate this question as that of estimating 
\begin{equation*}
S(\mathscr{A},\mathfrak{p},z) =| \{ a: a \in \mathscr{A},(a, P(z))=1\}
|\tag{9.6}\label{eq9.6} 
\end{equation*}

The\pageoriginale required estimates would be naturally dependent on the various
parameters describing $\mathscr{A}, \mathfrak{p}$, and also on
$z$. However, we would be interested in bounds, for \eqref{eq9.6}, which do
not involve the special features of these defining arguments
($\mathscr{A}$ and $\mathfrak{p}$). To make  this
remark clearer, we introduce the following notation.  

Let
\begin{equation*}
\mathscr{A}_d: = \{ a: a \in \mathscr{A}, a \equiv  0 \mod d \}\quad for\quad
d \in \mathbb{N}.\tag{9.7}\label{eq9.7} 
\end{equation*}

First we choose a convenient approximation $X$ to $| \mathscr{A}|$,
requiring 
\begin{equation*}
X > 1.\tag{9.8}\label{eq9.8}
\end{equation*}
we write for the remainder 
\begin{equation*}
R_1: = | \mathscr{A}| -X.\tag{9.9}\label{eq9.9}
\end{equation*}

Next, for each prime $p \in \mathfrak{p}$, we choose $\omega(p) (\in
\mathbb{R})$ so that $\dfrac{\omega(p)}{p}X$ is close to
$|\mathscr{A}_p|$, and set  
\begin{equation*}
R_p: |\mathscr{A}_p|-\frac{\omega(p)}{p}X, \forall p \in
\mathfrak{p}.\tag{9.10}\label{eq9.10} 
\end{equation*}

Further, denoting by 
\begin{equation*}
\overline{\mathfrak{p}}\tag{9.11}\label{eq9.11}
\end{equation*}
the complement of $\mathfrak{p}$ with respect to the set of all
primes, we also put  
\begin{equation*}
\omega(p) = \qquad \forall p \in
\overline{\mathfrak{p}}.\tag{9.12}\label{eq9.12} 
\end{equation*}
(This is consistent with our interest being only with regard to the
distribution of numbers of $\mathscr{A}$ in the residue class $0$
modulo primes from $\mathfrak{p}$.) If we now define  
\begin{equation*}
\omega(d): = \prod_{p|d} \omega(p) \qquad \forall d \in \mathbb{N}
\text{ with }\mu (d) \neq 0, \omega(1):=1\tag{9.13}\label{eq9.13} 
\end{equation*}
(and $\omega(d) = 0$ if  $\mu (d)= 0$), we see that 
\begin{equation*}
\omega \in \mathscr{M} \tag{9.14}\label{eq9.14}
\end{equation*}
and\pageoriginale also that the definition
\begin{equation*}
R_d : = | \mathscr{A}|- \frac{\omega(d)}{d} X \; \forall d \in \mathbb{N}
\text{ with }\mu (d) \neq 0,\tag{9.15}\label{eq9.15}
\end{equation*}
is consistent with \eqref{eq9.9} and \eqref{eq9.10}. (Note that $\omega$ may
depend on both $\mathscr{A}$ and $\mathfrak{p}$. Now we can
elaborate a little on our remarks made subsequent to \eqref{eq9.6}. The
estimates for \eqref{eq9.6} are allowed to depend on $X$, $\omega$ (and
consequently on $R$), but not on the particular structures of
$\mathscr{A}$ and $\mathfrak{p}$ (apart from those which
yields information towards the most appropriate choices for $X$ and
$\omega$ as introduced above).  

For the purposes of the method we also require to fulfill the
condition  
\begin{equation*}
(\Omega_1 ) \qquad  0 \leq \frac{\omega(p)}{p} \leq 1-
  \frac{1}{A_1} \text{  with some constant } A_1 \geq 1 \qquad 
  \tag*{$(9.16_a)$}\label{eq9.16}  
\end{equation*}
or equivalently 
\begin{equation*} 
(\Omega_1 ) \qquad\quad   1 \leq \frac{1}{1- \frac{\omega(p)}{p}} \leq
  A_1 \text{ with  some constant } A_1 \geq 1 . \quad \qquad
  \tag*{$(9.16_b)$}\label{eq9.16b}  
\end{equation*}

We introduce further 
\begin{equation*}
g(d): = \prod_{p|d} \frac{\omega(p)}{p- \omega(p)} \; \forall d \in
\mathbb{N}\quad \text{with}\quad \mu (d) \neq 0,\tag{9.17}\label{eq9.17} 
\end{equation*}
which is well-defined (because of $(\Omega_1)$). The product $g(d)$
reminds us of the function occurring in Theorem \ref{chap7-thm7.1}, but the
advantage here is that $\omega (p)$ need no longer be
integer-valued. On the other hand, when compared with Theorem
\ref{chap7-thm7.2}, 
the condition $(\Omega_1)$ already prevents us from dealing with `too
large a sieve' 

By \eqref{eq9.12} we see that 
\begin{equation*}
g(d) = 0\quad \text{if}\quad ( d,\bar{\mathfrak{p}}) \neq
1.\tag{9.18}\label{eq9.18} 
\end{equation*}
and also (from \eqref{eq9.13}) that 
\begin{equation*}
g(d) = 0 \Longleftrightarrow \omega(d) = 0.\tag{9.19}\label{eq9.19} 
\end{equation*}
(Here and in what follows $(d,\bar{\mathfrak{p}})=1$ means that no $p
\in \bar{\mathfrak{p}}$ divides $d$.) 

Finally,\pageoriginale we put
\begin{gather*}
W(z): = \prod_{p<z}(1- \frac{\omega(p)}{p}).\tag{9.20}\label{eq9.20}\\
G(z) : = \sum_{ d < z } \mu^2(d) g(d).\tag{9.21}\label{eq9.21}
\end{gather*}
and more generally 
\begin{equation*}
G_k(x) : = \sum_{\substack{d < x \\ {(d,k)=1}}} \mu^2(d) g(d), \;\;  0< x
\in \mathbb{R}, k \in \mathbb{N}.\tag{9.22}\label{eq9.22} 
\end{equation*}

In view of \eqref{eq9.6} we could start with the identity (cf. \eqref{eq1.20})
\begin{equation*}
S (\mathscr{A}, \mathfrak{p}, z) = \sum_{a \in \mathscr{A}} \sum_{d|(a,
  P(z))} \mu(d);\tag{9.23}\label{eq9.23} 
\end{equation*}
in fact, this is the sieve formula of Eratosthenes-Legendre. Selberg's
sieve, for obtaining an upper bound for $S(\mathscr{A},
\mathfrak{p},z)$, consists in the introduction of 
arbitrary real numbers $\lambda_d$ with the only condition  
\begin{equation*}
\lambda_1 = 1.\tag{9.24}\label{eq9.24}
\end{equation*}
which already implies that 
\begin{equation*}
S (\mathscr{A}, \mathfrak{p}, z)\leq  \sum_{a \in
  \mathscr{A}}(\sum_{\substack{d | x \\ {d|P(z)}}} \lambda_d)^2 =
\sum_{\substack{d_{\nu} |P(z) x \\ {\nu = 1,2}}} \lambda_{d_1}
\lambda_{d_{2}} \sum_{\substack{a \in \mathscr{A} \\ {a= 0 \mod
      [d_1,d_2]}}}\tag{9.25}\label{eq9.25} 
\end{equation*}  

Hence, by \eqref{eq9.7} and \eqref{eq9.15},
{\fontsize{10pt}{12pt}\selectfont
\begin{equation*}
\begin{cases}
S (\mathscr{A}, \mathfrak{p}, z)\leq X
\sum\limits_{\substack{d_{\nu}| P(z) \\ {\nu  = 1,2 }}}
\lambda_{d_{1}}\lambda_{d_{2}}
\frac{\omega([d_1,d_2])}{[d_1,d_2]}+\sum\limits_{\substack{d_{\nu}|
    P(z) \\ {\nu  = 1,2 [d_1,d_2]}}}|\lambda_{d_{1}}\lambda_{d_{2}}
R_{[d_1,d_2]}|= 
 X  \sum\limits_1 + \sum \limits_2 
\end{cases}\tag{9.26}\label{eq9.26}
\end{equation*}}\relax
say. With a view to keep $\sum_2$ small one takes in this method 
\begin{equation*}
\lambda_d = 0\quad \text{for}\quad d \geq z \tag{9.27}\label{eq9.27}
\end{equation*}
and then the remaining $\lambda_d's(2 \leq d < z d|P(z))$ are chosen
so as to minimize $\sum_1$ 

This\pageoriginale leads to the choice 
\begin{equation*}
\lambda_d = \mu(d) \prod_{p|d} \frac{p}{p- \omega(p)} \frac{G_d
  (\frac{z}{d})}{G(z)}.\tag{9.28}\label{eq9.28} 
\end{equation*}

Note here that \eqref{eq9.28} includes both \eqref{eq9.24} and
\eqref{eq9.27}. Next, it can be shown by the argument of
\eqref{eq3.26} that   
\begin{equation*}
|\lambda_d| \leq 1.\tag{9.29}\label{eq9.29}
\end{equation*}

Now, with the choice \eqref{eq9.28}, one obtains 
\begin{equation*}
\sum_1 = \frac{1}{G(z)}\tag{9.30}\label{eq9.30}
\end{equation*}
and further \eqref{eq9.27} and \eqref{eq9.29} give 
\begin{equation*}
\sum_2 \leq \sum_{\substack{d_{\nu}<z \\ {d_{\nu}|P(z) } \\ \nu = 1,2}}|
R_{[d_1,d_2]}|.\tag{9.31}\label{eq9.31} 
\end{equation*}

Here the numbers $d=[d_1,d_2]$ are $< z^2$ and divide $P(z)$. Since
$d$ is square free, the number of terms with the same $d$ is atmost  
\begin{equation*}
|\{d_1,d_2  : [d_1,d_2] = d \} = 3^{\nu (d)}\tag{9.32}\label{eq9.32}
\end{equation*}

From \eqref{eq9.26}, \eqref{eq9.30} and \eqref{eq9.31} we now obtain
(in view of \eqref{eq9.32}) 
 
\setcounter{section}{9}
\setcounter{theorem}{0}
\begin{theorem}\label{chap9-thm9.1}%The 9.1
$(\Omega_1)^1$: We have, in the above notation \footnote{By this
    notation, which is also employed in a similar way later, we mean
    that the subsequent statement is valid subject to the conditions
    in parentheses.}. 
\begin{equation*}
\begin{cases}
S (\mathscr{A}, \mathfrak{p}, z)\leq \frac{X}{G(z)} +
\sum\limits_{\substack{d_{\nu} z \\ {d_{\nu}|P(z)  }\\{\nu=1,2}}}
|R_{[d_1,d_2]}| \leq \frac{X}{G(z)}\\
 +\sum \limits_{\substack{d< z^2
    \\ {d|P(z) }}}3^{\nu (d)}|R_d|\leq \frac{X}{G(z)} + \sum
\limits_{\substack{d <  z^2 \\ {(d,,\bar{\mathfrak{p}}) =1 }}}
\mu^2(d)3^{\nu (d)}|R_d|. 
\end{cases}\tag{9.33}\label{eq9.33} 
\end{equation*}
\end{theorem}

Now\pageoriginale we give two important special cases of Theorem
\ref{chap9-thm9.1}. 

\begin{theorem}\label{chap9-thm9.2}%the 9.2
Suppose that 
\begin{equation*}
\omega(d) = 1 \text{~ and~ } |R_d| \leq 1, \text{~ if~ } \mu (d) \neq 0
\text{~ and~ } ( d, \bar{\mathfrak{p}} ) = 1.\tag{9.34}\label{eq9.34} 
\end{equation*}

Then
\begin{equation*}
S(\mathscr{A},\mathfrak{p},z) \leq \frac{X}{(\prod
  \limits_{\substack{p< z \\ {p \in \mathfrak{p}}}}(1-\frac{1}{p}))
  \log z}+ z^2.\tag{9.35}\label{eq9.35} 
\end{equation*}
\end{theorem}

\begin{proof}%pro
From our assumption on $\omega$ in \eqref{eq9.34} it follows that the
condition $(\Omega_1)$ is fulfilled and further  
\begin{equation*}
G(z) = \sum_{\substack{d< z \\ {(d,k) =1}}} \frac{\mu^2
  (d)}{\varphi(d)},\tag{9.36}\label{eq9.36} 
\end{equation*}
where
\begin{equation*}
k = \prod \limits_{\substack{p< z \\ {p \notin \mathfrak{p}}}}
p.\tag{9.37}\label{eq9.37} 
\end{equation*}

Therefore, by \eqref{eq3.26}, we have 
\begin{equation*}
G(z) \geq \frac{\varphi(k)}{k} \log z = \prod\limits_{\substack{p< z
    \\ {p \notin \mathfrak{p}}}}(1- \frac{1}{p}) \log
z,\tag{9.38}\label{eq9.38}  
\end{equation*}
so that first inequality of \eqref{eq9.33} yields \eqref{eq9.35}
(since $|R_d| \leq 1$ by \eqref{eq9.34}). 

Let us set 
\begin{equation*}
\mathfrak{p}_K= \{ p : p \nmid K \}, \; K \in
\mathbb{Z}\tag{9.39}\label{eq9.39} 
\end{equation*}
\end{proof}

\begin{theorem}\label{chap9-thm9.3}%the 9.3
Let $K(\neq 0)$ be an even integer and suppose that 
\begin{equation*}
\omega(p) = \frac{p}{p-1} \text{ for } p \in
\mathfrak{p}_K.\tag{9.40}\label{eq9.40} 
\end{equation*}

Then, we have 
\begin{equation*}
S(\mathscr{A},\mathfrak{p}_K,z) \leq \mathfrak{S} (K) \frac{X}{\log
  z}(1+O(\frac{1}{\log z}))+ \prod \limits_{\substack{p< z^2 \\ (d,K)
    =1}} \mu^2 (d) 3^{\nu d}|R_d|,\tag{9.41}\label{eq9.41} 
\end{equation*}
where 
\begin{equation*}
\mathfrak{S}(K) = 2 \prod_{p>2}(1-\frac{1}{(p-1)^2})  \prod_{2<p|K}
\frac{p-1}{p-2}\tag{9.42}\label{eq9.42} 
\end{equation*}
\end{theorem}

\begin{proof}%pro
We use the last bound given in \eqref{eq9.33}. Clearly, we need only
show that  
\begin{equation*}
\frac{1}{G(z)} \leq 2 \prod_{p>2}(1-\frac{1}{(p-1)^2})  \prod_{2<p|K}
\frac{p-1}{p-2} \frac{1}{\log z}(1+0 (\frac{1}{\log
  z})),\tag{9.43}\label{eq9.43} 
\end{equation*}\pageoriginale
since $(\Omega_1)$ is satisfied with $A_1 = 2$ in view of $2|K$. we
note that  
\begin{equation*}
g(p) = \frac{1}{p-2}=\frac{1}{p-1}(1+\frac{1}{p-2})=
\frac{1}{\varphi(p)}(1+g(p)) \text{ for } p\not\mid
K,\tag{9.44}\label{eq9.44}  
\end{equation*}
and so (cf. \eqref{eq9.17})
\begin{equation*}
g(d) = \frac{1}{\varphi(p)} \sum_{l|d}\mu^2 (l) g(l) \text{~ if~ } (d,K)
= 1.\tag{9.45}\label{eq9.45} 
\end{equation*}

Then, by \eqref{eq3.26}.
{\fontsize{10pt}{12pt}\selectfont
\begin{equation*}
G(z)= \sum \limits_{\substack{\ell< z \\ (\ell,K) =1}}
\frac{\mu^2(l)g(l)}{\varphi (\ell)} \sum \limits_{\substack{m<
    \frac{z}{\ell} \\ (m,lK) =1}} \frac{\mu^2(m)}{\varphi (m)}\geq
\prod_{p|K}(1-\frac{1}{p}) \prod \limits^{\infty}_{\substack{\ell=1
    \\ (\ell,K) =1}} \frac{\mu^2(\ell)g(\ell)}{ (l)} \log
(\frac{z}{\ell})\tag{9.46}\label{eq9.46} 
\end{equation*}}\relax
on observing that for $\ell > z$ the lower bound log
$(\dfrac{z}{l})$, for the (empty) inner sum, is negative, Further, by
\eqref{eq9.44} and \eqref{eq1.18}. 
\begin{equation*}
\sum^{\infty}_{\substack{l=1 \\ (l,K) =1}} \frac{\mu^2(\ell)g(\ell)}{
  (\ell)} = \prod_{p \not\mid K}(1+
\frac{1}{p(p-2)})\tag{9.47}\label{eq9.47} 
\end{equation*}
and 
\begin{equation*}
\begin{cases}
\prod\limits^{\infty}_{\substack{\ell=1 \\ (\ell,K) =1}}
\frac{\mu^2(l)g(l)}{ (\ell)} \log \ell =
\prod\limits^{\infty}_{\substack{\ell=1 \\ (\ell,K) =1}} 
\frac{\mu^2(l)g(\ell)}{ (=\ell)} \sum_{p|\ell} \log p\\
= 
\sum_{ p \not\mid K} \frac{\log p}{p(p-2)} \frac{1}{1+(p(p-2))^{-1}}
\prod\limits_{p' \nmid K}(1+ \frac{1}{p'(p'-2)}). 
\end{cases}\tag{9.48}\label{eq9.48}
\end{equation*}

Using \eqref{eq9.47} and \eqref{eq9.48} in \eqref{eq9.46} we obtain 
\begin{equation*}
G(z) \geq \prod_{p|K} (1- \frac{1}{p})\prod_{p \nmid K} (1+
\frac{1}{p(p-2)})  \{ \log z - \sum_p \frac{\log p}{(p-1)^2}
\},\tag{9.49}\label{eq9.49}   
\end{equation*}
which upholds \eqref{eq9.43}. Thus the theorem is completely proved. 
\end{proof}

\begin{center}
\textbf{NOTES}
\end{center}

Selberg's sieve occurs for the first time in Selberg \cite{key1} (cf. Selberg
\cite{key3}, \cite{key4}, \cite{key5}, \cite{key6}). 

For the content of this chapter we refer the reader to Halberstam and
Richert\pageoriginale \cite{key1} (Chapter \ref{chap3}). 

\eqref{eq9.29}:~ The details leading to \eqref{eq9.29} are the
following. In view of \eqref{eq9.28} consider (for only squarefree $d$'s) 
\begin{equation*}
\begin{cases}
\prod_{p|d} \frac{p}{p-\omega(p)}G_d(\frac{z}{d}) = ( \prod_{p|d}(1+
g(p))( \sum \limits_{\substack{d_1<z/d \\ (d_1,d) =1}}
\mu^2(d_1)g(d_1))\\
 = 
(\sum\limits_{d_2|d}\mu^2(d_2)g(d_2))(( \sum
\limits_{\substack{d_1<z/d \\ (d_1,d) =1}}\mu^2(d_1)g(d_1)).
\end{cases}\tag{9.50}\label{eq9.50}
\end{equation*}

Now multiplying out the last expression and comparing with $G(z)$ we
obtain \eqref{eq9.29} 

\ref{chap9-thm9.1}:~ cf. Halberstam and Richert \cite{key1} (Theorem
3.2) 

The observation due to Kobayashi, which we have mentioned earlier
(cf. notes for Chapters \ref{chap7} and \ref{chap8}), consists in noticing  
\begin{equation*}
\begin{cases}
G(z) \sum \limits_{\substack{d|F(n) \\ d|P(z) }} \lambda_d =
\sum\limits_{q < z} \mathop{\sum{}'}\limits^q_{l=1} b_{q,l}e(-n \frac{l}{q})\\  
\hspace{1cm} b_{q,l}= \frac{1}{q} \prod\limits_{p|q}(1-\frac{
  \rho(p)}{p})^{-1} \sum\limits^{q}_{\substack{h=1 \\(F(h), q) = 1}}
e(\frac{lh}{q}).  
\end{cases}\tag{9.51}\label{eq9.51}
\end{equation*}
and 
\begin{equation*}
\mu^2 (q) g(q) =
\mathop{\sum{}'}\limits^q_{l=1}|b_{q,l}|^2.\tag{9.52}\label{eq9.52} 
\end{equation*}
and using the duality principle
(cf. \eqref{eq2.47}--\eqref{eq2.48}). Here $F$ denotes 
an integer-valued polynomial and $\rho(p)$ is the number of solutions
of $F(n)  \equiv 0 \mod p$. Actually, Kobayashi \cite{key1} proves the
following dual form of the large sieve (in our notation) 
\begin{equation*}
\sum_{M<n \leq M+N}| \sum^R_{r=1}a_re(-nx_r)|^2 = (N +O (\delta^{-1}))
| \sum^R_{r=1}a_r|^2, \forall a_r \in \mathbb{C},\tag{9.53}\label{eq9.53} 
 \end{equation*} 
using the upper bound form of the large sieve as well as a smoothing
technique. of Bombieri \cite{key4}, for a lower bound. (The $O$-constant in
\eqref{eq9.53} is absolute.) From\pageoriginale this he derives
Selberg's sieve (cf. Mathews \cite{key3}), from which on using
\eqref{eq2.90} one obtains, instead of \eqref{eq9.33}, 
 \begin{equation*}
S(\mathscr{A},\mathfrak{p},z) \leq
\frac{X+z^2}{G(z)}.\tag{9.54}\label{eq9.54}  
 \end{equation*} 
(cf. Halberstam and Richert \cite{key1} (pp. 125-126)) a result that should
 be compared with \eqref{eq7.6}. It is also possible similarly to get the
 stronger form \eqref{eq7.8} by defining $\lambda_{d}$'s in  Selberg's sieve in
 a different way (cf. Halberstam and Richert \cite{key1} (p. 126)). 

\eqref{eq9.33}:~ For the second inequality in \eqref{eq9.33} we have used
that from $d_{\nu}|P(z)$, $\nu = 1,2$, one has $d: = [d_1, d_2]|P(z)$ and
also for each such $d$ 
 \begin{equation*}
|\{ d_1, d_2:d_{\nu}< d_{\nu}| P(z) ,[d_1, d_2]= d \} | \leq | \{d_1,
d_2:[d_1, d_2]= d  \}= 3^{\nu (d)}\tag{9.55}\label{eq9.55} 
 \end{equation*} 

\ref{chap9-thm9.2}:~ cf. Halberstam and Richert \cite{key1} (Theorem 3.3) 

\ref{chap9-thm9.3}:~ cf. Halberstam and Richert \cite{key1} (Theorem
3.10) 

