
\chapter{Arithmetical Aids}\label{chap1}%chap 1

IN\pageoriginale THIS chapter we shall collect for the reader's
convenience some of the results, which are use in later chapters, from
elementary number theory.  

\section{Multiplicative functions}\label{chap1-sec1}

By 
\begin{equation*}
\mathfrak{m} \tag{1.1}\label{eq1.1}
\end{equation*}
we denote the set of functions (defined on $\mathbb{N}$) 
\begin{equation*}%~Equation 1.2
f\nequiv 0 \tag{1.2}\label{eq1.2}
\end{equation*}
that satisfy 
\begin{equation*}%~Equation 1.3
f(nq)=f(n)f(q) \; \forall  n, \; q \in \mathbb{N} \tag{1.3}\label{eq1.3}
\end{equation*}
whenever
\begin{equation*}%~Equation 1.4
(n,q)=1. \tag{1.4}\label{eq1.4}
\end{equation*}

Since we have excluded the trivial $f$ by \eqref{eq1.2}, there is a $q$ such
that $f(q)\neq 0$ and so \eqref{eq1.3} implies for $n=1$ 
\begin{equation*} %~Equation 1.5
f \in \mathfrak{m}\Longrightarrow f(1) =1. \tag{1.5}\label{eq1.5}
\end{equation*}

Obviously
\begin{equation*}%~Equation 1.6
f(n)= n^z \in \mathfrak{m} \text{~ for every~ } z\in \mathbb{C},
\tag{1.6}\label{eq1.6} 
\end{equation*}
so, in particular, for $z=0$
\begin{equation*}%~Equation 1.7
f(n)=1 \in \mathfrak{m}, \tag{1.7}\label{eq1.7}
\end{equation*}
and for these functions the restriction \eqref{eq1.4} is not even
necessary. A non-trivial example is provided by the Mobius function,
defined by  
\begin{equation*}%~Equation 1.8
\mu (n)
\begin{cases}
0 \text{~ \; if $n$ is not squarefree, i.e., there is a prime~ }
 p:p|^2n,\\
(-1)^{\nu(n)} \text{~ if $n$ is squarefree}
\end{cases} \tag{1.8}\label{eq1.8}
\end{equation*}\pageoriginale
where, as usual. 
\begin{equation*}%~Equation 1.9
\nu(n)=\sum_{p|n}1  \quad (\nu(1)=0). \tag{1.9}\label{eq1.9}
\end{equation*}

For $\nu$, \eqref{eq1.3} is easily checked, because, if atleast one of the
numbers $n$ and $q$ is not squarefree we have zero on both sides, and
if both $n$ and $q$ are squarefree we get \eqref{eq1.3} subject to
\eqref{eq1.4} by using  
\begin{equation*}%~Equation 1.10
\nu(nq)=\nu(n) + \nu(q)\quad \text{for}\quad  (n,q)=1;
\tag{1.10}\label{eq1.10} 
\end{equation*}
hence 
\begin{equation*} %~Equation 1.11
\mu \in \mathfrak{m} \tag{1.11}\label{eq1.11}
\end{equation*}

A simple way of obtaining new functions is by multiplying together
\eqref{eq1.3} for any two such functions, so that  
\begin{equation*} %~Equation 1.2
f_1,f_2\in \mathfrak{m}\Rightarrow
f_1f_2 \in \mathfrak{m}. \tag{1.12}\label{eq1.12} 
\end{equation*}

A less trivial result is 
\begin{equation*}%~Equation 1.13
f_1,f_2 \in \mathfrak{m}\Rightarrow g(n):=
\sum_{d|n}f_1(d)f_2(\frac{n}{d})\in
\mathfrak{m}. \tag{1.13}\label{eq1.13} 
\end{equation*}

For a proof we first note that if $d|nq$ there is, in view of
\eqref{eq1.4}, a unique factorization $d=th$ with $t|n$, $h|q$.
Therefore, by considering \eqref{eq1.3}, for both functions, under the
condition \eqref{eq1.4}, we obtain   
\begin{equation*}%~Equation 1.14
\begin{cases}
g(nq)=\sum\limits_{t|n}\sum\limits_{h|q}f_1(th)f_2(\frac{n}{t}\frac{q}{h}) =\\
= \sum\limits_{t|n}f_1(t) f_2(\frac{n}{t})  \sum\limits_{h|q}f_1(h) f_2(\frac{q}{h})=g(n) g(q). 
\end{cases} \tag{1.14}\label{eq1.14}
\end{equation*}

For\pageoriginale a $f\in \mathfrak{m}$ it follows by repeated
application of \eqref{eq1.3}, subject to \eqref{eq1.4}, that  
\begin{equation*}%~Equation 1.5
f \in \mathfrak{m}\Rightarrow f(n)=\prod_{p|n} f(p^{a_p})\quad
\text{where}\quad n = \prod_{p|n} p^{a_p}. \tag{1.15}\label{eq1.15} 
\end{equation*}
so that these functions need only be known at prime-powers. In
particular, for squarefree  number one has  
\begin{equation*}%~Equation 1.16
f \in \mathfrak{m}\Rightarrow f(q)=\prod_{p|q} f(p) ~\text{ if }~ \mu (q)\neq
0. \tag{1.16}\label{eq1.16} 
\end{equation*}

A good illustration of this property of functions in $\mathfrak{m}$ is
given as follows. Suppose that $f_1$, $f_2 \in \mathfrak{m}$, and let $q$
be a squarefree number. Then $g\in \mathfrak{m}$, where $g$ is defined
by \eqref{eq1.13} abd from \eqref{eq1.16} we get  
\begin{equation*}%~Equation 1.7
\begin{cases}
g(q)=\prod\limits_{p|q} g(p)= \prod\limits_{p|q}(\sum\limits_{d|p}
f_1(d) f_2 (\frac{p}{d}))\\  
\hspace{0.7cm}=\prod\limits_{p|q} (f_1(p) f_2(1) + f_1(1) f_2(p))
\text{~ for~ } \mu(q)\neq 0. 
\end{cases}\tag{1.17}\label{eq1.17}
\end{equation*}

Therefore, in view of \eqref{eq1.5}, we have proved that 
\begin{equation*}%~Equation 1.18
\begin{cases}
f_1,f_2\in \mathfrak{m}\Rightarrow \sum\limits_{d|q}f_1(d) + f_2(\frac{q}{d})=\\
\hspace{1.8cm}= \prod\limits_{p|q}(f_1(p)f_2(p)). \text{~ if~ }
\mu(q) \neq 0. 
\end{cases} \tag{1.18}\label{eq1.18}
\end{equation*}

For an application we note that by \eqref{eq1.11} and \eqref{eq1.7}
one can take $\mu$ and $1$ for $f_1$ and $f_2$ respectively. Then,
denoting by $q(n)$, the `kernel' of $n$, i.e., the largest squarefree
divisor of $n$, \eqref{eq1.18} yields  
\begin{equation*}%~Equation 1.9
\sum_{d|n}\mu(d)= \sum_{d|q(n)}\mu(d)=\prod_{p|n}(\mu(p)+1),
\tag{1.19}\label{eq1.19} 
\end{equation*}
which gives the well-known formula 
\begin{equation*}%~Equation 1.20
\sum_{d|n}\mu(d)=
\begin{cases}
1 \text{~   for~ } n=1\\
0 \text{~   for~ } n>1.
\end{cases} \tag{1.20}\label{eq1.20}
\end{equation*}

\section{Ramanujan's function}\label{chap1-sec2}\pageoriginale

Recalling our notation (cf. \eqref{eq0.52})
\begin{equation*} % Equation 1.21 	
e(u):=e^{2\pi iu} \tag{1.21}\label{eq1.21}
\end{equation*}
we see, on considering the partial sums of geometric series, that
\begin{equation*} % Equation 1.22 
\sum^q_{\ell=1}e(n\frac{\ell}{q})= 
\begin{cases}
q \text{  for } q/n, \\
\hspace{2cm}\forall q\in \mathbb{N}, n\in \mathbb{Z}. \\
0 \text{  for } q \nmid n
\end{cases}\tag{1.22}\label{eq1.22}
\end{equation*}

The continuous analogue of \eqref{eq1.22} is 
\begin{equation*} %Equation 1.23
\int\limits_0^1 e(nx)dx=
\begin{cases}
1 \text{  for }n = 0, \\
1 \text{  for }n \neq 0, 
\end{cases}n\in\mathbb{Z}. \tag{1.23}\label{eq1.23}
\end{equation*}

Since $e(\dfrac{m}{q})$ has period $q$, \eqref{eq1.22} may also be
written as   
\begin{equation*} % Equation 1.24
\sum_{\ell \mod q}e(n\frac{\ell}{q})=
\begin{cases}
q \text{  for } q|n, \\
0 \text{  for } q \nmid n, 
\end{cases}\forall q\in \mathbb{N},
n\in\mathbb{Z}.\tag{1.24}\label{eq1.24} 
\end{equation*}
where $\ell$ runs through a complete system of residues modulo
$q$. The corresponding result for \eqref{eq1.23} is  
\begin{equation*} %Equation 1.25
\int\limits^{\alpha+1}_{\alpha} e(nx)dx=
\begin{cases}
1 \text{  for }n = 0, \\
0 \text{  for }n \neq 0, 
\end{cases}\forall n\in\mathbb{Z},
\alpha\in\mathbb{R}.\tag{1.25}\label{eq1.25} 
\end{equation*}

For any $a'_\ell s \in \mathbb{C}$ 
\begin{equation*}%~Equation 1.26
\sum^q_{n=1}\Big|\sum^q_{\ell=1} a_\ell e(\frac{n\ell}{q})\Big|^2 =
\sum^q_{\ell_1,\ell_2=1} a_{l_1} \overline{a}_{l_2} \sum^q_{n=1}
e((\ell_1-\ell_2)\frac{n}{q}),  \tag{1.26}\label{eq1.26} 
\end{equation*}
and because of $|\ell_1-\ell_2|<q$ in the innermost sum,
\eqref{eq1.22} gives  
\begin{equation*}%~Equation 1.27
\frac{1}{q}\sum^q_{\ell=1}|\sum^q_{\ell=1} a_\ell
e(\frac{n\ell}{q})\Big|^2 = \sum^q_{\ell=1} |a_\ell |^s, \; \forall
a_\ell \in \mathbb{C}. \tag{1.27}\label{eq1.27} 
\end{equation*}

Similarly, by \eqref{eq1.23}, assuming that $\sum_n
|a_n|^2<\infty$ (so, in particular, for any finite range for $m$), we
have  
\begin{equation*}%~Equation 1.28
\int\limits^1_0 |\sum\limits_ne(nx)|^2 dx=\sum_{n_1,n_2}
a_{n_1}\overline{a}_{n_2} \int\limits^1_0
e((n_1-n_2)x)dx=\sum_n|a_n|^2. \tag{1.28}\label{eq1.28} 
\end{equation*}

In\pageoriginale view of \eqref{eq1.25}, it is obvious that
\eqref{eq1.28} remains 
true if the integral is extend over any interval of length 1.
  
Ramanujan's sum is defined by 
\begin{equation*}%~Equation 1.29
\begin{cases}
c_q(n):\sum\limits^q_{\ell=1} e(n\frac{\ell}{q}) = \sum\limits'_{\ell
  \mod q} e(n \frac{\ell}{q}), \; \forall q\in \mathbb{N}, n\in
\mathbb{Z}\\ 
\text{and where
}\sum\limits'^{q}_{\ell=1}:\sum\limits^{q}_{\substack{\ell=1\\(\ell, q)=1}}
\end{cases}\tag{1.29}\label{eq1.29}
\end{equation*} 
and $\sum'\limits_{\ell \mod q}$ means summation as $\ell$ runs through
a reduced system of residues modulo $q$. In order to compute $c_q(n)$
we use \eqref{eq1.20}. Noting that always  
\begin{equation*}%~Equation 1.30
\sum_{d|q}f(d)=\sum_{d|q} f (\frac{q}{d}) \tag{1.30}\label{eq1.30}
\end{equation*}
and writing $\ell=\ell_1$. $\dfrac{q}{d}$ in the last step below we
find. by \eqref{eq1.20}, 
\begin{equation*}%~Equation 1.31
\begin{cases}
c_q(n)=  \sum\limits^{q}_{\ell = 1} e(n \frac{\ell}{q})
\sum\limits_{\substack{{d|q}\\d|\ell}} \mu (d) = \sum\limits_{d|q} \mu (d)
\sum\limits^{\ell}_{\substack{\ell =1 \\ \ell \equiv 0 \mod d}}
e(n\frac{\ell}{q})  =\\ 
\hspace{0.9cm}=\sum\limits_{d|q}\mu (\frac{q}{d}
)\sum\limits^q_{\substack{\ell=1\\ \ell=0 \mod \frac{q} {d}}} e(n
\frac{\ell}{q})= \sum\limits_{d|p}\mu (\frac{q}{d})
\sum\limits_{\ell_1=1}^{d}e(n\frac{\ell_1}{d}),  
\end{cases}\tag{1.31}\label{eq1.31}
\end{equation*}
so that, by \eqref{eq1.22}, we have 
\begin{equation*}%~Equation 1.32
c_q(n)= \sum_{\substack{d|q\\d|n}}\mu (\frac{q}{d})d  \quad  \forall
q\in \mathbb{N}, \;  n\in \mathbb{Z} \tag{1.32}\label{eq1.32} 
\end{equation*}

We note the following two special cases. For $n=0$, we see from
\eqref{eq1.29} that Ramanujan's sum becomes Euler's function $\varphi(q)$,
and \eqref{eq1.32} leads via \eqref{eq1.30} to the well-known formulae  
\begin{equation*}%~Equation 1.33
\begin{cases}
\varphi(q): =\sum\limits^q_{\ell=1}  1 = c_q(0)= \sum\limits_{d|q}\mu (\frac{q}{d})d=\sum\limits_{d|q}\mu(d) \frac{q}{d}=\\
\hspace{0.9cm}= q \sum\limits_{d|q(q)} \frac{\mu (d)}{d} = q \prod\limits_{
p|q(q)} (-\frac{1}{p}+1)= q\prod\limits_{p|q}(1-\frac{1}{P}),
\end{cases}\tag{1.33}\label{eq1.33}
\end{equation*}
where\pageoriginale we have used also \eqref{eq1.11}, \eqref{eq1.12},
\eqref{eq1.6} for 
$z=0$ and $z=0$ and $z=-1$, and \eqref{eq1.18}. Next, for $(n,q)=1$, the
right-hand side of \eqref{eq1.32} reduces to $\mu (q)$, so that via
\eqref{eq1.29} we also obtain  
\begin{equation*}%~Equation 1.34
\mu (q)=c_q(n)=\mathop{\sum{}'}\limits^q_{\ell=1} e(n\frac{\ell}{q}) \text{~ if~ }
(n,q)=1, q\in \mathbb{N}, n\in\mathbb{Z}. \tag{1.34}\label{eq1.34} 
\end{equation*}


\section{Dirichlet's characters and Gaussian Sums}\label{chap1-sec3} 

For each $q \in \mathbb{N}$ we define the arithmetic functions, names
`characters modulo $q$',  
\begin{equation*}%~Equation 1.35
\chi (m)(\in \mathbb{C}), \quad  \forall m \in
\mathbb{Z}. \tag{1.35}\label{eq1.35} 
\end{equation*}

For  an elementary introduction of these functions one requires
following four properties \eqref{eq1.36} through \eqref{eq1.39}: 
\begin{align*}%~Equation 1.36
\chi(1) & = 1 \tag{1.36}\label{eq1.36}\\
\chi (mn) & = \chi (m)\chi (n) \quad \forall m, n\in \mathbb{Z},
\tag{1.37} \label{eq1.37}
\end{align*}

i.e., $\chi\in \mathfrak{m}$ without the restriction $(m,n)=1$,
\begin{equation*}%~Equation 1.38
\chi (n)= \chi (\ell)   \text{~ for~ } n =\ell \mod q
\tag{1.38}\label{eq1.38} 
\end{equation*}
and 
\begin{equation*}%~Equation 1.39
\chi (n)=0    \text{~ for~ } (n,q)>1. \tag{1.39}\label{eq1.39}
\end{equation*}

The relations \eqref{eq1.38} and \eqref{eq1.36} imply 
\begin{equation*}%~Equation 1.40
\chi (m)=1 \quad \text{for}\quad m \equiv 1\mod q.
\tag{1.40}\label{eq1.40} 
 \end{equation*} 
 
By \eqref{eq1.39} any character vanishes for all $n$ which are not
coprime to $q$. On the other hand, for $(n,q)=1$, we have by Euler's
theorem $n^{\varphi (q)}\equiv 1 \mod q$, so that \eqref{eq1.37} and
\eqref{eq1.40} give  
 \begin{equation*}%~Equation 1.41
(\chi (n))^{\varphi(q)} = 1\quad \text{for}\quad
   (n,q)=1,\tag{1.41}\label{eq1.41} 
 \end{equation*} 
i,e., for  $(n, q) = 1$, $\chi(n)$ is a $\varphi(q)$-th root of unity;
in particular.  
\begin{equation*}%~Equation 1.42
|\chi (n)|=1\quad \text{for}\quad (n,q)=1. \tag{1.42}\label{eq1.42}
 \end{equation*}\pageoriginale 
 
It is  obvious that the function  
\begin{equation*}%~Equation 1.43
\chi_0(n)=  
\begin{cases}
1 \text{  if } (n,q)=1, \\ 
0 \text{  if  } (n,q)>1, \\ 
\end{cases}\tag{1.43}\label{eq1.43}
\end{equation*}
is a character modulo $q$; it is called the \textit{principal}
character $\mod q$. A simple formula is  
\begin{equation*}%~Equation 1.44
\sum^q_{\ell=1}\chi(\ell) = \mathop{\sum{}'}^q_{\ell=1} \chi(\ell)= 
\begin{cases}
 \varphi (q)  \text{   for } \chi=\chi_0, \\
o \text{   for   }\chi\neq \chi_0
\end{cases}\tag{1.44}\label{eq1.44}
\end{equation*}

The first statement is immediate from \eqref{eq1.39}, and for
$\chi=\chi_0$ the second statement follows by \eqref{eq1.43}. Next for
$\chi=\chi_0$ there must be a number $m$ such that  
\begin{equation*}%~Equation 1.45
(m,q)=1, \chi(m)\neq 1.\tag{1.45}\label{eq1.45}
\end{equation*}

So when $\ell$ runs through a reduced residue system $\mod q$ then the
same does $m \ell$ too. Hence, in view of \eqref{eq1.38} and
\eqref{eq1.37}, we have  
\begin{equation*}%~Equation 1.46
\chi(m) \sum'_{\ell \mod q} \chi(\ell) = \sum'_{\ell \mod q} \chi(\ell
m)= \sum'_{\ell \mod q} \chi(\ell),\tag{1.46}\label{eq1.46} 
\end{equation*}
which implies the second statement of \eqref{eq1.44} for $\chi\neq \chi_0$
because of $\chi(m) \neq 1$.  

It can be shown that for any given $q \in \mathbb{N}$ there are exactly
$\varphi(p)$ distinct functions fulfilling \eqref{eq1.36} through
\eqref{eq1.39}, 
i.e., there are $\varphi(q)$ characters $\mod q$, a fact that can be
stated in the form  
\begin{equation*}%~Equation 1.47
\sum_{\chi\mod q} 1=\varphi (q). \tag{1.47}\label{eq1.47}
\end{equation*}

It is easily checked that with $\chi(n)$ also $\bar{\chi}(n)$ is a character
$\mod q$. Also, with $\chi_1$, $\chi_2$ the function $\chi_1$, $\chi_2$ is a
character $\mod q$. The second remark leads us to the following
statement. If for certain  characters $\chi'$, $\chi''$, $\chi_1 \mod
q$, $\chi' (n) \chi_1(n)= \chi''(n)\chi_{1}(n)$ holds for all $n$, then
in view of \eqref{eq1.42} and \eqref{eq1.39}  one has
$\chi'=\chi''$. Hence,\pageoriginale if $\chi_1$ is a fixed character 
$\mod q$ and $\chi$ runs through all characters $\mod q$, then $\chi_1\chi$
also runs through all characters $\mod q$, a fact which can be expressed
through  
\begin{equation*}%~Equation 1.48
\chi_1(m) \sum_{\chi  \mod q}\chi(m)= \sum_{\chi  \mod
  q}{(\chi_1\chi)}(m)\sum_ {\chi  \mod q} \chi(m),\quad \forall m \in
\mathbb{Z} .  \tag{1.48}\label{eq1.48}
\end{equation*}

This leads us to the following counterpart of \eqref{eq1.44}:
\begin{equation*}%~Equation 1.49
\sum_{\chi \mod q}\chi(m) = 
\begin{cases} 
\varphi(q) & \text{for~ } m \equiv 1 \mod q, \\
0 & \text{for~  } m \nequiv 1 \mod q. 
\end{cases}\tag{1.49}\label{eq1.49}
\end{equation*}

For, when $m\equiv 1 \mod q$ this assertion follows from \eqref{eq1.40} and
\eqref{eq1.47} and for $(m,q)>1$ it is trivially true in view of
\eqref{eq1.39}. In this remaining case  
\begin{equation*}%~Equation 1.50
m \neq 1 \mod q, (m,q)=1, \tag{1.50}\label{eq1.50}
\end{equation*}
we need the result that, for any $m$ subject to \eqref{eq1.50}, there is a
character $\chi_1 \mod q$ such that $\chi_1(m) \neq 1$, and then the result
follows from \eqref{eq1.48}. 
  
Next we note the following equivalence:
\begin{equation*}%~Equation 1.51
\bar{\chi}(n)\chi_1(n)= \chi_0(n), \forall n \Leftrightarrow
\chi_1(n)= \chi(n).\quad \forall n. \tag{1.51}\label{eq1.51} 
\end{equation*}

This is clear for $(n,q)>1$. For  $(n,q)=1$ we multiply on the left by
$\chi(n)$ and use \eqref{eq1.42} and \eqref{eq1.43} (the latter
implies always that 
$\chi\chi_0 =\chi$), and this step can be reversed because $\chi(n)
\neq 0$. Since 
$\overline{\chi}\chi_1$ is a character, it may be used in \eqref{eq1.44}, and
because of \eqref{eq1.37} and \eqref{eq1.51} we obtain  
\begin{equation*}%~Equation 1.52
\mathop{\sum{}'}^q_{\ell=1} \bar{\chi}(\ell)\chi_1 (\ell) =  
\begin{cases}
\varphi(q) & \text{for~ } \chi=\chi_1,\\
0 & \text{for~ } \chi\neq \chi_1. 
\end{cases}\tag{1.52}\label{eq1.52}
\end{equation*}

Finally we prove that 
\begin{equation*}%~Equation 1.53
\sum_{\chi \mod q}\bar{\chi}(\ell)\chi (n)= 
\begin{cases}
\varphi (q) & \text{for~ } n \equiv \ell \mod q,\\ 
0 & \text{for~ } n \nequiv \ell \mod q 
\end{cases}\text{if~ } (\ell,q)=1.\tag{1.53}\label{eq1.53}
\end{equation*}

In\pageoriginale view of \eqref{eq1.39} we may assume that $(n,q)=1$. Since
$(\ell,q)=1$, 
we can determine $\ell'$ such that  
\begin{equation*}%~Equation 1.54
(\ell', q)=1, \ell \ell' \equiv 1 \mod q,\tag{1.54}\label{eq1.54}
\end{equation*}
and hence by \eqref{eq1.40} and \eqref{eq1.37} one has
$\chi(\ell)\chi(\ell')=1$, which 
yields after multiplication by $(\bar{\chi}(\ell))$, in view of
\eqref{eq1.42}, that $\bar{\chi}(\ell)=\chi(\ell')$, so that, by using
\eqref{eq1.37} again follows  
\begin{equation*}%~Equation 1.55
\sum_{\chi \mod q} \bar{\chi}(\ell)\chi(n) = \sum'_{\chi\mod q}
\chi(\ell'n).\tag{1.55}\label{eq1.55} 
\end{equation*}

On the otherhand, by the definition of $\ell'$, we have 
\begin{equation*}%~Equation 1.56
\ell'n\equiv 1 \mod q  \Leftrightarrow n \equiv \ell \mod
q. \tag{1.56}\label{eq1.56} 
\end{equation*}

Hence \eqref{eq1.53} follows from \eqref{eq1.49} with $m=\ell'n$,
using \eqref{eq1.55} and \eqref{eq1.56}.  

From \eqref{eq1.53} we now derive the following analogue of \eqref{eq1.27}:
\begin{equation*}%~Equation 1.57
\frac{1}{\varphi(q)} \sum_{\chi \mod q}| \sum^q_{\ell
  =1}\bar{\chi}(\ell)u_\ell |^2 = \mathop{\sum{}'}^q_{\ell
  =1}|u_{\ell}|^{2},\quad \forall u_\ell \in
\mathbb{C}. \tag{1.57}\label{eq1.57}  
\end{equation*}

Keeping \eqref{eq1.39} in mind, we see that the left-hand side equals
\begin{equation*}%~Equation 1.58
\frac{1}{\varphi(q)} \sum_{\chi \mod q} \sum'^q_{\ell, n
  =1}\bar{\chi}(\ell)\chi(n)u_\ell \bar{u}_n = \sum'^q_{\ell, n
  =1}u_\ell \bar{u}_n (\frac{1}{\varphi(q)} \sum_{\chi \mod q}
\bar{\chi}(\ell)\chi(n)). \tag{1.58}\label{eq1.58} 
\end{equation*}
so that \eqref{eq1.57} follows form \eqref{eq1.53}. 

If $d|q$ and $\chi_1$ is a character $\mod d$, then 
\begin{equation*}%~Equation 1.59
\chi(m):= 
\begin{cases}
\chi_1(m) & \text{for~ }(m, q)=1, \\
0 & \text{for~ } (m, q) >1, 
\end{cases} \tag{1.59}\label{eq1.59}
\end{equation*}
is a character $\mod q$, and we say that $\chi_1 \mod d$ induces the
character $\chi \mod q$. If $\chi \mod q$ is not induced by any character
$\chi_1 \mod d$  for any $d <q$, $\chi$ is then called a \textit{primitive}
character $\mod q$. The smallest $f$, $f|q$, such that a $\chi^*
\mod f$ induces\pageoriginale $\chi \mod q$ is called
\textit{conductor}of $\chi$.   

For any character $\chi \mod q $ the `Gaussian sum' is defined by 
\begin{equation*}%~Equation 1.60
\tau(\chi):=\sum^q_{\ell=1}\chi(\ell) e(\frac{\ell}{q})= \sum_{\ell
  \mod q} \chi(\ell) e(\frac{\ell}{q}),  \tag{1.60}\label{eq1.60} 
\end{equation*}
since both $\chi(\ell)$ and $e(\dfrac{\ell}{q})$ are of period $q$. By
\eqref{eq1.39}, $\ell$ runs actually through a reduced system of residues
$\mod q$, and so does $n \ell$ also if $(n,q)=1$. Therefore it
follows, by using \eqref{eq1.60} for $\bar{\chi}$, \eqref{eq1.37} and
\eqref{eq1.42}, that   
\begin{equation*}%~Equation 1.61
\begin{cases}
\tau(\bar{\chi}) \chi(n) & =\chi(n) \sum\limits_{\ell \mod q}
\bar{\chi}(n\ell) e(n\frac{\ell}{q})=\\ 
& = \sum\limits_{\ell \mod q}\bar{chi}(\ell)
e(n\frac{\ell}{q}), \text{ for any } \chi \mod q \text{ and } (n,q)=1. 
\end{cases}\tag{1.61}\label{eq1.61}
\end{equation*}

It requires a little more effort to prove that for primitive
characters $\chi$ \eqref{eq1.61}  holds even without the restriction
$(n,q)=1$, i.e,, 
\begin{equation*}%~Equation 1.62
\tau\bar{(\chi)}\chi(n) = \sum_{\ell \mod q} \bar{\chi}(\ell)
e(n\frac{\ell}{q}), \text{~ for primitive~ } \chi \mod q, n \in
\mathbb{Z}. \tag{1.62}\label{eq1.62} 
\end{equation*}

If we take $a_\ell = \bar{\chi}(\ell)$ in \eqref{eq1.27}, it follows from
\eqref{eq1.62} that for any primitive character $\chi \mod q$ 
\begin{equation*}%~Equation 1.63
\frac{1}{q}\sum_{n=1}^{q}|\tau (\bar{\chi})\chi(n)|^2=\varphi(q)
\tag{1.63}\label{eq1.63} 
\end{equation*}
or 
\begin{equation*}%~Equation 1.64
|\tau (\chi)^2|=q \text{~  for primitive~ } \chi \mod
q.\tag{1.64}\label{eq1.64} 
\end{equation*}

If $\chi$ is not a primitive character, let it be induced by 
\begin{equation*}%~Equation 1.65
\chi^* \mod f, q=rf, \tag{1.65}\label{eq1.65}
\end{equation*}
where $f$ is the conductor of $\chi$. It can be shown that 
\begin{equation*}%~Equation 1.66
\tau (\chi) =0 \quad \text{if~ } (r,f) >1 \tag{1.66}\label{eq1.66}
\end{equation*}
and that 
\begin{equation*} %~Equation 1.67
|\tau (\chi)|^2 = \mu^2(r) |\tau (\chi^*)|^2= \mu^2(r) f,\quad
\text{if~ } (r, f) =1,\tag{1.67}\label{eq1.67} 
\end{equation*} 
by\pageoriginale \eqref{eq1.64}, since $\chi^*$ is a primitive
character $\mod f$. Collecting together the results \eqref{eq1.64},
\eqref{eq1.66} 
and \eqref{eq1.67}. we have, for the Gaussian sums, the following  

\setcounter{section}{1}
\setcounter{lemma}{0}
\begin{lemma}\label{chap1-lem1.1} %lem 1.1
If $f$ is the conductor of $\chi \mod q$, then 
\begin{equation*}
q=rf \tag{1.68}\label{eq1.68}
\end{equation*}
and
\begin{equation*}
|\tau(\chi)^2 |= 
\begin{cases} 
\mu^2 (r)f  & \text{for~ } (r,f) =1,\\
0 & \text{otherwise}. 
\end{cases} \tag{1.69}\label{eq1.69}
\end{equation*}
\end{lemma}

We close this chapter with another application which demonstrates the
usefulness of our characters. 

\begin{lemma}\label{chap1-lem1.2} %lem 1.2
For any $a'_n s$ in $\mathbb{C}$ and any for character $\chi \mod q$ put
\begin{equation*}
S(x;q, \ell ):= \sum_{\substack{n \leq x \\ n \equiv \ell \mod q}} a_n
, S(x,\chi): = \sum_{n \leq x}a_n\chi(n). \tag{1.70}\label{eq1.70} 
\end{equation*}

Then 
\begin{equation*}
\mathop{\sum{}'}_{\ell = 1}^q | S(x; q, \ell )- \frac{S(x,\chi_0)}{\varphi
  (q)} |^2 = \frac{1}{\varphi (q)} \sum_{\chi \neq \chi_0} |S(x,
\bar{\chi}) |^2 \tag{1.71} \label{eq1.71}
\end{equation*}
\end{lemma}

\begin{proof} %pro
First we note that, by \eqref{eq1.53} (with $\chi$ being replaced by $\chi_1$
followed by taking the complex conjugate) and \eqref{eq1.52}, we have  
\begin{equation*}
\begin{cases}
\sum\limits_{\ell=1}^q \bar{x}(\ell ) S(x; q . \ell) &=\sum\limits_{\ell =1}^q \bar{x}(\ell ) \sum\limits_{n \leq x} a_n \frac{1}{\varphi (q)} \sum\limits_{x_1 \mod q} X_1 (\ell ) \bar{x}_1 (n) =\\
&=  \sum\limits_{n \le x} a_n \sum\limits_{x_1 \mod q} \bar{x}_1 (n) \frac{1}{\varphi (q)} \sum\limits_{\ell =1}^q{'} \bar{x}(\ell )x_1 (\ell )=\\
& =S(x, \bar{x}).
\end{cases} \tag{1.72}\label{eq1.72}
\end{equation*} 
\end{proof}

Next, taking $u_\ell = S (x; q, \ell)-
\dfrac{S(x,\chi_0)}{\varphi(q)}$ in \eqref{eq1.57}, it follows that  
{\fontsize{10}{12}\selectfont
\begin{equation*}
\sum_{\ell =1}^{q}{'}  \bigg | S(x; q. \ell )- \frac{S(x,\chi_0
  )}{\varphi (q)} \bigg |^2 =\frac{1}{\varphi (q)} \sum_{\chi \mod q}
\bigg |\sum_{\ell =1}^{q} \bar{\chi}(\ell) (S(x;q,\ell)-
\frac{S(x,\chi_0)}{\varphi (q)}) \bigg |^2. \tag{1.73}\label{eq1.73}  
\end{equation*}}

According\pageoriginale to \eqref{eq1.72} and \eqref{eq1.44}. we have 
\begin{equation*}
\sum_{\ell =1}^q \bar{\chi}(\ell ) (S(x;q, \ell)- \frac{S(x,\chi_0)}{\varphi (q)}) =
\begin{cases}
0 & \text{for } \chi=\chi_0,\\
S(x, \bar{\chi}) & \text{for } \chi \neq \chi_0,
\end{cases} \tag{1.74}
\end{equation*}
and using this in \eqref{eq1.73} we obtain \eqref{eq1.71}.

For the theory of prime numbers the important special case 
\begin{equation*}
\psi (x;q,\ell): = \sum_{\substack{n = x \\ n \equiv \ell \mod q}}
\Lambda (n), \psi (x,\chi):  \sum_{n \leq x} \Lambda (n)\chi (n)
\tag{1.75}\label{eq1.75}  
\end{equation*} 	
yields, on nothing that $\Lambda(n)$ is a real-valued function, the
identity 
\begin{equation*}
\mathop{\sum{}'}_{\ell}^{q} (\psi (x;q,\ell)- \frac{\psi (x,\chi_0)}{\varphi
  (q)})^2 =\frac{1}{\varphi(q)} \sum_{\chi \neq \chi_0} | \psi
(x,\chi)|^2. \tag{1.76}\label{eq1.76} 
\end{equation*}

This formula was the starting  point in the proof of the mean-value
theorem  of Davenport and Halberstam \cite{key2} (cf. \eqref{eq6.34}). 

\medskip

\begin{center}
\textbf{NOTES}
\end{center}

\eqref{eq1.27}: As another example, take in \eqref{eq1.27}
\begin{equation*}
a_\ell =
\begin{cases}
1 & \text{for } (\ell , q)=1,\\
0 & \text{for } (\ell , q)> 1.\\
\end{cases}
\end{equation*}

Then, in view of \eqref{eq1.29},
\begin{equation*}
\sum_{n=1}^{q} c^2_q(n) = q \varphi (q). \tag{1.77}
\end{equation*}


{\bf 3:} For the proofs omitted in this section  we refer, for
instance, to Davenport \cite{key1} (Chapter \ref{chap9}) and Huxley
\cite{key7} (Chapter \ref{chap3}).
