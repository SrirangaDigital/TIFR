
\chapter{Weighted Sieves}\label{chap12}%\chap 12

AS\pageoriginale ALREADY mentioned in the previous chapter our object
is to use the 
`final' results  there for  the proofs  
 of the results of next  chapter. For this purpose we need further
 improve to the quality  of these results,inspite of the concluding
 remarks of Chapter \ref{chap11}, and this is achie\-ved by considering a
 weighted sieve, which actually consists of a combination of various
 sifting functions (so that the counter-examples \eqref{eq11.86} collapse). 
 
 Before commencing the introduction of weighted  sieves for the
 purpose mentioned above, we briefly point out as to how the final
 result of Chapter \ref{chap11}, in particular \eqref{eq11.84},
 already yields a result in the direction of Theorem  
 \ref{chap13-thm13.2}. For this, let us consider
  \begin{equation*}
 \mathscr{A} := \bigg\{ p+2: p \leq x \bigg\}, \mathfrak{p} : =
 \mathfrak{p}_2 \tag{12.1}\label{eq12.1} 
 \end{equation*} 
(cf. \eqref{eq9.39}). Here we can take
\begin{equation*}
X=li \; x, \omega(p) = \frac{p}{p-1} \text{ for } p \in \mathfrak{p}
(i.e., \text{~ for~ } p\ge 2) \tag{12.2}\label{eq12.2} 
\end{equation*}  
and then the conditions $(\Omega_1)$, $(\Omega_2(1,L))$ are verified
easily for some fixed \textit{constants} $A_1$, $A_2$ and $L(\ge
1)$. Further, Bombieri's theorem (Lemma \ref{chap10-lem10.2}) enables
us to fulfill $(R(1, \alpha))$ with 
\begin{equation*}
\alpha = \frac{1}{2} \tag{12.3}\label{eq12.3}
\end{equation*}  
(for some suitably chosen absolute constants $A_3$ and $A_4$). 
  Also we find, by \eqref{eq10.8}, 
\begin{equation*}
\begin{cases}
W(z)= \prod\limits_{2<p<z} (1-\frac{1}{p-1})&=2
\prod\limits_{p<z}(1-\frac{1}{p}) \Pi _{2<p<z}
(\frac{p(p-2)}{(p-1)^2)})= \\ 
 & = 2e^{-\gamma}\prod\limits_{p>2}(1-\frac{1}{(p-1)^2})\frac{1}{\log
  z}(1+O(\frac{1}{\log z})).  
\end{cases}\tag{12.4}\label{eq12.4} 
\end{equation*}
  
Therefore,\pageoriginale taking
\begin{equation*}
z= x^{\frac{1}{u}}, u >4 ; u = 4.2\tag{12.5}\label{eq12.5}
\end{equation*}  
say, we obtain from \eqref{eq11.84} and \eqref{eq11.74} that, for some
positive constant $c_0$,	 
\begin{equation*}
S(\mathscr {A}, \mathfrak{p} ,z) \ge c_O \frac{x}{\log ^2 x}\text{~
  for~ } x \ge x_O . \tag{12.6}\label{eq12.6} 
\end{equation*}  
  
Now, note that the numbers counted on the left-hand side here have no
prime divisor $< z$ and so for each of these numbers we have  
\begin{equation*}
x+2 \ge p + 2 = p_1 \cdots p_r \ge x^{r/u} \tag{12.7}\label{eq12.7}
\end{equation*} 
which shows tht necessarily $r\leq u $ or by \eqref{eq12.5} that $r
\leq 4$. Thus, letting $x \rightarrow \infty $, we have that for
infinitely many primes $p$ holds  
\begin{equation*}
   p+2 = P_4 \tag{12.8}\label{eq12.8}
\end{equation*}   
  
At this point it is worthwhile to notice that Theorem
\ref{chap11-thm11.11} (instead 
of \eqref{eq11.84}) would have also led to \eqref{eq12.6}, though with
a smaller value for $c_0$, and so also to \eqref{eq12.8}.  We can
express this remark 
by saying that \eqref{eq12.8} follows from Selberg's upper bound sieve
combined with the (one-step) Buchstab-procedure on using
Bombieri's Prime number theorem (cf. also the remark involving
\eqref{eq13.21}, with respect to Theorem \ref{chap12-thm12.2}  below). 

Now we turn to the weighted  sieves. The first weighted sieve,
suitable for our purpose also, was introduced by Kuhn (in connection
with Brun's sieve).  Since then there have been  other more
sophisticated sieves invented. However, we shall need only a special
form of the simplest of these, namely Kuhn's sieve (cf. Theorem
\ref{chap12-thm12.1}). 

\setcounter{section}{12}
\setcounter{theorem}{0}
\begin{theorem}\label{chap12-thm12.1}
$(R(1,\alpha)):$ Let $h$ be an even integer (determined with respect
  to $x$)  satisfying  
\begin{equation*}
0 < |h| \leq x \tag{12.9}\label{eq12.9}
\end{equation*}\pageoriginale
and suppose that (associated with a sequence $\mathscr{A}$) we have
\begin{equation*}
X=\li x, \omega(p)=\frac{p}{p-1} \text{ for } p \in
\mathfrak{p}_h. \tag{12.10}\label{eq12.10} 
\end{equation*}

Let $u$ and $v$ be two real numbers (independent of $x$) such that
\begin{equation*}
\frac{1}{\alpha}<u<v. \tag{12.11}\label{eq12.11}
\end{equation*}

Define
\begin{equation*}
W(\mathscr{A};\mathfrak{p}_h,v ,u, \lambda): = \sum_{\substack{a \in
    \mathscr{A} \\ (a, P(x^{1/V}))=1}} \{1-\frac{1}{\lambda}
\sum_{\substack{x^{\frac{1}{v}} \leq p < x^{\frac{1}{u}}\\p | a \\ p
    \in \mathfrak{p}_h}}1, \tag{12.12}\label{eq12.12} 
\end{equation*}
\begin{equation*}
2 \leq \lambda \in \mathbb{R}. \tag{12.13}\label{eq12.13}
\end{equation*}
\end{theorem}

Then
\begin{gather*}
W(\mathscr{A}:\mathfrak{p}_h,v,u,\lambda) \geq e^{-\gamma}
\mathbb{C}(h)\frac{x}{\log^2 x}v\\
 \bigg\{f(\alpha v)-\frac{1}{\lambda}
\int _u ^v)F(v(\alpha-\frac{1}{t}))\frac{dt}{t}+O(\frac{1}{({\log
    x)^{1/15}}})\bigg\} \tag{12.14}\label{eq12.14} 
\end{gather*}
where $\mathbb{C}$ is defined by \eqref{eq9.42} and the $O$-constant depends
atmost on $u,v,A_3,A_4$ and $\alpha$. 

\begin{proof}
We may assume that
\begin{equation*}
x \geq X_0(u,v,A_3,A_4, \alpha). \tag{12.15}\label{eq12.15}
\end{equation*}

Let us set
\begin{equation*}
z:=x^{\frac{1}{v}}, \; y:=x^{1/u}. \tag{12.16}\label{eq12.16}
\end{equation*}

Then, by \eqref{eq12.12},
\begin{gather*}
W(\mathscr{A}:\mathfrak{p}_h,v,u,\lambda)=S(\mathscr{A},
\mathfrak{p}_h,z)\\
-\frac{1}{\lambda} \sum_{\substack{z \leq p < y \\ p
    \in \mathfrak{p}_h}} S(\mathscr{A}_p,
\mathfrak{p}_h,z)=S(\mathscr{A}, \mathfrak{p}_h,z)-\frac{1}{\lambda}
\sum_1, \tag{12.17}\label{eq12.17} 
\end{gather*}
say. Since $2 |h$, by \eqref{eq12.10}, the condition $(\Omega_1)$ is
satisfied with an \textit{absolute} constant\pageoriginale $A_1$. Also
$(\Omega_2(1,L))$ is fulfilled with some \textit{absolute} constant
$A_2$ and, by \eqref{eq12.9} with 
\begin{equation*}
L \leq O(1)+\sum_{p|h}\frac{\log p}{p-1} \ll \log 3|h| \ll \log \log
x. \tag{12.18}\label{eq12.18} 
\end{equation*}

Hence we get, by \eqref{eq11.84} (since $v>1 $ by \eqref{eq12.11}),
\begin{equation*}
S(\mathscr{A}, \mathfrak{p}_h,Z) \geq X W(z)\{f( \alpha v(\frac{\log
  \li x}{\log x}))+O(\frac{1}{(\log
  x)^{1/15}})\}. \tag{12.19}\label{eq12.19}  
\end{equation*}

We apply Theorem \ref{chap11-thm11.12} for the estimation of $\sum_1$,
and for this we define (in terms of the constants from ($R(1,
\alpha)$)) 
\begin{equation*}
\xi^2= \frac{x^{\alpha}}{\log ^{A_3+\alpha}X}. \tag{12.20}\label{eq12.20}
\end{equation*}

Note that for each $p$ in the range of $\sum_1$ we have
\begin{equation*}
\frac{\xi^2}{p} \geq \frac{x^{\alpha - \frac{1}{u}}}{\log ^{A_3+
    \alpha}X} \tag{12.21}\label{eq12.21} 
\end{equation*}

Therefore applying \eqref{eq11.79}, for each term in $\sum_1$, with
$\xi^2/p$ in place of $\xi^2,\mathfrak{p}=\mathfrak{p}_h$ (and $q=p$
so that $(Q)$ is satisfied because of $p \geq z$ and $p \in
\mathfrak{p}_h$) we obtain
{\fontsize{10pt}{12pt}\selectfont 
\begin{equation*}
\sum_1 \leq \sum_{z \leq p<Y} \frac{\omega(P)}{P}XW(z)\left\{
F(\frac{\log(\xi^2 / p)}{\log z})+O(\frac{1}{\log^{1/15}})\right \} +
\sum_{\substack{d < \xi^2 \\ (d, \mathfrak{p})=1}} \mu^2(d)3^{\gamma
  (d)} |R_d|. \tag{12.22}\label{eq12.22}  
\end{equation*}}\relax
in view of \eqref{eq12.21}, \eqref{eq12.11}, \eqref{eq11.81} and
\eqref{eq12.18}. Estimating the
last sum in \eqref{eq12.22} by means of $(R(1, \alpha))$
(cf. \eqref{eq11.62}), \eqref{eq12.20}
and \eqref{eq12.10} one gets 
\begin{equation*}
\sum_1 \leq XW(z)\left \{ \sum_{z \leq p < Y}
\frac{1}{p}F(\frac{\log(\xi^2/p)}{\log z})+ O(\frac{1}{(\log
  x)^{1/15}})\right \}+O(\frac{x}{\log^3
  x}). \tag{12.23}\label{eq12.23}  
\end{equation*}
after some simple considerations involving \eqref{eq12.10}, \eqref{eq11.73},
\eqref{eq11.75}, \eqref{eq11.75}, \eqref{eq12.16} and \eqref{eq10.22}. 

Now, from \eqref{eq12.10}, \eqref{eq10.8} and \eqref{eq12.16}, it
follows that (cf, \eqref{eq12.4}) 
\begin{equation*}
W(z)=\prod_{\substack{2 < p < z\\p \chi
    h}}(1-\frac{1}{p-1})=\frac{e^{-\gamma}}{\log z} \mathfrak{S}
(h)+O(\frac{\log \log x}{\log^2 x}) \tag{12.24}\label{eq12.24} 
\end{equation*}\pageoriginale
where $\mathfrak{S}$ is defined by \eqref{eq9.42} and (so) satisfies,
(cf. \eqref{eq12.9}),
\begin{equation*}
\mathfrak{S} (h)=O(\log \log 3|h|)=O(\log \log
x). \tag{12.25}\label{eq12.25} 
\end{equation*}

Using \eqref{eq12.24} in both \eqref{eq12.19} and \eqref{eq12.23} we
are led to (by \eqref{eq12.17} and \eqref{eq12.13}) 
\begin{gather*}
W(\mathscr{A}:\mathfrak{p}_h,v,u,\lambda)\geq \frac{x}{\log^2 x}e^{-
  \gamma}\mathbb{C}(h)v\\
 \left \{f(\alpha v)- \frac{1}{\lambda} \sum
_{Z \leq p < Y}\frac{1}{p}F(\frac{\log (x^{\alpha/P})}{\log
  x}v)+O(\frac{1}{(\log x)^{\frac{1}{15}}})\right\}
\tag{12.26}\label{eq12.26}  
\end{gather*}
by means of \eqref{eq11.76} and \eqref{eq12.20}. It remains only to
deal with the sum in \eqref{eq12.26} and for this we proceed as in
\eqref{eq10.24} obtaining thereby 
\begin{align*}
\sum_{z \leq p <y} \frac{1}{p}F(v
(\frac{\log(\frac{x^{\alpha}}{p})}{\log x}) &= \int\limits_z^{y} F(v
\frac{\log(\frac{x^\alpha}{w})}{\log x}) \frac{dw}{w \log w}+ O
(\frac{1}{\log z})\\
&=\int_u^v
F(v(\alpha-\frac{1}{t}))\frac{dt}{t}+O(\frac{1}{\log
  x}). \tag{12.27}\label{eq12.27}  
\end{align*}

Using this in \eqref{eq12.26} yields \eqref{eq12.14} and so the proof
is completed. 
\end{proof}


Regarding the terms containing the function $f$ and $F$ it turns out
that for a suitable range of $v$ one can have instead an expression
involving only elementary functions (cf. \eqref{eq11.73}) and
\eqref{eq11.74}). More precisely, we have 

\setcounter{lemma}{0}
\begin{lemma}\label{chap12-lem12.1}
Let
\begin{equation*}
\frac{1}{\alpha}<u<v,\frac{2}{\alpha}\leq v \leq \frac{4}{\alpha}
(\text{ for } 0 < \alpha \leq 1). \tag{12.28}\label{eq12.28} 
\end{equation*}

Then (for $2 \leq \lambda \in \mathbb{R}$)
\begin{equation*}
f(\alpha v)-\frac{1}{\lambda} \int_u^v F(v(\alpha -
\frac{1}{t}))\frac{dt}{t}=\frac{2e^{-\gamma}}{\alpha v}\left \{\log
(\alpha v-1)- \frac{1}{\lambda} \log
\frac{v-\frac{1}{\alpha}}{u-\frac{1}{\alpha}}\right\}. 
\tag{12.29}\label{eq12.29}  
\end{equation*}
\end{lemma}

\begin{proof}
Note that the arguments of the functions $f$ and $F$ in \eqref{eq12.29}
satisfy, by \eqref{eq12.28}, 
\begin{equation*}
2 \leq \alpha v \leq 4 \text{ and } 0<v(\alpha-\frac{1}{u})\leq
v(\alpha - \frac{1}{v})=\alpha v-1 \leq 3. \tag{12.30}\label{eq12.30} 
\end{equation*}

Hence, by \eqref{eq11.73} and \eqref{eq11.74} one has that the
left-hand side of \eqref{eq12.29} equals 
\begin{equation*}
\frac{2e^{\gamma}}{\alpha v}\left \{\log (\alpha
v-1)-\frac{1}{\lambda} \int\limits_u^v \frac{dt}{(t-\frac{1}{\alpha})}
\right \}. \tag{12.31}\label{eq12.31} 
\end{equation*}\pageoriginale

This proves \eqref{eq12.29}.
\end{proof}

For our use in the next chapter it suffices to have the specialization
of Theorem \ref{chap12-thm12.1} to the sequences 
\begin{equation*}
\mathscr{A}:=\{|p+h|:p \leq x \}, 2|h,0 < |h| \leq x,
\tag{12.32}\label{eq12.32} 
\end{equation*}
where $h$ is determined with respect to sufficiently large $x$, and
with $u$ and $v$ restricted by \eqref{eq12.28}. Here note that for
$\mathscr{A}$ one has \eqref{eq12.10} and $R_d=O(E(x,d))$ so that lemma
\ref{chap10-lem10.2} (with $k=1$) fulfills $(R(1, \alpha))$ for 
\begin{equation*}
\alpha=\frac{1}{2} \tag{12.33}\label{eq12.33}
\end{equation*}

Hence we have, by Theorem \ref{chap12-thm12.1} and Lemma
\ref{chap12-lem12.1} the required 

\begin{theorem}\label{chap12-thm12.2}
Let $h$ denote an even integer (determined with respect to $x$) satisfying
\begin{equation*}
0 < |h| \leq x. \tag{12.34}\label{eq12.34}
\end{equation*}
\end{theorem}

Let $u$ and $v$ be two real numbers (independent of $x$) subject to
\begin{equation*}
2 <u < v, 4 \leq v \leq 8 \tag{12.35}\label{eq12.35}
\end{equation*}
and let
\begin{equation*}
2 \leq \lambda \in \mathbb{R}. \tag{12.36}\label{eq12.36}
\end{equation*}

Then we have
\begin{gather*}
\sum_{\substack{p \leq x \\ (p+h,\prod p')=1 \\ p'<x^{1/v} \\ p'
    \nmid  h}} \bigg\{1-\frac{1}{\lambda}
\sum_{\substack{x^{1/v} \leq p < x^{1/u}\\ p' | p+h \\ p' \nmid h}}
\bigg\} \geq \frac{4x}{\log^2 x}\mathscr{S} (h)\\ 
 \left\{\log
(\frac{v}{2}-1)-\frac{1}{\lambda}\log \frac{v-2}{u-2}+0
(\frac{1}{(\log x)^{\frac{1}{15}}})\right\}. \tag{12.37}\label{eq12.37}
\end{gather*}
where $\mathfrak{S}$ is defined by \eqref{eq9.42} and the $O$-constant
depends atmost on $u$ and $v$. 

\medskip
\begin{center}
{\bf NOTES}\pageoriginale
\end{center}

\eqref{eq12.3}: Here and in the sequel one recognizes the effect of
Bombi\-eri's theorem (and possible improvements, for instance, like the
Elliott and Halberstam \cite{key2} conjecture mentioned in the notes
of Chapter \ref{chap6}, {\bf 3}) when used along with Selberg's sieve.

Among the various weighted sieves introduced successfully (for
applications) we mention first Kuhn \cite{key1}, \cite{key2},
\cite{key3}. Here the basic idea 
consists in forming the $W$-function (\eqref{eq12.12}). Next we have
Selberg's weights of the form 
\begin{equation*}
\sum_{a \in \mathscr{A}}\bigg \{1-\frac{1}{\lambda}D(a)\bigg
\}(\sum_{d|a} \lambda_d)^2 \tag{12.38}\label{eq12.38} 
\end{equation*}
(with  $\lambda_d$'s given by \eqref{eq9.28}). By taking
$\mathscr{A}=\{n(n+2): n \leq x,2 \nmid n \}$ and $D(a)=d(n)+d(n+2)$
here, Selberg \cite{key2} (cf. Selberg \cite{key4}, \cite{key5})
succeeded in proving that 
\begin{equation*}
n(n+2)=P_5 \tag{12.39}\label{eq12.39}
\end{equation*}
holds for infinitely many integers $n$. This method, in the case where
$D(a)$ has (apart from a term to take care of the `small' prime
divisors of $a$) the form of the inner sum in Kuhn's $W$ (cf. \eqref{eq12.12})
has been published by Miech \cite{key1}, \cite{key2}, and Porter
\cite{key1}, Next, Ankeny and Onishi \cite{key1} have replaced Kuhn's
constant weight 
$\dfrac{1}{\lambda}$, attached to the inner sum in \eqref{eq12.12}, by a
logarithmic weight which is more effective and also has a smoothing
effect on the prime divisors in that sum and this weight has been
generalized by Richert \cite{key1}. Both the Kuhn weight and the logarithmic
weight can also be used simultaneously in \eqref{eq12.38} (cf. Halberstam
and Richert \cite{key1} (Theorem 10.8)). In the first case, a
generalization and refinement of Selberg's second method (mentioned
above in connection with \eqref{eq12.39}) can be found in Bombieri
\cite{key6} (\S\ 8). Bombieri \cite{key6} (\S\ 9) (cf. \cite{key8})
has used this 
method, which is both elegant and comparatively simple (through some
what weaker than the other methods described above), for the problem
$p+2=P_4$.\pageoriginale (For a more general result which can be
obtained by this 
method, are Halberstam and Richart \cite{key1} (Theorem 10.9).) Buchstab
\cite{key2} has generalised Kuhn's idea of constant weights by splitting up
the inner sum in \eqref{eq12.12} into many parts and attaching different
constant weights to each part. This method is highly effective but is,
on the other hand, very complicated. Roughly speaking, it may be
described as splitting the inner sum into two parts and for one
portion attaching constant weights which approximate to the
logarithmic weight (thereby achieving a smoothing) while in the other
portion the attached weights approximate to a smoothing in the
opposite direction. 

Chen's ingenious idea, for which we refer to Chapter \ref{chap13}, leading to
an improvement in respect of some very prominent problems in additive
prime number theory does not need any of the more sophisticated
weighted sieves described above, but just the very special Kuhn's
sieve (in the form stated in Theorem \ref{chap12-thm12.2}). 

However, contrary to the other sieves methods mentioned above, this
method cannot be applied to a great variety of sieves problems. It can
be directly applied to the problems of the type 
\begin{equation*}
N=p+P_2.p+h=P_2, \quad ap+b=P_2 \tag{12.40}\label{eq12.40}
\end{equation*}
(cf. Theorem \ref{chap13-thm13.1}, \ref{chap13-thm13.2}; the last one
(cf. Theorem \ref{chap10-thm10.2}) according to Halberstam (oral
communication)). The problem of attacking other related problems by
this method has not yet been tackled (cf. Notes of Chapter
\ref{chap13}), and in this context the logarithmic weight procedure
still gives the best results known to date. In case of dimension
$\kappa=1$ we mention (Richert \cite{key1}): 

Let $F(n)$ be an irreducible polynomial of degree $g(\geq 1)$ with
integer coefficients. Let $\rho(p)$ denote the number of
solutions of the congruence 
\begin{equation*}
F(m) \equiv 0 \mod p. \tag{12.41}\label{eq12.41}
\end{equation*}\pageoriginale
and suppose that
\begin{equation*}
\rho(p)<p \text{ for all }p. \tag{12.42}\label{eq12.42}
\end{equation*}

Then, we have
\begin{equation*}
|\{n:1 \leq n \leq x, F(n)=P_{g+1}\}| \geq \frac{2}{3} \prod_{p}
\frac{(1-\frac{\zeta(p)}{p})}{(1-\frac{1}{P})} \frac{x}{\log x} \text{
  for } x \geq x_\circ(F) \tag{12.43}\label{eq12.43} 
\end{equation*}
and if further
\begin{equation*}
\rho (p)<p-1 \text{ for } p \nmid F(0) \text{ and } p \leq
g+1, \tag{12.44}\label{eq12.44} 
\end{equation*}
then (excluding the case $F(n)=\pm n$) we also have
\begin{gather*}
|\{p: p \leq x, F(p)=P_{2g+1}\}| \geq \frac{4}{3}\prod_{p \nmid F(0)}
\frac{(1-\frac{\rho(p)}{p-1})}{(1-\frac{1}{p})}\\
\prod_{p |F(0)}
\frac{(1-\frac{\rho(p)-1)}{p-1}}{(1-\frac{1}{p})} \frac{x}{\log^2 x}
\text{~ for~ } x \geq x_0(F). \tag{12.45}\label{eq12.45} 
\end{gather*}

Hence, in particular (if $\lim\limits_{x \to
  \infty}\dfrac{x_0(F)}{x}=0$), there are infinitely many natural
numbers $n$ such that 
\begin{equation*}
F(n)=P_{g+1}. \tag{12.46}\label{eq12.46}
\end{equation*}
and also infinitely many primes $p$ such that
\begin{equation*}
F(p)=P_{2g+1}. \tag{12.47}\label{eq12.47}
\end{equation*}
(Note that for \eqref{eq12.47} there is no need to exclude the case
$F(n)=\pm n$ as was done for \eqref{eq12.45}.) 

The corresponding problems for polynomials in two variables\break $F(m, n)$
canceled more successfully. In this connection we refer, for
instance, to the papers of Greaves \cite{key1}, \cite{key2}, Iwaniec
\cite{key2}, \cite{key3}, \cite{key5}, \cite{key6}, and Huxley and
Iwaniec \cite{key1}.  

Regarding the question of almost primes in arithmetic progressions, i.e.,
\begin{equation*}
P_r \equiv l \mod k, (l,k)=1, \tag{12.48}\label{eq12.48}
\end{equation*}\pageoriginale

Motohashi has proved, by averaging the logarithmic weights of Ric\-hert
\cite{key1}, that there is a 
\begin{equation*}
P_2 \leq k^{1.1}, \tag{12.49}\label{eq12.49}
\end{equation*}
and that there is a
\begin{equation*}
P_3 \leq k \log^{70} k \tag{12.50}\label{eq12.50}
\end{equation*}
for almost all $k$ and also the corresponding results valid for almost
all $\ell \mod k (k \to \infty)$ (Motohashi \cite{key12},
\cite{key13}, \cite{key15}). Without
any exceptions we have only the existence of 
\begin{equation*}
P_2 \ll k^{2.2}. \tag{12.51}\label{eq12.51}
 \end{equation*}
 \begin{equation*}
P_3 \ll k^{11/7} \tag{12.52}\label{eq12.52}
\end{equation*}  
and
\begin{equation*}
P_r \ll k^{1+\frac{1}{r-\frac{9}{7}}} \text{~ for~ } r>
2. \tag{12.53}\label{eq12.53} 
\end{equation*}
in \eqref{eq12.48}. (Actually, here the exponents can be replaced by
slightly smaller ones.) (Richert \cite{key1}. cf. Halberstam and
Richert \cite{key1} (Theorem 9.6)). These results be compared with the
corresponding ones for primes (cf. the remark following
\eqref{eq6.55}). 

For more details and various other applications of weighted sieves we
refer to Halberstam and Richert \cite{key1} (Chapter \ref{chap9} and
\ref{chap10}).  

As to the literature pertaining to the small sieves in general we
refer to the extensive list of references given in Halberstam and
Richert \cite{key1}. We take this opportunity to add the following list of
papers, which are neither included there nor are mentioned in these
lectures so far: 

Bombieri \cite{key7},\pageoriginale Buchstab \cite{key3}, Elliott
\cite{key3}, Hall 
\cite{key1}, Hooley \cite{key3}, \cite{key10}. Meijer \cite{key1},
\cite{key2}, Ramachandra \cite{key5}, Scourfield \cite{key1}, Wolke
\cite{key4}, \cite{key8}.

Lastly, we take up now again (cf. Notes of Chapters \ref{chap7},
\ref{chap8} and \ref{chap9}) the 
question of comparison of the large sieves and the small sieves. The
importance of the large sieves for applications in analytic number
theory should be clear from the exposition in the Chapters \ref{chap2} through
\ref{chap6}, and it can hardly be overestimated. Also in arithmetical
questions the large sieves turns out to be powerful when applied, in
an auxiliary capacity, along with `small' sieves, not only via the
Bombieri-type results of Chapter \ref{chap6}, {\bf 3} (cf. our condition $(R(1,
\alpha))$, \eqref{eq11.62}), and under \eqref{eq12.3} in these notes
above) but also 
in a wider sense as is reflected in the proof of Chen's theorem
(cf. Theorem \ref{chap10-thm10.3} and the beginning of notes for
Chapter \ref{chap13}). It 
is only in its arithmetical version (cf.Chapter \ref{chap7}) that the large
sieve, even in its most powerful form (namely, the weighted
Montgomery-Vaughan sieve) suffers from some deficiencies when compared
with Selberg's sieve (in particular, when compared with the weighted
form of the latter). Without repeating our remarks, made in the Notes
of Chapter \ref{chap7}, \ref{chap8} and \ref{chap9}, we mention only
the following facts: The 
large sieve (for example, Theorem \ref{chap7-thm7.1}) can be used (as
it is) to 
obtain only upper bounds, while Buchstab's method (cf. Lemma
\ref{chap11-lem11.1}) 
provides (at least in principle)a corresponding lower bound, however,
for $\dfrac{\log N}{\log z}>2$. Theorem \ref{chap11-thm11.13} (for
$X = N$) gives a
better estimate than does Theorem \ref{chap7-thm7.1}. Therefore, a
suitably iterated 
form of \eqref{eq7.9} should be investigated. With respect to the function
$\omega(p)$ the large sieve has the decisive advantage over the
Selberg's sieve in that Theorem \ref{chap7-thm7.1} imposes no
restriction on the order of magnitude of $\omega(p)$, so in particular
$\omega(p)$ need 
not be bounded on the `average' as\pageoriginale required by our condition
$(\Omega_2)$ of the small sieve. However, the large sieve cannot deal
with the very important case $\omega(p)=\dfrac{p}{p-1}$, for example
with the problem 
\begin{equation*}
F(p)=P_r. \tag{12.54}\label{eq12.54}
\end{equation*}

Here, the defect steams from the fact that the large sieves basically
requires that $n$ has to run through a sequence of
\textit{consecutive} integers. Thus the large sieve, while applicable
to the problem 
\begin{equation*}
F(n)=P_r. \tag{12.55}\label{eq12.55}
\end{equation*} 
(leading to a trivial estimate when applied directly to sieve the
sequence $\{F(p)\}$ requires that one has to sieve the sequence $\{n
F(n)\}$ for the problem $(12.54)$, but then (cf. our remarks following
\eqref{eq11.14}) the constant in the upper estimate is worsened by a
factor 2. 
