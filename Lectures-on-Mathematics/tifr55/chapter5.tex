

\chapter{The Hybrid Sieve}\label{chap5}%5

FOR\pageoriginale SOME very important applications to number theory,
via Dirichlet series, estimations of averages of the type 
\begin{equation*}
\sum_{q \leq Q}(\cdots) \sum^*_{X \mod q} \int\limits^T_{-T},
\tag{5.1}\label{eq5.1} 
\end{equation*}
which is a combined version of the forms of the large sieve considered
in Chapters \ref{chap3} and \ref{chap4}, are of much use. Prior to the
innovation of 
an ingenious method of Hal\'asz \cite{key1} there was no method of dealing
with this question without carrying out at least one of the operations
$\sum^*$ or $\int$ in a trivial fashion. Methods, for the purpose of
this hybrid sieve, were developed independently by Montgomery \cite{key2},
combining ideas of Hal\'asz with the large sieve, and by Jutila \cite{key1},
who used a method of Rodosskij with large sieve. Subsequently, a
common basis for both of these was provided by Gallagher \cite{key4}
through the introduction of new technical devices (see Lemmas
\ref{chap5-lem5.1} and \ref{chap5-lem5.2} below). 

We start with
\setcounter{section}{5}
\setcounter{lemma}{0}
\begin{lemma}\label{chap5-lem5.1} %Lem 5.1
Let
\begin{equation*}
D(t) : \sum_\nu c (\nu ) e(\nu t ), \tag{5.2}\label{eq5.2}
\end{equation*}
where $\nu$ runs through a countable set of real numbers and the
coefficients $c(\nu ) (\in \mathbb{C})$ are subjected to the condition 
\begin{equation*}
\sum_\nu \mid c (\nu ) \mid < \infty . \tag{5.3}\label{eq5.3}
\end{equation*}
\end{lemma}

Let $\delta$ and $T$ be positive real numbers satisfying
\begin{equation*}
\delta T \leq \frac{1}{2 \pi}. \tag{5.4}\label{eq5.4}
\end{equation*}

Then, for some absolute constant $c_0$, holds
\begin{equation*}
\int\limits^T_{-T} | D (t) |^2 dt \le c_0 \int\limits^\infty_{-
  \infty} | C_\delta (y) |^2 dy, \tag{5.5}\label{eq5.5} 
\end{equation*}\pageoriginale
where
\begin{equation*}
C_\delta(y) : \delta^{-1} \sum_{\mid y - \nu \mid < \frac{\delta}{2}}
c(\nu ). \tag{5.6}\label{eq5.6} 
\end{equation*}

\begin{proof} %pro
For a proof of \eqref{eq5.5} we use two results from the theory of
Fourier transforms. Introduce 
\begin{equation*}
F_\delta (y) : \begin{cases} \delta^{-1} & if \mid y \mid < \delta /
  2, \\ 0 & otherwise, \end{cases} (y \in \mathbb{R})
\tag{5.7}\label{eq5.7}  
\end{equation*}
so that
\begin{equation*}
C_\delta (y) = \sum_\nu c(\nu ) F_\delta (y - \nu ). \tag{5.8}\label{eq5.8}
\end{equation*}

In view of \eqref{eq5.3}, $C_\delta(y)$ is a bounded integrable function and
hence it belongs to $L_2(- \infty, \infty)$. Therefore, by
Plancherel's theorem, $C_\delta$ has a Fourier transform
$\hat{C}_\delta$ and further, by Parseval's formula, we have  
\begin{equation*}
\int\limits^\infty_{-\infty} \mid C_\delta (y) \mid^2 dy = \int
\limits^\infty_{-\infty} \mid \hat{C_\delta}(t) \mid^2
dt. \tag{5.9}\label{eq5.9}  
\end{equation*}

Now one has
\begin{equation*}
\begin{cases}
\hat{C_\delta}{(t)} &= \int \limits^\infty_{-\infty} C_\delta (y)e
(yt)dy = \sum_\nu c(\nu) \int \limits^\infty_{-\infty} F_\delta
(y-\nu)e (yt) dy = \\  
&= \sum_\gamma c(\nu ) e (\nu t) \int \limits^\infty_{-\infty} F_\delta (x) e
(xt)dx = D(t) \hat{F_\delta}(t) 
\end{cases}\tag{5.10} \label{eq5.10}
\end{equation*}
say, on using \eqref{eq5.8} and \eqref{eq5.2}. Also, by \eqref{eq5.7},
\begin{equation*}
\hat{F_\delta}(t) = \delta^{-1} \int\limits^{\delta/2}_{- \delta/2}
e(xt)dx = \frac{\sin (\pi \delta t)}{\pi \delta
  t}. \tag{5.11}\label{eq5.11}  
\end{equation*}

Thus \eqref{eq5.9} yields
\begin{equation*}
\int\limits^\infty_{-\infty} \mid C_\delta (y) \mid^2 dy = \int
\limits^\infty_{-\infty} \mid D(t) \hat{F_\delta}(t) \mid^2 dt \leq
\int\limits^T_{-T} \mid D(t) \hat{F_\delta}(t) \mid^2
dt. \tag{5.12}\label{eq5.12}  
\end{equation*}
because\pageoriginale of \eqref{eq5.10}. For $|t|\leq T$, we use
\eqref{eq5.11} to note (cf. \eqref{eq5.4}) 
\begin{equation*}
|\hat{F}_{\delta}(t)| \geq \frac{\sin (\pi \delta T)}{\pi \delta T}
\geq \frac{1}{\sqrt{c_0}}, \tag{5.13}\label{eq5.13} 
\end{equation*}
say, so that \eqref{eq5.5} follows from \eqref{eq5.12}.
\end{proof}

Following Gallagher we use Lemma \ref{chap5-lem5.1} to prove
(Gallagher \cite{key4}, Theorem 1). 

\begin{lemma}\label{chap5-lem5.2}
For $a_n \in \mathbb{C}$, let
\begin{equation*}
\sum_{n=1}^{\infty}|a_n < \infty.\tag{5.14}\label{eq5.14}
\end{equation*}
\end{lemma}

Then, for $T \geq 1$,
\begin{equation*}
\int\limits_{-T}^T \Big| \sum_{n=1}^{\infty} a_n n^{-it}\Big|^2 dt
\leq c_1 T^2 \int\limits_0^\infty \Big| \sum_{x<n<xe^{1/T}} a_n\Big|^2
\frac{dx}{x} \tag{5.15}\label{eq5.15} 
\end{equation*}
holds with some absolute constant $c_1$.

\begin{proof}
In Lemma \ref{chap5-lem5.1} we choose
\begin{equation*}
\nu = - \frac{1}{2\pi}~\log ~ n, ~ c(\nu ) = a_n, ~ n\in \mathbb{N}
\tag{5.16}\label{eq5.16} 
\end{equation*}
and note that \eqref{eq5.3} is satisfied because of
\eqref{eq5.14}. Further, we put 
\begin{equation*}
\delta = \frac{1}{2\pi T}, y = \frac{1}{2\pi}(\log x + \frac{1}{2T})
\quad (x > 0), \tag{5.17}\label{eq5.17} 
\end{equation*}
so that \eqref{eq5.4} is fulfilled and the condition of summation in
\eqref{eq5.6} reads 
\begin{equation*}
- \frac{\delta}{2} < y - \nu < \frac{\delta}{2} \Leftrightarrow \log x
< \log n < \frac{1}{T} + \log x. \tag{5.18}\label{eq5.18} 
\end{equation*}

Therefore \eqref{eq5.5} yields (with the above choice)
\begin{equation*}
\int\limits_{-T}^{T} \Big| \sum_{n=1}^{\infty} a_n n^{-it} \Big|^2 dt
\leq c_0 \int\limits_{0}^{\infty} 2\pi T^2 \Big| \sum_{x<n<xe^{1/T}}
a_n \Big|^2 \frac{dx}{x}. \tag{5.19}\label{eq5.19} 
\end{equation*}

This completes the proof of the lemma.
\end{proof}

Now we shall give two applications of Lemma \ref{chap5-lem5.2} for the
averages of the type \eqref{eq5.1}. 

\setcounter{theorem}{0}
\begin{theorem}\label{chap5-thm5.1}
For\pageoriginale $a_n \in \mathbb{C}$, let
\begin{equation*}
\sum_{n=1}^{\infty}|a_n| < \infty . \tag{5.20}\label{eq5.20}
\end{equation*}
\end{theorem}

Then, for $T \geq 1$,
\begin{equation*}
 \sum_{q\leq Q} \frac{q}{\varphi (q)}\sum_{\chi \mod ~ q}^*
 \int\limits_{-T}^{T} \Big| \sum^{\infty}_{n=1}a_n \chi (n)n^{-it}
 \Big|^2 dt \leq 2c_1 \sum_{n=1}^{\infty} (TQ^2 + n) | a_n |^2 \forall Q
 \in \mathbb{N} \tag{5.21}\label{eq5.21} 
\end{equation*}
holds with the constant $c_1$ of Lemma \ref{chap5-lem5.2}.

\begin{proof}
We use Lemma \ref{chap5-lem5.2} with $a_n \chi (n)$ instead of $a_n$
and then apply 
\begin{equation*}
\sum_{q\leq Q}\frac{q}{\varphi (q)} \sum\limits_{\chi \mod q}^*
\tag{5.22}\label{eq5.22} 
\end{equation*}
to (the resulting) \eqref{eq5.15}. Then, we have for the squared
expression on the righthand side of \eqref{eq5.15}, in the notation of
Theorem \ref{chap3-thm3.3}, $M = [x]$ and $N \leq x(e^{1/T}-1)+1$ so
that using \eqref{eq3.23} it follows that 
\begin{gather*}
\sum_{q\leq Q}\frac{q}{\varphi (q)}\sum_{\chi \mod  q}^*
\int\limits_{-T}^T \Big| \sum_{n=1}^{\infty} a_n \chi (n)n^{-it}
\Big|^2 dt\\
 \leq c_1 T^2 \int\limits_0^\infty (Q^2 + 1 +
x(e^{\frac{1}{T}}-1)x (\sum_{x<n<xe^{1/T}} | a_n |^2)
\frac{dx}{x}. \tag{5.23}\label{eq5.23} 
\end{gather*}

Herein, the factor of $| a-n |^2$ is 
\begin{gather*}
\Big\{ c_1 T^2 (Q^2 + 1) \int\limits^n_{ne^{-1/T}} \frac{dx}{x} + c_1
T^2 (e^{\frac{1}{T}}-1)  \int\limits^n_{ne^{-1/T}} dx\\
 = c_1 T(Q^2 + 1)
+ c_1 T^2(e^{\frac{1}{T}}-1)(1-e^{-\frac{1}{T}})n \leq 2c_1 (TQ^2 +
n), \tag{5.24}\label{eq5.24} 
\end{gather*}
where we have employed the estimate
\begin{equation*}
T^2(e^{1/T}-1)(1-e^{-1/T}) \leq 2 \text{ for } T \geq
1. \tag{5.25}\label{eq5.25} 
\end{equation*}
\end{proof}

Now, putting together \eqref{eq5.23} and \eqref{eq5.24} we obtain
\eqref{eq5.21}. 

\begin{theorem}\label{chap5-thm5.2}
Let $Q \in \mathbb{N}$. For $a_n \in \mathbb{C}$, let
\begin{equation*}
\sum_{n=1}^{\infty} | a_n | < \infty. \tag{5.26}\label{eq5.26}
\end{equation*}
and suppose that
\begin{equation*}
a_n = 0  \text{~ unless~ } (n,q) = 1 \text{~ for all~ } q \leq
Q. \tag{5.27}\label{eq5.27} 
\end{equation*}
\end{theorem}

Then,\pageoriginale for $T \geq 1$,
\begin{equation*}
\sum_{q\leq Q} \log \frac{Q}{q} \sum_{\chi \text{ mod }q}^*
\int\limits_{-T}^{T} | \sum^{\infty}_{n=1} a_n x(n)n^{-it} |^2 dt \leq
2c_1 \sum_{n=1}^\infty (TQ^2 + n) | a_n |^2 \tag{5.28}\label{eq5.28} 
\end{equation*}
holds with the constant $c_1$ of Lemma \ref{chap5-lem5.2}.

\begin{proof}%proof
We proceed as in the proof of Theorem \ref{chap5-thm5.1} but applying
\begin{equation*}
\sum_{q \leq Q} \log \frac{Q}{q} \sum_{\text{ mod }q}^{*}
\tag{5.29}\label{eq5.29}
\end{equation*}
instead of \eqref{eq5.22}. Now the condition \eqref{eq5.27} permits us
to use \eqref{eq3.5} of Theorem \ref{chap3-thm3.1} to obtain the same
bound as in \eqref{eq5.23} for 
the left-hand side of \eqref{eq5.28}. So the proof is again completed by
\eqref{eq5.24}. 
\end{proof}

\medskip
\begin{center}
{\bf NOTES}
\end{center}

It is possible to put Hal\'asz's method in an abstract form
(cf. Montgomery \cite{key5} (Lemma 1.7) and Huxley \cite{key7}
(p.~115)). Gallagher and Bom\-bieri (\cite{key3}) have observed that Bellman's
inequality \eqref{eq0.59} contains both the large sieve and the idea of
Hal\'asz. 

All the results of this chapter are due to Gallagher \cite{key4}. Clearly,
estimates corresponding to the other results of Chapter \ref{chap3} can be
derived in the same way. For general results conforming to the theme
of this chapter and also for more sophisticated forms of the hybrid
sieve, see Montgomery \cite{key5}. Huxley \cite{key7}, Gallagher
\cite{key4}, Forti and Viola \cite{key1} (cf. Bombieri \cite{key5} and
\cite{key6} (\S 5)), Huxley
\cite{key9}, \cite{key11}, Jutila \cite{key6} and Huxley \cite{key12}. 

\eqref{eq5.4}:~ This condition can be relaxed to $\delta T \leq 1 -
\epsilon$ for any $\epsilon > 0$ and then the constant $c_0$ of
\eqref{eq5.5} depends, as can be seen from the proof
(cf. \eqref{eq5.13}), on $\epsilon$. 

\eqref{eq5.5}:~ Since there is not need in applications we do not aim
at obtaining the best possible values (for instance, by a different
choice of $\delta$) for the constants\pageoriginale $c_0$ and $c_1$ in this
chapter. However, just for a complete form of the proof we obtain some
permissible values for these constants. From \eqref{eq5.13} and
\eqref{eq5.4} we see that one can take  
\begin{equation*}
c_0 = (2 \sin \frac{1}{2})^{-2} < 1.1. \tag{5.30}\label{eq5.30}
\end{equation*}

\begin{lemma}\label{chap5-lem5.3}
As Gallagher \cite{key4} has stated, if the $a_n$'s are irregular,
Lemma \ref{chap5-lem5.2} is more precise than Theorem
\ref{chap5-thm5.1}, since in Lemma \ref{chap5-lem5.2} the coefficients
are first smoothed and then squared. 
\end{lemma}

\eqref{eq5.14}:~ The condition (occurring in Theorems
\ref{chap5-thm5.1} and \ref{chap5-thm5.2}) 
\begin{equation*}
\sum_{n=1}^\infty | a_n | < \infty \tag{5.31}\label{eq5.31}
\end{equation*}
of this chapter stems from Lemma \ref{chap5-lem5.1}. Further, our
condition of Theorem \ref{chap4-thm4.3}, namely 
\begin{equation*}
\sum_{n=1}^{\infty} n | a_n |^2 \infty , \tag{5.32}\label{eq5.32}
\end{equation*}
need also be satisfied; for, otherwise, the theorems of this chapter
hold trivially true. Also, it may be noted from the examples 
\begin{equation*}
a_n = \frac{1}{n \log (n+1)} \text{~ and~ } a_n =
\begin{cases}
\frac{1}{m^2} & \text{if } n = 2^m, m \in \mathbb{N},\\
0 & \text{ otherwise},
\end{cases} \tag{5.33}\label{eq5.33}
\end{equation*}
that the conditions \eqref{eq5.31} and \eqref{eq5.32} are independent.

\eqref{eq5.15}:~ From \eqref{eq5.19} and \eqref{eq5.30} we see that 
\begin{equation*}
c_1 = 2\pi c_0 < 7. \tag{5.34}\label{eq5.34}
\end{equation*}


\eqref{eq5.25}:~ For  a proof of \eqref{eq5.25} if suffices to verify, by
differentiation and the use of $c^{1/x}> 1 + \dfrac{1}{x}(x>0)$, that
the function $x(1-e^{-1/x})$ is decreasing in $x > 0$ so that we have  
\begin{equation*}
x^2(e^{1/x} -1)(1-e^{-1/x}) = (x(1-e^{-1/x}))^2 e^{1/x} \leq
(e-1)(1-e^{-1}) \text{ for } x \geq 1, \tag{5.35}\label{eq5.35} 
\end{equation*}
which gives the upper bound in \eqref{eq5.25} because
\begin{equation*}
(e-1)^2 < 2e. \tag{5.36}\label{eq5.36} 
\end{equation*}
