
\chapter{A Generalized form of Selberg's Sieve}\label{chap11}%11

IN\pageoriginale CHAPTER \ref{chap9} we discussed a simple version of
Selberg's sieve along with two particular cases, corresponding to the
choices 
\begin{equation*}
 \omega(p) = 1 \text{ and } \omega(p) = \frac{P}{p-1} (\text{for} p \in
 \mathfrak{p}),\tag{11.1}\label{eq11.1} 
\end{equation*}
and applied the latter, in the next chapter, to obtain the important
Theorem \ref{chap10-thm10.3} for the purposes of Chapter
\ref{chap13}. In this chapter we 
continue the theme of the small sieve of Selberg with a view to
enunciating it in its best form (in a certain sense), with is useful
also in the proofs of the results of Chapter \ref{chap13}. However, our
account of this aspect of the Selberg's sieve here will be sketchy
(with relevant references being included in the Notes). 

At the out we recall that Selberg's Theorem \ref{chap9-thm9.1} was
proved subject 
to only the condition $(\Omega_1)$ (cf. \eqref{eq9.16} a) which
restriction can also stated in the form   
\begin{equation*}
0 \le \omega (p) \le \delta p\quad \text{for some}\quad \delta <
1.\tag{11.2} \label{eq11.2} 
\end{equation*}
(Though \eqref{eq11.2} does not make Selberg's sieve `too large a
sieve', nevertheless it leaves the sieve a `large' one still.)
However, to effectively deal with the function $G(z)$ and the
remainder terms of Theorem $9.1$ one needs information concerning the
average order of magnitude of $\omega$ (and also of $R_d$'s). In this
connection the condition  
 \begin{equation*}
(\Omega_2 (k, L))  -L \le \sum_{w \le p < z} \frac{\omega (p) \log
     p}{p} - k \log (\frac{z}{w}) \le A_2 \text{for} 2 \le w \le
   z,\tag{11.3}\label{eq11.3} 
\end{equation*}
which tells us that $\omega$ is `on the average' equal to $\kappa$ is
useful (when $R_d$'s are `small' atleast on the average) for obtaining
both the upper and lower estimates for\pageoriginale 
$S(\mathscr{A}, \mathfrak{p}
,z)$. Then the $O$-constants in these estimates are allowed to depend on
$A_1 (\text{from}\Omega_1))$ and $k$ (and hence, comparing with
\eqref{eq11.2}. the sieve is `small') $A_2$ (from $(\Omega_2(k, L))$), (and
some constants implicit in the restrictions on $R_d$'s
(cf. \eqref{eq11.62})) 
but not on any other parameter involved (in particular, independent of
$L$, which is of the nature of an error term, in practice thereby
requiring a separate consideration). 

The constant $\kappa$ in \eqref{eq11.3} is called the `dimension' of
the sieve problem. 

As we shall see later the equation of obtaining lower bounds for our
sifting function $S(\mathscr{A},\mathfrak{p},z)$ can be linked up, in
a significant way with the problem of finding good upper estimates for
it. Accordingly, we now deal with the latter problem. First, we
mention that when one combines for this purpose, Theorem
\ref{chap9-thm9.1} with $\Omega_2 (\kappa,L)$, or instead even with the one
- sided restriction   
\begin{equation*}
(\Omega_2 (\kappa)) \quad \sum_{w \le p < z}  \frac{\omega(p) \log p}{p}
  \le \kappa \log (\frac{z}{w}) + A_2 \quad\text{if}\quad 
2 \le w \le z,\tag{11.4} \label{eq11.4}
\end{equation*}
the results obtained are quite satisfactory. In fact, an elementary
reasoning gives, under the condition $(\Omega_2(k))$ (in addition to
$(\Omega_1)$), the estimate 
\begin{equation*}
\frac{1}{G(z)}\ll W (z)\tag{11.5}\label{eq11.5}
\end{equation*}
for the function $G(z)$ of Theorem \ref{chap9-thm9.1} in terms if
$W(z)$ defined by \eqref{eq9.20}, and with a little more effort, on
using $(\Omega_2(\kappa,L))$ in place of $\Omega_2(\kappa)$ here, one obtains  
\begin{equation*}
\frac{1}{G(z)} = W(z) e^{\gamma \kappa} \Gamma (\kappa + 1) \big \{ 1+ O
(\frac{L}{log z}) \big \}.\tag{11.6}\label{eq11.6} 
\end{equation*}

Thus we get, by means of (the last inequality of) Theorem
\ref{chap9-thm9.1} the following two theorems (respectively): 

\setcounter{section}{11}
\setcounter{theorem}{0}
\begin{theorem}\label{chap11-thm11.1}%the 11.1
$(\Omega_1)$, $(\Omega_2 (\kappa)):$
\begin{equation*}
S(\mathscr{A},\mathfrak{p}, z) \ll X W (z) + \sum_{\substack{d \leq z^2
    \\ (d, \mathfrak{p}) = 1}} \mu^2 (d) 3^{\nu (d)} | R_d
|\tag{11.7}\label{eq11.7}  
\end{equation*}
\end{theorem}
and\pageoriginale
\begin{theorem}\label{chap11-thm11.2}%the 11.2
$(\Omega_1)$, $(\Omega_2 (\kappa, L)):$
\begin{equation*}
S(\mathscr{A}, \mathfrak{p}, z) \le X W (z)e^{\gamma \kappa} \Gamma (k+1)
\big \{ 1+O (\frac{L}{\log z}) \big \}  + \sum_{\substack{d < z^2
    \\ (d, \bar{\mathfrak{p}}) = 1}} \mu^2 (d) 3^{\nu (d)}| R_d
|.\tag{11.8} \label{eq11.8}
\end{equation*}
\end{theorem}

Here, again, we point out the $O$-constants in these two theorems depend
atmost on $A_1$, $A_2$ and $\kappa$ inherent in $(\Omega_1)$ and
$\Omega(\kappa, L)$ (or $\Omega_2(\kappa)$). 

Next, in accordance with the remark involving \eqref{eq11.3}, we consider
the question of the magnitude of $R_d$'s There are many cases in which
one has the following information 
\begin{equation*}
(R) \qquad | R_d | \le \omega(d) \quad \text{if} \quad \mu(d) \neq 0
  \text{ and } (d. \bar{\mathfrak{p}}) = 1. \qquad \tag{11.9}\label{eq11.9} 
 \end{equation*} 
 
In such a situation one readily obtains, from (the second inequality
of) Theorem \ref{chap9-thm9.1} and \eqref{eq11.5}. 
 \begin{theorem}\label{chap11-thm11.3}%the 11.3
$(\Omega_1)$, $(\Omega_2 (\kappa))$, $(R)$:  For any $A > 0$
\begin{equation*}
S(\mathscr{A}, \mathfrak{p},z) \ll X \prod_{p < z} (1-\frac{\omega(p)}
{p})\text{ if } z \le X^A,\tag{11.10}\label{eq11.10} 
\end{equation*}
where the $\ll$-constant depends almost on $A$, $A_1$, $A_2$ and $\kappa$.  
\end{theorem}

In the literature, usually, the phrase `by Brun's sieve$\dots$'
refers to the statement \eqref{eq11.10}. Here notice that Theorem
\ref{chap11-thm11.3} is a available  the more convenient condition,
instead of $(\Omega_2(\kappa))$,  
\begin{equation*}
(\Omega_0)  \hspace{3cm} \omega(p) \le A_0 \hspace{3cm}
  \tag{11.11}\label{eq11.11} 
 \end{equation*} 
(since $(\Omega_0))$ implies $(\Omega_2 (\kappa)))$. Similar to Theorem
\ref{chap11-thm11.3}, on using \eqref{eq11.6} with Theorem
\ref{chap9-thm9.1} (cf. \eqref{eq10.8}), one has the explicit 

\begin{theorem}\label{chap11-thm11.4}
$(\Omega_1)$, $(\Omega_2 (\kappa, L))$, $(R)$:
\begin{gather*}
S (\mathscr{A}, \mathfrak{p},z) \le \Gamma (\kappa+1) \prod_{p} \{ (1-\frac
{\omega(p)} {p}) (1 -\frac{1}{p})^{-k}\} \frac{X}{\log^{k{_z}}}\\
 \{ 1+O
(\frac{\log \log  3z+L} {\log z})\} \; \text{ if } \; z \le
X^{^{\frac{1}{2}}},\tag{11.12}\label{eq11.12} 
\end{gather*} 
where\pageoriginale the infinite product converges and the $O$-constant
depends atmost on $A_1$, $A_2$ and $\kappa$.  
\end{theorem}

On the other hand, in absence of the information $(R)$, as happens
with more delicate problems, one has to seek atleast an average result
about $R_d$'s (cf. the remark containing \eqref{eq11.3}). It is at this stage
that the Bombieri-type results (cf. Chapter \ref{chap6}, {\bf 3}) 
are effective. It is easily seen from the last sums in Theorem
\ref{chap11-thm11.1} and \ref{chap11-thm11.2} that 
the size of $z^2$ (which is about  
\begin{equation*}
\frac{\sqrt{X}} {\log^C X}\tag{11.13}\label{eq11.13}
\end{equation*}
usually) is very important, since a smaller choice of
$z$ increases the leading  term view of the factor $W(z)$. For making
this remark a little more explicit notice that under $(R)$ one could
$z^2 \le X$ (cf. \eqref{eq11.12}), whereas the use of Bombieri's theorem
allows us, for example, upto the bound \eqref{eq11.13}, so that 
\begin{equation*}
z^2 \le \frac{\sqrt{X}} {\log^C X}\tag{11.14}\label{eq11.14}
\end{equation*} 
which worsens the leading term in \eqref{eq11.8} by a factor of
$2^{\kappa}$ (in view of the fact that $W_{(z)}$ behaves like
$c(\kappa) \log ^{-\kappa}_z$ under $\Omega_2 (\kappa, L)$). 
However, if we ask for a bound to the
primes represented by an (irreducible) integer-valued polynomial $F$
then we can sieve the sequence $\{F(p)\}$. Then (leaving minor
details apart) the dimension of the problem would be $1$ and one has
to use  Bombieri's theorem instead of $(R)$, thereby (apparently)
lose a factor $2$. Instead one can also sieve (as was necessary before
the availability of Bombieri's theorem) the sequence  $\{ nF(n)\}$. In
that event, the dimension becomes $2$ and consequently one loses a
factor $4$ instead of $2$. This example provides an instance of how
Bombieri's theorem permits linearizing a problem (-i.e., reduce the
dimension by one-) and thus save a factor of $2$ in the upper estimate.  

Now\pageoriginale returning to our problem of obtaining  good upper
estimates, with 
a view to achieve lower bounds (which are far more important), for $ S
(\mathscr{A}, \mathfrak{p},z)$ we find that it is helpful
(cf. \eqref{eq11.32}) to generalize the method of Chapter \ref{chap9} by the
introduction of a new parameter (which is possible because of the
dual role of $z$ there expressed through $d |P(z)$ and $d < z$)
in the following way.  

Again we start with \eqref{eq9.26} which holds true under the single
condition \eqref{eq9.24}, 
\begin{equation*}
\lambda_1  =  1,\tag{11.15}\label{eq11.15}
\end{equation*} 
and then require \eqref{eq9.27} with respect to some arbitrary 
\begin{equation*}
\xi >1,\tag{11.16}\label{eq11.16}
\end{equation*} 
instead of the $z$ inherent in $S (\mathscr{A}, \mathfrak{p},z)$, i.e., 
\begin{equation*}
\lambda_d = 0  \quad\text{for}\quad  d \ge \xiup. 
\tag{11.17}\label{eq11.17} 
\end{equation*}
(Note that \eqref{eq11.17} is consistent with \eqref{eq11.15} because
of \eqref{eq11.16}.) Proceeding as in Chapter \ref{chap9} one is now
led to the choices  
\begin{equation*}
\lambda_d : = \mu (d) \prod_{p|d} \frac{p} {p-\omega(p)} . \frac{G_d
  (\frac{\xi} {d}, z)} {G (\xi,z)},\tag{11.18}\label{eq11.18} 
\end{equation*} 
where
\begin{equation*}
G_k (x,z) : = \sum_{\substack{d < x\\ d|P(z)\\ (d,k) = 1}} g(d), ~ G
(x,z):= G_1(x,z) ~\text{for} ~ 0 < x \in
\mathbb{R}\tag{11.19}\label{eq11.19}  
\end{equation*} 
are generalizations of the functions in \eqref{eq9.22}
(and \eqref{eq9.21}). Also note here that \eqref{eq11.18} includes both
\eqref{eq11.15} and \eqref{eq11.17}. Now again, as before, one has   
\begin{equation*}
|\lambda_d | \le 1.\tag{11.20}\label{eq11.20}
\end{equation*}

Further, with the choice \eqref{eq11.18}, we get \eqref{eq9.26} with 
\begin{equation*}
\sum_1 = \frac{1} {G (\xi,z)} \tag{11.21}\label{eq11.21}
\end{equation*} 
and\pageoriginale also, corresponding to \eqref{eq9.31},
\begin{equation*}
\sum_2 \le \sum_{\substack{d_\nu \xi \\ d_\nu|P(z) \\ \nu = 1,2}}
\big| R_{d_1, d_2} \big| \le \sum_{\substack{d < \xi{^2} \\ d| P(z)}}
3 ^{\nu(d)}{_{|R_d|}},\tag{11.22}\label{eq11.22} 
\end{equation*} 
because of \eqref{eq11.20}, \eqref{eq11.17} and \eqref{eq9.32}. Now we
see that $\xi$ 
has, so to speak, taken over from $z$ the role of controlling the
order of magnitude of the remainder sum (cf. the remark preceding the
statement of \eqref{eq11.15}). 

Also we would formulate (cf. again \eqref{eq11.32}) Theorem
\ref{chap9-thm9.1} generalized further so as include all
$\mathscr{A}_q$'s for $q's$ restricted by  
\begin{equation*}
(Q) \hspace{1.2cm} \mu (q) \neq 0,  (q,P(z))=1,
  (q,\overline{\mathfrak{p}}) = 1. \hspace{1.2cm} 
\tag{11.23}\label{eq11.23} 
\end{equation*} 

Here we stress that $\mathscr{A}_q$'s are related to $\mathscr{A}$
through the approximations required by \eqref{eq9.15} (and \eqref{eq9.9}) and
consequently this step is not merely a change of notation. The
condition (Q) ensures, for this step, that the only changes required
in the previous generalization of Theorem \ref{chap9-thm9.1} are the
replacements of  
\begin{equation*}
\sum_1\quad \text{by}\quad \frac{\omega(q)}{q} \sum_1
\tag{11.24}\label{eq11.24} 
\end{equation*} 
and of
\begin{equation*}
R_d's \quad \text{by} \quad R_{qd}'s.\tag{11.25}\label{eq11.25} 
\end{equation*} 

Thus from \eqref{eq9.26}, \eqref{eq11.21}, \eqref{eq11.24},
\eqref{eq11.22} and \eqref{eq11.25},
in view \eqref{eq11.16}, the required generalization of Theorem
\ref{chap9-thm9.1} follows; namely, one has  

\begin{theorem}\label{chap11-thm11.5}
$(\Omega_1)$, $(Q)$: For any real number $\xi > 1$,
\begin{equation*}
S(\mathscr{A}_q, \mathfrak{p},z) \le \frac{\omega(q)} {q} \frac{X} {G
  (\xi,z)} + \sum_{\substack{d<\xi^2 \\ d|P(z)}} 3^{\nu(d)}
|R_{qu}|. \tag{11.26}\label{eq11.26} 
\end{equation*} 

Actually, for $q = 1$ and $\xi =z$, \eqref{eq11.26} is the second
inequality of Theorem \ref{chap9-thm9.1} since 
\begin{equation*}
\mathscr{A}_1 = \mathscr{A}\tag{11.27}
\end{equation*}\pageoriginale
and
\begin{equation*}
G (z,z) = G(z).\tag{11.28}\label{eq11.28}
\end{equation*} 
\end{theorem}

Here, by imposing the condition $(\Omega_2(\kappa))$ alone we cannot expect
to get very useful results. However,as a starting point for the method
of obtaining a lower bound for our sifting function, we use
$(\Omega_2(\kappa))$ only to derive a simple estimate for
$\dfrac{1}{G(\xi,z)}$ so that it follows from Theorem
\ref{chap11-thm11.5}.   


\begin{theorem}\label{chap11-thm11.6}%the11.6
$(\Omega_1)$, $(\Omega_2(\kappa))$, $(Q)$:  For
\begin{equation*}
\tau : = \frac{\log \xi^2} {\log z} \ge 2,\tag{11.29}\label{eq11.29} 
\end{equation*}
there holds
\begin{equation*}
 S (\mathscr{A}_q, \mathfrak{p}, z ) \le \frac{\omega(q)} {q} X W(z)
 \{ 1+ O (\exp \{-\frac{\tau} {2} (\log \frac{\tau} {2} +2)\}) \}
 +\sum_{\substack{d<\xi^2\\ d|P(z)}}
 3^{\nu(d)} |R_{qd}|.\tag{11.30} \label{eq11.30} 
\end{equation*} 
\end{theorem}

Now we are in a position to point out how by means of a certain very
effective combinatorial argument (in the from of identities
concerning our function $S(\mathscr{A}_q,\mathfrak{p},z)$ and
$W(z))$, which is really a result about arrangements from mathematical
logic,one can obtain some lower bounds and also improved estimates for
our sifting functions from the estimates of Theorem \ref{chap11-thm11.6}.  

Buchstab was the first to notice the fruitful utility of this
combinatorial result for the purposes of sieve methods. We state this
result as 

\setcounter{section}{11}
\setcounter{lemma}{0}
\begin{lemma}\label{chap11-lem11.1}%lem 11.1
$(Q)$:~ If 
\begin{equation*}
2 \le z_1 \le z,\tag{11.31}\label{eq11.31}
\end{equation*} 
then we have 
\begin{equation*}
 S (\mathscr{A}_q, \mathfrak{p}, z )  = S (\mathscr{A}_q,
 \mathfrak{p}, z_1)-\sum_{\substack{z_1 \le p < z\\ p \in
     \mathfrak{p}}} S (\mathscr{A}_{qp},
 \mathfrak{p},p)\tag{11.32}\label{eq11.32}  
\end{equation*} 
as well as
\begin{equation*}
W(z) = W(z_1) - \sum\limits_{z_1 \le p < z} \frac{\omega(p)} {p}~
W(p)\tag{11.33}\label{eq11.33} 
\end{equation*}\pageoriginale
\end{lemma}

\begin{proof}%pro
Let $p_1, p_2 \ldots$ be all the primes belonging of $\mathfrak{p}$,
written in their natural order, which are greater than or equal to
$z_1$. If $z \le p_1$ we have, by \eqref{eq9.15} and \eqref{eq9.12} 
\begin{equation*}
P(z) = P(z_1)  \text{~ and~ } \omega(p) = 0 \text{~ for~ } z_1 \le p<
z.\tag{11.34}\label{eq11.34}  
\end{equation*}

This gives, by \eqref{eq9.6} and \eqref{eq9.20}, 
\begin{equation*}
 S (\mathscr{A}_q, \mathfrak{p}, z )  = S (\mathscr{A}_q,
 \mathfrak{p}, z_1) \text{ and } W(z) = W(z_1), \text{ if } z \le
 p_1.\tag{11.35}\label{eq11.35}  
\end{equation*}

Thus Lemma \ref{chap11-lem11.1} is trivially true in this case $(z_1
\le z \le p_1)$.  

Now suppose that $p_1 < z$, so that defining the integer $N$ by $p_N <
z \le p_{_{N+1}}$ we have $N \ge 1$. Then, for each integer $\nu
=1,\ldots,N$, \eqref{eq9.6} yields 
\begin{equation*}
\begin{cases}
 S (\mathscr{A}_q, \mathfrak{p}_{\nu+1})  - S (\mathscr{A}_q,
 \mathfrak{p}_\nu)= \\ 
\quad = |\{a : a \in \mathscr{A}_q, a \equiv 0 \mod p_{\nu}, (a,
P(p_\gamma)) = 1\}|=\\ 
\quad = - S (\mathscr{A}_{qp_\gamma}, \mathfrak{p},p_\nu),
\end{cases}\tag{11.36}\label{eq11.36}
\end{equation*}
and, by \eqref{eq9.20}, we have
\begin{equation*}
W (p_{\nu +1}) - W (p_\nu) = - \frac{\omega(p_\nu)} {p_\nu}
W(p_\nu).\tag{11.37}\label{eq11.37} 
\end{equation*}

Summing up the identities \eqref{eq11.36} and \eqref{eq11.37} over $\nu =
1,\ldots, N$ and observing that $S(\mathscr{A}_q, \mathfrak{p}, z )
= S (\mathscr{A}_q, \mathfrak{p}_{N+1})$ and $W(z) = W(p_{N+1})$ we
obtain \eqref{eq11.32} and \eqref{eq11.33}. This completes the proof
of the lemma. 

To see as to how \eqref{eq11.32} links up the problem of obtaining a lower
bound for sifting functions to that of having good upper estimates,
suppose that one has a  lower bound for (the larger)  $S(\mathscr{A}_q,
\mathfrak{p}, z_1)$. Then upper bounds for $S(\mathscr{A}_{qp},
\mathfrak{p}, p)$'s enable us to obtain a lower bound for (the
smaller) $S(\mathscr{A}_{q}, \mathfrak{p}, z)$. (We\pageoriginale also have a
similar remark the problem of obtaining upper bounds by means of
\eqref{eq11.32}.) However, the significant part of \eqref{eq11.32} is its
iterative  aspect consisting of using \eqref{eq11.32} to (some of) the
$S$-functions with (the respective) $p$'s in place of $z$, thereby
obtaining more terms of both signs which in turn (after `many'
iterations) would provide a more effective scheme for the above
procedure. Actually , it is again  this iterative process because of
which \eqref{eq11.32} is stated with a (general) $q$ (rather than with $q =
1$).  

Since, to start with, we do not have general lower bound for  $S
(\mathscr{A}_q,\break \mathfrak{p}, z_1)$  one can only make the choice   
\begin{equation*}
z_1 = 2 \tag{11.38}\label{eq11.38}
\end{equation*}
so that, by \eqref{eq9.15},
\begin{equation*}
S (\mathscr{A}_q, \mathfrak{p}, z_1) = |\mathscr{A}_q| =
\frac{\omega(q)} {q} X + R_q.\tag{11.39}\label{eq11.39} 
\end{equation*}

Then, on multiplying \eqref{eq11.33} by $\dfrac{\omega(q)}{q}X$ and
subtracting the result from \eqref{eq11.32} we are led to deal with the
remainders $R_{qd}$'s only. Now, in order not to accumulate too many
terms from the upper estimate for the sum in \eqref{eq11.32}, we use Theorem
\ref{chap11-thm11.5} for  $S (\mathscr{A}_{qp}, \mathfrak{p},p )$ with $\xi^2$
replaced $\xi_0^2 / p$  (and also with $p$ instead of $z$, which
ensures $(Q)$ for $qp$ in place of $q$). Thus we arrive at a first
step lower bound corresponding to the upper bound of Theorem
\ref{chap11-thm11.6}, and so we state this result in the of an
asymptotic equality:  
\end{proof}

\begin{theorem}\label{chap11-thm11.7}%the11.7
$(\Omega_1)$, $(\Omega_2(\kappa))$, $(Q)$:  For
\begin{equation*}
\tau : = \frac{\log \xi^2} {\log z} \ge 2.\tag{11.40}\label{eq11.40} 
\end{equation*}
there holds
{\fontsize{10pt}{12pt}\selectfont
\begin{equation*}
 S (\mathscr{A}_q, \mathfrak{p}, z ) = \frac{\omega(q)} {q} X W(z)
 \{ 1+ O (\exp \{-\frac{\tau} {2} (\log \frac{\tau} {2} +2)\})\}
 + \theta \sum_{\substack{d<\xi^2\\ d|P(z)}} 3^{\nu(d)} |R_{qd}| \cdot
 |\theta|\le  1. \tag{11.41}\label{eq11.41}  
\end{equation*}}\relax 
\end{theorem}

It should be mentioned here that, with $q = 1$ and a suitable choice
for $\xi$. 

One\pageoriginale easily obtains from Theorem \ref{chap11-thm11.7}, under the
condition $(R)$, the 
so-called `Fundamental Lemma' (cf. Kubiliyus \cite{key2}, Lemma 1.4) 

\begin{theorem}\label{chap11-thm11.8}%the11.8
$(\Omega_1)$, $(\Omega_2(k))$, $(R)$:  For
\begin{equation*}
u : = \frac{\log X}{\log z} \ge 1.\tag{11.42}\label{eq11.42}
\end{equation*}
there hold
\begin{equation*}
S(\mathscr{A},\mathfrak{p},z) = X W(z) \{ 1+O(e^{ -\frac{1}{2} u \log
  u}) + O(e ^{- \sqrt{\log X}})\}\tag{11.43}\label{eq11.43} 
\end{equation*}
and 
\begin{equation*}
S(\mathscr{A},\mathfrak{p},z) = X W(z) \{ 1+O(e^{ -\frac{1}{2} u
}\}.\tag{11.44}\label{eq11.44} 
\end{equation*}
\end{theorem}

Observe that the preceding results (i.e., Theorem \ref{chap11-thm11.6},
\ref{chap11-thm11.7} and \ref{chap11-thm11.8}) are significant only if $u$
(or $\tau$) is large (which 
means that $z$ should be small in comparison with $X$ (or
$\xi$)). The reason for this limitation is, apart from the weak
condition $(\Omega_2(\kappa))$, mainly due to the fact that Lemma
\ref{chap11-lem11.1} 
has been used with a `trivial' choice for $z_{1}$ in deriving the
lower estimate. However, using the
stronger $\Omega_2(\kappa,L))$ instead of
$(\Omega_2(\kappa))$ we obtain a more precise information about $G (\xi,
z)$. And still more important is that Theorem \ref{chap11-thm11.7}
enables us to employ Lemma \ref{chap11-lem11.1} with a `non trivial'
choice for $z_1$ (cf. \eqref{eq11.52}). 

From this point onwards we shall confine ourselves to the case of
dimension  
\begin{equation*}
\kappa = 1,\tag{11.45}\label{eq11.45}
\end{equation*} 
and we owe some explanation for doing so. First of all, the upper
estimates for $G(\xi,z)$ under $(\Omega(\kappa,L))$ in the case of general
dimension $\kappa$ (and then so all for $S(\mathscr{A},\mathfrak{p},z)$ via
Theorem \ref{chap11-thm11.5}) become quite complicated and further, they do not
further, they do not yield satisfactory results (apart from the
particular cases of $\kappa = 1$ and $\kappa = \dfrac{1}{2}$), when
applied for obtaining lower bounds by means of Lemma
\ref{chap11-lem11.1}. Also, when one is 
interested solely in finding upper estimates (in the most interesting
questions)\pageoriginale the generalized Theorem \ref{chap11-thm11.5}
has no advantage over (the 
simple) Theorem \ref{chap9-thm9.1}. Finally, we have that most of the
prominent problems in prime number theory which can be attacked by Selberg's
sieve method are dimension 1. 
 
Now on imposing $(\Omega_2 (1,L))$ one obtains

\begin{lemma}\label{chap11-lem11.2}%lem 11.2
$(\Omega_1)$, $(\Omega_2(1,L))$: Let
\begin{equation*}
\tau : = \frac{\log \xi^2}{\log z} > 0.\tag{11.46}\label{eq11.46}
\end{equation*}
\end{lemma}

Then holds
\begin{equation*}
\frac{1}{G(\xi,z)} = W (z) \{ F_0(\tau) + O (\frac{L} {\log z} (
\tau^3 + \tau^{-2}))\},\tag{11.47}\label{eq11.47} 
\end{equation*}
where $F_0(\tau)$ is defined by
\begin{equation*}
F_0(\tau) = \frac{2e^\gamma}{\tau} \text{ for } o < \tau \le
2,\tag{11.48} \label{eq11.48} 
\end{equation*}
and by the differential-difference equation
\begin{equation*}
(\frac{1}{\tau F_0(\tau)})' = - \frac{1}{\tau^2 F_0 (\tau-2)}
  \text{for} ~\tau >2.\tag{11.49}\label{eq11.49} 
\end{equation*}

If we apply lemma \ref{chap11-lem11.2} in Theorem \ref{chap11-thm11.5}
we obtain, corresponding to Theorem \ref{chap11-thm11.6}, the following 

\begin{theorem}\label{chap11-thm11.9}%the 11.9
$(\Omega_1)$, $(\Omega_2(1,L))$, $(Q)$:  For any real number $\xi > 1$ and
\begin{equation*}
\tau : = \frac{\log \xi^2} {\log z} (> 0),\tag{11.50}\label{eq11.50} 
\end{equation*}
there holds
\begin{equation*}
 S (\mathscr{A}_q, \mathfrak{p}, z ) \le \frac{\omega(q)} {q} X W(z)
 \{ F_0 (\tau) + O (\frac{L}{\log z}(\tau^3 + \tau^{-2}))\}
 +\sum_{\substack{d<\xi^2\\ d|P(z)}}
 3^\nu(d){_{|R_{qd}|}},\tag{11.51}\label{eq11.51}    
\end{equation*} 
where $F_0(\tau)$ is defined by \eqref{eq11.48} and \eqref{eq11.49}.
\end{theorem}

We can now repeat that above Buchstab-procedure (cf. description
preceding Theorem \ref{chap11-thm11.7}) to obtain a general
non-trivial lower bound 
for\pageoriginale our sifting functions. To this end we choose in
\eqref{eq11.32}  
\begin{equation*}
z_1 =\exp \{ \frac{\log \xi} {\log \log \xi}\}\tag{11.52}\label{eq11.52}
\end{equation*} 
(and also assume $\xi$ to be sufficiently large; indeed, we can even
suppose that $\xi \ge z$ since the result below (Theorem
\ref{chap11-thm11.10}) is
otherwise trivial because of \eqref{eq11.59}). This choice enables us to
employ Theorem \ref{chap11-thm11.7}  for a lower estimate of  $S(\mathscr{A}_q,
\mathfrak{p}, z_1)$ and for the remaining part of the right-hand side
in \eqref{eq11.32} we  can use now Theorem \ref{chap11-thm11.9}
instead, but again with  
\begin{equation*}
\xi^2 \text{~ replaced by~ } \frac{\xi^2}{p} \text{~ and~ } z \text{~
  replaced by~ }p \tag{11.53}\label{eq11.53} 
\end{equation*} 
for the same reasons as before. However, here we encounter an
additional difficulty due to the presence of factors $F_0 (
\dfrac{\log(\xi^2 / p)}{p})$ (instead of 1 in the previous case) stemming
from our use of Theorem \ref{chap11-thm11.9} with \eqref{eq11.53}. 
This difficulty is overcome by deriving from \eqref{eq11.33} the following  

\begin{lemma}\label{chap11-lem11.3}%lem 11.3
$(\Omega_1)$, $(\Omega_2)1,L)):$ Suppose that
\begin{equation*}
2 \le z_1 \le z \le \xi,\tag{11.54}\label{eq11.54}
\end{equation*} 
and let $\psi (t)$ be a non-negative, monotonic and continuous
function for $t \ge 1$. Further, define 
\begin{equation*}
M: \max\limits_{z_1 w\le < z} \psi (\frac{\log (\xi^2 / w)} {\log
  w}),\tag{11.55}\label{eq11.55} 
\end{equation*}
\end{lemma}

Then holds 
\begin{gather*}
\sum_{z_1 \le p < z} \frac{\omega(p)}{p} W (p) \psi ( \frac{\log (
  \xi^2 / p)}{\log p})\\
= W(z) \frac{\log z}{\log \xi p} \int\limits^{\frac{\log \xi^2}{\log
  z_1}}_{\frac{\log \xi^2}{\log z}} \psi (t-1)dt+O ( 
\frac{LMW(z) \log z}{\log^2 z_1}).\tag{11.56}\label{eq11.56} 
\end{gather*}

Thus using Lemma \ref{chap11-lem11.3} (instead of \eqref{eq11.33})
along with \eqref{eq11.32} one obtain, by the above procedure, without
any more difficulty the required  

\begin{theorem}\label{chap11-thm11.10}
 $(\Omega _1)$,\pageoriginale $(\Omega_2(1, L)),(Q)$: For any real
  number $\xi>1$ and  
\begin{equation*}
\tau : = \frac{\log \log \xi^2}{\log z }(> 0).\tag{11.57}\label{eq11.57}
\end{equation*}
there holds
\begin{equation*}
S)\mathscr{A}_q.  \mathfrak{p} ,z) \ge \frac{\omega (q)}{q}X
W(z)\bigg\{ f_0 (\tau) + O (\frac{L(\log \log 3 \xi )^5}{\log
  \xi})\bigg\}- \sum_{\substack{d< \xi ^2\\d|P(z)}}
3^{\nu(d)}|R_{qd}|,\tag{11.58}\label{eq11.58} 
\end{equation*}
where $f_0(u)$ is defined by  
\begin{equation*}
f_0(\tau) =0   \text{ for }    0 < \tau\leq \nu_0 = 2.06 \cdots,
\tag{11.59}\label{eq11.59} 
\end{equation*}
and by, with the function $F_0$ of Lemma \ref{chap11-lem11.2},
\begin{equation*}
(\tau f_0 (\tau))' =F_0(\tau -1) \text { for } \tau \ge
  \nu_0. \tag{11.60}\label{eq11.60} 
\end{equation*}
\end{theorem}

Now, before starting the iteration of the Buchstab-procedure (cf. the
remarks made subsequent to the proof to Lemma \ref{chap11-lem11.1}), we make the
result of Theorem \ref{chap11-thm11.10} explicit (for $q=1$) by
imposing the following condition (about $R_d$'s):	 

Suppose that there are constants 
\begin{equation*}
0< \alpha \leqslant 1,    \quad A_3 (\geqslant 1) , \quad
A_4(\geqslant 1)\tag{11.61}\label{eq11.61} 
\end{equation*}
such that
\begin{equation*}
 (R(1, \alpha))\quad\sum_{d< \frac{X^\alpha}{{\substack{ \log
          ^{A_{3}} X\\(d,\mathfrak{p})=1}}}}\mu^2 (d) 3
  ^{\nu(d)}|R_d|\leqslant A_4 \frac{X}{\log ^2 X} \text{~ for~ } X
  \geqslant 2. \tag{11.62}\label{eq11.62} 
\end{equation*}

Then one gets from Theorem \ref{chap11-thm11.10} in the case $q=1$, on
making the choice  
\begin{equation*}
\xi ^2 = \frac{X^\alpha}{\log ^{A_{3}}X}  \tag{11.63}\label{eq11.63}
\end{equation*}
and noting   that $d|P(z)$  implies that $\mu(d)\neq 0$ and
$(d,\bar{\mathfrak{p}})=1$, 


\begin{theorem}\label{chap11-thm11.11}
$(\Omega_1)$, $(\Omega_ 2(1,L))$, $(R(1, \alpha))$:
\begin{equation*}
S(\mathscr{A},\mathfrak{p},z)\geqslant X W (z) \bigg\{ f_0 (\alpha
\frac{\log X}{\log z})+O(\frac{L (\log \log 3 X)^5}{\log X})\bigg\},
\tag{11.64}\label{eq11.64} 
\end{equation*}\pageoriginale
where  $f_0(\tau)$ is defined by \eqref{eq11.59} and \eqref{eq11.60}.
\end{theorem}

Returning now to the remarkable  iterative aspect of the Buchstab-
procedure we see that on using Theorem \ref{chap11-thm11.9}, 
with an appropriate\break
choice of $z_1$, one has an upper bound for the first term on the
right- hand side in \eqref{eq11.32} and also that using Theorem
\eqref{eq11.10}, with the replacements mentioned in
\eqref{eq11.53}. in combination with 
Lemma \ref{chap11-lem11.3} 
(for $f_0$ in place of $ \psi $) there follows a lower
bound for the sum in \eqref{eq11.32}. Thus we arrive at another form of
Theorem \ref{chap11-thm11.9}, where $F_0 (\tau$ is  replaced by some
other (similar) function $F_1(\tau)$ and this  in turn also yields
another form of 
Theorem \ref{chap11-thm11.10}, where $f_0 (\tau)$  is replaced by an 
$f_1 (\tau )$ (related to $F_1(\tau)$). 

Continuing this procedure we are led to results of the type Theorem
\ref{chap11-thm11.9} and \ref{chap11-thm11.10}, with (at the $\tau$ th
step) the following pair of functions (instead of $F_0 (\tau) f_0
(\tau)$ respectively)  
\begin{equation*}
F_\mu (\tau), f_\mu (\tau ), \quad \mu =0,1,2,\cdots,
\tag{11.65}\label{eq11.65} 
\end{equation*} 
where (analogous to the first step as indicated in \eqref{eq11.88})
\begin{equation*}
f_{\mu} (\tau): =
\begin{cases}
f_0 (\tau )= 0 \text{ for } \tau \leq \nu_\mu,  \\1- \frac{1}{\tau}
\int\limits_\tau ^\infty (F_\mu (t-1) \text { dt for } \tau \geq \nu
_\mu, 
\end{cases}\tag{11.66}\label{eq11.66} 
\end{equation*}
with the number $\nu _\mu$ defined by 
\begin{equation*}
\frac{1}{\nu_\mu}\int^{\infty}_ {\nu_{\mu}} (F_\mu (t-1) -1)  dt =1
\tag{11.67}\label{eq11.67} 
\end{equation*}
(cf. the remark following \eqref{eq11.88}) and similarly
 \begin{equation*}
F_{\mu +1}(\tau): = 
\begin{cases}
F_0 (\tau) \quad \text{ for } \tau \leq \nu '_\mu , \\
1- \dfrac{1}{\tau}  \int^{\infty}_{\tau}(f _\mu (t-1)-1) \text{ dt
  for }  \tau \ge  \nu ' _\mu 
\end{cases}\tag{11.68} \label{eq11.68}
 \end{equation*} 
with\pageoriginale suitably chosen numbers $\nu'_\mu$ for $\mu =
0,1,2,\ldots$  

The power of this procedure is demonstrated by the surprising fact
that, at each step, the quality of the respective forms of the
Theorems \ref{chap11-thm11.9}  and \ref{chap11-thm11.10}
improves. Further, the sequence of 
numbers $\{ \nu_\mu\}$  converges to 2 from above, as also does
$\{\nu'_ \mu\}$, and the pair of functions $\{F_\mu , f_\mu\}$
converges to a pair of limit functions $\{ F,f\}$ converges to a pair
of limit functions $\{ F,f\}$, as $\mu \rightarrow \infty:$ 
 \begin{equation*}
\lim\limits_{\mu \to \infty} \nu_\mu =2 = \lim\limits_{\mu \to \infty}
\nu ' _\mu , \lim\limits_{\mu \to \infty} F_\mu (\tau), = F (\tau),
\lim\limits_{\mu \to \infty}f_\mu =f(\tau). \tag{11.69}\label{eq11.69} 
 \end{equation*} 
 
Now, from \eqref{eq11.69}, \eqref{eq11.66}, \eqref{eq11.48} and
\eqref{eq11.59}, we find that 
  \begin{equation*}
F(\tau ) = \frac{2e ^\gamma}{\tau}, f (\tau)= 0 \text{ for } 0< \tau
\leq 2 \tag{11.70}\label{eq11.70} 
 \end{equation*} 
and 
\begin{equation*}
(\tau F(\tau )) ' = f(\tau -1) \text{~ and~ } (\tau f(\tau )) ' =  F(\tau
  -1), \text{~ for~ } \tau \ge 2, \tag{11.71}\label{eq11.71} 
 \end{equation*} 
 which gives on integrating from $2$ to $u$ 
\begin{equation*}
u F (u) -2 e^\gamma = \int ^{u-1}_1 f(t) dt \text{ and } uf (u)  =\int
_1^{u-1} F(t)dt, \text{ for } u \ge 2.\tag{11.72}\label{eq11.72} 
 \end{equation*} 
 
Then we obtain from \eqref{eq11.72}, in particular, (cf. \eqref{eq11.70})
\begin{equation*}
F(u)= \frac{2 \epsilon^\gamma}{u} \text{ for } 0<u \leq	  3
\tag{11.73}\label{eq11.73} 
\end{equation*} 
and (so) further 
\begin{equation*}
f(u) = \frac{2e^\gamma}{u} \log (u-1) \text{ for } 2 \leq u \leq
4.\tag{11.74}\label{eq11.74} 
\end{equation*} 

Also, we have
\begin{gather*}
F(u)\ge 0, f(u) \ge 0, F(u) \downarrow , f(u) \uparrow (\text{for } u> 0)\\
\text{and}\quad \lim\limits_{u \to \infty} F(u)= 1 = \lim\limits_{u \to
  \infty} f(u) \tag{11.75}\label{eq11.75} 
\end{gather*} 
and 
\begin{gather*}
0<  F(u_1) -F(u_2) \leq F(\delta) \frac{u_2- u_1}{u_1} ,\\ 0 \leq
f(u_2)-f(u_1)\leq 2e ^\gamma \frac {u_2- u_1}{u_1}, \text{ for } 0<
\delta \leq u_1 < u_2. \tag{11.76}\label{eq11.76} 
\end{gather*}
 
Actually,\pageoriginale when one knows these results (about $F$, $f$)
there is no need to consider the sequence of pairs of functions $\{
F_\mu , f_\mu\}$ (and so also  the numbers $ \nu_\mu$, $\nu '_\mu $). One
can instead iterate \eqref{eq11.32} as well as Lemma
\ref{chap11-lem11.3} (with $\psi=F$ 
and $\psi =f$) and apply the Buchstab-procedure (as was done, for
instance, to obtain Theorem \ref{chap11-thm11.10}) only once. More
precisely, in the 
iterated version of \eqref{eq11.32} among the various functions $S$
occurring with different signs, one can apply Theorem
\ref{chap11-thm11.7} for those 
which are within its domain (of applicability). For those of the
remaining $S's$ which are to  be  estimated from above one can use
Theorem \ref{chap11-thm11.9} while the trivial lower estimate $S \ge
0$ in used for 
the rest. It remains only to prove that this process converges.
However, this can be done if the dimension $\kappa$ satisfies 
 \begin{equation*}
\kappa < \kappa_0, \tag{11.77}\label{eq11.77}
\end{equation*} 
where $\kappa_0$ is some constant greater than 1. Therefore we can
succeed in our case \eqref{eq11.45}. 
  
 Now we can state the final result of the Buchstab-procedure obtained
 in the manner mentioned in our previous remark. 
 
\begin{theorem}\label{chap11-thm11.12}
$(\Omega_1)$, $(\Omega_2 (1,L))$, $(Q)$: For
 \begin{equation*}
\xi \ge z,\tag{11.78}\label{eq11.78}
 \end{equation*} 
 there hold
\begin{equation*}
S(\mathscr {A}_q,\mathfrak{p},z) \leq \frac{\omega(q)}{q} X W (z)
\bigg\{ F (\frac{\log \xi ^2}{\log z})+ O(\frac{I,}{(\log \xi )
  ^{1/14}})\bigg\}+ \sum_{\substack {d< \xi ^2\\d|P(z)}} 3
^{\nu(d)}|R_{qd}|, \tag{11.79}\label{eq11.79} 
\end{equation*} 
and 
\begin{equation*}
S(\mathscr {A}_q,\mathfrak{p},z) \ge \frac{\omega(q)}{q} X W (z)
\bigg\{ f (\frac{\log \xi ^2}{\log z})+ O(\frac{I,}{(\log \xi )
  ^{1/14}})\bigg\}+ \sum_{\substack {d< \xi ^2\\d|P(z)}} 3
^{\nu(d)}|R_{qd}|,\tag{11.80}\label{eq11.80} 
 \end{equation*} 
where the functions $F$, $f$ are defined by \eqref{eq11.70} and
\eqref{eq11.71}, and the $O$-constants 
depend\pageoriginale atmost on $A_1$ and $A_2$.
\end{theorem}

This theorem is also true if
\begin{equation*}
1< \xi < z \text { but } \ll \xi ^ \lambda \tag{11.81}\label{eq11.81}
\end{equation*}
with a positive constant $\lambda$, in which case the $O$-constant
in \eqref{eq11.79} depends also on $\lambda$. 

Similar to Theorem \ref{chap11-thm11.11} we can obtain, in the case
$q=1$, with the same choice as \eqref{eq11.63} the following final
result from Theorem \ref{chap11-thm11.12}. 

\begin{theorem}\label{chap11-thm11.13}
$(\Omega_1)$, $(\Omega_2(1,L))$, $(R(1, \alpha))$:  For
\begin{equation*}
z\leq X, \tag{11.82}\label{eq11.82}
\end{equation*}
we have 
\begin{equation*}
S (\mathscr{A},\mathfrak{p} , z) \leq X W (z) \bigg\{ F (\alpha \frac
{\log X}{\log z}) + O  (\frac{L}{(\log X)^{1/_{14}}})\bigg\},
\tag{11.83}\label{eq11.83} 
\end{equation*}
and
\begin{equation*}
S( \mathscr{A}, \mathfrak{p},  z) \ge X W (z) \bigg\{ f (\alpha \frac
{\log X}{\log z}) + O (\frac{L}{(\log X)^{1/_{14}}})\bigg\},
\tag{11.84}\label{eq11.84} 
\end{equation*}
where the functions $F$, $f$ are defined by \eqref{eq11.70} and
\eqref{eq11.71}, and the $O$-constants depend atmost on $A_i$,
$i=1,2,3,4$, and $\alpha$.  
\end{theorem}

Although the functions $F$, $f$ are invariant under the
Buchstab - procedure, in view of the fact that 
\begin{equation*}
f(u)<F(u) \quad \forall u>0 \tag{11.85}\label{eq11.85}
\end{equation*} 
and the procedure  described preceding \eqref{eq11.77} it is natural to have
some doubt as to whether the qualities of our final results
(viz. Theorems \ref{chap11-thm11.12} and \ref{chap11-thm11.13}) cannot
further be improved. However, it can be shown that, for the sets 
\begin{equation*}
\mathscr{A}= \mathscr{B}_\nu : = \bigg\{ n: | 1 \leq n \leq x,
\Omega(n) \equiv
| \nu \mod  2 \bigg\}, \nu =1,2 \text{ and } \mathfrak{p} =
\mathfrak{p}_1 \tag{11.86}\label{eq11.86} 
 \end{equation*} 
(cf. \eqref{eq9.39}). the relations \eqref{eq11.83} and
\eqref{eq11.84} hold with equality signs (upto the
leading\pageoriginale term) for $\nu = 1$ and $\nu=2$ respectively and
for all values of $u:\dfrac{\log X}{\log z}> 0$ in both  cases);
(Actually, here we take $X=\dfrac{x}{2}$, $\omega(p)=1$ and slightly
modify $R(1, \alpha)$, with $\alpha=1$, via Theorem
\ref{chap11-thm11.12}). It is in this sense that the final form, as
stated in Theorem \ref{chap11-thm11.13}, of the (proper Selberg sieve
is best possible. 

\medskip 
\begin{center}
\textbf{NOTES}
\end{center}
  
They survey of the Selberg's sieve given in this chapter follows the
approach of Halberstam and Richert \cite{key1} and we refer to (Chapters
\ref{chap4}, \ref{chap5}, \ref{chap6}, \ref{chap7}, \ref{chap8}, of)
this book for the content of this chapter as well 
as for the applications of  these results. So all the references below
(unless otherwise explicitly stated) are referred to by `l,c.'. 
\begin{description}
\item[\eqref{eq11.5}:] cf. Lemma \ref{chap4-lem4.1}
\item[\eqref{eq11.6}:] cf. Lemma 5.4
 \end{description}
 
Theorem \ref{chap11-thm11.1}: cf. l.c Theorem \ref{chap4-thm4.1}.

Theorem \ref{chap11-thm11.2}: cf. l.c. Theorem \ref{chap5-thm5.2} (see
also under \eqref{eq11.12} below.) 

Theorem \ref{chap11-thm11.3}: cf. l.c. Theorem
\ref{chap2-thm2.2}. Actually, Theorem \ref{chap11-thm11.3} holds for
$z \ge X^A$  also when the restriction in the product, $p<z$, is
replaced by $p<X$.  

Theorem \ref{chap11-thm11.4}; cf. l.c. Theorem \ref{chap5-thm5.1}.

\eqref{eq11.12}: Actually, in the $O$-term of \eqref{eq11.12}  (and
also of Theorem \ref{chap11-thm11.2} $L$ can   be replaced by  min
$(L, \log z)$ so that the Theorem \ref{chap11-thm11.4} (and
respectively Theorem \ref{chap11-thm11.2}) would include Theorem
\ref{chap11-thm11.3} (and respectively Theorem
\ref{chap11-thm11.1}). This is, however, not surprising because of the
fact that the condition $(\Omega_2 (\kappa,L))$ includes
$(\Omega_2(\kappa))$. 

\eqref{eq11.13}: For more details concerning the  description here, in
connection with  

\eqref{eq11.13}: see  l.c Chapter \ref{chap5}, {\bf 5.}

Theorem \ref{chap11-thm11.5}: cf. l.c. Theorem  6.1.

Theorem \ref{chap11-thm11.6}:\pageoriginale cf. l.c. Theorem 6.2. 

Lemma \ref{chap11-lem11.1}: cf. l.c. Lemma 7.1.

Theorem \ref{chap11-thm11.7}: cf. l.c Theorem \ref{chap7-thm7.1}.

Theorem \ref{chap11-thm11.8}: cf. l.c. Theorem
\ref{chap7-thm7.2}. Also cf. l.c. Theorem \ref{chap2-thm2.5} for a
slightly stronger result derived from Brun's sieve.  
 
Lemma \ref{chap11-lem11.2}: cf. l.c. Lemma 6.1 and also
l.c.  \eqref{eq4.18}  on p. 201. 


\eqref{eq11.48}, \eqref{eq11.49}: For the function $F_0$  defined
here, it can be proved that   
\begin{equation*}
F_0 (\tau) \ge 0, F_0 (\tau ) \downarrow (\text{ for } \tau >0 )
\text{~ and~  } \lim\limits_{\tau \to \infty} F_0 (\tau )
=1. \tag{11.87}\label{eq11.87} 
\end{equation*}

For more details about $F_0 (\tau )$, cf. l.c. Chapter
\ref{chap6}, \textbf{3} (where this function occurs as $\dfrac{1}{\sigma_1
  (\tau)}$).  

Theorem \ref{chap11-thm11.9}: cf. l.c. Theorem 6.3.

Lemma \ref{chap11-lem11.3}: cf. l.c. Theorem \ref{chap7-thm7.2}.

Theorem \eqref{eq11.10}: cf. l.c. Theorem 7.3.

\eqref{eq11.59}, \eqref{eq11.60}: To  form an idea of the introduction
of this function $f_0 (\tau)$ it is useful  to  observe that, by the
procedure leading to Theorem \ref{chap11-thm11.10} here, the
contributions  to the 
leading term (apart from the factor $\dfrac{\omega(q)}{q}X W (z)$)
are, in view of Lemma \ref{chap11-lem11.3} here (with $\psi (t)=F_0 (t)$) and
Theorem \ref{chap11-thm11.9} here, 
\begin{equation*}
1-\frac{1}{\tau} \int^{\infty}_{\tau} (F_0 (t-1) -1)  \text { dt } = :
f_0 (\tau ). \tag{11.88}\label{eq11.88} 
\end{equation*}

This gives \eqref{eq11.60} and the choice of \eqref{eq11.59} made
since $f_0 (\tau)$ is negative if $\tau \leq \nu _0$ (by
\eqref{eq11.88}), where $\nu _0$ is defined by  
\begin{equation*}
\frac{1}{\nu _0}\int^{\infty}_{\nu_{0}} (F_0 (t-1)-1)  dt= 1,
\tag{11.89}\label{eq11.89} 
\end{equation*}
while one has always, on the other hand, the trivial $S
(\mathscr{A}_q, \mathfrak{p},z) \ge 0$,  Now \eqref{eq11.89} yields   
$$
\nu_0 = 2.06.
$$

Further,\pageoriginale one can show, corresponding to \eqref{eq11.87} above
\begin{equation*}
f_0(\tau ) \ge 0, f_0 (\tau) \uparrow ( \text { for } \tau > 0) \text{
  and } \lim\limits_{ \tau \to \infty} f_0(\tau ) =
1. \tag{11.90}\label{eq11.90}  
\end{equation*}

\eqref{eq11.62}: Clearly, this condition has been modelled  such that
Bombi\-eri-type theorems are directly applicable. 

Theorem \ref{chap11-thm11.11}. cf. l.c. Theorem 7.4.
 
\eqref{eq11.70},...\eqref{eq11.76}: Jurkat and Richert \cite{key1}
(cf. l.c. (Chapter \ref{chap8}, \textbf{2})). 

Theorem \ref{chap11-thm11.12}: cf. l.c.  Theorem 8.3.

Theorem \ref{chap11-thm11.13}: Jurkat and Richert \cite{key1},
Halberstam, Jurkat and\break Richert \cite{key1} (cf. l.c. (Theorem
8.4)). Selberg \cite{key5} has pointed out that
(according to an unpublished paper of J.B. Rosser) this result can
also  be derived from Brun's sieve. This has been proved, even with a
better error term (namely, with $\dfrac{1}{14}$ replaced by $1$) by
Iwaniec \cite{key1}. Iwaniec \cite{key1} has also applied this
improved version of Theorem \ref{chap11-thm11.13} to  sharpen the
bounds for the Legendre-Jacobsthal function $C_0(r)$, the maximum
length a block of consecutive  integers each of which is divisible by
at least one of the first $r$ primes. His result is
$$
C_0 (r)\ll r^2 \log  ^2 r,
$$
whereas (our) Theorem \ref{chap11-thm11.13} leads only to  the estimate
$$
C_0 (r)\ll r^2 \exp \big\{  (\log r) ^{13/14}\big\}.
$$

In the opposite direction we have, by Rankin \cite{key1}. 
$$
C_0(r)>e^{\gamma -\epsilon} \frac{r \log ^2 r \log \log \log r}{(\log \log
  r)^2} 
$$
(cf. l.c. (p. 239)). (For the case $\kappa = \dfrac{1}{2}$ see Iwaniec
\cite{key2}, \cite{key6}, and for $\kappa< \dfrac{1}{2}$ see  Iwaniec
\cite{key7}. However, for $\kappa$  exceeding some constant
$\kappa_1(>1)$ Selberg's sieve seems to be always superior than Brun's
sieve (for instance, in sifting values of reducible polynomials). 

\eqref{eq11.86}: This\pageoriginale fact (about \eqref{eq11.86} was
established by 
Selberg  \cite{key3} (for $0<u \leq 2$), and he added the remark that
the sieve method ``cannot distinguish between numbers with  an odd or
an even  number of prime factors''. 

