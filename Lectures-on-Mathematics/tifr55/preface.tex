\thispagestyle{empty}

\begin{center}
{\Large\bf Lectures on}\\[5pt]
{\Large\bf Sieve Methods}
\vskip 1cm

{\bf By}
\medskip

{\large\bf H.E. Richert}
\vfill

{\bf Tata Institute of Fundamental Research}


{\bf Bombay}

{\bf 1976}

\end{center}
\eject


\thispagestyle{empty}
\begin{center}
{\Large\bf Lectures on}\\[5pt]
{\Large\bf Sieve Methods}
\vskip 1cm

{\bf By}
\medskip

{\large\bf H.E. Richert}
\vfill

{\bf Notes By}
\medskip

{\large\bf S. Srinivasan}
\vfill

{\bf Tata Institute of Fundamental Research}


{\bf Bombay}


{\bf 1976}

\end{center}
\eject


\thispagestyle{empty}

\begin{center}

\vfill
{\bf\copyright Tata Institute of Fundamental Research, 1976}
\vfill

\parbox{0.7\textwidth}{No part of this book may be reproduced
in any form by print, microfilm or any
other means without written permission
from the Tata Institute of Fundamental
Research, Colaba, Bombay 400 005}
\vfill

Printed In India

By

Anil D. Ved At Prabhat Printers. Bombay 400 004

And Published By

{\bf The Tata Institute of Fundamental Research}
\end{center}

\chapter{Preface}


THESE LECTURES were given during a seven-week course at the Tata
Institute of Fundamental Research. The aim was to provide an
introduction to modern sieve methods, i.e. to various forms of both
the large sieve (part I) and the small sieve (part II), as well as
their interconnections and applications. Being a were of the fact that
such a goal  cannot be reached in such a short time. I have tried to
compromise between an introduction and a survey. The difficult task of
deciding what to omit I have tried to overcome in most cases by
presenting the simplest approach in details and a sketch of the more
sophisticated results if their proof would have required too much
time. Nevertheless I have decided to include a chapter an the history
of the large sieve upto Bombieri's first paper, because I believe that
a student coming to a new branch of mathematics can learn much more
from the historical development in that then is generally
expected. The final chapter contains a proof of Chen's Theorem,
because I consider it the most beautiful example of the interaction
between various sieve methods and other powerful tools of analytic
number theory. 

I am indebted to my colleagues at the Tata Institute for their
generous  hospitality, particularly to K.G. Ramanathan and to
K. Ramachandra for many interesting discussions. 

The notes have been prepared by S.Srinivasan. His critical ability has
been of great value to me, and I wish to thank him his meticulous
handling of the manuscript. 
\bigskip

\hfill{\large\bf H.-E. Richert}

\noindent
\hfill Bombay, 
April 1976

