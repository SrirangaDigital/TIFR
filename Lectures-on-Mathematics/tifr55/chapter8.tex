
\chapter{The Brun-Titchmarsh Theorem}\label{chap8} %chap 8

IN\pageoriginale THIS	chapter we deal with an important application, of the
arithmetical sieve result of the previous chapter, in prime number
theory. One of the prominent problems of number theory is the study of
distribution of primes in arithmetic progressions; i.e., to investigate  
\begin{equation*}
\pi (x;k,\ell ): = \sum_{\substack {p \leq x \\ { p \equiv \ell \mod
      k}}} 1 , \tag{8.1}\label{eq8.1} 
\end{equation*}
in particular, to obtain estimates valid uniformly in (ranges of) $k$
(relative to $x$).  

In this direction there is the famous Siegel-Walfisz theorem that,
for any $C > 0$ and $U > 0$, 
\begin{equation*}
\pi (x;k,\ell)= \frac{li \; x}{\varphi (k)} + O_{U,C}(x \log^{-U} x)
\text{~ uniformly in~ } k \leq \log^C x \tag{8.2}\label{eq8.2} 
\end{equation*}
(with an ineffective $O$-constant due to the possible existence of a
Siegel-zero), which was one of the main tools in I.M. Vinogradov's
proof of the solubility of the famous equation 
\begin{equation*}
2N + 1 = p_1 + p_2 + p_3 \quad \text{for}\quad N \geq
N_0. \tag{8.3}\label{eq8.3}  
\end{equation*}

Titchmarsh used the generalized Riemann hypothesis to tackle the
divisor problem (since then named after him) 
\begin{equation*}
\sum_{p \leq x} d(p-1) \sim cx \text{~ as~ } x \to \infty \quad
(\text{certain~ } c > 0) \tag{8.4}\label{eq8.4} 
\end{equation*}
where he also employed (cf. \eqref{eq8.21}) the estimate
\begin{equation*}
\pi (x;k,\ell ) \ll_\alpha \frac{x}{\alpha (k) \log x}\quad
\text{for}\quad k
\leq x^\alpha\quad (\text{fixed~ } \alpha , 0 < \alpha < 1 ),
\tag{8.5}\label{eq8.5}  
\end{equation*}
which he obtained by Brun'sieve.

This\pageoriginale problem provides a good example to illustrate our
earlier remark 
that Bombieri's prime number theorem can replace the generalized
Riemann hypothesis on average, since \eqref{eq6.23} states that $\dfrac{li\
  x}{\varphi(k)}$ is the leading term for almost all $k \leq
\dfrac{x^{\frac{1}{2}}}{\log^C x}$. 

Now we are in a position to outline the \textit{unconditional} proof
of \eqref{eq8.4} employing \eqref{eq6.23} and \eqref{eq8.5}. To start
with have   
\begin{equation*}
\sum_{p \leq x} d(p-1) = \sum_{p \leq x}\sum_{d | p-1}1 = \sum_{d \leq
  x} \sum_{\substack {p \leq x \\ {p \equiv 1 \mod d}}}1 = \sum_{d
  \leq x} \pi (x;d,1) \tag{8.6}\label{eq8.6} 
\end{equation*}
(in the notation of \eqref{eq8.1}). Let $U > 0$ denote a number to be
suitably restricted and let $C = C(U)( > 1)$ be a value for which
\eqref{eq6.23} holds. Then splitting the range of $d$ in the last sum of
\eqref{eq8.6} into three parts 
\begin{equation*}
d \leq x^{\frac{1}{2}} \log^{-C}x,\quad x^{\frac{1}{2}} \log^{-C}x < d
\leq x^{\frac{1}{2}} \log^{C}x, x^{\frac{1}{2}} \log^{C}x < d \leq x
\tag{8.7}\label{eq8.7} 
\end{equation*}
and denoting the corresponding partial sums there by 
\begin{equation*}
\sum_1 , \sum_2, \sum_3 \tag{8.8}\label{eq8.8}
\end{equation*}
we see that, in view of \eqref{eq8.5}, one has
\begin{equation*}
\sum_2 \ll \frac{x}{\log x} \sum '' \frac{1}{\varphi (d)} \ll \frac{x
  \log \log x}{\log x}, \tag{8.9}\label{eq8.9} 
\end{equation*}
where $''$  denote the restriction of $d$ to the second range in
\eqref{eq8.7}. Here we have used the fact (cf. Estermann \cite{key1})
that   
\begin{equation*}
\sum_{d \leq y} \frac{1}{\varphi (d)} = c \log y + o(1) \text{~ as~ } y
\to \infty , c = \prod_p (1+\frac{1}{p(p-1)}). \tag{8.10}\label{eq8.10} 
\end{equation*}

Next consider (in a notation similar to the one above)
\begin{equation*}
\sum_3 = \sum^{"} \sum_{\substack {p \leq x \\ { p \equiv 1 \mod d }}}
1 =\sum_{p \leq x} \sum ^{'''}_{d|p-1} 1. \tag{8.11}\label{eq8.11} 
\end{equation*}

Treating $\sum_1$ similarly we see that (after a simple rearrangement)
since $C>1$, 
\begin{equation*}
\sum_1 -\sum_3 \ll \sum_{p\leq x} \mathop{\sum{}''}\limits_{d|p-1} 1 +
\sum_{p \leq x \log^{-2C}x} d(p-1) \leq \sum_2 + x \log^{-1} x
\tag{8.12}\label{eq8.12}  
\end{equation*}\pageoriginale
from which one obtains, with the help of \eqref{eq8.9} and \eqref{eq8.6}, 
\begin{equation*}
\sum_{p \leq x} d(p-1) = 2 \sum_1 + O (\frac{x \log \log x}{\log
  x}). \tag{8.13}\label{eq8.13} 
\end{equation*}

Now, by our choice of $C$ and \eqref{eq6.23}, we have
\begin{equation*}
\sum_1 = \sum' (\pi (x;d,1)- \frac{li~x}{\varphi (d)}) + li~x \sum
'\frac{1}{\varphi (d)} + li~x \sum '\frac{1}{\varphi (d)} + 0 (x
\log^{-U} x). \tag{8.14}\label{eq8.14} 
\end{equation*}

Using here \eqref{eq8.10} again we get, in view of \eqref{eq8.13}.
\begin{equation*}
\sum_{p \leq x} d(p-1) = 2 ~ li ~ x (c \log (x^\frac{1}{2}
\log^{-C}x)) + O(\frac{x \log \log x}{\log x}) \tag{8.15}\label{eq8.15} 
\end{equation*}
on taking $U=2$ say. Thus, after inserting the value of $c$ from
\eqref{eq8.10}, we have the asymptotic formula \eqref{eq8.4} in a more
precise form: 
\begin{equation*}
\sum_{p \leq x} d(p - 1) = x \prod_p (1 + \frac{1}{p(p-1)} + O(\frac{x
  \log \log x}{\log x}). \tag{8.16}\label{eq8.16} 
\end{equation*}

For number-theoretic purposes the `Brun-Titchmarsh Theorem'\break \eqref{eq8.5} is
very valuable being the only known result valid in such a wide range
for $k$. 

In the direction of \eqref{eq8.5} we shall prove the following result
of Montgomery and Vaughan \cite{key2} as an application of Theorem
\ref{chap7-thm7.1} (\eqref{eq7.8}):
 
\setcounter{section}{8}
\setcounter{theorem}{0}
\begin{theorem}\label{chap8-thm8.1}%the 8.1
For any positive real numbers $x$ and $y$, and for any $\ell ,k$ from
$\mathbb{N}$ with 
\begin{equation*}
(\ell ,k)=1, \tag{8.17}\label{eq8.17}
\end{equation*}
there is an absolute constant $c_0$ such that
\begin{equation*}
\pi (x + y ; k, \ell) -\pi (x; k, \ell) < \frac{2y}{\varphi (k)(\log
  (\frac{y}{k}) + \frac{13}{15})}, \tag{8.18}\label{eq8.18} 
\end{equation*}
provided
\begin{equation*}
\frac{y}{k} > c_0. \tag{8.19}\label{eq8.19}
\end{equation*}
\end{theorem}

\begin{remark*}
Under\pageoriginale the assumptions of Theorem \ref{chap8-thm8.1} we
have, in particular, the estimate \eqref{eq8.18} without the term
$\dfrac{13}{15}$. In this context, 
by adding some numerical computation, Montgomery and Vaughan have also
shown that $c_0$ can be taken equal to $1$, i.e,. we have the neat
result  
\begin{equation*}
\pi (x + y ; k, \ell ) -\pi (x; k,\ell ) < \frac{2y}{\varphi (k) \log
  (\frac{y}{k})}, 1 \leq \ell \leq k < y, (\ell ,k) = 1, \forall x >
0; \tag{8.20}\label{eq8.20} 
\end{equation*}
so, in particular, choosing $x = 0$ (and replacing $y$ by $x$)
\begin{equation*}
\pi (x; k. \ell ) < \frac{2y}{\varphi (k) \log (\frac{y}{k})}, 1 \leq
l \leq k < x, (\ell ,k) = 1. \tag{8.21}\label{eq8.21} 
\end{equation*}
\end{remark*}

\begin{proof}
Let $z > 0$ denote a number to be suitably chosen later. Consider the
set  
\begin{equation*}
\gamma : =\{ m: x < mk + \ell \leq x + y, ((mk +\ell ),
\prod^{\infty}_{\substack{p \leq z \\ p \not\mid k}}p) = 1 \}
. \tag{8.22}\label{eq8.22} 
\end{equation*}
\end{proof}

In the notation of Theorem \ref{chap7-thm7.1} we have 
\begin{equation*}
M \bigg[ \frac{x-l}{k} \bigg] , N = \bigg[ \frac{x+y-\ell}{k}\bigg]-
\bigg[ \frac{x-l}{k}\bigg] , \tag{8.23}\label{eq8.23} 
\end{equation*}
and so 
\begin{equation*}
\frac{y}{k}-1 < N < \frac{y}{k}+1. \tag{8.24}\label{eq8.24}
\end{equation*}

Note that $\gamma $ contains all those prime numbers counted in $\pi
(x+y;\break k,\ell ) - \pi (x; k,\ell )$ which are $ > z$, and also  
\begin{equation*}
\omega (p) \geq  \quad \forall p \leq z, p \not \mid
k. \tag{8.25}\label{eq8.25} 
\end{equation*}

Hence, by \eqref{eq7.8},
\begin{equation*}
\pi (x+y; k,\ell ) - \pi (x; k,\ell ) \leq S + z \leq
\frac{1}{L^*(z)}+ \leq \frac{N}{M_k(z)}+ z, \tag{8.26}\label{eq8.26} 
\end{equation*}
where, by \eqref{eq8.25}, 
\begin{equation*}
M_k(z) : \sum_{\substack{q\leq z \\ (q, k) =1}} ( 1 + \frac{3}{2} q
\frac{z}{N})^{-1} \frac{\mu^2 (q)}{\varphi
  (q)}. \tag{8.27}\label{eq8.27}  
\end{equation*}

We\pageoriginale now choose 
\begin{equation*}
z = (\frac{2}{3}N)^{\frac{1}{2}}\tag{8.28}\label{eq8.28}
\end{equation*}
so that 
\begin{equation*}
\frac{2}{3} \frac{z}{N}= z^{-1}\tag{8.29}\label{eq8.29}
\end{equation*}
which makes 
\begin{equation*}
M_k(z) = \sum_{\substack{q \leq z \\ {(q, k)=1}}} (1+ qz^{-1})^{-1}
\frac{\mu^2(q)}{\varphi(q)}.\tag{8.30}\label{eq8.30} 
\end{equation*}

The equality part of \eqref{eq3.26} and the observation that
$(1+qz^{-1})^{-1}$ decreases as $q$ increases, enable us to uphold  
\begin{equation*}
M_k(z) \geq \frac{\varphi(k)}{k} \sum_{q \leq z}(1+qz^{-1})^{-1}
\frac{\mu^2(q)}{\varphi(q)} = \frac{\varphi(k)}{k}
M_1(z).\tag{8.31} \label{eq8.31} 
\end{equation*}

Now we need, instead of the estimate in \eqref{eq3.26} which yielded only
the leading term, the more precise formula due to D.R. Ward \cite{key1},
namely that  
\begin{equation*}
\sum_{q \leq w}\frac{\mu^2(q)}{\varphi(q)} \log w + c_1+o (1) \text{~
  as~ } w \to \infty,\tag{8.32}\label{eq8.32} 
\end{equation*}
where 
\begin{equation*}
c_1 = \gamma + \sum_p \frac{\log p}{p(p-1)} =\lim_{u \to \infty} (\log u -
\sum_{p \leq u} \frac{\log p}{p}) = 1.33258 \cdots
\tag{8.33}\label{eq8.33} 
\end{equation*}
(cf. Rosser and Schoenfeld \cite{key1}). Using \eqref{eq8.32} for
partial summation we get as $N \to \infty$ 
\begin{equation*}
M_1(z) = \log z+c_1 - \log 2+ \circ (1)=\frac{1}{2} \log N+ c_1 - \log
2-\frac{1}{2} \log\frac{3}{2}+ \circ (1)\tag{8.34}\label{eq8.34} 
\end{equation*}
because of \eqref{eq8.28}, and consequently
\begin{equation*}
\frac{N-1}{N}M_1(z) = \frac{1}{2} \log (N+1) + c_2 + o(1) \text{ as }
N \to \infty \tag{8.35}\label{eq8.35} 
\end{equation*}
with, in view of \eqref{eq8.33},
\begin{equation*}
c_2 = c_1- \log 2-\frac{1}{2} \log \frac{3}{2} = c_1 -\frac{1}{2}
\log 6 > \frac{13}{30}.\tag{8.36}\label{eq8.36}  
\end{equation*}\pageoriginale

We obtain, on using \eqref{eq8.32}, \eqref{eq8.35} and \eqref{eq8.28},
\begin{equation*}
\frac{N-1}{N}M_k(z) \geq \frac{\varphi(k)}{2k}(\log (N+1) + 2c_2 +
o(1)), z \leq (N-1)^{\frac{1}{2}} \text{~ as~ } N \to
\infty. \tag{8.37}\label{eq8.37}  
\end{equation*}

Thus, on choosing a sufficiently large $c_0$, it follows from an use
of \eqref{eq8.37} in \eqref{eq8.26} 
\begin{gather*}
\pi (x+y; k, l)-\pi(x;k,l) \leq \frac{2k (N-1)}{\varphi(k)(\log
  (N+1)+2c_2+ \circ (1))}+(N-1)^{\frac{1}{2}}\\
< \frac{2y}{\varphi(k)(\log
  (\frac{y}{k})+\frac{13}{15})}\tag{8.38}\label{eq8.38} 
\end{gather*}
because of \eqref{eq8.24} and \eqref{eq8.19}. This completes the proof
of the theorem. 

\medskip
\begin{center}
\textbf{NOTES}
\end{center}

For the history of Brun-Titchmarsh theorem we refer to Halberstam and
Richert \cite{key1} (Chapter \ref{chap3}). 

\eqref{eq8.12}:~ The estimations of \eqref{eq8.12} are obtained as
follows. We have  
\begin{equation*}
\sum_1 - \sum_3 = \sum_{p \leq x}\left(\sum'_{d |p-1}1-\sum'''_{d
  |p-1}1\right).\tag{8.39}\label{eq8.39} 
\end{equation*}

For any fixed $p \leq x$, the difference of the sums inside is
precisely the number of $d's(\leq x^{\frac{1}{2}}\log^{-C}x)$
dividing p-1 and with $\dfrac{p-1}{d}\leq x^{\frac{1}{2}} \log^C
x$. Now splitting such d's into parts according as $\dfrac{p-1}{d}\leq
x^{\frac{1}{2}} \log^{-C} x$ or not and noting further further that the
second part is empty if $p > \log^{-2C}x$ we obtain the first
majorization. The remaining part employs only the simple  $\sum
\limits_{m \leq y} d(m) = O(y \log y)$. 

\eqref{eq8.16}:~ This unconditional result is due to Rodriquez
\cite{key1} and Halberstam \cite{key1} (cf. Halberstam and Richert
     \cite{key1} (Theorem 3.9). More general results of
     this type have already been mentioned in the notes of Chapter
     \ref{chap6}.  



Theorem \ref{chap8-thm8.1}:
The estimate \eqref{eq8.20} demonstrates the power of the weighted
sieve of Montgomery and Vaughan (cf. under Theorem \ref{chap7-thm7.1}
in the notes for Chapter \ref{chap7}. However, an estimate of the type
\eqref{eq8.18} can also be derived without the use of
obayashi's\pageoriginale results: cf. Halberstam and Richert
\cite{key1} (pp. 124-125)). On the other hand, it is easily checked that if
we use, at the beginning of the proof, the estimate \eqref{eq7.6} instead of
\eqref{eq7.8} we cannot obtain \eqref{eq8.20}, even subject to the condition
\eqref{eq8.19}, without an extra term on the right-hand side. 

\eqref{eq8.21}:~ A further improvement of the factor 2 in
\eqref{eq8.21} to a constant $c<2$ would have important consequences
concerning the Siegel-zeros of Dirichlet's $L$-functions, as has been
first pointed out by Rodosskij (cf. for example, Bombieri and
Davenport \cite{key2}). 

\eqref{eq8.36}:~ It is easily checked to have, instead of \eqref{eq8.36}, that 
\begin{equation*}
c_2>1,\tag{8.40}\label{eq8.40} 
\end{equation*}
so that one has neater \eqref{eq8.18} with $1$ in place of $\dfrac{13}{15}$,
it would be sufficient to improve the constant $\dfrac{3}{2}$ (at least
in \eqref{eq2.91}) (cf. notes for Chapter \ref{chap2}, under
\eqref{eq2.76}) to a constant $\Delta$ satisfying 
\begin{equation*}
\Delta \frac{1}{4} e^{2c_1 -1}= 1.32163 \cdots
\tag{8.41}\label{eq8.41} 
\end{equation*}

On the other hand (see the above remark) we point out that Selberg's
sieve permits one to replace 13/15 in \eqref{eq8.18} by any constant
$C$ (with a $c_0 = c_0(C)$ in \eqref{eq8.19}). 

Recently, starting from the works of Hooley and Motohashi (Hooley
\cite{key2}, \cite{key6}, and Motohashi \cite{key8}, \cite{key9},
\cite{key11}) there has been a
remarkable progress with respect to the Brun-Titchmarsh theorem. Some
of these results are concerning averages and certain others are valid
only for some ranges of $k$. As an example, we mention one of the most
recent results of the latter type (Goldfeld \cite{key4}): For every
(sufficiently small) $\epsilon >0$ holds, with a certain $c>0$, 
\begin{equation*}
\pi(x;k,l) \leq (1 +\epsilon) \frac{x}{\varphi(k) \log
  (\frac{x}{\sqrt{k^3}}}). if x^{\frac{2}{5} -c \epsilon} \leq k \leq
x^{\frac{1}{2}}\tag{8.42}\label{eq8.42}  
\end{equation*}
