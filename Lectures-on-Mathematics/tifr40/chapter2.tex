\chapter{Preschemes}\label{chap2}

\begin{definition}\label{chap2-defi2.1}
A\pageoriginale ringed space $(X,\mathscr{O}_{X})$ is called a {\em
  prescheme} if every point $x\in X$ has an open neighbourhood $U$
such that $(U,\mathscr{O}_{X}|U)$ is an affine scheme.
\end{definition}

An open set $U$ such that $(U,\mathscr{O}_{X}|U)$ is an affine scheme
is called an affine open set of $X$; such sets form a basis for the
topology on $X$.

\setcounter{section}{1}
\setcounter{defin}{0}
\begin{defin}\label{chap2-defi2.1.1}
A morphism $\Phi:(X,\mathscr{O}_{X})\to (Y,\mathscr{O}_{Y})$ of
preschemes is a morphism $(f,\varphi)$ of ringed spaces such that for
every $x\in X$, the stalk-map $\varphi_{x}:\mathscr{O}_{f(x)}\to
\mathscr{O}_{x}$ defined by $\Phi$ is a local homomorphism.
\end{defin}

Preschemes then form a category (Sch). In referring to a prescheme, we
will often suppress the structure sheaf from notation and denote
$(X,\mathscr{O}_{X})$ simply by $X$.

\subsection{}\label{chap2-sec2.1.2}%2.1.2
Suppose $\mathscr{C}$ is any category and $S\in\Ob \mathscr{C}$. We
consider the pairs $(T,f)$ where $T\in \Ob\cdot\mathscr{C}$ and $f\in
\Hom_{\mathscr{C}}(T,S)$. 

If $(T_{1},f_{1})$, $(T_{2},f_{2})$ are two such pairs, we define
$\Hom((T_{1},f_{1}),\break (T_{2},f_{2}))$ to be the set of
$\mathscr{C}$-morphisms $\varphi:T_{1}\to T_{2}$, making the diagram
\[
\xymatrix{
T_{1}\ar[rr]^{\varphi}\ar[dr]_{f_{1}} & & T_{2}\ar[dl]^{f_{2}}\\
 & S &
}
\] 
commutative.

This\pageoriginale way we obtain a category, denoted by $\mathscr{C}|S$. In the
special case $\mathscr{C}=(\Sch)$, the category $(\Sch/S)=(\Sch)|S$ is
called the category of $S$-preschemes; its morphisms are called
$S$-morphisms. $S$ itself is known as the base prescheme of the
category. 

\setcounter{remark}{2}
\begin{remark}\label{chap2-rem2.1.3}%2.1.3
Let $\Spec A$ be an affine scheme and $Y$ any prescheme. Then
$\Hom(A,\Gamma(Y,\mathscr{O}_{Y}))$ is naturally isomorphic to
$\Hom(Y,\Spec A)$.
\end{remark}

In fact, let $(U_{i})$ be an affine open covering of $Y$ and
$\varphi\in \Hom\break (A,\Gamma(Y,\mathscr{O}_{Y}))$. The composite maps
$$
\varphi_{i}:A\xrightarrow{\varphi}\Gamma(Y,\mathscr{O}_{Y})\xrightarrow{\text{restriction}}\Gamma(U_{i},\mathscr{O}_{Y})
$$
give morphisms ${}^{a}\varphi_{i}:U_{i}\to \Spec A$, for every $i$,
since the $U_{i}$ are affine. It is easily checked that
${}^{a}\varphi_{i}={}^{a}\varphi_{j}$ on $U_{i}\cap U_{j}$,
$\forall\ i,j$. We then get a morphism ${}^{a}\varphi:Y\to \Spec A$;
the map $\varphi\mapsto{}^{a}\varphi$ is a bijection from
$\Hom(A,\Gamma(Y,\mathscr{O}_{Y}))$ onto $\Hom(Y,\Spec A)$
(cf. (\ref{chap1-rems1.6.4})(c)).

It follows that every prescheme $X$ can be considered as a
$\Spec\mathbb{Z}$-prescheme in a natural way:
$$
(\Sch)=(\Sch/\Spec\mathbb{Z})=(\Sch/\mathbb{Z}).
$$

\begin{remark}\label{chap2-rem2.1.4}% 2.1.4
Let $(X,\mathscr{O}_{X})$ be a prescheme and $\mathscr{F}$ an
$\mathscr{O}_{X}$-module. Then it follows from (\ref{chap1-rems1.6.4})(d) that
$\mathscr{F}$ is quasi-coherent\pageoriginale $\Leftrightarrow$ for
every $x\in X$ and {\em any} affine open neighbourhood $U$ of $x$,
$\mathscr{F}|U\simeq \widetilde{M}_{U}$, for a
$\Gamma(U,\mathscr{O}_{X})$  - module $M_{U}$.
\end{remark}

We may take this as our definition of a quasi-coherent
$\mathscr{O}_{X}$-module.

\section{Product of Preschemes}\label{chap2-sec2.2}%2.2

\setcounter{subsection}{-1}
\subsection{}\label{chap2-sec2.2.0}%2.2.0

Suppose $(X,f)$, $(Y,g)$ are $S$-preschemes. We say that a triple
$(Z,p,q)$ is a product of $X$ and $Y$ over $S$ if:
\begin{itemize}
\item[(i)] $Z$ is an $S$-prescheme

\item[(ii)] $p:Z\to X$, $q:Z\to Y$ are $S$-morphisms and

\item[(iii)] for any $T\in(\Sch/S)$, the natural map:
\begin{gather*}
\Hom_{S}(T,Z)\to \Hom_{S}(T,X)\times \Hom_{S}(T,Y)\\
f\mapsto (p.f,q.f)
\end{gather*}
is a bijection.
\end{itemize}

The product of $X$ and $Y$, being a solution to a universal problem,
is obviously unique upto an isomorphism in the category. We denote the
product $(Z,p,q)$, if it exists by $\fprod{X}{Y}{S}$ and call it the
{\em fibre-product} of $X$ and $Y$ over $S$; $p$, $q$ are called
projection morphisms.

\begin{theorem}\label{chap2-thm2.2.1}
If $X$, $Y\in(\Sch/S)$, the fibre-product of $X$ and $Y$ over $S$
always exists.
\end{theorem}

We\pageoriginale shall not prove the theorem here. However, we observe
that if $X=\Spec A$, $Y=\Spec B$ and $S=\Spec C$ are all affine, then
$\Spec(\foprod{A}{B}{C})$ is a solution for our problem. In the
general case, local fibre-products are obtained from the affine case
and are glued together in a suitable manner to yield a fibre product
$\fprod{X}{Y}{S}$. 

(For details see EGA, Ch. I, Theorem (3.2.6)).

\begin{remarks*}
\begin{enumerate}
\renewcommand{\labelenumi}{(\theenumi)}
\item The underlying set of $\fprod{X}{Y}{S}$ is {\em not} the
  fibre-product of the underlying sets of $X$ and $Y$ over that of
  $S$. However if $x\in X$, $y\in Y$ lie over the same $s\in S$, then
  there is a $z\in\fprod{X}{Y}{S}$ lying over $x$ and $y$. (For a
  proof see Lemma (\ref{chap2-lem2.3.1})).

\item An open subset $U$ of a prescheme $X$ can be considered as a
  pres\-cheme in a natural way. Suppose $S'\subset S$, $U\subset X$,
  $V\subset Y$ are open sets such that $f(U)\subset S'$, $g(V)\subset
  S'$; we may consider $U$, $V$ as $S'$-preschemes. When this is done,
  the fibre-product $\fprod{U}{V}{S'}$ is isomorphic to the open set
  $p^{-1}(U)\cap q^{-1}(V)$ in $Z=\fprod{X}{Y}{S}$, considered as a
  prescheme.
\end{enumerate}
\end{remarks*}

This follows easily from the universal property of the fibre-product.

\setcounter{subsection}{1}
\subsection{Change of base:}\label{chap2-sec2.2.2}%2.2.2

Let $X$, $S'$ be $S$-preschemes. Then the fibe-product
$\fprod{X}{S'}{S}$ can be considered as an $S'$-prescheme in a natural
way:
\[
\xymatrix{
X\ar[d] & \fprod{X}{S'}{S}\ar[l]\ar[d]\\
S & \ar[l] S'
}
\]
When\pageoriginale this is done, we say that $\fprod{X}{S'}{S}$ is
obtained from $X$ by the base-change $S'\to S$ and denote it by
$X_{(S')}$. Note that, in the affine case, this corresponds to the
extension of scalars.

If $X$ is any prescheme, by its {\em reduction} $\mod p$,
$p\in\mathbb{Z}^{+}$, (\resp $\mod p^{2}$ and so on) we mean the
base-change corresponding to $\mathbb{Z}\to \mathbb{Z}_{/(p)}$ (\resp
$\mathbb{Z}\to \mathbb{Z}_{/(p^{2})}$ and so on).

If $f:X\to X'$., $g:Y\to Y'$ are $S$-morphisms $f$ and $g$ define, in
a natural way an $S$-morphism: $\fprod{X}{Y}{S}\to \fprod{X'}{Y'}{S}$,
which we denote by $\fprod{f}{g}{S}$ or by $(f,g)_{S}$. When
$g=I_{S'}:S'\to S'$, we get a morphism:
$\fprod{f}{I_{S'}}{S}=f_{(S')}:X_{(S')}\to X'_{(S')}$.

\section{Fibres}\label{chap2-sec2.3}

Let $(X,f)$ be an $S$-prescheme and $s\in S$ be any point. Let
$U\subset S$ be an affine open neighbourhood of $s$ and
$A=\Gamma(U,\mathscr{O}_{S})$. If $p_{s}$ is the prime ideal of $A$
corresponding to $s$, $\mathscr{O}_{s,S}$ is identified with
$A_{p_{s}}$. Denote by $k(s)$ the residue field of
$\mathscr{O}_{s,S}=A_{p_{s}}$. The composite $A\to A_{p_{s}}\to k(s)$
defines a morphism $\Spec k(s)\to \Spec A=U\subset S$; i.e. to say,
$\Spec k(s)$ is an $S$-prescheme in a natural way. Consider now the
base-change $\Spec k(s)\to S$:
\[
\xymatrix@C=1.5cm{
X\ar[d]^{f} & \ar[l]_-{p} X'=\fprod{X}{\Spec k(s)}{S}\ar[d]^{q}\\
S & \ar[l]\Spec k(s)
}
\]

The\pageoriginale first projection $p$ clearly maps $X'$ into the set
$f^{-1}(s)\subset S$. We claim that $p(X')=f^{-1}(s)$ and further
that, when we provide $f^{-1}(s)$ with the topology induced from $X$,
$p$ is a homomorphism between $X'$ and $f^{-1}(s)$.

To prove this, it suffices to show that for every open set $U$ in a
covering of $X$, $p$ is a homeomorphism from $p^{-1}(U)$ onto $U\cap
f^{-1}(s)$. In view of the remark (2) after Theorem
(\ref{chap2-thm2.2.1}) we may 
then assume that $X$, $S$ are affine say $X=\Spec A$, $S=\Spec
C$. That $p(X')=f^{-1}(s)$ will follow as a corollary to the following
more general result.

\begin{lemma}\label{chap2-lem2.3.1}%2.3.1
Let $X=\Spec A$, $Y=\Spec B$ be affine schemes over $S=\Spec
C$. Suppose that $x\in X$, $y\in Y$ lie over the same element $s\in
S$. Then the set $E$ of elements $z\in Z=\fprod{X}{Y}{S}$ lying over
$x$, $y$ is isomorphic to $\Spec (\foprod{k(x)}{k(y)}{k(S)})$ (as a
set). 
\end{lemma}

\begin{proof}
One has a homomorphism $\psi:\foprod{A}{B}{C}\to
\foprod{k(x)}{k(y)}{k(s)}$ which gives a morphism
$a_{\psi}:\foprod{\Spec k(x)}{k(y)}{k(s)}\to Z$; clearly the image of
$a_{\psi}$ is contained in the set $E=p^{-1}(x)\cap q^{-1}(y)$. That
$a_{\psi}$ is injective is seen by factoring $\psi$ as follows:
$\foprod{A}{B}{C}\to \foprod{A_{x}}{B_{y}}{C_{s}}\to
\foprod{k(x)}{k(y)}{k(s)}$. In order to see that $a_{\psi}$ is
surjective one remarks that for $z\in E$ the homomorphism $A\to
\foprod{A}{B}{C}\to k(z)$ factors through $k(x)$, similarly for $B$,
therefore we have for $\foprod{A}{B}{C}\to k(z)$ a factorisation
$\foprod{A}{B}{C}\to \foprod{k(x)}{k(y)}{k(s)}\to k(z)$.\hfill Q.E.D. 
\end{proof}

In\pageoriginale the above lemma if we take $B=k(s)$ it follows that
in the diagram
\[
\xymatrix@C=.1cm@R=1.5cm{
 & X'=\fprod{X}{\Spec
    k(s)}{S}=\Spec(\foprod{A}{k(s)}{C})\ar[dl]^{p}\ar[dr]_{q} &\\
X=\Spec A\ar[dr]^{f} & & Y=\Spec B\ar[dl]_{g}\\
& s\in S=\Spec C & 
}
\]
the map $p:X'\to f^{-1}(s)$ is a bijection.

\setcounter{subsection}{1}
\subsection{}\label{chap2-sec2.3.2}%2.3.2

Returning to the assertion that $X'$ is homeomorphic to the fibre
$f^{-1}(s)$ (with the induced topology from $X$) we note that if
$\varphi:C\to k(s)$ is the natural map, then $p:X'\to X$ is the
morphism corresponding to $1\otimes \varphi:A\to
\foprod{A}{k(s)}{C}$. To show that $p$ carries the topology over, it
is enough to show that any closed set of $X'$, of the form $V(E')$, is
also of the form $V((1\otimes\varphi)E)$ for some $E\subset A$.

Now, any element of $\foprod{A}{k(s)}{C}$ can be written in the form
$\sum\limits_{i}a_{i}\otimes\left(\dfrac{\overline{c}_{i}}{\overline{t}}\right)=\left(\sum\limits_{i}(a_{i}\otimes\overline{c}_{i})\right)\cdot
\left(1\otimes\dfrac{1}{\overline{t}}\right)$ with $a_{i}\in A$,
$c_{i}$, $t\in C$. Since\pageoriginale
$\left(1\otimes\dfrac{1}{\overline{t}}\right)$ is a unit of
$\foprod{A}{k(s)}{C}$, we can take for $E\subset A$, the set of
elements $\sum\limits_{i}a_{i}c_{i}$ where
$\sum\limits_{i}a_{i}\otimes\left(\overline{c}_{i}/\overline{t}\right)$
is an element of $E'$.\hfill Q.E.D.

\begin{note*}
The fibre $f^{-1}(s)$ can be given a prescheme structure through this
homeomorphism $p:X'\to f^{-1}(s)$. If, in the above proof, we had
taken $\mathscr{O}_{s}/\mathscr{M}_{s}^{n+1}$, instead of
$k(s)=\mathscr{O}_{s}/\mathscr{M}_{s}$ we would still have obtained
homeomorphisms
$p_{n}:\fprod{X}{\Spec}{S}\left(\mathscr{O}_{s}/\mathscr{M}_{s}^{n+1}\right)\to
f^{-1}(s)$. The prescheme structure on $f^{-1}(s)$ defined by means of
$p_{n}$, is known as the $n^{\rm th}$-{\em infinitesimal neighbourhood
  of the fibre}. 
\end{note*}

\section{Subschemes}\label{chap2-sec2.4}%2.4

\setcounter{subsection}{-1}
\subsection{}\label{chap2-sec2.4.0}%2.4.0

Let $X$ be a prescheme and $\mathscr{J}$ a quasi-coherent sheaf of
ideals of $\mathscr{O}_{X}$. Then the support $Y$ of the
$\mathscr{O}_{X}$-Module $\mathscr{O}_{X}/\mathscr{J}$ is closed in
$X$ and $(Y,\mathscr{O}_{X/\mathscr{J}}|Y)$ has a natural structure of
a prescheme. In fact, the question is purely local and we may assume
$X=\Spec A$. Then $\mathscr{J}$ is defined by an ideal $I$ of $A$ and
$Y$ corresponds to $V(I)$ which is surely closed. The ringed space
$(Y,\mathscr{O}_{X/\mathscr{J}}|Y)$ has then a natural structure of an
affine scheme, namely, that of $\Spec(A/I)$.

Such a prescheme is called a {\em closed subscheme} of $X$. An {\em
  open subscheme} is, by definition, the prescheme induced by $X$ on
an open subset in a natural way. A {\em subscheme} of $X$ is a closed
subscheme of an open subscheme of $X$.

\subsection{}\label{chap2-sec2.4.1}%2.4.1

A\pageoriginale subscheme may have the same base-space as $X$. For
example, one can 
show that there is a quasi-coherent sheaf $\mathscr{N}$ of ideals of
$\mathscr{O}_{X}$ such that $\mathscr{N}_{x}$ = nil-radical of
$\mathscr{O}_{x}$. $\mathscr{N}$ defines a closed subscheme $Y$ of
$X$, which we denote by $X_{\red}$ and which is {\em reduced}, in the
sense that the stalks $\mathscr{O}_{y,Y}$ of $Y$ have {\em no
  nilpotent elements}. $X$ and $Y$ have the same base space ($A$ and
$A_{\red}$ have the same prime ideals).

Consider a morphism $f:X\to Y$ of preschemes. Suppose
$\varphi_{X}:X_{\red}\to X$, $\varphi_{Y}:Y_{\red}\to Y$ are the
natural morphisms. Then $\exists$ a morphism $f_{\red}:X_{\red}\to
Y_{\red}$ making the diagram
\[
\xymatrix@=1.5cm{
X\ar[r]^{f} & Y\\
X_{\red}\ar[u]^{\varphi_{X}}\ar[r]_{f_{\red}} & Y_{\red}\ar[u]_{\varphi_{Y}}
}
\]
commutative. This corresponds to the fact that a homomorphism
$\varphi:A\to B$ of rings defines a homomorphism
$\varphi_{\red}:A_{\red}\to B_{\red}$ such that
\[
\xymatrix@R=.5cm@C=1cm{
A\ar[rrr]^{\varphi}\ar[ddd]_{\eta_{A}} & & & B\ar[ddd]^{\eta_{B}}\\
 & \ar@/^1pc/@{-_{>}}[dr] & & \\
 & & & \\
A_{\red}\ar[rrr]^{\varphi_{\red}} & & & B_{\red}
}
\]


\subsection{}\label{chap2-sec2.4.2}%2.4.2

A\pageoriginale morphism $f:Z\to X$ is called an {\em immersion} if it admits a factorization $Z\xrightarrow{f'}Y\xrightarrow{j}X$ where $Y$ is a sub-scheme of $X$, $j:Y\to X$ is the canonical inclusion and $f':Z\to Y$ is an isomorphism. The immersion $f$ is said to be {\em closed} (\resp {\em open}) if $Y$ is a closed subscheme (\resp open subscheme) of $X$.

\begin{example*}
Let $X\xrightarrow{f}S$ be an $S$-prescheme. Then, there is a natural $S$-morphism $\Delta:X\to \fprod{X}{X}{S}$ such that the diagram
\[
\xymatrix@=.6cm{
 & & X\ar[dll]_{I_{X}=\id_{X}}\ar[ddd]^{I_{X}}\ar[ddl]^{\Delta}\\
X\ar[ddd]_{f} & & \\
 & \fprod{X}{X}{S}\ar[ul]^{p_{1}}\ar[dr]_{p_{2}} & \\
 & & X\ar[lld]^{f}\\
S & &
}
\]
is commutative. $\Delta$ is called the diagonal of $f$. It is an immersion.
\end{example*}

\setcounter{subsubsection}{2}
\begin{subdefin}\label{chap2-defi2.4.2.1}%2.4.2.1
A morphism $f:X\to S$ is said to be {\em separated} (or $X$ is said to be an $S$-scheme) if the diagonal $\Delta:X\to \fprod{X}{X}{S}$ of $f$ is a {\em closed} immersion.
\end{subdefin}

A prescheme $X$ is called a {\em scheme} if the natural map $X\to \Spec\mathbb{Z}$ is separated.

\begin{remark*}
Let $Y$ be an affine scheme, $X$ any prescheme and
$(U_{\alpha})_{\alpha}$ an affine open $c$ over for $X$. One can then
show that a morphism $f:X\to Y$\pageoriginale  is separated if and
only if $\forall\ \alpha,\beta$,
\begin{itemize}
\item[\rm(i)] $U_{\alpha}\cap U_{\beta}$ is also affine

\item[\rm(ii)] $\Gamma(U_{\alpha}\cap U_{\beta},\mathcal{O}_{X})$ is
  generated as a ring by the canonical images of
  $\Gamma(U_{\alpha},\mathscr{O}_{X})$ and $\Gamma(U_{\beta},\mathscr{O}_{X})$.
\end{itemize}
\end{remark*}

(For a proof see EGA, Ch. I, Proposition (5.5.6)).

\subsection{Example of a prescheme which is not a
  scheme.}\label{chap2-sec2.4.3}%2.4.3 

Let $B=k[X]$, $C=k[Y]$ be polynomial rings over a field $k$. Then
$\Spec B_{X}$ and $\Spec C_{Y}$ are affine open sets of $\Spec B$ and
$\Spec C$ respectively; the isomorphism $\dfrac{f(X)}{X^{m}}\mapsto
\dfrac{f(Y)}{Y^{m}}$ of $B_{X}$ onto $C_{Y}$ defines an isomorphism
$\Spec C_{Y}\xrightarrow{\sim}\Spec B_{X}$. By recollement of $\Spec
B$ and $\Spec C$ through this isomorphism, one gets a prescheme $S$,
which is {\em not} a scheme; in fact, condition (ii) of the proceding
remark does {\em not} hold: for, $\Gamma(\Spec
B,\mathscr{O}_{S})\simeq B=k[X]$ and $\Gamma(\Spec
C,\mathscr{O}_{S})\simeq C=k[Y]$; the canonical maps from these into
$\Gamma(\Spec B\cap \Spec C,\mathscr{O}_{S})\simeq k[u,u^{-1}]$ are
given by $X\mapsto u$, $Y\mapsto u$ and the image in each case is
precisely $=k[u]$.

\section{Some formal properties of morphisms}\label{chap2-sec2.5}%2.5

\begin{itemize}
\item[(i)] every immersion is separated

\item[(ii)] $f:X\to Y$, $g:Y\to Z$ separated $\Rightarrow g\circ
  f:X\to Z$ separated.

\item[(iii)] $f:X\to Y$ a separated $S$-morphism $\Rightarrow
  f_{(S')}:X_{(S')}\to Y_{(S')}$ is separated for every
  base-change $S'\to S$.

\item[(iv)] $f:X\to Y$,\pageoriginale $f':X'\to Y'$ are separated
  $S$-morphisms $\Rightarrow \fprod{f}{f'}{S}:\fprod{X}{X'}{S}\to
  \fprod{Y}{Y'}{S}$ is separated.

\item[(v)] $g\circ f$ separated $\Rightarrow f$ is separated

\item[(vi)] $f$ separated $\Leftrightarrow f_{\red}$ separated.
\end{itemize}

The above properties are {\em not all independent}. In fact, the
following more general situation holds:

Let $P$ be a property of morphisms of preschemes.

Consider the following propositions:
\begin{enumerate}
\renewcommand{\theenumi}{\roman{enumi}}
\renewcommand{\labelenumi}{(\theenumi)}
\item every closed immersion has $P$

\item $f:X\to Y$ has $P$, $g:Y\to Z$ has $P\Rightarrow g\circ f$ has $p$

\item $f:X\to Y$ is an $S$-morphism having $P\Rightarrow f_{(S')}:X_{(S')}\to Y_{(S')}$ has $P$ for any base-change $S'\to S$.

\item $f:X\to Y$ has $P$, $f':X'\to Y'$ has $P\Rightarrow$ that $\fprod{f}{f'}{S}:\fprod{X}{X'}{S}\to \fprod{Y}{Y'}{S}$ has $P$

\item $g\circ f$ has $P$, $g$ separated $\Rightarrow f$ has $P$.

\item $f$ has $P\Rightarrow f_{\red}$ has $P$. 
\end{enumerate}

If we suppose that (i) and (ii) hold then (iii) $\Leftrightarrow$
(iv). Also, (v), (vi) are consequences of (i), (ii) and (iii) (or
(iv)). 

\begin{proof}
Assume (ii) and (iii). The morphism $\fprod{f}{f'}{S}$ admits a
factorization: 
\[
\xymatrix{
X\times X'\ar[dr]_{\fprod{f}{I_{X'}}{S}}\ar[rr]^{\fprod{f}{f'}{S}} & &
    \fprod{Y}{Y'}{S}\\
 & \fprod{Y}{X'}{S}\ar[ur]_{\fprod{I_{Y}}{f'}{S}} &
}
\]
By\pageoriginale (iii), the morphisms $\fprod{f}{T_{X'}}{S}$ and
$\fprod{I_{Y}}{f'}{S}$ have $P$ and so by (ii) $\fprod{f}{f'}{S}$ also
has $P$.
\end{proof}

On the other hand, assume (i) and (iv). $I_{S'}$ being a closed
immersion, has $P$ by (i) and so $f_{(S')}=\fprod{f}{I_{S'}}{S}$ has
$P$ by (iv).

Now assume (i), (ii) and (iii). If $g:Y\to Z$ is separated,
$\fprod{Y\xrightarrow{\Delta}{Y}}{Y}{Z}$ is a closed immersion and has
$P$ by (i); by making a base-change $X\xrightarrow{f}Y$ we get a
morphism $X\simeq \fprod{Y}{X}{Y}\xrightarrow{\Delta_{Y}\times
  I_{X}}\fprod{\fprod{Y}{Y}{Z}}{X}{Y}\simeq \fprod{X}{Y}{Z}$ which, by
(iii) has property $p$. The projection $p_{2}:\fprod{X}{Y}{Z}\to Y$
satisfies the diagram:
\[
\xymatrix@R=.5cm@C=1cm{
X\ar[rrr]^{\varphi}\ar[ddd]_{g\circ f} & & & \fprod{X}{Y}{Z}\ar[ddd]^{p_{2}}\\
 & \ar@/^1pc/@{-_{>}}[dr] & & \\
 & & & \\
Z\ar@{<-}[rrr]^{g} & & & Y
}
\]
i.e.\@ to say, $p_{2}$ is obtained from $g\circ f$ by the base-change
$Y\to Z$ and so, by (iii) has $P$.

Finally, $f:X\to Y$ is the composite of
$X\xrightarrow{\Delta_{Y}\times I_{X}}\fprod{X}{Y}{Z}$ and
$p_{2}:\fprod{X}{Y}{Z}\to Y$ and so by (ii) has $P$.

To\pageoriginale prove (vi) from  (i), (ii), (iii) use the diagram
\[
\xymatrix@R=.5cm@C=1cm{
X_{\red}\ar[rrr]^{f_{\red}}\ar[ddd]_{\varphi_{X}} & & & Y_{\red}\ar[ddd]^{\varphi_{Y}}\\
 &  & \ar@/_1pc/@{-_{>}}[dlr] & \\
 & & & \\
X\ar[rrr]^{f} & & & Y
}
\]
and the facts that the canonical morphisms $\varphi_{X}$,
$\varphi_{Y}$ are closed immersions and so have $P$, that
$\varphi_{Y}\circ f_{\red}=f\circ g$ has $P$ and that a closed
immersion is separated, then use (v).\hfill Q.E.D.

We now remark that if we replace (i) of the above propositions by

(i)$'$ every immersion has $P$,
then (i), (ii), (iii) imply (v)$'$ $g\circ f$ has $P$, $g$ has
$P\Rightarrow f$ has $P$.

\section{Affine morphisms}\label{chap2-sec2.6}%2.6

\begin{defin}\label{chap2-defi2.6.1}% 2.6.1
A morphism $f:X\to S$ of preschemes is said to be {\em affine} (or $X$
{\em affine over} $S$) if, for every affine open $U\subset S$,
$f^{-1}(U)$ is affine in $X$.
\end{defin}

It is enough to check that for an affine open cover $(U_{\alpha})$ of
$S$, the $f^{-1}(U_{\alpha})$ are affine.

\setcounter{subsection}{1}
\subsection{}\label{chap2-sec2.6.2}%2.6.2

Suppose that $\mathscr{B}$ is a quasi-coherent
$\mathscr{O}_{S}$-Algebra. Let $(U_{\alpha})$ be an affine open cover
of $S$; set $A_{\alpha}=\Gamma(U_{\alpha},\mathscr{O}_{S})$,
$B_{\alpha}=\Gamma(U_{\alpha},\mathscr{B})$ and $X_{\alpha}=\Spec
B_{\alpha}$. The homomorphism $A_{\alpha}\to B_{\alpha}$ defines a
morphism\pageoriginale $f_{\alpha}:X_{\alpha}\to U_{\alpha}$; the
$X'_{\alpha} s$ then patch up together to give an $S$-prescheme
$X\xrightarrow{f}S$; this prescheme $X$ is affine over $S$, is such
that $f_{\alpha}(\mathscr{O}_{X})\simeq \mathscr{B}$, and is
determined, by this property, uniquely upto an isomorphism. We denote
it by $\Spec\mathscr{B}$. conversely, every affine $S$-prescheme is
obtained as $\Spec \mathscr{B}$, for some quasi-coherent
$\mathscr{O}_{S}$-Algebra $\mathscr{B}$. (For details, see EGA Ch. II,
Proposition (1.4.3)).

\begin{remarks*}
\begin{enumerate}
\renewcommand{\theenumi}{\alph{enumi}}
\renewcommand{\labelenumi}{(\theenumi)}
\item Any affine morphism is separated. (Recall the remark at the end
  of (\ref{chap2-sec2.4.2}).)

\item If $S$ is an affine scheme, a morphism $f:X\to S$ is affine
  $\Leftrightarrow X$ is an affine scheme.

\item The formal properties (i) to (vi) of (\ref{chap2-sec2.5}) hold,
  when $P$ is the 
  property of being affine.

\item Suppose $X\xrightarrow{h}Y$ is an $S$-morphism. If $f$, $g$ are
  the structural morphisms of $X$, $Y$ resply, the homomorphism
  $\mathscr{O}_{Y}\to h_{\ast}(\mathscr{O}_{X})$ defined by $h$,
  given an $\mathscr{O}_{S}$-morphism
$$
\mathscr{A}(h):g_{\ast}(\mathscr{O}_{Y})\to
g_{\ast}(h_{\ast}(\mathscr{O}_{X}))=f_{\ast}(\mathscr{O}_{X}); 
$$
then we have a natural map: $\Hom_{S}(X,Y)\to
\Hom_{\mathscr{O}_{S}}(g_{\ast}(\mathscr{O}_{Y})\to
f_{\ast}(\mathscr{O}_{X}))$ (the latter in the sense of
$\mathscr{O}_{S}$-Algebras) defined by $h\mapsto \mathscr{A}(h)$. If
$Y$ is affine over $S$, it can be shown that this natural map is a
{\em bijection}. (EGA Ch II, Proposition (1.2.7)). (Also, compare with
remark (\ref{chap1-rems1.6.4}) (c) for affine schemes, and with remark
(\ref{chap2-rem2.1.3})). 
\end{enumerate}
\end{remarks*}

\section{The finiteness theorem}\pageoriginale\label{chap2-sec2.7}%2.7

\setcounter{defin}{-1}
\begin{defin}\label{chap2-defi2.7.0}%2.7.0
A morphism $f:X\to Y$ of preschemes is said to be of {\em finite
  type}, if, for every affine open set $U$ of $Y$, $f^{-1}(U)$ can be
written as $f^{-1}(U)=\bigcup\limits^{n}_{\alpha=1}V_{\alpha}$, with
each $V_{\alpha}$ affine open in $X$ and each
$\Gamma(V_{\alpha},\mathscr{O}_{X})$ a finite type
$\Gamma(U,\mathscr{O}_{Y})$-algebra. 
\end{defin}

It is again enough to check this for an affine open cover of
$Y$. $A_{n}$ affine morphism $f:X\to Y$ is of finite type
$\Longleftrightarrow$ the quasi-coherent $\mathscr{O}_{Y}$-Algebra
$f_{\alpha}(\mathscr{O}_{X})$ is an $\mathscr{O}_{Y}$-Algebra of
finite type. In particular, a morphism $f:\Spec B\to \Spec A$ is of
finite type $\Longleftrightarrow B$ is a finite type $A$-algebra
(i.e. is finitely generated as $A$-algebra).

\begin{defin}\label{chap2-defi2.7.1}%2.7.1
A morphism $f:X\to S$ is {\em universally closed} if, for every
base-change $S'\to S$, the morphism $f_{(S')}:\fprod{X}{S'}{S}\to S'$
is a closed map in the topological sense.
\end{defin}

\begin{defin}\label{chap2-defi2.7.2}%2.7.2
A morphism $f$ is {\em proper} if
\begin{itemize}
\item[\rm(i)] $f$ is separated

\item[\rm(ii)] $f$ is of finite type and

\item[\rm(iii)] $f$ is universally closed.
\end{itemize}
\end{defin}

The formal properties of (\ref{chap2-sec2.5}) hold then $P$ is the
property of being proper.

\begin{defin}\label{chap2-defi2.7.3}%2.7.3
A prescheme $Y$ is {\em locally noetherian}, if every $y\in Y$ has an
affine open neighbourhood $\Spec B$, with $B$ noetherian. It is said
to be {\em noetherian}, if it can be written as\pageoriginale
$Y=\bigcup\limits^{n}_{i=1}Y_{i}$ where the $Y_{i}$ are affine open
sets such that the $\Gamma(Y_{i},\mathscr{O}_{Y})$ are noetherian rings.
\end{defin}

If $f:X\to Y$ is a morphism of finite type and $Y$ is locally
noetherian, then $X$ is also locally noetherian.

\setcounter{subsection}{3}
\subsection{}\label{chap2-sec2.7.4}%2.7.4
Let $(X,\mathscr{O}_{X})$ and $(Y,\mathscr{O}_{Y})$ be ringed spaces
and $f$ a morphism from $X$ to $Y$. Let $\mathscr{F}$ be an
$\mathscr{O}_{X}$-Module. We then define, for every
$q\in\mathbb{Z}^{+}$, a presheaf of modules on $Y$ by defining:
$U\mapsto H^{q}(f^{-1}(U),\mathscr{F})$ for every open $U\subset Y$
(See EGA Ch 0, III \S\ 12). The sheaf that this presheaf defines on
$Y$ is called the $q^{\rm th}$-{\em direct image} of $\mathscr{F}$ and
is denoted by $R^{q}f_{\ast}(\mathscr{F})$.

\setcounter{theorem}{4}
\begin{theorem}\label{chap2-thm2.7.5}%2.7.5
Let $X$, $Y$ be preschemes, $Y$ locally noetherian, and $f$ a {\em
  proper} morphism from $X$ to $Y$. Then, if $\mathscr{F}$ is any
coherent $\mathscr{O}_{X}$-Module, the direct images
$R^{q}f_{\ast}(\mathscr{F})$ are all coherent
$\mathscr{O}_{Y}$-Modules. 
\end{theorem}

(For a proof see EGA Ch. III, Theorem (3.2.1)). This is the {\em
  theorem of finiteness} for proper morphisms.
