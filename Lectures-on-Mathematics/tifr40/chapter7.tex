\chapter{The Technique of Descents and Applications}\label{chap7}

\section{}\label{chap7-sec7.1}

Before\pageoriginale stating the problem with which we shall be
concerned in this section, in its most general form, we shall look at
it in three particular cases of interest.

\setcounter{exam}{0}
\begin{exam}\label{chap7-exam1}
Let $A$, $A'$ be rings and $\varphi:A\to A'$ be a
ring-homomorphism. Let $\mathscr{C}_{A}$ (resp. $\mathscr{C}_{A'}$) be
the category of $A$-(resp. $A'$-) modules. $\varphi$ defines a
covariant functor $\varphi^{\ast}:\mathscr{C}_{A}\to
\mathscr{C}_{A'}$, viz.,
$\varphi^{\ast}(M)=\foprod{M}{A'}{A}=M'$. Suppose $M$,
$N\in\mathscr{C}_{A}$ and $u:M\to N$ is an $A$-linear map. Then
$\varphi^{\ast}(u)=\foprod{u}{1_{A'}}{A}:M'\to N'$ is an $A'$-linear map.
\end{exam}

\begin{problem}\label{exam1-prob1}
Suppose $u':M'=\varphi^{\ast}(M)\to \varphi^{\ast}(N)=N'$ is an
$A'$-linear map. When can we say that $u'=\varphi^{\ast}(u)$ for an
$A$-linear map $u:M\to N$?
\end{problem}

\begin{problem}\label{exam1-prob2}
Suppose $M'\in\mathscr{C}_{A'}$. When can we say that
$M'=\foprod{M}{A'}{A}$ for an $M\in\mathscr{C}_{A}$?
\end{problem}

\begin{exam}\label{chap7-exam2}
Let $S$, $S'$ be preschemes and $\varphi$ be a morphism $\varphi:S'\to
S$. Let $\mathscr{C}_{S}$ (resp. $\mathscr{C}_{S'}$) be the category
of quasicoherent $\mathscr{O}_{S}$-(resp. $\mathscr{C}_{S'}$-)
Modules. $\varphi$ defines a (covariant) functor $\mathscr{C}_{S}\to
\mathscr{C}_{S'}$ given by $\mathscr{F}\mapsto
\varphi^{\ast}(\mathscr{F})$. We again have:
\end{exam}

\setcounter{problem}{0}
\begin{problem}
If $u':\mathscr{F}'=\varphi^{\ast}(\mathscr{F})\to
\mathfrak{g}'=\varphi^{\ast}(\mathfrak{g})$ is an
$\mathscr{O}_{S'}$-morphism when can we say that
$u'=\varphi^{\ast}(u)$ for an $\mathscr{O}_{S}$-morphism
$u:\mathscr{F}\to \mathfrak{g}$?
\end{problem}

\begin{problem}
If\pageoriginale $\mathscr{F}'$ is any quasi-coherent
$\mathscr{O}_{S'}$-Module, when can we say that
$\mathscr{F}'=\varphi^{\ast}(\mathscr{F})$ for a 
quasi-coherent $\mathscr{O}_{S}$-Module? 
\end{problem}

\begin{exam}\label{chap7-exam3}
Let $S$, $S'$ be preschemes and let $\mathscr{C}_{S}=(\Sch/S)$,
$\mathscr{C}_{S'}=(\Sch/S')$. Suppose $\varphi:S'\to S$ is a morphism;
$\varphi$ defines a (covariant) functor
$\varphi^{\ast}:\mathscr{C}_{S}\to \mathscr{C}_{S'}$ given by
$$
X\mapsto \varphi^{\ast}(X)=X'=\fprod{X}{S'}{S}.
$$
\end{exam}

We may again pose the two problems as in the previous examples. 

It is clear that the two problems posed are of the same nature in the
different examples and admit a generalisation in the following manner:

Let $\mathscr{C}$ be a category and suppose that for every $S\in
\mathscr{C}$, we are given a category $\mathscr{C}_{S}$ such that for
every morphism $\varphi:S'\to S$ in $\mathscr{C}$, we are given a
covariant functor $\varphi^{\ast}:\mathscr{C}_{S}\to
\mathscr{C}_{S'}$. Assume, in addition that
\begin{enumerate}
\renewcommand{\labelenumi}{(\theenumi)}
\item If $S''\xrightarrow[\psi]{}S'\xrightarrow[\varphi]{}S$ is a
  sequence in $\mathscr{C}$, there is a natural isomorphism:
  $\psi^{\ast}\circ \varphi^{\ast}\simeq (\varphi\circ\psi)^{\ast}$. 

\item If
  $S'''\xrightarrow[\varphi_{3}]{}S''\xrightarrow[\varphi_{2}]{}S'\xrightarrow[\varphi_{1}]{}S$
  is a sequence in $\mathscr{C}$, the diagram, obtained from (1),
\[
\xymatrix@=1.2cm{
 & (\varphi_{1}\varphi_{2}\varphi_{3})^{\ast} &\\
\varphi^{\ast}_{3}\circ(\varphi_{1}\varphi_{2})^{\ast}\ar[ur]^{\sim} &
& (\varphi_{2}\varphi_{3})^{\ast}\circ
\varphi^{\ast}_{1}\ar[ul]_{\sim}\\
 & \varphi^{\ast}_{3}\circ\varphi^{\ast}_{2}\circ
\varphi^{\ast}_{1}\ar[ul]_{\sim}\ar[ur]^{\sim} &
}
\]
is\pageoriginale {\em commutative.}

\item $(\Id)^{\ast}=\Id$. 

\item The isomorphism
  $\psi^{\ast}\circ\varphi^{\ast}\xrightarrow{\sim} (\varphi\circ
  \psi)^{\ast}$ of (1) has the property that if either $\psi$ or
  $\varphi$ is the identity, the isomorphism also becomes the identity.
\end{enumerate}

\setcounter{problem}{0}
\begin{problem}
Given $\xi$, $\eta\in \mathscr{C}_{S}$ and a morphism
$$
u':\xi'=\varphi^{\ast}(\xi)\to
\varphi^{\ast}(\eta)=\eta'\quad\text{in}\quad \mathscr{C}_{S'} 
$$
when can we say that $u=\varphi^{\ast}(u)$ for a morphism $u:\xi\to
\eta$ in $\mathscr{C}_{S}$?
\end{problem}

\begin{problem}
Given $\xi'\in\mathscr{C}_{S'}$, when is $\xi'$ of the form
$\varphi^{\ast}(\xi)$ for a $\xi\in\mathscr{C}_{S}$?
\end{problem}

We shall now obtain certain conditions necessary for the above two
questions to have an answer in the affirmative.

\medskip
\noindent
{\bf For Problem 1.}~Suppose $\exists\ u:\xi\to \eta$ in
$\mathscr{C}_{S}$ such that $u'=\varphi^{\ast}(u):\xi'\to
\eta'$. Assume that $\fprod{S'}{S'}{S}=S''$ exists in $\mathscr{C}$;
consider the sequence:
\[
\xymatrix{
S & \ar[l]^{\varphi} S' & \ar@<-.2em>[l]_-{p_{1}}\ar@<.2em>[l]^-{p_{2}}  S''
}
\]

Let $\varphi\circ p_{1}=\varphi\circ p_{2}=\psi$ and denote by
$\xi''$, $\eta''$ the elements $\psi^{\ast}(\xi)$, $\psi^{\ast}(\eta)$
of $\mathscr{C}_{S''}$. Consider the morphism
$p^{\ast}_{1}(u):p^{\ast}_{1}(\xi')\to p^{\ast}_{2}(\xi')$; according
to our assumption $p^{\ast}_{i}\circ
\varphi^{\ast}\xrightarrow{\sim}\psi^{\ast}$.\pageoriginale There is
then a (dotted) morphism making the diagram
\[
\xymatrix@=1.2cm{
p^{\ast}_{1}(\xi')\ar[r]^{p^{\ast}_{1}(u')}\ar[d]^{\wr} &
p^{\ast}_{1}(\eta')\ar[d]^{\wr}\\ 
\xi''=\psi^{\ast}(\xi)\ar@{-->}[r] & \psi^{\ast}(\eta)=\eta''
}
\] 
commutative. In the following we also denote this morphism by
$p^{\ast}_{1}(u')$, i.e., we identify $p^{\ast}_{1}(\xi')$ with
$\xi''$ and $p^{\ast}_{1}(\eta')$ with $\eta''$. We introduce
similarly $p^{\ast}_{2}(u')$. It is clear that with these notations we
must have $$
\underline{p^{\ast}_{1}(u')=p^{\ast}_{2}(u')};
$$ 
in fact
$\alpha$: both equal $\psi^{\ast}(u)$. 

If this necessary condition is also sufficient (as they turn out to be
in some cases) we say that $\varphi:S'\to S$ is a {\em morphism of
  descent.}

\medskip
\noindent
{\bf For Problem (2).}~Suppose $\exists\ \xi\in\mathscr{C}_{S}$, with
$\varphi^{\ast}(\xi)=\xi'$. Then, with the above notations, we have:
$$
p^{\ast}_{1}(\xi')\xrightarrow{\sim}\psi^{\ast}(\xi)\xleftarrow{\sim}p^{\ast}_{2}(\xi'), 
$$
i.e., an isomorphism $\alpha:p^{\ast}_{1}(\xi')\to p^{\ast}_{2}(\xi')$
making the\pageoriginale diagram
\[
\xymatrix@=1.2cm{
p^{\ast}_{1}(\xi')\ar[dr]^{\sim}\ar[rr]^{\alpha} & &
p^{\ast}_{2}(\xi')\ar[dl]_{\sim}\\
 & \psi^{\ast}(\xi) 
}
\]
commutative. We shall now get conditions on $\alpha$. Assume that
$S'''=\fprod{S'}{\fprod{S'}{S'}{S}}{S}$ {\em exists in $\mathscr{C}$};
  let $q_{i}$ be the $i^{\text{th}}$ projection $S'''\to S'$; also let
  $p_{ji}(j\geq i)$ be the morphism $(q_{i},q_{j})_{S}:S'''\to
  S''$. Consider the sequence:
\[
\xymatrix{
S & \ar[l]^{\varphi}S' & \ar@<-.2em>[l]_{p_{1}}\ar@<.2em>[l]^{p_{2}}
S'' & \ar@<-.4em>[l]_{p_{21}}\ar[l]^{p_{31}}\ar@<.9em>[l]^{p_{32}} S'''.
}
\] 

Let $\lambda=\varphi\circ p_{1}\circ p_{21}=\varphi\circ p_{1}\circ
p_{31}=\cdots :S'''\to S$. We then have commutative diagrams of the
type:
\[
\xymatrix{
 & q^{\ast}_{1}(\xi')\ar[dl]_{\sim} && p^{\ast}_{21}
p^{\ast}_{1}(\xi')\ar[dd]^<<<<<<<<<{p^{\ast}_{21}(\alpha)}\ar[ll]_{\sim}\ar[dr]^{\sim} &\\
\lambda^{\ast}(\xi) & & & & p^{\ast}_{21}\psi^{\ast}(\xi)\ar@/^1em/[llll]_-{\sim}\\
 & q^{\ast}_{2}(\xi)\ar[ul]_{\sim} &&
p^{\ast}_{21}p^{\ast}_{2}(\xi')\ar[ll]_{\sim}\ar[ur]^{\sim} & 
}
\]

We\pageoriginale also denote by $p^{\ast}_{21}(\alpha)$ the (dotted)
morphism $q^{\ast}_{1}(\xi')\to q^{\ast}_{2}(\xi')$ which will make
the diagram
\[
\xymatrix@=1.2cm{
q^{\ast}_{1}(\xi')\ar@{-->}[d] &
p^{\ast}_{21}p^{\ast}_{1}(\xi')\ar[l]_{\sim}\ar[d]^{p^{\ast}_{21}(\alpha)}\\ 
q^{\ast}_{2}(\xi') & p^{\ast}_{21}p^{\ast}_{2}(\xi')\ar[l]_{\sim}
}
\]
commutative. Using this diagram and the ``large'' diagram above we see
that
\[
\xymatrix@C=1.5cm{
 & q^{\ast}_{1}(\xi')\ar[dd]^{p^{\ast}_{21}(\alpha)}\ar[dl]_{\sim}\\
\lambda^{\ast}(\xi) & \\
 & q^{\ast}_{2}(\xi)\ar[ul]_{\sim}
}
\]
is commutative. We make similar conventions and get similar
commutative diagrams for $p^{\ast}_{32}(\alpha)$ and
$p^{\ast}_{31}(\alpha)$. Then we must have
$$
\underline{p^{\ast}_{32}(\alpha)p^{\ast}_{21}(\alpha)=p^{\ast}_{31}(\alpha)}
$$
(the so-called ``cocycle'' condition). Finally, if $\Delta:S'\to
S''=\fprod{S'}{S'}{S}$ is the diagonal, by the assumptions made at 
the\pageoriginale beginning we must have
$$
\underline{\Delta^{\ast}(\alpha)=\text{identity}.}
$$

If these necessary conditions are also sufficient and if $\varphi$ is
also a morphism of descent, we say that $\varphi$ is a {\em morphism
  of effective descent.} An $\alpha$ of the above type is then called
a {\em descent-datum} on $\xi'$.

\begin{remark*}
Let $\mathscr{C}$ be the category of preschemes; if we agree that we
take $S$ and $S'$ locally noetherian but if we {\em don't} assume
$S'\to S$ of finite type then it may very well happen that
$S''=\fprod{S'}{S'}{S}$ is {\em not} locally noetherian (e.g. $S=\Spec
A$, $S'=\Spec \widehat{A}$, $A$ a noetherian local ring). We shall
deal with this difficulty in the sections \ref{chap7-defi7.2.1.3},
\ref{chap7-lem7.2.1.4}, \ref{chap7-lem7.2.1.5}.
\end{remark*}

\medskip
\noindent
{\bf Example (1).}

\begin{prop}\label{chap7-prop7.1.1}
A faithfully flat ring-homomorphism $\varphi:A\to A'$ is a morphism of
effective descent.
\end{prop}

\begin{proof}
A (a) We shall first show that $\varphi$ is a morphism of descent. Let
$M$, $N$ be $A$-modules and $u':\foprod{M}{A'}{A}\to \foprod{N}{A'}{A}$
be an $A'$-linear map. Consider the commutative diagram
\[
\xymatrix@=1.2cm{
M\ar[r] &
\foprod{M}{A'}{A}=M'\ar[d]^{u'}\ar@<.2em>[r]^{p_{1}}\ar@<-.2em>[r]_{p_{2}} 
 &
\foprod{M}{\foprod{A'}{A'}{A}}{A}=M''\ar@<.2em>[d]^{p^{\ast}_{2}(u')}\ar@<-.2em>[d]_{p^{\ast}_{1}(u')}\\
N\ar[r] &
\foprod{N}{A'}{A}=N'\ar@<.2em>[r]^{p_{1}}\ar@<-.2em>[r]_{p_{2}} &
\foprod{N}{\foprod{A'}{A'}{A}}{A}=N'' 
}
\]

We\pageoriginale have to show that if
$p^{\ast}_{1}(u')=p^{\ast}_{2}(u')$ then $\exists$ an $A$-linear map
  $u:M\to N$ such that $u'=\foprod{u}{I_{A'}}{A}$. This will follow,
  if we show that
\begin{itemize}
\item[(i)] for any $M\in\mathscr{C}_{A}$, $M\to \foprod{M}{A'}{A}$ is
  an injection,

\item[(ii)] for any $N\in\mathscr{C}_{A}$, $\xymatrix{N\ar[r] &
  N'\ar@<.2em>[r]^{p_{1}}\ar@<-.2em>[r]_{p_{2}} & N''}$ is exact, and

\item[(iii)] $u'(M)\subset \ker (N',p_{1},p_{2})$.
\end{itemize}

\begin{itemize}
\item[(i)] We know that $A'$ is a direct factor of
  $\foprod{A'}{A'}{A}$ by means of the map $a'\mapsto a'\otimes 1$;
  thus, for any $M\in \mathscr{C}_{A}$, $\foprod{M}{A'}{A}\to
  \foprod{M}{\foprod{A'}{A'}{A}}{A}$ makes $\foprod{M}{A'}{A}$ a
  direct summand of $\foprod{M}{\foprod{A'}{A'}{A}}{A}$ and is in
  particular an injection. As $A'$ is faithfully $A$-flat it follows
  that $M\to \foprod{M}{A'}{A}$ is an injection.

\item[(ii)] We have to prove that the sequence
\[
\xymatrix{
N\ar[r] & N'\ar@<.2em>[r]^{p_{1}}\ar@<-.2em>[r]_{p_{2}} & N''
}
\]
is exact.


It suffices to prove this after tensoring the sequence with a ring $B$
faithfully flat over $A$; note that we get a similar situation with
the pair $B\to B'=\foprod{B}{A'}{A}$ as with $A\to A'$. Take for $B$
the ring $A'$ itself. But then $A'\to \foprod{A'}{A'}{A}=A''$ admits a
section $\lambda:A''\to A'$ given by $\lambda(a'_{1}\otimes
a'_{2})=a'_{1}a'_{2}$. We may thus assume without loss of generality
that $\varphi:A\to A'$ itself admits a section $\lambda:A'\to
A$.\pageoriginale Consider then the commutative diagram:
\[
\xymatrix@=1.2cm{
N\ar[r]^{\psi=1\otimes \varphi}\ar@{<->}_{\Id.}[dr] &
\foprod{N}{A'}{A}\ar[d]^{\mu =1\otimes
  \lambda}\ar@<.2em>[r]^{p_{1}}\ar@<-.2em>[r]_{p_{2}} &
\foprod{N}{\foprod{A'}{A'}{A}}{A}\ar[d]^{\mu\otimes 1}\\
 & N\otimes A\xrightarrow{\sim}N\ar[r] &
\foprod{N}{\foprod{A}{A'}{A}}{A}\xrightarrow{\sim}\foprod{N}{A'}{A}  
}
\]

Let $x'=\sum(x_{i}\otimes a'_{i})$ be an element of
$N'=\foprod{N}{A'}{A}$ such that $p_{1}(x')=p_{2}(x')$; that is,
$\sum(x_{i}\otimes a'_{i}\otimes 1)=\sum (x_{i}\otimes 1\otimes
a'_{i})$. On applying $\mu\otimes 1$ we obtain:
$$
\sum(x_{1}\otimes\lambda(a'_{i})\otimes 1)=\sum(x_{i}\otimes 1\otimes a'_{i});
$$
under the identification
$\foprod{N}{\foprod{A}{A'}{A}}{A}\xrightarrow{\sim}\foprod{N}{A'}{A}$,
this means that
$$
\sum(x_{i}\otimes a'_{i})=\sum(\lambda(a'_{i})x_{i}\otimes
1),\quad\text{i.e.,}\quad x'=\psi(\mu(x'))\in\psi(N).
$$

\item[(iii)] By the commutativity of the diagram at the beginning of
  the proof, we have: $p^{\ast}_{1}(u')\circ p_{1}=p_{1}\circ u'$,
  $p^{\ast}_{2}(u')\circ p_{2}=p_{2}\circ u'$ and by assumption
  $p^{\ast}_{1}(u')=p^{\ast}_{2}(u')$ while for $m\in M$ one has
  $p_{1}(m)=p_{2}(m)=m\otimes 1\otimes 1\in M''$. Thus, for $m\in M$,
  $p_{1}\circ u'(m)=p_{2}\circ u'(m)$ and (iii) is proved.
\end{itemize}

\noindent
(b)~It\pageoriginale remains to show that $\varphi$ is effective. Suppose $N\in
\mathscr{C}_{A'}$ and we have a commutative diagram:
\[
\xymatrix@C=1.2cm{
 & \foprod{N'}{A'}{A}=p^{\ast}_{1}(N')\ar[dd]^{\wr \alpha}\\
N'\ar[ur]^{p_{1}}\ar[dr]_{p_{2}} &\\
 & \foprod{A'}{N'}{A}=p^{\ast}_{2}(N')
}
\]
where $\alpha$ is an $A'$-isomorphism satisfying the ``cocycle''
condition. We want to find an $N\in\mathscr{C}_{A}$ such that
$\foprod{N}{A'}{A}\xrightarrow[\rho]{\sim}N'$ and such that
\[
\xymatrix@=1.2cm{
\foprod{N}{\foprod{A'}{A'}{A}}{A}\ar[d]^{\wr}_{\text{(natural)}}\ar[r]^{p^{\ast}_{1}(\rho)}
& \foprod{N'}{A'}{A}\ar[d]^{\wr \alpha}\\
\foprod{A'}{\foprod{N}{A'}{A}}{A}\ar[r]^{p^{\ast}_{2}(\rho)} &
\foprod{A'}{N'}{A} 
}
\]
commutes.

Set $N=\ker (\alpha\circ p_{1}-p_{2})\in\mathscr{C}_{A}$ (this choice
of $N$ is motivated by (ii) of (a)). We always have an
{\em $A'$-linear} map $\foprod{N}{A'}{A}\xrightarrow{\rho}N'$. To show
that $\rho$ is an $A'$-isomorphism, it is enough to show that $\rho$
is an $A$-isomorphism; and for this, we may assume, as in (a), that
there is a section $\lambda:A'\to A$ for $\varphi:A\to A'$.  

For\pageoriginale the sake of clarity and ease, we now go back to the
general case and prove:
\end{proof}

\setcounter{lemma}{1}
\begin{lemma}\label{lem7.1.2}
If $\varphi:S'\to S$ admits a section $\sigma:S\to S'$, then $\varphi$
is a morphism of effective descent.
\end{lemma}

\begin{proof}
Let $\xi'\in\mathscr{C}_{S'}$ and $\alpha:p^{\ast}_{1}(\xi')\to
p^{\ast}_{2}(\xi')$ be an isomorphism such that
$p^{\ast}_{32}(\alpha)p^{\ast}_{21}(\alpha)=p^{\ast}_{31}(\alpha)$
(notations as before). Our aim is to find an $\eta\in \mathscr{C}_{S}$
and an isomorphism $\eta'=\varphi^{\ast}(\eta)\xrightarrow{\rho}\xi'$
such that the diagram
\[
\xymatrix@C=1.5cm{
 & p^{\ast}_{1}(\eta')\ar[r]^{\sim}_{p^{\ast}_{1}(\rho)}\ar[dl]_{\sim} &
  p^{\ast}_{1}(\xi')\ar[dd]^{\wr\,\alpha} \\
\eta'' & &\\
& p^{\ast}_{2}(\eta')\ar[ul]_{\sim}\ar[r]^{\sim}_{p^{\ast}_{2}(\rho)} &
p^{\ast}_{2}(\xi') 
}
\]
is commutative.

We have the diagram:
\[
\xymatrix@=1.4cm{
S & S'\ar[l]_{\varphi} &
\fprod{S'}{S'}{S}\ar@<.2em>[l]^{p_{2}}\ar@<-.2em>[l]_{p_{1}} \\
 & S\ar[ul]^{\text{Id.}}\ar[u]_{\sigma} & \fprod{S}{S'}{S}\xrightarrow{\sim}S'\ar[l]_-{\mathscr{P}}\ar[u]_{\fprod{\sigma}{1}{S}}
}
\]
with\pageoriginale $p_{1}\circ(\sigma \times 1)=\sigma\circ\varphi$
and $p_{2}\circ(\sigma\times 1)=$ identity. Hence, if
$\eta=\sigma^{\ast}(\xi')$ we have
$\eta'=\varphi^{\ast}\sigma^{\ast}(\xi')=(p_{1}\circ (\sigma\times
1))^{\ast}(\xi')=(\sigma\times 1)^{\ast}p^{\ast}_{1}(\xi')$ and
$(*****)^{\ast}$ $(\sigma\times 1)^{\ast}p^{\ast}_{2}(\xi')=\xi'$. We
then get a $\theta:\eta'\xrightarrow{\sim} \xi'$, namely, $\theta=(\sigma\times
1)^{\ast}(\alpha)$. We obtain thus a $\beta:p^{\ast}_{1}(\eta)\to
p^{\ast}_{2}(\eta')$ which makes the diagram
\[
\xymatrix@=1.5cm{
p^{\ast}_{1}(\xi')\ar[r]^{\sim}_{\alpha} & p^{\ast}_{2}(\xi')\\
p^{\ast}_{1}(\eta')\ar[u]_{p^{\ast}_{1}(\theta)}^{\wr}\ar[r]^{\sim}_{\beta}
& p^{\ast}_{2}(\eta')\ar[u]_{\wr}^{p^{\ast}_{2}(\theta)}
}
\]
commutative. However, we do {\em not}, in general, have a commutative
diagram
\[
\xymatrix@R=1.5cm{
p^{\ast}_{1}(\eta')\ar[rr]^{\sim}_{\beta} & & p^{\ast}_{2}(\eta')\\
 & \eta''=(p_{1}\circ
\varphi)^{\ast}(\eta)\ar[ul]_{\sim}^{\text{(natural)}}\ar[ur]^{\sim}_{\text{(natural)}} &
}
\]

Therfore,\pageoriginale we want to modify $\theta:\eta'\to \xi'$ to a
$\rho:\eta'\xrightarrow{\sim} \xi'$ by means of an automorphism $\gamma:\eta'\to
\eta'$ (i.e., $\rho=\theta\circ\gamma$) such that the diagram
\[
\xymatrix@C=1.2cm{
p^{\ast}_{1}(\eta')\ar[rr]^{\sim}_{\beta} & & p^{\ast}_{2}(\eta')\\
p^{\ast}_{1}(\eta')\ar[u]^{p^{\ast}_{1}(\gamma)} & &
p^{\ast}_{2}(\eta')\ar[u]_{p^{\ast}_{2}(\gamma)}\\
 & \text{(natural)} & \\
 & \eta''\ar[uul]_{\sim}\ar[uur]^{\sim} &
}
\]
is commutative. This $\rho$ will satisfy our requirements. We first
observe that $\beta:p^{\ast}_{1}(\eta')\to p^{\ast}_{2}(\eta')$ also
satisfies the cocycle condition. We may consider $\beta$ as an element
of $\Aut (\eta'')$. To find a $\gamma$ in $\Aut\eta'$ such as above,
i.e., such that $\beta=p^{\ast}_{2}(\gamma)\circ
p^{\ast}_{1}(\gamma)^{-1}$ we have only to prove the following:
\end{proof}

\begin{lemma}\label{lem7.1.3}
Let $\eta\in\mathscr{C}_{S}$ and consider the ``complex''
\[
\xymatrix{
(\Aut)\ar@{-}@<.3em>[r]\ar@{-}[r]\ar@{-}@<-.3em>[r] & \Aut\eta'
  \ar@<.2em>[r]^{p^{\ast}_{1}}\ar@<-.2em>[r]_{p^{\ast}_{2}} & \Aut
  \eta''
\ar@<.8em>[r]^{p^{\ast}_{31}}\ar@<-.2em>[r]^{p^{\ast}_{21}}\ar@<-.6em>[r]_{p^{\ast}_{32}}
& \Aut \eta'''
}
\]
(notations as before). If there is a section $\sigma:S\to S'$ then 
$$
\underline{H^{1}(\Aut)=(e)}.
$$
\end{lemma}

\begin{proof}
By\pageoriginale definition, the $1$-cocycles are elements $\beta\in
\Aut \eta''$ for which the cocycle condition is satisfied and the
$1$-coboundaries are those $\beta\in \Aut \eta''$ which are of the
form $\beta=p^{\ast}_{2}(\gamma)\circ p^{\ast}_{1}(\gamma)^{-1}$, for
a $\gamma\in \Aut \eta'$. Consider now the corresponding complex in
$\mathscr{C}$; the section $\sigma:S\to S'$ defines a homotopy operator
for this complex in the following manner:
\[
\xymatrix@=1.25cm{
S & S'\ar[l]_{\varphi} & S''\ar@<-.2em>[l]_{p_{1}}\ar@<.2em>[l]^{p_{2}} &
S'''\ar@<-.7em>[l]_{p_{21}}\ar@<.2em>[l]_{p_{31}}\ar@<.6em>[l]^{p_{32}}
\mbox{$\begin{matrix}\cdots\cdots\\[-10pt] \cdots\cdots\end{matrix}$}\\ 
 & S\ar[ul]^{\text{Id.}}\ar[u]_{\sigma} &
    \fprod{S}{S'}{S}\xrightarrow{\sim}
    S'\ar[l]_{\varphi}\ar[u]_{\sigma\times 1} & \ar@<-.2em>[l]_{p_{1}}\ar@<.2em>[l]^{p_{2}}
    \fprod{S}{\fprod{S'}{S'}{S}}{S}\xrightarrow{\sim} S''
    \ar[u]_{\sigma\times 1\times 1} 
}
\]
with $p_{1}\circ(\sigma\times 1)=\sigma\circ \varphi$,
$p_{2}\circ(\sigma\times 1)=\Id$.,
$$
(\sigma\times 1)\circ p_{1}=p_{21}\circ(\sigma\times 1\times 1),\quad
(\sigma\times 1)\circ p_{2}=p_{31}\circ(\sigma\times 1\times 1),
$$
and
$$
p_{32}\circ(\sigma\times 1\times 1)=\Id.
$$

If $\beta$ is a $1$-cocycle, set $\gamma=(\sigma\times
1)^{\ast}(\beta)$. We then have:
\begin{align*}
p^{\ast}_{2}(\gamma)\circ p^{\ast}_{1}(\gamma)^{-1} &=
p^{\ast}_{2}(\sigma\times 1)^{\ast}(\beta)\circ
(p^{\ast}_{1}(\sigma\times 1)^{\ast}(\beta))^{-1}\\[5pt]
&= (((\sigma\times 1)_{p_{2}})^{\ast}(\beta))\circ(((\sigma\times
1)_{p_{1}})^{\ast}(\beta))^{-1}\\[4pt]
&= (\sigma\times 1\times
1)^{\ast}p^{\ast}_{31}(\beta)\circ((\sigma\times 1\times
1)^{\ast}p^{\ast}_{21}(\beta))^{-1}\\[4pt] 
&= (\sigma\times 1\times 1)^{\ast}(p^{\ast}_{31}(\beta)\cdot
p^{\ast}_{21}(\beta)^{-1})\\[4pt] 
&= (\sigma\times 1\times 1)^{\ast}p^{\ast}_{32}(\beta)=\beta.
\end{align*}
\hfill Q.E.D.
\end{proof}

\setcounter{exam}{1}
\begin{exam}%2
$\mathscr{C}=(\Sch)$,\pageoriginale $S$, $S'\in(\Sch)$;
  $\mathscr{C}_{S}$(resp. $\mathscr{C}_{S'}$) is the category of
  quasi-coherent $\mathscr{O}_{S}$-(resp. $\mathscr{O}_{S'}$-)-Modules.
\end{exam}

\setcounter{prop}{3}
\begin{prop}\label{chap7-prop7.1.4}
If $\varphi:S'\to S$ is faithfully flat, quasi compact then $\varphi$
is a morphism of effective descent.
\end{prop}

(A morphism $f$ is quasi-compact if $f^{-1}(U)$ is quasi-compact for
every quasi-compact: $U$).

({\em Note:}~We do {\em not} assume here that $\varphi$ is a morphism
of finite type-our hypothesis is much weaker).

\begin{proof}
{\bf Case~(a)} {\em $S$, $S'$ both affine.} The proposition in
  this case follows from Proposition \ref{chap7-prop7.1.1}.

\medskip
\noindent

{\bf Case~(b)} {\em $S$ affine.} There is a finite affine open
  cover $(S'_{i})_{i\in I}$ of $S'$ ($\varphi$ quasi-compact). Set
  $\widetilde{S}=\coprod\limits_{i\in I}S'_{i}$. Then $\widetilde{S}$
  is affine and the morphism $\widetilde{S}\xrightarrow{\psi}S$
  (composite of the natural map $\theta:\widetilde{S}\to S'$ and
  $\varphi$) is also faithfully flat. Now consider the diagram:
\[
\xymatrix@C=1.2cm{
 & \widetilde{S}\ar[dl]_{\psi}\ar[dd]_{\theta}^{\text{(natural)}} &
  \fprod{\widetilde{S}}{\widetilde{S}}{S}
\ar@<-.2em>[l]_{q_{1}}\ar@<.2em>[l]^{q_{2}}\ar[dd]
  & \fprod{\widetilde{S}}{\fprod{\widetilde{S}}{\widetilde{S}}{S}}{S} 
\ar@<-.7em>[l]_{q_{31}}\ar@<.2em>[l]_{q_{32}}\ar@<.6em>[l]^{q_{21}}\\
S & & \\
& S'\ar[ul]^{\varphi} & \fprod{S'}{S'}{S}
\ar@<-.2em>[l]_{p_{1}}\ar@<.2em>[l]^{p_{2}} &
\fprod{S'}{\fprod{S'}{S'}{S}}{S} \ar@<-.7em>[l]_{p_{31}}\ar@<.2em>[l]_{p_{32}}\ar@<.6em>[l]^{p_{21}}
}
\]

If\pageoriginale $\xi$, $\eta\in\mathscr{C}_{S}$ and $u':\xi\to \eta'$
is a $\mathscr{C}_{S'}$-morphism then $u'$ defines a
$\mathscr{C}_{\widetilde{S}}$-morphism
$\widetilde{u}:\psi^{\ast}(\xi)\to \psi^{\ast}(\eta)$. And from the
equality $p^{\ast}_{1}(u')=p^{\ast}_{2}(u')$ and the commutativity of
the above diagram follows
$q^{\ast}_{1}(\widetilde{u})=q^{\ast}_{2}(\widetilde{u})$. By case
(a), $\exists$ a $u:\xi\to \eta$, a $\mathscr{C}_{S}$-morphism, such
that $\widetilde{u}=\psi^{\ast}(u)$; it is immediate that
$u'=\varphi^{\ast}(u)$; in fact this holds in every open set
$S'_{i}$. 

If $\xi'\in\mathscr{C}_{S'}$ and $\alpha$ is a descent-datum for
$\xi'$, $\widetilde{\alpha}$ defined in the obvious way is a
descent-datum for $\theta^{\ast}(\xi')=\widetilde{\xi}$ and again by
case (a), $\exists \xi\in\mathscr{C}_{S}$ such that
$\widetilde{\xi}\xleftarrow{\sim}\psi^{\ast}(\xi)$. It is easy to see
that $\xi'\xleftarrow{\sim}\varphi^{\ast}(\xi)$. 

\medskip
{\bf Case~(c).} {\em $S$, $S'$ arbitrary.} Let $(T_{j})$ be an affine
open cover of $S$ and $T'_{j}=\varphi^{-1}(T_{j})$. Then the $T'_{j}$
form an open cover of $S'$. Let $T=\coprod\limits_{j}T_{j}$ and
$\theta:T\to S$ the natural map. If $\mathscr{F}$ and $\mathscr{G}$
are quasi-coherent $\mathscr{O}_{S}$-Modules denote by $\Phi(S)$ the
group $\Hom_{S}(\mathscr{F},\mathscr{G})$, by $\Phi(S')$ the group
$\Hom_{S'}(\mathscr{F}',\mathscr{G}')$ and so on. If we set
$T'=\coprod\limits_{j}T'_{j}$ and $\theta:T'\to S'$ is the natural
map, we have a commutative diagram:
\[
\xymatrix{
\Phi(S)\ar[d]\ar[r]^{\varphi^{\ast}} &
\Phi(S')\ar[d]\ar@<.2em>[r]^{p^{\ast}_{1}}\ar@<-.2em>[r]_{p^{\ast}_{2}}
& \Phi(S'')\\
\Phi(T)\ar[r]\ar@<.2em>[d]\ar@<-.2em>[d] &
\Phi(T')\ar@<.2em>[r]^{p^{\ast}_{1}}\ar@<-.2em>[r]_{p^{\ast}_{2}}
\ar@<.2em>[d]\ar@<-.2em>[d] &
\Phi(T'')\\
\Phi(\fprod{T}{T}{S})\ar[r] & \Phi(\fprod{T'}{T'}{S'}) & 
}
\]

By\pageoriginale case (b), the morphisms $T'_{j}\to T_{j}$ are
morphisms of effective descent and therefore clearly so is
$T'=\coprod\limits_{j}T'_{j}\to \coprod\limits_{j}T_{j}=T$. It follows
that the second row is exact while $\Phi(\fprod{T}{T}{S})\to
\Phi(\fprod{T'}{T'}{S'})$ is an injection. As it is clear that
$\theta:T\to S$ is a morphism of (effective) descent, the first column
is also exact. Usual diagram-chasing shows that the first row is also
exact, in other words, that $\varphi$ is a morphism of descent. We
show similarly that $\varphi$ is also effective.\hfill Q.E.D.
\end{proof}

\begin{exam}%3
We shall not discuss the problems (1) and (2) in their general form,
in this case. However, if we restrict ourselves to the case of {\em
  preschemes affine over} $S$, $S'$ then {\em a faithfully flat,
  quasi-compact morphism $S'\to S$ is a morphism of effective
  descent.} In fact, such preschemes are defined by quasi-coherent
$\mathscr{O}_{S}$-and $\mathscr{O}_{S'}$-Algebras and we are
essentially back to example (2). [It is not difficult to see that $u$
  (resp. $\mathscr{F}$) in example (2), problem (1) (resp. problem
  (2)) is an $\mathscr{O}_{S}$-Algebra homomorphism (resp. an
  $\mathscr{O}_{S}$-Algebra) provided we start with
  $\mathscr{O}_{S}$-Algebras and homomorphisms of
  $\mathscr{O}_{S}$-Algebras instead of $\mathscr{O}_{S}$-Modules
  (look at the proof of Proposition \ref{chap7-prop7.1.1})].
\end{exam}


\section{}\label{chap7-sec7.2}

Let $S$ be a locally noetherian prescheme and $X$, $Y$ be \'etale
coverings of $S$. Let $S_{0}\hookrightarrow S$ be a closed subscheme
of $S$ defined by a Nil-Ideal $\mathscr{F}$ of $\mathscr{O}_{S}$
($\mathscr{F}$ is a Nil-Ideal of $\mathscr{O}_{S}\Leftrightarrow
\mathscr{F}_{s}\subset$ nil-radical of $\mathscr{O}_{sS}$, $\forall
s\in S$; this is equivalent to saying that\pageoriginale $S_{0}$ and
$S$ have the same base space or $(S_{0})_{\red}=S_{\red}$). We then
have an obvious functor
$(\mathscr{E}t/S)\xrightarrow{\Phi}(\mathscr{E}t/S_{0})$ given by
$X\mapsto \fprod{X}{S_{0}}{S}=X_{0}$. We assert that $\Phi$ defines an
equivalence of categories in the sense of the following

\subsection{Main Theorem}\label{chap7-sec7.2.1}
\begin{itemize}
\item[(a)] $\Hom_{S}(X,Y)\xrightarrow{\sim}\Hom_{S_{0}}(X_{0},Y_{0})$

\item[(b)] If $X_{0}\in(\mathscr{E}t/S_{0})$, then $\exists\
  X\in(\mathscr{E}t/S)$ 
\end{itemize}
and an isomorphism $\fprod{X}{S_{0}}{S}\xrightarrow{\sim}X_{0}$. 

The theorem will follow from the series of lemmas below. 

\setcounter{subsection}{1}
\setcounter{sublemma}{0}
\begin{sublemma}\label{chap7-lem7.2.1.1}
Let $S$ be locally noetherian and $f:X\to S$ a separated morphism of
finite type. Then $f$ is an open immersion $\Leftrightarrow f$ is
\'etale and universally injective.
\end{sublemma}

\noindent
({\em Note:}~Universally injective - radiciel = injective + radiciel
residue field extensions).

\begin{proof}
$\Rightarrow$ clear.

$\Leftarrow : f$~ is an open map and thus by passing to an open
  sub-scheme of $S$, we may assume that $f(X)=S$. Clearly $f$ is then
  a homeomorphism onto. Since \'etale remains \'etale under
  base-change, $f_{(S')}$ is still a homeomorphism onto, for any
  base-change $S'\to S$; it follows that $f$ is universally closed and
  thus proper. By Chevalley's lemma we deduce that $f$ is
  finite. Suppose then that $X=\Spec \mathscr{A}$, where $\mathscr{A}$
  is a coherent $\mathscr{O}_{S}$-Algebra. As $f$ is
  \'etale\pageoriginale and universally injective, it follows that
  $\mathscr{A}$ is locally free of rank 1 at every point of $S$, hence
  $X\xrightarrow{\sim}S$.\hfill Q.E.D.
\end{proof}

\begin{sublemma}\label{chap7-lem7.2.1.2}
Let $f:X\to S$ be a separated \'etale morphism of finite type ($S$
locally noetherian). Suppoe $Y\to S$ is a morphism of finite type and
$Y_{0}=V(\mathscr{F})$ is a closed subscheme of $Y$ defined by a
Nil-Ideal $\mathscr{F}$ of $\mathscr{O}_{Y}$. Let $\alpha_{0}:Y_{0}\to
X$ be an $S$-morphism. Then $\exists$ a {\em unique} $S$-morphism
$\alpha:Y\to X$ making the diagram
\[
\xymatrix@=1.3cm{
X\ar[d]_{f} & Y_{0}\ar[l]_{\alpha_{0}}\ar@{_{(}->}[d]\\
S & Y\ar[l]\ar[ul]_{\alpha}
}
\]
commutative.
\end{sublemma}

\begin{proof}
By making the base-change $Y\to S$ and considering the morphism
$Y_{0}\to \fprod{X}{Y}{S}$ we are reduced to proving the lemma in the
case $S=Y$. That is, given a $Y$-morphism
$Y_{0}\xrightarrow{\alpha_{0}}X$, we want to extend $\alpha_{0}$ to a
section $\alpha$ of $X\to Y$.

Now suppose $\sigma:Y\to X$ is a section of $X\xrightarrow{f}Y$;
$f\cdot \sigma=$ identity is \'etale and $f$ is \'etale, hence
$\sigma$ is \'etale. Also $f\cdot \sigma$ is a closed immersion and
$f$ is separated and thus $\sigma$ is a closed immersion. $Y$ being
locally noetherian its connected components are open and we may then
assume $Y$ connected. $\sigma(Y)$ will then be a connected component
of $X$, isomorphic to $Y$ under $\sigma$; in view of lemma
\ref{chap7-lem7.2.1.1} then, the sections of $f$ are in $(1-1)$
correspondence\pageoriginale with the components $X_{i}$ of $X$ such
that $f|X_{i}$ {\em is surjective and universally injective} on
$X_{i}$ to $Y$.

Now $Y_{0}$, $Y$ have the same base-space and hence so have
$X_{0}=\fprod{X}{Y_{0}}{Y}$ and $X$; also, the morphism
$f_{(Y_{0})}:X_{0}\to Y_{0}$ obtained from $f$ by the base-change
$Y_{0}\to Y$ is topologically the same map as $f$; hence, $f$ is
universally injective and surjective on $X_{i}$ to $Y\Leftrightarrow
f_{(Y_{0})}$ is so on $(X_{i})_{0}$ to $Y_{0}$, i.e., the set of
sections of $X\xrightarrow{f}Y$ is the same as the set of sections of
$X_{0}\xrightarrow{f_{(Y_{0})}}Y_{0}$ and the $(1-1)$ correspondence
is given in the obvious way. The lemma now follows from the fact that
$Y_{0}\xrightarrow{\alpha_{0}}X$ can be considered as a section for
$X_{0}\xrightarrow{f_{(Y_{0})}}Y_{0}$.\hfill Q.E.D.
\end{proof}

This lemma proves part (a) of Theorem \ref{chap7-sec7.2.1}. To prove part
(b) we need a generalisation of the lemma.

\medskip
\noindent
{\bf Warning.}~In the following sections
\ref{chap7-defi7.2.1.3}--\ref{chap7-lem7.2.1.5} until the proof of (b)
of Theorem 
\ref{chap7-sec7.2.1}, we drop the assumptions made in
\ref{chap6-sec6.1} that the 
preschemes are locally noetherian and the morphisms are of finite
type. We begin by making the 

\setcounter{subdefin}{2}
\begin{subdefin}\label{chap7-defi7.2.1.3}
A morphism $f:X\to Y$ of preschemes is of {\em finite presentation} if
\begin{itemize}
\item[(i)] $f$ is quasi-compact

\item[(ii)] $\Delta:X\to \fprod{X}{X}{Y}$ is quasi-compact,

\item[(iii)] for every $x\in X$, $\exists$ a nbd. $U_{x}$ of $x\in X$
  and a nbd. $V_{f(x)}$ of $f(x)$ in $Y$ such that $f(U_{x})\subset
  V_{f(x)}$ and\pageoriginale $\Gamma(U_{x},\mathscr{O}_{X})$ is a
  $\Gamma(V_{f(x)},\mathscr{O}_{Y})$-algebra of finite presentation;
  i.e., $\Gamma(U_{x},\mathscr{O}_{X})\cong
  \Gamma(V_{f(x)},\mathscr{O}_{Y})[T_{1},\ldots,T_{s}]/\mathfrak{a}$
  where $\mathfrak{a}$ is a finitely generated ideal of a polynomial
  algebra\break $\Gamma(V_{f(x)},\mathscr{O}_{Y})[T_{1},\ldots,T_{s}]$. 
\end{itemize}
\end{subdefin}

An example of a morphism of finite presentation is a finite type
separated morphism $X\xrightarrow{f}Y$ with $Y$ locally noetherian.

\setcounter{sublemma}{3}
\begin{sublemma}\label{chap7-lem7.2.1.4}
Let $U$ be a {\em noetherian} prescheme and
$(\mathscr{A}_{\alpha})_{\alpha \in I}$ an inductive family of
quasi-coherent $\mathscr{O}_{U}$-Algebras. Let $V_{\alpha}$ be the
$U$-affine prescheme $\Spec \mathscr{A}_{\alpha}$ defined by
$\mathscr{A}_{\alpha}$, $\forall_{\alpha\in I}$, and
$V=\Spec\mathscr{A}$ where
$\mathscr{A}=\varinjlim_{\alpha}\mathscr{A}_{\alpha}$. 
\begin{itemize}
\item[\rm(a)] Suppose we have a diagram:
\[
\xymatrix@C=.3cm@R=1.2cm{
& & X_{\alpha}\ar[dr] & & Y_{\alpha}\ar[dl] & & X_{\beta}\ar[dr] & &
  X_{\beta}\ar[dl] & & X\ar[dr] & & Y\ar[dl]\\
U & & & V_{\alpha}\ar[lll] & & & & V_{\beta}\ar[llll] & & & & V'\ar[llll]  & 
}
\]
where $X_{\alpha}$, $Y_{\alpha}$ are finitely presented preschemes
over $V_{\alpha} \forall\ \alpha \in I$ such that $\forall\beta\geq
\alpha$, $X_{\beta}=\fprod{X_{\alpha}}{V_{\beta}}{V_{\alpha}}$
$$
Y_{\beta}=\fprod{Y_{\alpha}}{V_{\beta}}{V_{\alpha}};\quad\text{let}\quad
X=\fprod{X_{\alpha}}{V}{V_{\alpha}},\quad Y=\fprod{Y_{\alpha}}{V}{V_{\alpha}}
$$


Then,
$\varprojlim_{\alpha}\Hom_{V_{\alpha}}(X_{\alpha},Y_{\alpha})\xrightarrow{\sim}\Hom_{V}(X,Y)$.

\item[\rm(b)] Suppose\pageoriginale is a finitely presented prescheme
  over $V$. Then, for all large $\alpha\in I$, $\exists\
  X_{\alpha}/V_{\alpha}$, finitely presented, such that (i) $\forall
  \beta\geq \alpha$, $X_{\beta}\simeq
  \fprod{X_{\alpha}}{V_{\beta}}{V_{\alpha}}$ and (ii) $X\simeq
  \fprod{X_{\alpha}}{V}{V_{\alpha}}$.

\item[\rm(c)] With assumptions as in (a), if the $Y'_{\alpha}$s and $Y$ are
  locally noetherian, $X/Y$ \'etale covering $\Rightarrow
  X_{\alpha}/Y_{\alpha}$ \'etale covering already, for some $\alpha\in I$. 
\end{itemize}
\end{sublemma}

\begin{proof}
\begin{itemize}
\item[(a)]~ We shall be content with merely observing that it is
clear how to prove this in case everything is affine.

\smallskip

\item[(b)]~ Since $U$ is noetherian, in view of (a) we may assume that
  $U=\Spec C'$, $V_{\alpha}=\Spec A_{\alpha}$, $V=\Spec A$ where
  $A=\varprojlim_{\alpha}A_{\alpha}$. 
\end{itemize}

\noindent
{\bf Case (1).}~{\em Assume $X=\Spec B$.}

$X/V$ finitely presented $\Rightarrow B/A$ finitely presented; let
$$
B\cong A[T_{1},\ldots,T_{s}]/\mathfrak{a}, \mathfrak{a}
$$ 
finitely
generated, say, by $P_{1},\ldots,P_{n}\in
A[T_{1},\ldots,T_{s}]$. Choose $\alpha$ so large that the coefficients
of the $P_{i}$ come from $A_{\alpha}$. Consider the ideal
$\mathfrak{a}_{\alpha}=(P_{1},\ldots,P_{n})$ of
$A_{\alpha}[T_{1},\ldots,T_{s}]$ and set
$B_{\alpha}=A_{\alpha}[T_{1},\ldots,T_{s}]/\mathfrak{a}_{\alpha}$ and
$X_{\alpha}=\Spec B_{\alpha}$; for $\beta\geq \alpha$, set
$X_{\beta}=\fprod{X_{\alpha}}{V_{\beta}}{V_{\alpha}}$. 

\medskip
\noindent
{\bf Case (2).}~{\em $X$ arbitrary.}

Let $(X_{i})^{r}_{i=1}$ be a finite affine open cover of $X$. In view
of case (1) and (a), it is enough now to show that each $X_{i}\cap
X_{j}$ is quasi-compact. But the underlying space of $X_{i}\cap X_{j}$
is that of $\fprod{X}{(\fprod{X_{i}}{X_{j}}{V})}{(\fprod{X}{X}{V})}$
as is seen from the commutative diagram:\pageoriginale
\[
\xymatrix@=1.2cm{
\fprod{X}{(\fprod{X_{i}}{X_{j}}{V})}{(\fprod{X}{X}{V})}\ar[r]\ar[d] &
\fprod{X_{i}}{X_{j}}{V}\ar[d]^{\text{(natural)}}\\
X\ar[r]^{\Delta} & \fprod{X}{X}{V}
}
\]

As $\Delta$ is a quasi-compact morphism the ``top''-morphism is also
\break quasi-compact; since $\fprod{X_{i}}{X_{j}}{V}$ is quasi-compact, our
result follows.

\smallskip

(c)~ Here again we give only some indications and leave the details to
the reader. To start with we may assume everything affine. Say
$U=\Spec C$, $V_{\alpha}=\Spec A_{\alpha}$, $V=\Spec A$,
$X_{\alpha}=\Spec B^{\alpha}_{1}$, $Y_{\alpha}=\Spec B^{\alpha}_{2}$,
$X=\Spec B_{1}$, $Y=\Spec B_{2}$. We also note that we can assume
$V_{\alpha}=Y_{\alpha}$, and $V=Y$. One checks that
$\Omega_{B_{1/B_{2}}}\xleftarrow{\sim}\varprojlim_{\alpha}\Omega_{B^{\alpha}_{1}/B^{\alpha}_{2}}$,
  say. Now if $X/Y$ is unramified then $\Omega_{B_{1}/B_{2}}=0$. 
But since $\Omega_{B_{1}^{\alpha_{0}}/B^{\alpha_{0}}_{2}}$ is a finite
$B^{\alpha_{0}}_{1}$-module one has
$\Omega_{B^{\alpha}_{1}/B^{\alpha}_{2}}=0$  for large $\alpha$. Also,
if $X/Y$ is finite and flat then $B_{1}$ is a locally free
$B_{2}$-module of finite rank but then the same is true for
$B^{\alpha}_{1}$ with respect to $B^{\alpha}_{2}$ for large
$\alpha$. Assertion (c) follows.\hfill Q.E.D.
\end{proof}

\begin{sublemma}\label{chap7-lem7.2.1.5}
Suppose\pageoriginale we have a commutative diagram:
\[
\xymatrix@=1.2cm{
X\ar[d]^{f} & X_{(T)}=\fprod{X}{T}{S}\ar[l]\ar[d] &
Y_{0}=V(\mathscr{F})\ar@{_{(}->}[d]\ar[l]_{\alpha_{0}}\\
S & T\ar[l] & Y\ar[l]
}
\]
with: $S$ noetherian, $f$ \'etale and separated, $T\to S$ affine,
$Y\to T$ finitely presented and $Y_{0}=V(\widetilde{\mathscr{F}})$ where
$\mathscr{F}$ is a Nil-Ideal of $\mathscr{O}_{Y}$. Then $\exists$ a
{\em unique} $\alpha:Y\to X_{(T)}$ keeping the diagram still
commutative. 
\end{sublemma}

\begin{proof}
Since $T\to S$ is affine, $T=\Spec \mathscr{B}$, where $\mathscr{B}$
is a quasi-coherent $\mathscr{O}_{S}$-Algebra. Write
$\mathscr{B}=\varprojlim_{\lambda}\mathscr{B}_{\lambda}$ where the
$\mathscr{B}'_{\lambda}$s are $\mathscr{O}_{S}$-sub-Algebras, of
finite type, of $\mathscr{B}$. Set $T_{\lambda}=\Spec
\mathscr{B}_{\lambda}$. By (a) and (b) of Lemma \ref{chap7-lem7.2.1.4} we
have, for large $\lambda$, a situation of the following type:
{\fontsize{8}{10}\selectfont
\[
\xymatrix{
 & \fprod{X}{T_{\lambda}}{S}\ar[dl]\ar@{<--}@/^1em/[dr]^>>>>{\alpha_{\lambda}?}\ar[dd] & & \ar[ll]_{\alpha_{0,\lambda}}Y_{0,\lambda}\ar[dd]\\
SX\ar[dd] & & \fprod{X}{T}{S}\ar[ll]\ar[dd]\ar@{--}@/_1em/[dr] & & Y_{0}=V(\mathscr{F})=\fprod{Y_{0,\lambda}}{Y}{Y_{\lambda}}\ar[ul]\ar@{^{(}->}[dd]\\
 & \Spec \mathscr{B}_{\lambda}=T_{\lambda}\ar[dl] & &
Y_{\lambda}\ar[ll] &\\
S\ S & & \Spec\mathscr{B}=T\ar[ll]\ar[ul] & & Y=\fprod{Y_{\lambda}}{T}{T_{\lambda}}\ar[ll]\ar[ul]
}
\]}\relax

For\pageoriginale large $\lambda$, $\exists$ inductive families
$(Y_{\lambda})$, $(Y_{0,\lambda})$ and morphisms
$\alpha_{0,\pi}:Y_{0,\lambda}\to \fprod{X}{T_{\lambda}}{S}$ such that
$\varinjlim_{\lambda}\alpha_{0,\lambda}=\alpha_{0}$ (see
diagram). Also it is not very difficult to see (using arguments
similar to those in Lemma \ref{chap7-lem7.2.1.4}) that the $Y_{0,\lambda}$
are subschemes defined by Nil-Ideals $\mathscr{F}_{0,\lambda}$ of
$\mathscr{O}_{Y_{\lambda}}$. Now $T_{\lambda}$ is of finite type over
$S$ and is therefore noetherian; also, since $Y\to T$ is of finite
presentation so is $Y_{\lambda}\to T_{\lambda}$ and is in particular
of finite type. Hence Lemma \ref{chap7-lem7.2.1.2} applies and we get a
(dooted) morphism $\alpha_{\lambda}:Y_{\lambda}\to
\fprod{X}{T_{\lambda}}{S}$ keeping the diagram commutative. Passing to
the limit with respect to $\lambda$ we get an $\alpha:T\to
\fprod{X}{T}{S}$ making the diagram commutative. The uniqueness of
$\alpha$ follows from Lemma \ref{chap7-lem7.2.1.2} and (a) of Lemma
\ref{chap7-lem7.2.1.4}.\hfill Q.E.D.
\end{proof}

\setcounter{subsubsection}{5}
\subsubsection{Proof of (b) of Theorem
  \ref{chap7-sec7.2.1}}\label{chap7-sec7.2.1.6} 

~

\medskip
\noindent
{\bf Case 1.}~Assume $S=\Spec A$, $A$ a noetherian, complete local
ring; $S_{0}$ is then given by $\Spec (A/J)=\Spec A_{0}$, $J$ a
nil-ideal of $A$. Let $X_{0}\xrightarrow{f_{0}}S_{0}$ be the given
\'etale covering; assume that if $s_{0}\in S_{0}$ is the unique closed
point, the residual extensions at the points of $f^{-1}_{0}(s_{0})$
are all trivial.

Then $X_{0}$ is given by $\Spec B_{0}$ where $B_{0}$ is a finite
direct-product of copies of $A_{0}$, say
$B_{0}=\bigoplus^{r}_{i=1}A_{0}$; $X=\Spec B$, with
$B=\bigoplus^{r}_{i=1}A$ is then a solution for our problem.

\medskip
\noindent
{\bf Case 2.}~From\pageoriginale among the assumptions in case 1, drop
completeness of $A$ and the triviality of residual extensions along
the fibre $f^{-1}_{0}(s_{0})$.
\[
\xymatrix@R=1cm{
S=\Spec A & & \Spec A'=S'\ar[ll]_{\text{faithfully
    flat}}^{\text{quasi-compact}}\\
 & & \\
 & X_{0}\ar[d]^{f_{0}} &
X'_{0}=\fprod{X_{0}}{S'_{0}}{S_{0}}\ar[d]^{f'_{0}}\ar[l]\\
 &\ar@{^{(}->}[uuul] \Spec A_{0}=S_{0} & S'_{0}=\Spec \left(\frac{A'}{JA'}\right)\ar[l]
}
\]

In this case we can choose a complete, noetherian, local overring $A'$
of $A$ such that the following situation holds:
{\em and such that} if $s'_{0}\in S'_{0}$ is the unique closed point,
then the residual extensions along the fibre ${f'}^{-1}_{0}(s'_{0})$
{\em are all trivial.} By case 1, $\exists$ an \'etale covering
$X'/S'$ such that $X'_{0}\xleftarrow{\sim}\fprod{X'}{S'_{0}}{S'}$. 


We\pageoriginale have then the following situation:
\begin{landscape}
\begin{equation*}
\xymatrix@C=.6cm@R=1.1cm{
 & X'\ar[d] & p^{\ast}_{1}(X')\ar[dr] & & p^{\ast}_{2}(X')\ar[dl] & \\
S & \ar[l] S' & & \ar@<-.2em>[ll]_{p_{1}}\ar@<.2em>[ll]^{p_{2}} S''=\fprod{S'}{S'}{S} & &
S'''=\fprod{S'}{\fprod{S'}{S'}{S}}{S} \ar@<-.8em>[ll]_{p_{1}}\ar@<-.3em>[ll]^{p_{31}}\ar@<.6em>[ll]^{p_{2}}\\ 
 & & \\
 & X_{0}\ar[d]^{f_{0}} &
X'_{0}\ar[d]^{f'_{0}}\ar[l] & p^{\ast}_{1}(X'_{0})\ar[dr] & &
p^{\ast}_{2}(X'_{0})\ar[dl] &\\
 &\ar@{^{(}->}[uuul] S_{0} & S'_{0}\ar[l] & &
S''_{0}=\fprod{S'_{0}}{S'_{0}}{S_{0}}\ar@<-.2em>[ll]_{p_{1}}\ar@<.2em>[ll]^{p_{2}}
& & S'''_{0}=\fprod{S'''_{0}}{\fprod{S'_{0}}{S''_{0}}{S_{0}}}{S_{0}} 
\ar@<-.8em>[ll]_{p_{21}}\ar@<-.3em>[ll]^{p_{31}}\ar@<.6em>[ll]^{p_{32}}
}
\end{equation*}
\end{landscape}

Since $X'_{0}/S'_{0}$ ``comes from below'', it clearly has a
descent-datum
$\alpha_{0}:p^{\ast}_{1}(X'_{0})\xrightarrow{\sim}p^{\ast}_{2}(X'_{0})$
satisfying the cocycle condition. We want to ``lift'' this isomorphism
$\alpha_{0}$ to an $\alpha:p^{\ast}_{1}(X')\to p^{\ast}_{2}(X')$. For
this one may be tempted to use part (a) of Theorem \ref{chap7-sec7.2.1}
which we already have proved. But observe that we are now in a type of
situation we anticipated while making the remark preceding Proposition
\ref{chap7-prop7.1.1} -- we {\em cannot} make sure that $S''$, $S''_{0}$ are
locally noetherian and hence {\em cannot} apply part (a). {\em It is
  here that Lemma \ref{chap7-lem7.2.1.5} comes to our rescue.} By using this
lemma in the obvious way we get an isomorphism
$\alpha:p^{\ast}_{1}(X')\xrightarrow{\sim}p^{\ast}_{2}(X')$;
the\pageoriginale assertion of uniqueness in this lemma proves that
the $\alpha$ we obtained satisfies the cocycle condition. As $S'\to S$
is faithfully flat and quasi-compact, it is a morphism of effective
descent for \'etale coverings (an easy corollary to Proposition
\ref{chap7-prop7.1.4}) and thus $\exists$ an \'etale covering $X/S$ such
that $X'\xleftarrow{\sim}\fprod{X}{S'}{S}$. Therefore
$$
X'_{0}\xleftarrow{\sim} \fprod{X'}{S'_{0}}{S'}\simeq
\fprod{X}{\fprod{S'}{S'_{0}}{S'}}{S}\simeq
\fprod{(\fprod{X}{S_{0}}{S})}{S'_{0}}{S_{0}}. 
$$

Also, by construction, the descent-datum for $X'/S'$ goes down to that
for $X'_{0}/S'_{0}$. It follows that
$\fprod{X}{S_{0}}{S}\xrightarrow{\sim}X_{0}$. 

\medskip
\noindent
{\bf Case 3.}~{\em $S$ arbitrary.}
\smallskip

By part (a), which we have already proved, it is enough to prove the
existence of an \'etale covering $X/S$ (with the required property)
{\em locally.}

Let $s\in S$ and $U=\Spec A$ be an affine open neighbourhood of
$s$. We may assume $S=\Spec A$, A noetherian. The local ring $A_{s}$
is given by $A_{s}=\varinjlim_{\substack{f\in A\\ f(s)\neq 0}}
A_{f}$; we then have $S\leftarrow \Spec A_{f}\leftarrow \Spec A_{s}$
and the given \'etale covering $X_{0}$ over $S_{0}=\Spec (A/J)$, $J$ a
nil-ideal, defines an \'etale covering $X_{s,0}$ over $\Spec
(A_{s}/JA_{s})=\Spec (A/J)_{s}$ and then, by case 2, an \'etale
covering $X_{s}$ over $\Spec A_{s}$. By (b) and (c) of lemma
\ref{chap7-lem7.2.1.4}, $\exists f\in A$, $f(s)\neq 0$ and an \'etale
covering $X_{f}/\Spec A_{f}$ such that
$X_{s}\xleftarrow{\sim}\foprod{X_{f}}{A_{s}}{A_{f}}$. The\pageoriginale
covering $X_{f}$ is a solution for our problem in the neighbourhood
$\Spec A_{f}$ of $s$.\hfill Q.E.D.

\begin{remark*}
The Theorem \ref{chap7-sec7.2.1} shows that if $S_{0}\hookrightarrow S$ is
such that $(S_{0})_{\red}\break =S_{\red}$ then the natural functor
$(\mathscr{E}t/S)\xrightarrow{\Phi}(\mathscr{E}t/S_{0})$ is an
equivalence; in particular, it proves that
$\pi_{1}(S_{\red},s)\xrightarrow{\sim}\pi_{1}(S,s)$. 
\end{remark*}

\setcounter{prop}{1}
\begin{prop}\label{chap7-prop7.2.2}
Let $S'\xrightarrow{f}S$ be a faithfully flat, quasi-compact, radiciel
morphism. Then the natural functor
$(\mathscr{E}t/S)\xrightarrow{\Phi}(\mathscr{E}t/S')$ is an equivalence.
\end{prop}

\begin{proof}
Consider the diagonal $S'\xrightarrow{\Delta}S''=\fprod{S'}{S'}{S}$.

Since $f$ is radiciel $\Delta(S')=S''$ (Cor. (3.5.10) -- EGA, Ch.I),
i.e. $S'$ is a closed subscheme of $S''$, having the same base-space.
\begin{itemize}
\item[(a)] Given $X$, $Y\in (\mathscr{E}t/S)$, set
  $X'=\fprod{X}{S'}{S}$, $Y'=\fprod{Y}{S'}{S}$.
\end{itemize}

To prove $\Hom_{S}(X,Y)\xrightarrow{\sim}\Hom_{S'}(X',Y')$.

Since $f:S'\to S$ is faithfully flat, quasi-compact it is a morphism
of descent for \'etale coverings. We have thus only to show that if
$u'\in \Hom_{S'}(X',Y')$, and if $\xymatrix{S' &
  \ar@<-.2em>[l]_-{p_{1}}\ar@<.2em>[l]^-{p_{2}}
  S''=\fprod{S'}{S'}{S}}$ are the canonical projections, then,
considered as morphisms from $X''=\fprod{X'}{S''}{S'}$ to
$Y''=\fprod{Y'}{S''}{S'}$, $p^{\ast}_{1}(u)$ and $p^{\ast}_{2}(u)$ are
equal. We have\pageoriginale the diagram:
\[
\xymatrix@C=.4cm@R=1.4cm{
S && \ar[ll] S' && & \ar@<-.2em>[lll]\ar@<.2em>[lll]
S''=\fprod{S'}{S'}{S} &&& S'\ar[lll]_-{\Delta} &\\
& X'\ar[ur]\ar[rr]^{u'} && Y'\ar[ul] & 
X''\ar@<.2em>[rr]^{p^{\ast}_{1}(u')}\ar@<-.2em>[rr]_{p^{\ast}_{2}(u')}\ar[ur] &&
Y''\ar[ul] & X'\ar[rr]^{u'}\ar[ur] && Y'\ar[ul]
}
\]

In view of our remarks about $\Delta$, and Theorem \ref{chap7-sec7.2.1}, it
is enough to show that
$\Delta^{\ast}(p^{\ast}_{1}(u'))=\Delta^{\ast}(p^{\ast}_{2}(u'))$. But
each of $\Delta^{\ast}(p^{\ast}_{1}(u'))$,
$\Delta^{\ast}(p^{\ast}_{2}(u'))$ is equal to $u'$ considered as
morphisms form $X'$ to $Y'$.

\begin{itemize}
\item[(b)] Given $X'\in(\mathscr{E}t/S')$, to show that $\exists\
  X\in(\mathscr{E}t/S)$ such that
  $X'\xleftarrow{\sim}\fprod{X}{S'}{S}$. 
\end{itemize}

For this, again it is enough to find a descent-datum
$\alpha:p^{\ast}_{1}(X')\xrightarrow{\sim}p^{\ast}_{2}(X')$. We have
the diagram:
\[
\xymatrix@C=.2cm@R=1.1cm{
S & \ar[l]S' && \ar@<-.2em>[ll]_{p_{1}}\ar@<.2em>[ll]^{p_{2}} S'' && S'''\ar@<-1em>[ll]_{p_{31}}\ar@<-.6em>[ll]|{p_{32}}\ar@<-.2em>[ll]^{p_{21}} & \\
 & & p^{\ast}_{1}(X')\ar[ur] & & p^{\ast}_{2}(X')\ar[ul] & & S'\ar[ulll]^{\begin{smallmatrix}\Delta_{1}\\{\text{diagonal}}\end{smallmatrix}}\ar[ul]_{\begin{smallmatrix}\Delta_{2}\\{\text{diagonal}}\end{smallmatrix}}\\
X'\ar[uu] & & & & & & \\
 & & & & & &
\ar[uu]X'=\Delta^{\ast}_{1}p^{\ast}_{1}(X')=\Delta^{\ast}_{1}p^{\ast}_{2}(X')
}
\]
in\pageoriginale view of our remarks about $\Delta_{1}:S'\to S''$ and
Theorem \ref{chap7-sec7.2.1}, the identity morphism
$i:X'=\Delta^{\ast}_{1}p^{\ast}_{1}(X')\to
\Delta^{\ast}_{1}p^{\ast}_{2}(X')=X'$, lifts to an isomorphism
$\alpha:p^{\ast}_{1}(X')\to p^{\ast}_{2}(X')$. The other diagonal
morphism $\Delta_{2}:S'\to S'''=\fprod{S'}{\fprod{S'}{S'}{S}}{S}$ also
imbeds $S'$ as a closed subscheme of $S'''$ having the same
base-space. Then one checks easily that $\alpha$ satisfies the cocycle
condition again by using Theorem \ref{chap7-sec7.2.1}.\hfill Q.E.D.
\end{proof}

Proposition \ref{chap7-prop7.2.2} simply says that if $S'\to S$ is
faithfully flat, quasi-compact and radiciel, $s'\in S'$ and $s\in S$
its image, then $\pi_{1}(S',s')\xrightarrow{\sim}\pi_{1}(S,s)$.

\section{}\label{chap7-sec7.3}
Let $k$ be an algebraically closed field and $X$, $Y$ be connected
$k$-presche\-mes. Suppose $X$ is $k$-proper and $Y$ locally
noetherian. Let $a\in X$, $b\in Y$ be geometric points with values in
an algebraically closed field extension $K$ of $k$. Consider a
geometric point $c=(a,b)\in \fprod{X}{Y}{k}$ over $a$ and $b$. We
claim first that $\fprod{X}{Y}{k}$ is connected. Since $Y$ is
connected and $\fprod{X}{Y}{k}\to Y$ is proper, it is enough to show
that the fibres of $\fprod{X}{Y}{k}\to Y$ are connected and for this
one has only to show that for any field $k'\supset k$

\begin{itemize}
\item[(*)] $\fprod{X}{k'}{k}$ is connected.
\end{itemize}

The question is purely topological and we may assume
$X=X_{\red}$. Looking at the Stein-factorisation $X\to \Spec
\Gamma(X,\mathscr{O}_{X})\to k$ it\pageoriginale follows (by making
use of \ref{chap6-sec6.3.1.1} and the finiteness theorem) that
$\Gamma(X,\mathscr{O}_{X})=k$, since $X$ is connected and $k$
algebraically closed. On the other hand, if $X'=\fprod{X}{k'}{k}$, one
has
$\Gamma(X',\mathscr{O}_{X'})=\foprod{\Gamma(X,\mathscr{O}_{X})}{k'}{k}$
(flat base-change) hence $=k'$; again looking at the
Stein-factorisation we see that $X'$ is connected \ref{chap6-sec6.3.1.1}. 

We can then form $\pi_{1}(\fprod{X}{Y}{k},c)$ and form the product of
the natural maps $\pi_{1}(\fprod{X}{Y}{k},c)\to \pi_{1}(X,a)$, and
$\pi_{1}(\fprod{X}{Y}{k},c)\to \pi_{1}(Y,b)$. We then have the

\begin{prop}\label{chap7-prop7.3.1}
With the assumptions made above
$\pi_{1}(\fprod{X}{Y}{k},c)\xrightarrow{\sim}\pi_{1}(X,a)\times
\pi_{1}(Y,b)$. 
\end{prop}

\begin{proof}
We may assume $X=X_{\red}$.

\medskip
\noindent
{\bf Case~(1).}~{\em Assume $K=k$.}
\smallskip

The morphism $\fprod{X}{Y}{k}\to Y$ is proper and separable and the
fibre over $b\in Y$ is $\simeq$ to $\fprod{X}{k}{k}=X$. We then get
the exact sequence:
$$
\pi_{1}(X,a)\to \pi_{1}(\fprod{X}{Y}{k},c)\to \pi_{1}(Y,b)\to e\quad
\text{(Theorem \ref{chap6-thm6.3.2.1}).} 
$$

But the fibre over $b$, namely, $\fprod{X}{k}{k}=X$ is imbedded in
$\fprod{X}{Y}{k}$ and the composite $X=\fprod{X}{k}{k}\hookrightarrow
\fprod{X}{Y}{k}\xrightarrow{p_{1}}X$ is identity. This means that
$\exists$ continuous sections for the homomorphism $\pi_{1}(X,a)\to
\pi_{1}(\fprod{X}{Y}{k},c)$. Thus, we get:
\begin{itemize}
\item[(i)] $e\to \pi_{1}(X,a)\to \pi_{1}(\fprod{X}{Y}{k},c)\to \pi_{1}(Y,b)\to
e$ 


is exact,

\item[(ii)] and\pageoriginale the sequence splits (also {\em
  topologically}). 
\end{itemize}

\medskip
\noindent
{\bf Case (2).}~{\em $K$ arbitrary.}
\smallskip

The reasoning as in case (1) gives an isomorphism
$$
\pi_{1}(\fprod{X}{\fprod{Y}{K}{k}}{k},c')\xrightarrow{\sim}\pi_{1}(\fprod{X}{K}{k},a')\times\pi_{1}(\fprod{Y}{K}{k},b') 
$$
where $c'$, $a'$, $b'$ are points of $\fprod{(\fprod{X}{Y}{k})}{K}{k},
\fprod{X}{K}{k},\fprod{Y}{K}{k}$ respectively, above $c$, $a$,
$b$. The theorem then is a consequence of the following:
\end{proof}

\begin{prop}\label{chap7-prop7.3.2}
Let $k$ be an algebraically closed field and $X\to k$ a proper
connected $k$-scheme. Let $k'$ be an algebraically closed field
extension of $k$ and let $a'\in \fprod{X}{k'}{k}$ be any geometric
point. If $a\in X$ be the image of $a'$, then
$$
\pi_{1}(\fprod{X}{k}{k},a')\xrightarrow{\sim}\pi_{1}(X,a).
$$
\end{prop}


\begin{proof}
That $\fprod{X}{k'}{k}$ is connected follows from assertion (*) of
\ref{chap7-sec7.3}; the same assertion also proves that if $Z$ is a
connected \'etale covering of $X$ then $Z'=\fprod{Z}{k'}{k}$ is a
connected \'etale covering of $X'=\fprod{X}{k'}{k}$. In other words,
$\pi_{1}(\fprod{X}{k'}{k},a')\to \pi_{1}(X,a)$ is {\em surjective}
(cf. \ref{chap5-sec5.2.1}). We shall prove that it is injective by showing
that every connected \'etale covering $Z'$ of $X'=\fprod{X}{k'}{k}$ is
of the form $\fprod{Z}{k'}{k}$ for some $Z\in (\mathscr{E}t/X)$. 

By Lemma \ref{chap7-lem7.2.1.4} $\exists$ a $k$-algebra $A$ of {\em finite
  type}, $A\subset k'$, and $Z_{A}\in
(\mathscr{E}t/(\fprod{X}{A}{k}))$ such that
$Z'\xleftarrow{\sim}\fprod{Z_{A}}{k}{A}$. 
If\pageoriginale $Y=\Spec A$, $\exists\ y\in Y$ such that $k(y)=k$
[since $A$ is of finite type over (the algebraically closed) $k$]. One
can apply case (1) of Theorem \ref{chap7-prop7.3.1} to the $k$-rational point
$(a,y)\in\fprod{X}{Y}{k}$ to obtain
$$
\pi_{1}(\fprod{X}{Y}{k},(a,y))\xrightarrow{\sim}\pi_{1}(X,a)\times\pi_{1}(Y,y). 
$$

If $Z_{A}$ is defined by an open subgroup $H$ of
$\pi_{1}=\pi_{1}(\fprod{X}{Y}{k},(a,y))$ this means that $\exists$
open (normal) subgroups $G\subset \pi_{1}(X)$ and $G'\subset
\pi_{1}(Y)$ (defining galois coverings $\widetilde{X}/X$,
$\widetilde{Y}/Y$ respectively) such that $H\supset G\times G'$
(i.e. such that $Z_{A}$ is obtained as a quotient of the galois
covering $\fprod{\widetilde{X}}{\widetilde{Y}}{k}$ of $X\times
Y$). Let $\overline{Z}_{A}$ be the lift of the covering
$Z_{A/(\fprod{X}{Y}{k})}$ to $\fprod{X}{\widetilde{Y}}{k}$; we than
have the commutative diagram:
\[
\xymatrix@=.7cm{
 & \fprod{\widetilde{X}}{\widetilde{Y}}{k}\ar[dr]\ar[dl]\ar[d] & \\
\fprod{X}{\widetilde{Y}}{k}\ar[dr] & \overline{Z}_{A}\ar[l]\ar[r] &
Z_{A}\ar[dl]\\
 & \fprod{X}{Y}{k}
}
\]

We\pageoriginale claim that $\overline{Z}_{A}$ is connected.

In fact, let $y\in Y=\Spec A$ be the generic point of $Y$; then
$k(y)=$ field of fractions of $A\subset k$. For any
$\widetilde{y}\in\widetilde{Y}$ lying above $y\in Y$ we thus have
$k(\widetilde{y})\subset k'$ since $k'$ is algebraically closed. If we
then apply the base-change
$\fprod{X}{\fprod{\widetilde{Y}}{k'}{\widetilde{Y}}}{k}\to
  \fprod{X}{\widetilde{Y}}{k}$ to the \'etale covering
  $\overline{Z}_{A}\to \fprod{X}{\widetilde{Y}}{k}$, we obtain the
  \'etale covering
$$
\fprod{Z_{A}}{k'}{Y}=Z'\to X'
$$
which is connected. It follows that $\overline{Z}_{A}$ is
connected. Thus the morphism
$\fprod{\widetilde{X}}{\widetilde{Y}}{k}\to \overline{Z}_{A}$ is
surjective and $\overline{Z}_{A}$ is sandwiched between
$\fprod{\widetilde{X}}{\widetilde{Y}}{k}$ and
$\fprod{X}{\widetilde{Y}}{k}$; this implies that
$\overline{Z}_{A}/(\fprod{X}{\widetilde{Y}}{k})$ is defined by an open subgroup
of $\pi_{1}(\fprod{X}{\widetilde{Y}}{k})=\pi_{1}(X)\times G'$ which
contains $\pi_{1}(\fprod{\widetilde{X}}{\widetilde{Y}}{k})=G\times
G'$;
i.e. $\overline{Z}_{A}=\fprod{\widetilde{X}_{1}}{\widetilde{Y}}{k}$
for an $\widetilde{X}_{1}\in (\mathscr{E}t/X)$. We now obtain
\begin{align*}
(\overline{Z}_{A})_{\widetilde{y}} &= \text{the fibre of }
  \overline{Z}_{A}\text{ over } \widetilde{y}\\[4pt]
&=
  \fprod{\widetilde{X}_{1}}{\fprod{\widetilde{Y}}{k(\widetilde{y})}{\widetilde{Y}}}{k}=\fprod{\widetilde{X}_{1}}{k(\widetilde{y})}{k}\\[4pt] 
\text{and}\quad Z' &=
\fprod{Z_{A}}{k'}{Y}=\fprod{(Z_{A})_{y}}{k'}{k(y)}\\[4pt] 
&=
\fprod{(\overline{Z}_{A})_{\widetilde{y}}}{k}{k(\widetilde{y})}=\fprod{\widetilde{X}_{1}}{k'}{k} 
\end{align*}
\hfill Q.E.D.
\end{proof}



