\thispagestyle{empty}
\begin{center}
{\Large\bf Lectures on}\\[5pt] 
{\Large\bf An Introduction to Grothendieck's Theory of}\\[5pt]
{\Large\bf the Fundamental Group}
\vskip 1cm

{\bf By}

{\large\bf J.P. Murre}
\vfill

{\bf Notes by}

{\large\bf S. Anantharaman}
\vfill

\parbox{0.7\textwidth}{No part of this book may be reproduced
in any form by print, microfilm or any other means without written permission
from the Tata Institute of Fundamental Research, Colaba, Bombay 5}
\vfill

{\bf Tata Institute of Fundamental Research, Bombay}

{\bf 1967}
\end{center}
\eject

\thispagestyle{empty}


\chapter{Preface}


These lectures contain the material presented in a course given at the
Tata Institute during the period December 1964 - February 1965. The
purpose of these lectures was to give an introduction to
Grothendieck's theory of the fundamental group in algebraic geometry
with, as application, the study of the fundamental group of an
algebraic curve over an algebraically closed field of arbitrary
characteristic. All of the material (and much more) can be found in
the ``S\'eminaire de g\'eom\'etrie alg\'ebrique'' of Grothendieck,
1960-1961 Expos\'e V, IX and X.

I thank Mr. S. Anantharaman for the careful preparation of the notes.


\bigskip

\hfill{{\large\bf J.P. Murre}}



\chapter{Prerequisites}


We assume that the reader is somewhat familiar with the notion and
elementary properties of preschemes. To give a rough indication:
Chap. I, \S\ 1-6 and Chap. II, pages 1-14, 100-103 and 110-114 of the
EGA (``\'El\'ements de G\'eometrie Alg\'ebrique'' of Grothendieck and
Dieudon'e). We have even recalled some of these required elementary
properties in Chap. I and II of the notes but this is done very
concisely.

We need also all the fundamental theorems of EGA, Chap. III (first
part); these theorems are stated in the text without proof. We do not
require the reader to be familiar with them; on the contrary, we hope
that the applications which have been made will give some insight into
the meaning of, and stimulate the interest in, these theorems.



