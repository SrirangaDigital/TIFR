\thispagestyle{empty}

\chapter{Introduction}

The main topic of these notes is geodesics. Our aim is twofold. The
first is to give a fairly complete treatment of the foundations of
riemannian geometry through the tangent bundle and the geodesic flow
on it, following the path sketched in \cite{2} and \cite{19}. We
construct the canonical spray of a riemannian manifold $(M,g)$ as the
vector field $G$ on $T(M)$ defined by the equation 
$$
i(G)\cdot d\alpha =-\dfrac{1}{2}dE,
$$
where $i(G)\cdot d\alpha$ denotes the interior product by $G$ of the
exterior derivative $d\alpha$ of the canonical form $\alpha$ on
$(M,g)$ (see \textit{(III.6)}) and $E$ the energy (square of the norm)
on $T(M)$. Then the canonical connection is introduced as the unique
symmetric connection whose associated spray is~$G$.

The second is to give global results for riemannian manifolds which
are subject to geometric conditions of various types; these conditions
involve essentially geodesics.

These global results are contained in Chapters IV, VII and
VIII. Chapter IV contains first the description of the geodesics in a
symmetric compact space of rank one (called here an S.C.-manifold) and
the description of Zoll's surface (a riemannian manifold, homeomorphic
to the two-dimensional sphere, non isometric to it and all of whose
geodesics through every point are closed). Then we sketch results of
Samelson and Bott to the effect that a riemannian manifold all of
whose geodesics are closed has a cohomology ring close to that of an
S.C.-manifold. In Chapter VII are contained the Hopf-Rinow theorem,
the existence of a closed geodesic in a non-zero free homotopy class
of a compact riemannian manifold, and the isometry between two simply
connected and complete riemannian manifolds of the same constant
sectional curvature.

In Chapter VIII one will find theorems of Myers and Synge, the
Gauss-Bonnet formula and a result on complete riemannian manifolds of
non-positive curvature. Then come the theorem of L.W. Green, which
asserts that, on the two-dimensional real-projective space, a
riemannian structure all of whose geodesics are closed has to be
isometric to the standard one, and the theorem of $E$. Hopf: on the
two-dimensional torus, a riemannian structure all of whose geodesics
are without conjugate points has to be a flat one. As a counterpoint
we have quoted the work of Busemann which shows that the theorems of
Green and Hopf pertain to the realm of riemannian geometry, for they
no longer hold good in $G$-spaces (see {\em (VIII.10)}). However, the
result on complete manifolds with non-positive curvature is still
valid in $G$-spaces.

We have included in Chapter VIII theorems of Loewner and Pu, which are
``isoperimetric'' inequalities on the two dimensional torus and the
two-dimensional real projective space (equality implying isometry with
the standard riemannian structures on these manifolds). These results
do not involve geodesics explicitly, but have been included for their
great geometric interest. One should also note that the results of
Green, Hopf, Loewner and Pu are two-dimensional and so lead to
interesting problems in higher dimensions.

The tools needed for these results are developed in various chapters:
Jacobi fields, sectional curvature, the second variation formula play
an important role; see also the formulas in {\em (VIII.4)} and {\em
  (VIII.8)}.

The reference for calculus is \cite{36}; references for differential
and riemannian geometry are \cite{14}; \cite{16}, \cite{17},
\cite{18}, \cite{19}, \cite{21}, \cite{33}, \cite{35}, and lectures
notes of I.M.Singer and a seminar held at Strasbourg University. We
have used some of these references without detailed acknowledgement.

I am greatly indebted to N. Bourbaki, P. Cartier and J.L. Koszul for
communication and permission to use ideas and results of theirs on
connections. For most valuable help I am glad to thank here
P. Cartier, J.L. Koszul, N. Kuiper and D. Lehmann.

