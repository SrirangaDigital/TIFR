\chapter{Lecture}\label{part4:lec46} %%% 46
\markboth{\thechapter. Lecture}{\thechapter. Lecture}

All\pageoriginale the other sums that we have to estimate behave some
what similarly. We take as specimen $S_{20}$.
\begin{align*}
  S_{20} & = \frac{1}{D^{1/2}2^{r/2}} \mathop{\textstyle{\sum'}}_{o
    \leq h < k\leq N} e^{-2 \pi i \frac{h}{k}n} T_k (h, \underline{o})
  \times \frac{1}{k^r} \times \sum^{N-1}_{\ell=k_2}
  \int\limits^{\frac{1}{k(k+\ell)}}_{\frac{1}{k(k+\ell+1)}} \frac{e^{2
    \pi n \mathfrak{z}}}{\mathfrak{z}^{r/2}} d \varphi\\
  & = \frac{1}{D^{1/2} 2^{r/2}} \sum^N_{k=1} \frac{1}{k^r}
  \sum^{N-1}_{\ell=N-k+1} \int\limits^{\frac{1}{k(k+\ell)}}_{\frac{1}{k(k+
      \ell+1)}} \frac{e^{2 \pi n \mathfrak{z}}}{\mathfrak{z}^{r/2}}
  \sum_{\substack{h \mod k\\N-k < k_2 \leq \ell}} e^{-2 \pi i
    \frac{h}{k}n} T_k (h, \underline{o}) d \varphi
\end{align*}

The original interval for $k_2$ was bigger: $N - k < k_2 \leq N$. Now
the full interval is not permissible, i.e., we have admitted not all
residues modulo $k$, but only a part of these, and the $N$ may lie in
two adjacent classes of residues.

Here we have a new type of sum of interest. We know how to discuss
$T_k$; $h$ plays a role there. The sums we have now get are
$$
\sum_{\lambda \mod \wedge} T_k (\lambda, \underline{o})
\mathop{\textstyle{\sum'}}_{\substack{h \equiv \lambda
    \pmod{\wedge}\\N-k < k_2 \leq \ell}} e^{-2 \pi i \frac{h}{k}n}
$$

The\pageoriginale inner sum is an incomplete Ramanujan sum, with
restriction on $k_2$ implying (see lecture \ref{part4:lec45}) actually a restriction
on $h!$ The Kloosterman sums are a little more general:
$$
\sum_{h \bar{h} \equiv 1 \pmod{k}} e^{2 \pi \frac{i}{k} (uh +
    \mathscr{V}\bar{h})} 
$$

Our present sums are incomplete Kloosterman sums (with $\mathscr{V}=o$
and $u=1$), and the interesting fact is that they also permit the same
appraisal, viz.
$$
O \left(k^{r/2} k^{1- \alpha + \epsilon} (k, n)^\alpha \right)
$$

From there on things go just as smoothly as before.
$$
S_{2\underline{o}} =O \left(\sum^{\sqrt{n}}_{k=1} k^{- r/2} k^{1-
\alpha + \epsilon} (k, n)^\alpha
\int\limits^{\frac{1}{k(N+1)}}_{\frac{1}{k(k+N)}} \frac{d
  \varphi}{(\delta^2_N + \varphi^2)^{r/4}}\right)
$$
and here for convergence of the integral we want $r \geq 3$. This
would give again the old order. Similar estimates hold for the other
pieces:
$$
\sum_{\underline{m} \neq \underline{o}} S_{2 \underline{m}} = O
\left(n^{r/4} - \frac{\alpha}{2}+ \epsilon \right)
$$

(The incomplete Kloosterman sums here are actually incomplete
Ramanujan sums and so we may got a slightly better estimate; but this
is of no conse\-quence\pageoriginale as the other terms have a higher order).

We then have
$$
A_r (n) = S_{oo} + O \left(n^{r/4 - \alpha/2 - \epsilon} \right),
\alpha = \frac{1}{2}.
$$

Let us look at $S_{oo}$. It is classical, but not quite what we like
it to be.
$$
S_{oo} = \frac{1}{D^{1/2} 2^{r/2}} \sideset{}{'}\sum_{o \leq
  h < k \leq N} e^{- 2 \pi i \frac{h}{k} n} \frac{T_k (h,
  \underline{o})}{k^r}
\int\limits^{\frac{1}{k(k+N)}}_{\frac{1}{k(k+N)}} \frac{e^{2 \pi n
    \mathfrak{z}}}{\mathfrak{z}^{r/2}} d \varphi + O
\left(n^{r/4-\alpha/2 + \epsilon} \right) 
$$

Replace the integral by an infinite integral:
$$
\displaylines{\frac{1}{D^{1/2} 2^{r/2}} \sum^{\sqrt{n}}_{k=1}
  \frac{H_k (n)}{k^r} \int\limits^{\infty}_{- \infty} \frac{2 \pi n
    \mathfrak{z}}{\mathfrak{z}^{r/2}} d \varphi + O
  \left(n^{r/4- \alpha/2 + \epsilon} \right),\cr
  \text{with} \hfill H_k (n) = \sum_h e^{- 2 \pi i \frac{h}{k} n T_k (h,
    \underline{o})}\phantom{WWWWWWWi}\hfill \cr
  = O \left(k^{r/2} k^{1 - \alpha + \epsilon} (k, n)^\alpha \right),}
$$
thereby  adding an error term of order 
$$
O \left(\sum^{\sqrt{n}}_{k=1} k^{- \frac{r}{2} + 1 - \alpha +
  \epsilon}  (k, n)^\alpha \int\limits^\infty_{\frac{1}{kN}} \frac{d
  \varphi}{(\delta_N^2 + \varphi^2)^{r/4}}\right)
$$

Now
\begin{align*}
  \int\limits^\infty_{\frac{1}{kN}} \frac{d \varphi}{(\delta_N^2 +
    \varphi^2)^{r/4}} & = \int\limits^\infty_{\frac{1}{kN}} \frac{d
    \varphi}{\left( 1+ \left(\frac{\varphi}{\delta_N}
    \right)^2\right)^{r/4}} \delta_N ^{1- {r/2}}\\
  & = O \left(n^{\frac{r}{2}-1} \int\limits^\infty_{\frac{N}{k}}
  \frac{d \psi}{(1+ \varphi^2)^{r/4}} \right)
\end{align*}
with\pageoriginale $\psi = N^2 \varphi$. $\psi$ is never smaller than 1 as
$\frac{N}{k} >1$. So we can drop 1 in the denominator without
committing any error in the order of magnitude. So this gives
$$
O \left(n^{\frac{r}{2}-1} \int^\infty_{\frac{N}{k}} \frac{d
  \psi}{\psi^{r/2}} \right)
$$
and the integral converging for $r \geq 3$, it is equal to 
$$
O \left( n^{\frac{r}{2}-1} \left(\frac{\sqrt{n}}{k}
\right)^{-\frac{r}{2}+1} \right) = O \left(n^{\frac{r}{4} -
  \frac{1}{2}} k^{\frac{r}{2} -1} \right)
$$

Hence our new error term is 
$$
O \left(\sum^{\sqrt{n}}_{k=1} k^{-\alpha + \epsilon} (n, k)^\alpha
n^{\frac{r}{4} - \frac{1}{2}}\right) = O \left(n^{r/4 -
  \alpha/2 + \epsilon} \right)
$$
which is what has already appeared.

We than have on writing $2 \pi n \mathfrak{z} = \omega$,
$$
A_r (n) = \frac{1}{D^{1/2} 2^{r/2}} \sum^{\sqrt{n}}_{k=1}
\frac{H_k(n)}{k^r} \frac{1}{i} (2 \pi n)^{\frac{r}{2}-1}
\int\limits^{c+ i \infty}_{c- i \infty} \frac{e^\omega}{\omega^{r/2}}
d \omega + O \left(n^{r/4- \alpha/2 + \epsilon} \right),
$$\pageoriginale
and the integral being the Hankel integral for the gamma-function,
\begin{align*}
  A_r (n) & = \frac{(2 \pi)^{r/2} n^{r/2}-1}{D^{1/2} 2^{r/2}}
  \sum^{\sqrt{n}}_{k=1} \frac{H_k (n)}{k^r} \frac{1}{\Gamma(^{r/2})} +
  O \left( n^{r/4 - \alpha/2 + \epsilon}\right)\\
  & = \frac{\pi^{r/2}}{\Gamma\left( \frac{r}{2}\right)D^{1/2}}
  n^{r/2-1} \sum^\infty_{k=1} \frac{H_k (n)}{k^r}+ O \left(
  n^{r/4- \alpha /2 + \epsilon}\right)\\ 
  & \hspace{3cm}+ O \left(n^{\frac{r}{2}-1}
  \sum^\infty_{k= \sqrt{n}+1}  \frac{k^{\frac{r}{2} +1 - \alpha + \epsilon}
    (k, n)^\alpha}{k^r}\right)
\end{align*}

This new error term is 
$$
O \left( n^{\frac{r}{2}-1} \sum_{d\mid n} d^\alpha \sum_{q>
  \frac{\sqrt{n}}{d}} (qd)^{1- \frac{r}{2} - \alpha + \epsilon} \right) =
O \left(n^{\frac{r}{2}-1} \sum_{d\mid n} d^{1+ \epsilon - r/2} \sum_{q>
    \frac{\sqrt{n}}{d}} q^{- r/2}  \right)
$$

(This is because for the Ramanujan sum we have)
\begin{align*}
  \sideset{}{'}\sum_{h \mod k} e^{-2 \pi i \frac{h}{k}n} & =
  \sum_{d|(k, n)} d \mu \left(\frac{k}{d} \right)\\
  & = O \left((k, m) \sum_{d| (k, n)} 1\right)= O \left(
  (k, n)^{1+\epsilon}\right);
\end{align*}
and\pageoriginale then we use the old appraisal ($k^{1-\alpha + \epsilon}(k, n)^\alpha$
with $\alpha= 1+\epsilon$). So we have
$$
O \left( n^{\frac{r}{2}-1} \sum_{d|n} d^{1+ \epsilon-r/2}
  \left( \frac{\sqrt{n}}{d}\right)^{- r/2+1}\right) =
O \left(n^{\frac{r}{4}-\frac{1}{2}} \sum_{d|n} d^\epsilon\right) =
O \left( n^{r/4- 1/2+ 2 \epsilon}\right)
$$

This is of smaller order than the old error term. So we have our final
result:
$$
A_r (n) = \frac{\pi^{r/2}}{\Gamma (r/2) D^{1/2}} n^{r/2-1}
\sum^\infty_{k=1} \frac{H_k (n)}{k^r} + O \left(n^{r/4 -
  \alpha/2 + \epsilon} \right);
$$
the singular series plus the error term. 

What remains to be shown is that the singular series again enjoys the
multiplicative property: 
$$
H_{k_1 k_2} (n) = H_{k_1} (n) H_{k_2} (n)
$$

We shall then have it as the product
$$
\displaylines{\prod_p \gamma_p\cr
  \text{where} \hfill \gamma_p =1 + \frac{H_k (n)}{p^r} + \frac{H_{p^2}
    (n)}{p^{2r}} + \cdots \hfill }
$$

The\pageoriginale arithmetical interpretation now  becomes difficult,
because all the properties that the quadratic form may have will have
to show up. One or other of the factors $\gamma_p$ may be zero in
which case we have no representation.

We should like to throw some light on the Kloosterman sums. We take
for granted the estimate
$$
\mathop{\textstyle{\sum'}}_{\substack{h \mod k\\ h \bar{h} \equiv 1
    \pmod{k}}} e^{2 \pi \frac{i}{k} (u h + \mathscr{V}\bar{h})} =
O \left( k^{1- \alpha + \epsilon}\cdot (k, u)^\alpha \right)
$$
Kloosterman and Esterman (Hamburger Abhandlungen Vol.7) proved
$\alpha= \frac{1}{4}$; Salie' (Math. Zeit., vol. 36) proved $\alpha =
\frac{1}{3}$. Using the multiplicativity, in a certain sense, of the
sums, Salie' could prove that if $k= p^\beta$, $p$ prime and $\beta
\geq 2$, then $\alpha= \frac{1}{2}$ but he could not prove this in the
other cases. The difficult case was that of
$$
\mathop{\textstyle{\sum'}}_{h \mod p} e^{2 \pi i/p (u h + \mathscr{V}
  \bar{h})}.
$$

For this nothing better than $O \left(p^{2/3 + \epsilon} (p,
u)^{1/3} \right)$ could be obtained; and it defied all efforts until
A.Weil proved $\alpha = 1/2$ in all cases by using deep methods
(Proc. Nat. Acad. Sc.1948). Further application of the Kloosterman
sums offer no difficulty.

The (generalised) Kloosterman sums are symmetrical in $u$ and
$\mathscr{V}$, for
$$
\sideset{}{'}\sum_{\substack{h \equiv \lambda (\wedge)\\h
    \mod k}} e^{\frac{2 \pi i}{k} (uh + \mathscr{V}\bar{h})}=
\sideset{}{'}\sum_{\substack{\bar{h} \equiv \bar{\lambda}
    (\wedge)\\h \mod k}} e^{\frac{2 \pi i}{k} (u \bar{h} + \mathscr{V}h)}
$$ 
since\pageoriginale $(\lambda, \wedge)=1$, $h \equiv \lambda
\pmod{\wedge}$ and $h \bar{h} \equiv 1 \pmod{\wedge}$ imply $\bar{h}
\equiv \bar{\lambda} \pmod{\wedge}$ and $\lambda \bar{\lambda} \equiv
1 \pmod{\wedge}$. The last we can write as
$$
\mathop{\textstyle{\sum'}}_{h \mod k} g(\bar{h}) e^{\frac{2 \pi i}{k}
  (u h + \mathscr{V} \bar{h})},
$$
where $g(m)$ is the periodic function defined as 
$$
g(m)=
\begin{cases}
  1 & \text{if}~ m \equiv \bar{\lambda} \pmod{\wedge}\\
  0 & \text{otherwise}.
\end{cases}
$$
$g(m)$ has therefore the finite Fourier expansion
$$
g(m) = \sum^\wedge_{j=1} C_j e^{2 \pi i j \frac{m}{\wedge}}
$$

The coefficients $c_j$ can be calculated in the usual way:
$$
c_q = \frac{1}{\wedge} e^{\frac{- 2 \pi i q \bar{\lambda}}{\wedge}}m
q=1, 2, \ldots , \wedge
$$

Substituting for $C_q$, the sum becomes
{\fontsize{9}{11}\selectfont
$$
\sum_{j \mod \wedge} C_j \mathop{\textstyle{\sum'}}_{h \mod k} e^{2
  \pi i j \frac{\bar{h}}{\wedge}} e^{2 \pi \frac{i}{k}  (u h +
  \mathscr{V}\bar{h})} = \frac{1}{\wedge} \sum_{j \mod \wedge} e^{- 2
  \pi i j \frac{\bar{\lambda}}{\wedge}} \sum_{h \mod k} e^{2 \pi
  \frac{i}{k} \left(u h + (\mathscr{V} + \frac{jk}{\wedge}) \bar{h}\right)}
$$}\relax
so that the generalised sum becomes a finite combination of
undisturbed\break Kloosterman sums and so has the estimate $O \left(
k^{1- \alpha + \epsilon} (k, u)^\alpha\right)$

This\pageoriginale works just as well in the other case when there is
an inequality on $\bar{h}$.
$$
\displaylines{\sum_{\substack{h \equiv \lambda \pmod{\wedge}, h \mod k\\a \leq
    \bar{h} \leq b}} e^{\frac{2 \pi i}{k} (u h + \mathscr{V}
    \bar{h})}= \sum_{\substack{h \equiv \lambda \pmod{\wedge}\\h \mod
      k}} f(\bar{h}) e^{\frac{2 \pi i}{k}(uh + \mathscr{V} \bar{h})
  }\cr
  \text{where}\hfill f(m) =
  \begin{cases}
    1, & 0 < m\leq a,\\
    0, & a < m\leq k, 
  \end{cases}\hfill }
$$
and $f(m)$ is periodic modulo $k$.

Then
$$
f(m) = \sum^k_{j=1} c_j e^{2 \pi i j \frac{m}{k}}
$$
where
\begin{align*}
  c_j & = \frac{1}{k} \frac{e^{- \frac{2 \pi i j}{k}}- e^{- 2 \pi i j
      (a+1)/k}}{1- e^{-2 \pi i j /k}}, j \neq k,\\
  c_k & = \frac{a}{k}\\
  |c_j| & \leq \frac{2}{k \sin \pi j/k}
\end{align*}

The\pageoriginale sum becomes
\begin{multline*}
  \sum^{k-1}_{j=1} c_j \mathop{\textstyle{\sum'}}_{\substack{h \mod
      k\\ h \equiv \lambda \pmod{\wedge}}} e^{\frac{2 \pi i}{k}(uh +
    (\mathscr{V}+ j) \bar{h})} + c_k
  \mathop{\textstyle{\sum'}}_{\substack{h \mod k\\h \equiv \lambda
      \pmod{\wedge}}} e^{\frac{2 \pi i}{k} (uh + \mathscr{V}
    \bar{h})}\\
  = O \left( k^{1- \alpha + \epsilon} (h, k)^\alpha\right) \left\{1+
  \frac{1}{k} \sum^{k-1}_{j=1} \frac{1}{| \sin \frac{\pi j}{k}|}\right\}
\end{multline*}

Since $\sin \alpha \geq \frac{2}{\pi}$,
\begin{align*}
  2 \sum^{\frac{k-1}{j}}_{j=1} \frac{1}{\sin \frac{\pi j}{k}} & \leq 2
  \frac{\pi}{2} \sum \frac{1}{\frac{\pi j}{k}}\\
  & = k \sum_{j \leq \frac{k-1}{2}} \frac{1}{j} = O (k \log k)
\end{align*}
so that again the sum becomes
$$
O \left(k^{1- \alpha + \epsilon} (k, u)^\alpha \right)
$$

Kloosterman first discussed his method for a diagonal quadratic
form. Later on he applied it to modular forms and for this he could
derive on the investigations by Heeke comparing modular forms with
Eisenstein series. In this case the theory becomes simpler: we can
subtract suitable Eisenstein series and the principle term then
becomes zero. The $r$-fold theta-series that we had are in fact
modular forms.
