\chapter{Lecture}\label{part4:lec38} %% 38
\markboth{\thechapter. Lecture}{\thechapter. Lecture}

It\pageoriginale would be of interest to study $\gamma_2 (n)$ also
for $r=3, 4$.
$$
\gamma_2 (n) = 1+ V_2 (n) + V_{2^2}(n)+ \cdots 
$$

First consider the case $r=3, 2/n$. Then
$$
V_2 (n) =0.
$$

For $V_{2^\lambda} (n)$, $\lambda > 1$, we have to make a distinction
between $\lambda $ even and $\lambda$ odd.

\textit{$\lambda$ even}.

$$
V_{2^\lambda} (n) = 
\begin{cases}
  \quad 0, & \text{if}~ 2^{\lambda -2}\nmid n;\\
  \frac{\cos \frac{\pi}{4} (2 \nu - r)}{2^{(\lambda -1)(\frac{r}{2}
      -1)}}, & \text{if}~ 2^{\lambda -2}\nmid n, n= 2^{\lambda -2}.\nu.
\end{cases}
$$

\textit{$\lambda$ odd}.

$$
V_{2^\lambda} (n) = 
\begin{cases}
  \quad 0, & \text{if}~ 2^{\lambda -3}\nmid n;\\
  \quad 0, & \text{if}~ 2^{\lambda -3}\mid n, n= 2^{\lambda -3}\nu, \nu- r
  \nequiv 0 \pmod{4}\\
  \frac{(-)^{\frac{\nu-r}{4}}}{2^{(\lambda -1)(\frac{r}{2}
      -1)}}, & \text{if}~ 2^{\lambda -3}\mid n, n= 2^{\lambda -3}\nu, \nu
    - r\equiv 0 \pmod{4}
\end{cases}
$$

So for $r=3$,
\begin{align*}
  \gamma_2 (n) & = 1+ V_4 (n) + V_8 (n)\\
  & = 1+ \frac{\cos \frac{\pi}{4} (2n-3)}{\sqrt{2}} +
  \frac{(-)^{\frac{n-3}{4}}}{2}, 
\end{align*}
where\pageoriginale the last summand has to be replaced by 0 if
$(n-3)/4$ is not an integer. Since $2n -3$ is odd, we have
$$
|\cos \frac{\pi}{4} (2n-3)| = \frac{1}{\sqrt{2}},
$$
and thus clearly,
$$
|\gamma_2 (n)| \leq 1+ \frac{1}{2} + \frac{1}{2} = 2
$$

Moreover, $\gamma_2(n)$ can vanish. This would require
$$
\displaylines{(-)^{\frac{n-3}{4}}=1\cr
  \text{and} \hfill \cos \frac{\pi}{4} (2n -3) =- \frac{1}{\sqrt{2}}
  \hfill }
$$
simultaneously. But this is the case for 
$$
n \equiv 7 \pmod{8},
$$
as is easily seen. This corresponds to the fact that a number $n$, $n
\equiv 7 \pmod{8}$ cannot be represented as the sum of three squares.

Next take $r=4$. We distinguish between the cases $2\nmid n$ and
$2\mid n$.

\begin{enumerate}
\item $2\nmid n$. Then from relations (\ref{part4:lec37:eq*}) and
  (\ref{part4:lec37:eq**}) proved in lecture 37, we have
  \begin{align*}
    \gamma_2 (n) & = 1+ V_4 (n) + V_8 (n)\\
    & = 1 + \frac{\cos \frac{\pi}{4} (2n -4)}{2} =1 - \frac{1}{2} \cos
    \frac{\pi n}{2}\\
    & = 1
  \end{align*}
\item $2\mid n$\pageoriginale Let $n= 2^\alpha n'$, $2\mid n'$. Then
  (\ref{part4:lec37:eq*}) and (\ref{part4:lec37:eq**}) show that 
  $V_{2^\lambda} (n) =0$ for $\lambda > \alpha+3$. But actually $V_{2^
  \lambda} (n) =0$ also for $\lambda = \alpha+3$. Indeed, for $\alpha$
  odd, $\lambda = \alpha +1$ is the last even, $\lambda = \alpha+2$
  the last odd index for non-vanishing $V_{2^\lambda} (n)$. For $\alpha$
  even, $\lambda = \alpha +2$ is the last even index: $\lambda =
  \alpha +3$ is odd and since $4\nmid (n'-4)$, we have also $V_{2^\lambda} (n)
  =0$ for $\lambda= \alpha +3$.
\end{enumerate}
$$
\therefore \qquad \gamma_2 (2^\alpha n')= 1+ \sum^{\alpha+2}_{\lambda
=2} V_{2^\lambda} (n)
$$

Now, in $V_{2^\lambda} (n)$, for $\lambda$ even, 
\begin{align*}
  \cos \frac{\pi}{4} (2 \nu - r) & = - \cos \frac{\pi}{2} n' 2^{\alpha
  - \lambda +2}\\
  & = - \cos \pi n' 2^{\alpha - \lambda +1}\\
  & = 
  \begin{cases}
    -1, & \text{for}~ \lambda \leq \alpha,\\
    1, &  \text{for}~ \lambda = \alpha +1,\\
    0, & \text{for}~ \lambda = \alpha +2.
  \end{cases}
\end{align*}

Similarly in $V_{2^\lambda} (n)$, for $\lambda$ odd, 
\begin{align*}
(-)^{\frac{\nu-4}{4}} & = - (-)^{n^1. 2^{\alpha - \lambda +1}}\\
  & =
  \begin{cases}
    -1, & \text{for}~ \lambda \leq \alpha;\\
    1, & \text{for}~ \lambda = \alpha +1,
  \end{cases}
\end{align*}
and\pageoriginale $V_{2^\lambda} (n) =0$ for $\lambda = \alpha +2$
since then $4\nmid 2^{\alpha- \lambda +1}$. The numerators of the
non-vanishing $V_{2^\lambda} (n)$ are $-1$ upto the last one, which is
1. And thus
\begin{align*}
  \gamma_2 (2^\alpha n')& = 1 - \frac{1}{2} - \frac{1}{2^2} - \cdots -
  \frac{1}{2^{\alpha-1}} + \frac{1}{2^\alpha}\\
    & = \frac{1}{2^{\alpha-1}} + \frac{1}{2^\alpha} = \frac{3}{2^\alpha}
\end{align*}

Although here $\gamma_2 (2^\alpha n')> 0$, we see that for $\alpha$
sufficiently large $\gamma_2 (n)$ can come arbitrarily close to 0.

We now consider $\gamma_p (n)$ for $p\geq 3$.
$$
\displaylines{
  \gamma_p (n) = 1 + V_p (n) + V_{p^2} (n) + \cdots,\cr
  \text{where} \hfill V_{p^\lambda} (n) = \frac{1}{p^{\lambda r}}
  \sum_{\substack{h \mod p^\lambda\\p\nmid  h}} G(h, p^\lambda)^r
  e^{-2 \pi i \frac{h}{p^\lambda} n}\hfill }  
$$

Now 
\begin{align*}
  G(h, p^\lambda) & = \left(\frac{h}{p^\lambda} \right) G(1,
  p^\lambda)\\
  & = \left(\frac{h}{p} \right)^\lambda i^{\left(\frac{p^\lambda-1}{2}
    \right)^2} p^{\frac{\lambda}{2}}\\
  \therefore \quad V_{p^\lambda} (n) & =
  \frac{i^{r\left(\frac{p^\lambda -1}{2} \right)^2}}{p^{\lambda r/2}}
  \sum_{\substack{h \mod p^\lambda\\ p \nmid h}}
  \left(\frac{h}{p} \right) e^{- 2 \pi i\frac{h}{p^\lambda}n}
\end{align*}

We\pageoriginale have to distinguish between $\lambda r$ odd; and
$\lambda r$ even

1) $\lambda r$ \textit{even}. If $p^\lambda \equiv 1 \pmod{4}$, then
$$
(-)^{\frac{r}{2} \left(\frac{p^\lambda-1}{2} \right)^2}=
(-)^{\frac{r}{2} \frac{p^\lambda -1}{2}}
$$

So
$$
V_{p^\lambda} (n) = \frac{i^{r \left( \frac{p^\lambda
      -1}{2}\right)^2}}{p^{\lambda r/2}} \sum_{\substack{h \mod
    p^\lambda\\p \nmid h}} e^{- 2 \pi i \frac{h}{p^\lambda}n}
$$

2) $\lambda r$ \textit{odd}. In this case
$$
V_{p^\lambda} (n) = \frac{i^{r \left( \frac{p^\lambda
      -1}{2}\right)^2}}{p^{\lambda r/2}} \sum_{\substack{h \mod
    p^\lambda\\p \nmid h}} \left(\frac{h}{p}\right) e^{- 2 \pi i
  \frac{h}{p^\lambda}n} 
$$

The inner sum here is a special case of the so-called Ramanujan sums:
$$
C_k (n) = \sum_{\substack{h \mod k\\ (h, k)=1}} e^{2 \pi i \frac{h}{k}
n} 
$$

There sums can be evaluated. Look at the simpler sums
\begin{align*}
  S_k (n) & = \sum_{ \lambda \mod k} e^{2 \pi i \frac{\lambda}{k}n}\\
  & = 
  \begin{cases}
    k, & \text{if}~ k\mid n;\\
    0, & \text{if}~ k \nmid  n.
  \end{cases}
\end{align*}

Classify\pageoriginale the $\lambda's$ in $S_k(n)$ according to their
common divisor with $k$. Then
\begin{align*}
  S_k (n) & = \sum_{d\mid k} \sum_{\substack{\lambda \mod k\\(\lambda ,
      k)=d}} e^{2 \pi i\frac{\lambda}{k}n}\\
  & = \mathop{\sum_{d\mid k}\sum_{\lambda \mod
      k}}_{\left(\frac{\lambda}{k}, \frac{k}{d}\right)=1} e^{2 \pi i
    \frac{\lambda}{d} \cdot \frac{n}{k/d}}\\
  & = \mathop{\sum_{d\mid k}\sum_{\mu \mod
      \frac{k}{d}}}_{\left(\mu, \frac{k}{d} \right)=1} e^{2 \pi i
    \frac{\mu n}{k/d}}\\
  & = \sum_{d\mid k} C_{\frac{k}{d}} (n)\\
  & = \sum_{d\mid k} C_d (n).
\end{align*}

Now by M\"obious inversion formula,
$$
C_k (n) = \sum_{d\mid k} \mu \left(\frac{k}{d}\right) S_d (n),
$$
and $S_d(n)$ is completely known- it is either 0 or $d$; hence
\begin{align*}
  C_k (n) & = \sum_{d\mid k, d\mid n} d \mu \left(\frac{k}{d}\right)\\
  & = \sum_{d\mid (n, k)} d \mu \left(\frac{k}{d}\right).
\end{align*}

So\pageoriginale these are integers.

The M\"obious function which appears here arises as a coefficient in a
certain Dirichlet series; in fact
$$
\frac{1}{\zeta (s)} = \sum^\infty_{n=1} \frac{\mu (n)}{n^s}
$$

It is possible to build up a complete formal theory of Dirichlet
series as we had in the case of power series. Formal Dirichlet series
form a ring without null-divisors. The multiplication law is given by
$$
\displaylines{\sum \frac{a_n}{n^2} \sum \frac{b_n}{n^2} = \sum
  \frac{c_n}{n^s}\cr
  \text{where} \hfill c_n = \sum_{kj=n} a_k b_j\hfill }
$$

The relation 
$$
\displaylines{\sum \frac{\mu(n)}{n^s} \sum \frac{1}{n^s} =1 \cr
\text{then implies that} \hfill 0 = \sum_{jk=n} \mu (j)\cdot 1=
\sum_{d\mid n} \mu(d), n>1.\hfill }
$$
