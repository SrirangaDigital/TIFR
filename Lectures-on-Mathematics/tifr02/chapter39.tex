\chapter{Lecture}\label{part4:lec39} %%% 39
\markboth{\thechapter. Lecture}{\thechapter. Lecture}

For\pageoriginale $p \geq 3$ we had 
$$
\gamma_p (n)= 1+ V_p (n) + V_{p^2} (n) + \cdots
$$ 
where
\begin{align*}
  V_{p^\lambda} (n) & = \frac{1}{p^{\lambda r}} \sum_{\substack{h \mod
  p^\lambda\\(h, p)=1}} G(h, p^\lambda)^r e^{-2 \pi
    \frac{h}{p^\lambda}n}\\
  & = \frac{i^{\left( \frac{p^\lambda -1}{2}\right)^{2r}}}{p^{\lambda
      r/2}} 
  \begin{cases}
    \sum\limits_{\substack{h \mod p^\lambda\\p \nmid h}} e^{- 2 \pi i
      \frac{h}{p^\lambda}n}, & \lambda r ~\text{even};\\
    \sum\limits_{\substack{h \mod p^\lambda\\p \nmid h}}
    \left(\frac{h}{p}\right) e^{-2 \pi i \frac{h}{p^\lambda}n}, &
    \lambda r ~\text{odd}.
  \end{cases}
\end{align*}

For $\lambda r$ odd we have to evaluate this directly. If $\lambda r$
is even it is simpler; it is a special case of the Ramanujan sums:
$$
C_k (n) = \sum_{\substack{h \mod k\\(h, k)=1}} e^{2 \pi i \frac{h}{k}n}
$$
which could be evaluated by means of the M\"obious inversion formula:
$$
C_k (n) = \sum_{d\mid (k, n)} d \mu \left(\frac{k}{d}\right)
$$

So\pageoriginale if $k$ is a prime power, $k= p^\lambda$,
\begin{align*}
  \sum_{\substack{h \mod p^\lambda\\ p \nmid h}} e^{- 2 \pi i
    \frac{h}{p^\lambda} n}& = \sum_{d\mid(p^\lambda, n)} d \mu
  \left(\frac{p^\lambda}{d}\right) \\
  & = 
  \begin{cases}
    0 , & \text{if}~ \alpha < \lambda -1, n= p^\alpha n', p \nmid
    n';\\
    -1 \times p^{\lambda -1} = - - p^\alpha. & \text{if}~ \alpha =
    \lambda -1;\\
    -1 \times p^{\lambda -1} + p^\lambda\\
    = p^\lambda (1- \frac{1}{p}), & \text{if}~ \alpha \geq \lambda.
  \end{cases}
\end{align*}

For obtaining these values we observe that in the summation on the
right side we have to take into account only such divisors $d$ that
$\frac{p^\lambda}{\alpha}$ is at most $p$. This leads in the first
case $\alpha < \lambda -1$ to a vacuous sum. In the second case the
only admissible divisor is $p^{\lambda-1}$; in the last we have two
divisors $p^{\lambda -1}$ and $p^\lambda$. Thus
$$
V_{p^\lambda} (n)=0
$$ 
for $\lambda > \alpha +1$; we get again a finite sum for $\gamma_p
(n)$

We now take $\lambda r$ odd. We want
$$
\sum_{\substack{h \mod p^\lambda\\ p \nmid h}}
\left(\frac{h}{p}\right) e^{-2 \pi i \frac{h}{p^\lambda} n}
$$
$h$ modulo $p$ is periodic, and we emphasize this by writing
$$
h= r p+ s; s=1, 2, \ldots, p-1; r=1, \ldots, p^{\lambda-1}
$$\pageoriginale

So the above sum becomes
$$
\sum^{p^\lambda -1}_{r=1} \sum^{p-1}_{s=1} \left(\frac{s}{p}\right)
e^{- 2 \pi i \frac{(rp+s)}{p^\lambda}} = \sum^{p-1}_{s=1}
\left(\frac{s}{p}\right) e^{- 2\pi i \frac{s}{p^\lambda}}
\sum^{p^\lambda -1}_{r=1} e^{- 2 \pi i \frac{r}{p^\lambda -1}n}
$$

This is zero when $p^{\lambda-1}\nmid n$ (because the inner sum
vanishes). Otherwise, let $n= p^{\lambda -1}\nu$ and $p \nmid \nu$;
then it is again zero because we have only a sum of quadratic residue
symbols (since the character is not the principal character). If
$p\mid \nu$, the sum becomes
$$
p^{\lambda -1} \overline{G(\nu, p)}= p^{\lambda -1}
\left(\frac{\nu}{p}\right) i^{\left(\frac{p-1}{2}\right)^2} \sqrt{p}
$$

So if $n= p^\alpha \cdot n'$ where $p \nmid n'$, then
$$
V_{p^\lambda} (n)=
\begin{cases}
  \qquad 0 , & \text{if}~ \lambda -1 > \alpha;\\
  p^\alpha \left(\frac{n'}{p}\right) i^{\left(\frac{p-1}{2}\right)^2} \sqrt{p},
  & \text{if}~ \lambda -1 = \alpha;\\
  \qquad 0, & \text{if}~ 0 \leq \lambda -1 < \alpha. 
\end{cases}
$$

So the only non vanishing term in the case $\alpha + 1$ odd is
$V_{p^{\alpha+1}}(n)$.

Let us put things together now. Let $r$ be even. If $p \nmid n$,
then 
\begin{align*}
  \gamma_p = 1+ V_p & = 1- \frac{i^{\left(\frac{p-1}{2}\right)^2
      r}}{p^{r/2}}\\
  & = 1- \frac{(-)^{\frac{r}{2} \frac{p-1}{2}}}{p^{r/2}}
\end{align*}\pageoriginale

If $p\mid n$, $n= p^\alpha \cdot n'$, then
\begin{align*}
  \gamma_p & = 1+ V_p + V_{p^2} + \cdots + V_{p^\alpha} +
  V_{p^\alpha+1}\\
  & = 1+ \frac{\epsilon_p}{p^{r/2}} (p-1) + \frac{\epsilon_p^2}{p^2  r/2}
  p(p-1)+ \cdots\\
  & \hspace{4cm}+ \frac{\epsilon_p^\alpha}{p^\alpha r/2} p^{\alpha-1} (p-1)
  - \frac{\epsilon_p^{\alpha+1}}{p^{(\alpha+1) r/2}} p^\alpha,  
\end{align*}
where $\epsilon_p = (-)^{r (p-1)/4}$ for $r\neq 4$
\begin{align*}
  & = \left(1- \frac{\epsilon_p}{p^{r/2}}\right)+ \frac{\epsilon_p}{p^{r/2}-1}
  \left( 1- \frac{\epsilon_p}{p^{r/2}}\right)\\
  & \qquad +
  \frac{\epsilon_p^2}{p^2\left(\frac{r}{2}-1 \right)} \left(1- \frac{\epsilon
    _p}{p^{r/2}}\right)+ \cdots + \frac{\epsilon_p^\alpha}{p^{\alpha
      \left(\frac{r}{2}-1\right) }}
  \left(1-\frac{\epsilon_p}{p^{r/2}}\right)\\
  & = \left(1- \frac{\epsilon_p}{p^{r/2}}\right) \left(1-
  \frac{\epsilon_p^{\alpha+1}}{p^{(\alpha+1)\left(\frac{r}{2}-1\right)
  }}\right) ~\left(1- \frac{\epsilon_p}{p^{r/2-1}}\right)^{-1}
\end{align*}

For\pageoriginale $r=4$, the thing becomes critical: Let us look at
it more specifically.

$\dfrac{r(p-1)}{4}$ is even now and so $\epsilon_p=1$. Hence
$$
\gamma_p = \left(1- \frac{1}{p^2}\right) \frac{1-
  \frac{1}{p^{\alpha+1}}}{1- \frac{1}{p}}
$$ 

We go to the full singular series.
\begin{align*}
  S_4(n) & = \prod_p \gamma_p = \gamma_2 \prod_{p \geq 3} \gamma_p\\
  & = \gamma_2 \prod_{p \geq 3} \left(1- \frac{1}{p^2}\right) \prod_{p
  \geq 3} \frac{1- \frac{1}{p^{\alpha+1}}}{1- \frac{1}{p}}
\end{align*}

The product is convergent since $\sum \frac{1}{p^2} < \infty$. So 
\begin{align*}
  |S_4 (n)| & \geq \gamma_2 \prod_p \left(1- \frac{1}{p^2}\right)
  \prod_{p\mid n} \frac{1-\frac{1}{p^2}}{1- \frac{1}{p}}\\
  & \leq \gamma_2 \prod_p \left(1- \frac{1}{p^2}\right)^2 \prod_{p\mid
    n}  \frac{1}{1- \frac{1}{p}}
\end{align*}
$\prod {\left(1-\frac{1}{p}\right)}$ diverges to zero in the
infinite product senses. So $S_4(n)$ is not bounded. $S_4(n)$ could
become very small if we keep the odd factors fixed and introduce more
even factors.

$S_4(n)$ is unbounded in both senses; it can be as large as we please
or as small as zero.

For $r \geq 5$ we are again on the safe side. In this case the first
term comes from $V_{p^\lambda}$. We have
$$
\displaylines{S_5 (n) \thicksim \left(1\pm\frac{V_p}{p^{5/2}}\right)\cr
  \text{or} \hfill C_2 \prod \left(1+ \frac{1}{p^2}\right) < S_5 (n) <
  C_1 \prod \left(1- \frac{1}{p^2}\right)\hfill}
$$\pageoriginale 

For $r=7$ the situation is similar. For $r=6$ the series again
converges. So for $r \geq 5$.
$$
0 < C_1 < S_r (n) < C_2
$$

This is of importance in the application to our problem. 

We had 
$$
A_r(n) = \frac{\pi^{r/2}}{\Gamma(r/2)} n^{\frac{r}{2}-1} S_r(n) +
O (n^{r/4}) 
$$

If $r\geq 5, \frac{r}{2} -1 > \frac{r}{4}$, and since $S_{r(n)}$ being
bounded does not raise the order in the term,
$$
A_r (n) \thicksim \frac{\pi^{r/2}}{\Gamma (r/2)} n^{\frac{r}{2} -1}
S_r(n) 
$$ 

If, however, if $r=4$, the sharpness of the analysis is lost. Both the
first factor and the error term are $O (r)$ and $S_r(n)$ may
contribute to a decrease in the first term. If there are many odd
factors for $n$, the main term is still good. But if there are many
powers of 2, it would be completely submerged.

For $r=4$ the exact formula was given by Jacobi. 

We shall consider also representation of $n$ in the form $an_1^2 +
bn_2^2 + cn_3^2 + dn_4^2$ in which connection the Kloosherman sums
appear. We shall also cast a glance at the meaning of the singular
series in the sense of Siegel'e $p$-edic density.
