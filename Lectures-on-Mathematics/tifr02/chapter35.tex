\chapter{Lecture}\label{part4:lec35}
\markboth{\thechapter. Lecture}{\thechapter. Lecture}

After\pageoriginale we reduced our problem to the singular series in
which the Gaussian sums appear conspicuously, we have to do something about
them before we proceed further. The Gaussian sums are defined as 
$$
G(h, k)= \sum_{\ell \mod k} e^{2 \pi i \frac{h}{k} \ell^2}, (h, k)=1
$$

They obey a simple multiplication rule: if $k= k_1 k_2$, $(k_1,
k_2)=1$, then 
$$
G(h, k_1 k_2)= G(hk_1, k_2) \cdot G(hk_2, k_1).
$$

For, put $\ell = rk_1 + sk_2$; when $r$ runs modulo $k_2$ and $s$
modulo $k_1, \ell$ runs through a full residue system modulo $k_1
k_2$. Hence
\begin{align*}
  G(h, k_1 k_2) & = \sum_{k \mod k_2}~ \sum_{s \mod k_1} e^{2 \pi i
    \frac{h}{k_1 k_2} (k_1 r + k_2 s)^2}\\
    & = \sum_{r \mod k_2}~ \sum_{s \mod k_1} e^{2 \pi i \frac{h}{k_1
      k_2}(k_1^2 r^2 + k_2^2 s^2) }\\
  & = \sum_{r \mod k_2} e^{2 \pi i \frac{hk_1}{k_2} r^2} \sum_{s \mod
      k_1} e^{2 \pi i \frac{hk_2}{k_1}s^2}\\
  & = G(hk_1,k_2) G(hk_2, k_1).
\end{align*}

Ultimately, therefore, only prime powers have to be considered to
denominators. We have to distinguish the cases $p=2$ and $p> 2$, $p$
prime.

1) \pageoriginale Let $p \geq 3$, $k= p^\alpha$ with $\alpha>1$
$$
\displaylines{G(h, p^\alpha)= \sum_{l \mod p^r} e^{2 \pi i
    \frac{h}{p^\alpha} \ell^2}\cr
  \text{write} \hfill \ell = m p^{\alpha -1} + r;\hfill }
$$
$m=0, 1, \ldots, p-1$; $r=0, 1, \ldots p^{\alpha-1}-1$. Then this
becomes
$$
\sum^{p-1}_{m=0} \sum^{p^{\alpha -1}-1}_{r=0} e^{2 \pi i
  \frac{h}{p^\alpha} (m p^{\alpha-1}+ r)^2} = \sum^{p-1}_{m=0}
  \sum^{p^{\alpha-1-1}}_{r=0} e^{2 \pi i \frac{h}{p^\alpha}
    (m^2 p^{2 \alpha-1} + 2 mr p^{\alpha-1} + r^2)}
$$

Since $\alpha \geq 2$, $2 \alpha -2 \geq \alpha$ and so the first term
in the exponent may be omitted. This gives
$$
\sum^{p^{\alpha-1}-1}_{r=0} e^{2 \pi i
  \frac{h}{p^\alpha}r^2} \sum^{p-1}_{m=0} e^{2 \pi i \frac{h}{p} 2 m r}
$$

The inner sum is a sum of $p^{th}$ roots of unity; so it depends on
whether $p$ divides $2 rh$ or not. But $(h, p)=1$ and $p \nmid 2$. So we
need consider only the cases: $p\mid r$ and $p\nmid r$. However in the latter
case this sum is $0$ while in the former it is $p$. We therefore get,
when $p\mid r$, $r= ps$,
\begin{align*}
  p \sum^{p^{\alpha-1}-1}_{r=0, p\mid r} e^{2 \pi i
    \frac{h}{p^\alpha} r^2} & = p\sum^{p^{\alpha-2}-1}_{s=0}
    e^{2 \pi i \frac{h}{p^\alpha} p^2 s}\\
    & = p \sum^{p^{\alpha-2}-1}_{s=0} e^{2 \pi i
      \frac{h}{p^{\alpha-2}}s}\\
    & = p G(h, p^{\alpha-2})
\end{align*}

We\pageoriginale have therefore reduced the never of the denominator
by 2. We can 
repeat the process and proceed as long as we end with either the
$0^{th}$ or the $1^{st}$ power. So we have two chances. In the former
case, evidently $G(h, 1)=1$. So for $\alpha$ even, 
$$
G(h, p^\alpha) = p^{\alpha/2}
$$ 

On the other hand, if $\alpha$ is odd, we have
$$
G(h, p^\alpha) = p^{\frac{\alpha-1}{2}} G(h, p).
$$

These may be combined into the single formula
\begin{equation*}
  G(h, p^\alpha)= p^{[\frac{\alpha}{2}]} G \left(h,
  p^{\alpha-2[\frac{\alpha}{2}]}\right) \tag{1}\label{part4:lec35:eq1}
\end{equation*}

2) $p= 2^\lambda$, $\lambda \geq 2$. $h$ is now odd. Write 
\begin{gather*}
  \ell = m 2^{\lambda-1} + r; ~m =0 , 1; ~ r=0, 1, \ldots,
  2^{\lambda-1}-1\\
  G(h, 2^\lambda) = \sum^{2^{\lambda-1}-1}_{r=0} e^{2 \pi i
    \frac{h}{2^\lambda} r^2} + \sum^{2^{\lambda-1}-1}_{r=0}
  e^{2 \pi i \frac{h}{2^\lambda} (2^{\lambda-1}+r)^2}
\end{gather*}
since $\lambda \geq 2$, $2 \lambda -2 \geq \lambda$, in the second sum
it is only the exponent $r^2$ that contributes a non-zero term; and
this is\pageoriginale then the same the first. Altogether we have
then
\begin{equation*}
  2 \sum^{2^{\lambda-1}-1}_{r=0} e^{\pi i
    \frac{h}{2^\lambda-1} r^2} \tag{*}\label{part4:lec35:eq*}
\end{equation*}

This, however is not a Gaussian sum. The substitution for $\ell$ does
not work; to be effective, then we take
$$
\ell = m 2^{\lambda-2} + r; m=0, 1, 2, 3; ~r=0, 1,\ldots, 2^{\lambda-2}-1. 
$$

Now take $\lambda \geq 4$ and start again all over.
\begin{align*}
  G(h, 2^\lambda)& = \sum^3_{m=0} \sum^{2^{\lambda-2}-1}_{r=0} e^{2
    \pi i \frac{h}{2^\lambda} (m 2^{\lambda-2}+r)^2}\\
  & = \sum^3_{m=0} \sum^{2^{\lambda-2}-1}_{r=0} e^{2 \pi i
    \frac{h}{2^\lambda}(2^{\lambda-1}m r+ r^2)}, \quad (\text{for}~
  \lambda \geq 4 ~\text{i.e.,}~ 2 \lambda   -4 \geq \lambda).\\
  & = \sum^{2^{\lambda-2}-1}_{r=0} e^{2 \pi i \frac{h}{2^\lambda} r^2}
    \sum^3_{m=0} e^{\pi i h m r}\\
  & = \sum^{2^{\lambda-2}-1}_{r=0} e^{2 \pi i \frac{h}{2^\lambda} r^2}
    \sum^3_{m=0} (-)^{mn}\\
  & = 2 \sum^{2^{\lambda-2}-1}_{r=0} (-)^r e^{2 \pi i
      \frac{h}{2^\lambda} r^2} + 2 \sum^{2^{\lambda-2}-1}_{r=0} e^{2
      \pi i \frac{h}{2^\lambda} r^2}\\
  & = 4 \sum^{2^{\lambda-3}-1}_{s=0} e^{\pi i \frac{h}{2^{\lambda-3}}s^2}
\end{align*}

This\pageoriginale is not Gaussian sum either. But is is of the form
(\ref{part4:lec35:eq*}). We therefore have, for $\lambda \geq 4$,
$G(h, 2^\lambda)= 2 G(h, 
2^{\lambda -2})$. If $\lambda=4$, we need go down to only $2^2 =4$ and
if $\lambda =5$ to $2^3=8$. So we need separately $G(h, 8)$ and $G(h,
4)$; and of course $G(h, 2)$. These cases escape us, while formerly
only $G(h, p)$ did. For $\lambda \geq 4$, we may write
\begin{equation*}
  G(h, 2^\lambda)= 2^{\left[\frac{\lambda}{2}\right]-1} G\left(h,
  2^{\lambda-2 \left[\frac{\lambda}{2}+2 \right]}\right) \tag{2}\label{part4:lec35:eq2}
\end{equation*}

This supplements formula (\ref{part4:lec35:eq1}).

We now consider the special cases, $k=2, 4,8$. Here $h$ is odd.
\begin{align*}
  G(h, 2) & = 1+ e^{2 \pi i \frac{h}{2}}=0\\
  G(h, 4) & = 1+ e^{2 \pi i \frac{h}{4}\cdot 1} + e^{2 \pi i
    \frac{h}{4} \cdot 4}+ e^{2 \pi i \frac{h}{4} \cdot 9}\\
  & = 2 + 2 e^{\pi i \frac{h}{2}}\\
  & = 2\left( 1+ i^h\right)\\
  G(h, 8) & = 1+ 1+ 2 e^{\pi i h} + 4 e^{2 \pi i \frac{h}{8}}\\
  & \qquad (\text{since}~ 1^2, 3^2, 5^2, 7^2 ~\text{are all}~ \equiv 1
 ~\text{modulo}~ 8)\\
  & = 4 e^{\pi i \frac{h}{4}}= 4\left(\frac{1+ i}{\surd 2} \right)^2
\end{align*}

Before\pageoriginale we return to $G(h, p)$, $p > 2$, we shall a
digression an connect to the whole thing with the Legendre-Jacobi symbols
\begin{align*}
  G(h, p) & = \sum^{p-1}_{\ell=0} e^{2 \pi i \frac{h}{p} \ell^2}\\
  & = 1+ 2 \sum_a e^{2 \pi i \frac{h}{p} a},
\end{align*}
the summation over all quadratic residues a modulo $p$, since along
with $\ell$, $p- \ell$ is also a quadratic residue. We can write this
in a compact form, so arranging it that the non-residues get cancelled
and the residues appear twice:
\begin{align*}
  G(h, p) & = \sum_{r \mod p} \left\{ 1+ \left(
  \frac{r}{p}\right)\right\} e^{ 2\pi i \frac{h}{p}r}\\
  & = \sum_{r \mod p} \left( \frac{r}{p}\right) e^{2 \pi i \frac{h}{p}
  r}
\end{align*}

This would appear in a completely new aspect if we utilised the fact
that $hr$ runs through a full system of residues modulo $p$. Then 
\begin{align*}
  G(h, p) & = \sum_{k \mod p} \left(\frac{h}{p} \right)
  \left(\frac{hr}{p} \right) e^{2 \pi i \frac{h}{p} r}\\
  & = \left(\frac{h}{p} \right) \sum_{r \mod p} \left(\frac{r}{p}
  \right) e^{2 \pi i \frac{r}{p}}\\
  & = \left(\frac{h}{p} \right)  G(h, p).
\end{align*}

This\pageoriginale is very useful if we new go to the Jacobi symbol. For
prime $p$, the Legendre symbol has the multiplicative property:
$$
\left(\frac{r_1}{p} \right) \left(\frac{r_2}{p} \right)=
\left(\frac{r_1 r_2}{p} \right)    
$$
Jacobi has the following generalisation.

Define $\left(\frac{r}{pq} \right) $ by
$$
\left(\frac{r}{pq} \right)= \left(\frac{r}{p} \right)
\left(\frac{r}{q} \right).    
$$

Si it is $\pm 1$; if it is $+1$ it does not necessarily mean that $r$
is a quadratic residue modulo $pq$. The Jacobi symbol no longer
discriminates between residues and non residues. From the definition
then 
$$
\left(\frac{a}{p^\alpha q^\beta \cdots} \right)  = \left(\frac{a}{p}
\right)^\alpha \left(\frac{a}{q} \right)^\beta \cdots.   
$$

The Jacobi symbol has the properties of a character, as can be
verified by using the Chinese remainder theorem.

We can now write
$$
G(h, p^\alpha)= \left(\frac{h}{p} \right)^\alpha G(1, p^\alpha) 
$$
under all circumstances. How does this come about? Separate the cases:
$\alpha$ even, $\alpha$ odd.
\begin{alignat*}{4}
  G(g, p^\alpha) & = G(1, p^\alpha), && \alpha ~\text{even};\\
  & = p^{\frac{\alpha-1}{2}} G(h, p), && \alpha ~\text{odd},\\
  & = \left(\frac{h}{p} \right)  p^{\frac{\alpha-1}{2}}& G(1, p) =
  & \left(\frac{h}{p} \right)  G(1, p^\alpha)
 \end{alignat*}

We\pageoriginale can write both in one sweep as
\begin{align*}
  G(h, p^\alpha) & = \left(\frac{h}{p} \right)^\alpha G(1, p^\alpha)\\
  & = \left(\frac{h}{p^\alpha} \right) G(1, p^\alpha) 
\end{align*}

Now use the multiplicative law. If $p$, $q$ are odd primes, then 
\begin{align*}
  G(h, p^\alpha q^\beta) & = G(hp^\alpha, q^{\beta}) G(hq^\beta,
  p^\alpha)\\
  & = \left(\frac{hp^\alpha}{q^\beta} \right) G(1, q^\beta)
  \left(\frac{hq^\beta}{p^\alpha} \right) G(1, p^\alpha)   
\end{align*}

Since the Jacobi symbol is separately multiplicative in numerator and
denominator, but not both, this is equal to 
\begin{multline*}
  \left(\frac{h}{q^\beta} \right)
  \left(\frac{p^\alpha}{q^\beta}\right) G(1, q^\beta)
  \left(\frac{h}{p^\alpha}\right) \left(\frac{q^\beta}{p^\alpha}
  \right) G(1, p^\alpha)
  = \left(\frac{h}{q^\beta} \right) \left(\frac{h}{p^\alpha}
  \right)G(p^\alpha, q^\beta) G(q^\beta, p^\alpha),
\end{multline*}
taking the second and third factors together, and also the last
two. And this is 
$$
\left(\frac{h}{p^\alpha q^\beta} \right)  G(1, p^\alpha q^\beta)
$$
according to the multiplication law.

Suppose that we have
$$
G(h_1k_1)= \left(\frac{h}{k_1} \right)  G(1, k_1); G(h, k_2) =
\left(\frac{h}{k_2} \right) G(1, k_2).  
$$

We\pageoriginale go through the above worker; literally and get 
$$
G(h, k_1 k_2) = \left(\frac{h}{1,h_2} \right) G(1, k_1, k_2). 
$$

So we have proved in general that for odd $k$,
$$
G(h, k)= \left(\frac{h}{k} \right) G(1, k) 
$$

We can now return to $G(h, p)$.
