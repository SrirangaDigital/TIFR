\chapter{Lecture}\label{part4:lec34} %% 34
\markboth{\thechapter. Lecture}{\thechapter. Lecture}

We\pageoriginale had discussed the sum $\Theta \left( e^{2 \pi i
\frac{h}{k} - 2 \pi \mathfrak{z}}\right)$ and written it equal to 
$$
\displaylines{\frac{1}{k \sqrt{2\mathfrak{z}}} \left\{ G (h, k) + H
  (h, k; \mathfrak{z})\right\}\cr
  \text{where} \hfill |H (h, k; \mathfrak{z})| < C \sqrt{k} e^{-
    \frac{\pi}{2k^2} \mathscr{R} \frac{1}{\mathfrak{z}} } \hfill }
$$

If we apply this to the integral in which $\Theta^r$ appears,
$$
\Theta \left( e^{2 \pi i \frac{h}{k} - 2 \pi \mathfrak{z}}\right)^r =
\frac{1}{ k^r (2 \mathfrak{z})^{\frac{k}{2}}} \sum^r_{\lambda =0}
\binom{r}{\lambda} G(h, k)^{r- \lambda} H(h, k; \mathfrak{z}), 
$$
or, keeping the piece corresponding to $\lambda=0$ apart,
$$
\Theta \left(e^{2 \pi i \frac{h}{k}- 2 \pi \mathfrak{z}} \right)^r -
\frac{1}{k^r (2\mathfrak{z})^{\frac{r}{2}}} G(h, k)^r = \frac{1}{k^r
  (2 \mathfrak{z})^{\frac{r}{2}}} \sum^r_{\lambda=1}
\binom{r}{\lambda} G(h, k)^{r-\lambda} H (h, k; \mathfrak{z})^\lambda
$$

Let us appraise this. Since
\begin{align*}
\left| \Theta \left( e^{2 \pi i \frac{h}{k} - 2 \pi
  \mathfrak{z}}\right)^r - \frac{1}{k^r (2 \mathfrak{z})^{\frac{r}{2}}}
  G(h, k)^r\right|  & < C \frac{1}{k^r |\mathfrak{z}|^{\frac{r}{2}}}
    \sum^r_{\lambda=1} (\sqrt{k})^{r-\lambda} k^{\frac{\lambda}{2}}
    e^{-\frac{2 \pi}{2k^2}} \mathscr{R} \frac{1}{\mathfrak{z}}\\
    & < C \cdot \frac{1}{(k |\mathfrak{z}|)^{\frac{r}{2}}} e^{-
        \frac{\pi}{2}}  e^{- \frac{\pi}{2k^2}\mathscr{R}
      \frac{1}{\mathfrak{z}}} 
\end{align*}

Now\pageoriginale
$$
A_r (n) = \mathop{\textstyle\sum'}_{0 \leq h < k \leq N} e^{-2 \pi i \frac{h}{k} n}
\int\limits_{- \mathscr{V}'_{hk}}^{\mathscr{V}''_{hk}} 2 \pi
\mathfrak{z} n \Theta \left(e^{2 \pi i \frac{h}{k} n-2 \pi \mathfrak{z}} \right)
$$
where, of course, $\mathfrak{z}= \delta_N - i \varphi$. Hence
\begin{align*}
  & \left| A_r (n)  - \mathop{\textstyle\sum'}_{0 \leq h < k \leq N}
  e^{-2 \pi i \frac{h}{k}
    n} \int\limits^{\mathscr{V}''_{hk}}_{-\mathscr{V}'_{hk}}
  \frac{e^{2\pi n \mathfrak{z}}}{k^r(2\mathfrak{z})^{\frac{r}{2}}} G(h,
  k)^r d \varphi\right|\\
  & \leq C \mathop{\textstyle\sum'}_{0 \leq h < k \leq N}
  \int\limits_{-\mathscr{V}'_{hk}}^{\mathscr{V}''_{hk}} e^{2 \pi n
    \mathscr{R} \mathfrak{z}} \frac{e^{- \frac{2 \pi}{k^2}\mathscr{R}
      \frac{1}{\mathfrak{z}}}}{k^{r/2}|\mathfrak{z}|^{r/2}} d \varphi\\
  & \leq C \mathop{\textstyle\sum'}_{0 \leq h < k \leq N} e^{2 \pi n \delta_N}
  \int\limits_{- \mathscr{V}'_{hk}}^{\mathscr{V}''_{hk}} \frac{e^{-
      \frac{\pi}{2k^2} \frac{\delta \nu}{\delta_N^2 +
        \varphi^2}}}{\left[h^2 (\delta_N^2 + \varphi^2)
      \right]^{\frac{r}{4}}} d \varphi\\
  & = C \mathop{\textstyle\sum'}_{0 \leq h < k \leq N} e^{2 \pi n \delta_N} \delta_N^{-
    \frac{r}{4}} \int\limits_{-
    \mathscr{V}'_{hk}}^{\mathscr{V}''_{hk}} \left(\frac{\delta_N}{k^2
    (\delta_N^2 + \varphi^2)} \right)^{\frac{r}{4}} e^{-
    \frac{\pi}{2k^2} \frac{\delta_N}{\delta_N^2 + \varphi^2}} d \varphi
\end{align*}

Now\pageoriginale $\frac{1}{2k N} \leq \mathscr{V}_{hk} \leq
\frac{1}{kN}$ and $\mathscr{V}'_{hk} \leq \varphi \leq
\mathscr{V}''_{hk}$, while $\delta_N = \frac{1}{N^2}$. Putting $X=
  \frac{\delta_N}{k^2(\delta_N^2 + \varphi^2)}$, the integrand becomes
  $X^{\frac{r}{4}} e^{- \frac{\pi}{2} X}$ which remains bounded. (It
  was for this purpose that in our estimate of $H (h, k;
  \mathfrak{z})$ earlier we retained the factor $e^{- \pi/(2k^2) \cdot
  \mathscr{R} \frac{1}{\mathfrak{z}}}$). Hence the last expression is
  less than or equal to 
$$
C \sum_{o \leq h < k \leq N} e^{2 \pi \frac{n}{N^2}} N^{\frac{r}{2}}
\int\limits^{\mathscr{V}''_{hk}}_{-\mathscr{V}'_{hk}} d\varphi= C
e^{2\pi \frac{n}{N^2}} N^{\frac{r}{2}},
$$
since the whole Farey dissection exactly fills the interval $(0, 1)$. 

In the next stage of our argument we take the integral
$$
\int\limits_{\mathscr{V}'_{hk}}^{\mathscr{V}''_{hk}} \frac{e^{2 \pi n
    \mathfrak{z}}}{\mathfrak{z}^{\frac{k}{4}}} 
$$
and write it as
$$
\left( \int\limits^\infty_{- \infty} -
\int\limits^\infty_{\mathscr{V}''_{hk}} - \int\limits^{-
  \mathscr{V}'_{hk}}_{- \infty}\right) \frac{e^{2 \pi n
    \mathfrak{z}}}{\mathfrak{z}^{r/2}} d \varphi 
$$

The infinite integrals are conditionally convergent if $r > 0$
(because the numerator is essentially trigonometric), and absolutely
convergent\pageoriginale for $r>2$, so that we take $r$ at least equal to 3. Then
\begin{align*}
  \left| \int\limits^{\infty}_{\mathscr{V}''_{hk}} \frac{e^{2 \pi n
      \mathfrak{z}}}{\mathfrak{z}^{r/2}} d \varphi\right| & \leq
  e^{2\pi \frac{n}{N^2}} \int\limits^\infty_{\mathscr{V}''_{hk}}
  \frac{d \varphi}{(\delta^2_N + \varphi^2)^{\frac{r}{2}}}\\
  &  \leq e^{2 \pi \frac{n}{N^2}} \int\limits^\infty_{\frac{1}{2kN}}
  \frac{d \varphi}{(\delta^2_N + \varphi^2)^{\frac{r}{4}}}
\end{align*}

(Here and in the estimate of the other integral $\int\limits^{-
  \mathscr{V}'_{hk}}_{- \infty}$, we make use of the fact that the
interval from $\mathscr{V}'_{hk}$ to $\mathscr{V}''_{hk}$ is neither
too long nor too short. This argument arises also in Goldbach's problem
and Waring's problem). The right side is equal to 
\begin{align*}
  N^{r-2} e^{2 \pi \frac{n}{N^2}} \int\limits^\infty_{- \frac{1}{2kN}}
  \frac{N^2 d \varphi}{(1+ N^4 \varphi^2)^{\frac{r}{4}}} & = e^{2 \pi
    \frac{n}{N^2}} N^{r-2} \int\limits^\infty_{\frac{N}{2k}} \frac{d
    \psi}{(1+ \psi^2)^{r/4}}\\
  & < e^{2 \pi \frac{n}{N^2}} N^{r-2}
  \int\limits^\infty_{\frac{N}{2k}} \frac{d \psi}{\psi^{r/2}}
\end{align*}

This appears crude but is nevertheless good since $\varphi$ never
comes near $0$; $N/2k> \frac{1}{2}$, and the ratio of $\psi^2$ to
$1+ \psi^2$ is at least $\frac{1}{3}$ and so we lose no essential
order of magnitude. The last integral is equal to 
\begin{align*}
  Ce^{2\pi \frac{n}{N^2}} & N^{r-2} \left(\frac{N}{2k} \right)^{-
    \frac{r}{2}+1}, r \geq 3,\\ 
  & = C e^{2\pi \frac{n}{N^2}} N^{\frac{r}{2}-1} k^{\frac{r}{2}-1}
\end{align*}

A\pageoriginale similar estimate holds for $\int\limits^{-
  \mathscr{V}'_{hk}}_{- \infty}$ also. So, 
\begin{align*}
  & \left| A_r (n) - \mathop{\textstyle\sum'}_{0\leq h < k \leq N}
  e^{- 2 \pi i \frac{h}{k}
    n}\left( \frac{G(h, k)}{k \sqrt{2}}\right)^r
  \int\limits^\infty_{- \infty} \frac{e^{2 \pi n
      \mathfrak{z}}}{\mathfrak{z}^{r/2}} d \varphi\right|\\
  & < C e^{2 \pi \frac{n}{N^2}} N^{r/2} + C \mathop{\textstyle\sum'}_{0 \leq h< k \leq N}
  \frac{1}{k^{r/2}} e^{2 \pi \frac{n}{N^2}} N^{r/2-1} k^{r/2-1}\\
  & < C e^{2 \pi \frac{n}{N^2}} N^{r/2} + C e^{2 \pi \frac{h}{N^2}}
  N^{r/2-1} \mathop{\textstyle\sum'}_{0 \leq k \leq N}\\
  & = C e^{2 \pi \frac{n}{N^2}} N^{r/2}.
\end{align*}

This, however, does not go to zero as $N \to \infty$; we have no good
luck here as we had in partitions. So we make the best of it, and
obtain an asymptotic result. Let $n$ also tend to infinity. We shall
keep $n/N^2$ bounded, without lotting; it go to zero, as in the latter
case the exponential factor would become 1. We have to see to it
that $n \leq CN^2$ i.e., $N$ is at least $\sqrt{n}$. Otherwise the
error term would increase fast. Making $N$ bigger would not help in
the first\pageoriginale factor and would make the second worse. So
the optical choice for $N$ would be $N= [\sqrt{N}]$. The error would
now be 
$$
O \left( n^{\frac{r}{4}}\right)
$$

We next evaluate the integral
$$
\int\limits^\infty_{- \infty} \frac{e^{2 \pi n
    \mathfrak{z}}}{\mathfrak{z}^{r/2}} d \varphi
$$

This is the some as 
\begin{align*}
  \int\limits^\infty_{- \infty} \frac{e^{2 \pi n (\delta_N - i
      \varphi)}}{(\delta_N - i \varphi)^{r/2}} d \varphi & = -
  \int\limits^{- \infty}_\infty \frac{e^{2 \pi n (\delta_N + i
      \alpha)}}{(\delta_N + i \alpha)^{r/2}} d \varphi\\
  & = \frac{1}{i} \int\limits^{\delta_N+ i \infty}_{\delta_N - i
    \infty}\frac{e^{2 \pi n s}}{s^{r/2}} ds
\end{align*}

After a little embellishment this becomes a well-known integral. It is
equal to
$$
\frac{(2 \pi n)}{i}^{r/2} \int\limits^{2 \pi n \delta_N + i \infty}_{2 
\pi n \delta_N - i \infty} \frac{e^\omega}{\omega^{r/2}} d \omega
$$
which exists for $r>2$, and is actually the Hankel loop integral, and
hence equal to 
$$
\frac{2 \pi (2 \pi n)^{r/2}-1}{\Gamma(r/2)}
$$

Hence,\pageoriginale for $f \geq 3$. We hence the number of
representations of $n$ as the sum of $r$ squares:
$$
A_r (n) = \frac{(2 \pi)^{r/2}}{\Gamma (r/2)} \cdot
\frac{n^{\frac{r}{2}-1}}{2^{r/2}} \cdot \mathop{\textstyle\sum'}_{0 \leq h < k \leq N}
\frac{G(h, k)^r}{k^r} e^{-2ri \frac{h}{k}}+ O (n^{r/4}).
$$

One final step. Let us improve this a little further. Write
$$
\sum_{h \mod k} \frac{G(h, k)^r}{k^r} e^{- 2\pi i \frac{h}{k} n} =
V_k^{(r)} (n) = V_k (n)
$$

We have to sum $V_k(n)$ from $k=1$ to $k=N$. However, we sum from
$k=1$ to $k=\infty$, thereby incurring an error
$$
\left| \sum^\infty_{k=N+1} V_k (n)\right| \leq \sum^\infty_{k=N+1}
k^{- \frac{r}{2}+1},
$$
and this converging absolutely for $r\geq 5$ is 
$$
O \left( N^{- \frac{r}{2} +2}\right) = O \left( n^{-
  \frac{r}{4}+1}\right) 
$$

This along with the factor $n^{\frac{r}{2}-1}$ would give exactly
$O (n^{r/4})$. (We could have saved this for $r=4$ also if we
had been a little more careful). Thus, for $r\geq 5$, we
have\pageoriginale
$$
\displaylines{
  A_r(n) = \frac{\pi^{r/2}}{\Gamma(r/2)} n^{\frac{r}{2}-1} S_r (n) +
  O (n^{r/4}), \cr
  \text{where} \hfill S_r (n) = \sum^\infty_{k=1} V_k (n)\hfill }
$$

$S_r(n)$ is the singular series. We shall show that $S_r(n)$ remains
bounded at least for $r\geq 5$.
