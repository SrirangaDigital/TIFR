\part{Analysis}\label{part2}

\chapter{Lecture}\label{part2:lec9}
\markboth{\thechapter. Lecture}{\thechapter. Lecture}

\heading{Theta-functions}

A\pageoriginale power series hereafter shall for us mean something entirely
different from what it did hitherto. $x$ is a complex variable and
$\sum\limits^\infty_{n=0} a_n x^n$ will have a value, its sum, which
is ascertained only only after we introduce convergence. Then
$$
f(x) = \sum^\infty_{n=0} a_n x^n;
$$
$x$ and the series are coordinated and we have a function on the
complex domain. We take for granted the theory of analytic functions
of a complex variable; we shall be using Cauchy's theorem frequently,
and in a moment we shall have occasion to use Weierstrass's double
series theorem. 

Let us go back to the Jacobi identity:
\begin{align*}
  \prod^\infty_{n=1} (1-x^{2n}) (1+ \mathfrak{z}x^{2n-1})
  (1+\mathfrak{z}^{-1} x^{2n-1})
  & = \sum^\infty_{k=-\infty} \mathfrak{z}^k x^{k^2}\\
  & = 1+ \sum^\infty_{k=1} (\mathfrak{z}^k + \mathfrak{z}^{-k})
  x^{k^2}, (\mathfrak{z}\neq 0),
\end{align*}
which is a power series in $x$. Two questions arise. First, what are
the domains of convergence of both sides? Second, what does equality
between the two sides mean? Formerly, equality meant agreement of the
coefficients\pageoriginale up to any stage; what it means now we have
got to explore. The left side is absolutely convergent - and absolute
convergence is enough for us - for $|x|<1$; (for the infinite product
$\prod (1+ a_n)$ is absolutely convergent if $\sum |a_n|< \infty$; $z$
is a complex variable which we treat as a parameter). For the right
side we use the Cauchy-Hadamard criterion for the radius of
convergence:
\begin{align*}
  \rho & = \frac{1}{\llim\sqrt[n]{|a_n|}}\\
  & = \frac{1}{\llim \sqrt[k^2]{|\mathfrak{z}+ \mathfrak{z}^{-k}}|}
\end{align*}

Suppose $|\mathfrak{z}|>1$; then *****************, and 
$$
\displaylines{\hfill |\mathfrak{z}^k+ \mathfrak{z}^{-k}| <2 |
  \mathfrak{z}|^k, \hfill \cr
  \text{and} \hfill \sqrt[k^2]{|\mathfrak{z}^k + \mathfrak{z}^{-k}|} <
  \sqrt[k^2]{2} \sqrt[k]{|\mathfrak{z}|} \to 1 ~\text{as}~ k \to
  \infty \hfill \cr
  \hfill \therefore \quad \llim (\sqrt[k^2]{|\mathfrak{z}^k+
    \mathfrak{z}^{-k}|})\leq 1. \hfill }
$$

It is indeed $=1$, not $<1$, because ultimately, if $k$ is large
enough, $|\mathfrak{z}|^k> 1$, and so
$$
\frac{1}{2} |\mathfrak{z}|^k< |\mathfrak{z}^k+ \mathfrak{z}^{-k}|,
$$
and we have the reverse inequality. By symmetry in $\mathfrak{z}$ and
$1/\mathfrak{z}$, this holds also for $|\mathfrak{z}|<1$. The case
$|\mathfrak{z}|=1$ does not present any serious difficulty either. So
in all cases $\rho=1$. Thus both sides are convergent for $|x|<1$, and
indeed \textit{uniformly} in any closed circle $|x|\leq 1-\delta <
1$. 

The\pageoriginale next question is, why are the two sides equal in
the sense of function theory? This is not trivial. Here equality of
values of coefficients up to any definite stage is not sufficient as
it was before; the unfinished coefficients before multiplication may
go up and cannot be controlled. Here, however, we are in a strong
position. We have to prove that
$$
\prod^N_{n=1} (1-x^{2n}) (1+ \mathfrak{z}x^{2n-1})(1+\mathfrak{z}^{-1}
x^{2n-1}) \to 1+ \sum^\infty_{k=1} (\mathfrak{z}^k +
\mathfrak{z}^{-k}) x^{k^2}
$$
with increasing $N$, when $|x|<1$, and indeed uniformly so in $|x|\leq
1 - \delta < 1$. On the left side we have a sequence of polynomials:
$$
f_N (x) = \prod^N_{n=1} (1-x^{2n})(1+\mathfrak{z} x^{2n-1})
(1+\mathfrak{z}^{-1} x^{2n-1}) = \sum^\infty_{m=0} a_m^{(N)} x^m,
~\quad\text{say}. 
$$
(of course the coefficients are all zero beyond a certain finite
stage). Now we know that the left side is a partial product of a
convergent infinite product; in fact $f_N(x)$ tends uniformly to a
series, $f (x)$, say. Now what do we know about a sequence of analytic
functions on the same domain converging uniformly to a limit function?
The question is answered by Weierstrass's double series theorem. We
can assert that $f(x)$ is analytic in the same domain at least, and
further if $f(x) = \sum\limits^\infty_{m=0} a_m x^m$, then
$$
a_m = \lim\limits_{N \to \infty} a_m^{(N)}.
$$

The\pageoriginale coefficients of the limit function have got
something to do with the original coefficients. Now
$$
a_m^{(N)}= \frac{1}{2 \pi i} \int\limits_{|x|=1-\delta} \frac{f_N(x)}{x^{m+1}}dx
$$

Let $N \to \infty$; this is permissible by uniform convergence and the
$a_m^{(N)}, s$ in fact converge to 
$$
a_m = \frac{1}{2 \pi i} \int\limits_{|x|=1-\delta} \frac{f(x)}{x^{m+1}}dx.
$$

(Weierstrass's' own proof of this theorem was what we have given here,
in some disguise; he takes the values at the roots of unity and takes
a sort of mean value).

Now what are the coefficients in
$1+\sum(\mathfrak{z}^k+\mathfrak{z}^{-k})x^{k^2}$? Observe that the
convergence of $a_m^{(N)}$ to $a_m$ is a peculiar and simple
one. $a_m^{(N)}$ indeed converges to a known $a_m$; as a matter of
fact $a_m^{(N)}=a_m$ for $N$ sufficiently large. They reach a limit
and stay put. And this is exactly the meaning of our formal
identity. So the identity has been proved in the function-theoretic
sense:
$$
\prod^\infty_{n=1} (1-x^{2n}) (1+\mathfrak{z}
x^{2n-1})(1+\mathfrak{z}^{-1} x^{2n-1})= 1+ \sum^\infty_{k=1}
(\mathfrak{z}^k+ \mathfrak{z}^{-k}) x^{k^2} = \sum^\infty_{k=-\infty}
\mathfrak{z}^k x^{k^2}.
$$

These\pageoriginale things were done in full extension by Jacobi. Let
us  employ the usual symbols; in place of $x$ write $q, |q|<1$, and
put $z=e^{2 \pi i v}$. Notice that the right side is a Laurent
expansion in $z$ in $0< |\mathfrak{z}|< \infty$ ($v$ is unrestricted
because we have used the exponential). We write in the traditional
notation 
\begin{align*}
  \prod^\infty_{n=1} (1-q^{2n})& (1+q^{2n-1} e^{2 \pi i v})(1+ q^{2n-1}
  e^{-2\pi i v})\\
  & = \sum^\infty_{n=-\infty} q^{n^2} e^{2 \pi i n v}\\
  & = v_3 (v, q)
\end{align*}
$v_3$ (and in fact all the theta functions) are entire functions of
$v$. We have taken $|q|< 1$; it is customary to write $q=e^{\pi i
  \tau}$, so that $|q|<1$ implies
$$
\displaylines{\hfill |e^{\pi i \tau}|= e^{\mathscr{R}\pi i \tau},
  \mathscr{R} \pi i \tau < 0\hfill \cr
  \text{i.e.,} \hfill \mathscr{R} i \tau < 0 \quad \text{or}\quad
  \mathscr{I} m \tau > 0 \hfill }
$$ 
$\tau$ is a point in the upper half-plane. $\tau$ and $q$ are
equivalent parameters. We also write
$$
\mathscr{V}_3 (\mathscr{V}, q)= \mathscr{V}_3 (\mathscr{V}/\tau)
$$

(An excellent account of the $\mathscr{V}$-functions can be found in
Tannery and Molk: Fonctiones Elliptiques, in 4 volumes; the second
volume contains a very well organized collection of formulas).

One remark is immediate from the definition of $\mathscr{V}_3$, viz.
$$
\mathscr{V}_3 (\mathscr{V}+1, q)= \mathscr{V}_3(\mathscr{V}, q)
$$

On\pageoriginale the other hand,
\begin{align*}
  \mathscr{V}_3 (\mathscr{V}+\tau, q) & = \prod^\infty_{n=1}
  (1- q^{2n})(1-q^{2n-1}e^{2 \pi i \mathscr{V}} e^{2 \pi i \tau})\times (1+
  q^{2n-1} e^{-2 \pi i \mathscr{V}} e^{-2\pi i \tau})\\
  & = \sum^\infty_{n=-\infty} q^{n^2} e^{2 \pi i n \mathscr{V}} e^{2
    \pi i n \mathscr{V}},
\end{align*}
and since $q=e^{\pi i \tau}$,
$$
\prod^\infty_{n=1} (1-q^{2n}) (1+ q^{2n+1}e^{2 \pi i \mathscr{V}}) (1+
q^{2n-3}e^{-2\pi i \mathscr{V}}) = \sum^\infty_{n=- \infty} q^{n^2+
  2n} e^{2 \pi i n \mathscr{V}}
$$
or 
\begin{align*}
  \frac{1+q^{-1} e^{-2 \pi i \mathscr{V}}}{1+q e^{2 \pi i
      \mathscr{V}}} & \prod^\infty_{n=1} (1-q^{2n})(1+ q^{2n-1}e^{2 \pi i
    \mathscr{V}})(1+q^{2n-1} e^{-2\pi i \mathscr{V}})\\
  & = q^{-1} e^{-2 \pi i \mathscr{V}} \sum^\infty_{n=-\infty}
  q^{(n+1)^2} e^{2 \pi i (n+1)\mathscr{V}}\\
  & = q^{-1} e^{-2 \pi i \mathscr{V}} \mathscr{V}_3(\mathscr{V}, q)\\
  & = (q e^{2\pi i \mathscr{V}})^{-1} \mathscr{V}_3 (\mathscr{V}, q)
\end{align*}

So we have the neat result:
$$
\mathscr{V}_3 (\mathscr{V}+ \tau, q)= q^{-1} e^{-2 \pi i \mathscr{V}}
\mathscr{V}_3 (\mathscr{V}, q)
$$
1\pageoriginale is a period of $\mathscr{V}_3$ and $\tau$ resembles a period. It is
quite clear that we cannot expect 2 periods in the full sense, because
it is impossible for an entire function to have two periods. Indeed if
$\omega_1$ and $\omega_2$ are two periods of $f$, then $f(\mathscr{V}+
\omega_1) =f(\mathscr{V})$, $f(\mathscr{V}+ \omega_2)=
f(\mathscr{V})$, and $f(\mathscr{V}+ \omega_1+ \omega_2)=
f(\mathscr{V})$ and the whole module generated by $\omega_1$ and
$\omega_2$ form periods. Consider the fundamental region which is the
parallelogram with vertices at $0, \omega_1, \omega_2$, $\omega_1+
\omega_2$. If the function is entire it has no poles in the
parallelogram and is bounded there (because the parallelogram is
bounded and closed), and therefore in the whole plane. Hence by
Liouville's theorem the function reduces to a constant.

While dealing with trigonometric functions one is not always satisfied
with the cosine function alone. It is nice to have another function:
$\cos (x- \pi/2)= \sin x$. A shift by a half-period makes it
convenient for us. Let us consider analogously
$\mathscr{V}_3(\mathscr{V}+ \frac{1}{2},q)$,
$\mathscr{V}_3(\mathscr{V}+ \tau/2, q)$, and
$\mathscr{V}_3(\mathscr{V}+ \frac{1}{2} + \frac{\tau}{2}, q)$. Though
$\tau$ is not strictly a period we can still speak of the fundamental
region, because on shifting by $\tau$ we change only by a trivial
factor. Replace $\mathscr{V}$ by $\mathscr{V}+\frac{1}{2}$ and
everything is fine as 1 is a period.
\begin{align*}
  \mathscr{V}_3 (\mathscr{V}+ \frac{1}{2}, q) & = \prod^\infty_{n=1}
  (1-q^{2n}) (1-q^{2n-1} e^{2 \pi i \mathscr{V}})(1-q^{2n-1} e^{-2\pi
    i \mathscr{V}})\\
  & = \sum^\infty_{n=-\infty} (-)^n q^{n^2}e^{2 \pi i n \mathscr{V}}
\end{align*}
which is denoted $\mathscr{V}_4(\mathscr{V}, q)$

Again\pageoriginale
\begin{align*}
  \mathscr{V}_3 (\mathscr{V}+ \frac{\tau}{2}, q) & = \prod^\infty_{n=1}
  (1-q^{2n}) (1+q^{2n-1} e^{2 \pi i \mathscr{V}}e^{\pi i \tau})(1+q^{2n-1} e^{-2\pi
    i \mathscr{V}}e^{-\pi i \tau})\\
  & = \sum^\infty_{n=-\infty} q^{n^2} e^{2 \pi i n \mathscr{V}} e^{\pi
  i n \tau}\\
\text{i.e.,} \quad(1+ e^{-2 \pi i \mathscr{V}})
  &\prod\limits^\infty_{n=1} (1- q^{2n}) (1+q^{2n} e^{2 \pi i \mathscr{V}})
  (1+q^{2n} e^{- 2 \pi i \mathscr{V}}) \\
 & = \sum\limits^\infty_{n=- \infty} q^{n^2 + n} e^{2 \pi i n
   \mathscr{V}}\\
 & = q^{-1/4} e^{- \pi i\mathscr{V}} \sum^\infty_{n=-\infty} q^{(n+
  1/2)^2}e^{(2n+1)\pi i \mathscr{V}}\\
 & = q^{-1/4} e^{- \pi i \mathscr{V}} \mathscr{V}_2 (\mathscr{V}, q)
\end{align*}
where $\mathscr{V}_2(\mathscr{V}, q)= \sum\limits^\infty_{n=-\infty}
q^{(n+ 1/2)^2} e^{(2n+1)\pi i \mathscr{V}}$, by definition. (Here
$q^{-1/4}$ does not contain an unknown 4th root of unity as factor,
but is an abbreviation for $e^{- \pi i \tau/4}$, so that it is well
defined). So
$$
\mathscr{V}_2 (\mathscr{V}, q)= 2q^{1/4} \cos \pi \mathscr{V}
\prod^\infty_{n=1} (1- q^{2n})(1+q^{2n} e^{2 \pi \tau \mathscr{V}})(1+ q^{2n}
e^{-2 \pi i \mathscr{V}})
$$

Finally\pageoriginale
\begin{align*}
  \mathscr{V}_3 (\mathscr{V}+ \frac{1+\tau}{2}, q) & = q^{-1/4} e^{-
    \pi i (\mathscr{V}+\frac{1}{2})} \mathscr{V}_2 \left(\mathscr{V}+
  \frac{1}{2}, q\right)\\
    & = q^{-1/4} \frac{1}{i} e^{- \pi i \mathscr{V}} \mathscr{V}_2
  \left(\mathscr{V}+ \frac{1}{2}, q\right)\\
  & = q^{1/4} e^{- \pi i \mathscr{V}} \sum^\infty_{n=- \infty} (-)^n
  q^{\left( \frac{2n+1}{2}\right)^2} e^{(2n+1) \pi i \mathscr{V}}\\
  & = \frac{2}{i} \cos \pi \left(\mathscr{V}+ \frac{1}{2}\right)
  e^{-\pi i \mathscr{V}}\\ 
  & \hspace{2cm}\prod^\infty_{n=1} (1- q^{2n})\left(1- q^{2n} e^{2
    \pi i \mathscr{V}}\right)\times  \left(1 - q^{2n} e^{-2 \pi i
    \mathscr{V}}\right)  
\end{align*}

Now define
$$
\mathscr{V}_1 (\mathscr{V}, q)= \mathscr{V}_2 \left(\mathscr{V}+
\frac{1}{2}, q\right),
$$
or
\begin{align*}
  \mathscr{V}_1 (\mathscr{V}, q) & = 2q^{1/4} \sin \pi \mathscr{V}
  \prod^\infty_{n=1} (1- q^{2n}) (1+ q^{2n} e^{2 \pi i \mathscr{V}})
  (1- q^{2n} e^{- 2 \pi i \mathscr{V}})\\
  & = i q^{-1/4} \sum^\infty_{m=-\infty} (-)^n q^{\left(\frac{2n+1}{2}
    \right)^2} e^{(2n+1)\pi i \mathscr{V}}
\end{align*}

Collecting\pageoriginale together we have the four
$\mathscr{V}$-functions:
\begin{align*}
  \mathscr{V}_1 (\mathscr{V}, q) & = i q^{-1/4} \sum^\infty_{m=-\infty}
  (-)^n q^{\left( \frac{2n+1}{2}\right)^2} e^{(2n+1)\pi i \mathscr{V}}\\
  & = \sum^\infty_{n=0} (-)^n q^{\left( \frac{2n+1}{2}\right)^2} \sin
  (2n+1) \pi \mathscr{V}\\
  \mathscr{V}_2 (\mathscr{V}, q) & = 2 \sum^\infty_{n=0}
    q^{\left(\frac{2n+1}{2} \right)^2} \cos (2n+1) \pi \mathscr{V}\\
  \mathscr{V}_3 (\mathscr{V}, q) & = 1+ 2 \sum^\infty_{n=1} q^{n^2}
  \cos 2 n \pi \mathscr{V}\\
  \mathscr{V}_4 (\mathscr{V}, q) & = 1+ 2 \sum^\infty_{n=1} (-)^n
  q^{n^2} \cos 2 \pi n \mathscr{V}
\end{align*}

Observe that the sine function occurs only in $\mathscr{V}_1$. Also if
$q, \mathscr{V}$ are rel these reduce to trigonometric expansions. 
