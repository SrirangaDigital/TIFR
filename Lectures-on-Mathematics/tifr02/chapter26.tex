\chapter{Lecture}\label{part3:lec26} %%% 26
\markboth{\thechapter. Lecture}{\thechapter. Lecture}

We\pageoriginale had the formula for $B_k (\nu)$:
$$
B_k (\nu) = \frac{1}{4} \sqrt{\frac{k}{3}} \sum_{\ell^2 \equiv \nu
  \pmod{24 k}} \left( \frac{\ell}{3}\right) \left( \frac{-1}{3}\right)
e^{\pi i \ell/6k},
$$
with $\nu \equiv 1  \pmod{24}$. Writing $d= (24, k^3)$, we had the
following cases:

\begin{enumerate}[1)]
\item $d=1$
  $$
  B_k (\nu) = \left( \frac{3}{k}\right) \sqrt{k} \sum_{(24 \pi)^2
    \equiv \nu \pmod{k}} e^{4 \pi i r/k}
  $$
\item $d=3$
  $$
  B_k (\nu) = i \left( \frac{-1}{k}\right) \sqrt{\frac{k}{3}}
  \sum_{(24 r)^2\equiv \nu \pmod{3k}} \left( \frac{-1}{r}\right)
  e^{\pi i r/2k}
  $$
\item $d=8$
  $$
  B_k (\nu) = \frac{1}{4i} \left( \frac{k}{3}\right) \sqrt{k}
  \sum_{(3r)^2 \equiv \nu \pmod{8k}} \left( \frac{-1}{r}\right) e^{\pi
  i r/2k}
  $$
\item $d=24$. We do not get anything new.
\end{enumerate}

Assume $k= k_1 k_2$, $(k_1, k_2)=1$. We desire to write $B_k
(\nu_1)$. $B_{k_2}(\nu_2)= B_k (\nu)$, with a suitable $\nu$ to be
found out from $\nu_1$ and $\nu_2$. It cannot be foreseen. It is a
multiplication of a peculiar sort. Two cases arise. 
\begin{enumerate}[(i)]
\item  At least one
    of $k_1$, $k_2$ is prime to 24 and therefore to\pageoriginale 6, say
    $(k_1, 6)=1$.
\item None is prime to 6. But since $(k_1, k_2)= 1, 2/k_1,
  3/k_1$. Under the circumstances prevailing these are the two
  mutually exclusive cases.
\end{enumerate}

\setcounter{case}{0}
\begin{case}\label{part3:lec26:case1}
  Utilise $d=1$.
  \begin{align*}
    B_{k_1} (\nu_1) \cdot B_{k_2} (\nu_2) & = \left(
    \frac{3}{k_1}\right) \sqrt{k_1} \cdot \frac{1}{4} \sqrt{\frac{k_2}{3}}
    \sum_{(24 r)^2 \equiv \nu_1 \pmod{k_1}} e^{4 \pi i r/k_1}\\ 
    & \hspace{3cm} \cdot
    \sum_{\ell^2 \equiv \nu_2 \pmod{24 k_2}} \left(
    \frac{\ell}{3}\right) \left( \frac{-1}{\ell}\right) e^{\pi i
      \ell/6 k_2}\\
    & = \frac{1}{4} \left( \frac{3}{k_1}\right) \sqrt{\frac{k_1
        k_2}{3}}  \mathop{\sum\sum}_{\substack{(24r)^2\equiv \nu_1
    \pmod{k_1}\\ \ell^2 \equiv \nu_2 \pmod{24 k_2}}} e^{\frac{\pi
        i}{6k_1 k_2} (24 k_2 r+ k_1 l)} \left(\frac{l}{3}\right) \left(
      \frac{-1}{l}\right)
  \end{align*}
  $k_1$ and $24 k_2$ are coprime moduli. If $r$ runs modulo $k_1$ and
  $\ell$ runs modulo $24 k_2$, $24 k_2 r+k_1 \ell$ would then run
  modulo $24 k_1 k_2$.
\end{case}

Write
$$
24 k_2 r + k, \ell \equiv t \pmod{24 k_1 k_2}
$$

Then 
\begin{align*}
  t^2 & = (24 k_2 + k_1 \ell)^2 \equiv (24 k_2 r)^2 \pmod{k_1}\\
  & \equiv k^2_2 \nu_1 \pmod{k_1}, ~\text{since}~ (24 r)^2 \equiv
  \nu_1 \pmod{k_1}
\end{align*}

Similarly
\begin{align*}
  t^2 & \equiv (k_1 \ell)^2 \pmod{24 k_2}\\
  & \equiv k_1^2 \nu_2 \pmod{24 k_2}, \quad \text{since}~\ell^2 \equiv \nu_2
  \pmod{24 k_2}.
\end{align*}

So\pageoriginale in order to get both conditions of summation, we need only choose
$t^2 \equiv \nu \pmod{24k_1 k_2}$; and this can be done by the Chinese
remainder theorem. So
$$
B_{k_1} (\nu_1) B_{k_2} (\nu_2) = \frac{1}{4} \left(\frac{3}{k},
\right) \sqrt{\frac{k}{3}}  \sum_{t^2 \equiv \pmod{24 k, k_2}}
\left( \frac{\ell}{3}\right) \left( \frac{-1}{\ell}\right) e^{\pi i t/
6k} 
$$

This already looks very much like the first formula though not
quite. What we have in mind is to compare it with
$$
B_k (\nu) = \frac{1}{4} \sqrt{\frac{k}{3}} \sum_{t^2 \equiv \nu
  \pmod{24 k}} \left(
  \frac{t}{3}\right)\left( \frac{-1}{t}\right) e^{\pi i t/6k}  
$$

So find out
\begin{align*}
  \left( \frac{t}{3}\right) \left( \frac{-1}{t}\right) & = \left(
  \frac{24k_2 r+ k_1\ell}{3}\right) \left( \frac{-1}{24 k_2 r + k_1
    \ell}\right)\\
  & = \left( \frac{k_1\ell}{3}\right) \left(
  \frac{-1}{k_1\ell}\right)\\
  & = \left( \frac{k_1}{3}\right)\left( \frac{-1}{k_1}\right) \left(
  \frac{\ell}{3}\right) \left( \frac{-1}{\ell}\right)\\
  & = \left( \frac{3}{k_1}\right) \left( \frac{\ell}{3}\right) \left(
  \frac{-1}{\ell}\right), 
\end{align*}
by the reciprocity law. So the formulas agree: $B_{k_1} (\nu_1)
B_{k_2} (\nu_2)= B_k (\nu)$; and we have settled the affair in this
case by

\setcounter{thm}{0}
\begin{thm}\label{part3:lec26:thm1}
  If\pageoriginale $k_2^2 \nu_1 \equiv \nu \pmod{k_1}$ and $k_1^2 \nu_2 \equiv
  \nu \pmod{24 k_2}$, $(k, 6)=1$, then 
  $$
  B_{k_1} (\nu_1) B_{k_2}(\nu_2)= B_{k_1 k_2}(\nu)
  $$
\end{thm}

\begin{case}\label{part3:lec26:case2}
  This corresponds to $d=d_1=8$ and $d= d_2=3$.
  \begin{align*}
    B_{k_1}(\nu_1) \cdot B_{k_2}(\nu_2) & = \frac{1}{4} \left(
    \frac{k_1}{3}\right) \sqrt{k_1} \left( \frac{-1}{k_2}\right)
    \sqrt{\frac{k_2}{3}}\\
    & \qquad \sum_{(3r)^2 \equiv \nu_1 \pmod{8k_1}} \left( \frac{-1}{r}\right)
    e^{\pi i r/2k_1} \sum_{(8r)^2 \equiv \nu_2 \pmod{3k_2}} e^{\pi i s
    /3k_2}.\\
    & = \frac{1}{4} \left( \frac{k_1}{3}\right) \left(
    \frac{-1}{k_2}\right) \sqrt{\frac{k_1 k_2}{3}}\\
    & \qquad \mathop{\sum\sum}_{\substack{(3r)^2 \equiv \nu_1
      \pmod{8k_1}\\ (8r)^2 \equiv \nu_2 \pmod{3k_2}}} \left(
    \frac{-1}{3}\right) \left( \frac{s}{3}\right) e^{\frac{\pi i}{6
        k_1 k_2}(3 k_2 r + 8 k_1 s)}
  \end{align*}
\end{case}

Since $(k_1, k_2)=1$, $(8k_1, 3k_2)=1$ and so $3k_2 r + 8k_1 s=t$ runs
through a full system of residues modulo $24k_1 k_2$. So
$$
B_{k_1} (\nu_1) B_{k_2} (\nu_2) = \frac{1}{4} \left(
\frac{k_1}{3}\right)\left( \frac{-1}{k_2}\right)  \sqrt{\frac{k}{3}}
\sum_{t^2 \equiv \nu \pmod{24 k_1 k_2}} \left( \frac{-1}{r}\right)
\left( \frac{s}{3}\right)  e^{\pi i t/(6k_1 k_2)}
$$

As\pageoriginale before
\begin{align*}
  t^2 & = (3 k_2 r + 8 k_1 s)^2 \equiv (3 k_2 r)^2 \equiv (3k_2 r)^2
  \equiv k_2^2 \nu_1 \pmod{8k_1}\\
  t^2 & = (8k_1 s)^2 \equiv k_1^2 \nu_2 \pmod{3k_2} 
\end{align*}

Now determine $\nu$ such that $\nu \equiv k_2^2 \nu_1 \pmod{8k_1}$ and
$\nu \equiv k_1^2 \nu_2 \pmod{3k_2}$, again by the Chinese remainder
theorem. So $t^2 \equiv \pmod{24 k_1 k_2}$. Now
\begin{align*}
  \left( \frac{t}{3}\right)\left( \frac{-1}{t}\right) & = \left(
  \frac{8k_1 s}{3}\right) \left( \frac{-1}{3k_1r}\right)\\
  & = \left( \frac{k_1}{3}\right)\left( \frac{-1}{k_2}\right)\left(
  \frac{s}{3}\right) \left( \frac{-1}{r}\right) 
\end{align*}
(since 8 and $-1$ are quadratic non-residues modulo 3). So
\begin{align*}
  B_{k_1} (\nu_1) B_{k_2} (\nu_2) & = \frac{1}{4} \sqrt{\frac{k}{3}}
  \sum_{t^2 \equiv \nu \pmod{24k}} \left( \frac{t}{3}\right)  \left(
  \frac{-1}{t}\right) e^{\pi i t/6k}\\
  & = B_k (\nu)
\end{align*}
where $\nu$ is given. Hence

\begin{thm}\label{part3:lec26:thm2}
  If $k_2^2 \nu_1 \equiv \nu \pmod{8k_1}$ and $k_1^2 \nu_2 \equiv \nu
  \pmod{3k_2}$, then 
  $$
  B_{k_1} (\nu_1) B_{k_2}(\nu_2) = B_{k_1 k_2} (\nu)
  $$
\end{thm}

Let us give an example of what this is good for. Calculate
$A_{10}(26)$. Since we can reduce modulo 10, $A_{10}(26)= A_{10}(6)$. 
\begin{align*}
  \nu & = 1- 24n =- 143.\\
  A_{10}(26) & = A_{10}(6) = B_{10}(-143)= B_{10} (-23)\\
  & = B_5 (\nu_1) B_2 (\nu_2)
\end{align*}
where\pageoriginale $\nu_1$, $\nu_2$ are determined by the conditions
$$
\displaylines{\hfill 
  4 \nu_1 \equiv - 23 \pmod{5} ~\text{or}~ - \nu_1 \equiv -3 \pmod{5}
  \hfill \cr
  \text{and} \hfill 25 \nu_2 \equiv -23 \pmod{48} ~\text{or}~ \nu_2
  \equiv 1 \pmod{48} \hfill }
$$ 

So $A_{10} (26)= B_5 (3) B_2 (1)$, and these are explicitly
known. Since $ \left( \frac{3}{5}\right)=-1$, $B_5 (3)=0$. It is
actually not necessary now to calculate $B_2(1)$. 
$$
\displaylines{\hfill B_2 (1) = (-)^\lambda \left( \frac{-1}{r}\right)
  2^{\lambda/2} \sin \frac{\pi r}{2^{\lambda +1}} \hfill \cr
  \text{where} \hfill (3r)^2 \equiv \nu \pmod{2^{\lambda+3}}, (3r)^2
  \equiv 1 \pmod{16}, \hfill } 
$$
or $3r\equiv 1 \pmod{16}$, $r \equiv 11 \pmod{16}$. (there
being four solutions). Then
\begin{align*}
  B_2 (1) & = (-) (-) \sqrt{2} \sin \frac{11 \pi}{4} = 1 \times
  \sqrt{2} \cdot \frac{1}{\sqrt{2}} =1\\
  A_{10} (26) & =0.
\end{align*}

One more thing can be established now. We have the inequalities:
\begin{align*}
  |B_{2^\lambda} (\nu)| &\leq 2^{\lambda/2},\\
  |B_{3^\lambda} (\nu)| & \leq 3^{\frac{\lambda}{2}} 2\sqrt{3},\\
  |B_{p^\lambda} (\nu)| & \leq 2p ^{\frac{\lambda}{2}}, p > 3.
\end{align*}

By\pageoriginale the multiplicative property,
$$
\displaylines{\hfill 
  |B_k(\nu)|= |A_k (\nu)| \leq \sqrt{k} (2 \sqrt{3})^{\lambda(k)} \hfill
  \cr
  \text{where} \hfill \lambda(k) = \sum_{p\mid k}1. \hfill }
$$

This is a rough appraisal, but $\lambda(k)$ is in any case a small
number. So
$$
|B_k (\nu) < C \sqrt{k} \cdot k^\epsilon, \epsilon > 0, C= C_\epsilon.
$$

We see that although $A_n (n)$ has $\varphi(k)$ summands and in
general all that one knows is that $\varphi (k)\leq k-1$, because of
strong mutual cancellations among the roots of unity, the order is
brought down to that of $k^{\frac{1}{2}+ \epsilon}$. This reminds us of
other arithmetical sums like the Gaussian sums and the Kloosterman
sums. 
