\chapter{Lecture}\label{part4:lec42} %%% 42
\markboth{\thechapter. Lecture}{\thechapter. Lecture}

The\pageoriginale error term in the formula for the number of
representations of $n$ as the sum of $r$ squares, $r \geq 5$, was
$O (n^{r/4})$. For $r=4$ this did not suffice. We shall
therefore study the problem by Kloosterman's method and find out what
happens when we want to decompose $n$ in the form $n= n_1^2+ n_2^2 +
n_3^2 + n_4^2$. We shall see that we can diminish the order in the
error term by nearly $\frac{1}{18}$. When Kloosterman did this for the
first time (Act a Mathematic as 1927) he took a slightly more general
problem, that of representing $n$ in the form $n= an_1^2 + b_2^2 +
cn_3^2 +dn_4^2, a, b, c, d$ integers. This works nicely; we get the
singular series and an error term which is smaller than before. The
difficult not will be about the arithmetical interpretation. The
singular series will now be a difficult phenomenon; we shall have
multiplicativity, but the interpretation of the factors $\gamma_p$
becomes complicated. We shall content ourselves with the analytical
power of the discussion. The generating function which will have to be
discussed is quite clear:
$$
\displaylines{F(x) = \Theta (x^a) \quad \Theta (x^b) \quad \Theta
  (x^c) \quad \Theta (x^d)\cr
\text{where} \hfill \Theta (x) = \sum^\infty_{n=- \infty} x^{n^2}
\hfill } 
$$

And we will have 
$$
A_4 (n) = \frac{1}{2 \pi i} \int_C \frac{\Theta (x^a) \Theta(x^b)
  \Theta(x^c) \Theta(x^d)}{x^{n+1}} dx
$$
and the analysis goes on as before with Farey series.

We are here representing $n$ by a positive definite quadratic form
which is a diagonal form. Let us make the problem more general.

Let\pageoriginale us represent $n$ by a positive definite quadratic form with
integral coefficients. (We could very well ****** also the
`semi-integral' case). Let $S$ be a positive definite integral
symmetric matrix and \underline{$x$} a column vector with elements
$x_1, x_2, \ldots x_r$ in $r$-space. \underline{$x'$} is the
transposed row-vector. $\underline{x'} S \underline{x}$ is a quadratic
form in $r$ variables. The question is how often can we express an
integer $n$ by integer vectors with respect to this quadratic form in $r$
variables. 

The generating function to be studied this time is 
$$
F_r (x) = \sum_{\underline{n}} x^{\underline{n'} S\underline{n}}, |x|<1,
$$
the summation over all integral vectors $\underline{n}$. Convergence
is easily assured by positive definiteness. Indeed
$$
\underline{x'} S\underline{x} \geq C(x_1^2 + \cdots + x_r^2), C>0
$$

For $\underline{x'} S \underline{x}$ has a minimum $C> 0$ on $|x|=1$
by positive definiteness; the in-equality follows from the homogeneity
of the quadratic form. And $\sum x^{c(n_1^2 + \cdots + n_r^2)}$ is
trivially a product of convergent series.

In a later paper (Hamburger Abhandlungen, 1927) Kloosterman, on the
advice of Heeke, took up a more general problem. This would require a
little more preparation on modular forms. The generating function will
now be a modular form of dimension $- \frac{r}{2}$ of a certain
`stufe'; so we have to discuss modular forms not only with respect to
the full modular group, but also the substitutions
$$
\begin{pmatrix} a & b\\ c & d\end{pmatrix} \equiv 
\begin{pmatrix} 1 & 0\\ 0 & 1\end{pmatrix} \pmod{N},
$$
($N$\pageoriginale will the `stufe') which from a subgroup finite index
  in the modular group. Kloosterman's work goes through for all
  modular forms of this sort, but we should want generalisations of
  $\eta(\tau)$ and $\mathscr{V}(\tau)$. To do this we need a good deal
  of Heeke's theory about Eisenstein series of higher stufe of the
  type:
$$
\sum_{\substack{m_1\equiv a\pmod{N}\\m_2\equiv b \pmod{N}}}
\frac{1}{(m_1 + m_2 \tau)^r}
$$
which is a modular form of dimension $- \frac{r}{2}$ and stufe
$N$. These were investigated by Heeke in a famous paper (Hamburger
Abhandlungen 1927). Kloosterman could carry out his theory for these
also. We shall, however, compromise on the quadratic form.

We had the generating function
\begin{align*}
  F_r (x) & = \sum_{\underline{n}} x^{\underline{n}'S \underline{n}},
  |x|<1,\\
  & = 1+ \sum^\infty_{n=1} A_r (n) x^n.
\end{align*}

$F_r(x)$ is a modular form. This can be seen directly by the
transformation formulae. Let us start with Kloosterman's method and
see what happens. The problem is to get
$$
A_r (n) = \frac{1}{2 \pi i} \int_C \frac{F_n(x)}{x^{n+1}} dx
$$

At a certain moment later on we shall need a greater knowledge of $F_r
(x)$ 

Let us carry out the Farey dissection:
\begin{align*}
  x & = e^{2 \pi i \frac{h}{k} - 2 \pi \mathfrak{z}} = e^{2 \pi i
    \frac{h}{k} - 2 \pi (\delta_N - i \varphi)}\\
  A_r(n) & = \sum_{0 \leq h < k\leq N} e^{- 2\pi i \frac{h}{k} n}
  \int\limits^{\mathscr{V}''_{hk}}_{- \mathscr{V}'_{hk}} F_r (e^{2 \pi
  i \frac{h}{k} - 2 \pi \mathfrak{z}}) e^{2 \pi n \mathfrak{z} d
    \varphi} 
\end{align*}
with\pageoriginale $(h, k)=1$, $\mathscr{V}'_{hk}= \frac{1}{k(k_1+k)}$,
$\mathscr{V}''_{hk} = \frac{1}{k(k+ k_2)}$ where in the Farey
situation, $\frac{h_1}{k_1} < \frac{h}{k} < \frac{h_2}{k_2}$. The
refinement of Kloosterman consists in not merely making the rough
remark that
$$
\frac{1}{2kN} \leq \mathscr{V}'_{hk}, \mathscr{V}''_{hk} \leq
\frac{1}{k(N+1)}, 
$$
but in a finer following up of the number theoretical determination of
the adjacent fractions. We have
$$
\displaylines{ h_1 k - hk_1 =- 1, h k_2 - h_2 k=- 1;\cr
\text{i.e.,} \hfill hk_1\equiv 1 \mod k, hk_2 \equiv -1 \mod k\hfill} 
$$
$\frac{h}{k}$ is given. What we are worried about is, how long is its
environment. $k_1$ and $k_2$ are given as solutions of certain
congruences. We have the habit of calling $h'$ a number such that 
\begin{gather*}
  hh' \equiv -1 \pmod{k}; \quad \text{so let us write}\\
  k_1 \equiv -h' \pmod{k}, k_2 \equiv h' \pmod{k}
\end{gather*}

So we know in which residue class modulo $k$ $k_1$ and $k_2$ have to
lie. $k_1 + k$,\pageoriginale being the denominator of a mediant, had
to exceed  $N$. $N< k_1
+k \leq N + k$, or $N-k< k_1 \leq N$. So $k_1$ has a span of size
$k$. This along with $k_1 \equiv - h \mod{k}$ determines $k_1$
completely. Similarly, for $k_2$, $N-k< k_2 \leq N$ So there is no
uncertainty at all about $\mathscr{V}'_{hk}$, $\mathscr{V}''_{hk}$;
and we could single them out if we insisted on that.

For example, let $\frac{h}{k} = \frac{5}{9}$, $N=12$; what are the
neighbours? $\frac{h_1}{k_1}< 5/9< \frac{h_2}{k_2}$. First determine
$h'$. $5h'\equiv -1 \pmod{9}$ or $h'=7$. Then $12-9< k_1 \leq 12$ and
$k_1 \equiv - 7 \pmod{9}$, so $k_1=11$. Similarly $3 < k_2 \leq 12$,
$k_2\equiv 7 \pmod{9}$ so $k_2=7$. We need only $k_1$ and $k_2$; but
for our own enjoyment let us calculate $h_1$ and $h_2$.
$$
\begin{vmatrix}
  h, & 5\\
  11 & 9
\end{vmatrix}=- 1,
\begin{vmatrix}
  5 & h_2\\
  9 & 7
\end{vmatrix}=- 1,
$$
or $h_1 =6$, $h_2=4$, so that we have $\frac{6}{11} < \frac{5}{9} <
\frac{4}{7}$ as adjacent fractions in the Farey series of order 12. We
do not need to display the whole Farey series. 

Now utilise this in the following way.
$$
A_r (n) = \sideset{}{'}\sum_{o \leq h < k \leq N} e^{2 \pi i
  \frac{h}{k} n} \int\limits^{\frac{1}{k(k+k_2)}}_{-
    \frac{1}{k(k_1+k)}} F_r \left( e^{2 \pi i \frac{h}{k} - 2 \pi
    \mathfrak{z}}\right) e^{2 \pi n \mathfrak{z}} d \varphi
$$
Kloosterman does the following investigation. In any case we are sure
that $k_1$, $k_2$ can at most become $N$. If we take $k_1$ and $k_2$
big we have a small interval of integration. Since
\begin{gather*}
  k_1 + k < k_1+1 + k< \cdots < N+k,\\
  k_2 + k < k_2 +1 + k < \cdots < N+k,\\
  \frac{1}{k_1+k} > \frac{1}{N+k}, \frac{1}{k_2+k} > \frac{1}{N+k},
\end{gather*}\pageoriginale
so that the interval of integration should be at least as big as the
interval $-1/k(k+N)$ to $1/k(k+N)$. This interval is always present
whatever be $k_1$ and $k_2$. So $A_r(n)$ is equal to the always
present kernel
$$
\sideset{}{'}\sum_{0 \leq h < k \leq N} e^{2 \pi i
  \frac{h}{k} n} \int\limits^{\frac{1}{k(k+N)}}_{\frac{-1}{k(k+N)}}
(\cdots )d \varphi,
$$
with the possible additional terms
$$
\sideset{}{'}\sum_{0 \leq h < k \leq N} e^{-2 \pi i
  \frac{h}{k} n} \sum^{N-1}_{\ell =
  k_2}\int\limits^{\frac{1}{k(k+\ell)}}_{\frac{1}{k(k+\ell+1)}}
(\cdots) d \varphi 
\sum_{0 \leq h < k \leq N} e^{-2 \pi i
  \frac{h}{k} n} \sum^{N-1}_{\ell =
  k_1}\int\limits^{\frac{1}{k(k+\ell+1)}}_{\frac{-1}{k(k+\ell)}} 
(\cdots) d \varphi
$$

There is no doubt about the integrals. The limits are all
well-defined. This will help us to appraise certain roots of unity
closely-by the Kloosterman\pageoriginale sums.

We shall now return to the integrand; that is a $\mathscr{V}$-function
and requires the usual $\mathscr{V}$ treatment. Consider the $r$-fold
$\mathscr{V}$-series: 
$$
\Theta (t) = \sum_{\underline{n}} e^{- \pi t \underline{n}'
  S\underline{n}}, \mathscr{R}_e t> 0.
$$

Modify this slightly by introducing a vector $\alpha$ of real
numbers;\break  
$\underline{\alpha'} = (\alpha_1, \ldots , \alpha_r)$. Let 
$$
\Theta (t; \alpha_1, \ldots , \alpha_r)= \sum_{\underline{n}} e^{- \pi t
(\underline{n}' + \underline{\alpha}') S (\underline{n}+ \underline{\alpha})}
$$

This is periodic in $\alpha_j$, of period 1, and so permits a Fourier
expansion. The convergence is so good that the function is analytic in
each $\alpha_j$ and so we are sure that it is equal to the sum
$$
\sum_{\underline{m}} C(\underline{m}) e^{2 \pi i \underline{m}' \underline{\alpha}}
$$ 
where $C(\underline{m})$ is the Fourier coefficient:
\begin{align*}
  C(\underline{m})& = \int^1_0\cdots \int^1_0 \Theta (t; \beta_1,
  \ldots , \beta_r) e^{-2 \pi i \underline{m}' \underline{\beta}} d
  \beta_1 , \ldots d \beta_r\\
  & = \int^1_0 \cdots \int^1_0 \sum_{\underline{n}} e^{- \pi t
    (\underline{n}'+ \underline{\beta}') S(\underline{n}+
    \underline{\beta})} e^{-2 \pi i \underline{m}' \underline{\beta}}
  d \beta_1 , \ldots , d \beta_r\\
  &= \int^1_0 \cdots \int^1_0 \sum_{\underline{n}} e^{- \pi t
    (\underline{n}'+ \underline{\beta}') S(\underline{n}+
    \underline{\beta})} e^{-2 \pi i \underline{m}' (\underline{n}+
    \underline{\beta})}  d \beta_1 , \ldots , d \beta_r
\end{align*}\pageoriginale
which is an integral over the unit cube $W$, and so on translation
with respect to the vector $\underline{n}$, becomes
$$
\sum_{\underline{n}} \mathop{\int\cdots \int}_{W+ \underline{n}} e^{-
  \pi t (\underline{\mathscr{V}} S \underline{\mathscr{V}})} e^{-2 \pi
i \underline{m}' \mathscr{V}} d\underline{\mathscr{V}}_1 \cdots  d
\mathscr{V}_r
$$
(the exchange of integration and summation orders being trivial)
$$
=\int_{-\infty}^\infty \cdots \int^\infty_{- \infty} e^{- \pi t
  \mathscr{V}'D \mathscr{V}} e^{-2 \pi i \underline{m}'
  \underline{\mathscr{V}}} d \mathscr{V}_1 \ldots d \mathscr{V}_r. 
$$
