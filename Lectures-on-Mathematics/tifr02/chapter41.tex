\chapter{Lecture}\label{part4:lec41} %%% 41
\markboth{\thechapter. Lecture}{\thechapter. Lecture}

It\pageoriginale might be of interest to take $C_{r}(n)$, the main
term in the formula for $A_{r}(n)$ and make some remarks about it.
$$
C_r(n) = Q_r \sum_{d\mid n} (-)^{n+d+ \frac{r}{4} \left(\frac{n}{d} +1
  \right) d^{\frac{r}{2}-1}}
$$

Let us form the generating function
$$
H_r (x) = 1 + \sum^\infty_{n=1} C_r (n) x^n;
$$
this will give a sort of partial fraction decomposition. In the case
where $r\equiv 0 \pmod{8}$, it is simpler:
\begin{align*}
  H_r (x) & = 1+ Q_r \sum^\infty_{n=1} x^n \sum_{d\mid n} (-)^{n+d}
  d^{\frac{r}{2}-1}\\
  & = 1+ Q_r \sum^\infty_{n=1} (-x)^n \sum_{d\mid n} (-)^d
  d^{\frac{r}{2}-1}\\
  & = 1+ Q_r \sum^\infty_{d=1} (-)^d d^{\frac{r}{2}-1}
  \sum^\infty_{q=1} (-x)^{qd}\\
  & = 1+ Q_n \sum^\infty_{d=1} (-)^d d^{\frac{r}{2}-1}
  \frac{(-x)^d}{1-(-x)^d}\\
  & = 1+ Q_r \sum^\infty_{d=1} d^{\frac{r}{2}-1}
  \frac{x^d}{1-(-x)^d}\\
  & = 1+ Q_r \left\{\frac{1. x}{1+x} + 2^{\frac{r}{2}-1}
  \frac{x^2}{1-x^2}+ 3^{\frac{r}{2}-1} \frac{x^3}{1+x^3} + \cdots  \right\}
\end{align*}

This is a Lambert Series. Replacing $x$ by $e^{\pi i \tau}$, it
becomes 
$$
1+ Q_r \left\{ \frac{e^{\pi i \tau}}{1+e^{\pi i \tau}}+
2^{\frac{r}{2}-1} \frac{e^{2 \pi i \tau}}{1- e^{2 \pi i \tau}} +
\cdots \right\}
$$

The\pageoriginale series above can be transformed into an Eisen stein series. If $r$ is
taken to be 8, it is actually the 8th power of the
$\mathscr{V}-$function 

Next, take $r \equiv 4 \pmod{8}$
\begin{align*}
  G_r(x) & = 1+ Q_r \sum^\infty_{n=1} x^n \sum_{d\mid n} (-)^{n+d+
    \frac{n}{d}+1} d^{\frac{r}{2}-1}\\
  & = 1- Q_r \sum^\infty_{d=1} (-)^d d^{\frac{r}{2}-1} \sum_{d\mid n}
  (-x)^n (-)^{n/d}\\
  & = 1- Q_r \sum^\infty_{d=1} (-)^d d^{\frac{r}{2}-1}
  \sum^\infty_{q=1} (-x)^{qd} (-)^q\\
  & = 1+ Q_r \sum^\infty_{d=1} (-)^d d^{\frac{r}{2}-1}
  \frac{(-x)^d}{1+(-x)^d}\\
  & = 1+ Q_r \sum^\infty_{d=1} d^{\frac{r}{2}-1} \frac{x^d}{1-
    (-x)^d}\\
  & = 1+ Q_r \left\{\frac{1 \cdot x}{1-x} + 2^{\frac{r}{2}-1}
  \frac{x^2}{1+x^2} + 3^{\frac{r}{2}-1} \frac{x^3}{1-x^3} + \cdots \right\}
\end{align*}

This is again a Lambert Series. This shows that a $\mathscr{V}-$power
has to do with Lambert series which appears as an evaluation of
certain Eisenstein series not that they are identical.

We now go to something quite different. We had for $r \geq 5$,
\begin{equation*}
  A_r (n) \thicksim \frac{\pi^{r/2}}{\Gamma\left(\frac{r}{2} \right)}
  n^{\frac{r}{2}-1} S_r (n) \tag{*}\label{part4:lec41:eq*}
\end{equation*}

This\pageoriginale comes out as a nice formula. Now could we not
make some sense out of this formula? What is its inner meaning? We
shall show that the first factor $\left( \pi^{r/2}/ \Gamma
(r/2)\right)n^{\frac{r}{2}-1}$ gives the average value of the number
of representations of $n$ as the sum of $r$ squares; the second factor
also is an average, in the $p$-adic measurement. We shall show that
$$
\sum_{n \leq x} A_r (n) \thicksim \frac{\pi^{r/2}}{\Gamma (r/2)}
\sum_{n \leq x} n^{\frac{r}{2}-1}
$$

So for each individual $n$, $S_r (n)$ gives the deviation of $A_r(n)$
from \break $\left(\pi^{r/2}/\Gamma(r/2) \right)n^{\frac{r}{2}-1}$; but on
the average there is no deviation. 

Let us first look at $\sum_{n \leq x} A_r (n)$.
\begin{align*}
  \sum_{n \leq x} a_r (n) & = \sum_{n \leq x} \quad \mathop{\sum\quad
    1}_{m_1^2 + \cdots m_r^2=n}\\
  & = \mathop{\sum\quad 1}_{m_1^2 + \cdots + m_r^2 \leq x},
\end{align*}
which is the number of lattice-points in the $r$-sphere with centre at
the origin and radius $\sqrt{x}$, and so is proportional asymptotically
to a certain volume (because the point lattice has cells or volume 1
and to each points belongs a cell). So this is roughly the volume of
the sphere of radius $\sqrt{x}$ which is 
\begin{align*}
  & \mathop{\int \cdots\int}_{x_1^2 + \cdots + x_n^2 \leq x} dx_1 \cdots
  dx_r\\
  & = \frac{\pi^{r/2}}{\Gamma(r/2)}x^{r/2}
\end{align*}\pageoriginale

The difference will not be zero but of the order of magnitude of the
surface of the sphere, i.e., $O \left( x^{r/2}-1\right)$

Now consider the other side.
\begin{align*}
  \frac{\pi^{r/2}}{\Gamma (r/2)} \sum_{n \leq x} n^{\frac{r}{2}-1} &
  \thicksim \frac{\pi^{r/2}}{\Gamma(r/2)} \int_0^x
  \mathscr{V}^{\frac{r}{2}-1} d \mathscr{V}\\
  & = \frac{\pi^{r/2} x^{r/2}}{\Gamma\left(\frac{r}{2} +1 \right)} 
\end{align*}

So the first factor on the right of (\ref{part4:lec41:eq*}) gives the average. $S_r(n)$
has to be adjusted. $S_r (n)$ is also, surprisingly, an average. It
was defined as 
$$
S_r (n) = \gamma_2 (n) \gamma_3 (n) \gamma_5 (n) \cdots \gamma_p (n)
\cdots, 
$$
and $\gamma_p (n)$ in turn was given by\pageoriginale
\begin{align*}
  \gamma_p (n) & = 1+ \sum^\infty_{\lambda=1} V_{p^\lambda} (n)\\
  & = 1+ \sum^\infty_{\lambda=1} \frac{1}{p^{\lambda r}}
  \sum_{\substack{h \mod p^\lambda\\p \nmid h}} G(h, p^\lambda)^{r}
  e^{- 2 \pi i \frac{h}{p^\lambda}n} \frac{1}{p^{\lambda r}}
    \sum_{\substack{h \mod p^\lambda\\p \nmid h}} G(h, p^\lambda)^r
    e^{-2 \pi i \frac{h}{p^\lambda}n}\\
    & = \frac{1}{p^{\lambda r}} \sum_{\substack{h \mod p^\lambda\\p
        \nmid h}} \left\{ \sum_{\ell_1 \mod p^\lambda} e^{2 \pi i
      \frac{h}{p^\lambda}\ell_1} \sum_{\ell_2 \mod p^\lambda} e^{2 \pi
        i \frac{h}{p^\lambda} \ell_2}\right.\\
    & \hspace{5cm}\left.\sum_{\ell_r \mod p^\lambda}
      e^{2 \pi i \frac{h}{p^\lambda} \ell r^2} \right\} e^{- 2\pi i
            \frac{h}{p^\lambda}n}\\
    & =\frac{1}{p^{\lambda r}} \sum_{\ell_1 , \ldots , \ell_4 \mod
        p^\lambda}~ \sum_{h \mod p^\lambda} e^{2 \pi i
        \frac{h}{p^\lambda}}(\ell_1^2 + \cdots + \ell_r^2 -n)\\
    & = \frac{1}{p^{\lambda r}} \sum_{\ell_1, \ldots , \ell_r \mod
        p^\lambda}\\
    & \qquad \left\{ \sum_{s \mod p^\lambda} e^{2 \pi i \frac{s}{p^\lambda}
          (\ell_1^2 + \cdots + \ell_r^2 -n)}- \sum_{t \mod p^{\lambda
            -1}} e^{2 \pi i \frac{t}{p^{\lambda -1}} (\ell_1^2 +
          \cdots + \ell_r^2 -n)}\right\}\\
    & = \frac{1}{p^{\lambda r}} \sum_{\ell_1, \ldots \ell_n \mod
        p^\lambda} ~\sum_{s \mod p^\lambda} e^{2 \pi i
        \frac{s}{p^\lambda} (\ell_1^2 + \cdots + \ell_r^2 -n)}\\
    & \hspace{2cm} - \sum_{\ell_1, \ldots , \ell_r \mod p^{\lambda -1}}
      \sum_{t \mod p^{\lambda -1}} e^{2 \pi i \frac{t}{p^{\lambda -1}}
        (\ell_1^2 + \cdots + \ell_2^2 -n)}\\
    & = W_{p^\lambda} (n) - W_{p^\lambda-1} (n), ~\text{say},\\
    \therefore \quad & \quad 1+ V_p (n)+ V_{p^2}(n) + \cdots + V_{p^\lambda} (n) =
    W_{p^\lambda} (n) \to \gamma_p (n)       
\end{align*}

So\pageoriginale for $\lambda$ large enough $W_{p^\lambda}(n)=
V_{p^\lambda}(n)$: the partial sums get identical. The value of
$\lambda$ for which this occurs depends on the structure of $n$, on
how many primes that specific $n$ contains. Now 
\begin{align*}
  & \sum e^{2 \pi i \frac{s}{p^\lambda} (\ell_1^2 + \cdots +
  \ell_r^2-n)} = 0 ~\text{or}~p^\lambda \\
  \therefore \quad & \quad W_{p^\lambda}(n) =
  \frac{p^\lambda}{p^{\lambda r}} \sum_{\substack{\ell_1, \ldots ,
      \ell_r \mod p^\lambda\\\ell_1^2 + \cdots + \ell_r^2 \equiv n
      \pmod{p^\lambda}}}
\end{align*}

The sum on the right gives the number of times the congruence \break
$\ell_1^2 + \cdots + \ell_r^2 \equiv n \pmod{p^\lambda}$ can be
solved, $N_{p^\lambda} (n)$, say. Then
$$
W_{p^\lambda}(n) = \frac{1}{p^{\lambda(r-1)}} N_{p^\lambda}(n) 
$$

We have therefore divided the number of solutions of the congruence by
$p^{\lambda(r-1)}$. Now how many $\ell_1, \ldots , \ell_r \mod
p^\lambda$ do we have? There are $p^{\lambda r}$ possibilities
discarding $n$. $n$ is one of the numbers modulo $p^\lambda$. So
dividing by $p^r$, the average number of possibilities is $p^{\lambda
  (r-1)}$. Hence $\dfrac{N_{p^\lambda}(n)}{p^{\lambda (r-1)}}$ is the
  average density modulo $p^\lambda$ of the solution of the
  congruence. And since the $W_{p^\lambda} (n)$ eventually becomes
  $\gamma_p (n)$, each factor $\gamma_p (n)$ acquires a density
  interpretation, viz. the $p$-adic density of the lattice points
  modulo $p^\lambda$.

