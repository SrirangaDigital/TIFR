\chapter{Vanishing theorems for hermitian manifolds}\label{chap1}%%% chapter 1

\section{The space $\mathcal{L}^{p,q}$}%%% 1
 
 All\pageoriginale manifolds considered are assumed to be connected
 and  paracompact. 
 
 Let $\pi :E \rightarrow X$ be a holomorphic vector bundle on a
 paracompact  complex manifold $X$ of complex dimension $n$. 
 
 Let $\MG=\big\{{U_i}\big\}_{i \epsilon I}$  be a locally
 finite open coordinate covering of $X$ and  
 
 $\big\{e_{ij}:U_i \cap U _j \rightarrow GL (m\mathbb{C})
\big\}_{ij \in I \times I}$  
 transition functions defining $E$. A hermitian metric along the fibres
 of $E$ is a collection $\big\{h_i :U_i \rightarrow GL
 (m, \mathbb{C})\big\}_{i \epsilon I }$ of $C^{\infty}$
 -maps such that for $x \epsilon U_i, h_i (x)$ is a positive definite
 hermitian matrix and for $x \epsilon U _i \cap U _j, h_j (x)=
 ^{t}\overline {e_{ij}}(x) h_i (x) e_{ij}(x)$.  
 
\begin{lemma}%%% 1.1
  Every holomorphic vector bundle on a complex manifold  admits a
  hermitian metric.  
\end{lemma}
 
 \begin{proof}
   Let $\big\{ \rho k\big\}_{k \in I}$ be a $C^ \infty$-partition of
   unity subordinate to $\big\{U_k\big\}_{k 
     \in I} (= \MG)$. Let $\big\{h_i^o :U_i
   \rightarrow GL (m, \mathbb{C})\big\}_{i \in I}$  be any
   family of $C^\infty$-functions such that for $i \in I$, $x \in
   U_i$, $h^o_i (x)$ is a positive definite  hermitian matrix. Then
   the family $\big\{ h^o_i \big\}_{i\in I}$ defined by 
   $$
    h_i (x) = \sum\limits_k \rho_k(x) {}^{t}
    \overline{e_{ki}}(x)h^o_k(x)e_{ki}(x) 
   $$
   is a metric along the fibres of $E$.
 \end{proof}
 
In particular,\pageoriginale the holomorphic tangent bundle
$\mycirc{H} \to X $ admits a hermitian metric. Let
$\MG = \big\{U_i\big\}_{i  \in I}$ be a covering of $X$ by
means of coordinate open sets. Let $Z^1_i,\ldots ,Z^n_i$  be any system of
coordinates in $U_i$. Then $\mycirc{H}\rightarrow X $ is
defined with respect to $\MG$ by means of the transition
functions $J_{ij} :U_i \cap U_j \rightarrow GL
(n, \mathbb{C})$ defined by  
$$ 
J_{ij}(x)= \left( \frac{\partial Z^\alpha_j}{ \partial Z^\rho_l
  \beta }(x)\right)_{\alpha \beta} \text{ for  } x \in U _i \cap
U_j. 
$$ 

A hermitian metric on $\mycirc{H}$ is then a family of $
C^\infty$-hermitian -positive definite matrix-valued functions $g_i :
U_i \rightarrow GL (n, \mathbb{C})$ such that  
$$
g=^t\overline J_{ij} g_i J_{ij}
$$
Identifying hermitian matrices with hermitian bilinear forms on the
tangent spaces through the basis $\dfrac{\partial}{\partial Z^1} \ldots
\dfrac{\partial}{\partial Z^n}$ of the holomorphic tangent space, we
may write $g_i$ as 
$$
\sum\limits_{\alpha \beta } g_{i  \alpha \bar{\beta}}
dZ_i^\alpha  d \bar{Z}^\beta_\alpha.
$$

A hermitian metric on $X$ defines on $X$ (regarded as a differentiable
manifold), a Riemannian metric: in fact, if $z^\alpha_i =x^\alpha_i
+iy^\alpha_i$, then we have  
$$
ds^2 =\sum g_{i \alpha \overline\beta} \; dZ^\alpha_i \cdot dI^\beta_i
= \sum Re \quad g_{i \alpha \overline \beta}(dx^\alpha_i
dx^\beta_i + dy^\alpha_i dy^\beta_i)
$$ 
in the coordinate open set $U_i$, in terms of the local real
coordinates $x_i,y_i$. Since $X$ carries a complex structure, it
has a canonical orientation defined by this structure. Hence $X$ has a
canonical structure of an oriented Riemannian manifold. We denote by
$\ell$ the volume form on $X$ with respect to this oriented Riemannian
structure. 

Let\pageoriginale $C^r(X)$, $0\leq r \leq 2n$ ($n$= complex dimension of
$X$)  denote the 
space of complex valued exterior differential forms of degree $r$.
It is a module over the algebra, $F$, (over the complex
numbers) of complex valued $C^\infty$ function on $X$ 

\begin{definition}%%% 1.1
  The ``star operator'' $* :C^r(X)
  \rightarrow C^{m-r}(X)$  on an oriented Riemannian manifold $X$ of (real)
  dimension  $m$ is defined by the formula  
$$
(* \varphi)(t_1,\ldots , t_{m-r}) . \ell=\varphi\wedge \tau (t_1)\wedge
  \ldots \wedge \tau (t_{m-r)}
$$ 
where $\varphi \in C^r (X), t_1,\dots ,t_{m-r}$  are tangent vectors
  to $X$ at a point $P$ and $\tauup$ is the canonical isomorphism of the
  $F$-module of tangent vector fields on $X$ onto the $F$-module of
  1-forms. Clearly * is a real operator,
  i.e. $\overline{*\varphi}=\overline{* \varphi }$ (the bar denoting
  conjugation of complex numbers). Furthermore we have 
\end{definition}

\begin{lemma}%%% 1.2
  Let $(X, g)$ be an oriented Riemannian
  manifold of dimension $m$. Then 
  \begin{equation}
    **\varphi =(-1)^{r(m-1)}\varphi  \text{ for } \varphi \in  C^r
    (X).    \tag{1 \; I} 
  \end{equation}
\end{lemma}

\begin{proof}
  (\cite{key18}) Let $x \in X $ and $t_1,\ldots, t_m $ be an
  orthonormal basis of the tangent space at $x$ with respect 
  to the Riemannian metric. Then it is enough to check that 
  $$
  **(\tau(t_{i_1})\wedge \ldots \wedge \tau
  (t_{ir})) = (-1)^{r(m-1)} \tau (t_{i_1})\wedge \ldots \wedge \tau
  (t_{i_r}) 
  $$
  for every $0< i_1 < i_2 < \ldots < i_r \leq m$, or again that for every
  sequence  $0<j_1 < \ldots < j_r \leq m$,   
\begin{multline*}
  (**) \tauup\tau (t_{i_1}) \wedge \ldots \wedge \tauup (t_{i_r})(t_{j_1},
  \ldots t_{j_r})\\ 
  =\bigg\{\tauup (t_{i_1}) \wedge \ldots \wedge \tauup (t_{i_r})\bigg\}
  (t_{j_1},\ldots t_{j_r})\cdot (-1)^{r(m-1)}.
\end{multline*}
\end{proof}

Suppose\pageoriginale now that $\quad (j_1,...,j_r) \neq
(i_1,...,i_r)$. \quad then   
\begin{multline*}
  (**)(\tauup(t_{i_1})\wedge ..., \wedge\tauup(t_{i_r})(t_{j{_1}},...,t_{j_r}).l \\
  =*\left\{\tauup(t_{i_1})\wedge... \wedge \tauup(t_{i_r})\right\} \wedge
  \tauup (t_{j_1}) \wedge \dots \wedge \tauup (t_{j_r})
\end{multline*}
and the right hand side is zero because,
\begin{multline*}
  \left\{\tauup(t_{i_1})\wedge \dots \wedge \tauup(t_{i_r})\right\}
  (t_{\lambda_1 },\dots, t_{\lambda_{m-r}})l\\ 
  =\tauup(t_{i_1})\wedge...\wedge \tauup(t_{i_r})\wedge
  \tauup(t_{\lambda_1})\wedge...\wedge \tauup(t_{\lambda_{m-r}}) 
\end{multline*}
and here the right side is zero unless 
$\big\{i_1,..,i_r,\lambda_1,..,\lambda_{m-r}\big\}=\big\{1,2,...,m\big\}$.

$t_1,...,t_m$  are so chosen that $\tauup(t_1)\wedge...\wedge
\tauup(t_m)$ give the natural orientation, 
$$
\displaylines{\hfill  
  *(\tauup(t_{i_1})\wedge...\wedge
  \tauup(t_{i_r}))(t_{\lambda_1},..., t_{\lambda_{m-r}})=0\hfill \cr 
        \text{if } \hfill (i_1,...,i_r,
        \lambda_1,...,\lambda_{m-r})\neq (1,2,..m) \hspace{2cm}
        \text{and}\hfill }
$$
\begin{align*}
  *(\tauup(t_{i_1})\wedge ... &\wedge
  \tauup(t_{i_r}))(t_{\lambda_1},...t_{m-r}) l= \\ 
  & =\tauup(t_{i_1})\wedge \tauup(t_{i_r})\wedge
  \tauup(t_{\lambda_1})...(t_{\lambda_{m-r}}). 
\end{align*}

Now, the right side is clearly $\varepsilon \cdot l$ where $\varepsilon$ is
the signature of the permutation 
$$
(1,2,...,n)\rightsquigarrow(t_{i_1},...,t_{i_r},t_{\lambda_1},
... t_{\lambda_{m-r}}).  
$$

It follows that
$$
*(\tauup(t_{i_1})\wedge...\wedge \tauup(t_{i_r}))=\varepsilon
\tauup(t_{\mu_1})\wedge...\wedge \tauup(t_{\mu_{m-r}})
$$
where\pageoriginale $0 < \mu_1 < \ldots < \mu_{m-r}\le m$ are so chosen that
$$
\big\{t_{i_1},...,t_{i_r},
t_{\mu_1},..,t_{\mu_{m-r}}\big\}=(1,2,..,m).
$$

Hence 
\begin{align*}
  (**)(\tauup(t_{i_1})\wedge), & \wedge (t_{i_1},..., t_{i_r})\\
  &=\varepsilon\tauup(t_{\mu_1})\wedge.. \wedge \tauup(t_{\mu_r})\wedge
  \tauup(t_{i_1})\wedge...\wedge \tauup (t_{i_r})\\ 
  &=\varepsilon \varepsilon' \cdot \ell
\end{align*}
where $\varepsilon'$ is the signature of the permutation
$$
(1,2,...,m)\rightsquigarrow(t_{\mu_1},..,t_{\mu_r},
t_{i_1},..,t_{i_r}).
$$

Hence ~$\varepsilon \varepsilon'=(-1)^{r(m-r)}=(-1)^{r(m-1)}$. It follows that
$$
(**)(\tauup(t_{i_1})\wedge...\wedge \tauup(t_i)(t_{i_1},..,t_{i_r}) = 
(-1)^{r(m-1)} \left\{ \tauup(t_{i_1})\wedge...\wedge
\tauup(t_{i_r})\right\}(t_{i_1},...,t_{i_r}).
$$

For $(j_1, j_2,..,j_r)\neq (i_1,..,i_r)$ we have already seen
$$
(**)\tauup (t_{i_1})\wedge..\wedge \tauup(t_i))(t_{j_1},...,t_{j_r})=0.
$$

Evidently,
$$
\big\{\tauup (t_{i_1})\wedge...\wedge \tauup(t_i)\big\}(t_{j_1},..,t_{j_r})=0.
$$

Hence the lemma.

The isomorphisms $\tauup$ between the $F$-module of $C^\infty$ tangent
vector fields on $X$ onto the $E$-module $C^1$ extends to an isomorphism
between the $r$-th exterior power of these modules. 

Let $U$ be a coordinate open set on the differentiable manifold
$X$. Let $x^1,...,x^m$ be a coordinate system on $U$. Then for the fixed
basis $dx^1,...,dx^m$ for the space of differentials at every point of
$U$, any $r$-form\pageoriginale $\varphi \in C^r$ can be represented
in $U$ by  
$$
\phi=\phi_I dx^I,
$$
where $\phi_I$ is a $C^\infty$ function on $U$, $I=(i_1,...,i_r)$ is an
$r$-tuple of  indices $1 \le, i_1 < \ldots < i_r \le m=\; {\dim}_{\mathbb{R}}
X$, and $dx^I  = dx^{1_i} \wedge \ldots \wedge dx^{1_r}$. 

Let $g_{ij}$ be the metric tensor in $U$, and let $g^{ij}(i,j=1,...,m)$
be the elements of the inverse matrix $g^{-1}=(g_{ij})^{-1}$. Then
$\tauup^{-1}\phi$ is defined in $U$ by its components (with respect to
the basis $\dfrac{\partial}{\partial
  x^1},...,\dfrac{\partial}{\partial x^m}$) 
$$
\varphi^{i_1 \ldots i_r}=g^{i_1 k_1} \ldots g^{i_r k_r} \varphi_{k_1
  \ldots k_r}.
$$
If  
$$
(*\varphi)_J \; dx^J
$$
is the local representation in $U$ of the $(m-r)$-form $* \varphi, J$ being
the multi-index $J =(j_1, \ldots, j_{m-r})(1 \le j_1 < \ldots <
j_{m-r}\le  m)$, then 
$$
(* \varphi)_J = \ell_{IJ} \varphi^I.
$$
where $\ell=\ell_{1 \ldots_m} dx' \wedge...\wedge dx^m$ is the
representation of the volume element in
$U (\ell_1..._m=\sqrt{det(g_{ij})})$ and $\ell_{IJ}=\varepsilon
\ell_{1 \ldots m}' \varepsilon$ being the sign of the summation
$(1,...m)\rightsquigarrow(IJ)$. 

The proof is straightforward. It follows that for $\varphi, \psiup \in
C^r$, 
\begin{equation}
  \varphi \wedge * \psiup =\dfrac{1}{r!}\left\{ \sum\limits_I
  \varphi_I \psiup^I \right\}\ell \tag{1, II} 
 \end{equation}
the summation being over all multi-indices $I=(i_1,...,i_r)$ or 
$$
\varphi \wedge * \psiup=\sum \varphi_I \psiup^I
$$
if we extend the summation only to the multi-indices
$I=(i_1,...,i_r)$, $l \leq i_1 < \ldots < i_r \leq m$ clearly,  
\begin{equation}
  \left.
  \begin{aligned}
    {\rm clearly}, &\qquad  &  \varphi \wedge * \psiup &=\psiup\wedge * \varphi\\
    {\rm and } & \qquad &  *\varphi \wedge &** \psiup=\varphi \wedge * \psiup
  \end{aligned}
  \right\} \tag*{$\dots$ (1, III)}
\end{equation} 

We\pageoriginale now revert to the situation when the manifold $X$ is
a complex manifold and the Riemannian  
structure is the one canonically associated to a hermitian structure
on $X$ .  Let $C^{pq}(X)$ be the  
complex vector space of $C^\infty$ forms of type $(p,q)$ on $X$ .

\begin{lemma}%%% 1.3
  The $*$-operator maps $C^{pq}(X)$ into  $C^{n-q,n-p} (X)$. 
\end{lemma}

\begin{proof}  
  The isomorphism $\tauup$ of the (real) tangent space to
  $X$ at $x \epsilon X$ onto space of 
  differentials extends to a complex linear isomorphism of the
  ``complexified'' tangent space onto the space of complex  
  valued differentials.  Denoting the extension again by $\tauup$ and
  noting the fact that every complex valued differential 
  form may be regarded as a multilinear form on the complexified
  tangent space, we see that the formula  
  $$
  (*\varphi)(t_1,\dots,t_{m-r})\ell = \varphi\wedge \tauup(t_1)\wedge
  \dots \wedge \tauup(t_{m-r})
  $$
  holds for complex tangent vectors $(t_1,\dots,t_{m-r})$ as well.
  Moreover from the definitions of the associated 
  Riemannian structure, we see easily that for every
  $\alpha,\tauup\left(\dfrac{\partial}{\partial z^\alpha_i} \right)$ is a 
  linear combination of the $dz_i^\beta$ while
  $\tauup \left(\dfrac{\partial}{\partial
    \bar{z}^\alpha\alpha}\right)$ is a linear   
  combination of the $dz_i^\beta$.  Our assertion now follows from
  the definitions of forms of type $(p,q)$. 
\end{proof}

Now, let $E$ be a holomorphic vector bundle on $X$ defined by 
$(\MG = \{U_i\}_{i \in I}$; $e_{ij}:U_i \cap U_j
\rightarrow GL (n,\mathbb{C}))$.  

Let $h = (h _i)_{i \in I}$ be a hermitian metric along the fibres.
Let $C ^{p,q} (X,E)$ denote the  
space of $C^{\infty}$ - forms of type $(p,q)$ on $X$ with values in $E$ .
The operator * of the associated Riemannian 
structure defines also a canonical isomorphism 
$$
* : C^{p,q} (X,E) \rightarrow C^{n-p, n-p}(X,E)
$$ 
as follows.  If a form $\varphi$ in $C^{p,q}(X,E)$ is represented
in $U_i$ by a  
column\pageoriginale $\left (\vdots^{\varphi^1_i}_{\varphi^m_i}\right)$
where each $\varphi^k_i$ is a scalar $(p,q)$ form, then $* \varphi$ is
represented by $\left (\vdots^{ *\varphi^1_i}_{* \varphi^m_i}\right)$ 
in $U_i$.  Next, the hermitian metric h enables us to define an
operator 
$$ 
\# :  C^{p,q}(X,E)\rightarrow C^{q,p}(X,E^*),
$$ 
where $E^*$ is the dual-bundle to $E.(E^*$ is defined by the
transition functions $\{ x \rightsquigarrow t_{e_i (x)}^{-1}\}$ with
respect to the covering  
$\MG = \{ U_i\}_{i \in I}$).  In fact if we represent a
$(p,q)$-form $\varphi$ in $U_i$ by a column $\left
(\vdots^{\varphi^1_i}_{\varphi^m_i}\right)$, where each $\varphi^k_i$
is a scalar form of type $(p,q)$, then $\# \varphi$ in $ U_i$ 
is given by the column of $(p,q)$-forms 
$$ 
(\#  \varphi)_i  =
\begin{pmatrix} 
  h_i \varphi^1_i\\ 
  h_i \varphi^2_i\\ 
  h_i \varphi^m_i  
\end{pmatrix}
$$ 

That $\{(\# \varphi)_i \}_{i \in I}$ define forms with values in $E^*$
follows from the fact 
\begin{align*}
  \overline{h_i \varphi_i} =(t_{\overline{e}_{ji}}h_j e_{ji}
  e_{ij}\varphi_j) & =(t_{\overline{e_{ji}h_j}}\varphi_j)^-\\  
  & = e^{t-1}_{i_j} \overline{h_j \varphi_j}
\end{align*}

\begin{remark*}  
  $C^{p,q} (X,E)$\pageoriginale has a structure of an
  $F$-module where $F$ is the ring of global complex-valued
  $C^{\infty}$-functions. For this structure of $F$-modules on the
  $C^{p,q} (X,E)$
$$
 * : C^{p,q}(X,E) \rightarrow C^{n-q,n-p}(X,E)
$$
 is an $F$-linear isomorphism, while the operator 
$$
\# : C^{p,q}(X,E)\rightarrow C^{q,p}(X,E^*)
$$
 satisfies the condition 
 $$
 \# (f.\varphi) = \overline{f}(\# \varphi) ~\text{where} ~f \in F,
 \varphi \in C^{p,q}(X,E). 
 $$
\end{remark*}

It follows that the operators * and $\#$ define actually homomorphisms 
$$
\displaylines{\hfill 
* : A^{p,q}(E)\rightarrow A^{n-q,n-p}(E) \; $$
({\resp}~ A^{p,q}(E) \overset{\sharp}{\rightarrow }
A^{q,p}(E^*)\hfill \cr
\text{where}\hfill A^{rs}(E) ({\resp}~ A^{rs}(E^*))\hfill }
$$
is the vector-bundle of exterior differential forms of type $(r,s)$ with
values in $E$ (\resp $E^*$); here * is $\mathbb{C}$-linear while
$\#$ is anti linear. 

 In the special case when $E$ is trivial bundle, the map  $\# :
 C^{\circ\circ} (X, E) \to C^{\circ\circ} (X, E^*)$ is simply the map
 $$  
 f \rightsquigarrow \overline{-f}. 
 $$
 
 Let $\alpha \in C^{p,q} (X,E)$ and $\beta \in C^{rs}
 (E^*)$. Then $(\alpha \wedge \beta)$ is defined as a scalar form of
 type $(p+r, q+s)$. For $ \varphi, \psi  \in C^{pq} (X,E)$ we set 

 
 $A_E (\varphi,\psi) l = \varphi \wedge (\# * \psi)$. (We note that $\#
 * = * \#$.) If ($\supp \varphi \cap \supp \psi)$ is compact, then
 $$
 \Bigg\lmoustache_X |\varphi \wedge * \# \psi | =  \Bigg \lmoustache_X
 |A_E (\varphi,\psi)| <\infty
 $$
 
 Let $U$ be an open coordinate set in $X$ and let $\varphi, \psi$ be
 represented in $U$ by  
 $$
 \varphi=\left\{\frac{1}{p!q!} \varphi^{a}_{A\bar{B}}
 dz^A \wedge \overline{dz^B}\right\}, \psi= \left\{ \frac{1}{p!q!}
 \psi^a_{A \overline{B}} dz^A \wedge   \overline{dz^B}\right\}, 
 $$
where:\pageoriginale $a = 1,  \ldots, m = rank E$, $A = (\alpha_{1}, \ldots , 
\alpha_{p})$, $B = (\beta_{1}, \ldots , \beta_{q})$, $1 \leq \alpha_{i},
\beta_{i} \leq n$, $dz^A =dz^{\alpha_{1}} \wedge \ldots \wedge
dz^{\alpha_{p}}$, $dz^{\beta} = dz^{\beta_{1}} \wedge \ldots \wedge
dz^{\beta_{q}}$. 

Then it follows from (1 II) that 
$$ 
A(\varphi, \psi) = \frac{1}{p! \, q!} h_{\overline{b}a} \varphi^{a}_{A
  \overline{B}} \phi^{\overline{\psi^{b \overline{A}B}}},  
$$
$(h_{\overline{b}a})_{a,b = 1, \ldots , m}$ being the local
representation in $U$ of the metric along the fibres of $E$. We shall call
$A(\varphi , \varphi )^{\frac{1}{2}}$ the ``length of the form $\varphi$.  

Sometimes it will be denoted also by $ |\varphi |$.

Let $ \vartheta^{p,q} (X,E)$, be the space of $C^{\infty}-(p-q)-$
forms with compact support. Then for $\varphi, \psi \epsilon
\mathscr{D}^{pq} (X,E)$ we can define a scalar product:  
$$ 
(\varphi, \psi) = \underset{X}{\int} \varphi \wedge (* \# \psi ) =
\underset{X}{\int} A_{E}(\varphi, \psi). 
$$ 

Also, we denote $(\varphi , \varphi) $ by $ \| \varphi \|^{2};\| \quad
\|$ is in fact a norm: the hermitian bilinear form $(\varphi, \psi)$
defines on $ \mathscr{D}^{p,q}(X,E)$ the structure of a prehilbert
space over $\mathbb{C}$. The completion of $\mathcal{D}^{p,q}(X,E)$ is
denoted $\mathcal{L}^{p,q} (X,E)$. This latter space is referred to in the
sequel also as the space of square summable forms. Because of the
Riesze-Fisher theorem $ \mathcal{L}^{pq}(X,E) $ can be identified with
the space of square forms of type $(p,q)$-i.e. the space of $\varphi$
such that $\underset {X}{\int} A (\varphi , \varphi) < \infty$.    

\section{The space $ W^{p,q}$}\label{chap1:sec2} %%% 2

The operator $ \overline{\partial}: \underline{A}^{p,q}(E) \rightarrow
\underline{A}^{p,q+1} (E)$ defines homomorphisms (again denoted by
the same symbol) 
\begin{gather*}
  \overline{\partial}: C^{p,q} (X,E) \rightarrow C^{p,q+1} (X,E) \\
  \overline{\partial}: \mathscr{D}^{p,q} (X,E) \rightarrow
  \mathscr{D}^{p,q+1} (X,E).
\end{gather*}

We\pageoriginale can now define the adjoint of $ \overline{\partial}$
with respect to hermitian metric on $X$ and a hermitian metric along
the fibres of the vector bundle $ \pi : E \rightarrow X$, i.e. an
operator  
$$  
  \vartheta : C^{p,q}(X,E) \rightarrow C^{p,q-1}(X,E)
$$
satisfying
\begin{equation*}
 (\overline{\partial} \varphi, \psi) = 
  (\varphi, \vartheta \psi)  \tag{1.1}\label{eq1.1}
\end{equation*}
for \qquad $\varphi \epsilon \mathscr{D}^{p,q}(X,E)$, $\psi
  \epsilon \mathscr{D}^{p,q+1}(X,E)$. Such an operator, if it exists
  is immediately seen to be unique. For the existence we have the    

\begin{lemma}%% 1.4
  If $ \varphi \epsilon
  \mathscr{D}^{p,q}(X,E)$, $\psi \epsilon \mathscr{D}^{p,q+1} (X,E) $
  are Lipschitz continuous forms with compact support, then
  $(\overline{\partial} \varphi, \psi) = (\varphi, -* \#^{-1} 
  \overline{\partial} * \# \psi)$. 
\end{lemma}

\begin{proof} 
  We note first that the scalar products on both sides are
  defined. Firstly, the $*$ and $\#$  being $\mathbb{C}$-linear (\resp
  $\mathbb{C}$-anti linear)
  bundle homomorphisms, these can also be defined on arbitrary forms -
  that is arbitrary sections of the vector bundle $ A^{p,q}(E)$.
  Secondly, since $\varphi$ and $\psi$ are Lipschitz
  continuous $ \overline{\partial} \varphi$ and  $-^{*} \#^{-1}
  \overline{\partial} * \# $ are defined almost every where and are
  bounded measurable forms with compact support. We have moreover,     
{\fontsize{9}{11}\selectfont 
 $$ 
  \begin{aligned}
    d (\varphi \wedge * \# \psi) & = \overline{\partial}(\varphi \wedge *
    \# \psi) \\ 
    & = \left\{\overline{\partial} \varphi \wedge * \# \psi
    \right\}+\{(-1)^{p+q} 
    \varphi \wedge \overline{\partial}* \# \psi\}\qquad \\ 
    & = \left\{\overline{\partial} \varphi \wedge * \# \psi\right\} -
    \left\{\varphi \wedge * \# \vartheta \psi\right\} \\ 
    \text{since } \qquad ** = (-1)^{(p+q)(2n-p-q)} & =
    (-1)^{p+q} 
  \end{aligned} 
  $$}\relax 
  (here $n$ is the complex dimension of $X$). Applying Stoke's formula,
  we obtain (since $\varphi$ and $\psi$ are compact support), 
$$
0 = \underset{X}{\int} 
  (\overline{\partial} \varphi \wedge * \# \psi ) - \underset{X}{\int}
  \varphi \wedge * \# \vartheta \psi
$$
i.e. $(J \varphi, \psi) = (\varphi , \vartheta \psi)$,  which proves
the lemma.  
\end{proof}

  The\pageoriginale operator $ \vartheta $ depends on the hermitian
  metric on  $X$ and on the hermitian metric along the fibres of
  $E$. To emphasise this fact, we may write sometimes $ \vartheta_{E}$
  for $\vartheta $.   

We have already introduced the norm $ \| ~ \|$ on $
\mathscr{D}^{p,q}(X,E)$ and denoted the corresponding completion by $
\mathcal{L}^{p,q}(X,E)$. We will  now introduce another norm on $
\mathscr{D}^{p,q}(X,E)$.  

\begin{definition}%%% 1.2
  For $ \varphi,\psi \epsilon \mathscr{D}^{p,q}(X,E)$. 
$$
a (\varphi, \psi) = (\overline{\partial} \varphi , \overline {\partial} \psi) +
  (\vartheta \psi , \vartheta \psi) + ( \varphi, \psi).
$$
 and \; $N  (\varphi)^{2} = a( \varphi, \varphi)$.  
 
 As before we see that the hermitian scalar product  $a(\varphi ,
 \psi)$ defines a complex prehilbert-space structure on
 $\mathscr{D}^{p,q}(X,E)$.  We denote by $W^{p,q}\break (X,E)$ the
 corresponding completion. Clearly, we have for $\varphi , \psi
 \epsilon \mathscr{D}^{p,q}\break(X,E)$  the  inequality $\| \varphi
 \| \leq N(\varphi)$, so that the identity   
 $$ 
 \mathscr{D}^{p,q}(X,E)\rightarrow \mathscr{D}^{p,q}(X,E) 
 $$ 
 extends uniquely to a continuous linear map, 
 $$ 
 i : W^{p,q}(X,E) \rightarrow \mathcal{L}^{p,q}(X,E).  
 $$
\end{definition}

 \begin{prop}\label{chap1:prop1.1}%%% 1.1
   i is injective.
 \end{prop}
 
\begin{proof}
  Let $ \varphi_{\nu} \epsilon \mathscr{D}^{p,q}(X,E)$ be a Cauchy
  sequence in $N$ and assume that $\| \varphi_\nu \| \rightarrow 0$.
  Since $ \varphi_{\nu} $ is a Cauchy-sequence in $N$, $
  \overline{\partial} \varphi_{\nu}$ and $ \vartheta \varphi_{\nu}$
  are Cauchy sequences in the norm $\| \;\;  \|$. Hence
  $\overline{\partial} \varphi_{\nu}$ and $\vartheta \varphi_{\nu}$
  tend to a limit in $\mathcal{L}^{p,q+1}(X,E)$ and
  $\mathcal{L}^{p,q-1}(X,E)$ respectively.   

  We denote these limits by $u$ and $v$. By our identification of $
  \mathcal{L}^{p,q}\break(X,E)$ as the space of square summable $E$-valued
  forms, $u$ and $v$ may be regarded as forms on $X$ with values on $E$. 
\end{proof}

 Now\pageoriginale for any $ \psi \in \mathscr{D}^{p, q+1} (X, E)$
$$
 (u, \psi)
 =\underset{  \nu \to \infty }{\lim} (\overline \partial \varphi_\nu ,
 \psi) = \underset{  \nu \to \infty }{\lim} (\varphi_\nu , \vartheta
 \psi) = 0 
$$
since $\| \varphi_\nu \| \rightarrow 0$. Since $\psi$ is
 arbitrary in $\mathscr{D}^{p, q + 1} (X , E), U=0$. Similarly
 $v=0$. That is, $\overline \partial \varphi_\nu \rightarrow 0$ and $
 \vartheta \varphi_\nu \rightarrow 0 $ in $\mathcal{L}^{p, q+1} (X,
 E)$ and $\mathcal{L}^{p, q-1} (X, E)$ respectively. Thus
 $N(\varphi_\nu) \longrightarrow 0$ as $\nu \longrightarrow \infty,$
 hence the proposition. 
 
 By continuity we obtain from (\ref{eq1.1}) that, if $\varphi \in W^{p,q}
 (X, E), \psi \in W^{p,q+1} (X ,E)$, then 
 $$
 (\overline{\partial} \varphi, \psi) = ( \varphi, \vartheta \psi)
 $$
 
 Clearly, we may now regard $W^{p ,q} (X ,E)$ also as a space of
 measurable $E$-valued forms on $X$. From the definitions, it is moreover
 clear that if $f \in W^{p,q} (X, E)$, $\overline{\partial}f$ and
 $\vartheta f $ in the distribution sense are currents representable by
 square summable $(p ,q + 1)$ and $(p ,q - 1)$ forms respectively. In
 general $W^{ p,q} (X, E)$ is not the space of all square summable
 forms $\omega$ of type $(p,q)$ whose $\overline{\partial}$ and
 $\vartheta$ (in the sense of distributions) are again square
 summable. We have however, the 
 
 \begin{theorem}%%% 1.1
   If the Riemannian metric (associated to the hermitian metric) on $X$
   is complete, then 
   {\fontsize{10}{12}\selectfont
   $$
   W^{p,q} (X, E)= \left\{ \varphi \mid \varphi \in \mathcal{L}^{p,^\wedge
     q} (X, E); \overline{\partial} \varphi \in
   \mathcal{L}^{p,q+1}(X ,E); \vartheta \varphi \in
   \mathcal{L}^{p,q-1} (X, E) \right\}
   $$}\relax 
 \end{theorem}

\begin{proof}
  We will first establish three lemmas which are needed for the
  proof.
\end{proof}

\begin{alphlemma}\label{chap1:alphlemmaA}%%% A
  There exists $C_\circ > 0$ depending only on the dimension of X
  such that for any scalar form u and any $v \in C^{p,q} (X,
  E)$, 
  $$ 
  A_E(u \wedge {v}, u \wedge  {v})(x) \leq \{C_\circ \mid u
  \mid^2 A_E  {(v, v)}\} (x) 
  $$
where\pageoriginale $\mid u \mid^2$ denotes the length of the scalar
form a). (We recall that $A_E(\varphi , \psi)\ell = \varphi \wedge \ast \#
  \psi)$. This is simply a lemma on finite dimensional vector space
  with scalar products. We omit the proof. 
\end{alphlemma}


\begin{alphlemma}\label{chap1:alphlemmaB}%%% B
  Let $ p_\circ \in X$ be any point. The function $\rho(x) = d
  (p_\circ, x)=$ distance from $0$ of $x$, is locally Lipschitz continuous
  and wherever $\rho$ has partial derivatives, $\mid d \rho \mid^2 \leq
  2n.$ 
\end{alphlemma}

\begin{proof}
  We have by the triangle inequality 
  $$ 
  \mid \rho  {(x)} - \rho  {(y)} \mid \leq  {d(x ,y)}.
  $$

  Now if $U$ is a coordinate open set with coordinates $(X^1,\ldots ,
  X^m)$ ($X$ considered as a  differentiable manifold of dimension $m=2
  n$) and the Riemannian metric is given on $U$ by $\sum g_{ij}dx^ix^j$
  then for any  $V \subset 0 \subset U$, there exists $\lambda$ and
  $\mu$ such that  
  $$
  \lambda \sum dx^{i^2} \leq \sum  g_{ij}dx^i dx^j \leq \mu \sum
  dx^{i^2}
  $$
  so that if $V$ is a ball about the origin in $U$, there exist constants
  $C_1 C_2 > 0$ such that  
  $$ 
  C_1 \mid x-y \mid \leq d(x, y) \leq C_2 \mid x-y \mid  \text{ for  }
  x, y \in  V. 
  $$

  Thus $ \mid \rho (x) - \rho (y) \mid < C_2 \mid x-y \mid$ for $x, y \in
  V$. Hence the first assertion. 
\end{proof}

For the second assertion consider about a point $p$ a neighbourhood $U$ in
which we can introduce geodesic polar-coordinates. Let $(x^1,
\ldots , x^m)$ be a coordinate system such that $x^i(p)=0$
for $1 \leq i \leq m$.  

For $\varepsilon > 0$ we can choose $U$ so small that $\mid g_{ij}
\mid <\delta_{ij} +\varepsilon$, i.e.  $d(x,p)^2 \leq (1 + \varepsilon ) \sum
x^2_i$. In such a coordinate system, we have setting $e_i = (0,\ldots ,1
\ldots ,0)$ (1 at the $i^{\rm th}$ place)  
$$
\frac{\partial \varrho}{\partial x^i}(\rho) =\underset{h\rightarrow 
  0}{Lt} \frac{\rho (he_i) - \varrho (0)}{h}
$$ 

On\pageoriginale the other hand we have 
$$
 {\mid \rho (he_i)-\rho (0) \mid \leq d( 0, h,e_i) \leq (1 + \varepsilon ) h}
$$

Since we are working with a coordinate system such that $d(x ,0
  )\leq$ $(1 + \varepsilon \left(\sum x^2_i\right)^{\frac{1}{2}}$. Hence $\rho$ is
Lipschitz continuous and, if the derivatives exist, $
{\dfrac{\partial \rho}{\partial x^i} (p) \leq 1}$. Now, by definition
$\mid d \rho \mid^2 =\sum g_{ij} \frac{\partial \rho}{ \partial
    X^i} \frac{\partial \rho }{\partial X^j}$ and in the coordinate
system introduced above, $g_{ij}(0) = \delta_{ij}$ so that $
\mid \rho \mid^2 = \overset{m}{\underset{1=1}\sum} \big\lgroup
  \frac{\partial \rho }{\partial x^i}\big\rgroup^2 \leq m$. When $X$ is
a complex manifold of complex dimension $n$, we have, 
$$
\mid d \rho \mid^2 \leq 2n.
$$


\begin{alphlemma}\label{chap1:alphlemmaC}%%% C
Let  
$$
\begin{aligned}
  \dot{\mathscr{D}}^{p,q}_{\overline{\partial}} (X ,E) & = \left\{ \varphi
  \mid \varphi \in \mathcal{L}^{p,q} (X, E), \overline{\partial} \varphi
  \in \mathcal{L}^{p,q+1} (X, E), \Supp \varphi \subset \subset X\right \}\\[5pt]
   \dot{\mathscr{D}}^{p,q}_{\vartheta} (X ,E) & = \left\{ \varphi \mid \varphi
  \in \mathcal{L}^{p,q} (X, E),\right.\\
 &\qquad\left.\vartheta \varphi \in \mathcal{L}^{p,q}
  (X, E), \vartheta \varphi \in \mathcal{L}^{p,q-1} (X, E), 
  \supp \varphi \subset  X\right\} 
\end{aligned}
$$
(where $\overline{\partial}$ and $ \vartheta$ are in the sense of
distributions) and finally 
$$
\dot{\mathscr{D}}^{p,q}(X, E)
  = \dot{\mathscr{D}}^{p,q}_{\overline{\partial}} (X ,E) \cap
  \dot{\mathscr{D}}^{p,q}_\vartheta (X, E).
$$
Then
\begin{enumerate}
\item[{\rm (i)}] $\mathscr{D}^{p,q} (X, E)$ is dense in
  $\dot{\mathscr{D}}^{p,q}_{\overline{\partial}} (X ,E)$ w.r.t. the norm
$$
P(\varphi): P(\varphi)^2 = \| \varphi \|^2 + \| \overline{\partial}
\varphi \|^2
$$
and

\item[{\rm (ii)}] $\mathscr{D}^{p,q} (X, E)$ is dense in
  $\dot{\mathscr{D}}^{p,q}_{\vartheta} (X ,E)$  with the norm 
$$
Q(\varphi): Q(\varphi)^2 = \| \varphi \|^2 + \| \vartheta \varphi \|^2 
$$

\item[{\rm (iii)}] $\mathscr{D}^{p,q} (X, E)$ is dense in
  $\dot{\mathscr{D}}^{pq}(X ,E)$ with the norm 
$$
N: N(\varphi)^2 = \| \varphi \|^2 + \| \overline{\partial} \varphi 
\|^2 + | \vartheta \varphi \|^2.
$$ 
\end{enumerate}
\end{alphlemma}

\begin{proof}
  We will\pageoriginale prove (i) ; the proofs for the other two cases
  are similar.  
  Let $ \MG= \{U_i \}_{i \in I}$ be a locally finite covering of
  $X$ such that $E \mid_{U_i}$ is trivial. Let $m$ be the rank of $E$. Assume
  further that each $U_i$ is a coordinate open set. Let $\{ V_i \mid V_i
  \subset \subset U_i\}_{i \in I}$ be a shrinking of $U_i$. Let
  $(\rho_i)_{ i \in I} $ be a partition unity subordinate  to $\{V_{i}
  \}_{i \in I}$. Let $\varphi \in
  \dot{mathscr{D}}_{\overline{\sigma}}^{p,q(X,E)}$. $E \mid_{U_i}$
  being trivial, we may take $\rho_{i} \varphi_{i}$ to be a
  $\mathbb{C}^m$-valued 
  from on  $\mathbb{C}^n$ with compact support in $V_i$. For any
  $\varepsilon > 0$, we can 
  find a  $\mathbb{C}^m$ valued $C^\infty$ from $\psi_i$ with compact support in
  $V_i \subset \mathbb{C}^n$, such that $\mid \mid {\overline{\partial}}
  \rho_i \varphi_i - {\overline{\partial}} \psi_i \mid \mid < \varepsilon$
  and $\| \rho_i \varphi_i - \psi_i \| < \varepsilon$. This can be secured by
  the usual regularisation methods. Since the support of $\varphi$ is
  compact, we may take $\psi_i = 0$ except for a finite number of
  $i$. Since $E\mid _{U_i}$ is trivial and $\rho_i \varphi_i$ were
  regarded as scalar forms through suitable trivialisations, we may now
  revert the process and consider $\psi_i$ as $E$-valued $C^\infty$-forms
  with support in $V_i$. It follows that $\psi = \sum \psi_i$ is a
  $C^\infty$-form with compact support and 
$$
\| \varphi - \psi \| = \|
  \sum \rho_i \varphi_i - \sum \psi_i \| \leq {\underset{i}{\sum}}\|
  \rho_i \varphi_i - \psi_i \| < M\varepsilon
$$
 and similarly $\| {\overline{\partial}}\varphi -
 {\overline{\partial}} \psi \| < M \varepsilon$ where $M$ is the number of
 indices of the finite set  
 $$ 
 \{ i \mid i \in I, U_i \cap  ~~\text{ support } ~~ \varphi \neq
 \phi\}.
$$
Since $M$ is fixed for a given $\varphi$ and $\varepsilon$ is at our
 choice, the lemma is proved. 
\end{proof}
  
Lemma \ref{chap1:alphlemmaC} enables us to prove the following more
general statement. 

\setcounter{theorem}{0}
\begin{theorem}\label{chap1:thm1.1}%%% 1.1
  If\pageoriginale the Riemannian metric (associated to the hermitian
  metric) of $X$ complete, then 
  \begin{enumerate}[\rm i)]
  \item $\mathscr{D}^{p,q}(X, E)$ is dense in the space
    $$
    \left\{ \varphi \mid \varphi \varepsilon \mathcal{L}^{p,q} (X,E),
    \overline{\partial}\varphi \in  \mathcal{L}^{p,q+1} (X,E)\right\},
    $$ 
    with respect to the norm $P(\varphi)$;

  \item $\mathscr{D}^{p,q}(X,E)$ is dense in the space
    $$
    \left\{\varphi \mid \varphi \in \mathcal{L}^{p,q}(X,E), \vartheta
    \varphi \in \mathcal{L}^{p,q-1}(X,E)\right\},
    $$ 
    with respect to the norm $Q(\varphi)$;

  \item $\mathscr{D}^{p,q}(X,E)$ is dense in the space
    $$ 
    \left\{\varphi \mid \varphi \in \mathcal{L}^{p,q} (X,E),
     {\overline{\partial}} \varphi \in
     \mathcal{L}^{p,q+1}(X,E), \vartheta \varphi \in
     \mathcal{L}^{p,q-1}(X,E) \right\},
    $$ 
     with respect to the norm $N(\varphi)$.
  \end{enumerate}
\end{theorem}

\begin{proof} 
  We will prove i). The proofs of ii) and iii) are similar. 

  In view of Lemma \ref{chap1:alphlemmaC} it is sufficient to prove
  that every distributive 
  form $\varphi$ in $\mathcal{L}^{p,q}(X,E)$ such that
  ${\overline{\partial}} \varphi \in \mathcal{L}^{p,q+1}(X,E)$ can be
  approximated as closely as we want by forms $\psi$ such that $\psi ,
  {\overline{\partial}} \psi$ are square summable and supp $\psi ,
  \subset \subset X$. 
  
  Let $\mu: \mid R^1 \rightarrow [0,1]$ be a $C^\infty$ function such that 
  
  (i) $\mu (t) = 1$ if $t < 1$ and (ii) $\mu (t) =0$ if $t >2$.
  
  Let $M= {\underset{t}{\Supp}} \mid \frac{d \mu}{d t}\mid$. Let $d(x,y)$
  be the distance function defined by the complete Riemannian metric
  of $X$. Then we fix a point $p_\circ \in X$ 
 and\pageoriginale set $\rho (x) = d (x, p_\circ)$ for $x \in X$, the
 function  
 $$
 \omega_\nu (x) = \mu \left(\frac{\rho (x)}{\nu}\right)\nu
 > 0 
 $$
 is locally Lipschitz, and where the derivatives $\frac{\partial
   \rho}{\partial x^i}$ exist, then  
 $$
 \mid d \omega_\nu \mid^2 \leq \mid \frac{1}{\gamma} \frac{\partial
   \mu}{\partial t} \left(\frac{\rho (x)}{\nu} \right) d
 \rho \mid^2 \leq \frac{2nM^2}{\nu^2}
 $$
 in view of Lemma \ref{chap1:alphlemmaB}. Since the metric is complete
 the ball of centre $p$ 
 and radius $c$, 
 $$
 B_c =\{ p\mid d (p,p_\circ)< c\}
 $$
 is relatively compact in $X$ for all $c > 0$.

Consider now the form $\omega_\nu. \varphi = \varphi_\nu $;
$\varphi_\nu$ has compact support. It is easily seen that
$\omega_\nu. \varphi = \varphi_\nu $ is in $\dot{\mathscr{D}}^{p,q}(X,
E)$. We will now prove that $\varphi_\nu \rightarrow \varphi$ in the
norm $N$ as $\nu \rightarrow \infty$. In fact, we have first of all  
$$
\| \varphi - \varphi_\nu \| = \| (1 - \omega_\nu) \varphi \| _{X-B_
  \nu} \leq \| \varphi \|_ {X-B_\nu} \rightarrow 0  ~~ \text{as} ~~
\nu \rightarrow \infty.
$$

Secondly
\begin{align*}
  \| {\overline{\partial}} \varphi - {\overline{\partial}} \varphi_\nu
  \| &= \| {\overline{\partial}} \varphi - \omega_\nu
     {\overline{\partial}} \varphi - {\overline{\partial}} \omega_\nu
     \wedge \varphi \| \\ 
     &\leq \| {\overline{\partial}} \varphi \| _ {X-B_\nu} + \|
     {\overline{\partial}}\omega_\nu \wedge \varphi \|\\ 
  \text{Now}, \hspace{2cm} \| \overline{\partial} \omega_{\nu} \wedge
  \varphi \| &= A({\overline{\partial}}\omega_\nu \wedge \varphi,
   {\overline{\partial}} \omega_\nu \wedge \varphi) \\ 
   &\leq \mid {\overline{\partial}}\omega_\nu \mid^2 \cdot A(\varphi ,
   \varphi) \quad  \text{\; by Lemma \ref{chap1:alphlemmaA}. }
\end{align*}
On the other hand, one checks easily that almost everywhere
$$
\mid {\overline{\partial}}\omega_\nu \mid^2 \leq \mid d \omega_\nu
\mid^2 \leq \frac{2nM^2}{\nu^2},
$$ 
so that, we obtain
$$ 
\| {\overline{\partial}} \varphi_\nu - {\overline{\partial}} \varphi 
\| \leq \| {\overline{\partial}} \varphi\|_ {X-B_\nu} +
\frac{C'}{\nu} \| \varphi \|.
$$
Hence\pageoriginale $\mid \mid \overline{\partial} \varphi_\nu -
\overline{\partial}\varphi \mid \mid \rightarrow 0 $ as $\nu \rightarrow
\infty $. This completes the proof of Theorem \ref{chap1:thm1.1} (i). 
\end{proof}

\begin{definition}%% 1.3
  $\square : C^{p,q}(X,E)\rightarrow C^{p,q}(X,E)$ is the operator
  $\overline{\partial}\vartheta + \vartheta \overline{\partial}$. The
  following result is easily checked. 
\end{definition}

\begin{lemma}%% 1.5
  The operator $\square$ defined above is strongly elliptic.
\end{lemma}

The operator $\square$ depends on the metric on $X$ and on the metric
along the fibres of $E$. To emphasize this fact we may write
$\square_E$ for $\square$. 

For the operator $\square$ we have the following result.

\begin{theorem}[Stampacchia Inequality]\label{chap1:thm1.2}%%% 1.2
  We assume the Riemannian metric on $X$  (associated to the hermitian
  structure) to be complete. Let $p_o \in X$ and for $\nu>0$, let
  $B_\nu=\{x \mid d(x,p_o)<\nu \}$. Then there exists a constant $A>0$
  such that for every $\sigma >0$, for every choice $r<R$ of a pair of
  positive reals $r$, $R$ and every $\varphi \in C^{p,q}(X,E)$, 
  $$
  \mid \mid \overline{\partial}\varphi \mid \mid ^2_{B_{r}} + \mid
  \mid \partial \varphi \mid \mid^2_{B_{r}}\leq \sigma \mid \mid
  \square \varphi \|^2_ {B_{r}}+ \bigg (\frac{1}{\sigma}+
  \frac{A}{(R-r)^2}\bigg)\ \| \varphi \|^2 _{B_{R}} 
  $$
\end{theorem}

\begin{proof} 
  Let us choose as before a $C^\infty$ function $\mu: \mathbb{R}^1 \rightarrow
  [0,1]$ such that $\mu (t)=1$  for $t<1$ and $\mu (t)=0$ for $t>2$. 

  Let $M = sup |\frac{d \mu}{dt}|$. consider the function defined by 
  $$ 
  W( x)= \mu \left\{ \frac{\rho(x) + R - 2r  }{R-r}\right\}
  $$ 
  where $\sigma ( x)= d(p_o,x)$. Evidently then,  $\omega$   has
  support in $B_R$ and $ w \equiv 1$ on $B_r (B_R \subset \subset X)$. 
\end{proof}

Moreover,\pageoriginale $|d\omega|^2 = | \frac{1}{R-r} \frac{\partial
  \mu}{\partial t} 
\left\{\frac{\rho (x)+R-2r}{R-r}\right\} \cdot d_\rho|^2$. It follows 
 that $|d \omega|^2 \leq \frac{2nM^2}{(R-r)^2}$.

 Suppose now that $\varphi
 \in C^{pq}(X,E)$ and $\psi$ is any Lipschitz continuous form with
 compact support in $B_R$, then 
 $$
 (\overline{\partial}\varphi, \overline{\partial}\psi)_{B_R}+ 
 (\vartheta\varphi , \vartheta \psi)_{B_R}= (\square \varphi,
 \psi)_{B_R}.
 $$
 
 Set $\psi = \omega^2 \varphi$. We have then
 \begin{align*}
   \overline{\partial} \psi & = \omega^2 \overline{\partial}\varphi + 2
   \omega \overline{\partial}\omega \Lambda \varphi \\ 
   \vartheta \psi & =  \omega^2 \vartheta \varphi - * (2 \omega
   \partial \omega \Lambda * \varphi). 
 \end{align*} 
 This leads to 
 \begin{multline*}
   (\omega \overline{\partial}\varphi , \omega
   \overline{\partial}\varphi)_{B_R}+(\omega \vartheta \varphi ,
   \omega \vartheta \psi)_{B_R}\\ 
    =(\square \varphi, \omega^2 \varphi) -(\omega
   \overline{\partial}\varphi,2 \overline{\partial}\omega \Lambda
   \varphi)+(\omega \vartheta \varphi , *(2 \partial \omega
   \Lambda * \varphi)). 
 \end{multline*}
 
 Now, by Schuarz inequality,
\begin{gather*}
  \mid (\square \varphi, \omega^2 \varphi )_{B_R}\mid \leq \frac{1}{2}
  \sigma \| \square \varphi \|^2_{B_R} + \frac{1}{2\sigma}\|\omega^2
  \varphi \|_{B_R} \leqslant \frac{\sigma}{2}\| \square \varphi
  \|^2_{B_R}+ \frac{1}{2\sigma}\| \varphi \|^2_{B_R}\\
  \mid (\omega \overline {\partial} \varphi, 2
  \overline{\partial}\omega \Lambda \varphi)_{B_R}\mid \leqslant
  \frac{1}{2}\| \omega \overline {\partial} \varphi
  \|^2_{B_R}+\frac{4c_onM^2}{(R-r)^2} \|\varphi \|^2_{B_R}\\ 
 \text{and}\hspace{1cm} 
 \mid (\omega \vartheta \varphi , *(2\partial \omega \Lambda * \varphi
 ))\mid \leqslant \frac{1}{2}\| \omega \vartheta \varphi \|^2_{B_R}+
 \frac{4c_onM^2}{(R-r)^2}\| \varphi \|^2_{B_R}\hspace{1cm}
\end{gather*}
(where $c_o$ is the positive constant which has been introduced in
lemma \ref{chap1:alphlemmaA}) 

If follows\pageoriginale that
$$
\| \omega \overline{\partial}\varphi \|^2 _{B_R} + \|\omega \vartheta
\varphi \|^2_{B_R} \leq \sigma \|\square \varphi \|^2_{B_R} +
\sigma \|\varphi \|^2_{B_R}+ \frac{8c_onM^2}{(R-r)^2}\|\varphi
\|^2_{B_R}.
$$

The inequality follows now from
$$
\|\omega \overline{\partial} \varphi \|^2_{B_R} \geqslant \|
\overline{\partial}\varphi \|^2 _B ~~\text{and }~~\| \omega \vartheta
\varphi \|^2 _{B_R} \geqslant \|\vartheta \varphi \|^2 _{B_R}
$$

This completes the proof of theorem \ref{chap1:thm1.2}.

\begin{corollary}\label{chap1:coro1}%%% 1
  For $\varphi_2 \in C^{p,q}(X,E)$ and for any $\sigma > 0$,
  $$ 
  \|\overline {\partial} \varphi \|^2  + \|\vartheta \varphi \|^2 \leq
  \sigma \| \square \varphi \|^2 + \frac{1}{\sigma} \| \varphi \|^2.
  $$
\end{corollary}

\begin{proof}
  Set $R=2r$ in stampachia inequality and let $r \rightarrow \infty $.
\end{proof}

\begin{corollary}\label{chap1:coro2}%% 2
  If $\varphi \in C^{p,q}(X, E)$, $\|\varphi\|<\infty $ and $\|\square
  \varphi \| <\infty$, then 
  $\| \overline{\partial}\varphi\|<\infty$, $\|\vartheta \varphi \|
  <\infty$. If $square \varphi =0$, then
  $\overline{\partial}\varphi = \vartheta \varphi=0 $. 
\end{corollary}

\begin{proof}
  The first assertion follows from corollary \ref{chap1:coro1}. The second
  again from corollary \ref{chap1:coro1} since $\sigma$ is arbitrary.
\end{proof}

\begin{remark*}
  On any (paracompact) complex manifold $X$ there exists a
  complete hermitian metric. More exactly we shall prove that, given
  any hermitian metric $ds^2$ on $X$, there exists a $C^\infty$ function
  $F:X\rightarrow \mathbb{R}$ such that $F \cdot ds^2$ is a complete metric. 
\end{remark*}

\begin{proof}
  Let $\{B_\nu\}_{\nu \in \mathbb{N}}$ be a sequence of compact sets such that
  $$
  B_\nu \subset \overset{o}{B}_{\nu+1}, \cup B_\nu=X.
  $$

  Let $f_\nu : X\rightarrow \mathbb{R}$ be a $C^\infty$ function on $X$,
  satisfying the following\pageoriginale conditions:
  \begin{align*}
   & 0\leqslant f_\nu \leqslant 1,\\
   & f_\nu=1~ \text{on}~ B _{\nu + 1} -B_\nu ,\\
   & \Supp ~~ f_\nu \subset \overline {B_{\nu+2}-B_{\nu-1}}
  \end{align*}

  Let $d(x,y) \; (x,y \in X)$ be the distance function determined by the
  hermitian $ds^2$, and let 
  \begin{gather*}
    \varepsilon_\nu = \inf_{\substack{x \in B_\nu\\y \in
        \partial B_{\nu+1}}} d(x,y).
  \end{gather*}

  Then $\varepsilon_\nu > 0$. Consider the positive $C^{\infty}$ function 
  $$
  F(x)= {\overset{+\infty}{\underset{\nu=0}\sum}}
  \frac{f_\nu (x)} {\varepsilon_\nu}
  $$ 
  and the hermitian metric $\widetilde{ds}^2 =
  F(x)ds^2$. Let $\tilde{d}(x,y)$ be
  the distance function determined by $ds^2$. We have 
  \begin{align*}
    \overset{\backsim} {d}(\overline {B}_{\nu},
    B_{\nu+1}) \geqslant& \\ 
    \geqslant  & \quad x\in \frac{\inf}{
        B_{\nu+1}- B_\nu} F(x) {d}(\overline B_{\nu}, B_{\nu+1})\\ 
      \geqslant & \quad  \frac{1}{\varepsilon_{\nu}} \varepsilon_\nu = 1.
  \end{align*}  
  This implies that every Cauchy sequence for the distance
  $\tilde{d}(x,y)$ converges. Hence
  $\widetilde{ds}^2$ is a complete hermitian metric. 
\end{proof}

\section{W-ellipticity and a weak vanishing
  theorem}\label{chap1:sec3}\pageoriginale %% 3 

We now introduce a seminorm on ${W}^{p,q}(X,E)$ as follows:
for $\varphi, \psi \epsilon {W}^{p,q}\break(X,E)$,  let
  $b(\varphi, \psi) = (\overline{\partial}\varphi,
\overline{\partial}\psi) + (\vartheta \varphi, \vartheta \psi)$;  
then $b$ is a positive semi definite form and $b(\varphi \varphi)$
defines a semi-norm on $W^{p,q}(X,E)$. the map 
$$
j:W^{p,q} (X,E) \rightarrow W_b^{p,q}(X,E)
$$
where the latter space is $W^{p,q} (X,E)$ provided with
the (not necessarily Hausdorff) topology defined by the seminorm
$b(\varphi, \varphi)$ and $j$ is the identity is clearly continuous: in
fact, $b(\varphi, \varphi) \leqslant N(\varphi)$. 

\begin{definition}%% 1.4
  We say that $E$ is ${W}^{p,q}$ elliptic with respect to the
  hermitian metric on $X$ and the and the hermitian metric along the
  fibres of $E$, if $j$ admits a continuous inverse. 

  In particular, the $W^{p,q}$-ellipticity of $E$ implies that
  $b(\varphi, \varphi)^{\frac{1}{2}}$ is actually a norm. In fact, there
  is a $K>0$ such that $N(\varphi)^{2} \leqslant K b(\varphi,
  \varphi)$. Since by definition, $N(\varphi) = b(\varphi,
  \varphi) + \| \varphi\|^{2}, W^{pq}$-ellipticity is
  equivalent to following: there is a constant $C > 0( = K-1$
  when $K$ is as above) such that 
  \begin {equation*} 
    \| \varphi\|^{2} \leqslant C b (\varphi, \varphi) = C(\|
    \overline{\partial} \varphi \|^{2} + \| \vartheta \varphi
    \|^{2}).  \tag{1.2}\label{eq1.2}
  \end{equation*}
  We shall call $C$ a $W^{p,q}$-ellipticity constant.
\end{definition}

\begin{prop}\label{chap1:prop1.2}%%% 1.2
  Assume given a hermitian metric on $X$ and a hermitian metric along
  the fibres of $\pi: E \rightarrow {X}$. Suppose further
  that $E$ is $W^{pq}$ -elliptic with reference to these
  hermitian metrics.  
  Then there is a linear map $G : \mathcal{L}^{p,q} (X,E)
  \rightarrow W^{p,q} (X, E)$ such that, for 
  $f \in \mathcal{L}^{p,q} (X,E), \varphi \in
  W^{p,q} (X,E)$, we have 
  $$
  (f, \varphi) = (\overline{\partial} \;Gf, \;\overline{\partial}
  \varphi) + (\vartheta \,Gf, \,\varphi).
  $$ 
\end{prop}

The\pageoriginale linear map $G$ is continuous more exactly:
\begin{equation*}
  {b (Gf, Gf)} \leqslant C \|f\|^{2} \tag{1.3}\label{eq1.3}
\end{equation*}
$C$ being the constant which appears in (\ref{eq1.2}). 
Moreover $Gf$ is uniquely determined by the above formula.

\begin{proof}
  $\varphi \rightsquigarrow(\varphi,f)$ defines a linear form on
  $W^{p,q}(X,E)$ which is evidently continuous. By the Riesz
  representation theorem (since $E$ is  $W^{p,q}$ elliptic,
  $b(\varphi,\psi)$ defines a Hilbert space 
  structure on ${W}^{p,q}(X,E)$ equivalent to that defined by $N$),
  there is a unique element $Gf \in W^{p,q}
  (X,E)$ such that 
  $$
  (f, \varphi) = \,b (Gf, \,\varphi) = (\overline{\partial} \, Gf,\,
  \overline{\partial} \varphi) + (\vartheta Gf, \vartheta
  \varphi).
  $$ 

  We have, now
  \begin{align*}
    b(Gf, Gf) & = (\overline{\partial} Gf,
    \overline{\partial} Gf) + (\vartheta Gf, \vartheta Gf).\\ 
    &= (f, Gf)~\text{ by the above equation.}
  \end{align*} 
  Hence $b (Gf, Gf)^{2} \leqslant \| f\|^{2}. \| Gf\|^{2} \leqslant C
  \| f\|^{2} b (Gf, Gf)$ in view of $W^{{pq}}$-ellipticity. That is,
  $$
  b (Gf, Gf) \leqslant C \| f \|^{2}. 
  $$
  We obtain
  $$
  N (Gf)^{2} \leqslant (C+1) \| f\|^{2}. 
  $$
  This completes the proof of Proposition \ref{chap1:prop1.2}.
\end{proof}

\begin{coro*}
  $Gf = f$ (in the sense of distributions).
\end{coro*}

\begin{proof}
  For $u\in \mathscr{D}^{p,q} (X,E)$, we have
  $(f,u) = (\overline{\partial} Gf,
    \overline{\partial}u) = (\vartheta Gf, \square u)$. 
 
  This proves the lemma.
\end{proof}

\begin{prop}\label{chap1:prop1.3}%%% 1.3
  Let\pageoriginale $E$ be $W^{p,q}$-elliptic with respect to a given
  metric along the fibers and to a complete hermitian metric 
  on $X$. In the notation of Proposition \ref{chap1:prop1.2} we have
  the following. 
  \begin{enumerate}[\rm (i)]
  \item If $f \; \in \mathcal{L}^{p,q} (X,E)$ and 
    $\overline{\partial}f\in \mathcal{L}^{p,{q+1}} (X,E)$, then 
    \begin{multline*}
      \overline {\partial}{Gf}\in W^{p,{q+1}}  (X,E); \Box
      \overline{\partial} {Gf} = \overline{\partial}f;\\
      \| \vartheta \overline{\partial} {Gf} \|^{2} \leqslant
      \frac{1}{\sigma}  \| \overline{\partial f} \|^{2} +
      \sigma \| \overline{\partial}{Gf} \|^{2} ~\text{ for }~
      \sigma > 0.
    \end{multline*}

  \item If $f\in \mathcal{L}^{p,q-1}(X,E)$ and
  $\vartheta f \in \mathcal{L}^{p,q+1} (X,E)$, then  
    \begin{multline*}
      \vartheta Gf \in W^{p,{q-1}} (X,E); \Box \vartheta Gf = \vartheta f;\\
      || \bar{\partial }\vartheta Gf\; ||^2 \leqslant \sigma ||
      \vartheta \; f ||^2 + \sigma || \vartheta \; Gf\; ||^2
      for \sigma > 0.
    \end{multline*}
  \end{enumerate}
\end{prop}

\begin{proof}
  Since  the  metric is complete, by Theorem {1.1'} $\mathcal{D}^{p,q}
  (X,E)$  
  is dense in the space  $\left\{\varphi\ \mid \varphi \in
  \mathcal{L}^{p,q} (X,E) , \overline{\partial} \varphi \in
  \mathcal{L}^{p,{q+1}} (X,E)\right\}$ 
  provided with the norm $P(\varphi)^{2}= \| \varphi \|^{2}+\|
  \overline{\partial} \varphi \| ^{2}$. Let 
  $f\in \mathcal{L}^{p,q} ~(X,E )$ and  $\overline{\partial} {f}
  \in \mathcal{L}^{p,{q+1}}\break (X,E)$. Then there is a sequence 
  $f_n \in \mathcal {D}^{p,q}(X, E)$ such that  $\| f_{n}-f \|
  \rightarrow  0,\| \overline {\partial} f_n-\overline {\partial f}\|
  \rightarrow 0$ as $n\rightarrow \infty$. We have moreover, from
  Proposition \ref{chap1:prop1.2}, that 
  
  $N (G \varphi)^2 \leqslant{K} \| \varphi \| ^{2}$ for
  $\varphi \in \mathscr {L}^{p,q}(X,E)$. Hence 
  $$
  \displaylines{\hfill 
  \| Gf_n- Gf_m \| \leqslant K \| f_n -f_m \|\hfill \cr 
  \text{and}\hfill    
  \| \overline{\partial} Gf_{n}- \overline{\partial}
  Gf_n \| + \| \vartheta Gf_n- \vartheta Gf_n\|
  \leqslant K\| f_n-f_m\|. \hfill }
  $$

Applying now Stampachia inequality (Corollary \ref{chap1:coro2},
Theorem \ref{chap1:thm1.2}) to 
$\overline{\partial}Gf_n$, and taking into account the fact
that  $\Box$ and $\overline{\partial}$ commute, 
we obtain, for $\sigma > 0$,
$$ 
\| \vartheta \overline{\partial} (Gf_n - Gf_m) \|^{2} \leqslant
\frac{1}{\sigma} \bar{\partial}(f_\nu-f_\mu) \|^2 + \sigma \|
(\overline{\partial} G(f_\nu- f_\mu)\|^2
$$

Since the right hand side tends to 0 as $n, m\rightarrow \infty$, it
follows that $\bigg\{\vartheta \overline{\partial}f_n\bigg\}$ is a
Cauchy-sequence in $\mathcal{L}^{p,q}(X,E)$. It is 
immediate\pageoriginale then that $\vartheta \overline{\partial}Gf =
  \lim\limits_{n \rightarrow \infty}  \vartheta \overline
  {\partial} Gf_n \in  \mathcal{L}^{p,q} (X,E)$. On the 
other hand $\overline{\partial} (\overline{\partial} Gf) =0$. Hence
$\overline{\partial} Gf \in W^{p,{q+1}} (X,E)$. The equation 
$\Box \overline{\partial} Gf= \overline{\partial} f$ follows from the
fact that $\overline{\partial} ~~\text{and} ~~\Box$ commute and the
fact that $ \Box Gf=f$. The last inequality is the Stampachia inequality
applied to $\overline{\partial} Gf$. The proof of Part (ii) of the
proposition is entirely analogous. 
\end{proof}

\begin{theorem}\label{chap1:thm1.3} %%% 1.3
  Let $\pi : E \rightarrow X$ be a  holomorphic vector-bundle
  which is $W^{p,q}$-elliptic with respect to a complete hermitian
  metric on $X$ and 
  a hermitian metric along the fibres of $E$. Then if $q> 0$,
  given $f \in \mathcal{L}^{pq} (X,E)$ with  $\overline
       {\partial}  f=0$, there is a unique $x \in W^{pq}(X,E)$
         such that $f=  \overline{\partial} \vartheta x$ and
       $\overline{\partial}x=0$. Moreover, we have  
       $\| \vartheta x \|^{2} \leqslant C \| f \|^2$, $C$ being a
       $W^{p,q}$-ellipticity constant. 
\end{theorem}

\begin{proof}
  We set $x=Gf$. We have then $\Box  Gf=f$. (Corollary to Proposition
  \ref{chap1:prop1.2}). Clearly $\overline {\partial}f=0 \in
  \mathcal{L}^{p,{q+1}} (X, E)$. Hence by (i) of proposition
  \ref{chap1:prop1.3}, we have $ 
  \overline{\partial}Gf \in W^{p, q+1} (X,E)$ and further  
  $\| \vartheta \overline{\partial} Gf\|^2 \leqslant \sigma \|
  \Box  \overline {\partial} Gf\|^2 + \frac{1} {\sigma}\|
  \overline{\partial} Gf \|^2 $. On the other hand 
  since  $\Box$ and $\overline{\partial}$ commute, it follows again from
  the Corollary to Proposition \ref{chap1:prop1.2}, that $\Box \overline{\partial}
  Gf=0$. Since $\sigma$ 
  is arbitrary, $\vartheta \overline {\partial}x=0$. It follows that
  $(\overline {\partial}x, \overline {\partial}x)= (x,\vartheta
  \overline{\partial } x)=0$ (note that  $\overline{\partial}x \in
  W^{p,{q+1}}(X,E)$: Proposition \ref{chap1:prop1.3}). Hence $\overline{\partial}
  x=0$. Now $\Box = \overline{\partial} \vartheta + \vartheta \overline
  {\partial}$ so that $\Box Gf=f$ leads to 
  $(\overline{\partial} \vartheta + \vartheta \overline{\partial}) 
  Gf=f$ that is, $\overline {\partial} \vartheta Gf=f$. 
\end{proof}

Finally\pageoriginale (\ref{eq1.3}) yields
$$ 
\| \vartheta x \|^{2}\leqslant C \| f\|^{2}.
$$

The uniqueness of $x$ satisfying $\overline {\partial} \vartheta x=f$
and $\overline{\partial}x=0$ is easily checked. 

This completes the proof of the theorem.

It follows from the regularity theorem for elliptic systems that , if f
$\in  C^{p,q}$ $(X,E) \cap \mathcal{L}^{p,q} $ then $Gf$ (can be
modified on a null set so that it) will be of class $C^{\infty}$ on $X$.

In particular we have the 

\begin{coro*}
  If $f \in \mathcal{D}^{p,q} (X,E)$ and $\overline {\partial}
  f=0$ then there exists $\psi  \in C^{p,{q-1}}\break (X,E)$ such that $
  \overline{\partial} \psi =f$  
\end{coro*}

\begin{remark*}
  The corollary above clearly implies the following.
\end{remark*}

If $E$ is $W^{pq}$-elliptic with respect to a complete hermitian
metric, then the natural map 
$$
H^q_k (X,\Omega^p (E))\rightarrow H^q(X, \Omega^p(E))
$$
where the  left-side stands for the $q^{th}$ cohomology with compact
supports of $X$ with values in $\Omega^p(E)$, is the trivial map $\alpha
\rightsquigarrow 0$ for every $\alpha \in H^q_k (X,
\Omega^p  (E))$. 

\begin{remark*}
  Let $ \MG  =\bigg\{U_i\bigg\}$ be a covering of $X$ such that
  $\pi : \; E \rightarrow \; X $ is defined with respect
  to $\MG$ by holomorphic transitions $ e_{i_j} : U_i \cap U_j \rightarrow
  \; GL(m, \mathbb{C} ) (m=\text{ rank of }~ E)$.
\end{remark*}

 Let $\bigg\{h_i\bigg\}_{i \in I}$ be a hermitian metric along the fibers of $E$;
 then $h_i$ is a  $C^{\infty}$ function on $U_i$ whose values are
 positive definite hermitian matrices. 

We have\pageoriginale on $U_i\cap U_j$
$$ 
\displaylines{\hfill 
  h_i =^t_{~\overline{e}_{ji}} h_j e_{ji}, \hfill \cr 
  \text{and therefore } \hfill 
  t_{h^{-1}_i}=^t (^{\overline{{t} e^{-1}_{ji}}})^t h^{-1}_j\;{}^t
  e^{-1}_{ji}\phantom{wwwwwi}\hfill } 
$$

That means that $\left\{{}^t h^{-1}_i \right\}$ defines a matric along
the fibres of the dual bundle $E^*$. 

We note here for future reference the following identities, holding
with respect tot he metrics $\{h_i\}$ on $E$ and
$\{t_{h^{-1}_i}\}$ on $E^*$.  They follow immediately from the
definitions of the operators involved. 
\begin{align*}
  A_{E^*}(* \# \varphi,* \# \psi) & =  A_E(\varphi, \psi)
  \tag{1.4}\label{eq1.4}\\ 
  \overline {\partial} * \# \varphi & = (-1)^{p+q} * \# \vartheta_E
  \varphi,(\varphi, \psi \in C^{pq}(X,E)) \tag{1.5}\label{eq1.5}\\ 
  \vartheta_{E^*} \; * \# \varphi &= (-1)^{p+q+1} * \#
  \overline{\partial} \psi, \tag{1.6}\label{eq1.6}\\
 \Box_{E^*} *  \# \varphi & = * \# \Box_E \varphi. \tag{1.7}\label{eq1.7}
\end{align*}

As a corollary we have that $*\#$ defines an isometry of
$\mathcal{L}^{pq} (X,E)$ onto \break $\mathcal{L}^{ n-p \; n-q}(X,E)$
which maps $W^{pq}$ isometrically onto $W^{n-p \; n-q} (X,E^*)$. 

Furthermore, if $E$ is $W^{pq}$-elliptic with respect to the
metric $\{{h_i}\}$ on $E$, then $E^*$ is $W^{n-p \; n-q}$-
elliptic with respect to the metric $\{t_{h^{-1}_i}\}$ on $E^*$
the $W$-ellipticity constants being the same. 

\section{Carleman inequalities}\label{chap1:sec4}%%% 4

We\pageoriginale will now formulate certain further conditions on the
vector bundles and show how these more stringent conditions lead to
stronger vanishing theorems than the one above.  

\textit{We assume always that the hermitian metric denoted $ds^2$ on $X$ is
complete.} Let the given metric on $E$ be denoted by $h$.   

\begin{description} 
\item[$C_1$.] There is given a $ C^{\infty}$ function
  $\phi: X\rightarrow \mathbb{R}^+$. 

\item[$C_2$.] For every non decreasing convex $C^{\infty}$-function
  $\lambda: \mathbb{R} \rightarrow \mathbb{R}$.  

  $E$ is $W^{pq}$-elliptic with respect to $(ds^2, e^{\lambda
  \phi}_h)$.

\item[$C_3$.] The $W^{pq}$-ellipticity constant is independent of.
\end{description}

That is, there is $C > 0$ independent of $\lambda$ such that

$\parallel f \parallel^2_\lambda \leqslant C \bigg\{{\parallel
  \overline {\partial} f\parallel^2_\lambda + \parallel
  \vartheta_\lambda f\parallel^2_\lambda}\bigg\}$ for $f \in
\mathscr{D}^{pq} (X,E)$ where $\parallel \; \parallel_\lambda$ stands for
the norm with respect to ($ds^2, e^{\lambda(\phi)}$. 
$h$), and $\vartheta_\lambda$ denotes the $\vartheta$-operator with
respect to these metrics.  

Conditions $C_1, C_2, C_3$ imply $ C'_1,
C'_2, C'_3$ below.  This weaker set of conditions are sufficient
for the ``vanishing theorems'' we will now prove. 

\begin{description} 
\item[$C'_1$.] There is given a $C^\infty$ function
  $\phi: X\rightarrow \mathbb{R}^+$ (this is the same as $C_1$).  

\item[$C'_2$.] Given $C_o>0$, there is a non-decreasing
  $C^\infty$-function $\lambda:\mathbb{R} \rightarrow \mathbb{R}$
  that $\lambda(t)=0$ for $t\leqslant C_o$ and
  $\lambda(t)> 0$ for $t > C_o$, such that, $E$ is
  $W^{pq}$ elliptic with respect to $(ds^2,e^{\nu\lambda(\phi)} h)$
  for every positive integer $\nu$.  

\item[${ C'_3}$.] The\pageoriginale $W^{pq}$ ellipticity constant is
  independent of $\nu$ that is, there is a $C > 0$ such that for $f
  \in \mathscr{D}^{pq}(X,E)$ 
  $$ 
  \parallel f \parallel^2_\nu \leq C \left\{{\parallel \overline
    {\partial} f\parallel^2_\nu + \parallel\vartheta_\nu
    f\parallel^2_\nu}\right\},
  $$
  where $\parallel \; \parallel_\nu$ stands for $\parallel \;
  \parallel_{\nu \lambda}$ and $\vartheta_\nu$ for $\vartheta_{\nu
    \lambda}$. 
\end{description}

\begin{lemma}\label{chap1:lem1.6}%%% 1.6
  Assume given a constant $ C_o>0$ and that the condition $
  C'_1,  C'_2,  C'_3$ above are satisfied ($C_o$ in
  condition $C'_2$ is taken as the above constant). For $\lambda$ as
  in condition $C'_2$, let $\mathcal{L}^{p,q}_{\nu}(X,E)$ denote
  the space of $E$-valued forms on $X$ which are square-summable with
  respect to $(ds^2, e^{\nu \lambda(\phi)}. h)$.  Then
  for f$\in {\underset{\nu}{\cap}} \mathcal{L}^{p,q}_\nu(X,E) (q>0)$
  such that $\overline \partial f=0$, there exist $\Psi _\nu$ for every
  integer $\nu \geq 0$ such that $\Psi \nu \in
  \mathcal{L}^{p,q-1}_\nu(X,E)$, $\overline{\partial} \Psi_\nu =f$ and
  $\parallel \Psi_\nu\parallel_\nu \le C \parallel f \parallel_\nu$ (We
  assume that $ds^2$ is complete). 
\end{lemma}

\begin{proof}
  This follows from Theorem \ref{chap1:thm1.3}.
\end{proof}

\begin{theorem}\label{chap1:thm1.4} %% 1.4
  Assume that conditions $ C'_1, C'_2,  C'_3$ are
  satisfied. Then for every square-summable form f of type $(p,q)(q>0)$
  having compact support such that $\overline {\partial} f=0$, there
  exists $\Psi \in \mathcal{L}^{p,q-1}(X,E)$ such that, $f=\overline
  {\partial} \Psi$, $\parallel \Psi \parallel^2 \leq C \parallel f
  \parallel^2$ and (support of $\Psi$)  
$$
\subset\{x\|\phi(x)<(\sup \phi(y))  \; y \in  \text{ support } f)\}.
$$ 
\end{theorem}

\begin{proof}
  Let $C_o = \sup \phi (x), x \in$ support $f$.  Then we have 
  \begin{multline*}
  \parallel f \parallel^2_\nu = \int_X e^{\nu \lambda (\phi)} A
  (f,f)dx=  \int\limits_{\Supp.\;f} e^{\nu \lambda(\phi)} 
  A(f,f)dX\\ 
  =\int\limits_{\Supp f}  A(f,f) dx= \int\limits_{X} A(f,f)dX =
  \parallel f \parallel^2
  \end{multline*}

Since\pageoriginale $\nu \lambda (\varphi (x))  =  0 $ on the support
of $f$. It follows  that  f $ \epsilon \underset{\nu}{\cap}
\mathcal{L}^{p,q}_{\nu} (X,E) $ 
 for every integer  $ \nu > 0$. Hence by Lemma \ref{chap1:lem1.6}, we can find
 $\psi_{\nu} \epsilon$  $ \mathcal{L}^{p,q-1}_{\nu{_{2}}}  (X,E)$ 
 for every  $ \nu \epsilon \mathbb{Z}^{+} $ such that
 $\overset{-}{\partial} \psi_{\nu}  = f$ and  $\mid\mid \psi \mid\mid^2_\nu 
 \leq  C \mid\mid f \mid\mid^{2}$.  
  On the other hand, we have  $ \mid\mid \psi_{\nu} \mid\mid_{\nu}
  \ge  \mid\mid  \psi_{\nu} \mid\mid $ so that,  
  $\mid\mid \psi_{\nu} \mid\mid^{2} \leq    \mid\mid \psi_{\nu}
  \mid\mid^{2}_{\nu} \leq C \mid\mid f \mid\mid^{2} = C_{1}$  say. 
  It follows that we can assume  (by passing  to a  subsequence  if
  necessary) that  $\psi_{\nu}$  converges weakly in
  $\mathcal{L}^{p,q-1} (X,E) $ to a limit $\psi$. We have   
  $\mid\mid \psi \mid\mid < C \mid\mid f \mid\mid$. On the other hand,
  for every $\varepsilon  > 0$, 
  $$  
  \int_{\phi(x) > c_{0}+ \varepsilon}  e^{\nu \lambda (\phi)}  { A}
  (\psi_{\nu},\psi_{\nu}) d X \leq C_{1} 
  $$
  and since  $\lambda $ is non-decreasing, we have,
  $$ 
  e^{\nu \lambda (C_{0}+\varepsilon)} \underset{\varphi \ge C_{0}+
    \varepsilon}{\int}  A (\psi_{\nu},\psi_{\nu}) dX \leq C_{1}.
  $$
    
  It follows that $\underset{\varphi \ge C_{0}+\varepsilon}{\int} A
  (\psi_{\nu},\psi_{\nu}) dX$ tends to zero and hence
  $\psi_{\nu}\rightarrow 0 $ 
  almost everywhere in $ \varphi  \ge C_{0}+ \varepsilon$. (The integrand is
  positive for all $\nu$.)  
  Hence $ \psi =0 $ on $\{x \mid \varphi (x) \ge  C_{0} + \varepsilon \} $
  for every  $\varepsilon$. Hence   
  support  $ \psi \subset \{x \mid \phi' (x) \leq C_{0} \} $. Finally for
  \begin{align*}
    u  \in \mathscr{D}^{n-p,n-q} (X,E^*),(-1)^{p+q} \langle \psi,
    \overset{-}{\partial}u \rangle &= (-1)^{p+q}   \int \psi \wedge
    \overset{-}{\partial}u \\  
    &=Lt (-1)^{p+q} \int \psi_{\nu} \wedge \overset{-}{\partial}u\\
    &=\big \langle f,u \big \rangle.
  \end{align*}
     
  Hence $\bar {\partial} \psi =f$ in the sense of distributions. This
  completes the proof of the theorem.  
\end{proof}

\begin{remark*}
  The\pageoriginale proof contains the following lemma (which has no
  connection with $W^{p,q}$-ellipticity.  
\end{remark*}

\begin{lemma}\label{chap1:lem1.7}%% 1.7
  Suppose given a $C^{\infty} $  non-decreasing function  $
  \lambda:\mathbb{R} \rightarrow \mathbb{R} $    
  such that $\lambda (t) = 0$ for $t\leq C_{0} ,  \lambda (t) >0$
   for $t  >  C_{0}$, and a $C^{\infty}$ function $\varphi :
  X \rightarrow \mathbb{R}$. Suppose further that  there is given an   
  $f \in \mathcal{L}^{pq} (X,E)$  such that sup  $ \varphi (x)$ on the
support of $f$ is $C_{0}$. If for every integer $\nu > 0$ 
there exists  $\psi_{\nu}$   such that  $\bar{\partial} \psi_{\nu} =
f$ and $\mid\mid \psi_\nu  \mid\mid^{2}_{\nu} \leq C \mid\mid f \mid\mid^{2}$
for a constant $C$ independent of  $\nu$, then there  exists  $ \psi \in
\mathcal{L}^{p,q-1}(X,E) $  such that $ \bar{\partial} \psi = f,
\mid\mid \psi \mid\mid^{2} \leq C \mid\mid f \mid\mid^{2} $ and  
support $ \psi \subset \{x \mid \varphi (x) \leq C_{0} \}$. (Here by)
$\mid\mid \psi_{\nu} \mid\mid^{2}_{\nu} $   
we  mean  $ \int e ^{\nu \lambda (\phi)} A (\psi_{\nu},\psi_{\nu})
dX)$.
\end{lemma}

Suppose now that the function  $\phi$ in $C'_{1}$  satisfies the
following additional condition. 

$C'_{4}$.  For every  $ C > 0$, the set  $ \{ x \mid \phi(x) < C\}
\subset \subset X$. 

\begin{remark*}
  If  $C'_{4}$ is  satisfied in addition to $C'_{1}, C'_{2}$ and
  $C'_{3}$ then in Theorem \ref{chap1:thm1.4}, we can assert that the support
  of  $\psi$ is compact. 
\end{remark*}

Here we can state, as a corollary to Theorem \ref{chap1:thm1.4} the following 
Theorem {$1.4'$}. If the hypothesis  of Theorem  \ref{chap1:thm1.4}
are fulfilled and 
if $C'_{4}$ holds, then 
$$ 
H^q_{k} (X,\Omega ^{p} (E))=0.
$$

\begin{lemma}%% 1.8
  Let $ V \subset \mathcal{L}^{p,q} (X,E)$ be the set V=
  $V^{p,q}(X,E)=   \{ \varphi  \in \mathcal{L}^{pq}(X,E) \bigg|$ 
  there exists $ \psi \in \mathcal{L}^{p,q-1}(X,E) $  such that  $
  \overset{-}{\partial} \psi = \varphi \}$ 
  and\pageoriginale $N =N^{p,q}(X,E) = \{\varphi \in
  \mathcal{L}^{p,q}(X,E)\mid  \vartheta \varphi =0 \}$.  
  Then  $N$ is the orthogonal complement of $V$ provided that the metric
  is complete. 
\end{lemma}

\begin{proof}
  Let $\rho \in \mathcal{L}^{p,q}(X,E)$. Then $ \rho  \in \{$
  orthogonal  complement of $ V\} $ if  
  and only if  $(\rho ,\overset{-}{\partial } \psi)=0 $  for every
  $\psi \in \mathcal{L}^{p,q-1} (X,E)$ with  
  $\overset{-}{\partial} \psi \in  \mathcal{L}^{p,q-1}(X,E)$. Since
  the metric  is complete, this is equivalent to
  $(\rho,\overset{-}{\partial} \psi) = 0 $ for every $\psi \in
  \mathscr{D}^{p,q-1}(X,E)$. Hence the lemma.
\end{proof}

\begin{theorem}\label{chap1:thm1.5}%%  1.5
  Assume conditions  $C'_{1},C'_{2},C'_{3} $ to hold for 
  $\pi : E \rightarrow X$.  Let $f \in \mathcal{L}^{p,q+1}(X,E) $ be
  such that sup  $\varphi (x)$  on the support of f is $C_{0}$. 
  Suppose further that $(f,g) = 0$  for all $g$ which belong to some  $
  \mathcal{L}^{p,q+1}_{\nu},$ and are such that  
  $\vartheta_{\nu} g= 0 (\nu = 1,2,....)$(\footnote{In view of the
    choice of $\lambda$, we have $(f,g) = (f,g)_\nu$.}).  Then there exists
  $\psi \in \mathcal{L}^{p,q}(X,E)$ 
  such that  $ \overset{-}{\partial} \psi = f $ and support  $\psi
  \subset \{ X \mid \varphi (X) \leq C_{0} \}$. 
\end{theorem}
  
Let $V^{p \; q+1}_{\nu}  $ be the  space  $ V^{p \; q+1} $ constructed with
respect to the metrics  \break $(e^{ \nu \lambda (\phi)} h,ds^{2}) $. Since  
$ \lambda =0$ on $\supp f$, then  $f \epsilon  \mathcal{L}^{pq}_{\nu}
(X,E)$ for all $\nu > 0$.  
  
By the previous lemma
$$ 
f \in  V_{\nu}^{\frac{}{p, q+1}} \; ~\text {for}~ \;  \nu = 1,2,...
$$
  
First assume that  
$$ 
f \in V_{\nu}^{p \; q+1} \text{ for a particular } \nu. 
$$

In this case, there exists  $\varphi_{\nu} \in
\mathcal{L}^{p,q}_{\nu}(X,E)$ such that $f = \overset{-}{\partial}
\varphi_{\nu}$ 
Now by Proposition \ref{chap1:prop1.3}, there  is an  $X_{\nu} =
G_{\nu} \varphi_{\nu} \in   W^{p,q}_{\nu}(X,E)$.  

Such\pageoriginale that $\Box_\nu x_\nu = \varphi_\nu$. We now set $\psi_\nu
= \vartheta_\nu \bar{\partial}$. By Proposition \ref{chap1:prop1.3} $\psi_\nu \in 
\mathcal{L}^{pq}_\nu (X,E)$. Further we have $\bar{\partial}\psi_\nu =
\bar{\partial}\vartheta_\nu \bar{\partial}x_\nu =
\bar{\partial}(\bar{\partial} \vartheta_\nu  + \vartheta
\bar{\partial})x_\nu = \bar{\partial}\varphi_\nu = f$. Since
$\vartheta_\nu \psi_\nu = 0$ , then by Theorem \ref{chap1:thm1.1} $\psi_\nu \in
W^{pq}_\nu(X,E)$. On the other hand, in view of
$W^{p,q}_\nu$-ellipticity,   
$$
 \Arrowvert \psi_\nu \Arrowvert^2_\nu \leq C(\Arrowvert
\bar{\partial} \psi_\nu \Arrowvert^2_\nu + \Arrowvert \vartheta_\nu
\psi_\nu \Arrowvert^2_\nu) = C \Arrowvert f \Arrowvert^2_\nu =
C_1
$$
 say $C_1$ is independent of $\nu $ since $\lambda (\varphi(x)) =
0$ for $x\in $ support $f$. 

Suppose now $f \in V^{\overline{pq+1}}_\nu$, then there exists a
sequence f$^i_\nu \in V^{pq+1}_\nu$ such that $\Arrowvert$f$^i_\nu $ -
f $\Arrowvert_\nu \rightarrow 0$. Then choosing $\psi^i_\nu \in
W^{pq}_\nu  (X,E)$ is above for each $f^i_\nu (\in V^{pq+1}_\nu)$ we
have  
\begin{align*}
& \bar{\partial} \psi_\nu^i = f^i_\nu\\
& \qquad \Arrowvert \psi_\nu^i \psi_\nu^2 \leqslant C \Arrowvert f^i_\nu
\Arrowvert^2_\nu \; \text{ and } \;  \Arrowvert \psi^i_\nu - \psi^j_\nu
\Arrowvert^2_\nu \leqslant C \Arrowvert f^i_\nu  - f^l_\nu
\Arrowvert^2_\nu. 
\end{align*}

Hence $\psi^i_\nu$ converge to a limit $\psi_\nu$ in
$\mathcal{L}^{p,q}_\nu(X,E)$. Clearly $\bar{\partial}\psi_\nu =
F$. Further, from $\Arrowvert \psi^i_\nu \Arrowvert^2_\nu \leqslant C
\Arrowvert f^i_\nu \Arrowvert^2_\nu$, we deduce that $\Arrowvert
\psi_\nu \Arrowvert_\nu^2 \leqslant  C \Arrowvert f
\Arrowvert^2$. Now the proof follows from Lemma \ref{chap1:lem1.7}.  

The conditions $C_1^*$ , $C^*_2$ , $C^*_3$ below are dual to
$C_1'$ , $C'_2$ , $C'_3$ : that is if $C_1^*$ ,
$C^*_2$ , $C^*_3$  hold for $E$-valued $(p,q)$-forms, then $C'_1$,
$C'_2$ , $C'_3$ hold  for $E^*$ valued $(n-p,n-q)$ forms. 
\begin{enumerate}
\item[$C^*_1$.] $(= C'_1 = C_1)$ There is given a $C^\infty$ function
$\phi : X\rightarrow \mathbb{R}^+$.

\item[$C^*_2$.] For every $C_o \geqslant
0$, there exists a non-decreasing $C^\infty$-function, $\lambda :
\mathbb{R}\rightarrow \mathbb{R}$ such that $\lambda(t) = 0$ if
$t\leqslant C_0$ and $\lambda (t) > 0$ for $t >  C_0$ and $E$ is
$W^{pq}$-elliptic with respect to $(ds^2 ,e^{\nu \lambda(\phi)}h)$

\item[$C^*_3$.] The $W^{p,q}$ ellipticity constant is independent of $\nu =
1,2,\ldots$   
\end{enumerate}

Then\pageoriginale we have analogous to Theorem \ref{chap1:thm1.5}, the following 


\begin{theorem}\label{chap1:thm1.6}%%% 1.6
  Assume $C_1^*$ , $C^*_2$ , $C^*_3$. Then if $ f \in
  \mathcal{L}^{p,q-1} (X,E)$, is such that $\sup \phi (X)$ on support of
  $f= C_0$, and if further $(f,u) = 0$ for every u $\in
  \mathcal{L}^{p,q-1}_{-\nu} (X,E)$, such that $\bar{\partial} u=0$,
  then there exists $\chi \in \mathcal{L}^{p,q} (X,E)$ such that $f=
  \vartheta \chi$ and support $\chi \subset\{X|\phi(X)\leq  C
  \}$. 
\end{theorem}

\begin{theorem}\label{chap1:thm1.7}%%% 1.7
  Let $T= T^{p,q-1}$ be a distribution valued form type $(p,q-1)$ (that
  is a current of type $(p, q-1))$. Suppose further that $\bar{\partial}
  T \in C^{p,q}(X,E)$ and let $K=$ support of $T$. Then for any
  neighbourhood $U$ of $K$, there is a form $\eta \in C^{p,q-1}(X,E)$
  such that $\bar{\partial} T = \bar{\partial \eta}$ and support
  $\eta \subset V$. In particular if $K$ is compact, we can find an
  $\eta$ with the above property. 
\end{theorem}

\begin{proof}
  We recall (Prerequisites, 4) that we have fine resolutions 
$$
  \displaylines{\hfill 
    0 \rightarrow \Omega^p \rightarrow
    A^{p,o}\overset{\bar{\partial}}\longrightarrow A^{p,1} \rightarrow
    \ldots \overset{\bar{\partial}}\longrightarrow A^{p,n}\rightarrow 0
    \hfill \cr 
    \text{and} \hfill 0 \rightarrow \Omega^p \rightarrow
    K^{p,o}\overset{\bar{\partial}}\longrightarrow K^{p,1} \rightarrow
    \ldots \overset{\bar{\partial}}\longrightarrow K^{p,n}\rightarrow
    0.\hfill } 
$$
  Further there is canonical injection $A^{p,q}\rightarrow K^{p,q}$
  which is compatible with the operator $\bar{\partial}$. Hence by a
  standard result on cohomology of sheaves the induced map of
  complexes  
  $$
  \underset{q \geqslant 0}\sum \Gamma_\Phi (X,A^{p,q}) \rightarrow
  \sum \Gamma_\Phi (X,K^{p,q})
  $$ 
  is a homotopy equivalence of complexes for any para compactifying
  family $\Phi$ of closed sets on $X$. We consider in particular the
  family of all closed sets of $X$ which are contained in $U$. This is
  evidently a paracompactifying family and theorem follows from the general
  result stated above (for a more detailed proof, see \cite{key2},
  97-99). 

Theorems \ref{chap1:thm1.5}\pageoriginale and \ref{chap1:thm1.7}
together enable us to prove  
\end{proof}


\begin{theorem}\label{chap1:thm1.8}%%% 1.8	
  Let $\pi : E\rightarrow K$ be a holomorphic vector bundle. Assume
  that for a suitable complete hermitian metric $ds^2$ on $X$ and a
  suitable hermitian metric h along the fibres of $E$, conditions
  $C'_1$, $C'_2$, $C'_3$ and $C'_4$ hold. Then the image
  $\bar{\partial}\mathscr{D}^{pq}(X,E)$ of $\bar{\partial}$ in
  $\mathscr{D}^{p,q+1} (X,E)$ is a closed subspace for 
  the usual topology on $\mathscr{D}^{pq} (X,E)$. Hence
  $H^{q+1}_k(X,\Omega^p(E))$ has a structure of a separated
  topological vector space. 
\end{theorem}

\begin{proof}
  The ``usual'' topology on $\mathscr{D}^{pq}(H,E)$ may be described
  as follows. Let $\{ K_\nu \}$ be an increasing sequence of compact sets such
  that $\bar{K_\nu}\subset \overset{o}K{_\nu+1}$ and $U K_\nu =
  K$. Further, let $\MG = \{U_i \}_{i\in  I}$ be a locally
  finite covering of $X$ by open sets such that each $U_i$ is a
  relatively compact open subset of a coordinate open set $V_i$ on
  $X$. Let $\MG_{\nu} = \{ U_i \arrowvert U_i\cap K_\nu
  \neq \phi  \}$. We topologize each $\mathscr{D}^{pq}(K_\nu ,E) =
  \{ \varphi | \varphi \in \mathscr{D}^{pq} (X,E), \supp \varphi
  \subset_\nu \}$ as follows: for $U_i \in \MG_\nu , \varphi |
  U_i$ may be regarded as a $C^\infty$ vector-valued function
  $\varphi_i$ on $U_i$ ; a fundamental system of neighbourhoods of
  zero in $\mathscr{D}^{pq}(K_\nu , E)$ is given by,  
  \begin{equation*}
    \left\{ \varphi \in \mathscr{D}^{pq}(K_\nu ,E)\big |
    \arrowvert {}^{\bar{\alpha}} \varphi_i \big | < \varepsilon_\alpha,
    ~\text{ for every }~ U_i \in \MG_\nu \right\}  
  \end{equation*}	
  where $\{\varepsilon_\alpha \}$ is an arbitrary family of positive  reals. 
  
$\mathscr{D}^{pq}(K_\nu ,E)$ is a Frechet space. There is a natural injection 
$$
  \mathscr{D}^{pq}(K_\nu , E)\rightarrow \mathscr{D}^{pq}(K_{\nu+1},
  E). 
$$	
\end{proof}
		
The\pageoriginale image of $\mathscr{D}^{pq}(K_{\nu},E)$ is a closed
subset of 
$\mathscr{D}^{pq}(K_{\nu+1},E)$. The induced topology on the image
coincides with the topology of $\mathscr{D}^{pq}(K_{\nu},E)$. This
shows that $\mathscr{D}^{pq}(X,E)$ is a strict inductive 
limit of the Frechet spaces
$\mathscr{D}^{pq}(K_{\nu},E)$ \cite{key8} {66-67}; \cite{key9}
(225-227). The ``usual'' 
topology of $\mathscr{D}^{pq}(X,E)$ is the inductive limit topology. 

A subset of $\mathscr{D}^{pq+1}(X,E)$ is closed if, and only if,
it is sequentially closed (\cite{key19}, 228). Therefore it will be sufficient
to prove that $\bar{\partial}\mathscr{D}^{pq}\break(X,E)$ is sequentially
closed in $\mathscr{D}^{pq+1}(X,E)$. 

Let $\{\varphi_{i}\}_{i \in \mathbb{N}} $ be a sequence of $(p,q)$-
forms of $\mathscr{D}^{pq}(X,E)$. Such that the sequence
$\{\overline{\partial}\varphi_{i}\}_{i \in\mathbb{N}} \subset
\overline{\partial}\mathscr{D}^{pq}(X,E)$ converges to an element
$\varphi$ of $\mathscr{D}^{pq+1}\break(X,E)$. The sequence $\{ \xi
\varphi_{i}\}$ is a bounded set in $\mathscr{D}^{pq+1}(X,E)$. 

Hence there exists a compact $K_s$ such that
$\overline{\partial}\varphi_{i} \in \mathscr{D}^{pq+1}(K_{s},E)$ for $i
= 1,2, \ldots$, and $\varphi \in \mathscr{D}^{pq+1}(K_{s},E)$
(\cite{key8}, 70; \cite{key19}, 226). 

Let $c_o= \sup \phi (x)$ on $K_s$. Let $\lambda: \mathbb{R} \rightarrow
\mathbb{R}$ be a non decreasing $C^{\infty}$ function such that 
\begin{gather*}
\lambda(t) = 0 ~ \text{for ~t } \leqslant c_o,\\
\lambda(t) ~~ > ~~ 0  ~\text{for ~t} ~~ >  ~~c_o ,
\end{gather*}
and let us consider the spaces $\mathcal{L}^{pq+1}_{\nu}(X,E)=
\mathcal{L}^{pq+1}_{\nu \lambda(\varphi)}(X,E)$ for $\nu =\break 1,2,\ldots$. 

Let $g\in \mathcal{L}^{pq+1}_{\nu}(X,E)$ be such that
$$
\vartheta_{\nu}{g} = 0. 
$$

Then we have 
$$
(\varphi, g) = (\varphi, g)_\nu =
\lim(\overline{\partial}\varphi_{i},g)_{\nu} =
\lim(\varphi_{i},\vartheta_{\nu} g) = 0.  
$$

Thus by theorem \ref{chap1:thm1.5} there exists a $\psi \in \mathcal{L}^{pq}(X,E)$
with compact support\pageoriginale (since we have assumed
$c'_{4}$), such that 
$$
\varphi = \overline{\partial}\psi.
$$

Now, since in addition $\varphi$ is $C^{\infty}, \psi$ may be assumed
to be a $C^{\infty}$ form, i.e. $\psi \in\mathscr{D}^{pq}(X,E)$. Hence
$\in \overline{\partial}\mathscr{D}^{pq}(X,E)$. This completes the
proof of theorem \ref{chap1:thm1.8}.  
