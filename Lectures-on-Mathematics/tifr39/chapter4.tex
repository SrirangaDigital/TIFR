\chapter{Vanishing Theorems}\label{chap4}%%% 4

\setcounter{section}{11}
\section{$q$-complete manifolds}%%% 12

Let\pageoriginale $X$ be a complex manifold and $\phi:X \rightarrow
\mathbb{R}$ a $C^\infty$-function. The Levi form $\mathscr{L}(\phi)$
of $\phi$ is the hermitian quadratic differential form on $X$ defined
as follows: let $z^1,...,z^n$ be a coordinate system on an open set
$U \subset X$; then  
$$ 
\mathscr{L}(\phi)\left(\dfrac{\partial}{\partial z^\alpha},
\dfrac{\partial}{\partial \overline{z}^\beta}\right)=\frac{\partial^2
  \varphi}{\partial z^\alpha \partial\overline{z}^\beta}.
$$
($\mathscr{L}(\varphi)$ is thus a hermitian form on the  holomorphic
tangent space). 

\setcounter{definition}{0}
\begin{definition}\label{chap4:def4.1}%% 4.1
  A $C^\infty$ function $\phi:X \rightarrow \mathbb{R}$ is strongly
  $q$-pseudo-convex if the Levi form $\mathscr{L}(\phi)$ has at least
  $(n-q)$ positive eigen-values at every point of $X$. 
\end{definition}

\begin{definition}\label{chap4:def4.2}%%% 4.2
  A complex manifold $X$ is $q$-complete if there exists a $C^\infty$
  function $\phi:X \rightarrow \mathbb{R}$ such that 
  \begin{enumerate}[(i)]
  \item $\phi$ is strongly $q$-pseudo-convex
   
  \item for $c \in \mathbb{R}$, the set $\big\{ x\mid x \in X, \phi
    (x)< c\big\}$ is relatively compact in $X$. 
  \end{enumerate}
\end{definition}

\begin{lemma}[E. Calabi]\label{chap4:lem4.1}%%% 4.1 
  Let $H$ be a hermitian quadratic differential form on $X$ and $G$ a
  hermitian metric on $X$. Assume that $H$ has at least $p$ positive eigen
  values. Let ${\varepsilon_1(x),...,\varepsilon_n(x)}$ be the
  eigen values of $H$ (w.r.t. $G$) at $x$ in decreasing order: $
  {\varepsilon_r(x) \ge \varepsilon_{r+1}(x)}$ 

  Then\pageoriginale given $c_1, c_2 > 0$, $G$ can be so chosen that  
  $$
l_H(x) = c_1 \varepsilon_p(x)+ c_2  \Inf (0,\varepsilon_n(x))>0
 ~ \text{ for all } ~ x \in X.
$$  
\end{lemma}

\begin{proof}
  Let $G$ be any complete hermitian metric whatever. Let
  $\sigma_1(x),\ldots,\break\sigma_n(x)$ be the eigen-values of $H$ with
  reference to $G$ arranged in decreasing order. We construct now a
  metric $G$ on $X$ whose eigen values are functions of $
  {\big\{\sigma_i(x)\big\}_{1 \le i \le n}}$ as follows: let $
  {\lambda : X \to  \mathbb{R}}$ be a ${C^\infty-}$
  function (we will impose conditions on $\lambda$ letter); let $U$ be
  a  coordinate open set in $X$ with holomorphic coordinates
  $(z^1,...,z^n)$; in $U$, we have $G=G_{U \alpha 
      \overline{\beta}} dz^\alpha d\overline{z}^{\beta}$  so that $(G_{U
      \alpha \overline{\beta}})_{\alpha \beta}$ is a function whose
  values are positive definite hermitian matrices; then the matrix
  valued function ${\widehat{G}_U=(\widehat{G}_{U \alpha
      \overline{\beta}})_{\alpha \beta}}$ where $
  {\widehat{G}=\sum\widehat{G}_{U \alpha
      \overline{\beta}}dz^\alpha d\overline{z}^\beta }$ in $U$ is
  defined by  
  $$
  {\widehat{G}^{-1}_U = G^{-1}_U  \displaystyle\sum^\infty_{\substack{=
        \\ r=0}} \frac{\lambda (x)^r}{(r+1) !}(H_U G^{-1}_U)^r}
  $$ 
  where $H_U$ is the matrix valued function $(H_{U \alpha
      \beta})$ defined by  
  $$
  H=\sum H_{U \alpha \overline{\beta}}dz^\alpha d\bar{z}^\beta 
  $$
  in $U$.
\end{proof}

We now assert that $\widehat{G}_U$ define a global hermitian
defferential form on $X$ and under a suitable choice of $\lambda$, it is
positive definite. To see that $\widehat{G}_U$ defines a global
hermitian differential form on $X$, we need only prove the
following. Let $V$ be another coordinate open set with coordinates
complex $(w^1,...,w^n)$. Let
$J=\dfrac{\partial(z^1,...,z^n)}{\partial(w^1,...,w^n)}$ be the    
Jacobian\pageoriginale matrix. As before let $G_V = (G_{V \alpha
  \beta})$ be defined by $G=\sum G_{V \alpha \overline{\beta}}
dw^\alpha d \bar{w}^\beta$ in $V$. Then if $\widehat{G}_V$ is defined
starting from $G_V$ as  $\widehat{G}_U$ from $G_U$, we have  
$$
J\widehat{G}_U^t \overline{J}=\widehat{G_V}.
$$

We have in fact, writing $J^*$ for ${}^t {\bar{J}}^{-1}$,
$$
J^* {\widehat{G}}_U^{-1} J^{-1} = G^{-1}_V J \left(\sum\limits_{r=0}^{\infty}
\frac{\lambda(x)^1}{(r+1)!}(H_U G^{-1}_U)^r\right)J^{-1}, \; ~ {\rm
  since} ~ \; J G_U {}^{t}\bar{J} = G_V.
$$

It follows from the above that
\begin{align*}
  J^* \widehat{G}_U^{-1} J^{-1} & = G_V^{-1} \sum\limits_{r=0}^\infty
  \frac{\lambda(x)^r}{(r+1)!}  (J  H_U  G_U^{-1}  J^{-1})^r \\
    &= G_V^{-1} \sum\limits_{r=0}^\infty
  \frac{\lambda(x)^r}{(r=1)!}(H_V J^* G_U^{-1} J^{-1})^r 
  \end{align*}
since $J H_U^t \overline{J}H_V$. Hence we obtain
\begin{align*}
  J^*\widehat{G}_U^{-1} J^{-1} & =G_V^{-1} \sum\limits_{r=0}
  \frac{(\lambda(x))^r}{(r+1)!} (H_V G_V^{-1})^r\\ 
  &= \widehat{G}_V^{-1}
\end{align*}

This proves that $\widehat{G}_U$ defines on $X$ a global hermitian
differential. We next show that $\widehat{G}$ is positive definite. For
this we look for the eigen-values of $\widehat{G}$ with reference to $G$. TO
compute these, we may assume, in the above formula for $\widehat{G}_U$,
that $G_U$ is the identity matrix 

Then\pageoriginale we have
$$ 
\widehat{G}_U= \displaystyle \sum_{r=0}^{+ \infty}
\frac{\lambda(x)^r}{(r+1)!} H_U^r
$$

It follows that the eigen values  of $\widehat{G}_U$ are
$$ 
\left\{\sum\limits_{r=0}^{+ \infty} \frac{\lambda(x)^r}{(r+1)!} 
\sigma_q(x)^r \right\}_{1 \leq q \leq n}. 
$$ 

It is easily seen that  these are all strictly greater than zero: this
assertion simply means  this: $f(t)= \frac{e^t-1}{t}= 
\sum\limits^{\infty}_{r=0} \frac{t^r}{(r+1)!}$ for $t \neq 0, f(0)=1$ (which
is continuous in $t$) is everywhere greater than 0. 

We will now look for conditions on $\lambda$ such  that $\widehat{G}$
satisfies our requirements. From the formula for $\widehat{G},$ we have 
$$
H_U \widehat{G}_U^{-1}= \sum\limits_{r=0}^{+ \infty}
\frac{\lambda(x)^r}{(r+1)} (H_U G_U^{-1})^{r+1}.
$$

Now the eigen values of $H$ with respect to $\widehat{G}$ (\resp $G$) are
simply those of the matrix $H_U \widehat{G}_U^{-1}$ (\resp $H_U G_U^{-1}$),
Hence these eigen-values $\varepsilon_q(X)$ of $H$ with reference to
$\widehat{G}$ are 
$$
f(\lambda(x), \; \sigma_q(x))
$$
where $f(s,t)$ is the function on $\mathbb{R}^2$ defined by
$$
f(s,t)= \sum\limits_{r=0}^{t \infty} \frac{S^r}{(r+1)} t^{r+1} 
$$

Since $\frac{\partial f(s,r)}{\partial t}=e^{st} > 0$ for any,
$f(s,t)$ is monotone increasing in $t$.  

Hence\pageoriginale we have
$$
\varepsilon_r (x) \geqslant \varepsilon_{r+1} (x) ~ \text{ for } ~ 1
\leqslant r \leqslant n-1.  
$$

Moreover $f(s,t) \geqslant t$ for $s \geqslant 0$. Thus , if we choose
$\lambda(x) \geqslant 0$ for  every $x \epsilon X$, then
$\varepsilon_q(x)\geqslant \sigma_q(x)> 0$. 

The choice of $\lambda(x)$ is now made as follows . Let , for every
integer $\nu > 0$. $B_\nu= \bigg\{x \mid d(x,x_0)\leqslant \gamma
\bigg\}$ for some $x_0 \epsilon X$, the distance being $i$ the metric
$G$. The $B_{\nu}$ are then compact. Let $b_{\nu}=\underset{x \in
  B_{\gamma}}{\text{Inf}} (\sigma_p(x))$.

Then $b_1 \geqslant b_2 \geqslant \ldots \geqslant b_{\nu + 1} \geqslant
\ldots $. 

Let  $b(x)$ be a $C^\infty$ function on $X$ such that $b(x)>0$ for $x
\in X$ and $b(x)< b_{\nu}$ in $B_{\nu}-B_{\nu-1}$. Then clearly
$b(x) \leq  \sigma_p (x)$. 

Finally let $\rho(x)$ be a $C^\infty$ function on $X$ such that
$\rho(x)\geqslant d(x,x_0)$, and $k> \sqrt{\frac{C_2}{C_1}} b_1$ be a
real constant. Set $\lambda(x)=\frac{2ke^{\rho (x)}}{b^2(x)}$. We have
then 
$$
\displaylines{\hfill \varepsilon_q(x)=f(\lambda(x),
  \sigma_p(x))=\sigma_p(x)+\frac{\lambda(x)}{2!}\sigma_p(x)^2 + \cdots
  \hfill \cr   \text{so that}\hfill  \varepsilon_p (x) \geqslant
  \frac{ke^{\rho (x)}}{b^2(x)}   \sigma_p(x)^2 \geqslant ke^{\rho (x)}
  \geqslant k.\hfill }  
$$ 

On the other hand,
$$
\displaylines{\hfill 
  \varepsilon_n(x)=f(\lambda(x), \sigma_n(x))= \frac{1}{\lambda (x)}
  \bigg\{e^{\lambda (x) \sigma_n (x)_{-1}} \bigg\}\geqslant
  \frac{-1}{\lambda (x)}=- \frac{b^2(x)}{2ke^{\rho(x)}} \geqslant -
  \frac{b_1^2}{k},\hfill \cr  
  \hfill C_1 \varepsilon_p(x)+ C_2 \Inf (0, \varepsilon_n(x)) \geqslant C_1 k-C_2
  \frac{b_1^2}{k} \geqslant \frac{1}{k}(C_1k^2-C_2 b_1^2)>0.\hfill } 
$$

This proves the lemma.

Combining\pageoriginale this with the Remark at the end of \S
\ref{chap1:sec2}, we can 
assert moreover that $G$ can be chosen to be complete and satisfy the
condition of Lemma \ref{chap4:lem4.1}. 

Let $\phi :X \rightarrow \mathbb{R}$ be a strongly $q$-pseudoconvex
function on $X$. Assuming as a form $H$ in the above lemma the Levi form
$\mathscr{L} (\phi)$, and taking $p=n-q$,  $c_1 = 1$, $c_2 = n$, we
obtain 

\begin{lemma}\label{chap4:lem4.2}%%% 4.2
  Let $\phi : X \rightarrow \mathbb{R}$ be a strongly $q$-pseudoconvex
  function on $X$. Then there exists a complete hermitian metric on $X$
  such that at each point of $X$. 
  $$ 
  l_{\mathscr{L}(\phi)} (x) = \varepsilon_{n-q} (x) + n ~\text{ Inf }~
  (0,\varepsilon_n (x)) >   0. 
  $$
  Let $u \in C^{rs} (X,E)$ and let 
  $$
  \mathscr{L}(\phi) \{u,u\} = \frac{1}{p!q!} h_{\bar{b}a} \frac{\partial^2
    \phi} {\partial z^\alpha \partial \overline{z}^\beta} u^a_A{^
    \alpha}_{\overline{B}'} u^{\overline{b \bar{A} \beta B'}}
  $$
\end{lemma}

\begin{lemma}\label{chap4:lem4.3}%%%4.3
  If $\phi : X \rightarrow \mathbb{R}$ is strongly $q$-pseudoconvex then
  for any $u \in C^{rs}(X,E)$, with $s \ge q+1$ the following
  inequality holds at any point of $X$ 
  $$ 
  \mathscr{L}(\phi) \{ u,u \} \ge l_{\mathscr{L} (\phi)}^{A(u,u)}.
  $$
\end{lemma}

\begin{proof}
  At any point $x \in X$, we have
  $$ 
  \mathscr{L}(\phi) \{u,u\} = h_{\bar{b}a} \underset{\beta}\sum
  \varepsilon_\beta (x)u_{A \overline{\beta} \overline{B'}}
  u{^{\overline{\bar{A} \beta B'}}} 
  $$ 
  Since\pageoriginale 
  $$
  s \ge q+1 ,\qquad \text {i.e }  s+n-q\ge n+1,
  $$
  then every $s$-tuple of indices contains at least one the indices
  $1,2 \ldots, n-q$ of the positive eigenvalues $ \varepsilon_1(x),\ldots,
  \varepsilon_{n-q}(x)$. Hence    
  $$\frac{1}{p!q!} h_{\overline{b}a} \sum\limits^{n-q}_{\beta = 1}
  u_{A \overline{\beta} \overline{B}'} u^{\overline{\bar{A} \beta B'}}
  \geq A (u,u),$$   and therefore  
  \begin{multline*}
    \mathscr{L}(\phi) \{ u, u \} \ge \varepsilon_{n-q} (x)A (u,u) +
    \frac{1}{p!q!}\\ 
    \left\{\underset{n-q+1}{\varepsilon} (x) u_{A
      \overline{n-q+1 B'}} u {^{\overline{A n-q+1 B^1}}} \}  + \varepsilon_n (x)
    u_{A n \overline{B}'} \;  u^{\overline{\bar{A} n B'}} \right\}\\
    \ge  \varepsilon_{n-q} (x) A (u,u) + \Inf (o,\varepsilon_n(x)) A (u,u).
  \end{multline*}
  This proves our lemma.
\end{proof}

\begin{lemma}\label{chap4:lem4.4}%% 4.4
  $\phi$ be a strongly $q$-pseudo convex function on $X$ and $\mu$ a
  positive real valued function on $\mathbb{R}$ such that $\mu' (t) \ge
  0$, $\mu'' (t) > 0$. 

  Then
  $$ 
  \mathscr{L}(\mu (\phi)) (u,u) = \mu' (\phi) \mathscr{L}(\phi) (u,u)
   + (\mu'')(\phi) \Bigg| \frac{\partial \phi} {\partial z^\alpha}
  u^\alpha\Bigg | ^2
  $$

 Hence
 $$
 \displaylines{\hfill 
   \mathscr{L}(\mu(\phi)) (u,u) \ge \mu' (\phi) \mathscr{L} (\varphi)
   (u,u),\hfill \cr
   \text{and}\hfill 
   l_{\mathscr{L}(\mu(\phi))} \ge \mu' (\phi) l_{\mathscr{L}(\phi). \hfill }}
 $$
 
 The Lemma follows from a direct computation of the Levi form\break
 $\mathscr{L}(\mu(\phi))$ and from the fact since $\mu''(\phi)\Bigg|
 \dfrac{\partial \phi}{\partial z^\alpha} u ^\alpha\Bigg|^2$ is positive
 semi definite, the eigenvalues of $\mathscr{L}(\mu(\phi))$ are not
 than the corresponding\pageoriginale eigenvalues of $\mu'(\varphi)
 \mathscr{L} (\varphi)$.  
\end{lemma}

\begin{lemma}\label{chap4:lem4.5}%%% 4.5
   Let $X$ be a hermitian manifold and $\phi$ a strongly $q$-pseudo
   convex proper function such that at each $x \in X$ 
   $$ 
   l_{\mathscr{L}(\phi)} (x) = \varepsilon_{n-q} (X) + n \Inf
   (0,\varepsilon_n (x)) > 0,
   $$
   where $\{\varepsilon_p\}$ are the eigen values of $\mathscr{L}(\varphi)$
   with respect to the metric on X arranged in decreasing order. Then
   given any continuous function $g : X \rightarrow \mathbb{R}$
   there exists a sequence $\{ a_{\nu}\}_{1 \le \nu < \infty}$
   of real constants such that for any $C^{\infty}$ function $\mu :
   \mathbb{R} \rightarrow \mathbb{R}$ satisfying (i) $\mu'(t) > 0$ (ii)
    $\mu''(t)\ge 0$ and (iii) $\mu' (t) \ge a_{\nu}$ for
   $\nu \le t <\nu +1$, we have   
   $$
   l_{\mathscr{L}(\mu(\phi))} (x) > g(x).
   $$
 \end{lemma}

\begin{proof}
  Since the sets $K_\nu = \{x|\nu \leq \phi (x) \leq \nu + 1\}$ are
  compact and $l_{\mathscr{L}(\phi)} (x) > 0$ for all $x \in X$, we
  can find $a_\nu$ such that 
  $$ 
  a_\nu l_{\mathscr{L}(\phi)} (x) > g(x) \; \text{  for } \; x\in K_\nu
  $$
  the lemma now follows from Lemma \ref{chap4:lem4.4}
\end{proof}

\section[Holomorphic bundles over q-complete manifolds]{Holomorphic bundles 
over q-complete\hfil\break manifolds}%%% 13

\begin{lemma}\label{chap4:lem4.6}%%% 4.6
  Let $E$ be a holomorphic vector bundle on a $q$-complete complex
  manifold. Let $\varphi : X \rightarrow \mathbb{R}$ be a strongly
  $q$-pseudo-convex $C^\infty$ function on $X$ such that $ \{ x|\phi (x) < c\}
  \subset \subset X$ for every $c \in \mathbb{R}$. Let $ds^2$ be a
  hermitian metric 
  on $X$ such that $ l_{\mathscr{L}(\phi)} (x) > 0 $  with respect
  to this metric. Let $h$ be any hermitian metric on $X$. Then there is a
  sequence  $(a_\nu)_{1 \le \gamma < \infty}$  of real constants such
  that the following property holds.  

  Let\pageoriginale $\mu$ be any $C^\infty$ functions on
  $\mathbb{R}$ such that 
  $\mu' (t) \geq a_\nu$ for $\nu \leq t \leq \nu +1$ and $\mu' (t) > 0$,
  $\mu''(t) \geq 0$ for all $t$. Let $\mathcal{K}$ and
  (\resp $\mathcal{K}_{-\mu')}$ be the operators (linear over
  $C^{\infty}$ functions) from $C^{rs}(X, E) \longrightarrow C^{rs} (X
  ,E)$ defined in chapter \ref{chap3}. (See (\ref{eq3.12}) and
  (\ref{eq3.17})) with respect 
  to the hermitian metrics $ds^2$ and $h$ (\resp $h = e^{-\mu
    (\phi) h}$. Then for $\psi \in C^{r,s}(X, E)$ with $s > q$, we
  have  
  $$
  A_{-\mu} (\mathcal{K}_{-\mu} \psi, \psi) \geq e^{-\mu (\phi)} A(\psi
  , \psi). 
  $$
\end{lemma}

\begin{proof}
  From the Remark at the end of Chapter \ref{chap3}, 
  $$
  \mathcal{K}_{-\mu}\psi = \mathcal{K} \psi - \sum (-1)^i \frac{\partial^2\mu
    (\phi)}{\partial_z ^{\beta} \partial \overline{z}^{\beta_i}}
  \varphi^a_A \frac{\beta}{B'_i}.
  $$

  It follows that
  \begin{gather*}
    A_{-\mu}(\mathcal{K}_{-\mu} \psi , \psi) = e^{-\mu (\phi)} \{ \mathscr{L}(\mu
    (\phi)) (\psi, \psi)\} + e^{-\mu (\phi)} A(\mathcal{K} \psi ,\psi)\\
    \geq e^{-\mu (\phi)} \{l_{\mathscr{L}(\mu(\phi)} + f\} A(\psi, \psi),
  \end{gather*}
  
  Where $f : X \longrightarrow \varmathbb{R}$ is a continuous function
  depending on $\mathcal{K}$. By Lemma \ref{chap4:lem4.4}. $(a_\nu)_{1 \leq \nu <
    \infty}$ can be so chosen that for any convex $\mu$ satisfying 

  (i)~ $\mu' (t) > 0$,\qquad (ii)~ $\mu'' (t) \geq 0$\qquad (iii) ~$\mu'
  (t) > a_\nu$ for $\nu \leq t < \nu + 1$, we have  
  $$
  c_1 l_{\mathscr{L}(\mu (\varphi)} + f \geq 1.
  $$

  Hence
  $$
  A_{-\mu (\varphi)} (\mathcal{K}_{-\mu} \psi , \psi) \geq e^{-\mu
    (\varphi)} A (\psi , \psi).
  $$
\end{proof}

\begin{theorem}\label{chap4:thm4.1}%%% 4.1
  Let\pageoriginale $X$ be a $q$-complete manifold so that there exists
  $\varphi :X 
  \longrightarrow \varmathbb{R}$ satisfying : (i) $\varphi$ is
  strongly $q$-pseudoconvex and (ii), for $c \in \varmathbb{R}, \{ x
  \mid \varphi (x) \leq c \}$  is compact $X$. Let $E$ be a holomorphic
  vector bundle on $X$. Then there exists a complete hermitian metric
  $ds^2$ on $X$ and a hermitian metric $h$ along the fibres of $E$ and real
  constants $\{ a_\nu\}_{1 \leq \nu < \infty}$ and $c> 0$ such that
  for every $C^\infty$ function $\mu : \varmathbb{R}\longrightarrow
  \varmathbb{R}$ satisfying 
  
 {\rm (i)} ~$\mu' (t) > 0$, \quad {\rm (ii)}~ $\mu'' (t) \geq 0$ \quad
 and {\rm (iii)}~  $\mu' (t) \geq a_\nu$ for $v \leq t< \nu +1$ 
  we have
 $$
3(\| \overline{\partial} \psi \|_{-\mu} + \| \vartheta_{-\mu} \psi
  \|^2_{-\mu}) \geq \| \psi \| ^2_{-\mu}  \; \text{ for every } \; \psi  
 \in \mathscr{D}^{r,s }(X ,E), s > q.
$$
 
 Hence $E$ is $W^{rs}$-elliptic for $s> q$, with respect to the metric
 $e^{-\mu (\varphi)} h$. 
\end{theorem}

 \begin{proof}
   From proposition \ref{chap3:prop3.1}., we have
   $$ 
   (\mathcal{K}_{-\mu} \psi ,\psi) \leq 3 \left\{ \|
   \overline{\partial}\psi \|^2_-\mu  + \| \vartheta _{-\mu} \psi
   \|^2\right\}.  
   $$

(The subscript $-\mu$ means that the operators scalar product etc. are
   defined  with reference to the metric $ds^2$ on the base and
   $e^{-\mu (\varphi)}h$ along the fibres of $E$). The theorems then
   follow from Lemma \ref{chap4:lem4.2}, Lemma \ref{chap4:lem4.6} and
   the above inequality.  
 \end{proof}

\begin{remark*}
  If we choose a $\mu$ satisfying the conditions of Theorem \ref{chap4:thm4.1}, then
  for any function $\lambda : \varmathbb{R}\longrightarrow
  \varmathbb{R}$ such that $\lambda'(t) \geq 0, \lambda''(t) \geq 0$
  for $t \in \varmathbb{R}, \mu + \lambda$  again satisfies the
  conditions of the theorems. 
\end{remark*}

Consequently if we denote the metric on the base by $ds^2$ and the
hermitian\pageoriginale metric $e^{-\mu (\varphi)}h$ by $h'$, with
respect to $(ds^2 , h')$ we have the following properties for $s > q$:  
\begin{enumerate}[1)]
\item There is a $C^\infty$ function $\phi : X \to
  \mathbb{R}$ such that $\phi(x) \geq 0$ 

\item For every $C^\infty$ non decreasing convex function $\lambda
  :\mathbb{R} \rightarrow  \mathbb{R} E$ is $W^{rs}-$
  elliptic with respect to $(ds^2 , e^{-\lambda(\phi)}h')$ 

\item The $W^{rs}-$ ellipticity constant is independent of $\nu$.

\item The set $B_c=\{ x \mid x \in X, \varphi (x) \leq c\}$ is compact
  in $X$ for every $c \in \mathbb{R}$. These conditions imply
  conditions $C^*_1, C^*_2, C^*_3, C^*_4$ of \S \ref{chap1:sec4}. 
\end{enumerate}

From Theorem \ref{chap4:thm4.1} and from Theorem \ref{chap1:thm1.3} we
have the following corollary. 

\begin{theorem}\label{chap4:thm4.2}%%% 4.2
  Let $X$ be a $q$-complete complex manifold and $E$ a holomorphic vector
  bundle on $X$. Then for 
  $$
  s \geq q+1, r \geq 0, H^s(X, \Omega^r (E)) = 0.
  $$
\end{theorem}

\begin{proof}
  In fact, for $\varphi \in C^{rs} (X ,E) (s > q),$ there is a
  suitable $\mu$ as in Theorem \ref{chap4:thm4.1} above (i.e $\mu
  :\varmathbb{R}\longrightarrow \varmathbb{R}$ is a $C^\infty$
  function such that  $\mu' (t) > 0, \mu'' (t) \geq 0$ and $\mu' (t) >
  a_\nu $ for $\nu \leq t < \nu + 1 , \{a_\nu\}_{1 \leq  \nu <
    \infty}$ being chosen so that Theorem \ref{chap4:thm4.1} holds
  for this sequence  of real constants) such that 
  $$ 
  \| \varphi \| ^2_{-\mu} = \int_X e^{-\mu (\phi)} A(\varphi
  ,\varphi) dX < \infty.
  $$

On the other hand, Theorem \ref{chap4:thm4.1}. ensures us that with respect to the
metric $e^{-\mu (\varphi)} h$ ($h$ as in Theorem \ref{chap4:thm4.1}.) and the complete
hermitian metric $ds^2$ (as in Theorem \ref{chap4:thm4.1}.) on $X$, we have $W^{rs}-$
ellipticity for $s > q$. We deduce Theorem \ref{chap4:thm4.2}. from
Theorem \ref{chap1:thm1.3} of Chapter \ref{chap1}. 
\end{proof}

\begin{theorem}\label{chap4:thm4.3}%%% 4.3
  The\pageoriginale cohomology groups $H^s_k (X, \Omega^r (E))$
  (cohomology with compact supports) vanish for $s \leq  n -q -1$ for
  any vector bundle $E$ on the $q$-complete complex manifold
  $X$. Moreover, $H^{p-q}_k (X , \Omega^r (E))$ has a structure
  of a separated topological vector space.  
\end{theorem}

\begin{proof}
  The theorem can be proved by applying Serre's duality to Theorem
  \ref{chap4:thm4.2}. It can also be given the following direct proof, which is based
  on the results of \S \ref{chap1:sec4}. 
  
  Let us choose a metric $h'$ along the fibres of the dual bundle
  $E^*$ in such a way that conditions 1), 2), 3), 4) of the Remark
  after Theorem \ref{chap4:thm4.1}. be satisfied (by $E^*$). Corresponding to the
  metric $e^{-\lambda (\phi)} h'$ on $E^*$, we choose the ``dual''
  metric ${}^t (e^{-\lambda (\phi)}h')^{-1} = e^{\lambda (\phi)^t }
  h'^{-1}$ on $E$. Let $\varphi \in C^{rs} (X ,E)$, with $s 
  \leq n-q-1$. Then $\ast \# \varphi \in C^{n-r n-s}
  (X,E^\ast)$. Since $n-s \geq q+1, \ast \# \varphi$ satisfies the
  inequality 
  $$
  \displaylines{\hfill 
    A_{E^\ast, -\lambda} (\mathcal{K}_{E^\ast, -\lambda} \ast \# \varphi , \ast \#
    \varphi) \geq A_{E^\ast, -\lambda}(\ast \# \varphi, \ast \#
    \varphi),\hfill \cr
    \text{i.e.}\hfill 
    A_{E^\ast, -\lambda} (\mathcal{K}_{E^\ast, -\lambda} \ast \# \varphi , \ast \#
    \varphi) \geq A_{E, \lambda}( \varphi, \varphi).\hfill }
  $$
\end{proof}

Hence, for any $\varphi \in \mathscr{D}^{rs} (X,E)$, with $s \leq
n-q-1$ we have 
$$ 
\|\varphi \|^2 _{E, \lambda} \leq 3 ( \| \overline{\partial}\ast \#
\varphi \|^2 _{E^\ast , -\lambda}+ \| \vartheta_{E^\ast, -\lambda} \ast \#
\varphi \| ^2_{E^\ast, -\lambda})
$$ 
i.e. by (\ref{eq1.4}), (\ref{eq1.5}), (\ref{eq1.6}), 
$$
\| \varphi \|^2 _{E,\lambda}\leq 3 (\| \overline{\partial} \varphi \|
^2 _{E,\lambda} + \| \vartheta_{E,\lambda}\varphi \|^2)
$$

This\pageoriginale inequality holds for any $\varphi \in
\mathscr{D}^{rs} (X,E)$, 
with $s \leq n - q - 1$, and for any $C^\infty$, nondecreasing, convex
function $\lambda : \varmathbb{R} \rightarrow  \varmathbb{R}$. 

Our theorem follows from Theorem \ref{chap1:thm1.4}. of \S
\ref{chap1:sec4} and from Theorem \ref{chap1:thm1.8}.   

Let now $X$ be a $q$-complete manifold and $\phi : X \rightarrow
\mathbb{R}^+$ a strongly $q$-pseudo convex function on $X$ such that
for   
$$
c \in  \varmathbb{R}, B_c = \{x| x \in X, \phi (x) < c \}\subset
\subset X. ~\text{ We set }~ B_c =Y ~\text{ for some }~ c \in
\varmathbb{R}.
$$

\begin{lemma}\label{chap4:lem4.7}%%%  4.7
  $B_c$ is $q$-complete.
\end{lemma}

\begin{proof}
  Let $\mu$ be a $C^\infty$, nondecreasing convex function on
  $(-\infty, c)$ such that $\{t \mid t \in  \varmathbb{R}, \mu (t)
  < t_\circ \}$ is relatively compact in $(-\infty , c)$ for every
  $t_\circ \in  \varmathbb{R}$. Then the two conditions of Definitions
  \ref{chap4:def4.2} are satisfied by the function $\mu (\phi)$. This proves the
  lemma  

  We adopt the following notation
  \begin{align*}
  F^{rs} (X,E) & = \left\{ \psi \in \mathscr{L}^{rs}_{\loc}(X,E);
  \overline{\partial} \psi =0 \right\}.\\
Q^{rs} (X,E) & = \left\{ \psi \in \mathscr{L}^{rs}_{-\nu \phi} (X,E)
  \text{ for positive integer  } \nu, \overline{\partial}\psi =0
  \right \}.
  \end{align*}

  Finally $\rho : F^{rs} (X,E) \rightarrow F^{rs}(Y,E)$ is the
  restriction map. Clearly $Q^{rs} (X,E) \subset C F^{rs} (X,E)$ and we
  have 
\end{proof}

\begin{theorem}\label{chap4:thm4.4}%% 4.4
  If $s \geq q$, $\rho (Q^{rs} (X,E))$ is dense in
  $F^{rs}(Y,E)$. (The metric on $X$ and that along the fibres are chosen as
  ds$^2$ and h respectively in the Remark after Theorem \ref{chap4:thm4.1}.) 
\end{theorem}

\begin{proof}
  We denote $Q^{rs}(X,E)$ simply by $Q$. Let $\mu$ be a continuous
  linear functional on $F^{rs} (Y,E)$ which vanishes on $r(Q)$. It is
  sufficient to prove that $\mu$ vanishes on all of $F^{rs}(Y,E)$.  
  
  First,\pageoriginale we extend $\mu$ to a continuous linear form on $
  \mathscr{L}^{rs}_{\loc}(Y,E)$ (Hahn-Banach extension theorem). By the
  representation theorem it follows that there is a $\psi \in
  \mathscr{L}^{rs} (Y,E)$ with compact support in $Y$ such that $\mu (u)
  = (\psi , u)$ for every $ u \in \mathscr{L}^{rs}_{\loc}(Y,E)$. 

  Let $c_\circ = \sup \{ \varphi (x) \mid x \in$. Support $\psi\}$.
  Since Support $\psi$ is compact in $Y = B_c$, then 
  $$ 
  c_\circ < c.
  $$

Now, let $\lambda : \varmathbb{R} \rightarrow \varmathbb{R}$ be a real
$C^\infty$ nondecreasing, convex function such that 
\begin{align*} 
  \lambda (t) & = 0 ~\text { for }~ t \leq < c_\circ\\
  \lambda (t) & = 0 ~\text { for }~ t< c_\circ,\\
  0 & \leq \lambda' ~\text { (t) }~ \leq 1.
\end{align*}

Then, for $t > c_\circ$,
$$
\lambda (t) \leq t-c_\circ \leq t .
$$

Let  
$$  
u \in \mathscr{L}^{rs}_{-\nu \lambda (\phi)} (X,E) ~\text{ for some }~ \nu
> 0, \overline{\partial} u =0 . 
$$

Then $u \in Q$, because
$$
\| u \|^2 _{-\nu \phi} = \int_X e^{-\nu \phi} A(u ,u ) dX \leq
\int e^{-\nu \lambda (\phi) } A (u,u) dX = \| u\| ^2 _{-\nu \lambda
  (\phi)}
$$ 

Thus we have   
$$ 
(\psi , u) = \mu (\rho (u)) =0 
$$

Now\pageoriginale under these conditions, it follows from Theorem
\ref{chap1:thm1.6}, 
taking into account the remark after Theorem \ref{chap4:thm4.1} that there exists
$\mathcal{X} \in \mathscr{L}^{r, s+1}$ such that  

(i) $\psi = \vartheta \mathcal{X}$ and \quad (ii) support $\mathcal{X}
\subset \big\{ x| \varphi (x) \leqslant c_\circ \big\}$. 

Let $\text{c}_\circ < \text{c}_{1} <c$ and introduce a complete
hermitian metric on $Y$ which coincides with the given one
$B_{1}$. With respect to the new metric, 
$$
\psi \in  \mathscr{L}^{rs} (Y,E), \mathcal{X} \in
\mathscr{L}^{r, s+1} (Y,E)
$$

Now let  $u \in \text{F}^\text{rs}(Y,E)$ be an arbitrary form such
that $\overline{\partial}\text{u = 0}$. Then there is a
$C^\infty$, nondecreasing function $\sigma : (- \infty,
c) \rightarrow \varmathbb{R}$ such that $\sigma(t) = 0$ 
for $t \leqslant c_1$, and 
$$
u \in \mathscr{L}_{- \sigma(\phi)}^{rs} (Y,E).
$$

We have then 
$$ 
(\psi, u) =(\vartheta \mathcal{X}, u) = (\vartheta_{- \sigma(\phi)}
\mathcal{X}, u)_{-\sigma(\phi)} = (\mathcal{X}, 
\overline{\partial} u)_{-\sigma (\phi)}=0.
$$

Hence the linear form $\mu$ vanishes on $F^{rs}(Y,E)$. This proves
the theorem.  
\end{proof}

\setcounter{corollary}{0}
\begin{corollary}\label{chap4:coro1}%%% 1
  $Q^{rs} (X,E)$ is dense in $F^{rs} (X,E)$ (in the topology of
  $L^2-$ convergence on compact sets). 
\end{corollary}

\begin{proof}
  This is simply the case $X=Y$ (note that in the proof of  Theorem
  \ref{chap4:thm4.4},   the case $c = \infty$ is covered).  
\end{proof}

\begin{corollary}\label{chap4:coro2}%%% 2
  $\rho(F^{rs} (X,E))$\pageoriginale is dense in $(F^{rs} (Y,E)$. 
\end{corollary}

\begin{proof}
  In fact $\Omega \subset F^{rs} (X,E)$. 
\end{proof}

Suppose now that $X$ is a stein manifold. Then $X$ is is 0-complete. Hence
we can find a $C^\infty$ function $\phi: X \rightarrow
\mathbb{R}$ such that $\mathscr{L}(\phi)$ is positive definite and
such that for $c \in \mathbb{R}$, $\big \{x| x \in
\mathcal{X}, \phi \mathcal{X} <  c \big\}\subset\alpha X$. 
If then $E$ is a holomorphic vector bundle on $X$, we obtain from
Corollary \ref{chap4:coro1}, the following result. 

\begin{corollary}\label{chap4:coro3}%%% 3
  Let $Q(X,E)$ denote the space of holomorphic sections of $E$ over the
  Stein manifold $X$. There exists a complete hermitian metric on $X$ and
  a hermitian metric along the fibres of $E$ such that the set
  $\{f| f \in Q (X,E); \int_X  e^{-\nu \phi} |f|^{2}< \infty \;
  \text{ for some integer } \; \nu \}$ is 
  dense in $Q(X,E)$ in the topology of uniform convergence on compact
  sets. 
\end{corollary}

In view of Corollary \ref{chap4:coro1}, to prove Corollary
\ref{chap4:coro3}, we need only prove the following 

\begin{lemma}\label{chap4:lem4.8}%%% 4.8
  Let $D$ be a domain in $\mathbb{C}^n$  and $\{ f_m\}$ a sequence of
  holomorphic functions on $D$ such that for every compact
  $K \subset D, \{f_m\}$ converges in
  $L^2 (K)$ (= space of square summable functions on
  $K$). Then $\{ f_m\}$ converges uniformly on every compact set
  $K$. 
\end{lemma}

\begin{proof}
  Let $K$ be a compact set contained in D. Then there is a constant
  $c = c (K) > 0$ such that for every point
  $ z_\circ \in K$. The poly-disc $P{_z}_{_\circ}$  
  $= \{z \big| |z^i-z^i_o \mid \leq c \}$ of radius c is contained in
  $D$ and $\bigcup\limits_{z_\circ \in K} P_{z_o} \subset
  \subset D$.
  We have then from the Cauchy-integral formula,
 
  $f(z_\circ) = \frac{1}{(\pi c^2)^n} \int_{P_{z_o}} f(z)
  dv$. ($dv$ = Lebesque measure in $\mathbb{C}^n$) 
 
  It\pageoriginale follows on applying this to $f^2$, that
  $$
  \mid f (z_\circ)\mid^2 \leqslant
  \frac{1}{\text{vol} P{_z}_{_\circ}} \int_{P{_z}_{_o}} \mid f \mid ^2
  dv \leqslant \frac{1}{\text{vol}P{_z}_{_\circ}} \int_{P{_z}_{_o}} \|
  f||^2_{K'}
  $$
  where $K'$ is the (compact) closure of ${\underset{z_\circ \in
      \text{K}}{\bigcup}} P{_z}_{_o}$. The lemma is immediate from the
  above inequality. 
\end{proof}

In the special case $q=0$, Corollary \ref{chap4:coro2} together with the lemma above
yields the following result. 
 
\begin{theorem}\label{chap4:thm4.5}%%% 4.5
  Let $X$ be a Stein manifold and $\varphi : X \rightarrow
  \mathbb{R}$ a strongly 0-pseudo convex $C^\infty$
  function on $X$ such that for 

  $c\in \mathbb{R}, \big \{\mathcal{X} \mid \phi \mathcal{(X)} <
  c\big \} \subset  \subset X$. Then the pair $(X,
  B_c)$, $c \in \mathbb{R}$ is a Runge  pair. In other
  words, the set of holomorphic functions on $B_c$ which are
  restrictions of global holomorphic functions on $X$ is dense in the
  space of all holomorphic functions on $B_c$ in the topology of
  uniform convergence on compact subsets (of $B_c$). 
\end{theorem}

\section{Examples of $q$-complete manifolds}%%% 14

Let $X$ be an $m$-complete manifold and $f_1, \ldots, f_{q+1}$ be any
$q + 1$ holomorphic functions on $V$. Let $Z= \big\{ x | f_i (x) = 0
~ \text{ for } ~ 1 \leqslant i \leqslant q+1\big\}$. 

\begin{description}
\item[\underline{Assertion}] $Y = V - Z$ is $(m+q)$-complete.
\end{description}
  
\begin{proof}
  Let $\varphi :  V \rightarrow \mathbb{R} $ be a strongly
  $m$-pseudo-convex function $V$ such that for every $c \in
  \mathbb{R}$, 
  $$
  B_c = \big\{x \mid x \in V, \varphi
  (x) < c \big\} \subset \subset X.
  $$
  
  Because\pageoriginale of the second condition, we can choose a
  $C^{\infty}$ function 
  $$
  \lambda : \mathbb{R} \rightarrow \mathbb{R} ~\text{such that }~
  \lambda' (t) > 0, \lambda'' (t) \geq 0, 
  $$
  $ \lambda (t)> 0$  for  $t \in \mathbb{R}$, $\lambda (t)
  \rightarrow \infty$ as $t \rightarrow \infty$ and further  
  $$ 
  \lambda (\varphi) > \sum_{i=1}^{q+1} f_i  \overline{f}_i  ~ \text{ on }~
  Y. 
  $$

Set $\psi =\lambda (\varphi) - \log 
\sum\limits^{q+1}_{i=1}  f_i \bar{f}_i$. Since
$\mathscr{L} (\log \sum\limits^{q+1}_{i=1} f_i \bar{f}_i)$
is positive semi-definite with at most $q$ positive eigen-values, $\psi$
strongly $(m+q)$-pseudo convex. It remains to prove that 
$$ 
B'_c=\{X \mid  x\in  Y, \psi (x) < c \} \subset \subset y   
$$
for every $c\in \mathbb {R}$. Now $\psi (x)\leq c$ implies that
$$ 
\lambda (\varphi) - \sum\limits^{q+1}_{i=1} \;\log\;
f_i\; \bar{f}_i \leq c
$$
or again that
$$
\displaylines{\hfill 
  \frac{e^{\lambda (\phi)}}{\sum\limits^{q+1}_{i=1} f_i \overline{f_i}}
  \leq e^c,\hfill \cr 
  \text{i,e.,} \hfill \sum\limits^{q+1}_{i=1}  f_i \overline{f}_i \geq
  e^{\lambda (\varphi)-c} > \delta >0.\hfill }
$$ 
\end{proof}

Hence $\text{B}_c \subset X-U$, where $U$ is a neighborhood of $Z$. On
the  other hand; since  
$ \lambda (\varphi)> \overset{q+1}{\underset{i=1}\sum} f_i
\overline{f}_i, e^{\lambda (\varphi)} \slash \lambda (\varphi)< e^c$, 
hence\pageoriginale $\lambda (\varphi)< e^c = C$, on $B'_c$.

Hence $B'_c \subset B_c$. On the other hand since
$\overline{B'_c}$ is closed in $X-Z$ in $X-U$, it is closed in $X$. Hence
$\overline{B'_c}$ is compact. This concludes the proof of our assertion.

In particular, if $X$ is a Stein manifold, then $Y$ is $q$-complete. 
          
If $Z$ is a complete intersection, then $q+1$ is the complex codimension
of the  submanifold $Z$ of $X$. Let now $Z$ be any analytic subset of the
Stein manifold $X$, of complex codimension $q+1$ at each point. To $Y=X-Z$
$q$-complete? 
  
If $q=0$, and if $Z$ is a divisor, then the answer to this question is
positive  i.e. $Y=X -Z$ is a Stein manifold as it was shown by Serre
(\cite{key6} p. 50) R.R. Simha \cite{key31}  
has proved that, if $q=0$, if $X$ is a complex codimension 2 and if $Z$
is an analytic  set in $X$ of complex codimension 1 at each point, then
$Y=X-Z$ is a Stein space. Examples (\cite{key23} 
Satz 12, 17; \cite{key14}) show that this result is false when 
$\dim_{\mathbb{C}} X>Z$. 
  
An example, due to G. Sorani and V. Villani \cite{key33} , shows that
if  $q>0$ then $Y$ is not necessarily $q$-complete. Before discerning that
example, we shall establish the following 
  
\begin{theorem}[\cite{key32}]\label{chap4:thm4.6}%%% 4.6 
  Let $X$ be $q$-complete of complex dimension $n$.  
  Let $\Omega^{n}$  denote the sheaf of germs of holomorphic $n$-forms
  on $X$. The natural map 
  $$ 
  H^q (X,\Omega^n)\rightarrow H^{n+q} (X, \mathbb{C})
  $$
  is surjective.
\end{theorem}

\begin{proof}
  For\pageoriginale any complex manifold, we have spectral sequence
  $(E^{st}_r)_o\leq r \leq \infty$ such that $E_\infty$ is the
  associated graded group of  
  $H^* (X, \mathbb{C})$, and $E^{st}_1= H^t (X, \Omega^s) $
  \cite{key11}. In the  case when $X$  is 
  $q$-complete, $H^t (X,\Omega^s)=0$ for $t > q$. In particular for 
  $t+s=n+q$, $E^{st}_t=0$ for all $(s,t)$ different from $(n,q)$. Hence
  $E^{s,t}_r=0$ for $(s,t) \neq  (n,q), s+t=n+q$. Moreover, all
  of $E^{n,q}_1$ are 
  cycles for $d_1$ since $\Omega ^{n+1} =0$. Hence $E^{n,q}_1
  \rightarrow E^{n,q}_2$ is surjective.  For $ r \geq 2$, $E^{n-r,
    q+r-1}_r =0$ since $q+r-1 > q$ and $E^{n+r, q+r+1} =0$. Hence
  $E^{n,q}_2  \underline{ \backsim} 
  E^{n,q}_\infty$. 
  Hence $H^{n+q} (X, \mathbb{C}) \simeq
  E^{n,q}_\infty=E^{n,q}_2$. (We note that $E^{s,t}_\infty=0$ for
    $s+t =n+q$, $(s,t) \neq (n,q)$).
This proves the theorem.
\end{proof}

We consider now the subsets of $\mathbb{C}^{2n}$
\begin{align*}
  Z_1 &= \big\{z \in {\mathbb{C}}^{2n} \big|  z_i=0 ~ \text{ for\; i }~ > n
  \big\},\qquad X_1= {\mathbb{C}}^{2n}-Z_1,\\ 
  Z_2 &= \big\{z \in {\mathbb{C}}^{2n} \big|  z_i=0 ~ \text{ for\; i }~ \leq
  n \big\},\qquad X_2={\mathbb{C}}^{2n}-Z_2,\\ 
  Z &= Z_1 \cup Z_2,~\;{Y}=\mathbb{C}^{2n}-Z={X}_1 \cap {X}_2.
\end{align*}

The analytic set $Z \subset {\mathbb{C}}^{2n}$ has complex codimension n at
each  point. We shall prove that $Y$ is not $(n-1)$-complete, when $n
\geq 2$.

Since the point $\{0\}$ is a complete intersection of codimension $2n$ in
$\mathbb{C}^{2n}$, then $U= {\mathbb{C}}^{2n}-\{0\}$ is $(2n-1)$-complete.  
Let $\Omega^{2n}$ be the sheaf of group of holomorphic 2n-forms. Then by
Theorem \ref{chap4:thm4.6}. 
$$ 
H^{2n-1}(U,\Omega^{2n}) \rightarrow H^{2n-1} (U,\mathbb{C})
$$
is surjective.\pageoriginale On the other hand, since $U$ is
contractible on the unit sphere of $\mathbb{C}^{2n}$, then $H^{2n-1}
(U, \mathbb{C}) \approx \mathbb{C}$. Hence    
$$
H^{2n-1} (U,\Omega^{2n})  \neq 0.
$$

The exact cohomology sequence of Mayer-Victoris yields 
\begin{multline*}
   \rightarrow H^r(U,\Omega^{2n}) \rightarrow H^r(X_1,\Omega^{2n})
  \oplus H^r(X_z, \Omega^{2n})\\
 \rightarrow H^r (X{_1 \cap} X_2
  ,\Omega^{2n})\rightarrow H^{r+1} (U, \Omega^{2n}) \rightarrow
 \tag{4.1}\label{eq4.1}
\end{multline*}

Since $Z_i(i=1,2)$ is a complete intersection of codimension $n,X_i$
is $(n-1)$-complete. 

Hence
$$
   H^r(X_i , \Omega^{2n}) = 0 \quad for \quad r \geq n (i=1,2).
$$
It follows from (\ref{eq4.1})  that
$$ 
H^r(X_1 \cap X_2 , \Omega^{2n}) \approx H^{r+1}
    (U,\Omega^{2n}) \quad \text{for} \quad r \geq n. 
$$
Then
$$
H^{2n-2} (X_1 \cap X_2, \Omega^{2n}) \neq 0.
$$
 
If $n\geq 2$, then $2n-2\geq n$. This shows that $X_1 \cap X_2$ is
not $(n-1)$-complete. 

\section{A theorem on the supports of analytic functionals}%%% 15

 We shall apply the methods developed above to obtain-following the
 ideas of  \cite{key16}, \S 2.5  with minor technical changes - a
 generalization, due to Martinean \cite{key23} of a theorem
 originating with P\'olya \cite{key27}. We first prove two
 propositions which are of independent interest. 

\begin{prop}\label{chap4:prop4.1}%% 4.1
  Let\pageoriginale $\phi$ be a $C^\infty$ plurisubharmonic function
  on  $\mathbb{C}^n$. Let $\Arrowvert \omega \Arrowvert_{-\varphi}$
  denote the norm of 
  a $C^\infty$ form $\omega$ of type $(p,q)$ with respect to the
  Euclidean metric on the base and the metric $e^{-\phi}$ on the trivial
  line bundle. Let 
  $\psi = \phi +2 \log (1+\arrowvert$z$\arrowvert^2$). Then there is
  a constant $C > 0$ such that for any $C^\infty$ form $\omega$ of
  type $(p,q)$, $q \geq 1$, with $\bar{\partial}\omega =0$, $\Arrowvert
  \omega \Arrowvert_\phi < \infty$, there is a $C^\infty$ form $\omega'$ of
  type $(p,q - 1)$ with $\bar{\partial}\omega' =\omega$, $\Arrowvert
  \omega' \Arrowvert_{-\psi} \leq  C\Arrowvert \omega  \Arrowvert_{-\phi}$.  
\end{prop}

\begin{proof}
  Consider the operator $\mathcal{K}_{-\psi}$ with respect to the metric
  $e^{-\psi}$.  We have, since the Euclidean metric is flat,  
  \begin{align*}
     A_\psi(\mathcal{K}_{-\psi} u,u) & = e^{-\psi}
    \mathscr{L}(\psi) \{ u,u \} \\ 
    & \qquad \geq 2 e^{-\psi}(1+\bar{|} z |^2)^{-2} A(u,u) =
    2(1+|z|^2)^{-2} A A_{-\psi} (u,u) , 
  \end{align*}	
  since $\mathscr{L}(\psi) \{ u,u \} \geq 2 \mathscr{L} (log(1+|z|^2))
  \{ u,u \} \geq 2(q+|z|^2)^{-2}A(u,u)$. Hence $\mathcal{K}_{-\psi}$ is
    positive definite with eigenvalues $\geq 2(1+|z|^2)^{-2}$ at the
    point $z$; $\mathcal{K}^{-1}_{-\psi}$ is also positive definite, and
    its eigenvalues are $\leq \frac{1}{2}(1+|z|^2)^2$. By consequence
    we have 	 
    \begin{align*}
	 |A_{-\psi} (u,v)| &\leq A_{-\psi} (\mathcal{K}^{-1}_{-\psi} u,u)
         A_{-\psi}(\mathcal{K}_{-\psi} v,v)\\  
	 &\leq \frac{1}{2}A_{-\psi}((1+|z|^2)^2 u,u) A_{-\psi}
         (\mathcal{K}_{-\psi}v,v).
    \end{align*}
	  
    Hence, for forms $u$, $v$ of type $(p,q)$, we have    
    \begin{equation*}
      |(u,v)_{-\psi} |^2 \geq \frac{1}{2} \Arrowvert(1+|z|^2)u
      \Arrowvert^2_{-\psi}(\mathcal{K}_{-\psi} v,v)_{-\psi},  \tag{4.2}\label{eq4.2}
    \end{equation*}	
    whenever the forms $u$, $v$ are such that the norms on the right
    are finite. 	 
\end{proof}

On\pageoriginale $\mathscr{D}^{p,q} (\mathbb{C}^n)$, we introduce the
norm $M$ by  
$$
M(u)^2 = (\mathcal{K}_{-\psi}u,u)_{-\psi} + \Arrowvert \bar{\partial}u
\Arrowvert^2_{-\psi} + \Arrowvert \vartheta_{-\psi} u
\Arrowvert^2_{-\psi},
$$ 
and let $V^{p,q}$ be the completing of $\mathscr{D}^{p,q}$ with respect
to this norm. Obviously $V^{p q} \subset
\mathscr{L}^{pq}_{\loc}$. For $u\in \mathscr{D}^{p,q}$, we have,
by (\ref{eq3.18}),  
$$ 
(\mathcal{K}_{-\psi} u,u) \geq \Arrowvert \bar{\partial} u
\Arrowvert^2_{-\psi} + \Arrowvert \vartheta_{-\psi} u
\Arrowvert^2_{-\psi}. 
$$

Thus the norm $M$ on $\mathscr{D}^{p,q}$ is equivalent to that defined by
\begin{gather*}
M'(u)^2 = \Arrowvert \bar{\partial} u \Arrowvert^2_{-\psi} +
\Arrowvert \vartheta_{-\psi} u \Arrowvert^2_{-\psi}.
\end{gather*}

Let $\omega \in \mathscr{L}^{p,q}_{\loc}$, and let  $\Arrowvert
\omega \Arrowvert_{-\varphi} < \infty$. Then, for $u \in
\mathscr{D}^{p,q}$, we have, by (\ref{eq4.2}) (dropping the factor
$\frac{1}{2}$ as we may)  
\begin{align*}
     |(u,\omega)_{-\psi} |^2 & \leq \Arrowvert (1+|z|^2) \omega
    \Arrowvert^2_{-\psi} M' (u)^2 \\ 
    &=\Arrowvert \omega \Arrowvert^2_{-\phi} M'(u)^2 \tag{4.3}\label{eq4.3}
\end{align*}

Thus, the linear form $u \rightsquigarrow (u, \omega)_{-\psi}$ is
continuous on $V^{p,q}$ so that there exists a unique $x \in V^{p,q}$
with  
\begin{align*}
  (u,\omega)_{-\psi} & = (\bar{\partial} u, \bar{\partial}
  x)_{-\psi} + (\vartheta_{-\psi } u, \; \vartheta_{-\psi } \;  
 x )_{- \psi }\\  
  &= (\Box_{-\psi} u,x)_{-\psi} , u \in \mathscr{D}^{p,q}.
\end{align*}
Thus we have 
$$ 
\omega = \Box_{-\psi} x  
$$
and,\pageoriginale moreover, by (\ref{eq4.3})
$$
\displaylines{\hfill 
  M'(x)^2 =(x,\omega)_{-\psi}\leqslant \parallel \omega
  \parallel_{-\phi}M' (x),\hfill \cr
  \text{so that}\hfill 
  \parallel  \overline {\partial}x \parallel^2_{-\psi}~+\parallel
  \vartheta_{-\psi} x \parallel^2 _{-\psi}\leqslant \parallel \omega
  \parallel^2_{-\phi}.\hfill } 
$$ 

Let us assume now that $\omega ~is ~C^\infty$~ and that
~$\overline{\partial} \omega=o$. Then by  
Corollary \ref{chap1:coro1} to Theorem \ref{chap1:thm1.2}, we have,
for any $\sigma>0$, 
$$ 
\displaylines{\hfill 
  \parallel \vartheta_{-\psi} \overline \partial x \parallel_{\psi}
  \leqslant \frac{1}{\sigma} \parallel \overline \partial \omega
  \parallel^2 _{\psi} + \sigma \parallel \overline \partial x
  \parallel^2 _{-\psi}\hfill \cr 
  \hfill  =\sigma \parallel \overline {\partial} x
  \parallel_{-\psi} ,\hfill }
$$
and letting $\sigma \rightarrow 0$, we obtain
$$  
\displaylines{\hfill 
  \vartheta _{-\psi} \overline \partial x = 0,\hfill \cr
  \text{so that}\hfill 
  \Box_{-\psi} x = \overline{\partial} \vartheta_{-\psi}~x; \hfill }
$$
since, as proved above, $ \parallel \vartheta _{-\psi} x
\parallel_{-\psi} \le \parallel \omega \parallel_{- \phi}$, we obtain
the proposition  on setting $ \omega ' = \vartheta _{- \psi} x.$

We point out that the above proposition depends  on a weaker condition
than $ W^{p,q}$-ellipticity, in fact, proposition \ref{chap4:prop4.1}  
is a particular case of the following general statement, which can be
established by a similar argument. 

\begin{prop}\label{chap4:prop4.2}%%% 4.2
  Let\pageoriginale $X$  be a complex manifold  endowed with a complete
  hermitian metric. Let $\pi : E \rightarrow X$  be a holomorphic
  vector bundle on $X$ and let $h$ be a hermitian metric along the
  fibres of $E$. Assume that 
  $$ 
  A (\mathcal{K}\varphi, \varphi) > 0
  $$
  at each point of $X$ and for every $\varphi \in C^{pq}
  (X,E)$. Let $\varepsilon(x)$ be the least 
  eigenvalue  of the (positive definite) hermitian from $A
  (\mathcal{K} \varphi, \varphi)_x (x \in X)$ 
  acting on the space of the $(p,q)$-form with values in $E$. Then
  $\varepsilon (x)> 0$. 

  If $\varphi \in  C^{pq} (X,E)$, $\overline{\partial} \varphi=0$ is such that
  $$ 
  \int_X \frac{1}{\varepsilon (x)} A (\varphi,\varphi)dX < \infty,
  $$
  there exists a from $\psi \in C^{p,q-1} (X,E)$ such that
  \begin{gather*}
    \overline \partial \psi = \varphi,\\
    \int_X A (\psi,\psi) dX \le 3 \int_X \frac{1}{\varepsilon (x)} A
    (\varphi,\varphi) dX .
  \end{gather*}
\end{prop}

\begin{prop}%%% 4.3
  Let $\phi$  be a plurisubharmonic function on $\mathbb{C}^n$ such 
  that there exists a constant $C > 0$ for which $|\phi(z) -
  \phi(z')|\leq C$ 
  whenever $|z-z'| \leq 1$. Let $V$ be a (complex) subspace of  $\mathbb{C}^n$ of 
  codimension $k$, and $f$ a holomorphic function on $V$ such that
  $\int |f|^2 e^{-\varphi}  d_V v < \infty$ ($d_V v$ = Lebesgue measure on
  $V$). Then, there exits a holomorphic function $F$ on $\mathbb{C}^n$
  with $F|V =f$ 
  and 
  $$ 
  \int_{\mathbb{C}^n} |F|^2 e^{-\varphi} (1+|z|^2)^{3k} dv \leqslant M \int_V |f|^2
  e^{-\varphi}d_V v,
  $$
  where\pageoriginale $M$ is a constant independent of $f$; here
  $dv=d_{\mathbb{C}^n} v$ is the Lebesgue measure in ${\mathbb{C}}^n$ 
\end{prop}

\begin{proof}
  We may clearly suppose that $V$ has codimension 1 in
  $\mathbb{C}^n$. We assume that $V$ is the subspace $z^n =0$.  

  Then $f$ is a holomorphic function in $\mathbb{C}^n$, depending only on
  $z^1, \ldots,\break z^{n-1}$. Let $\lambda: \mathbb{R} \rightarrow
  \mathbb{R}$ be a $C^\infty$ function on $\mathbb{R}$,  such that 
  \begin{align*}
    0 & \leqslant \lambda(t) \leqslant 1,\\
    \lambda(t) & =1 ~ \text{ for } ~ t\leqslant \frac{1}{4}\\
    \lambda(t) & =0 ~ \text{ for } ~ t\geqslant 1.
  \end{align*}
  Let $C_1 = \Sup \left| \dfrac{d \lambda}{dt} \right|$.

 We will construct a function $\mu $ in such a way that 
  $$
F(z_1, \ldots, z_n) = \lambda (|z^n|^2)~ f (z_1, \ldots, z_{n-1}) - z^n \mu
  (z_1, \ldots, z_n)
  $$
  satisfies all the requirements of the lemma.

  Let 
  $$
  \sigma=\overline \partial \lambda(|z^n|^2)~f,
  $$
  $\sigma$ is a $C^\infty$, $\overline{\partial}$-closed form, such
  that $|z^n|\leq \frac{1}{2}$. Thus the form  
  $\omega=\frac{1}{z^n} \sigma$ is a $(0,1)$- from which is $
  C^\infty$ and $\overline{\partial}$-closed on $\mathbb{C}^n$,
  i.e. $\omega \in C^{01}(\mathbb{C}^n, \mathbb{C})$. Moreover 
  $$
  \Supp~\omega \subset  \bigg\{z=(z^1, \ldots, z^n)|\leqslant |z^n|
  \frac{1}{2} \leqslant   1\bigg \}. 
  $$
Since\pageoriginale
$$
\displaylines{\hfill 
\mid \phi (z,...,z^{n-1}, z^n) - \phi (z,...,z^{n-1}, 0) \mid \le C
~\text{ for }~ \mid z^n \mid \leqslant 1, \hfill \cr
\text{then}\hfill 
\int\limits_{\mid z^n \mid < 1} e^{- \varphi'} \mid f \mid ^2 ~ dv \leqslant
\pi e^C ~ \int\limits_V ~ e^{- \rho} \mid f \mid ^2  ~dv_V\hfill }
$$ 

On the other hand  
{\fontsize{10}{12}\selectfont
$$
\Arrowvert \omega \Arrowvert ^2_{- \phi} = \int\limits_{\frac {1}{2} < 
  \mid z^n \mid < 1} e^{- \phi} A (\omega, \omega) dv \leqslant
(2)^2  \int\limits_{\frac {1}{2} < \mid z^n \mid < 1} e^{- \phi} \mid f \mid
^2 d \leqslant C_2 \int\limits_V e^{- \phi} \mid f \mid ^2 dv_V 
$$}\relax  
$(C_2 = 4C^2_1 \pi e^C)$. In particular, $\Arrowvert \omega
\Arrowvert^2_{- \phi} < \infty $ 

Assume now that $\phi$ is $C^\infty$. By Proposition \ref{chap4:prop4.1}, there
exists a $C^\infty$ function $\mu : \mathbb{C}^n \rightarrow
\mathbb{C}$ such that 
\begin{align*}
& = \overline \partial \mu\\
\int\limits_{\mathbb{C}^n} \frac{e^{-\phi}}{(1+ \mid z \mid^2)^n} \mid \mu \mid^2
dv &  \leqslant \Arrowvert \omega \Arrowvert^2_{- \phi} \leqslant c_2
~~\int\limits_V e^{- \phi} \mid f \mid ^2 dv.
\end{align*} 

If $\phi$ is not $C^\infty$, we introduce a regularising function, 
i.e. a $C^\infty$ function $\alpha(z) = \alpha(\mid z \mid) 
\geqslant 0$, on $\mathbb{C}^n$, with support $\subset \{ \mid z \mid \leqslant 1
\}$, and such that $\int\limits_{\mathbb{C}^n} \alpha dv = 1$. Given
$0 < \varepsilon < 1$, the function 
$$
\varphi_\varepsilon (z) = \int\limits_{\mathbb{C}^n} \phi (z -
\varepsilon z') \alpha (z') dv (z') 
$$ 
is a $C^\infty$ plurisubharmonic function on $\mathbb{C}^n$, such that
$$
\phi_\varepsilon (z) \rightarrow \phi (z) ~ \text{ as } ~ \varepsilon \rightarrow 0.
$$ 

Moreover\pageoriginale
$$ 
\displaylines{\hfill 
  \mid \phi_\varepsilon (z) - \phi (z) \mid \leqslant
  \int\limits_{\mathbb{C}^n} \alpha (z') 
  \mid \phi (z - \varepsilon z') - \phi (z) \mid d (z') \leqslant C,\hfill \cr
  \text{and therefore}\hfill 
  \mid \phi_\varepsilon (z) - \phi_\varepsilon (z') \mid \leqslant 3C ~
  \text{ for } ~ \mid z-z'
  \mid \leqslant . \hfill}
$$

Hence there exists a $C^\infty$ function $\mu_\varepsilon :
\mathbb{C}^n \to \mathbb{C}$, such that
$$
\displaylines{\hfill 
  \overline{\partial} \mu_\varepsilon = \omega \hfill \cr
  \hfill \int\limits_{\varepsilon^n} \frac{e^{- \phi}}{(1+ \mid z
    \mid^2)^2} 
\mid \mu_\varepsilon \mid^2 dv \leqslant e^c
~~\int\limits_{\mathbb{C}^n}
 \frac{e^{- \phi}_\varepsilon}{(1+ \mid z \mid
    ^2)^2} \mid \mu_\varepsilon \mid^2 dv \leqslant C_3\hfill \cr
  \hfill \int\limits_V ~~e^{- \phi}
  \mid f \mid^2 ~~ dv_V. (C_3 = C^2_1 \pi e^{5c}) } 
$$

Letting $\varepsilon \rightarrow 0$, we can find a distribution $\mu$
such that 
$$
\displaylines{\hfill 
  \overline{\partial} \mu = \omega_1 \hfill \cr
  \hfill \int\limits_{\mathbb{C}^n} \frac{e^{- \phi}}{(1+ \mid z \mid
    ^2)^2} \mid \mu \mid ^2 
  dv \leqslant C_3 \int\limits_V e^{- \phi} \mid f \mid^2 dv_V.\hfil }
$$ 

Consider now the distribution
$$
F = \lambda(\mid z^n \mid ^2) f - z^n \mu. 
$$

We have 
$$
\overline{\partial} F = \overline{\partial} \lambda \cdot f - z^n
\overline{\partial} \mu = 0 .
$$
Hence $F$ is a holomorphic function $\mathbb{C}^n$. 

Furthermore\pageoriginale
{\fontsize{10}{12}\selectfont
\begin{align*}
  \int\limits_{\mathbb{C}^n} \frac{e^{- \phi}}{(1+\mid z \mid^2)^3} & - \mid F
  \mid^2 d v \\
  & \leqslant 2 \int\limits_{\mid z^n \mid < \mid} \frac {e^{-
      \phi}}{(1+\mid z \mid^2)^3 } - \mid f \mid^2 d v + 2 \int\limits_{\mathbb{C}}
  \frac{\mid z^n \mid^2 e^{- \phi}}{(1+\mid z \mid^2)^3 } \mid \mu
  \mid^2  dv\\ 
  & \leqslant 2 \pi e^C \int\limits_V e^{-\phi} \mid f \mid^2 dv_V + 2
  \int\limits_{\mathbb{C}^n} \frac{e^{- \phi}}{(1+ \mid z \mid ^2)^2} \mid \mu \mid^2
  dv\\ 
  & \leqslant C_4 \int\limits_V e^{-\phi} \mid f \mid^2 dv_V (C_4 = 2
  \pi e^C + 2C_3). 
\end{align*}}\relax
This proves our proposition.
\end{proof}

\begin{lemma}\label{chap4:lem4.9}%%% 4.9
  Let $\phi$ be as in Proposition \ref{chap4:prop4.2}. Let $F:\mathbb{C}^n \rightarrow
  \mathbb{C}$ be a holomorphic function such that
  $$ 
  \int\limits_{\mathbb{C}} \frac{e^{-2 \phi}}{(1+\mid z \mid^2)^N}
  \mid F \mid^2 dv = M^2  < \infty ,
  $$
  for some $N \geqslant 0$. Then there exists a positive constant
  $C_5$ depending only on $C$ and on $N$ such that 
  $$
  \mid F (z) \mid \leqslant C_5 M e^{\phi(z)} (1+\mid z \mid)^N
  e^{\phi(z)}.
  $$
\end{lemma}

\begin{proof}
  Let $z_0 \in C^n$ and let $P(z_0) $ be the poly-disc with centre $z_0$ 
  and radius 1. Then by Cauchy's formula we have 
$$
F (z_0)^2 = \frac {1}{\pi^n} \int_{P(z_0)} F(z)^2 dv,
$$
whence\pageoriginale
$$
\mid F(z_o)^2 \mid \leqslant \frac{1}{\pi^n} \int\limits_{P(z_o)} \mid F
 \mid^2 d.
$$

  Furthermore
  $$
  \displaylines{\hfill 
    e^{-\phi(z_o)} \leqslant e^\varepsilon e^{-\phi}(z) ~ \text{ for } ~ z \in P_1
    (z_o)\hfill \cr
    \text{and}\hfill 
    \frac{1}{(1+\mid z_o \mid)^2} \leqslant \bigg(\frac {1+2n}{1+ \mid z
      \mid}\bigg)^2 \leqslant \frac{(1+2n)^2} {1+ \mid z \mid ^2}~~for~~z
    \in P_1(z_o)\hfill }
  $$ 
   
  In conclusion 
\begin{align*}
  \frac{e^{-2 \phi (z_o)}}{(1+\mid z_o)^{2N}} \mid F(z_o)\mid ^2
  &\leqslant \frac {e^{2 \varepsilon} (1+2n)^{2N}}{\pi^n} \underset{P(z_o)}\int
  \frac{e^{-2 \phi}}{(1+\mid z \mid ^2)} \mid F^2 \mid dv\\ 
  &\leqslant
  \frac{e^{2\varepsilon} (1+2n)^{2N}}{ \pi^n } M^2 .
\end{align*}
  
  This concludes the proof of our lemma.
\end{proof}

Before going on to state the theorem of Martinean-p\'olya, we need some
definitions. 

Let $\mathscr{H}=\mathscr{H} (\mathbb{C}^n)$ denote the vector space of
holomorphic function on   $\mathbb{C}^n$ with the topology of compact
convergence. An \textit{analytic functional} $\mu$ is a continuous linear
functional on $\mathscr{H} \cdot \mu$ is said to be \textit{supported by a compact
set} $K$ if for any open set $U \supset K$ there is a constant $M_U$ such
that sort any $f \in \mathscr{H}$ we have 
$$
\mid\mu(f)\mid \le M_U \underset{z \in U}{\sup} \mid f(z)\mid .  
$$

For\pageoriginale $z=(z_1, \ldots, z_n)$, $\zeta =(\zeta_1, \ldots, \zeta_n)\in
\mathbb{C}^n$, let 
$\langle z,\zeta \rangle =\sum z_i \zeta_i$. For $\zeta \in \mathbb{C}^n$,
let $e_\zeta \in \mathscr{H}$ be defined by $e_\zeta (z)=e^{\langle z,
  \zeta \rangle}$. The \textit{Laplace} transform of $\mu$ is the holomorphic
function $\overset{\backsim}\mu$ on $\mathbb{C}^n$ defined by  
$$
\overset{\backsim}\mu (\zeta) = \mu (e_\zeta).
$$

Let $K$ be compact and $H$ the function defined by
$H_K(\zeta)=\underset{z \in K}{\sup}\, \re \break \langle z,\zeta
\rangle$. Clearly $H_K$ is continuous, and hence, being the supremum of
plurisubharmonic functions, is itself plurisubharmonic. If $K_\varepsilon$
denotes the set  $\{ z \in  \mathbb{C}^n \mid d(z,K)\le \varepsilon \}$, we have
$H_{K_{\varepsilon}} (\zeta) = H_K(\zeta) + \varepsilon \mid \zeta \mid$

\begin{theorem}[Martineao-Polya]\label{chap4:thm4.7} %%% 4.7
  In order that the analytic functional $\mu$ be supported by the
  convex compact set $K$, it is necessary and sufficient  that for
  every $\varepsilon > 0$, there exist a constant $C_\varepsilon > 0$ such that  
  $$
  \mid \overset{\backsim}\mu (\zeta)\mid \leqslant C_\varepsilon e^{H_K
    (\zeta)+\in \mid \zeta \mid}.
  $$
\end{theorem}

\begin{proof}
  Suppose that $\mu$ is supported by $K$. Then, by definition  
$$
|\tilde{\mu} (\zeta)| = |\mu (e_\zeta)| \leq C_\varepsilon
\sup\limits_{z \in K_\varepsilon} |e_\zeta (z)| = C_\varepsilon
e^{\{\sup\limits_{K_\varepsilon} \langle z, \zeta \rangle \}} \leq
C_\varepsilon e^{H_K (\zeta) + \varepsilon |\zeta|} 
$$

  To prove the converse, we proceed as follows. $\mathscr{H}$ is a
  closed subspace of the space $\mathscr{C}$ of continuous function on
  $\mathbb{C}^n$; hence (Hahn - Banach) $\mu$ extends to a linear
  functional $\mu : \mathscr{C} \to \mathbb{C}$, hence defines a
  measure. It is clearly sufficient to 
  prove that any $\varepsilon > 0$, there exists distribution $\nu$
  with support in $K_\varepsilon$ such that 
  $$
  \mu(f)=\nu ~ \text{ for all } ~ f \in \mathscr{H}.
  $$

  Let\pageoriginale $\xi'$ be the space of distributions with compact support in
  $\mathbb{C}^n$. For $\nu \in \xi'$, let $\hat{\nu}$
  denote the Fourier transform of $\nu$ considered as a
  function of $2n$ real variables $u_1, \ldots, u_{2n}$: 
  $$
  \displaylines{\hfill 
  \hat{\nu}(u_1, \ldots , u_{2n}) = \nu (e'_u), \hfill \cr
  \text{where} \hfill e'_u(z)=\exp \, (-i(u_1 \re z_1 + u_2 \iim
  z_1 + \ldots + u_{2n} \iim z_n)).\hfill }
  $$
\end{proof}

Clearly  $\widehat{\nu}$ has an extension to a holomorphic
function on $\mathbb{C}^{2n}$; further since linear combination of the function
$e_\zeta$ are dense in $\mathscr{H}$, we have $\nu(g)=\mu(g)$
for all $g \in \mathscr{H}$ if and only if $\mu
(e_\zeta)=\nu (e_\zeta)$ for all $\zeta$, i.e. if and only if    
\begin{equation*} 
  \overset{\backsim}\mu (\zeta)= \widehat{\nu}(i \zeta_1, -
  \zeta_i,...,i\zeta_n,-\zeta_n). \tag{4.4}\label{eq4.4}
\end{equation*}

Therefore, because of the Paley-Wiener theorem, it is sufficient to
construct an entire function $\widehat{\nu}$ on $\mathbb{C}^{2n}$
satisfying (\ref{eq4.4}), for which, we have further, 
\begin{equation*}
  \mid \widehat{\nu} (u) \mid \leqslant C_\varepsilon (1+\mid u
  \mid|)^{N\phi (u)} . \tag{4.5} \label{eq4.5}
\end{equation*}
for some $N > 0$; here $\phi (u) = \sup\limits_{x \in K_\varepsilon}
(x_1 \iim u_1 + \ldots +  x_{2n} \iim u_{2n} )$ and $z_j = x_{2j-1} +
i x_{2j}$.

Consider the subspace $V$ of $\mathbb{C}^{2n}$ consisting of points $(i
\zeta_i, -\zeta_1,...,i\break \zeta_n ,-\zeta_n)$, where
$\zeta=(\zeta_1,...,\zeta_n)\in \mathbb{C}^{n}$. If
$u=(i\zeta_1,...,-\zeta_n)$, we have
$\phi(u) = H_{K_\varepsilon}(\zeta) = H_K (\zeta) + \varepsilon |\zeta|$.

Further,\pageoriginale $\phi(u)$ is plurisubharmonic in
$\mathbb{C}^{2n}$ and there exists 
a constant $C > 0$ such that  $|\phi(u) - \phi(u')|\leqslant C$  for
$|u-u'|\leqslant 1$. Clearly $\overset{\backsim}\mu$  
defines a holomorphic function $f$ on $V$ if we set 
$f(i\zeta_1,...,-\zeta_n)=\overset{\backsim}\mu(\zeta_1,...,\zeta_n)$. Since
for any $ \delta >0$, 
$|\overset{\backsim}\mu(\zeta)|\leqslant C_{\delta}
e^{H_K(\zeta)+\delta|\zeta|}$ by hypothesis and  $\phi(u) = H_K(\zeta) +
\varepsilon|\zeta|$  on $V$, we have
$$
|f(u)|\leqslant C_{\delta} e^{\phi(u)+(\delta-\varepsilon)|u|} ~ \text{
  on  } ~ V.
$$

If $\delta< \varepsilon$, we clearly have therefore
$$
\int_V|f(u)|^2 e^{-2 \phi (u)} d_V v< \infty.
$$

By Proposition \ref{chap4:prop4.2} there exists $F$, holomorphic in
$\mathbb{C}^{2n}$ such that  
$F|V=f$, and 
$$
\int\limits_{\mathbb{C}^{2n}} |F(u)|^2  e^{-2 \phi (u)} (1+|u|^2)^{-3n}
dv < \infty. 
$$

By Lemma \ref{chap4:lem4.9} this implies that there exists a constant
$C > 0$ such that  
$$ 
|F(u)| \leqslant M (1+|u|)^{3n} e^{\phi(u)},
$$
so that if we set $\widehat{\nu} (u) = F(u)$, the conditions
(\ref{eq4.4}) and (\ref{eq4.5}) are satisfied. This completes the
proof of the theorem. 


\begin{thebibliography}{99}
\bibitem{key1}{A. Andreotti and H. Grauert}:\pageoriginale
  Th\'eor\`ems  de finitude pour la cohomogipdes
  espaces complexes, Bull. Soc. Math. Frances \underline{90}(1962),
  193-259.

\bibitem{key2}{A. Andreotti and E. Vesentini}: Carleman estimates for
  the Laplace Beltrami equation on comples manifoldes,
  instit. Heautes \'Etudes Sci., Publ. Math. 25.81-130:
  Erratum ibd, to appearss

\bibitem{key3}{A. Andreotti and E. Vesentini} Les
  th\'eor\`emes fondaientaux de la th\'eorie
  des espaces holomorhiquement complets, S\'eminaire de
  G\'eometrie Diff\'erentielle Vol IV (1962-63) 1-31;
  Paris, Secr\'etariat Math\'emateque.

\bibitem{key4}{S. Bochner}: Tensor fields with finite bases, Ann. of
  Math., 53(1951), 400-411.

\bibitem{key5}{E. Calabi and E. Vesentini}: On compact, locally
  symmetric Kahler manifolds, Ann. of Math., \underline{71}(1960),
  472-507.

\bibitem{key6}{H. Cartan}: S\'eminaire E.N.S., 1953-54; Paris,
  Secr\'etariat Math\'ematique. 

\bibitem{key7}{H. Cartan}: Vari\'et\'es analytiques
  complexes et cohomologie, Colloque sur les fonctiones de plusieurs
  Variables, Bruxelles, 1953; 41-55.

\bibitem{key8}{J. Deny and J.L. Lions}: Les espaces de B. Levi,
  Ann. Inst. Fourier. 5(1953-54), 305-370. 

\bibitem{key9}{J. Dieudonn\'e and L. schwartz}: La
  dualit\'e dans les espaces $(\mathfrak{F})$ et $(\mathscr{L}\mathfrak{F})$,
  Annales de I' Institut Fourier, 1(1949), 61-101.

\bibitem{key10}{K.O. Friedrichs}: On the differentiability of the
  solutions of linear elliptic differential equations, Comm. Pure
  Appl. Math., \underline{6}(1953), 299-325.

\bibitem{key11}{K.O. Friedrichs}: Differential forms on Riemannian
  manifolds, Comm. Pure. Appl. Math., \underline{8}(1955) 551-590.

\bibitem{key12}{A. Frolicher}: Relations between the cohomology groups
  of Dolbeault and topological invariants,
  Proc. Nat. Acad. Sci. U.S.A., \underline{41}(1955), 641-644.

\bibitem{key13}{M.P. Gaffney}:\pageoriginale A special Stokes's theorem for complete
  riemannian manifolds, Ann. of Math., \underline{60}(1954), 140-145. 

\bibitem{key14}{R. Godement}: Topologie alg\'ebrioue et
  th\'eorie et theorie des faisceaux, Hermann, Paris, 1958.

\bibitem{key15}{H. Grauert and R. Remmert}: Singularitaten komplexer
  Mannigfltigkeiten and Riemannscher Gebiete, Math. Zeitschr.,
  \underline{67}(1957), 103-128.

\bibitem{key16}{L. Hormander}: $L^2$ estimates and existence theorems
  for the $\overline{\partial}$-operator, Diffenrential Analysis,
  OXford University Press, 1964, 65-79.

\bibitem{key17}{L. Hormander}: Existence theorems for the
  $\overline{\partial}$-operator by $L^2$ methods, Acta
  mathematica, 113 (1965), 89-152.

\bibitem{key18}{K. Kodaira}: Harmonic fields in Riemannian manifolds
  (Generalized potential theory), Ann. of math., 50 (1959), 587-665.

\bibitem{key19}{J.L. Koszul}: Vari\'et\'es
  Kahl\'eriennes, S$\tilde{a}$o Paulo, 1957; mimeographed notes.  

\bibitem{key20}{G. K\"othe}: Topologische lineare R\"aume,
  Springer-Verlag, 1960.

\bibitem{key21}{A. Lichn\'erowicz}: Th\'eorie globale
  des connexion et des groupes d'holonmie, Cremonese, Roma, 1955. 

\bibitem{key22}{A. Lichn\'erowicz}:
  G\'eom\'etrie des groupes de transformationes, Dunod, Paris, 1958.     

\bibitem{key23}{B. Malgrange}: Lectures on the theory of functions of
  several complex variables, Tata Institute of Fundamental
  Research, Bombay, 1958. 

\bibitem{key24}{A. Martineau}: Sur les fonctionnelles anaytiques et la
  transformation de Fourier-Borel, Journal d'Analyse
  Math\'ematique, XI(1963), 1-164. 

\bibitem{key25}{T. Meis}: Die Minimale Blmtterzhl der
  Konkretisierungen einer kompakten Riemannschen Flmche Schriftenreihe
  des Mathematischen Instituts der Universitmt munster, Heft 16, 1960.

\bibitem{key26}{M.S. Narasimhan}:\pageoriginale A remark on curvature
  and the Dirichlet problem, Bull. Amer. Math. Soc., 65(1959),
  363-364.  

\bibitem{key27}{K. Nomizu}:  Lie groups and differential geometry,
  Publications of the Mathematical Society of Japan, 1956.

\bibitem{key28}{G. Polya}: Untersuchungen \"uber L\"ucken
  and Singularit\"aten von Potenzreihen, Math. Zeitschr.,
  29 (1929) 549-640. 

\bibitem{key29}{F. Rellich}: Ein Satz \"uber mittlere
  Konvergenz, Nachr. Ges. G\"ottingen (math.-phys. Kl.), 1930,
  30-35.  

\bibitem{key30}{G. de Rham}: Vari\'et\'es
  diff\'erentiables, Hermann,  Paris, 1955.

\bibitem{key31}{L. Schwartz}:  Th\'eories des distributioes,
  Hermann, Paris; Tome I, 1957; Tome II, 1959.

\bibitem{key32}{R.R. Simha}:  On the complement of a curve on a Stein
  space of dimension two, Math. Zeitschr., 82(1963), 63-66.

\bibitem{key33}{G. Sorani}: Homologie des $q$-paires de Runge, Annali
  Scuola Normale Superiore, Pisa, (3), 17 (1963), 319-322.

\bibitem{key34}{G. Sorani-and V. Vesentini}: Spazi $q$-completi e
  coomologia, Mimeographed notes, 1964.

\bibitem{key35}{E. Vesentini}: Sulla coomologia delle variet\'a
  differenziaoili, Rend. Seminario Matematico e fisico di Milano,
  34 (1964), 19-30.

\bibitem{key36}{A. Weil}: Introduction a l'\'etude des
  vari\'et\'es k\"ahl\'eri ennes,  Hermann, Paris, 1958. 

\bibitem{key37}{K. Yano and S. Bochner}: Curvature and Betti numbers,
  Ann. of Math. Studies, 32, Princeton, 1953.
\end{thebibliography}

