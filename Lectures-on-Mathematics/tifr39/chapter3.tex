\chapter{Local expressions for $\square$ and the main
  inequality}\label{chap3}%%% 3 

\setcounter{section}{8}
\section{Metrics and connections}
 
We\pageoriginale now go back to the case of holomorphic vector bundles
on a complex manifold.  

Let $x$ be a complex manifold and $\pi: E \rightarrow X$ a
holomorphic vector bundle. Let $\mathcal{U}=\{U_i\}_{i\epsilon I}$ be
a covering of $X$ such that  $\pi: E \rightarrow X$ is defined with
respect to $\mathcal{U}$ by transition functions $e_{ij} : U_i\cap
U_j \rightarrow G L$  ($m$, $\mathbb{C}$) ($E$ is of rank $m$). 

Let $(h_i)_{i\epsilon I}$ be a hermitian metric along the fibres of $E$:
then $h_i$ are $C^\infty$ functions $U_i$ whose values are positive
definite hermitian matrix such that. 
$$
h_i=\overset{t-}{e}_{ji} h_j e_{ji} = {}^t \overline{e}_{ij}^{-1}
h_j e_{ij}^{-1} ~ \text{ on  } ~ U_i \cap U_j.
$$ 

Consider the 1-form $l_i = h^{-1}_{i}\partial {h_i}$ where
$\partial$ is the exterior differentiation with respect holomorphic
coordinates. We have for this family of forms the following. 

\setcounter{lemma}{0}
\begin{lemma}\label{chap3:lem3.1}%%% 3.1
  If $l_i= h^{-1}_{i} \partial {h_i}$, then $l_i$ are $(1,0)$ forms with
  values in $M_m (\mathbb{C} )$ and on $U_i\cap U_j$ we, have  
  $$  
  l_i=e^{-1}_{ji} l_j e_{ji}+e^{-1}_{ji} \partial
  e_{ji}.
  $$
\end{lemma}

In\pageoriginale other words $l_i$ are the local representations of a
connection of type $(1,0)$ on the principal bundle associated to $E$.   

\begin{proof}
  We have  
  $$
  e^{-1}_{ji} \; l_{j}\; e_{ji}+\: e^{-1}_{ji}\:
  \partial e_{ji}\; = e^{-1}_{ji}\; \partial h_j e_{ji}\;
  +e^{-1}_{ji}\; \partial e_{ji}
  $$

  Now $h_i= {}^t {\overline{e}}_{ji}  \; h_j \; e_{ji}$, so that 
  $$
  h^{-1}_{i}\partial h_i = e^{-1}_{ji}\;  h^{-1}_{j}\;
  {}^t \bar{e}_{ji}^{-1}\; \bigg\{\partial
 {}^t {\overline{e}}_{ji} \cdot \;   h_j e_{ji}\;
  + {}^t {\overline{e}}_{ji}\; \partial h_j e_{ji}\; + {}^t
   {\overline{\; e}}_{ji}\;  h_j\partial  e_{ji}\; \bigg\}.
  $$

Since the $e_{ij}$ are holomorphic functions,
  $$
  \partial {}^t{\bar{e}}_{ji}\; =0. 
  $$

  Hence \;  $h^{-1}_{i}\partial h_i = e^{-1}_{ji}\; h^{-1}_{j}\;
  \partial h_j e_{ji} + e^{-1}_{ji} \partial e_{ij}$. 
\end{proof}

The principal bundles $\overset{\frown}{\omega}: P\longrightarrow X $
associated to $E$ is defined, with respect  to the covering
$\mathcal{U}=\{U_i\}$, by the transition function $\{e_{ij}\}$ acting
on the fibre $GL(m, \mathbb{C})$ as follows: 
$$
Z_i=e_{ij}Z_j \qquad (Z_i, Z_j \in  GL (m, \mathbb{C})).
$$

The computations above shows that 
$$
Z^{-1}_i (l_i Z_i+dZ_i) = Z^{-1}_j (l_jZ_j+dZ_j) ~ \text{ on } ~
\overset{\frown}{\omega}^{-1} (U_i \cap
\overset{\frown}{\omega}(U_j)
$$ 

Hence $\omega_i = Z^{-1}_{i}(l_i Z_i+dZ_i)$ is the local
representation on $\overset{\frown}{\omega}^{-1}(U_i)$ of a global
1-form $\omega$ with values in the Lie algebra of $GL(m, \mathbb{C})$.

Let\pageoriginale $P_\xi$ be the tangent space to $P$ at a point $\xi
\in  P$. Let $Q_\xi \subset P_\xi$ be the subspace annihilated by
$\omega$. The family of these subspaces defines a connection in $P$
\cite{key26}. This proves the lemma.   

The covariant derivation associated to the connection defined above is
a map 
$$
\triangledown : \Gamma (X , \mathcal{A} (E)) \rightarrow \Gamma (X ,
\mathcal{A}(E \otimes \Theta ^*)
$$ 
where for a vector-bundle $F$, $\mathcal{A} (F)$ denotes the sheaf of
germs of differentiable sections of $F$, and $\Theta^*$ is the bundle
of 1-forms with value in $\mathbb{C}$. (In the above we regard $E$ as a
differentiable vector bundle). $\triangledown$ is defined in an open
set $\{U_i\} $ of the covering $\mathcal{U}$ as follows. 

Let $z^1 ,\ldots , z^n$ be a system of local coordinates on $U_i$. Since
we are given a local trivialisation over $U_i$ of $E$, a section
$\sigma$ over $U_i$ of $E$ may be regarded as function on $X$ with values
in  $\mathbb{C}^m$ (rank $E =m$). We then have  
$$
\displaylines{\hfill 
  \triangledown_\alpha (\sigma) = \triangledown
  (\sigma)\left(\frac{\partial}{\partial z^\alpha}\right) = \partial \sigma
  \left(\frac{\partial}{\partial z^\alpha}\right) + l_i
  \left(\frac{\partial}{\partial z^\alpha}\right) (\sigma)\hfill \cr
  \text{and}\hfill  
  \triangledown_{\overline{\alpha}}(\sigma) = \triangledown
  (\sigma)\left(\frac{\partial}{\partial \overline{z}^\alpha}\right) =
  {\overline{\partial}} \sigma \left(\frac{\partial}{\partial
    z^\alpha}\right).\hfill } 
$$
 
It is easy to check that these local representation defined a global
map $\triangledown$ as above. 

We consider next the curvature form of the connection.
 
Let $\Omega $ be the curvature form of the connection form $ \omega$ 
in the\pageoriginale principal bundle $P$ associated to $E$. The values of
$\Omega$ on a pair, $u_1, u_2$, of tangent vector fields to $P$ is given
by the structure formula (\cite{key26}, 34-35):  
$$ 
\Omega (u_1 ,u_2)= d \omega (u_1 ,u_2 ) + \frac{1}{2} [\omega (u_1) , 
  \omega (u_2)].
$$ 
Hence the component of type (1,1) of $\Omega$ is given on
${\bar{\omega}}^{-1} (U_i)$ by  
 $$ 
\displaylines{\hfill 
  {\overline{\partial}}\omega_i = Z_i^{-1} s_i Z_i ,\hfill \cr
  \text{where} \hfill s_i = {\overline{\partial}} l_i. \hfill }
$$
 We shall call s the curvature form of the (connection $\{l_i\}$
 defined by the) metric $\{h_i\}$ along the fibers of $E$. On $U_i \cap
 U_j$ 
 $$
 s_i = e_{i j} \,s_j\, e_{j i}.
 $$
 Hence we have

 \begin{lemma}\label{chap3:lem3.2} %%% 3.2
   The curvature form of the connection defined above is given in $U_i$
   by the matrix valued 2-form 
   $$
   (s^a_t)_i = s_i ={\overline{\partial}} l_i 
   $$
   Hence it is a form of type (1,1) with values in the ``adjoint''
   bundle End $(E)$.
 \end{lemma}

   The bundle $ \Theta^*$ decomposes canonically as a direct sum
   $$  
   \Theta^* \backsimeq   \Theta^*_\circ \oplus \bar\Theta^* _ \circ
   $$
where\pageoriginale $\Theta_\circ$ (\resp $\overline{\Theta}_\circ$) is the
holomorphic tangent bundle (\resp anti holomorphic tangent bundle) of
$X$. (The two bundles in the above decomposition are regarded as
differentiable bundles. As a differentiable bundle $\Theta^*_\circ$
(\resp $ (\overline {\Theta}^*_\circ),$  is simply the bundle of
$C^\infty-$ forms on $X$ of type (1,0) (\resp type (0,1)). The
decomposition above gives a direct-sum representations 
$$
\mathcal{A} ( \Theta^*)\backsimeq \mathcal{A}(\Theta^*_\circ) \oplus
\mathcal{A}(\overline {\Theta}^*_\circ),
$$
and for any holomorphic vector bundle $E$ on $X$,
$$
\Gamma (X, \mathcal{A} (E \otimes \Theta^*)) \backsimeq \Gamma (X,
\mathcal{A}(E \otimes \Theta^*_\circ )) \oplus\Gamma (X, \mathcal{A}(E
\otimes \Theta^*_\circ )).
$$

Now if we are given a connection on $E$, it defines as we have remarked
a covariant derivation 
$$
\triangledown : \Gamma (X,\mathcal{A}(E)) \rightarrow \Gamma (X,
\mathcal{A}(E \otimes \Theta^*));
$$
composing with the natural projections defined by the direct sum
decomposition above we obtain maps 
$$
\displaylines{\hfill 
  \triangledown' : \Gamma (X,\mathcal{A}(E)) \rightarrow \Gamma (X,
  \mathcal{A}(E \otimes \Theta^*_\circ))\hfill \cr
  \text{and } \hfill 
  \triangledown'': \Gamma (X,\mathcal{A}(E)) \rightarrow \Gamma (X,
  \mathcal{A}(E \otimes \overline{\Theta}^*_\circ)).\hfill \phantom{and}}
$$ 

\begin{remarks*}
\begin{enumerate}
\renewcommand{\labelenumi}{\bf(\theenumi)}
\item As it has been remarked at the end of \S \ref{chap1:sec3}, the metric $
  {\{h_i\}}$ on $E$ induces a metric on the dual $E^*$: this is
  simply given by the family of positive definite matrix-valued
  functions  $\left\{ {}^t h^{-1}_i \right\}_{i \in I}$.
  The corresponding\pageoriginale connection is given locally on $ U_i
  $ by the form 
  $-^t l_i$.  It is also easy to see that the curvature form of this
  connection is given by $\{-^ts_i\}_{i \in I}$  

\item Let $\bar{E }$ denote the ``anti holomorphic'' bundle associated to
  $E$. $ \bar{E}$ is the bundle with transition functions  
  $ \bar{e}_{ij}$.  Then ${}^th_i =\bar{h}_i$ define a metric on
  $\bar{E}$ as well. In this case, the local forms  
  $$ 
  \bar{l}_i =  \bar{h}^{-1}_i ~~\bar{\partial} ~~\bar{h}_i
  $$  
  define a connection of type (0,1). When we speak of a connection on
  $\bar{E}$ without further  comment, it will always be of this
  connection (of course, it is necessary to assume given a metric
  or at least a connection on $E$.) 

\item Suppose that $E$ and $F$ are holomorphic vector bundles on $X$ and
  that we given hermitian metrics $\{h^1_{i}\}_{i \in I}$ and $
  \{h^2_{i}\}_{i \in I} $  
  on $E$ and $F$ respectively. (We assume here, as we may, that $E$ and $F$
  are defined by means of transition function $\{e_{ij}\}\{f_{ ij}\}$  
  with respect to the same covering $ \mathcal{U}=\{U_i\}_{i \in I}).$
  Then the family of matrix valued functions $\{h^1_{i} \otimes
  h^2_{i}\}$ defines a metric  $\{h_i\}$ along the fibres of
  $E \otimes F$. We have then
  $$
  \displaylines{\hfill  
    \partial (h_i) = \partial(h^1_{i} \otimes h^2_{i})  =\partial
    h^1_{i}\otimes  h^2_{i} + h^1_{i} \otimes \partial h^2_{i}\hfill \cr
    \text{so \; that}\hfill  l_i  =h^{-1}_{i} \partial h_i
      =h^{-1}_{i} \partial h^1_{i} \otimes I_{r_2} + I_{r_1} \otimes
      h^{2^{-1}}_{i} \partial h^2_{i} \hfill }
  $$
    where\pageoriginale $r_1$ (\resp ~ $r_2$) is the rank of $E$
    (\resp rank $F$) and for an  
    integer $m$, $I_m$ denotes the $(m \times m)$ identity matrix. One sees
    immediately then that the curvature form $s_i$ is given by the
    formula 
    $$ 
    {s_i = \overline{\partial}l_i = s^1_i \otimes I_{r_{2}} + I_{r_{1}}
      \otimes s^2_i.}
    $$

    It is clear that the above considerations can be carried over to the 
    case when one or both of $E$ and $F$ are anti holomorphic.
\end{enumerate}
\end{remarks*}

In the sequel, we call a connection on a (differential) vector
bundle (with complex fibres) a $\partial -$connection (\resp  $
{\overline {\partial}} $- connection) 
if the associated connection form is of type (1,0) (\resp type (0,1)).
The connection $ {l_i}$ on a holomorphic bundle $E$ defined with respect
to a metric $ {h_i}$ is then a $\partial -$ connection while that on
the conjugate bundle $ {\overline {E}}$ is a ${\overline{\partial}-}$
connection.   

Now if $ {\{ h_i \}_{i \in I}}$ is a hermitian metric along the fibres
of $E$, then it can be regarded as a section $h$ of the bundle $E^*
\otimes {\bar{E}}^*$. We have also a canonical connection on $E^*
\otimes {\bar{E}}^*$. We denote the covariant derivative on section of
$E^* \otimes \bar{E}^*$ again by $\triangledown$: 
$$ 
 \triangledown : \Gamma (X, \mathcal{A}(E^* \otimes {\bar{E}}^*))
\rightarrow \Gamma (X, \mathcal{A}(E^* \otimes {\bar{E}}^* \otimes \Theta^* )). 
$$

\setcounter{prop}{0}
\begin{prop}\label{chap3:prop3.1}%% 3.1
  ${\triangledown h = 0}$ 
\end{prop}

\begin{proof}
  Let $ {z^1,..,z^n}$ be a coordinate system on ${U_1}$. In this proof
  we write $l$ for $l_i$ and $h$ for ${h_i}$. 
  
  We\pageoriginale have 
  \begin{align*}
    \triangledown_{\alpha} h = ( \triangledown h )
    \left(\frac{\partial}{\partial z^{\alpha}} \right)   
     & = \partial_{\alpha}h_{ \overline{a}b}- \underset{c}{\sum}
     l^c _{b \alpha} h_{\overline{a}c}\\ 
     & = \partial_x h_{\overline{a}b}  -\sum
     (h^{-1})^{c\overline{d}}\partial_\alpha  h_{\overline{d}b}
     h_{\overline{a}c} \\   
     & = \partial_\alpha h_{\overline{a}b} -\sum \delta^d_a
     \partial_\alpha h_{\overline{d}b} = \partial_\alpha
     h_{\overline{a}b} -\partial_\alpha h_{\overline{a}b} =0. 
  \end{align*}
  Similarly one shows that
  $$ 
  \triangledown_{ \overline{\alpha}} h= \triangledown h
  (\frac{\partial}{\partial {\overline{z}}^\alpha})=0.  
  $$
  Hence the proposition.

  (Here $h$ is regarded as a function from $U_i$ into $\bigg\{ \mathbb{C}^m
  \otimes {\overline{\mathbb{C}}}^{m} \bigg \}^{*} $ using the given 
  trivialisation of $E \mid U_i$. A vector $t \in \mathbb{C}^m$ is an $m$-tuple
  denoted $(t^a)_{1 \leq a \leq m}$ and $\overline{t} \epsilon
  {\overline{\mathbb{C}}}^{n}$ is again an $m$-tuple but is denoted 
$(t^{\overline{a}})_{1 \leq a \leq m}$. Further conventions are as
  follows: a vector in the dual of  $\mathbb{C}^{m}$ is an $m$-tuple but with
  subscripts instead of super scripts. In other words we write  
  \begin{align*}
    t & = \sum t^{a} e_{a} ~\text{ for} ~t \in \mathbb{C}^{m}\\
    t &=  \sum t^{\overline{a}} \overline{e}_{a} ~\text{ for } t~ \in
    \bar{\mathbb{C}}^{m}\\ 
    t & = \sum t_{a} e^{*}_{a} ~\text{ for}~ t \in \mathbb{C}^{m^{*}}\\
    t &= \sum t_{\overline{a}}\overline{e}^{*}_{a} ~\text{ for } ~t
    \in {\overline{\mathbb{C}^{m^*}}} 
   \end{align*}
  where $ \{e_{a}\}$ (resp. $\{e_{a}\}$, $\{e^{*}_{a}\}$,
  $\{\overline{e}^{*}_{a}\}$ is the canonical (resp. conjugate of the
  canonical, dual of the canonical, conjugate dual of the canonical)
  basis of   $\mathbb{C}^{m}$).   
\end{proof}

\begin{prop}\label{chap3:prop3.2}%%% 3.2
  Let\pageoriginale $z^1,\ldots ,z^n$  be a coordinate system on
  $U_i$. Then in $U_i$, the curvature form can be written as  
  $$ 
    a_i=(s^a_{b\bar{\beta}\alpha})_{ab}d\bar{z}_\beta \wedge 
    dz_{\alpha,}
$$
where
\begin{equation*}
    (s^{a}_{b \bar{\beta} \alpha})_{ab} = \triangledown_{\bar{\beta}}
    \triangledown_\alpha- \triangledown_\alpha
    \triangledown_{\bar{\beta}} \tag{3.1}\label{eq3.1}
\end{equation*}
\end{prop}

\begin{proof}
  Let $t=(t^a)$ be a section of the bundle over $U_i$.
  
  Then 
  $$ 
  \triangledown {_{\bar{\beta}}} t = \partial {_{\bar{\beta}} } t =
  (\partial {_{\bar{\beta}}}t^a) 
  $$

  Hence 
  $$
  (\triangledown {_{\alpha} } \triangledown {_{\bar
      {\beta}}}t){^a} = \partial {_\alpha}
  {\partial_\frac{t^a}{\beta}}  + \underset{b}{\sum} 
  l^a_{b\alpha} {\partial {_{\bar{\beta}}}} t^b 
  $$
  where $l =\sum l^a_{b\beta}  \; dz_\alpha$ is the connection form.
  $$
  \displaylines{\hfill 
  \triangledown _\alpha t=\partial_\alpha
  t+(l^a_{b\alpha})(t)=\partial_\alpha
  t^a+\underset{b}{\sum} \; l^a_{b\alpha} \: t^b \hfill \cr
  \text{so that}\hfill\cr
  \hfill \begin{aligned}
    \triangledown{_{\bar{\beta}}}\triangledown{_\alpha}t &
    =\partial{_{\bar{\beta}}} \partial{_{\alpha}}t^a +
    \underset{b}{\sum}\partial{_{\bar{\beta}}}l^a_{b\alpha} 
    t^b+\underset{b}{\sum} \; l^a_{b\alpha} \partial 
    {_{\bar{\beta}}}t^b\\ 
    &=\partial{_{\bar{\beta}}}\partial{_\alpha}t^a+s^a_{b{\bar{\beta}}\alpha}
    t^b+\underset{b}{\sum} \; l^a_{b \alpha} \partial{_{\bar{\beta}}}
    t^b.
  \end{aligned} \hfill }
  $$

  
  Hence $-(\triangledown{_\alpha} \triangledown{_{\bar{\beta}}}
  t-\triangledown {_{\bar{\beta}}} \triangledown {_\alpha} t)^a =
  \underset{b}{\sum} s^a_{b{\bar{\beta}}\alpha} t^b$. 

  This proves the proposition. 
  
  The equation (\ref{eq3.1}) is known as the \textit{Bianchi Identity}.
  
  (We remark that $s$ is a form of type (1,1).).

  The\pageoriginale following lemma is easy to prove.
\end{proof}

\begin{lemma}\label{chap3:lem3.3}%% 3.3
  $\triangledown \#  = \# \triangledown$.

  Next we specialize the preceding considerations to the case $E =
  \Theta_o$ the homomorphic tangent bundle. Let $z^1, \ldots,z^n$ be a
  local coordinate system in an open set $U\subset X$. Let  
  $$
  {g} = \sum g_{\bar{\alpha}\beta}dz^{-\alpha}dz^{\beta}
  $$
  be the expression for the hermitian metric in this coordinate
  system. The associated $\partial$-connection is then given in $U$ by
  the (1,0)- form 
  $$ 
  \displaylines{\hfill  
  \sum C^{\alpha}_{\beta\gamma}d z^{\gamma}\hfill \cr
  \text{where}\hfill 
  C^{\alpha}_{\beta\gamma} = \sum g^{\bar{\tau}\alpha}\frac{\partial
    g_{\bar{\tau} \beta}}{\partial z^\gamma}.\hfill }
  $$
  (here, as is usual, $g^{\bar{\alpha}\beta}$ is defined by
  ($g^{\bar{\alpha}\beta}) (g_{\bar{\alpha}\beta}$) = Identity). 
\end{lemma}

That means that
\begin{equation*}
\begin{cases}
&    \triangledown_{\beta}\left(\frac{\partial}{\partial z^{\alpha}}\right)=
    \sum C^{\alpha}_{\beta \gamma} \frac{\partial}{\partial
      z^{\gamma}}\\  
&    \triangledown_{\bar{\beta}} ~\frac{\partial}{\partial z^{\alpha}}=
    0
\end{cases} \tag{3.2}\label{eq3.2}
\end{equation*}
where
$$
 \triangledown : \Gamma (X,\mathcal{A}(\Theta)) \rightarrow
   \Gamma (X,\mathcal{A}(\Theta_o \otimes\Theta^{*}))
  $$
is the covariant derivation.

The\pageoriginale $\partial$- connection defined above is not in general 
symmetric:  $ C^{\gamma}_{\alpha \beta}\neq C^{\gamma}_{\beta
  \alpha}$; in fact this connection is symmetric  
if and only if the hermitian metric $g$ is Kahler. We set then
$$ 
S=\sum \frac{1}{2} ( C^{\gamma}_{\beta \alpha}-C^{\gamma}_{\alpha
  \beta})dz^{\beta}\wedge dz^{\alpha}.
$$

Clearly $S$  is an alternating (2,0) form with values in the tangent
bundle $\Theta_0$. It is called the torsion form of the $\partial$-
connection. Its vanishing characterises the Kahler metrics. 

A hermitian metric on the holomorphic tangent bundle
defines a Riemannian metric as well on $X$. We have corresponding
to this metric a Riemannian connection on $X$. The associated
covariant derivative is a map 
$$  
D:\Gamma(X \mathcal{A}(\Theta)) \to \Gamma (X, \mathcal{A}(\Theta
{}\otimes \Theta^*)).
$$

In particular, we have
$$  
D:\Gamma (X \mathcal{A}(\Theta_0)) \to \Gamma (X, \mathcal{A}(\Theta_0
\otimes \Theta^*)).
$$

Denoting, as is usual, the Riemann-Christofel symbols by
$\Gamma^{\gamma}_{\alpha \beta}$, $\Gamma^{\gamma}_{\alpha \beta}$,.. etc, 
one can prove easily that these are related to the $C^{\gamma}_{\alpha
  \beta}$ as follows: 
\begin{align*}
  \Gamma^{\alpha}_{\beta \gamma}  &= \frac{1}{2} (C^{\alpha}_{\beta
    \gamma} + C^{\alpha}_{\gamma \beta})\\ 
  \sum\limits_\gamma \Gamma^{\bar{\gamma}}_{\beta \bar{\gamma}} &=-\sum\limits_r
  S^{\gamma}_{\beta \gamma } 
\end{align*}

Since $\Theta \backsimeq \Theta_\circ \oplus \overline{\Theta} _\circ$ as a
differentiable vector bundle, we have a direct sum decomposition  
$$
\Gamma (X,\mathcal{A}(\Theta \otimes \Theta^*))\backsimeq \Gamma
(X,\mathcal{A}(\Theta \otimes \Theta^*_\circ))\otimes\Gamma
(X,\mathcal{A}(\Theta \otimes \Theta^*_\circ))
$$
and\pageoriginale hence a natural projection
$$
\Gamma (X, \mathcal{A}(\Theta \otimes \Theta ^*))\rightarrow \Gamma
(X, \mathcal{A}(\Theta \otimes \Theta ^*));
$$
composing this projection with $D$ and $\triangledown$,  we obtain two
linear maps 

$\begin{aligned}
& \qquad  {\triangledown'' : \Gamma (X, \mathcal{A}(\Theta
    _o))\rightarrow \Gamma  (X,\mathcal{A}(\Theta_o \otimes
    \overline{\Theta}_o^{*}))}\\ 
{\rm and}  & \qquad    {D'' : \Gamma (X, \mathcal
  {A}(\Theta_o))\rightarrow \Gamma (X,\mathcal{A}(\Theta _o \otimes
  \overline{\Theta}_o^{*}))}. 
\end{aligned}$

If the metric is Kahler, these two maps coincide.
In a similar way, composing the natural projection
$$ 
\Gamma (X, \mathcal{A}(\Theta \otimes \Theta^*))\rightarrow \Gamma
(X,\mathcal{A}(\Theta \otimes \Theta_o^{*}))
$$
with $\triangledown$ and  $D$ we define two linear maps,
\begin{align*}
  \triangledown' &: \Gamma (X, \mathcal{A} \Theta_o))\rightarrow
  \Gamma (X,\mathcal{A}(\Theta_o \otimes \Theta_o{^*}))\\
  D' & : \Gamma (X, \mathcal{A} \Theta_o)) \rightarrow \Gamma (X,
  \mathcal{A}(\Theta_o \otimes \Theta_{o}^{*}))
\end{align*}
which coincide if the hermitian metric of $X$ is a Kahler metric.

Suppose now that we are given a vector bundle $E$ on $X$ and hermitian
metrics on $X$ and along the fibres of $E$. We then have canonical
$\partial$ - connections on the bundle $E$ and $\Theta_o$ and a
$\partial$- connection on $\Theta_o$. These connections extend
canonically to connections on each of the bundles 
$$
\overset{p}{\Lambda} \Theta _o^* \otimes 
\overset{q}{\Lambda}\overline{\Theta}_o^* \otimes E
$$

We\pageoriginale denote by $\triangledown$ the covariant derivation in
any of these bundles: 
$$
\triangledown: \Gamma(X,\mathcal{A}(\overset {p}{\wedge}
\Theta^*_o \otimes \overset {q}{\wedge} \overline
{\Theta}^*_o \otimes E))\rightarrow \Gamma (X,
\mathcal{A}(\overset{p}\wedge \Theta^*_o \otimes
\overset {q}{\wedge} \overline {\Theta}^*\otimes E
\otimes \Theta^*))
$$

Once again, composing with the natural projections, we can define 
\begin{align*}
\triangle''& : \Gamma(X,\mathcal{A}(\overset {p}{\wedge}
\Theta^*_o \otimes \overset {q}{\wedge} \overline
            {\Theta}_o \overset{*} \otimes E))\rightarrow \Gamma
            (X, \mathcal{A}(\overset{p}\wedge \Theta^*_o \otimes
            \overset {q}{\wedge} \overline{\Theta}^*_o \otimes
            E \otimes \overline{\Theta}^*_o)),\\
 \triangle'& : \Gamma(X,\mathcal{A}(\overset {p}{\wedge}
 \Theta_o \overset {q}{\wedge} \overline{\Theta}_o
 \otimes E)) \rightarrow \Gamma (X, \mathcal{A}(\overset{p}\wedge
 \Theta_o \overset{q}\wedge \Theta_o \otimes E \otimes E
 \otimes \Theta^*_o),
\end{align*}

Let $s= \sum\limits_{\alpha, \beta} s^a_{b \bar{\alpha} \beta}  d
{\bar{z}}^\alpha \wedge dz^\beta$ be the curvature form of the
$\partial$ -connection on $E$ in a co-ordinate open set $Y$ over which
the bundle $E$ is trivial, the complex analytic coordinates in $U$ being
$(z^1, \ldots, z^n)$ : if (rank $E$)$=m$, $a,b$ run through 1 to $m$ and for
fixed $\alpha$, $\beta$, $(s^a_{b \overline{\alpha} \beta})$ is an $(m
\times m)$-matrix. 

In the same coordinate neighborhood, let
$$
L=\sum\limits_{\sigma, \tau} L^{\alpha}_{\beta \bar{\sigma} \tau} 
 d {\bar{z}}^\sigma  \wedge dz^\tau
$$
be the curvature form of the $\partial$-connection on the
holomorphic tangent bundle $\Theta_o$. For fixed $\sigma,
\tau, (L^{\alpha}_{\beta \bar{\sigma} \tau})$ is an
$(n \times n)$-matrix. 


\section{Local expressions for $\overline{\partial}$, $\vartheta$ 
  and $\Box$} %%%% 10

We will now obtain a local expression for the operator 
$$
\Box = \overline{\partial} \vartheta + \vartheta \overline {\partial}
$$
in terms of $\triangledown_\alpha$, $\bar{\triangledown}_\alpha$ and
the forms $L$ and $s$ above. 

We\pageoriginale adopt the following notation : for a $p$- tuple 
 $A= (\alpha_1,\ldots \alpha_p)$, we set $dz^A =
dz^{\alpha_1}\wedge \ldots \wedge dz^{\alpha_p}$  
 (\resp $ dz^{-A} = d\bar{z}^{\alpha_1} \wedge \ldots \wedge d
\bar{z}^{\alpha_p}$). Then if for $ \phi  \epsilon C^{pq(X,E)}$,
we  set 
$$   
\displaylines{\hfill 
  {\varphi \left(\frac{\partial}{\partial z^{\alpha 1}},\ldots ,
    \frac{\partial}{\partial z ^{\alpha p}}, \frac{\partial}{\partial
    {\bar{z}}^{\beta 1}},\ldots , \ldots \frac{\partial}{\partial{\bar{z}}^{\beta
     q}}\right)=\varphi^{a}_{A\bar{B}}}\hfill \cr
  \text{we have} \hfill 
    {\varphi = \displaystyle\sum_{A,B}\frac{1}{p! q!}\varphi^a_{A\bar{B}}
     dz^A \wedge d\bar{z}^B} \hfill }
$$
where $A= (\alpha_1,\ldots ,\alpha_p)$ and $B= (\beta _1,\ldots
,\beta_q)$ for a $p$-tuple $A$ and a $(q+1)$- tuple $(\beta_1, \ldots, 
\beta _{q+1})$ 
we have  
$$   
(\bar{\partial}\varphi)^a _A \bar{\beta}_{1} \ldots
  \bar{\beta}_{q+1} = (-1)^p \sum\limits^{q+1}_{r=1} (-1)^{r-1}
  \bar{\partial}_{\bar{\beta}_r} \varphi^a_{A \bar{\beta}_1 \ldots
    \beta_r \ldots \beta_{q+1}}.
$$

On the other hand
\begin{align*}
  \nabla_{\bar{\beta}_r} \varphi^a_{\bar{\beta}_1 \ldots
    \hat{\bar{\beta}} r \ldots \beta_{q+1}} &
  =\partial_{\bar{\beta}_r}\varphi^a_{A\bar{\beta}_1 \ldots
    \hat{\bar{\beta}}_r\ldots \bar{\beta}_{q+1}} - \sum\limits_{i\neq
    r} C^{\bar{\alpha}}_{\beta_i\beta_r}\varphi^a_{A \bar{\beta}_1\ldots
    \bar{\beta}_{i-1}\bar{\alpha}\bar{\beta}_{i+1}\ldots
    \hat{\bar{\beta}}_r \ldots \bar{\beta}_{q+1}}\\ 
  & =\partial_{\bar{\beta} r} \varphi^a_{A \bar{\beta}_1\ldots
    \hat{\bar{\beta}}_r \ldots \bar{\beta}_{q+1}}\\ 
  & \qquad -\sum\limits_{i\neq r}\overline{B^\alpha_{\beta_i
      \beta_r}} + \overline{S}^\alpha_{\beta _i \beta _r} \varphi^a_{A 
    \bar{\beta}_1\ldots \bar{\beta}_{i-1} \bar{\alpha} \beta_{i+1}\ldots
    \hat{\bar{\beta}}_r \ldots \beta_{q+1}} 
\end{align*}\pageoriginale
where $C^\gamma_{\alpha \beta}$ has been defined before and
$$ 
\displaylines{\hfill 
  B^\gamma_{\alpha\beta}=\frac{1}{2}(C^{\gamma}_{\alpha\beta} + 
  C^{\gamma}_{\beta\alpha})\hfill \cr  
  \text{while} \hfill
  S^{\gamma}_{\alpha\beta}=\frac{1}{2}(C^{\gamma}_{\alpha
    \beta} - C^{\gamma}_{\beta \alpha})\hfill \cr  
  \text{so that}\hfill  S = \sum S^{\gamma}_{\alpha \beta
  }dz^{\alpha}\wedge dz^\beta \hfill }
$$ 
is the torsion of the connection on $\Theta_\circ$.

We have therefore
\begin{multline*}
  (\bar{\partial}\varphi^a)_{A  \bar{\beta}_1\ldots \bar{\beta}_{q+1}}=
  (-1)^p\sum (-1)^{r-1}\nabla\bar{\beta_r}
  \varphi^a_{A\bar{\beta}_1\ldots \hat{\bar{\beta}}_r \ldots
  \bar{\beta}_{q+1}}\\  
  +(-1)^p \sum\limits_{i\neq r
  }(-1)^{r-1}\bar{S}^\alpha_{\beta_i\beta_r} 
\varphi^a_{A\bar{\beta}_1\ldots(\alpha)_1\ldots \hat{\bar{\beta}}_r\ldots
  \bar{\beta}_{q+1}}. 
\end{multline*}

Let\pageoriginale
$$ 
S: C^{pq}(X,E)\rightarrow C^{p,q+1}(X,E)
$$
be the operator defined by 
\begin{equation*}
  (S\varphi)^a_{A \bar{\beta}_1\ldots
  \bar{\beta}_{q+1}} = (-1)^p \sum\limits_{i \neq r} 
(-1)^{r-1}\bar{S}^\alpha_{\beta _i \beta_r}
\varphi^a_{A\beta_1 \ldots \beta_{i-1} \alpha \beta_{i+1}\beta _\gamma
  \bar{\beta}_{--q-1}} \tag{3.4}\label{eq3.4}
\end{equation*}
and set
\begin{equation*}
  \bar{\partial}=\tilde{\partial}+S, \tag{3.5}\label{eq3.5}
\end{equation*}
so that
\begin{equation*}
  (\tilde{\partial\varphi})^a_{A \bar{\beta}_1\ldots
    \bar{\beta_{q+1}}}=(-1)^p\sum
  (-1)^{r-1} \nabla_{\bar{\beta_r}} \varphi^\alpha_{A \bar{\beta_1}\ldots
    \hat{\bar{\beta}}_r\ldots \beta_{q-1}} \tag{3.6}\label{eq3.6}
\end{equation*}
Let
\begin{equation*}
  \tilde{\vartheta}=-*\#^{-1}\tilde{\partial} * \#: 
  C^{pq}(X,E)\rightarrow C^{p,q-1}(X,E) \tag{3.7}\label{eq3.7}
\end{equation*}
and 
\begin{equation*}
  T=-*\#^{-1}  S * \# : C^{pq}(X,E)\rightarrow C^{p,q-1}(X,E)
  \tag{3.8}\label{eq3.8} 
\end{equation*}
so that
\begin{equation*}
  \vartheta = \tilde{\theta}+ T. \tag{3.9}\label{eq3.9}
\end{equation*}

Finally let  
\begin{equation*}  
  \tilde{\square} = \tilde{\partial}\tilde{\vartheta} +
  \tilde{\vartheta}\tilde{\partial}  
\end{equation*}

If\pageoriginale the hermitian metric on $X$ is a Kahler metric, then  
$$   
\bar{\partial}=   {\overset{\thicksim }\partial}, \quad  \vartheta  = 
{\overset {\thicksim} \vartheta} , \quad \Box =  {\overset{\thicksim} \Box}.
$$ 
For  \quad $\varphi \in C^{pq} (X,E)$
\begin{equation*}
\left(\tilde{\vartheta} \varphi \right)^a_{\overline{AB' }} = (-1)^{p-1}
\nabla_{\alpha} \varphi^a_A \alpha_{\overline{B'}}  , \tag{3.10}\label{eq3.10}
\end{equation*}
so that, exactly as in the case of the Laplacian $\triangle$ in
Chapter \ref{chap2},  we have 
\begin{equation*} 
  \left(\overset{\thicksim}{\Box} \varphi \right)^a_{\overline{AB}}  = -
  \nabla_{\alpha} \nabla^{\alpha} \varphi^a_{\overline{AB}} +
  \underset{r=1}{\overset{q}\sum } (-1)^{r-1} ( \nabla_\alpha
  \nabla_{\overline{\beta_r}} -
  \nabla_{\overline{\beta_r}}\nabla_\alpha ) \varphi^a_A
  \alpha_{\overline{B'_r}}   \tag{3.11}\label{eq3.11}
\end{equation*}
where 
$$ 
\displaylines{\hfill 
  \nabla^\alpha = g^{\alpha {\overline{\beta}}}
  \nabla_{\overline{\beta}}, \hfill \cr 
  \text{and}\hfill  
  A = ( \alpha_1,........\alpha_p), B= ( \beta_1,......\beta_q), B'_r =
  ( \beta_1,.......,\hat{\beta}_r ,...., \beta_q).\hfill}
$$

In view of the Ricci identity, the summand of (\ref{eq3.11}) can be expressed
by  
\begin{equation*}
  \underset{r=1}{\overset{q}\sum} (-1)^{r-1}( \nabla_\alpha
  \nabla_{{\overline{\beta_r}}} - \nabla_{{\overline{\beta_r}}}
  \nabla_\alpha ) \varphi^a_A \alpha _{\overline{B'_r}}= (\overset
        {\thicksim }{\mathcal{K}} \varphi)^a AB    \tag{3.12}\label{eq3.12}
\end{equation*}
where $\overset{\thicksim}{\mathcal{K}}$ is a mapping 
$$   
\overset {\thicksim}{\mathcal{K}} : C^{pq}(X,E) \rightarrow
C^{pq}(X,E),
$$
which is linear over $C^{\infty}$ functions, whose local expression
involves linearly (with integral coefficients) only the
coefficients of the curvature\pageoriginale forms, $s$ and $L$, of $E$
and $\Theta_{\circ}$. 

By Remark (3) after Lemma \ref{chap3:lem3.2} we have 
\begin{equation*} 
  \left(\overset{\thicksim} {\mathcal {K}} \varphi \right)^a_{A \overline{B}} =
  \overset{q} {\underset  {r=1}{\sum}}(-1)^r s^a_{b
    \overline{\beta}_{r} \alpha} \varphi^b_A \alpha_{\overline{B'_r}}
  + \left(\overset{\thicksim}{\mathcal{K_\circ}} \varphi \right)^a_{A
    \overline{B}},  \tag{3.13} \label{eq3.13}
\end{equation*} 
where  $\overset{\thicksim}{\mathcal{K_\circ}}$ involves only the
curvature tensor of $ \Theta_{\circ}$, and is completely independent
of $E$. 

Formula (\ref{eq3.11}) can be also written as 
\begin{equation*} 
  (\overset{ \thicksim}\Box \varphi)^a _{A \overline{B }}= -
  \nabla_\alpha \nabla^\alpha \varphi^a _{A \overline{B}} + ( \overset{
  \thicksim}{\mathcal{K}} \varphi)^a_{A \overline{B}}.
  \tag{3.14}\label{eq3.14} 
\end{equation*} 

Now 
 \begin{align*}
   \Box & = (\bar{\partial} \vartheta + \vartheta \bar {\partial})=  (
   \bar {\partial} + S) (\overset{\thicksim}{\vartheta} + T) + (
   \overset{\thicksim}\vartheta + T ) (\overset{\thicksim}\partial + S
   ) \\ 
   & = \overset{\thicksim}{\Box}+ \overset{\thicksim}{\partial} T +
   T\overset{\thicksim}{\partial} + \overset{\thicksim}{\vartheta}S +
   S\overset{\thicksim }{\vartheta}+ ST + TS.  
 \end{align*}
It follows that

\begin{lemma}\label{chap3:lem3.4}%%% 3.4
  For any $\varphi \epsilon C^{pq}(X,E)$ 
  $$  
  (\Box \varphi)^a _{A\overline{B}}= (\overset{\thicksim} {\Box}
  \varphi)^a _{A\overline{B}}+ (F_1 \varphi)^a _{A\overline{B}}+(F_2
  \nabla' \varphi)^a _{A\overline{B}}+ (F_3 \nabla'' \varphi)^a
  _{A\overline{B}} 
  $$
  where
  \begin{align*}
   F_1 &: C^{pq} (X,E) \rightarrow C^{pq}(X,E) ,\\
   F_2 &: C^{pq} (X, E \otimes \Theta^*_o ) \rightarrow C^{pq}(X,E), \\
  F_3 &: C^{pq}(X,E \otimes \Theta^*_o ) \rightarrow C^{pq }(X,E),
 \end{align*}
are\pageoriginale linear over $ C^\infty $ functions.  Their local
expression involves the tensor tensor and its first derivatives.   
\end{lemma}

If the metric on $X$ is a Kahler metric, then $F_1 \equiv 0, F_2
\equiv 0, F_3 \equiv 0$. 

Applying proposition \ref{chap3:prop3.1} to  the hermitian metric on
$X$ we have that   
$$ 
\nabla * = * \nabla. 
$$

Hence, by (\ref{eq3.14}),
\begin{equation*}
  \overset{\thicksim}\Box - *^{-1} \overset{\thicksim}\Box * =
  \overset {\thicksim} {\mathcal{K}-}*^{-1}\overset {\thicksim}
  {\mathcal{K}} *.  \tag{3.15}\label{eq3.15}
\end{equation*}

If the metric on $X$ is Kahler, this identity becomes
$$   
\Box - *^{-1}\Box * = \overset{\thicksim}{\mathcal {K}}
- *^{-1}\overset{\thicksim}{\mathcal {K}} *.
$$

A direct computation shows that, in this case,
$\overset{\thicksim}{\mathcal {K}}-*^{-1}\overset{\thicksim}{\mathcal 
  {K}}*$ does not depend on the curvature form of the Kahler
metric. We have in fact (see \cite{key2}, 103) 
$$ 
((\overset{\thicksim}{\mathcal {K}}-*^{-1}\overset{\thicksim}{\mathcal 
  {K}} *)    )^a _{A\overline{B}} = \overset{q}{\underset{i=1}\sum} 
(-1)^i s^a_{b \overline{\beta}_i \beta} \varphi^{b \beta}_{A
  \overline{B'_i}} + \overset{p}{\underset{j=1}\sum}(-1)^j
s^a_{b\overline{\alpha} \alpha_j} \varphi^{b
  \overline{\alpha}}_{A'_j\bar{B}} +
   s^a_{b \overline{\beta}}  \overline{\beta}
  \varphi^b_{A\overline{B}}
$$
where $A'_j= (\alpha_1,\ldots, \hat{\alpha_j} ,....,\alpha_p)$.

Starting from this formula, it is easy to check that 
$$   
  \overset{\thicksim}{\mathcal {K}}- *^{-1}\overset{\thicksim}{\mathcal
    {K}} *= e (s) \Lambda - \Lambda e (s),
$$
where\pageoriginale
$$
e(s):C^{pq}(X,E)\rightarrow C^{p+1, q+1}(X,E)
$$
is the linear mapping locally defined by
$$
 (e(s)\varphi)_{i}= \sqrt{-1} s ^{a}_{ib}\wedge \varphi^{b},
$$
and $\Lambda$ is the classical operator of the Kahler geometry (see
e.g. \cite{key35}, 42). Thus we have, in the Kahler case, 
$$
\Box - \ast^{-1} \Box * = e(s)\Lambda - \Lambda e(s)
$$
which, by (\ref{eq1.7}), can also be written
$$ 
\Box_{E} - \# ^{-1} \Box_{E}\ast \# = e(s) \Lambda - \Lambda e(s).
$$

This formula, which was first obtained in \cite{key4}, 483, yields
$W$-ellipti\-city conditions on K\"ahler manifolds (see \cite{key2}, ~).

If the metric on $X$ is not Kahler, then it follows from Lemma
\ref{chap3:lem3.4} and from (\ref{eq3.15}) that 
$$
\Box - \ast^{-1} \Box \ast = \overset{\thicksim}{K}-\ast^{-1}
\overset{\thicksim}{K}\ast + G_{1} \varphi + G_{2}\triangledown '
\varphi + G_{3}\nabla''  \varphi 
$$
where
\begin{align*}
  G_{1} &: C^{pq}(X, E)\rightarrow C^{pq}(X,E),\\
  G_{2} &: C^{pq}(X, E\otimes \Theta^{\ast}_{\circ})\rightarrow C^{pq}(X,E),\\
  G_{3} &: C^{pq}(X, E\otimes \Theta^{\ast}_{\circ})\rightarrow C^{pq}(X,E),
\end{align*}
are linear over $C^{\infty}$ functions. Their local expressions
involve the torsion tensor and its first covariant derivatives. If the 
hermitian metric\pageoriginale is a K\"ahler metric, then $G_{1}
\equiv 0$, $G_{2} \equiv 0$, $G_{3} \equiv 0$.  

\section{The main inequality}%% 11

We shall now establish an integral inequality which, under convenient
hypotheses, yields a sufficient condition for the $W^{pq}$-
ellipticity of $E$. 

Let $\varphi \in C^{pq}(X,E)(q > 0)$. Let $\xi$ and  $\eta$ be two
tangent vector fields to $X$, defined by 
\begin{align*}
  \xi & = (\xi^{\beta}=h_{\bar {ba}}\triangledown_{\bar
  {\gamma}}\varphi^{a}_{A}\beta_{\bar
  {B'}}\overline{\varphi^{b\bar{A}\gamma B'}},\xi^ {\bar{\beta}}=0)\\
  \eta & =(\eta^{\gamma}=0,
  \eta^{\bar{\gamma}}=h_{\bar{ba}} \triangledown_{\beta} \varphi^{a}_{A}
  \beta_{\bar{B'}}.\overline{\varphi^{b\bar{A}\gamma
      B'}}) 
\end{align*}

We have ($\eta $ being the complex dimension of $X$)
$$ 
div \xi = \displaystyle \sum_{i=1}^{2n} D_{i}
\xi^{i}=\sum^{n}_{\beta=1}\triangledown_{\beta}\xi^{\beta}-\sum_{\alpha,\beta}
S^\beta_{\alpha} \beta \xi^{\alpha},
$$ 
where
\begin{align*}
  \triangledown_{\beta}\xi^{\beta} & = h_{\bar{b}a}
  \triangledown_{\beta} \triangledown_{\bar{\gamma}} \varphi^{a}_{A}
  \beta_{\bar{B'}}. \overline{\varphi^{b\bar{a}\gamma B'}}+
  h_{\bar{b}a}\triangledown_{\bar{\gamma}}
  \varphi^{a}_{A}\beta_{\bar{B'}}. \overline{\triangledown_{\bar{B}} 
    \varphi^{b\bar{A}\gamma B'}}\\ 
& =h_{\bar{b}a}\triangledown_{\bar{\gamma}} \triangledown_{\beta}
  \varphi^{a}_{A} \beta_{\bar{B'}}. \overline{\varphi^{b\bar{a}\gamma B'}}+
  h_{\bar{b}a}\triangledown_{\bar{\gamma}} \varphi^{a}_{A}
  \beta_{\bar{B'}}. \overline{\triangledown_{\bar{B}} \varphi^{b\bar
      {A}\gamma B'}}\\   
 & \hspace{3cm} +h_{\bar{b}a}(\triangledown_{\beta} \triangledown_{\bar{\gamma}} -
  \triangledown_{\bar{\gamma}} \triangledown_{\beta}) \varphi^{a}_{A}
  \beta_{\bar{B'}}. \overline{\varphi^{b \bar{A} \gamma B'}}.
\end{align*}

The\pageoriginale last summand can be evaluated using the Ricci
identity. We have, by (\ref{eq3.12}).   
$$ 
h_{\bar{b}a}(\triangledown_{\beta} \triangledown_{\bar{\gamma}} - 
\triangledown_{\bar{\gamma}} \triangledown_{\beta})
\varphi^{a}_{A}\beta_{\bar{B'}}.\overline{\varphi^{b \bar{A}\gamma
    B}}=\frac{1}{q}
h_{ba}(\tilde{K}\varphi)^{a}_{A\bar{B}}\overline{\varphi^{b
    \bar{A}B}}= p!(q-1)! A(\tilde{K}\varphi,\varphi).
$$

Furthermore a direct computation, starting from (\ref{eq3.6}), shows that
$$ 
A(\tilde{\partial} \varphi, \tilde{\partial} \varphi) = A (
\triangledown '' \varphi , \triangledown '' \varphi) -
\frac{1}{p!(q-1)!}h_{\bar{b}a} \triangledown_{\bar{\gamma}}
\varphi^{a}_{A} \beta_{\bar{B'}}.\overline{\triangledown_{\bar{\beta}}
\varphi^{b\bar{A}\gamma B'}}.
$$

Hence we have 
\begin{align*}
\triangledown_{\beta}\xi^{\beta} &= h_{\bar{b}a}
\triangledown_{\bar{\gamma}} \triangledown_{\beta}
\varphi^{a}_{A}\beta_{\bar{B'}}. \overline{\varphi^{b\bar{A}\gamma
    B'}} + p!(q-1)!\\ 
&\qquad\left\{A(\tilde{K} \varphi,\varphi) + A(\triangledown''
\varphi,  \triangledown'' \varphi)- A (\tilde{\partial}\varphi,
\tilde{\partial}\varphi)\right\}.
\end{align*}

Now
\begin{align*} 
  \text{div}~ \eta & = \displaystyle \sum_{i=1}^{2n}D_{i} \eta^{i} = \sum
  \triangledown_ {\bar{\gamma}} \eta^{\bar{\gamma}} - 2
  \sum_{\gamma,\beta} \overline{S^{\beta}_{\gamma \beta}}\eta^{\bar{\gamma}},\\
  \triangledown_{\bar{\partial}}\eta^{\bar{\gamma}} &=
  h_{\bar{b}a} \triangledown_{\bar{\gamma}} \triangledown_{\beta}
  \varphi^{a}_{A} \beta_{\bar{B'}}.\overline{\varphi^{b\bar{A}\gamma 
  B'}} + h_{\bar{b}a} \triangledown_{\beta} \varphi^{a}_{A}
  \beta_{\bar{B'}}. \overline{\triangledown_{\gamma} \varphi^{b\bar{A}\gamma
    B'}}\\
& =h_{\bar{b}a} \triangle_{\bar{\gamma}} \triangle_{\beta}
  \varphi^{a}_{A} \beta_{\bar{B'}}.\overline{\varphi^{b\bar{A}\gamma
    B'}}+ p!(q-1)!A(\tilde{\vartheta} \varphi,\tilde{\vartheta}
  \varphi),\qquad \text{by (\ref{eq3.10}).}
\end{align*}


Thus\pageoriginale
\begin{align*}
  \text{div}~ \xi - \text{div}~ \eta &= p! (q-1) ! \left\{
  A(\overset{\thicksim} {K}  \varphi,\varphi) + A (\triangledown''
  \varphi,  \triangledown'' \varphi)\right.\\  
  &\left.-A (\overset{\thicksim}
  \partial \varphi, \overset{\thicksim} \partial \varphi) - A
  (\overset{\thicksim} \vartheta \varphi, \overset{\thicksim}
  \vartheta  \varphi)\right\}
  -2 \left(S_\alpha \beta_\beta \xi^\alpha -
  \overline{S^\beta_{\gamma\beta}}\eta\overline{^\gamma}\right). 
\end{align*}
Let $\varphi \epsilon \mathscr{D}^{pq} (X,E)$. Then by Stokes' theorem 
we have  
\begin{equation*}
  \|\triangledown'' \varphi\|^2 +
  (\overset{\thicksim}{K}\varphi,\varphi)  =  \|\overset{\thicksim}
  \partial \varphi\|^2 + \|\overset{\thicksim} \vartheta \varphi\|^2
  + \frac{2}{p!(q-1)!} \int_X \left(S_{\alpha \beta}^{\beta} \xi^\alpha -
  \overline{S^\beta  _{\gamma\beta}}\eta\overline{^\gamma}\right)dX.
  \tag{3.16}\label{eq3.16} 
\end{equation*}

Let $\mid  S \mid$  be the length of the torsion form. Applying lemma
\ref{chap1:alphlemmaA} of Chapter \ref{chap1} and the Schwartz
inequality we get from (\ref{eq3.4}), (\ref{eq3.5}) 
and from (\ref{eq3.8}), (\ref{eq3.9}) the following estimates 
\begin{align*}
  \|\overset{\thicksim}\partial \varphi\|^2 & \leq
  2(\overline{\partial}\varphi\|^2 + \|S\varphi\|^2)\leq 2
  \|\overline{\partial}\varphi\|^2 + c\int_X \mid S \mid^2 A(\varphi,
  \varphi)dX\\ 
  & \qquad = 2\|\overline{\partial}\varphi\|^2 +
  + c(\mid S \mid^2 \varphi,\varphi),\\
  \|\overset{\thicksim}\vartheta\varphi\|^2 & \leq
  2(\|\vartheta \varphi\|^2 + \|T\varphi\|^2) \leq 2\| \vartheta \varphi\|^2 +
  c\int_X \mid S \mid^2 A(\varphi, \varphi)dX\\ 
  & \qquad = 2\|\vartheta \varphi \|^2 + c(\mid S \mid^2 \varphi,
  \varphi), 
\end{align*}
$c$ being a universal positive constant (depending only on the
dimension of $X$). 

Furthermore, again by the Schwartz inequality, we have\pageoriginale
{\fontsize{10}{12}\selectfont
\begin{align*}
  \frac{1}{p!(q-1)!} \mid\sum S_{\alpha} \beta_{\beta} \xi^\beta \mid
  & \leq \frac{1}{p!(q-1)!} \mid S \| \xi\mid \leq q\mid S \mid
  A(\triangledown''\varphi,\triangledown''\varphi)^{\frac{1}{2}}
  A(\varphi,\varphi)^{\frac{1}{2}}\\ 
  & \leq \frac{q\varepsilon}{2}\mid S \mid^2 A (\varphi,\varphi) +
  \frac{q}{2\varepsilon} A(\triangledown'' \varphi,\triangledown'' \varphi),\\ 
  \frac{1}{p!(q-1)!}\mid \sum \overline{S
    \beta}_{\gamma_\beta}\eta^{\bar{\gamma}}\mid & \leq
  \frac{1}{p!(q-1)!}\mid S \|\eta\mid \leq q\mid S \mid
  A(\overset{\thicksim}{\vartheta} \varphi,
  \overset{\thicksim}{\vartheta} \varphi)^{\frac{1}{2}} 
  A(\varphi,\varphi)^{\frac{1}{2}} \\ 
  & \leq q\frac{\varepsilon}{2}\mid S \mid^2 A (\varphi, \varphi) +
  \frac{q}{2\varepsilon} A (\overset{\thicksim}{\vartheta} \varphi,
  \overset{\thicksim}{\vartheta}\varphi),
\end{align*}}\relax
for any $\varepsilon > 0.$

Substituting these estimates in (\ref{eq3.16}) we obtain for any $\varepsilon >
0$  
\begin{multline*}
  \left(1-\frac{q}{\varepsilon}\right) \|\bigtriangledown'' \varphi\|^2 +
  \left(\overset{\thicksim}{K}\varphi - \left(2c+2q\varepsilon +
  \frac{qc}{\varepsilon}\right) \mid S \mid^2  \varphi,\varphi\right)\\
  \leq 2
  \|\overline{\partial}\varphi\|^2 + 
  +2 \left(1 +\frac{q}{\varepsilon}\right) \|\vartheta \varphi\|^2.
\end{multline*}
Setting, for instance $\varepsilon = 2q$ , and\\

\begin{equation*} 
  \mathcal{K}=\overset{\thicksim}{\mathcal{K}} - \left(\frac{5}{2} c +
  4n^2\right) \mid S 
  \mid^2. Id : C^{pq} (X,E)\rightarrow C^{pq}(X,E)  \tag{3.17}\label{eq3.17}
\end{equation*}
we obtain the following

\begin{prop}%% 3.1
  For any $\varphi \epsilon \mathscr{D}^{pq} (X,E) \; (q>0)$ we have 
  $$
  \frac{1}{2}\|\triangledown''\varphi\|^2 +(\mathcal{K}\varphi,\varphi)\leq
  3 \left(\|\overline{\partial}\varphi\|^2 + \|\vartheta\varphi\|^2\right).
  $$
  If the hermitian metric on X is a K\"ahler metric, the above proposition
  can be considerably sharpened. In fact, 
  in this\pageoriginale case, $S \equiv 0$, $\overline{\partial}=\overline{0}$,
  $\overset{\thicksim}{\vartheta}=\vartheta$. It follows from (\ref{eq3.16}) 
  that for every $\varphi \epsilon \mathscr{D}^{pq}(X,E) \; (q>0)$
  satisfies the identity 
  \begin{equation*}
    \|\triangledown'' \varphi\|^2 + (\mathcal{K}\varphi,\varphi)=
    \|\overline{\partial}\varphi\|^2 + \|\vartheta \varphi\|^2 \quad
    \text{(K\"ahler)}  \tag{3.18}\label{eq3.18}
  \end{equation*} 
  where we have set
  $\mathcal{K}=\overset{\thicksim}{\mathcal{K}}$. Also in the
  hermitian case the choice of the numerical constants can be
  improved.   
\end{prop}

As a corollary to proposition \ref{chap3:prop3.1} we have the following 

\begin{theorem}%% 3.1
  If there exists a positive constant $C_2$ such that 
  $$
  A(\mathcal{K} \varphi,\varphi)\geq C_2 A(\varphi,\varphi)
  $$
  for every $\varphi \epsilon C^{pq}(X,E) \; (q>0)$ and at each point of
  $X$, then $E$ is $W^{p,q}$-elliptic. 
\end{theorem}

\begin{remark*}
  Effect on $\mathcal{K}$ of a change of the hermitian metric on $E$.
\end{remark*}

  Let $\varphi : X \mathbb{R}$ be any $C^{\infty}$- function. We
  wish to examine how the operator $\mathcal{K}$ above changes when we
  replace the hermitian metric $h$ on $E$ by $h'=e^\varphi h$. (The
  hermitian metric on the base $X$ remains undisturbed). First, the
  curvature form $s'$ of $h'$ is given by  
  $$
  s' = s + \sum\limits_{\alpha,\beta}\frac{\partial^2
    \varphi}{\partial \overline{z}^\alpha \partial
    z^\beta}\overline{dz^\alpha}\wedge dz^\beta. I_m , 
  $$
  where $s$, as above, is the curvature form of $h; (z^\alpha)_{1\leq
    \alpha \leq n}$ is a local coordinate system in an open set 
  $U\subset X$ on which a local trivialization of $E$ is assumed given;
  the symbol $I_m$ stands for the $(m \times m)$ identity matrix. It is
  immediate from (\ref{eq3.13}) that  
  $$
  (\mathcal{K}'\varphi)^a_{A\overline{B}}=( \mathcal{K}\varphi^a)_{A\overline{B}}+ 
  \displaystyle\sum_i(-1)^i \frac{\partial^2 \varphi}{\partial z^\alpha
    \partial \overline{z}^{\beta_i}}  \varphi^{a\alpha}_{A\overline{B'_i}}.
  $$
  where\pageoriginale $\mathcal{K}'$ is the analogue of $\mathcal{K}$,
  defined now with respect to the new hermitian metric $h'$.  
