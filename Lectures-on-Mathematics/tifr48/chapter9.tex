
\chapter{The Main Conjectures and the Main Theorem}\label{chap9}

Let\pageoriginale $G$ be a real analytic semi-simple group with no
center and no compact factors, and let $K$ be a maximal compact
subgroup. Let $X= G/K$ and let $\Gamma$, $\Gamma'$ be two discrete
subgroups of $G$, isomorphic under an isomorphism $\theta: \Gamma \to
\Gamma'$. We assume that $G/\Gamma$, $G/\Gamma'$ have finite Haar
measure. Let $\varphi:X \to X$ be a homeomorphism such that $\varphi
(\gamma x)= \theta (\gamma) \varphi (x)$ to $\forall \gamma \in \Gamma$
and $x \in X$. Then

\begin{conj}\label{chap9:conj1}
  $\theta$ extends to an analytic automorphism of $G$ provided $G$
  contains no factor locally isomorphic to $SL (2, \mathbb{R})$.
\end{conj}

\begin{conj}\label{chap9:conj2}
  Let $X_\circ$ be the unique compact $G$-orbit of a Satake
  compactification of $X$. Then $\varphi$ extends to a homeomorphism
  of $X \cup X_\circ$. Let $\varphi_\circ$ be the restriction to
  $X_\circ$ of the extension, then $\varphi_\circ G
  \varphi_\circ^{-1}=G$ as transformation of $X_\circ$, provided $G$
  has no factor locally isomorphic to $SL (2, \mathbb{R})$. 
\end{conj}

It is not difficult to see that Conjecture \ref{chap9:conj2} implies
Conjecture \ref{chap9:conj1}. Indeed we remark first that $G$ operates
faithfully on $X_\circ$, since $G$ has no compact factors and no
center. Since $X$ is topologically dense in $X \cup X_\circ$ we have
$\varphi_\circ (\gamma x)= \theta (\gamma) \varphi_\circ (x)$ for all
$x \in X_\circ$ and all $\gamma \in \Gamma$; that is, $\theta
(\gamma)= \varphi_\circ \gamma \varphi_\circ^{-1}$ as transformations
of $X_\circ$. If $\varphi_\circ G \varphi_\circ^{-1}=G$, then $g
\mapsto \varphi_\circ g \varphi_\circ^{-1}$ is a continuous
automorphism of $G$ with respect to the compact open topology of $G$
as a transformation group of $X_\circ$. As is well-known; this implies
that $g \mapsto \varphi_\circ g\varphi_\circ^{-1}$ is a continuous
automorphism of the analytic group $G$ and hence an analytic
automorphism. 

The\pageoriginale following example shows that $SL (2, \mathbb{R})/\pm
1$ violates the conjecture.

\begin{example}\label{chap9:exp9.1}
  Let $G= SL (2, \mathbb{R})/\pm 1$, $K= SO(2, \mathbb{R})/\pm
  1$. Then $X$ is the upper half plane with $G$ operating as linear
  fraction transformations $z \to \frac{az+b}{cz+d}$. Alternatively,
  we may identify $X$ with the interior of the unit ball in the
  plane. 
\end{example}

Let $S$ and $S'$ be two compact Riemann surfaces of genus $> 1$ which
are diffeomorphic but not conformally equivalent. Let $\Gamma =
\pi_1(S)$ and $\Gamma' = \pi_1 (S')$ be the fundamental groups of $S$
and $S'$. Let $\psi: S \to S'$ be a diffeomorphism, let $\theta:
\Gamma \to \Gamma'$ be the induced isomorphism of fundamental groups,
and let $\varphi : X \to X$ be the lift of $\psi$ to the simply
connected covering spaces of $S$ and $S'$; by uniformization theory,
the latter may be identified with $X$. Then $\varphi (\gamma x)=
\theta (\gamma) \varphi (x)$ for all $\gamma \in \Gamma$, $x \in
X$. As transformation groups on $X$ we can therefore write $\Gamma'=
\varphi \gamma \varphi^{-1}$. However $G \neq \varphi G \varphi^{-1}$
unless $\varphi$ is a Mobius transformation of $X$.

Pursuing the example further, the map $\varphi$ is a so-called
quasiconformal map (cf. next sections for definitions and properties)
and therefore induces a homeomorphism $\varphi_\circ$ of the boundary
$X_\circ$ of the unit ball. Then $\varphi_\circ \Gamma'
\varphi_\circ^{-1}= \Gamma'$ as transformations of $X_\circ$ since $X$
is dense in $X \cup X_\circ$. However $G \neq \varphi_\circ G
\varphi_\circ^{-1}$ unless $\varphi_\circ$ is a Moebius transformation
of the circle $X_\circ$.

The\pageoriginale following trivial example serves to illustrate that
once $\theta$ is given $\varphi_\circ$ is uniquely determined by
contrast with $\varphi$ which is not unique; and that $\varphi G
\varphi^{-1}=G$ is not necessary even when $\varphi_\circ G
\varphi_\circ^{-1}=G$.

\begin{example} \label{chap9:exp9.2}
  Let $\Gamma = \Gamma'$, $\theta=$ Identity, $\psi$ a homeomorphism
  which is the identity map except on some small neighbourhood of
  $X/\Gamma$. Then $\varphi G \varphi^{-1}\neq G$ since otherwise
  $\varphi$ would have to be the identity map. However $\varphi_\circ$
  is the identity map and in particular $\varphi_\circ G
  \varphi_\circ^{-1}=G$.
\end{example}

In these lectures we prove a slightly modified form of conjecture
\ref{chap9:conj2} for the group $G = 0(1, n)/\pm 1$ where $n >
2$. More precisely 

\begin{thm} \label{chap9:thm9.3}
  Let $G= 0(1, n)/\pm 1$, $n> 2$, and let $X$ be the associated
  Riemannian space. Let $\Gamma$, $\Gamma'$ be discrete subgroups such
  that $G/\Gamma$ and $G/\Gamma'$ have finite Haar measure. Let
  $\varphi: X \to X$ be a homeomorphism and $\theta: \Gamma \to
  \Gamma'$ an isomorphism such that $\varphi (\gamma x)= \theta
  (\gamma) \varphi (x)$ for all $\gamma \in \Gamma$, $x \in X$. Assume
  that $\varphi$ is quasi-conformal (cf. below for definition) then
  $\varphi$ induces a diffeomorphism $\varphi_\circ$ of the boundary
  component $X_\circ$ of the Satake compactification of $X$ and
  moreover $\varphi_\circ G \varphi_\circ^{-1}=G$. 
\end{thm}

\begin{note}
  The condition that $\varphi$ be quasi-conformal is automatically
  fulfilled if $G/\Gamma$ and $G/\Gamma'$ are compact and $\varphi$ is
  diffeomorphism. 
\end{note}

The proof of this theorem is based on the theory of quasi conformal
mappings cf. \cite{17}. In the following section we present a summary
of our proof.
