\chapter*{Introduction}

\addcontentsline{toc}{chapter}{Introduction}

These lectures are devoted to the proof of two theorems (Theorem 8.1,
the first main theorem and Theorem 9.3). Taken together these theorems
provide evidence for the following conjecture:

Let $Y$ and $Y'$ be complete locally symmetric Riemannian spaces of
non-positive curvature having finite volume and having no direct
factors of dimensions 1 or 2. If $Y$ and $Y'$ are homeomorphic, then
$Y$ and $Y'$ are isometric upto a constant factor (i.e., after
changing the metric on $Y$ by a constant).

The proof of the first main theorem is largely algebraic in nature,
relying on a detailed study of the restricted root system of an
algebraic group defined over the field $\mathbb{R}$ of real
numbers. The proof of our second main theorem is largely analytic in
nature, relying on the theory of quasi-conformal mappings in
$n$-dimensions.

The second main theorem verifies the conjecture above in case $Y$ and
$Y'$ have constant negative curvature under a rather weak
supplementary hypothesis.

The central idea in our method is to study the induced homeomorphism
$\varphi$ of $X$, the simply covering space of $Y$ and in particular
to investigate the action of $\varphi$ at infinity. More precisely
our method hinges on the question: Does $\varphi$ induce a smooth
mapping $\varphi_\circ$ of the (unique) compact orbit $X_\circ$ in a
Frustenberg-Stake compactification of the symmetric Riemannian space
$X$?

There are good reasons to conjecture that not only is $\varphi_\circ$
smooth, but that $\varphi_\circ G_\circ \varphi_\circ^{-1}= G_\circ$
where $G_\circ$ denotes the group of transformations of $X_\circ$
induced by $G$, provided of course that $X$ has no one or two
dimensional factors. The boundary behaviour of $\varphi$ thus merits
further investigation.

It is a pleasure to acknowledge my gratitude to Mr. Gopal Prasad who
wrote up this account of my lectures.

\bigskip
~\hfill {\large\bf G.D. Mostow}
