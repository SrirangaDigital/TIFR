
\chapter{$\mathbb{R}$-regular elements}\label{chap3}

When\pageoriginale $k= \mathbb{R}$ and $G$ is a semi-simple algebraic
$\mathbb{R}$-group we can give another description of
\textit{reductive} $\mathbb{R}$-regular elements.

Let $G$ be a semi-simple $\mathbb{R}$-group without loss of generality
we can (and we will) assume that $G$ is self adjoint
(cf. \cite{13}). Let $T$ be a maximal $\mathbb{R}$-split torus in
$G$. Let $A= (T_\mathbb{R})^\circ$.

We can assume that $A\subset P(n)$ [see Lemma \ref{chap1:lem1.4}]. Let
$\triangle$ be a fundamental system of restricted roots on $T$, let
$$
\displaylines{A^t = \{ x \big| x \in A \quad \alpha (x) > t \quad
  \forall \alpha \in \triangle \}\cr
  \text{then} \hfill A^1 = A_\triangle. \hfill \cr
  Z(T)_\mathbb{R} = Z(A)_\mathbb{R} =A. (Z(A) \cap 0 (n, \mathbb{R}))\cr
  \text{we put} \hfill Z(A) \cap 0 (n, \mathbb{R})=L. \hfill }
$$

Then $L$ is the unique maximum compact-subgroup of
$Z(A)_\mathbb{R}$. The only $\mathbb{R}$-regular elements in $A$ are
those in (Norm $A$) [$A_\triangle$]. More generally the
$\mathbb{R}$-regular elements in $Z(T)_\mathbb{R}$ are of the form m.a
with $m \in L$ and $a \in (\text{Norm}~ A) [A_\triangle]$. For given
such an element it lies in $P(\triangle')$ for any $\triangle' \subset
\triangle$. Moreover if $P$ is parabolic and $m. a\in P$ then $Z(m.a)
\subset P$
$$
\therefore\quad T \subset P \& P = P (\triangle') ~\text{for some}~
\triangle' \subset \triangle
$$

This implies that $m.a$ is $\mathbb{R}$-regular.

Since\pageoriginale all the max $\mathbb{R}$-split tori are conjugate by an element
from $G_\mathbb{R}$ it follows that the set of reductive
$\mathbb{R}$-regular elements in $G$ is
$G_\mathbb{R}[L. A_\triangle]$.

\begin{lemma}[Polar decomposition] \label{chap3:lem3.1}
  If $x$ is a reductive element of $GL\break (n, \mathbb{R})$ then $x$ can
  be written uniquely in the form $x=p.k$ with $p, k \in GL (n,
  \mathbb{R})$, the eigenvalues of $p$ are positive, the eigenvalues
  of $k$ are of absolute value 1 and $pk = kp$.
\end{lemma}

\begin{proof}
  Let $V$ be the underlying complex vector space.
\end{proof}

$\displaystyle{V = \oplus \sum_{\lambda} V_\lambda}$, whee $\lambda$
  varies over the eigenvalues of $x$.

Let 
\begin{align*}
  & p: v \mapsto |\lambda |V \quad \text{for}~ v \in V_\lambda\hspace{4cm}\\
\text{and} \hspace{2cm} & k: v \mapsto \frac{\lambda}{|\lambda|} v.
~\text{for}~ v \in V_\lambda. 
\end{align*}
Then $p$, $k$ satisfy the requirements of the lemma.

\begin{defi*}
  $p$ is called the \textit{polar part} of $x$.
\end{defi*}

The polar decomposition provides the following characterization of
$\mathbb{R}$-regular elements.

\begin{prop*} 
  $A$ reductive element is $\mathbb{R}-$regular iff its polar part is
  $\mathbb{R}$- regular. 
\end{prop*}

The rest of this section will be devoted to the proof of the 
\begin{thm} \label{chap3:thm3.2}
  Let $G$ be a semi-simple $\mathbb{R}-$group and $y$ be an
  $\mathbb{R}$-regular reductive element in $G_\mathbb{R}$. Then there
  is an algebraic subset $S_y$, not containing 1, such that for all
  large $n$, $xy^n$ is $\mathbb{R}$-regular, provided $x \in
  G_\mathbb{R}- S_y$.
\end{thm}

We\pageoriginale introduce the following new notations:

\begin{tabbing}
  $N^+$ \= = The unipotent analytic subgroup with Lie algebra
  $\displaystyle{\sum_{\alpha > 0} \dot{G}_\alpha}$\\
  $N^-$ \> = The unipotent analytic subgroup with Lie algebra
  $\displaystyle{\sum_{\alpha < 0} \dot{G}_\alpha}$\\
  $L_\mathbb{C}$ \> = The Zariski closure of $L$= maximal
    $\mathbb{R}$-compact subgroup of $Z(T)$.\\
    $F$ \> $ = N^-_{\mathbb{R}}\cdot N^+_{\mathbb{R}}$ 
\end{tabbing}
we have Bruhat's decomposition
\begin{align*}
  G & = N^- (\text{Norm}~ T) N^+\\
  G_\mathbb{R} & = N^-_\mathbb{R} N^+_{\mathbb{R}}.
\end{align*}
we need following Lemmas.

\begin{lemma} \label{chap3:lem3.3}
  Let $V$ be a finite dimensional vector space and let $v_i \in V$ and
  $d_i \in GL (V) i = 1, 2, \ldots$
  
  Assume 
  \begin{enumerate}[\rm (i)]
    \item $\lim\limits_{i \to \infty} v_i = v = \lim\limits_{i \to
      \infty} d_i v_i$ and
      \item $(d_i-1)^{-1}$ are bounded uniformly in $i$ then $v=0$.
  \end{enumerate}
\end{lemma}

\begin{proof}
  Set $w_i = (d_i -1)v_i$ then $w_i \to 0$ as $i \to \infty$
  $$
  \therefore \quad v_i = (d_i-1)^{-1} w_i \to 0 \quad \text{i.e.,} v=0.
  $$
\end{proof}

\begin{lemma} \label{chap3:lem3.4}
  Let $K$ be a compact subset of $G_\mathbb{R}$, let $W$ and
  $U_A$ be neighbourhoods of 1 in $G_\mathbb{R}$ and $A$
  respectively. Let $t > 1$, then there\pageoriginale is a nbd. 
  $U$ of 1 in $G$ such   that
  $$
  (kW) [La U_A] \supset k [La]. U ~\forall k \in K, a \in A^t.
  $$
\end{lemma}

\begin{proof}
  Since the rank of the map $(g, b) \to g(b)$ of $(W \cap F) \times
  LA^1$ into $G_\mathbb{R}$ at $(1, b)$ equals the dimension of
  $[\dot{G}_\mathbb{R}, \dot{L}+ \dot{A}] + \dot{L} + \dot{A}=
  \dot{G}$ the map is open in a nbd. of $(1, b)$. By taking a open
  subset of $U_A$ we can assume that $\forall a' \in U_A$ $t^{-1} <
  \alpha (a') < t$ $\forall \alpha \in \triangle$ and $\overline{U}_A$
  is compact. Then $\forall a \in A^t$ $a U_A \subset A^1$. If
  necessary, by passing to a open subset, we can assume that the above
  map has maximal rank on $(W \cap F) \times LA^1$, $\overline{W}$ is
  compact and $\overline{W} \cap ~\text{Norm}~ A \subset
  Z(A)_\circ$. Then the set $kW [La U_A]$ is a nbd. of identity. It
  remains only to show that
  $$
  \bigcap_{\substack{a \in A^t, k \in K\\m \in L}} (k [ma])^{-1} (kW)
         [La U_A] ~\text{is a nbd. of identity}.
  $$
\end{proof}

Since for any nbd. $U$ of 1, $\displaystyle{\cap_{k \in K}} k [U]$, for
$K$ compact, is a nbd. of 1, it is sufficient to show that
\begin{equation*}
  \bigcap_{a \in A^t, m \in L} (ma)^{-1} W[La U_A] ~\text{is a nbd. of
    1}. \tag{*}\label{chap3:eq*} 
\end{equation*}

Let 
$$
\displaylines{\pi: G_\mathbb{R} \to G_{\mathbb{R}/A}\cr
  \text{set} \hfill \widetilde{W} = \pi (W) \hfill \cr
  \text{define} \hfill f_{m, a}: \widetilde{W} \times U_A \times L \to
  G \hfill \cr
  \hfill f_{m, a}: (W A, a', m') \mapsto (ma)^{-1} (w (m' a a')) \hfill }
$$
then\pageoriginale (\ref{chap3:eq*}) is equivalent to 
\begin{equation*}
  \bigcap_{a \in A^t, m \in L} ~\text{Image}~ f_{m, a} ~\text{is a
    nbd. of 1}.\tag{**}\label{chap3:eq**}
\end{equation*}

It is easy to see that the condition (\ref{chap3:eq**}) fails iff
there is a sequence of points $x_i \in W \times U_A \times L$ and a
sequence $(m_i, a_i) \in L \times A^t$ such that $x_i \mapsto$
boundary of $=\widetilde{W} \times U_A \times L$ in $G_{\mathbb{R}/A}
\times A \times L$ and $\lim\limits_{i \to \infty} f_{m_i, a_i}
(x_i)=1$. Hence to prove (\ref{chap3:eq**}) it suffices to show that
if $\lim\limits_{i \to \infty} f_{m_i, a_i} (w_i A, a_i', m_i')=1$
with $a'_i \in U_A, m_i' \in L$ and if $\lim (w_i, a_i', m_i, m_i') =
(w, a', m, m')$ then $w=1$ and $a'=1$. For then it will follow that
$$
(w_i A, a_i', m_i') \to (A, 1, m)
$$
which is not a boundary point of $\widetilde{W} \times U_A \times L$.

The previous statement is equivalent to 
\begin{equation*}
  \begin{cases}
    \text{If}~  (m_i a_i)^{-1} (w_i [m_i' a_i a_i']) \mapsto 1
    ~\text{and if}\\
     (w_i, a_i', m_i, m_1') \to (w, a', m, m') \in (\overline{W} \cap
    F) \times \overline{U}_A \times L \times L\\
    \text{then}~  w=1 ~\text{and}~ a'=1.
  \end{cases}\tag{***}\label{chap3:eq***}
\end{equation*}

We prove (\ref{chap3:eq***}) 

From the uniqueness of Bruhat's decomposition it follows that
$N^-_{\mathbb{R}}\break LA N^+_{\mathbb{R}}$, being the image of
$N^-_{\mathbb{R}} \times LA \times N^+_{\mathbb{R}}$ under a
homeomorphism, is open (invariance of domain).

Let\pageoriginale $b \in A$ be close enough to the identity 1, so that 
$$
w[b] \in N^-_{\mathbb{R}} LA N^+_{\mathbb{R}} \quad \forall w \in \overline{W}.
$$

Then 
$$
\displaylines{ w_i [b] = p_i c_iq_i \quad p_i \in N^-_{\mathbb{R}},
  c_i \in LA, q_i \in N^+_{\mathbb{R}}\cr
  \text{and} \hfill w[b] = pcq \quad p \in N^-_{\mathbb{R}}~ c \in LA,
  q \in N^+_{\mathbb{R}}. \hfill \cr
  \text{Set} \hfill b_i = m_i a_i \quad \text{then} ~ b_i^{-1}= v_i
  (w_i [(m_i' a_i a_i')^{-1}]) \hfill \cr
  \text{where}\hfill v_i = b_i^{-1} (w_i [m_i' a_i a_i'])\hfill}
$$
\begin{align*}
  \therefore \quad (b_i^{-1} w_i) [b] & = v_i (w_i [(m_i'a_i
    a_i')^{-1}]) w_i b w_i^{-1} (w_i[m_i' a_i a_i']) v_i^{-1}.\\
  & = v_i w_i [b].
\end{align*}

Since $v_i \to 1$ and $w_i \to w$ we have
$$
\displaylines{ \lim\limits_{i \to \infty} (b_i^{-1} w_i) [b] = w[b]\cr
  \text{i.e.,} \hfill \lim\limits_{i \to \infty} b_i^{-1} w_i [b]=
  \lim\limits_{i \to \infty} w_i [b]\hfill\cr
  \text{but} \hfill b_i^{-1} [w_i [b]] = b_i^{-1} [p_i] \cdot c_i
  \cdot b_i^{-1} [q_i] \hfill \cr}
$$
so
\begin{align*}
  \lim b_i^{-1} [w_i [b]] & = \lim b_i^{-1} [p_i] \cdot c_i \cdot
  b_i^{-1} [q_i]\\
  & = \lim p_i \cdot c_i \cdot q_i
\end{align*}
$$
\displaylines{\therefore \qquad \lim b_i^{-1} [p_i] = p = \lim p_i\cr
  \text{and} \hfill \lim b_i^{-1} [q_i] = q = \lim q_i.\hfill }
$$

Since\pageoriginale $N^+$, $N^-$ are nilpotent. By induction on the lengths of the
descending central series of $N^-$, $N^+$ and using the previous lemma
we get
\begin{align*}
  \lim p_i & = p=1 = q = \lim q_i.\\
  \therefore \qquad w[b] & = c \in LA
\end{align*}
since $A$ is connected any nbd. of 1 in $A$ generates $A$ we have 
$$
\displaylines{w [A] \subset LA\cr
  \therefore \hfill w [A] = A \hfill \cr
  w \in \overline{W} \cap ~\text{Norm}~A \subset Z(A) \cr
  \therefore \hfill w \in \overline{W} \cap LA \cap F. \hfill }
$$

But from Bruhat's decomposition $\overline{W} \cap LA \cap F = \{
1\}$.

\noindent $\therefore$ \qquad  $w=1$.

From (\ref{chap3:eq***})
$$
b_i^{-1} (w_i [m_i' a_i a_i']) \to 1.
$$

Since $w_i \to 1$ we have 
$$
\displaylines{b_i^{-1} m_1' a_i a_i' \to 1\cr
  \therefore \hfill \substack{\underbrace{m_i^{-1} m_i'}\\\in L} ~~
  a_i' \to 1\hfill \cr
  \therefore \hfill a_i' \to 1 \qquad \text{since}~ L \cap A = \{ 1\}.
  \hfill \cr
  \text{i.e.,} \hfill a' =1.\hfill}
$$
This\pageoriginale proves the Lemma.

\begin{lemma} \label{chap3:lem3.5}
  Let $C$ be a compact subset of $N^-_{\mathbb{R}}$ and let $t >
  1$. Then there exists a compact subset $K \subset N^-_{\mathbb{R}}$
  such that
  $$
  Cb \subset K[b] \quad b \in LA^t. 
  $$
\end{lemma}

\begin{proof}
  (By induction on the length of the derived series of
  $N^-_{\mathbb{R}}$).

  Set 
  $$
  N_\circ = N^-_{\mathbb{R}} \quad N_{i+1}= [N_i, N_i]
  $$
  Suppose $N_\circ$ is abelian.
\end{proof}

Then 
\begin{align*}
  ub & = v[b] ~\text{iff}~ ub = vbv^{-1} b^{-1} b\\
  \text{i.e.,} \quad u & = vb~ v^{-1} ~b^{-1}\\
  & = v- ad ~b~v \quad (\text{written in additive from})\\
  & = (1- adb)v.
\end{align*}

Iff\pageoriginale $v= (1- adb)^{-1} v$ where $adb(x) = bxb^{-1}$.

Since $(1- adb)^{-1}$ is uniformly bounded for 
$$b \in LA^t,\quad {\bigcup_{b \in LA^t}(1- adb)^{-1}C}$$ 
is a subset of a
compact set $K$. This proves the statement for the case when $N_\circ$
is abelian. 

In general, given a fixed $b \in LA^t$, by applying the above argument
to $N_\circ/{N_1}$, we can find elements $v \in N_\circ$ and $n \in
N_1$ such that $nub = v[b]$; moreover since $N_\circ=
N^-_{\mathbb{R}}$, as $u$ varies over compact set $C$, $n$ and $v$
vary over compact sets.

$ub = n^{-1}v [b]= v[n_1 b]$ where $n_1 \in N_1$ and varies over a
compact set $K_1$.

By induction $n_1 b= v_1 [b]$ and $v_1$ varies over a compact set as
$n_1$ varies over compact set $K_1$ and $b$ over $LA^t$.

We have 
$$
ub =v [n_1b]= v [v_1 [b]]= v v_1 [b]
$$
as both $v$, $v_1$ vary over compact sets $v \cdot v_1$ varies over a
compact set proving the Lemma.

Now we prove Theorem \ref{chap3:thm3.2}. 

We can assume (see pp. 36-37) that $y \in LA^t$, $t > 1$. Let $S_y =G-
N^- Z(A)N^+$. Then since $S_y \Big| Z(A) N^+$ is union of $N^-$ orbits
of lower dimensions in $GZ(A)N$, $S_y$ is Zariski closed in $G$. Let
$$
x \in G_{\mathbb{R}}- S_y = N^-_{\mathbb{R}} Z(Z)_{\mathbb{R}} N^+_{\mathbb{R}}
$$
then\pageoriginale
\begin{align*}
  x & = u^- b u^+ ~\text{with}~ b \in LA, u^- \in N^-_{\mathbb{R}}
  \quad \& u^+ \in N^+_{\mathbb{R}}\\
  xy^n & = u^- b u^+ y^n = (u^- b y^n) (y^{-n} u^+ y^n)\\
  & = v[by^n](y^{-n} u^+ y^n) ~\text{for some $v \in K$ (by Lemma
    \ref{chap3:lem3.5})}.
\end{align*}

Since $y$ is $\mathbb{R}$-regular reductive element, given a nbd. $U$
of 1, $\exists n_\circ (U)$ such that $(y^{-n} u^+ y^n)\in U$ for $n >
n_\circ (U)$. Hence by Lemma \ref{chap3:lem3.4}, $\exists n_\circ (y,
x)$ such that $xy^n \in G [LA^t]$ if $n> n_\circ (y, x)$.

\begin{lemma}\label{chap3:lem3.6}
  $S_y$ contains no conjugacy class of $G$. 
\end{lemma}

\begin{proof}
  Suppose $E \subset S_y$ with $G[E]=E$. Since $S_y$ is Zariski closed
  we can also assume that $E= E^*$. Let $0 = G -E$, then for $g \in
  N^+$
  \begin{align*}
    g[N^- Z(A)N^+] & = g [G- S_y] \subset g [G-E] \subset G- E =0\\
    \therefore \qquad N^+ [N^- Z(A)N^+] & = N^+ N^- Z(A)N^+ \subset 0.
  \end{align*}
\end{proof}

But 
\begin{align*}
  N^+ N^- Z(A) N^+ & = N^+ N^- Z(A) Z(A) N^+ = N^+ Z(A) N^- Z(A) N^+\\
  & = \underline{N^+ Z(A) N^+}~ \underline{N^- Z(A)N^+} = J J^{-1}
  \subset 0\\
  & \qquad \text{where}~ J = N^+ Z(A)N^+
\end{align*}

Since $J$ is Zariski open in $G$, for any $g \in G$, $g J \cap J$,
being intersection of two Zariski open (hence dense) sets, is
nonempty.

Therefore
\begin{align*}
  g & \in J J^{-1} \subset 0.\\
  \therefore \quad 0 & = G\\
  \therefore \quad E & = G = 0 = \phi.
\end{align*}
This\pageoriginale proves the assertion.
