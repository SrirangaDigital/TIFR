
\chapter{The First Main Theorem} \label{chap8}

This\pageoriginale section is devoted to the proof of 

\begin{thm} \label{chap8:thm8.1}
  Let $G_*$ be a semi-simple analytic group with no compact factors
  and no center. $K$ be a maximal compact subgroup. Let $X=G/K$ and
  let $\Gamma, \Gamma'$ be two discrete subgroups of $G_*$, isomorphic
  under an isomorphism $\theta: \Gamma \to \Gamma'$. We assume that
  $G_*\big| \Gamma, G_* \big| \Gamma'$ have finite Haar measure. Let
  $X_\circ$ be the unique compact $G_*$ orbit in some
  Satake-compactification of $X$. Let $\varphi: X \to X$ be a
  homeomorphism such that (i) $\varphi (\gamma x)= \theta
  (\gamma)\varphi (x)~ \forall \gamma \in \Gamma$, $x \in X$: (ii)
  $\varphi$ extends to a homeomorphism of $X \cup X_\circ$ whose
  restriction to $X_\circ$ is a diffeomorphism of $X_\circ$, then
  $\theta$ extends to an automorphisms of $G_*$. 
\end{thm}

[Conjecture. Condition (ii) is superfluous if $G$ has no factors
  isomorphic to $PSL (2, \mathbb{R})$.]

For the proof of the theorem we need following lemmas.

\begin{lemma} \label{chap8:lem8.2}
  Let $G$ be a connected reductive linear algebraic group. Let $k^T$
  be a maximal $k$-split torus and $T$ be a maximal $k$-torus
  containing $k^T$. Let $t_1, t_2$ be elements in $k^T$ conjugate in
  $G$. Then $t_1, t_2$ are conjugate by an element in Norm $({}_K T)
  \cap$ Norm $T$. 
\end{lemma}

\begin{proof}
  From Bruhat's decomposition
  $$
  G= N^+ (\text{Norm}~ {}_k T) N^+.
  $$
  Suppose 
  $$
  \displaylines{
  x t_1 x^{-1} = t_2 ~\text{with}~ x \in G\cr
  \text{and} \hfill x = u~w~v \quad u, v \in N^+, w \in ~\text{Norm}~
  k^T \hfill }
  $$
  then\pageoriginale
  \begin{gather*}
    uwv t_1 = t_2 uwv\\
    \therefore \quad u \underline{w t_1} t_1^{-1} [v] = t_2 [u]
    \underline{t_2 w} v.
  \end{gather*}
\end{proof}

By the uniqueness of Bruhat's decomposition $wt_1 = t_2w$. Thus $t_1,
t_2$ are conjugate by $w$ (\text{Norm}~ $k^T$). Since $T$ and $w
Tw^{-1}$ are contained in $Z({}_k T)$, by the conjugacy of maximal tori,
$\exists \lambda \in Z({}_k T)$ such that $\lambda T \lambda^{-1}= w T
w^{-1}$ i.e., $\lambda^{-1} w T ew^{-1} \lambda =T$ i.e.,
$\lambda^{-1} w \in ~\text{Norm}~ (k^T) \cap ~\text{Norm}~T$. 

It is clear that $t_1$ and $t_2$ are conjugate by $\lambda^{-1} w$. 

\begin{lemma} \label{chap8:lem8.3}
  Let $G$ be the Zariski closure of a real linear algebraic group
  $G_*$, let ${}_\mathbb{R}T$ be a maximal $\mathbb{R}$-split torus in
  $G$ and $T$ be a maximal $\mathbb{R}$-torus containing
  ${}_\mathbb{R}T$. $W^A= \text{Norm}~ {}_\mathbb{R}T$. $P_*$ be the
  stabalizer in $G_*$ of a point in $X_\circ$, $P$ the Zariski
  closure of $P_*, U$ the unipotent radical of $P$. We assume that $P
  \supset T$. $\mathscr{R}_U$ the set of roots occurring in $U,
  \mathscr{R}_{N^+}$ the set of roots occurring in $N^+$, then
  $$
  W^A (\mathscr{R}_U)= \pm \mathscr{R}_{N^+}.
  $$
\end{lemma}

\begin{proof}
  From our description of Satake compactfication in \S\ \ref{chap2}, we know
  that $P = P (\triangle')$ for some $\triangle' \subset
  {}_\mathbb{R}\triangle$. Indeed in the notation of \S\ \ref{chap2}, $\triangle'
  = E_\rho$ where $\rho$ is an $\mathbb{R}$-irreducible representation
  with finite kernel, and thus $P(\triangle')$ contains no normal
  subgroup of positive dimension, equivalently, the subset
  $\triangle'$ contains no connected component of the fundamental
  system of restricted roots ${}_\mathbb{R}\triangle$.
\end{proof}

We\pageoriginale have $P (\triangle')= G (\triangle')$. $N(\triangle')$, $U =
N(\triangle')$ and $N^+= N (\phi)$. It is easy to see that if
${}_\mathbb{R} \triangle$ is connected, then $\mathscr{R}_U$ contains a
root whose restriction to ${}_\mathbb{R}T$ has length equal to the
length of any restricted root in ${}_\mathbb{R}\triangle$. We recall
that the Weyl group of a connected root system permutes transitively
all roots having the same length. Applying this observation to each
connected component of ${}_\mathbb{R} \triangle$, we find $W^A$
(restriction of $\mathbb{R}_U$ to ${}_\mathbb{R}T$)= all restricted
roots. Hence
$$
W^A (\mathbb{R}_U) = \pm \mathbb{R}_{U^+}.
$$

\begin{lemma} \label{chap8:lem8.4}
  Let $A= ({}_\mathbb{R} T {}_\mathbb{R})^\circ$ and let $b \in L. A =
  Z({}_\mathbb{R} T){}_\mathbb{R}$. Then $b$ is $\mathbb{R}$-regular
  iff $b$ keeps fixed exactly $m/m_\circ$ points [here $m= \sharp W^A$
  and $m_\circ$= order of the Weyl group of $G(\triangle')$] and on
  the tangent space at these points, the eigenvalues are different
  from 1 in absolute value.
\end{lemma}

\begin{proof}
  Let $U^-$ denote the opposite of $U$. Suppose $b$ is
  $\mathbb{R}$-regular then the eigenvalues of $b$ on tangent spaces
  at the points fixed under $b$ are the values of $W^A
  (\mathbb{R}_{U^-})$ on $b$. Conversely if $b \in LA$ has only
  finitely many fixed points on $X_\circ$, then $b$ is
  $\mathbb{R}$-regular. The Lemma is now clear.
\end{proof}

\begin{lemma} \label{chap8:lem8.5}
  Let $G_*$, $\Gamma$, $\Gamma'$ be as in the hypothesis of the
  Theorem 8.1. Let $\gamma$ be a reductive $\mathbb{R}$-regular
  element of $\Gamma$. Then $\theta (\gamma)$ is also reductive
  $\mathbb{R}$-regular. 
\end{lemma}

\begin{proof}
  Let\pageoriginale $p_\circ \in X_\circ$ and let $P_*$ be the stabilizer of
  $p_\circ$ in $G_*$. For $g \in P_*$ we denote by $\hat{g}$ the
  operation of $g$ on $\dot{G}_*/ \dot{P}_*$. An element $g \in P_*$
  is reductive $\mathbb{R}$-regular iff $Ad_N+g$ has eigenvalues
  different from 1 in absolute value; this will be true if $g$ keeps
  fixed $m/m_\circ$ points of $X_\circ = G_{*/P_*}$ and on each of the
  tangent spaces at the fixed points, takes eigenvalues $\neq 1$ in
  absolute value.
\end{proof}

Thus $g \in G$ is reductive $R$-regular iff it keeps fixed $m/m_\circ$
points in $X_\circ$ and on the tangent space each point has
eigenvalues $\neq 1$ in absolute value. From this it will follow that
if $\gamma$ is $\mathbb{R}$-regular then $\theta (\gamma)$ is also
$\mathbb{R}$-regular. 

\begin{remark*}
  If $G$ is a reductive algebraic group over any field $k$, then it
  follows immediately from definitions that an element of $G$ is
  $k$-regular iff it keeps only a finite number of points in $G/p$
  fixed for $\forall P = P (\triangle')$. $\triangle' \subset {}_k
  \triangle$. It can be proved that the element is reductive iff the
  number of fixed points is $\frac{\text{order of the Weyl group of
      $G$}}{\text{order of the Weyl group of P(i.e. of
      $C(\triangle')$)}}$; it is unipotent iff the number of fixed
  points is precisely 1.
\end{remark*}

\begin{lemma} \label{chap8:lem8.6}
  If $H= T^\circ_\mathbb{R}$, there exists an automorphism $\tau$ of
  $H$ and a Zariski dense subset $H_\tau$ of $\mathbb{R}$-regular
  elements in $H$ such that $\forall h \in H$, $h$ and $\tau(h)$
  operate equivalently on $X_\circ$, i.e., there exists a
  diffeomorphism $\Phi_\circ$ of $X_\circ$ such that $h=
  \Phi_\circ^{-1} \tau (h) \Phi_\circ$.
\end{lemma}

\begin{proof}
  Let $A^1 = \{ a; a \in A ~\alpha (a) >1 \forall \alpha \in
  {}_\mathbb{R} \triangle\}$. Let $K$ be a maximal compact subgroup of
  $G_*$. Recall $Z ({}_\mathbb{R} T){}_\mathbb{R}= L.A$. We can assume
  tht $K \supset L$. Let (1) be the projection of 1 in $X= G_*/K$.
\end{proof}

Let\pageoriginale $p_\circ= \lim\limits_{n \to \infty} a^n (1)$, $a \in A^1$ and let
$P_*$ be the stabilizer of $p_\circ$. 
$$
P= P (\triangle')= P (\Phi).
$$

Set $V=$ tangent space to $X_\circ$ at $p_\circ$, then $V \approx
\dot{G}_*/ \dot{P}_*$. Let $g \in P_*$ and let $\hat{g}$ denote the
operation of $g$ on $V$. If $H \subset P_*$, $\hat{H} \subset C$;
where $C$ is a Cartan subgroup of $GL (V)$.

Let $W= \frac{N(C)}{Z(C)}$ be the Weyl group of $C$. For any element
$\gamma \in \Gamma$ set $\gamma'= \theta (\gamma)$.

Given a reductive $\mathbb{R}$-regular element $\gamma$ of $\Gamma$,
there exists a $g \in G_*$ such that $g[\gamma]$ belongs to $H \cap
LA'$. The element $\theta (\gamma)$ is also reductive
$\mathbb{R}$-regular. Therefore $\exists g' \in G_*$ such that $g'
[\gamma'] \in H \cap LA^1$.

Since
\begin{align*}
  \varphi (\gamma p) & = \theta (\gamma) \varphi (p)m, ~\text{we can
    write}\\
  \theta (\gamma) & = \varphi \gamma \varphi^{-1} (=
  \varphi[\gamma])\\ 
  g' [\gamma'] & = g' [\varphi[\gamma]]= g' [\varphi[g^{-1} g
      [\gamma]]]\\ 
  \therefore \quad g' (\gamma') & = g' \varphi g^{-1} [g [\gamma]]\\
  g' \hat{[}\gamma'] & = \sigma^y g [\gamma\hat{]} (\sigma^y)^{-1}
\end{align*}
where $\sigma^\gamma$ is the differential of $g' \varphi g^{-1}$ at
$p_\circ$.

Therefore there is an element $\tau^\gamma$ in $W$, the Weyl group of
$C$ such that
$$
g'\widehat{[\gamma']}= \widehat{\tau \gamma (g [\gamma])}.
$$

For\pageoriginale any element $w \in W$, let $H_w$ denote the subset of $H \cap LA'
\cap G_* [\Gamma]$ on which the map $\gamma \to \tau^\gamma$ is
constant. Since $H \cap LA' \cap G_* [\Gamma]$ is Zariski dense in $H$
and $W$ is finite, there exists a $\tau \in W$ such that $H_\tau$ is
Zariski dense in $H$. Denoting Zariski closure by superscript $*$, we
can write
$$
H^*_\tau = H^*
$$
since 
$$
\displaylines{\tau (H_\tau) \subset H\cr
\therefore \quad \tau (H^*) = H^* ~\text{and therefore}~ \tau (H)= H.}
$$
Thus $\tau$ induces an automorphism of $H$, and by definition, $h$ and
$\tau(h)$ operate equivalently on $x_\circ$ for all $h \in H_\tau$.

\heading{Proof of the Theorem 8.1.}

Let $S_1= \displaystyle{\bigcup_{\substack{w \in W\\H_w ~\text{not
        Zariski dense}}}}H_w$ and let $S = S^*_1 \cup$ non
$\mathbb{R}$-regular elements in $H$.

Then clearly
$$
S^* \neq H^*.
$$

Let $\tau$ be an automorphism of $H$ given by the previous lemma. Then
$\tau$ permutes the roots $\Phi^*$, that is 
$$
\{ \alpha (h); h \in H_\tau, \alpha \in \Phi^*\} = \{ \alpha (\tau
(h)); \alpha \in \Phi^*, h \in H_\tau\}
$$

By\pageoriginale the lemma \ref{chap8:lem8.3} $\tau$ permutes $\Phi$. Hence $\Tr Ad
g [\gamma] = \Tr Ad (g' [\gamma'])$ that is, $\Tr Ad \gamma = \Tr Ad
\gamma' ~\forall g \in \Gamma \cap G_* [H_\tau]$. It follows that $\Tr
Ad \gamma = \Tr Ad \gamma'$ for all $\gamma \in \Gamma \cap G_* [H -
  S_1]$. 

Since $G$ is without center we can identify it with $Ad G$. 

Given $\gamma \in \Gamma$ and $S \subset H$ with $S^* \neq H^*$ and
$n$ any positive integer, by Lemma \ref{chap4:lem4.4}. $\exists
\gamma_\circ \in \Gamma \cap G[H-S]$ such that $\gamma_\circ,
\gamma^2_\circ, \ldots \gamma_\circ^n$, $\gamma \gamma_\circ$, $\gamma
\gamma_\circ^2, \ldots \gamma \gamma_\circ^n \in G [H-S]$. 

Let $n = \dim G$. Then $\Tr (\gamma \gamma_\circ^m)= \Tr \theta
(\gamma \gamma_\circ^m)= \Tr \theta (\gamma) \theta (\gamma_\circ)^m$
for $m=1, \ldots n$. We can write $1= c_1 \gamma_\circ + c_2
\gamma_\circ^2+ \cdots + c_n \gamma_\circ^n= f (\gamma_\circ)$ by
setting the characteristic polynomial of $\gamma_\circ$ equal to zero.

Then 
\begin{align*}
  \Tr \gamma = \Tr \gamma f(\gamma_\circ) & = \sum_{m=0}^n \Tr (c_m
  \gamma \gamma_\circ^n)\\
  & = \sum_{m=0}^n c_m \Tr \theta (\gamma) \theta (\gamma_\circ)^m\\
  & = \Tr \theta (\gamma) f (\theta (\gamma_\circ)).
\end{align*}

But $\Tr \gamma_\circ^m= \Tr \theta (\gamma_\circ)^m$ for $m=1, \ldots
, n$ and thus $\gamma_\circ$ and $\theta (\gamma_\circ)$ have the same
characteristic polynomial, by Newton's formulae.

Hence $f (\theta (\gamma_\circ))=1$ and $\Tr \gamma = \Tr \theta
(\gamma)$ for all $\gamma \in \Gamma$. 

Suppose $\displaystyle{\sum_{\gamma \in \Gamma}} \mathbb{C} \gamma
\gamma =0)$. Then $0= \Tr \left(\displaystyle{\sum_{\gamma \in
    \Gamma}}\mathbb{C}_\gamma \gamma \sum d_{\gamma^*} \gamma^* \right)$
$\forall d_{\gamma *} \in \mathbb{C}$ $\forall \gamma^* \in
\Gamma$. This will imply
$$
\Tr \left( \sum_{\gamma \in \Gamma} \mathbb{C}_\gamma \theta (\gamma)
\cdot \sum_{\gamma^* \in \Gamma} d_{\gamma *} \theta (\gamma^*)\right)=0
$$ 

Let\pageoriginale $\mathcal{E}$ denote the $\mathbb{C}$ linear span of
$\Gamma$. Clearly $\mathcal{E}$ is an associative matrix algebra. By
the density theorem (that the Zariski closure of $\Gamma$ is $G$), the
linear span of $\Gamma$ s linear span of $G_*$. Thus $\Tr
\displaystyle{\sum_{\gamma \in \Gamma} \mathbb{C}_\gamma \theta
  (\gamma) e= 0}$ for all $e \in \mathcal{E}$.

We can (and we will) assume that $Ad G$ is self adjoint. The we can
assert
$$
\Tr (\sum C_\gamma \theta (\gamma)) {}^t \sum C_\gamma \theta (\gamma)=0
$$

This implies that ${}^t \sum C_\gamma \theta (\gamma) =0$. Therefore
$\theta$ induces a linear isomorphism of $\mathcal{E}$ onto
$\mathcal{E}$, since $\sum C_\gamma \gamma =0$ implies $\sum C_\gamma
\theta (\gamma)=0$. Clearly $\theta$ is an $\mathbb{R}$-algebra
automorphism $\theta (\Gamma^*) \cap \mathcal{E}_{\mathbb{R}}= (\theta
(\Gamma))^* \cap \mathcal{E}_\mathbb{R}$ implies that $G_*=
G_\mathbb{R}^\circ = (G \cap \mathcal{E}_\mathbb{R})^\circ= \theta ((G
  \cap \mathcal{E}_\mathbb{R})^\circ) = \theta (G_*)$, since $\Gamma$
  and $\theta (\Gamma)$ are Zariski dense in $G$.

Thus we have proved that $\theta$ extends to an automorphism of
$G_*$. 
