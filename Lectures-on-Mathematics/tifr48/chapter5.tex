
\chapter{Some Ergodic Properties of Discrete Subgroups}\label{chap5}

\begin{lemma}[Mautner]\label{chap5:lem5.1}
  Given\pageoriginale a group $B \cdot A$, where $B$ is an additive group of reals
  or complex numbers and $A$ is an infinite cyclic subgroup of the
  multiplicative group of complex numbers a with $|a|<1$ and assume
  that

  \medskip
  \begin{tabular}{ll}
    & 0 is group operation is\\
    $a \circ b \circ a^{-1}= a. b$ & \\
    & ordinary multiplication in $\mathbb{C}$
  \end{tabular}
\end{lemma}

Let $V$ be a Hilbert space and let $\rho$ be a unitary representation
of $B \cdot A$ on $V$, then any element $v \in V$ whose line is fixed
under $A$ is fixed under $B$.

\begin{proof}
  Since $\rho$ is unitary
  \begin{align*}
    \rho (a) v & = \alpha v ~\text{with}~ |a|=1 ~\text{for}~ b \in B\\
    <\rho (b) v, v> & = <\rho(a) \rho (b) v, \rho (a) v>\\
    & = <\rho (a) \rho(b) \rho(a^{-1}) \rho (a) v, \rho (a) v>\\
    & = <\rho (a \circ b \circ a^{-1}) \alpha v, \alpha v> = <\rho (a
    \circ b \circ a^{-1}) v, v>
  \end{align*}
  So for $\forall n$ positive
  \begin{align*}
    <\rho (b)v, v> & = <\rho (a^n \circ b \circ a^{-n}) v, v>\\
    & = <\rho (a^n . b) v, v>\\
    & \quad \text{as}~ n \to \infty\\
    <\rho (b) v, v> & = <v, v>\\
    \therefore \quad \rho (b) v & = v. (\text{use Schwarz's inequality}).
  \end{align*}
  This\pageoriginale proves the assertion.
\end{proof}

\begin{lemma} \label{chap5:lem5.2}
  Let $G$ be an analytic semis-simple group having no compact
  factors. Let $\rho$ be a unitary representation of $G$ on a Hilbert
  space $V$, let $x$ be a reductive $\mathbb{R}$-regular element in
  $G$, if for some element $v \in V$, $\rho (x)v= \alpha v$ then $\rho
  (G)v=v$.
\end{lemma}

\begin{proof}
  Take the decomposition of $G$ with respect to $x$. Let $A$ be the
  group generated by $x$ and $B$ a root space. The previous Lemma applies.
\end{proof}

\begin{remark*}
  The above result holds for any $x$ not contained in a compact
  subgroup (see \cite{11}).
\end{remark*}

\begin{thm} \label{chap5:thm5.3}
  Let $x$ be a reductive $\mathbb{R}-$regular element of $G$. Then $x$
  operates ergodically on $G_* \big| \Gamma$, i.e. any measurable
  subset of $G_*\big| \Gamma$ stable under left translation by $x$ is
  either of measure zero or its complement has measure zero.
\end{thm}

\begin{proof}
  Let \quad $V= \mathscr{L}^2 (G_*/ \Gamma)$.
\end{proof}

Since the measure on $G_*/\Gamma$ is $G_*$-invariant, the canonical
action of $G_*$ on $V$ is unitary.

Let $Z\subset G_* /\Gamma$ with $x Z \subset Z$ and let $v$ be the
characteristic function of $Z$. Then since measure of $x^{-1}Z-Z$ is
zero
$$
x. v =v.
$$

Therefore\pageoriginale by the previous lemma
\begin{align*}
  G_* \cdot v & = v.\\
  \therefore \quad v & = 1 ~\text{almost every where}\\
  \text{or}\quad v & = 0 ~\text{almost every where}.
\end{align*}

This implies that either $Z$ or $G/\Gamma-Z$ has zero measure.

\begin{remark*}
  Let $M$ be a separable topological measure space [i.e. the open sets
  are measurable and have positive measure] and let $f:M \to M$ be a
  measurable transformation. Let $A^+ = \{ f^n, n =1,2, \ldots
  \}$. Then if $f$ is ergodic, for almost all $p \in M$, $A^+ p$ is
  dense in $M$.
\end{remark*}

\medskip
\noindent[\textit{Proof}: 
    Let $\{ U_i\} $ be a denumerable base of open sets. Let $\{ W_i =
    p|p \in M, A^+ p \cap U_i = \phi\}$ then $W_i$ is
    measurable. Also $p \in W_i \Rightarrow fp \in W_i$ therefore $f
    W_i \subset W_i$. Since $f$ is ergodic and $U_i \subset M - W_i,
    W_i$ is of measure zero.
    $$
    \therefore \quad E = \bigcup^\infty_{i=1} W_i \quad \text{has
      measure 0}
    $$
    $p \notin E$ implies $A^+ p \cap U_i \neq \phi ~\forall i$ and this
    proves that for almost all $p \in M$, $A^+ p$ is dense].

\begin{thm} \label{chap5:thm5.4}
  Let $G_*$ be a semi-simple analytic linear group. Let $\Gamma$ be a
  subgroup such that $G_*/\Gamma$ has a finite invariant measure. Let
  $P$ be a $\mathbb{R}$-parabolic subgroup of $G^*_*$ (= the
  complexification of $G_*$). Set $P_* = P \cap G_*$ then
  $\overline{\Gamma P_*}= G_*$.
\end{thm}

\begin{proof}
  Let\pageoriginale $T$ be a maximal $\mathbb{R}-$split torus in $P$. Let $x \in
  T^\circ_{\mathbb{R}}$  such that for any restricted root $\alpha$ on
  $T$ with $G_\alpha \subset U^+$ the unipotent radical of $P, \alpha
  (x) > 1$.
\end{proof}

Let $U^-$ be the opposite (i.e. $\dot{U}^- = \sum \dot{G}_{- \alpha}$
where $\dot{U}^+= \sum \dot{G}_\alpha$) of $U^+$ and $K_*$ a maximal
compact subgroup of $G_*$. Also let $W^-$ be a nbd. of 1 in $U^- \cap
G_*$ whose logarithm is a convex set. Since $W^-P_*$ is a nbd. of 1 in
$G_*$, and $K_*$ is compact $\exists$ a nbd. $W$ of 1 in $G_*$ with $W
\subset W^- P_*$ and $K_* [W]=W$.

$U x^n W \Gamma$ is stable under $x$ and contains an non-empty open
set, hence by Theorem \ref{chap5:thm5.3} it differs from $G\big|
\Gamma$ in a set of measure zero. Therefore $\exists~ n = n(k)$ such
that
\begin{gather*}
  W^{-1} k^{-1} \Gamma \cap x^n W \Gamma \neq \phi\\
  \therefore \quad \Gamma \cap k W x^n W \neq \phi,\\
  \text{but} \hspace{1cm} x^n W \subset x^n W^- P_* = x^n W^- x^{-n} \cdot
  P_* \subset W^{-} P_*\hspace{2cm}
\end{gather*}
(using the convexity of logarithm of $W^-$)
\begin{align*}
  \therefore \quad & \Gamma \cap k W W^- P_* \neq \phi\\
  \therefore \quad & \Gamma P_* ~\text{meets}~ k W W^-. 
\end{align*}
This proves that $K_* \subset \overline{\Gamma P_*}$. \quad $\therefore
K_* P_* \subset \overline{\Gamma P_*} \subset G_*$.

But we know that $K_* \cdot P_* = G_*$. 

Therefore $G_* = \overline{\Gamma P_*}$.

