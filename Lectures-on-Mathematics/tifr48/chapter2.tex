
\chapter{Intrinsic characterization of $K_*$ and $E$} \label{chap2}

$K_*$\pageoriginale is a maximal compact subgroup of $G_*$, equivalently the
complexification $K$ of $K_*$ is a maximal $\mathbb{R}$-compact
subgroup of $G$.
\begin{align*}
  & E= \Exp (\dot{G}_* \cap S(n))\\
  & \dot{G}_* \cap S(n) = \log E.
\end{align*}

$\log E$ is the orthogonal complement to $\dot{K}_*$ in $\dot{G}_*$
with respect to the Killing form (see \cite{13}).

\begin{thm}\label{chap2:thm2.1} %%% 2.1
  The maximal compact subgroups in a f.c.c. Lie group are conjugate by
  inner automorphism (\cite{13} or Chapter XV \cite{9}).
\end{thm}

For $g GL (n, \mathbb{R})$ we have a linear automorphism of $S(n) s
\mapsto g s {}^tg$ which leaves $P(n)$ stable. This operation of $GL
(n, \mathbb{R})$ on $S(n)$ is called the \textit{canonical action}.

Now let $G_*$ be an analytic semi-simple group with finite center and
let $\rho$ be a finite dimensional representation of $G$ with finite
kernel. By Theorem \ref{chap1:thm1.2} we can assume, after
conjugation that $\rho (G_*)$ is self adjoint.

We set
$$
K_* = \rho^{-1} (\rho (G_*) \cap 0(n, \mathbb{R}))
$$
$K_*$ is then a maximal compact subgroup of $G_*$, let $\varphi : G
\mapsto P(n)$ denote the map
$$
g \mapsto \rho (g) {}^t \rho (g)
$$
then\pageoriginale 
$$
\varphi (g_1 g_2)= \rho (g_1) \varphi (g_2) {}^t\rho (g_1).
$$

Thus under $\varphi$, left translation by $g$ corresponds to the
canonical action by $\rho (g)$ on $P(n)$. In addition
\begin{alignat*}{4}
  \varphi (gk) & = \varphi (g) & &\text{for}~ \quad k \in K_*\\
  \text{and} \qquad \varphi (g_1) & = \varphi (g_2) \qquad &
  &\text{iff}~\quad   g_1 K_* =  g_2 K_*
\end{alignat*}
therefore $\varphi$ induces an injection
$$
\overline{\varphi} : X = G_*/K_* \to P(n).
$$

Let $[S]$ denote the projective space of lines in $S(n)$ and let 
$$
\Pi : S(n) - 0 \to [S]
$$
be the natural projection and let $\psi = \pi \circ
\overline{\varphi}$ and be the composite
$$
G_* \to G_*/ K_* \to [S]
$$
then $\overline{\psi}$ is injective because if $p_1, p_2
\in \overline{\varphi} (X)$ with $\pi p_1 = \pi p_2$, then
since $p_1$, $p_2$ are positive definite matrices, $\exists\, c > 0$
such that 
\begin{alignat*}{5}
   & p_1 = cp_2 &\qquad &&&\\
  \text{so } & |p_1|= c^n |p_2| &&& \text{where}~ |p| & = \det p.\\
  \text{But since }  & p_1, p_2 \in
  \overline{\varphi} (G_*) &&&|p_1| &   = |p_2| = + 1. 
\end{alignat*}
[for $G_*$ being semi-simple, the commutator $[G_*, G_*]= G_*$ and so
  there does not exist a non-trivial homomorphism of $G_*$ into an
  abelian\pageoriginale group. Thus $g \mapsto |\rho (g)|$ is a trivial homomorphism
of $G_*$ into $\mathbb{R}^*$].

This implies that $c= +1$ i.e., $p_1=p_2$. The map
$\overline{\psi}$ is a $G$-map that is $\overline{\psi} (gx)=
g \overline{\psi} (x)$ for all $g \in G$, $x \in X$s
thus $\overline{\psi (X)}$ is stable under $G$.

\begin{defi*}
  If $\rho$ is irreducible over $\mathbb{R}$ then $\overline{\psi
    (X)}$ is called the stable compactification of $X$. This of course
  depends on $\rho$.
\end{defi*}

\begin{remark*}
  The above compactification was arrived in a measure theoretic way by
  Frusentenberg \cite{7}.
\end{remark*}

We shall now show that $X$ has the structure of a symmetric Riemannian
space and shall obtain a decomposition for $\overline{\psi (X)}$ in
terms of symmetric Riemannian spaces.

On $P(n)$ we introduce a infinitesimal metric
$$
ds^2 = \Tr (p^{-1} \dot{p})^2
$$
where $p(t)$ is a differentiable curve in $P(n)$ and $\dot{p}(t)=
\frac{dp}{dt}\Big|_t$.

It is easy to check that this metric is invariant under the action of
$GL(n, \mathbb{R})$ on $P(n)$ and also under the map $p \mapsto
p^{-1}$. This implies that $P(n)$ is a symmetric Riemannian
space. (see \cite{14}).

Let $G_*$ be a semi-simple analytic subgroup of $GL (n, \mathbb{R})$,
then by Theorem \ref{chap1:thm1.3}. 
$$
G_* = (G_* \cap P(n) \cdot (G_* \cap 0 (n))).
$$

Let $A$ be a maximal connected abelian subgroup of $P(n) \cap G_*$.

Since\pageoriginale any abelian subgroup of $P(n)$ can be (simultaneously)
diagonalized, we can assume that $A \subset D(n)$ the set of real
diagonal matrices.

Let $T$ be the Zariski closure of $A$ in $GL (n, \mathbb{C})$ then by
Lemma \ref{chap1:lem1.4}, $T$ is a maximal $\mathbb{R}$-split tours in
the Zariski closure $G$ of $G_*$ in $GL (n, \mathbb{C})$ and 
$$
A= (T_\mathbb{R})^\circ.
$$

Let $\triangle$ be a fundamental system of restricted roots on
$T$. There is a natural faithful representation of $GL (n,
\mathbb{C})$ and therefore of $G$ on $\mathbb{C}^n$. In this section
the complex vector space $\mathbb{C}^n$ considered as a $G$-module
under this representation will be denoted by $V$.

From the representation theory of semi-simple Lie algebras we have
$$
V = \oplus \sum V_\mu
$$
where $\mu'$s are ``weights'' (more precisely, restricted weights) on
$T$. The highest weight will be denoted by $\mu_\circ$. Also we know
that any other weight is of the form $\mu = \mu_\circ- \sum n_\alpha
\alpha$, where each $n_\alpha$ is a non-negative integer.

For $h \in A_\triangle$ we have clearly
\begin{equation*}
  \psi (h) = \pi
  \begin{pmatrix}
    \ddots & & 0\\
    & (\mu(n))^2 &\\
    0 & & \ddots
  \end{pmatrix}
\end{equation*}

After\pageoriginale a conjugation we can assume that the first diagonal entry is
$(\mu\circ(h))^2$.

So
\begin{equation*}
  \psi (h) = \pi
  \begin{pmatrix}
    1\ddots & & 0\\
    & ((\mu-\mu_\circ)(h))^2 &\\
    0 & & \ddots
  \end{pmatrix}
\end{equation*}

Let $\{ h_n\}$ be a sequence in $A_\triangle$ such that the sequence
$\psi(h_n)$ is convergent in the projective space $[S(n)]$. If
necessary by passing to a subsequence, we can assume that $\forall
\alpha \in \triangle \lim\limits_{n \to \infty} \alpha (h_n)$ exists
in $\mathbb{R} \cup \{ \infty\}$ and is equal to $\ell_\alpha$. For a
weight $\mu = \mu_\circ - \sum n_\alpha \alpha$, if we define $\Supp
\mu = \{ \alpha \big| h_\alpha \neq 0\}$, then clearly the diagonal
entry in $\lim\limits_{n \to \infty} \psi (h_n)$, corresponding to the
weight $\mu$ is zero iff $\Supp \mu$ contains some $\alpha$ with $I_\alpha =
\infty$.

\begin{notns}
  For a non-empty subset $\triangle'$ of $\triangle$, we write
  $$
  V (\triangle')= \sum_{\Supp \mu \subset \triangle'} V_\mu
  $$
  $p_{\triangle'}$= the projection of $V$ on $V(\triangle')$ with
  kernal $\displaystyle{\sum_{\Supp \mu \nsubset \triangle'}} V_\mu$

  $\pi_{\triangle'}= \pi (p_{\triangle'} h p_{\triangle'})$ for $h
  \in S(n)$ and let $\psi_{\triangle'}$ be the composite $G_* \to G_*
  /K_* \to P(n)\xrightarrow{P_{\triangle'}}[S(n)]$. $(K_* = G_* \cap 0
  (n, \mathbb{R}))$.
\end{notns}

Since\pageoriginale $V(\triangle')$ is stable under $A$, we note that
$p_{\triangle'} h = h p_{\triangle'}= p_{\triangle'}h
p_{\triangle'}$. The preceding remarks establish

\begin{lemma} \label{chap2:lem2.2}
  $$
  \overline{\psi (A_{\triangle'})}= \bigcup_{\triangle' \subset
    \triangle} \psi_\triangle (\overline{A}_\triangle)
  $$
\end{lemma}

  Also, if 
  $$
  K_* = G_* \cap 0 (n, \mathbb{R})
  $$
  we have by theorem \ref{chap1:thm1.6}
  \begin{align*}
    E & = K_* [\bar{A}_\triangle]\\
    G_* & = E \cdot K_* \\
    & = K_* [\bar{A}_\triangle] \cdot K_*\\
    \therefore \quad \overline{\psi (X)} & = \overline{\psi (G_*)}=
    \overline{\psi (K_* [\bar{A}_\triangle]\cdot K_*)}= \overline{\psi
    (K_* [\bar{A}_\triangle])}\\
    & = \overline{\psi (K_* [A_\triangle])} = \overline{\pi \psi K_*
      [A_\triangle]}= \overline{\pi (K_* [\varphi A_\triangle])}\\
    & = \overline{K_* \cdot (\pi \varphi A_\triangle)}= K_* \cdot
    \overline{\pi \varphi (A_\triangle)}
  \end{align*}

For $h$, $h' \in \dot{T} < h, h'> = T_r (hh')$ is a inner product on
$\dot{T}$. This inner product induces an inner product on $\Hom
(\dot{T}, \mathbb{C})$ and hence its restriction on $\Hom (T,
\mathbb{C}^*) \hookrightarrow \Hom (\dot{T}, \mathbb{C})$. This
restriction will again be denoted by $<, >$. 

\begin{lemma} \label{chap2:lem2.3}
  If $\dot{G}_\alpha V_\mu =0$, then the following two conditions are
  equivalent.
  \begin{enumerate}
    \item $\dot{G}_{- \alpha} V_\mu =0$.
      \item $<\mu, \alpha>=0$.
  \end{enumerate}
\end{lemma}

\begin{proof}
  We\pageoriginale can choose $X_\alpha \in \dot{G}_\alpha$ such that $\Tr (X_\alpha
  {}^tX_\alpha)=1$ set
  \begin{align*}
    [X_\alpha, {}^tX_\alpha] & = h'_\alpha, ~\text{then for $h \in
      \dot{T}$ we have}\\
    <h, h'_\alpha> & = \Tr hh'_\alpha= \Tr h[X_\alpha, {}^tX_\alpha]\\
    & = \Tr [h, X_\alpha]^t {}^rX_\alpha = \alpha (h) \cdot \Tr X_\alpha
    {}^t X_\alpha = \alpha (h)
  \end{align*}
$\therefore \quad h'_\alpha = h_\alpha$ where $h_\alpha$ is the dual
of $\alpha$ in the inner product. Therefore for any weight $\mu$,
$<\mu, \alpha>= \mu (h_\alpha)$.
\end{proof}

By considering the representation of 3-dimensional simple Lie algebra
generated by $\{ X_\alpha, h_\alpha, {}^tX_\alpha\}$ on
$\displaystyle{\sum_{n \in \mathbb{Z}}} V_{\mu + n \alpha}$ the result
follows immediately. (see \cite{10} or pp. IV-3 to IV-6 of
\cite{23}). 

\begin{defi*}
  Let $E_\rho= \{ \alpha \big| \alpha \in \triangle < \alpha ,
  \mu_\circ > =0$
\end{defi*}

$A$ subset $\triangle'$ of $\triangle$ is said to be $\rho-connected$
if $\triangle' \cup \{ \mu_\circ\}$ is connected in the sense of
Dynkin's diagram of $\triangle'$ lies in $E_\rho$.

For $\triangle' \subset \triangle$ we set $\widetilde{\triangle}'=
\triangle' \cup \{ \alpha \big| \alpha \in E_\rho \alpha \big|
\beta$ for $\forall \beta \in \triangle'\}$. The following is an easy
consequence of the previous lemma.

\begin{lemma} \label{chap2:lem2.4}
  $A$ subset $\triangle'$ of $\triangle$ is $\rho$-connected iff there
  is a weight $\mu$ with support $\mu = \triangle'$.
\end{lemma}

\begin{proof}
  By induction on $s=$ the cardinality of $\triangle'$. If $s=1$ the
  result follows at once from lemma \ref{chap2:lem2.3}. If $s>1$ then
  $\triangle'$ contains a $\rho$-connected subset $\triangle''$ of
  cardinal $s-1$, and hence there is a weight $\triangle'' = \mu_\circ - n_1
  \alpha_1 - \cdot n_{s-1} \alpha_{s-1}$, where $\triangle''= \alpha_1,
  \ldots , \alpha_{s-1}$.
\end{proof}

Let\pageoriginale $\alpha_s \in \triangle' - \triangle''$. Then $<\mu, \alpha_s>=
<\mu_\circ, \alpha_s>- \sum n_k <\alpha_k , \alpha_s > \geq 0$
and is not zero since $\triangle'' \cup \{ \alpha_s\}$ is
$\rho$-connected. Hence $\mu- \alpha_s$ is a weight of support
$\triangle'$. 

\begin{coro} \label{chap2:coro2.5}
  $V(\triangle')=V$ (largest $\rho$-conn. subset in $\triangle'$) and 
  $$
  \overline{\psi (A_\triangle)}= \bigcup_{\substack{\triangle' \subset
  \triangle\\\triangle'- \rho \,\text{conn}}.} \psi_{\triangle'}
  (\overline{A}_\triangle)
  $$
\end{coro} 

\begin{lemma} \label{chap2:lem2.6}
  $$
  \overline{\pi (E)}= \bigcup_{\substack{\triangle' \subset
      \triangle\\\triangle' \rho-\text{conn.}}} G_* \cdot
  \underset{\triangle'}{\pi} (1). 
  $$
\end{lemma}

\begin{proof}
  \begin{align*}
  \overline{\pi(E)} & = \overline{\psi (E)} = \overline{\pi K_*
    [\overline{A}_\triangle]}\\
  & = \overline{\psi K_* [A_\triangle]}= K_* \cdot \overline{\psi
    (A_\triangle)}\\
  & = K_* \cdot \bigcup_{\substack{\triangle' \subset
      \triangle\\ \triangle' \rho-\text{conn}}} \psi_{\triangle'}
  (\bar{A}_\triangle) =\bigcup_{\substack{\triangle' \subset
      \triangle\\ \triangle' \rho-\text{conn}.}} K_* \cdot
  \psi_{\triangle'} (\bar{A}_\triangle)\\
  & = \bigcup_{\triangle' \rho-\text{conn}} K_* (\bar{A}_\triangle
  \psi_{\triangle'} (1)) = \bigcup_{\substack{\triangle \subset
      \triangle\\ \triangle' \rho-\text{conn}.}} (K_*
  \bar{A}_\triangle) \psi_{\triangle'} (1)\\ % \tag{1}\label{chap2:lem2.6:eq1}\\
  & = \bigcup_{\substack{\triangle' \subset
      \triangle\\ \triangle' \rho-\text{conn}.}} G_* \cdot \pi_{\triangle'}(1). 
  \end{align*}
  Since $\triangle$ is finite there are only finitely many subsets
  $\triangle' \subset \triangle$. So this lemma in particular shows
  that $\overline{\psi (X)}= \overline{\pi (E)}$ consists of a finite
  number\pageoriginale of $G_*$ orbits.
\end{proof}

\begin{lemma} \label{chap2:lem2.7}
  \begin{enumerate}[\rm (i)]
    \item For $h \in \perp_{\triangle'}$ and $v: V (\triangle')$, $hv
      = \mu_\circ (h) v$
      \item $\dot{G}_\alpha V(\triangle')=0$ if $\alpha > 0$ and
        $\alpha \psi \{ \triangle\}$
        \item $\dot{G}_\alpha V (\triangle')=0$ if $\alpha \in \{ \widetilde{\triangle'}- 
          \triangle \} $ 
  \end{enumerate}
\end{lemma}

\begin{proof}
  Parts (i) and (ii) are immediate. (iii) follows from Lemma
  \ref{chap2:lem2.3} and (ii) of this lemma.
\end{proof}

For each restricted root $\alpha$, set $G_\alpha$ the group generated
by $\{ \Exp X, X \in \dot{G}_\alpha\}$. For a subset $\triangle'$ of
$\triangle$ let $G' (\triangle')$ be the group generated by
$G_\alpha$, $\alpha \in \{ \triangle' \}$ and let $K (\triangle')$ be
the subgroup generated by $\Exp (X- {}^tX) X \in \dot{G}_\alpha$,
$\alpha \in \{ \triangle'\}$ and maximum $\mathbb{R}$-compact subgroup
of $Z(T)$. $G'(\triangle')$ is semisimple.

We write
\begin{align*}
  G_* (\triangle') & = G (\triangle') \cap G_*; K_* (\triangle')=
  K(\triangle') \cap G_*\\
  G_*' (\triangle')& = G' (\triangle') \cap G_*; K_* = K_*
  (\triangle)= K(\triangle) \cap G_*\hspace{2cm}\\
  \text{and} \hspace{1cm} K'_* (\triangle') & = K_* \cap G'
  (\triangle'). 
\end{align*}

It is easy to see that $G(\triangle')= G' (\triangle')\cdot Z(T)$;
$G'_* (\triangle')= (G' (\triangle')_{\mathbb{R}})^\circ$ but
$G_*(\triangle')$ need not be connected. Also $K_* (\triangle')$ and
$K'_* (\triangle')$ are maximal compact subgroups of $G_*(\triangle')$
and $G'_*(\triangle')$ respectively.

\begin{remarks*}
~
  \begin{enumerate}[(i)]
    \item Since $g \in G_\alpha$ implies $Z^{-1} g Z \in G_\alpha
      \forall Z \in Z(T)$ we have $G_\alpha \cdot Z(T) = Z(T) \cdot
      G_\alpha$.
      \item Since $(\widetilde{\triangle}- \triangle')\perp \triangle'$, roots
        in $(\{ \widetilde{\triangle}'\} - \{ \triangle'\})=$ roots in
        $\{ \widetilde{\triangle}' - \triangle'\}$.
        \item The\pageoriginale Lie algebras of $G(\triangle')$ and $P(\triangle')$
          are respectively
          $$
          Z(T) +\sum_{\alpha \in \{ \triangle' \}} \dot{G}_\alpha
          ~\text{and}~ Z(T) + \sum_{\alpha > 0} G_\alpha +
          \sum_{\substack{\alpha< 0\\ \alpha \in \{
              \widetilde{\triangle}'\}}} G_\alpha
          $$
          \item $P (\triangle')$ is connected and for $\triangle'
            \supset \triangle''$ we have $P (\triangle') P
            (\triangle'')$. Now we prove following results, which
            allow us to determine the $G_*$ orbits in
            $\overline{\psi (X)}$.
  \end{enumerate}
\end{remarks*}

\begin{lemma} \label{chap2:lem2.8}
~
  \begin{enumerate}[\rm (i)]
    \item The stabalizer of $V (\triangle')$ is $P
      (\widetilde{\triangle}')$
      \item The stabalizer of $P (\triangle')\cdot \pi_{\triangle'}$
        (1) is $P(\widetilde{\triangle}')$.
        \item The stabalizer of the point $\pi_{\triangle'}$
          (1) in $G_*$ is 
          $$
          G_* (\widetilde{\triangle}'- \triangle') \cdot K_*
          (\triangle') N_* (\triangle') \cdot ({}^\perp \triangle'
          \cap A).
          $$
  \end{enumerate}
\end{lemma}

\begin{proof}
~
  \begin{enumerate}[(i)]
    \item It is clear that the stabalizer of $V(\triangle')$ contains
      $P(\triangle')$  hence is a parabolic group and therefore it is
      connected. $V(\triangle')$ is stable under a connected subgroup
      $H$ iff it is stable under $\dot{H}$. From this it can be easily
      proved that the stabalizer is $P(\widetilde{\triangle}')$.
      \item Let $S$ be the Stabalizer of $P(\triangle')
        \pi_{\triangle'}$ (1) and $S_{\triangle'}$
        the Stabalizer of $\pi_{\triangle'}$ (1)
  \end{enumerate}
\end{proof}

Clearly $S \supset P (\triangle')$. If $x$ stabalizer $P(\triangle')
\pi_{\triangle'}$ (1) $x. \pi_{\triangle'}$
(1) for some $p \in P (\triangle')$, this implies
that $p^{-1} x. \pi_{\triangle'}$ (1) =
$\pi_{\triangle'}$ (1) i.e. $p^{-1} x \in S_{\triangle'}$.

Hence $S= P (\triangle') \cdot (S_{\triangle'} \cap S)$.

We first prove that $S_{\triangle'} \subset P (\widetilde{\triangle}')$.

If\pageoriginale $g \in S_{\triangle'}$ then $g p_{\triangle'} {}^t g= c
p_{\triangle'}$ for some $c \in \mathbb{R}$
$$
\displaylines{\text{i.e.} \hfill g p_{\triangle'} = c p_{\triangle'}
  (^tg)^{-1} \hfill \cr
  \text{So}\hfill g V_{\triangle'} = g p_{\triangle'} V_{\triangle'} =
  c p_{\triangle'} (^tg)^{-1} V_{\triangle'} \subset V_{\triangle'}
  \hfill }
$$
\begin{align*}
  \therefore \qquad & g ~\text{the stabalizer of }~ V(\triangle') = P
  (\widetilde{\triangle}').\hspace{2cm}\\
  \therefore \qquad & S_{\triangle'} \subset P(\widetilde{\triangle}')\\
  \therefore \qquad & S \subset P(\triangle'). P(\widetilde{\triangle}') = P
  (\widetilde{\triangle}'). 
\end{align*}

From parts (ii) and (iii) of Lemma \ref{chap2:lem2.7} it follows
almost immediately that $N(\triangle') \subset S_{\triangle'}$ and
$G_\alpha \subset S_\triangle'$ and $G_\alpha \subset S_{\triangle'}$,
$\forall \alpha \in \{ \widetilde{\triangle}'- \triangle'\}$ (\& So
$G' (\widetilde{\triangle}'- \triangle') \subset S_{\triangle'}$)
i.e., $N(\triangle')$. $\pi_{\triangle'}$ (1)=
$\pi_{\triangle'}$ (1) = $G'
(\widetilde{\triangle'}- \triangle') \pi_{\triangle'}$
(1). Also from the remark (ii) after \S 2.7, we
get $G_\alpha \subset Z(G' (\triangle')) \forall \alpha \in \{
\widetilde{\triangle}' - \triangle'\}$.

Now we prove that $G' (\widetilde{\triangle}'- \triangle') \subset
S$. 

For $\alpha \in \{ \widetilde{\triangle}' - \triangle'\}$
\begin{align*}
  G_\alpha \cdot P(\triangle') \pi_{\triangle'}
  (1) & G_\alpha \cdot G(\triangle') N
  (\triangle'). \pi_{\triangle'} \text{(1)}\\
  & = G_\alpha G(\triangle') \pi_{\triangle'}
  (1)\\
  & = G_\alpha G' (\triangle') Z(T)
  \pi_{\triangle'}(1)\\ 
  & = G_\alpha G' (\triangle') G_\alpha Z(T) \pi_{\triangle'}
  (1)\\
  & = G' (\triangle') Z(T) \cdot G_{\alpha} \pi_{\triangle'}
  (1) = G' (\triangle') Z(T)
  \pi_{\triangle'} (1)\\
  & \subset P(\triangle') \pi_{\triangle'}
  (1). 
\end{align*}

This\pageoriginale proves that $\forall \alpha \in \{ \widetilde{\triangle}'-
\triangle'\} G_\alpha \subset S$ and therefore $G'
(\widetilde{\triangle}'- \triangle') \subset S$.

From the Lie algebra considerations it is easy to see that the group
given by $G' (\widetilde{\triangle}' - \triangle')$ and
$P(\triangle')$ is $P(\widetilde{\triangle}')$.

$\therefore$ \quad $P(\widetilde{\triangle}') \subset S$. This proves
that $S= P(\widetilde{\triangle}')$.

\noindent (iii) In (ii) we proved
$$
\displaylines{S_{\triangle'} \subset P(\widetilde{\triangle}').\cr
\text{As} \hfill P(\widetilde{\triangle}') = N
(\widetilde{\triangle}') \cdot  G (\widetilde{\triangle}')\hfill \cr
\text{and} \hfill S_{\triangle'} \supset N (\triangle') \supset N
(\widetilde{\triangle}')\hfill\cr
\text{we have} \hfill S_{\triangle'} = N (\widetilde{\triangle}')
\cdot (S_{\triangle'} \cap G(\widetilde{\triangle}'))\hfill \cr
\text{since} \hfill G(\widetilde{\triangle}') =
G'(\widetilde{\triangle}'- \triangle') \cdot G(\triangle') ~\text{and
  since}\hfill \cr
G' (\widetilde{\triangle}'- \triangle') \subset S_{\triangle'}\cr
G(\widetilde{\triangle}') \cap S_{\triangle'} = G'
(\widetilde{\triangle}'- \triangle') \cdot \{ G(\triangle') \cap
S_{\triangle'}\} \cr
\text{clearly} \hfill S_{\triangle'} \supset
K(\triangle')_{\mathbb{R}} \quad \text{and}\quad S_{\triangle'} \cap
T= {}^\perp \triangle'\hfill}
$$
\begin{align*}
  (G(\triangle')S_{\triangle'})_{\mathbb{R}} & =
  (S_{\triangle'})_{\mathbb{R}} \cap (G(\triangle'))_{\mathbb{R}}= (S_{\triangle'})_\mathbb{R}
  (K(\triangle')_{\mathbb{R}}  A K(\triangle')_{\mathbb{R}})\\
  & = K(\triangle')_{\mathbb{R}} \cdot ({}^\perp \triangle' \cap A)\\
  \therefore \quad (S_{\triangle'}) \cap G_* & = N_*
  (\widetilde{\triangle}') \cdot (S_{\triangle'} \cap G(\triangle'))\\
  & = N_* (\widetilde{\triangle}') \cdot G'_* (\widetilde{\triangle}'-
  \triangle') \cdot K_* (\triangle') \left(  {}^\perp \triangle' \cap
  A\right). 
\end{align*}\pageoriginale

\begin{lemma} \label{chap2:lem2.9}
~
  \begin{enumerate}[(1)]
    \item $G_*= K_* P_* (\triangle')$ \quad $\forall \triangle'
      \subset \triangle$
      \item dimension of $K_* (\triangle')\cdot N(\triangle')$ is
        independent of $\triangle'$.
  \end{enumerate}
\end{lemma}

\begin{proof}
  \begin{enumerate}[(i)]
  \item  Since $K_* \cdot P_* (\triangle') \subset G_*$ and since for
  $\triangle' \subset \triangle'' P_* (\triangle') \subset P_*
  (\triangle'')$ it is sufficient to prove that
  $$
  G_* = K_* \cdot P_* (\phi).
  $$
  As a vector space
  $$
  K_* \dot{(}\triangle)+ P_* \dot{(}\phi) = \dot{G}_*
  $$
  
  By implicit function theorem $K_* P_* (\phi)$ is open in $G_*$. Also
  it is closed, since $K_*$ is compact. Connectedness of $G_*$ implies
  the result.
  
\item For a positive root $\alpha$, let $\{ X^i_\alpha\}$, be a basis
  of $\dot{G}_\alpha$ then the set 
$${\bigcup_{\substack{
      \alpha  \in \{ \triangle'\}\\ \alpha > 0}} \{ X^i_\alpha-
  {}^tX^i_\alpha\} \cup \bigcup_{\substack{\alpha > 0\\ \alpha \notin \{
      \triangle'\}}} \{ X^i_{\alpha}\}}$$ 
is a basis for the Lie algebra of  $K_* (\triangle')\cdot N(\triangle')$. This shows that $\dim (K_*   
  (\triangle')N(\triangle'))= \dim \left(\displaystyle{\mathop{\sum}_{\alpha > 0}}
  \dot{G}_\alpha\right)$.
  \end{enumerate}
\end{proof}

\begin{defi*}
  If\pageoriginale $A$ is a right $H$-set and $B$ a left $H$-set, where $H$ is a
  group, then $A \times{}_H B$ denotes the set $(A \times
  B)_{/\mathbb{R}}$ where $\mathbb{R}$ is the equivalence relation
  $(ah, \underline{h^{-1}b) \sim (a, b)}$, $\forall h \in H$.
\end{defi*}

By the previous lemma
$$
G_* = K_* P_* (\triangle') \quad \forall \triangle' \subset \triangle 
$$

So the $G_*$ orbit of 
\begin{align*}
  \pi_{\triangle'} \text{(1)} & = G_* \cdot
  \pi_{\triangle'} \text{(1)}\\
  & = K_* \cdot P_* (\triangle') \pi_{\triangle'}
  \text{(1)} \\
  & = K_* (P_* (\triangle') \pi_{\triangle'}
  \text{(1)}) \\
  & \approx \frac{P_* (\triangle')}{K_* (\triangle') N_*
    (\triangle')({}^\perp \triangle' \cap A)}. K_* ~\text{(by part
    (iii) of Lemma \ref{chap2:lem2.8})}.\\
  & = \frac{G^*(\triangle')}{K_* (\triangle')} \times K_*.
\end{align*}

Since $P_* (\triangle')= G'_* (\triangle') \cdot N_* (\triangle')
\cdot (Z(T) \cap G_*)$. If we put
$$
\frac{G'_* (\triangle')}{K_* (\triangle')}= X (\triangle')
$$
we have the $G_*$ orbit of $\pi_{\triangle'}$ (1)
$\approx X(\triangle') \times \displaystyle{\mathop{K_*}_{K_*
  (\triangle')}}$

This is compact iff $X (\triangle')$ is a single point set,
equivalently iff\break $G'_* (\triangle') = K'_{*}(\triangle')$ i.e., iff
$\triangle'= \phi$.

Then\pageoriginale the orbit is the compact set $X_\circ = \frac{K_*}{K_* (\phi)}
\cdot \pi_{\triangle'}$ (1). Also from part (iii)
of Lemma \ref{chap2:lem2.8} it is clear that $\dim S_{\triangle'} \geq
\dim S_{\triangle''}$ if $\triangle' \subset \triangle''$.

So we have proved.

\begin{theorem*}[(Satake)]
  $\overline{\psi (X)}$ consists of a finite number of {\small $G_*$}
  orbits. Among these there is a unique compact orbit $X_\circ$, also
  characterized as the orbit of minimum dimension.
\end{theorem*}
