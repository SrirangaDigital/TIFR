\part{Theorem of Browder and Novikov}\label{part1}

\chapter{}

\markboth{\thepart. Theorem of Browder and Novikov}{}

\thispagestyle{plain}
\section{Preliminaries}\label{chap1:sec1}\pageoriginale

\subsection{The cap-froduct}\label{chap1:subsec1.1}%% 1.1

The homology and the cohomology groups we use are the singular
ones. Let $\mathbb{Z}$ denote the ring of integers and $\wedge$ an
arbitrary commutative ring with $1 \neq 0$. For any topological space
$X$ and any integer $n \geq 0$ the set of singular $n$-simplices of $X$
is denoted by $S_n (X)$. For any $s \in S_n (X)$ and any
integer $i$ satisfying $0 \leq i \leq n$ let $s(0, \ldots i)$ (resp. $s
(i,\ldots, n)$) denote the element of $S_i(X)$ (resp. $S_{n-i} (X)$) got
by restricting $s$ to the front $i$-dimensional (resp. The rear
$(n-i)$-dimensional) face of the standard $n$-simplex $\Delta_n$. Let
$C(X)$ denote the singular chain complex of $X'$ over $\mathbb{Z}$ and
$C = C(X) \otimes_{\mathbb{Z}} \wedge$ the chain complex of $X$ over
$\wedge$. The cochain complex of $X$ over $\wedge$ which is defined as
$\Hom_{\mathbb{Z}} (C (X), \wedge)$ is canonically isomorphic to
$\Hom_\wedge (C (X) \otimes_{\mathbb{Z}} \wedge, \wedge)$. The boundary
homomorphism $\delta$ in $C^* = \Hom _\wedge (C,\wedge)$ is given by $f
= (-1)^{n-1} f \circ \partial$ for every $f \in C^n (X, \wedge) = \Hom
(C_n, \wedge)$ where $\partial : C_n \to C_{n-1}$ is the boundary
homomorphism in $C$. As usual $C^*$ is considered as a chain complex
with $C^{*}_{-n} = C^n (X, \wedge)$. The evaluation $\map e: C^*
\otimes_{\wedge} C \to \wedge$ is defined by $e (f \otimes c) = f (c)
\forall f \in C^{*}_{-n}$ and $c \in C_n$ and $e |
C^*_{-p} \otimes C_q =0$ whenever $p\neq q$. Considering $\wedge$ as a
chain complex (with all its elements of degree zero) it is easily seen
that $e: C^* \otimes_{\wedge} C \to \wedge$ is a chain
homomorphism. 

For\pageoriginale any two chain complexes $A$ and $B$ over $\wedge$
let $\alpha : 
H(A) \otimes_{\wedge} H (E) \to H(A \otimes B)$ be the natural $\map$. If
$x \in H_p (A)$ and $y \in H_q (B)$ and if $z$ and $z'$ are
respectively cycles of $A$ and $B$ representing $x$ and $y$, then $z
\otimes z'$ is a cycle of $A\otimes B$ and the homology class of $z
\otimes z'$ is by definition $\alpha (x \otimes y)$. Let $T: A\otimes_{\wedge}
B\to B \otimes_\wedge A$ be the chain isomorphism given by $T(a
\otimes b)= (-1)^{pq} b \otimes a$ $\forall a \in A_p$, $b \in
B_q$. 

The Alexander-Whitney diagonal $\map m_0 : C \to C \otimes_{\wedge}
C$ is defined to be the unique $\wedge$-homomorphism satisfying $m_0
(s) = \sum\limits^{n}_{i=0} s (0,\ldots, i) \otimes_{\wedge} s
(i,\ldots,n) \; \forall s \in S_n  (X)$. It is well-known and is not
hard to check that $m_0$ is a chain $\map$. We denote the composition of
the chain homomorphism indicated in the following diagram 
\[
\xymatrix{
C^* \otimes_{\wedge} C \ar[r]^{Id_{C^*} \otimes m_0} & C^*
\otimes_{\wedge} C \otimes_{\wedge} C \ar[r]^{T\otimes Id_C} & 
C\otimes_{\wedge} C^* \otimes_{\wedge} C  \ar[r]^{Id_C \otimes e}
& C\otimes_{\wedge} \wedge = C 
}
\]
by $\bigcap : C^* \otimes_{\wedge} C\to C$. More explicitly this $\map$ is
given by  
{\fontsize{10}{12}\selectfont
$$
\bigcap (f \otimes s) = f \cap s=
\begin{cases}
(-1)^{q(n-q)} f (s (n-q, \ldots, n)). s (0, \ldots, n-q) \text{ if } n
  \geq q\\ 
o \text{ if  } n <  q
\end{cases}
$$}\relax    
for every $f \in C^q (X, \wedge)$ and $s \in S_n (X)$. Let
$H (\cap) : H (C^* \otimes_{\wedge} C) \to H (C)$ be the homomorphism
induced by `$\bigcap$'. For any\pageoriginale $a \in H^q(C^*) =
H_{-q} (C^*) = H^q (X, \wedge)$ and $u \in H_n (C) = H_n (X, \wedge)$ the
element $H(\bigcap) o \alpha (a \otimes u)$ is called the cap-product
of a by $u$ and is denoted by $a \cap u$. 

The chain $\map e : C^* \otimes_{\wedge} C \to \wedge $ induces a
homomorphism $H(e) : H (C^* \otimes_\wedge C) \to \wedge$. For any $a
\in H^q (X, \wedge)$ and $u \in H_q (X, \wedge)$ the image
$H(e) o \alpha (a \otimes u)$ is known as the value of the cohomology
class a on the homology class $u$ and is denoted by $a(u)$. 
 
\subsection{}\label{chap1:subsec1.2}%sec 1.2 
The following properties of the cap-product will be needed later.
\begin{enumerate}
\renewcommand{\labelenumi}{\bf(\theenumi)}
\item $(a  \cup b) \cap u=a \cap (b \cap u) \forall a \in H^p
  (X, \wedge)$, $b \in H^q (X, \wedge)$ and $u \in H_n (X,
  \wedge)$ with $p, q, n$ arbitrary integers. Here $a \cup b$
  denotes the Cup product of $a$ and $b$. 

\item For any continuous $\map f: Y\to X$, if the induced homomorphisms
  in homology and cohomology are denoted by $f_* : H (Y,\break \wedge) \to
  H(X, \wedge)$ and $f^* : H^* (X, \wedge) \to H^* (Y, \wedge)$, then
  for any $a \in H^q (X, \wedge)$ and $v \in H_n (Y,
  \wedge)$ 
$$
f_* (f^* a \cap v) = a \cap f_* (v). 
$$
\end{enumerate}


\subsection{Poincar'e Duality}\label{chap1:subsec1.3}%sec 1.3

When we refer to homology and cohomology groups without mentioning the
coefficients we mean integer coefficients. Let $M$ be a compact,
connected, orientable manifold (without boundary) of dimension
$n$. Then it is known that $H_n (M) \simeq \mathbb{Z}$. A choice of a
generator\pageoriginale $u$ for $H_n (M)$ is known as an orientation
for $M$. $M$ 
together with a chosen orientation is called an oriented manifold and
the distinguished element of $H_n (M)$ is called the fundamental class
of $M$ and is denoted by $[M]$. 

Let $h : \mathbb{Z} \to \wedge$ be the obvious ring homomorphism
(which sends 1 of $\mathbb{Z}$ into 1 of $\wedge$). Let $v = h_*
([M])$ where $h_* : H_n (M) \to H_n (M, \wedge)$ is the homomorphism
induced by $h$. Then Poincare duality can be stated as follows: 

The $\map \Delta : H^q (M, \wedge) \to H_{n-q} (M, \wedge)$ given by
$\Delta (x) = x \cap v$ is an isomorphism for all $q$. 

In case $M$ is not necessarily orientable it is true that $H_n
(M;\mathbb{Z}_2) \simeq \mathbb{Z}_2 $ and if $v$ denotes the non zero
element of $H_n (M; \mathbb{Z}_2)$ then $\bigcap v : H^q (M;
\mathbb{Z}_2) \to H_{n-q} (M; \mathbb{Z}_2)$ is an isomorphism for all
$q$. 

When $M$ is compact and not necessarily connected $M$ is orientable if
and only if each of its connected components is orientable. $M$ being
compact, the number of connected components is finite and denoting
them by $\{M_j\}^r_{j=1}$ we have $H_n (M) \simeq \oplus^{r}_{j=1}
H_n (M_j)$. If each $M_j$ is oriented and if $[M_j]$ is the
fundamental class of $M_j$ then $[M] = \sum\limits^{r}_{j=1} [M_j]\in
H_n (M) = \oplus^{r}_{j=1} H_n (M_j)$ is defined to be the
fundamental class of $M$.  

\subsection{}\label{chap1:subsec1.4} %sec 1.4
All the\pageoriginale vector bundles we consider are real vector
bundles. For any 
$X$ the trivial vector bundle of rank $\ell$ over $X$ will be denoted
by $\mathscr{O}^l_X$. The total space and the base space of any vector
bundle $\xi$ will be denoted by $E(\xi)$ and $B_{\xi}$ respectively. To
denote that $\xi$ is of rank $k$ we just write $\xi^k$. If $f : Y\to
X$ is a continuous $\map$ and $\xi$ any vector bundle over $X$ the pull
back bundle on $Y$ is denoted by $f' (\xi)$. If $\xi$ carries a
Riemannian metric, for any $\varepsilon > 0$ the subspace of $E(\xi)$
consisting of vectors of length $\leq \varepsilon$ is denoted by $E_\varepsilon
(\xi)$ and the boundary consisting of vectors of length $\varepsilon$ is
denoted by $\dot{E}_\varepsilon (\xi)$. When $B_\xi$ is compact the Thom
space $\xi$ denoted by $T(\xi)$ is defined to be the one point
compactification of $E(\xi)$. Let `$\infty$' denote the point at
infinity of $T(\xi)$. When $\xi$ carries a Riemannian metric we can
describe the Thom space alternatively as follows. Let $T_\varepsilon
(\xi)$ be the quotient space got from $E_\varepsilon (\xi)$ by collapsing
$\dot{E}_\varepsilon (\xi)$ to a point. The $\map \beta : E_\varepsilon
(\xi) \to T(\xi)$ defined by $\beta (v^{\rightarrow}) =
\dfrac{v^{\rightarrow}}{\varepsilon -||v^{\rightarrow}||}$ for $v^{\to}
\in E_\varepsilon (\xi) - \dot{E}_\varepsilon (\xi)$ and $\beta
(v^{\to}) = \infty$ for $\overrightarrow{v}\in \dot{E}_\varepsilon
(\xi)$ passes 
down to a homeomorphism $\Theta :T_\varepsilon (\xi) \to
T(\xi)$. Compactness of $B_\xi$ is essential for $\Theta$ to be a
homeomorphism.  

For any differential $(=C^\infty)$ manifold  $M$ the tangent bundle of
$M$ will be denoted by $\tau M$. The word differentiable will always
mean differentiable of class $C^\infty$ for us. For the rest of
\textit{this} sections $M$ denotes a compact, connected, oriented
differential manifolds\pageoriginale of dimension $n \geq 0$ with
$[M]$ as the fundamental class. By Whitney's imbedding theorem $M$ can be
differentially imbedded in $\mathbb{R}^{n+k}$. Except when $n=0$ the
compactness of $M$ automatically implies that $k \geq 1$. Even when
$n=0$ we can assume $k \geq 1$. Let $\nu$ be the normal bundle of this
imbedding. Then $\tau_M \oplus \nu \simeq
\mathscr{O}^{n+k}_{M}$. Since $\tau_M$ and $\mathscr{O}^{n+k}_{M}$ are
both orientable it follows that $\nu$ is an orientable vector
bundle. Identifying the tangent space  to $\mathbb{R}^{n+k}$ at any
point with $\mathbb{R}^{n+k}$ in the usual way and taking the usual
Riemannian metric on $\tau_{\mathbb{R}^{n+k}} \simeq \mathscr{O}^{2n
  + 2k}_{\mathbb{R}^{n+k}}$ any element of $E(\nu)$ can be thought of as a pair
$(x, \dfrac{\mathbb{R}}{v})$ with $x \in M$ and $v^{\to} \in  
\mathbb{R}^{n+k}$ in a directional normal to $M$ at $x$. Let $e :
E(\nu) \to \mathbb{R}^{n+k}$ be defined by $e (x, v) = x+v$. $\exists$
an $\varepsilon > 0$ such that $e$ is a diffeomorphism of the set
$E_\varepsilon (\nu)$ on to a neighbourhood $A$ of $M$. $A$ is called a
closed tabular neighbourhood of $M$. Let $\dot{A} = e
(\dot{E}_\varepsilon (\nu))$. Considering $S^{n+k}$ as the one point
compactification of $\mathbb{R}^{n+k}$ we can define a $\map C: S^{n+k}
\to T(\nu)$. This is the $\map$ got by collapsing the complement of $A
-\dot{A}$ in $S^{n+k}$ to a point. More precisely, $C| A = \beta
oe^{-1}$ and $C | (S^{n+k} - A) = \infty$.  

Let $\Phi : H_n (M) \to H_{n+k} (T (\nu))$ be the Thom isomorphism
\cite{c1:key5}.  


\setcounter{prop}{4}
\begin{prop}\label{chap1:prop1.5}%1.5.
$\Phi ([M]) = C_* (\iota)$ for a generator $\iota$ of $H_{n+k} (S^{n+k})$.  
 \end{prop} 
 
 \begin{proof}
We have only to show that $C_* : H_{n+k} (S^{n+k}) \to H_{n+k}
(T(\nu))$ is an isomorphism. We abbreviate $E_\varepsilon (\nu)$ by
$E_\varepsilon$ etc. Let $A_{\frac{1}{2}} = e(E_{\varepsilon/2})$. Clearly $\beta|
E_{\frac{\varepsilon}{2}}$ is a homeomorphism of $E_{\frac{\varepsilon}{2}}$ onto
the image\pageoriginale $\Gamma$ (say). Let $x$ be any point in $M$
(such a point exists because dim $M \geq 0$ by assumption) and $i_x :
S^{n+k} \to (S^{n+k}, S^{n+k} - x)$ and $j_x: (S^{n+k},
S^{n+k}-M) \to (S^{n+k}, S^{n+k}-x)$ the respective inclusions. Consider
the following commutative diagram.  

{\parfillskip=0pt
 The homomorphism indicated as $\beta_*$ is an isomorphism since
 $\beta : E_{\frac{\varepsilon}{2}} \to \Gamma$ is a homeomorphism. It
 follows that the monomorphism numbered \ding{172} is an
 isomorphism. The space $T(\nu) - M$ is contractible in
 itself\pageoriginale to 
 $\infty$. Hence the $\map H_{n+k} (T (\nu)) \to H_{n+k} (T (\nu)$, $T
 (\nu) -M)$ is an isomorphism. (The assumption $k  \geq 1$ is used
 here). Since $H_{n+k} (T(\nu)) \simeq H_n (M) \simeq \mathbb{Z}$ we
 have $H_{n+k} (S^{n+k}, S^{n+k} -M)$. Since $(i_x)_*$ is an
 isomorphism it follows that $j_*$ is a monomorphism and that image of
 $j_*$ is a direct summand of $H_{n+k} (S^{n+k}, S^{n+k} - M)$. The
 groups $H_{n+k} (S^{n+k})$ and\par}  

\begin{landscape}
\vbox to\textheight{\vfill%
\[
\xymatrix{
 & H_{n+k} (S^{n+k}, S^{n+k} -x) & & \\
H_{n+k} (S^{n+k}) \ar[r]_>>>>>{j_*} \ar[ur]^{(i_x)_*} \ar[d]_{C_*} &
H_{n+k} (S^{n+k}, S^{n+k} - M) \ar[u]_{(j_x)_*}
\ar[d]^{C_*}_{\text{\ding{172}}} & & 
H_{n+k} (A_{\frac{1}{2}} , A_{\frac{1}{2}} - M) \ar[d]_{(e^{-1})_*}
\ar[ll]_{\rm Excision}^{\approx}\\
H_{n+k} (T(\nu)) \ar[r]^>>>>>{\approx} & H_{n+k} (T (\nu), T(\nu) - M) &
H_{n+k} (\Gamma, \Gamma - M) \ar[l]_>>>>>{\approx}^>>>>>{\rm Excision} &H_{n+k}
(E_{\frac{\varepsilon}{2}}, E_{\frac{\varepsilon}{2}} - M )
\ar[l]_{\approx}^{\beta_*}
}
\]\vfill}
\end{landscape}
\noindent
$H_{n+k} (S^{n+k}, S^{n+k}-M)$ being
 both isomorphic to $\mathbb{Z}$ it follows that $j_*$ is an
 isomorphism. It now follows that $C_* : H_{n+k} (S^{n+k} \to H_{n+k}
 (T (\nu))$ is an isomorphism.

\end{proof}

\setcounter{subsection}{5}
 \subsection{The index of A 4d-dimensional
   manifold}\label{chap1:subsec1.6}%%% 1.6 
 
 Let $M$ be a compact, connected, oriented manifold of dimensional $4d$
 with $d$ an integer $\geq 0$ and let $[M]$ be the fundamental class of
 $M$. The image $h_*([M])$ of the fundamental class of $M$ under the
 inclusion $h:\mathbb{Z} \to \mathbb{Q}$ is called the fundamental
 class with coefficients in $\mathbb{Q}$ and is also denoted by
 $[M]$. The $\map (x, y) \rightsquigarrow (x \cup y)[M]$ of $H^{2d} (M,
 \mathbb{Q}) \times H^{2d} (M, \mathbb{Q}) \to \mathbb{Q}$ gives a
 symmetric, non degenerate bilinear form $H^{2d} (M,
 \mathbb{Q})$. Symmetry is clear from $x \cup y=(-1)^{2d \cdot 2d}  y \cup x =
 y \cup x$. That it is non degenerate is a consequence of Poinecare
 duality together with the fact that $(a, u) \rightsquigarrow a(u)$ is a
 bilinear non degenerate pairing of $H^{2d} (M, \mathbb{Q}) \times
 H_{2d} (M, \mathbb{Q}) \to \mathbb{Q}$. This latter fact is
 embodied in the  universal\pageoriginale coefficient theorem $H^{2d} (M,
 \mathbb{Q}) = \hom_{\mathbb{Q}} (H_{2d} (M, \mathbb{Q}),
 \mathbb{Q})$. The signature (i.e. the number of $+ve$ diagonal
 elements minus the number of $-ve$ general elements when diagonalised
 over $\mathbb{Q}$) of the bilinear form $(x, y) \rightsquigarrow (x
 \cup y) [M]$ on $H^{2d} (M, \mathbb{Q})$ is defined to be the index
 of $M$ and is denoted by $I(M)$. 
 
 In case $M$ is also differentiable we have the following Theorem of
 Hirzebruch's \cite{c1:key1}. 
 
\setcounter{theorem}{6}
 \begin{theorem}%1.7.
Let $L_k (p_1, \ldots,p_k)$ be the multiplicative sequence of
polynomials corresponding to the power series 
$$
\frac{\sqrt{t}}{\tanh \sqrt{t}} = 1 +\frac{1}{3} t- \frac{1}{45} t^2 +
\dots + (-1)^{k-1} \frac{2^{2k}}{(2k)!} B_k t^k + \cdots 
$$
(Here $B_k$ is the $k^{\rm th}$ Bernouilli number). Then the index $I(M)$
is equal to the $L$-genus of $M$ defined as $\bigg\{L_d(p_1 (\tau_M),
\ldots, p_d (\tau_M) \bigg\} ([M])$, where $p_i (\tau_M)$ is the
$i^{\rm th}$ Pontrjagin class of $\tau_M$.  
  
  For more information about the formalism of multiplicative sequen\-ces
  and the correspondence between power series and multiplicative
  sequence the reader is referred to \cite{c1:key1}, \cite{c1:key5}. 
  
  We just\pageoriginale content ourselves with the remark that $L_k (p_1,
  \ldots,p_k) $ are universally defined polynomials (i.e. independent
  of $M$) with coefficients in the indeterminates $p_1, p_2, \ldots
  $. The total weight of each term of $L_k(p_1,\ldots p_k)$ is $4k$ when
  $p_j$ is alloted the weight $4j$. The  first two of these
  polynomials are $L_1(p_1) = \dfrac{1}{3}p_1$; $L_2(p_1, p_2) =
  \dfrac{1}{45}(7 p_2-p^2_1)$.  
 \end{theorem}

\setcounter{subsection}{7}
\subsection{}\label{chap1:subsec1.8}%sec 1.8
\; We will be mainly concerned with a space $X$ which is a finite
simplicial complex. Given any vector bundle $\xi^k$ over $X$ there
exists a vector bundle $\eta$ over $X$ with $\xi \oplus \eta \simeq
\mathscr{O}_X $ (of some rank). In fact $\exists$ a $\map f : X \to
G_{k+\ell, k}$ (the Grassmann manifold of $k$-planes in
$\mathbb{R}^{k+\ell}$) for some $\ell$ such that $f! (\gamma^k) =
\xi$. Here $\gamma^k$ is the universal bundle on $G_{k+\ell, k}$. The
space $E(\gamma^k)$ is the subspace of $G_{k+\ell, k}
\times \mathbb{R}^{k+\ell}$ consisting of elements $(y, v^{\to})$ with 
$\overrightarrow{v}\in y$. Let $\tilde{\gamma}^\ell$ be the
vector bundle on 
$G_{k+\ell, k}$ consisting of elements $(y, \overrightarrow{w})$ with
$\overrightarrow{w} 
\in \mathbb{R}^{k+\ell}$ orthogonal to $y$. Then $\eta = f!
(\tilde{\gamma}^\ell)$ satisfies $\xi \oplus \eta \simeq
\mathscr{O}^{k+\ell}_x$. Two vector bundles $\xi$ and $\xi'$ over $X$ are
said to be stably equivalent if $\xi \oplus \mathscr{O}^\ell_X \simeq
\xi' \oplus \mathscr{O}^{\ell'}_X$ for some $\ell$ and $\ell'$. The stable
class of $\xi$ is denoted by $[\xi]$. If $\xi$ and $\xi'$ are stably
equivalent and if $\eta$ and $\eta'$ are such that $\xi \oplus \eta
\simeq \mathscr{O}^n$ and $\xi' \oplus \eta' \simeq \mathscr{O}^{n'}$
for some $n$ and $n'$ it is easy to see that $\eta$ and\pageoriginale
$\eta'$ are stably equivalent. The class of $\eta$ is denoted
by$-[\xi]$. It is known that the Pontrjagin classes of a vector bundle 
depend only on the stable class of the bundle. If $\bar{p}_1 (\xi)$,
$\bar{p}_2 (\xi),\ldots$ denote the Pontrjagin classes of some $\eta$
belonging to the class $-[\xi]$ it follows that the elements $L_k$,
$(\bar{p}_1,(\xi),\ldots, \bar{p}_k (\xi))$ depend only on the class
$[\xi]$ of $\xi$.  

 Referring to the situation where $M^{4d}$ is differentiably imbedded
 in $\mathbb{R}^{4d+k}$ with normal bundle $\nu$ we see that $L_k$,
 $(\bar{p}_1 (\nu), \ldots, \bar{p}_K (\nu )) = L_k$, $(p_1
 (\zeta_M), \ldots, p_k (\zeta_M)) \in H^{4k'} (M,
 \mathbb{Q})$. Thus Hirzebruch's theorem can be rephrased in terms of
 the normal bundle $\nu$ as $\big\{L_d (\bar{p}_1 (\nu), \ldots,
 \bar{p}_d (\nu))\big\}\break ([M]) = I (M)$. 
 

 \section{The main Theorem}\label{chap1:sec2}%%% 2
 Let $X$ be a connected finite simplicial complex with $\prod_1 (X)
 =0$. The theorem of Browder and Novikov deals with conditions under
 which $X$ will be of the same homotopy type as a compact
 differentiable manifold $M$ without boundary. Since $X$ is simply
 connected if such an $M$ exists it has to be orientable. We first
 state the theorem, which actually consists of two parts.  
 
 \begin{theorem}\label{chap1:thm2.1}%the 2.1
Let $X$ be a connected finite simplicial complex with\break $\prod_1
(X) = 0$. Suppose that the following two conditions are satisfied.  
  \begin{enumerate}[i)]
 \item $X$\pageoriginale satisfies Poincar\'e duality i.e. to say
   $\exists$  some 
   integer $n$ with\break $H_n(X)\simeq \mathbb{Z}$ and if $u$ is a
   generator, $\bigcap u : H^q (X) \to H_{n-q} (X)$ is an isomorphism
   for all $q$. 
 
 \item $\exists$ an oriented vector bundle $\xi^k$ over $X$ such that
   $\Phi (u) \in H_{n+k} (T (\xi))$ is spherical, $\Phi: H_n (X) \to
   H_{n+k} (T(\xi))$ being the Thom isomorphism. 
  \end{enumerate}  
 \end{theorem}

  Then if $n$ is odd $X$ is of the same homotopy type as a compact
  differentiable manifold $M$ of dimension $n$ under a homotopy
  equivalence $f: M\to X$ satisfying $[f ! (\xi)] = - [\tau_M]$. 
  
  The second part of the theorem is concerned with the case $n
  =4d$ with $d$ an integer $> 1$.  
  
  $X$ being a finite complex we have $H^q (X, \mathbb{Q})=H^{q}(X)
  \otimes \mathbb{Q}$ and $H_{i}(X,\break \mathbb{Q}) =H_{i}(X) \otimes
  \mathbb{Q}$. Denoting the image of $u$ in $H_{n}(X, \mathbb{Q})$
  under $h_{*}: H_{n}(X) \to H_{n}(X, \mathbb{Q})$ where
  $h:\mathbb{Z}\to \mathbb{Q}$ is the inclusion of $\mathbb{Z}$ into
  $\mathbb{Q}$ by $v$ we have $\cap v : H^{q}(X, \mathbb{Q})\to
  H_{n-q}(X, \mathbb{Q})$ an isomorphism for all $q$. Actually $\bigcap
  v$ can be identified with $(\bigcap u )\otimes \mathbb{Q}$. Thus
  assumption $i$) actually implies Poincare duality for coefficients
  in $\mathbb{Q}$. Actually, it is true that assumption $i$) implies
  Poincare duality for any arbitrary commutative coefficient ring
  $\wedge$ (with $1 \neq 0$). The procedure adopted to define the
  index $I(M^{4d})$ in \S \ref{chap1:subsec1.6} can now be used to
  define the index   $I(X)$ of $X$.   
  
Assume\pageoriginale in addition to i) and ii) we have the following
valid for $\xi$.

      iii)  $I(X) = \bigg\{ L_d (\bar{p}_1 (\xi),\ldots, \bar{p}_d (\xi)) 
  \bigg\} (v)$. 
  
  Then $X$ is of the same homotopy type as a compact differentiable
  manifold $M$ of dimension $4d$ under an equivalence $f : M \to X$
  satisfying $[f ! (\xi)] = - [\tau_M]$.  
  
  Part $I$ of these lectures is devoted to the proof of this
  theorem. From \S \ref{chap1:sec1} at actually follows that the conditions $i)$,
  $ii)$, and $iii)$ when $n=4d$, are necessary for the validity of the
  Theorem. 
  
  From the assumption $\prod_1 (X)=0$ it follows that the integer $n$
  satisfying condition $i)$ of Theorem \ref{chap1:thm2.1} has to be $\geq 3$
  whenever $n$ is odd. But for $n=3$ the condition $i)$ itself implies
  that $X$ is of the same homotopy type as $S^3$. Moreover every
  vector bundle on $S^3$ is trivial since $\prod_2 (So(k))=0$ for
  every integer $k \geq 0$. Thus for any vector bundle $\xi$ over $X$
  and any homotopy equivalence $f : S^3 \to X$ we have  $[f! (\xi)]
  =-[\tau_{S^3}]$. This shows that Theorem \ref{chap1:thm2.1} is trivially valid
  for $n=3$ and hence it only remains to prove the Theorem for $n \geq
  5$. But some of the Lemmas and propositions that will be proved here
  are valid for $n \geq 4$, and it will be clear later when exactly we
  need the assumption $n > 4$. 
  
\setcounter{subsection}{1}
  \subsection{}%sec 2.2 
  Realizing\pageoriginale $X$ as a subcomplex of a simplex $\Delta_N$
  for some 
  integer $N$ and imbedding $\Delta_N$ affinely in $\mathbb{R}^N$ we
  get an open set $U \supset X$ of $\mathbb{R}^N$ such that $X$ is
  a deformation retract of $U$. Let $j : X \to U$ be the inclusion and
  $r : U \to X$ the retraction (i.e. $roj=Id_x$) with $jor \sim
  Id_U$ ($\sim =$ `homomorphic to'). Let $\xi$ be a vector bundle on $X$
  satisfying condition $ii)$ of Theorem \ref{chap1:thm2.1}. Let $\xi' = r! (\xi)$. It
  is easy to see that $\xi'$   can be made into  
a differentiable vector bundle. Actually $\xi'$ is induced by a
certain map $g : U\to G_{k+\ell, k}$ for some integer $\ell$, form the
universal bundle $\gamma^k$ on $G_{k+\ell, k}$. Since the map $g$ can
be approximated by a differentiable map $g : U\to G_{k+\ell, k}$ with
$g \sim g'$, it follows that $\xi'$ can be made into a differentiable
vector bundle. The Thom space $T(\xi')$ of $\xi'$ is defined as
follows. Introducing a fixed $C^\infty$ Riemannian matric on $\xi'$,
let $E_1(\xi')$ be the subspace of $E(\xi')$ consisting of vectors of
length $\leq 1$ and $\dot{E}_1 (\xi')$ the boundary of $E_1(\xi')$
consisting precisely of vectors of length 1. The space $T(\xi')$ is
defined as the quotient space $E_1 (\xi')/ \dot{E}_1 (\xi')$. In this
case $T(\xi')$ is not the one point compactification of $E
(\xi')$. Still we denote the point of $T(\xi')$ to which $\dot{E}_1
(\xi')$ is collapsed by $''\infty''$. Clearly $T(\xi')-\infty$ is a
differentiable manifold.    
   
   Since $roj = Id_X$ we have $\xi' / X=\xi$. Taking the restriction
   to $\xi$ of the Riemannian metric on $\xi'$, and realizing $T(\xi)$
   as\pageoriginale $E_1 (\xi) / \dot{E}_1 (\xi)$ we see that the
   inclusion map $h : 
   E (\xi) \to E (\xi')$ induces a map $T(h) : T (\xi) \to T
   (\xi')$. The symbol $\Phi$ denotes throughout the Thom
   isomorphism. Let  $f : S^{n+k} \to T(\nu)$ be a map such that
   $f^*(\iota) = \phi (u)$, $\iota$  being a generator of $H_{n+k}
   (S^{n+k})$. By condition ii) such a map exists. The naturality of
   the Thom isomorphism yields $(T(h) of)_* (\iota) = \Phi (j_*
   (u))$. Denoting $T(h) o f$ by $f'$ we see that $f' : S^{n+k} \to T
   (\xi')$ is a map satisfying $f'_* (\iota) = \Phi (j_*(u))$. By the
   transverse regular approximation theorem \cite{c1:key4}, $\exists$ a
   differentiable map $f'' : S^{n+k}\to T(\xi')$ (whenever it makes
   sense i.e. on $f''^{-1}(T(\xi') - \infty)$) with $f'' \sim f'$ and
   $f''$ transverse regular on $U$. Clearly $f''^{-1} (U) \neq \emptyset$
   for if $f''(S^{n+k}) \cap U = \emptyset$ the map $f''_* : H_{n+k}
   (S^{n+k}) \to H_{n+k} (T (\xi'))$ would factor through $H_{n+k} (T (\xi') -
   U)=0$ (since $T (\xi')-U$ is contractible to ``$\infty$''). But
   $f''_* (\iota) = f'_*(\iota) = \Phi (j_* (u)) \neq 0$. Hence
   $M=f''^{-1}(U)$ is a differentiable manifold of codimension $k$
   in $S^{n+k}$ with normal bundle $\nu _M \simeq f'' ! (\xi')$. But $M$
   need not necessarily be connected. Since $f''(\xi')$ and
   $\tau_{S^{n+k}}$ are orientable and since $\tau_{S^{n+k}} \Big| M\simeq
   \tau_M \oplus f'' ! (\xi')$ we see that $\tau_M$ is
   orientable. Since $U$ is closed in $T(\xi)$ we have $M = f''^{-1}
   (U)$ closed in $S^{n+k}$ and hence $M$ is a compact, orientable
   differentiable manifold of dimensional $n$. Choose some $C^\infty$
   Riemannian  metric for\pageoriginale $\nu_M$. It is known that
   $\exists$ a tubular neighbourhood i.e. a diffeomorphism $D$ of
   $E_\varepsilon (\nu)$ for some $\varepsilon>0$ onto a closed neighbourhood $B$
   of $M$ in $S^{n+k}$, and map $\bar{f}: S^{n+k} \to T(\xi')$ satisfying
   the following conditions: 
\begin{enumerate}[1)]
\item $\bar{f}$ is differentiable on $\bar{f}^{-1}(T(\xi')-\infty)$ and
  transverse regular on $U$ 

\item $\bar{f}=f''$ on $M$ and $\bar{f}^{-1}(U)=f''^{-1}(U)=M$ 

\item $\bar{f} o D$ is a bundle map of $E_\varepsilon (\nu)$ onto the
  image (i.e. maps the fibre of $E_\varepsilon (\nu)$ at $x \in M$
  homeomorphically onto the image portion of the fibre at $f(x)$ in
  $E(\xi)$) 

\item $\bar{f}\sim f'' : S^{n+k}\to T (\xi')$. 
 \end{enumerate} 
     
For a proof refer to steps 1 and 2 of the proof of Theorem 3.16
in \cite{c1:key4}. 
    
From the compactness of $M$ it follows that $\exists$ a $\delta > 0$
with $\bar{f} o D\break(E_\varepsilon(\nu))\supset E_\delta (\xi')
\bigg|\bar{f}(M)$. Let $\{M_i\}_{i=1, , r}$ be the connected
components of $M$ and let $A_i=\bar{f}^{-1}(E_\delta(\xi'))\bigg|M_i$
and $\dot{A}_i = \bar{f}^{-1}(\dot{E}_\delta(\xi'))\bigg|M_i$. We will
write the same symbols $A_i$, $\dot{A_i}$ to denote $D^{-1}(Ai)$,
$D^{-1}(Ai)$ etc. In otherwords we identify $E_\varepsilon(\nu)$ and the
tubular neighbourhood $B$. 
     
We now introduce the following changes in notation. We write $\xi$,
$f$ and $u$ for $\xi$, $\bar{f}$ and $j_*(u)$. With this altered
notation $f: S^{n+k}\to T(\xi)$ is a map satisfying $\Phi (u) =
f_*(\iota)$, differentiable on $f^{-1}(T(\xi)-\infty)$, transverse
regular on $U$ and is\pageoriginale also a bundle map covering 
$f\bigg|M : M \to U$ on a tubular neighbourhood of $M$ in $S^{n+k}$.  
     
 \subsection{}%sec 2.3 
 We choose $\iota$ as the fundamental class
 $\left[S^{n+k}\right]$. Then each $(A_i, \dot{A_i})$ receives the
 induced orientation $\left[A_i, \dot{A_i}\right]$. Denoting by
 $\nu_i$ the restriction of $\nu$ to $M_i$ and by $\Phi_i :
 H_n(M_i)\to H_{n+k}(T(\nu_i))$ the Thom isomorphism, let $\psi_i:H_n
 (M_i)\to H_n(A_i,A_i -M_i)$ be the unique isomorphism making the
 following diagram commutative. 
\[
\xymatrix{
H_n (M_i) \ar[r]^{\Phi_i} \ar[d]_{\psi_i} & H_{n+k} (T (\nu_i))
\ar[d]^{\rm(incln)_*}_\approx\\
H_{n+k} (A_i, A_i - M_i) \ar[r]^>>>>>{\approx}_>>>>{\rm Excision} & H_{n+k}
(T(\nu_i), T (\nu_i) - M_i)
}
\]     
     
The homomorphisms $(j_i)_* : H_{n+k}(A_i, \dot{A}_i)\to H_{n+k}(A_i,
A_i - M_i)$ induced by inclusions are isomorphisms (since $\dot{A}_i$ is a
deformation retract of $A_i-M_i$). We choose orientations
$\left[M_i\right]$ for $M_i$ by the requirement that
$\psi_i(\left[M_i\right])=(j_i)_*\left(\left[A_i, \dot{A}_i\right]
\right)$ 
     
\setcounter{lemma}{3}
\begin{lemma}\label{chap1:lem2.4}%2.4
The map $f:M\to U$ is of degree 1 i.e. to say
$f_*(\left[M\right])=u$ with $\left[ M \right ] = \sum [M_i]$. 
\end{lemma} 
         
\begin{proof}
Let\pageoriginale $\psi : H_n (U)\to H_{n+k}(E_\delta(\xi),
E_\delta(\xi)-U)$ be the isomorphism making the square.  
\end{proof}             

\[
\xymatrix{
H_n(U) \ar[r]^>>>>>>{\psi} \ar[d]_{\Phi} & H_{n+k} (E_\delta (\xi), E_\delta
(\xi) - U)  \ar[d]^{\rm excision}_{\approx} \\
H_{n+k} (T(\xi)) \ar[r]^>>>>>>{\approx} & H_{n+k} (T (\xi), T (\xi) - U)
}
\]

\noindent
commutative. Naturality of the Thom isomorphism together with the fact
that $f\big|B$is a ``bundle map'' yield the following commutative
\hbox{diagram.}

\begin{landscape}
\[
\xymatrix{
& H_{n+k} (S^{n+k}) \ar[r]^{f_*} \ar[d]^{j_*} & H_{n+k} (T (\xi))
  \ar[d]^{(j_{\xi})_*}\\
& H_{n+k} (S^{n+k}, S^{n+k} - M) \ar[r]^{f_*}  & H_{n+k}(T(\xi),
  T(\xi)- U) \\
\bigoplus\limits^r_{i=1} H_{n+k} (A_i, \dot{A}_i)
\ar[r]^>>>>>>{(\oplus j_i)_*} & 
\bigoplus\limits^r_{i=1}  H_{n+k} (A_i, A_i - M_i) = H_{n+k}
(\bigcup\limits^r_{i=1} A_i, \bigcup\limits^r_{i=1} (A_i - M_i))
\ar[u]_{e_*(\rm excision)}
\ar[r]^>>>>>{f_*} & H_{n+k} (E_\delta (\xi), E_\delta (\xi) - U)
\ar[u]^{\approx}_{(e_\xi)_* (\rm excis)}\\
& \bigoplus\limits^r_{i=1} H_n (M_i) \ar[u]_{\oplus \psi_i}
\ar[r]^{f_*} & H_n(U) \ar[u]_{\psi}
}
\]
\begin{center}
{\bf Diagram 2.}
\end{center}
\end{landscape}

 

Let $f_* [M] = du$.\pageoriginale We have to show that $d=1$. We have
$(e^{-1}_*)j_*\break [S^{n+k}] = \sum\limits_{i} (j_i)_* (\left[A_i,
  \dot{A}_i\right])$. To show that $d=1$ it suffices to show that
$\psi f_* [M] = \psi (u)$. From Diagram 2 we  have
\begin{align*}
\psi f_* [M] & = f_* \left(\sum \psi_i [M_i]\right) = f_*
\left(\sum(j_i)_* ([A_i, \dot{A}_i])\right)\\ 
& = f_* (e_*)^{-1} j_* [S^{n+k}] = (e_{\xi_*})^{-1} (j_\xi)_* (f_*
     [S^{n+k}])\\ 
& = \left(e_{\xi_*}\right) ^{-1}  \left(j_\xi\right)_*
     \left(\Phi(u)\right). 
\end{align*}                 
But by definition of $\psi$ we have $\psi(u)= (e_{\xi_*})^{-1}
(j_\xi)_* \Phi (u)$. 

We change our notations again and write $f : M \to X$ for the map of
$rof$ where $r : U \to X$ is the homotopy equivalence chosen already and
write $u$ for the original generator of $H_n (X)$. Then $f$ is of
degree $1$. The homomorphism $H_q (M) \to H_q (X)$ induced by $f$ is
denoted by $f_q$. 

\begin{lemma}\label{chap1:lem2.5}%2.5.
There exist homomorphism $g_q : H_q (X) \to H_q (M)$ with $f_q o g_q =
Id_{H_q(X)}$ and hence $H_q (M)= $ Kar $f_q \oplus g_q (H_q (X))$.  
\end{lemma}

\begin{proof}
For any $x \in H_q (X)$ let $\gamma \in H^{n-q} (X)$ be the
element $\Delta^{-1} (x)$ where $\Delta : H^{n-q} (X) \to H_q (X)$ is
the Poincare isomorphism. Setting $g_q (x) = f^* (\gamma) \cap [M]$
we have $f_q g_q (x) = f_* (f^* (\gamma) \cap [M]) = \gamma \cap f_*
[M] = \gamma \cap u = x$.  
\end{proof}

The proof of this lemma uses only two facts $: \textcircled{a} X$
satisfies Poincare\pageoriginale duality and $\textcircled{b} f : M
\rightarrow X $ is a map of degree 1.   

Let $\eta'$ be a bundle over $X$ (of rank $\ell'$ say) such that $\xi
\otimes \eta' \simeq \mathscr{O}^{k + \ell'}_X$. Let $\eta = \eta' \otimes
\mathscr{O}_X^{k + n}$. Then $ [\eta] = [\eta'] = - [\xi]$ and  
\begin{gather*}
f!(\eta) = f! (\eta') \oplus \mathscr{O}^{k+n}_M \simeq f! (\eta') \oplus
\tau^{n}_M + \nu^k_M \simeq \oplus f! (\eta') \oplus f! (\xi)\\
 \simeq\tau^{n}_M \oplus f!  (\eta' \oplus \xi ) \simeq \tau^{n}_M
 \oplus \mathscr{O}^{k+\ell'}_M.  
\end{gather*}

Denoting $k+ \ell'$ by $\ell$ we have the following situation: 
 $\exists$ a vector bundle $\eta$ of rank $n + \ell$ on $X$ with
$[\eta] = - [\xi]$ and a map $f:M \to X$ of degree 1 satisfying $f!
(\eta) \approx \tau^n_M \otimes \mathscr{O}^\ell_M$. Without loss of
generality we can assume $\ell' \geq 1$. Our aim is to surgerize $M$
finitely many times and obtain a connected simply connected manifold
$M'$ together with a map $f' : M' \to X$ inducing isomorphisms in
homology and further satisfying $f'! (\xi)\approx \tau^{n}_{M'} \oplus
\mathscr{O}^\ell_{M'}$. In this is done the theorem is proved since $f'$
will then be a homotopy equivalence by a theorem of J.H.C. Whitehead
and the relation $f' !(\xi) = \tau^n_{M'} \otimes \mathscr{O}^\ell_{M'}$ implies
$[f' !(\xi)] = - [\tau^n_{M'}]$. In case $n$ is odd and $\geq 5$ we will be
able to achieve this using conditions i) and ii) and when $n = 4d$
with $d$ an integer $> 1$ we will also need condition iii) to do the
same.  

\section{Surgery or Spherical modification}\label{chap1:sec3} %Section 3

The unit\pageoriginale $disk \left\{(x_1,\ldots,x_n)\in
\mathbb{R}^n \big| \sum\limits^{n}_{i = 1} x^2_i \leq 1\right\}$ in
$\mathbb{R}^n$ is denoted by $D^n$ and the unit open ball
$\left\{(x_1,\ldots, x_n) \in \mathbb{R}^n
\bigg|\sum\limits^n_{i =1}x^2_i<1\right \}$ by 
$B^n$. For any real number $t > 0$ the closed disk and the open ball
of radius $t$ are denoted by $tD^n$ and $tB^n$ respectively. All the
manifolds we consider are oriented $C^\infty$ manifolds. We use the
letter $V$ to denote a compact manifold without boundary, of dimension
$n \geq 1$.  

\begin{definition}% definition 3.1
Given an orientation preserving differentiable imbedding $\varphi: S^q
\times \dfrac{3}{2}D^{n-q} \rightarrow V$ with $n > q \geq 0$ let
$\chi (V, \varphi)$ denote the quotient manifold obtained from the
disjoint union $V-\varphi(S^q \times \dfrac{1}{2}
D^{n-q})U\dfrac{3}{2} B^{q+1}\times S^{n-q-1}$ by identifying
$\varphi(x, t, y)$ with $(tx, y) \forall x \in S^q$, $y \in
S^{n-q-1}$ and $\dfrac{1}{2} < t < 3/2$.  
\end{definition}

It is easy to check that $\chi(V,\varphi)$ is Hausdorff. Since $\varphi
(x, ty) \rightsquigarrow(tx, y)$ is a diffeomorphism for $x\in
S^q$, $y\in S^{n-q-1}$ and $\dfrac{1}{2}<t<3/2$ it follows that
$\chi (V, \varphi)$ is a $C^\infty$-manifold. It is clearly compact
and oriented. The manifold $\chi(V, \varphi)$ is said to be got from
$V$ by a surgery of type $(q+1, n-q)$.  

Two compact if oriented manifolds $V$ and $V'$ are said to be
$\chi$-equi\-val\-ent if $\exists$ a finite sequence of manifolds $V_1 =
V_1, V_2, \ldots, V_r = V'$ such that $V_{i+1}$ is got from $V_i$ by a
surgery.  

\setcounter{lemma}{1}
\begin{lemma}\label{chap1:lem3.2}% lemma 3.2
Suppose\pageoriginale $V$ has $s$ connected components with $s \geq
2$ and 
$\varphi: S^o \times D^n \rightarrow V$ an orientation preserving
imbedding which carries the two components of $ S^o \times D^n$
into distinct components of $V$. Then $\chi (V,\varphi)$ has exactly
$(s-1)$ connected components.  
\end{lemma}

\begin{proof}% proof
Trivial for $n\geq 2$. For $n = 1$ we have to use the fact that every
component of $V$ is diffeomorphic to $S'$.  

Using conditions i) and ii) of Theorem \ref{chap1:thm2.1}. we obtained a compact
oriented manifold $M$ of dimension $n$, a vector bundle $\eta$ of
rank $(n + \ell)$ on $X$ with $[\eta] = - [\xi]$ and a map $f: M
\rightarrow X$ of degree 1 satisfying $f!(\eta)\approx \tau^n_M
\oplus \mathscr{O}^\ell_M$. Let $\varphi: S^q\times \dfrac{3}{2}
D^{n-q}\rightarrow M$ be an orientation preserving imbedding with $n>
q \geq 0$. Assume further that $f \circ \varphi(S^q
\times\dfrac{3}{2}D^{n-q}) = x^*$, a chosen base point for $X$. Let
$M' = \chi(M, \varphi)$ and let $f' : M' \rightarrow X$ be defined as
follows. Setting $M_o = M - \varphi(S^q \times B^{n-q})$ the map $f'$ is
given by $f'  | M_o = f|M_o$ and $f'|\varphi ' (D^{q+1}\times
S^{n-q-1}) = x^*$ where $\varphi': D^{q+1}\times S^{n-q-1}\rightarrow
M'$ denotes the imbedding induced by the inclusion $D^{q+1} \times 
S^{n-q-1} \to \dfrac{3}{2} B^{q+1} \times S^{n-q-1}$. Clearly $f'$ is
well defined and continuous.   
\end{proof}

\begin{lemma}\label{chap1:lem3.3}% lemma 3.3
 The map $f' : M' \rightarrow X$ is of degree 1.
\end{lemma}

\begin{proof}% proof
Consider\pageoriginale the following commutative diagram.  
\[
\xymatrix@R=0.9cm@C=1.4cm{
H_n(M) \ar[r]^>>>>>>>>>>>{j_*} & H_n (M, \varphi (S^q \times D^{n-q}))
\ar[r]^>>>>>>>{f_*} & H_n(X,x^\ast) \\
& H_n (M_o, \varphi (S^q \times S^{n-q-1})) \ar[u]^{e_*}_\approx
\ar[ur]^{f_*} \ar[d]^{e'_*}_{\approx} &   \\
H_n(M') \ar[r]^>>>>>>>{j'_*} & H_n (M', \varphi' (D^{q+1} \times S^{n-q-1}))
\ar[uur]_{f'_*}  & 
}
\]
\begin{center}
{\bf Diagram 3.}
\end{center}

Here $j_*$, $j'_*$,  $e_*$ and $e'_*$ are homomorphisms induced by the
respective inclusions. The maps $e_*$ and $e'_*$ are isomorphisms by
excision and homotopy. That $f'$ is of degree 1 now follows from
$e'^{-1}_* j'_{*} [M'] = e^{-1}_* j_* [M]$.  

Suppose $M$ is not connected. Choosing $\varphi: S^o \times
\dfrac{3}{2}D^n$ such that the two components of $S^o \times
\dfrac{3}{2}D^n$ go into distinct components of $M$ let $M' =
\chi(M,\varphi)$. Since $X$ is connected it follows that $f \circ\varphi:
S^o \times \dfrac{3}{2} D^n \rightarrow X$ is homotopic to constant
map. By homotopy extension property we can choose a map $g: M
\rightarrow X$ with $g\sim f$ and $g|\varphi (S^o \times
\dfrac{3}{2}D^n) = x^*$. Then clearly $g$ is of degree 1 and $g!(\eta)
\approx \tau^n_M \oplus \mathscr{O}^\ell_M$. Thus we can without
loss\pageoriginale of 
generality assume that $f$ itself satisfies the condition $f \varphi
(S^o \times \dfrac{3}{2}D^n) = x^*$. Let $f': M' \rightarrow X$ be the
associated map i.e. $f' | M_o = f | M_o$ and $f'| \varphi'(D' \times
S^{n-1}) = x^*$. 
\end{proof}

\begin{lemma}\label{chap1:lem3.4}% lemma 3.4
$f': M' \rightarrow X$ is of degree 1 and $f'! (\eta) \approx
  \tau^n_{M'} \oplus \mathscr{O}^\ell_{M'}$.  
\end{lemma}

\begin{proof}
That $f'$ is of degree 1 follows from Lemma \ref{chap1:lem3.3} Let $T_M = \tau^n_M
\oplus \mathscr{O}^\ell_{M'}$ and $T_{M}, = \tau^{n}_{M'} \otimes
\mathscr{O}^\ell_{M'}$ and $\psi:T_M \rightarrow f! (\eta)$ a bundle
isomorphism. Our aim is to get a bundle isomorphism  
$\psi': T_{M'} \rightarrow f'!(\eta)$. Since $T_{M'} | M_o = T_M |
M_o$ and $f'| M_o = f | M_o$ we can take $\psi' = \psi$ on $T_{M'}|
M_o$. We denote the image of $S^o \times D^n$ by $\varphi$ in $M$ by $\im
\varphi$ and the image of $D^1 \times S^{n-1}$ under $\varphi'$ in $M'$
by $\im \varphi'$. We identify $T_{M'}| \im \varphi = \tau_\varphi, (D^1
\times S^{n-1})$ with $(\tau_{\frac{3}{2}B^1\times \frac{3}{2} B^2} \Big| D^1
\times S^{n-1})\oplus \mathscr{O}_{D'\times S^{n-1}}^{\ell-1}$. Let
$w_1,\ldots,w_{n+\ell}$ be a trivialization of $\tau_{\frac{3}{2}B^1
  \oplus \frac{3}{2}B^n}\oplus \mathscr{O}^{\ell-1}_{\frac{3}{2}B^1 \times
  \frac{3}{2}B^n}$ and take the induced trivialization of $T'_M| \im
\varphi'$ to identify it with  
$D^1 \times S^{n-1}\times \mathbb{R}^{n+\ell}$. Let $e_1, \ldots,
e_{n+\ell}$ be a basis of the fibre of $\eta$ at $x$ and let $u_1,
\ldots, u_{n+\ell}$ be the pull back trivialisation of $f'!(\eta)|\im
\varphi'$. Using this trivialization we identify $f'!(\eta)|\im
\varphi'$ with $D^1\times S^{n-1}\times \mathbb{R}^{n+\ell}$. The map
$\psi:T_{M'}|$ Bdry $M_o \rightarrow f'!(\eta)|$ Bdry $M_o$ then
corresponds to an orientation preserving bundle map $\psi: S^o \times
S^{n-1}\times \mathbb{R}^{n+\ell}\rightarrow S^o \times S^{n-1}\times
\mathbb{R}^{n+\ell}$ and thus to a continuous map\pageoriginale
$\Theta: S^o \times 
S^{n-1}\rightarrow GL_+(n + \ell, \mathbb{R})$ given by $\psi(x,
\overrightarrow{v}) = (x,\Theta (x) \overrightarrow{v}) \; \forall
\overrightarrow{v} \in \mathbb{R}^{n+\ell}$. To get a bundle map
$T_{M'}\rightarrow f'! (\eta) $ extending $\psi': T_{M'} \Big|M_o
\rightarrow f'! (\eta) \big|M_o $ it suffices to get a continuous
extension of $\Theta$ into a map $D^1 \times S^{n-1}\rightarrow GL_+
(n+\ell,\mathbb{R})$. But we know that $\psi$ comes from a bundle map
$T_M | \im \varphi \rightarrow f!(\eta)| \im \varphi$. Since $f|\varphi
(S^o \times D^n) = x^*$ the trivialization $u_1, \ldots, u_{n+\ell}$ of
$T'_M |$ Bdry $M_o = T_M |$ Bdry $M_o $ extends to a trivialization of
$f! (\eta)| \im \varphi$. Also $T_M|\im \varphi = \tau_{\varphi (S^o
  \times D^n)}\oplus \mathscr{O}^\ell_{\varphi (S^o \times D^n)}$ can be
identified with $\left(\tau_{\frac{3}{2} B^1 \times \frac{3}{2}B^n} \oplus
\mathscr{O}^{\ell-1}_{\frac{3}{2}B^1 \times \frac{3}{2}B^n} \right) \Big|S^o \times
D^n$. Thus the trivialization $w_1, \ldots, w_{m+\ell}$ extends to a
trivialization of $T_M|\im \varphi$. Using these trivializations we see
that $\psi$ corresponds to a bundle map $S^o \times D^n \times
\mathbb{R}^{n+\ell}\rightarrow S^o \times D^n
\times\mathbb{R}^{n+\ell}$. In otherwords $\exists$ an extension
$\bar{\Theta}$ of $\Theta$ into a map $S^o \times D^n \rightarrow GL_+
(n+\ell,\mathbb{R})$. Since $GL_+(n+\ell,\mathbb{R})$ is connected and
$D^n$ contractible it follows that $\exists$ a map $D^1 \times D^n
\rightarrow GL_+ (n+\ell, \mathbb{R})$ extending $\bar{\Theta}$. This
complete the proof of Lemma \ref{chap1:lem3.4}.  
\end{proof} 

As an immediate consequence of lemmas \ref{chap1:lem3.2} and
\ref{chap1:lem3.4} we get the  
following:  

\setcounter{prop}{4}
\begin{prop}\label{chap1:prop3.5}%proposition 3.5
There\pageoriginale exists a connected, compact, oriented $C^\infty$
manifold $M'$ which is $\chi$-equivalent to $M$ and a map $f':
M'\rightarrow X$ of degree 1 with $f'!(\eta)\approx T_{M'} =
\tau^n_{M'}\oplus \mathscr{O}^\ell_{M'}$.  
\end{prop}

We now change our notations. We replace $M'$ by $M$ and $f'$ by
$f$. Thus $M$ is connected and $f: M \rightarrow X$ is of degree 1
with $f!(\eta)\approx\tau^n_M \oplus \mathscr{O}^\ell_M$.  

Let $\varphi: S^q \times \dfrac{3}{2} D^{n-q}\rightarrow M$ be an
orientation preserving imbedding where $n > q \geq 1$ and let us
assume $f \varphi (S^q \times \dfrac{3}{2} D^{n-q}) = x^*$. Let $f': M'
= \chi (M,\varphi) \rightarrow X$ be the associated map. In general
$f' ! (\eta)$ need not be isomorphic to $\tau^n_{m}, \oplus
\mathscr{O}^\ell_{M'}$. Consider the following alteration of the map $\varphi
$. Let $\alpha : S^q \rightarrow So (n-q)$ be a $C^\infty$ map 
and let $\varphi_\alpha: S^q \times \dfrac{3}{2}D^{n-q}\rightarrow M$
be given by $\varphi_\alpha (x,y) = \varphi (x, \alpha (x) y)$ $\forall
(x,y) \in S^q \times \dfrac{3}{2}D^{n-q} $. Clearly
$\varphi_\alpha $ is an imbedding, also satisfying, $f \varphi_\alpha
(S^q \times \dfrac{3}{2}D^{n-q}) = x^*$. Let $f'_\alpha: M'_\alpha =
\chi (M, \varphi_\alpha) \rightarrow X$ be the associated map. The
sets $\varphi (S^q \times D^{n-q})$ and $\varphi' (D^{q+1}\times
S^{n-q-1})$ (and similarly $\varphi_\alpha (S^q \times D^{n-q}) $ and 
$\varphi'_\alpha (D^{q+1}\times S^{n-q-1})$) are denoted by $\im
\varphi$ and $\im \varphi'$ respectively (similarly by $\im 
\varphi_\alpha$ and $\im \varphi'_\alpha$ respectively). Let $\psi'$ be
defined to be $\psi$ on $T_{M'}| M_o = T_M| M_o $ into $f'! (\eta)
|M_o = f!(\eta) |M_o$. Let $e_1, \ldots, e_{n+\ell}$ be a fixed basis
of\pageoriginale the fibre of $\eta$ at $x^*$ and $u_1, \ldots,
u_{n+\ell}$ the pull 
back trivialization of $f! (\eta) | \im \varphi $. Then $v_i =
\psi^{-1}(u_i) $ constitute a trivialization of $T_{M'}|\im \varphi =
T_M |\im \varphi$ and there exists a bundle isomorphism
$T_{M'}\rightarrow f'! (\eta) $ extending $\psi'$ if and only if the
trivialization $v_1, \ldots,  v_{n+\ell}$ of $T_{M'} | Bdry M_o$ extends
to a trivialization of $T_{M'}|\im \varphi'$. We identify $T_{M'}|\im
\varphi'$ with 
$$
 \left( \tau_{\frac{3}{2} B^{q+1} \times \frac{3}{2}B^{n-q}}  \oplus
 \mathscr{O}^{\ell-1}_{ \frac{3}{2} B^{q+1} \times \frac{3}{2}
   B^{n-q}} \right) \bigg| D^{q+1} \times S^{n-q-1}. 
$$

 Let $w_1, \ldots,w_{n+\ell}$ be any trivialization of
 
$\mathscr{L}_{\frac{3}{2} B^{q+1} 
\times \frac{3}{2} B^{n-q}} = (\tau \oplus
 \mathscr{O}^{\ell-1})_{\frac{3}{2} B^{q+1} \times \frac{3}{2}
   B^{n-q}}$. Then we get a  
continuous map $\Theta: S^q \times S^{n-q-1} \to GL_{+} (n+\ell ,
\mathbb{R})$ given by $v (x, y) = \Theta (x, y) w (x, y)$ $\forall (x,
y) \in S^q \times S^{n-q-1}$. If there is an extension of
$\Theta$ into a continuous map $D^{q+1}\times S^{n-q-1}\rightarrow
GL_+ (n+\ell,\mathbb{R})$ then $v_1, \ldots, v_{n+\ell}$ can be extended
to a trivialization of $T_{M'} \big| \im \varphi'$. But since $T_M| \im
\varphi$ is identifiable with $(\tau \oplus
\mathscr{O}^{\ell-1})_{\frac{3}{2}B^{q+1}\times \frac{3}{2}B^{n-q}}$ we see
  that $ \Theta$ 
admits of an extension $\bar{\Theta}: S^q \times D^{n-q}\rightarrow
GL_+ (n+\ell, \mathbb{R})$. Hence $\Theta: S^q \times
S^{n-q-1}\rightarrow GL_+ (n+\ell, \mathbb{R})$ admits of an
extension $D^{q+1}\times S^{n-q-1}\rightarrow GL_+(n+\ell,\mathbb{R})$
whenever $\bar{\Theta}$ 
admits\pageoriginale of an extension $D^{q+1} \times
D^{n-q}\longrightarrow GL_+ 
(n+\ell, \mathbb{R})$. Choosing a fixed point $y_0 = S^{n-q-1}$ the
obstruction to the existence of such an extension is given by the
homotopy class of the map $\gamma: S^q \longrightarrow GL_+ (n+\gamma
, \mathbb{R})$ where $\gamma (x) = \Theta (x, y_0)$. Let us denote
this obstruction class by $\gamma (\varphi)\in \Pi_q (GL_+
(n+ \ell, \mathbb{R}))$. Let the obstruction class for the
imbedding $\varphi_\alpha $ be denoted by $\gamma(\varphi_\alpha)$.  

\setcounter{lemma}{5}
\begin{lemma}\label{chap1:lem3.6}% lemma 3.6
The obstruction $\gamma(\varphi_\alpha)$ depends only on $\gamma
(\varphi)$ and the homotopy class $(\alpha)$ of $\alpha$ in $\Pi_q (S
0 (n-q))$. More precisely identifying $\pi_q(SO (n-q))$ with
$\pi_q(GL_+ (n-q), \mathbb{R})$ we have $\gamma (\varphi_\alpha) =
\gamma (\varphi) + s_* (\alpha)$ where $s_*: \pi_q(GL_+ (n-q,
\mathbb{R})) \rightarrow \pi_q (GL_+ (n+\ell, \mathbb{R})) $is the map
induced by the inclusion $s: GL_+ ((n-q) , \mathbb{R}) \rightarrow
GL_+ (n+\ell,\mathbb{R})$.  
\end{lemma}

\begin{proof}
Suppose $\varepsilon_1, \ldots, \varepsilon_{n+\ell}$ is any
trivialisation of $T_{M'} \big|\im \varphi'$ and suppose $\lambda: S^q
\times S^{n-q-1}\rightarrow GL_+ (n+\ell,\mathbb{R})$ the map given by
$v(x,y) = \lambda (x, y) \in (x, y)$ $\forall (x, y) \in
S^q \times S^{n-q-1}$. Then $\exists$ a counts map $P: D^{q+1}\times
S^{n-q-1}\rightarrow GL_+ (n+\ell, \mathbb{R})$ such that $\Theta(x,
y) = \lambda(x,y) p (x,y)$. Actually $P$ is the transformation
relating the frame $\varepsilon(x, y)$ to $v'(x, y) $. Hence the
homotopy class of $A| S^q \times yo$ is the same as that of $\lambda
\big|S^q\times yo$. Now let $\Phi': D^{q+1}\times (D^{n-q} - \{0\} )
\rightarrow M' \times \mathbb{R}$ be the map given by $\Phi'(x,y) =
(\varphi' (x, \frac{y}{|| y ||}), ||y||-1)$. Choosing some
trivialisation\pageoriginale $C_0, C_1 \ldots, C_{\ell-1}$ of
$\mathscr{O}^\ell_{\im\varphi'}$ we see that  
$$
 \frac{\partial \Phi'}{\partial \xi} = \left( \dfrac{\partial
  \Phi'}{\partial x_1}, \ldots, \dfrac{\partial \Phi'}{\partial
  x_{q+1}}, \ldots, \dfrac{\partial \Phi'}{\partial y_{n-q}}, C_1,
\ldots , C_{\ell-1} \right)
 $$
 can be chosen as a trivialization for
$T_{M'} \big| \im \varphi '$. Thus the obstruction $\gamma (\varphi)$ is
the class of the continuous map $\gamma(x)$ given by $\gamma(x) =
\left\langle \frac{\partial \Phi'}{\partial\xi}, v \right\rangle
(x)$, the matrix of $v$ w.r.t the basis $\frac{\partial
  \Phi'}{\partial \xi}$. The obstruction $\gamma (\varphi_\alpha)$ is
the homotopy class of the map $\gamma_\alpha(x) = \left\langle
\frac{\partial \Phi'_\alpha}{\partial \xi}, v \right\rangle (x)$ where
$\Phi'_\alpha $ is defined similar to $\Phi'$ using $\varphi'$. It is
easily seen that we have $\frac{\partial\Phi' \alpha}{\partial x_i}
= \frac{\partial\Phi'}{\partial x_i} + \sum\limits_{k}\frac{\partial
  \Phi'}{\partial y_k} a_{ki}$ (for some $a_{ki}$) $\frac{\partial
  \Phi'_\alpha}{\partial y_j} = \sum\limits_{k}
\frac{\partial\Phi'}{\partial y_k}A_{kj}$ where $(A_{kj}(x)) = \alpha
(x)$. If, for every $0 \leq t \leq 1$ the frame
$\left(\frac{\partial\Phi'_\alpha }{\partial \xi }\right)_t$ is defined
by $\left(\frac{\partial\Phi'_\alpha }{\partial x_i }\right)_t =
\frac{\partial \Phi'}{\partial x_i}+ t
\sum\limits_{k}\frac{\partial\Phi'}{\partial y_k} \; a_{ki}(i =1,2,\ldots
q+1) $ 
$\left(\frac{\partial \Phi'_\alpha}{\partial y_j} \right)_t =
\frac{\partial \Phi'_\alpha }{\partial y_j}(j = 1,2,\ldots n-q) $ and  
$ (C_\mu )_t = C_\mu ( = 1, 2, \ldots \ell-1)$.  

We see that $\gamma^t_\alpha (x) = \left\langle
\left(\dfrac{\partial\Phi'_\alpha}{\partial \xi } \right)_t, v
\right\rangle (x)$ gives a homotopy 
between the map $\gamma^0 _\alpha (x) = \gamma (x) $. $s(x) $ where
$s: GL_+ (n-q, \mathbb{R}) \rightarrow GL_+ (n+\ell, \mathbb{R}) $ is
the inclusion and $\gamma^1_\alpha (x) =
\gamma_\alpha(x)$.\pageoriginale Thus the homotopy class
$[\gamma_\alpha]$ is the same as $[\gamma]+s_*(\alpha)$. Thus is to
say $\gamma (\varphi_\alpha) = \gamma (\varphi) + s_* (\alpha)$.   

Perhaps we should have remarked earlier that while dealing with
oriented bundles the trivializations are supposed to be those
belonging to the orientation class. Since $s_*: \prod_q (SO (n-q))
\rightarrow \prod_q (SO (n+\ell)) $ is surjective for $q < n-q$ we
have the following:  
\end{proof}

\setcounter{prop}{6}
\begin{prop}\label{chap1:prop3.7}% proposition 3.7
If $q< \dfrac{n}{2}$ $\exists$ a $C^\infty map $ $\alpha: S^q \rightarrow SO
(n-q)$ such that $f'_\alpha : M'_\alpha = \chi (M, \varphi_\alpha )
\rightarrow X $ satisfies $ f'_\alpha ! (\eta) \approx
\tau^n_{M'_\alpha }\oplus \mathscr{O}^\ell_{M'_\alpha}$.  

Let now $V$ be connected of dimension $n\leq 4$ and $v^*$ some chosen
base point in $V$. Choose some base point $P^*$ in $S^1$ and let
$\varphi: S^1 \times \dfrac{3}{2}D^{n-1}\rightarrow V$ be an
orientation preserving imbedding such that $\varphi(p^*,0) = v^* $ and
$\varphi \big| S^1 \times 0 $ represents $\lambda \in \prod_1
(V,v^*)$. Let $V' = \chi (V, \varphi)$ and let $V_\circ$ and $\varphi': D^2
\times S^{n-2}\rightarrow V'$ have their usual meanings i.e. $V_\circ = V
- \varphi (S^1 \times B^{n-1})$ and $\varphi'$ is the imbedding of
$D^2 \times S^{n-2}$ into $V'$ induced by the inclusion of $D^2 \times
S^{n-2}$ in $\dfrac{3}{2}B^2 \times S^{n-2}$. Choose some fixed
$z^*\in S^{n-2}$ and choose $v'^* = \varphi (p^*,z^*) =
\varphi' (p^* , z^*) $ as the base point of $V'$. Let $\sigma$ be the
path in $V$ given by $\sigma (t) = \varphi (p^*, tz^*)$; it is a path
joining\pageoriginale $v^*$ to $v'^*$ in $V$ and let $\sigma_*:
\prod_1 (V, v^*) \rightarrow \prod_1 (V, v'^*)$ be the isomorphism
induced by $\sigma$.   
\end{prop}

\setcounter{lemma}{7}
\begin{lemma}\label{chap1:lem3.8}% lemma 3.8
Let $N(\lambda)$ be the normal subgroup of $\prod_1 (V, {v'}^*)$
generated by $\sigma_* (\lambda)$. Then $\prod_1 (V', {v'}^*)$ is
isomorphic to $\prod_1 (V, v'^\ast)/ N(\lambda)$. 
\end{lemma}

\begin{proof}% proof
Let $j_*: (V_\circ, v'^*)\rightarrow (V, v'^* )$ be the inclusion. We claim
that $j_*: \prod_1 (V_\circ, v'^*) \rightarrow \prod_1(V, v'^*)$ is an
isomorphism.  
 
In fact if $\Theta: (S^1, p^*) \rightarrow (V, v'^*)$ is any map and 
$\bar{\Theta}: (S^1, p^* ) \rightarrow (V, v'^*)$ a map homotopic to
$\Theta$ and transverse regular on $\varphi (S^1 \times 0)$ (such a
map exists since $v'^* \notin \varphi (S^1 \times 0))$, since Codim
$\varphi (S^1 \times 0)$ in $V$ is $\geq$ 2 (actually Codim $\varphi
(S^1 \times 0)$ in $V \geq 3$). We see that $\bar{\Theta}(S^1) \cap
\varphi(S^1 \times 0) = \phi$. Choosing a deformation retraction $r:
S^1 \times (D^{n-1}-0 ) \rightarrow S^1\times S^{n-2}$ we see that $r'
= \varphi r \varphi ^{-1}: \varphi (S^1 \times (D^{n-1}-0))
\rightarrow \varphi (S^1 \times S^{n-2})$ is a deformation retraction
and that $r'\bar{\Theta}$ is a map  homotopic to $\bar{\Theta}$ and
satisfying $r' \bar{\Theta}(S^1) \subset V_\circ$. Thus $j_*$ is
onto. Also if $\psi : (S^1 ,p^*) \rightarrow (V_\circ, v'^*) $ is a map
such that $j \psi$ is homotopic to a constant map then $\exists$ an
extension (also denoted by $\psi$) of $\psi $ into a map $\psi: D^2
\rightarrow V $ with $\psi (0) = {v'}^*$. We can get a map
$\bar{\psi}$ with $\bar{\psi}| S^1 \cup 0 + \psi | s^1 \cup 0 $ and
$\bar{\psi}$ transverse regular on $\varphi (S^1\times 0 )$. Since
Codim of $\varphi (S^1 \times 0)$ in $V \geq 3$ we see that
$\bar{\psi}(D^2) \cup \varphi (S^1 \times 0 ) = \phi$ and an argument
similar to the one above\pageoriginale yields a homotopy of $\psi:
(S^1, p^*) \rightarrow (V_\circ , v^*) $ with the constant map, taking
place on $V_\circ$ itself. This show that $j_* $ is a monomorphism.   

 We have $V' = V_\circ \cup \im \varphi'$ (as usual $\im \varphi' =
 \varphi' (D^2 \times S^{n-2})$) with $V_\circ \cap \im \varphi' = 
 \varphi (S^1 \times S^{n-2}) = \varphi' (S^1 \times
 S^{n-2})$. Clearly $V_\circ$, $\im \varphi'$ and $V_\circ \cap \im \varphi'$
 are connected. Lemma \ref{chap1:lem3.8} follows immediately from Van Kampen
 theorem. also, clearly $V'$ is connected.  
 
 As already remarked earlier by us Theorem \ref{chap1:thm2.1} needs to be proved only
 when $n \geq 5$. We have already obtained a compact, connected,
 oriented $C^\infty$ manifold $M$ of dimension $n$ and a map $f: M
 \rightarrow X$ of degree 1 with $f! (\eta) \simeq \tau^n_M \oplus
 \mathscr{O}^\ell_M $. (Refer Proposition \ref{chap1:prop3.5}.)  
\end{proof}

\setcounter{prop}{8}
 \begin{prop}\label{chap1:prop3.9}% proposition 3.9
There exists a connected simply connected manifold $M'$ which is
$\chi$-equivalent to $M$ and map $f':M'\rightarrow X$ of degree
1 satisfying $f'! (\eta) \simeq \tau^n_{M'}\oplus \mathscr{O}^\ell_{M'}$.  
 \end{prop} 
 
 \begin{proof}
Choose some base point $m^* \in M$. We can without loss of
generality assume that $f(m^*) = x^*$ for otherwise we can change $f$
to a homotopic map satisfying this condition. Since $M$ is a compact
manifold $\prod_1 (M, m^*)$ is finitely generated. Let $\lambda_1,
\ldots, \lambda_r$ be generators for $\prod_1 (M, m^*)$. We can get
an imbedding $\varphi: S^1 \rightarrow M$ representing $\lambda_1$
(for this $n\geq 3$ is sufficient). Since $M$ is oriented the normal
bundle of $\varphi$ in $M$ is trivial and hence it can be
extended\pageoriginale 
into an orientation preserving diffeomorphism $\varphi: S^1 \times
\dfrac{3}{2} D^{n-1}\rightarrow M$. Since $X$ is simply connected we
have $f \circ \varphi$ homotopic to the constant map. By changing $f$ if
necessary to a homotopic map we can assume $f \varphi (S^1 \times
\dfrac{3}{2} D^{n-1}) = x^*$. Now let $M'_\varphi = \chi (M, \varphi)$
and $f'_\varphi: M'_\varphi \rightarrow X$ be the map associated to
$f$. By proposition \ref{chap1:prop3.7} $\exists$ a $C^\infty$ map $\alpha: S^1
\rightarrow SO(n-1)$ such that $f'_\alpha: M'_\alpha = M'_{\varphi_\alpha} =
\chi (M, \varphi_\alpha ) \rightarrow X$ satisfies $f'_\alpha !
(\eta) \simeq \tau^n_{M'_\alpha} \oplus \mathscr{O}^\ell_{M'\alpha}$ and is
of degree 1. The map $\varphi_\alpha | S^1 \times 0$ is the same as
$\varphi | S^1 \times 0 = \varphi: S^1 \rightarrow
M$. Hence $\varphi_\alpha \big |S^1$ represents the same element as
$\varphi$ i.e. $\lambda_1$. By Lemma \ref{chap1:lem3.8} it follows that $\prod_1
(M'_\alpha )$ is isomorphic to $\prod_1 (M) /$ (Normal $s\cdot g$
generated by $\lambda_1$) and hence $\prod_1 (M'_\alpha)$ is
generated by $(r-1)$ elements. It now follows that after a finite
number of surgeries we can get a connected, simply connected manifold
$M'$ and a map $f': M' \rightarrow X$ satisfying the requirements of
the proposition.  
 \end{proof} 
  
\begin{remark*}% remark
For applying lemma \ref{chap1:lem3.8} we only nee that $\dim M = n
\geq 4$. Moreover we have so far used only conditions i) and ii) of
Theorem \ref{chap1:thm2.1}.   
 \end{remark*} 

 \section{Effect of surgery on homology}\label{chap1:sec4}% section 4 
  Let $A$ and $B$ be any two connected, simply connected topological
 spaces and $q$ an integer $\geq$ 2. Suppose $h: A \rightarrow B$ is a
 continuous map such that $h_*: H_i (A) \rightarrow H_i (B)$ is an
 isomorphism for $i < q$ and an epimorphism for $i = q$. Denote the
 Kernel of $h_q: H_q (A) \rightarrow H_q (B)$ by $K_q$. 
 
 \begin{lemma}\label{chap1:lem4.1}% lemma 4.1 
 Any $x \in K_q$\pageoriginale can be represented by a map
 $\Theta: S^q \rightarrow A$ (i.e. $\Theta_* (i_q) = x $ where $i_q$
 is a generator of $H_q(S^q)$) with $h o \Theta$ homotopic to a
 constant map.  
 \end{lemma} 
 
 \begin{proof}% proof
Without loss of generality we can assume $h$ to be an inclusion map,
for otherwise, we replace $h$ by the inclusion of $A$ into the mapping
cylinder of $h$. For the proof of Lemma \ref{chap1:lem4.1} we use the Relative
Hurewicz Theorem. Since $h_* : H_i (A) \rightarrow H_i(B)$ is an
isomorphism for $i<q$ and an epimorphism for $i = q$ it follows from
the exact homology sequence of the pair $(B, A)$ that $H_i (B, A) = 0$
for $i \leq q$. Hence by the relative Hurewicz Theorem $\prod_i (B, A)
= 0$ for $i \leq q$ and $\rho: \prod_{q+1}(B, A)\xrightarrow{\approx}
H_{q+1}(B, A)$ where $\rho$ is the Hurewicz homomorphism. Now consider
the following diagram.  
\[
\xymatrix{
\prod_{q+1} (B,A) \ar[r]^\partial \ar[d]^\rho_{\approx} & \prod_q (A)
\ar[r]^{h_*} \ar[d]^\rho & \prod_q (B) \ar[r] \ar[d]^\rho &
\prod_q(B,A) = 0 \ar[d]^\rho\\
H_{q+1} (B,A) \ar[r]^\partial & H_q (A) \ar[r]^{h_*} & H_q (B) \ar[r] &
H_q (B,A) =0
 }
\] 
\begin{center}
{\bf Diagram 4}
\end{center}

The maps indicated by $\rho$ are the Hurewicz homomorphisms. If
$x\in K_q$ then $\exists$ $y \in H_{q+1}(B, A)$ such that
$\partial y = x$. 

Let\pageoriginale $y^1 \in \prod_{q+1} (B, A)$ be given by
$\rho^{-1}(y)$. The 
element $z\in \prod_q (A)$ given by $z = \partial y^1$ satisfies
$\rho(z) = x$ and $h_* (z) = h_* (\partial y^1) = 0$. Hence if $
\Theta: S^q \rightarrow A$ represents $z\in \prod_q (A)$ then
$\Theta$ satisfies the requirements of the Lemma.  
 \end{proof}

 \begin{lemma}\label{chap1:lem4.2}% lemma 4.2
Suppose $\nu$ is a vector bundle of rank $(n-q)$ over $S^q$ which is
stably trivial. If $2q < n$ then $\nu$ itself is trivial.  
 \end{lemma} 
 
 \begin{proof}% proof 
Let $\nu$ be determined by the element $\mu$ of $\prod_{q-1}(SO
(n-q))$. Stable triviality of $\nu$ implies that $\exists$ an integer
$r \geq n-q$ such that $s_* (\mu) = 0$ where $s_*: \prod_{q-1}(SO
(n-q)) \rightarrow \prod_{q-1}(SO(r))$ is the homomorphism induced by
the inclusion $SO(n-q) \rightarrow (SO(r))$. But if $2 q< n$ the map
$s_* $ is an isomorphism. Hence $\mu = 0$.  
 
 Let $V$ be a compact, connected, oriented $C^\infty$ manifold with
 $\prod_1(V) = 0$ of dimension $n$ and let $B$ be any connected,
 simply connected space. Let $h: V \rightarrow B$ be a continuous map
 with $h_*: H_i (V) \rightarrow H_i (B)$ an isomorphism for $ i < q$
 and an epimorphism for $i = q$ where $q \geq 2$. Further assume
 $\exists $ a vector bundle $\zeta$ on $B$ with $[h: (\zeta)] =
 [\zeta_V]$. Denote the Kernel of $h_q$ by $K_q$.  
 \end{proof}

 \begin{lemma}\label{chap1:lem4.3}% lemma 4.3
If $2q <n $ any $x \in K_q $ can be represented by a $C^\infty $
imbedding $\varphi : S^q \rightarrow V $ whose normal bundle
$\nu_\varphi $ is trivial and which further satisfies $h \circ \varphi
\sim$ constant map.  
 \end{lemma} 
 
 \begin{proof}
By Lemma \ref{chap1:lem4.1}\pageoriginale $\exists $ a map $\Theta :
S^q \rightarrow V$ 
representing $x$ such that $h \circ \Theta$ is homotopically trivial. If
$2 q< n$  $\exists$ a $C^\infty $ imbedding $\varphi : S^q \rightarrow V$
with $\Theta \sim \varphi$. We have $\tau_V | \varphi (S^q )\simeq
\tau_{\varphi (S^q)} \oplus \nu_\varphi$ where $\nu_\varphi$ is the
normal bundle of the imbedding $\varphi$. Since $\tau_{\varphi
  (S^q)} \oplus \mathscr{O}_{\varphi (S^q)}\simeq
\mathscr{O}^{q+1}_{\varphi(S^q)}$, we see that $[ \tau_V | \varphi (S^q) ] =
[\nu_\varphi]$. But
$[\tau_V | \varphi (S^q) ] = [h! (\zeta)|\varphi (S^q) ]$. Since $h \circ
\varphi $ is homotopically trivial by construction we see that
$\nu_\varphi $ is stably trivial. Now Lemma \ref{chap1:lem4.2} yields that
$\nu_\varphi $ itself is trivial.  
 
 Assume $2q<n$. Let $x \in K_q$ and let $\varphi :S^q
 \rightarrow V$ be a $C^\infty$ imbedding representing $x$. Since
 the normal bundle $\nu_\varphi$ is trivial we can extend $\varphi$
 into a orientation preserving imbedding $\varphi : S^q \times
 \dfrac{3}{2} D^{n-q}\rightarrow V$. Since $h \circ \varphi$ is homotopic
 to the constant map, changing $h$ in its homotopy class we may assume
 $h \circ \varphi = $ Const $b^*$. Let $V' = \chi (V, \varphi)$ and $h'
 :V' \rightarrow B$ the associated map i.e. to say $h' | V_\circ = h |
 V_\circ $ and $h'| \im \varphi' = b^*$ where $V_\circ, \im \varphi$ and
 $\im \varphi'$ have their customary meanings.  
 \end{proof}

\setcounter{prop}{3}
\begin{prop}\label{chap1:prop4.4}%Prop 4.4
$h'_* : H_i (V') \rightarrow H_i (B) $ is an isomorphism for $i<q$
  and the Kernel $K'_q$ of $h'_q = H_q(V') \rightarrow H_q (B)$ is
  isomorphic to $K_q| (x)$, whenever $2q< n-1$. 
\end{prop}

\begin{proof}
Consider\pageoriginale the following commutative diagram. 
\begin{landscape}
\vbox to\textheight{\vfill%
\[
\xymatrix{
H_i (S^q \times D^{n-q}) \ar[r]^>>>>>>>{\varphi_*} & H_i (V) \ar[r]^>>>>>>{j_*}
\ar[d]^{h_*} & H_i (V, \im \varphi) \ar[r]^>>>>>>>>{\partial} \ar[d]^e_\approx
& H_{i-1} (S^q \times D^{n-q})\\
& H_i (B) & H_i (Vo, Bdry Vo) \ar[l] & \\
H_i (D^{q+1} \times S^{n-q-1}) \ar[r] & H_i (V') \ar[r]_>>>>>{j'_*}
\ar[u]_{h'_*} & H_i (V', \im \varphi') \ar[r]^>>>>>>{\partial}
\ar[u]^\approx_{e'} & H_{i-1}(D^{q+1} \times S^{n-q-1})  
}
\]
\begin{center}
{\bf Diagram 5}
\end{center}
\vfill}
\end{landscape}

Since by assumption $2q<n-1$, whenever $1 \leq i \leq q $ we have $H_i
(S^q \times D^{n-q}) = 0 = H_i (D^{q+1}\times S^{n-q-1}) $ and hence  

$j_*^{-1} \circ e^{-1} \circ e' j'_* = H_i (V') \rightarrow H_i (V) $ will
then be an isomorphism satisfying commutativity in  
\[
\xymatrix{
H_i (V') \ar[rr]^\approx \ar[dr]_{h'} & & H_i (V) \ar[dl]^{h_*}\\
& H_i (B) & 
}
\]

This shows that $h'^* $ is an isomorphism for $i <
q$. When\pageoriginale $i=q$ Diagram 5 yields the following diagram.   
\[
\xymatrix{
\mathbb{Z}_1 \ar[r]^>>>>>{\varphi_*}_>>>>>{\rightsquigarrow x} & H_q (V)
\ar[r]^{j_*}\ar[d]^{h_q} & H_q (V, \im \varphi) \ar[r]
\ar[d]^e_{\approx} & 0\\
& H_q(B) & H_q (V_o, \text{ Bdry } V_o) & \\
0 \ar[r] & H_q(V') \ar[r]_>>>>>{j'_*} \ar[u]^{h'_q} & H_q (V', \im
\varphi') \ar[r] \ar[u]^{\approx}_{e'} & 0
}
\]
\begin{center}
{\bf Diagram 6.}
\end{center}

The map $\varphi_* $ is given by $\varphi_* (1) = x$. We get an
isomorphism of $H_q(V)) / (x)$ (induced by $j_*)$ with $H_q (V, \im
\varphi ) $ and then we see that $\exists $ an isomorphism $H_q (V) /
(x) \xrightarrow{\approx}H_q (V') $ making  
\[
\xymatrix{
H_q(V)/(x) \ar[rr]^\approx \ar[dr]_{\text{got from } \;\; h_*\;\; } & & H_q
(V') \ar[dl]^{\;\; h'_* \;\; 
  \text{ commutative}} \\
& H_q (B) & 
}
\]
 This proves that $K'_q \approx K_q / (x)$. 
 
 Assuming\pageoriginale conditions i) and ii) of Theorem \ref{chap1:thm2.1} with $n
 \geq 4$ we have 
 obtained a compact, connected oriented $C^\infty$ manifold $M$ of
 dimension $n$ with $\prod_1 (M) = $ and a map $f: M \rightarrow X$ of
 degree 1 satisfying $f! (\eta) \approx \tau^n_M \oplus \mathscr{O}^\ell_M$.  
\end{proof}

 \begin{prop}\label{chap1:prop4.5}%prop 4.5
There exists a connected, simply connected manifold $M'$ which is
$\chi$ equivalent to $M$ and a map $f' = M' \rightarrow X $ of degree
1 such that $f' ! (\eta) \approx \tau^n_{M'} \oplus \mathscr{O}^\ell_M$, and
$f'_* : H_i (M') \rightarrow H)_i (X) $ an isomorphism for $i <
\dfrac{n}{2}$.   
 \end{prop} 

 \begin{proof}
For $n =4$ there is nothing to prove for $f:M \rightarrow X$ already
satisfies the requirements of the proposition. Since $M$ is compact
the homology groups $H_i (M) $ are all finitely generated. For $ n
\geq 5$ Proposition \ref{chap1:prop4.5} is a consequence of this fact,
Lemma \ref{chap1:lem4.3} and 
Propositions \ref{chap1:prop4.4} and \ref{chap1:prop3.7}.  
 \end{proof} 

\setcounter{dashremark}{4}
\begin{dashremark}\label{chap1:dashrem4.5'}%Remk 4.5'
\; If $f'_{*} : H_q (M') \rightarrow H_q (X) $ also is an isomorphism for
$q = [\dfrac{n}{2}]$ then $f' :M' \rightarrow X $ will be a homotopy
equivalence. To show this we have only to show that $f'_{*}: H_i (M')
\rightarrow H_i (X) $ is an isomorphism for every $i$. As already
proved (Lemma \ref{chap1:lem2.5}) the fact that $f'$ is of degree 1 implies that
$f'_{*} : H_i (M') \rightarrow H_i (X)$ is onto for every $i$. Let $a
\in H_i (M')$ be such that $f'_*(a)=0 (i>q)$. Let
$\alpha=\triangle^{-1}(a) \in H^{n-1}(M')$. Since $i > q$ we have
$n-i \leq q$. Since $f'_*: H_j (M') \to H_j(X)$ is an isomorphism for $j
\leq q$\pageoriginale we have $f'^*: H^j (X) \to H^j (M')$ an isomorphism
for $j \leq q$ by the Universal Coefficient
Theorem. Hence $\alpha$ can be 
written as $f'^* (\beta)$ for a unique $\beta \in
H^{n-i}(X)$. Then if $x = \beta \cap u \in H_i (X)$ by the
definition of $g$ given in Lemma \ref{chap1:lem2.5}, we have $g(x) = a$. But $H_i
(M')= \ker f'_* \oplus g_i H_i (X)$ (direct sum). This implies $a = 0$
and hence $f'_*$ an isomorphism for all $i$. 

Let $A$ be any connected topological space satisfying Poincare duality
with $u \in H_n (A) \simeq \mathbb{Z}$ as the fundamental class. 
 \end{dashremark}

\setcounter{definition}{5}
\begin{definition} %definition 4.6
Let $a \in H_i (A)$ and $b \in H_{n-i} (A)$. The homology
intersection of $a$ and $b$, denoted by $a$. $b$ is defined as
follows: We identify $H_0 (A)$ with $\mathbb{Z}$ with any element
(i.e. pt) $w$ of $A$ as a generator. Let $\alpha = \Delta^{-1}(a)$ and
$\beta = \Delta^{-1}(b)$ where $\Delta$ is the Poincare
isomorphism. Then $\alpha \cup \beta \in H^n (A)$. The homology
intersection $a$. $b$ is that integer which satisfies $(\alpha \cup
\beta )\cap u = (a. b)w$. Because of (1) \S \ref{chap1:subsec1.2} we see that $a$. $b$
can also be defined as the value $(\alpha \cup \beta ) [u]$ of $\alpha
\cap \beta$ on the homology class $u$.  

Let $V$ be a compact, connected, simply connected $C^\infty$ manifold
of dimension $n \geq 4$ and let $q = [\dfrac{n}{2}]$. 
\end{definition}

\setcounter{lemma}{6}
\begin{lemma}\label{chap1:lem4.7}% lemma 4.7
Let $a \in H_q (V)$ and suppose $\exists$ $b \in H_{n-q} (V)$
such that $a . b = 1$. Suppose also that a is represented by an
imbedding $\phi: S^q \times \dfrac{3}{2} D^{n-q} \to V$
(i.e. $\varphi | S^q \times 0$ represents a).  

Let\pageoriginale $V' = \chi (V, \varphi)$. Then Rank $H_q (V') <$
Rank $H_q (V)$ and $H_i (V') \approx H_i (V)$ for $i < q$.   
\end{lemma}

\begin{proof}
Let $V_\circ$, $\im \varphi$ and $\im \varphi'$ have their customary
meanings. By excision and homotopy we have $H_i (V, V_\circ)
\xleftarrow[\approx] {\varphi_*} H_i (S^q \times D^{n-q}, S^q \times
S^{n-q})$. 

Also
\begin{equation*}
H_i (S^q \times D^{n-q}, S^q \times S^{n-q-1})=
\begin{cases}
\mathbb{Z} \text{ if } i = n-q \text{ or } n\\
0 \text{ otherwise }
\end{cases}
\end{equation*}
\end{proof}


From the homology exact sequence of the pair $(V, V_\circ)$ we see that
$H_i (V_\circ) \xrightarrow{(i_\circ)_*} H_i (V)$ is an isomorphism whenever
$i \neq n - q$ and $n$. (Here $i_\circ: V_\circ \to V$ denotes the
inclusion). Also we have the following exact sequence:   
$$
0 \to H_{n-q} (V_\circ) \to H_{n-q } (V) \xrightarrow{j_*} H_{n-q} (V, 
V_\circ) \simeq \mathbb{Z} \xrightarrow{\partial} H_{n-q-1} (V) \to \cdots 
$$

 The homomorphism $j_*$: $H_{n-q} (V) \to H_{n-q} (V, V_\circ)$ can more
 explicitly be described as follows. Identifying $H_{n-q} (V, V_\circ)$
 with $H_{n-q} (S^q \times D^{n-q}, S^q \times S^{n-q-1})$ we see that
 $\varphi (x_0 \times D^{n-q})$ with $x_0$ some fixed base point in
 $S^q$, is a generator for the group $H_{n-q} (V, V_\circ) \simeq
 \mathbb{Z}$. Denoting this generator by 1 we have $j_* (y)= \pm
 a$. $y1$. In fact the intersection number of $\varphi (S^q \times 0)$
 with $\varphi (x_0 \times D^{n-q})$ being clearly $\pm$ 1 we have
 $j_* (y) = \pm a$. $y1$. 
 
The existence\pageoriginale of an element $b \in H_{n-q}(V)$ with
$a. b=1$ ensures that $j_*: H_{n-q} (V) \to \mathbb{Z}$ is an
epimorphism and hence we have the exact sequence  
$$
0 \to H_{n-q} (V_\circ) \to H_{n-q} (V) \xrightarrow{j_*} \mathbb{Z}
\to 0. 
$$
In particular Rank $H_{n-q} (V_\circ) <$ Rank $H_{n-q}(V)$

We have $V' = V_\circ \cup D^{q+1} \times S^{n-q-1}$ with $V_\circ \cap
D^{q+1} \times S^{n-q-1} = S^q \times S^{n-q-1}$. Letting $j_1$: $S^q
\times S^{n-q-1} \to D^{q+1} \times S^{n-q-1 }$ and $i' = V_\circ \to V'$
denote the respective inclusions we have the Mayer-Vietais sequence. 
\begin{align*}
H_i (S^q \times S^{n-q-1}) &\xrightarrow{(-j_1)_* \oplus \varphi_*} H_i
(D^{q+1} \times S^{n-q-1}) \oplus H_i (V_\circ)\\ 
&\xrightarrow{\varphi'_* +
  i'_*} H_i (V') \to H_{i-1} (S^q \times S^{n-q-1}) 
\end{align*}

It follows that if $1 < i < n-q-1$ we have 
$$
H_i (V_\circ) \xrightarrow{i'_*} H_i (V').
$$
Also if $i = 1$ and $i < n-q-1$ we have the exact sequence 
$$
0 \to 0 \oplus H_1 (V_\circ ) \xrightarrow{i'_*} H_1 (V') \to \mathbb{Z}
\xrightarrow{(-j_1)_* \oplus \varphi_*} \mathbb{Z} \oplus \mathbb{Z}
\xrightarrow{\varphi'_* +i'_*} \mathbb{Z} 
$$
The map $(-j_1)_* \oplus \varphi_*$ carries $1 \in \mathbb{Z} =
H_0 (S^q \times S^{n-q-1})$ into $(-1, 1)$ of $\mathbb{Z} \oplus
\mathbb{Z}$ and hence a monomorphism. Therefore $H_1 (V_\circ)
\xrightarrow{i'_*} H_1 (V')$ is also an isomorphism in this case. Thus
we see that if $i < n-q-1$ then $H_i (V_\circ) \xrightarrow{i'_*}H_i (V')$
is an isomorphism. We now consider the two\pageoriginale cases $n = 2q
+1$ and $n =2q$ separately.  


\medskip
\noindent{\textbf{Case (1)}}% case 1
$n = 2q +1$. Then $q = n-q-1$. We have already proved that $H_i (V_\circ )
  \xrightarrow{(i_\circ)_*} H_i (V)$ is an isomorphism for $i \neq n-q$
  and $n$. The Mayer-Victoris sequence for $ i = q$ yields the exact
  sequence $H_q (S^q \times S^q) \xrightarrow{(-j)_* \oplus \varphi_*}
  H_q (D^{q +1} \times S^q) \oplus H_q (V_\circ) \to H_q (V') \to
  0$. Writing $H_q (S^q \times S^q)$ as $\mathbb{Z} \oplus \mathbb{Z}$
  we see that $(-j_1)_* \oplus \varphi_*$ carries $(1,0)$ of
  $\mathbb{Z} \oplus \mathbb{Z}$ into $(i^{-1}_{\circ_*}, (a)$ of $H_q
  (D^{q+1} \times 
  S^q) \oplus H_q (V_\circ)$ and $(0, 1)$ into $(-1, 0)$. Since the
  intersection number $a \cdot b =1$ we see that a has to be of infinite
  order and the above sequence now yields $H_q (V') \simeq H_q (V_\circ) /
  (a)$. Observing that $(i_\circ)_*: H_q (V_\circ) \to H_q (V)$ is an
  isomorphism we see that Rank $H_q (V') <$ Rank $H_q (V)$. Actually
  $H_q (V') \simeq H_q (V)/ (a)$. 

\medskip
\noindent{\textbf{Case (2)}}%case 2
$n = 2q$. As already verified $H_i (V_\circ) \xrightarrow{i'_*} H_i (V')$
  is an isomorphism for $i <n-q-1=q-1$. Also $H_i (V_\circ)
  \xrightarrow{(i_\circ)_*} H_i (V)$ is an isomorphism for $i \neq q$ and
  $n$. Combining these $H_i (V) \xrightarrow{i'_* \circ (i_\circ)^{-1}_*}H_i
  (V')$ is an isomorphism for $i < q -1$. For $i = q-1$ the
  Mayer-Victoris sequence yields the exact sequence  
\begin{align*}
H_{q-1} (S^q \times S^{q-1}) &\xrightarrow{(-j_1)_* \oplus \varphi_*}
  H_{q-1}(D^{q+1} \times S^{q-1}) \oplus H_{q-1} (V_\circ)\\ 
&\to H_{q-1} (V') \to 0. 
\end{align*}

But\pageoriginale $H_{q-1} (S^q \times S^{q-1}) \simeq \mathbb{Z}, 
H_{q-1} (D^{q+1} 
\times S^{q-1}) \simeq \mathbb{Z}$ and the map $(-j_1)_* \oplus
\varphi_*$ carries 1 of $H_{q-1} (S^q \times S^{q-1})$ into $(-1,
0)$. Hence $i'_*: H_{q-1}\break (V_\circ) \to H_{q-1}(V')$ is an
isomorphism. Since $(i_\circ)_*: H_{q-1} (V_\circ)\to H_{q-1} (V)$ is also
an isomorphism we have $H_{q-1}(V) \xrightarrow{i'_* \cdot (i_\circ)^{-1}_*}
H_{q-1} (V')$ an isomorphism. For $i = q$ the Mayer-Victoris sequence
yields  

$H_q (S^q \times S^{q-1}) \to 0 \oplus H_q (V_\circ) \to H_q (V') \to
H_{q-1} (S^q \times S^{q-1}) \xrightarrow{\text{`mono'}} H_{q-1}
(D^{q+1} \times S^{q-1} ) \oplus H_{q-1}(V_\circ)$.  

The map $H_{q-1} (S^q \times S^{q-1}) \xrightarrow{(-j_1)_* \oplus
  \varphi_*} H_{q-1} (D^{q+1} \times S^{q-1}) \oplus H_{q-1}(V_\circ)$
which carries the generator 1 of $H_{q-1} (S^q \times S^{q-1})$ into
$(-1, 0)$ is clearly a monomorphism. Hence $H_q (S^q \times S^{q-1})
\to H_q (V_\circ) \xrightarrow{i'_*} H_q (V') \to 0$ is exact. It follows
that Rank $H_q (V')<$ Rank $H_q (V_\circ)$. The map composite $H_q (S^q \times
S^{q-1}) \to H_q (V_\circ)$ carries the generator of $H_q (S^q \times
S^{q-1})$ into `$a$', an element of infinite order. As already
verified Rank $H_q (V_\circ)<$ Rank $H_q (V)$ (since $q = n -q$, and
we actually verified Rank $H_{n-q} (V_\circ)<$ Rank $H_{n-q}(V)$). 

This completes the proof of Lemma \ref{chap1:lem4.7}


\section{Proof of the main theorem for $n = 4 d >
  4$}\label{chap1:sec5} %Section 5 

We\pageoriginale have already obtained a compact, connected, simply
connected 
$C^\infty$ manifold $M$ of dimension $4d$ and a map $f$: $M \to X$ of
degree 1 satisfying $f! (\eta ) \simeq \tau^n_M \oplus
\mathscr{O}^\ell_M$ and $f_*: H_i (M) \to H_i (X)$ an isomorphism
$\forall i < 2d$. (Proposition \ref{chap1:prop4.5}).  

Let $K_{2 d} = \text{ Ker } 
f_{2d} : H_{2d} (M) \to H_{2d} (X)$.  

\begin{lemma}\label{chap1:lem5.1}% lemma 5.1.
$K_{2d}$ is a free abelian group.
\end{lemma}

\begin{proof}
Since $H_{2d}(M)$ is finitely generated and $K_{2d}$ a direct summand
of $H_{2d}(M)$ (Lemma \ref{chap1:lem2.5}) it follows that $K_{2d}$ is finitely
generated. To prove that $K_{2d}$ is free it therefore suffices to
prove that $K_{2d}$ is torsion free. We write $g$ for
$2d$ for simplicity. If possible let $x \in K_q$ be any torsion
element and let $x^1 \in H^q (M)$ correspond to $x$ under
Poincare duality i.e. $x^1 \cap [M] = x$. $x^1$ is then a torsion
element of $H^q (M)$. By the Universal Coefficient Theorem for
cohomology we have the following commutative diagram. 
\end{proof}
\[
\xymatrix@R=1.5cm{
  0 \ar[r] & \Ext (H_{q-1} (M), \mathbb{Z}) \ar[r]^{\beta} & H^q(M)
  \ar[r]^{\alpha} & \Hom (H_q(M), \mathbb{Z}) \ar[r] & 0 \\
0 \ar[r] & \Ext (H_{q-1}(X),\mathbb{Z}) \ar[r]^{\beta}
\ar[u]_{\approx}^{\Ext (f_*, Id_{\mathbb{Z}})} & H^q (X)
\ar[r]^{\alpha} \ar[u]^{f_*} & \Hom(H_q (X), \mathbb{Z}) \ar[r]
\ar[u]_{\Hom (f_*,   Id_{\mathbb{Z}})} & 0   
}
\]
\begin{center}
{\bf Diagram 7}
\end{center}

Clearly,\pageoriginale $\Hom(H_q (M), \mathbb{Z})$ is torsion
free. Also for any 
finitely generated abelian group $A$ the group $\Ext(A, \mathbb{Z})$ is
a torsion group. It follows that $\beta(\Ext (H_{q-1}(M), \mathbb{Z}))$
is precisely the torsion subgroup of $H^q (M)$. Hence $\exists$ an
element $y^1 \in \Ext (H_{q-1} (M), \mathbb{Z})$ with $\beta (y^1)
= x^1$. Since $f_*$: $H_i (M) \to H_i (X)$ is an isomorphism for $i
\leq q-1$ we have
$$
\Ext(f_*, Id_{\mathbb{Z}}): \Ext (H_{q-1}(X),
\mathbb{Z}) \to \Ext (H_{q-1} (M), \mathbb{Z})
$$
an isomorphism. Let $z^1 \in H^q (X)$ be given by $z^1 = \beta \circ
(\Ext(f_*, Id_{\mathbb{Z}})^{-1}(y'))$. Then clearly $f^* (z^1) =
x^1$. Our aim is to show 
that $K_q$ has no torsion, or that $x = 0$. For this it suffices to
show that $x^1 = 0$ since $\cap [M]= \Delta 
: H^q (M) \to H_q (M)$ is an isomorphism. Now consider the element
$z^1 \cap u \in H_q (X)$. Since $f$ is of degree 1 we have $f_*
([M]) = u$. We have  
$$ 
0 = f_* (x) = f_* (x^1 \cap [M]) = f_* (f^* (z^1) \cap [M]) = z^1 \cap 
f_* [M] = z^1 \cap u.  
$$
But by assumption $\cap u$: $H^q (X) \to H_q (X)$ is an
isomorphism. Hence $z^1 = 0$ and therefore $x^1 = f^* (z^1) = 0$. This
completes the proof of Lemma \ref{chap1:lem5.1}. 

For the rest of \S \ref{chap1:sec5} we denote $2d$ by $q$. 

Let $H_q (M)= K_q \oplus gH_q (X)$ be the splitting given by Lemma
\ref{chap1:lem2.5}  

\begin{lemma}\label{chap1:lem5.2}% lemma 5.2
For any $a \in K_q$ and any $b \in g H_q (X)$ the
intersection number $a \cdot b = 0$. Also if $b_1 = g(c_1)$ and $b_2 =
g(c_2)$ with $c_1, c_2 \in H_q (X)$ then the intersection
number $b_1 \cdot b_2$ is the same is $c_1 \cdot c_2$. 
\end{lemma}

\begin{proof}
Let $b = g(c)$\pageoriginale with $ c \in H_q (X)$ ($c$ is unique
since $g$ is a mono). Let $\gamma \in H^q (X)$ be such that
$\gamma \cap u = 
c$. Then by the very definition of $g$ we have $b = f^* (\gamma) \cap
[M]$. To prove that $a \cdot b = 0$ it suffices to verify that $f_*
((\alpha \cup f^* (\gamma)) \cap [M]) = 0$ with $\alpha \in H^q
(M)$ satisfying $\alpha \cap [M] = a$. Since $q = 2d $ we have $\alpha
\cup f^* (\gamma ) = f^* (\gamma) \cup \alpha$. Hence $f_* ((\alpha
\cup f^* \gamma ) \cap [M])= (-1)^{q \cdot q} f_* ((f^* \gamma \cup \alpha )
     [M] ) = f_* (f^* \gamma \cap (\alpha \cap [M]))$ (since $q = 2d )
     = f_* (f^* \gamma \cap a ) = \gamma \cap f_* (a) = 0 $ since
     $f_*(a) =0$. Choosing $\gamma_1$, $\gamma_2$ in $H^q (X)$ with
     $\gamma_1 \cap u= c_1$, $\gamma_2 \cap u = c_2$ we have $b_1 =
     f^* (\gamma_1) \cap [M]$ and $b_2 = f^* (\gamma_2) \cap [M]$. Now 
\begin{align*}
f_* ((f^* \gamma_1 \cup f^* \gamma_2) \cap [M]) & = f_* (f^*(\gamma_1
\cup \gamma_2 ) \cap [M]) = (\gamma_1 \cup \gamma_2 ) \cap f_*
([M])\\ 
&= (\gamma_1 \cup \gamma_2) \cap u.
\end{align*}
From this the equality $b_1 \cdot b_2= c_1 \cdot c_2$ follows. 

Denoting by $T_q (M)$ and $T_q (X)$ respectively the torsion subgroup
of $H_q (M)$ and $H_q (X)$ we have $H_q (M)/_{T_q (M)} \simeq K_q
\oplus \dfrac{H_q (X)}{T_q (X)}$. (because of Lemma
\ref{chap1:lem5.1}). Lemma \ref{chap1:lem5.2} 
precisely states that we can find bases for $K_q $ and $\dfrac{H_q
  (X)}{T_q (X)}$ such that the matrix $A_M$ of the intersection
bilinear form on $H_q (M)/ T_q (M)$ take the form
$\left( \begin{smallmatrix} A_K & 0\\ 0&A_X\end{smallmatrix}
  \right)$\pageoriginale 
  where $A_K$ and $A_X$ are the matrices of the form restricted to
  $K_q$ and $H_q (X)/ T_q (X)$. Also the lemma asserts that the
  restriction of the intersection bilinear form on $H_q (M)/ T_q (M)$
  to $H_q (X)/ T_q (X)$ agrees with the intersection bilinear form on
  $H_q (X) / T_q (X)$ got from the fact that $X$ satisfies Poincare
  duality. Since intersection by definition corresponds to cup-product
  under Poincare duality we see that the signature of $A_M$ is the
  same as the index of the manifold $I(M)$ defined in 1.6 and
  similarly signature if $A_X$ is $I(X)$. Let us denote the signature
  of $A_K$ by $I(K)$. Then we have $I(X) + I(K)= I(M)$. 
\end{proof}

\begin{lemma}\label{chap1:lem5.3}% lemma 5.3
$I(K)$ is zero.
\end{lemma}

\begin{proof}
The assumption iii) of Theorem \ref{chap1:thm2.1} is actually used in
concluding that 
$I(K) = 0$. We have a map $f: M \to X$ of degree 1 with $f! (\eta
) = \tau^n_M \oplus \mathscr{O}^\ell_M$. Also $[\eta] = - [\xi]$. By
Hirzobruch's Index Theorem $I(M) = \left\{L_d (p_1 (\tau^n _M),
\ldots, p_d(\tau^n_M))\right\}[M]$. 

 But $L_d (p_1 (\tau^n_M), \ldots, p_d (\tau^n_M)) = L_d (p_1
 (f!(\eta)), \ldots, p_d (f! (\eta))$ (since\break $L_k (p_1(\lambda
 ),\ldots, p_k (\lambda))$ for any vector bundle $\lambda $ depends
 only on the stable class of $\lambda$). Hence  
\begin{align*}
I(M) & = \left\{L_d (p_1(f! (\eta ), \ldots, p_d (f!(\eta ))\right\}[M]\\
& = \left\{ L_d (p_1 (\eta ),\ldots, p_d (\eta))\right\}(f_*[M])\\
& = \left\{ L_d (\overline{p_1} (\xi), \ldots, \overline{p_1}(\xi))
\right\}(u)\\ 
& = I(X) \text{ by assumption } (iii).
\end{align*}
This proves that $I(K) = 0$.

Denote\pageoriginale the group $H^q (M)/ T^q (M)$ (where $T^q (M)$ is
the torsion of $H^q (M)$) By $B^q(M)$ and similarly the group $H_q
(M)/ T_q (M)$ by $B_q (M)$. Choosing any basis $x_1, \ldots, x_r $ for
$B^q$ we see that $y_i = x_i \cap [M]$ (actually $\cap [M]: H^q (M)
\to H_q (M)$ gives a well determined isomorphism also denoted by $\cap
M$ of $B^q 
(M)$ onto $B_q (M)$) form a basis for $B_q (M)$. Since $B^q (M) \simeq
\hom (B_q (M), \mathbb{Z})$ we can get elements $y^1_i, \ldots, y^1_r$
in $B^q$ such that $y^1_i (y_j) = \delta_{ij}$. The bilinear form $(x,
y) \rightsquigarrow (x \cup y) [M]$ on $B^q$ is easily seen to have
determinant $\pm 1$, for $(y^1_j \cup x_i )[M] = y^1_j (y_i) =
\delta_{ij}$. It follows that $A_M$ has determinant $\pm 1$. Similarly
$A_X$ has determinant $\pm 1$. It follows that $A_K$ has determinant
$\pm 1$. 
\end{proof}

\begin{lemma}\label{chap1:lem5.4}% lemma 5.4
If $B$ is a symmetric non-degenerate bilinear form on a finitely
generated free abelian group $H$. with determinant $\pm 1$ and if the
signature of $B$ is Zero then $\exists x \neq 0$ in $H$ such that
$B(x, x) = 0$.

A proof of this can be found in \cite{c1:key6}. As a corollary we see that if
$K_q \neq 0 \exists $ an element $a \neq 0$ in $K_q$ such that
$a$. $a=0$. Moreover we can choose `$a$' to be indivisible in
$K_q$. Then $K_q | (a)$ is free and hence we can find a basis of the
form $a$, $b_2, \ldots, b_r$ for $K_q$. Since $A_k$ has determinant
$\pm 1$ and $a$. $a = 0$ we cannot have $a \cdot b_j = 0$ $ \forall j$. If
$j_1, \ldots, j_r$, are the indices in $(2, \ldots, r)$ with $a. b_j
\neq 0$ then $\underset{ i = 1, \ldots r'}{g. c. d} (a. b_{j_i})$ has
to be 1 for otherwise this greatest common divisor will divide
determinant of $A_K$. 

Hence\pageoriginale $\exists$ integers $m_{j_i}$ such that $
\sum\limits^{r'}_{i=1} 
m_{j_i} (a. b_{j_i}) = 1$. The element $b \in K_q$ given by $ b =
\sum\limits^{r'}_{i=1} m_{j_i} (b_{j_i})$ satisfies $a$. $b=1$. 
\end{lemma}

\begin{lemma}\label{chap1:lem5.5}%lemma 5.5
If $d > 1$ there exists an imbedding $\varphi$: $S^q \to M^{4d} (q =
2d)$ representing a and further satisfying $f \circ \varphi \sim
\tilde{x}^*$ (where $\tilde{x}^*$ is the constant map $S^q \to x$
carrying the whole of $S^q$ into $x^*$.) 
\end{lemma}

\begin{proof}
It is for the proof of this lemma that we need $d$ to be 1. By Lemma
\ref{chap1:lem4.1} $\exists$ a continuous map $\Theta: S^q \to M$ representing
`$a$' and satisfying $f \circ \Theta \sim \tilde{x}^*$. We use the fact
that $M$ is simply connected. Also since $M$ is of dimension $4d$ with
$d$ an integer $ > 1$ it follows from Lemma 6 of \cite{c1:key6} that $\exists $ a
$C^\infty$ imbedding $\varphi: S^q \to M$ with $\varphi \sim
\Theta$. This proves Lemma \ref{chap1:lem5.5}. 
\end{proof}

\begin{remark*}
It is not true that a continuous map $\Theta$: $S^2 \to V^4$ is
homotopic to a $C^\infty$ imbedding even if $V^4$ is a compact, simply
connected $C^\infty$ manifold (if dimension 4). An example is given
by Kervaire and Milnor in \cite{c1:key3}.
\end{remark*}

\begin{lemma}\label{chap1:lem5.6}%lemma 5.6.
For any $C^\infty$ imbedding $\varphi: S^q \to M$ representing `$a$'
and satisfying $f \circ \varphi \sim \tilde{x}^*$ the normal bundle
$\nu_\varphi$ is trivial. 
\end{lemma}

\begin{proof}
We have $ \tau_M | \varphi (S^q) \simeq \tau^q_{\varphi (S^q)} \oplus
\nu^q_\varphi$. Since $M$ and $S^q$ are orientable it follows that
$\nu_\varphi $ in orientable. Also from\pageoriginale $f! (\eta)|
\varphi (S^q) 
\simeq (\tau^n_M \oplus \mathscr{O}^\ell_M)| \varphi (S^q)$, we have
\begin{align*}
& f! (\eta)| \varphi (S^q) \simeq \tau^q \varphi (S^q) \oplus
\nu^q_\varphi \oplus \mathscr{O}^\ell_{\varphi (S^q)} \approx
\tau_{\varphi (S^q)} \oplus \mathscr{O}_{\varphi (S^q)} \oplus \nu^q_\varphi
\oplus \mathscr{O}^{\ell-1}_{\varphi (S^q)}\\
& \simeq
\mathscr{O}^{q+1}_{\varphi (S^q)} \oplus \nu_\varphi \oplus
\mathscr{O}^{\ell-1}_{\varphi (S^q)} \simeq \nu_\varphi \oplus
\mathscr{O}^{q + \ell}_{\varphi (S^q)}. 
\end{align*}

But since $f \circ \varphi \sim \tilde{x}^*$ we have $f ! (\eta)| \varphi
(S^q) \simeq \mathscr{O}^{2q + \ell}_{\varphi (S^q)}$. 

Thus $\nu_\varphi \otimes \mathscr{O}^{q + \ell}_{\varphi (S^q)} \simeq
\mathscr{O}^{2q + \ell}_{\varphi (S^q)}$. Thus $\nu_\varphi$ is stably
trivial. If $\nu \in \Pi_{q-1} (SO_q)$ is the element
corresponding to the bundle $\nu_\varphi$ on $S^q$ we have $S_* (\nu)
= 0$ where $s_* $: $\Pi_{q-1} (SO_q ) \to \Pi_{q-1} (SO_{2q + \ell })$
is the homomorphism induced by the inclusion. Since
$\Pi_{q-1}(SO_{q+1}) \to \Pi_{q-1} (SO_{2q + \ell })$ is an
isomorphism it follows that $i_* (\nu) = 0$ where $i_*$: $\Pi_{q-1}
(SO_q) \to \Pi_{q-1} (SO_{q+1})$ is induced by the inclusion. Since
$SO_{q+1}/ SO_q = S^q$ we have a fibration of $SO_{q+1}$ by $SO_q$ as
the fibre and $S^q$ as the base. Consider the corresponding exact
sequence  
$$
\Pi_q (S^q) \xrightarrow{\partial} \Pi_{q-1} (SO_q) \xrightarrow{i_*}
\Pi_{q-1}(SO_{q+1}). 
$$ 
$\partial$ carries a generator of $\Pi_q (S^q)$ into the element
$\tau$ of $\Pi_{q-1}(SO_q)$ corresponding\pageoriginale to the tangent
bundle of $S^q$. Since $i_* (\nu) = 0$ it follows that $\nu-k \tau $
for some integer $k$. The map which assigns to an isomorphism class
$\lambda$ of an orientable vector bundle of rank $q$ over $S^q$ its
Euler class $\chi (\lambda)$ defines a homomorphism $\chi$: $\Pi_{q-1}
(SO_q) \to H^q (S^q)$. For the tangent bundle $\tau $ of $S^q$ the
class $\chi (\tau)$ is known to be twice a generator of $H^q
(S^q)$. (That $q = 2d$ is even, we use here). Thus the composition
$\Pi_q (S^q) \xrightarrow{\partial} \Pi_{q-1}(SO_q) \xrightarrow{\chi}
H^q (S^q)$ is a monomorphism and any element in the image of
$\partial$ is zero if and only if its Euler class is zero. The Euler
class of the normal bundle of the imbedding $\varphi$ representing
`$a$' can be identified with $a \cdot a$ times a generator of $H^q
(S^q)$. For, given a normal vector field with a finite number of zeros
on $\varphi (S^q)$ we can deform $\varphi (S^q)$ along these vectors
to obtain a new imbedding which intersects $\varphi (S^q)$ at only
finitely many places. The multiplicity of each such intersection is
equal to the index of the corresponding zero of the normal vector
field.  
\end{proof}

\begin{remark*}
A more `formal' proof for the fact that $\chi (\nu_\varphi) = a \cdot a$
times a generator of $H^q (S^q)$ can be given as follows. 
\end{remark*}

Denoting the imbedded manifold $\varphi (S^q)$ by $S^q$ itself, let
$\Phi$: $H^i\break (S^q) \to H^{q+i} (T(\nu))$ be the Thom isomorphism. If
$U= \Phi (1) \in H^q (T(\nu))$ then the Euler class of $\nu$ can
be defined by $\chi (\nu) = \Phi^{-1} (\bigcup \cup \bigcup
)$. \cite{c1:key5}. Taking a tubular neighbourhood $A$ of $S^q$ in $M$ and
collapsing the exterior of $A$ to a point we get\pageoriginale a map
$C: M \to T(\nu)$. If $\gamma \in H^q (M)$ is the class which corresponds
to `$a$' under Poincare duality (i.e. $\gamma \cap [M] = a$) it is
known that $C^* (\cup) = \gamma$ \cite{c1:key9}. Hence $C^* (\bigcup \cup
\bigcup) = \gamma \cup \gamma =a\cdot a[M]$ by the definition of the
intersection number. But from the diagram 
\[
\xymatrix@R=1.5cm{
H^q (S^q) \ar[r]^{\Phi}_{\approx} & H^{2q} (T (\nu)) \ar[dl]_{C^*} &
H^{2q} (T(\nu), T(\nu) - S^q) \ar[d]^{\rm excision}_{\approx}
\ar[l]_>>>>>>>>{\approx}\\ 
H^{2q}(M) & H^{2q} (M,M-S^q) \ar[r]^{\rm excision}_{\approx}
\ar[l]^>>>>>{j_*}  & H^{2q} (A, A-S^q)
}
\]
We see that $H^{2q} (M, M-S^q) \simeq \mathbb{Z}$. Taking any $pt$ $x
\in S^q$ we have the triangle: 
\[
\xymatrix@R=1.5cm{
\mathbb{Z} \simeq H^{2q} (M) & H^{2q} (M, M-S^q)\ar[l]_{j_*} \\
& H^{2q} (M,M-x) \ar[u] \ar[ul]^{\approx}
}
\]

Hence $H^{2q}(M, M-x) \simeq \mathbb{Z}$ has to be a direct summand of
$H^{2q} (M, M-S^q)$ which is also $\simeq \mathbb{Z}$. It follows that
$j^*: H^{2q} (M, M-S^q) \approx H^{2q}(M)$. Examining the diagram
again we see that $C^* = H^{2q} (T(\nu)) \approx H^{2q}(M)$. Hence
$\bigcup \cup \bigcup =a \cdot a$ times a generator of $H^{2q} (T(\nu))$
and $\Phi^{-1} (\bigcup \cup \bigcup) = a \cdot a$ times a generator of
$H^q (S^q)$. 

We ar now almost at the end of the proof of Theorem \ref{chap1:thm2.1} for the case
$n = 4d$. Choosing an indivisible $a \neq 0$ in $K_q$
with\pageoriginale $a \cdot a=0$ 
we saw that $\exists$ $b \in K_q$ with $a \cdot b = 1$. The existence of
such an `$a$' was guaranteed by Lemma \ref{chap1:lem5.4}. From Lemma
\ref{chap1:lem5.5} and \ref{chap1:lem5.6} we
see that $\exists$ an orientation preserving imbedding $\varphi: S^q
\times \dfrac{3}{2} D^q \to M$ with $f \circ \varphi \sim \tilde{x}^*$ and
representing `$a$'. Let now $M' = \chi (M, \varphi)$ and $f': M' \to
X$ the associated map which is constructed after altering $f$ in its
homotopy class so as to satisfy $f \circ \varphi = x^*$. By Lemma
\ref{chap1:lem3.3} $f'$ 
is of degree 1. To get an isomorphism $\tau^n_M$, $\oplus \mathscr{O}^\ell_M,
\to f' ! (\eta)$ we had an obstruction $\gamma \in \Pi_q
(SO_{n+\ell})$ and when $\varphi$ was replaced by $\varphi_\alpha $
given by $\varphi_\alpha (x, y) = \varphi (x, \alpha (x) y)$ with
$\alpha: S^q \to SO_q$ a $C^\infty$ map then the new obstruction
$\gamma_\alpha$ satisfied the relation $\gamma_\alpha = \gamma + s_*
(\alpha)$ where $S_* : \Pi_q (SO_q) \to \Pi_q (SO_{n +\ell})$ is the
homomorphism induced by the inclusion. (Lemma \ref{chap1:lem3.6}). Since $q$ is even
the homomorphism $\Pi_q (SO_q) \to \Pi_q (SO_{q+1})$ is onto
\cite{c1:key8}. Also 
$\Pi_q (SO_{q+1}) \to \Pi_q (SO_{n+ \ell})$ is onto. Thus there exists
an $\alpha$ such that $f'_\alpha: M' = \chi (M, \varphi_\alpha) \to
X$ satisfies the condition $f' \alpha ! (\eta ) \simeq \tau^n_M$,
$\oplus \mathscr{O}^\ell_M$, in addition to being of degree 1. Thus
without loss of generality we can assume that $f'$ itself was `good'
in the sense that $f' ! (\eta) \simeq \tau^n_M$,
$\mathscr{O}^\ell_M$. Denoting the inclusions of $M_o$ in $M$ and $M'$
respectively by $i$ and $i'$ we have the following commutative diagram
for every integer $j$. 
\[
\xymatrix@R=1.5cm@C=1.5cm{
H_j(M) \ar[dr]^{f_*} & \\
H_j (M_o) \ar[u]_{i_*} \ar[d]^{i'_*} &  H_j(X)\\
H_j(M') \ar[ur]_{f'_*} & 
}
\]
By\pageoriginale Case 2 of Lemma \ref{chap1:lem4.7} we have $i_*: H_j (M_o) \to
H_j (M)$ and 
$i'_*:H_j (M_o) \to H_j (M')$ to be isomorphisms for $j< q$. Since
$f_*: H_j (M) \to H_j (X)$ is an isomorphism for $j < q$ it follows
that $f': H_j (M')\to H_j(X)$ is an isomorphism for $j < q$. Also by
the same lemma $RK$ $H_q (M') < RK H_q (M)$. If $K'_q$ denotes the
Kernel of $f'_q = H_q (M') \to H_q (X)$ we have $K'_q$ free and of
rank $<$ rank of $K_q$. It follows that after a finite number of
spherical modifications we can obtain a manifold $M''$ and a map $f'':
M'' \to X$ with $\deg f'' = 1$, $f'' ! (\eta) \simeq \tau^{n}_{M''}
\oplus \mathscr{O}^\ell_{M''}$ and $K''_{q}= \ker f''_{q} = 0$. It
follows from the Remark \ref{chap1:dashrem4.5'}$'$ that $f'': M'' \to
X$ is a homotopy equivalence. This completes the proof of the main
theorem for $n= 4d > 4$.  

\section{Proof of the main theorem for $n = 2q +1$}\label{chap1:sec6}
%Section 6  

Throughout \S \ref{chap1:sec6} we will assume $n = 2q +1$ with $q$ an integer $\geq
2$. Let $W = W^{2q +2}$ be a compact orientable topological
manifold\pageoriginale 
of dimension $2q+2$ with boundary $bW$. Let $F$ be any fixed
field. The semi-characteristic $e^* (bW; F)$ of $bW$ with respect to
$F$ is defined to the residue class $\sum\limits^{q}_{i = 0}$ Rank
$H_i (bW; F)$ modulo 2. Let $\rho_F$ be the rank of the bilinear
pairing $H_{q+1}(W; F) \otimes H_{q+1} (W; F) \to F$ given by the
intersection number and $e (W)$ the Euler characteristic of $W$,  

\begin{lemma}\label{chap1:lem6.1}%lemma 6.1
 We have $e^* (bW; F) + e (W) \equiv \rho_F (\mod 2)$. 
\end{lemma}

\begin{proof}
Consider the homology exact sequence of the pair $(W, bW)$ with
coefficients in $F$,  
\begin{align*}
H_{q + 1}(W; F) &\xrightarrow{j_*} H_{q+1} (W, bW; F)\\
&\xrightarrow{\partial} H_q (bW; F)\\ 
&\to \cdots \to H_0 (W; bW; F) \to 0. 
\end{align*}
\end{proof} 

By Poincare-Lefschetz duality if $z \in H_{q+1} (W, bW; F)$ is
such that $x\cdot Z = 0$ $\forall$ $x \in H_{q+1} (W; F)$ then $Z =
0$. It follows from this remark and the relation $x \cdot y =
x \cdot j_*(y)$ for any $x, y \in H_{q+1} (W; F)$ that $\ker j_*$
is precisely the nullity of the intersection bilinear form on $H_{q+1}
(W; F)$. Hence  
\begin{align*}
\rho_F& = \dim H_{q+1} (W; F) - \dim \ker j_* = \dim \im j_* = \dim
\ker \partial\\ 
&= \dim H_{q+1} (W, bW; F)- \dim \im \partial 
\end{align*}
Denoting the dimensions of $H_j (W: F)$ and $H_j (W, bW; F)$ by $b_j
(W; F)$ and $b_j (W, bW; F)$ respectively we have  
$$
\rho_F = b_{q+1} (W, bW ; F) - b_q (bW ; F) + b_q (W; F)- b_q (W, bW;
F)+ \cdots. 
$$ 

But\pageoriginale $b_i (W, bW; F) = b_{2q +2-i} (W; F)$ by
Poincare-Lefschetz duality. Thus $\rho_F \equiv e^* (bW; F) + e (W)
(\mod 2)$.  

Let $V$ be a compact connected oriented $C^\infty$ manifold of
dimension $n = 2q +1$ and let $a \in H_q(V)$ be any torsion
element $\neq 0$. Suppose further $\varphi: S^q \times \dfrac{3}{2}
D^{n-q} \to V$ is an orientation preserving imbedding representing the
homology class `$a$'. Let $V' = \chi (V, \varphi)$.  

\begin{lemma}\label{chap1:lem6.2}%lemma 6.2.
If $q$ is even we have an exact sequence 
$$
0 \to \mathbb{Z} \to H_q (V') \to H_q (V)/ (a) \to 0
$$
where $(a)$ is the subgroup generated by a in $H_q (V)$
\end{lemma}

\begin{proof}
As usual let $V_\circ = V - \varphi (S^q \times B^{q+1})$ and let
$\varphi'$: $D^{q+1} \times S^q \to V'$ be the imbedding induced by
the inclusion of $D^{q+1} \times S^q$ in $\dfrac{3}{2} B^{q+1} \times
S^q$. We then have the following commutative diagram with exact
horizontal rows.  
\end{proof}
{\fontsize{10}{12}\selectfont
\[
\xymatrix{
\mathbb{Z} \simeq H_q (S^q \times D^{q+1}) \ar[r]^>>>>>{\varphi_*} & H_q(V)
\ar[r] &  H_q (V, \varphi (S^q \times D^{q+1})) \ar[r] & 0 \\ 
& & H_q (V_\circ, \varphi (S^q \times S^q)) \ar[u]_\approx & \\
\mathbb{Z} \simeq H_q (D^{q+1} \times S^q) \ar[r]^>>>>>{\varphi'_*} & H_q
(V') \ar[r] & H_{q} (V', \varphi' (D^{q+1} \times S^q)) \ar[r] & 0
 }
\]}\relax
\begin{center}
{\bf Diagram 8}
\end{center}

Since\pageoriginale $\varphi_* (1) = a$ by assumption it follows that
$H_q (V', \varphi' (D^{q+1} \times S^q)) H_q (V, \varphi (S^q \times
D^{q+1})) \simeq H_q (V)/(a)$. To prove Lemma \ref{chap1:lem6.2} we have only to show
that $\varphi'_* :\mathbb{Z} \to H_q (V')$ is a monomorphism. Since
`$a$' is a torsion element to show that $\varphi'_*$ is a monomorphism
we have only to prove that $b_q (V', \mathbb{Q}) \not\equiv b_q (V,
\mathbb{Q}) (\mod 2)$ where $b_q (V, \mathbb{Q})$ is the $q^{th}$
Bettinumber of $V$ i.e. the rank of $H_q (V, \mathbb{Q})$. Since $H_i
(V) \simeq  H_i (V, \varphi (S^q \times D^{q+1})) \simeq H_i (V_\circ, \varphi
(S^q \times S^q)) \simeq H_i (V', \varphi' (S^{q+1} \times S^q)) \simeq
H_i (V')$ for $i < q$ the statement $b_q (V', \mathbb{Q}) \not\equiv
b_q (V, \mathbb{Q}) (\mod 2)$ will follow if we show that
$\sum\limits^{q}_{i = 0} b_i (V', \mathbb{Q}) +\sum\limits^{q}_{i=0}
b_i (V, \mathbb{Q}) \not\equiv 0 (\mod 2)$. 

Let $W = I \times V \bigcup\limits_\varphi D^{q+1} \times D^{q+1}$ be
the topological manifold got as follows. We take the disjoint union of
$I \times V$ and $ D^{q+1} \times D^{q+1}$ and identify the points of
$S^q \times D^{q+1}$ with their images under $\varphi$ in $V \times
1$. Then $W$ is a compact orientable manifold of dimension $2q + 2$
with boundary consisting of the disjoint union of $V$ and $V'$. Hence
by Lemma \ref{chap1:lem6.1} we have $e^* (bW; \mathbb{Q}) + e(W) \equiv \rho(\mod 2)$
where $\rho$ is the rank of the intersection bilinear pai ring
$H_{q+1}(W, \mathbb{Q}) \times H_{q+1} (W, \mathbb{Q}) \to
\mathbb{Q}$. Since $q$ is even, this intersection bilinear pairing is
skew symmetric and hence $\rho$ is even. But 
$$
e^* (bW ; \mathbb{Q})
\equiv \sum\limits^{q}_{i=0} b_i (V', \mathbb{Q})+
\sum\limits^{q}_{i=0} b_i (V, \mathbb{Q}) \; (\rm{mod} ~ 2).
$$

 Also $W$ is of
the same homotopy type as the space got from $V$ by attaching
$D^{q+1}$ by means of $\varphi | S^q \times 0$ and hence $e(W) = e(V)
+ (-1)^{q+1}$. Since $V$ is of\pageoriginale dimension $2q+1$ by
Poincare duality we 
have $e (V) \equiv 0\break (\mod 2)$ and hence the relation $e^* (bW ;
\mathbb{Q}) + e (W) \equiv 0 (\mod 2)$ yields $\sum\limits^{q}_{i=0}
b_i (V' \mathbb{Q})+ \sum\limits^{q}_{i=0} b_i (V, \mathbb{Q}) +
(-1)^{q+1} \equiv (\mod 2)$ or $\sum\limits^{q}_{i=0} b_i (V'
\mathbb{Q})+ \sum\limits^{q}_{i=0} b_i (V,\break \mathbb{Q}) \not\equiv 0
(\mod 2)$. This completes the proof of Lemma \ref{chap1:lem6.2}. 

We now consider the case when $q$ is odd. Let $d$ be the order of
`$a$'. Since $a \neq 0$ and is a torsion element of $H_q (V)$, $d$ is
an integer $> 1$. Now suppose the imbedding $\varphi$: $S^q \times
\dfrac{3}{2} D^{q+1} \to V$ representing `$a$' is replaced by
$\varphi_\alpha$ given by $\varphi_\alpha (x, y) = \varphi (x, \alpha
(x).y)$ with $\alpha: S^q \to S O_{q+1}$ a $C^\infty $ map satisfying
$s_* (\alpha ) = 0$ where $s_*: \Pi_q (SO_{q+1}) \to \Pi_q
(SO_{2q+1+\ell})$ is the homomorphism induced by the inclusion $s:
SO_{q+1} \to SO_{2q +1+ \ell}$. Let $y^*$ be a base point chosen once
for all and let $j: SO_{q+1} \to S^q$ be the map given by $j
(w)= w y^*$. (We consider $y^*$ as a column vector in
$\mathbb{R}^{q+1}$ and the matrix $w$ operates on the right on
$y^*$). We want to study the $q^{th}$ homology of $V'_{\alpha} = \chi
(V, \varphi_\alpha)$. Clearly the manifold $V_\circ = V- \varphi_{\alpha}
(S^q \times B^{q+1})$ is independent of $\alpha$ and the meridian
$\varphi_\alpha (y^* \times S^q)$ of the torus $\varphi_\alpha (S^q
\times S^q)=$ Bdry $V_\circ$ as a point set does not depend on $\alpha$,
hence its homology class $\varepsilon'$ in $H_q (V_\circ)$ does not depend
on $\alpha$. On the other hand the homology class $\varepsilon_\alpha$
of $\varphi_\alpha (S^q \times y^*)$ in $H_q (V_\circ)$ does depend on
$\alpha$. Let $\varepsilon $ be the homology class of $\varphi (S^q
\times y^*)$ in $H_q (V_\circ)$. Then we have   
$$ 
\varepsilon_\alpha = \varepsilon + j_* (\alpha) \varepsilon' \text{
  where }  j_* : \Pi_q (SO_{q+1}) \to \Pi_q (S^q) \simeq \mathbb{Z}  
$$\pageoriginale
 is the homomorphism induced by $j$. 

We claim that $ \exists$ an integer $d'_{\alpha}$  such that $ d
\varepsilon_\alpha = d' _\alpha \varepsilon'$ in $H_q(V_\circ)$. Actually
in the homology exact sequence  
$$ 
\to H_{q+1} (V_\circ) \xrightarrow{i_*} H_{q+1} (V) \to H_{q+1} (V, V_\circ)
\xrightarrow{\partial}H_q (V_\circ) \xrightarrow{i_*} H_{q} (V) \to 
$$
identifying $ H_{q+1}(V, V_\circ)$ with $ \mathbb{Z} \simeq H_{q+1}(S^q
\times D^{q+1}, S^q \times S^q)$ by excision we saw that the
homomorphism $ H_{q+1}(V) \to H_{q+1} (V,  V_\circ)$ was given by $x
\rightsquigarrow \pm$ a.x(Refer to the proof of Lemma \ref{chap1:lem4.7}). Since
`$a$' is a torsion element we have $a . x =0$ and hence  
$$
0 \to \mathbb{Z} \simeq H_{q+1} (V, V_\circ) \xrightarrow{\partial} H_q
(V_\circ) \xrightarrow{i_*} H_q (V) \to \cdots 
$$
is exact. $\partial$ carries the generator $ \varphi (y ^* \times
D^{q+1})$ of the relative group $ H_{q+1} (V, V_\circ)$ into $
\varepsilon'$ in $ H_q (V_\circ)$. The element $ d \varepsilon$ of $ H_q
(V_\circ)$ gets mapped into $ da = 0$ by $ i_*$ and hence $ \exists$ an
integer $ d'$ such that $ d \varepsilon = d' \varepsilon'$. From $
\varepsilon _\alpha = \varepsilon + j_* (\alpha ) \varepsilon'$ we
have $ d \varepsilon_\alpha = d \varepsilon + dj_* (\alpha )
\varepsilon' = (d' + dj_* (\alpha )) \varepsilon'$. Thus $ d'_\alpha =
d' + dj_* (\alpha)$ satisfies the requirement $ d \varepsilon_\alpha =
d'_\alpha \varepsilon'$. Let $ a'_\alpha$ be the element $
(i'_\alpha)_* (\varepsilon') \in H_q (V'_\alpha)$ where $
i'_\alpha : V_\circ \to v'_\alpha$ is the inclusion. Then from the exact
sequence  
$$
H_{q+1} (V', V_\circ) \xrightarrow{\partial} H_q (V_\circ)
\xrightarrow{(j'_\alpha)_*} H_q (V'_\alpha) \to  0 
$$
we see\pageoriginale that $ (i'_\alpha)_* (d _\alpha \varepsilon') =
(i'_\alpha)_* 
(d \varepsilon_\alpha) = 0$ since $ \partial $ carries the generator $
\varphi'_\alpha (D^{q+1} \times y^* )$ of the relative group $
H_{q+1}(V', V_\circ)$ into the element $\varepsilon_\alpha \in H_q
(V_\circ)$ represented by $ \varphi_\alpha (S^q \times y^*)$. It follows
that $ a' _\alpha$ is of order $|d'. + dj_* (\alpha )|$ with $d' = $
the order of $a' \in H_q (V')$ represented by $\varphi' (y^*
\times S^q)$. 

Identifying the stable group $ \Pi_q (SO_{2q+1 + \ell})$ with $\Pi_q
(SO_{q+2})$ there is an exact sequence associated with the fibration $
SO_{q+2} / SO_{q+1} = S^{q+1}$: 
$$
\Pi_{q+1} ( S^{q+1}) \xrightarrow{\partial} \Pi_q (SO_{q+1})
\xrightarrow{s_*} \Pi_q (SO_{q+2}). 
$$
The composition $ \Pi_{q+1} (S^{q+1}) \xrightarrow{\partial} \Pi_q
(SO_{q+1})  \xrightarrow{j_*} \Pi_q (S^q)$ (for $q$ odd) carries a
generator of $ \Pi_{q+1}(S^{q+1})$ into twice a generator of $
\Pi_q(S^q)$. It follows that $ j_* (\alpha)$ with $\alpha \in
\ker s_*$ can take any even value. (\, $+$ ve or $-$ ve). Thus if $d'$
is not divisible by $d$ we can choose an $\alpha \in \Ker s_*$
such that the order $ | d'_\alpha |$ of $ a'_\alpha$ satisfies
$|d'_\alpha | < d$. Thus we have proved the following  

\begin{lemma}\label{chap1:lem6.3} %lemma 6.3
Let $q$ be odd and $ > 1$ and $\varphi : S^q \times \dfrac{3}{2}
D^{q+1} \to V$ an orientation preserving imbedding representing a
torsion element $ a \in H_q (V)$ of order $d>1$. Then the element
$ a' \in H_q (V')$ represented by $ \varphi' (y^* \times S^q ) $
is of finite order; moreover if $d'$ is the order\pageoriginale  of $
a' $ and if 
$d'$ is not divisible by $d$ then $\exists$ an $\alpha \in \Ker
s_*$ such that the element $ a'_\alpha$ in $ H_q (V'_\alpha) = H_q
(\chi (V, \varphi _\alpha))$ represented by $ \varphi'_\alpha (y^*
\times S^q)$ has order strictly less than that of a in $H_q (V)$.  
\end{lemma}

Next we deal with the case when $d'$ is divisible by $d$. We recall
the definition of linking numbers [Siefert-Threlfall \cite{c1:key7}] Let
$\lambda \in H_p (V)$ and $ \mu \in H_{n-p-1} (V) $ be
torsion classes in the respective groups. Associated with the
coefficient sequence  
$$
0 \to \mathbb{Z} \xrightarrow {h} \mathbb{Q} \to \mathbb{Q}/ 
\mathbb{Z}  \to 0 
$$
we have the exact homology sequence 
$$
\to H_{p+1} (V; \to \mathbb{Q}/ \mathbb{Z}) \xrightarrow{\partial} H_p
(V) \xrightarrow {h_*} H_p (V; \mathbb{Q}) \to \cdots 
$$
($h$ is the inclusion of $\mathbb{Z}$ in $\mathbb{Q}$). Since
$\lambda$ is a torsion element we have $h_*(\lambda ) = 0$. Therefore
$ \exists \nu \in H_{p+1} (V; \mathbb{Q} /\mathbb{Z}) $ such that
$ \partial (\nu ) = \lambda$. The pairing $(\mathbb{Q} /\mathbb{Z})
\otimes \mathbb{Z} \to  \mathbb{Q} /\mathbb{Z}$ defined by
multiplication gives an intersection pairing $H_{p+1} (V; \mathbb{Q}/
\mathbb{Z}) \otimes H_{n-p-1} (V) \to \mathbb{Q} /\mathbb{Z} $. We
denote this pairing by a dot $ '.'$\,. 

\setcounter{definition}{3}
\begin{definition} %definition 6.4
The linking number $L(\lambda, \mu )$ is the rational number modulo
$1$ defined by $ L( \lambda, \mu ) = \nu. \mu$. This linking number is
well-defined and satisfies the relation $L(\mu, \lambda) +
(-1)^{p(n-p-1)} L(\lambda, \mu ) =0 $ [Ref: Siefert-Threlfall \cite{c1:key7}]. 
\end{definition}


\setcounter{lemma}{4}
\begin{lemma}\label{chap1:lem6.5} %lemma 6.5
$L(a, a) = \pm d' d (\mod 1)$.\pageoriginale (This lemma is valid even
  if $d'$ is not divisible by $d$. In fact when $d'$ is divisible by
  $d$ this lemma asserts that $ L(a, a) = 0$).   
\end{lemma}

\begin{proof}
We have $ d \varepsilon - d' \varepsilon' =0$ in $ H_q
(V_\circ)$. Therefore the cycle $d \varphi (S^q \times y^* ) - d' \varphi
' (y^* \times S^q )$ bounds a chain $C$ in $ V_\circ$. Let $ C_1 = \varphi
(y^* \times D^{q+1})$ be the cycle in $\varphi (S^k \times D^{k+1})
\subset V$ with boundary $ \varphi (y^* \times  S^q)$. The chain $ C +
d' C_1$ has boundary $d \varphi (S^q \times y^*)$. Hence $\dfrac{C+ d'
  C_1}{d}$ has boundary $ \varphi (S^q \times y^*)$. Also $\varphi
(S^q \times 0)$ represents the same class $a \in H_q (V)$ as
$\varphi (S^q \times y^*)$. Taking the intersection of $ \dfrac{C+d'
  C_1}{d}$ with $\varphi (S^k \times 0)$ we get $ \pm d' / d$ since
$C$ is  disjoint from $ \varphi (S^k \times 0)$ and $C_1$ has
intersection number $ \pm 1$ with $ \varphi (S^k \times 0)$. Thus $ L
(a, a) = \pm d' / d (\mod 1)$.  
\end{proof}

\begin{lemma}\label{chap1:lem6.6} % lemma 6.6
Let $V = V^{2q+1}$ be a compact oriented $C^\infty$ manifold with
$q>1$ odd, and $f = V \to X$ a map of degree 1 satisfying the
following conditions.  
\begin{enumerate}[(1)]
\item $f_* : H_i (V) \to H_i (X)$ is an isomorphism for $i< q$ 

\item $k_q = \ker f_* : H_q (V) \to H_q (X) $ is a torsion
  group. Suppose further that $L (a, a) = 0$ $\forall$ $a \in
  k_q$. Then $K_q$ is a direct sum of a finite number of copies of
  $\mathbb{Z}_{2} = \mathbb{Z}/_{2 \mathbb{Z}}$. 
\end{enumerate}
\end{lemma}

\begin{remark*}
When\pageoriginale stating this lemma we have a complex $X$ satisfying the
conditions of Theorem \ref{chap1:thm2.1} in our mind. In particular $X$ satisfies
Poincare duality and it is only this that is needed for the validity
of Lemma \ref{chap1:lem6.6}. 
\end{remark*}

\begin{proof}
Since $X$ satisfies Poincare duality for integer coefficients it
follows that $ X$ satisfies Poincare duality for coefficients in any
arbitrary commutative ring. Using the fact that $f$ is of degree 1,
monomorphisms $g_j : H_j (X) \to H_j (V)$ were constructed satisfying
$H_j (V) = \ker f_j \oplus g_j (H_j (x))$ for every $j$ [Lemma \ref{chap1:lem2.5}].
The same procedure can be adopted to define monomorphisms
$g_{j,\wedge} : H_j (X, \wedge) \to H_j (V, \wedge )$ for any
commutative coefficient ring and we still have $H_j (V, \wedge) = \ker
f_{j, \wedge} \oplus g_{j, \wedge }(H_j (X, \wedge))$. Also the exact
sequences in homology corresponding to the exact coefficient sequence $
0 \to \mathbb{Z } \to \mathbb{Q} \to \mathbb{ Q}/\mathbb{Z } \to 0$
give rise to a commutative diagram.  
\end{proof}
\[
\xymatrix@R=1.5cm{
H_{q+1} (V, \mathbb{Q}/\mathbb{Z}) \ar[r]^{\partial} & H_q
(V;\mathbb{Z}) \ar[r]^{h_*} & H_q (
V,\mathbb{Q}) \ar[r]  & \\
H_{q+1} (X, \mathbb{Q}/\mathbb{Z}) \ar[r] \ar[u]_{g_q} & H_q (X,
\mathbb{Z}) \ar[r]^{h_*} \ar[u]_{g_q} & H_q(X, \mathbb{Q}) \ar[r]
\ar[u]_{g_q} & 
}
\]

Let\pageoriginale $T_q (V)$ and $T_q(X)$ denote the torsion subgroups
of $H_q (V)$ 
and $H_q (X)$ respectively. Then from assumption (2) we have $T_q (V)
= K_q \oplus g T_q (X)$. For any $b, b^1 T_q (V)$ let $L (b, b^1)$
denote their linking number. Then since $q$ is odd we have $L (b, b^1)
= L(b^1,b)$.   According to Poincar\'e duality theorem for  torsion group
\cite[p. 245]{c1:key7} $L$ defines a non degenerate pairing $T_q (V) \otimes T_q
(V) \to \mathbb{Q}/\mathbb{Z}$. We claim that $L | K_q \otimes K_q$
gives a non degenerate pairing $K_q \otimes K_q \to
\mathbb{Q}/\mathbb{Z}$. Let $b \in K_q$ satisfy $L(b, b^1) = 0$
$\forall$ $b^1 \in K_q$. We have to show that $L (b,c) = 0 \forall c
\in T_q (V)$. Since $T_q (V) = K_q \oplus g T_q (X)$ we have only
to prove that $L( b,  y) = 0$ $\forall$ $y \in g T_q (X)$. Let $y^1
\in T_q (X)$ be such that $g(y^1) = y$. Then $h_* (y^1) = 0$
(since $y^1$ is a torsion element) and therefore $\exists Z^1 \in
H_{q+1} (X, \mathbb{Q}/ \mathbb{Z})$ such that $\partial Z^1 =
y^1$. The element $Z \in H_{q+1} (V, \mathbb{Q}/ \mathbb{Z})$
given by $Z = g(Z^1)$ satisfies $\partial Z = y$. Now $L (b, y) = L
(y, b) = Z. b$ (this intersection is the one corresponding to the
pairing $(\mathbb{Q}/ \mathbb{Z}) \otimes \mathbb{Z} \to\mathbb{Q}/
\mathbb{Z}$). Thus we have only to verify $K_q . g (H_{q+1}(X,
\mathbb{Q}/ \mathbb{Z}) = 0$. This can be proved in a way similar to
Lemma \ref{chap1:lem5.2}. Thus $L | K_q \otimes K_q \to \mathbb{Q}/\mathbb{Z}$ gives
a nondegenerate pairing.  

We now claim that every element $a \in K_q$ is of order 2. In
fact for any $b \in K_q$ we have $0 = L (a+b, a+b) = L (a, b) + L
(b,a) = L (2a, b)$. Hence $2a = 0$. This completes the proof of Lemma
\ref{chap1:lem6.6}. 

\begin{lemma}\label{chap1:lem6.7} %lemma 6.7
Let\pageoriginale $f : V \to X$ be of degree 1 satisfying the
following conditions.  
\begin{enumerate}[1)]
\item $f_* : H_i (V) \to H_i (X)$ an isomorphism for every $i< q$

\item $K_q = \ker f_q : H_q (V) \to H_q (X)$ a direct sum of a finite
  number of copies of $\mathbb{Z}_2$ and that $\forall$ $a \in K_q$
  the linking number $L (a, a) = 0$.  
\end{enumerate}
Suppose $\varphi : S^q \times \dfrac{3}{2} D^{q+1} \to V$ is an
imbedding representing $a \neq 0$ in $K_q$. Then for the manifold $V'
= \chi (V, \varphi)$ the Bettinumber $b_q(V' ; \mathbb{Z}_2)$ (i.e. the
dimension of $H_q (V' ;  \mathbb{Z}_2))$ satisfies $b_q (V' ;
\mathbb{Z}_2) \not \equiv b_q (V ;  \mathbb{Z}_2 )(\mod 2)$. 
\end{lemma} 

\begin{proof}
Let $W = l \times V \cup_\varphi D^{q+1} \times D^{q+1} $ as in the
proof of Lemma \ref{chap1:lem6.2}. By Lemma \ref{chap1:lem6.1} we have
$e^*(V' :  \mathbb{Z}_2) + 
e^* (V;  \mathbb{Z}_2) + e(W) \equiv \rho (\mod 2)$ where $ \rho$ is
the rank of the intersection bilinear $H_{q+1} (W;  \mathbb{Z}_2)$. If
we show that $ \rho$ is even then as in the proof of Lemma \ref{chap1:lem6.2} it will
follow that $b_q (V' ;  \mathbb{Z}_2 ) \not\equiv b_q (V;
\mathbb{Z}_2) (\mod 2)$. Thus we have only to show that $\rho$ is
even. If for every $x \in H_{q+1} (W;  \mathbb{Z}_2)$ the
intersection $x \cdot x$ is zero then $ \rho$ will be even. Thus we have
only to show that $x \cdot x = 0$ $\forall$ $x \in  H_{q+1} (W;
\mathbb{Z}_2 )$. In the homology exact sequence for the pair $(W, V)$
with $ \mathbb{Z}_2$ coefficients  
$$
H_{q+1} (V;  \mathbb{Z}_2) \xrightarrow{j_*} H_{q+1} (W;
\mathbb{Z}_2) \to H_{q+1} (W, V;  \mathbb{Z}_2) \xrightarrow{\partial}
\to H_q (V; \mathbb{Z}_2) 
$$
the group $H_{q+1} (W, V; \mathbb{Z}_2) \simeq \mathbb{Z}_2$ with
$\varphi (D^{q+1} \times y^*)$ as generator and\pageoriginale
$\partial $ carries it 
into $a \neq 0$ in $H_q (V; \mathbb{Z}_2)$. Actually if we use
$\mathbb{Z}_2$ coefficients and take the kernel $K_q (\mathbb{Z}_2)$
of $f_* : H_q (V; \mathbb{Z}_2) \to H_q (X, \mathbb{Z}_2)$ it will be
isomorphic to $K_q$ since $K_q$ is a direct sum of a finite number of
copies of $\mathbb{Z}_2$ and $f_* : H_j (V) \to H_j (X)$ is an
isomorphism for $j < q$. Hence $\partial : H_{q+1} (W, V ;
\mathbb{Z}_2) \to H_q (V; \mathbb{Z}_2)$ is a monomorphism and
therefore $j_* : H_{q+1} (V; \mathbb{Z}_2)  \to H_{q+1} (W;
\mathbb{Z}_2)$ is onto. It is clear that $x.x =0 $ for elements of the
form $x = j_* (y)$ with $y \in H_{q+1} (V; \mathbb{Z}_2)$ because
a cycle representing $y$ can be deformed in $W$ so as not to intersect
$V$. This completes the proof of Lemma \ref{chap1:lem6.7}. 
\end{proof}

Now we go to the proof of Theorem \ref{chap1:thm2.1} when $n = 2q +1$ with $q \geq
2$. We have already obtained a connected simply connected, compact
oriented $C^\infty$ manifold $M$ of dimension $n$ and a map $f : M \to
X$ of degree 1 satisfying $f : (\eta) \simeq \tau ^n _M \oplus
\mathscr{O}_M^\ell$ and $f_* : H_j (M) \to H_j (X)$ isomorphism for $j
< q$. Let $K_q$ be the Kernel of $f_q : H_q (M) \to H_q (X)$. Let $K_q
= F_q \mathscr{O} T( K_q)$ with $F_q$ free and $T(K_q)$ the torsion
subgroup of $K_q$. Choose an element `$a$' forming part of a basis for
$F_q$. As an easy consequence of Poincare duality we get an element $b
\in H_{q+1} (M)$ such that $a . b =1$. By Lemma \ref{chap1:lem4.3} $\exists a
C^\infty$ imbedding $\varphi : S^q \to M$ representing `$a$' with
trivial normal bundle $\nu_\varphi$ and further satisfying $f \circ
\varphi \sim \tilde{x}^*$ (the constant map). Extending $\varphi$ to
an orientation preserving imbedding $\varphi : S^q \times
\dfrac{3}{2} D^{q+1} \to M$ and performing survey we get a manifold
$\chi (M, \varphi) = M'$\pageoriginale and a map $f' : M' \to X$ of
degree 1 with 
$f'_* : H_j (M') \to H_j (X)$ isomorphisms for $j < q$ and $K'_q =
\ker f'_q : H_q (M') \to H_q (X)$ isomorphic to $K_q / (a)$. (Refer to
case  (i) of Lemma \ref{chap1:lem4.7}). Changing $\varphi$ to $\varphi_\alpha$ if
necessary for a suitable $C^\infty$ map $\alpha :  S^q \to SO_{q+1}$
we may assume $f' ! (\eta ) \simeq \tau^n _{M'} \oplus
\mathscr{O}_{M'}^\ell$ (Proposition \ref{chap1:prop3.7}). Applying surgery successively
to `kill' elements of a basis of $F_q$ we get a connected, simply
connected compact oriented $C^\infty$ manifold $M''$ and a map $f'' :
M'' \to X$ of degree 1 satisfying the following conditions:  
\begin{enumerate}[1)]
\item $f''_* : H_j (M'') \to H_j (X)$ is an isomorphism $\forall j <q$
  and $K''_q = \ker f''_q : H_q (M'') \to H_q (X)$ is precisely the
  torsion subgroup of $K_q$.  

\item $f'' ! (\eta ) \simeq \tau^n _{M''} \oplus
  \mathscr{O}_{M''}^\ell$. 
\end{enumerate}
Thus changing notations we may assume that the original $f : M \to X$
itself satisfied the condition that $K_q$ is a torsion group. Now
assume $q$ even. Choosing an element $a \neq 0$ in $K_q$ and applying
surgery to `kill' $a'$ (this is possible because of Lemma  \ref{chap1:lem4.3}) we
introduce an additional $\mathbb{Z}$ to the kernel, but the torsion
subgroup of the Kernel becomes $K_q / (a)$. (Refer to Lemma \ref{chap1:lem6.2}) But
by our earlier remarks we can successfully apply surgery to kill
$\mathbb{Z}$. In other words by two suitable surgeries on $M$ we can
get a compact,\pageoriginale oriented, connected, simply connected
$C^\infty$ 
manifold $M'$ and a map $f^1 :  M^1 \to X$ of degree 1 with $f' :
(\eta) \approx \tau^n_{M'} \oplus \mathscr{O}_{M^1}^\ell, f^1_* : H_j
(M)  \to  H_j (X)$ isomorphism for $j < q$ and $K'_q = \ker f'_q  :
H_q (M^1) \to H_q (X)$ definitely smaller than $ K_q$. Iteration of
this procedure a finite number of times proves Theorem \ref{chap1:thm2.1} for $n = 2q
+1$ with $q$ even. 

We have still to consider the case $q$ odd. If $a \neq 0$ in $K_q$ is
of order $d$ when we perform surgery by means of an imbedding
$\mathscr{O} : S^q \times \dfrac{3}{2} D^{q+1} \to M$ representing
`$a$' and get $f^1 : M^1 = \chi (M, \varphi ) \to X$ we introduce a
new element of finite order in the kernel of $f^1$. To get $f^1 :
(\eta ) \simeq \tau^n _{M^1} \oplus \mathscr{O}_{M^1}^\ell$ we may have
to alter $\varphi$ into $\varphi_\alpha$ for a suitable $\alpha : S^q
\to SO_{q+1}$ and this can be done by Proposition
\ref{chap1:prop3.7}. We can assume 
that $\varphi$ itself satisfied this requirement also. However if we
change again $\varphi$ to $\varphi _\alpha$ with $\alpha \in \Ker
s_*$ there is no obstruction to getting an isomorphism of
$f^1_{\alpha^!} (\eta )$ with $\tau^n_{M'_\alpha} \oplus
\mathscr{O}_{M^1_\alpha}^\ell$. It is this freedom of choice of
$\alpha$ in $ \Ker s_*$ that helps in proving Theorem \ref{chap1:thm2.1} for $n = 2q
+ 1$ with $q$ odd $> 1$. If the order $d^1$ of $a^1 \in H_q
(M^1)$ represented by $\varphi^1 (y^* \times S^q )$ is not divisible
by $d$ then for a suitable $\alpha \in \Ker s_*$ the element
$a^1_\alpha \in  H_q (M^1_\alpha)$ will have order
strictly\pageoriginale less 
than $d$ (Lemma \ref{chap1:lem6.3}). It follows now from Lemma
\ref{chap1:lem6.5} and \ref{chap1:lem6.6} that we 
can get a manifold $M''$ which is $\chi-$-equivalent to $M$ and a map
$f'' : M'' \to X$ satisfying the following conditions.  
\begin{enumerate}
\item $M''$ is connected, simply connected and $f''$ is of degree 1.

\item $f''_* : H_j (M'') \to H_j (X)$  is an isomorphism for $j < q$;
  the kernel $K''_q$ of $f''_q : H_q (M'') \to H_q (X)$ is a direct
  sum of a finite number of copies of $\mathbb{Z}_2$. 

\item $f'' ! (\eta) \simeq \tau^n_{M''} \oplus \mathscr{O}^\ell_{M''}$.
\end{enumerate}

Lemma \ref{chap1:lem6.7} coupled with the observations made above helps in getting a
manifold $M'''$ which is connected and simply connected and
$\chi$-equivalent to $M''$ and a map $f''' : M''' \to X$ with $f'''_*
: H_j (M''') \to H_j (X)$ isomorphism for $j~~q$  and $f''' ! (\eta)
\simeq \tau_{M'''}^n \oplus \mathscr{O}^\ell_{M'''}$. From the remark
4.5 it follows that $f'''$ is a homotopy equivalence. This completes
the proof of Theorem \ref{chap1:thm2.1}. 

\begin{thebibliography}{99}
\bibitem{c1:key1} {F. Hirzebruch}:\pageoriginale Neue topologische methoden in der
  algebraischen geometrie, Springer 1962. 

\bibitem{c1:key2} {M.A. Kervaire \& J.W.Milnor}: Groups of homotopy spheres,
  1 Ann. Math. Vol. 77, 1963 pp.504-537. 

\bibitem{c1:key3} {M.A. Kervaire \& J.W.Milnor}: On 2-spheres in 4
  -manifolds, Proc. Nat. Acad. Sci. Vol. 47 (1961) pp. 1651-1657. 

\bibitem{c1:key4} {J.W. Milnor}: Lectures on differential topology, Princeton
  University, 1958. 

\bibitem{c1:key5} {J.W. Milnor}: Lectures on characteristic topology,
  Princeton University, 1957. 

\bibitem{c1:key6} {J.W. Milnor}: A procedure for killing homotopy groups of
  differentiable manifolds, Soc. 1961, pp. 39-55. 

\bibitem{c1:key7} {Siefert \& Threlfall}: Lehrbuch der topologie, Chelsea,
  1947. 

\bibitem{c1:key8} {N.E. Steenrod}: Topology of fibre bundles, Princeton
  University Press, 1951.
 
\bibitem{c1:key9} {R. Thom}: Quelques proprietes globales des varietes
  differentiables, Comm. Math. Helv. 1954, pp. 17-85.
 
\bibitem{c1:key10} {H. Whitney}: The self-intersections of a smooth
  n-manifold in 2n-space, Ann. Math., Vol.45, 1944, pp. 220-246. 
\end{thebibliography}

