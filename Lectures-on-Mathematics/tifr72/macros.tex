\usepackage{graphicx,xspace,fancybox}

\newcounter{pageoriginal}
\marginparwidth=15pt
\marginparsep=10pt
\marginparpush=5pt
%\renewcommand{\thepageoriginal}{\arabic{pageoriginal}}
\newcommand{\pageoriginale}{\refstepcounter{pageoriginal}\marginpar{\footnotesize\xspace\textbf{\thepageoriginal}}
} 
\let\pageoriginaled\pageoriginale

\newtheorem{prob}{Problem}
\newtheorem{lem}{Lemma}
\newtheorem{lemma}{Lemma}[section]
\newtheorem{sublemma}{Lemma}[subsection]
\newtheorem{thm}{Theorem}
\newtheorem{theorem}{Theorem}%[section]
\newtheorem{subtheorem}{Theorem}[subsection]
\newtheorem{autothm}{AUTOMORPHISM THEOREM}[subsection]
\newtheorem{proposition}{Proposition}
\newtheorem{prop}{Proposition}[section]
\newtheorem{subprop}{Proposition}[subsection]
\newtheorem{definition}{Definition}[section]
\newtheorem{subdefin}{Definition}[subsection]
\newtheorem{corollary}{Corollary}
\newtheorem{method}{METHOD}
%\newtheorem{coro}{Corollary}[section]
\newtheorem{subcoro}{Corollary}[subsection]
\newtheorem{defn}{Definition}
\newtheorem{gausslemma}{Gauss Lemma}[section]
\newtheorem{proof of}{Proof of}
\newtheorem{Basic Lemma}{Basic Lemma}

%\newtheorem{capdefin}{Definition}[chapter] 
%\newtheorem{chaptheorem}{Theorem}[chapter]
%\newtheorem{chapproposition}{Proposition}[chapter]
%\newtheorem{capproblem}{Problem}[chapter]
%\newtheorem{remarks}{Remarks}
\newtheorem{exercise}{Exercise}
\newtheorem{Lemma}{Lemma}
\newtheorem{Assumption}{Assumption}
%\newtheorem{observations}{Observations}
%\newtheorem{example}{Example}
%\newtheorem{generalizations}{Generalizations}
\newtheorem{unsolved problem}{Unsolved Problem}
\newtheorem{proof of theorem}{Proof of theorem}[section] 


\newtheorem{Prooff}{Proof of Lemma}
\newtheorem{proof of lemma}{Proof of lemma}
\newtheorem{proof of the lemma}{Proof of the lemma}
\newtheorem{proofofthelemma}{Proof of the lemma}

%\newtheorem{the exchange property}{The Exchange Property}[chapter]

%\newtheorem{sufficiency}{Sufficiency}
\DeclareMathOperator*{\uniformly}{uniformly}
\DeclareMathOperator{\supp}{supp}
\DeclareMathOperator{\diam}{diam}
\DeclareMathOperator{\Rep}{Rep}
\DeclareMathOperator{\loc}{loc}
\DeclareMathOperator{\Hom}{Hom}
\DeclareMathOperator{\re}{Re}
\DeclareMathOperator{\Inf}{Inf}
\DeclareMathOperator{\tr}{tr}
\DeclareMathOperator{\Tr}{Tr}
\DeclareMathOperator{\ass}{Ass}
\DeclareMathOperator{\im}{Im}
\DeclareMathOperator{\mes}{mes}
\DeclareMathOperator{\spec}{Spec}
\DeclareMathOperator{\resp}{resp}
\newtheorem{warning}{WARNING}
\newtheorem{Case}{Case}
\newtheorem{case}{Case}
\newtheorem{step}{Step}
\newtheorem{Axiom}{Axiom}
\newtheorem{question}{QUESTION}
%\newtheorem{remark}{Remark}
\newtheoremstyle{nonum}{}{}{\itshape}{}{\bfseries}{.}{ }{#1}
\theoremstyle{nonum}
\newtheorem{lemma0}{Lemma}
\newtheorem{problem}{Problem}
\newtheorem{Problem}{Problem}
\newtheorem{sketch of proof of Theorem}{Sketch of proof of Theorem}[chapter]
\newtheorem{proof of the corollary}{Proof of the Corollary}
\newtheorem{fact}{Fact}
\newtheorem{facts}{Facts}
\newtheorem{proof of the theorem}{Proof of the theorem}
\newtheorem{corollary to theorem}{Corollary to Theorem}
\newtheorem{proof of ex.}{Proof of ex.}[chapter]
\newtheorem {key lemma}{Key Lemma}
\newtheorem{uniqueness}{Uniqueness}

\newtheoremstyle{remark}{10pt}{10pt}{ }%
{}{\bfseries}{.}{ }{}
\theoremstyle{remark}

\newtheorem{Step}{Step}
\newtheorem{remark}{Remark}[section]
\newtheorem{remarks}{Remarks}[section]
\newtheorem{rem}{Remark}
\newtheorem{subremarks}{Remarks}[subsection]
\newtheorem{subremark}{Remark}[subsection]
\newtheorem{example}{Example}[section]
\newtheorem{subexample}{Example}[subsection]
\newtheorem{exam}{Example}
\newtheorem{examples}{Examples}[section]
\newtheorem{note}{Note}[section]
\newtheorem{claim}{Claim}[section]
\newtheorem{application}{Application}[section]
\newtheorem{proofoflemma}{Proof of Lemma}[section]
\newtheorem{Definition}{DEFINITION}[chapter]
\newtheorem{proofofsublemma}{Proof of Lemma}[subsection]
\newtheorem{proofofprop}{Proof of Proposition}
\newtheorem{prfofprop}{Proof of Proposition}
\newtheorem{proofoftheorem}{Proof of theorem}
\newtheorem{proofofthm}{Proof of theorem}[section]
\newtheorem{proofcontd}{Proof continued}[subsection]
\newtheorem{Proposition}{PROPOSITION}[chapter]
\newtheorem{Exercise}{EXERCISE}[chapter]
\newtheorem{experiment}{Experiment}
\newtheorem{Conjecture}{Conjecture}

	
\newtheoremstyle{nonum}{}{}{\itshape}{}{\bfseries}{\kern -2pt{\bf.}}{ }{#1 \mdseries
{\bf #3}}
\theoremstyle{nonum}
\newtheorem{Proof}{PROOF}
\newtheorem{conjecture}{CONJECTURE}
\newtheorem{lemma*}{Lemma}	
\newtheorem{theorem*}{Theorem}	
\newtheorem{Theorem}{THEOREM}[section]
\newtheorem{irreducibilitythm*}{IRREDUCIBILITY THEOREM}
\newtheorem{genirreducibilitythm*}{GENERIC IRREDUCIBILITY THEOREM}
\newtheorem{embedthm*}{EMBEDDING THEOREM}
\newtheorem{prop*}{Proposition}
\newtheorem{first step}{First Step}
\newtheorem{second step}{Second Step}
\newtheorem{third step}{Third Step}

\newtheorem{claim*}{Claim}
\newtheorem{notation}{Notation}
\newtheorem{defi*}{Definition}
\newtheorem{conjecture*}{CONJECTURE}
\newtheorem{coro*}{Corollary}
\newtheoremstyle{mynonum}{}{}{ }{}{\bfseries}{\kern -2pt{\bf.}}{ }{#1 \mdseries
{\bf #3}}
\theoremstyle{mynonum}
\newtheorem{remark*}{Remark}	
\newtheorem{remarks*}{Remarks}	
\newtheorem{exer*}{Exercise}	
\newtheorem{example*}{Example}	
\newtheorem{examples*}{Examples}	
\newtheorem{note*}{Note}
\newtheorem{proofoflemma*}{Proof of Lemma}
\newtheorem{proofofthm*}{Proof of Theorem}
\newtheorem{extension}{Extension}

\def\ophi{\overset{o}{\phi}}

\def\oval#1{\text{\cornersize{2}\ovalbox{$#1$}}}


\newcommand*\mycirc[1]{%
  \tikz[baseline=(C.base)]\node[draw,circle,inner sep=.7pt](C) {#1};\:
}

\DeclareMathOperator{\dxd}{dxd}
\DeclareMathOperator{\inz}{inz}
\DeclareMathOperator{\izy}{izy}
\DeclareMathOperator{\ixy}{ixy}
\DeclareMathOperator{\intt}{int}
\DeclareMathOperator{\spm}{Spm}

\DeclareMathOperator{\Min}{Min}
\DeclareMathOperator{\const}{Const}
\DeclareMathOperator*{\strong}{strong}

\DeclareMathOperator{\id}{id}
\DeclareMathOperator{\Id}{Id}
\DeclareMathOperator{\df}{df}
\DeclareMathOperator{\pr}{pr}
\DeclareMathOperator{\Df}{Df}
\DeclareMathOperator{\Dg}{Dg}
\DeclareMathOperator{\Ad}{Ad}
\DeclareMathOperator{\Sp}{Sp}
\DeclareMathOperator{\sh}{sh}
\DeclareMathOperator{\dt}{dt}
\DeclareMathOperator{\hyp}{hyp}
\DeclareMathOperator{\Hyp}{Hyp}
\DeclareMathOperator{\can}{can}
\DeclareMathOperator{\Ext}{Ext}
\DeclareMathOperator{\red}{Red}
\DeclareMathOperator{\rank}{rank}
\DeclareMathOperator{\codim}{codim}
\DeclareMathOperator{\grad}{grad}
\DeclareMathOperator{\rad}{Rad}
\DeclareMathOperator{\Spin}{Spin}
\DeclareMathOperator{\Spec}{Spec}
\DeclareMathOperator{\Ker}{Ker}
\DeclareMathOperator{\ring}{ring}
\DeclareMathOperator{\de}{def}
\DeclareMathOperator{\adj}{adj}
\DeclareMathOperator{\Vol}{Vol}
\DeclareMathOperator{\vol}{vol}
\DeclareMathOperator{\Int}{Int}
\DeclareMathOperator{\End}{End}
\DeclareMathOperator{\Ric}{Ric}
\DeclareMathOperator{\Trace}{Trace}
\DeclareMathOperator{\Char}{char}
\DeclareMathOperator{\mult}{mult}
\DeclareMathOperator{\Supp}{Supp}
\DeclareMathOperator{\Sup}{Sup}
\DeclareMathOperator{\Sin}{Sin}
\DeclareMathOperator{\Aut}{Aut}
\DeclareMathOperator{\Pic}{Pic}
\DeclareMathOperator{\Cl}{Cl}
\DeclareMathOperator{\ca}{ca}
\DeclareMathOperator{\cu}{cu}
\DeclareMathOperator{\idd}{id.}
\DeclareMathOperator{\transdeg}{trans{.}deg}
\DeclareMathOperator*{\Lt}{Lt}
\DeclareMathOperator*{\lt}{lt}
\DeclareMathOperator*{\lint}{\int\cdots\int}
\DeclareMathOperator*{\subs}{\subset}
\DeclareMathOperator*{\0pt}{0pt}


\def\uub#1{\underlie{\underline{#1}}}
\def\ub#1{\underline{#1}}
\def\oob#1{\overline{\overline{#1}}}
\def\ob#1{\overline{#1}}


\font\bigsymb=cmsy10 at 4pt
\def\bigdot{{\kern1.2pt\raise 1.5pt\hbox{\bigsymb\char15}}}
\def\overdot#1{\overset{\bigdot}{#1}}

\makeatletter
\renewcommand\subsection{\@startsection{subsection}{2}{\z@}%
                                     {-3.25ex\@plus -1ex \@minus -.2ex}%
                                     {-1.5ex \@plus .2ex}%
                                     {\normalfont}}%

%\renewcommand\thesection{\@arabic\c@section}
\renewcommand\thesubsection{({\thechapter.\thesection.\@arabic\c@subsection})}

\renewcommand{\@seccntformat}[1]{{\csname the#1\endcsname}\hspace{0.3em}}
\makeatother

\def\fibreproduct#1#2#3{#1{\displaystyle\mathop{\times}_{#3}}#2}
\let\fprod\fibreproduct

\def\fibreoproduct#1#2#3{#1{\displaystyle\mathop{\otimes}_{#3}}#2}
\let\foprod\fibreoproduct


\def\cf{{cf.}\kern.3em}
\def\Cf{{Cf.}\kern.3em}
\def\eg{{e.g.}\kern.3em}
\def\ie{{i.e.}\kern.3em}
\def\iec{{i.e.,}\kern.3em}
\def\idc{{id.,}\kern.3em}
\def\resp{{resp.}\kern.3em}


\def\bA{\mathbf{A}}
\def\bB{\mathbf{B}}
\def\bC{\mathbf{C}}
\def\bD{\mathbf{D}}
\def\bE{\mathbf{E}}
\def\bF{\mathbf{F}}
\def\bG{\mathbf{G}}
\def\bH{\mathbf{H}}
\def\bI{\mathbf{I}}
\def\bJ{\mathbf{J}}
\def\bK{\mathbf{K}}
\def\bL{\mathbf{L}}
\def\bM{\mathbf{M}}
\def\bN{\mathbf{N}}
\def\bO{\mathbf{O}}
\def\bP{\mathbf{P}}
\def\bQ{\mathbf{Q}}
\def\bR{\mathbf{R}}
\def\bS{\mathbf{S}}
\def\bT{\mathbf{T}}
\def\bU{\mathbf{U}}
\def\bV{\mathbf{V}}
\def\bW{\mathbf{W}}
\def\bX{\mathbf{X}}
\def\bY{\mathbf{Y}}
\def\bZ{\mathbf{Z}}

%\def\qedsymbol{\textup{Q.E.D.}}
%\let\QED\qed


\newcommand\Bpara[4]{%
\begin{picture}(0,0)%
        \setlength{\unitlength}{1pt}%
        \put(#1, #2){\rotatebox{#3}{\raisebox{0mm}[0mm][0mm]{%
        \makebox[0mm]{$\left\{\rule{0mm}{#4pt}\right.$}}}}%
\end{picture}}



\renewcommand\chaptermark[1]{\markboth{\thechapter. #1}{}}
\renewcommand\sectionmark[1]{\markright{\thesection. #1}}

