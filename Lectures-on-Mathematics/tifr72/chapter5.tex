\chapter{Zero-Density Theorems}\label{chap5}%chap V

THE\pageoriginale MAIN OBJECT of the present chapter is to prove in detail a
well-known zero density estimate for the Riemann zeta-function which
will be required in the next chapter. The argument has nothing to do
with the sieve methods developed in PART \ref{part1}, but the result will be
combined with the linear sieve to produce a deep consequence on the
difference between consective primes. 

We shall prove also a zero-density estimate of the Linnik type for
Dirichlet's $L$-functions; there the hybrid dual sieve for intervals
will play an impotant r$\hat{o}$le, and we shall have a nice instance
of a fruitful unification of sieve methods and analytic methods. 

\section{A Zero-Density Estimate for
  $\zeta(s)$}\label{chap5-sec5.1}%sec 5.1 

As usual, we denote by $N(\alpha, T)$ the number of zeros $\rho =
\beta + i \gamma$ of $\zeta (s)$ satisfying $\alpha \leq \beta \leq 1,
| \gamma |  \leq T$. It is expedient to consider the estimate of $N
(\alpha, T) - N (\alpha, T/2)$ instead of $N (\alpha,  T)$; so we
assume henceforth that $\rho$ satisfies  
\begin{equation*}
 \alpha \leq \beta \leq 1, \frac{T}{2} \leq | \gamma | \leq T,
 \tag{5.1.1}\label{eq5.1.1} 
\end{equation*}
and that $T$ is sufficiently large.

We\pageoriginale divide our discussion into three parts according to
the value of $\alpha$ 

\begin{Case}\label{chap5-case1}%case 1
$0 \leq \alpha \leq 3/4$.
\end{Case}

For a while, we assume further that 
\begin{equation*}
  \frac{1}{2} + (\log T)^{-1} \leq \alpha \leq
  \frac{3}{4}. \tag{5.1.2}\label{eq5.1.2} 
\end{equation*}

Let $x$, $y$ be two parameters such that $2 \leq x \leq y, \log xy = 0
(\log T)$, and put   
$$
M(s) = \sum_{n < x} \mu(n)n^{-s}.
$$

Then considering the Mellin integral 
$$
\frac{1}{2 \pi i} \int\limits^{2 + i \infty}_{2- i \infty} \zeta(\rho
+w) M(\rho + w) r (w) y^w dw 
$$
we get 
\begin{multline*}
  e^{-1/y} + \sum_{n \geq x}a (n) n^{-\rho} e^{-n/y}
  = M(1)y^{1- \rho } r (1 -\rho) \\
  + \frac{1}{2 \pi} \int\limits^{\infty}_{-\infty} \zeta
  \left(\frac{1}{2}+ i (\gamma + u)\right) M\left(\frac{1}{2} + i
  (\gamma + u)\right) y^{\frac{1}{2} - \beta + iu}
  r\left(\frac{1}{2}- \beta + iu \right) du,
\end{multline*}
where we should observe \eqref{eq5.1.2}, and $a(n) = \sum \limits_{\substack{
    d | n \\ {d < x}}} \mu (d)$.  On the left side, we may truncte the
sum at $y(\log T)^2$ with a negligible error; on the right side, the
first term is negligible while the  integral may be truncted at $u =
\pm (\log T)^2$. Thus we have   
\begin{multline*}
  1<< | \sum_{ x \leq n \leq y (\log T^2)} A^{ A (n) n^{-\rho_e - n/y}}|\\
  + y^{\frac{1}{2} -\alpha} \log T \int\limits^{(\log T)^2}_{-(\log
    T)^2}| \zeta \left(\frac{1}{2}+ i(\gamma + u)\right)
  M\left(\frac{1}{2} + i(\gamma + u)\right) |
  du. \tag{5.1.3} \label{eq5.1.3} 
\end{multline*}\pageoriginale
 
Next, from each each horizontal strip 
$$
2n + \nu \leq t < 2n + \nu +1(\nu = 0, ; n = 0, \pm 1, \pm 2, \ldots), 
$$
we pick up a zero of $\zeta (S)$ satisfying \eqref{eq5.1.1}, and let
$\mathscr{R}_{\nu}$ be the resulting set of zeros. Then we have  
\begin{equation*}
  N(\alpha, T) -N \left(\alpha,  \frac{T}{2}\right) \ll ( | \mathscr{R}_0 |  +
  |\mathscr{R}_1 | ) \log T \tag{5.1.4} \label{eq5.1.4}
\end{equation*}
since, as is well-known, $N(0, u + 1) - N(0, u) \ll \log (u + 2)$. Hence
it suffices to estimate $ | \mathscr{R}_{\nu} |$. By \eqref{eq5.1.3}
and H\"older's inequality, we have  
\begin{multline*}
  | \mathscr{R}_{\nu} | \ll | \mathscr{R}_{\nu} |^{\frac{1}{2}} \left\{
  \sum_{\rho \epsilon  \mathscr{R}_{\nu}}| \sum_{x \leq n \leq y
    (\log T^2)}a (n) n^{\rho} e^{-n/y}|^2 \right\}^{\frac{1}{2}}
  + y^{\frac{1}{2} -\alpha} \log T | \mathscr{R}_{\nu} |^{\frac{1}{4}}\\
  \int \limits^{(\log T)^2}_{-(\log T)^2} \left\{ \sum_{\rho \epsilon
    \mathscr{R}_\nu} |\zeta \left(\frac{1}{2} + i
  (\gamma +u)\right)^4 | \right\}^{\frac{1}{4}} \left\{ \sum_{\rho \epsilon 
    \mathscr{R}_{\nu}} | M \left(\frac{1}{2}+i(\gamma + u)\right)|^2
  \right\}^{\frac{1}{2}}du. \tag{5.1.5} \label{eq5.1.5} 
\end{multline*}

To proceed further, we require discrete mean-value theorems for the
Riemann zeta-function and Dirichlet polynomials: 

\begin{Lemma}\label{chap5-lem23}%lemma 23
  \begin{enumerate}[\rm (i)]
  \item Let $\{ t_r \}$ be a set of real numbers such that $ | t_r |
    \leq T$ and $| t_r -t_{r'} | \geq \delta > 0$ for $r \neq r'$. Then
    we have 
    $$
    \sum_r | \zeta \left(\frac{1}{2} + it_r\right) |^4 << (\delta^{-1} +
    \log T) T(\log T)^4. 
    $$
  \item Let\pageoriginale $\{ s_r \}$ be set of complex numbers such
    that $Re(s_r)\geq 0, | Im (s_r) | \leq T$, and $| Im (s_r) - Im
    (s_r') | \geq \Delta > 0$ for $r \neq r'$. Then we have, for
    arbitrary complex numbers $\{a_m\}$, 
    $$
    \sum_r | \sum_{M < m \leq 2M} a_m m^{-s_r} |^2 \ll (\delta^{-1} +
    \log M) (T + M) \sum_{M < m \leq 2M} | a_m |^2. 
    $$

  \item Let $\{ s_r \}$ be as above. Then we have also
    $$
    \sum_r | \sum_{M< m \leq 2M} a_m m ^{-s_r} |^2 << (\Delta^{-1}) (M +
    | \{ s_r \} |T^{\frac{1}{2}} \log T) \sum_{M < m \leq 2M} | a_m
    |^2. 
    $$
  \end{enumerate} 
\end{Lemma}

Applying (i) and (ii) of this lemma to \eqref{eq5.1.5}, we get readily
\begin{align*}
|\mathscr{R}_{\nu}|  &<<  |\mathscr{R}_{\nu} |^{\frac{1}{2}}
\left(y^{2(1-\alpha )} + T x^{1-2 \alpha}\right)^{\frac{1}{2}} \log^c T\\ 
  & \quad  + |\mathscr{R}_{\nu}|^{\frac{1}{4}}y^{\frac{1}{2} -\alpha}
  T^{\frac{1}{4}}(x + T)^{\frac{1}{2}} \log^c T. 
\end{align*}

Namely, we have 
$$
| \mathscr{R}_{\nu} | << \left(y^{2(1 -\alpha)} + Tx^{1-2 \alpha} +
T^{\frac{2}{3}} y^{\frac{2}{4} (1-2 \alpha)}\right) \log^c T; 
$$
in this, we set 
$$
x = T, y = T^{\frac{2}{2 (2- \alpha)}},
$$
getting
$$
| \mathscr{R}_{\nu} | \ll T^{\frac{3}{2- \alpha}(1- \alpha)} \log^cT.
$$

This\pageoriginale and \eqref{eq5.1.4} yield 
\begin{equation*}
  N (\alpha, T ) \ll {\frac{3}{T^{2 - \alpha}}(1- \alpha )} \log^c T
  \tag{5.1.6} \label{eq5.1.6}
\end{equation*}
in which we may now neglect \eqref{eq5.1.2} by an obvious reason.

\begin{Case}\label{chap5-case2}%case 2
  $5/6 \leq \alpha \leq 1$.
\end{Case}

We proceed as before, up to \eqref{eq5.1.3}: But this time, we set
there  
\begin{equation*}
  x = T^{\frac{6 \alpha - 5 }{12(3 \alpha -1)}} + \eta,  y =
  T^{\frac{3 }{4 (3\alpha -1 )}+2}  \eta  \tag{5.1.7} \label{eq5.1.7}
\end{equation*}
with a small fixed $\eta > 0$. Then recalling the well-known estimate
$\zeta (\frac{1}{2} + it) << ( | t | + 1)^{1/6} \log ( | t | + 2)$, we
see that the second term on the right side \eqref{eq5.1.3} is $(T^{-
  \eta/10})$. Hence we have  
$$
1 << | \sum_{x \leq n \leq y (\log T)^2} a (n) n^{-\rho}e^{-n/y} |.
$$

From this, we can infer that there exist an $N, x \leq N \leq y (\log
T)^2$ and a subset $\mathscr{R}^{(1)}_{\nu}$ of $\mathscr{R}_{\nu}$
such that  
\begin{equation*}
  |\mathscr{R}_{\nu} | <<_{\eta} | \mathscr{R}^{(1)}_{\nu} | \log T
  \tag{5.1.8}\label{eq5.1.8} 
\end{equation*}
and for all $\rho \epsilon  \mathscr{R}^{(1)}_{\nu}$
\begin{equation*}
(\log T)^{-1} <<_{\eta} | \sum_{ N \leq n \leq 2N} a (n)
  n^{-\rho}e^{-n/y} |.  \tag{5.1.9} \label{eq5.1.9}
\end{equation*}

To proceed further, we require a large-value theorem of Dirichalet
polynomials. 

\begin{Lemma}\label{chap5-lem24}%lemma 24 
Let\pageoriginale $\{ s_r \}$ satisfy the condition given in (ii) of
the previous lemm. And let us assume that there is a $V > 0$, such
that for all $r$ 
  $$
  V< | \sum_{M < m \leq 2M} a_m m^{-s_r}.
  $$
  Then we have 
  $$
  | \{s_r \} | << (1 + \delta^{-1})^3 M (V^{-2}G + V^{-6}T G^3 \log^2T)
  $$
  where
  $$
  G =| \sum_{M < m \leq 2M} | a_m |^2.
  $$
\end{Lemma}

Before applying this to our situation, we choose the integer $k$ such
that for the $N$ of \eqref{eq5.1.9} 
\begin{equation*}
  N^{k-1} < T^{\frac{1}{3 \alpha -1}} \leq N^k; \tag{5.1.10}\label{eq5.1.10}
\end{equation*}
obviously, we have $2 \leq k <<_{\eta} 1$. Then we raise the both
sides of \eqref{eq5.1.9} to $2k$-th power, and use LEMMA
\ref{chap5-lem24}, getting   
$$
| \mathscr{R}^{(1)}_{\nu} | <<_{\eta} \left(N^{2k(1 - \alpha)}+ (TN^{2k
  (2-3\alpha)}\right) \log^{c(\eta)}T. 
$$

Now if $k = 2$, then we have, by \eqref{eq5.1.7} and \eqref{eq5.1.10},
$$
T^{\frac{2}{3\alpha - 1}} \leq N^4 < y^4 (\log T)^8 \leq
T^{\frac{3}{3\alpha - 1} + 9 \eta}, 
$$
and if $k \geq 3$, then \eqref{eq5.1.10} implies
$$
T^{\frac{2}{3 \alpha -1}} \leq N^{2k} < T^{\frac{2}{3 \alpha -1}\left(1+
  \frac{1}{k-1}\right)}  \leq  T^{\frac{3}{3 \alpha-1}}. 
$$

Hence\pageoriginale we get 
$$
\mathscr{R}^{(1)}_{\nu} <<_{\eta} T^{\left(\frac{3}{3 \alpha-1} + 9 \eta\right)
  (1-\alpha)} \log^{c (\eta)}T. 
$$

Thus, by \eqref{eq5.1.4} and \eqref{eq5.1.8}, we obtain
\begin{equation*}
N(\alpha, T) << T^{\left(\frac{3}{3 \alpha -1} + \epsilon  \right) (1-\alpha )}
\log^{c(\epsilon  )}T. \tag{5.1.11} \label{eq5.1.11}
\end{equation*}

\begin{Case}\label{chap5-case3}%case 3
$3/4 \leq \alpha \leq 5/6$.
\end{Case}

In this case, we require a zero-detecting method different from the
above. We note first that, by an elementary consideration, we can
confine ourselves to those zeros $\rho = \beta + i \gamma$ of $\zeta
(s)$ that satisfy \eqref{eq5.1.1} and  
\begin{equation*}
  \zeta (s) \neq 0 \text{ for } \alpha + \eta^4 \leq \sigma \leq 1, |
  t- \gamma | < \log^2 T, \tag{5.1.12} \label{eq5.1.12}
\end{equation*}
where $\eta$ is a small positive parametr to be fixed later. We pick
up such a zero which lies in one of the horizontal strips $ 2n + \nu
\leq t< 2n + \nu +1 (\nu = 0,1; n = 0, \pm 1, \pm 2, \ldots)$, and let
$\tilde{\mathscr{R}_{\nu}}$ be the obtained set of zeros. Then we have  
$$
N(\alpha,T) - N(\alpha,  T/2) <<_{\eta}( | \tilde{\mathscr{R}_{0}} | +
| \tilde{\mathscr{R}_{1}} | ) \log^{c(\eta)}T. 
$$

Also, we remark that \eqref{eq5.1.12} implies 
\begin{equation*}
  \zeta (s)^{-1} = 0_{\eta}(T^{\eta^5}) \tag{5.1.13}\label{eq5.1.13}
\end{equation*}
in the region $\alpha + 2 \eta^4 \leq \sigma,  | t- \gamma | <
\dfrac{1}{2}(\log T)^2$. 

Then we put 
$$
N(s) = \sum^{\infty}_{n=1} \frac{\mu (n)}{n^s} \exp
\left(-\left(\frac{n}{x}\right)^2\right) 
$$
where $x$ is to be chosen later, and for a while, we assume only $2
\leq x \leq T$.\pageoriginale We have, for $Re(s) \leq 3/2$, 
$$
N(s) = \frac{1}{2 \pi i} \int \limits^{2 + i \infty}_{2- i \infty} 
\frac{1}{\zeta(W)} r \left( \frac{w-s}{2}\right) x^{w-s} dw.  
$$

In this formula, let us confine $s$ to the region $\sigma \leq \alpha
+ \eta^4, | t - \gamma | \leq \frac{1}{8}(\log T)^2$, and then shift
the line of integration to the broken line $\ni w = u + iv: u=2$, $| v-
\gamma | \geq \frac{1}{4}(\log T)^2$; $\alpha + 2 \eta^4 \leq u \leq 2$,
$v = \gamma  \pm \frac{1}{4}(\log T)^2$; $u = \alpha + 2 \eta^4$, $| v-
\gamma | \leq \frac{1}{4}(\log T)^2$. Then, by \eqref{eq5.1.13}, we get  
\begin{equation*}
  N(s) = 0_{\eta}\left(x^{\alpha + 2 \eta^4 - \sigma}
  T^{\eta5}\right).\tag{5.1.14}\label{eq5.1.14} 
\end{equation*}

Next, we consider the function
$$
\displaylines{\hfill 
  N(s) \zeta (s) = \sum^{\infty}_{n=1} b (n) n^{-s} (\sigma >1)
  \hfill \cr
  \text{where}\hfill
  b(n) = \sum_{d | n} \mu (d) \exp \left(
  -\left(\frac{d}{x}\right)^2\right).\qquad \quad \hfill} 
$$

As before, we consider the expression 
\begin{equation*}
  \sum^{\infty}_{n=1} b (n) n^{-\rho}e^{-n/y}
  = \frac{1}{2 \pi i } \int \limits^{2+ i \infty}_{2-i \infty} \zeta (s)
  N(s) r (s- \rho ) y^{s - \rho}ds \tag{5.1.15} \label{eq5.1.15}
\end{equation*}
where $\rho$ satisfies \eqref{eq5.1.12}. Again, we shift the line of
integration to the broken line $\ni s: \sigma  = 2, | t - \gamma |
\geq \dfrac{1}{8}(\log T)^2; \alpha + \eta^4 - \eta^2 \leq
\sigma \leq 2, t = \gamma \pm \dfrac{1}{8}(\log T)^2; \sigma = \alpha
+  \eta^4 - \eta^2, | t - \gamma | \leq \dfrac{1}{8}(\log T)^2$. 

Then using \eqref{eq5.1.14}, we see that the last integral is                 
$$
\displaylines{\hfill 
  0_\eta \left(y^{\eta^4 - \eta^2} x^{\eta^{4} + \eta^{2}} T^{\eta^5}
  C_p (T) \right),\hfill \cr
  \text{where}\hfill 
  C_p (T) = \max_{|t- \gamma| \leq \frac{1}{8}(\log T)^2} |\zeta (\alpha
  + \eta^4 - \eta ^2 + it)|.\hfill } 
$$\pageoriginale

But by a convexity argument we can show that
\begin{equation*}
  C_p (T) \ll_\eta T^{\frac{\eta^2}{2}+ \eta ^5},\tag {5.1.16}\label{eq5.1.16}
\end{equation*}
which implies that the  right side of \eqref{eq5.1.15} is $0_{\eta
}(1)$ if we set 
\begin{equation*}
  y = T^{\frac{1}{2}+ \eta}, x T^{\eta ^{4}}. \tag{5.1.17}\label{eq5.1.17}
\end{equation*}

Thus we put these values of $x$, $y$ in \eqref{eq5.1.15}; we can
truncate the sum 
on the left side at $y (\log T)^2$, and noting that $b (1) = 1 + 0
(x^{-2})$ and $b(n) = 0 (( n /x)^2)$ for $1<n\leq x$, we can infer
that 
$$
1 \ll _\eta | \sum_{x (\log T)^{-2} \leq n \leq y (\log T)^2 } b (n) n
^{- \rho _{e}-n /y}| 
$$
for all $\rho$ which satisfy \eqref{eq5.1.12}.

Then, as in CASE \ref{chap5-case2},  we have an $N,  x (\log T)^{-2}
\leq N \leq y (\log T)^2$, for which there exists a subset
$\tilde{\mathscr{R}_{\nu} (1)}$ of $\tilde{\mathscr{R}_{\nu}}$ such that
$|\tilde{\mathscr{R}}_{\nu} (1) | \gg |\tilde{\mathscr{R}}_{\nu}|\break
(\log T)^{-1}$ and for all $\rho \epsilon  \tilde{\mathscr{R}}_{\nu}
(1)$ 
$$
(\log T)^{-1} \ll _{\eta} |\sum_{N < n \leq 2N} b(n) n ^{- \rho _e -n /y}|.
$$

We\pageoriginale raise the both sides of this to $2k - th$ power so that $N ^{k-1} <
T^{\frac{1}{3 \alpha -1}} \leq N^k $. Obviously,  we have $2\leq k
\ll_{\eta} 1$. If $k \geq 3$, we can argue as before,  and get
\eqref{eq5.1.11} for our present value of $\alpha$. But, if $k= 2$, we have 
$$
T^{\frac{1}{3 \alpha -1}} \leq N^2 \leq T^{1 + 2\eta}(\log T)^4
$$
because of \eqref{eq5.1.17},  and using LEMMA \ref{chap5-lem24} we obtain
\begin{align*}
  |\tilde{\mathscr{R}}_{\nu}^(1) | &  \ll _{\eta} (\log T)^c \left(T ^{2 (1
    + 2\eta ) (1 - \alpha )}+ T^{\frac{3}{3 \alpha -1} (1-\alpha )}\right)\\ 
  &  \ll _{\eta} (\log T)^c T ^{\left(\frac{3}{3 \alpha -1} + 4 _{\eta} \right)(1 -\alpha )}
 \end{align*} 
  since $3/ (3 \alpha -1) \geq 2 $ for $\alpha \geq 5/6$. Thus we get
  \eqref{eq5.1.11} in CASE \ref{chap5-case3} as well. 
 
 Finally, taking into account the zero-free region of Vinogradov, we
 may summarize the above discussion as  

\begin{theorem}\label{chap5-thm14}
For $ 0 \leq \alpha \leq 1$, we have
$$
N(\alpha, T) \ll T^{(\phi (\alpha) + \epsilon ) (1 - \alpha)}
$$
where
$$
\phi (\alpha)= 
  \begin{cases}
    \frac{3}{3 \alpha -1} & {\rm if}~ 3/4 \leq \alpha \leq 1,\\
    \frac{3}{2-\alpha}  &{\rm if}~ 0 \leq \alpha \leq 3/4.
  \end{cases}
$$
\end{theorem} 
 
 The zero-density result which we shall require in the  next chapter
 is not this theorem but rather the following consequence of it. 
\begin{Lemma}\label{chap5-lem25}
Let\pageoriginale $\{a_n\}$ be arbitrary complex numbers such that
$|a_n | \ll  1$, and put  
   $$
   K(s) = \sum_{K < n \leq 2K} a_n n^{-s}.
   $$

   Then we have,  for  $0 \leq \alpha \leq 1$,
   $$
   \sum_{\substack{\rho \\ |\gamma | \leq T \\ \beta \geq \alpha}} |K (\rho)| \ll
   \begin{cases}
     \left(T^{\frac{6}{5}+ \epsilon  } K\right)^{1-\alpha} & {\rm
       if}~ T \leq K \leq T^c,\\ 
     \left(\frac{T^{\frac{16}{5}+ \epsilon }}{K}\right)^{1- \alpha} & {\rm if}~
     T^{\frac{4}{5}} \leq K \leq T 
   \end{cases} 
   $$
   where $\rho = \beta + i \gamma$ is a complex zero of $\zeta (s)$. 
 \end{Lemma}

To prove this, let us choose a $\rho $ among those in the rectangle
$2n + \nu \leq t < 2n + \nu +1, \alpha \leq \sigma \leq 1, (\nu =
0,1)$, for which $|K(\sigma)|$ is the greatest, and $\mathscr{R}'_\nu$
be the obtained set of zeros. Then we have obviously 
$$
 \sum_{\substack{\rho \\ |\gamma | \leq T \\ \beta \geq \alpha}} |K
 (\rho)| \ll \log T \left\{ \sum_{\rho \epsilon  \mathscr{R'}_0} + \sum_{\rho
   \epsilon  \mathscr{R'}_1}\right\} |K (\rho )|. 
$$

We have, by Schwarz's inequality,
$$
  \sum_{\rho \epsilon  \mathscr{R'}_{\nu}} |K(\rho )|  \leq N(\alpha
,  T)^{\frac{1}{2}} \left\{ \sum_{\rho \epsilon  \mathscr{R'}_\nu}
  |K(\rho)|^2 \right\}^{\frac{1}{2}}.
$$

If $K \geq T$, then, by (ii) of LEMMA \ref{chap5-lem23}, we get 
$$
  \sum_{\rho \epsilon  \mathscr{R'}_{\nu}}, |K (\rho)|^2 \ll K^{2
    (1-\alpha)} \log^c T ; 
$$
on the other hand, the last theorem implies
\begin{equation*}
  N(\alpha, T) \ll T^{\left(\frac{12}{5} + \epsilon  \right) (1 -
    \alpha)}, \tag{5.1.18} \label{eq5.1.18}
\end{equation*}\pageoriginale
whence the first assertion of the lemma.  As for the second we
consider three cases separately.  Firstly, if $0 \leq \alpha \leq
3/4$, then, by the last theorem,  
\begin{align*}
  \sum_{\rho \epsilon  \mathscr{R'}_\nu} |K (\rho)|  & \ll \left(TN
  (\alpha,  T) K ^{1-2\alpha}\right)^{\frac{1}{2}}\log^c T\\ 
  & \ll K^{\alpha -1}\left(T^{\frac{3 - 4 \alpha}{2 (1-\alpha)}
      T^{\frac{1}{T^{2 (1-\alpha)}}+ \frac{3}{2(2-\alpha)} +
      \epsilon }}\right)^{(1-\alpha) } \log ^c T\\ 
  & \ll K^{\alpha -1}\left(T^{\frac{3 - 4 \alpha}{2 (1-\alpha)}+
      \frac{1}{T^{2 (1-\alpha)}}+ \frac{3}{2(2-\alpha)} +
      \epsilon} \right)^{(1-\alpha) } \log ^c T\\ 
  & \ll \left(T^{\frac{16}{5}+\epsilon }/K\right)^{1-\alpha} \log^c T,
\end{align*}
since 
$$
2 + \frac{3}{2(2 -\alpha)} \leq 16/5
$$
if $0 \leq \alpha \leq 3/4$. Secondly,  if $3/4 \leq \alpha \leq
11/12$, then again, by the last theorem, 
\begin{align*}
  \sum_{\rho \epsilon  \mathscr{R}'_\nu} | K(\rho) | & \ll  K^{\alpha
    -1} \left(K^{\frac{3 - 4 \alpha}{2 (1-\alpha)}}T^{\frac{1}{2
        (1-\alpha)}+ \frac{3}{2(3-\alpha)} +
    \epsilon }\right)^{(1-\alpha) } \log ^c T\\ 
  & \ll K^{\alpha -1} \left(T^{\frac{6 - 8 \alpha}{5(1-\alpha )}+
      \frac{1}{2(1-\alpha)}+ \frac{3}{2(3\alpha-1)} +
    \epsilon }\right)^{(1-\alpha ) } \log^c T. 
\end{align*}

But, for $\alpha \leq 11/12$, we have
$$
\frac{6-8\alpha}{5(1-\alpha)} + \frac{1}{2(1-\alpha)} +
\frac{3}{2(3\alpha -1)} = \frac{8}{5}+ \frac{1}{10(1-\alpha)} +
\frac{3}{2(3 \alpha -1)} \leq 16/5. 
$$

Finally, if $11/12\leq \alpha \leq 1$, then we appeal to (iii) of
LEMMA \ref{chap5-lem23}, getting\pageoriginale 
$$
\sum_{\rho \epsilon  \mathscr{R}'_\nu} |K(\rho)| \ll (N (\alpha,  T)
(K + N (\alpha,  T) T ^{\frac{1}{2}}) K^{1-2\alpha})^{\frac{1}{2}}
\log^c T. 
$$

But, for $\alpha \geq 11/12$, we have by the last theorem
$$
T^{\frac{1}{2}} N (\alpha,  T) \ll T^{\frac{9}{14}+ \epsilon }< K.
$$

Hence the sum in question is
$$
\ll \left(K T^{\frac{3}{2(3 \alpha -1)}+ \epsilon }\right)^{1-
  \alpha} \log^c T, 
$$
and noting that for $\alpha \geq 11/12,  K \leq T$, we have
$$
KT^{\frac{3}{2(3\alpha -1)}} \leq T^{\frac{16}{5}} K^{-1},
$$
which ends the proof of the second assertion of the lemma.

Here we should note that in the statement of the last lemma we have
neglected log-factors, because of Vinogradov's zero-free region. 

\section{A Zero-Density Estimate of the Linnik
  Type}\label{chap5-sec5.2}%SEC 5.2 

Most of the estimates of $N(\alpha,  T)$ can be extended to those of
$N(\alpha, T \chi)$ the number of zeros of $L (s, \chi)$ in the
rectangle $\alpha \leq \sigma \leq 1, |t| \leq T$. But they are of
limited value, because the theory of Dirichlet $L$-functions is greatly
hampered by the lack of a zero-free region comparable to that of
Vinogradov for the Riceman zeta-function. For some important problems
in prime number theory, however, this deficiency can be circumscribed
by\pageoriginale the combination of the Deuring- Helibronn  phenomenon
\eqref{eq4.2.3} and the zero-density estimate of the Linnik type: 
\begin{equation*}
  \sum_{\chi\pmod{q}} N(\alpha, T,  \chi ) \ll (qT)^{c (1-\alpha)} (0
  \leq \alpha \leq 1), \tag{5.2.1} \label{eq5.2.1}
\end{equation*}
which is especially strong near the line $\sigma = 1$.

We have seen already that the Selberg sieve for multiplicative
functions is capable to yield a very simple proof of the
Deuring-Heilbronn. In this section, we shall apply again a similar
idea to $L$-functions, and show that the same holds for once-difficult
zero-density estimates of the Linnik type. 

Our main tool is inequality \eqref{eq1.2.10}, $k=1$, or more precisely, the
following one more step hybridized version of it. 
\begin{Lemma}\label{chap5-lem26}
Let $S(\chi)$ be a set of complex numbers such that for any $s, s'
\epsilon  S(\chi)$ we have $Re (s) \geq 0, | Im (s)| \leq T $, and
$|Im (s) - Im (s')| \geq \delta > 0$ if $s \neq s'$. Then we have,
for any complex numbers $\{ a_n\}$, 
\begin{multline*}
  \sum_{\substack{ rq <Q \\ (r,q ) =1}} \frac{\mu^2 (r) q}{\varphi
    (rq)} \sum^*_{\chi \pmod{q}} \sum_{s \epsilon  S (\chi)}
  |\sum_{n \leq N} a_n \chi (n) \psi_r (n) n^{-s}|^2\\ 
  \ll (\delta ^{-1} + \log N) \sum_{n \leq N} (n+Q^2 T) |a_n|^2 \left(1 +
  \log \left(\frac{\log^2 N}{\log 2n}\right)\right). 
\end{multline*}

Here,\pageoriginale  $\psi_r$ is, of course, the one the defined at
\eqref{eq1.2.11}. Here-after, we shall take $T$ for a sufficiently large
variable, and for the sake of simiplicity, we assume, up to
\eqref{eq5.2.7}, that all Dirichlet characters are non-principal and
have conductors less than $T$. 
\end{Lemma}

Now let us denote, by $\bar{N}(\alpha, T, \chi)$, the number of
zeros of $L (s, \chi)$ in the rectangle $\alpha \leq \sigma \leq 1,  |t|
\leq T$, save for the $T$-exceptional zero (cf. \S~
\ref{chap4-sec4.2}). We note first
that because of Page-Landau's theorem and a reason to be disclosed
later, we may assume that 
\begin{equation*}
  1-\eta \leq \alpha \leq 1 - \frac{c}{\log T} \tag{5.2.2}\label{eq5.2.2}
\end{equation*}
with a fixed small $\eta > 0$. Then, for each $\nu = 0,1$ we pick up a
zero of $L (s,  \chi)$ which lies in the above rectangle and also in
one of the horizontal strips 
$$
\frac{2n+\nu}{\log T} \leq t < \frac{2n + \nu + 1}{\log T}(n = 0, \pm
1, \pm 2,  \ldots), 
$$
and denote by $Z_\nu (\chi)$ the resulting set of zeros of $L(s,
\chi)$. Here, we should quote the zero-density lemma:  the number of
zeros of $L (s, \chi)$ contained in the disk $| s- (1 + iu)| \leq 1 -
\alpha$ is $0 (( 1-\alpha) \log T)$, if $-T \leq u \leq  T$,  and
$\alpha$ satisfies \eqref{eq5.2.2}. 

Thus we have
$$
\bar{N} (\alpha, T, \chi)\ll(1-\alpha)\log T(|Z_0(\chi)|+|Z_{1}(\chi)|).
$$

Next,\pageoriginale let us recall the formula \eqref{eq1.4.15}; there,
we set $f \equiv 1$ 
and $\xi = \Lambda^{(1)}$ (cf. THEOREM
\ref{part1-chap1:sec1.3:thm4}). Then LEMMA
\ref{part1-chap1:sec1.4:lem5} implies that, for each square-free $r$,
the function  
\begin{multline*}
  M_r (s,  \chi ; \Lambda^{(1)}) \\ 
  = \frac{1}{\varphi (r)}
  \sum_{d=1}^{\infty} \Lambda_d^{(1)} \mu ((r,d)) \frac{\chi (d)}{d^s}
  \prod_{\substack{ p \not\mid d \\ p | r}} \left(\left( 1- \frac{\chi
    (p)}{p^s}\right) \left( 1- \frac{1}{p}\right)^{-1} -1\right)
  \tag{5.2.3}\label{eq5.2.3}
\end{multline*}
satisfies
\begin{equation*}
  L(s, \chi ) M_r (s, \chi ; \Lambda^{(1)}) = \sum_{n=1}^{\infty}
  \chi(n) \psi_r (n) \left(\sum_{d|n}
  \Lambda_d^{(1)}\right)n^{-s}. \tag{5.2.4} \label{eq5.2.4} 
\end{equation*}
 
We note also that \eqref{eq5.2.3} gives, for $0 \leq \sigma \leq 1$,
 \begin{equation*}
   M_r (s, \chi ; \Lambda^{(1)}) \ll z^{(1 + \vartheta ) (1 - \sigma )
     + \epsilon  } r^{-\sigma -1 + \epsilon }\tag{5.2.5} \label{eq5.2.5}
  \end{equation*}  
  where $\vartheta,  z$ are the parameters appearing in the definition
  of $\Lambda^{(1)}$. 
  
  Then we consider the Mellin integral
  $$
  \frac{1}{2 \pi i} \int\limits_{2 - i \infty}^{2+ i \infty} L(s, \chi
  )M_r (s, \chi ; \Lambda^{(1)}) r (s- \rho ) x ^{s- \rho} ds 
  $$
  where $\rho \epsilon  Z_\nu (\chi)$; in this, we set
  \begin{equation*}
    \vartheta = \eta^2,  r \leq R = T^{\eta^2},  z = T^3 R^2, x =
    T^{\frac{7}{2} + \eta} \tag{5.2.6} \label{eq5.2.6}
  \end{equation*}  
  with the same $\eta$ as that of \eqref{eq5.2.2}. Shifting the line of
  integration $\sigma = (\log T)^{-1}$,  and noting \eqref{eq5.2.2} and
  \eqref{eq5.2.5}, we see that this integral is $0 (T^{-\eta /2})$. On the
  other hand, we have \eqref{eq1.3.11} and \eqref{eq5.2.4}. 
  
  Hence,\pageoriginale after some simple consideration, we get
  $$
  1 \ll | \sum_{z \leq n \leq x (\log T)^2} \chi (n) \psi_r (n) \left(
  \sum_{d|n} \Lambda_d^{(1)} \right)n^{\sigma_e -n /x}| 
  $$
  for any $\sigma \epsilon  Z_\nu (\chi)$ and square - free $r \leq R$.
  
  Now, noting \eqref{eq1.2.12}, we see that gives 
\begin{multline*}
  \log R \sum_{1< q<
    T} \sum^*_{\chi \pmod{q}} |Z_\nu (\chi)| 
  \ll \sum_{\substack{r \leq R\\ 1< q< T\\ (r,q)=1}} \frac{\mu^2 (r)
    q}{\varphi (rq)}\\ 
  \sum^*_{\chi \pmod{q}} \sum_{\rho \epsilon 
    Z_\nu (\chi)} | \sum_{z \leq n \leq x (\log T)}^2 \chi (n)\psi_r
  (n) \left(\sum_{d |n} \Lambda_d^{(1)}\right) n^{-\rho} e^{-n/x}|^2. 
\end{multline*}

Then, appealing to LEMMA \ref{chap5-lem26}, we infer that the right side is 
\begin{align*}
  & \ll \log T \sum_{z \leq n \leq x (\log T)^2} \left(\sum_{d |n}
    \Lambda_d^{(1)}\right)^2 n^{1-\alpha}\\ 
    & \ll (x \log^2 T)^{2 (1-\alpha)} \log T \sum_{n=1}^{\infty}
    \left(\sum_{d|n} \Lambda_d^{(1)}\right)^2 n ^{-1-(\log T)^-1}\\ 
    & \ll (x \log^2 T)^{2 (1-\alpha)} \log T ; 
  \end{align*}  
  the last line is due to THEOREM \ref{part1-chap1:sec1.3:thm4}. Hence
  we have 
  $$
  \sum_{1< q< T} \sum^*_{\chi \pmod{q}} |Z_\nu (\chi)| \ll_\eta T^{(7
    + 2\eta)(1-\alpha)}, 
  $$
  which implies
  \begin{equation*}
    \sum_{1< q< T} \sum^*_{\chi \pmod{q}} \bar{N}(\alpha, T, \chi)
    \ll_\eta T^{(7+3_\eta)(1-\alpha)}. \tag{5.2.7}\label{eq5.2.7} 
  \end{equation*}  
  
  But, as is well-knowm, we have, for $0 \leq \alpha \leq 1$,
  \begin{equation*}
    \sum_{q <Q} \sum^*_{\chi \pmod{q}} N (\alpha, T, \chi) \ll (Q^2
    T)^{\frac{5}{2}(1-\alpha)} \log^c QT. \tag{5.2.8} \label{eq5.2.8}
  \end{equation*}\pageoriginale  
  
  Thus, taking into account Vinogradov's zero-free region for $\zeta
  (s)$, we obtain, from \eqref{eq5.2.7} and \eqref{eq5.2.8}, 
  
\begin{theorem}\label{chap5-thm15}
  We have, for $0 \leq \alpha \leq 1$ and $T\geq 1$,
  $$
  \sum_{q < T} \sum^*_{\chi \pmod{q}} N (\alpha, T, \chi  ) \ll T^{8 (1 - \alpha)}.
  $$
\end{theorem}   

\begin{center} 
\bf{NOTES (V)}
\end{center}   

  The zero-density method originates in the discovery made by Bohr and
  Landau \cite{key3} of evidence which supports statistically Riemann's
  hypothesis. But the actual emergence of the zero-density method as
  an indispensable tool for the study of the distribution of primes
  started, when Hoheisel \cite{key23} found that the estimate of the type 
  $$
  N(\alpha,  T) \ll T^{\lambda (1 - \alpha)} \log^c T (0 \leq \alpha \leq 1)
  $$
 yields a result on the difference between consecutive primes which
 had never been obtained without assuming the Riemann hypothesis for
 $\zeta (s)$ or sometimes similar to it. Namely, Hoheisel found a way
 to avoid the Riemann hyothesis for the investigaton of the
 distribution of primes. Afterwards, the discovery of the
 zero-free\pageoriginale 
 for $\zeta(s)$ of Vinogradov's type made it clear that the smaller
 $\lambda$ in the above estimate of $N(\alpha, T)$ would yield the
 better results on prime numbers. 
  
  In this context, Huxley's result \cite{key25}:
 $$
  N(\alpha, T) \ll
  \begin{cases}
    T^{\frac{3}{2- \alpha}(1-\alpha)} \log^c T & \text{if}~ 0 \leq
    \alpha \leq 3/4\\ 
    T^{\frac{3}{3 \alpha -1} (1-\alpha)} \log^c T & \text{if}~ 3/4 \leq
    \alpha \leq 1  
  \end{cases}  
 $$
is so far the best among various estimates of $N (\alpha, T)$, for it
gives the smallest $\lambda$, i.e., $12/5$, ever obtained. 

In Huxley's proof of this, a difficult estimate of $\zeta
\left(\frac{1}{2} + \text {it }~\right)$ due to Haneke was employed.
The reason 
that we develped, for CASE \ref{chap5-case3}, a zero-detecting method
of Bombieri \cite{key5} is that we wanted to dispense with Haneke's
result. This caused 
a slight decrease in the quality of the obtained estimate,  THEOREM
\ref{chap5-thm14}, of $N(\alpha, T)$ if compared with Huxley's, but, for the
applications which we have in mind, this will give no difference. 


\eqref{eq5.1.13} can be proved in just the same way as in Titchmarsh
      [\cite{key79}, p. 77]; \eqref{eq5.1.16} can be proved similarly,
      but we need also the functional equation for $\zeta (s)$. 

It should be remarked that, in \S~\ref{chap5-sec5.1}, we used twice
the device of 
raising a Dirichlet polynomial to a high power so  as\pageoriginale to
let it take a 
form suitable for the application of LEMMA \ref{chap5-lem24} - the
large-value theorem of Huxley \cite{key25}; this nice idea is due to
Jutila \cite{key37}.  

LEMMA \ref{chap5-lem25} is due to Iwaniec and Jutila \cite{key34}; the
weighted version of 
the zero-density estimates was first considered and used by Jutila
\cite{key38}. 

What Hoheisel did for $\zeta (s)$ Linnik did for $L$-functions;
namely, he found the way to avoid the extended Riemann hypothesis in
the investigation of some important problems concerning primes in
arithmetic progressions. This possibility was first realized in  his
famous work [\cite{key47}, I] in which he proved a result similar to
\eqref{eq5.2.1}. But it should be stressed that it is Fogel's \cite{key12} who
actually obtained the estimate of the type \eqref{eq5.2.1}. In Fogel's argument,
Turan's idea \cite{key80} is vital, and this is the same in Gallagher's
important work \cite{key15} where an estimate similar to that of THEOREM
\ref{chap5-thm15} was first proved.

In Linnik's Tur\'an's and Fogels' works, a sieve result, i.e. the
Brun-Titchmarsh theorem occupies an important place; the same can be
said about Gallagher's quoted above, for he used Bombieri-Davenport's
theorem \eqref{eq1.2.13} which is apparently a large sieve extension of the
Brun-Tich\-marsh theorem. This sieve aspect of the theory is now made
more explicit in our proof of THEOREM \ref{chap5-thm15}, for, as we
have shown in \S~ \ref{part1-chap1:sec1.2}, the pseudo-character
$\psi_r$ is directly  
connected\pageoriginale with the Selberg siever for intervals.

LEMMA \ref{chap5-lem26} can readily be proved by combining \eqref{eq1.2.10},
$k=1$, with the 
argument of Montgomery [\cite{key48}, Chap. 7 and 8]. 

One should note that in our proof of THEOREM \ref{chap5-thm15} Selberg's
observation \eqref{eq5.2.4} is vital (cf. Montgomery \cite{key50}. For a more
refinned treatment of the matter related to THEOREM \ref{chap5-thm15},  see
Motohashi \cite{key53} and  Jutila \cite{key40}. 

For the history of the zero-density method, we refer   to Mongomery
\cite{key48} and Richert \cite{key66}. 
