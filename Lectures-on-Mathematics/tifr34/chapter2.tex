\chapter{Manifolds}\label{chap2}

\section{Basic definitions}\label{chap2:sec1}

\begin{defis*} % def 
  \begin{enumerate} [(1)]
  \item Let\pageoriginale $V$ be a hausdorff  topological space. It is said to be a
    $(C^0)$ manifold of dimension $n$ if each $x \in V$ has an open
    neighbourhood $U$, which is homeomorphic to an open set in
    $\mathbb{R}^b$. 
  \item If $V$ is a topological space which is hausdorff, $V$ is said
    to be a $C^k$ manifold, $(0 \le k \le \infty)$, of dimension $n$,
    or a differentiable manifold of class $C^k$, if there is given a
    family of pairs $(U_i, \varphi_i)$, $U_i$ an open set in $V$ and
    $\varphi_i$, a homeomorphism of $U_i$ onto an open set in
    $\mathbb{R} ^ n $ such that  
    \begin{align*}
      & \cup U_i = V \text  {and, if } U_i \cap U_j \neq \phi,\\
      & \varphi_j \circ \varphi^{-1}_i \big | \varphi_i (U_i \cap U_j)
      \text{is a~ } C^k \text  { map of } \varphi_i (U_i \cap U_j)
      \text{into~ } \mathbb{R}^n.  
    \end{align*}
  \item If $V$ is a $C^k$ manifold of dimension $n$, a $C^k$ atlas on
    $V$ is a maximal set $\{(U_i, \varphi_i)\}$ such that $\cup U_i =
    V$ and whenever $U_i \cap U_j \neq \phi$, 
  $$
  \varphi_j \circ \varphi^{-1}_i \big | \varphi_i (U_i \cap U_j) \text {is
    a } C^k \text { map of } \varphi_i (U_i \cap U_j) \text {into }
  \mathbb{R}^n.  
  $$
  \end{enumerate}
\end{defis*}

\begin{remarks*}
  \begin{enumerate} [1.]
  \item Any set of pairs as in (2) can be completed to a $C^k$
    atlas and conversely an atlas defines the structure of a $C^k$
    manifold. 
  \item The dimension of $V$ is independent of the ``coordinate
    systems'' $\{U_i, \varphi_i\}$ according to a theorem of
    $L.E.J$. Brouwer which asserts that if a non-empty open set in
    $\mathbb{R}^n$ is homeomorphic to one in $\mathbb{R}^m$, then $m =
    n$. We shall not prove this theorem here. For a proof, see eg
    \cite{18}. 
  \item A\pageoriginale hausdorff topological space $V$ is said to be a real
    analytic (complex analytic ) manifold if there is given a family
    of pairs $(U_i, \varphi_i)$, $U_i$ an open set in $V$,
    $\varphi_i$, a homeomorphism of $U_i$ onto an open set in
    $\mathbb{R}^n$ (an open set in $\mathbb{C}^n)$, such that $\bigcup
    U_i = V$ and whenever $U_i \cap U_j \neq \phi$, $\varphi_j \circ
    \varphi^{-1}_i \big | \varphi_i (U_i \cap U_j)$ is real analytic
    (complex analytic = holomorphic). 
  \item If $V$ is a $C^k$ manifold and $U$ an open set in $V$, a map
    $f: U \to \mathbb{R}$ is called $C^r$, $0 \le r \le k$ if for
    each coordinate neighbourhood $(U_i, \varphi_i)$, with $U_i \cap U
    \neq \phi$, $f \circ \varphi^{-1}_i \big | \varphi_i (U_i \cap U)$ is
    $C^r$. We denote the set of $C^r $ functions on $V$ by $C^r (V)$,
    $0 \le r \le k$. 
  \item If $V$ and $V'$ are two $C^k$ manifolds of dimensions $n$ and
    $m$ respectively, $U$, an open set in $V$, a map $f$: $U \to V'$
    is called $C^r$, $0 \le r \le k$ if for coordinate neighbourhoods
    $(U_i, \varphi_i) $ and $(U'_j, \varphi'_j)$ of $V$ and $V'$
    respectively, such that $U_i \cap U \neq \phi$ and $f(U_i \cap U)
    \subset U'_j$, the map $\varphi'_j \circ f \circ \varphi^{-1}_i \big |
    \varphi_i (U_i \cap U)$ is of class $C^r$. 
  \end{enumerate}
\end{remarks*}

We denote set of $C^k$ maps of $V$ into $W$ by $C^k (V, W)$. If a
$C^k$ map $f$: $V \to W$ is a bijection and $f^{-1}: W \to V$ is
also $C^k$, we say that $f$ is a $C^k$-diffeomorphism, (or
diffeomorphism or $C^k$-isomorphism) of $V$ onto $W$. Real analytic
and holomorphic mappings between real and complex analytic manifolds
may be defined in the same way. We also introduce real and complex
analytic isomorphisms between such manifolds just as we did
diffeomorphisms. 

\begin{examples*}%exam
\begin{enumerate}[1.]
  \item $S_1 = \{x \in \mathbb{R}^2 \big | || x || = 1\}$ is a
    $C^\infty$ manifold of dimension $1$. 
  \item  If $V$  is a $C^k$ manifold and $\tilde{V}$ a hausdorff space,
    $p: \tilde{V} \to V$ local homeomorphism,\pageoriginale there is a unique
    structure of $C^k$ manifold on $\tilde{V}$ such that for
    $\tilde{a} \in \tilde{V}$, $p (\tilde{a}) = a$, there exist
    neighbourhoods $\tilde{U}$ of $\tilde{a}$, $U$ of a such that $p$:
    $\tilde{U} \to U$ is a $C^k$ isomorphism. 
  \end{enumerate}
\end{examples*}

A more interesting class of Examples (Grassmann manifolds) is 
described at the end of the section. 

It is clear that a complex analytic manifold carries a natural real
analytic structure; a real analytic manifold a $C^\infty$ structure
and a $C^k$ manifold $(0 < k \le \infty)$ a $C^r$ structure $(0 \le r
< k)$. Conversely, it follows from results of $H$. Whitney \cite{47} that
any paracompact $C^1$ manifold carries a real analytic
structure. Further, the imbedding theorem of H.Grauert \cite{13} (see $\S
9$ for the statement) and the approximation theorem of Whitney
(Chap.I, \S\ 5) imply that this structure is unique. However, a $C^0$
manifold may have no differentiable structure (M. Kervaire \cite{20}) and
even when it has,this is not unique. For example, the sphere $S^7$ can
carry two differentiable structures such that there is \textit{no}
diffeomorphism of one onto the other (J. Milnor \cite{28}). The problem
of the existence and uniqueness of complex structures is a problem of
quite a different nature, and had given rise to a vast literature (see
in particular H. Hopf \cite{16}, K.Kodaria and D.C. Spencer \cite{21}). 

Let $a$ be a point in a $C^k$ manifold $V$. Consider all ordered pairs
$(f, U)$ where $U$ is an open set containing $a$ and $f$, $a$ $C^k$ map
$U \to \mathbb{R}$. In the set of these ordered pairs we define an
equivalence relation as follows. $(f, U) \sim (f', U') $ if there
exists an open set $\Omega$ containing $a$ such that $\Omega \subset U
\cap U'$, and such that $f \big | \Omega = f' \big | \Omega$. The
equivalence classes of these ordered pairs are called germs (of $C^k$
functions) at $a$. We shall frequently identify a germ with a function
defining\pageoriginale it when there is no fear of confusion. 

\begin{defi*} % defini 
  A germ $f$ of a $C^k$ functions, $k \ge 1$, at a is said to be
  stationary at a if there exists a coordinate neighbourhood $(U,
  \varphi)$ with $a \in U$ such that all the first partial derivatives
  of $f \circ \varphi^{-1}$ vanish at $a$. Here $(f, U)$ is a pair
  defining $f$. It is clear that the above definition depends only on
  the germ $f$. 
\end{defi*}

\begin{notation}
  $C^k_a$ denotes the set of all $C^k$ germs at $a$, $S^k_a = S_a$
  denotes the set of all stationary $C^k$ germs at a and $m^k_a =
  m_a$, the set of all $C^k$ germs vanishing at $a$. $C^k_a$ is a
  vector space over $\mathbb{R}$; $S^k_a$ and $m^k_a$ are subspaces.. 
\end{notation}

\begin{defi*}% def
  \begin{enumerate}[(1)]
  \item The quotient space $C^k_a / S_a$ is called the space of
    differentials (or cotangent vectors or co vectors) and is denoted
    by $T^*_a (V)$. The image of $f \in C^k_a$ in $T^*_a(V)$ is
    denoted by $(df)_a$. 
  \item The dual space of $T^*_a (V)$, i.e., the space of all linear
    functionals $X$: $C^k_a \to \mathbb{R} $ with $X (f) = 0$ for $f
    \in S_a$, is called the tangent space at a and is denoted by $T_a
    (V)$. A point in $T_a (V)$ is called a tangent vector. 
  \item A linear function $L$: $C^k_a \to \mathbb{R}$ is called a
    derivation if for $f$, $g \in C^k_a (V)$, 
    $$
    L (f. g) = L(f) ~ g(a) + f(a) L(g).
    $$
  \end{enumerate}
\end{defi*}

\setcounter{proposition}{0}
\begin{proposition}\label{chap2:sec1:prop1} % prop 1
  Any tangent vector $X$ in $T_a (V)$ is a derivation.
\end{proposition}

\begin{proof}
  For any $f$, $g \in C^k_a $ the function $\varphi$ given by
  $$
  \varphi = f g - f(a) g - f. g (a), \text {  is   in } S_a.
  $$
\end{proof}

Hence\pageoriginale $X (\varphi) = 0$ 

i.e.~ $X(fg) = f(a) X(g) + X(f). g(a)$.

\begin{defi*} % def
  If $(U, \varphi)$ is a coordinate neighbourhood and for a point $x
  \in U$, $(x_1, \ldots , x_n)$ are the coordinates of $\varphi (x)$
  in $\mathbb{R}^n$, for $a$ $C^1$ function $f$: $U \to \mathbb{R}$,
  $a \in U$, we define 
  \begin{align*}
    & \left(\frac{\partial f}{\partial x_1}\right)_a , \left(\frac{\partial
      f}{\partial x_2}\right)_a , \ldots , \left(\frac{\partial f}{\partial
      x_n}\right)_a \text { by }\\ 
    & \left(\frac{\partial f \circ \varphi^{-1}}{\partial
      x_1}\right)_{\varphi (a)} , 
    \ldots , \left(\frac{\partial f \circ \varphi^{-1}}{\partial
      x_n}\right)_{\varphi_{(a)}}   
  \end{align*}
  respectively. We define tangent vectors $\left(\dfrac{\partial}{\partial
    x_i}\right)_a$ at a by $\left(\dfrac{\partial}{\partial x_i}\right)_a f =
  \left(\dfrac{\partial f}{\partial x_i}\right)_a$. 
\end{defi*}

\begin{proposition}\label{chap2:sec1:prop2} % prop 2
  $\left(\dfrac{\partial}{\partial x_1}\right)_a, \ldots ,
  \left(\dfrac{\partial}{\partial x_n}\right)_a$ are linearly
  independent in $T_a (V)$ and span $T_a (V)$. 
\end{proposition}

\begin{proof}
  If $f \in C^k_a$, $g$ defined by
  $$
  g(x) = f(x) - f (a) - \sum x_i \left(\dfrac{\partial f}{\partial x_i
  }\right)_a, 
  $$
  is in $S_a$. Hence for $X \in T_a (V), ~ X(g) = 0$

  i.e. \qquad \quad $X(f) = \sum X(x_i) \left(\dfrac{\partial
    f}{\partial x_i}\right )_a$

  i.e. \quad \qquad $X = \sum X(x_i) ~ \left(\dfrac{\partial}{\partial
    x_i}\right )_a$.
\end{proof}

Therefore $\left(\dfrac{\partial}{\partial x_1}\right)_a , \ldots ,
\left(\dfrac{\partial}{\partial x_n}\right)_a$ span $T_a (V)$. Further
$\left(\dfrac{\partial x_j}{\partial x_i }\right)_a = \delta_{ij}$ hence
$\left\{\left(\dfrac{\partial}{\partial x_i }\right)_a\right\}$, $1
\le i \le n$, are linearly independent. 

\begin{coro*} % coro
$T_a (V)$\pageoriginale and $T^*_a(V)$ are $n$ dimensional vector spaces.
\end{coro*}

It follows from this that any tangent vector defines a linear function
on the germs of $C^1$ functions, which vanishes on stationary
functions and is a derivation. 

\begin{proposition}\label{chap2:sec1:prop3} % prop 3.
  If $X$ is a derivation of $C^{k - 1}_a$, $k \ge 1$, $X$ is in $T_a
  (V)$. [Note that there is a natural injection of $C^k_a $ in
    $C^{k-1}_a$]. 
\end{proposition}

\begin{proof}
  If $f \in S_a$, we can assume without loss of generality that $f$ is
  defined on an open set $U$ containing $a, (U, \varphi)$ a coordinate
  neighbourhood, and for some open set $U' \subset U$, with $a  \in
  U'$ and, if $x \in U'$, $t \varphi (x) \in \varphi (U)$, $0 \le t \le
  1$. Then for $x \in U'$, 
  \begin{align*}
    f(x) & = \int\limits_{0}^1 \frac{\partial f}{\partial t} [
      \varphi^{-1} (t \varphi (x)) ] dt\\ 
    & = \sum x_i g_i (x)
  \end{align*}
  where $g_i (x) = \int\limits_{0}^1 \dfrac{\partial f}{\partial x_i}
  [\varphi^{-1} (t \varphi (x)] dt$. 
\end{proof}

Clearly $g \in C^{k-1}_a$.

We may also assume that $\varphi (a) = 0$. Then
$$
X(f) = \sum x_i (a) X (g_i) + \sum X(x_i) g_i (a)
$$
but $x_i (a) = 0= g_i (a)~ 1 \le i \le n$.

Hence $X(f) = 0$ i.e. $X$ is a linear map $C^{k-1}_a \to \mathbb{R}$
which vanishes on $S^k_a $ i.e. $X$ is a tangent vector. 

\setcounter{corollary}{0}
\begin{corollary}\label{chap2:sec1:coro1}% coro 1
  If $V$ is a $C^\infty$ manifold and $f \in m^\infty_a$, then $f$ is
  stationary at a if and only if $f \in (m^\infty_a)^2$. 
\end{corollary}

\begin{proof}
  The\pageoriginale maps $x_i$ and $g_i$ in the above proof are in $m^\infty_a$,
  which proves the necessity. The sufficiency is trivial. 
\end{proof}

\begin{corollary}\label{chap2:sec1:coro2} %coro 2
  For a $C^\infty$ manifold; we have $T^*_a = m_a /(m_a)^2$.
\end{corollary}

One has also the following ``geometric'' definition of tangent
vectors. Let $\gamma$: $I \to V$ be a $C^k$ curve (i.e. a $C^k$ map of
a neighbourhood of the unit interval $I = [ 0, 1]$ on $\mathbb{R} $
into $V$). The \textit{tangent to} $\gamma$ \textit{at} $a = \gamma
(0)$ is the tangent vector $X$ at a defined by  
$$
X(f) = \frac{d}{dt} f \circ \gamma (t) \big |_{t = 0} \text {  for  }
f \in C^1_a. 
$$
[It is easily verified that this defines a tangent vector.] One has 

\begin{proposition}\label{chap2:sec1:prop4} % prop 4
  Any tangent vector at $a \in V$ is the tangent at $a$ to some curve
  $\gamma$ with $\gamma (0) = a$. 
\end{proposition}

\begin{proof}
  We may suppose that $V$ is the open cube $| x_i | < 1 $ in
  $\mathbb{R}^n$. Any tangent vector $X$ at $x = 0$ is of the form 
  $$
  X = \sum a_i \left(\frac{\partial}{\partial x_i}\right)_0.
  $$
\end{proof}

Let $\gamma_i$, $1 \le i \le n$, be $C^k$ functions in a neighbourhood
of $I$ with $| \gamma_i | < 1$, $\gamma_i (t) = a_i t$ in a
neighbourhood of $t = 0$. We may take for $\gamma$ the curve given by
$\gamma (t) = (\gamma_1(t) , \ldots , \gamma_n (t))$. 

Unless otherwise stated, in what follows $V$ denotes a $C^k$ manifold
of dimension $n$ and $W$ denotes a $C^k$ manifold of dimension
$m$. Let $F : V \to W$ be a $C^k$ map. Then the maps 
\begin{align*}
  & f_* : T_a (V) \to T_{f(a)} (W) \text { and }\\
  & f^* : T^*_{f(a)} (W)  \to T^*_a (V) \text {are defined by}\\
  & f_* (X) (g) = X(g ~  \circ ~ f) \text {and } (f^* (d \varphi))_{(a)} =
  [d(\varphi \circ f) ]_a 
\end{align*}
when\pageoriginale $g \in C^k_{f(a)}$, $(d \varphi) \in T^*_{f(a)} (W)$ and $X \in
T_a(V)$. Note that if $g \in S_{f(a)}$, $g \circ f \in S_a$. It is easily
verified that $f_*$ and $f^*$ are transposes of one another. 

Remark that if $f_1$: $V_1 \to V_2$, $f_2 : V_2 \to V_3$ are $C^k$
maps, then we have, for any $a \in V_1$, $(f_2 ~ \circ ~f_1)^*_a=
(f^*_2)_{f_1(a)} \circ (f_1)^*_a$. It follows that if $f: V \to W$ is a
$C^k$ isomorphism, then $f^*_a$ is an isomorphism for any $a$. Hence
$T_a (V)$ and $T_{f(a)}(W)$ have the same dimension; hence $V$, $W$
have the same dimension. Thus the fact that the dimension of a $C^k$
manifold, $k \ge 1$, is invariant of the $(C^k)$ local coordinates
chosen, is obvious. (Compare with Remark $2$ after the definition of a
manifold.) Let $T(V) = \bigcup\limits_{a \in V} T_a (V)$. We shall
prove the following  

\setcounter{theorem}{0}
\begin{theorem}\label{chap2:sec1:thm1} % the
  If $V$ is a $C^k$ manifold, $k \ge 1$, $T(V)$ carries a
  natural structure of a $C^{k-1}$ manifold dimension $2n$. 
\end{theorem}

\begin{proof}
  It follows from Proposition \ref{chap2:sec1:prop2} that, relative to
  a coordinate 
  system $(U_i, \varphi_i)$, with $a \in U_i$, a tangent vector $X$ in
  $T_a(V)$ is completely determined by $\{\alpha_\nu = X(x_\nu)_a \}_{1
    \le \nu \le n}$. Let $(U_j, \varphi_j)$ be another coordinate
  neighbourhood with $a \in U_i$ and let the tangent vector be given
  by $\{\beta _\nu = X (y_\nu)_a \}_{1 \leq \nu \leq n}$ with respect to $(U_j,
  \varphi_j)$. We denote by $(x_1, x_2, \ldots , x_n)$ and $(y_1, y_2,
  \ldots , y_n)$, the local coordinates of $\varphi_i (x)$ and
  $\varphi_j (x)$ respectively. Then for any $g \in C^k_a$, 
\begin{align*}
  X(g) = \sum_{i} \alpha_i \left(\frac{\partial g}{\partial x_i}\right)_a  &=
  \sum_j \beta_j \left(\frac{\partial g}{\partial y_j}\right)_a \\ 
  &= \sum_{j}\beta_j \left(\sum_\nu \left(\frac {\partial g}{\partial
    x_\nu}\right)_a \left( \frac{\partial x_\nu}{\partial
    y_j}\right)_a\right) \\  
  &= \sum_\nu \left( \sum_j \beta_j \left( \frac{\partial x_\nu}{\partial
    y_j}\right)_a \right) ~\left(\frac{\partial g}{\partial x_\nu}\right)_a 
\end{align*}
\end{proof}

Hence\pageoriginale 
\begin{equation*}
  \alpha_i = \sum_j \beta_j \left(\frac{\partial x_i}{\partial y_j}\right)_a
  \tag{1.1}\label{chap2:sec1:eq1.1}
\end{equation*}
i.e. 
\begin{equation*} 
  (\alpha_1,  \ldots , \alpha_n)= ( \beta_1 ,
  \ldots , \beta_n) (m_{ij})_a
\end{equation*}
where $(m_{ij})_a$ is the matrix $(\dfrac{ \partial x_j}{\partial
  y_i})_a$. Clearly $(m_{ij})_a$ is non-singular and $(m_{ij})_a
(m_{ji})_a = I$. Now consider the topological union $E- \bigcup_i (U_i
\times \mathbb{R}^n \times i)$ and define an equivalence relation,
$\sim$, by $(x, v, i) \sum (x', v', j)$ if $x=x'$ and $v
=v'(m_{ij})_x$, [where $(m_{ij})_x$ is the matrix defined
  above]. Clearly there is an obvious bijective map from $E/_\sim$
onto $T(V)$. It suffices to show that $E/_\sim$ carries a natural
structure of $C^{k-1}$ manifold. 

It is clear that $\sim$ is an open equivalence relation. Let $\eta :E
\to E/\sim$ denote the natural map, and let $p': E \to V$ the
continuous map $p((x,v,i)) =x$. Clearly $p'$ maps equivalent points
onto the same point in $V$, so that $p'$ defines a continuous map
$p$: $E/_\sim \to V$. Further $\eta_i = \eta \left\{ U_i \times
\mathbb{R}^n \times i \right\}$ is a homeomorphism onto $p^{-1}(U_i)$;
in particular $p^{-1}(U_i)$ is hausdorff; we identity $U_i \times
\mathbb{R}^n$ with $U_i \times \mathbb{R}^n \times i$. We assert\pageoriginale that
$E / _\sim$ is hausdorff: in fact if $e_1$, $e_2 \in E/\sim ~e_1 \neq
e_2$, then if $p(e_1)\neq p(e_2)$ and $\Omega_i$ is a neighbourhood of
$e_i$, $\Omega_1 \cap \Omega_2 = \phi$, then $p^{-1}(\Omega_1)$,
$p^{-1}(\Omega_2)$ are disjoint neighbourhoods of $e_1,$ $e_2$
respectively. If $p(e_1) = p(e_2)$, then $e_1$, $e_2 \epsilon p^{-1}(U_i)$
for some $i$, since $p^{-1}(U_i)$ is open in $E /_\sim$ and is
hausdorff, $e_1$, $e_2$ can be separated. 

If $\varphi_i$ is the given $C^k$ homeomorphism of $U_i$ onto an open
set $U'_i$ in $\mathbb{R}^n$, then $( \varphi_i \times id) \circ
\eta_i^{-1} = \Phi_i$ is a homeomorphism of $p^{-1}(U_i)$ onto $U'_i
\times \mathbb{R}^n$; that the mappings  $\Phi_j \circ \Phi_i^{-1}$ are
$C^{k-1}$ follows at once from (\ref{chap2:sec1:eq1.1}) [note that
  (\ref{chap2:sec1:eq1.1}) involves
  derivatives of $C^{k}$ functions]. 

We remark that the $C^{k-1}$ structure of $T(V)$ so obtained does not
depend on the system $\left\{ U_i , \varphi_i \right\}$ used. 

$T(V)$ is an example of a real vector bundle (see Chap.III,
\S\ \ref{chap3:sec1}). 

If $0 \leq p \leq n$, we consider the vector space $\Lambda^{p}
T^*_a(V)$. An element of this space is called  a \textit{$p$-co vector}
at the point $a$. If $(U, \varphi)$ is a coordinate system at $a$,
then the differentials $(dx_1)_a, \ldots , (dx_n)_a$ form  a basis of
$T^*_a(V)$. Hence a basis of $\Lambda^{p} T^*_a(V)$ is given by the
elements $(dx_{i_1})_a \Lambda \cdots \Lambda(dx_{i_p})_a$, $i_1 <
\cdots < i_p$. In exactly the same way as above, we prove the
following  

\begin{theorem}\label{chap2:sec1:thm2} % \thm 2
  The set $\overset{p}\Lambda T^*(V) = \bigcup\limits_{a \in V}
  \overset{p}\Lambda T^*_a(V)$ carries a natural structure of
  $C^{k-1}$ manifold [of dimension $n + (^n_p) $].  
\end{theorem}

\medskip
\noindent
\textbf{Grassmann manifolds}.

Let $0 <  r  < n$, and let $G_{r, n} $ denote the set of
$r$-dimensional linear subspaces of $\mathbb{R}^n$. We shall show
that $G_{r, n}$ carries a natural structure of real analytic
manifold. 

Let\pageoriginale $M(r, n)$ denote the space of $r \times n$ real matrices and $N =
N(r, n)$ the subset of matrices of rank $r$. $M(r, n)$ is clearly
homeomorphic to $\mathbb{R}^{rn}$ and $N(r, n)$ to an open subset. Let
$G= GL(r, \mathbb{R})$ denote the group of nonsingular $r \times r$
matrices. We have  natural map $G \times N \to N$ defined  by $(A, B)
\rightsquigarrow A.B$, where $A \in G$, $B \in N$. 

We assert that there is a natural bijection $p:  N /_{G} \to G_{r,
  n}$, where $N/_{G}$ is the quotient of  $N$ by the equivalence
relation: $B_1 \sim B_2$ if there is $A \in G$ with $B_2 = A$, $B_1$. 

\begin{proof}
  If $B \in N$ we may look upon  $B$ as a column 
  $\begin{pmatrix} v_1 \\ \vdots\\ v_r  \end{pmatrix}$
  where $v_\nu \in \mathbb{R}^n$; let $p(B)$ denote the subspace
  spanned by $v_1, \ldots , v_r$. If $B \in N$, this subspaces  has
  dimension $r$. The assertion that $p$ is a bijection is equivalent
  with the obvious assertion that the sets $(v_1, \ldots , v_r)$, $(w_1,
  \ldots , w_r)$ of points of $\mathbb{R}^n$ span the same
  $r$-dimensional subspace if and only if there is an $A \in G$ with  
  $$
  \begin{pmatrix}
    w_1 \\ \vdots \\ w_r
  \end{pmatrix}    
  =A
  \begin{pmatrix}
    v_1 \\ \vdots \\ v_r
  \end{pmatrix}    
  $$
\end{proof}  

We put on $G_{r, n}$ the quotient topology; clearly the equivalence
relation defined above is open. 

Let $K$ denote the set of $r$-tuples  $j_1 < \cdots < j_r$ of integers
$j_\nu$ with $1 \le j_\nu \le n$. For $\alpha \in K$, let $V$ be the
subset of $M(r, n)$ consisting of matrices $B=(b_{ij})_{1  \le i \le r,
  1 \leq j \leq m}$, for which  
  $$
  B^\alpha = (b_{ij_\nu})_{1 \le i \le r, 1 \le \nu \le r}, \alpha= (j_1, \ldots , j_r)
  $$
  is non-singular. We have $\bigcup V_\alpha =N$. It is clear that if
  $B_1$, $ B_2 \in N$ and  $A \in G$\pageoriginale satisfies $B_2= A B_1$, and if $B_1
  \in V_\alpha$, then $B_2 \in V_\alpha$ and we have $B^\alpha_2=
  AB^\alpha_1$. 

  For $B_1 \in V_\alpha$, we shall write symbolically, $B= (B^\alpha,
  C^\alpha)$, where $B$ is the matrix defined above and $C^\alpha$ is
  the $r \times (n-r)$ matrix  
  $$
  C^\alpha = (b_{ij_\nu}) \text{ with } 1 \le i \le r, 1 \le j_1
  <\cdots <  j_{n-r} \le n, j_\nu \notin \alpha. 
  $$  
  
  We shall identify $M(r, n-r)$ with $\mathbb{R}^{r(n-r)}$. Let
  $\psi_\alpha: V_\alpha \to \mathbb{R}^{r(n-r)}$ denote the mapping 
  $$
  \psi_\alpha (B) = (B^\alpha)^{-1} C^\alpha;
  $$
  $\psi_\alpha$ is clearly continuous and open. Then, if $U_\alpha$ is
  the subset $p(V_\alpha)$ of $G_{r,n}$, there is a homeomorphism 
  $$
  \varphi_\alpha:U_\alpha \to \mathbb{R}^{r(n-r)}
  $$ 
  such that, $\varphi_\alpha \circ p = \psi_\alpha$. In fact it is easy to
  verify that $\psi_\alpha(B_1)= \psi_\alpha(B_2)$ if and only  if
  $B_1 \sim B_2$, which gives us the existence of a bijection. This is
  continuous and open, since $\psi_\alpha$ is  
  
  We assert next that $G_{r,n}$ is Hausdorff. Since the equivalence
  relations is open, we have only  to prove that its graph $\Gamma$
  consisting of pairs $(B_1, B_2) \in N \times N$ with $B_1 \sim B_2$
  is closed in $N \times N$. Suppose that  
  $$
 ((B_1)_\nu , (B_2)_\nu) \in \Gamma, (B_i)_\nu \to B_i \in N
 $$
 and let $A_\nu \in G$ satisfy
 $$
 (B_2)_\nu= A_\nu (B_1)_\nu.
 $$
 Since\pageoriginale $B_1 \in N$, $B_1 \in V_\alpha$ for some $\alpha$. Then so does
 $(B_1)_\nu$ for sufficiently large $\nu$ and we have 
$$
(B_1)^\alpha_\nu \to  B^{\alpha}_1 \text{ as } \nu \to \infty
$$ 
Then we have $(B_2) \nu \in V_\alpha$ and 
$$
(B_2)^\alpha_\nu = A_\nu  (B_1)^\alpha_\nu.
$$
Since  $ (B_1)^\alpha_\nu \to  B_1^\alpha  \in G$, and since $
(B_2)^\alpha_\nu$ converges to a matrix $A^{(1)} \in M(r,r)$ (since,
by assumption, $(B_2)_\nu \to B_2$ in $N$), the matrix $A_\nu$
converges to $A= (B^\alpha_1)^{-1} A^{1}$ as $\nu \to \infty$. Since  
$$
(B_2)_\nu = A_\nu (B_1)_\nu,
$$
we deduce that $B_2= A B_1$. However, since $B_2$ has rank $r$, $A$
has rank $\ge r$; since $A\in  M(r,r)$, $A \in G$ so that $B_1 \sim
B_2$ and $(B_1, B_2) \in \Gamma$. 
  
  The covering $\{ U_\alpha\}_{\alpha \in K}$ and the homeomorphisms
  $\varphi_\alpha$: $U_\alpha \to \mathbb{R}^{r(n-r)}$ make of
  $G_{r,n}$ an $r(n-r)$ dimensional real analytic manifold. In fact
  the coordinate changes $\varphi_\alpha \circ \varphi^{-1}_\beta$ are
  easily seen to be  \textit{rational} functions. 
  
  Let $0(n)$ denote the orthogonal group of $\mathbb{R}^n$, i.e. the
  set of $n \times n$ matrices $A$ for which  
  $$
  A \cdot {}^t A =I;
  $$
  here $I$ is the unit $n \times n$ matrix and $^t A$ is the transpose
  of $A$. $0(n)$ acts on $G_{r,n}$: if $B_1 \in N$, $0 \in 0(n)$, then
  $B_1 0 \in N$ and, if  $B_1 \sim B_2$\pageoriginale we have $B_1 0 \sim B_2 0$. It
  is easy to show that $0(n)$ is compact and that it acts transitively
  on $G_{r,n}$. We deduce the following  

  \begin{prop*}
    The Grassmannian $G_{r,n}$ is a compact, real analytic manifold of
    dimension $r(n-r)$.  
  \end{prop*}    

\begin{remarks*} % rem 1
  \begin{enumerate}        
  \item The manifold $G_{1,n}$ is called $(n-1)$ -dimensional
    projective space $\mathbb{P}^{n-1}(\mathbb{R})$. 
  \item It can  be proved in the same way that the set $G_{r,n}(
    \mathbb{C})$ of complex $r$-dimensional subspaces of $\mathbb{C}^n$
    is a compact complex manifold of complex dimension $r(n-r)$,
    $G_{1,n}(\mathbb{C})$ is the complex projective space
    $\mathbb{P}^{n-1} ( \mathbb{C})$. 
  \end{enumerate}
\end{remarks*}
    
For much of the material contained in \S\ 1, 2 see Schwartz \cite{40}.
    
\section{Vector fields and differential forms}\label{chap2:sec2} % \sec 2
    
Let $V$ be a $C^k$ manifold and $p$: $T(V) \to V$ the projection given
by $p(X)=a$ for $X \in  T_a (V)$ for any  $ a \in V$. 
\begin{defi*}
  $A$ $C^r$ {\em Vector field} $X$, $0 \le r \le k-1$ {\em is, by
    definition}, a $C^r$ {\em map} $X: V \to T(V)$ {\em such that } 
  $$
  p \circ X = \text{ identity on } V.
  $$
\end{defi*}  
        
Clearly if $X$ is a vector field, $X(a) \in T_a (V)$ for any  $a \in
V$.  If $(U,\varphi)$ is a coordinate neighbourhood, we may represent
the vector field $X$ by the formula 
$$
X_a = \sum \xi_i (a) \left(\frac{\partial}{\partial x_i}\right)_a.
$$
    
Then\pageoriginale $X$ is of class $C^r$ if and only if the $\xi_i (a)$ are $C^r$ functions.
\begin{defi*}
  $A$ $p$ {\em differential form $\omega$ of class } $C^r$ {\em is a
  } $C^r$ {\em map} $\omega$: $V \to \overset{p}\Lambda T^*(V)$ {\em
    such that} $\omega (a) \in   \overset{p}\Lambda T^*_a (V)$ {\em
    for each } $a \in V$. 
\end{defi*}
    
If $(U, \varphi)$ is a coordinate neighbourhood $\omega$ has a representation
$$
\omega_a = \sum_{i_1 < i_2 <\cdots < i_p} \xi_{i_1\cdots i_p} (a)
(dx_{ij})_a \Lambda(dx_{i_{r}})_a \Lambda \cdots \Lambda (dx_{i_p}). 
$$ 
again  $\omega$ is of class $C^r$ if and if the $\xi_{i_1\cdots i_p}$
a $C^r$ functions. Let $\mathscr{G}$denote the module [over the ring
  $C^{k-1}(V)$ of $C^{k-1}$ functions on $V$] of $C^{k-1}$ vector
fields on $V$. If $\omega$ is a p-form on $V$, it defines a p-linear
map of $\mathscr{G}^p$ into $C^{k-1}(V)$; in fact we have only to set 
$$
\omega(X_1, \ldots , X_p) (a) = \omega_a ((X_1)_a , \ldots , (X_p)_a).
$$

[Note that $\overset{p}\Lambda T^*_a (V)$ is the dual of the space
  $\overset{p}\Lambda T_a (V)$.] This map his following two
properties: $(a)$ it is alternate; $(b)$ it is multilinear over
$C^{k-1}(V)$. Conversely, any alternate map $\varphi$ of
$\mathscr{G}^p$ into $C^{k-1}(V)$, which is multilinear over
$C^{k-1}(V)$ defines a differential p-form $\omega$; in  fact, if
$(X_1)_a , \ldots , (X_p)_a$ are vectors at $a \in V$, and if $X_1 ,
\ldots , X_p$ are vector fields on $V$ extending these  vectors, we
define the $p$-co vector $\omega_a$ by  
$$
\omega_a ((X_1)_A , \ldots , (X_p)_a) = \varphi (X_1, \ldots , X_p).
$$

It is easily verified, using the fact that $\varphi$ is
$C^{k-1}(V)$-linear that $\varphi (Y_1, \ldots , Y_p)= 0$ at a point
$b$ if $(Y_i)_b=0$ for some $i$, so that  the\pageoriginale above definition is
independent  of the extension of the vectors $(X_i)_a$ to vector
fields on $V$. If $f: V \to W$ is a $C^k$ map and  $a \in V$, $b = f
(a)$, we have defined linear maps $f_*: T_a (V) \to T_b(W)$ and
$f^*: T^*_{f(a)}(W) \to T^*_a(V)$. This defines a map, which denote
$f^*$, of $\overset{p}\Lambda T^*{f(a)}(W) \to \overset{p}\Lambda
T^*_{a}(V)$. $f^*$ is clearly an algebra homomorphism of $\Lambda
T^*_{f(a)} (W)$ into $\Lambda T^*_{a} (V)$.  
    
Hence if $\omega$ is  a $p$ form on $W$ of class $C^r$ we may
associate to any $ a \in V$ the $p$ co vector $f^* (\omega_{f(a)})$. It
is easy to see that this defines a $p$-form $f^*(\omega)$ of class
$C^r$ on $V$. However, the map $f_*$ does not in general, transform
vector fields. 

\begin{defi*}
  If $f: V \to W$ is  a $C^k$  map, $k \ge 1,$ $f$ is said
    to have rank $r$ at $a \in V$, if 
  $$
  \text{rank } f_* : T_a (V) \to T_{f(a)}(W) \text{ is } r.
  $$
\end{defi*}    
    
We can easily  calculate the map $f_*$ in terms of local coordinates
$(U, \varphi)$ at a  and $(U' , \varphi')$ at $b=f(a)$. In terms if
the bases  $$\left(\dfrac{\partial}{\partial x_1}\right)_a , \ldots,
\left(\dfrac{\partial}{\partial x_n}\right)_a\quad 
\mbox{and}\quad \left(\dfrac{\partial}{\partial 
  y_1}\right)_b, \ldots, \left(\dfrac{\partial}{\partial
  y_m}\right)_b$$ of $T_a(V)$ 
and $T_b(W)$, if $X= \sum a_i \left(\dfrac{\partial}{\partial x_i}\right)_a$, $g
\in C^k_b$ then  
\begin{align*}
  \sum b_j \left(\frac{\partial g}{\partial y_j}\right)_b &= f_*
  (X). (g) = X(g \circ f) \\ 
  &= \sum_i a_i  \sum_j \left(\frac{\partial g}{\partial y_j}\right)_b
  \left(\frac{\partial f_j}{\partial x_i}\right)_a  
\end{align*}    
so that, $f_* (a_1, \ldots , a_n)= (b_1, \ldots , b_n)$, with 
$$
b_j = \sum a_i \left(\frac{\partial f_j}{\partial x_i}\right)_a.
$$

This\pageoriginale was precisely the map $d(f)(a)$ defined in Chap. I
\S\ \ref{chap1:sec1}, if we
look upon $f$ as a map of an open set in $\mathbb{R}^n$ into
$\mathbb{R}^m$. We obtain therefore the following theorems form the
inverse function theorem, the rank theorem and Sard's theorem, proved
in Chap. I. 
    
\noindent
\textbf{Inverse function theorem}. \textit{If $V$ and  $W$  are $C^k$
  (real analytic) manifolds of dimension $n$ and $f:$ $V \to W$ a $C^k$
  real analytic map, and if $f_* : T_a (V) \to T_{f(a)}(W)$ is an
  isomorphism for some $a \in V$, then there exist neighbourhoods
  $\Omega $ and $\Omega'$ of a and $f(a)$ receptively, such that $f
  |\Omega$ is a $C^k$ (real analytic) isomorphism onto $\Omega'$}. 
    
\noindent
\textbf{Rank Theorem.} \textit{If $V^n$ and $W^m$ are $C^k$ (real
  analytic) manifolds and  $f$: $V \to W$, a $C^k$(real analytic) map
  such that rank  $f$ is a constant, $r$,  for all points in $V$, then
  for every point $a \in V$, there exists coordinate neighbourhoods
  $(U, \varphi), (U', \varphi')$ of a and $f(a)$ respectively such
  that $\varphi' \circ  f \circ  \varphi^{-1} | _{\varphi \Omega}$ is given
  by} 
$$\varphi'_1 \circ f \circ \varphi^{-1}(x_1, \ldots , x_n) = (x_1, x_2,
  \ldots , x_r, 0 , \ldots ,  0)
$$ 
    
\begin{defi*}
  If $V$ and $W$ are $C^1$ manifolds of dimension $n$ and $m$
  respectively, and $f: V \to W$ a $C^1$ map a point $a \in V$ is
  called critical if rank $ _a f<m$. 
\end{defi*}

\begin{defi*}
  If  $W$ is a  $C'$ manifolds of dimension $m$, countable at
  $\infty$, a set $E$ in $W$ is said to have measure zero in $W$ if
  for any coordinate neighbourhood $(U, \varphi)$, $\varphi (E \cap
  U)$ has measure zero in $\mathbb{R}^m$. 
\end{defi*}
    
It is clear that the notion of a set being of measure zero is
dependent of the coordinate neighbourhoods used in the
definition. 

\noindent
\textit{Sard's theorem}. \textit{If $V$ and $W$ are
  $C^\infty$ manifolds of dimension $n$ and $m$ respectively which are
  countable at infinity,and $f: V \to W a C^\infty$ map} and if $A$
is the set of critical points of $f$ in $V$, then $f(A)$ is of measure
zero in $W$. 
    
As\pageoriginale in Chapter I, we can prove the existence of partitions of  unity
: we have only to use the fact that if $(U, \varphi)$ is a coordinate
neighbourhood and $K \subset U$ is compact, then there is a $C^k$
function $\eta$ on $V$ with compact support $\subset U$ such that
$\eta (x) > 0$ for $x \in K$. We formulate this as separate theorem. 
    
\medskip
\noindent
\textbf{Partition of unity}. Given an open covering
$\{U_i\}_{i \in I}$ of a $C^k$ manifold $(0 \le k \le \infty) V$ which
is countable at  infinity, there exists a family $\{\varphi_i \}_{i
  \in I}$ of $C^k$ functions, $\varphi_i \ge 0$, with supp.
$\varphi_i \subset U_i$ such that the family  $\{ supp. \varphi_i\}$
is locality finite and $\sum \varphi_i (x) =1$ for any $x \in V$. 

\begin{coro*}
  If $F$ is a closed subset of $V$ and $U \supset F$ is open, there
  exists a $C^k$ function $\varphi$ on $V$ with $\varphi(x)
  = \begin{cases} 1 &\text{ if } x \in F \\ 0 &\text{ if } x \in V
    -U  \end{cases}$. 
\end{coro*}    
    
Let $V$ be a $C^k$ manifold, $k<2$. For any $C^{k-1}$vector field $X,a
 \in V$, let $x$ given by $X = \sum\limits_i \alpha_i
\dfrac{\partial}{\partial x_i}$ in a neighbourhood of  *****. 

Then for $f \in  C^k_a$, $X(f)$ can be considered as a function in
$C^{k-1}_a$, given by  
\begin{equation}
  X(f)(y) = \sum_i \alpha_i (y) \left(\frac{\partial f}{\partial
    x_i}\right) (
  y) \text { for } y \text{~ in a neighbourhood of}
  a. \tag{1.2}\label{chap2:sec2:eq1.2}  
\end{equation}   
      
If $Y$ be another $C^{k-1}$vector field given by $Y= \sum \beta_i 
\dfrac{\partial}{\partial x_i}$ in a neighbourhood of $a$. We define a
$C^{k-2}$ vector field [$X, Y$] by  
$$
[X, Y]_a (f) = X_a [Y(f)]- Y_a[X(f)]
$$  
and by (\ref{chap2:sec2:eq1.2}),
$$
Y_a [X(f)]= \sum_j \beta_j (a)  \left[ \sum_i \left\{ \left( \frac{\partial
    \alpha i}{\partial x_j}\right)_a \left(\frac{\partial f}{\partial
    x_i}\right )_a +
  \alpha _i (a) \left(\frac{\partial^2 f}{\partial x_j \partial x_i}\right)
  \right\}_a \right] 
$$
   
Hence\pageoriginale  
$$
[X,Y]_a (f) = \sum_i \left[ \sum_j \alpha_j (a) \left(\frac{\partial
    \beta_i}{\partial x_j}\right)_a - \beta_j (a) \left( \frac{\partial
    \alpha_i}{\partial x_j}\right)_a \right] \left( \frac{\partial f}{\partial
  x_i}\right)_a 
$$
It can be easily verified that for $C^{k-1}$ vector fields $X$, $Y$,
$Z$, $k \ge 3$, $[X, Y] =- [Y, X]$ and, if $k \ge 4$, 
$$
[X,  [Y,Z]] +  [Y, [Z,X]] +  [Z, [X, Y]] = 0.
$$
This is called the \textit{Jacobi identity}.
   

\heading{Differential forms on the product of two manifolds}
   
Let $V$ and $V'$ be $C^k$ manifolds (countable at infinity), $W = V
\times V'$, and $\pi$, $\pi'$ the projections of $W$ on $V$, $V'$
respectively. Then any $C^{k-1}$ form $\omega$ of degree $p$ on $V$
can be identified with the form $\pi^* (\omega)$on $W$; a similar
remark applies to $V'$.  

Let $A(V),\ldots$ denote the space of forms on $V, \ldots$ We
topologise $A(V)$ as follows: a sequence $\{ \omega_\nu\}$ of forms
$\omega_\nu \in A (V)$ tends to zero if, for any coordinate
neighbourhood $U$ on $V$ (coordinates $x_1,  \ldots , x_n)$ and any
compact subset $K$ of $U$, if $\omega^I_\nu$ denotes the coefficient
of $dx_{i_1} \Lambda \cdots \Lambda dx_{i_p} [I= (i_1, \ldots , i_p),
  i_1 < \cdots  < i_p, p =0, 1, \ldots  , n]$, then for any $I$,
$\omega^I_\nu$ and all its partial derivatives of order $< k$ tend to
zero  as $\nu \to \infty$. 

Using a partition of unity, we prove easily by applying
Cor. \ref{chap1:sec5:coro2} to
Theorem \ref{chap1:sec5:thm2} of Chap. I, \S\ 5, the following 
   
\setcounter{proposition}{0}
\begin{proposition}\label{chap2:sec2:prop1} % prop 1
  Finite\pageoriginale linear combinations of forms of the type $\prod^*(\omega)$
  $\Lambda \prod^*(\omega ')$, where $\omega$ is a form on $V$,
  $\omega'$ one on $V'$, are dense in $A(V \times V')$.  
\end{proposition}    

This implies of course that finite  linear combinations of forms of the
type $\prod^* ( \omega ) \Lambda \prod^*(\omega')$ where degree
$\omega +$ degree $\omega' =p$ are dense in the space of $p$-forms on
$W$; for $p=0$ this means that functions on $W$ can be approximated by
finite linear combinations of products of functions on $V$, $V'$
respectively. 
    
Corresponding statement for holomorphic forms on the product  of two
complex manifolds are also true. If $\mathscr{H}(V)$ denotes the space
of holomorphic forms on the complex manifold $V$, we topologies it by
means of convergence of the coefficient on compact subsets of
coordinate neighbourhood, just as we did above, (the convergence of
the derivatives is here a consequence of the convergence of the
coefficient since they are holomorphic functions). The density can be
proved along the lines of Theorem \ref{chap1:sec5:thm4} of Chap. I, \S\ 5; we have only
to introduce the Hilbert space corresponding to the space $A(\alpha)$
introduced in Chap. I, \S\ 5. 
    
Let $\{ U_i \}$ be a locally finite covering of $V$ by coordinate
neighbourhoods, and $\alpha$ a positive continuous functions on
$V$. Let $\omega$ be a homomorphic form on $V$, and let   
$$
\omega =  \sum_i \omega^{(i)}_I dz^{(i)}_I 
\begin{cases} 
  I = (i_1,\ldots , i_p), i_1 <\cdots < i_p, p =0, \ldots , n \\ 
  ds^{(i)}_I = dz^{(i)}_{i_1} \Lambda \cdots \Lambda dz^{(i)}_{i_p}
\end{cases} 
$$
in $U_i$. Let $\mathscr{H}_V (\alpha)$ denote the set of forms
$\omega$ for which 
$$
|| \omega ||^2 = \sum_i \sum_I \int\limits_{U_i}\left | \omega^{(i)}_I
\right|^2 \alpha (z^{(i)}) dv_z(i) < \infty. 
$$

Define\pageoriginale the Scalar products of $\omega, \omega' \in \mathscr{H}_V (\alpha)$ by
$$
(\omega , \omega') = \sum_i \sum_I \int\limits_{U_i}\omega^{(i)}_I
\overline{\omega'^{(i)}_I} \alpha (z^{(i)}) dv_z(i). 
$$

It is follows from Lemma \ref{chap1:sec5:lem2} of Chap. I, \S\ 5 that
convergence in 
$\mathscr{H}_V(\alpha)$ implies verified that $\mathscr{H}_V(\alpha)$
is complete. We can now prove, exactly as Theorem \ref{chap1:sec5:thm4} of Chap I, \S\
5 the following 

\begin{proposition}\label{chap2:sec2:prop2} % \prop 2
  If $\{ U_i \}, \{ U'_j\}$ are locally finite coverings of $V$, $V'$
  and $\alpha$, $\alpha'$ are positive continuous functions on $V$,
  $V'$, if $\{ \varphi'_\mu\}$, $\{ \varphi'_\nu\}$ are orthonormal
  bases for $\mathscr{H}_V (\alpha)$, $\mathscr{H}_{V'}$, $(\alpha')$,
  then $\prod^* (\varphi_\nu) A \prod^* (\varphi'_\mu)$ from an
  orthonormal basis for $\mathscr{H}_{V \times V'}(\alpha \times
  \alpha')$ with respect to the covering $U_i \times  U'_j$. Further,
  finite linear combinations of forms of the type $\prod^* (\omega )
  \Lambda^{' *}(\omega')$, $\omega$, $\omega'$ holomorphic forms on
  $V$, $V'$ respectively, are dense in the space of holomorphic forms
  on $V \times V'$. 
\end{proposition}

\section{Submanifolds}\label{chap2:sec3} % sec 3.

\begin{defi*}
  Let $V$ be a $C^k$ manifold, $k \ge 1$. $A$ $C^r$ submanifold of
  $V$, $0 < r \le k$ is a $C^r$ manifold $W$ and an injection $i: W
  \to V$ such that $i$ is a $C^r$ map and the map $i_*: T_A (W) \to
  T_{i (a)}(V)$ is an injection for every $a \in W$. 
\end{defi*}

We identity the submanifolds $(W_1, i_1)$ and $(W_2, i_2)$ if there
exists a $C^r$ isomorphism $h: W_1 \to W_2$ such that  $i_2 \circ h =
i_1$. 
\begin{remarks*} 
  \begin{enumerate}[1]
  \item It\pageoriginale follows immediately that 
    $$
    \dim. W \leq \dim V.
    $$
  \end{enumerate}
\end{remarks*}

Further, if the dimension of  $W = m$, from  the rank theorem it
follows that for $a \in W$, there exist coordinate neighbourhoods
$(U_1,  \varphi_1)$ of a and $(U_2, \varphi_2)$ of $i(a)$ such that, 
\begin{gather*}
  \varphi_2 \circ i \circ \varphi^{-1}_1 | \varphi_1 (U_1) \text{ is given by }\\
  \varphi_2 \circ i \circ  \varphi^{-1}_1 (x_1, \ldots , x_m) = (x_1, \ldots ,
  x_m , 0, \ldots , 0). 
\end{gather*}

Hence given a system of local coordinates at a, it can be ``extended
''(in an obvious sense) to a system at $i(a)$. 
\begin{enumerate}[2.]
\item If  $W$ is a closed subset of $V$, $V$ being a $C^k$ manifold of
  dimension $n$, if for each $a \in W$, there exists a coordinate
  neighbourhood $(U, \varphi)$, and if the local coordinate $(x_1,
  \ldots , x_n)$ in $U$ can be so chosen that  
  $$
  W \cap U = \left\{ x \bigg| x_{r+1} = x_{r+2}= \cdots = x_n = 0 \right\},
  $$
  then $W$ is  a $C^k$ manifold of dimension $r$, and is a submanifold of $V$.
\end{enumerate}

\begin{proof}
With the manifold structure defined in the obvious way, $W$ is a
$C^k$ manifold and the injection $i : W \to V$ is a $C^k$ map. It is
easily verified that $i_*$: $T_a (W)  \to T_{i (a)}(V)$ is an
injection. 
\end{proof} 

\begin{enumerate}[3]
\item If $V$ is a $C^k$ manifold of dimension $n$ and if  $f_{r+1},
  \ldots , f_n$ are $C^k$ functions on $V$ such that $df_{r+1}, df_{r+2}, \ldots
  , df_n$ are linearly independent at all points of  $W = \left\{ x
  \in V \bigg| f_{r+1} (x) = f_{r+2} (x) = \cdots = f_n (x) =0
  \right\}$, then $W$ is a submanifold of $V$, of dimension $r$. 
\end{enumerate}
 
\begin{proof}
  Since\pageoriginale $df_{r+1}, \ldots , df_n$ are locally independent at any point
  $a \in W$, we can find $C^k$ functions $f_1 , \ldots , f_r$ such
  that $(df_i)_a , 1 \le i \le n$,  are linearly independent at $a$;
  if $f= (f_1, \ldots , f_n)$ then, by the inverse function theorem,
  $f$ is a $C^k$ diffeomorphism in a neighbourhood of $a$, By the
  change of coordinates $(x_1,  \ldots ,  x_n) \to (y_1 , \ldots ,
  y_n), y_i = f_i (x_1,  \ldots , x_n)$, we have, 
  $$
  W \cap U = \left\{ x \bigg|  y_{r+1} = \cdots =  y_n  =  0\right\}.
  $$ 
\end{proof} 
 
Hence by remark 2), $W$ is a sub manifold of dimension $r$.
\begin{remark*}
  Similar definitions and results apply to real and complex analytic
  submanifolds. 
\end{remark*} 

\begin{coro*}
  In $\mathbb{R}^{n+1}$, the unit sphere given by 
  $$
  S^n =  \left\{ x \bigg| x^2_0 + x^2_1 + \cdots + x^2_n = 1 \right\},
  $$
  is a real  analytic submanifold of dimension $n$.
\end{coro*} 

\begin{proof}
  If $f$ is the function $x^2_0 + \cdots +  x^2_n -1$, df is $\neq 0$
  at all points of $S^n= \left\{ x \in \mathbb{R}^{n+1} \bigg|f(x) =
  0 \right\}$. 
\end{proof} 

\begin{enumerate}[4.]
\item If $V$, $V'$ are $C^k$ (real, complex analytic) manifolds, $V
  \times V'$ carriers a natural structure of $C^k$ (real, complex
  analytic) manifold. 
 \end{enumerate} 

\begin{defis*}
  \begin{enumerate}[1)]
  \item Let $V$ and $W$ be $C^k$ manifolds. Then a continuous map $f:V
    \to W$ is called locally proper if for every $y  \in  f(V)$, there
    exists a compact neighbourhood $U$ of $y$ in $W$ such that
    $f^{-1}(U)$ is compact. 

  \item If\pageoriginale $V$ and $W$ are $C^k$ manifolds then a continuous map $f: V
    \to W$ is proper if for every compact set $K$ in $W$, $f^{-1}(K)$ is
    compact. 
  \end{enumerate} 
\end{defis*} 

\begin{remark*}
  If $V$ and $W$ are $C^k$ manifolds and $f: V \to W$ is locally
  proper, then  $f$ is proper if and only if $f(V)$ is closed in $W$. 
\end{remark*}

\setcounter{proposition}{0}
\begin{proposition}\label{chap2:sec3:prop1} %\prop 1
  If $i$: $W \to V$ is a submanifold of $V^n$, then  the following
  statement are equivalent. 
  \begin{enumerate}[1)]
  \item $i$ is a homeomorphism of $W$ onto $i(W)$ with the induced
    topology from $V$.  
  \item The map $i$: $W \to V$ is locally proper.
  \end{enumerate} 
\end{proposition} 

\begin{proof}
  If the topology on $W$ is same as that on  $i(W)$, for any  $a \in
  W$ there exists a compact neighbourhood $K$ in $W$ for which $i(K)$
  is a compact neighbourhood of $i(a)$ in $i(W)$. Hence $i(K) \supset
  U_1 \cap i(W)$, $U_1$ open in $V$. Let $U_2$ be a relatively compact
  neighbourhood of $i(a)$ in $U_1$; then  
  
  $\bar{U}_2 \cap i (W) \subset \bar{U}_1 \cap i (W)$ and hence $i(K)$
  is compact and hence closed, $\bar{U}_1 \cap i (W) \subset i(K)$, 

  i.e. \quad $\bar{U}_2 \cap i (W) \subset i(K)$,

  i.e. \quad $i^{-1}(\bar{U}_2) \subset K$ and $i^{-1}(\bar{U}_2)$ is compact. 
\end{proof} 
 
\noindent
Hence 1) implies  2).
 
If the map $i$ is locally proper, for each $i(a)$ there exists a
compact neighbourhood $U$ in $V$ such that that $i^{-1}$ is
compact. Then\pageoriginale $i^{-1} (U)$ is a compact neighbourhood of a such that
$i \Big| i^{-1} (U)$ is a homeomorphism onto $i(U)$ since a continuous
bijective map form a compact space to a hausdorff space is a
homeomorphism. Hence 2) implies 1).  

Note that if 1) or 2) is satisfied, then $i(W)$ is locally closed in
$V$. The converse is, however, false. 

\begin{defi*}
  A submanifold $W$ of $V$ is called a closed submanifold if $i: W
  \to V$ is proper. 
\end{defi*}

We shall give an example of a submanifold for which the injection $i$
does not preserve the topology. For that we use the following 
\begin{theorem*}[(Kronecker)]
  Let $\alpha_1 ,\ldots , \alpha_n$ be $n$ real numbers which are
  linearly independent over the ring $\mathbb{Z}$ of integers, Let $T^n
  = S^1 \times \dots \times S^1 = \big \{ e^{i \theta_1}, \ldots ,
  e^{i \theta_n} )  | \theta_i$ real $\big \}$, and let $\omega$:
  $\mathbb{R} \to T^n$ denote the map $\omega (t) = (e^{i \alpha_1 t},
  \ldots , e^{e \alpha_n t})$. Then the image $\omega (\mathbb{R})$ is
  dense in  $T^n$. 
\end{theorem*}

The best proof of this theorem is, without question, that given by
H. Weyl \cite{44}. 

\begin{example*}
  $T^n$ defined above is a real analytic manifold of dimension
  $n$. Consider the map $\omega: \mathbb{R} \to T^n$ defined
  above. $\omega$ is an injection for if $\omega (x_1) = \omega
  (x_r)$, 
  $$
  \alpha_i x_1 = 2 \pi m_i + \alpha_i x_2, i = 1, \ldots ,n, m_i \in \mathbb{Z}
  $$
  and if $x_1 \neq x_2$, and $d_1,\ldots k_n$ are integers, not all
  zero, with $\sum k_i m_i = 0$, then $\alpha k_1 + \cdots + \alpha_n
  k_n = 0$, which contradicts the hypothesis  that\pageoriginale $\alpha_1, \ldots,
  \alpha_n$ are linearly independent over $\mathbb{Z}$. Also rank
  $(di)$ is maximal $= 1$ at all points of $\mathbb{R}$. Hence
  $\mathbb{R}$ is a submanifold of $T^n$.  Let $\omega (\mathbb{R}) =
  D$. If $\omega$ preserves the topology, by the proposition proved
  above, $D$ is locally closed and hence $D = \bar{D} \cap U$, $U$
  open in $T^n$. By Kronecker's theorem $D$ is dense in $T^n$ i.e. $D
  = T^n \cap U =U$. But $D$ is not open in $T^n$ if $n > 1$, and thus
  we arrive at a contradiction. 
\end{example*}

\begin{proposition}\label{chap2:sec3:prop2} % proposition 2.
  If $i: W \to V$ is a $C^k$ submanifold of $V$, and $M$ is a $C^k$
  manifold, then a continuous map $f$: $M \to W$ is $C^k$ if and only
  if $i \circ f: M \to V$ is $C^k$. 
\end{proposition}

\begin{proof}
  Let $a \in W$; choose coordinate neighbourhoods $U$ of a  in $W$ and
  $U'$ of $i (a)$ in $V$ such that $i \big| U \to U'$ is the map $i
  (x_1, \ldots,x_m) = (x_1, \ldots, x_m$, $0, \ldots , 0)$. We may
  restrict ourselves  to the subset $N = f^{-1} (U)$ of $M$. The
  proposition is than obvious since if $f : N \to U$ has components
  given by $f (u) = ( f_1 (u), \ldots , f_m(u))$, then $i \circ f: N \to
  U'$ has components given by  $i \circ f (u) = (f_1 (u), \ldots , f_m (u),
  ~ 0, \ldots, 0 )$. 
\end{proof}

\begin{proposition}\label{chap2:sec3:prop3} %3
  If $i: W \to V$ is a $C^k$ submanifold, then for a germ $g_a$ of a
  continuous function at $a \in W$ to be $C^k$, it is necessary and
  sufficient that there is a $C^k$ germ $G_b$ at $b = i (a)$ such that
  $G_b \circ i = g_a$. Conversely. if $i$ is a continuous injection of the
  $C^k$ manifold $W$ into $V$ having this property, then $i : W \to V$
  is a submanifold. 
\end{proposition}

\begin{proof}
  Let $i$ be a submanifold and choose coordinates at $a, ~ (U; x_1,
  \ldots$, $x_m), (U'; x_1, \ldots, x_n)$  at $b = i(a)$ such that $i
  \Big| U$ is the map $i (x_1, \ldots, x_m) = (x_1, \ldots,x_m, ~ 0,
  \ldots, 0)$. If $g$ is $C^k$ on $U$, and $G$ is the $C^k$ function
  on $U'$ defined by $G(x_1, \ldots, x_n) =  g (x_1, \ldots, x_m)$,
  clearly $G \circ i  = g$. 
\end{proof}

Conversely,\pageoriginale let $i: W \to V$ be an injection such that $C^k$ germs $g$
at a are precisely the germs $G ~ 0 ~ i$, $G$, a $C^k $ germ on $V$ at
$b = i (a)$. Then $i$ is $C^k$ for if, in terms of local coordinates
$(U'; x_1, \ldots, x_n)$ at $b$,  $i_1 \ldots, i_n$ are the
components of $i$, then $i_l = x_l \circ i$, and $x_l \in C^k$. We assert
that there exists a germ of $C^k$ map $p : V \to W$ at $b \in V$, $ p
(b) = a$, such that $p \circ i =$ identity near a in $W$. In fact, if $(U;
x_1, \ldots, x_m)$ are local coordinates at $a \in W$, then, by
hypothesis, there exist $C^k$ germs $P_l,  l = 1, \ldots , m$ at $b$
such that $x_l - p_l \\circ i$; the $p_l$ may be looked upon as the germ of
a $C^k$ mapping $p$: $V \to U$ for which $p (b) = a$, $p \circ i =$
identity near a in $U$. 

We then have 
$$
(p_*)_{i (a)} \circ (i_*)_a = \text{ identity on } T_a (W),
$$
so that $(i_*)_a$ is injective.

\begin{proposition}\label{chap2:sec3:prop4}% proposition 4.
  If $i$: $W \to V$ is a closed submanifold, i.e. $i$ is proper, then
  for any $C^k$ function $g$ on $W$, there exists a $C^k$ function $G$
  on $V$ such that $G \circ i = g$. 
\end{proposition}

\begin{proof}
  We identify $W$ with $i(W)$. Let $U_a$ be a neighbourhood of a in
  $V$, $G_a$  a $C^k$ function in $U_a$ with $G_a = g$ on $U_a \cap
  W$. Let  $\big\{ U_{a_{\alpha}}, V- W \big \}_{\alpha \in A}$ be a
  locally finite covering of $V$ such that for each $\alpha$,
  $U_{a_{\alpha}} \subset U_a$ for some $a \in W$. Let
  $(\varphi_\alpha, \varphi )$ be a $C^k$ partition of unity relative
  to this covering and $h_\alpha = \varphi_\alpha. G_{a_{\alpha}}$ in
  $U_{a_\alpha}$, 0 in $V -U_{a_{\alpha}}$. Clearly, if $G = \sum h_\alpha$,
  then $G$ is $C^k$ on $V$ and, for $x \in W$, $G (x) = \sum\limits_{x
    \in U_{a_{\alpha}}} h_\alpha (x) =  \sum \limits_{x \in
    U_{a_{\alpha}}} \varphi_\alpha (x)$. $G_{a_{\alpha}} (x) = g (x)
  \sum\limits_{x \in U_{a_{\alpha}}} \varphi_\alpha (x) = g (x)$. 
\end{proof}

\begin{remark*}
  Propositions\pageoriginale \ref{chap2:sec3:prop2} and
  \ref{chap2:sec3:prop3} and their proofs remain valid for real or 
  complex analytic manifolds. Prop. \ref{chap2:sec3:prop3} is true for
  real analytic 
  manifolds, but is very difficult to prove; see $H$. Cartan \cite{6} and
  $H$. Grauert \cite{13}, it is false for complex manifolds in general. A
  very important special case, due to $K$ Oka, for which it is true
  will be dealt with later (\S\ 7). 
\end{remark*}

\section{Exterior differentiation}\label{chap2:sec4} % \section 4.
 
If $V$ is a $C^k$ manifold, $A^p_r (V)$ denotes the $C^r$ differential
forms of degree $p$, on $V$, $0 \leq r < k $  if $p > 0$, $0 \leq r
\leq k$ if $p = 0$. In what follows $V$ shall denote a $C^k$ manifold
which is countable at $\infty$ with $k \geq 2$. 

\begin{defi*} 
  An exterior differentiation $d$ is a map $d$: $A^p_r (V) \to
  A^{p+1}_{r-1} (V)$ for each $p \geq 0$ and  $1 \leq r < k$ if $p >
  \circ$, $ 1 \leq r \leq k$ if $p = 0$, satisfying the following. 
  \begin{enumerate}[1)]
  \item $d$ \textit{is $\mathbb{R}$-linear, i.e. $d \left[ \alpha
      \omega_1 + \beta \omega_2 \right] = \alpha d \omega + \beta d
    \omega_2 $ for $\alpha$, $\beta \in \mathbb{R}$, $\omega_1$,
    $\omega_2 \in A^p_r (V)$}. 
  \item \textit{$d | A^0_k (V)$ is given by $(df)_a =$ the image of
    $f$ in $T^*_a (V)$}. 
  \item \textit{$d (df) = 0$ for $f \in C^k_a$}.
  \item \textit{If $\omega_1 \in A^p_r (V)$, $\omega_2 \in A^q_r (V)$,
    $d (\omega_1 \wedge \omega_2) = d \omega_1 \wedge \omega_2 +
    (-1)^p \omega_1 \wedge d \omega_2$}. 
  \end{enumerate}
\end{defi*}

We deduce the  following properties of an exterior differentiation
form its definition. 
\begin{enumerate}[I.]
\item $d$ \textit{is a local operator, i.e. if for an open set $U$ we
  have $\omega \nmid U = 0$. then $d \omega \nmid_ U = 0$}. 
\end{enumerate}

\begin{proof}
  If\pageoriginale $U'$ is a coordinate neighbourhood $\subset U$, and $U''$ is a
  relatively compact subset of $U$, there exists a $C^k$ function $f$
  on $U'$ which is ***** on $U'' = 1$ in a neighbourhood of $\partial
  U'$; hence there exists $f \in C^k (U)$ such that  
  
  $\begin{aligned}
    f (x) &= 0 \text{ for } x \in U''\\
    &= U \text{ for } x \in V -U.
  \end{aligned}$
\end{proof}

Hence if $\omega \nmid U = 0, \omega = f \omega$ so that $d \omega =
(df) \wedge \omega + f d \omega$ Since $f = 0$, and by $2)$, $df = 0$
on $U''$, we deduce that $d \omega \nmid U'' = 0$. It follows that $d
\omega$ vanishes in a neighbourhood of any point of $U$, so that $d
\omega | U = 0$. 

\begin{enumerate}[II.]
\item \textit{$d^2 = 0$}~ (if $k \geq 3)$. 
\end{enumerate}

\begin{proof}
  It is enough to prove this with $V$ replaced by a coordinate
  neighbourhood. Let 
  $$
  \omega= \sum_{i_1< \dots < i_p} f_{i_{1}\ldots i_p} dx_{i_{1}}
  \wedge dx_{i_{2}} \wedge \ldots \wedge dx_{i_{p}} \epsilon
  A^p_{k-1} (V). 
  $$
\end{proof}

Then
\begin{multline*}
  d(d \omega ) =  \sum\limits_{i_1< \dots < i_p}\\ 
  d  \big [ df_{i_{1}
  \ldots i_p} \wedge dx_{i_{1}} \wedge \ldots \wedge dx_{i_{p}} +
  f_{i_{1} \ldots i_p} d (dx_{1_{1}} \wedge \ldots \wedge dx_{i_{p}})
  \big]. 
\end{multline*}

Now, by 3) and 4),   $d (dx_{i_{1}}\wedge \ldots d x_{i_p})$
\begin{align*}
  &= \sum^{p}_{r=1} (-1)^{r-1} (dx_{i_{1}} \wedge \ldots \wedge
  d^2 x_{i_r} \wedge \dots \wedge dx_{i_{p}}) \\ 
  &= 0. \\
\end{align*}

Hence $d (d \omega) =  \sum\limits_{i_1< \dots < i_p} \big\{
df_{i_{1}\ldots i_p} \wedge d (dx_{i_{1}} \wedge \ldots dx_{i_{p}}) +
d^2 f_{i_{1}\dots i_p} \wedge dx_{i_{1}} \ldots \wedge dx_{i_{p}} \big
\} =0$. 

We\pageoriginale shall now prove \textit{the existence and uniqueness of the
  exterior differentiation}. It suffices to prove the existence and
uniqueness any coordinate neighbourhood. 

Define $d_1$ by $d_1 (\omega) = \sum \limits_{i_1< \dots < i_p} d
(f_{i_{1}\ldots i_p}) \Lambda dx_i \Lambda \ldots \Lambda dx_{i_{p}}$
where $\omega \in A^p_r (V)$ is given by  
$$
\omega = \sum_{i_1< \dots < i_p}  f_{i_{1}\ldots i_p} dx_{i_{1}}
\Lambda dx_{i_{2}} \Lambda \ldots \Lambda dx_{i_{p}}  
$$

It is easily seen that $d_1$ satisfies the conditions 1) and 2). As for 3)
\begin{align*}
  d_1 f &= \sum_{i} \frac{\partial f}{\partial x_i} dx_i \\
  \Rightarrow d^2_1 f &= \sum_{i} \left( \sum_{j} \frac{\partial^2
    f}{\partial x_j \partial x_i} dx_j \right) \Lambda dx_i \\ 
  &= \sum_{i, j} \frac{\partial^2 f}{\partial x_j \partial x_i} dx_j
  \Lambda dx_i = \sum_{j < i} \left(\frac{\partial^2 f}{\partial x_j
    \partial x_i} - \frac{\partial^2 f}{\partial x_i \partial x_j}\right)
  dx_j \Lambda dx_i \\ 
  &= 0
\end{align*}

We shall show that $d_1$ satisfies $4)$. It is enough to verify this
for $\omega_1 = f_1 dx_{I_{1}}$, $\omega_2 = f_2 dx_{I_{2}}$, where
$dx_{I_{1}} = dx_{i_{1}} \Lambda \ldots \Lambda dx_{i_{p}}$ and
$dx_{I_{2}} = dx_{j_{1}} \Lambda \ldots \Lambda dx_{j_{q}}$. 

Now,\pageoriginale
\begin{align*}
  \omega_1 \Lambda \omega_2 = f_1 f_2 &dx_{I_{1}} \Lambda dx_{I_{2}} \\
  d_1 (\omega_1 \Lambda \omega_2 ) &= d_1  (f_1f_2) \Lambda dx_1
  \Lambda dx_{I_{2}} \\ 
  &= \big [ (d_1 f_1). f_2 + f_1 (d_1 f_2) \big ] \Lambda dx_{I_{1}}
  \Lambda dx_{I_{2}} \\ 
  &=  d_1 f_1 \Lambda dx_{I_{1}} \Lambda  f_2  dx_{I_{2}}  + (-1)^p
  f_1dx_{I_{1}} \Lambda d_1 f_2 \Lambda dx_{I_{2}} \\ 
  &=  (d_1 \omega_1 ) \Lambda \omega_2 + (-1)^p \omega_1 \Lambda (d_1
  \omega_2 ).
\end{align*}

We are using the obvious fact that the $d$ in 2) of the definition of
exterior differentiation satisfies  $d (f_1 f_2) = f_1 df_2 + f_2
df_1$. Hence $d_1$ defined above is an exterior differentiation. If
$d_2$ is another exterior differentiation, $\omega = f  dx_{i_{1}}
\Lambda \cdots \Lambda dx_{i_{p}}$, it follows  from $4)$ that $d_2
\omega = d_2 f \Lambda dx_{i_{1}} \Lambda \cdots \Lambda dx_{i_{p}} +
\sum^{p}_{r=1} (-1)^{r-1} f  dx_{i_{1}}  \Lambda \cdots \Lambda d_2
(dx_{i_{r}} ) \Lambda \cdots \Lambda dx_{i_{p}}$. 

By 2) and 3) it follows that $d_2 f = d_1 f$ and $d_2 (dx_{i_{r}}) =
d_2 (d_2dx_{i_{r}} ) = 0$. Hence $d_2 \omega = df \Lambda dx_{i_{1}}
\Lambda \cdots \Lambda dx_{i_{p}}$, i.e. the exterior differentiation
is unique. 

We have already remarked that $T^*_a (V)$ is the dual of $T_a
(V)$. Consider $\overset{p}{\Lambda} T^*_a (V)$ as  the dual of
$\overset{p}{\Lambda} T_a (V)$, i.e. for every $p$-form
$\omega. \omega_a$ defines an alternate linear function of $\sum
\limits^{p}_{r=1} T_a (V)$ which determines $\omega_a$ uniquely. Hence
$\omega$ gives rise to an alternate mapping of $p$-tuples  of $C^{k-1}$ vector
fields into $C^{k-1}$ functions. 

\setcounter{proposition}{0}
\begin{proposition}\label{chap2:sec4:prop1}%proposition 1
  If\pageoriginale $\omega$  is a $p$-form, $X_1, \ldots X_{p+1}$, $C^{k-1}$ vector
  fields, then for any $a \in V$, $d \omega$ is the linear function
  given by  
  \begin{multline*}
    (d \omega ) (X_1, X_2, \ldots , X_{p+1})  = \sum^{p+1}_{1}
    (-1)^{i-1} X_i (\omega (X_1, \ldots X_i, \ldots ,x_{p+1}))\\ 
    +  \sum_{i
      < j} (-1)^{i -j} \omega ([ X_i, X_j ] ~ , X_1, \dots ,
    \hat{X}_i, \dots, \hat{X}_j, \dots, X_{p+1}).  
  \end{multline*}
\end{proposition}

Here the $\Lambda$ over a term means that this term is to be omitted.

We shall prove the proposition only when $\omega$ is a 1-form. The
general case involves more complicated calculations. It is sufficient
to prove the formula in a coordinate neighbourhood. By linearity, it
is enough to prove it for forms of the type $\omega = f\ dg$. If
$\omega = f dg$, $f$, $g$ functions, then $d \omega = df \Lambda dg$. 

Hence
\begin{align*}
  D &= (df \Lambda dg) (X_1, X_2) \\
  &= \det  \begin{vmatrix} (df) (X_1) &  (df) (X_2) \\  & \\ (dg)
    (X_1) & (dg) (X_2) \\ 
   \end{vmatrix} 
\end{align*}
where
$$
(df)  X_1 = X_1 (f).
$$

Hence
\begin{align*}
  D &= X_1 (f)  X_2  (g) - X_2  (f)  X_1 (g)\\
  &= X_1 (f X_2 (g) - X_2 (f ~ X_1 (g)) ~ - f X_1 (X_2 (g))) + f  X_2
  (X_1 (g)).\\ 
  &=  X_1 (\omega (X_2))  - X_2 (\omega (X_1))  \omega ( [X_1, X_2] )
\end{align*}
which is the required formula.

\begin{proposition}\label{chap2:sec4:prop2} % proposition 2.
  If\pageoriginale $V$ and $W$ are $C^k$ manifolds and $f: V \to W$
  a $C^k$ map,  we have, for any $p$ form $\omega$ on $W$, 
  \begin{equation}
    d_{(V)} (f^* (\omega )) = f^* (d_{(W)} (\omega
    )). \tag{3.1}\label{chap2:sec4:eq3.1} 
  \end{equation} 
\end{proposition}

\begin{proof}
  We may clearly suppose that $W$ is an open set in
  $\mathbb{R}^m$. Since $f^*$ is an algebra homomorphism of $\Lambda T^*
  (W)$ into $\Lambda T^* (V)$, it is enough to prove
  (\ref{chap2:sec4:eq3.1}) for a 
  system of generators of $\Lambda T^* (W)$ e.g. when $\omega$ is a
  function or is the exterior derivative of a function. If  $\omega =
  \varphi$ is a function, 
  $$
  (d \varphi)_{f (a)} \in C^k_{f (a)} /_{S^k_{f (a)}}
  $$
  and $f^* \big[ (d \varphi)_{f (a)} \big] = d (\varphi \circ f)_a$ by
  definition of $f^*$. 
\end{proof}

If $\omega = d\varphi$ where $\varphi$  is a function,
\begin{align*}
  &f^* (d (d \varphi )) = 0  \text{ and } \\
  &d [ f^* (d \varphi )] = d [ d (\varphi \circ f ) ] = 0.
\end{align*}
\hfill{q.e.d.}

For a somewhat different approach to exterior differentiation set
Koszul \cite{22}. 

\section{Orientation and Integration}\label{chap2:sec5} % section 5.

\begin{defi*}
  On a $C^k$ manifold $V$ with $k \geq 1$, a continuous $n$-form
  $\omega$ which is nowhere zero on $V$ is called an orientation on
  $V$ and if there exists an orientation on $V$, $V$ is called
  orientable. 
\end{defi*}

\setcounter{proposition}{0}
\begin{proposition}\label{chap2:sec5:prop1}% proposition 1.
  A\pageoriginale manifold $V$ is orientable if and only if there exists a system of
  coordinates $(U_i, \varphi_i)$, $\bigcup U_i = V$, such that the
  transformation $\varphi_i \circ \varphi^{-1}_j | \varphi_{j (U_i \cap
    U_j)}$ has  positive jacobian det $| d (\varphi_i \circ
  \varphi^{-1}_j) |$ whenever $U_i \bigcap U_j \neq \phi$. 
\end{proposition}

\begin{proof}
  If $\omega$ is an orientation of $V$, for any $a \in V$ there exists
  a connected coordinate neighbourhood $(U_a, \varphi)$ of a such that
  in terms of local coordinates $\omega_x = f (x) dx_1 \Lambda \cdots
  \Lambda dx_n$, for  $x \in U_a$. Further $\varphi$ can be so chosen
  that $f (x) > 0$ for $x \in U_a$ (change $x_1$ to $-x_1$ if
  necessary ). Consider a system of coordinate neighbourhoods $(U_i,
  \varphi_i )$, such that for any $x \in U_i$, $\omega_x$ in terms of
  local coordinates can be  written as $\omega_x = f_i (x) dx_1 ^{(i)}
  \Lambda \cdots \Lambda dx_n^{(i)}$, $f_i (x) > 0$. Then the jacobian
  of the transformation $\varphi_i \circ \varphi^{-1}_j$ is a quotient of
  the functions $f_i \circ \varphi^{-1}_i$ and  so $> d$. 
\end{proof}

Conversely if there exists a system of coordinate neighbourhoods
$(U_i, \varphi_i)$ with the above property, consider a partition of
unity $\big\{ \Psi_i \big\}$ subordinate to the covering $\big\{ U_i
\big\}$. Define $\omega_x$ in terms of local coordinates as $w_x =
\sum_{i} \Psi _i (x) dx_1^i \Lambda \cdots \Lambda dx^i_n$. 

Then $\omega_x$ is a continuous $n$ form which is  $> 0$ for
every $x$ and hence is an orientation of $V$. 
\begin{remark*}
  It follows that on a $C^k$ manifold, there is a $C^{k-1} n$ form
  which is nowhere zero. 
\end{remark*}

Let 
$$
\displaylines{\hfill 
  E = \big\{ \xi \in \overset{n}{\Lambda} T^* (V) | \xi \neq 0 \big
  \}\hfill \cr
  \text{and}\hfill 
  p (\xi) = a \text{ if } \xi \in \overset{n}{\Lambda} T^*_a (V).\hfill }
$$

Define\pageoriginale an equivalence relation in $E$ by
\begin{align*}
  \xi_1 \sim \xi_2  \text{ if }  ~ x &= p (\xi_1 ) = p (\xi_2 ),
  \text{ and there is } ~ \lambda > 0 \text{ with } \\ 
  \xi_1 &=  \lambda \xi_2. 
\end{align*}

Let $\tilde{V} = E /_{\sim}$.

\begin{proposition}\label{chap2:sec5:prop2} % proposition 2.
  $V$ is hausdorff and $\tilde{V} \to V$ is a covering.
\end{proposition}

\begin{proof}
  The equivalence relation is clearly open and it is easily seen that
  the graph of the equivalence relation in $E \times E$ is closed
  Hence $V$ is hausdorff. Let $(U, \varphi)$ be a coordinate
  neighbourhood of $a \in V$. Define $\xi$ and $\eta$ in terms of
  local coordinates as  
  $$
  \displaylines{\hfill 
  \xi_x = dx_1 \Lambda \cdots \Lambda dx_n \hfill \cr
  \text{and}\hfill 
  \eta_x = - dx_1 \Lambda \cdots \Lambda dx_n.\hfill }
  $$
\end{proof}

Then $p^{-1} (U) = \left(\bigcup \limits_{x \in U} \xi_x\right) \bigcup \left(
\bigcup \limits_{x \in U} \bar{\eta}_x\right)$ and $ \tilde{V}$ is a
covering. 

\setcounter{corollary}{0}
\begin{corollary}\label{chap2:sec5:coro1} % corollary 1.
  If $V$ is connected, $V$ is orientable if and only if  $\tilde{V}$
  is not connected. 
\end{corollary}

\begin{proof}
  If $V$ is connected and orientable, let $\omega$ be an
  orientation. Then  $\bigcup\limits_{x \in V} \overline{(\omega_x)}$
  is non-empty open and closed subset of $\tilde{V}$ and hence
  $\tilde{V}$ is not connected. 
\end{proof}

If $V$ is connected and $\tilde{V}$ not connected, consider
$\bar{\xi}_a \in \tilde{V}$ and let $U_a$ be the connected component
of $\bar{\xi}_a$ in $\tilde{V}$.  Then if $p| U_a = \pi$, $\pi$: $\to
V$ is a covering and if $p^{-1} (x) = \pi^{-1} (x)$ for some $x \in
V$, $ p^{-1} (y) = \pi^{-1} (y)$ for every $y \in V$ and $U_a =
\tilde{V}$ which is\pageoriginale a contradiction. Hence for any $x
\in V$, 
$\pi^{-1} (x)$ contains exactly one point, so that $\pi$ is a
homeomorphism. It is easily verified that there is a continuous $n$
form $\omega$ on $V$ for which $\omega_x \in \pi ^{-1} (x)$ for any
$x$; hence $V$ is  orientable. 

\begin{corollary}\label{chap2:sec5:coro2} % corollary 2.
  If $V$ is a simply connected manifold it is orientable.
\end{corollary}

\begin{proof}
  If  $V$ is simply connected, clearly $\tilde{V}$ is not connected
  and the proof follows form Corollary \ref{chap2:sec5:coro1}. 
\end{proof}

It can be be show  easily that $\tilde{V}$ is \textit{always}
orientable; in fact if, for $a \in \tilde{V}$, $ \omega_{p (a)}$ is an
$n$-co vector at $p (a)$ with $\omega_{p (a)} \in a$, we see at once 
(partition of unity) that  there is an $n$-form $\tilde{\omega}$ on
$\tilde{V}$ with  $\tilde{\omega}_a  = \lambda_a p^*  (\omega_{p
  (a)})$ with  $\lambda_a > 0$. 

Let\qquad $\mathbb{R}^n_+ = \big\{ (x_1, \ldots ,x_n) \in
\mathbb{R}^n | x_1 \geq \circ \big\}$. 

\begin{defi*}
  A hausdorff topological space $V$ is said to be a $C^k$ manifold
  with boundary and of dimension $n$ if there exists  a system of
  ``coordinate neighborhoods" $(U_i, \varphi_i )$, such that $\bigcup
  U_i = V$ and $\varphi_i$ is a homeomorphism of $U_i$ onto an open
  subset of $\mathbb{R}^n_+$ for which whenever  $U_i \cap U_j \neq
  0$, then map $\varphi_i \circ \varphi^{-1}_j \Big| \varphi_j (U_i
  \cap U_i )$ is a $C^k$ map of  $\varphi_j (U_i \cap U_j)$ as a
  subset of $\mathbb{R}^n_+$. 
\end{defi*}

If  $f$ is a real valued function on $\mathbb{R}^n_+$,
$\dfrac{\partial f}{\partial x_i}$, $ i \geq 2$ are defined in the
same way as for a function  on  $\mathbb{R}^n$ and  
$$
\dfrac{\partial
  f}{\partial x_1} \Big|_a = \lim\limits_{h \to  + 0} \dfrac{f
  (a_1 +h, ~ a_2, \ldots, a_n) ~ - f (a_1, \ldots , a_n)}{h}.
$$

For a  $C^k$ manifold $V$ with boundary, $k \geq 1$, $C^k$ functions,
tangent vectors, $T_a (V)$, differential forms etc. are defined in the
same way\pageoriginale as for a manifold. Orientation is  also defined in an
analogous way. Hereafter $V$ will denote a $C^k$ manifold with
boundary, which is countable $\infty$. 

\begin{defi*}
  A vector $X$ in $T_a (V)$ is called a positive tangent vector or an
  inner normal if for any $f \in m^k_a$ with $f (x) \geq 0$ for $x$ in
  a neighbourhood of $a$, we have $X (f) \geq 0$ and if there exists an
  $f \in m^k_a$, $f (x) \geq 0$ in a neighbourhood of $a$, for which
  $X (f) > 0$. $(m^k_a$ is the set of  $C^k$ germs at a which vanish
  at $a$.) A tangent vector $X$ is negative (or an outer normal) if
  $-X$ is positive. 
\end{defi*}

Let a be a point  in $V$. If there exists a coordinate neighbourhood
$(U, \varphi)$ of a, such that $\varphi (U)$ is an open set in
$\mathbb{R}^n$, for any $f \in m^k_a$ consider  $f$ as a function of
local coordinates. If $f (x) \geq 0$ for $x$ in a neighbourhood $U'$
of $a$, $U' \subset U$, $f$ has a minimum at a and  hence
$\dfrac{\partial f}{\partial x_i} \Big|_a = 0$, $1\leq i \leq
n$. Hence for any $X \in T_a (V)$, $X (f) = \sum x (x_i)
\dfrac{\partial f}{\partial x_i} |_a = 0$, i.e. there does not exist
a positive tangent vector in $T_a (V)$. 

\begin{proposition}\label{chap2:sec5:prop3}  % proposition .3
  Let $a \in V$ and suppose that there exists a coordinate
  neighbourhood $(U, \varphi )$ of a such that $\varphi (U)$ is an
  open set of $\mathbb{R}^n_+$ and $\varphi (a) \in \big\{ x \in
  \mathbb{R}^n_+ \Big| x_1 = 0 \big \}$, then a tangent vector $X_i
  \in T_a (V)$ given by  $X = \sum \alpha_i  \dfrac{\partial}{\partial
    x_i}$ is a positive tangent vector if and only if $\alpha_1 > 0$.  
\end{proposition}

\begin{proof}
  We may suppose that $\varphi (a) = 0$. If $f \in m^k_a$ and
  $f (x) \geq 0$ in a neighbourhood $U'$ of $a$, consider $f (a_1, x_2,
  \ldots , x_n)$ as a function of $(x_2, \ldots , x_n)$, in terms of
  local coordinates. The set $U_1 = \big\{ (x_2, \ldots , x_n)\break 
  (x_1, \ldots , x_n) \in \varphi (U) \big\}$, is open in
  $\mathbb{R}^{n-1}$ and by the same argument as above
  $\dfrac{\partial f}{\partial x_i} \Big|_a = 0$, $2 \leq i \leq
  n$. Now, if  $f \in m^k_a$, $f \geq 0$, we have  
  \begin{align*}
    \frac{\partial f}{\partial x_1} \Big|_a  &=  \lim\limits_{h \to +
      0}  \frac{f (h, 0, \ldots , 0) ~ - f (0, \ldots ,0)}{h} \\ 
    &= \lim\limits_{h \to + 0}  \frac{f (h, 0, \ldots , 0)}{h} \geq 0
  \end{align*}
  and\pageoriginale by choosing $f (x) = x_1$, we see that 
  $$
  \frac{\partial f}{\partial x_1} \Big|_a = 1,  ~ f \in m^k_a,  ~
  \text{ and } ~ f (x) \geq 0 \text{ for } x  \text{ in } U. 
  $$
\end{proof}

Then \qquad \quad $X (f) = \alpha_1 \left(\dfrac{\partial f}{\partial x_1}
\Big|_a \right)  = \alpha_1$. 

Hence if $\alpha_1 > 0$, $X$ is a positive tangent vector and
conversely if $X$ is a positive tangent vector $\alpha_1 > 0$. 
\begin{defi*}
  An element $\omega \in T^*_a (V)$ is called positive (negative) if
  $\omega (X) > 0 (< 0)$ for any positive $X \in T_a (V)$. 
\end{defi*}

In terms of local coordinates $(U, \varphi )$, $\varphi (U) \subset
\mathbb{R}^n_+$, $\varphi (a) = 0$, an element $\omega = \sum \alpha_i
dx_i $ is  $ > 0$ if and only if $\alpha_1 >  0$, $\alpha_2 =  \cdots
= \alpha_n = 0$. 

\begin{defi*}
  The set $\partial V = \big\{ x \in V  \Big|$ there exists a
  coordinate neighbourhood $(U, \varphi)$ of $x$ with $\varphi (x) = 0
  \big \}$ is called the boundary of $V$. 
\end{defi*}

\begin{remark*}
  It is clear from the above discussion that $x \in \partial V$ if and
  only if there exists a positive tangent vector in $T_x (V)$. $V$ is
  said to have no boundary if $\partial V = \phi$. 
\end{remark*}

\begin{proposition}\label{chap2:sec5:prop4} % proposition 4.
  $\partial V$ is a $C^k$ manifold of dimension $n-1$.
\end{proposition}

\begin{proof}
  If $a \in \partial V$, there exists a coordinate neighbourhood $(U,
  \varphi )$ of a such that $\varphi (a) = 0$. 
\end{proof}

Let\pageoriginale $U' = \Big\{ x \in U \Big|  \varphi (x) = \circ \Big\}$. Clearly $U' =
\partial V \cap U$. For $x \in U'$, define $\varphi'$ by 
$$
\varphi' (x) = (x_i)_{2 \leq i \leq n}
$$
$\varphi'$ is obviously a homeomorphism of $U'$ onto open set in
$\mathbb{R}^{n-1}$. If $(U_1, \varphi_1), ~ (U_2, \varphi_2)$ are
coordinates in $V$ inducing coordinates $(U'_1, \varphi'_1)$, $(U'_2,
\varphi'_2)$ on $\partial V$, the map $\varphi'_1 ~\circ
(\varphi'_2)^{-1}$ is the restriction of  $\varphi_1 \circ \varphi^{-1}_2$
to a submanifold of $\varphi_2 (U_2)$, and so is  $C^k$ and so is
$C^k$. Thus  $(U'_i, \varphi'_i)$ is a  system of coordinate
neighbourhood for  $\partial V$ and $\partial V$ is a $C^k$ manifold
of  dimension $n-1$. It is obvious that $\partial V$ has no boundary. 

Further $\partial V$ is clearly a $C^k$ submanifold of $V$ which is in
fact a closed submanifold. We shall therefore identify, for $a \in
\partial V$, the tangent space $T_a (\partial V)$ with a subspace of
$T_a (V)$.  

\begin{proposition}\label{chap2:sec5:prop5}% proposition 5.
  If $a \in \partial V$ and $X_1$, $X_2$ are positive tangent vectors
  in $T_a (V)$, then there exists $\alpha > 0$ such that  $X_1 -
  \alpha X_2 \in T_a (\partial V )$.  
\end{proposition}

\begin{proof}
  In terms of a local coordinate system $( U, \varphi )$ at $a$, let 
  $$
  X_1 = \sum \alpha_i  \frac{\partial}{\partial x_i}, X_2 ~ =  \sum
  \alpha_i  \frac{\partial}{\partial x_i}. 
  $$
\end{proof}

Then $\alpha_1 > 0, ~ \beta_1 > 0$. Let $\alpha =
\dfrac{\alpha_1}{\beta_1}$, then 
$$
X_1 - \alpha X_2 = \sum^{n}_{i=2} (\alpha_i  - \alpha \beta_i )
\frac{\partial}{\partial x_i}  \in T_a (\partial V). 
$$

\begin{defi*}
  $\xi \in  \overset{n-1}{\Lambda} T_a (V)$,  is called positive if
  for any outer normal $e \in T_a (V)$, $e \Lambda \xi$, as an element
  of $\overset{n}{\Lambda} T_a (V)$, is positive. 
\end{defi*}
 
It\pageoriginale is clear that we may look upon $\overset{n-1}{\Lambda} T_a
(\partial V)$ as a subspace of $\overset{n-1}{\Lambda}\break T_a (
V)$; similar remarks apply to  $\overset{n-1}{\Lambda} T_a^* (
\partial V)$ and $\overset{n-1}{\Lambda} T_a^* (V)$.

\begin{prop*}
  If $V$ is oriented, so is $\partial V$.
\end{prop*}

\begin{proof}
  As in the above definition, we say that $\omega_1 \in
  \overset{n-1}{\Lambda} T^*_a (\partial V)$ is  positive if for any  
  $$
  \omega_2 \in  T^*_a (V), ~ \omega_2 < 0, \text{ we have } ~ \omega_2
  \Lambda \omega_1 > 0.  
  $$
\end{proof}

Let $\omega$ be the orientation of $V$  and in terms of ``positive"
local coordinates  suppose that  
$$
\omega_x = f(x) dx_1 \Lambda \cdots \Lambda dx_n, ~  f (x) > 0 ~
\text{ for each } x. 
$$
then $\omega'_x  = -dx_2  \Lambda \cdots \Lambda dx_n$ is in
$\overset{n-1}{\Lambda} T_a^* (\partial V)$ and is positive for each
$x$ in $\partial V$. The condition of the positivity of the Jacobians
is trivially verified. 

\begin{remark*}
  If  $V$ is a $C^k$ manifold and $D$ an open set such that for any
  $a \in (\bar{D}- D)$, there exists a neighbourhood $U$ in $V$ and  a
  $C^k$ function $g$ in $U$ with $(dg)_a \neq 0$, $D \cap U = \big\{ x
  \in U ~  \Big| g (x) > 0 \big\}$, then $\bar{D}$ is a  manifold with
  boundary and $\partial D$ coincides with the topological boundary
  $(\bar{D}-D)$ of $D$. 
\end{remark*}

\setcounter{theorem}{0}
\begin{theorem}[Formula for change of  variable]\label{chap2:sec5:thm1}% theorem 1
  Let $\Omega, \Omega'$ be  open
  sets in $\mathbb{R}^n$ and $h$: $\Omega' \to \Omega$ a $ C^1$
  homeomorphism (so that $h$ is bijective and the jacobian det $d h
  (y) \neq 0$ for  $y \in \Omega'$). Then, if $f$ is a continuous
  function with compact support in $\Omega$, we have 
  \begin{equation}
    \int\limits_{\Omega} f (x) dx_1 \ldots dx_n = ~
    \int\limits_{\Omega'} (f \circ h) (y). | det  ~ dh (y) | dy_1
    \ldots dy_n. \tag{5.1}\label{chap2:sec5:eq5.1} 
  \end{equation}
\end{theorem}

\begin{proof}
  We\pageoriginale first prove the formula when $h$ is a linear transformation. Let
  $A$ denote the matrix of $h$ (with respect to the canonical basis of
  $\mathbb{R}^n$). By the elementary divisors theorem, $A$ can be
  written as a product of finitely many matrices $A_i$ each of which
  is either a diagonal matrix or an elementary matrix viz. the matrix
  corresponding to one of the linear transformations 
  \begin{enumerate}[(a)]
  \item $h (x_1, \ldots , x_n) = (x_1, \ldots , x_{i-1}, x_k, x_{i+1},
    \ldots , x_{k-1}, x_i, x_{k+1}, \ldots , x_n)$ 
  \item $h(x_1, \ldots , x_n) = (x_1 + x_2, x_2, \ldots , x_n)$.
  \end{enumerate}
\end{proof}

It is clearly sufficient to prove (1) for matrices of these special
kinds. For diagonal matrices of type (a), the formula (1) is a
trivial consequence of Fubini's theorem. For transformations $h$ of
type (b) we have, by Fubini's theorem 
\begin{align*}
  \int\limits_{\Omega'}&  (f \circ h) (y) | \det dh (y) | dy_1 \cdots dy_n \\
  & = \int \limits_{\mathbb{R}^n} f(x_1 + x_2, x_2 , \ldots , x_n)
  dx_1 \cdots dx_n\\ 
  & = \int \limits_{\mathbb{R}^{n-1}} dx_2 \cdots dx_n
  \int \limits_{\mathbb{R}^n} f(x_1 + x_2, \ldots x_n) dx_1 \\ 
  & =  \int \limits_{\mathbb{R}^{n - 1}} dx_2 \cdots dx_n \int
  \limits_{\mathbb{R}} f(x_1, \ldots , x_n) dx_1\\ 
  & \tag{\text{since
    Lebesque measure on $\mathbb{R}$ is translation invariant}}\\ 
  & = \int \limits_{\mathbb{R}^n} f(x) dx_1 \cdots dx_n = \int
  \limits_{\Omega} f(x) dx_1 \cdots dx_n. 
\end{align*}

To prove (\ref{chap2:sec5:eq5.1}) in general, we remark that it
suffices to prove the inequality 
\begin{equation}
  \int \limits_\Omega f(x) dx_1 \cdots dx_n \leq \int
  \limits_{\Omega'} (f \circ h) (y) |\det dh (y)| dy_1 \cdots dy_n
  \tag{5.2}\label{chap2:sec5:eq5.2} 
\end{equation}
for\pageoriginale all non-negative $f$ with compact support. (Apply the inequality
to $h ^{-1}$ to obtain equality.) Moreover, by the definition of the
Riemann integral, it is sufficient to prove the following statement
(\ref{chap2:sec5:eq5.3}) If $Q$ is a closed cube with equal sides
contained in $\Omega'$, we have  
\begin{equation*}
  \mu (h(Q)) \leq \int \limits_{Q} |\det dh (y)| dy_1 \cdots
  dy_n\label{chap2:sec5:eq5.3} 
\end{equation*}
here $\mu$ denotes Lebesgue measure in $\mathbb{R}^n$.

\medskip
\noindent
\textbf{Proof of (5.3).}
Let $K$ denote any closed cube, with equal sides say, $\delta$,
contained in $\Omega '$. For an $n \times n$ matrix $A = (a_{ij})$,
set $|| A || = \max \limits_{i} \sum \limits_{j} |a_{ij}|$. Note that
if $I$ is the unit matrix, we have $|| I || = 1$. 

Let $h = (h_1, \ldots , h_n)$. Taylor's formula shows that if $x$, $y
\in K$ 
$$
\displaylines{\hfill 
  h_i (x) - h_i (y) = \sum_j \frac{\partial h_i}{\partial x_j}
  (\theta_i) (x_j - y_j), \theta_i \in K, \hfill \cr
  \text{so that}\hfill 
  | h_i (x) - h_i (y) | \leq \sup_{a \in K} || dh (a) ||. \delta.\hfill }
$$

Consequently, $h(K)$ is contained in a cube of side $\delta. \sup
\limits_{a \in K} || dh (a) ||$ so that 
\begin{equation}
  \mu (h(K)) \leq \{ \sup_{a \in K} || dh (a) ||\}^n \mu (K)
  \tag{5.4}\label{chap2:sec5:eq5.4} 
\end{equation}

If we apply (\ref{chap2:sec5:eq5.4}) to the transformation $g = A.h$,
where $A$ is the 
inverse of the linear transformation $(d h) (a)$ for a fixed $a \in
K$, and observe\pageoriginale that, by (\ref{chap2:sec5:eq5.1})
applied to the linear transformation $A$ we have  
$$
\mu (g(K)) = | \det dh(a) |^{-1} \mu (h(K)),
$$
we obtain
\begin{equation}
  \mu (h(k)) \leq |\det dh(a)| \{ \sup_{b \in K} || (dh(a))^{-1} dh(b)
  || \}^n \mu (K). \tag{5.5}\label{chap2:sec5:eq5.5} 
\end{equation}

We observe that as the sides of $K$ tend to zero, $(dh(a))^{-1} dh(b)
\to I$, uniformly for $b$ in any compact subset of $\Omega '$
(\ref{chap2:sec5:eq5.3}) is
now easy to prove. Divide $Q$ into $\varepsilon^{-n}$ cubes $K_i$ of
side ($\varepsilon$. side of $Q$), and let $a_i \in K_i$. Then 
$$
\sup_{b \in K_i} || (dh(a_i))^{-1} dh(b) || \leq 1 + \alpha
(\varepsilon), \text{ where } \alpha (\varepsilon) \to 0 \text{ as  }
\varepsilon \to 0. 
$$
The inequality (5) now gives	
$$
\mu (h(Q)) \leq \sum_i \mu (h(K_i)) \leq (1 + \alpha (\varepsilon))^n
\sum_i |\det| dh(a_i) | \mu (K_i). 
$$

As $\varepsilon \to 0$, by definition, the sum on the right converges
to $\int \limits_{Q} |\det dh(y)| dy_{1}\cdots dy_n$, so that, since
$\alpha (\varepsilon) \to 0$, we obtain (5). 

\heading{Integration.}

Let $V$ be an oriented $n$ dimensional $C^k$ manifold (with or without
boundary) countable at infinity $(k \geq 1)$. Let $\omega$ be a
continuous $n$ form on $V$ with compact support. We shall define the
integral  
$$
\int \limits_V \omega
$$
as follows.

Let\pageoriginale $\{ U_i, \varphi_i\}$ be a locally finite family of coordinate
systems such that $\varphi_i$ induces the given orientation on $U_i$
from that  of $\mathbb{R}^n$; the Jacobians det $ | d (\varphi_i \circ
\varphi^{-1}_j |$ are then all positive. Let $\{\alpha_i \}$ be a
partition of unity relative to the covering $\{ U_i \}$; let
$\Omega_i = \varphi_i (U_i)$, and let $x^{(i)}_1 , \ldots , x^{(i)}_n$
denote the running coordinates 	in $\Omega_i$. Let $(\varphi^{-1}_i)^*
(\alpha_i \omega) = g_i (x^{(i)}) d x^{(i)}_1 \wedge \cdots \wedge
dx^{(i)}_n$. We set  
$$
\int \limits_{V} \omega = \sum_i \int \limits_{\Omega i} g_i (x^{(i)})
dx^{(i)}_1, \ldots , dx^{(1)}_n 
$$
(the latter integral being an ordinary Riemann or Lebesgue integral);
the sum is finite since $\omega$ has compact support. The integral so
defined is linear: we have $\int \limits_{V} (\omega_1 + \omega_2) =
\int \limits_{V} (\omega_1 +\omega_2) =  \int \limits_{V} \omega_1+
\omega_2$ and $\int\limits_V \lambda \omega = \lambda \int \limits_V
\omega$ for $\lambda \in \mathbb{R}$. It is, however, necessary in
applications to know that the definition above is independent of the
covering $U_i$, and the functions $\alpha_i$ used in the
definition. We shall denote the integral defined above temporarily by
$I (\omega)$. Since $I$ is linear, its invariance of the $\{ U_i ,
\alpha_i \}$ results at once from the following. 

\setcounter{lemma}{0}
\begin{lemma}\label{chap2:sec5:lem1}
  Let $(U, \varphi)$ be any coordinate system such that $\det |d (
  \varphi \circ \varphi^{-1}_i|$ is positive on $\varphi_i (U_i \cap U)$
  for each i. Let $\omega$ be an $n$ form with support in $U$, and
  in terms of the local coordinates in $\varphi (U) = \Omega $; let 
  $$
  \omega = f(x) dx_i \wedge \cdots \wedge dx_n.
  $$
\end{lemma}

Then we have
$$
\int \limits_{\Omega} f(x) dx_1 \cdots dx_n = I(\omega).
$$

\begin{proof}
  It\pageoriginale is enough to prove that if
  $$
  \alpha_i \omega = f_i (x) dx_1 \wedge \cdots \wedge dx_n = g_i
  (x^{(i)})  dx^{(i)}_1 \wedge \cdots \wedge dx^{(i)}_n 
  $$
  then
  $$
  \int \limits_{\Omega} f_i (x) dx_1, \ldots , dx_n = \int
  \limits_{\Omega_i} g_i (x^{(i)}) dx^{(i)}_1, \ldots , dx^{(i)}_n. 
  $$
\end{proof}

The integrals are respectively $\int \limits_{\varphi (U_i \cap U)}$
and $\int \limits_{\varphi_i (U_i \cap U)}$; let $h_i$: $\varphi (U
\cap U_i) \to \varphi (U \cap U_i)$ be the mapping $\varphi \circ
\varphi^{-1}_i$; since $\alpha_i \omega = f_i (x) dx_i \wedge \cdots
\wedge dx_n = g_i(x^{(i)}_i) dx^{(i)}_1 \wedge \cdots \wedge
dx^{(i)}_n$, we have 
$$
f_i \circ h(x^{(i)}), \det (dh_i) (x^{(i)}) = g_i (x^{(i)});
$$
however, by hypothesis, det $(dh_i) (x^{(i)}) > 0$, and the assertion
follows from the formula for change of variable. 

\begin{theorem}[Stokes' theorem]\label{chap2:sec5:thm2}% them 2
  If $V$ is an oriented manifold of
  dimension $n, V$ an (n - 1) form of class $C^1$, having compact
  support, we gave 
  $$
  \int \limits_{\partial V} \omega = \int \limits_{V} d \omega.
  $$
  In particular, the above formula holds for all $C^1$ forms $\omega$
  if $V$ is compact  
\end{theorem}

\begin{proof}
  If $(U_i, \varphi_i)$ is a locally finite system of coordinate
  neighbourhoods, $(\eta_i)$ a partition of unity subordinate to $\{
  U_i\}$, it is enough to prove that 
  $$
  \int \limits_{\partial V} \eta_i \omega = \int \limits_V d (\eta_i \omega)
  $$
\end{proof}

\begin{romancase}\label{chap2:sec5:caseI} %%% case1
  If\pageoriginale $\varphi_i (U_i)$ is open in $\mathbb{R}^n$,
  $$
  \int \limits_{\partial V} \eta_i \omega = 0
  $$
\end{romancase}

Further, if $\eta_i \omega = \sum \limits^n_{j = 1} f_j dx_1 \wedge
\cdots \wedge d\hat{x}_j \wedge \cdots \wedge dx_n$, we gave
$d(\eta_i \omega) = \sum \dfrac{\partial f_j}{dx_j} (-1)^{j-1} dx_1
\wedge \cdots \wedge dx_n$  
\begin{align*}
  \text{and} ~  \int \limits_V d(\eta_i \omega) & = \int
  \limits_{\varphi_i (U_i)} \sum (-1)^{j-1} \frac{\partial
    f_j}{\partial x_j} dx_1 \wedge \cdots \wedge dx_n \\ 
  & = \int \limits_{\mathbb{R}^n} \left(\sum (-1)^{j-1} \frac{\partial
    f_j}{\partial x_j}\right) dx_1 dx_2 \cdots dx_n 
\end{align*}
since $f_j$ has compact support for each $j, \int \limits
_{\mathbb{R}}\dfrac{\partial f_j}{\partial x_j} dx_j = 0$. Hence it
follows from Fubini's theorem that  
$$
\int \limits_{V} d (\eta_i \omega) = 0 = \int \limits_{\partial V}
\eta_i \omega. 
$$

\begin{romancase}\label{chap2:sec5:caseII} %%% case2
  If $\varphi_i (U_i)$ is not an open set in $\mathbb{R}^n$
  $$
  \int \limits_{V} d(\eta_i \omega) = \int \limits_{\mathbb{R}^n_+ }
  \left(\sum (-1)^{j-1} \frac{\partial f_j}{\partial x_j}\right) dx_1 \cdots
  dx_n. 
  $$
\end{romancase}

Now
$$
\int \limits_{\mathbb{R}^n_+} \frac{\partial f_j}{\partial x_j} dx_1
\cdots dx_n = 0 \text{ if } j=1 \text{ as in case \ref{chap2:sec5:caseI}}: 
$$
further, if $j \neq 1$, $f_j dx_1 \wedge \cdots \wedge d\hat{x}_j
\wedge \cdots \wedge dx_n| \partial V = 0$. Also  
$$
  \int \limits_{\mathbb{R}^n_+} \dfrac{\partial f_1}{\partial x_1}
  dx_1 \cdots dx_n 
  = \int \limits_{\mathbb{R}} dx_n \int \limits_{\mathbb{R}} dx_{n-1}
  \cdots \int \limits_{x_1 \geq 0} \frac{\partial f_1}{\partial x_1}
  dx_1. 
$$

Hence\pageoriginale 
\begin{align*}
  \int \limits_{\mathbb{R}^n_+} \dfrac{\partial f_1}{\partial
  x_1} dx_1 \cdots dx_n
  & = - \int \limits_{\mathbb{R}^{n-1}} f_1 (0, x_2, \ldots , x_n)
  dx_2 \cdots dx_n\\ 
  \text{ and } \hspace{2cm} \int \limits_{V} d (\eta_i \omega) & = - \int
  \limits_{\mathbb{R}^{n-1}} f_1 (0, x_2, \ldots , x_n) dx_2 \cdots
  dx_n\\ 
  & = \int \limits_{V} \eta_i \omega
\end{align*}

\section{One parameter groups and the theorem of
  Frobenius}\label{chap2:sec6} 

In what follows $V$ denotes a $C^k$ manifold countable at $\infty$
with $k \geq 3$. 
\begin{defi*}
  A $C^r$ map $\varphi$: $\mathbb{R} \times V \to V$, $0 < r \leq k$,
  is called a one parameter group of $C^r$ transformations of it
  satisfies the following conditions: 
  \begin{enumerate}[(1)]
  \item for every $t \in \mathbb{R}$, $\varphi (t, x) = \varphi_t (x)$
    is a $C^r$ diffeomorphism of $V$ onto itself; 
  \item $\varphi_{t + s} (x) = \varphi_t \circ \varphi_s (x)$ for $s$, $t
    \in \mathbb{R}$ and $x \in V$. 
  \end{enumerate}
\end{defi*}

\begin{defi*}
  If $U$ is an open subset of $V$, a local one parameter group of
  $C^r$ transformations of $U$ into $V$ is a $C^r$ map $\varphi$: $I_
  \varepsilon \times U \to V$, $I_{\varepsilon} = \{t \in \mathbb{R} \big | |t| <
  \varepsilon \}$, $\varepsilon > 0$, which satisfies the following
  conditions: 
  \begin{enumerate}
  \item for\pageoriginale any $t \in I_{\varepsilon}$, $\varphi (t, x) = \varphi_t
    (x)$ is a $C^r$ diffeomorphism of $U$ into $V$ (i.e. onto an open
    subset of $V$); 
  \item if $s$, $t$, $s+t \in I_ \varepsilon$ and $x$, $\varphi_t (x)
    \in U$, then $\varphi_{s + t}(x) = \varphi_S \circ \varphi_t (x)$ 
  \end{enumerate}
\end{defi*}

Given a one parameter group $\varphi$: $\mathbb{R} \times V \to V$ we
can associate to it a vector field $X_\varphi$ defined by $(X_
\varphi)_a (f) = \dfrac{\partial (f \circ \varphi_t)}{\partial t} \big
| (0, a)$ for $f \in C^k_a$;  i.e. $(X_{\varphi})_a$ is precisely the
tangent to the curve $t \to \varphi_t (a)$ at a, $ X_ \varphi $ is
called the vector field induced by $\varphi$. A local one parameter
group of transformations of $U$ into $V$ induces a vector field on $U$
in the same way. 

\setcounter{proposition}{0}
\begin{proposition}\label{chap2:sec6:prop1}% props 1
  Given a $C^{k-1}$ vector field $X$, there exists, for every $a \in
  V$, a neighbourhood $U$ of a and a local one parameter group of
  $C^{k-1}$ transformations of $U$, $\varphi : I_{\varepsilon} \times
  U$ which induces $X$ on $U$, i.e. we have  
  $$
  X_b (f) = \frac{\partial (f \circ \varphi)}{\partial t} (0, b) \text{
    for } b \in U \text{ and } f \in C^k_b. 
  $$
\end{proposition}

\begin{proof}
  Let the vector field $X$ be given by
  $$
  X = \sum a_i (x) \frac{\partial} {\partial x_i}
  $$
  in terms of local coordinates in an open set $U' \ni a$. We have
  then to solve the differential equation 
  $$
  \displaylines{\hfill 
    \sum \frac{\partial \varphi_i}{\partial t}
    \frac{\partial}{\partial x_i} = \sum a_i \frac{\partial}{\partial
      x_i} \hfill \cr 
    \text{i.e.} \hfill  \frac{\partial \varphi}{\partial t} = a
    (\varphi (t_1 , x))\hfill } 
  $$
  with the initial condition $\varphi (0, x) = x$; [here $\varphi$
    stands for an $n$-tuple of functions]. 
\end{proof}

Since\pageoriginale $X$ is a $C^{k-1}$ vector field, $(k \geq 3)$, $a_i \in C^{k-1}$
and by Chapter I, \S\ \ref{chap1:sec6}, there exists $\varepsilon > 0$, a
neighbourhood $U$ of $a$ and a unique $C^{k-1}$ map $\varphi$:
$I_{\varepsilon} \times U \to V$ satisfying the differential equation 
$$
\frac{\partial \varphi}{ \partial t} = a (\varphi (t,x)), \varphi (0,
x) = x. 
$$

For $s$, $t$, $s+ t \epsilon I_{\varepsilon}$ and $x$, $\varphi_t
(x) \epsilon U$, it can be easily verified that $\varphi_s \circ
\varphi_t (x)$ and $\varphi_{s+t} (x)$ are both solutions of the
differential equation 
$$
\frac{\partial \psi}{\partial s} = a (\psi (s,x)), \psi_i (0, x) =
\varphi_i (t,x). 
$$
Hence by the uniqueness of the solution of equations of this form, we have
$$
\varphi_s \circ \varphi_t (x) = \varphi_{s + t}(x) \text{ for } x,
\varphi_t (x) \in U. 
$$

It now remains to show that for $t \in I_{\varepsilon}$, $ \varphi_t
(x)$ is a diffeomorphism of $U$ into $V$. Since 
$$
(d_2 \varphi) (0, x) = \text{ identity, and } \varphi \in C^{k-1},
\text{ it } 
$$
follows that for sufficiently small $\varepsilon$, $t \in
I_{\varepsilon}$ implies $(d_2 \varphi) (t, x)$ is nonsingular and
hence, by the rank theorem, $\varphi_t (x)$ is a diffeomorphism of $U$
into $V$ if $U$ is chosen small enough (see also proof of the
following corollary). 

\begin{coro*}
  Given a $C^{k-1}$ vector field $X$ on $V$ and a relatively compact
  open set $U$, there exists a local one parameter group $\varphi_t$
  of $C^{k-1}$ transformations of $U$ into $V$ which induces $X$ on
  $U$. 
\end{coro*}

\begin{proof}
  Let\pageoriginale $U'$ be an open set with $U \subset \subset U' \subset \subset
  V$. (We write 
  $A \subset \subset B$ to mean that $A$ is relatively compact in
  $B$.)  For any a 
  $a \in \overline{U'}$ there is a neighbourhood $U_a$ in $V$ and a local
  one parameter group $\varphi^{(a)}_t$: $U_a \to V$, $|t|< \varepsilon
  (a)$, which induces $X$ on $U_a$. Suppose $a_1, \ldots , a_k$ so
  chosen that $\bigcup U_{a_i} \supset U'$. Let $\varepsilon ' = \min
  \varepsilon (a_i)$. If $U_{a_i} \bigcap U_{a_j} \neq \phi$, then
  $\varphi^{(a_i)}_t$, $\varphi^{(a_j)}_t $ induce $X$ on $U_{a_i}
  \cap U_{a_j}$, and hence coincide there. Define $\varphi_t(x) =
  \varphi^{(a_i)}_t (x)$ for $x \in U_{a_i}$. Let $\varepsilon <
  \varepsilon '$ be so small that $\varphi_t (U) \subset U'$ for $|t|
  < \varepsilon$. Since each $\varphi_t$ is a 1-parameter group, we
  have only to show that each $\varphi_t$ is injective on $U$. But
  this is obvious since $\varphi_{-t} (\varphi_t(x)) = x$ for $x \in
  U$. (Note that $\varphi_t (x) \epsilon U'$ and $\varphi_{-t}$ is
  defined on $U'$). 
\end{proof}

\begin{remark*}
  If $V$ is compact, $X$ gives rise to a global 1- parameter group $\psi_s$. In
  fact, as is easily deduced from the above corollary, there is
  $\varepsilon > 0$ such that $\varphi_t : V \to V$ is a
  diffeomorphism (onto) for $|t| < \varepsilon$. Given $s \in
  \mathbb{R}$, we set $\psi_s = (\varphi_{s/k})^k$ where $k$ is an
  integer so chosen that $|s/k| < \varepsilon$ and $(\varphi_{s/k})^k$
  denotes the composite of $(\varphi_{s/k})$ with itself $k$ times. $(
  \psi_s$ is independent of the $k$ chosen). 
\end{remark*}

We have remarked earlier that a differentiable map does not transfer
vector fields into vector fields. However, let $\sigma$ be a $C^r$
diffeomorphism of an open set $U \subset V$, into $V$ and $X$, a
$C^{r-1}$ vector field on $U$, let $U' = \sigma(U)$. The assignment to
$a \in U '$ of the vector $\overset{\sigma}{\ast} (X_{\sigma - 1(a)})$ at a, is
clearly a $C^{r-1}$ vector field on $U'$, denoted
by$\overset{\sigma}{\ast} (X)$ or
$\overset{\sigma}{\ast} X$. If $f$ is a $C^k$ function on $U'$, we have, 
$$
\sigma_* (X)(f) = X(f \circ \sigma) \circ \sigma^{-1}.
$$
If\pageoriginale $X$, $Y$ are two vector fields on $U$, we have
\begin{align*}
  [\overset{\sigma}{\ast} X, \sigma_* Y] (f) & =
  \overset{\sigma}{\ast}  (X) [Y (f \circ \sigma ) \circ
    \sigma^{-1}] - \overset{\sigma}{\ast} (Y) [X(f \circ \sigma )
    \circ \sigma^{-1}]\\  
  & = [X (Y(f \circ \sigma )) - Y (X(f \circ \sigma ))] \circ \sigma^{-1} \\
  & =  \overset{\sigma}{\ast}([X, Y]) (f), \\
  \text{i.e.}~ [\overset{\sigma}{\ast} X, \overset{\sigma}{\ast}  Y] &
  = \overset{\sigma}{\ast} [X, Y]. 
\end{align*}

\begin{proposition}\label{chap2:sec6:prop2} % props 2
  If $\sigma$ is a diffeomorphism $U \to U'$ and if a local one
  parameter group of transformations $\varphi$: $(U \cup U' ) \to V$
  induces the vector field $X$, then $\overset{\sigma}{\ast} X$ is
  induced by the 
  local one parameter group $\sigma \circ \varphi \circ \sigma^{-1}: U' \to U$. 
\end{proposition}

\begin{proof}
  \begin{align*}
    \overset{\sigma}{\ast} (X)(f) & = X (f \circ \sigma) \circ \sigma^{-1} \\
    & = \frac{\partial}{\partial t} (f \circ \sigma \circ \varphi_t)|_{t = 0}
    \circ \sigma^{-1} \\ 
    & = \frac{\partial}{\partial t} (f \circ \sigma \circ \varphi_t 0
    \sigma^{-1} |_{t = 0}  
  \end{align*}
\end{proof}

\setcounter{corollary}{0}
\begin{corollary}\label{chap2:sec6:coro1} %% coro1
  $\sigma$ commutes with $\varphi_t$ for every $t$ if and only if
  $\sigma_* (X) = X$. 
\end{corollary}

\begin{defi*}
  A local one parameter group $\varphi$ is said to leave a vector
  field $X$ invariant if $(\varphi_{t_*}) (X) = X$ for every $t$. 
\end{defi*}

\begin{remark*}
  If\pageoriginale $\varphi$ induces the vector field $X_\varphi$, $X_\varphi$ is
  invariant under $\varphi$. 
\end{remark*}

\begin{defi*}
  If $\varphi$ is a local one parameter group $U \to V$, of $C^2$
  transformations, and $Y$, a vector field on $V$, and if
  $(\varphi_{t})_* Y = Y_t$, we define the vector field $\dfrac{d
      Y_t}{dt}$ by 
  $$
  \left(\frac{dY_t}{dt}\right) (f) = \frac{d}{dt} [Y_t (f)].
  $$
\end{defi*}

\begin{proposition}\label{chap2:sec6:prop3} %props 3
  If $Y$ is a $C^{k-1}$ vector field on $V$, $k \geq 3$ and if a one
  parameter group $\varphi$ induces the $k-1$ vector field $X$ on $U$
  we gave 
  $$
  \frac{dY_t}{dt} \bigg|_{t_0} = [ Y_{t_0}, X]~ on ~ U.
  $$
\end{proposition}

\begin{proof}
  We shall first prove the result for $t_0 = 0$. We have
\end{proof}

\begin{align*}
  \frac{dY_t}{dt} \bigg|_0 (f) & = \lim_{t \to 0} \frac{1}{t} [Y_t - Y] (f) \\
  & = \lim_{t \to 0} \frac{1}{t} [Y [f \circ \varphi_t ] \circ
    \varphi_{-t} - Y(f)] \\ 
  & = \lim_{t \to 0} \frac{1}{t} [Y(f \circ \varphi_t ) - Y(f) \circ \varphi_t
  ] \circ \varphi_{-t} \\ 
  & = \lim_{t \to 0} \frac{1}{t} [Y(f \circ \varphi_t ) - Y(f) - \circ
    Y(f) \circ \varphi_t + Y (f) ] \circ \varphi_{-t} \\ 
  & = \lim_{t \to 0} \frac{Y (f \circ \varphi_t - f)}{t} - \lim_{t \to 0}
  \frac{Y(f) \circ \varphi_t -Y(f)}{t}, 
\end{align*}
since $\lim \limits_{t \to 0} \varphi_{-t} =$ identity. Now
$$
\lim_{t \to 0} \frac{Y(f) \circ \varphi_t - Y (f)}{t} = X(Y(f))
$$
by definition of $X$. Consider $h(t, x) = f \circ \varphi_t (x)$. Clearly
$h \in C^2$ since\pageoriginale $h \in C^{k-1}$ and 
$$
\frac{h(t, x) - h (0 , x)}{t} = \frac{f \circ \varphi_t - f}{t} \in C^1.
$$
\begin{alignat*}{4}
  \text{ Hence } &\hspace{1cm}&\lim \limits_{t \to 0} \frac{Y [ f \circ \varphi_t -
      f]}{t}&  = Y \left[\lim \limits_{t \to 0} \frac{f \circ \varphi_t -
      f}{t}\right]\\ 
  &&& = Y(X(f)).\\
  \text{ Hence } &&\frac{d Y_t} {dt} \bigg|_{t = 0} (f) & = Y [X (f)]
  - X [Y(f)] \\ 
  &&& = [Y_0 , X] (f). \\
  \text{i.e.}\quad && \frac{dY_t}{dt}\bigg|_{t = 0} & = [Y_0, X].
\end{alignat*}

For any $t_0$ in the interval of definition.
\begin{alignat*}{4}
&&(\varphi_{t_0})_* \left(\frac{dY_t}{dt}\right)_{t =0} & =
\left(\frac{dY_t}{dt}\right)_{t =t_0} \\
  \text{ and }&\hspace{2cm}& (\varphi_{t_0})_* [Y_0, X] & =
       [(\varphi_{t_0})_* Y_0, (\varphi_{t_0})_* X] \hspace{2cm}\\ 
 && & = [Y_{t_0}, X]. 
\end{alignat*}

Hence $\frac{d Y_t}{dt} |_{t = t_0} = [ Y_{t_0}, X]$.

\begin{coro*}
  If $X$, $Y$ are vector fields on $V$ which give rise to local one
  parameter groups $\varphi$ and $\psi$: $U \to V$ respectively, then
  for all $t$, $s$, $\varphi_t	$ and $\psi_s$ commute
  (i.e. $\varphi_t \circ \psi_s = \psi_s \circ \varphi_t$ on the common of
  definition) if and only if $[X, Y] = 0$. 
\end{coro*}

\begin{proof}
  If\pageoriginale $\varphi_t$ and $\psi_s$ commute for sufficiently small $t$ and
  $s, \varphi_t$ leaves $Y$ invariant. 
  
  Hence \hspace{2cm} $\frac{d Y_t}{dt} = [Y_t, X] = [Y, X] = 0.$\hfil 
\end{proof}

Conversely if $[Y, X] = 0$
\begin{align*}
  \frac{d Y_t}{dt} \bigg|_{t = t_0} & = [Y_{t_0}, X] = [Y_{t_0}, X_{t_0}] \\
  & =(\varphi_{t_0})_* [Y, X] = 0.
\end{align*}

Hence $\varphi_t$ leaves $Y$ invariant, which, with the corollary to
Prop.~2, completes the proof. 

In what follows we consider a $C^k$ manifold $V$. The vector fields
will be $C^{k-1}$ and differentiable functions, mappings will be
$C^k$. 

\begin{defi*}
  \begin{enumerate}
  \item  A distribution (or differential system) $\mathscr{D}$ of
    rank $p$, on (a $C^k$ manifold) $V$ is an assignment to each point
    $a \epsilon V$ of a subspace $\mathscr{D}(a)$ of $T_a(V)$, of
    dimension $p$. 
  \item A distribution $\mathscr{D}$ is called differentiable if for
    every $a \in V$ there exists a neighbourhood $U$ of a and
    differentiable vector fields $X_1$, $X_2, \ldots , X_p$ such that
    $X_{1_b}$, $X_{2_b}, \ldots X_{p_b}$ form a basis of
    $\mathscr{D}(b)$ for every $b \in U$. 
  \item A submanifold $i$: $W \to V$ of $V$ (more generally, a $C^k$
    mapping $i$: $W \to V)$ is called an integral of $\mathscr{D}$ if
    for $a \in W, i_* (T_a(W)) \subset \mathscr{D}(i(a))$. 
  \item A distribution $\mathscr{D}$ is said to be completely
    integrable if for every $a \in V$, there exists a neighbourhood
    $U_a$ and a system of local coordinates\pageoriginale $(x_1, \ldots, x_n)$, such
    that for sufficiently small $c_i$, $p+1 \leq i \leq n$, the
    submanifolds given by $U_c = \{x \epsilon U | x_i = c_i, i \geq
    p+1 \}$ are integrals of $\mathscr{D}$. 
  \end{enumerate}
\end{defi*}

\begin{remark*}
  Any submanifold of an integral is itself an integral.
\end{remark*}

\setcounter{lemma}{0}
\begin{lemma}\label{chap2:sec6:lem1}
  If $\mathscr{D}$ is a completely integrable differentiable
  distribution and if $W \subset U$ is a connected integral of
  $\mathscr{D}$, then $W \subset U_c$ for some $c = (c_i)_{p+1 \leq i
  \leq n}$, [where $U$ carries a coordinate system as in $(4)$
    above]. 
\end{lemma}

\begin{proof}
  We have $i_* (T_a(W)) \subset \mathscr{D} (i(a))$. Now for any $c,
  T_{i (a)} (U_c)$ has dimension $p$ and hence $T_{i(a)}(U_c) =
  \mathscr{D} (i(a))$ 
  
  Hence \hspace{2cm} $i_* (T_a(W)) \subset T_{i(a)} (U_c)$.  
\end{proof}

Now $T_{i(a)}(U_c)$ is the subspace of $T_{i(a)}(V)$, orthogonal to
the 1- forms $\{dx_i \}_{i > p}$. Hence $\{d x_{i}\}_{i > p}$ are
orthogonal to $i_* (T_a(W))$, i.e. $dx_i | W = 0, i > p$ and hence
$x_i = c_i$ for some constant $c_i, i > p$, since $W$ is connected.

\begin{defi*}
  A differentiable distribution $\mathscr{D}$ is called involutive (or
  complete) if for any $a \in V$, there is a neighbourhood $U$ and
  vector fields $X_1, \ldots , X_p$ generating $\mathscr{D}$ in $U$
  such that, we gave, for $b \in U$ 
  $$
  [X_i, X_j]_b \in \mathscr{D}(b) \text{ for }i,j \leq p.
  $$
\end{defi*}

Note that there then exist differentiable functions $a^k_{ij}$ in $U$
such \break $[x_i, x_j] = \sum \limits^p_{k=1} a^k_{ij} X_k$. 

\begin{remark*}
  The above definition is independent of the basics $X_1, \ldots , X_p$.
\end{remark*}

\begin{lemma}\label{chap2:sec6:lem2}%lem 2
  If\pageoriginale a differential system $\mathscr{D}$ is involutive, for any $a \in
  V$ there exists a neighbourhood $U$ of $a$ and $a$ basis $X_1,
  \ldots, X_p$ of $\mathscr{D}$ in $U$ such that $[X_i, X_j]=0$ in
  $U$.  
\end{lemma}

\begin{proof}
  Let $(Y_i)_{1 \leq i \leq p}$ be a basis of $\mathscr{D}\big|U$.
\end{proof}

In terms of local coordinates, let
$$
Y_i = \sum^n_{r=1} a_{ir} \frac{\partial}{\partial x_r}.
$$

We may assume without loss of generality that the matrix $(a_{ir}(x))=
A(x)$, $1 \leq i \leq p$, $1 \leq r \leq p$ is of rank $p$ at the
point $x=a$. If $U$ is small enough, $A(x)$ has rank $p$ for $x \in
U$. If $B(x) = (b_{ir})(x) = [A(x)]^{-1}$, then the $b_{ir}$ are
differentiable. Let 
\begin{alignat*}{4}
  && X_i &= \sum^p_{k=1} b_{ik} Y_k. \\
  \text{ Then } && X_i & = \frac{\partial}{\partial x_i} + \sum_{r >
    p} C_{ir}\frac{\partial}{\partial x_r} ~\text{ and }~ \quad (X_i)_{1
    \leq i \leq p} 
\end{alignat*}
form a basis of $\mathscr{D}\big|U$. Since $\mathscr{D}$ is
involutive, we have 
$$
[X_i, X_j] = \sum^{p}_{r=1} \lambda_r X_r.
$$

But $\left(\dfrac{\partial}{\partial x_i}\right)_{1 \leq i \leq p}$ commute with
each other and, if $[X_i, X_j] = \sum \limits^n_{r=1} \mu_r$
 $\left(\dfrac{\partial}{\partial x_r}\right)$, then $\mu_r = 0$ for $r \leq
p$. Clearly we therefore have  
$$
\lambda_r = \mu_r = 0 \text{ for } r \leq p.
$$

\begin{proposition}\label{chap2:sec6:prop4}%prop 4
  Let\pageoriginale $X_1, \ldots, X_p$ be vector fields on $V$ which are linearly
  independent at every point of $V$ and such that $[X_i, X_j]=0$, then
  for any $a \in V$ there exists a neighbourhood $U$ and coordinates
  $t_1$, $t_2, \ldots, t_p$, $x_{p+1}, \ldots, x_n$ in $U$ such that
  $X_i = \dfrac{\partial}{\partial t_i}$ for $i \leq p$. 
\end{proposition}

\begin{proof}
  We can assume that $X_1, \ldots, X_p$ are induced by local one
  parameter groups of transformations, $\varphi^{(1)}$, $\varphi^{(2)},
  \ldots, \varphi^{(p)}$ in a neighbourhood $U$ of $a$. We suppose
  that $\varphi^{(i)}_t$ are defined for $| t | < $. After
  a linear change of coordinates on $U$ we may suppose that the
  vectors 
  $$
  (X_1)_a , \ldots, (X_p)_a, \left(\frac{\partial}{\partial x_{p+1}}\right)_a,
  \ldots, \left(\frac{\partial}{\partial x_n}\right)_a 
  $$
  are linearly independent. We suppose further that the coordinates of
  a are zero. Let $U' \subset \mathbb{R}^{n-p}$ be the set of $x' =
  (x_{p+1}, \ldots, x_n)$ with $(0, x') \in U$, $Q \subset
  \mathbb{R}^p$, the set $|t_i|< \delta$ and let $h$: $Q \times U' \to
  U$ be the mapping 
  $$
  h(t_1, \ldots, t_p, x_{p+1}, \ldots , x_n) = \varphi^{(1)}_{t_1} \circ
  \ldots \varphi^{(p)}_{t_p}(0, x'), 
  $$
  $ \varepsilon$ being chosen so small that the composites are all defined.
\end{proof}

For any $C^k$ function $f$ on $U$, we have $\dfrac{\partial}{\partial
  t_1} [ f \circ h]_{t=0} = (X_1)_a (f)$, by definition of
$\varphi^{(1)}_t$, and since the $\varphi^{(i)}_{t_i}$ commute,
(because $[X_i, X_j] = 0$), we have 
$$
h_* \left[ \left(\frac{\partial}{\partial t_i}\right)_o\right] =
(X_i)_a, 1 \leq i \leq p. 
$$

It is obvious that
$$
h_* \left(\frac{\partial}{\partial x_i}\right)_0 = 
\left(\frac{\partial}{\partial x_i}\right)_0, i > p. 
$$

This,\pageoriginale however, implies that $h_*$ has the maximum rank $=n$ and is a
diffeomorphism in a neighbourhood of $0$. Hence $t_1, \ldots, t_p,
x_{p+1}, \ldots, x_n$ may be considered as local coordinates in $U$ if
$U$ is small enough. Further, exactly as above, we show that
$h_*(\frac{\partial}{\partial t_i}) = X_i$, $i \leq p$, which gives
the proposition. 

\setcounter{theorem}{0}
\begin{theorem}[Frobenius]\label{chap2:sec6:thm1}% thm 1
  A differential system on $V$ is involutive if and only
  it is completely integrable. 
\end{theorem}

\begin{proof}
  If $\mathscr{D}$ is a completely integrable system for $a \in V$
  there exists a neighbourhood $U$ of a such that for all sufficiently
  small $(C_i)_{p+1 \leq i \leq n}$, $U_c = \{ x \in U | x_i = s_i$,
  $i > p\}$ are integrals of $\mathscr{D}$. Hence
  $\left(\dfrac{\partial}{\partial x_i}\right)_{1 \leq i \leq p}$ form a basis of
  $\mathscr{D}| U$ and $\mathscr{D}$ is involutive. This together with
  Lemma \ref{chap2:sec6:lem2} and Proposition \ref{chap2:sec6:prop4}
  above proves the theorem.  
\end{proof}

\begin{remark*}
  We have proved the theorem of Frobenius for $C^2$ distributions,
  i.e. distributions having a basis of $C^2$ vector fields [We have
    used the condition essentially in the proof of
    Prop. \ref{chap2:sec6:prop3}.] However
  the theorem is valid also for $C^1$ vector fields. We have only to
  prove Prop. \ref{chap2:sec6:prop3} for $C^1$ vector fields. This can
  be by approximating 
  the fields by $C^2$ fields and using the results of Chap I,
  \S\ \ref{chap1:sec6}, to
  conclude that the local 1-parameter group associated to a vector
  field $X$ depends continuously on $X$. 
\end{remark*}

Let $\omega_{p+1}, \ldots, \omega_n$ be $1$-forms on $V$ which are
linearly independent at every point. We can define a distribution
$\mathscr{D}$ by setting 
$$
\mathscr{D}(a) = \{ X \in T_a (V) \bigg| (\omega_i)_a (X) = 0 ~\text{
  for }i = p + 1, \ldots, n \}. 
$$

If the $\omega_i$ are differentiable then so is $\mathscr{D}$. In
fact, considering suitable linear\pageoriginale combinations of the $\omega_i$ with
differentiable coefficients, we may suppose that, in a neighbourhood
of any given point a of $V$, we have 
$$
\omega_i = dx_i + \sum_{r \leq p}a_{ir} dx_r, i > p.
$$

Then $\mathscr{D}$ is the distribution spanned by the vector fields 
$$
X_r = \frac{\partial}{\partial x_r}- \sum_{j > p} a_{jr}
\frac{\partial}{\partial x_j}, 1 \leq r \leq p: 
$$
(it is obvious that the $X_r$ are orthogonal to the $\omega_i$ and
they are clearly linearly independent).   

For distributions given in this form, the theorem of Frobenius is as follows.

\begin{theorem}\label{chap2:sec6:thm2} % them 2
  Let $ \omega_{p+1}, \ldots , \omega_n$ be 1-forms which are linearly
  independent at every point. Then, in order that the distribution
  $\mathscr{D}$ defined by them be completely integrable, it is
  necessary and sufficient that every point $a \in V$ has a
  neighbourhood in which there exists 1-forms $\alpha^r_{j_n}$ such
  that, for $j > p$, 
  \begin{equation*}
    d \omega_j = \sum^n_{k=p+1} \omega_k \wedge \alpha^r_j;
    \tag{6.1}\label{chap2:sec6:eq6.1} 
  \end{equation*}
  i.e. $d \omega_j$ {\em belongs to the ideal generated by the} $\omega_k$.
\end{theorem}

[Note that the condition (\ref{chap2:sec6:eq6.1}) is invariant under
  `change of basis', 
  i.e. if $\eta_j$ are 1-forms which span the same subspace of
  $T^*_a (V)$ for any $a$, then the condition (\ref{chap2:sec6:eq6.1})
  is satisfied if 
  and only if the corresponding condition on the $\eta_j$ is.]
 
\begin{proof}
  If\pageoriginale $\mathscr{D}$ is completely integrable, and $a \in V$, choose
  coordinates at a such that the ``planes'' $x_{p+1} = c_{p+1}, \ldots,
  x_n = c_n$ are integrals of $\mathscr{D}$. Then $\mathscr{D}(b)$ is
  the space orthogonal to $(dx_{p+1})_b, \ldots, (dx_n)_b$. Hence
  $dx_{p+1}, \ldots, dx_n$ span the same subspace of $T^*_b(V)$ as
  $\omega_{p+1}, \ldots, \omega_n$ for $b \in U$. The equation
  (\ref{chap2:sec6:eq6.1})
  for the $dx_j$ is trivial. 
\end{proof}

Suppose conversely that there exist $\alpha^r_j$ satisfying
(\ref{chap2:sec6:eq6.1}). Let 
$X_1, \ldots$, $X_p$ be vector fields in a neighbourhood of a generating
$\mathscr{D}$. We have 
$$
(d \omega_k) (X_i, X_j) = X_i \omega_r (X_j) - X_j \omega_r
(X_i)-\omega_r ([X_i, X_j]). 
$$
Because of (\ref{chap2:sec6:eq6.1}) $(d \omega_k) (X_i, X_j) = 0$; and
by definition, 
$\omega_r (X_i) = \omega_r (X_j) = 0$. Hence $\omega_r ([X_i, X_j])=0$
so that $[X_i, X_j]_b$ is orthogonal to $(\omega_r)_b$ for all $r$, so
that $[X_i,X_j]_b \in \mathscr{D}(b)$. This proves that $\mathscr{D}$
is involutive, hence completely integrable 

One can prove that through any point passes a maximal integral. More
precisely, we have 

\begin{theorem*}[{\boldmath $3'$}]\label{chap2:sec6:thm3'} %thm 3'
  If $\mathscr{D}$ is completely integrable, then for any $a \in V$,
  there exists a connected integral $i$: $ W \to V$ of $\mathscr{D}$
  such that if $j: W' \to V$ is any connected integral of
  $\mathscr{D}$ with $j(a') =a$ then $W'$ is a submanifold of $W$. 
\end{theorem*}

\begin{proof}
  Let $W$ be the set of points $x$ of $V$ with the following property:
  there exists a chain of differentiable mappings $\gamma_i$: $I \to
  V$, $0 \leq i \leq N$ ($I$ being the closed unit interval) with
  $\gamma_\circ (0) = a$, $\gamma_N(1)=x$, $\gamma_{i+1}(0) = \gamma_i
  (1) (0 \leq i < N)$, such that each $\gamma_i$ is an integral of
  $\mathscr{D}$ (in the obvious sense). We topologize $W$ as
  follows. Let $x_0 \in W$, and $U$ an open set about $x_0$ carrying
  coordinates $x_1, \ldots, x_n, x_i(x_0) = 0$, such\pageoriginale that all the
  ``planes'' $U_c = \{ x_{p+1}, = c_{p+1},\ldots, x_n = c_n\}$ are integrals of
  $\mathscr{D}$. We may suppose that $U$ is a ``cube'', so that these
  planes are connected. Clearly every point of $U_0$ belongs to
  $W$. The sets $W_ \varepsilon (x_0) = U_0 \cap \{ x \in U \bigg | |
  x | < \varepsilon \}$ will, by definition, form a fundamental system
  of neighbourhoods of $x_0$ in $W$. [Note that by Lemma
    \ref{chap2:sec6:lem1} the 
    \textit{sets} $U_c$ are completely determined by $\mathscr{D}$]
  Also if $\gamma_0, \ldots, \gamma_N$ is a chain as in the definition
  of $W$, $\gamma_N (I) \subset U_0 \subset \bar{W}$. It is clear that
  this topology is Hausdorff. We make $W$ into a $C^{k-1}$ manifold by
  requiring that the obvious mappings $W_ \varepsilon (x_0) \to \{
  (x_1, \ldots, x_p) \in \mathbb{R}^p \bigg| | x_i | < \varepsilon \}$
  determine coordinates on $W$. It is then clear that $W$ is a
  connected integral of $\mathscr{D}$. 
\end{proof}

If $j:W' \to V$ is any connected integral with $j(a')=a$, let, for $w'
\in W', \gamma'_0, \gamma'_1, \ldots, \gamma'_N$ be diffeomorphism of
$I$ into $W'$ such that $\gamma'_0 (0) = a', \gamma'_{i+1}(0) =
\gamma'_i (1) = w'$. Let $w = j(w')$. Then $\gamma_i = j(\gamma'_i)$
is a chain as in the definition of $W$ joining a to $w$. Hence $w \in
W$. Thus there is a mapping $\eta : W' \to W$ with $i \circ \eta =
j$. Clearly $\eta$ makes of $W'$ a submanifold of $W$. 

Finally, we give the Frobinius theorem in another form. In this form,
it may be looked upon as a direct generalisation of the existence
theorem for ordinary differential equations proved in Chap. I,
\S\ \ref{chap1:sec6}. 

\setcounter{theorem}{3}
\begin{theorem}\label{chap2:sec6:thm4} %thm 4
  Let $\Omega$ be an open set in $\mathbb{R}^n$ with coordinates
  $(x_1, \ldots, x_n), \Omega'$ an open set in $\mathbb{R}^m$ with
  coordinates $(t_1, \ldots,t_m)$. Let $f_i: \Omega \times \Omega' \to
  \mathbb{R}^n$ be $C^k$ functions, $i = 1, \ldots, m(k \geq 2)$. In
  order that to every $t_0 \in \Omega'$ and $x_0 \in \Omega$  there is
  a neighbourhood $U$ of $t_0$ and a unique $C^k$ map 
  $ x : U \to \Omega$ such that 
  \begin{equation*}
    \frac{\partial x(t)}{\partial t_i}= f_i (x(t), t),~ i=1, \ldots, m,
    x(t_0) = x_0, \tag{6.2}\label{chap2:sec6:eq6.2} 
  \end{equation*}
  it\pageoriginale is necessary and sufficient that we have
  \begin{align*}
    \frac{\partial f_i}{\partial t_j} (x,t) & + (d_1 f_i) (x,t). f_j (x, t)\\
    =\frac{\partial f_j}{\partial t_i} (x,t) & + (d_1 f_j) (x,t). f_i (x, t)\\
  \end{align*}
  for $1 \leq i$, $j \leq m$, $(x, t) \in \Omega \times \Omega $
\end{theorem}

[Note that $d_1 f_i$ is a linear mapping of $\mathbb{R}^n$ into itself.]
\begin{proof}
  The uniqueness of the solution, if it exists, follows from the
  uniqueness theorem for solutions of ordinary differential equations
  pro\-ved in Cha. I, \S\ \ref{chap1:sec6}. If the equations
  (\ref{chap1:sec6:eq6.2}) are solvable, the
  equations (\ref{chap1:sec6:eq6.3}) hold; in fact the two sides of
  the equality at the 
  point $(x_0, t_0)$ are then simply 
  $$
  \frac{\partial^2 x(t)}{\partial t_j \partial t_i} \bigg|_{t=t_0}.
  $$
\end{proof}

To prove the converse, we proceed as follows. The equations
(\ref{chap2:sec6:eq6.2}) can be written 
\begin{equation*}
  \frac{\partial x_r}{\partial t_i} = f_{ir} (x, t), f_i = (f_{i1},
  \ldots, f_{in}), r=1, \ldots,n. \tag{6.4}\label{chap2:sec6:eq6.4} 
\end{equation*}
Consider the differential forms
\begin{equation*}
  dx_r - \sum^m_{i=1} f_{ir} (x, t) dt_i, r=1, \ldots, n
  \tag{6.5}\label{chap2:sec6:eq6.5} 
\end{equation*}
on $\Omega \times \Omega'$, and let $\mathscr{D}$ be the form
differential system of rank $m$ defined by them. If $\mathscr{D}$ has
an integral manifold of the form $x - \varphi (t) = 0$, where
$\varphi$\pageoriginale is a $C^k$ map of a neighbourhood of $t_0$ into $\Omega$
with $\varphi (t_0)=x_0$, then $x = \varphi$ is a solution of
(\ref{chap2:sec6:eq6.2}). Suppose now that $u_1, \ldots, u_n$ are
$C^k$ functions near 
$(x_0, t_0)$ such that $(du_1) (x_0, t_0), \ldots , (du_n) (x_0, t_0)$
are linearly independent. Then if the manifold $W= \{u_1 = \ldots =
u_n =0 \}$ [in a neighbourhood of $(x_0, t_0)$ this is a manifold by
  the rank theorem] is an integral of $\mathscr{D}$, it is clear that
the forms (\ref{chap2:sec6:eq6.5}) and the forms $du_1, \ldots ,du_n$ generate the same
subspace of $T^*_{(x_0, t_o)}(\Omega \times \Omega')$. Hence $(d_1
u_1) (x_0, t_0), \ldots ,(d_1 u_1)(x_0,t_0)$ are linearly
independent. Hence by the implicit function theorem, $W$ is given by
equations $x - \varphi(t) = 0$, and by our remark above, the equations
(\ref{chap2:sec6:eq6.2}) are solvable. Thus, if $\mathscr{D}$ is
completely integrable, 
then the equations (\ref{chap2:sec6:eq6.2}) are solvable. 

Now, as we have seen before, $\mathscr{D}$ has a basis given by the
vector fields 
\begin{equation*}
  X_i = \frac{\partial}{\partial t_i} + \sum^n_{r=1} f_{ir}(x, t)
  \frac{\partial}{\partial x_r}; \tag{6.6}\label{chap2:sec6:eq6.6} 
\end{equation*}
further, we have seen in the proof of Lemma \ref{chap2:sec6:lem2} that
$\mathscr{D}$ is completely integrable if and only if 
$$
[X_i, X_j] = 0 ~\text{ for }~ i, j \leq m.
$$

It is easily verified that these latter conditions are precisely the
condition (\ref{chap1:sec6:eq6.3}). Thus, if conditions
(\ref{chap1:sec6:eq6.3}) are satisfied,
$\mathscr{D}$ is completely integrable, and in particular the
equations (\ref{chap1:sec6:eq6.2}) are solvable.  

\begin{remark*}
Theorem \ref{chap2:sec6:thm4}\pageoriginale is true also for $C^1$ functions $f_i$; $a$ proof of this
  statement can be obtained by using the remark made after the proof
  of Frobenius' theorem. 
\end{remark*}

We remark that if, in Theorem \ref{chap2:sec6:thm4}, we take $n=1$ and
the $f_i$ to be 
functions independent of $x$, we obtain the following result. 

In order that there exist a $C^k$ function $x(t_1, \ldots, t_m)$ for
which, in a neighbourhood of $t_0$, we have 
$$
\frac{\partial x}{\partial t_i} = f_i(t),
$$
it is necessary and sufficient that 
$$
\frac{\partial f_i}{\partial t_i} = \frac{\partial f_j}{\partial t_i}
$$

This can be formulated as follows. Consider the 1-form $\omega = \sum
\limits^m_{i=1} f_i(t)$ $dt_i$. Then there is a function $f$ with
$df=\omega$ in a neighbourhood of any point if and only if $d \omega
= 0$.  

This result is a special case of Poincare's lemma, which we shall prove later.

For the material concerning 1-parameter groups, see Nomizu \cite{33}. A
different treatment of the Frobenius theorem (in the first form given
here) will be found in Chevalley \cite{7}. 

\section[Poincare's lemma, the type decomposition...]{Poincare's lemma, the type decomposition of complex
  co vectors, and Grothendieck's lemma}\label{chap2:sec7} % sec 7 

\begin{defi*}
  If $V$ is a $C^k$ manifold of dimension $n$ a differential form
  $\omega$, of degree $p$, is said to be closed if $d \omega =0$ and
  is said to be\pageoriginale exact if there exists a form $\omega_1$ of degree
  $p-1$, such that $d \omega_{11} = \omega$. 
\end{defi*}

Since $d^2 = 0$, an exact form is closed. We denote the set of closed
$p$-differential forms by $Z^p(V)$ and the set of exact $p$-differential
forms by $B^p(V)$. The quotient $H^p(V) = Z^p / B^p(V)$ is called the
$p^{th}$ de Rham group of $V$. A basic theorem of de Rham, which we
shall not prove here, implies that the $H^p(V)$ are topological
invariants. i.e., if $V$, $V'$ are homeomorphic, then $H^p(V) \approx
H^p(V')$. For a proof, see e.g. A.~Weil \cite{43}. 

\medskip
\noindent
\textbf{Poincare's lemma.}
If $D$ is a convex open set $\mathbb{R}^n$, every closed form of
degree $\geq 1$ on $D$ is exact, i,e. $H^p(D) = 0$ for $p \geq 1$. 

\begin{proof}
  We may suppose without loss of generality that $0 \in D$. Let $I=(0,
  1)$, be the open unit interval. Consider the map $h:D \times I \to
  D$ given by $h(x, t) = t.x$. 
\end{proof}

If $\omega$ is a closed $p$ form on $D$, $p \geq 1$, let $\omega =\sum
\limits_I a_I (x)$. $dx_I$ in terms of the coordinates of
$\mathbb{R}^n$. Then $h^* (\omega)$ is a form on $D \times I$ given by 
\begin{align*}
  h^* (\omega) & = \sum_I a_I (tx) d (tx_I), I = (i_1, \ldots, i_p),
  i_1 < i_2 \ldots < i_p \\ 
  &  = \sum_I a_I (tx) t^p dx_I + t^{p-1} \sum_I a_I (tx)(\sum_j
  (-1)^{j-1} x_j dt \wedge dx_{I, j}) 
\end{align*}
where
\begin{align*}
  dx_{I, j} & = dx_i \wedge \ldots \wedge d\hat{x}_j \wedge \ldots
  \wedge dx_{i_n} \text{ if } j \in I \\ 
  & = 0 \text{ otherwise}.
\end{align*}

Hence\pageoriginale $h^*(\omega) = \sum a_I (tx) t^p d(x_I) + dt \wedge \omega'$
where $\omega'$ is a $(p-1)$ form on $D \times I$. We have 
{\fontsize{10}{12}\selectfont
$$
\displaylines{\hfill 
  0 = h^* (d \omega ) = d (h^* (\omega )),\hfill \cr
  \text{so that } \hfill 
  \sum \limits_I \dfrac{\partial}{\partial t} (t^p a_I (tx)) dt \wedge
  dx_I + t^p \sum_{j, I} \frac{\partial}{\partial x_j} (a_I (tx)) dx_j
  \wedge dx_I - dt \wedge d \omega' =0.\hfill } 
$$} 

This implies that 
$$
\sum \frac{\partial}{\partial x_j}(a_I (tx)) dx_j \wedge dx_I = 0
$$
and that
\begin{gather*}
  \sum \frac{\partial}{\partial t}(t^p a_I (tx)) dt \wedge dx_I \\
  = dt \wedge d \omega'.
\end{gather*}
Since $dx_I$ does not contain $dt$, this implies that
$$
\displaylines{\hfill 
\frac{\partial}{\partial t} \left(\sum t^p a_I (tx) dx_I\right) = d_x
\omega' \hfill \cr
\text{where}\hfill d_x \omega' = \sum dx_i \wedge \dfrac{\partial
  \omega'}{\partial x_i}\hfill }
$$ 

Hence 
\begin{align*}
  \int \limits^1_0 \frac{\partial}{\partial t} \left(\sum a_I (tx)t^p
dx_I\right)dt &=\omega  ~ (\text{since}~ p \geq 1) \\
  & = \int \limits^1_0 d_x \omega' dt. \\
  & = d_x \bigg[ \int \limits ^1_0 \omega' dt \bigg ].
\end{align*}
i.e.\pageoriginale \qquad $\omega = d \omega_1$ where $\omega_1 = \int \limits^1_0
\omega' dt$. 

Compare this proof with the one given in A. Weil \cite{43}.

We introduce on $\mathbb{R}^{2n}$ the structure of a vector space over
$\mathbb{C} $ by means of the $\mathbb{R}$ isomorphism of
$\mathbb{R}^{2n}$ isomorphism of $\mathbb{R}^{2n}$ onto $\mathbb{C}^n$ 
given by  
$$
(x_1, \ldots, x_{2n}) \leftrightarrow (z_1, \ldots , z_n)
$$
where $z_j = x_{2{j-1}} + i x_{2j}$.

If $E$ is a vector space over $\mathbb{C} $, of dimension $n$,
consider the complex vector space $\mathscr{E}^* = \Hom_{\mathbb{R}}
(E,\mathbb{C})$, of $\mathbb{R}$-linear mappings of $E$ into
$\mathbb{C} $. 
\begin{align*}
  \text{ Let }~ F = & \{ f | f \text{ an } \mathbb{R} \text{ linear
    form }: E \to \mathbb{C} \text{ such that }~ f(iv) = if (v) \}. \\ 
  \bar{F} = & \{ f | f \text{ an } \mathbb{R} \text{ linear form }: E
  \to \mathbb{C} \text{ such that }~ f(iv) = - if (v) \}. 
\end{align*}
Then \qquad $\mathscr{E}^* = F \oplus \bar{F}$.

[For if $g \in \mathscr{E}^*$, consider $f'$ and $f''$ defined by 
\begin{align*}
  f' (v) & = \frac{1}{2} \{ g(v) - ig (iv) \}\\
  f'' (v) & = \frac{1}{2} \{ g(v) + ig (iv) \}.
\end{align*}
Then $ g= f' + f''$ and $f' \in F, f'' \in \bar{F}$]

We denote $F$ by $E(1, 0)$ and $\bar{F}$ by $E^{(0, 1)}$. Conjugation
$z \to \bar{z}$ in $\mathbb{C}$ defines an $\mathbb{R}$-isomorphism of
$F$ onto $\bar{F}$. Let $(e_1, e_2, \ldots, e_n)$ form a $\mathbb{C}$
basis of $F$. Then $(\bar{e}_1 , \ldots , \bar{e}_n)$ forms a $
\mathbb{C}$ basis of $\bar{F}$. 

We shall have to consider the vector space $\wedge^r
\mathscr{E}^*$. For fixed $p$, $q$ with $p+q = r$, let
$\mathscr{E}^*_{p, q}$ denote the complex subspace of $\wedge^r
\mathscr{E}^*$ generated\pageoriginale by the elements of the form  
$$
e_I \wedge \bar{e}_J = e_{i_1} \wedge \ldots \wedge_{e_{i_p}} \wedge
\bar{e}_{j_1} \ldots \wedge \bar{e}_{j_q} 
$$
where $i_1 < \ldots < i_p , j_1 < \ldots < j_q$ (but there is no
relation between the $i$ and the $j$). Then the elements $e_I \wedge
\bar{e}_J$ are linearly independent and span $\wedge ^r \mathscr{E}^*$
if $I$, $J$ run over all increasing sequences of $p$ and $q$ integers
respectively, so that $\wedge^r \mathscr{E}^* = \sum \limits_{p+q =r}
\mathscr{E}^*_{p,q}$. 

In what follows, $V$ is a complex analytic manifold of complex
dimension $n, (x_1, y_1, \ldots, x_n, y_n)$ denotes the real local
coordinates and $(z_1,\ldots,z_n), z_j = x_j + iy_j$, complex
coordinates. Let $T_a = T_a (V)$ be the tangent space to $V$ at a
considered as a $C^\infty$ manifold  of dimension $2n$ over
$\mathbb{R}$. 

Let $\mathscr{T}^*_a = \Hom_{\mathbb{R}}(T_a, \mathbb{C})$.

Clearly $\mathscr{T}^*_a$, as a vector space over $\mathbb{C}$ has
dimension $2n$. Since $(dx_j)_a, (dy_j)_a \in \Hom_\mathbb{R}(T_a,
\mathbb{R}) \subset \Hom_\mathbb{R} (T_a, \mathbb{C})$, the
expressions $(dz_j)_a = (dx_j)_a + i(dy_j)_a, (d \bar{z}_j)_a =
(dx_j)_a - i (dy_j)_a$ are well defined elements of $\mathbb{J}^*_a$;
it is clear that they form a $\mathbb{C}$ basis of
$\mathscr{T}^*_a$. Note that for any complex valued $C^\infty$
function $g$ on $V$, the differential $(dg)_a \in \mathscr{T}^*_a$. 

We note the mapping $T_a (V) \to \mathbb{R}^{2n}$ defined by $ X \to
(dx_1 (X)$, $dy_1 (X), \ldots, dx_n (X)$, $dy_n (X))$ is an
$\mathbb{R}$-isomorphism. Hence the map $x \to (dz_1 (X), \ldots, dz_n
(X)$ is an $\mathbb{R}$- isomorphism of $T_a (V)$ onto $\mathbb{C}^n$. This
isomorphism defines the structure of complex vector space on
$T_a(V)$. This structure is independent of the complex coordinate
system\pageoriginale used. It is seen at once that it is uniquely characterised by
the following property. 

If $f$ is a germ of holomorphic function at $a \in V$, we have
$$
(df)_a (( \alpha + i \beta ) X) = (\alpha + i \beta) (df)_a (X),
\alpha, \beta \in \mathbb{R}, X \in T_a (V). 
$$

We may also consider the space $\mathscr{T}_a(V) = T_a (V)
\otimes_\mathbb{R} \mathbb{C} = \Hom (\mathscr{T}^*_a$,
$\mathbb{C})$. This is called the space of complex tangent vectors at
$a$. $\mathscr{T}_a (V)$ has a basis. dual to the basis $dz_1, \ldots,
dz_n$, $d \bar{z}_1, \ldots d \bar{z}_n$ of $\mathscr{T}^*_a$; this
basis is denoted by $\dfrac{\partial}{\partial z_1}, \ldots ,
\dfrac{\partial}{\partial z_n}$, $\dfrac{\partial}{\partial
  \bar{z}_1}, \ldots$, $\dfrac{\partial}{\partial \bar{z}_n}$. It is
easily verified that, in terms of the tangent vectors
$\dfrac{\partial}{\partial x_j}$, $\dfrac{\partial}{\partial y_j}$
(which also form a basis of $\mathscr{T}_a (V)$ we have, 
$$
\dfrac{\partial}{\partial z_j}= \frac{1}{2} \left(\dfrac{\partial}{\partial
  x_j}- i \dfrac{\partial}{\partial y_j}\right),\dfrac{\partial}{\partial 
  \bar{z}_j}= \frac{1}{2} \left(\dfrac{\partial}{\partial x_j} + i
\frac{\partial}{\partial y_j}\right). 
$$
$\mathscr{T}^*_a$ is the complexification of $T^*_a$ and elements of
$\mathscr{T}^*_a$ are called complex co vectors at $a$. $\wedge^p
\mathscr{T}^* (V)$ is a $C^\infty$ manifold of real dimension $2n +
(^{4n}_p)$. Hereafter, by a $p$ differential form $\omega$, we mean a
complex $p$ differential form, i.e., a $C^\infty$ map $\omega : V \to
\wedge^p \mathscr{T}^* (V)$ such that $ \omega (a) \in \wedge^p
\mathscr{T}^*_a (V)$. 

We return now to our remarks on $\wedge^r \mathscr{E}^*$ for a complex
vector space $E$, where $\mathscr{E}^* = \Hom_\mathbb{R}(R,
\mathbb{C})$. We take for $E$, the space $T_a = T_a (V)$ with the
complex structure introduced above. It is immediate that
$\mathscr{E}^*_{1, 0}$ is the space spanned by $dz_1, \ldots, dz_n,
\mathscr{E}^*_{0, 1}$, that spanned by $d \bar{z}_1, \ldots,
d\bar{z}_n$. Hence $\mathscr{E}^*_{p, q}$ is spanned by the convectors
$ dz_I \wedge d\bar{z}_J =dz_{i_1} \wedge \cdots \wedge dz_{i_p}
\wedge d\bar{z}_{j_1} \wedge \cdots \wedge d \bar{z}_{j_q}$\pageoriginale 

A differential form $\omega$ is said to be of type $(p, q)$ if for
each $a \in V$, $\omega \in \mathscr{E}^*_{p, q}$ that is to say, 
$$
\omega_a = \sum^a_{\substack { i_1 < \ldots < i_p \\ { j_1 < \ldots <
      j_p}}} \omega_{IJ} dz_I \wedge d \bar{z}_J, I = (i_1, \ldots,
i_p), J = (j_1, \ldots, j_p). 
$$
The operator $d$ of exterior differentiation
defined on real valued forms extends obviously to a $\mathbb{C}$
linear map from $C^\infty p$ forms to $C^\infty (p+1)$ form, with
properties similar to those proved before. 

If $f$ is a complex valued function, we have a decomposition 
$$
df = \partial f + \bar{\partial}f
$$
where $\partial f$ is of type $(1, 0)$ and $\bar{\partial}f$ of type
$(0, 1)$, since the space $\mathscr{E}^* = \mathscr{E}_{1, 0} \oplus
\varepsilon^*_{1, 0}$. In terms of local coordinates, we have  
$$
\partial f = \sum \frac{\partial f}{\partial z_k} dz_k,
\bar{\partial}f = \sum \frac{\partial f}{\partial \bar{z}_k} d
\bar{z}_k. 
$$
If $\omega$ is a form of type $(p, q)$ say, 
\begin{align*}
  \omega & = \sum \omega_{IJ} dz_{i_1} \wedge \ldots \wedge dz_{i_p}
  \wedge d \bar{z}_{j_1}\ldots \wedge d \bar{z}_{j_q} \\ 
  & = \sum \omega_{IJ} dz_I \wedge d\bar{z}_J \text{ say },\\
  \text{ then } \quad d \omega & = \sum d \omega_{IJ} \wedge dz_I
  \wedge d \bar{z}_J \\ 
  & = \sum (\partial \omega_{IJ}+ \partial \omega_{IJ}) \wedge dz_I
  \wedge d \bar{z}_J 
\end{align*} 
so that $d \omega = \partial \omega + \bar{\partial} \omega$, where
$\partial \omega$ is of type $(p+1, q)$ and $\bar{\partial} \omega$ of
type $(p, q+1)$. From the fact that the decomposition $\wedge^r
\mathscr{E}^* = \sum \mathscr{E}^*_{p,q}$ is direct we see at once
that the fact that $d^2 = 0$ is equivalent with the three conditions 
$$
\partial^2 = 0, \partial \bar{\partial} + \bar{\partial} \partial = 0,
\bar{\partial}^2 = 0. 
$$

Note\pageoriginale further that we have $\bar{\partial}  f = \bar{\partial}
(\bar{f})$ [The operation $\bar{\partial f} $ is the conjugation
  $\mathscr{E}^*_{1,0} \rightarrow \mathscr{E}^*_{0,1}$ defined
  earlier]. 

\begin{defi*}%defi 0
  A differential form $\omega$ is holomorphic if $\omega$ is of type
  $(p,0)$ and $\bar{\partial}  \omega = 0$. 
\end{defi*}

\begin{remark*}%rem 0
  If $f$ is a $0-$ form, it is holomorphic if and only if $f$, as a
  function of $(z_1, \ldots, z_n)$, the complex local coordinates is
  holomorphic. Further if $\omega$ is of type $(p, 0)$, if $\omega =
  \sum f_I (z) dz_I$ in local coordinates, $\omega$ is holomorphic if
  and only if $f_I(z)$ is holomorphic for each $I$. 
\end{remark*}

We make two further remarks.

\begin{enumerate}
\item Any complex manifold is orientable. In fact the jacobian
  determinant of a holomorphic map $f: \Omega \rightarrow
  \mathbb{C}^n$, $\Omega$ open in $\mathbb{C}^n$, considered as a
  $C^\infty$ map of an open set in $\mathbb{R}^{2n}$ into
  $\mathbb{R}^{2n}$ (in terms of the 
  identification of $\mathbb{R}^{2n}$ and $\mathbb{C}^n $ made
  earlier) 
  is equal to $|D|^2$, where $D = \det \left(\dfrac{\partial
    f_i}{\partial x_j}\right)$.	 
\item Let $V$, $V'$ be complex manifolds, $f: V \rightarrow V'$ a
  holomorphic map. $f$ induces a $\mathbb{C}$ linear map $T_a (V)
  \rightarrow T_{f(a)}(V')$ since $\varphi \circ f$ is holomorphic for any
  holomorphic $\varphi$. Hence, the map $f^* : \mathscr{T}^*_{f (a)} \to
  \mathscr{T}^*_a$ maps $(\mathscr{E}^*_{1, 0})_{f(a)}$ into $(\mathscr{ 
    E}^*_{1, 0})_a$. Hence $f^*(\omega ')$ is of type $(p,q)$ if
  $\omega '$ is of type $(p,q)$. Since moreover, for any form $\omega
  '$ of type $(p,q)$ on $V'$, we have, 
  \begin{align*}
    \partial (f^*(\omega ')) + \bar{\partial}  (f^* (\omega ')) & = df^*
    (\omega ') = f^* (d \omega ') \\ 
    & = f^* (\partial \omega' ) + f^* (\bar{\partial} \omega'),
  \end{align*}
  and\pageoriginale $f^*$ preserves the type, we deduce that
  $$
  \partial f^* (\omega ') = f^* (\partial \omega '), \bar{\partial}  f^*
  (\omega') = f^* (\bar{\partial}  \omega'). 
  $$
\end{enumerate}

[Note that this is not true for any $C^\infty$ map $f$.] As in the
case of a $C^k$ manifold, we set $Z^{p,q}(V) = $ set of $C^\infty$
forms $\omega$ of type $(p,q)$ with $\bar{\partial} \omega = 0$ and
$B^{p,q}(V) = $ set of $C^\infty$ forms $\omega$ of type $(p,q)$, for
which there is a $C^\infty$ form $\omega'$ of type $(p,q - 1)$ with
$\bar{\partial} \omega = \omega$. Then, since $\bar{\partial}^2 = 0$,
we have $B^{p,q}(V) \subset Z^{p,q}(V)$. We set $H^{p,q}(V) =
Z^{p,q}(V) / B^{p,q}(V)$. These groups are called the Dolbeault groups
of $V$. 

These groups are not topological invariants of $V$. They depend
essentially on the holomorphic structure of $V$. 

We now look for an analogue of Poincare's lemma, i.e. for a class of
domains $D$ in $\mathbb{C}^n$  for which $H^{p,q}(D) = 0$ for $q*****1$. We begin
with the following lemma. 

\begin{lemma*}%lemma 0
  Let $K, L$ and $L'$ be compact sets in $\mathbb{C}, \mathbb{C}^r$
  and $\mathbb{R}^n$, respectively. We denote a point in $K \times L
  \times L'$ by $(z, w, t)$. If $g$ is a $C^\infty$ function defined
  in a neighbourhood of $K \times L \times L'$ and if $g$ is
  holomorphic in $w$ for each fixed $z$ and $t$, then there exists a
  $C^\infty$ function $f$ in a neighbourhood of $K \times L \times L'$
  which is holomorphic in $w$ for fixed $z$ and $t$ such that
  $\dfrac{\partial f}{\partial \bar{z}} = g$ in a neighbourhood of $K \times
  L \times L'$. 
\end{lemma*}

\begin{proof}
  We may assume that $g$ has compact support in $\mathbb{C}$ for any
  fixed $w$ and $t$ [Multiply $g$ if necessary, by $\varphi(z)$ where
    $\varphi$ has compact support and $= 1$ in a neighbourhood of
    $K$]. 
    Define 
  $$
  f(z, w, t) = - \frac{1}{\pi} \int\limits_{\mathbb{C}} \frac{g
    (\zeta, w,t)}{\zeta - z} d \xi \wedge d \eta 
  $$
\end{proof}

Then\pageoriginale 
\begin{align*}
  \frac{\partial f} {\partial \bar{z}} & = - \frac{1}{\pi}
  \int\limits_{\mathbb{C}} \frac{\partial g (\zeta + z, w, t)} {\partial
    \bar{z }} - \frac{1}{\zeta} d \xi \wedge d \eta \\ 
  & = \lim\limits_{\mathcal{E} \to 0}- \frac{1}{\pi} \int\limits_{|
    \zeta | \geq \mathcal{E}} \frac{\partial g (\zeta + z, w, t)}
             {\partial \bar{\zeta }} - \frac{1}{\zeta} d \xi \wedge d
             \eta \\ 
  & = \lim\limits_{\mathcal{E} \to 0}- \frac{1}{2 \pi i }
             \int\limits_{| \zeta | \geq \mathcal{E}} \frac{\partial g
               (\zeta + z, w, t)} {\partial \bar{\zeta }} -
             \frac{1}{\zeta} d \bar{\xi} \wedge d \eta \\ 
  & = \lim\limits_{\mathcal{E} \to 0}- \frac{1}{2 \pi i }
             \int\limits_{| \zeta | \geq \mathcal{E}} d\left(\frac{g
               (\zeta + z, w, t) d \zeta}{\zeta}\right). 
\end{align*}

Now by Stoke's theorem, 
\begin{align*}
  & -   \lim_{\mathcal{E} \to 0} \frac{1}{ 2 \pi i } \int\limits_{|
    \zeta | \geq \mathcal{E}}d \bigg( \frac{  g (\zeta + z, w, t) d
    \zeta  } {\zeta } \bigg)\\ 
  & = \lim_{\mathcal{E} \to 0} \frac{1}{2 \pi i } \int\limits_{|
    \zeta | = \mathcal{E}} \frac{d  (\zeta + z, w, t) d \zeta  }
  {\zeta }\\ 
  & = g(z).
\end{align*}

Clearly $f$ is a $C^\infty$ function and holomorphic in $w$ for fixed
$z$ and $t$ Grothendieck's lemma (or Poincare's lemma for
  $\bar{\partial}$). If $D = D_1 \times \cdots \times D_n$, where
  $D_i$ is a domain in $\mathbb{C}$, $1 \leq i \leq n$, then if
  $\overset{(p, q)}\omega$ is a $C^\infty$ differential form on $D$
  with $\bar{\partial} \omega = 0$, $q \geq 1$, then there exists a
  $C^\infty$ differential form $\omega'$ on $D$ such that
  $\bar{\partial}\omega = \omega$; in other words $H^{p, q}(D) = 0$
  for $q \geq 1$. 
\begin{proof}

To make clear the basic idea we shall first prove the lemma for $(0,
1)$ forms on $K = K_1 \times \cdots \times K_n$ where $K_i$ are
compact sets in. 
\end{proof}

[i.e. for forms in some neighbourhood of $K$, the above equations
  holding in some neighbourhood of $K$ (not necessarily the same)]. 

Let\pageoriginale $\omega = a_1 d \bar{z}_1 + \cdots + a_n d\bar{z}_n$. By the
lemma, there exists a $C^\infty$ function $b_n$ in a neighbourhood of
$K_1 \times \cdots \times K_n$ such that  
$$
\frac{\partial b_n}{\partial \bar{z}_n} = a_n.
$$
Let $\omega_{n-1} = (\omega - \bar{\partial}b_n )= a'_1d \bar{z}_n +
\cdots + a_{n-1} d\bar{z}_{n-1}$. 

Then $\bar{\partial} \omega = 0 \Rightarrow \bar{\partial} \omega_{n-1} = 0$.

Hence
$$
d \bar{z}_n \wedge \sum_{i \leq n-1} \frac{\partial a'_i}{\partial
  \bar{z}_n} d \bar{z}_i = 0 
$$
i.e. $a'_i$ are holomorphic in $z_n$.

Hence by the lemma there exists a $C^\infty$ function $b_{n-1}$ in a
neighbourhood of $K_1 \times \cdots \times K_n$ which is holomorphic
in $z_n$ and for which $\dfrac{\partial b_{n-1}}{\partial
  \bar{z}_{n-1}} = a'_{n-1}$. 

Let $\omega_{n-2} = \omega - \bar{\partial} b_n - \bar{\partial}b_{n-1}$.

Then $\omega_{n-2} = a''_1 d\bar{z}_1 + \cdots a''_{n-2}
d\bar{z}_{n-2}$ with $a''_1$ 
holomorphic in $z_{n-1}$, $z_{n-2}$. We continue the process and
obtain 
\begin{align*}
  \omega_1 & = \omega - \bar{\partial} b_n - \bar{\partial}b_{n-1}
  \cdots - \bar{\partial}b_1 = 0 \\ 
  \text{i.e.}\qquad  \omega &= \bar{\partial} (b_n + b_{n-1} \cdots + b_n).
\end{align*}
We shall now prove by induction the lemma for forms on 
$$
K = K_1 \times \cdots \times K_n, K_i \text{ compact in } \mathbb{C}.
$$
Let $\mathscr{O}_k = $ the set of differential forms of type $(p, q)$
not containing $d \bar{z}_k, \ldots d\bar{z}_n$ in their expressions
in local coordinates. 

Assume\pageoriginale that the lemma is proved for differential forms in
$\mathscr{O}_i$, $i \leq k$. (The lemma is trivial for
$\mathscr{O}_1$). Let $\omega$ be a differential form in
$\mathscr{O}_{k + 1}$. Then 
\begin{gather*}
  \omega  = d \bar{z}_k \wedge \omega_1 + \omega_2 \text{ where } \\
  \omega^{p,q-1}_1 , \omega^{p,q}_2 \in \mathscr{O}_k.
\end{gather*}

If $\bar{\partial} \omega = 0$, $d\bar{z}_k \wedge \bar{\partial}
\omega_1 + \bar{\partial} \omega_2 = 0$ hence $\dfrac{ \partial
  \omega_1}{\partial \bar{z}_j} = 0$ for $j > k$. 

Since, by assumption, $\dfrac{\partial \omega_2}{ \partial \bar{z}_j}
= 0$ for $j \geq k$. By the lemma, there exists  $\overset{p,
  q-1}\Phi$ in a neighbourhood of $K$, holomorphic in $z_j$, $j > k$
such that $\dfrac{\partial \Phi}{\partial \bar{z}_k} = \omega_1$. Then
$\omega - \bar{\partial} \Phi \in \mathscr{O}_k$, $\bar{\partial}
(\omega - \bar{\partial} \Phi) = 0$ and by the induction hypothesis
there exists $\psi$ such that 
\begin{align*}
  \bar{\partial} \psi & = \omega - \bar{\partial \Phi} \\
  \text{i.e.}\quad  \omega & = \bar{\partial} (\Phi + \psi ).
\end{align*}

We shall now prove the lemma for $D = D_1 \times \cdots \times
D_n$. Let $K^\nu_i$ be a sequence of compact sets, $K^\nu_i \uparrow
D_i$ as $\nu \to \infty$ and let $K^\nu = K^\nu_1 \times \cdots \times
K^\nu_n$. By what we have proved above, there exist differential forms
$\omega^\nu$ of type $(p, q-1)$ in neighbourhood of $K^\nu$ such that 
$$
\bar{\partial} \omega^\nu = \omega \text{ in a neighbourhood of } K^\nu.
$$
We shall consider two different cases
\begin{enumerate}
\item [(i)] $q \geq 2$ and (ii) q = 1
\item [(i)] If $q > 1, \bar{\partial} (\omega^{\nu+1} - \omega^\nu) = 0
  $ in a neighbourhood of $K^\nu$. 
\end{enumerate}

Since\pageoriginale $\omega^{\nu + 1} - \omega^\nu$ is of type of $(p, q-1)$ and
$q-1 \geq 1$, there exists a differential form $\varphi^{\nu + 1}$ of
type $(p, q - 1)$ in $D$ such that $\bar{\partial} \varphi^{\nu +1} =
\omega^{\nu + 1} - \omega^\nu$ on $K$. 

Let $\psi^{\nu + 1} = \omega^{\nu + 1} - \bar{\partial}\varphi^{\nu
  +1} - \cdots - \bar{\partial \varphi^1}$. 
\begin{align*}
  \text{Then} \hspace{2cm} \psi^{\nu + 1} - \psi^\nu & = \omega^{\nu + 1} -
  \omega^\nu - \bar{\partial} \varphi^{\nu +1}\hspace{2cm}\\ 
  & = 0  \text { in a neighbourhood of } K^\nu.
\end{align*}

Hence the form $\psi = \psi^\nu$ in $K^\nu$, $\nu \geq 1$, is well
defined, and $\bar{\partial} \psi = \omega$. 

We suppose that $K^\nu_i$ have the property that any holomorphic
function in a neighbourhood of $K^\nu_i$ can be approximated,
uniformly on $K^\nu_i$ by holomorphic functions in $D_i$ [It is a
  classical theorem that any domain in $\mathbb{C}$ can be
  approximated by such compact sets: this result is a consequence of
  the Runge theorem proved in Chap. III.] From Chap I, \S\ \ref{chap1:sec5}, it
follows that any holomorphic function on $K^\nu$ can be approximated
on $K^\nu$ by holomorphic functions in $D$ and in view of the remark
following the definition of holomorphic forms, there exist holomorphic
forms $\varphi^{\nu + 1}$ of type $(p,0)$ on $D$ such that
$||\varphi^{\nu + 1} - (\omega^{\nu +1} - \omega^\nu ) || <
\dfrac{1}{2^\nu}$ on $K^\nu$. [the inequality holding for all
  coefficients]. Hence $\sum \limits^\infty_1 \{ \varphi^{\nu + 1} -
(\omega^{\nu + 1} - \omega^\nu )\}$ is uniformly convergent on any
compact subset of $D$. 

Let $\omega' = \sum \limits^\infty_0 \{  \omega^{\nu +1} - \omega^\nu
- \varphi^{\nu + 1} \}$ where $\omega^0 = 0$; we have 
$$
\omega' = \omega^r - \varphi^r - \varphi^1 + \sum \limits^\infty_r  (
\omega^{\nu +1} - \omega^\nu  - \varphi^{\nu + 1}) 
$$
on $K^r$; since the $\varphi^\nu$ and $\sum \limits^\infty_r(
\omega^{\nu +1} - \omega^\nu  - \varphi^{\nu + 1})$ are holomorphic on
$K^r$ we conclude that $\bar{\partial} \omega' = \omega$ in $D$, and
that $\omega'$ is $C^\infty$. 

The\pageoriginale proof of Grothendieck's lemma on compact sets given above follows
essentially the exposition by Serre \cite{41}  of the original proof of
Grothendieck. It is to be remarked that also the proof of Poincare's
lemma (for cubes instead of arbitrary convex sets) can be given on the
same lines as that of the Grothendieck lemma. This is essentially the
proof given by E. Cartan \cite{5}; this proof of $E$. Cartan was in fact
the origin of the proof of the Grothendieck lemma. 

\section[Applications to complex analysis...]{Applications to 
complex analysis. Hartogs' continuation
  theorem and the Oka-Weil theorem}\label{chap2:sec8} %sec 8 

\setcounter{proposition}{0}
\begin{proposition}\label{chap2:sec8:prop1}%prop 1
  Let $\Omega$ be a convex open set in $\mathbb{C}^n$ and $\varphi$ a
  real valued $C^\infty$ function on $\Omega$. In order the there
  exist a holomorphic function $f$ on $\Omega$ such that Re $f =
  \varphi$, it is necessary and sufficient that 
  $$
  \frac{\partial^2 \varphi}{\partial z_i \partial \bar{z}_j} = 0
  \text{ for } 1 \leq i, j \leq n. 
  $$
\end{proposition}

\noindent \textit{Proof.}
  If $\varphi = Re \ f = \dfrac{1}{2} (f + \bar{f})$, then
  $\dfrac{\partial^2 \varphi}{\partial z_i \partial \bar{z}_j} = 0$
  since $\dfrac{\partial f}{\partial \bar{z}_j} = 0, \dfrac{\partial
    \bar{f}}{\partial z_i} = \dfrac{\overline{\partial f}} {\partial
    \bar{z}_i} = 0$. Suppose conversely that these equations are
  satisfied. We see at once that the form of type (\ref{chap2:sec1:eq1.1}) 
  \begin{equation*}
    \bar{\partial} \partial \varphi = 0.
  \end{equation*}

Since $d = \partial + \bar{\partial}$ and $\bar{\partial}^2 =0$, this
can be written $d \partial \varphi = 0$. By Poincare's lemma, there is
a complex valued function $g$ on $\Omega$ with  
$$
dg = \partial \varphi.
$$

Since\pageoriginale $\partial \varphi$ is of type $(1,0)$, we have $\partial g =
\partial \varphi, \bar{\partial} g = 0$, so that $g$ is
holomorphic. Further 
$$
d(g + \bar{g}) = dg + \overline{dg} = \partial \varphi +
\overline{\partial \varphi} = d \varphi, 
$$
so that $g + \bar{g} - \varphi$ is constant, and the proposition follows.

This implies the following

\begin{prop*}[{\boldmath $1'$}]\label{chap2:sec8:prop'}%prop 1'
  Let $\varphi$ be a $C^\infty$ real valued function on the complex
  manifold $V$. In order that $\varphi$ be locally the real part of a
  holomorphic function, it is necessary and sufficient that
  $\bar{\partial} \partial \varphi = 0$.  
\end{prop*}

\setcounter{lemma}{0}
\begin{lemma}\label{chap2:sec8:lem1}%lemm 1
  If $D = \{(z_1, \ldots , z_n) \in \mathbb{C}^n  | |z_i| < R_i \}$,
  $n \geq 2$, $U$ is a neighbourhood of $\partial D$ in $\mathbb{C}^n$
  and if $f$ is a holomorphic function in $U \cap D$, there exists a
  neighbourhood $V$ of $\partial D$ and a holomorphic function $F$ in
  $D$ such that $F | V \cap D = f$. 
\end{lemma}

\begin{proof}
  Let $\mathscr{E}_1$, $\mathscr{E}_2$ be two positive numbers such
  that if $U_1 = \{ (z_1, \ldots , z_n)$ $| R_1 - \mathscr{E}_1 < |z| <
  R_1 , |z_2| < R_2, \ldots , |z_n| < R_n   \}$ and $U_2 = \{ (z_1,
  \ldots , z_n)  |z_1| < R_1 , R_2 - \mathscr{E}_2 <  |z_2| < R_2,
  \ldots , |z_n| < R_n   \}$, then $U_1 \cup U_2 \subset U$. 
\end{proof}

For any holomorphic function $f$ on $U_1$ there exist holomorphic
functions $a_r$ in $\{ |z_2| < R_2, \ldots , |z_n| < R_n \}$ such that
$f(z) = \sum \limits^\infty_{-\infty} a_r(z') z^r_1$ where $z' = (z_2,
\ldots , z_n)$. Let $z' = (z_2, \ldots , z_n)$ be any point with  
$$
R_2 - \varepsilon_2 < |z_2| < R_2 , \ldots , |z_n| < R_n.
$$

Then $f(z_1, z')$ is holomorphic for $|z_1| < R_1$ since $f$ is
holomorphic in\pageoriginale $U_2$. Hence there can be no terms containing negative
powers of $z_1$ in the Laurent as expansion of $f$: thus $a_r (z') = 0
$ for $r < 0$, if $R_2 - \varepsilon_2 < |z_2| < R_2$. By the
principle of analytic continuation, this implies that $a_r(z') = 0$,
for $r < 0$, $|z_2| < R_2, \ldots |z_n| < R_n$, and  
$$
f(z) = \sum^\infty_0 a_r (z') z^r_1 ~\text{in}~ U_1 \cup U_2.
$$
By Abel's lemma, $ \sum\limits^\infty_0 a_r (z') z^r_1$ is uniformly
convergent on compact subsets of $D$ and hence	 
$$
F(z) =  \sum^\infty_0 a_r (z') z^r_1
$$
is a holomorphic extension of $f | U_1 \cup U_2$ to $D$. Hence $F = f$
in the connected component $\Omega$ of $U \cap D$ containing $U_1 \cup
U_2$; since $\partial D$ is connected, $\Omega = V \cap D$where $V$ is
a neighbourhood of $D$. 

\begin{lemma}\label{chap2:sec8:lem2}% lem 2
  If $\omega$ is a differential form of type $(0,1)$ with compact
  support in $\mathbb{C}^n$, $n \geq 2$, and if $\bar{\partial} \omega
  = 0$, there exists a $C^\infty$ function $\varphi$ on
  $\mathbb{C}^n$, with compact support, such that $\bar{\partial}
  \varphi = \omega$. 
\end{lemma}

\noindent\textit{Proof.}
  Choose $R > 0$ such that if
  \begin{equation*}
  D \left\{ (z_1, \ldots , z_n) \Big| |z_i| < R \right\}, \text{ then supp. }
  \omega \subset D.\tag*{$\Box$} 
  \end{equation*}

By Poincare's lemma for $\bar{\partial}$, there exists a $C^\infty$
function $f$ on $\mathbb{C}^n$ such that  
$$
\bar{\partial} f = \omega.
$$

Now we have $\omega = \bar{\partial} f = 0$ in a neighbourhood of
$\partial D$, i.e. $f$ is holomorphic in a neighbourhood of $\partial
D$. Hence by Lemma \ref{chap2:sec8:lem1} there exists a function $F$,
holomorphic\pageoriginale on $D$ 
such that $F(z) = f(z)$ for $x$ in a certain neighbourhood of
$\partial D$. Consider 
$$
\varphi (z) = \begin{cases} f(z) - F (z) & \text{for}~
  z \in D \\0 & \text{for}~ z \notin D. 
\end{cases}
$$ 

Then clearly $\varphi$ is
$C^\infty$ function with compact support and $\bar{\partial} \varphi =
\omega$. 

We shall now prove the following important theorem of Hartogs.

\setcounter{theorem}{0}
\begin{theorem}[Hartogs]\label{chap2:sec8:thm1}%the 1
  Let $D$ be a bounded open connected subset of $\mathbb{C}^n$, $n
  \geq 2$, such that  $\mathbb{C}^n - D$ is connected, and $U$, a
  neighbourhood of $\partial D$. If $f$ is a holomorphic function on
  $U$, then there exists a neighbourhood $V$ of $\partial D$ and a
  holomorphic function $F$ on $D$ such that $F | V \cap D = f$. 
\end{theorem}

\begin{proof}
  We can assume without loss of generality that $f \in C^\infty$ in
  $D$. [If not, multiply $f$ by a $C^\infty$ function $\alpha$ with
    compact support in $U$ such that $\alpha(z) = 1$ for $z$ in a
    neighbourhood of $\partial D$.] Let $\omega = \bar{\partial} f$ in
  $D$;  since $f$ is holomorphic near $\partial D, \omega$ has compact
  support in $D$. We extend it to $\mathbb{C}^n$ by setting $\omega =
  0$ outside $D$. 
\end{proof} 
 
 Then $\omega$ is of type $(0,1)$ and has compact support and
 $\bar{\partial} \omega = 0$. Hence by Lemma \ref{chap2:sec8:lem2}, there exists a
 $C^\infty$ function $\varphi$, in $\mathbb{C}^n$, with compact
 support such that $\bar{\partial} \varphi = \omega$. 
 
 In particular $\varphi$ is holomorphic on each open set on which
 $\omega$ vanishes and hence $\varphi$ is holomorphic in a
 neighbourhood of $\mathbb{C}^n - D$. Also $\varphi$ has compact
 support and $\mathbb{C}^n - D$ is connected. Hence, by the principle
 of analytic continuation $\varphi = 0$ in a connected neighbourhood
 of $\mathbb{C}^n - D$ and hence $\varphi = 0$ in a neighbourhood $V$
 of $\partial D$. Consider $F = f - \varphi$; we have  
 $$
 \bar{\partial} F = 0 \text{ in } D, F = f \text{ near } \partial D.
 $$
 Hence\pageoriginale $F$ is a holomorphic function with the required properties.
 
\begin{defi*}
  A domain $D$ in $\mathbb{C}^n$ is said to be a Cousin domain if
  given a differential form $\omega$ of type $(p,q)$, $q \geq 1$, $p
  \geq 0$, such that $\bar{\partial} \omega = 0$, there exists a
  differential form $\omega'$ of type $(p, q-1)$ such that
  $\bar{\partial} \omega' = \omega$; (in this case we shall also any
  that $D$ is Cousin). 
\end{defi*} 

\begin{theorem}[Oka]\label{chap2:sec8:thm2}%them 2
  Let $B = \{z \in \mathbb{C} \big| |z| < 1 \}$. If $D$ is a domain in
  $\mathbb{C}^n$ such that $D \times B$ is Cousin and if $f$ is a
  holomorphic function on $D$, $D_f = \{z \in D \big| |f(z)| < 1 \}$, then
  $D_f$ is Cousin. Further given a differential form $\overset{(p,q)}
  \omega$, $q \geq 0$ on $D_f$, such that $\bar{\partial} \omega = 0$,
  there exists a form $\Omega$ of type $(p,q)$ on $D \times B$ with
  $\bar{\partial} \Omega = 0$ such that if $i: D_f \rightarrow D \times
  B$ is the map given by $i(z) = (z, f(z))$, we have $i^*
  (\Omega) = \omega$. 
\end{theorem} 

\begin{proof}
  We begin with the remark that $i$: $D_f \rightarrow D \times B$ is
  injective and proper; further $i_*$ is injective at every point, so
  that $i(D_f)$ is a closed complex analytic submanifold of $D \times
  B$. Let $\pi$: $D \times B \rightarrow D$ be the projection 
  $$
  \pi (z, z') = z, (z, z') \in D \times B.
  $$
\end{proof} 
 
Let $\pi^{-1} (D_f) = D_f \times B = V$. Then $V$ is a neighbourhood
of $i (D_f)$ in $D \times B$. 
 
Let $V'$ be a neighbourhood of $i(D_f)$ in $V$ such that $\bar{V'}
\subset V$. Then there exists a $C^\infty$ function $\alpha$ on $D
\times B$ such that 
\begin{align*}
  \alpha (z, z') & = 1 \text{ if } (z, z') \text{ is in a
    neighbourhood of } i(Df) \\ 
  & = 0 \text{ if } (z, z') \notin V'.
\end{align*} 
 
Let $\varphi = \pi^*(\omega)$ on $V$; since $\pi$ is holomorphic,
$\varphi$ is of type $(p,q)$. Further, since $\pi \circ j =$ identity on
$D_f$, we have $i^* (\varphi) = \omega$. Then if $\varphi'$\pageoriginale is defined
on $D \times B$ as $\varphi' = \alpha \varphi$ on $V = 0$ outside $V$,
$\varphi'$ is a $C^\infty$ form of type $(p,q)$ on $D \times B$, and
since $\varphi' = \varphi$ near $i(D_f)$ we have $i^* (\varphi') =
\omega$. Let $\omega_1$ be the form defined on $D \times B$ by  
\begin{align*}
  \omega_1 & = \text{ in a neighbourhood of } i(D_f) \\
  & = \frac{ 1}{z' - f(z)} \bar{\partial}(\varphi') \text{ in } D
  \times B - i (D_f). 
\end{align*} 
 
Then $\bar{\partial} \omega_1 = 0$ and $\omega_1$ is of type $(p,
q+1), q \geq o$. [$\omega_1$ is $C^\infty$ since $\bar{\partial}
  (\varphi') = \bar{\partial} (\varphi) - 0$ in a neighbourhood of
  $i(D_f)$.] Hence there exists $\psi$ of type $(p,q)$ such that
$\bar{\partial} \psi = \omega_1$. 
 
Consider $\bar{\partial} (\varphi' + (z' - f(z)) \psi) $
$$
= \bar{\partial} (\varphi') - (z' - f(z)) \bar{\partial} \psi.
$$

Clearly $\bar{\partial} [ \varphi' + (z' - f (z)) \psi]
= 0$ on $D \times B$ and $i^* [\varphi' + (z' - f(z)) \psi] = i^*
(\varphi') = \omega$. 

Hence given a differential form $\overset{(p,q)} \omega$, $q \geq 0$ on
$D_f$ with $\bar{\partial} \omega = 0$ there exists a form $\Omega \{
= \varphi' + (z - f(z)) \psi \}$, on $D \times B$ such that
$i^*(\Omega) = \omega$ and $\bar{\partial} \Omega = 0$. Since $D
\times B$ is Cousin it follows immediately that $D_f$ is Cousin. 

\begin{coro*}
  With the same notation as in the theorem, if $D \times B^r$ is
  Cousin for every positive integer $r$, so is $D_f \times B^r$ for
  every positive integer $r$. 	 
\end{coro*} 

\begin{proof}
  Consider $D \times B^{r + 1}$, a point in $D \times B^{r + 1}$ being
  denoted by \break $(z, w_1, \ldots , w_{r+1})$. Then, if $D' = (D \times
  B^r)$, $D'_f = D_r \times B^r$ and by applying the lemma to $D'$ the
  corollary is proved. 
\end{proof} 

\begin{theorem}[Oka]\label{chap2:sec8:thm3}%the 3
  If\pageoriginale $z = (z_1, \ldots , z_n) \in \mathbb{C}^n$ and if $f_i(z)_{1 \leq
    i \leq r}$ are holomorphic functions in $\mathbb{C}^n$ and if $U =
  \left\{z \in \mathbb{C}^n \Big| |f_i (z)|<1 i, 1\leq i \leq r
  \right\}$ the map $U 
  \rightarrow \mathbb{C}^n \times B^r$ given by $i(z) = (z, f_1(z),
  \ldots , f_r (z))$, then  given a holomorphic function $g$ on $U$,
  there exists a holomorphic function $F$ on $\mathbb{C}^n \times B^r$
  such that $G \circ i = g$. 
\end{theorem} 

\begin{proof}
  Let
  $$
  D_0 = \mathbb{C}^n , D_{k+1} = \{z \in D_k \bigg| |f_{k+1}(z)| < 1
  \}; 0 \leq k < r. 
  $$
  Clearly $D_r = U$.
\end{proof} 
 
Let $i_r$ be the map $D_r \rightarrow D_{r-1} \times B$ defined by
$i_r (z) = (z, f_r(z))$, $i_{r-1}$: $D_{r-1} \times B \rightarrow
D_{r-2} \times B^2$ the map defined by $i_{r-1} (z, w_1) = (z, w_1,
f_{r-1} (z))$, and so on. Then we have $i = i_1 \circ i_2 \circ \cdots \circ
i_r$. Further, since $\mathbb{C}^n \times B^m$ is Cousin for every $m$
(by Poincare's lemma for$\bar{\partial}$), it follows by the corollary
to Theorem \ref{chap2:sec8:thm2} that $D_k \times B^m$ is Cousin for $0 \leq k \leq r$,
and all $m \geq 0$, so that, by Theorem \ref{chap2:sec8:thm2}, for any form $\omega_k$
of type $(p,q)$ on $D_k \times B^{r-k}$ with $\bar{\partial} \omega_k
= 0$, there is a form $\omega_{k+1}$ of type $(p,q)$ on $D_{k-1}
\times B^{r- k+1}$ with $\bar{\partial} \omega_{k+1} = 0$ and $i^*_k
(\omega_{k+1}) = \omega_{k}$. Hence, by induction for any
$\bar{\partial}$ closed form $\omega$ of type $(p,q)$ on $U = D_r$,
there is a $\bar{\partial}$ closed form $\Omega$ of type $(p,q)$ on
$\mathbb{C}^n \times B^r$ with $i^* (\Omega) =
\omega$. Theorem \ref{chap2:sec8:thm3} is
the special case of this for which $p = 0$, $q = 0$. 

\begin{theorem}[Oka - Weil approximation theorem]\label{chap2:sec8:thm4}% them 4
  If $\{f_i (z) \}_{1 \leq i \leq r}$ are entire functions in $z_1,
  z_2, \ldots, z_n$, and if $U = \left\{ z \in \mathbb{C}^n \bigg| |f_i
  (z)| < 1,1 \leq i \leq r \right\}$, then $U$ is a Runge domain. 
\end{theorem}  

\begin{proof}
  With\pageoriginale the notation of Theorem \ref{chap2:sec8:thm3},
  given a holomorphic function $G$ 
  on $\mathbb{C}^n \times B^r$ such that $G \circ i = g$. Let a point in
  $\mathbb{C}^n \times B^r$ be denoted by $(z, w)$. Then $G$ can be
  expanded in a uniformly convergent Taylor series, $G(z, w) = \sum
  a_{\alpha \beta} z^\alpha w^\beta$. 
\end{proof} 

Hence $G(z, w) = \lim \limits_{k \rightarrow \infty} g_k (z, w)$,
uniformly on compact sets where $g_k (z, w)$ are polynomials in $z_1,
\ldots , z_n, w_1,\ldots , w_r$. 
\begin{align*}
  \text{ Hence } \qquad g = G \circ i & = \lim_{k \rightarrow \infty} g_k \circ i \\
  & = \lim_{k \rightarrow \infty} \sum_{\alpha + \beta \leq
    k}a_{\alpha \beta} z^{\alpha} f^{\beta}, \text{ uniformly on
    compact } 
\end{align*} 
subsets of $U$, where $f = (f_1(z), \ldots , f_r (z))$. Thus $g$ can
be approximated by polynomials in $z_1, \ldots, z_n$, $f_1, \ldots,
f_1$. Since, the $f_i$ being entire, the $f_i$ can be approximated by
polynomials in $z_1, \ldots$, $z_n$, so can $g$, and $U$ is a Runge
domain. 

\begin{remark*}
  We have used Grothendieck's lemma for a domain of the form $D = D_1
  \times \cdots \times D_n$; however, for the proof of the Oka-weil
  theorem, it would suffice to use it for \textit{compact} sets $K =
  K_1 \times \cdots \times K_n$. However, the extension theorem of Oka
  (Theorem \ref{chap2:sec8:thm3}) is very important, so that we have given the proof for
  open, rather than compact, sets. 
\end{remark*} 
 
As a corollary to the Oka-Weil theorem we have the following 
\begin{proposition}\label{chap2:sec8:prop2} % prop 2
  A convex open set in $\mathbb{C}^n$ is a Runge domain.
\end{proposition}

\begin{proof}
  It is enough to prove that a bounded convex set in Runge. Consider
  $U$ as a convex set  in $\mathbb{R}^{2n}$. Then for any point $z_0$
  on the boundary, there exists a linear function $l, l(z) = \sum
  \limits^n_1 a_i x_i + \sum \limits^n_1 b_i y_i + c$ such that\pageoriginale $U
  \subset \left\{z \bigg| l (z) < 0 \right\}$ and $l(z_0) = 0$. Let $L(z)$ be a
  linear function, $L(z) = \sum \limits^n_i d_i z_i + e$, $d_i$, $e
  \in \mathbb{C}$ such that $1(z) = Re [ L(z)]$. Hence $U \subset \left\{ z
  \bigg| Re L(z) < 0 \right\}$, while $Re L(z_0) = 0$. Let $K$ be any compact
  subset of $U$. If $z_0 \in \partial U$, we may therefore find a
  linear function $L$ with $Re L(z) < 0$ for $z \in K$, $Re L(z_0) >
  0$ (replace the $L$ constructed above by $L + \delta$ where $\delta
  > 0$ is sufficiently small).Then Re $L(z) > 0$ for $z$ in a
  neighbourhood of $z_0$. Since $\partial U$ is compact, there exist
  finitely many linear functions $L_1, \ldots , L_r$ such that 
\end{proof} 
 
Re $L_i (z) < 0$ for $z \in K$, $Re L_j(z) > 0$ for at least one $j$
if $z \in \partial U$. Hence the set 
$$
\Omega_U = \left\{z \in U \bigg| Re L_i(z) < 0, i = 1, \ldots, r \right\}
$$
contains $K$ and is relatively compact in $U$. Since the set. $\Omega
= \{ z \in \mathbb{C}^n \bigg| Re L_i (z) < 0, i = 1, \ldots , r\}$ is
convex, hence connected and $\Omega \cap U = \Omega_U$ is relatively
compact in $U$, it follows that $\Omega \subset U$. Now 
$$
\Omega_U = \left\{z \in \mathbb{C}^n \bigg| |f_i(z) |< 0, i = 1,
\ldots, r \right\} 
$$
where $f_i(z) = e^{L_i (z)}$, so that $\Omega$ is Runge by theorem
\ref{chap2:sec8:thm4}. Hence any holomorphic function on $U (\supset \Omega)$can be
approximated, uniformly on $K$, by polynomials. Since $K$ is an
arbitrary compact subset of $U$, the proposition is proved. 
 
The\pageoriginale proof of Hartogs' theorem given here is suggested by the proof of
the Runge theorem of Malgrange-Lax (see Chap. III \S\ \ref{chap3:sec10}; also
Malgrange \cite{27}). That of the Oka-well theorem is merely a translation
of Oka's own proof \cite{34} into the language of differential forms. 

\section[Immersions and imbeddings: the theorems of Whitney]{Immersions and 
imbeddings: the theorems of\hfil\break Whitney}\label{chap2:sec9}
 
In what follows $V$, $V'$ are $C^k$ manifolds, $1 \leq k \leq \infty$
countable at infinity. 

\begin{defis*}
  \begin{enumerate}[(1)]
  \item A $C^k$ map $f: V \rightarrow V'$ is called an immersion if
    for every $a \in V$, $f_* $: $T_a (V) \rightarrow T_{f(a)}(W)$ is
    injective. If $f_*: T_a (V) \rightarrow T_{f(a)}(W)$ is
    injective for every a in a subset $E$ of $V$, we say that $f$ is
    regular on $E$. 
  \item A $C^k$ map $f: V \rightarrow V'$ is called an imbedding if
    $f$ is an immersion and $f$ is injective. 
  \item An imbedding (immersion) $f: V \rightarrow V'$ is called a
    closed imbedding (immersion) if $f$ is proper. 
  \end{enumerate}
\end{defis*} 
 
[Note that the set of points where $f$ is regular is open.]
 
Let $\{ U_i\}$ be a locally finite covering of $V$, $U_i$ being
relatively compact coordinate neighbourhoods. Then there exist compact
sets $K_i \subset U_i$ with $\cup K_i = V$. Let $\eta$ be a continuous
function on $V$, $\eta (x)> 0$ for all $x$, and $N$, a non- negative
integer $\leq k$. Given a $C^k$ function $f$ on $V$, another $C^k$
function $g$ is said to approximate $f$ within $\eta$ upto $N^{th}$
order (with respect to the covering $\{ U_i \}$), if  
$$
|D^\alpha f(x) - D^\alpha g(x) | < \eta (x) \text{ for } |\alpha| \leq
N \text{ and } x \in K_i 
$$
and we denote this fact by  $g$ approximates $f$ with respect to $(U_i
, \eta, N)''$. If\pageoriginale $\{U_i \}$ is given, we say that $g$ approximates $f$
within $\eta$ upto order $N$.  

\begin{remark*} 
  If$ \{ U_i \}$, $\{ U'_j \}$ are two locally finite coverings of $V$,
  $K_i \subset U_i$, $K_j \subset U_j$, $K_i$, $K'_j$ compact sets of
  $V$ such that $\cup K_i = \cup K'_j = V$ then there exists a
  positive continuous function $\delta$ such that if $g$ approximates
  $f$ with respect to $(U_i , \eta , N)$ then $g$ approximates $f$
  with respect to $(U'_j , \delta \eta ,N)$. 
\end{remark*}
 
\begin{proof}
  Since $\{U'_j \}$ is locally finite, it suffices to prove that if
  $\{y^j_1, \ldots, y^j_n \}$, $\{x^i_1, \ldots , x^i_n \}$ are
  coordinate in $U'_j, U_i$ respectively, then for any $C^k$ function
  $h$ on $V$, we have 
  $$
  \bigg| D^\alpha_{y^j} h(y) \bigg| \leq C_j \sup_{K'_j \cap U_i \neq
    0} \sum_{|\beta| \leq N} \bigg| D^\beta_{x^i} h(x^{(i)}) \bigg| 
  $$
  for $y$ in $K'_j$ and some constant $C_j$ independent of $h$. This
  is, however, obvious. 
\end{proof} 

This remark implies that if $\{ U_i \}$, $N$ are such that $f$ can be
approximated by functions $g$ in a given class $\mathscr{C}$ with respect to
$(\eta, N)$ for any $\eta$ then the same is true if $\{V)i \}$ is
replaced by any other locally finite covering $\{U'_j \}$ consisting
of relatively compact coordinate neighbourhoods. 

\setcounter{proposition}{0}
\begin{proposition}\label{chap2:sec9:prop1}%pro 1
  If  $f: V^n \rightarrow \mathbb{R}^p$ is a $C^1$ map which is an
  immersion, given any locally finite $\{U_i\}$ as above, there exists
  a positive continuous function $\eta$ on $V^n$ such that if $g$
  approximates $f$ with respect to $(U_i, \eta , 1)$, then $g$ is an
  immersion. 
\end{proposition}
 
\begin{proof}
  The rank $(df) (x) = n =\dim V$ for any $x \in V$. Hence there exists
  a locally finite covering $\{ U_i \}$, compact sets $K_i \subset
  U_i$, $\cup K_i = V$, and positive\pageoriginale numbers $\delta_i<1$ such that if
  $|D^ \alpha f(x) - D^ \alpha g(x)|< \delta_i$ for $x$ in $U_i
  ,|\alpha| \leq 1$, then rank $(dg)(x)=n$. Let $\{ \alpha_i \}$ be a
  partition of unity subordinate to $\{ U_i \}$ and $\delta'_1
  =\inf. \{ \delta_{i_1}, \delta_{1p} \}$, then infimum being over
  those $i_k$ for which $K_i \cap U_{i_k} \neq \phi$. We may then
  take $\eta = \sum \delta '_ i \alpha_i$. 
\end{proof}

\setcounter{lemma}{0}
\begin{lemma}\label{chap2:sec9:lem1} %lemma 1
  If $K$ is a compact set in $V$, $L$ a neighbourhood of $K$ and $f$:
  $V \to \mathbb{R}^p$ is an imbedding, there exists a positive number
  $\delta$ such that for any $C^1$ map $g$: $V \to \mathbb{R}^p$ such that
  $\parallel f-g \parallel ^L_1 < \delta$, $g | K$ is injective.  
\end{lemma}

\begin{proof}
  Since rank $(df)(x)=n$, for any $x$ in $V$, the rank theorem implies
  that for any $x \epsilon V$ there is a relatively compact
  neighbourhood $U$ and a positive number $\delta '$ such that
  $|f(x')-f(x'')|\geq \delta '|x'-x''|$ for $x'$, $x'' \in U$. Let $0
  < \varepsilon < \delta'$ and $\parallel g-f \parallel ^U_1$ is
  sufficiently small, and $h=g-f$, we have 
  $$
  |h(x')-h(x'')|\leq \varepsilon \parallel x'-x'' \parallel \text{ for
  } x',x'' \in U. 
  $$
\end{proof}

Then $|g(x')-g(x'')|\geq (\delta'-\varepsilon) |x'-x''|$, i.e. $g|U$ is
injective. Since $K$ compact, there exists a finite number of points
$x_1, \ldots x_n$ and neighbourhood $U_1, \ldots U_n$, $L \supset \cup
U_i \supset K$, such that if $\parallel g-f \parallel^{U_i}_1$ is
sufficiently small, $g|U_i$ is injective. Hence there exists a
neighbourhood $\Omega$ of the diagonal $\Delta$ in $K \times K$ and a
positive number $\delta_1$ such that if $\parallel g-f \parallel ^L_1
< \delta _1$, we have $g(x)\neq g(y)$ for any $(x,y)\in \Omega
-\Delta$. Again there exists $\delta _2 > 0$ such that for $(x,y)\in
K \times K- \Omega, |f(x)-f(y)|\geq \delta_2$. Let $\delta= \min
(\delta_1 ,\dfrac{\delta_2}{4})$. Then if $\parallel g-f \parallel^L_1
< \delta$, and $(x,y)\epsilon K \times K- \Omega$, $\parallel
g(x)-g(y) \parallel \geq \dfrac{ \delta_2}{2}$ and clearly $g|K$ is
injective. 

We\pageoriginale shall not need the next proposition, but have included it because
it is of interest and is useful in many questions. 

\begin{proposition}\label{chap2:sec9:prop2}% prop 2
  If $f$: $V^n \to \mathbb{R}^p$ is an imbedding and $f$ is locally
  proper, there exists continuous function $\eta$ on $V$ such that if
  $g$ approximates $f$ within $\eta$ upto $1^{st}$ order, then $g$ is
  an imbedding. 
\end{proposition}

\begin{proof}
  It follows from Proposition \ref{chap2:sec9:prop1} that there exists a continuous
  function $\eta_1$, such that if $g$ approximates $f$ within $\eta_1$,
  upto $1^{st}$ order, $g$ is an immersion. Now for $g$ satisfying
  this condition, we shall find a positive continuous function
  $\eta_2$ such that if $g$ approximates $f$ within $\eta_2$ upto
  $1^{st}$ order $g$ is an imbedding. Let $K_m$ be compact sets such
  that $K_m \subset \overset{\circ}{K}_{m+1}$ and $\cup K_m=V$. Define $L_m =
  \overline {K_{m+1}-K_m}$. Then since $f$ is locally proper,
  (therefore proper into and open set $\Omega$ in $\mathbb{R}^p$),
  there exist open sets $U_m $ in $\mathbb{R}^p$ such that
  $f(L_m)\subset U_m $ and $U_m \cap U_{m'}=\phi$ if $m'\geq m+2$. [This
    is because $\{ f(L_m) \}$ is a locally finite system of compact
    sets in $\Omega$ such that $f(L_m)\cap f(L_{m'})=\phi$ if $m' \geq
    m+2$]. Now choose $\delta_m >0$ such that 
  $$
  \parallel f-g \parallel ^{L_m}_1 < \delta_m ~\text{for all}~ m
  \Rightarrow g(L_m)\subset U_m 
  $$
  and $g|L_m \cup L_{m+1}$ is injective. Then if $\eta _2 (x)< \delta
  _m$ for $x$ in $L_m$ and $g$ approximates $f$ within $\eta_2$ upto
  $1^{st}$ order, $g$ is injective. For if $g(x)=g(y)$, $x \in L_m$,
  and $x \neq y$, since $g| L_m \cup L_{m+1}$ is injective $y \in
  L_{m'}$, where $m' \geq m+2$ or $m' \leq m-2$. But $g(L_m)\subset U_m$
  for every $m$ and $U_m \cap U_{m'}=\phi$ if $m' \geq m+2$ or $m'
  \leq m-2$. Hence we have a contradiction i.e. $g$ is injective. 
\end{proof}

The\pageoriginale proposition is false if we drop the assumption that $f$ is locally
proper.  Further even on compact subsets, an approximation to an
injective map (which is not regular) need not be injective. 

\begin{lemma}\label{chap2:sec9:lem2}% lemma 2
  If $\Omega$ is bounded open set in $\mathbb{R}^n$, $f$ a $C^k$ map:
  $\Omega \to \mathbb{R}^p$, $p\geq 2n$, then for any $\varepsilon >0$
  there exists a $C^k$map $g$: $\Omega \to \mathbb{R}^p$ such that
  $\parallel g-f \parallel^{\Omega}_1 < \varepsilon$ and
  $(\dfrac{\delta g}{\delta x_i})_{1 \leq i \leq n}$ are linearly
  independent at any point of $\Omega$. 
\end{lemma}

\begin{proof}
  We may suppose that $f \in C^2$ because of Whitney's approximation
  theorem (Chap. 1 \S\ \ref{chap1:sec5}). Let $f_0=f$. If $f_1, \ldots ,f_r$ are
  $C^k$ maps such that $\parallel f_s-f \parallel^\Omega_1 <
  \varepsilon$ and $\dfrac{\delta f_s}{\delta x_1}, \ldots
  \dfrac{\delta f_s}{\delta x_s}$ are linearly independent on
  $\Omega$, for $0 \leq s \leq r <n$ we shall define $f_{r+1}$ such
  that 
  $$
  \parallel f_{r+1}-f \parallel ^\Omega _1 < \varepsilon \text{ and }
  \frac{\partial f_{r+1} }{\partial x_1} , \ldots, \frac{\partial
    f_{r+1} }{\partial x_{r+1}} 
  $$
  are linearly independent on $\Omega$.
\end{proof}

Let $v_i (x)= \dfrac{\partial f_{r} }{\partial x_1} ~1 \leq i \leq n$.

Define
\begin{gather*}
  \varphi :\mathbb{R}^r \times \Omega \to
  \mathbb{R}^p \text{ by }\\ 
  \varphi (\lambda_1 ,\ldots ,\lambda_r,x)=\sum ^r_1 \lambda_i
  \frac{\partial f_r}{\partial x_i}-v_{r+1} (x). 
\end{gather*}

Now we have $\dim \mathbb{R}^r \times \Omega < p$ and $\varphi \in
C^1$. Hence the image of $\mathbb{R}^r \times \Omega$ by $\varphi$ has
measure zero in $\mathbb{R}^p$. Hence given any $\delta >0$, there
exists $a \in \mathbb{R}^p$ 
such that $\parallel a \parallel <\delta$ and $a \notin \varphi
(\mathbb{R}^r \times \Omega)$. For sufficiently small $\delta$, if we
define $f_{r+1}(x)=f_r(x)+a.x_{r+1}$, $a \in \mathbb{R}^p$ having the
above property,\pageoriginale we have $\dfrac{\partial f_{r+1}}{\partial
  x_i}=\dfrac{\partial f_{r}}{\partial x_i}$ for $i \leq r$ and
$\dfrac{\partial f_{r+1}}{\partial x_{r+1}}=v_{r+1}(x)+a$ which is
linearly independent of $\dfrac{\partial f_{r}}{\partial x_i}$, $1
\leq i \leq r$ since $a \notin \varphi (\mathbb{R}^r \times
\Omega)$. The lemma is proved with $g=f_n$. 

Note that in the above lemma, $g| \Omega$ is an immersion.

\setcounter{theorem}{0}
\begin{theorem}\label{chap2:sec9:thm1}%Thm 1
  If $p \geq 2n$, $f$: $V^n \to \mathbb{R}^p$ is a $C^k$ map if $\eta$
  is positive continuous function on $V$ and $\{ U_i \}$ any locally
  finite covering of $V$ by relatively compact coordinate
  neighbourhoods, then there exists an immersion $g$: $V^n \to
  \mathbb{R}^p$ such that $g$ approximates $f$ with respect to $(U_i,
  \eta,1)$.
\end{theorem}

\begin{proof}
  Because of the remark made at the beginning, we may replace $\{ U_i
  \}$ by any other similar covering. We may therefore suppose that $\{
  U_i \}$ is a locally finite covering of $V$ by relatively compact
  coordinate neighbourhoods such that $U_i$ are diffeomorphic to
  bounded open sets in $\mathbb{R}^n$. Let $K_i$ be compact sets with
  $K_i \subset U_i$ and $\cup K_i=V$. Let $f_0=f$. Assume that $f_1,
  \ldots ,f_m$ are defined and have the following properties 
  \begin{enumerate}[(i)]
  \item $f_m$ approximates $f$ with respect to $(U_i, \eta ,1)$,
  \item $f_m$ is regular on $\bigcup \limits_{i \leq m} K_i$,
  \item $\Supp.(f_{m+1}-f_m)\subset U_{m+1}$.
  \end{enumerate}
\end{proof}

Let $\alpha_m$ be a $C^\infty$ function: $V \to \mathbb{R}$, having
compact support in $U_{m+1}$, while $\alpha_m (x)=1$ for $x$ in a
neighbourhood of $K_{m+1}$. By the lemma proved above, $f_m| U_{m+1}$
has approximation $h_m$ within $\delta_m$ upto $1^{st}$ order such
that $h_m$ is regular on $U_{m 
  +1}$; let $\eta '$ be a positive continuous\pageoriginale function, $\eta' < \eta$
such that, if $g$ approximates $f_m$ within $\eta'$ upto $1^{st}$
order, then $g$ is regular on $\bigcup \limits _{i \leq m} K_i$ 

Define
$$
f_{m+1}=f_m+\alpha_m (h_m-f_m). 
$$
Then clearly if $\delta_m$ is small enough,
\begin{enumerate}[i)]
\item $f_{m+1}$ approximates $f$ within $\eta$ upto the $1^{st}$ order,
\item $f_{m+1}$ is regular on $\bigcup \limits _{i \leq m} K_i$ (since
  it approximates $f_m$ within $\eta'$) and $f_{m+1}=h_{m}$ in
  neighbourhood of $K_{m+1}$ and so regular on $\bigcup \limits _{i
    \leq m} K_i$, 
\item $\Supp (f_{m+1}-f_m) \subset U_{m+1}$.
\end{enumerate}

Hence by induction we have functions $\{ f_m \}_{m \geq 1}$ satisfying
(i), (ii) and (iii) above. We now define $g=\lim \limits_{m \to
  \infty} f_m$. Since $\{ U_i \}$ is locally finite and
$\Supp. (f_{m+1}-f_m)\subset U_{m+1}$, $g$ is well defined and it is
easily verified that $g$ satisfies the conditions stated in the
theorem. 

\begin{theorem}\label{chap2:sec9:thm2} %Thm 2
  Let $f$: $V^n \to \mathbb{R}^p$ be an immersion, $p \geq 2n+1$, $\{
  U_i \}$ a locally finite covering of $V$ by relatively compact
  coordinate neighbourhoods, $K_i$ compact sets, $K_i \subset U_i$,
  $\cup K_i=V$, such that $f|U_i$ is injective and let $\eta$ be a
  positive continuous function on $V$. Then there exists an imbedding
  $g$, approximating $f$ within $\eta$ upto $1^{st}$ order. 
\end{theorem}

\begin{proof}
  We shall define, by induction, regular maps $f_m: V \to
  \mathbb{R}$, $m \geq 1$, 
  \begin{enumerate}[(i)]
  \item $f_m |U_i$ is injective for each $i$,
  \item $f_m$ is injective on $\bigcup \limits_{i \leq m}K_i$,
  \item $f_m$\pageoriginale approximates $f$ within $\eta$ upto $1^{st}$ order and
    $\Supp.(f_{m+1}-f_m) \subset U_{m+1}$. 
  \end{enumerate}
  Let $f_0$ and assume that $f_1, \ldots ,f_n$ are define. Let
  $\alpha_m$ be a $C^k$ function $\alpha_m$: $V\to \mathbb{R}$ with
  compact support in $U_{m+1}$ such that $\alpha_m (x)=1$ in a
  neighbourhood of $K_{m+1}$. Let $\Omega$ be the open subset of $V
  \times V$ defined by $\Omega= \{ (x, y) | \alpha _m (m)\neq \alpha_m
  (y)\}$. Then $\Omega$ is a $C^k$ manifold of dimension $2n$; define
  $\varphi$: $\Omega \to \mathbb{R}^p$ by 
  $$
  \varphi (x, y) = \frac{f_m (y)-f_m (x)}{\alpha_m(x)-\alpha_m(y)}.
  $$
  
  Since $p \geq 2n+1$ and $\varphi \in C^1$, $\varphi (\Omega)$ has
  measure zero in $\mathbb{R}^p$. Hence we can choose $a \in
  \mathbb{R}^p$, arbitrarily near $0$, such that a $\notin \varphi
  (\Omega)$ and if $f_{m+1}(x)=f_m(x)+a \alpha_m (x)$, $f_{m+1}$
  approximates $f_m$ within  a suitable positive function $\eta'$ so
  that $f_{m+1}$ is regular and $f_{m+1}$ approximates $f$ within
  $\eta$. We shall now prove that $f_{m+1}$ thus defined satisfies
  (i), (ii) and (iii). If $f_{m+1}(x) = f_{m+1}(y)$, then 
  \begin{equation}
    a \{ \alpha_m (x) - \alpha_m (y) \} = f_ m (y)-f_m (x)
    \tag{9.1}\label{chap2:sec9:eq9.1} 
  \end{equation}
  and it follows from the choice of a that
  $$
  \alpha_ m (x)-\alpha_m(y)=0 \text{ i.e. } f_m(x)=f_m(y).
  $$
\end{proof}

Hence $f_{m+1}| U_i$ is injective for each if and $f_{m+1}| \bigcup
\limits_{i \leq m} K_i$ is injective. Moreover if $x \in K_{m+1}$ and
$f_{m+1}(x)=f_{m+1}(y)$ for $y \in \bigcup \limits_{i \leq m+1} K_i$
then $y \in U_{m+1}$, [for otherwise $\alpha_m (x)=1$ and $\alpha_m(y)=0$ which
  contradicts\pageoriginale the choice of because of
  (\ref{chap2:sec9:eq9.1})] and since $f_{m+1}| 
U_{m+1}$ is injective $x=y$ i.e. $f_{m+1}| \bigcup \limits_{i \leq
  m+1} K_i$ is injective. Hence we have, by induction, a family $\{
f_m \}$ satisfying (i) (ii) and (iii) and if $g \lim \limits _{m \to
\infty} f_m$, $g$ is seen to have the required properties. 

\begin{lemma}\label{chap2:sec9:lem3}% lemma 3
  If $f$: $V^n \to \mathbb{R}$ is continuous proper map and $g$: $V^n
  \to \mathbb{R}$ is a continuous map which satisfies $|f(x)-g(x)|<1$
  then $g$ is proper. 
\end{lemma}

\begin{proof}
  Clearly $\{ x \in V \big | |g(x) |\leq C \} \subset \{ x \in V \big|
  |f(x)|\leq C+1 \}$ and so is compact for every $C$. 
\end{proof}

\begin{theorem}[Whitney]\label{chap2:sec9:thm3} %Thm 3
  If $V$ is a $C^k$ manifold, $k \geq 1$, of dimension $n$, then there
  exists a closed immersion of $V$ into $\mathbb{R}^{2n}$ and there
  exists a closed imbedding of $V$ into $\mathbb{R}^{2n+1}$. 
\end{theorem}

\begin{proof}
  Let $\{ U_i \}$ be a locally finite covering of $V$ as before, and
  let $\{ K_i \}$ be compact sets, $K_i \subset U_i$ and $\cup K_i
  =V$. Let $\{ \alpha_i \}$ be $C^k$ functions, $supp. \alpha_i
  \subset U_i$ and $\alpha_i (x)=1$ for $x$ in a neighbourhood of
  $K_i$. Define 
  $$
  \varphi:V \to \mathbb{R} \text{ by } \varphi (x)= \sum_{i \geq 1}i \alpha_i (x).
  $$
  clearly $\varphi$ is $C^k$. Moreover if $x \in K_m$, we have
  $\varphi (x) \geq m \alpha_m (x)=m$. Hence $\varphi^{-1}[0, m]
  \subset \bigcup \limits_{i \leq m+1} K_i$ and so is compact. Hence
  $\varphi$ is proper. define $\varphi '$: $V \to
  \mathbb{R}^{2n}$ by $\varphi '(x)=(\varphi(x), 0, \ldots
  ,0)$. Choose $\eta_1$, a positive continuous function, with $0 <
  \eta_1(x)<1$. By the lemma above if $f$ approximates $\varphi'$
  within $\eta_1$, it is proper. Then by Theorem \ref{chap2:sec9:thm1}, there exists an
  immersion $f$ which approximates $\varphi'$ within\pageoriginale $\eta _1/2$ and
  this proves the first part of the theorem.
\end{proof}

Let $f$: $V \to \mathbb{R}^{2n}$ be a proper immersion. Choose a
locally finite covering $\{ U_i\}$ of $V$ such that $f| U_i$  is
injective and there exists compact sets $\{ K_i \}$, $K_i \subset
U_i$, $\cup K_i =V$. Define $F$: $V \to \mathbb{R}^{2n+1}$ by
$F(x)=(f_1(x),.,f_{2n}(x),0)$. Then by Theorem \ref{chap2:sec9:thm2}, there exists an
imbedding $g$, approximating $F$ within $\eta_1/2$ upto $1^{st}$
order. Hence $g$ approximates $\varphi'$ within $\eta_1$ and hence is
proper i.e. $g$: $V \to \mathbb{R}^{2n+1}$ is a closed imbedding. 

We add a note about th embedding of real analytic manifolds. Let $V$
be real analytic, and suppose that $V$ admits a proper real analytic
imbedding $i$ in $\mathbb{R}^p$ for some $p$. Then if $f$ is $C^
\infty$ on $V$, there exists $F \in C^ \infty$ on $\mathbb{R}^p$ with
$F \circ i = f$. If follows easily from this and Whitney's approximation
theorem (Chap. I, \S \ref{chap1:sec5}) that for any locally finite $\{ U_i \}$, $\eta
>0$ and $N>0$, and $C^ \infty$ function $f$ can be approximated a real
analytic function $g$ with respect to $\{ U_i, \eta, N \}$ (we have
only to approximate $F$ by $G$ and set $g = G \circ i$). Hence it follows,
from Whitney's Theorem \ref{chap2:sec9:thm3} and
Proposition \ref{chap2:sec9:prop2} that such a manifold has a
closed immersion in $\mathbb{R}^{2n}$, and a closed imbedding in
$\mathbb{R}^{2n+1}$. These results immersion in $\mathbb{R}^{2n}$, and
a closed imbedding in $\mathbb{R}^{2n+1}$. These results have been
completed by H.~Grauert \cite{13} by showing that any real analytic manifold
countable at $\infty$ can be analytically imbedded in $\mathbb{R}^p$
for some $p$. It follows from our remarks above that we have the
following theorem. 

\begin{theorem*}
  Any real analytic manifold of dimension $n$ which is countable at
  $\infty$ admits a real analytic closed immersion in $\mathbb{R}^{2n}$, and a
  real analytic closed imbedding in $\mathbb{R}^{2n+1}$. 
\end{theorem*}

The\pageoriginale problem of holomorphic imbeddings of complex manifolds is of a
different nature. Only so called  \textit{ Stein manifolds} (see
Chap. III for definition) can be imbedded as closed submanifolds of
$\mathbb{C}^p$. (See R. Narasimha \cite{31} and E. Bishop \cite{3}). 

Whitney \cite{49} has proved that if $V$ is a $C^k$ manifold $(k \geq 1)$
of dimension $n (\geq 2)$ and $g$: $V \to \mathbb{R}^{2n-1}$ is any
continuous map, then there is a $C^k$ immersion $f$: $V \to
\mathbb{R}^{2n-1}$ approximating $f$. From our remarks above it
follows that any real \textit{differentiable manifold} ($C^k$ or
analytic) \textit{admits a} a closed immersion in $
\mathbb{R}^{2n-1}$. (This is obviously false for $n=1$; the circle
cannot be immersed in the line.) He has further proved \cite{48} that any
$C^k$ manifold of dimension $n$ can be imbedded in $
\mathbb{R}^{2n}$. In particular, compact have \textit{closed}
imbeddings in $\mathbb{R}^{2n}$. These results have been completed by
M.W.~Hirsch \cite{15} by proving that \textit{a non-compact manifold of
  dimension $n$ has an imbedding in $ \mathbb{R}^{2n}$ (hence, a
  closed imbedding in $\mathbb{R}^{2n-1}$)}.  

These results are best possible.

\medskip
\noindent
{\bf Note.} The proof of the imbedding theorem (Theorem \ref{chap2:sec9:thm3} above) given
here is essentially that of Whitney \cite{50}. 
