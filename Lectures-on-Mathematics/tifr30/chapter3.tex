\chapter{Descent methods}\label{chap3}%Chap III

\section{Galoisian descent}\label{chap3:sec1}\pageoriginale%Sec 1 
 
 Let $A$ be a Krull ring and let $K$ be its quotient field. Let $G$ be
 a finite group of automorphisms of $A$. Let $A'$ denote the ring of
 invariants of $A$ with respect to $G$ and let $K'$ be the quotient
 field of $A'$. Then $A'= A \cap K'$, so that $A'$ is a Krull
 ring. Since $\prod\limits_{s\in G} (x-s(x)) = 0$, $x \in A$, we see that $A$
 is integral over $A'$. Thus  we have the homomorphism $j : D (A')
 \rightarrow D(A)$ and $\bar{j} : C(A') \rightarrow C(A)$ (see Chapter
 \ref{chap1} \S \ref{chap1:sec6}). We are interested in
 computing Ker $(\bar{j})$. Let $D_1 = 
 j^{-1} (F(A))$. Then Ker $(\bar{j}) = D_1/_{F(A)}$. Let $S$ be a
 system of  generators of $G$. Let $\underline{d} \in D_1$, with  $j
 (\underline{d}) = (a)$, $a \in K$. 
    
\noindent
The divisor $j(\underline{d})$ is invariant under $G$, i.e. $(s(a)) =
(a)$, $s \in G$. Hence $s(a) /a \in U$, the group of units of
$A$. Let $h$ denote the homomorphism $h : K^* \rightarrow (K^* )^S$
given by $x \rightsquigarrow (s(x){/_{x}})_{s \in S}$. Then $h(a) \in
h(K^*) \cap U^S$, Now if $a = a' u$, $a' \in K$, $u \in U$, then
$s(a)/{a}= s(a')/{a'} \cdot s(u)/{u'}$. Thus $h(a)$ is determined
uniquely modulo $h(U)$, and we therefore have a homomorphism $\varphi
: D_1 \rightarrow (h (K^*) \cap U^S)_{h(U)}$ with $\underline{d}
\rightsquigarrow h(a) ~ (\mod h(U))$, where $a (\underline{d}) = (a)$,
$a \in K$. 
   
\begin{theorem}\label{chap3:thm1.1} %the 1.1
The mapping $\varphi$ induces a monomorphism $\theta : $ Ker
  $(\bar{j}) \rightarrow \dfrac{(h(K^*) \cap U^S)}{h(U)}$. Furthermore,
  if no prime divisor of $A$ is ramified over $A'$, then $\theta$ is an
  isomorphism. 
\end{theorem}  
      
\begin{proof}
Let\pageoriginale $\underline{d} \in D_1$. Then
$\varphi(\underline{d}) = 0 
\Leftrightarrow h(a) = h(u)$, $u \in U \Leftrightarrow s (a) /{a} =
s(u){/_{u}}$, for all $s \in S$. 
    \begin{align*}
& \Leftrightarrow s(\frac{a}{u}) = \frac{a}{u} ~~\text { for all} ~~s
      \in G \Leftrightarrow \frac{a}{u} = a' \in K'\\ 
& \Leftrightarrow j (d) = (a)_A = (\frac{a}{u})_A = (a')_A =
      j((a')_{A'}). 
    \end{align*}    
    \end{proof}

    \noindent
    But, since $j$ is injective, we have $\underline{d} = (a')_{A'} $
    i.e. $\Ker (\varphi) = F(A')$. Hence $\theta$ is a monomorphism. 
    
    Now assume that no prime divisor of $A$ is ramified over $A'$. Let
    $\alpha \in (h(K^*) \cap U^S) \Big/ _{h(U)}$, $\alpha = h(a) ~
    (\mod h(U))$. Since $h(K^*) = h(A^*)$, we may assume that $a \in
    A$. Since $s(a)/a \in U$, for $s \in S$, the divisor  
\begin{enumerate}
\item[(a)] is invariant under $G$. Now, by hypothesis for any prime
  divisor $\mathscr{Y}' \in D(A') $, we have $j (\mathscr{Y}' ) =
  \mathscr{Y}_1 + \cdots + \mathscr{Y}_g$, where the $\mathscr{Y}_i$
  form a complete set of prime divisor lying over $\mathscr{Y}'$. Further the
  $\mathscr{Y}_i$ are conjugate to each other. Since the divisor (a)
  is invariant under $G$, the prime divisors which are conjugate to
  each other occur with the same coefficient in (a) so that (a) is
  the sum of divisors of form $j (\mathscr{Y}')$, $\mathscr{Y}' \in
  P(A')$. Hence $\theta$ is surjective and therefore an isomorphism. 
\end{enumerate}  
      
\setcounter{rem}{0}
\begin{rem}
For $S = G$ the group $(h(K^*) \cap (U)^G) \big/ h(U)$ is the
{\em cohomology} group $H^1(G, U)$: in fact a system
$(\dfrac{s(x)}{x})_{s \in G}$ for $x \in K^*$ is the most general
cocycle of $G$ in $K^*$ (since $H^1 (G_1 K^*) = 0$, as is well known),
whence  $h(K^*) \cap (U)^G = Z^1 (G, U)$; on the other hand $h(U)$ is
obviously the group $B^1 (G, U)$ of coboundaries. The preceding
theorem may also be proved by the following cohomological argument. As
usual, if $G$ operates on a set $E$, we denote by $E^G$ the set of
invariant elements of $E$;\pageoriginale we recall that $E^G = H^o (G,
E)$. Now, 
since $H^1 (G, K^*) = 0$, the exact sequence 
$$
0 \to U \to K^* \to F(A) \to 0 
$$
gives the exact cohomology sequence
$$
0 \to U^G \to (K^*)^G \to F(A)^G \to H^1 (G, U) \to 0. 
$$
\end{rem}
  
  \noindent
  On the other hand, since $U^G$ is the group of units in $A' = A^G$,
  we have 
  $$
  0 \to U^G \to (K^*)^G \to F(A') \to 0
  $$
  and therefore,
  $$
  0 \to F(A') \to F(A)^G \to H^1 (G, U) \to 0
  $$
  
 \noindent
  In other words, $H^1 (G, U) = $ (invariant principal divisors of
  $A$) / (divisors of $A$ induced by principal divisors of $A'$). This
  gives immediately a monomorphism $\theta: \ker (\bar{j}) \to H^1
  (G, U)$. If $A$ is divisorially unramified over $A'$, one sees, as
  in the theorem, that every invariant divisor of $A$ comes from $A'$,
  thus $\theta$ is surjective in this case. 
  
\begin{rem}\label{chap3:rem2} % rem 2
Suppose $G$ is a finite cyclic group generated by an element say
$s$. Then we may take $S = \{s \}$. By Hilbert's Theorem 90, the
group $h(K^*)$ is precisely the group of elements of norm 1. Thus
$(h(K^*) \cap U) / h(U)$ is the group of units of norm 1 modulo
$h(U)$. 
 \end{rem}  
      
 \begin{rem} % rem 3
The hypothesis of ramification is essential in the above theorem. For
instance let $A = \mathbb{Z} [i]$,  $i^2 = -1$, $G = \{1, \sigma\}$,
$\sigma (i) = - i$. Then $A' = \mathbb{Z}$, $C(A') = C(A) = 0$. Hence
$\Ker (\bar{j}) = 0$. However, $U \cap h(K^*) = \{1, -1, i, -i\}$, $h(U)
= \{1, -1\}$. Thus $(h(K^*) \cap U) \big/ h(U) \approx \mathbb{Z}
/_{(2)}$.  
 \end{rem}    
    
\noindent
We\pageoriginale note that the prime number 2 is ramified in $A$. 
   

\medskip    
\noindent\textbf{Examples: Polynomial rings.}
\begin{enumerate}[1.]
\item Let $k$ be a fied and $A = k \big[x_1, \ldots , x_d \big]$, the
  ring of polynomials in $d$ variables, $d \ge 2$. Let $n$ be an
  integer with $(n, p) = 1, p$ being the characteristic of $k$ and
  let $k$ contain a  primitive $n^{\rm th}$ root of unity $w$. Consider the
  automorphism $s : A \to A$ with $x_i \rightsquigarrow w x_i$, $1 \le i
  \le d$ and let $G$ be the cyclic group of order $n$ generated by
  $s$. Then the ring of invariants $A'$ is generated by the monomials
  of degree $n$ in the $x_i$; geometrically this is the $n$-tuple model
  of the projective space. Set $F_i (X) = X^n - x^n_i$. Now any
  ramified prime divisor of $A$ must contain $F'_i (x_i) =
  nx^{n-1}_i$. Thus there is no divisorial ramification in $A$. Here
  $U = k^*$ and the group of units of norm 1 is the group of $n^{\rm th}$
  roots of unity. Further $h(U) = \{1\}$, $(h(K^*) \cap U) \big / h(U)
  \approx \dfrac{\mathbb{Z}}{(n)}$, by Remark \ref{chap3:rem2}. Since
  $A$ is factorial, by Theorem \ref{chap3:prop3.1}, we have
  $C(A') \approx \mathbb{Z}  / (n)$. 

\item Let $k, w, n,$ be as in (1) and $A = k [x, y]$. Let $s$ be the
  $k$-automorphism of $A$ defined by $x \rightsquigarrow wx$, $y
  \rightsquigarrow w^{-1}y$. The ring of $G$-invariants $A' = k \big[
    x^n, y^n, xy \big]$, i.e. $A'$ is the affince coordinate ring of
  the surface  $Z^n = XY$. Again as in (1) there is no divisorial
  ramification, $U = k^*$ and $C(A') \approx \mathbb{Z}/ (n)$. 

\item Let $k$ be a field and $A = k \big[X_1, \ldots X_n \big]$. Let
  $A_n$ denote the alternating group. Now $A_n$ acts on $A$. If the
  characteristic of $k$ is $\neq 2$, then the ring of $A_n$-invariants
  is $A' = k \big [s_1, \ldots , s_n, \triangle \big]$ where $s_1
  \cdots s_n$ denote the elementary symmetric functions and
  $\triangle = \prod\limits_{1 < j)} (x_i - x_j)$.\pageoriginale If
  characteristic 
  $k = 2$, then $\triangle$ is also symmetric and $A' = k \big[ s_1,
    \ldots , s_n, \alpha \big]$, where $\alpha = \dfrac{1}{2}
  (\prod\limits_{i < j} (x_i - x_j) + \prod\limits_{i < j} (x_i +
  x_j))$.
\end{enumerate}

As the coefficients of $\prod\limits_{i < j}(x_i - x_j) +
  \prod\limits_{i < j} (x_i + x_j)$ are divisible by 2, the element
  $\alpha$ has a meaning in characteristic 2. Further there is no
  divisorial ramification in $A$ over $A'$. For the only divisorial
  ramifications of $A$ over $k \big[ s_1 , \ldots , s_n \big]$ are
  those prime divisors which contain $F' (x_i) = \prod\limits_{j \neq
    i}(x_j - x_i)$, where $F(X) = \prod (X- x_j)$. Since $\triangle =
  \prod\limits_{i < j} (x_i -x_j) \in A'$ (in characteristic 2,
  $\triangle$ is in fact in $k \big[s_1, \ldots , s_n \big]$), there
  is no divisorial ramification in $A$ over $A'$. Hence $C(A') \approx
  H^1 (A_n, U)$, by the remark following Theorem \ref{chap3:thm1.1}.
      But $U = k^*$ and $A_n$ acts trivially on $k^*$. Hence $C(A') \approx
H^1 (A_n, U)$ is the group of homomorphisms of $A_n$ into $k^*$. Thus
if $n \ge 5$, $A_n$ is simple and therefore $C(A') = 0$ i.e. $A'$ is
factorial. The only non-trivial cases we have to consider are, $n =
3,4$. For $n = 3$, $A_n$ is the cyclic group of order 3. Hence
$C(A') = 0$ if $k$ does not contain cube roots of unity, otherwise
$C(A') \approx \mathbb{Z}/(3)$. We now consider the case $n = 4$. We
have $[ A_4, A_4] = \big\{1, (2 \; 2) (3 \; 4), (1 \; 4) (2 \; 3), (1
\; 3) (2 \; 4) \big\}$ and  $A_4 \big/ \big[A_4, A_4 \big] \approx
\mathbb{Z}/  
(3)$. Now the group of homomorphisms of $A_4$ into $k^{\ast}$ is isomorphic
to the group of homomorphisms of $A_4 \big/ \big[A_4, A_4 \big]$
into $k^*$. Hence, is in the case $n = 3$, $C(A') = 0$ if $k$ does not
contain cube roots of unity; otherwise $C(A')_\approx \mathbb{Z}/
(2)$. 

           
\begin{example*} %exam
Power series rings. We first prove the following lemma.
\end{example*} 
       
\setcounter{lemma}{1}
\begin{lemma}%Lem 1.2
Let\pageoriginale $A$ be a local domain, $\mathscr{M}$ its maximal
ideal. Let $s$ be 
an automorphism of $A$ of order $n$, ($n$, char $(A/\mathscr{M})) =
1$. Let $U$ denote the group of units of $A$ and $h$ the mapping $K^*
\to K^*$ with $x \rightsquigarrow s(x) / x$, $K$ being the quotient
field of $A$. Then $(1 +  \mathscr{M}) \cap h(K^*) \subset h(U)$
(i.e. $\im (H'(G, I  + \mathscr{M}) \to H'(G, U)) = 0$, where $G$ is 
the group generated by $s$). 
    \end{lemma}  
      
\begin{proof}
Let $u \in 1  +  \mathscr{M}$, $u = s(x)/_{x}$, $x \in K^*$. Then $u$
has norm 1, i.e. $N(u) = u^{1+s  +  \cdots + s^{n-1}} = 1$. Set
$v = 1 +  u + u^{1+s}+  \cdots + u^{1+ \cdots + s^{n-2}}$. Then $v
\equiv n.1 (\mod \mathscr{M})$. Since $n$ is prime to the
characteristic of $A/ \mathscr{M}$, it follows that $v$ is a
unit. Further we have $s (v) = 1+u^s + u^{s+s^2}+ \cdots u^{s+1
  \cdots + s^{n-1}}$ and $u s(v) = v$ i.e. $u = s(v^{-1}) \big/v^{-1}$
$\in h(U)$ and the lemma is proved.  
    \end{proof}    

    
    In the examples (1) and (2) of polynomial rings we replace the
    rings $A = K \big[x_1, \ldots , x_d \big]$ and $A = K [x, y]$
    respectively by $A = K \big[[  x_1, \ldots , x_d]\big]$ and $A =
    K \big[[x, y ] \big]$. Since in $A$ we have $U \big/(1+
    \mathscr{M}) \approx k^*$, we obtain the same results as in the
    case of ring of polynomials, in view of the above lemma. 
    
\setcounter{prop}{2}
\begin{prop}\label{chap3:prop1.3} % prop 1.3
Let $A$ be a local ring, $\mathscr{M}$ its maximal ideal. Let $G$
  be a finite group of automorphisms of $A$, acting trivially on $k =
  A/\mathscr{M}$. Further, assume that there are no non-trivial
  homomorphisms of $G$ into $k^*$ and that $(Card (G), \text{ Char }
  (k)) = 1$. Then $H^1 ~ (G, U) = 0, U $ being the group of units of
  $A$. In particular, if $A$ is factorial, so is $A'$. 
    \end{prop} 
       
    \begin{proof}
Let $(u_s)_{s \in G}$, $u_s \in U$, be a 1-cocycle of $G$ with values
in $U$. Then $u_{ss'} = s(u_{s'}). u_s$. Reducing modulo
$\mathscr{M}$, we get $\bar{u}_{ss'} = \bar{u}_{s'} \cdot \bar{u}_s$,
since $G$ acts trivially on $k$. We have made the hypothesis that
there are no non-trivial homomorphisms of $G$ into $k^*$. Hence $u_s
\in 1 + \mathscr{M}$, $s \in G$. Set\pageoriginale $y = \sum\limits_{t
  \in G} 
u_t$. Then $y = $ Card $(G)\cdot 1$ $(\mod \mathscr{M})$. Thus $y \in
U$. Now $s (y) = \sum\limits_{ t\in G} s(u_t) = \sum\limits_{t}
\dfrac{u_{st}}{u_s} = \dfrac{1}{u_s} y $ i.e. $u_s = s(y^{-1}) \big/
y^{-1}$. Hence $H^1 (G, U) = 0$. 
\end{proof}   
     
\begin{coro*} %coro
Let $A = k \big[[x_1, \ldots , x_n ]\big]$. Let $k$ be of
characteristic $p > n$ or 0. Let $A_n$ be the alternating group on
$n$ symbols. Then for $n \ge 5$, $H^1 (G, U) = 0$, i.e. the ring of
invariants $A'$ is factorial. 
    \end{coro*}    
    
    For further information about the invariants of the alternating
    group we refer to Appendix 1. 
    

\section{The Purely inseparable case}\label{chap3:sec2}%Sec 2
 
 Let $A$ be a Krull ring of characteristic $p \neq 0$, and let $K$ be
 its quotient field. Let $\triangle$ be a derivation of $K$ such that
 $\triangle (A) \subset A$. Set $K' = \Ker (\triangle)$ and $A' = A
 \cap K'$. Then $A'$ is again  a Krull ring and $A^p \subset A'$, $K^p
 \subset K'$. In particular, $A$ is integral over $A'$. Hence the
 mapping $j : D(A') \to D(A)$ of the group of divisors goes down to a
 mapping $\bar{j} : C(A') \to C(A)$ of the corresponding divisor class
 groups. We are interested in computing $\Ker (\bar{j})$. Set $D_1 =
 j^{-1} (F(A))$, so that $\Ker (\bar{j}) = D_1 / F(A')$. Let
 $\underline{d} \in D_1$, and $ j(\underline{d}) = (a)$, $a \in
 K^*$. From the definition of $j$ it follows that $e_{\underline{p}}$
 divides $v_{\underline{p}}(a)$, where $\underline{p}$ is a prime
 divisor of $A$, $v_{\underline{p}}$ the corresponding valuation and
 $e_{\underline{p}}$ the  ramification index of
 $v_{\underline{p}}$. Hence there exists an $a' \in K'^*$ such that
 $v_{\underline{p}}(a) = v_{\underline{p}} (a')$, i.e. $a = a' . u$,
 $u$ being a unit in $A_{\underline{p}}$. Thus $\triangle a/a =
 \triangle a'/a' + \triangle u/u = \dfrac{\triangle u}{u}$. Since
 $\triangle (A_{\underline{p}}) \subset A_p$, it follows that $\triangle
 a/a \in A_{\underline{p}}$,\pageoriginale for all prime divisors
 $\underline{p}$ of 
 $A$, i.e. $\triangle a/a \in A$. We shall call $a x \in K$, a
 \textit{logarithmic derivative} if $x = \triangle t/t $ for some $t
 \in K^*$. The set of all logarithmic derivative is an additive
 subgroup of $K$. Set $\mathscr{L} = \big\{\triangle t/t \big |
 \triangle t / t \in A,  t \in K^{\ast} \big\}$. Let $U$ denote, as before,
 the group of units of $A$ and set $\mathscr{L}' = \big\{\triangle u/
 u \big| u \in U \big\}$. Now $\mathscr{L} ' \subset
 \mathscr{L}$. For $a \underline{d} \in D_i$ with $j (\underline{d}) =
 (a)$, $a \in K^*$, $\triangle a/a \in \mathscr{L}$ is uniquely determined
 modulo $\mathscr{L}'$. Let $\varphi$ denote the homomorphism  $: D_1
 \to \mathscr{L}/ \mathscr{L}'$, $\underline{d} \rightsquigarrow
 \triangle a/a(\mod \mathscr{L}')$ if $j (\underline{d}) = (a)$. Now $\varphi
 (\underline{d}) = 0 \Leftrightarrow \triangle a/a = \triangle u/ u$,
 for $u \in U \Leftrightarrow \triangle (a/ u) = 0$ i.e. $a / u = a'
 \in K' \Leftrightarrow (a)_A = (a')_A$. But $j ((a')_{A'}) =
 (a')_{A'}$ and $j $ is injective. Hence $\underline{d} = (a')_{A'}
 $. Thus $\Ker (\varphi) = F(A')$. We have proved the first assertion
 of the following theorem. 
 
\begin{theorem}\label{chap3:thm2.1}%%%% 2.1
\begin{enumerate}[(a)]
\item We have a canonical monomorphism $\varphi : \Ker (\bar{j})
  \to \mathscr{L} \big / \mathscr{L}'$. 

\item If $[K : K'] = p$ and if $\triangle (A)$ is not contained
  in any prime ideal of height 1 of $A$, then $\bar{\varphi} $ is an
  isomorphism. 
\end{enumerate}
\end{theorem}
       
\begin{proof}
To complete the proof of the theorem, we have only to show that
$\bar{\varphi}$ is surjective under the hypothesis of (b). As $K /
K'$ is a purely inseparable extension, every prime divisor of $A'$
uniquely extends to a prime divisor of $A$. Thus a divisor
$\underline{d} = \sum\limits_{\underline{p} \in P(A)}
n_{\underline{p}} \underline{p} \in \Iim (j)$ if and only if
$e_{\underline{p}} \big / n_{\underline{p}}$ where $e_{\underline{p}}$
denotes the ramification index of $\underline{p}$. Since $
A^p \subset A'$, it follows that for any prime division $\underline{p}$ of $A$,
$e_{\underline{p}} = 1$ or $p$. Let $a \in K^*$ be such that $\triangle
a / a \in A$. It is sufficient to prove that for a prime divisor
$\underline{p}$ of $A$, if $n = v_p (a)$ is not a multiple of $p$,
then $e_p = 1$. Let $t$ be a uniformising parameter of $v_p$. Let $a =
ut^n$, $u$ being a unit in $A_{\underline{p}}$. Then $\triangle u/u + n
~ \triangle t/t = \triangle a/a \in A_{\underline{p}}$. Hence,
$\triangle t/t \in A_{\underline{p}}$, i.e. $\triangle$\pageoriginale
induces a 
derivation $\bar{\triangle}$ on the residue class field $k =
A_{\underline{p}} \big/ t A_{\underline{p}}$. By hypothesis, since
$\triangle A \nsubset \underline{p}$, we have $\bar{\triangle} \neq
0$. Let $k'$ be the residue class field of $\underline{p} \cap
A'$. Then $k ' \subseteq \Ker (\bar{\triangle})
\underset{\neq}{\subset} k$. Thus $f 
= [k : k' ] \neq 1$. Since $[K : K'] = p$, and $k^p \subset k'$, we
have $f = p$. Now the inequality $e_{\underline{p}} f \le [K : K '] =
p$ gives $e_p = 1$. The proof of Theorem \ref{chap3:thm2.1} is complete.  
\end{proof}   
     

\section{Formulae concerning derivations}\label{chap3:sec3}%Sec 3
    
    Let $K$ be a field of characteristic $p \neq 0$ and let $D : K \to
    K$ be a derivation. Let $tD$ denote the derivation $x
    \rightsquigarrow t$. $D(x)$. We note that $D^p$, the $p^{\rm th}$
    iterate of $D$, is again a derivation. 
    
    \begin{prop}\label{chap3:prop3.1} % prop 3.1
Let  $D : K \to K$ be a derivation of $K$(char $K = p \neq
  0$). Assume that $[K : K'] = p$. Then 
\begin{enumerate}[(a)]
\item $ D^p = a D$, $a \in K' = \Ker D$.

\item If $A$ is a Krull ring with quotient field $K$ such that
  $D(A) \subset A$, that $a \in A' = K' \cap A$. 
\end{enumerate}
    \end{prop} 
      
\begin{proof}
\begin{enumerate}[(a)]
\item By hypothesis $K = K' (z)$, $z^p \in K'$. For any $K'$-
  derivation $\triangle$ of $K$, $\triangle = \dfrac{\triangle
    z}{D z} D$. In particular $D^p = a D$ for $a  \in
  K$. Hence $D_a^{p+1} = D(D^p a) = D(a \cdot Da) = (Da)^2 + aD^2 a$. On
  the other hand, $Da^{p+1} = D^p (Da) = a D^2 a$. Hence $Da = 0 $
  i.e. $a \in K'$. 

\item Since $A = \bigcap\limits_{p \in P (A)}
  A_{\underline{p}}$ and $D(A_{\underline{p}}) \subset
  A_{\underline{p}}$, we have only to deal with the case when $A$ is a
  discrete valuation ring. Let $v$ denote the corresponding
  valuation. Let $t \in A$, with $v(t) = 1$. We have $D^p t =
  a \cdot Dt$. If $v(Dt) = 0$, then $a = \dfrac{1}{Dt}$, $D^p t \in A$. Assume
  $v(Dt) > 0$. Then for $x \in A$, we have $v(Dx) \ge v(x)$. For is $x
  = u t^n$ with $v(u) = 0$. In particular, $v(D^p t) \ge v(Dt)$
  i.e. $a \in A$.  
\end{enumerate}
\end{proof}  
      
\begin{prop}\label{chap3:prop3.2} %prop 3.2
Let\pageoriginale $D : K \to K$ be a derivation of, $K$(char $K = p
\neq 
  0$). Let $K' = \Ker D$ and $[K : K'] = p$. An element $t \in K$ is a
  logarthmic derivative (i.e. there exists an $x \in K$ such that $t =
  D x/x)$ if and only if
$$
 D^{p-1} (t) - \text{ at } + t^p = 0,
$$
where $D^p = aD$.
    \end{prop}  
      
    \begin{proof}
We state first the following formula of Hochschild
(Trans. A.M.S. 79(1955), 477-489). 
    \end{proof}    
    
    \noindent
Let $K$ be a field of characteristic $p \neq 0$ and $D$ a derivation
of $K$. Then 
   $$
   (t D)^p = t^p D^p + (tD)^{p -1} (t). D
   $$
($t \in K$, $t D$ denotes the derivation $x \rightsquigarrow
t. Dx)$. We have to prove the proposition only in the case when $t
\neq 0$. 
    
    Let now $t$ be a logarthmic derivative, say $t = Dx/x$. Set
    $\triangle = \dfrac{1}{t} D$. Then by Hochschild's formula, we
    have, 
    \begin{align*}
D^p & = (t \triangle)^p = t^p \triangle^p + (t \triangle)^{p -1}
(t) \triangle.\\ 
& = t^p \triangle^p + D^{p-1} (t) \cdot  \triangle = aD. 
    \end{align*}    
    
    \noindent
    But $\triangle^n  x =x$, for $n \ge 1$. Hence $a. Dx = t^p x +
    D^{p-1} (t) x$, i.e. 
    $$
    t^p - \text{ at } + Dt^{p-1}(t) = 0.  
    $$
    
    Conversely assume that $t$ is such that $D^{p-1} t - \text{ at} + t^p =
    0$. Set $\triangle = \dfrac{1}{t}. D$. Again by Hoschschild's
    formula, we have $D^p = (t \triangle)^p = t^p \triangle^p + D^{p-1}
    (t). \triangle = aD = a.t \triangle$, i.e. $a.t \triangle = t^p
    \triangle^p + 
    (\text{at} - t^p)\triangle $ i.e. $t^p (\triangle^p - \triangle) = 0$,
    i.e. $\triangle^p - \triangle = 0$, i.e. $(\triangle - (p-1) I)
    \cdots (\triangle - 2 I) (\triangle - I) = 0$, where $I$ is the
    identity mapping of $K$ into $K$. Choose\pageoriginale $y \in K$
    with $y_1 = 
    \triangle y \neq 0$ and set $y_2 = (\triangle - I) y_1, \ldots ,
    y_p = (\triangle - (p-1)I) y_{p-1} (= 0)$. Then, there exists
    a $j$ such that $y_{j - 1} \neq 0$ and $y_j = (\triangle - j I)
    y_{j - 1} = 0$. 
    
    \noindent
    Hence
    $$
    \triangle y_{j - 1}  = jy_{j - 1}, ~~ \text{ i.e.} ~~\triangle
    y_{j-1} \big/ y_{j-1} = j \in \mathbb{F}_p^*, \mathbb{F}_p 
    $$
    being the prime field of characteristic $p$. Let $n$ be the
    inverse of $j$ modulo $p$. Set $x = y^n_{j - 1}$. Then
    $\dfrac{\triangle x}{x} =  n \dfrac{y_{j-1}}{y_{j-1}} = nj = 1$,
    i.e. $\triangle x = x $ i.e. $t = Dx/x$. 
    


\section{Examples: Polynomial rings}\label{chap3:sec4}%Sec 4
                                                                             
Let $k$ be a factorial ring of characteristic $p \neq 0$. Set $A =
k [x, y]$. Let $D$ be a $k$-derivation of $A$ and $A' = \Ker
(D)$. The group of units $U$ of $A$ is the group of units of
$k$. Hence here $\mathscr{L}' = 0$. Since $A$ is factorial, we by
Theorem \ref{chap3:thm2.1}, an injection of $C(A') = \Ker (\bar{j})$ into
$\mathscr{L}$. (We recall that $\mathscr{L}$ is the group of
logarthmic derivatives contained in $A$ and that $\mathscr{L}'$ is
the group of logarthmic derivatives of units.) We shall now
consider certain special $k$-derivations of $A$.  
\begin{enumerate}[(a)]
\item \textbf{The Surface {\boldmath$Z^p = XY$}.} Consider the
  derivation $D$ of 
  $A $ $k[x, y]$ with $Dx = x$ and $Dy = - y$. Then $k [x^p, y^p, xy]
  \subset A' = \Ker (D)$. Let $L, K, K'$ denote the quotient fields of
  $k, A, A'$ respectively. Now $L [x^p, y^p, xy]$ is the coordinate
  ring of the affine surface $Z^p = XY$. Since the surface $Z^p = XY$
  has only an isolated singularity (at the origin), it is normal.  But
  $k [x^p, y^p, xy] = L [x^p y^p, xy] \cap k [x, y]$. Hence $k [x^p,
    y^p, xy]$ is normal. Since $A'$ is integral over $k [x^p, y^p,
    xy]$ and has the same quotient field as $k [x^p, y^p, xy]$, we
  have $A' = k [x^p, y^p, xy]$. We note\pageoriginale that the
  hypothesis of Theorem \ref{chap3:thm2.1} (b) is satisfied
  here. Hence $C(A') = \mathscr{L}$. Now 
  $\mathscr{L} = \big\{DP/P \big| P\in K, DP/P \in A \big\}$. For $P
  \in A$, we have $d^o (DP) \le d^o (P)$. Hence $DP/P \in \mathscr{L}$
  if and only if $DP/P \in k$. The formula  $D(x^a y^b) = (a-b) x^a
  y^b$ shows that $\mathscr{L} = \mathbb{F}_p$, the prime field of
  characteristic $p$. Hence $C (A') \approx \mathbb{Z}/(p)$. 

\item \textbf {The surface {\boldmath$Z^p = X^i + Y^j $}}. Again we
  take $A = k 
  [x, y ]$, $k$ a factorial ring of characteristic $p \neq 0$. Let $D$
  be the $k$-derivation of $A$ given by $Dx = j y^{j - 1}$, $Dy = - i x^{i
    - 1}$, where $i$, $j$ are positive integers prime to $p$. Let $K,
  K', L$ denote the quotient fields of $A, A' = \Ker (D)$ and $k$
  respectively. We have $k[ x^p, y^p, x^i + y^j ] \subset A'$ and $[
    L (x^p, y^p, x^i +  y^j ) : K] = p$. Hence $K' = L (x^p, y^p, x^i
  + y^j)$. Now $L [x^p, y^p, x^i + y^j]$ is the coordinate ring of
  the affine surface $Z^p = X^i + Y^j$ which is normal since it has
  only an isolated singularity (at the origin). Hence $L [ x^p, y^p,
    x^i + y^j]$ is integrally closed. But $k [  x^p, y^p, x^i + y^j] =
  L [ x^p, y^p, x^i + y^j] \cap k [x, y]$. Hence $k [ x^p, y^p, x^i +
    y^j]$ is integrally closed. Since $A'$ is integral over $k [ x^p,
    y^p, x^i + y^j]$, we have $A' = k [ x^p, y^p, x^i + y^j]$. We
  remark that our $D$ satisfies the hypothesis of Theorem \ref{chap3:thm2.1}
  (b). Hence $C(A') = \big\{DP / P \big| P \in K, DP /P \in A \big
  \}$. We shall now compute $\mathscr{L}$. We attach weights $j$ and 
  $i$ to $x$ and $y$ respectively. By Proposition \ref{chap3:prop3.1},
  we have $D^p 
  = a D$ with $a \in A'$. It is easily checked that if $G$ is an
  isobaric polynomial of weight $w$, then $DG$ is isobaric weight
  $w+ij - i - j$ and therefore $D^p G$ is isobaric of weight $w + p(ij
  - i - j)$. Now $D^p x = aDx$. Comparing the weights we see that $a$ is
  isobaric of weight $(p -1) (ij - i - j)$. 
    \end{enumerate}    
    
    \noindent
    Let\pageoriginale $F$ be a polynomial which is a logarthmic
    derivative. Let 
    $F_\alpha$ of weight $\alpha$ (respectively $F_\beta$ of weight
    $\beta$) be the component of smallest (respectively largest)
    weight of $F$. By Proposition \ref{chap3:prop3.2}, $F$ is a logarthmic
    derivative if anly only if $D^{p - 1} F - a F = - F^p$. Comparing
    the weights of the components with smallest and largest weights on
    both sides, we get weight $(D^{p -1} F_\alpha - a F_\alpha ) \le$
    weight $(F^p_\alpha)$ and weight $(D^{p-1} F_\beta - a F_\beta$)
    $\ge $ weight $(F^p_\beta)$. That is $p \alpha \ge \alpha + (p -
    1) (ij - i - j)$, $p \beta \le \beta + (p - 1) (ij - i - j)$. Hence
    $ij - i - j \le \alpha \le \beta \le ij - i - j $ i.e. $\alpha =
    \beta = ij - i - j$. Hence $F$ must be isobaric of weight $ij - i
    - j$. Set $d = (i, j)$, $i = dr$, $j = ds$. Thus, the monomials that
    can occur in $F$ are of the form $x^\lambda y^\mu$, $\lambda  j +
    \mu i = ij - i - j $ i.e. $\lambda s + \mu r = drs - r -s $,
    i.e. $(\lambda + 1) s = (ds - \mu - 1) r$. Since $(r, s) = 1$,
    $\lambda + 1$ is a multiple of $r$. Thus the smallest value of
    $\lambda$ admissible is $r - 1$, the corresponding $\mu $ being
    $(d-1)s-1$. Thus $F$ is necessarily of the form $F =
    \sum\limits_{n = 1}^{d- 1} b_n x^{nr - 1} y ^{(d-n)s-1}$. If $d =
    1$, then $\mathscr{L} = 0$ and $A'$ is factorial. If $d > 1$, the
    coefficients of $D^{p-1} F - a F $ will be linear forms in $b_1,
    \ldots , b_{d-1}$ and those of $-F^p$ are $p^{\rm th}$ powers of $b_1,
    \ldots , b_{d-1}$. Thus $F$ is a logarthmic derivative if and only
    if $b^p_n = L_n (b)$, $L'_{n'} (b) = 0$, $1\le n \le d- 1$, $1 \le n'
    \le t$, where $L_n(b)$, $L'_{n'} (b)$ are linear forms occuring as
    the coefficient of $D^{p-1} F - a F$. 
    
    $L'_{n'} (b)$ indicates the ones which do not occur in $ -
    F^p$. The hypersurfaces $b^p_n = L_n (b)$ intersect at a finite
    number of points in the profective space $P^{d-1}$ and, by
    Bezout's theorem, the number of such points in the algebraic
    closure of $L$ is atmost $p^{d-1} $. As $\mathscr{L}$ is an
    additive subgroup of $A$, $\mathscr{L}$ is a $p$-group of type $(p,
    \ldots, p)$ of order $p^f$, $f \le d - 1$.\pageoriginale Hence we
    have proved the  
    
    \setcounter{theorem}{0}
    \begin{theorem}\label{chap3:thm4.1} % the 4.1
 Let $k$ be a factorial ring of characteristic $p \neq 0$, and let
  $i , j$ be two positive integers prime to $p$ and $d = (i, j)$. Then
  the group $C(A')$ of divisor classes of $A' = k [X, Y,Z]$ with $Z^p
  = X^i + Y^j $ is a finite group of type $(p, \ldots p)$ of order
  $p^f$ with $f \le d- 1$. In particular $A'$ is factorial if $i$ and
  $j$ are coprime. 
    \end{theorem} 
    
    We can say more about $C(A')$ in the case $p = 2$. Let $k$ be
    of characteristic 2. Then $D^2 = 0$, i.e. $a= 0$. The equation
    for the logarthmic derivative then becomes $DF = F^2$. As above
    $F$ is of the form $F = \sum\limits_{n = 1}^{d - 1} b_n x^{nr - 1}
    y^{(d - n) s - 1}$. Here $i$, $j$, $r$, $s$, $d$ are all odd integers. If
    $n$ is odd, then $D (b_n x^{nr - 1} y^{(d-n)s-1} = b_n x^{nr -
      1+dr-1} y^{(d-n) s - 2}$.
The corresponding term in $D^2 F$ is
    $b^2_m x^{2 mr - 2} y^{2(d-m)s-2}$, where $2m = n + d = 2q + 1+d
    (n = 2q +1)$, $b_n = b^2_m$. Set $d = 2c - 1$. Then $m = q+c$. Thus
    $b_{2q+1} = b^2_{q+c}$. On the other hand let $n$ be even, say $n
    = 2q$. The $D(b_n x^{nr -1} y^{(d-n)s-1}) = b_n x^{nr - 2}
    y^{(d-n)s-1+ds-1}$. The corresponding term in $D^2 F$ is $b^2_m
    x^{2m r-2} y^{2(d-m)s-2}$, where $b_n = b^2_m$ and $nr - 2 = 2
    mr-2 $ i.e. $2m = n = 2q$. Hence $b_{2q} = b^2_q$. Thus $F$ is a
    logarthmic derivative if and only if the equations $b_{2q +1} =
    b^2_{q+c}$ and $b_{2q} = b^2_q$, $d+1 = 2c$, are satisfied. 
    
    Consider the permutation $\prod $ of $(1, 2, \ldots , d - 1)$
    given by, $\prod (2q) = q$, $1 \le q \le c-1$, $\prod (2q - 1) = q+c$,
    $0 \le q \leq c-2$. Now the equations for the logarthmic derivative can be
    written as $b_q = b^2_{\Pi (q)}$, $1 \le q \le 2c - 2$. Let $U_1,
    \ldots, U_l$ be the orbits of the group generated by $\prod $
    and let Card $(U_i) = u(i) $.\pageoriginale Then $u(1)+ \cdots  +
    u(l) = 2c - 2 
    $. If $U_e = (q_1, \ldots q_{u(e)})$, then the equations $b_{q_1}
    = b^2_{\Pi (q_1)}, \ldots , b_{q_{u(e)}} = b^2_{\Pi (q_{u(e)})}$
    are equivalent to $b^{2^{u(e)}}_{q_1} = b_{q_1}$. Thus the
    solutions of $b^{2u (e)} = b$ give rise to solution of $b_m =
    b^2_{\Pi (m)}$, where $m \in U_l$. But the solutions of
    $b^{2^{u(e)}}= b $ in $k$ is the group $\mathbb{F}(2^{u(e)}) \cap
    k$, where $\mathbb{F} (2^{u(e)})$ is the field consisting of
    $2^{u(e)}$ elements. Hence the group $\mathscr{L}$ of logarthmic
    derivatives is isomorphic to $\prod\limits_{e = 1}^1 (\mathbb{F}
    (2^{u(e)} ) \cap k)$. Hence we have proved the following 
    
\begin{theorem}\label{chap3:thm4.2} % the 4.2
Let $k$ be a factorial ring of characteristic 2 and let $i$, $j$ 
  be odd integers and $d = (i, j)$. Let $A' = k [X, Y, Z]$, $Z^2 = X^i
  + Y^j$. The group $C(A')$ is of the type $(2, \ldots , 2)$ and of order
  $2^u$ with $u \le d - 1$. If $k$ contains the algebraic closure of
  the prime field, then the order of $C(A') $ is $2^{d-1}$. 
    \end{theorem} 
       
\begin{remark*} 
It would be interesting to know if the above theorem is true for
arbitrary non-zero characteristics. We remark that for $p = 3$ and for
the surfaces $Z^3 = X^2 + Y^4$, $Z^3 = X^4 + Y^8$, the analogue of the
above result can be checked. 
\end{remark*}   
 
\section{Examples: Power series rings}\label{chap3:sec5}
    
    Let $A$ be a Krull ring and $D:A \to A$, a derivation of $A$. Let
    $\mathscr{L}$ denote the group of logarthmic derivatives contained
    in $A$ and  $\mathscr{L}'$ the group of logarthmic derivatives of
    units of $A$. Set $\underline{q}= A \cdot D(A)$. We have, $\mathscr{L}'
    \subset \underline{q} \cap \mathscr{L}$. We prove the other
    inclusion in a particular case. 

\setcounter{lemma}{0}
    \begin{lemma}\label{chap3:lem5.1} % lem 5.1
Let\pageoriginale $A$ be a factorial ring of characteristic 2 and $D:
A \to A$ be 
a derivation of $A$ satisfying $D^2= aD$, with $a \in \Ker
(D)$. Assume that there exist $x, y \in \text{ Rad  }(A)$ such that
$\underline{q} = (Dx, Dy)$. Then $\mathscr{L}' = \mathscr{L} \cap
\underline{q}$. 
    \end{lemma}  
      

\begin{proof}
Let $t \in \mathscr{L} \cap \underline{q}$, say $t = cDx + d D y$.  If
$r = (Dx, Dy)$. By considering the derivation $\dfrac{1}{r} D$, we
may assume that $Dx$ and $Dy$ are relatively prime. Since $t \in
\mathscr{L}$, by Proposition \ref{chap3:prop3.2}, we have $Dt + at +t^2 =
0$. Substituting $t = cDx + d D y$ in this equation, we get 
$$
Dx (Dc + c^2 Dx) = Dy (Dd + d^2 Dy). 
$$
 \end{proof}    
    
\noindent 
Since $Dx$ and $Dy$ are relatively prime, there is an $\alpha \in A$
such that  
    $$
    Dc + c^2 Dx = \alpha  Dy, Dd + d^2 Dy = \alpha Dx.
    $$
    
    \noindent
    Set $u = 1 + cx + dy + (cd + \alpha ) xy$. The element $u$ is a
    unit in $A$. $A$ straight forward computation shows that $Du =
    tu$. The proof of the lemma is complete. 
\begin{enumerate}[(a)]
\item \textbf{The surface {\boldmath$Z^2 = XY$} in characteristic
  2.} Let $k$  
  be a regular factorial ring. Then, by Theorem \ref{chap3:thm2.1}, $A
  = k \big [ [x, 
      y ]\big]$ is factorial. Let $D$ be the k-derivation of $A$ given
  by $Dx = x, Dy = y$. Then as in  $\S 4, A' = \ker (D) = k \big
  [[x^2, y^2, xy ]\big] = k \big [ [ X, Y Z]\big], Z^2 = XY$. Here,
  $\underline{q} = (x, y)$ and $[K : K'] = 2$. Hence, by Theorem
  \ref{chap3:thm2.1},
  $C(A') \approx \mathscr{L}/ \mathscr{L}'$. By Lemma
  \ref{chap3:lem5.1}, we have 
  $\mathscr{L}/ \mathscr{L}' = \mathscr{L}\big / (\mathscr{L} \cap
  \underline{q}= (\mathscr{L} + \underline{q}\big /
  \underline{q}$. This, and the formula $D (x^a y^b) = (a-b) x^a y^b$
  show that $(\mathscr{L} + \underline{q}) / \underline{q} \approx
  \mathbb{F}_2 = \mathbb{Z}/ (2)$. 

\item \textbf{The Surface {\boldmath$Z^2 = X^{2 i +1}+ Y^{2 j + 1}$} in
    characteristic 2.}\pageoriginale 
   Let $k$ be a regular factorial ring and $A = k \big [[x, y
      ]\big]$. Let $D$ be the $k$-derivation defined by $Dx = y^{2j}, Dy
    = x^{2i}$. Then $A' = k \big [[x^2, y^2, x^{2i + 1}+ y^{2j +1}]
      \big] = k \big [[X, Y, Z]\big], Z^2 = X^{2i +1} + Y^{2j +1}$. We
    have $\underline{q} = A D(A) = (x^{2i}, y^{2j})$ and $[K : K'] =
    2$. Hence $C(A') \approx \mathscr{L}/ \mathscr{L}'$. Since $D^2=
    0$, an element $F \in A$ is a logarthmic derivative if and only if
    $DF = F^2$. We assign the weights $2j +1$ and $2 i +1$ to $x$ and
    $y$ respectively. For an $F \in A$ with $F = \sum\limits_{l \ge q}
    F_l$, where $F_l$ is an  isobaric polynomial of weight $l$, $F_q
    \neq 0$, we call $q$ the order of $F$, $0(F) = q$. As in Theorem
    \ref{chap3:thm4.1}, $D$ elevates the weight of an isobaric
    polynomial by $4 ij - 
    1$, Hence, if $F \in \mathscr{L}$ and $0(F) = q$, then $0(F)=2q =
    0(DF) = q + 4 ij - 1$. Hence $q \ge 4ij - 1$. 
    \end{enumerate}

    Let $\mathscr{L}_q = \big\{F \big | F \in \mathscr{L}, 0(F) \ge q
    \big \}$. Now $\big \{\mathscr{L}_q \big\}_{q \ge 4ij - 1}$
    filters $\mathscr{L}$ and $\mathscr{L}'_q = \mathscr{L}_q \cap
    \mathscr{L}'$ filters $\mathscr{L}'$. Hence $C(A') = \mathscr{L}/
    \mathscr{L}'$ is filtered by $C_q = (\mathscr{L}_q +
    \mathscr{L}')/_{\mathscr{L}' \approx} \mathscr{L}_{q/ \mathscr{L}'
      q}$. In view of Lemma \ref{chap3:lem5.1}, we have $\mathscr{L}_q
    =     \mathscr{L}' _q$ for $q$ large, ie.e $C_q = 0$, for $q$ 
    large. Since the $C_q$ are vector spaces over $\mathbb{F}_2$, the
    extension problem here is trivial. Hence $C(A') \approx
    \sum\limits_{q \ge 4ij -1} C_q / C_{q+1}$. Since $0 (x^{2i}) = 2i
    (2j +1) > 4ij$ and $0(y^{2j}) = 2j (2i +1) > 4ij$, we have
    $\mathscr{L}' = \mathscr{L} \cap \underline{q} \subset
    \mathscr{L}_{4ij}$. Therefore $C_{4ij - 1}\big / C_{4ij}=
    \mathscr{L}_{4 ij - 1}\big / \mathscr{L}_{4ij}$. By Theorem
    \ref{chap3:thm4.2},
    $\mathscr{L}_{4ij -1} \big / \mathscr{L}_{4ij}$ is a finite group
    of type $(2, \ldots , 2) $ of order $2^f$, with $f\le d - 1, d =
    (2 i +1, 2 j +1)$. We now determine $C_q \big / C_{q+1}$, for $q
    \ge 4ij$. Let $A^{(q)}$ denote the k-free module\pageoriginale generated by
    monomials of weight $q$. Let $\varphi_q : \mathscr{L}_q \to
    A^{(q)}$ be the homomorphism given by $\varphi_q (F) = $ component
    of $F$ of weight $q, F \in \mathscr{L}_q$. Then $\ker (\varphi_q)
    = \mathscr{L}_{q+1}$. We shall now prove 
        \begin{align*}
& \varphi_q (\mathscr{L}_q) = A^{(q)} \cap A', q \ge 4ij, \tag{*}\\ 
& \varphi_q (\mathscr{L}'_q) = A^{(q)} \cap A' \cap \underline{q}, q
          \ge 4ij.\tag{**} 
    \end{align*}    
    
    \noindent
    Note that $(**)$ is a consequence of $(*)$ and the fact that 
    $$
    \mathscr{L}'_q = \mathscr{L}_q \cap \underline{q}.
    $$

\begin{proofif}
Let $F = F_q +  F_{q+1} + \cdots \in \mathscr{L}_q, F_q$ being of
weight $q$. Since $DF = F^2$, and weight $DF_q = q + 4ij - 1 < 2q$, we
have $DF = 0$, i.e. $\varphi_q (F) = F_q \in A^{(q) } \cap
A'$. Conversely, let $F_q \in A^{(q)} \cap A'$. We have  to find $F_n,
n \ge q, F_n$  isobaric polynomial weight $n$, such that $F =
\sum\limits_{n \ge q} F _n \in \mathscr{L}_q$, i.e. $DF = F^2$. Hence
we have to determine $F_n$ such that $DF_n = 0$, if $n$ is even or $n
+ 4 ij -1 < 2q$ and $DF_n = F^2m$, if $2m = n + 4 ij -1 (m < n)$. Thus
$F_n$ have to determined by `integrating' the equation $DF_n = G^2$,
where $G = 0$ or an isobaric polynomial of weight $q$. Because of the
additivity of the derivation, we have only to handle the case  $G =
x^\alpha y^\beta, \alpha (2 j +1) + \beta (2 i +1) \ge q \ge 4ij$. In
this case, either $\alpha \ge i$ or $\beta \ge j$. If $\alpha \geq i$, we
take $F_n = x^{2 (\alpha - i)} y^{2 \beta +1}$ and if $\beta \ge j$,
we take $F_n = x^{2 \alpha + i } y^{2(\beta - j)}$. Thus proves $(*)$
and hence also $(**)$. This gives $C_q \big / C_{q+1} \approx (A^{(q)}
\cap A')  \big/  (A^{(q)} \cap A' \cap \underline{q})$, $q \ge 4 ij$. Hence
$C_q\big / C_{q+1}, q \ge 4ij$ is a $k$-free module of finite rank, say
$n(q)$. Hence  $C(A') \approx C_{4ij - 1} \big/ C_{4 ij} \oplus
C_{4ij}$,\pageoriginale where $C_{4ij}$ is a $k$-free module of finite
rank $N (i, j) = 
\sum\limits_{q 
  \ge 4ij} n(q)$. We now determine the integer $N(i, j)$. We  observe
that in $A$, the ideal $\underline{q}$ admits a supplement generated by
the monomials $x^q y^b$ such that $a < 2 i, b <  2 j$. Since $x^{2i
  +1} + y^{2i +1} \in q$, in $A'$, the ideal $\underline{q} \cap A'$
admits a supplement generated by the monomials $x^{2a} y^{2b}$ such
that $2a < 2i, 2b < 2 j$. Thus $N (i, j)$ is equal to the number of
monomials $x^{2a} y^{2b}$, with $0 \le 2 \alpha < 2i, 0 \le 2 \beta <
2 j$ and weight of $x^{2a} y^{2b} \ge 4ij$. Hence we have the 
    \end{proofif}  


\setcounter{theorem}{1}
    \begin{theorem}\label{chap3:thm5.2} % the 5.2
Let $k$ be a factorial ring of characteristic 2, and $i$, $j$ two
  integers with $(2 i + 1, 2j+1) = d$. Let $A' = k \big[[X, Y,
      Z]\big]$, where $Z^2 = X^{2i +1} + Y^{2 j +1}$. Then the divisor
  class group $C(A') \approx H \oplus G$, where, $H$ is a group of
  type $(2, \ldots , 2)$ of order $2^f$, $f \le d - 1$; (if $k$
  contains the algebraic closure of the prime field $\mathbb{F}_2$,
  then $H$ is of order $2^{d-1}$); further $G \approx k^{N (i, j)}$,
  where $N (i, j)$ is the number of pairs of integers $(a, b)$ with $0
  \le a < i$, $0 \le b < j$ and $(2j +1)a + (2i +1) b \ge 2ij$. 
    \end{theorem}  
      
\begin{remarks*} % rem
\begin{enumerate}[1)]
\item The function $N (i, j) \sim ij / 2$ 

\item  $N (i, j) = 0$ if and only if the pair $(a, b) = (i-1, ~ j - 1)
  $ does not satisfy the inequality $(2 j +1)a + (2i+1)b \ge 2ij$,
  i.e. if $(i, j)$ satisfies the inequality $2 ij - i - j < 2$. This
  is satisfied only by the pairs $(1, 1)$, $(1, 2)$ and $(1, 3)$,
  barring the trivial cases $i=0$ or $j = 0$. Hence, upto a
  permutation the only factorial ring we obtain is, except for the
  trivial cases, $k \big[[X, Y, Z]\big]$, $Z^2 = X^3 + Y^5$. In view
  of Theorem \ref{chap3:thm4.1} and Theorem
  \ref{chap3:thm5.2},\pageoriginale the 
  pairs $(2 i +1, 2j +1) \neq (3,5), (5, 3)$, for which $2i +1$ and
  $2j+1$ are relatively prime, provide examples of factorial rings
  whose completions are not factorial.  
\end{enumerate}
    \end{remarks*}    

\medskip
\noindent{\textbf{(c) Power series ring.}}
    Let $k$ be a regular factorial ring of characteristic 2. Let $A
    = k [x, y]$ (resp. $k \big[[x, y ]\big]$) and $R = A
    \big[[T]\big]$. We define a $k$-derivation $D : R \to R$ by $Dx  = 
    y^{2j}$, $Dy = x^{2i}$, $DT = 0$. Then $\Ker D = A' \big[[T]\big]$,
    where $A' = k[x^2, y^2, x^{2i+1} + y^{2j +1}]$ (Resp. $k\big
    [[x^2, y^2, x^{2i+1}+ y^{2i}] \big ]$). For a Krull ring $B$, let
    $\mathscr{L} (B)$ and $\mathscr{L}'(B)$  denote the group of
    logarthmic derivatives in $B$ and the  group of logarthmic
    derivatives of the units of $B$, respectively. We will compute
    $C(R) = \mathscr{L} (R) \big/\mathscr{L}'(R)$. An $F \in R$ is
    in $\mathscr{L} (R)$ if and only if $DF = F^2$ (since $D^2  =
    0)$. Let $F = \sum\limits_n a_n T^n$. Then $F \in \mathscr{L}(R)$
    if and  only if $Da_o = a^2_o$, $Da_{2n+1} = 0$, $Da_{2n} =
    a^2_n$. Since by Lemma \ref{chap3:lem5.1}, $\mathscr{L}' (R) =
    \mathscr{L}(R) 
    \cap \underline{q}$, where $\underline{q} = (Dx, Dy)$, we have $F
    \in \mathscr{L}' (R)$ if and only if $Da_o = a^2_o$, $Da_{2n +1} =
    0$, $Da_{2n} = a^2_n $ and $a_n \in (Dx, Dy)$. Thus $F \in
    \mathscr{L} (R)$ $(\resp . \mathscr{L}' (R))$ implies $a_o \in
    \mathscr{L}(A)$  $\resp. a_o \in \mathscr{L}' (A))$. Further,
    $\mathscr{L} (R) \big /\mathscr{L}'(R) \approx \mathscr{L}(A)
    \big/ \mathscr{L}'(A) \oplus \dfrac{\mathscr{L}(R) \cap 
      TR}{\mathscr{L}' (R) \cap TR}$. As before we assign weights $2j
    +1$, $2i+1$ to $x$ and $y$ respectively. Let $q(n) = 0(a_n)$. Now if $F
    \in \mathscr{L}(R)$, then $Da_n = a^2_n$. Hence $q(2n) + q \le
    2q(n)$, where $q = 4ij - 1$. That is , $q(2n) - q \le
    2(q(n)-q)$. By induction, we get $q(2^r n) - q \le 2^r (q(n) - q)$
    for $r \ge 1$. Since $q(2^r n) \ge 0$, we conclude that $q(n) \ge
    q$. A computation similar to that in Theorem \ref{chap3:thm5.2}
    shows that the 
    `integration' of $Da_{2n} = a^2_n$ is possible. Further if $a_n
    \in \underline{q} = (Dx, Dy)$, then $a_{2n}$ can be chosen in
    $\underline{q}$. 

    \noindent
    Let $A^{(q)}$\pageoriginale be the set of elements of order $\ge
    q$. In 
    computing $F \in \mathscr{L} (R)$, each integration introduces an
    `arbitrary element' of $A' \cap A^{(q)}$. In computing $F \in
    \mathscr{L}' (R)$, each integration introduces an arbitrary
    constant of $A' \cap A^{(q)} \cap \underline{q}$. Hence
    $(\mathscr{L} (R) \cap TR) \big/ \mathscr{L}' (R) \cap TR$ is the
    product of countably many copies of $V = (A' \cap A^{(q)} \big /
    (A' \cap A^{(q)} \cap \underline{q})$. As in the last example, $V$
    is a $k$-free module of rank equal to the number $N (i, j) $ of
    pairs $(a, b)$ with $0 \le a < i$, $0 \le b < j$, $(2j + 1) 2a + (2i
    +1) 2b \ge q = 4ij - 1$ and this inequality is equivalent to $(2j
    +1) a+ (2i +1)b \ge 2ij$. Hence we have the   
    
\begin{theorem} % the 5.3
Let $k$ be a factorial ring of characteristic 2, and $i$, $j$
  two integers. Let $A' = k [X, Y,Z]$ (or $k \big[[X, Y, Z]\big]$)
  with $Z^2 = X^{2i+1} + Y^{2j +1}$. Then $C(A' \big[[T]\big]) \big/
  C(A') \approx (k \big[[T]\big])^{N(i, j)}$ where  $N(i, j)$ is the
  number of pairs $(a, b)$ with $0 \le a < i$, $0 \le b < j$ and $(2 j
  +1) a+ (2i +1) b \ge 2ij$. 
\end{theorem}    
    
\begin{remarks*} % remks
\begin{enumerate}[(1)]
\item Take $A' = k \big[x^2, y^2, x^{2 i +1} + y^{2 j +1}\big]$ with
  $(2i +1, 2j +1)= 1$ and $N (i, j) > 0$. Then $A'$ is factorial, but  $A'
  \big[[T]\big]$ is not. (We have thus to exclude only $Z^2 = X^3 +
  Y^5$ and trivial cases.) 

\item Let $A'$ be the complete local ring $A' = k \big[[X, Y,
    Z]\big]$, $Z^2 = X^{2 i +1} + Y^{2 j +1}$. Then $A'$ and $A'
  \big[[T]\big]$ are simultaneously factorial or simultaneously 
  non-factorial. 

\item In general, the mapping $C(A') \to C(A' \big[[T]\big])$ is not
  surjective. 
\end{enumerate}
\end{remarks*}   
  
\chapter*{Appendix}\label{app}

\begin{center}
\Large{\textbf{The alternating group operating on a power series ring}}
\end{center}\pageoriginale
    
    We have seen (Chap. \ref{chap3}, \S  \ref{chap3:sec1}) that the
    ring $A'$ of invariants 
    of the alternating group $A_n$ operating on the polynomial ring $k
    [x_1, \ldots , x_n]$ is factorial for $n \ge 5$. Let us study the
    analogous question for the power series ring $A = k \big [[x_1,
        \ldots , x_n]\big]$; let $U$ be the group of units in $A$, $m$
    the maximal ideal of $A$, and $A'$ the ring of invariants of $A_n$
    (operating by permutations of the variables). We recall
    (Chap. \ref{chap3}, \S  \ref{chap3:sec1}) that $C(A')
    \approx H^1 (A_n, U)$ since $A$ is 
    divisorially unramified over $A'$.  We have already seen
    (Chap. \ref{chap3}, \S  \ref{chap3:sec1}, Corollary to Proposition
    \ref{chap3:prop1.3}) 
    that $H^1 (A_n 
    , U) = 0$ if the characteristic $p$ of $k$ is prime to the order
    of $A_n, $ i.e. if $p>n$. Thus what we are going to do concerns
    only fields of ``small'' characteristic. 
    
    \begin{theorem*} % the 
Suppose that  $p \neq 2$, $3$. Then with the notation as above,
  $A'$ is factorial for $n \ge 5$. For $n = 3$, $4$, $C(A')$ is isomorphic
  to the group of cubic roots of unity contained in $k$. 
     \end{theorem*}    
     
     Our statement means that $C(A') \approx H^1 (A_n , k^*) = Hom
     (A_n, k^*)$. In view of the exact sequence $0 \to 1 + \mathscr{M}
     \to U \to k^* \to 0$, we have only to prove that $H^1 (A_n, 1+
     \mathscr{M}) = 0$. For this it is sufficient to prove that 
     \begin{equation*}
H^1 (A_n, (1+_{ \mathscr{M} }s )\big/ (1+_{\mathscr{M}} s+1) = 0 \text{
  for every } j \ge 1. \tag{1}\label{c3:eq1}  
      \end{equation*}      
      
      In fact, given a cocycle $(x_s) $ in $1+ \mathscr{M} (s \in  A_n,
      x_s \in 1+ \mathscr{M})$, it is a\pageoriginale coboundary
      modulo $1+ 
      \mathscr{M}^2$, i.e. there exists $y_1 \in 1+ \mathscr{M}$ such
      that $x_s \equiv s(y_1) y^{-1}_1 \mod 1+ \mathscr{M}^2$. We set
      $x_{2, s} = x_s y_1 s(y_1)^{-1}$; now $x_{2, s}$ is a cocycle
      in $1 +  \mathscr{M}^2$, and therefore a coboundary modulo $1+
      \mathscr{M}^3$. By induction we find elements $y_1, \ldots ,
      y_{j1} \cdots (y_j \in 1+ \mathscr{M}^j)$ and $x_{js} \in 1+
      \mathscr{M}^j$ such that $x_{1, s} = x_s$ and $x_{j +1,s} =
      x_{j,s } y_j s (y^{-1}_j)$. The product $\prod\limits_{j =
        1}^\infty y_j$ converges since $A$ is \textit{complete};
      calling $y$ its value, we have $x_s y s(y^{-1}) = 1$ for every
      $s \in A_n$, which proves that $(x_s)$ is a coboundary. 
      
      In order to prove (\ref{c3:eq1}), we notice that the multiplicative group
      $(1+ \mathscr{M}^j) \big/ (1+ \mathscr{M}^{j+1})$ is isomorphic
      to the additive group $\mathscr{M}^j \big / \mathscr{M}^{j +1}$,
      i.e. to the vector space $W_j$ of homogeneous polynomials of
      degree $j$. Our theorem is thus a consequence of the following
      lemma: 
      
      \setcounter{lem}{0}
      \begin{lem}\label{app:lem1}%%% 1
Let $S_n (\resp . A_n)$ operate on $k \big [x_1, \ldots , x_n
    \big]$ by permutations of the variables, and let $W_j$ be the
  vector space of homogeneous polynomials of degree $j$. Then  
\begin{enumerate} [a)]
\item $H^1 (S_n, W_j) = 0 $ if the characteristic $p$ is $\neq
  2$; 

\item $H^1 (A_n, W_j) = 0$ if $p \neq 2, 3$
\end{enumerate}
       \end{lem}       
       
       We consider a monomial $x = x^{j(1)}_1 \cdots x^{j (n)}_1$ of
       degree $j$ and its transforms by $S_n (\resp. A_n)$. These
       monomials span a stable subspace $V$ of $W_j$, and $W_j$ is a
       direct sum of such stable subspaces $V$. We need only prove
       that $H^1 (S_n, V) = 0 (\resp. H^1 (A_n, V) = 0)$. Now the
       distict transforms $x_{\theta}$ of the monomial $x$ are indexed by $G/H
       (G = S_n $ or $A_n)$, where\pageoriginale $H$ is the stability
       group of $x$; we have $s(x_\theta) = x_{s \theta}$ for $a \in
       G$. We are going to prove, in a moment, that  
       \begin{equation*}
H^1 (G, V) = \text{ Hom } (H, k) ~ (G = S_n \text { or } A_n)
\tag{2}\label{c3:eq2} 
        \end{equation*}        
        
        Let us first see how (\ref{c3:eq2}) implies Lemma
        \ref{app:lem1}. The stability 
        subgroup $H$ is the set of all $s$ in $S_n$ (or $A_n$) such
        that $\prod\limits_i x^{j (i)}_i = x = s(x) = \prod\limits_i
        x^{j (i)}_{s(i)} = \prod\limits_i x^{j (s^{-1} (i))} $,
        i.e. such that $j (s^{-1}(i)) = j (i)$ for  every $i$. Thus $H$
        is the set of all $s$ in $S_n$ or $A_n$ which, for every
        exponent $r$, leave the set of indices $s^{-1} (\{r\})$
        globally invariant. Denote by $n(r)$ the cardinality of $s^{-1}
        (\{r\})$ (i.e. the number of variables $x_i$  having exponent
        $r$ in the monomial $x$). In the case of $S_n$, $H$ is the
        direct product of the groups $S_{n (r)}$; since a nontrivial
        factor group of $S_{n(r)}$ is necessarily cyclic of order 2,
        we have Hom$(H, K)= 0$ in characteristic $\neq 2$; hence we
        get $a)$ in Lemma \ref{app:lem1}. In the case of $A_n$, $H$ is
        the subgroup of $\prod\limits_r S_{n(r)}$ consisting of the
        elements $(s_r)$ such that the number of indices for which
        $s_r \in S_{n(r)} - A_{n(r)}$ is even; thus $H $ contains $H^1
        = \prod\limits_r A_{n(r)}$ as an invariant subgroup, and $H/H^1$
        is a commutative group of type $(2, 2, \ldots , 2)$; on the
        other hand a nontrivial commutative factor group of $A_{n(r)}$
        is necessarily cyclic of order 3 (this  happens only for
        $n(r) = 3, 4)$; thus, if $p \neq 2$ and 3, who have Hom $(H,
        k) = 0$, and this proves $b)$. 
        
        We are now going to prove (\ref{c3:eq2}). More precisely we have the
        following lemma (probably well known to specialists in
        homological algebra; probably, also, high-powered
        cohomological methods could make the proof less
        computational). 
        
        \begin{lem}\label{app:lem2} %lemma 2
Let\pageoriginale $G$ be a finite group, $H$ a  subgroup of $G, k$ a ring, $V$
  a free k-module with a basis $(e_\theta)$ indexed by $G/H$. Let $G$
  operate on $V$ by $s(e_\theta) = e_{s \theta}$. Then $H^1 (G, V)
  \approx \text{ Hom } (H, k)$. 
        \end{lem}        
        
        A system $(v_s = \sum\limits_{ \theta \in G/ H} a_{s, \theta}
        e _\theta )$ $( s \in G, a_{s, \theta} \in k )$ is a cocycle if
        and only if $v_{ss'} = v_s + s(v_{s'})$ i.e. if and only if  
        \begin{equation*}
a_{ss' , \theta } = a_{s, \theta} + a_{s', s^{-1}\theta}. \tag
{3}\label{c3:eq3} 
        \end{equation*}        
        
        \noindent
        It is a coboundary if and only if there exists $y =
        \sum\limits_{\theta \in G/H} b_ \theta ~ e_ \theta$ such that
        $v_s = s(y) - y$, i.e. if and only if there exist elements $b_
        \theta$ of $k$ such that 
        \begin{equation*}
a_{s, \theta } = b_{s^{-1} \theta} - b_\theta. \tag{4}\label{c3:eq4}
        \end{equation*}        
        
        Let $\varepsilon$ denote  the unit class $H$ in $G/H$ and,
        given a cocycle $(v_s)$ as above, set $\varphi_v(h) = a_{h,
          \varepsilon}$ for $h$ in $H$. Since $h \varepsilon =
        \varepsilon (h \in H)$, (\ref{c3:eq3}) shows that $\varphi_v$ is a
        homomorphism of $H$ into $k$. We obviously have $\varphi_{v
          +v'} = \varphi_v + \varphi_{v'}$, whence a homomorphism 
        $$
        \varphi : Z^1 (G, V) ~ (``\text{cocycles}'') \to \text{ Hom }
        (H, k).  
        $$
        
        By (\ref{c3:eq4}), we see that $\varphi$ is zero on the
        coboundaries. Conversely if $\varphi_v = 0$, we prove that
        $(v_s)$ is a coboundary. In fact, for $\theta \in G/H$, choose
        $t \in G$ such that $\theta = t^{-1} \varepsilon$, and set
        $b_\theta = a_{t, \varepsilon}$; this element does not depend
        on the choice of $t$ since, if $t^{-1} \varepsilon = u^{-1}
        \varepsilon  $, then $ut^{-1} \in H$ and $u = ht$ with $h \in
        H$; by (\ref{c3:eq3}), we have $a_{u, \varepsilon} = a_{h v,
          \varepsilon} = a_{h, \varepsilon} +  a_{t, h^{-1} \varepsilon}
        = a_{t, \varepsilon} $ (since $\varphi_a = 0)$. Now, if
        $\theta = t^{-1}_ \varepsilon$ and if $s \in G$,\pageoriginale  we have
        $s^{-1} \theta = (ts)^{-1} \theta $, whence $b_ \theta = a_{t,
          \varepsilon}$ and $b_{s^{-1} \theta} = a_{ts,
          \varepsilon}$. From (\ref{c3:eq3}) we get $b_{s^{-1} \theta} -
        b_\theta = a_{ts, \varepsilon} - a _{t, \varepsilon} = a_{s,
          t^{-1}\varepsilon} = a_{s, \theta}$, thus proving that
        $(v_s)$ is a coboundary.  
        
        Thus the proof of lemma \ref{app:lem2} will be complete if we
        show that 
        $\varphi$ is surjective. Let $c$ be a homomorphism of $H$ into
        $k$. For every $\theta$ in $G/H$, we choose $t(\theta)$ in $G$
        such that $\theta = t(\theta)^{-1} \varepsilon$. Then every
        $s \in G$ may be written uniquely as $s = h.t (\mu) ~ (h \in
        H, \mu = s^{-1} H)$. We set  
        \begin{equation*}
a_{s, \theta} = c(h),\tag{5}\label{c3:eq5} 
        \end{equation*}
        where $h$ is the unique element of $H$ such that $t(\theta)
        \cdot  s = h.t (s^{-1} \theta)$ (notice that $t (\theta). s.t (s^{-1}
        \theta)^{-1}. \varepsilon = t(\theta) s.s ^{-1} \theta = t
        (\theta). \theta = \varepsilon$, whence $t (\theta) . s. t\break
        (s^{-1} \theta )^{-1} \in H)$. Let us verify the ``cocycle
        condition'' (\ref{c3:eq3}). We have $a_{ss', \theta} = c(h), a_{s,
          \theta} = c(h_1)$ and $a_{s' , s^{-1} \theta} = c(h_2)$,
        with $t(\theta) ss' = h.t\break (s'^{-1} s^{-1} \theta)$, $t
        (\theta)s = t (~ ) s = h_1 t (s^{-1} \theta)$ and $t (s^{-1}
        \theta). s' = h_2 t(s'^{-1} s^{-1} \theta)$.  From this we
        immediately deduce that  $h = h_1 h_2$. Since $c$ is a
        homomorphism, we have $c(h) = c(h_1) + c(h_2)$, i.e. $a_{ss'},
        \theta = a_{s, \theta}+ a_{s' s ^{-1} \theta}$. Thus $v_s  =
        \sum\limits_{\theta} a_s \theta e_\theta$  is a  cocycle. For
        this cocycle, we have (for $h \in H) \varphi^\theta_v (h) =
        a_{h, \varepsilon} = c(h_1)$, where, by (\ref{c3:eq5}), $h_1$ is such
        that $t(\varepsilon). h = h_1 t (h^{-1} \varepsilon) = h_1
        t(\varepsilon)$; since the additive group of $k$ is
        commutative, we have $c(h) = c(h_1)$, whence $\varphi_v (h) =
        c(h) $ for every $h \in H$. 
        \hfill{Q.E.D}

