\chapter{Krull rings and factorial rings}\label{chap1}

In this\pageoriginale chapter we shall study some elementary
properties of Krull rings and factorial rings.   

\section{Divisorial ideals}\label{chap1:sec1}%Sec 1
Let $A$ be an integral domain (or a domain) and $K$ its quotient
field. \textit{A fractionary ideal} $\mathscr{U}$ is an
$A$-sub-module of $K$ for which there exists an element $d \in A$ ($d
\neq 0$) such that $d \mathscr{U} \subset A$ i.e. $\mathscr{U}$ has
``common denominator'' $d$). A fractionary ideal is called a 
\textit{principal ideal} if it is generated by one
element. $\mathscr{U}$ is said to be \textit{integral} if
$\mathscr{U}\subset A$. We say that 
$\mathscr{U}$ is \textit{divisorial} if $\mathscr{U} \neq (0)$ and if
$\mathscr{U}$ is an intersection of principal ideals. Let
$\mathscr{U}$ be a fractionary ideal and $\mathscr{V}$ non-zero
$A$-submodule of $K$. Then the set $\mathscr{U} : \mathscr{V} = \big
\{x \in K \big| x \mathscr{V} \subset \mathscr{U} \big\} $ is a
fractionary ideal. The following formulae are easy to verify. 
\begin{enumerate}[(1)]
\item $(\bigcap\limits_{i} \mathscr{U}_i : (\sum\limits_j
  \mathscr{V}_j) = \bigcap\limits _{i, j} ( \mathscr{U}_i :
  \mathscr{V}_j)$. 

\item $\mathscr{U} : \mathscr{V} \mathscr{V}' = (\mathscr{U} :
  \mathscr{V}) : \mathscr{V}'$. 

\item If $x \in K$, $x \neq 0$, then $\mathscr{V} : Ax = x^{-1}
  \mathscr{V}$. 
\end{enumerate}

\begin{lemma}\label{chap1:lem1.1} % lem 1.1
Let $\mathscr{U}$ be a fractionary ideal $\neq (0)$ and $\mathscr{V}$
a divisorial ideal. Then $\mathscr{V} : \mathscr{U}$ is divisorial. 
\end{lemma}

\begin{proof}
Let $\mathscr{V} = \bigcap\limits_{i} Ax_i $. Then $\mathscr{V} :
\mathscr{U} = \bigcap\limits_i (Ax_i : \mathscr{U}) = \bigcap\limits_i
(\bigcap\limits_{a \in \mathscr{U}} \dfrac{Ax_i}{a})$ 
\end{proof}

\begin{lemma}\label{chap1:lem1.2} % lem 1.2
\begin{enumerate}[(1)]
\item Let\pageoriginale $\mathscr{U} \neq (0)$ be a fractionary
  ideal. Then the 
  smallest divisorial ideal containing $\mathscr{U}$, denoted by
  $\bar{\mathscr{U}}$, is $A : (A : \mathscr{U})$. 

\item If $\mathscr{U}$, $\mathscr{V} \neq (0)$, then $\bar{\mathscr{U}}
  = \bar{\mathscr{V}} \Leftrightarrow A : \mathscr{U} = A :
  \mathscr{V}$.
\end{enumerate}
\end{lemma}

\begin{proof}
\begin{enumerate} [(1)]
\item By Lemma~\ref{chap1:lem1.1}, $A : (A : \mathscr{U})$ is
  divisorial. Obviously 
  $\mathscr{U} \subset A : (A : \mathscr{U})$. Suppose now that
  $\mathscr{U} \subset A x$, $x \neq 0$, $x \in K$. Then $A : \mathscr{U}
  \supset A : A x = A x^{-1}$. Thus $A : (A : \mathscr{U}) \subset A :
  A x^{-1} = Ax$ and (1) is proved. 

\item is a trivial consequence of (1) and the proof of 
Lemma~\ref{chap1:lem1.2} is complete.  
\end{enumerate}
\end{proof}


\section{Divisors}\label{chap1:sec2}%Sec 2 
Let $I(A)$ denote the set of non-zero fractionary ideals of the
integral domain $A$. In $I(A)$, we introduce an equivalence relation
$\sim$, called \textit{Artin equivalence} (or \textit{quasi
  Gleichheit}) as follows:  

\noindent
$\mathscr{U} \sim \mathscr{V} \Leftrightarrow \bar{\mathscr{U}} =
\bar{\mathscr{V}} \Leftrightarrow A : \mathscr{U} = A:
\mathscr{V}$. The quotient set $I(A)/\sim$ of $I(A)$ by the equivalence
relation $\sim$ is called the set of \textit{divisors of $A$}. Thus
there is an l-1 correspondence between the set $D(A)$ of divisors and
the set of divisorial ideals. Let $d$ denote the canonical mapping $d:
I (A) \rightarrow I(A) / \sim$. Now, $I(A)$ is partially ordered by
inclusion and we have $\mathscr{U} \subset \mathscr{V} \Rightarrow
\bar{\mathscr{U}} \subset \bar{\mathscr{V}}$. Thus this partial order
goes down to the 
quotient set $I(A) / \sim$ by $d$. If $\mathscr{U} \subset
\mathscr{V}$, we write $d (\mathscr{V}) \leq d(\mathscr{U})$. 
On $I(A)$ we have the structure of an ordered commutative monoid given
by the composition law $(\mathscr{U}, \mathscr{V}) \rightsquigarrow
\mathscr{U} \mathscr{V}$-with $A$ acting as the unit element. Let
$\mathscr{U}$, $\mathscr{U}'$, $\mathscr{V} \in I(A)$ and $\mathscr{U}
\sim \mathscr{U}' i.e. ~ A : \mathscr{U} = A : \mathscr{U}'$. We have
$A : \mathscr{U} \mathscr{V} = (A : \mathscr{U}) : \mathscr{V} = (A :
\mathscr{U}') : \mathscr{V} = A : \mathscr{U}' \mathscr{V}$. Hence
$\mathscr{U} \mathscr{V} \sim \mathscr{U}' \mathscr{V}$. Thus $D(A)$
acquires the structure of a commutative monoid, with\pageoriginale the
composition 
law $(\bar{\mathscr{U}}, \bar{\mathscr{V}}) \rightarrow
\overline{\mathscr{U} \mathscr{V}}$. We write the composition law in
$D(A)$ additively. Thus $d(\mathscr{U} \mathscr{V}) = d(\mathscr{U}) +
d (\mathscr{V})$ for $\mathscr{U}$, $\mathscr{V} \in I(A)$. Since the
order in $D(A)$ is compatible with the compositional law in $D(A)$, $D(A)$
is a commutative ordered monoid with unit. We note that
$$
d(\mathscr{U} \cap \mathscr{V}) \geq \sup (d (\mathscr{U}), d
(\mathscr{V})   
$$
and  
$$
d(\mathscr{U}+ \mathscr{V}) = \text{ inf } (d (\mathscr{U}), d
(\mathscr{V}). 
$$

\noindent
Let $K^*$ be the set of non-zero elements of $K$. For $x \in K^*$, we
write $d(x) = d(Ax); d(x)$ is called a \textit{principal} divisor. 

\begin{theorem}\label{chap1:thm2.1} % the 2.1
For $D(A)$ to be a group it is necessary and sufficient that $A$
  be completely integrally closed. 
 \end{theorem} 
 
 We recall that $A$ is said to be \textit{completely integrally}
 closed if whenever, for $x \in K$ there exist an $a \neq 0$, $a \in A
 s.t. ax^n \in A$ for every $n$, then $x \in A$. 
 
  We remark that if $A$ is completely integrally closed, then it is
  integrally closed. The converse also holds if $A$ is noetherian. A
  valuation ring of height $>1$ is an example of an integrally closed
  ring which is not completely integrally closed. 
  
\begin{proof} 
Suppose $D(A)$ is a group. Let $x \in K$ and a be a non-zero element of $A$
such that $ax^n \in A$, for every $n \geq 0$. Then 

\noindent
  $a \in \bigcap\limits_{n = 0} Ax^{-n} = \mathscr{V}$ which is
divisorial. Set $d(\mathscr{V}) = \beta $ and $\alpha = d(x^{-1})$. 
  \end{proof}  
  
\noindent
Now $\beta = \sup\limits_{n \ge 0} (n \alpha)$. But $\beta + \alpha =
\Sup\limits_{n \ge 0}((n + 1) \alpha)$ 
  
  \quad ~~$= \sup\limits_{q \ge 1} ~ (q \alpha)$. Thus $\beta +
  \alpha \leq \beta$. Since $D(A)$ is a group $-\beta$\pageoriginale
  exists, and   therefore 
  $$
  \alpha = (\beta + \alpha) - \beta \le \beta - \beta = 0\text{
    i.e. } d(x) \ge 0. 
  $$
  
  \noindent
  Hence $Ax \subset A$ i.e. ~ $x \in A$.
  
  \noindent
  Conversely suppose that $A$ is completely integrally closed. Let
  $\mathscr{U}$ be a divisorial ideal. Then $\mathscr{U} = x
  \mathscr{U}', \mathscr{U}' \subset A$. Since we already know that
  principal divisors are invertible, we have only to prove that
  integral divisorial ideals are invertible. Let $\mathscr{V}$ be a
  divisorial ideal $\subset A$. Then $\mathscr{V} . (A : \mathscr{V})
  \subset A$. Let $\mathscr{V} (A:  \mathscr{V}) \subset A x$, for
  some $x \in K$. Then $x^{-1} \mathscr{V} (A : \mathscr{V}) \subset
  A$. Thus $x^{-1} \mathscr{V} \subset A : (A : \mathscr{V}) =
  \mathscr{V}$, since $\mathscr{V}$ is divisorial. Thus $\mathscr{V}
  \subset \mathscr{V} x$. By induction $\mathscr{V} \subset
  \mathscr{V} x^n$, for every $n \ge 0$. Consider an element $b
  \neq 0$, $b \in \mathscr{V}$. Then $b (x^{-1})^n \subset
  \mathscr{V} \subset A$ for every $n \ge 0$. Hence $x^{-1} \in A$
  i.e. $Ax \supset A$. Thus $\mathscr{V} (A : \mathscr{V}) \sim
  A$. Theorem~\ref{chap1:thm2.1} is completely proved.  
  
\noindent
Notice that we have $d(\mathscr{U}) + d(A : \mathscr{U}) = 0$. 
  
Let us denote by $F(A)$ the subgroup generated by the principal
divisors. If $D(A)$ is a group, the quotient group $D(A)/F(A)$ is
called the \textit{divisor class group} of $A$ and is denoted by
$C(A)$. 
  
In this chapter we shall study certain properties of the group
$C(A)$. 

\section{Krull rings}\label{chap1:sec3}%Sec 3 
Let $\mathbb{Z}$ denote the ring of integers. Let $I$ be a
set. Consider the  abelian group $\mathbb{Z}^{(I)}$. We order
$\mathbb{Z}^{(I)}$ by means of the following relation:  
  
\noindent
  for $(\alpha_i),(\beta_i) \in \mathbb{Z}^{(I)}$, $( \alpha_i) \ge
  (\beta_i) $ ~~if~~ $\alpha_i \ge \beta_i, $ ~~for all~~ $i \in I$. 
  
The\pageoriginale ordered group $\mathbb{Z}^{(I)}$ has the following
properties: 
(a) any two elements of $\mathbb{Z}^{(I)}$ have a least upper bound
and a greatest lower bound i.e. $\mathbb{Z}^{(I)}$ is an ordered
lattice. (b) The positive elements of $\mathbb{Z}^{(I)}$ satisfy the
minimum condition i.e. given a nonempty subset of  positive elements
of $\mathbb{Z}^{(I)}$, there exists a minimal element in that
set. Conversely any ordered abelian group satisfying conditions (a)
and (b) is of the form $\mathbb{Z}^{(I)}$ for some indexing set $I$
(for proof see Bourbaki, Algebre, Chapter $VI$).  
  
Let $A$ be an integral domain. We call $A$ a \textit{Krull ring} if
$D(A) \thickapprox \mathbb{Z}^{(I)}$, the isomorphism being order
preserving. 

\begin{theorem}\label{chap1:thm3.1} % the 3.1
Let $A$ be an integral domain. Then $A$ is Krull if and only if
  the following two conditions are satisfied. 
\begin{enumerate}[(a)]
\item $A$ is completely integrally closed.

\item The divisorial ideals contained in $A$ satisfy the maximum
  condition. 
\end{enumerate} 
  \end{theorem}  
  
\noindent
In fact the above theorem is an immediate consequence of 
Theorem~\ref{chap1:thm2.1} and the characterization of the ordered group
  $\mathbb{Z}^{(I)}$ mentioned above. An immediate consequence of
  Theorem~\ref{chap1:thm3.1} is: 
  
\begin{theorem}\label{chap1:thm3.2} %the 3.2
$A$ noetherian integrally closed domain is a Krull ring. 
\end{theorem} 
  
\noindent 
  We remark that the converse of Theorem~\ref{chap1:thm3.2} is
  false. For example 
  the ring of polynomials in an infinite number of variables over a
  field $K$ is a Krull ring, but is not noetherian. In fact this ring
  is known to be factorial and we shall show later that any factorial
  ring is a Krull ring.  

  Let $e_i = ( \delta_{ij}) _{j \in I} \in \mathbb{Z}^{(I)}$, where
  $\delta_{ij}$ is the usual  
  
  \noindent
  Kronecker\pageoriginale delta. The $e_i$ are minimal among the
  strictly positive 
  elements. Let $A$ be a Krull ring and let $\varphi$ be the order
  preserving isomorphism $\varphi : D(A) \to \mathbb{Z}^{(I)}$. Let
  $\underline{P}_i = \varphi^{-1} (e_i)$. We call the divisors
  $\underline{P}_i$ the \textit{prime divisors}. Let $P(A)$ denote the
  set of prime divisors. Then any $\underline{d} \in D(A)$ can be
  written uniquely in the form 
  $$
  \underline{d} = \sum\limits_{\underline{P} \in P(A)}
  n_{\underline{P}} \underline{P} 
  $$
  where $n_{\underline{P}} \in \mathbb{Z}$ and $n_{\underline{P}} = 0$
  for almost all $\underline{P}$. Now let $x \in K^*$. Consider the
  representation 
  $$
  d(x) = \sum_{\underline{P} \in P(A)} v_{\underline{P}} (x)
  \underline{P}, v_{\underline{P}} (x) \in \mathbb{Z},
  v_{\underline{P}} (x) = 0 
  $$
  for almost  all $\underline{P} \in P(A)$. Since $d(xy)=d(x)+d(y)$ we
  have, $v_{\underline{P}}(xy) = v_{\underline{P}}(x) +
  v_{\underline{P}}(y)$ for all $\underline{P}$. Further $d(x+y) \ge
  d(Ax+Ay) = \inf (d(x), d(y))$. This, in terms of
  $v_{\underline{P}}$, means that $v_{\underline{P}} (x+y) \ge \inf
  (v_{\underline{P}}(x), v_{\underline{P}}(y))$. We set
  $v_{\underline{P}}(0) = + \infty$. Thus the $v_{\underline{P}}$ are
  all discrete valuations of $K$. These are called the
  \textit{essential valuations} of $A$.  
  
  Let $\underline{P}$ be a prime divisor. Let $\mathscr{Y}$ be the
  divisorial ideal corresponding to $\underline{P}$. As
  $\underline{P}$ is positive, $\mathscr{Y}$ is an integral ideal. We
  claim that $\mathscr{Y}$ is a \textit{prime} ideal. For let $x$, $y
  \in A$, $xy \in \mathscr{Y}$. Then $d(xy) \ge \underline{P}$ i.e. $d(x)
   + d(y) \ge \underline{P}$ i.e. $v_{\underline{P}} (x) +
  v_{\underline{P}}(y) \ge 1$. As $v_{\underline{P}} (x) \ge 0$, $v_p(y)
 \geq 0$, we have $v_{\underline{P}}(x) \ge 1$  or $v_{\underline{P}} (y) \ge 1$;
  i.e. $x \in \mathscr{Y}$ or $y \in \mathscr{Y}$. Further the
  divisorial ideal corresponding to $n \underline{P} $ is $\big\{ x
  \in A \big| v_{\underline{P}} (x) \ge n \big\}$,\pageoriginale $n
  \ge 0$. The 
  prime ideal $\mathscr{Y}$ is the centre of the valuation
  $v_{\underline{P}}$ on $A$ (i.e. the set of all elements $x \in A$
  s.t. $v_{\underline{P}} (x) \ge 0$). Since the prime divisors are
  minimal among the set of positive divisors, the corresponding
  divisorial ideals, which we call \textit{prime divisorial ideals},
  are maximal among the integral divisorial ideals. The following
  lemma shows that the divisorial ideals which are prime
  divisorial and this justifies the terminology 'prime divisorial'. 
  
\setcounter{lemma}{2}
\begin{lemma}% lem 3.3
Let $\mathscr{G} $ be a prime ideal $\neq (0)$. Then $\mathscr{G}$
contains some prime divisorial ideal. 
\end{lemma}  
  
\begin{proof}
Take an $x \in \mathscr{G}$, $x \neq 0$. Let $d(x) = \sum\limits_i n_i
P_i$, (finite sum), $n_i > 0$, $\underline{P}_i \in P(A)$. Let
$\mathscr{Y}_i$ be the prime ideal corresponding to
$\underline{P}_i$. Let $y \in \prod \mathscr{Y}^{n_i}_{i}$, $y \neq
0$. Then $v_{P_i} (y) \ge n_i$. Hence $d(y) \ge d(x)$ i.e. $Ay \subset
Ax$. Thus $\prod \mathscr{Y}^{n_i}_i \subset Ax \subset
\mathscr{G}$. As $\mathscr{G}$ is prime, $\mathscr{Y}_i \subset
\mathscr{G}$ for some $i$. 
\end{proof}  
  
\begin{coro*} % coro
A prime ideal is prime divisorial if and only if it is height 1.
\end{coro*}  
   
(We recall that a prime ideal is of \textit{height one} if it is
minimal among the non-zero prime ideals of $A$). 
  
\begin{proof}
Let $\mathscr{Y}$ be a prime divisorial ideal. If $\mathscr{Y}$ is not
of height 1, then $\mathscr{Y} \underset{\neq}{\supset}
\mathscr{G}$, where $\mathscr{G}$ is a non-zero prime ideal. By the
above lemma $\mathscr{G}$ contains a prime divisorial ideal
$\mathscr{Y}'$. Thus $\mathscr{Y} \underset{\neq}{\supset}
\mathscr{Y}'$. This contradicts the maximality of $\mathscr{Y}'$ among
integral divisorial ideals. Conversely let $\mathscr{Y}$ be a prime
ideal of height 1. Then, by the above lemma, $\mathscr{Y}$ contains
a prime divisorial ideal $\mathscr{Y}'$. Hence $\mathscr{Y } =
\mathscr{Y}'$ and the proof of the Corollary is complete. 
\end{proof}  
  
\begin{lemma} % lem 3.4
 Let\pageoriginale $\mathscr{Y}$ be a divisorial ideal corresponding
 to a prime divisor $\underline{P}$. Then the ring of quotients
 $A_\mathscr{Y}$ is the ring of $v_{\underline{P}}$.  
\end{lemma}  
  
\begin{proof}
Let $\dfrac{a}{s} \in A_\mathscr{Y}$, $a \in A$, $s \in A -
\mathscr{Y}$. Then $v_{\underbar{P}}(s) = 0$, $v_{\underbar{P}} (a)
\geq 0$,
$v_{P} \left(\dfrac{a}{s} \right) \ge 0$. Conversely let $x \in K^*$
with $v_{\underline{P}} (x) \ge 0$. Set $d(x) =
\sum\limits_{\underline{Q}} n (\underline{Q}) \underline{Q}$, and let
$\mathscr{G}$ 
be the prime divisorial ideal corresponding  to $\underline{Q}$. Let
$\mathscr{V} = \prod\limits_{n (\underline{Q}) < 0} \mathscr{G}^{-n
  (\underline{Q})} $. As the prime divisor $\underline{Q}$, with $n
(\underline{Q}) < 0$ are different from $\underline{P}$, we have
$\mathscr{V} \nsubset \mathscr{Y}$. Take $s \in \mathscr{V}$, $s \notin
\mathscr{Y}$. Then $v_{\underline{Q}}(sx) \ge 0$ for all
$\underline{Q}$ i.e. $d(sx) \ge 0$ i.e. $sx \in A$. Hence $x \in
A_\mathscr{Y}$. This proves the lemma. 
\end{proof}  
  
\begin{coro*}
$A = \bigcap\limits_{\mathscr{Y}} A_\mathscr{Y}$, $\mathscr{Y}$
    running through all prime ideals of height one. We shall now give
    a characterization of Krull rings in terms of discrete valuation
    rings. 
\end{coro*}  
  
\setcounter{theorem}{4}
\begin{theorem}[``valuation Criterion'']\label{chap1:thm3.5}%the 3.5
 Let $A$ be a domain. Then the following conditions are equivalent:  
\begin{enumerate}[(a)]
\item $A$ is a Krull ring.

\item  There exists a family $(v_i)_{i  \in I}$ of discrete
  valuations of $K$ such that 
\end{enumerate}
\begin{enumerate}[(1)]
 \item $ A = \bigcap\limits_i R_{v_i}$ (i.e. $x \in A$ if and only if
   $v_i (x) \ge 0$), where $R_{v_i}$ denotes  the ring of $v_i$, 

 \item For every $x \in A$, $v_i (x) = 0$, for almost all $i \in
   I$. 
\end{enumerate}
\end{theorem}   
  
  \begin{proof} % pro
  \begin{enumerate}[(a)]
 \item $ \Rightarrow (b)$. In fact $A = \bigcap\limits_{\underline{P}
   \in P(A)} R_{v_{\underline{P}}}$ and condition (2) of (b) is
   obvious from the very definition of $v_{\underline{P}}$. 

 \item $\Rightarrow (a)$.\pageoriginale Since a discrete valuation ring is
   completely integrally closed and any intersection of completely
   integrally closed domains is completely integrally closed, we
   conclude that $A$ is completely integrally closed. Now let $x \in
   K^*$. Then 
 $$
 Ax = \left \{y \in K \big| v_i (y) \ge v_i (x), \text{ for } i \in I
 \right\}. 
 $$
 Thus, because of condition (2), any divisorial ideal is of the form
 $\left\{ x \in K \big|  v_i (x) \ge n_i, i \in I, (n_i) \in
 \mathbb{Z}^{(I)} \right\}$, and conversely. Any integral divisorial
 ideal $\mathscr{V}$ is defined by the conditions $v_i (x) \ge n_i$,
 $n_i \ge 0$, $n_i = 0$ for almost all $i \in I$. There are only
 finitely many divisorial ideals $\mathscr{V}'$ containing
 $\mathscr{V}$ (in fact the number of such ideals is $\prod\limits_i
 (1+n_i))$. Hence $A$ satisfies the maximum condition for integral
 divisorial ideals  therefore by Theorem~\ref{chap1:thm3.1}, $A$ is a
 Krull ring.  
  \end{enumerate}
  \end{proof}  
  
\begin{remark*}
Let $\mathscr{G}$ be a prime divisorial ideal of $A$ defined by $v_i
(x) \ge n_i$, $n_i \ge 0$, $n_i = 0$ for almost all $i$. Let
$\mathscr{Y}_i$ be the centre of $v_i $ on $A$. Then $\prod\limits_i
\mathscr{Y}_i^{n_i} \subset \mathscr{G}$. Hence $\mathscr{Y}_i \subset
\mathscr{G}$ for some $i$. But height $\mathscr{G} = 1$. Hence
$\mathscr{Y}_i = \mathscr{G}$. Thus every prime divisorial ideal is
the centre of some $v_i$. Now $A_\mathscr{G} \subset R_{v_i}$. But
$A_\mathscr{G}$, being a discrete valuation ring, is a maximal subring
of $K$. Hence $A_\mathscr{G} = R_{v_i} = $ ring of
$v_{\underline{Q}}$, where $\underline{Q}$ is the prime divisor
corresponding to $\mathscr{G}$. Thus every essential valuation of $A$
is equivalent to some $v_i$. The family $(v_i)_{i \in I}$ may be
`bigger', but contains all essential valuations. 
\end{remark*}  
  

  \section{Stability properties}\label{chap1:sec4}%Sec 4
    In this section we shall see that Krull rings behave well under
  localisations polynomial extensions etc. 
 
\begin{prop}\label{chap1:prop4.1} % prop 4.1
Let\pageoriginale $K$ be a field and $A_\alpha$ a family of Krull
rings. Assume 
  that any $x \in B = \bigcap\limits_{\alpha} A_\alpha$, $x \neq 0$ is
  a unit in almost all $A_\alpha$. Then $B$ is a Krull ring.   
\end{prop}  

\begin{proof}
By theorem~\ref{chap1:thm3.5}, $A_{\alpha}= \bigcap\limits_{i \in
  I_\alpha} R (v_{\alpha, 
  i})$, where $R(v_{\alpha , i})$ are discrete valuation rings and
every $x \in A_\alpha$ is a unit in almost all $R(v_{\alpha , i})$, $i
  \in I$. Now $B = \bigcap\limits_{\alpha , i} R (v_{\alpha ,
  i})$. Let $x \neq 0$, $x \in B$. Then by assumption, $x$ is a unit in
almost all $A_{\alpha}$, i.e. $v_{\alpha, i} (x) = 0$, for all $i \in
I_\alpha$, and almost all $\alpha$. Then $x$ is not a unit in at most a
finite number of the $A_\beta$, say $A_{\beta_1} , \ldots
A_{\beta_t}$. Now $v_{\beta_{ji}} (x) = 0$ for almost all $i$, $j = 1,
\ldots , t$. Thus $v_{\alpha, i}(x) = 0$ for almost all $\alpha$ and
$i$. The proposition follows immediately from Theorem~\ref{chap1:thm3.5}. 
\end{proof}  
  
\begin{coro*} % coro 
\begin{enumerate}[(a)]
\item A finite intersection of Krull rings is a Krull ring.

\item Let $A$ be a Krull ring, $K$ its quotient field. Let
$L$ be a subfield of $K$. Then $A \cap L$ is a Krull ring.
\end{enumerate}
  \end{coro*} 
   
  \begin{prop} % prop 4.2
Let $A$ be a Krull ring. Let $S$ be any multiplicatively closed
  set with $0 \notin S$. Then the ring of quotients $S^{-1} A$ is
  again a Krull ring. Further the essential valuations of $S^{-1} A$
  are those valuation $v_P$ for which $\mathscr{Y} \cap S = \phi$,
  $\mathscr{Y}$ being the prime divisorial ideal corresponding to $P$. 
  \end{prop} 
   
\begin{proof}
By Theorem~\ref{chap1:thm3.5} we have only to prove that $S^{-1} A =
\bigcap\limits_{\mathscr{Y} \in I} A_\mathscr{Y}$, $I$, being the set of
prime ideals of height one which do not intersect $S$. Trivially,
$S^{-1} A \subset \bigcup\limits_{\mathscr{Y} \in I}
A_\mathscr{Y}$. Conversely let $x \in \bigcup\limits_{\mathscr{Y \in
    I}} A_\mathscr{Y}$. We have $v_P (x) \ge 0$,\pageoriginale for any
prime divisor 
corresponding to $\mathscr{Y} \in I$. For a prime divisorial ideal
$\mathscr{G}$ with $\mathscr{G} \cap S \neq\emptyset$, choose
$s_\mathscr{G} \in \mathscr{G} \cap S$. Set $s =
\prod\limits_{v_{Q}(x) < 0} s^{- v_Q (x)}_{\mathscr{G}}$, where
$Q$ is the prime divisor corresponding to $\mathscr{G}$. Then $sx \in
A $ i.e. $x \in S^{-1}_A$ and the proposition is proved. 
\end{proof}  
  
\begin{prop} % prop 4.3
Let $A$ be a Krull ring. Then the ring $A[X]$ of polynomials is again
a Krull ring. 
\end{prop}  
  
\begin{proof}
Let $A$ be defined by the discrete valuation rings $\big\{v_i\big\}_{i
  \in I}$. If $v$ is a discrete valuation of $A$, then $v$ can be
extended to $A[X]$, by putting $\bar{v} (a_o + a_1 x + \cdots + a_q
x^q) = \min\limits_{j}(v(a_j))$ and then to the quotient field $K(X)$
of $A[X]$ ($K$ is the quotient field of $A$) by putting $\bar{v}(f/g) =
\bar{v}(f) - \bar{v}(g)$.  Let $\Phi = \big \{  \bar{v}_i \big\}_{i
  \in I}$. On the other hand, let $\Psi$ denote the set $p(X)$-adic
valuation of $K(X)$, where $p(X)$, runs through all irreducible
polynomials $K[X]$. We now prove that $A[X]$ is a Krull ring with
$\Phi \bigcup \Psi$ as a set of valuations defining it. Let $f \in
K(X)$. If $\omega (f) \ge 0$ for all $\omega' \in \Psi$, then $f \in
K[X]$, say $f = a_o + a_1 X + \cdots + a_q X^q : a_i \in K$. If further
$v(f) \ge 0$, for all $v \in \Phi$, then $v(a_i) \ge 0$, $v \in \Phi$, $i
= 1, \ldots , q$. Since $A$ is a Krull ring, $a_i \in A$, $i =
1, \ldots , q$. Hence $f \in A [X]$. To prove that $A [X]$ is a Krull
ring, it only remains to verify that for $f \in A [X]$,  $v(f)= 0 =
\omega (f)$ for almost all $v \in \Phi$, $\omega \in \Psi$. Since $K 
X$ is a principal ideal domain, $\omega(f) = 0$, for almost all
$\omega \in \Psi$. Further $\bar{v}_i (f) = \min\limits_{j}(v_i (a_j))
= 0$ for almost all $i \in I$, since $A$ is a Krull ring. 
\end{proof}  
  
\begin{coro*} % coro
Let\pageoriginale $A$ be a Krull ring. Then $A \big[ X_1 , \ldots ,
  X_n \big]$ is a Krull ring.  
\end{coro*}  
  
\begin{remark*} % rem
Since, in a polynomial ring in a an infinite number of variables, a
given polynomial depends only on a finite number of  variables the
above proof shows that a polynomial ring in an infinite number of
variables over $A$ is also a Krull ring, 
\end{remark*} 
   
\begin{prop} %prop 4.4
Let $A$ be a Krull ring. Then the ring of formal power series $A
  [[X]]$ is again a Krull ring. 
\end{prop}  
  
\begin{proof}
We note that $A \big[[X] \big] = \bigcap V_\alpha [[X]]_S \bigcap K
\big[[X]\big]$, 
where $S$ is the multiplicatively closed set $\big\{1, x, x^2, \ldots
\big\}$, $K$, the quotient field of $A$ and $V_{\alpha}$ the essential
valuation rings of $A$. Now $V_\alpha \big[[X]\big]_S$ and $K \big
[[X] \big]$, being noetherian and integrally closed, are Krull rings,
Now the proposition is an immediate consequence of 
Proposition~\ref{chap1:prop4.1}.  
\end{proof}  
  
\begin{prop}\label{chap1:prop4.5} % prop 4.5
 Let $A$ be a Krull ring and $K$ its quotient field. Let $K'$ be 
  a finite algebraic extension of $K$ and $A'$, the integral closure
  of $A$ in $K'$. Then $A'$ is also a Krull ring.  
\end{prop}  
  
\begin{proof}
Let $K''$ be the least normal extension of $K$ containing $K'$ and let
$A''$ be the integral closure of $A$ in $K''$. Since $A' = A'' \bigcap
K'$, by Corollary $(b)$, Prop. \ref{chap1:prop4.1}, it is sufficient
to prove that 
$A''$ is a  Krull ring. Let $\Phi$ be the family of essential
valuations of $A$. Let $\Phi''$ be the family of all discrete
valuations of $K''$ whose restriction to $K$ are in $\Phi$. It is well
known (see Zariski-Samuel. Commutative  algebra Vol. 2) that every
discrete valuation $v$ of $K$ extends to a discrete valuation of $K''$
and that such extensions are finitely many in number.\pageoriginale We
shall show that $A''$ is a Krull ring by using the valuation criterion
with $\Phi''$ as family of valuations. We prove $(i) \quad x \in A''$
if and only if $\omega' (x) \ge 0$, for all $\omega \in \Phi''$  
\end{proof}  
  
\begin{proof}
Let $x \in A''$. Then $x$ satisfies a monic polynomial, say $x^n +
a_{n-1} x^{n-1} + \cdots + a_o = 0$, $a_i \in A$. If possible let
$\omega(x) < 0$, for some $\omega \in \Phi$. Then $\omega (-a_{n-1}
x^{n-1} - \cdots - a_o) \ge \inf (\omega (a_{n - 1} x^{n-1}) , \ldots ,
\omega(a_o) > \omega(x^n)$, since $\omega(a_i) \ge
0$. Contradiction. Conversely let $x \neq 0$, $x \in K''$ with $\omega
(x) \ge 0$, for all $\omega \in \Phi''$. Let $\sigma$ be any
$K$-automorphism of $K''$. Then $\omega o \sigma \in \Phi''$. Hence
$\omega (\sigma (x)) \ge 0$ for all $K$-automorphisms $\sigma$ of
$K''$. Let us now consider the minimal polynomial $f(X)$ of $x$ over
$K$; say $f(X) = X^r + \alpha_{r - 1} X^{r-1} + \cdots + \alpha_o$, $\alpha_i
\in K$. Since the $\alpha_i$ are symmetric polynomials in the $\sigma
(x)$, we have $v(\alpha_i) \ge 0$, for all $v \in \Phi$. Since $A$ is
a Krull ring, $\alpha_i \in A$ and $(i)$ is proved. $(ii)$ For $x \neq
0$, $x \in A''$, $\omega (x) = 0$, for almost all $\omega \in \Phi''$. 
\end{proof}  
  
\begin{proof}
Let $x^n + a_{n -1} x^{n-1} + \cdots + a_o = 0$ be an equation
satisfied by $x$ (which expresses the integral dependence of $x$). We
may assume $a_o \neq 0$. If $\omega (x) > 0$, then $\omega (a_o) =
\omega (x^n + a_{n - 1} x^{n-1} +  \cdots + a_1 x) \ge \omega (x) >
0$.  
Since $A$ is a Krull ring there are only a finitely many $v \in \Phi $
such that $v(a_o) > 0$ and since every $v \in \Phi$ admits only a
finite number of extensions, (ii) is proved and with it, the
proposition. 
\end{proof}  
  
\section{Two classes of Krull rings}\label{chap1:sec5}%Sec 5

\begin{theorem}\label{chap1:thm5.1} % the 5.1
Let $A$ be a domain. The following conditions are equivalent. 
\begin{enumerate}[(a)]
\item Every\pageoriginale fractionary ideal $\mathscr{V} \neq (0)$ of $A$ is
  invertible. (i.e. there exists a fractionary ideal
  $\mathscr{V}^{-1}$ such that $\mathscr{V} \mathscr{V}^{-1} = A)$. 

\item $A$ is a Krull ring and every non-zero ideal is
  divisorial. 

\item $A$ is a Krull ring and every prime ideal $\neq (0)$ is
  maximal (minimal). 

\item $A$ is a noetherian, integrally closed domain such that every
  prime ideal $\neq (0)$ is maximal. 
\end{enumerate}  
\end{theorem}  
  
\begin{proof}
(a) ~ $\Rightarrow$ ~ (b). Let $\mathscr{V}^{-1}$ exist. Then
    $\mathscr{V}^{-1} = A : \mathscr{V}$. For $\mathscr{V}
    \mathscr{V}^{-1} \subset A \Rightarrow \mathscr{V}^{-1} \subset A
    : \mathscr{V}$ and $\mathscr{V}\mathscr{V}^{-1} = A$ and
    $\mathscr{V} (A : \mathscr{V}) \subset A$ together imply $A :
    \mathscr{V} \subset \mathscr{V}^{-1}$. Further $\mathscr{V} = A :
    (A : \mathscr{V} )$; the condition that $\mathscr{U}$ and
    $\mathscr{V}$ are Artin equivalent becomes $`` \mathscr{U} =
    \mathscr{V}''$. Thus $D(A)$ is a group. The following lemma now
    proves the assertion (a) ~ $\Rightarrow$ ~ (b) because of 
Theorem~\ref{chap1:thm2.1}.   
\end{proof}  
  
\begin{lemma*} 
$\mathscr{V} \subset A $ invertible $\Rightarrow \mathscr{V}$ is
    finitely generated. (and thus (a) implies that $A$ is noetherian). 
\end{lemma*}  
  
\begin{proof} % pro
Since $\mathscr{V} \mathscr{V}^{-1} = A$, we have $1 = \sum\limits_1
x_i y_i$, $x_i \in \mathscr{V}$, $y_i \in \mathscr{V}^{-1}$. 
 For $x \in \mathscr{V}$, we have $x = \sum\limits_i x_i (y_i x)$ i.o.
$\mathscr{V} = \sum\limits_i Ax_i$. 
\end{proof}  
   
  \noindent
  (b) $\Rightarrow$ (c), Since $A$ is a Krull ring and every non-zero
  ideal is divisorial, every non-zero prime ideals is of height 1. 
  
\noindent
(c) $\Rightarrow$ (a). Let $\mathscr{V}$ be any fractionary
ideal. Then $\mathscr{V}(A: \mathscr{V}) \subset A$ and since
$D(A)$ is a group, $\mathscr{V} (A : \mathscr{V})$ is Artin-equivalent
to $A$. Hence $\mathscr{V}(A: \mathscr{V})$ is not contained in
any prime divisorial ideal. But by (c). Since every prime ideal
$\neq (0)$ is prime divisorial, $\mathscr{V} (A: \mathscr{V})$ is not
contained in any maximal ideal, and hence $\mathscr{V} (A :
\mathscr{V}) = A$. 
  
\noindent
  (a) $\Rightarrow$ (d).\pageoriginale That $A$ is noetherian is a
consequence of the lemma used in the proof of the implication $(a)
\Rightarrow (b)$. That $A$ is integrally closed and every prime ideal
$\neq (0)$ is maximal is a consequence  of the fact that $(a)
\Rightarrow (c)$.  
   
  \noindent
  (d) $\Rightarrow$ (c). This is an immediate consequence of the
  fact that a noetherian integrally closed domain is a Krull ring. The
  proof of Theorem~\ref{chap1:thm5.1} is now complete. 
  
\setcounter{definition}{1}
  \begin{definition} %def 5.2
$A$ ring $A$ is called a Dedekind ring if $A$ is a domain satisfying
    any one of the equivalent conditions of Theorem~\ref{chap1:thm5.1}. 
  \end{definition}
  
\begin{remark*} % rem 
\begin{enumerate} [(1)]
\item The condition $(c)$ can be restated as: $A$ is a Dedekind ring
  if $A$ is a Krull ring and its Krull dimension is atmost 1. 

\item Let $A$ be a Dedekind ring and $K$ its quotient field. Let $K'$
  be a finite algebraic extension of $K$ and $A'$, the integral
  closure of $A$ in $K'$. Then $A'$ is again a Dedekind ring. 
\end{enumerate}  
\end{remark*}  
  
\begin{proof}
By Proposition~\ref{chap1:prop4.5} it follows that $A'$ is a Krull
ring. For a 
non-zero prime ideal $\mathscr{Y}' $ of $A'$, we have, by the
Cohen-Seidenberg theorem, height $(\mathscr{Y}')$ = height
$(\mathscr{Y} ' \cap A) \le 1$. Now (2) is a consequence of Remark
(1). 
\begin{enumerate}
\item [(3)] Let $A$ be a domain and $\mathscr{U}$, a fractionary
  ideal. Then $\mathscr{U}$ is invertible if and only if $\mathscr{U}$
  is a projective $A$-module. If further, $A$ is noetherian, then
  $\mathscr{U}$ is projective if and only if $\mathscr{U}$
  is locally principal (i.e $\mathscr{U}_\mathscr{M}$ is principal for
  all maximal ideals $\mathscr{M}$ of $\triangle$). 
\end{enumerate}
\end{proof}  
  
  We shall say that a ring $A$ satisfies the \textit{condition} $(M)$
  if $A$ satisfies the maximum condition for principal ideals. For
  instance Krull rings satisfy $(M)$. 
  
\setcounter{theorem}{2}
 \begin{theorem}\label{chap1:thm5.3} % the 5.3
 For\pageoriginale a domain $A$, the following conditions are
 equivalent. 
 \begin{enumerate}[a)]
\item $A$ is a Krull ring and every prime divisorial ideal is
  principal.

\item $A$ is a Krull ring and every divisorial ideal is
  principal. 

\item $A$ is a Krull ring and the intersection of any two principal
  ideals is principal.

\item $A$ satisfies $(M)$ and any elements of $A$ have a least
  common multiple (1.c.m.) (i.e. $A a \cap Ab$ is principal for $a,
  b \in A$). 

\item $A$ satisfies $(M)$ and any two elements of $A$ have
  greatest common divisor $(g.c.d.)$. 

\item $A$ satisfies $(M)$ and every irreducible element of $A$
  is prime. 

(We recall that $a \in A$ is \em{irreducible} if $Aa$ is maximal
  among principal ideals, an element $p \in A$ is \em{prime} if $A~ p$ is
  a prime ideal). 

\item $A$ has the unique factorization property. More precisely, 
  there exists a subset $P \subset A$, $0 \notin P$ such that every $x
  \neq 0$, $x \in A$ can be written in on and only way as $x =
  u. \prod\limits_{p \in P} p^{n(p) }$, $n(p) \ge 0$, $n(p) = 0$ for
  almost all $p$, $u$ being a unit.
 \end{enumerate} 
 \end{theorem}  
 
\begin{proof} 
\begin{enumerate} [(a)]
\item $\Rightarrow$ (b). Since prime divisors generate $D(A)$, we have
  $D(A) = F(A)$. 

\item $\Rightarrow$ (c). Trivial.

\item $\Rightarrow$ (b). Let $\mathscr{V}$ be any divisorial ideal
  $\neq (0)$. 
\end{enumerate}
 \end{proof} 
 
 \noindent
 We shall show that $\mathscr{V}$ is principal. We may assume that
 $\mathscr{V}$ is integral. Let $\mathscr{V} = \bigcap\limits_{\lambda
   \in I} A c_\lambda$, $c_ \lambda \in K$. Consider the set of all
 divisorial ideals\pageoriginale $\mathscr{V}_J =
 \bigcap\limits_{\lambda \in J} A 
 c_ \lambda \bigcap A$, where $J$ runs over all finite subsets $I$. We
 have $\mathscr{V} \subset \mathscr{V}_J \subset A$, for all finite
 subsets $J \subset I$. Since $A$ is a Krull ring, any integral
 divisorial ideal is defined by the inequalities $v(x) \ge n_v$, $n_v >
 0$, $n_v = 0$ for almost all $v$, $v$ running through all essential
 valuations of $A$. Hence there are only finitely many divisorial
 ideals between $\mathscr{V}$ and $A$. Hence $\mathscr{V} =
 \bigcap_{\lambda \in J} A c$, for some finite set $J
 \subset I$. By choosing a suitable common denominator for
 $c_\lambda$, $\lambda \in J$, we may assume that $c_\lambda \in
 A$. Now $(c) \Rightarrow (b)$  is immediate.  
 \begin{enumerate}
\item [(b)] $\Rightarrow (a)$. Trivial. 

\item [(c)] $\Rightarrow (d)$. Trivial.

\item [(d)] $\Rightarrow (e)$. This is an immediate consequence of the
  following elementary property of ordered abelian groups, namely: in
  an ordered abelian group $G$, the existence of sup $(a, b)$ is
  equivalent to the existence of $\inf (a, b)$, for $a$, $b \in
  G$. (Apply this, for instance to $F(A)$) $(e) \Leftrightarrow (f)
  \Leftrightarrow (g)$. This follows from divisibility arguments used
  in elementary number theory. 

\item [(g)] $\Rightarrow (c)$. For $K \in A$, $x \neq 0$, we write $x =
  u_x \prod\limits_{p \in P} p ^{v_p(x)}$, $u$ being a unit. The set of
  all $ \{ v_p \} _{p \in P}$, defines a set $\Phi$ of discrete
  valuations of the quotient field $X$ of  $A$. It is clear that $A$
  satisfies the valuation criterion for Krull rings with $\Phi$ as
  the set of valuations. Further, for $a$, $b \in A$, $A a \cap Ab = Ac$,
  where $c = u_x u_y \prod\limits_{p \in P} p^{\max (v_P (x) v_P
    (g))}$. Hence $(g) \Rightarrow (c)$ and the proof of 
Theorem~\ref{chap1:thm5.3} is complete.  
 \end{enumerate} 
 
\setcounter{definition}{3}
 \begin{definition} % def 5.4
 $A$ is said to be factorial if $A$ is a domain satisfying any
   one of the conditions of Theorem~\ref{chap1:thm5.3}. 
 \end{definition} 
 
\setcounter{remark}{4}
 \begin{remark} % rem 5.5
Let\pageoriginale $A$ be a noetherian domain with the property that
every prime ideal of height 1 is principal. Then $A$ is factorial.  
 \end{remark} 
 
 \begin{proof} % prof
 We shall prove the condition $(f)$. Set $b \in A$ be an irreducible
 element. By Krull's Principal Ideal Theorem, $Ab \subset \mathscr{Y}$,
 $\mathscr{Y}$, a prime ideal of height 1. By hypothesis,
 $\mathscr{Y}$ is principal. Since $Ab$ is maximal among principal
 ideals, $Ab = \mathscr{Y}$. Of course $A$ satisfies $(M)$. 
 \end{proof} 

 
 \section{Divisor class groups}\label{chap1:sec6}%Sec 6 
 Let $A$ be a Krull ring. We recall that the divisor class group 
 $c(A)$ of $A$ is $\dfrac{D(A)}{F(A)}$, where $D(A)$ is the group of
 divisors of $A$ and $F(A)$ the subgroup of $D(A)$ consisting of
 principal divisors of $A$. If $A$ is a Dedekind ring, $C(A)$ is
 called the \textit{group of ideal classes}. By Theorem~\ref{chap1:thm5.3}, 
it is clear that a Krull ring $A$ is factorial if
 and only if $C(A) = 0$.  
  
  Let $A$ and $B$ be Krull rings, with $A \subset B$. From now on we
  shall use the same notation for a prime divisor and the prime
  divisorial ideal corresponding to it. Let $\underline{P}$,
  $\underline{p}$ be prime divisors of $B$ and $A$ respectively. We
  write $\underline{P}  \big| \underline{p}$ if $\underline{P}$ lies
  above $\underline{p}$  i.e. $\underline{P} \cap A =
  \underline{p}$. If $\underline{P} \big| \underline{p}$, the
  restriction of $v_{\underline{P}}$ to the 
  quotient field of $A$ is equivalent to $v_{\underline{p}}$, and we
  denote by $e(\underline{P} , \underline{p}$), the ramification index
  of $v_{\underline{p}}$ in $v_{\underline{P}}$. For a $\underline{p}
  \in P(A)$, we define 
$$
j(\underline{p}) = \sum\limits_{\underline{P} \big| \underline{p}}
e(\underline{P}, \underline{p}) \underline{P}, \underline{P} \in
\underline{P} (B).
$$
The above sum is finite since $x \in \underline{p}$, $x \neq 0$ is
  contained in only finite many $\underline{P}, \underline{P} \in
  \underline{P}(B)$. Extending $j$ by linearity we get a homomorphism
  of $D(A)$ into $D(B)$ which we also denote by $j$. We are
  interested\pageoriginale 
  in the case in which $j$ induces a homomorphism of $\bar{j}: C(A)
  \rightarrow C(B)$ i.e. $j (F (A)) \subset F(B)$. For $x \in A$, we
  write $d_A (x) = d (Ax) \in D(A) $ and $d_B(x) = d (Bx) \in D(B)$. 

  \begin{theorem}\label{chap1:thm6.1} %the 6.1
Let $A$ and $B$ be Krull rings with $A \subset B$. Then we have
  $j(d_A (x)) = d_B(x)$ if and only if the following condition is
  satisfied. $(NBU)$. For every prime divisor $P$ of $B$, height $(P
  \cap A) \le 1$.  
  \end{theorem}  
  
\begin{proof}
\begin{align*}
j (d_A(x) & = j(\sum_{\underline{p} \in P(A)} v_{\underline{p}} (x)
\underline{p})\\ 
& = \sum_{ \underline{p} \in P(A)} v_{\underline{p}} (x)
\sum_{\underline{P} | \underline{p}} e(\underline{P}, \underline{p})
\underline{P} = \sum_{\underline{p}, \underline{P} | \underline{p}}
v_{\underline{P}} (x) \underline{P}\\ 
& = \sum_{\underline{P}, \underline{P} \cap A \in P (A)}
v_{\underline{P}} (x) \underline{P}. 
\end{align*}
\end{proof}  
  
  \noindent
  If $\underline{P} \cap A = (0)$, then $v_{\underline{P}}(x) =
  0$. Therefore, 
  \begin{equation*}
j(d_A (x)) = \sum_{\text{height} (\underline{P} \cap A) \le 1}
v_{\underline{P}} (x) \underline{P}. \tag{1}\label{c1:eq1} 
  \end{equation*}   
  
  \noindent
  Now, if $(NBU)$ is true, then $j(d_A (x)) =
  \sum\limits_{\underline{P} \in P(B)} v_{\underline{P}}(x)
  \underline{P} = d_B (x) $. On the other hand let $j (d_A (x)) = d_B
  (x)$ for every $x \in A$. Let $\underline{P} \in P(B)$ with
  $\underline{P} \cap A \neq (0)$. Choose $x \in \underline{P} \cap A$,
  $x \neq 0$. We have by (1) above 
  $$
  j(d_A (x)) = \sum_{\text{height } (\underline{P} \cap A) \le 1}
  v_{\underline{P}} (x) \underline{P} = d_B(x) = \sum_{\underline{P}
    \in P (B)} v_{\underline{P}} (x) \underline{P}. 
  $$
  
  \noindent
  Since $v_{\underline{P}}(x) > 0$, we have height $(\underline{P}
  \cap A)=1$ and the theorem is proved. 
  
  When $(NBU)$ is true we have $j(F(A)) \subset F(B)$ and
  therefore $j$ induces a canonical homomorphism $\bar{j} : C(A)
  \rightarrow C(B)$. 
  
  We\pageoriginale now give two sufficient conditions in order that
  $(NBU)$ be true. 
  
  \begin{theorem}\label{chap1:thm6.2} %the 6.2
 Let $A$ and $B$ be as in Theorem~\ref{chap1:thm6.1}. Then $(NBU)$ is
  satisfied if any one of the following two conditions are satisfied.
\begin{enumerate}[(1)]
\item $B$ is integral over $A$.

\item $B$ is a flat $A$-module (i.e the functor
  $\underset{A}{\otimes} B$ is exact).  
\end{enumerate}
Further, if (2) is satisfied we have $j(\mathscr{U}) =
    \mathscr{U}B$, for every divisorial ideal $\mathscr{U}$ of $A$. 
  \end{theorem}  
  
\begin{proof} % proof
If (1) is satisfied, $(NBU)$ is an immediate consequence of the
Cohen-Seidenberg theorem. 
\end{proof}  
  
  Suppose now that (2) is satisfied. Let $\underline{P} \in P(B)$
  with $\mathscr{U} = \underline{P} \cap A \neq (0)$. Suppose $\mathscr{U}$ is
  not divisorial. Choose a non-zero element $x \in \mathscr{U}$. Let $d(x)
  = \sum\limits_{i = 1}^{n} v_{p_i}(x) \underline{p}_i
  \underline{p}_i \in P(A)$. Since height $\sigma > 1$ we have $\mathscr{U}
  \nsubset \underline{p}_i $ for $i = 1, \ldots, n$. By an easy
  reasoning on prime ideals there exists $a \; y \in \mathscr{U}$, $y \notin
  \bigcup\limits_{f  = 1}^{n} \underline{p}_i$. Then $d_A (x)$ and
  $d_A (y)$ do not have any component in common and therefore 
  $$
  d_A (xy) = d_A (x) + d_A (y) = {\rm Sup} (d_A (x), d_A(y)). 
  $$

  \noindent
  This, in terms of divisorial ideals, means that $Ax \cap Ay =
  Axy$. Since $B$ is A-flat, we have $Bxy = Bx \cap By$; that is $d_B
  (x) $ and $d_B (y)$ do not have any component in common. But $x$, $y
  \in \underline{P} \cap A$. Contradiction. 
  
  We shall now prove that for any divisorial ideal $\mathscr{U}
  \subset A$, $j (\mathscr{U}) = \mathscr{U} B$. Since $A$ is a Krull
  ring $\mathscr{U}$ is the intersection of finitely many principal
  ideals, say $\mathscr{U} = \bigcap\limits_{i = 1}^n$ $Ax_i$, so that
  $d_A (\mathscr{U}) = \sup (d_A (x_i))$.\pageoriginale Since $B$ is
  A-flat, we have 
  $B \mathscr{U} = \bigcap\limits_{i = 1}^n Bx_i$. 
  Thus $B \mathscr{U}$ is again divisorial. On the other hand, $d_B (B
  \mathscr{U}) = \sup\limits_i (d_B (x_i)) = \sup\limits_i (j (d_A (x
  ))$.  Noting that $j$ is order preserving and that any order
  preserving homomorphism of $\mathbb{Z}^{(I)}$ into
  $\mathbb{Z}^{(J)}$ is compatible with the formation of sup and inf
  (to prove this we have only to check it component - wise), we have 
  $$
  B \mathscr{U} = \sup\limits_i (j (d_A(x_i)) = j(\sup\limits_{i} (d_A
  (x_i))) = j (d_A (\mathscr{U})). 
  $$
   
  \begin{theorem}[Nagata]\label{chap1:thm6.3} % the 6.3
Let $A$ be a Krull ring and $S$, $a$ multiplicatively
      closed set in $A (0 \notin S)$. Consider the ring of quotients
      $S^{-1}A $ (which is A-flat). We have 
\begin{enumerate}[(a)]
\item  $\bar{j} : C(A) \rightarrow C(S^{-1} A)$ is surjective.

\item If $S$ is generated by prime elements then $\bar{j}$ is bijective.
\end{enumerate}
\end{theorem}  
  
\begin{proof}
(a)~ Since $P(S^{-1} A) = \big\{\underline{p}  S^{-1} A \big| 
    \underline{p} \in P (A), \underline{p} \cap S  = \phi \big\},
    \bar{j}$ is surjective by Theorem~\ref{chap1:thm6.2}, (2). 
\end{proof}  
  
  Let us look at the kernel of $\bar{j}$. Let $H$ be the subgroup of
  $D(A)$ generated by prime divisors $\underline{p}$ with
  $\underline{p} \cap S \neq \phi$. Then it is clear that 
  \begin{equation*} 
\Ker (\bar{j}) = \frac{(H + F(A))}{F(A)} \thickapprox \frac{H}{(H \cap
  F(A))} \tag{6.4}\label{c1:eq6.4} 
  \end{equation*}  

  \noindent
  Suppose that $S$ is generated by prime elements. Let $\underline{p}
  \in P(A)$, with $\underline{p} \cap S \neq \phi$, say $s_1 \cdots
  s_n \in \underline{p}$, where $s_i$ are prime elements. Then since
  $\underline{p}$ is minimal $\underline{p} = As_i $ for some
  $s_i$. Thus $H \subset F(A)$ and hence $\bar{j}$ is a bijection. 
  
\setcounter{dashthm}{2}
\begin{dashthm}[Nagata]\label{chap1:thm6.3'}%%% 6.3'
Let $A$ be a noetherian domain and $S$ a
      multiplicatively closed set of $A$ generated by prime
      elements $\big \{\underline{p}_i \big\}_{i \in I}$. If $S^{-1}
      A$ is a Krull ring then $A$ is a Krull\pageoriginale ring and
      $\bar{j}$ is 
      bijective.
\end{dashthm}
  
\begin{proof}
By virtue of Theorem~\ref{chap1:thm6.3}, we have only to prove that
$A$ is 
integrally closed (then it will be a Krull ring, since it is
noetherian). Now $A_{Ap_i}$ is a local noetherian domain whose
maximal ideal is principal and hence $A_{Ap_i}$ is a discrete
valuation ring. It suffices to show that $A = S^{-1} A \cap
(\bigcap\limits_{i \in I}A_{Ap_i})$. We may assume $Ap_{i}
\neq Ap_j$ for $i \neq j$. Let $a/s \in S^{-1} A \cap
(\bigcap\limits_{i} A_{Ap_{i}})$, $a \in A$, $s \in S$, $s = \prod_i
p_i^{n (i)}$. We have $v_{p_i} (a/s) \ge 0$, where $v_{p_i}$ is the
valuation corresponding to $A_{A_{p_{i}}}$. By our assumption,
$v_{p_i}(p_j) = 0$ for $j \neq i$. Hence $v_{p_i} (a) \ge v_{p_i}(s) =
n(i)$. Hence $S$ divides a i.e. $a/s \in A$. 
  \end{proof}  
  
\begin{coro*} % coro
Let $A$ be a domain and $S$ a multiplicatively closed set generated by
a set of prime elements. Let $S^{-1} A$ be factorial. If $A$ is
noetherian or a Krull ring, then $A$ is factorial. 
\end{coro*}  
  
\begin{proof}
By Theorems~\ref{chap1:thm6.3} and \ref{chap1:thm6.3'}, $\bar{j} :
C(A) \rightarrow C(S^{-1}A)$ is 
bijective. 
\end{proof} 
   
\setcounter{theorem}{3}
  \begin{theorem}[Gauss]\label{chap1:thm6.4} % the 6.4
 Let $R$ be a Krull ring. Then $\bar{j} : C(R) \rightarrow C(R [X])$
 is bijective. In particular, $R$ is factorial if and only if $R[X] $
 is factorial. 
  \end{theorem}  
  
(Since $R[X] $ is $R$-flat, $\bar{j}$ is defined).
  
\begin{proof}
Set $A = R[X]$, $S = R^*$, the set of non-zero elements of $R$. Then
$S^{-1} A = K [X]$, where $K$ is the quotient field of $R$. Thus
$C(S^{-1} A) = 0$ i.e. 
  $$
  C(A) = \Ker (\bar{j} : C(A) \rightarrow C(S^{-1}A)) = \frac{(H +
    F(A))}{F(A)} 
  $$
  where\pageoriginale $H$ is the subgroup of $D(A)$ generated by $P
  \in P(A)$, with 
  $P \cap R \neq (0)$ (see formula 6.4). Hence $D(A) = H +
  F(A)$. Since $R[X]$ is $R$-flat, by Theorem~\ref{chap1:thm6.2}, (2)
  we have  
  $$
  j(P \cap R) = (P \cap R) R[X] = P, \text{ for } P \in P(A), P \cap R \neq (0).
  $$
  
  \noindent
  Hence $j (D(R)) = H$ and therefore $\bar{j}$ is surjective, since
  $D(A) = H + F(A)$. Now an ideal $\mathscr{U}$ of $R$ is principal if
  and only if $\mathscr{U} R [X]$ is principal. Therefore $\bar{j}$ is
  injective. Thus $\bar{j}$ is bijective. 
  
  Let $A$ be a noetherian ring and $\mathscr{M}$ an ideal contained in
  the radical of $A$ (i.e. the intersection of all maximal ideals of
  $A$). If we put on $A$ the $\mathscr{M}$-adic topology, then
  $(A, \mathscr{M})$ is called a  \textit{Zariski ring}. The
  completion $\hat{A}$ of $A$ is again a Zariski ring and it is well
  known that $\hat{A}$ is $A$-flat and $A \subset \hat{A}$. 
\end{proof}

\setcounter{theorem}{4}
\begin{theorem}[Mori]\label{chap1:thm6.5} % the 6.5
Let $(A, \mathscr{M})$ be a Zariski ring. Then if
    $\hat{A}$ is a Krull ring, then so is $A$. Further $j : C(A)
    \rightarrow C(\hat{A})$ is injective. In particular if $\hat{A}$
    is factorial, so is $A$. 
  \end{theorem}  
  
\begin{proof}
Let $K$ and $L$ be the quotient fields of $A$ and $\hat{A}$
respectively; $K \subset L$. To prove that $A$ is a Krull ring we
observe that $A = \hat{A} \cap K$. For if $\dfrac{a}{b} \in \hat{A}
\cap K$, $a$, $b \in A$, then $a \in \hat{A}b \cap A = Ab$,
i.e. $\dfrac{a}{b} \in A$. Hence $A$ is a Krull ring. By virtue of
Theorem~\ref{chap1:thm6.2} (2), to prove that $\bar{j}$ is an
injection it is 
enough to show that an ideal $\mathscr{U}$ of $A$, is principal if
$\hat{A} \mathscr{U}$ is principal. Let $\hat{A} \mathscr{U} = \hat{A}
\alpha$, $\alpha \in \hat{A}$. Now $\dfrac{\mathscr{U}}{\mathscr{M}
  \mathscr{U}} \thickapprox \dfrac{\hat{A} \mathscr{U}}{\hat{A}
  \mathscr{M} \mathscr{U}}$ is generated by a single element as an
$\dfrac{A}{\mathscr{M}}$-module say by $x (\mod \mathscr{M}
\mathscr{U})$, $x \in \mathscr{U}$. Then $\mathscr{U} = Ax + \mathscr{M}
\mathscr{U}$. By Nakayama's-lemma $\mathscr{U} = Ax$ and the theorem
is\pageoriginale proved. 
\end{proof}  
  

\section{Applications of the theorem of Nagata}\label{chap1:sec7}%Sec 7 
    We recall that a ring $A$ is called \textit{graded} if $A =
  \sum\limits_{n \in \mathbb{Z}} A_n$, $A_n$  being abelian groups
  such that $A_p ~ A_q \subset A_{p+q}$, for $p$, $q \in \mathbb{Z}$,
  and the sum being direct. An ideal of $A$ is \textit{graded} if it
  generated by homogeneous elements. 

\begin{prop}\label{chap1:prop7.1} % prop 7.1
Let $\Lambda$ be a graded Krull ring. Let $DH(A)$ denote the 
  subgroup of $D(A)$ generated by graded prime divisorial ideals and 
  let $FH(A)$ denote the subgroup of $DH(A)$ generated by principal
  ideals. Then the canonical mapping $\dfrac{DH(A)}{FH(A)} \rightarrow
  C(A)$, induced by the inclusion $i : DH(A) \rightarrow D(A)$, is an
  isomorphism. 
\end{prop}  
  
\begin{proof}
If $A = A_o$, there is nothing to prove. Hence we may assume $A \neq
A_o$. Let $S$ be the set of non-zero homogeneous elements of $A$. Then
$S^{-1} A$ is again a graded ring; infact $S^{-1} A = \sum\limits_{j
  \in \mathbb{Z}} (S^{-1} A)_j$, where $(S^{-1}A)_j =
\big\{\dfrac{a}{b} \big| a, b \in A, a, b$ homogeneous, $d^o a - d^o
b = j \big \}$. 
  \end{proof}  
  
  \noindent
  Here $d^o x$ denotes the degree of a homogeneous element $x \in A$. We
  note that $(S^{-1} A)_o = K$ is a field and that $S^{-1} A
  \thickapprox K \big [t, t^{-1 } \big ]$, where $t$ is a homogeneous
  element of smallest strictly positive degree. Now $t$ is
  transcendental over $K$ and therefore $S^{-1}A$ is factorial. Hence
  $C(A) \thickapprox Ker (\bar{j})$, where $\bar{j}$ is the canonical
  homomorphism $\bar{j} : C(A) \rightarrow C(S^{-1} A)$, and
  $C(S^{-1}A) = 0$. 
  
  Hence $C(A) \thickapprox \dfrac{H}{(H \cap F(A))}$, where $H$ is the
  subgroup of $D(A)$, generated\pageoriginale by prime ideals
  $\mathscr{Y}$ of 
  height 1 with $\mathscr{Y} \cap S \neq \phi$. Since $DH(A) \cap
  F(A) = FH(A)$, and since prime divisorial ideals are of height 1,
  the proposition is a consequence of the following 


\setcounter{lemma}{1}
  \begin{lemma} %lem 7.2
Let $A = \sum\limits_{n \in \mathbb{Z}}A_n$ be a graded ring and
  $\mathscr{Y}$ a prime ideal in $A$ and let $\mathscr{U}$ be the ideal
  generated by homogeneous elements of $\mathscr{Y}$. Then $\mathscr{U}$ is
  a prime ideal. 
  \end{lemma}  
  
  \begin{proof} 
Let $x y \in \mathscr{U}$, $x = \sum x_i$, $y = \sum y_i$, $x \notin
\mathscr{U}$, $y 
\notin \mathscr{U}$. Let $x_{i_o} y_{j_o}$ be the lowest components of $x$,
$y$ such that $x_{i_o} \notin \mathscr{U}$, $y_{j_o} \notin \mathscr{U}$. Then
$x_{i_o} y_{j_o} \in \mathscr{U} \subset \mathscr{Y}$. Since $\mathscr{Y}$
is prime, $x_{i_o}$ or $y_{j_o} \in \mathscr{Y}$, say $x_{i_o} \in
\mathscr{Y}$. Then $x_{i_o} \in \mathscr{U}$, a contradiction. 
\end{proof}  

\begin{coro*} % coro
Let $A = \sum\limits_{n \in \mathbb{Z}} A_n$ be a graded ring and
$\mathscr{Y}$ a prime ideal of height 1. Then $\mathscr{Y}$ is
graded if and only if $\mathscr{Y} \cap S \neq \phi$. 
\end{coro*}  
  
\begin{remark*} % rem
If $\mathscr{U}$ is a graded ideal of $A$, then the least divisorial
ideal $A : (A : \mathscr{U})$ containing $\mathscr{U}$ is also graded
(straight forward proof). 
Thus the divisors corresponding to graded divisorial ideals of $A$
form a subgroup of $D(A)$; this subgroup obviously contains $DH(A)$;
furthermore, since, given a graded integral divisorial ideal
$\mathscr{U}$, all the prime divisorial ideals containing
$\mathscr{U}$ are graded (by the corollary), we see that this subgroup
is in fact $DH(A)$. 
\end{remark*}

The above proposition can be applied for instance to the homogeneous
coordinate ring of a projective variety. The following proposition
connects the divisor class group of a projective variety $V$ with
the  divisor class group of a suitable affine open subset of $V$. 

\setcounter{prop}{2}
\begin{prop}\label{chap1:prop7.3} % prop 7.3
Let\pageoriginale $A$ be a graded Krull ring and $p$ a prime homogeneous
   element $\neq 0$ with $d^o p = 1$. Let $A'$ be the subring of $K$
   (quotient field of $A$) generated by $A_o$ and $\dfrac{a}{p^{d^oa}
     }$, where $a$ runs over the non-zero homogeneous elements of
   $A$. Then $C(A') \thickapprox C(A)$. 
\end{prop}  
  
\begin{proof}
We note that $A' = (S^{-1} A)_o$ and that $p$ is transcendental
over $A'$. The inclusions $A' \rightarrow A'[p] \rightarrow A' \big[p,
  p^{-1}\big]$ induce isomorphisms $C(A') \thickapprox C(A' [p]), C(A'
[p]) \thickapprox C(A' [p, p^{-1}])$, the first isomorphism follows
from Theorem~\ref{chap1:thm6.4} and the second from 
Theorem~\ref{chap1:thm6.3}. Now $A 
\big[p^{-1}\big] = A' \big[p,\break p^{-1}\big]$. But again by 
Theorem~\ref{chap1:thm6.3}, $C(A) \thickapprox C(A \big[p^{-1} \big])$ and the
proof of the proposition is complete. 
\end{proof}  
  
  Let $V$ be an arithmetically normal projective variety. We prove
  that the homogeneous coordinate ring of $V$ is factorial if and only
  if the local ring of the vertex of the projecting cone is factorial;
  in fact we have the following 
  

\begin{prop}%prop 7.4
Let $A = A_o + A_1 +A_2 + \cdots$ be a graded Krull ring\break and
  suppose that $A_o$ is a field. Let $\mathscr{M}$ be the maximal
  ideal $A_1 + A_2  + \cdots$. Then $C(A) \thickapprox C(A
  \mathscr{M}$). 
\end{prop} 
   
\begin{proof}
We have only to prove that $\bar{j} : C(A) \rightarrow C(A
\mathscr{M})$ is injective. 
 Because of Proposition~\ref{chap1:prop7.1} and of the remark following it is
 sufficient to prove that if $\mathscr{V}$ is a graded divisorial
 ideal such that $\mathscr{V} A_{\mathscr{M}}$ is principal, then so
 is $\mathscr{V}$. Suppose that $\mathscr{V}$ is a graded divisorial
 ideal with $\mathscr{V} A_{\mathscr{M}}$ principal. Since
 $A_{\mathscr{M}}$ is a local ring, there is a homogeneous element $u
 \in \mathscr{V}$ such that $\mathscr{V} A_{\mathscr{M}} = u
 A_{\mathscr{M}}$. Let $x \in \mathscr{V}$ be any homogeneous
 element. Then $x = \dfrac{y}{z} u$, $y \in A$, $z \in A -
 \mathscr{M}$. Let   
\begin{align*}
& y = y_q + y_{q+1} + \cdots, z = z_o + z_1 + z_2 + \cdots , y_i \in
    A_i, i \ge q, y_q \neq 0,\\ 
& z_j \in A_j, z_o \neq 0. \text{ Thus } x(z_o + z_1 +  z_2 + \cdots )
    = (y_q + y_{q+1} + \cdots ) u. 
\end{align*}\pageoriginale  
  
  \noindent
  Hence $xz_o = y_q u$. Since $z_o$ is invertible, we conclude that
  $\mathscr{V} = An$ and the proposition is proved.  
  \end{proof}


\medskip  
\noindent
\textbf{Adjunction of indeterminates}.

  Let $A$ be a local ring and $\mathscr{M}$ its maximal ideal. We set
  $A(X)_{\loc} = A [X]_{\mathscr{M} A [X]}$ and by induction $A(X_1,
  \ldots , X_n)_{\loc} = A(X_1, \ldots , X_{n-1})_{\loc}\break
  (X_n)_{\loc}$. We remark that $A(X)_{\loc}$ is a local ring and that
  if $A$ is noetherian, properties of $A$ like its dimension,
  multiplicity, regularity and so on are preserved in passing from $A$
  to $A(X)_{\loc}$. Further $A(X)_{\loc}$ is A-flat (in fact it is
  faithfully flat). 
  
\begin{prop}\label{chap1:prop7.5}%prop 7.5
Let $A$ be a local Krull ring. Then 
$$\bar{j} : C(A) \rightarrow
  C(A(X_1, \ldots , X_n)_{\loc})$$ 
is an isomorphism. 
\end{prop}  
  
\begin{proof}
It is sufficient to prove this when $n = 1$. Since by 
Theorem~\ref{chap1:thm6.5} 
$C(A) \thickapprox C(A [X])$, we see that $\bar{j}$ is surjective. Let
$\mathscr{V}$ be a divisorial ideal of $A$ for which $\mathscr{V}
A(X)_{\loc}$ is principal. Since $A(X)_{\loc}$ is a local ring, we
may assume that $\mathscr{V} A(X)_{\loc} = A(X)_{\loc} \alpha$, $\alpha
\in \mathscr{V}$. Let $y \in \mathscr{V}$. Then $y =
\dfrac{f(X)}{g(X)}. \alpha$, where $f(X)$, $g(X) \in A[ X]$ and atleast
one of the coefficients of $g(X)$ is invertible in $A$. Looking at a
suitable power of $X$ in $y.g(X)= \alpha f(X)$ we see that $y \in A
\alpha $ i.e. $\mathscr{V} = A \alpha$. Hence $\bar{j}$ is
injective. This proves the proposition. 
\end{proof}  
  
\begin{prop} % prop 7.6
Let\pageoriginale $A$ be a domain and $a$, $b \in A$ with $Aa \cap Ab
= Aab$. 
\end{prop}  
  
\noindent
The following results hold.
\begin{enumerate}[(a)]
\item The element $aX-b$ is prime in $A[X]$. 

\item If further, we assume that $A$ is a noetherian
  integrally closed domain and that $Aa$ and $Aa + Ab$ are prime
  ideals, then the ring $A' = \dfrac{A[X]}{(a X-b)}$ is again
  integrally closed and the groups $C(A)$ and $C(A')$ are canonically
  isomorphic. 
 \end{enumerate}  

\begin{proof}
\begin{enumerate}
\item[(a)] Consider the $A$-homomorphism $\varphi : A [X] \rightarrow
  A \big[\dfrac{b}{a} \big]$ given by $\varphi(X) = \dfrac{b}{a}$. It
  is clear that the ideal $(a X -b) \subset \Ker (\varphi)$. Conversely
  we show by induction on the degree that if a polynomial $P(X) \in
  \Ker (\varphi)$, then $P(X) \in (a X-b)$. This is evident if $d^o (P)
  = 0$. If $P(X) = c_n X^n + c_{n-1} X^{n-1} + \cdots + c_o (n > 0)$,
  the relation $P(\dfrac{b}{a}) = 0$, shows that $b^n c_n \in
  Aa$. Since $Aa \cap Ab = Aab$, it follows that $c_n \in Aa$, say
  $c_n = d_n a$, $d_n \in A$. Then the polynomial $P_1(X) = P(X) - d_n (a
  X-b) X^{n-1} \in \Ker (\varphi)$ and has degree $\le n - 1$. By
  induction we have $P_1(X) \in (a X -b)$ and hence $P(X) \in (a
  X-b)$. Thus $(a X-b) = \Ker (\varphi)$ and $(a)$ is proved. 

\item We note that $A' \thickapprox A \big[ \dfrac{b}{a}\big] \subset
  A \big[\dfrac{1}{a}\big ] $ and $A \big[ \dfrac{1}{a}\big]
  \thickapprox A' \big[ \dfrac{1}{a}\big]$. By 
Theorem~\ref{chap1:thm6.3'}, the 
  proof of $(b)$ will be complete if we show that $a$ is a prime element
  in $A'$. But 
$$
\dfrac{A'}{A' a} \thickapprox \dfrac{A[X]}{(a, a X-b)} =  
\dfrac{A[X]}{(a, b)} \thickapprox \dfrac{A}{(a, b)} \big [ X \big]. 
$$
\end{enumerate}
  \end{proof}  
  
\noindent
By assumption $(a, b)$ is a prime ideal and therefore a is a prime
element in $A'$.\pageoriginale 

\begin{rem}% rem 1
In (b), if $a$, $b$ are contained in the radical of $A$, and if the
ideal $Aa+ Ab$ is prime, then $a$ and $b$ are prime elements (for
proof see $P$. Samuel: Sur les anneaux factoriels,
Bull. Soc. math. France, 89 (1961),  155-173). 
\end{rem}

\begin{rem} % rem 2
Let $A$ be a noetherian integrally closed local domain and let the
elements $a$, $b \in A$ satisfy the hypothesis of the above proposition.  
Set $A'' = ^{A(X)} \loc / (aX-b)$. Then it follows from (a) that
$A''$ is a Krull ring. We have a commutative diagram 
\[
\xymatrix{
A \ar[dr]\ar[r]^{\alpha} & A (X)_{\log} \ar[r]^{\beta} & A''\\
& A[X]/_{(aX-b)} \ar[ur]
}
\]

\noindent
Since $A''$ is a ring of quotients of $A [X]/_{(a X -b)}$, it follows
from (b) that $\beta o \alpha$ induces a surjective mapping
$\varphi : C(A) \rightarrow C(A'')$. We do not know if $\varphi $ is
an isomorphism. If $\varphi$ is an isomorphism we can get another
proof of the fact that a regular local ring is factorial (see
P. Samuel : Sur les anneaux factoriels, Bull. Soc. math. France,
t.89, 1961).  
\end{rem}

\begin{prop}[C.P. Ramanujam] % prop 7.7
Let $A$ be a noetherian analytically normal
      local ring and let $\mathscr{M}$ be its maximal ideal. Let $B =
      A \Big[\big[ X_1 , \ldots ,\break X_n \big]\Big]$. Then the
      canonical mapping $j : C(B) \rightarrow C(M_{\mathscr{M} B}$) is
      an isomorphism. 
\end{prop}  
  
\begin{proof}
By Theorem~\ref{chap1:thm6.3} (a), $\bar{j}$ is surjective and $\Ker(j) =
{(H+F(B))}/\break F(B)$, where $H$ is the subgroup of $D(B)$ generated by
prime ideals $\mathscr{Y}$ of height one in $B$ with $\mathscr{Y}
\nsubset \mathscr{M} B$ and $F(B)$ is the group of principal
ideals. Thus we have\pageoriginale only to prove that if $\mathscr{Y}$
is a prime 
ideal of height one of $B$ with $\mathscr{Y} \nsubset \mathscr{M} B$,
the then $\mathscr{Y}$ is principal. This, we prove in two steps.  
 \begin{enumerate}[(i)]
\item Assume that $A$ is complete. Let $\mathscr{Y}$ be a prime ideal
  of height one with $\mathscr{Y} \nsubset \mathscr{M} B$. It is clear
  that $\mathscr{Y}$ is generated by a finite number of elements $f_j
  \in \mathscr{Y}$ such that $f_j \notin \mathscr{M} B$. Set $R = A
  \Big [\big [X_1, \ldots , X_{n-1} \big] \Big]$ and let
  $\mathscr{M} (R)$ denote the maximal ideal of $R$. We claim that any
  $f \in B - \mathscr{M} B$ is an associate of a polynomial $g(X_n) =
  X^q_n + a_{q-1} X^{q-1}_n + \cdots + a_o$, $a_i \in \mathscr{M} (R)$. To prove
  this we first remark that by applying an A-automorphism of $B$ given
  by $X_i \rightsquigarrow$  $X_i + X^{t(i)}_n, t = 1, \ldots , n-1,
  X_n \rightsquigarrow X_n $ with $t(i)$ suitably chosen, we may
  assume that the series $f_j$ are regular in $X_n$, (for details
  apply Zariski and P.Samuel : Commutative algebra p.147, Lemma~3 to
  the product of the $f_j's$). Now since the Weierstrass 
  Preparation Theorem is valid for the ring of formal power series
  over a complete local ring, it follows that $f = u(X^q_n + a_{q-1}
  X^{q-1}_n + \cdots + a_o)$, $u$ invertible in $B = R \Big[\big [X_n
      \big]\Big]$, $a_i \in \mathscr{M} (R)$ and $q$ being the order
  of $f$ $\mod \mathscr{M}(R)$. Thus $\mathscr{Y}$ is generated by
  $\sigma = \mathscr{Y} \cap R [X_n]$. Now, since $B$ is $R [X_n]$ -
  flat it follows by Theorem~\ref{chap1:thm6.2}, (2), that $\sigma$ is
  divisorial. Now by Proposition~\ref{chap1:prop7.5}  $C(R [X_n] ) \rightarrow
  C(R(X_n)_{\loc})$ is an isomorphism. Since $\sigma \nsubset
  \mathscr{M} (R) ~ R [X_n]$, it follows that $\sigma $ is
  principal. Hence $\mathscr{Y}$ is principal. 

  \item Now we shall deal with the case in which $A$ is not
    complete. The completion $\hat{B}$ of $B$ is the ring $\hat {A}
    \Big [\big [Y_1, \ldots , Y_n \big] \Big]$. Let $\mathscr{Y}$ be a
    minimal prime ideal of $B$, with $\mathscr{Y} \nsubset
    \mathscr{M}_B$. Since $\hat{B}$ is $B$-flat, the ideal
    $\mathscr{Y} \hat{B}$ is 
    divisorial. Furthermore, since $\mathscr{Y} \nsubset \mathscr{M}
    B$, all the components $k$ of $\mathscr{Y}\hat{B}$ are such that
    $k \nsubset \mathscr{M} \hat{B}$, and therefore
    principal by $(i)$. Thus $\mathscr{Y} \hat{B}$ is principal. Hence
    $\mathscr{Y}$\pageoriginale is principal by Theorem~6.6. 
  \end{enumerate}
  \end{proof}

\section{Examples of factorial rings}\label{chap1:sec8}%Sec 8
  
\begin{theorem}\label{chap1:thm8.1} % the 8.1
Let $A$ be a factorial ring. Let $A \big[X_1 , \ldots , X_n \big]$ be
graded by assigning weights $\omega_i$ to $x_i ~ (\omega_i > 
  0)$. Let $F(X_1, \ldots ,  X_n)$ be an irreducible isobaric
  polynomial. Let $c$ be a positive integer prime to $\omega$, the
  weight of $F$. Set $B = A \big[X_1 , \ldots X_n, Z\big ]/(Z^c
  - F(X_1, \ldots , X_n)) ~ = A \big [x_1, \ldots , x_n, z\big ]$, $z^c
  = F(x_1, \ldots , x_n)$. Then $B$ is factorial in the following two
  cases. 
\begin{enumerate}[(a)]
\item $c \equiv 1 (\mod \omega)$

\item Every finitely generated projective A-module is free.
\end{enumerate}    
\end{theorem}   
     
\begin{proof}
\begin{enumerate}[(a)]
\item Since $B_{/z B} \thickapprox A \big [X_1, \ldots , X_n, Z] /
  (Z^c - F, Z) \thickapprox   A \big [X_1, \ldots , X_n \big] /\break
  (F)$, it follows that $z$ is prime in $B$. Now, set $x_i = z^{d
  \omega_i} x'_i$, where $c = 1 + d \omega$. Then $z^c = F(x_1, \ldots
  , x_n) = z^{c - 1} F(x'_1, \ldots , x'_n)$, i.e. $z = F(x'_1, \ldots
  , x'_n)$ so that $B \big[ z^{-1 } \big] = A \big [ x'_1, \ldots ,
  x'_n, F(x'_1, \ldots , x'_n)^{-1} \big]$. Since $x'_1, \ldots ,
  x'_n$ are algebraically independent over $A$, we see that $B \big
  [z^{-1} \big] $ is factorial. Now $B = B \big [ z^{-1}\big] \cap K
  \big[ x_1, \ldots , x_n, z \big]$, where $K$ is the quotient
  field of $A j$ for, let $\dfrac{y}{z^r} \in  K \big[x_1, \ldots , x_n,
    z \big]$ with $y \in B$. Then since $(z^r)$ is a primary ideal not
  intersecting $A$, we have $y \in B z^r = B \cap K \big[ x_1,
    \ldots , x_n, z \big] z^r$. Hence $B$ is a Krull ring and
  therefore factorial by Theorem~\ref{chap1:thm6.3'}. 

\item Since $c$ is prime to $\omega$, there exists a positive integer
  $e$ such that $c ~ e \equiv 1 (\mod \omega)$. Now by $(a) ~ B' = A
  \big[ x_1, \ldots , x_n, u \big]$, with $u^{ce} = F (x_1, \ldots
  , x_n)$ is factorial. Further $B' = B [u]$, $u^e = z$ and $B' $ is a
  free B-module with $1, u, u^2, \ldots , u^{e-1}$ as a basis. It
  follows that $B$ is the intersection of $B'$ and of the quotient
  field of $B$, and is therefore\pageoriginale a Krull ring. Now $B$
  can be graded 
  by attaching a suitable weight to $Z$. Let $\mathscr{U} $ be a
  graded divisorial ideal. Since $B'$ is factorial, $\mathscr{U} B'$ is
  principal. As $B'$ is free over $B$, $\mathscr{U}$ is a projective
  $B$-module. Now by Nakayama's lemma for graded rings it follows that
  $\mathscr{U}$ is free and therefore principal. The proof  of (b)
  is complete. 
\end{enumerate}
\end{proof} 
 
      
\begin{examples*}
\begin{enumerate}[(1)]
\item Let $a$, $b$, $c$ be positive integers which are pairwise relatively
  prime. Let $A$ be a factorial ring. Then the ring $B = A \big [x, y,
    z \big]$, with $z^c = x^a + y^b$, is factorial. 

\item Let $\mathbb{R}$ denote the field of real numbers. Then the ring
  $B = \mathbb{R} \big[ x, y, z \big]$ with $z^3 = x^2 + y^2$ is
  factorial. 
\end{enumerate}
\end{examples*}

\begin{theorem}[Klein-Nagata]%% 8.2
 Let $K$ be a field of characteristic $\neq 2$ and $A = K \big [x_1, \ldots ,
      x_n \big]$ with $F (x_1, \ldots , x_n) = 0$, where $F$ is a
    non-degenerate quadratic form and $n \ge 5$. Then $A$ is
    factorial. 
\end{theorem}

\begin{proof}
Extending the ground field $K$ to  a suitable quadratic extension $K'$ if
necessary, the quadratic form $F (X_1, \ldots , X_n)$ can be
transformed into $X_1 ~ X_2 - G(X_3, \ldots , X_n)$. Let $A' = K'
\underset{K}{\otimes} A = K' \big [x_1, \ldots , x_n \big]$, $x_1x_2 =
G(x_3, \ldots , x_n)$. Since $F$ is non-degenerate and $n \ge 5$, $G
(X_3, \ldots , X_n)$ is irreducible and therefore $x_l$ is a prime
element in $A'$. Now $A' \big [\dfrac{1}{x_1}\big] = K' \big [ x_1,
  x_3, \ldots , x_n , \dfrac{1}{x_1}\big]$. 
\end{proof}

\noindent
Since $x_1, x_3, \ldots , x_n$ are algebraically independent, it
follows from Theorem~\ref{chap1:thm6.3'} that $A'$ is factorial. Now
as $A'$ is 
A-free, for any graded divisorial ideal $\mathscr{U}$ of $A$,
$\mathscr{U} A'$ is divisorial and hence principal. Therefore
$\mathscr{U}$ is a projective ideal. Since a finitely generated graded
projective module is free over $A$, we conclude that $\mathscr{U}$
is principal. Thus\pageoriginale $A$ is factorial. 

\setcounter{rem}{0}
\begin{rem} % rem 1
The above theorem is not true for $n \le 4$. For instance, $A = K \big
[ x_1, x_2, x_3, x_4 \big]$ with $x_1 x_2 = x_3 x_4$ is evidently not
factorial.  
\end{rem}

\begin{rem} % rem 2
 We have proved that if $A$ is a homogeneous coordinate ring over a
 field $K$ such that $K' \bigotimes\limits_{K} ~ A$ is factorial for
 some ground field extension $K'$ of $K$, then $A$ is factorial. This
 is not true for affine coordinate rings (see the study of plane
 conics later in this section). 
\end{rem}

\begin{rem} % rem 3
The above theorem is a particular case of theorems of Severi, Lefshetz
and Andreotti, which in turn are particular cases of the following
general theorem proved by Grothendieck. 
\end{rem}

\begin{theorem*}
{\em (Grothendieck)}. 
 Let $R$ be a local domain which is a complete intersection
   such that $R_{\mathscr{Y}}$ is factorial for every prime ideal
   $\mathscr{Y}$ with height $\mathscr{Y} \le 3$. Then $R$ is
   factorial. 
\end{theorem*}

\noindent
(We recall that $R$ is a complete intersection if $R =
A / \mathscr{U}$, where $A$ is a regular local ring and
$\mathscr{U}$ an ideal generated by an A-sequence). 
(For proof of the above theorem see Grothendieck: Seminaire de
Geometric algebrique, expos\'e XI,  IHES (Paris), 1961-62). 

\medskip
\noindent
\textbf{Study of plane conics}. Let $C$ be a projective non singular
curve over a ground field $K$. Let $A$ be the homogeneous coordinate
ring of $C$. The geometric divisors of $C$ can be identified with
elements of $D ~ H (A)$. Then $F H(A) = G_1 (C) + \mathbb{Z}h$, where
$G_l (C)$ denotes set of divisors of $C$ linearly equivalent to zero
and $h$ denotes a hyper plane section. Let now $C$ be a conic in the
projective plane $P^2$. Since the genus of  $C$ is zero we have $G_l
(C) = G_o (C)$ where $G_o (C)$ is the set of divisors of
degree\pageoriginale zero 
of $C$. (i.e. its Jacobian variety is zero). Let $d$ denote the
homomorphism of $D H (A)$ into $\mathbb{Z}$ given by $d (\mathscr{U})=
$ degree of $\mathscr{U}$, for $\mathscr{U} \in D H(A)$. Then $d^{-1}
(2 \mathbb{Z}) = G_o (C) + \mathbb{Z} h = G_l (C) + \mathbb{Z}h = F H
(A)$. Hence $C(A) \thickapprox I md _{/ 2 \mathbb{Z}}$. Thus $A$ is
factorial if and only if $I md= 2 \mathbb{Z}$. 

Suppose $C$ does not carry any K-rational points. Then $A$ is
factorial. For if not, $C(A) \thickapprox {\mathbb{Z}}/(2)$ and
there exists a divisor $\mathscr{U} \in D H(A)$ with $d (\mathscr{U})
= 1$. By the Riemann-Roch Theorem, we have $l(\mathscr{U}) \ge
d(\mathscr{U}) - g + 1=2$, where $l(\mathscr{U})$ denotes the
dimension of the vector space of functions $f$ on $C$ with $(f) +
\mathscr{U} \ge 0$. Thus there exists a function $f$ on $C$ with $(f)
+ \mathscr{U} \ge 0$ and thus we obtain a positive divisor of degree
1, i.e. $C$ carries a rational point: Contradiction. Conversely if
$C$ carries a rational point $P$, then $P$ is a divisor of degree 1
and $C(A) \thickapprox {Z}/(2)$ i.e. $A$ is not factorial. Thus we
have proved (a) The homogeneous coordinate ring $A$ of $C$ is
factorial if and only if $C$ does not have rational points over $K$. 
  
  Let now $C'$ be a conic in the affine place over $K$. Let $A'$ be
  its coordinate ring. Let $C$ be its projective closure in $P^2$. Let
  $I$ be the subgroup of $D H(A)$ generated by the divisors at
  infinity ($A$ is the homogeneous coordinate ring of $C$). Then
  $C(A') \thickapprox D H (A)/ (FH(A) + I)$. Thus 
\begin{enumerate}[(i)]
\item if $C$ has no rational points over $K$, then by (a) $D H (A) = F
  H(A)$ and therefore $C(A') = 0$, so that $A'$ is factorial;
 
\item if $C$ has rational points over $K$ at infinity, then $DH(A) =
  FH(A) + I$ and $A'$ is factorial; 

\item if $C$\pageoriginale has rational points, but not at infinity,
  then $I \subset FH (A)$ and $C(A') \thickapprox C(A) \thickapprox 
  \mathbb{Z}_{/_{(2)}}$; in this case $A$ is not factorial. 
  \end{enumerate}
  
\begin{examples*} % exam
\begin{enumerate} [(i)]
\item $C' \equiv x^2 + 2y^2 + 1 =0$ over the rationals. Then $A'$ is
  factorial. However the coordinate ring of $C'$ over $\mathbb{Q}$ (i) is not
  factorial. 

\item $C' \equiv x^2 + y^2 - 1 = 0$. The coordinate ring of $C'$ over
  $\mathbb{Q}$ is not factorial. But the coordinate ring of $C'$ over
  $\mathbb{Q}$ (i) is factorial. 
\end{enumerate}
    \end{examples*}
    
    The above examples show that unique factorization is preserved
    neither by ground field extension nor by ground field
    restriction. 
    
\medskip
\noindent\textbf{Study of the real sphere}. Let $\mathbb{R}$ denote 
         the field of real numbers and $\mathbb{C}$, the field of
         complex numbers. We shall consider the coordinate ring of the
         sphere $X^2+Y^2+Z^2 = 1$ over $\mathbb{R}$ and $\mathbb{C}$.     

\setcounter{prop}{2}
\begin{prop} %prop 8.3
\begin{enumerate}[(a)]
\item The ring $A = \mathbb{R} \big[ x, y, z \big], x^2 + y^2 +
  z^2 = 1$ is factorial 
 
\item The ring $A = \mathbb{C} \big[x, y, z \big], x^2+y^2+z^2 =
  1$ is not factorial. 
\end{enumerate}
\end{prop}

\begin{proof}
\begin{enumerate}[(a)]
\item We have $A/ (z-1) \approx \mathbb{R} \big[X, Y, Z \big] \big/
  (Z-1, X^2 +  Y^2 + Z^2 -1)$ 
$$
\mathbb{R} \big[X, Y, Z \big]  \big/ (X^3+ Y^2, Z-1) \thickapprox
\mathbb{R} \big[X, Y \big] /(X^2+Y^2). 
$$

\noindent
Hence $Z-1$ is prime in $A$. Set $t =\dfrac{1}{z-1}$, so that
$z=1+\dfrac{1}{t}$. Now, since $x^2 + y^2 + z^2 - 1 = 0$, we have $x^2
+ y^2 + 1 + \dfrac{1}{t^2} + \dfrac{2}{t} - 1 = 0$ i.e. $(tx)^2 +
(ty)^2 = - 2t - 1$ i.e. $t \in \mathbb{R}[tx, ty]$. Now $A[t] =
\mathbb{R}[tx, ty, \dfrac{1}{t}]$ is factorial. Hence by 
Theorem~\ref{chap1:thm6.3'}, $A$ is factorial.
 
\item Since $(x + iy) (x - iy) = (z + 1) (z - 1)$, we conclude that $A
  = \mathbb{C} [x, y, z]$, $x^2 + y^2 + z^2 = 1$ is not factorial. 
\end{enumerate}
\end{proof}

Let\pageoriginale $K$ denote the field of complex numbers or the field
of reals, and 
$A= K$ $x, y, z$, $x^2 + y^2+ z^2=1$. Let $M$ be the module  $M = Adx
+ Ady + Adz$, with the relation $xdx + ydy + zdz = 0$. 


\begin{prop} % prop 8.4
The $A$-module $M$ is projective
\begin{enumerate}[(a)]
\item If $K = \mathbb{R}$, then $M$ is not free

\item If $K = \mathbb{C}$, then $M$ is free.
\end{enumerate}
\end{prop}  

  
\begin{proof}
Since the elements $v_1 = (0, z, -y)$, $v_2 = (-z, 0, x )$, $v_3 = (y, -x,
0)$ of $A^3$ satisfy the relation $xv_1 +  yv_2 + zv_3 =  0$, we have
a homomorphism $u : M \rightarrow A^3$, given by $u(dx) = v_1$, $u(dy) =
v_2$, $u(dz) = v_3$. Let $\nu$ be the homomorphism $\nu : A^3
\rightarrow M$ given by $\nu (a, b, c) = a(ydz - zdy)+ b(zdx - xdz) +
c (xdy - ydx)$. It is easy to verify that $\nu o u$ is the identity on
$M$. Hence $M$ can be identified with a direct summand of $A^3$. Hence
$M$ is projective. Now the linear form $\varphi : A^3 \rightarrow A$
given by $\varphi (a, b, c) = ax + by + cz$ is zero on $M$. But
$A^3/M$ is  a tossion-free module of rank 1. Hence $M = \ker
\varphi$. On the other hand we have $\varphi (A^3) = A$, since
$x^2+y^2+z^2=1$. Hence $M \oplus A \approx  A^3$. Thus $M$ is
equivalent to a free module. 
\end{proof}  

\begin{enumerate}[(a)]
\item If $K = \mathbb{R}$, then $M$ is not free. We remark that $M$ is
  the A-module of sections of the dual bundle of the targent bundle to
  the sphere $S_2$. Since there are no non-degenerate continuous
  vector fields on $S_2$, the tangent bundle is not trivial, nor is
  its dual. 

\item If $K = \mathbb{C}$, then $M$ is free. For, the tangent bundle
  to the complexification   of $S_2$ is trivial (this complexification
  being the product of two complex projective lines). 
\end{enumerate}  
  
\begin{remark*} %rem 
R.~Swan\pageoriginale (Trans, Amer. Math. Soc. 105(1962), 264-277(1962) has
proved the following. The ring $A = \mathbb{R} \big[x_1, x_2, \ldots
  , x_5 \big]$, $\sum\limits_{i = 0}^5 x^2_i = 1$ is factorial. Now
$S_7$ can be fibred by $S_3$, the base being $S_4$. Let $V$ be the
bundle of tangent vectors along the fibres for this fibration and $M$,
the corresponding module. Then $M$ is not free, where as $M
\bigotimes\limits_{\mathbb{R}} \mathbb{C}$ is free over $A
\bigotimes\limits_{\mathbb{R}} \mathbb{C}$. Further $A
\bigotimes\limits_{\mathbb{R}} \mathbb{C}$ is factorial. Moreover $M$
is not equivalent to a free-module. 
\end{remark*}  
  
\medskip
  \noindent
  \textbf{Grassmann varieties}. Let $E$ be a vector space of dimension
  $n$ over a field $K$. Let $G = G_{n, q}$ be the set of all $q$
  dimensional subspaces of $E (q \le n)$. Then set $G$ can be provided
  with a structure of a projective variety as given below. 
  
  We call an element $x \in \overset{q}{\wedge} E$ a decomposed
  multi-vector if $x$ is of the form $x_1 \wedge \cdots \wedge x_q$,
  $x_i \in E$. We have $x_1 \wedge \cdots  \wedge x_q = 0$ if and only
  if $x_1, \ldots , x_q$ are linearly dependent. Further $x_1 \wedge
  \cdots \wedge x_q = \lambda y_1 \wedge \cdots \wedge y_q$, $\lambda
  \in K^*$ if and only if $x_1 , \ldots , x_q$ and $y_1 , \ldots ,
  y_q$ generate the same subspace. In the set of all decomposed
  multivectors we introduce the equivalence relation $x_1 \wedge
  \cdots x_q \sim y_1 \wedge \cdots \wedge y_q $ if $x_1 \wedge \cdots
  \wedge x_q = \lambda y_1 \cdots y_q$ for some $\lambda \in K^*$. Then
  the set $G_{n, q} $ can be identified with the quotient set which is
  a subset of $P (\overset{q}{\wedge} E)$ the $(^n_q) - 1$ dimensional
  projective space defined by the vector space  $\overset{q}{\wedge}
  E$. It can be shown that with this identification,  $G_{n, q}$ is a
  closed subset of $P (^q_{\wedge} E)$ in the Zariski topology). The
  projective variety $G_{n, q}$ is known as the Grassmann variety. As
  $GL(n, K)$ acts transitively on $G_{n, q}$, it is non-singular. 
  
  \noindent
  Let\pageoriginale $L$ be a generic $q$-dimensional subspace of $E$
  with a basis 
  $x_1, \ldots x_q$, say $x_i = \sum\limits_{j = 1}^n \lambda_{ij }
  e_j$, $1 \le i \le q$, $\lambda_{ij} \in K$. Then 
$$
  x_1 \wedge \cdots \wedge x_q = \sum\limits_{i_1 \cdots i_q} d_{i_1, 
  \ldots, i_q} (\lambda) e_{i_1} \wedge \cdots \wedge e_{i_q},   
$$
  where $d_{i_1, \ldots , i_q} = \det (\lambda_{k i_j})$. Let $x_{ij}$,
  $1 \le i \le q$, $1 \le j \le n$ be algebraically independent elements
  over $K$. Let  $B = K \big[x_{ij}\big]_{\substack{1 \le i \le q
      \\ 1 \le j \le n}}$ the polynomial ring in $nq$ variables. For
  any subset $H = \big \{i_1, \ldots , i_q \big \}$, $i_1 < i_2 < \cdots
  < i_q$ of cardinality $q$, we denote by $d_H(x)$ the $q$ by $q$
  determinant $\det(x_{ki_j})$. It is clear that $A = K \big[d_H (x)
    \big]_{H \in J}'$ where $J$ is the set of all subsets of
  cardinality $q$ of $\big\{i, \ldots , n \big \}$, is the
  homogeneous coordinate ring of $G_{n, q}$. 
  
\begin{prop} %prop 8.5
The ring $A$ is factorial. 
\end{prop}  
      
\begin{proof}
It is known that the ring $A$ is normal (See $J$. Igusa: On the
arithmetic normality of Grassmann variety,
Proc. Nat. Acad. Sci. U.S.A. Vol. 40, 309 - 313). Consider the
element $d = d_{\{1, \ldots , q\}} (x) \in A$. We first prove that $d$ is
prime in $A$. Consider the subvariety $S$ of $G_{n, q}$ defined by $d=
0$. Let $E'$ be the subspace generated by $e_1, \ldots , e_q$ and
$E''$ the subspace generated by $e_{q+1}, \ldots , e_n$. (We recall
that $e_1, \ldots , e_n$ is a basis of $E$). Now $\alpha \in S$ if and
only if $\dim (pr_{E'} (\alpha )) < q$, i.e. if and only if $\alpha \cap
E'' \neq (0)$. Let $Z = (0, \ldots , 0, Z_{q+1}, \ldots , Z_n)$, where the
$Z_i$ are algebraically independent. Let $x_1, \ldots , x_{q-1}$ be
independent generic points of $E$, independent over
$k(z)$. Then\pageoriginale $Z 
\wedge x_1 \wedge \cdots \wedge x_{q-1}$ is a generic point of $S$,
and therefore $S$ is irreducible. Let $\mathscr{Y}$ be the prime ideal
defining $S$. Then $A.d = \mathscr{Y}^{(s)}$ for some $s$. We now look
at the zeros of $A.d$ which are singular points. These zeros are given
by the equations $\dfrac{\partial}{x_{it}}(d) = 0$, $1 \le i \le q$, $1 \le
t \le n$, or equivalently, by equating to 0 the sub-determinants of
$d$ of order $q-1$. Hence $\alpha$ is a singular zero of $A.d$ if and
only if $pr_E' (\alpha)$ has codimension $\ge 2$ i.e. if and only if
$\dim (\alpha \cap E'') \ge 2$. Hence $A.d$ has at least one simple
zero. That is, $s=1$ and $A.d$ is  a prime. 
\end{proof}    
    
    The co-ordinate ring of the affine open set $U$ defined by $d \neq
    0$ is the ring $A' = \bigg\{\dfrac{a}{d^{d^o (a) /q}}  \big| a
    \in A$, a homogeneous $\bigg\} = A_{/_{(1-d)}}$. We shall describe
    the ring $A'$ in another way. Let $\alpha \in G_{n, q}$. Then
    $\alpha \in U \Leftrightarrow \alpha \cap E'' = (0)$. Let $y_1 =
    (1, 0 , \ldots ,  0, y_{1 q+1}, \ldots y_{1n}), \ldots$, $y_q = (0,
    \ldots , 1, ~ y_{q, q+1}, \ldots , y_{qn})$, where the $y_{ij}$
    are algebraically independent over $K$. Then $y_1 \wedge \cdots
   \wedge y_q$ is a generic point of $U$. Set $y = (y_{ij})_{\substack{1 \le
        i \le q \\ 1 \le j \le n}}$ where $y_{ij} = \delta_{ij}$, $i
   \leq q$, $j \leq q$. Then
    $A' = K \big[d_H (y) \big]_{H \in J}$. But $d_{1, \ldots , i,
      \ldots , q, j} (y) = \pm y_{ij}$, $1 \le i \le q$, $q+1 \le j \le
    n$. Hence $A' = K \big[ y_{ij} \big]_{\substack{1 \le i \le q,
        \\ q+1 \le j \le n}}$. Hence $A'$ is factorial. Hence, by
    Proposition~\ref{chap1:prop7.3}, the ring $A$ is factorial. 
    
\setcounter{rem}{0}
\begin{rem} %rem 1
The ring $A$ provides an example of a factorial ring which is not a
complete intersection. 
\end{rem}  
      
\begin{rem} %rem 2
We do\pageoriginale not know any example of a factorial ring which is
not a Cohen-Macaulay ring.  
\end{rem}  
      
\begin{rem} % rem 3
We do not know any example of a factorial ring which is not a
Gorenstein ring.  
\end{rem}    
    
 A local ring $A$ is said to be a \textit{Gorenstein ring} if $A$ is
 Cohen-macaulay and every ideal generated by a system of parameters is
 irreducible. 
 

\section{Power series over factorial rings}\label{chap1:sec9}%Sec 9 

\begin{theorem}\label{chap1:thm9.1} % the 9.1
Let $A$ be a noetherian domain containing elements $x, y, z$ satisfying
\begin{enumerate} [(i)]
\item $y$ is prime, $Ax \cap Ay = Axy$;

\item $z^{i - 1} \notin Ax + Ay$, $z^i \in Ax^j + Ay^k$, where
  $i$, $j$, $k$ are integers such $ijk - ij - jk - ki \geq 0$.
\end{enumerate}
\end{theorem}    
    
    Then $A \big[[T]\big]$ \textit{is not factorial}.
    
    We first list here certain interesting corollaries of the above
    theorem. 
    
\begin{corollary}\label{chap1:coro1} % coro 1
There exist factorial rings $A$ (also local factorial ones) such that
$A\big[[T]\big]$ is not factorial. Let $k$ be a field and let $A' =
k \big[x, y, z \big]$ with $z^i = x^j + y^k$, $(i, j, k) = 1$, $ijk - ij
- jk -ki \geq 0 $ (for instance $i = 2$, $j = 5$, $k = 7$). Then by
Theorem~\ref{chap1:thm8.1} the ring $A'$ is factorial, and so is the
local ring $A= 
A'_{(x, y, z)}$. But $x, y, z$ satisfy the hypothesis of the above
theorem. Therefore $A' \big[[T]\big]$ and $A \big[[T]\big]$ are not
factorial. 
\end{corollary} 
   
\begin{corollary}\label{chap1:coro2} %coro 2
There\pageoriginale exists a local factorial ring $B$ such that its
completion 
$\hat{B}$ is not factorial. Set $A = A'_{(x, y, z)}$, $B = A
[T]_{({\mathscr{M}}, T)}$, where $A'$ is as in the proof of Corollary
\ref{chap1:coro1} 
and $\mathscr{M}$ is the maximal ideal of $A$. Then $\hat{B}$ is
factorial. Now $\hat{B} = \hat{A} \big[[T]\big]$. Further $\hat{B}$
is also the completion of the local ring $A \big[[T]\big]$. Thus if
$\hat{B}$ is factorial, so is $A \big[[T]\big]$ by \textit{Mori's
  Theorem} (see for instance, Sur les anneaux factorials,
Bull. Soc. Math. France, 89, (1961), 155 - 173). Contradiction. 
\end{corollary} 
  
\begin{corollary} %coro 3
There exists a local non-factorial ring $B$ such that its associated
graded ring $G(B)$ is factorial. 
\end{corollary}    
  
\noindent
We set $A_1 = k [ u, v, x, y, z]$, $z^7 = u^5 x^2 + v^4 y^3$. We observe
that $z$ is prime in $A_1$ and that $A_1 \big[\dfrac{1}{z}\big] = k
\big[x', y', u, v, \dfrac{1}{z} \big]$, $x = z^3 x'$,  $y = z^2y'$, $z =
u^5 x^{' 2}+ v^4y^3$. Hence $A_1 \big[ \dfrac{1}{z} \big]$ is
factorial and therefore  is $A_1$. Take $A = A_{1_{(u, v, x, y, z)}}$, $B
= A \big[[T]\big]$. Since $x, y, z$  $A$ satisfy the hypothesis of
the above theorem with $i = 7$, $j = 2$, $k = 3$, the ring  $B = A
\big[[T]\big]$ is not factorial. But $G(B) = G (A) [T] = A_1 [T]$ is
factorial.  

\begin{remark*} %rem
\begin{enumerate}[1.]
\item If $A$ is a regular factorial ring, then so is $A
  \big[[T]\big]$. (see Chapter \ref{chap2}, Theorem~\ref{chap1:thm2.1}).  

\item If $A$ is a noetherian factorial ring such that $A_{\mathscr{M}}
  \big[[T]\big]$ is factorial for every maximal ideal $\mathscr{M}$
  of $A$, then $A \big[[T]\big]$ is factorial.  

\item Suppose that $A$ is a factorial Macaulay ring such that
  $A_\mathscr{Y} \big[[T]\big]$ is  factorial for all prime ideals
  $\mathscr{Y}$ with height $\mathscr{Y} = 2$. Then $A \big[[T]\big]$ is
  factorial (for proofs of (2) and (3), see P. Samuol, on unique
  factorization domains, Illinois J.Math. 5(1961) 1-17). 

\item \textbf{Open question}.\pageoriginale Let $A$ be a
  \textit{complete} local ring which is factorial. Then is $A \big [
    [T] \big]$ factorial?  
\end{enumerate}
  \end{remark*}  
  
  In Chapter \ref{chap3} we shall see that at least in characteristic 2,
  the completion $\hat{A}$ of $A$ of Corrolary \ref{chap1:coro2} is not
  factorial. We shall also give examples to show that $C(A)
  \rightarrow C(A \big [[T]\big])$ is not surjective. Finally it may
  be of interest to note that $J$. Geiser has proved that there do not
  exist \textit{complete} factorial rings satisfying the hypothesis of
  Theorem~\ref{chap1:thm9.1}. 

\begin{proofofthm} % pro of the 9.1
Let $S$ denote the multiplicatively closed set $1, x, x^2,
\ldots$. Set $A' = S^{-1} A$, $B = A \big[[T]\big]$. Then $S^{-1} B
\subset A' \big[[T]\big]$; in fact $A' \big[[T]\big]$ is the
$T$-adic completion of $S^{-1} B$. But, however, $S^{-1}B$ is not a
Zariski ring with the $T$-adic topology. Let $S'$ denote the set of
elements of $B$, whose constant coefficients are in $S$. Then
$S'^{-1}B$ is a Zariski ring with the $T$-adic topology and its
completion is $A' \big[[T]\big]$. Consider the element $v = xy - z^{i
  - 1} T \in B$. 
\begin{enumerate}[(a)]
\item No power series $y + a_1 T + a_2 T^2 + \cdots B$ is an associate
  of $v = xy - z^{i - 1} T $ in $A'$   $T$ (nor, a fortiori in
  $S'^{-1} B$).  
\end{enumerate}
\end{proofofthm} 

\begin{proof}
If possible, suppose that $(xy - z^{i - 1} T) ~ (\dfrac{1}{x} +
\dfrac{c}{x^\alpha }T + \cdots ) ~ \in B$, with $\dfrac{1}{x}+
\dfrac{c}{x^\alpha} T + \cdots \in A' \big [[T]\big]$. Then
$\dfrac{cy}{x^{\alpha - 1}} - \dfrac{z^{i - 1}}{x} \in A $ i.e $cy -
z^{i - 1} x^{\alpha - 2} \in A x^{\alpha - 1}$. Since $z^{i -1} \notin
Ax + Ay$ we have $\alpha \ge 2$. Further since $Ax \cap Ay = Axy$ we
have $c\in Ax^{\alpha-2}$, say $c = c' x^{\alpha-2}$. Then $z^{i-1} - c' y \in
Ax$. Contradiction 
\begin{enumerate}
\item[(b)] There exists an integer $t$ and an element $v' =
  \dfrac{y^t}{x}+ \dfrac{b_1}{x^2} T + \cdots + \dfrac{b_n}{x^{n+1}}
  T^{n+1} + \cdots $ such that $u = v v' \in B$, where $v = xy -
  z^{i-1} T$.  
\end{enumerate}
\end{proof}

\begin{proof}
Take $t \geq ij$. We have to find elements $b_1, ~ b_2, \ldots b_n,
\ldots $ of $A$ such that 
$$
\frac{b_n}{x^{n+1}} xy - \frac{b_{n-1}}{x^n} z^{i - 1} \in A \text{
  i.e. } b_n y - b_{n-1} z^{i - 1} \in Ax^n 
$$\pageoriginale
for $n \ge 1$. We set $b_0 = y^t$. Assume that the $b_l$ for $l \le
nij$ have been determined and that $b_{nij} = y^{t(n)} F _n (x^j,
y^k)$, where $t(n) \ge ij$ and $F_n (X, Y)$ is a form of degree
$ni$. This is trivially verified for $n = 0$. 
\end{proof}

\noindent
The congruence $b_{ni j +1 } y - b_{nij} z^{i -1} \in Ax^{nij +1}$
may be solved by taking $b_{nij + 1} = y^{t(n) - 1} F_n (x^j, y^k)
z^{i - 1}$. Similarly 
    
\noindent
$b_{nij + r} = y^{t(n) -r} F_n (x^j, y^k) z^{r (i - 1)}$, $0 \le r <
ij$. Further the relation 
$$b_{(n+1)ij } y - b_{nij+ij-1} z^{i-1} \in
Ax^{(n + 1)ij}$$ 
implies that 
$$b_{(n+1)ij} y  - y^{t(n) - ij +1} F_n (x^j,
y^k) Z^{ij(i-1)} \in Ax^{(n+1)ij}.$$ 
But $z^i \in Ax^j + Ay^k$, say
$z^i = cx^j + dy^k$. Now we have to solve the congruence, $b_{(n + 1)
  ij} \equiv y^{t(n) - ij +1} G (\mod Ax^{(n+1)ij})$ where $G = (cx^j
+ dy^k)^{j (i - 1)} F_n (x^j, y^k)$. The form $G(X, Y) = (c X +
dY)^{j(i - 1)} F_n (X, Y)$ is of degree $ni ~ + ~ (i - 1)j$. The
monomials in $G(x^j, y^k)$ are of the form $x^{j \alpha} y^{k \beta}$,
$\alpha + \beta = ni + (i - 1)j$. By reading modulo $Ax^{(n + 1)ij}$ we
can `neglect' the terms for which $j \alpha \ge (n + 1) ij $
i.e. $\alpha \ge (n+1)i$. For the remaining terms we have $\beta > ni
+ (i - 1) j - (n + 1) i = ij - j - i$. Thus $G(x^j, y^k) \equiv y^{(ij
  - j - i)k} F_{n + 1} (x^j, y^k) (\mod Ax^{(n+1) ij})$, where
$F_{n+1}$ is a form of degree $ni + (i - 1)j - (ij-j-i) = (n+1)i$. Now
$b_{(n+1) ij} y \equiv y^{t(n) + (ijk - jk - ki -ij) + 1} F_{n+1} (x^j,
y^k) ~ (\mod Ax^{(n+1) ij})$. We may solve this by taking $b_{(n+1)ij}
= y^{t(n) + ijk -jk -ki -ij } F_{n+1} (x^j, y^k)$, i.e. we may take
$t(n+1) = t(n) + ijk - jk - ki - ij$ and (b) is proved. 

\begin{enumerate}[(c)]
\item $B$ is\pageoriginale not factorial. Suppose that, in fact, $B$
  were factorial. 
\end{enumerate}    
 
\noindent
Set $u = v v'$, with $v, v'$ as in (b). Let $u = u_1 , \ldots u_s$
be the decomposition of $u$ into prime factors in $B$; since the
constant term of $u$ is a power of $y$ and since $y$ is prime, the
constant term of each $u_l$ is a power of $y$. Consider $R = S'^{-1}
B; R$ is factorial. Now $v' \in \hat{R}$ and therefore $u ~ R ~ R v =
Rv$. Further $v$ is prime in $R$ (since the constant term of $v$ is
$y$ times an invertible element in $S^{-1} A$). Now unique
factorization in $R$ implies that $v$ is an associate of some $u_j$ in
$R$. This contradicts (a).  


