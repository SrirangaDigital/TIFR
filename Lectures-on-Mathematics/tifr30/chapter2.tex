\chapter{Regular rings}\label{chap2}%Chap II
    
Let\pageoriginale $A$ be a noetherian local ring and $\mathscr{M}$ its
maximal 
ideal. We say that $A$ is \textit{regular} if $\mathscr{M}$ is
generated by an $A$-sequence. We recall that $x_1, \ldots , x_r \in A$
is an A-sequence if, for $i = 0, \ldots , r- 1$, $x_{i +1}$ is not a
zero divisor in $A/(x_1, \ldots , x_i)$. It can be proved that a
regular local ring is a normal domain. Let $A$ be a noetherian
domain. We say that $A$ is regular if $A_\mathscr{M}$ is regular for
every maximal ideal $\mathscr{M}$ of $A$. 


\section{Regular local rings}\label{chap2:sec1} %Sec 1
 
Let $A$ be a noetherian local ring and $\mathscr{M}$ its maximal
ideal. Let $E$ be a finitely generated module over $A$. Let $x_1,
\ldots , x_n \in E$ be such that the elements $x_i ~ \mod \mathscr{M}
E$ form a basis for $E/\mathscr{M} E$; then the $x_i $ generate
$E$ (by Nakayama's lemma). Such a system of generators is called a
\textit{minimal} system of generators. Let $x_1, \ldots , x_n$ be a
minimal system of generators of $E$. Let $F = \sum\limits_{i = 1}^n
Ae_i$ be a free module of rank $n$. Then the sequence $0 \rightarrow
E_1 \rightarrow F_o \xrightarrow{\varphi} E \rightarrow 0$ is exact
where $\varphi (e_i) = x_i$. Now $E_1$ is finitely generated. Choosing
a minimal set of generators for $E_1$, we can express $E_1$ as a
quotient of a  free module $F_2$. Continuing in this fashion we get an
exact sequence of modules. 
$$
\cdots \rightarrow F_n \rightarrow F_{n-1} \rightarrow \cdots
\rightarrow F_o \rightarrow E \rightarrow 0; 
 $$
we call this a \textit{minimal resolution} of $E$. We say that the
homological dimension of $E$ (notation hd $E$) is $< n$ if $F_{n + 1}
= 0$ in a minimal resolution of $E$. If $F_i \neq 0$ for every $i$,
then we put $hd E = \infty$. It can be  proved that $hd$ $E$ does not
depend upon the minimal resolution (since two\pageoriginale minimal
systems of generators of $E$ differ by an automorphism of $E$). We
recall that  
    
$E$ is free $\Leftrightarrow $ the canonical mapping $\mathscr{M}
\otimes E \rightarrow E$ is injective $\Leftrightarrow$ Tor $^A_1 (E,
A_{/_{\mathscr{M}}} ) = 0$. 
    
From this it easily follows that for a finitely generated A-module $E$
we have $hd E < n$ if and only if Tor $^A_{n+1} (E, A_{/_{\mathscr{M}}})=0$. 

\setcounter{theorem}{0}
\begin{theorem}[Syzygies]%them 1.1
Let $A$ be a regular local ring and $d$ the number of
    elements in an $A$-sequence generating the maximal ideal
    $\mathscr{M}$ of $A$. Then for any finitely generated module
    $E$. We have $hd  E \leq  d$.
\end{theorem}

We state a lemma which is not difficult to prove.

\setcounter{lemma}{1}
\begin{lemma}\label{chap2:lem1.2} % lem 1.2
Let $A$ be a noetherian local ring and $G$ a finitely generated
$A$-module with $hd G < \infty$. Let a be a non-zero divisor for
$G$. Then $hd \; \dfrac{G}{aG} = hd \; G + 1$. 
\end{lemma}    
    
Now if $\mathscr{M} = (x_1, \ldots , x_d)$, where $x_1, \ldots , x_d$
is an A-sequence, then by means of an immediate induction and a use of
the above lemma we get $hd (A_{/_{\mathscr{M}}})= d$. Hence
Tor$^A_{d+1} ~ (E, A_{/_{\mathscr{M}}}) = 0$ for any module $E'$. Hence
$hd E \le d$. 

\setcounter{theorem}{2}
\begin{theorem}[Serre]\label{chap2:thm1.3} %the 1.3
Let $A$ be a local ring with maximal ideal $\mathscr{M}$ such that $hd
\mathscr{M} < \infty$. Then $A$ is regular. 
\end{theorem}    
    
We first observe that the hypothesis of the theorem implies that for
an $A$-module $E$, we have $hd E \le hd (A/{\mathscr{M}}) = hd ~ m
+1$. 
    
We prove the theorem by induction on the dimension $d$ of the $A/_{m}$
vector space $\mathscr{M}/_{\mathscr{M}^2}$. If $d = 0$, then
$\mathscr{M} = 0; A$ is a field and therefore regular. Suppose $d>
0$. Then we claim that under the hypothesis\pageoriginale of the
theorem there 
exist an element $b \in \mathscr{M} - \mathscr{M}^2$ which is not a
zero divisor. This follows from the following lemma. 

\setcounter{lemma}{3}
\begin{lemma} %lem 1.4
Let $A$ be local ring. Suppose that every $x \in \mathscr{M} -
\mathscr{M}^2$ is a zero divisor. Then any finitely generated module
of finite homological dimension is free. 
\end{lemma}

\begin{proof}
Let $G$ be a module with $hd G < \infty$. If $hd G > 0$, we find, by
resolving $G$, an A-module $E$ such that $hd  E= 1$. Let $0 \rightarrow
F_1 \rightarrow F_o \rightarrow E \rightarrow 0$ be a minimal
resolution of $E$. Then $F_1$ is free and $F_1 \subset \mathscr{M} F_o$. Now,
since every element of $\mathscr{M} - \mathscr{M}^2$ is a zero
divisor, it follows that every element of $\mathscr{M}$ is a zero
divisor and $\mathscr{M}$ is associated to (0). Hence there exists
an $a \neq 0$, such that $a \mathscr{M} = 0$. Hence a $F_1 = 0$. This
contradicts the fact that $F_1$ is free. Hence $hd G = 0$, and $G$ is
free. 
\end{proof}
    
Applying this lemma to `our' $A$ we see that if every element of
$\mathscr{M} - \mathscr{M}^2$ is a zero divisor, then $\mathscr{M}$ is
free. Since $\mathscr{M}$ consists of  zero divisors we have
$\mathscr{M}= 0$ i.e. $A$ is a field. Thus if $d > 0$ there is a $b
\in \mathscr{M} - \mathscr{M}^2$ such that $b$ is not a zero
divisor. Set $A' = A/_{Ab}$, $\mathscr{M}' = \mathscr{M}/Ab$. We claim
that $\mathscr{M} / Ab$ is a direct summand of
$\mathscr{M}/\mathscr{M}b$. 
Let $\psi$ denote the canonical surjection
$\mathscr{M}/_{\mathscr{M}^b} \rightarrow
\mathscr{M}_{/_{Ab}}$. Let $b, q_1, \ldots , q_{d-1}$ be a minimal set
of generators of $\mathscr{M}$. Set $\sigma = \sum\limits_{i = 1}^{d -
  1} Aq_i$. Let $\varphi$ be the canonical mapping $\sigma \rightarrow
\mathscr{M}/_{\mathscr{M}^b}$. Then $\Ker (\varphi) = \sigma \cap
\mathscr{M}^b \subset \sigma \cap Ab$. On the other hand if $\lambda b
\in \sigma$, then $\lambda b = \sum\limits_{i = 1}^{d-1} \lambda_i
q_i$, $\lambda_i \in A$. But $b, q_1, \ldots , q_d$ is a minimal set of
generators of $\mathscr{M}$. Hence $\lambda$, $\lambda_i \in
\mathscr{M}$. Thus $\sigma \cap \mathscr{M} b = \sigma \cap ~ Ab$. 

Thus\pageoriginale we have a canonical injection $\mathscr{M}/_{Ab} =
\dfrac{\sigma  +  Ab}{Ab} \xrightarrow{\theta}
\dfrac{\mathscr{M}}{\mathscr{M}^b}$, since $\sigma + Ab/_{Ab} \approx
\sigma/_{\sigma \cap Ab}= \sigma/_{\sigma \cap
    \mathscr{M}b}$. It is easy to see that $\psi o \theta =
I_{\mathscr{M}}$. Hence $\mathscr{M}/_{Ab}$ is a direct summand of
$\mathscr{M}/_{\mathscr{M}^b}$. We now have the following lemma
easily proved by induction on $hd (E))$. 

\begin{lemma}% lem 1.5
Let $A$ be a commutative ring and $E$ an A-module with $hd E <
\infty$. Let $b \in A$ be a non-zero divisor for $A$ and $E$. Then
$hd_{A/{Ab}} E/{bE} < \infty$. 
\end{lemma}    

From the above lemma, it follows that $hd_{A/bA}
\mathscr{M}/_{\mathscr{M}^b}< \infty$. Since $\mathscr{M}/_{Ab}$
is a direct summand of $\mathscr{M}/_{\mathscr{M}^b}$ we have $hd
_{A/{Ab}} ~ \mathscr{M}/{Ab} < \infty$. Since
$\dim_{A/_{\mathscr{M}}} \;\; \mathscr{M}/Ab \big/_{\mathscr{M}^2
  Ab / Ab} =  \dim_{A /\mathscr{M}}  \;\; \mathscr{M} /_{\mathscr{M}^2
  + Ab} = d-1$, $A/Ab$ is regular by induction hypothesis. Hence
$\mathscr{M}/Ab$ is generated by an $A/{Ab}$-sequence, say
$x_1, \ldots x_{d-1}$ modulo $Ab$. Then $b$, $x_1 , \ldots , x_{d-1}$ is
an $A$-sequence and generates $\mathscr{M}$. Thus $A$ is regular. 
    
For any local ring $A$ we define the global dimension of $A$ to be $hd
A/\mathscr{M}$, where $\mathscr{M}$ is the maximal ideal;
notation : $g l \dim A = \delta (A)$. For any $A$-module $E$ we have
$hd_A E \le \delta (A)$. 

\begin{coro*} % coro
Let $A$ be a regular local ring and $\underline{p}$ a prime ideal with
$\underline{p} \neq \mathscr{M}$, where $\mathscr{M}$ is the maximal
ideal of $A$. Then $A_{\underline{p}}$ is regular and $\delta
(A_{\underline{p}}) < \delta (A)$. 
\end{coro*}
    
\begin{proof}
We have $\delta (A_{\underline{p}}) = hd_{A_{\underline{p}}} ~
A_{\underline{p}}/_{\underline{p}A_{\underline{p}}} =
hd_{A_{\underline{p}}} A/_{\underline{p}} \otimes A_{\underline{p}}
\le hd_A A/_{\underline{p}}$, the last inequality, being a consequence
of the fact that $A_{\underline{p}}$ is A-flat. Choose\pageoriginale
$x \in \mathscr{M} - \underline{p}$. Since $x$ is not a zero divisor
for $A_{/_{\underline{p}}}$ we have by Lemma~\ref{chap2:lem1.2}, $hd_A 
A/_{\underline{p}}/_{x}(A/_{\underline{p}}) = hd_A
\dfrac{\underline{p} + x}{\underline{p}}=1 + hd_A A/_{\underline{p}}
\le \delta (A)$ i.e. $hd_A A/_{\underline{p}} \le \delta (A) -
1$. Hence $\delta (A_{\underline{p}}) < \delta (A)$. 
\end{proof}  

      
\setcounter{theorem}{5}
\begin{theorem}[Auslander-Buchsbaum]%%% 1.6
Any regular local ring is factorial.
\end{theorem}   
     
\begin{proof}
Let $A$ be a regular local ring. We prove the theorem by induction on
the global dimension $\delta (A) $ of $A$. If $\delta (A) = 0$, then
$A$ is a field and therefore factorial. Suppose $\delta (A) > 0$. Let
$x$ be an element of an A-sequence generating $\mathscr{M}$. Then $x$
is a prime element. By Nagata's theorem, we have only to prove that
$B= A \big[\dfrac{1}{x} \big]$ is factorial. For any maximal ideal
$\mathscr{M}$ of $B$, we have $B_{\mathscr{M}} = A_{\underline{p}}$,
where $\underline{p}$ is a prime ideal with $x \notin
\underline{p}$. By the corollary to Theorem~\ref{chap2:thm1.3}, we see that
$A_{\underline{p}}$ is regular and $\delta (A_{\underline{p}}) <
\delta (A)$. Hence by the induction hypothesis, $A_{\underline{p}}$ is
factorial. Thus $B_{\mathscr{M}}$ is factorial for every maximal ideal
$\mathscr{M}$ of $B$ (i.e. $B$ is locally factorial). Let $\sigma$ be
a prime ideal of height 1 in $B$. Then $\sigma$ is locally principal
i.e. $\sigma$ is a projective ideal. Now $B$ being a ring of quotients
of the regular local ring $A$, the ideal admits a finite free
resolution. By making an induction on the length of the free
resolution of $\sigma$ we conclude that there exist free modules $F$, $L$
such that $\sigma \oplus L \approx F$. By comparing the ranks we see
that $L \approx B^n$, $F \approx B^{n+1}$ for some $n$. Taking the
$(n+1)^{th}$ exterior power we have $\bigoplus\limits^n_{j=1}
  \overset{j}{\wedge} (\sigma) \otimes \overset{n+1-j}{\wedge}
(L) \approx B$. Since $\sigma$ is a modulo of rank $1 \overset{j}{\wedge}
(\sigma)$, $j \ge 2$ is a torsion-module. Hence $\sigma \approx B$ that
is, $\sigma$ is principal and therefore $B$ is factorial. Hence $A$ is
factorial. 
\end{proof}    
    

\section{Regular factorial rings}\label{chap2:sec2}%Sec 2
We recall\pageoriginale that a regular ring $A$ is a noetherian domain
such that 
$A_{\mathscr{M}}$ is regular for any maximal ideal $\mathscr{M}$ of
$A$. We say that domain $A$ is locally \textit{factorial} if
$A_{\mathscr{M}}$ is factorial for every maximal ideal $\mathscr{M}$ of
$A$. 


\setcounter{theorem}{0}
\begin{theorem}
If $A$ is a regular factorial ring then the rings $A[X]$ and
  $A \big[[X]\big]$  are regular factorial rings. 
\end{theorem} 
   
\begin{proof}
We first prove that $A [X] $ and $A \big[[X]\big]$ are regular. Let
$B = A [X]$. Let $\mathscr{M}$ be a maximal ideal of
$B$. Set $\underline{p}= A \cap \mathscr{M}$. Then $B_\mathscr{M} =
(A_{\underline{p}} [X])_{\mathscr{M} A_{\underline{p}}[X]}$. Since a
localisation of a regular ring  is again regular, we see that
$A_{\underline{p}}$ is regular. Thus to prove that $B$ is regular we
may assume  that $A$ is a local ring with maximal ideal $m(A)$  and
that $\mathscr{M} \cap A = m(A)$. Since $A$ is regular, $m(A)$ is
generated by  an $A$-sequence, say $a_1, \ldots, a_r$. Now
$B/_{m(A)B} \approx A/_{m(A)} [X]$. Thus $\mathscr{M}/_{m (A)
    B} = (\bar{F}(X))$, where $F (X) \in \mathscr{M}$ is such that the
class $\bar{F}(X)$ of  $F(X)$ $(\mod m (A))$ is irreducible in
$\dfrac{A}{m (A)} [X]$. Now $a_1, \ldots , a_r, F(X)$ is
$B_\mathscr{M}$-sequence and generates $\mathscr{M} / B_\mathscr{M}$
(in fact $\mathscr{M}= (a_1, \ldots , a_r F(X)))$. Hence
$B_\mathscr{M}$ is regular for every maximal ideal $\mathscr{M}$
i.e. $B$ is regular. We shall now prove that $C = A \big[[X]\big]$ is
regular. Let $\mathscr{M}$ be a maximal ideal of $C$. Since $X \in
\Rad (C)$, $\mathfrak{M} = \mathscr{M} + X C$, where $\mathscr{M}$ is a
maximal ideal of $A$. Now $A_\mathscr{M} \subset
C_{\mathfrak{M}}$. Since $A_\mathscr{M}$ is regular, $\mathscr{M}
A_\mathscr{M}$ is generated by an $A_\mathscr{M}$-sequence, say $m_1,
\ldots , m_d$. Then $m_1, \ldots , m_d$, $X$ is a 
$C_\mathfrak{M}$-sequence which generates $\mathfrak{M}
C_\mathfrak{M}$. Thus $C_\mathfrak{M}$ is regular i.e. $C$ is
regular. 
    \end{proof}    
    
    We\pageoriginale now prove that $B = A [X]$, $C = A \big[[X]\big]$ are
    factorial. That $B$ is factorial has already been proved (see
    Chapter $I$, Theorem~\ref{chap1:thm6.5}). To prove that $C$ is factorial, we
    note that $K$ is in the radical $\Rad (C)$ of $C$ and $\dfrac{C}{X
      C} \approx A$ is factorial. Now the following lemma completes
    the proof of the theorem. 
    
\setcounter{lemma}{1}
\begin{lemma}\label{chap2:lem2.2} % lem 2.2
Let $B$ be a locally factorial noetherian ring (for instance a regular
domain). Let $x \in \Rad (B)$. Assume that $B/{xB}$ is
factorial. Then $B$ is factorial. 
\end{lemma}    

\begin{proof}
Let $\sigma$ be a prime ideal of height 1 in $B$. Then $\sigma$ is
locally principal i.e. $\sigma$ is projective. If $x \in \sigma$, then
$\sigma = Bx$. If $x \notin \sigma$, then $\sigma \cap Bx = \sigma
x$. Thus $\sigma/\sigma x = \sigma{/_{(\sigma \cap Bx)}} \approx
(\sigma + Bx){/_{Bx}}$ i.e. $\sigma /_{\sigma x}$ is a projective
ideal in $B/_{Bx}$. Since $B/_{Bx}$ is factorial and $\sigma/_{\sigma
  x}$ divisorial, we see that $\sigma{/_{\sigma x}}$ is principal in
$B/_{Bx}$. Hence, by Nakayama's lemma, $\sigma$ is principal. 
\end{proof}  
      
\begin{coro*} % coro 
Let $A$ be a principal ideal domain. Then $A \big[[X_1, \ldots , X_n
  ]\big]$ is factorial.  
    \end{coro*}    
    
    In particular if $K$ is a field, then $K \big[[X_1, \ldots , X_n
      ]\big]$ is factorial. 

    
\section{The ring of restricted power series}\label{chap2:sec3}%Sec 3
 Let $A$ be a commutative ring and let $\mathscr{M}$ be an ideal of
 $A$. We provide $A$ with the $\mathscr{M}$ - adic topology. Let $f =
 \sum a_\alpha X^\alpha \in A \big[[X_1, \ldots , X_d]\big]$,
 $\alpha = (\alpha_1, \ldots , \alpha_d)$, $X^\alpha = X^{\alpha_1}_1
 \cdots X^{\alpha_d}_d$. We say that $f$ is a \textit{restricted
   power series} if $a_\alpha \rightarrow 0$ as $| \alpha |
 \rightarrow \infty$, $| \alpha | = \alpha_1 + \cdots + \alpha_d$. It is
 clear that the set of all restricted power series is a subring of $A
 \big[[X_1 , \ldots , X_d]\big]$ which we denote\pageoriginale by $A
 \big \{x_1, 
 \ldots , X_d \big\}$; we have the inclusions $A \big[X_1, \ldots ,
   X_d \big] \subset A \{X_1, \ldots X_d\} A \big[[X_1, \ldots ,
     X_d]\big]$. In fact $A \big\{X_1, \ldots , X_d\big \}$ is the
 $\mathscr{M} (X_1, \ldots,\break X_d)$- adic completion of $A \big[X_1,
   \ldots , X_d \big]$. In particular, if $A$ is noetherian so is $A
 \big\{X_1, \ldots , X_d \big\}$. Further $A \big[[X_1, \ldots ,
     X_d]\big]$ is the completion of $A \big\{X_1, \ldots , X_d
 \big\}$ for the $(X_1, \ldots , X_d)$-adic topology. But this is not
 of interest, since $A \big \{X_1, \ldots , X_d \big\}$ is not a
 Zariski ring with respect to the $(X_1, \ldots,\break X_d)$-adic
 topology. 
 
\begin{lemma}\label{chap2:lem3.1} %lem 3.1
Let $A$ be a commutative ring and $\mathscr{M}$ an ideal of $A$ with
$\mathscr{M} \subset \Rad (A)$. Let $A \big\{X_1, \ldots , X_d
\big\}$ denote the ring of restricted power series, $A$ being provided 
with the $\mathscr{M}$-adic topology. Then $\mathscr{M} \subset \Rad
(A \{ X_1, \ldots, X_d\})$. 
\end{lemma}  
      
    \begin{proof}
Let $m \in \mathscr{M}$. Consider $1+ ms (X)$, where $s (X) \in A \big
\{X_1, \ldots , X_d \big\}$. Set $s(X) = a_o + t(X)$, $t(X)$ being
without constant term. Since $1+ m a_o$ is invertible, we have $1 + ms
(X) = \dfrac{1}{1+ ma_o} (1- mu (X))$, where $u(X)$ is a restricted
series without constant term. Now, $\dfrac{1}{1-mu (X)}= 1+mu (X) +
m^2 u(X)^2 +  \cdots $ is clearly a restricted power series. Thus $1 +
ms (X)$ is invertible in $A \big\{X_1, \ldots , X_d \big\} $ i.e. $m
\in \Rad (A \big\{X_1, \ldots , X_d \big\})$. 
    \end{proof}  
    
\setcounter{theorem}{1}
    \begin{theorem}\label{chap2:thm3.2}[P. Salmon]%the 3.2
Let $A$ be a regular local ring and let
        $\mathscr{M}$ denote its maximal ideal. Then $R = A \{X_1,
        \ldots , X_d\}$ is regular and factorial, the power series
        being restricted with respected to the maximal ideal. 
    \end{theorem}  
      
\begin{proof}
Let $\underline{p}$ be a maximal ideal of $R$. By Lemma~\ref{chap2:lem3.1},
$\mathscr{M} R \subset \Rad (R) \subset \underline{p}$. Now
$p/_{\mathscr{M}R}$ is a maximal ideal of $R/_{\mathscr{M}R}=
(A_{/_{\mathscr{M}}}) \big[ X_1, \ldots , X_d \big]$. It is well
known that any maximal ideal of $(A{/_{\mathscr{M}}}) \big[X_1,
  \ldots , X_d \big]$ is generated  by $d$\pageoriginale elements
which form an 
$(A{/_{\mathscr{M}}}) \big[X_1, \ldots, X_d \big]$ -sequence. Thus
$\underline{p}/_{\mathscr{M}R}$ is generated by an
$R{/_{\mathscr{M}R}}$-sequence. But, $A$ being regular, $\mathscr{M} R$ is
generated by an $R$-sequence. Therefore $\underline{p}$ is generated by
an $R$-sequence. By passing to the localisation, we see that
$\underline{p} R_{\underline{p}}$ is generated by a
$R_{\underline{p}}$-sequence. Hence $R$ is regular. 
    \end{proof}    
    
    We now prove that $R$ is factorial. The proof is by induction on
    $gl. \dim A= \delta (A)$. If $\delta (A)= 0$, then $A$ is a field
    and $R = A \big[ X_1, \ldots , X_d \big]$ hence $R$ is
    factorial. Let $\delta = \delta (A) > 0$ and $m_1, \ldots ,
    m_\delta$ generate $\mathscr{M}$. Now $R$ is regular and therefore
    locally factorial. By Lemma~\ref{chap2:lem3.1} $m_1 \in \Rad
    (R)$. Further 
    $R{/_{m_1 R}} \approx (A/_{m A}) \big\{X_1, \ldots , X_d
    \big\}$, $\delta (A{/_{m_1 A}}) = \delta - 1$; hence, by induction
    hypothesis, $R{/_{m_1 R}}$ is factorial. Using 
    Lemma~\ref{chap2:lem2.2} we see  
    that $R$ is factorial. 
    
    \setcounter{rem}{0}
    \begin{rem} %rem 1
Let $A$ be a local ring which is factorial. Then it does not imply
that $A\{T\}$ is factorial. Take $A = k \big[x, y, ~ z \big]_{(x, y,
  z)}$, $z^2 = x^3 + y^7$. As in the proof of Theorem~\ref{chap1:thm9.1}, 
Chapter~\ref{chap1}, 
we can prove that there exist $b_1, b_2 , \ldots \in A$ such that $(xy
- z T) ~ (\dfrac{y}{x} + \dfrac{b_1}{x^2} T + \dfrac{b_2}{x^3} T^2 +
\cdots + \dfrac{b_n}{x^{n+1}} T^{n+1} + \cdots) = u \in B = A
\{T\}$. In fact it can be checked that we can take the elements $b_i$
such that $u = y^2 - x T^2 - xy T^8 - 3x y^2 T^{14} \cdots$, $- \alpha_n
xy^n T^{2+6n} \ldots$, where $\alpha_n$ is an integer such that $0
\le \alpha_n \le 2^{3n}$. By providing $A$ with the $(x, y, z)$-adic
topology, we see that the power series $u$ is restricted. Now the
proof of Theorem~\ref{chap1:thm9.1} verbatum carries over and we
conclude that the restricted power series ring $A \{T\}$ is not
factorial.   
\end{rem} 
       
\begin{rem} %rem 2
In the above example, if we take $k = \mathbb{R}$ or $\mathbb{C}$, the
real number field or the complex number field respectively, then we
can speak of the\pageoriginale convergent power series ring over
$A$. Now the above power series $u$ is convergent since $0 \le
\alpha_n \le 2^{3n}$. Hence the convergent power series ring over $A$
is also not factorial. 
\end{rem}  
      
