
\chapter[Geometry at the High School level...]{Geometry at the High School level (The Geometry Programme of
  the School Mathematics Study Group)}

\begin{center}
{\em By~} EDWIN E. MOISE
\end{center}

\setcounter{pageoriginal}{80}
\textsc{The}\pageoriginale work that I shall report on is a text-book
writing project which has been carried out, during the last two years,
by the School Mathematics Study Group (SMSG), an agency of the
National Science Foundation. For what appeared to be good reasons,
we were working in a considerable hurry ; and one of the things that
we did not do was to make an adequate survey of the work of the
people who were addressing themselves to the same problems in other
countries. I especially felt the lack of this, myself, because this
was my first experience in working on the problems of school
education. This gives me especially good reasons for considering it a
privilege to attend the present conference. I suspect that in
mathematical education, only the little problems are special and local
; thebig problems are as international as mathematics itself.

For the past forty years or so, for reasons too complicated to explain
here, American high-school education has operated in an unhealthy
degree of isolation from the learned world ; at times it has seemed
almost as if the schools and the universities were mutually turning
their backs on each other. More recently, for reasons which I cannot
claim really to understand, this trend has been reversed; and the
reverse trend has been helped by a heighened public consciousness of
educational problems. This, no doubt, was largely due to the
achievements of the Soviet rocket experts; the impact of the sputniks
on American public opinion would be hard to exaggerate.

One reflection of this state offairs was the creation of the SMSG,
under federal sponsorship. The task of this organization is to create
a new mathematics curriculum for the high-schools, and to write
text-books from which the new curriculum can be taught. The work was
begun in the Summer of 1958, at Yale University. There\pageoriginale
were about forty of us, and we split up immediately into four groups,
one for each of the high-school grades. Each of the resulting groups
of ten consisted of experienced high-school teachers and
mathematicians from the universities, in about equal numbers. In 1958
we worked for four weeks. In the Summer of 1959 a somewhat larger
group, of about the same composition, worked for nine weeks at the
University of Colorado, and finished four sample text-books. Each of
thse is now being used experimentally in about a hundred classrooms
distributed all over the country. As you can see, this was what is
referred to, in our current argot, as a ``crash programme''; and I am
sure that more leisurely efforts will permit considerable improvements
in what we did.

It should be understood that in matters of opinion, I am speaking only
for myself, and not for the SMSG as a group. With this proviso, I may
explain that the basic assumptions in our work were as follows :
\begin{itemize}
\item[(1)] Curricular proposals, in the abstract, are not likely to be
  very useful, even semantically : that is to say, reading a general
  account of what someone thinks that the schools ought to be doing,
  it is not always easy to tell with any exactitude what sort of
  Utopian vision the author has in mind. And the feasibility of a new
  scheme can only be judged in terms of an explicit description of how
  it is to be carried out. For this reason, SMSG was supposed to
  produce ``sample text-books''; these were not supposed to be highly
  polished jobs, but they were supposed to be good enough to form a
  basis for valid experiments in teaching.

\item[(2)] The new ideas were supposed to come mainly from the
  mathematicians. Adapting new mathematical material for the use of
  schools is a little like flying an airplane under conditions for
  which the plane was never designed : you have to understand the
  machine extremely well, to make it work. On the other hand, men
  whose only experience of teaching is in universities are not likely
  to have a very good practical sense of what can be done and how, in
  a high-school classroom. Thus each half of each of our groups began
  its\pageoriginale task with certain varieties of occupational
  naivet\'e. Our modus operandi was a long dialogue, in which it was
  hoped that each half would whittle away the naivet\'e of the
  other. In the tenth-grade geometry group, this process lasted for
  seven hours a day, five or more days a week, for four weeks. Its
  success depended, of course, on everybody's willingness to engage
  both in vigorous discussion and bona fide listening. Our practice
  agreed with the theory fairly well. This process of reciprocal
  education, however, took the whole four weeks of our 1958 working
  session ; and at the end of the summer we had merely a detailed
  outline of our programme, with only fragments of text, even in
  first-draft form.
\end{itemize}

In the second semester of 1958-59, I went on leave from my regular
academic duties at the University of Michigan, and wrote a first draft
of the entire text. This was in fact not quite a first draft, because
a number of revisions were made, a chapter at a time, in response to
very useful criticisms from Miss Hope Chipman, who teaches at our
University High School. The manuscript included problem material only
in places where rather novel problems were needed to accompany
especially novel treatments in the text. At the writing session at the
University of Colorado, there were a total of fourteen people working
on the text. We worked under a system of division of labour. First
there would be a general conference, on a given chapter, and revisions
would be decided on in general terms. When the second draft had been
written, one group wrote problem sets, another group wrote the
corresponding section of the teachers' manual, and yet a third group
undertook the final editorial job of getting the manuscript in shape
for final typing and photographing. I doubt that this is the best way
to write a text-book, but at least it was orderly, and it enabled us
to meet our dead-line.

So much for the way we worked. Let me now explain the substance of our
problems as we understood them. For a number of years, the traditional
courses in plane and solid geometry have been under heavy
criticism. In the first place, it has been pointed out that Euclid can
no longer be regarded as a model of deductive
reasoning. This,\pageoriginale of course, is true, and the same
objection applies with far greater force to the revisions,
abridgements and corruptions of Euclid which are currently used in
most American shcools. (I will return later to the question of what
sort of weight ought to be attached to this criticism.)


It has been objected, further, that Euclid is not \textit{modern.} I
should explain that in some American educational circles the ideal of
modernness is pursued almost as a fetish. Perhaps this is a reflection
of the spirit of obsessional contemporaneity which has done so much to
damage the teaching of the humanities. At any rate, those who object
to Euclidean geometry on this ground appear to have missed an
important point. I have heard it claimed, much more plausibly, that
Euclid is too Bourbakiste to be a proper part of elementary
education. In fact, Euclid and Bourbaki are written in much the same
spirit, and in the two millennia or so between them, there is no real
parallel to either of them. Both of them undertake to axiomatize
literally everything in sight, working from scratch, modulo zero,
starting with abstract logic. Both of them work their way to the
concrete mathematical results of their respective times by a prolonged
campaign of Satz and Beweis. In each case, the results are both
elegant and brilliant, but their appropriateness for the schools is
open to question. That is it may well be that Euclid is \textit{too}
modern.

It has been said rather plausibly that a deductive treatment of solid
geometry is not worth the semester which has usually been devoted to
it. What is really needed, in solid geometry, is the knowledge of a
rather small number of basic theorems, plus a good intuitive grasp of
space relationships, of the sort required to visualize
three-dimensional figures and sketch them. For this reason, there have
been proposals for an ``integrated'' course, treating plane and solid
geometry together, with the main emphasis on the former. This idea was
part of the plan that we eventually adopted.

Finally, it was proposed by one group that the ``integration'' idea be
carried even further, by introducing coordinate systems at an early
stage, and developing the synthetic method and the  analytic method
together, with heavy emphasis on the latter.

These\pageoriginale were the main criticisms of school geometry which
were popular at the time when the SMSG started its work. But we
agreed, almost from the outset, that the main problems were of another
sort. If the level of rigour in the treatment could be effectively
raised, well and good ; but we felt that, in a pinch, the rigour that
served as a model for Descartes and Spinoza ought to be good enough
for a tenth-grade child. (In the course of our work, it appeared that
the level of rigour could be raised very substantially, without any
loss in teachability ; we will know more about this from the results
of the first year of experimentation.) There were other aspects of the
traditional treatment that concerned us more seriously.
\begin{itemize}
\item[(1)] The use of language is peculiar. Here, as almost nowhere
  else in mathematics, the word \textit{equals} and the symbol = are
  used to denote, not the logical identity, but rather whatever
  equivalence relation the speaker happens to have in mind at the
  moment. The basic equivalence relation is congruence, in what turns
  out to be the sense of isometry ; but the word congruence is not
  used consistently to describe this relation : it is used for
  triangles, but not for segments or for angles. The symbolism is
  richly ambiguous. Thus, in standard texts, the symbol \textit{AB}
  may denote (a) the distance between the points $A$ and $B$, (b) the
  segment whose end-points are $A$ and $B$, (c) the line that contains
  the points $A$ and $B$ or (d) the ray which emanates from $A$ and
  contains $B$. Many other examples of this sort could be given. We
  considered that one of our main tasks was to devise a more
  manageable language, in which the words would match up with the
  concepts in a more natural way, and which would permit us to speak
  more simply and more explicitly. 

\item[(2)] As I have already indicated, we believe that the ideas of
  integration and modernness can be overdone. But many of the ideas of
  synthetic geometry are the prototypes of ideas that are important,
  in various extended forms, in modern mathematics ; and in such
  cases, we believed that the geometric ideas should be presented in
  such a way as to bring out the analogies and make the geometry an
  effective introduction to the later developments. One of the most
  striking\pageoriginale instances of this is the idea of congruence
  between triangles. It is customary to write $\Delta ~ABC \simeq
  \Delta ~DEF$, to indicate that the two triangles are congruent ; and
  in most treatments, we can then write equally well $\Delta ~ABC
  \simeq \Delta ~EFD$, or $\Delta~ CBA \simeq \Delta ~ DEF$, and so
  on. It is plain, however, that this relation of abstract congruence
  is not the idea that is really being used, because we nearly always
  go on to infer things about corresponding sides or corresponding
  angles. That is to say, what we really mean is that the triangles
  are congruent in a particular way, that is, under a particular
  correspondence. Thus, in this context, the student encounters for
  the first times the basic idea of a correspondence preserving
  certain properties ; this is the prototype of the isomorphisms, the
  isometries, the homeomorphisms, the order-preserving mappings, and
  so on, that are going to come up later.

For this reason, among others, we speak in our book not of congruence
between triangles, but of \textit{congruences} between
triangles. These are defined as one-to-one correspondence between the
vertices, preserving measures of the sides and the angles. When we
write $\Delta ABC \simeq \Delta DEF$, this means that the
correspondence $A \leftarrow \rightarrow D$, $B \leftarrow \rightarrow
E$, $C \leftarrow \rightarrow F$ is a congruence. (We claim no
originality for this.) The other reason for this presentation is that
we think it will  probably be easier for the student; it furnishes him
with a notation from which he can read off immediately the six
congruence relations between the corresponding sides and angles. The
fragmentary reports that we have had so far from the experimental
teaching tend to justify these hopes.

\item[(3)] Classical synthetic geometry involves some rather delicate
  concepts which get badly damaged when they are reworked for the
  benefit of the young. This is strikingly true of the ideas of
  congruence and distance. Euclid had no numbers, except for the
  numbers that he got geometrically; and he therefore could have no
  distance function, even if he had been willing to use non-invariant
  methods. The basic undefined concept was congruence ; and in the
  spirit of this treatment, we would have to say that the ``distance''
  between two points $A$, $B$ is the set of all point-pairs $X$, $Y$
  which are congruent to $A$, $B$.\pageoriginale (If we choose a unit,
  then numbers become the measures of distances.) By the time these
  subtleties reach the readers of standard high- school text-books,
  they have become hardly more than a conceptual blur, in which
  geometric objects---segments, for examples---are  being
  persistently confuesed with the numbers that measure them. One
  alternative to this would be to give an accurate and thorough
  treatement of such matters, in the spirti of Hilbert't
  \textit{Foundations of Geometry.} It seemed to us, however, that
  this would be extremely difficult, and probably not worth the time
  and effort that it would cost. The foundations of mathematics are
  not a part of elementary mathematics. They are, rather, a scholarly
  speciality, and a rather difficult one. It seemed more promising to
  recast the postulates in such a way that logical accuracy and
  pedagogic intelligibility would be in less conflict.
\end{itemize}

These considerations argued strongly for the use fo the metric
postulates for geometry which were devised in the early 1930's by
G.D. Birkhoff. In Birkhoff's treatment, it is assumed that the real
numbers are known. A distance function is chosen once for all. We can
then say that two segments are congruent if the distance between their
end-points is the same. The degree is chosen as a unit of measure for
angles. We can then say that two angles are congruent if their
degree-meausres are the same. The rest of the postulates are stated
within this conceptual framework. There is probably no need to repeat
them here \textit{in extenso.} It should be admitted that these
postulates do not have the delicacy and elegance of Hilbert's. But
they have an important advantage. They can be put to work quickly, and
they can be used accurately by a tenth-grade student. This is
especially important at the beginning.

One of the most serious hazards, in beginning the study of deductive
geometry, is that we seem to be spending  so much time in proving
trivial and obvious-looking theorems. We seem to be faced with a
choice of evils. If we prove these obvious theorems carefully, this is
likely to seem pointless to the student. If we turn them into
postulates, then the student is likely to miss the whole point of the
deductive method, and wonder why we don't just postulate the
whole\pageoriginale business. It is easy to be impatient with the
student's impatience, and to attribute it to his lack of appreciation
of lofty intellectual matters. It seems to me, however, that a
tenth-grade student's lack of interest in foundational proofs is
reasonable, in terms of the mathematical world that he lives in at
this point of his studies. When a mature mathematician is willing to
handle postulates with patience and accuracy, this is due in large
part to the fact that the world he lives in is littered with
pathological examples which have reminded him again and again that
strange and unexpected things can happen in situations which seemed at
the outset to be simple and intuitive. Young students cannot feel this
way, because they have not had such experiences. For this reason,
postulates to be used in high-schools should be powerful enough to
reduce foundational questions to a minimum. 

If we accept this viewpoint, then it may seem doubtful whether a
formally deductive treatment should be used at all. If the SMSG group
had arrived at this conclusion, I rather doubt that we could have
acted on it, because it surely requires too much artistry for anybody
to do it in a hurry. At any rate, we thought that there were good
reasons for using formal proofs based on postulates.
\begin{itemize}
\item[(1)] In American schools, a great deal of geometry is taught
  informally before the tenth grade. Given this background of
  intuitive knowledge, it seems reasonable to codify it and formalize it.

\item[(2)] There are a great many experimental curricula, now being
  tried, in which algebra is treated deductively. Let us hope that
  they work. But on the basis of our experience so far, synthetic
  geometry seems to offer the most favourable opportunity to introduce
  the student to the formal structure of mathematics. In this context,
  quite immature students seem to be quite able to accept and use the
  ideas of postulate, definition, theorem and proof. Perhaps this is
  due to a historical accident, but I suspect not. The postulates of
  geometry are abstract, logically speaking, but intuitively they are
  fairly concrete, and capable of being visualized. We should also
  remember that until the last century, they were not even thought of
  as\pageoriginale being postulates in the sense in which a modern
  mathematician understands the term ; they were considered to be
  ``self-evident truths'', from which other and more obscure truths
  could be deduced without the possibility of error. We know now that
  this viewpoint is philosophically wrong, but it remains useful and
  comforting in elementary study. Honesty forbids us to present this
  viewpoint but it is likely to linger in the back of the student's
  mind ; drastic and persistent measures would probably have to be
  taken in order to uproot it ; and in my own opinion, no such
  measures should be taken. The old view of postulates seems to help
  the student to behave like a mathematician, and behaving like a
  mathematician is a very good way of learning to understand mathematics.
\end{itemize}

For these reasons, we thought that plane geometry should be treated
deductively. About solid geometry, we felf doubtful~; deductive solid
geometry takes a great deal of time, and we wondered whether it was
worth it. What we really wanted to produce was space intuition. This
created a problem, because if you treat one sort of thing formally,
and another sort of thing informally, in the same book, the student is
likely to get confused about what the ground rules are. The best
solution to this problem that we were able to think of was to
introduce space concepts briefly and early, and them to use space
configurations in problem material throughout the rest of the
text. For this reason, a student in our course gets much more
experience with solid geometry than the meagre character of our
chapters on the subject might suggest. At some points, we found that
our ``integrated'' treatment actually had the advantages that it is
supposed to have. For example, the theorems on tangent lines to
circles are so closely analogous to the theorems on tangent planes to
spheres that quite long discussions could be repeated almost word for
word.

In all the above discussion, I have been trying to explain the sense
in which I think that high-school geometry ought to be \textit{formal,
modern, integrated} and \textit{rigorous.} But I have remarked at the
outset that it is usually hard to tell, from a general discussion of
curricular reforms, just what the author is driving at, and I may by
now be involved\pageoriginale in the same difficulty myself. It may
help to clarify matters if I give a short outline of our treatment of
plane area. This is merely an outline of the logic of the treatment,
not a sample of the style in which it is presented in the text.

A \textit{triangular region} is a figure that consists of a triangle
plus its interior. A \textit{polygonal region} is a figure that can be
cut up into a finite number of triangular regions, in such a way that
if two of the triangular regions intersect, then their intersection is
either an edge or a vertex of each of them. (Here, and throughout the
text, the word \textit{intersection} is used in a purely set-theoretic
sense, not involving any idea of crossing.) Hereafter, a polygonal
region will be referred to simply as a \textit{region}.

\textit{Postulate.} To every region $R$ there corresponds a unique
positive number $A$, called the area of $R$.

\textit{Postulate.} If two triangles are congruent, then the
corresponding triangular regions have the same area.

\textit{Postulate.} Suppose that the region $R$ consists of two
regions $R_1$ and $R_2$ which intersect only in edges and
vertices. Then the area of $R$ is the sum of the areas of $R_1$ and $R_2$.

\textit{Postulate.} The area of a  rectangle is the product of its
base and its altitude. 

The next few theorems give the usual formulas for the areas of
parallelograms, right triangles, triangles and trapezoids. We then use
the theory of area to get quick and easy proofs of (1) the Pythagorean
theorem and (2) the basic similarity theorem, which asserts that a
line which is parallel to the base of a triangle and intersects the
other two sides in different points, divides the other two sides into
proportional segments. The second of these proofs is borrowed from the
classic text-book of Legendre. The use of the area concept to replace
limiting processes in incommensurable cases is quite old, but lately
it has not been widely used.

This treatment will illustrate most of the viewpoints that I have been
describing.
\begin{itemize}
\item[(1)] Geometric\pageoriginale figures are described,
  matter-of-factly, as sets of points, but there is no fanfare about set-theory.

\item[(2)] The assumptions are explicitly stated. Euclid's famous
  axiom, to the effect that the whole is equal to the sum of its
  parts, is usually appealed to in place of the third of the above
  postulates. If this axiom is ``true,'' and if it really implies our
  postulate, then it equally well implies the non-existence of the
  Banach-Tarski Paradox. That is, Euclid's axiom does not state the
  conditions under which area is additive.

\item[(3)] The last postulate is strong ; we might have used only the
  formula $A = a^2$ for areas of squares. The proof of the area
  formula for rectangles, in the incommensurable case, would then have
  been quite difficult. 

\item[(4)] In a way it is inelegant to use area theory in proving the
  Pythagorean theorem and the basic similarity theorem. We are using
  more postulates than are needed. The difficulty was that we were
  unable to devise a teachable presentation of the proof of the basic
  similarity theorem, using merely the parallel postulate; and to give
  in, by calling this theorem a postulate, seemed even more inelegant
  than using area theory. Once we had decided, for this reason, to
  discuss area before similarity, we thought that we might as well use
  areas in proving the pythagorean theorem. (The proof that we use is
  the familiar one, in which a large square is decomposed into a small
  square plus four  congruent right triangles.)

\item[(5)] It is a fact, of course, that every simple closed polygon
  in the plane, convex or not, separates the plane into exactly two
  connected open sets ; that is, every polygon has an interior and an
  exterior. But to prove this to a tenth-grade class is impossible. It
  seemed to us that in general, a question should not be raised in the
  classroom unless it can be given an intelligible answer. For this
  reason, we arranged the treatment in such a way that the question
  of the plane-separation theorem does not arise. It is customary to
  speak of the areas of polygons, rather than the areas of polygonal
  regions, but everybody knows that the latter is what we really
  mean.\pageoriginale Therefore, if you are going to speak of areas of
  polygons, you have three choices : you can state a theorem without
  proof, or state a superfluous postulate, or limit yourself to convex
  polygons, for which the separation theorem can be proved. The
  disadvantages of the first two are clear, and the third appears to
  be even worse. The student knows quite well that non-convex
  polygonal regions have areas ; in simple cases he can calculate them
  ; and if we tell him that he can't talk about such things in our
  course, he is likely to get the idea that the phrase
  \textit{rigorous mathematics} is grammatically analogous to the
  phrase \textit{truncated cone}, in which the adjective damages the
  noun instead of modifying it in the usual way.
\end{itemize}

This should indicate the spirit in which we have attempted to
reconcile the objectives or rigour, clarity, and speed of
development. Following is an outline of the content of our book, in
the form of a list of the titles of the chapters, with brief
indications of their contents :

\smallskip
\textit{Chapter 1.} Some Geometry We Already Know. (An intuitive
review of similar right triangles, areas, volumes, etc. For some
students, of course, this material is new.)

\smallskip
\textit{Chapter 2.} Common Sense and Exact Reasoning. (Indications of
the power of formal and technical mathematics, versus commonsense
judgments. Explanation of the role of postulates, undefined terms,
definitions and theorems.)

\smallskip
\textit{Chapter 3.} Points, Lines and Planes in Space. (Incidence
postulates and their simplest consequences.)

\smallskip
\textit{Chapter 4.} Measurement of Distance. (Here we give a
distance-function postulate, followed by postulates which state, in
effect, that on any line we can set up a coordinate system with the
expected properties. Betweenness is not axiomatized, but is defined
in terms of distance).

\smallskip
\textit{Chapter 5.} Half-planes, Angles and Triangles. (Here we give
the basic definitions, and state postulates on the separation of
planes\pageoriginale by lines and of space by planes. We also give postulates which
state, in effect, that there is an abstract protractor, with the
expected properties, by which angles can be measured in degrees.)

\smallskip
\textit{Chapter 6.} Congruences. (The spirit of this chapter has
already been described.)

\smallskip
\textit{Chapter 7.} Geometric Inequalities. (Here the triangular
inequality turns out to be a theorem).

\smallskip
\textit{Chapter 8.} Perpendicular Lines and Planes in Space.

\smallskip
\textit{Chapter 9.} Parallel Lines in a Plane. 

\smallskip
\textit{Chapter 10.} Parallel Planes in Space.

\smallskip
\textit{Chapter 11.} Areas of Polygonal Regions.

\smallskip
\textit{Chapter 12.} Similarity.

\smallskip
\textit{Chapter 13.} Circles and Spheres.

\smallskip
\textit{Chapter 14.} Constructions with Straight-edge and Compass.

\smallskip
\textit{Chapter 15.} The Area of a Circle and Related Topics. 

\smallskip
\textit{Chapter 16.} Analytic Geometry. (A brief introduction.)

\smallskip

This is designed to be a programme for a year-course, in classes
meeting for five hours a week. We were afraid, at first, that we might
be putting in too much, but the first scattered reports that I have
heard so far from the teachers suggest that we did not. It is now too
soon to say how well the programme works ; in any case, we shall
revise it in the Summer of 1960, on the basis of this year's
experience. In the meantime it seems, at least, reasonable to hope
that our innovations and experiments will be of some value to the
writers of future text-books. This, indeed, was our main purpose.

\bigskip
\bigskip

{\fontsize{9pt}{12pt}\selectfont
University of Michigan}\relax







