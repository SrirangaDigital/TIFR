
\chapter{Second South Asian Conference on Mathematical Education}

\begin{center}
\textbf{BOMBAY, 20-27 JANUARY 1960}

\medskip

\textbf{\Large{INAUGURAL ADDRESS}}

\medskip

{\em By~} \textbf{H. J. BHABHA}
\end{center}

\textsc{The} First South Asian Conference on Mathematical Education
was held here in February, 1956, and was attended by 75 mathematicians
from 20 countries. This Conference is a sequel to the first, and has
also been organised by the Tata Institute of Fundamental Research,
with the financial assistance of the Sir Dorabji Tata Trust. We are
glad to have with us on this occasion representatives of the
Governments of Ceylon, Malaya, Pakistan, Singapore and Thailand, and,
on behalf of the Government of India, I would like to thank these
Governments for having accepted our invitation to send representatives
to this Conference. We are also glad to have with us a delegate from
the Science Council of Japan, two delegates from the Mathematical
Society of Thailand, and one from the International Commission on
Mathematical Instruction. The Conference  is attended, in addition, by
eminent mathematicians from Denmark, France, Germany, the USSR, the
United Kingdom, and the USA. I extend a warm welcome to all our
distinguished guests from overseas, as also to many delegates from the
Indian universities, who are participating in this Conference.

About 10 days ago, we had the inauguration of the International
Colloquium on Function Theory and many of the distinguished
mathematicians who attended that meeting are also present with us
today. Hearing of the International Colloquium on Function Theory and
this Conference on Mathematical Education, the Prime Minister,
Jawaharlal Nehru, sent a message for both these meetings, which I read
out at the inauguration of the Colloquium. He has asked me to extend
on his behalf and on behalf of the Government of India a warm welcome
to those attending this Conference. `The study of mathematics', the
Prime Minister's message says, `has always been of great
importance. Today, its importance has increased even more, and has
become the basis of the great advances that science has made in recent
years.' I would like to take this opportunity to tell you of the
Government of India's very active interest in the development of
mathematics, and its intention to do whatever it can for encouraging
the teaching and study of mathematics on the widest possible basis in
the universities, and in promoting research of high quality.

Since April 1959, the Government of India's responsibility for the
promotion of advanced study and research in mathematics has been
allotted to the Department of Atomic Energy. This is largely because
the Department has been administratively responsible at Government
level for the Tata Institute of Fundamental Research, which is the
National Centre of the Government of India for Advanced Study and
Fundamental Research in Mathematics. I can assure you that money will
not be lacking for supporting any worthwhile project which may be put
forward for the promotion of mathematics in this country. Fortunately,
the amount of money required for supporting advanced mathematical
study and research on a widespread basis throughout the country is
relatively small compared with the money required for technological
research, for which we are going in in a large way. The problem of
finding money for mathematics is the easiest one to solve. Far more
difficult is the problem of growing mathematicians of quality able to
make important new contributions to the development of
mathematics. This is necessarily a slow and hard process. The
Department will be glad to support research projects in the
universities, which are soundly conceived and certified to be so by
experts in the field, and I once more give the assurance that lack of
financial support will not come in the way of any good project. Those
who wish to receive grants will have to show that they have
mathematical work to their credit, which is considered valuable by
experts in their respective fields either in India or abroad.

In their enthusiasm many people, who should know better, put forward
proposals which have not been properly conceived and which do not
stand careful examination. Suggestions have, for example, been made
that four mathematical research institutes should be set up in the
country. We can see no more reason for setting up more than one
national centre for mathematics or theoretical physics at the present
state of our development than for setting up more than one National
Physical Laboratory or National Chemical Laboratory. The construction
of a building with a library of books, a few blackboards and chalk
does not produce new mathematics. It is good mathematicians alone who
can do this, and we hardly have enough of them to man more than one
national research centre at the present stage.

Teaching and research are inseparable. The main centres of
mathematical research should be the universities, where the main
burden of teaching rests. This research should be done in the
mathematical departments themselves, and what is required is that the
universities should have a sufficient number of posts in mathematics
at all levels so that their mathematical staff, instead of being
deluged with teaching, as at present, have sufficient time and
facilities for following the new advances in mathematics and making
their contribution to them. A national research centre must work in
cooperation with the universities to produce a sufficiently large
number of good mathematicians to staff the universities. As the
national centre for mathematics, the Tata Institute of Fundamental
Research has always followed this policy and arranged a large number
of lectures and seminars at various levels and issued dozens of
volumes of lectures notes intended to facilitate the educational
process particularly at the higher levels.

We in the Department of Atomic Energy have had many talks with the
Chairman of the University Grants Commission on how we could assist in
raising the standard of teaching and research in the
universities. There is no difference of opinion between us on this
subject. We have agreed to receive members of the staff of the
universities who wish to devote their entire time to studying the
latest developments on their subject and doing research in the
laboratories of the Atomic Energy Commission. The Tata Institute of
Fundamental Research has followed a similar practice with respect to
teachers in mathematics from the universities, and with the increased
space that will become available when the School of Mathematics move
to the new buildings of the Institute, I am sure the Institute will be
able to render an increasing service to mathematics in the country in
this respect. 

About a year ago the Department of Atomic Energy instituted a scheme
of scholarships in the universities at the undergraduate and
post-graduate levels. Under this scheme two or three scholarships are
given in each subject, in which a particular Indian university is
specially good, for enabling the two or three best young students in
that subject to carry on their studies without financial worry up to
the stage of their B. Sc. degree and then later up to the
M. Sc. degree. The scheme has been started on the basis of a hundred
scholarships a year distributed over all the universities in the
country and we would be quite prepared to increase the number, if we
find that the scheme is successful and brings good results. Those who
successfully complete their education in this manner will, on the one
hand, be assured of a suitable job with the Department of Atomic
Energy, and on the other will undertake to serve the Department for a
limited period of years. This does not mean that they will all have to
work in the Atomic Energy Commission's own institutions. The
Commission could send them on deputation to work in any university,
which would like to have them, and in this manner the Commission hopes
to be of assistance to the universities. This scheme now covers a
dozen scientific subjects, and we now propose to extend it to
mathematics also.

The purpose of this Conference is to discuss in detail the problems of
mathematical education with special reference to South Asia and to
formulate plans for its sound development. It will carry forward the
work done by the last. It will discuss the teaching of the various
branches of mathematics. The presence of nearly 30 eminent
mathematicians from abroad will make it profitable to utilise their
experience in the discussions. It is hoped that the Conference will
help the university teachers to come to their own conclusions as to
how they should handle the teaching of any particular subject. 

We realise that work in the universities is greatly influenced by the
quality of the instruction imparted in schools. It is our business
therefore to take interest in school education to the extent to which
we can. It is clear that in the nuclear age in which we live, school
education needs to be reoriented. In particular a good deal more
mathematics needs to be taught in order to discipline the mind of the
student and to prepare it for the precise study of nature. 

I hope this Conference will be of use to all those who are
participating and that some of the ideas which will be developed here
will benefit the countries of the region whose governments have been
good enough to send representatives to participate in this
Conference. I hope above all that the contacts made here will lead to
closer collaboration and interchanges in the field of mathematics
between the workers of the participating countries and also others in
this important region of the world.


