
\chapter{Mathematics in the Humanities}\footnote{I am indebted to
    N. D. Andreyev and V. A. Zalgaller for the analysis of the
    curricula in mathematical linguistics and economics.}

\begin{center}
{\em By~} A. D. ALEXANDROV
\end{center}

\textsc{The}\pageoriginale deep penetration of mathematics---its ideas
and methods, the very mathematical ways of thinking---into various
fields of knowledge is one of the most characteristic features of the
modern development of science. The recent rise and rapid development
of mathematical linguistics and mathematical economics or econometrics
are, may be, the most interesting instances of this general process. 

This process sets new problems of mathematical education at the
university level, especially the problem of mathematical instruction
for the students who take linguistics or economics as their
speciality. We are in need of a new kind of scholar and practical
specialist in linguistics and economics, who can master mathematics to
such a degree that he can understand, apply, and, according to his
personal abilities, take part in the theoretical development of
mathematical linguistics and economics, of their principles and
applications.

This new problem of mathematical education seems to be of certain
general interest and I have chosen it as the topic of my present
address. I shall touch upon not only some general principles of
solving this educational problem, but I am going, as well, to tell how
we try to solve the problem practically at our Leningrad University. 

The system of university education in the USSR differs from that in
many other countries. Thus it would not be superfluous to recall first
some of the basic features of the Soviet system, in order to point
out, in particular, the position of mathematical linguistics and of
mathematical economics within the general scheme. 

\begin{center}
\textsc{Some\pageoriginale basic features of university Education in the\\ USSR. The
  position of mathematical linguistics and\\ economics within the
  general scheme.}
\end{center}

The very concept of a university is understood in the USSR somewhat
differently from that accepted in many other countries.

A Soviet university does not include, as a rule, either technical or
medical or any, in the strict, sense, practical higher education. We
have a lot of special institutions for these kinds of higher education
: technical, medical, pedagogical, agricultural and others. 

A Soviet university is, first of all, an institution of scientific
education. It corresponds, in a sense, to the faculties of sciences
and arts of the universities in many other countries. Some Soviet
universities include law, economics and philosophy.

The subdivision of a Soviet university into faculties is comparatively
a detailed one. For instance, at our Leningrad University we have 12
faculties : 6 in sciences : mathematics (including mechanics and
astronomy), physics, chemistry, biology, geology, geography; and 6 in
humanities : philology, history, economics, law, philosophy, and a
faculty of oriental studies.

Mathematics is taught now at all the faculties with the exception of
those of history, of law, and of oriental studies. But the interest in
cybernetics among the scholars in law proves that some mathematical
instruction will take place at the faculty of law. 

Most faculties are subdivided into departments, as for instance, the
mathematical faculty has three departments : mathematics, mechanics
and astronomy, while mathematics is subdivided into two subdepartments
: those of pure and of computing mathematics. 

At the beginning of the past academic year, i.e. in autumn 1958, new
departments of mathematical linguistics and of mathematical economics
were set up at the philological and economical faculties,
respectively. As far as the mathematical part of the curriculum is
concerned, it is, quite naturally, sponsored by the mathematical
faculty. 

Each\pageoriginale faculty has, apart from the basic course, an
evening course, and most faculties have a correspondence course,
too. Thus, for instance, at Leningrad University we have about 13,500
students among whom about 5,500 are at the evening and correspondence
courses. But mathematical linguistics and economics are taught at the
basic course only.

We have a university extension scheme too. And in its scope there are,
for instance, courses in mathematical methods o economics given for
our graduates and practical workers in economics. 

Apart from the teaching staff there is at the Leningrad University a
staff of research workers who have no obligatory teaching work. To the
departments of mathematical linguistics and economics there correspond
two research groups engaged in research work in these fields, and they
offer a kind of basis for the special instruction and practical work
of the students at these departments. In these groups there work a few
post-graduate students.

It is clear that it would be quite impossible to promote the
instruction in any field of science, especially in a new one, without
intensive research work.

The whole university course takes, as a rule, five  years. It includes
basic education in the speciality, in social sciences and in a foreign
language. The course are obligatory. The deeper instruction in the
speciality starts at the third year. It includes optional special
courses, seminars and a period of practice, when a student works in
his speciality. At the final stage a student prepares a diploma paper
which is expected to be, and in most cases is, a kind of research
work. The paper is read or, as we put it, ``defended'' at a meeting
convened by the corresponding chair.

The level of the graduates compared to that in other countries is near
that of a master by American standards, as I am judge by my personal
knowledge.

Post-graduate courses at Soviet universities last three years, and
during this period of time a post-graduate student has to pass a few
advanced\pageoriginale examinations in his speciality (and in
philosophy), and to prepare an original scientific paper which is
presented to the faculty as a candidate thesis. A good average
candidate paper is at the level of a doctor thesis in some other
countries, as I can judge by discussing this topic with my American
colleagues who knew those Russian papers in geometry which had played
the role of candidate dissertations.

\begin{center}
On the mathematical basis in applied linguistics
\end{center}

Mathematical linguistics is a part of the science of language which
treats linguistic problems by mathematical methods.

While solving many problems it is convenient to examine the language
in terms of code and message. Analysing a complex of texts, i.e. a
sufficiently great set of messages, we ascertain that the letters of
the alphabet are met with certain frequencies, i.e. we ascertain the
existence of a certain ``\textit{probability distribution of the code
  elements}''.

Further examination shows that the words or grammatical elements have
their own distributions of frequencies and that, in general, every
level in the hierarchy of the language elements has its ``\textit{own
  statistical picture}''.

Thus, there arises a complex picture of over-lapping statistical
regularities, which by their interaction, determine not only the
functioning but also the ways of development of the linguistic system.

The comprehension of this picture in the whole and the knowledge of
its separate parts is essential for such technical applications of
linguistics as the improvement of the functioning of means of
communication, machine translation, automatic information service,
managing of machines by speech, etc.

That is why the first and the main part of mathematics applied to
linguistic research is the theory of probabilities with the elements
of the theory of information. The theory of probability being founded
on mathematical analysis, it is natural that everybody working in
the\pageoriginale field of mathematical linguistics should know the
foundations of the differential and integral calculus.

Any utterance in any language is a result of some generating
process. This process determines the forms of words, connections
between them, and in accordance with the given connections, the
distribution of words in a linear sequence, called a phrase or
sentence. The choice of elements, their change, their
combination---all this is determined in the speaker's brain by certain
synthesizing operators ; these operators are used not freely, but in a
fixed order, which, in its turn, is determined by the application of
operators of the second order, etc.

To put it more simply, the use of the whole ensemble of operators is a
determined process, which we could rightly call the algorithm of the
synthesis of speech.

At the other end of the communication canal another problem is facing
us : to decompose the heard message into elements, to determine
connections between them, and then to define the meaning of the
message. This process is carried out by means of the recognizing
operators ; the use of which is also a  determined process, which
should be called the algorithm of the analysis of speech. 

Every speaker in any language possesses two algorithms : synthetic and
analytic. 

Mechanical translation is nothing else but the more or less successful
use of the models of these two algorithms in a cybernetic device.

The application of such models for the mechanical translation from one
language into another requires the mastering of, first, methods of
constructing linguistic algorithms, secondly, methods of their
programming on the electronic computers.

As appears from the above, the second necessary field of mathematics
for a person working in applied linguistics is the theory of
algorithms and computation mathematics.

The\pageoriginale accumulation of information by means of cybernetic
machines requires a preliminary formal analysis of the meaning of
messages. Such an analysis can be done by various means, because
principles laid in the basis of the analysis can essentially vary from
time to time.

Thus, the third necessary field is mathematical logic.

The informational accumulation is not only the most difficult, but
also the \textit{most important} field of the application of
mathematical linguistics.

It is extremely necessary therefore to give such instruction to a
student in mathematical linguistics, which would allow him to
participate in the practical solution of this central problem of
applied linguistics.

Thus, for instance, some classes of linguistic objects and operators
behave as rings, others as fields, others as groups, others as
lattices. That is why it is necessary for those engaged in
mathematical linguistics to acquire a sound knowledge of higher
algebra, theory of groups and lattices included.

\begin{center}
The Curricula in Mathematical Linguistics\\
and Economics

\textbf{The curriculum in mathematical linguistics}

5$\frac{1}{2}$ year course
\end{center}
\begin{center}
{\fontsize{9pt}{11pt}\selectfont
  \tabcolsep=1.5pt
%\renewcommand{\arraystretch}{1}
\begin{tabular}{llccccl}
&& &\textit{lectures} & & \textit{seminars} & \textit{years} \\
I. &Linguistics, languages and literature & \ldots & 710 &+& 780 & \\
&various obligatory courses\\
II. & Social sciences and philosophy & \ldots & 300 & + & 160 &\\
& \quad + logic &... & ~~36 & && \\
III. & Mathematics (obligatory) & \ldots & 720 & +& 590 & \\
& 1. Analysis (with ordinary and partial \\
& \quad differential equations) & \ldots & 274 & + & 306 & I,II, III.\\
& 2. Analytic geometry & \ldots & ~~36 & + & ~~36 & \\
& 3. Algebra & \ldots & 100 & +& 100 & I (1st sem.), \\
& & & & & & I (2nd sem.) II.\\
& 4. Probability and statistics  & \ldots & 124 & +& 80 & II, IV.\\
& 5. Mathematical logic & \ldots & 68 &&& III (2nd)---IV (1st).\\
& 6. Theory of machine translation & \ldots & 56 & & & IV (2nd)---V
(1st).\\
& 7. Computing machines and\\ 
& \quad  programming & \ldots & 68 & + & 68 & III---IV (1st). \\
IV. & Optional courses & \ldots & 254 & + & 254 & II---V.\\
& For exmaple : \\
& 1. Information theory\\
&2. Theory of algorithms \\
&3. Stochastic processes \\
&4. Math. logic (additional) \\
&5. Acoustical phonetics
\end{tabular}}\relax
\end{center}\pageoriginale

\eject

\begin{center}
\textbf{The curriculum in mathematical economics}

5 year course
\end{center}
\begin{center}
{\fontsize{8pt}{10pt}\selectfont
\tabcolsep=2pt
\begin{tabular}{llcr@{~~~~}l}
& & &\textit{seminars and} & \textit{years} \\[-0.1cm]
& & &\textit{~~practice~~} &\\
I. & Social sciences and general economics & && \\
& or political economy & \ldots & 470 + 260 & \\
II. & Special economics and statistics  & \ldots & 590 + 270 & \\
III. & Foreign language & \ldots & 360 & \\
IV. & Mathematics (general obligatory & & & \\[-0.1cm]
& courses & \ldots & 390 + 310 & \\
& 1.~ Analysis & \ldots & 150 + 150 & I, II\\
& 2.~ Analytic geometry and & & & \\
& \quad~ linear algebra & \ldots & 120 + 80 & I (2nd)---II (1st)\\
& 3. Probability and statistics & \ldots & 60+60 & II (2nd)---III
(1st)\\
& 4. Introduction to modern analysis & \ldots & 60 + 20 & III\\
V. & Mathematics special and computing & \ldots & 280 +220 & \\
& 1. Elementary technique of computations  & \ldots & 70 & I \\
& 2. Computing machines & \ldots \Bpara{5}{-13}{0}{15} & 70 & I\\[-0.1cm]
& \quad~~ and & & & III\\
& 3. their applications to economical & & &\\[-0.1cm]
& \quad ~~problems & \ldots & 60 +90 & IV\\
& 4. Linear programming & \ldots \Bpara{5}{-7}{0}{10}& 40 + 40 & II\\
& 5. Dynamic programming & \ldots & 60 + 30 & III\\
& 6. Investigation of operations \\
& \quad~~ and theory of games & \ldots & 50 +40 & IV\\
VI. & Optional special courses & \ldots & 240 + 130 & IV-V\\
\end{tabular}}\relax
\end{center}

\textit{The\pageoriginale courses in mathematics} (general) (390+310
hours\footnote[1]{The first number means the number of lectures and the
  second the number of seminars or exercises}).
\begin{itemize}
\item[(1)] \textit{Mathematical analysis} (150 + 150 hours\footnotemark[1]). It includes ordinary differential and
  integral calculus, more detailed study of the functions of several
  variables; maximum and minimum problems; the Lagrange multiplier
  method which is useful ins some economic problems. (Parts less
  interesting for the economist are shortened, in particular, the
  theory of series).

\item[(2)] \textit{Analytic geometry and linear algebra} (120 + 80
  hours).

The course is read in the following order : plain analytic geometry ;
vector calculus ; analytic geometry in space ; foundations
(essentials) of convex solids ; the theory of determinants and
matrices ; analytic geometry of $n$-spaces ; vectors and convex solids
in $n$-space ; general linear spaces ; linear transformations ;
quadratic forms. The aim of the course is to prepare students for the
active perception of the ideas of linear programming ; the development
of geometric imagination ; the mastering of vectors and matrices. 

\item[(3)] \textit{The theory of probability and mathematical
  statistics}  (60 + 60 hours).

\item[(4)] \textit{Introduction to modern analysis} (60 + 20 hours).
\end{itemize}

It is intended here to give essentials of functional analysis in order
to have students prepared for reading scientific literature.

Many now quickly developing questions are connected with the extremal
problems in functional spaces.

The second group of subjects---\textit{specific parts of mathematics
  which are developing in immediate connection with economic problems
  and computations} (350 + 270  hours).
\begin{itemize}
\item[(1)] \textit{Linear programming} (40 + 40 hours). Elementary
  economic problems are given here; essential theorems of linear
  programming ; calculation methods. 

\item[(2)] \textit{Dynamic programming}\pageoriginale (60 + 30 hours). Bellman's
  schemes of extremal problems concerning many-staged processes.

\item[(3)] \textit{Investigation of operations and theory of games.}

\item[(4)] \textit{Electronic computers} (70 + 70 hours).

Practical work at the university computing centre is included.

\item[(5)] \textit{Application of mathematical methods and electronic
  computers in economic planning and economic analysis (60 +40 hours).}

The third group of subjects consists of 

\item[(6)] \textit{Optional special courses and seminars,}

(e.g., input-output analysis, problems of mass service ; etc.).
\end{itemize}

Finally, the independent work of students consists of the following
tasks :
\begin{itemize}
\item[(1)] Practical work at electronic computers.

\item[(2)] Practical work in industry dealing with the organization of
  production at machinery-building works.

\item[(3)] Practical work in industry aimed at the application of
  mathematical methods. 

\item[(4)] University diploma paper, which is to be carried out under
  the direction of an instructor or scientific research worker of the
  faculty laboratory.
\end{itemize}

\begin{center}
\textsc{Conclusion}
\end{center}

The problems of mathematical education that we have considered and
which may seem somewhat too special are, in fact, of vital
importance. 

The planning of economics at every level must be based upon exact
scientific methods. This is even more true of the planning of
economics for the state or country as a whole, which, though most
characteristic of socialistic economics, is inevitable for any country
which tries to rise rapidly to a high economic level. The exact
scientific\pageoriginale methods being just mathematical methods, mathematical
economics and therefore the instruction in it thus proves to be of
great importance which will grow in the next few years.

Mathematical linguistics is expected to acquire great practical
importance too. The rapid development of science, and the growing
number of nations who begin to play a considerable part in this
development, leads to such a growth of the amount of scientific
publication in various languages, that it becomes quite impossible for
a scientist to read all the papers in his speciality.

Some twenty years ago the western scientists could restrict themselves
to the knowledge of western European languages which are very close to
each other and have, at least, one and the same alphabet and a lot of
words and expressions in common. For instance an English-speaking
scientist can read a special paper in French or German. But now
East-European languages, especially the Russian language, have taken
the floor, and western scientists are compelled to study this
language, which proves to be difficult for them. 

The rise of Asiatic and African nations will lead to a situation where
we can e confronted with important contributions to any branch of
knowledge published in a language entirely  alien to the European ones.

Chinese, Hindi, Bengali, Japanese, Indonesian, Arabic are languages
spoken by such numbers of men that it would be strange to expect that
all their scientific product will be presented in languages foreign to
them. At the same time the scholars and technicians of these nations
are confronted with the same problem. 

Therefore one can expect that the problems of concise information, and
of translation from any language into any other one, will soon grow to
an overwhelming size.

A solution of the problem is offered by mathematical linguistics which
looks for the most concise forms of language to accumulate the
information for a mediator-language and for machine translation.

One\pageoriginale can expect that we shall publish papers
simultaneously in the mother language and in the artificial mediator
language especially\break adapted for machine translation. This will reduce
the problem to industrial machine work instead of intellectual efforts
of a scientist or an interpreter.

I think that these remarks on the significance of mathematical
economics and linguistics give a justification of the choice of the
topic of my address, especially at this Conference on Mathematical
Education in Southern Asia.

\bigskip
\bigskip

\noindent
{\fontsize{9pt}{11pt}\selectfont
University of Leningrad}\relax
