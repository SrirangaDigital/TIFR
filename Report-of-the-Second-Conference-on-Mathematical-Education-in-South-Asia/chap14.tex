
\chapter{Progress of Mathematical Education in Indonesia}

\begin{center}
{\em By~} IR. SOEHAKSO
\end{center}

\textsc{During} the Bombay Conference on Mathematical Education, 1956,
Indonesia was represented by a four-man delegation, of which the
present reporter was a member. The  delegation fully agreed with the
resolutions adopted at the conference. The attempts at putting into
practice those resolution and the difficulties thereby encountered
will now be reported.

\medskip
\noindent
\textsc{Scheme of education in Indonesia.}

\smallskip
\noindent
Elementary school : 6 years.
\begin{tabbing}
Secondary school \= : \= 6 years. \= S. M. P. level (first level) 3 years,\\
\> \> \> S. M. A. level (second level) 3 years.\\[0.1cm]
University. \>\>\> Mathematics department Bachelor's\\
\>\>\> and Master's degree 5 years $(3+2)$,\\
\>\>\> Department of technology 4 years, \\
\>\>\> Other departments 4---6 years.
\end{tabbing}

The secondary school, first and second level is divided into
Department A (intended for future students in law, arts, etc.)
Department B (for future students in mathematics, science,
technology), Department C (for future students in economics, etc.). We
will begin with the secondary level, where the difficulties
encountered are the most severe. 

The main difficulties are :
\begin{itemize}
\item[(i)] The lack of sufficiently trained teachers of mathematics. 

\item[(ii)] The general economic situation (high living costs etc.).
\end{itemize}

Whereas (i) is common to most countries in South-East Asia, the
peculiarity for Indonesia is that in our country the difficulties to
be surmounted are tremendous.

\noindent
\textsc{The cause.}\pageoriginale The reason for this is that the
increase of pupils and schools surpasses by far the supply of
adequately trained teachers. To give an idea of the proportions
involved, let us mention that during the period before independence
only the main towns of Djakarta, Bandoeng, Semarang, Soerabaja and
Jogjakarta had secondary schools, S.M.A. level. Moreover 90\% or more
of the teaching staff was Dutch.

After a very short period, almost every town of any significance will
have its S.M.A.

\noindent
\textsc{Supply of teachers.} Waiting for the graduates from the
universities and teachers' training colleges takes too long a time. To
meet the urgent need, the government is forced to set up ``emergency
courses'', the so called B I and B II courses. Even this
unsatisfactory solution is  insufficient to meet the great demand for
teachers, especially in the outer islands. Therefore, second year
students of the universities are recruited as volunteers to give two
or three years service as secondary school teachers. Needless to say,
this state of affairs is most unsatisfactory from the view point of
upholding the standards of the secondary schools, though it is the
best and only possible solution.

On the one hand, it is necessary for Indonesia to accelerate the
increase of schools, but on the other one may raise the question
whether too great an acceleration at the cost of standards it wise
policy. This, however, is for the Ministry of Education to decide.

In attempting to put the ideas of Professor Choquet (Teaching in
secondary schools and research, Bombay Congress, 1956) into practice,
we have modernized the courses for the Teacher Training Faculties so
as to include subjects such as modern algebra, set theory, etc.

The main problem is however to reach the teachers \textit{now} is
charge of teaching mathematics in secondary schools. The majority of
these teachers indeed lack a knowledge of the trends in modern
mathematics. We have proposed do set up vacation-courses in the
centres at Bandoeng and Jogjakarta.

\medskip
\noindent
\textsc{The budgetary difficulty.}\pageoriginale With this we come to
the most difficult problems of all for Indonesia, those arising from
the general economic situation. These problems however lie outside
our scope. But it may be noted that if a teacher, after his full daily
task, is forced (in order to be able to support his family) to work
till late in the evening every day of the week (often giving lessons
in private schools) then he is not entirely to be blamed if he lacks
the energy and physical resources to study and keep himself up to
date. (The average salary of a secondary school teacher is
Rp. 1000---to Rp. 2000---monthly. Free market rate of change :
Rp. 200---equivalent to 1 dollar.)

\medskip
\noindent
\textsc{Courses and Textbooks.} There exist several translations of
Dutch textbooks.

Subjects taught at the secondary school level are : plane and solid
geometry, coordinates and some analytical geometry, descriptive
geometry, arithmetic, algebra and some calculus. This is in fair
accordance with the subjects outlined in Bombay in 1956. Statistics is
not taught at all, not because we do not realize its importance, but
simply because of the total lack of trained teachers. 

We now turn to the university level, where the difficulties
encountered are less severe, though still very
considerable. Considering the general economic situation, partly the
consequence of shortage in industries, engineers, technologists, it is
understandable that administrators urge a catching up in technology in
the shortest possible time. However, the over repeated, and ever
stronger, appeal from the side of government officials to scientists
that ``they do not lose themselves in speculative and pure sciences,
but direct their efforts to advancing technology'', contains, what has
been called by Professor Stone (Some crucical aspects of mathematical
education, Bombay, 1956), the technological fallacy.

The matter was discussed at length in a special session of the science
section of the Mipi congress (Congress of Indonessian Scientists,
Malang, 1958), participated in also by the present reporter. The
meeting agreed that a ``first rate technology cannot be
expected\pageoriginale beside third rate mathematics and pure sciences
in general'' (Presidential address, Professor Chandrasekharan, Bombay,
1956). It is justifiable, for Indonesia, to give every priority to
technological research, which however \textit{must not mean} a neglect
of the pure sciences, including mathematics. The problem of
mathematical education was also discussed during the Malang Congress
by Professor Dr. Wesley L.Orr, visiting professor from the University
of California, in his address : Education for professional
engineers. Professor Orr's proposal was that the mathematics and
science taught in  undergraduate programs ought to be at a more
advanced level, with much of what is now taught by the colleges being
forced back into the secondary schools.

Indonesia lacks a mathematical tradition like those of India, Japan
and China. It seems therefore wise for us to send our postgraduates
abroad, at least during the near future. In fact the mathematics
departments (Bachelor's and Master's degrees) in Jogjakarta and
Bandoeng are very young. Both were established after 1950. The
teaching staff consists of Indonesians and some American
specialists. (Kentucky team for Bandoeng ; U.C.L.A. and I.C.A. teams
for Jogjakarta). The universities are free to set up their own
curricula. Where possible we have followed the outline of the Bombay
resolution. For example, following the suggestion of Professor Stone,
mathematical logic, upto then not taught in the mathematics
department,is now a subject. Examinations, both written and oral, are
given twice a year. 

\bigskip
\noindent
\textsc{Library.} In one word, our libraries are insufficient,
especially for research. the University of Indonesia in Bandoeng is a
little better off than that in Jogjakarta, because it has taken over
the mathematics library of the Bandoeng technological faculty from the
Dutch. This is a point where we sincerely hope to get foreign aid from
universities and institutions abroad. 

Other difficulties are that too much lecture time is required from the
staff, that too little time is available for research, that there is
too\pageoriginale much bureaucracy (for instance in the ordering of
periodicals, books, equipments, etc.).

We have outlined above frankly our difficulties. The task lying ahead
of us, namely, the rapid development of the mathematical sciences in
Indonesia, is huge. The difficulties are many and diverse, several of
them quite beyond the powers of the scientist. But with God's help, we
are confident of surmounting them.


\bigskip
\bigskip

\noindent
{\fontsize{9pt}{11pt}\selectfont
University of Gadjah Mada\\
Jogjakarta}\relax
