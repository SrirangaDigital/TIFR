\chapter{Reports of Working Groups}

\begin{center}
{\large\bf 5}
\end{center}
\medskip

\setcounter{pageoriginal}{192}
\noindent
\textsc{Chandrasekharan}~: We\pageoriginale shall first discuss the introducting of differential geometry for the first degree course (18-20 years). At present some formulae about plane curves are discussed, and the student is confused as to whether he is studying differential geometry or calculus. Should there be a bifurcation into analysis and differential geometry ? In the conference held here four years ago Professor Alexandrov pointed out that the differential geometry of curves and surfaces should be introduced as a part of the first degree course. I do not know how it is in France.

\smallskip
\noindent
\textsc{Cartan}~: It is the same in France.

\smallskip
\noindent
\textsc{Newman}~: We do not have either the differential geometry of curves or surfaces. Elementary things about curves, but nothing more.

\smallskip
\noindent
\textsc{Chandrasekharan}~: To teach relativity in the M.A. one needs differential geometry.

\smallskip
\noindent
\textsc{Newman}~: Some differential geometry topics are treated in applied mathematics !

\smallskip
\noindent
\textsc{Fenchel}~: Differential geometry is taught as a continuation of analytic geometry in Denmark, in the second year at the university. Kinematics, fields of vectors, etc. are introduced at this stage.

\smallskip
\noindent
\textsc{Vaidya}~: The teaching of differential geometry presupposes a knowledge of analytic geometry but one does this only at 18. It is difficult to introduce differential geometry of curves and surfaces at this stage. Is it advisable ?

\smallskip
\noindent
\textsc{Chandrasekharan}~: We are discussing the introduction of differential geometry at 20 not at 18, so the problem does not arise.

\smallskip
\noindent
\textsc{Nagappa}~: Differential geometry is studied only as a special subject in Mysore.

\smallskip
\noindent
\textsc{Nagendranath}~: Does\pageoriginale the present B.A. degree course in Bombay include differential geometry ?

\smallskip
\noindent
\textsc{Chandrasekharan}~: The B.A. syllabus includes simple theorems on the differential geometry of plane curves.

Let us now turn to topology. General topology has only recently been introduced at the M.A., M.Sc. stage, at 20 or 21. Very few universities have offered it even as an optional subject. No one however has dared to introduce algebraic topology at any stage in any form, e.g. the Jordan curve theorem is never considered---not even stated. Can algebraic or geometrical topology be introduced at any stage in the curriculum, or a full paper on set topology ?

\smallskip
\noindent
\textsc{Newman}~: The course should be capable of being altered from year to year. We do this in Manchester. The amount in the syllabus can gradually be increased from year to year instead of in big jumps. It takes about 5 to 10 years to change the syllabus completely in this way.

\smallskip
\noindent
\textsc{Chandrasekharan}~: What about the topology of metric spaces ? Can't it come in an analysis course ?

\smallskip
\noindent
\textsc{Newman}~: No.

\smallskip
\noindent
\textsc{Chandrasekharan}~: Not even in a Master's degree course ?

\smallskip
\noindent
\textsc{Newman}~: Then, perhaps. First I introduce set topology, then geometric topology; first the students should familiarize themselves with the notions of set topology, such as mappings.

\smallskip
\noindent
\textsc{Chandrasekharan}~: In Bombay set topology comes in twice; one repeats for metric spaces what was done in analysis for the real line.

\smallskip
\noindent
\textsc{Stone}~: What do you do in Malaya and Singapore ?

\smallskip
\noindent
\textsc{Eliezer}~: In Singapore in the honours course (equivalent to the British honours course) there is a course in set topology, with a little\pageoriginale algebraic topology. They do quite an advanced set topology course.

\smallskip
\noindent
\textsc{Chandrasekharan}~: Do they do Tychonoff's theorem, compactification theorem, etc. ?

\smallskip
\noindent
\textsc{Eliezer}~: Yes.

\smallskip
\noindent
\textsc{Chandrasekharan}~: Then it is the same as the M.Sc. in Madras.

\smallskip
\noindent
\textsc{Artin}~: Why should there be difficulties ? I have never met them with students.

\smallskip
\noindent
\textsc{Newman}~: I cannot agree: students will reproduce what they are told, but with the right questions their lack of knowledge will show.

\smallskip
\noindent
\textsc{Chandrasekharan}~: I tend to agree with Professor Newman. A student who had a full year's course in set topology and who when asked whether the set of integers was connected, and whether the set of rationals was connected, could not give any answer, said that the set of real numbers was not connected. The students know the accurate definitions, not examples. This is a result of bad teaching.

\smallskip
\noindent
\textsc{Cartan}~: It is very difficult to compare France and India. In France the general notions of topology are taught and its applications to analysis are shown with many examples. They know connectedness with examples, (so they will know the reals are connected). At this level topology should be taught, not for itself, but for its use in analysis.

\smallskip
\noindent
\textsc{Chandrasekharan}~: Metric spaces then suffice.

\smallskip
\noindent
\textsc{Cartan}~: Oh yes!

\smallskip
\noindent
\textsc{Chandrasekharan}~: This is what Professor Oke also suggested for Bombay.

\smallskip
\noindent
\textsc{Akizuki}~: In Japan metric spaces are included in the analysis course. The course in topology includes both general and algebraic topology.

\smallskip
\noindent
\textsc{Krull}~: I can't see any difficulty in introducing set topology at an early stage.

\smallskip
\noindent
\textsc{Chandrasekharan}~: No.\pageoriginale In fact we have introduced set topology in Bombay. In India an M.Sc. is a necessary and sufficient condition for one to be a university teacher. Is it so in the Sorbonne ? Here one can take an M.Sc. at 19 or 20 and become a lecturer.

\smallskip
\noindent
\textsc{Cartan}~: No.

\smallskip
\noindent
\textsc{Alexandrov}~: In Soviet universities, coordinate geometry is introduced in the 1st year (at 18), differential geometry in the 2nd year: curves in space, Gauss-Bonnet theorem if there is time (but I could not prove it in the time) and geodesics are introduced. In the 3rd or 4th year a course on topology, rather as Professor Newman suggested, is given. Elements of general topology are given in the course of analysis. This is necessary for functional analysis, integral equations, etc. It is quite impossible to avoid it.

\smallskip
\noindent
\textsc{Chandrasekharan}~: We have an optional paper on functional analysis and integral equations in Bombay.

\smallskip
\noindent
\textsc{Krull}~: At Bonn a 2nd year course was given on elementary general topology, with fixed point theorems included, and was greatly enjoyed by the students.

\smallskip
\noindent
\textsc{Newman}~: I am all in favour of having topology in an analysis course, but not general topology.

\smallskip
\noindent
\textsc{Fenchel}~: I want to ask Professor Newman in connection with his lecture what he does before geometric topology.

\smallskip
\noindent
\textsc{Newman}~: All we have before the course I outlined is: closed and open sets on the line or in the plane, frontier, closure and interior.

\smallskip
\noindent
\textsc{Artin}~: In the U.S.A. and Germany properties of closed bounded sets and open sets are used in analysis; it depends on the way the calculus is taught. Metric spaces are given only as examples of general topological spaces.

\smallskip
\noindent
\textsc{Stone}~: I disagree with this point. It is a waste of time developing topology first for the line, then for the plane, then for 3 dimensional space.\pageoriginale The proper thing is the metric space, followed by uniform convergence. Show that the geometrical methods which work in the line or plane hold for infinite dimensional spaces also. Discuss examples and finally generalize to a topological space. If you want to teach deeper aspects of 2-dimensional topology leave it till later.

\smallskip
\noindent
\textsc{Newman}~: Do the thing for 2-dimensions but so that it can go over to an arbitrary metric space.

\smallskip
\noindent
\textsc{Stone}~: This kind of discussion arises in other cases. Should you integrate functions of one variable before of several variables, and those of several variables before those on an abstract space ? It is my opinion that a sufficient degree of generality should be introduced as early as possible.

\smallskip
\noindent
\textsc{Chandrasekharan}~: Let us now turn to methods of mathematical physics. In many Indian universities the courses for mathematics are composite. But in some universities like Calcutta there is a clear division. In 1956 we thought that everyone should have some knowledge of ``mathematical methods'', and ``mechanics of continuous media'', since some of those may become engineers or teachers of engineers. We grouped under the first heading various topics: ordinary differential equations (existence theorems), boundary value problems, operational calculus (essentially Fourier-Laplace transforms).

\smallskip
\noindent
\textsc{Artin}~: No Hilbert space ?

\smallskip
\noindent
\textsc{Chandrasekharan}~: No, but in Bombay there is a compulsory paper on real variables and an optional paper on functional analysis. 

\smallskip
\noindent
\textsc{Racine}~: Is an M.A. in mathematics ever a preparation for engineering ?

\smallskip
\noindent
\textsc{Chandrasekharan}~: Yes, for a degree for engineering research. Also the teachers for engineers are required. For example, for many years this Institute did not take any research students in theoretical physics who had not had a Master's degree in mathematics. Research in applied mathematics is poor in India. There is no point in doing statics, dynamics of a particle, dynamics of a rigid body,\pageoriginale etc. from Loney, if at the end of the course, when a student is asked whether a handkerchief is a rigid body, he answers `Yes'. This ``methods-paper'' is really a paper on differential equations.

\smallskip
\noindent
\textsc{Racine}~: Many applied mathematicians do not know any pure mathematics except some methods of solving some differential equations.

\smallskip
\noindent
\textsc{Newman}~: In Manchester there are 2 common years for both pure and applied mathematics. In the final year the applied people are given a strong course in mathematical methods which teaches them a lot of mathematics which the pure people don't learn.

\smallskip
\noindent
\textsc{Racine}~: What about integral equations ?

\smallskip
\noindent
\textsc{Chandrasekharan}~: They come in the functional analysis paper for which we follow the recent book by Kolmogorov and Fomin. What we need really is a paper on differential equations.

\smallskip
\noindent
\textsc{Racine}~: Potential theory ?

\smallskip
\noindent
\textsc{Chandrasekharan}~: Optional.

\smallskip
\noindent
\textsc{Akizuki}~: Group representations.

\smallskip
\noindent
\textsc{Chandrasekharan}~: Not possible, but we do it at a later stage, for instance in this Institute.

\smallskip
\noindent
\textsc{Fenchel}~: Concerning methods of mathematical physics we find we cannot satisfy the physicists. Much of what they do (e.g. quantum mechanics) is not mathematics at all, and we do not know how to translate it into mathematics. But we put in a selection of ordinary and partial differential equations. Fourier analysis and some functional analysis. The rest is left to the physicists, and to optional courses.

\smallskip
\noindent
\textsc{Chandrasekharan}~: We do much the same; we do {\em not} call it `methods of mathematical physics', but just `mathematical methods'.

\smallskip
\noindent
\textsc{Fenchel}~: We taught mechanics, but the physicists complained, so we gave it over to them.

\smallskip
\noindent
\textsc{Chandrasekharan}~: Let\pageoriginale us turn now to real variables. Should differentiation be included in a course on Lebesgue integration ? And should the integral be done for the plane, or for $n$-dimensional space or for an abstract space ? We have stuck to the plane in Bombay, pointing out that it carries over for $n$-dimensions. Should the differentiation theorem for monotone (or monotone continuous) functions be proved ? Should we do Fubini's theorem ?

\smallskip
\noindent
\textsc{Racine}~: Teach a little but teach it well. This is not so in Madras; a bit more so in Bombay.

\smallskip
\noindent
\textsc{Chandrasekharan}~: In Madras the M.Sc. is too specialised.  The M.A. student follows his teacher. He may know a great deal of algebra but no topology or Lebesgue integration or vice versa. If we were going to teach Lebesgue integration how should we do it ?

\smallskip
\noindent
\textsc{Stone}~: There is no difference between one and infinitely many dimensions; after all a measure preserving transformation takes one case into the other. Take the abstract approach. Regarding differentiation, from the point of view of completeness, put differentiation at the beginning as in Riesz-Nagy. But differentiation is a blind alley.

\smallskip
\noindent
\textsc{Chandrasekharan}~: We have done precisely that in Bombay.

\smallskip
\noindent
\textsc{Artin}~: It does not matter whether or not we do one particular theorem.

\smallskip
\noindent
\textsc{Cartan}~: What is meant by differentiation ?

\smallskip
\noindent
\textsc{Chandrasekharan}~: Differentiation of a function of one real variable, or differentiability of the integral.

\smallskip
\noindent
\textsc{Cartan}~: How is it possible at this stage to do Lebesgue integration ? It takes a lot of time, it is too much!

\smallskip
\noindent
\textsc{Chandrasekharan}~: But a great deal of stuff on line integrals, Stokes' theorem (with incorrect proof) etc. is cut out. This is not a first course on integration. The Riemann integral has already been done\pageoriginale at the B.A. stage. This course is at the age of 20. It is the last regular university course.

\smallskip
\noindent
\textsc{Artin}~: In Princeton it is minimal !

\smallskip
\noindent
\textsc{Stone}~: Do physicists learn Lebesgue integration ?

\smallskip
\noindent
\textsc{Chandrasekharan}~: They need Stieltjes integration.

\smallskip
\noindent
\textsc{Stone}~: Why should there be any difficulty in introducing Lebesgue-Stieltjes integration along with Lebesgue integration ?

\smallskip
\noindent
\textsc{Chandrasekharan}~: Let us turn to complex function theory. In Bombay we have a relatively simple compulsory paper and an advanced optional paper.

\smallskip
\noindent
\textsc{Artin}~: The minimal amount necessary is what is in Knopp Vol. I puls $\frac{1}{4}$ of Vol. II.

\smallskip
\noindent
\textsc{Chandrasekharan}~: I have one objection to this book, to the chapter on Riemann surfaces. It is a bit vague.



