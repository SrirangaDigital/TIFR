
\chapter{Contents and Methods of an Algebra Course}

\begin{center}
{\em By~} EMIL ARTIN
\end{center}


\setcounter{pageoriginal}{4}
The\pageoriginale planning of a year's course for graduate students giving an
introduction to abstract algebra will depend on many details which one
cannot consider at our conference since they will depend on local
conditions. The preparation of the students will have to be taken into
account, their general ability, the completeness of the library and
other things will play a role.

I am therefore looking at my address merely as a kind of report on
personal experiences with such courses in the United States and in
Germany. In the case of an algebra course we are in the happy position
that we barely will have to quarrel about the axiomatic approach. If
you drop the axiomatic method the whole course loses its sense and
will become extremely clumsy. Opponents of this methods frequently
point out difficulties for the student. In my opinion they are
nonexistent. I have had very poorly prepared students; it suffices in
such a case to make a few concessions at the beginning and to go a
little slower. If the student is not only poorly prepared but also
poorly gifted, he should be discouraged from studying mathematics.

A point that I wish to mention is the question of text books. I am
opposed to them, probably because of my European background. In my
lectures I merely give a few reference books which the student may
consult. I find it hateful go give a course where I have to plough may
way through chapter after chapter of a given book. The liveliness of
the lecture, which is meant to give an impetus to the students, would
suffer tremendously. I have never used a text, not even in calculus
courses in the United States. As a compromise I write much more on the
blackboard than is customary, so that a student who takes down these
notes can reconstruct at home all the details of the lecture without
difficulty. One has tried for many years to improve\pageoriginale the calculus
courses. The main obstacle in the eyes of many teachers has always
been the lack of suitable text books.

Before we discuss in detail the subject matter of such a course allow
me one more remark. By now algebra has also become a tool to be used
in other branches of mathematics. The algebraist should therefore have
in mind the needs in other fields such as topology, analysis and
algebraic geometry. The planning of the course will therefore depend
again on local conditions of the university.

If you glance at the topics I am going to list you will see that they
cannot possibly be taught in the time at our disposal. On the other
hand one should not devote more time to a general algebra course since
the fantastic growth of mathematics makes it imperative that the
student devotes his time also to other subjects of mathematics. The
lecturer will therefore have to make a choice among the topics and to
compromise in many cases. One way out is to change the content each
time the course is given, another to relegate some topics to
seminars. Let us now look at the details. Since the fundamentals are
of prime importance I shall devote more space to them than to the
later parts.

\begin{enumerate}[(1)]
\item I usually start the course with a review of elementary set
  theory, giving the symbols for union, intersection, inclusion,
  cartesian product and similar things. The notion of an equivalence
  relation $R$ in a set $S$ leads to the quotient space $S/R$ of the
  equivalence classes of $S$ with respect to $R$. The notion of a map
  $f: S \to T$ of a set $S$ into a set $T$ is introduced, together
  with such notions as ontoness of the map, one to one correspondence
  and inverse image of a subset of $T$. For the later discussion of
  homomorphisms it is convenient to discuss already on the set
  theoretical level the canonical decomposition of $f$ into three maps :
$$
S \to S/R \to f (S) \to T.
$$

The notions of a partial, respectively total ordering of a set belong
to this section. As a transfinite axiom one will need Zorn's
lemma. Because of lack of time I only mention its equivalence with the
axiom of choice.

\item In\pageoriginale preparation for the usual topics of algebra one may study in
  general compositions of elements of sets. Given three sets $A$, $B$,
  $C$ a composition would be a map of the cartesian product $A \times
  B$ into the set $C$ and be denoted either by a neutral symbol as $a
  \vee b = c$ or by addition $a + b = c$ or multiplication $ab
  =c$. Examples of ``natural'' compositions are easy to
  give. Frequently $A = B = C$ in which case the composition is called
  internal. If $B = C$ it is called external and $A$ may also be
  called an operator domain of $B$. One may now discuss the
  associative law and its simplest consequences, in particular for a
  composition of a finite number of elements. The meaning of a neutral
  element (zero element or unit element) follows. If two compositions
  are given one can discuss the distributive laws.

\item Next comes the notion of a group. It is convenient also to
  introduce the term semigroup, if the composition satisfies only the
  associative law. The decomposition of a group into cosets with
  respect to a subgroup is the first important theorem.

In order to save time later on it is advisable to introduce the
concept of a group with operators and to keep the operator domain
fixed for all the groups to be considered.

It is time to study homomorphisms $f: G \to H$ between two groups
with the same operator domain, and the notion of the kernel of $f$. As
an example of a homomorphism we have the canonical map $G \to G/ U$
onto the factor group with respect to a normal subgroup $U$. Finally
one proves the fundamental theorems for homomorphisms.

For an immediate application one can now define rings and modules over
a ring and study homomorphisms between these objects.

\item Specializing rings one obtains the notion of a field $F$ (not
  necessarily commutative) and that of a vector space over $F$.

At this point we must deviate into a general discussion : 

In most countries it is by now customary to give the undergraduate
course on analytic geometry in the spirit of the book by Schreier\pageoriginale and
Sperner. This method has such tremendous advantages over the
old-fashioned way that I shall assume here that the students have had
a course that is at least mildly in this vein. The students will
therefore be already acquainted with vectors, matrices and
determinants. Nevertheless a review of linear algebra should be given
and the main theorems on linear equations be derived again (without
determinants). This may be supplemented by other notions of linear
algebra, as that of the dual of a space and the study of the
endomorphisms of a space. 

\item Ideals in a commutative ring $R$. One would begin with the
  residue class ring $R/\mathfrak{a}$ of $R$ modulo an ideal
  $\mathfrak{a}$ and the canonical map $R \to R / \mathfrak{a}$ with
  all its properties. This gives the interconnection of the ideals of
  $R/\mathfrak{a}$ and those of $R$. The most important concept is
  that of a prime ideal and it is necessary to prove corresponding
  existence theorems. This may be done most efficiently as follows :
  If $M$ is a non-empty multiplicative semigroup contained in $R$ and
  $\mathfrak{a}$ an ideal disjoint from $M$ on can prove by Zorn's
  lemma the existence of an ideal $\mathfrak{p}\supset \mathfrak{a}$
  which is still disjoint from $M$ and maximal with respect to this
  property. It is easy to see that $\mathfrak{p}$ is a prime
  ideal. This theorem has several consequences. If one defines the
  radical of $\mathfrak{a}$ as the set of all $\alpha \in R$ such that
  some power $\alpha^i \in \mathfrak{a}$ one sees easily that this
  radical is also the intersection of all prime ideals which contain
  $\mathfrak{a}$. Should $R$ have a unit element one may take $M$ to
  consist only of this element and one sees that any ideal
  $\mathfrak{a} \neq R$ is contained in a maximal ideal. For an ideal
  $\mathfrak{a}$ to be maximal it is necessary and sufficient that
  $R/\mathfrak{a}$ be a field.


As an exercise for the students they should prove that the set of all
primes above $\mathfrak{a}$ satisfies the condition of Zorn's lemma if
one orders the prime ideals by inclusion but in the descending
direction. One obtains now the minimal prime ideals above $\mathfrak{a}$.

These theorems have found important applications in all branches of
mathematics. 

It is clear what is meant by the ideal generated by a subset $S$ of
$R$ and how the elements of this ideal can be expressed in terms of $S$.

\item Construction\pageoriginale of new algebraic structures.

\begin{itemize}
\item[(a)] The quotient field of an integral domain. As an exercise to
  this construction the student may work out for himself the following
  generalization :

Let $R$ be a commutative ring, $S$ a non-empty multiplicative
semigroup contained in $R$. It is desired to construct a new ring
$R_s$ consisting of formal quotients $\dfrac{a}{s}$ where $a \in R$,
$s \in S$ with the following equivalence relation $: \dfrac{a_1}{s_1}
\sim \dfrac{a_2}{s_2}$ shall mean the existence of an $s_3 \in S$ such
that $s_3 (s_2 a_1 - s_1 a_2) = 0$. Defining addition and
multiplication in the obvious way one obtains a ring $R_s$. The ring
$R$ may be mapped homomorphically (not any longer isomorphically) into
$R_s$ by $a \to \dfrac{as}{s}$.

\item[(b)] The construction of the ring of polynomials with coefficients in
  a given ring $R$ may be generalized by considering a semigroup $S$
  disjoint from $R$. By $R[S]$ one understands naively all formal sums
  $\sum\limits_s a_s s $ with $a_s \in R$ and $a_s =0$ for almost all
  $s$. Addition and multiplication are defined in the obvious
  way. This is a good opportunity to show the students how this is
  changed into a rigorous construction. Since the $a_s$ describe the
  element, it is also given by a map $f: S \to R$ such that $f(s) = 0$
  for almost all $s$ ; the idea being that $f(s) = a_s$. Now one
  defines $f + g$ and $fg$ in such a way that they agree with the
  result in the naive terminology. The student should verify the ring
  axioms for $R[S]$.

Special cases of this construction are the group ring if $S$ is a
group and the of polynomials in several variables for a suitable
semigroup $S$. By an easy change of this setup one obtains also the
ring of formal power series.

The fact that one can substitute special values for the variables in a
polynomial identity leads to the following generalization : Any\pageoriginale
homomorphism $S \to R'$ of $S$ into an extension ring $R'$ of $R$
induces a homomorphism of $R[S]$ into $R'$.

\item[(c)] Solution of sets of equations. Let $F$ be a given
  commutative field, $F[X]$ the polynomial ring in several (possibly
  infinitely many) variables, $X_i$ which we abbreviate by the symbol
  $X$. Let $S$ be a subset of $F[X]$ and $\mathfrak{a}$ the ideal of
  $F[X]$ generated by S. Does there exist an extension field $E$ of
  $F$ such that the polynomials of $S$ have a common zero $x$ in $E$ ?
  Any such zero is also a zero of $\mathfrak{a}$ and
  conversely. Should $\mathfrak{a} = F [X]$ then $1 \in \mathfrak{a}$
  and a common zero does not exist. If $\mathfrak{a} \neq F [X]$ let
  $\mathfrak{p}$ be a maximal ideal containing $\mathfrak{a}$. The
  field $E = F [X] / \mathfrak{p}$ will contain an isomorphic replica
  $F'$ of $F$. Replacing $F'$ in $E$ by $F$, the canonical image of
  $X$ in $E$ given a common zero of $\mathfrak{p}$ hence of
  $\mathfrak{a}$. The necessary and sufficient condition for the
  existence of such an extension field $E$ is therefore the following
  : There should \textit{not} be an identity
\begin{equation*}
1 = \sum G (X) H(X) \tag{1}
\end{equation*}
with $G(X) \in F [X]$ and $H(X) \in S$.

\item[(d)] Construction of an algebraically closed extension field $K$
  of $F$. Let $f$ range over all non-constant polynomials in one
  variable with coefficients in $F$, but use for each $f$ another
  variable $X_f$. The totality of all $X_f$ is abbreviated by $X$. Let
  $S$ consist of all $f(X_f)$. By a simple division argument and
  consideration of the degrees, one sees that an identity (1) does not
  exist for this $S$. Thus there exists an extension field $F_1$ in
  which each $f$ has a zero. Replacing $F$ by $F_1$ one finds a field
  $F_2$ in which each polynomial of $F_1$ has a zero. Thus we obtain a sequence
$$
F \subset F_1 \subset F_2 \subset \ldots
$$

The union $K$ of all the $F_i$ is obviously algebraically closed. At
this stage one cannot yet prove that one can find an algebraic
extension of this type.

\item[(e)] As a preparation for topology or for homological algebra
  one could discuss multilinear maps and the tensor product of modules.
\end{itemize}

\item By a\pageoriginale principal ideal ring $R$ one means an
  integral domain with a unit element such that every ideal of $R$ is
  principal. The aim is to prove that every prime is maximal and that
  $R$ is a ring with unique factorization into primes. Examples like
  the ring of integers, that of Gauss integers and the ring of
  polynomials in one variable over a field are obvious. As a
  preparation for many other fields of mathematics one may discuss the
  main theorem on abelian groups, generalized to $R$-modules.

\item Let $R$ be an integral domain with unique factorization and
  quotient field $F$. The ring $R[X]$ of polynomials in one variable
  with coefficients in $R$ has again this property. The factorization
  of a polynomial in $R[X]$ is closely connected with the
  factorization in $F[X]$. Although one is usually pressed for time
  one should not omit this section for obvious reasons : One obtains
  the unique factorization of polynomials in several variables and one
  is in  a position to give examples of irreducible polynomials (Eisenstein).

\item Extensions of commutative fields. To this section belong the
  notions of degree of an extension, of algebraic and transcendental
  extensions and that of the degree of transcendency. One should also
  prove that any finite multiplicative subgroup of a field is
  cyclic. After defining  the characteristic of a field one way
  disucss the behaviour of the roots of unity. Many theorems of
  elementary number theory can now be obtained. 

The topics which we have covered up to now have given to the student a
solid foundation so that he should be able to go into more special
theories. I shall not be as detailed in the following sections, since
it is clear in most cases how one would proceed.

\item Galois theory. One of the key theorems is the following : Let
  $\sigma : F \to F'$ be an isomorphism of a field $F$ onto a field
  $F'$, $f(x)$ an irreducible polynomial in $F, f'(x)$ its image under
  $\sigma$. Let $\alpha$ be a root of $f(x)$, $\alpha'$ one of
  $f'(x)$, in some extension field of $F$ respectively $F'$; then one
  can extend $\sigma$ to an isomorphism of $F(\alpha)$ onto
  $F'(\alpha')$ which maps $\alpha$ onto $\alpha'$. This theorem
  together with the fact that different isomorphisms of one field into
  another are linearly independent givens an easy access to the proof
  of the main theorems of Galois\pageoriginale theory. After having
  proved these theorems for separable extensions a brief discussion of
  inseparable extensions should not be omitted, since these extensions
  are needed in algebraic geometry.

\item The elements of valuation theory. The equivalence of the notions
  place, valuation ring and valuation (into some ordered group) would
  be the first topic of this section. Then one proves the extension
  theorem which states that any homomorphism (not trivial) of a
  subring of a field into some other field is induced by a place. As
  an immediate application one can define the integral closure of a
  ring in an extension field and derive its main properties. Further
  applications can be given if one decides to bring in some algebraic geometry.

I shall only mention other topics which I have discussed in various
courses of this type.

They are group theory, the theory of algebras, Noetherian rings  and
modules, homological algebra and algebraic geometry. In my most recent
courses I always concentrated on algebraic geometry due to the rapid
development of this field. The choice of a particular topic is of
course a question of personal taste and of the particular needs of the
department of mathematics.

I have not touched the questions of a suitable bibliography. If need
be, we could work out such a bibliography at this conference.

In closing allow me a few words on the pedagogical side of the
course. We all know that the best planned course can be ruined by a
poor presentations. It is my experience that nothing can be done about
it. The first seminar talk of a student has always revealed to me his
future teaching abilities. I have often tried to improve this by
taking to the student and explaining to him his mistakes. I have had
little or no success whatsoever. On the other hand we know that with
very good students notoriously poor teachers have had frequently
remarkable success. I am therefore of the opinion that we may safely
leave aside any considerations of this problem.
\end{enumerate}


{\fontsize{9pt}{12pt}\selectfont
University of Hamburg}\relax
