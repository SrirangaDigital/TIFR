
\chapter{The Teaching of Probability Theory and Mathematical Statistics}

\begin{center}
{\em By~} B. V. GNEDENKO
\end{center}

\setcounter{pageoriginal}{114}
\noindent
1. \textsc{Introductory Remarks.}\pageoriginale It is known that the
role of any given mathematical subject in the educational process is
determined, no solely and not so much by its logical completeness and
inner beauty, as by its significance for the development of science,
engineering, social science and mathematics itself. The extent to
which a given subject furthers progress in human knowledge, the extent
to which it helps to effect the great task of subjugating the forces
of nature, should determine, and in the final analysis does determine,
its position in the system of general and mathematical
education. Indeed, it is these two requirements that constitute the
basic criterion of the role and position of any subject in the system
of instruction. It is self-understood that on accepting this
principle, we inevitably arrive at the conclusion that the nature of
instruction must be constantly changing, that the leading role must
pass from one subject to another in the course of time. And indeed,
the history of education shows us this process rather
vividly. Properly speaking, progress consists in the fact that within
one system of opinions, there grows up---under the influence of social
experience, under the impact of scientific discoveries and technical
inventions---a new set of views which allows us to regard the old
subjects from a different, broader standpoint, and to see what has
been concealed from the mental vision of preceding generations. 

The foregoing is not meant to belittle the importance of subjects of
general educative value, nor of subjects helping to develop our
conceptions of beauty. There can be no harmony in the development of a
human being, if he is unable to feel the charm of beautiful
architectural ensembles, to admire the beauties of poetry, to derive
pleasure from enchanting melodies or from works of art. I do not
mention this for the sole reason that mathematics has acquired a
tremendous utilitarian significance in our age of exceptional
scientific, social\pageoriginale and technical development. The
serious development of physics, most fields of engineering, biology
and even linguistics is based to a considerable extent on the level of
development of mathematics, on the potentialities of its
representatives. The importance of rearing in students the capacity
for theoretical probability ideas and intuition, training them in the
conceptions which elucidate the role of stochastic elements in the
formation of natural phenomena and technical processes, is becoming
clearer from day to day. This task arises when teaching
mathematicians and physicists, as well as biologists, engineers,
economists, and linguists.

The theory of probability has greatly gained in importance during the
past few decades in the development of many domains  of science. One
readily recalls the exceptional progress made by physicists since the
time when the atomistic conception of the structure of matter became
the dominating one and statistical methods began to occupy their
merited position. The considerable successes of astronomy are also
linked with the utilisation of the conceptions and methods of the
theory of probability. We may add that present day engineering, with
its trend towards ever greater automation, is constantly making greater
and more diversified demands on mathematics. And in as much as making
allowance for random deviations of a process from its normal course is
one of the crucial problems of modern mass production technique, the
role of the theory of probability in this field becomes
obvious. Biology and medicine are essentially concerned with the laws
of mass phenomena and must, therefore, inevitably resort to the
methods and results of the theory of probability. Interesting and
promising investigations involving the statistical structure of
language, which are being conducted almost everywhere at present, are
acquiring a special significance in connection with researches on
automatic translation. The branches of the science of economics,
playing so prominent a part in the life of society, resort more and
more to the theory of probability and mathematical statistics in
solving their basic  problems. Finally, we must note the importance of
the theory of probability\pageoriginale in mathematics itself. During
the past years, elegant and very promising new branches of science
have arisen under the direct influence of the theory of probability,
particularly in the theory of numbers and the theory of functions.

The foregoing gives us only a very general idea of the role of the
theory of probability in modern science. This role is already
tremendous and is bound to increase with time. It is self-understood
that this circumstance exerts a great influence on the interest in the
theory of probability, not only as a subject of scientific research,
but as a subject for study. It is not by chance that it is now
included in the curricula in physics, chemistry, economics and
biology, departments of universities and the vast majority of
technical schools.

The problems of teaching the theory of probability in their various
aspects have interested me for over twenty years. I am happy to have
the opportunity of delivering a report to the present congress, in
which I intend to expound the principles of mathematical education in
the Soviet Union and their application to the teaching of the theory
of probability. I should like to express my gratitude to  the
organizers of the congress for inviting me to deliver this report.

\medskip
\noindent
\textsc{2. Some conclusions from the discussion on education.} During
the past year and a half a comprehensive discussion has been held in
the Soviet Union on the nature and subject matter of education at all
its stages. This discussion embraced hundreds of thousands of people
and was conducted both on the pages of special pedagogical journals,
and newspapers and magazines. Educators and scientists, as well as
representatives of all professions and population groups, expressed
their points of view. In a discussion of so wide a scope concerning
one of the most urgent questions of modern times, many diverse and,
sometimes, contradictory points of view were, naturally, expressed and
in some cases quite convincingly motivated. This discussion could not,
of course, help involving the teaching of mathematics. Without
attempting a complete exposition of the question, I shall endeavour to
sum up the final conclusions arrived at by the vast majority of
persons participating in the discussion.
\begin{itemize}
\item[(a)] At the\pageoriginale first stages, the teaching of
  mathematics should be as grap\-hic and as simple to understand as
  possible.

\item[(b)] In senior classes of secondary school the pupils should be
  given specimens of the logical development of mathematics. At the
  same time, the connection of mathematics and its fundamental ideas
  with the problems of practical life should be elucidated.

\item[(c)] The idea of functional dependence should predominate,
  beginning with the primary grades. The principles of mathematical
  analysis---elements of the theory of limits, differential and
  integral calculus---should be presented in the senior grades of
  secondary school.

\item[(d)] Pupils who show an interest in and ability for mathematics
  should be admitted to physics-mathematics and mechanics-mathematics
  departments of universities immediately after finishing secondary
  school.

\item[(e)] The level of mathematical education should be elevated in
  technical, biological, agricultural and economic institutions of
  higher education. In particular, the training of specialists in the
  theory of probability and statistics should be improved.

\item[(f)] The coordination of abstract mathematical theories with
  practical problems should be furthered in all ways during the
  process of instruction. With this end in view mathematics students
  should engage in practical work at various research institutes,
  experimental stations and industrial plants.
\end{itemize}

It is interesting to note that many scientists, especially physicists
and engineers, spoke of the necessity for raising the role of the
theory of probability and mathematical statistics in university
education and for introducing the principles of the theory of
probability into the secondary school curriculum.

In a number of cities of the country the experiment is now being made
of organizing classes in mathematics, where the pupil will be able to
obtain a broader knowledge of the ideas of mathematical analysis and
the principles of numerical analysis, as well as of programming for
electronic computers. 

The\pageoriginale theory of probability has not hitherto been taught
at secondary school, nor are its elements included in the curriculum
at present. Many schoolboys, however, are making their first
acquaintance with this subject at numerous mathematical circles which
are functioning at schools as well as at universities. I conducted one
such circle in my time at the University of Moscow together with
Professor A.A. Liapunov. Its members were pupils from various Moscow
schools. Many years have elapsed since then, and the members of this
circle have managed not only to finish school, but to graduate from
institutions of higher learning as well. Some of them have become
physicists, biologists, physicians. It is interesting to note that
although most of them did not become mathematicians, the seeds sown in
their schooldays have borne fruit. I have recently met a former member
of the circle. He is now working as a surgeon in Leningrad, and has
the reputation of a good specialist. On talking with him I learned
that he endeavours to apply statistical methods to medicine.

Admittance to higher educational institutions in the Soviet Union is
based on entrance examinations. For the vast majority of higher
educational institutions mathematics is one of the basic subjects, the
knowledge of which must be verified on entrance. At present the
requirements at the examinations are very high: elementary algebra,
geometry and trigonometry, elements of the theory of limits. Within
these limits the candidate for matriculation must not only be able to
prove theorems, but to solve problems and clearly represent space
figures. Such a method of selecting university students does not
always ensure the admittance into the university of the most gifted
mathematicians, nor can we be certain that all capable young men have
been enrolled. In recent years many schools have begun the practice of
recommending capable pupils to the corresponding departments. This
approach to the selection of mathematically gifted pupils, combined
with a subsequent talk with the university professor can, evidently,
be regarded as expedient.
\begin{landscape}
{\fontsize{9pt}{11pt}\selectfont
\tabcolsep=4pt
\renewcommand{\arraystretch}{1.7}
\begin{longtable}[l]{llc@{\,}rrrcccccccccccc}
\toprule
& &  &\multicolumn{15}{c}{NUMBER OF HOURS}\\\cline{4-18}
 & & & Total & Lec- & Exer- & Labo- & \multicolumn{11}{c}{Every week} \\[-0.3cm]
 \multicolumn{3}{c}{Subject} &  & tures & cises& ratory & \multicolumn{11}{c}{Semester}\\
& & & & & & & 1 & 2 & 3 & 4 & 5 & 6 & 7 & 8 & 9 & 1 0 & 11 \\
\midrule
1. & Analytical geometry & \ldots & 188 & 102 & 86 & --- & 5 & 6 &
---&---&---&---&---&---&---&---&---\\
2. & Mathematical analysis & \ldots & 528 & 284 & 244 & --- & 8 & 9 &
8 & 6 & ---&---&---&---&---&---&---\\
3. & Higher algebra & \ldots & 208 & 142 & 66 & --- & 6 & 4 & 2 &
---&---&---&---&---&---&---&---\\
4. & Differential geometry & \ldots & 108& 82 & 26 & ---&---&---& 6 &
---&---&---&---&---&---&---& ---\\
5. & Differential equations & \ldots & 136 & 68 & 68 & ---&---&---& 4
& 4 & ---&---&---&---&---&---&---\\ 
6. & Equations of mathematical &&&&&&&&&&&&\\[-0.3cm]
& ~~physics & \ldots & 152 & 112 & 40 & ---&---&---&---&---&---& 5 & 4 &
---&---&---&---\\
7. & Theory of functions of a &&&&&&&&& \\[-0.3cm]
& ~~complex variable & \ldots & 102 & 68 & 34& ---&---&---&---&---&
3&3&---&---&---&---&---\\
8. & Theoretical mechanics & \ldots & 236 & 140 & 96 &
---&---&---&---&5&6&3&---&---&---&---&---\\
9. & Theory of probability & \ldots & 54 &36 &18
&---&---&---&---&---&3& ---&---&---&---&---&---\\ 
10. & Calculus of variations & \ldots & 36 & 36 &
---&---&---&---&---&---& 2 & ---&---&---&---&---&---\\
11. & Physics & \ldots & 282 & 112 & 54 & 116 & ---&---&---&
6&5&6&---&---&---&---&---\\
12. & Selected chapters of theo-& &&&&&&&&\\[-0.3cm]
& ~~retical physics & \ldots & 100 & 100 &
---&---&---&---&---&---&---&---& 2 &4&---&---&---\\
 13. & Theory of functions of a &&&&&&&&\\[-0.3cm]
& ~~real variable  and functio-&&&&&&&&\\[-0.3cm]
& ~~nal analysis & \ldots & 86 & 86 &
 ---&---&---&---&---&2&3&---&---&---&---&---&---\\
14. & Foundations of geometry & \ldots & 72 & 72 &
---&---&---&---&---&---&---&---& 4 & ---&---&---&---\\
15. & Theory of numbers &\ldots & 40 &40
&---&---&---&---&---&---&---&---&---&---&---&4&---\\
16. & Special courses & \ldots & 490 & 290 & 200 &
---&---&---&---&---&---& 4&2&2&8&8&8\\ 
17. & Electronic digital computer&&&&&&&\\[-0.3cm]
& ~~and the preparation of &&&&&&&\\[-0.3cm]
& ~~programmes & \ldots & 104 & 72& ---&32 &
---&---&---&---&---&---&4&2&---&---&---\\
18. & Methods of numerical &&&&&\\[-0.3cm]
& ~~analysis & \ldots & 68 & 68 &
---&---&---&---&---&---&---&---&2&2&---&---&---\\
19. & Mathematical practice & \ldots & 208 & ---&---&208 &
---&---&---&---&6&4&2&---&---&---&---\\
\end{longtable}}\relax
\end{landscape}

\medskip
\setcounter{pageoriginal}{121}
\noindent
3.\pageoriginale \textsc{The curriculum of mathematical education at the
  universities of the Soviet Union.} The course of study of
mathematics at the university lasts 5.5 years, the academic year
contains 34 weeks of studies and a total of about 7 weeks devoted to
examinations (twice a year). Attendance at lectures is obligatory for
students of the day division. Students who attain exceptional success
in their studies may receive permission to take examinations without
attending lectures, but they are then required to submit an individual
plan of their work. The training of specialists in mathematics and
mechanics is conducted in accordance with various syllabi and
curricula. I shall here confine myself to listing the subjects
prescribed for specialists in mathematics, indicating the number of
hours devoted to lectures and exercises, the distribution of courses
by semesters and the number of hours per week for each course. (See
table on pages 120 and 121.)


The diversity of special courses and seminars is very great. They are
not prescribed beforehand and depend to a great extent on the
interests of the professors working at the given university. In
addition, a considerable role is played by the mathematical
specialization developed at the university.

Besides the prescribed courses there are optional courses which the
student may attend if he wishes. Among the optional courses are
Topology, History of Mathematics, Automatic control, etc. During the
seventh semester the students are obliged to write a so called course
theme. The course theme should contain either a scientific result
independently arrived at, or an exposition of some scientific theory
studied independently. As a rule, the course theme is the beginning of
work on the diploma theme, which concludes the course of study at the
university (the diploma theme may be replaced by state examinations).

Of special interest is the so called mathematical practice obligatory
for all mathematics students. It lasts three semesters, each student
is given three or four individual rather complicated tasks requiring
the application of knowledge gained in mathematical subjects which the
student has already passed. As a rule, the student is
required\pageoriginale not only to obtain a formal solution, but to
carry out computations and obtain a numerical result. Several such
tasks were presented by me in the article ``Mathematical education
in the USSR'' (\textit{American Mathematical Monthly,} LXIV, (1957)
389-408). We may add that the tasks for mathematical practice are
compiled anew each year. Somewhat later I shall cite some examples of
tasks assigned to students of the seventh semester at the
Mechanics-Mathematics department of Moscow University specializing
in the theory of probability. In my opinion mathematical practice is
an interesting and useful innovation in mathematical education,
developing scientific initiative. It deserves a much more detailed
elucidation in a special paper, citing a greater number of variations
of the tasks for students of different specializations and years of
study.

Beginning with the fifth semester, each student chooses a narrow
mathematical specialization---theory of probability, functional
analysis, theory of differential equations, theory of functions of a
real or complex variable, the methods of numerical analysis, topology,
etc. Special courses and special seminars are selected by students in
accordance with their narrow speciality. The content of the
mathematical practice is also linked with the specialization in the
fourth year of study. 

In the 8th semester all students of mathematics take practice in
modern computing machines. The students who have chosen the speciality
of secondary school teacher go through teaching practice lasting 22
weeks during the 8th and the 9th semesters. Mathematicians of other
specialities take production practice lasting 39 weeks. The nature of
the productive practice is extremely diversified. It is passed at
industrial laboratories, research institutes, statistical boards,
experimental agricultural stations, etc. 

Students of both mathematics and mechanics take a prescribed 54 hours
course on the theory of probability. One third of the hours are
devoted to exercises, the other two thirds to lectures on theoretical
material. The course, which is prescribed for all mathematics
students, is not extensive. It is confined to the first eight
chapters\pageoriginale of the text-book \textit{Course of the theory
  of probability} by B. V. Gnedenko (2nd edition, Moscow, 1954). This
book has been translated in the Korean People's Democratic Republic,
the Chinese People's Democratic Republic, the German Democratic
Republic and, to the best of my knowledge, is being translated in the
united States of America and the Democratic Republic of Viet-Nam. To
go into detail, the course embraces the fundamental concepts of the
theory of probability expounded on the basis of the axiomatics of
A. N. Kolmogorov, an outline of the sequence of independent tests with
proof of the local and integral theorems of Moivre-Laplace, elements
of Markov chains, elements of distribution functions, the concept of
moments and the fundamental theorems involving them, the law of large
numbers, elements of the theory of characteristic functions and
Liapunov's theorem, elements of the theory of stochastic processes.

\bigskip
\noindent
4. \textsc{Curriculum of specialization in the theory of probability.}
 During the first two years of study the training of all mathematics
 students is conducted in conformity with the general plan outlined
 in the preceding section. The training of specialists along different
 lines begins in the third year. Students specializing in the field of
 the theory of probability take the general course of the theory of
 probability in the fifth semester together with all mathematics
 students, but they have 2 hours per week of exercises (instead of 1
 hour per week in the case of other specialities). In the sixth
 semester probability students take a course of ``Supplementary
 chapters of the theory of probability'' (2 hours of lectures and 2
 hours of exercise per week). In the fourth year of study at Moscow
 University students specializing in the theory of probability are
 given the following prescribed courses :

Stochastic processes (2 hours per week for one year).

Mathematical statistics (7th semester, 2 hours of lectures and 2 of
exercises per week).

Supplementary chapter of mathematical statistics (8th semester, 2
hours of lectures per week).

This\pageoriginale curriculum is cited as an example only, since at
other universities there are different lists of prescribed courses for
theory of probability specialists. Thus, a short prescribed course in
the theory of information (36 hours) is given at kiev university in
addition to the courses mentioned.

In mathematical education an important part is played by special
courses and seminars, which introduce the student directly to the
crucial problems of contemporary research. It is very essential to
offer the student a great diversity of courses and seminars, so that
he can select those which are most to his taste. This is why we try to
have as great a number as possible of different special courses. With
this end in view specialists are invited who work in other educational
and research institutions. To give an idea of the nature of the
special course, we present here a list of special courses and seminars
delivered at Moscow University during the 1958-1959 academic year, as
well as in the current year. 

Special courses during the 1958-1959 academic year.
\begin{itemize}
\item[(a)] Markov processes and differential equations (one year
  course for the 5th year);

\item[(b)] Gaussian stochastic processes (half-year course for the 3rd
  and 4th years);

\item[(c)] Branching stochastic  processes (half-year course for the
  3rd and 4th years);

\item[(d)] Monte-Carlo methods (one year course for the 4th and 5th years);

\item[(e)] Measures in topological spaces (half-year course for the
  4th and 5th years);

\item[(f)] Supplementary chapters of mathematical statistics
  (half-year course for the 4th and 5th years);

\item[(g)] Limit theorems for sums of random variables (one year
  course for the 4th and 5th years);

\item[(h)] Theory\pageoriginale of stationary stochastic processes
  (half-year course for the 5th year).
\end{itemize}

Special courses during the 1959-1960 academic year. 
\begin{itemize}
\item[(a)] Integral transformations (one year course for the 3rd and
  4th years);

\item[(b)] Theory of games (one year course for the 4th and 5th years);

\item[(c)] Monte-Carlo methods (half-year course for the 4th and 5th years);

\item[(d)] Limit theorems and elements of the statistics of stationary
  processes (one year course for the 5th year);

\item[(e)] Theory of information (half-year course for the 5th year).
\end{itemize}

Special seminars during the 1958-1959 academic year. 
\begin{itemize}
\item[(a)] Theory of games (one year, for the 3rd year);

\item[(b)] Selected problems of analysis (one year, for the 3rd year);

\item[(c)] Metric theory of dynamic systems (one year, for the 4th year);

\item[(d)] Applications of the theory of probability and the theory of
  information (one year, for the 4th and 5th years); 

\item[(e)] Limiting theorems (one year, for the 4th and 5th years);

\item[(f)] Mathematical statistics in biological problems (one year,
  for the 4th and 5th years);

\item[(g)] Markov processes and differential equations (one year, for
  the 5th year).
\end{itemize}

Special seminars during the 1959-1960 academic year.
\begin{itemize}
\item[(a)] Selected problems of analysis (one year, for the 3rd year);

\item[(b)] Theory of information (one year, for the 4th year);

\item[(c)] Branching processes (one year, for the 4th and 5th years);

\item[(d)] Stochastic processes (one year, for the 5th year and graduates);

\item[(e)] Probability methods in radio engineering (one year, for the
  5th year and graduates).
\end{itemize}


At\pageoriginale Kiev University seminars were conducted on the theory
of queues, asymptotic expansions, statistics of stationary processes
and others. A seminar has been proposed on the statistical problems of
economics and planning, as well as on linear and dynamic
programming. Students are constantly invited to participate in the
research and reviewing seminars of the statistical department. 

\bigskip
\noindent
5. \textsc{Tasks of mathematical practice.} To give a clearer idea of
the nature and peculiarities of the tasks assigned in mathematical
practice, I shall present some examples of problems proposed in the
fourth year of study at Moscow University in the 1957-1958 academic
year.

\medskip
\textit{Variation} 1.
\begin{problem}
Find the asymptotic formula for the density of a Student distribution
with $n$ degrees of freedom containing terms of the order $n^{-1}$
inclusive. By means of this formula find the asymptotic formulae for 
$$
z = z (x) = \phi^{-1} (F_n(x)), ~~ x= F^{-1}_n(\phi(z)),
$$
($x$ and $z$ are mutually inverse functions). Let $\phi$ be a normal
function of distribution with assembly average 0 and dispersion 1; and
$F_n$ the Student distribution function. The asymptotic formula for
$x(z)$ may be employed for the approximate calculation of the
quantilia of Student distribution with large $n$.
\end{problem}

\begin{problem}%%% 2
A particle moves at random along a finite segment $(x_1, x_2)$. Let
$p(x,y,t)$ be the density of probability of the coordinates of $y$ of
the randomly moving particle at the moment $t$, if at $t =0$ the
particle was at point $x$. The density satisfies the equation
$$
\frac{\partial p}{\partial t} = - a \frac{\partial p}{\partial y} +
\frac{b}{2} \frac{\partial^2 p}{\partial y^2} .
$$
At the absorbing point of the boundary $y = x_i$,
$$
p(x, x_i, t) = 0.
$$
At the reflecting point of the boundary $y = x_i$,
$$
\left[\frac{\partial p (x,y,t)}{\partial y} \right]_{y=x_i} = 0.
$$\pageoriginale
Let us denote by $P(x,t)$ the probability that the particle, beginning
its motion from point $x$ at $t=0$, will be absorbed at the moment
$t$. Let us introduce the limiting conditional density
$$ 
p^\ast (x,y) = \lim\limits_{t \to \infty} \frac{p(x,y,t)}{1 - P (x,t)}
$$
and the limit density
$$
p^{\ast\ast} (x,y) = \lim\limits_{t \to \infty} p (x,y,t) .
$$
Find $p(x,y,t)$, $P(x,t)$ and $p^{\ast} (x,y)$ assuming $a = - 2 $, $b
=1$; the boundary point $x_1 = 0$ reflecting, the point $x_2 = L$
absorbing. 
\end{problem}

\begin{problem}%%% 3
Let $x(t)$ be a stationary Gaussian process. Investigate the
distribution and moments of the following random variable
$$
\int\limits^t_0 x^2 (s) ds.
$$
\end{problem}

\begin{problem}%%% 4
Calculate the limit
$$
\lim\limits_{T \to \infty} \frac{1}{T} M \int\limits^T_0
\int\limits^T_0 \frac{e^{-|t-s|}}{|\vec{\xi} (t) - \vec{\xi} (s)|}
dt \; ds,
$$
where $\xi(t)$ is a three-dimensional random Brownian trajectory.
\end{problem}

\begin{problem}%%% 5
Let $\xi (t)$ be a Markov, stationary, Gaussian process with mean
value $0; \eta(t) = \xi^3(t)$. Find the correlation function and
spectrum of the process $\eta(t)$. Find the best linear and best
nonlinear extrapolation formulae for this process; compare the errors
of the best linear and nonlinear prognosis.
\end{problem}

\medskip
\textit{Variation 2.}
\setcounter{problem}{0}
\begin{problem}%%% 1
The variables $x$ and $y$ are connected by the linear relationship $y=
ax +b$ (the parameters $a$ and $b$ are unknown). For fixed equidistant
values of $x (x_k = x_0 + kh, \; k = 0,1,2,\ldots, n)$ the values,
$\eta_k = y_k + \delta_k = ax_k + b + \delta_k$ are observed, where
$\delta_0, \delta_1, \delta_2, \ldots, \delta_n$\pageoriginale are
mutually independent random errors, subjected to normal distribution
with parameters $(0,\sigma^2)$.

Find effective estimates for $a$ and $b$ and construct on the segment
$(x_0, x_n)$ a five per cent confidence region for the function $y =
ax + b$, both in the case of a known $\sigma$, as well as in the case
of an unknown $\sigma$. The confidence region should have a minimum area.
\end{problem}

\begin{problem}%%% 2
In Problem 2, Variation 1, determine $p (x,y,t), P(x,t)$ and
$p^{\ast\ast} (x,y)$, assuming that $a = 1, ~b=2$, the boundary points
$x_1 =0$ and $x_2 = L$ reflecting. 
\end{problem}

\begin{problem}%%% 3
Let $(u(t), v(t))$ be a two-dimensional Gaussian process with a given
two-dimensional spectral function. Find the spectrum of the process
$X(t) = u(t).v(t)$. Investigate special cases qualitatively. 
\end{problem}

\begin{problem}%%% 4
Calculate the Laplace transformation of the random variable
$$
\int\limits^1_0 \infty^1_0 \left[\xi (s,t) \right]^2 dt \; ds
$$
where $\xi(t,s)$ is a Wiener field of two variables.
\end{problem}

\begin{problem}%%% 5
Let $\xi_0 (t)$ be a Gaussian stochastic stationary process with an
unknown mean value and correlation function $B(\tau)  = C
e^{-\alpha|\tau|}$, and $\xi_1(t)$ a process all sample functions of
which are jump functions with interruptions at random points $\ldots,
t_{-2}, t_{-1}, t_0, t_1, t_2, \ldots$ distributed in accordance with
Poisson's law with parameter $\alpha$; in the intervals between the
interruption points, the values of $\xi_1(t)$ are mutually independent
Gaussian random variables with dispersion $\sigma^2$ and unknown mean
value. The processes $\xi_0 (t)$ and $\xi_1(t)$ are observed on the
segment $0 \leqslant t \leqslant T$. Find for the one and the other
process the best linear estimates of the mean value and estimates by
the method of maximum likelihood; compare the dispersion of these estimates.
\end{problem}

\medskip
\noindent
6. \textsc{Training scientists in the field of the theory of
  probability.} University graduates who have an interest and capacity
for scientific research are, if they so desire, retained at
universities and\pageoriginale research institutes to continue their
training for research work as post-graduate students. Post-graduate
students pass through a three-year course of study under the guidance
of some outstanding specialist in the given branch of science. The
post-graduate student must in the course of three years prepare to
pass examinations in several special subjects and carry out an
independent scientific investigation, which is to be defended in
public. If he has successfully completed his course of studies and
defended his thesis, the post-graduate is granted the first scientific
degree---candidate of sciences (in the case of a mathematical
thesis---candidate of physico-mathematical sciences). Persons who
desire to do so may be admitted to a post-graduate studentship some
time after graduation from the university (or any institution of
higher education). In that case they must pass entrance examination
and submit to the selection commission an independent paper (of a
research or review nature). The post-graduate's curriculum is not
prescribed beforehand, but is determined solely by the scientific
interests of the student. As an example I present the curriculum of
I.N. Kovalenko, who entered upon his post-graduate course under my
guidance in 1957.
\begin{enumerate}
\item \textit{Functional analysis.} Recommended literature :

E. Hille, \textit{Functional analysis and semi-groups} (selected chapters),

L. H. Loomis, \textit{An introduction to abstract harmonic analysis}
(selected chapters),

N. I. Akhiezer and I. M. Glazman, \textit{Theory of linear operators,}

L. Schwartz, \textit{Th\'eorie des distributions,}

papers in journals on the methods of applying functional analysis in
the theory of stochastic processes.

\item \textit{Stochastic processes.} Recommended literature:

J. L. Doob, \textit{Stochastic processes,}

A. M. Yaglom, \textit{Introduction to the theory of stationary
  stochastic functions,}

A. Blanc-Lapierre\pageoriginale et R. Fort\'et, \textit{Th\'eorie des
  functions al\'eatoires} (selected chapters).

\item \textit{Theory of games and decision functions.} Recommended
  literature:

D. Blackwell and M. A. Girshick, \textit{Theory of games and
  statistical decisions,}

A. Wald, \textit{Statistical decision functions,}

papers in journals on the theory of games.

\item \textit{Theory of queues.} Recommended literature :

A. Y. Khinchin, \textit{Mathematical methods of the theory of queues,}

papers in journals on the theory of queues and on
associated\break branches of radio engineering.
\end{enumerate}

In addition review reports were planned at the seminar of the section
on the theory of games and the theory of queues. 

The thesis prepared by I. N. Kovalenko deals with the solution of
several problems of the theory of queues by the methods of Markov
processes and the theory of games.

Besides mathematical subjects each post-graduate must study and pass
examinations on philosophy and one foreign language. 

Since the nature of candidates' theses in the theory of probability
may be of some interest, I shall mention some that have already been
published after the defence. Of course, the scientific value of these
researches is far from uniform, but a single precise measure can
hardly be found for creative work. In the short list appended below,
there are theses defended in Moscow, Leningrad and Kiev. 
\begin{itemize}
\item[(a)] A. A. Petrov, Verification of statistical hypotheses on the
  type of distribution based on small samples, \textit{Journal for the
  Theory of Probability and its Applications,}  1 (1956), 258-271.

\item[(b)] A. V. Scorohod, Limit theorems for stochastic processes,
  \textit{Journal\break for the Theory of Probability and its Applications} 1,
  (1956), 289-319, and Limit theorems for stochastic processes with
  independent increments, ibid. 2 (1957), 146-177.

\item[(c)] V. S. Mihalevic,\pageoriginale Consecutive Bayes solutions
  and optimal methods of statistical acceptance control,
  \textit{Journal for the Theory of Probability and its Applications,}
  1 (1956) 437-465.

\item[(d)] A. A. Iliachenko, Des lois asymptotiques pour les chaines
  de\break Markoff avec un nombre fini d'\'etats, \textit{Ukranian
    Mathematical Journal, X} (1958), 23-36.

\item[(e)] V. Richter, Local limit theorems for large deviations,
  \textit{Journal for the Theory of Probability and its Applications,}
  2 (1957), 224-229, and Multidimensional local theorems for large
  deviations, ibid. 3 (1958), 107-114.
\end{itemize}

The following scientific degree in the USSR after that of candidate is
the doctor's degree. To receive the doctor's degree a research must be
carried out that constitutes a serious contribution to some field of
science. As an example of a research for which the degree of doctor of
physico-mathematical sciences has been awarded in the field of the
theory of probability we may point to the paper of Y. V. Prokhorov,
Convergence of random processes and limit theorems of probability
theory, \textit{Journal for the Theory of Probability and its
  Applications,} 1 (1958), 177-238.


\bigskip
\noindent
7. \textsc{Text-books on the theory of probability.} To get a clear
idea of the principles on which education is based, one must
inevitably turn to the text-books employed in teaching. I should like
to point out several Soviet books which played the principal part in
the education of mathematicians at various periods in the course of
the past twenty years. I shall later point out the books that played
and are playing a part in the teaching of the theory of probability
and mathematical statistics for non-mathematical specialists. 

The principal role in the development of interest in the theory  of
probability was played by S. N. Bernstein's book \textit{Theory of
  probability,} the first edition of which, appeared in 1927, and the
fourth in 1946. This book pursued a number of aims, of which I regard
the following as being the basic ones. First of all, the author
endeavoured to show how the theory of probability can be built on a
strictly axiomatic basis just as geometry, algebra, mechanics
and\pageoriginale other mathematical subjects. The author's viewpoint
is that the initial exposition cannot be detached from customary
conceptions and therefore approaches the formulation of axioms only
after a detailed discussion and qualitative examination of numerous
examples readily comprehensible to the beginner. Further, Bernstein's
book systematically follows the principle that general theoretical
results should be interpreted from the point of view of their
applications. With this aim the author discusses in his text numerous
illustrations taken chiefly from problems of biology and demography. A
third feature of Bernstein's book is that it endeavours to keep up to
the level of the latest advancements in our science. The author,
therefore, introduced substantial supplementary material into each new
edition. However, the original structure of the book was preserved
unaltered. This led to the fact that this text-book diverged from the
ideals of the structure of the theory of probability which were
developed in Moscow.

The first attempt to reflect the viewpoint of the Moscow
mathematicians on the system of the initial exposition of the theory
of probability was made by V. I. Glivenko in his \textit{Course of the
theory of probability} (Moscow, 1939). This book is the first to base
the principles of the theory of probability on a theory of set
approach.

The ideas of building a course of the theory of probability in Moscow
during the past twenty-five years were most completely followed in the
text-book of B. V. Gnedenko \textit{Course of the theory of
  probability} (2nd edition, 1954). These ideas can be formulated in
the following propositions :
\begin{enumerate}
\item General concepts should arise from concrete data taken from the
  most vital problems in science, engineering and mathematics. 

\item At the initial stage of instruction it is best to proceed from
  the particular to the general.

\item General results should be made clear by a detailed discussion of
  examples, after which problems should be proposed for independent
  solution.

\item The\pageoriginale text-book should not confine itself to the
  exposition of concepts and theorems and to discussions of examples,
  but should contain a sufficiently distinct formulation of
  methodological conclusions as to the scientific value of the
  developed theory.

\item The theory of set approach to the construction of the principles
  of the theory of probability is the most natural, making it possible
  not only to integrate the various mathematical subjects most
  completely, but to interpret the rules of operation on random
  variables.

\item A university text-book on mathematics should given an idea of
  the most essential branches of the modern theory of probability
  without excessively detailed discussion of the results of any of them.
\end{enumerate}

At present a number of prominent Soviet specialists in the theory of
probability consider it necessary to compile a text-book which would
expose the logical structure of the subject irrespective of the
discussion of real examples. M. Lo\`eve, \textit{Probability theory,}
is named as a book which approaches this ideal. In my opinion, three
types of text-books, taking into account the psychological
peculiarities of the students and their varied interests, are
necessary for university education.

The text-books of the first type should be built along lines of
abstract construction of theory. They should proceed from broad
general theoretical conceptions and should feel no compunction against
application of the results of other mathematical subjects (topology,
functional analysis, geometry, differential and integral equations,
etc.). Acquaintance with the primary sources of concepts and
formulation of problems is not obligatory in a text-book of this
type. Such a manual is particularly useful for older students, who are
prepared for the apprehension of general theories and are able to
perceive concrete phenomena and diversified applications behind the
abstract concepts and general results.

The text-books of the second type should concern themselves chiefly
with the elucidation of the subject matter based on consideration of
physical phenomena and technical processes. This type of text-book\pageoriginale may
be permitted a less rigorous exposition of logical principles and may
confine itself, in some cases, to the formulation of some results
without proof. The chief aim of such books consists in giving a broad
conception of the subject and a fairly detailed elucidation of the
possibilities possessed by the theory for application of its results
in science, engineering and economics. It is self-understood that
text-books constructed along these lines are particularly essential
fro physicists, astronomers, biologists and engineers. At the same
time, I find them very useful for mathematicians as
well. G. P. Boyev's text-book \textit{Theory of probability,}
published in the USSR, is of this type.

The third type of text-book should combine the essential properties of
the first two types of manuals. Here we proceed from concrete
phenomena and on this basis build up general concepts, then going back
from general concepts to the consideration of concrete phenomena and
problems of practical application. I regard textbooks of this type as
being best adapted for the initial instruction in mathematical
subjects. I consider my \textit{Course of the theory of probability}
as approaching this third type of manual.

There is no doubt that at present a single manual, even if it is a
very good one, cannot suffice for a comprehensive acquaintance with
the subject matter of the theory of probability. A series of
monographs should be written on the crucial problems of the theory of
probability, introducing the reader, fairly quickly and completely, to
the modern trends in theoretical and applied research. Efforts are
being made in the USSR at present to implement this pedagogic and, at
the same time, scientific task. 

We cannot, of course, limit ourselves to writing text-books solely for
education of mathematicians. It is necessary, at the same time, to
produce a series of manuals for specialists in other
fields---biologists, physicists, economists, textile engineers,
mechanical engineers, etc. A number of such books have already
appeared, but there is still in the Soviet Union a lack of books on
probability theory designed for particular specialities.

Mathematical\pageoriginale statistics has not yet attained its merited
development as a subject of instruction in the universities of the
Soviet Union. I consider that we are as taking only the first steps in
this respect. If in the theory of probability we have some fine
traditions, the training of specialists in the field of mathematical
statistics leaves much to be desired. More has been done along these
lines in Uzbekistan than elsewhere, V. I. Romanovsky having created
some traditions there in the domain of statistics, chiefly
applied. The ways of training specialists in mathematical statistics,
both applied and theoretical, are now being seriously discussed at a
number of universities (Moscow, Leningrad, Kiev, Erevan, Vilnius and others).

We in the Soviet Union have a high regard for the scientific results
and traditions attained by Indian scientists in the field of
mathematical statistics. I hope that the knowledge of the principles
of university education which I shall acquire here at the congress and
by visiting the universities of India, will allow me to pass on the
experience of Indian universities to my Soviet colleagues.


\bigskip
\bigskip
{\fontsize{9pt}{11pt}\selectfont
University of Kiev}\relax

