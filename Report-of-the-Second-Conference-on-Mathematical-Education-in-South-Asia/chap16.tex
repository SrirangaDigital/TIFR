\chapter{Progress of Mathematical Education in Pakistan}

\begin{center}
{\em By}~ A. L. SHAIKH
\end{center}
\medskip

\setcounter{pageoriginal}{160}
I\pageoriginale must apologize for not being able, due to short notice, to present to you a detailed report on the progress of mathematical education in Pakistan. I shall content myself with giving an outline of this progress since independence over twelve years ago, particularly with reference to the changing background of general education, which is due to an earnest desire to overhaul the whole system of education in Pakistan with a view to the realization of the new aspirations and ideals of our people.

I must say that our problems and difficulties are similar to those which exist here in this country. Upto the matriculation or even the Intermediate stage the work done in mathematics is about the same as is done here. For the matriculation examination, mathematics is not compulsory, but it is obligatory for those who proceed to take up science. The subjects usually studied are arithmetic, algebra, geometry and trigonometry. Since quite a large number of students wish to take science for further study, about fifty per cent of the students offer mathematics.

After matriculation, a student studies for two more years, and then
appears for the Intermediate examination in mathematics. Those who
wish to take medicine, study biology instead of mathematics. The
subjects usually studied during these two years are algebra,
geometry,\break trigonometry, coordinate geometry, pure solid geometry, calculus and
mechanics. The content and the standard reached in these subjects have
not changed very much during the past decade. Variations however occur
from place to place, e.g. elementary mechanics is compulsory at some
places while it is not so at others. The concept of limit in calculus
is emphasised and explained but $\epsilon$ and $\delta$ methods are
not required and so also proofs of limit theorems etc. 

After\pageoriginale passing the Intermediate examination, the student
goes to engineering or the B.A. or the B.Sc, degree. We have different
courses for the Pass and Honours degrees. The duration also varies at
present from two to three years, but very soon the duration will be
made uniform for three years. For the Honours degree in mathematics,
courses vary from place to place, but it is significant to note that
mathematics courses for the Honours degree are being gradually raised
and modernised. Originally the difficulty was to get properly
qualified young men to teach modern branches of mathematics. But it is
significant of the urge for reform that comparatively a large number
of university teachers in mathematics have either gone abroad on their
own or under the various aid programmes for advanced study and
research in mathematics, with the result that at almost all the
universities there are persons qualified to do this work. And the
stream continues to flow, and we may confidently look forward to
building up strong departments of mathematics at various centres. 

The M.A. or M.Sc. degree in mathematics requires generally one year's study after the B.A. or B.Sc. Honours examination, and two years in case of those who have taken the Pass examinations. The contents of the courses in mathematics for the Master's examination have been rapidly undergoing change, the older branches being replaced by modern subjects such a abstract algebra, group theory, elements of set theory and topology, differential geometry, etc. subject to local conditions and availability of qualified teachers. During the past five or six years the movement for reform and improvement of mathematical work at the university level is going ahead with increasing momentum, and I have no doubt that during the coming few years great progress will be achieved.

Further a Pakistan Mathematical Society was formed last year, which has brought together a number of enthusiastic students, and we expect it will develop and grow in the coming years.

I must also mention here that last year the Government appointed a commission to examine the entire system of education in Pakistan and\pageoriginale make recommendations for bringing it up to the needs and ideals of the nation. The report has been recently published, and the whole system of education from the lowest to the highest stage will be overhauled immediately.

One important consequence of this is especially noteworthy, viz. the great importance attached to advanced study and research. To achieve this the universities will be freed from the control and responsibility for education upto the Intermediate stage, which will be handed over to boards of secondary education, which is already the case in some places but not in all, and the universities will devote all their resources and man power to higher learning and research. There are also in operation schemes for rapid expansion and provision of facilities for such work in all the universities.

In conclusion I wish to state on behalf of my colleagues and myself that we are thankful to the organizers of this conference and particularly to Professor Chandrasekharan for inviting us to participate in the deliberations of this very important conference, and thus affording us a valuable opportunity of meeting and sharing in the discussions of eminent mathematicians from different parts of the world. I am confident that the knowledge we have acquired about the phenomenal changes and development in mathematical thinking and education in the advanced countries of the world today, will be of immense benefit and assistance in shaping the course of mathematical education in our own country.

\bigskip
\medskip
{\fontsize{9pt}{11pt}\selectfont
Sind University

Hyderabad (Sind)
}\relax
