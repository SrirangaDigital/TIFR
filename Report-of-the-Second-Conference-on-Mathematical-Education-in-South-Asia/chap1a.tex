
\chapter{Second South Asian Conference on Mathematical Education} 

\begin{center}
\textbf{BOMBAY, 20-27 JANUARY 1960}

\medskip

\textbf{REPORT}
\end{center}

\begin{enumerate}
\item  A Conference on Mathematical Education in South Asia was held
  at the Tata Institute of Fundamental Research, Bombay, on 20-27
  January 1960. It was a sequel to the First South Asian Conference on
  Mathematical Education held at the Institute on 22-28 February
  1956. It was attended by over sixty mathematicians representing a
  dozen countries: Ceylon, Denmark, France, West Germany, India,
  Japan, Malaya, Pakistan, Singapore, Thailand, the United Kingdom,
  the United States of America, and the Union of Soviet Socialist
  Republics. The Governments of Ceylon, Pakistan, Malaya, Singapore,
  and Thailand sent in official delegations, Professor
  K. Chandrasekharan was the President of the Conference.

The Conference was jointly sponsored and financially supported by the
Tata Institute of Fundamental Research and the Sir Dorabji Tata
Trust. An Organizing Committee consisting of Professor
K. Chandrasekharan (Chairman), Professor T. A. A. Broadbent, Professor
C. Racine, Professor K. G. Ramanathan and Professor M. H. Stone was in
charge of the Conference programme. The purpose of the Conference was
to discuss the teaching of various branches of mathematics in the
light of modern developments, to provide a lead to teachers of
mathematics in South Asia, and to stimulate the vocation of teaching
among gifted researchers in the region.

\item The Conference programme consisted of (i) a series of
  \textit{invited addresses,} (ii) discussions in \textit{working
    groups}, and (iii) reports on the progress of mathematical
  education in various countries in South Asia. Invited addresses
  lasted forty minutes each, and were mimeographed and distributed to
  the members immediately after delivery. Working groups met in
  sessions of seventy five minutes each. Half-hour reports were
  presented on the progress of mathematical education in Ceylon,
  Indonesia, Malaya-Singapore, Pakistan and Thailand. The following
  addresses were given.

Professor E. Artin (Hamburg) : \textit{Contents and methods of an
  algebra course}

Professor M. H. Stone (Chicago) : \textit{A beginning course in
  analytic geometry and calculus}

Professor  T. A. A. Broadbent (London) : \textit{The place and
  teaching of co-ordinate geometry in the upper school}

Professor Y. Akizuki (Kyoto) : \textit{Introduction to determinants}

Professor W. Fenchel (Copenhagen) : \textit{A beginner's course in
  analytic geometry and linear algebra}

Professor Wolfgang Krull (Bonn) : \textit{Analytical and projective geometry}

Professor A. Lichnerowicz (Pairs) : The role of the exterior calculus
in geometry

Professor E. E. Moise (Ann Arbor) : \textit{Geometry at the high
  school level}

Professor M. H. A. Newman (Manchester) : \textit{Topology in the first
degree course}

Professor A. D. Alexandrov (Leningrad) : \textit{Mathematics in the
  humanities}

Professor B. V. Gnedenko\footnote{The address was not given since
  Professor Gnedenko did not attend the Conference} (Kiev) : \textit{The teaching of
  probability theory and mathematical statistics}

Professor C. J. Eliezer : \textit{Progress of mathematical education
  in Malaya and Singapore}

Ir. Soehakso\footnote{This address was read by Professor Stone in the
  absence of Ir. Soehakso.} : \textit{Progress of mathematical education
  in Indonesia}

Professor S. Tanbunyuen : \textit{Progress of mathematical education
  in Thailand}

Professor A. L. Shaikh : \textit{Progress of mathematical education in
Pakistan}

Principal D. G. Sugathadasa : \textit{Progress of mathematical
  education in Ceylon}

These addresses were followed by discussions, which continued in the
sessions of the working groups.

\item The Conference was inaugurated on 20 January 1960 at 11 a.m. by
  Dr. H. J. Bhabha, Secretary to the Government of India in the
  Department of Atomic Energy. Professor K. Chandrasekharan delivered
  the Presidential Address and Professor S. Tanbunyuen made a brief
  speech on behalf of the delegates. The Conference closed with a
  plenary session at 12 noon on Wednesday, 27 January 1960, at which
  the President expressed his thanks to all the participants for their
  co-operationo. At this session, the Conference passed a resolution
  expressiong the opinion that a seminar for teachers of mathematics
  in South Asia be organized sometime during the next four years.

\item Two meetings of the Committee for Mathematics in South Asia
  which had been set up in terms of a resolution passed by the First
  South Asian Conference (cf. Report of a Conference on Mathematical
  Education in South Asia, Tata Institute of Fundamental Research,
  Bombay, 1956, p. 172) wee held at the Institute on Sunday, 24
  January 1960 at 11 a.m. and on Wednesday, 27 January 1960 at 2.30
  p.m. The meetings were attended by the following members:  Professor
  K. Chandrasekharan (Chairman), Professor C. J. Eliezer
  (Malaya-Singapore), Professor M. H. Stone, Principal
  D. G. Sugathadasa (Ceylon), and Professor S. Tanbunyuen
  (Thailand). Professor Y. Akizuki (Japan) and Professor A. L. Shaikh
  (Pakistan) attended the meetings of the Committee by special
  invitation. Detailed discussions took place regarding the proposed
  seminar for teachers of mathematics in South Asia, and the
  preparation of text-books. The committee decided to maintain close
  relations with the International Commission on Mathematical
  Instruction, and to seek its advice whenever necessary. 

\item The social programme included a dinner at Juhu for the delegates
  on 19 January, a reception by the Vice-Chancellor of the University
  of Bombay on 20 January, a performance by the Little Ballet Troupe
  of the puppet dance drama `Ramayana' on 20 January, a performance of
  classical Indian dances, Bharata Natyana and Manipuri, on 22
  January, an excursion to Elephanta on 23 January and a banquet to
  the delegates on 27 January.
\end{enumerate}


