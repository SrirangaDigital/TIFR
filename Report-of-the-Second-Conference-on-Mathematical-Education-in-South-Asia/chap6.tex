
\chapter{A Beginners' Course in Analytic Geometry and Linear Algebra}

\begin{center}
{\em By~} W. FENCHEL
\end{center}

Considering\pageoriginale the tremendous number of text books on
analytic geometry and linear algebra published in many languages
during the last decades, it may seem hard to justify a discussion of
the content of beginner courses on these branches of
mathematics. However, the great diversity of these books reveals a
thorough disagreement of many of their authors in almost every
respect. To a certain extent this is due, of course, to the
differences in the knowledge presupposed and, thus, in the curriculae
of the secondary schools. Further, the aim depends on whether the
course is  intended for mathematicians only or also for physicists and
for engineers. There are, however, deeper reasons. Two tendencies
represented with strongly varying weights in the text books stand out
rather clearly, namely to make the students familiar on one hand with
the basic elementary \textit{results} (concerning lines, planes,
circles, spheres, conics, quadrics, determinants, matrices, linear
equations, quadratic forms), on the other hand with the general
\textit{notions} and \textit{methods} which have proved so fruitful in
many branches of modern mathematics and which nowadays are
indispensable even for physicists. This state of affairs reflects the
difficulties we are facing whenever we have to plan courses in
analytic geometry and in linear algebra. There are therefore good
reasons for a discussion of the matter at this conference.

It is not my intention in this lecture to advocate the only perfect
solution of the problem, simply because such a solution does not exist
for several obvious reasons. You will certainly not expect me, either,
to tell you anything mathematically new about these thoroughly studied
subjects. All I can do is to contribute to a discussion by giving an
account of what we normally are doing at the two Danish universities,
and, to a certain extent, at the Danish Institute of Technology.

The\pageoriginale problems in question are, of course, not of a recent
date. They turned up gradually several decades ago, but the
development of modern mathematics has made them more and more
prominent and difficult. A basic question is, roughly speaking, how
far we can go in the reduction of the ``concrete'' material in favour
of the ``abstract'' notions and methods. (For the sake of brevity I am
using the words ``concrete'' and ``abstract'' to characterize the two
principal requirements and tendencies, well aware of the fact that
this distinction does not make much sense in mathematics.) Another
important question is, what and how much should be presupposed, in
particular from elementary geometry. This depends, of course,
primarily on the curriculae of the secondary schools, but whatever
these are, neither a rigorous treatment nor a sufficiently broad
knowledge of three-dimensional elementary geometry can be
expected. Finally I mention the more practical question whether the
two subjects should be treated in two separate courses or in the same
course. It is easy to point out advantages as well as disadvantages
in both cases.

In several of the well-known text books we find excellent
\textit{extreme} answers to these questions. On one hand we have the
books dealing with only one of the subjects, starting at school level,
developing the methods and techniques gradually, and containing a
wealth of beautiful and interesting special facts. On the other hand
there are the axiomatic treatments of linear and affine spaces, in
principle requiring no prerequisites at all and deriving the more
``concrete'' facts as very special cases of general results. Now in
Denmark one has chosen an admittedly not very exciting compromise. It
has, however, the advantage of being somewhat flexible and can thus be
adjusted to varying conditions. This aspect is of particular
importance just now, because all of the curriculae in mathematics from
the primary schools to the universities are under lively discussion in
committees and otherwise. Within a few years a considerable change, in
particular in the schools, will have taken place.

An\pageoriginale essential feature of the course in question is,
roughly speaking, to present ``concrete'' material in such a manner
that the underlying ``abstract'' notions and methods become clear and
may be introduced if and when the teacher considers it
appropriate. Considering the great proportion of students who have
serious difficulties in handling the ``abstract'' notions, no
essential part of the course is based on these notions.

Several Danish text books have been written in this spirit. Here I
have to mention first of all a text book in geometry by B. Jessen, the
first edition of which was published almost 20 years ago.

2.~ Let me begin by saying a few words about the mathematical
knowledge we can expect our students to have when they enter the
university, normally at the age of 19. In a small country like Denmark
it is possible not only to have the same curriculum but also common
problems for the final examination at all secondary schools. There are
three ``lines'' at these schools, two language lines (classical and
modern) and a science line. Our university courses in the sciences are
based on the curriculum of the latter (though students from the
language lines are admitted ; but it is left to them to acquire the
knowledge presupposed in the courses). This curriculum comprises a
rigorous treatment  of the differential and integral calculus for
functions of one variable based on some continuity axiom, complex
numbers, a complete discussion of systems of linear equations with two
unknowns, plane analytic geometry (equations and parametric
representations of lines, circles and the various types of conics in
rectangular coordinates), and finally, in a less satisfactory manner,
the elements of solid geometry.

At the universities the students of mathematics and physics as well as
some of the students of chemistry attend the same courses in the three
sciences during their first two years and partly their third year of
study. [A considerable change will take place next year ; the common
  courses will be reduced to one year, and the mathematics courses in
  the second and third years will be intended only\pageoriginale for
  the students of mathematics and (theoretical) physics. For the
  present discussion these changes are however of minor importance.]
There are two beginners' courses, analysis, and geometry and linear
algebra, both extending over four terms with 6, 4, 3, 3 and 4, 4, 3, 3
hours per week, respectively. The latter comprises, besides analytic
geometry and linear algebra, also differential geometry and
kinematics. Here we are only concerned with the two first named
subjects. More than half of the total time, say  $4+4$ hours per week,
is normally  devoted to them.

3.~ The following headings, where some minor sections dealing with
more isolated topics like polar coordinates have been omitted, give a
rough idea of the content of the course :
\begin{itemize}
\item[{I.}] Vector algebra (including scalar and vector products) with
  various applications to analytic geometry.

\item[{II.}] Systems of linear equations, matrices, determinants.
 
\item[{III.}] Linear mappings (of the geometrical vector space and of
  number spaces), eigenvalues and eigenvectors.

\item[{IV.}] Quadratic forms, conics and quadrics.
\end{itemize}

As to I, in the definitions of the basic notions of \textit{vector
  algebra} free use is made of elementary solid geometry, the
euclidean three-space and its simple properties are taken for
granted. Vectors are defined in the usual way as equivalence classes
of ordered pairs of points. The sum of two vectors is likewise defined
in the usual way by means of representing pairs of points, and its
independence of the choice of the representatives is discussed. Then
the commutative and associative laws, the existence of a zero vector
and the existence of the opposite vector are pointed out. Thus, it is
shown that the vectors form and \textit{abelian group}. According to
the preference of the teacher, this notion, and other examples, may be
mentioned at this stage or postponed.

The next step consists in the definition of the product of a vector by
a scalar. Then using similar triangles, the defining relations of a
\textit{vector space} are proved :
$$
\bl \bv = \bv, \lambda (\mu \bv) =(\lambda \mu)\bv, \lambda (\bu +
\bv) = \lambda \bu + \lambda \bv, (\lambda + \mu) \bv = \lambda \bv +
\mu \bv .
$$\pageoriginale
Now the notions of a vector space consisting of ``geometrical''
vectors and its subspaces are introduced. Again, the teacher may, if
it seems appropriate to him at that stage, talk about abstract vector
spaces, mentioning examples of function spaces and spaces of
sequences.

Thereafter \textit{linear dependence} of ``geometrical'' vectors is
defined, discussed and interpreted geometrically. The dimension of a
vector space is defined as the maximal number of linearly independent
vectors. It is shown, using elementary geometry, that the four vectors
in the space are linearly dependent. This is used in the introduction
of coordinate systems (bases). Given any three linearly independent
vectors $\be_1$, $\be_2$, $\be_3$, every vector $\bx$ can be written
uniquely as a linear combination $x_1 \be_1 + x_2 \be_2 + x_3
\be_3$. The coefficients $x_1, x_2, x_3$ are called the coordinates of
$\bx$. Similarly coordinate systems are introduced in the one-and
two-dimensional subspaces. After an origin has been chosen in the
ordinary space (on a line, in a place), the coordinates of a point are
defined as those of its position vector. As an obvious application,
the linear parametric representations of lines and planes are derived,
and by elimination of the parameter(s) it is shown that lines and
planes can also be represented by linear equations.

From the systematic point of view one would like to proceed within
affine geometry. However, as a concession to the physicists, the two
products of vectors are  are introduced as early as possible. The
\textit{scalar product} is defined as the product of the lengths of
the two vectors and the cosine of the angle between, and again using
elementary geometry, it is shown that the scalar product $\bu.\bv$ is
a symmetric bilinear function of the vectors $\bu$ and $\bv$. The
obvious applications to the determination of distances and angles are
made, and, specializing to rectangular coordinate systems, the
equations of a line in the plane and of a plane in the Hesse normal
form are derived. 

The introduction of the \textit{vector product} is preceded by a
discussion of the signed area of a polygon in an oriented plane. The
product $\bu \times \bv$\pageoriginale is defined geometrically. The only
non-obvious property which has to be proved on the basis of elementary
geometry is the distributive law. By means of it the expressions for
the rectangular coordinates of the product vector are easily
obtained. After the introduction of the signed volume of an oriented
polyhedron in the oriented space, the mixed product $\bu.\bv \times
\bw$ is considered. Its interpretation as such a volume is given, and
it is shown that it is a trilinear  skewsymmetric function of the
vectors. A preliminary discussion of determinants of order 3 may
follow here.

Finally coordinate transformations are studied.

The topics named under II may be dealt with in the following
order. First vector spaces consisting of ordered $m$-uples of (real or
complex) numbers are defined. (Here again there is an opportunity to
emphasize the abstract point of view.) The fundamental fact to be
established is that any $m+1$ such vectors are linearly dependent,
that is, a system of $m$ homogeneous linear equations with $n$ unknowns
has a non-zero solution if $n>m$. This can easily be proved by
induction with respect to $m$ (elimination of one of the
unknowns). From this theorem the ``determinant free'' theory of
arbitrary \textit{systems of linear equations} is derived. The notions
of a \textit{matrix} and its rank, the maximal number of linearly
independent columns, are introduced and used right from the
beginning. Then the algebra of matrices is
studied. \textit{Determinants} are treated in the traditional way,
using the explicit expression in terms of the elements as the
definition. The set of non-singular square matrices of a given order
yields an example of a non-abelian \textit{group} and examples of
group theoretic arguments. Finally orthogonal matrices are defined and
studied.

As to III, \textit{linear mappings}, it is convenient to start with
the ``geometrical'' vector spaces. A linear ``vector function'' or
(mixed) \textit{tensor} (of order two) is by definition a function
$\bv' = f(\bv)$ from a subspace of the ordinary vector space into this
space such that $f(\bu + \bv) = f(\bu) + f(\bv)$ and $f(\lambda \bv) =
\lambda f (\bv)$. The well-known simple properties of such mappings
are derived. Now affine mappings of the space\pageoriginale into
itself (or of a plane into a plane, a line into a line) are defined as
follows : Choose two points $O$ and $O'$, and let a linear vector
function $\bv' = f(\bv)$ be given. The image of a point $P$ is that
point $P'$ for which $O'P' = f(\vec{OP})$. It is shown that the
mapping so defined remains unchanged if the pair of points $O,O'$ is
replaced by any other pair $Q,Q'$, where $Q'$ is the image of
$Q$. Properties of the affine mappings : lines are mapped into lines,
parallel lines into parallel lines, invariance of the ratio in which a
point divides a segment, all volumes of polyhedra are multiplied by
the same number etc., are easily deduced from the properties  of
$f$. The introduction of coordinate systems leads to the
representation of a linear vector function by a matrix equation $\bx'
=\bA \bx$, where $\bx'$ and $\bx$ stand for the column matrices of the
coordinates of the vectors in question. The effect of a coordinate
transformation on the matrix $\bA$ is derived, and thus, the
connection with the definition of a tensor by transformation
properties (a definition which, however, is only given in more
advanced courses mainly intended for physicists.)

The eigenvalue problem is, to begin with, also treated for
``geometrical'' linear mappings. In this way its intrinsic character
becomes clear. One asks for vector $\bv$ different from zero for which
$f(\bv) = \lambda \bv$ for some number $\lambda$. There are many
examples of mappings, familiar to the students, where the eigenvectors
can be found directly ; e.g. rotations, reflections, projections,
affine mappings leaving a line pointwise fixed. After the introduction
of coordinates and the generalization of the problem to number spaces
of arbitrary dimension, the characteristic polynomial is studied in
the usual way.

With regard to the reduction of quadratic forms ``symmetric linear
vector functions'' are considered. Of course, this notion makes  sense
only in euclidean spaces, i.e. if there is a scalar product. A purely
linear approach would be more satisfactory. However, the introduction
and study of (intrinsically defined) bilinear forms seems to add
considerably to the students' difficulties, and this is certainly even
more true of the dual space.

Throughout\pageoriginale this chapter there is ample opportunity to
talk about various important abstract notions like isomorphism,
homomorphism, kernel (nulspace) of a linear mapping, quotient space etc.

The topics named under IV, reduction of quadratic forms by orthogonal
transformations, conics, quadrics, are treated in a rather traditional
way. A conic is by definition the set of points whose coordinates
satisfy a quadratic equation. The invariance of this notion under
coordinate transformations and affine mappings is proved. The
reduction of the equation in rectangular coordinates by orthogonal
coordinate transformations leads to one of the types known to the
students from the school. A similar treatment of the quadrics is
preceded by an enumeration and a description of the various types
based on their equations in suitable rectangular coordinate systems.

4.~ As things are now, most of our students come to the university
with the impression or even conviction that mathematics has only to do
with numbers and geometrical configurations. The analysis course is
felt as a natural continuation of what has been done in the school,
while notions as mildly abstract as vector and matrix seem strange and
difficult to many students. Abstract vector spaces and, in particular,
groups present very serious difficulties to a larger proportion of the
beginners than we can afford to discourage. On the other hand the
gifted students profit enormously form an abstract set-up

According to the new curriculum for secondary schools, which is under
preparation, the elements of set theory and of vector algebra will be
taught and applied in the schools. It is our hope that this will
enable us in the university courses to shift the emphasis more to the
``abstract'' side. 

\bigskip
\bigskip

\noindent
{\fontsize{9pt}{11pt}\selectfont
University of Copenhagen}\relax
