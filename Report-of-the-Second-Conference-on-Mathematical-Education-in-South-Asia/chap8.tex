
\chapter{The Role of the Exterior Calculus in Geometry}

\begin{center}
{\em By~} ANDR\'E \ LICHNEROWICZ
\end{center}


In\pageoriginale the French universities first year's teaching, the
exterior calculus, with its algebraic side and its analytic side, now
plays explicitly an important part. This main instrument of the
present differential geometry enters also in mechanics and
mathematical physics and gives the full comprehension of the notion of
multiple integral.

In its algebraic part, the exterior calculus comes from the work of
Grassmann. Systematizing the notion of ``multilinear covariant'' under
the same of exterior differential, Elie Cartan has given to the set of
the exterior calculus its remarkable power. We may say that Elie
Cartan's work relies almost completely upon the use of the exterior
calculus. 

This calculus, which appears nowadays very early in our university
teaching, is developed of course, by means of various methods,
according to the teachers. For instance, the exterior algebra can be
constructed as the theory of an hypercomplex system or rely upon the
tensor algebra.

Some of our secondary school's teachers know that one of their tasks,
in the final school year, is to prepare psychologically its
elaboration. The notions of vector product, mixed product and
potential of a vector field, for instance, must be developed under
this point of view. 

In this address, my purpose is to point our briefly how the exterior
calculus can be constructed, how it throws light retrospectively, for
our students, upon the most elementary considerations, and conversely,
how is prepares the future for them. 
\begin{enumerate}
\item \textsc{Tensor Algebra.}
\begin{itemize}
\item[(a)] The notion of vector space on the
  real field being admitted, it is easy to define the \textit{tensor
    product} of two spaces.

It $L$\pageoriginale is an arbitrary set, we consider the set $M(L)$
of the mappings $f$ of $L$ into $R$ which are non-vanishing only for a
finite number of elements of $L$. The set $M(L)$ admits a natural
structure of a vector space over the reals. If $f_x$ is the mapping
which is different from zero only for the element $x$ of $L$ and which
is such that $f_x(x) =1$, the $f_x$'s define a base of $M$. If we
identify $f_x$ with $x$, $M(L)$ can be considered as the space of the
finite linear combinations of elements of $L$.

Let $E$ and $F$ be two vectors spaces ; $M(E \times F)$ is the space
of the finite linear combinations of couples $(x,y) ~ (x \in E, y \in
F)$. Let us consider the subspace $N$ of $M$ generated by the
following elements
\begin{align*}
(x+x',y) - (x,y) - (x', y), &\quad (x,y + y') - (x,y) - (x, y'),\\
(\lambda x, y) - \lambda (x,y), & \quad(x, \lambda y) - \lambda (x,y).
\end{align*}

The \textit{tensor product} $E \otimes F$ is the quotient $M/N$. The
class of $(x,y)$ modulo $N$ is the tensor product $x \otimes y$. The
mapping $(x,y) \to x \otimes y$ is \textit{bilinear} by construction.

If $f$ is a linear mapping of a vector space $E$ into another $E'$,
$g$ a linear mapping of $F$ into $F'$, there is one and only one
linear mapping of $F$ into $F'$, there is one and only one linear
mapping $h$ of $E\otimes F$ into $E' \otimes F'$ such that 
$$
h (x \otimes y) = f(x) \otimes g(y).
$$
$h$ is the \textit{tensor product} of $f$ by $g$.

\item[(b)] If $E$ and $F$ have \textit{finite dimensions,} the tensor
  products of elements of two bases respectively of $E$ and $F$ define
  a base of $E \otimes F$. Conversely the bilinearity and this
  property of the bases define the tensor product up to an
  isomorphism. 

These definitions can be extended to a finite number of vector spaces,
the tensor product being associative. 

\item[(c)] If $E$ is a vector of dimension $n$, a \textit{tensor} of
  order $p$ on $E$ is an element of one of the vector spaces
  constructed by a tensor product of $p$ vector spaces identical to
  $E$ or to the dual space $E^\ast$. 

If a\pageoriginale base of $E$ is chosen, we define our tensor by its
components with respect to the base constructed by the tensor product
from the base of $E$ and the dual base of $E^\ast$.

\item[(d)] The tensors $t$, elements of $E^{\ast(2)}$ for instance,
  can be identified with the linear mappings of $E^{(2)}$ into $R$ or
  with the bilinear forms $t(x,y)$ on $E$. More generally an element
  of $E^{(p)} \otimes E^{(q)}$ can be identified with a multilinear
  form of order $r \leqslant q$ on $E$ with values in $E^{(p)} \otimes
  E^{\ast(q-2)}$. The parts played by $E$ and $E^\ast$ can be
  interchanged. 

The tensors $t$, elements of $E \otimes E^\ast$, can be identified
with the endomorphisms $f_t$ of $E$. The scalar $Tr.f_t$ is the
\textit{contraction} of $t$; for any base, it is equal to the trace of
the matrix of the components. The contraction can be extended to
elements of $E^{(p)} \otimes E^{\ast(q)} (p, q \neq 0)$   according to
the previous interpretation in terms of multilinear forms, and gives
elements of $E^{(p-1)} \otimes E^{\ast(q-1)}$.

\item[(e)] A tensor $t$, element of $E^{\ast(2)}$, is
  \textit{symmetric} if the corresponding bilinear form is symmetric,
  that is to say if, for any couple $(x,y)$,
$$
t(x,y) = t (y,x).
$$
It is \textit{skewsymmetric} if 
$$
t(x,y) = - t (y, x).
$$
These definitions can be extended to an arbitrary tensor.
\end{itemize}

\item \textsc{Exterior Algebra.}
\begin{itemize}
\item[(a)] The exterior algebra is defined by the tensors, elements
  of $E^{(q)}$ or $E^{\ast(q)} (0 \leqslant q \leqslant n)$, which are
  completely \textit{skewsymmetric}. To the elements of $E^{\ast(q)}$
  we give the name of exterior forms of grade $q$ or $q$-forms : the
  0-\textit{forms} are the scalars and the 1-forms are the linear
  forms. The $q$-forms generate a subspace $\Lambda^{\ast(q)}$ of
  $E^{\ast(q)}$ of dimension $C^q_n$. To any element of $E^{\ast(q)}$
  corresponds ``by skewsymmetrisation'' an element of
  $\Lambda^{\ast(q)}$.

It is possible to define on the forms a new operation, the
\textit{exterior product}, by means of the skewsymmetrisation of the
tensor product. If $\alpha$ and $\beta$ are respectively a $q$-form
and a $q'$-form, the exterior product $\alpha \Lambda \beta$ defines a
$(q+q')$-form and has the following properties : it is
\textit{bilinear, associative}, and satisfies the rule of commutation :
$$
\alpha \Lambda \beta = (-1)^{qq'} \beta \Lambda \alpha.
$$\pageoriginale
Thus, it is possible to define on the set of the forms a structure of
a graduated ring.

The same considerations are available for the skewsymmetric  tensor
elements of $E^{(q)}$ which generate a subspace $\Lambda^{(q)}$. By a
canonical isomorphism $\Lambda^{\ast(q)}$ can be identified with the
dual space of $\Lambda^{(q)}$.

\item[(b)] The exterior product of $q$ vectors of $E$ has the
  following property : the vectors are linearly independent if and
  only if their exterior product is different from zero. To any base
  of $E$ correspond, by exterior product of $q$ vectors, $C^q_n$
  distinct elements which define a base of $\Lambda^q$. As element of
  $\Lambda^q$, a skewsymmetric tensor $t$ is defined by $C^q_n$
  convenient components, the \textit{strict components} of $t$. In
  particular an element of $\Lambda^{(n)}$ is defined by \textit{one}
  strict component. 

If $u_{(i)} (i = 1, 2, \ldots, n)$ are $n$ vectors of $R^n$, the
strict component of the exterior product of these $n$ vectors defines
the \textit{determinant} of the matrix of the components of the
vectors and it is easy to obtain from this definition the main
properties of the determinants.

If $\tau$ and $\tau'$ are the strict components of an $n$-form with
respect to two bases and if $A$ is the matrix of the change of base
\begin{equation*}
\tau' =\Delta \tau \quad (\text{with } \Delta = \det A). \tag{1}
\end{equation*}
Conversely if to any base is associated a number $\tau$, these numbers
are the strict components of a fixed $n$-form if (1) is satisfied for
any change of base.
 
\item[(c)] The set of the bases of a vector space $E$ can be divided
  into two classes such that the determinant $\Delta$ of the
  corresponding matrix is positive for two bases of the same class,
  negative in the converse case.

An \textit{orientation} of $E$ is defined by the association to any
base of a number $\epsilon$ which is $+1$ for the bases of a class, $-1$
for the bases of the other class. If $\epsilon$, $\epsilon'$ correspond to two
bases,
\begin{equation*}
\epsilon' = \frac{\Delta}{|\Delta|} \epsilon. \tag{2}
\end{equation*}
\end{itemize}

\item \textsc{Euclidean vector space.}\pageoriginale 
\begin{itemize}
\item[(a)] An euclidean space is a vector space $E$ which admits a
  scalar product, that is to say a non-degenerate symmetric bilinear
  form, the metric form. This scalar product defines a canonical
  isomorphism of $E^\ast$ onto $E$ ; thus, by tensor product, any
  tensor of order $p$ can be identified with an element of $E^{(p)}$,
  and $E^{(p)}$ admits a natural structure of euclidean space.

\item[(b)] We know that, in elementary geometry, the discriminant $g$ of
  the metric form is equal to the square of the volume of the
  parallelepiped constructed on the three vectors of the base. It
  leads us to study the transformation of $g$ in a change of base. We
  have thus
$$
\surd |g'| = |\Delta| \surd |g|.
$$
If we choose an orientation of $E$ :
$$
\epsilon' \surd |g'| = \Delta \epsilon \surd |g|,
$$
and the $\epsilon \surd |g|$ are the strict components of an $n$-form
$\eta$, the volume element of the oriented euclidean space. 

\item[(c)] If $\alpha$ is a skewsymmetric tensor of order $q (0 \leqslant q
  \leqslant n)$, we can associate to $\alpha$ a skewsymmetric tensor
  $\ast\alpha$ of order $(n-q)$ by tensor product and total
  contraction with the volume element ; $\ast \alpha$ is the adjoint
  tensor of $\alpha$. It is easy to prove that :
$$
\ast\ast \alpha = \frac{g}{|g|} (-1)^{q(n-1)} \alpha .
$$
Thus
$$
\ast^{-1} = \frac{g}{|g|} (-1)^{q(n-q)}\ast.
$$
In the space $E_3$ supposed oriented and corresponding to the
elementary geometry, the \textit{vector product} of two vectors is
exactly the \textit{adjoint vector} of the exterior product of the two
vectors. We see how the existence of the vector product is connected
with the dimension 3.

In the space-times of special relativity which is an euclidean space
of dimension 4, an electromagnetic field is defined by a skewsymmetric
tensor $F$ of order two. The adjoint tensor, which is also of
order\pageoriginale tow, $\ast F$, plays too, an important part. The
Lorentz invariant scalars of the electromagnetic field $\vec{E^2} -
\vec{H^2}$ and $2 \vec{E}. \vec{H}$ are defined by the scalar products
of $F$ by itself and by $\ast F$.
\end{itemize}

\item \textsc{Exterior Differentiation.} Let $\mathscr{E}_n$ be, for
  instance, an affine space of dimension $n$. Let us consider, on a
  domain of $\mathscr{E}_n$, an exterior differential form $\alpha$ of
  order $q$, that is to say a differentiable $q$-form field. It is
  possible to define on these differential forms an operation which
  generalizes the differentiation of a function, that is to say of a
  0-form. It is sufficient to assume that to any $q$-form $\alpha$
  corresponds a $(q+1)$-form $d\alpha$, the differentiation $d$ having
  the following properties:
\begin{itemize}
\item[(1)] $d (\alpha + \beta) = d\alpha + d \beta$,
\item[(2)] $d(\alpha \wedge \beta) = d \alpha \wedge \beta + (-1)^{g
  r\alpha} \alpha \wedge d \beta$,
\item[(3)] $d^2 = 0$.
\end{itemize}
There exists one and only one operation with these properties. This
operation is the \textit{exterior differentiation}. These
considerations are available for a differentiable manifold : the
exterior product and the exterior differentiation are permutable with
the operation of inverse image of a differentiable mapping.

For a 0-form $\phi$ we obtain the \textit{gradient} of $\phi$ ; for an
1-form we obtain the skewsymmetric tensor of order two which is the
\textit{curl tensor}. If $\mathscr{E}_n$ admits a metric structure and
if $n =3$, it is possible to introduce its adjoint vector which is the
curl vector. With the same assumption, but for an arbitrary $n$, the
divergence of an 1-form $\alpha$ is defined by :
$$
\Div \alpha = \ast^{-1} d \ast \alpha .
$$
We obtain thus with the exterior differentiation the synthesis of the
different ordinary differential operators. 

If there is, for a given form $\alpha$, a form $\beta$ such that
$\alpha = d \beta$, it is necessary that $d \alpha =0$ and this
condition is locally sufficient. The conditions relative to the
existence of a potential or of a potential vector come from this result.

Curvilinear\pageoriginale integrals, surface integrals and so on, must
be considered as integrals of forms, and the origin of the formula of
the change of variables appears clearly. The classical formulas of
Riemann, Green, Stokes... can be gathered in the unique formula
(general formula of Stokes)
$$
\int\limits_{\partial C} \alpha = \int\limits_{C} ~~ d \alpha, 
$$
where the integration field $C$ is oriented and where $\partial C$ is
the boundary of $C$.

\textsc{Conclusion.} We have seen that elementary and important
notions such as the vector product, the curl vector, the operators
grad, div, find their true light in the exterior calculus. This
calculus is conceptually simple and experience proves that the
students are able very early to grasp this instrument and to use it on
many occasions : classical and modern differential geometry, calculus
of variations, physics, elements of the theory of Lie groups or of
topology and so on.

Mathematical education is a whole which is defined by the set of its
finalities, that is to say by the set of its long-term
goals. Therefore, I think that a systematic introduction of the
exterior calculus must be studied, step by step, and that pedagogic
experiences on the different methods of exposition can perhaps begin
in all countries.
\end{enumerate}


\bigskip
\bigskip

{\fontsize{9pt}{11pt}\selectfont
College de France 

Paris }\relax
