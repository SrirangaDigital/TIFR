\title{MUMFORD'S CONJECTURE FOR $GL(2)$ AND APPLICATIONS}
\markright{Mumford's Conjecture for $GL(2)$ and Applications}

\author{By~~ C. S. Seshadri}
\date{}

\maketitle


\setcounter{pageoriginal}{346}
\textsc{In}\pageoriginale \cite{art18-key12}, it was shown that on a smooth projective curve $X$ of genus $\geq 2$ over $\bfC$, there is a natural structure of a normal projective variety on the isomorphic classes of unitary vector bundles of a fixed rank. This can also be given a purely algebraic formulation, namely that on the classes of semi-stable vector bundles of a fixed rank and degree zero, under a certain equivalence relation, there is a natural structure of a normal projective variety when $X$ is defined over $\bfC$. In fact this was used in \cite{art18-key12}. It is then natural to ask whether this algebraic result holds good in arbitrary characteristic. The main obstacles to extending the proof of \cite{art18-key12} to arbitrary characteristic are as follows:
\begin{enumerate}
\renewcommand{\labelenumi}{(\theenumi)}
\item to carry over the results of Mumford (obtained in characteristic 0) on quotient spaces of the $N$-fold product of Grassmannians for the canonical diagonal action of the full linear group (c.f. \S4, Chap. 4, \cite{art18-key5}), to arbitrary characteristic, and

\item to find a substitute for unitary representations which have been used in \cite{art18-key12}, mainly to show that the varieties in question are complete.
\end{enumerate}

It is not hard to see how to set about (2). One has to show that a certain morphism is proper (see \S\ref{art18-sec3}, Lemma \ref{art18-lem2}). This is not difficult but requires some careful analysis and it is an improvement upon some of the arguments in \cite{art18-key12}. The difficulty (1) appears to be more basic. If Mumford's conjecture generalizing complete reducibility to reductive groups in arbitrary characteristic (cf. \S1, Def. 3) is solved for all special linear groups, (1) would follow. In this we have partial success, namely we solve Mumford's conjecture for $GL(2)$, which allows us to solve (1) for the case of a product of Grassmannians of two planes. Consequently the results of \cite{art18-key12} carry over to the case of vector bundles of {\em rank} 2 in arbitrary characteristic.

The\pageoriginale proof of Mumford's conjecture for $GL(2)$ is rather elementary and we give it in \S\ref{art18-sec1}. As for applications to vector bundles, only the solution of (2) above is given in detail (\S\ref{art18-sec3}, Lemma \ref{art18-lem2}, (3)). The other points are only sketched and proofs for most of these can be found in \cite{art18-key5} or \cite{art18-key12}.

The algebraic schemes that we consider are supposed to be defined over an algebraically closed field $\bfK$ and of finite type over $\bfK$. The points of an algebraic scheme are the geometric points in $\bfK$ and the algebraic groups considered are {\em reduced} algebraic group schemes. By a rational representation of an algebraic group $G$ in a finite dimensional vector space. $V$, we mean a homomorphism $\rho:G\to \Aut V$ of algebraic groups.

\section{Mumford's conjecture for $GL(2)$}\label{art18-sec1}

\begin{definition}\label{art18-defi1}
An algebraic group $G$ is said to be reductive if it is affine and $\rad G$ (radical of $G$) is a torus, i.e. a product of multiplication groups.
\end{definition}

\begin{definition}\label{art18-defi2}
An algebraic group $G$ is said to be linearly reductive if it is affine and every rational representation of $G$ in a finite dimensional vector space is completely reducible.
\end{definition}

It is a classical result of H. Weyl that if the characteristic of the base field is zero, every reductive group is linearly reductive. A torus group is easily seen to be linearly reductive in arbitrary characteristic. If the characteristic $p$ of the base field is {\em not zero}, there are not many more linearly reductive groups other than the torus groups; in fact, there is the following result due to Nagata: an algebraic group $G$ is linearly reductive if and only if the connected component $G^{0}$ of $G$ through identity is a torus and the order of the finite group $G/G^{0}$ is prime to $p$ (c.f. \cite{art18-key6}).

It is proved easily that an affine algebraic group $G$ is linearly reductive if and only if any one of the two following properties holds :
\begin{enumerate}
\renewcommand{\labelenumi}{(\theenumi)}
\item for every rational representation of $G$ in a finite dimensional vector space $V$ and a one dimensional $G$-invariant linear subspace $V_{0}$ of\pageoriginale $V$, there exists a $G$-invariant linear subspace $V_{1}$ of $V$ such that $V=V_{0}\oplus V_{1}$;

\item for every rational representation of $G$ in a finite dimensional vector space $V$ and a $G$-invariant point $v\in V$, $v\neq 0$, there exists a $G$-invariant linear form $f$ on $V$ such that $f(v)\neq 0$.
\end{enumerate}

\begin{definition}\label{art18-defi3}
An algebraic group $G$ is said to be geometrically reductive if it is affine and for every rational representation of $G$ in a finite dimensional vector space $V$ and a $G$-invariant point $v\in V$, $v\neq 0$, there exists a $G$-invariant polynomial $f$ on $V$ such that $f(v)=1$ and $f(0)=0$ or, equivalently, there is a $G$-invariant homogeneous form $f$ on $V$ such that $f(v)=1$.
\end{definition}

Let $G$ be a geometrically reductive algebraic group acting on an affine algebraic scheme $X$ (we can even take $X$ to be an arbitrary affine scheme over the base field $\bfK$, i.e. not necessarily of finite type over $\bfK$) and $X_{1}$, $X_{2}$ two $G$-invariant closed subsets of $X$ such that $X_{1}\cap X_{2}$ is empty. Then there exists a $G$-invariant $f\in A(X=\Spec A)$ such that $f(X_{1})=0$ and $f(X_{2})=1$. This is proved easily as follows : there exists an element $g\in A$ (not necessarily $G$-invariant) such that $g(X_{1})=0$ and $g(X_{2})=1$. Now the translates of $g$ by elements of $G$ span a {\em finite}-dimensional $G$-invariant linear subspace $W$ of $A$. For every $h\in W$, $h(X_{1})=0$ and $h(X_{2})$ is a constant. We have a canonical rational representation of $G$ on $W$ and therefore also on the dual $W^{*}$ of $W$. The canonical inclusion $W\subset A$ defines a $G$-morphism $\phi:X\to W^{*}$ of $X$ into the affine scheme $W^{*}$ (to be strict the affine scheme whose set of geometric points is $W^{*}$) and we have $\phi (X_{1})=0$ and $\phi(X_{2})=w$, $w\neq 0$. Now by the geometric reductivity of $G$, there exists $h$ in the coordinate ring of $W^{*}$ such that $h(0)=0$ and $h(w)=1$. Now if $f$ is the image of $h$ in $A$ by the canonical homomorphism of the coordinate ring of $W^{*}$ in $A$, then $f$ has the required properties.

The following statements are proved easily.
\begin{enumerate}
\renewcommand{\labelenumi}{(\theenumi)}
\item $G$ is geometrically reductive if and only if for every rational representation of $G$ in a finite-dimensional vector space $V$ and a {\em semi-invariant} point $v\in V$, $v\neq 0$ (i.e. the one-dimensional linear subspace\pageoriginale of $V$ spanned by $v$ is $G$-invariant), there is a semi-invariant homogenous form $f$ on $V$ such that $f(v)=1$.

\item $G$ is geometrically reductive if and only if for every rational representation of $G$ in a finite-dimensional vector space $V$, a $G$-invariant linear subspace $V_{0}$ of $V$ of codimension one and $X_{0}$ an element of $V$ such that $X_{0}$ and $V_{0}$ span $V$ and $X_{0}$ is $G$-invariant modulo $V_{0}$, there exists a $G$-invariant $F\in S_{m}(V)$ ($m^{\text{th}}$ symmetric power) for some $m\geq 1$, such that $F$ is monic in $X_{0}$ when $F$ is written with respect to a basis $X_{0}$, $X_{1},\ldots, X_{n}\in V$, $X_{i}\in V_{0}$, $i\geq 1$.

\item Let $N$ be a normal algebraic subgroup of an affine algebraic group $G$ such that $N$ and $G/N$ are geometrically reductive. Then $G$ is geometrically reductive. In particular, a finite product of geometrically reductive groups is geometrically reductive.

\item Let $G$ be a reductive group. Then $G$ is geometrically reductive if and only if $G/\rad G$ is so.

\item A linearly reductive group is geometrically reductive. A finite group is geometrically reductive.
\end{enumerate}

The conjecture of Mumford states that a reductive group is geometrically reductive (c.f. Preface, \cite{art18-key3}). On the other hand it can be shown that a geometrically reductive group is necessarily reductive (c.f. \cite{art18-key8}).

\begin{theorem}\label{art18-thm1}
The full linear group $GL(2)$ of $2\times 2$ matrices is geometrically reductive.
\end{theorem}

\begin{proof}
Let $G$ be an affine algebraic group and $\rho$, $\rho'$ rational representations of $G$ in finite-dimensional vector spaces $W$, $W'$ respectively. Let $\phi:W\to W'$ be a homomorphism of $G$-modules and $w$, $w'$ semi-invariant points of $W$, $W'$ respectively such that $w'=\phi(w)$, $w'\neq 0$. Now if there is a semi-invariant polynomial $f$ on $W'$ such that $f(w')=1$ and $f(0)=0$, then there is a semi-invariant polynomial $g$ on $W$ such that $g(w)=1$ and $g(0)=0$; in fact we can take $g$ to be the image of $f$ under the canonical homomorphism induced by $\phi$ of the coordinate ring of $W'$ into that of $W$. Using this\pageoriginale simple remark, the proof of the geometric reductivity of $GL(2)$ can be divided into the following steps.
\begin{enumerate}
\renewcommand{\labelenumi}{(\theenumi)}
\item It is a well-known fact (c.f. \S1, expos\'e 4, Prop. 4, \cite{art18-key2}) that if $G$ is an affine algebraic group and $\rho$ a rational representation of $G$ in a finite dimensional vector space $W$, then the $G$-module $W$ can be imbedded as a submodule of $A^{n}$ ($n$-fold direct sum of $A$), where $A$ is a submodule of the coordinate ring of $G$, considered as a $G$-module for the regular representation (we should fix the right or the left regular representation). Thus to prove geometric reductivity of $G$, we have only to consider submodules $A$ of the coordinate ring of $G$ such that there exists a semi-invariant $a\in A$, $a\neq 0$.

\item Let $G=GL(n)$, $R$ the coordinate ring of $G$ and $(X_{ij})$, $1\leq i\leq n$, $1\leq j\leq n$, the canonical coordinate functions on $G$. The linear space generated by $X_{ij}$ is a $G$-module and we can identify it with the $G$-module $V^{n}=V\oplus\cdots\oplus V$ ($n$ times), where $V$ is an $n$-dimensional vector space and $G$ is represented as $\Aut V$. Let $\xi$ be the function $\det |X_{ij}|$ and $L$ the 1-dimensional $G$-submodule of $R$ spanned by $\xi$. Now if $W$ is a finite-dimensional linear subspace of $R$, there exists an integer $m\geq 1$ such that for any $g\in W$, $g\xi^{m}$ is a polynomial in $(X_{ij})$. A polynomial in $(X_{ij})$ can be uniquely expressed as a sum of multihomogenous forms in the sets of variables
\begin{align*}
& Y_{1}=(X_{11},X_{21},\ldots,X_{n1}), \ Y_{2}=(X_{12},X_{22},\ldots,X_{n2}),\ldots\\
& Y_{n}=(X_{1n},X_{2n},\ldots,X_{nn})(Y_{i}-i^{\text{th}}\text{~column of~}(X_{ij})).
\end{align*}
The space of multihomogenous forms in $(X_{ij})$ of degree $m_{i}$ in $Y_{i}$ can be identified with the $G$-module $W(m_{1},\ldots,m_{n})$, where
$$
W(m_{1},\ldots,m_{n})=\bigoplus\limits^{n}_{i=1}S^{m_{i}}(V)(S^{m_{i}}(V)-m^{\text{th}}_{i}\text{~symmetric power of } V).
$$
Thus if $W$ is a finite dimensional $G$-invariant linear subspace of $R$, $W\otimes L^{(m)}$ can be embedded as a $G$-submodule of a finite direct sum of $G$-modules of the type $W(m_{1},\ldots,m_{n})$, where $L^{(m)}$ denotes the $1$-dimensional $G$-module $L\otimes\cdots\otimes L$ ($m$ times). Thus to prove the geometric reductivity of $GL(n)$, it suffices to consider the $G$-modules\pageoriginale of the form $W(m_{1},\ldots,m_{n})$ such that there is a non-zero semi-invariant element in it.
\end{enumerate}

Now it is easy to see that $W(m_{1},\ldots,m_{n})$ has a non-zero semi-invariant element $v$ if and only if $m_{1}=m_{2}=\ldots=m_{n}=m$ and then that $v$ is in the $1$-dimensional linear subspace spanned by $\xi^{m}$ $(\xi=\det |X_{ij}|)$. This is an immediate consequence of the following remarks:
\begin{itemize}
\item[(i)] every $1$-dimensional $G$-module (given by a rational representation) is isomorphic to $L^{(n)}$ for some $n\in \bfZ$ and

\item[(ii)] the only $G$-invariant elements of $R$ are the scalars.
 
Thus to prove the geometric reductivity of $GL(n)$, we have only to consider the $G$-modules $W(m)$,
$$
W(m)=W(m,\ldots,m)=\otimes S^{m}(V)(n\text{-fold tensor product of } S^{m}(V))
$$
with the semi-invariant element being $\xi^{m}$, $\xi=\det |X_{ij}|$.

\item[(iii)] Let $G=GL(2)$. Let $J:W(m)\to S^{2m}(V)$ be the canonical homomorphism, where for an element $f$ in $W(m)$ being considered as a multi-homogeneous polynomial of degree $m$ in $Y_{1}=(X_{11},X_{21})$, $Y_{2}=(X_{12},X_{22})$, $j(f)$ is the homogeneous polynomial of degree $2m$ in two variables obtained by setting $Y_{1}=Y_{2}$. Now $j$ is a $G$-homomorphism. Let $\theta_{m-1}:W(m-1)\to W(m)$ be the homomorphism defined by $\theta_{m-1}(f)=f\xi$, $f\in W(m-1)$. Now $\theta_{m-1}$ is a homomorphism of the underlying $SL(2)$ modules and it ``differs'' from a $GL(2)$ homomorphism only upto a character of $GL(2)$. Consider the following sequence
\begin{equation*}
0\to W(m-1)\xrightarrow{\theta_{m-1}}W(m)\xrightarrow{j}S^{2m}(V)\to 0.\tag{*}
\end{equation*}
We claim that this sequence is exact. It is clear that $\theta_{m-1}$ is injective. Further the kernel of $j$ consists precisely of those polynomials $f$ in $(X_{ij})$ which belong to $W(m)$ and such that $f$ vanishes when we set $(X_{ij})$ to be a singular matrix. Therefore $f=g\xi$, which means that $\ker j=\theta_{m-1}W(m-1)$. Now $\dim W(m)=(m+1)^{2}$, $\dim W(m-1)=m^{2}$ and $\dim S^{2m}(V)=(2m+1)$, so that $\dim W(m)=\dim W(m-1)+\dim S^{2m}(V)$. From this one concludes that (*) is exact.

We\pageoriginale shall now show that the exact sequence (*) has a ``quasi-splitting'', i.e. there is a closed $G$-invariant subvariety of $W(m)$ such that the canonical morphism of this subvariety into $S^{2m}(V)$ is surjective and {\em quasi-finite} i.e. every fibre under this morphism consists only of a finite number of points. Let $D_{m}$ be the subset of $W(m)$ consisting of decomposable tensors, i.e. $D_{m}=\{f|f=g\otimes h, g,h\in S^{m}(V)\}$. Then $D_{m}$ is obviously a $G$-invariant subset of $W(m)$. We have a canonical morphism
$$
\Psi : S^{m}(V)\times S^{m}(V)\to S^{m}(V)\otimes S^{m}(V)=W(m)
$$
and $D_{m}=\Psi(S^{m}(V)\times S^{m}(V))$. From the fact that $\Psi$ is bilinear, we see that $D_{m}$ is the cone over the image of $\Psi'$, where $\Psi'$ is the canonical morphism
$$
\Psi':\bfP(S^{m}(V))\times \bfP(S^{m}(V))\to \bfP(W(m))
$$
induced by $\Psi$, $\bfP$ indicating the associated projective spaces. It follows now that $D_{m}$ is a {\em closed} $G$-invariant subvariety of $W(m)$. The morphism $j_{1}:D_{m}\to S^{2m}(V)$ induced by $j$ is surjective, because every homogeneous form in two variables over an algebraically closed field can be written as a product of linear forms, in particular as a product of two homogeneous forms of degree $m$. We see also easily that $j_{1}:D_{m}\to S^{2m}(V)$ is quasi-finite (it can also be shown without much difficulty that $j_{1}$ is proper so that $j_{1}$ is indeed a {\em finite} morphism but we do not make use of it in the sequel). An element $f\otimes g\in D_{m}$ becomes zero when we set $Y_{1}=Y_{2}$ if and only if $f$ and $g$ are zero, i.e. we have $D_{m}\cap \theta_{m-1}(W(m-1))=(0)$.

\item[(iv)] Let $G=GL(2)$. We shall now show by induction on $m$, that there exists a closed $G$-invariant subvariety $H_{m}$ of $W(m)$ passing through $0$ and not through $\xi^{m}$. This will imply that $GL(2)$ is geometrically reductive.
\end{itemize}

For $m=0$, the assertion is trivial. Let $H_{m-1}$ be a homogeneous $G$-invariant hypersurface of $W(m-1)$ not passing through $\xi_{m-1}$. Let $H_{m}$ be the join of $\theta_{m-1}(H_{m-1})$ and $D_{m}$, i.e.
$$
H_{m}=\{\lambda+\mu|\lambda\in \theta_{m-1}(H_{m-1}), \mu\in D_{m}\}.
$$
We\pageoriginale shall now show that $H_{m}$ is a homogeneous $G$-invariant hypersurface of $W(m)$ not passing through $\xi^{m}$. It is immediate that $\xi^{m}$ is not in $H_{m}$ for if $\xi^{m}=\lambda+\mu$, $\lambda$ in $\theta_{m-1}(H_{m-1})$, $\mu\in D_{m}$, then by setting $Y_{1}=Y_{2}$ since $\xi^{m}$ and $\lambda$ become zero, we conclude that $\mu$ becomes zero. As remarked before, this implies that $\mu$ itself is zero so that $\xi^{m}\in \theta_{m-1}(H_{m-1})$. It would then follow that $\xi^{m-1}\in H_{m-1}$, which leads to a contradiction so that we conclude that $\xi^{m}$ is not in $H_{m}$. The subset $H_{m}$ is $G$-invariant and also invariant under homothecy. Thus to complete the proof of our assertion it suffices to show that $H_{m}$ is closed and of codimension one in $W(m)$. This is an immediate consequence of the following lemma, since $H_{m}$ is the join of the two homogeneous subvarieties $\theta_{m-1}(W(m-1))$ and $D_{m}$ whose common intersection is (0).
\end{proof}

\begin{lemma}\label{art18-lem1}
Let $Q_{1}$, $Q_{2}$ be closed subvarieties of a projective space $\bfP$ such that $Q_{1}\cap Q_{2}$ is empty. Then the join $Q$ of $Q_{1}$ and $Q_{2}$ is a closed subvariety of $\bfP$ and $\dim Q=\dim Q_{1}+\dim Q_{2}+1$.
\end{lemma}

\noindent
{\bf Proof of Lemma.} Let $\Delta$ be the diagonal in $\bfP\times \bfP$ and $R=(\bfP\times \bfP-\Delta)$. If $r=(p_{1}p_{2})$, $p_{i}\in \bfP$, let $L(r)$ be the line in $\bfP$ joining $p_{1}$ and $p_{2}$. Then the mapping $r\to L(r)$ defines a correspondence between $R$ and $\bfP$, and it is seen easily that this is defined by a {\em closed} subvariety of $R\times \bfP$. Since $Q_{1}\cap Q_{2}$ is empty, we have $Q_{1}\times Q_{2}\subset R$. Let $\Gamma_{1}=\pr^{-1}_{1}(Q_{1}\times Q_{2})$, $\pr_{1}$ being the canonical projection of $R\times \bfP$ onto the first factor. Now the join $Q=\pr_{2}(\Gamma_{1})$, $\pr_{2}$ being the projection of $R\times \bfP$ onto the second factor. Since $Q_{1}\times Q_{2}$ is complete, it follows that $Q$ is a closed subvariety of $\bfP$.

We see that $\dim Q_{1}+\dim Q_{2}\leq \dim Q\leq \dim Q_{1}+\dim Q_{2}+1$. Therefore to show that $\dim Q=Q\dim Q_{1}+\dim Q_{2}+1$, it suffices to show that $\dim Q\geq \dim Q_{1}+\dim Q_{2}+1$. Since $Q_{1}\cap Q_{2}$ is empty, we cannot have $\dim \bfP=\dim Q_{1}+\dim Q_{2}$. If $\dim \bfP=\dim Q_{1}+\dim Q_{2}+1$, we see that the lemma is true in this case. Suppose then that $\dim \bfP>\dim Q_{1}+\dim Q_{2}+1$. Then there is a point $p\in \bfP$ which is not in the join $Q$ of $Q_{1}$ and $Q_{2}$. Let us now project $Q_{1}$ $Q_{2}$ and $Q$ from $p$ in a hyperplane $H$ not passing through $p$. Let $Q'_{1}$, $Q'_{2}$ and $Q'$ be the images of $Q_{1}$, $Q_{2}$ and $Q$ respectively in $H$. Then $Q'_{1}$, $Q'_{2}$\pageoriginale %page 355
