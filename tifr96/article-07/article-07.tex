\title{A DEFORMATION THEORY FOR SINGULAR HYPERSURFACES}
\markright{A Deformation Theory for Singular Hypersurfaces}

\author{By~~ B. Dwork}

\date{}

\maketitle

\setcounter{pageoriginal}{84}
\textsc{In}\pageoriginale previous articles \cite{art07-key1}, \cite{art07-key2}
 we have given a theory for the zeta function of non-singular hypersurfaces defined over a finite field. Here we shall discuss the zeta function of the complement, $\overline{V'}$, in projective $n$-space of the algebraic set
$$
X_{1}X_{2}\ldots X_{n+1}\overline{f}(X)=0
$$
defined over $GF[q]$, where $\overline{f}$ is homogeneous of degree $d$. Let $f$ be a lifting of $\overline{f}$. We shall make no hypothesis that $f$ be non-singular in general position and shall study the variation of the zeta function of $\overline{V}$ as $f$ varies. This involves a generalization of the non-singular case and we will review the situation for that case.

\section{Notaton}\label{art07-sec1}

In the following the field of coefficients will be a suitably chosen field of characteristic zero. The precise choice of field will usually be clear from the context.

\smallskip

$\underline{L}^{*}$ = all infinite sums of the form $\Sigma_{dw_{0}=w_{1}+\cdots+w_{n+1}}A_{w}X^{-w}$, $w_{i}\geq 0$, $\forall i$;

\smallskip

$\underline{L}$ = all finite sums of the form $\Sigma A_{w}X^{w}$, the range of $w$ being as above;

\smallskip

$\underline{K}=\{\xi^{*}\in L^{*}|D^{*}_{i}\xi^{*}=0,i=1,2,\ldots,n+1\}$;

\smallskip

$D^{*}_{i}=\gamma_{-}\circ (E_{i}+{}_{\pi}X_{0}E_{i}f)$, $D_{i}=E_{i}+{}_{\pi}X_{0}E_{i}f$;

\smallskip

$E_{i}=X_{i}\dfrac{\partial}{\partial X_{i}}$;

\smallskip

$\pi^{p-1}=-p$;

\smallskip

$\gamma_{-}X^{w}=\begin{cases} X^{w} & \text{if each } w_{i}\leq 0,\\
0 & \text{otherwise;}\end{cases}$

\smallskip

$\Phi X^{w}=X^{qw}$;

\smallskip

$\alpha^{*}=\gamma_{-}\circ \exp \{\pi X_{0}f(X)={}_{\pi}X^{q}_{0}f(X^{q})\}\circ \phi$\pageoriginale

\smallskip

$\underline{K}^{\infty}=\{\xi^{*}\in \underline{L}^{*}|D^{*v}_{i}\xi^{*}=0, \ \forall i, \forall v\text{~ large enough}\}$;

\smallskip

$\underline{L}^{*}_{g}$ = elements of $\underline{L}^{*}$ with suitable growth conditions;

\smallskip

$\underline{K}^{\infty}_{g}$ = elements of $\underline{K}^{\infty}$ with suitable growth conditions;

\smallskip

(suitable growth simply means that $\alpha^{*}$ operates.)

\smallskip

$R$ = resultant of $(f,E_{1}f,\ldots,E_{n}f)$;

\smallskip

$K$ = algebraic number field;

\smallskip

$\underline{O_{K}}$ = ring of integers of $K$;

\smallskip

$f\in \underline{O_{K}}[X_{1},\ldots,X_{n+1}]$;

\smallskip

$b(\lambda)$ is a suitably chosen element of $\underline{O_{k}}[\lambda]$.

\smallskip

For each prime $p$ we assume an extension of $p$ to $K$ has been chosen and that $q$ is the cardinality of the residue class field.

\section{Non-singular case}\label{art07-sec2}

In this paragraph we suppose that $f$ defines a non-singular hypersurface in general position (hence $R\neq 0$). In this case $\dim \underline{K}=d^{n}$ and if $\overline{f}$ is non-singular and in general position (i.e. $\overline{R}\neq 0$) then all elements of $\underline{K}$ satisfy growth conditions 
\begin{equation}
\ord A_{w}=-0(\log w_{0}).\label{art07-eq1}
\end{equation}
The zeta function of $\overline{V'}$ is given by the characteristic polynomial of $\alpha^{*}|\underline{K}$. The Koszul complex of $D^{*}_{1}$, $D^{*}_{2},\ldots,D^{*}_{n+1}$ acting on $\underline{L^{*}}$ and $L^{*}_{g}$ is acyclic.

If $f_{\lambda}\in \underline{O}_{K}[\lambda,X]$ where $\lambda=(\lambda^{(1)},\ldots,\lambda^{(\mu)})$ is a set of independent parameters and if $R(\lambda)$ is not identically zero then as above we may define $\underline{K_{\lambda}}$ as a $K(\lambda)$ space and any basis has the form
\begin{equation}
\xi^{*}_{i,\lambda}=b^{-1}\Sigma_{w}G^{(i)}_{w}(\lambda)X^{-w}(\pi R(\lambda))^{-w_{0}}, i=1,2,\ldots,d^{n},\label{art07-eq2}
\end{equation}
with $G^{(i)}_{w}\in \underline{O}_{K}[\lambda]$, $i=1,\ldots,d^{n}$ and $\deg G^{(i)}_{w}\leq kw_{0}$ for suitable constant $k$. For $\lambda$ $p$-adically close to $\lambda_{0}$ we have the map of $\underline{K_{\lambda_{0}}}$ onto $\underline{K_{\lambda}}$ (with suitable extension of field of coefficients)
$$
T_{\lambda_{0},\lambda}=\gamma_{-}\circ \exp \pi X_{0}(f_{\lambda_{0}}-f_{\lambda})
$$
and\pageoriginale we have the commutative diagram
\[
\xymatrix@=1.5cm{
K_{\lambda^{q}_{0}}\ar[r]^-{T_{\lambda^{q}_{0},\lambda^{q}}}\ar[d]_-{\alpha^{*}_{\lambda_{0}}} & K_{\lambda^{q}}\ar[d]^-{\alpha^{*}_{\lambda}} \\
K_{\lambda_{0}}\ar[r]_-{T_{\lambda_{0},\lambda}} & K_{\lambda}
}
\]
where $\alpha^{*}_{\lambda}$ is defined by modifying the formula for $\alpha^{*}$, replacing $f(X)$ by $f_{\lambda}(X)$ and $f(X^{q})$ by $f_{\lambda^{q}}(X^{q})$. The matrix $C_{\lambda}$ of $T_{\lambda_{0},\lambda}$ (relative to the bases \eqref{art07-eq2}) is the solution matrix of a system of linear differential equations (with coefficients in $K(\lambda)$)
\begin{equation}
\dfrac{\partial}{\partial \lambda^{(t)}}\underline{X}=\underline{X}B^{(t)}, t=1,2,\ldots,\mu,\label{ART07-EQ3}
\end{equation}
which is independent of $p$ and $\lambda_{0}$. The matrix of $\alpha^{*}_{\lambda}$ relative to our which is independent of $p$ and $\lambda_{0}$. The matrix of $\alpha^{*}_{\lambda}$ relative to our basis is holomorphic (as function of $\lambda$) in a region
\begin{equation}
W=\{\lambda|(R(\lambda))|>1-\epsilon, |\lambda|<1+\epsilon\}\label{art07-eq4}
\end{equation}
for some $\epsilon>0$. It follows from Krasner that for $\lambda\in W$, $|\lambda|=1$, the zeta function of $\overline{V}'_{\lambda}$ is determined by \eqref{ART07-EQ3} and the matrix of $\alpha^{*}_{\lambda_{0}}$ for one specialization of $\lambda_{0}$ in $W$. This has seemed remarkable and it is this situation which we wish to extend to the singular case.

\section{Explanation of equation \texorpdfstring{\eqref{ART07-EQ3}}{eq3} (Katz \texorpdfstring{\cite{ART07-KEY5}}{key5})}\label{art07-sec3}

$\underline{K}$ is dual to $\underline{L}/\Sigma D_{i}\underline{L}$. Again let $w$ be in $Z^{n+2}$ such that $dw_{0}=w_{1}+\cdots+w_{n+1}$. Let $\widetilde{\underline{L}}^{0}$ be the span of all $X^{w}$ such that $w_{0}>0$ (but $w_{1},\ldots,w_{n+1}$ may be negative). It is shown by Katz that
$$
X^{w}\to \dfrac{(w_{0}-1)!}{(-\pi)^{w_{0}-1}}\dfrac{d(X_{1}/X_{n+1})}{(X_{1}/X_{n+2})}\wedge\ldots\wedge\dfrac{d(X_{n}/X_{n+1})}{(X_{n}/X_{n+1})}\dfrac{X^{w}}{(X_{0}f)^{w}},
$$
modulo exact $n-1$ forms gives an isomorphism
$$
\widetilde{\underline{L^{0}}}\Sigma^{n+1}_{i=1}\quad D_{i}\widetilde{\underline{L^{0}}}\simeq H^{n}(V')
$$
and\pageoriginale if $f$ is replaced by $f_{\lambda}$ then the endomorphisms $\sigma_{t}=\dfrac{\partial}{\partial \lambda^{(t)}}+\pi X_{0}\dfrac{\partial f_{\lambda}}{\partial \lambda^{(t)}}$ on the left is transformed into differentaition of the $n^{\text{th}}$ de-Rham space on the right with respect to $\lambda^{(t)}(t=1,2,\ldots,\mu)$. This is valid without restriction on $f$ but if $f$ is non-singular in general position then by comparison of dimensions, he showed that $\underline{L}'/\Sigma D_{i}\underline{L}$ is identified with the factor space of $H^{n}(V')$ modulo an $n+1$ dimensional subspace consisting of invariant classes. ($L'$ is the set of all elements of $\underline{L}$ with zero constant term.) This identifies Equation \eqref{ART07-EQ3} with the ``dual'' of the Fuchs-Picard equation of $H^{n}(V')$ if we use the fact that \eqref{ART07-EQ3} is equivalent to the fact that $\sigma_{t}\circ T_{\lambda_{0},\lambda}$ annihilates $\underline{K}_{\lambda_{0}}$.

\section{Singular case \texorpdfstring{\cite{ART07-KEY3}}{key3}}\label{art07-sec4}

Here we know finiteness of the Koszul complex for $\underline{L}^{*}$ and for $\underline{K}^{\infty}$ but not in general for $\underline{L}^{*}_{g}$ or for $\underline{K}^{\infty}_{g}$. However $\underline{K}^{\infty}=\underline{K}^{\infty}_{g}$ for almost all primes $p$ and in this way the theory has been developed only for a generic prime.

We mention that this restriction could be removed if we could show:

\begin{conjecture*}
A linear differential operator in one variable with polynomial coefficients operating on functions holomorphic in an ``open'' disk has finite cokernel. (This is known in the complex case. In the $p$-adic case it is known only for disks which are either small enough or large enough. It is true without restriction if the coefficients are constants. The conjecture is false for ``closed'' disks.)
\end{conjecture*}

In any case the zeta function of $\overline{V}'$ is given by the action of $\alpha^{*}$ on the factor spaces of the Koszul complex of $\underline{K}^{\infty}_{g}$. The first term, i.e. the characteristic polynomial of $\alpha^{*}|\underline{K}_{g}$ dominates the zeta function in that up to a factor of power of $q$, the zeros and poles of the zeta function occur in this factor. In the following we consider the variation of $\underline{K}$ (and of the corresponding factor of the zeta function) as $f$ varies.

If we again consider $f_{\lambda}\in K[\lambda,X]$, (but now $R(\lambda)$ may be identically zero) then we may again construct $\underline{K}_{\lambda}$; its dimension $N$ over $K(\lambda)$\pageoriginale is not less than $N_{\lambda_{0}}=\dim_{K(\lambda_{0})}\underline{K}_{\lambda_{0}}$ for each specialization, $\lambda_{0}$, of $\lambda$.

\begin{theorem*}
Each basis $\{\xi^{*}_{i,\lambda}\}^{N}_{i=1}$ of $\underline{K}_{\lambda}$ is of the form
\begin{equation}
\xi^{*}_{i,\lambda}=b(\lambda)^{-1}\Sigma M_{w}^{(i)}(\lambda)X^{-w}/(\pi G(\lambda))^{w_{0}}\label{art07-eq5}
\end{equation}
where $G(\lambda_{0})=0$ if and only if $N_{\lambda_{0}}<N$. If $N_{\lambda_{0}}=N$ then the basis may be chosen such that $b(\lambda_{0})\neq 0$. For each pair $(i,w)$, $M^{(i)}_{w}$ is a polynomial whose degree is bounded by a constant multiple of $w_{0}$.
\end{theorem*}

For almost all $p$
\begin{equation}
\ord'_{p}(M^{(i)}_{w}/(\pi G)^{w_{0}})=0(\log w_{0})\label{art07-eq6}
\end{equation}
where the left side refers to the $p$-adic ordinal extended to $K(\lambda)$ in a formal way (generic value on circumference of unit poly disk). We again have the mapping $T_{\lambda_{0},\lambda}$ of $\underline{K}_{\lambda_{0}}$ into $\underline{K}_{\lambda}$ for $\lambda$ close enough to $\lambda_{0}$ and if $N_{\lambda_{0}}=N$ then the matrix of this mapping is again a solution matrix of Equation \eqref{ART07-EQ3}. Also the matrix of $\alpha^{*}_{\lambda}$ is holomorphic in a region of the same type as before,
$$
|G(\lambda)|>1-\epsilon, |\lambda|<1+\epsilon,
$$
for some $\epsilon>0$ (this region may be empty for a finite set of $p$) and the theory of Krasner may again be applied. This completes our statement of results.

We now discuss equation \eqref{art07-eq5}. If $f_{\lambda}$ is generically singular, choose a new family, $f_{\lambda,\Gamma}$, which is generically non-singular and which coincides with $f_{\lambda}$ when $\Gamma=0$. We have the mapping $T^{(\lambda,\Gamma)}$ of $\underline{K}_{\lambda}$ into $\underline{K}_{\lambda,\Gamma}$ given by $\gamma_{-}\circ \exp (\pi X_{0}(f_{\lambda}-f_{\lambda,\Gamma}))$ and for $\xi^{*}$ in $\underline{K}_{\lambda}$ we may write
\begin{equation}
T^{(\lambda,\Gamma)}\xi^{*}=\Sigma_{j}\underline{X}_{j}\xi^{*}_{j,\lambda,\Gamma}\label{art07-eq7}
\end{equation}
where $\{\xi^{*}_{j,\lambda,\Gamma}\}^{d^{n}}_{j=1}$ is a basis of $\underline{K}_{\lambda,\Gamma}$ given by \eqref{art07-eq2}. The left side of \eqref{art07-eq6} lies in $K(\lambda)[[\Gamma,X^{-1}]]$, $\underline{X}_{j}$ lies in $K(\lambda)[[\Gamma]]$ and $\xi^{*}$ may be recovered by setting $\Gamma=0$ (i.e. by determining the coefficient of $\Gamma^{0}$ on the right side).

Our only information about the vector $\underline{X}=(\ldots,\underline{X}_{j},\ldots)$ is that it satisfies a differential equation
\begin{equation}
\dfrac{\partial \underline{X}}{\partial \Gamma}=\underline{X}B\label{art07-eq8}
\end{equation}\pageoriginale
when $B$ is rational in $\lambda$, $\Gamma$. If this equation has (for generic $\lambda$) a regular singular point at $\Gamma=0$ (which is not clear since the theorem of Griffiths need not apply to $V'_{\lambda,\Gamma}$, even though $f_{\lambda,\Gamma}$ is generically non-singular) and if further $\Gamma B=B_{0}(\lambda)+\Gamma B_{1}(\lambda)+\cdots$ then each formal power series solution $\sum\limits^{\infty}_{s=0}A_{s}\Gamma^{s}$ must have the form (for some $h\in K[\lambda]$, $v\in Z_{+}$)
$$
A_{s}=\text{~polynomial~}/\left\{h(\lambda)^{s}\prod\limits^{s}_{t=v}\det (tI-B_{0})\right\}.
$$
Since $B_{0}$ is a function of $\lambda$ this leads to the possibility that the singular locus (in $\lambda$) of the formal solution is an infinite union of varieties,
$$
\{\det (tI-B_{0})=0\}^{\infty}_{t=v},
$$
and this would leave the same possibility for the singular locus of the coefficients of $\xi^{*}$. We indicate two methods by which this difficulty may be overcome.

\begin{method}\label{art07-method1}
In the above analysis, the hypothesis of regularity of singularity of \eqref{art07-eq8} at $\Gamma=0$ was not essential but now we use this hypothesis to conclude (with the aid of \S3) that for fixed $\lambda$, the zeros of the polynomial $\det(tI-B_{0})$ (i.e. the roots of the indicial polynomial of \eqref{art07-eq7}) are related to the eigenvalues of the monodromy matrix for $H^{n}(V'_{\lambda,\Gamma})$ for a circuit about $\Gamma=0$. Since this matrix can be represented by a matrix with integral coefficients which is continuous as function of $\lambda$ for $\lambda$ near a generic point, the conclusion is that the polynomial $\det (tI-B_{0})$ is independent of $\lambda$. With this conclusion the method of the previous paragraph easily leads to equation \eqref{art07-eq5}. However as noted a (probably not serious) gap remains in this treatment since the question of regularity of singularity of \eqref{art07-eq7} is not settled.
\end{method}

\begin{method}\label{art07-method2}
By means of Equations \eqref{art07-eq6} and \eqref{art07-eq7} together with crude estimates for growth conditions of formal power series solutions of ordinary differential equations we show for each prime $p$, a\pageoriginale constant $c_{p}$ and an element $a_{0}$ of $K[\lambda]$ such that each $\Sigma A_{w}X^{-w}$ in $\underline{K}_{\lambda_{0}}$ satisfies growth conditions
\begin{equation}
\ord A_{w}\geq -c_{p}w_{0}+0(1)\label{art07-eq9}
\end{equation}
provided
\begin{itemize}
\item[(i)] $\lambda_{0}$ lies in a certain Zariski open set defined over $K$,

\item[(ii)] $|\lambda_{0}|\leq 1$,

\item[(iii)] $a_{0}(\lambda_{0})$ is a unit,

\item[(iv)] $p$ is not one of a certain finite set of primes.
\end{itemize}
\end{method}

Now let $\lambda_{0}$ be algebraic over $K$, in the Zariski open set of (i) and such that $a_{0}(\lambda_{0})\neq 0$. We may choose $p$ such that conditions (ii), (iii), (iv) are satisfied and then for $\lambda_{1}$ close enough to $\lambda_{0}$ the conditions (i)-(iv) remain satisfied. We may put upon $\lambda_{1}$ the further conditions that $\ord (\lambda_{1}-\lambda_{0})>c_{p}$ and that $\lambda_{1}$ be of maximal transcendence degree over $K$. We conclude that the dimension of $\underline{K}_{\lambda_{1}}$ (over $K(\lambda_{1})$) is $N$, that the elements of $\underline{K}_{\lambda_{1}}$ satisfy \eqref{art07-eq9} and hence that $T_{\lambda_{1},\lambda_{0}}$ is defined. We conclude (since $T_{\lambda_{1},\lambda_{0}}$ is injective) that the dimension of $K_{\lambda_{0}}$ is $N$. From this we conclude that
$$
N_{\lambda_{0}}=N
$$
for all $\lambda_{0}$ in a Zariski open set. This is the central point (which one might expect to follow from general principles); from this and equations \eqref{art07-eq6} and \eqref{art07-eq7} the remainder of the results may be deduced. The details are explained in \cite{art07-key4}.

\begin{thebibliography}{99}
\bibitem{art07-key1} \textsc{B. Dwork :} On the zeta function of a hypersurface, {\em I.H.E.S. No.} 12, {\em Paris} 1962.

\bibitem{art07-key2} \textsc{B. Dwork :} On the zeta function of a hypersurface II, {\em Ann. of Math.} 80 (1964), 227-299.

\bibitem{ART07-KEY3} \textsc{B. Dwork :} On the zeta function of a hypersurface III, {\em Ann. of Math.} 83 (1966), 457-519.

\bibitem{art07-key4} \textsc{B. Dwork :}\pageoriginale On the zeta function of a hypersurface IV, {\em Ann. of Math.} (to appear).

\bibitem{ART07-KEY5} \textsc{N. Katz :} On the differential equations satisfied by period matrices, {\em I.H.E.S.} (to appear).
\end{thebibliography}

\medskip
\noindent
{\small Princeton University}

\noindent
{\small Princeton, N. J., U.S.A.}
