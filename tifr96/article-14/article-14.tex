\title{ON CANONICALLY POLARIZED VARIETIES}
\markright{On Canonically Polarized Varieties}

\author{By~~ T. Matsusaka$^{*}$}

\footnotetext{This work was done while the author was partially supported by N. S. F.}
\date{}

\maketitle

\setcounter{pageoriginal}{264}
\section*{Brief summary of notations and conventions}
\pageoriginale

We shall follow basically notations and conventions of \cite{art14-key21}, \cite{art14-key25}, \cite{art14-key32}. For basic results on specializations of cycles, we refer to \cite{art14-key24}. When $U$ is a complete variety, non-singular in codimension $1$ and $X$ a $U$-divisor, the module of functions $g$ such that $\div (g)+X\succ 0$ will be denoted by $L(X)$. We shall denote by $l(X)$ the dimension of $L(X)$. The complete linear system determined by $X$ will be denoted by $\Lambda(X)$. A finite set of functions $(g_{i})$ in $L(X)$ defines a rational map $f$ of $U$ into a projective space. $f$ will be called a {\rm rational map of $U$ defined by $X$.} When $(g_{i})$ is a basis of $L(X)$, it will be called a {\em non-degenerate map}. $X$ will be called {\em ample} if a non-degenerate $f$ is a projective embedding. It will be called {\em non-degenerate} if a positive multiple of $X$ is ample. (In the terminology of Grothendieck, there are called {\em very ample} and {\em ample}). Let $W$ be the image of $U$ by $f$ and $\Gamma$ the closure of the graph of $f$ on $U\times W$. We shall denote by $\deg (f)$ the number $[\Gamma:W]$. For any $U$-cycle $Y$, we shall denote by $f(Y)$ the cycle $pr_{2}(\Gamma\cdot (Y\times W))$. We shall denote by $\mathfrak{K}(U)$  a {\em canonical divisor} of $U$. When $X$ is a Cartier divisor on $U$, we shall denote by $\mathscr{L}(X)$ the invertible sheaf defined by $X$. When $U$ is a subvariety of a projective space, $C_{U}$ will denote a hyperplane section of $U$. When $U$ is a polarized variety, a {\em basic polar divisor} will be denoted by $X_{U}$.

By an {\em algebraic family of positive cycles}, we shall understand the set of positive cycles in a projective space such that the set of Chow-points forms a {\em locally closed} subset of a projective space. By identifying these cycles with their Chow-points, some of the notations and results on points can be carried over to algebraic families and this will be done frequently.

Finally, $\mathfrak{G}_{l}$, $\mathfrak{G}_{a}$, $\sim$ will denote respectively the group of divisors linearly equivalent to zero, the group of divisors algebraically equivalent to zero and the linear equivalence of divisors.

\section*{Introduction}\pageoriginale

Let $V^{n}$ be a polarized variety and $X_{V}$ a basic polar divisor on $V$. Then the Euler-Poincar\'e characteristic $\chi(V,\mathscr{L}(mX_{V}))$ is a polynomial $P(m)$ in $m$. We have defined this to be the {\em Hilbert characteristic polynomial} of $V$ (c.f. \cite{art14-key16}). If $d$ is the rank of $V$, i.e. $d=X^{(n)}_{V}$, any algebraic deformation of $V$ of rank $d$ has the same Hilbert characteristic polynomial $P(m)$ (c.f. \cite{art14-key16}). As we pointed out in \cite{art14-key16}, if we can find a constant $c$, which depends only on $P(m)$, such that $mX_{V}$ is ample for $m\geq c$, then the existence of a universal family of algebraic deformations of $V$ of bounded ranks follows. The existence of such a constant is well-known in the case of curves and Abelian varieties. We solved this problem for $n=2$ in \cite{art14-key17} (compare \cite{art14-key9}, \cite{art14-key10}, \cite{art14-key12}). But the complexity we encountered was of higher order of magnitude compared with the case of curves. The same seems to be the case for $n\geq 3$ when compared with the case $n=2$. One of the main purposes of this paper is to solve the problem for $n=3$ when $V$ is ``generic'' in the sense that $\mathfrak{K}(V)$ is a non-degenerate polar divisor.

In general, let us consider the following problems.
\begin{itemize}
\item[$(A_{n})$] Find a constant $c$, which depends on the polynomial $P(x)$ only, such that $h^{i}(V,\mathscr{L}(mX_{V}))=0$ for $i>0$ whenever $m\geq c$.

\item[$(A'_{n})$] Find two constants $c$, $c'$, which depend on $P(x)$ only, such that $h^{i}(V,\mathscr{L}(mX_{V}))<c'$ for $i>0$ whenever $m\geq c$.

\item[$(A''_{n})$] Find two constants $c$, $c'$, which depend on $P(x)$ only, such that $|l(mX_{V})-P(m)|<c'$ whenever $m\geq c$.

\item[$(B_{n})$] Find a constant $c$, which depends on $P(x)$ only, such that $mX_{V}$ defines a birational map of $V$ whenever $m\geq c$.

\item[$(C_{n})$] Find a constant $c$, which depends on $P(x)$ only, such that $mX_{V}$ is ample whenever $m\geq c$.
\end{itemize}
As we mentioned, what we are interested in is the solution of $(C_{n})$. But $(A_{n})$, $(B_{n})$ could be regarded as step-stones for this purpose. It is easy to see that the solution of $(C_{n})$ implies the solutions of $(A_{n})$ and $(B_{n})$. In the case of characteristic zero and $\mathfrak{K}(V)$ a non-degenerate\pageoriginale polar divisor, $(A_{n})$ can be solved easily (\S1). Assuming that $(A''_{n})$ has a solution, we shall show that $(B_{n})$ has a solution for $n=3$ when the characteristic is zero. Assuming that $(A_{n})$ and $(B_{n})$ have solutions and that $\mathfrak{K}(V)$ is a non-degenerate polar divisor, we shall show that $(C_{n})$ has a solution when the characteristic is zero. Hence $(C_{3})$ has a solution when the characteristic is zero and $\mathfrak{K}(V)$ is a non-degenerate polar divisor.

\bigskip

\begin{center}
{\Large\bf Chapter I. \boldmath$(A''_{n})$ and $(B_{n})$}\labeltext{1}{art14-chap1}
\end{center}

\section{Canonically polarized varieties}\label{art14-chap1-sec1}

We shall first recall the definition of a polarized variety as modified in \cite{art14-key16}. Let $V^{n}$ be a complete non-singular variety and $\mathscr{M}$ a finite set of prime numbers consisting of the characteristic of the universal domain (or the characteristics of universal domains) and the prime divisors of the order of the torsion group of divisors of $V$. Let $\mathscr{X}$ be a set of $V$-divisors satisfying the following conditions: (a) $\mathscr{X}$ contains an ample divisor $X$; (b) a $V$-divisor $Y$ is contained in $\mathscr{X}$ if and only if there is a pair $(r,s)$ of integers, which are prime to members of $\mathscr{M}$, such that $rY\equiv sX\mod \mathfrak{K}_{a}$. When there is a pair $(\mathscr{M},\mathscr{X})$ satisfying the above conditions, $\mathscr{X}$ is called a {\em structure set} of polarization and $(V,\mathscr{X})$ a {\em polarized variety}. A divisor in $\mathscr{X}$ will be called a {\em polar divisor} of the polarized variety. There is a divisor $X_{V}$ in $\mathscr{X}$ which has the following two properties: (a) a $V$-divisor $Y$ is in $\mathscr{X}$ if and only if $Y\equiv rX_{V}\mod \mathfrak{K}_){a}$ where $r$ is an integer which is prime to members of $\mathscr{M}$; (b) when $Z$ is an ample polar divisor, there is a positive integer $s$ such that $Z\equiv sX_{V}\mod \mathfrak{G}_{a}$ (c.f. \cite{art14-key16}). $X_{V}$ is called a {\em basic polar divisor}. The self-intersection number of $X_{V}$ is called the {\em rank} of the polarized variety. A polarized variety will be called a {\em canonically polarized variety} if $\mathfrak{K}(V)$ is a polar divisor.

\begin{lemma}\label{art14-lem1}
Let $V^{n}$ be a canonically polarized variety and $P(x)=\Sigma_{\gamma_{n-i}}x^{i}$ the Hilbert characteristic polynomial of $V$. Then $\mathfrak{K}(V)\equiv \rho X_{V}\mod \mathfrak{K}_{a}$ where $rho$ is a root of $P(x)-(-1)^{n}\gamma_{n}=0$.
\end{lemma}

\begin{proof}
It follows from Serre's duality theorem that 
$$
\chi(V,\mathscr{L}(mX_{V}))=(-1)^{n}\chi(V,\mathscr{L}(\mathfrak{K}(V)-mX_{V}))\quad (\text{c.f. \cite{art14-key23}}).
$$\pageoriginale
We may replace $\mathfrak{K}(V)$ by $\rho X_{V}$ in this equality since $\chi$ is invariant under algebraic equivalence of divisors (c.f. \cite{art14-key4}, \cite{art14-key16}). Then we get $P(m)=(-1)^{n}P(\rho-m)$. Setting $m=0$, we get $(-1)^{n}P(\rho)=\gamma_{n}$. Our lemma is thereby proved.
\end{proof}

\begin{proposition}\label{art14-prop1}
Let $V$ be a canonically polarized variety in characteristic zero and $P(x)$ the Hilbert characteristic polynomial of $V$. Then there is a positive integer $\rho_{0}$, which depends on $P(x)$ only, such that $h^{i}(V,\mathscr{L}(Y))=0$ for $i>0$ and $h^{0}(V,\mathscr{L}(Y))=l(Y)>0$ whenever $m\geq \rho_{0}$ and $Y\equiv mX_{V}\mod \mathfrak{K}_{a}$.
\end{proposition}

\begin{proof}
Let $\gamma_{n}$ be the constant term of $P(x)$ and $s_{0}$ the maximum of the roots of the equation $P(x)-(-1)^{n}\gamma_{n}=0$. Let $\mathfrak{K}(V)$ be a canonical divisor of $V$ and $k$ an algebraically closed common field of rationality of $V$, $X_{V}$ and $\mathfrak{K}(V)$. There is an irreducible algebraic family $\mathfrak{H}$ of positive divisors on $V$, defined over $k$, such that, for a fixed $k$-rational divisor $C_{0}$ in $\mathfrak{H}$, the classes of the $C-C_{0}$, $C\in \mathfrak{H}$, with respect to linear equivalence exhausts the points of the Picard variety of $V$ (c.f. \cite{art14-key15}). We shall show that $2s_{0}$ can serve as $\rho_{0}$.

Take $t$ so that $t-(s_{0}-\rho)=t'>0$ and $t-2(s_{0}-\rho)=m>0$ where $\mathfrak{K}(V)\equiv \rho X_{V}\mod \mathfrak{K}_{a}$. For any $C'$ in $\mathfrak{H}$, $tX_{V}+C'-C_{0}+\mathfrak{K}(V)\equiv (t'+s_{0})X_{V}\mod \mathfrak{K}_{a}$ and $t'+s_{0}>0$. Hence $tX_{V}+C'-C_{0}+\mathfrak{K}(V)$ is non-degenerate (c.f. \cite{art14-key16}, Th. 1) and the higher cohomology groups of the invertible sheaf $\mathscr{A}$ determined by $tX_{V}+C'-C_{0}+2\mathfrak{K}(V)$ vanish by Kodaira vanishing theorem (c.f. \cite{art14-key11}). $tX_{V}+C'-C_{0}+2\mathfrak{K}(V)\equiv (m+2s_{0})X_{V}\mod \mathfrak{K}_{a}$ and $\chi(V,\mathscr{A})=P(m+2s_{0})$ since $\chi$ is invariant by algebraic equivalence of divisors. Moreover, $P(m+2s_{0})>0$ by our choice of $m$ and $s_{0}$. It follows that $h^{0}(V,\mathscr{A}>0$. Therefore, in order to complete our proof, it is enough to prove that a $V$-divisor $Z$ such that $Z\equiv (m+2s_{0})X_{V}\mod \mathfrak{K}_{a}$ has the property that $Z\sim tX_{V}+C'-C_{0}+2\mathfrak{K}(V)$ for some $C'$ in $\mathfrak{H}$. Clearly, such $Z$ is algebraically equivalent to $tX_{V}+2\mathfrak{K}(V)$. Hence
$$
Z-(tX_{V}+2\mathfrak{K}(V))\sim C'-C_{0}
$$
for some $C'$ in $\mathfrak{H}$. Our proposition is thereby proved.
\end{proof}

\section{Estimation of \texorpdfstring{$l(C_{U})$}{CU} on a projective surface}\label{art14-sec2}\pageoriginale

Let $V$ be a non-singular surface in a projective space and $\Gamma$ a curve on $V$. Let $\mathfrak{R}$ be the intersection of local rings of $\Gamma$ at the singular points of $\Gamma$. Using only those functions of $\Gamma$ which are in $\mathfrak{R}$, we can define the concept of complete linear systems and associated sheaves, as on a non-singular curve. Let $\Gamma'$ be a $V$-divisor such that $\Gamma'\sim \Gamma$, that $\Gamma'$ and $\Gamma$ intersect properly on $V$ and that no singular point of $\Gamma$ is a component of $\Gamma\cdot \Gamma'$. Similarly let $\mathfrak{K}(V)$ be such that $\mathfrak{K}(V)$ and $\Gamma$ intersect properly on $V$ and that no singular point of $\Gamma$ is a component of $\Gamma\cdot \mathfrak{K}(V)$. Then $\Gamma\cdot (\Gamma'+\mathfrak{K}(V))=\mathfrak{K}(\Gamma)$ is a canonical divisor of $\Gamma$, $p_{a}(\Gamma)=1+\frac{1}{2}\deg (\mathfrak{K}(\Gamma))$ and the generalized Riemann-Roch theorem states that $l(\mathfrak{m})=\deg (\mathfrak{m})-p_{a}(\Gamma)+1+l(\mathfrak{K}(\Gamma)=\mathfrak{m})$ for a $\Gamma$-divisor $\mathfrak{m}$ (c.f. \cite{art14-key20}, \cite{art14-key22}).

If $C$ is a complete non-singular curve, the theorem  of Clifford states that $\deg (\mathfrak{m})\geq 2l(\mathfrak{m})-2$ for a {\em special} $C$-divisor $\mathfrak{m}$. We shall first extend this to $\Gamma$.

\begin{lemma}\label{art14-lem2}
Let $V$, $\Gamma$ and $\mathfrak{K}(\Gamma)$ be as above and $\mathfrak{m}$ a special positive divisor on $\Gamma$ (i.e. $l(\mathfrak{K}(\Gamma)-\mathfrak{m})>0$). Then $\deg (\mathfrak{m})\geq 2l(\mathfrak{m})-2$.
\end{lemma}

\begin{proof}
When $l(\mathfrak{m})=1$, our lemma is trivial since $\mathfrak{m}$ is positive. Therefore, we shall assume that $l(\mathfrak{m})>1$.

Let $\Gamma^{*}$ be a normalization of $\Gamma$, $\alpha$ the birational morphism of $\Gamma^{*}$ on $\Gamma$ and $T$ the graph of $\alpha$. For any $\Gamma$-divisor $\mathfrak{a}$, we set $\mathfrak{a}^{*}=\alpha^{-1}(\mathfrak{a})=pr_{\Gamma^{*}}((\mathfrak{a}\times \Gamma^{*})\cdot T)$. When the $f$ are elements of $L(\mathfrak{a})$, i.e. elements of $\mathfrak{R}$ such that $\div (f)+\mathfrak{a}\succ 0$, the $f\circ \alpha=f^{*}$ generate a module of functions on $\Gamma^{*}$ which we shall denote by $\alpha^{-1}L(\mathfrak{a})$. The module $L(\mathfrak{K}(\Gamma)-\mathfrak{m})$ is not empty since $\mathfrak{m}$ is special. Hence it contains a function $g$ in $\mathfrak{R}$. Let $\div(g^{*})=\mathfrak{m}^{*}+\mathfrak{n}^{*}-\mathfrak{K}(\Gamma)^{*}$. Let $N$ be the submodule of functions $f^{*}$ in $\alpha^{-1}L(\mathfrak{K}(\Gamma))$ defined by requiring $f^{*}$ to pass through $\mathfrak{n}^{*}$, i.e. requiring $f^{*}$ to satisfy: coefficient of $x$ in $\div (f^{*})\geq $ coefficient of $x$ in $\mathfrak{n}^{*}$ for each component $x$ of $\mathfrak{n}^{*}$. Let $\dim N=\dim L(\mathfrak{K}(\Gamma))-t$. Then $\mathfrak{n}^{*}$ imposes $t$ linearly independent conditions in $\alpha^{-1}L(\mathfrak{K}(\Gamma))$. When these $t$ linear conditions are imposed on the vector subspace $\alpha^{-1}L(\mathfrak{K}(\Gamma)-\mathfrak{m})$, we get the vector space generated by $g^{*}$ over the universal domain. It follows that $l(\mathfrak{K}(\Gamma)-\mathfrak{m})-t\leq 1$,\pageoriginale i.e. $t\geq l(\mathfrak{K}(\Gamma)-\mathfrak{m})-1$. Since $l(\mathfrak{K}(\Gamma))=p_{a}(\Gamma)$, we then get $p_{a}(\Gamma)-\dim N\geq (\mathfrak{K}(\Gamma)-m)-1$. $L(\mathfrak{m})$ has a basis $(h_{i})$ from $\mathfrak{R}$. The $h_{i}\cdot g$ are elements of $L(\mathfrak{K}(\Gamma))$ and $h_{i}^{*}\cdot g^{*}\in N$. Hence the multiplication by $g^{*}$ defines an injection of $\alpha^{-1}L(\mathfrak{m})$ into $N$. It follows that $l(\mathfrak{m})\leq \dim N$ and $p_{a}(\Gamma)-l(\mathfrak{m})\geq l(\mathfrak{K}(\Gamma)-\mathfrak{m})-1$. By the generalized Riemann-Roch theorem, we have $l(\mathfrak{K}(\Gamma)-\mathfrak{m})=l(\mathfrak{m}-\deg (\mathfrak{m})+p_{a}(\Gamma)-1$. When this is substituted above, we get the required inequality.
\end{proof}

\begin{proposition}\label{art14-prop2}
Let $V$ be a non-singular surface in a projective space such that $p_{g}\geq 1$. Let $C_{0}$ be a curve on $V$ such that the complete linear system $\Lambda (C_{0})$ is without fixed point and that $C^{(2)}_{0}>0$. Then $\dim \Lambda (C_{0})\leq \frac{1}{2}C^{(2)}_{0}+1$.
\end{proposition}

\begin{proof}
If $\dim \Lambda(C_{0})=0$, there is nothing to prove. Therefore we shall assume that $\dim \Lambda (C_{0})>0$. Let $\mathfrak{R}$ be the intersection of local rings of $C_{0}$ at the singular points of $C_{0}$. There is a canonical divisor $\mathfrak{K}(V)$ of $V$ whose support does not contain any singular point of $C_{0}$. Such $\mathfrak{K}(V)$ and $C_{0}$ intersect properly on $V$. There is a member $C$ of $\Lambda (C_{0})$ which does not contain any singular point of $C_{0}$. Such $C$ and $C_{0}$ intersect properly on $V$. We have 
$$
p_{a}(C_{0})=1+\frac{1}{2}\deg (C_{0}\cdot (C+\mathfrak{K}(V))
$$
and $C_{0}\cdot (C+\mathfrak{K}(V))=\mathfrak{k}$ is a canonical divisor of $C_{0}$. Let $\mathfrak{m}=C_{0}\cdot C$. By the generalized Riemann-Roch theorem, we get $l(\mathfrak{m})=\deg (\mathfrak{m})-p_{a}(C_{0})+1+l(\mathfrak{k}-\mathfrak{m})$ and $\mathfrak{k}-\mathfrak{m}=C_{0}\cdot \mathfrak{K}(V)$. By our assumption there is a function $f$, other than $0$, in the module $L(\mathfrak{K}(V))$. Since the support of $\mathfrak{K}(V)$ does not contain any singular point of $C_{0}$, $f$ is regular at these points. Let $f'$ be the function on $C_{0}$ induced by $f\cdot f'$ is then regular at every singular point of $C_{0}$ and is contained in $\mathfrak{R}$. Since $\div (f)+\mathfrak{K}(V)\succ 0$ it follows that $\div(f')+C_{0}$. $\mathfrak{K}(V)\succ 0$. Hence $f'\in L(C_{0}\cdot \mathfrak{K}(V))=L(\mathfrak{k}-\mathfrak{m})$, which proves that $l(\mathfrak{k}-\mathfrak{m})>0$ and that $\mathfrak{m}$ is a special $C_{0}$-divisor.

By Lemma \ref{art14-lem2} we have $\deg (\mathfrak{m})=C^{(2)}_{0}\geq 2l(\mathfrak{m})-2$. Every function $g$ in $L(C)$ induces a function $g'$ on $C_{0}$ contained in $\mathfrak{R}$ since $C_{0}\cdot C$ has no singular component on $C_{0}$. Hence $\div (g')+\mathfrak{m}\succ 0$ and it follows that $g'\in L(\mathfrak{m})$.\pageoriginale If $g'=0$, $\div(g)=C_{0}-C$. Consequently $l(\mathfrak{m})\geq l(C)-1$ and $C^{(2)}_{0}\geq 2l(C)-4$. Our proposition follows at once from this.

We shall recall here the definition of the {\em effective geometric genus} of an algebraic variety $W$. Let $W$ and $W'$ be complete normal varieties and assume that there is a birational morphism of $W'$ on $W$. Then $p_{g}(W)\geq p_{g}(W')$. Hence there is a complete normal variety $W''$, birationally equivalent to $W$, such that $p_{q}(W'')$ has the minimum value $p_{g}$ among the birational class of $W$. This $p_{g}$ is called the {\em effective geometric genus} of $W$. When $W$ is non-singular, $p_{g}(W)=p_{g}$ (c.f. \cite{art14-key13}, \cite{art14-key31}).
\end{proof}

\begin{proposition}\label{art14-prop3}
Let $U$ be an algebraic surface in a projective space and $C_{U}$ a hyperslane section of $U$. Assume that the effective geometric genus $p_{g}$ of $U$ is at least $1$. Let $\Lambda$ be the linear system of hyperplane sections of $U$ and denote by $C^{(2)}_{U}$ the degree of $U$. Then $\dim \Lambda \leq \frac{1}{2}C^{(2)}_{U}+1$.
\end{proposition}

\begin{proof}
Let $V$ be a non-singular surface in a projective space and $f$ a birational morphism of $V$ on $U$ (c.f. \cite{art14-key34}). Let $k$ be an algebraically closed common field of rationality of $U$, $V$ and $f$. Let $T$ be the group of $f$ on $V\times U$ and $P$ the ambient projective space of $U$. We may assume that $U$ is not contained in any hyperplane. Then $T$ and $V\times H$ intersect properly on $V\times P$ for every hyperplane $H$. We set $f^{-1}(H)=\pr_{V}(T\cdot (V\times H))$. Let $\Lambda^{*}$ be the set of $f^{-1}(H)$. It is a linear system on $V$. Since $f$ is a morphism, it has no fixed point. Therefore it has no fixed component in particular. Let $H$ be a generic hyperplane over $k$ and $C_{U}=U\cdot H$. The $\pr_{V}(T\cdot (V\times H))=\pr_{V}(T\cdot (V\times C_{U}))$ where the latter intersection-product is taken on $V\times U$ (c.f. \cite{art14-key25}, Chap. VIII). Setting $f^{-1}(C_{U})=\pr_{V}(T\cdot (V\times C_{U}))$, $f^{-1}(C_{U})$ is a generic member of $\Lambda^{*}$ over $k$. Since $f$ is a birational transformation, every component of $f^{-1}(C_{U})$ has to appear with coefficient 1. It follows that $f^{-1}(C_{U})$ is an irreducible curve by the theorem of Bertini (c.f. \cite{art14-key25}, Chap. IX) and $\Lambda^{*}$ has no fixed point. Moreover, $p_{g}(V)=p_{g}$ and $\dim \Lambda = \dim \Lambda^{*}$.

Let $C_{U}$ and $C'_{U}$ be two independent generic members of $\Lambda$ over $k$. When $Q$ is a component of $C_{U}\cap C'_{U}$, it is a generic point of $U$ over $k$ and\pageoriginale is a proper component of multiplicity 1 on $U$. When that is so, we get $f^{-1}(C_{U})\cdot f^{-1}(C'_{U})=f^{-1}(C_{U}\cdot C'_{U})$ (c.f. \cite{art14-key25}, Chap. VIII). Then $\deg (C_{U}\cdot C'_{U})=\deg (f^{-1}(C_{U})\cdot f^{-1}(C'_{U}))$ and our proposition follows from these and from Proposition \ref{art14-prop2}.
\end{proof}

\section{A discussion on fixed components}\label{art14-sec3}

Let $V^{n}$ be a complete variety, non-singular in codimension 1 and $X$ a divisor on $V$. We denote by $\Lambda(X)_{\red}$ the reduced linear system determined uniquely by $\Lambda(X)$. Then $\Lambda(X)=\Lambda(X)_{\red}+F$ and $F$ is called the {\em fixed part} of $\Lambda(X)$. A component of $F$ is called {\em a fixed component} of $\Lambda(X)$.

\begin{lemma}\label{art14-lem3}
Let $V^{n}$ be a complete variety, non-singular in codimension $1$, $X$ a positive $V$-divisor and $F=\Sigma^{l}_{l}a_{i}F_{i}$ the fixed part of $\Lambda(X)$. Assume that $l((\alpha-1)X+F)>l((\alpha-1)X)$ for some positive integer $\alpha>1$ and that $X\neq F$. Then we have the following results: {\rm(a)} there is a positive divisor $F'=\Sigma^{l}_{1}a'_{i}F_{i}$ such that $F-F'\succ 0$, that $l((\alpha-1)X+F')=l((\alpha-1)X+F)$ and that $l((\alpha-1)X+F'-F_{j})<l((\alpha-1)X+F)$ for all $j$ with $F'-F_{j}\succ 0$; {\rm(b)} let $I$ be the set of indices $i$ such that $a'_{i}\neq 0$. Then the $F_{i}$, $i\in I$, are not fixed components of $\Lambda((\alpha-1)X+F')$.
\end{lemma}

\begin{proof}
(a) follows immediately from our assumption. Let $k$ be an algebraically closed common field of rationality of $V$, $X$ and for the components of $F$. Let $L$ be a generic divisor of $\Lambda((\alpha-1)X+F')_{\red}$ over $k$. The fixed part of $\Lambda((\alpha-1)X+F')$ is obviously of the form $\Sigma^{l}_{1}b_{s}F_{s}$. Suppose that $b_{i}\neq 0$ for some $i\in I$. Then $L+\Sigma^{l}_{1}b_{s}F_{s}-F_{i}$ is positive and is a member of $\Lambda((\alpha-1)X+F'-F_{i})$. Hence $l((\alpha-1)X+F')=l((\alpha-1)X+F'-F_{i})$ which is contrary to our choice of $F'$. Our lemma is thereby proved.
\end{proof}

\begin{lemma}\label{art14-lem4}
Using the same notations and assumptions of Lemma \ref{art14-lem3}, let $\beta=\Pi_{I}a'_{i}$. Then there is the smallest positive integer $\gamma$ satisfying $\beta(a_{i}-a'_{i})-\gamma a'_{i}\geq 0$ for $i\in I$ and $\beta(a_{i_{0}}-a'_{i_{0}})-\gamma a'_{i_{0}}=0$ for some $i_{0}\in I$. Moreover, $\Lambda(\beta\alpha X+\gamma(\alpha-1)X)$ has the property that $F_{i_{0}}$ is not a fixed component of it.
\end{lemma}

\begin{proof}
Let $Z$ be a generic divisor of $\Lambda(X)_{\red}$ over $k$. Then $\alpha X\sim (\alpha-1)X+\Sigma^{l}_{1}a_{i}F_{i}+Z=((\alpha-1)X+\Sigma_{I}a'_{i}F_{i})+\Sigma_{I}(a_{i}-a'_{i})F_{i}+F''+Z$ for\pageoriginale some positive divisor $F''$ which does not contain the $F_{i}$, $i\in I$, as components. By the above lemma, $\Lambda((\alpha-1)X+\Sigma_{I}a'_{i}F_{i})$ has the property that no $F_{i}$, $i\in I$, is a fixed component of it. Let $m$ be a positive integer. Then $(\beta+m)((\alpha-1)X+\Sigma_{I}a'_{i}F_{i})+\beta Z+\Sigma_{I}(\beta a_{i}-\beta a'_{i}-ma'_{i})F_{i}+\beta F''\sim \beta\alpha X+m(\alpha-1)X$. By what we have seen above. the $F_{i}$, $i\in I$, are not fixed components of the complete linear system determined by $(\beta+m)((\alpha-1)X+\Sigma_{I}a'_{i}F_{i})+\beta Z$. By the definition of $\beta$, the $\beta(a_{i}-a'_{i})$ are divisible by the $a'_{i}$. Hence we can find the smallest positive integer $\gamma$ as claimed in our lemma. Then there is an index $i_{0}\in I$ such that $F_{i_{0}}$ is not a component of $\Sigma_{I}(\beta a_{i}-\beta a'_{i}-\gamma a'_{i})F_{i}$. From these our lemma follows at once.
\end{proof}

\begin{proposition}\label{art14-prop4}
Let $V^{n}$ be a complete non-singular variety, $X$ a positive non-degenerate divisor on $V$ and $F$ the fixed part of $\Lambda(X)$. Assume that there is an integer $\alpha >1$ such that $l((\alpha-1)X+F)>l((\alpha-1)X)$ and that $X\neq F$. Let $d=X^{(n)}$ and $\mu(X,\alpha)=(d^{d}\alpha+d^{d+1}(\alpha-1))!$. Then there is a component $F_{i}$ of $F$ such that it is not a fixed component of the complete linear system determined by $\mu(X,\alpha)X$.
\end{proposition}

\begin{proof}
We shall estimate $\beta$ and $\gamma$ of Lemma \ref{art14-lem4}. Let $Z$ be as in the proof of Lemma \ref{art14-lem4} and $I(~ ,~ )$ denote the intersection number. Since $X\sim Z+\Sigma^{l}_{1}a_{i}F_{i}$, $X^{(n)}=I(X^{(n-1)},Z)+\Sigma^{l}_{1}a_{i}I(X^{(n-1)},F_{i})=d$. Hence $a_{i}>0$ and $\Sigma^{l}_{1}a_{i}<d$. It follows that $\beta<\Pi_{I}a_{i}<d^{d}\cdot \gamma$ satisfies $\beta(a_{i}-a'_{i})\geq \gamma a'_{i}$. Hence $\gamma\leq \beta(a_{i}-a'_{i})<\beta a_{i}<d^{d+1}$. Our proposition now follows from these and from Lemma \ref{art14-lem4}.
\end{proof}

\section{Estimation of some intersection numbers and its application}\label{art14-sec4}

Let $W^{n}$ be a non-singular projective variety and $T^{n-1}$ a subvariety of $W$. Let $Y$ be a non-degenerate divisor on $W$ and $k$ a common field of rationality of $W$, $T$ and $Y$. Let $m$ be a positive integer such that $l(mY)>1$ and $A_{1},\ldots,A_{n} n$ independent generic divisors of $\Lambda(mY)$ over $k$. Let the $D_{i}$ be the proper components of $|A_{1}|\cap \cdots \cap |A_{s}|$ on $W$ and $a_{i}=\text{coef.}_{D_{i}}(A_{1}\ldots A_{s})$. Let $I$ be the set of indices $i$ such that $D_{i}$ contains a generic point of $W$ over $k$. Then we define the symbol $I(A_{1}\ldots A_{s}/W,k)$ to denote $\Sigma_{I}a_{i}$. We shall denote by $(A_{1}\ldots A_{s}/W,k)$ the $W$-cycle $\Sigma_{I}a_{i}D_{i}$. Let the $E_{j}$ be the proper components of $|A_{1}|\cap \cdots \cap | A_{s}|\cap |T|$ on $W$ and $b_{j}=\text{coef.}_{E_{j}}(A_{1}\ldots A_{s}\cdot T)$. Let\pageoriginale $J$ be the set of indices $j$ such that $E_{j}$ contains a generic point of $T$ over $k$. Then we denote by $(A_{1}\ldots A_{s}\cdot T/T,k)$ and by $I(A_{1}\ldots A_{s}\cdot T/T,k)$ the $W$-cycle $\Sigma_{J}b_{j}E_{j}$ and the number $\Sigma_{J}b_{j}$ respectively.

\begin{lemma}\label{art14-lem5}
Let $W^{n}$ be a non-singular projective variety and $T^{n-1}$ a subvariety of $W$, both defined over a field $k$. Let $Y$ be a non-degenerate $W$-divisor, rational over $k$, and $m$ a positive integer such that $l(mY)>1$. Let $A_{1},\ldots,A_{s}$, $s\leq n$, be $s$ independent generic divisors of $\Lambda (mY)$ over $k$. Then we have the following inequalities: {\rm(a)} $I(A_{1}\ldots A_{s}/W,k)\leq m^{s}Y^{(n)}$; {\rm(b)} $I(A_{1}\ldots A_{s}\cdot T/T,k)\leq m^{s}I(Y^{(n-1)},T)$.
\end{lemma}

\begin{proof}
We shall prove only {\rm(b)}. {\rm(a)} can be proved similarly. We set $\Sigma b_{i}E_{i}=A_{1}*\ldots * A_{s}*T$. If the $b_{i}$ are zero for all $i$, our lemma is obviously true. Hence we shall assume that the $b_{i}$ are positive.

Let $r$ be a large positive integer such that $rm\ Y$ is ample and $C_{1},\ldots,\break C_{s},C'_{1},\ldots,C'_{n-s-1}n-1$ independent generic divisors of $\Lambda(rm\ Y)$ over $k$. Since $(A_{1}\ldots A_{s}\cdot T/T, k)\prec A_{1}*\ldots * A_{s} * T$, it follows that $I(A_{1}\ldots A_{s}\cdot T/T,k)<\Sigma b_{i}$. Since $Y$ is non-degenerate, $\Sigma b_{i}I(E_{i},Y^{(n-s-1)})\geq \Sigma b_{i}$. We have
{\fontsize{10pt}{12pt}\selectfont
$$
(1/(rm)^{n-s-1})\deg \{(A_{1}*\ldots * A_{s} * T)\cdot C'_{1}\ldots C'_{n-s-1}\}=\Sigma b_{i}I(E_{i},Y^{(n-s-1)}).
$$}\relax
The left hand side can be written as
$$
(1/r^{s}(rm)^{n-s-1})\deg \{(rA_{1} * \ldots * rA_{s} * T)C'_{1}\ldots C'_{n-s-1}\}.
$$

The $rA_{i}$ are members of $\Lambda(rm\ Y)$. Hence $((rA_{i}), T, (C'_{j}))$ is a specialization of $((C_{i}),T,(C'_{j}))$ over $k$. Let $\mathfrak{m}=C_{1}\ldots C_{s}\cdot T\cdot C'_{1}\ldots C'_{n-s-1}$ and $\mathfrak{m}'$ an arbitrary specialization of $\mathfrak{m}$ over $k$ over the above specialization. Then, when $Q$ is a component of $rA_{1}*\ldots * rA_{s}$. $T\cdot C'_{1}\ldots C'_{n-s-1}$ with the coefficient $v$, $Q$ appears exactly $v$ times in $\mathfrak{m}'$ by the compatibility of specializations with the operation of inter-section-product (c.f. \cite{art14-key24}). It follows that $\deg(\mathfrak{m})=(mr)^{n-1} I(Y^{(n-1)}, T)\geq \deg \{(rA_{1}*\ldots * rA_{s} * T)C'_{1}\ldots C'_{n-s-1}\}$. (b) follows easily from these.
\end{proof}

As an application of Lemma \ref{art14-lem5}, we shall prove the following proposition which we shall need later.

\begin{proposition}\label{art14-prop5}
Let\pageoriginale $W^{n}$ be a non-singular projective variety and $Y$ a non-degenerate divisor on $W$. Let $m$ be a positive integer such that $l(mY)>Y^{(n)}m^{n-1}+n-1$. Let $f$ be a non-degenerate rational map of $W$ defined by $mY$. Then $\deg (f)\neq 0$, i.e. the image of $W$ by $f$ has dimension $n$.
\end{proposition}

\begin{proof}
By our assumption, $l(mY)>Y^{(n)}m^{s}+s$ for $1\leq s\leq n-1$. Let $U^{s}$ be the image of $W$ by $f$. Then $\Lambda(mY)_{\red}$ consists of $f^{-1}(H)$ where $H$ denotes a hyperplane in the ambient space of $U$. Let $k$ be a common field of rationality of $W$, $Y$ and $f$ and $A_{1},\ldots,A_{s}$ (resp. $B_{1},\ldots,B_{s}$) independent generic divisors of $\Lambda(mY)$ (resp. $\Lambda(mY)_{\red}$) over $k$. As is well known and easy to prove by means of the intersection theory. $\deg(U)\leq I(B_{1}\ldots B_{s}/W,k)$. Moreover, $I(B_{1}\ldots B_{s}/W,k)=I(A_{1}\ldots A_{s}/W,k)$ as can be seen easily. Then we get $\deg (U)\leq m^{s}Y^{(n)}$ by Lemma \ref{art14-lem5}. Let $\Lambda$ be the linear system of hyperplane sections of $U$. We have $\dim \Lambda \leq m^{s}Y^{(n)}+s-1$ (c.f. \cite{art14-key17}). On the other hand, $l(mY)=\dim \Lambda (mY)+1=\dim \Lambda(mY)_{\red}+1=\dim \Lambda +1$. This contradicts our assumption if $s<n$. Our proposition is thereby proved.
\end{proof}

\section{A solution of \texorpdfstring{$(B_{3})$}{B3}, (I)}\label{art14-sec5}

First, we shall fix some notations which shall be used through the rest of this chapter. Let $V^{3}$ be a polarized variety of dimension 3 and $P(m)=\Sigma^{3}_{0}\gamma_{3-i}m^{i}$ the Hilbert characteristic polynomial of $V$. Let $d=X^{(3)}_{V}$. As is well known, $\gamma_{0}=d/3$! As we pointed out in our introduction, {\em we shall solve} $(B_{3})$ {\em under the assumption that $(A''_{3})$ has a solution}. As we showed in Proposition \ref{art14-prop1}, $(A''_{3})$ has a solution when $V$ is canonically polarized and the characteristic is zero. Therefore, {\em we shall assume that there are two constants $c_{0}$ and $c$, which depend on the polynomial $P(x)$ only, such that $|l(mX_{V})-P(m)|<c$ whenever $m\geq c_{0}$}. We shall denote by $\Sigma$ the set of polarized varieties of dimension 3 such that $P(x)$ is their Hilbert characteristic polynomial. We shall use $V$ to denote a ``variable element'' of $\Sigma$.

From our basic assumption, we can find a positive integer $e_{0}\geq c_{0}$, which depends on $P(x)$ only, such that $l(mX_{V})>dm^{2}+2$ for $m\geq e_{0}$. For such $m$ {\em a non-degenerate rational map of $V$ defined by $mX_{V}$\pageoriginale maps $V$ generically onto a variety of dimension $3$} by Proposition \ref{art14-prop5}. Moreover, when that is so, $\Lambda(mX_{V})_{\red}$ is irreducible, i.e. it contains an irreducible member (c.f. \cite{art14-key25}, Chap. IX).

The following lemma has been essentially proved in the course of the proof of Proposition \ref{art14-prop4}.

\begin{lemma}\label{art14-lem6}
Let $r\geq e_{0}$ and $F'=\Sigma^{v}_{1}b_{i}F_{i}$ the fixed part of $\Lambda (rX_{V})$. Then $v\leq r^{3}d$, $\Sigma^{v}_{1}b_{i}<r^{3}d$ and $b_{i}<r^{3}d$.
\end{lemma}

When $r$ is a positive integer, we shall set $P(rx)=P_{r}(x)$ to regard it as a polynomial in $x$. For two positive integers $r$ and $m$, we set $\delta(r,m)=3(\gamma_{0}r^{3})m^{2}-(1/rd)m^{2}+2c+2$; $\delta'(r,m)=2(\gamma_{0}r^{3})m^{2}+2c+2$.

\begin{lemma}\label{art14-lem7}
When a positive integer $r$ is given, it is possible to find a positive integer $\alpha'$, which depends on $P(x)$ and $r$ only, such that $P_{r}(m)-P_{r}(m-1)>\delta(r,m)$ and $>\delta'(r,m)$ whenever $m\geq \alpha'$.
\end{lemma}

\begin{proof}
$P_{r}(m)=\gamma_{0}r^{3}m^{3}+\cdots$ and $P_{r}(m)-P_{r}(m-1)=3(\gamma_{0}r^{3})m^{2}+\cdots$ Therefore such a choice of $\alpha'$ is possible.
\end{proof}

\begin{proposition}\label{art14-prop6}
There are positive integers $v$, $\overline{v}$, which depend on $P(x)$ only, and $e$, $\alpha$, which depend on a member $V$ of $\Sigma$, with the following properties : {\rm(a)} $e_{0}\leq e\leq v$, $\alpha \leq \overline{v}$; {\rm(b)} when $F=\Sigma^{l}_{1}a_{i}F_{i}$ is the fixed part of $\Lambda(e_{0}X_{V})$, there is a positive integer $t$ such that $t\leq 1<e^{3}_{0}d$ and that $F_{1},\ldots,F_{t}$ are not fixed components of $\Lambda(eX_{V})$; {\rm(c)} $P_{e}(m)-P_{e}(m-1)>\delta (e,m)$ and $>\delta'(e,m)$ for $m\geq \alpha$; {\rm(d)} when $F^{*}$ is the fixed part of $\Lambda(eX_{V})$, $l((\alpha-1)eX_{V}+F^{*})=l((\alpha-1)eX_{V})$.
\end{proposition}

\begin{proof}
We can choose $\alpha_{0}$, depending on $P(x)$ only, such that $P_{e_{0}}(m)-P_{e_{0}}(m-1)>\delta (e_{0},m)$ and $>\delta'(e_{0}.m)$ whenever $m\geq \alpha_{0}$ by Lemma \ref{art14-lem7}. Assume that $l((\alpha_{0}-1)e_{0}X_{V}+F)>l((\alpha_{0}-1)e_{0}X_{V})$. Then, after rearranging indices if necessary, we may assume that $F_{1}$ is not a fixed component of the complete linear system determined by $\mu(e_{0}X_{V},\alpha_{0})e_{0}X_{V}$ by Proposition \ref{art14-prop4}, where
$$
\mu (e_{0}X_{V}, \alpha_{0})=(s^{s}\alpha_{0}+s^{s+1}(\alpha_{0}-1)!\text{~~ and~~ } s=e^{3}_{0}d.
$$

Let $e_{1}=e_{0}\mu(e_{0}X_{V},\alpha_{0})$ and $F'$ the fixed part of $\Lambda (e_{1}X_{V})$. By Lemma \ref{art14-lem7} we can find a positive integer $\alpha_{1}$, depending on $P(x)$ only, so that $P_{e_{1}}(m)-P_{e_{1}}(m-1)>\delta (e_{1},m)$ and $>\delta'(e_{1},m)$ whenever $m\geq \alpha_{1}$.\pageoriginale Assume still that $l((\alpha_{1}-1)e_{1}X_{V}+F')>l((\alpha_{1}-1)e_{1}X_{V})$. Then the complete linear system determined by $\mu(e_{1}X_{V},\alpha_{1})e_{1}X_{V}$ has not $F_{2}$ as a fixed component, after rearranging indices if necessary (c.f. Proposition \ref{art14-prop4}).

Since $l<e^{3}_{0}d$ by Lemma \ref{art14-lem6}, this process has to terminate by at most $e^{3}_{0}d-1$ steps. Positive integers $e_{i}$, $\alpha_{i}$ we choose successively can be chosen so that they depend only on $P(x)$. To fix the idea, let us choose $\alpha_{i}$ as follows. Let $r_{i}$ be the largest among the set of roots of the equations $P_{e_{i}}(x)-P_{e_{i}}(x-1)-\delta (e_{i},x)=0$, $P_{e_{i}}(x)-P_{e_{i}}(x-1)-\delta'(e_{i},x)=0$ and let $\alpha_{i}=|r_{i}|+1$. Suppose that our process terminates when we reach to the pair $(e_{t},\alpha_{t})$. Then $t\leq l<e^{3}_{0}d$. From the definition of the function $\mu$ (c.f. Proposition \ref{art14-prop4}) and from our choice of the $\alpha_{i}$, we can find easily upperbounds $v$, $\overline{v}$ for $e_{t}$, $\alpha_{t}$ which depend on $P(x)$ only and not on $t$. Moreover, when $F''$ is the fixed part of $\Lambda(e_{t}X_{V})$, $F_{1},\ldots,F_{t}$ are not components of $F''$ and $l((\alpha_{t}-1)e_{t}X_{V}+F'')=l((\alpha_{t}-1)e_{t}X_{V})$ by our assumption made above. Our proposition follows from these at once.
\end{proof}

\section{A solution of \texorpdfstring{$(B_{3})$}{B3}, (II)}\label{art14-sec6}

In this paragraph, we shall keep the notations of Proposition \ref{art14-prop6}. Let $k$ be an algebraically closed common field of rationality of $V$ and $X_{V}$ and $T$ a generic divisor of $\Lambda(eX_{V})_{\red}$ over $k$. $\Lambda(\alpha eX_{V})$ is canonically associated to the module $L(\alpha eX_{V})$. Let $K$ be a field of rationality of $T$ over $k$ and $L'$ the module of rational functions on $T$ induced by $L(\alpha eX_{V})$. Let $g$ be a non-degenerate rational map of $T$ defined by $L'$, defined over $K$. Let $U$ be the image of $T$ by $g$. A non-degenerate rational map of $V$ defined by $meX_{V}$ maps $V$ generically onto a variety of dimension 3 since $me\geq e\geq e_{0}$ by Proposition \ref{art14-prop5}, whenever $m>0$. Since $T$ is a generic divisor of $\Lambda(eX_{V})_{\red}$ over $k$, it follows that $\dim U=2$. Let $h$ be a non-degenerate rational map of $T$ defined by the module of rational functions on $T$ which is induced by $L(eX_{V})$. Then the image $W$ of $T$ by $h$ has dimension 2 also as we have seen above.

We shall establish some inequalities.
\begin{equation}
\dim T_{r_{T}}\Lambda (\alpha eX_{V})>\frac{1}{2}e^{3}\alpha^{2}d-(1/ed)\alpha^{2}+1.\label{art14-eq6.1}
\end{equation}
In\pageoriginale fact, $\dim Tr_{T}\Lambda(\alpha eX_{V})=l(\alpha eX_{V})-l(\alpha eX_{V}-T)-1=l(\alpha eX_{V})-l((\alpha-1)eX_{V}+F^{*})-1=l(\alpha eX_{V})-l((\alpha-1)eX_{V})-1$ by our choice of $\alpha$ and $e$ (c.f. Proposition \ref{art14-prop6}). By Proposition \ref{art14-prop6}, (c) and by the equality $\gamma_{0}=d/6$, $l(\alpha eX_{V})-l((\alpha-1)eX_{V})>P_{e}(\alpha)-P_{e}(\alpha-1)-2c>\frac{1}{2}e^{3}\alpha^{2}d-(1/ed)\alpha^{2}+2$. Our inequality is thereby proved.

The following formulas are well known and easy to prove by means of the intersection theory (c.f. \cite{art14-key25}, Chap. VII).
\begin{align}
& \deg (g)\cdot \deg (U)=I(A_{1}\cdot A_{2}\cdot T/T,K),\notag\\
& \deg (h)\cdot \deg (W) = I(B_{1}\cdot B_{2}\cdot T/T,K),\label{art14-eq6.2}
\end{align}
where $A_{1}$, $A_{2}$ (resp. $B_{1}$, $B_{2}$) are independent generic divisors of $\Lambda (\alpha eX_{V})$ (resp. $\Lambda (eX_{V})$) over $K$.

Next we shall find an upperbound for $\deg(U$).
\begin{align}
& \deg (U)\leq (1/\deg(g))(e^{3}\alpha^{2}d-e^{2}\alpha^{2})\text{~~ if~~ } F^{*}\neq 0;\label{art14-eq6.3}\\
& \deg (U)\leq (1/\deg (g))e^{3}\alpha^{2}d\text{~~ if~~ } F^{*}=0.\notag
\end{align}
In fact, $A_{i}\sim \alpha eX_{V}$ and $X_{V}$ is non-degenerate. Applying Lemma \ref{art14-lem5} to our case, we get $I(A_{1}\cdot A_{2}\cdot T/T,K)\leq e^{2}\alpha^{2}I(X^{(2)}_{V}, T)$ and $T+F^{*}\sim eX_{V}$. We have $I(X^{(2)}_{V},T)=I(X^{(2)}_{V},eX_{V})-I(X^{(2)}_{V},F^{*})=ed-I(X^{(2)}_{V},F^{*})$. \eqref{art14-eq6.3} follows from these.

Let $\Lambda$ be the linear system of hyperplane sections of $U$. From \eqref{art14-eq6.3} and from Theorem 3 in \cite{art14-key17}, we immediately get
\begin{align}
& \dim \Lambda \leq (1/\deg (g))(e^{3}\alpha^{2}d-e^{2}\alpha^{2})+1\text{~~ if~~ }F^{*}\neq 0;\notag\\
& \dim \Lambda \leq (1/\deg(g))e^{3}\alpha^{2}d+1\text{~~ if~~ } F^{*}=0.\label{art14-eq6.4}
\end{align}

In order to estimate an upper bound for $\deg(g)$, it is enough to do so for $\deg(h)$. By doing so, we shall get
\begin{equation}
\deg (g)\leq e^{3}d.\label{art14-eq6.5}
\end{equation}
From \eqref{art14-eq6.2} we get $\deg(h)\leq I(B_{1}\cdot B_{2}\cdot T/T,K)$. By Lemma \ref{art14-lem5}, the latter is bounded by $e^{2}I(X^{(2)}_{V},T)$. This, in turn, is bounded by $e^{2}I(X^{(2)}_{V},T+F^{*})=e^{3}d$. This proves our inequality since $\deg(g)\leq \deg(h)$.

Combining \eqref{art14-eq6.4} and \eqref{art14-eq6.5}, we get
\begin{equation}
\dim \Lambda \leq (1/\deg (g))e^{3}\alpha^{2}d-(1/ed)\alpha^{2}+1\text{~~ if~~ } F^{*}\neq 0. \label{art14-eq6.6}
\end{equation}

As\pageoriginale is well known, $\dim Tr_{T}\Lambda(\alpha eX)=\dim \Lambda$. Consider now the case $F^{*}\neq 0$ first. From \eqref{art14-eq6.1} and \eqref{art14-eq6.6}, we get $(1/\deg(g))e^{3}\alpha^{2}d-(1/ed)\alpha^{2}+1>\frac{1}{2}e^{3}\alpha^{2}d-(1/ed)\alpha^{2}+1$. Hence $\deg(g)<2$ and conseqently $\deg(g)=1$  and $g$ is birational. From the definition of $g$, it follows that a non-degenerate rational map $f$ of $V$ defined by $\alpha eX_{V}$ induces on $T$ a birational map. When that is so, $f$ is a birational map since $T$ is a generic divisor of the linear system $\Lambda(eX_{V})_{\red}$ of positive dimension over $k$, a proof of which will be left as an exercise to the reader.

Next, consider the case when $F^{*}=0$. By Proposition \ref{art14-prop6}, we have $P_{e}(\alpha)-P_{e}(\alpha-1)>\delta'(e,\alpha)$. Then we get $\dim Tr_{T}\Lambda (\alpha eX_{V})>(1/3)e^{3}\alpha^{2}d+1$, a proof of which is quite similar to that of \eqref{art14-eq6.1}. Combining this with \eqref{art14-eq6.4}, we get $(1/\deg (g))e^{3}\alpha^{2}d>(1/3)e^{3}\alpha^{2}d$. Hence $\deg(g)<3$. This proves that $\deg(f)\leq 2$, a proof of which will be left to the reader again.

By Proposition \ref{art14-prop6}, $e$ is bounded by $v$ and $\alpha$ is bounded by $\overline{v}$. Let $\rho_{1}=(v \ \overline{v})$! We shall summarize the results of this paragraph as follows.

\begin{proposition}\label{art14-prop7}
There is a constant $\rho_{1}$, which depends on $P(x)$ only, such that the following properties hold: {\rm(a)} if $\Lambda(eX_{V})$ has a fixed component, $\rho_{1}X_{V}$ defines a birational transformation of $V$; {\rm(b)} if $\Lambda(eX_{V})$ has no fixed component, a non-degenerate rational map of $V$ defined by $\rho_{1}X_{V}$ is of degree at most $2$; {\rm(c)} $\rho_{1}$ is divisible by $e$ and $\alpha$.
\end{proposition}

\section{A solution of \texorpdfstring{$(B_{3})$}{B3}, (III)}\label{art14-sec7}

Proposition \ref{art14-prop7} solves our problem in the case when $\Lambda(eX_{V})$ has a fixed component. In order to treat the other case, we shall review the concept of minimum sums of linear systems and fix some notations.

In general, let $W$ be a variety and $\mathscr{M}$, $\mathscr{N}$ two modules of rational functions of finite dimensions on $W$. We shall denote by $\Lambda(\mathscr{M})$, $\Lambda(\mathscr{N})$ the reduced linear systems on $W$ defined by these two modules. Let $\mathscr{R}$ be the module generated by $f\cdot g$ with $f\in \mathscr{M}$ and $g\in \mathscr{N}$. Then the reduced linear system $\Lambda(\mathscr{R})$ is known as the {\em minimum sum} of $\Lambda(\mathscr{M})$ and $\Lambda(\mathscr{N})$. We shall denote this minimum sum by $\Lambda(\mathscr{M})\oplus \Lambda(\mathscr{N})$. When\pageoriginale $\Lambda$ is a reduced linear system on $W$ and $\Lambda'$ the minimum sum of $r$ linear systems equal to $\Lambda$, we shall write $\oplus^{r}\Lambda$ for $\Lambda'$.

Let $W'$ be another variety and $\beta$ a rational map of $W'$ into $W$. Let $\Lambda$ be the closure of the graph of $\beta$ on $W'\times W$ and $\Lambda$ a linear system of divisors on $W$. Let $\Lambda'$ be the set of $W'$-divisors $\beta^{-1}(Z)=\pr_{W'}(\Gamma\cdot (W'\times Z))$ with $Z\in \Lambda$. Then $\Lambda'$ is a linear system of $W'$-divisors and this will be denoted by $\beta^{-1}(\Lambda)$.

\begin{lemma}\label{art14-lem8}
Let $f$ be a non-degenerate rational map of $V$ defined by $\rho_{1}X_{V}$ and $W$ the image of $V$ by $f$. When $N$ is the dimension of the ambient space of $W$, the following inequalities hold:
$$
P(\rho_{1})-c-1\leq N\leq P(\rho_{1})+c-1; \ \ \deg(W)\leq \rho^{3}_{1}d.
$$
\end{lemma}

\begin{proof}
$N=l(\rho_{1}X_{V})-1$. Hence the first inequality follows from $(A''_{3})$. Let $k$ be a common field of rationality of $V$, $X_{V}$ and $f$ and $Z_{1}$, $Z_{2}$, $Z_{3}$ independent generic divisors of $\Lambda(\rho_{1}X_{V})$ over $k$. Then $\deg(W)\leq I(Z_{1}\cdot Z_{2}\cdot Z_{3}/V,k)\leq \rho^{3}_{1}d$ by Lemma \ref{art14-lem5}. Our lemma is thereby proved.
\end{proof}

\begin{coro*}
Let $k_{0}$ be the algebraic closure of the prime field. There is a finite union of irreducible algebraic families of irreducible varieties in projective spaces, all defined over $k_{0}$, such that, when $V\in \Sigma$ and when $f$, $W$ are as in our lemma, $W$ is a member of at least one of the irreducible families.
\end{coro*}

\begin{proof}
This follows at once from our lemma and from the main theorem on Chow-forms (c.f. \cite{art14-key3}).
\end{proof}

In the following lemma, {\em we shall assume that global resolution, dominance and birational resolution in the sense of Abhyankar hold for algebraic varieties of dimension $n$} (c.f. \cite{art14-key35}). These hold for characteristic zero (c.f. \cite{art14-key5}) and for algebraic varieties of dimension 3 when the characteristic is different from 2, 3 and 5 (c.f. \cite{art14-key35}).

\begin{lemma}\label{art14-lem9}
Let $\mathfrak{F}$ be an irreducible algebraic family of irreducible varieties in a projective space and $k$ an algebraically closed field over which $\mathfrak{F}$ is defined. Then $\mathfrak{F}$ can be written as a finite union $\bigcup_{j}\mathfrak{F}_{j}$ of irreducible algebraic families, all defined over $k$, with the following properties:\pageoriginale {\rm(a)} $\mathfrak{F}_{i}\cap \mathfrak{F}_{j}=\emptyset$ whenever $i\neq j$; {\rm(b)} for each $j$, there is an irreducible algebraic family $\mathfrak{H}_{j}$ of non-singular varieties in a projective space, defined over $k$; {\rm(c)} when $W_{i}$ is a generic member of $\mathfrak{F}_{i}$ over $k$, there is a generic member $W^{*}_{i}$ of $\mathfrak{H}_{i}$ over $k$ and a birational morphism $\phi_{i}$ of $W^{*}_{i}$ on $W_{i}$ such that $W^{*}_{i}$, $\phi_{i}$ are defined over an algebraic extension of the smallest field of definition of $W_{i}$ over $k$; {\rm(d)} when $W'_{i}$ is a member of $\mathfrak{F}_{i}$, $\Gamma_{i}$ the graph of $\phi_{i}$ and when $(\Gamma'_{i},W^{*'}_{i})$ is an arbitrary specialization of $(\Gamma_{i},W^{*}_{i})$ over $k$ over the specialization $W_{i}\to W'_{i}$ ref. $k$, $W^{*'}_{i}$ is a member of $\mathfrak{H}_{i}$ and $\Gamma'_{i}$ is the graph of a birational morphism $\phi'_{i}$ of $W^{*'}_{i}$ on $W_{i}$; {\rm(e)} when $C'_{i}$ is a generic hyperplane section of $W'_{i}$ over a filed of definition of $W'_{i}$, ${\phi'}^{-1}_{i}(C'_{i})$ is non-singular.
\end{lemma}

\begin{proof}
Let $W$ be a generic member of $\mathfrak{F}$ over $k$ and $K$ the smallest field of definition of $W$ over $k$. There is a non-singular variety $W^{*}$ in a projective space and a birational morphism $\phi$ of $W^{*}$ on $W$, both defined over $\overline{K}$. For the sake of simplicity, replace $W^{*}$ by the graph of $\phi$. Then $\phi$ is simply the projection map. Therefore, we can identify the graph of $\phi$ with $W^{*}$. A multiple projective space cna be identified with a non-singular subvariety of a projective space by the standard process. Chow-points of positive cycles in a multiple projective space can then be defined by means of the above process. Let $C$ be a hyperplane section of $W$, rational over $K$. $W^{*}$ can be chosen in such a way that $\phi^{-1}(C)=U$ is non-singular. Let $w$, $w^{*}$, $u$ be respectively the Chow-points of $W$, $W^{*}$, $U$ and $\overline{T}$, $\overline{T}^{*}$ the locus of $w$, $(w^{*},u)$ over $k$. An open subset $T$ of $\overline{T}$ over $k$ is the Chow-variety of $\mathfrak{F}$. The set $T^{*}$ of points on $\overline{T}^{*}$ corresponding to pairs of non-singular varieties is $k$-open on $\overline{T}^{*}$ as can be verified without much difficulty. Let $Z$ be the locus of $(w,w^{*},u)$ over $k$ on $T\times T^{*}$. Let $T_{1}$ be the set of points of $T$ over which $Z$ is complete (i.e. proper). Then $T_{1}$ is non-empty and $k$-open (c.f. \cite{art14-key25}, Chap. VII, Cor., Prop. 12). The set-theoretic projection of the restriction $Z_{1}$ of $Z$ on $T_{1}\times T^{*}$ on $T_{1}$ contains a non-empty $k$-open set $F_{1}$. Let $H_{1}$ be the locus of $w^{*}$ over $k$. Then the families $\mathfrak{F}_{1}$, $\mathfrak{H}_{1}$ defined by $F_{1}$, $H_{1}$ satisfy (b), (c), (d) and (e) of our lemma which is not difficult to verify. $T_{1}-F_{1}$ can be written as a finite union of locally closed irreducible\pageoriginale subvarieties of $T$, defined over $k$, such that no two distinct components have a point in common. Then we repeat the above for each irreducible component to obtain the lemma.
\end{proof}

\begin{coro*}
There are two finite sets of irreducible algebraic families $\{\mathfrak{F}_{i}\}$, $\{\mathfrak{H}_{i}\}$ with the following properties : {\rm(a)} when $V\in \Sigma$ and $f$ a non-degenerate rational map of $V$ defined by $\rho_{1}X_{V}$, there is an index $i$ such that the image $W$ of $V$ by $f$ is a member of $\mathfrak{F}_{i}$; {\rm(b)} every member of the $\mathfrak{H}_{i}$ is a non-singular subvariety of a projective space; {\rm(c)} the $\mathfrak{F}_{i}$ and the $\mathfrak{H}_{i}$ satisfy {\rm(c), (d)} and {\em(e)} of our lemma.
\end{coro*}

\begin{proof}
This follows from the Corollary of Lemma \ref{art14-lem7}, Lemma \ref{art14-lem8} and from the basic assumptions on resolutions of singularities.
\end{proof}

\begin{lemma}\label{art14-lem10}
Let the characteristic be zero, $V\in\Sigma$ and $f$ a non-degene\-rate rational map of $V$ defined by $\rho_{1}X_{V}$. Let $W$ be the image of $V$ by $f$. Then there is a non-singular projective variety $W^{*}$ and a birational morphism $\phi$ of $W^{*}$ on $W$ with the following properties: {\rm(a)} when $k$ is a common field of rationality of $V$, $X_{V}$, $f$ and $\phi$ and when $C_{W}$ is a generic hyperplane section of $W$ over $k$, $\phi^{-1}(C_{W})$ is non-singular; {\rm(b)} when $C_{W}$ and $C'_{W}$ are independent generic over $k$, $\phi^{-1}(C_{W})\cdot \phi^{-1}(C'_{W})$ is non-singular; {\rm(c)} there is a positive integer $\pi$, which depends only on $P(x)$, such that $|p_{a}(\phi^{-1}(C_{W}))|<\pi$.
\end{lemma}

\begin{proof}
When a non-singular subvariety of a projective space is specialized to another such variety over a discrete valuation ring, the virtual arithmetic genus is not changed (c.f. \cite{art14-key2}, \cite{art14-key4}). Take the $\mathfrak{F}_{i}$, $\mathfrak{H}_{i}$ as in the Corollary of Lemma \ref{art14-lem9} and take $W^{*}$ from a suitable $\mathfrak{H}_{i}$. Then there is a birational morphism $\phi$ of $W^{*}$ on $W$, satisfying (a). (c) follows from (a) when we take the above remark into account. (a) and (b) follow easily also from the theorem of Bertini on variable singularities since the characteristic is zero.
\end{proof}

\begin{lemma}\label{art14-lem11}
There is a constant $\rho_{2}$, which depends on $P(x)$ only, such that $m\rho_{1}X_{V}$ has the following properties for $m\geq \rho_{2}$, provided that it does not define a birational map and the characteristic is zero: {\rm(a)} when $f'$ is a non-degenerate rational map of $V$ defined by $m\rho_{1}X_{V}$, $k'$ a field of rationality of $V$ and $X_{V}$ and $T$ a generic divisor of\pageoriginale $\oplus^{m}\Lambda(\rho_{1}X_{V})$ over $k'$, $T$ is irreducible and the effective geometric genus of the proper transform of $T$ by $f'$ is at least $2$; {\rm(b)} $\deg (f')=2$ and $f'$ induces on $T$ a rational map of degree $2$.
\end{lemma}

\begin{proof}
Let $f$, $W$, $W^{*}$, $\phi$, $k$ be as in Lemma \ref{art14-lem10}. Let $C_{W}$ be generic over $k$ and $U=\phi^{-1}(C_{W})$. Let $U'$ be a generic specialization of $U$ over $k$, other than $U$. By the modular property of $p_{a}$\footnote{This can be proved exactly in the same way as Lemma 5 of \cite{art14-key31} because of our Lemma \ref{art14-lem10}.}, we get $p_{a}(mU)=mp_{a}(U)+\Sigma^{m-1}_{1}p_{a}(sU'\cdot U)$. Applying the modular property again to $p_{a}(sU'\cdot U)$ on $U$, which is non-singular by Lemma \ref{art14-lem10} we get the following equality:
$$
sp_{a}(U'\cdot U)+\frac{1}{2}s(s-1)(U'\cdot U)^{(2)}-s-1=p_{a}(sU'\cdot U).
$$
From the definition of $U$, it is clear that $U^{(3)}>0$ on $W^{*}$. Hence $(U'\cdot U)^{(2)}=U^{(3)}>0$. Moreover $|p_{a}(U)|<\pi$, by Lemma \ref{art14-lem10}. Using these and $\Sigma^{m-1}_{1}s(s-1)=(m-1)m(2m-1)/6$, we get
$$
p_{a}(mU)>-m(\pi+1)-m(m-1)+(m-1)m(2m-1)/12+1.
$$
We can find a positive integer $\rho_{2}$, which depends on $P(x)$ only, such that the right hand side of the above inequality is at least 2 whenever $m\geq \rho_{2}$. When that is so, any member $A$ of $\Lambda(mU)$ satisfies $p_{a}(A)>1$ since the virtual arithmetic genus of divisors is invariant with respect to linear equivalence.

Let $\Lambda$ be the linear system of hyperplane sections of $W$. Clearly $\phi^{-1}(\oplus^{m}\Lambda)=\phi^{m}\phi^{-1}(\Lambda)$ and the latter contains a non-singular member $A$ by the theorem of Bertini on variable singularities. Let $p_{g}$, $p_{a}$, $q$ denote respectively the effective geometric genus, effective arithmetic genus and the irregularity of $A$. Then $q=p_{g}-p_{a}$ and $p_{a}(A)=p_{a}$. Since $q\geq 0$, it follows that $p_{g}>1$ whenever $m\geq \rho_{2}$. When $C_{m}$ is a generic divisor of $\oplus^{m}\Lambda$ over $k$, we can take for $A$ the variety $\phi^{-1}(C_{m})$. Therefore, the effective geometric genus of $C_{m}$ is at least 2 when $m\geq \rho_{2}$.

Assume that $f'$ is not birational for some $m\geq \rho_{2}$ and rational over $k$. Then $\Lambda(m\rho_{1}X_{V})$ has no fixed component and $\deg(f')=2$ by Proposition \ref{art14-prop7}. By the same proposition, the same is true for $\Lambda(\rho_{1}X_{V})$\pageoriginale and $f$. Let $W'$ be the image of $V$ by $f'$. Since $\deg(f)=\deg(f')$, there is a birational transformation $h$ between $W'$ and $W$ such that $f=h\circ f'$ holds generically. Then $f^{-1}(C_{m})=T$ is irreducible (c.f. \cite{art14-key25}, Chap. IX) and is a generic divisor of $\oplus^{m}f^{-1}(\Lambda)=\oplus^{m}\Lambda(\rho_{1}X_{V})$ over $k$. Let $L$ be the proper transform of $T$ by $f'$. $C_{m}$ and $L$ are birationally corresponding subvarieties of $W$ and $W'$ by $h^{-1}$ and, when that is so, the effective geometric genus of $L$ is at least $2$. Our lemma follows easily from this.

Let now $f'$ denote a non-degenerate rational map of $V$ defined by $\rho_{2}\rho_{1}X_{V}$, and {\em assume that $f'$ is not birational}. By Proposition \ref{art14-prop7}, $\deg\break (f')=2$ and $\Lambda(\rho_{2}\rho_{1}X_{V})$ has no fixed component. By Lemma \ref{art14-lem11}, the complete linear system contains a linear pencil whose generic divisor $T$ has the property that its proper transform $D$ by $f'$ has the effective geometric genus which is at least 2.

Let $f$ be a non-degenerate rational map of $V$ into a projective space defined by $m\rho_{2}\rho_{1}X_{V}$ and {\em assume that $f$ has still the property that $\deg\break (f)=2$}. Let $E$ be the proper transform of $T$ by $f$ and $g$ the rational map induced on $T$ by $f$. Then $D$ and $E$ are clearly birationally equivalent and the effective geometric genus of $E$ is at least $2$. As in \eqref{art14-eq6.1}, $\dim Tr_{T}\Lambda(m\rho_{2}\rho_{1}X_{V})=l(mT)-l(mT-T)-1>P_{\rho_{1}\rho_{2}}(m)-P_{\rho_{1}\rho_{2}}(m-1)-2c-1$. The leading coefficient of the right hand side of the above inequality is given by $\frac{1}{2}(\rho_{1}\rho_{2})^{3}d$.
\end{proof}

Let $K$ be the smallest field of rationality of $T$ over $k$ and $Z_{1}$, $Z_{2}$ two independent generic divisors of $\Lambda(mT)$ over $K$. Then exactly as in \eqref{art14-eq6.2}, we get $\deg(g)\deg(E)=I(Z_{1}\cdot Z_{2}\cdot T/T,K)$. By Lemma \ref{art14-lem5}, the latter is bounded by $(\rho_{1}\rho_{2})^{3}m^{2}d$. By Lemma \ref{art14-lem11} and by our assumption, $\deg(g)=2$. Hence $\deg(E)\leq \frac{1}{2}(\rho_{1}\rho_{2})^{3}m^{2}d$. Let $\Lambda$ be the linear system of hyperplane sections of $E$. By Proposition \ref{art14-prop3}, $\dim \Lambda\leq \frac{1}{4}(\rho_{1}\rho_{2})^{3}m^{2}d+1$. Since $g$ is defined by $Tr_{T}\Lambda (mT)$, it follows that $\dim Tr_{T}\Lambda(mT)=\dim \Lambda$. Therefore,
$$
P_{\rho_{1}\rho_{2}}(m)-P_{\rho_{1}\rho_{2}}(m-1)-2c-1<\frac{1}{4}(\rho_{1}\rho_{2})^{3}m^{2}d+1.
$$
Since the leading coefficient of the left hand side is $\frac{1}{2}(\rho_{1}\rho_{2})^{3}d$, we can find a constant $\rho_{3}$, which depends on $P(x)$ only, such that the above inequality does not hold for $m\geq \rho_{3}$. For such $m$, $g$ and hence $f$ has to\pageoriginale be birational. Setting $\frac{1}{2}\rho_{4}=\rho_{1}\rho_{2}\rho_{3}$ and combining the above result with that of Proposition \ref{art14-prop7}, we get

\begin{theorem}\label{art14-thm1}
Let the characteristic be zero, $V^{3}$ a polarized variety, $P(m)\break =\chi(V,\mathscr{L}(mX_{V}))$ and assume that $(A''_{3})$ is true. Then there is a constant $\rho_{4}$, which depends on $P(x)$ only, such that $mX_{V}$ defines a birational transformation of $V$ when $m\geq \rho_{4}$.
\end{theorem}

\begin{coro*}
Let the characteristic be zero and $V^{3}$ be canonically polarized. Then $(B_{3})$ is true.
\end{coro*}

\bigskip

\begin{center}
{\Large\bf Chapter II. The Problem \boldmath$(C_{n})$.}\labeltext{2}{art14-chap2}
\end{center}

In this chapter, we shall solve $(C_{n})$ for canonically polarized varieties $V^{n}$ under the following assumptions: {\em $(A_{n})$ and $(B_{n})$ are true; theorems on dominance and birational resolution in the sense of Abhyankar hold for dimension $n$.} As we remarked already, this is the case when the characteristic is zero (c.f. \cite{art14-key5}) or when $n=1,2,3$ if the characteristics 2, 3 and 5 are excluded for $n=3$ (c.f. \cite{art14-key35}).

\section{Preliminary lemmas}\label{art14-sec8}

\begin{lemma}\label{art14-lem12}
Let $U$ and $U'$ be two non-singular subvarieties of projective spaces and $g$ a birational transformation between $U$ and $U'$. Then we have the following results: {\rm(a)} $g(\mathfrak{K}(U))+E'\sim \mathfrak{K}(U')$ where $E'$ is a positive $U'$-divisor whose components are exceptional divisors for $g^{-1}$; {\rm(b)} $l(m\mathfrak{K}(U))=l(m\mathfrak{K}(U'))$ for all positive integers $m$; {\rm(c)} $\Lambda(m\mathfrak{K}(U'))=\Lambda(g(m\mathfrak{K}(U)))+mE'$ for all positive integers $m$.\footnote{A subvariety of codimension 1 of $U'$ is called exceptional for $g^{-1}$ if the proper transform of it by $g^{-1}$ is a subvariety of $U$ of codimension at least 2.} 
\end{lemma}

\begin{proof}
These results are well known for characteristic zero. (b) and (c) are easy consequences of (a). (a) can be proved as in \cite{art14-key33} using fundamental results on monoidal transformations (c.f. \cite{art14-key29}, \cite{art14-key33}) and the theorem of dominance.
\end{proof}

\begin{lemma}\label{art14-lem13}
Let $U$ be a non-singular subvariety of a projective space such that $C_{U}\sim m\mathfrak{K}(U)$ for some positive integer $m$. Let $U'$ be a non-singular\pageoriginale subvariety of a projective space, birationally equivalent to $U$. Then $m\mathfrak{R}(U')$ defines a non-degenerate birational map $h'$ of $U'$, mapping $U'$ generically onto a non-singular subvariety $U^{*}$ of a projective space such that $C_{U^{*}}\sim m\mathfrak{K}(U^{*})$. Moreover, $U$ and $U^{*}$ are isomorphic.
\end{lemma}

\begin{proof}
Let $g$ be a birational transformation between $U$ and $U'$. Then $g(C_{U})\sim g(m\mathfrak{K}(U))$ and $\Lambda(g(\mathfrak{K}(U)))+mE'=\Lambda(m\mathfrak{K}(U'))$ where $E'$ is a positive $U'$-divisor whose components are exceptional divisors for $g^{-1}$ by Lemma \ref{art14-lem12}. Assume first that the set of hyperplane sections of $U$ forms a complete linear system. Then $g(C_{U})$ is irreducible for general $C_{U}$ (c.f. \cite{art14-key25}, Chap. IX). Hence $mE'$ is the fixed part of $\Lambda(m\mathfrak{K}(U'))$. Since $l(m\mathfrak{K}(U))=l(m\mathfrak{K}(U'))$ by Lemma \ref{art14-lem12}, it follows that all members of $\Lambda(g(m\mathfrak{K}(U)))$ are of the form $g(C_{U})$. This proves that $g^{-1}$ is a non-degenerate rational map defined by $m\mathfrak{K}(U')$. If our assumption does not hold for $U$, apply a non-degenerate map of $U$, defined by $C_{U}$, to $U$. This amp is obviously an isomorphism and the image of $U$ by this clearly satisfies our assumption.
\end{proof}

\begin{lemma}\label{art14-lem14}
Let $U^{n}$ (resp. ${U'}^{n}$) be a complete non-singular variety, $\mathfrak{K}(U)$ (resp. $\mathfrak{K}(U')$) a canonical divisor of $U$ (resp. $U'$) and $\mathfrak{O}$ a discrete valuation ring with the quotient field $k_{0}$ and the residue field $k'_{0}$. Assume that $U$, $\mathfrak{K}(U)$ are rational over $k_{0}$ and that $(U',\mathfrak{K}(U'))$ is a specialization of $(U,\mathfrak{K}(U))$ over $\mathfrak{O}$. Assume further that the following conditions are satisfied: {\rm(i)} there is a positive integer $m_{0}$ such that $l(m_{0}\mathfrak{K}(U))=l(m_{0}\mathfrak{K}(U'))$; {\rm(ii)} a non-degenerate rational map $h$ (resp. $h'$) defined by $m_{0}\mathfrak{K}(U)$ (resp. $m_{0}\mathfrak{K}(U')$) is birational; {\rm(iii)} $h'(U')=W'$ is non-singular and $C_{W'}\sim m_{0}\mathfrak{K}(W')$. Then the following two statements are equivalent: {\rm(a)} There is a birational map $g$ of $U$, between $U$ and a non-singular subvariety $W^{*}$ of a projective space such that $C_{W^{*}}\sim t\mathfrak{K}(W^{*})$ for some positive integer $t$ and $\mathfrak{K}(W')^{(n)}=\mathfrak{K}(W^{*})^{(n)}$; {\rm(b)} $\deg(h(U))=\deg (h'(U'))$. Moreover, when {\rm(a)} or {\rm(b)} is satisfied, $h(U)=W$ is non-singular, $C_{W}\sim m_{0}\mathfrak{K}(W)$ and $\mathfrak{K}(W)^{(n)}=\mathfrak{K}(W')^{(n)}$.
\end{lemma}

\begin{proof}
First assume (a). $h$ is uniquely determined by $m_{0}\mathfrak{K}(U)$ up to a projective transformation. Therefore we get $\deg(h(U))\geq \deg (h'(U'))$\pageoriginale by Proposition \ref{art14-app-prop2.1} of the Appendix since specializations are compatible with the operation of algebraic projection (c.f. (\cite{art14-key24}). Let the $Z_{i}$ be $n$ independent generic divisors of $\Lambda(m_{0}\mathfrak{K}(U))$ over $k_{0}$. Then $\deg(h(U))=I(Z_{1}\ldots Z_{n}/U,k_{0})$ since $h$ is birational. Let $d_{0}=\mathfrak{K}(W')^{(n)}=\mathfrak{K}(W^{*})^{(n)}$. Then $\deg(h'(U'))=m_{0}{}^{n}d_{0}$ by (iii). Hence $\deg (h(U))\geq m_{0}{}^{n}d_{0}$. By Lemma \ref{art14-lem11}, $\Lambda (g(m\mathfrak{K}(U)))+mE^{*}=\Lambda(m\mathfrak{K}(W^{*}))$ for all positive $m$ where $E^{*}$ is as described in the lemma. Let $L$ be a common field of rationality of $W^{*}$ and $g$ over $k_{0}$ and the $Y_{i}$ (resp. $Y^{*}_{i}$) $n$ independent generic divisors of $\Lambda(m\mathfrak{K}(U))$ (resp. $\Lambda(m\mathfrak{K}(W^{*}))$) over $L$. Then we have $I(Y_{1}\ldots Y_{n}/U,L)=I(Y^{*}_{1}\ldots Y^{*}_{n}/W^{*},L)$. By Lemma \ref{art14-lem5}, 
$$
I(Y^{*}_{1}\ldots Y^{*}_{n}/W^{*},L)\leq m^{n}d_{0}.
$$ 
Setting $m=m_{0}$, we therefore get $I(Y_{1}\ldots Y_{n}/U,L)\leq m_{0}{}^{n}d_{0}$. The left hand side of this is obviously $I(Z_{1}\ldots Z_{n}/U,k_{0})$. Combining the two inequalities we obtained, we get $\deg(h(U))=m_{0}{}^{n}d_{0}=\deg(h'(U'))$. Hence (a) implies (b).

Now we assume (b). Let $W=h(U)$, $C=C_{W}$, $C'=C_{W'}$. By Proposition \ref{art14-app-prop2.1} of the Appendix and by the compatibility of specializations with the operation of algebraic projection, we get $(U,\mathfrak{K}(U), W)\to (U',\mathfrak{K}(U'),W')$ ref. $\mathfrak{O}$. Since $W'$ is non-singular, $W$ is non-singular too. Since $h$ is defined by $m_{0}\mathfrak{K}(U)$, there is a positive $U$-divisor $F$ such that $h^{-1}(C)+F\sim m_{0}\mathfrak{K}(U)$. Hence there is a positive divisor $T$ with $h(m_{0}\mathfrak{K}(U))\sim C+T$. There is a positive divisor $E$ such that $C+T+E\sim m_{0}\mathfrak{K}(W)$ by Lemma \ref{art14-lem11}. Let $C''+T'+E'$ be a specialization of $C+T+E$ over $\mathfrak{O}$ over the specialization under consideration. Since linear equivalence is preserved by specializations (c.f. \cite{art14-key24}), $C'\sim C''$ and $C'+T'+E'\sim m_{0}\mathfrak{K}(W')$ (c.f. Lemma \ref{art14-app-lem1.1} of the Appendix; $U$, $U'$ are clearly non-ruled since $l(m\mathfrak{K}(U))$, $l(m\mathfrak{K}(U'))$ are positive for large $m$). Since $m_{0}\mathfrak{K}(W')\sim C'$ by (iii), it follows that $T'$, $E'$ are positive and $T'+E'\sim 0$. This proves that $T=E=0$ and $m_{0}\mathfrak{K}(W)\sim C$. Our lemma is thereby proved.
\end{proof}

\section{A proof of \texorpdfstring{$(C_{n})$}{Cn}}\label{art14-sec9}

In order to solve $(C_{n})$, we shall fix some notation. We shall denote by $\Sigma$ the {\em set of canonically polarized varieties with the fixed Hilbert characteristic polynomial} $P(x)$ and by $V^{n}$ a ``variable element'' of $\Sigma$. As we have shown in Lemma \ref{art14-lem1}, there is a root $\rho$ of the\pageoriginale equation $P(x)-(-1)^{n}\gamma_{n}=0$ such that $\mathfrak{K}(V)\equiv \rho X_{V}\mod G_{a}$. Then $\Sigma$ can be expressed as a union of subspaces $\Sigma_{\rho}$ corresponding to $\rho$. In order to solve $(C_{n})$, we may restrict our attention to $\Sigma_{\rho}$. From our basic assumptions stated at the beginning of this chapter, there is a constant $\rho_{5}$, which depends on $P(x)$ only, having the following properties: (a) higher cohomology groups of $\mathscr{L}(Y)$ vanish and $l(Y)>0$ whenever $Y\equiv mX_{V}\mod \mathfrak{G}_{a}$ and $m\geq \rho_{5}$; (b) such $Y$ defines a birational transformation of $V$. When $V\in \Sigma_{\rho}$ and when $X_{V}$ is replaced by $\mathfrak{K}(V)$, (a) and (b) still hold since $\mathfrak{K}(V)\equiv\rho X_{V}\mod G_{a}$ and $\rho$ is a positive integer by the definition of a basic polar divisor. From now on, we shall restrict ourselves to the study of $\Sigma_{\rho}$ and $V$ shall denote a ``variable element'' of this set. We set $d_{0}=\mathfrak{K}(V)^{(n)}$ and $\rho_{6}=\rho\cdot \rho_{5}$.

Let $f$ be a non-degenerate birational map of $V$ defined by $\rho_{5}\mathfrak{K}(V)$. $f$ maps $V$ into the projective space of dimension $P(\rho_{6})-1$ and the degree of the image is bounded by $\rho_{6}{}^{n}X_{V}^{(n)}=\rho_{5}{}^{n}d_{0}$, which follows easily from Lemma \ref{art14-lem5}. When we do this for each member of $\Sigma_{\rho}$, we see that each such image is contained in a finite union $\mathfrak{F}$ of irreducible algebraic families of irreducible varieties by the main theorem on Chow-forms (c.f. \cite{art14-key3}). Let $\mathfrak{A}'$ be the set of images of members of $\Sigma_{\rho}$ in $\mathfrak{F}$ thus obtained. Applying Lemma \ref{art14-lem8} to $\mathfrak{F}$, we get immediately the following results.

\begin{lemma}\label{art14-lem15}
There are finite unions $\bigcup_{i}\mathfrak{F}_{i}$ and $\bigcup_{i}\mathfrak{H}_{i}$ of irreducible algebraic families of irreducible varieties in projective spaces with the following properties: {\rm(a)} each $\mathfrak{F}_{i}$ contains some members of $\mathfrak{A}'$ and $\mathfrak{A}'$ is contained in $\bigcup_{i}\mathfrak{F}_{i}$; {\rm(b)} $\bigcup_{i}\mathfrak{H}_{i}$ consists of non-singular varieties; {\rm(c)} when $W$ is a member of $\mathfrak{F}_{i}$, there is a member $U$ of $\mathfrak{H}_{i}$ and a birational morphism of $U$ on $W$.
\end{lemma}

Let $\mathfrak{H}$ be a finite union of irreducible algebraic families of positive cycles in projective spaces and $\mathfrak{u}$ a set of members of $\mathfrak{H}$. We shall say that $\mathfrak{H}$ is $\mathfrak{u}$-{\em admissible} if each component family of $\mathfrak{H}$ contains some members of $\mathfrak{u}$. We shall denote by $\mathfrak{A}$ the set of members $U$ of $\bigcup_{i}\mathfrak{H}_{i}$ such that there is a birational morphism of $U$ on a member of $\mathfrak{A}'$.\pageoriginale Then $\bigcup_{i}\mathfrak{H}_{i}$ is $\mathfrak{A}$-admissible by Lemma \ref{art14-lem15}. Elements $U$ of $\mathfrak{A}$ satisfy the following three conditions (c.f. Lemma \ref{art14-lem13}):
\begin{itemize}
\item[(I)] $m\mathfrak{K}(U)$ defines a birational transformation $f_{m}$ of $U$ for large $m$; 

\item[(II)] $f_{m}(U)$ is non-singular for large $m$;

\item[(III)] $l((\rho_{5}+m)\mathfrak{K}(U))=P_{\rho}(\rho_{5}+m)$ for $m>0$.
\end{itemize}
{\em We shall find a subset of $\bigcup_{i}\mathfrak{H}_{i}$, containing $\mathfrak{A}$, satisfying the above three conditions, which can be expressed as a finite union of irreducible algebraic families of non-singular varieites. From this we shall recover members of $\Sigma_{\rho}$ in some definite projective space, up to isomorphisms, in such a way that hyperplane sections are some fixed multiple of canonical divisors.} This is the main idea of the rest of this paragraph.

\begin{lemma}\label{art14-lem16}
There is a finite union $\bigcup_{i}\mathfrak{J}_{j}$ of irreducible algebraic families contained in $\bigcup_{i}\mathfrak{H}_{i}$ having the following properties: {\rm(a)} $\bigcup_{j}\mathfrak{J}_{j}$ is $\mathfrak{A}$-admissible; {\rm(b)} every member of $\bigcup_{j}\mathfrak{J}_{j}$ satisfies the condition {\rm(III)}.
\end{lemma}

\begin{proof}
Since $\bigcup_{i}\mathfrak{H}_{i}$ is $\mathfrak{A}$-admissible, there is a member $V$ of $\Sigma_{\rho}$ and a member $U$ of $\mathfrak{H}_{i}$ such that $V$ and $U$ are birationally equivalent. Then $l((\rho_{5}+m)\mathfrak{K}(V))=l((\rho_{5}+m)\mathfrak{K}(U))$ for all positive integers $m$ by Lemma \ref{art14-lem12}. Setting $m_{0}=\rho_{5}$, $m_{i}=\rho_{5}+i$, $q_{m_{i}}=P_{\rho}(\rho_{5}+i)$ and then $q_{m_{i}}=P_{\rho}(\rho_{5}+i)+1$ in the Corollary to Proposition \ref{art14-app-prop1.1} in the Appendix, we get our lemma easily.
\end{proof}

\begin{lemma}\label{art14-lem17}
There is a finite union $\bigcup_{i}\mathfrak{J}_{i}$ of irreducible algebraic families contained in $\bigcup_{i}\mathfrak{J}_{j}$ having the following properties: {\rm(a)} $\bigcup_{i}\mathfrak{J}_{i}$ is $\mathfrak{A}$-admissible; {\rm(b)} every member of $\bigcup_{i}\mathfrak{J}_{i}$ satisfies the condition {\rm(I)}.
\end{lemma}

\begin{proof}
Let $k$ be a common field of definition of the component families $\mathfrak{J}_{i}$. Let $U_{0}\in \mathfrak{J}_{j}\cap \mathfrak{A}$ and $U$ a generic member of $\mathfrak{J}_{j}$ over $k$. $m\mathfrak{K}(U_{0})$ defines a birational map for large $m$ by Lemma \ref{art14-lem12}. Therefore $m\mathfrak{K}(U)$ defines a rational map $f$ of $U$ such that $f$ does not decrease the dimension by Lemma \ref{art14-app-lem1.1} and the Corollary to Proposition \ref{art14-app-prop2.1} of the Appendix. Then we see that $m\mathfrak{K}(U)$ defines a birational map for large $m$ which is an easy consequence of the technique of normalization in a finite algebraic extension of the function field (c.f. \cite{art14-key25}, Appendix I).

Fix\pageoriginale a positive integer $m_{0}$ such that $m_{0}\mathfrak{K}(U)$, $m_{0}\mathfrak{K}(U_{0})$ both define birational transformations. Then $m_{0}\mathfrak{K}(U)$ is, in particular, linearly equivalent to a positive $U$-divisor $Y$. Consider an algebraic family with divisors over $\overline{k}$ such that $(U,Y)$ is a generic element of it over $\overline{k}$ and apply Proposition \ref{art14-app-prop2.2} and its Corollary \ref{art14-app-coro3} of the Appendix to it (c.f. also Lemma \ref{art14-app-lem1.1} of the Appendix). Then we see that there is an irreducible algebraic family $\mathfrak{J}'_{j}$ such that the Chow-variety of $\mathfrak{J}'_{j}$ is $k$-open on that of $\mathfrak{J}_{j}$ and that $m_{0}\mathfrak{K}(U')$ defines a birational map of $U'$ whenever $U'$ is in $\mathfrak{J}'_{j}\cdot \mathfrak{J}_{j}-\mathfrak{J}'_{j}$ is a finite union of irreducible algebraic families. Apply the above procedure to all those components of $\mathfrak{J}_{j}-\mathfrak{J}'_{j}$ which contain some members of $\mathfrak{A}$. This process cannot continue indefinitely. Doing the same for each $\mathfrak{J}_{j}$, we get easily our lemma.
\end{proof}

\begin{lemma}\label{art14-lem18}
Let $\mathfrak{K}$ be a finite union of irreducible algebraic families in projective spaces and $\mathfrak{B}$ a set of members of $\mathfrak{K}$. Assume that the following conditions are satisfied : {\rm(i)} $\mathfrak{K}$ consists of non-singular varieties; {\rm(ii)} $\mathfrak{K}$ is $\mathfrak{B}$-admissible; {\rm(iii)} $\mathfrak{B}$ is a subset of $\mathfrak{A}$; {\rm(iv)} every member of $\mathfrak{K}$ satisfies the conditions {\rm(I)} and {\rm(III)}. Then there is a finite union $\mathfrak{K}^{*}$ of irreducible algebraic families, contained in $\mathfrak{K}$, having the following properties: {\rm(a)} $\mathfrak{K}^{*}$ is $\mathfrak{B}$-admissible; {\rm(b)} for each component $\mathfrak{K}^{*}_{i}$ cf $\mathfrak{K}^{*}$, there is a $U_{i}\in \mathfrak{K}^{*}_{i}\cap \mathfrak{B}$ and a positive integer $m_{i}\geq \rho_{5}$ such that $m_{i}\mathfrak{K}(U_{i})$ defines a non-degenerate birational map $h_{i}$ of $U_{i}$ such that $h_{i}(U_{i})=W_{i}$ is non-singular and that $C_{W_{i}}\sim m_{i}\mathfrak{K}(W_{i})$; {\rm(c)} when $U$ is a generic member of $\mathfrak{K}^{*}_{i}$ over a common field $k$ of definition of the $\mathfrak{K}^{*}_{i}$, $m_{i}\mathfrak{K}(U)$ defines a non-degenerate birational map $h$ of $U$ such that $\deg(W_{i})=\deg(W)$, where $W=h(U)$; {\rm(d)} for each member $U'$ of $\mathfrak{K}^{*}_{i}$, $m_{i}\mathfrak{K}(U')$ defines a birational map.
\end{lemma}

\begin{proof}
We proceed by induction on the dimension of $\mathfrak{K}$. When the dimension of $\mathfrak{K}$ is zero, our lemma is trivial. Therefore we assume that our lemma is true for dimension up to $s-1$ and set $\dim \mathfrak{K}=s$. In order to prove our lemma, it is clearly enough to do so when $\mathfrak{K}$ is an irreducible algebraic family.

Let $Y$ be a positive divisor on $U$ such that $Y\sim \rho_{5}\mathfrak{K}(U)$, where $U$ denotes a generic member of $\mathfrak{K}$ over $k$. Then we consider an algebraic family\pageoriginale with divisors defined over $\overline{k}$ such that $(U,Y)$ is a generic member of it over $\overline{k}$ and apply Corollary \ref{art14-coro2} to Proposition \ref{art14-app-prop2.2} in the Appendix to our situation. By doing so, we can find a positive integer $m_{0}$ such that $m\mathfrak{K}(U')$ defines a birational map for every member $U'$ of $\mathfrak{K}$ whenever $m\geq m_{0}$.

Let $U_{1}$ be a member of $\mathfrak{B}$. Since $\mathfrak{B}$ is contained in $\mathfrak{A}$, there is a positive integer $m_{1}\geq \rho_{5}$, $m_{0}$ with the following properties: $m_{1}\mathfrak{K}(U_{1})$ defines a non-degenerate birational map $h_{1}$ of $U_{1}$ and $W_{1}=h_{1}(U_{1})$ is non-singular; $C_{W_{1}}\sim m_{1}\mathfrak{K}(W_{1})$ (c.f. Lemma \ref{art14-lem13}). Let $h$ be a non-degenerate birational map defined by $m_{1}\mathfrak{K}(U)$. Since $h$ is determined uniquely by $m_{1}\mathfrak{K}(U)$ up to a projective transformation, we see that $\deg(h(U))\geq \deg(h_{1}(U_{1}))=\deg (W_{1})$, $W_{1}=h_{1}(U_{1})$, by applying Proposition \ref{art14-app-prop2.1} of the Appendix and using the compatibility of specializations with the operations of algebraic projection.

Let $\mathfrak{K}'$ be the set of members $U'$ of $\mathfrak{K}$ with the following properties: when $h'$ is a non-degenerate birational map of $U'$ defined by $m_{1}\mathfrak{K}(U')$, then $\deg(h'(U'))\leq \deg(W_{1})$. {\em We claim that $\mathfrak{K}'\supset \mathfrak{B}$ and is a finite union of irreducible algebraic families.} We consider the same algebraic family with divisors as above, which is defined over $\overline{k}$, having $(U,Y)$ as a generic element over $\overline{k}$. In applying Corollary \ref{art14-coro1} to Proposition \ref{art14-app-prop2.2} in the Appendix to our situation, we let $\deg(h_{1}(U_{1}))=s_{0}$. In view of Lemma \ref{art14-app-lem1.1} of the Appendix, it is then easy to see that the set of Chow-points of members of $\mathfrak{K}'$ is a closed subset of that of $\mathfrak{K}$ over $k$. Let $U''$ be a member of $\mathfrak{B}$. Then it is contained in $\mathfrak{A}$ and there is a member $V''$ of $\Sigma_{\rho}$ such that there is a birational map $f''$ of $V''$, mapping $V''$ generically onto $U''$. Let $k'$ be a common field of rationality of $U''$, $V''$ and $f$ over $k$. By Lemma \ref{art14-lem11}, $\Lambda(f''(m_{1}\mathfrak{K}(V'')))+m_{1}E=\Lambda(m_{1}\mathfrak{K}(U''))$ where $E$ is a positive $U''$-divisor whose components are exceptional divisors for ${f''}^{-1}$. Let $h''$ be a non-degenerate rational map of $U''$ defined by $m_{1}\mathfrak{K}(U'')$ and the $Z_{i}$ (resp. $Z'_{i}$) independent generic divisors of $\Lambda(m_{1}\mathfrak{K}(V''))$ (resp. $\Lambda(m_{1}\mathfrak{K}(U''))$) over $k'$. Then the above relation between two complete linear systems show that $I(Z_{1}\ldots Z_{n}/V'',k')=I(Z'_{1}\ldots Z'_{n}/U'',k')$. Moreover, $h''$ is birational by our choice of $m_{1}$, $\deg(h''(U''))\leq I(Z'_{1}\ldots Z'_{n}/U'',k')$\pageoriginale and $I(Z_{1}\ldots Z_{n}/V'',k')\leq m_{1}{}^{n}d_{0}$ by Lemma \ref{art14-lem5}. It follows that $\deg(h''(U''))\leq m_{1}{}^{n}d_{0}$. On the other hand, $U_{1}$ is the underlying variety of some member of $\Sigma_{\rho}$ by Lemma \ref{art14-lem13}. Consequently $\deg(W_{1})=m^{n}_{1}d_{0}$ by the same lemma. This proves that $U''$ is contained in $\mathfrak{K}'$. Our contention is thereby proved.
\end{proof}

Let $\mathfrak{K}''$ be the union of those components of $\mathfrak{K}'$ which contain $U_{1}$. Denote by $U''$ now a generic member of a component of $\mathfrak{K}''$ over $\overline{k}$. As before, from Proposition \ref{art14-app-prop2.1} and from the compatibility of specializations with the operation of algebraic projection, we see that $\deg(h''(U''))\break \geq \deg (h_{1}(U_{1}))$ and consequently $\deg(h''(U''))=\deg (h_{1}(U_{1}))$. Moreover, $\mathfrak{K}''$ is $\mathfrak{K}''\cap \mathfrak{B}$-admissible.

We have $\mathfrak{B}=(\mathfrak{K}''\cap \mathfrak{B})+((\mathfrak{K}'-\mathfrak{K}'')\cap \mathfrak{B})$. $\mathfrak{K}'-\mathfrak{K}''$ is a finite union of irreducible algebraic families. Call $\mathfrak{K}_{1}$ the union of those components of $\mathfrak{K}'-\mathfrak{K}''$ which contain some members of $\mathfrak{B}$ and set $\mathfrak{B}_{1}=\mathfrak{B}\cap \mathfrak{K}_{1}$. We have $\dim \mathfrak{K}>\dim \mathfrak{K}_{1}$. By our induction assumption, (a), (b), (c), (d) are satisfied by $\mathfrak{K}_{1}$, $\mathfrak{B}_{1}$. We have shown that $\mathfrak{K}''$, $\mathfrak{K}''\cap \mathfrak{B}$ satisfy these too by Lemma \ref{art14-lem14}. Thus our lemma is proved.

From our lemma, Lemma \ref{art14-lem14} and from Lemma \ref{art14-app-lem1.1} of the Appendix, we get

\begin{corollary}\label{art14-coro1}
$U$ and $h$ in {\rm(c)} of our lemma further satisfy the following properties : $W=h(U)$ is non-singular, $C_{W}\sim m_{i}\mathfrak{K}(W)$ and $\mathfrak{K}(W)^{(n)}=\mathfrak{K}(W_{i})^{(n)}=d_{0}$.
\end{corollary}

\begin{corollary}\label{art14-coro2}
In our Lemma \ref{art14-lem18} and Corollary \ref{art14-coro1} above, $m_{i}$ may be replaced by a positive integer which is a multiple of $m_{i}$.
\end{corollary}

\begin{proof}
This is an easy consequence of Lemma \ref{art14-lem11}.
\end{proof}

\begin{theorem}\label{art14-thm2}
Let $V^{n}$ be a canonically polarized variety and $P(x)$ its Hilbert characteristic polynomial. Assume that $(A_{n})$ and $(B_{n})$ have solutions for $V$ and that theorems on dominance and birational resolution in the sense of Abhyankar hold for dimension $n$. Then there is a constant $\rho_{7}$ which depends on $P(x)$ only such that $\rho_{7}X_{V}$ defines a non-degenerate projective embedding of $V$.
\end{theorem}

\begin{proof}
As\pageoriginale we pointed out at the beginning of this paragraph, it is enough to prove this for $V\in \Sigma_{\rho}$. By Lemmas \ref{art14-lem15}, \ref{art14-lem15}, \ref{art14-lem16}, the finite union $\mathfrak{J}$ of irreducible families constructed in Lemma \ref{art14-lem17}, together with $\mathfrak{A}$, satisfies the requirements of Lemma \ref{art14-lem18}. Therefore, there is a finite union $\mathfrak{M}$ of irreducible algebraic families satisfying the conclusions of Lemma \ref{art14-lem18}. For the sake of simplicity, we shall say that a {\em non-singular projective variety $D$ has a property} (*) {\em with respect to $t'$} if $t'\mathfrak{K}(D)$ defines a non-degenerate birational map $h$ of $D$ such that $A=h(D)$ is non-singular, is the underlying variety of a member of $\Sigma$ and that $C_{A}\sim t'\mathfrak{K}(A)$.

Let the $\mathfrak{M}_{i}$ be the components of $\mathfrak{M}$ and the $U_{i}$, $m_{i}$ as in Lemma \ref{art14-lem18}, (b) ($\mathfrak{K}^{*}_{i}$ in the Lemma is our $\mathfrak{M}_{i}$). By Corollary \ref{art14-lem2} of Lemma \ref{art14-lem19}, $m_{i}$ may be replaced by $t=\Pi_{i}m_{i}$ or by any positive multiple of $t$. Let $k$ be an algebraically closed common field of rationality of the $\mathfrak{M}_{i}$ and $U$ a generic member of $\mathfrak{M}_{i}$ over $k$. $U$ has the property (*) with respect to $t$ by Lemma \ref{art14-lem18} and its corollaries. Let $U'\in \mathfrak{M}_{i}$ and $h'$ a non-degenerate birational map defined by $t\mathfrak{K}(U')$. Assume that $h'$ has the properties that $\deg(h'(U'))=\deg (h(U))$ and that $h'(U')$ is non-singular. $h$ is uniquely determined by $t\mathfrak{K}(U)$ up to a projective transformation. Therefore, we may assume without loss of generality transformation. Therefore, we may assume without loss of generality that $W'=h'(U')$ is a specialization of $W=h(U)$ over $k$ by Lemma \ref{art14-app-lem1.1} and Proposition \ref{art14-app-prop2.1} of the Appendix, since specializations are compatible with the operation of algebraic projection. Since self-intersection numbers and linear equivalence are preserved by specializations, it follows that $W'$ has also the property (*) with respect to $t$ (c.f. Lemma \ref{art14-app-lem1.1} of the Appendix and \cite{art14-key2}).

Let $Y$ be a member of $\Lambda(t\mathfrak{K}(U))$ and consider an algebraic family with divisors, defined over $k$, with a generic element $(U,Y)$ over $k$. We apply Corollary \ref{art14-app-coro3} to Proposition \ref{art14-app-prop2.3} in the Appendix to this. By doing so, we can find an irreducible algebraic family $\mathfrak{M}'_{i}$ of non-singular varieties, having the following properties: (a) the Chow variety of $\mathfrak{M}'_{i}$ is $k$-open on that of $\mathfrak{M}_{i}$; (b) when $U'\in \mathfrak{M}'_{i}$, $t\mathfrak{K}(U')$ defines a non-degenerate birational map of $U'$ such that $h'(U')$ is non-singular; (c) $\deg(h'(U'))=\deg (h(U))$. Let $\mathfrak{M}'=\bigcup_{i}\mathfrak{M}'_{i}$ and $\mathfrak{A}^{*}=\mathfrak{A}\cap \mathfrak{M}'$.\pageoriginale As we have shown above $U'\in \mathfrak{A}^{*}$ has the property (*) with respect to $t$.

Let $\mathfrak{B}=(\mathfrak{M}-\mathfrak{M}')\cap \mathfrak{A}\cdot \mathfrak{M}-\mathfrak{M}'$ is a finite union of irreducible algebraic families. When we remove from it those components which do not contain members of $\mathfrak{B}$, we get a finite union $\mathfrak{N}$ of irreducible algebraic families, which is contained in $\mathfrak{M}$, $\mathfrak{B}$-admissible and satisfies $\dim \mathfrak{M}>\dim \mathfrak{N}$. When we apply our process to $\mathfrak{N}$ and $\mathfrak{B}$ and continue it, applying Lemma \ref{art14-lem18} and its corollaries, it has to terminate by a finite number of steps. Consequently, we can find a positive integer $t'$ such that a member of $\mathfrak{B}$ has the property (*) with respect to $t'$. When we set $\rho_{7}=t\cdot t'\cdot \rho$, this constant satisfies the requirements of our theorem.
\end{proof}

\begin{coro*}
Let the characteristic be zero, $V^{3}$ a canonically polarized variety and $P(x)$ the Hilbert characteristic polynomial of $V$. Then $(C_{3})$ is true for $V^{3}$ and $P(x)$.
\end{coro*}

\begin{proof}
This follows easily from our theorem, Theorem \ref{art14-lem1} and from Proposition \ref{art14-prop1}.
\end{proof}

\bigskip

\begin{center}
{\Large\bf Appendix}
\end{center}

\setcounter{section}{0}
\section{}\label{art14-app-sec1}
\begin{sublemma}\label{art14-app-lem1.1}
Let $U^{n}$ and ${U'}^{n}$ be non-singular and non-ruled subvarieties of projective spaces such that $U'$ is a specialization of $U$ over a discrete valuation ring $\mathfrak{O}$. Let $\mathfrak{K}(U)$ be a canonical divisor of $U$ and $(U',Y)$ a specialization of $(U,\mathfrak{K}(U))$ over $\mathfrak{O}$. Then $Y$ is a canonical divisor of $U'$.
\end{sublemma}

\begin{proof}
When $n=1$, the complete linear system of canonical divisors on $U$ (resp. $U'$) is characterized by the fact that it is a complete linear system of positive divisor of degree $2g-2$ and dimension at least $g-1$. Hence our lemma is easily seen to be true in this case.

Assume that our lemma is true for dimensions up to $n-1$. Let $k$ (resp. $k'$) be a common field of rationality of $U$ and $\mathfrak{K}(U)$ (resp. $U'$ and $Y$) and $C$, $C^{*}$ (resp. $C'$, ${C'}^{*}$) independent generic hypersurface sections\pageoriginale of $U$ (resp. $U'$) over $k$ (resp. $k'$). Then $(U',Y,C',{C'}^{*})$ is a specialization of $(U,\mathfrak{K}(U), C,C^{*})$ over $\mathfrak{O}$. $C\cdot (C^{*}+\mathfrak{K}(U))$ is a canonical divisor of $C$ (c.f. \cite{art14-key31}) and this has the unique specialization $C'\cdot ({C'}^{*}+Y)$ over the above specialization with reference to $\mathfrak{O}$, since specializations and intersection-product are compatible operations. It follows that $C'\cdot ({C'}^{*}+Y)$ is a canonical divisor of $C'$. When $\mathfrak{K}(U')$ is a canonical divisor of $U'$, rational over $k'$, $C'\cdot ({C'}^{*}+Y)\sim C'\cdot ({C'}^{*}+\mathfrak{K}(U'))$. $C'$ is a generic hypersurface section of $U'$ over a filed of rationality of ${C'}^{*}$ over $k'$. Moreover, when the degree of the hypersurface is at least two, a generic linear pencil contained in the linear system of hypersurface sections consists of irreducible divisors (c.f. \cite{art14-key19}). It follows that $Y\sim \mathfrak{K}(U')$ by an equivalence criterion of Weil (c.f. \cite{art14-key27}, Th. 2).\footnote{In \cite{art14-key27}, Th. 2, it is claimed that $Y-\mathfrak{K}(U'\sim \Sigma m_{i}T_{i}$ where the $T_{i}$ are some subvarieties of $U'$. But these $T_{i}$ are components of reducible members of such a pencil contained in the linear system of hyperplane secitons. We can eliminate them using linear systems of hypersurface sections.}

We shall consider a family of varieties with divisors on them. Let $\mathfrak{A}$ be an irreducible algebraic family of subvarieties in a projective space, $A$ the Chow-variety of it and $a$ a generic point of $A$ over a field of definition $k$ of $A$, corresponding to a variety $U(a)$. Let $X(a)$ be a divisor on $U(a)$ and $X(a)=X(a)^{+}-X(a)^{-}$ the reduced expression for $X(a)$ where $X(a)^{+}$, $X(a)^{-}$ are both positive divisors. Let $u$ (resp. $v$) be the Chow-point of $X(a)^{+}$ (resp. $X(a)^{-}$) and $z=(u,v)$. Let $A'$ be the locus of $(a,z)$ over $\overline{k}$. When $(a',z')$ is a point of $A'$, $a'$ defines a cycle $U(a')$ in the projective space uniquely such that the support of $Y$ is contained in the support of $U(a')$. When every member of $\mathfrak{A}$ is irreducible. $A'$ defines an irreducible family whose member is a variety with a chain of codimension 1 on it. We shall call this an {\em irreducible family of varieties with chains of codimension $1$.} When $k'$ is a field of definition for $A'$, we shall call $k'$ a field of definition or rationality of the family.
\end{proof}

\begin{subprop}\label{art14-app-prop1.1}
Let $\mathfrak{A}'$ be an irreducible algebraic family of non-sin\-gular varieties $U(a')$ with divisors $X(a')$ and $\{q_{m_{i}}\}$ an increasing sequence\pageoriginale of positive integers starting with $q_{m_{0}}>1$. Assume that there is a member $(U(a_{0}), X(a_{0}))$ such that $l(m_{i}X(a_{0}))\geq q_{m_{i}}$ for all $i$. Let $(U(a), X(a))$ be a generic member of $\mathfrak{A}'$ over a common field $k$ of rationality of $\mathfrak{A}'$, $a_{0}$ and assume that $l(m_{i}X(a))\geq q_{m_{i}}$ for $0\leq i\leq s-1$ but $l(m_{s}X(a))<q_{m_{s}}$. Then, there is a finite union $\mathfrak{E}$ of irreducible families, defined over $\overline{k}$ and contained in $\mathfrak{A}'$, such that a member $(U(a'),X(a'))$ of $\mathfrak{A}'$ is in $\mathfrak{E}$ if and only if $l(m_{i}X(a'))\geq q_{m_{i}}$ for $0\leq i\leq s$.
\end{subprop}

We shall prove this by a series of lemmas.

Let $\dim U(a_{0})=n$ and $H_{1},\ldots,H_{n-1}$ independent generic hyper-surfaces of degree $t$ over $k$. Let $H^{(1)}=H_{1}\ldots H_{n-1}$ and $H^{(1)},\ldots,H^{(r)}$ $r$ independent generic specializations of $H^{(1)}$ over $k$. For each point $a'$ of $A'$, we set $U(a')\cdot H^{(i)}=C(a')_{i}$ whenever the intersection is proper. We take $r$ and $t$ sufficiently large.

\begin{sublemma}\label{art14-lem1.2}
Let $B$ be the set of points $a'$ on $A'$ satisfying the following conditions: {\rm(i)} $C(a')_{i}$ is defined for all $i$; {\rm(ii)} the $C(a')_{i}$ are non-singular for all $i$; {\rm(iii)} $X(a')$ and $C(a')_{i}$ intersect properly on $U(a')$ for all $i$. Let $k'$ be an algebraically closed common field of rationality for the $H^{(i)}$ over $k$. Then $B$ is a $k'$-open subset of $A'$.
\end{sublemma}

\begin{proof}
These are well-known and easy exercises. Therefore, we shall omit a proof.
\end{proof}

We shall show that the set of points $a'$ on $B$ such that $l(m_{s}X(a'))\geq q_{m_{s}}$ forms a $k'$-closed subset of $B$. We can cover $A'$ by open sets $B$ by changing the $H^{(i)}$. Therefore our problem is reduced to the similar problem on the family defined by $B$. In order to solve our problem on this family we may replace $B$ by a variety with a proper and surjective morphism on it. Therefore, we may assume without loss of generality that the $C(a')_{i}$ carry rational points over $k'(a')$.

For each $a'$ in $B$, let $J(a')_{i}$ be the Jacobian variety of $C(a')_{i}$ and $\Gamma(a')_{i}$ the graph of the canonical map $\phi(a')_{i}$ of $C(a')_{i}$ into $J(a')_{i}$. We assume that these are constructed by the method of Chow so that these are compatible with specializations (c.f. \cite{art14-key1}, \cite{art14-key8}). In order to simplify the notations, we simply denote by $\mathfrak{I}(Y\cdot C(a')_{i})$ the {\em Abelian sum} of $Y\cdot C(a')_{i}$ on $J(a')_{i}$, whenever $Y$ is a $U(a')$-divisor such\pageoriginale that $Y\cdot C(a')_{i}$ is defined. It should be pointed out here that the $J(a')_{i}$ and the $\Gamma(a')_{i}$ are rational over $k'(a')$.

Let $P$ be the ambient projective space of the $U(a)$ and $F^{*}$ the closed subset of a projective space, consisting of Chow-points of positive cycles in $P$ which have the same dimension and degree as members of $\Lambda(m_{s}X(a_{0}))$. There is a closed subset $T^{*}$ of $B\times F^{*}$ such that a point $(a,y)$ of $B\times F^{*}$ is in $T^{*}$ if and only if $U(a)$ carries the cycle $Y(y)$ defined by $y$ (c.f. \cite{art14-key3}). Let $F$ be the geometric projection of $T^{*}$ on $F^{*}$ and $T=B\times F\cap T^{*}$. $T$ is a $k'$-closed subset of $B\times F$. Let $a$ be a generic point of $B$ over $k'$. Since the $J(a')_{i}$ are defined over $k'(a')$ for $a'\in B$, there is a subvariety $Z$ of $B\times \Pi_{i}P_{i}$, where the $P_{i}$ are ambient spaces for the $J(a)_{i}$, such that $Z(a)=\Pi_{i}J(a)_{i}$ (c.f. \cite{art14-key25}, Chap. VIII).

\begin{sublemma}\label{art14-app-lem1.3}
Let $U$ and $U'$ be non-singular subvarieties of a projective space such that $U'$ is a specialization of $U$ over a field $k$. Let $X$ (resp. $X'$) be a divisor on $U$ (resp. $U'$) such that $(U',X')$ is a specialization of $(U,X)$ over $k$. Let $u'$ be a given point of $U'$. Then there are divisors $D$, $E$ (resp. $D'$, $E'$) on $U$ (resp. $U'$) with the following properties: {\rm(a)} $X\sim D-E$ on $U$ and $X'\sim D'-E'$ on $U'$; {\rm(b)} the supports of $D'$, $E'$ do not contain $u'$; {\rm(c)} $(U',X',D',E')$ is a specialization of $(U,X,D,E)$ over $k$.
\end{sublemma}

\begin{proof}
Let $C$ (resp. $C'$) be a hypersurface section of $U$ (resp. $U'$) by a hypersurface of degree $m$. Then, as is well known, $X+C$ (resp. $X'+C'$) is ample on $U$ (resp. $U'$) and $l(X+C)=l(X'+C')$, $l(C)=l(C')$ when $m$ is sufficiently large (c.f. \cite{art14-key25}, Chap. IX, \cite{art14-key31}, \cite{art14-key21}, \cite{art14-key4}). Denote by $G(*)$ the support of the Chow-variety of the complete linear system determined by $^{*}$. Since linear equivalence is preserved by specializations (c.f. \cite{art14-key24}), it follows that $(U',X',G(X'+C'),G(C'))$ is a specialization of $(U,X,G(X+C),G(C))$ over $k$. When a point $x'$ in $G(X'+C')$ and a point $y'$ in $G(C)$ are given, there is a point $x$ in $G(X+C)$ and a point $y$ in $G(C)$ such that $(x,y)\to (x',y')$ ref. $k$ over the above specialization. We can choose $x'$, $y'$ so that the corresponding divisors $D'$, $E'$ do not pass through $u'$. Since $X'\sim S'+C'-C'$ and $X\sim X+C-C$, our lemma follows at once from the above observations. 
\end{proof}

\begin{coro*}
Let\pageoriginale $T_{\alpha}$ be a component of $T$ and $(a,y)$ a generic point of $T_{\alpha}$ over $k'$. There is a rational map $f_{\alpha}$ of $T_{\alpha}$ into $Z$ such that $f_{\alpha}(a,y)=(a,\ldots,\mathscr{S}(Y(y)\cdot C(a)_{i}),\ldots)$.\footnote{$Y(y)$ denotes the divisor defined by $y$.} Moreover, $f_{\alpha}$ satisfies the following conditions: {\rm(a)} when $(a',y')\in T_{\alpha}$, $f_{\alpha}(a,y)$ has a unique specialization $(a',Q')$ over $k'$ over $(a,y)\to (a',y')$ ref. $k'$; {\rm(b)} when $Y'$ is a $U(a')$-divisor such that $Y'\sim Y(y')$ and that $Y'$ and the $C(a')_{i}$ intersect properly on $U(a')$, $Q'=(a',\ldots,\mathscr{S}(Y',C(a')_{i}),\ldots)$; {\rm(c)} the locus $L$ of $(a,\ldots,\mathscr{S}(m_{s}X(a)\cdot C(a)_{i}),\ldots)$ over $k'$ is a subvariety of $Z$ and contains $(a',\ldots,\mathscr{S}(m_{s}X(a')\cdot C(a)_{i}),\ldots)$ whenever $a'\in B$ and the latter is a unique specialization of the former over $k'$, over $a\to a'$ {\rm ref.} $k'$.
\end{coro*}

\begin{proof}
This follows easily from Lemma \ref{art14-app-lem1.3}, from the compatibility of specializations with the Chow-construction of Jacobian varieties, the operation of intersection-product (c.f. \cite{art14-key24}), the Abelian sums and from the invariance of linear equivalence by specializations.
\end{proof}

Let $W^{*}_{\alpha}$ be the closure of the graph of $f_{\alpha}$ on $B\times F\times Z$, $W^{*}$ the union of the $W^{*}_{\alpha}$ and $W=W^{*}\cap B\times F\times L$. $W$ is a $k'$-closed subset of $B\times F\times L$.

\begin{sublemma}\label{art14-app-lem1.4}
The set $E'$ of points $a'\in B$ such that $W\cap a'\times F\times L$ has component of dimension at least $q_{m_{s}}-1$ forms a $k'$-closed proper subset of $B$.
\end{sublemma}

\begin{proof}
When $a$ is a generic point of $B$ over $k'$, the projection of the intersection on $F$ is the support of the Chow-variety of $\Lambda(m_{s}X(a))$ (c.f. \cite{art14-key14}, \cite{art14-key26}). Then our lemma follows at once from \cite{art14-key27}, Lemma \ref{art14-lem7}, applied to $B\times F\times \overline{L}$ where $\overline{L}$ denotes the closure of $L$ in its ambient space.
\end{proof}

\begin{sublemma}\label{art14-app-lem1.5}
Let $a'\in B$ such that $l(m_{s}X(a'))<q_{m_{s}}$. Then $a'\not\in E'$.
\end{sublemma}

\begin{proof}
Assume the contrary. Then the intersection $W\cap a'\times F\times L$ contains a component $a'\times D$ of dimension at least $q_{m_{s}}-1$ by Cor., Lemma \ref{art14-app-lem1.3}. Let $k''$ be an algebraically closed field, containing $k'$, over which $D$ is defined and $Q'$ a generic point of $D$ over $k''$. It is of the form $(y',e')$ where $e'=(a',\ldots,\mathscr{S}(m_{s}X(a')\cdot C(a')_{i}),\ldots)$ (c.f. Cor., Lemma \ref{art14-app-lem1.3}).\pageoriginale Since $r$ is sufficiently large, there is an index $i$ such that $H^{(i)}$ is generic over $k(a',y')$ (c.f. \cite{art14-key26}, Lemma 9). Then Cor., Lemma \ref{art14-app-lem1.3} implies that $\mathscr{S}(m_{s}X(a')\cdot C(a')_{i})=\mathscr{S}(Y(y')\cdot C(a')_{i})$. Since $t$ is sufficiently large and since $H^{(i)}$ is generic over $k(a',y')$, it follows that $m_{s}X(a')\sim Y(y')$ by an equivalence criterion of Weil (c.f. \cite{art14-key27}, Th. 2; see also the footnote for Lemma \ref{art14-app-lem1.1}). Hence $l(m_{s}X(a'))\geq q_{m_{s}}$ and this contradicts our assumption.
\end{proof}

\begin{sublemma}\label{art14-app-lem1.6}
Let $a'\in B$ such that $l(m_{s}X(a')\in)\geq q_{m_{s}}$. Then $a'\in E'$.
\end{sublemma}

\begin{proof}
Let $D'$ be the Chow-variety of the complete linear system\break $\Lambda(sX(a'))$. Then $\dim D'\geq q_{m_{s}}-1>0$. Let $k''$ be an algebraically closed field, containing $k'$, over which $D'$ is rational. Let $y'$ be a generic point of $D'$ over $k''$. Then $(a',y')$ is contained in some component $T_{\alpha}$ of $T$. Let $(a,y)$ be a generic point of $T_{\alpha}$ over $k''$. Let 
$$
e'=(a',\ldots,\mathscr{S}(m_{s}X(a')\cdot C(a')_{i}),\ldots)
$$ 
and $e=(a,\ldots,\mathscr{S}(Y(y)\cdot C(a)_{i}),\ldots )$. Then $(a,y,e)\to (a',y',e')$ ref. $k''$ by Cor., Lemma \ref{art14-app-lem1.3}. Hence $(a',y',e')$ is a point of $W$. It follows that $W\cap a'\times F\times L$ contains $a'\times D'\times e'$ and $a'$ is contained in $E'$. Our lemma is thereby proved.
\end{proof}

As we have pointed out, Lemmas \ref{art14-app-lem1.4}, \ref{art14-app-lem1.5}, \ref{art14-app-lem1.6} prove our proposition.

\medskip
\noindent
{\bf Corollary to Proposition \ref{art14-app-prop1.1}.}~ {\em Let the notations and assumptions be as in our proposition. There is a finite union $\mathfrak{E}$ of irreducible families, defined over $\overline{k}$ and contained in $\mathfrak{A}'$, such that a member $(U(a'),X(a'))$ of $\mathfrak{A}'$ is in $\mathfrak{E}$ if and only if $l(mX(a'))\geq q_{m}$ for all $m$.}

\section{}\label{art14-app-sec2}

\begin{subprop}\label{art14-app-prop2.1}
Let $V^{n}$ (resp. ${V'}^{n}$) be a complete abstract variety, non-singular in codimension $1$, and $X$ (resp. $X'$) a divisor on $V$ (resp. $V'$). Let $k$ be a common field of rationality of $V$ and $X$, $\mathfrak{O}$ a discrete valuation ring of $k$ and assume that $(V',X')$ is a specialization of $(V,X)$ over $\mathcal{O}$ and that $l(X)=l(X')$. Let $\Gamma'$ be the closure of the graph of a non-degenerate rational map of $V'$ defined by $X'$. Then, there is a field $K$ over $k$, a discrete valuation ring $\mathcal{O}'$ of $K$ dominating $\mathfrak{O}$ and the closure $\Gamma$ of the graph of a non-degenerate rational map of $V$ defined\pageoriginale by $X$ such that $(V',X',\Gamma'+Z')$ is a specialization of $(V,X,\Gamma)$ over $\mathfrak{O}'$, where $Z'$ is such that $pr_{V'}Z'=0$.
\end{subprop}

\begin{proof}
Let $k'$ be the residue field of $\mathfrak{O}$. Since $X'$ is rational over $k'$, $\Lambda(X')$ is defined over $k'$. Let $g'_{1}=1$, $g'_{2},\ldots,g'_{N+1}$ be functions on $V'$ which define $\Gamma'$. From Lemmas 4 and 5, \cite{art14-key16}, we can see easily that there is a filed $K$ over $k$, a discrete valuation ring $\mathfrak{O}'$ of $K$ which dominates $\mathfrak{O}$ and a basis $(g_{i})$ of $L(X)$ over $K$ such that $(V',X',(g'_{i}))$ is a specialization of $(V,X,(g_{i}))$ over $\mathfrak{O}'$. Let $\Gamma$ be the closure of the graph of a non-degenerate rational map of $V$ defined by $X$, determined in terms of $(g_{i})$. Let $T$ be a specialization of $\Gamma$ over $\mathfrak{O}'$. It is clear and easy to see that $\Gamma'$ is contained in the support of $T$. Therefore, $\Gamma'$ is a component of $T$. When that is so, our proposition follows from the compatibility of specializations with the operation of algebraic projection (c.f. \cite{art14-key24}).

In the discussions which follow, we shall need the following definition. Let $U$ and $W$ be two abstract varieties, $f$ a rational map of $U$ into $W$ and $U'$ a subvariety of $U$ along which $f$ is defined. Let $f'$ be the restriction of $f$ on $U'$ and $W'$ the geometric image of $U'$ by $f'$. We shall denote by $f[U']$ the variety $W'$ if $\dim U'=\dim W'$ and $0$ otherwise.
\end{proof}

\begin{coro*}
Notations and assumptions being the same as in our propositions, let $f$ (resp. $f'$) be a non-degenerate rational map of $V$ (resp. $V'$) defined by $X$ (resp. $X'$). When $f'[V']\neq 0$, then $f[V]\neq 0$.
\end{coro*}

\begin{proof}
Let $k'$ be the residue field of $\mathfrak{O}$ and $Q'$ a generic point of $V'$ over $k'$. Then there are $n$ independent generic divisors $X'_{i}$ of $\Lambda(X')$ over $k'$ such that $Q'$ is a proper point of intersection of $\bigcap_{i}X'_{i}$. Let the $X_{i}$ be $n$ independent generic divisors of $\Lambda(X)$ over $k$ and $\mathfrak{O}^{*}$ a discrete valuation ring, dominating $\mathfrak{O}$, such that $(V',X',(X'_{i}))$ is a specialization of $(V,X,(X_{i}))$ over $\mathfrak{O}^{*}$ (c.f. \cite{art14-key16}). By the compatibility of specializations with the operation of intersection-product (c.f. \cite{art14-key24}, in particular, Th. 11, Th. 17), there is a point $Q$ in $V$ such that it is a proper component of $\bigcap_{i}X_{i}$ and that $Q'$ is a specialization of $Q$ over the above specialization with reference to $\mathfrak{O}^{*}$. This proves our corollary.
\end{proof}

We\pageoriginale shall consider again, as in \S\ref{art14-chap1-sec1}, an algebraic family (irreducible) $\mathfrak{A}'$ with divisors in a projective space. We shall assume that every member $(U(a),X(a))$ satisfies the conditions that $U(a)$ {\em is non-singular in codimension $1$ and that $X(a)$ is a positive divisor on $U(a)$.} Therefore, $a$ is a pair of the Chow-point of $U(a)$ and that of $X(a)$. Let $k$ be an algebraically closed field of rationality of $\mathfrak{A}'$ and $A'$ the locus of $a$ over $k$, where $a$ corresponds to a generic member of $\mathfrak{A}'$ over $k$.

\begin{subprop}\label{art14-app-prop2.2}
$\mathfrak{A}'$ and $A'$ being as above, assume that the following conditions are satisfied: {\rm(i)} when $a'\in A'$, then $l(X(a'))=l(X(a))$ where $a$ is a generic point of $A'$ over $k$; {\rm(ii)} when $f$ is a non-degenerate rational map of $U(a)$ defined by $X(a)$, then $f[U(a)]\neq 0$. Then the set $E$ of points $a'$ in $A'$ such that a non-degenerate rational map $f'$ of $U(a')$ defined by $X(a')$ has the property $\deg (f'[U(a')])<s=\deg (f[U(a)])$ is a $k$-closed subset of $A'$.
\end{subprop}

\begin{proof}
Since $X(a)$ is rational over $k(a)$, there is a non-degenerate rational map $f$ of $U(a)$, defined by $X(a)$, which is defined over $k(a)$. Let $\Gamma$ be the closure of the graph of $f$ and $t$ the Chow-point of $\Gamma$. We shall denote $\Gamma$ by $\Gamma(t)$. Let $w$ be the Chow-point of $f[U(a)]$. We shall denote $f[U(a)]$ also by $W(w)$. Let $T$ (resp. $W$ be the locus of $t$ (resp. $w$) over $k$ and $D$ the locus of $(a,t,w)$ over $k$. $D$ is then a subvariety of $A'\times T\times W$.

Let $W_{0}$ be the set of points $w'$ such that the corresponding $W(w')$ with the Chow-point $w'$ is irreducible and not contained in any hyperplane. Let $T_{0}$ be the set of points $t'$ such that the corresponding $\Gamma(t')$ with the Chow-point $t'$ is irreducible and $D_{0}=D\cap A'\times T_{0}\times W_{0}$. As is well known, $W_{0}$ is $k$-open on $W$ and $T_{0}$ is $k$-open on $T$. Hence $D_{0}$ is a closed subvariety of $A'\times T_{0}\times W_{0}$, defined over $k$. The set-theoretic projection of $D_{0}$ on $A'$ contains a $k$-open subset of $A'$ (c.f. \cite{art14-key28}). Let $D'$ be the largest $k$-open subset of $A'$ contained in this projection.

Let $a'\in D'$. There is a point $(a',t',w')\in D_{0}$. By our choice of $W_{0}$, $T_{0}$ and $D_{0}$, $\Gamma(t')$ is irreducible, $W(w')$ is irreducible, $\deg(W(w'))=\deg(W(w))=s$ and $W(w')$ is not contained in any hyper-plane.\pageoriginale Moreover, $(U(a'), X(a'),\Gamma(t'),W(w'))$ is a specialization of $(U(a), X(a),\Gamma(t),\break W(w))$ over $k$. Since linear equivalence is preserved by specializations and since specializations are compatible with the operations of intersection-product and algebraic projection (c.f. \cite{art14-key24}), it follows that $\Gamma(t)'$ is the closure of the graph of a non-degenerate map of $U(a')$ determined by $X(a')$ and $\pr_{2}\Gamma(t')=mW(w')$ if $pr_{2}\Gamma(t)=mW(w)$. Thus a point of $E$ cannot be contained in $D'$.

$A'-D'$ is a $k$-closed subset of $A'$. Let $A''$ be a component of it and $a'$ a generic point of $A''$ over $k$. Let $f'$ be a non-degenerate rational map of $U(a')$ defined by $X(a')$ and assume that $f'[U(a')]=0$. If $A''$ has another point $a''$, let $f''$ be a similar map of $U(a'')$ defined by $X(a'')$. We consider a curve $C$ on $A''$ which contains $a'$ and $a''$. The existence of such a curve is well known and easy to prove by using the theorem of Bertini. Normalizing $C$ and localizing it at a point corresponding to $a''$, we apply the result of Proposition \ref{art14-app-prop2.1}. Then we see that $f''[U(a'')]=0$ since specializations are compatible with the operation of algebraic projection. Assume this time that $f'[U(a')]\neq 0$. Consider a curve $C$ on $A'$ which contains $a$ and $a'$ and proceed as above. Then we see that $\deg(f[U(a)])=s\geq \deg(f'[U(a')])$. When $a''$, $f''$ are as above, we see also that $\deg(f'[U(a')])\geq \deg (f''[U(a'')])$ by the same technique. Therefore, choosing only those $A''$ such that $\deg(f'[U(a')])=s$ and repeating the above process, we get our proposition easily.
\end{proof}

\setcounter{corollary}{0}
\begin{corollary}\label{art14-app-coro1}
Let $s_{0}\leq s$ be a non-negative integer. Then the set $E_{s_{0}}$ of points $a'$ of $A'$ such that a non-degenerate rational map $f'$ defined by $X(a')$ has the property $\deg(f'[U(a')])\leq s_{0}$ is a $k$-closed subset of $A'$.
\end{corollary}

\begin{proof}
This follows easily from our proposition.
\end{proof}

\begin{corollary}\label{art14-app-coro2}
With the same notations and assumptions of our proposition, assume further that $f$ is a birational map. Then there is a $k$-open subset $A'_{0}$ of $A'$ such that the following conditions are satisfied by points $a'$ of $A'_{0}$: When $f'$ is a non-degenerate map of $U(a')$ defined by $X(a')$, $f'$ is a birational map and $\deg(f(U(a)))=\deg (f'(U(a')))$.
\end{corollary}

\begin{proof}
Using the same notations of the proof of our proposition, let $a'\in D'$. Then a point $(a',t',w')\in D_{0}$ was such that $\Gamma(t')$ is irreducible,\pageoriginale $W(w')$ is irreducible, $\deg(W(w))=\deg(W(w'))=s$ and $W(w')$ is not contained in any hyperplane. Then $\Gamma(t')$ is the closure of the graph of a birational map defined by $X(a')$. Therefore, it is easy to see that $D'$ satisfies our requirement as $A'_{0}$.
\end{proof}

\begin{corollary}\label{art14-app-coro3}
With the same notations and assumptions of our proposition, assume further that $f$ is a birational map and that $f(U(a))$ is non-singular. Then there is a $k$-open subset $A'_{0}$ of $A'$ such that the following conditions are satisfied by points $a'$ of $A'_{0}$: When $f'$ is a non-degenerate rational map of $U(a')$ defined by $X(a')$, $f'$ is a birational map, $f'(U(a'))$ is non-singular and that $\deg(f(U(a)))=\deg (f'(U(a')))$.
\end{corollary}

\begin{proof}
In the proofs of our proposition and corollary, above take $W_{0}$ to be the set of points $w'$ such that $W(w')$ is irreducible, non-singular and not contained in any hyperplane. $W_{0}$ is also a $k$-open subset of $W$. The rest of our proof will then be exactly the same as that of the above corollary.
\end{proof}

\begin{corollary}\label{art14-app-coro4}
With the same notations and assumptions of our proposition, assume that $f$ is not birational. Then there is a $k$-open subset $A''_{0}$ of points $a'$ of $A'$ with the following property: When $f'$ is a non-degenerate rational map defined by $X(a')$, $f'$ is not birational and $\deg (f[U(a)])=\deg(f'[U(a')])$.
\end{corollary}

\begin{proof}
The proof of Corollary \ref{art14-app-coro2} above goes through almost word for word when we make the following change: (i) ``birational'' should be changed to ``not birational''. It should be noted that $\pr_{2}\Gamma(t)=mW(w)$, $\pr_{2}\Gamma(t')=mW(w')$ and $m>1$ in the proof of our proposition since $f$ in our case is not birational.
\end{proof}

\begin{thebibliography}{99}
\bibitem{art14-key1} \textsc{W. L. Chow :}\pageoriginale The Jacobian variety of an algebraic curve, {\em Amer. Jour. Math.} 76 (1954), 453-476.

\bibitem{art14-key2} \textsc{W. L. Chow} and \textsc{J. Igusa :} Cohomology theory of varieties over rings, {\rm Proc. Nat. Acad. Sci. U.S.A.} 44 (1958), 1244-1248.

\bibitem{art14-key3} \textsc{W. L. Chow} and \textsc{v. d. Waerden :} Zur Algebraischen Geometrie IX, {\em Math. Ann.} (1937), 692-704.

\bibitem{art14-key4} \textsc{A. Grothendieck} et \textsc{J. Dieudonn\'e :} {\em El\'ements de g\'eom\'etrie alg\'ebrique}, Publ. Math. de l'inst. des Hautes E\'t. Sci., Paris, No. 4, 8, 11, 17, 20, 24, etc.

\bibitem{art14-key5} \textsc{H. Hironaka :} Resolution of singularities of an algebraic variety over a field of characteristic zero I-II, {\em Ann. Math.} 79 (1964), 109-326.

\bibitem{art14-key6} \textsc{W. V. D. Hodge :} {\em The theory and applications of harmonic integrals,} Cambridge Univ. Press (1958).

\bibitem{art14-key7} \textsc{J. Igusa :} On the Picard varieties attached to algebraic varieties, {\em Amer. Jour. Math.} 74 (1952), 1-22.

\bibitem{art14-key8} \textsc{J. Igusa :} Fibre systems of Jacobian varieties, {\em Amer. Jour. Math.} 78 (1956), 171-199.

\bibitem{art14-key9} \textsc{K. Kim :} Deformations, related deformations and a universal subfamily, {\em Trans. Amer. Math. Soc.} 121 (1966), 505-515.

\bibitem{art14-key10} \textsc{S. Kleiman :} {\em Toward a numerical theory of ampleness,} Thesis at Harvard Univ. (1965).

\bibitem{art14-key11} \textsc{K. Kodaira :} On a differential-geometric method in the theory of analytic stacks, {\em Proc. Nat. Acad. Sci. U.S.A.} 39 (1953), 1268-1273.

\bibitem{art14-key12} \textsc{K. Kodaira :} Pluricanonical systems on algebraic surfaces of general type, {\em to appear.}

\bibitem{art14-key13} \textsc{S. Koizumi :} On the differential forms of the first kind on algebraic varieties, {\em Jour. Math. Soc. Japan,} 1 (1949), 273-280.

\bibitem{art14-key14} \textsc{S. Lang :} {\em Abelian varieties,} Interscience Tracts, No. 7 (1959).

\bibitem{art14-key15} \textsc{T. Matsusaka :}\pageoriginale On the algebraic construction of the Picard variety, I-II, {\em Jap. J. Math.} 21 (1951), 217-236, Vol. 22 (1952), pp. 51-62.

\bibitem{art14-key16} \textsc{T. Matsusaka :} Algebraic deformations of polarized varieties, {\em to appear} in {\em Nagoya J}.

\bibitem{art14-key17} \textsc{T. Matsusaka} and \textsc{D. Mumford :} Two fundamental theorems on deformations of polarized varieties, {\em Amer. J. Math.} 86 (1964), 668-684.

\bibitem{art14-key18} \textsc{D. Mumford :} Pathologies III, {\em Amer. J. Math.} 89 (1967), 94-104.

\bibitem{art14-key19} \textsc{A. N\'eron} and \textsc{P. Samuel :} La vari\'et\'e de Picard d'une vari\'et\'e normale, {\em Ann. L'inst. Fourier}, 4 (1952), pp. 1-30.

\bibitem{art14-key20} \textsc{M. Rosenlicht :} Equivalence relations on algebraic curves, {\em Ann. Math.,} (1952), 169-191.

\bibitem{art14-key21} \textsc{J-P. Serre :} Faisceaux alg\'ebriques coherents, {\em Ann. Math.} 61 (1955), 197-278.

\bibitem{art14-key22} \textsc{J-P. Serre :} Groupes alg\'ebriques et corps de classes, {\em Act. Sci. Ind.} No. 1264.

\bibitem{art14-key23} \textsc{J-P. Serre :} Un th\'eor\`eme de dualit\'e, {\em Comm. Math. Helv.} 29 (1955), 9-26.

\bibitem{art14-key24} \textsc{G. Shimura :} Reduction of algebraic varieties with respect to a discrete valuation of the basic field, {\em Amer. J. Math.} 77 (1955), 134-176.

\bibitem{art14-key25} \textsc{A. Weil :} {\em Foundations of Algebraic Geometry}, Amer. Math. Soc. Col. Publ., No. 29 (1960).

\bibitem{art14-key26} \textsc{A. Weil :} Vari\'et\'es Abeliennes et Courbes Alg\'ebriques, {\em Act. Sci. Ind.} No. 1064 (1948).

\bibitem{art14-key27} \textsc{A. Weil :} Sur les crit\`eres d'equivalence en g\'eom\'etrie alg\'ebriques, {\em Math. Ann. 128} (1954), 95-127.

\bibitem{art14-key28} \textsc{A. Weil :} On algebraic groups of transformations, {\em Amer. J. Math.} 77 (1955), 355-391.

\bibitem{art14-key29} \textsc{O. Zariski :} Foundations of a general theory of birational correspondences, {\em Trans. Amer. Math. Soc.} 53 (1943), 490-542.

\bibitem{art14-key30} \textsc{O. Zariski :}\pageoriginale Reduction of singularities of algebraic three-dimensional varieties, {\em Ann. Math.} 45 (1944), 472-542.

\bibitem{art14-key31} \textsc{O. Zariski :} Complete linear systems on normal varieties and a generalization of a lemma of Enriques-Severi, {\em Ann. Math.} 55 (1952), 552-592.

\bibitem{art14-key32} \textsc{O. Zariski :} Scientific report on the second summer Institute, III, {\em Bull. Amer. Math. Soc.} 62 (1956), 117-141.

\bibitem{art14-key33} \textsc{O. Zariski :} Introduction to the problem of minimal models in the theory of algebraic surfaces, {\em Publ. Math. Soc. Japan,} No. 4 (1958).

\bibitem{art14-key34} \textsc{S. Abhyankar :} Local uniformization on algebraic surfaces over ground fields of characteristic $p\neq 0$, {\em Ann. Math.} 63 (1956), 491-526.

\bibitem{art14-key35} \textsc{S. Abhyankar :} {\rm Resolution of singularities of embedded algebraic surfaces,} Academic Press (1966).

\end{thebibliography}

\bigskip
\noindent
{\small Brandeis University.}
