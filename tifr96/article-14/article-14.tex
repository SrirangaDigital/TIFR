\title{ON CANONICALLY POLARIZED VARIETIES}
\markright{On Canonically Polarized Varieties}

\author{By~~ T. Matsusaka$^{*}$}

\footnotetext{This work was done while the author was partially supported by N. S. F.}
\date{}

\maketitle

\setcounter{pageoriginal}{264}
\section*{Brief summary of notations and conventions}
\pageoriginale

We shall follow basically notations and conventions of \cite{art14-key21}, \cite{art14-key25}, \cite{art14-key32}. For basic results on specializations of cycles, we refer to \cite{art14-key24}. When $U$ is a complete variety, non-singular in codimension $1$ and $X$ a $U$-divisor, the module of functions $g$ such that $\div (g)+X\succ 0$ will be denoted by $L(X)$. We shall denote by $l(X)$ the dimension of $L(X)$. The complete linear system determined by $X$ will be denoted by $\Lambda(X)$. A finite set of functions $(g_{i})$ in $L(X)$ defines a rational map $f$ of $U$ into a projective space. $f$ will be called a {\rm rational map of $U$ defined by $X$.} When $(g_{i})$ is a basis of $L(X)$, it will be called a {\em non-degenerate map}. $X$ will be called {\em ample} if a non-degenerate $f$ is a projective embedding. It will be called {\em non-degenerate} if a positive multiple of $X$ is ample. (In the terminology of Grothendieck, there are called {\em very ample} and {\em ample}). Let $W$ be the image of $U$ by $f$ and $\Gamma$ the closure of the graph of $f$ on $U\times W$. We shall denote by $\deg (f)$ the number $[\Gamma:W]$. For any $U$-cycle $Y$, we shall denote by $f(Y)$ the cycle $pr_{2}(\Gamma\cdot (Y\times W))$. We shall denote by $\mathfrak{K}(U)$  a {\em canonical divisor} of $U$. When $X$ is a Cartier divisor on $U$, we shall denote by $\mathscr{L}(X)$ the invertible sheaf defined by $X$. When $U$ is a subvariety of a projective space, $C_{U}$ will denote a hyperplane section of $U$. When $U$ is a polarized variety, a {\em basic polar divisor} will be denoted by $X_{U}$.

By an {\em algebraic family of positive cycles}, we shall understand the set of positive cycles in a projective space such that the set of Chow-points forms a {\em locally closed} subset of a projective space. By identifying these cycles with their Chow-points, some of the notations and results on points can be carried over to algebraic families and this will be done frequently.

Finally, $\mathfrak{G}_{l}$, $\mathfrak{G}_{a}$, $\sim$ will denote respectively the group of divisors linearly equivalent to zero, the group of divisors algebraically equivalent to zero and the linear equivalence of divisors.

\section*{Introduction}\pageoriginale

Let $V^{n}$ be a polarized variety and $X_{V}$ a basic polar divisor on $V$. Then the Euler-Poincar\'e characteristic $\chi(V,\mathscr{L}(mX_{V}))$ is a polynomial $P(m)$ in $m$. We have defined this to be the {\em Hilbert characteristic polynomial} of $V$ (c.f. \cite{art14-key16}). If $d$ is the rank of $V$, i.e. $d=X^{(n)}_{V}$, any algebraic deformation of $V$ of rank $d$ has the same Hilbert characteristic polynomial $P(m)$ (c.f. \cite{art14-key16}). As we pointed out in \cite{art14-key16}, if we can find a constant $c$, which depends only on $P(m)$, such that $mX_{V}$ is ample for $m\geq c$, then the existence of a universal family of algebraic deformations of $V$ of bounded ranks follows. The existence of such a constant is well-known in the case of curves and Abelian varieties. We solved this problem for $n=2$ in \cite{art14-key17} (compare \cite{art14-key9}, \cite{art14-key10}, \cite{art14-key12}). But the complexity we encountered was of higher order of magnitude compared with the case of curves. The same seems to be the case for $n\geq 3$ when compared with the case $n=2$. One of the main purposes of this paper is to solve the problem for $n=3$ when $V$ is ``generic'' in the sense that $\mathfrak{K}(V)$ is a non-degenerate polar divisor.

In general, let us consider the following problems.
\begin{itemize}
\item[$(A_{n})$] Find a constant $c$, which depends on the polynomial $P(x)$ only, such that $h^{i}(V,\mathscr{L}(mX_{V}))=0$ for $i>0$ whenever $m\geq c$.

\item[$(A'_{n})$] Find two constants $c$, $c'$, which depend on $P(x)$ only, such that $h^{i}(V,\mathscr{L}(mX_{V}))<c'$ for $i>0$ whenever $m\geq c$.

\item[$(A''_{n})$] Find two constants $c$, $c'$, which depend on $P(x)$ only, such that $|l(mX_{V})-P(m)|<c'$ whenever $m\geq c$.

\item[$(B_{n})$] Find a constant $c$, which depends on $P(x)$ only, such that $mX_{V}$ defines a birational map of $V$ whenever $m\geq c$.

\item[$(C_{n})$] Find a constant $c$, which depends on $P(x)$ only, such that $mX_{V}$ is ample whenever $m\geq c$.
\end{itemize}
As we mentioned, what we are interested in is the solution of $(C_{n})$. But $(A_{n})$, $(B_{n})$ could be regarded as step-stones for this purpose. It is easy to see that the solution of $(C_{n})$ implies the solutions of $(A_{n})$ and $(B_{n})$. In the case of characteristic zero and $\mathfrak{K}(V)$ a non-degenerate\pageoriginale polar divisor, $(A_{n})$ can be solved easily (\S1). Assuming that $(A''_{n})$ has a solution, we shall show that $(B_{n})$ has a solution for $n=3$ when the characteristic is zero. Assuming that $(A_{n})$ and $(B_{n})$ have solutions and that $\mathfrak{K}(V)$ is a non-degenerate polar divisor, we shall show that $(C_{n})$ has a solution when the characteristic is zero. Hence $(C_{3})$ has a solution when the characteristic is zero and $\mathfrak{K}(V)$ is a non-degenerate polar divisor.

\bigskip

\begin{center}
{\Large\bf Chapter I. \boldmath$(A''_{n})$ and $(B_{n})$}\labeltext{1}{art14-chap1}
\end{center}

\section{Canonically polarized varieties}\label{art14-chap1-sec1}

We shall first recall the definition of a polarized variety as modified in \cite{art14-key16}. Let $V^{n}$ be a complete non-singular variety and $\mathscr{M}$ a finite set of prime numbers consisting of the characteristic of the universal domain (or the characteristics of universal domains) and the prime divisors of the order of the torsion group of divisors of $V$. Let $\mathscr{X}$ be a set of $V$-divisors satisfying the following conditions: (a) $\mathscr{X}$ contains an ample divisor $X$; (b) a $V$-divisor $Y$ is contained in $\mathscr{X}$ if and only if there is a pair $(r,s)$ of integers, which are prime to members of $\mathscr{M}$, such that $rY\equiv sX\mod \mathfrak{K}_{a}$. When there is a pair $(\mathscr{M},\mathscr{X})$ satisfying the above conditions, $\mathscr{X}$ is called a {\em structure set} of polarization and $(V,\mathscr{X})$ a {\em polarized variety}. A divisor in $\mathscr{X}$ will be called a {\em polar divisor} of the polarized variety. There is a divisor $X_{V}$ in $\mathscr{X}$ which has the following two properties: (a) a $V$-divisor $Y$ is in $\mathscr{X}$ if and only if $Y\equiv rX_{V}\mod \mathfrak{K}_){a}$ where $r$ is an integer which is prime to members of $\mathscr{M}$; (b) when $Z$ is an ample polar divisor, there is a positive integer $s$ such that $Z\equiv sX_{V}\mod \mathfrak{G}_{a}$ (c.f. \cite{art14-key16}). $X_{V}$ is called a {\em basic polar divisor}. The self-intersection number of $X_{V}$ is called the {\em rank} of the polarized variety. A polarized variety will be called a {\em canonically polarized variety} if $\mathfrak{K}(V)$ is a polar divisor.

\begin{lemma}\label{art14-lem1}
Let $V^{n}$ be a canonically polarized variety and $P(x)=\Sigma_{\gamma_{n-i}}x^{i}$ the Hilbert characteristic polynomial of $V$. Then $\mathfrak{K}(V)\equiv \rho X_{V}\mod \mathfrak{K}_{a}$ where $rho$ is a root of $P(x)-(-1)^{n}\gamma_{n}=0$.
\end{lemma}

\begin{proof}
It follows from Serre's duality theorem that 
$$
\chi(V,\mathscr{L}(mX_{V}))=(-1)^{n}\chi(V,\mathscr{L}(\mathfrak{K}(V)-mX_{V}))\quad (\text{c.f. \cite{art14-key23}}).
$$\pageoriginale
We may replace $\mathfrak{K}(V)$ by $\rho X_{V}$ in this equality since $\chi$ is invariant under algebraic equivalence of divisors (c.f. \cite{art14-key4}, \cite{art14-key16}). Then we get $P(m)=(-1)^{n}P(\rho-m)$. Setting $m=0$, we get $(-1)^{n}P(\rho)=\gamma_{n}$. Our lemma is thereby proved.
\end{proof}

\begin{proposition}\label{art14-peop1}
Let $V$ be a canonically polarized variety in characteristic zero and $P(x)$ the Hilbert characteristic polynomial of $V$. Then there is a positive integer $\rho_{0}$, which depends on $P(x)$ only, such that $h^{i}(V,\mathscr{L}(Y))=0$ for $i>0$ and $h^{0}(V,\mathscr{L}(Y))=l(Y)>0$ whenever $m\geq \rho_{0}$ and $Y\equiv mX_{V}\mod \mathfrak{K}_{a}$.
\end{proposition}

\begin{proof}
Let $\gamma_{n}$ be the constant term of $P(x)$ and $s_{0}$ the maximum of the roots of the equation $P(x)-(-1)^{n}\gamma_{n}=0$. Let $\mathfrak{K}(V)$ be a canonical divisor of $V$ and $k$ an algebraically closed common field of rationality of $V$, $X_{V}$ and $\mathfrak{K}(V)$. There is an irreducible algebraic family $\mathfrak{H}$ of positive divisors on $V$, defined over $k$, such that, for a fixed $k$-rational divisor $C_{0}$ in $\mathfrak{H}$, the classes of the $C-C_{0}$, $C\in \mathfrak{H}$, with respect to linear equivalence exhausts the points of the Picard variety of $V$ (c.f. \cite{art14-key15}). We shall show that $2s_{0}$ can serve as $\rho_{0}$.

Take $t$ so that $t-(s_{0}-\rho)=t'>0$ and $t-2(s_{0}-\rho)=m>0$ where $\mathfrak{K}(V)\equiv \rho X_{V}\mod \mathfrak{K}_{a}$. For any $C'$ in $\mathfrak{H}$, $tX_{V}+C'-C_{0}+\mathfrak{K}(V)\equiv (t'+s_{0})X_{V}\mod \mathfrak{K}_{a}$ and $t'+s_{0}>0$. Hence $tX_{V}+C'-C_{0}+\mathfrak{K}(V)$ is non-degenerate (c.f. \cite{art14-key16}, Th. 1) and the higher cohomology groups of the invertible sheaf $\mathscr{A}$ determined by $tX_{V}+C'-C_{0}+2\mathfrak{K}(V)$ vanish by Kodaira vanishing theorem (c.f. \cite{art14-key11}). $tX_{V}+C'-C_{0}+2\mathfrak{K}(V)\equiv (m+2s_{0})X_{V}\mod \mathfrak{K}_{a}$ and $\chi(V,\mathscr{A})=P(m+2s_{0})$ since $\chi$ is invariant by algebraic equivalence of divisors. Moreover, $P(m+2s_{0})>0$ by our choice of $m$ and $s_{0}$. It follows that $h^{0}(V,\mathscr{A}>0$. Therefore, in order to complete our proof, it is enough to prove that a $V$-divisor $Z$ such that $Z\equiv (m+2s_{0})X_{V}\mod \mathfrak{K}_{a}$ has the property that $Z\sim tX_{V}+C'-C_{0}+2\mathfrak{K}(V)$ for some $C'$ in $\mathfrak{H}$. Clearly, such $Z$ is algebraically equivalent to $tX_{V}+2\mathfrak{K}(V)$. Hence
$$
Z-(tX_{V}+2\mathfrak{K}(V))\sim C'-C_{0}
$$
for some $C'$ in $\mathfrak{H}$. Our proposition is thereby proved.
\end{proof}

\section{Estimation of $l(C_{U})$ on a projective surface}\label{art14-sec2}\pageoriginale

Let $V$ be a non-singular surface in a projective space and $\Gamma$ a curve on $V$. Let $\mathfrak{R}$ be the intersection of local rings of $\Gamma$ at the singular points of $\Gamma$. Using only those functions of $\Gamma$ which are in $\mathfrak{R}$, we can define the concept of complete linear systems and associated sheaves, as on a non-singular curve. Let $\Gamma'$ be a $V$-divisor such that $\Gamma'\sim \Gamma$, that $\Gamma'$ and $\Gamma$ intersect properly on $V$ and that no singular point of $\Gamma$ is a component of $\Gamma\cdot \Gamma'$. Similarly let $\mathfrak{K}(V)$ be such that $\mathfrak{K}(V)$ and $\Gamma$ intersect properly on $V$ and that no singular point of $\Gamma$ is a component of $\Gamma\cdot \mathfrak{K}(V)$. Then $\Gamma\cdot (\Gamma'+\mathfrak{K}(V))=\mathfrak{K}(\Gamma)$ is a canonical divisor of $\Gamma$, $p_{a}(\Gamma)=1+\frac{1}{2}\deg (\mathfrak{K}(\Gamma))$ and the generalized Riemann-Roch theorem states that $l(\mathfrak{m})=\deg (\mathfrak{m})-p_{a}(\Gamma)+1+l(\mathfrak{K}(\Gamma)=\mathfrak{m})$ for a $\Gamma$-divisor $\mathfrak{m}$ (c.f. \cite{art14-key20}, \cite{art14-key22}).

If $C$ is a complete non-singular curve, the theorem  of Clifford states that $\deg (\mathfrak{m})\geq 2l(\mathfrak{m})-2$ for a {\em special} $C$-divisor $\mathfrak{m}$. We shall first extend this to $\Gamma$.

\begin{lemma}\label{art14-lem2}
Let $V$, $\Gamma$ and $\mathfrak{K}(\Gamma)$ be as above and $\mathfrak{m}$ a special positive divisor on $\Gamma$ (i.e. $l(\mathfrak{K}(\Gamma)-\mathfrak{m})>0$). Then $\deg (\mathfrak{m})\geq 2l(\mathfrak{m})-2$.
\end{lemma}

\begin{proof}
When $l(\mathfrak{m})=1$, our lemma is trivial since $\mathfrak{m}$ is positive. Therefore, we shall assume that $l(\mathfrak{m})>1$.

Let $\Gamma^{*}$ be a normalization of $\Gamma$, $\alpha$ the birational morphism of $\Gamma^{*}$ on $\Gamma$ and $T$ the graph of $\alpha$. For any $\Gamma$-divisor $\mathfrak{a}$, we set $\mathfrak{a}^{*}=\alpha^{-1}(\mathfrak{a})=pr_{\Gamma^{*}}((\mathfrak{a}\times \Gamma^{*})\cdot T)$. When the $f$ are elements of $L(\mathfrak{a})$, i.e. elements of $\mathfrak{R}$ such that $\div (f)+\mathfrak{a}\succ 0$, the $f\circ \alpha=f^{*}$ generate a module of functions on $\Gamma^{*}$ which we shall denote by $\alpha^{-1}L(\mathfrak{a})$. The module $L(\mathfrak{K}(\Gamma)-\mathfrak{m})$ is not empty since $\mathfrak{m}$ is special. Hence it contains a function $g$ in $\mathfrak{R}$. Let $\div(g^{*})=\mathfrak{m}^{*}+\mathfrak{n}^{*}-\mathfrak{K}(\Gamma)^{*}$. Let $N$ be the submodule of functions $f^{*}$ in $\alpha^{-1}L(\mathfrak{K}(\Gamma))$ defined by requiring $f^{*}$ to pass through $\mathfrak{n}^{*}$, i.e. requiring $f^{*}$ to satisfy: coefficient of $x$ in $\div (f^{*})\geq $ coefficient of $x$ in $\mathfrak{n}^{*}$ for each component $x$ of $\mathfrak{n}^{*}$. Let $\dim N=\dim L(\mathfrak{K}(\Gamma))-t$. Then $\mathfrak{n}^{*}$ imposes $t$ linearly independent conditions in $\alpha^{-1}L(\mathfrak{K}(\Gamma))$. When these $t$ linear conditions are imposed on the vector subspace $\alpha^{-1}L(\mathfrak{K}(\Gamma)-\mathfrak{m})$, we get the vector space generated by $g^{*}$ over the universal domain. It follows that $l(\mathfrak{K}(\Gamma)-\mathfrak{m})-t\leq 1$,\pageoriginale i.e. $t\geq l(\mathfrak{K}(\Gamma)-\mathfrak{m})-1$. Since $l(\mathfrak{K}(\Gamma))=p_{a}(\Gamma)$, we then get $p_{a}(\Gamma)-\dim N\geq (\mathfrak{K}(\Gamma)-m)-1$. $L(\mathfrak{m})$ has a basis $(h_{i})$ from $\mathfrak{R}$. The $h_{i}\cdot g$ are elements of $L(\mathfrak{K}(\Gamma))$ and $h_{i}^{*}\cdot g^{*}\in N$. Hence the multiplication by $g^{*}$ defines an injection of $\alpha^{-1}L(\mathfrak{m})$ into $N$. It follows that $l(\mathfrak{m})\leq \dim N$ and $p_{a}(\Gamma)-l(\mathfrak{m})\geq l(\mathfrak{K}(\Gamma)-\mathfrak{m})-1$. By the generalized Riemann-Roch theorem, we have $l(\mathfrak{K}(\Gamma)-\mathfrak{m})=l(\mathfrak{m}-\deg (\mathfrak{m})+p_{a}(\Gamma)-1$. When this is substituted above, we get the required inequality.
\end{proof}

\begin{proposition}\label{art14-prop2}
Let $V$ be a non-singular surface in a projective space such that $p_{g}\geq 1$. Let $C_{0}$ be a curve on $V$ such that the complete linear system $\Lambda (C_{0})$ is without fixed point and that $C^{(2)}_{0}>0$. Then $\dim \Lambda (C_{0})\leq \frac{1}{2}C^{(2)}_{0}+1$.
\end{proposition}

\begin{proof}
If $\dim \Lambda(C_{0})=0$, there is nothing to prove. Therefore we shall assume that $\dim \Lambda (C_{0})>0$. Let $\mathfrak{R}$ be the intersection of local rings of $C_{0}$ at the singular points of $C_{0}$. There is a canonical divisor $\mathfrak{K}(V)$ of $V$ whose support does not contain any singular point of $C_{0}$. Such $\mathfrak{K}(V)$ and $C_{0}$ intersect properly on $V$. There is a member $C$ of $\Lambda (C_{0})$ which does not contain any singular point of $C_{0}$. Such $C$ and $C_{0}$ intersect properly on $V$. We have 
$$
p_{a}(C_{0})=1+\frac{1}{2}\deg (C_{0}\cdot (C+\mathfrak{K}(V))
$$
and $C_{0}\cdot (C+\mathfrak{K}(V))=\mathfrak{k}$ is a canonical divisor of $C_{0}$. Let $\mathfrak{m}=C_{0}\cdot C$. By the generalized Riemann-Roch theorem, we get $l(\mathfrak{m})=\deg (\mathfrak{m})-p_{a}(C_{0})+1+l(\mathfrak{k}-\mathfrak{m})$ and $\mathfrak{k}-\mathfrak{m}=C_{0}\cdot \mathfrak{K}(V)$. By our assumption there is a function $f$, other than $0$, in the module $L(\mathfrak{K}(V))$. Since the support of $\mathfrak{K}(V)$ does not contain any singular point of $C_{0}$, $f$ is regular at these points. Let $f'$ be the function on $C_{0}$ induced by $f\cdot f'$ is then regular at every singular point of $C_{0}$ and is contained in $\mathfrak{R}$. Since $\div (f)+\mathfrak{K}(V)\succ 0$ it follows that $\div(f')+C_{0}$. $\mathfrak{K}(V)\succ 0$. Hence $f'\in L(C_{0}\cdot \mathfrak{K}(V))=L(\mathfrak{k}-\mathfrak{m})$, which proves that $l(\mathfrak{k}-\mathfrak{m})>0$ and that $\mathfrak{m}$ is a special $C_{0}$-divisor.

By Lemma \ref{art14-lem2} we have $\deg (\mathfrak{m})=C^{(2)}_{0}\geq 2l(\mathfrak{m})-2$. Every function $g$ in $L(C)$ induces a function $g'$ on $C_{0}$ contained in $\mathfrak{R}$ since $C_{0}\cdot C$ has no singular component on $C_{0}$. Hence $\div (g')+\mathfrak{m}\succ 0$ and it follows that $g'\in L(\mathfrak{m})$.\pageoriginale If $g'=0$, $\div(g)=C_{0}-C$. Consequently $l(\mathfrak{m})\geq l(C)-1$ and $C^{(2)}_{0}\geq 2l(C)-4$. Our proposition follows at once from this.

We shall recall here the definition of the {\em effective geometric genus} of an algebraic variety $W$. Let $W$ and $W'$ be complete normal varieties and assume that there is a birational morphism of $W'$ on $W$. Then $p_{g}(W)\geq p_{g}(W')$. Hence there is a complete normal variety $W''$, birationally equivalent to $W$, such that $p_{q}(W'')$ has the minimum value $p_{g}$ among the birational class of $W$. This $p_{g}$ is called the {\em effective geometric genus} of $W$. When $W$ is non-singular, $p_{g}(W)=p_{g}$ (c.f. \cite{art14-key13}, \cite{art14-key31}).
\end{proof}

\begin{proposition}\label{art14-prop3}
Let $U$ be an algebraic surface in a projective space and $C_{U}$ a hyperslane section of $U$. Assume that the effective geometric genus $p_{g}$ of $U$ is at least $1$. Let $\Lambda$ be the linear system of hyperplane sections of $U$ and denote by $C^{(2)}_{U}$ the degree of $U$. Then $\dim \Lambda \leq \frac{1}{2}C^{(2)}_{U}+1$.
\end{proposition}

\begin{proof}
Let $V$ be a non-singular surface in a projective space and $f$ a birational morphism of $V$ on $U$ (c.f. \cite{art14-key}). Let $k$ be an algebraically closed common field of rationality of $U$, $V$ and $f$. Let $T$ be the group of $f$ on $V\times U$ and $P$ the ambient projective space of $U$. We may assume that $U$ is not contained in any hyperplane. Then $T$ and $V\times H$ intersect properly on $V\times P$ for every hyperplane $H$. We set $f^{-1}(H)=\pr_{V}(T\cdot (V\times H))$. Let $\Lambda^{*}$ be the set of $f^{-1}(H)$. It is a linear system on $V$. Since $f$ is a morphism, it has no fixed point. Therefore it has no fixed component in particular. Let $H$ be a generic hyperplane over $k$ and $C_{U}=U\cdot H$. The $\pr_{V}(T\cdot (V\times H))=\pr_{V}(T\cdot (V\times C_{U}))$ where the latter intersection-product is taken on $V\times U$ (c.f. \cite{art14-key25}, Chap. VIII). Setting $f^{-1}(C_{U})=\pr_{V}(T\cdot (V\times C_{U}))$, $f^{-1}(C_{U})$ is a generic member of $\Lambda^{*}$ over $k$. Since $f$ is a birational transformation, every component of $f^{-1}(C_{U})$ has to appear with coefficient 1. It follows that $f^{-1}(C_{U})$ is an irreducible curve by the theorem of Bertini (c.f. \cite{art14-key25}, Chap. IX) and $\Lambda^{*}$ has no fixed point. Moreover, $p_{g}(V)=p_{g}$ and $\dim \Lambda = \dim \Lambda^{*}$.

Let $C_{U}$ and $C'_{U}$ be two independent generic members of $\Lambda$ over $k$. When $Q$ is a component of $C_{U}\cap C'_{U}$, it is a generic point of $U$ over $k$ and\pageoriginale is a proper component of multiplicity 1 on $U$. When that is so, we get $f^{-1}(C_{U})\cdot f^{-1}(C'_{U})=f^{-1}(C_{U}\cdot C'_{U})$ (c.f. \cite{art14-key25}, Chap. VIII). Then $\deg (C_{U}\cdot C'_{U})=\deg (f^{-1}(C_{U})\cdot f^{-1}(C'_{U}))$ and our proposition follows from these and from Proposition \ref{art14-prop2}.
\end{proof}

\section{A discussion on fixed components}\label{art14-sec3}

Let $V^{n}$ be a complete variety, non-singular in codimension 1 and $X$ a divisor on $V$. We denote by $\Lambda(X)_{\red}$ the reduced linear system determined uniquely by $\Lambda(X)$. Then $\Lambda(X)=\Lambda(X)_{\red}+F$ and $F$ is called the {\em fixed part} of $\Lambda(X)$. A component of $F$ is called {\em a fixed component} of $\Lambda(X)$.

\begin{lemma}\label{art14-lem3}
Let $V^{n}$ be a complete variety, non-singular in codimension $1$, $X$ a positive $V$-divisor and $F=\Sigma^{l}_{l}a_{i}F_{i}$ the fixed part of $\Lambda(X)$. Assume that $l((\alpha-1)X+F)>l((\alpha-1)X)$ for some positive integer $\alpha>1$ and that $X\neq F$. Then we have the following results: {\rm(a)} there is a positive divisor $F'=\Sigma^{l}_{1}a'_{i}F_{i}$ such that $F-F'\succ 0$, that $l((\alpha-1)X+F')=l((\alpha-1)X+F)$ and that $l((\alpha-1)X+F'-F_{j})<l((\alpha-1)X+F)$ for all $j$ with $F'-F_{j}\succ 0$; {\rm(b)} let $I$ be the set of indices $i$ such that $a'_{i}\neq 0$. Then the $F_{i}$, $i\in I$, are not fixed components of $\Lambda((\alpha-1)X+F')$.
\end{lemma}

\begin{proof}
(a) follows immediately from our assumption. Let $k$ be an algebraically closed common field of rationality of $V$, $X$ and for the components of $F$. Let $L$ be a generic divisor of $\Lambda((\alpha-1)X+F')_{\red}$ over $k$. The fixed part of $\Lambda((\alpha-1)X+F')$ is obviously of the form $\Sigma^{l}_{1}b_{s}F_{s}$. Suppose that $b_{i}\neq 0$ for some $i\in I$. Then $L+\Sigma^{l}_{1}b_{s}F_{s}-F_{i}$ is positive and is a member of $\Lambda((\alpha-1)X+F'-F_{i})$. Hence $l((\alpha-1)X+F')=l((\alpha-1)X+F'-F_{i})$ which is contrary to our choice of $F'$. Our lemma is thereby proved.
\end{proof}

\begin{lemma}\label{art14-lem4}
Using the same notations and assumptions of Lemma \ref{art14-lem3}, let $\beta=\Pi_{I}a'_{i}$. Then there is the smallest positive integer $\gamma$ satisfying $\beta(a_{i}-a'_{i})-\gamma a'_{i}\geq 0$ for $i\in I$ and $\beta(a_{i_{0}}-a'_{i_{0}})-\gamma a'_{i_{0}}=0$ for some $i_{0}\in I$. Moreover, $\Lambda(\beta\alpha X+\gamma(\alpha-1)X)$ has the property that $F_{i_{0}}$ is not a fixed component of it.
\end{lemma}

\begin{proof}
Let $Z$ be a generic divisor of $\Lambda(X)_{\red}$ over $k$. Then $\alpha X\sim (\alpha-1)X+\Sigma^{l}_{1}a_{i}F_{i}+Z=((\alpha-1)X+\Sigma_{I}a'_{i}F_{i})+\Sigma_{I}(a_{i}-a'_{i})F_{i}+F''+Z$ for\pageoriginale some positive divisor $F''$ which does not contain the $F_{i}$, $i\in I$, as components. By the above lemma, $\Lambda((\alpha-1)X+\Sigma_{I}a'_{i}F_{i})$ has the property that no $F_{i}$, $i\in I$, is a fixed component of it. Let $m$ be a positive integer. Then $(\beta+m)((\alpha-1)X+\Sigma_{I}a'_{i}F_{i})+\beta Z+\Sigma_{I}(\beta a_{i}-\beta a'_{i}-ma'_{i})F_{i}+\beta F''\sim \beta\alpha X+m(\alpha-1)X$. By what we have seen above. the $F_{i}$, $i\in I$, are not fixed components of the complete linear system determined by $(\beta+m)((\alpha-1)X+\Sigma_{I}a'_{i}F_{i})+\beta Z$. By the definition of $\beta$, the $\beta(a_{i}-a'_{i})$ are divisible by the $a'_{i}$. Hence we can find the smallest positive integer $\gamma$ as claimed in our lemma. Then there is an index $i_{0}\in I$ such that $F_{i_{0}}$ is not a component of $\Sigma_{I}(\beta a_{i}-\beta a'_{i}-\gamma a'_{i})F_{i}$. From these our lemma follows at once.
\end{proof}

\begin{proposition}\label{art14-prop4}
Let $V^{n}$ be a complete non-singular variety, $X$ a positive non-degenerate divisor on $V$ and $F$ the fixed part of $\Lambda(X)$. Assume that there is an integer $\alpha >1$ such that $l((\alpha-1)X+F)>l((\alpha-1)X)$ and that $X\neq F$. Let $d=X^{(n)}$ and $\mu(X,\alpha)=(d^{d}\alpha+d^{d+1}(\alpha-1))!$. Then there is a component $F_{i}$ of $F$ such that it is not a fixed component of the complete linear system determined by $\mu(X,\alpha)X$.
\end{proposition}

\begin{proof}
We shall estimate $\beta$ and $\gamma$ of Lemma \ref{art14-lem4}. Let $Z$ be as in the proof of Lemma \ref{art14-lem4} and $I(~ ,~ )$ denote the intersection number. Since $X\sim Z+\Sigma^{l}_{1}a_{i}F_{i}$, $X^{(n)}=I(X^{(n-1)},Z)+\Sigma^{l}_{1}a_{i}I(X^{(n-1)},F_{i})=d$. Hence $a_{i}>0$ and $\Sigma^{l}_{1}a_{i}<d$. It follows that $\beta<\Pi_{I}a_{i}<d^{d}\cdot \gamma$ satisfies $\beta(a_{i}-a'_{i})\geq \gamma a'_{i}$. Hence $\gamma\leq \beta(a_{i}-a'_{i})<\beta a_{i}<d^{d+1}$. Our proposition now follows from these and from Lemma \ref{art14-lem4}.
\end{proof}

\section{Estimation of some intersection numbers and its application}\label{art14-sec4}

Let $W^{n}$ be a non-singular projective variety and $T^{n-1}$ a subvariety of $W$. Let $Y$ be a non-degenerate divisor on $W$ and $k$ a common field of rationality of $W$, $T$ and $Y$. Let $m$ be a positive integer such that $l(mY)>1$ and $A_{1},\ldots,A_{n} n$ independent generic divisors of $\Lambda(mY)$ over $k$. Let the $D_{i}$ be the proper components of $|A_{1}|\cap \cdots \cap |A_{s}|$ on $W$ and $a_{i}=\text{coef.}_{D_{i}}(A_{1}\ldots A_{s})$. Let $I$ be the set of indices $i$ such that $D_{i}$ contains a generic point of $W$ over $k$. Then we define the symbol $I(A_{1}\ldots A_{s}/W,k)$ to denote $\Sigma_{I}a_{i}$. We shall denote by $(A_{1}\ldots A_{s}/W,k)$ the $W$-cycle $\Sigma_{I}a_{i}D_{i}$. Let the $E_{j}$ be the proper components of $|A_{1}|\cap \cdots \cap | A_{s}|\cap |T|$ on $W$ and $b_{j}=\text{coef.}_{E_{j}}(A_{1}\ldots A_{s}\cdot T)$. Let\pageoriginale $J$ be the set of indices $j$ such that $E_{j}$ contains a generic point of $T$ over $k$. Then we denote by $(A_{1}\ldots A_{s}\cdot T/T,k)$ and by $I(A_{1}\ldots A_{s}\cdot T/T,k)$ the $W$-cycle $\Sigma_{J}b_{j}E_{j}$ and the number $\Sigma_{J}b_{j}$ respectively.

\begin{lemma}\label{art14-lem5}
Let $W^{n}$ be a non-singular projective variety and $T^{n-1}$ a subvariety of $W$, both defined over a field $k$. Let $Y$ be a non-degenerate $W$-divisor, rational over $k$, and $m$ a positive integer such that $l(mY)>1$. Let $A_{1},\ldots,A_{s}$, $s\leq n$, be $s$ independent generic divisors of $\Lambda (mY)$ over $k$. Then we have the following inequalities: {\rm(a)} $I(A_{1}\ldots A_{s}/W,k)\leq m^{s}Y^{(n)}$; {\rm(b)} $I(A_{1}\ldots A_{s}\cdot T/T,k)\leq m^{s}I(Y^{(n-1)},T)$.
\end{lemma}

\begin{proof}
We shall prove only {\rm(b)}. {\rm(a)} can be proved similarly. We set $\Sigma b_{i}E_{i}=A_{1}*\ldots * A_{s}*T$. If the $b_{i}$ are zero for all $i$, our lemma is obviously true. Hence we shall assume that the $b_{i}$ are positive.

Let $r$ be a large positive integer such that $rm\ Y$ is ample and $C_{1},\ldots,C_{s},C'_{1},\ldots,C'_{n-s-1}n-1$ independent generic divisors of $\Lambda(rm\ Y)$ over $k$. Since $(A_{1}\ldots A_{s}\cdot T/T, k)\prec A_{1}*\ldots * A_{s} * T$, it follows that $I(A_{1}\ldots A_{s}\cdot T/T,k)<\Sigma b_{i}$. Since $Y$ is non-degenerate, $\Sigma b_{i}I(E_{i},Y^{(n-s-1)})\geq \Sigma b_{i}$. We have
$$
(1/(rm)^{n-s-1})\deg \{(A_{1}*\ldots * A_{s} * T)\cdot C'_{1}\ldots C'_{n-s-1}\}=\Sigma b_{i}I(E_{i},Y^{(n-s-1)}).
$$
The left hand side can be written as
$$
(1/r^{s}(rm)^{n-s-1})\deg \{(rA_{1} * \ldots * rA_{s} * T)C'_{1}\ldots C'_{n-s-1}\}.
$$

The $rA_{i}$ are members of $\Lambda(rm\ Y)$. Hence $((rA_{i}), T, (C'_{j}))$ is a specialization of $((C_{i}),T,(C'_{j}))$ over $k$. Let $\mathfrak{m}=C_{1}\ldots C_{s}\cdot T\cdot C'_{1}\ldots C'_{n-s-1}$ and $\mathfrak{m}'$ an arbitrary specialization of $\mathfrak{m}$ over $k$ over the above specialization. Then, when $Q$ is a component of $rA_{1}*\ldots * rA_{s}$. $T\cdot C'_{1}\ldots C'_{n-s-1}$ with the coefficient $v$, $Q$ appears exactly $v$ times in $\mathfrak{m}'$ by the compatibility of specializations with the operation of inter-section-product (c.f. \cite{art14-key24}). It follows that $\deg(\mathfrak{m})=(mr)^{n-1} I(Y^{(n-1)}, T)\geq \deg \{(rA_{1}*\ldots * rA_{s} * T)C'_{1}\ldots C'_{n-s-1}\}$. (b) follows easily from these.
\end{proof}

As an application of Lemma \ref{art14-lem5}, we shall prove the following proposition which we shall need later.

\begin{proposition}\label{art14-prop5}
Let\pageoriginale $W^{n}$ be a non-singular projective variety and $Y$ a non-degenerate divisor on $W$. Let $m$ be a positive integer such that $l(mY)>Y^{(n)}m^{n-1}+n-1$. Let $f$ be a non-degenerate rational map of $W$ defined by $mY$. Then $\deg (f)\neq 0$, i.e. the image of $W$ by $f$ has dimension $n$.
\end{proposition}

\begin{proof}
By our assumption, $l(mY)>Y^{(n)}m^{s}+s$ for $1\leq s\leq n-1$. Let $U^{s}$ be the image of $W$ by $f$. Then $\Lambda(mY)_{\red}$ consists of $f^{-1}(H)$ where $H$ denotes a hyperplane in the ambient space of $U$. Let $k$ be a common field of rationality of $W$, $Y$ and $f$ and $A_{1},\ldots,A_{s}$ (resp. $B_{1},\ldots,B_{s}$) independent generic divisors of $\Lambda(mY)$ (resp. $\Lambda(mY)_{\red}$) over $k$. As is well known and easy to prove by means of the intersection theory. $\deg(U)\leq I(B_{1}\ldots B_{s}/W,k)$. Moreover, $I(B_{1}\ldots B_{s}/W,k)=I(A_{1}\ldots A_{s}/W,k)$ as can be seen easily. Then we get $\deg (U)\leq m^{s}Y^{(n)}$ by Lemma \ref{art14-lem5}. Let $\Lambda$ be the linear system of hyperplane sections of $U$. We have $\dim \Lambda \leq m^{s}Y^{(n)}+s-1$ (c.f. \cite{art14-key17}). On the other hand, $l(mY)=\dim \Lambda (mY)+1=\dim \Lambda(mY)_{\red}+1=\dim \Lambda +1$. This contradicts our assumption if $s<n$. Our proposition is thereby proved.
\end{proof}

\section{A solution of $(B_{3})$, (I)}\label{art14-sec5}

First, we shall fix some notations which shall be used through the rest of this chapter. Let $V^{3}$ be a polarized variety of dimension 3 and $P(m)=\Sigma^{3}_{0}\gamma_{3-i}m^{i}$ the Hilbert characteristic polynomial of $V$. Let $d=X^{(3)}_{V}$. As is well known, $\gamma_{0}=d/3$! As we pointed out in our introduction, {\em we shall solve} $(B_{3})$ {\em under the assumption that $(A''_{3})$ has a solution}. As we showed in Proposition \ref{art14-prop1}, $(A''_{3})$ has a solution when $V$ is canonically polarized and the characteristic is zero. Therefore, {\em we shall assume that there are two constants $c_{0}$ and $c$, which depend on the polynomial $P(x)$ only, such that $|l(mX_{V})-P(m)|<c$ whenever $m\geq c_{0}$}. We shall denote by $\Sigma$ the set of polarized varieties of dimension 3 such that $P(x)$ is their Hilbert characteristic polynomial. We shall use $V$ to denote a ``variable element'' of $\Sigma$.

From our basic assumption, we can find a positive integer $e_{0}\geq c_{0}$, which depends on $P(x)$ only, such that $l(mX_{V})>dm^{2}+2$ for $m\geq e_{0}$. For such $m$ {\em a non-degenerate rational map of $V$ defined by $mX_{V}$\pageoriginale maps $V$ generically onto a variety of dimension $3$} by Proposition \ref{art14-prop5}. Moreover, when that is so, $\Lambda(mX_{V})_{\red}$ is irreducible, i.e. it contains an irreducible member (c.f. \cite{art14-key25}, Chap. IX).

The following lemma has been essentially proved in the course of the proof of Proposition \ref{art14-prop4}.

\begin{lemma}\label{art14-lem6}
Let $r\geq e_{0}$ and $F'=\Sigma^{v}_{1}b_{i}F_{i}$ the fixed part of $\Lambda (rX_{V})$. Then $v\leq r^{3}d$, $\Sigma^{v}_{1}b_{i}<r^{3}d$ and $b_{i}<r^{3}d$.
\end{lemma}

When $r$ is a positive integer, we shall set $P(rx)=P_{r}(x)$ to regard it as a polynomial in $x$. For two positive integers $r$ and $m$, we set $\delta(r,m)=3(\gamma_{0}r^{3})m^{2}-(1/rd)m^{2}+2c+2$; $\delta'(r,m)=2(\gamma_{0}r^{3})m^{2}+2c+2$.

\begin{lemma}\label{art14-lem7}
When a positive integer $r$ is given, it is possible to find a positive integer $\alpha'$, which depends on $P(x)$ and $r$ only, such that $P_{r}(m)-P_{r}(m-1)>\delta(r,m)$ and $>\delta'(r,m)$ whenever $m\geq \alpha'$.
\end{lemma}

\begin{proof}
$P_{r}(m)=\gamma_{0}r^{3}m^{3}+\cdots$ and $P_{r}(m)-P_{r}(m-1)=3(\gamma_{0}r^{3})m^{2}+\cdots$ Therefore such a choice of $\alpha'$ is possible.
\end{proof}

\begin{proposition}\label{art14-prop6}
There are positive integers $v$, $\overline{v}$, which depend on $P(x)$ only, and $e$, $\alpha$, which depend on a member $V$ of $\Sigma$, with the following properties : {\rm(a)} $e_{0}\leq e\leq v$, $\alpha \leq \overline{v}$; {\rm(b)} when $F=\Sigma^{l}_{1}a_{i}F_{i}$ is the fixed part of $\Lambda(e_{0}X_{V})$, there is a positive integer $t$ such that $t\leq 1<e^{3}_{0}d$ and that $F_{1},\ldots,F_{t}$ are not fixed components of $\Lambda(eX_{V})$; {\rm(c)} $P_{e}(m)-P_{e}(m-1)>\delta (e,m)$ and $>\delta'(e,m)$ for $m\geq \alpha$; {\rm(d)} when $F^{*}$ is the fixed part of $\Lambda(eX_{V})$, $l((\alpha-1)eX_{V}+F^{*})=l((\alpha-1)eX_{V})$.
\end{proposition}

\begin{proof}
We can choose $\alpha_{0}$, depending on $P(x)$ only, such that $P_{e_{0}}(m)-P_{e_{0}}(m-1)>\delta (e_{0},m)$ and $>\delta'(e_{0}.m)$ whenever $m\geq \alpha_{0}$ by Lemma \ref{art14-lem7}. Assume that $l((\alpha_{0}-1)e_{0}X_{V}+F)>l((\alpha_{0}-1)e_{0}X_{V})$. Then, after rearranging indices if necessary, we may assume that $F_{1}$ is not a fixed component of the complete linear system determined by $\mu(e_{0}X_{V},\alpha_{0})e_{0}X_{V}$ by Proposition \ref{art14-prop4}, where
$$
\mu (e_{0}X_{V}, \alpha_{0})=(s^{s}\alpha_{0}+s^{s+1}(\alpha_{0}-1)!\text{~~ and~~ } s=e^{3}_{0}d.
$$

Let $e_{1}=e_{0}\mu(e_{0}X_{V},\alpha_{0})$ and $F'$ the fixed part of $\Lambda (e_{1}X_{V})$. By Lemma \ref{art14-lem7} we can find a positive integer $\alpha_{1}$, depending on $P(x)$ only, so that $P_{e_{1}}(m)-P_{e_{1}}(m-1)>\delta (e_{1},m)$ and $>\delta'(e_{1},m)$ whenever $m\geq \alpha_{1}$.\pageoriginale Assume still that $l((\alpha_{1}-1)e_{1}X_{V}+F')>l((\alpha_{1}-1)e_{1}X_{V})$. Then the complete linear system determined by $\mu(e_{1}X_{V},\alpha_{1})e_{1}X_{V}$ has not $F_{2}$ as a fixed component, after rearranging indices if necessary (c.f. Proposition \ref{art14-prop4}).

Since $l<e^{3}_{0}d$ by Lemma \ref{art14-lem6}, this process has to terminate by at most $e^{3}_{0}d-1$ steps. Positive integers $e_{i}$, $\alpha_{i}$ we choose successively can be chosen so that they depend only on $P(x)$. To fix the idea, let us choose $\alpha_{i}$ as follows. Let $r_{i}$ be the largest among the set of roots of the equations $P_{e_{i}}(x)-P_{e_{i}}(x-1)-\delta (e_{i},x)=0$, $P_{e_{i}}(x)-P_{e_{i}}(x-1)-\delta'(e_{i},x)=0$ and let $\alpha_{i}=|r_{i}|+1$. Suppose that our process terminates when we reach to the pair $(e_{t},\alpha_{t})$. Then $t\leq l<e^{3}_{0}d$. From the definition of the function $\mu$ (c.f. Proposition \ref{art14-prop4}) and from our choice of the $\alpha_{i}$, we can find easily upperbounds $v$, $\overline{v}$ for $e_{t}$, $\alpha_{t}$ which depend on $P(x)$ only and not on $t$. Moreover, when $F''$ is the fixed part of $\Lambda(e_{t}X_{V})$, $F_{1},\ldots,F_{t}$ are not components of $F''$ and $l((\alpha_{t}-1)e_{t}X_{V}+F'')=l((\alpha_{t}-1)e_{t}X_{V})$ by our assumption made above. Our proposition follows from these at once.
\end{proof}

\section{A solution of $(B_{3})$, (II)}\label{art14-sec6}

In this paragraph, we shall keep the notations of Proposition \ref{art14-prop6}. Let $k$ be an algebraically closed common field of rationality of $V$ and $X_{V}$ and $T$ a generic divisor of $\Lambda(eX_{V})_{\red}$ over $k$. $\Lambda(\alpha eX_{V})$ is canonically associated to the module $L(\alpha eX_{V})$. Let $K$ be a field of rationality of $T$ over $k$ and $L'$ the module of rational functions on $T$ induced by $L(\alpha eX_{V})$. Let $g$ be a non-degenerate rational map of $T$ defined by $L'$, defined over $K$. Let $U$ be the image of $T$ by $g$. A non-degenerate rational map of $V$ defined by $meX_{V}$ maps $V$ generically onto a variety of dimension 3 since $me\geq e\geq e_{0}$ by Proposition \ref{art14-prop5}, whenever $m>0$. Since $T$ is a generic divisor of $\Lambda(eX_{V})_{\red}$ over $k$, it follows that $\dim U=2$. Let $h$ be a non-degenerate rational map of $T$ defined by the module of rational functions on $T$ which is induced by $L(eX_{V})$. Then the image $W$ of $T$ by $h$ has dimension 2 also as we have seen above.

We shall establish some inequalities.
\begin{equation}
\dim T_{r_{T}}\Lambda (\alpha eX_{V})>\frac{1}{2}e^{3}\alpha^{2}d-(1/ed)\alpha^{2}+1.\label{art14-eq6.1}
\end{equation}
In\pageoriginale fact, $\dim Tr_{T}\Lambda(\alpha eX_{V})=l(\alpha eX_{V})-l(\alpha eX_{V}-T)-1=l(\alpha eX_{V})-l((\alpha-1)eX_{V}+F^{*})-1=l(\alpha eX_{V})-l((\alpha-1)eX_{V})-1$ by our choice of $\alpha$ and $e$ (c.f. Proposition \ref{art14-prop6}). By Proposition \ref{art14-prop6}, (c) and by the equality $\gamma_{0}=d/6$, $l(\alpha eX_{V})-l((\alpha-1)eX_{V})>P_{e}(\alpha)-P_{e}(\alpha-1)-2c>\frac{1}{2}e^{3}\alpha^{2}d-(1/ed)\alpha^{2}+2$. Our inequality is thereby proved.

The following formulas are well known and easy to prove by means of the intersection theory (c.f. \cite{art14-key25}, Chap. VII).
\begin{align}
& \deg (g)\cdot \deg (U)=I(A_{1}\cdot A_{2}\cdot T/T,K),\notag\\
& \deg (h)\cdot \deg (W) = I(B_{1}\cdot B_{2}\cdot T/T,K),\label{art14-eq6.2}
\end{align}
where $A_{1}$, $A_{2}$ (resp. $B_{1}$, $B_{2}$) are independent generic divisors of $\Lambda (\alpha eX_{V})$ (resp. $\Lambda (eX_{V})$) over $K$.

Next we shall find an upperbound for $\deg(U$).
\begin{align}
& \deg (U)\leq (1/\deg(g))(e^{3}\alpha^{2}d-e^{2}\alpha^{2})\text{~~ if~~ } F^{*}\neq 0;\label{art14-eq6.3}\\
& \deg (U)\leq (1/\deg (g))e^{3}\alpha^{2}d\text{~~ if~~ } F^{*}=0.\notag
\end{align}
In fact, $A_{i}\sim \alpha eX_{V}$ and $X_{V}$ is non-degenerate. Applying Lemma \ref{art14-lem5} to our case, we get $I(A_{1}\cdot A_{2}\cdot T/T,K)\leq e^{2}\alpha^{2}I(X^{(2)}_{V}, T)$ and $T+F^{*}\sim eX_{V}$. We have $I(X^{(2)}_{V},T)=I(X^{(2)}_{V},eX_{V})-I(X^{(2)}_{V},F^{*})=ed-I(X^{(2)}_{V},F^{*})$. \eqref{art14-eq6.3} follows from these.

Let $\Lambda$ be the linear system of hyperplane sections of $U$. From \eqref{art14-eq6.3} and from Theorem 3 in \cite{art14-key17}, we immediately get
\begin{align}
& \dim \Lambda \leq (1/\deg (g))(e^{3}\alpha^{2}d-e^{2}\alpha^{2})+1\text{~~ if~~ }F^{*}\neq 0;\notag\\
& \dim \Lambda \leq (1/\deg(g))e^{3}\alpha^{2}d+1\text{~~ if~~ } F^{*}=0.\label{art14-eq6.4}
\end{align}

In order to estimate an upper bound for $\deg(g)$, it is enough to do so for $\deg(h)$. By doing so, we shall get
\begin{equation}
\deg (g)\leq e^{3}d.\label{art14-eq6.5}
\end{equation}
From \eqref{art14-eq6.2} we get $\deg(h)\leq I(B_{1}\cdot B_{2}\cdot T/T,K)$. By Lemma \ref{art14-lem5}, the latter is bounded by $e^{2}I(X^{(2)}_{V},T)$. This, in turn, is bounded by $e^{2}I(X^{(2)}_{V},T+F^{*})=e^{3}d$. This proves our inequality since $\deg(g)\leq \deg(h)$.

Combining \eqref{art14-eq6.4} and \eqref{art14-eq6.5}, we get
\begin{equation}
\dim \Lambda \leq (1/\deg (g))e^{3}\alpha^{2}d-(1/ed)\alpha^{2}+1\text{~~ if~~ } F^{*}\neq 0. \label{art14-eq6.6}
\end{equation}

As\pageoriginale is well known, $\dim Tr_{T}\Lambda(\alpha eX)=\dim \Lambda$. Consider now the case $F^{*}\neq 0$ first. From \eqref{art14-eq6.1} and \eqref{art14-eq6.6}, we get $(1/\deg(g))e^{3}\alpha^{2}d-(1/ed)\alpha^{2}+1>\frac{1}{2}e^{3}\alpha^{2}d-(1/ed)\alpha^{2}+1$. Hence $\deg(g)<2$ and conseqently $\deg(g)=1$  and $g$ is birational. From the definition of $g$, it follows that a non-degenerate rational map $f$ of $V$ defined by $\alpha eX_{V}$ induces on $T$ a birational map. When that is so, $f$ is a birational map since $T$ is a generic divisor of the linear system $\Lambda(eX_{V})_{\red}$ of positive dimension over $k$, a proof of which will be left as an exercise to the reader.

Next, consider the case when $F^{*}=0$. By Proposition \ref{art14-prop6}, we have $P_{e}(\alpha)-P_{e}(\alpha-1)>\delta'(e,\alpha)$. Then we get $\dim Tr_{T}\Lambda (\alpha eX_{V})>(1/3)e^{3}\alpha^{2}d+1$, a proof of which is quite similar to that of \eqref{art14-eq6.1}. Combining this with \eqref{art14-eq6.4}, we get $(1/\deg (g))e^{3}\alpha^{2}d>(1/3)e^{3}\alpha^{2}d$. Hence $\deg(g)<3$. This proves that $\deg(f)\leq 2$, a proof of which will be left to the reader again.

By Proposition \ref{art14-prop6}, $e$ is bounded by $v$ and $\alpha$ is bounded by $\overline{v}$. Let $\rho_{1}=(v \ \overline{v})$! We shall summarize the results of this paragraph as follows.

\begin{proposition}\label{art14-prop7}
There is a constant $\rho_{1}$, which depends on $P(x)$ only, such that the following properties hold: {\rm(a)} if $\Lambda(eX_{V})$ has a fixed component, $\rho_{1}X_{V}$ defines a birational transformation of $V$; {\rm(b)} if $\Lambda(eX_{V})$ has no fixed component, a non-degenerate rational map of $V$ defined by $\rho_{1}X_{V}$ is of degree at most $2$; {\rm(c)} $\rho_{1}$ is divisible by $e$ and $\alpha$.
\end{proposition}

\section{A solution of $(B_{3})$, (III)}\label{art14-sec7}

Proposition \ref{art14-prop7} solves our problem in the case when $\Lambda(eX_{V})$ has a fixed component. In order to treat the other case, we shall review the concept of minimum sums of linear systems and fix some notations.

In general, let $W$ be a variety and $\mathscr{M}$, $\mathscr{N}$ two modules of rational functions of finite dimensions on $W$. We shall denote by $\Lambda(\mathscr{M})$, $\Lambda(\mathscr{N})$ the reduced linear systems on $W$ defined by these two modules. Let $\mathscr{R}$ be the module generated by $f\cdot g$ with $f\in \mathscr{M}$ and $g\in \mathscr{N}$. Then the reduced linear system $\Lambda(\mathscr{R})$ is known as the {\em minimum sum} of $\Lambda(\mathscr{M})$ and $\Lambda(\mathscr{N})$. We shall denote this minimum sum by $\Lambda(\mathscr{M})\oplus \Lambda(\mathscr{N})$. When\pageoriginale $\Lambda$ is a reduced linear system on $W$ and $\Lambda'$ the minimum sum of $r$ linear systems equal to $\Lambda$, we shall write $\oplus^{r}\Lambda$ for $\Lambda'$.

Let $W'$ be another variety and $\beta$ a rational map of $W'$ into $W$. Let $\Lambda$ be the closure of the graph of $\beta$ on $W'\times W$ and $\Lambda$ a linear system of divisors on $W$. Let $\Lambda'$ be the set of $W'$-divisors $\beta^{-1}(Z)=\pr_{W'}(\Gamma\cdot (W'\times Z))$ with $Z\in \Lambda$. Then $\Lambda'$ is a linear system of $W'$-divisors and this will be denoted by $\beta^{-1}(\Lambda)$.

\begin{lemma}\label{art14-lem8}
Let $f$ be a non-degenerate rational map of $V$ defined by $\rho_{1}X_{V}$ and $W$ the image of $V$ by $f$. When $N$ is the dimension of the ambient space of $W$, the following inequalities hold:
$$
P(\rho_{1})-c-1\leq N\leq P(\rho_{1})+c-1; \ \ \deg(W)\leq \rho^{3}_{1}d.
$$
\end{lemma}

\begin{proof}
$N=l(\rho_{1}X_{V})-1$. Hence the first inequality follows from $(A''_{3})$. Let $k$ be a common field of rationality of $V$, $X_{V}$ and $f$ and $Z_{1}$, $Z_{2}$, $Z_{3}$ independent generic divisors of $\Lambda(\rho_{1}X_{V})$ over $k$. Then $\deg(W)\leq I(Z_{1}\cdot Z_{2}\cdot Z_{3}/V,k)\leq \rho^{3}_{1}d$ by Lemma \ref{art14-lem5}. Our lemma is thereby proved.
\end{proof}

\begin{coro*}
Let $k_{0}$ be the algebraic closure of the prime field. There is a finite union of irreducible algebraic families of irreducible varieties in projective spaces, all defined over $k_{0}$, such that, when $V\in \Sigma$ and when $f$, $W$ are as in our lemma, $W$ is a member of at least one of the irreducible families.
\end{coro*}

\begin{proof}
This follows at once from our lemma and from the main theorem on Chow-forms (c.f. \cite{art14-key3}).
\end{proof}

In the following lemma, {\em we shall assume that global resolution, dominance and birational resolution in the sense of Abhyankar hold for algebraic varieties of dimension $n$} (c.f. \cite{art14-key35}). These hold for characteristic zero (c.f. \cite{art14-key5}) and for algebraic varieties of dimension 3 when the characteristic is different from 2, 3 and 5 (c.f. \cite{art14-key35}).

\begin{lemma}\label{art14-lem9}
Let $\mathfrak{F}$ be an irreducible algebraic family of irreducible varieties in a projective space and $k$ an algebraically closed field over which $\mathfrak{F}$ is defined. Then $\mathfrak{F}$ can be written as a finite union $\bigcup_{j}\mathfrak{F}_{j}$ of irreducible algebraic families, all defined over $k$, with the following properties:\pageoriginale {\rm(a)} $\mathfrak{F}_{i}\cap \mathfrak{F}_{j}=\emptyset$ whenever $i\neq j$; {\rm(b)} for each $j$, there is an irreducible algebraic family $\mathfrak{H}_{j}$ of non-singular varieties in a projective space, defined over $k$; {\rm(c)} when $W_{i}$ is a generic member of $\mathfrak{F}_{i}$ over $k$, there is a generic member $W^{*}_{i}$ of $\mathfrak{H}_{i}$ over $k$ and a birational morphism $\phi_{i}$ of $W^{*}_{i}$ on $W_{i}$ such that $W^{*}_{i}$, $\phi_{i}$ are defined over an algebraic extension of the smallest field of definition of $W_{i}$ over $k$; {\rm(d)} when $W'_{i}$ is a member of $\mathfrak{F}_{i}$, $\Gamma_{i}$ the graph of $\phi_{i}$ and when $(\Gamma'_{i},W^{*'}_{i})$ is an arbitrary specialization of $(\Gamma_{i},W^{*}_{i})$ over $k$ over the specialization $W_{i}\to W'_{i}$ ref. $k$, $W^{*'}_{i}$ is a member of $\mathfrak{H}_{i}$ and $\Gamma'_{i}$ is the graph of a birational morphism $\phi'_{i}$ of $W^{*'}_{i}$ on $W_{i}$; {\rm(e)} when $C'_{i}$ is a generic hyperplane section of $W'_{i}$ over a filed of definition of $W'_{i}$, ${\phi'}^{-1}_{i}(C'_{i})$ is non-singular.
\end{lemma}

\begin{proof}
Let $W$ be a generic member of $\mathfrak{F}$ over $k$ and $K$ the smallest field of definition of $W$ over $k$. There is a non-singular variety $W^{*}$ in a projective space and a birational morphism $\phi$ of $W^{*}$ on $W$, both defined over $\overline{K}$. For the sake of simplicity, replace $W^{*}$ by the graph of $\phi$. Then $\phi$ is simply the projection map. Therefore, we can identify the graph of $\phi$ with $W^{*}$. A multiple projective space cna be identified with a non-singular subvariety of a projective space by the standard process. Chow-points of positive cycles in a multiple projective space can then be defined by means of the above process. Let $C$ be a hyperplane section of $W$, rational over $K$. $W^{*}$ can be chosen in such a way that $\phi^{-1}(C)=U$ is non-singular. Let $w$, $w^{*}$, $u$ be respectively the Chow-points of $W$, $W^{*}$, $U$ and $\overline{T}$, $\overline{T}^{*}$ the locus of $w$, $(w^{*},u)$ over $k$. An open subset $T$ of $\overline{T}$ over $k$ is the Chow-variety of $\mathfrak{F}$. The set $T^{*}$ of points on $\overline{T}^{*}$ corresponding to pairs of non-singular varieties is $k$-open on $\overline{T}^{*}$ as can be verified without much difficulty. Let $Z$ be the locus of $(w,w^{*},u)$ over $k$ on $T\times T^{*}$. Let $T_{1}$ be the set of points of $T$ over which $Z$ is complete (i.e. proper). Then $T_{1}$ is non-empty and $k$-open (c.f. \cite{art14-key25}, Chap. VII, Cor., Prop. 12). The set-theoretic projection of the restriction $Z_{1}$ of $Z$ on $T_{1}\times T^{*}$ on $T_{1}$ contains a non-empty $k$-open set $F_{1}$. Let $H_{1}$ be the locus of $w^{*}$ over $k$. Then the families $\mathfrak{F}_{1}$, $\mathfrak{H}_{1}$ defined by $F_{1}$, $H_{1}$ satisfy (b), (c), (d) and (e) of our lemma which is not difficult to verify. $T_{1}-F_{1}$ can be written as a finite union of locally closed irreducible\pageoriginale subvarieties of $T$, defined over $k$, such that no two distinct components have a point in common. Then we repeat the above for each irreducible component to obtain the lemma.
\end{proof}

\begin{coro*}
There are two finite sets of irreducible algebraic families $\{\mathfrak{F}_{i}\}$, $\{\mathfrak{H}_{i}\}$ with the following properties : {\rm(a)} when $V\in \Sigma$ and $f$ a non-degenerate rational map of $V$ defined by $\rho_{1}X_{V}$, there is an index $i$ such that the image $W$ of $V$ by $f$ is a member of $\mathfrak{F}_{i}$; {\rm(b)} every member of the $\mathfrak{H}_{i}$ is a non-singular subvariety of a projective space; {\rm(c)} the $\mathfrak{F}_{i}$ and the $\mathfrak{H}_{i}$ satisfy {\rm(c), (d)} and {\em(e)} of our lemma.
\end{coro*}

\begin{proof}
This follows from the Corollary of Lemma \ref{art14-lem7}, Lemma \ref{art14-lem8} and from the basic assumptions on resolutions of singularities.
\end{proof}

\begin{lemma}\label{art14-lem10}
Let the characteristic be zero, $V\in\Sigma$ and $f$ a non-degenerate rational map of $V$ defined by $\rho_{1}X_{V}$. Let $W$ be the image of $V$ by $f$. Then there is a non-singular projective variety $W^{*}$ and a birational morphism $\phi$ of $W^{*}$ on $W$ with the following properties: {\rm(a)} when $k$ is a common field of rationality of $V$, $X_{V}$, $f$ and $\phi$ and when $C_{W}$ is a generic hyperplane section of $W$ over $k$, $\phi^{-1}(C_{W})$ is non-singular; {\rm(b)} when $C_{W}$ and $C'_{W}$ are independent generic over $k$, $\phi^{-1}(C_{W})\cdot \phi^{-1}(C'_{W})$ is non-singular; {\rm(c)} there is a positive integer $\pi$, which depends only on $P(x)$, such that $|p_{a}(\phi^{-1}(C_{W}))|<\pi$.
\end{lemma}

\begin{proof}
When a non-singular subvariety of a projective space is specialized to another such variety over a discrete valuation ring, the virtual arithmetic genus is not changed (c.f. \cite{art14-key2}, \cite{art14-key4}). Take the $\mathfrak{F}_{i}$, $\mathfrak{H}_{i}$ as in the Corollary of Lemma \ref{art14-lem9} and take $W^{*}$ from a suitable $\mathfrak{H}_{i}$. Then there is a birational morphism $\phi$ of $W^{*}$ on $W$, satisfying (a). (c) follows from (a) when we take the above remark into account. (a) and (b) follow easily also from the theorem of Bertini on variable singularities since the characteristic is zero.
\end{proof}

\begin{lemma}\label{art14-lem11}
There is a constant $\rho_{2}$, which depends on $P(x)$ only, such that $m\rho_{1}X_{V}$ has the following properties for $m\geq \rho_{2}$, provided that it does not define a birational map and the characteristic is zero: {\rm(a)} when $f'$ is a non-degenerate rational map of $V$ defined by $m\rho_{1}X_{V}$, $k'$ a field of rationality of $V$ and $X_{V}$ and $T$ a generic divisor of\pageoriginale $\oplus^{m}\Lambda(\rho_{1}X_{V})$ over $k'$, $T$ is irreducible and the effective geometric genus of the proper transform of $T$ by $f'$ is at least $2$; {\rm(b)} $\deg (f')=2$ and $f'$ induces on $T$ a rational map of degree $2$.
\end{lemma}

\begin{proof}
Let $f$, $W$, $W^{*}$, $\phi$, $k$ be as in Lemma \ref{art14-lem10}. Let $C_{W}$ be generic over $k$ and $U=\phi^{-1}(C_{W})$. Let $U'$ be a generic specialization of $U$ over $k$, other than $U$. By the modular property of $p_{a}$\footnote{This can be proved exactly in the same way as Lemma 5 of \cite{art14-key31} because of our Lemma \ref{art14-lem10}.}, we get $p_{a}(mU)=mp_{a}(U)+\Sigma^{m-1}_{1}p_{a}(sU'\cdot U)$. Applying the modular property again to $p_{a}(sU'\cdot U)$ on $U$, which is non-singular by Lemma \ref{art14-key10} we get the following equality:
$$
sp_{a}(U'\cdot U)+\frac{1}{2}s(s-1)(U'\cdot U)^{(2)}-s-1=p_{a}(sU'\cdot U).
$$
From the definition of $U$, it is clear that $U^{(3)}>0$ on $W^{*}$. Hence $(U'\cdot U)^{(2)}=U^{(3)}>0$. Moreover $|p_{a}(U)|<\pi$, by Lemma \ref{art14-lem10}. Using these and $\Sigma^{m-1}_{1}s(s-1)=(m-1)m(2m-1)/6$, we get
$$
p_{a}(mU)>-m(\pi+1)-m(m-1)+(m-1)m(2m-1)/12+1.
$$
We can find a positive integer $\rho_{2}$, which depends on $P(x)$ only, such that the right hand side of the above inequality is at least 2 whenever $m\geq \rho_{2}$. When that is so, any member $A$ of $\Lambda(mU)$ satisfies $p_{a}(A)>1$ since the virtual arithmetic genus of divisors is invariant with respect to linear equivalence.

Let $\Lambda$ be the linear system of hyperplane sections of $W$. Clearly $\phi^{-1}(\oplus^{m}\Lambda)=\phi^{m}\phi^{-1}(\Lambda)$ and the latter contains a non-singular member $A$ by the theorem of Bertini on variable singularities. Let $p_{g}$, $p_{a}$, $q$ denote respectively the effective geometric genus, effective arithmetic genus and the irregularity of $A$. Then $q=p_{g}-p_{a}$ and $p_{a}(A)=p_{a}$. Since $q\geq 0$, it follows that $p_{g}>1$ whenever $m\geq \rho_{2}$. When $C_{m}$ is a generic divisor of $\oplus^{m}\Lambda$ over $k$, we can take for $A$ the variety $\phi^{-1}(C_{m})$. Therefore, the effective geometric genus of $C_{m}$ is at least 2 when $m\geq \rho_{2}$.

Assume that $f'$ is not birational for some $m\geq \rho_{2}$ and rational over $k$. Then $\Lambda(m\rho_{1}X_{V})$ has no fixed component and $\deg(f')=2$ by Proposition \ref{art14-prop7}. By the same proposition, the same is true for $\Lambda(\rho_{1}X_{V})$\pageoriginale and $f$. Let $W'$ be the image of $V$ by $f'$. Since $\deg(f)=\deg(f')$, there is a birational transformation $h$ between $W'$ and $W$ such that $f=h\circ f'$ holds generically. Then $f^{-1}(C_{m})=T$ is irreducible (c.f. \cite{art14-key25}, Chap. IX) and is a generic divisor of $\oplus^{m}f^{-1}(\Lambda)=\oplus^{m}\Lambda(\rho_{1}X_{V})$ over $k$. Let $L$ be the proper transform of $T$ by $f'$. $C_{m}$ and $L$ are birationally corresponding subvarieties of $W$ and $W'$ by $h^{-1}$ and, when that is so, the effective geometric genus of $L$ is at least $2$. Our lemma follows easily from this.

Let now $f'$ denote a non-degenerate rational map of $V$ defined by $\rho_{2}\rho_{1}X_{V}$, and {\em assume that $f'$ is not birational}. By Proposition \ref{art14-prop7}, $\deg(f')=2$ and $\Lambda(\rho_{2}\rho_{1}X_{V})$ has no fixed component. By Lemma \ref{art14-lem11}, the complete linear system contains a linear pencil whose generic divisor $T$ has the property that its proper transform $D$ by $f'$ has the effective geometric genus which is at least 2.

Let $f$ be a non-degenerate rational map of $V$ into a projective space defined by $m\rho_{2}\rho_{1}X_{V}$ and {\em assume that $f$ has still the property that $\deg(f)=2$}. Let $E$ be the proper transform of $T$ by $f$ and $g$ the rational map induced on $T$ by $f$. Then $D$ and $E$ are clearly birationally equivalent and the effective geometric genus of $E$ is at least $2$. As in \eqref{art14-eq6.1}, $\dim Tr_{T}\Lambda(m\rho_{2}\rho_{1}X_{V})=l(mT)-l(mT-T)-1>P_{\rho_{1}\rho_{2}}(m)-P_{\rho_{1}\rho_{2}}(m-1)-2c-1$. The leading coefficient of the right hand side of the above inequality is given by $\frac{1}{2}(\rho_{1}\rho_{2})^{3}d$.
\end{proof}

Let $K$ be the smallest field of rationality of $T$ over $k$ and $Z_{1}$, $Z_{2}$ two independent generic divisors of $\Lambda(mT)$ over $K$. Then exactly as in \eqref{art14-eq6.2}, we get $\deg(g)\deg(E)=I(Z_{1}\cdot Z_{2}\cdot T/T,K)$. By Lemma \ref{art14-lem5}, the latter is bounded by $(\rho_{1}\rho_{2})^{3}m^{2}d$. By Lemma \ref{art14-lem11} and by our assumption, $\deg(g)=2$. Hence $\deg(E)\leq \frac{1}{2}(\rho_{1}\rho_{2})^{3}m^{2}d$. Let $\Lambda$ be the linear system of hyperplane sections of $E$. By Proposition \ref{art14-prop3}, $\dim \Lambda\leq \frac{1}{4}(\rho_{1}\rho_{2})^{3}m^{2}d+1$. Since $g$ is defined by $Tr_{T}\Lambda (mT)$, it follows that $\dim Tr_{T}\Lambda(mT)=\dim \Lambda$. Therefore,
$$
P_{\rho_{1}\rho_{2}}(m)-P_{\rho_{1}\rho_{2}}(m-1)-2c-1<\frac{1}{4}(\rho_{1}\rho_{2})^{3}m^{2}d+1.
$$
Since the leading coefficient of the left hand side is $\frac{1}{2}(\rho_{1}\rho_{2})^{3}d$, we can find a constant $\rho_{3}$, which depends on $P(x)$ only, such that the above inequality does not hold for $m\geq \rho_{3}$. For such $m$, $g$ and hence $f$ has to\pageoriginale be birational. Setting $\frac{1}{2}\rho_{4}=\rho_{1}\rho_{2}\rho_{3}$ and combining the above result with that of Proposition \ref{art14-prop7}, we get

\begin{theorem}\label{art14-thm1}
Let the characteristic be zero, $V^{3}$ a polarized variety, $P(m)=\chi(V,\mathscr{L}(mX_{V}))$ and assume that $(A''_{3})$ is true. Then there is a constant $\rho_{4}$, which depends on $P(x)$ only, such that $mX_{V}$ defines a birational transformation of $V$ when $m\geq \rho_{4}$.
\end{theorem}

\begin{coro*}
Let the characteristic be zero and $V^{3}$ be canonically polarized. Then $(B_{3})$ is true.
\end{coro*}

\bigskip

\begin{center}
{\Large\bf Chapter II. The Problem $(C_{n})$.}\labeltext{2}{art14-chap2}
\end{center}

In this chapter, we shall solve $(C_{n})$ for canonically polarized varieties $V^{n}$ under the following assumptions: {\em $(A_{n})$ and $(B_{n})$ are true; theorems on dominance and birational resolution in the sense of Abhyankar hold for dimension $n$.} As we remarked already, this is the case when the characteristic is zero (c.f. \cite{art14-key5}) or when $n=1,2,3$ if the characteristics 2, 3 and 5 are excluded for $n=3$ (c.f. \cite{art14-key35}).

\section{Preliminary lemmas}\label{art14-sec8}

\begin{lemma}\label{art14-lem12}
Let $U$ and $U'$ be two non-singular subvarieties of projective spaces and $g$ a birational transformation between $U$ and $U'$. Then we have the following results: {\rm(a)} $g(\mathfrak{K}(U))+E'\sim \mathfrak{K}(U')$ where $E'$ is a positive $U'$-divisor whose components are exceptional divisors for $g^{-1}$; {\rm(b)} $l(m\mathfrak{K}(U))=l(m\mathfrak{K}(U'))$ for all positive integers $m$; {\rm(c)} $\Lambda(m\mathfrak{K}(U'))=\Lambda(g(m\mathfrak{K}(U)))+mE'$ for all positive integers $m$.\footnote{A subvariety of codimension 1 of $U'$ is called exceptional for $g^{-1}$ if the proper transform of it by $g^{-1}$ is a subvariety of $U$ of codimension at least 2.} 
\end{lemma}

\begin{proof}
These results are well known for characteristic zero. (b) and (c) are easy consequences of (a). (a) can be proved as in \cite{art14-key33} using fundamental results on monoidal transformations (c.f. \cite{art14-key29}, \cite{art14-key33}) and the theorem of dominance.
\end{proof}

\begin{lemma}\label{art14-lem13}
Let $U$ be a non-singular subvariety of a projective space such that $C_{U}\sim m\mathfrak{K}(U)$ for some positive integer $m$. Let $U'$ be a non-singular\pageoriginale subvariety of a projective space, birationally equivalent to $U$. Then $m\mathfrak{R}(U')$ defines a non-degenerate birational map $h'$ of $U'$, mapping $U'$ generically onto a non-singular subvariety $U^{*}$ of a projective space such that $C_{U^{*}}\sim m\mathfrak{K}(U^{*})$. Moreover, $U$ and $U^{*}$ are isomorphic.
\end{lemma}

\begin{proof}
Let $g$ be a birational transformation between $U$ and $U'$. Then $g(C_{U})\sim g(m\mathfrak{K}(U))$ and $\Lambda(g(\mathfrak{K}(U)))+mE'=\Lambda(m\mathfrak{K}(U'))$ where $E'$ is a positive $U'$-divisor whose components are exceptional divisors for $g^{-1}$ by Lemma \ref{art14-lem12}. Assume first that the set of hyperplane sections of $U$ forms a complete linear system. Then $g(C_{U})$ is irreducible for general $C_{U}$ (c.f. \cite{art14-key25}, Chap. IX). Hence $mE'$ is the fixed part of $\Lambda(m\mathfrak{K}(U'))$. Since $l(m\mathfrak{K}(U))=l(m\mathfrak{K}(U'))$ by Lemma \ref{art14-lem12}, it follows that all members of $\Lambda(g(m\mathfrak{K}(U)))$ are of the form $g(C_{U})$. This proves that $g^{-1}$ is a non-degenerate rational map defined by $m\mathfrak{K}(U')$. If our assumption does not hold for $U$, apply a non-degenerate map of $U$, defined by $C_{U}$, to $U$. This amp is obviously an isomorphism and the image of $U$ by this clearly satisfies our assumption.
\end{proof}

\begin{lemma}\label{art14-lem14}
Let $U^{n}$ (resp. ${U'}^{n}$) be a complete non-singular variety, $\mathfrak{K}(U)$ (resp. $\mathfrak{K}(U')$) a canonical divisor of $U$ (resp. $U'$) and $\mathfrak{O}$ a discrete valuation ring with the quotient field $k_{0}$ and the residue field $k'_{0}$. Assume that $U$, $\mathfrak{K}(U)$ are rational over $k_{0}$ and that $(U',\mathfrak{K}(U'))$ is a specialization of $(U,\mathfrak{K}(U))$ over $\mathfrak{O}$. Assume further that the following conditions are satisfied: {\rm(i)} there is a positive integer $m_{0}$ such that $l(m_{0}\mathfrak{K}(U))=l(m_{0}\mathfrak{K}(U'))$; {\rm(ii)} a non-degenerate rational map $h$ (resp. $h'$) defined by $m_{0}\mathfrak{K}(U)$ (resp. $m_{0}\mathfrak{K}(U')$) is birational; {\rm(iii)} $h'(U')=W'$ is non-singular and $C_{W'}\sim m_{0}\mathfrak{K}(W')$. Then the following two statements are equivalent: {\rm(a)} There is a birational map $g$ of $U$, between $U$ and a non-singular subvariety $W^{*}$ of a projective space such that $C_{W^{*}}\sim t\mathfrak{K}(W^{*})$ for some positive integer $t$ and $\mathfrak{K}(W')^{(n)}=\mathfrak{K}(W^{*})^{(n)}$; {\rm(b)} $\deg(h(U))=\deg (h'(U'))$. Moreover, when {\rm(a)} or {\rm(b)} is satisfied, $h(U)=W$ is non-singular, $C_{W}\sim m_{0}\mathfrak{K}(W)$ and $\mathfrak{K}(W)^{(n)}=\mathfrak{K}(W')^{(n)}$.
\end{lemma}

\begin{proof}
First assume (a). $h$ is uniquely determined by $m_{0}\mathfrak{K}(U)$ up to a projective transformation. Therefore we get $\deg(h(U))\geq \deg (h'(U'))$\pageoriginale by Proposition \ref{art14-prop2.1} of the Appendix since specializations are compatible with the operation of algebraic projection (c.f. (\cite{art14-key24}). Let the $Z_{i}$ be $n$ independent generic divisors of $\Lambda(m_{0}\mathfrak{K}(U))$ over $k_{0}$. Then $\deg(h(U))=I(Z_{1}\ldots Z_{n}/U,k_{0})$ since $h$ is birational. Let $d_{0}=\mathfrak{K}(W')^{(n)}=\mathfrak{K}(W^{*})^{(n)}$. Then $\deg(h'(U'))=m_{0}{}^{n}d_{0}$ by (iii). Hence $\deg (h(U))\geq m_{0}{}^{n}d_{0}$. By Lemma \ref{art14-lem11}, $\Lambda (g(m\mathfrak{K}(U)))+mE^{*}=\Lambda(m\mathfrak{K}(W^{*}))$ for all positive $m$ where $E^{*}$ is as described in the lemma. Let $L$ be a common field of rationality of $W^{*}$ and $g$ over $k_{0}$ and the $Y_{i}$ (resp. $Y^{*}_{i}$) $n$ independent generic divisors of $\Lambda(m\mathfrak{K}(U))$ (resp. $\Lambda(m\mathfrak{K}(W^{*}))$) over $L$. Then we have $I(Y_{1}\ldots Y_{n}/U,L)=I(Y^{*}_{1}\ldots Y^{*}_{n}/W^{*},L)$. By Lemma \ref{art14-lem5}, $I(Y^{*}_{1}\ldots Y^{*}_{n}/W^{*},L)\leq m^{n}d_{0}$. Setting $m=m_{0}$, we therefore get $I(Y_{1}\ldots Y_{n}/U,L)\leq m_{0}{}^{n}d_{0}$. The left hand side of this is obviously $I(Z_{1}\ldots Z_{n}/U,k_{0})$. Combining the two inequalities we obtained, we get $\deg(h(U))=m_{0}{}^{n}d_{0}=\deg(h'(U'))$. Hence (a) implies (b).

Now we assume (b). Let $W=h(U)$, $C=C_{W}$, $C'=C_{W'}$. By Proposition \ref{art14-prop2.1} of the Appendix and by the compatibility of specializations with the operation of algebraic projection, we get $(U,\mathfrak{K}(U), W)\to (U',\mathfrak{K}(U'),W')$ ref. $\mathfrak{O}$. Since $W'$ is non-singular, $W$ is non-singular too. Since $h$ is defined by $m_{0}\mathfrak{K}(U)$, there is a positive $U$-divisor $F$ such that $h^{-1}(C)+F\sim m_{0}\mathfrak{K}(U)$. Hence there is a positive divisor $T$ with $h(m_{0}\mathfrak{K}(U))\sim C+T$. There is a positive divisor $E$ such that $C+T+E\sim m_{0}\mathfrak{K}(W)$ by Lemma \ref{art14-lem11}. Let $C''+T'+E'$ be a specialization of $C+T+E$ over $\mathfrak{O}$ over the specialization under consideration. Since linear equivalence is preserved by specializations (c.f. \cite{art14-key24}), $C'\sim C''$ and $C'+T'+E'\sim m_{0}\mathfrak{K}(W')$ (c.f. Lemma \ref{art14-lem1.1} of the Appendix; $U$, $U'$ are clearly non-ruled since $l(m\mathfrak{K}(U))$, $l(m\mathfrak{K}(U'))$ are positive for large $m$). Since $m_{0}\mathfrak{K}(W')\sim C'$ by (iii), it follows that $T'$, $E'$ are positive and $T'+E'\sim 0$. This proves that $T=E=0$ and $m_{0}\mathfrak{K}(W)\sim C$. Our lemma is thereby proved.
\end{proof}

\section{A proof of $(C_{n})$}\label{art14-sec9}

In order to solve $(C_{n})$, we shall fix some notation. We shall denote by $\Sigma$ the {\em set of canonically polarized varieties with the fixed Hilbert characteristic polynomial} $P(x)$ and by $V^{n}$ a ``variable element'' of $\Sigma$. As we have shown in Lemma \ref{art14-lem1}, there is a root $\rho$ of the\pageoriginale %page 288
