\title{RESOLUTION OF SINGULARITIES OF ALGEBRAIC SURFACES}
\markright{Resolution of Singularities of Algebraic Surfaces}

\author{By~~ Shreeram Shankar Abhyankar}

\date{}

\maketitle

\setcounter{pageoriginal}{0}
\section{Introduction}\label{art01-sec1}
\pageoriginale 

The theorem of resolution of singularities of algebraic surfaces asserts the following :

\textsc{Surface Resolution.} {\em Given a projective algebraic irreducible surface $Y$ over a field $k$, there exists a projective algebraic irreducible nonsingular surface $Y'$ over $k$ together with a birational map of $Y'$ onto $Y$ (without fundamental points on $Y'$).}

For the case when $k$ is the field of complex numbers, after several geometric proofs by the Italians (see Chapter I of \cite{art01-key15}), the first rigorous proof of Surface Resolution was given by Walker \cite{art01-key14}. For the case when $k$ is a field of zero characteristic, Surface Resolution was proved by Zariski (\cite{art01-key16},\cite{art01-key17}); and for the case when $k$ is a perfect field of nonzero characteristic, it was proved by Abhyankar (\cite{art01-key2},\cite{art01-key3},\cite{art01-key4}).

A stronger version of Surface Resolution is the following~:

\medskip
\textsc{Embedded Surface Resolution.} {\em Let $X$ be a projective algebraic irreducible nonsingular three-dimensional variety over a field $k$, and let $Y$ be an algebraic surface embedded in $X$. Then there exists a finite sequence $X\to X_{1}\to X_{2}\to \ldots\to X_{t}\to X'$ of monoidal transformations, with irreducible nonsingular centres, such that the total transform of $Y$ in $X'$ has only normal crossings and the proper transform of $Y$ in $X'$ is nonsingular.}

For the case when $k$ is of zero characteristic and $Y$ is irreducible, the part of Embedded Surfaces Resolution concerning the proper transform of $Y$ was proposed by Levi \cite{art01-key13} and proved by Zariski \cite{art01-key18}. Again for the case when $k$ is of zero characteristic, Hironaka \cite{art01-key12} proved Embedded Resolution for algebraic varieties of any dimension. For the case when $k$ is a perfect field of nonzero characteristic, Embedded Surface Resolution was proved by Abhyankar (\cite{art01-key7},\cite{art01-key9},\cite{art01-key10},\cite{art01-key11}).

Then\pageoriginale in January 1967, in a seminar at Purdue University, I gave a proof of Embedded Surface Resolution (and hence {\em a fortiori} also of Surface Resolution) for an arbitrary field $k$, i.e. without assuming $k$ to the perfect. The details of this proof will be published elsewhere in due course of time. This new proof is actually only a modification of my older proofs cited above. One difference between them is this. In the older proofs I passed to the albegraic closure of $k$, did resolution there, and then pulled it down to the level of $k$. In case $k$ is imperfect, the pulling down causes difficulties. In the new modified proof I work directly over $k$. In this connection, a certain lemma about polynomials in one indeterminate (with coefficients in some field) plays a significant role.

In order that my lecture should not get reduced to talking only in terms of generalities, I would like to show you, concretely, the proof of something. The said lemma being quite simple, I shall now state and prove it.

\section{The lemma}\label{art01-sec2}

Let $k'$ be a field and let $q$ be a positive integer. Assume that $q$ is a power of the characteristic exponent of $k'$; recall that, by definition : (the characteristic exponent of $k'$) = (the characteristic of $k'$) if $k'$ is of nonzero characteristic, and (the characteristic exponent of $k'$) = 1 if $k'$ is of zero characteristic. Let $A$ be the ring of all polynomials in an indeterminate $T$ with coefficients in $k'$. As usual, by $A^{q}$ we denote the set $\{h^{q}:h\in A\}$. Note that then $A^{q}$ is a subring of $A$; this is the only property of $q$ which we are going to use. For any $h\in A$, by $\deg h$ we shall denote the degree of $h$ in $T$; we take $deg 0 = -\infty$. Let $f\in A$ and let $d=\deg f$. Assume that $f\not\in k$, i.e. $d>0$. Let $r:A\to A/(fA)$ be the canonical epimorphism. For any nonnegative integer $e$ let
$$
W(q,d,e)=[e/(dq)]q+[\max \{q-(q/d),(e/d)-[e/(dq)]q\}]
$$
where the square brackets denote the integral part, i.e. for any real number $a$, by $[a]$ we denote the greatest integer which is $\leq a$.

\medskip
\textsc{The Lemma.} {\em Given any $g\in A$ with $g\not\in A^{q}$, let $e=\deg g$. Then we can express $g$ in the form}
$$
g=g'+g^{*}f^{u}
$$\pageoriginale
{\em where $g'\in A^{q}$, $g^{*}\in A$ with $g^{*}\not\in fA$, and $u$ is a nonnegative integer such that}
$$
u\leq W(q,d,e)
$$
{\em and: either $u\nequiv 0(q)$ or $r(g^{*})\not\in (r(A))^{q}$.}


Before proving the lemma we shall make some preliminary remarks.

\begin{remark}\label{art01-rem1}
The assumption that $g\in A$ and $g\not\in A^{q}$ is never satisfied if $q=1$. Thus the lemma has significance only when $k'$ is of nonzero characteristic and $q$ is a positive power of the characteristic of $k'$. The lemma could conceivably be generalized by replacing $A^{q}$ by some other subset of $A$.
\end{remark}

\begin{remark}\label{art01-rem2}
For any integers $q$, $d$, $e$ with $q>0$, $d>0$, $e\geq 0$, we clearly have $W(q,d,e)\geq 0$.
\end{remark}

\begin{remark}\label{art01-rem3}
The bound $W(q,d,e)$ for $u$ can be expressed in various other forms. Namely, {\em we claim that for any integers $q$, $d$, $e$ with $q>0$, $d>0$, $e\geq 0$, we have}
$$
W(q,d,e)=W_{1}(q,d,e)=W_{2}(q,d,e)=W_{3}(q,d,e)
$$
{\em where}
\begin{align*}
W_{1}(q,d,e) &= [[e/d]/q]q+[\max\{q-(q/d),[e/d]-[[e/d]/q]q\}],\\
W_{2}(q,d,e) &= [[e/d]/q]q+\max \{[q-(q/d)],[e/d]-[[e/d]/q]q\},\\
W_{3}(q,d,e) &= \max \{[e/d],[[e/d]/q]q+[q-(q/d)]\}.
\end{align*}
To see this, first note that by the division algorithm we have
$$
e=[e/d]d+j\text{~~ with~~ } 0\leq j\leq d-1
$$
and
$$
[e/d]=[[e/d]/q]q+j'\text{~~ with~~ } 0\leq j'\leq q-1;
$$
upon substituting the second equation in the first equation we get 
$$
e=[[e/d]/q]dq+(dj'+j)\text{~~ and~~ } 0\leq dj'+j\leq dq-1
$$
and hence
$$
[[e/d]/q]=[e/(dq)].
$$\pageoriginale
For any two real numbers $a$ and $b$ we clearly have $[\max \{a,b\}]=\max \{[a],[b]\}$, and hence in view of the last displayed equation we see that
$$
W(q,d,e)=W_{2}(q,d,e)=W_{1}(q,d,e).
$$
For any real numbers $a$, $b$, $c$ we clearly have
$$
a+\max \{b,c\}=\max \{a+b,b+c\},
$$
and hence we see that
$$
W_{2}(q,d,e)=W_{3}(q,d,e).
$$
Thus our claim is proved. Clearly $[e/d]\leq W_{3}(q,d,e)$; since $W(q,d,e)=W_{3}(q,d,e)$, we thus get the following
\end{remark}

\begin{remark}\label{art01-rem4}
For any integers $q$, $d$, $e$ with $q>0$, $d>0$, $e\geq 0$, we have $[e/d]\leq W(q,d,e)$.
\end{remark}

\begin{remark}\label{art01-rem5}
Again let $q$, $d$, $e$ be any integers with $q>0$, $d>0$, $e\geq 0$. Concerning $W(q,d,e)$ we note the following.

If $d=1$ then clearly $W(q,d,e)=e$.

If $d>1$ and $e\leq q$ then~: $[e/(dq)]=0$ and $\max\{q-(q/d),e/d\}<q$, and hence $W(q,d,e)\leq q-1$.

If $d>1$ and $e>q$ then~:
\begin{align*}
[[e/d]/q]q+[q-(q/d)] &\leq (e/d)+q-(q/d)\\
                     &= (e/d)+(q/d)(d-1)\\
                     &< (e/d)+(e/d)(d-1)\\
                     &= e
\end{align*}
and hence
$$
[[e/d]/q]q+[q-(q/d)]\leq e-1;
$$
also $[e/d]\leq e-1$, and hence $W_{3}(q,d,e)\leq e-1$ where $W_{3}$ is as in Remark \ref{art01-rem3}; since $W(q,d,e)=W_{3}(q,d,e)$ by Remark \ref{art01-rem3}, we get that $W(q,d,e)\leq e-1$.

Thus~: {\em if $d=1$ then $W(q,d,e)=e$; if $d>1$ and $e\leq q$ then $W(q,d,e)\leq q-1$; if $d>1$ and $e>q$ then $W(q,d,e)\leq e-1$.}

In\pageoriginale particular~: {\em either $W(q,d,e)/q\leq e/q$ or $W(q,d,e)/q<1$; if $d>1$ then either $W(q,d,e)/q<e/q$ or $W(q,d,e)/q<1$.}
\end{remark}

Thus the lemma has the following

\begin{coro*}
The same statement as that of the lemma, except that we replace the inequality $u\leq W(q,d,e)$ by the following weaker estimates: either $u/q\leq e/q$ or $u/q<1$; if $d>1$ then either $u/q<e/q$ or $u/q<1$.
\end{coro*}

For applications, this corollary is rather significant.

\begin{remark}\label{art01-rem6}
For a moment suppose that $r(g)\in (r(A))^{q}$. Then $r(g)={h'}^{q}$ for some $h'\in r(A)$. Now there exists a unique $h^{*}\in A$ such that $\deg h^{*}\leq d-1$ and $r(h^{*})=h'$. Let $h=h^{*q}$. Then $h\in A^{q}$, $\deg h\leq dq-q$, and $g-h\in fA$. Since $g\not\in A^{q}$ and $h\in A^{q}$, we must have $g-h\not\in A^{q}$; hence in particular $g-h\neq 0$. Since $0\neq 01-h\in fA$, there exists $g_{1}\in A$ and a positive integer $v$ such that $g_{1}\not\in fA$ and $g-h=g_{1}f^{v}$. Since $g_{1}f^{v}=g-h\not\in A^{q}$ we get that: either $v\nequiv 0(q)$ or $g_{1}\not\in A^{q}$. Now $g_{1}f^{v}=g-h$, $\deg f=d$, $\deg g=e$, and $\deg h\leq dq-q$; therefore upon letting $e_{1}=\deg g_{1}$ we have
$$
dv+e_{1}=\deg (g_{1}f^{v})=\deg(g-h)
\left\{
\begin{array}{ll}
=e & \text{if~ } e>dq-q\\
\leq dq-q & \text{if~ } e\leq dq-q.
\end{array}\right.
$$
Thus we have proved the following
\end{remark}

\begin{remark}\label{art01-rem7}
If $r(g)\in (r(A))^{q}$ then $g=h+g_{1}f^{v}$ where $h\in A^{q}$, $g_{1}\in A$ with $g_{1}\not\in fA$, $v$ is a positive integer, and letting $e_{1}=\deg g_{1}$ we have
$$
dv+e_{1}
\left\{
\begin{array}{ll}
=e & \text{if~ } e>dq-q\\
\leq dq-q & \text{if~ } e\leq dq-q
\end{array}\right.
$$
and: either $v\nequiv 0(q)$ or $g_{1}\not\in A^{q}$.
\end{remark}

\noindent
\textsc{Proof of the Lemma.} We shall make induction on $[e/(dq)]$. First consider the case when $[e/(dq)]=0$. If $r(g)\not\in (r(A))^{q}$ then, in view of Remark \ref{art01-rem2}, it suffices to take $g'=0$, $g^{*}=g$, $u=0$. If $r(g)\in (r(A))^{q}$ then let the notation be as in Remark \ref{art01-rem7}; since $[e/(dq)]=0$ and by Remark \ref{art01-rem7} we have $dv\leq \max \{dq-q,e\}$, we see that $v\leq W(q,d,e)$ and\pageoriginale $v<q$; since $0<v<q$, we see that $v\nequiv 0(q)$; therefore it suffices to take $g'=h$, $g^{*}=g_{1}$, $u=v$.


Now let $[e/(dq)]>0$ and assume that the assertion is true for all values of $[e/(dq)]$ smaller than the given one. If $r(g)\not\in (r(A))^{q}$ then, in view of Remark \ref{art01-rem2}, it suffices to take $g'=0$, $g^{*}=g$, $u=0$. So now suppose that $r(g)\in (r(A))^{q}$ and let the notation be as in Remark \ref{art01-rem7}. Since $[e/(dq)]>0$, by Remark \ref{art01-rem7} we have
\begin{equation*}
dv+e_{1}=e.\tag{*}
\end{equation*}
Therefore $v\leq [e/d]$ and hence if $v\nequiv 0(q)$ then, in view of Remark \ref{art01-rem4}, it suffices to take $g'=h$, $g^{*}=g_{1}$, $u=v$. So now also suppose that
\begin{equation*}
v\equiv 0(q).\tag{**}
\end{equation*}
Then by Remark \ref{art01-rem7} we must have $g_{1}\not\in A^{q}$; since $v>0$, by (*) and (**) we see that $[e_{1}/(dq)]<[e/(dq)]$; therefore by the induction hypothesis we can express $g_{1}$ in the form
$$
g_{1}=g'_{1}+g^{*}f^{u_{1}}
$$
where $g'_{1}\in A^{q}$, $g^{*}\in A$ with $g^{*}\not\in fA$, and $u_{1}$ is a nonnegative integer such that $u_{1}\leq W(q,d,e_{1})$ and either $u_{1}\nequiv 0(q)$ or $r(g^{*})\not\in (r(A))^{q}$. Let $u=u_{1}+v$. Then $u$ is a nonnegative integer, and in view of (**) we see that $u\equiv 0(q)$ if and only if $u_{1}\equiv 0(q)$; consequently: either $u\nequiv 0(q)$ or $r(g^{*})\not\in (r(A))^{q}$. Let $g'=h+g'_{1}f^{v}$; since $h$ and $g'_{1}$ are in $A^{q}$, by (**) we get that $g'\in A^{q}$. Clearly
$$
g=g'+g^{*}f^{u}.
$$
By (*) and (**) we get that
$$
e\equiv e_{1}\mod dq \text{~~ and~~ } (e-e_{1})/(dq)=v/q,
$$
and hence
$$
[e/(dq)]=[e_{1}/(dq)]+(v/q)
$$
and
$$
(e/(dq))-[e/(dq)]=(e_{1}/(dq))-[e_{1}/(dq)];
$$
therefore
$$
W(q,d,e)=W(q,d,e_{1})+v;
$$
since\pageoriginale $u=u_{1}+v$ and $u_{1}\leq W(q,d,e_{1})$, we conclude that $u\leq W(q,d,e)$.


\section{Use of the lemma}\label{art01-sec3}

To give a slight indication of how the lemma is used, let $R$ and $R^{*}$ be two-dimensional regular local rings such that $R^{*}$ is a quadratic transform of $R$. Let $M$ and $M^{*}$ be the maximal ideals in $R$ and $R^{*}$ respectively. Let $k'=R/M$ and let $J$ be a coefficient set of $R$, i.e. $J$ is a subset of $R$ which gets mapped one-to-one onto $k'$ by the canonical epimorphism $R\to R/M$. We can take a basis $(x,y)$ of $M$ such that $MR^{*}=xR^{*}$. Then $y/x\in R^{*}$. Let $s:R[y/x]\to R[y/x]/(xR[y/x])$ be the canonical epimorphism and let $T=s(y/x)$. Then $s(R)$ is naturally isomorphic to $k'$ and, upon identifying $s(R)$ with $k'$ and letting $A=k'[T]$, we have that $T$ is transcendental over $k'$, $s(R[y/x])=A$, and there exists a unique nonconstant monic irreducible polynomial
$$
f=T^{d}+f_{1}T^{d-1}+\cdots+f_{d}\text{~~ with~~ } f_{i}\in k'
$$
such that $s(R[y/x]\cap M^{*})=fA$. Take $f'_{i}\in J$ with $s(f'_{i})=f_{i}$, and let 
$$
y^{*}=(y/x)^{d}+f'_{1}(y/x)^{d-1}+\cdots+f'_{d}.
$$
Now $R^{*}$ is the quotient ring of $R[y/x]$ with respect to the maximal ideal $R[y/x]\cap M^{*}$ in $R[y/x]$; consequently $(x,y^{*})$ is a basis of $M^{*}$, and, upon letting $s^{*}:R^{*}\to R^{*}/M^{*}$ be the canonical epimorphism and identifying $s^{*}(R)$ with $k'$, we have that $R^{*}/M^{*}=k'(s^{*}(y/x))$, $s^{*}(y/x)$ is algebraic over $k'$, and $f$ is the minimal monic polynomial of $s^{*}(y/x)$ over $k'$.

Given any element $G$ in $R$ we can expand $G$ as a formal power series $H(x,y)$ in $(x,y)$ with coefficients in $J$; since $G\in R^{*}$, we can also expand $G$ as a formal power series $H^{*}(x,y^{*})$ in $(x,y^{*})$ with coefficients in a suitable coefficient set $J^{*}$ of $R^{*}$. In our older proofs we needed to show that if $H$ satisfies certain structural conditions than $H^{*}$ satisfies certain other structural conditions (for instance see (2.5) of \cite{art01-key4}, (1.5) of \cite{art01-key5}, and \S7 of \cite{art01-key11}); there we were dealing with the case when $R/M$ is algebraically closed (and hence with the case when $d=1$, the equation $y^{*}=(y/x)+f'_{1}$ expressing the quadratic transformation is linear in $y/x$, $R^{*}/M^{*}=R/M$, and one may\pageoriginale take $J^{*}=J$). The lemma enables us to do the same sort of thing in the general case, i.e. when $R/M$ is not necessarily algebraically closed and we may have $d>1$.

In passing, it may be remarked that if $R$ is of nonzero characteristic and $R/M$ is imperfect then, in general, it is not possible to extend a coefficient field of the completion of $R$ to a coefficient field of the completion of $R^{*}$.

\section{Another aspect of the new proof}\label{art01-sec4}

Another difference between the new modified proof and the older proofs is that the new modified proof gives a unified treatment for zero characteristic and nonzero characteristic; this is done by letting the characteristic exponent play the role previously played by the characteristic. To illustrate this very briefly, consider a hypersurface given by $F(Z)=0$ where $F(Z)$ is a nonconstant monic polynomial in an indeterminate $Z$ with coefficient in a regular local ring $R$, i.e.
$$
F(Z)=Z^{m}+\sum\limits^{m}_{i=1}F_{i}Z^{m-i}\quad\text{with}\quad F_{i}\in R.
$$
Let $M$ be the maximal ideal in $R$ and suppose that $F_{i}\in M^{i}$ for all $i$. Now the hypersurface given by $F(Z)=0$ has a point of multiplicity $m$ at the ``origin'', and one wants to show that, by a suitable sequence of monoidal transformations, the multiplicity can be decreased.

In the previous proofs of this, dealing with zero characteristic (for instance see \cite{art01-key16}, \cite{art01-key18}, \cite{art01-key12}, and (5.5) to (5.8) of \cite{art01-key10}), $F_{1}$ played a dominant role. In our older proofs, dealing with nonzero characteristic (for instance see \cite{art01-key9} and \cite{art01-key11}), the procedure was to reduce the problem to the case when $m$ is a power of the characteristic and then to do that case by letting $F_{m}$ play the dominant role.

In the new modified proof we directly do the general case by letting $F_{q}$ play the dominant role where $q$ is the greatest positive integer such that $q$ is a power of the characteristic exponent of $R/M$ and $q$ divides $m$. Note that on the one hand, if $R/M$ is of zero characteristic (or, more generally, if $m$ is not divisible by the characteristic of $R/M$) then $q=1$; and on the other hand, if $R/M$ is of nonzero characteristic\pageoriginale and $m$ is a power of the characteristic of $R/M$ then $q=m$.

\section{More general surfaces}\label{art01-sec5}

Previously, in (\cite{art01-key1}, \cite{art01-key5}, \cite{art01-key6}, \cite{art01-key7}, \cite{art01-key8}), I had proved Surface Resolution also in the arithmetical case, i.e. for ``surfaces'' defined over the ring of integers; in fact what I had proved there was slightly more general, namely, Surface Resolution for ``surfaces'' defined over any pseudogeometric Dedekind domain $k$ satisfying the condition that $k/P$ is perfect for every maximal ideal $P$ in $k$. In view of the new modified proof spoken of in \S1, this last condition can now be dropped. The final result which we end up with can be stated using the language of models (alternatively, one could use the language of schemes), and is thus:

\medskip
\textsc{Surface Resolution Over Excellent Rings.} {\em Let $k$ be an excellent (in the sense of Grothendieck, see $(1.2)$ of \cite{art01-key10}) noetherian integral domain. Let $K$ be a function field over $k$ such that $\dim_{k}K=2$; (by definition, $\dim_{k}K=$ the Krull dimension of $k+{}$ the transcendence degree of $K$ over $k$). Let $Y$ be any projective model of $K$ over $k$. Then there exists a projective nonsingular model $Y'$ of $K$ over $k$ such that $Y'$ dominates $Y$.}

In (\cite{art01-key7}, \cite{art01-key9}, \cite{art01-key10}, \cite{art01-key11}) I had proved Embedded Surface Resolution for models over any excellent noetherian integral domain $k$ such that for every maximal ideal $P$ in $k$ we have that $k/P$ has the same characteristic as $k$, and $k/P$ is perfect. In view of the new modified proof spoken of in \S1 the condition that $k/P$ be perfect can now be dropped. What we end up with can be stated thus:

\medskip
\textsc{Embedded Surface Resolution Over Equicharacteristic Excellent Rings.} {\em Let $k$ be an excellent noetherian integral domain such that for every maximal ideal $P$ in $k$ we have that $k/P$ has the same characteristic as $k$. Let $K$ be a function field over $k$ such that $\dim_{k}K=3$. Let $X$ be a projective nonsingular model of $K$ over $k$, and let $Y$ be a surface in $X$. Then there exists a finite sequence $X\to X_{1}\to X_{2}\to \ldots \to X_{t}\to X'$ of monoidal transformations, with irreducible nonsingular centers, such that the total transform of $Y$ in $X'$ has only normal crossings and the proper transform of $Y$ in $X'$ is nonsingular.}

\begin{thebibliography}{99}\pageoriginale
\bibitem{art01-key1} \textsc{S. S. Abhyankar :} On the valuations centered in a local domain, {\em Amer. J. Math.} 78 (1956), 321-348.

\bibitem{art01-key2} \textsc{S. S. Abhyankar :} Local uniformization on algebraic surfaces over ground fields of characteristic $p\neq 0$, {\em Annals of Math.} 63 (1956), 491-526. Corrections: {\em Annals of Math.} 78 (1963), 202-203.

\bibitem{art01-key3} \textsc{S. S. Abhyankar :} On the field of definition of a nonsingular birational transform of an algebraic surface, {\em Annals of Math.} 65 (1957), 268-281.

\bibitem{art01-key4} \textsc{S. S. Abhyankar :} Uniformization in $p$-cyclic extensions of algebraic surfaces over ground field of characteristic $p$, {\em Math. Annalen,} 153 (1964), 81-96.

\bibitem{art01-key5} \textsc{S. S. Abhyankar :} Reduction to multiplicity less than $p$ in a $p$-cyclic extension of a two dimensional regular local ring ($p={}$ characteristic of the residue field), {\em Math. Annalen,} 154 (1964), 28-55.

\bibitem{art01-key6} \textsc{S. S. Abhyankar :} Uniformization of Jungian local domains, {\em Math. Annalen,} 159 (1965), 1-43. Correction : {\em Math. Annalen,} 160 (1965), 319-320.

\bibitem{art01-key7} \textsc{S. S. Abhyankar :} Uniformization in $p$-cyclic extensions of a two dimensional regular local domain of residue field characteristic $p$, {\em Festschrift zur Ged\"achtnisfeier f\"ur Karl Weierstrass} 1815-1965, {\em Wissenschaftliche Abhandlungen des Landes Nordrhein-Westfalen,} 33 (1966), 243-317, Westdeutscher Verlag, K\"oln und Opladen.

\bibitem{art01-key8} \textsc{S. S. Abhyankar :} Resolution of singularities of arithmetical surfaces, {\em Arithmetical Algebraic Geometry}, 111-152, Harper and Row, New York, 1966.

\bibitem{art01-key9} \textsc{S. S. Abhyankar :} An algorithm on polynomials in one indeterminate with coefficients in a two dimensional regular local domain, {\em Annali di Mat. Pura ed Applicata,} Serie IV, 71 (1966), 25-60.

\bibitem{art01-key10} \textsc{S. S. Abhyankar :} {\em Resolution of singularities of embedded algebraic surfaces,} Academic Press, New York, 1966.

\bibitem{art01-key11} \textsc{S. S. Abhyankar :}\pageoriginale Nonsplitting of valuations in extensions of two dimensional regular local domains, {\em Math. Annalen,} 170 (1967), 87-144.

\bibitem{art01-key12} \textsc{H. Hironaka :} Resolution of singularities of an algebraic variety over a field of characteristic zero, {\em Annals of Math.} 79 (1964), 109-326.

\bibitem{art01-key13} \textsc{B. Levi :} Resoluzione delle singolarita puntuali delle superficie algebriche, {\em Atti Accad. Sc. Torino,} 33 (1897), 66-86.

\bibitem{art01-key14} \textsc{R. J. Walker :} Reduction of singularities of an algebraic surface, {\em Annals of Math.} 36 (1935), 336-365.

\bibitem{art01-key15} \textsc{O. Zariski :} Algebraic Surfaces, {\em Ergebnisse der Mathematik und ihrer Grenzgebiete}, vol. 3, 1934.

\bibitem{art01-key16} \textsc{O. Zariski :} The reduction of singularities of an algebraic surface, {\em Annals of Math.} 40 (1939), 639-689.

\bibitem{art01-key17} \textsc{O. Zariski :} A simplified proof for the reduction of singularities of algebraic surfaces, {\em Annals of Math.} 43 (1942), 583-593.

\bibitem{art01-key18} \textsc{O. Zariski :} Reduction of singularities of algebraic three-dimensional varieties, {\em Annals of Math.} 45 (1944), 472-542.
\end{thebibliography}

\medskip
\noindent
Purdue University

\noindent
Lafayette, Indiana, U.S.A.


