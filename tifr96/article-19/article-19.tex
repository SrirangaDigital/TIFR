\title{THE UNIPOTENT VARIETY OF A SEMISIMPLE GROUP}
\markright{The Unipotent Variety of a Semisimple Group}

\author{By~~ T. A. Springer}
\date{}

\maketitle

\setcounter{pageoriginal}{372}
\textsc{If}\pageoriginale $G$ is a connected semisimple linear algebraic group over a field of characteristic $0$, one easily sees that the variety $V$ of unipotent elements of $V$ is isomorphic to the variety $\mathfrak{B}$ of nilpotent elements of its Lie algebra $\mathfrak{g}$; moreover one can choose the isomorphism so as to be compatible with the canonical actions of $G$ on $V$ and $\mathfrak{B}$. In this note the analogous situation in characteristic $p>0$ will be discussed. We have to restrict $p$ to be ``good'' for $G$ (see \ref{art19-sec0.3}). This is not so surprising, since in ``bad'' characteristics there are anomalies in the behaviour of unipotent elements.

Due to technical difficulties, we cannot prove the {\em isomorphism} of $V$ and $\mathfrak{B}$ for $p>0$, but only a slightly weaker result (Theorem \ref{art19-thm3.1}). This is, however, sufficient for the applications which are discussed in \S\ref{art19-sec4}.

\setcounter{section}{-1}
\section{Notations and recollections}\label{art19-sec0}

\subsection{}\label{art19-sec0.1} 
$k$ denotes a field, $\overline{k}$ an algebraic closure of $k$ and $k_{s}$ a separable closure. $p$ is the characteristic of $k$.

An algebraic variety $V$ defined over $k$ (or a $k$-variety) is a scheme which is of finite type and absolutely reduced over $k$. $V(k)$ denotes its set of $k$-rational points. We may and shall identify $V$ with $V(\overline{k})$, or $V(k_{s})$.

An algebraic group $H$ defined over $k$ (or a $k$-group) will mean here a {\em linear} algebraic group, i.e. an affine group scheme, of finite type and smooth over $k$. $H^{0}$ denotes the identity component of $H$.

The Lie algebra of an algebraic group $H$ will be denoted by the corresponding German letter $\mathfrak{h}$. $H$ acts on $\mathfrak{h}$ via the adjoint representation Ad. If $x\in H$, $Z(x)$ denotes the centralizer of $x$ in $H$, $\mathfrak{z}(x)=\{X\in \mathfrak{h}|\Ad (x)X=X\}$ the centralizer of $x$ in $\mathfrak{h}$. If $X\in \mathfrak{h}$, $Z(X)=\{x\in H|\Ad (x)X=X\}$ is the centralizer of $X$ in $H$, $\mathfrak{z}(X)$ its\pageoriginale centralizer in $\mathfrak{h}$ and $N(X)=\{x\in H|\Ad (x)X\in \overline{k}^{*}X\}$ its normaliser in $H$.

\subsection{}\label{art19-sec0.2}
$G$ always denotes a connected semisimple linear algebraic group, defined over a field $k$.

Let $T$ be a maximal torus of $G$, $B$ a Borel subgroup containing $T$ and $U$ the unipotent part of $B$. Denote by $\Sigma$ the set of roots of $G$ with respect to $T$. $B$ determines an order on $\Sigma$, let $\Sigma^{+}$ be the set of positive roots. We denote by $r_{1},\ldots,r_{l}$ the corresponding simple roots. For $r\in \Sigma$ there is an isomorphism $x_{r}$ of the additive group $\bfG_{a}$ onto a closed subgroup $G_{r}$ of $G$, such that
$$
t.x_{r}(\xi).t^{-1}=x_{r}(t^{r}\xi)\quad (\xi\in \overline{k}),
$$
where $t^{r}$ denotes the value of the character $r$ of $T$ in $t\in T$. $U$ is generated by the $G_{r}$ with $r>0$. $G_{r}$ and $G_{-r}$ generate a subgroup $P_{r}$ which is connected semisimple of type $\bfA_{1}$. $X_{r}\in \mathfrak{g}$ will denote a nonzero tangent vector to $G_{r}$. We will say that $G$ is simple if $\Sigma$ is irreducible.

\subsection{}\label{art19-sec0.3}
Let $G$ be simple. Then there is a unique highest root $r$ in $\Sigma$, for the given order. Express $r$ as an integral linear combination of the $r_{i}$. The characteristic $p$ of $k$ is called {\em bad} for $G$, if $p$ is a prime number dividing one of the coefficients in this expression. Otherwise $p$ is called {\em good} for $G$. If $p$ is good and if moreover $p$ does not divide the order of the centre of the simply connected covering of $G$, then $p$ is called {\em very good} for $G$. $p=0$ is always very good. For the simple types, the bad $p>0$ are:

$\bfA_{l}$: none; $\bfB_{l}$, $\bfC_{l}$, $\bfD_{l}$ : $p=2$; $\bfE_{6}$, $\bfE_{7}$, $\bfF_{4}$, $\bfG_{2} : p = 2$, $3$; $\bfE_{8} : p=2,3,5$. A good $p$ is very good, unless $G$ is of type $\bfA_{l}$ and $p$ divides $l+1$. If $G$ is arbitrary, $p$ is defined to be good or very good for $G$ if it is so far all simple normal subgroups of $G$.

\subsection{}\label{art19-sec0.4}
$x\in G$ is called regular, if $\dim Z(x)$ equals rank $G$. We shall have to make extensive use of the properties of regular unipotent elements of $G$, which are established in \cite{art19-key14}, \cite{art19-key15}.

\section{The unipotent variety of \texorpdfstring{$G$}{G}}\label{art19-sec1}

Let $V(G)$ (or $V$, if no confusion can arise) denote the set of unipotent elements of $G$. Then $V(G)$ is closed\pageoriginale in $G$. We call $V(G)$ the {\em unipotent variety} of $G$. The following result justifies this name.

\begin{proposition}\label{art19-prop1.1}
$V(G)$ is an irreducible closed subvariety of $G$ of dimension $\dim G - \rank G$. $G$ acts on $V(G)$ by inner automorphisms. $V(G)$ and the action of $G$ on it are defined over $k$.
\end{proposition}

Except for the last statement, this contained in (\cite{art19-key14}, 4.4, p. 131). We sketch the proof since we have to refer to it later on.

Suppose first that $G$ is quasi-split over $k$. Let $B$ be a Borel subgroup of $G$, which is defined over $k$. Then $U$ is also defined over $k$. Consider the subset $W$ of $G/B\times G$, consisting of the $(gB,x)$ such that $g^{-1}xg\in U$. This is a closed subvariety of $G/B\times G$ (\cite{art19-key5}, exp. 6, p. 12, 1.13). It follows from loc. cit. that $W$ is irreducible and defined over $k$. $V$ is the projection of $W$ onto the second factor of $G/B\times G$, let $\pi :W\to V$ be the corresponding morphism. That $V$ has the asserted dimension is proved in (\cite{art19-key14}, loc. cit). $G/B$ being a projective $k$-variety, $\pi$ is proper, defined over $k$. $V$ is closed in $G$ and defined over $k$.

Let $G$ act on $G/B\times G$ by $(h,(gB,x))\to (hgB,hxh^{-1})$. This action is defined over $k$, $W$ is stable and $\pi$ is a $G$-equivariant $k$-morphism $W\mapsto V$, $G$ acting on $V$ as in the statement of the proposition.

If $k$ is arbitrary, $G$ splits over $k_{s}$ (see \cite{art19-key2}, 8.3 for example). It follows that $V$ is defined over $k_{s}$. Since $V$ is clearly stable under $\Gamma=\Gal (k_{s}/k)$, it is defined over $k$. $W$ is also defined over $k$, moreover if $s\in \Gamma$ there exists $g_{s}\in G(k_{s})$ such that
$$
{}^{s}B=g_{s}Bg^{-1}_{s}, \ {}^{s}U=g_{s}Ug^{-1}_{s}.
$$
Define a new action $(s,w)\mapsto {}_{s}w$ of $\Gamma$ on $W$ by ${}_{s}(gB,x)=({}^{s}g\cdot g_{s}B,{}^{s}x)$. $W$ is stable under this action of $\Gamma$, hence this defines a structure of $k$-variety on $W$, such that the projection $\pi : W\to V$ is a $G$-equivariant $k$-morphism.

If $G$ and $G'$ are two semisimple $k$-groups, and $f$ a $k$-homomorphism, there exists an induced $k$-morphism $V(f):V(G)\to V(G')$.

\begin{proposition}\label{art19-prop1.2}
\begin{itemize}
\item[\rm(i)] If $f$ is a separable central isogeny, then $V(f)$ is an isomorphism, compatible with the actions of $G$ and $G'$, 

\item[\rm(ii)] $V(G\times G')$\pageoriginale is isomorphic to $V(G)\times V(G')$ as a $k$-variety, the isomorphism being compatible with the actions of $G$, $G'$, $G\times G'$. 
\end{itemize}
\end{proposition}

The proof of Proposition \ref{art19-prop1.2} is easy.

The following results are essentially contained in \cite{art19-key15}.

\begin{proposition}\label{art19-prop1.3}
\begin{itemize}
\item[\rm(i)] $V(G)$ is nonsingular in codimension $1$;

\item[\rm(ii)] $V(G)$ is normal if $G$ is simply connected, or if $p$ does not divide the order of the centre of the simple connected covering of $G$.
\end{itemize}
\end{proposition}

We may assume $k$ to be algebraically closed. In $V$ we have the open subvariety $O$ of the regular elements. $O$ is an orbit of $G$ and all its elements are simple points of $V$ (see \cite{art19-key15}, 1.2, \ref{art19-prop1.5} for these statements). (i) then follows from the fact, proved loc. cit. (6.11 e), that the irreducible components of $V-O$ have codimension $\geq 2$.

If $G$ is simply connected, then by (\cite{art19-key15}, 6.1, 8.1) $V$ is a complete intersection in $G$. The first statement in (ii) then follows from known normality criteria (e.g. \cite{art19-key7}, iv, 5.8.6, p. 108) and the second one is a consequence of \ref{art19-prop1.2} (i).

We now prove some properties of the proper $k$-morphism $\pi:W\to V$, introduced in the proof of \ref{art19-prop1.1}.

\begin{proposition}\label{art19-prop1.4}
If $G$ is adjoint, then $\pi$ is birational.
\end{proposition}

Since a regular unipotent element is contained in exactly one Borel subgroup, $\pi$ is bijective on $\pi^{-1}(O)$ ($O$ denoting as before the variety of regular elements). \ref{art19-prop1.4} will then follow, if we show that $\pi$ is separable (see e.g. \cite{art19-key17}, III, 4.3.7, p. 133). We may assume $k$ to be algebraically closed. Let $\phi$ be the morphism $G\times U\to G$ such that $\phi(g,u)=gug^{-1}$. From the definition of $\pi$ it follows that there exists a morphism $\psi : G\times U\to W$ such that $\phi=\pi\circ\psi$. To prove the separability of $\pi$, it suffices to prove that in some point $a\in G\times U$, the tangent map $(d\phi)_{a}:T(G\times U)_{a}\to T(G)_{\phi(a)}$ has image of dimension $\dim V$.

Let $u$ be a regular unipotent element in $U$. We will take $a=(e,u)$. Identify the tangent space $T(G\times U)_{a}$ with $\mathfrak{g}\oplus \mathfrak{u}$ (via a right translation with $(e,u)$), identify $T(G)_{\phi(a)}$ with $\mathfrak{g}$. Then $(d\phi)_{a}$ becomes the homomorphism $\alpha :\mathfrak{g}\oplus \mathfrak{u}\to \mathfrak{g}$, which sends $(X,U)$ into $(1-\Ad(u)^{-1})X+U$.\pageoriginale It follows that $\alpha(\mathfrak{g}\oplus \mathfrak{u})$ contains $(\Ad (u)-1)\mathfrak{g}$, which, by (\cite{art19-key15}, 4.3, p. 58) has dimension equal to $\dim V$. (Since $G$ is adjoint, the $\mathfrak{z}$ of loc. cit. is now the zero element). This implies the result.

\begin{remark*}
By dimensions one finds also that $\mathfrak{u}\subset (\Ad (u)-1)\mathfrak{g}$.
\end{remark*}

\begin{proposition}\label{art19-prop1.5}
The fibres of $\pi$ are connected.
\end{proposition}

We may assume $k$ to be algebraically closed. In view of the definition of $W$ this can be stated in another way, namely that, $B$ denoting some Borel subgroup of $G$, the fixed point set in $G/B$ of any unipotent element $g\in G$ is connected. Identifying $G/B$ with the variety of Borel subgroups of $G$, \ref{art19-prop1.5} can also be interpreted as follows: the closed subvariety of $G/B$ consisting of the Borel subgroups containing $g$, is connected. \ref{art19-prop1.5} follows, if $G$ is simply connected, from \ref{art19-prop1.3} (ii) and \ref{art19-prop1.4} by Zariski's connectedness theorem (see \cite{art19-key7}, 4.3.7, p. 133). The general case then follows at once, since central isogenies do not affect the statement.

Another proof of \ref{art19-prop1.5} was given by J. Tits. Since his method of proof will be useful in \S\ref{art19-sec2}, we will reproduce his proof here. We interpret $G/B$ as the variety of Borel subgroups of $G$. Let $g\in B$ be a unipotent element of $G$, let $B'$ be another Borel subgroup containing $g$. By Bruhat's lemma, $B\cap B'$ contains a maximal torus $T$. Let $N$ be its normalizer and $\mathscr{W}=N/T$ be the Weyl group. For $w\in \mathscr{W}$, denote by $n_{w}$ a representative in $N$. There exists then $w\in \mathscr{W}$ such that $B'=n_{w}Bn^{-1}_{w}$. $B$ determines an order on the root system $\Sigma$, let $w_{1},\ldots,w_{l}$ ($l=\rank G$) be the reflections in $\mathscr{W}$ defined by the corresponding simple roots. Since the $w_{i}$ generate $\mathscr{W}$, we can write $w$ as a product $w=w_{i_{1}}\ldots w_{i_{t}}$. We take $t$ as small as possible. Put $v_{0}=1$, $v_{h}=w_{i_{1}}\ldots w_{i_{h}}(1\leq h\leq t)$, $B_{h}=n_{v_{h}}Bn^{-1}_{v_{h}}$, so that $B_{0}=B$, $B_{t}=B'$. Let $\Sigma_{h}$ denote the set of $r\in \Sigma$ such that $r>0$, $v_{h}r<0$. It is known that the minimality of $t$ implies that $\Sigma_{h}\subset \Sigma_{h+1}$ (this follows e.g. from \cite{art19-key5}, p. 14-06, lemma). The intersection $B_{0}\cap B_{h}$ is generated by $T$ and the subgroups $G_{r}$ (see \ref{art19-sec0.2}) with $r\not\in \Sigma_{h}$ (\cite{art19-key5}, exp. 13, No. 2). Hence $B_{h}\supset B_{0}\cap B_{t}=B\cap B'$, in particular $g$ belongs to all $B_{h}$. Let $X\subset G/B$ be the variety of Borel subgroups containing $g$. It suffices\pageoriginale to show that we can connect $B_{h}$ and $B_{h+1}$ inside $X$ by a projective line.

Put $u=v_{h}w_{h+1}v^{-1}_{h}$, then $B_{h+1}=n_{u}B_{h}n^{-1}_{u}$, moreover $u$ is a reflection in a simple root for the order on $\Sigma$ determined by $B_{h}$. Changing the notation, we are reduced to proving that $B$ and $B'$ are connected inside $X$ by a projective line, if $w$ is a reflection in a {\em simple} root $r>0$. Then let $P_{r}$ be the subgroup of $G$ generated by $G_{r}$ and $G_{-r}$. $P_{r}$ is of type $\bfA_{1}$ and we may take $n_{w}\in P_{r}$. One easily checks that $hBh^{-1}\supset B\cap B'$ for all $h\in P_{r}$. $P_{r}\cap B$ is a Borel subgroup of $P_{r}$. Let $\psi:P_{r}/P_{r}\cap B\to G/B$ be the canonical immersion. $L=P_{r}/P_{r}\cap B$ is a projective line and $\psi(L)$ contains both $B$ and $B'$. This establishes our assertion.

\section{The nilpotent variety of \texorpdfstring{$G$}{G}}\label{art19-sec2}

We discuss now the Lie algebra analogues of the results of \S\ref{art19-sec1}. We recall that an element $X\in \mathfrak{g}$ is called {\em nilpotent} if it is tangent to a unipotent subgroup of $G$. Equivalently, $X$ is nilpotent if it is represented by a nilpotent matrix in any matrix realization of $G$ (see \cite{art19-key1}, \S1, pp. 26-27). Let $\mathfrak{B}(G)$ (or $\mathfrak{B}$) denote the set of nilpotent elements in $\mathfrak{g}$, $\mathfrak{g}$ being endowed with the obvious structure of affine space over $k$. Then $V$ is a closed subset of $\mathfrak{g}$.

\begin{proposition}\label{art19-prop2.1}
$\mathfrak{B}(G)$ is an irreducible closed subvariety of $\mathfrak{g}$ of dimension $\dim G -\rank G$. $G$ acts on $\mathfrak{B}(G)$ via the adjoint representation of $G$. $\mathfrak{B}(G)$ and the action of $G$ on it are defined over $k$.
\end{proposition}

The proof is similar to that of \ref{art19-prop1.1}. First let $G$ be quasi-split over $k$. We use the notations of the proof of \ref{art19-prop1.1}. Instead of $W$, we consider now the closed subvariety $\mathfrak{W}$ of $G/B\times \mathfrak{g}$, consisting of the $(gB,X)$ such that $\Ad(g)^{-1}X\in \mathfrak{u}$. $G$ acts on $\mathfrak{W}$ by
$$
(h,(gB,X))\mapsto (hgB,\Ad(h)X).
$$
The projection of $G/B\times \mathfrak{g}$ onto its second factor induces a $G$-equivariant proper morphism $\tau : \mathfrak{W}\to \mathfrak{B}$. The argument parallels now that of the proof of \ref{art19-prop1.1} (see \cite{art19-key2}, \S2, where a similar situation is discussed).

If\pageoriginale $f:G\to G'$ is a $k$-homomorphism of semisimple $k$-groups, there exists an induced $k$-morphism $\mathfrak{B}(f):\mathfrak{B}(G)\to \mathfrak{B}(G')$ (by \cite{art19-key2}, 3.1).

\begin{proposition}\label{art19-prop2.2}
\begin{itemize}
\item[\rm(i)] If $f$ is a separable isogeny, then $\mathfrak{B}(f)$ is an isomorphism, compatible with the actions of $G$ and $G'$;

\item[\rm(ii)] $\mathfrak{B}(G\times G')$ is isomorphic to $\mathfrak{B}(G)\times \mathfrak{B}(G')$ as a $k$-variety, the isomorphism being compatible with the actions of the $G$, $G'$, $G\times G'$.
\end{itemize}
\end{proposition}

The proof is left to the reader.

\ref{art19-prop1.3} can only be partially extended to $\mathfrak{B}$.

\begin{proposition}\label{prop2.3}
Let $p$ be good for $G$. Then $\mathfrak{B}(G)$ is nonsingular in codimension $1$.
\end{proposition}

This will not be needed, so we only indicate briefly how this can be proved. If $p$ is good, there exist in $\mathfrak{g}$ regular nilpotent elements (by \cite{art19-key14}, 5.9 b, p. 138, this is also a consequence of \cite{art19-key10}, 5.3, p.8). The orbit $\mathfrak{O}$ in $\mathfrak{B}$ of such an element is open and consists of nonsingular points. One then uses the method of \cite{art19-key15} to prove that all irreducible components of $\mathfrak{B}-\mathfrak{O}$ have codimensions at least $2$.

It is likely that $\mathfrak{B}(G)$ is normal if $G$ is simply connected and $p$ is good. However we are not able to prove this. A proof along the lines of that of \ref{art19-prop1.3} would require the analogue of \ref{art19-prop1.3} (ii). In characteristic $0$ this is a result of Kostant (\cite{art19-key9}). For a proof of the corresponding fact in positive characteristics it seems that one needs detailed information about the ring of $G$-invariant polynomial functions on $\mathfrak{g}$.

If the normality of $\mathfrak{B}$ were known. Theorem \ref{art19-thm3.1} could be ameliorated and its proof could be simplified.

\begin{proposition}\label{art19-prop2.4}
Suppose that $p$ is very good for $G$. Then $\tau :\mathfrak{W}\to \mathfrak{B}$ is birational.
\end{proposition}

The proof is similar to that of \ref{art19-prop1.4}. Instead of results on regular unipotent elements, one now uses those on regular nilpotent elements of $\mathfrak{g}$, which are discussed in (\cite{art19-key14}, 4, p. 138).

\begin{proposition}\label{art19-prop2.5}
The fibres of $\tau : \mathfrak{W}\to \mathfrak{B}$ are connected.
\end{proposition}

A\pageoriginale proof based on Zariski's connectedness theorem, as in \ref{art19-prop1.5}, cannot be given here since we cannot use normality of $\mathfrak{B}$. But Tits' proof works in the case of $\mathfrak{B}$ and carries over with some obvious modifications.

\section{Relation between \texorpdfstring{$V$}{V} and \texorpdfstring{$\mathfrak{B}$}{B}}\label{art19-sec3}

In this number we shall prove the following theorem.

\begin{theorem}\label{art19-thm3.1}
Suppose that $G$ is simply connected and that $p$ is good for $G$. Then there exists a $G$-equivariant $k$-morphism $f:V\to \mathfrak{B}$, which induces a homeomorphism $V(\overline{k})\to \mathfrak{B}(\overline{k})$.
\end{theorem}

The normality of $\mathfrak{B}$ would imply that $f$ is an isomorphism. However \ref{art19-thm3.1} is already sufficient for the applications we want to make. In characteristic $0$, one easily gives a proof of \ref{art19-thm3.1}, using the logarithm in some matrix realization of $G$. For the proof we need a number of auxiliary results. The first three give some rationality results on regular unipotents and nilpotents.

\begin{proposition}\label{art19-prop3.2}
Suppose that $G$ is adjoint and that $p$ is very good for $G$. Let $X$ be a regular nilpotent element of $\mathfrak{g}(k)$. Then its centralizer $Z(X)$ is connected, defined over $k$ and is a $k$-split unipotent group.
\end{proposition}

Recall that a connected unipotent $k$-group is called $k$-split, if there exists a composition series of connected $k$-groups, such that the successive quotients are $k$-isomorphic to $\bfG_{a}$ (see \cite{art19-key11}, p. 97). That $Z(X)$ is connected unipotent is proved in (\cite{art19-key14}, 5.9b, p. 138). Len $N(X)$ be the normalizer of $X$ in $G$ (see \ref{art19-sec0.1}). Under our assumptions, $N(X)$ is also defined over $k$ (\cite{art19-key10}, 6.7, p. 11). Moreover, $N(X)$ is connected. In fact, if $S$ is a maximal torus of its identity component $N(X)^{0}$, then for any $g\in N(X)$, there exists $s\in S$ such that $\Ad(g)X=\Ad(s)X$, whence $N(X)\subset S\cdot Z(X)\subset N(X)^{0}$, since $Z(X)$ is connected.

Now $N(X)$ contains a maximal torus $S$ which is defined over $k$ (by a theorem of Rosenlicht-Grothendieck, see \cite{art19-key1}). Define a character $a$ of $S$ by $\Ad(s)X=s^{a}X$. Then $a$ is clearly defined over $k$; moreover since $Z(X)$ is unipotent and since $N(X)/Z(X)$ has dimension $1$, we have that $\dim S=1$. It follows that $S$ is $k$-split. $S$ acts on $Z(X)$ by inner\pageoriginale automorphisms. We claim that $S$ acts without fixed points. To prove this, we may assume $k$ algebraically closed, moreover it suffices to prove that $S$ acts without fixed points on the Lie algebra $\mathfrak{z}$ of $Z(X)$ (\cite{art19-key3}, 10.1, p. 127).

Since all regular nilpotents are conjugate (\cite{art19-key14}, 5.9c, p. 130), it suffices to prove the assertion for a particular $X$. We may take, with the notations of \ref{art19-sec0.2}, $X=\sum\limits^{l}_{i=1}X_{r_{i}}$ (loc. cit. p. 138). Then one may take for $S$ the subtorus of $T$, which is the identity component of the intersection of the kernels of all $r_{i}-r_{j}$. This $S$ acts without fixed points on $\mathfrak{u}$, hence also on $\mathfrak{z}$, since $\mathfrak{z}\subset \mathfrak{u}$ (\cite{art19-key14}, 5.3, p. 138). By the conjugacy of maximal tori, the assertion now follows for an arbitrary maximal torus of $N(X)$.

$S$ acting without fixed points on $Z(X)$, it follows that $Z(X)$ is $k$-split (\cite{art19-key2}, 9.12). This concludes the proof of \ref{art19-prop3.2}.

The following result generalizes (\cite{art19-key13}, 4.14, p. 135).

\begin{corollary}\label{art19-coro3.3}
Under the assumptions of \ref{art19-prop3.2}, let $Y$ be another regular unipotent in $\mathfrak{g}(k)$. Then there exists $g\in G(k)$ such that $Y=\Ad(g)X$.
\end{corollary}

Let $P=\{g\in G|\Ad (g)X=Y\}$. $P$ is defined over $k$. This is proved in the same way as the fact that $Z(X)$ is defined over $k$ (\cite{art19-key10}, 6.7, p. 11, see \cite{art19-key2}, 6.13 for a similar situation). $P$ is a principal homogeneous space of the $k$-split unipotent group $Z(X)$, hence $P$ has a $k$-rational point $g$ (\cite{art19-key13}, III-8, Prop. 6), which has the required property.

\begin{proposition}\label{art19-prop3.4}
Suppose that $G$ is quasi-split over $k$. Then 
\begin{itemize}
\item[\rm(i)] $G(k)$ contains a regular unipotent element;

\item[\rm(ii)] if $p$ is good, $\mathfrak{g}(k)$ contains a regular nilpotent element.
\end{itemize}
\end{proposition}

Replacing $G$ by its simply connected covering (which is defined over $k$) we may assume $G$ to be simply connected. Then we also may assume $G$ to be simple over $k$ and even absolutely simple (\cite{art19-key18}, 3.1.2, p. 6).
\begin{itemize}
\item[(i)] First\pageoriginale assume $G$ is not of type $\bfA_{l}$ ($l$ even). Let $B$ be a Borel subgroup of $G$ which is defined over $k$. With the notations of \ref{art19-sec0.2}, we take $x=\prod\limits^{l}_{i=1}x_{r_{i}}(\xi_{i})$, where the order of the product and the $\xi_{i}\in k^{*}_{s}$ are chosen such that $x\in G(k)$. This is possible, see (\cite{art19-key15}, proof of 9.4, p. 72 for a similar situation). If $G$ is of type $\bfA_{l}$ ($l$ even) a slightly different argument is needed, similar to the one of (loc. cit. 9.11, p. 74). One could also prove (i) in that case by an explicit check in case $G$ is a special unitary group.
\end{itemize}

The proof of (ii) is similar (but simpler). Take as regular nilpotent $X=\sum\limits^{l}_{i=1}\xi_{i}X_{r_{i}}$, with suitable $\xi_{i}\in k^{*}_{s}$. $X$ is regular by (\cite{art19-key14}, p. 138).

In the next result we shall be dealing with the unipotent part $U$ of a Borel subgroup of a $k$-split $G$, and with its Lie algebra. Notations being as in \ref{art19-sec0.2}, we have the following formula,
\begin{equation}
x_{r}(\xi)x_{s}(\eta)x_{r}(\xi)^{-1}x_{s}(\eta)^{-1}=\prod\limits_{i,j>0}x_{ir+js}(C_{ijrs}\xi^{i}\eta^{j})(\xi,\eta\in k),\label{art19-eq1}
\end{equation}
where $r$, $s\in \Sigma$, $r+s\neq 0$. The product is taken over the integral linear combinations of $r$, $s$ which are in $\Sigma$, and the $C_{ijrs}$ are integers. We presuppose a labelling of the roots in taking the product, the labelling being such that the roots with lower height come first. The height of a positive root $r=\sum\limits^{l}_{i=1}n_{i}(r)r_{i}$ is defined as $h(r)=\sum\limits_{i=1}n_{i}(r)$. Now \eqref{art19-eq1} shows that there exists a groupscheme $U_{0}$ over $\bfZ$, such that $U=\fprod{U_{0}}{k}{\bfZ}$. The same is true for $B$, so that $B=\fprod{B_{0}}{k}{\bfZ}$. $U_{0}$ is isomorphic, as a scheme, to an affine space over $\bfZ$, $B_{0}$ is isomorphic as a scheme to $U_{0}\times \bfG^{1}_{m}$. $B_{0}$ acts on $U_{0}$.

For simplicity, we shall identify $U_{0}(B_{0})$ here with the sets $U_{0}(K)$ ($B_{0}(K)$) of points with values in some algebraically closed filed $K\supset \bfZ$, likewise for $\bfG_{a}$.

Let $s$ be the product of the bad primes for $G$, let $R=\bfZ_{s}$ be the ring of fractions $n/s^{k}(n\in \bfZ)$. Put $U_{1}=\fprod{U_{0}}{R}{\bfZ}$, $B_{1}=\fprod{U_{0}}{R}{\bfZ}$. The homomorphism $x_{r}:G_{a}\to U$ comes from a homomorphism of group schemes over $\bfZ:\bfG_{a}\to U_{0}$, which leads to a homomorphism over\pageoriginale $R:\bfG_{a}\to U_{l}$. The latter one will also be denoted by $x_{r}$, the image of $x_{r}$ is also denoted by $G_{r}$. Let $\mathfrak{u}_{1}$ be the Lie algebra of $U_{1}$. It is a free $R$-module, having a basis consisting of elements tangent to the $G_{r}$. We denote these basis elements by $X_{r}$, as in \ref{art19-sec0.2}. We endow $\mathfrak{u}_{1}$ with its canonical structure of affine space over $R$.

After these preparations, we can state the next result.

\begin{proposition}\label{art19-prop3.5}
There exists a $B_{1}$-equivariant isomorphism of $R$-sche\-mes $\phi:U_{1}\to \mathfrak{u}_{1}$.
\end{proposition}

This is proved by exploiting the argument of (\cite{art19-key14}, pp. 133-134) used to determine the centralizer of regular unipotent and nilpotent elements in good characteristics.

Define $v\in U_{1}(R)$ by $v=\prod\limits^{l}_{i=1}x_{r_{i}}(1)$. Then the argument of loc. cit. extends to the present case and shows that the centralizer $Z$ of $v$ in $U_{l}$ is a closed sub-groupscheme of $U_{1}$, isomorphic as a scheme to $1$-dimensional affine space over $R$. Moreover, since $\fprod{Z}{K}{R}$ is commutative (\cite{art19-key8}, 5.8, p. 1003) it follows that $Z$ is commutative.

We claim that there is a homomorphism of $R$-groupschemes $\psi:\bfG_{a}\to Z$, such that $\psi(\xi)=\prod\limits_{r}x_{r}(F_{r}(\xi))(\xi\in l)$, where $F_{r}$ is a polynomial in $R[T]$ such that $F_{r_{i}}=T(1\leq i\leq l)$. This can be proved by the method of (\cite{art19-key14}, pp. 133-134), defining $F_{r}$ by induction on the height of $r$. It follows, that the Lie algebra $\mathfrak{z}$ of $Z$ contains an element of the form $X=\sum\limits_{r>0}\xi_{r}X_{r}$, with $\xi_{r}\in R$ and $\xi_{r_{i}}=1(1\leq i\leq l)$. Since $X$ is in the Lie algebra of the commutative group scheme $Z$, we have $\Ad(Z)X=X$.

But the same which has been said above about the centralizer of $v$ applies to the centralizer $Z_{1}$ of $X$: this is also a closed sub-group scheme of $U_{1}$, isomorphic to $l$-dimensional affine space over $R$ (since the argument of \cite{art19-key14} applies also to nilpotent elements like $X$).

$Z_{1}$ and $Z$ have the same Krull dimension $l+1$. But since $Z_{1}$ is a closed subscheme of $Z_{1}$, we must have $Z=Z_{1}$. Let $F$ be the closed subscheme of $U_{1}$ consisting of the $\prod\limits_{r>0}x_{r}(\xi_{r})$ such that $\xi_{r_{i}}=1$ $(1\leq i\leq l)$.\pageoriginale Using again the method of (\cite{art19-key14}, p. 133) one defines a morphism $\chi : F\to U_{1}$, such that $\chi(x)v\chi(x)^{-1}=x(x\in F)$.

Let $O$ be the open subscheme of $U_{1}$ consisting of the $\prod\limits_{r>0}x_{r}(\xi_{r})$ such that $\xi_{r_{i}}\neq 0(1\leq i\leq l)$. $\chi$ is easily seen to extend to a morphism $\chi : O\to B_{1}$ such that $\chi(x)v\chi(x)^{-1}=x(x\in O)$. It follows that $O$ is the orbit of $v$ under $B_{1}$. Define a morphism $\phi:O\to \mathfrak{u}_{1}$ by
\begin{equation}
\phi(bvb^{-1})=\Ad (b)X\quad (b\in B_{1}).\label{art19-eq2}
\end{equation}
From the preceding remarks it follows that $\phi$ is well-defined, moreover $\phi(\prod\limits_{r>0}x_{r}(\xi_{r}))$ is a polynomial function in $\xi_{r}$, $\xi^{-1}_{r_{i}}(1\leq i\leq l)$.

We want to show that $\phi$ extends to a morphism $\phi:U_{1}\to \mathfrak{u}_{1}$, satisfying \eqref{art19-eq2}. Now there is a $\fprod{B_{1}}{K}{R}$-equivariant $K$-morphism $\phi_{1}:\fprod{U_{1}}{K}{R}\to \fprod{u_{1}}{K}{R}$ given by the logarithm is a suitable matrix realization of the algebraic group $\fprod{U_{1}}{K}{R}$. $\fprod{\phi}{\Iid}{R}$ extends to a $B_{1}\times K$-equivariant $K$-morphism of an open set of $U_{1}\times K$ which contains $v$ into $\mathfrak{u}_{1}\times K$.

We have that $\phi_{1}(v)$ and $\phi\times\Iid (v)$ are conjugate in $U_{1}(K)$ (by \cite{art19-key14}, 5.3, 5.9 c pp. 137-138). But since $\phi_{1}$ is completely determined by $\phi_{1}(v)$, we have that $\phi\times \Iid=\Ad(b)\circ \phi_{1}$, for suitable $b\in B_{1}(k)$. It follows that $\phi\times \Iid$ can be extended to all of $U_{1}\times K$. Hence $\phi$ can be extended to a morphism $U_{1}\to \mathfrak{u}_{1}$, as desired.

So we have a $B_{1}$-equivariant $R$-morphism $\phi:U_{1}\to \mathfrak{u}_{1}$, with $\phi(v)=X$. Reversing the roles of $U_{1}$ and $\mathfrak{u}_{1}$, one gets in the same manner a $B_{1}$-equivariant $R$-morphism $\phi':\mathfrak{u}_{1}\to U_{1}$ such that $\phi'(X)=v$. But then $\phi$ and $\phi'$ are inverses, so that is an isomorphism. This concludes the proof of \ref{art19-prop3.5}.

\begin{remark*}
The analogue of \ref{art19-prop3.5}, with $R$ replaced by $\bfZ$ and $U_{1}$ by $U_{0}$, $\mathfrak{u}_{1}$ by $\mathfrak{u}_{0}$, is false. In fact, this would imply that, over any field, the centralizer of a regular unipotent element would be connected. This is not true in bad characteristics (\cite{art19-key14}, 4.12, p. 134).
\end{remark*}

We can now prove \ref{art19-thm3.1}. First let $G$ be split over $k$. We use the notations of \ref{art19-sec0.2}. From \ref{art19-prop3.5} we get a $B$-equivariant $k$-isomorphism $\lambda:U\to \mathfrak{u}$. Let $W$, $\pi$; $\mathfrak{W}$, $\tau$ be as in \S\S\ref{art19-sec1} and \ref{art19-sec2}. Then
$$
\theta:(gB,x)\to (gB,\Ad(g)\lambda(g^{-1}xg))
$$\pageoriginale
defines a $G$-equivariant $k$-isomorphism of $W$ onto $\mathfrak{W}$. Denote by $\mathcal{O}_{V}$ the sheaf of local rings on $V$. Since $\pi$ and $\tau$ are proper, we can apply Grothendieck's connectedness theorem (\cite{art19-key7}, III, 4.3.1, p. 130). Using \ref{art19-prop1.3}(ii) and \ref{art19-prop1.4} we find that the direct image
$\pi_{*}(\mathcal{O}_{W})=\mathcal{O}_{V}$. Let $\mathfrak{W}\xrightarrow{\tau_{1}}\mathfrak{B}'\xrightarrow{\tau_{2}}\mathfrak{B}$ be the Stein factorization of $\tau$ (loc.cit. p.131). Then $(\tau_{1})_{*}(\mathcal{O}_{\mathfrak{W}})=\mathcal{O}_{\mathfrak{B}'}$ and $\tau'_{2}$ is finite. The definition of $\mathfrak{B}'$ (\cite{art19-key7}, III, p. 131) shows that $G$ acts on it, and that $\tau_{1}$, $\tau_{2}$ are $G$-equivariant. $V$ and $\mathfrak{B}$ are affine varieties (\ref{art19-prop1.1} and \ref{art19-prop2.1}). Also, since $\mathfrak{B}'$ is finite over $\mathfrak{B}$, $\mathfrak{B}'$ is affine (\cite{art19-key7}, III, 4.4.2, p. 136). It follows then from the definition of direct image that the ring of sections $\Gamma(W,\mathcal{O}_{W})$ is isomorphic to $\Gamma(V,\mathcal{O}_{V})$, likewise that $\Gamma(\mathfrak{W},\mathcal{O}_{\mathfrak{W}})$ is isomorphic to $\Gamma(\mathfrak{B},\mathcal{O}_{\mathfrak{B}'})$, these isomorphisms being compatible with the canonical actions of $G$. This is obvious in the first case, and in the second case it follows again from the definition of $\mathfrak{B}'$. But $V$ and $\mathfrak{B}'$ being affine. $\Gamma(V,\mathcal{O}_{V})$ and $\Gamma(\mathfrak{B}',\mathcal{O}_{\mathfrak{B}'})$ determine $V$ and $\mathfrak{B}'$ completely. Also, $W$ and $\mathfrak{W}$ are isomorphic via $\theta$. Putting this together, we get a $G$-equivariant $k$-isomorphism $\mu:V\to \mathfrak{B}'$.

By (\cite{art19-key7}, III, 4.3.3, p.131), for any $x\in \mathfrak{B}$, the number of connected components of $\tau^{-1}(x)$ equals the number of points of $(\tau_{2})^{-1}(y)$. By \ref{art19-prop2.5} this implies that $\tau_{2}$ is bijective on $\mathfrak{B}'(\overline{k})$. Then $f=\tau_{2}\circ \mu$ satisfies the requirements of \ref{art19-thm3.1}.

Notice that $f$ is not unique, but is completely determined by $f(v)$, where $v$ is a given regular unipotent element. We now turn to the case that $G$ is arbitrary, not necessarily split over $k$. $G$ being simply connected, we may as well suppose that $G$ is absolutely simple (\cite{art19-key10}, 3.1.2, p. 46). We first dispose of the case that $G$ is of type $\bfA$. Then $G$ is a $k$-form of $\bfS\bfL_{n}$. Now there is, in the case of the split group of type $\bfS\bfL_{n}$, a very simple argument to prove \ref{art19-thm3.1}. Identifying in that case $G$ and $\mathfrak{g}$ with subsets of a matrix algebra, $V$ becomes the set of unipotent matrices, $V$ that of nilpotent matrices and we can take $f(v)=v-1$.

If\pageoriginale we have another $k$-form $G$ of $\bfS\bfL_{n}$, then it is obtained from $\bfS\bfL_{n}$ by a twist using a cohomology class in $H^{1}(k,\Aut \bfS\bfL_{n})$. The corresponding form $\mathfrak{g}$ is obtained from $\mathfrak{s}\mathfrak{l}_{n}$ by the same twist. The above $f$ then clearly induces an isomorphism $V\to \mathfrak{B}$ having the required properties.

We may now assume $G$ to be absolutely simple, but not of type $\bfA$. Then if $p$ is good for $G$, it is also very good. Suppose that $G$ is quasi-split over $k$. Let $g\in G(k)$ be regular unipotent, let $X\in \mathfrak{g}(k)$ be regular nilpotent (they exist by \ref{art19-prop3.4}). Since $G$ splits over $k_{s}$, we have, by the first part of the proof, a $G$-equivariant $k_{s}$-isomorphism $f:V\to \mathfrak{B}$. By \ref{art19-coro3.3} there exists $h\in G(k_{s})$ such that $\Ad(h)f'(v)=X$. But then $f=f'\circ \Ad(h)$ is $G$-equivariant, defined over $k_{s}$ and satisfies ${}^{s}f=f$ for all $s\in \Gal(k_{s}/k)$. Hence $f$ is defined over $k$. Finally, an arbitrary $k$-form $G$ is obtained by an inner twist from a quasi-split $k$-form $G$ (this is implicit, for example, in \cite{art19-key18}, 3). Let $f_{1}$ have the required properties for $G_{1}$. One easily checks then that $f_{1}$ determines an $f$ having the properties of \ref{art19-thm3.1}. This concludes the proof of \ref{art19-thm3.1}.

\begin{corollary}\label{art19-coro3.6}
Suppose that $G$ is adjoint and that $p$ is very good for $G$. Let $x$ be a regular unipotent element of $G(k)$. Then its centralizer $Z(x)$ is connected, defined over $k$ and is a $k$-split unipotent group.
\end{corollary}

Let $f$ be as in \ref{art19-thm3.1}. Then $X=f(x)$ is a regular nilpotent element in $\mathfrak{g}$. We have $Z(x)=Z(X)$. The assertion then follows from \ref{art19-prop3.2}.

\begin{corollary}\label{art19-coro3.7}
Under the assumptions of \ref{art19-coro3.6}, let $y$ be another regular unipotent element in $G(k)$. Then there exists $g\in G(k)$ such that $y=gxg^{-1}$.
\end{corollary}

The proof is similar to that of \ref{art19-coro3.3}.

\begin{remark}\label{art19-rem3.8}
The condition in \ref{art19-coro3.6} and \ref{art19-coro3.7} that $p$ be a very good prime cannot be relaxed. As an example, consider the case where $G=\mathbf{PSL}_{2}$ and where $k$ is a non-perfect field of characteristic $2$. The ring of regular functions of $\bfS\bfL_{2}$ being identified to $A=k[X,Y,Z,U]/(XU-YZ-1)$, that of $G$ is isomorphic to the subring of $A$ generated by the products of an even number of variables.\pageoriginale Hence one can identify $\mathbf{PSL}_{2}(k)$ with the subgroup of $\mathbf{SL}_{2}(\overline{k})$, consisting of the matrices $a$ such that $\rho a\in GL_{2}(k)$ for some $\rho\in \overline{k}$ with $\rho^{2}\in k^{*}$.
\end{remark}

In our situation, let $\rho\in \overline{k}$, $\rho^{2}\in k^{*}$. Then $\left(\begin{matrix} 1 & 1\\ 0 & 1\end{matrix}\right)$ and $\left(\begin{matrix} 0 & \rho\\ \rho^{-1} & 0\end{matrix}\right)$ are both regular unipotents in $\mathbf{PSL}_{2}(k)$, but it is easily checked that they are not conjugate by an element of $\mathbf{PSL}_{2}(k)$. On the other hand, if $k$ is perfect, \ref{art19-coro3.6} and \ref{art19-coro3.7} are already true if $p$ is good. But if $p$ is bad, both \ref{art19-coro3.6} and \ref{art19-coro3.7} are false (see \cite{art19-key14}, 4.14, p. 135 for the first statement and 4.12, p. 134, 4. 15c, p. 136 for the others).

\begin{corollary}\label{art19-coro3.9}
With the notations of \ref{art19-thm3.1}, we have $f(e)=O$.
\end{corollary}

For $e$ is the only unipotent element of $G$ in the centre of $G$ and $O$ is the only nilpotent element of $\mathfrak{g}$ invariant under $\Ad(G)$ (the last assertion follows for example, by using the fact that any nilpotent element is contained in the Lie algebra of a Borel subgroup).

\section{Applications}\label{art19-sec4}

First we give some applications of \ref{art19-thm3.1} to rationality problems.

\begin{proposition}\label{art19-prop4.1}
Let $k$ be a finite field with $q$ elements. Suppose that $G$ is simply connected and that $p$ is good for $G$. Then the number of nilpotent elements in $\mathfrak{g}(k)$ is $q^{\dim G-\rank G}$.
\end{proposition}

Steinberg has proved that the number of unipotent elements in $G(k)$ is $q^{\dim G-\rank G}$ (\cite{art19-key16}, 15.3, p. 98). The assertion then follows from \ref{art19-thm3.1}.

\begin{proposition}\label{art19-prop4.2}
Suppose that $p$ is very good for $G$. Then the following conditions are equivalent:
\begin{itemize}
\item[\rm(i)] $G$ is anisotropic;

\item[\rm(ii)] $G(k)$ does not contain unipotent elements $\neq e$.
\end{itemize}
\end{proposition}

We recall that $G$ is called anisotropic, if $G$ does not contain a nontrivial $k$-subtorus $S$, which is $k$-split, i.e. $k$-isomorphic to a product of multiplicative groups.

If $G(k)$ contains a unipotent $\neq e$, then \ref{art19-thm3.1} and \ref{art19-rem3.8} imply that $\mathfrak{g}(k)$ contains a nilpotent $\neq O$. One then argues as in (\cite{art19-key10}, 6.8, p. 11) to show that $G$ contains a $k$-split sub-torus $S$. Hence (i) $\Rightarrow$ (ii). Conversely,\pageoriginale if $G$ contains such a subtorus $S$, then $G$ has a proper parabolic $k$-subgroup (\cite{art19-key3}, 4.17, p. 92). Its unipotent radical $R$ is a $k$-split unipotent group (\cite{art19-key3}, 3.18, p. 82) and it follows that $R(k)\neq \{e\}$, so that $G$ has a rational unipotent $\neq e$.

For perfect $k$ and good $p$, \ref{art19-prop4.2} was proved in (\cite{art19-key10}, 6.3, p. 10). More general results were announced in \cite{art19-key17}.

\begin{proposition}\label{art19-prop4.3}
Suppose that $p$ is good for $G$. For any $g\in G$, $\dim G-\dim Z(g)$ is even.
\end{proposition}

This was conjectured for arbitrary $p$ in (\cite{art19-key15}, 3.10, p. 56) and is known to be true in characteristic 0 (\cite{art19-key9}, Prop. 15, p. 364).

We may assume that $k$ is algebraically closed and $G$ simply connected. Let $g=g_{s}g_{u}$ be the decomposition of $G$ into its semisimple and unipotent parts. Since $Z(g_{s})$ is reductive and of the same rank on $G$, it follows readily that $\dim G-\dim Z(g_{s})$ is even. Because $Z(g)$ is the centralizer of $g_{u}$ in $Z(g_{s})$, it follows that it suffices to consider the case that $g$ is unipotent. Then \ref{art19-thm3.1} implies, that \ref{art19-prop4.2} is equivalent to the assertion that $\dim G-\dim Z(X)$ is even for any nilpotent $X\in \mathfrak{g}$. But $\dim Z(X)$ equals the dimension of the Lie algebra centralizer $\mathfrak{z}(X)$ of $X$ (\cite{art19-key10}, 6.6, p. 11). So we have to prove that $\dim \mathfrak{g}-\dim \mathfrak{z}(X)$ is even if $X$ is nilpotent in $g$. This we do by an adaptation of the method used in characteristic 0 (\cite{art19-key9}, loc. cit.), even for arbitrary $X$. We use the following lemma.

\begin{lemma}\label{art19-lem4.4}
\begin{itemize}
\item[\rm(i)] Suppose $G$ is simple, not of type $\bfA_{n}$. If $p$ is good for $G$, there exists a nondegenerate, symmetric bilinear form $F$ on $\mathfrak{g}$, which is invariant under $\Ad(G)$.

\item[\rm(ii)] There exists a nondegenerate symmetric bilinear form $F$ on $\mathfrak{g}\mathfrak{l}_{l}(k)$, which is invariant under $\Ad GL_{l}(k)$.
\end{itemize}
\end{lemma}

\begin{description}
\item[Proof of \ref{art19-lem4.4} {\rm (i)}]
If $G$ is of type $\bfB_{l}$, $\bfC_{l}$, $\bfD_{l}$, then $p\neq 2$ and we can represent $G$ as a group of orthogonal or symplectic matrices in a vector space $A$, and $\mathfrak{g}$ by a Lie algebra of skewsymmetric linear transformations with respect to the corresponding symmetric or skewsymmetric bilinear form on $A$. $F(X,Y)=\Tr(XY)$ satisfies then our conditions.

If\pageoriginale $G$ is of type $\bfE_{6}$, $\bfE_{7}$, $\bfE_{8}$, $\bfF_{4}$, $\bfG_{2}$, the Killing form on $\mathfrak{g}$ is non-degenerate if $p$ is good (\cite{art19-key12}, p. 551) and can be taken as our $F$.

\item[{\rm(ii)}] $F(X,Y)=\Tr(XY)$ satisfies our conditions.
\end{description}

To finish the proof of \ref{art19-prop4.3}, we can assume $G$ to be simple. Let $X\in \mathfrak{g}$. First if $G$ is not of type $\bfA_{l}$, we let $F$ be as in \ref{art19-lem4.4} (i). consider the skewsymmetric bilinear form $F_{1}$ on $\mathfrak{g}$ defined by $F_{1}(Y,Z)=F([XY],Z)$. Then $\mathfrak{z}(X)=\{Y\in \mathfrak{g}|F_{1}(Y,Z)=0\text{~ for all~ } Y\in \mathfrak{g}\}$. Since the rank of $F_{1}$ is even, $\dim \mathfrak{g}-\dim \mathfrak{z}(X)$ is even, which is what we wanted to prove. If $G$ is of type $\bfA_{l}$, we apply the same argument, however not for $\mathfrak{s}\mathfrak{l}_{l+1}$ but for $\mathfrak{g}\mathfrak{l}_{l+1}$, using \ref{art19-prop4.3} (ii).

\begin{proposition}\label{art19-prop4.5}
Suppose that $G$ is adjoint and that $p$ is good for $G$. Let $g$ be a unipotent element of $G$. Then $g$ lies in the identity component $Z(g)^{0}$ of its centralizer $Z(g)$.
\end{proposition}

Let $f:V\to \mathfrak{B}$ be as in \ref{art19-thm3.1}. Put $X=f(g)$, let $A=f^{-1}(\overline{k}X)$. Since $f$ is a homeomorphism, this is a closed connected subset of $V$, containing (by \ref{art19-rem3.8}) $e$ and $g$. Moreover, since $Z(X)=Z(g)$, we find from the $G$-equivariance of $f$ that $A\subset Z(g)$. It follows that $g\in Z(g)^{0}$.

\begin{remark*}
In bad characteristics the assertion of \ref{art19-prop4.5} is not true (\cite{art19-key14}, 4.12, p. 134).
\end{remark*}

The number of unipotent conjugacy classes in $G$ (resp. of nilpotent conjugacy classes in $\mathfrak{g}$) has been proved to be finite in good characteristics by Richardson (\cite{art19-key10}, 5.2, 5.3, p. 8). By \ref{art19-thm3.1}, these two numbers are equal. In characteristic 0, there is a bijection of the set of unipotent conjugacy classes in $G$ onto the set of conjugacy classes (under inner automorphisms) of 3-dimensional simple subgroups of $G$ (see e.g. \cite{art19-key8}, 3.7, p. 988). Representatitives for the classes of such subgroups are known (see \cite{art19-key6}). In characteristic $p>0$ it is not advisable to work with 3-dimensional subgroups, one has to deal then directly with the unipotent elements. We will discuss this in another paper. Here we only want to point out one consequence of \ref{art19-thm3.1}. We define a unipotent element $g\in G$ to be {\em semi-regular} if its centralizer $Z(g)$ is the product of the center of $G$ with a unipotent group. Regular\pageoriginale unipotent elements are semi-regular (as follows from \cite{art19-key15}, 3.1, 3.2, 3.3, pp. 54-55). The converse, however, fails already in characteristic 0. In that case, it has been proved by Dynkin (\cite{art19-key6}, 9.2, p. 169 and 9.3, p. 170) that for $G$ simple semi-regular implies regular if and only if $G$ is of type $\bfA_{l}$, $\bfB_{l}$, $\bfC_{l}$, $\bfF_{4}$, $\bfG_{2}$. The result we want to prove is the following one, which extends (\cite{art19-key14}, 4.11, p. 134).

\begin{proposition}\label{art19-prop4.6}
Suppose that $G$ is adjoint and that $p$ is good for $G$. Let $g$ be a semi-regular unipotent element of $G$. Then the centralizer $Z(g)$ is connected.
\end{proposition}

By \ref{art19-thm3.1}, $Z(g)\subset V$ is homeomorphic to the variety $A$ of fixed points of $\Ad(g)$ in $\mathfrak{B}$. We claim that $A$ is the set of fixed points of $\Ad(g)$ in the whole of $\mathfrak{g}$. For let $X\in \mathfrak{g}$, $\Ad(g)X=X$. Let $X=X_{s}+X_{n}$ be the Jordan decomposition of $X$(\cite{art19-key1}, 1.3, p. 27), then $\Ad(g)X_{s}=X_{s}$. But this means that $X_{s}$ in the Lie algebra of $Z(g)$ (\cite{art19-key10}, 6.6, p. 11). $Z(g)^{0}$ being unipotent, it follows that $X_{s}=0$. This establishes our claim. It follows that $A$ is a linear subspace of $\mathfrak{g}$, so that it is connected, hence so is $Z(g)$.

\begin{thebibliography}{99}
\bibitem{art19-key1} \textsc{A. Borel} and \textsc{T. A. Springer :} Rationality properties of linear algebraic groups, {\rm Proc. Symp. Pure Math.} IX (1966), 26-32.

\bibitem{art19-key2} \textsc{A. Borel} and \textsc{T. A. Springer :} Rationality properties of algebraic groups II, {\em Tohoku Math. Jour. 2nd Series,} 20 (1968), 443-497, to appear.

\bibitem{art19-key3} \textsc{A. Borel} and \textsc{J. Tits :} Groupes reductifs, {\em Publ. Math. I.H.E.S.} 27 (1965), 55-150.

\bibitem{art19-key4} \textsc{C. Chevalley :} Sur certains groupes simples, {\em Tohoku Math. Journal,} 2nd Series 7 (1955), 14-66.

\bibitem{art19-key5} \textsc{C. Chevalley :}\pageoriginale {\em S\'eminaire sur la classification des groupes de Lie alg\'ebriques}, 2 vol., Paris, 1958.

\bibitem{art19-key6} \textsc{E. B. Dynkin :} Semisimple subalgebras of semisimple Lie algebras, {\em Amer. Math. Soc. Transl. Ser.} 2, 6 (1957), 111-245 (={\em math. Sbornik} N. S. 30 (1952), 349-362).

\bibitem{art19-key7} \textsc{A. Grothendieck} and \textsc{J. Dieudonn\'e :} {\em \'El\'ements de g\'eom\'etrie alg\'ebrique,} III, Publ. Math. I.H.E.S., No. 11 (1961); IV (seconde partie), {\em ibid.,} No. 24 (1965).

\bibitem{art19-key8} \textsc{B. Kostant :} The principal three-dimensional subgroup and the Betti numbers of a complex simple Lie group, {\em Amer. J. Math.} 81 (1959), 973-1032.

\bibitem{art19-key9} \textsc{B. Kostant :} Lie group representations in polynomial rings {\em ibid.,} 85 (1963), 327-404.

\bibitem{art19-key10} \textsc{R. W. Richardson, Jr. :} Conjugacy classes in Lie algebras and algebraic groups, {\em Ann. of Math.} 86 (1967), 1-15. 

\bibitem{art19-key11} \textsc{M. Rosenlicht :} Questions of rationality for solvable algebraic groups over nonperfect fields, {\em Annali di Mat.} (IV), 61 (1963), 97-120.

\bibitem{art19-key12} \textsc{G. B. Seligman :} Some remarks on classical Lie algebras, {\em J. Math. Mech.} 6 (1957), 549-558.

\bibitem{art19-key13} \textsc{J.-P. Serre :} {\em Cohomologie Galoisienne :} Lecture Notes in mathematics, No. 5, Springer-Verlag, 1964.

\bibitem{art19-key14} \textsc{T. A. Springer :} Some arithmetical results on semisimple Lie algebras, {\rm Publ. Math. I.H.E.S. No.} 30, (1966), 115-141.

\bibitem{art19-key15} \textsc{R. Steinberg :} Regular elements of semisimple algebraic groups, {\em Publ. Math. I.H.E.S. No.} 25 (1965), 49-80.

\bibitem{art19-key16} \textsc{R. Steinberg :} Endomorphisms of algebraic groups, {\em Memoirs Amer. Math. Soc. No.} 80 (1968).

\bibitem{art19-key17} \textsc{J. Tits :} Groupes semi-simples isotropes. {\em Colloque sur la th\'eorie des groupes alg\'ebriques,} C.B.R.M., Bruxelles, 1962, 137-147.

\bibitem{art19-key18} \textsc{J. Tits :} Classification of algebraic semisimple groups, {\em Proc. Symp. Pure Math.} IX (1966), 33-62.

\end{thebibliography}

\bigskip
\noindent
{\small University of Utrecht.}

