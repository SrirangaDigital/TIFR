\title{THE UNIPOTENT VARIETY OF A SEMISIMPLE GROUP}
\markright{The Unipotent Variety of a Semisimple Group}

\author{By~~ T. A. Springer}
\date{}

\maketitle

\setcounter{pageoriginal}{372}
\textsc{If}\pageoriginale $G$ is a connected semisimple linear algebraic group over a field of characteristic $0$, one easily sees that the variety $V$ of unipotent elements of $V$ is isomorphic to the variety $\mathfrak{B}$ of nilpotent elements of its Lie algebra $\mathfrak{g}$; moreover one can choose the isomorphism so as to be compatible with the canonical actions of $G$ on $V$ and $\mathfrak{B}$. In this note the analogous situation in characteristic $p>0$ will be discussed. We have to restrict $p$ to be ``good'' for $G$ (see \ref{art19-sec0.3}). This is not so surprising, since in ``bad'' characteristics there are anomalies in the behaviour of unipotent elements.

Due to technical difficulties, we cannot prove the {\em isomorphism} of $V$ and $\mathfrak{B}$ for $p>0$, but only a slightly weaker result (Theorem \ref{art19-thm3.1}). This is, however, sufficient for the applications which are discussed in \S\ref{art19-sec4}.

\setcounter{section}{-1}
\section{Notations and recollections}\label{art19-sec0}

\subsection{}\label{art19-sec0.1} 
$k$ denotes a field, $\overline{k}$ an algebraic closure of $k$ and $k_{s}$ a separable closure. $p$ is the characteristic of $k$.

An algebraic variety $V$ defined over $k$ (or a $k$-variety) is a scheme which is of finite type and absolutely reduced over $k$. $V(k)$ denotes its set of $k$-rational points. We may and shall identify $V$ with $V(\overline{k})$, or $V(k_{s})$.

An algebraic group $H$ defined over $k$ (or a $k$-group) will mean here a {\em linear} algebraic group, i.e. an affine group scheme, of finite type and smooth over $k$. $H^{0}$ denotes the identity component of $H$.

The Lie algebra of an algebraic group $H$ will be denoted by the corresponding German letter $\mathfrak{h}$. $H$ acts on $\mathfrak{h}$ via the adjoint representation Ad. If $x\in H$, $Z(x)$ denotes the centralizer of $x$ in $H$, $\mathfrak{z}(x)=\{X\in \mathfrak{h}|\Ad (x)X=X\}$ the centralizer of $x$ in $\mathfrak{h}$. If $X\in \mathfrak{h}$, $Z(X)=\{x\in H|\Ad (x)X=X\}$ is the centralizer of $X$ in $H$, $\mathfrak{z}(X)$ its\pageoriginale centralizer in $\mathfrak{h}$ and $N(X)=\{x\in H|\Ad (x)X\in \overline{k}^{*}X\}$ its normaliser in $H$.

\subsection{}\label{art19-sec0.2}
$G$ always denotes a connected semisimple linear algebraic group, defined over a field $k$.

Let $T$ be a maximal torus of $G$, $B$ a Borel subgroup containing $T$ and $U$ the unipotent part of $B$. Denote by $\Sigma$ the set of roots of $G$ with respect to $T$. $B$ determines an order on $\Sigma$, let $\Sigma^{+}$ be the set of positive roots. We denote by $r_{1},\ldots,r_{l}$ the corresponding simple roots. For $r\in \Sigma$ there is an isomorphism $x_{r}$ of the additive group $\bfG_{a}$ onto a closed subgroup $G_{r}$ of $G$, such that
$$
t.x_{r}(\xi).t^{-1}=x_{r}(t^{r}\xi)\quad (\xi\in \overline{k}),
$$
where $t^{r}$ denotes the value of the character $r$ of $T$ in $t\in T$. $U$ is generated by the $G_{r}$ with $r>0$. $G_{r}$ and $G_{-r}$ generate a subgroup $P_{r}$ which is connected semisimple of type $\bfA_{1}$. $X_{r}\in \mathfrak{g}$ will denote a nonzero tangent vector to $G_{r}$. We will say that $G$ is simple if $\Sigma$ is irreducible.

\subsection{}\label{art19-sec0.3}
Let $G$ be simple. Then there is a unique highest root $r$ in $\Sigma$, for the given order. Express $r$ as an integral linear combination of the $r_{i}$. The characteristic $p$ of $k$ is called {\em bad} for $G$, if $p$ is a prime number dividing one of the coefficients in this expression. Otherwise $p$ is called {\em good} for $G$. If $p$ is good and if moreover $p$ does not divide the order of the centre of the simply connected covering of $G$, then $p$ is called {\em very good} for $G$. $p=0$ is always very good. For the simple types, the bad $p>0$ are:

$\bfA_{l}$: none; $\bfB_{l}$, $\bfC_{l}$, $\bfD_{l}$ : $p=2$; $\bfE_{6}$, $\bfE_{7}$, $\bfF_{4}$, $\bfG_{2} : p = 2$, $3$; $\bfE_{8} : p=2,3,5$. A good $p$ is very good, unless $G$ is of type $\bfA_{l}$ and $p$ divides $l+1$. If $G$ is arbitrary, $p$ is defined to be good or very good for $G$ if it is so far all simple normal subgroups of $G$.

\subsection{}\label{art19-sec0.4}
$x\in G$ is called regular, if $\dim Z(x)$ equals rank $G$. We shall have to make extensive use of the properties of regular unipotent elements of $G$, which are established in \cite{art19-key14}, \cite{art19-key15}.

\section{The unipotent variety of $G$}\label{art19-sec1}

Let $V(G)$ (or $V$, if no confusion can arise) denote the set of unipotent elements of $G$. Then $V(G)$ is closed\pageoriginale in $G$. We call $V(G)$ the {\em unipotent variety} of $G$. The following result justifies this name.

\begin{proposition}\label{art19-prop1.1}
$V(G)$ is an irreducible closed subvariety of $G$ of dimension $\dim G - \rank G$. $G$ acts on $V(G)$ by inner automorphisms. $V(G)$ and the action of $G$ on it are defined over $k$.
\end{proposition}

Except for the last statement, this contained in (\cite{art19-key14}, 4.4, p. 131). We sketch the proof since we have to refer to it later on.

Suppose first that $G$ is quasi-split over $k$. Let $B$ be a Borel subgroup of $G$, which is defined over $k$. Then $U$ is also defined over $k$. Consider the subset $W$ of $G/B\times G$, consisting of the $(gB,x)$ such that $g^{-1}xg\in U$. This is a closed subvariety of $G/B\times G$ (\cite{art19-key5}, exp. 6, p. 12, 1.13). It follows from loc. cit. that $W$ is irreducible and defined over $k$. $V$ is the projection of $W$ onto the second factor of $G/B\times G$, let $\pi :W\to V$ be the corresponding morphism. That $V$ has the asserted dimension is proved in (\cite{art19-key14}, loc. cit). $G/B$ being a projective $k$-variety, $\pi$ is proper, defined over $k$. $V$ is closed in $G$ and defined over $k$.

Let $G$ act on $G/B\times G$ by $(h,(gB,x))\to (hgB,hxh^{-1})$. This action is defined over $k$, $W$ is stable and $\pi$ is a $G$-equivariant $k$-morphism $W\mapsto V$, $G$ acting on $V$ as in the statement of the proposition.

If $k$ is arbitrary, $G$ splits over $k_{s}$ (see \cite{art19-key2}, 8.3 for example). It follows that $V$ is defined over $k_{s}$. Since $V$ is clearly stable under $\Gamma=\Gal (k_{s}/k)$, it is defined over $k$. $W$ is also defined over $k$, moreover if $s\in \Gamma$ there exists $g_{s}\in G(k_{s})$ such that
$$
{}^{s}B=g_{s}Bg^{-1}_{s}, \ {}^{s}U=g_{s}Ug^{-1}_{s}.
$$
Define a new action $(s,w)\mapsto {}_{s}w$ of $\Gamma$ on $W$ by ${}_{s}(gB,x)=({}^{s}g\cdot g_{s}B,{}^{s}x)$. $W$ is stable under this action of $\Gamma$, hence this defines a structure of $k$-variety on $W$, such that the projection $\pi : W\to V$ is a $G$-equivariant $k$-morphism.

If $G$ and $G'$ are two semisimple $k$-groups, and $f$ a $k$-homomorphism, there exists an induced $k$-morphism $V(f):V(G)\to V(G')$.

\begin{proposition}\label{art19-prop1.2}
\begin{itemize}
\item[\rm(i)] If $f$ is a separable central isogeny, then $V(f)$ is an isomorphism, compatible with the actions of $G$ and $G'$, 

\item[\rm(ii)] $V(G\times G')$\pageoriginale is isomorphic to $V(G)\times V(G')$ as a $k$-variety, the isomorphism being compatible with the actions of $G$, $G'$, $G\times G'$. 
\end{itemize}
\end{proposition}

The proof of Proposition \ref{art19-prop1.2} is easy.

The following results are essentially contained in \cite{art19-key15}.

\begin{proposition}\label{art19-prop1.3}
\begin{itemize}
\item[\rm(i)] $V(G)$ is nonsingular in codimension $1$;

\item[\rm(ii)] $V(G)$ is normal if $G$ is simply connected, or if $p$ does not divide the order of the centre of the simple connected covering of $G$.
\end{itemize}
\end{proposition}

We may assume $k$ to be algebraically closed. In $V$ we have the open subvariety $O$ of the regular elements. $O$ is an orbit of $G$ and all its elements are simple points of $V$ (see \cite{art19-key15}, 1.2, 1.5 for these statements). (i) then follows from the fact, proved loc. cit. (6.11 e), that the irreducible components of $V-O$ have codimension $\geq 2$.

If $G$ is simply connected, then by (\cite{art19-key15}, 6.1, 8.1) $V$ is a complete intersection in $G$. The first statement in (ii) then follows from known normality criteria (e.g. \cite{art19-key7}, iv, 5.8.6, p. 108) and the second one is a consequence of \ref{art19-sec1.2} (i).

We now prove some properties of the proper $k$-morphism $\pi:W\to V$, introduced in the proof of \ref{art19-sec1.1}.

\begin{proposition}\label{art19-prop1.4}
If $G$ is adjoint, then $\pi$ is birational.
\end{proposition}

Since a regular unipotent element is contained in exactly one Borel subgroup, $\pi$ is bijective on $\pi^{-1}(O)$ ($O$ denoting as before the variety of regular elements). \ref{art19-sec1.4} will then follow, if we show that $\pi$ is separable (see e.g. \cite{art19-key17}, III, 4.3.7, p. 133). We may assume $k$ to be algebraically closed. Let $\phi$ be the morphism $G\times U\to G$ such that $\phi(g,u)=gug^{-1}$. From the definition of $\pi$ it follows that there exists a morphism $\psi : G\times U\to W$ such that $\phi=\pi\circ\psi$. To prove the separability of $\pi$, it suffices to prove that in some point $a\in G\times U$, the tangent map $(d\phi)_{a}:T(G\times U)_{a}\to T(G)_{\phi(a)}$ has image of dimension $\dim V$.

Let $u$ be a regular unipotent element in $U$. We will take $a=(e,u)$. Identify the tangent space $T(G\times U)_{a}$ with $\mathfrak{g}\oplus \mathfrak{u}$ (via a right translation with $(e,u)$), identify $T(G)_{\phi(a)}$ with $\mathfrak{g}$. Then $(d\phi)_{a}$ becomes the homomorphism $\alpha :\mathfrak{g}\oplus \mathfrak{u}\to \mathfrak{g}$, which sends $(X,U)$ into $(1-\Ad(u)^{-1})X+U$.\pageoriginale It follows that $\alpha(\mathfrak{g}\oplus \mathfrak{u})$ contains $(\Ad (u)-1)\mathfrak{g}$, which, by (\cite{art19-key15}, 4.3, p. 58) has dimension equal to $\dim V$. (Since $G$ is adjoint, the $\mathfrak{z}$ of loc. cit. is now the zero element). This implies the result.

\begin{remark*}
By dimensions one finds also that $\mathfrak{u}\subset (\Ad (u)-1)\mathfrak{g}$.
\end{remark*}

\begin{proposition}\label{art19-prop1.5}
The fibres of $\pi$ are connected.
\end{proposition}

We may assume $k$ to be algebraically closed. In view of the definition of $W$ this can be stated in another way, namely that, $B$ denoting some Borel subgroup of $G$, the fixed point set in $G/B$ of any unipotent element $g\in G$ is connected. Identifying $G/B$ with the variety of Borel subgroups of $G$, \ref{art16-sec1.5} can also be interpreted as follows: the closed subvariety of $G/B$ consisting of the Borel subgroups containing $g$, is connected. \ref{art19-sec1.5} follows, if $G$ is simply connected, from \ref{art19-sec1.3} (ii) and \ref{art19-sec1.4} by Zariski's connectedness theorem (see \cite{art19-key7}, 4.3.7, p. 133). The general case then follows at once, since central isogenies do not affect the statement.

Another proof of \ref{art19-sec1.5} was given by J. Tits. Since his method of proof will be useful in \S\ref{art19-sec2}, we will reproduce his proof here. We interpret $G/B$ as the variety of Borel subgroups of $G$. Let $g\in B$ be a unipotent element of $G$, let $B'$ be another Borel subgroup containing $g$. By Bruhat's lemma, $B\cap B'$ contains a maximal torus $T$. Let $N$ be its normalizer and $\mathscr{W}=N/T$ be the Weyl group. For $w\in \mathscr{W}$, denote by $n_{w}$ a representative in $N$. There exists then $w\in \mathscr{W}$ such that $B'=n_{w}Bn^{-1}_{w}$. $B$ determines an order on the root system $\Sigma$, let $w_{1},\ldots,w_{l}$ ($l=\rank G$) be the reflections in $\mathscr{W}$ defined by the corresponding simple roots. Since the $w_{i}$ generate $\mathscr{W}$, we can write $w$ as a product $w=w_{i_{1}}\ldots w_{i_{t}}$. We take $t$ as small as possible. Put $v_{0}=1$, $v_{h}=w_{i_{1}}\ldots w_{i_{h}}(1\leq h\leq t)$, $B_{h}=n_{v_{h}}Bn^{-1}_{v_{h}}$, so that $B_{0}=B$, $B_{t}=B'$. Let $\Sigma_{h}$ denote the set of $r\in \Sigma$ such that $r>0$, $v_{h}r<0$. It is known that the minimality of $t$ implies that $\Sigma_{h}\subset \Sigma_{h+1}$ (this follows e.g. from \cite{art19-key5}, p. 14-06, lemma). The intersection $B_{0}\cap B_{h}$ is generated by $T$ and the subgroups $G_{r}$ (see \ref{art19-sec0.2}) with $r\not\in \Sigma_{h}$ (\cite{art19-key5}, exp. 13, No. 2). Hence $B_{h}\supset B_{0}\cap B_{t}=B\cap B'$, in particular $g$ belongs to all $B_{h}$. Let $X\subset G/B$ be the variety of Borel subgroups containing $g$. It suffices\pageoriginale to show that we can connect $B_{h}$ and $B_{h+1}$ inside $X$ by a projective line.

Put $u=v_{h}w_{h+1}v^{-1}_{h}$, then $B_{h+1}=n_{u}B_{h}n^{-1}_{u}$, moreover $u$ is a reflection in a simple root for the order on $\Sigma$ determined by $B_{h}$. Changing the notation, we are reduced to proving that $B$ and $B'$ are connected inside $X$ by a projective line, if $w$ is a reflection in a {\em simple} root $r>0$. Then let $P_{r}$ be the subgroup of $G$ generated by $G_{r}$ and $G_{-r}$. $P_{r}$ is of type $\bfA_{1}$ and we may take $n_{w}\in P_{r}$. One easily checks that $hBh^{-1}\supset B\cap B'$ for all $h\in P_{r}$. $P_{r}\cap B$ is a Borel subgroup of $P_{r}$. Let $\psi:P_{r}/P_{r}\cap B\to G/B$ be the canonical immersion. $L=P_{r}/P_{r}\cap B$ is a projective line and $\psi(L)$ contains both $B$ and $B'$. This establishes our assertion.

\section{The nilpotent variety of $G$}\label{art19-sec2}

We discuss now the Lie algebra analogues of the results of \S\ref{art19-sec1}. We recall that an element $X\in \mathfrak{g}$ is called {\em nilpotent} if it is tangent to a unipotent subgroup of $G$. Equivalently, $X$ is nilpotent if it is represented by a nilpotent matrix in any matrix realization of $G$ (see \cite{art19-key1}, \S1, pp. 26-27). Let $\mathfrak{B}(G)$ (or $\mathfrak{B}$) denote the set of nilpotent elements in $\mathfrak{g}$, $\mathfrak{g}$ being endowed with the obvious structure of affine space over $k$. Then $V$ is a closed subset of $\mathfrak{g}$.

\begin{proposition}\label{art19-prop2.1}
$\mathfrak{B}(G)$ is an irreducible closed subvariety of $\mathfrak{g}$ of dimension $\dim G -\rank G$. $G$ acts on $\mathfrak{B}(G)$ via the adjoint representation of $G$. $\mathfrak{B}(G)$ and the action of $G$ on it are defined over $k$.
\end{proposition}

The proof is similar to that of \ref{art19-prop1.1}. First let $G$ be quasi-split over $k$. We use the notations of the proof of \ref{art19-prop1.1}. Instead of $W$, we consider now the closed subvariety $\mathfrak{W}$ of $G/B\times \mathfrak{g}$, consisting of the $(gB,X)$ such that $\Ad(g)^{-1}X\in \mathfrak{u}$. $G$ acts on $\mathfrak{W}$ by
$$
(h,(gB,X))\mapsto (hgB,\Ad(h)X).
$$
The projection of $G/B\times \mathfrak{g}$ onto its second factor induces a $G$-equivariant proper morphism $\tau : \mathfrak{W}\to \mathfrak{B}$. The argument parallels now that of the proof of \ref{art19-prop1.1} (see \cite{art19-key2}, \S2, where a similar situation is discussed).

If\pageoriginale $f:G\to G'$ is a $k$-homomorphism of semisimple $k$-groups, there exists an induced $k$-morphism $\mathfrak{B}(f):\mathfrak{B}(G)\to \mathfrak{B}(G')$ (by \cite{art19-key2}, 3.1).

\begin{proposition}\label{art19-prop2.2}
\begin{itemize}
\item[\rm(i)] If $f$ is a separable isogeny, then $\mathfrak{B}(f)$ is an isomorphism, compatible with the actions of $G$ and $G'$;

\item[\rm(ii)] $\mathfrak{B}(G\times G')$ is isomorphic to $\mathfrak{B}(G)\times \mathfrak{B}(G')$ as a $k$-variety, the isomorphism being compatible with the actions of the $G$, $G'$, $G\times G'$.
\end{itemize}
\end{proposition}

The proof is left to the reader.

\ref{art19-prop1.3} can only be partially extended to $\mathfrak{B}$.

\begin{proposition}\label{prop2.3}
Let $p$ be good for $G$. Then $\mathfrak{B}(G)$ is nonsingular in codimension $1$.
\end{proposition}

This will not be needed, so we only indicate briefly how this can be proved. If $p$ is good, there exist in $\mathfrak{g}$ regular nilpotent elements (by \cite{art19-key14}, 5.9 b, p. 138, this is also a consequence of \cite{art19-key10}, 5.3, p.8). The orbit $\mathfrak{O}$ in $\mathfrak{B}$ of such an element is open and consists of nonsingular points. One then uses the method of \cite{art19-key15} to prove that all irreducible components of $\mathfrak{B}-\mathfrak{O}$ have codimensions at least $2$.

It is likely that $\mathfrak{B}(G)$ is normal if $G$ is simply connected and $p$ is good. However we are not able to prove this. A proof along the lines of that of \ref{art19-prop1.3} would require the analogue of \ref{art19-prop1.3} (ii). In characteristic $0$ this is a result of Kostant (\cite{art19-key9}). For a proof of the corresponding fact in positive characteristics it seems that one needs detailed information about the ring of $G$-invariant polynomial functions on $\mathfrak{g}$.

If the normality of $\mathfrak{B}$ were known. Theorem \ref{art19-thm3.1} could be ameliorated and its proof could be simplified.

\begin{proposition}\label{art19-prop2.4}
Suppose that $p$ is very good for $G$. Then $\tau :\mathfrak{W}\to \mathfrak{B}$ is birational.
\end{proposition}

The proof is similar to that of \ref{art19-prop1.4}. Instead of results on regular unipotent elements, one now uses those on regular nilpotent elements of $\mathfrak{g}$, which are discussed in (\cite{art19-key14}, 4, p. 138).

\begin{proposition}\label{art19-prop2.5}
The fibres of $\tau : \mathfrak{W}\to \mathfrak{B}$ are connected.
\end{proposition}

A\pageoriginale %page 380 
