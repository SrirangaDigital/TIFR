\title{SOME RESULTS ON ALGEBRAIC CYCLES ON ALGEBRAIC MANIFOLDS}
\markright{Some Results on Algebraic Cycles on Algebraic Manifolds}

\author{By~~ Phillip A. Griffiths}

\date{}

\maketitle

\setcounter{pageoriginal}{92}
\setcounter{section}{-1}
\section{Introduction}\label{art08-sec0}\pageoriginale

The basic problem we have in mind is the classification of the algebraic cycles on a algebraic manifold $V$. The first invariant is the homology class $[Z]$ of a cycle $Z$ on $V$; if $Z$ has codimension $q$, then $[Z]\in H_{2n-2q}(V,\bfZ)(n=\dim V)$. By analogy with divisors (c.f. \cite{art08-cite18}), and following Weil \cite{art08-key22}, if $[Z]=0$, then we want to associate to $Z$ a point $\phi_{q}(Z)$ in a complex torus $T_{q}(V)$ naturally associated with $V$. The classification question then becomes two problems :
\begin{itemize}
\item[(a)] Find the image of $\phi_{q}$ (inversion theorem);

\item[(b)] Find the equivalence relation given by $\phi_{q}$ (Abel's theorem).
\end{itemize}

We are unable to make substantial progress on either of these. On the positive side, our results do cover the foundational aspects of the problem and give some new methods for studying subvarieties of general codimension. In particular, the issue is hopefully clarified to the extent that we can make a guess as to what the answers to (a) and (b) should be. This supposed solution is a consequence of the (rational) Hodge conjecture; conversely, if we know (a) and (b) in suitable form, then we can construct algebraic cycles.

We now give an outline of our results and methods.

For the study of $q$-codimensional cycles on $V$, Weil introduced certain complex tori $J_{q}(V)$; as a real torus,
$$
J_{q}(V)=H^{2q-1}(V,\bfR)/H^{2q-1}(V,\bfZ).
$$
These tori are abelian varieties. We use the same real torus, but with a different complex structure (c.d. \S\S1,2); these tori $T_{q}(V)$ vary holomorphically with $V$ (the $J_{q}(V)$ don't) and have the necessary functorial properties. In general, they are not abelian varieties, but have an $r$-convex polarization \cite{art08-key9}. However, the polarizing line bundle is positive on the ``essential part'' of $T_{q}(V)$.\pageoriginale Also, $T_{1}(V)=J_{1}(V)$ (= Picard variety of $V$) and $T_{n}(V)=J_{n}(V)$ (= Albanese variety of $V$).

Let $\Sigma_{q}$ be the cycles of codimension $q$ algebraically equivalent to zero on $V$. There is defined a homomorphism $\phi_{q}:\Sigma_{q}\to T_{q}(V)$ by $\phi_{q}(Z)=\left[\begin{smallmatrix}:\\ \int\limits_{\Gamma}\omega^{\alpha}\\ :\end{smallmatrix}\right]$/(periods), where $\Gamma$ is a $2n-2q+1$ chain with $\partial \Gamma=Z$ and $\omega^{1},\ldots,\omega^{m}\in H^{2n-2q+1}(V,\bfC)$ are a basis for the holomorphic one-forms on $T_{q}(V)$. Using the torus $T_{q}(V)$, this mapping is holomorhic and depends only on the complex structure of $V$ (c.f. \S3); this latter result follows from a somewhat interesting theorem on the cohomology of K\"ahler manifolds given in the Appendix following \S10. In \S3, we also give the infinitesimal calculation of $\phi$; the transposed differential $\phi^{*}$ is essetially the Poincar\'e residue operator (c.f. (3.8)). For hypersurfaces $(q=1)$, the Poincar\'e residue and geometric residue operators coincide, and the (well-known) solutions to (a) and (b) follow easily.

In \S4, we relate the functorial properties of the tori $T_{q}(V)$ to geometric operations on cycles. The expected theorems turn up, but the proofs require some effort. We use the calculus of differential forms with singularities. In particular, the notion of a residue operator associated to an irreducible subvariety $Z\subset V$ appears. Such a residue operator is given by a $C^{\infty}$ form $\psi$ on $V-Z$ such that: (1) $\psi$ is of type $(2q-1,0)+\cdots+(q,q-1)$; (2) $\partial \psi=0$ and $\overline{\partial}\psi$ is a $C^{\infty}$ $(q,q)$ form on $V$ which gives the Poincar\'e dual $\mathscr{D}[Z]\in H^{q,q}(V)$ of $[Z]$; and (3) for $\Gamma$ a $2n-k$ chain on $V$ meeting $Z$ transversely and $\eta$ a smooth $2q-k$ form on $V$, we have the residue formula: $\lim \int\limits_{\epsilon\to 0\Gamma\cdot (\partial T_{\epsilon})}\psi \wedge \eta=\int\limits_{\Gamma\cdot Z}\eta$, where $T_{\epsilon}$ is the $\epsilon$-neighborhood of $Z$ in $V$. The construction of residue operators is done using Hermitian differential geometry; the techniques involved give a different method of approaching the theorem of Bott-Chern \cite{art08-key4}. One use of the residue operators is the explicit construction, on the form level, of the Gysin homomorphism $i_{*}:H^{k}(Z)\to H^{2q+k}(V)$ where we can keep close track of the complex\pageoriginale structure (c.f. the Appendix to \S4, section (e)). This is useful in proving the functorial properties.

In \S5 we give one of our basic constructions. If $[Z]=0$ in $H_{2n-2q}(V,\bfZ)$, and if $\psi$ is a residue operator for $Z$, we may assume that $d\psi=0$. Then $\psi$ is the general codimensional analogue of a logarithmic integral of the third kind (\cite{art08-key17}). The trouble is that $\psi$ has degree $2q-1$ and so cannot directly be integrated on $V$ to give a function. However, $\psi$ can be integrated on the set of algebraic cycles of dimension $q-1$ on $V$. We show then that $Z$ defines a divisor $D(Z)$ on a suitable Chow variety associated to $V$, and that $\psi$ induces an integral of the third kind on this Chow variety. The generalization of Abel's theorem we give is then : $D(Z)$ is linearly equivalent to zero if $\phi_{q}(Z)=0$ in $T_{q}(V)$. As in the classical case, the proof involves a bilinear relation between $\psi$ and the holomorphic differentials on $T_{q}(V)$. Also, as mentioned above, the ``only if'' part of this statement (which is trivial when $q=1$) depends upon the Hodge problem. Our conclusion from this, as regards problem (b) is: The equivalence relation defined by $\phi$ should be linear equivalence on a suitable Chow variety. In particular, we don't see that this equivalence should necessarily be rational equivalence on $V$.

In \S6 we give our main result trying to determine the image of $\phi$. To explain this formula (given by (6.8) in \S6) we let $\{\bfE_{\lambda}\}$ be a holomorphic family of holomorphic vector bundles over $v$. We denote by $Z_{q}(\bfE_{\lambda})$ the $q^{\text{th}}$ Chern class in the rational equivalence ring, so that $\{Z_{q}(\bfE_{\lambda})\}$ gives a family of codimension $q$ cycles on $V$. Our formula gives a method for calculating the infinitesimal variation of $Z_{q}(\bfE_{\lambda})$ in $T_{q}(V)$; it involves the curvature matrix $\Phi$ in $\bfE_{\lambda}$ and the Kodaira-Spencer class giving the variation of $\bfE_{\lambda}$.

The crux of this formula is that it relates the Poincar\'e and geometric residues in higher codimension. The proof involves a somewhat delicate computation using forms with singularities and the curvature in $\bfE_{\lambda}$. In \S8 we give the argument for the highest Chern class of an ample bundle. In \S7 it is shown that we need only check the theorem for ample bundles; however, in general the Chern classes, given by Schubert cycles, will be singular, except of course for\pageoriginale the highest one. So, to prove our formula in general we give in \S9 an argument, which is basically differential-geometric, but which requires that we examine the singularities of $Z_{q}(\bfE_{\lambda})$.

The reason for proving such a formula is that the Chern classes $Z_{q}(\bfE_{\lambda})$ generate the rational equivalence ring on $V$. So, if we could effectively use the main result, we could settle problem (a). For example, for line bundles $(q=1)$, the mapping in question is the identity; this gives once more the structure theorems of the Picard variety. However, we are unable to make effective use of the formula, except in rather trivial cases, so that our result has more of an intrinsic interest and illuminating proof than the applications we would like.

In the last part of \S9 we give an integral-geometric argument, using the transformation properties of the tori $T_{q}(V)$ and the relation of these properties to cycles, of the main formula (6.8).

Finally, in \S10 we attempt to put the problem in perspective. We formulate possible answers to (a) and (b) and show how these would follow if we knew the Hodge problem. The construction of algebraic cycles, assuming the answer to (a) and (b), is based on a generalization of the Poincar\'e normal functions (c.f. \cite{art08-key19}) and will be given later.

To close this introduction, I would like to call attention to the paper of David Lieberman \cite{art08-key20} on the same subject and which contains several of the results given below. Lieberman uses the Weil Jacobians \cite{art08-key22} to study intermediate cycles; however, his results are equally valid for the complex tori we consider. His methods are somewhat different from the ones used below; many of our arguments are computational whereas Lieberman uses functorial properties of the Weil mapping and his proofs have an algebro-geometric flavor.

More specifically, Lieberman proves the functorial properties of the Weil mapping in somewhat more precise form than given below. Thus his results include the functorial properties \eqref{art08-sec4-eq4.2} (the hard one arising from the Gysin map) and \eqref{art08-sec4-eq4.14} (the easy one using restriction of cohomology), as well as (\ref{art08-sec4-rem4.12}) which we only state conjecturally. From\pageoriginale the functorial properties and the fact that the Weil mapping is holomorphic for codimension one, Lieberman concludes the analyticity of this mapping (given by (3.2)) in general. (It is interesting to contrast his conceptual argument with the computational one given in \cite{art08-key9}.) In summary, Lieberman's results include the important general properties of the intermediate Jacobians given in \S1-4 below. Also, the conjectured Abelian variety for which the inversion theorem ((a) above) holds was found by Lieberman using his Poincar\'e divisor, and the proof of (10.4) is due to Lieberman.

The reason for this overlap is because this manuscript was done in Berkeley, independently but at a later time than Lieberman (most of his results are in his M. I. T. thesis). By the time we talked in Princeton, this paper was more or less in the present form and, because of the deadline for these proceedings, could not be rewritten so as to avoid duplication.


\tableofcontents

\section{Complex Tori associated to Algebraic Manifolds}\label{art08-sec1}\pageoriginale

Let $V$ be an $n$-dimensional algebraic manifold and $\bfL\to V$ the {\em positive line bundle} giving the polarization on $V$. The {\em characteristic class} $\omega \in H^{1,1}(V)\cap H^{2}(V,\bfZ)$ may be locally written as $\omega=\dfrac{i}{2}\{\Sigma g_{\alpha\overline{\beta}}dz^{\alpha}\wedge d\overline{z}^{\beta}\}$ where\pageoriginale $\sum\limits_{\alpha,\beta}g_{\alpha\overline{\beta}}dz^{\alpha}d\overline{z}^{\beta}$ gives a K\"ahler metric on $V$.

According to Hodge, the cohomology group $H^{s}(V,\bfC)$ decomposes as a sum:
$$
H^{s}(V,\bfC)=\sum\limits_{p+q=s}H^{p,q}(V),
$$
where $H^{p,q}(V)$ are the cohomology classes represented by differential forms of type $(p,q)$. Under complex conjugation, $\overline{H^{p,q}(V)}=H^{q,p}(V)$. 

Consider now the cohomology group
\begin{equation}
H^{2n-2q+1}(V,\bfC)=\sum\limits^{2n-2q+1}_{r=0}H^{2n-2q+1-r,r}(V)\label{art08-sec1-eq1.1}
\end{equation}
and choose a complex subspace $S\subset H^{2n-2q+1}(V,\bfC)$ such that
\begin{equation}
S\cap \overline{S}=0\text{~~ and ~~} S+\overline{S}=H^{2n-2q+1}(V,\bfC);\label{art08-sec1-eq1.2}
\end{equation}
\begin{equation}
S=\sum\limits^{2n-2q+1}_{r=0}S\cap H^{2n-2q+1-r,r}(V)\label{art08-sec1-eq1.3}
\end{equation}
(i.e. $S$ is compatible with the {\em Hodge decomposition} \eqref{art08-sec1-eq1.1});
\begin{equation}
H^{n-q+1,n-q}(V)\subset S.\label{art08-sec1-eq1.4}
\end{equation}
Under these conditions we shall define a complex torus $T_{q}(S)$ such that the space of holomorphic $1$-forms on $T_{q}(S)$ is just $S$. There are three equivalent definitions of $T_{q}(S)$.

\begin{definition}\label{art08-sec1-defi1}
Choose a basis $\omega^{1},\ldots,\omega^{m}$ for $S$ and define the lattice $\Gamma(S)\subset \bfC^{m}$ of all column vectors $\pi_{\gamma}=\left[\begin{smallmatrix} \int\limits_{\gamma}\omega^{1}\\ \vdots\\ \int\limits_{\gamma}\omega^{m}\end{smallmatrix}\right]$ where $\gamma\in H_{2m-2q+1}(V,\bfZ)$. To see that $\Gamma(S)$ is in fact a lattice, we observe tht rank $(H_{2n-2q+1}(V,\bfZ))=2m$ and so we must show :
\end{definition}

If $\gamma_{1},\ldots,\gamma_{k}\in H_{2n-2q+1}(V,\bfZ)$ are linearly independent over $\bfR$, then $\pi_{\gamma_{1}},\ldots,\pi_{\gamma_{k}}$ are also linearly independent over $\bfR$. But if 
$$
\sum\limits^{k}_{j=1}\alpha_{j}\int\limits_{\gamma_{j}}\omega^{\alpha}=\int\limits_{\sum\limits^{k}_{j=1}\alpha_{j}\gamma_{j}}
$$\pageoriginale
then also
$$
\int\limits_{\sum\limits^{k}_{j=1}\alpha_{j}\gamma_{j}}\overline{\omega}^{\alpha}=0,\quad\text{since}\quad \overline{\alpha}_{j}=\alpha_{j}.
$$
This says that $\sum\limits^{k}_{j=1}\alpha_{j}\gamma_{j}$ is orthogonal to $S+\overline{S}=H^{2n-2q+1}(V,\bfC)$ and so $\sum\limits^{k}_{j=1}\alpha_{j}\gamma_{j}=0$.

If then $T_{q}(S)=\bfC^{m}/\Gamma(S)$, then $T_{q}(S)$ is a complex torus associated to $S\subset H^{2n-2q+1}(V,\bfC)$.

\begin{definition}\label{art08-sec1-defi2}
Let $H_{2n-2q+1}(V,\bfC)=H^{2n-2q+1}(V,\bfC)^{*}$ be the dual space of $H^{2n-2q+1}(V,\bfC)$ so that $0\to S\to H^{2n-2q+1}(V,\bfC)$ dualizes to 
\begin{equation}
0\leftarrow S^{*}\leftarrow H_{2n-2q+1}(V,\bfC).\label{art08-sec1-eq1.5}
\end{equation}
Then $H_{2n-2q+1}(V,\bfZ)\subset H_{2n-2q+1}(V,\bfC)$ projects onto a lattice $\Gamma(S)\subset S^{*}$ and $T_{q}(S)=S^{*}/\Gamma(S)$. (Proof that Definition \ref{art08-sec1-defi1} = \ref{art08-sec1-defi2}: choosing a basis $\omega^{1},\ldots,\omega^{m}$ for $S$ makes $S^{*}\simeq \bfC^{m}$ by $l(\omega^{\alpha})=l_{\alpha}$ where $=\left[\begin{smallmatrix} l_{1}\\ \vdots \\ l_{m}\end{smallmatrix}\right]$. Then $\pi_{\gamma}(\omega^{\alpha})=\int\limits_{\gamma}\omega^{\alpha}=\langle \omega^{\alpha},\gamma\rangle$ so that $\Gamma(S)$ is the same lattice in both cases.)
\end{definition}

\begin{definition}\label{art08-sec1-defi3}
Let $\mathscr{D}:H_{2n-2q+1}(V,\bfC)\to H^{2q-1}(V,\bfC)$ be the {\em Poincar\'e duality isomorphism} and $0\leftarrow S^{*}\leftarrow H^{2q-1}(V,\bfC)$ the sequence corresponding to \eqref{art08-sec1-eq1.5}, $\Gamma(S)\subset S^{*}$ the lattice corresponding to $\Gamma(S)$. Then $T_{q}(S)=S^{*}/\Gamma(S)$.
\end{definition}

Observe that if $H^{r,s}(V)\subset S$, then $H^{n-r,n-s}(V)\subset S^{*}$ and vice-versa. In particular, $S$ is the dual space of $S^{*}$ by:
\begin{equation}
\langle \omega,\phi\rangle =\int\limits_{V}\omega\wedge \phi\qquad (\omega\in S, \ \ \phi\in S^{*}).\label{art08-sec1-eq1.6}
\end{equation}
Thus\pageoriginale $S\cong H^{1,0}(T_{q}(S))$, the space of holomorphic 1-forms on $T_{q}(S)$.

\section{Special Complex Tori}\label{art08-sec2}

The choice of $S\subset H^{2n-2q+1}(V,\bfC)$ depends on the properties we want $T_{q}(S)$ to have; the results on algebraic cycles will be essentially independent of $S$ because of condition \eqref{art08-sec1-eq1.4}.

\setcounter{definition}{0}
\begin{example}\label{art08-sec2-exam1}
We let $S=\sum\limits^{n-q}_{r=0}H^{n-q+r+1,n-q-r}$ and set $T_{q}(S)=T_{q}(V)$. These tori have been studied in \cite{art08-key9}, where it is proved that $T_{q}(V)$ varies holomorphically with $V$.
\end{example}

The trouble with $T_{q}(V)$ is that it is not polarized in the usual sense; however, for our purposes we can do almost as well as follows. Recall \cite{art08-key23} that there is defined on $H^{2q-1}(V,\bfC)$ a quadratic form $Q$ with the following properties:
\setcounter{equation}{0}
\begin{equation}
\left.
\begin{array}{cl}
{\rm(a)} & Q \text{~ is skew-symmetric and integral on ~}\\
         & H^{2q-1}(V,\bfZ)\subset H^{2q-1}(V,\bfC);\\[4pt]
{\rm(b)} & Q(H^{r,s}(V),\overline{H^{r',s'}(V)})=0\text{~ if~ } r\neq r', s\neq s';\\[4pt]
{\rm(c)} & Q(H^{r,s}(V),\overline{H^{r,s}(V)})\text{~ is nonsingular; and}\\[4pt]
{\rm(d)} & iQ(H^{q-1,q}(V), \ \overline{H^{q-1,q}(V)})>0.
\end{array}
\right\}\label{art08-sec2-eq2.1}
\end{equation}
It follows that $Q(S^{*},S^{*})=0$ and that, choosing a basis $\omega^{1},\ldots,\omega^{m}$ for $S$, there is a complex line bundle $\bfL\to T_{q}(S)$ whose characteristic class $\omega(\bfL)\in H^{2}(T_{q}(S),\bfZ)\cap H^{1,1}(T_{q}(S))$ is given by
$$
\omega(\bfL)=\dfrac{i}{2\pi}\left\{\sum\limits^{m}_{\alpha,\beta=1}h_{\alpha\overline{\beta}}\omega^{\alpha}\wedge \overline{\omega}^{\beta}\right\},
$$
where the matrix $H=(h_{\alpha\overline{\beta}})=\{iQ(e_{\alpha},\overline{e}_{\beta})\}^{-1}$ and $\int\limits_{V}\omega^{\alpha}\wedge e_{\beta}=\delta^{\alpha}_{\beta}$. Diagonalizing $H$, we may write
\begin{equation}
\omega(\bfL)=\dfrac{i}{2\pi}\left\{\sum\limits^{m}_{\alpha=1}\epsilon_{\alpha}\omega^{\alpha}\wedge \overline{\omega}^{\alpha}\right\},\label{art08-sec2-eq2.2}
\end{equation}
where $\epsilon_{\alpha}=\pm 1$ and $\epsilon_{\alpha}=+1$ if $\omega^{\alpha}\in H^{n-q+1,n-q}(V)$. Thus we may say that :
\begin{equation}
\text{There is a natural } r\text{-{\em convex polarization} \cite{art08-key10}~~ } \bfL\to T_{q}(V)\label{art08-sec2-eq2.3}
\end{equation}\pageoriginale
($r$ = number of $\alpha$ such that $\epsilon_{\alpha}=-1$) and the characteristic class of $\bfL$ is positive on the translates of $H^{q-1,q}(V)$.

\begin{example}
We let $S=\sum\limits_{r} H^{n-q+2r+1,n-q-2r}$ and set $J_{q}(V)=T_{q}(S)$. This torus is Weil's {\em intermediate Jacobian} \cite{art08-key22} and from \eqref{art08-sec2-eq2.1} we find :
\end{example}

There is a natural $0$-{\em convex polarization} (= {\em positive line bundle})
\begin{equation}
\bfK\to J_{q}(V).\label{art08-sec2-eq2.4}
\end{equation}

Referring to \eqref{art08-sec2-eq2.2}, we let $\phi^{\alpha}=\omega^{\alpha}$ if $\epsilon_{\alpha}=+1$, $\phi^{\alpha}=\overline{\omega}^{\alpha}$ if $\epsilon_{\alpha}=-1$. Then the $\phi^{\alpha}$ give a basis for $H^{1,0}(J_{q}(V))$ and
\begin{equation}
\omega (\bfK)=\dfrac{i}{2\pi}\left\{\sum\limits^{m}_{\alpha=1}\phi^{\alpha}\wedge \overline{\phi}^{\alpha}\right\}.\label{art08-sec2-eq2.5}
\end{equation}

We recall \cite{art08-cite23} that $H^{s}(J_{q}(V),\mathscr{O}(\bfK^{\mu}))=0$ for $\mu>0$, $s>0$ and that $H^{0}(J_{q}(V),\mathscr{O}(\bfK^{\mu}))$ has a basis $\theta_{0},\ldots,\theta_{N}$ given by {\em theta functions of weight $\mu$.}

\medskip
\noindent
{\bf Comparison of $T_{q}(V)$ and $J_{q}(V)$.}~ In \cite{art08-key9} it is proved that there is a {\em real} linear isomorphism $\xi:T_{q}(V)\to J_{q}(V)$ such that
\begin{equation}
\left.
\begin{array}[l]{@{}rl}
{\rm(i)} & \xi^{*}\phi^{\alpha}=\omega^{\alpha}\text{~ if~ } \epsilon_{\alpha}=+1\text{~ and~ } \xi^{*}\phi^{\alpha}=\overline{\omega}^{\alpha}\text{~ if~ } \epsilon_{\alpha}=-1;\\
{\rm(ii)} & \xi^{*}(\bfK)=\bfL;~ \text{and}\\
{\rm(iii)} & \text{if~~} \Omega_{p}=\xi^{*}(\theta_{p})\left\{\prod\limits_{\epsilon_{\alpha}=-1}\right\}, \text{~ then the } \Omega_{p}\text{~ give a basis of}\\
 & H^{r}(T_{q}(V),\mathscr{O}(\bfL^{\mu})),\text{~~ and~~ }H^{s}(T_{q}(V),\mathscr{O}(\bfL^{\mu}))=0\text{~~ for}\\
& \mu>0, s\neq r.
\end{array}
\right\}\label{art08-sec2-eq2.6}
\end{equation}

\noindent
{\bf Some Special Cases.}~ For $q=1$, $T_{1}(V)=J_{1}(V)=H^{0,1}(V)/H^{1}(V,\bfZ)$ is the {\em Picard variety} of $V$ \cite{art08-key22}. For $q=n$, $T_{n}(V)=J_{n}(V)=H^{n-1,n}(V)/H^{2n-1}(V,\bfZ)$ is the {\em Albanese variety} \cite{art08-key3} of $V$. For $q=2$, $T_{2}(V)=H^{1,2}(V)+H^{0,3}(V)/H^{3}(V,\bfZ)$ and $J_{2}(V)=H^{1,2}(V)+H^{3,0}(V)/H^{3}(V,\bfZ)$; this is the simplest case where $T_{q}(V)\neq J_{q}(V)$.

\medskip
\noindent
{\bf Some Isogeny Properties.}~ We let $S_{q}\subset H^{2n-2q+1}(V,\bfC)$ be the subspace corresponding to either $T_{q}(V)$ or $J_{q}(V)$ constructed above, and we let $S^{*}_{q}\subset H^{2q-1}(V,\bfC)$ be the dual space. Then we have 
\[
\left.
\begin{array}{c}
\xymatrix{
H^{2q-1}(V,\bfC)\ar[r] & S^{*}_{q}\ar[r] & 0\\
H^{2q-1}(V,\bfZ)\ar[u] & &}
\end{array}\right\}
\]\pageoriginale
and $T_{q}(V)$ or $J_{q}(V)$ is given as $S^{*}_{q}/\Gamma^{*}_{q}$ where $\Gamma^{*}_{q}$ is the projection of $H^{2q-1}(V,\bfZ)$ on $S^{*}_{q}$.

Suppose now that $\psi\in H^{p,p}(V)\cap H^{2p}(V,\bfZ)$. Then, by cup-product, we have induced~:
\begin{equation}
\vcenter{
\xymatrix@=1.2cm{
S^{*}_{q}\ar[r]^-{\psi} & S^{*}_{p+q}\\
\Gamma^{*}_{q}\ar[r]^-{\psi}\ar[u] & \Gamma^{*}_{p+q}\ar[u]
}}\label{art08-sec2-eq2.7}
\end{equation}
which gives $\psi:T_{q}(V)\to T_{p+q}(V)$ or $\psi:J_{q}(V)\to J_{p+q}(V)$. We want to give this mapping in terms of the coordinates given in the first definition of paragraph 1.

Let $\omega^{1},\ldots,\omega^{m}=\{\omega^{\alpha}\}$ be a basis for $S_{q}\subset H^{2n-2q+1}(V,\bfC)$ and $\phi^{1},\ldots,\phi^{k}=\{\phi^{\rho}\}$ be a basis for $S_{p+q}\subset H^{2n-2p-2q+1}(V,\bfC)$. Then $\psi\wedge \phi^{\rho}=\sum\limits_{\alpha}m_{\rho\alpha}\omega^{\alpha}$ and 
\begin{equation}
\int\limits_{\gamma}\psi \wedge \phi^{\rho}=\int\limits_{\gamma\cdot \mathscr{D}(\psi)}\phi^{\rho},\label{art08-sec2-eq2.8}
\end{equation}
where $\mathscr{D}(\psi)\in H_{2n-2p}(V,\bfZ)$ and $\gamma\in H_{2n-2q+1}(V,\bfZ)$. Now $M=(m_{\rho\alpha})$ is a $k\times m$ matrix giving $\psi:\bfC^{m}\to \bfC^{k}$ by $\psi \left(\begin{smallmatrix} :\\ \lambda^{\alpha} \\ :\end{smallmatrix}\right)=\left(\begin{smallmatrix} :\\ \sum\limits^{m}_{\alpha=1} m_{\rho\alpha}\lambda^{\alpha}\\ :\end{smallmatrix}\right)$
 and
$$
\psi\left(\begin{matrix} 
:\\ \int\limits_{\gamma}\omega^{\alpha}
\end{matrix}\right)
=
\left(\begin{matrix}
:\\
\Sigma m_{\rho\alpha}\int\limits_{\gamma}\omega^{\alpha}\\
:
\end{matrix}\right)
=
\left(\begin{matrix}
:\\
\int\limits_{\gamma}\psi \wedge \phi^{\rho}\\
:
\end{matrix}\right)
=
\left(\begin{matrix}
:\\
\int\limits_{\gamma\cdot \mathscr{D}(\psi)} \phi^{\rho}\\
:
\end{matrix}\right),
$$
so that $\phi(\Gamma_{q})\subset \Gamma_{p+q}$. It follows that, in terms of the coordinates in Definition \ref{art08-sec1-defi1}, $\psi$ is given by the matrix $M$.

Now\pageoriginale suppose that $\psi:H^{2q-1}(V,\bfC)\to H^{2p+2q-1}(V,\bfC)$ is an isomorphism. Then $\psi:S^{*}_{q}\cong S^{*}_{p+q}$ and $\psi(\Gamma^{*}_{q})$ is of finite index in $\Gamma^{*}_{p+q}$. Thus $\psi:T_{q}(V)\to T_{p+q}(V)$ is an {\em isogeny}, as is also $\psi:J_{q}(V)\to J_{p+q}(V)$. Taking $\psi=\omega^{n-2q+1}$, where $\omega$ is the polarizing class, and using \cite{art08-key23}, page 75, we have :
\begin{equation}
\left.
\begin{array}{c}
\omega^{n-2q+1}:T_{q}(V)\to T_{n-q+1}(V),\text{~~ and}\\[3pt]
\omega^{n-2q+1} : J_{q}(V)\to J_{n-q+1}(V)
\end{array}\right\}\label{art08-sec2-eq2.9}
\end{equation}
are both isogenies for $q\leq \left[\dfrac{n+1}{2}\right]$.

Finally, using \cite{art08-key23}, Chapter IV, we have :

For $p\leq n-2q+1$, the mappings
\begin{equation}
\left.
\begin{array}{c}
\omega^{p}:T_{q}(V)\to T_{p+q}(V),\text{~~ and}\\[4pt]
\omega^{p}: J_{q}(V)\to J_{p+q}(V)
\end{array}
\right\}\label{art08-sec2-eq2.10}
\end{equation}
make $T_{q}(V)$ isogenous to a sub-torus of $T_{p+q}(V)$, and similarly for $J_{q}(V)$ and $J_{p+q}(V)$.

\medskip
\noindent
{\bf Some Functionality Properties.}~ Given a holomorphic mapping $f:V'\to V$, there is induced a cohomology mapping $f^{*}:H^{2q-1}(V,\bfC)\to H^{2q-1}(V',\bfC)$ with $f^{*}(S^{*}_{q}(V))\subset S^{*}_{q}(V'),f^{*}(\Gamma^{*}_{q}(V))\subset \Gamma^{*}_{q}(V')$ (using the obvious notation).

This gives
\begin{equation}
\left.
\begin{array}{c}
f^{*}:T_{q}(V)\to T_{q}(V'),\text{~~ and}\\[4pt]
f^{*}:J_{q}(V)\to J_{q}(V').
\end{array}
\right\}\label{art08-sec2-eq2.11}
\end{equation}

On the other hand, if $\dim V=n$ and $\dim V'=n'$, we set $k=n-n'$ and from $f_{*}:H_{2n'-2q+1}(V',\bfC)\to H_{2n-2(k+q)+1}(V,\bfC)$ we find a mapping
\begin{equation}
\left.
\begin{array}{c}
f_{*}:T_{q}(V')\to T_{q+k}(V),\text{~~ and}\\[4pt]
f_{*}:J_{q}(V')\to J_{q+k}(V).
\end{array}
\right\}\label{art08-sec2-eq2.12}
\end{equation}

Suppose now that $f:V'\to V$ is an embedding so that $V'$ is an algebraic submanifold of $V$. Then $V'$ defines a class $[V']\in H_{2n-2k}(V,\bfZ)$\pageoriginale and $\mathscr{D}[V']=\Psi\in H^{2k}(V,\bfZ)\cap H^{k,k}(V)$. We assert that :

In \eqref{art08-sec2-eq2.11} and \eqref{art08-sec2-eq2.12}, the composite mapping
\begin{equation}
\begin{array}{p{8cm}}
$f_{*}f^{*}:T_{q}(V)\to T_{q+k}(V)\text{~~ is just~~ }\Psi:T_{q}(V)\to T_{q+k}(V)\text{~~ as given by \eqref{art08-sec2-eq2.7} (and similarly for~~ $J_{q}(V)$)}$
\end{array}\label{art08-sec2-eq2.13}
\end{equation}

\begin{proof}
We have to show that the composite
\begin{equation}
H^{2q-1}(V,\bfC)\xrightarrow{f^{*}}H^{2q-1}(V',\bfC)\xrightarrow{f^{*}}H^{2q+2k-1}(V,\bfC)\label{art08-sec2-eq2.14}
\end{equation}
is cup product with $\Psi$. In homology \eqref{art08-sec2-eq2.14} dualizes to 
\begin{equation}
H_{2q-1}(V,\bfC)\xleftarrow{f_{*}}H_{2q-1}(V',\bfC)\xleftarrow{f^{*}}H_{2q+2k-1}(V,\bfC)\label{art08-sec2-eq2.15}
\end{equation}
where $f^{*}$ is defined by
\begin{equation}
\vcenter{\xymatrix@=1.2cm{
H_{2q+2k-1}(V,\bfC)\ar[d]^-{\mathscr{D}}\ar[r]^-{f^{*}} & H_{2q-1}(V',\bfC)\\
H^{2n-2q-2k+1}(V,\bfC)\ar[r]^-{f^{*}} & H^{2n-2k-2q+1}(V',\bfC)\ar[u]_{\mathscr{D}^{-1}}
}}\label{art08-sec2-eq2.16}
\end{equation}

If we can show that $f_{*}f^{*}(\gamma)=[V']\cdot \gamma$ for $\gamma \in H_{2q+2k-1}(V,\bfC)$, then $\int\limits_{\gamma}f_{*}f^{*}\phi=\int\limits_{f_{*}f^{*}\gamma}\phi=\int\limits_{[V']\cdot \gamma}\phi=\int\limits_{\gamma}\Psi \wedge \phi(\phi\in H^{2q-1}(V,\bfC))$, and we are done. So we must show that, in \eqref{art08-sec2-eq2.15}, $f^{*}$ is intersection with $V'$, and this a standard result on the {\em Gysin homomorphism} \eqref{art08-sec2-eq2.16} (c.f. (4.11) and the accompanying Remark).
\end{proof}

\section{Algebraic Cycles and Complex Tori}\label{art08-sec3}

Let $V=V_{n}$ be an algebraic manifold, $S\subset H^{2n-2q+1}(V,\bfC)$ a subspace satisfying \eqref{art08-sec1-eq1.2}-\eqref{art08-sec1-eq1.4}, and $T_{q}(S)$ the resulting complex torus. We choose a suitable basis $\omega^{1},\ldots,\omega^{m}$ for $S\cong H^{1,0}(T_{q}(S))$ and let $\Sigma_{q}$ = \{set of algebraic cycles $Z\subset V$ which are of codimension $q$ in $V$ and are homologous to zero\}. Following Weil \cite{art08-key22}, we define
\setcounter{equation}{0}
\begin{equation}
\phi_{q}:\Sigma_{q}\to T_{q}(S)\label{art08-sec3-eq3.1}
\end{equation}
as follows: if $Z\in \Sigma_{q}$, then $Z=\partial C_{2n-2q+1}$ for some $2n-2q+1$ chain $C$, and we set 
\begin{equation}
\phi_{q}(Z)=
\begin{bmatrix}
\vdots\\
\int\limits_{C}\omega^{\alpha}\\
\vdots
\end{bmatrix}.\label{art08-sec3-eq3.2}
\end{equation}\pageoriginale
Since $C$ is determined up to cycles, $\phi_{q}(Z)$ is determined up to vectors $\left[\begin{smallmatrix}\vdots\\ \int\limits_{\gamma}\omega^{\alpha}\\ \vdots\end{smallmatrix}\right](\gamma\in H_{2n-2q+1}(V,\bfZ))$, and so $\phi_{q}$ {\em is defined and depends on the subspace of the closed $C^{\infty}$ forms spanned by} $\omega^{1},\ldots,\omega^{m}$; this restriction will be removed in the Appendix to \S\ref{art08-sec3}.

Now, while it should be the case that $\phi_{q}$ is holomorphic, we shall be content with recalling from \cite{art08-key9} a special result along these lines. Consider on $V$ an analytic family $\{Z_{\lambda}\}_{\lambda\in \Delta}$($\Delta=\text{disc in~}\lambda\text{-plane}$) of $q$-codimensional algebraic subvarieties $Z_{\lambda}\subset V$. Locally on $V$, $\{Z_{\lambda}\}_{\lambda\in \Delta}$ is given by the vanishing of analytic functions $f_{1}(z^{1},\ldots,z^{n};\lambda),\ldots,f_{l}(z^{1},\ldots,z^{n};\lambda)$. We define $\phi:\Delta\to T_{q}(S)$ by $\phi(\lambda)=\phi_{q}(Z_{\lambda}-Z_{0})$. Using  \eqref{art08-sec1-eq1.4}, we have proved in \cite{art08-key9} that
\begin{equation}
\left.
\begin{array}{l}
\phi :\Delta\to T_{q}(S)\text{~~ is holomorphic and}\\[4pt]
\phi_{*}\{\bfT_{\lambda}(\Delta)\}\subset H^{q-1,q}(V).
\end{array}\right\}\label{art08-sec3-eq3.3}
\end{equation}

We may rephrase \eqref{art08-sec3-eq3.3} by saying that $\phi^{*}:S_{q}\to \bfT_{\lambda}(\Delta)^{*}$ is determined by $\phi^{*}|H^{n-q+1,n-q}(V)$ (c.f. \eqref{art08-sec1-eq1.4}).

\medskip
\noindent
{\bf Continuous Systems and The Infinitesimal Calculation of {\boldmath$\phi_{q}$.}}~ Suppose that the $Z_{\lambda}\subset V$ are all nonsingular and $Z=Z_{0}$. We let $\bfN\to Z$ be the {\em normal bundle} of $Z\subset V$, so that we have the exact sheaf sequence
\begin{equation}
0\to \mathscr{O}_{Z}(\bfN^{*})\to \Omega^{1}_{V|Z}\to \Omega^{1}_{Z}\to 0.\label{art08-sec3-eq3.4}
\end{equation}
Since $\dim Z=n-q$, from \eqref{art08-sec3-eq3.4} we have induced the {\em Poincar\'e residue operator}
\begin{equation}
\Omega^{n-q+1}_{V|Z}\to \Omega^{n-q}_{Z}(\bfN^{*})\to 0\label{art08-sec3-eq3.5}
\end{equation}
as follows : Let $\phi\in \Omega^{n-q+1}_{V|Z}$; $\tau_{1},\ldots,\tau_{n-q}$ be tangent vectors to $Z$; $\eta$ a normal vector to $Z$. Lift $\eta$ to a tangent vector $\widehat{\eta}$ on $V$ along $Z$.\pageoriginale Then $\langle \phi, \tau_{1}\wedge\ldots\wedge \tau_{n-q}\otimes \eta\rangle=\langle \phi,\tau_{1}\wedge\ldots\wedge \tau_{n-q}\wedge\widehat{\eta}\rangle$, where $\phi\in \Omega^{n-q+1}_{V|Z}$.

From \eqref{art08-sec3-eq3.5} and $\Omega^{n-q+1}_{V}\to \Omega^{n-q+1}_{V|Z}$, we have
\begin{equation}
H^{n-q}(V,\Omega^{n-q+1}_{V})\xrightarrow{\xi^{*}}H^{n-q}(Z,\Omega^{n-q}_{Z}(\bfN^{*})).\label{art08-sec3-eq3.6}
\end{equation}

On the other hand, in \cite{art08-key16} Kodaira has defined the {\em infinitesimal displacement mapping}
\begin{equation}
\rho : \bfT_{0}(\Delta)\to H^{0}(Z,\mathscr{O}_{Z}(\bfN)).\label{art08-sec3-eq3.7}
\end{equation}
To calculate $\phi^{*}$, we have shown in \cite{art08-key9} that the following diagram commutes:
\begin{equation}
\vcenter{\xymatrix@R=1.2cm{
 & H^{n-q+1,n-q}(V)=H^{n-q}(V,\Omega^{n-q+1}_{V})\ar[dd]^-{\xi^{*}}\ar[dl]_-{\phi^{*}}\\
\bfT_{0}(\Delta)^{*} & \\
& H^{n-q}(Z,\Omega^{n-q}_{Z}(\bfN^{*}))=H^{0}(Z,\mathscr{O}_{Z}(\bfN))^{*}.\ar[ul]_-{\rho^{*}}
}}\label{art08-sec3-eq3.8}
\end{equation}
In other words, infinitesimally $\phi$ is eseentially given by by $\xi^{*}$ in \eqref{art08-sec3-eq3.6}.

\medskip
\noindent
{\bf Some Special Cases.}
\begin{itemize}
\item[(i)] In case $q=n$, $Z$ is a finite set of points $z_{1},\ldots,z_{r}$ ($Z$ is a zero-cycle) and \eqref{art08-sec3-eq3.6} becomes:
\begin{equation}
H^{1,0}(V)\xrightarrow{\xi^{*}}\sum\limits^{r}_{j=1}\bfT_{z_{j}}(V)^{*}\label{art08-sec3-eq3.9}
\end{equation}
where $\xi^{*}(\omega)=\sum\limits^{r}_{j=1}$, $\omega\in H^{1,0}(V)$ being a holomorphic 1-form on $V$. In particular, $\phi^{*}$ is onto if $\xi^{*}$ is injective.

\item[(ii)] In case $q=1$, $Z\subset V$ is a nonsingular hypersurface. Then there is a holomorphic line bundle $\bfE\to V$ and a section $\sigma\in H^{0}(V,\mathscr{O}_{V}(\bfE))$ such that $Z=\{z\in V:\sigma(z)=0\}$. From the exact sheaf sequence $0\to \mathscr{O}_{V}\xrightarrow{\sigma}\mathscr{V}(\bfE)\to \mathscr{O}_{Z}(\bfN)\to 0$, we find
\begin{equation}
H^{0}(Z,\mathscr{O}_{Z}(\bfN))\xrightarrow{\xi}H^{1}(V,\mathscr{O}_{V}),\label{art08-sec3-eq3.10}
\end{equation}
where we claim that $\xi$ in \eqref{art08-sec3-eq3.10} is (up to a constant) the dual of $\xi^{*}$ in \eqref{art08-sec3-eq3.6} (using $H^{0,1}(V)=H^{n,n-1}(V)^{*}$).
\end{itemize}

\begin{proof}
We\pageoriginale may choose a covering $\{U_{\alpha}\}$ of $V$ by polycylinders such that $Z\cap U_{\alpha}$ is given by $\sigma_{\alpha}=0$ where $\sigma_{\alpha}$ is a coordinate function if $U_{\alpha}\cap Z\neq \emptyset$ and $\sigma_{\alpha}\equiv 1$ if $U_{\alpha}\cap Z=\emptyset$. Then $\sigma_{\alpha}/\sigma_{\beta}=f_{\alpha\beta}$ where $\{f_{\alpha\beta}\}\in H^{1}(V,\mathscr{O}^{*}_{V})$ and gives the transition functions of $\bfE\to V$. Let $\theta=\{\theta_{\alpha}\}\in H^{0}(Z,\mathscr{O}_{Z}(\bfN))$ and $\omega\in H^{n,n-1}(V)$. We want to show that, for a suitable constant $c$, we have
\begin{equation}
\int\limits_{V}\xi(\theta)\wedge \omega=c\int\limits_{Z}\langle \theta,\xi^{*}\omega\rangle.\label{art08-sec3-eq3.11}
\end{equation}
If $Z\cap U_{\alpha}\neq \emptyset$, we may write $\omega=\omega_{\alpha}\wedge d\sigma_{\alpha}$ where $\omega_{\alpha}$ is a $C^{\infty}(n-1,n-1)$ form in $U_{\alpha}$ such that $\omega_{\alpha}|Z\cap U_{\alpha}$ is well-defined. In $U_{\alpha}\cap U_{\beta}$, $\omega=\omega_{\alpha}\wedge d\sigma_{\alpha}=\omega_{\alpha}\wedge d(f_{\alpha\beta}\sigma_{\beta})=\omega_{\alpha}\sigma_{\beta}\wedge df_{\alpha\beta}+f_{\alpha\beta}\omega_{\alpha}\wedge d\sigma_{\beta}$ so that $\omega_{\alpha}|Z\cap U_{\alpha}\cap U_{\beta}=f^{-1}_{\alpha\beta}\omega_{\beta}|Z\cap U_{\alpha}\cap U_{\beta}$. This means that $\{\omega_{\alpha}|Z\cap U_{\alpha}\}$ gives an $(n-1,n-1)$ form on $Z$ with values in $\bfN^{*}$, and so $\{\theta_{\alpha}\omega_{\alpha}|Z\cap U_{\alpha}\}$ gives a global $C^{\infty}(n-1,n-1)$ form on $Z$ (since $\theta_{\alpha}=f_{\alpha\beta}\theta_{\beta}$ on $Z\cap U_{\alpha}\cap U_{\beta}$). It is clear that $\langle \theta,\xi^{*}\omega\rangle |Z=\{\theta_{\alpha}\omega_{\alpha}\}$ so that the right hand side of \eqref{art08-sec3-eq3.11} is
\begin{equation}
\int\limits_{Z}\{\theta_{\alpha}\omega_{\alpha}\}.\label{art08-sec3-eq3.12}
\end{equation}

On the other hand, choose a $C^{\infty}$ section $\Theta=\{\Theta_{\alpha}\}$ of $\bfE\to V$ with $\Theta|Z=\theta$. Then $\overline{\partial \Theta}=\sigma\xi(\theta)$ where $\xi(\theta)$ is a $C^{\infty}(0,1)$ form giving a Dolbeault representative of $\xi(\theta)\in H^{1}(V,\mathscr{O}_{V})$ in \eqref{art08-sec3-eq3.10}. Let $T_{\epsilon}$ be an $\epsilon$-tube aroung $Z$ and $\psi=\dfrac{\Theta}{\sigma}$. Then the left hand side of \eqref{art08-sec3-eq3.11} is $\int\limits_{V}\xi(\theta)\wedge \omega=\lim\limits_{\epsilon\to 0}\int\limits_{V-T_{\epsilon}}\xi(\theta)\wedge \omega=-\lim\limits_{\epsilon\to 0}\int\limits_{\partial T_{\epsilon}}\psi \wedge \omega$ (since $d(\psi\wedge \omega)= \overline{\partial}(\psi\wedge \omega)=\xi(\theta)\wedge \omega$). Locally $\psi\wedge \omega=\Theta_{\alpha}\omega_{\alpha}\wedge \dfrac{d\sigma_{\alpha}}{\sigma_{\alpha}}$ so that $\lim\limits_{\epsilon\to 0}\int\limits_{\partial T_{\epsilon}}\{\Theta_{\alpha}\omega_{\alpha}\}\wedge \dfrac{d\sigma_{\alpha}}{\sigma_{\alpha}}=2\pi i \int\limits_{Z}\{\Theta_{\alpha}\omega_{\alpha}|Z\cap U_{\alpha}\}=2\pi i\int\limits_{Z}\{\theta_{\alpha}\omega_{\alpha}\}$, which, by \eqref{art08-sec3-eq3.12}, proves \eqref{art08-sec3-eq3.11}.
\end{proof}

\medskip
\noindent
{\bf Appendix to \S\ref{art08-sec3}~: Some Remarks on the Definition of {\boldmath$\phi_{q}$}.}~ At the beginning of Paragraph 3 where $\phi_{q}:\Sigma_{q}\to T_{q}(S)$ was defined, it was stated that $\phi_{q}$ depended on the vector space spanned by $\omega^{1},\ldots,\omega^{m}$ and not just on $S$. This is because, if we replace $\omega^{\alpha}$ by $\omega^{\alpha}+d\eta^{\alpha}$, then $\int\limits_{C_{2n-2q+1}}\omega^{\alpha}+d\eta^{\alpha}=\int\limits_{C}\omega^{\alpha}+\int\limits_{Z}\eta^{\alpha}$ (Stokes' Theorem).

One\pageoriginale way around this is to use the K\"ahler metric on $V$ and choose $\omega^{1},\ldots,\omega^{m}$ to be {\em harmonic}. This has the disadvantage that harmonic forms are {\em not} generally preserved under holomorphic mappings. However, if we agree to use the torus $T_{q}(V)(S=\sum\limits_{r}H^{n-q+1+r,n-q-r}(V))$ constructed in Example \ref{art08-sec2-exam1} of \S\ref{art08-sec2}, it is possible to given $\phi_{q}:\Sigma_{q}\to T_{q}(V)$ {\em purely in terms of cohomology}, and so remove this problem in defining $\phi_{q}$.

To do this, we shall use a theorem on the cohomology of algebraic manifolds which is given in the Appendix below. Let then $\Omega^{q}$ be the sheaf of holomorphic $q$-forms on $V$ and $\Omega^{q}_{c}\subset \Omega^{q}$ the subsheaf of closed forms. There is an exact sequence:
\begin{equation*}
0\to \Omega^{q}_{c}\to \Omega^{q}\xrightarrow{d}\Omega^{q+1}_{c}\to 0\tag{A3.1}\label{art08-sec3-eqA3.1}
\end{equation*}
({\em Poincar\'e lemma}), which gives in cohomology (c.f. (A.7)):
\begin{equation*}
0\to H^{p-1}(V,\Omega^{q+1}_{c})\xrightarrow{\delta}H^{p}(V,\Omega^{q}_{c})\to H^{p}(V,\Omega^{q})\to 0.\tag{A3.2}\label{art08-sec3-eqA3.2}
\end{equation*}

From \eqref{art08-sec3-eqA3.2}, we see that there is a diagram (c.f.(A.16) in the Appendix):
\begin{equation*}
\vcenter{
\xymatrix@C=-.02cm{
H^{r}(V,\bfC) & = & H^{r}(V,\bfC)\\
H^{r-1}(V,\Omega^{1}_{c})\ar[u]_-{\delta} & \subset & H^{r}(V,\bfC)\ar@{=}[u]\\
\vdots\ar[u] & & \vdots\ar@{=}[u]\\
H^{r-q}(V,\Omega^{q}_{c})\ar[u] & \subset & H^{r}(V,\bfC)\ar@{=}[u]\\
\vdots\ar[u]_-{\delta} & & \vdots\ar@{=}[u]\\
H^{0}(V,\Omega^{r}_{c})\ar[u] & \subset & H^{r}(V,\bfC)\ar@{=}[u]\\
0\ar[u] & &
}}\tag{A3.3}\label{art08-sec3-eqA3.3}
\end{equation*}

Thus\pageoriginale $\{H^{r-q}(V,\Omega^{q}_{c})\}$ gives a {\em filtration} $\{F^{r}_{q}(V)\}$ of $H^{r}(V,\bfC)$; and\break $F^{r}_{q}(V)/F^{r}_{q+1}(V)\cong H^{r-q}(V,\Omega^{q})$. It is also true that $F^{r}_{q}(V)$ depends holomorphically on $V$ \cite{art08-key9}.

To calculate $F^{r}_{q}(V)$ using differential forms, we let $A^{s,r-s}$ be the $C^{\infty}$ forms of type $(s,r-s)$ on $V$, $B^{r,q}=\sum\limits_{s\geq q}A^{s,r-s}$, and $B^{r,q}_{c}$ the $d$-closed forms in $B^{r,q}$. Then $dB^{r,q}\subset B^{r+1,q}_{c}$, and it is shown in the Appendix (c.f. (A.18)) that
\begin{equation*}
F^{r}_{q}(V)\cong B^{r,q}_{c}/dB^{r-1,q}\subset H^{r}(V,\bfC).\tag{A3.4}\label{art08-sec3-eqA3.4}
\end{equation*}

We conclude then from \eqref{art08-sec3-eqA3.4} that:
\begin{equation*}
\left.
\begin{array}{@{}l}
\text{A class~ } \phi\in F^{r}_{q}(V)\subset H^{r}(V,\bfC)\text{~ is represented by a closed } C^{\infty}\\[4pt]
\text{form~ } \phi=\sum\limits_{s\geq q}\phi_{s,r-s}(\phi_{s,r-s}\in A^{s,r-s},\text{~ defined up to forms}\\[4pt]
d\eta=\sum\limits_{s\geq q}d\eta_{s,r-1-s}. 
\end{array}
\right\}\tag{A3.5}\label{art08-sec3-eqA3.5}
\end{equation*}

In particular, look at $F{2n-2q+1}_{n-q+1}(V)\cong \sum\limits_{r\geq 0}H^{n-q+1+r,n-q-r}(V)$. $A\phi \in B^{2n-2q+1,n-q+1}_{c}$ is defined up to 
$$
\sum\limits_{s\geq 0}d\eta_{n-q+1+s,n-q-1-s}
$$ 
and 
$$
\int\limits_{Z}\eta_{n-q+1+s,n-q-1-s}=0
$$ 
for an algebraic cycle $Z$ of codimension $q$ ($Z$ is of type $(n-q,n-q)$). Thus $\int\limits_{C_{2n-2q+1}}\phi(\partial C=Z)$ depends {\em only} on the class of $\phi\in F^{2n-2q+1}_{n-q+1}(V)\subset H^{2n-2q+1}(V,\bfC)$. This proves that:
\begin{equation*}
\left.
\begin{array}{l}
\text{For the torus~ } T_{q}(V)\text{~ constructed in \S\ref{art08-sec3}, the mapping}\\[3pt]
\phi_{q}:\Sigma_{q}\to T_{q}(V)\text{~ depends only on the complex structure of } V.
\end{array}\right\}\tag{A3.6}\label{art08-sec3-eqA3.6}
\end{equation*}

For the general tori $T_{q}(S)$ we may prove the analogue of \eqref{art08-sec3-eqA3.6} as follows. First, we may make the forms $\omega^{1},\ldots,\omega^{m}$ subject to $\partial \omega^{\alpha}=0$, $\overline{\partial}\omega^{\alpha}=0$, because $S=\sum\limits_{r}S\cap H^{2n-2q+1-r,r}(V)$ and so $\omega^{\alpha}=\mathscr{H}(\omega^{\alpha})+d\xi^{\alpha}$ ($\mathscr{H}$ = harmonic part of $\omega^{\alpha}$) and $\mathscr{H}(\omega^{\alpha})=\sum\limits_{r}\mathscr{H}(\omega^{\alpha}_{n-q+1+r,n-q-r})$ with $\partial (\omega^{\alpha}_{n-q+1+r,n-q-r})=0=\overline{\partial}\mathscr{H}(\omega^{\alpha}_{n-q+1+r,n-q-r})$. Thus we may choose a basis $\omega^{1},\ldots,\omega^{m}$ for $S$ with $\partial \omega^{\alpha}=0=\overline{\partial}\omega^{\alpha}$.

Second,\pageoriginale let $\eta$ be a $C^{\infty}$ form on $V$ with $\partial d\eta=0=\overline{\partial}d\eta$. We claim that $d\eta=\partial\overline{\partial}\xi$ for some $\xi$. Since $d\eta=\partial\eta+\overline{\partial}\eta$, it will suffice to do this for $\partial\eta$. Now write $\eta=\mathscr{H}_{\partial}\eta+\partial\partial^{*}G_{\partial}\eta+\partial^{*}\partial G_{\partial}\eta$, where $\mathscr{H}_{\partial}$ is the {\em harmonic projector} for $\boxvoid_{\partial}=\partial\partial^{*}+\partial^{*}\partial$ and $G_{\partial}$ is the corresponding {\em Green's operator}. Then $\partial_{\eta}=\partial\partial^{*}\partial G_{\partial}\eta$. On the other hand, since $\overline{\partial}\partial=0$, $\partial\eta=\mathscr{H}_{\partial}(\partial\eta)+\overline{\partial}\partial^{*}G_{\partial}\partial\eta$. But $\mathscr{H}_{\partial}=\mathscr{H}_{\partial}$ and $G_{\partial}=G_{\partial}$ so that $\partial\eta=\partial\overline{\partial}(\overline{\partial}^{*}G_{\partial}\eta)$ as desired.

Finally, let $\omega\in S$ satisfy $\partial \omega=0=\overline{\partial}\omega$ and change $\omega$ to $\omega+d\eta$ with $\partial(\omega+d\eta)=0=\overline{\partial}(\omega+d\eta)$. Then $\omega+d\eta=\omega+\partial\overline{\partial}\xi$ for some $\xi$. We claim that $\int\limits_{C}\omega=\int\limits_{C}\omega+\partial\overline{\partial}\xi$, where $C$ is a $2n-2q+1$ chain with $\partial C=Z$. If $\xi=\sum\limits_{r}\xi_{n-q+r,n-q-1-r}$, then
$$
\int\limits_{C}\partial\overline{\partial}\xi=\int\limits_{C}d(\overline{\partial}\xi)=\int\limits_{Z}(\overline{\partial}\xi)_{n-q,n-q}=\int\limits_{Z}\overline{\partial}\xi_{n-q,n-q-1}=0
$$
since $Z\subset V$ is a complex submanifold. This proves that:
\begin{equation*}
\left.
\begin{array}{@{}l@{}}
\text{If, in the definition of  $\phi_{q}:\Sigma_{q}\to T_{q}(V)$  in \eqref{art08-sec3-eq3.1}, we make}\\
\text{the $\omega^{\alpha}$ subject to $\partial\omega^{\alpha}=0=\overline{\partial}\omega^{\alpha}$, then $\phi_{q}$ is well-defined}\\
\text{and depends only on the complex structure of $V$.}
\end{array}
\right\}\tag{A.3.7}\label{art08-sec3-eqA3.7}
\end{equation*}

This is the procedure followed by Weil \cite{art08-key22}.

\medskip
\noindent
{\bf Remark. A(3.8).}~ Let $D=\partial\overline{\partial}$; then $D:A^{p,q}_{c}\to A^{p+1,q+1}_{c}$ and $D^{2}=0$. If $H^{r}_{D}(V)$ are the cohomology groups constructed from $D=\partial\overline{\partial}$ and $H^{r}_{d}(V)$ the deRham groups, there is a natural mapping:
\begin{equation*}
H^{r}_{D}(V)\xrightarrow{\alpha}H^{r}_{d}(V).\tag{A3.9}\label{art08-sec3-eqA3.9}
\end{equation*}

\section{Some Functorial Properties}\label{art08-sec4}
(a)~ Let $W_{n-k}\subset V_{n}$ be an algebraic submanifold of codimension $k$. We shall assume for the moment that there is a holomorphic vector bundle $\bfE\to V$, with fibre $\bfC^{s}$, and holomorphic sections $\sigma_{1},\ldots,\sigma_{s-k+1}$ of $\bfE$ such that $W$ is given by $\sigma_{1}\wedge\ldots\wedge \sigma_{s-k+1}=0$. Thus, the homology class carried by $W$ is the $k$-{\em th Chern class} of $\bfE\to V$ (c.f. \cite{art08-key5}). Following \eqref{art08-sec2-eq2.11} there is a mapping
\setcounter{equation}{0}
\begin{equation}
T_{q}(V)\xrightarrow{i^{*}}T_{q}(W)\label{art08-sec4-eq4.1}
\end{equation}
induced\pageoriginale from $H^{2q-1}(V,\bfC)\to H^{2q-1}(W,\bfC)$. We want to interpret this mapping geometrically.

For this, let $\{Z_{\lambda}\}_{\lambda\in \Delta}$ be a {\em continuous system} as in paragraph 3. Assume that each intersection $Y_{\lambda}=Z_{\lambda}\cdot W$ is transverse so that $\{Y_{\lambda}\}_{\lambda\in D}$ gives a continuous system of $W$. Letting $\phi_{q}(V)(\lambda)=\phi_{q}(Z_{\lambda}-Z_{0})\in T_{q}(V)$ and $\phi_{q}(W)(\lambda)=\phi_{q}(Y_{\lambda}-Y_{0})\in T_{q}(W)$, we would like to show that the following diagram commutes:
\begin{equation}
\vcenter{\xymatrix@C=1.5cm{
 & T_{q}(V)\ar[dd]^{i^{*}}\\
\Delta\ar[ur]^{\phi_{q}(V)}\ar[dr]^{\phi_{q}(W)} & \\
 & T_{q}(W)
}}\label{art08-sec4-eq4.2}
\end{equation}
This would interpret \eqref{art08-sec4-eq4.1}

\begin{proof}
Let $S_{V}=\sum\limits_{r\geq 0}H^{n-q+1+r,n-q-r}(V)\subset H^{2n-2q+1}(V,\bfC)$ be the space of holomorphic 1-forms on $T_{q}(V)$ and $\omega^{1},\ldots,\omega^{m}$ a basis for $S_{V}$. If $C_{\lambda}$ is a $2n-2q+1$ chain on $V$ with $\partial C_{\lambda}=Z_{\lambda}-Z_{0}$ then $\phi_{q}(V)(\lambda)=\left[\begin{smallmatrix} :\\ \int\limits_{C_{\lambda}}\omega^{\alpha}\\ :\end{smallmatrix}\right]$. Similarly, let $S_{W}\subset H^{2n-2q-2k+1}(W,\bfC)$ be the holomorphic 1-forms on $T_{q}(W)$ and $\phi^{1},\ldots,\phi^{r}$ a basis for $S_{W}$. Letting $D_{\lambda}=C_{\lambda}\cdot W$, $\partial D_{\lambda}=Y_{\lambda}-Y_{0}$ and $\phi_{q}(W)(\lambda)=\left[\begin{smallmatrix} :\\ \int\limits_{D_{\lambda}}\phi^{rho}\\ :\end{smallmatrix}\right]$.

Actually, in line with the Appendix to \S\ref{art08-sec3}, we should use the isomorphisms $F^{2n-2q+1}_{n-q+1}(V)\cong S_{V}$, $F^{2n-2k-2q+1}_{n-k-q+1}(W)\cong S_{W}$ (c.f. (A3.5)), and choose $\omega^{1},\ldots,\omega^{m}$ and $\phi^{1},\ldots,\phi^{r}$ as bases of $F^{2n-2q+1}_{n-q+1}(V)$ and $F^{2n-2k-2q+1}_{n-k-q+1}(W)$ respectively. We assume this is done.

We now need to give $i^{*}:\bfC^{m}\to \bfC^{r}$ explicitly using the above bases. Let $e_{1},\ldots,e_{m}\in S^{*}_{V}\subset H^{2q-1}(V,\bfC)$ be dual to $\omega^{1},\ldots,\omega^{m}$ and $f_{1},\ldots,f_{r}\in S^{*}_{W}\subset H^{2q-1}(W,\bfC)$ dual to $\phi^{1},\ldots,\phi^{r}$. Then $\int\limits_{V}\omega^{\alpha}\wedge e_{\beta}=\delta^{\alpha}_{\beta}$, $\int\limits_{W}\phi^{\rho}\wedge f_{\sigma}=\delta^{\rho}_{0}$.\pageoriginale Now $i^{*}(e_{\alpha})=\sum\limits^{r}_{\rho=1}m_{\rho\alpha}f_{\rho}$ for some $r\times m$ matrix $M$, and $i^{*}:T_{q}(V)\to T_{q}(W)$ is given by $M:\bfC^{m}\to \bfC^{r}$ (c.f. just below \eqref{art08-sec2-eq2.7}).

To calculate $M\left[\begin{smallmatrix} : \\ \int\limits_{\gamma}\omega^{\alpha}\\ :\end{smallmatrix}\right]$, we let $i_{*}:H^{2n-2k-2q+1}(W,\bfC)\to H^{2n-2q+1}(V,\bfC)$ be the {\em Gysin homomorphism} defined by:
\begin{equation}
\vcenter{\xymatrix{
H^{2n-2k-2q+1}(W,\bfC)\ar[r]^-{i_{*}} & H^{2n-2q+1}(V,\bfC)\\
H_{2q-1}(W,\bfC)\ar[u]_{\mathscr{D}_{W}}\ar[r]^-{i_{*}} & H_{2q-1}(V,\bfC).\ar[u]_{\mathscr{D}_{V}}
}}\label{art08-sec4-eq4.3}
\end{equation}

Then, $i_{*}(\phi^{p})=\sum\limits^{m}_{\alpha=1}m_{p\alpha}\omega^{\alpha}$, and $M\left[\begin{smallmatrix} :\\ \int\limits_{\gamma}\omega^{\alpha}\\ :\end{smallmatrix}\right]=\left[\sum\limits^{m}_{\alpha=1}m_{\rho\alpha}\int\limits_{\gamma}\omega^{\alpha}\right]=\left[\begin{smallmatrix}:\\ \int\limits_{\gamma}i_{*}(\phi^{\rho})\\ :\end{smallmatrix}\right]=\left[\begin{smallmatrix} :\\ \int\limits_{W\cdot \gamma}\phi^{\rho}\\ :\end{smallmatrix}\right]$ (c.f. \eqref{art08-sec2-eq2.16}) where $\gamma\in H_{2n-2q+1}(V,\bfZ)$ is a cycle on $V$. This gives the equation
\begin{equation}
M\begin{bmatrix}
:\\
\int\limits_{\gamma}\omega^{\alpha}\\
:
\end{bmatrix}
=
\begin{bmatrix}
:\\
\int\limits_{\gamma}i_{*}\phi^{\rho}\\
:
\end{bmatrix}
=
\begin{bmatrix}
:\\
\int\limits_{W\cdot \gamma} \phi^{\rho}\\
:
\end{bmatrix},\label{art08-sec4-eq4.4}
\end{equation}
for $\gamma\in H_{2n-2q+1}(V,\bfZ)$. To prove \eqref{art08-sec4-eq4.2}, we must prove \eqref{art08-sec4-eq4.4} for the chain $C_{\lambda}$ with $\partial C_{\lambda}=Z_{\lambda}-Z_{0}$; this is because, in \eqref{art08-sec4-eq4.2},
$$
i^{*}\phi_{q}(V)(\lambda)=M
\begin{bmatrix}
:\\
\int\limits_{C_{\lambda}}\omega^{\alpha}\\
:
\end{bmatrix}
\text{~~ and~~} \phi_{q}(W)(\lambda)=
\begin{bmatrix}
:\\
\int\limits_{W\cdot C_{\lambda}}\phi^{\rho}\\
:
\end{bmatrix}.
$$
Thus, to prove the formula \eqref{art08-sec4-eq4.2}, we must show :

The {\em Gysin homomorphism}
\begin{equation}
i_{*}:H^{2n-2k-2q+1}(W,\bfC)\to H^{2n-2q+1}(V,\bfC)\label{art08-sec4-eq4.5}
\end{equation}
(given\pageoriginale in \eqref{art08-sec4-eq4.3}) has the properties :
\begin{equation}
i_{*}:F^{2n-2q-2k+1}_{n-q-k+1}(W)\to F^{2n-2q+1}_{n-q+1}(V);\label{art08-sec4-eq4.6}
\end{equation}
and
\begin{equation}
\int\limits_{C}i_{*}\phi=\int\limits_{W\cdot C}\phi,\label{art08-sec4-eq4.7}
\end{equation}
where $C$ is a $2n-2q+1$ chain on $V$ with $\partial C=Z$, $Z$ being an algebraic cycle on $V$ meeting $W$ transversely.

This is where we use the bundle $\bfE\to V$. Namely, it will be proved in the Appendix to \S\ref{art08-sec4} below that there is a $C^{\infty}(k,k-1)$ form $\psi$ defined on $V-W$ having the properties :

$\partial \psi=0$ and $\overline{\partial}\psi=\Psi$ is a $C^{\infty}$ form on $V$ which represents the
\begin{equation}
\text{Poincar\'e dual~ } \mathscr{D}(W)\in H^{k,k}(V,\bfC)\cap H^{2k}(V,\bfZ);\label{art08-sec4-eq4.8}
\end{equation}
to give $i_{*}$ in \eqref{art08-sec4-eq4.6}, we let $\phi\in B^{2n-2k-2q+1,n-k-q+1}_{c}(W)$ represent a class in $F^{2n-2k-2q+1}_{n-k-q+1}(W)$ and choose $\widehat{\phi}\in B^{2n-2k-2q+1,n-k-q+1}(V)$ with $\widehat{\phi}|W=\phi$. Then $d(\psi\wedge \phi)$ is a {\em current} on $V$ and
\begin{align}
& i_{*}(\phi)=d(\psi\wedge\widehat{\phi});\quad\text{and}\label{art08-sec4-eq4.9}\\
& \lim\limits_{\epsilon \to 0}\int\limits_{c\cdot \partial T_{\epsilon}}\psi\wedge\eta =\int\limits_{C\cdot W}\eta,\label{art08-sec4-eq4.10}
\end{align}
where $T_{\epsilon}$ is the $\epsilon$-tube around $W$ and $\eta\in B^{2n-2k-2q+1,n-k-q+1}(V)$.
\end{proof}

\begin{remark*}
The composite
\begin{equation}
F^{2n-2k-2q+1}_{n-k-q+1}(V)\xrightarrow{i^{*}}F^{2n-2k-2q+1}_{n-k-q+1}(W)\xrightarrow{i_{*}}F^{2n-2q+1}_{n-q+1}(V)\label{art08-sec4-eq4.11}
\end{equation}
is given by $i_{*}i^{*}\eta=d(\psi\wedge \eta)=\Psi\wedge \eta(\eta\in B^{2n-2k-2q+1,n-k-q+1}_{c}(V))$; this should be compared with \eqref{art08-sec2-eq2.16} above.
\end{remark*}

\medskip
\noindent
{\bf Proof of \eqref{art08-sec4-eq4.6} and \eqref{art08-sec4-eq4.7} from \eqref{art08-sec4-eq4.8}--\eqref{art08-sec4-eq4.10}.}~ Since $i_{*}(\phi)=d(\psi\wedge \widehat{\phi})$ and $\psi\wedge\widehat{\phi}\in B^{2n-2q,n-q+1}$, $i_{*}(\phi)\in B^{2n-2q+1,n-q+1}_{c}(V)$ which proves \eqref{art08-sec4-eq4.6} (c.f. (e) in the Appendix to Paragraph 4).

To prove \eqref{art08-sec4-eq4.7}, we will have
$$
\int\limits_{C}i_{*}phi=\lim\limits_{e\to 0}\int\limits_{C-C\cdot T_{\epsilon}}i_{*}(\phi)=(\text{by Stokes' theorem})\int\limits_{\partial(C-C\cdot T_{\epsilon})}\psi\wedge \widehat{\phi}.
$$
But $\partial(C-C\cdot T_{\epsilon})=(Z-Z\cdot T_{\epsilon})-C\cdot \partial T_{\epsilon}$\pageoriginale and so $\int\limits_{C}i_{*}\phi=-\lim\limits_{\epsilon\to 0}\int\limits_{C\cdot \partial T_{\epsilon}}\psi \wedge \widehat{\phi}$ (since $\int\limits_{Z-Z\cdot T_{\epsilon}}\psi \wedge \widehat{\phi}=0$) $=\int\limits_{C\cdot W}\phi$ by \eqref{art08-sec4-eq4.10}.

\setcounter{theorem}{11}
\begin{remark}\label{art08-sec4-rem4.12}
Actually \eqref{art08-sec4-eq4.2} will hold in the following generality. Let $V$, $V'$ be algebraic manifolds and $f:V'\to V$ a holomorphic mapping. Let $\Sigma_{q}(V)$ be the algebraic cycles $Z\subset V$ of codimension $q$ which are homologous to zero and similarly for $\Sigma_{q}(V')$. Then there is a commutative diagram :
\[
\xymatrix@=1.2cm{
\Sigma_{q}(V)/\text{S.E.R.}\ar[d]^-{f^{*}}\ar[r]^-{\phi_{q}(V)} & T_{q}(V)\ar[d]^-{F^{*}}\\
\Sigma_{q}(V')/\text{S.E.R.}\ar[r]^-{\phi_{q}(V')} & T_{q}(V')
}
\]
where S.E.R. = suitable equivalence relation (including rational equivalence), and where $f^{*}(Z)=f^{-1}(Z)=\{z'\in V':f(z)\in V\}$ in case $Z$ is transverse to $f(V')$.

(b) Keeping the notation and assumptions of (4a) above, following \eqref{art08-sec2-eq2.12} we have:
\begin{equation}
i_{*}:T_{q}(W)\to T_{q+k}(V),\label{art08-sec4-eq4.13}
\end{equation}
and we want also these maps geometrically. For this, let $\{Y_{\lambda}\}_{\lambda\in \Delta}$ be a continuous system of subvarieties $Y_{\lambda}\subset W$ of codimension $q$. Then $Y_{\lambda}\subset V$ has codimension $k+q$ and so we may set
$$
\phi_{q+k}(V)(\lambda)=\phi_{q}(Y_{\lambda}-Y_{0})\in T_{q+k}(V),\phi_{q}(W)(\lambda)=\phi_{q}(Y_{\lambda}-Y_{0})\in T_{q}(W).
$$
We assert that the following diagram commutes:
\setcounter{equation}{13}
\begin{equation}
\vcenter{\xymatrix@C=1.5cm{
 & T_{q}(W)\ar[dd]^-{i_{*}}\\
\Delta\ar[ur]^-{\phi_{q}(W)}\ar[dr]_-{\phi_{q+k}(V)} &\\
 & T_{q+k}(V)
}}\label{art08-sec4-eq4.14}
\end{equation}
This\pageoriginale interprets $i_{*}$ in \eqref{art08-sec4-eq4.13} as ``{\em inclusion of cycles lying on $W$ into $V$}''.
\end{remark}

\begin{proof}
As in the proof of \eqref{art08-sec4-eq4.2}, we choose bases $\omega^{1},\ldots,\omega^{m}$ for $S_{V}\subset H^{2n-2k-2q+1}(V,\bfC)$ and $\phi^{1},\ldots,\phi^{r}$ for $S_{W}\subset H^{2n-2k-2q+1}(W,\bfC)$. Then
$$
\phi_{q}(W)(\lambda)=
\begin{bmatrix}
:\\
\int\limits_{C_{\lambda}}\phi^{\rho}\\
:
\end{bmatrix}
\text{~~ and~~ } \phi_{q+k}(V)(\lambda)=
\begin{bmatrix}
:\\
\int\limits_{C_{\lambda}}\omega^{\alpha}\\
:
\end{bmatrix}
\text{~~ where~~ } \partial C_{\lambda}=Y_{\lambda}-Y_{0}.
$$

We now need $i_{*}$ explicitly. Let $e_{1},\ldots,e_{m}$ be a dual basis in $S^{*}_{V}\subset H^{2q+2k-1}(V,\bfC)$ to $\omega^{1},\ldots,\omega^{m}$ and $f_{1},\ldots,f_{r}$ in $S^{*}_{W}\subset H^{2q-1}(W,\bfC)$ be a dual basis to $\phi^{1},\ldots,\phi^{r}$. Then $i_{*}$ in \eqref{art08-sec4-eq4.14} is induced by the Gysin homomorphism \eqref{art08-sec4-eq4.6} $i_{*}:H^{2q-1}(W,\bfC)\to H^{2q+2k-1}(V,\bfC)$. Write $i_{*}(f_{\rho})=\sum\limits_{\alpha=1}m_{\alpha\rho}e_{\alpha}$ so that $M=(m_{\alpha\rho})$ is an $m\times ?$ matrix $M:\bfC^{r}\to \bfC^{m}$ which gives $i_{*}T_{q}(W)\to T_{q+k}(V)$.

Now $M\phi_{q}(W)(\lambda)=\left[\begin{smallmatrix}:\\ \sum\limits^{r}_{\rho=1}m_{\alpha\rho}\int\limits_{C}\phi^{\rho}\\ :\end{smallmatrix}\right]$ so that, to prove \eqref{art08-sec4-eq4.14}, we 
\begin{equation}
\int\limits_{\gamma}\omega^{\alpha}=\sum\limits^{r}_{\rho=1}m_{\alpha\rho}\int\limits_{\gamma}\phi^{\rho}\label{art08-sec4-eq4.15}
\end{equation}
for $\gamma$ a suitable $2n-2k-2q+1$ chain on $W$. Since
$$
i^{*}:H^{2n-2k-2q+1}(V,\bfC)\to H^{2n-2k-2q+1}(W,\bfC)
$$
satisfies $\int\limits_{W}i^{*}(\omega)\wedge \phi=\int\limits_{V}\omega \wedge i_{*}\phi$, we have $i^{*}\omega^{\alpha}=\sum\limits^{r}_{\rho=1}m_{\alpha\rho}\phi^{\rho}$ in $H^{2n-2k-2q+1}(W,\bfC)$. On the other hand, since $i^{*}$ satisfies $i^{*}\{F^{r}_{q}(V)\}\subset F^{r}_{q}(W)$, we have, as forms 
$$
i^{*}\omega^{\alpha}=\sum\limits^{r}_{\rho=1}m_{\alpha\rho}\phi^{\rho}+d\mu^{\alpha}(\mu^{\alpha}\in B^{2n-2k-2q+1, n-k-q+1}(W))
$$ 
so that $\int\limits_{C_{\lambda}}\omega^{\alpha}=\sum\limits^{r}_{\rho=1}m_{\alpha\rho}\int\limits_{C_{\lambda}}\phi^{\rho}$ as needed.
\end{proof}

\begin{remark*}
To\pageoriginale prove \eqref{art08-sec4-eq4.14} for $J_{q}(W)$ and $J_{q+k}(V)$, we use that $i^{*}\partial=\partial i^{*}$ and $i^{*}\overline{\partial}=\overline{\partial}i^{*}$ on the form level, so that $i^{*}\omega^{\alpha}=\sum\limits^{r}_{\rho=1}m_{\alpha\rho}\phi^{\rho}+\partial \overline{\partial}\xi^{\alpha}$ and then, as before,
$$
\int\limits_{C_{\lambda}}\omega^{\alpha}=\sum\limits^{r}_{\rho=1}m_{\alpha\rho}\int\limits_{C_{\lambda}}\phi^{\rho}.
$$
\end{remark*}

(c)~ We now combine (a) and (b) above. Thus let $W\subset V$ be a submanifold of codimension $k$ and $\{Z_{\lambda}\}_{\lambda\in \Delta}$ be a continuous system of codimension $q$ on $V$ such that $Z_{\lambda}\cdot W=Y_{\lambda}$ is a proper intersection. Then $\{Z_{\lambda}\}_{\lambda\in\Delta}$ defines $\phi_{q}:\Delta\to T_{q}(V)$ and $\{Y_{\lambda}\}_{\lambda\in \Delta}$ defines $\phi_{q+k}:\Delta\to T_{q+k}(V)$. Combining \eqref{art08-sec4-eq4.2} and \eqref{art08-sec4-eq4.14}, we find that the following is a commutative diagram:
\begin{equation}
\vcenter{\xymatrix@C=1.5cm{
 & T_{q}(V)\ar[d]^-{i^{*}}\\
\Delta\ar[r]^{\phi_{q}(W)}\ar[dr]_-{\phi_{q+k}(V)}\ar[ur]^-{\phi_{q}(V)} & T_{q}(W)\ar[d]^-{i_{*}}\\
 & T_{q+k}(V)
}}\label{art08-sec4-eq4.16}
\end{equation}
Combining \eqref{art08-sec2-eq2.13} with \eqref{art08-sec4-eq4.16}, we have the following commutative diagram:
\begin{equation}
\vcenter{\xymatrix@C=1.5cm{
 & T_{q}(V)\ar[dd]^-{\Psi}\\
\Delta \ar[ur]_{\phi_{q}}\ar[dr]_-{\phi_{q+k}} & \\
 & T_{q+k}(V)
}}\label{art08-sec4-eq4.17}
\end{equation}
where $\Psi\in H^{k,k}(V)\cap H^{2k}(V,\bfZ)$ is the Poincar\'e dual of $W\in H_{2n-2k}(V,\bfZ)$ (c.f. \eqref{art08-sec2-eq2.7}).

\begin{remark*}
Actually, we see that \eqref{art08-sec4-eq4.17} holds for {\em all} algebraic cycles $W_{n-k}\subset V_{n}$, provided we assume a foundational point concerning the {\em suitable equivalence relation} (= S.E.R.) in Remark \ref{art08-sec4-rem4.12}. Let $\Sigma_{q}(V)$ be the algebraic cycles of codimension $q$ which are homologous\pageoriginale to zero, and assume that S.E.R. has the property that, for any $W_{n-k}\subset V_{n}$, the mapping $\Sigma_{q}(V)/\text{S.E.R.}\xrightarrow{W}\Sigma_{q+k}(V)/\text{S.E.R.}$ is defined and $W(Z)=W\cdot Z$ if the intersection is proper $(Z\in \Sigma_{q}(V))$. Then we have that: The following diagram commutes:
\begin{equation}
\vcenter{\xymatrix{
\Sigma_{q}(V)/\text{S.E.R.}\ar[dd]^-{W}\ar[r]^-{\phi_{q}} & T_{q}(V)\ar[dd]^-{\Psi} &\\
 & & (\Psi=\mathscr{D}(W)).\\
\Sigma_{q+k}(V)/\text{S.E.R.}\ar[r]^-{\phi_{q+k}} & T_{q+k}(V) &
}}\label{art04-sec4-eq4.18}
\end{equation}
\end{remark*}

\begin{proof}
The proof of \eqref{art08-sec4-eq4.17} will show that \eqref{art08-sec4-eq4.18} commutes when $W$ is a Chern class of an {\em ample bundle} \cite{art04-key11}. However, by \cite{art08-key12} the Chern classes of ample bundles generate the {\em rational equivalence ring} on $V$, so that \eqref{art08-sec4-eq4.18} holds in general.
\end{proof}

\medskip
\noindent
{\bf Appendix to Paragraph \ref{art08-sec4}.}~ Let $\bfE\to V$ be a holomorphic vector bundle with fibre $\bfC^{k}$, and $\sigma_{1},\ldots,\sigma_{k-q+1}$ holomorphic cross-sections of $\bfE\to V$ such that the subvariety $W=\{\sigma_{1}\wedge\ldots\wedge \sigma_{k-q+1}=0\}$ is a generally singular subvariety $W_{n-q}\subset V_{n}$ of codimension $q$. (Note the shift in indices from \S\ref{art08-sec4}.) Then the homology class $W\in H_{2n-2k}(V,\bfZ)$ is the {\em Poincar\'e dual} of the $q^{\text{th}}$ {\em Chern class} $c_{q}\in H^{2q}(V,\bfZ)$ (c.f. \cite{art08-key11}). We shall prove: There exists a differential form $\psi$ on $V$ such that
\begin{align*}
& \psi \text{~ is of type~ }(q,q-1), \text{~ is~ } C^{\infty}\text{~ in~ } V-W,\text{~ and~ } \partial \psi=0;\tag{A4.1}\label{art08-sec4-eqA4.1}\\
& \overline{\partial}\psi =d\psi \text{~ is~ } C^{\infty}\text{~ on~ } V\text{~ and represents~} c_{q}~\text{(via deRham)};\tag{A4.2}\label{art08-sec4-eqA4.2}\\
& \psi \text{~ has a pole of order~ } 2q-1\text{~ along~ } W \text{~ and, if~ }\omega \text{~ is any closed}\\
& 2n-2q\text{~ form on~ } V,\int\limits_{V}c_{a}\wedge\omega=\lim\limits_{\epsilon\to 0}\int\limits_{\partial T_{\epsilon}}\psi\wedge \omega\text{~ where~ } T_{\epsilon}\subset V\text{~ is}\\
& \text{the~ }\epsilon\text{-tubular neighbourhood around~ } W.\tag{A4.3}\label{art08-sec4-eqA4.3}
\end{align*}

\begin{proof}
For a $k\times k$ matrix $A$, define $P_{q}(A)$ by:
\begin{equation*}
\det\left(\frac{i}{2\pi}A+tI\right)=\sum\limits^{k}_{q=0}P_{q}(A)t^{k-q}.\tag{A4.4}\label{art08-sec4-eqA4.4}
\end{equation*}
Let\pageoriginale $P_{q}(A_{1},\ldots,A_{q})$ be the invariant, symmetric multilinear form obtained by polarizing $P_{q}(A)$ (for example, 
$$
P_{k}(A_{1},\ldots,A_{k})=1/k!\sum\limits_{\pi_{1},\ldots,\pi_{k}}\det(A^{1}_{\pi_{1}},\ldots,A^{k}_{\pi_{k}})
$$ 
where $A^{\alpha}_{\pi_{\alpha}}$ is the $\alpha^{\text{th}}$ column of $A_{\pi_{\alpha}}$; cf. (6.5) below). Choose an Hermitian metric in $\bfE\to V$ and let $\Theta\in A^{1,1}(V,\Hom (\bfE,\bfE))$ be the curvature of the metric connection. Then (c.f. \cite{art08-key11}):

$c_{q}\in H^{2q}(V,\bfC)$ is represented by the differential form
\begin{equation*}
P_{q}(\Theta)=P_{q}(\Theta,\ldots,\Theta).\tag{A4.5}\label{art08-sec4-eqA4.5}
\end{equation*}
\end{proof}

What we want to do is to construct $\psi$, depending on $\sigma_{1},\ldots,\sigma_{k-q+1}$ and the metric, such that \eqref{art08-sec4-eqA4.1}--\eqref{art08-sec4-eqA4.3} are satisfied. The proof proceeds in four steps.


(a)~ \textsc{Some Formulae in Local Hermitian Geometry.} Suppose that $Y$ is a complex manifold ($Y$ will be $V-W$ in applications) and that $\bfE\to Y$ is a holomorphic vector bundle such that we have an exact sequence :
\begin{equation*}
0\to \bfS\to \bfE\to \bfQ\to 0\tag{A4.6}\label{art08-sec4-eqA4.6}
\end{equation*}
(in applications, $\bfS$ will be the trivial sub-bundle generated by $\sigma_{1},\ldots,\sigma_{k-q+1}$). We assume that there is an Hermitian metric in $\bfE$ and let $D$ be the metric connection \cite{art08-key11}. Let $e_{1},\ldots,e_{k}$ be a unitary frame for $\bfE$ such that $e_{1},\ldots,e_{s}$ is a frame for $\bfS$. Then $De_{\rho}=\sum\limits^{k}_{\sigma=1}\theta^{\alpha}_{\rho}e_{\sigma}$ where $\theta^{\sigma}_{\rho}+\overline{\theta}^{\rho}_{\sigma}=0$. By the formula $D_{\bfS}e_{\alpha}=\sum\limits^{s}_{\beta=1}\theta^{\beta}_{\alpha}e_{\beta}(\alpha=1,\ldots,s)$, there is defined a connection $D_{\bfS}$ in $\bfS$, and we claim that $D_{\bfS}$ is the connection for the induced metric in $\bfS$ (c.f. \cite{art08-key11}, \S1.d).

\begin{proof}
Choose a {\em holomorphic} section $e(z)$ of $\bfS$ such that $e(0)=e_{\alpha}(0)$ (this is over a small coordinate neighborhood on $Y$). Then $D''e=0$ since $D''=\overline{\partial}$. Thus, writing $e(z)=\sum\limits^{s}_{\alpha=1}\xi^{\alpha}e_{\alpha}$, $0=D''e=\sum\limits^{s}_{\alpha=1}\overline{\partial}\xi^{\alpha}e_{\alpha}+\sum\limits^{s}_{\alpha,\beta=1}\xi^{\alpha}\theta^{\beta''}_{\alpha}e_{\beta}+\sum\limits^{k}_{\mu=s+1}\sum\limits^{s}_{\alpha=1}\xi^{\alpha}\theta^{\mu''}_{\alpha}e_{\mu}$. At $z=0$, this gives $\sum\limits^{s}_{\beta=1}(\overline{\partial}\xi^{\beta}(0)+\theta^{\beta''}_{\alpha})e_{\beta}+\sum\limits^{k}_{\mu=s+1}\theta^{\mu''}_{\alpha}e_{\mu}=0$. Thus $\theta^{\mu''}_{\alpha}=0$ and, since $(D''-D''_{\bfS})e_{\alpha}=\sum\limits_{\mu=s+1}\theta^{\mu''}_{\alpha}e_{\mu}$,\pageoriginale $D''_{\bfS}=\overline{\partial}$. By uniqueness, $D_{\bfS}$ is the connection of the induced metric in $\bfS$.
\end{proof}

\medskip
\noindent
{\bf Remark. (A4.7)}~ Suppose that $\bfS$ has a {\em global holomorphic frame} $\sigma_{1},\ldots,\sigma_{s}$. Write $\sigma_{\alpha}=\sum\limits^{s}_{\beta=1}\xi^{\beta}_{\alpha}e_{\beta}$. From $0=\overline{\partial}\sigma_{\alpha}=D''\sigma_{\alpha}=\sum\limits^{s}_{\beta=1}(\overline{\partial}\xi^{\beta}_{\alpha}+\sum\limits^{s}_{\gamma=1}\xi^{\gamma}_{\alpha}\theta^{\beta''}_{\gamma})e_{\beta}$, we get $\overline{\partial}\xi+\theta''_{\bfS}\xi=0$ or $\theta''_{\bfS}=-\overline{\partial}\xi\xi^{-1}$. This gives
\begin{equation*}
\theta_{\bfS}={}^{t}\overline{\xi}^{-1}\partial \overline{\xi}-\overline{\partial}\xi\xi^{-1}.\tag{A4.8}\label{art08-sec4-eqA4.8}
\end{equation*}

Now write $\theta=\left(\begin{smallmatrix} \theta^{1}_{1} & \theta^{1}_{2}\\ \theta^{2}_{1} & \theta^{2}_{2}\end{smallmatrix}\right)$ where $\theta^{1}_{1}=(\theta^{\alpha}_{\beta})$, $\theta^{2}_{1}=(\theta^{\mu}_{\alpha})$, etc. Then $\theta^{2''}_{1}=0=\theta^{1'}_{2}$ (since $\theta^{2}_{1}+{}^{t}\overline{\theta}^{1}_{2}=0$). Let $\phi=\left(\begin{smallmatrix} 0 & - \theta^{1}_{2}\\ -\theta^{2}_{1} & 0\end{smallmatrix}\right)$ and $\widehat{\theta}=\theta+\phi=\left(\begin{smallmatrix} \theta^{1}_{1} & 0\\ 0 & \theta^{2}_{2}\end{smallmatrix}\right)$. Then $\theta$ and $\widehat{\theta}$ give connections $D$ and $\widehat{D}$ in $\bfE$ with curvatures $\Theta$ and $\widehat{\Theta}$. Setting $\theta_{t}=\theta+t\phi$, we have a homotopy from $\theta$ to $\widehat{\theta}$ with $\overdot{\theta}_{t}=\phi\left(\overdot{\theta}_{t}=\dfrac{\partial \theta_{t}}{\partial t}\right)$.

Now let $P(A)$ be an invariant polynomial of degree $q$ (c.f. \S\ref{art08-sec6} below) and $P(A_{1},\ldots,A_{q})$ the corresponding invariant, symmetric, multilinear from (c.f. (6.5) for an example). Thus $P(A)=P(\underbrace{A,\ldots,A}_{q})$. Set
\begin{equation*}
Q_{t}=2\sum\limits^{q}_{j=1}P(\theta_{t},\ldots,\phi'_{j},\ldots,\Theta_{t}),\tag{A4.9}\label{art08-sec4-eqA4.9}
\end{equation*}
and define $\Psi_{1}$ by:
\begin{equation*}
\Psi_{1}=\int\limits^{1}_{0}Q_{t}dt.\tag{A4.10}\label{art08-sec4-eqA4.10}
\end{equation*}
What we want to prove is (c.f. \cite{art08-key11}, \S\ref{art08-sec4}):
\begin{equation*}
\left.
\begin{array}{l}
\Psi_{1}\text{~ is a~ } C^{\infty}\text{~ form of type~ } (q,q-1)\text{~ on~ } Y\text{~ satisfying}\\[4pt]
\partial\Psi_{1}=0, \ \overline{\partial}\Psi_{1}=P(\Theta)-P(\widehat{\Theta}).
\end{array}
\right\}\tag{A4.11}\label{art08-sec4-eqA4.11}
\end{equation*}

\begin{proof}
It will suffice to show that
\begin{align*}
& \partial Q_{t}=0,\quad\text{and}\tag{A4.12}\label{art08-sec4-eqA4.12}\\[3pt]
& \overdot{P}(\Theta_{t})=\overline{\partial}Q_{t}.\tag{A4.13}\label{art08-sec4-eqA4.13}
\end{align*}

By\pageoriginale the {\em Cartan structure equation,} $\Theta_{t}=d\theta_{t}+\theta_{t}\wedge \theta_{t}=d(\theta+t\phi)+(\theta+t\phi)\wedge (\theta+t\phi)=d\theta+\theta\wedge\theta+t(d\phi+\phi\wedge \theta+\theta\wedge\phi)+t^{2}\phi\wedge \phi=\Theta+tD\phi+t^{2}\phi\wedge\phi$. Now
$$
\phi\wedge \phi=\left(\begin{matrix} \theta^{1}_{3}\theta^{2}_{1} & 0 \\ 0 & \theta^{2}_{1}\theta^{1}_{2}\end{matrix}\right)~\text{and}~ D\phi=\left(\begin{matrix} 0 & -\Theta^{1}_{2}\\ -\Theta^{2}_{1} & 0^{2}\end{matrix}\right)+2\left(\begin{matrix} -\theta^{1}_{2}\theta^{2}{1} & 0\\ 0 & \theta^{2}_{1}\theta^{1}_{2}\end{matrix}\right).
$$
This gives
\begin{equation*}
\Theta_{t}=\Theta+t\left(\begin{matrix} 0 & -\Theta^{1}_{2}\\ -\Theta^{2}_{1} & 0\end{matrix}\right)+(t^{2}-2t)\left(\begin{matrix} \theta^{1}_{2}\theta^{2}_{1} & 0\\ 0 & \theta^{2}_{1}\theta^{1}_{2}\end{matrix}\right);\tag{A4.14}\label{art08-sec4-eqA4.14}
\end{equation*}
\begin{equation*}
D'\phi'=0;\quad\text{and}\tag{A4.15}\label{art08-sec4-eqA4.15}
\end{equation*}
\begin{equation*}
D''\phi'=\left(\begin{matrix} 0 & 0 \\ -\Theta^{2}_{1} & 0\end{matrix}\right)+\left(\begin{matrix}-\theta^{1}_{2}\theta^{2}_{1} & 0\\ 0 & -\theta^{2}_{1}\theta^{1}_{2}\end{matrix}\right)\tag{A4.16}\label{art08-sec4-eqA4.16}
\end{equation*}
It follows that $\Theta_{t}$ is of type $(1,1)$ and so $Q_{t}$ is of type $(q,q-1)$, as is $\Psi_{1}$.

By symmetry, to prove \eqref{art08-sec4-eqA4.12} it will suffice to have $\partial P(\Theta_{t},\ldots,\Theta_{t},\phi')=0$. Let $D_{t}=D'_{t}+D''_{t}$ be the connection corresponding to $\theta_{t}$. Then $D'_{t}\Theta_{t}=0$ (\textit{Bianchi identity}) and $\partial P(\Theta_{t},\ldots,\Theta_{t},\phi')=\Sigma P(\Theta_{t},\ldots,D'_{t}\Theta_{t},\ldots,\Theta_{t},\phi')+P(\Theta_{t},\ldots,\Theta_{t},D'_{t}\phi')=P(\Theta_{t},\ldots,\Theta_{t},D'_{t}\phi')$. But $D'_{t}\phi'=D'\phi'+t[\phi,\phi']'=0+t[\phi',\phi']=0$ by \eqref{art08-sec4-eqA4.15}. This proves \eqref{art08-sec4-eqA4.12}.

We now calculate $\overline{\partial}P(\Theta_{t},\ldots,\phi',\ldots,\Theta_{t})=\Sigma P(.,D''_{t}\Theta_{t},\ldots,\phi',\ldots,\Theta_{t})+P(\Theta_{t},\ldots,D''_{t}\phi',\ldots,\Theta_{t})+\Sigma P(\Theta_{T},\ldots,phi',\ldots,D''_{t}\Theta_{t},.)=P(\Theta_{T},\ldots,D''_{t}\phi',\ldots,\Theta_{t})$ (since $D''_{t}\Theta_{T}=0$ by Bianchi). Then we have $D''_{t}\phi'=D''\phi'+t[\phi,\phi']''=$
\begin{gather*}
(\text{by~ \eqref{art08-sec4-eqA4.16}}) \ \ \left(\begin{matrix} 0 & 0 \\ -\Theta^{2}_{1} & 0\end{matrix}\right)+(t-1)\left(\begin{matrix} \theta^{1}_{2} \theta^{2}_{1} & 0 \\ 0 & \theta^{2}_{1}\theta^{1}_{2}\end{matrix}\right). \ \ \text{But, by~ \eqref{art08-sec4-eqA4.14},}\\[3pt]
\overdot{\Theta}_{t}=\left(\begin{matrix} 0 & -\Theta^{1}_{2}\\ - \Theta^{2}_{1} & 0\end{matrix}\right)+2(t-1)\left(\begin{matrix} \theta^{1}_{2}\theta^{2}_{1} & 0\\ 0 & \theta^{2}_{1}\theta^{1}_{2}\end{matrix}\right),
\end{gather*}
so that
\begin{equation*}
2D''_{t}\phi'=\overdot{\Theta}_{T}=\left(\begin{matrix} 0 & \Theta^{1}_{2}\\ -\Theta^{2}_{1} & 0\end{matrix}\right)=[\pi,\Theta],\tag{A4.17}\label{art08-sec4-eqA4.17}
\end{equation*}
where $\pi=\left(\begin{smallmatrix} 1 & 0\\ 0 & 0\end{smallmatrix}\right)$. Using \eqref{art08-sec4-eqA4.17}, $\overline{\partial}Q_{t}-\overdot{P}(\Theta_{t})=\Sigma\{2P(\Theta_{t},\ldots,D''_{t}\phi',\ldots,\Theta_{t})-P(\Theta_{t},\ldots,\overdot{\Theta}_{t},\ldots,\Theta_{T})\}=\Sigma P(\Theta_{t},\ldots,[\phi,\Theta_{t}],\ldots,\Theta_{t})=0$. This proves \eqref{art08-sec4-eqA4.13} and completes the proof of \eqref{art08-sec4-eqA4.11}.
\end{proof}

Return\pageoriginale now to the form $Q_{t}$ defined by \eqref{art08-sec4-eqA4.9}. Since $\Theta_{T}=\Theta+tD\phi+t^{2}\phi\wedge \phi$, we see that $\underline{Q_{t}}$ is a polynomial in the differential forms $\Theta^{\rho}_{\sigma}$, $\theta^{\mu}_{\alpha}$, $\theta^{\alpha}_{\mu}(1\leq \rho, \sigma\leq k; 1\leq \alpha\leq s;s+1\leq \mu\leq k)$. Write $Q_{t}\equiv 0(l)$ to symbolize that each term in $Q_{t}$ contains {\em no more} than $l-t$ terms involving the $\theta^{\mu}_{\alpha}$ and $\theta^{\alpha}_{\mu}$. We claim that
\begin{equation*}
Q_{t}\equiv 0(2q-1).\tag{A4.18}\label{art08-sec4-A4.18}
\end{equation*}

\begin{proof}
The term of highest order (i.e. containing the most $\theta^{\mu}_{\alpha}$ and $\theta^{\alpha}_{\mu}$) in $Q_{t}$ is $(t^{2}/2)^{q-1}\Sigma P([\phi,\phi],\ldots,\phi',\ldots,[\phi,\phi])$. Now, by invariance,
\begin{align*}
P([\phi,\phi],\ldots,\phi',\ldots,[\phi,\phi]) &= -P(\phi,[\phi,\phi],\ldots,[\phi,\phi'],\ldots,[\phi,\phi])\\[3pt]
&= -\frac{1}{2}P(\phi,[\phi,\phi],\ldots,[\phi,\phi])
\end{align*}
since $[\phi,[\phi,\phi]]=0$ and $[\phi,\phi']=\frac{1}{2}[\phi,\phi]$. But, by invariance again, $P(\phi,[\phi,\phi],\ldots,[\phi,\phi])=0$. Since all other terms in $Q_{t}$ are of order $2q-2$ or less, we obtain \eqref{art08-sec4-eqA4.18}.

It follows from \eqref{art08-sec4-eqA4.10} that
\begin{equation*}
\Psi_{1}\equiv 0(2q-1).\tag{A4.19}\label{art08-sec4-eqA4.19}
\end{equation*}

(b) \textsc{Some further formulae in Hermitian geometry.} Retaining the situation \eqref{art08-sec4-eqA4.6}, we have from \eqref{art08-sec4-eqA4.11} and \eqref{art08-sec4-eqA4.19} that:
\begin{equation*}
P(\Theta)-P(\widehat{\Theta})=\overline{\partial}\Psi_{1}\text{~~ where~~ } \partial\Psi_{1}=0,\Psi_{1}\equiv 0(2q-1).\tag{A4.20}\label{art08-sec4-eqA4.20}
\end{equation*}
Now suppose that $\bfS$ has fibre dimension $k-q+1$ and that:
\begin{equation*}
\bfS\text{~~ has a global holomorphic frame~~ } \sigma_{1},\ldots,\sigma_{k-q+1}.\tag{A4.21} \label{art08-sec4-eqA4.21}
\end{equation*}
Let $\bfL_{1}\subset \bfS$ be line bundle generated by $\sigma_{1}$ and $\bfS_{1}=\bfS/\bfL_{1}$. Then $\sigma_{2}$ gives a non-vanishing section of $\bfS_{1}$ and so generates a line bundle $\bfL_{2}\subset \bfS_{1}$. Continuing, we get a diagram:
\begin{equation*}
\left.
\begin{array}{l@{\;}l@{\;}l@{\;}l}
0\longrightarrow \bfL_{1} & \longrightarrow \bfS & \longrightarrow \bfS_{1} & \longrightarrow 0\\
0\longrightarrow \bfL_{2} & \longrightarrow \bfS_{1} & \longrightarrow \bfS_{2} & \longrightarrow 0\\
 & \multicolumn{1}{c}{$\vdots$} & & \\
0\longrightarrow \bfL_{k-q} & \longrightarrow \bfS_{k-q-1} & \longrightarrow \bfS_{k-q} & \longrightarrow 0\\
0\longrightarrow \bfL_{k-q+1} & \longrightarrow \bfS_{k-q} & \longrightarrow 0 &
\end{array}
\right\}\tag{A4.22}\label{art08-sec4-eqA4.22}
\end{equation*}
All\pageoriginale the bundles in \eqref{art08-sec4-eqA4.22} have metrics induced from $\bfS$; as a $C^{\infty}$ bundle, $\bfS\cong \bfL_{1}\oplus\cdots\oplus \bfL_{k-q+1}$ (this is actually true as holomorphic bundles, but the splitting will {\em not} be this orthonormal one).

Now suppose that we use unitary frames $(e_{1},\ldots,e_{k-q+1})$ for $\bfS$ where $e_{\alpha}$ is a unit vector in $\bfL_{\alpha}$. If $\theta_{\bfS}=(\theta^{\alpha}_{\beta})$ is the metric connection in $\bfS$, then $\theta^{\alpha}_{\alpha}$ gives the connection of the induced metric in $\bfL_{\alpha}$ (c.f. (a) above). This in turn gives a connection
\begin{align*}
\gamma_{\bfS} &=
\begin{bmatrix}
\gamma^{1}_{1} & & 0\\
 & \ddots & \\
0 & & \gamma^{k-q+1}_{k-q+1}
\end{bmatrix}\quad 
(\gamma^{\alpha}_{\alpha}=\theta^{\alpha}_{\alpha})\text{~ with curvature~ }\\
\Gamma_{\bfS} &=
\begin{bmatrix}
\Gamma^{1}_{1} & & 0\\
        & \ddots & \\
0 & & \Gamma^{k-q+1}_{k-q+1}
\end{bmatrix}
\end{align*}
in $\bfS$. Now the connection $\widehat{\theta}=\theta_{\bfS}\oplus \theta_{\bfQ}$ in $\bfE$ has curvature $\widehat{\Theta}=\left[\begin{smallmatrix} \Theta_{\bfS} & 0\\ 0 & \Theta_{\bfQ}\end{smallmatrix}\right]$. We let $\Gamma=\left[\begin{smallmatrix} \Gamma_{\bfS} & 0\\ 0 & \Theta_{\bfQ}\end{smallmatrix}\right]$ be the curvature of the connection $\left[\begin{smallmatrix} \gamma_{\bfS} & 0\\ 0 & \theta_{\bfQ}\end{smallmatrix}\right]$ in $\bfE$. Then the same argument as used in (a) to prove \eqref{art08-sec4-eqA4.20}, when iterated, gives
\begin{equation*}
P(\widehat{\Theta})-P(\Gamma)=\overline{\partial}\Psi_{2}\quad\text{where}\quad \partial \Psi_{2}=0\text{~ and~ } \Psi_{2}\equiv 0(2q).\tag{A4.23}\label{art08-sec4-eqA4.23}
\end{equation*}

The congruence $\Psi_{2}\equiv 0(2q)$ is trivial in this case since $\deg \Psi_{2}=2q-1$. Adding \eqref{art08-sec4-eqA4.20} and \eqref{art08-sec4-eqA4.23} gives:
\begin{equation*}
P(\Theta)-P(\Gamma)=\overline{\partial}(\Psi_{1}+\Psi_{2}).\tag{A4.24}\label{art08-sec4-eqA4.24}
\end{equation*}

The polynomial $P(A)$ is of degree $q$, and we assume now that: 
\begin{equation*}
P(A)=0 \text{~ if~ } A=\left(\begin{matrix}0 & 0\\ 0 & A'\end{matrix}\right)\text{~ where~ } A' \text{~ is a~ } (q-1)\times (q-1)\text{~ matrix.}\tag{A4.25}\label{art08-sec4-eqA4.25}
\end{equation*}
\end{proof}

We claim then that
\begin{equation*}
P(\Gamma)=\overline{\partial}\Psi_{3}\text{~ where~ } \partial \Psi_{3}=0\text{~ and~ }\Psi_{3}\equiv 0(2q).\tag{A4.26}\label{art08-sec4-eqA4.26}
\end{equation*}

\begin{proof}
Each line bundle $\bfL_{\alpha}$ has a holomorphic section $\sigma_{\alpha}=|\sigma_{\alpha}|e_{\alpha}$. From $0=\overline{\partial}\sigma_{\alpha}=\overline{\partial}|\sigma_{\alpha}|e_{\alpha}+|\sigma_{\alpha}|\theta^{\alpha''}_{\alpha}e_{\alpha}$, we find $\theta^{\alpha''}_{\alpha}=-\overline{\partial}\log |\sigma_{\alpha}|$ and 
\begin{align*}
& \theta^{\alpha}_{\alpha}=(\partial - \overline{\partial})\log |\sigma_{\alpha}|,\quad\text{and}\tag{A4.27}\label{art08-sec4-eqA4.27}\\[3pt]
& \Gamma^{\alpha}_{\alpha}=2\overline{\partial}\partial \log |\sigma_{\alpha}|.\tag{A4.28}\label{art08-sec4-eqA4.28}
\end{align*}

Now\pageoriginale 
$$
P(\Gamma)=P\underbrace{(\Gamma_{S}+\Theta_{\bfQ},\ldots,\Gamma_{\bfS}+\Theta_{\bfQ})}_{q}=\sum\limits_{\substack{r+s=q\\ r>0}}\binom{q}{r}P(\underbrace{\Gamma_{S}};\underbrace{\Theta_{\bfQ}}_{s})
$$ 
(since $P(\underbrace{\Theta_{\bfQ},\ldots,\Theta_{\bfQ}}_{q})=0$ by \eqref{art08-sec4-eqA4.25}). 
Let 
$$
\xi=2\left[\begin{array}{llll} \theta^{1'}_{1} & &\\
 & \ddots & 0 & 0\\
 & &\theta^{k-q+1'}_{k-q+1} &\\
 & & 0 & \theta^{\mu}_{\nu}
 \end{array}\right].
$$
in $\bfL_{1}\oplus\cdots\oplus \bfL_{k-q+1}\Theta\bfQ$. Also $D''_{\gamma}\xi=\left(\begin{smallmatrix} \Gamma_{\bfS} & 0\\ 0 & 0\end{smallmatrix}\right)$. Set
\begin{equation*}
\Psi_{3}=\sum\limits_{\substack{r+s=q\\ r>0}}\binom{q}{r}P(\xi,\underbrace{\Gamma_{\bfS}}_{r-1};\underbrace{\Theta_{\bfQ}}_{s}).\tag{A4.29}\label{art08-sec4-eqA4.29}
\end{equation*}
Then $\partial \Psi_{3}=0$ since $D'_{\gamma}\xi=0=D'_{\gamma}\Gamma_{\bfS}=D'_{\gamma}\Theta_{\bfQ}$, and 
$$
\overline{\partial}\Psi_{3}=\sum\limits_{\substack{r+s=q\\ r>0}}\binom{q}{r}P(\underbrace{\Gamma_{\bfS}}_{r};\underbrace{\Theta_{\bfQ}}_{s})=P(\Gamma)
$$ 
since $D''_{\gamma}\xi=\Gamma_{\bfS}$, $D''_{\gamma}\Gamma_{\bfS}=0=D''_{\gamma}\Theta_{\bfQ}$.
This shows that $\Psi_{3}$ defined by \eqref{art08-sec4-eqA4.29} satisfies \eqref{art08-sec4-eqA4.26}.

Combining \eqref{art08-sec4-eqA4.24} and \eqref{art08-sec4-eqA4.26} gives:
\begin{equation*}
P(\Theta)=\overline{\partial}\Psi \text{~ where~ } \Psi=\Psi_{1}+\Psi_{2}+\Psi_{3},\partial \Psi=0,\Psi\equiv 0(2q).\tag{A4.30}\label{art08-sec4-eqA4.30}
\end{equation*}

Let $\Psi$ be given as just above by \eqref{art08-sec4-eqA4.30}; $\Psi$ is a form of type $(q,q-1)$ on $Y$. Suppose we refine the congruence symbol $\equiv$ so that $\eta\equiv 0(l)$ means that $\eta$ contains at most $l=1$ terms involving $\theta^{1'}_{1}$, $\theta^{k-2+q}_{1},\ldots,\theta^{k}_{1}$, $\theta^{1}_{k-q+2},\ldots,\theta^{1}_{k}$. Then, for some constant $c$,
\begin{equation*}
\Psi \equiv c\theta^{1'}_{1}\theta^{k-2+q}_{1}\ldots\theta^{k}_{1}\theta^{1}_{k-q+2}\ldots\theta^{1}_{k}(2q-1).\tag{A4.31}\label{art08-sec4-eqA4.31}
\end{equation*}
We want to calculate $c$ when $P(A)=P_{q}(A)$ corresponds to the $q^{\text{th}}$ Chern class (c.f. \eqref{art08-sec4-eqA4.4}). By \eqref{art08-sec4-eqA4.19}, $\Psi_{1}\equiv 0(2q-1)$ and an inspection of \eqref{art08-sec4-eqA4.9} shows that $\Psi_{2}\equiv 0(2q-1)$. Thus $\Psi\equiv \Psi_{3}(2q-1)$.

To calculate $\Gamma_{\bfS}$, we have $\Gamma^{\alpha}_{\alpha}=d\theta^{\alpha}_{\alpha}=d\theta^{\alpha}_{\alpha}+\sum\limits^{k}_{\rho=1}\theta^{\alpha}_{\rho}\wedge \theta^{\rho}_{\alpha}-\sum\limits^{k}_{\rho=1}\theta^{\alpha}_{\rho}\wedge \theta^{\rho}_{\alpha}=\Theta^{\alpha}_{\alpha}-\sum\limits^{k}_{\rho=1}\theta^{\alpha}_{\rho}\wedge \theta^{\rho}_{\alpha}$. Thus $\Gamma^{\alpha}_{\alpha}\equiv 0(0)$ for $\alpha>1$ and 
$$
\Gamma^{1}_{1}\equiv -\sum\limits^{k}_{\mu=k-q+2}\theta^{1}_{\mu}\wedge \theta^{\mu}_{1}(0).
$$\pageoriginale
It follows that $P_{q}(\xi,\underbrace{\Gamma_{\bfS}}_{r-1}\underbrace{\Theta_{\bfQ}}_{s})\equiv 0(2q-1)$ if $r-1>0$, and so $\Psi_{3}\equiv P_{q}(\xi, \underbrace{\Theta_{\bfQ}}_{q-1})(2q-1)$.

Now, by the definition of $P_{q}$, $P_{q}(\xi,\underbrace{\Theta_{\bfQ}}_{q-1})=(i/2\pi)^{q}(1/q)\theta^{1'}_{1}\det(\Theta_{\bfQ})$. But $(\Theta_{\bfQ})^{\mu}_{\nu}=\Theta^{\mu}_{\nu}+\sum\limits^{k-q+1}_{\alpha=1}\theta^{\mu}_{\alpha}\wedge \theta^{\alpha}_{\nu}$, so that $(\Theta_{\bfQ})^{\mu}_{\nu}\equiv \theta^{\mu}_{1}\wedge\theta^{1}_{\nu}(0)$. Combining these relations gives $\Psi_{3}\equiv (i/2\pi)^{q}(1/q)\theta^{1'}_{1}\det (\theta^{\mu}_{1}\theta^{1}_{\nu})(2q-1)$ or 
\begin{equation*}
\Psi_{3}\equiv \left(\frac{1}{2\pi i}\right)^{q}(-1)^{q(q-1)/2}\theta^{1'}_{1}\theta^{k-q+2}_{1}\ldots\theta^{k}_{1}\theta^{1}_{k-q+2}\ldots\theta^{1}_{k}(2q-1).\tag{A4.32}\label{art08-sec4-eqA4.32}
\end{equation*}

Combining \eqref{art08-sec4-eqA4.30} and \eqref{art08-sec4-eqA4.32} gives
\begin{equation*}
P_{q}(\Theta)=\overline{\partial}\Psi\quad\text{where}\quad \partial \psi=0\tag{A4.33}\label{art08-sec4-eqA4.33}
\end{equation*}
and 
\begin{equation*}
\left.
\begin{array}{c}
\Psi\equiv -\Gamma(q)\theta^{1'}_{1}\theta^{k-q+2}_{1}\ldots\theta^{k}_{1}\theta^{1}_{k-q+2}\ldots\theta^{1}_{k}(2q-1)\\[3pt]
(\Gamma(q)=(1/2\pi i)(-1)^{q(q-1)/2}).
\end{array}\right\}\tag{A4.34}\label{art08-sec4-eqA4.34}
\end{equation*}
\end{proof}

(c)~ \textsc{Reduction to a local problem.} Return now to the notation and assumptions at the beginning of the Appendix to \S\ref{art08-sec4}. Taking into account \eqref{art08-sec4-eqA4.6}, \eqref{art08-sec4-eqA4.21}, \eqref{art08-sec4-eqA4.30}, and letting $Y=V-W$, we have constructed a $(q,q-1)$ form $\psi$ on $V-W$ such that $\partial\psi=0$, $\overline{\partial}\psi=P_{q}(\Theta)=c_{q}$, and such that $\psi\equiv 0(2q)$. This proves \eqref{art08-sec4-eqA4.1}, \eqref{art08-sec4-eqA4.2}, and, in the following section, we will interpret $\psi\equiv 0(2q)$ to mean that $\psi$ has a pole of order $2q-1$ along $W$.

Let $\omega$ be a closed $2n-2q$ form on $V$ and $T_{\epsilon}$ the $\epsilon$-tube around $W$. Then $\int\limits_{V}c_{q}\wedge\omega =\lim\limits_{\epsilon\to 0}\int\limits_{V-T_{\epsilon}}c_{q}\wedge \omega=\lim\limits_{\epsilon\to 0}-\int\limits_{\partial T_{\epsilon}}\psi\wedge \omega$ since $d(\psi\wedge \omega)=P_{q}(\Theta)\wedge \omega$. This proves \eqref{art08-sec4-eqA4.3}.

For the purposes of the proof of \eqref{art08-sec4-eq4.2}, we need a stronger version of \eqref{art08-sec4-eqA4.3}; namely, we need that
\begin{equation*}
\lim\limits_{\epsilon\to 0}-\int\limits_{C\cdot \partial T_{\epsilon}}\psi \wedge \eta=\int\limits_{C\cdot W}\eta,\tag{A4.35}\label{art08-sec4-eqA4.35}
\end{equation*}
where\pageoriginale $\eta\in B^{2m-2k-2q+1,n-k-q+1}(V)$ and $C$ is the $2n-2q+1$ chain on $V$ used in the proof of \eqref{art08-sec4-eq4.2}. In other words, we need to show that {\em integration with respect to $\psi$ is a residue operator along $W$}. Because both sides of \eqref{art08-sec4-eqA4.35} are linear in $\eta$, we may assume that $\eta$ has support in a coordinate neighborhood. Also, because $\psi$ will have a pole only of order $2q-1$ along $W$, it will be seen that both sides of \eqref{art08-sec4-eqA4.35} will remain unchanged if we take out of $W_{n-q}$ an algebraic hypersurface $H_{n-q-1}$ which is in general position with respect to $C$. Thus, to prove \eqref{art08-sec4-eqA4.35}, we may assume that:
\begin{equation*}
\left.
\begin{array}{l}
\eta\text{~ has support in a coordinate neighborhood on~ } V\\[3pt]
\text{where~ } \sigma_{2}\wedge\ldots\wedge \sigma_{k-q+2}\neq 0, \ \ \sigma_{1}\neq  0.
\end{array}\right]\tag{A4.36}\label{art08-sec4-eqA4.36}
\end{equation*}

This is a local question which will be resolved in the next section.

We note in passing that \eqref{art08-sec4-eq4.9} follows from \eqref{art08-sec4-eq4.10} when $C$ is a cycle on $V$, so that \eqref{art08-sec4-eq4.2} will be completely proved when \eqref{art08-sec4-eqA4.35} is proved in the local form \eqref{art08-sec4-eqA4.35} above.

(d)~ \textsc{Completion of the proof.} Over $\bfC^{n}$ consider the trivial bundle $\bfE=\bfC^{n}\times \bfC^{k}$ in which we have a Hermitian metric $(h_{\rho\sigma}(z))$ ($z=\left[\begin{smallmatrix} z^{1} & \vdots & z^{n}\end{smallmatrix}\right]$ are coordinates in $\bfC^{n}$; $1\leq \rho$, $\sigma\leq k$). We suppose that there are holomorphic sections $\sigma_{2},\ldots,\sigma_{k-q+1}$ generating the sub-bundle $\bfS'=\bfC^{n}\times \{\bfO^{q}\times \bfC^{k-q}\}$ of $\bfE$, and let $\sigma_{1}$ be a holomorphic section of the form $\sigma_{1}(z)=\left[\begin{smallmatrix} z^{1}\\ \vdots\\ z^{q}\\ 0\\ \vdots\\ 0\end{smallmatrix}\right]$. Then the locus $\sigma_{1}\wedge\ldots\wedge \sigma_{k-q+1}=0$ is given by $z^{1}=\ldots=z^{q}=0$, so that we have the local situation of \eqref{art08-sec4-eqA4.6} ($\bfS$ is generated by $\bfS'$ and $\sigma_{1}$ on $\bfC^{n}-\bfC^{n-q}$), \eqref{art08-sec4-eqA4.21} and \eqref{art08-sec4-eqA4.36}. We consider unitary frames $e_{1},\ldots,e_{k}$ for $\bfE$ where $e_{1}=\dfrac{\sigma_{1}}{|\sigma_{1}|}$, and $e_{2},\ldots,e_{k-q+1}$ is a frame for $\bfS'$. Thus $e_{1},\ldots,e_{k-q+1}$ is a frame for $\bfS | \bfC^{n}-\bfC^{n-q}$.

Write\pageoriginale $De_{1}=\sum\limits^{k}_{\rho=1}\theta^{\rho}_{1}e_{\rho}$ ($\theta^{\rho}_{1}$ is of type $(1,0)$ for $\rho>1$) and set $\Omega=-\Gamma(q)\theta^{1'}_{1}\theta^{k-q+2}_{1}\ldots\theta^{k}_{1}\theta^{1}_{k-q+2}\ldots\theta^{1}_{k}$. If $\eta$ is a compactly support $2n-2q$ form on $\bfC^{n}$, we want to show:
\begin{equation*}
\int\limits_{\bfC^{n-q}}\eta=-\lim\limits_{\epsilon\to 0}\int\limits_{\partial T_{\epsilon}}\Omega\wedge \eta,\tag{A4.37}\label{art08-sec4-eqA4.37}
\end{equation*}
where $T_{\epsilon}$ is an $\epsilon$-ball around $\bfC^{n-q}\subset \bfC^{n}$. Having done this, we will, by almost exactly the same argument, prove \eqref{art08-sec4-eqA4.35}.

Using the metric connection, we write $De_{\rho}=\sum\limits^{k}_{\sigma=1}\theta^{\sigma}_{\rho}e_{\sigma}$. Then the 1-forms $\theta^{\rho}_{\alpha}$ are smooth on $\bfC^{n}$ for $\rho\neq 1$, $\sigma\neq 1$. If we can show that the $\theta^{\rho}_{1}$ have a first order pole along $\bfC^{n-q}\subset \bfC^{n}$, then it will follow that $\Omega$ has a pole of order $2q-1$ along $\bfC^{n-q}$ and that our congruence symbol ``$\equiv$'' (c.f. just above \eqref{art08-sec4-eqA4.31}) refers to the order of pole along $\bfC^{n-q}$. We consider each vector $e_{\rho}=\left[\begin{smallmatrix} e^{1}_{\rho}(z)\\\vdots\\e^{k}_{\rho}(z)\end{smallmatrix}\right]$ as a vector field $f_{\rho}=\sum\limits^{k}_{\sigma=1}e^{\sigma}_{\rho}(z)\dfrac{\partial}{\partial z^{\sigma}}$ on $\bfC^{n}$ and, letting $f_{a}=\dfrac{\partial}{\partial z^{a}}(a=k+1,\ldots,n)$, we have a {\em tangent vector frame} $f_{1},\ldots,f_{n}$ on $\bfC^{n}$ such that $f_{2},\ldots,f_{k-q+1}$, $f_{k+1},\ldots,f_{n}$ are tangent to $\bfC^{n-q}\subset \bfC^{n}$ along $\bfC^{n-q}$. Let $\omega^{1},\ldots,\omega^{n}$ be the co-frame of $(1,0)$ forms; then if $z=\sum\limits^{n}_{j=1}z^{j}\dfrac{\partial}{\partial z^{j}}$, $dz=\sum\limits^{n}_{j=1}f_{j}\omega^{j}$. But $z=\sigma_{1}+\sum\limits^{k-q+2}_{\alpha=2}\gamma_{\alpha}e_{\alpha}+\sum\limits^{n}_{a=k+1}z^{a}f_{a}$, so that $dz=D'\sigma_{1}\sum\limits^{k-q+2}_{\alpha=2}(\partial \gamma_{\alpha}e_{\alpha}+\gamma_{\alpha}D'e_{\sigma})+\sum\limits^{n}_{a=k+1}f_{a}\omega^{a}$. This gives:
\begin{equation*}
\left.
\begin{array}{@{}ll}
\omega^{1}=2|\sigma_{1}|\theta^{1'}_{1}; & \\[4pt]
\omega^{\mu}=|\sigma_{1}|\theta^{\mu}_{1}+\sum\limits^{k-q+1}_{\alpha=2}\gamma_{\alpha}\theta^{\mu}_{\alpha} & (\mu=k-q+2,\ldots,k);\\[4pt]
\omega^{\alpha}=\partial \gamma_{\alpha}+\sum\limits^{k-q+1}_{\beta=2}\gamma_{\beta}\theta^{\alpha'}_{\beta} & (\alpha=2,\ldots,k-q+1);\text{~ and}\\
\omega^{a}= dz^{a} & (a=k+1,\ldots,n).
\end{array}
\right\}\tag{A4.38}\label{art08-sec4-eqA4.38}
\end{equation*}
It\pageoriginale follows that $\theta^{1}_{1}$, $\theta^{\mu}_{1}$ have a first order pole along $\bfC^{n-q}$ and that 
\begin{equation*}
\Omega\equiv \left(\dfrac{i}{2}\right)^{q}(-1)^{q(q-1)/2}\dfrac{1}{|\sigma_{1}|^{2q-1}}\omega^{1}\omega^{k-q+2}\ldots\omega^{k}\overline{\omega}^{k-q+2}\ldots\overline{\omega}^{k}(2q-1).\tag{A4.39}\label{art08-sec4-eqA4.39}
\end{equation*}

The situation is now this: On $\bfC^{n}$, let $f_{1},\ldots,f_{n}$ be a tangent frame such that $f_{q+1},\ldots,f_{n}$ is a frame for $\{\bfQ^{q}\times \bfC^{n-q}\}\subset \bfC^{n}$ (thus $f_{1},\ldots,f_{q}$ is a normal frame for $\bfC^{n-q}\subset \bfC^{n}$). Let $\omega^{1},\ldots,\omega^{n}$ be the dual co-frame and $\eta$ be a compactly supported $2n-2q$ form. Then we need
\begin{equation*}
-\lim\limits_{\epsilon \to 0}\int\limits_{\partial T\epsilon}\eta \wedge \Lambda=\int\limits_{\bfC^{n-q}}\eta\tag{A4.40}\label{art08-sec4-eqA4.40}
\end{equation*}
where $\Lambda =\left(\dfrac{i}{2}\right)^{q}(-1)^{q(q-1)/2}\dfrac{1}{|\sigma|^{2q-1}}\omega^{1}\omega^{2}\ldots\omega^{q}\overline{\omega}^{1}\ldots\overline{\omega}^{q}$, $T_{\epsilon}$ is the $\epsilon$-tube around $\bfC^{n-q}\subset \bfC^{n}$, and $\sigma=\left[\begin{smallmatrix} z^{1}\\\vdots\\z^{q}\\0\\\vdots\\0\end{smallmatrix}\right]$, $f_{1}=\dfrac{\sigma}{|\sigma|}$.

If now the metric in the tangent frame is the flat Euclidean one and $T_{\epsilon}$ the normal neighborhood of radius $\epsilon$, then $\Lambda$ is minus the volume element on the normal sphere of radius $\epsilon$. Writing $\eta=(\eta(0,z)+|f_{1}|\widehat{\eta})\omega^{q+1}\wedge\ldots\wedge \omega^{n}\wedge \overline{\omega}^{q+1}\wedge\ldots\wedge\overline{\omega}^{n}+\eta'$ where $\eta'\equiv 0(\omega^{1},\ldots,\omega^{q},\overline{\omega}^{1},\ldots,\overline{\omega}^{q})$, it follows that
$$
-\lim\limits_{\epsilon\to 0}\int\limits_{\partial T_{\epsilon}}\eta \wedge \Lambda =\int\limits_{\bfC^{n-q}}\eta(0,z)\omega^{q+1}\ldots\omega^{n}\overline{\omega}^{q+1}\ldots\overline{\omega}^{n}=\int\limits_{\bfC^{n-q}}\eta.
$$ 
On the other hand, if $\widehat{T}_{\epsilon}$ is another $\epsilon$-tube aroung $\bfC^{n-q}$, by Stokes' theorem
$$
|\int\limits_{\partial T_{\epsilon}}\eta \wedge \Lambda-\int\limits_{\partial \widehat{T}_{\epsilon}}\eta \wedge \Lambda |\leq |\int\limits_{T_{\epsilon}\cup \widehat{T}_{\epsilon}}d(\eta\wedge\Lambda)|.
$$
Since $\eta$ is smooth and $d\Lambda$ has a pole of order $\leq 2q-1$ along $\bfC^{n-q}$ (in fact, we may assume $d\Lambda=0$), $\lim\limits_{\epsilon\to 0}|\int\limits_{T_{\epsilon}\cup \widehat{T}_{\epsilon}}d(\eta\wedge\Lambda)|=0$. Thus, the limit on the left hand side of \eqref{art08-sec4-eqA4.40} is independent of the $T_{\epsilon}$ (as should be the case).

Now\pageoriginale $z=\sum\limits^{q}_{\alpha=1}\gamma_{\alpha}f_{\alpha}(z)+\sum\limits^{m}_{\mu=q+1}\gamma_{\mu}f_{\mu}(z)$, and we set $z_{\eta}=\sum\limits^{q}_{\alpha=1}\gamma_{\alpha}f_{\alpha}$; then the left hand side of \eqref{art08-sec4-eqA4.40} is $-\lim\limits_{\epsilon\to 0|z_{\eta}|=\epsilon}\eta\wedge \Lambda$. But $z_{\eta}=|z_{\eta}|f_{1}$, and by iterating the integral, we have
\begin{align*}
& -\lim\limits_{\epsilon\to 0}\int\limits_{T_{\epsilon}}\eta\wedge \Lambda=\int\limits_{\bfC^{n-q}}\left\{-\lim\limits_{\epsilon\to 0}\int\limits_{\substack{|z_{\eta}|=\epsilon\\ z-z_{\eta}=\text{~constant}}}\eta\wedge \Lambda\right\}=\\[4pt]
& \int\limits_{\bfC^{n-q}}\eta(0,z)\left\{-\lim\limits_{\epsilon\to 0}\int\limits_{\substack{|z_{\eta}|=\epsilon\\ z-z_{\eta}=\text{~constant}}}\right\}\omega^{q+1}\ldots\omega^{n}\overline{\omega}^{q+1}\ldots \overline{\omega}^{n}=\int\limits_{\bfC^{n-q}}\eta.
\end{align*}

To prove \eqref{art08-sec4-eqA4.35}, we refer to the proof of \eqref{art08-sec4-eq4.2} (c.f. the proof of \eqref{art08-sec3-eq3.3} in \cite{art08-key9}) and see that we may assume that $\bfC$ is a (real) manifold with boundary $\partial C=Z$. In this case the argument is substantially the same as that just given.

(e)~ \textsc{Concluding Remarks on Residues, Currents, and the Gysin Homomorphism.} Let $V$ be an algebraic manifold and $W\subset V$ an irreducible subvariety which is the $q^{\text{th}}$ Chern class of an ample bundle $\bfE\to V$. Given an Hermitian metric in $\bfE$, the differential form $P_{q}(\Theta)$ ($\Theta$ = curvature form in $\bfE$) represents the Poincar\'e dual $\mathscr{D}(W)\in H^{2n-2q}(V,\bfZ)$ of $W\in H_{2n-2q}(V,\bfZ)$. The differential form $\psi$ (having properties \eqref{art08-sec4-eq4.8}-\eqref{art08-sec4-eq4.10} which we constructed is a {\em residue operator for} $W$; that is to say:
\begin{equation*}
\begin{array}{l}
\psi\text{~ is a~ }C^{\infty}(q,q-1)\text{~ form on~ } V-W\\
\text{which has a pole of order~ }2q-1\text{~ along~ } W;
\end{array}\tag{A4.41}\label{art08-sec4-eqA4.41}
\end{equation*}
\begin{equation*}
\partial \psi=0\text{~ and~ } d\psi=\overline{\partial}\psi=P_{q}(\Theta) \text{~ is the Poincar\'e dual of~ } W;\tag{A4.42}\label{art08-sec4-eqA4.42}
\end{equation*}
and for any $2n-k$ chain $\Gamma$ meeting $W$ transversely and any smooth $2n-2q-k$ form $\eta$,
\begin{equation*}
\lim\limits_{\epsilon\to 0}-\int\limits_{\Gamma\cdot \partial T_{\epsilon}}\psi\wedge \eta=\int\limits_{\Gamma\cdot W}\eta\quad(\text{\em Residue formula}).\tag{A4.43}\label{art08-sec4-eqA4.43}
\end{equation*}

This formalism is perhaps best understood in the language of {\em currents} \cite{art08-key14}. Let then $C^{m}(V)$ be the currents of degree $m$ on $V$; by definition, $\theta\in C^{m}(V)$ is a linear form on $A^{2n-m}(V)$ (the $C^{\infty}$ forms of\pageoriginale degree $2n-m$) which is continuous in the distribution topology (c.f. Serre \cite{art08-key21}). The derivative $d\theta\in C^{m+1}(V)$ is defined by
\begin{equation*}
\langle d\theta,\lambda\rangle =\langle \theta,d\lambda\rangle \text{~~ for all~~ } \lambda\in A^{2n-m-1}(V).\tag{A4.44}\label{art08-sec4-eqA4.44}
\end{equation*}
Of course we may define $\partial \theta$, $\overline{\partial}\theta$, and speak of currents of type $(r,s)$, etc. If $Z^{m}(V)\subset C^{m}(V)$ are the closed currents $(d\theta=0)$, then we may set $\mathscr{H}^{m}(V)=Z^{m}(V)/dC^{m-1}(V)$ ({\em cohomology computed from currents}), and it is known that (c.f. \cite{art08-key14})
\begin{equation*}
H^{m}(V)\cong \mathscr{H}^{m}(V).\tag{A4.45}\label{art08-sec4-eqA4.45}
\end{equation*}

Now $P_{q}(\Theta)$ gives a current in $C^{q,q}(V)$ by $\langle P_{q}(\Theta),\lambda\rangle=\int\limits_{V}P_{q}(\Theta)\wedge \lambda (\lambda\in A^{2n-2q}(V))$. By Stokes' theorem, $dP_{q}(\Theta)$ in the sense of currents is the same as the usual exterior derivative. Thus $dP_{q}(\Theta)=0$ and $P_{q}(\Theta)\in H^{q,q}(V)$.

Also, $W$ gives a current in $C^{q,q}(V)$ by $\langle W,\lambda\rangle=\int\limits_{W}\lambda(\lambda\in A^{2n-2q}(V))$. By Stokes' theorem again, $dW=0$ (if $W$ were a manifold with boundary, then $dW$ would be just $\partial W$).

Now $\psi$ gives a current in $C^{q,q-1}(V)$ by $\langle \psi,\lambda\rangle=\int\limits_{W}\psi\wedge\lambda$ (this is because $\psi$ has a pole of order $2q-1$). To compute $d\psi\in C^{q,q}(V)$, we have, for any $\lambda\in A^{2n-2q}(V)$,
\begin{align*}
& \int\limits_{V}\psi\wedge d\lambda=\lim\limits_{\epsilon\to 0}\int\limits_{V-T_{\epsilon}}\psi\wedge d\lambda=\lim\limits_{\epsilon\to 0}\left\{\int\limits_{V-T_{\epsilon}}-d(\psi\wedge\lambda)+d\psi\wedge\lambda\right\}\\[4pt]
&=\lim\limits_{\epsilon\to 0}\int\limits_{V-T_{\epsilon}}d(\psi\wedge\lambda)+\lim\limits_{\epsilon\to 0}\int\limits_{V-T_{\epsilon}}P_{q}(\Theta)\wedge\lambda=-\int\limits_{W}\lambda+\int\limits_{V}P_{q}(\Theta)\wedge\lambda,
\end{align*}
which says that, {\em in the sense of currents},
\begin{equation*}
d\psi=P_{q}(\Theta)-W.\tag{A4.45}\label{art08-sec4-eqA4.45}
\end{equation*}

Thus, among other things, the residue operator $\psi$ expresses the fact that, in the cohomology group $\mathscr{H}^{q,q}(V)$, $P_{q}(\Theta)=W$ (which proves also that $P_{q}(\Theta)=\mathscr{D}(W)$). The point in the above calculation is that $d\psi$ in the sense of currents is {\em not} just the exterior derivative of $\psi$; the singularities force us to be careful in Stokes' theorem, so that we get \eqref{art08-sec4-eqA4.45}.

Suppose\pageoriginale that $W$ is non-singular and consider the Gysin homomorphism $H^{k}(W)\to H^{k+2q}(V)$. Given a smooth form $\phi\in A^{k}(W)$ which is closed, we choose $\widehat{\phi}\in A^{k}(V)$ with $\widehat{\phi}|W=\phi$. Then the differential form $i_{*}(\phi)=d(\psi\wedge \widehat{\phi})=P_{q}(\Theta)\wedge \widehat{\phi}-\psi\wedge d\widehat{\phi}$ will have only a pole of order $2q-2$ along $W$ (since $d\widehat{\phi}|W=0$, the term of highest order in $\psi$ involves only {\em normal} differentials along $W$, as does $d\widehat{\phi}$), and so $i_{*}(\phi)$ is a current in $C^{k+2q}(V)$. We claim that, in the sense of currents, $di_{*}(\phi)=0$.

\begin{proof}
$\int\limits_{V}d(\psi\wedge \widehat{\phi})\wedge d\lambda=\lim\limits_{\epsilon\to 0}\int\limits_{V-T_{\epsilon}}d(\psi \wedge \widehat{\phi})\wedge d\lambda=-\lim\limits_{\epsilon\to 0}\int\limits_{\partial T_{\epsilon}}d(\psi\wedge\widehat{\phi})\wedge \lambda=\lim\limits_{\epsilon\to 0}\int\limits_{\partial T_{\epsilon}}\psi\wedge d\widehat{\phi}\lambda$

(since $d\psi\wedge \widehat{\phi}=P_{q}(\Theta)\wedge \widehat{\phi}$ is smooth). But $\psi\wedge d\widehat{\phi}$ has a pole of order $2q-2$ along $W$ so that $\lim\limits_{\epsilon\to 0}\int\limits_{\partial T_{\epsilon}}\psi\wedge d\widehat{\phi}\wedge\lambda=0$).

Thus $i_{*}(\phi)$ is a {\em closed current} and so defines a class in $\mathscr{H}^{k+2q}(V)\cong H^{k+2q}(V)$; because of the residue formula \eqref{art08-sec4-eqA4.43}, $i_{*}(\phi)$ is the Gysin homomorphism on $\phi\in H^{k}(V)$.

Of course, if we are interested only in the de Rham groups $H^{k}(W)$, we may choose $\widehat{\phi}$ so that $d\widehat{\phi}=0$ in $T_{\epsilon}$ for small $\epsilon$ (since $W$ is a $C^{\infty}$ {\em retraction} of $T_{\epsilon}$). Then $d(\psi\wedge \widehat{\phi})$ is smooth and currents are unnecessary. However, if we want to keep track of the complex structure, we must use currents because $W$ is generally {\em not a holomorphic retraction} of $T_{\epsilon}$. Thus, if $\phi\in F^{k}_{l}(W)$ (so that $\phi=\phi_{k,0}+\cdots+\phi_{l,k-l}$), we may choose $\widehat{\phi}\in F^{k}_{l}(V)$ with $\widehat{\phi}|W=\phi$, but we {\em cannot} assume that $d\widehat{\phi}=0$ in $T_{\in}$. The point then is that, if we let $\mathscr{F}^{k}_{l}(W)$ and $\mathscr{F}^{k+2q}_{l+q}(V)$ be the cohomology groups computed from the {\em Hodge filtration} using currents, then we have
\begin{equation*}
\mathscr{F}^{k+2q}_{l+q}(V)\cong F^{k+2q}_{l+q}(V);\tag{A4.46}\label{art08-sec4-eqA4.46}
\end{equation*}
and the Gysin homomorphism $i_{*}:H^{k}(W)\to H^{k+2q}(V)$ satisfies $i_{*}:F^{k}_{l}(W)\to F^{k+2q}_{l+q}(V)$ and is given, as explained above, by
\begin{equation*}
i_{*}(\widehat{\phi})=d(\psi\wedge \phi)\in \mathscr{F}^{k+2q}_{l+q}(V).\tag{A4.47}\label{art08-sec4-eqA4.47}
\end{equation*}

In\pageoriginale other words, by using residues and currents, we have proved that the Gysin homomorphism is compatible with the complex structure and can be computed using the residue form.
\end{proof}

\section{Generalizations of the Theorems of Abel and Lefschetz}\label{art08-sec5}

Let $V=V_{n}$ be an algebraic manifold and $\bfZ=\bfZ_{n-q}$ an {\em effective algebraic cycle} of codimension $q$; thus $\bfZ=\sum\limits^{l}_{\alpha=1}n_{\alpha}\bfZ_{\alpha}$ where $\bfZ_{\alpha}$ is irreducible and $n_{\alpha}>0$. We denote by $\Phi=\Phi (\bfZ)$ an irreducible component containing $\bfZ$ of the {\em Chow variety} \cite{art08-key13} of effective cycles $Z$ on $V$ which are algebraically equivalent to $Z$. If $Z\in \Phi$, then $Z-\bfZ$ is homologous to zero and so, as in \S\ref{art08-sec3}, we may define $\phi_{q}:\Phi\to T_{q}(V)$. Letting $\Alb(\Phi)$ be the {\em Albanese variety} of $\Phi$, we in fact have a diagram of mappings~:
\setcounter{equation}{0}
\begin{equation}
\vcenter{
\xymatrix@C=1.8cm@R=.5cm{
 & \Alb(\Phi)\ar[dd]^-{\alpha_{\Phi}}\\
\Phi\ar[ur]^{\delta_{\Phi}}\ar[dr]_{\phi_{q}} &\\
 & T_{q}(\Phi,V).
}}\label{art08-sec5-eq5.1}
\end{equation}
Here $T_{q}(\Phi,V)$ is the torus generated by $\phi_{q}(\Phi)$ and $\delta_{\Phi}$ is the usual mapping of an irreducible variety to its Albanese. Thus, if $\psi^{1},\ldots,\phi^{m}$ are a basis for the holomorphic $1$-forms on $\Phi$, then $\delta_{\Phi}(Z)=\left[\begin{smallmatrix}\vdots\\\int^{Z}_{Z}\phi^{\rho}\\\vdots\end{smallmatrix}\right]$, where $\int^{Z}_{Z}\phi^{\rho}$ means that we take a path on $\Phi$ from $\bfZ$ to $Z$ and integrate $\psi^{\rho}$. We may assume that $\psi^{1}=\phi^{*}_{q}(\omega^{1}),\ldots,\psi^{k}=\phi^{*}_{q}(\omega^{k})$ where $\omega^{1},\ldots,\omega^{k}$ give a basis for the holomorphic $1$-forms on $T_{q}(\Phi,V)$ $(\omega^{\alpha}\in H^{n-q+1,n-q}(V))$, and then $\alpha_{\Phi}\delta_{\Phi}(Z)=\alpha_{\Phi}\left[\begin{smallmatrix}\int^{Z}_{\bfZ}\psi^{1}\\\vdots\\\int^{Z}_{\bfZ}\psi^{m}\end{smallmatrix}\right]=\left[\begin{smallmatrix}\int^{Z}_{\bfZ}\omega^{1}\\\vdots\\\int^{Z}_{\bfZ}\omega^{k}\end{smallmatrix}\right]$, where $\int^{Z}_{\bfZ}\omega^{\alpha}$ means $\int_{\Gamma}\omega^{\alpha}$ if $\Gamma$ is a $2n-2q+1$ chain on $V$ with $\partial \Gamma=Z-\bfZ$.

Let now $\bfW=\bfW_{q-1}$ be a sufficiently general irreducible subvariety of dimension $q-1$ (codimension $n-q+1$) and $\Sigma=\Sigma(\bfW)$ an irreducible component of the Chow variety of $\bfW$. Each $Z\in \Phi$ defines a divisor $D(Z)$ on $\Sigma$ by letting $D(Z)=\{\text{all~}W\in \Sigma$ such that $W$\pageoriginale meets $Z\}$. Thus, if $Z=\sum\limits^{l}_{\alpha=1}n_{\alpha}Z_{\alpha}$, $D(Z)=\sum\limits^{l}_{\alpha=1}n_{\alpha}D(Z_{\alpha})$. Letting ``$\equiv$'' denote {\em linear equivalence of divisors}, we will prove as a generalization of {\em Abel's theorem} that:
\begin{equation}
\begin{array}{l}
D(Z)\text{~ is algebraically equivalent to~ } D(\bfZ),\\[2pt]
\text{even if we only assume that $Z$ is homologous to $\bfZ$};
\end{array}\label{art08-sec5-eq5.2}
\end{equation}
and
\begin{equation}
D(Z)\equiv D(\bfZ)\text{~~ if~~ }\phi_{q}(Z)=0\text{~ in~ }T_{q}(\Phi,V).\label{art08-sec5-eq5.3}
\end{equation}

\setcounter{definition}{0}
\begin{example}\label{art08-sec5-exam1}
Suppose that $\bfZ$ is a divisor on $V$; then $\Phi$ is a projective fibre space over (part of) $\Pic(V)$ (= Picard variety of $V$) and the fibre through $Z\in \Phi$ is the {\em complete linear system} $|Z|$. Now $\bfW$ is a point on $V$ and $\Sigma=V$, and $D(Z)=Z$ as divisor on $\Sigma$. In this case, \eqref{art08-sec5-eq5.3} is just the classical Abel's theorem for divisors \cite{art08-key17}; \eqref{art08-sec5-eq5.2} is the statement (well known, of course) that homology implies algebraic equivalence. The converse to \eqref{art08-sec5-eq5.3}, which reads :
\begin{equation}
\phi_{q}(Z)=0\text{~~ if~~ } D(Z)\equiv D(\bfZ),\label{art08-sec5-eq5.4}
\end{equation}
is the trivial part of Abel's theorem in this case.
\end{example}

\begin{remark*}
We may give \eqref{art08-sec5-eq5.3} as a functorial statement as follows. The mapping $\Phi\to \Div(\Sigma)$ (given by $Z\to D(Z)$) induces $\Phi\to \Pic(\Sigma)$. From this we get $\Alb(\Phi)\to \Alb(\Pic(\Sigma))=\Pic(\Sigma)$, which combines with \eqref{art08-sec5-eq5.1} to give
\begin{equation}
\vcenter{\xymatrix{
\Alb(\Phi)\ar[d]^-{\alpha_{\Phi}}\ar[r]^-{\xi_{\Phi}} & \Pic(\Sigma)\\
T_{q}(\Phi,V)\ar@{--}[ur]_{\zeta_{\Phi}} &
}}\label{art08-sec5-eq5.5}
\end{equation}
Then \eqref{art08-sec5-eq5.3} is equivalent to saying that $\xi_{\Phi}$ factors in \eqref{art08-sec5-eq5.5}.
\end{remark*}

\begin{proof}
For $z_{0}\in \Alb(\Phi)$, there exists a zero-cycle $Z_{1}+\cdots+Z_{N}$ on $\Phi$ such that $z_{0}=\delta_{\Phi}(Z_{1}+\cdots+Z_{N})$. Let $Z=Z_{1}+\cdots+Z_{N}$ be the corresponding subvariety of $V$. Then $\alpha_{\Phi}(Z)=\phi_{q}(Z-N\bfZ)$ and, assuming \eqref{art08-sec5-eq5.3}, if $\phi_{q}(Z-N\bfZ)=0$, then $\xi_{\Phi}(Z)=0$ in $\Pic(\Sigma)$. Thus, if \eqref{art08-sec5-eq5.3} holds, $\ker \alpha_{\Phi}\supset \ker \xi_{\Phi}$ and so $\xi_{\Phi}$ factors in \eqref{art08-sec5-eq5.5}.
\end{proof}

\begin{example}\label{art08-sec5-exam2}
Let\pageoriginale $\bfZ=$ point on $V$ so that $\Phi=V$, $\Alb(\Pi)=\Alb(V)$. Choose $\bfW$ to be a very ample divisor on $V$; then $\Sigma$ is a {\em projective fibre bundle} over $\Pic(V)$ with $|W|$ as fibre through $W$ (c.f. \cite{art08-key18}). Now $D(Z)$ consists of all divisors $W\in \Sigma$ which pass through $Z$. In this case, \eqref{art08-sec5-eq5.3} reads:
\begin{equation}
\begin{array}{l}
\text{Albanese equivalence of points on $V$ implies linear}\\[2pt]
\text{equivalence of divisors on~ } \Sigma.
\end{array}\label{art08-sec5-eq5.6}
\end{equation}
\end{example}

\begin{remark*}
There is a reciprocity between $\Phi$ and $\Sigma$; each $W\in \Sigma$ defines a divisor $D(W)$ on $\Phi$ so that we have $\Alb(\Sigma)\xrightarrow{\xi_{\Sigma}}\Pic(\Phi)$. Then \eqref{art08-sec5-eq5.5} dualizes to give~:
\begin{equation}
\vcenter{\xymatrix{
\Alb(\Sigma)\ar[r]^-{\xi_{\Sigma}}\ar[d]^{\alpha_{\Sigma}} & \Pic(\Phi)\\
T_{n-q+1}(\Sigma,V)\ar@{--}[ur]_{\zeta_{\Sigma}} &
}}\label{art08-sec5-eq5.7}
\end{equation}

For example, suppose that $\dim V=2m+1$ and $q=m+1$. We may take $\bfW=\bfZ$, $\Sigma=\Phi$, and then \eqref{art08-sec5-eq5.5} and \eqref{art08-sec5-eq5.7} coincide to give~:
\begin{equation}
\vcenter{\xymatrix{
\Alb(\Phi)\ar[r]^-{\xi_{\Phi}}\ar[d]^-{\alpha_{\Phi}} & \Pic(\Phi)\\
T_{q}(\Phi,V)\ar@{--}[ur]_-{\zeta_{\Phi}} & 
}}\label{art08-sec5-eq5.8}
\end{equation}

Given $\bfZ$, $\Phi$ as above, there is a mapping
\begin{equation}
H_{r}(\Phi,\bfZ)\xrightarrow{\tau}H_{2n-2q+r}(V,\bfZ)\label{art08-sec5-eq5.9}
\end{equation}
as follows. Given an $r$-cycle $\Gamma$ on $\Phi$, $\tau(\Gamma)$ is the cycle traced out by the varieties $Z_{\gamma}$ for $\gamma\in \Gamma$. Suppose that $\Phi$ is nonsingular. Then the adjoint $\tau^{*}:H^{2n-2q+r}(V)\to H^{r}(\Phi)$ is given as follows. On $\Phi\times V$, there is a cycle $T$ with $\pr_{V}T\cdot\{Z\times V\}=Z(Z\in \Phi)$. We then have\pageoriginale $\xymatrix{T\ar[d]^{\pi}\ar[r]^-{\widetilde{\omega}} & V\\ \Phi &}$ and~:
\begin{equation}
\tau^{*}=\pi_{*}\widetilde{\omega}^{*}:H^{2n-2q+r}(V)\to H^{r}(\Phi)\label{art08-sec5-eq5.10}
\end{equation}
(here $\pi_{*}$ is {\em integration over the fibre}). Since $\Phi$ is nonsingular $D(\bfW)$ (= divisor on $\Phi$) gives a class in $H^{1,1}(\Phi)$. In fact, we will show, as a generalization of the {\em Lefschetz theorem} \cite{art08-key19}, that
\begin{equation}
\tau^{*}:H^{n-q+s,n-q+t}(V)\to H^{s,t}(\Phi);\label{art08-sec5-eq5.11}
\end{equation}
and, if $\omega\in H^{n-q+1,n-q+1}(V)$ is the dual of $\bfW\in H_{q-1,q-1}(V)\cap H_{2q-2}(V,\bfZ)$, then~:
\begin{equation}
\text{The dual of~ } D(\bfW)\text{~ is ~} \tau^{*}\omega\in H^{1,1}(\Phi).\label{art08-sec5-eq5.12}
\end{equation}

In other words, an {\em integral cohomology class} $\omega$ of type $(n-q+1, n-q+1)$ on $V$ defines a {\em divisor} on $\Phi$.
\end{remark*}

\begin{remark*}
In \eqref{art08-sec5-eq5.11}, we have
\begin{equation}
\tau^{*}:H^{n-q+1,n-q}(V)\to H^{1,0}(\Phi);\label{art08-sec5-eq5.13}
\end{equation}
this $\tau^{*}$ is just $\phi^{*}_{q}:H^{1,0}(T_{q}(V))\to H^{1,0}(\Alb(\Phi))$ where $\phi_{q}$ is given by \eqref{art08-sec5-eq5.1}.
\end{remark*}

\setcounter{theorem}{13}
\begin{remark}\label{art08-sec5-rem5.14}
The gist of \eqref{art08-sec5-eq5.2}, \eqref{art08-sec5-eq5.3} and \eqref{art08-sec5-eq5.11}, \eqref{art08-sec5-eq5.12} may be summarized by saying: The cohomology of type $(p,p)$ gives algebraic cycles, and the equivalence relation defined by the tori $T_{q}(V)$ implies rational equivalence, both on suitable Chow varieties attached to the original algebraic manifold $V$.
\end{remark}

The problem of dropping back down to $V$ still remains of course.

(a)~ \textsc{A generalization of interals of the 3rd kind to higher codimension.} We want to prove \eqref{art08-sec5-eq5.2} and \eqref{art08-sec5-eq5.3} above. Since changing $\bfZ$ or $Z$ by rational equivalence will change $D(\bfZ)$ or $D(Z)$ by linear equivalence and will not alter $\phi_{q}(Z)$, and since we may add to $\bfZ$ and $Z$ a common cycle, we will assume that $\bfZ=\sum\limits^{l}_{\alpha=1}n_{\alpha}\bfZ_{\alpha}$, $Z=\sum\limits^{k}_{\rho=1}m_{\rho}Z_{\rho}$ where the $\bfZ_{\alpha}$, $Z_{\rho}$ are {\em Chern classes of ample\pageoriginale bundles} (c.f. \S\ref{art08-sec4} above) and that all intersections are transversal. To simplify notation then, we write $Y=Z-\bfZ$ and $Y=\sum\limits^{l}_{j=1}n_{j}Y_{j}$ where the $Y_{j}$ are nonsingular Chern classes which meet transversely. We also set $|Y|=\bigcup\limits^{l}_{j=1}Y_{j}$, $V-Y=V-|Y|$.

A {\em residue operator for} $Y$ (c.f. Appendix to \S\ref{art08-sec4}, section (e) above) is given by a $C^{\infty}$ differential form $\psi$ on $V-Y$ such that~:
\begin{itemize}
\item[(i)] $\psi$ is of degree $2q-1$ and $\psi=\psi_{2q-1,0}+\cdots+\psi_{q,q-1}$ ($\psi_{s,t}$ is the part of $\psi$ of type $(s,t)$);

\item[(ii)] $\partial \psi=0$ and $\overline{\partial}\psi=\Phi$ where $\Phi$ is a $C^{\infty}(q,q)$ form on $V$ giving the Poincar\'e dual of $Y\in H_{2n-2q}(V,\bfZ)$;

\item[(iii)] $\psi-\psi_{q,q-1}$ is $C^{\infty}$ on $V$ and $\psi$ has a pole of order $2q-1$ along $Y$; and

\item[(iv)] for any $k+2q$ chain $\Gamma$ on $V$ which meets $Y$ transversely and smooth $k$-form $\omega$ on $V$
\setcounter{equation}{14}
\begin{equation}
\int\limits_{\Gamma\cdot Y}\omega=-\lim\limits_{\epsilon\to 0}\int\limits_{\Gamma\cdot \partial T_{\epsilon}}\psi \wedge\omega (\textit{\em Residue formula})\label{art08-sec5-eq5.15}
\end{equation}
where $T_{\epsilon}$ is an $\epsilon$-tube around $Y$.
\end{itemize}

From \eqref{art08-sec4-eqA4.1}-\eqref{art08-sec4-eqA4.3} and \eqref{art08-sec4-eqA4.35}, we see that a residue operator $\psi_{j}$ for each $Y_{j}$ exists. Then $\psi=\sum\limits^{l}_{j=1}n_{j}\psi_{j}$ is a residue operator for $Y$ (the formula \eqref{art08-sec5-eq5.15} has to be interpreted suitably).
\begin{equation}
\text{If }Y=0\text{ in }H_{2n-2q}(V,\bfZ),\text{ then we may assume that } \overline{\partial}\psi=0.\label{art08-sec5-eq5.16}
\end{equation}

\begin{proof}
$\overline{\partial}\psi=\Phi$ is a $C^{\infty}$ form and $\Phi=0$ in $H^{q,q}_{\overline{\partial}}(V)$. Then $\Phi=\overline{\partial}\eta$ where $\eta=\overline{\partial}^{*}G_{\overline{\partial}}\Phi$ and $\partial \eta=0$ since $\partial \overline{\partial}^{*}=-\overline{\partial}^{*}\partial$, $\partial G_{\overline{\partial}}=G_{\overline{\partial}}\partial$, $\partial \Phi=0$. Since $\eta$ is of type $(q,q-1)$, we may take $\psi-\eta$ as our residue operator.
\end{proof}

\setcounter{theorem}{16}
\begin{remark}\label{art08-sec5-rem5.17}
If $Y$ is a divisor which is zero in $H_{2n-2}(V,\bfZ)$, then $\psi$ is a holomorphic differential on $V-Y$ having $Y$ as its {\em logarithmic residue locus} (c.f. \cite{art08-key18}).
\end{remark}

\begin{remark}\label{art08-sec5-rem5.18}
Let\pageoriginale $Y$ be homologous to zero and $\psi$ be a residue operator for $Y$ with $d\psi=0$ (c.f. \eqref{art08-sec5-eq5.16}). Then $\psi$ gives a class in $H^{2q-1}(V-Y)$, and $\psi$ is determined up to $H^{2q-1,0}(V)+\cdots+H^{q,q-1}(V)$. We claim that

$H^{2q-1}(V-Y)$ is generated by $H^{2q-1}(V)$ and the $\psi_{j}$.
\end{remark}

\begin{proof}
Let $\delta_{j}$ be a normal sphere to $Y_{j}$ at some simple point not on any of the other $Y_{k}$'s. We map $\bfZ^{(l)}=\underbrace{\bfZ\oplus\cdots\oplus}_{l}\bfZ$ into $H_{2q-1}(V-Y)$ by $(\alpha_{1},\ldots,\alpha_{l})\to \sum\limits^{l}_{j=1}\alpha_{j}\delta_{j}$. Since $\int_{\delta_{j}}\psi_{j}=+1$, we must show that the sequence
\setcounter{equation}{18}
\begin{equation}
\bfZ^{(l)}\to H_{2q-1}(V-Y)\xrightarrow{i_{*}}H_{2q-1}(V)\to 0\label{art08-sec5-eq5.19}
\end{equation}
is exact. By dimension, $H_{2q-1}(V-Y)$ maps onto $H_{2q-1}(V)$. If $\sigma\in H_{2q-1}(V-Y)$ is an integral cycle which bounds in $V$, then $\sigma=\delta \gamma$ for some $2q$-chain $\gamma$ where $\gamma$ meets $Y$ transversely in nonsingular points $p_{\rho}\in Y$. If $p_{\rho}\in Y_{j(\rho)}$, then clearly $\sigma\sim \sum\limits_{\rho}\delta_{j(\rho)}$ so that $Z^{(l)}$ generates the kernel of $i_{*}$ in \eqref{art08-sec5-eq5.19}.

Consider now our subvariety $\bfW=\bfW_{q-1}$ with Chow variety $\Sigma$. We may assume that $\bfW$ lies in $V-Y$ and, for $W\in \Sigma$, $W$ lying in $V-Y$, we may write $W-\bfW=\partial\Gamma$ where $\Gamma$ is a $2q-1$ chain not meeting $Y$. Clearly $\Gamma$ is determined up to $H_{2q-1}(V-Y)$. We will show :
\begin{equation}
\begin{array}{l}
\text{There exists an integral of the $3^{\text{rd}}$ kind $\theta$ on $\Sigma$ whose logarithmic}\\[2pt]
\text{residue locus is $D(Y)$, provided that $Y=0$ in $H_{2n-2q}(V,\bfZ)$.}
\end{array}\label{art08-sec5-eq5.20}
\end{equation}
\end{proof}

\begin{proof}
Let $\psi$ be a residue operator for $Y$ with $d\psi=0$. Define a 1-form $\theta$ on $\Sigma-D(Y)$ by~:
\begin{equation}
\theta=d\left\{\int\limits^{W}_{\bfW}\psi\right\}=d\left\{\int\limits_{\Gamma}\psi\right\}.\label{art08-sec5-eq5.21}
\end{equation}
This makes sense since $d\psi=0$. We claim that
$$
\theta\text{~\em is holomorphic on~ } \Sigma-D(Y).
$$
\end{proof}

\begin{proof}
Let $\Sigma^{*}=\Sigma-D(Y)$ and $T^{*}\subset \Sigma^{*}\times V$ the graph of the correspondence $(W,z)(z\in W)$ (i.e. $W\in \Sigma^{*}$ is a subvariety of $V$ and $z\in V$ lies on $W$). Then we have
\[
\xymatrix{
T^{*}\ar[r]^-{\widetilde{\omega}}\ar[d]^-{\pi} & V.\\
\Sigma & 
}
\]\pageoriginale
Now $\widetilde{\omega}^{*}:A^{r,s}(V)\to A^{r,s}(T^{*})(A^{r,s}(*)=C^{\infty}$ forms of type $(r,s)$ on$^{*}$); since $\widetilde{\omega}$ is holomorphic, $\widetilde{\omega}^{*}\overline{\partial}=\overline{\partial}\widetilde{\omega}^{*}$. On the other hand, the {\em integration over the fibre} $\pi_{*}:A^{r+q-1,s+q-1}(T^{*})\to A^{r,s}(\Sigma^{*})$ is defined and is determined by the equation :
\setcounter{equation}{21}
\begin{equation}
\int\limits_{\Sigma^{*}}\pi_{*}\phi\wedge \eta=\int\limits_{T^{*}}\phi\wedge \pi^{*}\eta,\label{art08-sec5-eq5.22}
\end{equation}
where $\eta$ is a compactly supported form on $T^{*}$. Since $\int\limits_{\Sigma^{*}}\overline{\partial}\pi_{*}\phi\wedge \eta=(-1)^{r+s}\int\limits_{T^{*}}\phi\wedge \pi^{*}\overline{\partial}\eta=\int\limits_{T^{*}}\partial \phi\wedge \pi^{*}\eta=\int\limits_{\Sigma^{*}}\pi_{*}(\overline{\partial}\phi)\wedge \eta$ for all $\eta$, $\overline{\partial}\pi_{*}=\pi_{*}\overline{\partial}$. Let $\tau^{*}:A^{r+q-1,s+q-1}(V)\to A^{r,s}(\Sigma^{*})$ be the composite $\pi_{*}\widetilde{\omega}^{*}$. Then $\overline{\partial}\tau^{*}=\tau^{*}\overline{\partial}$ (this proves \eqref{art08-sec5-eq5.11}).

Now let $\psi\in A^{q,q-1}(V-Y)$ be a residue operator for $Y$. Then by the definition \eqref{art08-sec5-eq5.21}, $\theta=\tau^{*}\psi\in A^{1,0}(\Sigma^{*})$ and $d\theta=\tau^{*}d\psi=0$. This proves that $\theta$ is holomorphic on $\Sigma^{*}$.

Now $Y=\sum\limits_{j=1}n_{j}Y_{j}$ where the $Y_{j}$ are subvarieties of codimension $q$ on $V$. We have that $D(Y)=\sum\limits^{l}_{j=1}n_{j}D(Y_{j})$ and $\psi=\sum\limits_{j=1}n_{j}\psi_{j}$. We will prove that $\psi$ has a pole of order one on $Y_{j}$ with logarithmic residue $n_{j}$ there.
\end{proof}

\begin{proof}
We give the argument when $Y$ is irreducible and $q=n$. From this it will be clear how the general case goes.

Let $\Delta$ be the unit disc in the complex $t$-plane and $\{W_{t}\}_{t\in \Delta}$ a holomorphic curve on $\Sigma$ meeting $D(Y)$ simply at the point $t=0$. Then $W_{0}$ meets $Y$ simply at a point $z_{0}\in V$. We may choose local coordinates $z^{1},\ldots,z^{n}$ on $V$ such that $z_{0}=Y$ is the origin. Now $\psi=\dfrac{1}{|z|^{2n-1}}\left\{\sum\limits^{n}_{\alpha=1}\psi_{\alpha}dz^{1}\ldots dz^{n}d\overline{z}^{1}\ldots d\widehat{\overline{z}}^{\alpha}\ldots d\overline{z}^{n}\right\}$ where $|z|^{2}=\sum\limits^{n}_{\alpha=1}|z^{\alpha}|^{2}$ and\pageoriginale $\psi_{\alpha}$ is smooth. We may assume that $W_{t}$ is given by $z^{1}=t$, and, to prove that $\theta$ has a pole of order one at $t=0$, it will suffice to show that $\iint\limits_{\Delta}|\theta\wedge d\overline{t}|$ is finite. It is clear, however, that $\iint\limits_{\Delta}|\theta\wedge d\overline{t}|$ will be finite if $\int\limits_{|z^{\alpha}|<1}|\psi\wedge d\overline{z}^{1}|$ is finite. But
$$
|\psi \wedge d\overline{z}^{1}|\leq c\left\{\dfrac{|dz^{1}\ldots dz^{n} \ d\overline{z}^{1}\ldots d\overline{z}^{n}|}{|z|^{2n-1}}\right\}\quad(c=\text{constant}),
$$
so that $\int\limits_{|z^{\alpha}|<1}|\psi\wedge d\overline{z}^{1}|$ is finite.

We now want to show that $\int\limits_{|t|=1}\theta=+1$ (i.e. $\theta$ has logarithmic residue $+1$ on $D(Y)$). Let $\delta=\bigcup\limits_{|t|=1}W_{t}$. Then $\int\limits_{|t|=1}\theta=\int\limits_{\delta}\psi$. If $T_{\epsilon}=\{z:|z|<\epsilon\}$, then setting $\Gamma=\left\{\bigcup\limits_{|t|\leq 1}W_{t}\right\}-T_{\epsilon},\partial \Gamma=\delta+\partial T_{\epsilon}$. Thus $\int\limits_{\delta}\psi=-\int\limits_{\partial T_{\epsilon}}\psi=+1$ as required.
\end{proof}

\begin{remark*}
Let $Y\subset V$ be as above but without assuming that $Y=0$ in $H_{2n-2q}(V,\bfZ)$. Let $\psi$ be a residue operator for $Y$ and $\theta=\tau^{*}(\psi)=\widetilde{\omega}_{*}\pi^{*}\psi$. The above argument generalizes to prove : 
\begin{equation}
\theta=\tau^{*}(\psi)\text{~ \em is a residue operator for~ } D(Y).\label{art08-sec5-eq5.23}
\end{equation}

We have now proved \eqref{art08-sec5-eq5.20}, and with it have proved \eqref{art08-sec5-eq5.2}, since $D(Y)$ will be algebraically equivalent to zero on $\Sigma$ because of the existence of an integral of the $3^{\text{rd}}$ kind associated to $D(Y)$.
\end{remark*}

\noindent
{\bf Proof of \eqref{art08-sec5-eq5.12}.}~ Let $Y\subset V$ be as above and interchange the roles of $Y$ and $\bfW$ in the statement of \eqref{art08-sec5-eq5.12}. Let $\omega\in H^{q,q}(V)$ be the dual of $Y\in H_{2n-2q}(V,\bfZ)$ and let $\psi$ be a residue operator for $Y$. Then (c.f. \eqref{art08-sec5-eq5.23} above) $\tau^{*}\psi=\theta$ is a residue operator for $D(Y)\subset \Sigma$, and so (c.f. Appendix to \S\ref{art08-sec4}, section (e)) $\overline{\partial}\theta$ is the dual of $D(Y)\in H_{2N-2}(\Sigma,\bfZ)(N=\dim \Sigma)$. But $\overline{\partial}\theta=\tau^{*}\overline{\partial}\psi=\tau^{*}\omega$, and so \eqref{art08-sec5-eq5.12} is proved.\hfill$\square$

\medskip
(b)~ \textsc{Reciprocity Relations in Higher Codimension.} Let $Y=Z-\bfZ$ be as in beginning of \S\ref{art08-sec5}, section (a) above. We assume that $Y=0$ in $H_{2n-2q}(V,\bfZ)$ so that $D(Y)$ is algebraically equivalent to zero on $\Sigma=\Sigma(\bfW)$. Let $\psi$ be a residue operator for $Y$ and $\theta=\tau^{*}\psi$ be\pageoriginale defined by \eqref{art08-sec5-eq5.21}. Then (c.f. \eqref{art08-sec5-eq5.20}) $\theta$ is an integral of the $3^{\text{rd}}$ kind on $\sigma$ whose logarithmic residue locus in $D(Y)$.

Now $\psi$ is determined up to $S=H^{2p-1,0}(V)+\cdots+H^{p,p-1}(V)$. Since $\tau^{*}(H^{p+r,p-1-r}(V))=0$ for $r>0$, $\theta$ is determined up to $\tau^{*}(S)$ where only $\tau^{*}(H^{p,p-1}(V))\subset H^{1,0}(\Sigma)$ (c.f. \eqref{art08-sec5-eq5.11}) counts. Let us prove now :
\begin{equation}
\begin{array}{l}
D(Y)\equiv 0\text{~\em on~} \Sigma \text{~\em if, and only, if, there exists~}\omega\in H^{1,0}(\Sigma)\\
\text{\em such that~}\int_{\delta}\theta+\omega\equiv 0(1)\text{~\em for all~}\delta\in H_{1}(\Sigma-D(Y),\bfZ).
\end{array}\label{art08-sec5-eq5.24}
\end{equation}

\begin{proof}
If $\omega$ exists satisfying $\int_{\delta}\theta+\omega\equiv 0(1)$ for all $\delta\in H_{1}(\Sigma-D(Y),\bfZ)$, then we may set :
\begin{equation}
f(W)=\exp \left(\int\limits^{W}_{\bfW}\theta+\omega\right),\quad (\exp \xi=e^{2\pi i\xi}).\label{art08-sec5-eq5.25}
\end{equation}
This $f(W)$ is a single-valued meromorphic function and, by \eqref{art08-sec5-eq5.20}, $(f)=D(Y)$.

Conversely, assume that $D(Y)=(f)$. Then $\theta-\dfrac{1}{2\pi i}\dfrac{df}{f}=-\omega$ will be a holomorphic 1-form in $H^{1,0}(\Sigma)$, and for $\delta\in H_{1}(\Sigma-D(Y),\bfZ)$, $\int\limits_{\gamma}\theta+\omega=\dfrac{1}{2\pi i}\int\limits_{\gamma}\dfrac{df}{f}=\dfrac{1}{2\pi i}\int\limits_{\gamma}d\log f\equiv 0(1)$. This proves \eqref{art08-sec5-eq5.24}.

Suppose we can prove :
\begin{equation}
\begin{array}{l}
\text{\em There exists $\eta\in S$ such that $\int_{\Gamma}\psi+\eta\equiv 0(1)$ for all}\\[2pt]
\Gamma\in H_{2q-1}(V-Y,\bfZ)\text{~\em if and only if,~}\phi_{q}(Y)=0\text{~\em in~} T_{q}(\Phi,V)\subset T_{q}(V).
\end{array}\label{art08-sec5-eq5.26}
\end{equation}
\end{proof}

Then we can prove the Abel's theorem \eqref{art08-sec5-eq5.3} as follows.

\begin{proof}
If $\phi_{q}(Y)=0$ in $T_{q}(\Phi,V)$, then by \eqref{art08-sec5-eq5.26} we may find $\eta\in S$ such that $\int_{\Gamma}\psi+\eta\equiv 0(1)$ for all $\Gamma\in H_{2q-1}(V-Y,\bfZ)$. Set $\omega=\tau^{*}\eta\in H^{1,0}(\Sigma)$. Then, for $\delta\in H_{1}(\Sigma-D(Y),\bfZ)$, $\int_{\delta}\theta+\omega=\int_{\delta}\tau^{*}(\psi+\eta)=\int_{\tau(\delta)}\psi+\eta\equiv 0(1)$, where $\tau$ is given by \eqref{art08-sec5-eq5.9}. Using \eqref{art08-sec5-eq5.24}, we have proved \eqref{art08-sec5-eq5.2}.
\end{proof}

\setcounter{theorem}{26}
\begin{remark}\label{art08-sec5-rem5.27}
The converse to Abel's theorem \eqref{art08-sec5-eq5.2}, which reads :
\setcounter{equation}{27}
\begin{equation}
\phi_{q}(Y)=0\text{~ in~ } T_{q}(Y)\text{~ if~ } D(Y)\equiv 0\text{~ in~ }\Sigma,\label{art08-sec5-eq5.28}
\end{equation}
will\pageoriginale be true, up to isogeny, if we have :%page 141
\end{remark}
