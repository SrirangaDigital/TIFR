\title{SOME RESULTS ON ALGEBRAIC CYCLES ON ALGEBRAIC MANIFOLDS}
\markright{Some Results on Algebraic Cycles on Algebraic Manifolds}

\author{By~~ Phillip A. Griffiths}

\date{}

\maketitle

\setcounter{pageoriginal}{92}
\setcounter{section}{-1}
\section{Introduction}\label{art08-sec0}\pageoriginale

The basic problem we have in mind is the classification of the algebraic cycles on a algebraic manifold $V$. The first invariant is the homology class $[Z]$ of a cycle $Z$ on $V$; if $Z$ has codimension $q$, then $[Z]\in H_{2n-2q}(V,\bfZ)(n=\dim V)$. By analogy with divisors (c.f. \cite{art08-cite18}), and following Weil \cite{art08-key22}, if $[Z]=0$, then we want to associate to $Z$ a point $\phi_{q}(Z)$ in a complex torus $T_{q}(V)$ naturally associated with $V$. The classification question then becomes two problems :
\begin{itemize}
\item[(a)] Find the image of $\phi_{q}$ (inversion theorem);

\item[(b)] Find the equivalence relation given by $\phi_{q}$ (Abel's theorem).
\end{itemize}

We are unable to make substantial progress on either of these. On the positive side, our results do cover the foundational aspects of the problem and give some new methods for studying subvarieties of general codimension. In particular, the issue is hopefully clarified to the extent that we can make a guess as to what the answers to (a) and (b) should be. This supposed solution is a consequence of the (rational) Hodge conjecture; conversely, if we know (a) and (b) in suitable form, then we can construct algebraic cycles.

We now give an outline of our results and methods.

For the study of $q$-codimensional cycles on $V$, Weil introduced certain complex tori $J_{q}(V)$; as a real torus,
$$
J_{q}(V)=H^{2q-1}(V,\bfR)/H^{2q-1}(V,\bfZ).
$$
These tori are abelian varieties. We use the same real torus, but with a different complex structure (c.d. \S\S1,2); these tori $T_{q}(V)$ vary holomorphically with $V$ (the $J_{q}(V)$ don't) and have the necessary functorial properties. In general, they are not abelian varieties, but have an $r$-convex polarization \cite{art08-key9}. However, the polarizing line bundle is positive on the ``essential part'' of $T_{q}(V)$.\pageoriginale Also, $T_{1}(V)=J_{1}(V)$ (= Picard variety of $V$) and $T_{n}(V)=J_{n}(V)$ (= Albanese variety of $V$).

Let $\Sigma_{q}$ be the cycles of codimension $q$ algebraically equivalent to zero on $V$. There is defined a homomorphism $\phi_{q}:\Sigma_{q}\to T_{q}(V)$ by $\phi_{q}(Z)=\left[\begin{smallmatrix}:\\ \int\limits_{\Gamma}\omega^{\alpha}\\ :\end{smallmatrix}\right]$/(periods), where $\Gamma$ is a $2n-2q+1$ chain with $\partial \Gamma=Z$ and $\omega^{1},\ldots,\omega^{m}\in H^{2n-2q+1}(V,\bfC)$ are a basis for the holomorphic one-forms on $T_{q}(V)$. Using the torus $T_{q}(V)$, this mapping is holomorhic and depends only on the complex structure of $V$ (c.f. \S3); this latter result follows from a somewhat interesting theorem on the cohomology of K\"ahler manifolds given in the Appendix following \S10. In \S3, we also give the infinitesimal calculation of $\phi$; the transposed differential $\phi^{*}$ is essetially the Poincar\'e residue operator (c.f. (3.8)). For hypersurfaces $(q=1)$, the Poincar\'e residue and geometric residue operators coincide, and the (well-known) solutions to (a) and (b) follow easily.

In \S4, we relate the functorial properties of the tori $T_{q}(V)$ to geometric operations on cycles. The expected theorems turn up, but the proofs require some effort. We use the calculus of differential forms with singularities. In particular, the notion of a residue operator associated to an irreducible subvariety $Z\subset V$ appears. Such a residue operator is given by a $C^{\infty}$ form $\psi$ on $V-Z$ such that: (1) $\psi$ is of type $(2q-1,0)+\cdots+(q,q-1)$; (2) $\partial \psi=0$ and $\overline{\partial}\psi$ is a $C^{\infty}$ $(q,q)$ form on $V$ which gives the Poincar\'e dual $\mathscr{D}[Z]\in H^{q,q}(V)$ of $[Z]$; and (3) for $\Gamma$ a $2n-k$ chain on $V$ meeting $Z$ transversely and $\eta$ a smooth $2q-k$ form on $V$, we have the residue formula: $\lim \int\limits_{\epsilon\to 0\Gamma\cdot (\partial T_{\epsilon})}\psi \wedge \eta=\int\limits_{\Gamma\cdot Z}\eta$, where $T_{\epsilon}$ is the $\epsilon$-neighborhood of $Z$ in $V$. The construction of residue operators is done using Hermitian differential geometry; the techniques involved give a different method of approaching the theorem of Bott-Chern \cite{art08-key4}. One use of the residue operators is the explicit construction, on the form level, of the Gysin homomorphism $i_{*}:H^{k}(Z)\to H^{2q+k}(V)$ where we can keep close track of the complex\pageoriginale structure (c.f. the Appendix to \S4, section (e)). This is useful in proving the functorial properties.

In \S5 we give one of our basic constructions. If $[Z]=0$ in $H_{2n-2q}(V,\bfZ)$, and if $\psi$ is a residue operator for $Z$, we may assume that $d\psi=0$. Then $\psi$ is the general codimensional analogue of a logarithmic integral of the third kind (\cite{art08-key17}). The trouble is that $\psi$ has degree $2q-1$ and so cannot directly be integrated on $V$ to give a function. However, $\psi$ can be integrated on the set of algebraic cycles of dimension $q-1$ on $V$. We show then that $Z$ defines a divisor $D(Z)$ on a suitable Chow variety associated to $V$, and that $\psi$ induces an integral of the third kind on this Chow variety. The generalization of Abel's theorem we give is then : $D(Z)$ is linearly equivalent to zero if $\phi_{q}(Z)=0$ in $T_{q}(V)$. As in the classical case, the proof involves a bilinear relation between $\psi$ and the holomorphic differentials on $T_{q}(V)$. Also, as mentioned above, the ``only if'' part of this statement (which is trivial when $q=1$) depends upon the Hodge problem. Our conclusion from this, as regards problem (b) is: The equivalence relation defined by $\phi$ should be linear equivalence on a suitable Chow variety. In particular, we don't see that this equivalence should necessarily be rational equivalence on $V$.

In \S6 we give our main result trying to determine the image of $\phi$. To explain this formula (given by (6.8) in \S6) we let $\{\bfE_{\lambda}\}$ be a holomorphic family of holomorphic vector bundles over $v$. We denote by $Z_{q}(\bfE_{\lambda})$ the $q^{\text{th}}$ Chern class in the rational equivalence ring, so that $\{Z_{q}(\bfE_{\lambda})\}$ gives a family of codimension $q$ cycles on $V$. Our formula gives a method for calculating the infinitesimal variation of $Z_{q}(\bfE_{\lambda})$ in $T_{q}(V)$; it involves the curvature matrix $\Phi$ in $\bfE_{\lambda}$ and the Kodaira-Spencer class giving the variation of $\bfE_{\lambda}$.

The crux of this formula is that it relates the Poincar\'e and geometric residues in higher codimension. The proof involves a somewhat delicate computation using forms with singularities and the curvature in $\bfE_{\lambda}$. In \S8 we give the argument for the highest Chern class of an ample bundle. In \S7 it is shown that we need only check the theorem for ample bundles; however, in general the Chern classes, given by Schubert cycles, will be singular, except of course for\pageoriginale the highest one. So, to prove our formula in general we give in \S9 an argument, which is basically differential-geometric, but which requires that we examine the singularities of $Z_{q}(\bfE_{\lambda})$.

The reason for proving such a formula is that the Chern classes $Z_{q}(\bfE_{\lambda})$ generate the rational equivalence ring on $V$. So, if we could effectively use the main result, we could settle problem (a). For example, for line bundles $(q=1)$, the mapping in question is the identity; this gives once more the structure theorems of the Picard variety. However, we are unable to make effective use of the formula, except in rather trivial cases, so that our result has more of an intrinsic interest and illuminating proof than the applications we would like.

In the last part of \S9 we give an integral-geometric argument, using the transformation properties of the tori $T_{q}(V)$ and the relation of these properties to cycles, of the main formula (6.8).

Finally, in \S10 we attempt to put the problem in perspective. We formulate possible answers to (a) and (b) and show how these would follow if we knew the Hodge problem. The construction of algebraic cycles, assuming the answer to (a) and (b), is based on a generalization of the Poincar\'e normal functions (c.f. \cite{art08-key19}) and will be given later.

To close this introduction, I would like to call attention to the paper of David Lieberman \cite{art08-key20} on the same subject and which contains several of the results given below. Lieberman uses the Weil Jacobians \cite{art08-key22} to study intermediate cycles; however, his results are equally valid for the complex tori we consider. His methods are somewhat different from the ones used below; many of our arguments are computational whereas Lieberman uses functorial properties of the Weil mapping and his proofs have an algebro-geometric flavor.

More specifically, Lieberman proves the functorial properties of the Weil mapping in somewhat more precise form than given below. Thus his results include the functorial properties (4.2) (the hard one arising from the Gysin map) and (4.14) (the easy one using restriction of cohomology), as well as (4.12) which we only state conjecturally. From\pageoriginale the functorial properties and the fact that the Weil mapping is holomorphic for codimension one, Lieberman concludes the analyticity of this mapping (given by (3.2)) in general. (It is interesting to contrast his conceptual argument with the computational one given in \cite{art08-key9}.) In summary, Lieberman's results include the important general properties of the intermediate Jacobians given in \S1-4 below. Also, the conjectured Abelian variety for which the inversion theorem ((a) above) holds was found by Lieberman using his Poincar\'e divisor, and the proof of (10.4) is due to Lieberman.

The reason for this overlap is because this manuscript was done in Berkeley, independently but at a later time than Lieberman (most of his results are in his M. I. T. thesis). By the time we talked in Princeton, this paper was more or less in the present form and, because of the deadline for these proceedings, could not be rewritten so as to avoid duplication.


\tableofcontents

\section{Complex Tori associated to Algebraic Manifolds}\label{art08-sec1}\pageoriginale

Let $V$ be an $n$-dimensional algebraic manifold and $\bfL\to V$ the {\em positive line bundle} giving the polarization on $V$. The {\em characteristic class} $\omega \in H^{1,1}(V)\cap H^{2}(V,\bfZ)$ may be locally written as $\omega=\dfrac{i}{2}\{\Sigma g_{\alpha\overline{\beta}}dz^{\alpha}\wedge d\overline{z}^{\beta}\}$ where\pageoriginale $\sum\limits_{\alpha,\beta}g_{\alpha\overline{\beta}}dz^{\alpha}d\overline{z}^{\beta}$ gives a K\"ahler metric on $V$.

According to Hodge, the cohomology group $H^{s}(V,\bfC)$ decomposes as a sum:
$$
H^{s}(V,\bfC)=\sum\limits_{p+q=s}H^{p,q}(V),
$$
where $H^{p,q}(V)$ are the cohomology classes represented by differential forms of type $(p,q)$. Under complex conjugation, $\overline{H^{p,q}(V)}=H^{q,p}(V)$. 

Consider now the cohomology group
\begin{equation}
H^{2n-2q+1}(V,\bfC)=\sum\limits^{2n-2q+1}_{r=0}H^{2n-2q+1-r,r}(V)\label{art08-sec1-eq1.1}
\end{equation}
and choose a complex subspace $S\subset H^{2n-2q+1}(V,\bfC)$ such that
\begin{equation}
S\cap \overline{S}=0\text{~~ and ~~} S+\overline{S}=H^{2n-2q+1}(V,\bfC);\label{art08-sec1-eq1.2}
\end{equation}
\begin{equation}
S=\sum\limits^{2n-2q+1}_{r=0}S\cap H^{2n-2q+1-r,r}(V)\label{art08-sec1-eq1.3}
\end{equation}
(i.e. $S$ is compatible with the {\em Hodge decomposition} \eqref{art08-sec1-eq1.1});
\begin{equation}
H^{n-q+1,n-q}(V)\subset S.\label{art08-sec1-eq1.4}
\end{equation}
Under these conditions we shall define a complex torus $T_{q}(S)$ such that the space of holomorphic $1$-forms on $T_{q}(S)$ is just $S$. There are three equivalent definitions of $T_{q}(S)$.

\begin{definition}\label{art08-sec1-defi1}
Choose a basis $\omega^{1},\ldots,\omega^{m}$ for $S$ and define the lattice $\Gamma(S)\subset \bfC^{m}$ of all column vectors $\pi_{\gamma}=\left[\begin{smallmatrix} \int\limits_{\gamma}\omega^{1}\\ \vdots\\ \int\limits_{\gamma}\omega^{m}\end{smallmatrix}\right]$ where $\gamma\in H_{2m-2q+1}(V,\bfZ)$. To see that $\Gamma(S)$ is in fact a lattice, we observe tht rank $(H_{2n-2q+1}(V,\bfZ))=2m$ and so we must show :
\end{definition}

If $\gamma_{1},\ldots,\gamma_{k}\in H_{2n-2q+1}(V,\bfZ)$ are linearly independent over $\bfR$, then $\pi_{\gamma_{1}},\ldots,\pi_{\gamma_{k}}$ are also linearly independent over $\bfR$. But if 
$$
\sum\limits^{k}_{j=1}\alpha_{j}\int\limits_{\gamma_{j}}\omega^{\alpha}=\int\limits_{\sum\limits^{k}_{j=1}\alpha_{j}\gamma_{j}}
$$\pageoriginale
then also
$$
\int\limits_{\sum\limits^{k}_{j=1}\alpha_{j}\gamma_{j}}\overline{\omega}^{\alpha}=0,\quad\text{since}\quad \overline{\alpha}_{j}=\alpha_{j}.
$$
This says that $\sum\limits^{k}_{j=1}\alpha_{j}\gamma_{j}$ is orthogonal to $S+\overline{S}=H^{2n-2q+1}(V,\bfC)$ and so $\sum\limits^{k}_{j=1}\alpha_{j}\gamma_{j}=0$.

If then $T_{q}(S)=\bfC^{m}/\Gamma(S)$, then $T_{q}(S)$ is a complex torus associated to $S\subset H^{2n-2q+1}(V,\bfC)$.

\begin{definition}\label{art08-sec1-defi2}
Let $H_{2n-2q+1}(V,\bfC)=H^{2n-2q+1}(V,\bfC)^{*}$ be the dual space of $H^{2n-2q+1}(V,\bfC)$ so that $0\to S\to H^{2n-2q+1}(V,\bfC)$ dualizes to 
\begin{equation}
0\leftarrow S^{*}\leftarrow H_{2n-2q+1}(V,\bfC).\label{art08-sec1-eq1.5}
\end{equation}
Then $H_{2n-2q+1}(V,\bfZ)\subset H_{2n-2q+1}(V,\bfC)$ projects onto a lattice $\Gamma(S)\subset S^{*}$ and $T_{q}(S)=S^{*}/\Gamma(S)$. (Proof that Definition \ref{art08-sec1-defi1} = \ref{art08-sec1-defi2}: choosing a basis $\omega^{1},\ldots,\omega^{m}$ for $S$ makes $S^{*}\simeq \bfC^{m}$ by $l(\omega^{\alpha})=l_{\alpha}$ where $=\left[\begin{smallmatrix} l_{1}\\ \vdots \\ l_{m}\end{smallmatrix}\right]$. Then $\pi_{\gamma}(\omega^{\alpha})=\int\limits_{\gamma}\omega^{\alpha}=\langle \omega^{\alpha},\gamma\rangle$ so that $\Gamma(S)$ is the same lattice in both cases.)
\end{definition}

\begin{definition}\label{art08-sec1-defi3}
Let $\mathscr{D}:H_{2n-2q+1}(V,\bfC)\to H^{2q-1}(V,\bfC)$ be the {\em Poincar\'e duality isomorphism} and $0\leftarrow S^{*}\leftarrow H^{2q-1}(V,\bfC)$ the sequence corresponding to \eqref{art08-sec1-eq1.5}, $\Gamma(S)\subset S^{*}$ the lattice corresponding to $\Gamma(S)$. Then $T_{q}(S)=S^{*}/\Gamma(S)$.
\end{definition}

Observe that if $H^{r,s}(V)\subset S$, then $H^{n-r,n-s}(V)\subset S^{*}$ and vice-versa. In particular, $S$ is the dual space of $S^{*}$ by:
\begin{equation}
\langle \omega,\phi\rangle =\int\limits_{V}\omega\wedge \phi\qquad (\omega\in S, \ \ \phi\in S^{*}).\label{art08-sec1-eq1.6}
\end{equation}
Thus\pageoriginale $S\cong H^{1,0}(T_{q}(S))$, the space of holomorphic 1-forms on $T_{q}(S)$.

\section{Special Complex Tori}\label{art08-sec2}

The choice of $S\subset H^{2n-2q+1}(V,\bfC)$ depends on the properties we want $T_{q}(S)$ to have; the results on algebraic cycles will be essentially independent of $S$ because of condition \eqref{art08-sec1-eq1.4}.

\setcounter{definition}{0}
\begin{example}\label{art08-sec2-exam1}
We let $S=\sum\limits^{n-q}_{r=0}H^{n-q+r+1,n-q-r}$ and set $T_{q}(S)=T_{q}(V)$. These tori have been studied in \cite{art08-key9}, where it is proved that $T_{q}(V)$ varies holomorphically with $V$.
\end{example}

The trouble with $T_{q}(V)$ is that it is not polarized in the usual sense; however, for our purposes we can do almost as well as follows. Recall \cite{art08-key23} that there is defined on $H^{2q-1}(V,\bfC)$ a quadratic form $Q$ with the following properties:
\setcounter{equation}{0}
\begin{equation}
\left.
\begin{array}{cl}
{\rm(a)} & Q \text{~ is skew-symmetric and integral on ~}\\
         & H^{2q-1}(V,\bfZ)\subset H^{2q-1}(V,\bfC);\\[4pt]
{\rm(b)} & Q(H^{r,s}(V),\overline{H^{r',s'}(V)})=0\text{~ if~ } r\neq r', s\neq s';\\[4pt]
{\rm(c)} & Q(H^{r,s}(V),\overline{H^{r,s}(V)})\text{~ is nonsingular; and}\\[4pt]
{\rm(d)} & iQ(H^{q-1,q}(V), \ \overline{H^{q-1,q}(V)})>0.
\end{array}
\right\}\label{art08-sec2-eq2.1}
\end{equation}
It follows that $Q(S^{*},S^{*})=0$ and that, choosing a basis $\omega^{1},\ldots,\omega^{m}$ for $S$, there is a complex line bundle $\bfL\to T_{q}(S)$ whose characteristic class $\omega(\bfL)\in H^{2}(T_{q}(S),\bfZ)\cap H^{1,1}(T_{q}(S))$ is given by
$$
\omega(\bfL)=\dfrac{i}{2\pi}\left\{\sum\limits^{m}_{\alpha,\beta=1}h_{\alpha\overline{\beta}}\omega^{\alpha}\wedge \overline{\omega}^{\beta}\right\},
$$
where the matrix $H=(h_{\alpha\overline{\beta}})=\{iQ(e_{\alpha},\overline{e}_{\beta})\}^{-1}$ and $\int\limits_{V}\omega^{\alpha}\wedge e_{\beta}=\delta^{\alpha}_{\beta}$. Diagonalizing $H$, we may write
\begin{equation}
\omega(\bfL)=\dfrac{i}{2\pi}\left\{\sum\limits^{m}_{\alpha=1}\epsilon_{\alpha}\omega^{\alpha}\wedge \overline{\omega}^{\alpha}\right\},\label{art08-sec2-eq2.2}
\end{equation}
where $\epsilon_{\alpha}=\pm 1$ and $\epsilon_{\alpha}=+1$ if $\omega^{\alpha}\in H^{n-q+1,n-q}(V)$. Thus we may say that :
\begin{equation}
\text{There is a natural } r\text{-{\em convex polarization} \cite{art08-key10}~~ } \bfL\to T_{q}(V)\label{art08-sec2-eq2.3}
\end{equation}\pageoriginale
($r$ = number of $\alpha$ such that $\epsilon_{\alpha}=-1$) and the characteristic class of $\bfL$ is positive on the translates of $H^{q-1,q}(V)$.

\begin{example}
We let $S=\sum\limits_{r} H^{n-q+2r+1,n-q-2r}$ and set $J_{q}(V)=T_{q}(S)$. This torus is Weil's {\em intermediate Jacobian} \cite{art08-key22} and from \eqref{art08-sec2-eq2.1} we find :
\end{example}

There is a natural $0$-{\em convex polarization} (= {\em positive line bundle})
\begin{equation}
\bfK\to J_{q}(V).\label{art08-sec2-eq2.4}
\end{equation}

Referring to \eqref{art08-sec2-eq2.2}, we let $\phi^{\alpha}=\omega^{\alpha}$ if $\epsilon_{\alpha}=+1$, $\phi^{\alpha}=\overline{\omega}^{\alpha}$ if $\epsilon_{\alpha}=-1$. Then the $\phi^{\alpha}$ give a basis for $H^{1,0}(J_{q}(V))$ and
\begin{equation}
\omega (\bfK)=\dfrac{i}{2\pi}\left\{\sum\limits^{m}_{\alpha=1}\phi^{\alpha}\wedge \overline{\phi}^{\alpha}\right\}.\label{art08-sec2-eq2.5}
\end{equation}

We recall \cite{art08-cite23} that $H^{s}(J_{q}(V),\mathscr{O}(\bfK^{\mu}))=0$ for $\mu>0$, $s>0$ and that $H^{0}(J_{q}(V),\mathscr{O}(\bfK^{\mu}))$ has a basis $\theta_{0},\ldots,\theta_{N}$ given by {\em theta functions of weight $\mu$.}

\medskip
\noindent
{\bf Comparison of $T_{q}(V)$ and $J_{q}(V)$.}~ In \cite{art08-key9} it is proved that there is a {\em real} linear isomorphism $\xi:T_{q}(V)\to J_{q}(V)$ such that
\begin{equation}
\left.
\begin{array}[l]{@{}rl}
{\rm(i)} & \xi^{*}\phi^{\alpha}=\omega^{\alpha}\text{~ if~ } \epsilon_{\alpha}=+1\text{~ and~ } \xi^{*}\phi^{\alpha}=\overline{\omega}^{\alpha}\text{~ if~ } \epsilon_{\alpha}=-1;\\
{\rm(ii)} & \xi^{*}(\bfK)=\bfL;~ \text{and}\\
{\rm(iii)} & \text{if~~} \Omega_{p}=\xi^{*}(\theta_{p})\left\{\prod\limits_{\epsilon_{\alpha}=-1}\right\}, \text{~ then the } \Omega_{p}\text{~ give a basis of}\\
 & H^{r}(T_{q}(V),\mathscr{O}(\bfL^{\mu})),\text{~~ and~~ }H^{s}(T_{q}(V),\mathscr{O}(\bfL^{\mu}))=0\text{~~ for}\\
& \mu>0, s\neq r.
\end{array}
\right\}\label{art08-sec2-eq2.6}
\end{equation}

\noindent
{\bf Some Special Cases.}~ For $q=1$, $T_{1}(V)=J_{1}(V)=H^{0,1}(V)/H^{1}(V,\bfZ)$ is the {\em Picard variety} of $V$ \cite{art08-key22}. For $q=n$, $T_{n}(V)=J_{n}(V)=H^{n-1,n}(V)/H^{2n-1}(V,\bfZ)$ is the {\em Albanese variety} \cite{art08-key3} of $V$. For $q=2$, $T_{2}(V)=H^{1,2}(V)+H^{0,3}(V)/H^{3}(V,\bfZ)$ and $J_{2}(V)=H^{1,2}(V)+H^{3,0}(V)/H^{3}(V,\bfZ)$; this is the simplest case where $T_{q}(V)\neq J_{q}(V)$.

\medskip
\noindent
{\bf Some Isogeny Properties.}~ We let $S_{q}\subset H^{2n-2q+1}(V,\bfC)$ be the subspace corresponding to either $T_{q}(V)$ or $J_{q}(V)$ constructed above, and we let $S^{*}_{q}\subset H^{2q-1}(V,\bfC)$ be the dual space. Then we have 
\[
\left.
\begin{array}{c}
\xymatrix{
H^{2q-1}(V,\bfC)\ar[r] & S^{*}_{q}\ar[r] & 0\\
H^{2q-1}(V,\bfZ)\ar[u] & &}
\end{array}\right\}
\]\pageoriginale
and $T_{q}(V)$ or $J_{q}(V)$ is given as $S^{*}_{q}/\Gamma^{*}_{q}$ where $\Gamma^{*}_{q}$ is the projection of $H^{2q-1}(V,\bfZ)$ on $S^{*}_{q}$.

Suppose now that $\psi\in H^{p,p}(V)\cap H^{2p}(V,\bfZ)$. Then, by cup-product, we have induced~:
\begin{equation}
\vcenter{
\xymatrix@=1.2cm{
S^{*}_{q}\ar[r]^-{\psi} & S^{*}_{p+q}\\
\Gamma^{*}_{q}\ar[r]^-{\psi}\ar[u] & \Gamma^{*}_{p+q}\ar[u]
}}\label{art08-sec2-eq2.7}
\end{equation}
which gives $\psi:T_{q}(V)\to T_{p+q}(V)$ or $\psi:J_{q}(V)\to J_{p+q}(V)$. We want to give this mapping in terms of the coordinates given in the first definition of paragraph 1.

Let $\omega^{1},\ldots,\omega^{m}=\{\omega^{\alpha}\}$ be a basis for $S_{q}\subset H^{2n-2q+1}(V,\bfC)$ and $\phi^{1},\ldots,\phi^{k}=\{\phi^{\rho}\}$ be a basis for $S_{p+q}\subset H^{2n-2p-2q+1}(V,\bfC)$. Then $\psi\wedge \phi^{\rho}=\sum\limits_{\alpha}m_{\rho\alpha}\omega^{\alpha}$ and 
\begin{equation}
\int\limits_{\gamma}\psi \wedge \phi^{\rho}=\int\limits_{\gamma\cdot \mathscr{D}(\psi)}\phi^{\rho},\label{art08-sec2-eq2.8}
\end{equation}
where $\mathscr{D}(\psi)\in H_{2n-2p}(V,\bfZ)$ and $\gamma\in H_{2n-2q+1}(V,\bfZ)$. Now $M=(m_{\rho\alpha})$ is a $k\times m$ matrix giving $\psi:\bfC^{m}\to \bfC^{k}$ by $\psi \left(\begin{smallmatrix} :\\ \lambda^{\alpha} \\ :\end{smallmatrix}\right)=\left(\begin{smallmatrix} :\\ \sum\limits^{m}_{\alpha=1} m_{\rho\alpha}\lambda^{\alpha}\\ :\end{smallmatrix}\right)$
 and
$$
\psi\left(\begin{matrix} 
:\\ \int\limits_{\gamma}\omega^{\alpha}
\end{matrix}\right)
=
\left(\begin{matrix}
:\\
\Sigma m_{\rho\alpha}\int\limits_{\gamma}\omega^{\alpha}\\
:
\end{matrix}\right)
=
\left(\begin{matrix}
:\\
\int\limits_{\gamma}\psi \wedge \phi^{\rho}\\
:
\end{matrix}\right)
=
\left(\begin{matrix}
:\\
\int\limits_{\gamma\cdot \mathscr{D}(\psi)} \phi^{\rho}\\
:
\end{matrix}\right),
$$
so that $\phi(\Gamma_{q})\subset \Gamma_{p+q}$. It follows that, in terms of the coordinates in Definition \ref{art08-sec1-defi1}, $\psi$ is given by the matrix $M$.

Now\pageoriginale suppose that $\psi:H^{2q-1}(V,\bfC)\to H^{2p+2q-1}(V,\bfC)$ is an isomorphism. Then $\psi:S^{*}_{q}\cong S^{*}_{p+q}$ and $\psi(\Gamma^{*}_{q})$ is of finite index in $\Gamma^{*}_{p+q}$. Thus $\psi:T_{q}(V)\to T_{p+q}(V)$ is an {\em isogeny}, as is also $\psi:J_{q}(V)\to J_{p+q}(V)$. Taking $\psi=\omega^{n-2q+1}$, where $\omega$ is the polarizing class, and using \cite{art08-key23}, page 75, we have :
\begin{equation}
\left.
\begin{array}{c}
\omega^{n-2q+1}:T_{q}(V)\to T_{n-q+1}(V),\text{~~ and}\\[3pt]
\omega^{n-2q+1} : J_{q}(V)\to J_{n-q+1}(V)
\end{array}\right\}\label{art08-sec2-eq2.9}
\end{equation}
are both isogenies for $q\leq \left[\dfrac{n+1}{2}\right]$.

Finally, using \cite{art08-key23}, Chapter IV, we have :

For $p\leq n-2q+1$, the mappings
\begin{equation}
\left.
\begin{array}{c}
\omega^{p}:T_{q}(V)\to T_{p+q}(V),\text{~~ and}\\[4pt]
\omega^{p}: J_{q}(V)\to J_{p+q}(V)
\end{array}
\right\}\label{art08-sec2-eq2.10}
\end{equation}
make $T_{q}(V)$ isogenous to a sub-torus of $T_{p+q}(V)$, and similarly for $J_{q}(V)$ and $J_{p+q}(V)$.

\medskip
\noindent
{\bf Some Functionality Properties.}~ Given a holomorphic mapping $f:V'\to V$, there is induced a cohomology mapping $f^{*}:H^{2q-1}(V,\bfC)\to H^{2q-1}(V',\bfC)$ with $f^{*}(S^{*}_{q}(V))\subset S^{*}_{q}(V'),f^{*}(\Gamma^{*}_{q}(V))\subset \Gamma^{*}_{q}(V')$ (using the obvious notation).

This gives
\begin{equation}
\left.
\begin{array}{c}
f^{*}:T_{q}(V)\to T_{q}(V'),\text{~~ and}\\[4pt]
f^{*}:J_{q}(V)\to J_{q}(V').
\end{array}
\right\}\label{art08-sec2-eq2.11}
\end{equation}

On the other hand, if $\dim V=n$ and $\dim V'=n'$, we set $k=n-n'$ and from $f_{*}:H_{2n'-2q+1}(V',\bfC)\to H_{2n-2(k+q)+1}(V,\bfC)$ we find a mapping
\begin{equation}
\left.
\begin{array}{c}
f_{*}:T_{q}(V')\to T_{q+k}(V),\text{~~ and}\\[4pt]
f_{*}:J_{q}(V')\to J_{q+k}(V).
\end{array}
\right\}\label{art08-sec2-eq2.12}
\end{equation}

Suppose now that $f:V'\to V$ is an embedding so that $V'$ is an algebraic submanifold of $V$. Then $V'$ defines a class $[V']\in H_{2n-2k}(V,\bfZ)$\pageoriginale and $\mathscr{D}[V']=\Psi\in H^{2k}(V,\bfZ)\cap H^{k,k}(V)$. We assert that :

In \eqref{art08-sec2-eq2.11} and \eqref{art08-sec2-eq2.12}, the composite mapping
\begin{equation}
\begin{array}{p{8cm}}
$f_{*}f^{*}:T_{q}(V)\to T_{q+k}(V)\text{~~ is just~~ }\Psi:T_{q}(V)\to T_{q+k}(V)\text{~~ as given by \eqref{art08-sec2-eq2.7} (and similarly for~~ $J_{q}(V)$)}$
\end{array}\label{art08-sec2-eq2.13}
\end{equation}

\begin{proof}
We have to show that the composite
\begin{equation}
H^{2q-1}(V,\bfC)\xrightarrow{f^{*}}H^{2q-1}(V',\bfC)\xrightarrow{f^{*}}H^{2q+2k-1}(V,\bfC)\label{art08-sec2-eq2.14}
\end{equation}
is cup product with $\Psi$. In homology \eqref{art08-sec2-eq2.14} dualizes to 
\begin{equation}
H_{2q-1}(V,\bfC)\xleftarrow{f_{*}}H_{2q-1}(V',\bfC)\xleftarrow{f^{*}}H_{2q+2k-1}(V,\bfC)\label{art08-sec2-eq2.15}
\end{equation}
where $f^{*}$ is defined by
\begin{equation}
\vcenter{\xymatrix@=1.2cm{
H_{2q+2k-1}(V,\bfC)\ar[d]^-{\mathscr{D}}\ar[r]^-{f^{*}} & H_{2q-1}(V',\bfC)\\
H^{2n-2q-2k+1}(V,\bfC)\ar[r]^-{f^{*}} & H^{2n-2k-2q+1}(V',\bfC)\ar[u]_{\mathscr{D}^{-1}}
}}\label{art08-sec2-eq2.16}
\end{equation}

If we can show that $f_{*}f^{*}(\gamma)=[V']\cdot \gamma$ for $\gamma \in H_{2q+2k-1}(V,\bfC)$, then $\int\limits_{\gamma}f_{*}f^{*}\phi=\int\limits_{f_{*}f^{*}\gamma}\phi=\int\limits_{[V']\cdot \gamma}\phi=\int\limits_{\gamma}\Psi \wedge \phi(\phi\in H^{2q-1}(V,\bfC))$, and we are done. So we must show that, in \eqref{art08-sec2-eq2.15}, $f^{*}$ is intersection with $V'$, and this a standard result on the {\em Gysin homomorphism} \eqref{art08-sec2-eq2.16} (c.f. (4.11) and the accompanying Remark).
\end{proof}

\section{Algebraic Cycles and Complex Tori}\label{art08-sec3}

Let $V=V_{n}$ be an algebraic manifold, $S\subset H^{2n-2q+1}(V,\bfC)$ a subspace satisfying \eqref{art08-sec1-eq1.2}-\eqref{art08-sec1-eq1.4}, and $T_{q}(S)$ the resulting complex torus. We choose a suitable basis $\omega^{1},\ldots,\omega^{m}$ for $S\cong H^{1,0}(T_{q}(S))$ and let $\Sigma_{q}$ = \{set of algebraic cycles $Z\subset V$ which are of codimension $q$ in $V$ and are homologous to zero\}. Following Weil \cite{art08-key22}, we define
\setcounter{equation}{0}
\begin{equation}
\phi_{q}:\Sigma_{q}\to T_{q}(S)\label{art08-sec3-eq3.1}
\end{equation}
as follows: if $Z\in \Sigma_{q}$, then $Z=\partial C_{2n-2q+1}$ for some $2n-2q+1$ chain $C$, and we set 
\begin{equation}
\phi_{q}(Z)=
\begin{bmatrix}
\vdots\\
\int\limits_{C}\omega^{\alpha}\\
\vdots
\end{bmatrix}.\label{art08-sec3-eq3.2}
\end{equation}\pageoriginale
Since $C$ is determined up to cycles, $\phi_{q}(Z)$ is determined up to vectors $\left[\begin{smallmatrix}\vdots\\ \int\limits_{\gamma}\omega^{\alpha}\\ \vdots\end{smallmatrix}\right](\gamma\in H_{2n-2q+1}(V,\bfZ))$, and so $\phi_{q}$ {\em is defined and depends on the subspace of the closed $C^{\infty}$ forms spanned by} $\omega^{1},\ldots,\omega^{m}$; this restriction will be removed in the Appendix to \S\ref{art08-sec3}.

Now, while it should be the case that $\phi_{q}$ is holomorphic, we shall be content with recalling from \cite{art08-key9} a special result along these lines. Consider on $V$ an analytic family $\{Z_{\lambda}\}_{\lambda\in \Delta}$($\Delta=\text{disc in~}\lambda\text{-plane}$) of $q$-codimensional algebraic subvarieties $Z_{\lambda}\subset V$. Locally on $V$, $\{Z_{\lambda}\}_{\lambda\in \Delta}$ is given by the vanishing of analytic functions $f_{1}(z^{1},\ldots,z^{n};\lambda),\ldots,f_{l}(z^{1},\ldots,z^{n};\lambda)$. We define $\phi:\Delta\to T_{q}(S)$ by $\phi(\lambda)=\phi_{q}(Z_{\lambda}-Z_{0})$. Using  \eqref{art08-sec1-eq1.4}, we have proved in \cite{art08-key9} that
\begin{equation}
\left.
\begin{array}{l}
\phi :\Delta\to T_{q}(S)\text{~~ is holomorphic and}\\[4pt]
\phi_{*}\{\bfT_{\lambda}(\Delta)\}\subset H^{q-1,q}(V).
\end{array}\right\}\label{art08-sec3-eq3.3}
\end{equation}

We may rephrase \eqref{art08-sec3-eq3.3} by saying that $\phi^{*}:S_{q}\to \bfT_{\lambda}(\Delta)^{*}$ is determined by $\phi^{*}|H^{n-q+1,n-q}(V)$ (c.f. \eqref{art08-sec1-eq1.4}).

\medskip
\noindent
{\bf Continuous Systems and The Infinitesimal Calculation of {\boldmath$\phi_{q}$.}}~ Suppose that the $Z_{\lambda}\subset V$ are all nonsingular and $Z=Z_{0}$. We let $\bfN\to Z$ be the {\em normal bundle} of $Z\subset V$, so that we have the exact sheaf sequence
\begin{equation}
0\to \mathscr{O}_{Z}(\bfN^{*})\to \Omega^{1}_{V|Z}\to \Omega^{1}_{Z}\to 0.\label{art08-sec3-eq3.4}
\end{equation}
Since $\dim Z=n-q$, from \eqref{art08-sec3-eq3.4} we have induced the {\em Poincar\'e residue operator}
\begin{equation}
\Omega^{n-q+1}_{V|Z}\to \Omega^{n-q}_{Z}(\bfN^{*})\to 0\label{art08-sec3-eq3.5}
\end{equation}
as follows : Let $\phi\in \Omega^{n-q+1}_{V|Z}$; $\tau_{1},\ldots,\tau_{n-q}$ be tangent vectors to $Z$; $\eta$ a normal vector to $Z$. Lift $\eta$ to a tangent vector $\widehat{\eta}$ on $V$ along $Z$.\pageoriginale Then $\langle \phi, \tau_{1}\wedge\ldots\wedge \tau_{n-q}\otimes \eta\rangle=\langle \phi,\tau_{1}\wedge\ldots\wedge \tau_{n-q}\wedge\widehat{\eta}\rangle$, where $\phi\in \Omega^{n-q+1}_{V|Z}$.

From \eqref{art08-sec3-eq3.5} and $\Omega^{n-q+1}_{V}\to \Omega^{n-q+1}_{V|Z}$, we have
\begin{equation}
H^{n-q}(V,\Omega^{n-q+1}_{V})\xrightarrow{\xi^{*}}H^{n-q}(Z,\Omega^{n-q}_{Z}(\bfN^{*})).\label{art08-sec3-eq3.6}
\end{equation}

On the other hand, in \cite{art08-key16} Kodaira has defined the {\em infinitesimal displacement mapping}
\begin{equation}
\rho : \bfT_{0}(\Delta)\to H^{0}(Z,\mathscr{O}_{Z}(\bfN)).\label{art08-sec3-eq3.7}
\end{equation}
To calculate $\phi^{*}$, we have shown in \cite{art08-key9} that the following diagram commutes:
\begin{equation}
\vcenter{\xymatrix@R=1.2cm{
 & H^{n-q+1,n-q}(V)=H^{n-q}(V,\Omega^{n-q+1}_{V})\ar[dd]^-{\xi^{*}}\ar[dl]_-{\phi^{*}}\\
\bfT_{0}(\Delta)^{*} & \\
& H^{n-q}(Z,\Omega^{n-q}_{Z}(\bfN^{*}))=H^{0}(Z,\mathscr{O}_{Z}(\bfN))^{*}.\ar[ul]_-{\rho^{*}}
}}\label{art08-sec3-eq3.8}
\end{equation}
In other words, infinitesimally $\phi$ is eseentially given by by $\xi^{*}$ in \eqref{art08-sec3-eq3.6}.

\medskip
\noindent
{\bf Some Special Cases.}
\begin{itemize}
\item[(i)] In case $q=n$, $Z$ is a finite set of points $z_{1},\ldots,z_{r}$ ($Z$ is a zero-cycle) and \eqref{art08-sec3-eq3.6} becomes:
\begin{equation}
H^{1,0}(V)\xrightarrow{\xi^{*}}\sum\limits^{r}_{j=1}\bfT_{z_{j}}(V)^{*}\label{art08-sec3-eq3.9}
\end{equation}
where $\xi^{*}(\omega)=\sum\limits^{r}_{j=1}$, $\omega\in H^{1,0}(V)$ being a holomorphic 1-form on $V$. In particular, $\phi^{*}$ is onto if $\xi^{*}$ is injective.

\item[(ii)] In case $q=1$, $Z\subset V$ is a nonsingular hypersurface. Then there is a holomorphic line bundle $\bfE\to V$ and a section $\sigma\in H^{0}(V,\mathscr{O}_{V}(\bfE))$ such that $Z=\{z\in V:\sigma(z)=0\}$. From the exact sheaf sequence $0\to \mathscr{O}_{V}\xrightarrow{\sigma}\mathscr{V}(\bfE)\to \mathscr{O}_{Z}(\bfN)\to 0$, we find
\begin{equation}
H^{0}(Z,\mathscr{O}_{Z}(\bfN))\xrightarrow{\xi}H^{1}(V,\mathscr{O}_{V}),\label{art08-sec3-eq3.10}
\end{equation}
where we claim that $\xi$ in \eqref{art08-sec3-eq3.10} is (up to a constant) the dual of $\xi^{*}$ in \eqref{art08-sec3-eq3.6} (using $H^{0,1}(V)=H^{n,n-1}(V)^{*}$).
\end{itemize}

\begin{proof}
We\pageoriginale % page 108
\end{proof}
