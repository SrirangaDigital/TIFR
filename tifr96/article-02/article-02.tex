\title{THE IMPLICIT FUNCTION THEOREM IN ALGEBRAIC GEOMETRY}
\markright{The Implicit Function Theorem in Algebraic Geometry}

\author{By~~ M. Artin\footnote{Sloan foundation fellow.}}

\date{}

\maketitle

\setcounter{pageoriginal}{12}
Several\pageoriginale years ago, Matsusaka introduced the concept of $Q$-{\em variety} in order to study equivalence relations on algebraic varieties. $Q$-{\em varieties} are essentially quotients of algebraic varieties by algebraic equivalence relations. The theory of these structures is developed in Matsusaka \cite{art02-key24}. In this paper, we discuss a special case of Matsusaka's notion in the context of arbitrary schemes. We call the structure {\em scheme for the etale topology,} or {\em algebraic space.} One obtains an algebraic space from affine schemes via gluing by etale algebraic functions, i.e. via an etale equivalence relation. Thus the concept is similar in spirit to that of {\em Nash manifold} \cite{art02-key29}, \cite{art02-key5}. It is close to the classical concept of variety, and gives a naturally geometric object. In particular, an algebraic space over the field of complex numbers has an underlying analytic structure (1.6).

We have tried to choose the notion most nearly like that of scheme with which one can work freely without projectivity assumptions. The assertion that a given object is an algebraic space will thus contain a lot of information. Consequently, the definition given is rather restrictive, and interesting structures such as Mumford's moduli topology \cite{art02-key26} have been excluded (for the moment, let us say), as being not scheme-like enough.

Our point of view is that a construction problem should be solved first in the context of algebraic spaces. In the best cases, one can deduce a posteriori that the solution is actually a scheme. We give some criteria for this in Section \ref{art02-sec3}, but the question of whether a given algebraic space is a scheme may sometimes be very delicate. Thus a construction as algebraic space simply ignores a difficult and interesting side of the problem. On the other hand, it cannot be said that a construction as scheme solves such a problem\pageoriginale completely either. For one wants to prove that the result is projective, say, and where possible to give a description via explicit equations; and projectivity can presumably be shown as easily for an algebraic space as for a scheme (cf. (3.4) in this connection). The question of what constitutes a solution is thus largely a matter of fashion.

We give here an outline of a theory of algebraic spaces, of which details will be published elsewhere. The foundations of this theory, very briefly indicated in Sections \ref{art02-sec1}, \ref{art02-sec2}, are being developed jointly with D. Knutson.\footnote{Cf. D. Knutson: {\em Algebraic spaces,} Thesis, M.I.T. 1968 (to appear).} Section \ref{art02-sec4} contains a fundamental result on approximation of formal sections locally for the etale topology, with some applications. In Section \ref{art02-sec5}, we give the basic existence theorem (\ref{art02-thm5.2}) for algebraic spaces. This theorem allows one to apply deformation theory methods directly to global modular problems, in the context of algebraic spaces. Various applications are given in Section \ref{art02-sec6}.

In Sections \ref{art02-sec4}-\ref{art02-sec6}, we assume that the schemes considered are locally of finite type over a field. The techniques are actually available to treat the case of schemes of finite type over an excellent discrete valuation ring, so that the case that the base in $\Spec \bfZ$ should be included. However, all details have not been written out in that case.

\section{Schemes for the etale topology}\label{art02-sec1}

We assume throughout that the base scheme $S$ is {\em noetherian.}

Let
\begin{equation}
F:(S\text{-schemes})^{0}\to (\text{Sets})\label{art02-eq1.1}
\end{equation}
be a (contravariant) function. When $X=\Spec A$ is an affine $S$-scheme, we will often write $F(A)$ for $F(X)$. Recall that $F$ is said to be a {\em sheaf for the etale topology} if the following condition holds:

\setcounter{subsection}{1}
\subsection{}\label{art02-sec1.2}

Let $U_{i}\to V(i\in I)$ be etale maps of $S$-schemes such that the union of their images in $V$. Then the canonical sequence of maps 
$$
F(V)\to \prod\limits_{i} F(U_{i})\rightrightarrows \prod\limits_{i,j}F(\fprod{U_{i}}{U_{j}}{V})
$$\pageoriginale
is exact.

We assume that reader familiar with the basic properties of this notion (cf. \cite{art02-key6} I, II, \cite{art02-key7} IV).

A morphism $f:U\to F$ (i.e. an element $f\in F(U)$) of an $S$-scheme to a functor $F$ \eqref{art02-eq1.1} is said to be {\em representable by etale surjective maps} if for every map $V\to F$, where $V$ is an $S$-scheme, the fibred product $\fprod{U}{V}{F}$ (considered as a contravariant functor) is representable by a scheme, and if the projection map $\fprod{U}{V}{F}\to V$ is an etale surjective map.

\setcounter{definition}{2}
\begin{definition}\label{art02-defi1.3}
An $S$-scheme for the etale topology, or an algebraic space over $S$, locally of finite type, if a functor
$$
X:(S\text{-schemes})^{0}\to (\text{Sets})
$$
satisfying the following conditions :
\begin{itemize}
\item[\rm(1)] $X$ is a sheaf for the etale topology on ($S$-schemes).

\item[\rm(2)] $X$ is locally representable : There exists an $S$-scheme $U$ locally of finite type, and a map $U\to X$ which is representable by etale surjective maps.
\end{itemize}
\end{definition}

It is important not to confuse this notion of algebraic space $S$ with that of scheme over $S$ whose structure map to $S$ is etale. There is scarcely any connection between the two. Thus {\em any} ordinary $S$-scheme locally of finite type is an algebraic space over $S$.

We will consider only algebraic spaces which are locally of finite type over a base, and so we drop that last phrase.

Most algebraic spaces $X$ we consider will satisfy in addition some separation condition :

\setcounter{sconditions}{3}
\begin{sconditions}\label{art02-sc1.4}
With the notation of \eqref{art02-defi1.3}, consider the functor $\fprod{U}{U}{X}$. This functor is representable, by (1.3) \cite{art02-key2}. The algebraic space $X$ is said to be
\begin{itemize}
\item[(i)] {\em separated} if $\fprod{U}{U}{X}$ is represented by a closed subscheme of $\fprod{U}{U}{S}$;

\item[(ii)] {\em locally\pageoriginale separated}, if $\fprod{U}{U}{X}$ is represented by a locally closed subscheme of $\fprod{U}{U}{S}$;

\item[(iii)] {\em locally quasi-separated}, if the map $\fprod{U}{U}{X}\to \fprod{U}{U}{S}$ is of finite type.
\end{itemize}

It has of course to be shown that these notions are independent of $U$. Although the general case, and the case that (iii) holds, are of considerable interest, we will be concerned here primarily with cases (i) and (ii).

Note that $\fprod{U}{U}{X}=R$ is the graph of an {\em etale} equivalence relation on $U$ (meaning that the projection maps are etale). It follows from general sheaf-theoretic considerations \cite[II.4.3]{art02-key6} that in fact $X$ is the quotient $U/R$ as sheaf for the etale topology on the category ($S$-schemes). Conversely, any etale equivalence relation defines an algebraic space $X=U/R$. Thus we may view an algebraic space as given by an {\em atlas} consisting of its {\em chart} $U$ (which may be taken to be a sum of affine schemes) and its {\em gluing data} $R\rightrightarrows U$, an etale equivalence relation. The necessary verifications for this are contained in the following theorem, which is proved by means of Grothendieck's descent theory \cite[VIII]{art02-key14}.
\end{sconditions}

\setcounter{theorem}{4}
\begin{theorem}\label{art02-thm1.5}
Let $U$ be an $S$-scheme locally of finite type and let $R\rightrightarrows U$ be an etale equivalence relation. Let $X=U/R$ be the quotient as sheaf for the etale topology. Then the map $U\to X$ is represented by etale surjective maps. Moreover, for any maps $V\to X$, $W\to X$, where $V$, $W$ are schemes, the fibred product $\fprod{V}{W}{X}$ is representable.
\end{theorem}

One can of course re-define other types of structure, such as that of analytic space by introducing atlases involving etale equivalence relations $R\rightrightarrows U$. However, it is a simple exercise to check that in the separated and locally separated cases, i.e. those in which $R$ is immersed in $U\times U$, this notion of analytic space is not more general than the usual one, so that every separated analytic etale space is an ordinary analytic space. Thus we obtain the following observations, which is important for an intuitive grasp of the notion of algebraic space.

\begin{corollary}\label{art02-coro1.6}
Suppose\pageoriginale $S=\Spec \bfC$, where $\bfC$ is the field of complex numbers. Then every (locally) separated algebraic space $X$ over $S$ has an underlying structure of analytic space.
\end{corollary}

\begin{examples}\label{art02-exam1.7}
\begin{itemize}
\item[(i)] If $G$ is a finite group operating freely on an $S$-scheme $U$ locally of finite type, then the resulting equivalence relation $R=G\times U\rightrightarrows U$ is obviously etale, hence $U/R$ has the structure of an algebraic space. Thus one can take for instance the example of Hironaka \cite{art02-key16} of a nonsingular 3-dimensional variety with a free operation of $\bfZ/2$ whose quotient is not a variety.

\item[(ii)] Let $S=\Spec \bfR [x]$, $U=\Spec \bfC[x]$. Then $\fprod{U}{U}{S}\approx U\Perp U$, where say the first $U$ is the diagonal. Now put $V=\Spec \bfC[x,x^{-1}]$. Then $R=U\Perp V\to U\Perp U$ is an etale equivalence relation on $U$. The quotient $X$ is locally separated. It is isomorphic to $S$ outside the origin $x=0$. Above the origin, $X$ has two geometric points which are conjugate over $\bfR$. Here $\bfR$ and $\bfC$ denote the fields of real and complex numbers respectively.

\item[(iii)] Let $S=\Spec k$, where $k$ is a field of characteristic not 2, $U=\Spec k[x]$, and let $R=\nabla \Perp \Gamma\to U\times U$, where $\nabla$ is the diagonal, and where $\Gamma$ is the complement of the origin in the anti-diagonal
$$
\Gamma=\{(x,-x)|x\neq 0\}.
$$
Then $X=U/R$ is locally quasi-separated, but not locally separated.
\end{itemize}
\end{examples}

\begin{proposition}\label{art02-prop1.8}
Let
\[
\xymatrix{
X\ar[dr] & & Z\ar[dl]\\
 & Y &
}
\]
be a diagram of algebraic spaces over $S$. Then the fibred product $\fprod{X}{Z}{Y}$ is again an algebraic space.
\end{proposition}

\section{Elementary notions}\label{art02-sec2}

\begin{definition}\label{art02-defi2.1}
A property $P$ of schemes is said to be local for the etale topology if
\begin{itemize}
\item[\rm(i)] $U'\to U$\pageoriginale 
\end{itemize}
\end{definition}




