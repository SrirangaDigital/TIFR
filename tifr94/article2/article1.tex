\title{C.P. Ramanujam}
\markright{C.P. Ramanujam}

\author{By~ K.G. Ramanathan}
\markboth{K.G. Ramanathan}{C.P. Ramanujam}

\date{}
\maketitle


\setcounter{page}{1}

Chidambaram\pageoriginale Padmanabhan Ramanujam was born in Madras on January 9, 
1938, as the eldest son of Shri C. Padmanabhan, an advocate in the 
High Court, Madras. Ramanujam had his primary and secondary education 
at Ewart's School, Madras and later joined the Sir M. Ct. Muthiah 
Chetty High School at Vepery, Madras. He passed his High School 
examination at the early age of 14 in 1952 and joined, as many 
talented young students in Madras did, the Loyola College, Madras, for 
his Intermediate and then the Mathematics Honours. He passed out of 
the Loyola College with the B.A. (Hons.) degree in 1957 with a second 
class which was unusual for a gifted student like him. During his High 
School and College days he was interested in Chemistry and Tennis. He 
had set up, in his house, a small laboratory in Chemistry and would 
perform experiments along with his friend Raghavan Narasimhan, now at 
Chicago University. The Laboratory was closed down after the 
`inevitable', namely, a small accident.

During his Honours years he came under the guidance of the late 
Reverend Fr. C. Racine of the Loyola College with whom he kept up 
regular correspondence almost until his (Ramanujam's) death. He had, 
as many students of Fr. Racine's had, a great regard and respect, 
bordering on affection for Fr. Racine. On Fr. Racine's suggestion he 
applied in 1957 for admission to the School of Mathematics of the Tata 
Institute of Fundamental Research at Bombay. In his letter to the 
Institute recommending Ramanujam, Fr. Racine wrote, ``He has certainly 
originality of mind and the type of curiosity which is likely to 
suggest that he will develop into a good research worker if given 
sufficient opportunity''. Ramanujam amply justified Fr. Racine's 
estimate of him. Ramanujam, before he entered the Tata Institute, also 
came, briefly, under the influence of the late Professor T. 
Vijayaraghavan, the eminent mathematician who was, until 1954, the 
Director of the Ramanujan Institute of Mathematics\pageoriginale at 
Madras. Ramanujam (and his friend Raghavan Narasimhan who also joined 
the Tata Institute in 1957) displayed a deep knowledge of Analytic 
Number Theory even at the time of joining the Tata Institute.

At the Tata Institute he learnt mathematics with an avidity and speed 
that was often frightening. He had given expert Colloquium talks and 
participated in seminars and displayed within two years of stay, 
versatility and depth in mathematics which are rare. Unfortunately 
however, he soon found himself in possession of a vast amount of 
mathematical expertise which he could not `cash in', by solving 
problems of importance. The success of a few others at the Institute 
led to great frustration and Ramanujam decided to leave the Institute. 
He felt, wrongly of course, that he was inadequate in solving problems 
in mathematics and that he would be, perhaps, a better teacher in a 
university or a college. He began applying to various colleges and 
universities in India and it was lucky for the Tata Institute and 
Ramanujam that he was not selected in any of the places he had applied 
to. 

It was then that he came to work with me and he started with a problem 
relating to Lie groups and Differential geometry connected with the 
work of C.L. Siegel. Early in 1961 he took up the problem regarding 
Diophantine equations especially related to those over algebraic 
number fields. The important problem was that raised by C.L. Siegel 
regarding the generalisation of Waring's problem to algebraic number 
fields, namely, to find a $g=g(m)$ \emph{independent of the degree of 
the algebraic number field $K$}, so that every totally positive 
integer $v$ with sufficiently large norm, which is in the ring 
$J_m (K)$ generated by $m$th powers of integers of $K$ can be written
$$
v=x_1^m+\cdots +x_g^m
$$
as the sum of $m$th powers of totally positive integers $x_i$. At that 
time many significant results were being obtained by Davenport and his 
co-workers D.J. Lewis and B.J. Birch. Davenport had proved that every 
cubic form with rational coefficients in at least $g=32$ 
variables\pageoriginale had a non-trivial rational zero. Ramanujam set 
about the task of trying to generalise Davenport's method to cubic 
forms over algebraic number fields by using Siegel's generalisation of 
the major and minor arcs of the Hardy-Littlewood-Ramanujan circle 
method. In this attempt he succeeded eminently by first simplifying 
Siegel's method and then proving that every cubic form in 54 variables 
over any algebraic number field $K$ had a non-trivial zero over that 
field. Ramanujam himself knew that the number 54 could be considerably 
reduced, even to Davenport's 29; but he was not interested in doing 
this. Recently Ramanujam's idea of proof was adapted by Pleasants who 
showed that Ramanujam's result holds with $g=16$. 

However, Waring's problem in algebraic number fields was always at the 
back of his mind. In the meantime some very interesting results had 
been published. Rose Marie Stemmler had written a thesis on the easier 
Waring's problem in algebraic number fields, namely, of expressing 
$v\succ 0$ in the form $x_1^m\pm x_2^m\pm\cdots\pm x_g^m$. She showed 
that $g$ can be bounded by an integer depending only on $m$ and 
\emph{not} on the degree of the field $K$ provided $m$ is, for 
instance, a prime number. Birch then used a generalisation of Hua's 
mean value theorem to show that indeed the method of Stemmler gives 
Siegel's conjecture in the case $m$ is a prime $p$, and that every 
totally positive integer of sufficiently large norm, in $J_p(K)$ is 
indeed a sum of at most $2^p+1$ totally positive $p$th powers. The 
extension of this result for arbitrary natural number $m$ was tied up 
with obtaining a local to global theorem as was shown by Birch and 
K\"orner. Ramanujam attacked the general problem of representation of 
elements of $J_m(A)$, the ring generated by $m$th powers of elements 
of a complete discrete valuation ring $A$, as sums of a fixed numbers 
of $m$th powers. He showed that whatever $A$ might be, $8m^5$ summands 
would be enough. Birch had shown, almost simultaneously, that if 
$s>2^m +1$, then every integer of sufficiently large norm which is sum 
of $s$, $m$th powers modulo all powers of prime ideals, is itself such a 
sum of $m$th powers of integers in $K$. The problem raised by Siegel 
was thus solved. Independently, at about\pageoriginale the same time, 
Birch also solved this $p$-adic problem by an entirely different 
method.

Ramanujam was promoted as Associate Professor for his brilliant work 
on Number Theory. He protested very strongly against this but was 
prevailed upon by friends and colleagues to accept the position. This 
promotion, besides being deserved in his case, also showed, as was the 
intention of the School of Mathematics, that promotion to any position 
in the Institute was possible on just brilliant accomplishments. He 
wrote his thesis in 1966 and early in 1967 he took his Doctoral 
examination at which Carl. L. Siegel was one of the examiners. Siegel 
very greatly appreciated not only Ramanujam's knowledge in Number 
Theory but also his general mathematical culture especially in 
Analysis.

Ramanujam had attended during the first few years of his stay at the 
Institute a large number of courses of lectures by visiting 
Professors. He wrote the notes of Lectures of Max Deuring during 
1958--59 on ``the theory of algebraic functions of one variable'' and 
in 1964--65 of the lectures of I.R. Shafarevich on ``minimal models 
and birational transformations of two dimensional schemes''. Professor 
Shafarevich was very appreciative of Ramanujam's contributions to the 
Lecture notes. He wrote, ``I want to thank him (\emph{Ramanujam}) for 
the splendid job he has done. He not only corrected several mistakes 
but also complemented proofs of many results that were only stated in 
oral exposition. To mention some of them, he has written the proofs of 
the Castelnuovo theorem ... of the chain condition ..., the example of 
Nagata of a non-projective surface ... and the proof of Zariski's 
theorem ...''. In the case of Mumford's lectures on Abelian Varieties 
which he gave in 1968 at the Tata Institute, Mumford wrote, ``...these 
lectures were subsequently written up, and improved in many ways, by 
C.P. Ramanujam. The present text is the result of a joint effort'', 
and further, ``... C.P. Ramanujam continuing my lectures at the Tata 
Institute lectured on and wrote up notes on Tate's theorem on 
homomorphisms between abelian varieties over finite fields''.  

The\pageoriginale contacts with Shafarevich and Mumford led him on to 
Algebraic Geometry and his progress and deep understanding in this 
field was phenomenal. Of his work in Algebraic Geometry, Mumford has 
written an excellent account which also is included in this volume. 
His influence on the members of the School of Mathematics was indeed 
very great. He was a source of inspiration and of ideas to the members 
of the School of Mathematics. He used to hold seminars on Commutative 
Algebra and on aspects of Algebraic Geometry. To talk to him on 
mathematics, especially Algebraic Geometry, was to get one's ideas 
clarified and sometimes get one's problems solved. He was very 
generous in giving his ideas, especially to the younger members of the 
School.

In 1964 he participated in the International Colloquium on 
Differential Analysis at the Institute and in 1968, the International 
Colloquium on Algebraic Geometry at which Grothendieck and Mumford, 
among others, greatly admired Ramanujam's knowledge and virtuosity. 
Both of them became fast friends and admirers of his and invited him 
to Paris and Harvard. 

It was in 1964 that he had the first of the many attacks of the cruel 
malady that was ultimately to take his life. The ailment was diagnosed 
by the medical officer to be ``Schizophrenia with severe depression''. 
This depression made him feel that he was ``inadequate'' for 
mathematical research and he therefore decided to take a position in a 
University. In July 1965 he was offered a Professorship, with tenure, 
at the Panjab University in Chandigarh. However, he went there only 
for a year. He was immensely liked by colleagues and students there, 
but the illness struck him again and amidst tragic circumstances he 
had to cut short his stay there after about 8 months. Shortly after he 
rejoined the Tata Institute in June 1965, he received an invitation 
from the Institut des Hautes Etudes Scientifiques, Paris, to spend 
six months there. The Tata Institute deputed him to Bures-Sur-Yvette 
with full salary but again he returned before the projected period of 
stay was over, since the cruel illness struck him. I still remember 
his telephoning to me after arrival at Bombay 
airport---sans\pageoriginale his baggage---saying that he would write 
to me about his `whereabouts in a few days'. Such bouts with his 
unfortunate illness continued throughout his short life. In  February 
1970, he wrote to the Director of the Institute, ``due to personal 
reasons, I have to leave the Institute immediately. Please accept this 
letter of resignation''. The Director, rightly, refused to accept the 
resignation and wrote that ``in his disturbed state of mind the above 
cannot be treated straightaway as meaningful''. However, he later 
resigned from the Institute and went for some time to Warwick 
University, at the instance of Mumford during the Algebraic Geometry 
year there. He also visited Italy for some time and made many friends 
there. When he came back, he was requested to accept a Professorship 
at the Tata Institute of Fundamental Research but stay at Bangalore 
where the Institute had just started the programme in applications of 
mathematics. He lectured there on Analysis regularly and won great 
respect from the students as well as from many members of the Indian 
Institute of Science.

During the years he stayed at the Institute he had the attacks of his 
cruel illness several times. At those times, he resolved to take his 
life and it appears that once he \emph{did} make a serious effort to 
end his life but the attempt was discovered and he was treated 
promptly. Due to great frustration he again decided to leave Bangalore 
and sever his connexion with the Tata Institute and had written 
several letters to me about it. He was recommended to the Indian 
Institute for Advanced Studies at Simla for a permanent Professorship. 
He died on October 27, 1974, after taking an overdose of barbiturates, 
before the offer was made by that Institute. 

During his Bangalore days he became increasingly interested in 
mysticism and especially in `miracles' which were being talked about 
in Bangalore at that time. He had many discussions with me about these 
`miracles' and he said that he would investigate the truth about these 
miracles. He wanted to do this, both because of his intense curiosity 
and also because he was interested in finding out whether his malady 
could be cured through these methods. He was\pageoriginale becoming 
somewhat `religious-minded' and began also chanting sacred hymns.

Ramanujam was intensely human in his relations with others and honest 
to a fault both in mathematics and in his dealings with fellow 
mathematicians and students---qualities that are rare, more so 
now-a-days. He was modest and self-effacing. In mathematics he had 
very high standards, though he would not make caustic remarks about 
the work of others especially if it concerned a young man who was 
trying to come up. He would always, as he told me many a time, think 
about himself and his own work, which curiously enough, he felt, was 
incommensurate with his position at the Institute. However, when it 
came to promotions he applied very strict standards. He was an 
understanding colleague, warm-hearted, and an evening with him could 
be rewarding mathematically and socially. He liked high class 
classical Indian music and loved books. Buying books, mostly on 
mathematics but sometimes on art, mysticism etc., especially first 
editions or rare books became an obsession with him. His enormous 
private library of mathematical books was bought by the Tata Institute 
of Fundamental Research.

Ramanujam was a very good lecturer and loved his students. His 
colleagues and friends loved him and got great benefit in mathematics 
from discussion with him. He was elected Fellow of the Indian Academy 
of Sciences in 1973. In his death the country in general, and the Tata 
Institute in particular, has lost an outstanding mathematician, a warm 
colleague and a great human being. He is mourned by one and all. 

\vfill\eject

~\phantom{A}
\thispagestyle{empty}


