
\title{AUTOMORPHY FACTORS OF HILBERT'S}
\markright{AUTOMORPHY FACTORS OF HILBERT'S}

\author{By~ EBERHARD FREITAG}
\markboth{EBERHARD FREITAG}{AUTOMORPHY FACTORS OF HILBERT'S}

\date{}
\maketitle

%\setcounter{page}{9}
\setcounter{pageoriginal}{8}

\noindent
\textbf{Introduction.}
Let $\Gamma$\pageoriginale be a group of analytic automorphisms of a domain $D \subset \bC^n$. By an automorphy factor of $\Gamma$ we understand a family of functions $I(z,\gamma)$, $z \in D$, $\gamma \in \Gamma$ holomorphic on $D$ and without zeros, which satisfy the condition
$$
I(z, \gamma' \gamma) = I (z,\gamma) I (\gamma z, \gamma').
$$
The most-occurring factors are the following ones
\begin{enumerate}
\item[1)] \textit{The trivial factors}
$$
I (z, \gamma) = \frac{h(\gamma z)}{h(z)}.
$$
Here $h$ is a holomorphic function on $D$ without zeros.

\item[2)] {\it The powers of the complex functions determinant} (Jacobian).

\item[3)] {\it The abelian characters $v$ of $\Gamma$}
$$
I(z, \gamma) = v(\gamma).
$$
\end{enumerate}
The determination of all automorphy factors belonging to a \textit{discontinuous} group is a difficult problem in general. It is roughly equivalent to the calculation of 
$$
\Pic D/\Gamma = \text{ group of analytic line bundles on $D/\Gamma$}.
$$
More precisely, if $\Gamma$ operates without fixed points, we have
$$
\Pic D/\Gamma = \frac{\text{group of automorphy factors}}{\text{subgroup of trivial factors}}
$$
There is a well-known isomorphism
$$
\Pic D/\Gamma=H^1 (D/\Gamma, \theta^\ast)
$$
($\cO$ = sheaf of automorphic functions,

\noindent
$\cO^\ast$ = sheaf of invertible automorphic functions).

By\pageoriginale means of the exact sequence
$$
0 \to \bZ \to \theta \xrightarrow{\exp} \theta^\ast \to 0,
$$
we reduce the original problem to the calculation of 
\begin{itemize}
\item[a)] the singular cohomology of $D/\Gamma$

\item[b)] the analytical cohomology $H^{\bigdot} (D/\Gamma , \theta)$.
\end{itemize}
This program could be carried out almost completely for the domain
$$
D = H^n = H \times \ldots \times H, H \text{ the usual upper half-plane.}
$$
Matsushima and Shimura succeeded in calculating those groups in case of a compact quotient by means of the Hodge theory \cite{art2-key3}. As for the non-compact quotients $D/\Gamma$ (Hilbert's modular groups) similar complete results have been found.
\begin{itemize}
\item[a)] The singular cohomology was inverstigated by G. Harder \cite{art2-key2}. 

Let us give a very brief indication of the specific problems arising in the non-compact case.

By ``cutting off cusps'' of $D/\Gamma$ one gets a manifold with boundary.

There is a natural mapping from the cohomology of the whole space $D/\Gamma$ to the (well-known) cohomology of the boundary. In the mentioned paper, Harder determined the image and the kernel of this map. His detailed study of this problem leads into the theory of non-analytic modular forms, especially into the theory of Eisenstein series.

\item[b)] The analytical cohomology was determined in \cite{art2-key1}.

To overcome the discrepancy between the standard compactification of $D/ \Gamma$ and a non-singular model, we had to carry out a thorough investigation of the algebraic nature of the cups. But it was not necessary to get a concrete resolution of the cusps.
\end{itemize}

\noindent
\textbf{1. The main result.}
In the following let $\Gamma$ be a group of simultaneously fractional linear substitutions
$$
M (z_1, \ldots, z_n) = \left(\frac{a_1 z_1 + b_1}{c_1 z_1 + d_1} , \ldots, \frac{a_n z_n + b_n}{c_n z_n + d_n} \right)
$$
of the\pageoriginale half space
$$
H^n = \{(z_1, \ldots , z_n) \in \bC^n ;  \Iim z_v \geq 0 \text{ for } 1 \leqslant v \leqslant n \}.
$$
We are only interested in the case in which $\Gamma$ is commensurable with Hilbert's modular group of a totally real number field. We define the complex power
$$
a^b = e^{b \log a}, \;\; a \neq  0
$$
by the principal branch of the logarithm.

\begin{theorem}\label{art2-thm1}%% 1
In case of $n \geqslant 3$\footnote{The method used for the proof is valid also in case of $n \leq 3$. But one has to carry out some separate investigations because in this case the first cohomology groups are not trivial.} the only automorphy factors of $\Gamma$ are:
$$
I (z, M) = v (M) \prod^n_{v=1} (c_v z_v + d_v)^{2r_v} \frac{h(M_z)}{h(z)}
$$
where 
\begin{itemize}
\item[a)] $r = (r_1 , \ldots , r_n)$ is a vector of rational numbers;

\item[b)] $\{v (M)\}_{M \in \Gamma}$ is a system of complex numbers of absolute value one;

\item[c)] $h$ is a holomorphic invertible function on $H^n$.
\end{itemize}
This factorisation of $I$ is unique.
\end{theorem}

By a system of multipliers of weight $r = (r_1, \ldots, r_n)$ we understand a family $\{v (M)\}_{M \in \Gamma}$ of complex numbers of absolute value one, such that
$$
I(z, M) = v(M) \prod^n_{v=1} (c_v z_v + d_v)^{2r_v}
$$
is a factor of automorphy.

\medskip
\noindent
\textsc{\textbf{Amendment to Theorem \ref{art2-thm1}.}}

\begin{itemize}
\item[1)] \textit{The group of abelian characters of $\Gamma$ is finite.}\footnote{A more general result has been proved by Serre \cite{art2-key4}.}

\item[2)] \textit{If $r = (r_1, \ldots, r_n)$ is the weight of a multiplier system, the components $r_v$ have to be rational and their denominators are bounded (by a number which may depend on the group)}.
\end{itemize}

We now\pageoriginale discuss an application of the main theorem.

A meromorphic modular form with respect to $\Gamma$ is a meromorphic function on $H^n$ satisfying the functional equations:
$$
 f(Mz) = v (M) \prod^n_{v=1} (c_v z_v + d_v)^{2r_v} f (z) \text{ for } M \in \Gamma.
$$
We cell $r = (r_1, \ldots, r_n)$ the weight and $v(M)$ the multiplier system of $f$.

We are interested in the zeros and poles of $f$ which we describe by a divisor ($f$) as usual.

By a divisor we understand a formal and locally finite sum
$$
D = \sum\limits_Y n_Y T, \;\; n_Y \in \bZ 
$$
the summation being taken over irreducible closed analytic sub-varieties of codimension one.

\begin{theorem}\label{art2-thm2}%%% 2
Let $\sD$ be a $\Gamma$-invariant divisor on $H^n$, $n \geqslant 3$. There exists a meromorphic modular form $f$ with the property
$$
\sD = (f).
$$
\end{theorem}

\begin{proof}
The space $H^n$ is a topologically trivial Stein-space. Therefore we can find a meromorphic function $g$ on $H^n$ with
$$
\sD  = (g).
$$
The function
$$
I (z, M) = \frac{g(Mz)}{g(z)}, \;\; M \in \Gamma
$$
is without poles and zeros because $\sD$ is $\Gamma$-invariant. We therefore can apply Theorem \ref{art2-thm1}. Put
$$
f = \frac{g}{h}.
$$
\end{proof}

\medskip
\noindent
\textbf{2. Sketch of proof.} The group $\Gamma$ operates in a natural way on the multiplicative group $H^0 (D, \cO^\ast)$ of holomorphic invertible functions on $D = H^n$.

The\pageoriginale automorphy factors are nothing else but the 1-cocycles with regard to the standard complex and the trivial factors $h(\gamma z)/ h(z)$ are the 1 - coboundaries, \iec
$$
H^1 (\Gamma, H^0 (D, \cO^\ast)) = \frac{group of automorphy factors}{subgroup of trivial factors}
$$
Theorem \ref{art2-thm1} may thus be formulated as follows

\begin{theorem}\label{art2-thm3}
The group $H^1 (\Gamma, H^0 (D, \cO^\ast))$ is finitely generated and of free rank $n$.
\end{theorem}

We now want to pass on a subgroup $\Gamma_0 \subset \Gamma$ of finite index in order to eliminate the elements of finite order. Let $\Gamma_0 \subset \Gamma$ be a normal subgroup of finite index. Putting
$$
A = H^0 (D, \cO^\ast)
$$
we obtain by means of the Hochschild-Serre sequence
$$
0 \to H^1 (\Gamma /\Gamma_0, \bC^\ast) \to H^1 (\Gamma, A) \to H^1 (\Gamma_0, A)^{\Gamma/\Gamma_0} \to H^2 (\Gamma/\Gamma_0, \bC^{\ast}).
$$

(Since every holomorphic modular function is constant, we have
$$
A^{\Gamma_0} = \bC^\ast.)
$$

The groups $H^v (\Gamma /\Gamma_0, \bC^\ast)$ are finite. This is proved by means of the sequence
$$
0 \to \bZ \to \bC \to \bC^\ast \to 0.
$$
In general, the cohomology groups of a finite group which acts trivially on $\bZ$, are finite.

We therefore can assume without loss of generality:

\textit{The group $\Gamma$ is a congruence-subgroup of Hilbert's modular group without torsion.}

In the case at hand it is easy to be seen
$$
H^1 (\Gamma, H^0 (D, \cO^\ast)) = H^1 (D/ \Gamma, \cO^\ast),
$$
\ie there is a one-to-one correspondence between the factor classes and the classes of isomorphic analytical line bundles on $X_0 = D/\Gamma$.

We now\pageoriginale  treat the group
$$
\Pic X_0 = H^1 (X_0, \cO^\ast), \;\; X_0 = D/ \Gamma
$$
by means of the sequence
$$
0 \to \bZ \to \cO  \xrightarrow{\exp} \cO^\ast \to 0.
$$
Hereby $\cO$ is the sheaf of holomorphic functions on $X_0$. From the long cohomology sequence results
$$
H^1 (X_0, \cO) \to \Pic X_0 \to H^2 (X_0, \bZ).
$$
We thus have to calculate the groups $H^1 (X_0, \cO)$ and $H^2 (X_0, \bZ)$.

\begin{theorem}\label{art2-thm4}
The groups $H^v (X_0, \cO)$ vanish for $1 \leqslant v \leqslant n -2$.
\end{theorem}

\begin{proof}
Let $S$ be the finite set of cusp classes of $\Gamma$ and 
$$
X = X_0 \cup S
$$
the standard compactification of $X_0 = D/\Gamma$. There is a long exact sequence, which combines the cohomology with supports in $S$ with the usual cohomology of sheafs
$$
H^v_S (X, \cO) \to H^v (X, \cO) \to H^v (X_0, \cO) \to H^{v+1}_S  (X, \cO) \to H^{v+1} (X,\cO).
$$
From my paper \cite{art2-key1} the results (Theorem 7.1).
$$
H^v_S (X, \cO) \simeq H^v (X, \cO) \text{ for } 1 \leqslant v \leq n
$$
is taken.

An analysis of the proof shows that this isomorphism is induced by the natural mapping 
$$
H^v_S (X, \cO) \to H^v (X, \cO).
$$
In case of $n \geqslant 3$ we now obtain the exact sequence
$$
0 \to \Pic X_0 \to H^1 (X_0, \bZ).
$$
Obviously the free rank of $\Pic X_0$ is not smaller than $n$ because the automorphy factors
$$
I_v (z, M) = (c_v z_v + d_v)^2 \quad (1 \leqslant v \leqslant n)
$$
are independent of each other.
\end{proof}

Therefore\pageoriginale Theorem \ref{art2-thm3} has been proved if one knows that $H^2 (X_0, \bZ)$ is of free rank $n$. That means

\begin{theorem}\label{art2-thm5} 
In case of $n \geqslant 3$ we have
$$
\dim_{\bC} H^2 (X_0, \bC) = n, \;\; X_0 = D/\Gamma.
$$ 
\end{theorem}

\begin{proof}
We derive the calculation of the 2nd Betti number of $X_0$ from Harder's investigations on the singular cohomology of $X_0 = D/\Gamma$ \cite{art2-key2}.

This will be explained briefly in the following.

By cutting off cusps we obtain a bounded manifold $X^\ast$ which is homotopically equivalent to $X_0$. (The boundary component at the cusp is given by
$$
\prod^n_{v=1} \Iim z_v = C; \; \; C \gg 0.)
$$

In the paper quoted above, Harder gives a decomposition of the singular cohomology of $X_0$
\begin{align*}
H^\bigdot (X_0, \bC) & = H^\bigdot (X^\ast, \bC)\\
& = H^\bigdot_{\inf} (X^\ast, \bC) \bigoplus H^\bigdot_{\text{univ}} (X^\ast, \bC) \bigoplus H^\bigdot_{\text{cusp}} (X^\ast,\bC).
\end{align*}

This decomposition has the following properties:
\begin{enumerate}
\item[(1)] The canonical mapping 
$$
\zeta^\ast: H^\bigdot (x^\ast, \bC) \to H^\bigdot (\partial X^\ast, \bC)
$$
defines an isomorphism of $H_{\inf} (X_0, \bC)$ onto the image of $\zeta^\ast$.

\item[(2)] $H_{\text{univ}} (X_0, \bC)$ is a subring, generated by the cohomology-classes attached to the universal harmonic forms 
$$
\frac{dx_v\wedge dy_v}{y^2_v}, \; 1 \leqslant v \leqslant n.
$$

 \item[(3)] The cohomology classes in $H^\bigdot_{\text{cusp}} (X_0, \bC)$ can be represented by harmonic cusp-forms (which are rapidly decreasing at infinity).
\end{enumerate}

The image of $\zeta^\ast$ can be represented by means of the theory of Eisenstein-series. One has
$$
H^v_{\inf} (X_0, \bC) = 0 \text{ for } 1 \leqslant v \leqslant n -1.
$$\pageoriginale
For the subspace of cusp forms $H^\bigdot_{\text{cusp}} (X_0, \bC)$ one has a Hodge decomposition
$$
H^r_{\text{cusp}} = \bigoplus_{p+q=r} H^{p,q}_{\text{cusp}}.
$$
This part of the theory coincides with the investigations of Matsushima and Shimura \cite{art2-key3} who treated the case of a compact quotient $D/\Gamma$. One has 
\begin{align*}
H^v_{\text{cusp}} (X_0, \bC) = 0 \text{ for } v \neq n \text{ and therefore}\\
H^2 (D/\Gamma, \bC) = H^2_{\text{univ}} (X_0,\bC) (n \geqslant 3).
\end{align*}

A basis of this vector space is represented by the harmonic forms
$$
\frac{dx_v \wedge dy_v}{y^2_v} \;\; 1 \leqslant v \leqslant n.
$$
\end{proof}

\begin{remark*}
The dimension of $H^{p,q}_{\text{cusp}}$ can be calculated explicitly using the methods of \cite{art2-key1}. One obtains an expression in terms of Shimizu's rank polynomials.
\end{remark*}

\medskip
\noindent
\textbf{3. Line bundles on the standard compactificaiton $D/\Gamma$.} Every $\Gamma$-invariant divisor $\sD$ on $H^n (n \geqslant 3)$ can be represented by a modular form of a certain weight $r = (r_1, \ldots, r_n)$ according to Theorem \ref{art2-thm2}.

Finally we investigate the problem in which case the weight $r$ satisfies the condition
$$
r_1 = \ldots = r_n.
$$
Firstly, $\sD$ defines a divisor on $X_0=H^n/\Gamma$ which can be continued to a divisor on the standard compactification $X = D/\Gamma$ due to a well-known theorem of Remmert.

Then, we call $\sD$ a Cartier divisor, if the associated divisorial sheaf on $X$ is a line bundle, \ie for each point $x \in X$ (even if it is a cusp) exists an open neighbourhood $U$ and a meromorphic function $f:U \to \bC$ which represents $\sD$:
$$
(f) = \sD/U.
$$
At the regular points, this is automatically the case.

\begin{theorem}\label{art2-thm6}
In the\pageoriginale case of $n \geqslant 3$ the following two conditions are equivalent for a $\Gamma$-invariant divisor:
\begin{itemize}
\item[i)] The divisor $\sD$ is defined by a meromorphic modular form of the type
\begin{multline*}
f(Mz) = v(M) \left[\prod^n_{v=1} (c_v z_v + d_v)^2 \right]^r f(z) \text{ for } M \in \Gamma \\
(|v(M)| = 1; r \in \bQ).
\end{multline*}

\item[ii)] A suitable multiple $k \sD$, $k \in \bZ$, of $\bD$ is a Cartier divisor.
\end{itemize}
\end{theorem}

\begin{proof}
The condition (ii) is only relevant in the case of cusps. One has to ovserve that in $H^n/\Gamma$ only a finite number of quotient singularities occur.

Firstly we show i) $\Rightarrow$ ii).

We may assume that $r$ is integer because we can replace $f$ by a suitable power. In that case $v$ has to be an abelian character.

This character has --- owing to amendment 2 of Theorem \ref{art2-thm1} --- finite order. Therefore we may assume $v =1$.

Let $\infty$ be a cusp of $\Gamma$.

The form $f$ is invariant by the affine substitutions of $\Gamma$
$$
z \to \epsilon z + \alpha
$$
and therefore induces a meromorphic function in a neightbourhood of the cusp, which obviously represents the divisor $\sD$.

ii) $\Rightarrow$ i):

We may assume that $\sD$ is a Cartier divisor. This means for the cusp: There exists a meromorphic function
$$
g : U_c \to \bC; U_c = \{z \in H^n ; N (\Iim z) > C\} 
$$
with the properties
\begin{itemize}
\item[a)] $g$ is $\Gamma$-invariant,

\item[b)] $(g) = \sD$ in $U_c$.
\end{itemize}
The\pageoriginale function $h = \dfrac{f}{g}$ is holomorphic and without zeros. The transformation law
$$
h(\epsilon z + \alpha) = v 
\begin{bmatrix} 
\epsilon^{1/2} & \epsilon^{-1/2} \alpha\\
0 & \epsilon^{-1/2}
\end{bmatrix} \prod^n_{v=1} \epsilon^{r_v}_v h(z)
$$
is valid. By means of a variant of Koecher's principle one sees, that the limit value
$$
\lim\limits_{N (\Iim z) \to \infty} h (z) = C
$$
exists and is finite.

The same argument also holds for the function $\dfrac{1}{h}$ instead of $h$.
Therefore $C$ has to be different from zero. It follows that
$$
v \begin{bmatrix}
\epsilon^{1/2} & \epsilon^{-1/2} \alpha\\
0 & \epsilon^{-1/2}
\end{bmatrix} \prod^n_{v=1} \epsilon^{r_v}_v = 1
$$
and therefore
$$
r_1 = \ldots = r_n.
$$

Finally we mention an interesting application without supplying the proof.
\end{proof}

\medskip
\noindent
\textsc{\textbf{Conclusion to Theorem \ref{art2-thm4}.}}
\textit{In case of $n \geqslant 3$ we have}
$$
\Pic X = \bZ, \; X = \overline{H^n/\Gamma} \;\; (\text{standard-compactification}).
$$

We also have got some information about a generator of $\Pic X$. Choose a natural number $e$, such that 
$$
\gamma^e = id \quad \text{for } \gamma \Gamma.
$$
Then $\sK^e$ ($\sK$ = canonical divisor) defines a line-bundle on $X$.

This bundle generates a subgroup of finite index in $\Pic X$
$$
N = [\Pic X : \{\sK^v, v \equiv 0 \mod e\}].
$$
It is possible to calculate the Chern-class of a generator of $\Pic X$ 
$$
c (\sK^{e/N}) = \frac{1}{2\pi i} \frac{e}{N} \sum^{n}_{v=1} \frac{dz_v \wedge d \bar{z}_v}{y^2_v}.
$$\pageoriginale 
This defines in fact a cohomology class  on $X$.

We now make use of the fact that Chern-classes are always integral, \ie
$$
\int\limits_{\gamma} c (\sK^{e/N}) \in \bZ,
$$
where $\gamma$ is a two-dimensional cycle on $\bX$. Such cycles can be constructed by means of certain specializations. The simplest case is 
$$
z_1 = \ldots = z_n.
$$

If one carries out the integration, one obtains conditions for $N$.

\begin{example*}
If $\Gamma$ If $\Gamma$ is the full Hilbert-modular group, then 
$$
\int\limits_{z_1 = \ldots = z_n} c(\sK^{e/N}) = \frac{en}{3N} \in \bZ.
$$
\end{example*}

\begin{thebibliography}{99}
\bibitem{art2-key1} \textsc{E. Freitag:} Lokale und globale Invarianten der Hilbertschen Modulgruppe, \textit{Invent. math.} 17, (1972) 106-134.

\bibitem{art2-key2} \textsc{G. Harder:} On the cohomology of $Sl (2, \fo)$ (preprint).

\bibitem{art2-key3} \textsc{Y. Matsushima} and \textsc{G. Shimura}: On the cohomology groups attached to certain vector-valued differential forms on the product of the upper half planes. \textit{Ann. of Math.} 78, (1963), 417-449.

\bibitem{art2-key4} \textsc{J. P. Serre:} Le probl\'eme des groups de congruence pour $Sl_2$. \textit{Ann. of Math.,} 92  (1970), 489-527.
\end{thebibliography}
