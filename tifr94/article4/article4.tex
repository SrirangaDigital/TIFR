
\title{RECENT DEVELOPMENTS IN HODGE THEORY: A DISCUSSION OF TECHNIQUES AND RESULTS}
\markright{RECENT DEVELOPMENTS IN HODGE THEORY: A DISCUSSION OF TECHNIQUES AND RESULTS}

\author{By~ PHILLIP GRIFFITHS and WILFRIED SCHMID}
\markboth{PHILLIP GRIFFITHS and WILFRIED SCHMID}{RECENT DEVELOPMENTS IN HODGE THEORY: A DISCUSSION OF TECHNIQUES AND RESULTS}

\date{}
\maketitle

%\setcounter{page}{21}
\setcounter{pageoriginal}{20}


\section*{Introduction}\label{art4-sec1}
In\pageoriginale this paper, we shall review several recent developments in \textit{Hodge theory}, as applied to the study of the cohomology of algebraic varieties. In some sense, we are continuing the report \cite{art4-key21} of the first author, in which the then current work in Hodge theory was discussed without proof and a number of open problems were raised. Here we shall be concerned primarily with \textit{methods of proof}, \iec understanding in as transparent terms as possible the techniques utilized in this recent work in Hodge theory. We shall also present some results, due to the second author \cite{art4-key41}, which have just now been published, and shall bring up to date the status of the problems raised in \cite{art4-key21}.

One of the recent developments we shall discuss is Deligne's theory of \textit{mixed Hodge structures} (\cite{art4-key12}, \cite{art4-key}, \cite{art4-key14}). In this work, Deligne extends classical Hodge theory first to open, smooth varieties \cite{art4-key13}, then to complete, singular varieties \cite{art4-key14}, and finally to general varieties, also in \cite{art4-key14}. The heuristic reasoning explaining why such a theory should be possible is given in \cite{art4-key12}.

Deligne's technique is to use \textit{resolution of singularities} \cite{art4-key29}, in order to be able in each case to write the cohomology of the variety in question as being derived from the cohomology of K\"ahler manifolds by homological algebra. Typically this process gives the cohomology of the variety as the abutment of a spectral sequence whose $E_1$ or $E_2$ term is the cohomology of a smooth projective variety. Thus the $E_1$ or $E_2$ term has a \textit{Hodge structure}, and in order for this structure to survive as a Hodge structure on $E_\infty$, inducing the desired mixed Hodge structure on the cohomology of the variety, it is necessary that the spectral sequence degenerates. Following a discussion of the formalism of Hodge structures and mixed Hodge structures in \S 1, we have in \S 2 (a), \S 4, and \S 5 (d)  presented several typical degeneration arguments in as direct a manner as we could.

In \S \ref{art4-sec4} we construct the mixed Hodge structure on the cohomology of the simplest singular complete varieties, namely those having only \textit{normal crossings as singularities}. Here the main reason for the various degeneration theorems can be clearly isolated. The result in \S \ref{art4-sec4} stops far short of proving the existence of a mixed Hodge structure on the cohomology of a general singular variety \cite{art4-key14}. However, it is the method by which one most frequently \textit{calculates} this mixed Hodge structure (cf. \cite{art4-key10}, for instance), once it is known to exits.

In \S \ref{art4-sec5}, we have reproved the main result in the open case \cite{art4-key13} from a more analytic and less homological point of view. Our main idea is, instead of using the customary de Rham complex of $C^\infty$ forms on a compact K\"ahler manifold, to utilize a larger complex containing $L^1$-forms with certain precise types of singularities, and where the \textit{Gysin map} can be given on the form level preserving the Hodge filtration. This complex is discussed in \S\ref{art4-sec2}(b), where it is pointed out that the introduction of singular forms is necessary in order to have such a Gysin map on the form level. Operating inside this complex allows us to see clearly the differentials in the relevant spectral sequence in the open case, and to conclude the degeneracy result from the principle of two types (\S\S\ref{art4-sec5}(d), (e)).

Section 6 is devoted to some applications of Deligne's theory. First in \S \ref{art4-sec6}(a), we give his ``theorem on the fixed part'', which is the main tool in Deligne's study of the moduli of Hodge structures. Then, in \S\ref{art4-sec6}(b), we give a direct proof of an interesting result from \ref{art4-13}, concerning meromorphic differential forms on algebraic verieties; and finally we discuss an application of mixed Hodge structures to \textit{intermediate Jacobians} in \S \ref{art4-sec6}(c).

The second technique which we shall explore in some depth is the use of \textit{hyperbolic complex analysis}, as it applies to variation of Hodge structure. Hyperbolic complex analysis is the study of the influence of \textit{negative curvature} on holomorphic mappings. The classifying spaces for variation of Hodge structure are negatively curved, relative to the holomorphic maps which might arise in algebraic geometry (\cf. \cite{art4-key11}, \cite{art4-key25}, and \S \ref{art4-sec3}(a), (b)), and so it is natural to apply the general philosophy in this case.

Following a discussion of the basic \textit{Ahlfors lemma} and its variants in \S 
%%%%% 33 page






\begin{thebibliography}{99}
\bibitem{art4-key1}
\end{thebibliography}





