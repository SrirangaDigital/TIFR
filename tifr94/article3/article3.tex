
\title{ON THE COHOMOLOGY OF DISCRETE SUBGROUPS OF SEMI-SIMPLE LIE GROUPS}
\markright{ON THE COHOMOLOGY OF DISCRETE SUBGROUPS OF SEMI-SIMPLE LIE GROUPS}

\author{By~ HOWARD GARLAND}
\markboth{HOWARD GARLAND}{ON THE COHOMOLOGY OF DISCRETE SUBGROUPS OF SEMI-SIMPLE LIE GROUPS}

\date{}
\maketitle

%\setcounter{page}{21}
\setcounter{pageoriginal}{20}


\section{Introduction}%% 1
We\pageoriginale begin with a classical situation. Thus, let $M$ be a compact Riemannian manifold, let $C^q (M)$ denote the space of $C^\infty$ $q$-forms on $M$, and let $(\;,\;)$ denote the positive-definite inner product on $C^q (M)$, coming from the Riemannian metric $ds^2$. Let $d: C^q (M) \to C^{q+1} (M)$ denote exterior differentiation, let $\delta$ denote the adjoint of $d$ (with respect to $({\;},{\;})$) and let $\Delta$ denote the Laplacian $\Delta = d \delta + \delta d$, Let $\bH^q =$ kernel  $(\Delta : C^q (M) \to C^q (M))$ then if $H^q (M, \bR)$ denotes the $q^{\text{th}}$ cohomology group of $M$ with real coefficients, we have the Hodge-deRham Theorem:
\begin{equation}
H^1 \simeq H^q (M,\bR). \label{art3-eq1.1}
\end{equation}

At the same time, there is, after lifting to the orthonormal frame bundle, a well-known decomposition of the Laplacian (see \cite{art3-key4})
$$
\Delta = \Delta^+ + R.
$$
where $(\Delta^+ \Phi, \Phi)$, $\Phi \in C^q (M)$, is positive semi-definite, and $R$ is defined in terms of the Riemannian curvature. Thus, if $(R \Phi, \Phi)$ is positive-definite, we have that $H^q (M,\bR) = 0$, thanks to \ref{art3-eq1.1}. This idea is due to Bochner. A variant of this idea was applied by Calabi and Weil to the study of the cohomology of certain local coefficient systems, for the purpose of proving a rigidity theorem for discrete subgroups of Lie groups (see \cite{art3-key11}). Here, I wish to discuss a method for applying Bochner's idea to the study of the cohomology of discrete subgroups of $p$-adic groups. In particular, I will introduce a notion of ``$p$-adic curvature'', which plays a role in cohomology vanishing theorems for discrete subgroups of $p$-adic groups, analogous to the role played by $R$ in the argument sketched above. The details will appear in \cite{art3-key7}.

We mention that for rank 2 $p$-adic groups, the $p$-adic curvature coincides with a certain incidence matrix used by Feit, Higman, and Tits\pageoriginale to study finite simple groups (thus they proved the following \textit{rigidity theorem}: A finite $B-N$ pair of rank $\geqslant$ 3 is of Lie type!). Before giving a more extensive discussion of the $p$-adic case, we will begin with the case of discrete subgroups of real semi-simple Lie groups. This will motivate the analogy between the real and $p$-adic cases.

\section{The real case}%% 2
Let $G$ be a real semi-simple, linear Lie group with no compact factors, let $K \subset G$ be a maximal compact subgroup, and $\Gamma \subset G$ a discrete subgroup such that $G/\Gamma$ is compact. The space $X = K / \Gamma$ is topologically a cell. For simplicity, we assume $\Gamma$ is torsion-free. Then the action of $\Gamma$ on $X$ is free and proper, and hence $X \to X /\Gamma$ is a covering. Thus we have an isomorphism
$$
H^q (\Gamma, \bB) \simeq H^q (X/\Gamma, \bR),
$$
where $H^q (\Gamma, \bR)$ denotes the $q^{\text{th}}$ Eilenberg-MacLane group of $\Gamma$ with respect to trivial action on $\bR$. On the other hand, the space $X$ is a Riemannian symmetric space, and hence $X/\Gamma$ is a compact, Riemannian, locally symmetric space. Hence one is tempted to apply Bochner's idea, as described earlier, in order to compute the cohomology groups of $X/\Gamma$. However, in this case, the curvature form ($R \Phi$, $\Phi$) is negative, and hence Bochner's idea does not apply---at least not at first glance. Matsushima was not discouraged by this, and was inspired, in part by the computation of Calabi and Weil, to develop an ingenious modification of the Bochner idea. He then succeeded in calculating $H^q (X/\Gamma, \bR)$ in a large number of cases (see Matsushima \cite{art3-key9} and Nagano-Kaneyuki \cite{art3-key10}).

To give a rough description of Matsushima's results, we let $X_u$ denote the compact dual of $X$. Now the group $G$ acts on $X$ (to the right). We may identify $I^q$, the space of $G$-invariant differential $q$-forms on $X$, with a space of $q$-forms on $X/\Gamma$, and we do so whenever convenient. In fact, making this identification, $I^q$ is contained in the space of harmonic $q$-forms on $X/\Gamma$. On the other hand $I^q$ is isomorphic (in a natural way) to the space of all harmonic $q$-forms on $X_u$, and hence we have a natural map
$$
\varphi : H^q (X_u, \bR) \to H^q (X/ \Gamma, \bR).
$$
Matsushima\pageoriginale proved that in a large number of cases the map $\varphi$ is an isomorphism. Roughly, his idea was the following: he considered the lift of a harmonic form on $X/\Gamma$ to $G/\Gamma$, expanded the lifted form in terms of a basis of right invariant forms, and proved the derivatives with respect to right invariant vector fields, of the resulting coefficients were zero, \iec he applied the Bochner idea to the right-invariant derivatives of the coefficients. 

%%%% 23
