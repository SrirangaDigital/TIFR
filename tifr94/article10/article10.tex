\title{SOME ARITHMETICAL DISCRETE GROUPS IN LOBA$\hat{\text{C}}$EVSKI$\hat{\text{I}}$ SPACES}
\markright{SOME ARITHMETICAL DISCRETE GROUPS IN LOBA$\hat{\text{C}}$EVSKI$\hat{\text{I}}$ SPACES}

\author{By~ \`E. B. VINBERG}
\markboth{\'E. B. VINBERG}{SOME ARITHMETICAL DISCRETE GROUPS IN LOBA$\hat{\text{C}}$EVSKI$\hat{\text{I}}$ SPACES}

\date{}
\maketitle

\textsc{Some terminology.}\pageoriginale 

A reflection (in a vector space or in a simply connected Riemannian space of constant curvature)---a reflection with respect to a hyperplane (the mirror).

A reflection group---group generated by reflections. 

An integral quadratic from---a from 
$$
f(x) = \Sigma a_{ij} x_i x_j \text{ where } a_{ij} = a_{ji} \in \bZ.
$$

An integral automorphism, or a unit, of the form $f$---an integral linear transformation, which preserves this form.

\textsc{Introduction.} The subject of this report is an application of the theory of discrete reflection groups to the study of the groups of units of some indefinite integral quadratic forms.

The basic propositions of Coxeter's theory of discrete reflection groups in Euclidean spaces \cite{art10-key1} may be transferred without difficulty to discrete reflection grups in Loba$\hat{\text{c}}$evski$\hat{\text{i}}$ spaces. This enables us to find a fundamental polyhedron, generators and defining relations of any such group.

On the other hand, let $f$ be an integral quadratic form of signature $(n,1)$, \ie equivalent over $\bR$ to the form 
$$
f_n (x) = - x^2_0 + x^2_1 + \ldots + x^2_n.
$$
Then the group $\sO(f, \bZ)$ of units of the form $f$ or, more precisely, its subgroup of index 2, may be regarded as a discrete group of motions of $n$-dimensional
Loba$\hat{\text{c}}$evski$\hat{\text{i}}$ space. It is known that it has a fundamental domain of finite volume \cite{art10-key2}. If the group $\sO(f, \bZ)$ contains a reflection subgroup of finite index, we have the means for defining its fundamental domain, generators and relations.

An integral\pageoriginale quadratic form $f(x) = \Sigma a_{ij} x_i x_j$ is called unimodular if $\det (a_{ij}) = \pm 1$. For unimodular forms of signature $(n,1)$ the following two cases are possible \cite{art10-key6}:
\begin{itemize}
\item[(1)] $f$ is odd; then $f$ is equivalent over $\bZ$ to $f_n$;

\item[(2)] $f$ is even; then $n = 8k +1$ and when $n$ is fixed, all such forms are equivalent over $\bZ$.
\end{itemize}

For the group $\sO(f, \bZ)$ or units of an unimodular integral quadratic form $f$ of signature $(n,1)$ we shall prove the two following theorems. 


\begin{alphtheorem}\label{art10-alphathmA}
If  $n \leqslant 17$, the group $\sO (f, \bZ)$ contains a reflection subgroup of finite index.
\end{alphtheorem}

The fundamental polyhedron of the maximal reflection subgroup of $\sO (f, \bZ)$ will be described explicitly in all these cases.

\begin{alphtheorem}\label{art10-alphathmB}
If $n \geqslant 25$, the group $\sO (f, \bZ)$ contains no reflection subgroup of finite index.
\end{alphtheorem}

For $18 \leqslant n \leqslant 24$ the question is open.\footnote{J. M. Kaplinskaya has proved that the groups $\sO(f_{18}, \bZ)$ and $\sO (f_{19}, \bZ)$ contain a reflection subgroup of finite index. On the other hand, it follows from a consideration communicated to me by M. Kneser, and Theorem 3.5 below that the group $\sO (f_{20}. \bZ)$ contains no such subgroup. Thus the group $\sO (f, \bZ)$ (where $f$ is such as in the text) contains a reflection subgroup of finite index if and only if $n \leqslant 19$.}

A more detailed exposition of these and some other results will appear in \cite{art10-key13, art10-key14}.

\section{Discrete reflection groups.}\label{art10-sec1}

\begin{description}
\item[1] Let $X^n$ be an $n$-dimensional simply connected Riemannian space of constant curvature, \ie a sphere $S^n$, Euclidean space $E^n$ or Loba$\hat{\text{c}}$evski$\hat{\text{i}}$ space $\Lambda^n$.

Let $\Gamma$ be any discrete reflection group (d.r.g.) in the space $X^n$. The mirrors of all reflections belonging to $\Gamma$ decompose $X^n$ into $\Gamma$-equivalent convex polyhedra, called $\Gamma$-cells. Each of these cells is a fundamental domain for $\Gamma$.
 
For some\pageoriginale $\Gamma$-cell $P$, let us denote

$P_i (i \in I)$---all ($n-1$)-dimensional faces of $P$,

$H_i(i \in I)$---the corresponding hyperplanes,

$R_i(i \in I)$---the corresponding reflections.

It is known that 
\begin{itemize}
\item[(1)] for any pair $\{P_i, P_j\}$ of adjacent faces the angle between $P_i$ and $P_j$ is of the form $\dfrac{\pi}{n_{ij}}$, where $n_{ij} \in Z$;

\item[(2)] the $R_i$ generate $\Gamma$;

\item[(3)] the relations 
\end{itemize}
$$
R^2_i = 1, \quad (R_i R_j)^{n_{ij}} =1
$$
are defining relations for the $R_i$.

It is also known \cite{art10-key9, art10-key10} that if $P_i$ and $P_j$ are not adjacent, then $H_i$ and $H_j$ are either parallel, or diverging (in the case $X^n = \Lambda^n$).

Conversely, any convex polyhedron, all the dihedral angles of which are submultiples of $\pi$, is a cell of some d.r.g.

\textit{The Coxeter's diagram $\Sigma (\Gamma)$ of a} d.r.g. $\Gamma$. In the above notation, $\Sigma (\Gamma)$ is a graph with vertices $v_i (i \in I)$ which are joined as follows:
{
\tabcolsep=3pt
%\fontsize{9}{11}\selectfont
\setcounter{table}{0}
\begin{longtable}{@{}l|l@{}}
\hline
&  \multicolumn{1}{c}{the vertices $v_i$ and $v_j$} \\[-0.4cm]
\multicolumn{1}{c|}{if} & \\
& \multicolumn{1}{c}{are joined} \\\hline
$P_i$ and $P_j$ are adjacent and the & by an $(n_{ij}-2)$-tuple line or by a \\
angle between them is equal  & simple line with index $n_{ij}$\\
to $\pi / n_{ij}$ & \\\hline
$H_i$ and $H_j$ are parallel & by a thick line or by a simple\\
& line with index $\infty$\\\hline
$H_i$ and $H_j$ are diverging & by a dotted line\\\hline
\end{longtable}}\relax

If $P_i$ and $P_j$ are orthogonal $(n_{ij} =2)$, $v_j$ and $v_j$  are not joined. 

\textit{The cosines matrix}\pageoriginale $\cos \Gamma$ \textit{of a} d.r.g. $\Gamma$ is defined as follows:
$$
\cos \Gamma = \left(-\cos \frac{\pi}{n_{ij}} \right)
$$
where we put $n_{ij} =1$ and $\cos \dfrac{\pi}{n_{ij}} =1$ if $n_{ij}$ is not defined. Evidently the cosines matrix may be reconstructed after the Coxeter's diagram. 

\item[(2)] We shall consider now three remarkable classes of d.r.g.

\textsc{Finite reflection groups.} Such are all the d.r.g. in $S^n$ and those d.r.g. in $E^n$ and $\Lambda^n$, which have a fixed point. The number of generators of a finite reflection group is called its rank.

It is known \cite{art10-key1} that a d.r.g. is finite if and only if its cosines matrix is positive definite, so the finiteness property of a d.r.g. depends only on its diagram. 

A diagram of a finite reflection group is called an elliptic diagram. Its rank is by definition the rank of the corresponding group. It is equal to the number of vertices.

A Coxeter's diagram is elliptic if and  only if all its connected components are such. All the connected elliptic diagram are given in Table 1.

\medskip
\noindent
\textsc{Parabolic Reflection Groups.} A diagram of a d.r.g. in $E^n$ with a bounded cell is called a parabolic diagram or rank $n$. A d.r.g. is said to be parabolic reflection group of rank $n$ if its diagram is parabolic or rank $n$. For example, such is a d.r.g. in $\Lambda^n$ with a fixed improper point if it has a bounded cell on an orysphere with center at this point.

A Coxeter's diagram is parabolic if and only if all its connected components are such. Its rank is equal to the number of vertices minus the number of connected components. All connected parabolic diagrams are  given in Table 2.

It is known \cite{art10-key1} that the connected Coxeter's diagram is parabolic if and only if the corresponding cosines matrix is degenerate non-negative definite.

\medskip
\noindent
\textsc{Lanner's Groups.} In\pageoriginale the work \cite{art10-key3} by Lanner were firstly enumerated all d.r.g. in $\Lambda^n$ with simplicial bounded cells. We shall the diagrams of these groups the Lanner's diagrams. They are given in Table 3. Any d.r.g., whose Coxeter's diagram is a Lanner's diagram, will be called a Lanner's group. 

\item[(3)] Let $\Gamma$ be a d.r.g. in $X^n$. It is known that the stable subgroup $\Gamma_x \subset \Gamma$ of any point $x \in X^n$ is generated by reflections. More precisely, let $P$ be a $\Gamma$-cell containing $x$; $P_i$, $H_i$ and $R_i (i \in I)$  are the same as in 1.1. Let us denote by $H^-_i$ the halfspace bounded  by $H_i$ and containing $P$. So $P= \bigcap\limits_{i \in I} H^-_i$. If $J = \{i \in I : H_i \ni x\}$, then $\Gamma_x$ is generated by the reflections $R_i$, $i \in J$, and $P_x = \bigcap\limits_{i \in I} H^-_i$. If $J = \{i \in I: H_i \ni x\}$, then $\Gamma_x$ is generated by the reflections $R_i$, $i \in J$, and $P_x = \bigcap\limits_{i \in J} H^-_i$. If $J = \{i \in I: H_i \ni x\}$, then $\Gamma_x$ is generated by the reflections $R_i$, $i \in J$, and $P_x = \bigcap\limits_{i \in J}H^-_i$  is a $\Gamma_x$-cell.

In the case $X^n = \Lambda^n$ the above assertions hold true for some improper points namely those for which $\Gamma_x$ has a bounded fundamental domain on an orysphere with  center at $x$. Such improper points will be said to possess the compactness property with respect to $\Gamma$. 

\item[(4)] Let $\Theta$ be an arbitrary discrete group in $X^n$. We denote by $\Gamma$ the group generated by all reflections belonging to $\Theta$. Let $P$ be a $\Gamma$-cell and Sym $P$ be the symmetry group of $P$.

It is trivial that $\Gamma$ is a normal subgroup of $\Theta$ and that $\Theta$ is a semi-direct product 
\begin{equation}
\Theta = \Gamma \cdot H, \label{art10-eq1.1} 
\end{equation}
where $H$ is some subgroup in Sym $P$.

How to find $P$? Let us fix a point $x_0 \in X^n$ and suppose that for any $r >0$ we can enumerate all reflections in $\Theta$, whose mirrors $H$ satisfy the condition $\rho (x_0, H) \leqslant r$, where $\rho$ denotes the distance. Then we can determine $P$ as described below.

\medskip
\textsc{Some conventions.} For any hyperplane $H$, we shall denote by $H^+$ and $H^-$ the halfspaces bounded by $H$. We shall say that halfspaces $H^-_1$ and $H^-_2$ are opposite if one of the following cases take place:
\begin{itemize}
\item[(1)] $H_1$ and $H_2$ are\pageoriginale crossing and the dihedral angle $H^-_1 \cap H^-_2$ does not exceed $\dfrac{\pi}{2}$;

\item[(2)] $H^-_1 \supset H_2$ and $H^-_2 \supset H_1$;

\item[(3)] $H^-_1 \cap H^-_2 = \empty$ .
\end{itemize}

\medskip
\noindent
\textsc{An algorithm for constructing a $\Gamma$-cell.}

``Firs we consider the stable subgroup $\Gamma_0$ of $x_0$ in $\Gamma$, \ie the group generated by all reflections in $\Theta$, whose mirrors contain $x_0$. Let 
$$
P_0 \bigcap\limits^k_{i} H_i
$$
be some $\Gamma_0$-cell, each of the $H_i$ being essential. There exists a unique $\Gamma$-cell containing $x_0$ and contained in $P_0$. This $\Gamma$-cell we denote by $P$.

Now we shall construct one by one hyperplanes $H_{k+1}, H_{k+2}, \ldots$ and halfspaces $H^-_{k+1}, H^-_{k+2} \ldots$ such that 
$$
P = \bigcap\limits_{\text{all } i} H^-_i
$$
each of the $H^-_j$ being essential. The $H_i$ will be ordered by increase of $\rho (x_0, H_i)$.

Thus rules for constructing $H_m$ and $H^-_m$ for $m\geqslant k +1$ are the following.
\begin{itemize}
\item[($1^0$).] $H^-_m$ is that of two halfspaces bounded by $H_m$, which contains $x_0$.

\item[($2^0$).] If the $H^-_i$ with $i < m$ have been constructed, then $H_m$ is chosen as the nearest to $x_0$ mirror of a reflection belonging to $\Theta$ for which the halfspaces $H^-_m$ and $H^-_i$ are opposite for all $i < m$.
\end{itemize}

The procedure may be finite or infinite.''

In the rule $2^0$ one may consider only those $H_i$ for which $\rho(x_0, H_i) < \rho (x_0 , H_m)$.

In the case $X^n = \Lambda^n$ one may take for $x_0$ an improper point possessing the compactness property with respect to $\Gamma$. The Algorithm remains in force if we replace $\rho (x_0, H) {}^{\text{def}}_= \min\limits_{x \in H} \rho (x_0, x) $ by $b(x_0, H) {}^{\text{def}}_= \min\limits_{x \in H} b(x_0 ,x)$,\pageoriginale where $b(x_0, x)$ is a positive function satisfying the following conditions:
\begin{itemize}
\item[(a)] for each motion $\varphi$ of space $\Lambda^n$ leaving $x_0$ fixed, there exists such a number $c > 0$ that
$$
b (x_0 , \varphi x) = c b( x_0 , x)
$$
for all $x \in \Lambda^n$;

\item[(b)] $x_p \to x_0$, $x_p \in \Lambda^n$, implies $b(x_0 , x_p) \to 0$. 
\end{itemize}

(Explicit formulas for $b (x_0, x)$ and $b(x_0, H)$ see in \ref{art10-eq2.1}).

Finding the Algorithm is not much longer than its formulation. Obviously $P\subset \cap H^-_i$. It is sufficient to prove that each of the $H_i$ bounds $P$. Let $m$ be the smallest index, for which $H_m$ does not bound $P$, and let $x$ be the point of $H_m$ nearest to $x_0$. It is easy to prove.
\end{description}

\begin{lemma}\label{art10-lem1.4}
Let $\{H^-_i\}$ be a set of halfspaces, each two of them being opposite. If $x_0 \in \cap H^-_i$ and $x$ is the point of $H_m$ nearest to $x_0$, then $x \in \cap H^-_i$. Moreover, $x \in H_i$, $i \neq m$, implies $x_0 \in H_i$.
\end{lemma}

Thus $x \in \cap H^-_i$. From the rule $2^0$ of choosing $H_m$ we can deduce that neither hyperplane bounding $P$ separates $x$ and $x_0$. Furthermore if a hyperplane bounding $P$ contains $x$, then it is one of the hyperplanes $H_1,\ldots, H_k$. Hence $P$ contains some neighbourhood of $x$ in $P_0$. This is evidently impossible.

If the procedure by the Algorithm is finite, how to recognize its end? This problem is of peculiar interest for the case $X^n = \Lambda^n$.

We shall discuss it in 2.4.

\section{The Gram matrix of a convex polyhedron in Loba$\hat{\text{c}}$evski$\hat{\text{i}}$ space.}\label{art10-sec2}

\begin{description}
\item[(1)] A \textsc{model of Loba$\hat{\textsc{c}}$evski$\hat{\textsc{i}}$ space.} Let $E^{n,1}$ be an $(n+1)$-dimensional vector space with scalar multiplication of signature $(n,1)$. We have
$$
\{v \in E^{n,1}: (v, v) < 0\} = \fC \cup (-\fC),
$$
where $\fC$ is an open convex cone. Let us denote by $\bP$ the group of positive real numbers. Then we identify $n$-dimensional Loba$\hat{\text{c}}$evski$\hat{\text{i}}$ space $\Lambda^n$ with$\fC/\bP$ in such a way that motions of $\Lambda^n$ are induced by orthogonal transformations of $E^{n,1}$ preserving $\fC$.

If we consider\pageoriginale ``projective sphere'' 
\end{description}

%%%% 130 page



%\begin{thebibliography}{99}
%\bibitem{}
%\end{thebibliography}
