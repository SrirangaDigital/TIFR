
\title{DISCRETE GROUPS AND Q-STRUCTURES ON SEMI-SIMPLE LIE GROUPS}
\markright{DISCRETE GROUPS AND Q-STRUCTURES ON SEMI-SIMPLE LIE GROUPS}

\author{By~ M. S. RAGHUNATHAN}
\markboth{M. S. RAGHUNATHAN}{DISCRETE GROUPS AND Q-STRUCTURES ON SEMI-SIMPLE LIE GROUPS}

\date{}
\maketitle


%\setcounter{page}{21}
\setcounter{pageoriginal}{224}

\noindent
\textsc{Introduction.} The\pageoriginale main aim of this paper is to establish the following result.

\begin{maintheorem*}
Let $G$ be a connected linear semisimple algebraic group defined over $\bR$ of $\bR$-rank $\geqslant$ 2 and with trivial centre. Let $G$ be the connected component of the identity in $\bG_{\bR}$, the group of $\bR$-rational points of $\bG$. Assume that $G$ has no non-trivial compact connected normal subgroup. Let $\Gamma \subset G$ be a discrete subgroup such that $G/\Gamma$ is non-compact ans has finite Haar measure. Assume further that $\Gamma \cap G'$ is finite for every connected proper normal subgroup $G'$ of $G$. Then there exists a $\bQ$-algebraic group $\bG^\ast$ and an isomorphism (defined over $\bR$)  $f: \bG^\ast \to \bG$ such that $f^{-1} (\Gamma) \subset G^\ast_\bQ$ and for every proper $\bQ$-parabolic subgroup $\bP$ of $G^\ast$, $\bP \cap f^{-1} (\Gamma)$ is an arithmetic subgroup of $\bP$.

If in addition, there exists a unipotent $\theta \in \Gamma$ which is contained in two distinct maximal unipotent subgroups of $\theta \in \Gamma$ which is contained in two distinct maximal unipotent subgroups of $\Gamma$, then $f^{-1}(\Gamma)$ is an arithmetic subgroup of $\bG^\ast$; in this case, $\bQ$-rank $\bG^\ast \geqslant 2$.
\end{maintheorem*}

The main theorem with the exception of the last assertion was announced by Margolis \cite{art9-key1} in 1969. (In a recent letter Margolis informs the author that he has proved the arithmeticity of $f^{-1}(\Gamma)$ in all cases.) As no proofs were available for a long time the present author pursued the problem on his own despite the announcement. Most of the work in this paper was done while the author was visiting at Osaka University during March-June 1972. The author would like to take this opportunity to express his warmest thanks to the J.S.P.S. and Professor Murakami and his colleagues for their king hospitality. Later, in January 1973, the author spoke on the work at the International Colloquium on ``Discrete subgroups of Lie groups'' held in Bombay.\footnote{The proofs given here are somewhat different from those outlined in the talk.}

Margolis' proof is apparently based on the following result:

\begin{theorem*}
Let $X \in M(n, \bR)$\pageoriginale be a nilpotent matrix. Then the map $f: \bR \to SL (n, \bR)/ SL (n, \bZ)$ defined by $f(t) = \pi(\exp t X)$, where $\pi: SL (n, \bR) \to S L (n, \bR)/ SL (n,\bZ)$ is the natural map, is not proper.
\end{theorem*}

(A proof of the theorem by Margolis \cite{art9-key2} has appeared. The result was conjectured first by Pjatetski-Shapiro \cite{art9-key1} and some years later, independently by H. Garland and the author (see Raghunathan \cite{art9-key4})).

The author tried without success to prove this result and eventually obtained the main theorem without the use of the lemma. Some parts of this paper especially in \S \ref{art9-sec2} can perhaps be shortened with the aid of this theorem; but it does not help shorten the paper on the whole significantly. A theorem proved by the author (Raghunathan \cite{art9-key1}, Theorem 11.16) can in some ways be regarded as the starting point. Repeated use is also made of the basic result on the existence of unipotent elements in discrete groups due to Kazdan-Margolis \cite{art9-key1}.

We now indicate how the main result is proved explaining in the process also how the paper is organised.

The main result above is formulated for lattices. However, we will work with what we call ``$L$-subgroups''. The class of $L$-subgroups is \textit{a priori} a wider class than that of lattices. (However, at the end of the paper one finds that $L$-subgroups are indeed lattices). In \S \ref{art9-sec1} we generalise Borel's density theorem (Borel \cite{art9-key1}) to $L$-subgroups. A justification for including this material here can perhaps be found in the following: that arithmetic subgroups are $L$-subgroups is essentially the Godement compactness-criterion which is only the first step towards proving that arithmetic groups are indeed lattices.\footnote{Curiously enough, we never use the fact, that arithmetic groups are lattices, in this paper.} The impatient reader can skip this section beyond \S \ref{art9-subsec1.8} altogether and go on to \S \ref{art9-sec2}. He need only read ``lattice'' for ``$L$-subgroup'' everywhere.

The material in \S \ref{art9-sec2} contains the most difficult steps.  We prove here the following (notation as in the statement of the main theorem):

There exists an $\bR$-parabolic subgroup $\bN$ of $\bG$ and a normal $\bR$-subgroup $\bN_0$ of codimension  1 in $\bN$ containing the unipotent radical\pageoriginale  $\bF$ of $\bN$ with the following properties: let $\bL$ be the Zariski-closure of $\bN \cap \Gamma$  in $\bG$ and $\bE$ the centre of $\bF$; then $\bL/\bE$ and $\bF$ carry natural $\bQ$-structures such that (a) $\bF \cap \Gamma$ (\resp. image of $\bL\cap \Gamma$ in $\bL/ \bE$)is arithmetic and (b) the natural action of $\bL/\bE$ on $\bF$ is defined over $\bQ$; moreover if $L = \bL \cap G$ and $N_0 = \bN_0 \cap G$, then $L \subset N_0$ and $N_0/ L$ is compact.

The proof is involved and rather difficult to outline. It makes use of the techniques---not merely the final results---of Kazdan and Margolis \cite{art9-key1}. In this section, we also introduce the notion of rank for an $L$-subgroup. $L$-subgroups of rank 1 and rank $> 1$ are treated separately in the subsequent sections.

\S\ref{art9-sec3} begins with the following characterisation of $L$-subgroups of rank 1. 

$\Gamma \subset G$ is ($L$-subgroup) of rank 1 if and only if every unipotent ($\neq$ identity) in $\Gamma$ is contained in a unique maximal unipotent subgroup of $\Gamma$.

The case of rank 1 $L$-subgroups is then handled as follows. One starts with a parabolic subgroup $\bN$ as described above and shows that a conjugate of $\bN$ by any element of $\Gamma$ not in $\bN$ is opposed to $\bN$; as a consequence one obtains a semi-direct product decomposition $\bN = \bM. \bF$ with $\bF$ the unipotent radical such that $(\bM \cap \Gamma) (\bF \cap \Gamma)$  has finite index in $\bN \cap \Gamma$. Also the assumption that $\bR$-rank $\bG>1$ guarantees that $\bM \cap \Gamma$ is sufficiently big (to guarantee that its centraliser in $\bF$ is in the centre of $\sF$). If $\bM_1$ is the Zariski-closure of $\bM \cap \Gamma$, then $\bD = \bM_1 \cdot \bF$ has a natural $\bQ$-structure in which $\bD \cap \Gamma$ is arithmetic. Now let $\theta \in \Gamma -N$; then $\bM^\theta_1 = \theta \bN \theta^{-1} \cap \bD$ is shown to be a maximal reductive $\bQ$-subgroup of $\bD$ so that one can find $x$, $y \in \bF_\bQ$ such that $x \bM^\theta_1 x^{-1} = \bM_1$, $y \bM_1 y^{1} = \theta^{-1} (\bM^\theta_1) \theta$, leading to: $x\theta^{-1} y$ is in the normaliser of $\bM_1$. From this we immediately obtain, what one expects should be the Bruhat-decomposition of elements of $\Gamma$. Using this decomposition, it is then shown that trace $(\ad(\Ad \theta (X)) \ad Y) \in \bQ$, for every pair $X$, $Y \in \ff$ (= Lie algebra of $\bF$) such that exp $X$, exp $Y \in \bF \cap \Gamma$.

\S \S \ref{art9-sec4}-\ref{art9-sec5} are independent of \S \ref{art9-sec3} (and can be read without reading \S \ref{art9-sec3}).

In the\pageoriginale higher rank case we start again with a parabolic group $\bN$ as above. The problem of finding $\bM \subset \bN$ such that $\bM \cap \Gamma$. $\bF \cap \Gamma$ has finite index in $\bN \cap \Gamma$, now presents certain difficulties (because no conjugate of $\bN$ need be opposed to $\bN$). However in \S \ref{art9-sec4},  we prove the main theorem for $L$-subgroups of rank $\geqslant 2$ under the additional hypothesis that we can indeed find such an $\bM$ in $\bN$. Under this additional assumption, one constructs a complete set of what one would expect to be the conjugacy classes maximal $\bQ$-parabolics---if there was a $\bQ$-structure on $\bG$ with $\Gamma$ arithmetic. Once this is done, there are ``enough'' unipotents to play around with to obtain the main theorem (including the arithmeticity of $\Gamma$).

The problem of finding $\bM$ as described in the paragraph above is taken up in \S \ref{art9-sec5}. It is interpreted as a cohomology-vanishing result and the requisite vanishing theorem are proved when $\bN$ is not conjugate to any opposing parabolic subgroup. Where $\bN$ is conjugate to an opposing group, the problem is handled as in the rank 1 case.

The appendix contains the proofs of two results used in the main body of the paper.

The following notational conventions are used.  As usual $\bQ$ (\resp. $\bR$, \resp. $\bC$) denotes the field of rational (\resp. real, resp. complex) numbers and $\bZ$, the ring of integers. By an algebraic group we mean a complex algebraic subgroup of $GL(n, \bC)$ in general. However (especially in \S \ref{art9-sec1} and \ref{art9-sec2}) the term algebraic group sometimes refers to the group of $\bR$-rational points $\bG_\bR$ of an algebraic group $\bG$ defined over $\bR$ or even a subgroup of $\bG_\bR$ containing the identity component of $\bG_\bR$. Correspondingly the meaning of terms like ``Zariski-dense'' and ``Zariski-closure'' will depend on the context. To avoid (or atleast reduce the possibilities of) confusion, we adopt the following convention: bold face capital roman letters are used to denote complex algebraic groups while real Lie groups are usually denoted by ordinary capital \textit{italic} letters. Further, if $\bG$ is an algebraic group defined over $\bR$ and $H \subset G_\bR$ is any ``algebraic'' subgroup, $\bH$, the corresponding bold face letter is used to denote its Zariski-closure in $\bG$. Greek capital letters are usually used to denote discrete groups: as far as possible we use corresponding roman (bold face) capital letters\pageoriginale  to denote their ``real'' (complex) Zariski-closures (in algebraic groups that contain them). Lie algebras over $\bR$ (\resp. $\bC$) are denoted by gothic lower case (\resp. capital gothic) letters. The Lie algebra of a Lie group is denoted by the gothic equivalent of the roman letter used to denote the group. For a Lie group $H$, $H^0$ denotes its identity component. (On the whole the notation used is inevitably somewhat unwieldy and, possibly a little confusing; the author seeks the reader's indulgence for this).

Standard results on algebraic groups are used without giving any references. They can in any event be found in Borel \cite{art9-key3} or Borel-Tits \cite{art9-key2}. Proofs of most of the results on discrete groups used in this paper can be found in Raghunathan \cite{art9-key1}. Some results of which we make frequent use are listed below for convenient reference.

\begin{romanlemma}\label{art9-romanlem1}
Let $G$ be a Lie group and $\Gamma \subset G$ a discrete subgroup. Let $\Phi \subset \Gamma$ be any finite set and $Z$ be the centraliser of $\Phi$ in $G$. Then the map $Z/ Z \cap \Gamma \to G/ \Gamma$ is proper.
\end{romanlemma}

(\begin{proof}
It suffices to show that $Z\Gamma$ is closed in $G$. Let $z_n \in Z$, $\gamma_n \in \Gamma$ be sequences such that $z_n \gamma_n$ converges to a limit
\end{proof})


%%%% 229 page 







\begin{thebibliography}{99}
\bibitem{art9-key1} 
\end{thebibliography}

