
\title{DISCRETE GROUPS AND Q-STRUCTURES ON SEMI-SIMPLE LIE GROUPS}
\markright{DISCRETE GROUPS AND Q-STRUCTURES ON SEMI-SIMPLE LIE GROUPS}

\author{By~ M. S. RAGHUNATHAN}
\markboth{M. S. RAGHUNATHAN}{DISCRETE GROUPS AND Q-STRUCTURES ON SEMI-SIMPLE LIE GROUPS}

\date{}
\maketitle


%\setcounter{page}{21}
\setcounter{pageoriginal}{224}

\noindent
\textsc{Introduction.} The\pageoriginale main aim of this paper is to establish the following result.

\begin{maintheorem*}
Let $G$ be a connected linear semisimple algebraic group defined over $\bR$ of $\bR$-rank $\geqslant$ 2 and with trivial centre. Let $G$ be the connected component of the identity in $\bG_{\bR}$, the group of $\bR$-rational points of $\bG$. Assume that $G$ has no non-trivial compact connected normal subgroup. Let $\Gamma \subset G$ be a discrete subgroup such that $G/\Gamma$ is non-compact ans has finite Haar measure. Assume further that $\Gamma \cap G'$ is finite for every connected proper normal subgroup $G'$ of $G$. Then there exists a $\bQ$-algebraic group $\bG^\ast$ and an isomorphism (defined over $\bR$)  $f: \bG^\ast \to \bG$ such that $f^{-1} (\Gamma) \subset G^\ast_\bQ$ and for every proper $\bQ$-parabolic subgroup $\bP$ of $G^\ast$, $\bP \cap f^{-1} (\Gamma)$ is an arithmetic subgroup of $\bP$.

If in addition, there exists a unipotent $\theta \in \Gamma$ which is contained in two distinct maximal unipotent subgroups of $\theta \in \Gamma$ which is contained in two distinct maximal unipotent subgroups of $\Gamma$, then $f^{-1}(\Gamma)$ is an arithmetic subgroup of $\bG^\ast$; in this case, $\bQ$-rank $\bG^\ast \geqslant 2$.
\end{maintheorem*}

The main theorem with the exception of the last assertion was announced by Margolis \cite{art9-key1} in 1969. (In a recent letter Margolis informs the author that he has proved the arithmeticity of $f^{-1}(\Gamma)$ in all cases.) As no proofs were available for a long time the present author pursued the problem on his own despite the announcement. Most of the work in this paper was done while the author was visiting at Osaka University during March-June 1972. The author would like to take this opportunity to express his warmest thanks to the J.S.P.S. and Professor Murakami and his colleagues for their king hospitality. Later, in January 1973, the author spoke on the work at the International Colloquium on ``Discrete subgroups of Lie groups'' held in Bombay.\footnote{The proofs given here are somewhat different from those outlined in the talk.}

Margolis' proof is apparently based on the following result:

\begin{theorem*}
Let $X \in M(n, \bR)$\pageoriginale be a nilpotent matrix. Then the map $f: \bR \to SL (n, \bR)/ SL (n, \bZ)$ defined by $f(t) = \pi(\exp t X)$, where $\pi: SL (n, \bR) \to S L (n, \bR)/ SL (n,\bZ)$ is the natural map, is not proper.
\end{theorem*}

(A proof of the theorem by Margolis \cite{art9-key2} has appeared. The result was conjectured first by Pjatetski-Shapiro \cite{art9-key1} and some years later, independently by H. Garland and the author (see Raghunathan \cite{art9-key4})).

The author tried without success to prove this result and eventually obtained the main theorem without the use of the lemma. Some parts of this paper especially in \S \ref{art9-sec2} can perhaps be shortened with the aid of this theorem; but it does not help shorten the paper on the whole significantly. A theorem proved by the author (Raghunathan \cite{art9-key1}, Theorem 11.16) can in some ways be regarded as the starting point. Repeated use is also made of the basic result on the existence of unipotent elements in discrete groups due to Kazdan-Margolis \cite{art9-key1}.

We now indicate how the main result is proved explaining in the process also how the paper is organised.

The main result above is formulated for lattices. However, we will work with what we call ``$L$-subgroups''. The class of $L$-subgroups is \textit{a priori} a wider class than that of lattices. (However, at the end of the paper one finds that $L$-subgroups are indeed lattices). In \S \ref{art9-sec1} we generalise Borel's density theorem (Borel \cite{art9-key1}) to $L$-subgroups. A justification for including this material here can perhaps be found in the following: that arithmetic subgroups are $L$-subgroups is essentially the Godement compactness-criterion which is only the first step towards proving that arithmetic groups are indeed lattices.\footnote{Curiously enough, we never use the fact, that arithmetic groups are lattices, in this paper.} The impatient reader can skip this section beyond \S \ref{art9-subsec1.8} altogether and go on to \S \ref{art9-sec2}. He need only read ``lattice'' for ``$L$-subgroup'' everywhere.

The material in \S \ref{art9-sec2} contains the most difficult steps.  We prove here the following (notation as in the statement of the main theorem):

There exists an $\bR$-parabolic subgroup $\bN$ of $\bG$ and a normal $\bR$-subgroup $\bN_0$ of codimension  1 in $\bN$ containing the unipotent radical\pageoriginale  $\bF$ of $\bN$ with the following properties: let $\bL$ be the Zariski-closure of $\bN \cap \Gamma$  in $\bG$ and $\bE$ the centre of $\bF$; then $\bL/\bE$ and $\bF$ carry natural $\bQ$-structures such that (a) $\bF \cap \Gamma$ (\resp. image of $\bL\cap \Gamma$ in $\bL/ \bE$)is arithmetic and (b) the natural action of $\bL/\bE$ on $\bF$ is defined over $\bQ$; moreover if $L = \bL \cap G$ and $N_0 = \bN_0 \cap G$, then $L \subset N_0$ and $N_0/ L$ is compact.

The proof is involved and rather difficult to outline. It makes use of the techniques---not merely the final results---of Kazdan and Margolis \cite{art9-key1}. In this section, we also introduce the notion of rank for an $L$-subgroup. $L$-subgroups of rank 1 and rank $> 1$ are treated separately in the subsequent sections.

\S\ref{art9-sec3} begins with the following characterisation of $L$-subgroups of rank 1. 

$\Gamma \subset G$ is ($L$-subgroup) of rank 1 if and only if every unipotent ($\neq$ identity) in $\Gamma$ is contained in a unique maximal unipotent subgroup of $\Gamma$.

The case of rank 1 $L$-subgroups is then handled as follows. One starts with a parabolic subgroup $\bN$ as described above and shows that a conjugate of $\bN$ by any element of $\Gamma$ not in $\bN$ is opposed to $\bN$; as a consequence one obtains a semi-direct product decomposition $\bN = \bM. \bF$ with $\bF$ the unipotent radical such that $(\bM \cap \Gamma) (\bF \cap \Gamma)$  has finite index in $\bN \cap \Gamma$. Also the assumption that $\bR$-rank $\bG>1$ guarantees that $\bM \cap \Gamma$ is sufficiently big (to guarantee that its centraliser in $\bF$ is in the centre of $\sF$). If $\bM_1$ is the Zariski-closure of $\bM \cap \Gamma$, then $\bD = \bM_1 \cdot \bF$ has a natural $\bQ$-structure in which $\bD \cap \Gamma$ is arithmetic. Now let $\theta \in \Gamma -N$; then $\bM^\theta_1 = \theta \bN \theta^{-1} \cap \bD$ is shown to be a maximal reductive $\bQ$-subgroup of $\bD$ so that one can find $x$, $y \in \bF_\bQ$ such that $x \bM^\theta_1 x^{-1} = \bM_1$, $y \bM_1 y^{1} = \theta^{-1} (\bM^\theta_1) \theta$, leading to: $x\theta^{-1} y$ is in the normaliser of $\bM_1$. From this we immediately obtain, what one expects should be the Bruhat-decomposition of elements of $\Gamma$. Using this decomposition, it is then shown that trace $(\ad(\Ad \theta (X)) \ad Y) \in \bQ$, for every pair $X$, $Y \in \ff$ (= Lie algebra of $\bF$) such that exp $X$, exp $Y \in \bF \cap \Gamma$.

\S \S \ref{art9-sec4}-\ref{art9-sec5} are independent of \S \ref{art9-sec3} (and can be read without reading \S \ref{art9-sec3}).

In the\pageoriginale higher rank case we start again with a parabolic group $\bN$ as above. The problem of finding $\bM \subset \bN$ such that $\bM \cap \Gamma$. $\bF \cap \Gamma$ has finite index in $\bN \cap \Gamma$, now presents certain difficulties (because no conjugate of $\bN$ need be opposed to $\bN$). However in \S \ref{art9-sec4},  we prove the main theorem for $L$-subgroups of rank $\geqslant 2$ under the additional hypothesis that we can indeed find such an $\bM$ in $\bN$. Under this additional assumption, one constructs a complete set of what one would expect to be the conjugacy classes maximal $\bQ$-parabolics---if there was a $\bQ$-structure on $\bG$ with $\Gamma$ arithmetic. Once this is done, there are ``enough'' unipotents to play around with to obtain the main theorem (including the arithmeticity of $\Gamma$).

The problem of finding $\bM$ as described in the paragraph above is taken up in \S \ref{art9-sec5}. It is interpreted as a cohomology-vanishing result and the requisite vanishing theorem are proved when $\bN$ is not conjugate to any opposing parabolic subgroup. Where $\bN$ is conjugate to an opposing group, the problem is handled as in the rank 1 case.

The appendix contains the proofs of two results used in the main body of the paper.

The following notational conventions are used.  As usual $\bQ$ (\resp. $\bR$, \resp. $\bC$) denotes the field of rational (\resp. real, resp. complex) numbers and $\bZ$, the ring of integers. By an algebraic group we mean a complex algebraic subgroup of $GL(n, \bC)$ in general. However (especially in \S \ref{art9-sec1} and \ref{art9-sec2}) the term algebraic group sometimes refers to the group of $\bR$-rational points $\bG_\bR$ of an algebraic group $\bG$ defined over $\bR$ or even a subgroup of $\bG_\bR$ containing the identity component of $\bG_\bR$. Correspondingly the meaning of terms like ``Zariski-dense'' and ``Zariski-closure'' will depend on the context. To avoid (or atleast reduce the possibilities of) confusion, we adopt the following convention: bold face capital roman letters are used to denote complex algebraic groups while real Lie groups are usually denoted by ordinary capital \textit{italic} letters. Further, if $\bG$ is an algebraic group defined over $\bR$ and $H \subset G_\bR$ is any ``algebraic'' subgroup, $\bH$, the corresponding bold face letter is used to denote its Zariski-closure in $\bG$. Greek capital letters are usually used to denote discrete groups: as far as possible we use corresponding roman (bold face) capital letters\pageoriginale  to denote their ``real'' (complex) Zariski-closures (in algebraic groups that contain them). Lie algebras over $\bR$ (\resp. $\bC$) are denoted by gothic lower case (\resp. capital gothic) letters. The Lie algebra of a Lie group is denoted by the gothic equivalent of the roman letter used to denote the group. For a Lie group $H$, $H^0$ denotes its identity component. (On the whole the notation used is inevitably somewhat unwieldy and, possibly a little confusing; the author seeks the reader's indulgence for this).

Standard results on algebraic groups are used without giving any references. They can in any event be found in Borel \cite{art9-key3} or Borel-Tits \cite{art9-key2}. Proofs of most of the results on discrete groups used in this paper can be found in Raghunathan \cite{art9-key1}. Some results of which we make frequent use are listed below for convenient reference.

\begin{romanlemma}\label{art9-romanlem1}
Let $G$ be a Lie group and $\Gamma \subset G$ a discrete subgroup. Let $\Phi \subset \Gamma$ be any finite set and $Z$ be the centraliser of $\Phi$ in $G$. Then the map $Z/ Z \cap \Gamma \to G/ \Gamma$ is proper.
\end{romanlemma}

\begin{proof}
(It suffices to show that $Z\Gamma$ is closed in $G$. Let $z_n \in Z$, $\gamma_n \in \Gamma$ be sequences such that $z_n \gamma_n$ converges to a limit $x \in G$. For $\theta \in \Phi$, $\gamma^{-1}_n z^{-1}_n \theta z_n \gamma_n = \gamma_n^{-1} \theta \gamma_n$ is a convergent sequence contained in $\Gamma$. Since $\Phi$ is finite, there exists $m$ such that $\gamma^{-1}_n \theta \gamma_n = \gamma^{-1}_m \theta \gamma_m$ for all $\theta \in \Phi$ and $n \geqslant m$ \ie $\gamma_n \gamma^{-1}_m \in Z$ consequently $x = x \gamma^{-1}_m \gamma_m \in Z \gamma$).
\end{proof}

\begin{romantheorem}{\rm (Malcev \cite{art9-key1}).}\label{art9-romanthm2}
Let $N$ be a nilpotent connected simply connected Lie group and $\fn$ its Lie algebra and $\exp: \fn \to N$, the exponential map. Let $\Gamma \subset N$ be any discrete subgroup. Let $\bN$ be a unipotent algebraic subgroup such that $N \subset \bN_\bR$. Then we have the following results.
\begin{itemize}
\item[(i)] $\Gamma$ is finitely generated.

\item[(ii)] $N/\Gamma$ is compact if and only if $\Gamma$ is Zariski-dense in $\bN$.

\item[(iii)] If $N/ \Gamma$ is compact, the $\bZ$-linear span of $\exp^{-1} (\Gamma)$ is a lattice in the (vector space) $\fn$. The $\bQ$-Linear span of $\exp^{-1} (\Gamma)$ is a $\bQ$-Lie subalgebra of $\fn$. Consequently $\bN$ acquires a $\bQ$-structure. For a subgroup $U \subset N$, $U/U \cap \Gamma$ is compact if and only if the Zariski-closure $\bU$ of $\bU$\pageoriginale is defined over $\bQ$. In particular if $E$ denotes the centre of $N$, $E/E \cap \Gamma$ is compact.
\end{itemize}
\end{romantheorem}

Proofs are also available in Raghunathan (\cite{art9-key1}, Chapter II).

\begin{romantheorem}{\rm (Auslander \cite{art9-key1}, Wang \cite{art9-key1}).}\label{art9-romanthm3}
Let $G$ be a connected Lie group and $\Gamma$ a lattice in $G$. Let $R$ (\resp. $N$) be the radical (\resp. maximal connected nilpotent normal subgroup of $G$). Let $S$ be a maximal connected semisimple subgroup of $G$ and $S'$ the kernel of the action of $S$ on $R$. If $S'$ has no nontrivial compact connected normal subgroup, then $R/R \cap \Gamma$ and $N/N \cap \Gamma$ are compact.
\end{romantheorem}

A proof is given in Chapter VIII of Raghunathan (\cite{art9-key1}, Corollary 8.28).

\begin{romantheorem}{\rm (Garland-Raghunathan \cite{art9-key1}).}\label{art9-romanthm4}
Let $G$ be a connected linear semisimple Lie group and $U \subset G$ a connected unipotent subgroup. Let $Z$ be the centraliser of $U$ in $G$ and $B$ a maximal reductive subgroup of $Z$. Then the kernel of the action of $B$ on $U$ is normal in $G$. If $U$ is not contained in any proper normal subgroup of $G$, this kernel is discrete (hence central).
\end{romantheorem}

A proof can also be found in Chapter XI of Raghunathan (\cite{art9-key1}, Proposition 11.19).

\begin{romantheorem}{\rm (Borel-Tits \cite{art9-key2})}\label{art9-romanthm5}
Let $G$ be a connected linear semisimple Lie group and $U$ be a unipotent subgroup. Let $P$ be the normaliser of $U$. Let $\fp$ (\resp $\fu$) be the Lie algebra of $P$ (\resp $\bU$). If $U$ is the unipotent radical of $P$, we can find $X \in \fg$ with the following properties:
\begin{itemize}
\item[(i)] $\ad X$ is semisimple and all its eigen-values are real;

\item[(ii)] for $\lambda \in \bR$, let $\fg(\lambda) = \{v \in \fg  \big| [X, v] = \lambda v\}$; then
$$
\fu = \sum\limits_{\lambda > 0} \fg (\lambda), \;\; \fp = \sum\limits_{\lambda \geq 0 } \fg (\lambda).
$$
\end{itemize}
 \end{romantheorem}

For a proof see Borel-Tits \cite{art9-key1} (The theorem proved in this reference is for algebraic groups, but the present formulation is an immediate consequence; see also Mostow \cite{art9-key1} and Raghunathan (\cite{art9-key1}, Chap. XII).


\begin{romancorollary}\label{art9-romancoro6}
Let $G$ be a connected linear semisimple Lie group and $H \subset G$ an algebraic subgroup with a non-trivial unipotent radical $V$.
Let $B$\pageoriginale be a maximal reductive subgroup. Then there exists $X \in \fg$ (= Lie algebra of $G$) such that
\begin{itemize}
\item[(i)] $\ad X$ is semisimple and has all eigen-values real,

\item[(ii)] for $\lambda \in \bR$, let $\fg (\lambda) = \{v \in \fg \big [X, v] = \lambda, v\}$; then the Lie algebra of $V$ is contained in $\sum\limits_{\lambda > 0} \fg (\lambda)$, 

\item[(iii)] $\{\exp t X\}_{- \infty < t < \infty}$ centralises $B$.
\end{itemize}
\end{romancorollary}

\begin{romancorollary}{\rm (Mahler's criterion).}\label{art9-romancoro7}
Let 
$$
\pi : S L (n, \bR) \to SL (n, \bR) / SL (n, \bZ)
$$
denote the natural map. Let $\{x_n\}_{1 \leq n < \infty}$ be any sequence in $SL (n, \bR)$. Then the following two conditions are equivalent:
\begin{itemize}
\item[(i)] $\pi (x_n)$ has no convergent subsequence;

\item[(ii)] There exist unipotents $\theta_n \in\Gamma - \{e\}$ such that $x_n \theta_n x^{-1}_n$ tends to $e$.
\end{itemize}
\end{romancorollary}

For a proof, see for instance, Raghunathan (\cite{art9-key1}, Chapter X).

\begin{romantheorem}{\rm (Zassenhaus \cite{art9-key1}, Kazdan-Margolis \cite{art9-key1}).}\label{art9-romanthm9}
Let $G$ be a Lie group. Then there exists a neighbourhood $\Omega$ of $e$ in $G$ with the following property: if $\Gamma \subset G$ is any discrete subgroup, $\Gamma \cap \Omega$ is contained in a connected nilpotent subgroup of $G$.
\end{romantheorem}

A neighbourhood $\Omega$ of $e$ in $G$ as in the above theorem will be called a \textit{Zassenhaus neighbourhood} of $e$ in $G$.

A proof is given in Raghunathan (\cite{art9-key1}, Chapter VIII).

\section{$L$-subgroup: a density theorem and consequences.}\pageoriginale\label{art9-sec1}

\subsection{}\label{art9-subsec1.1}
Let $G$ be a connected linear semisimple Lie group without compact factors. Let $\fg$ be the Lie algebra of $G$. We fix once for all a norm, $||\;\; ||$, on $\fg$. For $a>0$, let $V(a) = \{X \in \fg ||X|| < a \}$. Fix also a constant $a_0>0$ such that the exponential map $\exp : \fg \to G$ maps $V = V(a_0)$ diffeomorphically onto an open subset $U$ of $G$. For $a< a_0$, let $U (a) = \exp V(a)$. For $g \in U$, we set $|g| = ||X||$ where $X \in V$ is the unique element such that $\exp  X = g$. We need two definitions for the formulation of the first result.

\setcounter{definition}{1}
\begin{definition}\label{art9-def1.2}
For constants $C > c > 1$ and $a > 0$ with $0 < a < b = aC < a_0$ and a discrete subgroup $\Phi$ of $G$, an element $g \in G$ is $(a, c, C)$-adapted to $\Phi$ if the following conditions hold:
\begin{itemize}
\item[(i)] $\Ad g(V(a)) \cup \Ad g^{-1} (V(a)) \subset V (b)$,
$$
(\text{hence } g U (a) g^{-1} \cup g^{-1} U (a) g \subset U (b)) \text{ where } b = a C;
$$

\item[(ii)] for $x \in \Phi \cap U (b)$, $g x g^{-1} \in U$ and $c|x| < | gxg^{-1} |< C |x|$.
\end{itemize}
\end{definition}

\begin{definition}\label{art9-def1.3}
A discrete subgroup $\Gamma \subset G$ is an $L$-subgroup if it has the following property:

Let $p : G \to G / \Gamma$ be the natural map and $E \subset G$ any subset; then $p (E)$ is relatively compact if and only if there exists a neighbourhood $W$ of $e$ in $G$ such that $g \Gamma g^{-1} \cap W = \{e\}$ for all $g \in E$.

(Equivalently, for a sequence $g_n \in G$, $p (g)_n$ has no convergent subsequence if and only if there exists a sequence $\theta_n \in \Gamma-\{e\}$ such that $g_n \theta_n g^{-1}_n$ converges to $e$).
\end{definition}

\setcounter{subremark}{3}
\begin{subremark}\label{art9-subrem1.4}
A lattice in $G$ is an $L$-subgroup. An arithmetic subgroup of $G$ is an $L$-subgroup (this last fact can be proved directly without appealing to the fact that arithmetic subgroups are lattices).
\end{subremark}

The next lemma is essentially due to Kazdan and Margolis \cite{art9-key1}. We give a proof, as the technique of the proof itself is needed later. Kazdan and Margolis \cite{art9-key1} formulate and prove the lemma for lattices; the present formulation is to be found in Raghunathan (\cite{art9-key1}, Ch. XI).

\begin{lemma}\label{art9-lem1.5}
Fix\pageoriginale constants $C$, $c$, a with $C > c>1$ and $0 < aC = b < a_0$. Let $\Gamma \subset G$ be an $L$-subgroup and $E$ a compact subset of $G$ such that
$$
E \Gamma \supset \{g \in G \big| g \Gamma g^{-1} \cap U (a) = (e)\}.
$$
Let $S'$ be the finite set $\{\gamma \in \Gamma \big| E_\gamma \cap E U (b) \neq \empty\}$ and $S$ the set of unipotents in $S'$. Let $d'> 0$, $a > d'$, be a constant such that no element of $S'$---$S$ has a conjugate in $U (d')$. Let $m$ be the least positive integer such that $c^{m+1} d' \geqslant a$ and let $d = C^{-(m)+1}a$. Suppose now that we are given elements $g, \{g_1 \big| \; \big| \leqslant r < \infty\}$ with the following properties: let $h_r = g_r h_{r-1}$ with $h_1 = g_1$ and $\Gamma_r = h_r g \Gamma g^{-1} h_r^{-1}$; then $g_{r+1}$ is $(C, c, a)$-adapted to $\Gamma_r$ and $g_{r+1} = (e)$ if $\Gamma_r \cap U (a) = (e)$.

Then, there exists a minimal integer $k \geqslant 0$ such that $\Gamma_{k+1} \cap U (a) = (e)$. Further we can find $\theta \in \Gamma$ such that $h_{k+1} g \theta \in E$. Moreover if $g\Gamma g^{-1} \cap U (d) \neq (e)$, then for any $x \in \Gamma$ with $h_{k-1} g x g^{-1} h^{-1}_{k-1} \in U (a)$, $g x g^{-1} \in U (d')$ and $\theta^{-1} x \theta \in S$. In particular, $x$ is unipotent.
\end{lemma}

\begin{proof}
Let $a' = \inf \left\{\left|g x g^{-1} \right|  \; \big| x \in \Gamma - (e), g x g^{-1} \in U (a) \right\}$. Let $p$ be an integer such that $c^p a' > a$. We then claim that $\Gamma_p \cap U (a) = (e)$. For this we observe first that if for some integer $r > 0$, we have $h_r g x g^{-1} h^{-1}_r \in U (a)$ then $|g x g^{-1}| \leqslant c^{-r} |h_r g x g^{-1} h^{-1}_r|$. To see this we argue by induction on $r$. In fact if
$$
h_r g x g^{-1} h^{-1}_r \in U (a),
$$
we have 
$$
h_{r-1} g x g^{-1} h^{-1}_{r-1} = g^{-1}_r (h_r g x g^{-1} h_r^{-1}) g_r \in U (b).
$$
Since $g_r$ is $(C, c, a)$-adapted to $\Gamma_{r-1}$ we have 
\begin{equation*}
c|h_{r-1} g x g^{-1} h^{-1}_{r-1}| < | h_r g x g^{-1} h_r^{-1}|. \tag*{($\ast$)}
\end{equation*}
Thus $h_{r-1} g x g^{-1} h^{-1}_{r-1} \in U (a)$ so that by induction hypothesis,
\begin{equation*}
|h_{r-1} g x g^{-1} h^{-1}_{r-1}|  \geqslant c^{r-1} |g x g^{-1}|.
\tag*{($\ast\ast$)}
\end{equation*}
The claim now follows from ($\ast$) and ($\ast\ast$). Consider now $\Gamma_p \cap U (a)$. If $h_p g x g^{-1} h^{-1}_p \in U (a)$, we conclude that $|g x g^{-1}| < c^{-p} a < a'$ (by virtue of our choice of $p$); this implies that $g x g^{-1} = (e)$ in view of the definition of $a'$.
\end{proof}

This proves the first assertion.

Let\pageoriginale $k$ be the minimal integer such that $\Gamma_{k+1} \cap U (a) = (e)$. This means that
$$
(h_{k+1} g \Gamma g^{-1} h^{-1}_{k+1}) \cap U (a) = (e)
$$
and by the definition of $E$, we can find $\theta \in \Gamma$ such that $q = h_{k+1} g \theta \in E$. This proves the second assertion.

Suppose now $g \Gamma g^{-1} \cap U (d) \neq (e)$. Let $r$ be the smallest integer such that for $x \in \Gamma$ with $g x g^{-1} \in U (d)$, $h_{r+1} gx g^{-1} h^{-1}_{r+1} \not\in U (a)$. Evidently $r \leqslant k$. Moreover since $h_r g x g^{-1} h_r \in U (a)$, we conclude  that $h_p g x g^{-1} h^{-1}_p \in U (a)$ for all $p \leqslant r$ and hence that for $0 \leqslant p \leqslant r$, we have
$$
|h_p g x g^{-1} h^{-1}_p| \leqslant C^{p} |x|.
$$
It follows that $C^{r+1} |x|>a$. Since $|x|<d$, we conclude that $r \geqslant m$. Evidently $k \geqslant r$ so that $k \geqslant m$. Now if $x \in \Gamma$ is such that $h_k g x g^{-1} h^{-1}_k \in U (a)$, we see that $|g x g^{-1}| < c^{-k} \cdot a \leqslant c^{-m} \cdot a\leqslant d'$. This proves that $|g xg^{-1}| < d'$ \ie, $gxg^{-1} \in U (d')$. Finally we note that we have
$$
h_{k+1} g x g^{-1} h^{-1}_{k+1} = q \theta^{-1} x \theta q^{-1} \in U (b) (= U (aC))
$$
where $q \in E$. Thus
$$
E \theta^{-1} x \theta \cap U (b) E \neq \empty
$$
\ie, $\theta^{-1} x \theta \in S'$. Since $\theta^{-1} x \theta$ has a conjugate $g x g^{-1}$ in $U(d')$, $\theta^{-1} x \theta \in S$.

\begin{remark}\label{art9-rem1.6}
According to Kazdan and Margolis \cite{art9-key1} we can find $C, c$, a with $1 < c < C$, $0 < a C = b < a_0$ such that for any discrete subgroup $\Phi$ of $G$, we can find $g \in G (a, c, C)$-adapted to $\Phi$.
\end{remark}

As a consequence we observe that we have

\begin{theorem}\label{art9-thm1.7}
Let $\Gamma \subset G$ be an $L$-subgroup and $\pi : G \to G/\Gamma$ the natural map. Let $g_n \in G$ be any sequence. Then $\pi(g)_n$ has no convergent subsequence if and only if we can find unipotents $\theta_n \in \Gamma$---$\{e\}$ such that $g_n \theta_n g^{-1}_n$ converges to $e$.
\end{theorem}

(Theorem \ref{art9-thm1.7} again was established by Kazdan and Margolis when $\Gamma$ is a lattice).


This theorem guarantees the existence of nontrivial unipotent elements in an $L$-subgroup $\Gamma \subset G$ such that $G/\Gamma$ is not compact.

We will\pageoriginale  now obtain a generalisation to $L$-subgroups of a theorem of Borel for lattices.

\begin{theorem}\label{art9-thm1.8}
An $L$-subgroup $\Gamma \subset G$ is Zariski-dense in $G$.
\end{theorem}

\begin{proof}
Let $H$ be the Zariski-closure of $G$. Then $H$ has finitely many connected components. Replacing $\Gamma$ by a subgroup of finite index, we can assume that $H$ is connected: note that a subgroup of finite index in $\Gamma$ is again an $L$-subgroup. Assume that $H \neq G$. We assert then that $H$ is contained in a proper reductive subgroup of $G$. Now according to Mostow \cite{art9-key1}, a maximal connected subgroup $M$ of $G$ is either reductive or its Lie algebra $\fb$ is of the following form: there exists $X \in \fg$ such that $\ad X$ is semisimple and has all eigen-values real and $\fm$ is the span of the eigenspaces of $\Ad X$ corresponding to the non-negative eigen-values of $\ad X$. Thus to prove that last assertion we may without loss of generality assume that there exists $X \in \fg$ with the following properties;
\begin{itemize}
\item[(i)] $\ad X$ is semisimple;

\item[(ii)] the eigen-values of $\ad X$ are real;

\item[(iii)] for $\lambda \in \bR$, let $\fg^\lambda = \{z \in \fg | \; [X, Z] = \lambda Z\}$ and $\fm = \sum\limits_{\lambda \leq 0} \fg^\lambda$; then $\fh \subset \fm$ where $\fh$ is the Lie algebra of $H$.
\end{itemize}

Now let $\fn = \sum\limits_{\lambda > 0} \fg^{\lambda}$ and $\fm_0 = fg^0$. Let $M$, $M_0$ and $N$ be the Lie subgroups corresponding to $\fm$, $\fm_0$ and $\fn$ respectively. Then $M$ is the semi-direct product of $M_0$ and $N$. It is a closed subgroup of $G$ containing $H$. Further $N$ is unipotent and the exponential maps $\fn$ homeomorphically onto $N$. Let $g = \exp X$. Clearly then if $||\;\;||$ denotes a norm on $\fg$ with respect to which $\ad X$ is symmetric, then
\begin{equation*}
\text{and } 
\left.
\begin{aligned}
&|| \Ad g^r (Z) || \geqslant || Z||\\
& g^r x g^{-r} = x \text{ for } x \in M_0. 
\end{aligned}
\right\}
\tag*{$(\ast)$}
\end{equation*}
Consider now the sequence $\{g^r\}_{ 1 \leqslant r < \infty}$ in $G$. From $(\ast)$ one sees easily that $g^r \Gamma g^{-r} \cap W = \{e\}$ for a suitable neighbourhood $W$ of $e$ in $G$: note that\pageoriginale $\Gamma \subset H$ so that every element of $\Gamma$ is of the form $x \cdot y$ with $x \in M_0$ and $y \in N$. Since $\Gamma$ is an $L$-subgroup we can find $\{\alpha_n \in \Gamma (\subset H)\}_{1 \leqslant n < \infty}$ such that $g^n \alpha_n \in E$, where $E$ is a compact subset of $G$. Passing to a subsequence, we see that we obtain a sequence $\lambda_n$ of integers such that $g^\lambda n \theta_n$ converges to a limit $x (\in H)$ where $\theta_n = \alpha_{\lambda_n}$. Now let $\Phi \subset \Gamma$ be a finite set and consider the sequence $g^{\lambda_n} \theta_n \varphi \theta^{-1}_n g^{\lambda_n}$, $\varphi \in \Phi$; these sequences converge to limits. On the other hand for $\varphi \in \Phi$, $\theta_n \varphi \theta^{-1}_n \in H \subset M$, so that we have $\theta_n \varphi \theta^{-1}_n = x_n \exp Y_n $ where $x_n \in M_0$, $Y_n \in \fn$. It follows that 
$$
g^{\lambda_n} \theta_n \varphi \theta^{-1}_n g^{-\lambda_n} = x_n \cdot \exp \Ad g^{\lambda_n} (Y_n).
$$
Thus we conclude that $x_n$ and $\Ad g^{\lambda_n} (Y_n)$ must be convergent sequences. In view of ($\ast$), this implies that $Y_n$ must converge as well. But then $\theta_n \varphi \theta^{-1}_n$ must converge as well. Since $\theta_n \varphi \theta^{-1}_n \in \Gamma$, we conclude that for all large $n$ and $\varphi \in \Phi$, we have
$$
\theta_n \varphi \theta^{-1}_n = \theta_{n+1} \varphi \theta^{-1}_{n+1} = \varphi^\ast, \text{ say.}
$$
Now assume that $\Phi$ is so chosen that the centraliser of $\Phi$ in $G$ is the same as that of $H$(Such a finite set $\Phi$ can be found: in fact if $\rho : G \to \Aut_\bR F$ is a faithful linear representation, we need only choose $\Phi$ such that $\rho (\Phi)$ generates the associative subalgebra of $\End_\bR (F)$ generated by $\rho (H)$). Clearly then an element $h \in G$ commutes with $H$ if and only if $h$ commutes with $\Phi^\ast = \{\varphi^\ast \big| \varphi \in\Phi\}$. Consider now the sequence $g^{\lambda_n}\varphi^\ast g^{-\lambda_n}$. Let $\varphi^\ast = x^\ast \exp Y^\ast \in \Phi^\ast$ where $x^\ast \in M_0$ and $Y^\ast \in \fn$. Then
$$
g^{\lambda_n} \varphi^\ast g^{-\lambda_n} = x^\ast. \exp \; (\Ad g^{\lambda_n} (Y^\ast)).
$$
From ($\ast$) we conclude then that this sequence converges if and only if $Y^\ast =0$. Since $g^{\lambda_n} \varphi^\ast g^{-\lambda_n} = g^{\lambda_n} \theta_n \varphi \theta^{-1}_n g^{-\lambda_n}$ for all large $n$, this sequence does converge so that $Y^\ast = 0$ \ie $\varphi^{\ast} = x^\ast \in M_0$. Thus $g$ commutes with $\Phi^\ast$ \ie $H$ commutes with $g$. From this one concludes that $H \subset M_0$. We have thus shown that if $H \neq G$, then $H$ is contained in a proper reductive subgroup of $G$.

Now assume that $H \neq G$ and let $M$ be a proper reductive subgroup of $G$ containing $H$. $M$ being \textit{reductive}, we can find a representation $\rho : G \to \Aut_\bR F$ of $G$ and a vector $f_0 \in F$ such that 
$$
M = \{x \in G \big|~ \rho (x) f_0 = f_0 \}.
$$\pageoriginale
Now if $G/M$ is compact,
$$
E = \left\{f \in F \big|~\rho (G) f \text{ is relatively compact in } F \right\}
$$
is a $G$-stable subspace which is \textit{non-zero}. The representation of $G$ on $E$ is then equivalent to a unitary representation. Since $G$ has no compact factors, $G$ acts trivially on $E$, a contradiction since $f_0 \in E$. Thus $G/M$ is not compact. Since $M \supset \Gamma$, $G / \Gamma$ is not compact. Thus $\Gamma$ contains a unipotent. Now let $S (\neq e)$ be the minimal connected normal semisimple subgroup of $M$ containing all the unipotents in $\Gamma$. We can then write $M = S C$ where $C$ is the centraliser of $S$ in $M$ and $S \cap C$ is finite. Let $\fs$ be the Lie subalgebra of $\fg$ corresponding to $S$. We have then 
$$
\fg = \fs \oplus \ff = \fs \oplus \fc \oplus \ff'
$$
where $\ff$ and $\ff'$ are $S$-stable and $\fc$ is the Lie subalgebra corresponding to $C$ and
$$
\ff = \fc \oplus \ff'.
$$
Now if $\ff'$ is a trivial $\fs$-module, $\fs$ is an ideal in $\fg$ so that $G = S \cdot S'$ where $S'$ is the centraliser of $S$ in $G$. Since $G/ M$ is non-compact, we can find $g_n \in S'$ such that $p (g_n)$ (where $p: G \to G/\Gamma$ is the natural map) has no convergent subsequence; then there exist unipotents $\theta_n \in \Gamma$---(e) such that $g_n \theta_n g^{-1}_n$ converges to $e$, a contradiction since $\theta_n \in \Gamma \subset S$ and $S$ and $S'$ commute. Thus we conclude $\ff'$ is a non-trivial $\fs$-module. Now from the representation theory of real semisimple Lie algebras one concludes easily the following:

Let $\Phi \subset \Gamma$ be a maximal unipotent subgroup. Then there exists $Y \in \ff'$, $Y \neq 0$, such that (i) $\ad Y$ is nilpotent and (ii) $\Ad \varphi (Y) = Y$ for all $\varphi \in \Phi$. Consider now again a representation $\rho : G \longmapsto \Aut F$  such that there exists a vector $f_0 \in F$ such that
$$
M = \{M \in G \big|~ \rho (x) f_0 = f_0\}.
$$
Now the orbit map $t \longmapsto \rho (\exp t Y) f_0$ gives a homeomorphism of $R$ onto a closed subset of $F$ (see Rosenlicht \cite{art9-key2}). We conclude from this that $\{(\exp t Y) M\}_{- \infty < t < \infty}$ is a closed subset of $G$. In particular, the sequence\pageoriginale $\{\exp n Y\}_{1 \leqslant n < \infty}$ has no convergent subsequent modulo $M$ hence, \textit{a fortiori}, modulo $\Gamma$. Let $g_n = \exp n Y$. We assume, as we may without loss of generality, that a finite system $\Sigma$ of generators for $\Phi$ is contained in a Zassenhaus neighbourhood $\Omega$ of $e$ in $G$. Now in view of Theorem \ref{art9-thm1.7}, we can find unipotents $\theta_n \in \Gamma$---$\{e\}$ such that $g_n \theta_n g^{-1}_n$ converges to $e$. For large $n$, thus $g_n \theta_n g^{-1}_n \in \Omega$. Since $g_n \Sigma g^{-1}_n = \Sigma \subset \Omega$, $\theta_n$ and $\Sigma$ generate a unipotent subgroup of $\Gamma$. In view of the maximality of $\Phi$, $\theta_n \in \Phi$. But then $g_n \theta_n g^{-1}_n = \theta_n$, a contradiction. This completes the proof of \ref{art9-thm1.8}.
\end{proof}

As a consequence, we obtain the following corollaries. We omit their proofs which are analogous to those in the case of lattices. For proofs for lattices see for instance Raghunathan [1, Chapter V]; the proofs given there carry over with minor verbal changes.

\setcounter{coro}{8}
\begin{coro}\label{art9-coro1.9}
Let $\Gamma \subset G$ be an $L$-subgroup. Then the centraliser of $\Gamma$ is the centre of $G$. The normaliser of $\Gamma$ in $G$ is discrete.
\end{coro}

\begin{coro}\label{art9-coro1.10}
Let $\Gamma \subset G$ be an $L$-subgroup and $H$ a connected normal subgroup of $G$. Let $H'$ be the connected centraliser of $H$ in $G$. Let $f$ (\resp. $f'$) be the natural map of $G$ on $G/H$ (\resp. $G/H'$). The following conditions on $\Gamma$ are equivalent:
\begin{itemize}
\item[(1)] $f(\Gamma)$ is a discrete subgroup of $G / H$.

\item[(2)] $f (\Gamma')$ is a discrete subgroup of $G/H'$.

\item[(3)] $\Gamma \cap H$ is an $L$-subgroup of $H$.

\item[(4)] $\Gamma \cap H'$ is an $L$-subgroup of $H'$.

\item[(5)] $\Gamma$ contains $(\Gamma \cap H)$ $(\Gamma \cap H')$ as a subgroup of finite index.

\item[(6)] $\Gamma \cap H$ is Zariski-dense in $H$.

\item[(7)] $\Gamma \cap H'$ is Zariski-dense in $H'$.
\end{itemize}
\end{coro}

\setcounter{definition}{10}
\begin{definition}\label{art9-def1.11}
An $L$-subgroup $\Gamma \subset G$ is reducible if there exist proper normal subgroups  $H$, $H'$ in $G$ as in \ref{art9-coro1.10} such that $\Gamma$ satisfies one of the equivalent conditions (1) --- (7) of that  corollary. If $\Gamma$ is not reducible we will say that $\Gamma$ is irreducible.
\end{definition}

We have then 

\setcounter{coro}{11}
\begin{coro}\label{art9-coro1.12}
The following\pageoriginale conditions on an $L$-subgroup $\Gamma$ of $G$ are equivalent:
\begin{itemize}
\item[(1)] $\Gamma$ is irreducible,

\item[(2)] if $H$ is any proper connected normal subgroup of $G$, $H \cap \Gamma$ is not an $L$-subgroup of $H$, 

\item[(3)] if $H$ is any proper connected normal subgroup of $G$, $H \cap \Gamma$ is central in $H$,

\item[(4)] if $H$ is any proper connected normal subgroup of $G$, $H \Gamma$ is not closed,

\item[(5)] under the same hypothesis as in (4), $H \Gamma$ is dense in $G$.
\end{itemize}
\end{coro}

With the Definition \ref{art9-def1.11} we can formulate a decomposition theorem.

\begin{coro}\label{art9-coro1.13}
Let $\Gamma \subset G$ be an $L$-subgroup; then we can find connected normal subgroups $\{H_i \big| 1 \leqslant i \leqslant r\}$ of $G$ such that 
\begin{itemize}
\item[(i)] $H_1 . H_2 \ldots H_r = G$,

\item[(ii)] $H_i \cap H_1 . H_2 \ldots H_{i-1}. H_{i+1} \ldots H_r$ is finite,

\item[(iii)] $H_i \cap \Gamma$ is an irreducible $L$-subgroup of $H_i$,

\item[(iv)] $\prod\limits^{r}_{i=1} (H_i \cap \Gamma)$ has finite index in $\Gamma$. 
\end{itemize}
\end{coro}

\begin{coro}\label{art9-coro1.14}
Let $\Gamma \subset G$ be a non-uniform $L$-subgroup. Then $\Gamma$ is irreducible if and only if no unipotent element of $\Gamma$ belongs to a proper (connected) normal subgroup of $G$.
\end{coro}

\section{$P$-subgroups.}\label{art9-sec2}
From now on, we fix once for all a non-uniform irreducible $L$-subgroup $\Gamma$ of $G$. We state below two results which are proved for lattices Raghunathan (\cite{art9-key1}, Chapter XI). The proofs given there carry over with minor verbal changes to the case of $L$-subgroups.


\begin{theorem}\label{art9-thm2.1}
Let $\Delta$ be a subgroup of $\Gamma$ and $\Phi$ a unipotent subgroup of $\Gamma$ maximal among all unipotent subgroups of $\Gamma$ normalised by $\Delta$. Assume that $\Phi \neq (e)$. Let $D(\resp. F)$ be the zariski-closure of $\Delta$ (\resp. $\Phi$). Let $\ff$ be the Lie algebra of $F$ and $\sigma : D \longmapsto G L(\ff)$ the adjoint\pageoriginale representation of $D$ on $\ff$. Let $M$ be a maximal reductive subgroup of $D$. Then $M \cap$ (kernel $\sigma$) is finite.
\end{theorem}

The second result we need is a consequence of the above theorem (\cf 1.1 for notation not explained above).

\begin{theorem}\label{art9-thm2.2}
There exists a constant $\beta > 0$, $\beta < a_0$ such that an element $\gamma \in \Gamma$ has a conjugate in $U (\beta)$ (by an element of $G$) if and only if $\gamma$ is unipotent.
\end{theorem}

We will not sharpen Theorem \ref{art9-thm2.1} further.

\begin{theorem}\label{art9-thm2.3}
The hypothesis and notation are those of Theorem \ref{art9-thm2.1}. Then Centraliser $F$ = Centre of $G$. Centre of $F$. Consequently $M \cap ker \sigma \subset$ Centre of $G$. 
\end{theorem}

We begin with 

\setcounter{claim}{3}
\begin{claim}\label{art9-claim2.4}
Let $Z$ = Centraliser of $F$. Then $Z/Z \cap Z$ is compact.
\end{claim}

We observe first that the map
$$
Z/Z \cap \Gamma \to G/ \Gamma
$$
is proper (in fact $Z$ = Centraliser of $\Phi$ and $\Phi$ is finitely generated; (\ref{art9-romanthm2}); hence Lemma \ref{art9-romanlem1} applies). Suppose now $z_n \in Z$ is any sequence such that $p (z_n)$, $p: Z \to Z/Z \cap \Gamma$ being the natural map, has no convergent subsequence; then we can find $\theta_n \in\Gamma$---$e,\theta_n$ unipotent, such that $z_n \theta_n z_n^{-1}$ converges to $e$. Now we assume, as we  may, that a finite set of generators $\Sigma$ of $\Phi$ is contained in a Zassenhaus neighbourhood $\Omega$ of $e$ in $G$. Then for $\theta \in \Sigma$, $z_n \theta z_n^{-1}=\theta$ so that for all large $n, z_n \theta_n z^{-1}_n$  and $\Sigma = z_n \Sigma z^{-1}_n$ generate a nilpotent (hence unipotent) subgroup. Forming successive commutators one obtains a sequence $\varphi_n \in Z \cap \Gamma$---$(e)$ of unipotents such that $z_n \varphi_n z^{-1}_n$ converges to $e$. It is evidently sufficient to prove the following now.

\setcounter{assertion}{4}
\begin{assertion}\label{art9-asser2.5}
Any unipotent elements in $Z \cap \Gamma$ belongs to $\Phi$.
\end{assertion}

To prove this, we argue as follows. Let $S$ be the Zariski-elosure of $Z \cap \Gamma$. Since $\Delta$ normalises $\Phi$, $\Delta$ normalises $Z$ as well as $Z \cap \Gamma$. Let $\Delta' = \Delta$. $Z \cap \Gamma$ and $D'$ the Zariski-closure of $\Delta'$. Now $\Phi$ is evidently maximal among unipotent subgroups normalised by $\Delta'$. It follows that any maximal reductive subgroup of $D'$ and hence, \textit{a fortiori}, any maximal\pageoriginale reductive subgroup of $S$ acts on $F$ with finite kernel. Since $S$ acts trivially on $F$ we conclude that any maximal reductive subgroup of $S$ is finite. It follows that the set of all unipotent elements in $Z \cap \Gamma$ generate a \textit{unipotent} subgroup $\Psi$. Evidently $\Psi$ normalises $\Phi$ so that $\Psi \cdot \Phi$ is a unipotent subgroup of $\Gamma$. The group $\Psi$ is normalised by $\Delta$ so that $\Psi \cdot \Phi$ is normalised by $\Delta$. In view of the maximality of $\Phi$, $\Psi \Phi = \Phi$ \ie $\Psi \subset \Phi$. This proves the assertion and hence the claim.

\setcounter{subsection}{5}
\subsection{}\label{art9-subsec2.6}
To establish Theorem \ref{art9-thm2.3} we argue as follows: $M \cap \ker \sigma \subset Z$; let $B$ be a maximal reductive subgroup of $Z$ containing this finite group. Then the kernel of the action of $B$ on the unipotent radical $N$ of $Z$ is a normal subgroup of $G$ (cf. Theorem I), hence a semisimple subgroup without compact factors. It follows from Theorem I that (since $Z/Z \cap \Gamma$ is compact) $N/N \cap \Gamma$ is compact. Now $N \cap \Gamma \subset \Psi \subset \Phi$ and since $B \subset Z$, $B$ acts trivially on $\Phi$. Further $N \cap \Gamma$ is the Centre of $\Phi$ hence non-trivial. According to Corollary \ref{art9-coro1.14}, $N \cap \Gamma$ is not contained in any proper connected normal subgroup of $G$. It follows now (again from Theorem I) that $B$ is normal in $G$ and hence discrete and central. This proves Theorem \ref{art9-thm2.3}.

\subsection{}\label{art9-subsec2.7}
Suppose now that $\Phi$ is a unipotent subgroup of $\Gamma$ and $F$ is the Zariski-closure of $\Phi$. Assume that the centraliser of $F$ is (centre of $G$. centre of $F$) (this holds in particular when $\phi$ ($\neq (e)$) is maximal among unipotent groups normalised by a subgroup $\Delta$ of $\Gamma$: Theorem \ref{art9-thm2.3}). Let $N^\ast$ denote the normaliser of $F$ and $N$ the identity component of $N^\ast$. Let $\sigma$ denote the adjoint action of $N^\ast$ on $F$, as well as on the Lie algebra $\ff$ of $F : \sigma : N^\ast \to G L (\ff)$. Let 
\begin{align*}
N^\ast_0  & = \left\{x \in N^\ast \big| \det \sigma (x) = \pm 1 \right\}\\
N_0 & = \left\{x \in N \big| \det \sigma (x) = 1 \right\}
\end{align*}
(note that $N_0 = N^\ast_0 \cap N$). Finally, let $\sL \subset \ff$ denote the $\bZ$-span of $\exp^{-1} \Phi$; $\sL$ is a lattice in $\ff$ (see Theorem \ref{art9-romanthm2}). We have then 

\setcounter{proposition}{7}
\begin{proposition}\label{art9-prop2.8}
A sequence $x_n \in N_0$ is relatively compact modulo $\Gamma$ if and only if $\sigma(x_n)$ (in $GL(\ff)$) is relatively compact modulo $GL(\sL)$.
\end{proposition}

The proofs below, of this proposition as well as the next one, were obtained in collaboration with H. Garland.

\setcounter{subsection}{8}
\subsection{}\label{art9-subsec2.9}
Let $\{x_n\}_{1 \leqslant n < \infty}$ be a sequence in $N_0$ such that $\sigma (x_n)$ is relatively compact modulo $GL(\sL)$. Assume that $x_n$ is \textit{not} relatively compact modulo $\Gamma$. Replacing  $\{x_n\}_{1\leq n < \infty}$ by a subsequence if necessary, we can assume that there exist $\theta_n \in \Gamma$---$e$, $\theta_n$  unipotent such that $x_n \theta_n x^{-1}_n$ converges to $e$. On the other hand, we can find $u_n \in GL(\sL)$ such that $\sigma (x_n) u_n$ is relatively compact. It follows that we can find a sequence of bases $\{e_n (i) \big| 1 \leqslant i \leqslant p\}_{1 \leqslant n < \infty}$ of $\sL$ and a compact subset $E$ of $\ff$ such that $\sigma (x_n) e_n (i) \in E$ for $1 \leqslant i \leqslant p$ and $1 \leqslant n < \infty$. Now we can find an integer $r >0$ such that $\exp r. x \in \Phi$ for all $x \in \sL$. Let $\varphi_n (i) = \exp r. e_n (i)$. Evidently then, we can find a compact subset $E'\subset F$ such that $x_n \varphi_n (i) x^{-1}_n \in E'$ for $1 \leqslant i \leqslant p$ and $1 \leqslant n < \infty$. Let $g \in G$ be an element such that $g E' g^{-1} \subset \Omega$, a Zassenhaus neighbourhood of $e$ in $G$. Then for large $n$, $gx_n \varphi_n (i) x^{-1}_n g^{-1}$  as well as $gx_n \theta_m x^{-1}_n g^{-1}$ belong to $\Omega$. Thus $x_n \varphi_n (i)x^{-1}_n$ and $\theta_n$ generate together a nilpotent, hence unipotent group. Forming successive commutators of $\theta_n$ with the $\varphi_n (i)$ we obtain a unipotent sequence $\Psi_n (\neq e)$ of elements in $\Gamma$ centralising the $\{\varphi_n (i) big| 1 \leqslant i \leqslant p\}$ such that $x_n \Psi_n x^{-1}_n$ tends to $e$ as $n$ tends to $\infty$. Now the group generated by $\varphi_n (i)$, $1 \leqslant i \leqslant p$, it is easily seen, is Zariski-dense in $F$. Thus $\Psi_n \in $ centraliser of $F$. In view of Theorem \ref{art9-thm2.3}, $\Psi_n \in F$. Let $X_n \in \sL$ be the unique (non-zero) element such that exp $X_n = \Psi_n$. Then $\sigma (x_n) X_n$ tends to zero as $n$ tends to $\infty$. On the other hand $\sigma(x_n)$. $X_n = \sigma (x_n). u_n. u^{-1}_n X_n$ where $\sigma (x_n). u_n$ is relatively compact in $GL(\ff)$ and $u_n \in G L (\sL)$ so that $u^{-1}_n X_n \in \sL$. Since $\sL$ is a lattice in $\ff$, $\sigma (x_n) X_n$ cannot converge to $e$ as $n$ tends to $\infty$, a contradiction. Thus if $\sigma (x_n)$ is relatively compact modulo $GL(\sL)$, $x_n$ is relatively compact modulo $\Gamma$. Conversely if $x_n$ is relatively compact modulo $\Gamma$, $\sigma (x_n)$ is relatively compact modulo $GL(\sL)$. In fact if $\sigma (x_n)$ is not relatively compact, passing to a sub-sequence if necessary, we can find $e_n \in \sL$, $e_n \neq 0$, such that $\sigma (x_n) e_n$ converges to 0. This follows from Mahler's compactness criterion (note that $\sigma (N_0) \subset SL(\ff)$). Fixing an integer $r$ such that $\exp r \sL \subset \Gamma$ and setting $\varphi_n = \exp r e_n$, we find that $x_n \varphi_n x^{-1}_n$ converges to $e$; a contradiction, since $\Gamma$ is an $L$-subgroup. This proves Proposition \ref{art9-prop2.8}.
   
\setcounter{proposition}{9}
\begin{proposition}\label{art9-prop2.10}
The natural maps 
$$
N_0 / N_0 \cap \Gamma \to G / \Gamma
$$
and 
$$
N_0 / N_0 \cap \Gamma \to G L (\ff) /  Ga L (\sL)
$$
are both proper.
\end{proposition}

\setcounter{subsection}{10}
\subsection{}\label{art9-subsec2.11}
In view of Proposition \ref{art9-prop2.8}, it suffices to show that
$$
N_0/ N_0 \cap \Gamma \to G / \Gamma
$$
is proper. Let $x_n \in N_0$, $\gamma_n \in \Gamma$ be sequences such that $x_n \gamma_n$ converges to a limit. According to Proposition \ref{art9-prop2.8}, we can find a sequence of bases $\{e_n(i) \big| 1 \leqslant i \leqslant p \}$ of $\sL$ such that $\sigma (x_n) e_n (i) \in E (1 \leqslant i \leqslant p)$ where $E$ is a fixed compact subset of $\ff$. Let $r > 0$ be an integer such that exp $r \sL \subset \Phi \subset \Gamma$ and set $\varphi_n (i) =\exp r e_n (i)$. Then there is a compact subset $E' \subset F$ such that $x_n \varphi_n (i) x^{-1}_n \in E'$. Now we have
$$
x_n \varphi_n (i) x^{-1}_n = x_n \gamma^{-1}_n \gamma_n \varphi_n (i) \; \gamma^{-1}_n \gamma_n x^{-1}_n \in E'.
$$
It follows that $\gamma_n \varphi_n(i) \gamma^{-1}_n \in E''$ where $E''$ is a relatively compact subset of $G$. Since $\varphi_n(i) \in \Gamma$, we conclude that we can find a finite number $\{\theta_m (i) \big| 1 \leqslant i \leqslant p, \; 1 \leqslant m \leqslant q\}$ such that for $1 \leqslant n < \infty$, we have $\gamma_n \varphi_n(i) \gamma^{-1}_n = \theta_{m(n)} (i)$ for some $m(n)$ with $1 \leqslant m (n) \leqslant q$. For each $m$ with $1 \leqslant m \leqslant q$, choose an integer $v(m)$ such that $m(v(m)) = m$. Then we have
$$
\gamma_m \varphi_n (i) \gamma^{-1}_n = \gamma_v \varphi_v (i) \gamma^{-1}_v
$$
where $v = v (m(n))$. Since the $\varphi_n (i)$, $1 \leqslant i \leqslant p$, generate a Zariski-dense subgroup of $F$, we see that $\xi_n = \gamma_n \cdot \gamma^{-1}_v$ normalises $F$ \ie $\xi_n \in N^\ast$. Evidently $x_n \cdot \xi_n$ is relatively compact. The group $N^\ast \cap \Gamma$ leaves $\Phi$ stable. Thus $\sigma (N^\ast \cap \Gamma) (\sL) = \sL$. It follows that $\det \sigma (x) = \pm 1$ for all  $x \in N^\ast$ \ie $N^\ast \cap \Gamma \subset N^\ast_0$  so that $\xi_n \in N^\ast_0$. Finally, since $N_0$ has finite index in $N^\ast_0$, $N_0 \cap \Gamma$ has finite index in $N^\ast_0 \cap \Gamma$. It follows that we can find $\xi'_n \in N_0 \cap \Gamma$ such that $x_n \xi'_n$ is relatively compact. This completes the proof of Proposition \ref{art9-prop2.10}.

\subsection{}\label{art9-subsec2.12}
From now on $G$ \textit{is assumed to have trivial centre}. Let
\begin{align*}
\sU^\ast & = \text{ set of all subgroups $\Delta$ of $\Gamma$ which normalise a nontrivial}\\
& \text{\qquad  unipotent subgroup of $\Gamma$.}\\
\sU & =  \{\Delta \in \sU^\ast \big| \Delta \text{ is generated by unipotents} \}.
\end{align*}
For $\Delta \in \sU^\ast$, let
$$
\sU (\Delta) = \left\{\Phi \subset \Gamma | \Phi \text{ unipotent subgroup normalised by $\Delta$} \right\}.
$$\pageoriginale
Note that we have for $\Delta \in \sU^\ast$ and $\Phi \in \sU (\Delta)$,
$$
\Phi \in \sU (\Delta \Phi) \subset \sU (\Delta).
$$
Also, if in addition $\Delta \in \sU$, we have $\Delta \Phi \in \sU$.   

\setcounter{definition}{12}
\begin{definition}\label{art9-def2.13}
A subgroup $\Delta \in \sU^\ast$ is full if every $\Phi$ in $\sU(\Delta)$ is a subgroup of $\Delta$. Equivalently the maximal unipotent normal subgroup  of $\Delta$ is a (and hence the {\rm unique}) maximal element in $\sU (\Delta)$. 
\end{definition}

\begin{lemma}\label{art9-lem2.14}
Given $\Delta' \in \sU^\ast$ (\resp. $\sU$), there exists a full $\Delta$ in $\sU^\ast$ (\resp. $\sU$) such that $\Delta' \subset \Delta$.
\end{lemma}

\begin{proof}
In fact, let $\Phi$ be a maximal element in $\sU(\Delta')$. Then $\Delta' \Phi \in \sU^\ast$ (\resp. $\sU$ if $\Delta' \in \sU$). Evidently $\Phi$ is the maximum unipotent normal subgroup of $\Delta = \Delta' \Phi$. Hence the lemma. 
\end{proof}

\subsection{}\label{art9-subsec2.15}
Suppose now that $\Delta \in \sU^\ast$ is any element with Zariski-closure $D$. Then $D = M. U$ where $M$ is reductive algebraic, $U$ is the unipotent radical of $D$ and $M \cap U = (e)$. Moreover the $e$-component of $M$ itself decomposes further into an almost direct product $S.C$ where $S$ is semisimple and $C$ is central in $M$. Also, $SC$ has finite index in $M$. Further if $\Delta \in \sU$, it is easily seen that $D$ and $M$ are connected, $M=S$ and $C = \{e\}$. For $\Delta \in \sU$, we define the \textit{content} of $\Delta$ as the dimension of $M(=S)$ and denote it $c(\Delta)$. Finally let $p(\Gamma) = \max \{c (\Delta) \big| \Delta \in \sU\}$. (Note that $c(\Delta)$ has \textit{not} been defined for $\Delta \in \sU^\ast$). Occasionally, we refer to $p(\Gamma)$ as the \textit{parabolic content} of $\Gamma$.

\setcounter{definition}{15}
\begin{definition}\label{art9-def1.15}
A subgroup $\Delta$ of $\Gamma$ is a $P$-subgroup if
\begin{itemize}
\item[(i)] $\Delta \in \sU$,

\item[(ii)] $c (\Delta) = p (\Gamma)$,
 
and 

\item[(iii)] $\Delta$ is full.
\end{itemize}
\end{definition}


%%%%% 245 page




\begin{thebibliography}{99}
\bibitem{art9-key1} 
\end{thebibliography}

