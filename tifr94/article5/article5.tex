
\title{ON THE COHOMOLOGY OF DISCRETE ARITHMETICALLY DEFINED GROUPS}\footnote{Supported by the ``Sonderforschungsbereich f\"ur theoretische Mathematik an der Universit\"at Bonn'' and NSF grant GP-36418X.}
\markright{ON THE COHOMOLOGY OF DISCRETE ARITHMETICALLY DEFINED GROUPS}

\author{By~ G. HARDER} 
\markboth{G. HARDER}{ON THE COHOMOLOGY OF DISCRETE ARITHMETICALLY DEFINED GROUPS}

\date{}
\maketitle


%\setcounter{page}{21}
\setcounter{pageoriginal}{128}

\textsc{Introduction}
In this paper\pageoriginale I want to come back to the questions which I discussed in \cite{art5-key7}. These questions arise from the study of the cohomology of discrete arithmetically defined groups $\Gamma$. To investigate the cohomology of $\Gamma$ one considers the action of $\Gamma$ on the corresponding symmetric space $X$ and makes use of the fact that $H^\bigdot (X / \Gamma \bR) = H^\bigdot (\Gamma, \bR)$. If this quotient $X / \Gamma$ is compact then the Hodge theory (comp. \cite{art5-key12} \S 31) is a powerful tool for the investigation of these cohomology groups. But in general the quotient $X/ \Gamma$ is not compact and my concern in this paper are those phenomena which are due to this noncompactness. The Hodge theory fails in this case and I want to find a substitute for it or in other words I want to get control of the deviation from Hodge theory. The basic idea is to make use of Langland's theory of Eisenstein series (comp. \cite{art5-key8}) to describe the cohomology at ``infinity''.

In this paper I mainly consider the case that $\Gamma$ is of rank one, i.e. the semi-simple group $G/ K$ which defines $\Gamma$ is of rank one over the algebraic number field $k$. The main result is Theorem 4.6. This theorem is to some extent a generalization of Theorem 2.1 in \cite{art5-key7} which is stated there without proof. On the other hand the results in \cite{art5-key7} are still much more precise, because in the case of $G = S L_2 / k$ the intertwining operators $c (s)$ (comp. \S 3) are accessible for explicit computations. I hope to come back to these problems later.

\section{Basic notions and results on the cohomology of $\Gamma$.}\label{art5-sec1}

Let $G_\infty$ be a real Lie group acting transitively from the right on a contractible $C^\infty$-manifold $X$. Let $x_0 \in X$ and let us assume that the stabilizer of $x_0$ is a compact subgroup $K$. Therefore we have $K / G_\infty = X$. 

Let\pageoriginale $\Gamma \subset G_\infty$ be a discrete subgroup without torsion, let us assume

%%%% 130 page


\eqref{art5-eq}




