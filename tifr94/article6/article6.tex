
\title{ON MODULAR CURVES OVER FINITE FIELDS}
\markright{ON MODULAR CURVES OVER FINITE FIELDS}

\author{By~ YASUTAKA IHARA}
\markboth{YASUTAKA IHARA}{ON MODULAR CURVES OVER FINITE FIELDS}

\date{}
\maketitle


%\setcounter{page}{21}
\setcounter{pageoriginal}{160}

\noindent
\textsc{Introduction.}\pageoriginale We shall prove a certain basic theorem on modular curves over finite fields. This gives a solution of a conjecture raised at the Boulder summer school in 1965 \cite{art6-key7}. The congruence subgroup property of the modular group of degree two over $\bZ(1/p)$ is faithfully reflected in this theorem.

Let $p$ be a fixed prime number, $F$ be the algebraic closure of the prime field of characteristic $p$, and $F_{p^m} (m \geqslant 1)$ be the finite subfield of $F$ with $p^m$ elements. Let $j$ be a variable over $F$. For each positive integer $n$ with $p\nmid n$, let $L_n$ be the field of modular functions of level $n$ over $F(j)$ (with respect to $j$) in the sense of Igusa \cite{art6-key5}. Then $L_n / F (j)$ is a Galois extension with Galois group isomorphic to $SL_2 (\bL/ n) / \pm I$. We know that there is a \textit{canonical} choice of a Galois extension $K_n / F_{p^2} (j)$ with the same Galois group as $L_n / F(j)$, satisfying $K_n \cdot F= L_n$ and $K_n \cap F = F_{p^2}$ (\cite{art6-key7}, \cite{art6-key8}; see \S 1). The field $K_n$ is obtained as the fixed field of the action of $p^{\bZ} on L_n$. Call a prime divisor $P$ of $K_n$ \textit{supersingular} if the residue class of $j$ at $P$ is a supersingular ``$j$-invariant''. An important fact, which becomes apparent by lowering the field of modular functions from $L_n$ to $K_n$, is that \textit{all supersingular prime divisors of $K_n$ are of degree one over $F_{p^2}$}. This proof is in \cite{art6-key8}. For $n > 1$, the number of supersingular prime divisors of $K_n$ is equal to (1/12) $(p-1) [K_n: K_1]$. We shall prove that following 

\begin{theorem*}
There is no non-trivial unramified extension of $K_n$ in which all supersingular prime divisors are decomposed completely.
\end{theorem*}

This, combined with Igusa's result \cite{art6-key5} on the ramifications of $L_n/L_1$ (which says that the ramifications of $L_n/L_1$ are almost analogous to those of the corresponding situation in characteristic 0), gives the following complete characterization of the Galois extension $K_\infty/ K_r$, where $r >1$ and $K_\infty$ is the composite of $K_n$ for all $n$:

\begin{coro*}
$K_\infty/ K_r$ is the maximum Galois extension of $K_r$ satisfying the following two properties; (i) it is tamely ramified, and unramified outside\pageoriginale the cusps of $K_r$; (ii) the supersingular prime divisors $af$ $K_r$ are decomposed completely.
\end{coro*}     

If one wants to emphasize the arithmetic nature of the theorem, one may restate the theorem in the following way: \textit{``the Galois group of the maximum unramified Galois extension of $K_n$ is topologically generated by the $p^2$-th power Frobenius automorphisms of (all extensions of) the supersingular prime divisors of $K_n$.''} Some direct applications of this to the distribution of supersingular prime divisors are given in \S 2. If one wants to look at the theorem as a theorem on the fundamental group, one may formulate it as follows. let $\Gamma_1 = PSL_2 (Z^{(z)})$ be the modular group over $\bZ^{(p)} = \bZ[1/p]$ and $\Gamma_n$ the principle congruence subgroup of $\Gamma_1$ with level $n (p\nmid n)$. Let $X_n$ be a complete non-singular model of $K_n$. Then, for $r > 1$, \textit{``there is a categorical equivalence between subgroups of $\Gamma_r$ with finite indices and those finite separable irreducible coverings of $X_r$ defined over $F_{p^2}$ and satisfying the above two conditions} (i), (ii).'' (The second condition (ii) is geometrically stated as ``all points lying on the supersingular points of $X_r$ are $F_{p^2}$-rational''.) In this sense, the modular group of degree two over $\bZ^{(p)}$ is \textit{the fundamental group} defined by (i), (ii).

We shall mention here that it is essential to consider the groups over $\bZ^{(p)}$ and the curves over $F_{p^2}$. It cannot be replaced by the groups over $\bZ$ and the curves $F$. Roughly speaking, considering the group over $\bZ$ corresponds to considering the condition (i) alone. The condition (i) alone cannot characterize the system of coverings defined by the modular curves $X_n$, as each $X_n$ for $n \geqslant 6$ has so many non-trivial unramified coverings and they are all from \textit{outside} the system (see \S 1. 2, \S 2). Such unramified coverings of $X_n$ correspond to \textit{non-congruence subgroups} of $PSL_2 (\bZ)$. The passage from $PSL_2 (\bZ)$ to $PSL_2 (\bZ^{(p)})$ kills all non-congruence subgroups. Our theorem says, as a corresponding geometric fact, that all non-trivial unramified coverings of $X_n$ are killed by the condition (ii), \ie, by \textit{the super-singular Frobeniuses}. The connection between the function field $K_n$ (or the curve $X_n$) and $\Gamma_n$ is essential, as our previous studies \cite{art6-key7}, \cite{art6-key8}, and also the proof of the present theorem show.

In \S\ref{art6-sec1}, we\pageoriginale shall give some preliminaries, and in \S \ref{art6-sec2}, we shall state our theorem in several different forms, with some corollaries. The proof will be given in he rest of the paper, in \S\S \ref{art6-sec3}-\ref{art6-sec6}.

\textsc{Method for proof.} The modular group over $\bZ^{(p)}$ has the congruence subgroup property (Mennicke, Serre), and our theorem is a faithful reflection of this property. The proof goes through directly by chasing away the obstacles (clouds!). It consists of the following steps (a)-(d).
\begin{itemize}
\item[(a)] A geometric interpretation of the congruence subgroup property as the ``simply-connectedness'' of the system $\{R_n \xleftarrow{\varphi} R^0_n \xrightarrow{\varphi'} R'_n\}$ of three compact Riemann surfaces. The Riemann surfaces $R_n$, $R'_n$ and $R^0_n$ are defined by the fuchsian groups $\Delta_n, \Delta'_n$ and $\Delta^0_n$, where $\Delta_n$ is the principal congruence subgroup of $PSL_2 (\bZ)$ of level $n$,
$$
\Delta'_n = 
\begin{pmatrix}
p & 0 \\
0 & 1
\end{pmatrix}^{-1}
\Delta_n 
\begin{pmatrix}
p & 0 \\
0 & 1
\end{pmatrix}, \text{ and }
\Delta^0_n = \Delta_n \cap \Delta'_n.
$$
The system is essentially the graph of the modular correspondence ``$T(p)$ for level $n$''. (However, $R^0_n$ is not immersed in the product $R_n \times R'_n$.) The interpretation is made possible due to the fact that $\Gamma_n$ is the \textit{free} product of $\Delta_n$ and $\Delta'_n$ with amalgamated subgroup $\Delta^0_n$. And the congruence subgroup property is used in the following form that $\Gamma_n$ is generated only by the parabolic elements (\S 3).


\item[(b)] A geometric interpretation of the theorem to be proved as the ``simply-connectedness'' of the system $\{X_n \xleftarrow{\psi} X^0_n\xrightarrow{\psi'} X'_n\}$ of three $F_{p^2}$-curves. Here, $X_n$ is as above, $X'_n$ is its conjugate over $F_p$, and $X^0_n$ is the union of the graphs $\Pi$, $\Pi'$ of two $p$-th power morphisms $X_n \to X'_n$, $X'_n \to X_n$ (both graphs being taken on $X_n \times X'_n$. Actually, $X^0_n$ is partially normalized so that $\Pi$ and $\Pi'$ are \textit{crossing only at the supersingular points} (\S 4).

\item[(c)] The bridge connecting these two systems, \ie, the congruence relation ``$R^0_n \equiv \Pi + \Pi' (\mod p)$''. This is presented in \S 5.1. Due to our canonical choice of the curves over $F_{p^2}$ and their $p$-adic lifting, where the action of $p^\bZ$ is trivialized, the congruence relation is in its \textit{symmetric} form. Also, we use the non-singular curve $R^0_n$ instead of its image in $R_n \times R'_n$, and this corresponds to that $\Pi, \Pi'$ are crossing\pageoriginale only at the supersingular points.\footnote{We learnt this from Igusa \cite{art6-key5}, p. 472 (footnote); cf. also Deligne [i], No. 4} Besider the general formulations of congruence relations by Shimura \cite{art6-key18}, it is also meaningful to draw attention to this canonical and natural form of the congruence relation.

\item[(d)] Passing the bridge, using a celebrated theorem of Grothen-dieck on the unique $p$-adic liftability of \'etale coverings (\S 6).

I am very grateful to Professor P. Deligne who read my first version of the proof and kindly pointed out to me a simplification by an extended use of a theorem of Grothendieck (see \S 6.2).
\end{itemize}

\section{Preliminaries.}\label{art6-sec1}
 We shall review the definitions and some basic facts related to the fields of modular functions over finite fields.

\subsection{}\label{art6-subsec1.1}
Let $p$ be the fixed prime number, and $n$ be a positive integer not divisible by $p$. In \S \ref{art6-subsec1.1}, $F_p$ will denote any field of characteristic either 0 or $p$, satisfying the following condition. Let $W_n$ be the group of $n$-th roots of unity in the fixed separable closure of $F_p$, let $W_\infty$ be the inductive union of $W_n$ for all $n$ (with $p \nmid n$), and consider the cyclotomic extension $F = F_p (W_\infty)$. Then our condition on $F_p$ is that the Galois group $G(F/ F_p)$ contains, and is topologically generated by, a special element $\sigma$ that acts on $W_\infty$ as $\zeta \to \zeta^p$. This being assumed, put $F_{p^m} = F_p (W_{p^m -1})$, for each positive integer $m$. Then $F_{p^m}$ is a cyclic extension of $F_p$ with degree $m$, and it is the fixed field of $\sigma^m$ in $F$. The Krull topology of $G(F/F_p)$ induces, \textit{via} $\sigma^\bZ$, the product of $l$-adic topologies of $\bZ$ over all prime numbers $l$ (including p), and $G(F/F_p)$ is canonically isomorphic to the completion $\hat{\bZ}$ of $\bZ$ by this topology. For each $a \in \hat{\bZ}$, the corresponding element of $G (F/ F_p)$ will be denoted by $\sigma^a$, and the automorphism of $W_\infty$ induced by $\sigma^a$ will be denoted simply as ``$p^a$''. Examples of $F_p$ are the prime field of characteristic $p$, the $p$-adic field $\bQ_p$, the decomposition field of $p$ in the cyclotomic field $\bQ(W_\infty)$ over the rational number field $\bQ$, etc.

Now let $j$ be a variable over $F_p$, and $E$ be any elliptic curve over the rational function field $F_p(j)$, having $j$ as its absolute invariant. The equation of Tate,
\begin{equation} 
Y^2 + XY = X^3 - \frac{36}{j - 1728} X - \frac{1}{j - 1728},
\tag{$(T)_j$}
\end{equation}\pageoriginale 
gives an example of $E$. For each $n$, let $E_n$ be the group of $n$-th division points of $E$, and put $E_\infty = \bigcup\limits_{p\nmid n} E_n$. Then, since $E_n \simeq (\bZ/ n)^2$  the determinant give a homomorphism $\Aut E_n \to (\bZ / n)^\times$. Identifying the two groups $(\bZ/ n)^\times$ and $\Aut W_n$ in the canonical way, we shall consider the determinant as giving a homomorphism $\Aut E_n \to \Aut W_n$, which will be filtrated to the homomorphism at infinity,
\begin{equation}
\det: \Aut E_\infty \to \Aut W_\infty. \label{art6-eq1.1.1}
\end{equation}
Let $\fg$ be the Galois group over $F_p(j)$ of its separable closure. Then $\fg$ acts on $E_\infty$ and $W_\infty$, and the two actions are compatible with the homomorphism \eqref{art6-eq1.1.1}. In particular, by our assumption on $F_p$, the determinant of the action of each element of $\fg$ on $E_\infty$ must belong to $p^{\hat{\bZ}}$.

\setcounter{subprop}{1}
\begin{subprop}[Igusa]\label{art6-subprop1.1.2}
The subgroup of $\Aut E_\infty$ generated by the change of sign $-1_\infty$ ($1_\infty$: the identify map of $E_\infty$) and the actions of all elements of $\fg$ on $E_\infty$ consists of all those elements of $\Aut E_\infty$ whose determinants belong to $p^{\hat{\bZ}}$.
\end{subprop}

\begin{proof}
By Igusa \cite{art6-key5}, this subgroup contains all elements with determinant unity, as $j$ is a variable over $F$. The Proposition follows immediately from this and our assumption on $F_p$. (For characteristic 0, see also Shimura \cite{art6-17}, \cite{art6-18}.)
\end{proof}

\begin{coro*}
For at least one choice of the sign $\pm$, there exists an element of $\fg$ acting on $E_\infty$ as $\pm p \cdot 1_\infty$.
\end{coro*}

\begin{definitions*}
The fields on the left are by definition the fixed fields of the groups on the right. The groups are closed subgroups of $\fg$ defined by their actions on $E_\infty$ and $W_\infty$ (the two actions being connected by the determinant). The symbol ``1'' denotes the identity map of the indicated group.
$$
\xymatrix@R=0.1cm@C=0.5cm{
& L_\infty \ar@{-}[dd]\\
K_\infty \ar@{-}[ur] \ar@{-}[dd]& \\
& L_n \ar@{-}[dd] \\
K_n \ar@{-}[ur] \ar@{-}[dd]&\\
& L_1 = F(j)\\
K_1 - F_{p^2} (j) \ar@{-}[ur]& 
}
\quad
\xymatrix@R=0.1cm@C=0.5cm{
& \{ \pm 1 \; on\; E_\infty \} \ar@{-}[dd]\\
\{\pm p^{\hat{\bZ}} \;on\; E_\infty\}\ar@{-}[ur] \ar@{-}[dd]& \\
& \{\pm 1 \;on\; E_n \cdot 1 \; on \; W_\infty\} \ar@{-}[dd] \\
\{(1.1.3)\}  \ar@{-}[ur] \ar@{-}[dd]&\\
& \{1 \; on \; W_\infty \}\\
\{p^{2 \hat{\bZ}} \; on \; W_\infty\}  \ar@{-}[ur]& 
}
$$
The group\pageoriginale corresponding to $K_n$ is defined by 
\setcounter{equation}{2} 
\begin{equation}
\{\pm p^a \; on \; E_n \; and \; p^{2a} \;on\; W_\infty \textit{\; for some common } a \in \hat{\bZ}\}. \label{art6-eq1.1.3}
\end{equation}
It is the composite of the groups corresponding to $K_\infty$ and $L_n$. 

Thus, $K_n$ is a finite Galois extension of $F_p(j)$, and it is an algebraic function field of one variable with exact constant field $F_{p^2}$. The field $L_n$ is the field of modular functions of level $n$ over $F(j)$ in the sense of \cite{art6-key5}, and the relations between $L_n$ and $K_n$ are $L_n = K_n \cdot F$, $K_n = L_n \cap K_\infty$. The fields $L_\infty$, $K_\infty$ are the composites of $L_n $, $K_n$ for all $n$, and $L_\infty = K_\infty \cdot F$.

The Galois group $G(K_n/ K_1)$ is canonically isomorphic to $G(L_n / L_1)$, and hence by Prop. \ref{art6-subprop1.1.2}., canonically isomorphic to $\Aut^1 E_n/\pm I$, where $\Aut^1 E_n$ is the group of all automorphisms of $E_n$ with determinant unity. So, each choice of an isomorphism $\epsilon_n : E_n \tilde{\to} (\bZ/n)^2$  defines an isomorphism $\alpha_n : G (K_n / K_1) \tilde{\to} S L_2 (\bZ/n) / \pm I$. If we fix an isomorphism $\omega_n : W_n \tilde{\to} \bZ/n$, we can impose on $\epsilon_n$ the following condition that 
\begin{equation}
\omega_n (e(u,v)) = |\epsilon_n u, \epsilon_n v|, \; (u, v \in E_n),
\label{art6-eq1.1.4}
\end{equation}
where $e : E_n \times E_n \to W_n$ is the Weil's Riemannian form with respect to the divisor ($\equiv -$) class of \textit{degree one} (Weil \cite{art6-key19} Ch. IX), and is the matricial  determinant. This condition \eqref{art6-eq1.1.4} defines a unique $SL_2 (\bZ/n)$-class of $\epsilon_n$, and hence a unique class of $\alpha_n$ modulo inner automorphisms. We shall fix an isomorphism
\begin{equation}
\omega : W_\infty\tilde{\to} \underrightarrow{\lim} \bZ / n, \label{art6-eq1.1.5}
\end{equation}
which defines a unique class of isomorphisms
\begin{equation}
\alpha : G (K_\infty / K_1) \tilde{\longrightarrow} \underleftarrow{\lim} (SL_2 (\bZ/ n) / \pm I)
\label{art6-eq1.1.6}
\end{equation}
modulo inner automorphisms.

Let $\Gamma_1 = PSL_2 (\bZ^{(p)})$ be the modular group over $\bZ^{(p)} = \bZ[1/p]$ and $\Gamma_n$ be the principal congruence subgroup of level $n$. Then $\Gamma_1/ \Gamma_n$ is canonically isomorphic to $SL_2 (\bZ/n) \pm I$; hence the right side of \eqref{art6-eq1.1.6} is canonically isomorphic to $\underleftarrow{\lim} (\Gamma_1/\Gamma_n)$. Therefore, \eqref{art6-eq1.1.6} induces an injection
\begin{equation}
\iota : \Gamma_1 \hookrightarrow G (K_\infty / K_1).  \label{art6-eq1.1.7}
\end{equation}
which\pageoriginale is intrinsic up to inner automorphisms of $G(K_\infty/ K_1)$, once $\omega$ is fixed. The congruence subgroups of $\Gamma_1$ are in one-to-one correspondence with finite extensions of $K_1$ in $K_\infty$. In particular, $\Gamma_n$ corresponds to $K_n$ (for any choice of $\omega$).

It is easy to check that the fields $K_n$, $L_n$ are independent of the special choice of $E$. Also, the classes of $\alpha$ and $\iota$, considered up to inner automorphisms of $G(K_\infty/ K_1)$, depend only on $\omega$ and not on the special choice of $E$.

We shall call $K_n$ the field of modular functions of level $n$ over $F_{p^2}$.
\end{definitions*}

\subsection{}\label{art6-sbusec1.2}
Now we shall specify $F_p$ as the prime field of characteristic $p$, so that $F$ is the algebraic closure of $F_p$ and $W_\infty = F^\times$. For each $a \in F \cup (\infty)$, $P_a$ will denote the prime divisor of the function field $K_1 = F_{p^2} (j)$ defined by $j \equiv a (\mod P_a)$. So, for $a, b \in F$, $P_a = P_b$ holds if and only if $a$, $b$ are conjugate over $F_{p^2}$. The prime divisor $P_\infty$ is called cuspidal, and $P_a$ is called supersingular if $a$ is a super-singular ``$j$-invariant'' In the latter case, we have $a \in F_{p^2}$, so that all supersingular prime divisors of $K_1$ are of degree one over $F_{p^2}$. Other special prime divisors of $K_1$ are $P_{1728}$ and $P_0$. When $p=2$ or 3, $P_{1728} = P_0$ is the unique supersingular prime divisor of $K_1$. In all other cases, $P_{1728}(\resp. P_0)$  is supersingular if and only if $p \equiv -1 (\mod 4) (\resp. p \equiv -1 (\mod 3))$. For any intermediate field $K$ of $K_\infty/ K_1$ and a prime divisor $P$ of $K$, we call $P$ cuspidal (\resp. super-singular) when its restriction to $K_1$ is cuspidal (\resp. supersingular). We shall review some basic facts about the Galois extension $K_\infty/ L_1$.

\noindent
\textsc{The Ramification.} 
It is independent of the constant field. Hence the ramification of $K_\infty/ K_1$ is described by that of $L_\infty/ L_1$, which was determined by Igusa \cite{art6-key5} as follows.

\noindent
{\bf (IG. 1)} $P= P_\infty$. It is tamely ramified in $K_\infty / K_1$, and has an extension to $K_\infty$ whose inertia group is topologically generated by $\iota \left(\begin{pmatrix}
1&1\\
0&1
\end{pmatrix} \right)\footnote{This last point is not explicitly stated in \cite{art6-key5}, but follows easily from the arguments used there. See also \cite{art6-key8}, II, Ch. 5, \S 28 for an alternative proof.}$.

In particular,\pageoriginale the ramification index of $P_\infty$ in $K_n / K_1$ is equal to $n$, and moreover, $K_n$ is the maximum extension of $K_1$ in $K_\infty$ with this property. The second point follows immediately from the \textit{local} congruence subgroup property of $PSL_2$, that the principal congruence subgroup of level $n$ of the group $\underleftarrow{\lim} (SL_2 (\bZ/n'))$ is normally generated by $\begin{pmatrix}
1 & n \\ 0 & 1\end{pmatrix}$ in the topological sense. Thus:

\begin{subprop}
There is no non-trivial unramified extension of $K_n$ contained in $K_\infty$.
\end{subprop}

This implies that all non-trivial unramified extensions of $K_n$, for each $n$, are \textit{outside} $K_\infty$.

\noindent
\textbf{(IG. 2)} $P=P_{1728}$, $P_0$. If $p \neq 2, 3$, these prime divisors are tamely ramified in $K_\infty$ with indices 2, 3, respectively. If $p=2$ or $p=3$, $P_{1728} = P_0$ is wildly ramified with indices $12(p=2)$ or $6(p=3)$. In all cases, the ramifications of $P_{1728}$, $P_0$ in $K_\infty/ K_1$ are accomplished in $K_n/K_1$ for any $n > 1$, so that each extension of $P_{1728}$ or $P_0$ is unramified in $K_\infty/ K_n$ for $n>1$.

\noindent
\textbf{(IG. 3)} $P \neq P_\infty$, $P_{1728}$, $P_0$. These prime divisors $P$ are unramified in $K_\infty / K_1$. 

\subsection{}\label{art6-subsec1.3}
\textit{The decomposition.} The following assertion is an immediate consequence of Theorem 3, 4, 5 of our previous note \cite{art6-key8}, II, Ch. 5 (\S\S 24, 26, 28). It can also be proved directly without any essential difficulty.

\begin{subprop}\label{art6-subprop1.3.1}
A prime divisor of $K_\infty$ is of degree one over $F_{p^2}$ if and only if it is supersingular.
\end{subprop}

(This includes in particular that the cuspidal prime divisors of $K_\infty$ are \textit{not} of degree one over $F_{p^2}$.)

As a direct formal consequence of Prop. \ref{art6-subprop1.3.1}., we have:

\medskip
\noindent
\textbf{Proposition 1.3.1$'$.} 
\textit{There exists a non-zero ideal $n_0\bZ$ of $\bZ$ such that for each $b \in n_0 \bZ$, the prime divisors of $K_n$ of degree one over $F_{p^2}$ are precisely the supersingular prine divisors of $K_n$.}

Let\pageoriginale $h_n$ denote the number of supersingular prime divisors of $K_n$. It is well known that
\begin{align*}
h_1 & = \frac{1}{12} (p+13) - \frac{1}{4} \left(1 + \left(\frac{-1}{p} \right)\right) - \frac{1}{3} \left(1+ \left(\frac{-3}{p} \right) \right), \tag*{$(1.3.2)_1$}\\
h_n & = \frac{1}{12} (p-1) [K_n : K_1], \quad (n > 1). \tag*{$(1.3.2)_n$}
\end{align*}
The second formula follows from the first by using (IG2), (IG3).

In \cite{art6-key8}, Ch. 5, more intimate arithmetic relations between the group $\Gamma_n$ and the field $K_n$ are presented, but they will not be directly used here.


\section{The main theorem.}\label{art6-sec2}

\subsection{}\label{art6-subsec2.1}
For each positive integer $n$ with $p \nmid n$, let $K_n$ be the modular function field of level $n$ over the finite field $F_{p^2}$ defined in $\S 1$. Then, $K_n$ has a special finite set of prime divisors $s_1, \ldots, s_{h_n}$ of degree one, the supersingular prime divisors, where $h_n = \dfrac{1}{12} (p-1) [K_n : K_1]$ for $n >1$ (see \S \ref{art6-subsec1.3}).


\noindent
\textsc{Main Theorem (First formulation).} 
\textit{Let $M_n$ be the maximum unramified Galois extension of $K_n$, and $s_1, \ldots, s_{h_n}$ be the set of all supersingular prime divisors of $K_n$. Then the Frobenius conjugacy classes $\left\{\dfrac{M_n/K_n}{S_i} \right\}$ of $s_i (i=1, \ldots, h_n)$ generated, in the topological sense, the Galois group of $M_n / K_n$.}

(Since $M_n/K_n$ is generally non-abelian, the Frobenius automorphism associates to each prime divisor of $K_n$ a conjugacy class of the Galois group, which is called the Frobenius conjugacy class.)

In a more suggestive way, one may state it as ``\textit{the non-abelian divisor class group of $K_n$ is generated by the supersingular prime divisors $s_1,\ldots, s_{h_n}$.''}

The restriction of the theorem to abelian extensions gives:

\begin{corollary}\label{art6-coro1}
The divisor class group of $K_n$ is generated by the classes of supersingular prime divisors $s_1, \ldots, s_{h_n}$.
\end{corollary}

\begin{remark}\label{art6-rem1}
According\pageoriginale to the Grothendieck theory of fundamental groups, the Galois group of the maximum unramified Galois extension of an algebraic function field of genus $g > 0$ (over any algebraically closed constant field) is isomorphic to the projective limit of \textit{some subsystem} of the system of all finite factor groups of the abstract group defined by $2g$ generators $\alpha_1, \ldots, \alpha_g$, $\beta_1, \ldots, \beta_g$ with a single relation
$$
\alpha_1 \beta_1 \alpha^{-1}_1 \beta^{-1}_1 \ldots \alpha_g \beta_g \alpha_g^{-1} \beta^{-1}_g = 1.
$$
For characteristic 0, it is the projective limit of \textit{all} finite factor groups (as is classically well known), and for characteristic $p>0$, the best (general type) information on this subsystem is that it contains all those finite factor groups whose orders are not divisible by $p$. This gives a general idea on the structure of the Galois group. Note that \textit{our result is of a completely different type}. It also gives some topological generators of the Galois group, but its aim is not to describe the structure of the Galois group, but \textit{to describe the arithmetic distriction of supersingular prime divisors and to characterize the field $K_n$ by ramifications and supersingular complete decompositions} (see [MT 3] below). While the structure of the Galois group of $M_n/K_n$  itself is similar to the case of characteristic 0, our assertion is what properly belongs to characteristic $p>0$.

It is necessary to look at our theorem from several different angles, and so we shall present various different (but equivalent) versions of the theorem under the names [MT i]. The first formulation will hereafter be cited as [MT 1]$_n$. For example, and immediate variation is the following:

[MT 2]$_n$. \textit{There is no non-trivial unramified extension of $K_n$ in which all supersingular prime divisors $s_1, \ldots, s_{j_n}$ of $K_n$ are decomposed completely.}
\end{remark}

\begin{remark}\label{art6-rem2}
If $K'$ is a finite unramified extension of $K_n$ in which $s_1, \ldots, s_{h_n}$ are decomposed completely, the curve for $K'$ has at least $[K': K_n]. h_n$ number of $F_{p^2}$-rational points. It should be mentioned here that [MT 2]$_n$ \textit{cannot} be proved only by the estimation of the number of rational points.

The above\pageoriginale two formulations of the theorem are convenient for the presentations of the \textit{new} part of its content. But to exhibit its natural form, we must let $n \to \infty$. As noted in \S \ref{art6-sec1}, the constant field extension $L_n = K_n \cdot F$ is nothing but the modular function field of level $n$ over $F(j)$ in the sense of \cite{art6-key5}. Let $M_n$ be, as in [MT 1]$_n$, the maximum unramified Galois extension of $K_n$ (or equivalently, of $L_n$), and take the composites $M_\infty$, $L_\infty$, $K_\infty$ of $M_n$, $L_n$, $K_n$ for all $n$ (with $p\nmid n$):
$$
\xymatrix@R=-0.15cm@C=0.5cm{
& & & M_\infty \ar@{-}[ddll]\\
& & \ar@{-}[ddd] &\\
& L_\infty \ar@{-}[ddd]& &\\
K_\infty\ar@{-}[ddd]\ar@{-}[ur] & & &\\
& & M_n \ar@{-}[dl] &\\
& L_n \ar@{-}[ddd]& &\\
K_n \ar@{-}[ddd]\ar@{-}[ur] & & &\\
& & &&\\
& L_1 = F (j) & &\\
K_1 = F_{p^2} (j)\ar@{-}[ur] & & &
}
$$
By Prop. \ref{art6-subprop1.2.1}, we have $M_n \cap K_\infty = K_n$ and (equivalently) $M_n \cap L_\infty =L_n$. No, [MT 1]$_n$ gives immediately:

[MT 1]$_\infty$ (\resp. [MT 2]$_\infty$). \textit{One can replace $n$ by $\infty$ in} [MT 1]$_n$ (\resp. [MT 2]$_n$).

Here, by definition, the unramified extensions of $K_\infty$ are those extensions of $K_\infty$ contained in $M_\infty$.

Fix any positive integer $r$ with $r > 1$ and $p \nmid r$. Let $n$ be any positive integer with $p \nmid n$. Then, as reviewed in \S \ref{art6-subsec1.2}, $L_{nr}/ L_r$ is unramified outside the cusps, and the ramification index at each cusp is precisely $n$. In particular, the cusps are tamely ramified. Therefore, $M_{nr}$ (\resp. $M_\infty$) is the maximum Galois extension of $K_r$ satisfying the following condition (i 1)(i 2)$_n$ (\resp. (i 1) (i 2)$_\infty$);
\begin{itemize}
\item[(i 1)~] \textit{it is unramified outside the cusps of $K_r$,}

\item[(i 2)$_n$] \textit{the ramification index of each cusp is a factor of $n$, \resp.}

\item[(i 2)$_\infty$] \textit{each cusp is at most tamely ramified.}
\end{itemize}
Therefore, we obtain from [MT 2] the following simple characterization of $K_{nr}$ (\resp. $K_\infty$).
\end{remark}

\medskip
[MT 3] $K_nr$ (\resp $K_\infty$) \textit{is the maximum Galois extension of $K_r$ satisfying} (i 1)(i 2)$_n$ (\textit{\resp} (i 1)(i 2 )$_\infty$) \textit{and the following property} (ii):

\begin{itemize}
\item[(ii)] \textit{the supersingular prime divisors of $K_r$ are decomposed completely.}
\end{itemize}

Of course, to be able to say that ``[MT 3] characterizes $K_{nr}$ (or $K_\infty$)'', it is necessary that the base field $K_r$, its cusps, and its supersingular prime divisors are all explicitly presentable. A well-known case when they are explicitly presented is where $r = 2 (p \neq 2)$. Then $K_2 = F_{p^2} (\lambda)$, the rational function field, with cusps at $0, 1, \infty$, and supersingular prime divisors at the zeros of the polynomial
$$
P(\lambda) = \sum\limits^b_{i=0} \begin{pmatrix}
b \\
i
\end{pmatrix}^2  \lambda^i, \quad (b = \frac{1}{2} (p-1)),
$$
the prime divisors of $K_2$ being expressed by the residue classes of $\lambda$. Therefore, this special case gives the following.
 
\textit{Let $p$ be an odd prime and $n$ be a positive integer with $p \nmid n$. Then $K_\infty (\resp. K_{2n})$ is the maximum Galois extension of $F_{p^2}(\lambda)$ satisfying:}
\begin{itemize}
\item[(i 1)] \textit{it is unramified outside 0, 1, $\infty$:}

\item[(i 2)] (\resp (i 2)$_n$) 0, 1, $\infty$ \textit{are tamely ramified (\resp. 0, 1, $\infty$ are tamely ramified with ramification index dividing $n$);}

\item[(iii)] \textit{the zeros of $P(\lambda)$ are decomposed completely.}
\end{itemize}

One may restate [MT 3] in a more suggestive way as ``\textit{the subgroup of the non-abelian divisor class group of $K_r$ with conductor $\prod\limits^\nu_{i=1} c_i$, the product of all distinct cusps of $K_r$, and generated by the supersingular prime divisors $s_1, \ldots, s_{h_r}$ of $K_r$, is precisely that group corresponding to the Galois extension $K_\infty$ of $K_r$''.}

The restriction to abelian extensions gives:

\begin{corollary}\label{art6-coro2}
The subgroup of the divisor class group of $K_r$ with cuspidal conductor $\prod\limits^\nu_{i=1} c_i$ and generated by the supersingular prime divisors $s_1, \ldots, s_{h_r}$ is precisely that group corresponding to the maximum abelian extension of $K_r$ contained in $K_\infty$.
\end{corollary}

For example,\pageoriginale if $r = 2$, the maximum abelian extension of $K_2$ in $K_\infty$ is given by 
\begin{gather*}
F_{p^2 } (\lambda; (-\lambda)^{1/8}, (-\lambda')^{1/8}, (\frac{1}{2} \lambda \lambda')^{1/3}) \qquad (p \neq 2, 3), \\
F_{p^2 } (\lambda; (-\lambda)^{1/8} , (-\lambda')^{1/8}) \qquad (p = 3),
\end{gather*}
where $\lambda' = 1 -\lambda$. Therefore, we obtain easily from Corollary \ref{art6-coro2} the following multiplicative property of the set of supersingular ``$\lambda$-invariants''.

\begin{corollary}[$p \neq 2,3$]\label{art6-coro3}
Consider the group $\tilde{G} = (F^\times_{p^2})^8 \times (F^\times_{p^2})^8$ and the subgroup $H= \{(x,y) \in \tilde{G} | xy \in (F^\times_{p^2})^{24} \}$ of $\tilde{G}$ with index 3. Define an $H$-coset $H'$ by $H' = \{(x,y) \in \tilde{C} | \frac{1}{2} xy \in (F^\times_{p^2})^{24}\}$, and let $G$ be the subgroup of $\tilde{G}$ generated by $H'$. Let $S$ be the set of all zeros of $P(\lambda)$. For each $s \in S$ put $g_s = (-s, -s')$, where $s' = 1 -s$. Then 
\begin{description}
\item[\quad \qquad {\rm (a)}] $g_s \in H'$ for any $s \in S$,

\item[\quad \qquad {\rm (b)}] $g_s \cdot g^{-1}_t (s, t \in S)$  generate $H$;

\item[{\rm hence ~~(c)}]  $g_s (s \in S)$ generate $G$.
\end{description}
\end{corollary}

Finally, let $\Gamma_1 = PSL_2 (\bZ^{(p)})$ and $\Gamma_n$ be the principal congruence subgroup of level $n$. Then, in view of the isomorphism $G (K_\infty/ K_1) \tilde{\to} \underleftarrow{\lim} \Gamma_1 / \Gamma_n$ (\S \ref{art6-subsec1.1}) and the congruence subgroup property of $\Gamma_1$, [MT 3] can also be formulated as follows:

[MT 4]. \textit{There is a categorical equivalence between subgroups with finite indices of $\Gamma_r$ and finite extensions of $K_r$ satisfying (i 1), (i 2) and (ii).}

Or in short,  ``\textit{$\Gamma_r$ is the strict fundamental group of $K_r$ defined by} (i 1), (i 2) {\it and} (ii)''.

In \S \ref{art6-sec4}, a more geometric version of [MT]'s will be given. 


\begin{remark}\label{art6-rem3}
[MT 3] characterizes the extensions $K_{nr}/ K_r$, for $r > 1$. If one wants to take $K_1$ as the base field, one must first observe that not only the cusp $P_\infty$ but also $P_{1728}$ and $P_0$ are ramified in $K_\infty/ K_1$. If $p\neq 2$, 3 and $n >1$, the ramification indices of these three prime divisors of $K_1$ in $K_n/K_1$ are $n$, 2, 3, respectively (and all other prime divisors of $K_1$ are unramified). So, by [MT 2]$_n$, $K_n$ is a \textit{maximal} extension\pageoriginale of $K_1$ having (i) this ramification property, and (ii) the property that all prime divisors lying on the supersingular prime divisors of $K_1$ are of degree one over $F_{p^2}$. But here, we cannot replace ``\textit{maximal}'' by ``\textit{maximum}''. In fact, due to the situation that the ramifying prime divisors $P_{1728}$ or $P_0$ can be supersingular (for  $p\equiv -1 (\mod 4)$ or $-1 (\mod 3)$ respectively), the composite of two extensions of $K_1$ satisfying above two properties may \textit{not} satisfy them. This is why we did not formulate the characterization of the extension $K_{nr}/K_r$ for $r =1$. Our theorem was formulated as a conjecture in \cite{art6-key7}, \cite{art6-key8} and \cite{art6-key9}. Among them, the formulation in \cite{art6-key8}, II, Ch. 5, ``$\hat{\fK} = \fK$ ?'', contains an error in the definition of $\hat{\fK}$ arising from the misobservation of this subtle situation which, of course, can be easily corrected.\footnote{The second definition on p. 179 (\cite{art6-key8} II) is the correct one.}
\end{remark}

\section{The complex system $\{R_n {\displaystyle{\mathop{\longleftarrow}^\varphi}} R^0_n  {\displaystyle{\mathop{\longrightarrow}^{\varphi'}}} R'_n \}$.}\label{art6-sec3}
In this section, we shall prove a lemma in characteristic 0, which is \textit{the complex version} of the theorem to proved. It is a reflection of the congruence subgroup property of the modular group over $\bZ^{(p)}$. This property is in fact faithfully reflected due to the fact that the modular group over $\bZ^{(p)}$ is a free product of two modular groups over $\bZ$ with amalgamation. Our theorem will finally be reduced to this lemma.

\subsection{}\label{art6-subsec3.1}
Consider a system $\{R {\displaystyle{\mathop{\longleftarrow}^\varphi}} R^0  {\displaystyle{\mathop{\longrightarrow}^{\varphi'}}} R' \}$ of compact Riemann surfaces $R$, $R'$, $R^0$ and surjective morphisms $\varphi$, $\varphi'$. \textit{unramified covering} of such a system $\{R {\displaystyle{\mathop{\longleftarrow}^\varphi}} R^0  {\displaystyle{\mathop{\longrightarrow}^{\varphi'}}} R' \}$ is defined by a commutative diagram
\setcounter{equation}{0}
\begin{equation}
\vcenter{
\xymatrix@R=-0.05cm@C=1cm{
& R^{\ast^0} \ar[dr]^{\varphi^{\ast'}} \ar[dl]_{\varphi^\ast} \ar[ddd]_{f^0} & \\
R^\ast \ar[ddd]_{f} &&  R^{\ast'} \ar[ddd]_{f'}\\
& & &\\
& R^0 \ar[dr]_{\varphi'}  \ar[dl]^{\varphi}& \\
R & & R'}
}\label{art6-eq3.1.1}
\end{equation}
of surjective\pageoriginale morphisms beween compact Riemann surfaces satisfying the following conditions (a), (b), (c):
\begin{itemize}
\item[(a)] \textit{The degrees of morphisms indicated by parallel arrows are equal.}

\item[(b)] \textit{Let $F$, $F'$ be the smallest Galois coverings of $R$, $R'$ containing $f$, $f'$ as subcoverings, respectively. Then $F$ and $\varphi$, $F'$ and $\varphi'$, are both linearly disjoint.}

\item[(c)] \textit{The vertical coverings $f$, $f'$ and $f^0$ are unramified.}
\end{itemize}

The degree of the vertical coverings is called \textit{the degree of the covrring} \eqref{art6-eq3.1.1}. Equivalence of two such coverings is defined by a commutative diagram connecting the two covering systems by three isomorphisms. For example, if $\epsilon^0$ is an automorphism of $R^{\ast^0}$ commuting with $f^0$, then replacing $\varphi^\ast$ by  $\varphi^\ast \circ \epsilon^0$ (and leaving all others fixed) only gives an equivalent covering. In fact, by (b), there is some automorphism $\epsilon$ of $R^\ast$ commuting with $f$, such that $\varphi^\ast \circ \epsilon^0 = \epsilon \circ \varphi^\ast$. But then, it is clear that the new covering is equivalent to the old one. \textit{Up to equivalence, giving an unramified covering \eqref{art6-eq3.1.1} is the same thing as giving a pair $\{f, f'\}$ of unramified coverings $f: R^\ast \longrightarrow R$, $f' : R^{\ast'} \longrightarrow R'$ satisfying (b), for which the fiber products $\fprod{R^\ast}{R^0}{R}$ and $\fprod{R^0}{R^{\ast'}}{R'}$ are isomorphic as coverings of $R^0$.}
(These products are \textit{connected} by the linear disjointness, and \textit{non-singular} by the unramifiedness of $f$, $f'$; so that they are also compact Riemann surfaces.)
\begin{equation*}
\vcenter{
\xymatrix@R=-0.05cm@C=0.9cm{
& \fprod{R^\ast}{R^0}{R} \ar[dddr] \ar[dl]  & \simeq & \fprod{R^0}{R^{\ast}}{R'} \ar[dr] \ar[dddl] & \\
R^\ast\ar[dddr]_-f & & & & R^{\ast'}\ar[dddl]^-{f'}\\
& & & & & \\
& & R^0 \ar[dr] \ar[dl] & & \\
& R & & R' &
}}\tag*{(3.1.1$'$)}\label{art6-eq3.1.1'}
\end{equation*}
By the above remark, the choice of $\simeq$ in \eqref{art6-eq3.1.1'} does not affect the equivalence class.

\begin{defi*}
The system $\{R \xleftarrow{\varphi} R^0 \xrightarrow{\varphi'} R'\}$ will be called `simply-connected, if it has no other unramified covering than that of degree one.
\end{defi*}

\subsection{}\label{art6-subsec3.2}
Now let\pageoriginale $n$ be any positive integer with $p \nmid n$ and $\Delta_n$ be the principle congruence subgroup of level $n$ of the modular group $\Delta_1 = PSL_2 (\bZ)$ over $\bZ$. Put
\begin{gather*}
\Delta'_n  = 
\begin{pmatrix} 
p & 0\\
0 & 1
\end{pmatrix}^{-1}
\Delta_n
\begin{pmatrix} 
p & 0 \\
0 & 1
\end{pmatrix} = 
\left\{
\left.
\begin{pmatrix} 
a & p^{-1} & b\\
pc & d
\end{pmatrix}
\right| 
\begin{pmatrix} 
a & b \\
c & d
\end{pmatrix} \in 
\Delta_n
 \right\},\\
\Delta^0_n = \Delta_n \cap \Delta'_n = \left\{ \left. 
\begin{pmatrix} 
a & b \\
c & d
\end{pmatrix} \in \Delta_n \right| c \equiv 0 \; (\mod p)
 \right\}.
\end{gather*}
Then they define a system $\{\Delta_n \leftarrow \Delta^0_n \rightarrow \Delta'_n\}$ of fuchsian groups, the arrows indicating the inclusions. Note that $\Delta^0_n$ is of index $p+1$ in $\Delta_n$ and $\Delta'_n$. Let $\{R_n \xleftarrow{\varphi} R^0_n \xrightarrow{\varphi'} R'_n\}$ be the corresponding system of compact Riemann surfaces, \ie, $R_n$, $R'_n$, $R^0_n$ are the compactified quotients of the complex upper half plane by $\Delta_n$, $\Delta'_n$, $\Delta^0_n$; and $\varphi$, $\varphi'$ are the canonical coverings defined by the inclusions $\Delta^0_n \hookrightarrow \Delta_n$, $\Delta^0_n \hookrightarrow \Delta'_n$.

\setcounter{lemma}{1}
\begin{lemma}\label{art6-lem3.2}
For each positive integer $n$ with $p \nmid n$, the system $\{R_n \xleftarrow{\varphi} R^0_n \xrightarrow{\varphi'} R'_n \}$ is simply-connected.
\end{lemma}

Let $\Gamma_1 = PSL_2 (\bZ^{(p)})$ be the modular group over $\bZ^{(p)}$ and $\Gamma_n$ be the principal congruence subgroup of level $n$. The proof is based on the following properties (A), (B) of $\Gamma_n$.
\begin{itemize}
\item[(A)] $\Gamma_n$ is the free product of $\Delta_n$ and $\Delta'_n$ with  amalgamated subgroup $\Delta^0_n$.

\item[(B)] $\Gamma_n$ is normally generated by $\begin{pmatrix}
1 & n \\0 & 1\end{pmatrix}$ in $\Gamma_1$;  in particular, it is generated only by parabolic elements.
\end{itemize}

The property (A) was proved in our previous work. It is a corollary of the corresponding property of the local groups (\cite{art6-key8}, I, Ch. 2, \S 28, p. 111, or a more full-fledged geometric exposition in \cite{art6-key15}). The property (B) was proved by Mennicke \cite{art6-key12} and Serre \cite{art6-key16}. It is equivalent to the congruence subgroup property of $\Gamma_1$ modulo the local congruence subgroup property.

\subsection{\textsc{Proof of Lemma} 3.2.}\label{art6-subsec3.3}
Write
$$
\left\{R \xleftarrow{\varphi} R^0 \xrightarrow{\varphi'} R' \right\} \text{~~ and ~~} \left\{\Delta \longleftarrow \Delta^0 \longrightarrow \Delta' \right\}
$$
instead of \pageoriginale of writing with suffix $n$. Suppose that there is an unramified covering \eqref{art6-eq3.1.1} of degree $m$. Then there is a corresponding commutative diagram of inclusions of fuchsian groups
\setcounter{equation}{0}
\begin{equation}
\vcenter{
\xymatrix@R=-0.05cm@C=1cm{
& \Delta^{\ast 0} \ar[dr] \ar[dl] \ar[ddd] & \\
\Delta^\ast \ar[ddd] &&  \Delta^{\ast'} \ar[ddd]\\
& & &\\
& \Delta^0 \ar[dr]  \ar[dl] & \\
\Delta & & \Delta'}
}\label{art6-eq3.3.1}
\end{equation}
with 
\begin{gather}
(\Delta: \Delta^\ast) = (\Delta' : \Delta^\ast) = (\Delta^0 : \Delta^{\ast 0}) = m, \label{art6-eq3.3.2}\\
\Delta^\ast \cdot \Delta^0 = \Delta, \quad \Delta^{\ast'} \cdot \Delta^0 = \Delta', \tag*{$|$\quad~}\\
\Delta^\ast \cap \Delta^0 = \Delta^{\ast'} \cap \Delta^0 = \Delta^{\ast 0}, \label{art6-eq3.3.3}\\
Pb(\Delta) = Pb (\Delta^\ast), \;Pb (\Delta') = Pb (\Delta^{\ast'}) , \;Pb (\Delta^0) = Pb (\Delta^{\ast 0}), \label{art6-eq3.3.4}
\end{gather}
where $Pb (\;)$ is the set of all parabolic elements of the group inside the parenthesis. The first two equalities of \eqref{art6-eq3.3.3} are the consequences of the condition (b), the rest of \eqref{art6-eq3.3.3} is the consequence of the first two equalities and the condition (a), and \eqref{art6-eq3.3.4} is a consequence of (c).

Now let $\Gamma^\ast$ be the subgroup of $\Gamma = \Gamma_n$ generated by $\Delta^\ast$ and $\Delta^{\ast'}$. We shall deduce from \eqref{art6-eq3.3.2}, \eqref{art6-eq3.3.3} and the property (A) of $\Gamma$ that 
\begin{equation}
(\Gamma: \Gamma^\ast ) = m, \label{art6-eq3.3.5}
\end{equation}
and from \eqref{art6-eq3.3.4} that 
\begin{equation}
Pb (\Gamma) = P b(\Gamma^\ast). \label{art6-eq3.3.6}
\end{equation}
But then, by the property (B) of $\Gamma$, \eqref{art6-eq3.3.6} would imply $\Gamma = \Gamma^\ast$, and hence $m =1$ by \eqref{art6-eq3.3.5}. So it remains to prove \eqref{art6-eq3.3.5} and \eqref{art6-eq3.3.6}. 

Let $1 = M_0, M_1, \ldots, M_p$ (\resp $1 = M'_0, M'_1, \ldots, M'_p$) be representatives of the coset spaces $\Delta^{\ast 0} /\Delta^\ast$ (\resp $\Delta^{\ast 0} / \Delta^{\ast'}$). Then by \eqref{art6-eq3.3.3}, they are also representatives of $\Delta^0/ \Delta$ (\resp $\Delta^0/ \Delta'$). Since $\Gamma$ is generated by $\Delta$ and $\Delta'$, each element $\gamma$ of $\Gamma$ can be expressed in the form
\begin{equation}
\gamma = \delta_{0} M_{i_1} M'_{j_1} \ldots M_{i_r} M'_{j_r} \quad (\delta_0 \in \Delta^0, j_1, \ldots , i_r \neq 0). 
\label{art6-eq3.3.7}
\end{equation}
The set\pageoriginale of all those $\gamma \in \Gamma$ having an expression \eqref{art6-eq3.3.7} with $\delta_0 \in \Delta^{\ast 0}$ forms a subgroup of $\Gamma$, since $M_i$ and $M'_j$ belong to $\Delta^\ast$ and $\Delta^{\ast'}$ respectively. It is clear that this subgroup is $\Gamma^\ast$. Therefore, $\Delta^0 \cdot \Gamma^\ast  = \Gamma$; hence $(\Gamma : \Gamma^\ast) = (\Delta^0 : \Delta^0 \cap \Gamma^\ast)$. But \textit{by the property} (A) \textit{ of $\Gamma$, the expression \eqref{art6-eq3.3.7} is unique} (see Krosh \cite{art6-key10} II, for the general theorems on free products with amalgamations). Therefore, we have $\Delta^0 \cap \Gamma^\ast = \Delta^{\ast 0}$, which gives $(\Gamma : \Gamma^\ast) = (\Delta^0 : \Delta^{\ast 0}) =m$. Thus,  \eqref{art6-eq3.3.5} is settled.

To check \eqref{art6-eq3.3.6}, let $\gamma \in \Gamma$ be parabolic, and put $\gamma = \pm I + p^{-k}A (k \geqslant 1)$, where $I$ is the identity matrix and $A$ is a $\bZ$-integral matrix with $A^2=0$. Then $\gamma^{p^k} = \pm I + A \in PSL_2 (\bZ)$. Hence $\gamma^{p^k}$ belongs to $\Delta$ and hence to $\Delta^\ast$ by \eqref{art6-eq3.3.4}. Therefore, $\gamma^{p^k} \in \Gamma^\ast$. But for any parabolic element $\gamma$ of $\Gamma_1$, its order relative to any subgroup with finite index of $\Gamma_1$ is not divisible by $p$. In fact, by the above argument, $\gamma^{p^k} \in PSL_2 (\bZ)$, and hence $\gamma^{p^k}$ is conjugate to an integral power of $\begin{pmatrix}1 & 1 \\0 & 1\end{pmatrix}$. Therefore, $\gamma$ is conjugate to 
$\begin{pmatrix}
1 & b \\ 0 & 1 \end{pmatrix}$ with some $b \in \bZ^{(p)}$. But $\begin{pmatrix}
1 & b\\ 0 & 1
\end{pmatrix}$ is conjugate to its $p^2$-th power $\left(\text{by } \begin{pmatrix}
p^{-1} & 0\\ 0 & p
\end{pmatrix}\right)$; hence $\gamma$ is also conjugate to its $p^2$-th power $\gamma^{p^2}$. This gives the above assertion, and hence also that $\gamma \in \Gamma^\ast$. Therefore, \eqref{art6-eq3.3.6} is also settled. The lemma follows.


\subsection{}\label{art6-subsec3.4}
As for the covering \eqref{art6-eq3.1.1} of those systems obtained from the system of fuchsian groups $\{\Delta \leftarrow \Delta^0 \rightarrow \Delta'\} - \{\Delta_n \leftarrow \Delta^0_n \rightarrow \Delta'_n\}$, \textit{the linear disjointness of $F$ and $\varphi$ (\resp. $F'$ and $\varphi'$) follows automatically from the unramifiedness of $f$ (\resp. $f'$)}. In fact, let $\Delta^\ast$ be the fuchsian group corresponding to $R^\ast$, and $\Delta^{\ast\ast}$ be the intersection of all conjugates of $\Delta^\ast$ in $\Delta$. Then the unramifiedness of $f$ gives $Pb (\Delta) = Pb (\Delta^\ast)$; hence also $Pb(\Delta) = Pb (\Delta^{\ast\ast})$. But $Pb(\Delta^0) \subsetneq Pb(\Delta)$, as $\begin{pmatrix}
1&0\\ n&1
\end{pmatrix} \not\in \Delta^0$. In particular, $\Delta^{\ast \ast}$ is not contained in $\Delta^0$. But as is well known, there is no proper intermediate group between $\Delta$ and $\Delta^0$. (This is clear since $\Delta' / \Delta$ can be identified with the projective line over $F_p$ by $\begin{pmatrix}
a & b \\c & d 
\end{pmatrix} \rightsquigarrow dc^{-1} (\mod p)$, and $\Delta$ acts doubly transitively on this line as a group of linear fractional transforms.) Therefore, $\Delta^0 \bigdot \Delta^{\ast\ast} \Delta$, which implies the linear disjointness of $F$ and $\varphi$.


\subsection{}\label{art6-subsec3.5}
\textit{The general picture.}\pageoriginale  To clarify the general situation, we note the following. As in \S \eqref{art6-subsec3.2} , let $\Gamma_1 = PSL_2 (\bZ^{(p)})$, and put $\Delta_1 = PSL_2 (\bZ)$, $\Delta'_1 = \begin{pmatrix}
p &0\\
0&1
\end{pmatrix}^{-1} \Delta_1 
\begin{pmatrix}
p&0\\
0&1
\end{pmatrix}$ and $\Delta^0_1 = \Delta_1 \cap \Delta'_1$. Then, (i) $\Gamma_1$ is a free product of $\Delta_1$ and $\Delta'_1$ with amalgamated subgroup $\Delta^0_1$, and (ii) for any subgroup $\Gamma$ of $\Gamma_1$ with finite index, it holds that $\Delta^0_1  \cdot \Gamma = \Gamma_1$. (Since $\Gamma_1$ is dense in $PSL_2 (\bQ_p)$, $\bQ_p$  being the $p$-adic field, the topological closure of $\Gamma$ in $PSL_2 (\bQ_p)$ is of finite index in $PSL_2 (\bQ_p)$. But since $PSL_2(\bQ_p)$ is an infinite simple group, it has no non-trivial subgroups with finite indices. Therefore, $\Gamma$ is \text{dense} in $PSL_2 (\bQ_p)$. This gives $\Delta^0_1 \bigdot \Gamma = \Gamma_1$.) From these two properties (i), (ii), it follows directly that the associateion
\begin{equation}
\Gamma \rightsquigarrow \{\Delta, \Delta'\} ; \; (\Delta = \Gamma \cap \Delta_1, \; \Delta' = \Gamma \cap \Delta'_1) 
\label{art6-eq3.5.1}
\end{equation}
gives a one-to-one correspondence between subgroups $\Gamma$ of $\Gamma_1$ with finite indices and the pairs $\{\Delta, \Delta'\}$ of subgroups $\Delta \subset \Delta_1$, \; $\Delta' \subset \Delta'_1$ with finite indices satisfying $\Delta^0_1 \bigdot \Delta = \Delta_1$, $\Delta^0_1 \bigdot \Delta' = \Delta'_1$ and $\Delta^0_1 \cap \Delta = \Delta^0_1 \cap \Delta'$. It is also easy to check, by the argument similar to that used in the proof of Lemma 3.1, that $\Gamma$ is a free product of $\Delta$ and $\Delta'$ with amalgamated subgroup $\Delta^0 = \Delta \cap \Delta'$, and that $(\Gamma_1 : \Gamma) =(\Delta_1 : \Delta) = (\Delta'_1: \Delta') = (\Delta^0_1: \Delta^0)$. We may call $\{\Delta, \Delta'\}$ \textit{the canonical generating pair of fuchsian groups for} $\Gamma$. For $\Gamma = \Gamma_n$, it is nothing but $\{\Delta_n, \Delta'_n\}$ defined in \S \ref{art6-subsec3.2}.

Now let $\{R \xleftarrow{\varphi} R^0 \xrightarrow{\varphi'} R'\}$ be the system of compact Riemann surfaces corresponding to $\{\Delta \leftarrow \Delta^0 \rightarrow \Delta'\}$. Then there is a canonical covering 
\setcounter{equation}{1}
\begin{equation}
\vcenter{
\xymatrix{
R \ar[d]_-f & R^0\ar[r]^-{\varphi'}\ar[d]_-{f^0}\ar[l]_-{\varphi} & R' \ar[d]_-{f'}\\
R_1 & R^0_1 \ar[r]_-{\varphi'_1} \ar[l]^-{\varphi_1} & R'_1
}}
\textit{corresponding to }
\vcenter{
\xymatrix{
\Delta \ar[d] & \Delta^0\ar[r] \ar[d] \ar[l]& \Delta' \ar[d]\\
\Delta_1 & \Delta^0_1 \ar[r] \ar[l]& \Delta'_1
}}\label{art6-eq3.5.2}
\end{equation}
The covering \eqref{art6-eq3.5.2} satisfies the conditions (a), (b) of \S \ref{art6-subsec3.1}, and instead of (c), the following weaker condition:

%%%%%% 180 page

\eqref{art6-eq}

%\rightsquigarrow


