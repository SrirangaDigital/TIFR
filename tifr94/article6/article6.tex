
\title{ON MODULAR CURVES OVER FINITE FIELDS}
\markright{ON MODULAR CURVES OVER FINITE FIELDS}

\author{By~ YASUTAKA IHARA}
\markboth{YASUTAKA IHARA}{ON MODULAR CURVES OVER FINITE FIELDS}

\date{}
\maketitle


%\setcounter{page}{21}
\setcounter{pageoriginal}{160}

\noindent
\textsc{Introduction.}\pageoriginale We shall prove a certain basic theorem on modular curves over finite fields. This gives a solution of a conjecture raised at the Boulder summer school in 1965 \cite{art6-key7}. The congruence subgroup property of the modular group of degree two over $\bZ(1/p)$ is faithfully reflected in this theorem.

Let $p$ be a fixed prime number, $F$ be the algebraic closure of the prime field of characteristic $p$, and $F_{p^m} (m \geqslant 1)$ be the finite subfield of $F$ with $p^m$ elements. Let $j$ be a variable over $F$. For each positive integer $n$ with $p\nmid n$, let $L_n$ be the field of modular functions of level $n$ over $F(j)$ (with respect to $j$) in the sense of Igusa \cite{art6-key5}. Then $L_n / F (j)$ is a Galois extension with Galois group isomorphic to $SL_2 (\bL/ n) / \pm I$. We know that there is a \textit{canonical} choice of a Galois extension $K_n / F_{p^2} (j)$ with the same Galois group as $L_n / F(j)$, satisfying $K_n \cdot F= L_n$ and $K_n \cap F = F_{p^2}$ (\cite{art6-key7}, \cite{art6-key8}; see \S 1). The field $K_n$ is obtained as the fixed field of the action of $p^{\bZ} on L_n$. Call a prime divisor $P$ of $K_n$ \textit{supersingular} if the residue class of $j$ at $P$ is a supersingular ``$j$-invariant''. An important fact, which becomes apparent by lowering the field of modular functions from $L_n$ to $K_n$, is that \textit{all supersingular prime divisors of $K_n$ are of degree one over $F_{p^2}$}. This proof is in \cite{art6-key8}. For $n > 1$, the number of supersingular prime divisors of $K_n$ is equal to (1/12) $(p-1) [K_n: K_1]$. We shall prove that following 

\begin{theorem*}
There is no non-trivial unramified extension of $K_n$ in which all supersingular prime divisors are decomposed completely.
\end{theorem*}

This, combined with Igusa's result \cite{art6-key5} on the ramifications of $L_n/L_1$ (which says that the ramifications of $L_n/L_1$ are almost analogous to those of the corresponding situation in characteristic 0), gives the following complete characterization of the Galois extension $K_\infty/ K_r$, where $r >1$ and $K_\infty$ is the composite of $K_n$ for all $n$:

\begin{coro*}
$K_\infty/ K_r$ is the maximum Galois extension of $K_r$ satisfying the following two properties; (i) it is tamely ramified, and unramified outside\pageoriginale the cusps of $K_r$; (ii) the supersingular prime divisors $af$ $K_r$ are decomposed completely.
\end{coro*}     

If one wants to emphasize the arithmetic nature of the theorem, one may restate the theorem in the following way: \textit{``the Galois group of the maximum unramified Galois extension of $K_n$ is topologically generated by the $p^2$-th power Frobenius automorphisms of (all extensions of) the supersingular prime divisors of $K_n$.''} Some direct applications of this to the distribution of supersingular prime divisors are given in \S 2. If one wants to look at the theorem as a theorem on the fundamental group, one may formulate it as follows. let $\Gamma_1 = PSL_2 (Z^{(z)})$ be the modular group over $\bZ^{(p)} = \bZ[1/p]$ and $\Gamma_n$ the principle congruence subgroup of $\Gamma_1$ with level $n (p\nmid n)$. Let $X_n$ be a complete non-singular model of $K_n$. Then, for $r > 1$, \textit{``there is a categorical equivalence between subgroups of $\Gamma_r$ with finite indices and those finite separable irreducible coverings of $X_r$ defined over $F_{p^2}$ and satisfying the above two conditions} (i), (ii).'' (The second condition (ii) is geometrically stated as ``all points lying on the supersingular points of $X_r$ are $F_{p^2}$-rational''.) In this sense, the modular group of degree two over $\bZ^{(p)}$ is \textit{the fundamental group} defined by (i), (ii).

We shall mention here that it is essential to consider the groups over $\bZ^{(p)}$ and the curves over $F_{p^2}$. It cannot be replaced by the groups over $\bZ$ and the curves $F$. Roughly speaking, considering the group over $\bZ$ corresponds to considering the condition (i) alone. The condition (i) alone cannot characterize the system of coverings defined by the modular curves $X_n$, as each $X_n$ for $n \geqslant 6$ has so many non-trivial unramified coverings and they are all from \textit{outside} the system (see \S 1. 2, \S 2). Such unramified coverings of $X_n$ correspond to \textit{non-congruence subgroups} of $PSL_2 (\bZ)$. The passage from $PSL_2 (\bZ)$ to $PSL_2 (\bZ^{(p)})$ kills all non-congruence subgroups. Our theorem says, as a corresponding geometric fact, that all non-trivial unramified coverings of $X_n$ are killed by the condition (ii), \ie, by \textit{the super-singular Frobeniuses}. The connection between the function field $K_n$ (or the curve $X_n$) and $\Gamma_n$ is essential, as our previous studies \cite{art6-key7}, \cite{art6-key8}, and also the proof of the present theorem show.

In \S\ref{art6-sec1}, we\pageoriginale shall give some preliminaries, and in \S \ref{art6-sec2}, we shall state our theorem in several different forms, with some corollaries. The proof will be given in he rest of the paper, in \S\S \ref{art6-sec3}-\ref{art6-sec6}.

\textsc{Method for proof.} The modular group over $\bZ^{(p)}$ has the congruence subgroup property (Mennicke, Serre), and our theorem is a faithful reflection of this property. The proof goes through directly by chasing away the obstacles (clouds!). It consists of the following steps (a)-(d).
\begin{itemize}
\item[(a)] A geometric interpretation of the congruence subgroup property as the ``simply-connectedness'' of the system $\{R_n \xleftarrow{\varphi} R^0_n \xrightarrow{\varphi'} R'_n\}$ of three compact Riemann surfaces. The Riemann surfaces $R_n$, $R'_n$ and $R^0_n$ are defined by the fuchsian groups $\Delta_n, \Delta'_n$ and $\Delta^0_n$, where $\Delta_n$ is the principal congruence subgroup of $PSL_2 (\bZ)$ of level $n$,
$$
\Delta'_n = 
\begin{pmatrix}
p & 0 \\
0 & 1
\end{pmatrix}^{-1}
\Delta_n 
\begin{pmatrix}
p & 0 \\
0 & 1
\end{pmatrix}, \text{ and }
\Delta^0_n = \Delta_n \cap \Delta'_n.
$$
The system is essentially the graph of the modular correspondence ``$T(p)$ for level $n$''. (However, $R^0_n$ is not immersed in the product $R_n \times R'_n$.) The interpretation is made possible due to the fact that $\Gamma_n$ is the \textit{free} product of $\Delta_n$ and $\Delta'_n$ with amalgamated subgroup $\Delta^0_n$. And the congruence subgroup property is used in the following form that $\Gamma_n$ is generated only by the parabolic elements (\S 3).


\item[(b)] A geometric interpretation of the theorem to be proved as the ``simply-connectedness'' of the system $\{X_n \xleftarrow{\psi} X^0_n\xrightarrow{\psi'} X'_n\}$ of three $F_{p^2}$-curves. Here, $X_n$ is as above, $X'_n$ is its conjugate over $F_p$, and $X^0_n$ is the union of the graphs $\Pi$, $\Pi'$ of two $p$-th power morphisms $X_n \to X'_n$, $X'_n \to X_n$ (both graphs being taken on $X_n \times X'_n$. Actually, $X^0_n$ is partially normalized so that $\Pi$ and $\Pi'$ are \textit{crossing only at the supersingular points} (\S 4).

\item[(c)] The bridge connecting these two systems, \ie, the congruence relation ``$R^0_n \equiv \Pi + \Pi' (\mod p)$''. This is presented in \S 5.1. Due to our canonical choice of the curves over $F_{p^2}$ and their $p$-adic lifting, where the action of $p^\bZ$ is trivialized, the congruence relation is in its \textit{symmetric} form. Also, we use the non-singular curve $R^0_n$ instead of its image in $R_n \times R'_n$, and this corresponds to that $\Pi, \Pi'$ are crossing\pageoriginale only at the supersingular points.\footnote{We learnt this from Igusa \cite{art6-key5}, p. 472 (footnote); cf. also Deligne [i], No. 4} Besider the general formulations of congruence relations by Shimura \cite{art6-key18}, it is also meaningful to draw attention to this canonical and natural form of the congruence relation.

\item[(d)] Passing the bridge, using a celebrated theorem of Grothen-dieck on the unique $p$-adic liftability of \'etale coverings (\S 6).

I am very grateful to Professor P. Deligne who read my first version of the proof and kindly pointed out to me a simplification by an extended use of a theorem of Grothendieck (see \S 6.2).
\end{itemize}

\section{Preliminaries.}\label{art6-sec1}
 We shall review the definitions and some basic facts related to the fields of modular functions over finite fields.

\subsection{}\label{art6-subsec1.1}
Let $p$ be the fixed prime number, and $n$ be a positive integer not divisible by $p$. In \S \ref{art6-subsec1.1}, $F_p$ will denote any field of characteristic either 0 or $p$, satisfying the following condition. Let $W_n$ be the group of $n$-th roots of unity in the fixed separable closure of $F_p$, let $W_\infty$ be the inductive union of $W_n$ for all $n$ (with $p \nmid n$), and consider the cyclotomic extension $F = F_p (W_\infty)$. Then our condition on $F_p$ is that the Galois group $G(F/ F_p)$ contains, and is topologically generated by, a special element $\sigma$ that acts on $W_\infty$ as $\zeta \to \zeta^p$. This being assumed, put $F_{p^m} = F_p (W_{p^m -1})$, for each positive integer $m$. Then $F_{p^m}$ is a cyclic extension of $F_p$ with degree $m$, and it is the fixed field of $\sigma^m$ in $F$. The Krull topology of $G(F/F_p)$ induces, \textit{via} $\sigma^\bZ$, the product of $l$-adic topologies of $\bZ$ over all prime numbers $l$ (including p), and $G(F/F_p)$ is canonically isomorphic to the completion $\hat{\bZ}$ of $\bZ$ by this topology. For each $a \in \hat{\bZ}$, the corresponding element of $G (F/ F_p)$ will be denoted by $\sigma^a$, and the automorphism of $W_\infty$ induced by $\sigma^a$ will be denoted simply as ``$p^a$''. Examples of $F_p$ are the prime field of characteristic $p$, the $p$-adic field $\bQ_p$, the decomposition field of $p$ in the cyclotomic field $\bQ(W_\infty)$ over the rational number field $\bQ$, etc.

Now let $j$ be a variable over $F_p$, and $E$ be any elliptic curve over the rational function field $F_p(j)$, having $j$ as its absolute invariant. The equation of Tate,
\begin{equation} 
Y^2 + XY = X^3 - \frac{36}{j - 1728} X - \frac{1}{j - 1728},
\tag{$(T)_j$}
\end{equation}\pageoriginale 
gives an example of $E$. For each $n$, let $E_n$ be the group of $n$-th division points of $E$, and put $E_\infty = \bigcup\limits_{p\nmid n} E_n$. Then, since $E_n \simeq (\bZ/ n)^2$  the determinant give a homomorphism $\Aut E_n \to (\bZ / n)^\times$. Identifying the two groups $(\bZ/ n)^\times$ and $\Aut W_n$ in the canonical way, we shall consider the determinant as giving a homomorphism $\Aut E_n \to \Aut W_n$, which will be filtrated to the homomorphism at infinity,
\begin{equation}
\det: \Aut E_\infty \to \Aut W_\infty. \label{art6-eq1.1.1}
\end{equation}
Let $\fg$ be the Galois group over $F_p(j)$ of its separable closure. Then $\fg$ acts on $E_\infty$ and $W_\infty$, and the two actions are compatible with the homomorphism \eqref{art6-eq1.1.1}. In particular, by our assumption on $F_p$, the determinant of the action of each element of $\fg$ on $E_\infty$ must belong to $p^{\hat{\bZ}}$.


\setcounter{subprop}{1}
\begin{subprop}[Igusa]\label{1.1.2}
The subgroup of $\Aut E_\infty$ generated by the change of sign $-1_\infty$ ($1_\infty$: the identify map of $E_\infty$) and the actions of all elements of $\fg$ on $E_\infty$ consists of all those elements of $\Aut E_\infty$ whose determinants belong to $p^{\hat{\bZ}}$.
\end{subprop}

\begin{proof}
By Igusa \cite{art6-key5}, this subgroup contains all elements with determinant unity, as $j$ is a variable over $F$. The Proposition follows immediately from this and our assumption on $F_p$. (For characteristic 0, see also Shimura \cite{art6-17}, \cite{art6-18}.)
\end{proof}

\begin{coro*}
For at least one choice of the sign $\pm$, there exists an element of $\fg$ acting on $E_\infty$ as $\pm p \cdot 1_\infty$.
\end{coro*}

\begin{definitions*}
The fields on the left are by definition the fixed fields of the groups on the right. The groups are closed subgroups of $\fg$ defined by their actions on $E_\infty$ and $W_\infty$ (the two actions being connected by the determinant). The symbol ``1'' denotes the identity map of the indicated group.
$$
\xymatrix@R=0.1cm{
& L_\infty \ar@{-}[dd]\\
K_\infty \ar@{-}[ur] \ar@{-}[dd]& \\
& L_n \ar@{-}[dd] \\
K_n \ar@{-}[ur] \ar@{-}[dd]&\\
& L_1 = F(j)\\
K_1 - F_{p^2} (j) \ar@{-}[ur]& 
}
$$


%%%%  166 page

\end{definitions*}


\eqref{art6-eq}
