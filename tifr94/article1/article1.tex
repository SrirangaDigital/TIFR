\title{FOURIER COEFFICIENTS OF EISENSTEIN SERIES ON THE ADELE GROUP}
\markright{FOURIER COEFFICIENTS OF EISENSTEIN SERIES ON THE ADELE GROUP}

\author{By~ WALTER L. BAILY, JR.}
\markboth{WALTER L. BAILY, JR.}{FOURIER COEFFICIENTS OF EISENSTEIN SERIES ON THE ADELE GROUP}

\date{}
\maketitle

\textsc{Much of what}\pageoriginale I wish to present in this lecture will shortly appear elsewhere \cite{art1-key3}, so for the published part of this presentation I shall confine myself to a restatement of certain definitions and results, concluding with a few remarks on an area that seems to hold some interest. As in \cite{art1-key3}, I wish to add here also that many of the actual proofs are to be found in the thesis of L. C. Tsao  \cite{art1-key8}.

Let $G$ be a connected, semi-simple, linear algebraic group defined over $\bQ$, which, for simplicity, we assume to be $\bQ$-simple (by which we mean $G$ has no proper, connected, normal subgroups defined over $\bQ$). We assume $G$ to be simply-connected, which implies in particular that $G_{\bR}$ is connected \citep[Ch. 7, \S 5]{art1-key}. Assume that $G_{\bR}$ has no compact (connected) simple factors and that if $K$ is a maximal compact subgroup of it, then $X=K/G_{\bR}$ has a $G_{\bR}$-invariant complex structure, \iec $X$ is Hermitian symmetric. Then \cite{art1-key6} strong approximation holds for $G$. We assume, finally, that $rk_{\bQ}(G)$ (the common dimension of all maximal, $\bQ$-split tori of $G$) is $> 0$ and that the $\bQ$-relative root system ${}_{\bQ}\sum$ of $G$ is of type $C$ (in the Cartan-Killing classification). Then there exists a totally real algebraic number field $k$ and a connected, almost absolutely simple, simply-connected algebraic group $G'$ defined over $k$ such that $G=\sR_{k/\bQ}G'$; therefore, if $G$ is written as a direct product $\Pi G_i$ of almost absolutely simple factors $G_i$, then each $G_i$ is defined over a totally real algebraic number field, each $G_i$ is simply-connected, each $G_{i\bR}$ is connected and the relative root systems ${}_{\bR}\sum_i={}_{\bR}\sum(G_i)$ are of type $\bC$ \cite{art1-key4}.

Letting $K_i$ denote a maximal compact subgroup of $G_{i\bR}$, the Hermitian symmetric space $X_i =K_i/G_{i\bR}$ is isomorphic to a tube domain since ${}_{\bR}\sum_i$ is of type $C$ \cite{art1-key7}, hence $X=\Pi_i X_i$ is bi-holomorphically equivalent to a tube domain
$$
\fT = \{Z = X + i Y \in  \bC^n | Y \in \fR \},
$$
where\pageoriginale $\fR$ is a certain type of open, convex cone in $\bR^n$. Let $H$ be the group of linear affine transformations of $\fT$ of the form $Z \longmapsto A Z + B$, where $B \in \bR^n$, and $A$ is a linear transformation of $\bR^n$ carrying $\fR$ onto itself, and let $\tilde{H}$ be its complete pre-image in $G_{\bR}$ with respect to the natural homomorphism of $G_{\bR}$ into $\Hol(\fT)$, the group of biholomorphic automorphisms of $\fT$. Then $\tilde{H}=P_{\bR}$, where $P$ is an $\bR$-parabolic subgroup of $G$, and from our assumption that ${}_{\bQ}\sum$ is of type $C$, it follows that we may assume $P$ to be defined over $\bQ$ (the reasons for which are somewhat technical, but may all be found in \cite{art1-key4}).

Assume $G \subset G L (V)$, where $V$ is a finite-dimensional, complex vector space with a $\bQ$-structure. Let $\Lambda$ be a lattice in $V_{\bR}$, \iec a discrete subgroup such that $V_{\bR}/\Lambda$ is compact, and suppose that $\Lambda \subset V_{\bQ}$. Let $\Gamma =\{\gamma \in G_{\bQ} ~|~ \gamma \cdot \Lambda = \Lambda \}$, and for each finite prime $\bp$, let $\Lambda_{\bp} = \Lambda \bigcirc_{\bZ} \bZ_{\bp}, K_{\bp} = \{\gamma \in G_{\bQ_{\bp}} ~|~ \gamma \cdot \Lambda_{\bp} = \Lambda_{\bp}\}$. It may be seen, since strong approximation holds for $G$, that $K_{\bp}$ is the closure $\Gamma_{\bp}$ of $\Gamma$ in $G_{\bQ_{\bp}}$ (in the ordinary $\bp$-adic topology). Now the adele group $G_A$ of $G$ is defined as $\Pi' G_{\bQ_{\bp}}$, where $\Pi'$ denotes restricted direct product with respect to the family $\{K_{\bp}\}$ of compact sub-groups. Define $K_{\infty} = K$, $K^{\ast} ={\displaystyle{\mathop{\Pi}\limits_{\bp \leqslant \infty}}} K_{\bp}$ (Cartesian product).

For all but a finite number of finite $\bp$, we have $G_{\bQ_{\bp}} = K_{\bp} \cdot P_{\bQ_{\bp}}$, and by changing the lattice $\Lambda$ at a finite number of places, we may assume \cite{art1-key5} that $G_{\bQ_{\bp}} = K_{\bp} \cdot P_{\bQ_{\bp}}$ for \textit{all} finite $\bp$. In addition, from the Iwasawa decomposition we have $G_{\bR} = K_{\infty} \cdot P^0_{\bR}$, where $P^0_{\bR}$ denotes the identity component of $P_{\bR}$.

We may write the Lie algebra $\fg_{\bC}$ of $G$ as the direct sum of $\fk_{\bC}$, the complexification of the Lie algebra $\fk$ of $K$, and of two Abelian subalgebras $\fp^+$ and $\fp^-$, both normalized by $\fk$, such that $\fp^+$ may be indentified with $\bC^n \supset \bT$. Let $K_{\bC}$ be the analytic subgroup of $G_{\bC}$ with Lie algebra $\fk_{\bC}$ and let $P^{\pm} = \exp (\fp^{\pm})$; then $K_{\bC} \cdot P^+$ is a parabolic subgroup of $G$ which we may take to be the same as $P$, and $P^+ = U$ is its unipotent radical. Now $\fp^+$ has the structure of a Jordan\pageoriginale algebra over $\bC$, supplied with a homogeneous norm form $\sN$ such that $\Ad K_{\bC}$ is contained in the similarity group
$$
\sS= \{g \in G L (n, \bC) = G L (\fp^+) ~|~ \sN (gX) = v (g) \sN (X) \}
$$
of $\sN$, where $v: \sS \to \bC^{\times}$ is a rational character \cite{art1-key7}, defined over $\bQ$ if we arrange things such that $K_{\bC} = L$ is a $\bQ$-Levi subgroup of $P$. (Note that $K$ and $L_{\bR}$ are, respectively, compact and non-compact real forms of $K_{\bC}$.) Define $v_{\infty}$ as the character on $K_{\bC}$ given by $v_{\infty} (k) = v (\Ad_{\fp^+} k)$. Define $v_{\infty}$ as the character on $K_{\bC}$ given by $v_{\infty}(k)=v(\Ad_{\fp}+k)$. If $\bp \leqslant \infty$ let $|{\;}|_{\bp}$ be the ``standard'' $\bp$-adic norm, so that the product formula holds. We define (for $\bp \leqslant \infty$) $\chi_{\bp}$ on $P_{\bQ_{\bp}}$ by $\chi_{\bp}(ku)=|v(\Ad_{\fp}+k)|_{\bp}$, $k \in L_{\bQ_{\bp}}$, $u \in U_{\bQ_{\br}}$ and $\chi_A$ on $P_A$ by $\chi_A ((p_{\bp}))=\Pi_{\bp} \chi_{\bp} (p_{\bp})$, which is well defined since for $(p_{\bp}) \in P_A$, we have $\chi_{\bp}(p_{\bp}) = 1$ for all but a finite number of $\bp$. Now $v_{\infty}$ is bounded on $K$ and $v$ takes positive real values on $P^{0}_{\bR}$, hence $v_{\infty} (K \cap P^0_{\bR})=\{1\}$. Moreover, $K_{\bp}$ is compact and therefore $\chi_{\bp}(K_{\bp} \cap P_{\bQ_{\bP}}) = \{1\}$. Now let $m$ be any positive integer. Define $P^0_A = \{(p_{\bp}) \in P_A | p_{\infty} = 1\}$, so that $P_A = P_{\bR \cdot P^0_A}$, and put $P^\ast_A = P^0_R P^0_A$. From our previous discussion it is clear that $G_A = K^{\ast} \cdot P^{\ast}_A$. Define the function $\varphi_{m}$ of $G_A$ by
$$
\varphi_m (k^\ast \cdot p_{\ast}) = v_{\infty}(k_{\infty})^{-m} \chi_A (p_{\ast})^{-m},
$$
where $p_{\ast} \in P^\ast_A$, $k^\ast \in K^\ast$, $k^\ast = (k_{\bp})$. It follows from the preceding that $\varphi_{m}$ is well defined.

By the product formula, $\chi_A (p) = 1$ for $p=p_{\bQ}$. Define
$$
\tilde{\sE}_m (g) = \sum\limits_{\gamma \in G_{\bQ}/P_{\bQ}} \varphi_m(g\gamma), \;\; g \in G_A.
$$
By a criterion of Godement, this converges normally on $G_A$ if $m$ is sufficiently large.

Let $g \in G_{\bR}$, $Z \in \fT \subset \fp^+$; we may write \cite{art1-key4}
$$
\exp (Z) \cdot g = p^- \cdot k (Z, g) \exp (Z \cdot g),
$$
where $p^- \in P^-$, $k(Z, g) \in K_{\bC}$, $Z \cdot g \in \fT$ (this is, in fact, the definition of the operation of $G_{\bR}$ on $\fT$ ($\loc. \cit.$)); we then define $v(Z, g)= v_{\infty} (k (Z,g))$. $v (Z, g_1) v (Z \cdot g_1, g_2)$ . Letting $E$ denote the (suitably chosen) identity of the Jordan algebra $\fp^+$, we have $iE \in \fT$, $iE$ is\pageoriginale the unique fixed point of $K$, and it follows from the definitions that if $(iE) \cdot g = Z$, $g \in G_{\bR}$, and if we put
$$
\sE_m (Z) = v (iE, g)^m \tilde{\sE}_m(g),
$$
then $\sE_m$ is a holomorphic function on $\fT$ and we have, in fact, 
\begin{equation}
\sE_m (Z) = \sum\limits_{\gamma \in G_{\bQ}/P_{\bQ}} v (Z, \gamma)^{-m} | v(p_{\gamma}) | v(p_{\gamma})|^{-m}_A,  \label{art1-eq1}
\end{equation}
where $p_{\gamma} \in P^0_A$ is such that $\gamma \in K^\ast \cdot p_{\gamma} \cdot P^0_{\bR}, |\;|_A$ being the adelic norm. Thus $\sE_m$ is a holomorphic automorphic form on $\fT$ with respect to $\Gamma$.

If we let $\Omega = \Gamma \cap U$, then $\sE_m(Z + \theta) = \sE_m (Z)$ for all $\theta \in \Omega$, since $v (Z, \theta)^m = 1$ and $\sE_m$ is an automorphic form with respect to the system of automorphy factors $\{v(Z,\gamma)^m\}$; hence, $\sE_m$ has a Fourier expansion
$$
\sE_m (Z) = \sum\limits_{T \in \Theta'} a_m (T) \sE ((T,Z)),
$$
where $\epsilon (\;) = e^{2 \pi i (\;)}$, $(\;,\;)$ is an inner product on $\fp^+$ such that $(\fp^+_{\bQ}, p^+_{\bQ}) \subset \bQ$, with respect to which the cone $\fk$ is selfdual, and $\Theta'$ is the lattice dual to $\Theta$ with respect to $(\;,\;)$. It follows from the ``regular'' behaviour of $\sE_m$ ``at infinity''
 (Koecher's principle) that $a_m (T) = 0$ unless $T \in \Theta' \cap \fk$. Our main concern is with arithmetic problems related to the Fourier coefficients $a_m(T)$.

One can show that the Fourier coefficients of $\sE_m$ are all rational numbers. To do this, we first apply the Bruhat decomposition to group the terms in the series (1). We have $G_{\bQ} = \bigcup\limits_{0\leqslant j \leqslant r} P_{\bQ} \iota_j P_{\bQ}$, where $r = r k_{\bQ} (G)$ and $\iota_j$ runs over a certain set of double coset representatives of the relative $\bQ$-Weyl group of $G$ with respect to the relative $\bQ$-Weyl group of $L$. In particular, $\iota_r$ represents the total involution of ${}_{\bQ}W(G) = {}_{\bQ} W$, and induction shows that it is sufficient, for proving rationality, to consider the series over the biggest cell $\fC^\ast = P_{\bQ} \iota_r P_{\bQ}$:
\begin{equation}
\sE_m^\ast (Z) = \sum\limits_{\gamma \in \fC^\ast / P_{\bQ}}  v (Z, \gamma)^{-m} | v (p_{\gamma})|^{-m}_A. \label{art1-eq2}
\end{equation}

On the other hand, writing $\iota = \iota_r$, we can see that $\iota$ normalizes $L$, so that $\fC^{\ast}/P_{\bQ} \cong U_{\bQ}\iota$ (as a set of coset representatives). We obtain
\begin{equation}
\sE^\ast_m (Z) = \sum_{\gamma \in U_{\bQ}\iota} v (Z, \gamma)^{-m} |v(p_{\gamma})|^{-m}_A, \label{art1-eq3}
\end{equation}\pageoriginale
where $|\;|_A$ is the adelic norm. If $u \in U_{\bQ_\bp}$, then $u \iota \in G_{\bQ_{\bp}}$, so that we may write $u \iota = k \cdot p$, where $k \in K_{\bp}$, $p \in P_{\bQ_{\bp}}$, and we define $\kappa_{\bp} (u) = |v (p)|^{-1}_{\bp}$. Now $U_k$, for any field $k$, is a vector space over $k$, and in this sense for $u = U_{\bQ_{\bp}}$, $\alpha u$ is defined for any $\alpha \in \bQ_{\bp}$. Crucial for developments is the

\begin{prop*}
If $\mu \in \bZ^{\times}_{\bp}$ (the group of $\bp$-adic units) and $u \in U_{\bQ_{\bp}}$, then $\kappa_{\bp}(\mu u) = \kappa_{\bp} (u)$.
\end{prop*}

This can be proved in most cases, at least, by appealing to classification; however, the smoothest and most general proof  \cite{art1-key8} appears to depend on the results of Bruhat and Tits  \cite{art1-key5,art1-key9}.

One may show, using the Poisson summation formula along the lines of Siegel  \cite{art1-key10}, that each of the Fourier coefficients $a^{\ast}_m (T)$ of the series $E^{\ast}_m (Z)$ has an ``Euler product expansion'':
$$
a^\ast_m (T) = S_{\infty} (T) \cdot {\displaystyle{\mathop{\Pi}\limits_{\bp < \infty}}} S_{\bp} (T).
$$

Here $S_{\infty} (T)$ is a fractional power of the Jordan algebra norm $N (T)$ of $T$ multiplied by a product of ``gamma'' factors independent of $T$, and $S_{\bp}(T)$ is given by the formula
$$
S_{\bp} (T) = \sum\limits_{u \in U_{\bQ_{\bp/\Lambda_{\bp}}}} \epsilon_{\bp} ((T, u)) \kappa_{\bp} (u)^m,
$$
where $(\;,\;)$ is the inner product on the Jordan algebra mentioned previously and $\epsilon_p$ is a character of absolute value one on the additive group $\bQ_{\bp}/\bZ_{\bp}$ whose restriction to $p^{-1} \bZ_{\bp}/\bZ_{\bp}$ is non-trivial. By using the proposition, one may see that each factor $S_{\bp}(T)$ is a rational number for each finite $\bp$; and then by specific calculation of $S_{\bp}(T)$ for all but a finite number of ``bad'' $\bp$ and by using known facts about special values of $\zeta-$ and $L$-functions, one may prove that all $a^{\ast}_m (T)$ are rational numbers. For further details, we refer to \cite{art1-key8,art1-key3}.

We wish to draw attention to the functions $\kappa_{\bp} (u)$. It will be noticed that the definition of $\kappa_{\bp}$ depends on the choice of local lattice $\Lambda_{\bp}$; in other words, the collection of functions $\{\Kappa_{\bp}\}$ depends on the choice of arithmetic group $\Gamma$, and now we want to ask if there is any natural choice for $\Gamma$.

We have $G \subset G L (V)$, $G$ is simply connected, and $G = R_{k/\bQ} G'$, where $k$ is a totally real algebraic number field and $G' \subset G L (V')$ is almost absolutely simple and simply-connected. And we have assumed conditions that imply for each Galois injection $\sigma : k \to \bQ$ (the algebraic closure of $\bQ$) that $({}^{\sigma} G')_{\bR}$ is non-compact, thus strong approximation holds for $G$. Now $\Lambda$ is a lattice in $V_{\bR}$ contained in $V_{\bQ}$ and $\Gamma$ is the stabilizer of $\Lambda$. Let $\Gamma_{\bp}$ be the closure of $\Gamma$ in $G_{\bQ_{\bp}}$ with respect to the $\bp$-adic topology. It can be shown that $\Gamma_{\bp}= K_{\bp}$ is ``special'' maximal compact \cite{art1-key5,art1-key9} for all but a finite number of finite $\bp$ (H. Hijikata has communicated a proof of this to me), and thus $G_{\bQ_{\bp}} = K_{\bp} \cdot P_{\bQ_{bp}}$ at least for all but a finite number of finite $\bp$. Now by ``adjusting'' the lattice $\Lambda$ at a finite number of places, we can assume this to be so for all $\bp$. If $\Gamma_{\bp}$ is special maximal compact in $G_{\bQ_{\bp}}$ for all finite $\bp$. If $\Gamma_{\bp}$ is special maximal compact in $G_{\bQ_{\bp}}$ for all finite $\bp$, then we call $\Gamma_{\bp}$ is special maximal compact in $G_{\bQ_{\bp}}$ for all finite $\bp$, then we call $\Gamma$ a ``special arithmetic subgroup'' of $G_{\bR}$. Let $\Gamma'$ be another special arithmetic subgroup. Then $\Gamma_{\bp}=\Gamma'_{\bp}$ (since $\Lambda_{\bp} = \Lambda'_{\bp}$) for all but a finite number of $\bp$. If $S$ is the finite set of finite $\bp$ for which this is {\it not} true, we know, in {\it some} cases at least, that for each $\bp \in S$ there exits an outer automorphism $\alpha_{\bp}$ of $G_{\bQ_{\bp}}$ such that $\alpha_{\bp} (\Gamma_{\bp}) = \Gamma'_{\bp}$. In virtue of strong approximation, we can then (in such cases), up to an automorphism of the $\bQ_{\bp}$-relative , extended Dynkin diagram for a finite number of $\bp$, pass from the one special arithmetic group to the other by an inner automorphism of $G_{\bQ}$. This makes it reasonable to wonder whether this is a general phenomenon, and if so, what smallest class of automorphisms would suffice to identify all special arithmetic groups if $G$ is absolutely simple. Moreover, it suggests something ``natural'' about the arithmetic structure associated to a special arithmetic group. It would seem inviting to look more closely at such groups to see, for example, if or when they are maximal discrete in $G_{\bR}$ or to see if the normalizers of their images in $\Hol(\fT)$ are maximal discrete in the latter. This might be worthwhile in connection with results of \cite{art1-key1} giving\pageoriginale

%%%% 7 







\vfill\eject

~\phantom{A}
\thispagestyle{empty}


