
\title{STRONG RIGIDITY OF DISCRETE SUBGROUPS AND QUASI-CONFORMAL MAPPINGS OVER A DIVISION ALGEBRA}
\markright{STRONG RIGIDITY OF DISCRETE SUBGROUPS AND QUASI-CONFORMAL MAPPINGS OVER A DIVISION ALGEBRA}

\author{By~ G. D. MOSTOW}
\markboth{ G. D. MOSTOW}{STRONG RIGIDITY OF DISCRETE SUBGROUPS AND QUASI-CONFORMAL MAPPINGS OVER A DIVISION ALGEBRA}

\date{}
\maketitle


%\setcounter{page}{21}
\setcounter{pageoriginal}{202}

\textsc{Let $G$ be}\pageoriginale a semi-simple analytic linear group having no compact normal subgroup other than (1) and let $\Gamma$ be a lattice in $G$, that is, a discrete subgroup such that $G/\Gamma$ has finite Haar measure. The pair $(G, \Gamma)$ is called \textit{strongly rigid} if it is uniquely determined by $\Gamma$; that is, given two such pairs $(G, \Gamma)$  and $(G',\Gamma')$, and given an isomorphism $\theta: \Gamma \to \Gamma'$, then $\theta$ extends to an analytic isomorphism of $G$ to $G'$. In [3d], I announced the strong rigidity of $(G,\Gamma)$ if $G/\Gamma$ is compact and $G$ has no $\bR$-rank 1 factors. In this paper, I will indicate the proof of strong rigidity in the case of $\bR$-rank 1 groups other than $PL (2, \bR)$-rigidity being false for this latter group, as is seen from the example of two analytically inequivalent compact Riemann surfaces $S$ and $S'$ of genus greater than one: $(PL (2, \bR), \Gamma)$ is not rigid if $\Gamma$ is the fundamental group of the surface $S$.

My method of proving strong rigidity evolved from an effort to understand the failure of rigidity in $PL(2, \bR)$ from a geometric viewpoint. If $X$ denotes the simply connected covering space of $S$ and $S'$, then we may regard $X$ as the interior of the unit disc in the complex plane. As differentiable transformation groups on $X$, $\Gamma$ and $\Gamma'$ are equivalent. Why then are they inequivalent in the complex analytic sense? or to rephrase the question, why are they not conjugate in $PL (2,\bR)$? The natural conjecture is: because they are not differentiably equivalent on the \textit{boundary} of $X$.

This fact is quite general. In [3a], I proved:

Let $(G,\Gamma)$ and $(G', \Gamma')$ be two pairs as above. Let $X$ be the symmetric Riemannian space associated to $G$; that is $X = G/ K$ where $K$ is a maximal compact subgroup of $G$. Let $X_0$ denote the unique compact $G$-orbit in a maximal Satake compactification of $G$; or equivalently, the Furstenberg maximal boundary of $X$ (\cf \cite{art7-key1}, \cite{art7-key5}, [3c]). Let $X'$ and $X'_0$\pageoriginale be the corresponding spaces associated with $G'$. Let $\theta: \Gamma \to \Gamma'$ be an isomorphism and let $\varphi: X \to X'$ be a $\Gamma$-space morphism. Then if $\varphi$ extends to a diffeomorphism of $X_0$ to $X'_0$, $\theta$ extends to an analytic isomorphism of $G$ to $G'$.

The question then arose: given two pairs $(G, \Gamma)$ and $(G', \Gamma')$, can one always find a $\Gamma$-space morphism $\varphi: X \to X'$ which extends to a homeomorphism of $X_0$ to $X'_0$? Under the added hypothesis that $G/ \Gamma$ and $G'/ \Gamma'$ are compact, the answer turned out to be affirmative.

Actually, the hypothesis that $G/ \Gamma$ and $G'/\Gamma'$ are compact was necessary for establishing only one lemma, albeit a central lemma, which I shall now describe.

It is known that we lose no generality in assuming that $\Gamma$ is torsion-free. Therefore we assume that $\Gamma$ operates freely on $X$. As is well known, the space $X$ is homemorphic to euclidean space. Thus both $X$ and $X'$ are universal principal $\Gamma$-bundles, and therefore there is a $\Gamma$-space morphism $\varphi: X \to X'$; that is, a continuous (but not necessarily injective) map such that
$$
\varphi (\gamma\; x) =  \theta (\gamma) \varphi (x)
$$
for all $\gamma \in \Gamma$, $x \in X$. Under the hypothesis that $G/\Gamma$ and $G' / \Gamma'$ are compact, one can prove that $\varphi$ can be chosen to be a \textit{pseudo-isometry}.

I call a $\map \varphi : X \to X'$ between two metric spaces a $(k, b)$ pseudoisometry if
\begin{itemize}
\item[(i)] $d_{X'} (\varphi(x), \varphi(y)) \leqslant k d_X (x,y)$,  \hfill all $x, y$  in  $X$ 

\item[(ii)] $k^{-1} d_X (x,y) \leqslant d_{X'} (\varphi (x), \varphi (y))$ \hfill for all  $x, y$ in  $X$
\end{itemize}
 such that $d_X (x,y) \geqslant b$.

A pseudo-isometry is a map which is a $(k, b)$ pseudo-isometry for some $(k,b)$.

In [3d], we have

\begin{theorem*}
Let $G$ and $G'$ be linear analytic groups, let $\Gamma$ and $\Gamma'$ e lattices in $G$ and $G'$ respectively. Let $\theta: \Gamma \to \Gamma'$ be an isomorphism, and let $\varphi: X \to X'$ and $\varphi' : X' \to X$ be $\Gamma$-space pseudo-isometries. Then $\varphi$ extends to a homeomorphism $\varphi: X_0 \to X'_0$.
\end{theorem*}

The proofs\pageoriginale of the announcements in [3d] are in [3e]. The question arises, what about the smoothness of $\varphi_0$? If $G$ has no $\bR$-rank 1 factors, then not only is $\varphi_0$ smooth, but it in fact provides the desired extension of the isomorphism $\theta$ of $\Gamma$ to $\Gamma'$.

\begin{theorem*}
The map $\varphi_0$ induces an isomorphism of the Tits geometry of $G$ onto the Tits geometry of $G'$. Assume $G$ and $G'$ have no compact normal subgroups other than (1). If $G$ has no factors of $\bR$-rank 1, then
$$
\varphi_0 \circ G_{X_0} \circ \varphi^{-1}_0 = G'_{X'_0}
$$
and thus $\theta$ is the restriction to $\Gamma$ of the continuous (and hence analytic) isomorphism
$$
g_{x_0} \to \varphi_0 \circ g_{X_0} \circ \varphi^{-1}_0,
$$
where $g_{X_0}$ denotes the action of $g$ on $X_0$.
\end{theorem*}

Two clarifying comments are in order. First, the kernel of the homomorphism $G \to G_{X_0}$ is the maximum normal compact subgroup of $G$; by hypothesis, it consists only of the identity element. Secondly, the isomorphism between the Tits geometries induced by $\varphi_0$ is continuous, and by the generalized Fundamental Theorem of Projective Geometry for Tits geometries, induces an isomorphism of $G$ to $G'$ provided there are no $\bR$-rank 1 factors. The reason for this exception is that a continuous incidence-preserving bijective map between two projective $n$-planes need not be a projective transformation if $n=1$.

We now take up the case of $\bR$-rank 1 groups $G$. In this case the symmetric space $X$ associated to $G$ is hyperbolic space $H^n_\bK$ where $\bK$ is the field of real numbers $\bR$ or complex numbers $\bC$, the quaternions $\bH$, or the Cayley numbers $\bO$. That is, $\bK$ is a composition division algebra over $\bR$ with a positive definite norm.

In case $\bK = \bR$, $H^n_{\bR}$ is the $n$-dimensional simply connected space of constant negative curvature. As a model for $H^n_{\bR}$, we can take the interior of the unit ball in $\bR^n$ with the $G$-invariant metric 
$$
ds = \frac{|dx|}{1- |x|^2}
$$
where\pageoriginale $|dx|^2 = dx^2_1 + \ldots + dx^2_n$. This metric  $ds$ differs from the euclidean metric by merely a point-function multiplicative factor. Therefore any $\map  \varphi: H^n_{\bR} \to H^n_{\bR}$ which is conformal with respect to $ds$ is conformal with respect to the euclidean metric $|dx|$. More generally, any homeomorphism $\varphi : H^n_{\bR} \to H^n_{\bR}$ which is \textit{quasi-conformal} with respect to the $G$-invariant metric is quasi-conformal with respect to the euclidean metric, quasi-conformal is defined as follows:

A homeomorphism $\varphi : X \to X'$ between two metric spaces is called \textit{$k$-quasi-conformal} if: (1) ${\displaystyle{\mathop{\lim\sup}_{t \to 0}}} \dfrac{L(p,t)}{l(p,t)} < \infty$ at all $p\in X$ where $L(p,t)$ (\resp $l(p,t)$) is the radius of the circumscribed (\resp inscribed) ball with center $\varphi(p)$ of the image $\varphi (B (p,t))$, where $B(p,t)$ is the ball with center $p$ and radius $t$;

(2) ${\displaystyle{\mathop{\lim\sup}_{t \to 0}}} \dfrac{L (p,t)}{l(p,t)} \leqslant t$ \quad for almost all $p$.

A map $\varphi$ is called quasi-conformal if it is $k$-quasi-conformal for some $k$.

In [3b], I proved: If $\varphi$ is a quasi-conformal $\Gamma$-space morphism of $H^n_\bR$ to $H^n_{\bR}$ then the boundary $\map \varphi_0: X_0 \to X_0$ is a m\"obius  map. Thus 
$$
\varphi_0 \circ G_{X_0} \circ \varphi_0^{-1}  = G_{X_0}
$$
in this case too, and $\theta : \Gamma \to \Gamma'$ extends to an analytic automorphism of $G$.

The proof makes essential use of the theory of quasi-conformal mappings for euclidean space.

Now in case $\bK = \bC, \bH$, or $\bO$, we can again represent $H^n_\bK$ as the interior of the unit ball in $\bK^n$, but, as is well known, the $G$-invariant metric is no longer related to the euclidean metric by a multiplicative point-function factor. If $\bK = \bR$, $\bC$ or $\bH$, the metric is
$$
ds^2 = (1 - |x|^2)^{-1} (|dx|^2 + (1-|x|^2)^{-1} | (dx, x)|^2).
$$
One can calculate the hyperbolic distance between two points $x$ and $y$ of $H^n_\bK$:
$$
d(x,y) = \cosh^{-1} \frac{|1 - (x,y)|}{(1 - |x|^2)^{1/2} (1 - |y|^2)^{1/2}}.
$$
If $\bK = \bO$,\pageoriginale then $n \leqslant 2$ and we have
$$
d(x,y) = \cosh^{-1} \frac{(|1-(x,y)|^2 + R (x, y)^{1/2})}{(1-|x|^2)^{1/2} (1-|y|^2)^{1/2}}
$$
where $x = (x_1, x_2)$, $y = (y_1, y_2)$, and
$$
2^{-1} R(x,y)  = \Re (x_1 \bar{x}_2) (y_2 \bar{y}_1) - \Re (\bar{x}_2 y_2) (\bar{y}_1 x_1)
$$
$\{$Note that for any $p, q, r$ in $\bO, \Re p (q \; r) = \Re q(r p)$$\}$.

It is of some interest therefore that one can successfully develop an analogue of quasi-conformal mappings for $\bK$ as well as for $\bR$.

Let $d_0$ denote the positive real valued function on $\bK^n$ defined by the formula
\begin{align*}
4^{-1} d_0 (x,y)^4 & = |1 - (x,y)^2| + R (x,y) - (1-|x|^2)(1-|y|^2)\\
& = |x-y|^2 - (|x|^2 |y|^2 - |(x,y)|^2)  + R (x,y).
\end{align*}
For any $x$, $y$ on the unit sphere, 
$$
d_0 (x,y) \geqslant |x-y|.
$$

We can regard $\bK^n$ as a subset of projective $n$-space over $\bK$, and the $\bK$-lines in $\bK^n$ are subsets of dimension $k = \dim_\bR \bK = 1, 2, 4$, or 8. The intersection of a $\bK$-line with $H^n_{\bK}$ is a geodesic subspace. The intersection of $\bK$-lines with $X_0$ are called $\bK$-\textit{spheres}. By a real 2-plane in $H^n_{\bK}$ we mean a 2-dimensional hyperbolic subspace which lies in no $\bK$-line. The boundary of a real two-plane is called an $\bR$-\textit{circle} (even though it is not always a euclidean circle). If, however, the $\bK$-line (\resp. real two-plane) passes through the origin, then the corresponding $\bK$-sphere (\resp. $\bR$-circle) is a great $\bK$-sphere (\resp. $\bR$-circle).

We have

$d_0 (x,y) = |x-y|$, if $x$, $y$ are on the same great $\bR$-circle,

$d_0 (x,y) = (2 |x-y|)^{1/2}$, if $x,y$ are on the same great $\bK$-sphere.

If we attempted to form a metric from $d_0$, it would yield the usual geodesic are length along great $\bR$-circles, but would yield infinite are length along great $\bK$-spheres. Thus we call $d_0$ a \textit{semi-metric}. We set for any $p\in X_0$
$$
\bK (p, t) = \{q \in X_0 ; \;\; d_0 (p, q) < t\}
$$
and we call $\bK (p,t)$ the $d_0$-ball of center $p$ and radius $t$.

Given $\bR$-rank 1 groups $G$ and $G'$ containing isomorphic lattices $\Gamma$ and $\Gamma'$, let $\varphi: X \to X'$ be a $\Gamma$-space pseudo-isometry as above, and let $\varphi_0 : X_0 \to X'_0$ be the induced boundary homeomorphism.

We set for any $p \in X_0$,
\begin{align*}
L (p,t) & = \inf \{s; \varphi_0 (\bK (p,t)) \subset \bK' (\varphi_0 (p), s)\}\\
l(p,t) & = \sup \{s; \varphi_0 (\bK (p,t)) \supset \bK' (\varphi_0 (p), s)\}
\end{align*}
Then one proves

\begin{theorem*}
There is a positive constant $K$ such that for all $p \in X_0$
$$
{\displaystyle{\mathop{\lim\sup}_{t \to 0}}} \frac{L(p,t)}{l (p,t)} < K.
$$
\end{theorem*}

We can describe this result by saying that the \textit{boundary map} is $K$-\textit{quasi-conformal with respect to the semi-metric} $d_0$.

This theorem has striking consequences. It allows us to prove in the first instance

\begin{theorem*}
The boundary $\map \varphi_0$ is absolutely continuous on almost all $\bR$-circles, if $H^n_\bK \neq H^2_\bR$.
\end{theorem*}

This result in turn allows us to form non-zero directional derivatives of $\varphi_0$ on sets of positive measure along $\bR$-circles.

Upon utilizing the fact that $\Gamma$ operates ergodically at the boundary, we can succeed ultimately in proving that the $\map \varphi_0$ is induced by an isomorphism of $G$. Thus once again we find 
$$
\varphi_0 \circ G_{X_0} \circ \varphi^{-1}_0 = G'_{x_0}.
$$

Putting together our results on $\bR$-rank greater than one hand on $\bR$-rank 1, our method yields the 

\begin{theorem*}
Let $G$ be a semi-simple analytic linear group having no compact normal subgroups other than 1, and let $\Gamma$ be a discrete cocompact subgroup. Assume that there is no continuous homomorphism $\pi : G \to PL (2, \bR)$ with $\pi (\Gamma)$ discrete. Then the pair $(G,\Gamma)$ is rigid (\cf [3e]).
\end{theorem*}

Recent results of Margulis and Raghunathan establish the above theorem if we replace ``cocompact'' by ``non-compact lattice'', and add the hypothesis that $G$ has no $\bR$-rank 1 factors.

As pointed\pageoriginale out above, we use the cocompactness of $\Gamma$ only to construct the pseudo-isometry $\varphi$. However, if sufficient information is available about a fundamental domain for $\Gamma$, one can construct a pseudo-isometry in the non-compact case as well. This was recently verified by Gopal Prasad for the case of lattices in groups with a $\bR$-rank 1 factor.

Thus, putting together the results of Margulis, Raghunathan, Prasad, and [3e] we can weaken the hypothesis of ``cocompact'' in the above theorem to merely ``lattice''.


\begin{thebibliography}{99}
\bibitem{art7-key1} \textsc{H. Furstenberg:} A Poisson formula for semi-simple Lie groups, \textit{Ann. of Math.,} 77 (1963), 335-386.

\bibitem{art7-key2} \textsc{F. I. Mautner:} Geodesic flows on symmetric Riemann spaces, \textit{Ann. of Math.,} 65 (1957), 416-431.

\bibitem{art7-key3} \textsc{G. D. Mostow:} (a) On the conjugacy of subgroups of semi-simple groups, \textit{Proc, of symposia in Pure Math.,} 9 (1966), 413-419.

(b) Quasi-conformal mapppings in $n$-space and the rigidity of hyperbolic space forms, \textit{Publ. I. H. E. S.,} 34, 53-104.

(c) \textit{Lectures on discrete subgroups,} Tata Institute of Fundamental Research. (1970).

(d) The rigidity of locally symmetric spaces, \textit{Proc. International Congress of Math.,} (1970), vol. 2, 187-197.

(e) Strong rigidity of locally symmetric spaces (to appear).

\bibitem{art7-key4} \textsc{G. Prasad:} Strong rigidity of $\bQ$-rank 1 lattices, \textit{Inv. Math.,} 21 (1973), 255-286.

\bibitem{art7-key5} \textsc{I. Satake:} On representations and compactifications of symmetric Riemannian spaces, \textit{Ann. of Math.,} 71 (1960), 77-110.
\end{thebibliography}

