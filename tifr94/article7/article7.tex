
\title{STRONG RIGIDITY OF DISCRETE SUBGROUPS AND QUASI-CONFORMAL MAPPINGS OVER A DIVISION ALGEBRA}
\markright{STRONG RIGIDITY OF DISCRETE SUBGROUPS AND QUASI-CONFORMAL MAPPINGS OVER A DIVISION ALGEBRA}

\author{By~ G. D. MOSTOW}
\markboth{ G. D. MOSTOW}{STRONG RIGIDITY OF DISCRETE SUBGROUPS AND QUASI-CONFORMAL MAPPINGS OVER A DIVISION ALGEBRA}

\date{}
\maketitle


%\setcounter{page}{21}
\setcounter{pageoriginal}{202}

\textsc{Let $G$ be}\pageoriginale a semi-simple analytic linear group having no compact normal subgroup other than (1) and let $\Gamma$ be a lattice in $G$, that is, a discrete subgroup such that $G/\Gamma$ has finite Haar measure. The pair $(G, \Gamma)$ is called \textit{strongly rigid} if it is uniquely determined by $\Gamma$; that is, given two such pairs $(G, \Gamma)$  and $(G',\Gamma')$, and given an isomorphism $\theta: \Gamma \to \Gamma'$, then $\theta$ extends to an analytic isomorphism of $G$ to $G'$. In [3d], I announced the strong rigidity of $(G,\Gamma)$ if $G/\Gamma$ is compact and $G$ has no $\bR$-rank 1 factors. In this paper, I will indicate the proof of strong rigidity in the case of $\bR$-rank 1 groups other than $PL (2, \bR)$-rigidity being false for this latter group, as is seen from the example of two analytically inequivalent compact Riemann surfaces $S$ and $S'$ of genus greater than one: $(PL (2, \bR), \Gamma)$ is not rigid if $\Gamma$ is the fundamental group of the surface $S$.

My method of proving strong rigidity evolved from an effort to understand the failure of rigidity in $PL(2, \bR)$ from a geometric viewpoint. If $X$ denotes the simply connected covering space of $S$ and $S'$, then we may regard $X$ as the interior of the unit disc in the complex plane. As differentiable transformation groups on $X$, $\Gamma$ and $\Gamma'$ are equivalent. Why then are they inequivalent in the complex analytic sense? or to rephrase the question, why are they not conjugate in $PL (2,\bR)$? The natural conjecture is: because they are not differentiably equivalent on the \textit{boundary} of $X$.

This fact is quite general. In [3a], I proved:

Let $(G,\Gamma)$ and $(G', \Gamma')$ be two pairs as above. Let $X$ be the symmetric Riemannian space associated to $G$; that is $X = G/ K$ where $K$ is a maximal compact subgroup of $G$. Let $X_0$ denote the unique compact $G$-orbit in a maximal Satake compactification of $G$; or equivalently, the Furstenberg maximal boundary of $X$ (\cf. \cite{art7-key1}, \cite{art7-key5}, [3c]). Let $X'$ and $X'_0$\pageoriginale



%%%% 204 page





