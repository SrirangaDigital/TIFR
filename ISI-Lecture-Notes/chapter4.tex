\chapter{}\label{chap4}

\section{Projective Modules}\label{c4:s1}

In this section we recall some definitions and simple facts about
projective modules and rephrase Theorem~\ref{c2:thm2.2} of
Chapter~\ref{chap2} in term of projective modules.

\begin{dfn}\label{c4:dfn1.1}
An $A$-module $F$ is said to be \textit{free} if there exists a subset
$S=\{x_{\alpha}\}_{\alpha\varepsilon I},x_{\alpha}\varepsilon F$ such
that 
\begin{enumerate}[(i)]
\item the set $S$ generates $F$ as an $A$-module.
\item the elements of $S$ are linearly independent over $A$.
\end{enumerate}

The set $S=\{x_{\alpha}\}_{\alpha \varepsilon I}$ is called a
\textit{basis} for $F$.
\end{dfn}

\begin{Prop}\label{c4:Prop1.2}
Let $F$ be a free $A$-module with basis $S=\{x_{\alpha}\}_{\alpha
  \varepsilon I}$. Let $M$ be any $A$-module and $f:S\rightarrow M$ be
a map given by $x_{\alpha}\mapsto m_{\alpha},m_{\alpha}\varepsilon
M$. Then there exists a unique $A$-linear map $g:F\rightarrow M$ such
that $goi=f$ where $i:S\hookrightarrow F$ is the canonical inclusion.
\end{Prop}

\begin{Proof}
Define $g:F\rightarrow M$ as $\sum a_{\alpha}x_{\alpha}\mapsto \sum
a_{\alpha}m_{\alpha}$. Then $g$ is an $A$-linear map and
$g\circ i=f$. Further if there exists $g':F\rightarrow M$ such that
$g'\circ i=f$ then $g\mid S = g'\mid S$. Hence $g'\left(\sum
a_{\alpha}x_{\alpha}\right)=\sum a_{\alpha}g'(x_{\alpha})=\sum
a_{\alpha}g(x_{\alpha})=g\left(\sum a_{\alpha}x_{\alpha}\right)$. Therefore
$g=g'$. 

We remark that if $F$ is a free module with basis consisting of $n$
elements then $F\approx A^{n}$ and we say $F$ is free of \textit{rank} $n$. We
note that rank is well defined because if $A^{n}\approx A^{m}$ then
$n=m$ (go modulo a maximal ideal!)
\enprf
\end{Proof}

\begin{dfn}\label{c4:dfn1.3}
An $A$-module $P$ is said to be \textit{projective} if for any
surjective $A$-linear map $f:M\rightarrow M'$ and an $A$-linear map
$g:P\rightarrow M'$ there exists an $A$-linear map
$\tilde{G}:P\rightarrow M$ such that $f\circ \tilde{g}=g$. 
\end{dfn}

\begin{lem}\label{c4:lem1.4}
For an $A$-module $P$, the following are equivelent
\begin{enumerate}[(i)]
\item $P$ is projective $A$-module.
\item Every exact sequence $0\rightarrow N\rightarrow
  M\xrightarrow{f}p\rightarrow 0$ of $A$-modules splits (i.e. there
  exists an $A$-linear map $g:P\rightarrow M$ such that $f\circ
  g=1_p$).
\item P is a direct summand of a free $A$-module.
\end{enumerate}
\end{lem}

\begin{Proof}
(i)$\Rightarrow$ (ii) Let $1_p:P\rightarrow P$ be the identity map. Then
  by definition there exists an $A$-linear map $g:P\rightarrow M$ such
  that $f\circ g=1_p$. 

(ii)$\Rightarrow$ (iii) Let $F$ be a free module on a set of generators
  of $p$. Then we have an exact sequence $F\xrightarrow{f}p\rightarrow
  0$. Let $K=\ker f$. So the exact sequence
$$
0\rightarrow K\rightarrow F\xrightarrow{f} p\rightarrow 0
$$
splits i.e. there exists $g:P\rightarrow F$ such that $f\circ
g=1_p$. Now we note that $g(P)\cap K=0$, for if $x \varepsilon
g(P)\cap k$, then $x=g(p)=k,p\varepsilon P,k\varepsilon K$ so
$f(x)=(fg)(p)=0\Rightarrow p=0$, so $x=0$. Also $F=g(P)+K$, because
write $x=gf(x)+(x-gf(x))$ for $x\varepsilon F$. Therefore
$F=g(P)\oplus K\approx P\oplus K$. 

(ii)$\Rightarrow$ (i) Let $Q$ be an $A$-module such that $P\oplus Q=F$,
where $F$ is a free $A$-module. Let $M\xrightarrow{f}M'\rightarrow 0$
be exact and $g:P\rightarrow M'$ be an $A$-linear map. Let
$\phi:F\rightarrow M'$ be an $A$-linear map given by $\phi\mid P=g$
and $\phi\mid Q=0$. 

Let $\{x_{\alpha}\}_{\alpha \epsilon I}$ be a basis of $F$ and
$\psi:F\rightarrow M$ be a map given by $\psi(x_{\alpha})=m_{\alpha}$
where $f(m_{\alpha})=\phi(x_{\alpha}),\alpha\varepsilon I$. Let
$\tilde{g}=\psi \mid P$. Then clearly $f\circ \tilde{g}=g$. Hence $P$
is projective.
\enprf
\end{Proof}

\begin{rem}\label{c4:rem1.5}
An $A$-module $P$ is finitely generated projective if and only if
there exists a finitely generated $A$-module $Q$ such that $P\oplus
Q\approx A^{n}$. 

We recall that for any commutative ring
$A,\underline{a}=(a_1,\ldots,a_r)\varepsilon A^{r}$ is said to be
completable if $(a_1,\ldots,a_r)$ is the first column of an $r\times
r$ invertible matrix with entries in $A$. 

Let $U_r(A)$ denote the set of unimodular elements in $A^{r}$. We
remark that 

(1.6) Let $\underline{a}\varepsilon A^{r}$. Then
$\underline{a}\varepsilon U_r(A)$ if and only if there exists an
$A$-linear map $f:A^{r}\rightarrow A$ such that $f(\underline{a})=1$. 
\end{rem}

\begin{Proof}
Let $\underline{a}=(a_1,\ldots,a_r)\varepsilon U_r(A)$. Therefore
there exists $b_i\varepsilon A$, $1\leq i \leq r$ such that $\sum
b_ia_i=1$. So define $f:A^{r}\rightarrow A$ as $e_i\mapsto b_i,1\leq
i\leq r$ where $e_i$ are the canonical basis of $A^{r}$. Clearly
$f(\underline{a})=1$. Conversely, let $f(e_i)=b_i$. Then
$f(\underline{a})=\sum b_ia_i=1$, so $\underline{a}\varepsilon
U_r(A)$. 

(1.7) $\underline{a}\varepsilon U_r(A)$ is completable if and only if
$\underline{a}$ can be completed to a base of $A^{r}$. 
\enprf
\end{Proof}

\setcounter{dfn}{7}
\begin{dfn}\label{c4:dfn1.8}
An $A$-module $P$ is said to be \textit{stably free} if there exists
integers $n,m\geq 0$ such that $P\oplus A^{n}\approx A^{m}$
\end{dfn}

\begin{Prop}\label{c4:Prop1.9}
For any commutative ring $A$, the following are equivalent
\begin{enumerate}[(i)]
\item Every $\underline{a}\varepsilon U_r(A)$ is completable, $r\geq
  1$
\item Every stably free module over $A$ is free.
\end{enumerate}
\end{Prop}

\begin{Proof}
(i)$\Rightarrow$ (ii) By induction enough to show that $P\oplus
  A\approx A^{n+1}$ implies $P$ is free. Let $\phi$:$P\oplus
  A\xrightarrow{\sim} A^{n+1}$ and
  $\phi(0,1)=\underline{a}\varepsilon A^{n+1}$. Identifying through
  $\phi$, we can write $P \oplus A \underline{a}=A^{n+1}$. Let
  $f:A^{n+1}\rightarrow A $ be an $A$-linear map given by $f\mid p=0$
  and $f(\underline{a})=1$. Then by (1.6), $\underline{a}\varepsilon
  U_{n+1}(A)$. By hypothesis there exists unimodular elements
  $\underline{a}_2,\ldots,\underline{a}_{n+1} \varepsilon U_{n+1}(A)$
  such that $\underline{a},\underline{a}_2,\ldots,\underline{a}_{n+1}$
  form a basis for $A^{n+1}$. Hence $P\approx
  \dfrac{A^{n+1}}{A\underline{a}\approx
    \sum\limits_{i=2}^{n+1}A\underline{a}_i\approx A^{n}}$.

(ii)$\Rightarrow$(i) Let $\underline{a}\varepsilon U_r(A)$. By(1.6),
  there exists an $A$-linear map $f:A^{r}\rightarrow A'$ such that
  $f(\underline{a})=1$. Let $P=\ker f$. Since the exact sequence
$$
0\rightarrow P\rightarrow A^{r}\xrightarrow{f} A\rightarrow 0
$$
splits, we have $P\oplus A\approx A^{r}$. By (ii), $P$ is free, and in
fact $P\approx A^{r-1}$. Let
$(\underline{a}_2,\ldots,\underline{a}_r)$ be a basis for $P$. Then
$(\underline{a}, \underline{a}_2,\ldots,\underline{a}_r)$ is a basis
for $A^{r}$. So $\underline{a}$ is completable.
\enprf
\end{Proof}

\begin{thm}\label{c4:thm1.10}
Let $k$ be any field and $A=k[X_1,\ldots,X_n]$. Then any stably free
$A$-module is free.
\end{thm}

\begin{Proof}
Follows from Proposition~\ref{c4:Prop1.9} and Theorem~\ref{c2:thm2.2}
of Chapter~\ref{chap2}.

We remark that a stably free module need not be free.
\enprf
\end{Proof}

\begin{exple}\label{c4:exple1.11}
Let $A$ be the co-ordinate ring of the real sphere $S^{2}$
i.e. $A=\mathbb{R}[x,y,z]$ with relations $x^{2}+y^{2}+z^{2}=1$. Then
$\underline{a}=(x,y,z)\varepsilon A^{3}$ is unimodular. But it is
known that $\underline{a}$ is not completable.
\end{exple}

\section{Serre's Problem}\label{c4:s2}

In this section we prove a theorem of Quillen and Suslin:

Let $k$ be any field. Then any finitely generated projective module
over $k[X_1,\ldots,X_n]$ is free.

This we prove by using Towber presentation.


\begin{lem}\label{c4:lem2.1}
Let $A$ be an integral domain and $P$ a finitely generated projective
$A$-module. Then there is a free $A$-module $F$ of finite rank such
that $aF\subset P\subset F$, for some $0\neq a\varepsilon A$. 
\end{lem}

\begin{Proof}
Let $x_1,\ldots,x_m\varepsilon P$ generate $P$. Without loss of
generality we may assume that $x_1,\ldots, x_r$ is a maximal set of
linearly independent elements among $x_1,\ldots,x_m$. Then
$F=\sum\limits_{i=1}^{r} Ax_i$ is free. Clearly by maximality of
$x_,\ldots,x_r$ there exists an $a\varepsilon A$, $a\neq 0$ such that
$aP\subset F$. So $aF\subset aP\subset F$. But since $P\approx aP$,
writing $P$ for $aP$, the lemma follows. 
\enprf
\end{Proof}

\begin{Prop}\label{c4:Prop2.2}
(Towber presentation). Let $A$ be a ring such that every finitely
generated projective $A$-module is free. Let $B=A[X]$ and $P$ be a
finitely generated projective $B$-module and let $F=B^{r}$,
$h\varepsilon A[X]$ be a unitary polynomial such that 
$$
hF\subset P \subset F.
$$

Then there exists an exact sequence
$$
0\rightarrow B^{m}\rightarrow B^{m+r}\rightarrow P\rightarrow 0.
$$

In particular $P$ is stably free.
\end{Prop}

\begin{Proof}
Let $f_1,\ldots,f_r$ be the standard basis for $B^{r}$ and let $F_k$
denote the $A$-module
$$
\left\{\sum\limits_{i=1}^{r}h_if_i\mid h_i\varepsilon A[X],\deg
h_i\leq k\right\}
$$

Let $\deg h=n$. Then since $h$ is unitary, we have 
$$
F=F_{n-1}+hF\quad \text{ (use division algorithm) }
$$
and
$$
P=P_1+hF, P_1=P\cap F_{n-1}.
$$

It is clear that the above is a direct sum decomposition as
$A$-modules. Since $P$ is a projective $A$-module, it follows that
$P_1$ is a projective $A$-module. Further since $P_1\approx
\dfrac{P}{hF}$, $\dfrac{P}{hF}$ is a finitely generated
$\dfrac{A[X]}{(h)}$-module and $\dfrac{A[X]}{(h)}$ is a finitely
generated $A$-module, it follows that $P_1$ is a finitely generated
$A$-module. Hence by hypothesis $P_1$ is a free $A$-module. Let
$e_1,\ldots,e_m$ be a basis for $P_1$. Then $P$ as an $A$-module is
generated by $e_1,\ldots,e_m$ and $hf_1,\ldots,hf_r$. In fact
$$
p=\sum\limits_{i=1}^{m}Ae_i+h\sum\limits_{i=1}^{r}Bf_i
$$

Further since $P_1\subset F_{n-1}$, we have
$$
XP_1\subset F_n=F_{n-1}+hF_o,\left(F_o=\sum\limits_{i=1}^{r}Af_i\right)
$$
and since $XP_1\subset P$ and $hF_o\subset P$, we have 
$$
XP_1\subset P_1+hF_o
$$

In particular, Let
$$
Xe_i=-\sum\limits_{j=1}^{m}a_{ji}e_j-h\sum\limits_{j=1}^{r}b_{ji}f_j\quad
(1\leq i\leq m)
$$
where $a_{ji}$, $b_{ji}\varepsilon A$, and let $\phi :
B^{m+r}\rightarrow p$ a map given by $\phi(u_i)=e_i$ and
$\phi(v_j)=hf_j$, where $u_1,\ldots,u_m$, $v_1,\ldots, v_r$ is a base
for $B^{m+r}$. Clearly
$$
\alpha_i=\sum\limits_{j=1}^{m}(a_{ji}+\delta_{ij}X)u_j+\sum\limits_{j=1}^{r}
b_{ji}v_j\varepsilon \ker \phi
$$
for $1\leq i\leq m$. Since $\det(a_{ji}+\delta_{ij}X)$ is a monic
polynomial, the elements $\alpha_1,\ldots,\alpha_m$ are linearly
independent over $B$. Now we claim that $\ker \phi$ is generated by
$\alpha_1,\ldots,\alpha_m$. Since 
$$
\begin{aligned}
&Xu_i\varepsilon
A\alpha_i+\sum\limits_{j=1}^{m}Au_j+\sum\limits_{k=1}^{r}Av_k,\\
&{} m=\sum\limits_{i=1}^{m} B\alpha_i+\sum\limits_{j=1}^{m}Au_j+\sum\limits_{k=1}^{r}Bv_k
\end{aligned}
$$
is a $B$-module, hence $M=B^{r+m}$. So 
$$
\ker \phi=\sum\limits_{i=1}^{m}B\alpha_i+(N\cap \ker \phi)
$$
where $N=\sum\limits_{j=1}^{m}Au_j+\sum\limits_{k=1}^{r}Bv_k$. Now
suppose $\sum\limits_{j=1}^{m}a_je_j+\sum\limits_{k=1}^{r}h h_kf_k=0$
with $a_j\varepsilon A$. Since $P_1\cap hF=0$, we have
$\sum\limits_{j=1}^{m}a_je_{j}=\sum\limits_{k=1}^{r} h h_kf_k=0$. But
$e_j's$ and $f_k's$ are bases for $p_1$ and $B^{r}$ respectively and
$h$ is a unitary polynomial, so it follows that $a_j=0$ and $h_k=0$
for $1\leq j\leq m$, $1\leq k\leq r$.This shows that $N\subset \ker
\phi=0$. Hence we get an exact sequence
$$
0\rightarrow B^{m}\xrightarrow{\psi}B^{m+r}\xrightarrow{\phi}
P\rightarrow 0
$$
where $\psi$ is given by the $(m+r)\times m$  matrix
$$
\begin{bmatrix}
S & + & XI_m\\
& T &
\end{bmatrix}
$$
where $S=(a_{ij})$, $a_{ij}\varepsilon A, 1\leq i,j\leq m$ and
$T=(b_{ij})$, $b_{ij}\varepsilon A$, $1\leq i\leq r, 1\leq j\leq m$.
\enprf
\end{Proof}


\begin{thm}\label{c4:thm2.3}
Every finitely generated projective module over\\ $k[X_1,\ldots,X_n]$ is
free $k$ a field.
\end{thm}

\begin{Proof}
We prove this by induction on $n$.

Let $P$ be a finitely generated projective module over
$B=\\k[X_1,\ldots,X_n]$. Now by Lemma~\ref{c4:lem2.1}, we may assume
that there exists a free module $F$ of finite rank such that 
$$
hF\subset P\subset F
$$
for some $h\neq 0$, $h\varepsilon B$. By Lemma~\ref{c2:lem2.6} of
Chapter~\ref{chap2}, we may assume that $h$ is a unitary polynomial in
$X_n$ with coefficients in $A=k[X_1,\ldots,X_{n-1}]$. Now by the
induction hypothesis and Lemma~\ref{c3:lem2.2} $P$ is stably
free. Hence $P$ is free by Theorem~\ref{c4:thm1.10}.
\enprf
\end{Proof}
