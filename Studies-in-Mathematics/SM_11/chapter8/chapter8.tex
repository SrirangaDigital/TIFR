\title{Some Metrics on Picard Bundles}\label{chap8}
\markright{Some Metrics on Picard Bundles}

\author{By G.R. Kempf}
\markboth{G.R. Kempf}{Some Metrics on Picard Bundles}

\date{}
\maketitle

\setcounter{page}{165}

\footnotetext[1]{Partly supported by NSF grant \#MPS 75-05578.}
\setcounter{pageoriginal}{216}
THE\pageoriginale PICARD BUNDLES on the Jacobian $J$ of a smooth
complete curve $C$ of genus $g$ have been studied by various
methods \cite{chap8-key1}, \cite{chap8-key2},  but their nature still
remains mysterious. In this paper, I will present some naturally
occurring Hermitian metrics on certain Picard bundles in the complex
case. These metrics are special among the myriad of such
metrics. Hopefully, they will be useful in the application of the
methods of metric geometry to the study of Picard bundles.

These metrics are induced by metrics on the analog of Picard bundles
on the Jacobian. The rich geometry of abelian varieties provides
metrics on these analogs. The connection between these analogs and the
Picard bundles themselves has been developed by
R.C. Gunning \cite{chap8-key1}. The first two sections of this paper
are algebraic and apply to the geometry over an arbitrary field
$k$. In them, I will develop the abstract version of Gunning's
connection. In the third section, we will be working over the complex
field to define the matrics.

\section{Some linear systems on \texorpdfstring{$C$}{C} and \texorpdfstring{$J$}{J}}\label{chap8-sec1}

Let $V$ denote the set of effective divisors $D$ on $C$ of degree $g$
such that the complete linear system $|D|$ consists of one point. We
may regard $V$ as an open dense subset of the $g$-th symmetric product
$C^{(g)}$. Let $\int_{g}:C^{(g)}\to \Pic_{g}$ be the universal abelian
integral onto the $g$-th Picard variety of $C$, which classifies
isomorphism classes of invertible sheaves on $C$ of degree $g$. Then
$\int_{g}$ induces an isomorphism between $V$ and an open dense subset
$U$ of $\Pic_{g}$. 

Let $E$ be a divisor on $C$ of degree $ng$ for some integer $n\geq
2$. Then we have 

\begin{lemma*}
The\pageoriginale divisors $D_{1}+\cdots+D_{n}$ where the $D_{i}$'s in
$V$ are dense in the complete linear system $|E|$.
\end{lemma*}

\begin{proof}
Let $F$ be a member of $|E|$. We may write $F=G_{1}+\cdots+G_{n}$
where the $G_{i}$'s are effective divisors of degree $g$. Let
$\mathscr{L}_{1},\ldots,\mathscr{L}_{n}$ denote invertible sheaves on
$C$ of degree zero such that $\bigotimes\limits_{1\leq i\leq
n}\mathscr{L}_{i}\approx \mathscr{O}_{C}$. As $n\geq 2$ for a general
choice of the $\mathscr{L}_{i}$'s, each sheaf $\mathscr{L}_{i}(G_{i})$
is contained in $U$. In this case
$\mathscr{L}_{i}(G_{i})\approx \mathscr{O}_{C}(D_{i})$ where $D_{i}$
is in $V$. Hence $D_{i}+\cdots+D_{n}\in |E|$ and $F$ is contained in
the closure of such divisors as $V$ is dense in $C(g)$.\hfill {\em Q.E.D.}
\end{proof}

Recall that the Jacobian $J$ possesses a principal polarization; i.e.,
there is a divisor $\theta$ on $J$ which is determined upto
translation with\break $\dim \Gamma(J,\mathscr{O}_{J}(\theta))=1$. For any
integer $n>0$, we have the multiple polarization. Any divisor in this
class is linearly equivalent to $n\theta$ for some $\theta$. Choose an
element $\alpha$ of $\Pic_{1}$. Then we have the embedding
$i:C\hookrightarrow J$ given by $i(C)=\int (C)-\alpha$. With this
embedding, we can study the relation between the linear systems on $J$
and those on $C$.

\begin{prop*}
If $n\geq 2$, the linear system $|n\theta|$ cuts out a complete linear
system on $C$ of degree $ng$. Equivalently, the restriction
$\Gamma(J,\mathscr{O}_{j},(n\theta))\to \Gamma(C,\mathscr{O}_{J}(n\theta)|_{C})$
is surjective.
\end{prop*}

\begin{proof}
First recall \cite[for instance]{chap8-key4} that the linear
equivalence class $i^{\ast}\theta$ is of degree $g$. Furthermore, if
the inverse image $i^{1}(\theta+j)$ is defined for a point $j$ of $J$,
it is a divisor in $V$ and all divisors in $V$ occur this way. Let $E$
denote a member of the class $i^{*}(n\theta)$. Take a divisor
$D_{1}+\cdots+D_{n}$ in $|E|$ with the $D_{i}$'s in $V$. Thus we may
find points $j_{i}$ of $J$ such that $i^{-1}(\theta+j_{i})$ is defined
and equals $D_{i}$. Therefore,
$$
D_{1}+\cdots+D_{n}=i^{-1}((\theta+j_{1})+\cdots+(\theta+j_{n})).
$$
As $(\theta+j_{1})+\cdots+(\theta+j_{n})$ is algebraically equivalent
to $n\theta$ and its intersection with $C$ is linearly equivalent to
$E$, $(\theta+j_{i})+\cdots+\cdots+(\theta+j_{n})$ is contained in
$|n\theta|$ by the autoduality\pageoriginale of the Jacobian. As
$|n\theta|$ cuts out a linear system, this linear system is complete
by the Lemma.\hfill Q.E.D.
\end{proof}


\section{The connection between the Picard sheaves}\label{chap8-sec2}

Let $\mathscr{L}$ be an invertible sheaf on $C\times J$ which is a
universal family of invertible sheaves on $C$ of degree $d$. If
$d>2g-2$, the direct image
$\mathscr{W}_{d}\equiv \pi_{J_{\ast}}\mathscr{L}$ is a locally free
sheaf on $J$ of rank $d+1-g$. This sheaf $\mathscr{W}_{d}$ is called a
Picard bundle of degree $d$. Furthermore, the formation of the direct
image commutes with base extension. In particular, for any point $j$
of $J$, we have a natural isomorphism 
$$
\mathscr{W}_{d}\otimes
k(j)\xrightarrow{\approx}\Gamma(C,\mathscr{L}|_{C\times \{j\}}).
$$
Thus $\mathscr{W}_{d}$ arises naturally when one is studying
variational problems involving complete linear systems of degree $d$
on $C$. Let $X$ be an abelian variety with dual
$X\sphat$. Similarly, let $\mathscr{M}$ be an invertible sheaf on
$X\times X$ which gives a universal family of invertible sheaves on
$X$ which are algebraically equivalent to
$\mathscr{N}\equiv\mathscr{M}|_{X\times 0}$. We will assume that
$\mathscr{M}$ is isomorphic to
$\pi_{X}^{\ast}\mathscr{N}\otimes \mathscr{P}$ where $\mathscr{P}$ is
a Poincar\'e sheaf on $X\times X\sphat$. If $\mathscr{N}$ is ample on
$X$, the direct image
$\mathscr{V}\{\mathscr{N}\equiv \pi_{X\sphat}\ast \mathscr{M}$ is a
locally free sheaf on $X\sphat$ of rank $\dim \Gamma
(J,\mathscr{N})$ \cite{chap8-key3}. Furthermore, the formation of this
direct image also commutes with base extension. In particular, for
each point $y$ of $x$, we have a natural isomorphism
$$
\mathscr{V}\{\mathscr{N}\}\otimes
k(y)\to \Gamma(X,\mathscr{M}|_{X\times y})
$$ 

In the special case $X=J$, we have $X\sphat \cong J$ as $J$ is
principally polarized. If
$\mathscr{N}\approx \mathscr{O}_{J}(n\theta)$ for some integer $n>0$,
the sheaf $V\{\mathscr{N}\}$ will be denoted by $\mathscr{V}_{n}$. As
$\dim \Gamma(J,\mathscr{O}_{J}(n\theta))=n^{g}$, $\mathscr{V}_{n}$ is
$a$ locally free sheaf on $J$ of rank $n^{g}$. This sheaf contains
variational information about the various linear systems $|n\theta|$. 

Now consider the embedding $C\hookrightarrow J$, the sheaf
$C_{J}(n\theta)\otimes \mathscr{P}|_{C\times J}$ is a universal family
of invertible sheaves on $C$ of degree $ng$ paramerized by $J$ because
of the autoduality of $J$. Furthermore, we have the natural
restriction of direct images.
\begin{align*}
& \alpha:\mathscr{V}_{n}=\pi_{2^{\ast}}(\pi^{*}_{1}\mathscr{O}_{J}(n\theta)\otimes \mathscr{P})\to\\
& \to \pi_{J^{\ast}}(\pi^{*}_{1}\mathscr{O}_{J}(n\theta)\otimes \mathscr{P}|_{C\times J})=\mathscr{W}_{ng}.
\end{align*}\pageoriginale
The main connection between these two kinds of Picard sheaves is the following.

\setcounter{proposition}{1}
\begin{proposition}\label{chap8-prop2}
If $n\geq 2$, the above homomorphism $\alpha:\mathscr{V}_{n}\to \mathscr{W}_{ng}$ is surjective. 
\end{proposition}

\begin{proof}
It will suffice to show $\alpha\otimes k(j)$ is surjective for each
point $j$ of $J$. As the formation of both sheaves commutes with base
extension, this is equivalent to the surjectivity of
$\Gamma(J,\mathscr{H})\to \Gamma(C,\mathscr{H}|_{C})$ where
$\mathscr{H}\approx \mathscr{O}_{J}(n\theta)$ for some choice of
$\theta$. Thus this proposition follows from the last one.\hfill
Q.E.D. 
\end{proof}

The sheaf $\mathscr{V}\{\mathscr{M}\}$ has a clear description which
we will use in the next section. The last proposition may be used to
give a description of the Picard sheaf $\mathscr{W}_{4}$ when $C$ is a
curve of genus 2. In this case $\mathscr{W}_{4}$ has rank 8 and
$\mathscr{V}_{2}$ has rank 4.

\begin{claim*}
\begin{itemize}
\item[(a)] We have an exact sequence
$$
0\to \mathscr{O}_{J}(-\theta)\xrightarrow{\beta}\mathscr{V}_{2}\xrightarrow{\alpha}\mathscr{W}_{4}\to 0.
$$

\item[(b)] Any homomorphism $\mathscr{O}_{J}(-\theta)$ to
$\mathscr{V}_{2}$ is a multiple of $\beta$.
\end{itemize}
\end{claim*}

\begin{proof}
As $g=2$, we can assume that $C=\theta$.

On $J\times J$ we have an exact sequence
$$
0\to \pi^{*}_{1}\mathscr{O}_{J}(\theta)\otimes \mathscr{P}\to \pi^{*}_{1}\mathscr{O}_{J}(2\theta)\otimes \mathscr{P}\to \pi^{*}_{1}\mathscr{O}_{J}(2\theta)\otimes \mathscr{P}|_{C}\to 0.
$$
The sequence in (a) is just the direct image sequence. To see the
claim (a) it will suffice to compute
$\pi_{2*|}(\pi^{*}_{1}\mathscr{O}_{J}(\theta)\otimes \mathscr{P})$. Here
we may take
$$
\mathscr{P}=(\pi_{1}+\pi_{2})^{*}\mathscr{O}_{J}(\theta)\otimes \pi^{*}_{1}\mathscr{O}_{J}(-\theta)\otimes \pi^{*}_{2}\mathscr{O}_{J}(-\theta). 
$$
Thus we need to compute
$\pi_{2*}((\pi_{1}+\pi_{2})^{*}\mathscr{O}_{J}(\theta)\otimes \pi^{*}_{2}\mathscr{O}_{J}(-\theta))$
which equals
$\pi_{2*}(\pi_{1}+\pi_{2})^{*}\mathscr{O}_{J}(\theta))\otimes \mathscr{O}_{J}(-\theta)$
by the projection formula. Lastly, we need to show that
$\pi_{2}^{*}((\pi_{1}+\pi_{2})^{*}\mathscr{O}_{J}(\theta))$ is
trivial.\pageoriginale To see this, note that we have the $\pi_{2}$ -
isomorphism $(\pi_{1}+\pi_{2},\pi_{2})$ between $\pi_{1}+\pi_{2}$ and
$\pi_{1}$. Hence
$$
\pi_{2*}((\pi_{1}+\pi_{2})^{*}\mathscr{O}_{J}(\theta))\approx \pi_{2*}(\pi^{*}_{1}\mathscr{O}_{J}(\theta))=\mathscr{O}_{J}\otimes_{k}\Gamma(J,\mathscr{O}_{J}(\theta))
$$

The desired result follows because
$\Gamma(J,\mathscr{O}_{J}(\theta))=k$. 

For (b), note that $\Hom(\mathscr{O}_{J}(-\theta),\mathscr{V}_{2})=$
\begin{align*}
& \Gamma(J,\Hom\mathscr{O}_{J}(\mathscr{O}_{J}(-\theta),\pi_{2*}\mathscr{O}_{J}(2\theta)\otimes \mathscr{P})))\\
= & \Gamma(J\times
J,\pi^{*}_{2}\mathscr{O}_{J}(\theta)\otimes \pi^{*}_{1}\mathscr{O}_{J}(2\theta)\otimes \mathscr{P}).
\end{align*}
Using our formula for $\mathscr{P}$, we need to show that the space of
global sections of
$(\pi_{1}+\pi_{2})^{*}\mathscr{O}_{J}((\theta)\otimes \pi^{*}_{1}\mathscr{O}_{J}(\theta)$
is one dimensional. Using the isomorphism $(\pi_{1},\pi_{1}+\pi_{2})$
of $J\times J$ the last space is isomorphic to
$$
\Gamma(J\times
J,\pi^{*}_{1}\mathscr{O}_{J}(\theta)\otimes \pi^{*}_{2}\mathscr{O}_{J}(\theta))=\Gamma(J,\mathscr{O}_{J}(\theta))=\otimes
(J\Gamma(J,\mathscr{O}_{J}(\theta))=k 
$$
by the Kunn\"eth formula. This proves (b).
\end{proof}

\section{The natural Hermitian metrics on the Picard
bundles}\label{chap8-sec3}

We return to the situation of the last section where $\mathscr{N}$ is
an ample invertible sheaf on an abelian variety $X$ with dual
$X\sphat$. Recalling from \cite{chap8-key5} that we have an isogeny
$\phi_{\mathscr{N}}:X\to X$ which sends a point $x$ in $X$ to the
isomorphism class $T_{x}\mathscr{N}\otimes \mathscr{N}^{\otimes -1}$
where $T_{x}$ denotes translation by $x$. The kernel of
$\phi_{\mathscr{N}}$ is the finite group scheme $H$. Furthermore, we
have Mumford's theta group $G$ which is given by a central extension
$$
1\hookrightarrow \mathbb{G}_{m}\to G\to H\to 0.
$$
An element of $G$ is a specific isomorphism
$\alpha:T_{k}\mathscr{N}\cong \mathscr{N}$ for some point $k$ of
$H$. In fact, if $G$ acts on $X$ via translation by $H$, we have a
$G$-linearization of the sheaf $\mathscr{N}$ on $X$ where the centre
$\mathbb{G}_{m}$ of $G$ acts by multipliacation. Consequently,
$\Gamma(X,\mathscr{N})$ is a representation of $G$ where $G_{m}$ acts
by multiplication. This representation is the unique irreducible
representation of $G$ with this condition on $G_{m}$ and order
$(H)=[\alpha\Gamma,??? \Gamma(X,\mathscr{N})]^{2}$. 

Consider\pageoriginale the sheaf
$\Gamma(X,\mathscr{N})\otimes_{k}\mathscr{N}^{\otimes-1}$ on $J$. This
sheaf possesses a natural action of $H$. Just let $G$ act naturally on
$\Gamma(X\mathscr{N})$ and contragradiently on $\mathscr{N}^{\otimes
-1}$; then the tensor product has an induced $G$-linearization given
by the tensor product of the two actions. As the center of $G$ acts
trivially on the tensor product, we have an action of $H$ on the
sheaf. A central result about Picard sheaves on abelian varieties is

\begin{proposition}[See \cite{chap8-key3}]\label{chap8-prop3}
We have an $H$-isomorphism
$$
\phi^{*}_{\mathscr{N}}(\mathscr{V}\{\mathscr{N}\})\xrightarrow{\approx}F(X\mathscr{N})\otimes_{k}\mathscr{N}^{\otimes -1}
$$
where the $H$-action on
$\phi^{*}\mathscr{N}(\mathscr{V}\{\mathscr{N}\})$ is the tautological
one which determines $\mathscr{V}\{\mathscr{N}\}$ by descent theory.
\end{proposition}

From now on, we will assume that the ground field $k$ is the complex
numbers $\mathbb{C}$. In this case, all the group schemes are
reduced. The invertible sheaf $\mathscr{N}$ (or any invertible sheaf,
for that matter) on our abelian variety $X$ possesses an almost
canonical Hermitian metric by a result of
Mumford's \cite{chap8-key6}. This metric is determined upto a positive
real multiple by the condition that its curvature is an invariant
differential form on the abelian group $X$. The group $G$ possesses a
maximal compact subgroup $K$ which is an extension of $H$ by $\{c\in
G|~|c|=1\}$. In fact, $K$ consists of elements of $G$ where the
isomorphism $\alpha$ is an isometry. Also this metric globally induces
a Hermitian inner product on the vector space $\Gamma(J\mathscr{N})$
which is invariant under $K$. Here one uses an invariant normalized
Haar measure $\mu$ on $X$ and defines
$$
\langle \alpha,\beta\rangle
=\int_{X}\langle \alpha_{X},\beta_{X}\rangle_{X}\mu. 
$$

Returning to our sheaf
$\Gamma(J\mathscr{N})\otimes_{\mathbb{C}}\mathscr{N}^{\otimes -1}$ we
may give this sheaf a metric by taking the tensor product of the one
on $\Gamma(J,\mathscr{N})$ with the dual metric on $\mathscr{N}^{\otimes-1}$
for a given choice of a Mumford\pageoriginale type metric on
$\mathscr{N}$. The metric is immediately seen to be independent of the
last choice and to be invariant under the action of $H$ on
$\Gamma(J,\mathscr{N})\otimes_{\mathbb{C}}\mathscr{N}^{\otimes
-1}$. Using Proposition \ref{chap8-prop3} and descent theory, we may
descent this metric to get one on the quotient
$\mathscr{V}\{\mathscr{N}\}$. Thus we have

\begin{proposition}\label{chap8-prop4}
The Picard sheaf $\mathscr{V}\{\mathscr{N}\}$ possesses a canonical
Hermitian metric which pulls back via $\phi_{\mathscr{N}}$ to the one
given above.
\end{proposition}

Returning to the curve case, we have

\begin{coro*}
If $n\geq 2$, the Picard sheaf $\mathscr{W}_{ng}$ possesses a
canonical Hermitian metric.
\end{coro*}

\begin{proof}
By Proposition \ref{chap8-prop2}, we may use the restriction
$\mathscr{N}_{n}\to\to \mathscr{W}_{ng}$ to give a quotient metric
from the canonical metric on $\mathscr{V}_{n}$.\hfill Q.E.D.
\end{proof}

\begin{remark*}
One may check that the Chern forms (elementary invariants of the
curvature) of the Hermitian sheaf $\mathscr{V}\{\mathscr{N}\}$ are
invariant differential forms. This follows immediately after pulling
back via $\phi_{\mathscr{N}}$. One may ask for a computation of the
Chern forms of the Picard shaves $\mathscr{W}_{ng}$. Their cohomology
class is well-known. 
\end{remark*}

\begin{thebibliography}{}
\bibitem{chap8-key1} R.C.Gunning, On generalized theta functions, {\em Amer. J. Math.,} 104 (1982) pp.~183-208.

\bibitem{chap8-key2} G.~Kempf, Inversion of Abelian integrals, {\em Bulletin of A.M.S.,} 9(1982) pp.~25-32.

\bibitem{chap8-key3} G.~Kempf, Toward the inversion of Abelian integrals II, {\em Amer. J. Math.,} 101(1979) pp.~184-202.

\bibitem{chap8-key4} G.~Kempf,\pageoriginale Inverse images of theta divisors, {\em Ill. J. Math.,} to appear.

\bibitem{chap8-key5} D.~Mumford, {\em Abelian Varieties,} Oxford University Press, Oxford, 1970.

\bibitem{chap8-key6} D. Mumford, On the equations defining abelian varieties III, {\em Invent. Math.,} 3(1967) pp.~215-244.
\end{thebibliography}

\vskip 1cm

\noindent
Johns Hopkins University\\
Baltimore, Maryland, U.S.A.


