\title{On The Moduli Space Of Bundles On $K3$ Surfaces, I}
\markright{On The Moduli Space Of Bundles On $K3$ Surfaces, I}

\author{By S. Mukai}
\markboth{S. Mukai}{On The Moduli Space Of Bundles On $K3$ Surfaces, I}

\date{}
\maketitle

\setcounter{page}{263}
\setcounter{pageoriginal}{340}

IN\pageoriginale \cite{key12}, WE have shown that the moduli space $M_S$ of stable
sheaves on a $K3$ or abelian surface $S$ is smooth and has a natural
symplectic structure. In this article, we shall study $M_S$ more
precisely in the case $S$ is of type $K3$. We shall show that every
compact $2$ dimensional component of $M_S$ is a $K3$ surface isogenous
to $S$ (Definition~\ref{dfn1.7} and \ref{dfn1.8}) and describe its
period explicitly (Theorem~\ref{Theorem1.4}). As an application of
this result, we shall show that certain Hodge cycles on a product of
two $K3$ surfaces are algebraic (Theorem~\ref{Theorem1.9}). As a
corollary, we have that two $K3$ surfaces with Picard number $\geqq
11$ are isogeneous in our sense if and only if their transcendental
Hodge structures $T_S$ and $T_{S'}$ are isogenous, i.e., isomorphic
over $\mathbb{Q}$ (Corollary~\ref{cor1.10}).

%%%footnote
This work was done during the author's stay at the Institute for
Advanced Study in 1981-2, at the Max Planck Institut fur Mathematik
first in 1982 and later in 1983 and at the Mathematics Institute in
University of Warwick in 1982-3. The stay at MPI was partially
supported by SFB 40 and others by Educational Projects for Japanese
mathematical Scientists. 

\section{Introduction}\label{s1}\pageoriginale

Let $S$ be an algebraic $K3$ surface over the complex number field
$\tau$. The cohomology group $H^{2}(S,\mathbb{Z})$ with the cup
product pairing is an even unimodular lattice and isomorphic to
$\wedge=U^{\perp 3}\perp E^{\perp 2}_8$ which we call a $K3$ lattice,
where $U$ is the hyperbolic lattice $\begin{bmatrix}
0 & 1\\
1 & 0\end{bmatrix}$ and $E_8$ is an even unimodular negative definite
lattice of rank $8$. We define a bilinear form and a Hodge structure
of weight $2$ on the cohomology ring $H^{\ast}(S,\mathbb{Z})$. The
integral bilinear form $(\cdot)$ on $H^{\ast}(S,\mathbb{Z})$ is
defined by 
\begin{equation}\label{eqn1.1}
(\alpha\cdot\beta)=-\alpha^{0
    \cup}\beta^{4}+\alpha^{2\cup}\beta^{2}-\alpha^{4\cup}\beta^{0}\in H^{4}(S,\mathbb{Z})\cong\mathbb{Z}
\end{equation}
for every $\alpha=\left(\alpha^{0}, \alpha^{2},\alpha^{4}\right)$ and
$\beta=\left(\beta^{0}, \beta^{2}, \beta^{4}\right)$ in
$H^{\ast}(S,\mathbb{Z})$, where we identify $H^{4}(S,\mathbb{Z})$ with
$\mathbb{Z}$ by the fundamental cocycle $\omega\in
H^{4}(S,\mathbb{Z})$. The Hodge decomposition of
$H^{\ast}(S,\mathbb{C})=H^{\ast}(S,\mathbb{Z})\oplus \mathbb{C}$ is
defined by 
\begin{equation}\label{eqn1.2}
\begin{aligned}
&H^{\ast, 2, 0}(S,\mathbb{C})=H^{2,0}(S,\mathbb{C}),\\
&{}H^{\ast, 0, 2}(S,\mathbb{C})=H^{0,2}(S,\mathbb{C})\\
\text{and}\\
&{}H^{\ast,1,1}(S,\mathbb{C})=H^{0}(S,\mathbb{C})\oplus
  H^{1,1}(S,\mathbb{C})\oplus H^{4}(S,\mathbb{C}).
\end{aligned}
\end{equation}
$H^{\ast}(S,\mathbb{Z})$ with the bilinear form \eqref{eqn1.1} and the
Hodge structure \eqref{eqn1.2} is denoted by
$\widetilde{H}(S,\mathbb{Z})\cdot H^{2}(S,\mathbb{Z})$ is a sublattice
and a Hodge substructure of $\widetilde{H}(S,\mathbb{Z})$. 

Let $E$ be a sheaf on $S$. Since $H^{2}(S,\mathbb{Z})$ is an even
lattice, the Chern\pageoriginale character $ch(E)$ of $E$ belongs to
$H^{\ast}(S,\mathbb{Z})$. We denote $ch(E)\cdot \sqrt{td_S}\in
H^{\ast}(S,\mathbb{Z})=\widetilde{H}(S,\mathbb{Z})$ by $\upsilon(E)$
(Definition~\ref{dfn2.1}) The $H^{0}(S)$ -component of $\upsilon(E)$ is the
rank $r(E)$ of $E$ (at the generic point) and $H^{2}(S)$-component is
the $1$st Chern class $c_1(E)$. The $H^{4}(S)$-component of $\upsilon(E)$ is
denoted by $s(E)$. By the Riemann-Roch theorem, we have
$s(E)=r(E)+ch^{2}(E)=\mathcal{X}(E)-r(E)\cdot v(E)$ is of type (1,1)
with respect to the Hodge structure defined in (\ref{eqn1.2}). For
sheaves $E$ and $F$ on $S,\mathcal{X}(E,F)$ denotes the alternating
sum $\sum\limits_{i}(-1)^{i}\dim \ext^{i}\mathscr{O}_S(E,F)$. By the
Riemann-Roch theorem, we have (see Proposition~\ref{Prop2.2})
$$
\mathcal{X}(E,F)=-(\upsilon(E)\cdot \upsilon(F)).
$$
Let $\upsilon$ be a vector of $\widetilde{H}(S,\mathbb{Z})$ of Hodge type
(1,1), and let $M_A(\upsilon)$ be the moduli space of stable sheaves $E$ on
$S$ with $\upsilon(E)=\upsilon$ which are stable with respect to $A$ in the
sense of \cite{key2}. Then $M_A(\upsilon)$ is smooth and each component has
dimension $(\upsilon^{2})+2$. Assume that $\upsilon$ is isotropic, i.e.,
$(\upsilon^{2})=0$ and that $\upsilon$ is primitive, i.e., not divisible by any
integer $\geqq 2$. Then $M_A(\upsilon)$ is 2-dimensional. The orthogonal
complement $\upsilon^{\perp}$ of $\upsilon$ in $\widetilde{H}(S,\mathbb{Z})$ contains $\upsilon$
and the quotient $\dfrac{\upsilon^{\perp}}{\mathbb{Z} \upsilon}$ is a free
$\mathbb{Z}$-module of rank 22. The quadratic form on
$\widetilde{H}(S,\mathbb{Z})$ defined in \eqref{eqn1.1} induces a
quadratic form on $\dfrac{\upsilon^{\perp}}{\mathbb{Z}\upsilon}$ with signature
(3, 19). Since $\upsilon$ is of type \eqref{eqn1.1}, the Hodge
decomposition of $\widetilde{H}(S,\mathbb{C})$ induces that of
$\left(\dfrac{\upsilon^{\perp}}{\mathbb{Z}\upsilon}\right)\otimes
\mathbb{C}$. Hence $\dfrac{\upsilon^{\perp}}{\mathbb{Z}\upsilon}$ carries the
polarized Hodge structure of the same kind as $H^{2}(S,\mathbb{Z})$. 


\setcounter{dfn}{3}
\begin{Theorem}\label{Theorem1.4}
Let $S$ be an algebraic $K3$ surface and $\upsilon$ a primitive isotropic
vector of $\widetilde{H}(S,\mathbb{Z})$. Assume that the moduli space\pageoriginale
$M_A(\upsilon)$ is nonempty and compact. Then $M_A(\upsilon)$ is
irreducible and is a (minimal) $K3$ surface. Moreover, there is an
isomorphism of Hodge structures between
$H^{2}(M_A(\upsilon),\mathbb{Z})$ and
$\dfrac{\upsilon^{\perp}}{\mathbb{Z}\upsilon}$ which is compatible
with the cup product pairing on $H^{2}(M_A(\upsilon))$ and the
bilinear form $\dfrac{\upsilon^{\perp}}{\mathbb{Z}\upsilon}$ induced
by that on $\widetilde{H}(S,\mathbb{Z})$. 
\end{Theorem}

The above theorem and the Torelli theorem for $K3$ surfaces
(\cite{key7},\\\cite{key20}) determine the isomorphism class of
$M_A(\upsilon)$ uniquely There are many pairs of $\upsilon$ and $A$
for which the moduli spaces $M_A(\upsilon)$ are compact
(Proposition~\ref{Prop4.1} and Proposition~\ref{Prop4.3}).

\begin{Remark}
Even if $M_A(\upsilon)$ is not compact, every component of
$M_A(\upsilon)$ is birationally equivalent to a $K3$ surface $M$ and
the period of $M$ is isomorphic to
$\dfrac{\upsilon^{\perp}}{\mathbb{Z}\upsilon}$.

Now we show how the isomorphism between
$H^{2}(M_A(\upsilon),\mathbb{Z})$ and
$\dfrac{\upsilon^{\perp}}{\mathbb{Z}\upsilon}$ is obtained. The
isomorphism is induced by a natural algebraic cycle on $S\times
M_A(\upsilon)$. There exists a sheaf $\mathscr{E}$ on $S\times
M_A(\upsilon)$ which we call a quasi-universal sheaf (Definition~A.\ref{dfnn4}
and Theorem~A.\ref{Thm5}). $\mathscr{E}$ is flat over $M_A(\upsilon)$ and the
restriction to $S\times m$ is isomorphic to $E^{\oplus\sigma}_m$ for
every point $m\in M_A(\upsilon)$, where $E_m$ is a stable sheaf in
$M_A(\upsilon)$ corresponding to $m$. The integer
$\sigma=\sigma(\mathscr{E})$ does not depend on $m$ and is called the
similitude of $E$. Let $ch(\mathscr{E})\in H^{\ast}(S\times
M_A(\upsilon),\mathbb{Q})$ be the Chern character of
$\mathscr{E}$. Put
$Z_{\mathscr{E}}=\left(\pi^{\ast}_S\sqrt{td_S}\right)\cdot
ch(\mathscr{E})\cdot\left(\dfrac{\pi^{\ast}_{M}\sqrt{td_M}}{\sigma(\mathscr{E})}\right)$,
where $td_S$ is the Todd class of $S$ and $M=M_A(\upsilon)\cdot
Z_{\mathscr{E}}$ is an algebraic cycle on $S\times M_A(\upsilon)$
(with $\mathbb{Q}$ coefficient) and induces the homomorphism 
$$
\xymatrix{f_{Z\mathscr{E}}:\widetilde{H}(S,\mathbb{Q})\ar@{=}[d]\ar[r]& \widetilde{H}(M_A(\upsilon),\mathbb{Q}\ar@{=}[d])\\
H^{\ast}(S,\mathbb{Q})\ar[r]&H^{\ast}(M_A(\upsilon),\mathbb{Q}).\\
\mbox{\rotatebox{90}{$\in$}} & \mbox{\rotatebox{90}{$\in$}}\\
t\ar[r]&\pi_{M,\ast}(Z_{\mathscr{E}}\cdot \pi^{\ast}_St)}
$$
$f_{Z_{\mathscr{E}}}$\pageoriginale is a homomorphism of Hodge
structures. $f_{Z_{\mathscr{E}}}$ sends $\upsilon$ to the fundamental
cocycle $w\in H^{4}(M_A(\upsilon),\mathbb{Z})$
(Lemma~\ref{lemma4.11}) and maps $\upsilon^{\perp}$ into
$H^{0}(M_A(\upsilon),\mathbb{Q})\oplus H^{2}(M_A(\upsilon),
\mathbb{Q})$. Hence $f_{Z_{\mathscr{E}}}$ induces the homomorphism
$\varphi_{\mathbb{Q}}=\dfrac{\left(\upsilon^{\perp}\otimes \mathbb{Q}\right)}{\mathbb{Q}\upsilon}
\to H^{2}\left(M_A(\upsilon),\mathbb{Q}\right)$. 
\end{Remark}

\begin{Theorem}\label{Theorem1.5}
Assume that $\upsilon$ is an isotropic vector and that $M_A(\upsilon)$
is nonempty and compact. Then we have 
\begin{enumerate}
\renewcommand{\labelenumi}{(\theenumi)}
\item $\varphi_{\mathbb{Q}}$ does not depend on the choice of a
  quasi-universal family $\mathscr{E}$ on $S\times M_A(\upsilon)$, 
\item $\varphi_{\mathbb{Q}}$ is an isomorphism of Hodge structures and
  compatible with the bilinear forms on
  $\dfrac{(\upsilon^{\perp}\otimes \mathbb{Q})}{\mathbb{Q}\upsilon}$
  and $H^{2}(M_A(\upsilon),\mathbb{Q})$, and 
\item $\varphi_{\mathbb{Q}}$ is defined over $\mathbb{Z}$, i.e.,
  $\varphi_{\mathbb{Q}}\left(\dfrac{\upsilon^{\perp}}{\mathbb{Z}\upsilon}\right)=H^{2}(M_A(\upsilon),
  \mathbb{Z})$. 
\end{enumerate}
\end{Theorem}

If $\mathscr{E}$ is a universal family (i.e.,
$\sigma(\mathscr{E})=1$), then $Z_{\mathscr{E}}$ is integral and
$f_{Z\mathscr{E}}$ gives an Hodge isometry of between
$\widetilde{H}(S,\mathbb{Z})$ and $\widetilde{H}(M,\mathbb{Z})$ (Theorem~\ref{Theorem4.9}).

\begin{remark}\label{remark1.6}
The\pageoriginale relation between the periods of a variety $X$ and the moduli space
of bundles on $X$ was studied in the case $X$ is a curve in
\cite{key16}: Let $M$ be the moduli space of stable rank $2$ bundles
with a fixed determinant $\xi$. If $\deg \xi$ is odd, then $M$ is
compact and the two polarized Hodge structures $H^{1}(C,\mathbb{Z})$
and $H^{3}(M,\mathbb{Z})$ are isomorphic and the isomorphism is given
by using the Chern class of a universal family on $C\times M$. (Since
the weights are odd, in this case, the polarization is not symmetric
but skew symmetric).

The following is a natural analogue of the notion of isogeny of
abelian surfaces.
\end{remark}

\begin{dfn}\label{dfn1.7}
An algebraic cycle $Z\in H^{4}(S\times S',\mathbb{Q})$ on a product of
two $K3$ surfaces $S$ and $S'$ is an isogeny, if the homomorphism
$f_z:H^{2}(S,\mathbb{Q})\to H^{2}(S',\mathbb{Q}),t\to
\pi_{S},\cdot_{\ast}(Z.\pi^{\ast}_St)$, is an isometry, i.e. an
isomorphism compatible with cup product pairings. 

$f_Z$ is an isometry if and only if so is the homomorphism
$f'_Z:H^{2}\\(S',\mathbb{Q})\to H^{2}(S,\mathbb{Q})$, $t'\to
\pi_{S},\ast(Z,\pi^{\ast}_St')$ because $f_Z$ and $f'_Z$ are adjoint to
each other with respect to the cup product pairings. In fact, we have
$(t'\cdot f_Z(t))=(\pi_{S'}^{\ast}t'\cdot Z\cdot
\pi^{\ast}_St)=(f'_Z(t)\cdot t)$ for every $t\in H^{2}(S,\mathbb{Q})$
and $t'\in H^{2}(S',\mathbb{Q})$.
\end{dfn}

\begin{dfn}\label{dfn1.8}
Two $K3$ surfaces $S$ and $S'$ are isogenous if there exists an
isogeny $Z\in H^{4}(S\times S',\mathbb{Q})$ on $S\times S'$. 

Let $N_S$ be the N\'{e}ron-Severi group of $S.N_S$ is canonically\pageoriginale
isomorphic to $H^{1,1}(S,\mathbb{Z})$ and is a primitive sublattice of
$H^{2}(S,\mathbb{Z})$. The orthogonal complement $T_S$ of $N_S$ is
called the \textit{transcendental lattice}  of $S$. Every cohomology
class in $N_S$ is of type (1,1) and any cohomology class in $T_S$ is
not so. $H^{2}(S,\mathbb{Z})$ contains $N_S\perp T_S$ as a sublattice
of a finite index and $H^{2}(S,\mathbb{Q})$ is isomorphic to
$(N_S\times \mathbb{Q})\perp (T_S\times \mathbb{Q})$. Hence the
cohomology group $H^{4}(S\times S',\mathbb{Q})$ is the direct sum of
$4$ vector spaces $N_S\times N_{S'}\otimes \mathbb{Q}, N_S\otimes
T_{S'}\otimes \mathbb{Q}$, $T_S\otimes N_{S'}\otimes \mathbb{Q}$ and
$T_S\otimes T_{S'}\otimes \mathbb{Q}$. Neither $N_S\otimes T_{S'}\otimes
\mathbb{Q}$ nor $T_S\otimes N_{S'}\otimes \mathbb{Q}$ contains a
cohomology class of type (2,2). Hence if $Z\in H^{4}(S\times
S',\mathbb{Q})$ is a Hodge cycle, then $Z$ is the sum of
$Z_{\upsilon}\in N_S\otimes N_{S'}\otimes \mathbb{Q}$ and $Z_{\tau}\in
T_S\otimes T_{S'}\otimes \mathbb{Q}$. $Z_{\upsilon}$ is always an
algebraic cycle. Hence a Hodge cycle $Z$ is algebraic if and only if
so is $Z_{\tau}\cdot Z_{\tau}$ induces the homomorphism
$f^{\tau}_Z:T_S\otimes \mathbb{Q}\to T_{S'}\otimes \mathbb{Q}$. In
particular, $S$ and $S'$ are isogeneous if and only if there exists an
algebraic cycle $Z$ on $S\times S'$ such that $f^{\tau}_Z:T_S\otimes
\mathbb{Q}\to T_{S'}\otimes \mathbb{Q}$ is an isometry. By
Theorem~\ref{Theorem1.5}, $Z_{\mathscr{E}}$ is an isometry and $S$ and
$M_A(\upsilon)$ are isogeneous. As an application of this fact, we
have 
\end{dfn}

\begin{Theorem}\label{Theorem1.9}
Let $S$ and $S'$ be algebraic $K3$ surfaces and $Z\in H^{4}(S\times
S',\mathbb{Q})$ a Hodge cycle on $S\times S'$. Assume that
$f^{\tau}_Z:T_S\otimes \mathbb{Q}\to T_{S'}\otimes \mathbb{Q}$ is an
isometry and that the lattice $T=T_S\cap(f^{\tau}_Z)^{-1}T_S'$ can be
primitively embedded into a $K3$ lattice $\wedge$. Then $Z$ is an
algebraic cycle.
\end{Theorem}

If $\rho(S)\geqq 11$, then rank $T\leqq 11$ and $T$ can be primitively
embedded into $\wedge$ by Corollary 1.12.3 in
\cite{key17}. Hence we have 


\begin{cor}\label{cor1.10}
If $\rho(S)\geqq 11$ and if $f^{\tau}_Z:T_S\otimes \mathbb{Q}\to
T_S\otimes \mathbb{Q}$\pageoriginale is an isometry, then the Hodge cycle $Z$ is
algebraic. 
\end{cor}

\begin{remark}\label{remark1.11}
By the corollary, two $K3$ surfaces $S$ and $S'$ with $\rho\geq 11$
are isogenous if and only if the Hodge structures $T_S$ and $T_{S'}$ are
so. This partially answers to the question posed in \cite{key21}. For
$K3$ surfaces with $\rho=20$, this has been proved by Shioda-Inose
\cite{key22}. Moreover, Inose \cite{key4} has proved that if $T_S$ and
$T_{S'}$ are isogenous for such two $K3$ surfaces $S$ and $S'$, then
there exist rational maps of finite degree from $S$ to $S'$ and from
$S'$ to $S$. 

In \cite{key10}, Morrison has proved that if $T_S$ has a primitive
embedding $T_S\hookrightarrow U^{\perp 3}$, then there exist an
abelian surface $A$ and a certain algebraic correspondence on $S\times
A$ which induces $T_S\cong T_A$. By this result and the above
corollary, we have 
\end{remark}

\begin{Theorem}\label{Theorem1.12}
Let $S$ be an algebraic $K3$ surface. If $T_S\otimes \mathbb{Q}$ can
be embedded into $(U\otimes \mathbb{Q})^{\perp 3}$ as a lattice, then
there exists an algebraic cycle on $S \times A$ which induces an
isometry between $T_S\otimes \mathbb{Q}$ and $T_A\otimes \mathbb{Q}$ 
\end{Theorem}

This was conjectured in \cite{key10} by modifying Oda's conjecture in
\cite{key19}.

\begin{notation}
A $K3$ surface always means a minimal algebraic $K3$ surface over
  $\mathbb{C}$, throughout this article. For a complex manifold $X$
  over $\mathbb{C},\\H^{\ast}(X,\mathbb{Z})$ is the cohomology ring of
  $X$. The even (resp. odd)\pageoriginale part of $H^{\ast}(X,\mathbb{Z})$ is
  denoted by $H^{e\upsilon}(X,\mathbb{Z})$
  (resp. $H^{odd}(X,\mathbb{Z})$. $\ast$ is the involution of
  $H^{e\upsilon}(X,\mathbb{Z})$ which is $+1$ on
  $\bigoplus\limits_{n}H^{4n}(X)$ and $-1$ on
  $\bigoplus\limits_{n}H^{4n+2}(X)$. 

A sheaf on $X$ is a choerent
$\mathscr{O}_{\mathcal{X}}$-module. $h^{i}(E)$ is the dimension of the
cohomology group $H^{i}(X,E)$ and $\mathcal{X}(E)$ is the alternating
sum $\sum(-1)^{i}\\h^{i}(E)$. For an ample line bundle $A$ and a
nontorsion sheaf $E$, the rational number $\dfrac{(c_1(E)\cdot A^{\dim
  X-1})}{r(E)}$ is called the slope of $E$ with respect to $A$ and
denoted by $\mu_A(E)$. A torsion free sheaf $E$ is $\mu$-stable
(resp. $\mu$-semi-stable) with respect ot $A$, if $\mu_A(F)>\mu_A(E)$
(resp. $\mu_A(F)\geqq \mu_A(E))$ for every proper nontorsion quotient
sheaf $F$ of $E$. The set of isomorphism classes of all $\mu$-stable
(resp.$\mu$-semi-stable) sheaves on $X$ is denoted by $M^{\mu}_X$
$\left(\text{resp.} SM^{\mu}_X\right)$. $M^{\mu}_{X}$ is an open subset of the moduli
space $M_X$ of stable (in Gieseker's sense) sheaves on $X$. For a
sheaf $E$ on $X,E^{\vee}$ denotes the dual sheaf
$\hom_{\mathscr{O}X}(E,\mathscr{O}_X)\cdot ch(E)\in
H^{e\upsilon}(X,\mathbb{Q})$ is the Chern character of $E$. If $E$ is locally
free, then we have $ch(E^{v})=ch(E)^{\ast}$. 

A \textit{lattice} over a ring $R$ is a free $R$-module $L$ with a symmetric
bilinear form $(\cdot):L\times L\to R$ and a lattice means a lattice
over $\mathbb{Z}$. A sublattice $L_0$ of $L$ is primitive if
$\dfrac{L}{L_0}$ has no torsion and a vector $\upsilon$ of $L$ is
primitive if $\mathbb{Z} \upsilon$ is a primitive sublattice. An
isomorphism $f:L\xrightarrow{\sim}L'$ between two lattices $L$ and
$L'$ is an isometry if $f$ is compatible with the bilinear forms on
$L$ and $L$. 

For\pageoriginale an algebraic variety $X$, the N\'{e}ron-Severi group
$N_X$ is the Picard group Pic ($X$) modulo algebraic equivalence. The
Picard number $\rho(X)$ is the rank of $N_X$. If $S$ is a $K3$
surface, then the natural map Pic$(S)\to N_S$   is a bijection. For
$\ell \in N_S$, we denote by $\mathscr{O}_S(\ell)$ the line bundle
corresponding to $\ell$. 
\end{notation}


\section{Generalities}\label{s2}

In this section, we assume that $S$ is an abelian or $K3$ surface. The
Todd class $td_S$ of $S$ is equal to $1+2\epsilon w$, where $1\in
H^{0}(S,\mathbb{Z})$ is the unit element of the cohomology ring
$H^{\ast}(S,\mathbb{Z})$, $w \in H^{4}(S,\mathbb{Z})$ is the
fundamental cocycle of $S$ and $\epsilon$ is equal to $0$ or $1$
according as $S$ in abelian or of type $K3$. The positive square root
$\sqrt{td_S}=1+\epsilon w$ lies in the even part
$H^{e\upsilon}(S,\mathbb{Z})$ of $H^{\ast}(S,\mathbb{Z})$. Let $E$ be
a sheaf on $S$. Then the Chern character $ch(E)$ belongs to
$H^{e\upsilon}(S,Z)$. 

\begin{dfn}\label{dfn2.1}
For a sheaf $E$, we put $\upsilon(E)=ch(E)$. $\sqrt{td_S}\in
H^{e\upsilon}(S,\mathbb{Z})$ and call it the vector associated to
$E$. 

We define a symmetric integral bilinear form ( $\cdot$ ) on
$H^{e\upsilon}(S,\mathbb{Z})$ by 
$$
(u\cdot u')=\alpha^{\cup}\alpha'-r^{\cup}s'-s^{\cup}r'\in
H^{4}(S,\mathbb{Z})\cong \mathbb{Z}
$$
for every $u=(r,\alpha, s)$ and $u'=\left(r',\alpha',s'\right)\in
H^{0}(S,\mathbb{Z})\oplus H^{2}(S,\mathbb{Z})\oplus
H^{4}(S,\mathbb{Z})$. We denote $H^{e\upsilon}(S,\mathbb{Z})$ with
this inner product ( $\cdot$ ) by
$\widetilde{H}(S,\mathbb{Z})$. $\widetilde{H}(S,\mathbb{Z})$\pageoriginale is an
even lattice of rank $8(1+2\epsilon)$ and isomorphic to $U^{\perp
  4}\perp E_8^{\perp 2}\epsilon$ as an abstract lattice. The inner
product $(u\cdot u')$ is equal to the $H^{4}(S,\mathbb{Z})$-component
of $-u^{\ast}\cdot u\in H^{e\upsilon}(S,\mathbb{Z})$. Hence, for
sheaves $E$ and $F$ on $S,(\upsilon(E)\cdot \upsilon(F))$ is equal to
the $H^{4}(S)$-component of $-ch(E)^{\ast}\cdot ch(F)\cdot
td_S$. Therefore, by the Riemann-Roch theorem, we have 
\end{dfn}

\begin{Prop}\label{Prop2.2}
Let $E$ and $F$ be sheaves on $S$ and put
$\mathcal{X}(E,F)=\sum\limits_{i}(-1)^{i}\dim
\ext^{i}_{\mathscr{O}_{S}}(E,F)$. Then we have
$\mathcal{X}(E,F)=-(\upsilon (E)\cdot \upsilon(F))$. 
\end{Prop}

\begin{Proof}
If $E$ is locally free, then $\ext^{i}_{\mathscr{O}_S}(E, F)$ is
canonically isomorphic to $H^{i}(S,E^{\vee}\otimes F)$ for every $i$ and
$-ch(E)^{\ast}\cdot ch(F)\cdot td_S$ is equal to $-ch(E\otimes F)\cdot
td_S$. Hence our assertion follows from the usual Riemann-Roch
theorem. If $0\to E'\to E\to E''\to 0$ is an exact sequence, then
$\mathcal{X}(E,F)$ and $(\upsilon (E)\cdot\upsilon (F))$ are equal to
$\mathcal{X}(E',F)+\mathcal{X}(E'',F)$ and $(\upsilon (E')\cdot
\upsilon(F))+(\upsilon(E'')\cdot \upsilon(F))$, respectively. Since
$E$ has a resolution by locally free, sheaves, we have our assertion
for every sheaves $E$ and $F$.
\enprf
\end{Proof}

The dualizing sheaf $\omega_S$ of $S$ is trivial. Hence the Serre
duality is simple in form and is a very effective tool of our study.

\begin{Prop}\label{Prop2.3}
Let $E$ and $F$ be sheaves on $S$. Then the pairing
$\ext^{i}_{\mathscr{O}_S}\\(E,F)\times
\ext^{2-i}_{\mathscr{O}_S}(F,E)\to H^{2}(\mathscr{O}_S), (\alpha,
\beta)\to tr^{2}(\alpha \circ \beta)$ is\pageoriginale nondegenerate for every $i$,
where $tr^{2}:\ext^{2}_{\mathscr{O}_S}(F,F)\to H^{2}(\mathscr{O}_S)$ is the trace
homomorphism of $\ext^{2}_{\mathscr{O}_S}(F,F)$. In particular we
have $\dim \ext^{2}_{\mathscr{O}_S}(E,F)=\dim
\hom_{\mathscr{O}_S}(F,E)$ and $\dim \ext^{1}(E,F)=\dim
\ext^{1}_{\mathscr{O}_{S}}(F,E)$. 
\end{Prop}

\begin{Proof}
The usual Serre duality says that the natural pairing
$H^{i}(S,G)\times \ext_{\mathscr{O}_S}^{2-i}(G,\omega_S)\to
H^{2}(S,\omega_S)$ is nondegenerate for every sheaf $G$ on $S$. In the
case where $E$ is locally free, applying this Serre duality for
$G=E^{\vee}\otimes F$, we have our proposition. In the general case, take
locally free resolutions $0\to E^{m}\to E^{m-1}\to\cdots \to E^{0}\to
E\to0$ and $0\to F^{m}\to F^{n-1}\to\cdots F^{0}\to F\to 0$ of $E$ and
$F$, and apply the Serre duality for $\hom^{\bigdot}_{\mathscr{O}_S}(E^{\bigdot},F^{\bigdot})$ in
the derived category $D(S)$ of $S$(\cite{key3}), where
$E^{\bigdot}=[0\to E^{m}\to E^{m-1}\to \cdots \to E^{0}\to 0]$ and
$F^{\bigdot}=[0\to F^{n}\to F^{n-1}\to\cdots \to F^{0}\to 0]$. Then we
have our proposition.
\enprf
\end{Proof}

In the special case where $E=F$, the Serre pairing is a non degenerate
bilinear form on $\ext^{1}_{\mathscr{O}_S}(E,E)$ which we call the
Serre bilinear form. This form is skew symmetric.

By Proposition~\ref{Prop2.2} and \ref{Prop2.3}, we have 

\begin{Prop}\label{Prop2.4}
$(\upsilon(E)\cdot\upsilon(F))=\dim \ext^{1}_{\mathscr{O}_S}(E,F)\dim \hom_{\mathscr{O}_S}(E,F)-\dim\hom_{\mathscr{O}_{S}}(F,E)$. 
\end{Prop}

\begin{cor}\label{cor2.5}
$\dim \ext^{1}_{\mathscr{O}_S}(E,E)=(\upsilon(E)^{2})+2\dim
\End_{\mathscr{O}_S}(E)$\pageoriginale for every sheaf $E$ on $S$. In particular,
$\dim \ext^{1}_{\mathscr{O}_S}(E,E)$ is always an even integer. If $E$
is simple, then
$\dim \ext^{1}_{\mathscr{O}_{S}}(E,E)=$($\upsilon(E)^{2}+2$) and hence
$(\upsilon(E)^{2})\geqq -2$.
\end{cor}

The tangent space of Spl$_{S}$ (or $M_A$) at the point $[E]\in$
Spl$_S$ is canonically isomorphic to
$\ext^{1}_{\mathscr{O}_S}(E,E)$. Since Spl$_S$ is smooth
(\cite{key12}), we have 

\begin{cor}\label{cor2.6}
Let $\upsilon$ be a vector of $\widetilde{H}(S,\mathbb{Z})$. Then
every component of Spl$_S(\upsilon)$ is smooth and has dimension
$\left(\upsilon^{2}\right)+2$. 
\end{cor}

Next we prove some inequalities for $\left(\upsilon(E)^{2}\right)$ and
$\dim\ext^{1}_{\mathscr{O}_S}(E,E)$ which play an important role for
our study of sheaves on $S$. 

\begin{Prop}\label{Prop2.7}
Let $X:\to F\xrightarrow{f} E\xrightarrow{g} G\to 0$ be an exact
sequence of sheaves on $S$ such that
$\hom_{\mathscr{O}_S}(F,G)=0$. Define
$i:\ext^{1}_{\mathscr{O}_S}(G,F)\to \ext^{1}_{\mathscr{O}_S}(E,E)$ and
$j:\ext^{1}_{\mathscr{O}_S}(E,E)\to \ext^{1}_{\mathscr{O}_S}(F,G)$ by
$i(\alpha)=f\circ \alpha \circ g$ and $j(\beta)=g\circ \beta \circ
f$. Let I be the image of $i$ and $J$ the kernel of $j$. Then we have 
\begin{enumerate}
\renewcommand{\labelenumi}{(\theenumi)}
\item $I\subset J$ and the quotient $\dfrac{J}{I}$ is isomorphic to
$$
\ext^{1}_{\mathscr{O}_S}(F,E)\oplus \ext^{1}_{\mathscr{O}_S}(G,G),
$$ 

\item Let $e\in \ext^{1}_{\mathscr{O}_S}(G,F)$ be the extension class
of $X$ and define the homomorphism $h:\End_{\mathscr{O}_S}(F)\oplus
\End_{\mathscr{O}_S}(G)\to \End_{\mathscr{O}_S}(G,F)$ by $h(e_{F},
e_{G})=e_{F}\circ e-e\circ e_{G}$. Then the sequence (2.7.1) $0\to
\End_{\mathscr{O}_S}(E)\to \End_{\mathscr{O}_S}(F)\oplus
\End_{\mathscr{O}_{S}}(G)\xrightarrow{h} \ext_{\mathscr{O}_S^i}(G,F)\xrightarrow{i}\ext_{\mathscr{O}_{S}}\\(E.E)$\pageoriginale
is exact (since $\hom_{\mathscr{O}_S}(F,G)=0$, every endomorphism of
$E$ preserves $X$ and induces endomorphisms of $F$ and $G$), and 
\item $J$ is the orthogonal complement $I^{\perp}$ of $I$ with respect
to the Serre bilinear form on $\ext^{1}_{\mathscr{O}_S}(E,E)$ and $I$
is totally isotropic. 
\end{enumerate}
\end{Prop}

\begin{Proof}
\begin{enumerate}
\renewcommand{\labelenumi}{(\theenumi)}
\item Since $g\circ f=0$, $j\circ i=0$ and $J$ contains $I$. We show
that $\dfrac{J}{I}$ is isomorphic to
$\ext^{1}_{\mathscr{O}_S}(F,F)\oplus \ext^{1}_{\mathscr{O}_S}(G,G)$. If
$\alpha \in \ext^{1}_{\mathscr{O}_S}(E,E)$ belongs to $J$, then
$(g\circ \alpha)\circ f=0$. Hence there exists
$\alpha_G\in \ext^{1}_{\mathscr{O}_S}(G,G)$ such that
$g\circ \alpha=\alpha_G\circ g$. Since $\hom_{\mathscr{O}_S}(F,G)=0$,
such an $\alpha_G$ is unique. In a similar way, there exists a unique
$\alpha_F\in \ext^{1}_{\mathscr{O}_S}(F,F)$ such that $\alpha\circ
f=f\circ \alpha_F$. It is easy to see that the map
$\varphi:J\to \ext^{1}_{\mathscr{O}_S}(F,F)\oplus \ext^{1}_{\mathscr{O}_S}(G,G), \alpha \mapsto
(\alpha_F,\alpha_G)$ is a homomorphism.

\begin{claim}
$\ker \ \varphi=I$.
\end{claim}

If $\alpha\in I$, then $g\circ \alpha=\alpha\circ f=0$. Hence
$\alpha_F=\alpha_G=0$ and $I$ is contained in $\ker \varphi$. Assume
that $\alpha$ belongs to $\ker \varphi$. Then we have $\alpha\circ
f=g\circ \alpha=0$. Hence there exists
$\beta \in \ext^{1}_{\mathscr{O}_S}(E,F)$ such that
$\alpha=f\circ \beta$. Since $f\circ (\beta \circ f)=\alpha\circ f=0$
and since $\ext^{1}_{\mathscr{O}_S}(F,F)$
$f^{\circ}\xrightarrow{\ast}\ext^{1}_{\mathscr{O}_S}(F,E)$ is injective,
we have $\beta\circ f=0$. Hence $\beta=\gamma\circ g$ for some
$g\in \ext^{1}_{\mathscr{O}_S}(F,G)$. Therefore, $\alpha$ is equal to
$f\circ \gamma\circ g$ and belongs to $I$. 

\begin{claim}
$\varphi$ is surjective.
\end{claim}

By\pageoriginale the Serre duality and by our assumption, we have
$\ext^{2}_{\mathscr{O}_S}\\(G,F)=0$. Hence the homomorphism
$\ext^{1}_{0_S}(E,F)\xrightarrow{\ast \circ
f}\\\ext^{1}_{\mathscr{O}_S}(F,F)$ is surjective. Therefore, for every
$\alpha_F \in \ext^{1}_{\mathscr{O}_S}(F,F)$, there exists
$\beta \in \ext^{1}_{\mathscr{O}_S}(E,F)$ such that
$\alpha_F=\beta\circ f$. Put $\alpha=f\circ \beta \in
\ext^{1}_{\mathscr{O}_S}(E,E)$. Then it is easy to see that
$\varphi(\alpha)=(\alpha_F,0)$. In a similar way, for every
$\alpha_G\in \ext^{1}_{\mathscr{O}_S}(G,G)$, we obtain
$\alpha\in \ext^{1}_{\mathscr{O}_S}(E,E)$ such that
$\varphi(\alpha)=(0,\alpha_G)$. Hence $\varphi$ is surjective.

\item If $h(e_{F},e_{G})=0$, then $e_{F}\circ e=e\circ e_{G}$ which
means that two endomorphisms $e_F$ and $e_G$ of $F$ and $G$ are
compatible with respect to the extension class of $X$. Hence there
exists an endomorphism of $E$ which induces $e_F$ and
$e_G$. Therefore, the sequence (2.7.1) is exact at
$\End_{\mathscr{O}_S}(F)\oplus \End_{\mathscr{O}_S}(G)$. Since $f\circ
e=e\circ g=0$, we have $h\circ i=0$. Assume that
$\alpha\in \ext^{1}_{\mathscr{O}_S}(G,F)$ and $i(\alpha)=0$, i.e.,
$f\circ(\alpha \circ g)=0$. Then there exists
$\beta \in \hom_{\mathscr{O}_S}(E,G)$ such that $\alpha\circ
g=e\circ \beta$. Since $\hom_{\mathscr{O}_S}(F,G)=0$, there exists an
endomorphism $\gamma_G$ of $G$ such that $\beta=\gamma_{G}\circ
g$. Since $(\alpha-e\circ \gamma_G)\circ g=0$, there exists an
endomorphism $\gamma_F$ of $F$ such that
$\alpha-e^{\circ}\gamma_G=\gamma_F\circ e$. Therefore, $\alpha$ lies
in the image of $h$ and the sequence (2.7.1) is exact at
$\ext^{1}_{\mathscr{O}_S}(G,F)$.
 
\item Since $\omega_S$ is trivial, the homomorphisms $i$ and $j$ are
dual to each other by the Serre duality. Hence $I$ and
$\ext^{1}_{\mathscr{O}_S}\dfrac{(E,E)}{J}$ are dual to each other. If
$\alpha \in I$ and $\beta \in J$, then
$\alpha\circ \beta \in \ext^{2}_{\mathscr{O}_S}(E,E)$ is zero. Hence
$I$ and $F$ are perpendicular with respect to the Serre bilinear form
on $\ext^{1}_{\mathscr{O}_S}(E,E)$. Since the Serre bilinear form is
nondegenerate, $J$ coincides with $I^{\ell}$.
\end{enumerate}
\enprf
\end{Proof}

\begin{cor}\label{cor2.8}
(\cite{key11})\pageoriginale Let $X$ be same as above. Then we have
$$
\dim \ext^{1}_{\mathscr{O}_S}(F,F)+\dim\ext^{1}_{\mathscr{O}_{S}}(G,G)\leqq \dim\ext^{1}_{\mathscr{O}_S}(E,E).
$$  
\end{cor}

\begin{remark}\label{remark2.9}
If $S$ is a surface and $\mid K_S\mid \neq \phi$, then
$\hom_{\mathscr{O}_S}(F,G)=0$ implies
$\ext^{2}_{\mathscr{O}_S}(G,F)=0$. Hence (1) and (2) of the
proposition and the corollary are true for such surface (1) of the
proposition says that every infinitesimal deformation of $F$ and $G$
can be lifter to an infinitesimal deformation of $E$. 

The following proposition and its proof are quite similar to above
ones. In fact, these propositions are equivalent if one consider them
in the derived category $D(S)$ of $S$. 
\end{remark}

\begin{Prop}\label{Prop2.10}
Let $X:0\to E\xrightarrow{g} G\xrightarrow{e}F\to 0$ be an exact
sequence of sheaves on $S$ such that
$\ext^{1}_{\mathscr{O}_S}(F,G)=0$. Let
$f \in \ext^{1}_{\mathscr{O}_S}(F.E)$ be the extension class of
$X$. Define
$i=\hom_{\mathscr{O}_S}(G,F)\to \ext^{1}_{\mathscr{O}_S}(E,E)$ and
$j:\ext^{1}_{\mathscr{O}_S}(E,E)\to \ext^{2}_{\mathscr{O}_S}(F,G)$ by
$i(\alpha)=f\circ \alpha \circ g$ and $j(\beta)=g\circ \beta \circ
f$. Let $I$ be the image of $i$ and $J$ the kernel of $j$. Then we
have (1) and (3) in Proposition~\ref{Prop2.7} and (2) define the
homomorphism $h:\End_{\mathscr{O}_S}(F)\oplus
\End_{\mathscr{O}_S}(G)\to \hom_{\mathscr{O}_S}(G,F)$ by
$h(e_F,e_G)=e_F\circ e-e\circ e_G$ for $e_F\in \End_{\mathscr{O}_S}(F)$
and $e_G\in \End_{\mathscr{O}_S}(G)$. Every endomorphism of $E$ is
induced by that of $G$ and the sequence $0\to
\End_{\mathscr{O}_S}(E)\to \End_{\mathscr{O}_S}(F)\oplus
\End_{\mathscr{O}_S}(G)\xrightarrow{h} \ext^{1}_{\mathscr{O}_S}(G,F)^{i}\to \ext^{1}_{\mathscr{O}_S}(E,E)$\pageoriginale
is exact. In particular, if $I=0$, then $h$ is surjective.
\end{Prop}

\begin{Proof}
By the Serre duality, we have $\ext^{1}_{\mathscr{O}_S}(G,F)=0$. (1)
and (3) can be proved in a similar way to Proposition~\ref{Prop2.7}.
Since $\ext^{1}_{\mathscr{O}_S}(F,G)=0$, the map
$\End_{\mathscr{O}_S}(G)\to \hom_{\mathscr{O}_S}(E,G)$ is
surjective. Hence every endomorphism of $E$ is a restriction of an
endomorphism of $G$. Hence the homomorphism $\End_{\mathscr{O}_S}(E)\to
\End_{\mathscr{O}_S}(F)\oplus \End_{\mathscr{O}_S}(G)$  is well
defined. The exactness of the sequence can be proved in a similar way
to Proposition~\ref{Prop2.7}.
\enprf
\end{Proof}

\begin{cor}\label{cor2.11}
Let $X$ be same as above. Then we have
$\dim \ext^{1}_{\mathscr{O}_S}(F,F)\\+\dim \ext^{1}_{\mathscr{O}_S}(G,G)\leqq
\dim \ext^{1}_{\mathscr{O}_S}(E,E)$. 
\end{cor}

Let $E$ be a torsion free sheaf and $\widetilde{E}$ the double dual of
$E$. Then the natural homomorphism $E\to \widetilde{E}$ is injective
and the cokernel $M$ is of finite length. We have the exact sequence 
$$
0\to E\to \widetilde{E}\xrightarrow{e} M\to 0.
$$
Since $\widetilde{E}$ is locally free, we have
$\ext^{1}_{\mathscr{O}_S}\left(M,\widetilde{E}\right)\cong \ext^{1}_{\mathscr{O}_S}\left(\widetilde{E},
M\right)^{\vee}=0$. Since
$\left(\upsilon(M)^{2}\right)=0$. $\dim \ext^{1}_{\mathscr{O}_S}(E,E)$ is
equal to $2 \dim \End_{\mathscr{O}_S}(E)$  by
Corollary~\ref{cor2.5}. Hence we have 

\begin{cor}\label{cor2.12}\pageoriginale
Let $E$ be a torsion free sheaf on $S$ and $\widetilde{E}$ and $M$ be
as above. Then we have 
$$
\dim \ext^{1}_{\mathscr{O}_S}\left(\widetilde{E},\widetilde{E}\right)+2
\dim \End_{\mathscr{O}_S}(M)\leqq \dim \ext^{1}_{\mathscr{O}_S}(E,E).
$$
If equality holds in the above relation, then the natural homomorphism
$\End_{\mathscr{O}_S}\left(\widetilde{E}\right)\oplus
\End_{\mathscr{O}_S}(M)\to \hom_{\mathscr{O}_S}\left(\widetilde{E},M\right)$,
$(\alpha,\beta)\mapsto e^{\circ}\alpha-\beta\circ e$, is surjective. 
\end{cor}

\begin{lemma}\label{lemma2.13}
Let $(R,\mathfrak{m})$ be a local ring and $M$ an artinian
$R$-module. Then we have length $(\End_R(M))\geqq $ length $(M)$. If
equality holds, then $M$ is isomorphic to $\dfrac{R}{I}$ for an ideal
$I$ of $R$.
\end{lemma}

\begin{Proof}
We prove by induction on length $(M)$. Let $M_0$ be the submodule
$\{x\in M;\mathfrak{m} x=0\}$ of $M$. Every endomorphism of $M$ maps
$M_0$ into itself. Hence we have the exact sequence 
$$
0\to \hom_R(M,M_0)\to \End_R(M)\to
\End_R \left(\dfrac{M}{M_0}\right)\to 0.
$$
Since $M$ is artinian, $M_0$ is nonzero. Hence by induction
hypothesis, we have length $\left(\End_R\left(\dfrac{M}{M_0}\right)\right)\geqq$
length $\left(\dfrac{M}{M_0}\right)$. Since $\mathfrak{m} M_0=0$, every
homomorphism from $M$ to $M_0$ factors through
$\dfrac{M}{\mathfrak{m}M}$. Hence $\hom_R(M,M_0)$ is isomorphic to the
vector space
$\dfrac{\hom_R}{\mathfrak{m}(M/\mathfrak{m}M,M_0)}$. Therefore, we
have 
$$
\begin{aligned}
&\text{ length } (\End_R(M))= \text{ length } \left(\End_R\left(\dfrac{M}{M_0}\right)\right)+
\text{ length } (\hom_R(M,M_0))\\
&\geqq \text{  length } \left(\dfrac{M}{M_0}\right)+\text{ length
} \left(\dfrac{M}{\mathfrak{m}M}\right) \text{ length }(M_0)\\
&\geqq \text{ length }(M)
\end{aligned}
$$\pageoriginale
which shows the first half of the lemma. If equalities hold in the
above relations, then we have length
$\left(\End_R\left(\dfrac{M}{M_0}\right)\right)=\text{ length
}\left(\dfrac{M}{M_0}\right)$ and length
$\left(\dfrac{M}{\mathfrak{m}M}\right)=1$. By the latter equality and
the Nakayama's lemma, $M$ is generated by one element. Hence $M$ is
isomorphic to $\dfrac{R}{I}$ for an ideal $I$.
\enprf
\end{Proof}

By Corollary~\ref{cor2.12} and the above lemma, we have 

\begin{Prop}\label{Prop2.14}
Let $E$ be a torsion free sheaf on $S,\widetilde{E}$ the double dual
of $E$ and $M=\dfrac{\widetilde{E}}{E}$. Then we have 
$$
\dim \ext^{1}_{\mathscr{O}_S}\left(\widetilde{E},\widetilde{E}\right)+2\text{
length }(M)\leqq \dim\ext^{1}_{\mathscr{O}_S}(E,E).
$$
If equality holds, then the natural map
$\End_{\mathscr{O}_S}\left(\widetilde{E}\right)\oplus
\End_{\mathscr{O}_S}(M)\to \\\hom_{\mathscr{O}_S}\left(\widetilde{E},M\right)$
is surjective and $M$ is isomorphic to
$\dfrac{\mathscr{O}_S}{\mathscr{I}}$ for an ideal $\mathscr{I}$ of
$\mathscr{O}_S$. 
\end{Prop}

\begin{remark}\label{remark2.15}
Since $\widetilde{E}$ is locally free,
$\ext^{1}_{\mathscr{O}_S}\left(\widetilde{E},M\right)=\ext^{1}_{\mathscr{O}_S}\left(M,\widetilde{E}\right)=0$
for any surface $S$. Hence Corollary~\ref{cor2.12} and the above
proposition are true for any (smooth) surface. 

Let $0\to F\to E\to G\to 0$ be an exact sequence of nontorsion sheaves
on $S$. Since $\upsilon(E)=\upsilon(F)+\upsilon(G)$ and
$r(E)=r(F)+r(G)$, we have 
$$
\begin{aligned}
&\dfrac{\left(\upsilon(F)^{2}\right)}{r(F)}+\dfrac{\left(\upsilon(E)^{2}\right)}{r(G)}-\dfrac{\left(\upsilon(E)^{2}\right)}{r(E)}=\dfrac{r(F)r(G)}{r(E)}\left(\dfrac{\upsilon(F)}{r(F)}\dfrac{\upsilon(G)}{r(G)}\right)^{2}\\
&{}\text{ Since
} \dfrac{\upsilon(F)}{r(F)}-\dfrac{\upsilon(G)}{r(G)}=\left(0,\dfrac{c_1(F)}{r(F)}-\dfrac{c_1(G)}{r(G)},\dfrac{s(F)}{r(F)}-\dfrac{s(G)}{r(G)}\right), 
\end{aligned}
$$\pageoriginale
the right hand side of the above equality is equal to
$\dfrac{r(F)r(G)}{r(E)}\\\left(\dfrac{c_1(F)}{r(F)}-\dfrac{c_1(G)}{r(G)}\right)^{2}$. Hence
we have 
\end{remark}

\begin{Prop}\label{Prop2.16}
Let $0\to F\to E\to G\to 0$ be an exact sequence of nontorsion
sheaves. Then we have 
$$
\dfrac{\left(\upsilon(F)^{2}\right)}{r(F)}+\dfrac{\left(\upsilon(G)^{2}\right)}{r(G)}-\dfrac{\left(\upsilon(E)^{2}\right)
}{r(E)}=\dfrac{r(F)r(G)}{r(E)}\left(\dfrac{c_1(F)}{r(F)}-\dfrac{c_1(G)}{r(G)}\right)^{2}
$$
\end{Prop}

If $\rho(S)=1$, then the right hand side is always nonnegative
because we are assuming that $S$ is algebraic. Hence we have 

\begin{cor}\label{cor2.17}
If ($S$ is algebraic and) $\rho(S)=1$, then
$\dfrac{\left(\upsilon(F)^{2}\right)}{r(F)}+\dfrac{\left(\upsilon(G)^{2}\right)
}{r(G)}\geqq \dfrac{\left(\upsilon(E)^{2}\right)}{r(E)}$. Here equality
holds if and only if $\dfrac{c_1(F)}{r(F)}=\dfrac{c_1(G)}{r(G)}$.
\end{cor}

If $F$ and $G$ have the same slope with respect to an ample line
bundle $A$ i.e., $\mu_A(F)=\mu_A(G)$, then we have
$\left(A.\dfrac{c_1(F)}{r(F)}-\dfrac{c_1(G)}{r(G)}\right)=0$. Hence,
by the Hodge index theorem
$\left(\dfrac{c_1(F)}{r(F)}-\dfrac{c_1(G)}{r(G)}\right)^{2}$ is always
nonpositive and is equal to zero if and only if
$\dfrac{c_1(F)}{r(F)}=\dfrac{c_1(G)}{r(G)}$. Hence we have 

\begin{cor}\label{cor2.18}\pageoriginale
Assume that $F$ and $G$ have the same slope with respect to an ample
line bundle. Then we have 
$$
\dfrac{\left(\upsilon(F)^{2}\right)}{r(F)}+\dfrac{\left(\upsilon(G)^{2}\right)}{r(G)}\leqq \dfrac{\left(\upsilon(E)^{2}\right)}{r(E)}. 
$$
and equality holds if and only if
$\dfrac{c_1(F)}{r(F)}=\dfrac{c_1(G)}{r(G)}$. 
\end{cor}

Let $E$ be a $\mu$-semi-stable sheaf. Then there is a filtration 
$$
E_{\ast}:0=E_0\subset E_1\subset\ldots\subset E_n=E
$$
such that every successive quotient $F_i=\dfrac{E_i}{E_{i-1}}$ is
$\mu$-stable and has the same slope as $E$. Such a filtration
$E_{\ast}$ is called a $\mu-JHS$ filtration of $E$. Applying the above
corollary repeatedly for this filtration, we have the following: 

\begin{Prop}\label{Prop2.19}
Let $E$ be a $\mu$-semi-stable sheaf and $F_i(1\leqq i\leqq n)$ the
successive quotients of a $\mu$-JHS filtration of $E$. Then we have 
$$
\sum\limits_{i=1}^{n}\dfrac{\left(\upsilon(F_i)^{2}\right)}{r(F_i)}\leqq \dfrac{\left(\upsilon(E)^{2}\right)}{r(E)}
$$ 
Equality holds if and only if $\dfrac{c_1(F_i)}{r(F_i)}$ is equal to
$\dfrac{c_1(E)}{r(E)}$ for every $1\leqq i\leqq n$. 
\end{Prop}

\begin{remark}\label{remark2.20}
If $E$ is a semi-stable sheaf. Then there is a filtration
$$
0=E_0\subset E_1\subset\ldots \subset E_n=E
$$\pageoriginale
such that $F_i$ is stable, has the same slope as $E$ and
$\dfrac{s(F_i)}{r(F_i)}=\dfrac{s(E)}{r(E)}$ for every $i=1,\ldots,n$.
Such a filtration is called a $JHS$ filtration of $E$. The above
proposition is also true for a semi-stable sheaf $E$ and its $JHS$
filtration.

Now we assume that $S$ is a $K3$ surface and prove a result which we
shall need in \S\ 5. Let $F$ be a sheaf on $S$ which satisfies 
\end{remark}

\setcounter{equation}{20}
\begin{equation}\label{eqn2.21}
\begin{aligned}
&\text{ the canonical homomorphism }
f:H^{0}(S,F)\otimes \mathscr{O}_S\to F\\ 
&{}\text{ is injective and }
H^{2}(S,F)=0.
\end{aligned}
\end{equation}

We construct a sheaf $E$ on $S$ from $F$, which we call
the \textit{reflection} of $E$ (from the left), such that
$r(E)=-s(F)$, $c_1(E)=c_1(F)$ and $s(E)=-r(F)$. We show that $E$ is
simple if and only if $F$ is so. This result is a very special case of
the theory of the reflection functor of $S$, which we will discuss
systematically in \cite{key14}.


Let $\overline{F}$ be the cokernel of the canonical homomorphism
$$
f:H^{0}(S,F)\otimes \mathscr{O}_S\to F.
$$ 
We have the exact sequence 
\begin{equation}\label{eqn2.22}
0\to H^{0}(S,F)\otimes \mathscr{O}_S\xrightarrow{f}
F\to \overline{F}\to 0.
\end{equation}
Since $H^{1}(S,\mathscr{O}_S)=H^{2}(S,F)=0$, the above sequence
induces the exact sequence
\begin{equation}\label{eqn2.23}
\xymatrix@=.55cm{0\ar[r]&H^{1}(S,F)\ar[r]^{\alpha}&H^{1}\left(S,\overline{F}\right)\ar[r]&
H^{0}(S,F)\otimes\ar@{=}[d]^{\wr} H^{2}(S,\mathscr{O}_S)\ar[r]&0.\\
&&&H^{0}(S,F)&}
\end{equation}\pageoriginale
Construct an exact sequence
\begin{equation}\label{eqn2.24}
0\to \overline{F}\to E\to H^{1}(S,F)\to
H^{1}(S,F)\otimes \mathscr{O}_S\to 0
\end{equation}
so that the coboundary map $\delta:H^{1}(S,F)\otimes
H^{0}(S,\mathscr{O}_S)\to H^{1}(S,\overline{F})$ is equal to
$\alpha$. We call this extension $E$ of
$H^{1}(S,F)\otimes \mathscr{O}_S$ be $\overline{F}$ the reflection of
$F$ (from the left). Since $H^{2}(S,F)=0$ by our assumption,
$\mathcal{X}(F)$ is equal to $h^{0}(F)-h^{1}(F)$. Hence we have.
$$
\begin{aligned}
\upsilon(E)&=\upsilon(F)+h^{1}(F)\upsilon(\mathscr{O}_S)\\
&{}=\upsilon(F)-h^{0}(F)\upsilon(\mathscr{O}_S)+h^{1}(F)\upsilon(\mathscr{O}_S)\\
&{}=\upsilon(F)-\mathcal{X}(F)\upsilon(\mathscr{O}_S).
\end{aligned}
$$
Since $\mathcal{X}(F)=r(F)+s(F)$ and
$\upsilon(\mathscr{O}_S)=(1,0,1)$, we have $r(E)=-s(F),c_1(E)=c_1(F)$
and $s(E)=-r(F)$. (By our assumption, $\mathcal{X}(F)\leqq
h^{0}(F)\leqq r(F)$. Hence $s(F)$ is nonpositive.)

\setcounter{dfn}{24}
\begin{Prop}\label{Prop2.25}
Assume that $F$ satisfies (\ref{eqn2.21}) and let $E$ be the
reflection of $F$. Then we have $\End_{\mathscr{O}_S}(E)\cong
\End_{\mathscr{O}_S}(F)$. 
\end{Prop}

\begin{Proof}
We have constructed $E$ canonically from $F$. It is almost clear that
every endomorphism of $F$ induces an endomorphism of $E$. Let
$\varphi$ be an endomorphism of $E$. We show that $\varphi$ is induced
by an endomorphism of $F$. Since
$\hom_{\mathscr{O}_S}(F,\mathscr{O}_S)=0$ by our assumption and the
Serre duality, we have
$\hom_{\mathscr{O}_S}(F,\mathscr{O}_S)=0$. Hence $\varphi$ preserves
the exact sequence \eqref{eqn2.24} and induces an endomorphism
$\overline{\psi}$ of $\overline{F}$ and $f_1$ of $H^{1}(S,F)$. Since
$\overline{\psi}$ and $f_1$ are induced by $\varphi$, the following
diagram  
$$
\xymatrix{H^{1}(S,F)\otimes\ar[d]_{f_1} H^{0}(S,\mathscr{O}_S)\ar[r]^-{\delta=\alpha}&H^{1}\left(S,\overline{F}\right)\ar[d]^{H^{1}\left(\overline{\psi}\right)}\\
H^{1}(S,F)\otimes H^{0}(S,\mathscr{O}_S)\ar[r]_-{\delta=\alpha}&H^{1}\left(S,\overline{F}\right)}
$$\pageoriginale
is commutative. Hence $f_1$ preserves the exact sequence
\eqref{eqn2.23} and induces an endomorphism $f_0$ of
$H^{0}(S,F)$. From the long exact sequence
$\ext^{\ast}_{\mathscr{O}_S}$ (\eqref{eqn2.22}, $\mathscr{O}_S$), we
obtain the exact sequence 
$$
0\to
H^{0}(S,F)^{\vee}\xrightarrow{\delta'}\ext^{1}_{\mathscr{O}_S}\left(\overline{F},\mathscr{O}_S\right)\to \ext^{1}_{\mathscr{O}_S
}(F,\mathscr{O}_S)\to 0. 
$$
This sequence is the dual of the exact sequence \eqref{eqn2.22} via
the Serre duality. Hence we have the following commutative diagram:
$$
\xymatrix{H^{0}(S,F)^{\vee}\ar[d]^{f^{\vee}_0}\ar[r]^{\delta}&\ext^{1}_{\mathscr{O}_S}\left(\overline{F},\mathscr{O}_S\right)\ar[d]^{\ext^{1}_{\mathscr{O}_S}\left(\overline{\psi},\mathscr{O}_{S}\right)}\\
H^{0}(S,F)^{\vee}\ar[r]^{\delta'}&\ext^{1}_{\mathscr{O}_S}\left(\overline{F},\mathscr{O}_S\right)}
$$
Therefore, there exists an endomorphism $\psi$ of $F$ which Preserves
the exact sequence \eqref{eqn2.22} and induces $\overline{\psi}$ on
$F$ and $f_0$ on $H^{0}(S,F)$. By our construction, this $\psi$
induces $\varphi$. 
\end{Proof}

For our requirements in $\S 4$, we show a vanishing of higher direct
image sheaf $R^{i}f_{\ast}F$, which was essentially proved in \cite{key15}.

\begin{Prop}\label{Prop2.26}
Let $f:X\to Y$ be a proper morphism of noetherian\pageoriginale schemes and $F$ a
$Y$-flat coherent $\mathscr{O}_X$ -module. Let $Z$ be a closed
subscheme which is locally complete intersection in $Y$. For $y\in Y$,
let $F_Y$ be the restriction of $F$ to the fibre
$X_y=f^{-1}(y)$. Assume that $H^{i}(X_y,F_y)$ vanishes for every $i<$
codim $Z$ and $y\in Y-Z$. Then $R^{i}f_{\ast}F=0$ for every $i<$ codim
$Z$. 
\end{Prop}

\begin{Proof}
We may assume that $Y=\spec A$ is affine and $Z$ is defined by a
regular sequence $x_1,\ldots x_n\in A$, $n=$ codim $Z$. By the theorem
in \S\ 5 \cite{key15}, there exists a finite complex $K^{\bigdot}$ of
finitely generated projective $A$-modules such that
$H^{i}(K^{\bigdot})\cong R^{i}f_{\ast}F$. By the base change theorem and
by our assumption $R^{i}f_{\ast}F$ has a support on $Z$ for every
$i<n$. Hence there exists an integer $N$ such that
$\mathfrak{a}^{N}H^{i}(K^{\bigdot})=0$ for every $i<N$, where
$\mathfrak{a}=(x_1,\ldots,x_n)A$. Our proposition follows from the
following: 
\end{Proof}

\begin{lem}
\textit{Let $K$. be a finite complex of finitely generated projective
$A$-module and $\mathfrak{a}$ an ideal of $A$ generated by a regular
sequence $x_1\ldots x_n$ of $A$. If
$\mathfrak{a}^{N}H^{i}(K^{\bigdot})=0$ for every $i<n$, then
$H^{i}(K^{\bigdot})=0$ for every $i<n$.}
\end{lem}

This can be proved in the same way as the lemma in (\cite[p.127]{key15})
by using induction on $n$.
\enprf

\section{Semi-rigid sheaf}\label{s3}

In this section, we shall study sheaves $E$ on a $K3$ surface $S$ with\pageoriginale
small $\ext^{1}_{\mathscr{O}_S}(E,E)$. 

\begin{dfn}\label{dfn3.1}
A sheaf $E$ on $S$ is \textit{rigid} if
$\ext^{1}_{\mathscr{O}_S}(E,E)=0$. By Proposition 2.5, we
have 
\end{dfn}

\begin{Prop}\label{Prop3.2}
If $E$ is simple, then the following are equivalent:
\begin{enumerate}
\renewcommand{\labelenumi}{(\theenumi)}
\item $E$ is rigid,
\item $\left(\upsilon(E)^{2}\right)=-2$, and 
\item $\left(\upsilon(E)^{2}\right)<0$.
\end{enumerate}
\end{Prop}

By Proposition~\ref{Prop2.14}, we have 

\begin{Prop}\label{Prop3.3}
If $E$ is rigid and torsion free, then $E$ is locally free.
\end{Prop}

If $E$ is a rigid sheaf and if $\upsilon(F)=a\upsilon(E)$ for a
rational number $a$, then $\mathcal{X}(E,F)$ is equal to
$a\mathcal{X}(E,E)$ and is positive. Hence we have 

\begin{Prop}\label{Prop3.4}
Let $E$ be a rigid sheaf and $F$ a sheaf with
$\upsilon(F)=a\upsilon(E), a\in \mathbb{Q}$. Then either
$\hom_{\mathscr{O}_S}(E,F)\neq 0$ or $\hom_{\mathscr{O}_S}(F,E)\neq
0$. 
\end{Prop}

If $E$ is stable and $F$ is semi-stable and if
$\upsilon(E)=\upsilon(F)$, then every\pageoriginale nonzero homomorphism between $E$
and $F$ is an isomorphism. Hence we have 

\begin{cor}\label{cor3.5}
Let $E$ be a stable rigid bundle. If $F$ is semi-stable and
$\upsilon(F)=\upsilon(E)$, then $F$ is isomorphic to $E$. 
\end{cor}

\begin{cor}\label{cor3.6}
Let $\upsilon$ be a vector of $\widetilde{H}^{1,1}(S,\mathbb{Z})$ with
$\left(\upsilon^{2}\right)=-2$. Then the moduli space $M_A(\upsilon)$
is empty or a reduced one point. 
\end{cor}

\begin{Proof}
By Corollary~\ref{cor3.5}, if $M_A(\upsilon)$ is nonempty, then
$M_A(\upsilon)$ is one point. The tangent space of $M_A(\upsilon)$ at
the point $[E]\in M_A(\upsilon)$ is canonically isomorphic to
$\ext^{1}_{\mathscr{O}_S}(E,E)=0$. Hence $M_A(\upsilon)$ is reduced.
\enprf
\end{Proof}

$\dim\ext^{1}_{\mathscr{O}_S}(E,E)$ is always an even integer
(Corollary~\ref{cor2.5}). Hence if $\ext^{1}_{\mathscr{O}_S}(E,E)\neq
0$, then $\dim\ext^{1}_{\mathscr{O}_S}(E,E)\geqq 2$.

\begin{dfn}\label{dfn3.7}
A simple sheaf $E$ on $S$ is semi-rigid if $E$ satisfies the following
equivalent conditions: 
\begin{enumerate}
\renewcommand{\labelenumi}{(\theenumi)}
\item $\dim\ext^{1}_{\mathscr{O}_S}(E,E)=2$, and 
\item $\upsilon(E)\in \widetilde{H}^{1,1}(S,\mathbb{Z})$ is isotropic,
i.e. $\left(\upsilon(E)^{2}\right)=0$. 
\end{enumerate}
\end{dfn}

Proposition~\ref{Prop3.3} is not true for semi-rigid sheaf. In fact,
there\pageoriginale is a semi-rigid torsion free sheaf which is not locally
free. The simplest example is a maximal ideal $\mathfrak{m}$ of
$\mathscr{O}_S$. We can construct many such semi-rigid sheaves from a
rigid bundle. Let $F$ be a simple rigid vector bundle of rank
$r$. Take a point $s\in S$ and put $V=F\otimes k(s)$ and
$\widetilde{F}=F\otimes_k V^{\vee}\cdot \widetilde{F}$ is a rigid
bundle of rank $r^{2}$ and $\widetilde{F}\otimes k(s)$ is isomorphic to
$\End(V)$. Let $E$ be the kernel of the homomorphism
$f:\widetilde{F}\to \widetilde{F}\otimes k(s)\cong
\End(V)\xrightarrow{tr}k(s)$, where $tr$ is the trace map of
$\End(V)$. Every endomorphism of $E$ is induced by an endomorphism
$\alpha$ of $\widetilde{F}$. Since $\alpha$ preserves $f,\alpha$ is a
constant multiplication and hence $E$ is simple. It is easy to check
that $\upsilon(E)$ is isotropic. We call this $E$
the \textit{semi-rigid sheaf associated to $F$}. We have proved the
following:             

\begin{Prop}\label{Prop3.8}
Let $F$ be a simple rigid bundle of rank $r$. Then, for every point
$s\in S$, there exists a semi-rigid sheaf $E$ of rank $r^{2}$ and an
exact sequence 
$$
0\to E\to F^{\oplus r}\to k(s)\to 0.
$$
\end{Prop}

The above examples of semi-rigid torsion free sheaves are locally free
except at one point. This is true in general. In fact, by
Proposition~\ref{Prop2.14}, we have 

\begin{Prop}\label{Prop3.9}
Let $E$ be a torsion free sheaf with
$\dim \ext^{1}_{\mathscr{O}_S}(E,E)=2$. Let $\widetilde{E}$ be the
double dual of $E$ and assume that\pageoriginale $E$ is not locally free. Then the
quotient $\dfrac{\widetilde{E}}{E}$ is isomorphic to $k(s)$ for a
point $s\in S$. Moreover, $E$ is a rigid vector bundle and the natural
homomorphism
$\alpha:\End_{\mathscr{O}_S}\left(\widetilde{E}\right)\to \hom_{\mathscr{O}_S}\left(\widetilde{E},
k(s)\right)$ induced by the exact sequence $0\to E\to \widetilde{E}\to
k(s)\to 0$ is surjective.
\end{Prop}

\begin{cor}\label{cor3.10}
Let $E$ be a $\mu$-stable semi-rigid sheaf. If $E$ is not locally
free, then $r(E)=1$ and $E$ is isomorphic to $L\otimes\mathfrak{m}$
for a line bundle $L$ and a maximal ideal $\mathfrak{m}$ of
$\mathscr{O}_S$. 
\end{cor}

\begin{Proof}
Since $E$ is $\mu$-stable, so is $E$. Hence $E$ is simple. Since
$\alpha$ is surjective and $\dim
\End_{\mathscr{O}_S}\left(\widetilde{E}\right)=1$, we have
$\dim \hom_{\mathscr{O}_S}\left(\widetilde{E},k(s)\leqq 1\right)$. Therefore,
$\widetilde{E}$ is a line bundle.
\enprf
\end{Proof}

\begin{remark}\label{remark3.11}
If $F$ is a stable rigid bundle, then the semi-rigid sheaf $E$
associated to $F$ is stable. Hence the above corollary is not true for
stable semi-rigid sheaves.

If $E$ is semi-rigid and $\upsilon(F)=\upsilon(E)$, then
$\mathcal{X}(E,F)=-(\upsilon(E).\upsilon(F))=0$. Hence, if
$\hom_{\mathscr{O}_S}(E,F)=\hom_{\mathscr{O}_S}(F,E)=0$, then we have
$\ext^{1}_{\mathscr{O}_S}\\(E,F)=0$. 
\end{remark}

\begin{Prop}\label{Prop3.12}
Let $E$ be a stable semi-rigid sheaf and $F$ a semi-stable sheaf with
$\upsilon(F)=(\upsilon(E))$. If $E$ is not isomorphic to $F$, then
$\ext^{i}_{\mathscr{O}_S}(E,F)$ and $\ext^{i}_{\mathscr{O}_S}(F,E)$
vanish for every $i$. 
\end{Prop}

\begin{Proof}
By\pageoriginale the assumption of (semi-) stablility of $E$ and $F$, every
homomorphism between $E$ and $F$ is either zero or an
isomorphism. Hence, if $E\notcong F$, then
$\hom_{\mathscr{O}_S}(E,F)=\hom_{\mathscr{O}_S}(F,E)=0$. Since
$\mathcal{X}(E,F)=\mathcal{X}(F,E)=0$, we have our assertion by
Proposition~\ref{Prop2.4}
\enprf
\end{Proof}

If $M_A(\upsilon) \neq \phi$, then $M_A(a\upsilon)$ is empty for every
$a\neq 1$, In fact, we have 

\begin{Prop}\label{Prop3.13}
Let $E$ be a stable semi-rigid sheaf and $F$ a simple semi-stable with
$\upsilon(F)=a\upsilon(E)$, $\in \mathbb{Q}$. Then every nonzero
homomorphism between $E$ and $F$ is an isomorphism. 
\end{Prop}

\begin{Proof}
Let $f:E\to F$ be a nonzero homomorphism. Then $f$ is injective and
the cokernel of $F$ is semi-stable by our assumption on (semi-)
stability of $E$ and $F$.
\end{Proof}

\begin{claim}
$F$ is $E$-potent, i.e., has a filtration $0=F_0\subset
F_1\subset\ldots \subset F_n=F$  such that $\dfrac{F_i}{F_{i-1}}\cong
E$ for every $i=1,\ldots,n$.

We define $F_1=\Im(f)$ and $F_i$ inductively for $i\geqq 2$. Assume
that $F_i$ has been defined and $F_i\neq F$. Let $G_i$ be the quotient
$\dfrac{F}{F_i}$, Since $E$ is simple,
$\hom_{\mathscr{O}_S}(G_i,E)=0$. Since $G_i\neq 0$ and $F$ is simple,
the exact sequence $0\to F_i\to F\to G_i\to 0$ does not split. Hence
$\ext^{1}_{\mathscr{O}_S}(G_i,F_i)\neq 0$. Since $F_i$ is $E$-potent
and $\ext^{1}_{\mathscr{O}_S}$ is\pageoriginale an additive functor, we have
$\ext^{1}_{\mathscr{O}_S}(G_i,E)\neq 0$. Since
$\mathcal{X}(G_i,E)=-(\upsilon(G_i).\upsilon(E))=(i-a)\left(\upsilon(E)^{2}\right)=0$,
we have
$\dim\hom_{\mathscr{O}_S}(E,G_i)=\dim \ext^{1}_{\mathscr{O}_S}(G_i,E)-\dim\hom_{\mathscr{O}_S}(G_i,E)>0$. Hence
there exists a nonzero homomorphism $f_i:E\to G_i$. Let $F_{i+1}$ be
the pull-back of $\Im(f_i)$ by $F\to\to G_i$. Since $G_i$ is
semi-stable, $f_i$ is injective and $\dfrac{F_{i+1}}{F_i}$ is
isomorphic to $E$. So $F_{i+1}$ is well defined.

If $g:F\to E$ is a nonzero homomorphism, then $g$ is surjective. By the
same argument, we have our claim in this case. Since $F$ is simple,
$F$ is isomorphic to $E$ by the above and $f$ and $g$ are
isomorphisms.
\enprf
\end{claim}

Next we investigate the stability of semi-rigid sheaves.

\begin{Prop}\label{Prop3.14}
Let $S$ be an algebraic $K3$ surface with Picard number $1$ and $E$ a
simple torsion free sheaf on $S$. Assume that $E$ is rigid or
semi-rigid and that $\upsilon(E)$ is primitive in
$\widetilde{H}^{1,1}(S,\mathbb{Z})$. Then $E$ is stable.
\end{Prop}

\begin{Proof}
Since $\rho(S)=1$ and $\upsilon=\upsilon(E)$ is primitive, every semi-stable
sheaf $E'$ with $V(E')=\upsilon$ is stable. Hence it suffices to
show that $E$ is semi-stable. Assume that $E$ is not so. Let $F_1$ be
the $\beta$-subsheaf of $E$, i.e., $F_1$ maximizes the polynomial
$\dfrac{X(F_1(n))}{r(F_1)}$\pageoriginale among all subsheaves  of $E$ and then
maximizes $r(F_1)$ among such subsheaves. The quotient
$F_2=\dfrac{E}{F_1}$ is torsion free and
$\hom_{\mathscr{O}_S}(F_1,F_2)=0$ by our choice of $F_1$. Hence, by
Corollary~\ref{cor2.8}, we have 
$(\ast)
\dim \ext^{1}_{\mathscr{O}_S}(F_1,F_1)+\dim \ext^{1}_{\mathscr{O}_S}(F_2,F_2)\leqq
\dim \ext^{1}_{\mathscr{O}_S}(E,E)$. 
Since $\dim \ext^{1}_{\mathscr{O}_S}(E,E)=\left(\upsilon(E)^{2}\right)+2\leqq 2$,
we have $\dim \ext^{1}_{\mathscr{O}_S}(F_i,F_i)\leqq 2$ for both $i=1$
and $2$. Hence
$$
\left(\upsilon\left(F_i\right)^{2}\right)=\dim \ext^{1}_{\mathscr{O}_S}(F_i,F_i)-2\dim\End_{\mathscr{O}_S}(F_i)\leqq 0
$$ 
for both $i=1$ and $2$. Since
$r(F_i)<r(E)$, we have, by Corollary~\ref{cor2.17}, 
$$
\left(\upsilon(F_1)^{2}\right)+\left(\upsilon(F_2)^{2}\right)\geqq \left(\upsilon(E)^{2}\right).
$$
Hence we have 
$$
\begin{aligned}
&\dim \ext^{1}_{\mathscr{O}_S}(F_1,F_1)+\dim\ext^{1}_{\mathscr{O}_S}(F_2,F_2)\\
&{}\geqq \dim\ext^{1}_{\mathscr{O}_S}(E,E)+2\dim
\End_{\mathscr{O}_S}(F_1)\\
&{}+2\dim \End_{\mathscr{O}_S}(F_2)-2\\
&{}>\dim \ext^{1}_{\mathscr{O}_S}(E,E),
\end{aligned}
$$
Which contradicts ($\ast$).
\enprf
\end{Proof}

\begin{remark}\label{remark3.15}
If $F$ is a rigid bundle of rank $\geqq 2$, then the semi-rigid sheaf
$E$ associated to $F$ is not $\mu$-stable. Hence, even if\pageoriginale $\rho(S)=1$,
it is not always true that simple semi-rigid torsion free sheaf is
$\mu$-stable.

In the following two propositions, we consider the case where $c_1(E)$
is ample and study the stablility of $E$ with respect to $c_1(E)$. 
\end{remark}

\begin{Prop}\label{Prop3.16}
Let $E$ be a semi-rigid sheaf with $\upsilon(E)=(r,\ell, s)$. Assume
that $\ell$ is ample and $E$ is stable with respect to $\ell$. If $s$
is divisible by $r$ and $\upsilon(E)$ is primitive, then $E$ is
$\mu$-stable with respect to $\ell$. 
\end{Prop}

\begin{Proof}
Assume that $E$ is not $\mu$-stable. Then $E$ has a proper quotient
sheaf $E_1$ with $\mu(E_1)=\mu(E)$. We choose $E_1$ so that $r(E_1)$
is minimum among such quotients. Put
$\upsilon(E_1)=(r_1,\ell_1,s_1)$. Since $\mu(E_1)=\mu(E)$, we have
$(\ell\cdot \ell_1-r_1\ell/r)=0$. Since $E$ is semi-rigid, we have
$\ell^{2}=2rs$. Therefore, we have 
$$
\begin{aligned}
\left(\upsilon(E_1)^{2}\right)&=\left(\left(\ell_1-r_1\dfrac{\ell}{r}\right)+r_1\dfrac{\ell}{r}\right)^{2}-2r_1s_1\\
&{}=\left(\ell_1-r_1\dfrac{\ell}{r}\right)^{2}+\left(r_1\dfrac{\ell}{r}\right)^{2}-2r_1s_1\\
&{}=\left(\ell_1-r_1\dfrac{\ell}{r}\right)^{2}+2r_1\left(\dfrac{r_1s}{r-s_1}\right).
\end{aligned}
$$
Since $\upsilon(E)$ is primitive, $r$ and $\ell$ are coprime. Hence
$\ell_1-r_1\dfrac{\ell}{r}$ is not zero. Since
$\left(\ell_1-r_1\dfrac{\ell}{r.\ell}\right)=0$ and  $\ell$ is ample,
$\left(\ell_1-r_1\dfrac{\ell}{r}\right)^{2}$ is\pageoriginale negative by the Hodge
index theorem. On the other hand, since $E$ is stable, the integer
$\dfrac{r_1s}{r-s_1}$ is negative. Therefore, we have
$(\upsilon(E_1)^{2})<-2r_1\leqq -2$, which contradicts
Corollary~\ref{cor2.5} because $E_1$ is $\mu$-stable and simple by our
choice. 
\enprf
\end{Proof}

\begin{Prop}\label{Prop3.17}
Let $\upsilon=(r,\ell,s)$ be a primitive isotropic vector of
$\widetilde{H}^{1,1}(S,\mathbb{Z})$ and $E$ a sheaf with
$\upsilon(E)=\upsilon$. Assume that $\ell$ is ample and $E$ is
semi-stable but not stable with respect to $\ell$. Let 
$$
0=E_0\subset E_1\subset\ldots \subset E_n=E, n\geqq 2
$$
be a JHS-filtration of $E$. Then the successive quotients
$F_i=\dfrac{E_i}{E_{i-1}}$ are rigid for every $i=1,\ldots,n$. 
\end{Prop}

\begin{Proof}
By Proposition~\ref{Prop2.19} and Remark~\ref{remark2.20}, we have
$\left(\upsilon (F_i)^{2}\right)\leqq 0$ for every $i$. Since
$\upsilon$ is primitive, equality is not attained for any $i$. Hence
$F_i$ is rigid by Proposition~\ref{Prop3.2}
\enprf
\end{Proof}

\begin{cor}\label{cor3.18}
Let $\upsilon$ be as above. Then the complement of
$M_{\ell}(\upsilon)$ in the moduli space
$\overline{M}_{\ell}(\upsilon)$ of semi-stable sheaves $E$ with
$\upsilon(E)=\upsilon$ is a $0$-dimensional set. 
\end{cor}


\section{Surface components of the moduli space}\label{s4}\pageoriginale

Let $\upsilon=(r,\ell,s)$ be an isotropic vector of
$\widetilde{H}^{1,1}(S,\mathbb{Z})$ and $A$ an ample line bundle. Then
each component of $M_A(\upsilon)$ has dimension 2. In this section, we
study $M_A(\upsilon)$ in the case it is compact and we prove
Theorem~\ref{Theorem1.4} and Theorem~\ref{Theorem1.5}. By Langton's
result \cite{key6} (see also \cite[\S\ 5]{key9}), the moduli space of
semi-stable sheaves on $S$ is compact. Hence we have 


\begin{Prop}\label{Prop4.1}
$M_A(\upsilon)$ is compact if and only if every semi-stable sheaf $E$
with $\upsilon(E)=\upsilon$ is stable. This is the case, e.g., if the
greatest common divisor of $r$, $(\ell. A)$ and $s$ is equal to $1$.
\end{Prop}

The above is true for every vector $\upsilon$. Using this proposition,
we give some further sufficient conditions for $M_A(\upsilon)$ to be
compact for given primitive isotropic vector $\upsilon$. Let $c$ be
the greatest common divisor of $r$, $(\ell, m)$ and $s$, where $m$
runs over all divisor class of $S$. Then there exists an ample line
bundle $A$ such that the greatest common divisor of $r,(A,\ell)$ and
$s$ is equal to $c$. Hence if $c=1$, then $M_A(\upsilon)$ is compact
for such an ample line bundle $A$. For an application in \S\ 6, we
consider the case $c\geqq 2$. We show that $M_A(\upsilon)$ is compact
for an ample line bundle $A$ in this case, too. Let $N_S$ be the
Neron-Severi group of $S$. $N_S$ is a sublattice of
$H^{2}(S,\mathbb{Z})$. Let $N$ be the sub module generated\pageoriginale by $N_S$
and $\dfrac{\ell}{c}$ in $N_S\otimes \mathbb{Q}$. Since
$\left(\ell^{2}\right)=2rs,\left(\ell^{2}\right)$ is divisible by
$2c^{2}$. By the definition of $c$, the bilinear form on
$N_S\otimes \mathbb{Q}$ is integral and even on $N$. Hence $N$ is an
even lattice which contains $N_S$ as a sublattice of index $c$. 

\begin{Prop}\label{Prop4.2}
Let $A$ be an ample line bundle  on $S$ such that G.C.D. $(r,
(A.\ell), s)=c$. If there are no $(-2)$ vectors $\alpha$ in $N$ with
$(A.\alpha)=0$, then $M_A(V)$ is compact.
\end{Prop}

\begin{Proof}
Let $E$ be a semi-stable sheaf with $\upsilon(E)=\upsilon$ and 
$$
E_{\ast}:0=E_0\subset E_1\subset\ldots \subset E_n=E
$$
be a JHS-filtration of $E$ We show that $n=1$. Put
$F_i=\dfrac{E_i}{E_{i-1}}$ and $\upsilon(F_i)=(r_i,\ell_i,s_i)$ for
every $i=1,\ldots,n$. Since $E_{\ast}$ is a JHS-filtration, we have
$r_i:(A.\ell_i):s_i=r:(A.\ell):s$ for every $i$. There exists an
integer $a_i$ such that $r_i=a_i\dfrac{r}{c}$, $\dfrac{(A.\ell)}{c}$
and $s_i=\dfrac{a_is}{c}$. Put $m_i=\ell_i-\dfrac{a_i\ell}{c}\in
N$. Then we have $(A.m_i)=0$ and
$\left(\upsilon(F_i)^{2}\right)=\left(m_i+a_i\dfrac{\ell}{c}\right)^{2}-r_is_i=\left(m^{2}_i\right)+\dfrac{2a_i(m_i.\ell)}{c}$. Since
$\sum\limits_{i=1}^{m}m_i=0$, there exists an $i$ such that
$(m_i.\ell)\leqq 0$. For this $i$, we have
$\left(m^{2}_i\right)\geqq \left(\upsilon(F_i)^{2}\right)\geqq-2$. Since
$(A.\mathfrak{m}_i)=0$, $\left(m^{2}_i\right)$ is non-positive by the
Hodge index theorem. Hence by our assumption, we have
$\left(m^{2}_i\right)=0$ and $m_i=0$ by the Hodge index
theorem. Therefore,\pageoriginale we have
$\upsilon(F_i)=a_i\dfrac{\upsilon}{c}$. Since $\upsilon$ is primitive,
$\upsilon(F_i)$ is equal to $\upsilon$. Hence $E$ is stable.
\enprf
\end{Proof}

As an application of the above, we have the following proposition: 

\begin{Prop}\label{Prop4.3}
Assume that there exists a semi-rigid sheaf $E$ with
$\upsilon(E)=\upsilon$ which is $\mu$-stable with respect to an ample
line bundle $A'$. Then  there exists an ample line bundle $A$ such
that 
\begin{enumerate}
\renewcommand{\labelenumi}{(\theenumi)}
\item $E$ is $\mu$-stable with respect to $A$, and

\item $M_A(\upsilon)$ is nonempty and compact. 
\end{enumerate}
\end{Prop}

\begin{Proof}
There exists a neighbourhood $U$ of $A'$ in
$\mathbb{P}(N_S\otimes \mathbb{R})$ such that $E$ is $\mu$-stable with
respect to $A$ for every ample line bundle $A\in U$. Let
$\alpha_1,\ldots,\alpha_n$ be all the $(-2)$ vectors in $N$ which are
perpendicular to $A'$. If $A_1$ is an ample line bundle in
$U-\bigcup\limits^{n}_{i=1}\alpha^{\perp}_i$ and if $A_1$ is sufficiently
near $A'$, then $(A_1.\alpha)\neq 0$ for any $(-2)$ vector $\alpha$ in
$N$. Take such $A_1$ from
$U-\bigcup\limits_{i=1}^{n}\alpha^{\perp}_i$, and take an ample line
bundle $A_2$ such that G.C.D $(r,(A_2.\ell),s)=c$. If $n$ is
sufficiently large, then $A=n c A_1+A_2$ belongs to $U$ and satisfies
the last assumption of the preceding proposition. There are infinitely
many $n$'s such that G.C.D. $(r,(A.\ell),s)=c$.\pageoriginale Hence there exists
an integer $n$ such that $M_A(\upsilon)$ is compact and nonempty.
\enprf
\end{Proof}

If $M_A(\upsilon)$ is compact, then $M_A(\upsilon)$  is
irreducible. In fact, we have 

\begin{Prop}\label{Prop4.4}
Assume that $M_A(\upsilon)$ contains a connected component $M$ which
is compact and every member of $M$ is locally free. Then we have 
\begin{enumerate}
\renewcommand{\labelenumi}{(\theenumi)}
\item $M_A(\upsilon)$ is irreducible, and 
\item every semi-stable sheaf $E$ with $\upsilon(E)=\upsilon$ is
stable. 
\end{enumerate}
\end{Prop}

\begin{Proof}
Since $M_A(\upsilon)$ is smooth, $M$ is irreducible. We show that
every semi-stable sheaf $F$ with $\upsilon(F)=\upsilon$ belongs to
$M$. Let $\mathscr{E}$ be the restriction to $S\times M$ of a
quasi-universal family on $S\times M_A(\upsilon)$ (see Appendix
$2$). We consider the functor
$\Phi^{i}(F)=R^{i}\pi_{M},\ast\left(\mathscr{E}^{\vee}\otimes \pi_S^{\ast}F\right),
i=0,1$ and $2$, of $\mathscr{O}_S$-module $F$ into the category of
$\mathscr{O}_M$ -modules. If $F$ is semi-stable, then, for every
stable sheaf $E$ with $\upsilon(E)=\upsilon(F)$,
$H^{l}\left(S,E^{\vee}\otimes F\right)\neq 0$ is equivalent to $F\cong E$. Hence
if $F$ is semi-stable and $\upsilon(F)=\upsilon$, then $\Phi^{i}(F)$
is supported at most one point. Therefore, by
Proposition~\ref{Prop2.26}, we have $\Phi^{0}(F)=\Phi^{1}(F)=0$. Since
$\dim S=2$, $\Phi^{2}(F)$ is canonically isomorphic to
$H^{2}(S,E^{\vee}\otimes F)$ at the point $[E]$ of $M$, that is,
$\Phi^{2}(F)\otimes k([E])\cong H^{2}\left(S,E^{\vee}\otimes F\right)$. Hence
$\Phi^{2}(F)$ is nonzero if and only if $F$ is stable and belongs to
$M$. On the other hand, the cohomology class
$\alpha(F)=ch\left(\Phi^{0}(F)\right)-ch\left(\Phi^{1}(F)\right)+ch\left(\Phi^{2}(F)\right)
\in H^{\ast}(M,\mathbb{Q})$\pageoriginale does not depend on $F$ but depends only on
$\upsilon(F)$ by the Grothendieck-Riemann-Roch theorem. If $F$ belongs
to $M$, the $\alpha(F)$ is nonzero. Hence $\alpha(F)$ is nonzero for
every sheaf $F$ with $\upsilon(F)=\upsilon$. Therefore every
semi-stable sheaf $f$ with $\upsilon(F)=\upsilon$ is stable and
belongs to $M$, which proves $(1)$ and $(2)$.
\enprf
\end{Proof}

\begin{remark}\label{remark4.5}
In the above proposition, the assumption that every member of $M$ is
locally free is superfluous. The proof works without this assumption,
if one defines that functor $\Phi^{i}$ by
$\Phi^{i}(F)=\pi_M-\ext\left(\mathscr{E},\pi^{\ast}_SF\right)$, where
$\pi_M \ext(\ast,\ast)$  is the sheaf associated to the presheaf
assigning 
$$
\ext_{\mathscr{O}_{S\times U'}}(\ast\mid_{S\times
U'}\ast\mid_{S\times U})
$$ 
for every open subset $U$ of $M$. 
\end{remark}

\begin{cor}\label{cor4.6}
If every semi-stable sheaf $E$ with $\upsilon(E)=\upsilon$ is stable,
the $M_A(\upsilon)$ is compact and irreducible.
\end{cor}

We assume that the moduli space $M=M_A(\upsilon)$ is compact. Since
the canonical bundle of $M$ is trivial, (\cite[Corollary 0.2]{key12}),
$M_A(\upsilon)$ is abelian or of type $K3$. We first consider the case
where a universal family exists on $S\times M$. 

\begin{lemma}\label{lemma4.7}
For every sheaf $\mathscr{E}$ on $S\times M$, the Chern character
$ch(\mathscr{E})$ of $E$ is integral, i.e., belongs to
$H^{\ast}(S\times M,\mathbb{Z})$. 
\end{lemma}

\begin{Proof}
Put
$ch(\mathscr{E})=\sum\limits_{i=0}^{4}ch^{i}(\mathscr{E})\in \bigoplus\limits^{4}_{i=0}H^{2
i}(S\times M,\mathbb{Q}).ch^{1}(\mathscr{E})$ is\pageoriginale the first Chern class
$c_1(\mathscr{E})$ of $\mathscr{E}$ and is integral. Since
$H^{1}(S)=0$, $H^{2}(S\times M)$ is the direct sum of $H^{2}(S)$ and
$H^{2}(M)$. Hence $c_1(\mathscr{E})$ is equal to
$c_1,s(\mathscr{E})+c_1,M(\mathscr{E})\in H^{2}(S,\mathbb{Z})\oplus
H^{2}(M,\mathbb{Z})$. Since both $S$ and $M$ have trivial canonical
bundles, both $c_1,s\left(\mathscr{E}\right)^{2}$ and
$c_{1,M}\left(\mathscr{E}\right)^{2}$ are even. Hence
$ch^{2}(\mathscr{E})=\dfrac{1}{2}
c_1\left(\mathscr{E}\right)^{2}-c_2(\mathscr{E})$  is integral. By
the Grothendieck-Riemann-Roch theorem, the $H^{\ast}(S)\otimes
H^{4}(M)$ -component of $ch(\mathscr{E})\cdot t d_M$ is equal to
$\left(\sum\limits_{j}(-1)^{j}ch(R^{j}\pi_S,\ast\mathscr{E}\right)\otimes w$,
where $w \in H^{4}(M)$ is the fundamental cocycle  of $M$. Hence
$ch^{2}(\mathscr{E})$ and the $H^{2}(S)\otimes H^{4}(M)$-component of
$ch^{3}(\mathscr{E})$ are integral. Interchanging $S$ and $M$, we have
that the $H^{4}(S)\otimes H^{2}(M)$ component of $ch^{3}(\mathscr{E})$
is also integral. Since $H^{6}(S\times M)$ is the direct sum of
$H^{2}(S)\oplus H^{4}(M), ch^{3}(\mathscr{E})$ is integral.
\enprf
\end{Proof}

Let $\mathscr{E}$ be a universal family on $S\times M$. Put
$\mathbb{Z}=\pi_S\ast\sqrt{t d_S}
ch(\mathscr{E})^{\ast}\cdot \pi_M\ast\sqrt{td}_M$. By the lemma,
$\mathbb{Z}$ belongs to $H^{\ast}(S\times M,\mathbb{Z}).Z$ defines a
homomorphism
\setcounter{equation}{7}
\begin{equation}
\xymatrix{f:H^{\ast}(S,\mathbb{Z})\ar[r]& H^{\ast}(M,\mathbb{Z}).\\
\mbox{\rotatebox{90}{$\in$}} & \mbox{\rotatebox{90}{$\in$}}\\
\alpha\ar@{|->}[r]&\pi_M,\ast(\mathbb{Z}.\pi_S\ast \alpha)}
\end{equation}

\setcounter{dfn}{8}
\begin{Theorem}\label{Theorem4.9}
Under the above situation, we have 
\begin{enumerate}
\renewcommand{\labelenumi}{(\theenumi)}
\item $M$ is $K3$ surface, 

\item $f$\pageoriginale is an isometry form $\widetilde{H}(S,\mathbb{Z})$ onto
$\widetilde{H}(M,\mathbb{Z})$ with respect to the quadratic forms
defined in $(1.1)$, and 

\item the inverse of $f$ is equal to the homomorphism 
$$
\xymatrix{f':H^{\ast}(M,\mathbb{Z})\ar[r]&H^{\ast}(S,\mathbb{Z})\\
\mbox{\rotatebox{90}{$\in$}}&\mbox{\rotatebox{90}{$\in$}}\\
\beta\ar@{|->}[r]&\pi_S,\ast\left(Z'\cdot \pi^{\ast}_M\beta\right)}
$$
\end{enumerate}
defined by $Z'=\pi_S^{\ast}\sqrt{td_S}\cdot
ch(\mathscr{E})\cdot \pi_{M}^{\ast}\sqrt{td}_M$. 
\end{Theorem}

For the proof, the following is essential.

\begin{Prop}\label{Prop4.10}
Let $\mathscr{E}$ be a universal family on $S\times M$. Let $\pi_{12}$
and $\pi_{13}$ be the two projections of $S\times M\times M$ onto
$S\times M$. Then $\pi_{M\times
M}-\ext^{i}\left(\pi_{1,2}^{\ast}\mathscr{E},\pi^{\ast}_{13}\mathscr{E}\right)$
is zero if $i\neq 2$ and $\pi_{M\times
M}-\ext^{2}\left(\pi_{12}^{\ast}\mathscr{E},\pi^{\ast}_{13}\mathscr{E}\right)$ is supported on
the diagonal sub scheme  $\Delta$ of $M\times M$ and is a line bundle
on $\Delta$.
\end{Prop}

\begin{Proof}
If $E,F\in M_A(\upsilon)$ and $E\notcong F$, then
$\ext^{i}_{\mathscr{O}_S}(E,F)=0$ for every $i$ by
Proposition~\ref{Prop3.8}. Hence the relative Ext-sheaf $\pi_{M\times
M}-\ext^{i}\left(\pi_{12}^{\ast}\mathscr{E},\pi^{\ast}_{1
3}\mathscr{E}\right)$ has a support on $\Delta$. Since $\Delta$ is
locally complete intersection, the relative Ext-sheaf is zero for both
$i=1$ and $2$, by Proposition~\ref{Prop2.26}. By the base change
theorem, $\pi_{M\times M}-\ext^{2}\left(\pi^{\ast}_{1
2}\mathscr{E},\pi^{\ast}_{1 3}\mathscr{E}\right)$ is canonically
isomorphic to the $1$-dimensional vector space
$\ext^{2}_{\mathscr{O}_S}(E,E)\cong \End_{\mathscr{O}_S}(E)^{\vee}$
at the point $([E], [E])\in \Delta$. Since $M$ is a moduli space and
$\mathscr{E}$ is a universal family, the sheaf $\pi_{M\times
M}-\ext^{2}\left(\pi^{\ast}_{1 2}\mathscr{E},\pi^{\ast}_{1
3}\mathscr{E}\right)$ is annihilated by the ideal
$\mathscr{I}_{\Delta}$ of $\Delta$. Therefore,\pageoriginale $\pi_{M\times
M}-\ext^{2}\left(\pi^{*}_{1 2}E,\pi^{\ast}_{1 3} E\right)$ is a line
bundle on $\Delta$. 
\enprf
\end{Proof}

\begin{pf}
The following is the key to our proof. 
\begin{claim}
The endomorphism $f\circ f'$ of $H^{\ast}(M,\mathbb{Z})$ is the identity.
\end{claim}

The homomorphisms $f$ and $f'$ are given by cycles $Z$ and $Z'$ on
$S\times M$. Using the projection formula, it can be easily shown that
$f\circ f'$ is given by the cycle $\widetilde{Z}=\pi_{M\times
M},\ast\left(\pi^{\ast}_{12}Z\cdot \pi^{\ast}_{1 3} Z'\right)$, where
$\pi_{1 2}$ and $\pi_{1 3}$ are same as in the above
proposition. Precisely speaking, $(f\circ
f')(\beta)=\pi_{1,\ast}\left(\widetilde{Z}\cdot \pi^{\ast}_2\beta\right)$
for every $\beta \in H^{\ast}(M,\mathbb{Z})$, where $\pi_1$ and
$\pi_2$ are two projections of $M\times M$ onto $M$. By the definition
of $Z$ and $Z'$, we have $\widetilde{Z}=\left(\pi^{\ast}_1\sqrt{t
d_M}\right).\left(\pi_2^{\ast}\sqrt{td}_M\right)$ $\pi_{M\times
M},\ast (U)$, where $U=\dfrac{\left(\pi_{1
2}^{\ast}ch(\mathscr{E})^{\ast}\right)}{\pi_S^{\ast}td_S.\left(\pi^{\ast}_{1
3}ch\left(\mathscr{E}\right)\right)}$. By the Grothendieck-Riemann-Roch
theorem, the cycle $\pi_{M\times M}\ast(U)$ is rationally equivalent
to $\sum\limits_{i}(-1)^{i}ch\left(\pi_{M\times
M}\ext^{i}\left(\pi^{\ast}_{1
2}\mathscr{E},\pi_{13}\ast\mathscr{E}\right)\right)$. By the above
proposition, $\widetilde{Z}$ is rationally equivalent to
$\pi_1^{\ast}\sqrt{td}_M.ch(\delta_{\ast}L).\pi_2\ast\sqrt{td}_M$,
where $L$ is a line bundle $M$ and $\delta:M\to M\times M$ is the
diagonal embedding. Therefore, $f\circ f'$ is the multiplication by
$ch(L)\in H^{\ast}(M)$, i.e., $(f\circ f')(\beta)=\beta\cdot ch(L)$
for every $\beta\in H^{\ast}(M,\mathbb{Z})$. Let $\rho$ be the factor
change of $M\times M$. Then $(1\times \rho)^{\ast}U$ is equal to
$U^{\ast}$. Hence, we have $\rho^{\ast}(\pi_{M\times M},\ast
U)=(\pi_{M\times M},\ast U)^{\ast}$. On the other hand, since
$\pi_{M\times M},\ast U$ has a support on $\Delta$, we have
$\rho^{\ast}(\pi_{M\times M},\ast U)=\pi_{M\times M}\ast U$. Hence we
have $ch(\delta_{\ast}L)^{\ast}=ch(\delta_{\ast}L)$. Since $S$ is a
$K3$ surface, the line bundle $L$ is trivial. Therefore, $f\circ f'$
is the identity. 

By\pageoriginale the claim, $H^{\ast}(M,\mathbb{Z})$ is a direct
summand of $H^{\ast}(S,\mathbb{Z})$. Since $Z$ and $Z'$ belong to
$H^{e\upsilon}(S\times M,\mathbb{Z}),f$ and $f'$ preserve the
decompositions $H^{\ast}=H^{e\upsilon}\oplus H^{odd}$ of the
cohomology groups $H^{\ast}(M,\mathbb{Z})$ and
$H^{\ast}(S,\mathbb{Z})$. Hence $H^{odd}(M,\mathbb{Z})$ is a direct
summand of $H^{odd}(S,\mathbb{Z})$ which is zero, since $S$ is a $K3$
surface. Since $M$ has a trivial canonical bundle, we have, (1). By
(1), $H^{\ast}(M,\mathbb{Z})$ and $H^{\ast}(S,\mathbb{Z})$ have the
same rank $(=24)$. Therefore, $f$ is an isomorphism, which shows
(3). Let $\gamma=\gamma_S$:$S\to \spec \mathbb{C}$ be the structure
morphism of $S$. Then our inner product $(\alpha.\alpha')$ on
$\widetilde{H}(S,\mathbb{Z})=\widetilde{H}^{\ast}(S,\mathbb{Z})$ is equal to
$\gamma_{\ast}(\alpha^{\ast}.\alpha')$ Hence, by the projection
formula, we have 
$$
\begin{aligned}
(\alpha.f'(\beta))&=\gamma_{S,\ast}\left(\alpha^{\ast}\cdot \pi_{S,\ast}\left(\pi^{\ast}_S\sqrt{tdS}\cdot
ch(\mathscr{E})\cdot \pi^{\ast}_M\sqrt{td_M}\cdot \pi^{\ast}_M\beta\right)\right)\\
&{}=\gamma_S,\ast \pi_S,\ast\left(\pi^{\ast}_S\alpha^{\ast}\cdot \pi^{\ast}_S\sqrt{tdS}\cdot
ch(\mathscr{E})\cdot \pi^{\ast}_M\sqrt{td}_m\cdot \pi^{\ast}_M\beta\right)\\
&{}=\gamma_{S\times
M,\ast}\left(\pi^{\ast}_S\alpha^{\ast}\cdot \pi^{\ast}_M\beta\cdot
ch(\mathscr{E}).\sqrt{td_{s\times m}}\right).
\end{aligned}
$$
for every $\alpha\in H^{\ast}(S,Z)$ and $\beta \in H^{\ast}(M,Z)$. In a
similar way, we have 
$$
(\beta\cdot f(\alpha))=\gamma_{S\times
M,\ast}\left(\pi^{\ast}_M\beta^{\ast}\cdot \pi^{\ast}_S\alpha\cdot
ch(\xi)^{\ast}\cdot\sqrt{td_{S\times M}}\right).
$$
Therefore $(\alpha\cdot f'(\beta))=(f(\alpha).\beta)$ for every
$\alpha\in H^{\ast}(S\cdot \mathbb{Z})$ and $H^{\ast}(M,\mathbb{Z})$,
that is, $f$ and $f'$ are adjoint to each other with respect to the
inner products ($\cdot$) on $H^{\ast}(S,\mathbb{Z})$ and
$H^{\ast}(M,\mathbb{Z})$. By (3), $f'\circ f$ is the identity. Hence
we have $(f(\alpha)\cdot f(\alpha'))=(\alpha\cdot
f'(f(\alpha)))=(\alpha\cdot \alpha')$ for every $\alpha,\alpha'\in
H^{\ast}(S,\mathbb{Z})$, which proves (2).
\enprf                   
\end{pf}

Now\pageoriginale we assume only that $M=M_A(\upsilon)$ is compact and that
$\mathscr{E}$ is a quasi-universal family on $S\times M$ and prove
Theorem~\ref{Theorem1.4} and \ref{Theorem1.5}. Let
$\sigma(\mathscr{E})$ be the similitude of $\mathscr{E}$ and put
$Z=\pi^{\ast}_{S}\sqrt{td}_Sch(\mathscr{E}).\pi^{\ast}_M\dfrac{\sqrt{td_M}}{\sigma(\mathscr{E})}\in
H^{e\upsilon}(S\times M,\mathbb{Q}). Z$ induces a homomorphism
$$
\xymatrix{f:H^{\ast}(S,\mathbb{Q})\ar[r]&H^{e\upsilon}(M,\mathbb{Q})\\
\mbox{\rotatebox{90}{$\in$}} & \mbox{\rotatebox{90}{$\in$}}\\
\alpha\ar@{|->}[r]&\pi_M,\ast(Z.\pi^{\ast}_S\alpha)}
$$
The $H^{0}(M,\mathbb{Q})$ -component of $f(\alpha)$ is equal to
$(\upsilon.\alpha)$. Hence the orthogonal complement $\upsilon^{\perp}$
of $\upsilon$ in $H^{\ast}(S,\mathbb{Q})$ is sent into
$H^{2}(M,\mathbb{Q})\oplus H^{4}(M,\mathbb{Q})$ by $f$. 

\begin{lemma}\label{lemma4.11}
$f(v)$ is equal to the fundamental cocycle $\omega \in H^{4}(M,\mathbb{Z})$.
\end{lemma}

\begin{Proof}
Let $F$ be a member of $M=M_A(\upsilon)$ and let $\Phi^{2}(F)$ be same
as in the proof of Proposition~\ref{Prop4.4} and
Remark~\ref{remark4.5}. By the Grothendieck-Riemann-Roch theorem, we
have $ch(\Phi^{2}(F))=\pi_M,\ast
(ch(\mathscr{E})^{\ast}\cdot \pi^{\ast}_S(ch(F)\cdot
td_S))=\sigma(\mathscr{E})\sqrt{td_M}^{-1}\cdot
f(ch(F)\cdot \sqrt{td_S})=\sigma(\mathscr{E})\sqrt{td_M}^{-1}f(\upsilon)$. Now
$\Phi^{2}(F)$ has a support at the point $x\in M$ corresponding to $F$
and $\Phi^{2}(F)\otimes k(x)$ is canonically isomorphic to
$\ext^{2}_{\mathscr{O}_S}\left(\mathscr{E}_{S\times x},F\right)$. Since
$\mathscr{E}$ is a quasi-universal family, $\mathscr{E}\mid_{S\times
x}$ is isomorphic to $F^{\oplus \sigma (\mathscr{E})}$. Hence
$\Phi^{2}(F)\otimes k(x)$ is a $\sigma(\mathscr{E})$ dimensional
vector space. On the other hand, since $M$ is the moduli space and
$\mathscr{E}$ is a quasi-universal family, $\Phi^{2}(F)$ is
annihilated by the maximal ideal at $x$. Hence $\Phi^{2}(F)$ is
isomorphic\pageoriginale to $k(x)^{\oplus} \sigma(\mathscr{E})$ and
$ch(\Phi^{2}(F))=\sigma(\mathscr{E})\omega$, which proves our lemma.
\enprf
\end{Proof}

By this lemma, we see that $f$ induces a homomorphism
$$
\varphi_{\mathbb{Q}}:\dfrac{(\upsilon^{\perp} \text{ in }
H^{\ast}(S,\mathbb{Q}))}{\mathbb{Q}\upsilon} \to H^{2}(M,\mathbb{Q}).
$$
Proof of Theorem~\ref{Theorem1.4} and Theorem~\ref{Theorem1.5} : If
$\mathscr{E}$ is a quasi-universal family on $S\times M$. then so is
$\mathscr{E}\otimes \pi^{\ast}_M V$ for every vector bundle $V$ on
$M$. We first show that the two homomorphisms $\varphi_{\mathbb{Q}}$
and $\varphi_{\mathbb{Q},V}$ for $\mathscr{E}$ and
$\mathscr{E}\otimes \pi^{\ast}_M V$ are same. The similitude
$\sigma(\mathscr{E}\mid_{\otimes}\pi^{\ast}_M V)$ is equal to
$\sigma(\mathscr{E})r(V)$. Hence
$\dfrac{ch(\mathscr{E}\otimes \pi^{\ast}_M
V)}{\sigma(\mathscr{E}\otimes \pi^{\ast}_M V)}$ is equal to
$\dfrac{ch(\mathscr{E})}{\sigma(\mathscr{E})}.\pi^{\ast}_M \left(\dfrac{Ch(V)}{r(V)}\right)$. Therefore,
we have
$f_{\mathbb{Q},v}(\alpha)=f_{\mathbb{Q}}(\alpha)\left(\dfrac{ch(v)}{r(V)}\right)$
for every $\alpha\in H^{\ast}(S,\mathbb{Q})$. If
$(\upsilon. \alpha)=0$, then $H^{0}(M)$-component of
$f_{\mathbb{Q}}(\alpha)$ is zero. Hence the $H^{2}(M)$ component of
$f_{\mathbb{Q},V}(\alpha)$ is same as that of
$f_{\mathbb{Q}}(\alpha)$. Therefore, $\varphi_{\mathbb{Q},V}$ and
$\varphi_{\mathbb{Q}}$ are same. If $\mathscr{E}$ and $\mathscr{F}$
are quasi-universal families on $S\times M$, then there exist vector
bundles $U$ and $V$ on $M$ such that
$\mathscr{E}\otimes \pi^{\ast}_{M}U=\mathscr{F}\otimes \pi^{\ast}_M V$
(Definition A.4). Hence, by what we have shown, the two homomorphisms
$\varphi_{\mathbb{Q}}s$ for $\mathscr{E}$ and $\mathscr{F}$ are same,
which shows (1) of Theorem~\ref{Theorem1.5}

We prove (2) and (3) of Theorem~\ref{Theorem1.5} by a deformation
argument. Both are reduced to the case where a universal family exists
on $S\times M$. Let $T$ be the moduli space of $K3$ surface $S'$ with
isometric markings $i':H^{2}(S',\mathbb{Z})\to
H^{2}(S',\mathbb{Z})$. Let $T_0$ be the subspace of $T$ consisting of
$(S',i')$'s for which $i'(c_1(A))$ and $\ell=i'(\ell)$ lie in
$H^{1,1}(S')$ and $i'(c_1(A))$ is positive. $T_0$ contains $(S,id)$\pageoriginale
and has dimension 18 or 19 according as $c_1(A)$ and $\ell$ are
linearly independent or not. Let $A'$ be an ample divisor on $S'$ such
that $c_1(A')=i'(c_1(A))$ and put $\upsilon'=(r,\ell',s)$. The family
of moduli spaces $M_{A'}(\upsilon')$ is smooth over an etale covering of
$T_0$ (\cite{key12} Theorem 1.17). There exists a family
of quasi-universal families $\mathscr{F}_t$ on $S_t\times
M_{A_t}(\upsilon_t), t\in T_0$, which is flat over an etale covering of
$T_0$. By Proposition~\ref{Prop4.1}, the compactness of $M_{A'}(V')$ is
an open condition: There exists an open neighbourhood $U$ of $(S,id)$
such that $M_{A'}(\upsilon')$ is compact for every $(S',i')\in U$. On
the other hand the set of $(S',i^{'})$ which satisfy

($\ast$) there exists a divisor class $m\in H^{1,1}(S',\mathbb{Z})$
such that G.C.D. $(r,(\ell.m),s)=1$. 

is dense in $T_0$. By Theorem A.6 and Remark A.7, for such $S'$, there
exists a universal family on $S\times M_{A'}(\upsilon')$. Hence there
exists a pair $(S',i')$ for which $M'=M_{A'}(\upsilon')$ is compact and
a universal family $\mathscr{E}'$ exists on $S\times M'$. By
Theorem~\ref{Theorem4.9}, $M'$ is a $K3$ surface and $(2)$ and $(3)$
of Theorem~\ref{Theorem1.5} are true for this $S'$ and
$\mathscr{E}$. Hence $M$ is a $K3$ surface and $(2)$ and $(3)$ of
Theorem~\ref{Theorem1.5} are true for this $S'$ and for every
quasi-universal family $\mathscr{F}'$ on $S\times M'$. Since $(S,id)$
and $\mathscr{F}$ is a flat deformation of $(S',i')$ and
$\mathscr{F}'$, (2) and (3) are also true for $S$. The second half of
Theorem~\ref{Theorem1.4} follows from (2) and (3) of Theorem~\ref{Theorem1.5}
\enprf.

\section[Existence of simple...]{Existence of
simple \texorpdfstring{$\mu$}{eq}-semi-stable\\ semi-rigid
sheaves}\label{s5}

In this section, we show the existence of simple $\mu$-semi-stables
sheaves $E$ with $\upsilon(E)=\upsilon$ for primitive isotropic
vectors $\upsilon$ of $\widetilde{H}^{1,1}(S,\mathbb{Z})$. 

\begin{Theorem}\label{Theorem5.1} 
Let\pageoriginale $\upsilon=(r,\ell, s)$ be a primitive isotropic vector of\\
$\widetilde{H}^{1,1}(S,\mathbb{Z})$ of rank $r\geqq 1$ and $A$ an
arbitrary ample divisor. Then there exists a simple $\mu$-semi-stable
sheaf $E$ with $\upsilon(E)=\upsilon$, i.e. $SM_A(\upsilon)$ is
nonempty.
\end{Theorem}

By virtue of Theorem A. 1, this theorem is equivalent to the following
stronger version:

\begin{Theorem}\label{Theorem5.2}
Let $m$ be a divisor class of $S$. Then the simple $\mu$-semi-stable
sheaf $E$ can be chosen so that $E$ satisfies the following condition:

$(\ast)$~ $\dfrac{(c_1(F)\cdot m)}{r(F)}\geqq \dfrac{(c_1(E)\cdot
m)}{r(E)}$ holds for every non-torsion quotient sheaf $F$ of $E$ with
$\mu(F)=\mu(E)$. 
\end{Theorem}

In fact, if $n\gg 0$, then $nA+m$ is ample. By
Theorem~\ref{Theorem5.1}, there exists a simple sheaf $E_n$ with
$\upsilon(E_n)=\upsilon$ and which is $\mu$-semi-stable with respect
to $A+\dfrac{1}{n}m$. By Theorem A.1, there exists a simple sheaf $E$
which is $\mu$-semi-stable with respect to infinitely many
$A+\dfrac{1}{n}m$. It is easy to see that this $E$ satisfies ($\ast$)
in Theorem~\ref{Theorem5.2} We prove these theorems by induction on
$r$. In the case $r=1$, $E=\mathscr{O}_S(\ell)\otimes \mathfrak{m}$
satisfies our requirement for a maximal ideal $\mathfrak{m}$ of
$\mathscr{O}_S$. In fact, $\upsilon(E)=n$ and $\upsilon$ is $\mu$-stable with
respect to any ample line bundle. Assume that Theorem~\ref{Theorem5.2}
is true in the case of rank $<r$. Under this assumption, we shall show
that Theorem~\ref{Theorem5.1} is true fore every $\upsilon$ of rank
$r$.

Step I. Assume that $-r<s<0$ and $(\ell.A)=0$. Then there exists a
simple $\mu$-semi-stable sheaf $E$ with $\upsilon(E)=\upsilon$. 

\begin{Proof}\pageoriginale
By the induction hypothesis, there exists a simple $\mu$-semi-stable
sheaf $F$ with $\upsilon(F)=(-s,\ell,-r)$. Since $\mu(F)=0$, the
canonical homomorphism $f:H^{0}(S,F)\otimes \mathscr{O}_S\to F$ is
injective and for every nonzero homomorphism $g:F\to \mathscr{O}_S$,
the cokernel of $g$ is of finite length. Here we apply
Theorem~\ref{Theorem5.2}, putting $m=-\ell$. Then we can take $F$ so
that 
$$
\dfrac{-(c_1(G)\cdot \ell)}{r(G)}\geqq\dfrac{-(\ell^{2})}{r(F)}
$$
holds for every nontorsion quotient $G$ of $F$ with
$\mu(G)=\mu(F)$. Since $(\ell^{2})=2rs<0,(c_1(G)\cdot \ell)$ is
negative. Hence, for this $F$, we have
$\hom_{\mathscr{O}_S}(F,\mathscr{O}_S)=0$. Therefore, by the Serre
duality, $H^{2}(S,F)=0$ and $F$ satisfies (\ref{eqn2.21}). Let $E$ be
the reflection of $F$ (see \S\ 2). Then $\upsilon(E)=\upsilon$ and
there is an exact sequence
$$
0\to H^{0}(S,F)\otimes \mathscr{O}_S\xrightarrow{f} F\to E\to
H^{1}(S,F)\otimes \mathscr{O}_S\to 0.
$$
Since $F$ is $\mu$-semi-stable and $\mu(F)=\mu(\mathscr{O}_S)$, the
cokernel of $f$ is torsion free and $\mu$-semi-stable. Hence $E$ is
torsion free and $\mu$-semi-stable. By Proposition~\ref{Prop2.25}, $E$
is simple.
\enprf
\end{Proof}

We do not use the full strength of the above step but only the
existence of simple torsion free sheaves on monogonal $K3$ surfaces. A
quasi-polarized $K3$ surface $(S,A)$ is called \textit{monogonal} if
there exists a smooth elliptic curve $C$ on $S$ with $(A.C)=1$.  put
$g=\dfrac{1}{2}\left(A^{2}\right)+1$. Then $(A-gC)^{2}=-2$ and
$(C.A.-gC)=1$. Hence there exists an effective divisor $D$ such that
$D\sim A-gC$.  

If\pageoriginale $\rho(S)=2$, then Pic $S$ is generated by $C$ and $D$ and $D$ is a
smooth rational curve. $S$ is a double cover of the
$\mathbb{P}^{1}$-bundle
$\mathbb{F}_2=\mathbb{P}(\mathscr{O}\oplus \mathscr{O}(2))$ over
$\mathbb{P}^{1}$. A divisor $aC+b(C+D)$ on $S$ is ample if and only if
$a>b>0$.  

Step.II. \textit{Assume that $S$ is monogonal and $\rho(S)=2$. Then
there exists a simple torsion free sheaf $E$ on $S$ with $\upsilon(E)=\upsilon$.}

\begin{Proof}
$\ell$ is equal to $aC+b(C+D)$ for some integers $a$ and $b$. Take an
integer $b'$ so that $b'\equiv b \mod r$  and $|b'| \leqq
r/2$. Then take an integer $a'$ congruent to a modulo $r$ so that
$r/2< |a'|\leqq 3r/2$ and $a' b'<0$ if $b'\neq 0$ and so that
$-r<a'\leqq 0$ if $b'=0$. Put $\ell'=a'C+b'(C+D)\cdot \ell'$ is
congruent to $\ell$ modulo $r$ and $s'=\left({\ell'}^{2}\right)/2r$ is
an integer. We show the existence of a simple torsion free sheaf $E'$
on $S$ with $\upsilon(E')=(r,\ell',s')$. Then
$E=E'\otimes \mathscr{O}_S\left(\dfrac{(\ell-\ell')}{r}\right)$ is a
simple torsion free sheaf and satisfies $\upsilon(E)=\upsilon$. If
$b'\neq 0$, then $\dfrac{-3r^{2}}{4}\leqq a' b'<0$ by our choice of
$a'$ and $b'$. Since $\left(\ell'^{2}\right)=2a' b'$,  we have
$\dfrac{-3r}{4}\leqq s'<0$. Put $H=a'C-b'(C+D)$. If $b'\neq 0$, then
$H$ or $-H$ is ample. Since $(H\cdot\ell')=0$, there exists a simple
torsion free sheaf $E'$ with $\upsilon(E')=(r,\ell',s')$ by Step I. If
$b'=0$, then $s'=0$. Since $\upsilon'$ is primitive, $r$ and $a'$ are
coprime. Hence there exists a simple vector bundle $\xi$ on the
elliptic curve $C$ of rank $-a'$ and degree $r$ by \cite{key1} (see
also \S\ 2 \cite{key18}). $\xi$ is generated by global sections and
$H^{1}(C,\xi)=0$ (see Lemma~\ref{lemma5.3} below). We regard $\xi$ as
a sheaf on $S$ supported by $C$. Let $E'$ be the kernel of the natural
homomorphism $\varphi:H^{0}(S,\xi)\otimes \mathscr{O}_{S}\to \xi$. 
Then $\varphi$ is
surjective and $E'$ is a vector bundle. Since $\dim H^{0}(S,\xi)=\dim
H^{0}(C.\xi)=r$. the rank of $E'$ is equal to $r$. Since $\xi$ is a
simple sheaf and since $H^{1}(S,\xi)=0,E'$\pageoriginale is simple. (Every
endomorphism of $E$ comes from that of $\xi$.)
\enprf
\end{Proof}

\begin{lemma}\label{lemma5.3}
Let $E$ be an indecomposable  vector bundle of rank $r$ and degree $d$
on an elliptic curve $C$. If $d>r$. then $E$ is generated by global
sections and $H^{1}(C,E)=0$.
\end{lemma}

\begin{Proof}
Let $h$ be the greatest common divisor of $r$ and $d$. Then $E$ has a
filtration
$$
0=E_o\subset E_1\subset\ldots\subset E_b=E
$$
such that $\dfrac{E_i}{E_{i-1}}$ is indecomposable and has rank
$\dfrac{r}{h}$ and degree $\dfrac{d}{h}$ for every
$i=1,2,\ldots,h$. Hence we may assume that $r$ and $d$ are
coprime. Then, by Lemma 2.2 \cite{key1}, $E$ is simple. Let
$\dfrac{d'}{r'}$ be the greatest irreducible fraction with
$\dfrac{d'}{r'}<\dfrac{d}{r}$ and $0<r'<r$. There exists a simple
vector bundle $E'$ on $C$ with rank $r'$ and degree $d'$. Since $r'
d-rd'=1$, we have $\mathcal{X}(E',E)=1$ by the Riemann-Roch
theorem. Applying Part II \cite{key1} for $E'^{\vee}\otimes E$, we
have $\ext^{1}_{\mathscr{O}_C}(E',E)=0$ and
$\dim\hom_{\mathscr{O}_C}(E' E)=1$. 

Since $E'$ and $E$ are stable, the canonical homomorphism
$\varphi:E'\otimes \hom_{\mathscr{O}_C}(E',E)\to E$ is injective and
the cokernel $E''$ has no torsion. Since
$\hom_{\mathscr{O}_C}(E'',E')=0$, we have
$\ext^{1}_{\mathscr{O}_C}(E',E'')=0$ by the Serre duality. Hence every
endomorphism of $E''$ is induced by that of $E$. Therefore, $E''$ is
simple. So we have obtained an exact sequence of simple vector bundles 
$$
0\to E'\to E\to E''\to 0
$$
Case\pageoriginale for which $\dfrac{d'}{r'}>1:$ By the induction hypothesis, our
assertion is true for $E'$ and $E''$. Hence so is for $E$.

Case for which $\dfrac{d'}{r'}=1$: By our choice of $\dfrac{d'}{r'}$,
we have $r'=d'=1$ and $d=r+1$. $E'$ is a line bundle of degree $1$ and
isomorphic to $\mathscr{O}_C(p)$ for a point $p$ on $C$. By the
induction hypothesis, $E''$ is generated by global sections. Hence $E$
is generated by global sections except at $p$. Let $L$ be the kernel
of the canonical homomorphism $\psi:H^{0}(C,E)\otimes \mathscr{O}_C\to
E$. Since $\psi$ is generically surjective and since
$h^{0}(C,E)=r(E)+1$, $L$ is a line bundle. If
$\dim\hom_{\mathscr{O}_C}(L,\mathscr{O}_C)<h^{0}{(C,E)}$, then $E$
would be decomposable. Hence we have $h^{0}(C,L^{-1})\geqq d=r+1$. By
the Riemann-Roch theorem we have $\deg L\leqq -d$ and $\deg$ (Image
$\psi$)$\geqq d$. Hence $\psi$ is surjective.
\enprf
\end{Proof}

Next we study the case where $A$ is primitive and $\ell$ is a multiple
of $A$, say $\ell=k A$ for an integer $k$. In this case, the moduli
space $M_{S,A}(\upsilon)$ is defined for every polarized $K3$ surface
$(S,A)$ of a fixed degree, say $d=(A^{2})$. Let $F_d$
(resp. $\overline{F}_d$) be the moduli space of polarized
(resp. quasi-polarized) $K3$ surfaces $(S,A)$ of degree $d$. By the
Torelli theorem (\cite{key7}, \cite{key20}), $F_d$ and $F_d$ are
irreducible. 

Step III. There is a nonempty open subset $U$ of $F_d$ such that
$M_{S,A}$ is nonempty for every polarized $K3$ surface $(S,A)\in U$.

\begin{Proof}
If $(S,A)$ is monogonal and $\rho(S)=2$, then there exists\pageoriginale simple
torsion free sheaf $E$ with $\upsilon(E)=\upsilon$. Since
$\{Spl_S(\upsilon)\}_{(S,A)\in F_d}$ is a smooth family over an etale
covering of $F_d$ (\cite[Theorem1.17]{key12}), there
exists a simple torsion free sheaf $E'$ on $S'$ with
$\upsilon(E')=(r,kA',s)$ for every small deformation $(S',A')$ of
$(S,A)$. The polarized $K3$ surfaces $(S',A')$ with $\rho(S')=1$ form
a dense subset in $F_d$. Hence there exists a polarized $K3$ surface
$(S',A')$ with $\rho(S')=1$ and a simple torsion free sheaf $E'$ on $S'$
with $\upsilon(E')=(r,kA',s)$. Since $(\nu(E')^{2})=0$ and
$\rho(S')=1$, $E'$ is stable, by virtue of Proposition~\ref{Prop3.14}
Since $\{M_{S,A}(\upsilon)\}_{(S,A)\in F_d}$ is a smooth family over
an etale covering of $F_{d'}$ there exists an open neighbourhood $U$
of $(S',A')$ which satisfies our requirement.
\enprf
\end{Proof}

Step IV. \textit{If $\ell$ is a multiple of $A$, then there exists a
sheaf $E$ with $\upsilon(E)=\upsilon$ and which is stable with respect
to $A$, i.e., $M_{S,A}(\upsilon)$ is nonempty for every $(S,A)$.}

\begin{Proof}
By Langton's theorem (\cite{key6} see also \cite{key9} \S\ 5), the
family 
$$
\{\overline{M}_{S,A}(\upsilon)\}_{(S,A)\in F_d}
$$ 
of the
moduli spaces of semi-stable sheaves is proper over $F_d$. By Step
III, $\overline{M}_{S,A}(\upsilon)$ is nonempty over a dense open
subset of $F_d$. Therefore $\overline{M}_{S.A}(\upsilon)$ is nonempty
for every $(S,A)\in F_d$. Let $\pi:\mathscr{S}\to F$ be a family of
polarized $K3$ surfaces. Then, by Maruyama \cite{key9}\S\ 4, the
(coarse) moduli space $\prod:\overline{M}_{\mathscr{S}/F}\to F$ to
semi-stable sheaves on $\dfrac{\mathscr{S}}{F}$ exists and each fibre of
$\prod$ is canonically isomorphic to the moduli space of semi-stable
sheaves on the corresponding fibre of $\pi$. In particular, the
function $F_d\ni (S,A)\mapsto \dim \overline{M}_{S,A}(\upsilon$ is
upper semi-continuous. Since
$\dim \overline{M}_{S,A}(\upsilon)\geqq \dim \overline{M}_{S,A}(\upsilon)=2$
for every\pageoriginale member $(S,A)$ of $U$ in Step II, we have
$\dim \overline{M}_{S,A}(\upsilon)\geqq 2$ for every polarized $K3$
surface $(S,A)$. By Proposition~\ref{Prop3.14}, the complement of
$M_{S,A}(\upsilon)$ in $\overline{M}_{S,A}(\upsilon)$ discrete. Hence
$M_{S,A}(\upsilon)$ is nonempty for every $(S,A)\in F_d$.
\enprf
\end{Proof}

Now we return to the general case.

Step V. There exists a simple sheaf $E$ with $\upsilon(E)=\upsilon$
and which is $\mu$-semi-stable with respect to $A$. 


\begin{Proof}
If a sheaf $E$ is stable with respect to $A$, then $E\otimes L$ is
simple and $\mu$-stable with respect to $A$ for every line bundle
$L$. Hence, by Step IV, our assertion is true of $\ell\equiv k A \mod
r$  for an integer $k$. In particular, $SM^{\mu}_{rn A+\ell}(r,\ell,
s)$ is nonempty for every $n\gg 0$. Since the sequence $\{A+\ell/rn\}$
$\mathbb{Q}$-divisors converges to $A$, we have, by Theorem A. $1,
SM^{\mu}_A(r,\ell, s)$ is nonempty.
\enprf
\end{Proof}

We have completed the proof of Theorem~\ref{Theorem5.1} and
Theorem~\ref{Theorem5.2} By Step IV, we have also proved the
following. 


\begin{Theorem}\label{Theorem5.4}
Let $\upsilon=(R,\ell,s)$ be a primitive isotropic vector of\break
$\widetilde{H}^{1,1}(S,\mathbb{Z})$ and assume that $\ell$ is
ample. Then there exists a sheaf $E$ with $\upsilon(E)=\upsilon$ and
stable with respect to $\ell$, i.e., $M_{\ell}(r,\ell,s)\neq \phi$.
\end{Theorem}

\section{Application to the Hodge conjecture}\label{s6}

In this section, we apply the results in \S\ 4 and $5$ to show
that\pageoriginale certain Hodge cycles $Z$ on a product $S\times S'$ of two
algebraic $K3$ surfaces $S$ and $S'$ are algebraic
(Theorem~\ref{Theorem1.9}). We first consider the special case for
which $T_S\cong \varphi(T_S)$, where $\varphi=f^{\tau}_Z$ as in
Theorem~\ref{Theorem1.9}.

Step I. \textit{Let $\varphi:T_S\xrightarrow{\sim} T_{S'}$ be a Hodge
isometry between the transcendental lattices of $S$ and $S'$. Then
there exists an algebraic cycle $\omega \in H^{4}(S\times S', \mathbb{Q})$ on
$S\times S'$ such that
$\varphi(\alpha)=\pi'_{S'}\ast(W\cdot \pi_S^{\ast}\alpha)$ for every
$\alpha \in T_S$}. 

We remark that there exists an isomorphism $f:S'\to S$ such that
$f^{\ast}=\varphi$ on $T_S$ if $\rho(S)>11$
(Proposition~\ref{Prop6.2}). But this is not true in general if
$\rho(S)\leqq 11$. In fact, there is a pair of $K3$ surfaces $S$ and
$S'$ such that $T_S\cong T_{S'}$ but $N_S\notcong N_{S'}$ as lattices. We
note that two lattices $\widetilde{H}^{1,1}(S,\mathbb{Z})$ and
$\widetilde{H}^{1,1}(S',\mathbb{Z})$ are isomorphic to each other,
which is the key of our proof of  Step I. More  strongly, by Theorem
1.14.2 and 1.14.4 in \cite{key17}, we have 

\begin{Prop}\label{Prop6.1}
Let $\varphi_1,\varphi_2:T\to H$ be two primitive embeddings of a
lattice $T$ into an even unimodular lattice. $H$. Assume that the
orthogonal complement $N$ of $\varphi_1(T)$ in $H$ satisfies one of
the following:
\begin{enumerate}
\renewcommand{\labelenumi}{(\theenumi)}
\item $N$ contains the hyperbolic lattice $U=\begin{bmatrix}
0 & 1\\
1 & 0
\end{bmatrix}$ as a sublattice or 
\item $N$ is indefinite and rank $N\geqq$ rank $T+2$.
\end{enumerate}
Then\pageoriginale $\varphi_1$ and $\varphi_2$ are equivalent, i.e., there exists an
isometry $\gamma$ of $H$ such that
$\varphi_{1}=\gamma \circ \varphi_2$. 
\end{Prop}

We give a proof  of the fact remarked above, which is a prototype of
our proof of Step I.

\begin{Prop}\label{Prop6.2}
Let $S$ and $S'$ be algebraic $K3$ surfaces and $\varphi:T_S\to T_{S'}$
be  a Hodge isometry. If $\rho(S)>11$, then there exists an
isomorphism $f:S\to S$ such that $f^{\ast}=\varphi$ on $T_S$. 
\end{Prop}

For the proof, we need a version of Torelli theorem of $K3$ surfaces: 

\begin{Prop}\label{Prop6.3}
Let $S$ and $S'$ be $K3$ surfaces and $\psi:H^{2}(S,\mathbb{Z})\to
H^{2}(S',\mathbb{Z})$ be a Hodge isometry. Then there exists an
isomorphism $f:S'\to S$ such that $f^{\ast}=\psi$ on $T_S$. 
\end{Prop}

\begin{Proof}
By the strong Torelli theorem (\cite{key7}), there exists an
isomorphism $f:S''\to S$ and reflections $r_i(i=1,\ldots, n)$ by
$(-2)$ curves $C_i\cong \mathbb{P}^{1}$ on $S$ such that
$\psi=f^{\ast}\circ r_1\circ \cdots\circ r_n$. Since $[C_i]$ is
perpendicular to $T_S$, $r_i$ is identity on $T_S$ for every
$i=1,\ldots,n$. Hence we have our proposition.
\enprf
\end{Proof}

\begin{pf1}
Apply $(2)$ of
Proposition~\ref{Prop6.1} to two primitive embeddings
$T_S\underline{\hookrightarrow}H^{2}(S,\mathbb{Z})$ and $T_S,\hookrightarrow
H^{2}(S,\mathbb{Z})$. Since $H^{2}(S,\mathbb{Z})$ and
$H^{2}(S',\mathbb{Z})$ are isomorphic to each other as lattices, we
obtain isometry $\widetilde{\varphi}:H^{2}(S,\mathbb{Z})\to
H^{2}(S',\mathbb{Z})$ such that\pageoriginale $\widetilde{\varphi}\mid
T_S=\varphi$. By the above proposition, there exists an isomorphism
$f:S'\to S$ such that $f^{\ast}=\widetilde{\varphi}$ which proves our
proposition.
\enprf
\end{pf1}

\begin{pf2}
The orthogonal complement of $T_S$ in the extended $K3$ lattice
$\widetilde{H}(S,\mathbb{Z})$ is isomorphic to $N_S\perp U$. Applying
(1) of Proposition~\ref{Prop6.1} to the embedding of $T_S$ and $T_{S'}$
into $\widetilde{H}(S,\mathbb{Z})$ and $\widetilde{H}(S',\mathbb{Z})$,
we see that there exists an isometry
$\Phi:\widetilde{H}(S,\mathbb{Z})\to \widetilde{H}(S,\mathbb{Z})$ such
that $\Phi\mid T_S=\varphi$. Put $\upsilon =\Phi(0,0,1)=(r,\ell, s)$
and $u=\Phi(1,0,0)=(p,k,q)$. $\Phi$ maps $\widetilde{H}^{1,1}\cdot
(S')$ onto $\widetilde{H}^{1,1}(S)$. Hence both $\ell$ and $k$ are
divisor classes on $S$. Let $m$ be a divisor class on $S$. The Chern
character $e^{m}$ of the line bundle $\mathscr{O}_S(m)$ is a unit of
the cohomology ring $H^{\ast}(S,\mathbb{Z})$. Hence the multiplication
by $e^{m}$ induces a Hodges isometry $\Phi_m$,
$\Phi_m(r,\ell,s)=\left(r,\ell+rm,s+(m.\ell)+\dfrac{r}{2}\left(m^{2}\right)\right)$ of the
extended $K3$ lattice $\widetilde{H}(S,\mathbb{Z})\cong
H^{\ast}(S,\mathbb{Z})$. Replacing $\Phi$ by $\Phi_m\circ \Phi$  for a
sufficiently ample divisor $m$, we can choose $\Phi$ so that $s$ is
positive. Since the change of $r$ and $s$ is an Hodge isometry, we
choose $\Phi$ so that $r$ is positive. Since $(u.v)=-1$, the greatest
common divisor $r$, $(\ell, k)$ and $s$ is equal to $1$. 
\end{pf2}

\begin{claim}
There exists an integer $n$ such that $r$ and $s+n(\ell.k)$  are
coprime. 
Let $d$ be the greatest common divisor of $s$ and $(\ell. k)$. Since
$\dfrac{s}{d}$ and $\dfrac{(\ell.k)}{d}$ are coprime, there exists an
integer $n$ such that $r$ and $\dfrac{s}{d}+\dfrac{n(l.k)}{d}$ are
coprime. Since $r$ and $d$ are coprime, so are $r$ and $s+n(\ell.k)$. 

Take\pageoriginale $n$ as in the claim and replace $\Phi$ by $\Phi_{n
k}\circ \Phi$. Then by the claim, $r$ and $s$ are coprime. Replace
$\Phi$ by $\Phi_{r A}\circ \Phi$ again for a sufficiently ample
divisor $A$. Then $r$ and $s$ are still coprime and $\ell$ become
ample. Let $M$ be the moduli space $M_{\ell}(\upsilon)$ of sheaves $E$
with $\upsilon(E)=\upsilon$ which is stable with respect to $\ell$. By
Theorem~\ref{Theorem5.4} $M$ is nonempty. Since $r$ and $s$ are
coprime, every semi-stable sheaf is stable. Hence $M$ is compact and
hence irreducible by Corollary~\ref{cor4.6} By Theorem A.6 and Remark
A.7, there exists a universal family $\mathscr{E}$ on $S\times M$. By
Theorem~\ref{Theorem4.9}, the cycle $Z=\pi_S\ast\sqrt{td}_S\cdot
ch(\mathscr{E})\pi^{\ast}_M\sqrt{td}_M$ induces a Hodge isometry
$\psi:\widetilde{H}(M,\mathbb{Z})\to \widetilde{H}(S,\mathbb{Z})$, with
$\Psi(\delta)=\upsilon$, where $\delta=(0,0,1).\Phi^{-1}\circ \Psi$ is
an isometry and sends $\delta$ to $\delta$. Hence
$\Phi^{-1}\circ \Psi$ induces a Hodge isometry from
$H^{2}(M,\mathbb{Z})=\Psi^{-1}\left(\dfrac{\upsilon^{\perp}}{\mathbb{Z}
\upsilon}\right)$ onto
$H^{2}(S',\mathbb{Z})=\Phi^{-1}\left(\dfrac{\upsilon^{\perp}}{\mathbb{Z}\upsilon}\right)$. 

By Proposition~\ref{Prop6.3}, there exists an isomorphism $f:S'\to M$
such that $f^{\ast}:H^{2}(M,\mathbb{Z})\to H^{2}(S',\mathbb{Z})$
coincides with $\Phi^{-1}\circ \Psi$ on $T_M$. Then the Chern
character $ch((i\times f)\ast \mathscr{E})\in H^{\ast}(S\times
S', \mathbb{Z})$ of $(1\times f)^{\ast}E$ induces a Hodge isometry
$\Psi':\widetilde{H}(S',\mathbb{Z})\to \widetilde{H}(S,\mathbb{Z}$
which coincides with $\Phi$ (or equivalently $\varphi$) on $T_S'$. The
$H^{4}(S\times S')$ component $W$ of $Z$ induces a homomorphism $\tau$
of the Hodge structure $H^{2}(S',\mathbb{Z})$ to
$H^{2}(S',\mathbb{Z})$. $\tau$ maps $T_S'$ onto $T_S$ and coincides
with $\phi$ on $T_S'$. 
\enprf
\end{claim}

Let $\upsilon=(r,\ell, s)$ be a primitive isotropic vector of
$\widetilde{H}^{1,1}(S',\mathbb{Z})$ and assume that the moduli space
$M=M_A(\upsilon)$ of stable sheaves $E$ with $\upsilon(E)=\upsilon$ is
nonempty and compact. Then, by Theorem~\ref{Theorem1.5}, there exists
an algebraic cycle $Z$ on $S\times M$ defined by using the\pageoriginale Chern
character of a quasi-universal family and $Z$ induces a Hodge isometry
$\phi:\dfrac{\upsilon^{\perp}}{\mathbb{Z}\upsilon}\to
H^{2}(M,\mathbb{Z})$. The transcendental lattice $T_S$ $\text{ regarded as a
sublattice of }\widetilde{H}(S,\mathbb{Z})$ is perpendicular to $\nu$
and $T_S\cap \mathbb{Z}\nu=0$. Hence
$\dfrac{\upsilon^{\perp}}{\mathbb{Z}\upsilon}$ contains a sublattice
isomorphic to $T_S$ and $\varphi$ induces a Hodge isometry
$\varphi:T_S\to T_M\cdot \varphi$ is injective but not surjective in
general. 

\begin{Prop}\label{Prop6.4}
Let $\upsilon=(r,\ell, s)$ $m$ and $\varphi$ be as above. Let
$n=n(\upsilon)$ be the minimum of $|(u.v)|$, where $u$ runs over all
vectors of $\widetilde{H}^{1,1}(S,\mathbb{Z})$ with $(u.v)\neq
0$. Then we have 
\begin{enumerate}
\renewcommand{\labelenumi}{(\theenumi)}
\item the cokernel of $\varphi$ is a cyclic group of order $n$, 

\item there exists a transcendental cycle $\lambda \in T_S$ such that
$\ell+\lambda \in H^{2}(S,\mathbb{Z})$ is divisible by $n$, and 

\item if $\lambda$ satisfies (2), then  $\varphi(\lambda)\in T_M$ is
divisible by $n$ and $\dfrac{\varphi(\lambda)}{n}$ generates the
cokernel of $\varphi$. 
\end{enumerate}
\end{Prop}

\begin{Proof}
For every $\upsilon \in H^{1,1}(S,\mathbb{Z})$, $\dfrac{(u.v)}{n}$ is
an integer. Since $\widetilde{H}(S,\mathbb{Z})$ is unimodular and
$\widetilde{H}^{1,1}(S,\mathbb{Z})$ is a primitive sublattice, there
exists $w\in \widetilde{H}(S,\mathbb{Z})$ such that
$\dfrac{(u.\upsilon)}{n}=(w.v)$ for every
$\upsilon \in \widetilde{H}^{1,1}(S,\mathbb{Z}).\lambda=nw-u\in \widetilde{H}(S,\mathbb{Z})$
is perpendicular to $\widetilde{H}^{1,1}(S,\mathbb{Z})$ and hence lies
in $T_S$. It is clear that $\lambda$ satisfies (2). Assume that
$\lambda$ satisfies (2). Then $w=\dfrac{(\lambda+\upsilon)}{n}$ lies
in $\upsilon^{\perp}$ and $nw$ is congruent to $\lambda$ modulo
$\mathbb{Z}\upsilon$. Hence $\dfrac{\varphi(\lambda)}{n}$ lies in
$T_M$. We show that $\dfrac{\varphi(\lambda)}{n}$ generates the
cokernel  of $\varphi$. The transcendental lattice $T_M$ is isomorphic
to
$\dfrac{\left(\upsilon^{\perp}\cap \widetilde{H}^{1,1}(S,\mathbb{Z})\right)^{\perp}}{\mathbb{Z}\upsilon}\cong
(\mathbb{Q}\upsilon \oplus
T_S\otimes \mathbb{Q})\in \dfrac{\widetilde{H}(S,\mathbb{Z})}{\mathbb{Z}\upsilon}$. Let
$\alpha$ be a vector of $(\mathbb{Q} \upsilon \oplus
T_S\otimes \mathbb{Q})\cap \widetilde{H}(S,\mathbb{Z})$. Then
$\alpha=a\upsilon +v$ for $a\in \mathbb{Q}$ and\pageoriginale $\upsilon \in
T_S\otimes \mathbb{Q}$. Take a vector
$u\in\widetilde{H}^{1,1}(S,\mathbb{Z})$ such that
$(u.\upsilon)=n$. Then we have an
$=a(u.\upsilon)=(\alpha.\upsilon)\in \mathbb{Z}$. Since
$\upsilon=nw-\lambda$, we have
$\alpha=(an)w+(\upsilon-a\lambda)$. Since an is an integer,
$\upsilon-a\lambda$ lies in $T_S$ and $\alpha$ is congruent to $(an)w$
modulo $T_S$. Hence $\dfrac{\varphi(\lambda)}{n}$ generates the
cokernel of $\varphi$, which shows $(3)$. If
$\dfrac{\mathfrak{m}\varphi(\lambda)}{n}$ lies in $T_S$, then $mw$
lies in $T_S+\mathbb{Z}\upsilon$ and is equal to $\lambda'+b\upsilon$
for $\lambda'\in T_S$ and $b\in \mathbb{Z}$. We have
$m(\lambda+\upsilon)=n)(\lambda'+b v)$. Since $T_S\cap Zv=0 $, $m$ is
equal to $nb$ and divisible by $n$. Hence
$\dfrac{\varphi(\lambda)}{n}$ has order $n$ is Coke $\varphi$, which
shows (1)
\enprf
\end{Proof}

We have thus proved the following 

\begin{cor}\label{cor6.5}
Let $M$ be a compact surface component  of the moduli space of stable
sheaves on $S$. Then there exists an algebraic cycle on $S\times M$
which induces a homomorphism $\varphi: T_S\to T_M$ such that
$\varphi\times \mathbb{Q}$ is an isometry and the cokernel of
$\varphi$ is a finite cyclic group. 
\end{cor}

Conversely, we have 

\begin{Prop}\label{Prop6.6}
Let $S$ be an algebraic $K3$ surface and $\Psi:T_S\to T$ be an
embedding of the transcendental lattice $T_S$ of $S$ into an even
lattice $T$. Assume that the cokernel of $\Psi$ is a cyclic group of
order $r<\infty$. Then there exists a compact component $M$ of the
moduli space of stable sheaves of rank $r$ on $S$ which satisfies the
following: 
\begin{enumerate}
\renewcommand{\labelenumi}{(\theenumi)}
\item there is an isometry $i:T\xrightarrow{\sim} T_M$ and 

\item there\pageoriginale is an algebraic cycle on $S\times M$ which induces
$i\circ \psi$. 
\end{enumerate}
\end{Prop}

\begin{Proof}
Take a transcendental cycle $\tau\in T_S\otimes \mathbb{Q}$ so that
$(\psi\otimes \mathbb{Q})(\tau)$ belongs to $T$ and generates $T$
modulo $\psi(T_S)$. By our assumption, $\lambda=r\tau$ belongs to
$T_S$. Since $\psi\otimes \mathbb{Q}$ is an isometry, $(\tau. \beta)$
is equal to $((\psi \otimes \mathbb{Q})(\tau)\cdot \psi (\beta))$ and
is an integer for every $\beta \in T_S$. Since $H^{2}(S,\mathbb{Z})$
is a unimodular lattice and since $T_S$ is a primitive sublattice of
$H^{2}(S,\mathbb{Z})$, there exists a cycle $\alpha \in
H^{2}(S,\mathbb{Z})$ such that $(\alpha\cdot \beta)=(\tau\cdot \beta)$
for every transcendental cycle $\beta \in T_S$. Then, the cycle
$\ell=r(\alpha-\tau)$ belongs to $H^{2}(S,\mathbb{Z})$ and
perpendicular to $T_S$. Hence $\ell$ is a divisor class of
$S$. Moreover, $\ell+\lambda$ is equal to $r\alpha$ and divisible by
$r$ in $H^{2}(S,\mathbb{Z})$. Replacing $\alpha$ by $\alpha+$(a
sufficiently ample divisor), we can choose $\alpha$ so that $\ell$
becomes an ample divisor class. We put
$s=\dfrac{(\ell^{2})}{2r}=\dfrac{r(\alpha-\tau)^{2}}{2}$ and $\upsilon
=(r,\ell, s)\in \widetilde{H}^{1,1}(S,\mathbb{Z})$ and consider the
moduli space $M=M_A(\upsilon)$ of stable sheaves $E$ with
$\upsilon(E)=\upsilon$. Since $(\tau^{2})$ is an even integer, so is
$(\alpha-\tau)^{2}$. Hence $s$ is divisible by $r$. Since $\tau$ is
transcendental, $(\ell\cdot m)$ is equal to $r(\alpha\cdot m)$ and
hence divisible by $r$ for every divisor class $m$ of $S$. Hence the
number $n(\upsilon)$ (see Proposition~\ref{Prop6.4}) is equal to
$r$. $M_{\ell}(\upsilon)$ is nonempty, by Theorem~\ref{Theorem5.4} and
$M^{\mu}_{\ell}(\upsilon)$ is nonempty, by
Proposition~\ref{Prop3.16}. Hence by Proposition~\ref{Prop4.3}, there
exists an ample line bundle $A$ such that $M=M_A(\upsilon)$ is
nonempty and compact and irreducible. By Proposition~\ref{Prop6.4},
there exists an isometry $i:T\to T_M$ such that $\varphi=i\cdot \Psi$
and $\varphi$ is induced by an algebraic cycle on $S\times M$. 
\enprf
\end{Proof}

Step II. Let $\varphi:T_S\to T_S$ be a homomorphism of Hodge
structures\pageoriginale and assume that $\varphi \otimes \mathbb{Q}$ is an
isometry. Then there exists an algebraic cycle $W\in H^{4}(S\times
S',\mathbb{Q}\mid)$ on $S\times S'$ which induces $\varphi$. 

\begin{Proof}
We prove our assertion by induction on the length $\ell$ of the
cokernel of $\varphi$. In the case $\ell=1$, our assertion was proved
in Step I. Hence we assume that $\ell>1$. Take a sublattice $T$ of
$T_{S'}$ such that $\varphi(T_S)\displaystyle\mathop{\subset}_{\neq} T$
and $\dfrac{T}{\varphi(T_S)}$ is a cyclic group. Then, by
Proposition~\ref{Prop6.6}, there exists a $K3$ surface $M$ which is a
compact component of the moduli space of stable sheaves such that
$T_M\cong T$ and there exists an algebraic cycle $W_1$ on $S\times M$
which induces $T_S\to T\cong T_M$. By induction hypothesis, there
exists an algebraic cycle $W_2$ on $M\times S'$ which induces
$T_M\cong T\to T_{S'}$. Then, the cycle $Z=\pi_{S\times
S'},\ast(\pi^{\ast}_{S\times M}W_1\cdot \pi_{M\times S'}\ast W_s)$ on
$S\times S'$ is algebraic and induces  $\varphi$ on $T_S$.
\enprf
\end{Proof}

\begin{pf3}
By our assumption, there exists a primitive embedding
$T\hookrightarrow \wedge$ of $T$ into a $K3$ lattice $\wedge$. Since
$T\otimes \mathbb{Q}\cong T_S\otimes \mathbb{Q}$ the Hodge decomposition of
$T_S\otimes \mathbb{C}$ induces that of $T\otimes \mathbb{C}$ We
regard $T$ as a polarized Hodge structure by this Hodge
decomposition. The orthogonal complement of $T$ in $\wedge$ is a
hyperbolic lattice, i.e., has signature $(1,\ast)$. By virtue of the
surjectivity theorem of the period map for $K3$ surfaces \cite{key23},
there exists a $K3$ surface $S''$ and an isometry $i:\wedge\to
H^{2}(S'',\mathbb{Z})$ such that $i(T)=T_{S''}$ and $i\mid_T$ is a
homomorphism of Hodge structures. Both $T_S$ and $T_{S'}$ contain
$T_{S''}$ as a sublattice of finite index. By Step II, there exist
algebraic cycles on $S''\times S$ and on $S''\times S'$ which induce
the isometries $T_{S''}\hookrightarrow T_S$ and
$T_{S''}\hookrightarrow T_{S'}$,\pageoriginale respectively. Therefore, the
composition of the two algebraic cycles induces the Hodge isometry
between $T_S\otimes \mathbb{Q}$ and $T_{S'}\otimes \mathbb{Q}$. 
\enprf
\end{pf3}

\begin{app}\label{app1}
Boundedness and existence of $\mu$-semi-stable sheaves.
\end{app}

In this section, $S$ is an arbitrary complete algebraic surface over
$\mathbb{C}$. We study the behaviour of moduli spaces of
$\mu$-semi-stable sheaves with respect to $A_n,n=1,2,3,\ldots,$ when
ample $\mathbb{Q}$ -divisors $A_n$ converge to an ample divisor $A$. 

\begin{Thm}\label{Thm1}
Let $\{A_n\}$ be a sequence of ample $\mathbb{Q}$-divisors which
converges to an ample divisor $A$. Let $c_1$ be a numerical
equivalence class of divisors and $c_2$ an integer. Assume that, for
every $n$, there exists a sheaf $E_n$ on $S$ with Chern classes $c_1$
and $c_2$ (modulo numerical equivalence) and which is
$\mu$-semi-stable with respect to $A_n$. Then there exists a sheaf $E$
on $S$ which satisfies the following: 
\begin{enumerate}
\renewcommand{\labelenumi}{(\theenumi)}
\item there exists an infinite subsequence $\{A_{n_{k}}\}$ of $\{A_n\}$
such that $E$ is $\mu$-semi-stable with respect to every $A_{n_{k}}$,
and 
\item $E$ is $\mu$ semi-stable with respect to $A$. 
\end{enumerate}
Let $P$ be a Zariski-open condition for sheaves on $S$ which is
independent of $A_n$, e.g., simpleness or local freeness. If the open
condition $P$ holds for every $E_n$, then $E$ can be chosen so that
$E$ satisfies $P$. 
\end{Thm}

For\pageoriginale the proof of the above theorem, a certain boundedness of
$\mu$-semi-stable sheaves is essential. Let $\mathscr{A}$ be the ample
cone in $H^{1,1}(S,\mathbb{R})$ and $\mathscr{A}$ its closure.

\begin{Thm}\label{Thm2}
Let $H$ be an ample divisor and $B$ a bounded subset of
$\overline{\mathscr{A}}\cap H^{2}(S,\mathbb{Q})$. Let
$S^{r}_A(c_1,c_2)$ denote the set of isomorphic classes of rank $r$
sheaves with Chern classes $c_1$ and $c_2$ modulo numerical
equivalence and which are $\mu$-stable with respect to an ample
$\mathbb{Q}$-divisor $A$. Then the union $\bigcup\limits_{b\in
B}S^{r}_{H+b}(c_1,c_2)$ is bounded. 
\end{Thm}

In the case $B=\{0\}$, this was proved by Maruyama in \cite{key8} and
our proof of Theorem~A.\ref{Thm2} is quite parallel to his proof in \S\
2 \cite{key8}. Let $\alpha_1,\ldots,\alpha_{r-1}$ be a sequence of
$r-1$ rational numbers and let
$$
S^{r}_B(\alpha_1,\ldots,\alpha_{r-1}:c_1,c_2)
$$ 
be the set of
isomorphism classes of rank $r$ torsion free sheaves of type
$\alpha_1,\ldots,\alpha_{r-1}$ with respect to $H+b$ for some $b\in B$ 
(\cite[see p.28]{key8}) and with Chern classes $c_1$ and $c_2$ modulo
numerical equivalence. Our Theorem~A. \ref{Thm2} is a special case of the
boundedness of $S^{r}_B(\alpha_1,\ldots,\alpha_{r-1}:c_1,c_2)$ which
follows from Theorem A.~\ref{Thm3} below and Theorem 1.14
in \cite{key8}.

\begin{Thm}\label{Thm3}
Let $a$ be an integer and let
$S^{r}_{B,a}(\alpha_1,\ldots,\alpha_{r-1}:c_1)$ be the union of
$S^{r}_B(\alpha_1,\ldots,\alpha_{r-1}:c_1,c_2)$  for all $c_2\leqq
a$. Then there are two constants $b_0$ and $b_1$ (independent of each
$c_2$) such that for any member $E$ of $S^{r}_{B,a\mid}
(\alpha_1,\ldots,\alpha_{r-1}:c_1),\dim H^{0}(S,E)\leqq b_0$ and $\dim
H^{0}(C,E\otimes \mathscr{O}_C)\leqq b_1$ for any curve $C$ in an open
set $U(E)$  of $|H|$, where $U(E)$ may depend on $E$. 
\end{Thm}


\begin{Proof}
Our\pageoriginale proof is quite similar to that of Theorem 2.5
in \cite{key8}. We only indicate the parts to be modified. It suffices
to show the theorem for the subset
$VS^{r}_{B,a}(\alpha_1,\ldots,\alpha_{r-1}:c_1)$ of
$S^{r}_{B,a}(\alpha_1,\ldots \alpha_{r-1}:c_1)$ consisting of vector
bundles in $S_{B,a}(\alpha_1,\ldots, \alpha_{r-1}:c_1)$. We prove our
theorem by induction on $r$. Assume that the theorem is true in the
case rank $r-1$. Under this assumption, we shall show that our theorem
holds for $VS^{r}_{B,a}(\alpha_1,\ldots,\alpha_{r-1}:c_1)$. Since $B$
is bounded, there exists an integer $n$ such that $H^{0}(S,E(n))\neq
0$ for every member $E$ of
$VS^{r}_{B,a}(\alpha_1,\ldots, \alpha_{r-1}:c_1)$
(cf. Lemma 2.1 in\cite{key8}), where $E(n)$ is the
abbreviation of $E\otimes H^{\otimes n}$. Hence, for every member $E$
of $VS^{r}_{B,a}(\alpha_1,\ldots,\alpha_{r-1}:c_1)$, there exists an
exact sequence 
$$
0\to \mathscr{O}_S(D)\otimes H^{\otimes(-n)}\to E\to F\to 0
$$
where $D$ is an effective divisor and $F$ is a torsion free sheaf of
rank $r-1$. Let $L$ be the set of effective divisors $D$ such that
$\mathscr{O}_{S}(D)\otimes H^{\otimes(-n)}$  is contained in some
member $E$ of $VS^{r}_{B,a}(\alpha_1,\ldots,\alpha_{r-1}:c_1)$.

\begin{claim}
$L$ is bounded.
\end{claim}
$\mathscr{O}_S(D)$ is a subsheaf of $E(n)$ and $E(n)$ is of type
$\alpha_1,\ldots,\alpha_{r-1}$ with respect to $H+b$ for some $b\in
B$. Hence we have 
$$
\begin{aligned}
(D\cdot
H+b)&\leqq \mu_{H+b}(E(n))+\dfrac{\alpha_{r-1}}{(r-1)}=\dfrac{(c_1\cdot
H+b)}{r+n(H\cdot H+b)}+\dfrac{\alpha_{r-1}}{(r-1)}\\
&{}=\dfrac{(c_1\cdot H)}{r+n(H^{2})}+\left(\dfrac{c_1}{r}+nH.b\right)+\dfrac{\alpha_{r-1}}{(r-1)}.
\end{aligned}
$$
\pageoriginale
Since $B$ is bounded, $R=\displaystyle\mathop{sup}_{b\in 
B}(c_1/r+nH\cdot b)<\infty$. Since $b$ belongs to $\mathscr{A}$ and $D$ is
effective, $(b\cdot D)$ is nonnegative. Hence we have 
$$
(D\cdot H)\leqq (D\cdot H+b)\leqq \dfrac{(c_1\cdot H)}{r+n(H^{2})}+R+\dfrac{\alpha_{r-1}}{(r-1)}.
$$
Therefore, $L$ is bounded. 

Let $G$ be a rank $s$ quotient sheaf of $F$. Since $G$ is a quotient of
$E$ and since $E$ is of type $\alpha_1,\ldots \alpha_{r-1}$ with
respect to $H+b$, we have 
$$
\mu_{H+b}(E)-\alpha_s\leqq \mu_{H+b}(G).
$$
Put $\alpha_{s,D,b}=\alpha_S+\{n(H\cdot H+b)+(c_1/r-D\cdot
H+b)\}/(r-1)$. Then we have
$\mu_{H+b}(E)-\alpha_s=\mu_{H+b}(F)-\alpha_{s,D,b}$. Put
$\alpha'_s=\displaystyle\mathop{\sup}_{D\in L, b\in B} \alpha_{s,D,b.}$
Then we obtain $\mu_{H+b}(F)-\alpha_s\leqq \mu_{H+b}(G)$. Hence $F$ is
of type $\alpha_1,\ldots,\alpha_{r-2}$ with respect to $H+b$. Let $Q$
be the set of isomorphic classes of $F's$ which are obtained from some
$E$ in $VS^{r}_{B,a}(\alpha_1,\ldots,\alpha_{r-1}:c_1)$ as
above. Then, by the above result, $Q$ is a subset of 
$$
\coprod\limits_{\lambda \in \wedge}\coprod\limits_{c_2\leqq \alpha+\beta}S^{r-1}_B\left(\alpha'_1,\ldots, \alpha'_{r-2}:c_1-\lambda+nc_1(H),c_2\right)
$$
where\pageoriginale $\wedge=L/(\text{ (numerical equivalence) }$ and
$\beta=\displaystyle\mathop{\max}_{D\in L}\{-(c_1-D+nH\cdot
D-nH)\}$. By induction hypothesis, our theorem is true for any member
$F$ of $Q$ and our proof can be completed in the same way as
Theorem 2.5 in \cite{key8}.
\enprf
\end{Proof}


\begin{pf4}
Take an integer $N$ so that $NA_n-A$ is ample for every $n$. Applying
Theorem A.\ref{Thm2} for $H=A$ and $B=\{NA_n-A\}$, we see that the set
$\mathscr{E}$ of isomorphic classes of sheaves on $S$ which are
$\mu$-semi-stable with respect to $A_n$ for some $n$ is bounded. All
$E_n s$ belong to $\mathscr{E}$ and hence there exists a subfamily
$\{F_t:t\in V\}$ of $\mathscr{E}$ parametrized by a variety $V$ which
contains $E_n$ for infinitely many $n$, say, for $n=n_1,n_2,\ldots$
Since $\mu$-semi-stability is an open condition, for each $n_k$, there
exists a Zariski open set $U_k$ of $V$ such that $F_u$ is $\mu$-semi
stable with respect to $A_{n_{k}}$ (and satisfies the property $P$)
for every $u\in U_k$. $V$ is a variety over $\mathscr{C}$ and is a
Baire space. Hence the intersection of all $U_k$ $s$ is
nonempty. Therefore, we have (1). (2) follows immediately from (1),
because $\mu_A(F)=\lim\limits_{k\to \infty}\mu_{A_{n_{k}}}(F)$ for every
sheaf $F$ on $S$
\enprf
\end{pf4}

\begin{app}
Existence of a (quasi-) universal family 
\end{app}

Let $X$ be a scheme and $\mathscr{M}$ a connected component of the
moduli functor $\mathscr{S}pl_X$ of simple sheaves on $S$. 

\setcounter{dfnn}{3}
\begin{dfnn}\label{dfnn4}
\begin{enumerate}
\renewcommand{\labelenumi}{(\theenumi)}
\item Let $T$ be a scheme. A sheaf $\mathscr{E}$ on $X\times T$ is a
quassi-family of sheaves in $\mathscr{M}$ if $\mathscr{E}$ is $T$-flat
and if, for every $t\in T$ there exists an integer $\sigma$ and a
member $E$ of $\mathscr{M}$ such that $\mathscr{E}\mid_{X\times
T}\cong E^{\oplus \sigma}$.\pageoriginale If $T$ is connected, then the positive
integer $\sigma$ does not depend on $t\in T$ and called the similitude
of $\mathscr{E}$. 
\item Two quasi-families $\mathscr{E}$, $\mathscr{E}'$ of sheaves in
$\mathscr{M}$ on $X\times T$ are equivalent if there exist vector
bundles $V$ and $V'$ on $T$ such that
$\mathscr{E}\otimes \pi^{\ast}_TV\cong \mathscr{E}\otimes \pi^{\ast}_TV'$. 
\item A sheaf $\mathscr{E}$ on $X\times M$ is a quasi-universal family
of sheaves in $\mathscr{M}$ if $\mathscr{E}$ is a quasi-family and,
for every scheme $T$ and quasi family $\mathscr{F}$ on $X\times T$,
there exists a unique morphism $f:T\to M$ such that
$f^{\ast}\mathscr{E}$ and $\mathscr{F}$ are equivalent. 
\end{enumerate}
\end{dfnn}

By definition, if $\mathscr{E}$ on $X\times M$ and $\mathscr{E'}$ on
$X\times M'$ are quasi-universal families, the  $M$ and $M'$ are
isomorphic to each other and $\mathscr{E}$ and $\mathscr{E'}$ are
equivalent. 

\setcounter{Thm}{4}
\begin{Thm}\label{Thm5}
Assume that $\mathscr{M}$ is representable by a scheme $M$ of finite
type in the usual topology (if $k=\mathbb{C}$) or in the etale
topology. Then there exists a quasi-universal family on $X\times M$. 
\end{Thm}

\begin{Proof}
For simplicity, we assume that $k=\mathbb{C}$ and $M$ is representable
in the usual topology. There exists an open covering
$M=\bigcup\limits_{i}U_i$ (in the usual topology) and a universal
family $\mathscr{E}_i$ on $U_i\times X$ for every $i$. Take a
sufficiently ample line bundle $L$ such that all higher cohomology
groups $H^{i}(X,E\otimes L)$ vanish for every member $E$ of $M$. By
the base change theorem, the direct image
$V_i=\pi_i\ast(\mathscr{E}_i\otimes L)$ is a vector bundle on $U_i$,
where $\pi_i$ is the projection of $X\times U_i$ onto $U_i$ Shrink the
covering $\bigcup\limits_{i} U_i$\pageoriginale so that $\Pic (U_i\cap U_j)=0$ for
every $i\neq j$. Then there exists an isomorphism
$f_{ij}:\mathscr{E}_i\mid_{X\times (U_i\cap
U_j})\xrightarrow{\sim}\mathscr{E}_j\mid_{X\times (U_i\cap
U_j)}$. $f_{ij}$ induces an isomorphism
$\overline{f}_{ij}=\pi_{ij}\ast (f_{ij}\otimes
L):V_i\xrightarrow{\sim}V_j$, on $U_i\cap U_j$ where $\pi_{ij}$ is the
projection of $X\times (U_i\cap U_j)$ onto $U_i\cap U_j$. We put
$\Phi(f_{ij})=f_{ij}\otimes \pi^{\ast}_{ij}\left(\overline{f}^{-1}_{ij}\right)^{\vee}:\mathscr{E}_i\otimes
\pi^{\ast}_iV_i^{\vee}\mid_{X\times (U_i\cap
U_j)}\xrightarrow{\sim}\mathscr{E}_j\otimes \pi^{\ast}_jV^{\vee}_j\mid_{X\times
(U_i\cap U_j)}$.

\begin{claim}
$\Phi (f_{ij})\circ \Phi(f_{jk})\circ \Phi(f_{ki})$ is identity over
$X\times (U_i\cap U_j\cap U_k)$ for every $i,j$ and $k$. 
\end{claim}

By the functoriality of $\Phi$,
$\Phi(f_{ij})\circ \Phi(f_{ik})\circ \Phi(f_{ki})$ is equal to
$\Phi(g_{ijk})$, where $g_{ijk}=f_{ij}\circ f_{jk}\circ f_{ki}\cdot
g_{ijk}$ is an automorphism  of $\mathscr{E}_i\mid_{X\times (U_i\cap
U_j\cap U_k)}$ over $U_i\cap U_j\cap U_k$. Since $\mathscr{E}_i$ is
simple over $U_i$, the automorphism $g_{ijk}$ of $\mathscr{E}_i$ over
$U_i\cap U_j\cap U_k$ is multiplication by an invertible element of
$\mathscr{O}_{U_{i}}\cap U_j \cap U_k$. Hence $\Phi(g_{ijk})$ is
identity.

By the claim, $\mathscr{E}_i\otimes \pi^{\ast}_{i}V^{\vee}_i$ can be
glued together by $\Phi(f_{ij})$'s. We obtain a sheaf $\mathscr{E}$ on
$X\times M$ whose restriction to $U_i\times X$ is isomorphic to
$\mathscr{E}_i\otimes \pi^{\ast}_iV^{\vee}_i$ for every $i$. We show
that $\mathscr{E}$ is a quasi-universal family. Let $\mathscr{F}$ be a
quasi-family of sheaves in $M$ on $X\times T$. Since $\mathscr{E}_i$
are universal families, there exist a unique morphism $f:T\to M$, a
vector bundle $F_i$ on $f^{-1}(U_i)$ and an isomorphism
$h_i:\mathscr{F}\mid_{X\times f^{-1}(U_i)}\xrightarrow{\sim}((1\times
f)^{\ast}\mathscr{E}_i)\otimes \mathscr{O}_T F_i$ for every $i$. We
show that two quasi-families $\mathscr{F}$ and $\mathscr{G}=(1\times
f)^{\ast}\mathscr{E}$ on $X\times T$ are equivalent. Define the
homomorphism
$\varphi:\mathscr{G}\otimes\pi^{\ast}\pi_{\ast}\hom\mathscr{O}_{X\times
T}(\mathscr{G},\mathscr{F})\to \hom\mathscr{O}_{X\times
T}(\pi^{\ast}\pi_{\ast}\End_{\mathscr{O}_{X\times
T}}(\mathscr{G}),\mathscr{F})$ by $\varphi(g\otimes f)(e)=f(e(g))$\pageoriginale for
every $g\in \mathscr{G}$,
$f\in \pi^{\ast}\pi_{\ast}\hom\mathscr{O}_{X\times
T}(\mathscr{G},\mathscr{F})$ and
$e\in \pi^{\ast}\pi_{\ast}\End\mathscr{O}_{X\times T}(\mathscr{G})$,
where $\pi$ is the projection of $X\times T$ onto $T$. By using the
isomorphisms $h_i$, it can be easily checked that this $\varphi$ is an
isomorphism. Since $\pi_{\ast}\hom_{\mathscr{O}_{X\times T}}
(\mathscr{G},\mathscr{F})$ and $\pi_{\ast}\End_{\mathscr{O}_{X\times
T}}(\mathscr{G})$ are vector bundles on $T$, two quasi-families $F$
and $G$ are equivalent.
\enprf
\end{Proof}

A quasi-universal family of similitudes $1$ is nothing but a universal
family. On the existence of a universal family, we have the following
by an argument similar to the above and by an idea in \cite{key16}
(and its improvement Theorem 6.11 in \cite{key9}).

\begin{Thm}\label{Thm6}
Let the assumption be same as in above theorem. Let $\mu$ be the
greatest common divisor of $\mathcal{X}(E\otimes N)$, where $E$ is a
member of $\mathscr{M}$ and $N$ runs over all vector bundles on
$X$. If $\mu=1$, then there exists a universal on $X\times M$. 
\end{Thm}

\begin{Proof}
Let $\mu_0$ be the greatest common divisor of $\mathcal{X}(E\otimes
N)$, where $N$ runs over all vector bundles on $X$ which satisfy 
($\ast$) all higher cohomology groups $H^{i}(X, E\otimes N)$ vanish
for every member $E$ of $M$. 

We show that $\mu_0=1$. Let $\mathscr{O}_X(1)$ be an ample line bundle
on $X$. Then there exists an integer $m_0$ such that $N(m)$ satisfies
$(\ast)$ for every $m\geqq m_0$. $\mathcal{X}(E\otimes N(m))$ is
divisible by $\mu_0$ for every $m\leqq m_0$ by definition. Since
$Q\mathcal{X}(E\otimes N(m))$ is a numerical polynomial on $m$,
$\mathcal{X}(E\otimes N)$ is divisible by $\mu_0$. Since $N$ is an
arbitrary vector bundle, $\mu_0$ divides $\mu$ and hence $\mu_0=1$ by
our assumption.\pageoriginale Therefore, there exist vector bundles
$N_j$ with the property $(\ast)$ and integers
$a_{\upsilon}(1\leqq \upsilon \leqq n)$ 
such that $\sum\limits_{\upsilon=1}a_{\upsilon}\mathcal{X}(E\otimes
N_{\upsilon})=-1$. Let $M=\cup_{i}U_i,\mathscr{E}_i$ and
$f_{ij}:\mathscr{E}_{i \ X\times (U_i\cap
U_j)}\xrightarrow{\sim}\mathscr{E}_{j \ X\times (U_i\cap U_j)}$ be same as
in the proof of Theorem A.\ref{Thm5}. By the Property
$(\ast),\pi_{i,\ast}(\mathscr{E}_i\otimes \pi_{X}\ast N_{\upsilon})$
is a vector bundle of rank $\mathcal{X}(E\otimes N_{\upsilon})$ on
$U_i$ for every $i$ and $\upsilon$. Put $L_i=\bigotimes\limits_{\upsilon
=1}^{n}\det(\pi_{i,\ast}(\mathscr{E}_i\otimes \pi^{\ast}_{X}N_{\upsilon}))^{\otimes
a_{\upsilon}}$, where $\det$ denotes the highest nonzero exterior power of
a vector bundle. The isomorphism $f_{ij}$ induces the isomorphism
$p_{ij}:L_{i}\mid_{U_i\cap U_j}\xrightarrow{\sim}L_j\mid_{U_i\cap
U_j}$ for every $i,j$. By the same argument as in
Theorem~A.\ref{Thm5}, we can show that
$\mathscr{E}_i\otimes \pi^{\ast}_{i}L_i$ on $X\times U_i$ can be glued
together by the isomorphisms $f_{ij}\otimes p_{ij}$ and we obtain a
sheaf $\mathscr{E}$ on $X\times M$ whose restriction to $X\times U_i$
is isomorphic to $\mathscr{E}_i\otimes \pi_i^{\ast}L_i$ for every
$i$. Since $\mathscr{E}_i$ are universal families, $\mathscr{E}$ is a
universal family.
\enprf
\end{Proof}

\setcounter{rem1}{6}
\begin{rem1}
If $X$ is smooth, then every sheaf on $X$ has a resolution by a
locally free sheaves. Hence $\mu$ in the theorem is equal to the
greatest common divisor of $\mathcal{X}(E\otimes N')$ where
$E\in \mathscr{M}$ and $N'$ runs over all sheaves on $X$. If $X$ is
smooth and $\dim X=2$, then $\mu$ is equal to the greatest common
divisor of $r(E), (c_1(E).D)$ and $\mathcal{X}(E)$, where $D$ runs
over all divisors of $X$. 
\end{rem1}

\begin{thebibliography}{99}
\bibitem{key1}
{Atiyah, M. F.}: Vector bundles over an elliptic curve,
\textit{Proc.London Math. Soc.}, 7(1957)., 414--452.

\bibitem{key2}
{Gieseker, D.}: On the moduli space of vector bundles on an
algebraic surface, \textit{Ann. of Math}., 106(1977), 45--60.

\bibitem{key3}
{Hartshorne, R.}:\pageoriginale \textit{Residues and duality}, Lecture Notes
in Mathematics, Vol. 20. Berlin-Heidelberg-Now York: Springer 1966.

\bibitem{key4}
{Inose, H.}: Defining equations of singular $K3$ surfaces and a
notion of isogeny, \textit{Intl. Symp. on Algebraic Geometry} Kyoto
1977, 495--502: Kinokuniya 1978.

\bibitem{key5}
{Kleiman, S.}: Les th\'{e}or\'{e}mes des finitude pour le
foncteur de Picard, \textit{S\'{e}minaire de Geometrie Alg\'{e}brique
  du Bois Marie}, 1966/67 (S.G.A.6), Expos\'e XIII, Lecture Notes in
Math., Vol. 225, Berlin-Heidelberg-New York: Springer 1971.

\bibitem{key6}
{Langton, S. G.}: Valuative criteria for families of vector
bundles on algebraic varieties, \textit{Ann. of math.,} 101, (1975)
88--110.

\bibitem{key7}
{Looijenga, E., Peters. C.}: Torelli theorems for K\"{a}haler
$K3$ surfaces, \textit{Compos. Math.,} 42(1981), 145--186.

\bibitem{key8}
{Maruyama, M.}: Stable vector bundles on an algebraic surface,
\textit{Nagoya Math. J}., 58(1975), 25--68.

\bibitem{key9}
{Maruyama, M.}: Moduli of stable sheaves, II, \textit{J. Math. Kyoto
Univ.,} 18(1978), 557--614.

\bibitem{key10}
{Morrison. D. R.}: On $K3$ surfaces with large Picard number
\textit{Invent. Math.,} 75(1984), 105--121.

\bibitem{key11}
{Mukai, S.}:\pageoriginale On the classification of vector bundles over an
abelian surface (in Japanese), \textit{Recent Topics in Algebraic
  Geometry}, RIMS Kokyuroku 409(1989), 103--127. 

\bibitem{key12}
{Mukai, S.}: Symplectic structure of the moduli space of
sheaves on an abelian or $K3$ surface, \textit{Invent, math.,} 77
(1984), 101--116.

\bibitem{key13}
{Mukai, S.}: On the moduli space of bundles on $K3$ surfaces,
II (in preparation)

\bibitem{key14}
{Mukai, S.}: On reflection functors. (in preparation)

\bibitem{key15}
{Mumford, D.}: \textit{Abelian varieties}, Oxford University
Press 1974.

\bibitem{key16}
{Mumford, D., Newstead, P. E.}: Periods of a moduli space of
bundles on curves, \textit{Amer. J. Math.,} 90 (1968), 1200--1208.

\bibitem{key17}
{Nikulin, V. V.}: Integral symmetric bilinear forms and some of
their applications, English translation, \textit{Math. USSR
  Izvestija}, 14(1980), 103--167.

\bibitem{key18}
{Oda, Tadao}: Vector bundles on an elliptic curve,
\textit{Nagoya Math. J.,} 43(1971), 41--171.

\bibitem{key19}
{Oda, Takayuki}: A note on the Tate conjecture for $K3$
surfaces, \textit{Proc. Japan Acad.,} Ser. A 56(1980), 296--300.

\bibitem{key20}
{Piateckii-Shapiro, I., Shafarevitch, I. R.}:\pageoriginale A Torelli theorem
for algebraic surfaces of type $K3$, English translation,
\textit{math. USSR Izvestija } 5(1971), 547--587.

\bibitem{key21}
{Shafarevitch, I. R.}: Le th\'{e}or\`{e}me de Torelli pour
les surfaces algebraique de type $K3$, \textit{Acte Congr\'{e}s
  Intl. Math.,} 1(1970), 413--417.

\bibitem{key22}
{Shioda, T., Inose, H.}: On singular $K3$ surfaces,
\textit{Complex analysis and Algebraic Geometry}, Iwanami-Shoten,
Tokyo (1977), 113--136.

\bibitem{key23}
{Todorov, A. N.}: Applications of the K\"{a}hler-Einstein
Calabi-Yau metric to moduli of $K3$ surfaces, \textit{Invent. math.,}
61(1980), 251--265.
\end{thebibliography}

\vskip 1cm

\noindent
Department of Mathematics\\
Nagoya University\\
Furo-cho, Chikusa-ku\\
Nagoya, 464 Japan
