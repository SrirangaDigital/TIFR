\title{Line Bundles On Schubert Varieties}
\markright{Line Bundles On Schubert Varieties}

\author{By C. S. Seshadri}
\markboth{C. S. Seshadri}{Line Bundles On Schubert Varieties}

\date{}
\maketitle

\setcounter{page}{379}
\setcounter{pageoriginal}{498}

\section{Introduction}\label{s1}\pageoriginale

Recently a gap has been found by V. Kac in the proof of the main
results of the important work of Demazure \cite{key1} on Schubert
varieties in flag varieties associated to a semi-simple algebraic
group $G$. The purpose of this paper is to justify this work of
Demazure; in fact, the conjectures stated $b$. Demazure in his paper,
which essentially mean that his main results hold in arbitrary
characteristic or even over $\spec\ \mathbb{Z}$, also follow.    

The main contribution of this paper is the proof of the normality of a Schubert variety (in arbitrary characteristic). This, together with a recent beautiful work of V. B. Mehta and A. Ramanathan \cite{key4}, give the required justification of the work of Demazure, as well as his conjectures over $\spec\ \mathbb{Z}$. 

Recently, A. Joseph \cite{key2} has justified ``Demazure's character formula'' for large dominant weights (in characteristic zero). It can be easily seen that this is equivalent to showing that a Schubert\pageoriginale variety is normal (in characteristic zero). Combined with the above cited work of V. B. Mehta and A. Ramanathan, this work of Joseph also leads to a justification of the main results of Demazure \cite{key1}, but not his conjectures over $\spec\ \mathbb{Z}$. 

When $G$ is a \textit{classical group}, the work of Demazure (as well as his conjectures over $\spec\ \mathbb{Z}$) is also a consequence of ``Standard Monomial Theory'' (cf. \cite{key3}, \cite{key5}, \cite{key6})\footnote{In the lecture, we spoke only about this. The written version presents a later development.}

\bigskip
\noindent
\textbf{Addendum (January, 1985)}

This paper was written around May 1984 and represents the state of affairs around that time in justifying the work of Demazure \cite{key1}. Soon after, S. Ramanan and A. Ramanathan (cf. Projective normality of Flag Varieties and Schubert Varieties'' Invent. math., 79(1985), 217--224) proved Theorem~\ref{thm2} of this paper as well as the normality of Schubert varieties by methods related to \cite{key4}. Recently, A. Ramanathan (cf. ``Schubert varieties are arithmetically cohen-Macaulay''. Invent. Math. 80 (1985) 283--294) has proved the arithmetic Cohen-Macaulay nature of Schubert varieties in \textit{arbitrary characteristic} (cf. Remark~\ref{rem5} of this paper for the case of characteristic zero). The arithmetic Cohen-Macaulay nature of Schubert varieties (for arbitrary characteristic) was known earlier only for special classes of Schubert varieties as a consequence of standard monomial theory (e.g. $G=SL(n))$. 

\section{Normality of a Schubert variety}\label{s2}\pageoriginale

Let $G$ be a semi-simple, simply-connected, Chevalley group defined over a field $k$. Let us fix a maximal torus $T$ and a Borel subgroup $B$ containing $T$. Let $W=\dfrac{N(T)}{T}$ be the Weyl group, $N(T)$ being the normalizer of $T$. We talk of roots, weights etc. with respect to $T$, $B$. 

For $w\in W$, let $e(w)$ denote the image in $\dfrac{G}{B}$ of a representative of $w$ in $N(T)$. It is a fixed point for the canonical action of $T$ on $\dfrac{G}{B}$. Let $X(w)$ denote the closure in $\dfrac{G}{B}$ of the ``Bruhat cell'' $B\ e(w)$ $\left(\text{ in the Zariski topology of } \dfrac{G}{B}\right)$, endowed with canonical structure of a reduced subscheme of $\dfrac{G}{B}$. We call $X(w)$ the \textit{Schubert variety} assoicated to $w\in W$. Then we have the following: 

\begin{thm}\label{thm1}
Every Schubert variety $X(w)$ in $\dfrac{G}{B}\;(w \in W)$ is normal. 
\end{thm}

\begin{Proof}
The proof is by decreasing induction on the length 
$$
l(w)=\dim X(w)\quad\text{of}\quad X(w).
$$ 
If $w=w_0$, the element of maximal length in $W$; $X(w_0)=\dfrac{G}{B}$ which is smooth, in particular, therefore normal. We assume now that every Schubert variety $X$, such that $\dim X>l(w)$, is normal. 

Let $\tau=s_{\alpha}w$  be such that $\alpha$ is a simple root and $l(\tau)=l(W)+1$, so that $X(w)\subset X(\tau)$ and $l(w)=l(\tau)-1$. By the induction hypothesis, $X(\tau)$ is normal. Let $P_{\alpha}$ be the minimal parabolic subgroup of $G$, associated to $\alpha$. Then one knows that $X(\tau)$ is $P_{\alpha}$-stable. Then we have a canonical morphism:
$$
\begin{cases}
P_{\alpha}\times X(w)\to X(\tau) \text{ defined by }\\
(p,x)\to p\cdot x
\end{cases}
$$
\pageoriginale
Set $Z=P_{\alpha}\times ^{B} X(w)$ i.e. the set of equivalence classes under the equivalence relation 
$$
(p,x)\sim \left(pb, b^{-1} x\right)\text{ for some } b\in B (p \in P_{\alpha}, X \in X w).
$$
Then the above morphism goes down to a morphism: 
$$
\Psi: Z \to X(\tau)
$$
It is seen without much difficulty that $\Psi$ is a birational $P_{\alpha}$ morphism and that we have a canonical morphism: 
$$
p:Z\to \mathbb{P}^{1}, \mathbb{P}^{1}\simeq \dfrac{P_{\alpha}}{B}
$$
We note that $p$ is a (locally trivial) fibre space with base $\mathbb{P}^{1}$ and fibre $X(w)$; in fact, it is the fibre space with fibre $X(w)$, associated to the principal fibration $P_{\alpha} \to \mathbb{P}^{1}$ with structure group $B$ ($B$ acts on $X(w)$). 

Let us take the normalization morphism $\widetilde{X(w)}\to X(w)$. Observe that we have a natural action of $B$ on $\widetilde{X(w)}$ and that this map is a $B$-morphism. Set $\widetilde{Z}=P_{\alpha}\times^{B} \widetilde{X(w)}$. We denote by $\widetilde{\Psi}$ the canonical morphism: 
$$
\widetilde{\Psi}:\widetilde{Z}\to X(\tau)
$$
Again\pageoriginale we have a canonical morphism 
$$
\widetilde{p}:\widetilde{Z}\to \mathbb{P}^{1}
$$
which is a fibre space over $\mathbb{P}^{1}$ with fibre $\widetilde{X(w)}$. Then we have the following: 
\begin{enumerate}
\renewcommand{\theenumi}{\roman{enumi}}
\renewcommand{\labelenumi}{(\theenumi)}
\item $\widetilde{Z}\xrightarrow{v}Z\xrightarrow{\Psi}X(\tau), \widetilde{\Psi}=\overline{\Psi}\circ v$

\item $\xymatrix{
\widetilde{Z}\ar[dr]_{\widetilde{P}}\ar@{->}[rr]^{v}&&Z\ar[dl]^{P} & \widetilde{p}=p\circ v\\
&\mathbb{P}^{1}&}$
\end{enumerate}

Let $\mathscr{F}$ be an object on $X(w)$
$\left(\resp \widetilde{X(w)}\right)$, say a line bundle on which $B$
acts consistent with the canonical action of $B$ on
$X(w) \left(\resp \widetilde{X(w)}\right)$. We denote by $\mathscr{F}$
the ``associated'' object on $Z$ $\left(\resp \widetilde{Z}\right)$
i.e. $\mathscr{F}=P_{\alpha}\times^{B}\mathscr{F}$. For a line bundle
$M$ on $\dfrac{G}{B}$, we denote by $M(\tau) \left(\resp M(w)\right)$
the restriction of $M$ to $X(\tau)$ $\resp X(w)$) and by
$\widetilde{M(w)}$ the pullback of $M(w)$ to $\widetilde{X(w)}$. We
have then the following
\end{Proof}

\begin{lem}\label{lem1}
\begin{enumerate}
\renewcommand{\theenumi}{\roman{enumi}}
\renewcommand{\labelenumi}{(\theenumi)}
\item $\Psi^{\ast}(M(\tau))\simeq M(w)^{\sharp}$
\item $\widetilde{\Psi}^{\ast}(M(\tau))\simeq \widetilde{M(w)}^{\sharp}$
\end{enumerate}
\end{lem}

\begin{Proof}
We observe that $\Psi$ and $\widetilde{\Psi}$ are $P_{\alpha}$-morphisms. Now $M$ is a homogenous line bundle on $G/B$ i.e. $G$ acts on $M$. Hence $P_{\alpha}$ acts on $M(\tau)$, so that $P_{\alpha}$ acts on $\Psi^{\ast}(M(\tau)) \left(\resp \widetilde{\Psi}^{\ast}(M(\tau))\right)$. Suppose now that $H$ is a group, $K$ a subgroup and $\mathscr{G}$ an object on\pageoriginale the space $H/K$ on which $H$ acts (consistent with its action on $H/K$). Let $\mathscr{G}_0$ denote the ``fibre'' of $\mathscr{G}$ over the point $e \in H/K$ corresponding to the coset $K$. We see that the isotropy subgroup of $H$ at $e$, namely $K$, acts on $\mathscr{G}_0$. If $\mathscr{G}_0^{\sharp}$ denotes the object over $H/K$. ``associated'' to the $K$-principal fibre space $H\to H/K$, we see easily that $\mathscr{G}\simeq \mathscr{G}_0^{\sharp}$ Applying this principle to our case, by taking $H=P_{\alpha}$, $K=B$ and $\mathscr{G}=\Psi^{\ast}(M(\tau))\left(\resp \widetilde{\Psi}^{\ast}(M(\tau))\right)$, we find that $\mathscr{G}_0\simeq M(w) \left(\resp \widetilde{M(w)}\right)$ and the lemma follows. 

Let us fix an ample line bundle $L$ on $G/B$ (say, associated to \\$\rho=\left(\dfrac{1}{2} \text{sum of positive roots}\right)$. We use also the notation of Serre, say $\mathscr{O}_{X(\tau)}(m)$ etc. for the restriction of $L^{m}$ to $X(\tau)$ (or the associated coherent sheaves). Set 
$$
V_m=H^{0}(X(w), \mathscr{O}_{X(w)}(m)), \widetilde{V_m}=H^{0}\left(\widetilde{X(w)}, \mathscr{O}_{\widetilde{X(w)}}(m)\right)
$$
where $\mathscr{O}_{\widetilde{x(w)}}(1)$ denotes the pull-back of $L$ to $\widetilde{X(w)}$ (note that it is again ample). Now $V_m$ and $\widetilde{V_m}$ are $B$ modules. We denote by $V_m$ $\left(\resp \widetilde{V}_m\right)$ the vector bundles on $\mathbb{P}^{1}$, associated to the principal $B$-fibration $P_{\alpha}\to \mathbb{P}^{1}$ with fibre $V_m$ $\left(\resp \widetilde{V}_m\right)$. Then we have the following: 
\end{Proof}

\begin{lem}\label{lem2}
$H^{0}\left(\mathbb{P}^{1}, V_m\right)\simeq H^{0}\left(\mathbb{P}^{1},\widetilde{V}_m\right)\simeq H^{0} (X(\tau),\mathscr{O}_{X(\tau)}(m))$ for $m\geq 0$. 
\end{lem}

\begin{Proof}
One knows that the morphism $\Psi$ and $\widetilde{\Psi}$ are birational. Since $X(\tau)$ is normal, it follows that 
$$
\Psi_{\ast}(\mathscr{O}_z)\left(\resp \widetilde{\Psi_{\ast}}(\mathscr{O}_{\widetilde{Z}})\right) \simeq \mathscr{O}_{X(\tau)}.
$$
Hence
\begin{equation*}\label{eq1}
\begin{cases}
H^{0}(Z,\Psi^{\ast}\left(L(\tau)^{m}\right))\simeq H^{0}(X(\tau),(L(\tau)^{m}))\\
=H^{0}(X(\tau), \mathscr{O}_{X(\tau)}(m))\\
H^{0}(\widetilde{Z}, \widetilde{\Psi}^{\ast} (L(\tau)^{m})\simeq H^{0}(X(\tau),L(\tau)^{m})\\
=H^{0}(X(\tau), \mathscr{O}_{X(\tau)}(m))
\end{cases}\tag{1}
\end{equation*}
\pageoriginale
We observe now that $P_{\alpha}$ acts on $p_{\ast}\left(\Psi^{\ast}(L(\tau)^{m})\right)\left(\resp \widetilde{p}_{\ast}\left(\widetilde{\Psi}^{\ast}(L(\tau)^{m})\right)\right)$ consistent with its action on $\mathbb{P}^{1}$, since $\Psi$ and $p (\resp \widetilde{\Psi}$ and $(\widetilde{p})$ are $P_{\alpha}$ -morphisms. It follows then easily that these coherent sheaves on $\mathbb{P}^{1}$ are \textit{locally free}; in fact, by Lemma~\ref{lem1} (and an argument similar to its proof), we see that 
\begin{equation*}\label{eq2}
\begin{cases}
p_{\ast}\left(\Psi^{\ast}(L(\tau)^{m})\right) \simeq V_m\\
\widetilde{p}_{\ast} \left(\widetilde{\Psi}^{\ast}(L(\tau)^{m})\right)\simeq \widetilde{V}_m
\end{cases}\tag{2}
\end{equation*}
On the other hand, we have 
\begin{equation*}\label{eq3}
\begin{cases}
H^{0}\left(\mathbb{P}^{1}, p_{\ast}\left(\Psi^{\ast}(L)(\tau)^{m}\right)\right) \simeq H^{0}(Z, \Psi^{\ast}(L(\tau)^{m}))\\
\left(\resp H^{0}\left(\mathbb{P}^{1}, p_{\ast}\left(\widetilde{\Psi}^{\ast}\left(L(\tau)^{m}\right)\right)\right) \simeq H^{0}\left(\widetilde{Z}, \widetilde{\Psi}^{\ast}\left(L(\tau)^{m}\right)\right)\right)
\end{cases}\tag{3}
\end{equation*}
Now (\ref{eq1}), (\ref{eq2}) and (\ref{eq3}) imply Lemma~\ref{lem2}.
\end{Proof}

\begin{rem}\label{rem1}
For the proof of Theorem~\ref{thm1}, it suffices to use Lemma~\ref{lem2} for $m>>0$ 

Let\pageoriginale us now set 
$C=\mathscr{O}_{\widetilde{X}(w)}/\mathscr{O}_{X(w)}$ a coherent $\mathscr{O}_{X(w)}$ module. We have then the exact sequence of $\mathscr{O}_{X(w)}$ modules: 
$$
0\to \mathscr{O}_{X(w)}\to \mathscr{O}_{\widetilde{X(w)}}\to C\to 0
$$
Tensoring by $L(w)^{m}$, we obtain the exact sequence of $\mathscr{O}_{\widetilde{X(w)}}$-modules: 
\begin{equation*}\label{eq4}
0\to \mathscr{O}_{X(w)}(m)\to \mathscr{O}_{\widetilde{X(w)}}(m)\to C\to 0\tag{4}
\end{equation*}
(we use the principle that if $\delta:\widetilde{X(w)}\to X(w)$ is the canonical morphism and $N$ a line bundle (or invertible sheaf) on $X(w)$, we have 
\begin{equation*}
\delta_{\ast}\left(\delta^{\ast}N\simeq \mathscr{O}_{\widetilde{X(w)}}\otimes_{\mathscr{O}_{X(w)}} N\right)
\end{equation*}
We observe that 
\begin{equation*}
\begin{aligned}
&H^{0}(X(w), \mathscr{O}_{\widetilde{X(w)}}(m))\simeq H^{0} \left(\widetilde{X(w)}, \widetilde{L(w)}^{m}\right)\simeq \widetilde{V}_m\\ &{}\left(\widetilde{L(w)} \text{ pull-back of } L(w) \text{ on } \widetilde{X(w)}\right).
\end{aligned}
\end{equation*}
Let us now choose $m_0$ such that for $m\geq m_0$, one has 
\begin{equation*}\label{eq5}
\begin{cases}
(a)~ H^{1}(X(w), \mathscr{O}_{X(w)}(m))=0, \text{ and }\\
(b)~ H^{0}(X(\tau), \mathscr{O}_{X(\tau)}(m))\to H^{0}(X(w), \mathscr{O}_{X(w)}(m))\to 0\\
\text{ is exact. }
\end{cases}\tag{5}
\end{equation*}
Then\pageoriginale writing the cohomology exact sequence of (\ref{eq4}), we get by (\ref{eq5})\\(a): 
\begin{equation*}\label{eq6}
\begin{cases}
0\to V_m\to \widetilde{V}_m \to H^{0}(X(w), C(m))\to 0\\
\text{ exact for } m\geq m_0. 
\end{cases}\tag{6}
\end{equation*}
Set 
\begin{equation*}
W_m=H^{0}(X(w), C(m))
\end{equation*}
Now $C(m)$ is a coherent $\mathscr{O}_{X(w)}$ -module on which $B$ operates consistent with its action on $X(w)$ (since $\widetilde{X(w)}\to X(w)$ is a $B$ morphism etc.). Hence $W_m$ is also a $B$-module. We denote by $W_m$ the vector bundle on $\mathbb{P}^{1}$, associated to the $B$ module $W_m$ Then we get the following exact sequence of vector bundles 
\begin{equation*}\label{eq7}
0\to V_m\to \widetilde{V}_m\to W_m\to 0\tag{7}
\end{equation*}
Now we claim that 
\begin{equation*}\label{eq8}
H^{1}\left(\mathbb{P}^{1}, V_m\right)=0, m\geq m_0. \tag{8}
\end{equation*}
To prove (\ref{eq8}), let $E=H^{0}(X(\tau)$, $\mathscr{O}_{X(\tau)}(m))$. Then the canonical map $E\to V_m$ is \textit{surjective} for $m\geq m_0$ by (\ref{eq5}) (b). let $K=\ker (E\to V_m)$. Observe that $E$ is a $P_{\alpha}$ -module (since $X(\tau)$ is $P_{\alpha}$ -stable). Hence the associated vector bundle $E$ on $\mathbb{P}^{1}$ is trivial. Thus we get the following exact sequence of vector bundles on $\mathbb{P}^{1}$: 
\begin{equation*}\label{eq9}
0\to K\to E\to V_m\to 0\tag{9}
\end{equation*}
\pageoriginale
Since $E$ is trivial, $H^{1}\left(\mathbb{P}^{1}, E\right)=0$. Hence, writing the cohomology exact sequence of (\ref{eq9}), we deduce that $H^{1}\left(\mathbb{P}^{1}, V_m\right)=0$ for $m\geq m_0$ (using the fact $H^{2}\left(\mathbb{P}^{1}, K\right)=0$. This proves the assertion (\ref{eq8}). 

Using the assertion (\ref{eq8}), the cohomology exact sequence of (\ref{eq7}) leads to the following exact sequence: 
\begin{equation*}\label{eq10}
0\to H^{0}\left(\mathbb{P}^{1}, V_{m}\right)\to H^{0}\left(\mathbb{P}^{1}, V_{m}\right)\to H^{0}\left(\mathbb{P}^{1}, W_m\right)\to 0, m\geq m_0, \tag{10}
\end{equation*}
Now Lemma~\ref{lem2} implies that 
\begin{equation*}
H^{0}\left(\mathbb{P}^{1}, V_m\right)\xrightarrow{\sim} H^{0}\left(\mathbb{P}^{1}, \widetilde{\mathbb{V}}_m\right), m\geq 0. 
\end{equation*}
Hence, we conclude that 
\begin{gather*}
H^{0}\left(\mathbb{P}^{1}, W_m\right)=0 \text{ for } m\geq m_0.\\
(\text{Recall that } W_m=H^{0} (X(w), C(m))). 
\end{gather*}
Observe that 
\begin{equation*}
C\neq 0 \Leftrightarrow X(w) \text{ is not normal. }
\end{equation*}
Further,\pageoriginale if $Q$ is the parabolic subgroup of $G$ generated by $B$ and the minimal parabolic subgroups $P_{\beta}$ ($\beta$ simple) such that $P_{\beta}$ leaves $X(w)$-stable, we see that $Q$ leaves $X(w)$ stable and $C$ is a $Q-\mathscr{O}_{X(w)}$ -module. Hence, the required normality of $X(w)$ is a consequence of the following: 
\end{rem}

\begin{lem}\label{lem3}
Let $\mathscr{F}$ be a coherent $Q-\mathscr{O}_{X(w)} $ module. Let $W_m=H^{0}(X(w)$, $\mathscr{F}(m))$ (note that $\mathscr{F}(m)$ is also a $Q-\mathscr{O}_{X(w)}$-module). Then if $\mathscr{F}\neq 0$, there is a simple root $\alpha$ which moves $X(w)$ (i.e. if $\tau=s_{\alpha}w, \tau >w$) such that 
$$
H^{0}\left(\mathbb{P}^{1}, W_m\right)\neq 0 \text{ for } m>> 0
$$
(where $W_m$ is the vector bundle on $\mathbb{P}^{1}=P_{\alpha}/B$, defined as above, associated to $W_m$). 
\end{lem}

\begin{Proof}
Let $J=\Ann \mathscr{F}$ (annihilator of $\mathscr{F}$ - an ideal sheaf of $\mathscr{O}_{X(w)}$). Let $\mathscr{O}_Y=\mathscr{O}_{X(w)/J}$. We observe that $J$ is a $Q$-sheaf or that $Y$ is a $Q$-scheme. Then $Y_{\red}$ is also a $Q$-scheme. We observe that $\mathscr{F}$ ``lives on'' $Y$ i.e. it is the canonical extension of a $Q-\mathscr{O}_Y$-module. We denote this $Q-\mathscr{O}_Y$ -module by the same letter $\mathscr{F}$. Let $I$ be the ideal sheaf of $\mathscr{O}_Y$ such that 
$$
\mathscr{O}_{Y_{\red}}.=\dfrac{\mathscr{O}_Y}{I}
$$
We observe that $I$ is a $Q-\mathscr{O}_Y$-module and that there exists an integer $n$ such that 
$$
I^{n} \mathscr{F}=(0)\text{ and } I^{n-1}\mathscr{F}\neq (0), n\geq 1. 
$$
(note\pageoriginale $\mathscr{F}\neq (0)$). If $n=1$, the above relation means that $I\cdot \mathscr{F}=(0)$. Set $\mathscr{G}=I^{n-1}\mathscr{F}$. Since all the powers of $I$ are again $Q-\mathscr{O}_Y$ sheaves, we see that $\mathscr{G}$ is also a $Q-\mathscr{O}_Y$ sheaf. Further $\mathscr{G}\neq (0)$. We observe that it suffices to prove the lemma for $\mathscr{G}$ for if $W'_m=H^{0}(X(w), \mathscr{G}(m))$, we have the exact sequence: 
$$
0\to H^{0}(X(w), \mathscr{G}(m))\to H^{0}(X(w), \mathscr{F}(m))
$$
which gives the exact sequence of vector bundles on $\mathbb{P}^{1}$. 
$$
0\to W'_m\to W_m
$$
which, in turn, gives the exact sequence 
$$
0\to H^{0}\left(\mathbb{P}^{1}, W'_m\right)\to H^{0}\left(\mathbb{P}^{1}, W_m\right)
$$
Hence 
$$
H^{0}\left(\mathbb{P}^{1}, W_m\right) \neq (0)\Rightarrow H^{0}\left(\mathbb{P}^{1}, W_m\right) \neq (0). 
$$
This proves the above claim that it suffices to prove the lemma for $\mathscr{G}$. 

We observe that $I\cdot \mathscr{G}=0$ i.e. $\mathscr{G}$ lives on $Y_{\red}$. Let $Y_1,\ldots, Y_r$ denote the distinct irreducible components of $Y_{\red}$. These are all $Q$-stable ($Q$ is connected). We can also suppose that $G$ does not live on $Y'$ such that $Y'\neq Y$ and $Y'$ is the union of some irreducible components of $Y$ (otherwise, we replace $Y$ by the corresponding union etc.). Let $I_1$ be the ideal sheaf\pageoriginale of $\mathscr{O}_{Y_{\red}}$ defining the closed subscheme $Y_1$ of $Y_{\red}$ and $I_2$ the ideal sheaf of $\mathscr{O}_{Y_{\red}}$ defining the closed subschemes $Y_2\cup\cdots \cup Y_r$ of $Y_{\red}$. Set $\mathscr{G}_1=I_2\cdot \mathscr{G}$. Then $\mathscr{G}_1$ is a $Q-\mathscr{O}_{Y_{\red}}$ subsheaf of $\mathscr{G}$ (the sheaf of sections of $\mathscr{G}_1$ are those of $\mathscr{G}$ which vanish on $Y_2\cup\cdot \cup Y_r)$. Observe that $I_1$ annihilates $\mathscr{G}_1$, so that $\mathscr{G}_1$ lives on $Y_1$ $\left(\text{ for } \mathscr{O}_{Y_1}=\dfrac{\mathscr{O}_{Y_{\red}}}{I_1}\right)$. Now $\mathscr{G}_1\neq (0)$ for if $G_1=I_2 \mathscr{G}=(0)$, $\mathscr{G}$ would live on $Y_2\cup\cdots \cup Y_r$, which is not the case. Thus, by an argument as above, it suffices to prove the lemma for $\mathscr{G}_1$. 

Let $\mathscr{G}$ denote the torsion subsheaf of $\mathscr{G}_1$, $\mathscr{G}_1$ being now considered as a $Q-\mathscr{O}_{Y_1}$ -module $(Y_1-a Q \text{ stable Schubert subvariety of } X(w))$. We observe that $\mathscr{G}'$ is a $Q-\mathscr{O}_{Y_1}$ -sub-module of $\mathscr{G}_1$. If $\mathscr{G}'\neq (0)$, it suffices to prove the lemma for $\mathscr{G}'$. Observe that the support of $\mathscr{G}'$ is a closed subscheme of $Y_1$, \textit{properly contained in} $Y_1$. Repeating the above procedure, we would get a $Q-\mathscr{O}_{Y_2}$ - module $\mathscr{G}_2$, $\mathscr{G}_2\neq (0)$, such that $Y_2$ is a Schubert variety \textit{properly contained in} $Y_1$, and it would suffice to prove the lemma for $\mathscr{G}_2$. If we repeat this procedure and everytime we get that the torsion subsheaves $\mathscr{G}'_2\mathscr{G}'_3,\ldots$ etc. are $\neq(0)$, we would get an infinite strictly decreasing chain of Schubert varieties, which would lead to a contradiction. Therefore, at some point the torsion sheaf would be (0). 

Thus, as a consequence of the above method, it would suffice to prove the lemma when $\mathscr{F}$ lives on a Schubert subvariety $X(\theta)$ of $X(w)$, such that considered as a sheaf on $X(\theta)$, $\mathscr{F}$ is \textit{torsion free} (in particular $\neq (0)$). Note that $\mathscr{F}$ is a $Q-\mathscr{O}_{X(\theta)}$-module. Hence $X(\theta)$ is $Q$-stable. Let $\alpha$ be a simple root which moves $X(\theta)$\pageoriginale i.e. if $\varphi=s_{\alpha}\theta$, then $\varphi>\theta$. Note that $\alpha$ also moves $X(w)$, for otherwise, $P_{\alpha}$ leaves $X(w)$ stable i.e. $P_{\alpha}\subset Q$ and hence $P_{\alpha}$ leaves $X(\theta)$ stable, which is not the case. 

Now by the Borel fixed point theorem, $\mathscr{F}(m_0)$ and a non-zero section $s\in \mathscr{F}(m_0)$ such that the line through $s$ is $B$-fixed. Now multiplication by $s$ induces an inclusion 
$$
\begin{aligned}
j:\mathscr{O}_{X(\theta)}&\to \mathscr{F}(m_0),\\
f&\to f\cdot s \cdot
\end{aligned}
$$ 
Let $\mathscr{G}$ denote the image of $\mathscr{O}_{X(\theta)}$ in $\mathscr{F}(m_0)$. Now $\mathscr{G}$ is a $B-\mathscr{O}_{X(\theta)}$ submodule of $\mathscr{F}(m_0)$. Note that $\mathscr{G}$ need not be $B$-isomorphic to $\mathscr{O}_{X(\theta)}$ (the action of $B$ on $\mathscr{G}$ differs from that on $\mathscr{O}_{X(\theta)}$ by a character of $B$, namely  the character $\mathcal{X}$ which defines the action of $B$ on the line through $s$). It would suffice to prove the lemma for $\mathscr{G}$ i.e. if 
$$
W'_m=H^{0}(X(\theta), \mathscr{G}(m))=H^{0}(X(w), \mathscr{G}(m)),  
$$
($\mathscr{G}$ being considered canonically as an $\mathscr{O}_{X(w)}$ -module), then $H^{0}\left(\mathbb{P}^{1}. W_m\right)\\\neq (0)$ for $m>>0$. Note that the action of $B$ on $\mathscr{G}(m)$ (for all $m$) differs by the same character $\mathcal{X}$ from the action on $\mathscr{O}_{X(w)}(m)$. 
Consider the canonical morphism 
$$
\Psi: Z=P_{\alpha}\times ^{B} X(\theta)\to X(\varphi)~(\varphi=s_{\alpha}\theta)
$$
We\pageoriginale note that $\Psi$ is \textit{birational} (since $\varphi > \theta$). We have again the fibration: 
$$
\begin{aligned}
&p:Z\to \mathbb{P}^{1}\left(=\dfrac{P_{\alpha}}{B}\right)\\
&(\text{fibre type } X(\theta))
\end{aligned}
$$
Consider the line bundle $\mathscr{G}(m)^{\sharp}$ on $Z$, which is associated to the $B-\mathscr{O}_{X(\theta)}$ -module $\mathscr{G}(m)$. We see that 
$$
\mathscr{G}(m)^{\sharp}=\left(\mathscr{O}_{X(\theta)}(m)^{\sharp}\right)\otimes N
$$
where $N$ is a line bundle on $Z$ (or rather the corresponding sheaf), which comes from $\mathbb{P}^{1}$ ($N$ is the line bundle on $\mathbb{P}^{1}=\dfrac{P_{\alpha}}{B}$, associated to the character $\mathcal{X}$ of $B$). We have seen that (by Lemma~\ref{lem1})
$$
\left(\mathscr{O}_{X(\theta)}(m)\right)^{\sharp}=\Psi^{\ast}\left(\mathscr{O}_{X(\phi)}(m)\right). 
$$
Then we get
$$
p_{\ast}\left(\mathscr{G}(m)^{\sharp}\right)=p_{\ast}\left(\mathscr{O}_{X(\varphi)}(m)^{\sharp}\right)\otimes N
$$ 
where we use the same notation $N$ for the line bundle on $\mathbb{P}^{1}$ as well as its inverse image by p. Setting $V_m$ to be $B$-module 
$$
V_m=H^{0}(X(\theta), \mathscr{O}_{X(\theta)}(m))
$$
and $V_m$ the vector bundle on $\mathbb{P}^{1}$, associated to the principal $B$-fibration $P_{\alpha}\to \mathbb{P}^{1}=\dfrac{P_{\alpha}}{B}$, we see that 
$$
p_{\ast}\mathscr{O}_{X(\theta)}(m)^{\sharp}\cong V_m
$$
the\pageoriginale proof being as in Lemma~\ref{lem2} (see (\ref{eq2}) in the proof of Lemma~\ref{lem2}). Hence 
$$
p_{\ast}\left(\mathscr{G}(m)^{\sharp}\right)\simeq V_m\otimes N.
$$
If now 
$$
W_m=H^{0}(X(w),\mathscr{G}(m)). 
$$
We see that 
$$
p_{\ast}\left(\mathscr{G}(m)^{\sharp}\right)\simeq W_m\simeq V_m\otimes N 
$$
We see also that 
$$
H^{0}\left(\mathbb{P}^{1}, W_m\right)\simeq H^{0}\left(Z,\mathscr{G}(m)^{\sharp}\right)
$$
Thus to show that 
$$
H^{0}\left(\mathbb{P}^{1}, W_m\right)\neq (0) \text{ for } m>>0
$$
it suffices to show that 
$$
H^{0}\left(Z,\mathscr{G}(m)^{\sharp}\right)\neq (0) \text{  for } m>>0. 
$$
i.e.
$$
H^{0}\left(Z, \Psi^{\ast}\left(\mathscr{O}_{X(\varphi)}(m)\right)\otimes \mathscr{O}_Z N\right) \neq (0) \text{ for }m >> 0. 
$$
Now\pageoriginale 
$$
H^{0}\left(Z, \Psi^{\ast}\left(\mathscr{O}_{X(\varphi)}(m)\right)\otimes_{\mathscr{O}_Z}N\right)=H^{0}(X(\varphi), \Psi_{\ast}(\triangle))
$$
where
$$
\triangle=\Psi^{\ast}\left(\mathscr{O}_{X(\varphi)}(m)\right)\otimes_{\mathscr{O}_Z}N. 
$$
We find that 
$$
\Psi_{\ast}(\triangle)=\Psi_{\ast}(N)\otimes \mathscr{O}_{X(\varphi)}\mathscr{O}_{X(\varphi)}(m). 
$$
Since $\Psi$ is birational and $N$ is a \textit{line bundle}, we see that $\Psi_{\ast}(N)\neq (0)$. Hence by Serre's theorems, we have 
$$
H^{0}(X(\varphi), \Psi_{\ast}(N)\otimes \mathscr{O}_{X(\varphi)}(m))\neq (0)\text{ for } m>> 0. 
$$
This proves $(\ast)$. Hence Lemma~\ref{lem3}. follows and therefore Theorem~\ref{thm1} as well. 
\end{Proof}

\section{The work of Demazure.}\label{s3}


\begin{Prop}\label{Prop1}
Let $\tau$, $w\in W$ be a such that $\tau=s_{\alpha} w$, $\alpha$ simple and $\tau> w$. Let $Z=P_{\alpha} \times ^{B} X(w)$ $(P_{\alpha}$ -minimal parabolic subgroup of $G$ associated to $\alpha$) and $\Psi$ the birational morphism $\Psi:Z\to X(\tau)$ (as in Theorem~\ref{thm1}). Then we have the following: 
\begin{enumerate}
\renewcommand{\theenumi}{\roman{enumi}}
\renewcommand{\labelenumi}{(\theenumi)}
\item $R^{0}\Psi_{\ast}\left(\mathscr{O}_Z\right)=\mathscr{O}_{X(\tau)}$
\item $R^{q}\Psi_{\ast}\left(\mathscr{O}_Z\right)=0, q>0$
\item For\pageoriginale any line bundle $M$ on $X(\tau)$, we have $H^{i}(X(\tau), M)\simeq H^{i}\\\left(Z,\Psi^{\ast}(M)\right)\forall i$. 
\end{enumerate}
\end{Prop}

\begin{Proof}
As is well-known (iii) is a consequence of (i) and (ii) and (i) follows from the normality of $X(\tau)$ (of Theorem~\ref{thm1}). Hence we have only to prove (ii). We fix an ample line bundle $L$ on $G/B$ and denote by $\mathscr{O}_{X(\tau)}(m)$ the restriction of $L^{m}$ to $X(\tau)$. Set $V_m=H^{0}(X(w), \mathscr{O}_{X(w)}(m))$ and $V_m$ the bundle on $\mathbb{P}^{1}=\dfrac{P_{\alpha}}{B}$, associated to the principal fibration $p:P_{\alpha}\to \dfrac{P_{\alpha}}{B}$. Let $p$ denote the canonical morphism $p:Z\to \mathbb{P}^{1}$. We claim that 
\begin{equation*}\label{eqn1}
R^{q}\left(p_{\ast}\Psi^{\ast}\left(\mathscr{O}_{X(\tau)}(m)\right)\right)=0, q>0 \text{ and } m> >0. \tag{1}
\end{equation*}
To see this, we first observe that by Lemma~\ref{lem1}
$$
\Psi^{\ast}\left(\mathscr{O}_{X(\tau)}(m)\right)\simeq \mathscr{O}_{X(w)}(m)^{\sharp}
$$
$\mathscr{O}_{X(w)}(m)^{\sharp}$ being the line bundle on $Z$, ``associated to'' the line bundle $\mathscr{O}_{X(w)}(m)$ on $X(w) \left(\text{ for the fibration } P_{\alpha}\to \dfrac{P_{\alpha}}{B}\right)$. 
Now
$$
H^{i}(X(w), \mathscr{O}_{x(w)}(m))=0, i>0, m>>0. 
$$
Now (i) is an immediate consequence of the fact that $p$ is a locally trivial fibre space of fibre type $X(w)$ (in fact, the fibre of $p$ over the point corresponding to the coset $B$ can be canonically identified with $X(w)$). By the usual Leray spectral sequence argument, $(\ast)$ implies that 
\begin{equation*}\label{eqn2}
\begin{cases}
H^{P}\left(\mathbb{P}^{1}, p_{\ast}\Psi^{\ast}\left(\mathscr{O}_{X(\tau)}(m)\right)\right)\simeq H^{P}\left(Z, \Psi^{\ast}\left(\mathscr{O}_{X(\tau)}(m)\right)\right)\\
\text{ for all } p.
\end{cases}\tag{2}
\end{equation*}
Now\pageoriginale by Lemma~\ref{lem1} and (\ref{eq8}) of Theorem~\ref{thm1}, we have 
\begin{equation*}
\begin{aligned}
{\rm(a)}~ &p_{\ast}\Psi^{\ast}\left(\mathscr{O}_{X(\tau)}(m)\right)\simeq V_m, \text{ and }\\
{\rm(b)}~ &{}H^{i}\left(\mathbb{P}^{1}, V_m\right)=0, i>0, m>>0.
\end{aligned}
\end{equation*}
Hence by (\ref{eqn2}), we conclude that 
\begin{equation*}\label{eqn3}
H^{p}\left(Z, \Psi^{\ast}\left(\mathscr{O}_{X(\tau)}(m)\right)\right)=0, p>0, m>>0.\tag{3}
\end{equation*}
We claim now that 
\begin{equation*}\label{eqn4}
\begin{cases}
H^{0}(X(\tau), \left(R^{9}\Psi_{\ast}(\mathscr{O}_{Z})\right)\otimes \mathscr{O}_{X(\tau)}(m))\simeq\\
H^{q}(Z, \Psi^{\ast}\left(\mathscr{O}_{X(\tau)}(m)\right), q>0, m>>0.
\end{cases}\tag{4}
\end{equation*}
To prove this consider the Leray spectral sequence 
\begin{equation*}\label{eqn5}
\begin{cases}
H^{p}\left(X(\tau), R^{q}\Psi_{\ast}\left(\Psi^{\ast}\left(\mathscr{O}_{X(\tau)}(m)\right)\right)\right)\\
\Rightarrow H^{p+q}\left(Z, \Psi^{\ast}\left(\mathscr{O}_{X(\tau)}(m)\right)\right)
\end{cases}\tag{5}
\end{equation*}
Now we have 
$$
R^{q}\Psi_{\ast}\left(\Psi^{\ast}\left(\mathscr{O}_{X(\tau)}(m)\right)\right)\simeq \left(R^{q}\Psi_{\ast}(\mathscr{O}_Z)\right)\otimes \mathscr{O}_{X(\tau)}(m).
$$
so that\pageoriginale 
\begin{equation*}\label{eqn6}
\begin{aligned}
&H^{p}(X(\tau), R^{q}\Psi_{\ast}\left(\Psi^{\ast}\left(\mathscr{O}_{X(\tau)}(m)\right)\right)\simeq\\
&{}H^{p}(X(\tau), \left(R^{q}\Psi_{\ast}(\mathscr{O}_Z)(m)\right)
\end{aligned}\tag{6}
\end{equation*}
One has 
$$
H^{p}(X(\tau), \left(R^{q}\Psi_{\ast}(\mathscr{O}_Z)\right)(m))=0, p>0, m>>0.
$$
Hence the spectral sequence (\ref{eqn5}) degenerates and we get 
\begin{equation*}\label{eqn7}
\begin{aligned}
H^{0}(X(\tau), \left(R^{q}\Psi_{\ast}(\mathscr{O}_Z)\right)(m))\simeq\\
H^{q}\left(Z, \Psi^{\ast}\left(\mathscr{O}_{\mathcal{X(\tau)}}(m)\right)\right), m>>0.
\end{aligned}\tag{7}
\end{equation*}
Using (\ref{eqn3}), we deduce that 
\begin{equation*}\label{eqn8}
H^{0}(X(\tau), \left(R^{q} \Psi_{\ast}(\mathscr{O}_Z)\right)(m))=0, q>0, m>>0.\tag{8}
\end{equation*}
Now (\ref{eqn8}) implies that 
$$
R^{q}\Psi_{\ast}(\mathscr{O}_Z)=0, q>0.
$$
This completes the proof of Proposition~\ref{Prop1}. 
\end{Proof}

\begin{thm}\label{thm2}
Let $X(\tau)$, $\tau \in W$, be a Schubert variety in $G/B$ and $L$ a line bundle on $G/B$ associated to a dominant weight. Then we have the following: 
\begin{enumerate}
\renewcommand{\theenumi}{\roman{enumi}}
\renewcommand{\labelenumi}{(\theenumi)}
\item The\pageoriginale canonical restriction map 
$$
H^{0}\left(\dfrac{G}{B}, L\right)\to H^{0}(X(\tau), L)
$$
is surjective, and 
\item $H^{i}(X(\tau), L)=0, i>0$. 
\end{enumerate}
\end{thm}

\begin{Proof}
These follow essentially from the results of V. B. Mehta and A. Ramanathan \cite{key4} where they prove the analogues of (i) and (ii) for an \textit{ample} line bundle $L$ on $G/P$, $P$ a parabolic subgroup of $G$. To apply their result, if $L$ is as in the above theorem, note that it comes from an ample line bundle $L'$ on $G/P$ for a well determined parabolic subgroup $P$ of $G$. Let $X'$ denote the image of $X$ in $G/P$. One knows that the cohomology groups of $L'$ are preserved by pull-back to $\dfrac{G}{B}$; in fact, by Theorem~\ref{thm1} and proofs similar to that of Prop.\ref{Prop1}, one can show easily that the cohomology groups of $L'/X'$ are also preserved by pull-back to $X$. This proves Theorem~\ref{thm2}. 

Let $\mathbb{Z}[N]$ denote the group ring of the multiplicative group exp $N$, where 
$$
\exp N=\{\exp \lambda\mid \lambda \in N\} \text{ and } N=\hom\ (T, \mathbb{G}_m).
$$
Let $X(w)$ and $X(\tau)$ be Schubert varieties (as in Prop.~\ref{Prop1}) such that $\tau=S_{\alpha}w, \tau > w$, $\alpha$ simple. Let $L_{\lambda}$ denote the line bundle on $G/B$, ``associated'' to a \textit{dominant weight} $\lambda$ (we adopt the convention that when the base field is of characteristic zero, $H^{0}(G/B, L_{\lambda})$\pageoriginale is the dual of the irreducible module with highest weight $\lambda$). Now the ``characters'' of the $T$ modules $H^{0}(X(w), L_{\lambda})$ and $H^{0}(X(\tau), L_{\lambda})$ are elements of $\mathbb{Z}[N]$ and are denoted respectively by $F(w)$ and $F(\tau)$. Let $L_{\alpha}$ be the linear operator $L_{\alpha}:\mathbb{Z}[N]\to \mathbb{Z}[N]$ defined by 
$$
L_{\alpha}(\exp \lambda)=\dfrac{\exp \lambda - \exp (S_{\alpha} \lambda)}{1-\exp \alpha}, \lambda \in N 
$$
Let $M_{\sim}$  be the operator $M_{\alpha}:\mathbb{Z}[N]\to \mathbb{Z}[N]$, defined by 
$$
\begin{cases}
M_{\alpha}(\exp \lambda)=(\exp p). L_{\alpha}(\exp (\lambda-p))\\
\rho=\dfrac{1}{2}- \text{ sum of positive roots }
\end{cases}
$$
\end{Proof}

\begin{thm}\label{thm3}
We have the following ``character formula''. 
$$
M_{\alpha}(F(w))=F(\tau) \text{ with } F~ (\text{identity}) = \exp(-\lambda).
$$
\end{thm}

\begin{Proof}
By Theorem~\ref{thm2}, the following sequence is exact: 
\begin{equation*}
H^{0}(X(\tau), L_{\lambda})\to H^{0}(X(w), L_{\lambda})\to 0\tag{$\ast$}
\end{equation*}
Set
\begin{equation*}\label{eqnt1}
E=H^{0}(X(\tau), L_{\lambda}), V=H^{0}(X(w), L_{\lambda}).\tag{1}
\end{equation*}
We denote by $\mathbb{E}$ and $\mathbb{V}$ the vector bundles on $\mathbb{P}^{1}=\dfrac{P_{\alpha}}{B}$, associated\pageoriginale to the principal $B$-fibration $P_{\alpha}\to \dfrac{P_{\alpha}}{B}$ We see that $\mathbb{E}$ is trivial and we have the following exact sequence of vector bundles on $\mathbb{P}^{1}$:
\begin{equation*}\label{eqnt2}
0\to K\to \mathbb{E}\to V\to 0\tag{2}
\end{equation*}
Writing the cohomology exact sequence for (\ref{eqnt2}), then we conclude (as for the proof of (\ref{eqn8}) of Theorem~\ref{thm1}) that 
\begin{equation*}\label{eqnt3}
H^{i}\left(\mathbb{P}^{1}, \mathbb{V}\right)=0, i\geq 1.\tag{3}
\end{equation*}
If $\Psi$ and $p$ denote the canonical maps 
$$
\Psi:Z\to X(\tau) \text{ and } p:Z\to \mathbb{P}^{1}\left(=\dfrac{P_{\alpha}}{B}\right)
$$
We conclude as in the proof of Lemma~\ref{lem2}, that 
\begin{equation*}\label{eqnt4}
p_{\ast}\left(\Psi^{\ast}(L_{\lambda})\right)\simeq \mathbb{V}\tag{4}
\end{equation*}
Now (\ref{eqnt4}) implies that 
\begin{equation*}\label{eqnt5}
H^{0}\left(Z, \Psi^{\ast}(L_{\lambda})\right)\simeq H^{0}\left(\mathbb{P}^{1}, \mathbb{V}\right)\tag{5}
\end{equation*}
and by Prop.~\ref{Prop1}, we have 
$$
H^{0}\left(Z, \Psi^{\ast}(L_{\lambda})\right)\simeq H^{0}(X(\tau), L_{\lambda})
$$
so that we find that 
\begin{equation*}\label{eqnt6}
H^{0}\left(\mathbb{P}^{1}, \mathbb{V}\right)\simeq H^{0}(X(\tau), L_{\lambda})\tag{6}
\end{equation*}
In\pageoriginale view of (\ref{eqnt3}) and (\ref{eqnt6}), we have:
\begin{equation*}\label{eqnt7}
\begin{cases}
\text{ Char } H^{0}(X(\tau), L_{\lambda})=\text{ Char } X\left(\mathbb{P}^{1}, \mathbb{V}\right)\\
V=H^{0}(X(w), L_{\lambda})
\end{cases}\tag{7}
\end{equation*}

Now it is easily seen that Char $X\left(\mathbb{P}^{1}, \mathbb{V}\right)$ is obtained by applying the operator $M_{\alpha}$ to $V$ (essentially the Weyl character formula for $SL(2)$, for more details, see \cite{key6}). This proves Theorem~\ref{thm3}. 
\end{Proof}

\begin{rem}\label{rem2}
The formula of Theorem~\ref{thm3} is essentially the Demazure character formula (of \cite{key1} and \cite{key5}). If the dominant weight $\lambda$ is ``sufficiently large''. ($\ast$) in the proof of Theorem~\ref{thm2} is an immediate consequences of the basic theorems of Serre on projective varieties (for this purpose, we may have to work on $\dfrac{G}{P}$, $P$ being a parabolic setgroup, such that $L_{\lambda}$ comes from an ample line bundle on $\dfrac{G}{P}$ etc.). Hence, Demazure's character formula for sufficiently large $\lambda$, is a consequence of the normality of Schubert varieties and one does not need Theorem~\ref{thm2}. On the other hand, as we mentioned in the introduction, the Demazure character formula for sufficiently large $\lambda$, implies the  normality of Schubert varieties. The proof is by increasing induction on the dimension of Schubert varieties. With the notations as in Theorem~\ref{thm1}, it suffices to show that $X(\tau)$ is normal supposing that $X(w)$ is normal. Then $Z$ is normal. We see easily (as in the above proofs) that for sufficiently large $\lambda$ (say for $L_{m \lambda}, m> >0 \ L_{\lambda}$ ample on $G/B$), $H^{0}\left(Z, \Psi^{\ast} (L_{m \lambda})\right)$ is given by the Demazure character formula. Hence, by our hypothesis, we deduce that
$$
H^{0}(X(\tau), L_{m \lambda})\xrightarrow{\sim} H^{0}(Z, \Psi^{\ast}(L_{m\lambda})), m>>0. 
$$
\pageoriginale
From this we deduce easily that $\Psi_{\ast}(\mathscr{O}_Z)\simeq \mathscr{O}_{X(\tau)}$, which implies that $X(\tau)$ is normal, since $Z$ is normal.
\end{rem}

\begin{rem}\label{rem3}
Let $G_{\mathbb{Z}}$ denote the semi-simple, simply connected Chevalley group scheme over $\mathbb{Z}$ such that $G=G_{\mathbb{Z}}\times _{\spec\ \mathbb{Z}}\spec\ k$. We have a Borel subgroup scheme $B_{\mathbb{Z}}$ of $G_{\mathbb{Z}}$, corresponding to $B$ and we have the ``flag scheme'' $\dfrac{G_{\mathbb{Z}}}{B_{\mathbb{Z}}}$ which behaves well under base change with respect to any field e.g. $\dfrac{G_{\mathbb{Z}}}{B_{\mathbb{Z}}}\times_{\spec\ \mathbb{Z}} \spec\ k\simeq G/B$. For any $\tau\in W$, we define the Schubert subscheme $X_{\mathbb{Z}}(\tau)$ of $\dfrac{G_{\mathbb{Z}}}{B_{\mathbb{Z}}}$ as the ``flat closure'' in $\dfrac{G_{\mathbb{Z}}}{B_{\mathbb{Z}}}$ of the Schubert scheme associated to $\tau$ in $\dfrac{G_{\mathbb{Z}}}{B_{\mathbb{Z}}}\times _{\spec\ \mathbb{Z}}\spec\ \mathbb{Q}$. We claim that 
\begin{equation*}\label{equation1}
X_{\mathbb{Z}}(\tau)\times_{\spec\ \mathbb{Z}}\spec\ k \simeq X(\tau)\tag{1}
\end{equation*}
We refer to (\ref{equation1}) as saying that the Schubert scheme $X_{\mathbb{Z}}(\tau)$ ``behaves well'' under base change with respect to any field. To prove (\ref{equation1}) let us denote the following base changes by $\spec\ k$ as 
\begin{equation*}\label{equation2}
\begin{cases}
\overline{X(\tau)}=X_{\mathbb{Z}}(\tau)\times_{\spec\ \mathbb{Z}}\spec\ k\\
\overline{L}_{\lambda} = L_{\lambda, \mathbb{Z}}\times_{\spec\ \mathbb{Z}} \spec\ k 
\end{cases}\tag{2}
\end{equation*}
Then we see easily that 
\begin{equation*}\label{equation3}
\overline{X(\tau)}_{\red}=X(\tau)\tag{3}
\end{equation*}
Let\pageoriginale now $\lambda$ be a dominant weight such that the
associated line bundle $L_{\lambda}$ on $G/B$ is very ample. One can
also associate to $\lambda$ a line bundle $L_{\lambda, \mathbb{Z}}$ on
$\dfrac{G_{\mathbb{Z}}}{B_{\mathbb{Z}}}$, which is relatively ample
with respect to $\mathbb{Z}$. Let us denote by $X_{\mathbb{Q}}(\tau)$
$\left(\resp L_{\lambda, \mathbb{Q}}\right)$ the base change of
$X_{\mathbb{Z}}(\tau)\left(\resp  L_{\lambda, \mathbb{Z}}\right)$ by $\spec\ \mathbb{Q}\to \spec\ \mathbb{Z}$. Then since $X_{\mathbb{Z}}(\tau)$ is $\mathbb{Z}$ -flat, we see that 
\begin{equation*}\label{equation4}
\dim H^{0}\left(\overline{X(\tau)}, \overline{L}^{m}_{\lambda}\right)=\dim H^{0}\left(X_{\mathbb{Q}}(\tau), L^{m}_{\lambda, \mathbb{Q}}\right), m>> 0. \tag{4}
\end{equation*}
On the other hand, by Theorem~\ref{thm2}, we see that the character formula is independent of the base field. 

This fact implies that 
\begin{equation*}\label{equation5}
\dim H^{0}\left(X_{\mathbb{Q}}(\tau), L^{m}_{\lambda, \mathbb{Q}}\right)=\dim H^{0}\left(X(\tau), L^{m}_{\lambda}\right), m>> 0. \tag{5}
\end{equation*}
Combining (\ref{equation4}); (\ref{equation5}) we get 
\begin{equation*}\label{equation6}
\dim H^{0}\left(\overline{X(\tau)}, \overline{L}^{m}_{\lambda}\right)=\dim H^{0}\left(X(\tau), L^{m}_{\lambda}\right), \text{ for } m> >0.\tag{6}
\end{equation*}
Since $X(\tau)_{\red}=X(\tau)$, we deduce easily from (\ref{equation6}) that $X(\tau)=\overline{X(\tau)}$. This proves (\ref{equation1}). 
\end{rem}

\begin{rem}\label{rem4}
Let $\lambda$ be a dominant weight and let us use the notations as in Remark~\ref{rem3}. Then as a consequence of Remark~\ref{rem3} and the vanishing theorem (see (ii) of Theorem~\ref{thm2}), we deduce that 
\begin{equation*}\label{equations1}
H^{0}\left(X_{\mathbb{Z}}(\tau), L_{\lambda, \mathbb{Z}}\right) \otimes_{\mathbb{Z}}k \simeq H^{0}(X(\tau), L_{\lambda})\tag{1}
\end{equation*}
Further,\pageoriginale as a consequence of the exactness of 
\begin{equation*}\label{equations2}
H^{0}(G/B,L_{\lambda})\to H^{0}(X(\tau), L_{\lambda})\to 0\tag{2}
\end{equation*}
(see (\ref{equations1}), Theorem~\ref{thm2}), we deduce easily that the following sequence is exact. 
\begin{equation*}\label{equations3}
H^{0}\left(\dfrac{G_{\mathbb{Z}}}{B_{\mathbb{Z}}}, L_{\lambda, \mathbb{Z}}\right)\to H^{0}\left(X_{\mathbb{Z}}(\tau), L_{\lambda, \mathbb{Z}}\right)\to 0\tag{3}
\end{equation*}
Let us now set 
\begin{enumerate}
\renewcommand{\theenumi}{\alph{enumi}}
\item $V_{\lambda, \mathbb{Z}}(\tau)=\left(H^{0}\left(X_{\mathbb{Z}}(\tau), L_{\lambda, \mathbb{Z}}\right)\right)^{\ast}$ (dual)
\item $V_{\lambda, \mathbb{Z}}(w_0)=V_{\lambda, \mathbb{Z}}=\left(H^{0}\left(\dfrac{G_{\mathbb{Z}}}{B_{\mathbb{Z}}}, L_{\lambda, \mathbb{Z}}\right)\right)^{\ast}$
\item $V_{\lambda, k}(\tau)=V_{\lambda, \mathbb{Z}}\otimes_{\mathbb{Z}}k = \left(H^{0}\left(X(\tau), L_{\lambda}\right)\right)^{\ast}$
\end{enumerate}
Then because of (\ref{equations1}), (\ref{equations2}) and (\ref{equations3}), we deduce that 
\begin{equation*}\label{equations4}
V_{\lambda, \mathbb{Z}}(\tau) \text{ is a direct summand in } V_{\lambda, \mathbb{Z}}\tag{4}
\end{equation*}
Now $V_{\lambda, \mathbb{Z}}(\tau)$ has a more concrete description as follows. 

Let $V_{\lambda, \mathbb{Q}}$ denote the irreducible $G_{\mathbb{Q}}=G_{\mathbb{Z}}\times_{\spec\ \mathbb{Z}} \spec\ \mathbb{Q}$ module with highest weight $w_{0}(\lambda)$ ($w_{0}$= weyl involution). Fix a highest weight vector, say $e\in V_{\lambda, \mathbb{Q}}$. Let $U$ denote the enveloping algebra of the Lie algebra of $G_{\mathbb{Q}}$ and $U_{\mathbb{Z}} (\resp U^{+}_{\mathbb{Z}}$, $U^{-}_{\mathbb{Z}})$ the $\mathbb{Z}$ a subalgebras generated by $\dfrac{X_{\alpha}}{n!}$, $\alpha$ any root (resp. positive root, negative root). We now set 
\begin{equation*}\label{equations5}
V'_{\lambda, \mathbb{Z}}=U_{\mathbb{Z}}e(= U^{-}_{\mathbb{Z}} e)\tag{5}
\end{equation*}\pageoriginale
Now any $\tau\in W$ can be represented by a $\mathbb{Z}$ -valued point of $G_{\mathbb{Z}}$ and hence the element $\tau$. $e$ is determined upto the factor $\pm 1$. Let us write $e(\tau)$ for $\tau.e.$ Set 
\begin{equation*}\label{equations6}
V'_{\lambda, \mathbb{Z}}(\tau)=U^{+}_{\mathbb{Z}}e(\tau)\tag{6}
\end{equation*}
Now as a consequence of (\ref{equations3}), and Theorem~\ref{thm2}, it can be shown easily that (for details, see \cite{key6})
\begin{equation*}\label{equations7}
\begin{aligned}
&V_{\lambda, \mathbb{Z}}=V'_{\lambda, \mathbb{Z}}\\
& V_{\lambda, \mathbb{Z}}(\tau) = V'_{\lambda, \mathbb{Z}}(\tau)
\end{aligned}\tag{7}
\end{equation*}
Hence $V'_{\lambda, \mathbb{Z}}(\tau)$ is a direct summand in $V'_{\lambda, \mathbb{Z}}$. This was conjectured by Demazure in \cite{key1}. 
\end{rem}

\begin{rem}\label{rem5}
One can construct canonical desingularigations of $X(\tau)$. This can be done by either following Demazure \cite{key1} or refining the construction of $Z$ inductively as follows. Let $\tau=s_{\alpha}w$ with $\alpha$ simple and $w<\tau$. Suppose we have constructed a desingularigation 
$$
\Psi_w:Z(w)\to X(w)
$$
such that $Z_{id}=$ point. Then we define 
$$
Z(\tau)=P_{\alpha}\times^{B} Z(w)
$$
and\pageoriginale the morphism $\Psi_{\tau}:Z(\tau)\to X(\tau)$ in the obvious manner. We note that the morphism $\Psi_{\tau}$ depends upon the choice of a reduced decomposition of $\tau$. By repeating the proof of Prop.\ref{Prop1}, one deduces easily that 
\begin{enumerate}
\renewcommand{\theenumi}{\roman{enumi}}
\renewcommand{\labelenumi}{(\theenumi)}
\item $\left(\Psi_{\tau}\right)_{\ast}(\mathscr{O}_{z(\tau)})=\mathscr{O}_{X(\tau)}$, 
\item $\left(R^{q} \Psi_{\ast}\right)(\mathscr{O}_{Z(\tau)}=0, q>0$. 
\end{enumerate}

When the base field is of \textit{characteristic zero}, we see that (as in Demazure\cite{key1}) (i) and (ii) imply that $Z(\tau)$ has only \textit{rational singularities}, in particular that $Z(\tau)$ in Cohen-Macaulay. 
\end{rem}

\begin{thebibliography}{99}
\bibitem{key1}
{M. Demazure.} Desingularisation des varieties de Schubert generalisees, \textit{Ann. Sc. Ec. Norm. Sup. t.} 7, 1974, p. 53--58. 

\bibitem{key2}
{A. Joseph.} On the Demazure character formula (preprint)

\bibitem{key3}
{V. Lakshmibai and C. S. Seshadri.} Geometry of $G/P-V$ Journal of Algebra, 100(1986), p. 362--457. 

\bibitem{key4}
{V. B. Mehta and A. Ramanathan.} Frobenius splitting and cohomology vanishing for Schubert varieties, \textit{Ann. Math,} 122(1985), p. 27--40.

\bibitem{key5}
{V. Lakshmibai, C. Musili and C. S. Seshadri}.\pageoriginale Geometry of $G/P-IV$ (Standard monomial theory for classical types), \textit{Proc. Indian Acad. Sci.} 88A(1979), P. 280--362.

\bibitem{key6}
{C. S. Seshadri.} Standard monomial theory and the work of Demazure - Advanced Studies in Pure Mathematics I, 1983, \textit{Algebraic Varieties and Analytic Varieties}, p. 355--384. 
\end{thebibliography}

\vskip 1cm

\noindent
School of Mathematics,\\
Tata Institute of Fundamental Research,\\
Bombay - 400 005.

\vskip .5cm

\noindent
Present Address:\\
The Institute of Mathematical Sciences,\\
Madras - 600 113.

\newpage

~\phantom{a}
\thispagestyle{empty}
