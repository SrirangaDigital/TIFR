\title{Compactification of $M(0,2)$}
\markright{Compactification of $M(0,2)$}

\author{By M. S. Narasimhan and G. Trautmann}
\markboth{M. S. Narasimhan and G. Trautmann}{Compactification of $M(0,2)$}

\date{}
\maketitle

\setcounter{page}{325}
\setcounter{pageoriginal}{428}
\section{Introduction}\label{s1}\pageoriginale

Let $M(0,2)$ denote the quasi-projective variety of isomorphism
classes of stable rank $2$ vector bundles on $\mathbb{P}_3(\mathbb{C})$
with $C_1=0$, $C_2=2$. This variety was investigated in detail by
Hartshorne \cite{key1}. See also \cite{key2}. He proved that $M(0,2)$
has the structure of a fibre space over the $9$-dimensional variety
$R$ of reguli (i.e., the space of smooth quadrics in $\mathbb{P}_3$
with a distinguished system of generating lines), the fibre being an
open subset of a smooth quadric in $\mathbb{P}_5$ If $\sigma$ is the
smooth conic in the grassmannian $\mathbb{G}$ of lines in
$\mathbb{P}_3$ given by the generators of a regulus $\rho$, then the
fibre over $\rho$ consists of the space of smooth conics $\gamma$ such
that $\sigma$ and $\gamma$ are Poncelet related, with $\sigma$ as the
inner conic: a triangle can be inscribed in $\gamma$ which
circumscribes $\sigma$. 

In the present paper, we study the natural compactification of
$M(0,2)$ and the degeneration of bundles to sheaves with
singularities. Geometrically, one has to first compactify the fibres
over $R$ which is easy and is done by taking all conics $\gamma$,
smooth or not, which are Poncelet related to $\sigma$; the fibre over
$\sigma$ is then a smooth quadric in $\mathbb{P}_5$, which may be
called the Poncelet\pageoriginale
 quadric associated with $\sigma$. Next, one has to
take a good compactification of the space of reguli $R$. There is a
`naive' compactification of $R$, namely the ramified $2$-sheeted
covering of the space of all quadrics in $\mathbb{P}_3$ defined by the
space of singular quadrics. This is not the right one to take and has
to be `modified' and we take as the compactification of $R$ the
Hilbert scheme $\mathscr{C}(\mathbb{G})$ of all conics contained in
the grassmannian $\mathbb{G}=G(2,4)$. The variety $\mathscr{C}(\mathbb{G})$ is in fact smooth and there
is a tautological conic bundle over $\mathscr{C}(\mathbb{G})$. There
is then a Poncelet quadric bundle over $\mathscr{C}(\mathbb{G})$
associated with this conic bundle; it is constructed by considering
also the space of conics which are Poncelet related to a singular
conic, which turns out to be a pair of hyperplanes in $\mathbb{P}^{5}$
in the case of a pair of lines and a `doubled' hyperplance in the case
of a double line. This Poncelet quadric bundle over
$\mathscr{C}(\mathbb{G})$ is the sought for compactification of
$M(0,2)$. We prove that this space is the coarse moduli scheme for
semi-stable rank $2$ sheaves on $\mathbb{P}_3$ with $C_1=0$, $C_2=2$,
$C_3=0$ which are limits of stable vector bundles (see the statement
of Theorem in \S\ 3). The non-singular points of this compactification
correspond exactly to stable sheaves. It should be remarked that the
moduli space of semi-stable rank 2 sheaves with $C_1=0$, $C_2=2$,
$C_3=0$ is not irreducible. 

We proceed to describe the results and methods of proof in more
detail. 

\section{Poncelet pairs and the Poncelet quadric.}\label{s2}

 Two smooth conics $(\sigma, \gamma)$ in $\mathbb{P}_2$ are said to be
 Poncelet  related\pageoriginale
 if we can inscribe a triangle in $\gamma$ which
 circumscribes $\sigma$ We have also to deal with the case when
 $\sigma$ is degenerate. For this purpose it is convenient to consider
 the Poncelet relation as one between conics in $\mathbb{P}_2$ and
 those in the dual projective space $\mathbb{P}^{\ast}_{2}$. For
 instance, in the case above, if we consider the polar dual
 $\displaystyle\mathop{\gamma}^{\vee}$  of $\gamma$ with respect to
 the smooth conic $\sigma$, the Poncelet relation between $(\sigma,
 \gamma)$ says that there are three points in $\sigma$ such that the
 dual triangle has vertices on
 $\displaystyle\mathop{\gamma}^{\vee}$. This relation is symmetric. 

If $W$ is a $3$-dimensional vector space (over $\mathbb{C}$), the
Poncelet relation defines a correspondence between
$\mathbb{P}\left(S^{2}(W)\right)$ and
$\mathbb{P}\left(S^{2}\left(W^{\ast}\right)\right)$ i.e., a
correspondence between the space of conics in $\mathbb{P}(W)$ and in
$\mathbb{P}\left(W^{\ast}\right) $. In particular if $\sigma$ is a
conic in $\mathbb{P}(W)$ we get a quadric in
$\mathbb{P}\left(S^{2}(W)\right)$, the Poncelet quadric corresponding
to $\sigma$. If we denote also by $\sigma$ a quadratic form on $W$
defining $\sigma$, a quadratic form $Q$ on $S^{2}(W)$ defining the
Poncelet quadric associated $\sigma$ is given by: 
$$
\begin{aligned}
Q(X,Y,X'.Y')&=\dfrac{1}{}[\sigma(X,X')\sigma(Y,Y')+\sigma(X,Y')\sigma(Y,X')\\
&{}-\sigma (X,Y)\sigma (X',Y')]
\end{aligned}
$$
This construction can be relativised: given a conic bundle $C\to S$,
we can construct the Poncelet quadric bundle over $S$ associated to
$C$. 

\section{Statement of the main theorem.}\label{s3}\pageoriginale


\begin{THM}
\textit{Let $\mathbb{G}$ be the grassmannian of lines in
  $\mathbb{P}_3(\mathbb{C})$, $\mathscr{C}(\mathbb{C})$ the Hilbert
  scheme of conics contained in $\mathbb{G}$ and $Q\to
  \mathscr{C}(\mathbb{G})$ associated to the tautological conic bundle
over $\mathscr{C}(\mathbb{G})$. We then have:}
\begin{enumerate}
\renewcommand{\labelenumi}{(\theenumi)}
\item \textit{The variety $Q$ is the space of $S$-equivalence classes of
  semi-stable sheaves on $\mathbb{P}_3(\mathbb{C})$ which are limits,
  under flat deformations, of bundles in $M(0,2)$ }

\item \textit{$Q$ has the `coarse moduli property' for flat families
  $\{F_s\}_{s\in S}$ of \\torsion-free semi-stable rank $2$ sheaves
  with $C_1=0$, $C_2=2$, $C_3=0$ for which the subset of $S$,
  consisting of those $s\in S$ with $F_s$ a stable bundle
  $\epsilon M(0,2)$ is dense in $S$ where $S$ is normal. }
\item \textit{$Q$ is the normalisation of a component of the Maruyama scheme
  of all semi-stable torsion free rank $2$ sheaves on $\mathbb{P}_3$
  with $C_1=0$, $C_2=2$ $C_3=0$. }
\item \textit{The non-singular points of $Q$ correspond precisely to stable
 \\sheaves}. 
\end{enumerate}
\end{THM}

In the rest of the paper, we give a brief account of some of the ideas
involved in the proof of the theorem. 


\section[Geometric invariant theoretic...]{Geometric invariant theoretic (G.I.T)\\ description of the
  Poncelet bundle over the\\ space of conics in \texorpdfstring{$\mathbb{P}_2$}{eq}}\label{s4}\pageoriginale


In order to relate the Hilbert scheme $\mathscr{C}(\mathbb{G})$ and
the associated Poncelet bundle to sheaves on $\mathbb{P}_3$ we will
need a geometric invariant theoretic description of these spaces. As a
motivation for this description, which will be given in the next
section, we give in this section a G.I.T. parametrisation of the
Poncelet bundle over the space of conics in $\mathbb{P}_2$. 

We first give a G.I.T. parametrisation of conics in
$\mathbb{P}_2$. Let $W$ be a $3$-dimensional vector space and let
$\widetilde{W}=W\oplus W=W\otimes \mathbb{C}^{2}$ The action of
$SL(2,\mathbb{C})$ on $\mathbb{C}^{2}$ induces an action on
$\widetilde{W}$. This in turn gives an action of $SL(2,\mathbb{C})$ on
the grassmannian $G_4\left(\widetilde{W}\right)$ of $4$ dimensional
subspaces of $\widetilde{W}$, which is linearised with respect to the
natural polarisation on $G_4\left(\widetilde{W}\right)$. The
G.I.T. quotient $\dfrac{G_4\left(W\oplus W\right)^{ss}}{SL(2)}$ is
then identified with the space of conics in $\mathbb{P}_2$ (As usual,
the superscript ``ss'' denotes the set of semi-stable points). In
fact, if for a $4$-dimensional subspace $\Gamma \subset W\oplus W$, we
consider 
$$
\sigma(\Gamma)=\left\{L\in \mathbb{P}(W)\mid \dim
\left[\left(L\otimes \mathbb{C}^{2}\right)\cap \Gamma\right]\geq 1\right\}
$$
one sees that $\sigma(\Gamma)\neq \mathbb{P}(W)$ if and only if
$\Gamma$ is semi-stable and that $\sigma(\Gamma)$ is a conic if
$\Gamma$ is semi-stable. Moreover $\Gamma$ is stable if and only if
$\sigma(\Gamma)$ is a smooth conic. 

We next consider the Poncelet correspondence. (See \S 2). For this, we
consider the grassmannian $G_2\left(\widetilde{W}\right)$ of
$2$-dimensional subspaces\pageoriginale
 of $W\oplus W$ and observe, as above, that
$\dfrac{G_2\left(\widetilde{W}\right)^{ss}}{SL(2)}$ can be identified
with the space of conics in $\mathbb{P}\left(W^{\ast}\right)$. To
obtain the Poncelet correspondence between
$\mathbb{P}\left(S^{2}(W)\right)$ and
$\mathbb{P}\left(S^{2}\left(W^{\ast}\right)\right)$, we consider the
flag manifold $F\subset G_2\left(\widetilde{W}\right)\times
G_4\left(\widetilde{W}\right)$ consisting of $(M,\Gamma)$ with
$M\subset \Gamma$ and show that, for the action of $SL(2)$ on
$F,\dfrac{F^{ss}}{SL(2)}$ is isomorphic to the Poncelet subvariety
contained in $\mathbb{P}\left(S^{2}\left(W\right)\right)\times
\mathbb{P}\left(S^{2}\left(W^{\ast}\right)\right)$. Moreover the
natural map $F^{ss}\to G_4(W)^{ss}$ induces the Poncelet quadric
bundle over the space of conics in $\mathbb{P}(W)$. 

\section[G.I.T. description of the Hilbert scheme...]{G.I.T. description of the Hilbert scheme of conics in
  \texorpdfstring{$\mathbb{G}$}{eq} and the associated Poncelet bundle.}

Let $V$ be a $4$-dimensional vector space and
$\mathbb{G}=G_2(V)\subset
\mathbb{P}\left(\displaystyle\mathop{\wedge}^{2} V\right)$. We first
give a G.I.T. description of the Hilbert scheme of conics in
$\mathbb{P}\left(\displaystyle\mathop{\wedge}^{2}V\right)$. Note that
this scheme is a $\mathbb{P}^{5}$ bundle over the grassmanian $Z=G_3
\left(\displaystyle\mathop{\wedge}^{2}V\right)$ of planes in
$\mathbb{P}\left(\displaystyle\mathop{\wedge}^{2}V\right)$, the fibre
over a plane consisting of the space of conics in that plane. As such,
in analogy with \S\ 4, this scheme has the following
G.I.T. description. Let $U$ be the universal rank $3$ bundle over the
grassmannian $Z$; $U$ is a subbundle of the trivial bundle
$\displaystyle\mathop{\wedge}^{2}V\otimes \mathscr{O}_Z$ over
$Z$. Then $SL(2,\mathbb{C})$ operates on the relative grassmannian
bundle $G_4(U\oplus U)_2$ and the G.I.T quotient is the Hilbert scheme
of conics in
$\mathbb{P}\left(\displaystyle\mathop{\wedge}^{2}V\right)$. We also
have an obvious G.I.T. description of the associated Poncelet bundle
as in \S\ 4. 

To obtain the Hilbert scheme $\mathscr{C}(\mathbb{G})$ of conics in
$G$ as a G.I.T. quotient we define a subvariety $Y$ of $G_4(U\oplus
U)$ as follows. If $(z,\Gamma)\in G_4(U\oplus U)$, with $z\in Z$,
then $(z,\Gamma)$ defines\pageoriginale
 a quadratic form $\sigma(\Gamma)$ on the
fibre $U$ of $U$ at $z$, and the Pl\"{u}cker quadric $G$ defines a
quadratic form $\rho(z)$ on $U_z$ (Here
$\rho(z)$ could be zero which means that $\mathbb{P}(U_z)\subset G$
and $\sigma (\Gamma)$ could be zero which means that $\Gamma$ is not
semi-stable). We define $Y$ by the condition that $(z,\Gamma)\in Y$ if
and only if $\sigma(\Gamma)$ and $\rho(z)$ are linearly dependent. Let 
$$
Y^{ss}=Y\cap G_4 (U\oplus U)^{ss}\text{ and } Y^{s}=Y\cap G_4(U\oplus U)^{s}.
$$
We thee see that the natural map 

$\dfrac{Y^{ss}}{SL(2)}\to \mathscr{C}(\mathbb{G})$ induces an
isomorphism. 

To get the Poncelet bundle over $\mathscr{C}(\mathbb{G})$, we look at
the (relative) flag variety
$$
X\subset G_4(U\oplus U)\times _Z G_2(U\oplus U)
$$
consisting of $(z,\Gamma, M)$ with $(z,\Gamma)\in Y$ and $M\subset
\Gamma$. The morphism $X\to Y$ is a bundle with $G(2,4)$ as
fibres. The G.I.T. quotient $\dfrac{X^{ss}}{SL(2)}$ is the Poncelet
bundle $Q$ associated to the `universal' conic bundle on
$\mathscr{C}(\mathbb{G})$; moreover, $Q\to \mathscr{C}(\mathbb{G})$ is
induced by $X^{ss}\to Y^{ss}$. It turns out that the singular points
of $Q$ correspond exactly the non-stable points of $X^{ss}$. 

\section[\texorpdfstring{$\mathscr{C}(\mathbb{G})$}{eq} as the moduli...]{\texorpdfstring{$\mathscr{C}(\mathbb{G})$}{eq} as the moduli space of rank \texorpdfstring{$4$}{eq}
  `kernel'\\ sheaves on \texorpdfstring{$\mathbb{P}_3$}{eq}.}\label{s6}

The connection between the foregoing considerations and sheaves\pageoriginale
 on
$\mathbb{P}_3$ will be made by constructing, in the next section, a family of monads paramterised by $X^{ss}$ whose `cohomology' will  give the required rank $2$ semi-stable sheaves on $\mathbb{P}_3$. 

In this section we outline the construction of (the family of) the two modules (and the `arrow') on the right side of the monad. These are parameterised actually by $Y^{ss}$ (The space $\mathscr{C}(\mathbb{G})$ is, in some sense, the moduli space of the kernel sheaves of the monads). 

For doing this, it is convenient to study a G.I.T. parametrisation of the `naive' compactification of the space reguli and relate it to $\mathscr{C}(\mathbb{G})$ which is a modification of it. Consider the action of $SL(2)$ on $G=G_2(V\oplus V)$. Then $\dfrac{G_2(V\oplus V)^{ss}}{SL(2)}$ gives the naive compactification. The canonical map $\left(V^{\ast}\oplus V^{\ast}\right)\otimes \mathscr{O}_g\to A^{\ast}$, where $A$ is the tautological rank $2$ bundle on $G$ may be interpreted as a map 
$$
\alpha:\left(\displaystyle\mathop{\wedge}^{3}V\oplus \displaystyle\mathop{\wedge}^{3}V\right)\otimes \mathscr{O}_G\to \displaystyle\mathop{\wedge}^{4} V\otimes A^{\ast}
$$
on observing that $\displaystyle\mathop{\wedge}^{3} V\cong V^{\ast}\otimes \displaystyle\mathop{\wedge}^{4} V$. Note also that we have a morphism of bundles 
$$
\left(\displaystyle\mathop{\wedge}^{2}V\oplus \displaystyle\mathop{\wedge}^{2}V\right)\otimes O_G\to \displaystyle\mathop{\wedge}^{3} V\otimes A^{\ast}
$$
The relation between $\mathscr{C}(\mathbb{G})$ and $\dfrac{G_2(V\oplus V)^{ss}}{SL(2)}$ comes from 
\begin{lem}
\textit{In $\Gamma \in G_4 (U\oplus U) \subset G_4\left(\displaystyle\mathop{\wedge}^{2} V\oplus \wedge V\right)$. Define $N_{\Gamma} \subset V\oplus V$\pageoriginale
 to be the subspace of all pairs $(x,y),x,y \in V$, satisfying }
$$
\xi \wedge x+\eta \wedge y=0 \text{ for any }  (\xi, \eta)\in \Gamma.
$$
\textit{If $\Gamma \in Y^{ss}$, then $N_{\Gamma}$ is $2$-dimensional.}
\end{lem}

Using the lemma we define a morphism $Y^{ss}\to G_2(V\oplus V)^{ss}$ (which in turn defines the modification map $\mathscr{C}(\mathbb{G})\to \dfrac{G_2(V\oplus V)^{ss}}{SL(2)}$. We consider the diagram  
{\fontsize{8}{10}\selectfont
$$
\xymatrix@C=.5cm{Y^{ss}\ar@{-}@/^3pc/[rrr]_{\gamma} 
  \ar[r]\ar[d]^{q}&G_4(U\oplus U)\ar[r]&G_3\left(\displaystyle\mathop{\wedge}^{2} V\right)\times G_4\left(\displaystyle\mathop{\wedge}^{2}V\oplus \displaystyle\mathop{\wedge}^{2} V\right)\ar[r]&G_4\left(\displaystyle\mathop{\wedge}^{2} V\oplus \displaystyle\mathop{\wedge}^{2}V\right)\\
G=G_2(V\oplus V)\leftarrow A && & C\ar[u]}
$$}\relax
where $A$ and $C$ are the tautological subbundles on the grass mannians and 
$$
\gamma:Y^{ss}\to G_4\left(\displaystyle\mathop{\wedge}^{2}V\oplus \displaystyle\mathop{\wedge}^{2} V\right) 
$$
the composite map. 

We are now in a position to define the right hand side of the family of monads. Lift the map $\alpha\left(\displaystyle\mathop{\wedge}^{3}V\oplus \displaystyle\mathop{\wedge} ^{3}V\right) \otimes \mathscr{O}_G\to \displaystyle\mathop{\wedge}^{4} V\otimes A^{\ast}$ (defined above) first to $Y^{ss}$ by the morphism $Y^{ss}\to G$ and then to\pageoriginale $\mathbb{P}_3\times Y^{ss}$ by the projection $pr$ onto $Y$. We get an epimorphism 
$$
\left(\displaystyle\mathop{\wedge}^{3}V\oplus \displaystyle\mathop{\wedge}^{3} V\right)\otimes \mathscr{O}_{\mathbb{P}^{3}\times Y^{ss}}\to \displaystyle\mathop{\wedge}^{4}V\otimes pr^{\ast}q^{\ast}A^{\ast}
$$
Similarly, starting from the Koszul homomorphism 
$$
\displaystyle\mathop{\wedge}^{3} V \otimes \mathscr{O}_{\mathbb{P}_3}\to \displaystyle\mathop{\wedge}^{4} V\otimes \mathscr{O}(1) \text{ on } \mathbb{P}_3,
$$
we get an epimorphism 
\begin{equation}
\left(\displaystyle\mathop{\wedge}^{3}V\oplus \displaystyle\mathop{\wedge}^{3}V\right)\otimes \mathscr{O}_{\mathbb{P}_3\times Y}\to \left(\displaystyle\mathop{\wedge}^{4}V\otimes \displaystyle\mathop{\wedge}^{4}V\right) \otimes \mathscr{O}_{\mathbb{P}_3\times Y}
\end{equation}
Taking the sum of these homomorphisms, we obtain a homomorphism over $\mathbb{P}_{3}\times Y^{ss}$: 
$$
\begin{aligned}
\widetilde{\alpha}&:\left(\displaystyle\mathop{\wedge}^{3} V\oplus \displaystyle\mathop{\wedge}^{3}V\right)\otimes \mathscr{O}_{\mathbb{P}_3\times Y} \to \left(\displaystyle\mathop{\wedge}^{4}V\oplus \displaystyle\mathop{\wedge}^{4} V\right)\otimes\\ &{}\mathscr{O}_{\mathbb{P}_3\times Y}(1) \oplus \left(\displaystyle\mathop{\wedge}^{4} V\otimes pr^{\ast} q^{\ast}A^{\ast}\right).
\end{aligned}
$$
This gives the required family of `partial' monads. 

Let $\mathscr{N}=\ker \alpha$ and $\mathscr{A}=Im\widetilde{\alpha}$. We also observe that under the natural map 
$$
pr^{\ast}\gamma^{\ast}C(-1)\to \left(\displaystyle\mathop{\wedge}^{2}V\oplus \displaystyle\mathop{\wedge}^{2}V\right) \otimes \mathscr{O}_{\mathbb{P}_3}(-1)\to \left(\displaystyle\mathop{\wedge}^{3}V\oplus \displaystyle\mathop{\wedge}^{3}V\right)\otimes \mathscr{O}_{\mathbb{P}_3\times Y}, 
$$
$pr^{\ast}\gamma^{\ast}C(-1)$\pageoriginale injects into $\mathscr{N}$ and we define $\mathscr{G}$ by the exact sequence 
$$
(\ast)\quad 0\to pr^{\ast}\gamma^{\ast}C(-1)\to \mathscr{N}\to \mathscr{G}\to 0. 
$$
We have 

\begin{Prop}
\begin{enumerate}
\renewcommand{\labelenumi}{(\theenumi)}
\item \textit{The sheaves $\mathscr{N}$, $\mathscr{A}$ and $\mathscr{C}$ are flat over $Y^{ss}$}
\item \textit{For any $y=(z,\Gamma)\in Y^{ss}$, if $\mathscr{N}_y$ denotes the restriction of $\mathscr{N}$ to $\mathbb{P}_3\times y$, then the natural homomorphism}
$$
\Gamma \to H^{0}(\mathbb{P}_3,\mathscr{N}_y(1))(\text{ given by } (\ast))
$$
\textit{is an isomorphism. Moreover}
$$
C_1(\mathscr{N}_y)=-2,C_2(\mathscr{N}_y)=3,C_3(\mathscr{N}_y)=-4.
$$
\item \textit{The sheavs $\mathscr{N}_y$ are semi-stable in the sense of Gieseker.}
\end{enumerate}
\end{Prop}

\section[Construction of a family of monads...]{Construction of a family of monads and the universal rank \texorpdfstring{$2$}{eq} sheaf on \texorpdfstring{$\mathbb{P}_3$}{eq}}\label{s7}

We consider the diagram
$$
\xymatrix{
&& &\ar[d]B\\
X^{ss}\ar@/^2pc/[urrd]^-{\beta_0}\ar@/^4pc/[urrrd]^-{\beta} \ar[r]\ar[d]^{p}&Z\times F\ar[r]\ar[dr]_{id\times p_4}& F\ar[r]\ar[dr]^{p_{4}}& G_{2}\left(\displaystyle\mathop{\wedge}^{2} V\oplus \displaystyle\mathop{\wedge}^{2} V\right)\\ 
Y^{ss}\ar[r]& G_4(U\oplus U)\ar[r]& Z\times G_4\left(\displaystyle\mathop{\wedge}^{2} V\oplus \displaystyle\mathop{\wedge}^{2}V\right)\ar[r]& G_4\left(\displaystyle\mathop{\wedge}^{2}V\oplus \displaystyle\mathop{\wedge}^{2}V\right)\\
& & & \ar[u]C}
$$
Here\pageoriginale $F$ denotes the flag manifold of pairs $(\Gamma,
M)$ with $M\subset \Gamma$ and $p_2$, $p_4$ are the natural
projections. The map $X\to Z\times F$ is induced by 
$$
\xymatrix{G_4(U\oplus U) X_ZG_2(U\oplus U)\ar[r]& Z\times G_4\left(\displaystyle\mathop{\wedge}^{2} V\oplus \displaystyle\mathop{\wedge}^{2} V\right)\times G_2\left(\displaystyle\mathop{\wedge}^{2}V\oplus \displaystyle\mathop{\wedge}^{2}V\right)\\
x\ar[r]\ar[u]^{\leftarrow}& Z\times F\ar[u]}
$$
As before, $B$ and $C$ denote the tautological bundles. From the
diagram, we get an exact sequence of vector bundles on $F$: 
$$
0\to P^{\ast}_2 B\to P^{\ast}_4 C\to \displaystyle\mathop{B}^{\vee}\to 0.
$$
Lifting via $\beta_0$ and $pr:\mathbb{P}_3\times X^{ss}\to X^{ss}$, 

We obtain a diagram 
$$
\xymatrix{& 0\ar[d] & 0\ar[d]\\
&pr^{\ast}\beta^{\ast}B(-1)\ar@{=}[r]\ar[d]& pr^{\ast}\beta^{\ast}B(-1)\ar[d]\\
0\ar[r]& pr^{\ast}p^{\ast}\gamma^{\ast}C(-1)\ar[r]\ar[d]& P^{\ast}\mathscr{N}\ar[r]\ar[d]& p^{\ast}\mathscr{G}\ar[r]\ar@{=}[d]&0\\
0\ar[r]&pr^{\ast}\beta^{\ast}B^{\vee}\ar[r]\ar[d]& \mathscr{F}\ar[r]& p^{\ast}\mathscr{G}\ar[r]\ar[d]&0\\
& 0 & & 0\\
}
$$
in\pageoriginale which $\mathscr{F}$ is defined at the same time as a quotient of $p^{\ast}\mathscr{N}$ and as an extension of $p^{\ast}\mathscr{G}$ for each point $x=(z,\Gamma, M)\in X$ with $y=p(x)=(z,\Gamma)$, we obtain the induced diagram on the fibre $\mathbb{P}_3=\mathbb{P}_3\times x$
$$
\xymatrix{& 0\ar[d] &0\ar[d]\\
& M\otimes \mathscr{O}(-1)\ar@{=}[r]\ar[d]&M\otimes \mathscr{O}(-1)\ar[d]\\
0\ar[r]& \Gamma \otimes \mathscr{O}(-1)\ar[r]\ar[d]& \mathscr{N}_y\ar[r]\ar[d]& \mathscr{G}_y\ar[r] &0\\
0\ar[r]& \displaystyle\mathop{M}^{\vee}\otimes \mathscr{O}(-1)\ar[r]\ar[d]& \mathscr{F}_x\ar[r]\ar[d]& \mathscr{G}_y\ar[r]&0\\
&0&0}
$$

\begin{Prop}
\begin{enumerate}
\renewcommand{\labelenumi}{(\theenumi)}
\item \textit{For $x\in X^{ss}$, the sheaf $\mathscr{F}_x$ is a torsion free semi-stable sheaf with $C_1=0$, $C_2=2$ and $C_3=0$.}
\item \textit{For $x,x'\in X^{ss}$, $\mathscr{F}_x$ and $\mathscr{F}_x$ are $S$-equivalent if and only if $x$ and $x'$ are equivalent in the sense of G.I.T. i.e. $x,x'$ have the same image in $\frac{X^{ss}}{SL(2)}$}.
\end{enumerate}
\end{Prop}

\section{Limit Sheaves}\label{s8}

The `limit sheaves' which occur in the  compactification can be explicitly described. Such a description is also useful in carrying out some of the proofs. We\pageoriginale give below some examples of limit sheaves. 

\begin{EXP1}
Suppose that $\sigma$ is the smooth conic in $G$ given by the generators of a given regulus. The fibre over $x$ of the compacification consists of, in addition to some elements of $M(0,2)$, sheaves which are obtained from the rank $2$ trivial bundle on $\mathscr{P}_3$ or a null correlation bundle $(C_1=0$, $C_2b 1)$ by a Hecke transformation. More precisely, the sheaves $\mathscr{F}$ are of the form 
$$
\begin{aligned}
&0\to \mathscr{F}\to E\to 0_L(1)\to 0\\
&{}0\to \mathscr{F}\to 2\mathscr{O}\to 0_{L_1}(1)\oplus 0_{L_2}(1)\to 0\\
&{}0\to \mathscr{F}\to 2\mathscr{O}\to \mathscr{R}\to 0,
\end{aligned}
$$
\end{EXP1}
where $E$ is a null correlation bundle, $L,L_1,L_2$, are lines in $\mathbb{P}_3$ with $L_1\cap L_2=\emptyset$  and $\mathscr{R}$ is an extension of the form 
$$
0\to 0_L(1)\to \mathscr{R}\to 0_L(1)\to 0.
$$
The elements of $M(0,2)$ correspond to smooth conics which are Poncelet related to $\sigma$ and the other cases correspond to the various subcases of degenerate conics which are Poncelet related to $\sigma$. All these sheaves are stable. 

\begin{EXP1}
Let $S_1$ and $S_2$ be two planes in $\mathbb{P}_3$ intersecting along a line and let $p$ and $q$ be two distinct points on $S_1\cap S_2$. Then  considering the lines in $S_1$ passing through $p$ and lines in $S_2$\pageoriginale through $q$ we get a conic $\sigma$ in $G$, which is a pair of lines. Let $\mathbb{P}$ be the plane in $\mathbb{P}\left(\displaystyle\mathop{\wedge}^{2} V\right)$ in which $\sigma$ is contained and let $\mathbb{P}^{\ast}$ be the dual plane. Then $\sigma$ determines two points $e,f$ in $\mathbb{P}^{\ast}$ and the conics $\gamma$ in $\mathbb{P}^{\ast}$ passing through either $e$ or $f$ are those which are Poncelet related to $\sigma$. Take for $\gamma$ a pair of lines in $\mathbb{P}^{\ast}$ one passing through $e$, the other through $f$ and intersecting outside $e\cup f$. Then $\gamma$ defines lines $L$ and $K$ in $\mathbb{P}_3$ with $L$ (resp $K$) contained in $S_1$ (resp $S_2$) and passing through $p$ (resp $q$). 
\end{EXP1}

Corresponding to $\gamma$ we have in the compactification the sheaf $I_{L U q}\oplus I_{K U p}$, where $I_{L I q}$ (resp $I_{K U p}$) denotes the ideal sheaf of $LUq$ (resp $KUp$).


\begin{thebibliography}{99}
\bibitem{key1}
{R. Hartshorne.} Stable vector bundle of rank $2$ on
$\mathbb{P}_3$, \textit{Math. Ann}., (238), 229--280 (1978).

\bibitem{key2}
{A. Hirschowitz and M. S. Narasimhan.} Fibres de't Hooft
speciaux et applications, in \textit{Enumerative Geometry and
  Classical Algebraic Geometry}, Birkhauser 1982.
\end{thebibliography}

\vskip 1cm

\noindent
Tata Institute of Fundamental Research\\
Universitat Kaiserslautern.
