\title{Anisotropic Quadratic Spaces Over The Plane}
\markright{Anisotropic Quadratic Spaces Over The Plane}

\author{By M. Ojanguren, R. Parimala and R. Sridharan}
\markboth{M. Ojanguren, R. Parimala and R. Sridharan}{Anisotropic Quadratic Spaces Over The Plane}

\date{}
\maketitle

\setcounter{page}{353}

\setcounter{pageoriginal}{464}
\section*{Introduction}\pageoriginale

Let $K$ be a field of characteristic $\neq 2$. It is well-known that any quadratic space over the affine line $\mathbb{A}^{1}_K$ is extended from $K$. It was proved in \cite{key7} that any \textit{isotropic} quadratic space over $\mathbb{A}^{n}_K$ is extended from $K$ for all $n$. However, it $n\geq 2$, there do exist, in general, anisotropic quadratic spaces over $\mathbb{A}^{n}_K$ which are not extended from $K$ (\cite{key5}, \cite{key9}). In \cite{key8}, we constructed positive definite quadratic spaces of arbitrarily large ranks over the real affine plane $\mathbb{A}^{2}_{\mathbb{R}}$. More generally, if $K$ is any field which admits of an anisotropic quadratic space $q_0$ of rank $\geqq 3$, then there exist \cite{key10} indecomposable quadratic spaces over $\mathbb{A}^{2}_K$, with $q_o$ as the ``form on the fibre''.

The aim of this paper is to give a classification of anisotropic\\ quadratic spaces over $\mathbb{A}^{2}_K$ in terms of linear algebraic data. Our method is to exploit a theorem \cite[theorem 2.1]{key6}, which reduces the classification problem to that of quadratic spaces over the projective plane and then use suitable adaptations of the work of Barth-Hulek (\cite{key1}, \cite{key2}, \cite{key4}). Using this classification, we show that the set of isometry classes of indecomposable quadratic spaces of second Chern class $4$ (see \S\ 2 for definition) over $\mathbb{A}^{2}_K$ with a given form $q_o$ on the fibre is in bijection with the orbit of an anisotropic conic $C_o$ over $K$. 

For vector bundles over $\mathbb{P}^{2}$, we follow generally the definitions and noatation of Barth-Hulek \cite{key2} and Hulek \cite{key4}.

\section{Classification of Quadratic spaces over \texorpdfstring{$\mathbb{P}^{2}_K$}{eq}}\label{s1}\pageoriginale

We recall, in this section, some results of Barth-Hulek regarding the classification of quadratic spaces over $\mathbb{P}^{2}_K$ in  terms of linear algebra (\cite{key2}, \cite{key4}). We remark that though these results are stated by them only for the field of complex numbers, they are valid for any field. Throughout this section, we denote by $K$ a field of characteristic $\neq 2$. 

By a \textit{skew-symmetric, pre-stable triple} $\alpha$ we mean a triple $\alpha=\left(\alpha_0,\alpha_1,\alpha_2\right)$, $\alpha_i\in M_n(K)$, $\alpha^{t}_i=-\alpha_i$ such that for any $v\in K^{n}$, $V\neq 0$, the subspace of $K^{n}$ spanned by $\{\alpha_0 v, \alpha_1 v,\alpha_2v\}$ is at least two-dimensional. Two pre-stable triples $\alpha=\left(\alpha_0,\alpha_1,\alpha_2\right)$ and $\beta=(\beta_0,\beta_1,\beta_2)$ are said to be equivalent, if there exists $u\in GL_n(K)$ such that $\alpha_i=u\beta_iu^{t}$, $0\leq i\leq 2$. We denote the set of equivalence classes of such triples by $\mathscr{K}(n)$. Given any skew-symmetric, pre-stable triple $\alpha$, we shall associate to it a quadratic space over $\mathbb{P}^{2}_K$. 

Let $\mathfrak{o}$ be the structure sheaf of $\mathbb{P}^{2}_K$ and let $Z_0,Z_1,Z_2$ be the homogeneous coordinates, so that $\Gamma(\mathfrak{o}(1))$ has $Z_0,Z_1,Z_2$ as a basis. Let $V=(\Gamma(\mathfrak{o}(1))^{\ast}$ and let $\{v_0,v_1,v_2\}$ be the basis of $V$, dual to the basis $\{Z_0,Z_1,Z_2\}$. We define a linear map $A(\alpha):K^{n}\otimes V\to K^{n}\otimes V^{\ast}$ by $A(\alpha) (\varphi \otimes v_i)=\alpha_{i+1}(\varphi)\otimes Z_{i\ 1}-\alpha_{i-1}(\varphi)\otimes Z_i$, where $i=0,1,2\pmod 3$.

For the choice of the canonical basis of $K^{n}$. and the basis for $V$ and $V^{\ast}$ defined above, the matrix of $A(\alpha)$ is given by 
$$
\begin{pmatrix}
0 &\alpha_2 &-\alpha_1\\
-\alpha_2 & 0 &\alpha_0\\
\alpha_1& -\alpha_0 & 0
\end{pmatrix}
$$
Let\pageoriginale $U$ denote the image of $A(\alpha)$. Let $a:K^{n}\otimes \mathfrak{o}(-1)\to U\otimes \mathfrak{o}$ be the composite $K^{n}\otimes \mathfrak{o}(-1)\xrightarrow{1\otimes s} K^{n}\otimes V\otimes \mathfrak{o}\xrightarrow{A(\alpha)\otimes 1} U\otimes\mathfrak{o}$, and $c:U\otimes \mathfrak{o}\to K^{n}\otimes \mathfrak{o}(1)$, the restriction to $U\otimes \mathfrak{o}$ of the map $K^{n}\otimes V^{\ast}\otimes \mathfrak{o}\xrightarrow{1\otimes s^{t}} K^{n} \otimes \mathfrak{o}(1)$, where $s:\mathfrak{o}(-1)\to V\otimes \mathfrak{o}$ is induced by the multiplication $\Gamma(\mathfrak{o}(1))\otimes \mathfrak{o}(-1)\to \mathfrak{o}$ and $s^{t}$ denotes the transpose of $s$. 

We have a self-dual monad (\cite[p.10]{key2})
$$
\xymatrix{M(\alpha):K^{n}\ar@{=}[d]\otimes \mathfrak{o}(-1)\ar[r]^-{a}&U\otimes\ar[d]_{\varphi} \mathfrak{o}\ar[r]^-{c}&K^{n}\ar@{=}[d]\otimes \mathfrak{o}(1)\\
M(\alpha)^{\ast}:K^{n}\otimes \mathfrak{o}(-1)\ar[r]^-{c^t}&U^{\ast}\otimes \mathfrak{o}\ar[r]^-{a^t}&K^{n}\otimes \mathfrak{o}(1),}
$$
the map $\varphi:U\xrightarrow{\sim}U^{\ast}$ being defined by $\varphi(x)=i^{t}(y)$, where $i:U\to K^{n}\otimes V^{\ast}$ denotes the canonical inclusion and $y$ is any preimage of $x$ under $A(\alpha)$. Let $\mathscr{E}(\alpha)$ denote the cohomology of the monad $M(\alpha)$. The isomorphism $\varphi$ of $M(\alpha)$ with $M(\alpha)^{\ast}$ yields an isomorphism $q:\mathscr{E}(\alpha)\xrightarrow{\sim}\mathscr{E}(\alpha)^{\ast}$. Using the fact that $A(\alpha)$ is symmetric, one verifies that $q^{t}=q$, so that $\mathscr{E}(\alpha)$ carries a non-singular quadratic structure. We note \cite{key4} that $c_1(\mathscr{E}(\alpha))=0$, $c_2(\mathscr{E}(\alpha))=n$, and rank $\mathscr{E}(\alpha)$ = rank $A(\alpha)-2n$ and $\mathscr{E}(\alpha)$ is $s$-stable, i.e. $H^{o}(\mathscr{E}(\alpha))=0$. 

Let $\alpha$ and $\alpha'$ be equivalent skew-symmetric pre-stable triples and let $u\in GL_n(K)$ be such that $u^{t}\alpha'_iu=\alpha_i$, $0\leq i\leq 2$. We then have an isomorphism of the corresponding self-dual monads: 
$$
\xymatrix{&K^{n}\ar[dl]_{u\otimes 1}\otimes\mathfrak{o}\mid(-1)\ar@{-}[d]_{Id}\ar[r]^{a}&U\ar[dl]_-{\left(u^{t}\right)^{-1}\otimes 1}\otimes\ar@{-}[d]^{\varphi \otimes 1}\mathfrak{o}\ar[r]^{c}& K^{n}\otimes \mathfrak{o}\ar[dl]_{\left(u^{t}\right)^{-1}\otimes 1}\ar@{->}[dd]_{Id}(1)\\
K^{n}\otimes \mathfrak{o}(-1)\ar@{->}[dd]_{Id}\ar[r]^{a'}&U'\otimes\ar@{-}[d] \mathfrak{o}\ar[r]^{c'}\ar[d]&K^{n}\otimes \mathfrak{o}\ar@{-}[d](1)\\
& K^{n}\otimes\ar[dl]_{u\otimes 1} \mathfrak{o}(-1)\ar[r]^{c^{t}}\ar[d]&U^{\ast}\ar[dl]_{u\otimes 1}\otimes \mathfrak{o}\ar[r]^{a^{t}}\ar[d]&K^{n}\otimes\ar[dl]_{\left(u^{t}\right)^{-1}\otimes 1} \mathfrak{o}(1)\\
K^{n}\otimes \mathfrak{o}(-1)\ar[r]^{c^{'t}}&U'^{\ast}\otimes\mathfrak{o} \ar[r]^{a^{'t}}& K^{n}\otimes \mathfrak{o}(1)}
$$
\pageoriginale
where $u:U\to U'$ is induced by the map $u\otimes 1: K^{n}\otimes V^{\ast}\to K^{n}\otimes V^{\ast}$. This yields an isometry $u:\mathscr{E}(\alpha)\to \mathscr{E}(\alpha')$, i.e. we have a commutative diagram. 
$$
\xymatrix{\mathscr{E}(\alpha)\ar[d]_{q}\ar[r]^{u}&\mathscr{E}(\alpha')\ar[d]^{q}\\
\mathscr{E}(\alpha)^{\ast}&\ar[l]_{u^{t}}\mathscr{E}(\alpha')^{\ast}}
$$
Let $Q(n)$ denote the set of isometry classes of s-stable quadratic spaces $(\mathscr{E},q)$ over $\mathbb{P}^{2}_K$ with $c_2(\mathscr{E})=n$. The assignment $\alpha\to(\mathscr{E}(\alpha),q)$ defines a map $\mathscr{K}(n)\xrightarrow{m}Q(n)$. 

We first show that $m$ is injective. Suppose $\alpha$, $\alpha'\in \mathscr{K}(n)$ with $(\mathscr{E}(\alpha),q)\approx (\mathscr{E}(\alpha'),q')$. By definition, $\mathscr{E}(\alpha)$ and $\mathscr{E})(\alpha)$ are the cohomologies\pageoriginale of the self-dual monads $M(\alpha)$, $M(\alpha')$ respectively. In view of (\cite[prop. 4]{key2}), an isometry $f:\mathscr{E}(\alpha)\to \mathscr{E}(\alpha')$ is induced by a unique isomorphism of monads.
$$
\xymatrix{K^{n}\otimes\ar[d]^{u\otimes 1} \mathfrak{o}(-1)\ar[r]^-{a}&U\otimes\ar[d]^{\beta} \mathfrak{o}\ar[r]^-{c}&K^{n}\otimes\ar[d]^{\left(u^{t}\right)^{-1}\otimes 1} \mathfrak{o}(1)\\
K^{n}\otimes \mathfrak{o}(-1)\ar[r]^-{a'}&U'\otimes\mathfrak{o}\ar[r]^-{c'}& K^{n}\otimes \mathfrak{o}(1)}
$$
We therefore have a commutative diagram 
$$
\xymatrix{K^{n}\otimes V\ar[d]_{u\otimes 1}\ar[r]^{A(\alpha)}&K^{n}\otimes\ar[d]^{\left(u^{t}\right)^{-1}\otimes 1} V^{\ast}\\
K^{n}\otimes V^{A}\ar[r]^{\left(\alpha'\right)}&K^{n}\otimes V^{\ast},}
$$
so that $u^{t}\alpha'_iu=\alpha_i$, $0\leq i\leq 2$. This $\alpha'\sim\alpha$. 

We next show that $m$ is surjective.

Let $(\mathscr{E},q)$ be a quadratic space over $\mathbb{P}^{2}_K$ which is $s$-stable and with $c_2(\mathscr{E})=n$. Then $\dim \left(H^{1}(\mathscr{E}(-2))\right)=\dim \left(H^{1}(\mathscr{E}(-1))\right)=n$. We have multiplication maps 
$$
\alpha_{Z_{i}}:H^{1}(\mathscr{E}(-2))\to H^{1}(\mathscr{E}(-1))
$$
for $0\leq i \leq 2$. Let $\theta:H^{1}(\mathscr{E}(-1))\to H^{1}(\mathscr{E}(-2))$ be the composite $H^{1}(\mathscr{E}(-1))\xrightarrow{H^{1}(q(-1))}H^{1}\left(\mathscr{E}^{\ast}(-1)\right)\xrightarrow{S}H^{1}(\mathscr{E}(-2))^{\ast}$, where $s$ is the\pageoriginale Serre-duality map. If $\{e_i\}$ is a basis of  $H^{1}(\mathscr{E}(-1))$ and $\{f_i\}$ a basis of $H^{1}(\mathscr{E}(-2))$ dual to the basis $\{\theta(e_i)\}$, then $\alpha_{Z_{i}}$ are represented with respect to these bases by matrices $\alpha_i\in M_n(K)$ with $\alpha^{t}_i=-\alpha_i$(\cite[Th.20]{key8}). Let $\alpha=(\alpha_i)$. One can show (\cite[Proof of theorem 1.5.2]{key4}), that $\mathscr{E}$ belongs to the monad 
$$
H^{1}(\mathscr{E}(-2))\otimes \mathfrak{o}(-1)\to H^{1}(\mathscr{E}\otimes \Omega)\otimes \mathfrak{o}\to H^{1}(\mathscr{E}(-1))\otimes\mathfrak{o}(1). 
$$
If we identify  $H^{1}(\mathscr{E}(-1))$ with $H^{1}(\mathscr{E}(-2))^{\ast}$ through $\theta$ and identify both of these spaces with $K^{n}$ through the choices of the bases described above, then $\mathscr{E}$ belongs to the monad 
$$
M(\alpha):K^{n}\otimes \mathfrak{o}(-1)\xrightarrow{a}U\otimes\mathfrak{o}\xrightarrow{c}K^{n}\otimes \mathfrak{o}(1), 
$$
which is self dual with respect to $(\mathscr{E},q)$. Thus $(\mathscr{E},q)\xrightarrow{\sim}\mathscr{E}(\alpha)$ and we have the following 

\begin{thm}\label{thm1.1}
The map $[\alpha]\mapsto [\mathscr{E}(\alpha)]$ is a bijection between $\mathscr{K}(n)$ and $Q(n)$. 
\end{thm}

\section[Anisotropic quadratic spaces...]{Anisotropic quadratic spaces over the affine\texorpdfstring{\\}{eq} plane}\label{s2}

By an \textit{anisotropic quadratic space} over \textit{an irreducible scheme}, we mean a quadratic space which is anisotropic over the function field. 

The problem of classification of anisotropic quadratic spaces over $\mathbb{A}^{2}_K$ is equivalent to the problem of classification of anisotropic quadratic spaces over $\mathbb{P}^{2}_K$ thanks to the following 

\begin{thm}\label{thm2.1}
(\cite[Theorem 2.1]{key6}),\pageoriginale Every quadratic space over $\mathbb{A}^{2}_K$ extends to $\mathbb{P}^{2}_K$ and the extension is unique upto isometry if the space is anisotropic. 
\end{thm}

The following lemma describes the type of bundles on $\mathbb{P}^{2}_K$ one obtains, by extending anisotropic quadratic spaces from $A^{2}_K$. 

\begin{lem}\label{lem2.2}
Let $q$ be an anisotropic quadratic space over $\mathbb{A}^{2}_K$ which does not represent any unit of $K$. Let $\mathscr{E}(q)$ be the extension of $q$ to $\mathbb{P}^{2}_K$. Then 
\begin{enumerate}
\renewcommand{\theenumi}{\roman{enumi}}
\renewcommand{\labelenumi}{(\theenumi)}
\item $\mathscr{E}(q)$ is $s$-stable (in the sense of Hulek) i.e., $H^{\circ}(\mathscr{E}(q))\\=H^{\circ}(\mathscr{E}(q)^{\ast})=0$. 

\item $c_1(\mathscr{E}(q))=0$

\item The bundle $\mathscr{E}(q)$, restricted to every line in $\mathbb{P}^{2}_K$, defined over $K$, is trivial. 
\end{enumerate}
\end{lem}

\begin{Proof}
Since $\mathscr{E}(q)\xrightarrow{\sim}\mathscr{E}(q)^{\ast}$, to check that $\mathscr{E}(q)$ is $s$-stable, it is enough to prove that $H^{\circ}(\mathscr{E}(q))=0$. Let $s$ be a global section of $\mathscr{E}(q)$. Then, the map $x\mapsto q(s(x))$ is a global function on $\mathbb{P}^{2}_K$ and is hence a constant. If this constant is non-zero, then $s$ defines a trivial line sub-bundle which is an orthogonal summand of $\mathscr{E}(q)$. This contradicts (in view of Theorem~\ref{thm2.1}) the assumption that $q$ does not represent a unit. If the constant is zero, since $q$ is anisotropic, we have $s(x)=0$ for all $x$, implying $s=0$. 

(ii) Since $\mathscr{E}(q)$ supports a non-singular quadratic form, $\det \mathscr{E}(q)$ is\pageoriginale trivial and hence $c_1(\mathscr{E}(q))=0$. 

(iii) Let $L\subset \mathbb{P}^{2}_K$ be a line defined over $K$. Since $\mathscr{E}(q)\mid L$ supports an anisotropic form, the underlying bundle of $\mathscr{E}(q)\mid L$ is trivial. 

Motivated by the above lemma, we define an anisotropic quadratic space $q$ over $\mathbb{A}^{2}_K$ to the $s$-stable if it does not represent a unit of $K$. For any anisotropic quadratic space $q$ over $\mathbb{A}^{2}_K$, we define its \textit{second Chern class} $c_2(q)$ to be $c_2(\mathscr{E}(q))$ 
\end{Proof}

\begin{Prop}\label{Prop2.3}
Let $q$ be an $s$-stable quadratic space over $\mathbb{A}^{2}_K$. Then $C_2(q)$ is an even integer. 
\end{Prop}

\begin{Proof}
Let $\mathscr{E}=\mathscr{E}(q)$ denote the extension of $q$ to $\mathbb{P}^{2}_K$. Let $Z_i$, $0\leq i\leq 2$ be the co-ordinates of $\mathbb{P}^{2}_K$ and $\alpha_i$,$ 0\leq i\leq 2$ be the skew symmetric matrices representing the multiplication maps $\alpha_{Z_{i}}:H^{1}(\mathscr{E}(-2))\to H^{1}(\mathscr{E}(-1))$ with respect to suitable bases, as described in \S\ 1. We recall (\cite[\S\ 1.7.1]{key4}) that for $\lambda$, $\mu$, $v\in K$, the bundle $\mathscr{E}\mid Z=\lambda Z_0+\mu Z_1+vZ_2$ is trivial if and only if $\alpha_{\mathbb{Z}}$ is an isomorphism. By Lemma~\ref{lem2.2} (iii), $\mathscr{E}\mid Z_i$ is trivial and  hence $\alpha_i$ are non-singular for $0\leq i\leq 2$. Since $\alpha_i$ are skew symmetric and non-singular, it follows that $n=c_2(\mathscr{E})$  is even. In fact, if $K=\mathbb{R}$, the field of real numbers, the following stronger result, which, however, is not needed in what follows, holds.
\end{Proof}

\begin{Prop}\label{Prop2.4}
For any $s$-stable quadratic space $q$ over $\mathbb{A}^{2}_{\mathbb{R}}$, $c_2(q)$ is divisible by $4$. 
\end{Prop}

\begin{Proof}
Let $\pi:\mathbb{P}^{2}_{\mathbb{C}}\to \mathbb{P}^{2}_{\mathbb{R}}$ denote the projection. The set of all lines $Z=\lambda Z_o+\mu Z_1+vZ_2$ in $\mathbb{P}^{2}_{\mathbb{C}}$, such that $\pi^{\ast}\mathscr{E}(q)Z$ is not trivial\pageoriginale is a curve in the dual plane, defined by the equation $\det (X_o\alpha_o+X_1\alpha_1+X_2\alpha_2)=0$, unless it is the whole plane. The latter possibility does not arise because $\alpha_i$ being non-singular. $\det (X_o\alpha_o+X_1\alpha_1+X_2\alpha_2)\neq 0$. In view of (\cite[\S\ 1.7.1]{key4}) $c_2(q)=deg \det(X_o\alpha_o+X_1\alpha_1+X_2\alpha_2)$. We have $\det(X_o\alpha_0+X_1\alpha_1+X_2\alpha_2)=(Pf(X_0\alpha_0+X_1\alpha_1+X_2\alpha_2))^{2}$, where $Pf$ denotes the Pfaffian. Since $\mathscr{E}(q)$ restricted any $\mathbb{R}$ rational line is trivial, the curve $pf(X_o\alpha_o+X_1\alpha_1+X_2\alpha_2)=0$ which is defined over $\mathbb{R}$, has no $\mathbb{R}$-rational point. Thus, $Pf(X_o\alpha_o+X_1\alpha_1+X_2\alpha_2)=0$ is a curve of even degree, so that $c_2(q)=2deg pf(X_o\alpha_o+X_1\alpha_1+X_2\alpha_2)$ is divisible by $4$. 

Let $(\mathscr{E},q)$ be a quadratic space over $X=\mathbb{A}^{2}_K$ or $\mathbb{P}^{2}_K$. Then, there exists a quadratic form $q_o$ over $K$ such that at every point $x\in X$, the quadratic space $(\mathscr{E}_x,q_x)$ which is the fibre at $x$, is isometric to $q_O$. This can be deduced if $X=\mathbb{A}^{2}_K$ from the fact (Karoubi's theorem) that $(\mathscr{E},q)$ is ``stably extended'' from a form $q_O$ over $K$. If $X=\mathbb{P}^{2}_K$, and if $q_1$, $q_2$, $q_3$ are the forms over $K$  corresponding to the restrictions of $(\mathscr{E},q)$ to the affine planes, then the connectedness of $\mathbb{P}^{2}_K$ shows that $q_1\approx q_2\approx q_3$ We call $q_o$ the ``form on the fibre of $(\mathscr{E},q)$''. Clearly $(\mathscr{E},q)$ is anisotropic if and only if $q_o$ is anisotropic. 
\end{Proof}

\begin{Prop}\label{Prop2.5}
Let $(\mathscr{E},q)$ be an $s$-stable quadratic space over $\mathbb{P}^{2}_K$ with $\alpha=(\alpha_i)$ as the corresponding skew-symmetric triple. Then $\omega=\alpha_2\alpha_0^{-1}\alpha_1-\alpha_1\alpha_0^{-1}\alpha_2$ is symmetric and $\omega$ modulo its radical is the form on the fibre of $(\mathscr{E}, q)$. 
\end{Prop}

\begin{Proof}
Let $U$ be the image of the map $A(\alpha):K^{3n}\to K^{3n}$, where $A(\alpha)$ is as in \S\ 1. Let $\alpha:K^{n}\to U$ be defined as $a=\begin{pmatrix}
-\alpha_1\\
\alpha_0\\
0
\end{pmatrix}$

Let\pageoriginale $c:U\to K^{n}$ be the restriction of the map $K^{3n}\xrightarrow{(0,0,1)}K^{n}$; then the fibre of $\mathscr{E}(q)$ over $(0,0,1)$ is the homology of the complex 
$$
K^{n}\xrightarrow{a}U\xrightarrow{c}K^{n}
$$
with the form on the fibre induced by the map $\varphi:U\to U^{\ast}$
defined by $\varphi(x)(z)=<z,y>$, where $A(\alpha)(y)=x$, and $\langle ,\rangle$ denotes the canonical inner product on $K^{3n}$. Let 
$$
T=
\begin{pmatrix}
u & u\alpha_1\alpha^{-1}_0&u\alpha_2\alpha_0^{-1}\\
0 & 1 & 0\\
0 & 0 &1
\end{pmatrix}
$$
where $u u^{t}=\begin{pmatrix}
\widetilde{\omega} & 0\\
0 & 0
\end{pmatrix}$ and $\widetilde{\omega}$ is an invertible $r$ by $r$ matrix Transforming $K^{3n}$ by $T$,we get an isometric space given as the homology of the complex 
$$
K^{n}\xrightarrow{\begin{pmatrix}
0\\
\alpha_0\\
0
  \end{pmatrix}
} K^{r}\oplus K^{n}\oplus K^{n}\xrightarrow{(0,0,1)} K^{n}(\ast)
$$
Since 
$$
TA(\alpha)T^{t}=
\begin{pmatrix}
\omega & 0 & 0\\
0 & 0 &\alpha_0\\
0 & \alpha_0 & 0
\end{pmatrix}
$$
$\left(T^{t}\right)^{-1}\varphi T^{-1}$\pageoriginale is represented by the matrix $\begin{pmatrix}
\widetilde{\omega}^{-1} &  0 & 0\\
0 & 0 &-\alpha^{-1}_0\\
0 & \alpha^{-1}_0 &0
\end{pmatrix}$ so that the form on the fibre of the complex is given by $\widetilde\omega^{-1}$. This proves the Proposition. 

We shall now fix an anisotropic form $q_o$ over $K$ of rank $r\geq 2$ and an integer $n\geq r$. Let $q$ be an $s$-stable quadratic space over $\mathbb{A}^{2}_K$ with $q_o$ as the form on the fibre and with $c_2(q)=n$. Let $(\alpha_i)$, $0\leq i\leq 2$ be a (skew-symmetric) triple corresponding to the bundle $\mathscr{E}(q)$. By Proposition~\ref{Prop2.5}, the form on the fibre of $\mathscr{E}(q)$ is $q_1$, where $q_1\perp <0,\ldots,0>_{n-r}\xrightarrow{\sim}\omega$ Let $u\in GL_n(K)$ be such that $u\omega u^{t}=q_o\perp <0,\ldots0>_{n-r}$. If $\beta_i=u\alpha_iu^{t}$, $0\leq i\leq 2$, then $\beta_2\beta^{-1}_0\beta_1-\beta_1\beta_0^{-1}\beta_2=q_o\perp <0,\ldots,0>_{n-r}$. Thus in the equivalence class of $\alpha$, there exists a triple $\beta$ with $\beta_2\beta^{-1}_0\beta_1-\beta_1\beta^{-1}_0\beta_2=q_0\perp <0,\ldots 0>_{n-r}$. 

Let $\mathscr{K}(n,q_0)=\{\alpha=(\alpha_0,\alpha_1,\alpha_2)\mid \alpha_i\in GL_i(K)$, $\alpha$ pre-stable, $\alpha^{t}_i=-\alpha i$ and $\alpha_2 \alpha_0^{-1}\alpha_1-\alpha_1\alpha^{-1}_0\alpha_2=q_0\perp <0,\ldots,0>_{n-r}\}$ modulo the equivalence relation $\alpha\sim \beta$ if and only if there exist $u\in GL_n(K)$, $\lambda \in K^{\ast}$ such that $\beta_i=\lambda u^{t}\alpha_i u$, $0\leq i\leq 2$. We set $Q(n, q_0)$= set of similitude classes of $s$-stable quadratic spaces over $\mathbb{A}^{2}_K$ with $q_o$ as the form on the fibre and $c_2=n$, similitude being isometry upto a scalar of $K^{\ast}$ 

As an immediate consequence of Theorem~\ref{thm1.1}. we have the following 
\end{Proof}

\begin{thm}\label{thm2.6}
The\pageoriginale assignment $[q]\mapsto [\alpha(\mathscr{E}(q))]$ gives rise to a bijection between the sets $Q(n,q_0)$ and $\mathscr{K}(n,q_0)$. 
\end{thm}

\begin{remark}
The condition of pre-stability of $\alpha$ in the definition of $Q(n,q_0)$ can be dropped if $n$=rank $q_0$. 
\end{remark}

\section[Quaternion algebras associated to triples...]{Quaternion algebras associated to triples of\\ skew symmetric matrices}\label{s3}

\begin{dfn}\label{dfn3.1}
A triple $\alpha=(\alpha_0,\alpha_1,\alpha_2)$, $\alpha_i\in GL_n(K)$ is anisotropic if $\alpha$ is pre-stable, $\alpha^{t}_i=-\alpha_i$, $0\leq i\leq 2$ and $\widetilde{\omega}$ is anisotropic, where $\omega=\alpha_2\alpha^{-1}_0\alpha_{1}-\alpha_1\alpha_0^{-1}\alpha_2$, and $\widetilde{\omega}$ denotes $\omega$ modulo its radical. 

We observe that a pre-stable triple $\alpha$ is anisotropic if and only if the corresponding quadratic space $\mathscr{E}(\alpha)$ is anisotropic. 
\end{dfn}


\begin{lem}\label{lem3.2}
Let $\alpha=(\alpha_0,\alpha_1,\alpha_2)$ be an anisotropic triple. Then, for $\lambda$, $\mu$, $v\in K$, not all zero, $\lambda \alpha_0+\mu\alpha_1+v\alpha_2$ is non-singular.  
\end{lem}

\begin{Proof}
The triple $\alpha$ defines an anisotropic quadratic space $\mathscr{E}(\alpha)$ over $\mathbb{P}^{2}_K$. Since any anisotropic quadratic space over $\mathbb{P}^{1}_K$ is trivial as a vector bundle, $\mathscr{E}(\alpha)$, restricted to any $K$ rational line $Z$ of $\mathbb{P}^{2}_K$ is trivial. Hence the corresponding multiplication map $\alpha_Z:H^{1}(\mathscr{E}(-2))\to H^{1}(\mathscr{E}(-1))$ is an isomorphism. For $Z=\lambda Z_0+\mu Z_1+v Z_2$, $\alpha_Z$ is represented by the matrix $\lambda\alpha_0+\mu\alpha_1+v\alpha_2$ for a suitable choice of bases. Hence, it follows that $\lambda \alpha_0+\mu\alpha_1+v\alpha_2$ is non-singular. 
\end{Proof}


\begin{Prop}\label{Prop3.3}
Let $\alpha=(\alpha_0, \alpha_1, \alpha_2)$ be an anisotropic triple with $\alpha_i\in GL_4(K)$, $0\leq i\leq 2$. Then $\omega=\alpha_2\alpha^{-1}_0\alpha_1\alpha_1 \alpha^{-1}_0\alpha_2$ is non-singular and determinant $\omega$ is a square.
\end{Prop}

\begin{Proof}
If rank $\omega \leq 2$, since $\alpha$ is pre-stable, it gives rise
to an $s$-stable quadratic space of rank $\leq 2$ over
$\mathbb{A}^{2}_K$. This contradicts the\pageoriginale fact (\cite[Proposition 1.1]{key9}) that any rank $2$ quadratic space over $\mathbb{A}^{2}_K$ is extended from $K$. Hence rank $\omega \geq 3$. Let $U\in GL_4(K)$ be such that $U\omega U^{t}=\theta <1$, $-a, -b, c>$, with $\theta, a, b, c, \in K^{\ast}$. Replacing $\alpha_i$ by $U\alpha_i U^{t}$, we assume, without loss of generality, that 
\begin{equation*}
\alpha_2\alpha^{-1}_0\alpha_1-\alpha_1\alpha^{-1}_0\alpha_2=\theta <1, -a, -b, c>\tag{$\ast$}
\end{equation*}
Replacing $\alpha_i$ by $\alpha_i+x\alpha_0$, $i=1,2$ or $\alpha_1$ by $\alpha_1+x\alpha_2$, $x\in K$ neither changes the relation $(\ast)$ nor the non-singularity of the $\alpha_i$, in view of Lemma~\ref{lem3.2}. We therefore assume that $\alpha_i$ have the following form 
$$
\alpha^{-1}_0=
\begin{pmatrix}
\lambda \epsilon & A\\
-A^{t}& \mu \epsilon
\end{pmatrix},
\alpha_1=
\begin{pmatrix}
0 & B\\
-B^{t} & 0
\end{pmatrix} 
\alpha_2=
\begin{pmatrix}
0 & C\\
-C^{t} & v \epsilon
\end{pmatrix}
$$
where $\lambda$, $\mu$, $v\in K$, $A\in M_2(K)$, $B, C\in GL_2(K)$ and $\epsilon=\begin{pmatrix}
0 & 1\\
-1 & 0
\end{pmatrix}$.

The condition $(\ast)$ yields: 
\begin{equation}\label{eqn1}
\mu(C\in B^{t}-B\in C^{t})=\theta \begin{pmatrix}
1 & 0\\
0 & -a
\end{pmatrix}
\end{equation}
\begin{equation}\label{eqn2}
\lambda\left(C^{t}\in B-B^{t}\in C\right)+v\left(\epsilon A^{t}B-B^{t}A\epsilon\right)=\theta \begin{pmatrix}
-b & 0\\
0 & c
\end{pmatrix}
\end{equation}
\begin{equation}\label{eqn3}
CA^{t}B-BA^{t}C=\mu v B
\end{equation}

From (\ref{eqn3}), we get $CA^{t}-BA^{t}C B^{-1}=\mu v$. Comparing the traces, we have $2\mu v=0$. From (\ref{eqn1}), it follows that $\mu\neq 0$ since $\theta$, $a\in K^{\ast}$, so that $v=0$. From (\ref{eqn2}), it follows that $\lambda\neq 0$ since $\theta$, $b\in K^{\ast}$.\pageoriginale We get from (\ref{eqn1}) and (\ref{eqn2}).
$$
\begin{aligned}
&(\det B)C-(\det C)B C^{-1}B=\theta \mu^{-1}\begin{pmatrix}
1 & 0\\
0 & -a
  \end{pmatrix} \epsilon^{-1} B\\
&{}(\det C) BC^{-1}B-(\det B)C=\theta \lambda^{-1}B\epsilon^{-1}\begin{pmatrix}
-b & 0\\
0 & c
\end{pmatrix}
\end{aligned}
$$
Thus
$$
0\mu^{-1}\begin{pmatrix}
1 & 0\\
0 & -a
\end{pmatrix} \epsilon^{-1}B=-\theta \lambda^{-1}B\epsilon^{-1}\begin{pmatrix}
-b & 0\\
0 & c
\end{pmatrix}
$$
Comparing the determinants, we obtain $\lambda^{2}a=\mu^{2}bc$. Since $\lambda, a\in K^{\ast}$, it follows that $c\neq 0$, so that $\omega$ is non-singular. Further $\det \omega=\theta^{4}abc=\theta^{4}\lambda^{-2}\mu^{2}(bc)^{2}$ which is a square. 
\end{Proof}

\begin{lem}\label{lem3.4}
Let $q=\theta <1,-a,-b, ab>$, with $\theta, a, b, \in K^{\ast}$. 

Let $\alpha\in M_4(K)$ which satisfies $\alpha+q\alpha^{t}q^{-1}.=0$
and $\alpha\cdot q\alpha^{t}q^{-1}\in K$. Then $\alpha$ has either of the following forms 
\begin{equation*}
\begin{pmatrix}
0 & \mu & x & y\\
a\mu& 0 & -ay & -x\\
xb & yb & 0 & \mu\\
-yab & -xb & a\mu & 0
\end{pmatrix}\tag{$\ast$}
\end{equation*}
\begin{equation*}
\begin{pmatrix}
0 & \mu & x & y\\
a\mu & 0 & ay & x\\
xb & -yb & 0 &-\mu\\
-yab & xb & -a\mu & 0
\end{pmatrix}\tag{$\ast\ast$}
\end{equation*}
\end{lem}

\begin{Proof}
Let\pageoriginale $\eta_a\begin{pmatrix}
1 & 0\\
0 & -a
\end{pmatrix}, \alpha=\begin{pmatrix}
\alpha_1 & \alpha_2\\
\alpha_3 & \alpha_4
\end{pmatrix}$ with $\alpha_i\in M_2(K)$.

Then, $\alpha+q\alpha^{t}q^{-1}=0$ gives 
\begin{equation*}\label{eqnnt1}
\begin{aligned}
&\alpha_1\eta_a+\eta_a\alpha^{t}_1=0\\
&{}\alpha_4\eta_a+\eta_a\alpha^{t}_4=0\\
&{}\eta_a\alpha^{t}_3-b\alpha_2\eta_a=0
\end{aligned}\tag{1}
\end{equation*}
The fact that $\alpha q\alpha^{t}q^{-1}\in K$ gives 
\begin{equation*}\label{eqnnt2}
\begin{aligned}
&\alpha^{2}_1+\alpha_2\alpha_3=\alpha_3\alpha_2+\alpha^{2}_4\in K\\
&{}\alpha_1\alpha_2+\alpha_2\alpha_4=\alpha_3\alpha_1+\alpha_4\alpha_3=0
\end{aligned}\tag{2}
\end{equation*}
From (\ref{eqnnt1}), one obtains $\alpha_1=\begin{pmatrix}
0 & \mu\\
a\mu & 0
\end{pmatrix}, \alpha_4=\begin{pmatrix}
0 & 0\\
a0 & 0
\end{pmatrix}$
From (\ref{eqnnt2}), it follows that $\alpha^{2}_1-\alpha^{2}_4=\alpha_3\alpha_2-\alpha_2 \alpha_3$. Comparing the trace we get $Tr\left(\alpha^{2}_1\right)=Tr(\alpha^{2}_{4})$, i.e. $\mu^{2}=\theta^{2}$. According as $\mu=\pm \theta$, we get $\alpha$ to be of the form ($\ast$) or ($\ast\ast$) respectively. 
\end{Proof}

\begin{thm}\label{thm3.5}
Let $\alpha=(\alpha_0,\alpha_1,\alpha_2)$ be an anisotropic triple with $\alpha_i\in GL_4(K)$, $0\leq i\leq 2\leq$. Let $\alpha_2\alpha^{-1}_0\alpha_1-\alpha_1\alpha_0^{-1}\alpha_2=\omega$

Then 
\begin{enumerate}
\renewcommand{\theenumi}{\roman{enumi}}
\renewcommand{\labelenumi}{(\theenumi)}
\item The K-subalgebra $H(\alpha)$ of $M_4(K)$ generated by $\alpha_1\alpha_0^{-1}$ and $\alpha_2\alpha_0^{-1}$ is a quaternion algebra over $K$. 

\item The\pageoriginale form $\omega$ is non-singular and the involution on $M_4(K)$ given by $x\to \omega x^{t}\omega^{-1}$ restricts to the canonical involutions on $H(\alpha)$ and $H(\alpha)^{c}$, where $H(\alpha)^{c}$ denotes the commutant of $H(\alpha)$ in $M_4(K)$. 

\item The reduced norm on $H(\alpha)$ is isometric to $\omega$ up to a scalar. 

\item For any element $x=\lambda+\mu \alpha_1\alpha_0^{-1}+v\alpha_2\alpha_0^{-1}\in H(\alpha)$; $\lambda, \mu, v\in K$, we have $Nrd(x)=Pf(\lambda \alpha_0+\mu\alpha_1+v\alpha_{2})$. 
\end{enumerate}
\end{thm}

\begin{Proof}
By Proposition~\ref{Prop3.3}, it follows that $\omega$ is non-singular
and that disc $\omega=1$. Replacing $\alpha_i$ by $u\alpha_i u^{t}$
and $\omega$ by $u\omega u^{t}$ changes the algebra $H(\alpha)$ to
$uH(\alpha)u^{-1}$. We assume without loss of generality that
$\omega=\theta<1,-a,-b,ab>$. Changing $\alpha_i$, $i=1,2$ to
$\alpha_i+x\alpha_0$ or $\alpha_1+x\alpha_2$, $x\in K$, does not
change $\omega$ or the algebra $H(\alpha)$, nor does it affect the
invertibility of the $\alpha_i$. we thus assume, following the proof
of Proposition 3.2, that $\alpha_i$ have the form 
$$
\alpha^{-1}_0=\begin{pmatrix}
\lambda \epsilon & A\\
-A^{t}& \mu \epsilon
\end{pmatrix}\alpha_1=\begin{pmatrix}
0 & B\\
-B^{t} & 0
\end{pmatrix}, \alpha_2=\begin{pmatrix}
0 & C\\
-C^{t} &0
\end{pmatrix}
$$
where $A\in M_2(K), B, C\in GL_2(K), \lambda, \mu \in K^{\ast}$, $\lambda^{2}=\mu^{2}b^{2}$ with 
\begin{equation*}\label{eqnt1}
(\det B)CB^{-1}-(\det C)BC^{-1}=\theta \mu^{-1}\eta_a\epsilon^{-1}\tag{1}
\end{equation*}
\begin{equation*}\label{eqnt2}
(\det C)C^{-1}B-(\det B)B^{-1}C=\theta \lambda^{-1}\epsilon^{-1}(-b \eta_a)\tag{2}
\end{equation*}
\begin{equation*}\label{eqnt3}
CA^{t}B=BA^{t}C.\tag{3}
\end{equation*}
Let\pageoriginale $\lambda =e\mu b$ where $e=\pm 1$. From (\ref{eqnt1}) and (\ref{eqnt2}), we get 
$$
\begin{aligned}
&\eta_a\epsilon^{-1}B=e B\epsilon^{-1}\eta_a\\
&{}\eta_a\epsilon^{-1}C=e C\epsilon^{-1}\eta_a
\end{aligned}
$$
These imply that $B=\begin{pmatrix}
x & y\\
-ey & -eax
\end{pmatrix},c=\begin{pmatrix}
x' & y'\\
-ey' & -eax'
\end{pmatrix}, 
\\x, y, x', y' \in K$. Let $A=\begin{pmatrix}
p & q\\
r & s
\end{pmatrix}$ Using (\ref{eqnt1}) and (\ref{eqnt3}), we get $q=-er$, $p=-eas$. Thus $A=\begin{pmatrix}
-eas & -er\\
r & s
\end{pmatrix}$
Let $W$ denote the $K$-subspace of $M_4(K)$ spanned by $\{1,\omega \alpha^{-1}_0, \alpha_1\alpha_0^{-1}, \alpha_2\alpha_0^{-1}\}$. Then $W\subset H(\alpha)$, since $w\alpha^{-1}_0=\alpha_2\alpha_0^{-1}\alpha_1\alpha_0^{-1}-\alpha_1\alpha_0^{-1}\alpha_2\alpha_0^{-1}\in H(\alpha)$. We first show that $W=K+X$ or $K+X'$, where $K\subset M_4(K)$ as scalar matrics and $X$ is the set of all matrices of the form $(\ast)$ and $X'$ is the set of all matrices of the form $(\ast\ast)$. 
$$
\begin{aligned}
\omega\alpha^{-1}_0&=\theta\begin{pmatrix}
\eta_a & 0\\
0 & -b\eta_a
\end{pmatrix}\begin{pmatrix}
e\mu b\epsilon & A\\
-A^{t} &\mu \epsilon
\end{pmatrix}\\
&=\theta
\begin{pmatrix}
0 & e\mu b &-eas  &-er\\
e\mu ab & 0  &-ar & -as\\
-eabs & br  &0 &-\mu b\\
eabr & -abs & -\mu ab & 0
\end{pmatrix}\\
\alpha_1\alpha_0^{-1}&=
\begin{pmatrix}
0 & B\\
-B^{t} & 0
\end{pmatrix} \begin{pmatrix}
e\mu b\epsilon & A\\
-A^{t} & \mu \epsilon
\end{pmatrix}
\end{aligned}
$$
$$
=(eaxs+eyr).1+
\begin{pmatrix}
0 & -(xr+ys)&-\mu y & \mu x\\
-a(xr+ys) & 0 &e\mu ax & -e\mu y\\
-\mu b y & -e \mu b x & 0 & e(xr+ys)\\
-\mu a b x & -e \mu b y & ea(xr+ys) & 0
\end{pmatrix}
$$
\pageoriginale
The matrix $\alpha_2 \alpha_0^{-1}$ has a form similar to that of $\alpha_1\alpha_0^{-1}$ above, with $x,y$ replaced by $x',y'$. These matrices belong to $K+X$. if $e=1$ and to $K+X'$ if $e=-1$. 

We claim that $1$, $\omega \alpha^{-1}_0\cdot \alpha_1\alpha^{-1}_0$, $\alpha_2\alpha_0^{-1}$ are linearly independent over $K$. In fact, if $x_O1+x_1\omega \alpha^{-1}_O+x_2\alpha_1\alpha^{-1}_0+x_3\alpha_2\alpha_0^{-1}=0$, with $x_i\in K$, then $x_0\alpha_0+x_1\omega +x_2\alpha_1+x_3\alpha_2=0$. Comparing traces, we get $x_1(1-a-b+ab)=0$. Since $\omega$ is anisotropic, we get $x_1=0$. From the forms of $\alpha_i$, $0\leq i \leq 2$, we get $x_0=0$ and $x_2 B=x_3 C$. If $x_2$ and $x_3$ are not both zero, equation (\ref{eqnt1}) is contradicted. Thus $x_i=0$ for $0\leq i \leq 3$ and $\dim W=4$. Since $K+X$ and $K+X'$ are both of dimension $4$ over $K$, it follows that $W=K+X$ or $W=K+X'$ according as $e=\pm 1$. It is easy to check that $K+X$ and $K+X'$ are both $K$-subalgebras of $M_4(K)$ and hence the inclusion $W\subset H(\alpha)\subset K+X$ or $W\subset H(\alpha)\subset K+X'$ gives that $H(\alpha)=K+X$ or $H(\alpha)=K+X'$. It is also easy to see that $K+X$ and $K+X'$ are quaternion algebras with the canonical involution given by $x\to \omega x^{t}\omega^{-1}$. The algebra $K+X$ is generated by the two elements 
$$
i=
\begin{pmatrix}
0 & 1 & 0 & 0\\
a & 0 & 0 & 0\\
0 & 0 & 0 & 1\\
0 & 0 & a & 0
\end{pmatrix},j=\begin{pmatrix}
0 & 0 &1 & 0\\
0 & 0 & 0 & -1\\
b & 0 & 0 & 0\\
0 & -b & 0 & 0
\end{pmatrix}
$$
with\pageoriginale the relations $i^{2}=a$, $j^{2}=b$, $ij+ji=0$. Hence its norm with respect to the basis $1, i, j, ij$ is given by $\langle 1, -a, -b, ab\rangle \approx \theta^{-1}_{\omega}$. 

One can similarly show that the norm in $K+X'$ is also isometric to $\theta^{-1}\omega$. The commutant of $K+X$ in $M_4(K)$ is checked to be $K+X'$. To complete the proof of the theorem, we need only to check (iv) which follows from the following 
\end{Proof}

\begin{lem}\label{lem3.5}
Let $\alpha$, $\beta$ be two skew symmetric matrices in $M_{2 n}(R)$, where $R$ is any commutative ring with $\alpha$ non-singular. Then, the matrix $\beta \alpha^{-1}$ satisfies the equation $Pf(x\alpha-\beta)=0$. 
\end{lem}

\begin{Proof}
Given a skew symmetric matrix $\gamma \in M_{2n}(R)$, if $\gamma^{\ast}\in M_{2n}(R)$ is defined by 
$$
\begin{aligned}
\gamma^{\ast}_{ij}&=(-1)^{i+j-1}Pf(\Gamma_{ij}); i\leq j\\
&{}=(-1)^{i+j}Pf(\Gamma_{ij}),i>j,
\end{aligned}
$$
where $\Gamma_{ij}$ is the skew symmetric matrix obtained from
$\gamma$ by suppressing the $i^{th}$ and $j^{th}$ rows and columns,
then
$\gamma \gamma^{\ast}=\gamma^{\ast}\gamma=Pf(\gamma)$. $Id$ (see\cite{key11},
(24)). Thus, if $\gamma$ is treated as an endomorphism of $R^{2n}$,
then coker $\gamma$ is annihilated by $Pf(\gamma)$. Given skew
symmetric matrices $\alpha, \beta \in M_{2n}(R)$ with $\alpha$
non-singular, then we have an exact sequence \cite[p. 630]{key3} of $R[x]$- modules 
$$
0\to R[x]\xrightarrow{2nx\alpha-\beta} R[x]^{2n}\xrightarrow{\varphi} R^{2n}\to 0, 
$$ 
where\pageoriginale $\varphi$ is defined by $\varphi\left(\sum a_i x^{i}\right)=\sum\left(\beta\alpha^{-1}\right)^{i}(a_i)$, $a_i\in R^{2n}$. Here $R^{2n}$ is treated as an $R[x]$ -module by letting $x$ operate as $\beta \alpha^{-1}$. By the above remark, $f(x)=Pf(x\alpha-\beta)$ annihilates $R^{2n}$, i.e. $f\left(\beta\alpha^{-1}\right)=0$. This proves the lemma. 
\end{Proof}

\section{Anisotropic quadratic spaces with \texorpdfstring{$c_2\leq 4$}{eq}}\label{s4}

We have already seen (Proposition~\ref{Prop2.3}) that the second chern class $c_2$ of an $s$-stable quadratic space over $\mathbb{A}^{2}_K$ is even. we denote by $q_0$ an anisotropic quadratic form over $K$. Since every quadratic space of rank $2$ over $\mathbb{A}^{2}_K$ is extended from $K$ \cite[Prop 1.1]{key9}, $\mathscr{K}(n,q_0)=\phi$ if rank $q_0=2$. In particular, for $c_2=2$, $\mathscr{K}(2,q_0)=\phi$. We next consider the case $c_2=4$. In view of Proposition~\ref{Prop3.3}, $\mathscr{K}(4,q_0)=\phi$, unless rank $q_0=4$ and disc $q_0=1$. In this case, $q_0$ is, upto a scalar, the norm from a  quaternion algebra $H_0$. Let $C_0$ be the conic in $\mathbb{P}^{2}_K$ giving the norm on any three dimensional subspace of $H_0$. We shall show that $\mathscr{K}(4,q_0)$ is in bijection with the orbit of $C_0$ under the action of $GL_3(K)$. 

Let $q_0=\theta <1,-a, -b, ab>$. Let $C_0$ be the conic in $\mathbb{P}^{2}_K$ defined by $a z^{2}_0+b z^{2}_1-z^{2}_2=0$. We define a map $c:\mathscr{K}(4, q_0)\xrightarrow{K}$ $\left\{ \text{set of conics in} \mathbb{P}^{2}_K\right\}$ as follows. Let $[\alpha]\in \mathscr{K}(4,q_0)$. Then, $Pf(Z_0\alpha_0+Z_1\alpha_1+Z_2\alpha_2)$ is a homogenous polynomial, which is not zero since $\alpha_i$ are non-singular, and is of degree $2$. We define $c([\alpha]):Pf(Z_0\alpha_0+Z_1 \alpha_1+Z_2 \alpha_2)=0$. The map is well-defined on $\mathscr{K}(4,q_0)$ since if $\beta_i=\lambda u\alpha_i u^{t}$, $Pf(Z_0 \beta_0+Z_1\beta_1+Z_2\beta_2)=\lambda^{2}\det u. Pf(Z_0\alpha_0+Z_1\alpha_1+Z_2\alpha_2)$ which again determines the same conic. 

\begin{THM}\pageoriginale
\textit{The map $c$ induces a bijection between $\mathscr{K}(4,q_0)$ and the orbit of the conic $C_0$ under the action of $GL_3(K)$. The image of $c$ is precisely $\dfrac{GL_3(K)}{(subgroup fixing C_0)}$}.
\end{THM}

\begin{Proof}
Suppose $[\alpha]$, $[\beta]\in \mathscr{K}(4,q_0)$ with $c(\alpha)=c(\beta)$. Then $Pf(Z_0\alpha_0+Z_1\alpha_1+Z_2\alpha_2)=\lambda. Pf(Z_0\beta_0+Z_1\beta_1+Z_2\beta_2)$, with $\lambda \in K^{\ast}$. If $Pf\alpha_0=a$, $Pf\beta_0=b$, then $a^{-1}Pf(Z_0\alpha_0+Z_1\alpha_1+Z_2\alpha_2)=b^{-1}Pf(Z_0\beta_0+Z_1\beta_1+Z_2\beta_2)$. The map $\lambda +\mu \alpha_1\alpha_0^{-1}+v\alpha_2\alpha_0^{-1}\to \lambda +\mu \beta_1\beta^{-1}_0+v\beta_2\beta_0^{-1}$ gives an isometry from a $3$-dimensional subspace of $H(\alpha)$ onto the corresponding subspace of $H(\beta)$ which maps the identity elements onto each other. This can be extended to an isomorphism of $H(\alpha)$ onto $H(\beta)$. 

Let $u\in GL_4(K)$ with $uH(\alpha)u^{-1}=H(\beta)$. Then 
\begin{equation*}\label{eq1}
\begin{aligned}
u\alpha_1\alpha^{-1}_0 u^{-1}&=\beta_1\beta^{-1}_0\\
u \alpha_2\alpha^{-1}_0u^{-1}&=\beta_2\beta_0^{-1}
\end{aligned}\tag{1}
\end{equation*}

To show that $u\alpha_i u^{t}=\lambda \beta_i$, $\lambda \in K^{\ast}$, $0\leq i \leq 2$, it suffices to show that $u\alpha_0 u^{t}=\lambda \beta_0$. From (\ref{eq1}), we get 
\begin{equation*}\label{eq2}
u q_0\alpha^{-1}_0u^{-1}=u\left(\alpha_2\alpha_0^{-1}\alpha_1-\alpha_1\alpha_0^{-1}\alpha_2\right)\alpha^{-1}_0 u^{-1}=\beta^{-1}_0\tag{2}
\end{equation*}
We have the following commutative diagram: 
$$
\xymatrix{H(\alpha)\ar[d]_{\Int u}\ar[r]^{\ast}&H(\alpha)\ar[d]^{\Int u}\\
H(\beta)\ar[r]^{\ast}& H(\beta),}
$$
(since\pageoriginale int $u$ is an isomorphism of $H(\alpha)$ onto $H(\beta)$, it commutes with the canonical involutions$^{\ast}$ of these algebras). By theorem~\ref{thm3.5}, the involutions are precisely given by $x\mapsto q_0 x^{t}q_0^{-1}$. Thus 
$$
u q_0 x^{t} q^{-1}_0 u^{-1}=q_0 u^{t-1} x^{t} u^{t} q^{-1}_0,\, \forall x \in H(\alpha),
$$
i.e.
$$
\left(u^{t} q^{-1}_0 u q_0\right) x^{t}=x^{t}\left(u^{t}q^{-1}_0 u q_0\right)\, \forall x \in H(\alpha), 
$$
i.e.
$$
x\left(q_0 u^{t}q^{-1}_0 u\right)=\left(q_0 u^{t} q^{-1}_0 u\right)x\, \forall x\in H(\alpha), 
$$
i.e.
$$
q_0u^{t}q^{-1}_0u\in H(\alpha)^{c} \text{ and } \left(q_0 u^{t} q^{-1}_0 u\right)^{\ast}=q_0 u^{t} q^{-1}_0 u.
$$

Thus the element $q_0 u^{t} q^{-1}_0 u\in H(\alpha)^{c}$ is invariant under the canonical involution of $H(\alpha)^{c}$ and hence is a constant, say $\lambda \in K^{\ast}$ so that 
$$
u q_0=\lambda q_0 u^{t-1}
$$
Substituting this in (\ref{eq2}), we get 
$$
\lambda q_0 u^{t^{-1}}\alpha^{-1}_0 u^{-1}=q_0 \beta^{-1}_0, \text{ i.e. } u\alpha_0 u^{t}=\lambda \beta_0.
$$

We next show that for $[\alpha], [\beta] \in \mathscr{K}(4, q_0), C(\alpha)=\lambda u^{t} C(\beta)u$ for some $u\in GL_3(K)$, $\lambda \in K^{\ast}$. Since the norms in $H(\alpha)$ and $H(\beta)$ are isometric to $\theta^{-1}q_0$, $H(\alpha)$ and $H(\beta)$ are isomorphic. Hence the reduced norms restricted to $K+K \alpha_1 \alpha^{-1}_0 + K\alpha_2\alpha_0^{-1}$ and $K+K \beta_1 \beta^{-1}_0+K \beta_2\beta^{-1}_0$ are isometric upto a scalar. Let $u \in GL_3(K)$ be the matrix of transformation of these spaces with respect to the bases $\left(1, \alpha_1 \alpha^{-1}_0, \alpha_2 \alpha^{-1}_0\right)$ and $\left(1,\beta_1 \beta^{-1}_0. \beta_2\beta_0^{-1}\right)$\pageoriginale respectively. Then $u Pf\left(Z_0 \alpha_0+Z_1\alpha_1+Z_2\alpha_2\right)u^{t}=\lambda Pf\left(Z_0\beta_0+Z_1\beta_1+Z_2\beta_2\right)$. 

Let now $c([\alpha])=C_1$. Then, the full orbit of $C_1$ under $GL_3(K)$ is contained in the image of $C$. In fact, if $C_2=u C_1 u^{t}$, $u\in GL_{3}(K)$, and $\beta_1=\sum\limits_{j=0}^{2}u_{ij}\alpha_j$, $0\leq i \leq 2$, where $u=(u_{ij})$, then, $c(\beta)=u C_1 u^{t}$. Thus we have shown that the image of $c$ is a full orbit of a conic provided it is non-empty. In fact, if $C_1$: $4a Z^{2}_0+bZ^{2}_1-Z^{2}_2=0$, then $C_1$ is equivalent to $C_0$ and $C_1$ is the image of $[\alpha]$ defined by 
$$
\alpha_0=
\begin{pmatrix}
0 & 2\theta^{-1} & 0 & 0\\
-2\theta^{-1} & 0 & 0 & 0\\
 0 & 0 & 0 & -2\theta^{-1}a^{-1}\\
0 & 0 & 2\theta^{-1}a^{-1}& 0
\end{pmatrix}
$$
$$
\alpha_1=
\begin{pmatrix}
0 & 0 & 1 & 0\\
0 & 0 & 0 & b\\
-1 &  0 & 0 & 0\\
0 & -b & 0 & 0
\end{pmatrix}\alpha_2=
\begin{pmatrix}
0 & 0 & 0 &1\\
0 & 0 & 1 & 0\\
0 & -1 & 0 & 0\\
-1 & 0 & 0 & 0
\end{pmatrix}
$$
\end{Proof}

\begin{thebibliography}{99}
\bibitem{key1}
W. Barth,\pageoriginale Moduli of vector bundles on the projective plane, \textit{Invent. Math}. 42, 63--91 (1977).

\bibitem{key2}
W. Barth and K. Hulek, Monads and moduli of vector bundles, \textit{Manuscripta Math}. 25, 323--347 (1978)

\bibitem{key3}
H. Bass, \textit{Algebraic K-Theory}, Benjamin 1968.


\bibitem{key4}
K. Hulek, On the classification of stable rank-r vector bundles over the projective plane, \textit{Vector Bundles and Differential Equations}, Proc. Nice 1979, Progress in Mathematics $7$, Brikhauser

\bibitem{key5}
M. A. Knus and M. Ojanguren, Modules and quadratic forms over polynomial algebras, \textit{Proc. Amer. Math. Soc}. 66, 223--226(1977)


\bibitem{key6}
M. A. Knus, R. Parimala and R. Sridharan, Non-free projective modules over $\mathbb{H}[X,Y]$ and stable bundles over $\mathbb{P}_2(\mathbb{C})$, \textit{Invent. Math.} 65, 13--27 (1981)

\bibitem{key7}
M. Ojanguren, Formes quadratiques sur less alg\'{e}bres de polyn\^{o}mes, \textit{C.R.Acad.Sc.} 287, (Serie A) 695--698 (1978)

\bibitem{key8}
M. Ojanguren, R. Parimala and R. Sridharan, Indecomposable quadratic bundles of rank $4n$ over the real affine plane, \textit{Invent. Math}. 71, 643--653 (1983)

\bibitem{key9}
R. Parimala, Failure of a quadratic analogue of Serre's conjecture, \textit{Amer. J. of Math}. 100, 913--924 (1978)

\bibitem{key10}
R. Parimala,\pageoriginale Indecomposable quadratic spaces over the affine plane, to appear in \textit{Advances in Mathematics}. 

\bibitem{key11}
E. Von Weber, \textit{Vorlesungen \"uber} das Pfaff'sche Problem, Leipzig, 1900. 
\end{thebibliography}

\vskip 1cm

\noindent
\begin{tabular}{cl}
** & Institut de Mathematiques\\
   & Facult\'e des Sciences\\
   & Universit\'e de Lausanne\\
   & 1015 Lausanne --- Dorigny\\
   & Switzerland\\[.6cm]
*  & School of Mathematics\\
   & Tata Institute of Fundamental Research\\
   & Homi Bhabha Road\\
   & Bombay 400 005\\
   & India
\end{tabular}

\newpage
~\phantom{a}
\thispagestyle{empty}
