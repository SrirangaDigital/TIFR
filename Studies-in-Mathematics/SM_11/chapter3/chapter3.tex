\title{Some Results on a Question of Quillen}\label{chap3}
\markright{Some Results on a Question of Quillen}

\author{By S.\@ M. Bhatwadekar}
\markboth{S. M. Bhatwadekar}{Some Results on a Question of Quillen}

\date{}
\maketitle

\setcounter{page}{79}

\setcounter{pageoriginal}{106}

\section{Introduction}\pageoriginale\label{chap3-sec1}

This paper intends to bring to your attention some recent developments
on an approach suggested by Quillen to the following well-known
conjecture of H.\@ Bass and D.\@ Quillen. 

{\em Bass-Quillen Conjecture : Let $S$ be a regular local ring. Then
every finitely generated projective $S[T]$- module is free.}
  
In view of a monic inversion criterion of G.\@ Horrocks
(see \cite{chap3-H}) it would suffice to show that the extension of
the projective module to $S(T)$ (= localization of $S[T]$ at the
multiplicatively closed set of monic polynomials) is free. In
particular, if all finitely generated projective $S(T)$-modules are
free then all finitely generated projective $S[T]$-modules are free.

Let $R=S[T^{-1}]_{(n,T^{-1})}$ where $\mathfrak{n}$ denotes the
maximal ideal of $S$. Then, since $S$ is local,
$S(T)=R_{T^{-1}}$. Therefore in \cite{chap3-Q}, Quillen posed the
following stronger question. 

{\em Question:-\pageoriginale Are all finitely generated projective
$R_{f}$-modules free,\break when $R$ is a regular local ring and $f$ is a
regular parameter of $R$?}

Since then, the Bass-Quillen Conjecture has been settled
in various 
cases. However in all these cases, the methods employed there throw no
light on the corresponding Quillen question. Therefore this question
still merits interest. This motivated us (I and R.\@ A.\@ Rao) to
investigate the Quillen question at least in those cases where
Bass-Quillen Conjecture was known. We find that in some cases the
Quillen question is more difficult than the corresponding Bass-Quillen
Conjecture and needs subtle arguments. In some other cases, which I
shall be describing in this paper, the Quillen question has provided
alternative and neater arguments to settle the corresponding
Bass-Quillen Conjecture.

When $S$ is a local ring of an affine algebra at a regular prime
ideal, the Bass-Quillen Conjecture has been proved by H.\@ Lindel
(\cite[Theorem]{chap3-L-1}). With Rao, I have settled the
corresponding Quillen question. My joint work with Rao had two,
perhaps surprising, outcomes. Firstly the special case of the Quillen
question - the $S(T)$ case-which implies the Bass-Quillen Conjecture,
is actually equivalent to it \cite[Theorem 2.4]{chap3-B-R}. The other
novelty observed was that the Quillen question in geometric situations
(like in Lindel's Theorem) was, up to an etale extension, like
$S(T)$-case! This was shown using a refined ``Lindel subring''
approximation of a geometric local ring.  

Even\pageoriginale prior to Quillen's question, Horrocks has proved
that when $\dim R=3$ and $R$ contains a coefficient field, then every
projective $R$-module of rank 2 is free (\cite[Theorem 2]{chap3-H}). In
this article, I generalize this result by showing that every
projective $R_{f}$-module of rank $d-1$ ($d=\dim R$) is free when $R$
contains a coefficient field (see Theorem~\ref{chap3-thm3.5}). Moreover when $\dim 
R=3$, I prove the result of Horrocks under the weaker assumption that
$R$ is unramified (see Theorem~\ref{chap3-thm3.7}). The existence of a coefficient
field enables me to approximate (in a broader sense) the ring by a
good geometric ring. Using the rank condition, I reduce the problem to
one over such an ``approxiamte'' ring.

During this colloquium, Professor M.\@ Ojanguren brought to my notice
a beautiful result of O.\@ Gabber in \cite{chap3-G}, where he answers
the Quillen question affirmatively when $\dim R=3$.

Finally, let me summarise known results regarding the Bass-Quillen
Conjecture and the Quillen question in the following table:
\begin{center}
\begin{tabular}{@{}|>{\raggedright}p{2.25cm}|>{\raggedright}p{2.2cm}|>{\raggedright}p{2.2cm}|p{2.2cm}<{\raggedright}|@{}}
\hline
{\bf References} & {\bf Bass-Quillen Conjecture
$p\in \mathbb{P}(R[T];\newline \dim R=d$} & {\bf Quillen question
$P\in \mathbb{P}(R_{f}),\dim R=d+1$} & {\bf References}\\
\hline
Horrocks \cite{chap3-H} & $d=2$, $R$ contains a coefficient field
 & $d=2$, $R$ contains a coefficient field & Horrocks \cite{chap3-H}\\
Murthy \cite{chap3-Mu} & $d=2$ & $d=2$, $R$ unramified & Bhatwadekar
 (this paper)\\
 & & $d=2$ & Gabber \cite{chap3-G}\\
\hline 
Lindel-Lutke-bohmert \cite{chap3-L-L} & \multicolumn{2}{p{5cm}|}{$R$: power
series ring over a field} & Mohan Kumar \cite{chap3-MK}\\
Mohan Kumar \cite{chap3-MK} & \multicolumn{2}{c|}{} &\\
\hline
Lindel \cite{chap3-L-1} & \multicolumn{2}{p{5cm}|}{$R$: local ring of an
affine algebra at a regular prime ideal} &
Bhatwadekar-Rao \cite{chap3-B-R}\\
\hline
Lindel \cite{chap3-L-2} & \multicolumn{2}{p{5cm}|}{$R$ contains a
coefficient field and rank $P=d$} & Bhatwadekar (this paper)\\
\hline
\end{tabular}
\end{center}

This\pageoriginale paper has been organized as
follows. In \S\ \ref{chap3-sec2}, we record some definitions and
results. In \S\ \ref{chap3-sec3}, we prove our main theorems
(Theorem \ref{chap3-thm3.5} and
Theorem \ref{chap3-thm3.7}). In \S\ \ref{chap3-sec4}, we prove a
result about projective modules over a Laurent polynomial extension of
$S(T)$ when $\dim S=2$. (Theorem \ref{chap3-thm4.2}).

My warmest thanks are due to Ravi Rao, whose intuition and enthusiasm
led me to consider Theorem \ref{chap3-thm3.5} of this paper. I also
thank Amit Roy for encouragement.

\section{Preliminaries}\label{chap3-sec2}

Throughout this paper, all rings will be commutative noetherian and
all modules are finitely generated. Further all {\em projective}
modules will be assumed to be of constant rank.

In this section, we collect some definitions and results for later
use. $A$ will denote a commutative noetherian ring.

\subsection{}\label{chap3-sec2.1}
Given an ideal $\mathfrak{a}$ of $A$, $ht(\mathfrak{a})$ will denote
the height of $\mathfrak{a}$ and $\mu(\mathfrak{a})$ will denote the
minimal number of generators of $\mathfrak{a}$.

\subsection{}\label{chap3-sec2.2}
Given a projective $A$-module $P$ and an element $p$ of $P$ we define
$O_{P}(p)=\{\beta(p)|\beta\in \Hom_{A}(P,A)\}$. We say that $p$ is
{\em uni-modular} if $O_{P}(p)=A$. 

\subsection{}\label{chap3-sec2.3}
Given a free $A$-module $A^{d}$ we denote an element of $A^{d}$ by a
row vector $[a_{1},\ldots,a_{d}]$. We identify $\Aut_{A}(A^{d})$ with
$GL_{d}(A)$; the group of $d\times d$ invertible matrices over $A$. 

\subsection{}\label{chap3-sec2.4}
projective\pageoriginale $A$-module $P$ is said to be $A$-{\em
cancellative} if $P\oplus A^{d}\simeq Q\oplus A^{d}$ implies $P\simeq
Q$. 

\subsection{}\label{chap3-sec2.5}
Given a set $\mathfrak{P}$ of prime ideals of $A$ and a function
$\delta:\mathfrak{P}\to \mathbb{N}\cup \{0\}$ we define a partial
order $\ll$ on $\mathfrak{P}$ by setting
$\mathfrak{p}\ll \mathfrak{a}$ if $\mathfrak{p}\subset \mathfrak{a}$
and $\delta(\mathfrak{p})>\delta (\mathfrak{a})$. 

The function $\delta$ is called a {\em generalised dimension function}
on $\mathfrak{P}$ if for any ideal $\mathfrak{a}$ of $A$,
$V(\mathfrak{a})\cap \mathfrak{P}$ has only a finite number of minimal
elements with respect to $\ll$.

\subsection{}\label{chap3-sec2.6}
Next we quote a theorem from \cite{chap3-P}:

{\em Eisenbud-Evans Theorem: Let $P$ be a projective $A$-module, let
$\mathfrak{P}$ be a subset of $\Spec (A)$ and $\delta$ a generalized
dimension function on $\mathfrak{P}$. Assume rank
$P\geq \delta+1$. Let $(p,a)\in P\oplus A$ be uni-modular at all
$\mathfrak{p}\in \mathfrak{P}$. Then there exists an element $q\in P$
such that $p+aq$ is unimodular at all $\mathfrak{p}\in \mathfrak{P}$.}

\subsection{}\label{chap3-sec2.7}
Let $A$ be a PID. We say that $A$ is a special PID if
$SL_{r}(A)=E_{r}(A)$ for all $r\geq 2$.

\subsection{}\label{chap3-sec2.8}
Let $A[Y]$ be a polynomial algebra in one variable over $A$. Then
$A(Y)$ will denote the ring obtained from $A[Y]$ by inverting all
monic polynomials in $Y$.

\subsection{}\label{chap3-sec2.9}
{\em Quillen-Suslin Theorem} (\cite{chap3-Q}, Theorem 3 and
and \cite{chap3-Su}, Theorem 1).

{\em Let\pageoriginale $P$ be a projective $A[Y]$-module. If
$A(Y)\bigotimes\limits_{A[Y]} P$ is free, then $P$ is free.} 

\section{Main Theorem}\label{chap3-sec3}

In this section, we prove the main theorem
(Theorem~\ref{chap3-thm3.5}). For the proof of this theorem we need
lemmas and propositions.

\begin{proposition}\label{chap3-prop3.1}
Let $A$ be a ring and $P$ be a projective $A$-module of rank $d$. Let
$T$ be a multiplicatively closed subset of $A$ and let $(p',a')$ be a
unimodular element of $P_{T}\oplus A_{T}$. Then there exists
$\sigma\in \Aut_{A_{T}}(P_{T}\oplus A_{T})$ such that
\begin{itemize}
\item[\rm(1)] $\sigma(p',a')=(p,a)\in P\oplus A$

\item[\rm(2)] $ht\ O_{P\oplus A}(p,a)\geq d+1$
\end{itemize}
\end{proposition}

\begin{proof}
Without loss of generality, we can assume that
$T=\{1,t,t^{2},\ldots\}$ for some $t\in T$.


Let $I=\{\sigma/\sigma \in \Aut_{A_{t}}(P_{t}\oplus A_{t})$;
$\sigma(p',a')\in P\oplus A\}$. It is obvious that $I\neq \phi$. For
$\sigma\in I$, if $\sigma(p',a')=(\widetilde{p},\widetilde{a})\in
P\oplus A$ then let $N(\sigma)$ denote $ht\ O_{P\oplus
A}(\widetilde{p},\widetilde{a})$. Then it is enough to prove that
there exists $\sigma\in I$ such that $N(\sigma)\geq d+1$. This is
proved by showing that for any $\sigma\in I$ with $N(\sigma)\leq d$,
there exists $\sigma_{1}\in I$ such that $N(\sigma_{1})>N(\sigma)$. 

Let $\sigma\in I$ be such that $N(\sigma)\leq d$. Let
$\sigma(p',a')=(p,a)\in P\oplus A$. Let\pageoriginale $\mathfrak{P}$
be a set of prime ideals of $A$ of height $\leq d-1$ and let
$\mathfrak{P}_{1}=\mathfrak{P}\cap D(a)$. Then there is a generalized
dimension function $\delta:\mathfrak{P}_{1}\to \mathbb{N}\cup \{0\}$
such that $\delta\leq d-1$. Moreover since $\mathfrak{P}_{1}$ is a
subset of $D(a)$, $(p,a)$ is unimodular at all prime ideals belonging
to $\mathfrak{P}_{1}$. Further rank $P=d$. So,
by \eqref{chap3-sec2.6}, there exists a $q\in P$ such that $p+aq$ is
unimodular at all prime ideals belonging to $\mathfrak{P}_{1}$. Since
$(p,a)$ can be mapped in to $(p+aq,a)$ by an $A$-automorphism $\tau$
of $P\oplus A$ we have $O_{P\oplus A}(p,a)=O_{P\oplus
A}(p+aq,a)$. Therefore $\tau_{t}\sigma\in I$ and
$N(\sigma)=N(\tau_{t},\sigma)$. Hence, if necessary, replacing
$\sigma$ by $\tau_{t}\sigma$ and $(p,a)$ by $(p+aq,a)$, we can assume
that $p$ is unimodular at all prime ideals belonging to
$\mathfrak{P}_{1}$. This, in particular, implies that if a prime ideal
$\mathfrak{p}$ of $A$ contains $O_{P}(p)$ but does not contain the
element a then $ht\ \mathfrak{p}\geq d$. Therefore, since $d\geq
N(\sigma)\geq ht\ O_{P}(p)$ we have $N(\sigma)=ht\ O_{P}(p)$. 

Let $J$ denote the set of minimal prime ideals of $O_{P}(p)$. Let
$J_{1}$ be a subset of $J$ consisting of those members $\mathfrak{p}$
of $J$ which contain the element a and let $J_{2}=J-J_{1}$. Since
$N(\sigma)=ht\ O_{P}(p)$, $J_{1}$ is not empty. Moreover since
$(p,a)(=\sigma(p',a'))$ is a unimodular element of $P_{t}\oplus
A_{t}$, we have $t^{k}\in O_{P\oplus A}(p,a)$ for some positive
integer $k$. Therefore $t\in \mathfrak{p}$ for every $\mathfrak{p}\in
J_{1}$. Since rank $\mathfrak{p}=d$, $\mathfrak{p}\in J$ implies that
$ht\ \mathfrak{p}\leq d$ and therefore for every $\mathfrak{p}\in
J_{2}$ we have $ht\ \mathfrak{p}=d$. Hence
$\bigcap\limits_{\mathfrak{p}\in
J_{2}}\mathfrak{p}\nsubset \bigcup\limits_{\mathfrak{p}\in
J_{1}}\mathfrak{p}$. Let $x$ be an element of
$\bigcap\limits_{\mathfrak{p}\in J_{2}}\mathfrak{p}$ such that
$x\not\in \bigcup\limits_{\mathfrak{p}\mid \epsilon
J}\mathfrak{p}$. Since $t\in \mathfrak{p}$ for every $\mathfrak{p}$ in
$J_{1}$ we have $(tx)^{r}\in O_{P}(p)$ for some positive integer $r$. 

Let $\beta:P\to A$ be an $A$-linear map such that
$\beta(p)=(tx)^{r}$. Let $\widetilde{\theta}$ be an $A$-automorphism
of $P\oplus A$ defined as:
$\widetilde{\theta}(q,b)=(q,b+\beta(q))$\pageoriginale 
and let $\theta_{1}$ be an $A_{t}$-automorphism of $P_{t}\oplus A_{t}$
defined as: $\theta_{1}(q',a')=(q',t'a')$. Let $\sigma_{1}$ denote
$\theta_{1}^{-1}\widetilde{\theta}_{t}\theta_{1}\sigma$, then
$\sigma_{1}(p',a')=\theta^{-1}_{1}\widetilde{\theta}_{t}\theta_{1}(p,a)=(p,a+x^{r})$. Therefore,
$\sigma_{1}\in I$ and 
$$
N(\sigma_{1})=ht\ O_{P\oplus A}(p,a+x^{r})>ht\ O_{P}(p)=N(\sigma).
$$

This completes the proof of Proposition \ref{chap3-prop3.1}.
\end{proof}

As a consequence of Proposition \ref{chap3-prop3.1}, we get the
following result.

\begin{corollary}\label{chap3-coro3.2}
Let $A$ be a ring of dimension $d$ and let $P$ be a projective
$A$-module of rank $d$. Let $T$ be a multiplicatively closed subset of
$A$. If $P$ is $A$-cancellative then $P_{T}$ is $A_{T}$-cancellative. 
\end{corollary}

\begin{proof}
As before, without loss of generality, we can assume that
$T=\{1,t,t^{2},\ldots\}$ for some $t\in T$.

Let $(p',a')\in P_{t}\oplus A_{t}$ be a unimodular element. We want to
show that there exists $\theta\in \Aut_{A_{t}}(P_{t}\oplus A_{t})$
such that $\theta(p',a')=(0,1)$. 

By Proposition \ref{chap3-prop3.1}, there exists
$\sigma\in \Aut_{A_{t}}(P_{t}\oplus A_{t})$ such that
\begin{itemize}
\item[(1)] $\sigma(p',a')=(p,a)\in P\oplus A$

\item[(2)] $ht\ O_{P\oplus A}(p,a)\geq d+1$.
\end{itemize}

Since\pageoriginale $\dim A=d$, this implies that $(p,a)$ is a
unimodular element of 
$P\oplus A$. Since $P$ is $A$-cancellative, there exists
$\tau\in \Aut_{A}(P\oplus A)$ such that $\tau(p,a)=(0,1)$. Let
$\theta=\tau_{t}\sigma$. Then $\theta\in \Aut_{A_{t}}(P_{t}\oplus
A_{t})$ and $\theta(p',a')=(0,1)$. 

As an application of Proposition \ref{chap3-prop3.1} we prove the
following result which is a variant of result of Lindel (\cite[2.8 Satz]{chap3-L-2}).
\end{proof}

\begin{proposition}\label{chap3-prop3.3}
Let $(R,\mathfrak{m})$ be a local ring of dimension $d$ and let
$(R',\mathfrak{m}')$ be a local subring of $(R,\mathfrak{m})$ such
that $\widehat{R}'=\widehat{R}$. Let $f$ be an element of
$\mathfrak{m}'$ and let $Q$ be a stably free $R_{f}$-module of rank
$d-1$. Then there exists a stably free $R'_{f}$-module $Q'$ such that
$Q\simeq R_{f}\otimes_{R'_{f}}Q'$. 
\end{proposition}

\begin{proof}
Since $Q$ is stably free of rank $d-1\geq \dim R_{f'}$ there exists an
$R_{f}$-isomorphism $\Psi:Q\oplus R_{f}\to
R^{d}_{f}(=(R^{d-1})_{f}\oplus R_{f})$. Let
$\Psi(0,1)=[a_{1},\ldots,a_{d}]$. Then, by
Proposition \ref{chap3-prop3.1}
$$ 
(\text{taking~ } P=R^{d-1}, \ 
p'=[a_{1},\ldots,a_{d-1}], \ a'=a_{d}),
$$ 
there exists $\sigma\in
GL_{d}(R_{f})$ such that
\begin{itemize}
\item[(1)] $[a_{1},\ldots,a_{d}]\sigma=[b_{1},\ldots,b_{d}]\in R^{d}$

\item[(2)] $ht\ O_{R^{d}}[b_{1},\ldots,b_{d}]\geq d$.
\end{itemize}
If $ht\ O_{R^{d}}[b_{1},\ldots,b_{d}]>d$, then $[b_{1},\ldots,b_{d}]$
is a unimodular element of $R^{d}$. Since $R$ is local, there exists
$\tau\in GL_{d}(R)$ such that
$[b_{1},\ldots,b_{d}]\tau=[0,\ldots,1]$. This shows that $Q\simeq
R^{d-1}_{f}$. Then taking $Q'={R'}^{d-1}_{f}$ we are through. So we
assume $ht\ O_{R^{d}}[b_{1},\ldots,b_{d}]=d$.

Let\pageoriginale $\mathfrak{a}$ be an ideal of $R$ generated by
$b_{1},\ldots,b_{d}$. Then since
$\mathfrak{a}=O_{R}d[b_{1},\ldots,b_{d}]$, we have
$ht\ \mathfrak{a}=d$. Therefore $\mathfrak{a}$ is
$\mathfrak{m}$-primary ideal of $R$ and hence
$\mathfrak{m}^{\ell}\subset \mathfrak{a}$ for some positive integer
$l$. Moreover $d=\mu(\mathfrak{a})=ht(\mathfrak{a})$. 

Since $\widehat{R}'=\widehat{R}$ we have
${\mathfrak{m}'}^{\ell}R=\mathfrak{m}^{\ell}$ and
$R'/{\mathfrak{m}'}^{\ell}{\displaystyle{\mathop{\hookrightarrow}^{\approx}}}
R/\mathfrak{m}^{\ell}$. So if $\mathfrak{b}=\mathfrak{a}\cap R'$ then
(1) $\mathfrak{b}R=\mathfrak{a}$ and (2)
$d=\mu(\mathfrak{b})=ht(\mathfrak{b})$. 

Let $c_{1},\ldots,c_{d}$ be elements of $\mathfrak{b}$ which generate
$\mathfrak{b}$. Then as elements of $R$, $c_{1},\ldots,c_{d}$ generate
$\mathfrak{a}$ also. Since $R$ is local and $\mu(\mathfrak{a})=d$,
there exists $\theta\in GL_{d}(R)$ such that
$[b_{1},\ldots,b_{d}]\theta=[c_{1},\ldots,c_{d}]$. 

Consider the following short exact sequence
\begin{gather*}
0\to R'_{f}\to {R'}^{d}_{f}\to Q'\to 0\\[3pt]
1\to [c_{1},\ldots,c_{d}]
\end{gather*}

Since $f^{\ell}\in \mathfrak{a}\cap R'=\mathfrak{b}$,
$[c_{1},\ldots,c_{d}]$ is a unimodular element of
${R'}^{d}_{f}$. Therefore $Q'$ is stably free
$R'_{f}$-module. Obviously $Q\simeq R_{F}\otimes_{R'_{f}}Q'$. 

Let $(R,\mathfrak{m})$ be a regular local ring of dimension $d$ and
let $f$ be a regular parameter of $R$. If $R$ contains a local ring
$(R'\mathfrak{m}')$ such that (1) $f\in R'$ (2)
$\widehat{R}'=\widehat{R}$, then Proposition \ref{chap3-prop3.3},
shows that to study projective $R_{f}$-modules of rank $d-1$ it is
enough to consider projective $R'_{f}$-modules of rank $d-1$. The
following technical\pageoriginale lemma gives a sufficient condition
for $R$ to contain a local ring $R'$ such that (1) $f\in R'$, (2)
$\widehat{R}'=\widehat{R}$ and (3) every projective $R'_{f}$-module of
rank $d-1$ is free. 
\end{proof}

\begin{lemma}\label{chap3-lem3.4}
Let $(R,\mathfrak{m})$ be a regular local ring of dimension $d$ and
let $f$ be a regular parameter of $R$. Suppose $R$ contains a local
ring $(S,\mathfrak{n})$ of dimension $d_{o}$ such that
\begin{itemize}
\item[\rm(i)] $S\hookrightarrow R$ is a faithfully flat extension

\item[\rm(ii)] $R/\mathfrak{n}|R$ is a regular local ring with
$\overline{f}$ (= image of $f$ in $R/\mathfrak{n}R$) as its one of
regular parameters. 

\item[\rm(iii)]
$S/\mathfrak{n}{\displaystyle{\mathop{\hookrightarrow}^{\approx}}}R/\mathfrak{m}$ 

\item[\rm(iv)] $d_{o}\leq d-2$.
\end{itemize}
Then $R$ contains a regular local ring $(R',\mathfrak{m})$ such that
\begin{itemize}
\item[\rm(1)] $f$ is a regular parameter of $R'$

\item[\rm(2)] $\widehat{R}'=\widehat{R}$

\item[\rm(3)] Every projective $R'_{f}$-module of rank $d-1$ is free.
\end{itemize}
\end{lemma}

\begin{proof}
Since $S\hookrightarrow R$ is a faithfully flat extension and $R$ is
regular, so also $S$. Let $d-d_{o}=k\geq 2$. Let
$f=f_{1},f_{2},\ldots,f_{k}$ be elements of $R$ such that
$\{\overline{f}_{1},\overline{f}_{2},\ldots,\overline{f}_{k}\}$ is a
regular system of parameters of $R/\mathfrak{n}$ $R$(note that $\dim
R/\mathfrak{n}R=k$). 

Let\pageoriginale $B=S[f_{1},\ldots,f_{k}]$ and
$\mathfrak{p}=\mathfrak{n}B+(f_{1},\ldots,f_{k})B$. Then it is easy to
see that $B$ is a {\em polynomial algebra} in $k$ variables over $S$
and $\mathfrak{p}$ is a maximal ideal of $B$. Let
$R'=B_{\mathfrak{p}}$,
$\mathfrak{m}'=\mathfrak{p}B_{\mathfrak{p}}$. Then obviously $R$
contains $R'$ and $\widehat{R}'=\widehat{R}$. Moreover, $f=f_{1}$ is a
regular parameter of $R'$.

Let $A=S[f]$, $\mathfrak{p}_{o}=\mathfrak{n}A+fA$ and
$\widetilde{R}=A_{\mathfrak{p}_{o}}$. Then $\widetilde{R}_{f}$ is a
regular ring of dimension $d_{o}$ and
$\widetilde{R}_{f}[f_{2},\ldots,f_{k}]$ is a {\em polynomial algebra}
in $k-1$ variables over $\widetilde{R}_{f}$. Moreover, $R'_{f}$ is a
localization of $\widetilde{R}_{f}[f_{2},\ldots,f_{k}]$. 

Since
$K_{0}(\widetilde{R}_{f}[f_{2},\ldots,f_{k}])=K_{0}(\widetilde{R}_{f})=\mathbb{Z}$
and $\dim \widetilde{R}_{f}=d_{o}<d-1$ (by \cite[Theorem 1.1]{chap3-Sw})
every projective $\widetilde{R}_{f}[f_{2},\ldots,f_{k}]$-module of
rank $d-1$ is free. Now we are through in view of
Corollary \ref{chap3-coro3.2}, if we note that
$K_{0}(R'_{f})=\mathbb{Z}$ and
$d-1=\dim \widetilde{R}_{f}[f_{2},\ldots,f_{k}]$. 
\end{proof}

Now we prove the main theorem which is a generalization of Theorem 2
of Horrocks (\cite{chap3-H}).

\begin{theorem}\label{chap3-thm3.5}
Let $(R,\mathfrak{m})$ be a regular local ring of dimension $d$ and
let $f$ be a regular parameter of $R$. If $R$ contains a field $L$
such that $L\hookrightarrow R/\mathfrak{m}$ is a finite separable
extension then every projective $R_{f}$-module of rank $\geq d-1$ is
free. 
\end{theorem}

\begin{proof}
Let $Q$ be a projective $R_{f}$-module of rank $\geq d-1$. Since
$K_{0}(R_{f})=K_{0}(R)=\mathbb{Z}$, $Q$ is stably free. If, rank
$Q>d-1=\dim R_{f}$ then by (\cite[Corollary 3.5, p. 184]{chap3-B}) $Q$ is
free. So we assume that\pageoriginale rank $Q=d-1$. Since stably free
modules of rank one are always free we assume that $d-1\geq 2$
i.e. $d\geq 3$.

Since $L\hookrightarrow R/\mathfrak{m}$ is a finite separable
extension, $R/\mathfrak{m}$ is a simple extension of $L$ say
$L[\alpha]$. Let $\varphi(Y)$ be a minimal polynomial of $\alpha$ over
$L$. Let $a\in R$ be a lift of $\alpha$. Then $\alpha$ is separable
over $L$ and $\varphi$ is its minimal polynomial implies
$\varphi(a)\in \mathfrak{m}$ but $\dfrac{\partial \varphi}{\partial
Y}(a)\not\in \mathfrak{m}$. 

If $\varphi(a)$ and $f$ are linearly dependent modulo
$\mathfrak{m}^{2}$ then taking $x\in \mathfrak{m}$ such that $f$ and
$x$ are linearly independent modulo $\mathfrak{m}^{2}$, we see that
$\varphi(a+x)(=\varphi(a)+\dfrac{\partial\psi}{\partial
Y}(a)x+hx^{2})$ and $f$ are linearly independent modulo
$\mathfrak{m}^{2}$. Therefore replacing a by $a+x$ (if necessary) we
can assume that $\varphi(a)$ and $f$ are linearly independent modulo
$\mathfrak{m}^{2}$. 

Let $B=L[a]$ and $\mathfrak{p}=\varphi(a)B$. Then since $\varphi(a)$
is a regular parameter of $R$, $B$ is a polynomial algebra in one
variable over $L$ and $\mathfrak{p}$ is a maximal ideal of $B$. Let
$S=B_{\mathfrak{p}}$,
$\mathfrak{n}=\mathfrak{p}B_{\mathfrak{p}}$. Then $(S,\mathfrak{n})$,
is a one dimensional regular local ring contained in $R$. Moreover
$S/\mathfrak{n}=L[\alpha]=R/\mathfrak{m}$. It is easy to see that
$(S,\mathfrak{n})$ also satisfies the conditions (i) and (ii) of
Lemma \ref{chap3-lem3.4}. 

Since $Q$ is stably free we apply Lemma \ref{chap3-lem3.4} and
Proposition \ref{chap3-prop3.3} to conclude that $Q$ is free.
\end{proof}

\begin{corollary}\label{chap3-coro3.6}
Let $(S,\mathfrak{n})$ be a regular local ring of dimension $d$. Let
$S[Y]$ be a polynomial algebra in one variable over $S$. If $S$
contains\pageoriginale a field $L$ such that $L\hookrightarrow
S/\mathfrak{n}$ is a finite seperable extension then every projective
$S(Y)[X_{1},\ldots,X_{r}]$-module of rank $d$ is free.
\end{corollary}

\begin{proof}
Let $R=S[Y^{-1}]_{(n,Y^{-1})}$ and $Y^{-1}=f$. Then
$R_{f}=S(Y)$. Therefore by Theorem \ref{chap3-thm3.5} every projective
$S(Y)$-module of rank $d$ is free. Hence it is enough to prove that
every projective $S(Y)[X_{1},\ldots,X_{r}]$-module of rank $d$ is
extended from $S(Y)$. This can be proved using the same arguments as
in the proof of 4.1 Satz of Lindel (\cite{chap3-L-2}). 
\end{proof}

Now we conclude this section with the following theorem.

\begin{theorem}\label{chap3-thm3.7}
Let $R$ be a regular local ring of dimension $3$ and let $f$ be a
regular parameter of $R$. Assume that $R$ is unramified. Then every
projective $R_{f}$-module is free.
\end{theorem}

\begin{proof}
Let $\{f=f_{1},f_{2},f_{3}\}$ be a regular system of parameters of
$R$. Let $R^{0}=R$ and let $R^{i}$ denote the $f_{i}$-adic completion
of $R^{i-1}$ for $1\leq i\leq 3$. Then $R^{i-1}\hookrightarrow R^{i}$,
$f_{i}$ is a non-zero-divisor of $R^{i}$ and
$R^{i-1}/(f_{i})\xrightarrow{\approx}R^{i}/(f_{i})$. Hence
by (\cite[\S\ 2]{chap3-R}) we get the following {\em cartesian square}
(see \cite[p.~359]{chap3-B})
\[
\xymatrix@R=1.2cm@C=3cm{
\mathbb{P}(R^{i-1})\ar[d]\ar[r] & \mathbb{P}(R^{i})\ar[d]\\
\mathbb{P}(R^{i-1}_{f_{i}})\ar[r] & \mathbb{P}(R^{i}_{f_{i}})
}
\]
where, for a ring $A$, $\mathbb{P}(A)$ denote the category of all
finitely generated projective $A$-modules.

Therefore,\pageoriginale as $R^{i-1}$ is local for $1\leq i\leq
3$. every projective $R^{i-1}_{f_{i}}$-module is free if every
projective $R^{i}_{f_{i}}$-module is free. Moreover by
(\cite[Proposition $2'$]{chap3-Mu}) every projective
$R^{i}_{f_{i}}$-module is free if every projective
$R^{i}_{f_{i+1}}$-module is free.

The above discussion shows that to prove the theorem it is enough to
prove that every projective $R^{3}_{f_{3}}$-module is free. But
$R^{3}$ is a complete, regular local, unramified ring of dimension
3. Thereofre $R^{3}=S[[X,Y]]$ where $S$ is a regular local ring of
dimension 1. Again by (\cite[Proposition $2'$]{chap3-Mu}), we can
assume that $f_{3}=Y$. Now we are through in view of
Lemma \ref{chap3-lem3.4} and Proposition \ref{chap3-prop3.3}.
\end{proof}

\section{Projective modules over
{\fontsize{12}{14}\selectfont\texorpdfstring{$R_{f}[X_{1},\ldots,X_{r},Z^{\pm 1}_{1}\ldots,Z^{\pm 1}_{k}]$}{Rf}}}\label{chap3-sec4}

Let $R$ be a regular local ring and let $f$ be a regular parameter of
$R$. We want to study projective modules over
$R_{f}[X_{1},\ldots,X_{r},Z^{\pm 1}_{1},\ldots,Z^{\pm 1}_{k}]$. 
When $R$ is a power series over a field or a geometric local ring over
an infinite field then all projective modules over
$R_{f}[X_{1},\ldots,X_{r},Z^{\pm 1}_{1},\ldots,Z^{\pm 1}_{k}]$ are
free (see \cite[Theorem 3.1 and Theorem 2.2]{chap3-B-R}). These
results and Corollary \ref{chap3-coro3.6} lead us to ask the following
question: 

\noindent 
{\em Question. Let $R$ be regular local ring of dimension $d$
satisfying the hypothesis of Theorem \ref{chap3-thm3.5}. Let $f$ be a
regular parameter of $R$. Are all projective
$R_{f}[X_{1},\ldots,X_{r},Z^{\pm 1}_{1},\ldots,Z^{\pm 1}_{k}]$-modules
of rank $d-1$ free?}

\begin{remark}\label{chap3-rem4.1}
Swan has shown that all projective 
$$
R_{f}[X_{1},\ldots,X_{r},Z^{\pm
1}_{1},\ldots,Z^{\pm 1}_{k}]\text{-modules}
$$
\pageoriginale of rank $>\dim
R_{f}$ are free (\cite[Theorem 1.1.]{chap3-Sw}).
\end{remark}

In this section, we show that when $d=3$ and $R_{f}$ is of special
type, then the above question has an affirmative answer. More
precisely. 

\begin{theorem}\label{chap3-thm4.2}
Let $S$ be a regular local ring of dimension $2$. Let $S[Y]$ be a
polynomial algebra in one variable over $S$. Let $P$ be a projective
$S(Y)[X_{1},\ldots,X_{r},Z^{\pm 1}_{1},\ldots,Z^{\pm
1}_{k}]$-module. Then $P$ is free.
\end{theorem}

For the proof of this theorem we need the following proposition.

\begin{proposition}\label{chap3-prop4.3}
Let $A$ be a UFD and let $\pi$ be an element of $A$ such that
$A/(\pi)$ is a special PID. Let $P$ be a projective
$A[X_{1},\ldots,X_{r},Z^{\pm 1}_{1},\ldots,Z^{\pm 1}_{k}]$-module such
that $P_{\pi}$ is free. Then $P$ is free.
\end{proposition}

\begin{proof}
We prove the result by induction on $r+k$, the case $r+k=0$ being a
result of Bass and Murthy (\cite[Proposition 9.6]{chap3-B-M}). Let $B$
denote the ring $A[X_{1},\ldots,X_{r},Z^{\pm 1}_{1},\ldots,Z^{\pm
1}_{k}]$. 

\medskip
\noindent
{\em Case $(1):r>0$}

Let $T$ denote the multiplicatively closed subset of $B$ consisting of
monic polynomials in $X_{1}$ with coefficients in
$A$. Then\pageoriginale 
$$
B_{T}=A(X_{1})[X_{2},\ldots,X_{r},Z^{\pm
1}_{1},\ldots,Z^{\pm 1}_{k}].
$$ 
Moreover by (\cite[Proposition
$1'$]{chap3-Mu}) $A(X_{1})/(\pi)=A/(\pi)(X_{1})$ is a special
$PID$. Obviously $P_{T_{\pi}}$ is free. Therefore by induction $P_{T}$
is free and hence by (\ref{chap3-sec2.9}) $P$ is free.

\medskip
\noindent
{\em Case $(2):r=0$.}

Let $T'$ denote the multiplicatively closed subset of $B$ consisting
of monic polynomials in $Z_{1}$ with coefficients in $A$. Then
$$
B_{T'}=A(Z_{1})[Z^{\pm 1}_{2},\ldots,Z^{\pm 1}_{k}]
$$ 
and as before we
conclude that $P_{T}$ is free. Therefore by (\cite[Lemma
1.3]{chap3-Sw}) there exists a projective $A[Z^{-1}_{1},Z^{\pm
1}_{2},\ldots,Z^{\pm 1}_{k}]$-module $Q$ such that $P\simeq B\otimes
Q$. Since $P_{\pi}\simeq B_{\pi}\otimes Q_{\pi}$ is free,
by (\ref{chap3-sec2.9}) $Q_{\pi}$ is free. Therefore by Case (1) $Q$ is
free and hence $P$ is free.
\end{proof}

\medskip
\noindent
{\bf Proof of Theorem \ref{chap3-thm4.2}.} Let $\pi$ be a regular
parameter of $S$. Then $S/(\pi)$ is a discrete valuation ring and
therefore a special $PID$. Hence by (\cite[Proposition $1'$]{chap3-M})
$S(Y)/(\pi)=S/(\pi)(Y)$ is a special PID.

Let $L$ denote the quotient field of $S[Y]$. Then
$S_{\pi}[Y]\hookrightarrow S(Y)_{\pi}\hookrightarrow L$. Moreover,
$S(Y)_{\pi}$ is a$UFD$ of dimension 2 (in fact $S(Y)$ is a $UFD$ of
dimension 2). Therefore by (\cite[Proposition 2.1]{chap3-B-R})
$P_{\pi}$ is free. Hence by virtue of Proposition \ref{chap3-prop4.3},
$P$ is free. 


\begin{thebibliography}{}\pageoriginale
\itemsep=0pt
\bibitem{chap3-B} Bass, H. - {\em Algebraic $K$-theory}, Benjamin, New
York, 1968.



\bibitem{chap3-B-M} Bass, H. and Murthy, M. P. - Grothendieck groups
and Picard groups of abelian groups rings, {\em Ann. of Math.,}
vol. 86 (1967), 16-73.

\bibitem{chap3-B-R} Bhatwadekar, S. M. Rao, R. A. - On a question of
Quillen, Trans. {\em Amer. Math. Soc.,} vol. 279 (1983), 801-810.

\bibitem{chap3-G} Gabber, O. - Groupe de Brauer, {\em S\'eminaire, Les
Plans-sur-Bex,} Lecture Notes in Mathematics 844, 129-209,
Springer-Verlag (1980).

\bibitem{chap3-H} Horrocks, G. - Projective modules over an extension
of a local ring, {\em Proc. London Math. Soc.,} vol. 14 (1964), 714-718.

\bibitem{chap3-L-1} Lindel, H. - On a question of Bass - Quillen and
Suslin concerning projective modules over polynomial rings, {\em
Invent. Math.,} vol. 65 (1981), 319-323.

\bibitem{chap3-L-2} Lindel, H. - Erweiterungskriterien fur stabilfrei
Moduln uber Polynomringen, {\em Math. Annalen,} vol. 250 (1980), 99-108.

\bibitem{chap3-L-L} Lindel, H. and L\"utkebohmert, W. - Projective
Moduln \"uber Polynomialen\pageoriginale Erweiterungen von
Potenzreihenalgebren, {\em Arch. Math.,} vol. 28 (1977), 51-54.

\bibitem{chap3-MK} Mohan Kumar, N. - On a question of Bass-Quillen,
{\em J. Indian Math. Soc.,} vol. 43 (1979), 13-18.

\bibitem{chap3-Mu} Murthy, M. P. - Projective $A[X]$-modules, {\em
J. London Math. Soc.,} vol. 41 (1966), 453-456.

\bibitem{chap3-P} Plumstead, B. R. - The conjectures of Eisenbud and
Evans, {\em Amer. J. of Math.,} vol. 105 (1983), 1417-1433.

\bibitem{chap3-Q} Quillen, D. - Projective modules over polynomial
rings, {\em Invent. Math.,} vol. 36 (1976), 167-171.

\bibitem{chap3-R} Roy, A. - Application of patching diagrams to some
questions about projective modules {\em J. Pure and Applied Algebra},
vol. 24 (1982), 313-319.

\bibitem{chap3-Su} Suslin, A. A. - Projective modules over a
polynomial ring are free, {\em Soviet Math. Doklady,} vol. 17 (1976),
1160-1164 (English translation).

\bibitem{chap3-Sw} Swan, R. G. - Projective modules over Laurent
polynomial rings, {\em Trans. Amer. Math. Soc.,} vol. 237 (1978), 111-120.
\end{thebibliography}

\vskip .3cm

\noindent
School of Mathematics\\
Tata Institute of Fundamental Research\\
Homi Bhabha Road\\
Bombay 400 005\\
India


















