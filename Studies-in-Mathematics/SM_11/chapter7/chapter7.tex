\title{Vector Bundles on the Punctured Spectrum of a Local Ring II}\label{chap7}
\markright{}

\author{By G. Horrocks}
\markboth{G. Horrocks}{Vector Bundles on the Punctured Spectrum of a Local Ring II}

\date{}
\maketitle

\setcounter{page}{155}

\setcounter{pageoriginal}{206}
GIVEN\pageoriginale A SCHEME and a divisor what are the obstructions
to extending a bundle supported by the divisor to a bundle on the
ambient scheme? Schwarzenberger's integrality
conditions \cite{chap7-key5} give congruences for the Chern classes
when the divisor is a hyperplane in $\mathbb{P}^{d+1}$, and
in \cite{chap7-key3} Atiyah and Rees obtain an independent mod 2
condition on the dimensions of the holomorphic cohomology spaces when
the divisor has odd dimension and the bundle is self-dual in a
suitable sense.

In the present article, the ambient variety $Y$ is the punctured
spectrum of a local ring and the divisor $X$ belongs to a regular
element. There are no additive obstructions to extending a bundle from
$X$ to $Y$. So consider self-dual bundles on $X$ and whether they have
self-dual extensions. When $X$ has odd Krull dimension, there are
again no additive obstructions but for even dimension, the length mod
2 of the middle cohomology group of the bundle is the unique additive
obstruction to extendibility. For the complex field this obstruction
is also a topological invariant and may be identified with an element
of $K\widetilde{S}$ or $K\widetilde{O}$ depending on the mod 4 residue
class of the dimension. For arbitrary residue fields of the local
ring, the obstruction is invariant for self-dual algebraic
equivalence. More generally, it is invariant for confluence of
bundles. 

Finally there are non-additive obstructions to extendibility for
bundles which need not be self-dual, for example, the $k$-th exterior
power of a bundle with rank $2k$ is self-dual and for some of these
the obstruction to self-dual extendibility is non-zero. 

\section{No additive obstructions}\label{chap7-sec1}\pageoriginale


Let $B$ be a regular local ring of dimension $d+2\geq 4$,
$\mathfrak{n}$ be its maximal ideal, $Y=\Spec B-\{\mathfrak{n}\}$, $x$
be an element of $\mathfrak{n}-\mathfrak{n}^{2}$, $A=B/xB$,
$\mathfrak{m}=\mathfrak{n}/xB$, and $X$ the divisor of $x$ in $Y$. So
$\dim X=d$.

Complexes are cochain complexes indexed by $\mathbb{Z}$ and their
differentials are indexed by the degree of their source.

Identify vector bundles on $Y$ with reflexive $B$-modules locally free
on $Y$, and call them $Y$-bundles. An $X$-bundle $E$ extends to a
$Y$-bundle $F$ if $E$ is isomorphic to the $A$-bidual
$(F/xF)^{**}$. There are no additive obstructions to extendibility
with values in an abelian group because of:

\begin{theorem}\label{chap7-thm1.1}
Let $E$ be an $X$-bundle. Then there exists an extendible $X$-bundle
$E'$ such that $E\oplus E'$ is extendible.
\end{theorem}

\begin{proof}
In the equicharacteristic case, we may assume $A$ complete (\cite[\S\
8]{chap7-key6}) and $B=A[[x]]$. Let $V$ be the complex with non-zero
components $V^{-1}=B$, $V^{\circ}=B$ and differential $\partial$
multiplication by $x$. Take the dual $P$ of an $A$-projective
resolution of $E^{*}$ as in \cite{chap7-key6} and form the complex
$(\foprod{B}{P}{A})\otimes V$. Then the module of cycles of degree
zero extends $E\oplus \Ker (P^{1}\to P^{2})$, and the theorem follows
by induction on the projective dimension of $E^{*}$. 

In the general case, take $P^{i}(i\geq 0)$ as before and in lower
degrees take it to be an $A$-projective resolution of $E$. Let $Q\to
P$ be a homomorphism of a $B$-projective complex onto $P$ inducing
isomorphisms of cohomology groups. Take the tensor product
$\otimes_{B}A$ in the derived category of $B$-complexes to obtain a
morphism $\foprod{Q}{A}{B}\to \foprod{P}{V}{B}$ inducing isomorphisms
of cohomology. So $E\oplus \Ker (P^{1}\to P^{2})$ extends after adding
a free summand and the theorem follows by induction on projective
dimension. 
\end{proof}

\section{Self-dual bundles}\label{chap7-sec2}\pageoriginale

Self-dual bundles are assumed to have a given pairing. This is
described by an isomorphism between the bundle and its dual. A
self-dual $X$-bundle $E$ is said to be extendible if it has a
$Y$-bundle extension $F$ and its pairing $E\cong E^{*}$ extends to a
pairing $F\cong F^{*}$. 

\begin{theorem}\label{chap7-thm2.1}
Suppose that $d$ is odd and let $E$ be a self-dual $X$-bundle. Then
there exists an extendible self-dual $X$-bundle $\widehat{E}$ such
that $E\oplus \widehat{E}$ extends as a self-dual bundle.
\end{theorem}

\begin{proof}
First construct a self-dual $A$-projective complex $J$ such that:
$$
H^{\circ}(J)\cong E; \ H^{i}(J)=0(i<0); \ J^{i}=0\text{~~~ if~~~ } |i|>d/2.
$$ 
The construction starts from the complex $P$ in the proof
of \ref{chap7-thm1.1}. Kill the cohomology of $P$ in dimensions less
than $d/2$ to obtain a complex $Q$. Lift the isomorphisms in
dimensions greater than $d/2$ to a homomorphism of $P$ into $Q$ and
add free modules to $P$ so that the homomorphism is surjective. Then
$J$ is its kernel.

Since $J$ is self-dual, identify $J^{-i}=(J^{i})^{\ast} (i<0)$ and
$(\partial^{-i-1}=\partial^{i})^{\ast} (i>0)$. The pairing determines an
isomorphism $q:(J^{\circ})^{\ast}\cong J^{\circ}$. The differential
$\partial^{-1}$ is $q(\partial^{\circ})^{*}$.

In the equicharacteristic case put $K=\foprod{J}{B}{A}$, and denote
the induced differential and pairing by $\partial$, $q$ also. Define a
self-dual complex $L$ with differential $d$ by
\begin{align*}
& L^{\circ}=K^{-1}\oplus K^{\circ}\oplus K^{1}, L^{i}=K^{i}\oplus
K^{i+1}(i>0), L^{-i}=(L^{i})^{\ast}(i>0),\\[3pt]
&
d^{\circ}(a,b,c)=(\partial^{\circ}b+xc,\partial^{1}c),d^{i}(a,b)=(\partial^{i}a+xb,\partial^{i+1}b)(i>0),\\[3pt]
& d^{-i-1}=(d^{i})^{\ast}(i>0), r(a,b,c)=(c,qb,a)\text{~~ and~~ }
d^{-1}=r(d^{\circ})^{\ast}. 
\end{align*}
Then $H^{\circ}(L)$ is a self-dual $Y$-bundle extending $E\oplus
E'\oplus (E')^{\ast}$ where $E=\Ker (J^{1}\to J^{2})$. 

In\pageoriginale the general case, modify this procedure as in the
proof of \ref{chap7-thm1.1}. First let $J^{+}$ be the complex obtained
from $J$ by replacing $J^{i}$ by zero if $i<0$. Then construct a
complex of free $B$-modules $Q^{+}$ and an epimorphism $Q^{+}\to
j^{+}$ inducing isomorphisms of cohomology in degrees greater than
zero. Let $s:(Q^{\circ})^{\ast}\cong Q^{\circ}$ left
$q:(J^{\circ})^{\ast}\cong J^{\circ}$. Let $\delta$ be the
differential of $Q^{+}$. Define a self-dual complex $L$ by
\begin{align*}
& L^{i}=Q^{i},
L^{-i}=(L^{i})^{\ast}(i>0), \partial^{i}=\delta^{i}\partial^{-i-1}=(\partial^{i})^{\ast}i>0,\\[3pt]
& L^{\circ}=Q^{\circ}\oplus Q^{\circ},\partial^{\circ}(a,b)=\delta^{\circ}a-\delta^{\circ}b,\partial^{-1}(a)=(s(\delta^{\circ})^{\ast}a,s(\delta^{\circ})^{\ast}a).
\end{align*}
Then $H^{\circ}(L)$ extends $E\oplus E'\oplus (E')^{\ast}\oplus$
free. So the proof follows from \ref{chap7-thm1.1}.

Now suppose that $d=2r$ is even. There is an additive obstruction to
extendibility for self-dual bundles. For suppose $F$ extends $E$. The
exact cohomology sequence 
$$
H^{r}(F)\to H^{r}(F)\to H^{r}(E)\to H^{r+1}(F)\to H^{r+1}(F)
$$
and Serre-Grothendieck duality show that $H^{r}(E)$ has even length as
an artinian $A$-module. For any self-dual $X$-bundle $E$, define the
obstruction $\mu(E)$ to be the length of $H^{r}(E)\text{mod~}2$.
\end{proof}

\begin{theorem}\label{chap7-thm2.2}
Suppose that $d$ is even and the residue field $k$ is algebraically
closed. Let $E$ be a self-dual bundle with $\mu(E)=0$. Then there
exists an extendible self-dual bundle $\widehat{E}$ such that
$E\oplus \widehat{E}$ extends as a self-dual bundle.
\end{theorem}

\begin{proof}
Since $k$ is algebraically closed, $H^{r}(E)$ has a submodule $M$
whose quotient is isomorphic to $M^{\ast}$ via the isomorphism induced
by the paining. Construct a self-dual $A$-projective complex $J$ such
that: $H^{\circ}(J)\cong E$; $H^{i}(J)=0\,(i<0)$; $H^{r}(J)\cong M$. The
construction is similar to the first stage of the proof
of \ref{chap7-thm2.1}. The change is to construct the complex $Q$ by
killing the cohomology of $P$ in dimensions less than $r$ and only the
part $M$ in\pageoriginale dimension $r$. The proof can now be
completed as for \ref{chap7-thm2.1}.

To generalize this result to arbitrary fields, it is necessary to use
the Witt group for bilinear forms.
\end{proof}


\section{Confluence and algebraic equivalence}\label{chap7-sec3}

Suppose that $F$ is a reflexive $B$-module locally free except on a
1-dimen\-sional closed subset of $\Spec B$ intersecting $X$ in the empty
set. Its ideal is an intersection of distinct 1-dimensional primes
$\mathfrak{p}_{1},\ldots,\mathfrak{p}_{s}$ of $B$ which do not contain
$x$. Localizing $B$, $F$ at $\mathfrak{p}_{i}$ gives a regular local
ring $A_{i}$ and a vector bundle $E_{i}$ on the punctured spectrum of
$A_{i}$. The bundle $E=(F/xF)^{\ast}$ is called a confluence of
$E_{1},\ldots,E_{s}$. The following result has been proved
in \cite{chap7-key4}: 

\begin{theorem}\label{chap7-thm3.1}
Suppose that $d$ is even and $F$ is self-dual; then
$$
\mu(E)=\sum\limits^{s}_{i=1}\mu(E_{i}).
$$
\end{theorem}

\begin{proof}
The biduality spectral sequence gives exact sequences
\begin{align*}
& 0\to \Ext^{d+2}_{B}(H^{r+1}(F),B)\to H^{r}(F)\to \Ext^{d+1}(H^{r}(F),B)\to 0\\[3pt]
& 0\to \Ext^{d+2}_{B}(H^{r}(F),B)\to H^{r+1}(F)\to \Ext^{d+1}(H^{r-1}(F),B)\to 0.
\end{align*}
In both cases the kernels have support $\mathfrak{n}$ and the
cokernels are $x$-torsion free. So the two kernels are dual to each
other. The result now follows from the exact cohomology sequence for
$F/xF$. 

There is a simple application to the extension of bundles from
$\mathbb{P}^{d}$ to $\mathbb{P}^{d+1}$ ($d$ even). Let $\mathscr{E}$
be a self-dual bundle on $\mathbb{P}^{d}$. It always has a self-dual
extension as a sheaf to $\mathbb{P}^{d+1}$, for, example by extending
the pull-back of $\mathscr{E}$ to the punctured cone on
$\mathbb{P}^{d}$ over the vertex. 

Let $\mathscr{F}$ be any self-dual extension to $\mathbb{P}^{d+1}$
which is locally free except for singularities at points
$a_{1},\ldots,a_{s}$ of $\mathbb{P}^{d+1}$. The sheaf $\mathscr{F}$\pageoriginale
determines reflextive sheaves $\mathscr{F}_{1},\ldots,\mathscr{F}_{s}$
at each of these points. Apply \ref{chap7-thm3.1} to the punctured
cone over $\mathscr{P}^{d+1}$ in the neighbourhood of the vertex. We
find that
$$
\sum^{s}_{i=1}\mu (\mathscr{F})=\mu(\mathscr{E})
$$
where $\mu(\quad)$ is defined by lifting $\mathscr{E}$ to the
punctured cone on $\mathbb{P}^{d}$.

A second application is to algebraic equivalence. Define two
$X$-bundles to be algebraically equivalent if they can be joined by a
sequence of confluences. Let $E_{1}$, $E_{2}$ be two $X$-bundles, not
necessarily self-dual, with even rank $2t$ and assume $d$ is even. The
$t$-th exterior power of a rank $2t$ bundle has a natural pairing and
$\gamma(E)=(\mu(\Lambda^{t}E))$ is an obstruction to extending
$E$. Applying \ref{chap7-thm3.1} shows that if $E_{1}$, $E_{2}$ are
algebraically equivalent then $\gamma(E_{1})=\gamma(E_{2})$. In
particular let $E$ be the $X$-bundle of rank $d$ which is the second
syzygy of an ideal of A generated by a system of parameters for
$A$. Then $\gamma(E)$ is the multiplicity of the ideal mod 2 and $E$
is not algebraically equivalent to a trivial bundle if the
multiplicity is odd.

When $E$ is a rank $2X$-bundle coming by pull-back from
$\mathbb{P}^{d}$ and $d\geq 4$, $M$. Cohen has shown by means of the
Riemann-Roch Theorem that $\mu(E)=0$.
\end{proof}

\section{Formalism for \texorpdfstring{$\mu$}{mu}}\label{chap7-sec4}

Assume $d=2r$ is even. Form the exterior algebra $\Lambda (d+1)A)$ and
let $\xi$ be an element of $(d+1)A$ whose coordinates generate
$m$. Put
$$
\Theta=\Ker ({}_{\wedge}\xi:\Lambda^{r+1}\to \Lambda^{r+2}).
$$
It is a self-dual $X$-bundle with a pairing induced by the exterior
algebra structure and its cohomology is given by
$$
H^{i}(\Theta)=0(1\leq i\leq d-1, i\neq r), H^{r}(\Theta)=k.
$$
So $\mu(\Theta)=1$. Now let $E$ be any self-dual $X$-bundle and put
$\rho(E)$\pageoriginale equal to the rank of $E$ modulo two, and put
$$
\widehat{\mu}(E)=\rho(E)+\mu(E)t\in \mathbb{Z}_{2}[t],t^{2}=0
$$

\begin{theorem}\label{chap7-thm4.1}
\begin{itemize}
\item[\rm(i)] $\rho(E)=\mu(E\otimes \Theta)$.

\item[\rm(ii)] $\mu(\Theta^{p})=0$, $p\geq 2$.

\item[\rm(iii)] $\widehat{\mu}$ is a homomorphism of the ring of
self-dual vector bundles onto $\mathbb{Z}_{2}[t]$.

\item[\rm(iv)] $\mu(\Lambda^{p}\Theta)=0$, $p\geq 2$.
\end{itemize}
\end{theorem}

\begin{proof}
First prove (ii) for $p=2$. Since $\Theta$ extends to a vector bundle
on $\Spec A$ when $\mathfrak{m}$ is blown up, it is sufficient to
consider the graded case. The dualizing line bundle for $\Theta^{2}$
regarded as a sheaf on $\mathbb{P}^{d}$ has even degree and the
dualizing line bundle for $\mathbb{P}^{d}$ has odd degree. Serre
duality implies that $\mu(\Theta^{2})$ vanishes. 

Now we prove (i). By \ref{chap7-thm2.2}, there is an extendible
self-dual $X$-bundle $E'$ such that $E\oplus\mu(E)\Theta\oplus E'$ is
self-dually extendible. So tensoring with $\Theta$ and using (ii) with
$p=2$ shows that it is sufficient to note that $\rho(\Theta)=0$ and to
prove (i) for an extendible self-dual $X$-bundle $E$. Let
$\mathfrak{q}$ be the ideal generated by a base of $\mathfrak{m}$
lifted to $B$. Construct a $B$-module $\Phi$ from the exterior algebra
on $(d+1)B$ using this lifted base in the same way as $\Theta$ was
constructed from $(d+1)A$ and a base for $\mathfrak{m}$. The module
$\Phi$ is reflexive, free outside the variety of $\mathfrak{q}$, and
extends $\Theta$. Choose a self dual $Y$-bundle $F$ extending $E'$ and
apply \ref{chap7-thm3.1} to $F\otimes \Phi$. Since $F$ is locally free
except at $\mathfrak{n}$, the localization $F_{\mathfrak{q}}$ is free
with the same rank as $E$. So
$$
\mu(E\otimes \Theta)=\mu(F_{\mathfrak{q}}\otimes \Phi_{\mathfrak{q}})=\rho(E)\mu(\Phi_{\mathfrak{q}})=\rho(E),
$$
because $\mu(\Phi_{\mathfrak{q}})=1$.

Since $\rho(\Theta)=0$, (ii) now follows for $p>2$.

To\pageoriginale prove (iii) it is sufficient to show that
$\widehat{\mu}=0$ defines an ideal. This is easily reduced to showing
that if $E$ is extendible and of even rank then
$\widehat{\mu}(E\otimes \Theta)=0$. But this follows (i).

Finally (iv)  follows by reduction to the graded case as in the proof
of (ii). Serre duality and consideration of the total weights of the
indecomposable representations contained in $H^{r}(\Lambda^{p}\Theta)$
show that the representations occur in dual pairs. So
$\mu(\Lambda^{p}\Theta)=0$. 
\end{proof}

\section{Topological invariance}\label{chap7-sec5}

Take the residue field of $A$ to be $\mathbb{C}$. The category of
$X$-bundles is determined up to a canonical equivalence by the
completion of $A$ and it is easily verified that an $X$-bundle $E$
determines up to bundle isomorphism a topological bundle on a
punctured neighbourhood of the origin of $\mathbb{C}^{d+1}$. So $E$
determines a topological bundle $|E|$ on $S^{2d+1}$. If $E$ extends
then $|E|$ is trivial. 

Suppose that $d=2r$. The $K$-group for self-dual bundles has been
defined in \cite[p. 636]{chap7-key1}. For $S^{4r+1}$ it is isomorphic
to $\mathbb{Z}_{2}$. So the topological invariance of $\mu$ is
equivalent to:

\begin{theorem}\label{chap7-thm5.1}
$\Theta$ maps to the non-trivial element of the $K$-group of self-dual
bundles on $S^{4r+1}$.
\end{theorem}

\begin{proof}
At the conference I gave a proof via \cite[Theorem 4.2]{chap7-key3}
 which depends on the Atiyah-Singer Index
Theorem. Briefly let $W$ be the non-K\"ahler manifold obtained by
factoring $\mathbb{C}^{d+1}-\{0\}$ by an infinite cyclic subgroup of
$\mathbb{C}^{*}$. It has a holomorphic fibration
$f:W\to \mathbb{P}^{d}$ with a topological factorization through
$S^{4r+1}$. Now $\Theta$ is the pull back to $X$ of $\theta$ the
$r$-th exterior power of the tangent bundle of $\mathbb{P}^{d}$. The
stable homotopy invariant $\beta$ of \cite{chap7-key3} (the
holomorphic semicharacteristic) is now easily computed for
$f\ast \theta$ and seen to be non-zero. So, $\Theta$ is non-trivial
 in\pageoriginale the sense of the stable homotopy theory of bundles
 with a pairing.

In the course of the conference, M.F. Atiyah gave me a direct proof of
this result which I outline here. First the $K$-group of self-dual
bundles on $S^{4r+1}$ can be identified with
$K\widetilde{O}(S^{8m+1})$ if $r=2m$ and with
$K\widetilde{S}(S^{8m+5})$ if $r=2m+1$. In each case $S^{4r+1}$ is a
homogeneous space for the appropriate group (Spin or
Symplectic). Consider the case $r=2m$. Then $S^{8m+1}=\Spin
(8m+2)/\Spin (8m+1)$ and the representations of $\Spin (8m+1)$
determine real vector bundles on $S^{8m+1}$. The generator of
$K\widetilde{C}$ corresponds to the basic spin representation of
dimension $2^{4m}$ [\cite{chap7-key2}, \cite[p.270]{chap7-key7}]. Now
$\mathbb{P}^{4m}=U(4m+1)/U(4m)\times U(1)$ and the exterior power
representation $\Lambda^{2m}$ of $U(4m)$ determines the bundle
$\theta$. To compare $\theta$ with the generator of $KO$ express
$S^{8m+1}$ as $SU(4m+1)/SU(4m)$. An easy character computation shows
that the basic spin representation and $\Lambda^{2m}$ give equivalent
representations of $SU(4m)$ modulo sums of pairs of dual
representations. The case $r=2m+1$ can in a similar way be reduced to
a character computation. 
\end{proof}

\begin{thebibliography}{}
\bibitem{chap7-key1} D.W. Anderson, The real $K$-theory of classifying
spaces, {\em Proc. N. A. S.} Vol. 51 (1964), 634-6.

\bibitem{chap7-key2} M.F. Atiyah, R Bott and A. Shapiro, Clifford
modules, {\em Topology}, Vol. 3, Suppl. 1, 1964, 3-38.

\bibitem{chap7-key3} M.F. Atiyah and E. Rees, Vector bundles on
projective 3-space, {\em Invent. Math.} 35, 131-53 (1976).

\bibitem{chap7-key4} M. Cohen,\pageoriginale Algebraic families of
vector bundles on projective space and the punctured spectrum of a
local ring, Ph.D. thesis, Newcastle upon Tyne, 1978.

\bibitem{chap7-key5} F. Hirzebruch, {\em Topological Mehtods in
Algebraic Geometry,} 3rd Ed., Springer-Verlag, Berlin 1966.

\bibitem{chap7-key6} G. Horrocks, Vector bundles on the punctured
spectrum of a local ring, {\em Proc. London Math. Soc.} 1964,
689-713. 

\bibitem{chap7-key7} H. Weyl, {\em The Classical Groups}, Princeton
Math. Series, Princeton 1946.
\end{thebibliography}

\vskip 1cm

\noindent
School of Mathematics,\\
The University,\\
Newcastle upon Tyne,\\
NE1 7RU, England.

\newpage

~\phantom{A}
\thispagestyle{empty}






