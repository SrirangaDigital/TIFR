\chapter{ON P-ADIC REPRESENTATIONS ASSOCIATED WITH $Z_{p}$-EXTENSIONS}\label{art-4}

\begin{center}
{\large By~ Kenkichi Iwasawa}
\end{center}

\bigskip

\setcounter{pageoriginal}{140}
\textsc{In the present}\pageoriginale paper, we shall discuss some results on the $p$-adic representations of Galois groups, associated with so-called cyclotomic $Z_{p}$-extensions of finite algebraic number fields.

\smallskip
{\bf 1.}~Let $p$ be a prime number which will be fixed throughout the following, and let $Z_{p}$ and $\mathbb{Q}_{p}$ denote the ring of $p$-adic integers and the field of $p$-adic numbers respectively. A Galois extension $K/k$ is called a $Z_{p}$-extension if its Galois group is isomorphic to the additive group of the compact ring $Z_{p}$\footnote[1]{For various definitions and results on $Z_{p}$-extensions referred to throughout the following, see Iwasawa \cite{art4-key5} or Lang \cite{art4-key6}.}. Let $\Omega$ denote the field of all algebraic numbers, i.e., the algebraic closure of the rational field $\mathbb{Q}$ in the field $\mathbb{C}$ of all complex numbers, and let $W_{\infty}$ be the group of all $p^{n}$-th roots of unity in $\Omega$ for all $n\geq 0$. Then the field $\mathbb{Q}(W_{\infty})$ contains a unique subfield $\mathbb{Q}_{\infty}$ which is a $Z_{p}$-extension over $\mathbb{Q}$. In fact, $\mathbb{Q}_{\infty}$ is the unique $Z_{p}$-extension over $\mathbb{Q}$ contained in $\Omega$, and the degree of the extension $\mathbb{Q}(W_{\infty})/\mathbb{Q}_{\infty}$ is either $p-1$ or $2$ according as $p>2$ or $p=2$. For any finite extension $k$ of $\mathbb{Q}$, the composite $k_{\infty}=k\mathbb{Q}_{\infty}$ is then a $Z_{p}$-extension over $k$ and it is called the cyclotomic $Z_{p}$-extension over $k$. For each integer $n\geq 0$, there then exists a unique intermediate field $k_{n}$ with $[k_{n}K]=p^{n}$, and
$$
k=k_{0}\subset k_{1}\subset\ldots\subset k_{n}\subset\ldots\subset k_{\infty}=\bigcup\limits_{n\geq 0}k_{n}.
$$
Let $C_{n}$ denote the Sylow $p$-subgroup of the ideal class group of $k_{n}$. For $n\leq m$, $k_{n}\subseteq k_{m}$, there exists a natural homomorphism $C_{n}\to C_{m}$, and these homomorphisms define the direct limit
$$
C_{\infty}=\varinjlim C_{n}.
$$ 
Clearly $C_{\infty}$ is a $p$-primary abelian group and its Tate module $T(C_{\infty})$ is a $Z_{p}$-module. It is known that 
$$
T(C_{\infty})\simeq Z^{\lambda}_{p}
$$
where\pageoriginale $\lambda=\lambda_{p}(k)$ is a non-negative integer, called the $\lambda$-invariant of $k$ for the prime number $p$. Hence
$$
\foprod{V=T(C_{\infty})}{\mathbb{Q}_{p}}{Z_{p}}
$$
is a $\lambda$-dimensional vector space over $\mathbb{Q}_{p}$. Let
$$
\Gamma=\Gal(k_{\infty}/k)=\varprojlim \Gal(k_{n}/k)
$$
so that $\Gamma\simeq Z_{p}$. Clearly $\Gal(k_{n}/k)$ acts on $C_{n}$ for each $n\geq 0$ and hence $\Gamma$ acts on $C_{\infty}=\varinjlim C_{n}$ in the natural manner. Therefore $\Gamma$ acts also on $T(C_{\infty})$ and $V$. Thus we have a natural continuous finite dimensional $p$-adic representation of the Galois group $\Gamma=\Gal(k_{\infty}/k)$ on the $\lambda$-dimensional vector space $V$ over $\mathbb{Q}_{p}$. We shall next investigate the properties of the $p$-adic representation space $V$ for $\Gamma$.

\smallskip
{\bf 2.}~Let us first consider the special case where $p>2$ and where $k=\mathbb{Q}(\sqrt[p]{1})=$ the cyclotomic field of $p$-th roots of unity.

Let
$$
K=k_{\infty}=k\mathbb{Q}_{\infty}=\mathbb{Q}(W_{\infty}).
$$
In this case, $K/\mathbb{Q}$ is an abelian extension and
$$
G=\Gal(K/\mathbb{Q})=\Gamma\times \Delta
$$
where $\Gamma=\Gal(K/k)\simeq \mathbb{Z}_{p}$  and where $\Delta=\Gal(K/\mathbb{Q}_{\infty})=\Gal(k/\mathbb{Q})$ is a cyclic group of order $p-1$. Let $\widehat{\Delta}$ denote the character group of $\Delta$; we may identify $\widehat{\Delta}$ with $\Hom(\Delta,\mathbb{Z}^{\times}_{p})$ where $\mathbb{Z}^{\times}_{p}$ denotes the multiplicative group of all $p$-adic units in $\mathbb{Q}_{p}$. It is well known that $\widehat{\Delta}$ may be identified also with the group of all Dirichlet characters to the modulus $p$ and that it is generated by a special character $\omega$ called the Teichmuller character for $p$. A character $\chi$ in $\widehat{\Delta}$ is called even or odd according as $\chi(-1)=1$ or $\chi(-1)=-1$ respectively.

As one sees immediately, in this special case, not only $\Gamma=\Gal(K/k)$ but $G=\Gal(K/\mathbb{Q})$ also acts on $C_{\infty}$, $T(C_{\infty})$, and $V=\foprod{T(C_{\infty})}{\mathbb{Q}_{p}}{Z_{p}}$ naturally. Hence $V$ is again a $p$-adic representation space for $G$. For each $\chi$ in $\widetilde{\Delta}$, let
$$
V_{\chi}=\{v|v\in V,\delta\cdot v=\chi(\delta)v\text{~~ for all~~} \delta \text{~~ in ~~}\Delta\}.
$$
Since\pageoriginale $G=\Gamma\times \Delta$, $V_{\chi}$ is then a $\Gamma$-subspace of $V$ and
$$
V={\displaystyle{\mathop{\otimes}\limits_{\chi}}}V_{\chi},\quad \chi \in \widehat{\Delta}.
$$
Let $\gamma_{o}$ denote the element of $\Gamma$ such that $\gamma_{0}(\zeta)=\zeta^{1+p}$ for all $\zeta$ in $W_{\infty}$. $\gamma_{o}$ is a topological generator of $\Gamma$; namely, the cyclic subgroup generated by $\gamma_{0}$ is dense in $\Gamma$. For each $\chi$ in $\widehat{\Delta}$, let
$$
g_{\chi}(X)=\text{~the characteristic polynomial of } \gamma_{0}-1\text{~ acting on~ } V_{\chi}
$$
and let
\begin{align*}
g(X) &= \text{the characteristic polynomial of $\gamma_{0}-1$ acting on $V$}\\
&= \prod\limits_{\chi}g_{\chi}(X).
\end{align*}
On the other hand, let $L_{p}(s;\chi)$ denote the $p$-adic $L$-function for the Dirichlet character $\chi$ in $\widehat{\Delta}$. It is known in the theory of $p$-adic $L$-functions\footnote[2]{See Iwasawa \cite{art4-key4} or Lang \cite{art4-key6}.} that for each such $\chi$, there exists a power series $\xi_{\chi}(T)$ in the ring $Z_{p}[[T]]$ of all formal power series in an indeterminate $T$ with coefficients in $Z_{p}$ such that
$$
L_{p}(s;\chi)
\begin{array}{ll}
=\xi_{\omega_{\chi^{-1}}}((1+p)^{s}-1) & ,\text{~ for } \chi\neq 1, s\in Z_{p},\\
=\xi_{\omega}((1+p)^{s}-1)/((1+p)^{1-s}-1), & ,\text{~ for } \chi=1, s\in Z_{p},s\neq 1.
\end{array}
$$
Since $L_{p}(s;\chi)\nequiv 0$ if $\chi$ is even but $L_{p}(s;\chi)\equiv 0$ if $\chi$ is odd, $\xi_{\chi}(T)\equiv 0$ if $\chi$ is even and $\xi_{\chi}(T)\nequiv 0$ if $\chi$ is odd. By Weierstrass' preparation theorem, $\xi_{\chi}(T)$ for add $\chi$ can be uniquely written in the form
$$
\xi_{\chi}(T)=\eta_{\chi}(T)p^{e_{\chi}}f_{\chi}(T)
$$
where $\eta_{\chi}(T)$ is an invertible power series in the ring $Z_{p}[[T]]$, $e_{\chi}$ is a non-negative integer\footnote[3]{A recent theorem of B. Ferrero and L. Washington implies that $e_{x}=0$ for all odd $\chi$.}, and $f_{\chi}(T)$ is a so-called distinguished polynomial in $Z_{p}[T]$. The next theorem tells us that there exists a relation between the $p$-adic representation of $\Gamma=\Gal(K/k)$ on $V$ and the $p$-adic $L$-functions $L_{p}(s;\chi)$ for the characters $\chi$ in $\widehat{\Delta}$, or, more precisely, between the polynomials $g_{\chi}(X)$ and $f_{\chi}(T)$ defined above. Namely, we have the following result\footnote[4]{See Iwasawa \cite{art4-key3}.}: 

\medskip
\noindent
{\bf Theorem \thnum{1}.\label{art4-thm1}}~{\em Let\pageoriginale $k^{+}$ denote the maximal real subfield of the cyclotomic field $k=\mathbb{Q}(\sqrt[p]{\sqrt{1}})$ and let $h^{+}$ be the class number of $k^{+}$. Assume that $h^{+}$ is not divisible by $p$. Then}
\begin{align*}
g_{\chi}(X) &= 1, V_{\chi}=0,\quad\!\text{for all even~ } \chi\text{~ in~ }\widehat{\Delta},\\[3pt]
g_{\chi}(X) &= f_{\chi}(X),\qquad \text{~~~for all odd~ } \chi \text{~ in~ }\widehat{\Delta}.
\end{align*}

The assumption $p\nmid h^{+}$ in the theorem is known as Vandiver's conjecture. It has been verified by numerical computation for all primes $p<125,000$, and no counter example is yet found. On the other hand, if we define, following Leopoldt, the $p$-adic zeta function $\zeta_{p}(s;k^{+})$ of the totally real field $k^{+}$ by
$$
\zeta_{p}(s;k^{+})=\prod\limits_{\chi}{}^{+} \ \ L_{p}(s;\chi),\quad \chi\in \widehat{\Delta}, \chi(-1)=1,
$$
then the theorem implies that under the assumption $p\nmid h^{+}$, $\zeta_{p}(s;k^{+})$ is essentially equal to the characteristic polynomial $g(X)$ of $\gamma_{0}-1$ acting on the representation space $V$ over $\mathbb{Q}_{p}$, up to the change of variables $s\to (1+p)^{s}$. The result is mysteriously analogous to a well known theorem of A. Weil which states that a similar relation exists between the zeta function of an algebraic curve defined over a finite field and the characteristic polynomial of the Frobenius endomorphism acting on the $p$-adic representation space defined by the Jacobian variety of that curve.

Now, although the above theorem is proved only for a very special case (and even that under the assumption $p\nmid h^{+}$), we feel that it is not just an isolated fact for $k=\mathbb{Q}(\sqrt[p]{1})$, but is rather a part of a much more general result on teh cyclotomic $Z_{p}$-extensions over finite algebraic number fields. In fact, Greenberg \cite{art4-key2} generalizes Theorem \ref{art4-thm1} to the case where the ground field $k$ is a certain type of finite abelian extension over the rational field $\mathbb{Q}$, and Coates \cite{art4-key1} also discusses such a generalization for an abelian extension $k$ of an arbitrary totally real field. In the following, we shall report some results on cyclotomic $Z_{p}$-extensions, related to some further generalization of the above Theorem \ref{art4-thm1}.

\medskip
{\bf 3.}~We now assume that $p$ is an odd prime, $p>2$\footnote[5]{The case $p=2$ can be treated similarly but with some modifications.}, and consider as our ground\pageoriginale field a finite algebraic number field $k$ with the following properties:
\begin{itemize}
\item[(i)] $k$ is a Galois extension of the rational field,

\item[(ii)] $k$ contains primitive $p$-th roots of unity so that it is a totally imaginary field,

\item[(iii)] $k$ also contains a totally real subfield $k^{+}$ with $[k:k^{+}]=2$; namely, $k$ is a number field of C-M type.
\end{itemize}

In general, let $J$ denote the automorphism of the complex field $\mathbb{C}$ which maps each complex number $\alpha$ to its complex conjugate $\overline{\alpha}$. For simplicity, the restriction of $J$ on any subfield of $\mathbb{C}$, invariant under $J$, will be denoted again by $J$. Let
$$
\Delta =\Gal(k/\mathbb{Q})
$$
for the field $k$ mentioned above. Then by (ii) and (iii), $J$ is an element in the center of $\Delta$ and $J\neq 1$, $J^{2}=1$. As in \S1, let $K=k_{\infty}=k\mathbb{Q}_{\infty}$ denote the cyclotomic $Z_{p}$-extension over $k$. Since $k$ contains $p$-th roots of unity, $K=k(W_{\infty})$. Similarly, let $K^{+}=k^{+}_{\infty}=k^{+}\mathbb{Q}_{\infty}$ be the cyclotomic $Z_{p}$-extension over $k^{+}$. Then $K^{+}$ is a totally real subfield of the totally imaginary field $K$ with $[K;K^{+}]=2$. Clearly $K/\mathbb{Q}$ is a Galois extension because both $k/\mathbb{Q}$ and $\mathbb{Q}_{\infty}/\mathbb{Q}$ are Galois extensions. Hence, let
$$
G=\Gal(K/\mathbb{Q}),\quad \Gamma=\Gal(K/k)\simeq Z_{p}.
$$
Then we see immediately that $\Gamma$ is a central subgroup of $G$ and 
$$
\Delta=\Gal(k/\mathbb{Q})=G/\Gamma.
$$
As in the special case of \S2, the Galois group $G$ acts on $C_{\infty}$, $T(C_{\infty})$, and $V=\foprod{T(C_{\infty})}{\mathbb{Q}_{p}}{Z_{p}}$ so that $V$ provides us with a finite dimensional $p$-adic representation space for $G=\Gal(K/\mathbb{Q})$.

\medskip
\noindent
{\bf Theorem \thnum{2}.\label{art4-thm2}}~{\em Assume that $\lambda_{p}(k^{+})=0$ and that the so-called Leopoldt's conjecture holds for all intermediate fields $k^{+}_{n}$, $n\geq 0$, of the extension $K^{+}/k^{+}$. Then $V=\foprod{T(C_{\infty})}{\mathbb{Q}_{p}}{Z_{p}}$ is cyclic over $G=\Gal(K/\mathbb{Q})$; namely, there exists a vector $v_{o}$ in $V$ such that the whole space $V$ is spanned over $\mathbb{Q}_{p}$ by the vectors $\sigma\cdot v_{0}$, $\sigma\in G$.}

\smallskip

Recall that $\lambda_{p}(k^{+})$ denotes the $\lambda$-invariant of the totally real field $k^{+}$ for the prime $p$ and that Leopoldt's conjecture for $k^{+}_{n}$ states that any set of\pageoriginale units in $k^{+}_{n}$, multiplicatively linearly independent over the ring of rational integers $Z$, remains multiplicatively linearly independent over $Z_{p}$ when these units are imbedded in the multiplicative group of the algebra $\foprod{k^{+}}{\mathbb{Q}_{p}}{\mathbb{Q}}$. We note that both these assumptions are conjectured to be true for any totally real number field $k^{+}$. Note also that since $T(C_{\infty})\simeq Z^{\lambda}_{p}$, the conclusion of the theorem is equivalent to say that there exists an element $v_{0}$ in $T(C_{\infty})$ such that the elements of the form $\sigma\cdot v_{0}$, $\sigma\in G$, generate over $Z_{p}$ a submodule of finite index in $T(C_{\infty})$. The proof of the theorem will be briefly indicated in the next section.

In general, let $G$ be any profinite group and let $G=\varprojlim G_{i}$ with a family of finite groups $\{G_{i}\}$. The homomorphisms $G_{j}\to G_{i}$, $i\leq j$, which define the inverse limit, induce the homomorphisms $Z_{p}[G_{j}]\to Z_{p}[G_{i}]$ of the group rings of finite groups over $Z_{p}$, and they in turn define
$$
Z_{p}[[G]]=\varprojlim Z_{p}[G_{i}].
$$
$Z_{p}[[G]]$ is a compact topological algebra over $Z_{p}$ and it depends only upon $G$ and is independent of the family $\{G_{i}\}$ such that $G=\varprojlim G_{i}$. 

We apply the above general remark for $G=\Gal(K/\mathbb{Q})$ in Theorem \ref{art4-thm2} and define
$$
R=Z_{p}[[G]],\quad R'=\foprod{R}{\mathbb{Q}_{p}}{Z_{p}}.
$$
Let $G_{n}=\Gal(k_{n}/\mathbb{Q})$, $R_{n}=Z_{p}[G_{n}]$, $n\geq 0$. Since $G=\varinjlim G_{n}$, we then have
$$
R=\varprojlim R_{n}.
$$
Since $C_{n}$ is an $R_{n}$-module in the obvious manner, $C_{\infty}=\varprojlim C_{n}$ is an $R$-module. Hence $T(C_{\infty})$ also is an $R$-module and $V=\foprod{T(C_{\infty})}{\mathbb{Q}_{p}}{Z_{p}}$ is an $R'$-module. We next define a subset $A_{n}$ of $R_{n}$ by
$$
A_{n}=\{\alpha |\alpha \in (1-J)R_{n}, \ \alpha\cdot C_{n}=0\}.
$$
Note that $J=J|k_{n}$ is contained in the center of $G_{n}$ so that $A_{n}$ is a two-sided ideal of $R_{n}$, contained in $(1-J)R_{n}$. Furthermore, if $n$ is large enough and $m\geq n$, then the homomorphism $R_{m}\to R_{n}$ maps $A_{m}$ into $A_{n}$. Therefore
$$
A=\varprojlim A_{n}
$$
is\pageoriginale defined and it is a two-sided ideal of $R$, contained in $(1-J)R$. Let
$$
A'=\foprod{A}{\mathbb{Q}_{p}}{Z_{p}}.
$$
Clearly $A'$ is a two-sided ideal of $R'=\foprod{R}{\mathbb{Q}_{p}}{Z_{p}}$, contained in $(1-J)R'$. Moreover, it can also be proved that
$$
A'=\{\alpha'|\alpha'\in (1-J)R', \ \alpha'\cdot V=0\},
$$
namely, that $A'$ is the annihilator of the $R'$-module $V$ in $(1-J)R'$. Let
$$
d=[k:\mathbb{Q}].
$$
Using Theorem \ref{art4-thm2}, we can then easily prove the following

\medskip
\noindent
{\bf Theorem \thnum{3}.\label{art4-thm3}}~{\em Let}
$$
V'=(1-J)R'/A'.
$$
{\em Under the same assumptions as in Theorem \ref{art4-thm2}, there exist exact sequences of $R'$-modules}
$$
V'\to V\to 0,\quad 0\to V'\to V^{d}.
$$
{\em In particular, $V'$ is a finite dimensional vector space over $\mathbb{Q}_{p}$, and as $p$-adic representation spaces for $G=\Gal(K/\mathbb{Q})$, $V$ and $V'$ have the same composition factors.}

\smallskip
At this point, let us consider again the special case where $k=\mathbb{Q}(\sqrt[p]{1})$, $p>2$; the field $\mathbb{Q}(\sqrt[p]{1})$ certainly satisfies the conditions (i), (ii), and (iii) stated at the beginning of this section. In this case, $K=k_{\infty}$, $K^{+}=k^{+}_{\infty}$, and $k^{+}_{n}$, $n\geq 0$, are all abelian extensions over $\mathbb{Q}$, and Leopoldt's conjecture for $k^{+}_{n}$ is known to be true by a theorem of Brumer. On the other hand, it is easy to deduce $\lambda_{p}(k^{+})=0$ from Vandiver's conjecture $p\nmid h^{+}$ for the class number $h^{+}$ of $k^{+}$. Therefore we know by Theorem \ref{art4-thm2} that under the assumption $p\nmid h^{+}$, $V$ is cyclic over $G=\Gal(K/\mathbb{Q})$, namely,
$$
V=R'v_{0}
$$
with some vector $v_{0}$ in $V$. Now, $\lambda_{p}(k^{+})=0$ also implies $V=(1-J)V$ so that $V=(1-J)R'v_{0}$. Since $G=\Gal(K/\mathbb{Q})$ is an abelian group in this case, both $R$ and $R'$ are commutative rings. Hence it follows from the above that the map $\alpha'\to \alpha'v_{0}$, $\alpha'\in (1-J)R'$, induces an $R'$-isomorphism\pageoriginale
$$
V'=(1-J)R'/A'\xrightarrow{\sim}V.
$$
Furthermore, we know in this special case that there are many explicitly described elements in the ideal $A_{n}$ of $R_{n}$, $n\geq 0$, called Stickelberger operators for $k_{n}$, and that the $p$-adic $L$-functions $L_{p}(s;\chi)$ for $\chi$ in $\widehat{\Delta}=\Hom(\Delta,\Omega^{+}_{p})$ can be constructed by means of such Stickelberger operators\footnote[6]{See Iwasawa \cite{art4-key4} or Lang \cite{art4-key6}.}. Thus we obtain a relation between the $p$-adic representation space $V'$ and the $p$-adic $L$-functions $L_{p}(s;\chi)$, and hence between $V$ and $L_{p}(s;\chi)$ through the above isomorphism. This is the way how Theorem \ref{art4-thm1} is proved, and the proof is similar for Greenberg's generalization.

We now consider again the general case where $k$ is any finite algebraic number field satisfying the conditions (i), (ii), and (iii). For each C-M sub-field $k'$ of $k$ such that $k/k'$ is abelian, Stickelberger operators for $k_{n}/k'$ are still defined, and it is proved by Deligne and Ribet that such Stickelberger operators are related to abelian $p$-adic $L$-functions for $k'\cap k^{+}$ in much the same way as in the special case mentioned above. However, it is not known whether such general Stickelberger operators belong to the ideal $A_{n}$ and provide us with any essential part of $A_{n}$ defined above\footnote[7]{See the discussions in Coates \cite{art4-key1}.}. This prevents us from obtaining any nice relation between the $p$-adic representation space $V'$ and $p$-adic $L$-functions. On the other hand, we can find examples of $k/\mathbb{Q}$, satisfying (i), (ii), (iii) and also the assumptions in Theorem \ref{art4-thm2}, such that the representation spaces $V$ and $V'$ for $G=\Gal(K/\mathbb{Q})$ in Theorem \ref{art4-thm3} are not isomorphic to each other. Thus we see that the results of Theorems \ref{art4-thm2}, \ref{art4-thm3} tell us much less on the nature of the $p$-adic representation space $V$ for $G=\Gal(K/\mathbb{Q})$ than Theorem \ref{art4-thm1} for the special
 case $k=\mathbb{Q}(\sqrt[p]{1})$. Nevertheless, we still feel and hope that those theorems would be of some use in the future investigations to obtain a full generalization of Theorem \ref{art4-thm1} in \S2.

We also note in this connection that in such a generalization of Theorem \ref{art4-thm1}, one has certainly to consider $p$-adic (non-abelian) Artin $L$-functions. Given any Galois extension $L/K$ of totally real finite algebraic number fields, it is not difficult to define $p$-adic Artin $L$-function $L_{p}(s;\chi)$ for each\pageoriginale character $\chi$ of the Galois group $\Gal(L/K)$ so that $L_{p}(s;\chi)$ is related to the classical Artin $L$-function $L(s;\chi)$ in the usual manner and that those $L_{p}(s;\chi)$ share with the classical functions $L(s;\chi)$ all essential formal properties such as the formula concerning induced characters. One can even formulate the $p$-adic Artin conjecture for such $L$-functions; the conjecture is not yet verified and, in fact, it is closely related to the above mentioned problem of generalizing Theorem \ref{art4-thm1}. For all these, we refer the reader to forth-coming papers by R. Greenberg and B. Gross, noting here only that Weil's solution of Artin's conjecture for $L$-functions of algebraic curves defined over finite fields is based upon the study of the representations of Galois groups on the spaces similar to $V$ mentioned above.

\medskip
{\bf 4.}~We shall next briefly indicate an outline of the proof of Theorem \ref{art4-thm2}\footnote[8]{Cf. the proof of Theorem \ref{art4-thm5} in Greenberg \cite{art4-key2}.}. Following the general definition in \S3, let
$$
\Lambda = Z_{p}[[\Gamma]]
$$
for the profinite group $\Gamma=\Gal(K/k)$, and let $\gamma_{0}$ be any topological generator of $\Gamma\simeq Z_{p}$. Let $Z_{p}[[T]]$ denote as in \S2 the ring of all formal power series in $T$ with coefficients in $Z_{p}$. Then it is known that there is a unique isomorphism of compact algebras over $Z_{p}$:
$$
\Lambda = Z_{p}[[\Gamma]]\xrightarrow{\sim}Z_{p}[[T]]
$$  
such that $\gamma_{0}\to 1+T$. Hence fixing a topological generator $\gamma_{0}$, we may identify $\Lambda=Z_{p}[[\Gamma]]$ with $Z_{p}[[T]]$ so that $\gamma_{0}=1+T$. Then 
$$
\Lambda'=\foprod{\Lambda}{\mathbb{Q}_{p}}{Z_{p}}=\foprod{Z_{p}[[T]]}{\mathbb{Q}_{p}}{Z_{p}}
$$
and it is easy to see that $\Lambda'$ is a principal ideal domain. One also proves immediately that $\Lambda=Z_{p}[[\Gamma]]$ is a central subalgebra of $R-Z_{p}[[G]]$ and that the latter is a free $\Lambda$-module of rank $d=[k:\mathbb{Q}]=[G:\Gamma]$. Hence $R'=\foprod{R}{\mathbb{Q}_{p}}{Z_{p}}$ is an algebra over $\Lambda'=\foprod{\Lambda}{\mathbb{Q}_{p}}{Z_{p}}$ and it is a free module of rank $d$ over the principal ideal domain $\Lambda'$.

Now, let $L$ denote the maximal unramified abelian $p$-extension (i.e., Hilbert's $p$-class field) over $K$, and $M$ the maximal $p$-ramified abelian $p$-extension\pageoriginale 
