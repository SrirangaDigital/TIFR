\chapter{PERIOD INTEGRALS OF COHOMOLOGY CLASSES WHICH ARE REPRESENTED BY EISENSTEIN SERIES}

\begin{center}
{\large By~ G. Harder}
\end{center}

\bigskip

\setcounter{pageoriginal}{40}
\section*{Introduction}\pageoriginale
Our starting point is a very general question. Let $\Gamma$ be an arithmetic subgroup of a reductive Lie group $G_{\infty}$. Then the group $\Gamma$ acts on the symmetric space $X=G_{\infty}/K_{\infty}$ where $K_{\infty}\subset G_{\infty}$ is a maximal compact subgroup. Since $X$ is contractible one knows that the rational cohomology and homology groups of $\Gamma$ are isomorphic to the (co) homology groups of the quotient $\Gamma\backslash X$, i.e.
$$
H^{\nu}(\Gamma,\mathbb{Q})\simeq H^{\nu}(\Gamma\backslash X,\mathbb{Q})
$$
(Comp. \cite{art2-key21}, 1.6.).

In general the quotient space $\Gamma\backslash X$ is not compact. Borel and Serre have constructed a natural compactification $\Gamma\backslash X\hookrightarrow \Gamma\backslash \overline{X}$ where $\Gamma\backslash \overline{X}$ is a manifold with corners and where the inclusion is a homotopy equivalence. (Comp. \cite{art2-key3}). In various papers it has been shown that we can construct cohomology classes on $\Gamma\backslash X$ by starting from cohomology classes on the boundary. Roughly speaking we associate to a cohomology class $\psi$ on the boundary an Eisenstein series $E(\psi,s)$ which is a differential form depending on a complex parameters $s$. For a special value $s_{\psi}$ of our complex parameter this form may become a closed form. This closed form represents a cohomology class and its restriction to the boundary is related to our original class $\psi$(\cite{art2-key7}, \cite{art2-key8} and \cite{art2-key18}). We look at this as a procedure to construct cohomology classes on $\Gamma\backslash X$.

On the other hand we have another construction which gives us homology classes. To get these homology classes we start from lower dimensional reductive subgroups $M_{\infty}\hookrightarrow G_{\infty}$ for which $\Gamma_{M}=\Gamma\cap M_{\infty}$ is an arithmetic subgroup. If $X_{M}$ is the corresponding symmetric space we get a map $\Gamma_{M}\backslash X_{M}\to \Gamma\backslash X$. We even can find cases where $\Gamma_{M}\backslash X_{M}$ is compact and then the fundamental class of $\Gamma_{M}\backslash X_{M}$ gives us a homology class on $\Gamma\backslash X$. Our problem is to find situations where the dimension of $\Gamma_{M}\backslash X_{M}$\pageoriginale---which is also the dimension of the homology class--equals the dimension of an Eisenstein class. If this is the case we can ask for the value of the Eisenstein class on the above homology class which amounts to evaluating the integral
$$
\int\limits_{\Gamma_{M}\backslash X_{M}}E(\psi,s_{\psi})
$$
This idea of constructing cycles by means of subgroups $M_{\infty}\hookrightarrow G_{\infty}$ appears already in \cite{art2-key2} and \cite{art2-key16}.

In this paper we shall not consider the general problem but only a very special example. We take the group $G_{\infty}=\PGL_{2}(\mathbb{C})$ and $\Gamma$ will be a member of a very specific class of congruence subgroups of $\PGL_{2}(\bZ[i])$. If $\gamma\in \Gamma$ and if $\gamma$ is not unipotent then it generates a quadratic field extension $E(\gamma)$ in the matrix ring $M_{2}(\mathbb{Q}(i))$ which defines a reductive subgroup in $\PGL_{2}(\mathbb{C})$. Then the quotient $\Gamma_{M}\backslash X_{M}$ in this case will simply be a circle and we shall compute the integrals of Eisenstein classes over these circles. It will turn out that these period integrals are expressible in terms of values of $L$-functions with Grossencharaktere of type $A_{o}$. The results are stated in section \ref{art2-sec3}.

Actually we have much more general results. We have a clear picture for those arithmetic groups which come from the group $\GL_{2}$ over an arbitrary algebraic number field. It is planned to write a paper in which we treat this more general situation. But it is clear that this paper will be very long, very difficult to write and certainly also not easy to read. For instance we shall have to use adeles, we have to introduce coefficient systems and so on. That paper will contain proofs of the results announced in \cite{art2-key7} and the results in there have to be generalized. Therefore I made up my mind and decided to write a paper where all this is discussed in a special case. I tried to give many details which will cause some repetition and overlap with older papers and the one planned. But the degree of complexity in the general situation is very high and I think it might be useful to discuss one special case.

During the preparation of this paper here I became aware that also the theory of Eisenstein classes which has been announced in \cite{art2-key7} has some interesting arithmetic aspects. We shall devote a large part of this paper to\pageoriginale recall the theory of these Eisenstein classes and to discuss these arithmetic aspects which also concern values of some $L$-series. Therefore the title of the paper is not quite appropriate.

I want to thank D. Zagier for several discussions and for pointing to me how to compute the period integrals by a method that goes back to E. Hecke. (\cite{art2-key10}, 200).

\setcounter{section}{1}
\setcounter{subsection}{-1}
\subsection{Some Notations.}\label{art2-sec1.0}
If $R$ is any commutative ring with identity we denote its group of invertible elements by $R^{x}$.

The field $\mathbb{Q}[i]$ will be denoted by $F$, throughout this paper we consider $F$ as a subfield of $\mathbb{C}$, i.e. we fix an embedding of $F$ into $\mathbb{C}$. The ring of Gaussian integers $\bZ[i]\subset \mathbb{Q}(i)$ will be denoted by $\mathscr{O}$. More general if $E$ is any algebraic number field, we denote by $\mathscr{O}_{E}$ its ring of algebraic integers.

The finite places of $F$ will be denoted by $\sfp,\sfq\ldots$. The finite places of an extension $E/F$ will be denoted by capital letters $\sfP,\sfQ\ldots$. We denote by $E_{P}$ the completion at $\sfP$, by $\mathscr{O}_{\sfp}\subset F_{\sfp}$ the ring of $\sfp$-adic integers and by $\mathscr{O}_{E,\sfP}=\mathscr{O}_{\sfP}$ the ring of $\sfP$-adic integers. We drop the index $E$ if it is clear which filed we refer to.

We put $U_{P}=\mathscr{O}^{x}_{\sfP}$ and $U_{\sfP}=\mathscr{O}^{x}_{\sfP}$. The place of $F$ at infinity will be denoted by $\infty$ and the completion $F_{\infty}$ is canonically identified with $\mathbb{C}$.

The ring of adeles of $F$ is denoted by $\bA$ and by the letter we denote the group ideles of $F$. If we refer to another filed $E$ we write $\bA_{E}$, $I_{E}$. Elements of adele rings or idele groups will be denoted by underlined latin letters $\underline{x}$, $\underline{a}$, $\underline{u},\ldots$. If $\underline{x}\in \bA$ then we write
$$
\underline{x}=(x_{\infty},\ldots,x_{p},\ldots,x_{q},\ldots)
$$
i.e. $x_{\sfp}$, $x_{\sfq}$ are the $\sfp$, $\sfq$ components. By $\bA^{f}$ (resp $I^{f}$) we denote the ring (resp. group) of finite adeles (finite ideles) where we drop the component at $\infty$. Therefore
$$
\bA=\mathbb{C}\times \mathbb{A}^{f}, \ I=\mathbb{C}^{x}\times I^{f}
$$
and for $\underline{x}\in \bA$ we write $\underline{x}^{f}$ for its finite component, so that we have $\underline{x}=(x_{\infty},\underline{x}^{f})$.

By\pageoriginale $\sfU^{f}$ we denote the maximal compact subgroup of units in $I^{f}$, i.e. $\sfU^{f}=\Pi_{\sfp}U_{\sfp}$ and then $\sfU=U_{\infty}\times \sfU^{f}$ is the maximal compact subgroup in $I$, where $U_{\infty}$ is the circle group.

We start from the group $G_{o}/F=\PGL_{2}/F$. Then $B_{o}/F$, $U_{o}/F$ and $T_{o}/F$ will be the standard Borelsubgroup of upper triangular matrices, its unipotent radical and the standard diagonal torus. Sometimes it will be convenient to look at $G_{o}/F$ as a group over $\mathbb{Q}$, this means we put $G/\mathbb{Q}=R_{F/\mathbb{Q}}(G_{o}/F)$ where $R_{F/\mathbb{Q}}$ is the functor of restriction of scalars. (\cite{art2-key27}, 1.3.).

For any group scheme $H/A$ over any ring and any extension $A\to A_{1}$, we denote the group of points of $H$ with values in $A_{1}$ by $H(A_{1})$.

\subsection{The Cohomology of $\Gamma$ and the space $\Gamma\backslash X$.}\label{art2-sec1.1}
~

Let us put
$$
\Gamma_{o}=\PGL_{2}(\mathscr{O})=\PGL_{2}(Z[i])=\GL_{2}(\mathscr{O})/Z
$$
where $Z=\left\{\left(\begin{smallmatrix} i^{m} & 0\\ 0 & i^{m}\end{smallmatrix}\right)|m\in \bZ/4\bZ\right\}$. We have $\Gamma_{o}\subset \PGL_{2}(\mathbb{C})$ and the group $\Gamma_{o}$ acts on the three dimensional hyperbolic space $X=\PGL_{2}(\mathbb{C})/K_{\infty}$ where $K_{\infty}$ is the projective unitary group $\SU(2)/$centre = $\SO(3)$. We choose the standard embedding
$$
\SU(2)_=\left\{\left(\frac{\alpha}{-\beta}\frac{\beta}{\alpha}\right) \Big| \alpha\beta\in \mathbb{C}, \alpha\overline{\alpha}+\beta\overline{\beta}=1\right\}\subset \SL_{2}(\mathbb{C})
$$
We choose an ideal $\sfa\in \mathscr{O}$ which has to satisfy one of the following conditions
\begin{align}
\sfa &= ((1+i)^{3})\label{art2-eq1.1.1}\\
\text{or}\qquad \sfa & \text{~is an odd prime where $N(\sfa)=p$}\notag\\
                     & \text{~is a prime in $\bZ$ and $p\nequiv 1\mod 8$.}\notag
\end{align}
This condition \eqref{art2-eq1.1.1} implies that the group $W=\mathscr{O}^{x}=\{i,i^{-1},1,-1\}$ injects into the quotient $(\mathscr{O}/\sfa)^{x}$ and that $i$ is not a square in $(\mathscr{O}/\sfa)^{x}$.

Our main object of study are the congruence subgroups
$$
\Gamma=\Gamma (\sfa)=\left\{\left(\begin{matrix}a & b\\ c & d\end{matrix}\right)\in \Gamma_{o}\Big| \left(\begin{matrix} a & b\\ c & d\end{matrix}\right)=\text{Id~}\mod \sfa\right\}
$$
this\pageoriginale means that $\Gamma$ is the kernel of the natural homomorphism
$$
\Gamma_{o}=\PGL_{2}(\mathscr{O})\xrightarrow{p}\PGL_{2}(\mathscr{O}/\sfa)
$$

\smallskip
\noindent
{\bf Lemma \thnum{1.1.2}.\label{art2-lem1.1.2}}~{\em The homomorphism $p$ is surjective.}

\begin{proof}
It is very easy to see that the map
$$
\SL_{2}(\mathscr{O})\to \SL_{2}(\mathscr{O}/\sfa)
$$
is surjective. The image of $\SL_{2}(\mathscr{O}/\sfa)$ in $\PGL_{2}(\mathscr{O}/\sfa)$ is of index $2$ and the factor group is $(\mathscr{O}/\sfa)^{x}/((\mathscr{O}/\sfa)^{x})^{2}$. Then we see that $p\left(\left(\begin{smallmatrix} i & 0\\ 0 & 1\end{smallmatrix}\right)\right)\not{\in}$ image of $\SL_{2}(\mathscr{O}/\sfa)$ and this proves the lemma.

Let $R$ be any ring in $\mathbb{C}$. We want to assume always that the primes which divide the order of the finite group $\PGL_{2}(\mathscr{O}/\sfa)$ are invertible in $R$. We are interested in the cohomology group $H^{\nu}(\Gamma,R)$ and we can identify
$$
H^{\nu}(\Gamma,R)\simeq H^{\nu}(\Gamma\backslash X,R)
$$
since $\Gamma$ has no torsion, as one easily checks.

First of all we want to summarize some basic facts and definitions of the cohomology theory. If $M$ is a projective $R$-module on which we have an action of the finite group $\overline{G}=\Gamma_{o}/\Gamma=\PGL_{2}(\mathscr{O}/\sfa)$ we can define the cohomology groups
$$
H^{\nu}(\Gamma_{o},M)
$$
We will mainly be concerned with $H^{1}(\Gamma_{o},M)$ and we recall the definition in this case:

We write the action of $\overline{G}$ on $M$ by $(\overline{g},m)\to \overline{g}\cdot m$ and define
$$
Z^{1}(\Gamma_{0},M)=\{\Phi:\Gamma_{o}\to M|\Phi(\gamma_{1}\gamma_{2})=\Phi(\gamma_{1})+\gamma_{1}\Phi(\gamma_{2})\}
$$
This is the module of $1$-cocycles. We have a map
\begin{align*}
& M \xrightarrow{\delta_{o}}Z^{1}(\Gamma_{o},M)\\
& \delta : m\to \{\gamma\to (m-\gamma m)\}
\end{align*}
and $H^{1}(\Gamma_{o},M)=Z^{1}(\Gamma_{o},M)/\delta_{o}(M)$. 

There\pageoriginale is another way to define these cohomology groups: We look at the projection
$$
\overline{\pi}:X\to \Gamma_{o}\backslash X
$$
and we define a sheaf $\widetilde{M}$ on $\Gamma_{o}\backslash X$ as follows. For any open set $U\subset \Gamma_{o}\backslash X$ we define
$$
\widetilde{M}(U)=\left\{m : \overline{\pi}^{-1}(U)\to M 
\left|\begin{tabular}{l}
$m(\gamma u)=\gamma\cdot m(u)\text{~ and}$\\
$m\text{~ is locally constant.}$
\end{tabular}\right.\right\}
$$
It is well known that under the given assumptions we have (\cite{art2-key21}, 1.6).
$$
H^{\nu}(\Gamma_{o}\backslash X,\widetilde{M})\simeq H^{\nu}(\Gamma_{o},M)
$$

Let us look at the special case where $M=R[\overline{G}]$ is the group ring of the finite group $\overline{G}$. In this case we have two actions of $\overline{G}$ on $M$ namely by right and left multiplication
$$
(g_{1},g_{2}):\sum\limits_{\gamma\in G}a_{\gamma}\gamma\to \sum a_{\gamma}g_{1}\gamma g_{2}^{-1}
$$

We define the cohomology groups
$$
H^{1}(\Gamma_{o},R[\overline{G}])
$$
by the module structure given by right multiplication; so if $m=\sum\limits_{\gamma\in \overline{G}}a_{\gamma}\gamma\in R[\overline{G}]$ and $\gamma_{1}\in \overline{G}$ we have
$$
\gamma_{1}m=\sum\limits_{\gamma}a_{\gamma}\gamma\gamma^{-1}_{1}=\sum\limits_{\gamma}a_{\gamma\gamma_{1}}\gamma
$$
The well known Lemma of Shapiro tells us that
\setcounter{equation}{2}
\begin{equation}
H^{1}(\Gamma,R)\simeq H^{1}(\Gamma_{o},R[\overline{G}])\label{art2-eq1.1.3}
\end{equation}
and it is very easy to make this isomorphism explicit. If $\Phi:\Gamma_{o}\to R[G]$ is a 1-cocycle and if we write $\Phi(\gamma)=\sum\limits_{\sigma\in \overline{G}}\Phi_{\sigma}(\gamma)\sigma$. Then the cocycle relation tells us that $\Phi_{\sigma}(\gamma_{1})+\Phi_{\sigma\gamma_{1}}(\gamma_{2})=\Phi_{\sigma}(\gamma_{1}\gamma_{2})$ for all $\gamma_{1}$, $\gamma_{2}\in \Gamma$ and all $\sigma\in \overline{G}$. If we restrict $\Phi$ to the subgroup $\Gamma$ all the $\Phi_{\sigma}$ are homomorphisms. It follows from the cocycle relation that for $\gamma\in \Gamma$, $\eta\in \Gamma_{o}$ and $\overline{\eta}=\eta\mod \Gamma$
$$
\Phi_{\sigma}(\eta\gamma\eta^{-1})=\Phi_{\sigma\overline{\eta}}(\gamma)
$$
This\pageoriginale tells us that $\Phi_{1}$ determines the $\Phi_{\sigma}$ for $\sigma\neq 1$ and it is easy to see that $\Phi_{1}:\Gamma\to R$ is the image of the class represented by $\Phi$ and the Shapiro isomorphism \eqref{art2-eq1.1.3}.

The group $\overline{G}$ acts on the cohomology groups $H^{1}(\Gamma,R)=H^{1}(\Gamma\backslash X,R)$ where the action is induced by conjugation. On the other hand the action of $\overline{G}$ by left multiplication induces an action of $\overline{G}$ on $H^{1}(\Gamma_{o},R[\overline{G}])$ and it is not hard to check that \eqref{art2-eq1.1.3} commutes with these actions.

This isomorphism \eqref{art2-eq1.1.3} allows us to decompose the cohomology, we have
$$
R[\overline{G}]=\bigoplus\limits_{\theta}M_{\theta}
$$
where the $M_{\theta}$ are irreducible $\overline{G}\times \overline{G}$-modules. (Here we use our assumption that $1/|\overline{G}|\in R$). Then we get a decomposition
$$
H^{1}(\Gamma,R)=H^{1}(\Gamma_{o},R[\overline{G}])=\bigoplus\limits_{\theta}H^{1}(\Gamma_{o},M_{\theta})
$$
If we assume in addition that $R$ contains enough roots of unity, then the $M_{\theta}$ will be absolutely irreducible and we get
$$
M_{\theta}=M_{\widehat{\delta}}\otimes M_{\delta}
$$
where $\delta$ runs over the irreducible $\overline{G}$-modules and $\widehat{\delta}$ is the contragriedient module. Therefore we get
$$
H^{1}(\Gamma,R)=\bigoplus\limits_{\delta\in \widehat{\overline{G}}}H^{1}(\Gamma_{o},M_{\delta})\otimes M_{\widehat{\delta}}
$$
and the action of $\overline{G}$ on the right hand side is trivial on the first factor and the given action on $M_{\widehat{\delta}}$.
\end{proof}

\subsection{The Compactification of $\Gamma\backslash X$ and the Cohomology at Infinity}\label{art2-sec1.2}
It is well known that in this case the space $\Gamma\backslash X$ is not compact. It has a finite number of cusps which are in one-to-one correspondence with the $\Gamma$-conjugacy classes of Borel subgroups $B\subset G/F$. (\cite{art2-key1}) Borel and Serre developed a general theory of compactification of such spaces $\Gamma\backslash X$. They proved in \cite{art2-key3} that we have a homotopy equivalence
$$
\Gamma\backslash X\hookrightarrow \Gamma\backslash \overline{X}
$$
where in this special case $\Gamma\backslash \overline{X}$ is a compact manifold with a boundary. The\pageoriginale boundary components are in one-to-one correspondence with the $\Gamma$-conjugacy classes of Borel subgroups, i.e. they correspond to the cusps. We want to give a precise description of all this in our special situation.

Let $B$ be any Borel subgroup defined over the group field $F$. Let $U\subset B$ be its unipotent radical. It follows from the Iwasawa decomposition that the group $B(\mathbb{C})$ acts transitively on $X$. The positive root defines a homomorphism
$$
\alpha : B\to G_{m}
$$
and from this we get a homomorphism
$$
\alpha : B(\mathbb{C})\to G_{m}(\mathbb{C})=\mathbb{C}^{x}
$$
We put
$$
B^{(1)}(\mathbb{C})=\{b\in B(\mathbb{C})|~|\alpha (b)|_{\mathbb{C}}=1\}
$$
where $|z|_{\mathbb{C}}=z\overline{z}$ for $z\in \mathbb{C}^{x}$. The group
$$
B(\mathbb{C})\cap K_{\infty}=B^{(1)}(\mathbb{C})\cap K_{\infty}=K^{B}_{\infty}
$$
is a one dimensional circle and it is clear that we have a semidirect product 
$$
B^{(1)}(\mathbb{C})=U(\mathbb{C})\cdot K^{B}_{\infty}
$$
Therefore we have with $x_{o}=K_{\infty}\in G/K_{\infty}$
$$
X^{(1)}_{B}=B^{(1)}(\mathbb{C})\cdot x_{o}=U(\mathbb{C})\cdot x_{o}\subset X
$$
and $X^{(1)}_{B}\simeq U(\mathbb{C})\simeq \mathbb{C}$. If we put $\Gamma_{B}=B(\mathbb{C})\cap \Gamma$ then we get a homotopy equivalence
$$
\Gamma_{B}\backslash X_{B}^{(1)}=\Gamma_{B}\backslash U(\mathbb{C})\hookrightarrow \Gamma_{B}\backslash X
$$
and the Borel-Serre theory gives us that $\Gamma_{B}\backslash X^{(1)}_{B}$ is diffeomorphic to the boundary component $Y_{B}$ of $\Gamma\backslash \overline{X}$ which corresponds to $B$ (\cite{art2-key3}). Since $\Gamma_{B}\simeq \bZ\oplus \bZ$ we get that $Y_{B}$ is a product of two circles.

\bigskip
\noindent
{\bf Remarks}
\begin{enumerate}
\renewcommand{\labelenumi}{(\theenumi)}
\item Our congruence condition guarantees that $\Gamma\cap B(\mathbb{C})=\Gamma\cap U(\mathbb{C})$ since the image of $\Gamma\cap B(\mathbb{C})$ in $B(\mathbb{C})/U(\mathbb{C})$ has to consist of units in $\mathscr{O}$.

\item To\pageoriginale give the reader a better feeling for the Borel-Serre compactification we add a few more comments.
\end{enumerate}

We mentioned already that $B(\mathbb{C})$ acts transitively on $X$, we use this fact to define the function
\begin{align*}
& h_{B}:X\to \mathbb{R}^{x}_{t}\\
& h_{B}:x=bx_{o}\to |\alpha(b)|_{\mathbb{C}}
\end{align*}
We introduce the sets
$$
X^{B}(c)=\{x\in X|h_{B}(x)\geq c\}
$$
and the reduction theory tells us (\cite{art2-key1}, and \cite{art2-key5}, 1.2.) that for $c$ sufficiently large we have an embedding
$$
\Gamma_{B}\backslash X^{B}(c)\hookrightarrow \Gamma\backslash X
$$
and using the geodesic action or the vector field $dh^{B}$ we find
$$
\Gamma_{B}\backslash X^{B}(1)=\Gamma_{B}\backslash X^{(1)}_{B}\times [1,\infty)
$$
The Borel-Serre compactification in this case simply consists of adding $\infty$ in the second factor
$$
\Gamma_{B}\backslash X^{B}(1)=\Gamma_{B}\backslash X^{(1)}_{B}\times [1,\infty]\hookrightarrow \Gamma_{B}\backslash X^{(1)}_{B}\times [1,\infty]
$$
and $Y_{B}=\Gamma_{B}\backslash X^{(1)}_{B}\times \{\infty\}$.

The first part of the paper is devoted to the study of the map
$$
H^{1}(\Gamma\backslash X,R)\xrightarrow{\sim}H^{1}(\Gamma\backslash \overline{X},R)\to H^{1}(\partial (\Gamma\backslash \overline{X}),R)\xrightarrow{\sim}\bigoplus\limits_{B}H^{1}(Y_{B},R)
$$
where $B$ runs over a set of representatives for the $\Gamma$ conjugacy classes of Borel subgroups. The group $\Gamma_{B}$ is free abelian of rank $2$ and therefore we have
$$
H^{1}(Y_{B},R)=\Hom (\Gamma_{B},R)=R^{2}
$$

If we want to describe the cohomology of the boundary we have to describe the set of cusps or the set of $\Gamma$-conjugacy classes of Borel subgroups. This is very simple in this case since $\mathscr{O}$ has class number one. Actually we shall do a little bit better. We know that $H^{1}(\mathscr{O}(\Gamma\backslash \overline{X}), R)$ is a $\Gamma_{o}/\Gamma=\overline{G}$-module and we give a description of this $\overline{G}$-module.

Since\pageoriginale $\mathscr{O}$ has class number one it follows that the group $\Gamma_{o}$ acts transitively on the set of boundary components. It is easy to see (and will also follow from considerations in section \ref{art2-sec1.3}) that the stabilizer of the boundary component $Y_{B_{o}}$ is the group
$$
\overline{U}_{+}=\overline{U}_{o}\cdot W=\left\{\left(\begin{matrix} i^{m} & \overline{u}\\ 0 & 1\end{matrix}\right) \Big| \overline{u}\in \mathscr{O}/\sfa, i=i\mod \sfa\right\}
$$
where $W=\left\{\left(\begin{smallmatrix} i^{m} & 0\\ 0 & 1\end{smallmatrix}\right)\Big| m\in \bZ\right\}$. The group $\overline{U}_{+}$ acts on $H^{1}(Y_{B_{o}},R)$ and it follows from general principles of representation theory that we have an $\overline{G}$-module isomorphism
$$
H^{1}(\partial (\Gamma\backslash \overline{X}), R)\xrightarrow{\sim}\Ind^{\overline{G}}_{\overline{U}_{+}}H^{1}(Y_{B_{o}},R)
$$
where the induced module is the space of functions
$$
\left.
\begin{array}{r}
\Ind^{\overline{G}}_{\overline{U}_{+}}H^{1}(Y_{B_{o}},R)=\{h:\overline{G}\to H^{1}(Y_{B_{o}},R)|h(\overline{g}\overline{u}^{-1})=\overline{u}h(\overline{g})\\
\text{for~ } \overline{g}\in \overline{G}\text{~and~} \overline{u}\in\overline{U}_{+}
\end{array}
\right\}
$$
The group $\overline{G}$ acts on these functions by left translations.

It is easy to decompose this module into irreducible modules. We assume that $R$ contains the $|(\mathscr{O}/\sfa)^{*}|$-roots of unity. The group $\overline{U}_{+}=\overline{U}_{o}\cdot W$ and $U_{o}$ acts trivially on $H^{1}(Y_{B_{o}},R)$. Under the action of $W$ we have a decomposition $H^{1}(Y_{B_{o}},R)=\Hom(\Gamma_{B_{o}},R)=L_{+}\oplus L_{-}$ where $\left(\begin{smallmatrix} i & 0\\ 0 & 1\end{smallmatrix}\right)$ acts on $L_{+}$ by multiplication by $i$ and on $L_{-}$ by multiplication by $-i$.
$$
\left(
\begin{matrix}
i & 0\\
0 & 1
\end{matrix}
\right)
1_{+}=i1_{+};\quad \left(
\begin{matrix}
i & 0\\
0 & 1
\end{matrix}
\right)1_{-}=-i 1_{-}
$$
We look at the characters $\phi:(\mathscr{O}/\sfa)^{x}\to S^{1}$ for which $\phi(i)=i$. For each such character we have a subspace
$$
\left.
\begin{array}{r}
M^{*}_{\phi}=\{h:\overline{G}\to L_{+}|h(\overline{g}\overline{b}^{-1})=\phi(\overline{b})h(\overline{g})\\
\text{for~ } \overline{b}\in \overline{B}_{o}\text{~ and~ } \overline{g}\in \overline{G}
\end{array}
\right\}
$$
and $\Ind^{\overline{G}}_{\overline{U}_{+}}L_{+}$ and analogously we define $M^{*}_{\overline{\phi}}\subset \Ind^{G}_{\overline{U}_{+}}L_{-}$. This gives us a decomposition
\setcounter{equation}{0}
\begin{equation}
\Ind^{\overline{G}}_{\overline{U}_{+}}H^{1}(Y_{B_{o}},R)=H^{1}(\partial (\Gamma\backslash \overline{X}),R)=\bigoplus\limits_{\substack{\phi:(\mathscr{O}/\sfa)^{x}\to S^{1}\\ \phi(i)=i}}(M^{*}_{\phi}\oplus M^{*}_{\overline{\phi}})\label{art2-eq1.2.1}
\end{equation}\pageoriginale
where the $M^{*}_{\phi}$ and $M^{*}_{\overline{\phi}}$ are irreducible $\overline{G}$-modules. (\ref{art2-sec1.2.2} and \cite{art2-key25}, Cor. 4.11.) Here we profit from the fact that $\phi$ cannot be a trivial or a quadratic character.

\setcounter{subsubsection}{1}
\subsubsection{}\label{art2-sec1.2.2}
 At this point I want to give an idea of one of the main questions of this paper. As we have seen already we can study the restriction map
$$
H^{1}(\Gamma\backslash \overline{X},R)\to H^{1}(\partial (\Gamma\backslash \overline{X}),R)
$$
and we have decomposed the right hand side into irreducible modules \eqref{art2-eq1.2.1}. Let us assume that we have selected generators $e_{+}\in L_{+}$ and $e_{-}\in  L_{-}$ (We shall see later that we have a rather canonical choice, see \ref{art2-prop1.6.1}) then we can identify $M^{*}_{\phi}$ with the induced representation
$$
M_{\phi}=\{\psi :\overline{G}\to R|\psi(\overline{g}\overline{b}^{-1})=\phi(\overline{b})\psi(\overline{g})\}
$$
by mapping $\psi\to \{g\to \psi(g)\cdot g\cdot e_{+}\}$. One knows that $M_{\phi}$ and $M_{\overline{\phi}}$ are irreducible $\overline{G}$-modules and they are isomorphic. The operator
\begin{align*}
& T_{\phi}:M_{\phi}\to M_{\overline{\phi}}\\
& T_{\phi}:\psi\to T_{\phi}\psi(\overline{g})=\sum\limits_{u\in U_{o}}\psi(w\overline{ug})
\end{align*}
with $w=\left(\begin{smallmatrix} 0 & 1\\ -1 & 0\end{smallmatrix}\right)$ is a non zero interwining operator (\cite{art2-key25}, \S5).

Since there are no other isomorphisms among these induced representations the decomposition \eqref{art2-eq1.2.1} is isotypical.

Let us denote the quotient field of $R$ by $K$. For any $\phi$ we pick the isotypical component of $M_{\phi}$ in $H^{1}(\Gamma\backslash X,K)$ and get a map
$$
H^{1}(\Gamma\backslash X,K)_{\phi}\to M_{\phi}\otimes K\oplus M_{\overline{\phi}}\otimes K
$$

It follows from topological reasons that the image of the restriction map is of multiplicity one (namely $\frac{1}{2}\times$ the multiplicity of $M_{\phi}\otimes K\oplus M_{\overline{\phi}}\otimes K$ which is two) (comp. \cite{art2-key20} 3.4). Therefore the image is of the form (Schur's lemma)
$$
\{(\psi,c_{\phi}T_{\phi}\psi)|\psi\in M_{\phi}\otimes K\}\subset M_{\phi}\otimes K\oplus M_{\overline{\phi}}\otimes K
$$
where\pageoriginale $c_{\phi}\in K$ or $c_{\phi}=\infty$ in which case the image would be the second component. What is the value of $c_{\phi}$?

This problem will be attacked by transcendental methods, the theory of Eisenstein series will give us an expression for $c_{\phi}$ in terms of values of $L$-functions.

\subsubsection{}\label{art2-sec1.2.3}
Before I conclude this section I want to translate the questions and assertions \ref{art2-eq1.2.1} and \ref{art2-sec1.2.2} in the language of cohomology groups with coefficients.

We have the isomorphism \eqref{art2-eq1.1.3} and we put $\Gamma_{o,B_{o}}=B_{o}(F)\cap \Gamma_{o}$. Now we want to give a detailed description of the different isomorphisms in the following commutative diagram
\begin{equation*}
\vcenter{\xymatrix{
\Sh : H^{1}(\Gamma_{o},R[\overline{G}])\ar[r]^-{\sim} \ar[d]^{\res} & H^{1}(\Gamma,R)\simeq H^{1}(\Gamma\backslash X,R)\ar[d]\\
\partial \Sh:H^{1}(\Gamma_{o,B_{o}},R[\overline{G}])\ar[r]^-{\sim} & H^{1}(\partial(\Gamma\backslash \overline{X}),R)\ar[d]\\
 & \bigoplus\limits_{\substack{\phi:(\partial/\sfa)^{x}\to S^{1}\\ \phi(i)=i}}(M^{*}_{\phi}\oplus M^{*}_{\overline{\phi}})
}}\tag{1.2.3.1}\label{art2-eq1.2.3.1}
\end{equation*}

In this context it is convenient to identify the group ring $R[\overline{G}]$ with the ring of $R$ valued functions on $\overline{G}$ which is denoted by $\sfC(\overline{G})$ and
\begin{align*}
& \sfC(\overline{G})\xrightarrow{\sim}R[\overline{G}]\\
\text{by}\qquad & f\to \sum\limits_{\sigma\in \overline{G}}f(\sigma)\cdot \sigma
\end{align*}
Now let us assume that
$$
\Phi :\Gamma_{o}\to \sfC(\overline{G})
$$
is a 1-cocycle. Then for $\gamma\in \Gamma_{o}$ the value $\Phi(\gamma)$ is a function of $\overline{G}$ and the value of this function at $\sigma\in \overline{G}$ will be denoted by
$$
\Phi(\gamma) \ (\sigma)
$$
If we restrict $\Phi$ to $\Gamma$ then the map $\gamma\to \Phi(\gamma)(\sigma)$ is a homomorphism for any\pageoriginale $\sigma\in\overline{G}$ and we have seen that $\Sh$ is given by
$$
[\Phi]\to \{\gamma\to \Phi(\gamma)(1)\}
$$
where $[\Phi]$ is the class defined by $\Phi$. The class $[\Phi]$ defines a class on the boundary and we get a family of homomorphisms
\begin{align*}
& \phi_{B}:\Gamma_{B}=\Gamma\cap B(F)\to R\\
& \phi_{B}(\gamma_{B})=\Phi(\gamma_{B})(1)
\end{align*}
where $B$ runs over a set of $\Gamma$ conjugacy classes of Borel subgroups. If we write $\Gamma_{B}=\eta\Gamma_{B_{o}}\eta^{-1}$ with $\eta\in \Gamma_{o}$ then we get a homomorphism
\begin{gather*}
\Gamma_{B_{o}}\to R\\
\gamma_{o}\to \phi_{B}(\eta\gamma_{o}\eta^{-1})=\Phi(\eta\gamma_{o}\eta^{-1})(1)
\end{gather*}

But for $\gamma_{o}\in \Gamma_{B_{o}}=B_{o}(F)\cap \Gamma$ we have
$$
\Phi(\eta\gamma_{o}\eta^{-1})=\eta\Phi(\gamma_{o})=\overline{\eta}\Phi(\gamma_{o})
$$
where $\overline{\eta}$ is the image of $\eta\in \Gamma_{o}$ in $\overline{G}$. Therefore we have
$$
\Phi(\eta\gamma_{o}\eta^{-1})(1)=\Phi(\gamma_{o})(\overline{\eta})
$$
and this tells us that the cocycle $\Phi$ defines a map
\begin{align*}
& h_{\Phi}:\overline{G}\to \Hom(\Gamma_{B_{o}},R)\\
& h_{\Phi}:\sigma\to \{\gamma_{o}\to (\gamma_{o})(\sigma)\}
\end{align*}

This map is also defined for cocycles on $\Gamma_{o,B_{o}}$ with values in $R[\overline{G}]$ and the map $[\Phi]\to h_{\phi}$ gives us a direct realisation of $\partial \Sh$ and makes the commutativity of the diagram clear.

This means that the study of our restriction map can be reduced to the inverstigation of maps
$$
H^{1}(\Gamma_{o},M)\to H^{1}(\Gamma_{o,B_{o}},M)
$$
where $M$ is a projective $R$-module on which we have an irreducible $\overline{G}$-action, i.e. $M\bigotimes\limits_{R}K$ is an irreducible $\overline{G}$-module. Again we want to assume that $R$ contains enough roots of unity.

We\pageoriginale consider $H^{1}(\Gamma_{o,B_{o}},M)$. We have
$$
\Gamma_{o,B_{o}}=\left\{\left(\begin{matrix} t & u\\ 0 & 1\end{matrix}\right)|t\in \mathscr{O}^{x},u\in \mathscr{O}\right\}=\Gamma_{o,U_{o}}\cdot W
$$
where $\Gamma_{o,U_{o}}=\left\{\left(\begin{smallmatrix} 1 & u\\ 0 & 1\end{smallmatrix}\right)|u\in \mathscr{O}\right\}$ and $W$ is cyclic of order four generated by $\left(\begin{smallmatrix} i & 0\\ 0 & 1\end{smallmatrix}\right)$.

We always identify $W\subset \Gamma_{o,B_{o}}$ with its image in $\overline{B}_{o}=\left\{\left(\begin{smallmatrix} t & u\\ 0 & 1\end{smallmatrix}\right) | t\in (\mathscr{O}/\sfa)^{x}, \overline{u}\in \mathscr{O}/\sfa\right\}$.

Since we assume that $|\overline{G}|$ is invertible in our ring $R$ we see that the action of $\overline{U}_{o}$ on $M$ is semisimple and it is obvious that
$$
H^{1}(\Gamma_{o,U_{o}},M)=\Hom (\Gamma_{o,U_{o}},M^{\overline{U}_{o}})
$$
where we have to take into account that $M^{\overline{U}_{o}}=M^{\Gamma_{o},U_{o}}$. (The notation $M^{\overline{U}_{o}}$ means of course that we take the invariants). Therefore we can restrict our attention to those modules where $M^{\overline{U}_{o}}\neq (0)$. It is well known that in this case $M$ has to be a submodule of an induced module $N_{\chi}$ where $\chi$ is a character $\chi:\overline{B}_{o}\to \overline{B}_{o}/\overline{U}_{o}\to S^{1}$ and
$$
N_{\chi}=\{f:\overline{G}\to R|f(\overline{b}\overline{g})=\chi(\overline{b})f(\overline{g})\}
$$
The module $N^{\overline{U}_{o}}_{\chi}$ is easy to compute. We have the Bruhat decomposition $\overline{G}=\overline{B}_{o}w\overline{U}_{o}\cup \overline{B}_{o}$ with $w=\left(\begin{smallmatrix} 0 & 1\\ -1 & 0\end{smallmatrix}\right)$ and we put
\begin{align*}
f_{o} &= 
\begin{cases}
\overline{b}w\overline{y} &\to \chi(\overline{b})\\
\overline{b} &\to 0
\end{cases}\\
f_{\infty} &= 
\begin{cases}
\overline{b}w\overline{u} & \to 0\\
\overline{b} & \to \chi(\overline{b})
\end{cases}
\end{align*}
Then $N^{\overline{U}_{o}}_{\chi}=Rf_{o}\oplus Rf_{\infty}$. The group $\overline{G}$ acts on $N_{\chi}$ by right translations, if we restrict this action to $\overline{B}_{o}$, then $N^{\overline{U}_{o}}_{\chi}$ is an invariant subspace and
$$
\overline{b}f_{o}=\chi(\overline{b})^{-1}f_{o}, \ \overline{b}f_{\infty}=\chi(\overline{b})f_{\infty}
$$
Since $\Gamma_{o,B_{o}}=\Gamma_{o,U_{o}}\cdot W$ we have obviously
$$
H^{1}(\Gamma_{o,B_{o}},M)=\Hom(\Gamma_{o,U_{o}},M^{\overline{U}_{o}})^{W}
$$

The group $W$ acts on $\Gamma_{o,U_{o}}$ by means of the adjoint action and the module\pageoriginale $\Gamma_{o,U_{o}}\otimes \mathscr{O}$ decomposes into two spaces on which $\left(\begin{smallmatrix} i & 0\\ 0 & 1\end{smallmatrix}\right)\in W$ acts by the eigenvalues $i$, $-i$. Then it becomes clear, that $H^{1}(\Gamma_{o,B_{o}},M)\neq 0$ if and only if $\chi(i)=\pm i$. We assume $\chi(i)=i$ ans we call this character $\phi$ again. So $\phi=\chi$, then we have that $N_{\phi}$ is irreducible (\cite{art2-key25} 4.11.) and $M=N_{\phi}$.

We find
$$
\Hom (\Gamma_{o,U_{o}},N^{\overline{U}_{o}}_{\phi})^{W}=\Hom(\Gamma_{o,U_{o}},Rf_{o})^{W}\oplus \Hom(\Gamma_{o,U_{o}},Rf_{\infty})^{W}
$$
and
\begin{align*}
&\Hom(\Gamma_{o,U_{o}},Rf_{o})=\Hom(\Gamma_{o,U_{o}},R)=\Hom(\Gamma_{B_{o}},R)\\
&\Hom(\Gamma_{o,U_{o}},Rf_{\infty})=\Hom(\Gamma_{o,U_{o}},R)=\Hom(\Gamma_{B_{o}},R)
\end{align*}
But we have to keep in out mind that $W$ acts {\em non trivially} on $Rf_{o}$, $Rf_{\infty}$ and it acts trivially on $R$. If we take up our earlier notations we find
\begin{align*}
&\Hom(\Gamma_{o,U_{o}},Rf_{o})^{W}=L_{+}\subset \Hom(\Gamma_{B_{o}},R)\\
&\Hom(\Gamma_{o,U_{o}},Rf_{\infty})^{W}=L_{-}\subset \Hom(\Gamma_{B_{o}},R)
\end{align*}
We constructed an identification
$$
H^{1}(\Gamma_{o,B_{o}},N_{\phi})=L_{-}\oplus L_{-}=Re_{+}\oplus Re_{-}
$$
if we take up the notations in \ref{art2-sec1.2.2}.

We look again at our restriction map
$$
H^{1}(\Gamma_{o},N_{\phi})\to H^{1}(\Gamma_{o,B_{o}},N_{\phi})=Re_{+}\oplus Re_{-}
$$
and we want to relate this to \ref{art2-sec1.2.2}.

Let us pick the isotypical component $R[\overline{G}]_{\phi}$ in $R[\overline{G}]$ then we get
$$
\xymatrix{
H^{1}(\Gamma_{o},R[\overline{G}]_{\phi})\ar@{=}[d] &\\
H^{1}(\Gamma,R)_{\phi}\ar[r] & M^{*}_{\phi}\oplus M^{*}_{\overline{\phi}}=M_{\phi}\oplus M_{\overline{\phi}}
}
$$
On the other hand we realized our given induced representation as a submodule of $R[\overline{G}]_{\phi}$ namely
$$
N_{\phi}\hookrightarrow R[\overline{G}]_{\phi}
$$

Therefore\pageoriginale we get a diagram
\[
\xymatrix{
H^{1}(\Gamma_{o},N_{\phi})\ar[r]\ar@{^(->}[d] & L_{+}\oplus L_{-}=\rRe_{+}\oplus\rRe_{-}\ar@<2ex>@{^(->}[d]^{\lambda}\ar@<-2ex>@{^(->}[d]^{\lambda}\\
H^{1}(\Gamma_{o},R[\overline{G}]_{\phi})\ar[r] & M^{*}_{\phi}\oplus M^{*}_{\overline{\phi}}=M_{\phi}\oplus M_{\overline{\phi}}
}
\]
and we have to compute the inclusions $\lambda$ and $\lambda_{1}$.

To get these inclusions we observe that a generator of $L_{+}$ is given by the cocycle
\begin{align*}
& \Gamma_{U_{o}}\to Rf_{o}\subset N_{\phi}\\
& \gamma\to e_{+}(\lambda)\cdot f_{o}
\end{align*}
This defines \eqref{art2-eq1.2.3.1} a map
\begin{align*}
& h:\sigma \to \{\lambda\to e_{+}(\lambda)\cdot f_{o}(\sigma)\}\\
& h:\overline{G}\to \Hom(\Gamma_{U_{o}},R)
\end{align*}
Now we observe that $f_{o}\in N_{\phi}$ is also an element in $M_{\phi}$ and we see that $\lambda_{1}:e_{+}\to f_{o}\in M_{\phi}$ and $\lambda_{1}:e\to f_{\infty}\in M_{\overline{\phi}}$. The intertwining operator $T_{\phi}:M_{\phi}\to M_{\overline{\phi}}$ maps $T_{\phi}(f_{o})=N(\sfa)f_{\infty}$ and we get the proposition.

\medskip
\noindent
{\bf Proposition \thnum{1.2.4}.\label{art2-prop1.2.4}}~{\em The image of the restriction map}
$$
H^{1}(\Gamma_{o},N_{\phi}\otimes K)\to Ke_{+}+Ke_{-}
$$
{\em is spanned by the vector $(e_{+},c_{\phi}N(\sfa)e_{-})$ where $e_{\phi}\in K\cup \{\infty\}$.}

\smallskip
What is all this good for? If we want to compute explicitely with cocycles it seems to be convenient to work with $\Gamma_{o}$ instead of $\Gamma$ since it has less generators. We pay for it by introducing coefficients. Later on we shall compute $H^{1}(\Gamma_{o},N_{\phi})$ in some simple cases and we are then able to compute the number $c_{\phi}$.

\subsection{Adeles and the Description of the Set of Cusps}\label{art2-sec1.3}

In the adele group $G_{o}(\bA)=\PGL_{2}(\bA)$ we have the maximal compact subgroup
$$
\sfK=K_{\infty}\cdot \sfK^{f}=K_{\infty}\times \prod\limits_{\sfp\text{~finite}}\PGL_{2}(\mathscr{O}_{\sfp})
$$
where\pageoriginale $K_{\infty}=PU(2)(1.1)$. The ideal $\sfa$ defines a congruence subgroup $\sfK^{f}(\sfa)\subset \sfK^{f}$ namely
$$
\sfK^{f}(\sfa)=\{\underline{k}^{f}\in \sfK^{f}|\overline{k}^{f}\equiv 1\mod \sfa\}
$$

\noindent
{\bf Lemma \thnum{1.3.1}.\label{art2-lem1.3.1}.}~{\em Every element $\underline{x}\in G_{o}(\bA)$ can be written}
$$
\underline{x}=a\cdot (y_{\infty},\underline{k}^{f})
$$
{\em with $\underline{k}^{f}\in \sfK^{f}(\sfa)$.}

\begin{proof}
We represent $\underline{x}\in G_{o}(\bA)$ by an element $\widetilde{\underline{x}}\in \GL_{2}(\bA)$. If $\widetilde{\underline{x}}\in \SL_{2}(\bA)$, i.e. $\det(\widetilde{\underline{x}})=1$ then the assertion follows from strong approximation for $\SL_{2}$. We may modify $\widetilde{\underline{x}}$ by an element $\underline{z}\in I$ which we consider as an element of the center of $\GL_{2}(\bA)$, then $\det(\widetilde{\underline{x}})$ gets multiplied by $z^{2}$. So the obstruction to get the determinant $\widetilde{\underline{x}}$ equal to one sits in $I/I^{2}$. We may modify $\widetilde{\underline{x}}$ by an element in $\GL_{2}(F)$ and by an element in the inverse image of $\sfK^{f}(\sfa)$ in $\GL_{2}(\bA)$. This means that the obstruction against writing $\underline{x}$ in the above form lies in
$$
I/I^{2}\cdot F^{x}\cdot \sfU^{f}(\sfa)
$$
where $\sfU^{f}(\sfa)=\{\underline{t}\in \sfU^{f}|\underline{t}\equiv 1\mod \sfa\}$. Using the fact that $F$ has class number one we find that this group is equal to
$$
\mathbb{C}^{x}\times \sfU^{f}/\sfU^{f}(\sfa)\cdot W\cdot (\sfU^{f})^{2}
$$
where $W=\{i\}\subset F^{x}$. Since nothing is claimed at the infinite place we may drop the infinite component and the obstruction sits in $(\mathscr{O}/\sfa)^{x}/((\mathscr{O}/\sfa)^{x})^{2}\cdot W=1$. The lemma is proved.

The lemma says simply that
$$
G_{o}(F)\backslash G_{o}(\bA)/G_{o}(\mathbb{C})\times \sfK^{f}(\sfa)=\{1\}.
$$
Now we consider the double coset decomposition
$$
B_{o}(F)\backslash G_{o}(\bA^{f})/\sfK^{f}(\sfa)
$$
Let us write
$$
G_{o}(\bA^{f})=\bigcup\limits_{\underline{\xi}}B_{o}(F)\underline{\xi}\sfK^{f}(\sfa)
$$
We extent $\underline{\xi}$ to an element $\underline{\xi}'$ of $G_{o}(A)$ by $(\underline{\xi}')_{\infty}=1$. According to our previous\pageoriginale lemma we may write
$$
\underline{\xi}'=a\cdot (a^{-1},\underline{k}^{f}).
$$
with $\underline{k}^{f}\in \sfK^{f}(\sfa)$. Then $a^{-1}B_{o}a=B$ is a Borel subgroup over $F$. We see that the $\Gamma$-conjugacy class of this Borel subgroup depends only on $\underline{\xi}$. If we pick a Borel subgroup $B/F$ then we find an $a\in G_{o}(F)$ such that $B=a^{-1}B_{o}a$. We choose $\underline{\xi}'=(1,a,\ldots,a,\ldots)$ and therefore we get a bijection between the set of $\Gamma$-conjugacy classes of Borel subgroups $B/F$ of $G/F$ and the set of double cosets
$$
B_{o}(F)\backslash G_{o}(\bA^{f})/\sfK^{f}(\sfa)
$$
We are now able to settle the minor point left open in \ref{art2-sec1.2} concerning the stabilizer of the boundary component $Y_{B_{o}}$ under the action of $\overline{G}$. We simply count the number of cusps. We have a map
$$
B_{o}(F)\backslash G_{o}(\bA^{f})/\sfK^{f}(\sfa)\to B_{o}(\bA^{f})\backslash G_{o}(\bA^{f})/\sfK^{f}(\sfa)
$$
which is surjective.

Since $B_{o}(\bA^{f})\cdot \sfK^{f}=G_{o}(\bA^{f})$ we get
$$
B_{o}(\bA)\backslash G_{o}(\bA^{f})/\sfK^{f}(\sfa)=\sfK^{f}/\sfK^{f}(\sfa)\cap B_{o}(\bA^{f})=\overline{G}/\overline{B}_{o}
$$
The fibers of this map are equal to
$$
B(F)\backslash B(\bA^{f})/B(\bA^{f})\cap \sfK^{f}(\sfa)
$$
where $B$ is any Borel subgroup corresponding to a point in the fiber. Since the unipotent radical has strong approximation we find for these fibers that they are equal to
$$
I^{f}/F^{x}\cdot \sfU^{f}(\sfa)=\sfU^{f}/W\sfU^{f}(\sfa)=(\mathscr{O}/\sfa)^{x}/W
$$
and that proves that the number of cusps is equal to $[\overline{G}:\overline{U}_{+}]$
\end{proof}

\subsection{Differential Forms and De Rham Cohomology}\label{art2-sec1.4}

We should look at $G_{o}/F$ as group over the rationals and therefore we introduce $G/\mathbb{Q}=R_{F\backslash \mathbb{Q}}(G_{o}/F)$. The Lie algebra $\sfg=\Lie (G/\mathbb{Q})$ is a $\mathbb{Q}$-vector space and we define $\sfg_{\infty}=\sfg\bigotimes\limits_{\mathbb{Q}}\mathbb{R}$. Then $\sfg_{\infty}$ is the Lie algebra of the real group $G_{\infty}=G(\mathbb{R})=G_{o}(\mathbb{C})$. (We shall sometimes denote the group of complex points\pageoriginale of groups over $F$ by the subscript $\infty$ and then we stress the point of view that they may also be considered as real points of a group over $(\mathbb{Q})$. Now
$$
\sfg_{\infty}=\left\{\left(\begin{matrix} \alpha & \beta\\ \gamma & -\alpha\end{matrix}\right)\Big| \alpha,\beta,\gamma\in \mathbb{C}\right\}
$$
and the Cartan involution obtained from our given maximal compact group is
$$
\theta:X=\left(\begin{matrix} \alpha & \beta\\ \gamma & -\alpha\end{matrix}\right)\to - {}^{t}\overline{X}=\left(\begin{matrix} -\overline{\alpha} & -\overline{\gamma}\\ -\overline{\beta} & \overline{\alpha}\end{matrix}\right)
$$
We get the Cartan decomposition
$$
\sfg_{\infty}=\sfk_{\infty}+\sfp
$$
where $\sfk_{\infty}=\Lie (K_{\infty})$. The real vector space $\sfp$ has the basis
$$
H=\left(\begin{matrix} 1 & 0\\ 0 & -1\end{matrix}\right), \ E_{1}=\left(\begin{matrix} 0 & 1\\ 1 & 0\end{matrix}\right), \ E_{2}=\left(\begin{matrix} i & 0\\ 0 & -i\end{matrix}\right)
$$
The group $K_{\infty}=PU(2)=SU(2)/\{+\iId\}$ acts on $\sfp$ by the adjoint action. If $k_{\infty}\in PU(2)$ is represented by the matrix $\left(\begin{smallmatrix} \alpha & \beta\\ -\overline{\beta} & \overline{\alpha}\end{smallmatrix}\right)\in SU(2)$ then
\setcounter{equation}{0}
\begin{align}
& \ad(k_{\infty})H=(\alpha\overline{\alpha}-\beta\overline{\beta})H-2\rRe(\alpha\beta)E_{1}-2\iIm(\alpha\beta)E_{2}\notag\\
& \ad(k_{\infty})E_{1}=2\rRe(\alpha\beta)H+\rRe(\alpha^{2}-\beta^{2})E_{1}+\iIm(\alpha^{2}+\beta^{2})E_{2}\label{art2-eq1.4.1}\\
& \ad(k_{\infty})E_{2}=2\iIm(\overline{\alpha}\beta)H=\iIm(\alpha^{2}-\beta^{2})E_{1}+\rRe(\alpha^{2}+\beta^{2})E_{2}\notag
\end{align}
The normalised Killing form on $\sfg_{\infty}$
$$
\langle X,Y\rangle =\frac{1}{16} \text{~trace~}(\ad X\cdot \ad Y)
$$
induces a $K_{\infty}$ invariant, positive definite symmetric quadratic form on $\sfp$. With respect to this form our three vectors $H$, $E_{1}$, $E_{2}$ form an orthonormal basis. We shall use this form to identify the space $\sfp$ with its dual space.

The projection map
$$
\pi:G_{\infty}\to G_{\infty}/K_{\infty}=X
$$
defines an isomorphism
$$
(d\pi)_{e}:\sfp\to T_{x_{o}}=\check{T}_{x_{o}}
$$
between $\sfp$ and the tangent space of $X$ at the point $x_{o}$.

This\pageoriginale allows us to identify the space of differential $p$-forms on $X$ with a certain space of $\Lambda {}^{p}\sfP$-valued functions on the group $G_{\infty}$. To be more precise we can identify the space $\Omega^{p}(X)$ of $C^{\infty}$-$p$-forms on $X$ and the space of $C^{\infty}$-functions
$$
C^{p}(G_{\infty},\Lambda^{p}\ad,\Lambda^{p}\sfp)=\{\omega:G_{\infty}\to \Lambda^{p}\sfp|\omega(g_{\infty}k_{\infty})=\Lambda^{p}\ad(k^{-1}_{\infty})\omega(g_{\infty})\}
$$
(\cite{art2-key8}, 1.3). We want to make this identification perfect in the sense that we do not distinguish between the $p$-form and the function on $G_{\infty}$. The identification goes as follows: Let $\omega:G_{\infty}\to \Lambda {}^{p}\sfp$ which satisfies $\omega(g_{\infty}k_{\infty})=\Lambda^{p}\ad(k^{-1}_{\infty})\omega(g_{\infty})$. If $x\in X$ and $g_{\infty}\in G_{\infty}$ satisfies $g_{\infty}x_{o}=x$ then the left translation $y\to g_{\infty}y$ on $X$ induces an isomorphism of tangent spaces
$$
dL_{g_{\infty}}:T_{x_{o}}\xrightarrow{\sim}T_{x}
$$
If $t_{x}\in \Lambda^{p}T_{x}$ then $\omega$ considered as a $p$-form has to have a value on $t_{x}$
\begin{equation}
\omega(x)(t_{x})=\langle \omega(g_{\infty}),\Lambda^{p}dL_{g_{\infty^{-1}}}(t_{x})\rangle\label{art2-eq1.4.2}
\end{equation}
This identification is compatible with the action of $G_{\infty}$ from the left on $X$, so we may divide by $\Gamma$ and get
$$
\Omega^{p}(\Gamma\backslash X)=C^{p}(\Gamma\backslash G_{\infty},\Lambda^{p}\ad,\Lambda^{p}\sfp)
$$
It is important that we can write this also as a space of function on the adele group. Using lemma \ref{art2-lem1.3.1} we find
\begin{align*}
& C^{p}(\Gamma\backslash G_{\infty},\Lambda^{p}\ad,\Lambda^{p}\sfp)=\\
& C^{p}(G_{o}(F)\backslash G_{o}(\bA)/\sfK^{f}(\sfa),\Lambda^{p}\ad,\Lambda^{p}\sfp)=\\
& 
\left\{
\begin{array}{l}
\omega:G_{o}(F)\backslash G_{o}(\bA)\to \Lambda^{p}\sfp|\omega\text{~ is~ } C^{\infty}\text{~ in the infinite component}\\
\text{and~ }\omega(\underline{g}\underline{k})=\Lambda^{p}\ad(k^{-1}_{\infty})\omega(g)\text{~ where~ } \underline{k}=(k_{\infty},\underline{k}^{f})\text{~ and}\\
\sfk^{f}\in \sfK^{f}(\sfa)
\end{array}\right\}
\end{align*}

\subsection{De Rham Cohomology at Infinity :}\label{art2-sec1.5}
Let $B/F$ be any Borel subgroup of $G_{o}/F$. This Borel subgroup defines a boundary component $Y_{B}\subset \partial (\Gamma\backslash X)$ and we want to describe the cohomology of this boundary component in terms of differential forms.

We still fix our base point $x_{o}\in X$. We have seen that the boundary component\pageoriginale associated to $B$ is diffeomorphic to
$$
\Gamma_{B}\backslash X^{(1)}_{B}=\Gamma_{B}\backslash U(\mathbb{C})x_{o}=\Gamma_{B}\backslash U(\mathbb{C})
$$
and we have homotopy equivalences
$$
\Gamma_{B}\backslash X^{(1)}_{B}\hookrightarrow \Gamma_{B}\backslash X\hookrightarrow \Gamma_{B}\backslash X\cup Y_{B}
$$
(\ref{art2-sec1.2} Remark \ref{art2-rem2}). The group $B_{\infty}=B(\mathbb{C})$ acts transitively on $X$ and we put
$$
K^{b}_{\infty}=B_{\infty}\cap K_{\infty}
$$
Then $K^{B}_{\infty}$ is a circle. This allows us another description of the space of $C^{\infty}$-$p$-forms on $\Gamma_{B}\backslash X$:
\begin{align*}
& \Omega^{p}(\Gamma_{B}\backslash X)=C^{p}(\Gamma_{B}\backslash B_{\infty};\Lambda^{p}\ad,\Lambda^{p}\sfp)=\\
& \qquad\qquad
\left\{
\begin{array}{l}
\omega|\omega:\Gamma_{B}\backslash B_{\infty}\to \Lambda^{p}\sfp;\omega\text{~ is~ }C^{\infty}\text{~ and}\\
\qquad\qquad \omega(b_{\infty}k_{\infty})=\Lambda^{p}\ad(k^{-1}_{\infty})\omega(b_{\infty})\\
\qquad\qquad \text{for~ } k_{\infty}\in K^{B}_{\infty}.
\end{array}
\right\}
\end{align*}
Under the action of $K^{B}_{\infty}$ we have a canonical decomposition of
$$
\sfp=\sfp_{o,B}\oplus \sfp_{1,B}=\sfp_{o}\oplus \sfp_{1}
$$
where $\sfp_{o}$ is of dimension 1 and $K^{B}_{\infty}$ acts trivially and $\sfp_{1,B}$ is two dimensional irreducible. In the case of $B=B_{o}$ this decomposition becomes
$$
\sfp=\mathbb{R} H \oplus (\mathbb{R} E_{1}\oplus \mathbb{R}  E_{2})
$$
Therefore we get for any 1-form on $\Gamma_{B}\backslash X$ a decomposition
$$
\omega=\omega_{o,B}+\omega_{1,B}=\omega_{o}+\omega_{1}
$$
It is clear that the $\omega_{o}$ component vanishes if we restrict it to the ``slices''
$$
\Gamma_{B}\backslash X_{B}^{(1)}\to \Gamma_{B}\backslash X
$$
and this tells us that our decomposition does not depend on the choice of the base point $x_{o}$.

Let us assume that $\omega\in\Omega^{1}(\Gamma_{B}\backslash X)$ is a closed 1-form. Then $\omega$ defines a cohomology class $[\omega]\in H^{1}(\Gamma_{B}\backslash X;\mathbb{R})=H^{1}(\Gamma_{B}\backslash X_{B}^{(1)};\mathbb{R})$. We want to compute this class. The group $U_{\infty}$ acts on $\Gamma_{B}\backslash X$ by translations and from this\pageoriginale it follows that the cohomology class $[\omega]$ is also represented by the form
$$
\omega^{(0)}(b_{\infty})=\int\limits_{\Gamma_{B}\backslash U_{\infty}}(u_{\infty}b_{\infty})du_{\infty}
$$
where the volume $\Gamma_{B}\backslash U_{\infty}$ is normalized to be equal to 1. If we restrict this 1-form $\omega^{(0)}$ to $\Gamma_{B}\backslash X^{(1)}$ we get
$$
\omega^{(0)}\left| \Gamma_{B}\backslash X^{(1)}=\omega^{(0)}_{B}\right| \Gamma_{B}\backslash X^{(1)}
$$
and $\omega^{(0)}_{1,B}$ is translation invariant and therefore constant. This means
$$
\omega^{(0)}_{1,B}(u_{\infty})=\omega^{(0)}_{1,B}(1)\in \sfp_{1,B}
$$
This element $\omega^{(0)}_{1,B}$ defines a homomorphism from $\Gamma_{B}$ into $R$:

Every element $\gamma\in \Gamma_{B}$ can be written in the form $\gamma=\exp \log \gamma$ where $\log\gamma=\iId-\gamma\in \Lie (R_{F/\mathbb{Q}}(U/F))$ and the homomorphism is
\begin{gather*}
\gamma\to \langle \log \gamma,\omega_{1,B}^{(0)}(1)\rangle =\\
=\langle \log \gamma, \omega^{(0)}(1)\rangle
\end{gather*}
Since we have $H^{1}(\Gamma_{B},\mathbb{R})=\Hom(\Gamma_{B},\mathbb{R})$ we find the formula
\setcounter{equation}{0}
\begin{equation}
[\omega](\gamma)=\langle \log \gamma,\omega^{(0)}(1)\rangle=\langle\log \gamma,\omega^{(0)}_{1,B}(1)\rangle\label{art2-eq1.5.1}
\end{equation}

We consider the group $B_{\infty}$ as a real algebraic subgroup of $\PGL_{2}(\mathbb{C})=G(\mathbb{R})$ where $G=R_{F/\mathbb{Q}}(G_{o})$. The centralizer of $K^{B}_{\infty}$ is a real torus $T_{\infty}$ which is of dimension 2 and decomposes into a one dimensional split torus and a one dimensional anisotropic torus. Therefore we have
\begin{gather*}
T_{\infty}\xrightarrow{\sim}\mathbb{C}^{x}=\mathbb{R}^{x}\times S^{1}\\
t_{\infty}\xrightarrow{\sim}(t'_{\infty},k(t_{\infty}))
\end{gather*}
If $\omega(1)\in \sfp_{1,B}$ we construct for any complex number $s\in \mathbb{C}$ a form
$$
\omega_{s}:\Gamma_{B}\backslash B_{\infty}\to \sfp_{1,B}\otimes C
$$
by
$$
\omega_{s}(b_{\infty})=\omega_{S}(u_{\infty}t_{\infty})=|t_{\infty}|^{\frac{1}{2}+\frac{s}{2}}_{\mathbb{C}}\ad(k(t_{\infty})^{-1})\omega(1)
$$
where as before $|z|_{\mathbb{C}}=z\overline{z}$ for $z\in \mathbb{C}$.

\medskip
\noindent
{\bf Lemma \thnum{1.5.2}.\label{art2-lem1.5.2}}~{\em The\pageoriginale 1-form $\omega_{s}$ is closed if and only if $s=0$.}
\smallskip

This is an easy computation (see also \cite{art2-key8}, Lemma 3.1).

This lemma allows us to go back and forth from forms on $\Gamma_{B}\backslash X^{(1)}_{B}$ to forms on $\Gamma_{B}\backslash X_{B}$.

\subsection{The Adelic Description of the Cohomology at the Boundary}\label{art2-sec1.6}
In the last section we gave a discussion of the de Rham cohomology of an individual boundary component. Now we want to look at all the boundary components and to describe the cohomology in terms of differential forms which depend on adelic variables.

We start from our standard Borel subgroup $B_{o}$ and as in \ref{art2-sec1.5} we decompose $\sfp=\sfp_{o}\oplus \sfp_{1}=\sfp_{o,B_{o}}\oplus \sfp_{1,B_{o}}$. We define $B_{o,\infty}^{(1)}=\{b_{\infty}\in B_{o,\infty}|~|\alpha(b_{\infty})|=1\}$. We introduce the space of maps
$$
\sfH_{\infty}=
\left\{
\begin{array}{l}
\omega:U_{o}(\bA)B_{o}(F)\backslash B_{o,\infty}^{(1)}\cdot G_{o}(\bA^{f})\to \sfp_{1}\otimes \mathbb{C}|\\[4pt]
\omega(\underline{g}\underline{k})=\ad(k^{-1}_{\infty})\omega(g)\text{~ for~ }\underline{k}=(k_{\infty},\underline{k}^{f})\\[4pt]
\text{and~ } k_{\infty}\in K^{B,o}_{\infty},\underline{k}^{f}\in \sfK^{f}(\sfa)
\end{array}
\right\}
$$
We want to show that we have a natural identification
$$
\sfH_{\infty}\xrightarrow{\sim}H^{1}(\partial(\Gamma\backslash \overline{X}),\mathbb{C})
$$
To get this identification we start from a computation which is heuristical at the moment, but will also be used later.

Let us assume we have a 1-form \ref{art2-sec1.4}
$$
\omega:G_{o}(F)\backslash G_{o}(\bA)/\sfK^{f}(\sfa)\to\sfp
$$
We recall the double coset decomposition (\ref{art2-sec1.3})
$$
G_{o}(\bA^{f})=\cup B_{o}(F)\cdot \underline{\xi}\sfK^{f}(\sfa)
$$
where the double cosets are in $1-1$ correspondence to the cusps. Let us pick an element $b_{\infty}\in B_{o,\infty}$ and we compute $\omega(b_{\infty}\underline{\xi})$. As in (\ref{art2-sec1.3}) we write $\underline{\xi}'=(1,\underline{\xi})=a\cdot (a^{-1},\underline{k}^{f})$ and get
$$
\omega(b_{\infty}\underline{\xi})=\omega(\underline{b}_{\infty}\cdot a\cdot (a^{-1},1))\text{~ where}
$$
$\underline{b}_{\infty}=(b_{\infty},1,\ldots,1,\ldots)$. Then $b'_{\infty}=a^{-1}b_{\infty}a\in B_{\infty}$ where $B$ is a representative\pageoriginale for the $\Gamma$-conjugacy class of Borel subgroups corresponding to $\underline{\xi}$.

Then
$$
\omega(b_{\infty}\underline{\xi})=\omega(\underline{b}'_{\infty}\cdot (a^{-1},1)=\omega(b'_{\infty}\cdot a^{-1})
$$
where we observe that the adele $\underline{b}'_{\infty}\cdot (a^{-1},1)$ is 1 at the finite components. We write $a^{-1}=b_{a^{-1}}\cdot k_{a^{-1}}$ with $b_{a^{-1}}\in B_{\infty}$ and $k_{a^{-1}}\in K_{\infty}$ and find
$$
\omega(b_{\infty}\underline{\xi})=\ad(k_{a^{-1}}^{-1})\omega(b'_{\infty}b_{a^{-1}})
$$
We substitute $b'_{\infty}b_{a^{-1}}=b''_{\infty}$ and get
$$
\omega(b''_{\infty})=\ad(k_{a^{-1}})\cdot \omega(ab''_{\infty}b^{-1}_{a^{-1}}a^{-1}\cdot \underline{\xi})
$$

Our forms in $\sfH_{\infty}$ are not defined on all of $G_{\infty}$ but only on $B_{o,\infty}^{(1)}$. Therefore we do the following:

We write
$$
G_{o}(\bA^{f})=\bigcup\limits_{\underline{\xi}}B_{o}(F)\underline{\xi}\sfK^{f}(\sfa)
$$
and
$$
\xi'=(1,\underline{\xi})=a(a^{-1},\underline{k}_{f})
$$
and
$$
B=a^{-1}B_{o}a
$$
and
$$
a^{-1}=b_{a}\cdot k_{a}\qquad b_{a}\in B_{\infty}, \ k_{a}\in K_{\infty}
$$
then we put
\begin{gather*}
\omega^{B}:B_{\infty}\to \sfp_{1,B}\\[3pt]
\omega^{B}:b''_{\infty}\to \omega(ab''_{\infty}b^{-1}_{a^{-1}}a^{-1}\cdot \underline{\xi})
\end{gather*}
One checks that
$$
\omega^{B}(b''_{\infty}k_{\infty})=\ad(k^{-1}_{\infty})\omega^{B}(b''_{\infty})
$$
for $k_{\infty}\in B_{\infty}\cap K_{\infty}=K^{B}_{\infty}$ and that
$$
\omega^{B}(u''_{\infty}b''_{\infty})=\omega^{B}(b''_{\infty})
$$

Therefore\pageoriginale we get for any $\omega\in \sfH_{\infty}$ a collection of differential forms $\omega^{B}\in \Omega^{1}(\Gamma_{B}\backslash X^{(1)}_{B})$ which are $U_{\infty}$ invariant and represent cohomology classes of the corresponding boundary component at $\infty$. \eqref{art2-eq1.5.1}. This gives us a map
$$
\sfH_{\infty}\to \bigoplus\limits_{B}H^{1}(Y_{B},\mathbb{C})
$$
which is obviously an isomorphism and does not depend on any choice. Let us assume that we have a 1-form
$$
\omega : G(F)\backslash G(\bA)/\sfK^{f}(\sfa)\to \sfp
$$
which is closed (\ref{art2-sec1.4}). So it represents a cohomology class $[\omega]$. We know that the restriction of $[\omega]$ to the boundary is given by an element in $\sfH_{\infty}$, we want to compute that element. On the adele group $U_{o}(\bA)$ we choose a Haar measure $d\underline{u}$ so that the volume $U_{o}(F)\backslash U_{o}(\bA)$ becomes equal to 1. Then we compute
$$
\omega^{(0)}(g)=\int\limits_{U_{p}(F)\backslash U_{o}(\bA)}\omega(\underline{u}\underline{g})d\underline{u}
$$
If we restrict $\omega^{(0)}$ to $B_{o,\infty}^{(1)}\cdot G_{o}(\bA^{f})$ we can project the values to $\sfp_{1}=\sfp_{1,B_{o}}$ and get
$$
\omega^{(0)}_{1}:U_{o}(\bA)\cdot B_{o}(F)\backslash B_{o,\infty}^{(1)}G_{o}(\bA^{f})\to \sfp_{1}
$$
which is an element in $\sfH_{\infty}$.

\medskip
\noindent
{\bf Proposition \thnum{1.6.1}.\label{art2-prop1.6.1}}~{\em Under the natural identification constructed above the element $\omega^{(0)}_{1}$ corresponds to the restriction of $[\omega]$ to the boundary.}
\smallskip

This follows from \ref{art2-sec1.5} where we did the corresponding thing for the individual cusps and the computation at the beginning of this section. The normalisation of the measure corresponds exactly to the one in \ref{art2-sec1.5}. We have the decomposition \eqref{art2-eq1.2.1} for the cohomology of the boundary. For the rest of this section we want to analyse our identification
$$
\sfH_{\infty}\xrightarrow{\sim}H^{1}(\partial(\Gamma\backslash X);\mathbb{C})
$$
from the point of view of \eqref{art2-eq1.2.1}. Actually we shall very explicitely associate to any element $\psi\in M^{*}_{\phi}$ or $M^{*}_{\overline{\phi}}$ an element $\omega(\psi)$ of $\sfH_{\infty}$.

We\pageoriginale have
$$
K^{B_{o}}_{\infty}=\left\{\left(
\begin{matrix}
e^{i\theta} & 0\\
0 & 1
\end{matrix}
\right)\Big| \theta\in \mathbb{R}\mod 2\pi\right\}
$$
The group acts on $\sfp_{1}\otimes \mathbb{C}=\mathbb{C}E_{1}\oplus E_{2}$ and the vectors
\begin{align*}
& e_{+1}=E_{1}-i\otimes E_{2}\\
& e_{-1}=E_{1}+i\otimes E_{2}
\end{align*}
are eigenvectors with respect to this action:
$$
\ad \left(\begin{matrix} e^{i\theta} & 0\\ 0 & 1\end{matrix}\right)e_{+1}=e^{i\theta}\cdot e_{+1}, \ \ad\left(\begin{matrix} e^{i\theta} & 0\\ 0 & 1\end{matrix}\right)e_{-1}=e^{-i\theta}e_{-1}
$$
The two elements $e_{+1}$, $e_{-1}$ define homomorphisms from $\Gamma_{B_{o}}$ to $R$ (\ref{art2-sec1.5}) and we shall use them as canonical generators of the two modules $L_{+}$ and $L_{-}$ \eqref{art2-eq1.2.2}.

Therefore we have now established the identification
$$
M^{*}_{\phi}=M_{\phi}; \ M^{*}_{\overline{\phi}}=M_{\overline{\phi}}
$$
in \ref{art2-eq1.2.2}. Now we shall give an explicit formula for the identification maps
$$
H^{1}(\partial (\Gamma\backslash \overline{X}),\mathbb{C})\xrightarrow{\sim}\bigoplus\limits_{\substack{\phi:(\mathscr{O}/\sfa)^{x}\to S^{1}\\ \phi(i)=i}}(M_{\phi}\otimes \mathbb{C}\oplus M_{\phi}\otimes \mathbb{C})\to \sfH_{\infty}
$$
The crucial point is the following simple

\medskip
\noindent
{\bf Lemma \thnum{1.6.2}.\label{art2-lem1.6.2}}~{\em To any $\phi:(\mathscr{O}/\sfa)^{x}\to S^{1}$ which satisfies $\phi(i)=i^{\pm 1}$ there exists exactly one character}
$$
\widetilde{\phi}:I/F^{x}\sfU^{f}(\sfa)\to S^{1}
$$
{\em for which}
$$
\widetilde{\phi}|\sfU^{f}/\sfU^{f}(\sfa)=\widetilde{\phi}|(\mathscr{O}/\sfa)^{x}=\phi
$$
{\em any for $z\in \mathbb{C}^{x}$}
$$
\widetilde{\phi}((z,1,\ldots,1))=\left(\frac{z}{|z|}\right)^{\mp 1}.
$$

\begin{proof}
As in \ref{art2-sec1.3} we start from
$$
I/F^{x}\sfU^{f}(\sfa)\xrightarrow{\sim}\mathbb{C}^{x}x(\mathscr{O}/\sfa)^{x}/W
$$
Since\pageoriginale we have to have
$$
\widetilde{\phi}((i,\ldots,i,\ldots))=1
$$
and $\phi(i)=i^{\pm 1}$ we get existence and uniqueness easily.

To any of our characters $\phi:(\mathscr{O}/\sfa)^{x}\to S^{1}$ for which $\phi(i)=i^{\pm 1}$ we introduce the number $\epsilon (\phi)=\pm 1$ such that $\phi(i)=i^{\epsilon(\phi)}$.

We have
$$
U_{o}(\bA)\cdot B_{o}(F)\backslash B_{o}(\bA)\simeq T_{o}(\bA)/T_{o}(F)=I/F^{x}
$$
and therefore we may also look at $\widetilde{\phi}$ as a character
$$
\widetilde{\phi}:B_{o}(F)\backslash B_{o}(\bA)\to S^{1}
$$
which is trivial on $U_{o}(\bA)$.

To any $\psi\in M_{\phi}$ where $\phi(i)=i^{\pm 1}$ we associate an element $\omega( \ \ , \phi,\psi)\in \sfH_{\infty}$ by the formula
\begin{align*}
& \omega(b_{\infty}\underline{g}^{f},\phi,\psi)=\omega(b_{\infty}\underline{b}^{f}\underline{k}^{f},\phi,\psi)=\\
& \widetilde{\phi}(b_{\infty}\underline{b}^{f})\cdot\psi ((\underline{k}^{f})^{-1})\cdot e_{\epsilon(\phi)}
\end{align*}
where we identify $\sfK^{f}/\sfK^{f}(\sfa)=\overline{G}$. It's of course pure routine but we want to check whether this is well defined and the signs are correct.

If $b_{\infty}=\left(\begin{matrix} e^{i\theta} & 0\\ 0 & 1\end{matrix}\right)=h(\theta)$ then we should have
\begin{align*}
& \omega(h(\theta)\underline{g}^{f},\phi,\psi)=\ad(h(\theta)^{-1})\cdot \omega((1,\underline{g}^{f},\phi,\psi)=\\
& \ad(h(\theta)^{-1})\cdot \phi(\underline{b}^{f})\cdot \psi(\underline{k}^{f})^{-1})\cdot e_{\epsilon(\phi)}=\\
& \widetilde{\phi}(\underline{b}^{f})\cdot \psi(\underline{k}^{f})^{-1})\cdot e^{-\epsilon(\phi)\theta}\cdot e_{\epsilon(\phi)}
\end{align*}
and on the other hand we have
$$
\widetilde{\phi}(h(\theta)\cdot \underline{b}^{f})=e^{-\epsilon(\phi)}\cdot \widetilde{\phi}(\underline{b}^{f})
$$
so the component at infinity is ok. To prove that it is well defined we have to write
$$
\underline{g}^{f}=\underline{b}^{f}\cdot \underline{k}^{f}=\underline{b}^{f}\cdot \underline{b}^{f}_{1}\cdot (\underline{b}^{f}_{1})^{-1}\cdot \underline{k}^{f}
$$
and\pageoriginale get from the finite places
\begin{align*}
& \widetilde{\phi}(\underline{b}^{f}\underline{b}^{f}_{1})\cdot \psi(((\underline{b}^{f}_{1})^{-1}\underline{k}^{f})^{-1})=\\
& \widetilde{\phi}(\underline{b}^{f})\cdot \widetilde{\phi}(\underline{b}^{f}_{1})\cdot \psi((\underline{k}^{f})^{-1}\cdot \underline{b}^{f}_{1})=\\
& \widetilde{\phi}(\underline{b}^{f})\cdot \phi(\underline{b}^{f}_{1})\cdot \widetilde{\phi}(\underline{b}^{F}_{1})^{-1}\cdot \psi((\underline{k}^{f})^{-1})=\\
&\widetilde{\phi}(\underline{b}^{f})\cdot \psi((\underline{k}^{f})^{-1}).
\end{align*}
and this proves that $\omega(\quad,\phi,\psi)$ is well defined.

The map
$$
\bigoplus\limits_{\phi(i)=i}(M_{\phi}\oplus M_{\overline{\phi}})\otimes \mathbb{C}\to \sfH_{\infty}
$$
which maps $\psi\in M_{\phi}$ and $\psi'\in M_{\overline{\phi}}$ to $\omega(\quad,\phi,\psi)$ and $\omega(\quad,\overline{\phi},\psi')$ is equal to the identification between $H^{1}(\partial(\Gamma\backslash\overline{X}),\mathbb{C})$ and $\sfH_{\infty}$ if we take \eqref{art2-eq1.2.1} and \eqref{art2-eq1.2.2} into account.

One remark concerning the notation. The $\psi$ is always an element in $M_{\phi}$ where $\phi(i)=\pm i$ so $\phi$ is determined by $\psi$. But I think it is better always to keep in mind from which space the $\psi$ has been taken, so therefore we keep the $\phi$ in the notation.
\end{proof}

\section{The Eisenstein Series}\label{art2-sec2}

We start from a cohomology class at infinity. We have the identifications \ref{art2-eq1.2.1} and \ref{art2-eq1.2.2} and we have seen how to associate to a class $\psi\in M_{\phi}$ a map
$$
\omega(\quad,\phi,\psi):U_{o}(\bA)\cdot B_{o}(F)\backslash B_{o,\infty}^{(1)}G_{o}(\bA^{f})\to \mathbb{C} e_{\epsilon(\phi)}\subset \sfp_{1}\otimes \mathbb{C}
$$

We extend this to a map from $G(\bA)$ to $\sfp\otimes \mathbb{C}$. To get this extension we choose a complex number $s\in \mathbb{C}$. We have seen that $B_{o,\infty}=B_{o,\infty}^{(1)}\cdot \mathbb{R}^{x}$ (\ref{art2-sec1.5} and \ref{art2-lem1.5.2}) and $G_{\infty}=B_{o,\infty}K_{\infty}$. We write an element $g_{\infty}\in G_{\infty}$ as $g_{\infty}=b_{\infty}\cdot t_{\infty}\cdot k_{\infty}$ where $t_{\infty}\in (\mathbb{R}_{+})^{x}$, $b_{\infty}\in B^{(1)}_{\infty}$ and $k_{\infty}\in K_{\infty}$ and put
\begin{gather*}
\omega_{s}((g_{\infty},\underline{g}^{f}),\phi,\psi)=\omega_{s}((b_{\infty}t_{\infty}k_{\infty},\underline{g}^{f}),\phi,\psi)=\\
|t_{\infty}|_{\mathbb{C}}^{\frac{1}{2}+\frac{s}{2}}\cdot \ad(k_{\infty}^{-1})\cdot \omega((b_{\infty},\underline{g}^{f}),\phi,\psi)
\end{gather*}
Now we are in the position to define the Eisenstein series. For $\rRe(s)>1$ the series
$$
E(\underline{g},\phi,\psi,s)=\sum\limits_{a\in B_{o}(F)\backslash G_{o}(F)}\omega_{s}(a\underline{g},\phi,\psi)
$$
is\pageoriginale absolutely and locally uniformly convergent. Moreover it is known that our series has a meromorphic continuation into the entire $s$-plane (\cite{art2-key9}, Thm. 7., \cite{art2-key13}, Chap. 6). We can interpret $E(\underline{g},\phi,\psi,s)$ as a 1-form on $\Gamma\backslash X(1.4)$ and this 1-form is closed for $s=0$. (\cite{art2-key8}, 4.3). It is well known that $E(\underline{g},\phi,\psi)$ is holomorphic at $s=0$.

If we want to know the restriction of the Eisenstein class $[E(g,\phi,\psi,0)]$ to the boundary we have to compute the constant term (Prop. \ref{art2-prop1.6.1}).
$$
\int\limits_{U_{o}(F)\backslash U_{o}(A)}E(\underline{u}\underline{g},\phi,\psi,0)d\underline{u}=E^{(0)}(\underline{g},\phi,\psi,0)
$$
We do this by analytic continuation and compute for $\underline{g}=(b_{\infty},\underline{g}^{f})$ with $b_{\infty}\in B_{\infty}$ and $\rRe(s)>1$
$$
\int\limits_{U_{o}(F)\backslash U_{o}(\bA)}E(\underline{u}\underline{g},\phi,\psi,s)d\underline{u}
$$

This computation has been carried out at several places (\cite{art2-key6}, 1.6., \cite{art2-key11}, 6, and \cite{art2-key13}). So we recall only the main steps in the computation. We start from the Bruhat decomposition $G_{0}(F)=B_{0}(F)\cup B_{0}(F)\left(\begin{smallmatrix} 0 & 1\\ -1 & 0\end{smallmatrix}\right)U_{0}(F)$ and substitute the definition of the Eisenstein series into the integral. Then we get two terms
$$
\int\limits_{U_{o}(F)\backslash U_{o}(\bA)}\omega_{s}(\underline{u}\underline{g},\phi,\psi)d\underline{u}+\int\limits_{U_{o}(\bA)}\omega(w\underline{u}\underline{g},\phi,\psi)d\underline{u}
$$
where $w=\left(\begin{smallmatrix} 0 & 1\\ -1 & 0\end{smallmatrix}\right)$. The first integral is constant and therefore we find
$$
\omega_{s}(\underline{g},\phi,\psi)+\int\limits_{U_{o}(\bA)}\omega_{s}(w\underline{u}\underline{g},\phi,\psi)d\underline{u}
$$
We have a map $B_{o}(\bA)\xrightarrow{\alpha}I$ defined by the positive root and for $b\in \underline{B}_{o}(\bA)$ we define $|\underline{b}|=|\alpha(\underline{b})|$ where $|\underline{x}|=$ idelenorm of $\underline{x}\in I$. We write $\underline{g}=$
\begin{align*}
& (b_{\infty},\underline{g}^{f})=(b_{\infty},\underline{b}^{f})\cdot (1,\underline{k}^{f})=\underline{b}\cdot \underline{k} \text{~~ and get}\\
& \omega_{s}(\underline{g},\phi,\psi)=|\underline{b}|_{\mathbb{C}}^{\frac{1}{2}+\frac{s}{2}}\cdot \widetilde{\phi}(\underline{b})\cdot \psi(\underline{k}^{f})^{-1})\cdot e_{\epsilon(\phi)}
\end{align*}
and
\begin{gather*}
\int\limits_{U_{o}(\bA)}\omega_{s}(w\underline{u}\underline{b}\underline{k},\phi,\psi)du\\[4pt]
=|b|_{\mathbb{C}}^{\frac{1}{2}-\frac{s}{2}}\widetilde{\phi}(\underline{b})^{-1}\int\limits_{U_{o}(\bA)}\omega_{s}(\underline{u}\underline{k},\phi,\psi)d\underline{u}
\end{gather*}

The\pageoriginale functions $\omega_{S}(\underline{g},\phi,\psi)$ are product of local functions
$$
\omega_{s}(g,\phi,\psi)=\omega^{(\infty)}_{s}(g_{\infty},\phi,\psi)\prod\limits_{\sfp\text{~finite}}\omega_{s}^{(\sfp)}(g_{\sfp},\phi,\psi)
$$
This is so since they are defined by $\widetilde{\phi}$ and $\psi$ which are both products of local functions
\begin{align*}
& \psi(\underline{k}^{f})=\psi(k_{\sfp_{o}})\text{~~ where~~ }\{\sfp_{o}\}=\text{supp~}(\sfa)\\[3pt]
& \widetilde{\phi}(\underline{x})=\widetilde{\phi}_{\infty}(x_{\infty})\cdot \prod\limits_{\sfp}\widetilde{\phi}_{\sfp}(x_{\sfp}).
\end{align*}
We have for $\sfp\nmid \sfa$
$$
\omega^{(\sfp)}_{s}(g_{\sfp},\phi,\psi)=\omega_{s}^{(\sfp)}(b_{\sfp}k_{\sfp},\phi,\psi)=\widetilde{\phi}_{\sfp}(b_{\sfp})|b_{\sfp}|_{\sfp}^{\frac{1}{2}+\frac{s}{2}}
$$
for $\sfp|\sfa$
$$
\omega_{s}^{(\sfp)}(g_{\sfp},\phi,\psi)=\widetilde{\phi}_{\sfp}(b_{\sfp})|b_{\sfp}|_{\sfp}^{\frac{1}{2}+\frac{s}{2}}\cdot \psi(k^{-1}_{\sfp})
$$
and
\begin{align*}
& \omega_{s}^{(\infty)}(g_{\infty},\phi,\psi)=\omega_{s}^{(\infty)}(b_{\infty}\cdot t_{\infty}k_{\infty},\phi,\psi)=\\[3pt]
& |t_{\infty}|^{\frac{1}{2}+\frac{s}{2}}_{\mathbb{C}}\cdot \widetilde{\phi}_{\infty}(b_{\infty})\cdot \ad(k^{-1}_{\infty})e_{\epsilon(\phi)}
\end{align*}
Therefore the integral decomposes into a product of local integrals. We have to write the measure as a product of local measures and we are in the fortunate case that we can take
$$
d\underline{u}=d\underline{u}_{\infty}\prod\limits_{\sfp}du_{\sfp}
$$
where $\vol_{du_{\sfp}}(\mathscr{O}_{\sfp})=1$ for all $\sfp$ and $du_{\infty}=\dx \ \dy$ (Actually there should be a $\frac{1}{2}$ at $(1+i)$ and a 2 at infinity but they cancel).

For those $\sfp$ which do not divide $\sfa$ (these are all except one) we find
$$
\int\limits_{U_{o}(F_{\sfp})}\omega^{(\sfp)}_{s}(wu_{\sfp},\phi,\psi)du_{\sfp}=\frac{1-\widetilde{\phi}_{\sfp}(\pi_{\sfp})^{2}|\pi_{\sfp}|^{1+s}_{\sfp}}{1-\widetilde{\phi}_{\sfp}(\pi_{\sfp})^{2}|\pi_{\sfp}|^{s}_{\sfp}}
$$
and this follows from a standard computation (\cite{art2-key11}, \S6).

What happens at $\sfp_{o}$ where $\sfp_{o}$ is the prime dividing $\sfa$? In this case we note\pageoriginale that $U_{o}(F_{\sfp_{o}})=F_{\sfp_{o}}$ and our integral is a sum
\begin{gather*}
\int\limits_{\mathscr{O}_{\sfp_{o}}}\omega_{s}^{(\sfp_{o})}(wu_{\sfp_{o}},k_{\sfp_{o}}\phi,\psi)du_{\sfp_{o}}+\\[4pt]
\sum\limits^{\infty}_{n=1}\int\limits_{\mathscr{O}^{x}_{\sfp_{o}}}\omega^{(\sfp_{o})}_{s}\left(w\left(\begin{matrix} 1 & \pi^{-n}_{\sfp_{o}}  \epsilon_{\sfp_{o}}\\
0 & 1
\end{matrix}\right)k_{\sfp_{o}},\phi,\psi\right)d\epsilon_{\sfp_{o}}
\end{gather*}
and it is for $n>0$
$$
\left(\begin{matrix}
0 & 1\\
-1 & 0
\end{matrix}\right)\left(
\begin{matrix}
1 & \pi^{-n}_{\sfp_{o}}\epsilon_{\sfp_{o}}\\
0 & 1
\end{matrix}
\right)k_{\sfp_{o}}
=\left(
\begin{matrix}
\pi^{n}_{\sfp_{o}}\epsilon^{-1}_{\sfp_{o}} & 1\\
0 & \pi^{-n}_{\sfp_{o}}\epsilon_{\sfp_{o}}
\end{matrix}
\right)
\left(
\begin{matrix}
-1 & 0\\
\pi_{\sfp_{o}}\epsilon^{-1}_{\sfp_{o}} & -1
\end{matrix}\right)k_{\sfp_{o}}
$$
We substitute this into the integrals of the infinite sum. We get that each integral in the infinite sum has value zero since $\phi^{2}$ is not a trivial character. We have only the first term and get
\begin{gather*}
\int\limits_{\mathscr{O}_{\sfp_{o}}}\omega_{s}^{(\sfp_{o})}(w,u_{\sfp_{o}}k_{\sfp_{o}},\phi,\psi)du_{\sfp_{o}}=\\
\frac{1}{N(\sfa)}\sum\limits_{\overline{u}\in \mathscr{O}_{\sfp_{o}}/\sfp_{o}}\psi(k^{-1}_{\sfp_{o}}\overline{u}^{-1}w)=\frac{1}{N(\sfa)}T_{\phi}\psi(k^{-1}_{\sfp_{o}})
\end{gather*}
where $T_{\phi}$ is the intertwining operator constructed in \eqref{art2-eq1.2.2}.

At the infinite place we have to compute
$$
\int\limits_{\mathbb{C}}\omega_{s}^{(\infty)}\left(w\left(\begin{matrix} 1 & z\\ 0 & 1\end{matrix}\right)\right),\phi,\psi)\dx \dy
$$
where $z=x+iy$. We introduce polar coordinates and get
$$
\int\limits^{\infty}_{0}\int\limits^{2w}_{0}\omega_{s}^{(\infty)}\left(w\left(\begin{matrix} 1x & e^{i\theta}\\ 0 & 1\end{matrix}\right),\phi,\psi\right)x\dx \text{d}\theta
$$
and we have
$$
\left(
\begin{matrix}
1 & xe^{i\theta}\\ 0 & 1
\end{matrix}
\right)
=
\left(
\begin{matrix}
e^{1\theta} & 0\\
0 & 1
\end{matrix}
\right)
\cdot
\left(
\begin{matrix}
1 & x\\ 0 & 1
\end{matrix}
\right)
\left(
\begin{matrix}
e^{-i\theta} & 0\\
0 & 1
\end{matrix}
\right)
$$
This gives us
$$
\int\limits^{\infty}_{0}\int\limits^{2\pi}_{0}e^{+\epsilon(\phi)i\theta}\cdot \ad
\left(\left(
\begin{matrix}
e^{i\theta} & 0\\
0 & 1
\end{matrix}
\right)\right)
\omega_{s}^{(\infty)}\left(w\cdot
\left(
\begin{matrix}
1 & x\\ 0 & 1
\end{matrix}
\right),\phi,\psi\right)
x\dx d\theta
$$
Let\pageoriginale us write
$$
\omega^{(\infty)}_{s}\left(w
\left(
\begin{matrix}
1 & x\\ 0 & 1
\end{matrix}
\right)\phi,\psi\right)=A(x)\cdot e_{\epsilon(\phi)}+B(x)\cdot H+C(x)\cdot e_{-\epsilon(\phi)}
$$
Integrating the first two terms over $\theta$ we find zero, so we are left with
\begin{gather*}
\int\limits^{\infty}_{0}\int\limits^{2\pi}_{0}\omega^{(\infty)}_{s}\left(w
\left(
\begin{matrix}
1 & x\cdot e^{i\theta}\\ 0 & 1
\end{matrix}
\right)\right),\phi,\psi,x\dx d\theta=\\
2\pi\left(\int\limits^{\infty}_{0}|b(x)|^{\frac{1}{2}+\frac{s}{2}}_{\mathbb{C}}\cdot C(x)x\dx\right)e_{-\epsilon(\phi)}
\end{gather*}
We have to start from the Iwasawa decomposition
$$
w
\left(
\begin{matrix}
1 & x\\
0 & 1
\end{matrix}
\right)
=
\left(
\begin{matrix}
(1+x^{2})^{-1/2} & x\\
0 & (1+x^{2})^{1/2}
\end{matrix}
\right)
\cdot
\left(
\begin{matrix}
-x(1+x^{2})^{-1/2} &-(1-x^{2})^{-1/2}\\
(1+x^{2})^{-1/2} & -x(1+x^{2})^{-1/2}
\end{matrix}
\right)
$$
Then $b(x)=(1+x^{2})^{-1}$ and a simple computation using \eqref{art2-eq1.4.1} yields $C(x)=-(1+x^{2})^{-1}$.

We get for our integral
$$
\left(-2\pi \int\limits^{\infty}_{0}(1+x^{2})^{-1-s}(1+x^{2})^{-1}x\dx\right)e_{-\epsilon(\phi)}=-\frac{\pi}{s+1}e_{-\epsilon(\phi)}
$$
Multiplying all this together we find for $\rRe(s)>1$ and $g=(b_{\infty},\underline{g}^{f})$
\begin{gather*}
\int\limits_{U_{o}(F)\backslash U_{o}(\bA)}E(\underline{u}\underline{g},\phi,\psi,s)d\underline{u}=\\
\omega_{s}(\underline{g},\phi,\psi)-\frac{\pi}{s+1}\cdot \frac{L(\widetilde{\phi}^{2},s)}{L(\widetilde{\phi}^{2},s+1)}\cdot \omega_{-s}(\underline{g},\overline{\phi},T_{\phi}\psi)
\end{gather*}
where the $L$-function is defined as
$$
L(\widetilde{\phi}^{2},s)=\prod\limits_{\sfp\neq \sfp_{o}}(1-\widetilde{\phi}^{2}_{\sfp}(\pi_{\sfp})|\pi_{\sfp}|^{+s})^{-1}
$$
(\cite{art2-key12}, X/V, \S8) Since both sides have meromorphic continuation into the entire $s$-plane we find that the equality holds for all $s$.

Before stating our main result we look at the expression
$$
\left.-\frac{\pi}{s+1}\frac{L(\widetilde{\phi}^{2},s)}{L(\widetilde{\phi}^{2},s+1)}\right|_{s=0}
$$
a little bit more closely. The first crucial fact is that $L(\widetilde{\phi}^{2},1)\neq 0$ (\cite{art2-key12}, XV, \S4).

So\pageoriginale we have to compute
$$
-\pi\frac{L(\widetilde{\phi}^{2},0)}{L(\widetilde{\phi}^{2},1)}
$$
Now we exploit the functional equation. Let us assume that $\sfa=\sfp_{o}$ is an odd prime and $N(\sfp_{o})=p$. If we follow the instructions in \cite{art2-key12}, p. 299 carefully we find
$$
L(\widetilde{\phi}^{2},0)=+\overline{W(\widetilde{\phi}^{2})}\sqrt{p}\cdot \pi^{-1}L(\widetilde{\phi}^{2},1)
$$
and therefore
$$
-\pi \frac{L(\widetilde{\phi}^{2},0)}{L(\widetilde{\phi}^{2},1)}=-\overline{W(\widetilde{\phi}^{2})}\cdot \sqrt{p}\frac{L(\overline{\widetilde{\phi}}^{2},1)}{L(\widetilde{\phi}^{2},1)}
$$
We apply the formula for the number $W(\widetilde{\phi}^{2})$ given in \cite{art2-key12}, p. 300 and get $W(\widetilde{\phi}^{2})=i^{2}\cdot \tau(\widetilde{\phi}^{2})\dfrac{1}{\sqrt{p}}\cdot \widetilde{\phi}^{2}(\sfD^{-1}_{(1+i)})$ where $\tau(\widetilde{\phi}^{2})$ is a Gaussain sum.

(\cite{art2-key12}, XIV, \S4). Now $i^{2}=-1$ and $\widetilde{\phi}^{2}(\sfD^{-1}_{(1)})=\widetilde{\phi}^{2}((1,\frac{i}{2},1,\ldots,))$ where the $i/2$ stands at the $(1+i)$th componente of the idele. Then this is
$$
\widetilde{\phi}^{2}((-2i, 1, -2i,\ldots))=(-1)\cdot \phi^{2}(-2i)
$$
where the last $-2i$ is the residue class of $-2i$ in $\mathscr{O}/\sfp_{o}$. Therefore we find
$$
-\pi \frac{L(\overline{\widetilde{\phi}}^{2},0)}{L(\widetilde{\phi}^{2},1)}=\overline{-\tau(\widetilde{\phi}^{2})}\cdot \overline{\phi^{2}(-2i)} \frac{L(\overline{\widetilde{\phi}}^{2},1)}{L(\widetilde{\phi}^{2},1)}
$$
If we have $\sfa=(1+i)^{3}$ we find
\begin{align*}
& -\pi \frac{L(\overline{\widetilde{\phi}}^{2},0)}{L(\widetilde{\phi}^{2},1)}=\overline{-W(\widetilde{\phi}^{2})\cdot 2}\frac{L(\overline{\widetilde{\phi}}^{2},1)}{L(\widetilde{\phi}^{2},1)}=\\[4pt]
& \overline{\tau(\widetilde{\phi}^{2})}\frac{L(\overline{\widetilde{\phi}}^{2},1)}{L(\widetilde{\phi}^{2},1)}=-2\frac{L(\overline{\widetilde{\phi}}^{2},1)}{L(\widetilde{\phi}^{2},1)}
\end{align*}
Now we can state the first main theorem of the paper. In the statement we refer to the different identifications made before.

\medskip
\noindent
{\bf Theorem \thnum{2.1}.\label{art2-thm2.1}}~{\em For\pageoriginale $\psi\in M_{\phi}\otimes \mathbb{C}$ the Eisenstein series $E(\underline{g},\phi,\psi,0)$ is a closed $1$-form and the cohomology class $[E(g,\phi,\psi,0)]$ restricted to the boundary is equal to}
$$
[E(\underline{g},\phi,\psi,0)]_{\partial(\Gamma\backslash\overline{X})}=\psi-\overline{\widetilde{\phi}^{2}(-2i)}\overline{\tau(\widetilde{\phi}^{2})}\cdot \frac{L(\overline{\widetilde{\phi}}^{2},1)}{L(\widetilde{\phi}^{2},1)}\cdot \frac{1}{p}T_{\phi}\psi
$$
{\em if $\sfa=\sfp_{o}$ is prime and $N(\sfp_{o})=p$ and equal to}
$$
\psi-\frac{L(\overline{\widetilde{\phi}}^{2},1)}{L(\widetilde{\phi}^{2},1)4}\frac{1}{4}T_{\phi}\psi
$$
{\em if $\sfa=(1+i)^{3}$.}

\setcounter{subsection}{1}
\subsection{Arithmetic Applications}\label{art2-sec2.2}
In this section we assume that $\sfa=\sfp_{o}$ is an odd prime. The theorem gives us the value of the number $c_{\phi}$ in \ref{art2-sec1.2.2}, we get with $p=N(\sfp_{o})$
\setcounter{equation}{0}
\begin{equation}
c_{\phi}=-\frac{\overline{\phi^{2}(-2i)\tau(\widetilde{\phi}^{2})}}{p}\cdot \frac{L(\overline{\widetilde{\phi}}^{2},1)}{L(\widetilde{\phi}^{2},1)}\label{art2-eq2.2.1}
\end{equation}

\medskip
\noindent
{\bf Corollary \thnum{2.2.1}.\label{art2-coro2.2.1}}~{\em We have}
$$
|c_{\phi}|=\frac{1}{\sqrt{p}}
$$
{\em and in particular $c_{\phi}\neq 0$, $\infty$.}
\smallskip

This is a consequence of the properties of the Gaussian sums.

To give another interpretation of the Corollary \ref{art2-coro2.2.1} we recall that we have a scalar product on $M_{\phi}$
$$
\langle \psi,\psi\rangle = \int\limits_{\overline{G}/\overline{B}_{o}}\psi(\overline{g})\overline{\psi(\overline{g})}
$$
and the norm of the operator $T_{\phi}$ is obviously $\sqrt{p}$. So the $\sqrt{p}$ cancels and we find that the Corollary says that
$$
c_{\phi}T_{\phi}:M_{\phi}\otimes \mathbb{C}\to M_{\overline{\phi}}\otimes \mathbb{C}
$$
is an unitary operator.

I\pageoriginale was unable to see this form a topological point of view and shall come back to this kind of questions later\footnote[1]{Added in Proof: This is actually very easy to see.}.

But we can also reverse the argument. We had the identifications \eqref{art2-sec1.2.1}
$$
H^{1}(\partial(\Gamma\backslash \overline{X}),R)\xrightarrow{\sim}\bigoplus\limits_{\substack{\phi\\ \phi(i)=i}}(M^{*}_{\phi}\oplus M^{*}_{\phi})\xrightarrow{\sim}\bigoplus\limits_{\substack{\phi\\ \phi(i)=i}}(M_{\phi}\oplus M_{\phi})
$$
where the last identification has been by means of the elements $e_{+1}$, $e_{-1}\in \sfp \otimes \mathbb{C}$ (see \ref{art2-sec1.2.2} and \ref{art2-sec1.6}). Of course we must have $c_{\phi}\in R\bigotimes\limits_{Z}\mathbb{Q}$ and therefore we get the information that
$$
\overline{\tau(\widetilde{\phi}^{2})}\frac{L(\overline{\widetilde{\phi}}^{2},1)}{L(\widetilde{\phi}^{2},1)}\in R\otimes \mathbb{Q}
$$
But we can do better. The cohomology
$$
H^{1}(\partial (\Gamma\backslash\overline{X}),R)=H^{1}(\partial(\Gamma\backslash\overline{X}),Z)\otimes R
$$
and we have an action of the Galois group Gal $(K/\mathbb{Q})$ on the cohomology where $K$ is the filed of fractions of $R$. But this galois group is also acting on 
$$
\bigoplus\limits_{\substack{\phi\\ \phi(i)=i}}(M^{*}_{\phi}+M^{*}_{\overline{\phi}})
$$
in an obvious way and the action is compatible with the identification, moreover we see that $e_{+1}$ and $e_{-1}$ are both defined over $\mathbb{Q}(i)$ and the complex conjugation interchanges these two homomorphisms. Therefore we can say that also the last identification is compatible with the action of the galois group. The galois group Gal $(K/\mathbb{Q})$ acts on our character $\phi$ simply by acting on the values
$$
\phi^{\sigma}(x)=\phi(x)^{\sigma}
$$
and $\sigma\in \Gal(K/\mathbb{Q})$ maps $M_{\phi}$ into $M_{\phi}\sigma$. It is clear that $T^{\sigma}_{\phi}=T_{\phi\sigma}$ and all this tells us

\medskip
\noindent
{\bf Corollary \thnum{2.2.3}.\label{art2-coro2.2.3}}~{\em We have}
$$
\phi^{2}(-2i)\overline{\tau(\widetilde{\phi}^{2})}\cdot \frac{L(\overline{\widetilde{\phi}}^{2},1)}{L(\phi^{2},1)}\in R\otimes \mathbb{Q}
$$
{\em and if $\sigma\in \Gal(K/Q)$ and $\phi_{1}=\phi^{\sigma}$ then}
$$
\left(\overline{\phi^{2}(-2i)\tau(\widetilde{\phi}^{2})}\dfrac{L(\overline{\widetilde{\phi}}^{2},1)}{L(\widetilde{\phi}^{2},1)}\right)^{\sigma}=\overline{\phi^{2}_{1}(-2i)\tau(\phi^{2}_{1})}\cdot \dfrac{L(\overline{\widetilde{\phi}}^{2}_{1}),1}{L(\widetilde{\phi}^{2}_{1},1)}
$$

This\pageoriginale follows of course from the observation that the image
$$
H^{1}(\Gamma\backslash \overline{X},R)\to H^{1}(\partial(\Gamma\backslash \overline{X}),R)
$$
has to be invariant under the action of the galois group.

This corollary is related to results of Damerell, Shimura and Razar. Damerell's result is to some extent much stronger since it says that 
$$
L(\widetilde{\phi}^{2},1)=\omega^{2}\cdot \alpha
$$
where $\omega=\int\limits^{1}_{0}\dfrac{dx}{\sqrt{x-x^{3}}}$ and where $\alpha$ is an algebraic number whose denominator can be bounded in terms of our data (\cite{art2-key4}, II, Thm. 2). But on the other hand it seems to be so that our information concerning the ratio $L(\overline{\widetilde{\phi}}^{2},1)/L(\widetilde{\phi}^{2},1)$ is much more precise and I do not know whether this can be deduced from his methods. There is also a certain relation to the results of Shimura and Razar. Shimura considers Dirichlet $L$-series corresponding to modular forms (\cite{art2-key23})
$$
\sum\limits^{\infty}_{n=1}a_{n}n^{-s}=L(f,s)
$$
and twists them by Dirichlet characters
$$
D(f,\phi,s)=\sum a_{n}\phi(n)n^{-s}, \ \sum a_{n}\psi(n)n^{-s}=D(f,\psi,s)
$$
Then he is able to say something about the values
$$
\dfrac{D(f,\phi,s)}{D(f,\psi,s)}
$$
at special values of $s$ and then his results becomes very similar to ours. But I do not see whether his result implies Corollary \ref{art2-coro2.2.2} or whether it can be obtained from his methods.

To conclude this section I want to discuss a few examples very explicitely. We start from the following general remark: The cohomology $H^{1}(\Gamma\backslash X,R)=H^{1}(\Gamma_{o},R[G])$ can be computed in principle in an effective way once our data-this means $\sfa$-are given. This will be discussed in the thesis of E. Mendoza (\cite{art2-key15}). This means we are also able to compute\pageoriginale the number $c_{\phi}$ in a given case and this gives an effective way of computing the ratios $L(\overline{\widetilde{\phi}}^{2},1)/L(\phi^{2},1)$. (Comp. also \cite{art2-key23}, Intr.). We want to discuss this computation in a couple of cases where we chose a slightly different method than the one suggested by \cite{art2-key15}. We compute $H^{1}(\Gamma_{o},N_{\phi})$\eqref{art2-sec1.2.3} by starting from the cochain complex. We look at
\[
\xymatrix{
 & N_{\phi}\ar@{=}[d] & & &\\
0\ar[r] & C^{o}(\Gamma_{o},N_{\phi})\ar@{=}[d]\ar[r] & C^{1}(\Gamma_{o},N_{\phi})\ar[r]\ar[d] &\\
0\ar[r] & C^{o}(\Gamma_{o,B_{o}},N_{\phi})\ar[r] & C^{1}(\Gamma_{o,B_{o}},N_{\phi})\ar[r] &
}
\]
We computed $H^{1}(\Gamma_{o,B_{o}},N_{\phi})=\Hom(\Gamma_{o,U_{o}},N^{\overline{U}_{o}}_{\phi})^{W}=\Hom(\Gamma_{o,U_{o}},\Rf_{o}\oplus \Rf_{\infty})^{W}$. Let $\Phi\in \Hom(\Gamma_{0},B_{0},N_{\phi})$ and let
\begin{align*}
& A=\left(\begin{matrix} 1 & 1\\ 0 & 1\end{matrix}\right), \ C=\left(\begin{matrix} i & 0\\ 0 & 1\end{matrix}\right), \ B=\left(\begin{matrix} 0 & 1\\ -1 & 0\end{matrix}\right)\\[4pt]
& \Phi : \left(\begin{matrix} 1 & 1\\ 0 & 1\end{matrix}\right)=A\to af_{o}+bf_{\infty}\\[4pt]
& \Phi : \left(\begin{matrix} 1 & i\\ 0 & 1\end{matrix}\right)=CAC^{-1}\to cf_{o}+df_{\infty}.
\end{align*}
Such a $\Phi$ is invariant under $W$ if and only if $c=ia$ and $d=-ib$. This means that a cohomology class in $H^{1}(\Gamma_{o,B_{o}},N_{\phi})$ is canonically represented by a cocycle
\begin{align*}
& \Phi : A\to af_{o}+bf_{\infty}\\
& \Phi : C\to 0\\
& \Phi : \left(\begin{matrix} 1 & i\\ 0 & 1\end{matrix}\right)\to +iaf_{o}\to ibf_{\infty}
\end{align*}
In \ref{art2-sec1.6} we introduce $e_{+1}$, $e_{-1}\in \sfp_{1}\otimes \mathbb{C}$ and they define homomorphisms (\ref{art2-sec1.5})
\begin{align*}
& e_{+1} : 
\begin{cases}
\left(\begin{matrix} 1 & 1\\ 0 & 1\end{matrix}\right)\to \frac{1}{2}\\
\left(\begin{matrix} 1 & i\\ 0 & 1\end{matrix}\right)\to -i/2
\end{cases}\\
& e_{-1}\begin{cases}
\left(\begin{matrix} 1 & 1\\ 0 & 1\end{matrix}\right)\to \frac{1}{2}\\
\left(\begin{matrix} 1 & i\\ 0 & 1\end{matrix}\right)\to i/2
\end{cases}
\end{align*}
Therefore\pageoriginale we have in the notations of \ref{art2-sec1.2.3}, if we put $e_{+}=e_{+1}$ and $e_{-}=e_{-1}$ (what we did all the time) that
$$
\Phi=2ae_{+}+2be_{-}
$$
and $c_{\phi}=ba^{-1}\cdot N(\sfp_{o})^{-1}$ and our problem to compute $c_{\phi}$ amounts to: When can we extend the cocycle
$$
\Phi:A\to af_{o}+bf_{\infty};\Phi:C\to 0
$$
to a cocycle on $\Gamma_{o}$ with values im $N_{\phi}$? The only thing we have to do is we have to give the value $\Phi(B)\in N_{\phi}$. But we have certain restrictions for this value. These restrictions come from the relations
$$
B^{2}=1\quad BC-C^{-1}B\quad (AB)^{3}=1
$$
which imply
\begin{align*}
& \Phi(B)=C\Phi(B),\quad \Phi(B)+B\Phi(B)=0\text{~~ and}\\
& \Phi(AB)+AB\Phi(AB)+(AB)^{2}\Phi(AB)=0
\end{align*}
(These are not all relations, but they are sufficient in our special cases)

We stick to the case $\sfa=\sfp_{o}$ is an odd prime and introduce a basis in $N_{\phi}$. The basis consists of the functions $\delta_{\overline{u}}$ where $\overline{u}\in \mathscr{O}/\sfp_{o}=F_{p}$ and $\delta_{\infty}$ and
\begin{align*}
& \delta_{\overline{u}}:
\begin{cases}
w\cdot \left(\begin{matrix} 1 & \overline{u}\\ 0 & 1\end{matrix}\right) & \to 1\\
w\cdot \left(\begin{matrix} 1 & \overline{\nu}\\ 0 & 1\end{matrix}\right) & \to 0\qquad \text{for~ } \overline{\nu}\neq \overline{u}\\
\quad~ \left(\begin{matrix} 1 & 0\\ 0 & 1\end{matrix}\right) & \to 0
\end{cases}\\
&\delta_{\infty}: 
\begin{cases}
w \left(\begin{matrix} 1 & \overline{u}\\ 0 & 1\end{matrix}\right) & \to 0\\
\quad \left(\begin{matrix} 1 & 0\\ 0 & 1\end{matrix}\right) & \to 1
\end{cases}
\end{align*}
The group acts as follows
\begin{align*}
& A\delta_{\overline{u}}=\delta_{\overline{u}-1}, \ A\delta_{\infty}=\delta_{\infty};\\[3pt]
& B\delta_{o}=\delta_{\infty}, B\delta_{\infty}=\delta_{o}, \ B\delta_{\overline{u}}=\phi(\overline{u}^{2})\delta_{-\overline{u}^{-1}}\quad (u\neq 0),\\
& C\delta_{\overline{u}}=\phi(i)^{-1}\cdot \delta_{i\overline{u}}, \ C\delta_{\infty}=\phi(i)\delta_{\infty}
\end{align*}
The cocycles we are looking for are
\begin{align*}
&\Phi :A\to a\left(\sum\limits_{\overline{u}\in F_{p}}\delta_{\overline{u}}\right)+b\delta_{\infty}\\
&\Phi : C\to 0\\
&\Phi : B\to ?
\end{align*}

We\pageoriginale
%page 0079.JPG
