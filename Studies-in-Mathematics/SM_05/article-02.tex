\chapter{PERIOD INTEGRALS OF COHOMOLOGY CLASSES WHICH ARE REPRESENTED BY EISENSTEIN SERIES}

\begin{center}
{\large By~ G. Harder}
\end{center}

\bigskip

\setcounter{pageoriginal}{40}
\section*{Introduction}\pageoriginale
Our starting point is a very general question. Let $\Gamma$ be an arithmetic subgroup of a reductive Lie group $G_{\infty}$. Then the group $\Gamma$ acts on the symmetric space $X=G_{\infty}/K_{\infty}$ where $K_{\infty}\subset G_{\infty}$ is a maximal compact subgroup. Since $X$ is contractible one knows that the rational cohomology and homology groups of $\Gamma$ are isomorphic to the (co) homology groups of the quotient $\Gamma\backslash X$, i.e.
$$
H^{\nu}(\Gamma,\mathbb{Q})\simeq H^{\nu}(\Gamma\backslash X,\mathbb{Q})
$$
(Comp. \cite{art2-key21}, 1.6.).

In general the quotient space $\Gamma\backslash X$ is not compact. Borel and Serre have constructed a natural compactification $\Gamma\backslash X\hookrightarrow \Gamma\backslash \overline{X}$ where $\Gamma\backslash \overline{X}$ is a manifold with corners and where the inclusion is a homotopy equivalence. (Comp. \cite{art2-key3}). In various papers it has been shown that we can construct cohomology classes on $\Gamma\backslash X$ by starting from cohomology classes on the boundary. Roughly speaking we associate to a cohomology class $\psi$ on the boundary an Eisenstein series $E(\psi,s)$ which is a differential form depending on a complex parameters $s$. For a special value $s_{\psi}$ of our complex parameter this form may become a closed form. This closed form represents a cohomology class and its restriction to the boundary is related to our original class $\psi$(\cite{art2-key7}, \cite{art2-key8} and \cite{art2-key18}). We look at this as a procedure to construct cohomology classes on $\Gamma\backslash X$.

On the other hand we have another construction which gives us homology classes. To get these homology classes we start from lower dimensional reductive subgroups $M_{\infty}\hookrightarrow G_{\infty}$ for which $\Gamma_{M}=\Gamma\cap M_{\infty}$ is an arithmetic subgroup. If $X_{M}$ is the corresponding symmetric space we get a map $\Gamma_{M}\backslash X_{M}\to \Gamma\backslash X$. We even can find cases where $\Gamma_{M}\backslash X_{M}$ is compact and then the fundamental class of $\Gamma_{M}\backslash X_{M}$ gives us a homology class on $\Gamma\backslash X$. Our problem is to find situations where the dimension of $\Gamma_{M}\backslash X_{M}$\pageoriginale---which is also the dimension of the homology class--equals the dimension of an Eisenstein class. If this is the case we can ask for the value of the Eisenstein class on the above homology class which amounts to evaluating the integral
$$
\int\limits_{\Gamma_{M}\backslash X_{M}}E(\psi,s_{\psi})
$$
This idea of constructing cycles by means of subgroups $M_{\infty}\hookrightarrow G_{\infty}$ appears already in \cite{art2-key2} and \cite{art2-key16}.

In this paper we shall not consider the general problem but only a very special example. We take the group $G_{\infty}=\PGL_{2}(\mathbb{C})$ and $\Gamma$ will be a member of a very specific class of congruence subgroups of $\PGL_{2}(\bZ[i])$. If $\gamma\in \Gamma$ and if $\gamma$ is not unipotent then it generates a quadratic field extension $E(\gamma)$ in the matrix ring $M_{2}(\mathbb{Q}(i))$ which defines a reductive subgroup in $\PGL_{2}(\mathbb{C})$. Then the quotient $\Gamma_{M}\backslash X_{M}$ in this case will simply be a circle and we shall compute the integrals of Eisenstein classes over these circles. It will turn out that these period integrals are expressible in terms of values of $L$-functions with Grossencharaktere of type $A_{o}$. The results are stated in section \ref{art2-sec3}.

Actually we have much more general results. We have a clear picture for those arithmetic groups which come from the group $\GL_{2}$ over an arbitrary algebraic number field. It is planned to write a paper in which we treat this more general situation. But it is clear that this paper will be very long, very difficult to write and certainly also not easy to read. For instance we shall have to use adeles, we have to introduce coefficient systems and so on. That paper will contain proofs of the results announced in \cite{art2-key7} and the results in there have to be generalized. Therefore I made up my mind and decided to write a paper where all this is discussed in a special case. I tried to give many details which will cause some repetition and overlap with older papers and the one planned. But the degree of complexity in the general situation is very high and I think it might be useful to discuss one special case.

During the preparation of this paper here I became aware that also the theory of Eisenstein classes which has been announced in \cite{art2-key7} has some interesting arithmetic aspects. We shall devote a large part of this paper to\pageoriginale
%page 0043
