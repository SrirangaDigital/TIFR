\chapter{CRYSTALLINE COHOMOLOGY, DIEUDONN\'E MODULES, AND JACOBI SUMS}\label{art-6}

\begin{center}
{\large By~ Nicholoas M. Katz}
\end{center}

\setcounter{pageoriginal}{164}
\pageoriginale

%\tableofcontents


\section*{Introduction}
\pageoriginale
Hasse \cite{art6-key20} and Hasse-Davenport \cite{art6-key21} were the first to realize the connection between exponential sums over finite fields and the theory of zeta and $L$-functions of algebraic varieties over finite fields. This connection was exploited to Weil; one of the very first applications that Weil gave of the then newly proven ``Riemann Hypothesis'' for curves over finite fields was the estimation of the absolute value of Kloosterman sums (cf \cite{art6-key46}). The basic idea (cf \cite{art6-key20}) is that by using the theory of $L$-functions, one can express the negative of such an exponential sum as the sum of certain of the reciprocal zeroes of the zeta function itself; because the magnitude of these zeroes is given by the ``Riemann Hypothesis,'' one gets an estimate. In a fixed characteristic $p$, the estimate one gets in this way for all the fintie fields $\bF_{p^{n}}$ is best possible. On the other hand, very little is known about the variation with $p$ of the absolute values, even for Kloosterman sums, though in this case there is a conjecture, of Sato-Tate type, which seems inaccessible at present.

One case in which the problem of unknown variation with $p$ does not arise is when the expression of the exponential sum as a sum of reciprocal roots of zeta reduces to a sum consisting of a {\em single} reciprocal root; then the Riemann Hypothesis tells us the exact magnitude of the exponential sum. Conversely, an elementary argument shows that in a certain sense, this is the only case in which such exact knowledge of the magnitude of exponential sums can arise, and it shows further that a theorem of Hasse-Davenport type always results from such exact knowledge. Examples of exponential sums of this sort are Gauss sums and Jacobi sums.

Honda was the first to suggest that the identification of say, Jaboci sums, with reciprocal zeroes of zeta functions could also lead to significant non-archimedean information about Jacobi sums. A few years before his untimely death, Honda conjectured a $p$-adic limit formula for Jacobi sums in terms of ratios of binomail coefficients (\cite{art6-key23}). I gave an over-complicated proof (in a letter to Honda of Nov. 1971) which managed to shed no light whatever on the meaning of the formula. Recently, B. H. Gross and N. Koblitz \cite{art6-key14} showed that Honda's limit formula was really an {\em exact} $p$-adic formula for Jacobi sums in terms of products of values of Morita's $p$-adic $\Gamma$-function; as such, it constituted the first improvement in this century over Stickelberger's formula which gave the $p$-adic\pageoriginale valuation and the {\em first} non-vanishing $p$-adic digit in the $p$-adic expansion of a Jacobi sum!

In this paper, I will discuss the cohomological genesis of formulas of the sort discovered by Honda. The basic idea is that the reciprocal zeroes of zeta are the eigenvalues of the Frobenius endomorphism of a suitable cohomology group; if this group, together with the action of Frobenius upon it, can be made sufficiently explicit, one obtains the desired ``explicit formulas''.

There are two approaches to the question, which differ more in style than in substance. The first and longer is based on Honda's explicit construction of the Dieudonn\'e module of a formal group in terms of ``formal de Rham cohomology''. The second, less elementary but more efficient, is grounded in crystalline cohomology, particularly in the theory of the de Rham-Witt complex. I hope the reader will share my belief that there is something to be gained from each of the approaches, and pardon my decision to discuss both of them.

I would like to thank B. Dwork for many helpful discussions concerning the original proof of Honda's conjecture. Whatever I know of the Grothendieck-Mazur-Messing approach to Dieudonn\'e theory through exotic Ext's, I was taught by Bill Messing. I would also like to thank Spencer Bloch for his encouragement when I was trying to understand Honda's explicit Dieudonn\'e theory, and Luc Illusie for gently correcting some extravagent assertions I made at the Colloquium.

Finally, I would like to dedicate this paper to the memory of T. Honda.

\medskip
\noindent
{\bf I. Elementary Axiomatics, and the Hasse-Davenport Theorem.}
Consider a projective, smooth and geometrically connected variety $X$, say of dimension $d$, over a finite field $\bF_{q}$. For each integer $n\geq 1$, we denote by $X(\bF_{q^{n}})$ the finite set of points of $X$ with values in $\bF_{q^{n}}$, and by $\sharp X(\bF_{q^{n}})$ the cardinality of this set. The zeta function $Z(X/\bF_{q},T)$ of $X$ over $\bF_{q}$ is the formal power series in $T$ with $Q$-coefficients defined as
$$
Z(X/\bF_{q},T)=\exp \left(\sum\limits_{n\geq 1}\frac{T^{n}}{n}\sharp X(\bF_{q^{n}})\right).
$$
Thanks\pageoriginale %page no. 168
