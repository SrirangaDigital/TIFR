\chapter{CRYSTALLINE COHOMOLOGY, DIEUDONN\'E MODULES, AND JACOBI SUMS}\label{art-6}

\begin{center}
{\large By~ Nicholoas M. Katz}
\end{center}

\setcounter{pageoriginal}{164}
\pageoriginale

%\tableofcontents


\section*{Introduction}
\pageoriginale
Hasse \cite{art6-key20} and Hasse-Davenport \cite{art6-key21} were the first to realize the connection between exponential sums over finite fields and the theory of zeta and $L$-functions of algebraic varieties over finite fields. This connection was exploited to Weil; one of the very first applications that Weil gave of the then newly proven ``Riemann Hypothesis'' for curves over finite fields was the estimation of the absolute value of Kloosterman sums (cf \cite{art6-key46}). The basic idea (cf \cite{art6-key20}) is that by using the theory of $L$-functions, one can express the negative of such an exponential sum as the sum of certain of the reciprocal zeroes of the zeta function itself; because the magnitude of these zeroes is given by the ``Riemann Hypothesis,'' one gets an estimate. In a fixed characteristic $p$, the estimate one gets in this way for all the fintie fields $\bF_{p^{n}}$ is best possible. On the other hand, very little is known about the variation with $p$ of the absolute values, even for Kloosterman sums, though in this case there is a conjecture, of Sato-Tate type, which seems inaccessible at present.

One case in which the problem of unknown variation with $p$ does not arise is when the expression of the exponential sum as a sum of reciprocal roots of zeta reduces to a sum consisting of a {\em single} reciprocal root; then the Riemann Hypothesis tells us the exact magnitude of the exponential sum. Conversely, an elementary argument shows that in a certain sense, this is the only case in which such exact knowledge of the magnitude of exponential sums can arise, and it shows further that a theorem of Hasse-Davenport type always results from such exact knowledge. Examples of exponential sums of this sort are Gauss sums and Jacobi sums.

Honda was the first to suggest that the identification of say, Jaboci sums, with reciprocal zeroes of zeta functions could also lead to significant non-archimedean information about Jacobi sums. A few years before his untimely death, Honda conjectured a $p$-adic limit formula for Jacobi sums in terms of ratios of binomail coefficients (\cite{art6-key23}). I gave an over-complicated proof (in a letter to Honda of Nov. 1971) which managed to shed no light whatever on the meaning of the formula. Recently, B. H. Gross and N. Koblitz \cite{art6-key14} showed that Honda's limit formula was really an {\em exact} $p$-adic formula for Jacobi sums in terms of products of values of Morita's $p$-adic $\Gamma$-function; as such, it constituted the first improvement in this century over Stickelberger's formula which gave the $p$-adic\pageoriginale valuation and the {\em first} non-vanishing $p$-adic digit in the $p$-adic expansion of a Jacobi sum!

In this paper, I will discuss the cohomological genesis of formulas of the sort discovered by Honda. The basic idea is that the reciprocal zeroes of zeta are the eigenvalues of the Frobenius endomorphism of a suitable cohomology group; if this group, together with the action of Frobenius upon it, can be made sufficiently explicit, one obtains the desired ``explicit formulas''.

There are two approaches to the question, which differ more in style than in substance. The first and longer is based on Honda's explicit construction of the Dieudonn\'e module of a formal group in terms of ``formal de Rham cohomology''. The second, less elementary but more efficient, is grounded in crystalline cohomology, particularly in the theory of the de Rham-Witt complex. I hope the reader will share my belief that there is something to be gained from each of the approaches, and pardon my decision to discuss both of them.

I would like to thank B. Dwork for many helpful discussions concerning the original proof of Honda's conjecture. Whatever I know of the Grothendieck-Mazur-Messing approach to Dieudonn\'e theory through exotic Ext's, I was taught by Bill Messing. I would also like to thank Spencer Bloch for his encouragement when I was trying to understand Honda's explicit Dieudonn\'e theory, and Luc Illusie for gently correcting some extravagent assertions I made at the Colloquium.

Finally, I would like to dedicate this paper to the memory of T. Honda.

\medskip
\noindent
{\bf I. Elementary Axiomatics, and the Hasse-Davenport Theorem.}
Consider a projective, smooth and geometrically connected variety $X$, say of dimension $d$, over a finite field $\bF_{q}$. For each integer $n\geq 1$, we denote by $X(\bF_{q^{n}})$ the finite set of points of $X$ with values in $\bF_{q^{n}}$, and by $\sharp X(\bF_{q^{n}})$ the cardinality of this set. The zeta function $Z(X/\bF_{q},T)$ of $X$ over $\bF_{q}$ is the formal power series in $T$ with $Q$-coefficients defined as
$$
Z(X/\bF_{q},T)=\exp \left(\sum\limits_{n\geq 1}\frac{T^{n}}{n}\sharp X(\bF_{q^{n}})\right).
$$
Thanks\pageoriginale to Deligne \cite{art6-key6}, we know that this zeta function has a unique expression as a finite alternating product of polynomials $P_i (T) \in \bZ [T]$, $i = 0, \ldots, 2d$:
$$
Z (X/ \bF_q, T) = \prod\limits^{2d}_{i=0} P_i (T)^{(-1)^{i+1}} = \frac{P_1 P_3 \ldots P_{2d-1}}{P_0 P_2 \ldots P_{2s}}
$$
in which each polynomial $P_i (T) \in\bZ[T]$ is of the form 
$$
P_i (T) = \prod\limits^{\deg P_i}_{j=1} (1 -\alpha_{i,j} T)
$$
with $\alpha_{i,j}$ algebraic integers such that 
$$
|\alpha_{i,j}| = \sqrt{q}^i
$$
for \textit{any} archimedean absolute value $|\;|$ on the field $\bar{\bQ}$ of all algebraic numbers. The extreme polynomials $P_0$, $P_{2d}$ are given explicitly:
$$
P_0 (T) = (1-T) , P_{2d} (T) = (1 - q^d \cdot T)
$$

Despite this apparently ``elementary'' characterization of the polynomials $P_i (T)$, their true genesis is cohomological. Let us recall this briefly.

For each prime number $l$ different from the characteristic $p$ of $\bF_q$, let us denote by $H^i_l(X)$ the finitely generated $\bZ_l$-module defined as 
$$
H^i_l (X) =  \varprojlim_n H^i_{\text{etale}} (X \otimes \bar{\bF}_q, \bZ / l^n \bZ ).
$$
Corresponding to the prime $p$ itself, we denote by $W(\bF_q)$ the ring of $p$-Witt vectors of $\bF_q$, and by $H^i_{\text{cris}} (X)$ the finitely generated $W(\bF_q)$-module defined as 
$$
H^i_{\text{cris}} (X) = \varprojlim_n H^i_{\text{cris}} (X / W_n (\bF_q)).
$$
The Frobenius endomorphism $F$ of $X$ relative to $\bF_q$ acts, by functoriality, on these various cohomology groups $H^i_l(X)$ for $l\neq p$, and $H^i_{\text{cris}} (X)$; and $F$ induces automorphisms of the corresponding vector spaces $H^i_l(X) \bigotimes_{Z_l} Q_l$, $H^i_{\text{cris}} (X) \bigotimes_{W(\bF_q)} K$ ($K$ denoting the fraction field of $W(\bF_q)$). The polynomial $P_i (T) \in \bZ [T]$ which occurs in the factorization of the zeta function is then given cohomologically by the formulas 
\begin{gather*}
P_i(T) = \det (1 - T F \big| H^i_l(X) \otimes \bQ_l) \text{ for } l \neq p\\
P_i (T) = \det (1 - T F \big| H^i_{\text{cris}} (X) \otimes K).
\end{gather*}\pageoriginale 

The resulting formula for zeta as the alternating product of characteristic polynomials of $F$ on the $H^i$, in each of the cohomology theories $H^i_l (X) \otimes Q_l$ for $l \neq p$, $H^i_{\text{cris}} (X) \otimes K$, is equivalent, via logarithmic differentiation, to the identities in those theories 
$$
\neq X (\bF_{q^n}) = \sum (-1)^i \text{ trace } (F^n \big| H^i). \text{ for all } n \geq 1.
$$
By viewing the set $X(\bF_{q^n})$ as the set of fixed points of $F^n$ acting on $X(\bar{\bF}_q)$, this identity becomes a Lefschetz trace formula
$$
\# \text{ Fix } (F^n ) =\sum(-1)^i \text{ trace } (F^n \big| H^i) \text{ all } n \geq 1
$$
for $F$ and its iterates in each of our cohomology theories. If we take as \textit{given} these Lefschetz trace formulas, then the identification of $P_i$ with $\det (1 - F T \big| H^i)$ is equivalent to the assertion:
\begin{align*}
&\text{On any of the groups $H^i_l(X) \otimes Q_l$ with $l \neq p$,}\\
&\text{$H^i_{\text{cris}} (X) \otimes K$, the eigenvalues of $F$ are alge-}\\
&\text{braic integers all of whose archimedean absolute}\\
&\text{values are $\sqrt{q}^i$.}
\end{align*}
In fact, there is not a great deal more that is known about the action of $F$ on the $H^i_l(X) \otimes Q_l$ for $l \neq p$, and on $H^i_{\text{cris}} (X)  \otimes K$. It is still \textit{not} known, for example, whether the action of $F$ on these cohomology groups is always semi-simple when $i>1$. (That it is when $i =1$ results from the theory of abelian varieties).

Suppose that a finite group $G$ operates on $X$ by $\bF_q$-automorphisms. Let us choose a number field $E$ big enough that all complex representations of $G$ are realizable over $E$, and whose residue fields at all $p$-adic places contain $\bF_q$. (For example, the field $Q(\zeta_{q-1}, \zeta_N)$, where $N$ is the $l$.c.m. of the orders of elements of $G$, is such an $E$). We denote by $\lambda$ an $l$-adic place of $E$, $l \neq p$, and by $\sfP$ a $p$-adic place of $E$. Thus $E_\lambda$ is a finite extension of $Q_l$, and $E_\sfP$ is a finite extension of $K$.

Let $M$ be a finite dimensional $E$-vector space given with an action of $G$, say $\rho : G \to \Aut_E (M)$. The associated $L$-function $L(X/\sF_q, \rho, T)$ is the formal power series with $E$-coefficients defined as 
$$
L(X/\sF_q, \rho, T) = \exp \left(\sum\limits_{n \geq 1} \frac{T^n}{n} \cdot \frac{1}{\# G} \sum\limits_{g \in G} tr (\rho (g^{-1})) \# Fix (F^n g) \right)
$$\pageoriginale
where Fix $(F^n g)$ denotes the finite set of fixed points of $F^ng$ acting on $X(\bar{\bF}_q)$. We recover the zeta function of $X/\bF_q$ by taking for $\rho$ the regular representation of $G$. The usual formalism of zeta and $L$-functions gives 
$$
Z(X/ \bF_q, T) = \prod\limits_{\rho \text{ irred}} L (X / \bF_q, \rho, T)^{\deg (\rho)}
$$

It follows from Deligne's results that for any representation $\rho$, we have a unique expression for the corresponding $L$-function as an alternating product of polynomials $P_{i,\rho}(T) \in E [T]$,
$$
L(X/ \bF_{q}, \rho, T) =\prod\limits^{2d}_{i=0} P_{i,\rho} (T)^{(-1)^{i+1}},
$$
which are of the form 
$$
P_{i,\rho} (T) = \prod\limits^{\deg P_{i,\rho}}_{j=1} (1 - \alpha_{i, j, \rho} T)
$$
with algebraic integers $\alpha_{i, j,\rho}$ such that 
$$
|\alpha_{i, j, \rho}| = \sqrt{q}^i
$$
for any archimedean absolute value $|\;|$ on the field $\bar{Q}$ of all algebraic numbers. 

The cohomological expression of there $P_{i,\rho}$ is straighforward  (\cf\cite{art6-key18}). Because the action of $G$ is ``defined over $\bF_q$'' it commutes with $F$, and therefore the induced action of $G$ on the cohomology commutes with the action of $F$. Therefore $G$, acting by composition, induces automorphisms of the $E_\lambda$-vector spaces,$l \neq \rho$,
$$
\Hom_{E_\lambda [G]} (M\bigotimes\limits_E E_\lambda, H^i_l (X) \bigotimes\limits_{Z_l} E_\lambda).
$$
and of the $E_\sfP$-vector spaces
$$
\Hom_{E_{\sfP} [G]} (M \bigotimes\limits_{E} E_\sfP, H^i_{\text{cris}} (X) \bigotimes\limits_{W(\sF_q)} E_\sfP). 
$$
The polynomials $P_{i, \rho} (T) \in E [T]$ are given by the formulas 
\begin{align*}
P_{i,\rho} (T) & = \det (1 - T F \big| \Hom_{E_\lambda [G]} (M \bigotimes\limits_{E} E_\lambda, H^i_l (X) \bigotimes\limits_{Z_l} E_\lambda )) \text{ for } l \neq \rho \\
P_{i, \rho} (T) & = \det (1 - TF \big| \Hom_{E_\sfP [G]} (M\bigotimes\limits_E E_\sfP, H^i_{\text{cris}} (X) \bigotimes\limits_{W (\sF_q)} E_\sfP)).
\end{align*}

Let us\pageoriginale recall the derivation of these formulas. We first observe that the characteristic polynomial of $F$ on $\Hom_G (M, H^i) \simeq ({\displaystyle{\mathop{M}^v}} \otimes H^i)^G \subset {\displaystyle{\mathop{M}^v}} \otimes H^i$ \textit{divides} $\det (1-F T \big| H^i)^{\dim {\displaystyle{\mathop{M}^v}}}$, and hence the eigenvalues of $F$ on $\Hom_G (M, H^i)$ are algebraic integers, all of whose archimedean absolute values are $\sqrt{q}^i$. So it remains only to verify that the alternating product of those characteristic polynomials is indeed the $L$-function, \ie. that 
$$
L(X \backslash \bF_q, \rho, T) = \prod \det (1 - F T \big| ({\displaystyle{\mathop{M}^v}} \otimes H^i)^{G} )^{(-1)^{i+1} },
$$
Equivalently, we must check that 
\begin{align*}
\frac{1}{\# G} \sum \text{ trace }\rho  (g^{-1}) &\;\# \text{ Fix } (F^n g)\\
& = \sum (-1)^i \text{ trace } (1 \otimes F^n \big| ({\displaystyle{\mathop{M}^v}} \otimes H^i)^G)\\
& = \sum (-1)^i \frac{1}{\# G} \sum\limits_{g \in G} \text{ trace } (g \otimes F^n g \big|{\displaystyle{\mathop{M}^v}} \otimes H^i)\\
& = \sum (-1)^{i} \frac{1}{\# G} \sum\limits_{g \in G} \text{ trace } {\displaystyle{\mathop{\rho}^v}} (g) \cdot \text{ trace } (F^n g \big| H^i)\\
& = \frac{1}{\# G} \sum\limits_{g \in G}  \text{ trace } \rho (g^{-1}) \sum (-1)^i \text{ trace } (F^n g \big| H^i). 
\end{align*}
To check this last equality, we would like to invoke the Lefschetz trace formula, not for $F^n$, but for $F^ng$, with $g$ an automorphism of \textit{finite order} which commutes with $F$; this amounts to invoking the Lefschetz trace formula for $Fg$ on $X$ and on all its ``extensions of scalars'' $X \otimes \bF_{q^n}$. But an elementary descent argument shows that given an automorphism $g$ of finite order which commutes with $F$, there is \textit{another} variety $X'/ \bF_q$ together with an isomorphism $X \otimes \bar{\bF}_q \simeq X' \otimes \bar{\bF}_q$ under which $F g \otimes 1$ corresponds to $F \otimes 1$. Because this isomorphism also induces isomorphisms of cohomology groups 
\begin{gather*}
H^i_l (X)^{\text{dfn}} H^i_{\text{et}} (X' \otimes \bar{\bF}_q, \bZ_l) \simeq H^i (X \otimes \bar{\bF}_q, \bZ_l)^{\text{dfn}} H^i_l(X),\\
H^i_{\text{cris}} (X') \otimes W (\bar{\bF}_q) \simeq H^i_{\text{cris}} (X' \otimes \bar{\bF}_q) \simeq H^i_{\text{cris}} (X \otimes \bar{\bF}_q) \simeq\\
\simeq H^i_{\text{cris}} (X) \otimes W (\bar{\bF}_q),
\end{gather*}
the truth\pageoriginale of the Lefschetz formula for $Fg$ on $X$ results from its truth for $F$ on $X'$.

Let us now consider in greater detail the case of an irreducible $\rho$. Then $P_{i, \rho}$ is a polynomial whose degree is the common \textit{multiplicity} of $\rho$ in any of the $H^i_l(X) \otimes E_\lambda$, $l \neq \rho$, or in $H^i_{\text{cris}} (X) \otimes E_\sfP$. Decomposing the regular representation leads to the factorization
$$
P_i (T) = \prod\limits_{\rho \text{ irred}} P_{i, \rho} (T)^{\deg (\rho)}
$$
The coarser factorization 
$$
P_i (T) = \prod\limits_{ \rho \text{irred}} (P_{i,\rho} (T)^{\deg(\rho)})
$$
corresponds to the decomposition of $H^i_l(X) \otimes E_\lambda$, $\resp H^i_{\text{cris}} (X) \otimes E_\sfP$, into $\rho$-isotypical components
\begin{gather*}
H^i_l (X) \otimes E_\lambda \simeq \bigotimes\limits_{\text{irred} \rho } \left(H^i_l(X) \times E_\lambda \right)^\rho\\
H^i_{\text{cris}} (X) \otimes E_\sfP \simeq \bigotimes\limits_{\text{irred} \rho} \left(H^i_{\text{cris}} (X) \otimes E_\sfP \right)^\rho
\end{gather*}
Indeed the corresponding identities, for $\rho$ irreducible, are 
\begin{align*}
P_{i,\rho} (T)^{\deg(\rho)}  & = \det (1 - TF \big| (H^i_l (X) \otimes E_\lambda)^\rho) l \neq p\\
P_{i, \rho} (T)^{\deg (\rho)} & = \det (1 - TF \big| (H^i_{\text{cris}} (X) \otimes E_\sfP)^\rho).
\end{align*}

Let us denote by $S(X/ \bF_q, \rho, n)$ the exponential sums used to define the $L$-function:
$$
S (X/ \sF_q , \rho, n) = \frac{1}{\# G} \sum\limits_{g \in G} tr (\rho (g)) \# \text{ Fix }(F^n g^{-1}). 
$$
The following lemma gives the cohomological meaning of theorems of Hasse-Davenport type (\cf \cite{art6-key20}).

\medskip
\noindent
{\bfseries Lemma \thnum{1.1}:\label{art6-lem1.1}}
\textit{Let $X/\bF_q$ be projective and smooth. Let a finite group $G$ operate on $X$ by $\bF_q$-automorphisms, and let $p$ be an irreducible complex representation of $G$. Fix an integer $i_\circ$, and denote by $H^{i_\circ}$ any one of the cohomology groups $H^{i_\circ}_l(X) \bigotimes_{Z_l} E_\lambda$ with $l \neq p$, or $H^{i_\circ}_{\text{cris}}(X) \bigotimes\limits_{W(\bF_q)} E_\sfP$. Let $|\;|$ be any archimedean absolute value on the filed $\bar{Q}$ of all algebraic numbers. The following conditions are equivalent:}
\begin{enumerate}
\item[(1)] \textit{The\pageoriginale multiplicity of $\rho$ in $H^{i_\circ}$ is one, and the multiplicity of $\rho$ in $H^i$ is zero if $i \neq i_\circ$.}

\item[(2)] \textit{For all $n \geq 1$, we have }
\begin{gather*}
(-1)^{i_\circ} S (X / \bF_q, \rho,n) = ((-1)^{i_\circ} S (X / \bF_q, \rho, 1))^n,\\
\text{and } \hspace{3cm}|S (X / \bF_q, \rho, 1)| = \sqrt{q}^{i_\circ} \hspace{3cm}
\end{gather*}

\item[(3)] \textit{For all $n \geq 1$, we have}
$$
|S(X/ \bF_q, \rho, n)| = \sqrt{q}^{i_\circ n}
$$

\item[(4)] \textit{For all $n \geq 1$, we have }
\begin{gather*}
|S (X/ \bF_q, \rho, n)| = |S(X/ \bF_q,  \rho, 1)|^n\\
\text{and } \hspace{3cm} \sqrt{q}^{i_\circ} \leq |S (X / \bF_q, \rho, 1)| < \sqrt{q}^{1+ i_\circ}\hspace{3cm}
\end{gather*}

\item[(5)] \textit{The polynomial $P_{i_{\circ, \rho}} (T)$ is given by}
$$
P_{i_{\circ, \rho}} (T) = 1 - (-1)^{i_\circ} S (X / \bF_q, \rho, 1) T 
$$
\textit{and for $i \neq i_\circ$, we have $P_{i,\rho} (T) =1$.}

\item[(6)] \textit{The $\rho$-isotypical component $(H^i)^\rho =0$ for $i \neq i_\circ$, $(H^{i_\circ})^\rho$ has dimension $ = \deg (\rho)$, and $F$ operates on $(H^{i_\circ})^\rho$ as the scalar $(-1)^{i_\circ} S (X/ \bF_q, \rho, 1)$.}
\end{enumerate}

\begin{proof}
This is an easy exercise, using the basic identities:
$$
\left\{
\begin{aligned}
&\exp \left(\sum \frac{T^n}{n}  S (X / \bF_q, \rho, n)\right) = L (X/ \bF_q, \rho, T) = \prod\limits_{i} P_{i,\rho} (T)^{(-1)^{i+1}}\\
& \quad P_{i,\rho} (T) = \prod\limits_j (1 - \alpha_{i,j,\rho} T), |\alpha_{i,j,\rho}| = \sqrt{q}^i\\
& \deg P_{i,\rho} = \textit{ multiplicity of $\rho$ in } H^i = \frac{1}{\deg (\rho)} \cdot \dim ((H^i)\rho).
\end{aligned}
\right.
$$

Suppose, first, that (1) holds, or equivalently that for $i \neq i_\circ$, $P_{i,\rho} (T) =1$, while $P_{i_{\circ, \rho}}$ is a linear polynomial $P_{i_{\circ,\rho}} (T) = (1-A T)$ with $|A| = \sqrt{q}^{i_\circ}$. The cohomological expression for $L$ then becomes 
$$
\exp \left(\sum \frac{T^n}{n} S (X/ \bF_q, \rho, n) \right) = \left(\frac{1}{1-AT} \right)^{(-1)^{i_\circ}}.
$$
Taking\pageoriginale logarithms and equating coefficients, we find 
$$
(-1)^{i_\circ} S (X/ \bF_q, \rho, n) = A^n \quad \text{ for all } n \geq 1. 
$$
 
In particular (2) and (5) hold.

The implications (5) $\Rightarrow$ (1), (6) $\Rightarrow$ (1) are obvious. Also (5) $\Rightarrow$ (6), for if $P_{i_{\circ, \rho}}$ is linear, then $\rho$ has multiplicity one in $H^{i_\circ}$, so that $(H^{i_\circ})^\rho$ is $G$-irreducible, and hence $F$ must operate on $(H^{i_\circ})^\rho$ as a \textit{scalar}, which we compute by the formula
$$
P_{i_{\circ, \rho}} (T)^{\deg (\rho)} = \det (1 - TF \big| (H^{i_\circ})^\rho).
$$

Clearly we have (2) $\Rightarrow$ (3) $\Rightarrow$ (4). We must show that if (4) holds, then exactly \textit{one} of the $P_{i,\rho}$ is $\neq 1$, and that one is linear. Logarithmically differentiating the cohomological formula for $L$, we find 
$$
S(X/ \bF_q, \rho, n) = \sum\limits_i (-1)^i \sum\limits^{\deg P_{i, \rho}}_{j=1} (\alpha_{i, j, \rho})^n, \quad |\alpha_{i,j,\rho}| = \sqrt{q}^i.
$$
We must show that if (4) holds, then the double sum has only a single term in it. Separating the $\alpha_{i, j, \rho}$ according to the \textit{parity} of $i$, we get two disjoint sets of non-zero complex numbers (disjoint because their absolute values are disjoint), to which we apply the following lemma.


\end{proof}


%%%% 174 page



%%%%%%%%%%
\medskip
\noindent
{\bfseries Theorem \thnum{1}:\label{art11-thm1}}
