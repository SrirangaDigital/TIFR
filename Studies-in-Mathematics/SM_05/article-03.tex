\chapter{WAVE FRONT SETS OF REPRESENTATIONS OF LIE GROUPS}

\begin{center}
{\large By~ Roger Howe\footnote[1]{Partially supported by NSF Grant MCS 7610435}}
\end{center}

\bigskip

\setcounter{pageoriginal}{116}
\section*{Introduction}\pageoriginale
In the past few years the concept of wave front set \cite{art3-D} has proved fruitful for the theory of distributions and P.D.E. It seems it might also be of use in the representation theory of Lie groups. Its close relative, the singular spectrum of a hyperfunction, has already been discussed in a special context in \cite{art3-K-V}, which served as the catalyst for this note. The purpose here is to define and discuss general properties of wave front sets of representations, and to give some examples.

I would like to thank Nolan Wallach for very valuable discussions regarding this paper. Especially, the principle of proof of proposition \ref{art3-prop2} comes from him. Also I thank Richard Beals for valuable technical discussions.

\section{Generalities}
Let $\rho$ be a representation of the Lie group $G$. For convenience we shall assume $\rho$ is unitary, although this is not strictly necessary. Let $H$ be the Hilbert space on which $\rho$ acts, and let $J_{1}(H)=J_{1}$ be the trace class operators on $H$. Given $T\in J_{1}$, put
\begin{equation*}
\tr_{\rho}(T)(g)=\tr(\rho(g)T)\quad g\in G\tag{1.1}\label{art3-eq1.1}
\end{equation*}
where $\tr$ is the usual trace functional on $J_{1}$. Then
\begin{equation*}
\tr_{\rho}:J_{1}\to \mathbb{C}_{b}(G)\tag{1.2}\label{art3-eq1.2}
\end{equation*}
where $\mathbb{C}_{b}(G)$ is the space of bounded functions on $G$, is a norm-decreasing map. The image of $\tr_{\rho}$ is called the {\em space of (continuous) matrix coefficients of $\rho$.}

We may also regard $\tr_{\rho}(T)$ as a distribution on $G$ by integration, in the usual fashion
\begin{equation*}
\tr_{\rho}(T)(f)=\int\limits_{G}f(g)\tr_{\rho}(T)dg=\tr(\rho(f)T)\quad f\in \mathbb{C}^{\infty}_{c}(G).\tag{1.3}\label{art3-eq1.3}
\end{equation*}
Here\pageoriginale $dg$ is Haar maasure on $G$. Since $\tr_{\rho}(T)$ is a distribution on $G$, we may consider its wave front set $WF(\tr_{\rho}(T))$. Our basic reference for wave front sets is \cite{art3-D} and we shall recall their basic definitions and properties as they are needed. For now, recall $WF(\tr_{\rho}(T))$ is a closed, conical (i.e., closed under positive dilations in the fibers) set in $T^{*}G$, the cotangent bundle of $G$.

\begin{defi*}
$WF_{\rho}$ is the closure of the union of $WF(\tr_{\rho}(T))$ as $T$ varies over $J_{1}$.
\end{defi*}

Thus $WF_{\rho}$ is also a closed conical set of $T^{*}G$.

\begin{remark*}
This is not the same as the wave front set defined in \cite{art3-H1}, which is sort of a dual notion to the present one.
\end{remark*}

\medskip
\noindent
{\bf Proposition \thnum{1.1}.\label{art3-prop1.1}}~{\em $WF_{\rho}$ is invariant under left and right translations of $G$ on $T^{*}G$.}

\begin{proof}
Define as usual left and right translations on functions and distributions:
\begin{equation*}
\begin{cases}
L_{g}(f)(g')=f(g^{-1}g'):R_{g}(f)(g')=f(g'g)\quad f\in \mathbb{C}^{\infty}_{c}(G)\\
L_{g}(D)(d)=D(L_{g^{-1}}f); \ R_{g}(D)(f)=D(R_{g^{-1}}f)D\in D(G).
\end{cases}\tag{1.4}\label{art3-eq1.4}
\end{equation*}
Then we have the well-known relations
\begin{equation*}
L_{g}\tr_{\rho}(T)=\tr_{\rho}(T_{\rho}(g)^{-1}); \ R_{g}\tr_{\rho}(T)=\tr_{\rho}(\rho(g)T)\tag{1.5}\label{art3-eq1.5}
\end{equation*}

Left and right translations of $G$ also induce in the usual way transformations $L^{*}_{g}$ and $R^{*}_{g}$ on $T^{*}G$. By the naturality of the wave front set (\cite{art3-D}, proposition 1.3.3.) one has, for a distribution $D$ on $G$.
\begin{equation*}
WF(L_{g}D)=L_{g}^{*}(WF(D))\quad\text{and}\quad WF(R_{g}(D))=R_{g}^{*}(WF(D)).\tag{1.6}\label{art3-eq1.6}
\end{equation*}
The proposition follows directly from the definition and equations \eqref{art3-eq1.5} and \eqref{art3-eq1.6}.

Let $\mathfrak{g}$ be the Lie algebra of $G$, and let $\mathfrak{g}^{*}$ be the dual of $\mathfrak{g}$. Let $\Ad$ be the adjoint action of $G$ on $\mathfrak{g}$, and let $\Ad^{*}$ be the contragredient action on $\mathfrak{g}^{*}$. We can identify $\mathfrak{g}^{*}$ with the left invariant exterior 1-forms on $G$. This leads to an identification
\begin{equation*}
T^{*}G\simeq G\times \mathfrak{g}^{*}\tag{1.7}\label{art3-eq1.7}
\end{equation*}\pageoriginale
Thus if $\psi\in \mathbb{C}^{\infty}_{c}(G)$, we can regard $d\psi$, the differential of $\psi$, as a $g^{*}$-valued function on $G$. Doing so, we have the following behaviour under right and left translations
\begin{equation*}
d(L_{g}\psi)=L_{g}d\psi\quad d(R_{g}\psi)=\Ad g(R_{g}d\psi)\tag{1.8}\label{art3-eq1.8}
\end{equation*}

One sees from \eqref{art3-eq1.8} that a bi-invariant set in $T^{*}(G)$ is identified via \eqref{art3-eq1.7} with $G\times X$ where $X\subseteq \mathfrak{g}^{*}$ is an $\Ad^{*}G$ invariant set. Thus we can associate to $WF_{\rho}$ a closed conical $\Ad^{*}G$-invariant subset of $\mathfrak{g}^{*}$, to be denoted $WF^{o}_{\rho}$. The set $WF^{o}_{\rho}$ then determines $WF_{\rho}$ via \eqref{art3-eq1.7}.

It is conceivable that $WF_{\rho}$ could be very uninteresting--it might always be all of $\mathfrak{g}^{*}$ for example. Thus it may be instructive to point out at the beginning that for irreducible $\rho$ at least, $WF_{\rho}$ is limited to a certain characteristic and non-trivial behavior.

Let $U(\mathfrak{g})$ be the universal enveloping algebra of $\mathfrak{g}$. It is well known that there is a canonical linear isomorphism, the symmetrization map
\begin{equation*}
\sigma : U(\mathfrak{g})\xrightarrow{\sim}S(\mathfrak{g})\simeq P(\mathfrak{g}^{*})\tag{1.9}\label{art3-eq1.9}
\end{equation*}
where $S(\mathfrak{g})$ is the symmetric algebra of $\mathfrak{g}$, and $P(\mathfrak{g}^{*})$ the polynomial algebra on $\mathfrak{g}^{*}$, the two algebras being identified in the standard way. The symmetrization $\sigma$ is an intertwining map for the adjoint actions of $G$ on $U(\mathfrak{g})$ and on $P(\mathfrak{g}^{*})$. Thus $\sigma$ restricts to a linear isomorphism between $ZU(\mathfrak{g})$, the center of $U(\mathfrak{g})$, and $IP(\mathfrak{g}^{*})$, the $\Ad^{*}G$ invariants in $P(\mathfrak{g}^{*})$.

The map $\sigma$ has a natural interpretation in terms of P.D.E. We can identify each $u\in U(\mathfrak{g})$ to a left invariant differential operator $R_{u}$ on $G$. If $R_{u}$ has order $m$, then the leading symbol of $R_{u}$, in the sense of P.D.E \cite{art3-D}, will be a left-invariant section of $S^{m}T(G)$, the $m$-th symmetric power of the tangent bundle of $G$. Thus the symbol of $R_{u}$ is determined by its value at the identity, which will be an element of $S^{m}\mathfrak{g}\simeq P^{m}(\mathfrak{g}^{*})$. It is known and easy to check from the definitions that the symbol of $R$ is just the $m$-th homogeneous part of $\sigma(u)$.

Let $V(\mathfrak{g}^{*})$ denote the set of common zeroes of the homogeneous elements of positive degree of $IP(\mathfrak{g}^{*})$. We call $V$ the {\em characteristic variety} of $\mathfrak{g}^{*}$ (or of $G$).
\end{proof}

Let\pageoriginale $\rho$ be as above a unitary representation of $G$.

\medskip
\noindent
{\bf Proposition \thnum{1.2}.\label{art3-prop1.2}}~{\em Let $\rho$ be irreducible. Then}
\begin{equation*}
WF^{o}_{\rho}\subseteq V(\mathfrak{g}^{*})\tag{1.10}\label{art3-eq1.10}
\end{equation*}

\begin{proof}
Since $\rho$ is irreducible, the action of $ZU(\mathfrak{g})$ on the smooth vectors of $\rho$ is by scalars \cite{art3-Se}. Say $\rho(z)x=\mu(z)x$ for $x$ a smooth vector and $z\in ZU(\mathfrak{g})$, where $\mu:ZU(\mathfrak{g})\to \mathbb{C}$ is the infinitesimal character of $\rho$. Thus let $x$, $y$ be smooth vectors in $H$, the space of $\rho$. Let $E_{x,y}$ be the dyad
\begin{equation*}
E_{x,y}(u)=(u,x)y\qquad u\in H\tag{1.11}\label{art3-eq1.11}
\end{equation*}
Then
\begin{equation*}
\tr_{\rho}(E_{x,y})(g)=(\rho(g)y,x)\tag{1.12}\label{art3-eq1.12}
\end{equation*}
It follows by differentiating \eqref{art3-eq1.5} that
\begin{equation*}
R_{z}\tr_{\rho}(E_{x,y})=\tr_{\rho}(E_{x,\phi(z)y})=\mu(z)\tr_{\rho}(E_{x,y})\tag{1.13}\label{art3-eq1.13}
\end{equation*}
Here $R_{z}$ is as above, the right convolution operator on $G$ corresponding to $z$.

Since every element in $J_{1}$ is a limit in the trace norm of sums of smooth dyads, and $\tr_{\rho}$ is norm-decreasing, we find that
\begin{equation*}
R_{Z}\tr_{\rho}(T)=\mu(z)\tr_{\rho}(T)\quad T\in J_{1}\tag{1.14}\label{art3-eq1.14}
\end{equation*}
That is, the $\tr_{\rho}(T)$ are all eigendistributions for $ZU(\mathfrak{g})$. Since as $z$ varies in $ZU(\mathfrak{g})$, the symbol in the sense of P.D.E. will vary through all homogeneous elements of $IP(\mathfrak{g}^{*})$, we see that $V(\mathfrak{g}^{*})$ is just the intersection of all the characteristic directions of the $R_{z}$, $z\in ZU(\mathfrak{g})$. Hence by \cite{art3-D}, proposition 5.1.1, we have the inclusions $WF_{\tr_{\rho}}(T)\subseteq G\times V(\mathfrak{g}^{*})$ for all $T$ in $J_{1}$. By definition of $WF^{o}_{\rho}$, the inclusion \eqref{art3-eq1.10} follows.
\end{proof}

\begin{remark*}
We can formulate a relative version of this also. Let $N\subseteq G$ be a normal subgroup. Let $ZU(\sfN)^{G}$ be the $\Ad G$ invariants in $ZU(\sfN)$, where $\sfN$ is the Lie algebra of $N$. The corresponding sub-algebra of $P(\sfN^{*})$ is clearly $IP(\sfN^{*})^{G}$, the $\Ad^{*}G$ invariants in $IP(\sfN^{*})$. Let $V(N^{*};G)$ be the intersection of the zeroes of the homogeneous elements of positive degree in $IP(\sfN^{*})$. Then by the same proof as for the above propositions, we may assert: If $\rho$ is an irreducible representation of $G$, and $\rho|_{H}$ is the restriction\pageoriginale of $\rho$ to $N$, then
\begin{equation*}
WF^{o}(\rho|H)\subseteq V(\sfN^{*};G)\tag{1.15}\label{art3-eq1.15}
\end{equation*}
\end{remark*}

Next we observe that $WF_{\rho}$ behaves very simply under direct sums. If $\rho$ is a representation of $G$, let $n\rho$, where $n$ is a natural number or $\infty$, denote the $n$-fold direct sum of $\rho$ with itself. If $\rho_{1}$ and $\rho_{2}$ are two representations, recall that $\rho_{1}$ and $\rho_{2}$ are called quasi-equivalent if $\infty \rho_{1}$ and $\infty \rho_{2}$ are equivalent.

\medskip
\noindent
{\bf Proposition \thnum{1.3}.\label{art3-prop1.3}}
\begin{itemize}
\item[(a)] {\em If $\rho_{1}$ and $\rho_{2}$ are quasi-equivalent, then $WF^{o}_{\rho_{1}}=WF^{o}_{\rho_{2}}$.}

\item[(b)] {\em In general $WF^{o}(\rho_{1}\oplus \rho_{2})=WF^{o}_{\rho_{1}}\cup WF^{o}_{\rho_{2}}$}
\end{itemize}

\begin{proof}
To prove (a), it is enough to show that $WF^{o}_{\rho}=WF^{o}(\infty\rho)$; but this is clear because $\rho$ and $\infty \rho$ have the same matrix coefficients. Similarly, the general matrix coefficient of $\rho_{1}\oplus \rho_{2}$ is easily seen to have the form
$$
\tr_{\rho_{1}}(T_{1})+\tr_{\rho_{2}}(T_{2}),\quad T_{i}\in J_{1}(H_{i})
$$
where $H_{1}$ is the space of $\rho_{i}$. Setting $T_{1}=0$ and letting $T_{2}$ vary, then vice-versa, we see $WF^{o}_{\rho_{i}}$ is contained in $WF^{o}(\rho_{1}\oplus \rho_{2})$. On the other hand, \cite{art3-D}, definition \ref{art3-defi1.3.1} assures us of the other inclusion necessary for statement (b). This concludes the proposition.

We will now give a technical result offering various descriptions of $WF^{o}_{\rho}$.

Recall that if $f$ is a function of a positive real variable $t$, then $f$ is {\em rapidly decreasing} as $t\to \infty$ if
\begin{equation*}
\sup \{|f(t)|t^{n}:t\geq 1\}=\gamma_{n}(f)<\infty,\quad\text{all $n$ in $\bZ$}
\end{equation*}
Let $e$ denote the identity element of $G$. Let $\supp(\varphi)$ denote the support of $\varphi\in \mathbb{C}^{\infty}_{c}(G)$.
\end{proof}

\medskip
\noindent
{\bf Theorem \thnum{1.4}.\label{art3-thm1.4}}~{\em Let $U\subseteq \mathfrak{g}^{*}$ be an open set. The following conditions on $U$ are all equivalent}
\begin{itemize}
\item[(i)] {\em $U\cap WF^{o}_{\rho}$ is empty}

\item[(ii)] {\em For\pageoriginale any $T$ in $J_{1}(H)$, and every real-valued $\psi\in \mathbb{C}^{\infty}(G)$ such that $d\psi(e)\in U$, there is an open neighborhood $V$ of $e$ such that for any $\varphi\in \mathbb{C}^{\infty}_{c}(V)$ the integral}
\begin{equation*}
I(\varphi,\psi,T)(t)=\int\limits_{G}\tr_{\rho}(T)(g)\varphi(g)e^{it\psi(g)}dg\tag{1.17}\label{art3-eq1.17}
\end{equation*}
{\em is rapidly decreasing as $t\to \infty$. Furthermore, if $\psi=\psi_{\alpha}$ and $\varphi=\varphi_{\alpha}$ depend smoothly on a parameter $\alpha$ varying in a neighborhood of $0$ in $\mathbb{R}^{k}$, then for some perhaps smaller neighborhood $Y$ of $0$ in $\mathbb{R}^{k}$, the neighborhood $V$ and the quantities $\gamma_{n}(I(\varphi_{\alpha},\psi_{\alpha},T))$ can be chosen independently of $\alpha$ in $Y$.}

\item[(iii)] {\em For all $T$ in $J_{1}$, for all $\varphi \in \mathbb{C}^{\infty}_{c}(G)$ and for all real-valued $\psi\in \mathbb{C}^{\infty}_{c}(G)$ such that $d\psi(\supp \varphi)\subseteq U$, the integral $I(\varphi,\psi,T)$ is rapidly decreasing as $t\to \infty$. If $\varphi$ and $\psi$ depend on a parameter $\alpha$ as in {\rm(ii)}, then there is uniformity in $\alpha$ as described there.}

\item[(iv)] {\em The same as {\em(iii)}, but is enough to choose an open neighborhood $V$ of $e$ and choose $\varphi\in \mathbb{C}^{\infty}_{c}(V)$.}

\item[(v)] {\em The same as {\em(iii)}, but we have the estimates}
\begin{equation*}
\gamma_{n}(I(\varphi,\psi,T))\leq c_{n}(\varphi,\psi)||T||_{1}\tag{1.18}\label{art3-eq1.18}
\end{equation*}
{\em for some number $c_{n}(\varphi,\psi)$. If $\alpha$ is an auxiliary parameter as described in {\em(ii)}, then the numbers $c_{n}(\varphi_{\alpha},\psi_{\alpha})$ may be bounded uniformly on compact sets of $\alpha$'s.}

\item[(vi)] {\em For $\varphi\in \mathbb{C}^{\infty}_{c}(G)$ and real-valued $\psi\in \mathbb{C}^{\infty}(G)$ such that $d\psi(\supp \varphi)\subseteq U$, the norm of the operator $\rho(\varphi e^{it\phi})$ is rapidly decreasing as $t\to \infty$. If $\varphi$ and $\psi$ depend on a parameter $\alpha$ as in {\em(ii)}, then the quantities $\gamma_{n}(||\rho(\varphi e^{it\psi}||)$ can be bounded uniformly on compact sets of $\alpha$.}

\item[(vii)] {\em Same as {\rm(vi)}, except it is enough to choose a neighborhood $V$ of $e$ and varify {\em(vi)} for $\varphi\in \mathbb{C}^{\infty}_{c}(V)$.}
\end{itemize}

\begin{proof}
First we will check that statements (ii) though (vii) are equivalent, then we will compare them with (i). It is immediate that (v) implies (iii) and that (iii) implies (iv). Likewise (vi) clearly implies (vii). Also, in view of formula \eqref{art3-eq1.3} and the duality between $J_{1}(H)$ and the space $L(H)$ of all bounded operators on $H$, we see that (v) and (vi) are equivalent. If $\psi\in \mathbb{C}^{\infty}(G)$, and $d\psi(e)\in U$, then $d\psi^{-1}(U)=V_{1}$ is a neighborhood of $e$\pageoriginale in $G$. If $V$ is as in (iv), then $V\cap V_{1}$ will be a neighborhood that works for (ii). Hence (iv) implies (ii).

Fix $\varphi\in \mathbb{C}^{\infty}_{c}(G)$ and $\psi\in \mathbb{C}^{\infty}(G)$. Suppose for any $T\in J_{1}$, the integral $I(\varphi,\psi,T)(t)$ is rapidly decreasing as $t\to \infty$. For any $n$, and number $a>0$, the set $X_{a}$ of $T$ such that $\gamma_{n}(I(\varphi,\psi,T))\leq a$ is convex and symmetric around $0$. Since $I(\varphi,\psi,T)(t)$ is continuous on $J_{1}$, we see that $X_{a}$ is also closed. Since $X_{ab}=bX_{a}$, and $\bigcup\limits_{a\geq 0}X_{a}=J_{1}$ by assumption, we see $X_{a}$ contains a neighborhood of the origin in $J_{1}$. Thus we see that (iii) implies (v).

We will show that (ii) implies (iii) by a partition of unity argument. Observe the identity
\begin{equation*}
I(\varphi,\psi,T)=I(L_{g}\varphi,L_{g}\psi,T\rho(g)^{-1}).\tag{1.19}\label{art3-eq1.19}
\end{equation*}
This follows from the definition of $I(\varphi,\psi,T)$ and formula \eqref{art3-eq1.5}. Suppose that $d\psi(\supp \varphi)\subseteq U$, so that $\varphi$ and $\psi$ satisfy the hypotheses of (iii). By formula \eqref{art3-eq1.8}, we see $d(L_{g-1}\psi)(e)\in U$ if $g\in \supp \varphi$. Then (ii) tells us that given $T\in J_{1}$, there is a neighborhood $V=V(L_{g-1}\psi,T\rho(g))$ such that if $\varphi'\in \mathbb{C}^{\infty}_{c}(V)$, then $I(\varphi',L_{g-1}\psi,T\rho(g))$ is rapidly decreasing. From \eqref{art3-eq1.19} we can conclude that for $g\in \supp \varphi$, there is a neighborhood $V_{g}$ of $g$ such that if $\varphi''\in \mathbb{C}^{\infty}_{c}(V_{g})$, then $I(\varphi'',\psi,T)$ is rapidly decreasing. We can cover $\supp \varphi$ with a finite number of the neighborhoods $V_{g}$, and construct a partition of unity subordinate to this cover of $\supp \varphi$. That is, we can find $g_{i}$ such that the $V_{g_{i}}$ cover $\supp \varphi$, and we can find $\varphi''_{i}\in \mathbb{C}^{\infty}_{c}(V_{g_{i}})$ such that $\sum\limits_{i}\varphi''_{i}=1$ on $\supp \varphi$. Then
$$
I(\varphi,\psi,T)=\sum\limits_{i}I(\varphi\varphi''_{i},\psi,T)
$$
so $I(\varphi,\psi,T)$ is rapidly decreasing. Clearly we can do this uniformly in some auxiliary parameter $\alpha$. Thus we see that (ii) implies (iii). A completely analogous, slightly simpler argument shows that (vii) implies (vi). Hence all conditions (ii) through (vii) are equivalent.

Finally we observe that by \cite{art3-D}, proposition \ref{art3-prop1.3.2}, for any point $\lambda\in U$, the condition that the point $(e,\lambda)\in T^{*}G$ not belong to $WF(\tr_{\rho}T)$ for $T\in J_{1}$, is just statement (ii) restricted to those $\psi$ such that $d\psi(e)=\lambda$ (and with parameter $\alpha$). Thus we see that (i) implies (ii). Conversely, (ii) certainly implies that $(e,\lambda)\not\in WF(\tr_{\rho}T)$ for any $T\in J_{1}$ and any $\lambda\in U$. Since\pageoriginale $U$ is open and $WF_{\rho}$ is $G$-biinvariant, we find that also (ii) implies (i). Thus the theorem is proved.

Using theorem \ref{art3-thm1.4} we can establish a relation between the wave front set of a representation and that of its restriction to a subgroup. Let $H\subseteq G$ be a Lie subgroup of $H$, with Lie algebra $\sfh$. We have the restriction map
$$
q:\mathfrak{g}^{*}\to \sfh^{*}
$$
Note that $q$ is $\Ad^{*}H$-equivariant, and if $H$ is normal in $G$, then $q$ is $\Ad^{*}G$-equivariant.
\end{proof}

\medskip
\noindent
{\bf Proposition \thnum{1.5}.\label{art3-prop1.5}}~{\em We have the inclusion}
\begin{equation*}
q(WF^{o}_{\rho})\subseteq WF^{o}(\rho\backslash H)\tag{1.21}\label{art3-eq1.21}
\end{equation*}

\begin{proof}
Let $U$ be an open subset of $h^{*}$ not intersecting $WF^{o}(\rho|H)$. Choose a small neighborhood $V$ of the identity in $G$ so that in $VH$ there is a smooth cross-section $Y$ to $H$, so we can write uniquely
$$
v=yh\quad v\in V, \ y\in Y, \ h\in H.
$$
Choose $\varphi\in \mathbb{C}^{\infty}_{c}(V)$ and let $\psi\in \mathbb{C}^{\infty}(V)$ be such that $d\psi\subseteq q^{-1}(U)$. We can compute
\begin{align*}
\rho(\varphi e^{it\psi}) &= \int\limits_{G}\varphi(g)e^{it\psi(g)}\rho(g)dg\tag{1.22}\label{art3-eq1.22}\\[4pt]
&= \int\limits_{Y}\int\limits_{H}\varphi(yh)e^{it\psi(yh)}\rho(yh)dydh\\[4pt]
&= \int\limits_{Y}\rho(y)\left(\rho|H(\varphi_{y}e^{it\psi}y)\right)dy
\end{align*}
where we have written
$$
\varphi_{y}(h)=\varphi(yh)\quad \psi_{y}(h)=\psi(yh)
$$
As $y$ varies, $\varphi_{y}$ varies smoothly in $\mathbb{C}^{\infty}_{c}(H)$ and $\psi$ varies smoothly in $\mathbb{C}^{\infty}(H)$, with $d\psi_{y}(\supp \varphi_{y})\subseteq U$. Hence by theorem \ref{art3-thm1.4}, part (vi), the norms of the operators
$$
\rho|H(\varphi_{y}e^{it\psi_{y}})
$$
are\pageoriginale rapidly decreasing as $t\to \infty$, with uniform estimates at least locally in $y$. Since $\varphi_{y}\equiv 0$ for $y$ outside a compact set we see from (22) that $\rho(\varphi e^{it\psi})$ is also rapidly decreasing at $t\to \infty$, whence $q^{-1}(U)$ is disjoint from $WF^{o}_{\rho}$, by theorem 4, part (vii).

An interesting aspect of proposition \ref{art3-prop1.5} is that it proceeds in the opposite direction from the standard results (\cite{art3-D} proposition 1.33, see also \cite{3-R-S}, section 1X.9) concerning restrictions of distributions and wave front sets. This contrast allows us to prove a partial converse to proposition \ref{art3-prop1.5}.

Let $\sfh^{\perp}$ be the kernel of the projection map $q$ of \eqref{art3-eq1.20}. We will say $H$ is {\em crosswise to} $\rho$ if $\sfh^{\perp}\cap WF^{o}_{\rho}=\{0\}$. When $H$ is crosswise to $\rho$ we can, according to \cite{art3-D}, proposition 1.3.3, restrict $\tr_{\rho}T$ (or any of its derivatives) to $H$. The wave front set of $(\tr_{\rho}T)|_{H}$, which will be the same as the wave front set of $\tr_{\rho}|_{H}(T)$, will then be contained in $q(WF(\tr_{\rho}T))$. Combining this with proposition \ref{art3-prop1.5} we may assert.
\end{proof}

\medskip
\noindent
{\bf Proposition \thnum{1.6}.\label{art3-prop1.6}}~{\em If $H$ is crosswise to $WF^{o}_{\rho}$, then}
\begin{equation*}
WF^{o}(\rho|H)=q(WF^{o}_{\rho})\tag{1.23}\label{art3-eq1.23}
\end{equation*}

Note that if $H$ is crosswise to $F$, the projection $q(WF_{\rho})$ will be closed.

In particular if $\sfh^{\perp}\cap V(\mathfrak{g}^{*})=\{0\}$, then $H$ will be crosswise to all irreducible $\rho$.

Proposition \ref{art3-prop1.5} also implies a restriction on the wave front set of (outer) tensor products. Let $G_{1}$ and $G_{2}$ be two Lie groups, and $\rho_{i}$ unitary representations of $G_{i}$ on spaces $H_{i}$. We can form the tensor product representation $\rho_{1}\otimes \rho_{2}$ of $G_{1}\times G_{2}$ on $H_{1}\times H_{2}$.

\medskip
\noindent
{\bf Proposition \thnum{1.7}.\label{art3-prop1.7}}~{\em We have the inclusion}
\begin{equation*}
WF^{o}(\rho_{1}\otimes \rho_{2})\subseteq WF^{o}_{\rho_{1}}\times WF^{o}_{\rho_{2}}\subseteq \mathfrak{g}_{1}\times \mathfrak{g}_{2}\tag{1.24}\label{art3-eq1.24}
\end{equation*}

\begin{proof}
We have $(\rho_{1}\otimes \rho_{2})|_{G_{1}}\simeq (\dim \rho_{2})\rho_{1}$. Hence by propositions \ref{art3-prop1.5} and \ref{art3-prop1.3}, we see
$$
WF^{o}(\rho_{1}\otimes \rho_{2})\subseteq WF^{o}_{\rho_{1}}\times \mathfrak{g}^{*}_{2}.
$$
Interchanging\pageoriginale $G_{1}$ and $G_{2}$, repeating and intersecting gives (24).

We remark that the inclusion \ref{art3-eq1.2.4} can be strict. An example of this will be found in part II.

For certain representations there is a plausible alternate definition of wave front set. We consider this and compare it with our first notion given above. Recall that there is an antiautomorphism$^{*}$ on $U(\mathfrak{g})$ defined property that it is $-1$ on $\mathfrak{g}$:
$$
x^{*}=-x\quad x\in \mathfrak{g}.
$$
If $\rho$ is a unitary representation of $G$, then
\begin{equation*}
\rho(u^{*})=\rho(u)^{*}\quad u\in U(\mathfrak{g})\tag{1.26}\label{art3-eq1.26}
\end{equation*}
where the$^{*}$ on the right-hand side indicates the restriction of the adjoint of $\rho(u)$ to the space of smooth vectors of $\rho$. Thus if $u=u^{*}$, then $\rho(u)$ is a symmetric operator, and elements of the form $u^{*}u$ are mapped to non-negative symmetric operators, and so are sums of such elements. We call sums $\sum u_{i}^{*}u_{i}$ in $U(\mathfrak{g})$ {\em formally positive}. Evidently the formally positive elements form a cone in $U(\mathfrak{g})$, invariant by$^{*}$.

In the following discussion we take $G$ to be unimodular for convenience. 

We will say that $\rho$ is of {\em strong trace class} if there is some formally positive element $v$ of $U(\mathfrak{g})$ such that $\rho(v)$ (with domain understood to the the smooth vectors of $\rho$) is essentially self-adjoint, and invertible with trace class inverse. We note irreducible representations are often of strong trace class.

If $\rho$ is of strong trace class, then for all $\varphi$ in $\mathbb{C}^{\infty}_{c}(G)$, the operator $\rho(\varphi)$  will be trace class, with trace norm satisfying
\begin{equation*}
||\rho(\varphi)||_{1}\leq ||\rho(v)^{-1}||_{1}||\rho(R_{v}\varphi)||\leq ||\rho(v)^{-1}||_{1}||R_{v}\varphi||_{1}\tag{1.27}\label{art3-eq1.27}
\end{equation*}
where $||\rho(\varphi)||_{1}$ indicates the trace norm on $J_{1}(H)$, and $v\in U(\mathfrak{g})$ is a formally positive element which makes $\rho$ strongly trace class, and $R_{v}$ is the left-invariant operator on $G$ corresponding to $v$, and $||\rho(R_{v}\varphi)||$ is the usual operator norm of $\rho(R_{v}\varphi)$ and $||R_{v}\varphi||_{1}$ is the $L^{1}$-norm of $R_{v}\varphi$ as a function on $G$. It is clear from \eqref{art3-eq1.27} that the trace linear functional 
\begin{equation*}
\chi_{\rho}(\varphi)=\tr_{\rho}(\varphi)\tag{1.28}\label{art3-eq1.28}
\end{equation*}\pageoriginale
is a distribution on $G$. We of course call it the {\em character} of $\rho$. We note that $\chi_{\rho}$ is a conjugation invariant distribution, in the sense that
\begin{equation*}
\chi_{\rho}(\Ad g(\varphi))=\chi_{\rho}(\varphi)\tag{1.29}\label{art3-eq1.29}
\end{equation*}
where $\Ad g(\varphi)=L_{g}R_{g}(\varphi)$.

If $\rho$ is of strong trace class, so that its character $\chi_{\rho}$ is well-defined as a distribution, then in the context of this paper, an obvious thing to do is to consider the wave front set $WF(\chi_{\rho})$. This will be a conjugation invariant set in $T^{*}G$. In particular the intersection of $WF(\chi_{\rho})$ with the cotangent space at the identity, which is canonically identifiable with $\mathfrak{g}^{*}$, defines a closed, $\Ad^{*}G$-invariant, conical set in $\mathfrak{g}^{*}$. Denote this set by $WF^{o}(\chi_{\rho})$. It is natural to compare this with our $WF^{o}_{\rho}$ defined earlier.
\end{proof}

\medskip
\noindent
{\bf Theorem \thnum{1.8}.\label{art3-thm1.8}}~{\em When $\rho$ is of strong trace class with distributional character $\chi_{\rho}$, we have}
\begin{equation*}
WF^{o}(\chi_{\rho})=WF^{o}_{\rho}.\tag{1.30}\label{art3-eq1.30}
\end{equation*}

\begin{proof}
Let $v$ be a formally positive element of $U(\mathfrak{g})$ with respect to which $\rho$ is strongly trace class. Write $\rho(v)^{-1}=T\in J_{1}(H)$. Then for $\varphi$ in $\mathbb{C}^{\infty}_{c}(G)$ we have
\begin{align*}
\chi_{\rho}(\varphi) &= \tr(\rho(\varphi))=\tr(T_{\rho}(v)\rho(\varphi))\\[3pt]
&= \tr(\rho(L_{v}(\varphi))T)=\tr_{\rho}(T)(L_{v}\varphi)\\[3pt]
&= L_{v^{*}}(\tr_{\rho}(T))(\varphi).
\end{align*}
In other words
\begin{equation*}
\chi_{\rho}=L_{v^{*}}(\tr_{\rho}(T))\tag{1.31}\label{art3-eq1.31}
\end{equation*}
Since action by differential operators does not increase the wave front set, we see
$$
WF(\chi_{\rho})\subseteq WF(\tr_{\rho}(T))\subseteq WF_{\rho}.
$$
Hence, looking at the fibre of $T^{*}G$ over the identity of $G$ we see that the left side of \eqref{art3-eq1.30} is contained in the right side.

To prove the reverse inclusion, consider a point $p$ in $\mathfrak{g}^{*}-WF^{o}(\chi_{\rho})$. 

Let\pageoriginale $U$ be a neighborhood of $p$ with compact closure disjoint from $WF^{o}(\chi_{\rho})$. Since $WF(\chi_{\rho})$ is closed in $T^{*}G$, there is a neighborhood $V$ of the identity $e$ in $G$ such that $V\times U\subseteq T^{*}G$ is disjoint from $WF(\chi_{\rho})$. It follows that for $\varphi$ in $\mathbb{C}^{\infty}_{c}(V)$ and real-valued $\psi$ in $\mathbb{C}^{\infty}(V)$ such that $d\psi$ $(\supp \varphi)\subseteq U$, one has that $\chi_{\rho}(\varphi e^{it\psi})$ is rapidly decreasing as $t\to \infty$, with estimates uniform in smooth parametrized families of $\varphi$'s and $\psi$'s. Let $V_{1}$ be a symmetric neighborhood of $e$ such that $V^{2}_{1}\subseteq V$. Then if $\varphi\in \mathbb{C}^{\infty}_{c}(V_{1})$, we see that $\chi_{\rho}(L_{g}(\varphi e^{it\psi}))$ is rapidly decreasing in $t$, uniformly in $g$ in $V_{1}$ and in any other auxiliary parameter of interest. Set
$$
\varphi e^{it\psi}=\varphi_{t}\quad\text{and}\quad \varphi^{*}_{t}(g)=\overline{\varphi}_{t}(g^{-1})
$$
where------indicates complex conjugation.

Integrating, we find
\begin{align*}
&\int\limits_{G}\overline{\varphi}_{t}(g^{-1})\chi_{\rho}(L_{g}\varphi_{1})dg=\int\limits_{G}\chi_{\rho}(\overline{\varphi}_{t}(g^{-1})(L_{g}\varphi_{t})dg\tag{1.32}\label{art3-eq1.32}\\[3pt]
&\quad = \chi_{\rho}(\varphi^{*}_{t}*\varphi_{t})=\tr(\rho(\varphi^{*}_{t}*\varphi_{t}))=\tr(\rho(\varphi_{t})^{*}\rho(\varphi_{t}))
\end{align*}
is rapidly decreasing as $t\to \infty$. Here $\varphi^{*}_{t}*\varphi_{t}$ indicates the convolution of these functions. But the final expression in \eqref{art3-eq1.32} is just the Hilbert-Schmidt norm of $\rho(\varphi_{t})$. Since it is rapidly decreasing, the operator norm of $\rho(\varphi_{t})$ is also. Hence criterion (vii) of Theorem \ref{art3-thm1.4} tells us $U$ is disjoint from $WF^{o}_{\rho}$, and Theorem \ref{art3-thm1.8} is established.
\end{proof}

Before concluding this section, let us mention two plausible general properties of wave front sets not established here. First, is it true that $WF^{o}(\rho_{1}\otimes \rho_{2})\subseteq (WF^{o}_{\rho_{1}}+WF^{o}_{\rho_{2}})\overline{\quad}$(the $\overline{\quad}$ here denoting closure) for an inner tensor product? Second, is it true that $WF^{o}(\ind^{G}_{H}\sigma)\supseteq \sfh^{\perp}$?

\section{Examples}\label{art3-sec2}

Here we will show how to compute $WF^{o}_{\rho}$ for various familiar classes of groups, and examine the possibilities for $WF^{o}_{\rho}$ in some interesting cases.

\medskip
\noindent
{\bf A. Abelian Groups.}
\smallskip

If $G$ is abelian, then $G$ is a homomorphic image of a vector space $V$, so we may as well assume $G=V$. Then we may identify $V$ with its Lie algebra.\pageoriginale Also the dual vector space $V^{*}$ can be identified with $\widehat{V}$, the Pontrjagin dual of $V$, by the usual method. Define
$$
\alpha : V^{*}\to \widehat{V}
$$
by
\begin{equation*}
\alpha(\lambda)(v)=e^{2\pi i\lambda(v)}\quad \lambda\in V^{*},v\in V.\tag{2.1}\label{art3-eq2.1}
\end{equation*}

Define Fourier transform from $L_{1}(V)$ to $\mathbb{C}_{0}(V^{*})$ by the usual recipe:
\begin{equation*}
\widehat{\varphi}(\lambda)=\int\limits_{V}\varphi(v)e^{-2\pi i\lambda(v)}dv\quad \varphi\in L_{1}(V), \ \lambda\in V^{*}\tag{2.2}\label{art3-eq2.2}
\end{equation*}
Then the inverse Fourier transform is
\begin{equation*}
\widehat{f}^{-1}(v)=\int\limits_{V^{*}}f(\lambda)e^{2\pi i\lambda(v)}dv\quad f\in L_{1}(V^{*}), \ v\in V\tag{2.3}\label{art3-eq2.3}
\end{equation*}

Let $\rho$ be a unitary representation of $V$ on the Hilbert space $H$. Take $T\in J_{1}(H)$, and consider the matrix coefficient $\tr_{\rho}(T)$. According to Bochner's Theorem \cite{art3-R-S}, $\tr_{\rho}(T)\spcheck$ exits as a finite measure on $V^{*}$, positive if $T$ is. Moreover from our formulas \eqref{art3-eq1.5} and \eqref{art3-eq2.2} we can compute that
\begin{equation*}
\tr_{\rho}(\rho(\varphi)T)\sphat \  =(\check{\varphi})\sphat \ \ (\tr_{\rho}T)\sphat\tag{2.4}\label{art3-eq2.4}
\end{equation*}
where
\begin{equation*}
\varphi(v)=\check{\varphi}(-v)\tag{2.5}\label{art3-eq2.5}
\end{equation*}

We define $\supp \rho$ to be the closure of the union of the supports of the measures $(\tr (T))\sphat$. It is clear from \eqref{art3-eq2.4} that
\begin{equation*}
|\big| \rho(\varphi)\big||=\sup \{(\check{\varphi})\sphat \ (\lambda):\lambda \in \supp \rho\}\tag{2.6}\label{art3-eq2.6}
\end{equation*}

Given a set $S$ in a vector space $U$, define $AC(S)$, the {\em asymptotic cone} of $S$ as follows. Given $u$ in $U$, if any cone containing a neighborhood of $u$ intersects $S$ in an unbounded set, then $u$ is in $AC(S)$.

In terms of these objects we can give the not unexpected description of $WF_{\rho}$.

\medskip
\noindent
{\bf Proposition \thnum{2.1}.\label{art3-prop2.1}}~{\em For a unitary representation $\rho$ of a vector space $V$, one has}
\begin{equation*}
WF^{o}_{\rho}=-AC(\supp \varphi)\tag{2.7}\label{art3-eq2.7}
\end{equation*}\pageoriginale

\begin{remark*}
The minus sign in \eqref{art3-eq2.7} is an artifact of our conventions and could be eliminated by appropriate juggling.
\end{remark*}

\begin{proof}
We will apply criterion (vi) of Theorem \ref{art3-thm1.4}, with $\psi=2\pi \lambda$, $\lambda\in V^{*}$. (We will actually use the definition 1.3.1 of \cite{art3-D} rather than proposition 1.3.2 used for Theorem 4). Take $\varphi$ in $\mathbb{C}^{\infty}_{c}(V)$. Then one sees from \eqref{art3-eq2.2} that
\begin{equation*}
\left((\varphi e^{2\pi i\lambda})\spcheck\right)\sphat \ (\lambda')=(\check{\varphi})\sphat \ (\lambda'+t\lambda)\tag{2.8}\label{art3-eq2.8}
\end{equation*}
Suppose that $\lambda_{0}\not\in -AC(\supp \rho)$. Then we can choose a small neighborhood $U$ of $\lambda_{0}$ such that the distance between $-t\lambda$ and $\supp\rho$ (in any convenient norm) increases linearly in $t$. Therefore $-t\lambda$ has a ball around it of size $\geq \gamma t$, $\gamma$ being some constant independent of $\lambda$, disjoint from $\supp\rho$. Since $(\check{\varphi})\sphat$ is rapidly decreasing for $\varphi\in \mathbb{C}^{\infty}_{c}(V)$, we see from formulas \eqref{art3-eq2.6} and \eqref{art3-eq2.8} that $||\rho(\varphi e^{2\pi it\lambda})||$ decreases rapidly as $t\to \infty$. This shows the left side of \eqref{art3-eq2.7} is contained in the right side. The reverse inclusion is equally easy. If $-\lambda_{0}\in AC(\supp \rho)$, then no matter how small a neighborhood $U$ of $\lambda_{0}$ we choose, the cone on $-U$ will intersect $\supp \rho$ in a non-bounded set. This means we can choose $t$ arbitrarily large, and $\lambda_{1}$ in $U$, such that $-t\lambda_{1}$, is in $\supp \rho$. We may assume for convenience that $\varphi$ is positive-definite, so that $\widehat{\varphi}(0)=||\widehat{\varphi}||_{\infty}$. Then we see that $||\rho(\varphi e^{2\pi it \lambda_{1}})||=||\widehat{\varphi}||_{\infty}$ by formulas \eqref{art3-eq2.6} and \eqref{art3-eq2.8}, so that $U$ violates condition (vi) of Theorem \ref{art3-thm1.4}. Hence the right side of \eqref{art3-eq2.7} is contained in the left side, and the proposition is proved.
\end{proof}

We will use proposition \ref{art3-prop2.1} to give an example of strict inclusion in proposition \ref{art3-prop1.7}. Let $V=\mathbb{R}$, and let $\sfN$ be the direct sum of the characters
$$
t\to e^{2\pi in!t}\quad n\geq 1
$$
Then 
$$
\supp \rho=\{n!,n\in \mathbb{Z}^{+}\}
$$
Hence by proposition \ref{art3-prop2.1}, we have
$$
WF^{o}(\rho)=-AC(\supp \rho)=\mathbb{R}^{-}=\{t\in \mathbb{R}, t\leq 0\}.
$$
Consider\pageoriginale the tensor product $\rho\otimes \rho$ as a representation of $\mathbb{R}^{2}$. Then clearly
$$
\supp(\rho\otimes \rho)=\{(n!,m!):n,m\in \mathbb{Z}^{+}\}.
$$
It is easy to see that $AC(\supp (\rho\otimes \rho))$ consists of the positive $x$-axis, the positive $y$-axis, and the positive ray of the $45^{\circ}$ line $x=y$. Thus $WF^{o}(\rho\otimes\rho)$, being the negatives of these 3 rays, is properly contained in $WF^{o}_{\rho}\times WF^{o}_{\rho}$, which is the whole southwest quadrant.

\bigskip
\noindent
{\bf B: Nilpotent Groups.}
\smallskip

We will discuss only irreducible representations of general nilpotent groups. Let $N$ be a nilpotent Lie group, assumed to be connected and simply connected for simplicity. Let $\sfN$ be its Lie algebra, and $\exp:\sfN\to N$ the exponential map. Let $\rho$ be an irreducible representation of $N$. It is known that $\rho$ is of strong trace class, and according to the orbit theory of Kirillov \cite{art3-K}, there is an $\Ad^{*}N$ orbit $O(\rho)=O$ in $\sfN^{*}$, such that
\begin{equation*}
\chi_{\rho}(\varphi)=\int\limits_{O}(\varphi O \exp)\sphat \ \ (\lambda) do(\lambda) \ \ \varphi\in \mathbb{C}^{\infty}_{c}(N)\tag{2.9}\label{art3-eq2.9}
\end{equation*}
where $\chi_{\rho}$ is the character of $\rho$, as in \eqref{art3-eq1.27}, and$\sphat$ \ is as in \eqref{art3-eq2.2}, and do is a properly normalized $\Ad^{*}N$ invariant measure on $O$. Given formula \eqref{art3-eq2.9} and theorem \ref{art3-thm1.8}, it is an easy matter to establish the following result. We omit the details.

\medskip
\noindent
{\bf Proposition \thnum{2.2}.\label{art3-prop2.2}}~{\em If $\rho$ is an irreducible representation of $N$, and $O\subseteq \sfN^{*}$ is the associated orbit, then}
\begin{equation*}
WF^{o}_{\rho}=-AC(O).\tag{2.10}\label{art3-eq2.10}
\end{equation*}

\bigskip
\noindent
{\bf C. Compact Groups.}
\smallskip

Let $K$ be a compact connected Lie group, and let $T\subseteq K$ be a maximal torus. Let $W$ be the Weyl group of $T$, the normalizer of $T$ modulo the centralizer of $T$. Let $\sft$ and $\sfk$ be the Lie algebras of $T$ and $K$. If $K$ is semi-simple we can identify $\sfk$ with $\sfk^{*}$ via the Killing form. In general, we will suppose given some $\Ad K$-invariant, negative definite, bilinear form on $\sfk$ allowing us to identify $\sfk$ and $\sfk^{*}$. Then we can also identify $\sft$ and $\sft^{*}$, and may regard $\sft^{*}$ as a subspace of $\sfk^{*}$, and we will have
\begin{equation*}
\Ad^{*}K(\sft^{*})=\sfk^{*}\tag{2.11}\label{art3-eq2.11}
\end{equation*}
Thus\pageoriginale any $\Ad^{*}K$ invariant set in $\sfk^{*}$ is determined by its intersection with $\sft^{*}$, and this intersection will be a Weyl group invariant set. Fix a Weyl chamber $C^{+}$ in $\sft^{*}$, and fix an ordering of the roots of $\sft$ by letting this chosen Weyl chamber be positive. We have
\begin{equation*}
\Ad^{*}W(C^{+})=\sft^{*}\tag{2.12}\label{art3-eq2.12}
\end{equation*}
so that an $\Ad^{*}K$ invariant set in $\sfk^{*}$ is determined by its intersection with $C^{+}$.

The irreducible representations of $K$ are described by the celebrated highest weight theory of Cartan and Weyl. Let $\widehat{T}$ be the character group of $T$. Since $T$ is a quotient of $\sft$ via the exponential map, we can as described in paragraph IIA identify $\widehat{T}$ with a lattice in $\sft^{*}$, the so-called lattice of weights. The intersection
$$
\widehat{T}^{+}=\widehat{T}\cap C^{+}
$$
is called the set of dominant weights. The dominant weights parametrize the set $\widehat{K}$ of irreducible unitary representations of $K$. We recall how.

Let $\sfk_{\mathbb{C}}$ be the complexification of $\sfk$. We can write
\begin{equation*}
\sfk_{\mathbb{C}}=\sft_{\mathbb{C}}\otimes \sum\limits_{\alpha}L_{\alpha}\tag{2.13}\label{art3-eq2.13}
\end{equation*}
where the $L_{\alpha}$ are the root spaces, that is, the non-trivial eigenspaces of $\Ad T$ acting on $\sfk_{\mathbb{C}}$. We parametrize $L_{\alpha}$ by the character $\alpha$ it defines, and we regard $\alpha$ as an element of $\sft^{*}$ as explained above. We call a root $\alpha$ positive if $(\alpha,c)\leq 0$ for all $c\in C^{+}$, where (,) is the posited bilinear form by means of which we identified $\sfk$ and $\sfk^{*}$.

Denote the set of positive roots by $\sum^{+}$. Put
\begin{equation*}
\sfN^{+}=\bigoplus\limits_{\alpha\in \sum^{+}}L_{\alpha}\tag{2.14}\label{art3-eq2.14}
\end{equation*}
Then $\sfN^{+}$ is a nilpotent subalgebra of $\sfk_{\mathbb{C}}$, and it is known that
\begin{equation*}
\sfk_{\mathbb{C}}=\sft_{\mathbb{C}}\oplus \sfN^{+}\oplus \sfN^{-}\tag{2.15}\label{art3-eq2.15}
\end{equation*}
where $\sfN^{-}$ is the image of $\sfN^{+}$ under complex conjugation in $\sfk_{\mathbb{C}}$. Let $\rho$ be a representation of $K$ on a Hilbert space $H$. Denote by $H^{+}$ the subspace of $H$ annihilated by all elements of $\sfN^{+}$. The space $H^{+}$ is the space of highest weight vectors for $\rho$. Clearly $H$ is invariant by $\rho(T)$, so it may be\pageoriginale decomposed into a direct sum
\begin{equation*}
H^{+}=\sum\limits_{\gamma}H^{+}_{\gamma}\tag{2.16}\label{art3-eq2.16}
\end{equation*}
where $H^{+}_{\gamma}$ is the eigenspace of $T$ on which $T$ acts by the character $\gamma\in \widehat{T}$. The highest weight theory asserts the following facts:
\begin{itemize}
\item[(i)] Each $\gamma$ is in $\widehat{T}^{+}$

\item[(ii)] If $\rho$ is irreducible, then $\dim H^{+}=1$, so that $H^{+}=H^{+}_{\gamma}$ for some well-defined $\gamma$.

\item[(iii)] The map from $\widehat{K}$ to $\widehat{T}^{+}$ implied by (ii) is a bijection.
\end{itemize}

Now consider an arbitrary unitary representation $\rho$ of $K$. Denote the set of highest weights of $\rho$ by $\supp \rho$. Thus
$$
\supp \rho\subseteq \widehat{T}^{+}\subseteq C^{+}
$$
The following result is very closely akin to results in \cite{art3-K-V}.

\medskip
\noindent
{\bf Proposition \thnum{2.3}.\label{art3-prop2.3}}~{\em For a unitary representation $\rho$ of $K$, we have}
\begin{align*}
& -WF^{o}_{\rho}\cap C^{+}=AC(\supp \rho),\quad\text{or}\tag{2.17}\label{art3-eq2.17}\\[3pt]
& WF^{o}_{\rho}=\Ad^{*}K(-AC(\supp \rho))
\end{align*}

\begin{proof}
By proposition \ref{art3-prop1.3}, it suffices to prove this when $\rho$ is multiplicity free, that is, when $\rho$ contains only one copy of each of its irreducible constituents. Then $H^{+}$ (the space of $\rho$ being $H$ as usual) will be multiplicity free under the action of $T$. Let $\sigma$ denote the representation of $T$ on $H^{+}$. Then by definition $\supp \sigma=\supp \rho$.

Let $x$ and $y$ be two vectors in $H^{+}$. Consider the matrix coefficient $\tr_{\rho}(E_{x,y})$. Since the intersection of the characteristics of the elements of $\sfN^{+}$ is just $\sft^{*}$, and since $\tr_{\rho}(E_{x,y})$ is annihilated by $\sfN^{+}$ (acting either on the right or the left), we see by \cite{art3-D}, proposition 5.1.1, that the wave-front set of $\tr_{\rho}(E_{x,y})$ at the identity of $K$ is contained in $\sft^{*}$. This also implies by \cite{art3-D}, proposition 1.3.3, that $\tr_{\rho}(E_{x,y})$ restricts to $T$; this restriction must of course just equal to $\tr_{\sigma}(E_{x,y})$. One then has again by \cite{art3-D}, proposition 1.3.3.
\begin{equation*}
WF^{o}(\tr_{\sigma}(E_{x,y}))\subseteq WF^{o}(\tr_{\rho}(E_{x,y}))\subseteq WF^{o}_{\rho}\tag{2.18}\label{art3-eq2.18}
\end{equation*}
where in the first two expressions the $^{o}$ in $WF^{o}$ mean we are looking at the\pageoriginale fibre over the identity in $K$. From \eqref{art3-eq2.18} we immediately have
\begin{equation*}
WF^{o}\sigma \subseteq WF^{o}_{\rho}\tag{2.19}\label{art3-eq2.19}
\end{equation*}
Since $WF^{o}$ is $\Ad^{*}K$ invariant, we see by proposition \ref{art3-prop2.1} that the left side of \eqref{art3-eq2.17} contains the right side.

On the other hand, since $\rho$ is multiplicity free, it is of strong trace class, so to compute $WF^{o}_{\rho}$, it is enough by Theorem \ref{art3-thm1.8} to compute $WF^{o}(\chi_{\rho})$. Let $\Delta$ be the element in $U(\sfk)$ corresponding to our given bilinear form. Then $R_{\Delta}$ is elliptic, and $\rho(1+\Delta)$ is positive definite, and some power of $\rho(1+\Delta)$ has trace class inverse. Standard and straightforward arguments allow us to find $x\in H^{+}$ such that for some sufficiently large $l$ we have
\begin{equation*}
\chi_{\rho}=R_{(1+\Delta)}l\int\limits_{K}\Ad K(\tr_{\rho}(E_{x,x}))dk\tag{2.20}\label{art3-eq2.20}
\end{equation*}
Since $R_{\Delta}$ is elliptic, we have
\begin{align*}
WF(\chi_{\rho}) &= WF\left(\int\limits_{K}\Ad K(\tr_{\rho}(E_{x,x}))dk\right)\tag{2.21}\label{art3-eq2.21}\\
&\subseteq \Ad K(WF(\tr_{\rho}(E_{x,x})))
\end{align*}
Hence if we can show
\begin{equation*}
WF^{o}(\tr_{\rho}(E_{x,x}))\subseteq WF^{o}\sigma\tag{2.22}\label{art3-eq2.22}
\end{equation*}
we will be done. In fact \eqref{art3-eq2.22} is proven in just the same manner as proposition \ref{art3-prop1.5}. The reasoning is exactly the same as in equation \eqref{art3-eq1.22}, except instead of considering simply $\rho(\varphi e^{it\psi})$, one looks at the product $\rho(\varphi e^{it\psi})E_{x,x}$.
\end{proof}

\bigskip
\noindent
{\bf D: Semisimple Groups.}
\smallskip

We come now to the motivating examples of this paper. Let $G$ be a semisimple Lie group with finite center and with Iwasawa decomposition 
\begin{equation*}
G=KAN\quad \mathfrak{g}=\sfk\oplus \sfa\oplus \sfN\tag{2.23}\label{art3-eq2.23}
\end{equation*}
One also has the Cartan decomposition
\begin{equation*}
\mathfrak{g}=\sfk\oplus \sfp=\sfk\oplus \Ad K(\sfa)\tag{2.24}\label{art3-eq2.24}
\end{equation*}
where\pageoriginale $\sfp$ is the orthogonal complement of $\sfk$ with respect to the Killing form of $\mathfrak{g}$. We identify $\mathfrak{g}$ with $\mathfrak{g}^{*}$ via the Killing form. Thus in what follows we will speak of $\mathfrak{g}$ when strictly we should say $\mathfrak{g}^{*}$.

Let $N$ be the nilpotent set of $\mathfrak{g}$. It is well known that $N=V(\mathfrak{g})$ is the characteristic variety of $\mathfrak{g}$, in the sense of proposition \ref{art3-prop1.2}. It is also known that there are only finitely many conjugacy classes of nilpotent elements. Thus from proposition \ref{art3-prop1.2} we have the following result.

\medskip
\noindent
{\bf Proposition \thnum{2.4}.\label{art3-prop2.4}}~{\em If $\rho$ is an irreducible unitary representation of $G$, then}
\begin{equation*}
WF^{o}_{\rho}\subseteq N.\tag{2.25}\label{art3-eq2.25}
\end{equation*}
{\em In particular, there are only finitely many possibilities for $WF^{o}_{\rho}$.}

Let $\rho$ be irreducible, and consider $\rho/K$. It is a classic result of Harish-Chandra (see \cite{art3-W}) that $\rho/K$ contains each irreducible representation of $K$ a finite number of times, and that in fact $\rho/K$ is of strong trace class. These facts are also reflected in the behavior of wave front sets. From the Cartan decomposition \eqref{art6-eq2.24}, noting that $\sfp$ consists of semisimple elements, we see that $\sfk$ is crosswise to $WF^{o}_{\rho}$ in the sense of proposition \ref{art3-prop1.6}. Thus we have the following immediate consequence of that result.

\medskip
\noindent
{\bf Proposition \thnum{2.5}.\label{art3-prop2.5}}
\begin{itemize}
\item[(a)] {\em For irreducible $\rho$ we have}
\begin{equation*}
WF^{o}(\rho/K)=q(WF^{o}\rho)\tag{2.26}\label{art3-eq2.26}
\end{equation*}
{\em where $q$ is orthogonal projection of $\mathfrak{g}$ onto $\sfk$ (with kernel $\sfp$)}.

\item[(b)] {\em In particular $WF^{o}(\rho/K)$ is the orthogonal projection on $K$ of certain nilpotent orbits in $\mathfrak{g}$, and is one of only finitely many possibilities.}
\end{itemize}

\begin{remark*}
Part (b) of this proposition is very similar to proposition of \cite{art3-K-V}. However, the proof is substantially different from the proof in \cite{art3-K-V}. Also, Kashiwara-Vergne do not relate (their version of) $WF^{o}(\rho/K)$ to an object on $G$ attached intrinsically to $\rho$. We note that $WF^{o}_{\rho}$ is a finer invariant than $WF^{o}(\rho/K)$, as simple examples already on $SL_{3}(\mathbb{R})$ show. (However $WF$ is not finer than $\rho/K$, when multiplicities are taken into account. It seems to be roughly equivalent to $WF^{o}(\rho/K)$ plus some rough information on multiplicities. See the discussion below of the analogy with the $op$-adic case). Furthermore $WF^{o}_{\rho}$ provides a link between the\pageoriginale $N$-spectrum and the $K$-spectrum of $\rho$, empirical observation of which was the original motivation of Kashiwara-Vergne. Indeed applying proposition \ref{art3-prop1.5} to $N$, and using proposition \ref{art3-prop2.5} we arrive at the following fact. Note that the Killing form induces the identification
\begin{equation*}
\sfN^{*}\simeq \mathfrak{g}/(\sfa\oplus \sfN)\tag{2.27}\label{art3-eq2.27}
\end{equation*}
Let
\begin{equation*}
q':\mathfrak{g}\to \mathfrak{g}/(\sfa\oplus \sfN)\tag{2.28}\label{art3-eq2.28}
\end{equation*}
be the natural quotient map.
\end{remark*}

\medskip
\noindent
{\bf Proposition \thnum{2.6}.\label{art3-prop2.6}}~{\em We have the inclusion}
\begin{equation*}
WF^{o}(\rho/K)\subseteq q({q'}^{-1}(WF^{o}(\rho/N))\tag{2.29}\label{art3-eq2.29}
\end{equation*}

Acutally, this proposition is true with $N$ replaced by any subgroup of $G$. 

To illustrate the above results, we offer some observations about the symplectic group $\Sp_{2n}(\mathbb{R})=\Sp$. This is the subgroup of $\GL_{2n}$ preserving the standard symplectic form on $\mathbb{R}^{2n}$. Similar ideas apply to other classical groups. Each element of $\sfs\sfp$ may be regarded as a linear transformation on $\mathbb{R}^{2n}$ in the obvious way, and as such may be assigned its rank, a positive integer. Given an irreducible representation $\rho$ of $\Sp$, we define the {\em singular rank} of $\rho$ to be the maximum of the rank of the elements of $WF^{o}_{\rho}$.

There are other useful notions of rank for $\rho$ also. Let $X$ be a maximal isotropic subspace of $\mathbb{R}^{2n}$, and let $N_{1}$ be the subgroup of $\Sp$ that leaves $X$ fixed pointwise. It is well known that $N_{1}\simeq S^{2}(X)$, the second symmetric power of $X$. Hence $\sfN^{*}_{1}$ is identifiable to the space of symmetric bilinear forms on $X$, and each of its elements has a well-defined rank. We will say a representation $\rho$ of $\Sp$ has $N_{1}$-{\em rank} $j$ if $WF^{o}(\rho/N_{1})$ contains elements of rank $j$, but none of rank greater than $j$. By means of Proposition \ref{art3-prop2.1} and a little symplectic geometry, the notion of $N_{1}$-rank can be made considerably more concrete. It is discussed at more length in \cite{art3-H2}.

The singular rank of $\rho$ can vary from zero to $2n-1$, while the $N_{1}$-rank can vary only from zero to $n$. However, within their common range, they are closely related.

\medskip
\noindent
{\bf Proposition \thnum{2.7}.\label{art3-prop2.7}}~{\em Given\pageoriginale irreducible $\rho$ with $N_{1}$ rank less than $n$, one has the inequality}
\begin{equation*}
\text{singular rank $(\rho)\leq N_{1}$ rank $(\rho)$.}\tag{2.30}\label{art3-eq2.30}
\end{equation*}

Probably \eqref{art3-eq2.30} should be an equality.

\begin{proof}
By applying proposition \ref{art3-prop1.5}, we reduce the proof to an exercize in symplectic geometry. Let $Y$ be an isotropic subspace of $\mathbb{R}^{2n}$ complementary to $X$. Let $N^{-}_{1}$ be the subalgebra of $\sfs\sfp$ annihilating $Y$, and let $\sfm$ be the subalgebra of $\sfs\sfp$ stabilizing $X$ and $Y$. Then
\begin{equation*}
\sfs\sfp=N^{-}_{1}\oplus \sfm \oplus \sfN_{1}.\tag{2.31}\label{art3-eq2.31}
\end{equation*}
Also the orthogonal complement of $\sfN_{1}$ in $\sfs\sfp$ with respect to the Killing form is just $\sfm\oplus \sfN_{1}$, so we may identify $\sfN^{-}_{1}$ with $\sfN^{*}_{1}$. Thus what we want to show is that in a nilpotent $\Ad \Sp$ orbit in $\sfs\sfp$ consisting of elements of rank $l<n$, there are elements whose $N^{-}_{1}$ component in the decomposition \eqref{art3-eq2.31} has rank $l$ also. The rank of the $\sfN^{-}_{1}$ component of $s\in \sfs\sfp$ is easily seen to be
\begin{equation*}
\rank (s/X)-\dim (s(X)\cap X)\tag{2.32}\label{art3-eq2.32}
\end{equation*}
Hence, reversing the roles of $s$ and $X$, it will suffice, given $s$ of rank $l<n$ to find a maximal isotropic $X$ such that \eqref{art3-eq2.32} also equals $l$. This entails
\begin{equation*}
\rank (s/X)=l\quad s(X)\cap X=0.\tag{2.33}\label{art3-eq2.33}
\end{equation*}
Consider the action of $s$. It is elementary that
$$
\text{im~}s=(\ker s)^{\perp}
$$
where $^{\perp}$ indicates orthogonal complement for the standard symplectic form on $\mathbb{R}^{2n}$. Hence $Z=\text{im~}s\cap\ker s$ is isotropic, and
$$
\text{im~}s+\ker s=Z^{\perp}.
$$
We can write
$$
\mathbb{R}^{2n}=Z\oplus V_{1}\oplus V_{2}\oplus \widetilde{Z}
$$
where $V_{1}$ is a complement to $Z$ in im $s$, and $V_{2}$ is a complement to $Z$ in $\ker s$, and $\widetilde{Z}$ is an isotropic complement to $Z$ in $(V_{1}\oplus V_{2})^{\perp}$. The assumption\pageoriginale that rank $s<n$ implies $\dim V_{1}<\dim V_{2}$. Hence we may choose an embedding $\alpha:V_{1}\to V_{2}$ such that $\langle \alpha(v_{1}),\alpha(v'_{1})\rangle=-\langle v_{1},v'_{1}\rangle$, where $\langle ,\rangle$ denotes the symplectic form on $\mathbb{R}^{2}$. The space $U_{1}$ of points
$$
U_{1}=\{v+\alpha(v):v\in V_{1}\}
$$
is then isotropic. Let $U_{2}$ be a maximal isotropic subspace of $\alpha(V_{1})^{\perp}\cap V_{2}$. Then $X=U_{1}\oplus U_{2}\oplus \widetilde{Z}$ is a maximal isotropic subspace of $\mathbb{R}^{2n}$, and it is easy to check it satisfies the conditions \eqref{art3-eq2.33}.

To illustrate proposition \ref{art3-prop2.7}, consider the two components of the oscillator representation \cite{art3-H3}. It is an easy matter to compute their $N_{1}$ spectrum, and in particular to see they have $N_{1}$ rank equal to one, hence by the proposition, singular rank equal to one. (Singular rank zero would imply only a finite number of $K$-types, hence finite dimensionality). There are only two conjugacy classes of rank one nilpotent elements, the transvections
$$
ty:x\to \langle x,y\rangle y\quad x, \ y\in \mathbb{R}^{2n}
$$
where $\langle,\rangle$ denotes the symplectic form, forming one and their negatives forming the other. The two holomorphic oscillator representations have the class of transvections for their wave front set, and the antiholomorphic oscillator representations have the negatives of the transvections for their wave front set. More generally the representation of $\Sp_{2n}$ coming from its pairing with $O_{p,q}$ inside $\Sp_{2n(p+q)}$ (see \cite{art3-H3}) has $N_{1}$ rank equal to $p+q$, with obvious consequences for the wave-front sets of its irreducible components if $p+q<n$.

We will conclude the paper with a few remarks. First, the analogy of the present discussion with the results of \cite{art3-H4} and \cite{art3-HC} should be pointed out. In those papers it is shown that for an irreducible representation $\rho$ of a reductive $p$-adic group $G$, over a field of characteristic zero, the character $\chi_{\rho}$ of $\rho$ has an ``asympotic expansion'', valid in a neighborhood of the identity. This expansion expresses $\chi_{\rho}$ as a linear combination of distributions attached in a direct way to nilpotent conjugacy classes in the Lie algebra $\mathfrak{g}$ of $G$. From this expansion, one can read off directly information about the asymptotics of the $K$-spectrum, or the $N$-spectrum, where $K$ is a maximal compact subgroup of $G$, and $N$ a unipotent subgroup. This expansion\pageoriginale is thus comparable to the wave-front set, but it is more precise in two ways. First, it permits more precise description of the $K$ -- or $N$ -- spectra than does the wave front set. Secondly, it attaches to $\rho$ not simply a closed set of nilpotent orbits, but a collection of individual orbits with numbers, which might be thought of as multiplicities, attached. It would clearly be desirable to have an analogue of this expansion for groups over $\mathbb{R}$. It seems that Barbasch and Vogan \cite{art3-BV} have established the existence of such an analogue.

A second analogy that might be made is with the characteristic variety of a primitive ideal of a semisimple Lie algebra, as discussed by Borho and Kraft in \cite{art3-B-K}. It would seem the wave front set is the analytical analogue of their construction.  
\end{proof}

\begin{thebibliography}{99}
\bibitem[BV]{art3-BV} BARBASCH D. and D. VOGAN, {\em The local structure of characters,} preprint.

\bibitem[BK]{art3-BK} BORHO W. and H. KRAFT, Uber die Gelfand-Kirillov--Dimension, {\em Math. Ann.} 22 (1976), 1--24.

\bibitem[D]{art3-D} DUISTERMAAT J. {\em Fourier Integral Operators}, Courant Institute of Math. Sci., New York 1973.

\bibitem[HC]{art3-HC} HARISH CHANDRA, The characters of reductive p-adic groups, in {\em Contributions to Algebra}, Academic Press, New York (1977), 175-182.

\bibitem[H1]{art3-H1} HOWE R. On a connection between nilpotent groups and oscillatory integrals associated to singularities, {\em Pac. J. Math.} 73 (1977), 329--364. 

\bibitem[H2]{art3-H2} HOWE R. On the $N$-spectrum of representations of semisimple groups, (in preparation.)

\bibitem[H3]{art3-H3} HOWE R. $\theta$-{\em series and Invariant Theory}, to appear Proc. Symp. Pure Math, A.M.S., Providence, R.I., (1979).

\bibitem[H4]{art3-H4} HOWE R. The Fourier Transform and Germs of Characters, {\em Math. Ann.} 208 (1974), 305--322.

\bibitem[J]{art3-J} JOSEPH A.\pageoriginale {\em A. Characteristic Variety for the Primitive Spectrum of a Semisimple Lie Algebra}, Springer Lecture Notes in Math. 587, 102--118.

\bibitem[KV]{art3-KV} KASHIWARA M. and M. VERGNE, $K$-types and singular spectrum, Springer Lecture Notes Vol. 728.

\bibitem[K]{art3-K} KIRILLOV A.A. Unitary representations of nilpotent Lie groups, {\em Usp. Mat. Nauk} 106 (1962), 57--110.

\bibitem[RS]{art3-RS} REED M. and B. SIMON, {\em Methods of Modern Mathematical Physics}, Vol. II, Fourier Analysis, Self Adjointness.

\bibitem[Se]{art3-Se} SEGAL I. Hypermaximality of certain operators on Lie groups, {\em P.A.M.S.} 3 (1952), 13--15.

\bibitem[W]{art3-W} WARNER G. {\em Harmonic Analysis on Semisimple Lie Groups} I, Grund, Math. Wiss. 188, Springer Verlag, New York, Berlin, Heidelberg 1972.


\end{thebibliography}

