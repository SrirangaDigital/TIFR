\chapter{ON SHIMURA'S CORRESPONDENCE FOR MODULAR FORMS OF HALF-INTEGRAL WEIGHT$^{*}$}
\footnotetext[1]{Talk presented by S.G.}

\begin{center}
{\large By~ S. Gelbart and I. Piatetski-Shapiro}
\end{center}

\bigskip

\section*{Introduction}\pageoriginale

G. Shimura has shown how to attach to each holomorphic cusp form of half-integral weight a modular form of even integral weight. More precisely, suppose $f(z)$ is a cusp form of weight $k/2$, level $N$, and character $\chi$.\pageoriginale Suppose also that $f$ is an eigenfunction of all the Hecke operators $T^{N}_{k,\chi}(p^{2})$, say $T(p^{2})f=\omega_{p}f$. If $k\geq 5$, then the $L$-function
$$
\sum\limits^{\infty}_{n=1}A(n)n^{-s}=\prod\limits_{p<\infty}\left(1-\omega_{p}p^{-s}+\chi(p)p^{2k-2-2s}\right)^{-1}
$$
is the Mellin transform of a modular cusp form of weight $k-1$, level $N/2$, and character $\chi^{2}$. for further details, see \cite{Shim} or \cite{Niwa}.

Our purpose in this paper is to establish a Shimura correspondence for any (not necessarily holomorphic) cusp form of half-integral weight defined over a global field $F$(not necessarily $\mathbb{Q}$). Our approach is similar to Shimura's in that we use $L$-functions. Out point of view is new in that we use the theory of group representations.

Roughly speaking, suppose $\overline{\pi}=\bigotimes\limits_{v}\overline{\pi}_{v}$ is an automorphic cuspidal representation of the metaplectic group which doesn't factor through $\GL_{2}$. Then we introduce an $L$-factor $L(s,\overline{\pi}_{v})$ for each $v$ and we prove that the $L$-function
$$
L(s,\overline{\pi})=\prod\limits_{v}L(s,\overline{\pi}_{v})
$$
belongs to an automorphic representation of $\GL_{2}(\mathbb{A}_{F})$ in the sense of \cite{Jacquet-Langlands}. Since we characterize those $\overline{\pi}$ which correspond to cuspidal (as opposed to just automorphic) representations of $\GL_{2}(\mathbb{A}_{F})$ we refine as well as generalize Shimura's results.

Let us now describe our correspondence in more detail. Suppose $\overline{\pi}$ is an automorphic cuspidal representation of the metaplectic group. Since $\overline{\pi}$ is determined by its local components $\overline{\pi}_{v}$, we want to describe its ``Shimura image'' $S(\overline{\pi})$ in purely {\em local} terms. Thus we construct a local correspondence
$$
S:\overline{\pi}_{v}\to \pi_{v}
$$
by ``squaring'' the representation $\overline{\pi}_{v}$; if $\overline{\pi}_{v}$ is an induced representation, this means squaring the characters of $F^{x}_{v}$ which parametrize $\overline{\pi}_{v}$. In general, this process of ``squaring'' tends to smooth out representations, as we shall now explain.

Suppose\pageoriginale we consider the theta-representations of the metaplectic group. These representations generalize the classical modular forms of half-integral weight given by the theta-series
$$
\theta_{\chi}(z)=\sum\limits^{\infty}_{n=-\infty}\chi(n)e^{2\pi in^{2}_{z}}
$$
where $\chi$ is an (even) Dirichlet character of $Z$. Since these representations arise by pasting together a grossencharacter $\chi$ of $F$ with the ``even or odd'' part of the canonical metaplectic representation constructed in \cite{Weil}, we denote these representations by $r_{\chi}$ and call them Weil representations. Locally, $r_{\chi_{v}}$ is supercuspidal when $\chi_{v}(-1)=-1$. Almost everywhere, however, $\chi_{v}(-1)=1$, $r_{\chi_{v}}$ is the class $1$ quotient of a reducible principal series representation at $s=1/2$, and the global representation
$$
r_{\chi}=\bigotimes\limits_{v}r_{\chi_{v}}
$$
is ``distinguished'' from several different points of view. Most significantly, these $r_{\chi}$ exhaust the automorphic forms of half-integral weight which are determined by just one Fourier coefficient; this is the principal result of \cite{Ge PS2}.

Now if $\overline{\pi}_{v}$ is an even Weil representation $r_{\chi_{v}}$ (i.e. $\chi_{v}(-1)=1$), its Shimura image will be the one-dimensional representation $\chi_{v}$ of $\GL_{2}(F_{v})$, whereas if $\overline{\pi}_{v}$ is an ``odd'' Weil representation, $S(\overline{\pi}_{v})$ will be the special representation $\Sp(\chi_{v})$; cf. \S\ref{art1-sec7}. The Shimura correspondence thus takes cuspidal $r_{\chi}$ to automorphic representations of $\GL(2)$ which almost everywhere are one-dimensional and hence {\em not} cuspidal. The main result of this paper, however, guarantees that these representations are the only cuspidal $\overline{\pi}$ which map to non-cuspidal automorphic forms of $\GL(2)$. This explains the restriction $k\geq 5$ in \cite{Shim} and ultimately resolves ``Open question (C)'' of that paper; cf. \S\ref{art1-sec16}.

We mention also that the cuspidal representations $r_{\chi}$ contradict the Ramanujan-Petersson conjecture, in complete analogy to the counter-examples of \cite{Ho PS} for $\Sp(4)$. In particular, the $L$-function we attach to a supercuspidal component $r_{\chi_{v}}$ {\em can have a pole}; cf. \S\ref{art1-sec6}. Thus these\pageoriginale representations $r_{\chi}$ distinguish themselves in yet another way, and the regularizing nature of the local correspondence $S$ evidence itself (by ``lifting'' a supercuspidal representation to a non-supercuspidal one).

For a leisurely account of how classical modular forms of half-integral weight can be defined as representations of Weil's metaplectic group the reader is referred to \cite{Ge}. Most of the results described in the present paper were first announced in \cite{Ge PS}.

We note Chapter \ref{art1-chap-I} is purely local: after describing the local metaplectic group, and the notion of Whittaker models for its irreducible admissible representations, we introduce $L$ and $\epsilon$ factors and describe the local Shimura correspondence. In Chapters \ref{art1-chap-II} and \ref{art1-chap-III} we piece together these notions to obtain a global correspondence. In the process of doing so, we develop a Jacquet-Langlands theory for the metaplectic group. Details and related results are to be found in \cite{Ge}, \cite{GeHPS}, and \cite{GePS 2}. The principal contribution of the present paper is the proof of the global Shimura correspondence in Chapter \ref{art1-chap-III}.

It is with pleasure that we acknowledge our indebtedness to G. Shimura and R. P. Langlands. Shimura had already suggested the possibility of a representation-theoretic and adelic approach to his results in \cite{Shim}. On the other hand, the concrete suggestions and inspiration of Langlands first brought one of us close to the metaplectic group and got this project started. Langlands also suggested how the Selberg trace formula could be used to obtain (and in fact go beyond) our present results; this suggestion has just recently been developed by Flicker, whose results---improvements of our own---will appear in a forthcoming paper [Flicker].

\bigskip
\begin{center}
{\large\bfseries Chapter \thnum{I}.\label{art1-chap-I} Local Theory}
\end{center}
\smallskip

Throughout this Chapter $F$ will denote a local field of characteristic not equal to two. By $Z_{2}$ we shall denote the group of square roots of unity.

\section{The Metaplectic Group}\label{art1-sec1}

\subsection{}\label{art1-sec1.1} Let $H^{2}(\SL_{2}(F).Z_{2})$ denote the two-dimensional continuous cohomology group of $\SL_{2}(F)$ with coefficients in $Z_{2}$. From \cite{Weil} and \cite{Moore} it follows that if $F\neq \mathbb{C}$, $H^{2}(\SL_{2}(F),Z_{2})=Z_{2}$.

If\pageoriginale $F=\mathbb{C}$, let $\overline{\SL}_{2}(F)$ denote the group $\SL_{2}(F)\times Z_{2}$. If $F\neq \mathbb{C}$, let $\overline{\SL}_{2}(F)$ denote the non-trivial central extension of $\SL_{2}(F)$ by $Z_{2}$ determined by the non-trivial element of $H^{2}(\SL_{2}(F),Z_{2})$.

In all cases, we have an exact sequence of topological groups
$$
1\to Z_{2}\to \overline{\SL}_{2}(F)\to \SL_{2}(F)\to 1.
$$

\subsection{}\label{art1-sec1.2}
We want to extend Weil's metaplectic group to $\GL_{2}$. To do this, we use the fact that any automorphism of $\SL_{2}(F)$ lifts uniquely to an automorphism of $\overline{SL}_{2}(F)$.

Let $D$ denote the group
$$
D=\left\{\left(\begin{matrix} a & 0\\ 0 & 1\end{matrix}\right): a\in F^{x}\right\}
$$
Each element of $D$ operates on $\SL_{2}(F)$ by conjugation, hence lifts to an automorphism of $\overline{SL}_{2}$. If $\overline{G}$ denotes the resulting semi-direct product of $D$ and $\overline{\SL}_{2}(F)$, we obtain an exact sequences of locally compact groups
\begin{equation}
1\to Z_{2}\to \overline{G}\to \GL_{2}(F)\to 1.\label{art1-eq1.2.1}
\end{equation}
Note $\overline{G}$ is a non-trivial extension of $\GL_{2}(F)$ unless $F=\mathbb{C}$.

\subsection{}\label{art1-sec1.3}
The sequence \eqref{art1-eq1.2.1} splits over the following subgroups of $\GL_{2}(F)$:
\begin{align*}
N &= \left\{ \left(\begin{matrix} 1 & x\\ 0 & 1\end{matrix}\right):x\in F\right\}\\[4pt]
D &= \left\{ \left(\begin{matrix} a & 0\\ 0 & 1\end{matrix}\right):a\in F^{x}\right\}\\[4pt]
Z^{2} &= \left\{ \left(\begin{matrix} \lambda & 0\\ 0 & \lambda\end{matrix}\right):\lambda\in (F^{x})^{2}\right\}
\end{align*}
and (if $F$ is non-archimedean, of odd residual characteristic, and $O_{F}$ is the ring of integers of $F$),
$$
K=\GL_{2}(O_{F}).
$$

If $H$ is any subgroup of $\GL_{2}(F)$, let $\overline{H}$ denote its full inverse image in $\overline{G}_{F}$. If $H$ is such that the sequence \eqref{art1-sec1.2.1} splits over it, then $\overline{H}$ is the direct product of $Z_{2}$ with a subgroup of $\overline{G}$ which we again denote by $H$.

\subsection{}\label{art1-sec1.4}
The {\em center} of $\overline{G}_{F}$ is
$$
\overline{Z}^{2}=Z^{2}\times Z_{2}.
$$
On\pageoriginale the other hand, if
$$
Z=\left\{\left(\begin{matrix} \alpha & 0\\ 0 & \alpha\end{matrix}\right):\alpha\in F^{x}\right\},
$$
the group $\overline{Z}$ is abelian but {\em not} central in $\overline{G}$. When convenient, we confuse $Z$ with the group $F^{x}$, and $Z^{2}$ with the subgroup $(F^{x})^{2}$.

\subsection{}\label{art1-sec1.5}
If $\varphi : \overline{G}\to W$ is any function on $\overline{G}$, with values in a vector space $W$, we say $\varphi$ is {\em genuine} (or {\em doesn't factor through} $\GL_{2}$) if 
$$
\varphi(g\zeta)=\zeta\varphi(g),\quad\text{for all}\quad g\in \overline{G}, \ \zeta\in Z_{2}.
$$
Unless specified otherwise, we henceforth deal only with genuine objects on $\overline{G}_{F}$.

\section{Admissible Representations}\label{art1-sec2}

\subsection{}\label{art1-sec2.1}
By modifying the definitions in \cite{Jacquet-Langlands}, we can define, for each local $F$, the notion of an irreducible admissible representation $\overline{\pi}$ of $\overline{G}_{F}$.

\subsection{}\label{art1-sec2.2}
If $F$ is archimedean, we shall assume $\pi$ is actually irreducible unitary, or perhaps the restriction of such a representation to ``smooth'' vectors. Since $\overline{G}_{\mathbb{C}}=\GL_{2}(\mathbb{C})\times Z_{2}$ we shall have little to say about the case when $F$ is complex.

\subsection{Induced Representations.}\label{art1-sec2.3}
Let $B$ denote the Borel subgroup of $\GL_{2}(F)$. Although $\overline{B}$ is not abelian, it contains a convenient subgroup of finite index which {\em is} abelian, and even ``splits'' in $\overline{G}$. Indeed let $B_{0}$ denote the subgroup of $B$ consisting of matrices $\left(\begin{smallmatrix} a_{1} & x\\ 0 & a_{2}\end{smallmatrix}\right)$ where $a_{1}$ and $a_{2}$ have even $p$-adic order. If $F$ has even residual characteristic we also require that $a_{1}$ be a square modulo $1+4O_{F}$. If $F$ is real we simply require that $a_{1}>0$. In any case, $\overline{B}_{0}=B_{0}\times Z_{2}$, and the index of $\overline{B}_{0}$ in $\overline{B}$ is the index of $(F^{x})^{2}$ in $F^{x}$.

For any pair of quasi-characters $\mu_{1}$, $\mu_{2}$ of $F^{x}$, let $\mu_{1}\mu_{2}$ denote the (genuine) character of $\overline{B}_{0}/N$ whose restriction to $B_{0}/N$ is given by the formula
$$
\mu_{1}\mu_{2}\left(\left(\begin{matrix} a_{1} & 0\\ 0 & a_{2}\end{matrix}\right)\right)=\mu_{1}(a_{1})\mu_{2}(a_{2}).
$$
The\pageoriginale induced representation
\setcounter{equation}{0}
\begin{equation}
\overline{\rho}(\mu_{1},\mu_{2})=\Ind (\overline{G}_{F},\overline{B}_{0},\mu_{1}\mu_{2})\label{art1-eq2.3.1}
\end{equation}
is admissible and
$$
\overline{\rho}(\mu_{1},\mu_{2})\approx \overline{\rho}(\nu_{1},\nu_{2})
$$
not only if $\mu_{1}=\nu_{2}$ and $\mu_{2}=\nu_{1}$, but also if
\begin{equation}
\mu^{2}_{i}=\nu^{2}_{i}, \ i=1,2.\label{art1-eq2.3.2}
\end{equation}
cf. \S2 of \cite{GePS2} and \S5 of \cite{Ge}. Moreover, $\overline{\rho}(\mu_{1},\mu_{2})$ is irreducible unless $\mu^{2}_{1}\mu^{-2}_{2}(x)=|x|^{1}$ or $|x|^{-1}$ (or all integral points in the real case). In any case, the composition series has length at most 2; cf. \cite{Moen} and \cite{GeSa}.

\subsection{Classification of Representations}\label{art1-sec2.4}
If $\overline{\rho}(\mu_{1},\mu_{2})$ is irreducible, we denote it by $\overline{\pi}(\mu_{1},\mu_{2})$ and call it a {\em principal series} representation. If $\overline{\rho}(\mu_{1},\mu_{2})$ is reducible, we let $\overline{\pi}(\mu_{1},\mu_{2})$ denote its unique irreducible subrepresentation. In all cases, $\overline{\pi}(\mu_{1},\mu_{2})$ defines an infinite-dimensional irreducible admissible representation of $\overline{G}_{F}$. If $\mu^{2}_{1}\mu^{-2}_{2}(x)=|x|^{1}$ we call $\overline{\pi}(\mu_{1},\mu_{2})$ a {\em special} representation; it is equivalent to the unique quotient of $\overline{\rho}(\mu_{2},\mu_{1})$.

Suppose $(\overline{\pi},V)$ is any irreducible admissible (genuine) representation of $\overline{G}_{F}$. Then $\overline{\pi}$ is automatically infinite-dimensional. If it is not of the form $\overline{\pi}(\mu_{1},\mu_{2})$ for some pair $(\mu_{1},\mu_{2})$, we say $\overline{\pi}$ is {\em supercuspidal}. If $F$ is archimedean, no such representations exist. On the other hand, if $F$ is non-archimedean, $\overline{\pi}$ is supercuspidal if and only if for every vector $v$ in $V_{\overline{\pi}}$,
$$
\int\limits_{U}\overline{\pi}(u)v\ du = 0
$$
for some open compact subgroup $U$ of $N\subset \overline{G}_{F}$; cf. \cite{Ge}, \S5.

The construction and analysis of such supercuspidal representations is carried out in \cite{RS} and \cite{Meister}.

From \cite{Ge} Section 5, and \cite{Meister}, it follows that:

\subsubsection{}\label{art1-sec2.4.1}
An irreducible admissible representation $\overline{\pi}$ is class 1 if and only\pageoriginale if it is of the form $\overline{\pi}(\mu_{1},\mu_{2})$ with $\mu^{2}_{1}$ and $\mu^{2}_{2}$ unramified and $\mu^{2}_{1}\mu^{-2}_{2}(x)\neq |x|$, i.e., $\overline{\pi}$ is not special.

\subsection{Class 1 Representations}\label{art1-sec2.5}
Suppose $F$ is non-archimedean and of odd residual characteristic. If $\overline{\pi}$ is an admissible representation of $\overline{G}_{F}$, recall $\overline{\pi}$ is {\em class 1}, or spherical, if its restriction to $K_{F}$ contains the identity representation (at least once). If $\overline{\pi}$ is also irreducible, it can be shown that $\overline{\pi}$ then contains the identity representation {\em exactly} once; cf. \cite{Ge} and \cite{Meister}.

In particular, suppose $\overline{1}_{K}$ denotes the idempotent of the Hecke algebra of $\overline{G}_{F}$ belonging to the trivial representation of $K_{F}$, i.e.,
$$
1_{K}(g)=
\begin{cases}
1 & \text{if~ } g\in K\\
-1 & \text{if~ } g\in K\times \{-1\}\\
0  & \text{if otherwise}
\end{cases}
$$
Then $\overline{\pi}$ class 1 implies $\overline{\pi}(\overline{1}_{K})$ has non-zero range, and $\overline{\pi}$ class 1 irreducible implies the range is one-dimensional.

\section{Whittaker Models}\label{art1-sec3}

Fix once and for all a non-trivial additive character $\psi$ of $F$.

\subsection{Definition}\label{art1-defi3.1}
Suppose $\overline{\pi}$ is an irreducible admissible representation of $\overline{G}_{F}$. By a $\psi$-{\em Whittaker model for} $\overline{\pi}$ we understand a space $W(\overline{\pi},\psi)$ consisting of continuous functions $W(g)$ on $\overline{G}$ satisfying the following properties:

\subsubsection{}\label{art1-sec3.1.1}
$W\left(\left(\begin{smallmatrix} 1 & x\\ 0 & 1\end{smallmatrix}\right)g\right)=\psi(x)W(g)$;

\subsubsection{}\label{art1-sec3.1.2}
If $F$ is non-archimedean, $W$ is locally constant, and if $F$ is archi\-medean, $W$ is $C^{\infty}$;

\subsubsection{}\label{art1-sec3.1.3}
The space $W(\overline{\pi},\psi)$ is invariant under the right action of $\overline{G}_{F}$, and the resulting representation in $W(\overline{\pi},\psi)$ is equivalent to $\overline{\pi}$.

\subsection{}\label{art1-sec3.2}
In \cite{GeHPS} we prove that a $\psi$-Whittaker model always exists. If $W(\overline{\pi},\psi)$ is unique, we say $\overline{\pi}$ is {\em distinguished}. Note that if $\overline{\pi}$ is not genuine, i.e., if $\overline{\pi}$ defines an ordinary representation of $\GL_{2}(F)$, then $\overline{\pi}$ is always distinguished: this is the celebrated ``uniqueness of Whittaker models'' result of \cite{Jacquet-Langlands}.

In\pageoriginale general, if $\overline{\pi}$ is genuine (as we are assuming it is), it is not distinguished. To recapture uniqueness, we need to refine our notion of Whittaker model.

\subsection{}\label{art1-sec3.3}
Let $\omega_{\overline{\pi}}$ denote the {\em central character of} $\overline{\pi}$. This is the genuine character of $(F^{x})^{2}xZ_{2}$ determined by the formula
\setcounter{equation}{0}
\begin{equation}
\overline{\pi}\left(\begin{matrix} a^{2} & 0\\ 0 & a^{2}\end{matrix}\right)=\omega_{\overline{\pi}}(a^{2})I.\label{art1-eq3.3.1}
\end{equation}

Let $\Omega(\omega_{\overline{\pi}})$ denote the (finite) set of genuine characters of $\overline{Z}$ whose restriction to $\overline{Z}^{2}$ agrees with $\omega_{\overline{\pi}}$.

\subsection{Definition.}\label{art1-sec3.4}
For each $\mu$ in $\Omega(\omega_{\overline{\pi}})$, let $\mathscr{W}(\overline{\pi},\psi,\mu)$ denote the space of continuous functions $W(g)$ on $\overline{G}_{F}$ which, in addition to satisfying conditions \eqref{art1-sec3.1.1}-\eqref{art1-sec3.1.3}, also satisfy the condition
\setcounter{equation}{0}
\begin{equation}
W\left(\overline{z}\left[\begin{smallmatrix} 1 & x\\ 0 & 1 \end{smallmatrix}\right]g\right)=\mu(\overline{z})\psi(x)W(g),\quad\text{for}\quad \overline{z}\in \overline{Z}.\label{art1-eq3.4.1}
\end{equation}

In \cite{GeHPS} we prove that {\em such} a Whittaker model {\em is} unique. More precisely, there is {\em at most one} such model, and {\em for at least one} $\mu$ in $\Omega(\omega_{\overline{\pi}})$, a $(\psi,\mu)$-Whittaker model always exists.

\subsection{}\label{art1-sec3.5}
Let $\Omega(\pi)=\Omega(\overline{\pi},\psi)$ denote the set of $\mu$ in $\Omega(\omega_{\overline{\pi}})$ such that $\mathscr{W}(\overline{\pi},\psi,\mu)$ exists. This set depends on $\psi$, but its cardinality does not. Indeed if $\lambda\in F^{x}$, and $\psi^{\lambda}$ denotes the character
\setcounter{equation}{0}
\begin{equation}
\psi^{\lambda}(x)=\psi(\lambda x),\label{art1-eq3.5.1}
\end{equation}
then $\mathscr{W}(\overline{\pi},\psi,\mu)$ is mapped isomorphically to $\mathscr{W}(\overline{\pi},\psi^{\lambda},\mu^{\lambda})$ via the map
\begin{equation}
W(g)\to W^{\lambda}(g)=W\left(\left[\begin{matrix} \lambda & 0\\ 0 & 1\end{matrix}\right]g\right).\label{art1-eq3.5.2}
\end{equation}
Here $\mu^{\lambda}$ denotes the character
\begin{equation}
\mu^{\lambda}(\overline{z})=\mu\left(\left(\begin{matrix} \lambda & 0\\ 0 & 1\end{matrix}\right)^{-1}\overline{z}\left(\begin{matrix} \lambda & 0\\ 0 & 1\end{matrix}\right)\right)\label{art1-eq3.5.3}
\end{equation}
with the conjugation carried out in $\overline{G}$. The existence of the isomorphism \eqref{art1-eq3.5.2} means that $\mu\in \Omega(\overline{\pi},\psi)$ iff $\mu^{\lambda}\in \Omega(\overline{\pi},\psi^{\lambda})$.

\subsection{Remark.}\label{art01-sec3.6}
$\Omega(\overline{\pi},\psi)$ is a singleton set if and only if $\overline{\pi}$ is distinguished.

All possible examples of distinguished $\overline{\pi}$ are described in the next Section.

\section{The Theta-Representations $r_{\chi}$}\label{art1-sec4}\pageoriginale

These representations are indexed by characters of $F^{x}$ and treated in complete detail in \cite{Ge PS2}. We simply recall their definition and basic properties.

\subsection{}\label{art1-sec4.1}
In \cite{Weil} there was constructed a genuine admissible representation of $\overline{\SL}_{2}(F)$. We call this representation the basic Weil representation and denote it by $r^{\psi}$; it depends on the non-trivial additive character $\psi$ and splits into two irreducible pieces, one ``even'', one ``odd''.

If $\chi$ is an even (resp. odd) character of $F^{x}$, we can ``tensor'' $\chi$ with the even (resp. odd) piece of $r^{\psi}$ to obtain a representation $r^{\psi}_{\chi}$ of $\overline{G}_{F}^{*}$, the semi-direct product of $\overline{\SL}_{2}(F)$ with $\left\{\left(\begin{smallmatrix} 1 & 0\\ 0 & a^{2}\end{smallmatrix}\right):a\in F^{x}\right\}$. Inducing up to $\overline{G}_{F}$ produces an irreducible admissible representation which is independent of $\psi$ and denoted $r_{\chi}$. The restriction of $r_{\chi}$ to $\overline{\SL}_{2}(F)$ is the direct sum of a finite number of inequivalent representations, namely
$$
\{r^{\psi^{\lambda}}\}_{\lambda\in \Lambda},
$$
with $\Lambda$ an index set for the cosets of $(F^{x})^{2}$ in $F^{x}$.

\subsection{Each $r_{\chi}$ is a distinguished representation of $\overline{G}_{F}$.}\label{art1-sec4.2}
In particular, for each non-trivial character $\psi$ of $F$, let $\gamma(\psi)$ denote the eighth root of unity introduced in \cite{Weil}, Section 14.

Then
$$
\Omega(r_{\chi},\psi)=\{\chi_{\mu_{\psi}}\},
$$
with $\mu_{\psi}$ the projective character of $F^{x}$ defined by
\setcounter{equation}{0}
\begin{equation}
\mu_{\psi}(a)=\dfrac{\gamma(\psi)}{\gamma(\psi^{a})}\label{art1-eq4.2.1}
\end{equation}

We note that the restriction of $\mu_{\psi}$ to $(F^{x})^{2}$ is trivial. Moreover, if $\psi$ has conductor $O_{F}$, and $F$ is of odd residual characteristic, $\mu_{\psi}$ is also trivial on units.

\subsection{}\label{art1-sec4.3}
{\em When $\chi$ is unramified, and $F$ has odd residue characteristic, $r_{\chi}$ is class 1.} More generally, if $\chi$ is an even character, $r_{\chi}$ is the unique irreducible subrepresentation of $\overline{\pi}(\chi^{1/2}|~|^{-1/4}_{F},\chi^{1/2}|~|_{F}^{1/4})$.

\subsection{}\label{art1-sec4.4}\pageoriginale
If $\chi$ is an odd character, i.e., $\chi(-1)=-1$, then $r_{\chi}$ is super-cuspidal; cf. \cite{Ge}.

\subsection{}\label{art1-sec4.5}
Having observed that each $r_{\chi}$ is distinguished, we conjectured that the family $\{r_{\chi}\}_{\chi}$ exhausts the irreducible admissible distinguished representations of $\overline{G}$.

When $F$ is non-archimedean and of odd residue characteristic, the supercuspidal part of this conjecture is established in \cite{Meister}; the non-supercuspidal part is treated in \cite{Ge PS2}.

\section{A Functional Equation of Shimura Type}\label{art1-sec5}

As always, $F$ is a local field of characteristic not equal to 2 and $\psi$ is a fixed non-trivial character of $F$.

\subsection{}\label{art1-sec5.1}
Suppose $\overline{\pi}$ is any irreducible admissible representation of $\overline{G}_{F}$, and $\chi$ is any quasi-character of $F^{x}$. Recall the sets $\Omega(\overline{\pi},\psi)$ and $\Omega(r_{\chi},\psi)$ introduced in \eqref{art1-sec3.5}. In general, $\Omega(\overline{\pi},\psi)=\Omega(\omega_{\overline{\pi}})$. However, $\Omega(r_{\chi},\psi)=\{\chi \mu_{\psi}\}$.

To attach an $L$-factor to $\overline{\pi}$ and $\chi$, we fix some $\mu$ in $\Omega(\pi,\psi)$ and introduce the zeta-functions
\setcounter{equation}{0}
\begin{equation}
\Psi (s,W,W_{\chi},\Phi)=\int\limits_{N\backslash G} W(g)W_{\chi}(g)|\det (g)|^{s}\Phi((0.1)g)dg.\label{art1-eq5.1.1}
\end{equation}

Here $W(g)$ is any element of $\mathscr{W}(\pi,\psi,\mu)$, $W_{\chi}$ is any element of $\mathscr{W}(r_{\chi},\psi^{-1},\chi_{\mu_{\psi-1}})$, $\Phi\in \mathscr{S}(F\times F)$, and $s\in \mathbb{C}$. Since $W$ and $W_{x}$ are genuine, and transform contravariantly under $N$, their product actually defines a function on $N\backslash G$.

Similarly, we define
\begin{equation}
\widetilde{\Psi}(s,W,W_{\chi},\Phi)=\int\limits_{N\backslash G}W(g)W_{\chi}(g)|\det g|^{s}\omega^{-1}_{*}(\det g)\Phi((0,1)g)dg\label{art1-eq5.1.2}
\end{equation}
with 
\begin{equation}
\omega_{*}=\mu \chi \mu_{\psi^{-1}}.\label{art1-eq5.1.3}
\end{equation}
Note that $\omega_{*}$ is an ordinary character of $F^{x}$ whose restriction to $(F^{x})^{2}$ is $\chi \omega_{\overline{\pi}}$.

\subsection{}\label{art1-sec5.2}\pageoriginale
For $\Re(s)$ sufficiently large, and $g$ in $\GL_{2}(F)$, the integrals
\setcounter{equation}{0}
\begin{equation}
|\det g|^{s}\int\limits_{F^{x}}\Phi ((0.,t)g)|t|^{2s}\omega_{*}(t)\dt=f_{s}(g)\label{art1-eq5.2.1}
\end{equation}
and
\begin{equation}
|\det g|^{s}\omega^{-1}_{*}(\det g)\int\limits_{F^{x}}\Phi((0,t)g)|t|^{2s}\omega^{-1}_{*}(t)\dt = h_{s}(g)\label{art1-eq5.2.2}
\end{equation}
converge and define elements in the space of the induced representations $\rho(s-1/2, (1/2-s)\omega^{-1}_{*})$ and $\rho(\omega^{-1}_{*}(s-1/2),1/2-s)$ respectively. Cf. \cite{Ja}, 14. Moreover, for such $s$, the integrals defining $\Psi$ and $\widetilde{\Psi}$ converge.
\begin{equation}
\Psi(s,W,W_{\chi},\Phi)=\int\limits_{NZ\backslash G}W(g)W_{\chi}(g)f(g)dg\label{art1-eq5.2.3}
\end{equation}
and
$$
\widetilde{\Psi}(s,W,W_{\chi},\Phi)=\int\limits_{NZ\backslash G}W(g)W_{\chi}(g)h(g)dg
$$
Modifying the methods of \cite{Ja} we obtain :

\setcounter{theorem}{2}
\begin{theorem}\label{art1-thm5.3}
\begin{itemize}
\item[(a)] The functions $\Psi(s,W,W_{\chi},\Phi)$ and $\widetilde{\Psi}(s,W,W_{\chi},\Phi)$ extend meromorphically to $\mathbb{C}$;

\item[(b)] There exist Euler factors $L(s,\overline{\pi},\chi)$ and $\widetilde{L}(s,\overline{\pi},\chi)$ such that for any $W$, $W_{\chi}$, $\Phi$, $\psi$, and $\mu$, the functions
$$
\frac{\Psi(s,W,W_{\chi},\Phi)}{\widetilde{L}(s,\overline{\pi},\chi)}\quad\text{and}\quad \dfrac{\widetilde{\Psi}(s,W,W_{\chi},\Phi)}{\widetilde{L}(s,\overline{\pi},\chi)}
$$
are entire;

\item[(c)] There is an exponential factor $\epsilon(s,\overline{\pi},\chi,\psi)$ such that for all $W$, $W_{\chi}$ and $\Phi$ as above,
\setcounter{subsection}{3}
\setcounter{equation}{0}
\begin{equation}
\dfrac{\widetilde{\Psi}(1-s,W,W_{\chi},\widehat{\Phi})}{\widetilde{L}(1-s,\overline{\pi},\chi)}=\epsilon(s,\overline{\pi},\chi,\psi)\dfrac{\Psi(s,W,W_{\chi},\Phi)}{L(s,\overline{\pi},\chi)},\label{art1-eq5.3.1}
\end{equation}
with
$$
\widehat{\Phi}(x,y)=\iint \Phi (u,v)\psi(uy-vx)dudv.
$$
\end{itemize}
\end{theorem}

\setcounter{subsection}{3}
\subsection{}\label{art1-sec5.4}
The factor $\epsilon(s,\overline{\phi},\chi,\psi)$ might depend on the choice of $\mu$ as well as\pageoriginale $\psi$. Therefore, to be precise, we should write $\epsilon(s,\overline{\pi},\chi,\psi,\mu)$ in place of $\epsilon(s,\overline{\pi},\chi,\psi)$. However, a straightforward computation shows that
\setcounter{equation}{0}
\begin{equation}
\epsilon(s,\overline{\pi},\chi,\mu^{\lambda})=\omega_{\overline{\pi}}(\lambda^{-2})\chi^{-2}(\lambda)|\lambda|^{2-4s}\epsilon (s,\overline{\pi},\chi,\psi^{\lambda},\mu).\label{art1-eq5.4.1}
\end{equation}
Also, as we shall see, {\em globally} $\epsilon(s,\overline{\pi},\chi,\psi,\mu)$ is easily seen to be independent of both $\psi$ and $\mu$; cf. Remark \ref{art1-rem13.4}.

\subsection{}\label{art01-sec5.5}
If we introduce the ``gamma factor''
$$
\gamma(s,\overline{\pi},\chi,\psi)=\dfrac{\epsilon(s,\overline{\pi},\chi,\psi)\widetilde{L}(1-s,\overline{\pi},\chi)}{L(s,\overline{\pi},\chi)}
$$
then the functional equation \eqref{art1-eq5.3.1} takes the simpler form
$$
\widetilde{\Psi}(1-s,W,W_{\chi},\widehat{\Phi})=\gamma(s,\overline{\pi},\chi,\psi)\Psi(s,W_{1},W_{2},\Phi).
$$

\section{$L$ and $\epsilon$-Factors}\label{art1-sec6}

Let $\overline{\pi}$, $\chi$ and $\psi$ be as in the last section. In this section we collect together the values of $L(s,\overline{\pi},\chi)$, $\widetilde{L}(s,\overline{\pi},\chi)$, and $\epsilon(s,\overline{\pi},\chi,\psi)$ for most representations $\overline{\pi}$. To compute the factors $L$ and $\widetilde{L}$ we need to analyze the possible poles of $\Psi(s,W,W_{\chi},\Phi)$ and $\widetilde{\Psi}(s,W,W_{\chi},\Phi)$. To compute $\epsilon(s,\overline{\pi},\chi,\psi)$ we need to compute the functions $\Psi$ and $\widetilde{\Psi}$ explicitly, for judicious choices of $W$, $W_{\chi}$ and $\Phi$.

Suppose first that $F$ is non-archimedean.

\subsection{}\label{art1-sec6.1}
Suppose $\overline{\pi}$ is a supercuspidal. If $\overline{\pi}$ is not of the form $r_{\nu}$ for any quasi-character $\nu$, then
$$
L(s,\pi,\chi)=1=\widetilde{L}(s,\overline{\pi},\chi),\quad\text{for all}\quad \chi.
$$
On the other hand, if $\overline{\pi}=r_{\nu}$, then
$$
L(s,\overline{\pi},\chi)=L(2s,\chi\nu),
$$
and
$$
\widetilde{L}(s,\overline{\pi},\chi)=L(2s,\chi^{-1}\nu^{-1})
$$

If $\chi\nu$ is unramified,
$$
\epsilon(s,\overline{\pi},\chi,\psi)=\frac{\epsilon(2s,\chi\nu,\psi)\epsilon(2s-1,\chi\nu,\psi)L(1-2s,\nu^{-1}\chi^{-1})}{L(2s-1,\nu\chi)}
$$
whereas\pageoriginale if $\chi\nu$ is ramified
$$
\epsilon(s,\overline{\pi},\chi,\psi)=\epsilon(2s,\chi\nu,\psi)\epsilon(2s-1,\chi\nu,\psi).
$$

Here, as throughout, the factors $L(s,\omega)$ and $\epsilon(s,\omega,\psi)$ are the familiar $L$ and $\epsilon$ factors attached to each quasi-character $\omega$ of $F^{x}$; cf. \cite[pp. 108-109]{Jacquet-Langlands}.

\subsection{}\label{art1-sec6.2}
Suppose $\overline{\pi}$ is of the form $\overline{\pi}(\mu_{1},\mu_{2})=\overline{\phi}(\mu_{1},\mu_{2})$. Then
$$
L(s,\overline{\pi},\chi)=L(2s-1/2,\mu^{2}_{1}\chi)L(2x-1/2,\mu^{2}_{2}\chi),
$$
and
\setcounter{equation}{0}
\begin{equation}
\widetilde{L}(s,\overline{\pi},\chi)=L(2s-1/2,\mu^{-2}_{1}\chi)L(2s-1/2,\mu^{-2}_{2}\chi)\label{art1-eq6.2.1}
\end{equation}
If we set $s'=2s-1/2$, then
\begin{equation}
\epsilon(s,\overline{\pi},\chi,\psi)=\epsilon(s',\mu^{2}_{1}\chi,\psi)\epsilon(s',\mu^{2}_{2}\chi,\psi)\label{art1-eq6.2.2}
\end{equation}

In particular, suppose $F$ is class 1, $\chi$(also $\mu$) is trivial on units, $\psi$ has conductor $O_{F}$, $\Phi$ is the characteristic function of $O_{F}\times O_{F}$, and $W$ and $W_{\chi}$ are normalized $K_{F}$-fixed vectors in $W(\overline{\pi},\psi,\mu)$ and $W(r_{\chi},\psi^{-1})$. Then
\begin{equation}
\begin{cases}
\Psi(s,W_{1},W_{2},\Phi)=L(s,\overline{\pi},\chi)\\
\widetilde{\Psi}(s,W_{1},W_{2},\widetilde{\Phi})=\widetilde{L}(s,\overline{\pi},\chi)
\end{cases}\label{art1-eq6.2.3}
\end{equation}
and
$$
\epsilon(s,\overline{\pi},\chi,\psi)=1.
$$

\subsection{}\label{art1-sec6.3}
Suppose $\overline{\pi}$ is the special representation
$$
\overline{\pi}=\overline{\pi}(\mu_{1},\mu_{2}),\text{~ with~ }\mu^{2}_{1}\mu^{-2}_{2}(x)=|x|^{1}_{F},\text{~ and~ } \mu_{1}(x)=\nu(x)|x|^{1/4}_{F}
$$

Then
$$
\begin{cases}
L(s,\overline{\pi},\chi)=L(2s,\chi\nu^{2}),\\
\widetilde{L}(s,\overline{\pi},\chi)=L(2s,\nu^{-2}\chi^{-1}),
\end{cases}
$$
and-if $\pi(\nu^{2})$ denotes the special representation $\pi(\nu^{2}|~|^{1/2},\nu^{2}|~|^{-1/2})$ of $\GL_{2}(F)$,
$$
\epsilon(s,\overline{\pi},\chi,\psi)=\epsilon(s',\pi(\nu^{2})\otimes \chi,\psi).
$$

\subsection{}\label{art1-sec6.4}
If $\overline{\pi}$ is of the form $r_{\nu}$, with $\nu(-1)=1$, then
$$
\begin{cases}
L(s,\overline{\pi},\chi)=L(2s-1,\chi\nu)L(2s,\chi\nu),\\
\widetilde{L}(s,\overline{\pi},\chi)=L(2s-1,\chi^{-1}\nu^{-1})L(2s,\chi^{-1}\nu^{-1}),
\end{cases}
$$
and
$$
\epsilon(s,\overline{\pi},\chi,\psi)=\epsilon(2s-1,\chi\nu,\psi)\in (2s,\chi\nu,\psi)
$$\pageoriginale

\subsection{}\label{art1-sec6.5}
Suppose now that $F$ is archimedean. Then each $\overline{\pi}$ occurs as the subrepresentation of some $\overline{\rho}(\mu_{1},\mu_{2})$, with each $\mu_{i}$ determined up to a character of order $2$. Let $S(\overline{\pi})$ denote the unique irreducible admissible representation of $\GL_{2}(F)$ which appears as a subrepresentation of $\rho(\mu^{2}_{1},\mu^{2}_{2})$. Then
$$
\begin{cases}
L(s,\overline{\pi},\chi)=L(s,S(\overline{\pi})\otimes \chi),\\
\widetilde{L}(s,\overline{\pi},\chi)=L(s,S(\overline{\pi})\otimes \chi^{-1}),
\end{cases}
$$
and
$$
\epsilon(s,\overline{\pi},\chi,\psi)=\epsilon(s,S(\overline{\pi})\otimes \chi, \psi),
$$
the $L$ and $\epsilon$ factors on the right being those of \cite{Jacquet-Langlands}.

\subsection{Stability}\label{art1-sec6.6}

Given $\overline{\pi}$ and $\psi$, it can be shown that if $F$ is non-archimedean, and $\chi$ is sufficiently highly ramified, the corresponding $L$ and $\epsilon$-factors stabilize. More precisely, for all $\chi$ sufficiently highly ramified,
$$
L(s,\overline{\pi},\chi)=1=\widetilde{L}(s,\overline{\pi},\chi),
$$
and
\setcounter{equation}{0}
\begin{equation}
\epsilon(s,\overline{\pi},\chi,\psi)=\epsilon(s,\omega_{\pi}\chi,\psi)\in (s,\chi,\psi)\label{art1-eq6.6.1}
\end{equation}

In \eqref{art1-eq6.6.1}, $\omega_{\pi}$ is the character of $F^{x}$ defined by the equation
\begin{equation}
\omega_{\pi}(a)=\omega_{\overline{\pi}}(a^{2})\label{art1-eq6.6.2}
\end{equation}

\section{A Local Shimura Correspondence}\label{art1-sec7}

Suppose $\overline{\pi}$ is an irreducible admissible (genuine) representation of $\overline{G}_{F}$ and $\omega_{\overline{\pi}}$ is its central character.

\subsection{}\label{art1-sec7.1}\pageoriginale
Fixing a non-trivial character $\psi$ of $F$, we call an irreducible admissible representation $\pi$ of $G_{F}$ a {\em Shimura image of $\overline{\pi}$} if

\subsubsection{}\label{art1-sec7.1.1}
the central character $\omega_{\pi}$ of $\pi$ is such that
$$
\omega_{\pi}(a)=\omega_{\overline{\pi}}(a^{2}), \ a\in F^{x};
$$

\subsubsection{}\label{art1-sec7.1.2}
for any quasi-character $\chi$ of $F^{x}$,
$$
\begin{cases}
L(s,\overline{\pi},\chi)=L(s,\pi\otimes \chi),\\
\widetilde{L}(s,\overline{\pi},\chi)=L(s,\widetilde{\pi}\otimes \chi^{-1}),
\end{cases}
$$
and
$$
\epsilon(s,\overline{\pi},\chi,\psi)=\epsilon(s,\pi\otimes \chi,\psi).
$$

\subsection{}\label{art1-sec7.2}
If the Shimura image of $\overline{\pi}$ exists, it is unique, and independent of $\psi$. We denote it by $S(\overline{\pi})$.

\subsection{}\label{art1-sec7.3}
From Section \ref{art1-sec6} it follows that $S(\overline{\pi})$ exists whenever $\overline{\pi}$ is not a supercuspidal representation (not of the form $r_{\nu}$). Indeed in this case,
$$
\overline{\pi}=\overline{\pi}(\mu_{1},\mu_{2})\text{~ implies~ } S(\overline{\pi})=\pi(\mu^{2}_{1},\mu^{2}_{2}).
$$
In particular,
$$
\overline{\pi}=r_{\nu}(\nu(-1)=-1)\text{~ implies~ } (\overline{\pi})=\pi(\nu|~|^{1/2}_{F},\nu|~|_{F}^{1/2}).
$$
On the other hand, as we shall see, if $\overline{\pi}$ is supercuspidal (but not of the form $r_{\nu}$) its image $S(\overline{\pi})$ must also be supercuspidal.

\subsection{}\label{art1-sec7.4}
In case $F=\mathbb{R}$, and $\overline{\pi}$ corresponds to a discrete series representation of ``lowest weight $k/2$'', $S(\overline{\pi})$ corresponds to a discrete series representation of lowest weight $k-1$; cf. \cite{Ge}, \S4.

\subsection{Connections with Shimura's theory}\label{art1-sec7.5}
~

The fact that $S$ takes $\overline{\pi}(\mu_{1},\mu_{2})$ to $\pi(\mu^{2}_{1},\mu^{2}_{2})$ means (in the non-archimedean unramified situation) that eigenvalues for the Hecke algebras are preserved. See \S5.3 of \cite{Ge} for a careful analysis of this phenomenon. Keeping in mind (\ref{art1-sec7.4}), it follows that our local Shimura correspondence is consistent with the map defined globally (and classically) in \cite{Shim}.

\subsection{}\label{art1-sec7.6}\pageoriginale
Summing up, Shimura's correspondence operates locally as follows:
\begin{center}
\tabcolsep=10pt
\renewcommand{\arraystretch}{1.1}
\begin{tabular}{c|c}
$\overline{\pi}$ & $\pi=S(\overline{\pi})$\\
\hline
principal series & principal series\\
$\overline{\pi}(\mu_{1},\mu_{2})$ & $\pi(\mu^{2}_{1},\mu^{2}_{2})$\\
\hline
special representation & special rep\\
$\overline{\pi}(\nu|~|^{1/4},\nu|~|^{-1/4})$ & $\Sp(\nu^{2})$\\
\hline
Weil $r_{\nu}$ & special rep\\
$(\nu(-1)=-1)$ & $\Sp(\nu)$\\
\hline
Weil $r_{\nu}$ & one-dimensional rep\\
$(\nu(-1)=1)$ & $\nu\circ \det$
\end{tabular}
\end{center}
Note {\em all} special representations arise as Shimura images (whereas a principal series thus arises if it corresponds to even-or squared-characters of $F^{*}$); for the supercuspidal representations, see \cite{Flicker} and \cite{Meister}.

\bigskip
\begin{center}
{\large\bfseries Chapter \thnum{II}.\label{art1-chap-II} Global Theory}
\end{center}
\smallskip

Throughout this Chapter, $F$ will denote an arbitrary $A$-field of characteristic not equal to two, $\mathbb{A}$ its ring of adeles, and
$$
\psi=\prod\limits_{v}\psi_{v}
$$
a non-trivial character of $F\backslash \mathbb{A}$.

\section{The Metaplectic Group}\label{art1-sec8}

For each place $v$ of $F$ we defined in \S\ref{art1-sec1} a ``local'' metaplectic group $\overline{G}_{v}=\overline{G}_{F_{v}}$. Roughly speaking, the adelic metaplectic group $\overline{G}_{A}$ is a product of the local groups $\overline{G}_{v}$.

More precisely, recall that if $v$ is non-archimedean and ``odd'', $\overline{G}_{v}$ splits over $K_{v}=\GL_{2}(O_{F_{v}})$. Thus we can consider the restricted direct product
$$
\widetilde{G}=\prod\limits_{v}\overline{G}_{v}(K_{v}).
$$

The\pageoriginale metaplectic group $\overline{G}_{A}$ is obtained by taking the quotient of $\widetilde{G}$ by
$$
\widetilde{Z}_{e}=\left\{\prod\limits_{v}\epsilon_{v}\in \prod\limits_{v}Z_{2}:\epsilon_{v}=1\text{~ for all but an even number of } v\right\}.
$$
In particular, we can view $\overline{G}_{A}$ as a group of pairs $\{(h,\zeta):h\in G_{A},\zeta\in Z_{2}\}$, with multiplication given by
$$
(h_{1},\zeta_{1})(h_{2},\zeta_{2})=(h_{1}h_{2},\beta(h_{1},h_{2})\zeta_{1}\zeta_{2}),
$$
and $\beta$ a product of the local two-cocycles defining $\overline{G}_{v}$. The fact that the exact sequence
$$
1\to Z_{2}\to \overline{G}_{A}\to G_{A}\to 1
$$
splits over the discrete subgroup
$$
G_{F}=\GL_{2}(F)
$$
is equivalent to the quadratic reciprocity law for $F$; cf. \cite{Weil}.

\section{Automorphic Representations of Half-Integral Weight}\label{art1-sec9}

\subsection{}\label{art1-sec9.1}
Recall that $\overline{G}_{A}$ is the quotient of $\prod\limits_{v}\overline{G}_{v}=\widetilde{G}_{A}$ by the subgroup $\widetilde{Z}_{e}$.

\subsection{}\label{art1-sec9.2}
Suppose that for each place $v$ of $F$ we are given an irreducible admissible genuine representation $(\overline{\pi}_{v},V_{v})$ of $\overline{G}_{v}$. Suppose also that for almost every finite $v$, $\overline{\pi}_{v}$ is class 1. Then for almost every $v$ we can choose a $K_{v}$-fixed vector $e_{v}$ in $V_{v}$ and define a restricted tensor product space 
$$
V=\bigotimes\limits_{v}V_{v}(e_{v}).
$$
The resulting representation of $\widetilde{G}_{A}$ in $V$ given by
\setcounter{equation}{0}
\begin{equation}
\overline{\pi}=\bigotimes\limits_{v}\overline{\pi}_{v}\label{art1-eq9.2.1}
\end{equation}
is trivial on $\widetilde{Z}_{e}$ and defines an irreducible admissible representation of $\overline{G}_{A}$.

Conversely, suppose $\overline{\pi}$ is an irreducible unitary representation of $\overline{G}_{A}$. Following step by step the arguments of \S9 of \cite{Jacquet-Langlands} we can show that $\overline{\pi}$ must be of the form \eqref{art1-eq9.2.1} with each $\overline{\pi}_{v}$ determined uniquely by $\overline{\pi}$.

\subsection{}\label{art1-sec9.3}
Let $\omega$ denote a character of $(\mathbb{A}^{x})^{2}$ trivial on $(F^{x})^{2}$. Proceeding as in \S10 of \cite{Jacquet-Langlands} we can introduce a space $\overline{A}(\omega)$ of {\em automorphic forms on} $\overline{G}_{A}$. Each $\varphi$ in $\overline{A}(\omega)$ is a genuine $C^{\infty}$ function on $G_{F}\backslash \overline{G}_{A}$ which is\pageoriginale ``slowly increasing'' and transforms under the center (of $\overline{G}_{A}$) according to $\omega$. The group $\overline{G}_{A}$ acts as expected in $\overline{A}(\omega)$ by right translations.

%page 019

