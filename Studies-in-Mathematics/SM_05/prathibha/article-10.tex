\chapter{EISENSTEIN SERIES AND THE RIEMANN ZETA-FUNCTION}


\begin{center}
{\large By~ D. Zagier\footnote{\textit{Supported by the Sonderforschungsbereich ``Theoretische Mathematik'' at the University of Bonn.}}}
\end{center}

\bigskip

\setcounter{pageoriginal}{274}

\textsc{In this paper}\pageoriginale we will consider the functions $E(z, \rho)$ obtained by setting the complex variable $s$ in the Eisenstein series $E(z,s)$  equal to a zero of the Riemann zeta-function and will show that these functions satisfy a number of remarkable relations. Although many of these relations are consequences of more or less well known identities, the interpretation given here seems to be new and of some interest. In particular, looking at the functions $E(z, \rho)$ leads naturally to the definition of a certain representation of $SL_2 (R)$ whose spectrum is related to the set of zeroes of the zeta-function.

We recall that the Eisenstein series $E(z,s)$ is defined for $z = x + iy \in \sfH$ (upper half-plane) and $s \in \bbC$ with $\re (s) > 1$ by
\begin{equation*}
E (z,s) = \sum\limits_{ \gamma \in \Gamma_\infty /\Gamma} \Iim (\gamma z)^s = \frac{1}{2} \sum\limits_{\substack{c, d \varepsilon Z\\ (c,d)=1}} \frac{y^s}{|cz + d|^{2s}} \tag{1}\label{art10-eq1}
\end{equation*}
where $\Gamma = PSL_2 (\bZ)$, $\Gamma_\infty = \left\{\pm \left(\begin{matrix}1 & n \\0 & 1 \end{matrix}\right) \big| n \in \bZ \right\} \subset \Gamma$. If we multiply both sides of \eqref{art10-eq1} by $\zeta(2s) = \sum\limits^\infty_{r =1} r^{-2s}$ and write $m = rc$, $n = rd$, we obtain 
\begin{equation}
\zeta (2s) E (z,s) = \frac{1}{2} \sum\limits_{m,n}' \frac{y^2}{|mz+n|^{2s}},  \label{art10-eq2}
\end{equation}
where $\sum'$ indicates summation over all $(m,n) \in \bZ^2 / \{(0,0)\}$. The function $\zeta(2s) E (z,s)$ has better analytic properties than $E(z,s)$; in particular, it has a holomorphic continuation to all $s$ except for a simple pole at $s =1$.

There is thus an immediate connection between the Eisenstein series at $s$ and the Riemann zeta-function at $2s$. This relationship has been made use of by many authors and has several nice consequences, two of which will be mentioned in $\S 1$. Our main theme, however, is that there is also a relationship between the Eisenstein series and the zeta function at the \textit{same} argument. We will give several examples of this in $\S 2$. Each takes the form that a certain linear operator on the space of functions on $\Gamma/H$, when applied to $E(\cdot, s)$, yields a function of $s$ which is divisible by $\zeta(s)$. Then this operator annihilates all the $E (\cdot, \rho)$, and it is natural to look for a space $\sE$ of functions of $\Gamma / H$ which contains all the $E(\cdot, \rho)$ and which is annihilated by the operators in question. Such a space is defined in \S 3. In \S 4 we show that $\sE$ is the set of $K$-fixed vectors of a certain $G$-invariant subspace $\sV$ of the space of functions on $\Gamma / G$ (where $G = PSL_2 (\bR)$, $K = PSO(2)$). Then $\sV$ is a representation of $G$ whose spectrum with respect to the Casimir operator contains $\rho (1-\rho)$  discretely with multiplicity (at least) $n$ if $\rho$ is an $n$-fold zero of $\zeta(s)$. In particular, if (as seems very unlikely) one could show that $\sV$ is unitarizable, \ie if one could construct a positive definite $G$-invariant scalar product on $\sV$, then the Riemann hypothesis would follow.

The paper ends with a discussion of some other representations of $G$ related to $\sV$ and reformulation in the language of adeles. 

\smallskip
\S~1. We begin by reviewing the most important properties of Eisenstein series. 

a) \textit{Analytic continuation and functional equation.}

The function $E(z,s)$ has a meromorphic continuation to all $s$, the only singularity for $\re (s) > \frac{1}{2}$ being a simple pole at $s =1$ whose residue is independent of $z$:
\begin{equation*}
\res_{s=1} E(z,s) = \frac{3}{\pi} (\forall z \in \sfH). \tag{3}\label{art10-eq3}
\end{equation*}

The modified function
\begin{equation*}
E^\ast (z,s) = \pi^{-s} \Gamma (s) \zeta(2s) E (z,s) \tag{4}\label{art10-eq4} 
\end{equation*}
is regular except for simple poles at $s =0$ and $s =1$ and satisfies the functional equation 
\begin{equation*}
E^{\ast} (z,s) = E^\ast (z, 1 - s). \tag{5}\label{art10-eq5}
\end{equation*}

These statements are proved in a way analogous to Riemann's proof of the\pageoriginale analytic continuation and functional equation of $\zeta(s)$; we rewrite \eqref{art10-eq2} as
\begin{equation*}
E^\ast (z, s) = \frac{1}{2} \pi^{-s}  \Gamma (s) \sum\limits'_{m,n} Q_z (m,n)^{-s} = \frac{1}{2} \int\limits^\infty_{o} (\Theta_z (t) -1)^{s-1} dt, \tag{6}\label{art10-eq6}
\end{equation*}
where $Q_z (m,n) (z \in \sfH)$ denotes the quadratic form 
\begin{equation*}
Q_z (m,n) = \frac{|mz+n|^2}{y } \tag{7} \label{art10-eq7}
\end{equation*}
of discriminant $-4$ and $\Theta_z (t) = \sum\limits_{m,n \in z} e^{-\pi t Q_z(m,n)}$ the corresponding theta-series; then the Poisson summation formula implies $\Theta_z (\frac{1}{t}) = t \Theta_z (t)$ and the functional equation and other properties of $E(z,s)$ follow from this and equation \eqref{art10-eq6}.

b) ``\textit{Rankin-Selberg method}''.

Let $F : \sfH \to \bbC$ be a $\Gamma$-invariant function which is of rapid decay as $y \to \infty$ (\ie. $F (x+ iy) = 0(y^{-N})$ for all $N$). Let
\begin{equation*}
C (F; y) = \int\limits^1_0 F (x + iy) d x \qquad (y>0) \tag{8} \label{art10-eq8}
\end{equation*}
be the constant term of its Fourier expansion and 
\begin{equation*}
I (F; s) = \int\limits^\infty_o C (F; y) y^{s-2} dy \; (\re (s) >1) \tag{9} \label{art10-eq9}
\end{equation*}
the Mellin transform of $C (F; y)$. From \eqref{art10-eq1} we obtain
\begin{equation*}
I(F;s) = \int\limits_{\Gamma_\infty/ \sfH} F (z) y^s d z = \int\limits_{\Gamma / \sfH} F (z) E (z,s) dz, \tag{10}\label{art10-eq10}
\end{equation*}
where $dz$ denotes the invariant volume element $\dfrac{dxdy}{y^2}$. Therefore the properties of $E (z,s)$ given in a) imply the corresponding properties of $I(F;s)$: it can be meromorphically continued, has a simple pole at $s =1$ with 
\begin{equation*}
\res_{s=1} I(F;s) = \frac{3}{\pi} \int\limits_{\Gamma/ \sfH} F (z) d z, \tag{11}\label{art10-eq11}
\end{equation*}
and the function
\begin{equation*}
I^\ast (F;s) = \pi^{-s} \Gamma (s) \zeta (2s) I (F;s)\tag{12}\label{art10-eq12}
\end{equation*}
is regular\pageoriginale for $s \neq 0$, 1 and satisfies 
\begin{equation}
I^\ast (F;s) = I^\ast (F; 1-s) \tag{13}\label{art10-eq13}
\end{equation}

c) \textit{Fourier development.}

The function $E^\ast(z,s)$ defined by \eqref{art10-eq4} has the Fourier expansion
\begin{gather*}
E^\ast (z,s) = \zeta^\ast (2s) y^s + \zeta^\ast (2s -1 ) y^{1-s} \tag{14}\label{art10-eq14}\\
+ 2 \sqrt{y} \sum\limits^\infty_{n=1} n^{s-1/2} \sigma_{1-2s} (n) K_{s-1/2} (2 \pi n y) \cos 2 \pi n x,
\end{gather*}
where 
\begin{align*}
\zeta^\ast (s) & = \pi^{-s/2} \Gamma (\frac{s}{2}) \zeta(s) \qquad (s \in \bbC), \tag{15}\label{art10-eq15}\\
\sigma_v (n) & = \sum\limits_{d |n} d^v \qquad (n\in N, \; v \in \bbC),\\
K_v (t) & = \int\limits^\infty_o e^{-t \cosh u} \cosh v u \; du \qquad (v \in \bbC, \; t > 0).
\tag{16}\label{art10-eq16}
\end{align*}

The expansion \eqref{art10-eq14}, which can be derived without difficulty from \eqref{art10-eq2}, gives another proof of the statements in a); in particular, the functional equation \eqref{art10-eq5} follows from \eqref{art10-eq14} and the functional equations 
$$
\zeta^\ast (s) = \zeta^\ast (1-s) , \sigma_v (n) = n^v \sigma_{-v} (n), K_v (t) = K_{-v} (t). 
$$

Because of the rapid decay of the $K$-Bessel functions \eqref{art10-eq16}, equation \eqref{art10-eq14} also implies the estimates
\begin{align*}
\frac{\partial^n}{\partial s^n} E (z,s) = O (y^{\max (\sigma , 1 -\sigma)} \log^n y) (n = 0, 1, 2,\ldots , \sigma &= \re (s), \tag{17} \label{art10-eq17}\\
y & = \Iim (z) \to \infty)
\end{align*}
for the growth of the Eisenstein series and its derivatives. Finally, it follows from \eqref{art10-eq14} or directly from \eqref{art10-eq1} or \eqref{art10-eq2} that the Eisenstein series $E (z,s)$ are eigenfunctions of both the Laplace operator 
$$
\Delta = y^2 \left(\frac{\partial^2}{\partial x^2} + \frac{\partial^2 }{\partial y^2} \right)
$$
and the Hecke operators 
$$
T (n): F (n) \to \sum\limits_{\substack{ad = n \\ a, d >0}} \sum\limits_{\substack{b (\mod d)}} F (\frac{az+b}{d}) \quad (n>0), 
$$
namely\pageoriginale
\begin{equation*}
\Delta E (z,s) = s (s-1) E (z,s), \quad T (n) E (z,s) = n^s \sigma_{1-2s} (n) E (z,s) . \tag{18}\label{art10-eq18}
\end{equation*}

We now come to the promised applications of the relationship between $E (z,s)$ and $\zeta(2s)$. The first (which has been observed by several authors and greatly generalized by Jacquet and Shalika \cite{art10-3}) is a simple proof of the non-vanishing of $\zeta(s)$ on the line $\re (s) =1$. Indeed, if $\zeta (1+it) = 0$, the \eqref{art10-eq14} implies that the function $F(z) = E (z, \frac{1}{2} + \frac{1}{2}$ it) is of rapid decay, does not vanish identically, and has constant term $C(F;y)$ identically equal to 0. But then $I(F;s) =0$ for all $s$, and takings $s = \dfrac{1}{2} - \dfrac{1}{2}$ it in \eqref{art10-eq10} we find $\int\limits_{\Gamma / \sfH} |F(z)|^2 d z = 0$, a contradiction.

The second ``application'' is a direct but striking consequence of the Rankin-Selberg method. Let $\sC_y \subset \Gamma / \sfH$ be the horocycle $\Gamma_\infty/ (\bR+ iy)$; it is a closed curve of (hyperbolic) length $\dfrac{1}{y}$. The claim is that, as $y \to 0$, the curye $\sC_y$ ``fills up'' $\Gamma / \sfH$ in a very uniform way: not only does $\sC_y$ meet any open set $U \subset \Gamma / \sfH$ for $y$ sufficiently small, but the fraction of $\sC_y$ contained in $U$ tends to $\vol (U)/ \vol (\Gamma / \sfH)$ as $y \to 0$ and in fact
$$
\frac{\text{length }(\sC_y \cap U)}{\text{length} (\sC_y)} = \frac{\vol (U)}{\vol (\Gamma/ \sfH)} + 0 (y^{\frac{1}{2} - \varepsilon}) \quad (y \to 0);
$$
moreover, if the error term in this formula can be replaced by $0(y^{3/4 - \varepsilon})$ for all $U$, then the Riemann hypothesis is true! To see this, take $F (z)$ in b) to be the characteristic function $\chi_U$ of $U$. Then $C (F; y) = \dfrac{\text{length} (\sC_y \cap U)}{\text{length } (\sC_y)}$ and the Mellin transform $I (F, s)$ of this is holomorphic in $\re (s) > \frac{1}{2} \Theta$ (where $\Theta$ is the supremum of the real parts of the zeroes of $\zeta (s)$) except for a simple pole of residue $\kappa = \frac{3}{4} \int\limits_{\Gamma / \sfH} F (z) d z = \dfrac{\vol (U)}{\vol (\Gamma / \sfH)}$ at $s =1$. If $F$ were sufficiently smooth (say twice differentiable) we could deduce that $I(F; \sigma + it) = 0 (t^{-2})$ on any vertical strip $\re(s) = \sigma > \frac{1}{2} \Theta$, and the Mellin inversion formula would give $C (F; y) = \kappa + 0 (y^{1-\frac{1}{2} \Theta -\varepsilon})$. For $F = \chi_U$ we can\pageoriginale prove only $C (F; y) = \kappa + 0 (y^{\frac{1}{2} - \varepsilon})$; conversely, however, if $C (F; y) = \kappa + 0(y^\alpha)$ then $I (F; s) - \dfrac{\kappa}{s-1}$ is holomorphic for $\re(s) >1-\alpha$, and if this holds for all $F = \chi_U$ we obtain $\Theta \leqslant 2 (1-\alpha)$.

\S 2. In this section we give examples of special properties of the functions $E^\ast(z, \rho)$ or, more generally, of the functions 
\begin{equation*}
\left. F(z) = F_{\rho, m} (z) = \frac{\partial^m}{\partial s^m} E^\ast (z,s) \right|_{s = \rho} \quad (0 \leq m \leq n_\rho -1), 
\tag{19}\label{art10-eq19}
\end{equation*}
where $\rho$ is a non-trivial zero of $\zeta(s)$ of order $n_\rho$.

\medskip
\noindent
{\bfseries Example \thnum{1}:\label{art10-exam1}}


Let $D < 0 $ be the discriminant of an imaginary quadratic field $K$. To each positive definite binary quadratic form $Q(m,n)= am^2 + b mn+ cn^2$ of discriminant $D$ we associate the root $z_Q  = \dfrac{-b+ \sqrt{D}}{2a} \in \sfH$. The $\Gamma$-equivalence class of $Q$ determines uniquely an ideal class $A$ of $K$ such that the norms of the integral ideals of $A$ are precisely the integers represented by $Q$. Also, the form $Q_{z_Q}$ defined by \eqref{art10-eq7} equals $\dfrac{2}{\sqrt{|D|}} Q$. Therefore \eqref{art10-eq6} gives 
\begin{align*}
E^\ast (z_Q , s) & = \frac{1}{2} \left(\dfrac{|D|}{4} \right)^{s/2} \pi^{-s} \Gamma (s) \sum\limits'_{m,n} Q (m,n)^{-s} \\
& = \frac{w}{2} \left(\frac{|D|}{4} \right)^{s/2} \pi^{-s} \Gamma (s) \zeta (A, s), 
\end{align*}
where $w (= 2, 4$ or 6) is the number of roots of unity in $K$ and $\zeta (A, s) = \sum\limits_{\fa}^{-s}$  is the zeta-function of $A$. (Note that this equation makes sense because $E^\ast (z_Q, s)$ depends only on the $\Gamma$-equivalence class of $z_Q$ and hence of $\bQ$.) Thus if $Q_1, \ldots, Q_{h(D)}$ are representatives for the equivalence classes of forms of discriminant $D$, we have 
\begin{align*}
\sum\limits^{h(D)}_{i=1} E^\ast (z_{Q_i}, s) & = \frac{w}{2} \left(\frac{|D|}{4} \right)^{s/2} \quad \pi^{-s} \Gamma (s) \sum\limits^{h(D)}_{i=1} \zeta (A_i,s)\\
& = \frac{w}{2} \left(\frac{|D|}{4} \right)^{s/2} \quad \pi^{-s} \Gamma (s) \zeta_K (s) \\
& = \frac{w}{2} \left(\frac{|D|}{4} \right)^{s/2} \quad \pi^{-s} \Gamma (s) \zeta (s) L(s, D),
\end{align*}
where\pageoriginale $\zeta_K (s)$ is the Dedekind zeta-function of $K$ and $L(s,D)$ the $L$-series $\sum\limits^\infty_{n=1}\left(\dfrac{D}{n}\right) n^{-s}$. Since the latter is holomorphic, we deduce that the function $\sum\limits^{h(D)}_{i=1} E^{\ast} (z_{Q_i}, s)$ is divisible by $\Gamma (s) \zeta (s)$, \ie that it vanishes with multiplicity $\geq n_\rho$ at a non-trivial zero $\rho$ of $\zeta (s)$. A similar statement holds for any negative integer $D$ congruent to 0 or 1 modulo 4 (not necessarily the discriminant of a quadratic field) if we replace $\zeta_K (s)$ in the equation above by the function 
\begin{equation}
\zeta(s, D) = \sum\limits^{h(D)}_{i =1} \sum\limits_{\substack{(m,n) \varepsilon Z^2 / \Gamma_{Q_i}\\ Q_i (m,n)>0}} \frac{1}{Q_i (m, n)^s}
\tag{20}\label{art10-eq20}
\end{equation}
where $Q_i(i=1,\ldots, h (D))$ are representatives for the $\Gamma$-equivalence classes of binary quadratic forms of discriminant $D$ and $\Gamma_{Q_i}$ denotes the stabilizer of $Q_i$ in $\Gamma$. Again the quotient $L (s, D) = \zeta (s, D) / \zeta(s)$ is entire (\cite{art10-11}, Prop. 3, ii), p. 130). This proves

\medskip
\noindent
{\bfseries Proposition \thnum{1}:\label{art10-prop1}}
\textit{Each of the functions \eqref{art10-eq19} satisfies}
\begin{equation}
\sum\limits^{h(D)}_{i=1} F(z_{Q_i}) = 0 \tag{21}  \label{art10-eq21}
\end{equation}
\textit{for all integers $D <0$, where $z_{Q_1}, \ldots, z_{Q_{j(D)}}$ are the points in $\Gamma / \sfH$ which satisfy a quadratic equation with integral coefficients and discriminant $D$.}

Notice how strong condition \eqref{art10-eq21} is: the points satisfying some quadratic equation over $\bZ$ (``points with complex multiplication'') lie dense in $\Gamma/ \sfH$, so that it is not at all clear $a$ priori that there exists any non-zero continuous function $F: \Gamma / \sfH \to \bbC$ satisfying eq. \eqref{art10-eq21} for all $D <0$.

\medskip
\noindent
{\bfseries Example \thnum{2}:\label{art10-exam2}}

This is the analogue of Example \ref{art10-exam1} for positive discriminants. Let  $D > 0$ be the discriminant of a real quadratic field $K$ and $Q_1, \ldots, Q_{h(D)}$ representatives for the $\Gamma$-equivalence classes of quadratic forms of discriminant $D$. To each $Q_i$  we associate, not a point $z_{Q_i} \in \Gamma / \sfH$ as before, but a closed curve $C_{Q_i} \subset \Gamma / \sfH$ as follows: Let $w_i$, $w'_i \in \bR$ be the roots of the quadratic\pageoriginale equation $Q_i(x,1) = a_i x^2 + b_i x + c_i = 0$ and let $\Omega_i$ be the semicircle in $\sfH$ with endpoints $w_i$ and $w'_i$. The subgroup
\begin{equation*}
\Gamma_{Q_i} = 
\left\{
\left.  \pm 
\left(\begin{matrix} 
\frac{1}{2} (t - b_i u) & -c_i u \\
a_i u & \frac{1}{2} (t + b_i u)   
\end{matrix}\right) 
\right|
t, u \in \bZ , t^2 - D u^2  = 4 \right\}\tag{22}
\label{art10-eq22}
\end{equation*}
of $\Gamma$, which is isomorphic to $\{$units of $K\} / \{\pm 1\}$  and hence to $\bZ$, maps $\Omega_i$ to itself, and $C_{Q_i}$ is the image $\Gamma_{Q_i} / \Omega_i$ of $\Omega_i$ in $\Gamma / \sfH$. On $C_{Q_i}$ we have a measure $|d_{Q_i} z|$, unique up to a scalar factor, which is invariant under the operation of the group $\Gamma_{Q_i} \otimes \bR$  obtained by replacing $\bZ$ by $\bR$ in \eqref{art10-eq22}; if we parametize $\Omega_i$ by
$$
z = \frac{w_i i p + w'_i }{i p+1} \qquad (0 < p < \infty), 
$$
then $\Gamma_{Q_i}$ acts by $p \to \varepsilon^2 p$ ($\varepsilon = \dfrac{t + u \sqrt{D}}{2}$ a unit of $K$)  and $|d_{Q_i} z| = \dfrac{dp}{p}$. A theorem of Hecke (\cite{art10-2}, p. 201) asserts that the zeta-function of the ideal class $A_i$ of $K$ corresponding to $Q_i$ is given by 
$$
\zeta(A_i, s) = \frac{\pi^s}{\Gamma (\frac{s}{2})^2} D^{-s/2}\int\limits_{C_{Q_i}} E^\ast (z,s) |d_{Q_i} z|
$$
(\cf \cite{art10-10},  \S 3 for a sketch of the proof). Thus
$$
\sum\limits^{h(D)}_{i=1} \int\limits_{C_{Q_i}} E^\ast (z,s) |d_{Q_i} z| = \pi^{-s} D^{s/2} \Gamma (\frac{s}{2})^2 \zeta_K (s),
$$
which again is divisible by $\zeta(s)$, and as before we can take for $D$ any positive non-square congruent to 0 or 1 modulo 4 and get a similar identity with $\zeta_K(s)$ replaced by the function \eqref{art10-eq20}. Thus we obtain

\medskip
\noindent
{\bfseries Proposition \thnum{2}:\label{art10-prop2}}
\textit{Each of the functions \eqref{art10-eq19} satisfies}
\begin{equation}
\sum\limits^{h(D)}_{i=1} \int_{C_{Q_i}} F (z) |d_{Q_i} z| = 0 \tag{23}\label{art10-eq23} 
\end{equation}
\textit{for all non-square integers $D < 0$, where $Q_i (m,n) = a_i m^2 + b_i mn + c_i n^2$ $(i=1,\ldots, h(D))$\pageoriginale are representatives for the $\Gamma$-equivalence classes of binary quadratic forms of discriminant $D$, $C_{Q_i}$ is the image of $\{\left.z = x+ iy \in \sfH \right| a_i |z|^2 +  b_i x+ c_i =0\}$ in $\Gamma / \sfH$, and}
$$
|d_{Q_i} z| = \frac{\sqrt{D}}{|a_i z^2 + b_i z + c_i|} ((dx)^2 + (dy)^2)^{1/2}.
$$

\medskip
\noindent
{\bfseries Example \thnum{3}:\label{art10-exam3}}
The third example comes from the theory of modular forms. Let $f(z)$ be a cusp form of weight $k$ on $SL_2 (\bZ)$ which is a normalized eigenfunction of the Hecke operators, \ie $f$ satisfies 
$$
f\left(\frac{az+b}{cz+d} \right) = (cz + d)^k f(z) \quad (\forall z \in \sfH, \quad \left(\begin{matrix} a & b \\c & d 
\end{matrix}\right) \in SL_2 (\bZ))
$$
and has a Fourier development of the form 
$$
f(z) = \sum\limits^\infty_{n=1} a_n e^{2 \pi in  z}
$$
with $a_1 = 1$ and $a_{nm} = a_n a_m$ if $(n,m) =1$. Define $D_f (s)$ by 
$$
D_{f} (s) = \prod\limits_p \frac{1}{(1-\alpha^2_p p^{-s}) (1-\alpha_p \beta_p p^{-s})(1-\beta^2_p p^{-s})}  (\re (s) >> 0),
$$
where $\alpha_p$ and $\beta_p$ are the roots of $X^2 - a_p X + p^{k-1} = 0$; it is easily checked that 
$$
D_f (s) =\frac{\zeta (2s - 2k + 2)}{\zeta (s-k +1)} \sum\limits^\infty_{n=1} |a_n|^2 n^{-s}.
$$

Applying the Rankin-Selberg method \eqref{art10-eq10} to the $\Gamma$-invariant function $F(z) = y^k |f(z)|^2$ with constant term $C(F: y^k) = \int\limits^\infty_{n=1} |a_n|^2 e^{-4 \pi n y}$, we find $\int\limits_{\Gamma/ \sfH} y^k |f(z)|^2 E^\ast (z, s) d z = I^\ast (F;s)$
\begin{align*}
& = \pi^{-s} \Gamma (s) \zeta (2s) \cdot (4 \pi)^{-s-k+1} \Gamma (s+k -1) \sum\limits^\infty_{n=1} \frac{|a_n|^2}{n^{s+k-1}}\\
& = 4^{-s-k+1} \pi^{-2s -k +1} \Gamma (s) \Gamma (s+k-1) \zeta (s) D_f (s+k-1).
\end{align*}

This\pageoriginale formula, which was the original application of the Rankin-Selberg method (\cite{art10-5}, \cite{art10-6}), shows that the product $\zeta(s) D_f (s+k-1)$ is holomorphic except for a simple pole at $s =1$. It was proved by Shimura \cite{art10-7} and also by the author \cite{art10-11} that in fact $D_f (s)$ is an entire function of $s$. Thus the above integral is divisible by $\Gamma (s) \Gamma (s+ k -1) \zeta(s)$, and we obtain

\medskip
\noindent
{\bfseries Proposition \thnum{3}:\label{art10-prop3}}
\textit{Each of the functions \eqref{art10-eq19} satisfies}
\begin{equation*}
\int\limits_{\Gamma/ \sfH} y^k |f(z)|^2 F (z) dz = 0 \label{art10-eq24}
\end{equation*}
\textit{for every normalized Hecke eigenform $f$ of level 1 and weight $k$.}

The statement of Proposition \ref{art10-prop3} remains true if $f$ is allowed to be a non-holomorphic modular form (Maass wave-form); the proof for $k=0$ is given in \cite{art10-12} in this volume and the general case is included in the results of \cite{art10-1} or \cite{art10-4}.

Finally, we can extend our list of special properties of the functions \eqref{art10-eq19} by observing that each of these functions is an eigenfunction of the Laplace and Hecke operator (eq. \eqref{art10-eq18}) and hence (trivially) has the property that
\begin{equation*}
\Delta^i F (z) \text{ and } T (n) F (z) \text{ satisfy Proposition \ref{art10-prop1}-\ref{art10-prop3} for all } i \geqslant 0, \; n \geqslant 1. \tag{25}\label{art10-eq25}
\end{equation*}

Note that for general functions $F: \Gamma / \sfH \to \bbC$ (not eigenfunctions), eq. \eqref{art10-eq25} expresses a property no contained in Proposition \ref{art10-prop1} to \ref{art10-prop3}: for example, eq. \eqref{art10-eq21} for $D = - 4$ says that $F (i) =0$, and this does not imply $\Delta F(i) =0$.

\S~3. In \S~2 we proved that the functions $E^\ast (z, \rho)$, and more generally the functions \eqref{art10-eq19}, satisfy a number of special properties. In this section we will both explain and generalize these results by defining in a natural way a space $\sE$ of functions in $\Gamma / \sfH$ which contains the functions $\eqref{art10-eq19}$ and has the same special properties. 

Let $D$ be an integer congruent to 0 or 1 modulo 4. For $\Phi: \bR \to \bbC$ a function satisfying certain restrictions (e.g. \eqref{art10-eq27} and \eqref{art10-eq29} below) we define a new function $\sL_D \Phi: \sfH \to \bbC$ by
\begin{equation*}
\sL_D \Phi (z) = \frac{1}{2} \sum\limits_{\substack{a, b, c \in Z\\ b^2- 4 a c = D}} \Phi \left(\frac{a|z|^2 + b z +c}{y} \right) \quad (z = x + iy \in \sfH), \tag{26}\eqref{art10-26}
\end{equation*}
where the summation extends over all integral binary quadratic forms $Q(m,n) = am^2 + bmn + cn^2$ of discriminant $D$. Since $Q$ and $-Q$ occur together in the sum, we may assume that $\phi$ is an even function; the factor $\frac{1}{2}$ has then been included in the definition to avoid counting each term twice. 

The sum \eqref{art10-eq26} converges absolutely for all $z \in \sfH$ if we assume that 
\begin{equation*}
\Phi (X) = O(|X|^{-1-\varepsilon}) \quad (|X| \to \infty) \tag{27}\label{art10-eq27}
\end{equation*}
for some $\varepsilon > 0$. Moreover, the expression $\dfrac{a|z|^2 + bx +c}{y}$ is unchanged if one acts simultaneously on $x + iy \in \sfH$ and $Q(m,n) = am^2 + bmn + cn^2$ by an element $\gamma \in \Gamma$. Hence $\sL_D \Phi (\gamma z) = \sL_D \Phi (z)$, so $\sL_D$ is an operator from functions on $\bR$ satisfying \eqref{art10-eq27} to functions on $\Gamma / \sfH$.

Before going on, we need to know something about the growth of $\sL_D\Phi$ in $\Gamma / \sfH$. If $D$ is not a perfect square, then $a \neq 0$ in \eqref{art10-eq26}, so (for $\Phi$ even)
\begin{align*}
\sL_D \Phi (z)  & = \sum\limits^\infty_{a=1} \sum\limits^\infty_{\substack{b = -\infty\\ b^2 \equiv D (\mod 4a)}} \Phi \left(ay + \frac{a(x+ b/ 2 a)^2 - D / 4a}{y} \right)\\
& = \sum\limits^\infty_{a=1} \sum\limits^\infty_{\substack{b = - \infty\\ b^2 \equiv D (\mod 4a)}} O \left(\left(ay+ \frac{a(x + b / 2 a)^2 - D / 4a}{y} \right)^{-1-\varepsilon}\right)\\
& = \sum\limits^\infty_{a=1} O \left(n_D (a) \int\limits^\infty_{-\infty} \left(ay + \frac{ax^2}{y} \right)^{-1-\varepsilon} dx\right)
\end{align*}
as $y = \Iim (z) \to \infty$, where 
\begin{equation*}
n_D (a) = \neq \left\{\left. b (\mod 2 a) \right| b^2 \equiv D (\mod 4 a) \right\}. \tag{28} \label{art10-eq28}
\end{equation*}
Since the integral is $O(a^{-1-\varepsilon} y^{-\varepsilon})$ and $\sum\limits^\infty_{a=1} \dfrac{n_D (a)}{a^{1+\varepsilon}}$ converges, we find $\sL_D \Phi (z) = O (y^{-\varepsilon})$.\pageoriginale If $D$ is a square, then the same argument applies to the terms in \eqref{art10-eq26} with 
$a \neq 0$ and we are left with the sum 
$$
\frac{1}{2} \sum\limits_{b^2 =D} \sum\limits^\infty_{c = - \infty} \Phi (\frac{bx+c}{y})
$$
to estimate. If $\Phi$ is sufficiently smooth (say twice differentiable), then the inner sum differs by a small amount from the corresponding integral 
$$
\int\limits^\infty_{-\infty} \Phi (\frac{c}{y}) dc = y \cdot \frac{\infty}{-\infty} \Phi (X) d X,
$$
and this will be small as $y \to \infty$ only if $\int\limits^\infty_{-\infty} \Phi (X) dX$ vanishes. Thus with the requirement
\begin{equation*}
\Phi \text{ is } C^2 \text{ and } \int\limits^\infty_{-\infty} \Phi (X) d X = 0 \text{ if } D \text{ is a square } \tag{29}\label{art10-eq29}
\end{equation*}
as well as \eqref{art10-eq27} we have $\sL_D \Phi (z) = O(y^{-\varepsilon})$ as $y \to \infty$ for all $D$, and so the scalar product of $\sL_D \Phi$ with $F$ in $\Gamma / \sfH$ converges for any $F :\Gamma / \sfH \to \bbC$ satisfying $F(z) = O(y^{1-\varepsilon})$ for some $\varepsilon > 0$ (or even $F(z) - O(y)$). Therefore the definition of $\sE$ in the following theorem makes sense. 

\begin{theorem*}
\textit{For each integer $D \in Z$, $D \equiv 0$ or 1 ($\mod 4$), let }
$$
\sL_D \left\{\begin{aligned}
& \text{even functions } \Phi: \bR \to\bbC\\
& \text{satisfying \eqref{art10-eq27} and \eqref{art10-eq29}}
\end{aligned}
\right\} \longrightarrow \{\text{functions } \Gamma /\sfH \to \bbC \}
$$
\end{theorem*}
\textit{be the operator defined by \eqref{art10-eq26}}. Let $\sE$ be the set of functions $F: \Gamma / \sfH \to \bbC$ such that 
\begin{itemize}
\item[a)] \textit{$F(z) = O(y^{1-\varepsilon})$ for some  $\varepsilon > 0$}

\item[b)] \textit{$F (z)$  is orthogonal to  $\sum\limits_D \Iim (\sL_D)$, \ie $\int\limits_{\Gamma / \sfH} \sL_D \Phi (z) F(z) d z = 0$ for all $D \in \bZ$ and all $\Phi$ satisfying \eqref{art10-eq27} and \eqref{art10-eq29}.}
\end{itemize}

\textit{Then }
\begin{itemize}
\item[i)] \textit{$\sE$ contains the functions \eqref{art10-eq19};}

\item[ii)] \textit{$\sE$ is closed under the action of the Laplace and Hecke operators;}

\item[iii)] \textit{Any $F \in \sE$ satisfies \eqref{art10-eq21}, \eqref{art10-eq23} (for all $D$) and \eqref{art10-eq24} (for all $f$).}
\end{itemize}

\begin{proof}
i) The functions \eqref{art10-eq19} satisfy a) because of equation \eqref{art10-eq17}, since $0< \re (\rho)<1$. To prove b), we must show that the integral of any function\pageoriginale $\sL_D \Phi (z)$ against $E^\ast (z,s)$ is divisible by $\zeta(s)$. Consider first the case when $D$ is not a square. Let $\Phi$ be any function satisfying \eqref{art10-eq27} and $F : \Gamma / \sfH \to \bbC$ a function which is $O(y^\alpha)$ as $y \to \infty$ for some $\alpha \leqslant 1 + \varepsilon$. If $Q_i (m, n) = a_i m^2 + b_i mn + c_i n^2 (i = 1, \ldots, h (D))$ are representatives for the classes of binary quadratic forms of discriminant $D$, then any form of discriminant $D$ equals $Q_i \circ \gamma$ for a unique $i$ and $\gamma \in \Gamma_{Q_i} / \Gamma$ ($\Gamma_{Q_i} =$ stabilizer of $Q_i$ in $\Gamma$). Hence
$$
\sL_D \Phi (z) = \sum\limits^{h(D)}_{i=1} \sum\limits_{\gamma \in \Gamma_{Q_i}/ \Gamma} \Phi \left(\frac{a_i |\gamma z|^2+ b_i \re (\gamma z) + c_i}{\Iim (\gamma z)} \right)
$$
and so 
$$
\int\limits_{\Gamma/\sfH} \sL_D \Phi (z) F (z) dz = \sum\limits^{h(D)}_{i=1} \int\limits_{\Gamma_{Q_i}/ \sfH} \Phi \left(\frac{a_i |z|^2 + b_i x +c_i}{y} \right) F (z) dz.
$$ 
Taking $F(z) = \zeta(2s) E (z,s)$ with $1 \leqslant \re (s) < 1 + \varepsilon$ and using \eqref{art10-eq2}, we find that the right-hand side of this equations equals
$$
\frac{1}{2} \sum\limits^{h(D)}_{i=1} \sum\limits_{(m,n) \in Z^2 / \Gamma_{Q_i}} \int\limits_{\sfH} \Phi \left(a_i |z|^2+ b_i x + c_i \right)  \frac{y^s}{|mz + n|^2{2s}} d z.
$$

Since $D$ is not a square, $Q_i (n, -m)$ is different from 0 for all $(m,n) \neq (0,0)$, so, since $\Phi$ is an even function, we can restrict the sum to $(m,n) \in Z^2$ with $Q_i (n, -m)>0$ if we drop the factor $\frac{1}{2}$. Then the substitution $z \to \dfrac{nz - \frac{1}{2} b_i n + c_i m}{-mz+ a_i n - \frac{1}{2} b_i m}$ introduced in \cite{art10-11}, p. 127, maps $\sfH$ to $\sfH$ and gives 
\begin{multline*}
\int\limits_H \Phi \left(\frac{a_i |z|^2 + b_i x + c_i}{y} \right) \frac{y}{|mz+ n|^{2s}} dz\\
=Q_i (n, -m)^{-s} \int_\sfH \Phi \left(\frac{|z|^2 -D/4}{y} \right) y^s dz. 
\end{multline*}
Therefore we have 
\begin{equation}
\zeta(2s) \int\limits_{\Gamma / \sfH} \sL_D \Phi (z) E (z, s) dz = \zeta (s, D) \int\limits_\sfH \Phi \left(\frac{|z|^2- D /4}{y} \right) y^s dz \tag{30}  \label{art10-eq30}
\end{equation}
for $1 < \re (s) <1 + \varepsilon$, with $\zeta (s,D)$ defined as in \eqref{art10-eq20}. Since $\zeta(2s)$ and $\zeta(s, D)$\pageoriginale have meromorphic continuations to all $s$ and both integrals in \eqref{art10-eq30} converge for $0< \re (s) < 1+ \varepsilon$, we deduce that the identity is valid in this larger range; the divisibility of $\zeta (s; D)$ by $\zeta(s)$ now implies the orthogonality of the functions \eqref{art10-eq19} with $\sL_D \Phi (z)$.

If $D$ is a square, we would have to treat the terms with $Q_i (n,-m)=0$ in the above sum separately (as in \cite{art10-11}, pp. 127-128). We prefer a different method, which in fact works for all $D$. By the Rankin-Selberg method, we know that $\int \sL_D \Phi (z) E (z,s)dz$ equals the Mellin transform of the constant term of $\sL_D \Phi$, and writing $\sL_D \Phi (z)$ as
\begin{multline*}
\sum\limits^\infty_{a=1} \sum\limits_{\substack{b(\mod 2 a)\\ b^2 \equiv D (\mod 4a)}} \sum\limits^\infty_{n = - \infty} \Phi \left(\frac{a\left|z  + \frac{b}{2a} + n\right|^2 - D / 4a}{y} \right)\\
+ \frac{1}{2} \sum\limits_{b^2 = D} \sum\limits^\infty_{c = - \infty} \Phi \left(\frac{bx+c}{y} \right), 
\end{multline*}
we see that this constant term is given by 
\begin{gather*}
C(\sL_D \Phi; y) = \sum\limits^\infty_{a =1} n_D (a) \int\limits^\infty_{-\infty} \Phi \left(\frac{ax^2 + a y^2 - D / 4 a}{y} \right) d x\\
+\left\{
\begin{aligned}
& 0 \text{ if } D \neq m^2,\\
& y. \int\limits^\infty_{-\infty} \Phi (X) d X \text{ if } D = m^2 > 0,\\
& \frac{1}{2} \sum\limits^\infty_{c = - \infty} \Phi \left(\frac{c}{y} \right) \text{ if } D = 0,
\end{aligned}
\right.
\end{gather*} 
where $n_D (a)$ is defined by \eqref{art10-eq28}. The Mellin transform of the first term is 
$$
\left(\sum\limits^\infty_{a=1} \frac{n_D (a)}{a^s} \right) \cdot \int\limits^\infty_{0} \int\limits^\infty_{-\infty} \Phi \left(\frac{x^2 + y^2 - D/4}{y} \right) y^{s-2} d x dy,
$$
and since $\sum n_D (a) a^{-s} = \zeta(s, D)/ \zeta (2s)$ (\cite{art10-11}, Prop. 3, i), p. 130) we recover eq. \eqref{art10-eq30} if $D$ is not a square. The second term vanishes if $D = m^2 \neq 0$ because of the assumption \eqref{art10-eq29}, so eq. \eqref{art10-eq30} remains valid in this case. If $D =0$, then, using equation \eqref{art10-eq29} and the Poisson summation formula,\pageoriginale we see that the second term in the formula for $C (\sL_0 \Phi; y)$ equals 
$$
\frac{1}{2} \sum\limits^\infty_{c = - \infty} \Phi (\frac{c}{y}) - \frac{1}{2} y \int\limits^\infty_{-\infty} \Phi (X) dX =  y \sum\limits^\infty_{n =1} \tilde{\Phi} (ny), 
$$
where 
$$
\tilde{\Phi} (y) = \int\limits^\infty_{-\infty} \Phi (X) e^{2 \pi i X y} d X
$$
is the Fourier transform of $\Phi$. The Mellin transform of this is $\zeta(s)$ times the Mellin transform of $\tilde{\Phi}$, so we obtain 
\begin{align*}
\zeta (2s) \int\limits_{\Gamma / \sfH} \sL_0 \Phi (z) E (z,s)& d z  = \zeta (s, 0) \int\limits_{\sfH} \Phi \left(\frac{|z|^2}{y} \right) y^s dz\\
& + \zeta(s) \zeta(2s) \int\limits^\infty_{0} \tilde{\Phi} (y) y^{s-1} dy \tag{31}\label{art10-eq31}
\end{align*}
for $1 < \re (s) < 1+ \varepsilon$. Again both sides extend meromorphically to the critical strip and, since $\zeta(s,0) = \zeta (s) \zeta (2s-1)$, we again find that the integral on the left is divisible by $\zeta(s)$, \ie that the functions \eqref{art10-eq19} are orthogonal to the image of $\sL_0$. This completes the proof of i).

We observe that the same calculations as in \cite{art10-12}, \S 4, allow us to perform one of the integrations in the double integral on the right-hand side of \eqref{art10-eq30}, obtaining 
\begin{align*}
& \int\limits_{\Gamma/ \sfH} \sL_D \Phi (z) E^\ast (z,s) dz \tag{32}\label{art10-eq32}\\
& = \left\{
\begin{aligned}
& (2\pi)^{1-s} |D|^{s/2} \Gamma (s) \zeta (s, D) \int\limits^\infty_1 P_{s-1} (t) \phi (|D|^{\frac{1}{2}} t) dt \qquad \text{ if } D < 0, \\
& \frac{1}{2} \pi^{-s} D^{s/2} \Gamma \left(\frac{s}{2} \right)^2 \zeta (s,D) \int\limits^\infty_0 F \left(\frac{s}{2}, \frac{1-s}{2}; \frac{1}{2}; -t^2 \right) \Phi (D^{\frac{1}{2}} t) dt ~ \text{if } D > 0, 
\end{aligned}
\right.
\end{align*}
where $P_{s-1} (t)$ and $F \left(\dfrac{s}{2}, \dfrac{1-s}{2}; \dfrac{1}{2}; - t^2 \right)$ denote Legendre and hypergeometric functions, respectively; since both of these functions are invariant under $s \to 1 - s$, we see that \eqref{art10-eq30} is compatible with (and indeed gives another proof of) the functional equation of $\zeta (s, D)$ for $D \neq 0$ (\cite{art10-11}, Prop. 3, ii), p. 130).\pageoriginale We can also make the functional equation apparent in the case $D = 0$ by substituting $-\frac{1}{z}$ for $z$ in the first integral on the right-hand side of \eqref{art10-eq31} and using the identity
$$
\int\limits^\infty_{0} y^{s-1} \cos 2 \pi X y \; dy = \frac{1}{2} \pi^{\frac{1}{2} -s} \frac{\Gamma (\frac{s}{2})}{\Gamma (\frac{1-s}{2})} |X|^{-s} \quad (0 < \re (s) < 1)
$$
in the second; this gives 
\begin{align*}
& \int\limits_{\Gamma / \sfH} \sL_0 \Phi (z) E^{\ast} (z,s) dz \tag{33}\label{art10-eq33}\\
& = \zeta (s) \zeta^\ast (2 - 2 s) \int\limits^\infty_{0} \Phi (X) X^{s-1} dX + \zeta (1-s) \zeta^{\ast} (2s) \int\limits^\infty_{0} \Phi (X) X^{-s} d X
\end{align*}
for $0 < \re (s) <1$, with $\zeta^\ast (s)$ as in eq. \eqref{art10-eq15}.

A calculation similar to the one given here can be found in \S 2 of Shintani \cite{art10-8}.

\medskip
ii) Since both the Laplace and the Hecke operators are self-adjoint, it is sufficient to show that the space $\sum\limits_D \Iim (\sL_D)$, or a dense subspace of it, is closed under the action of these operators. An elementary calculation shows that 
\begin{equation*}
\Delta \Phi \left(\frac{a|z|^2 + b x+ c}{y} \right) = \Phi_1 \left(\frac{a|z|^2 + b x + c}{y} \right) \tag{34}\label{art10-eq34}
\end{equation*}
with 
\begin{equation*}
\Phi_1 (X) = 2 X \Phi'(X) + (X^2+ D) \Phi'' (X). \tag{35} \label{art10-eq35}
\end{equation*}
Hence $\Delta \sL_D \phi (z) = \sL_D \Phi_1 (z)$. If $\Phi$ is $C^\infty$ and of rapid decay, then $\Phi_1$ also is and satisfies conditions \eqref{art10-eq27} and \eqref{art10-eq29}, and since such $\Phi$ form a dense subspace the first assertion of ii) is proved.

The calculation for the Hecke operators is harder. It suffices to treat the operators $T(p)$ with $p$ prime, since these generate the Hecke algebra. We claim that 
\begin{equation*}
T (p) \circ \sL_D =\sL_{Dp^2} \circ  \alpha_p + \left(\frac{D}{p} \right) \sL_D + p\sL_{D/p^2} \circ \beta_p
\tag{36} \label{art10-eq36}
\end{equation*}
where\pageoriginale $\alpha_p$ and $\beta_p$ denote the operators
$$
\alpha_p \Phi (X) = \Phi (X/ p), \beta_p \Phi (X) = \Phi (pX),
$$
$\left(\dfrac{D}{p} \right)$ is the Legendre symbol, and $\sL_{D/p^2}$ is to be interpreted as 0 if $p^2 \nmid D$.  To prove this write
\begin{align*}
T (p) \sL_D \Phi (z) & = \sL_D \Phi (pz) + \sum\limits^p_{j=1} \sL_D \phi \left(\frac{z+j}{p} \right)\\
& = \sum\limits_{b^2 - 4 a c = D} \left\{\Phi \left(\frac{ap^2 |z|^2 + bpx + c}{py} \right) \right.\\
&\left.  + \sum\limits^p_{j=1} \Phi \left(\frac{a|z|^2 + (2aj + bp) x + (aj^2 + b jp + cp^2)}{py} \right) \right\}\\
& = \sum\limits_{b^2 - 4 a c = D p^2} n(a, b, c) \Phi \left(\frac{a|z|^2 + b x+ c}{py} \right)
\end{align*}
with 
$$
n (a, b, c) = \varepsilon \left(\frac{a}{p^2}, \frac{b}{p}, c \right) + \sum\limits^p_{j=1} \varepsilon \left(a, \frac{b - 2 aj}{p}, \frac{c-bj+ aj^2}{p^2} \right)
$$
(where $\varepsilon (a, b, c)$ equals 1 if $a, b, c$ are integral, 0 otherwise). To prove \eqref{art10-eq36} we must show that 
\begin{gather*}
n (a, b, c) = 1+ \left(\frac{D}{p} \right) \varepsilon \left(\frac{a}{p}, \frac{b}{p}, \frac{c}{p} \right) + p \varepsilon \left(\frac{a}{p^2}, \frac{b}{p^2}, \frac{c}{p^2} \right)\\
(a, b, c \in \bZ, \; b^2 - 4 a c = Dp^2).
\end{gather*}
For $p$ odd, this follows from the following table, in which $v_{p^1} (m)$ denotes the exact power of $p$ dividing an integer $m$.
\begin{center}
{\fontsize{9}{11}\selectfont
\renewcommand{\arraystretch}{1.2}
\tabcolsep=3pt
\begin{tabular}{ccc|cccc}
$v_{p^1} (a)$ & $v_{p^1}(b)$ & $v_{p'}(c)$ & $\varepsilon \left(\frac{a}{p^2}, \frac{b}{p}, c \right) \sum\limits^p_{j=1} $  & $\varepsilon \left(a, \frac{b - 2 a j}{p} , \frac{c - bj + aj^2}{p^2}\right)$  & $\varepsilon\left(\frac{a}{p}, \frac{b}{p}, \frac{c}{p} \right) $ & $\varepsilon \left(\frac{a}{p^2}, \frac{b}{p^2}, \frac{c}{p^2} \right)$\\\hline
0 & $\geqslant 0$ & $\geqslant 0$ & 0 & 1 & 0 & 0 \\
1 & $\geqslant 1$ & $\geqslant 1$ & 0 & $1 + \left(\frac{D}{p} \right)$ & 1 & 0\\
$\geqslant 2$ & $\geqslant 1$ & 0 & 1 & 0 & 0 & 0\\
$\geqslant 2$ & 1 & $\geqslant 1$ & 1 & 1 & 1 & 0\\
$\geqslant 2$ & $\geqslant 2$ & 1 & 1 & 0& 1 & 0\\
$\geqslant 2$ & $\geqslant 2$ & $\geqslant 2$ & 1 & $ p$ & 1 & 1
\end{tabular}}
\end{center}

The proof for $p=2$ is similar but there are more cases to be considered.

iii) We will show that each of the properties in question is implied by the orthogonality of $F$ with $\sL_D \Phi$ for special choices of $D$ and $\Phi$. 

For \eqref{art10-eq21} we choose 
$$
\Phi (X) = \delta (X^2 + D), 
$$
where $\delta$ is the Dirac delta-function. From the identity
\begin{equation*}
\left(\frac{a|z|^2 + b x + c}{y} \right)^2 + D = \frac{|az^2 + bz + c|^2}{y^2} \tag{37}\label{art10-eq37}
\end{equation*}
we see that the support of $\sL_D \Phi$ is the set of points in $\sfH$ satisfying some quadratic equation of discriminant $D$, and an easy calculation shows that 
\begin{equation*}
\int\limits_{\Gamma / \sfH} F (z) \sL_D \Phi (z) dz = \frac{\pi}{2 \sqrt{|D|}} \sum\limits^{h(D)}_{ i=1} F (z_{Q_i})  \tag{38}\label{art10-eq38}
\end{equation*}
for any continuous $F: \Gamma / \sfH \to \bbC$. (Of course, $\delta (X^2 + D)$ is not a function, and equation \eqref{art10-eq38} must be interpreted in the sense that it holds in the limit $n \to \infty$ if we choose $\Phi (X) = \delta_n (X^2 + D)$ where $\{\delta_n\}$ is a sequence of smooth, even functions with integral 1 and support tending to $\{0\}$.) Hence any $F \in(\Iim \sL_D)^\perp$ satisfies \eqref{art10-eq21}.

The\pageoriginale case $D > 0$, $D$ not a square, is similar; here we choose $\Phi (X) = \delta (X)$. so that $\sL_D \Phi (z)$ is supported on the semicircles $a|z|^2 + b x + c =0$ $(a, b, c \in \bZ, b^2 - 4ac =D)$, and find 
\begin{equation*}
\int\limits_{\Gamma / \sfH} F(z) \sL_D \Phi (z) dz = \frac{1}{\sqrt{D}} \sum\limits^{h(D)}_{i=1} \int\limits_{C_{Q_i}} F (z) |d_{Q_i} z|, \tag{39}
\label{art10-eq39} 
\end{equation*}
where the equation is to be interpreted in the same way as \eqref{art10-eq38}. Thus $F \in (\Iim \sL_D)^\perp$ implies $\eqref{art10-eq23}$.

It remains to prove that any $F \in \sE$ satisfies equation \eqref{art10-eq24}. We follow the proof of the divisibility of $\sum\limits^\infty_{n=1} |a_n|^2 / n^{s+k-1}$ by $\zeta(s)$ given in \cite{art10-11}. Equations \eqref{art10-eq37} and \eqref{art10-47} of that paper give the identity
\begin{align*}
&\sum\limits^r_{i=1} \frac{a_i(m)}{(f_i, f_i)} y^k |f_i (z)|^2 \tag{40} \label{art10-40}\\
& \frac{(-1)^{k/2}}{\pi} 2^{k-4} m^{k-1} (k-1) \sum\limits^\infty_{t = - \infty} \sL_{t^2 - 4 m} \Phi_{k,t} (z) \quad (z \in \sfH)
\end{align*}
for all integers $m > 0$, where $r = \dim S_k (SL_2 (Z))$, $f_i(z)  =\sum\limits^\infty_{m=1} a_i (m) e^{2 \pi i m z} (i = 1, \ldots, r)$ are the normalized Hecke eigenforms of weight $k,\;(f_i, f_i) = \int\limits_{\Gamma / \sfH} y^k |f_i(z)|^2 dz$, and $\Phi_{k,t} (X) = (X - it)^{-k} + (X + it)^{-k}$. Thus any function in $\sE$ is orthogonal to the sum on the left-hand side of \eqref{art10-eq40} and therefore, since the Fourier coefficients $a_i(m)$ are linearly independent, to each of the functions $y^k |f_i(z)|^2$.

Using the computations of \cite{art10-12} and an extension of the Rankin-Selberg method \cite{art10-13} , it seems to be possible to prove the orthogonality of $F \in \sE$ with $|f(z)|^2$ also for Maass eigenforms (= non-holomorphic cusp forms which are eigenvalues of the Laplace and Hecke operators) of weight 0.
\end{proof}

\S 4. Let $G = PSL_2 (\bR)$ and $K = SO (2) / \{\pm 1\}$ its maximal compact subgroup, and identity the symmetric space $G/ K$ with $\sfH$ by $gK = \left(\begin{matrix}
a & b \\
 c & d
\end{matrix}\right)K \leftrightarrow g \cdot i = \dfrac{ai + b}{ci+b}$. In this section we will construct a representation $\sV$ of $G$ in the space of functions of $\Gamma / G$ whose space of $K$-fixed vectors $\sV^K$ is $\sE$.

Let\pageoriginale
$$
\sX_R = \{
\left.
\left(\begin{matrix}
a & b/2\\
b/2 & c
\end{matrix} 
\right) \right| a, b, c \in \bR
\}
$$
be the 3-dimensional vector space of symmetric real $2 \times 2$ matrices and $\sX_Z \subset \sX_R$ the lattice consisting of matrices with $a, b, c \in \bZ$. The group $G$ acts on $\sX_R$ by $g \circ M = g^t Mg (g \in G , M \in \sX_R)$, and $\sX_Z$ is stable under the action of the subgroup $\Gamma$. For $M \in \sX_R$ and $g \in G$, the expression $tr (g^t Mg)$ depends only on the right coset $g K$ (since $k^t = k^{-1}$ for $k \in K$), \ie only on $g \cdot i \in \sfH$. An easy calculation shows that 
\begin{equation*}
tr (g^t Mg) = \frac{a|z|^2 + b x + c}{y} (M = \left(\begin{matrix}
a & b/2\\
b/2 & c
\end{matrix}
\right) \in \sX_R, z = g \cdot i \in \sfH).\tag{41}
\label{art10-eq41}
\end{equation*}
This explains where the strange expression $\dfrac{a|z|^2 + b x + c}{y}$ in the definition of $\sL_D$ comes from and also why this expression is invariant under the simultaneous operation of $\Gamma$ on the upper half-place $(gK \to \gamma g K)$ and on binary quadratic forms $(M \to (\gamma^{-1})^t M \gamma^{-1})$..

Using \eqref{art10-eq41}, we can rewrite the definition of $\sL_D$ as 
$$
\sL_D \Phi (gK) = \sum\limits_{\substack{M \in \sX_Z\\\det M = - D /4}} \Phi (tr (g^t M g)).
$$
To pass from functions on $\sfH$ to functions on $G$, we replace the special function $M \to \Phi (tr (M))$ by an arbitrary function $\Phi$ on the 2-dimensional submanifold
$$
\sX_R (D) = \{
\left.
\left( 
\begin{matrix}
a & b/2\\
b/2& c
\end{matrix}
\right)
 \in \sX_R \right| b^2 - 4 a c = D \}
$$
of $\sX_R$. Thus we extend $\sL_D$ to an operator (still denoted $\sL_D$) from the space of nice functions on $\sX_R (D)$ to the space of functions on $\Gamma / G$ by setting 
\begin{equation*}
\sL_D \Phi (g) =\sum\limits_{M\in \sX_Z (D)} \Phi (g^t Mg) \quad (g \in G), \tag{42}\label{art10-eq42}
\end{equation*}
where $\sX_Z (D) = \sX_R (D) \cap \sX_Z$. Here ``nice'' means that $\Phi$ satisfies the obvious extensions of \eqref{art10-eq27} and \eqref{art10-29}, \ie it must be of sufficiently rapid decay in $\sX_R (D)$ and, if $D$ is a square, must be smooth and have zero integral along each of the lines $l_{g, \varepsilon} = g^t \left(
\begin{matrix}
0 & \frac{1}{2} \varepsilon \sqrt{D}\\
\frac{1}{2} \varepsilon \sqrt{D} & \bbR
\end{matrix}
\right) g (g \in G, \varepsilon = +1)$ on the\pageoriginale ruled surface $\sX_R(D)$.

It is clear that $\sL_D \Phi (g)$ is left $\Gamma$-invariant, since $\sX_Z(D)$ is stable under $\Gamma$ and the sum \eqref{art10-eq42} is absolutely convergent. Also, the image of $\sL_D$ is stable under the representation $\pi$ of $G$ given by right translation, since 
$$
\pi (g) \circ \sL_D = \sL_D \circ \pi'(g) \quad  (g \in G),
$$
where $\pi'(g) \Phi (M)= \Phi (g^t M g)$. Hence the space
\begin{equation*}
\sV = \bigcap\limits_{ D \in Z} (\Iim \sL_D)^\perp \tag{43} \label{art10-eq43}
\end{equation*}
of functions $F: \Gamma / G \to \bbC$ satisfying an appropriate growth condition and such that 
\begin{equation*}
\int\limits_{\Gamma / G} \sL_D \Phi (g) F (g) d g = 0 \quad (dg= \text{Haar measure}) \tag{44}\label{art10-eq44}
\end{equation*}
for all $D \in Z$ and all nice functions $\Phi$ on $\sX_R (D)$ , is also stable under $G$.\footnotetext[1]{Since the various manifolds $\sX_R(D) \subset \sX_R$ are disjoint, we can also define $\sV$ as $(\Iim \sL)^\perp$, where $\sL$ is the operator from nice functions on $\sX_R$ to functions on $\Gamma / G$ defined by
$$
\sL \Phi (g) = \sum\limits_{M \in \sX_Z} \Phi (g^1 M g).
$$
We may also identify $\sX_R$ with the Lie algebra $i_R = \{\left. \left(\begin{matrix} -\frac{1}{2} b & -c\\ a& \frac{1}{2}b \end{matrix}
\right) \right|  a, b, c \in R\}$ of $G$ by $M \to M \left(\begin{matrix}
0 & -1\\ 1 & 0\end{matrix}\right)$; then the operation $M \to g' Mg$  of $G$ on $\sX_R$ becomes the adjoint representation $X \to g^{-1} X g $ of $G$ on $i_R$, and $\sV$ is the set of functions on $\Gamma / G$ orthogonal to all functions of the form $g \to \sum\limits_{X \in i_Z} \Psi (Ad (g) X)$, where $\Psi$ is a nice function on $i_R$.}  Also, it is clear that $\sV^K$ coincides with the space $\sE$ defined in \S 3. In particular, $\sV$ contains the vectors $v_\rho: g \to E^\ast (g \cdot i, \rho)$ ($\rho$ a non-trivial zero of $\zeta(s)$) and more generally $v_{\rho, m} \cdot g \to F_{\rho, m} (g \cdot i)$ ($F_{\rho, m}$ as in \eqref{art10-eq19}).

On the other hand, because the function $z \to E (z,s)$ is an eigenfunction of the Laplace operator on $H$, the representation theory of $G$ tells us that (at least for $s \not\in Z$) the smallest $G$-invariant space of functions on $\Gamma / G$ containing the function $g \to E (g \cdot i, s)$ is an irreducible representation isomorphic to the principal series representation $\sP_s$. (Recall that $\sP_s$ is the representation of $G$ by right translations on the set of functions $f$: $G \to \bbC$ satisfying $f\left(\left(\begin{matrix}a & b \\ 0 & a^{-1} \end{matrix}\right) g\right) = |a|^{2s} f (g)$ and $f\big|_K \in L^2 (K)$). Thus \textit{$\sV$ contains the principal series representation $\sP_\rho$ for every non-trivial zero $\rho$ of the Riemann-zeta-function.}

On the\pageoriginale other hand, $\sP_s$ is unitarizable if and only if $s (1-s) >0$, \ie $s \in (0,1)$ or $\re (s) =\frac{1}{2}$. Thus the existence of a unitary structure on $\sV$ would imply the Riemann hypothesis.

However, the following argument suggests that it may be unlikely that such a unitary structure can be defined in a natural way. If $\rho$ is a zero of $\zeta(s)$ of order $n>1$, then the functions $F_{\rho, m} (m = 0, \ldots, n -1)$ belong to $\sV^K$, and these functions are not eigenfunctions of $\Delta$, through the space they generate is stable under $\Delta$ (for example, differentiating \eqref{art10-eq18} with  respect to $s$ we find $\Delta F_{\rho, 1} = \rho (\rho -1) F_{\rho, 1} + (2 \rho -1) F_{\rho, 0}$. Therefore $\sV$ contains a $G$-invariant subspace $\sV_{\rho, n}$ corresponding to the eigenvalues $\rho (1-\rho)$ which is reducible but is not a direct sum of irreducible representations (we have dim $\sV^K_{\rho, n} = n$ and $\sV_{\rho,n} \supset \sV_{\rho, n -1} \supset \ldots \supset \sV_{\rho, 1} \supset \sV_{\rho,0}  =\{0\}$ with $\sV_{\rho,m} / \sV_{\rho, m -1} \cong \sP_\rho$), and such a representation cannot have a unitary structure. Thus the unitarizability of $\sV$ would imply not only the Riemann hypothesis, but also the simplicity of the zeroes of $\zeta(s)$. Since an analogue of $\sV$ can be defined for any number field or function field (\cf. \S 5), and since there are examples of such fields whose zeta-functions are known to have multiple zeroes, there cannot be any generally applicable way of putting a unitary structure on $\sV$. Of course, this does not preclude the possibility that our particular $\sV$ (for the filed $\bQ$) has a unitary structure defined in some special way, and indeed, if the zeros of $\zeta (s)$ are simple and lie on the critical line and if (as seems likely) $\sE$ is spanned by the $V_\rho$, then $\sV$ is in fact unitarizable, indeed in infinitely many ways, since we are essentially free to choose the norm of $v_\rho$. For various reasons, a natural choice seems to be $||v_\rho|| = |\zeta^\ast(2\rho)|$.

Finally, we should mention that the construction of $\sV$ is closely related to the Weil representation. The functions $\sL_D \Phi (g)$ are essentially the Fourier coefficients for a ``lifting'' operator from functions on $\Gamma / G$ to autormorphic forms on the metaplectic group, in analogy with the construction of Shintani \cite{art10-9} in the holomorphic case; thus the space $\sV$ can be interpreted as the kernel of the lifting.

\S 5. The proof of the theorem in \S~3 shows that almost every statement of the theorem can be strengthened to a statement about the individual spaces 
$$
\sE_D = (\Iim \sL_D)^\perp \quad (D \in \bZ)
$$
rather\pageoriginale than just their intersection $\sE$. Thus in part iii) of the theorem, to prove that a function $F$ satisfies \eqref{art10-eq21} or \eqref{art10-eq23} we needed only $F \in\sE_D$ for the value of $D$ in question, and it was only for \eqref{art10-eq24} that $F \in \bigcap_D \sE_D$ was needed. Similarly in part ii), equation \eqref{art10-eq34} shows that each space $\sE_D$ is stable under the Laplace operator. The same is not true for the Hecke operators, since $T(p)$ maps $\Iim (\sL_D)$ to $\Iim (\sL_{Dp^2}) + \Iim (\sL_D) + \Iim (\sL_{D/p^2})$ but the intersection of the spaces $\sE_D$ for all $D$ with a common squarefree part \textit{is} stable under the Hecke algebra. Finally---and most interesting---from equation \eqref{art10-eq30} or \eqref{art10-eq32} we see that $\sE_D(D \neq 0)$ contains $E^\ast (z, \rho)$ whenever $\rho$ is a zero of $\zeta (s,D)$ (\resp $\left.\dfrac{\partial^i}{\partial s^i} E^\ast (z,s) \right|_{s=\rho}$ for $0\leqslant i \leqslant n -1$ if $\rho$ is a zero of multiplicity $n$). Since $\zeta(s,D) = \zeta(s) L(s,D)$ and $L(s,D)$ has infinitely many zeroes in the critical strip, this shows that $\sE_D$ contains many more Eisenstein series than just the functions \eqref{art10-eq19}. (This conclusion holds also when $D$ is a square; in this case $L(s,D)$ is equal to $\zeta(s)^2$ up to an elementary factor and we do not get any new zeroes, but they all occur with twice the multiplicity and so we get twice as many functions as in \eqref{art10-eq19}. For $D =0$, however, we get only the functions \eqref{art10-eq19}, since the expression on the right-hand side of \eqref{art10-eq31} or \eqref{art10-eq33} is a linear combination of $\zeta(s) \zeta(2s)$ and $\zeta(s) \zeta(2s -1)$ rather than a multiple of $\zeta(s, 0)$.)

The functions $\zeta(s,D)$ for two discriminants $D$ withthe same square-free part differ by a finite Euler product and have the same non-trivial zeroes $\rho$. This, together with eq. \eqref{art10-eq36}, suggests that the most natural thing to do is to put together the corresponding spaces $\sE_D$. Thus we let $E$ denote either a quadratic extension of $\bQ$, or $\bQ+ \bQ$, or $\bQ$, and define 
$$
\sE(E) = \bigcap^\infty_{f=1} \sE_{df^2},
$$
where $d$ denotes the discriminant of $E$, or 1, or 0, respectively. Then the above discussion can be summarized as follows:
\begin{itemize}
\item[i)] Each of the spaces $\sE (E)$ is stable under the Laplace and Hecke operators;

\item[ii)] $\bigcap_E \sE (E) = \sE$;

\item[iii)] $\sE (E)$\pageoriginale contains $\left. \dfrac{\partial^i}{\partial s^i} E^\ast (z,s) \right|_{s = \rho}$ for $0 \leqslant i \leqslant n -1$ if $\rho$ is an $n$-fold zero of $\zeta_E(s)$, where $\zeta_E (s)$ denotes the Dedekind zeta-function of $E$ if $E=Q$ or $E$  is q quadratic field and $\zeta_{\bQ + \bQ} (s) = \zeta(s)^2$.
\end{itemize}
Of course, we can also define representations $\sV_D = (\Iim \sL_D)^\perp$ and $\sV (E)= \cap \sV_{df^2}$ similarly; then $\sV (E)^K = \sE (E)$ and $\sV (E)$ is a representation of $PSL_2 (\sR)$ whose spectrum is related to the zeroes of $\zeta_E(s)$ in the same way as that of $\sV$ to those of $\zeta(s)$. 

The representations $\sV (E)$ have a very nice interpretation in the language of adeles; we end the paper by describing this. As motivation, we recall that our starting point for the definition of $\sE$ was the fact that the zeta-function of a quadratic field $E$ can be written as the integral of $E (z,s)$ over a certain set $\sS_E \subset \Gamma / \sfH$ consisting of a finite number of points if $E$ is imaginary and of a finite number of closed curves if $E$ is real (Proposition \ref{art10-prop1} and \ref{art10-prop2}). Hence the functions $E(z, \rho)$, $\rho$  a zero of $\zeta_E (s)$, belong to the space of functions whose integral over $\sS_E$ vanishes. 

Now let $G$ denote the algebraic group $GL(2)$, Z its center, and $\bA$ the ring of adeles of $\bQ$. Choosing a basis of $E$ over $\bQ$ gives an embedding of $E^\times$ in $GL (2, \bQ)$ and a non-split torus $T \subset G$ with $T (\bQ) = E^{\times}$. There is a projection $G (\bQ) Z(\bA)/ G (\bA) \to \Gamma / \sfH$ and under this projection $T (\bQ) Z (\bA) / T (\bA)$ maps to $\sS_E$. The adelic analogue of Proposition \ref{art10-prop1} and \ref{art10-prop2} is the fact that the integral of an Eisenstein series over $T (\bQ)Z (\bA) / T (\bA)$ is a multiple of the zeta-function of $E$. To prove it, we must recall the definition of the Eisenstein series. Let $\Phi$ be a Schwartz-Bruhat function on $\bA^2$; then the Eisenstein series $E (g, \Phi, s)$ is defined for $g \in G(\bA)$ and $s \in \bbC$ with sufficiently large real part by 
\begin{equation*}
E (g , \Phi, s) = \int\limits_{Z (Q) / Z (A)} \sum\limits_{\xi \in Q^2 \{0\}} \Phi [\xi z g] |\det z g|^s_Q dz, \tag{45} \label{art10-eq45}
\end{equation*}
where $|{\;\;}|_Q$ denote the idele norm and $dz$ the Haar measure on $Z$. (This definition is the analogue of equation \eqref{art10-eq2}. The more usual definition of $E(g, \Phi, s)$, analogous to eq. \eqref{art10-eq1}, is
$$
E (g, \Phi , s) = \sum\limits_{ \gamma \in P (Q) / G (Q)} f (\gamma g, \Phi , s),
$$
where $P = \left\{\left(\begin{matrix}\ast & \ast \\ 0 & \ast \end{matrix}\right)\right\}$ and 
$$
F (G, \Phi, s) = |\det g|^s_Q \int\limits_{A^\times } \Phi [(0, a)g]|a|^{2s} d^\times a,
$$
which is easily seen to agree with \eqref{art10-eq45}; note that $f(g, \Phi, s)$ equals $\zeta_Q (2s)$ times an elementary function of $s$ by Tate theory, and that $f\left(\left(\begin{matrix} 
a & x \\ 0 & b
\end{matrix}
\right) g, \Phi, s\right) =  \left|\dfrac{a}{b} \right|^s f(g, \Phi, s)$, so the analogue of $f (g, \Phi, s)$ in the upper half-plane is the function $\zeta(2s) \Im (g \cdot i)^s$.)  Identifying $\bQ^2 \{0\}$ with $E^\times$ and observing that the $\bQ$-idele norm of $\det t (t \in T (\bA))$ equals the $E$-idele norm of $t$ under the identification of $T (\bA)$ with $\bA_E^\times$, we find 
\begin{align*}
\int\limits_{T(Q) Z(A)/ T (A)}  E(tg , \Phi, s) dt & =  \int\limits_{T (Q)/ T (A)} \sum\limits_{\xi \in Q^2 \{0\}} \Phi [\xi t g ] |\det tg|^s_Q dt\\
& = |\det g|^s_Q \int\limits_{E^\times / A^\times_E} \sum\limits_{\xi \in E^\times} \Phi [\xi t g] |t|^s_E dt \\
& = |\det g|^s_Q \int\limits_{A^\times_E} \Phi [eg] |e|^s_E d^\times e. 
 \end{align*}
Since $e \to \Phi [eg]$ is a Schwartz-Bruhat function on $\bA_E$, this is precisely the Tate integral for $\zeta_E (s)$. (The computation just given is the basis for Harder's computations of period integrals in this volume as well as for the generalization of the Selberg trace formula in \cite{art10-4}.) In particular, it follows that the integral of 
$$
\left. F(g) = \frac{\partial^i}{\partial s^i} E (g, \Phi, s)  \right|_{s = \rho}
$$
over $T(\bQ) Z (\bA) / T (\bA) g$ vanishes if $\rho$ is a zero of $\zeta_E (s)$ of multiplicity $\geqslant i +1$. The natural adelic definition of $\sV (E)$ is thus as the space of functions $F: G (\bQ) Z (\bA) / G (\bQ) \to \bbC$ satisfying
\begin{equation*}
\int\limits_{T(Q) Z (A) / T (A)} F (tg) dt = 0 \qquad (\forall g \in G (A)) \tag{46} \label{art10-eq46}
\end{equation*}
as well as some appropriate growth condition. The space $\sV (E)$ then contains irreducible principal series representations corresponding to the zeroes of $\zeta_E (s)$. Condition \eqref{art10-eq46} is similar to the condition
$$
\int\limits_{N(Q)/ N (A)} F(ng) dn =0 \qquad (\forall g \in G (A))
$$
defining cusp forms (where $N$ is the unipotent radical of a parabolic subgroup of $G$), so the space $\sV (E)$ can be thought of as an analogue of the space $L^2_0(G (\bQ) Z (\bA) / G (\bA))$ of cusp forms. Like $L^2_0$, it probably has a discrete spectrum. The difference is that, whereas $L^2_0$ has a unitary structure, the corresponding statement for $\sV (E)$ would imply the Riemann hypothesis and the simplicity of the zeroes of $\zeta_Q (s)$. We call functions $F$ satisfying \eqref{art10-eq46} \textit{toroidal forms} (in analogy with the French terminology of ``formes paraboliques'' for cusp forms) and the $\sV (E)$ \textit{toroidal representations}.

The calculation given above is unchanged if we replace $\bQ$ by any global field $F$ and take $E$ to be a quadratic extension of $F$. In the case where $F$  is the functional field of a curve $X$ over a finite field, there are only finitely many zeroes of $\zeta_F (s)$, their number being equal to the first Betti number of $X$. Then the $K$-finite part of our representation $\sV= \bigcap_E \sV (E)$ is a complex vector space of dimension at least (and hopefully exactly) equal to this Betti number, and Barry Mazur pointed out that this space might have a natural interpretation as a complex cohomology group $H^1 (X; \bbC)$. It is not yet clear whether this point of view is tenable. At any rate, however, from conversations with Harder and Deligne it appears that it will at least be possible to show that the dimension of the space in question is finite. 

\begin{thebibliography}{99}
\bibitem[1]{art10-1} \textsc{Gelbart, S.} and \textsc{H. Jacquet,:} A relation between automorphic representations of $GL(2)$ and $GL(3)$. \textit{Ann. Sc. Ec. Norm. Sup. 11 (1978) 471-542}.

\bibitem[2]{art10-2} \textsc{Hecke, E.:} \"Uber die Kroheckersche Grenzformel f\"ur reelle quadratische K\"orper und die Klassenzahl relativ-abelscher K\"orper, \textit{Verhandl. d. Naturforschenden Gesell. i. Basel} 28, 363-372 (1971). Mathematische Werke, pp. 198-207. Vandenhoeck \& Ruprecht, G\"ottingen 1970.

\bibitem[3]{art10-3} \textsc{Jacquet, H.} and \textsc{J. Shalika},: A non-vanishing theorem for zeta functions of $GL_2$. \textit{Invent. Math.} 38 (1976) 1-16.

\bibitem[4]{art10-4} \textsc{Jacquet, H.} and \textsc{D. Zagier,:} Eisenstein series and the Selberg trace formula II. In preparation. 

\bibitem[5]{art10-5} \textsc{Rankin, R.:} Contributions to the theory of Ramanujan's function $\tau(n)$ and similar arithmetical functions. I. \textit{Proc. Cam. Phil. Soc.} 35 (1939) 351-372.

\bibitem[6]{art10-6} \textsc{Selberg, A.:} Bemerkungen \"uber eine Dirichletsche Reihe, die mit der Theorie der Modulformen nahe verbunden ist. \textit{Arch. Math. Naturvid.} 43 (1940) 47-50.

\bibitem[7]{art10-7} \textsc{Shimura, G.:} On the holomorphy of certain Dirichlet series, \textit{Proc. Lond. Math. Soc. 31 (1975), 79-98}.

\bibitem[8]{art10-8} \textsc{Shintani, T.:} ON zeta-functions associated with the vector space of quadratic forms \textit{J. Fac. Science Univ.} Tokyo 22 (1975), 25-65.

\bibitem[9]{art10-9} \textsc{Shintani, T.:} On construction of holomorphic cusp forms of half integral weight. \textit{Nagoya Math.} J. 58 (1975) , 83-126.

\bibitem[10]{art10-10} \textsc{Zagier, D.:} A Kronecker limit formula for real quadratic fields. \textit{Math. Ann.} 213 (1975), 153-184.

\bibitem[11]{art10-11} \textsc{Zagier, D.:} Modular forms whose Fourier coefficients involve zeta- functions  of quadratic fields. In \textit{Modular Functions of One Variable VI,} Lecture Notes in Mathematics No. 627, Springer, Berlin-Heidelberg-New York 1977, pp. 107-169.

\bibitem[12]{art10-12} \textsc{Zagier, D.:} Eisenstein series and the Selberg trace formula I. This volume, pp. 303-355.

\bibitem[13]{art10-13} \textsc{Zagier, D.:} The Rankin-Selberg method for automorphic functions which are not of rapid decay. In preparation.
\end{thebibliography}
