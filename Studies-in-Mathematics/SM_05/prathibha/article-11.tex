\chapter{EISENSTEIN SERIES AND THE SELBERG TRACE FORMULA I}

\begin{center}
{\large By~ Don Zagier$^\ast$}
\end{center}

\bigskip

\setcounter{pageoriginal}{302}

\setcounter{section}{-1}
\section{Introduction.}\label{art11-sec0}
The integral\pageoriginale $\int K_\circ (g,g) E (g,s) dg$. Let $G = S L_2 (R)$ and $\Gamma$ be an arithmetic subgroup of $G$ for which $\Gamma / G$ has finite volume but is not compact. The space $L^2 (\Gamma / G)$ has the spectral decomposition (with respect to the Casimir operator)
$$
L^2 (\Gamma / G) = L^2_\circ (\Gamma / G) \bigoplus L^2_{sp} (\Gamma / G) \bigoplus L^2_{\text{cont}} (\Gamma / G),
$$
where $L^2_\circ (\Gamma / G)$ is the space of cusp forms and is discrete, $L^2_{sp} (\Gamma / G)$ is the discrete part of $(L^2_\circ)^\perp$, given by residues of Eisenstein series, and $L^2_{\text{cont}}$ is the continuous part of the spectrum, given by integrals of Eisenstein series. If $\varphi$ is a function of compact support or of sufficiently rapid decay on $G$, then convolution with $\varphi$ defines an endomorphism $T_\varphi$ of $L^2 (\Gamma / G)$, and the kernel function 
\begin{equation*}
K(g, g') = \sum\limits_{\gamma \in \Gamma} \varphi (g^{-1} \gamma g') \qquad (g, g' \in G) \tag{0.1} \label{art11-eq0.1}
\end{equation*}
of $T_\varphi$ has a corresponding decomposition as $K_\circ + K_{\text{sp}} + K_{\text{cont}}$, where $K_{\text{sp}}$ and $K_{\text{cont}}$ can be described explicitly using the theory of Eisenstein series. The restriction of $T_\varphi$ to $L^2_\circ (\Gamma / G)$ is of trace class; its trace is given by 
\begin{equation*}
Tr (T_\varphi, L^2_\circ )  = \int\limits_{\Gamma / G} K_\circ (g, g) dg. \tag{0.2} \label{art11-eq0.2}
\end{equation*}
The Selberg trace formula is the formula obtained by substituting $K(g,g) - K_{sp} (g,g) - K_{\text{cont}} (g, g)$ for $K_\circ (g,g)$ and computing the integral. However, although $K_\circ (g,g)$ is of rapid decay in $\Gamma / G$, the individual terms $K(g,g)$, $K_{sp} (g,g)$ and $K_{\text{cont}} (g,g)$ are not, so that to carry out the integration one has to either delete small neighbourhoods of the cusps form a fundamental domain or else ``truncate'' the kernel functions by subtracting off their constant terms in such neighbourhoods, and then to compute the limit as these neighbourhoods shrink to points. This procedure is perhaps somewhat unsatisfactory, both from an aesthetic point of view and because of the analytical difficulties it involves.

To get\pageoriginale around these difficulties we introduce the integral
\begin{equation*}
I(s) = \int\limits_{\Gamma / G} K_\circ (g,g) E (g,s) dg, \tag{0.3}\label{art11-eq0.3}
\end{equation*}
where $E(g,s) \; (g \in G, s \in C )$ denotes an Eisenstein series. The idea of integrating a $\Gamma$-invariant function $F(g)$ against an Eisenstein series was introduced by Rankin \cite{art11-5} and Selberg \cite{art11-6}, who observed that in the region of absolute convergence of the Eisenstein series this  integral equals the Mellin transform of the constant term in the Fourier expansion of $F$ (see \S 2 for a more precise formulation). Applying this principle to $F(g)= K_\circ (g,g)$ we can calculate $I(s)$ for $\re (s) >1$ as a Mellin transform, obtaining a representation of $I(s)$ as an infinite series of terms. Each of these terms can be continued meromorphically to $\re (s) \leqslant 1$; in particular, the contribution of a hyperbolic or elliptic conjugacy class of $\gamma$'s in \eqref{art11-eq0.1} is the product of a certain integral transform of $\varphi$ with the Dedekind zeta-functio
 of the corresponding real or imaginary quadratic field. Since the residue of $E(g,s)$ at $s =1$ (\resp the value of $E(g,s)$ at $s=0$) is a constant function, we recover the Selberg trace formula by computing $\res_{s=1} (I(s))$ (\resp. $I(0)$). This proof of the trace formula is more invariant and in some respects computationally simpler than the proofs involving truncation. It also gives more insight into the origin of the various terms in the trace formula; for instance, the class numbers occurring there now appear as residues of zeta-functions. 

However, the formula for $I(s)$ has other consequences than the trace formula. The most striking is that $I(s)$ (and in fact each of the infinitely many terms in the final formula for $I(s)$) is divisible by the Riemann zeta-function, \ie the quotient $I(s)/\zeta(s)$ is an entire function of $s$. Interpreting this as the statement that the Eisenstein series $E(g, \rho)$ is orthogonal to $K_\circ (g, g)$ (in fact, to each of infinitely many functions whose sum equals $K_\circ (g, g)$) whenever $\zeta(\rho) =0$, one is led to the construction of a representation of $G$ whose spectrum is related to the set of zeros of the Riemann zeta-function (\cf \cite{art11-11} in this volume).

On the other hand, the formula for $I(s)$ can be used to get information about cusp forms. The function $K_\circ (g,g')$ is a linear combination of terms $f_j (g) f_j (g')$, where $\{f_j\}$ is an orthogonal basis for $L^2_\circ (\Gamma / G)$ and where the coefficients depend on the function $\varphi$ and on the eigenvalues of $f_j$ (``Selberg transform''). Moreover, applying the Rankin-Selberg method to the function $F(g) = |f_j (g)|^2$ one finds that the integral of this function against $E (g,s)$ equals the ``Rankin zeta-function'' $R_{f_j}(s)$ (roughly speaking, the Dirichlet series $\sum\limits^\infty_{n=1} |a_n|^2n^{-s}$, where the $a_n$ are the Fourier coefficients off); indeed, this is the situation for which the Rankin-Selberg method was introduced. Thus $I(s)$ is a linear combination of the functions $R_{f_j}(s)$, and so one can get information about the latter from a knowledge of $I(s)$. In particular, using a ``multiplicity one'' argument one can deduce from the divisibility of $I(s)$ by $\zeta(s)$ that in fact each $R_{f_j(s)}$ is so divisible (this result had been proved by another method by Shimura \cite{art11-8} for holomorphic cusp forms and by Gelbart and Jacquet \cite{art11-2} in the general case). Other applications of the results proved here might arise by comparing them with the work of Goldfeld \cite{art11-1}. It does not seem impossible that the formula of $I(s)$ can be used to obtain information about the Fourier coefficients of cusp forms.

The idea we have described can be applied in several different situations:
\begin{itemize}
\item[1.] By working with an appropriate kernel function, we can isolate the contribution coming from holomorphic cusp forms of a given weight $k$ (discrete series representations in $L^2 (\Gamma / G)$). This case was treated in \cite{art11-10}. The computation of $I(s)$ here is considerably easier than in the general case because there is no continuous spectrum and only finitely many cusp forms $f_j$ are involved. We can therefore represent each Rankin zeta-function $R_{f_j}(s)$ as an infinite linear combination of zeta-functions of real and imaginary quadratic fields. Moreover, for certain odd positive values of $s$ the contributions of the hyperbolic conjugacy classes in $\Gamma$ to $I(s)$ vanish and one is left with an identity expressing $R_{f_j}(s)$ as a \textit{finite} linear combination of special values of zeta-functions of imaginary quadratic extensions of $\bQ$. As a corollary of this identity one obtains the algebraicity (and behaviour under $Gal (\bar{\bQ}/\bQ)$) of $\dfrac{1}{(f_j, f_j)}R_{f_j} (s)/ \pi^{k-1} \zeta(s)$ for the values of $s$ in question (\cite{art11-10}, Corollary to Theorem 2, p. 115), a result proved independently by Sturm \cite{art11-9} by a different method.

\item[2.] The first\pageoriginale case involving the continuous spectrum is that of Maass wave forms of weight zero, \ie cusp forms in $L^2 (\Gamma / G / K) = L^2 (\Gamma / \sfH)$, where $K$ denotes $SO(2)$ and $\sfH = G/ K$ the upper half-plane. This is the case treated in the present paper (with $\Gamma = SL_2 (\bZ)$).
 
\item[3.] Next, one can replace $SL_2 (\bR)$ and $SL_2 (\bZ)$ by $GL_2 (2, \bA)$ and $GL (2, F)$, respectively, where $F$ is a global field and $\bA$ the ring of adeles of $F$. This case, which is the most general one as far as $GL(2)$ is concerned, will be treated in a joint paper with Jacquet \cite{art11-3}. It includes as special cases 1 and 2, as well as their generalizations to holomorphic and non-holomorphic modular forms of arbitrary weight and level, Hilbert modular  forms, and automorphic forms over function fields.
 
\item[4.] Finally, the definition of $I(s)$ makes sense in any context where Eisenstein series can be defined, so it may be possible to apply the method sketched in this introduction to discrete subgroups of algebraic groups other than $GL(2)$.
\end{itemize}

\section{Statement of the main theorem.}\label{art11-sec1}
In this section we describe the main result of this paper, namely a formula for $I(s)$ in the critical strip $0< \re (s)<1$. In order to reduce the amount of notation and preliminaries needed, we will state the formula in terms of a certain holomorphic function $h(r)$; the relationship of $h(r)$ to the function $\varphi (g)$ of the introduction (Selberg transform) is well-known and will be reviewed in \S~2. Except at the end of \S~5, we will always consider only forms of weight 0 on the full modular group $\Gamma = SL_2(\bZ)/ \{\pm1\}$. The results for congruence subgroups are similar but messier to state and in any case will be subsumed by the results of \cite{art11-3}.

Any continuous $\Gamma$-invariant function $f: \sfH \to \bC$ has a Fourier expansion of the form 
\begin{equation*}
f(z) = \sum\limits^\infty 2 (\_{n = - \infty} A_n (f;y) e^{2\pi in x} \quad (z\in \sfH) \tag{1.1}\label{art11-eq1.1}
\end{equation*}
(here and in future we use $x$ and $y$ to denote the real and imaginary parts of $z \in \sfH$). We denote by $L^2 (\Gamma/ \sfH)$ the Hilbert space of $\Gamma$-invariant functions $f: \sfH \to \bC$ such that $(f,f) = \int\limits_{\Gamma / \sfH} |f(z)|^2dz$ is finite $\left(dz = \frac{dx dy}{y^2}\right)$ and by\pageoriginale $L^2_\circ (\Gamma / \sfH)$ the subspace of functions with $A_\circ (f;y ) \equiv 0$. The space $L^2_\circ (\Gamma / \sfH)$ is stable under the Laplace operator
$$
\Delta = y^2 \left(\frac{\partial^2}{\partial x^2} + \frac{\partial^2}{\partial y^2} \right)
$$
and has a basis $\{f_j\}_{j \geqslant 1}$ consisting of eigenforms of $\Delta$(see \cite{art11-4}, \S~5.2). 

We write 
\begin{equation*}
\Delta f_j = -\left(\frac{1}{4}  +r^2_j \right)f_j \quad  (j = 1, 2, \ldots ) \tag{1.2} \label{art11-eq1.2}
\end{equation*}
where $r_j \in \bC$. Since $\Delta$ is negative definite, we have $r^2_j + \dfrac{1}{4} \geqslant 0$, \ie $r_j$ is either real or else pure imaginary of absolute value $\leqslant \frac{1}{2}$. In fact it is known that the $r_j$ are real for $SL_2 (\bZ)$, but the corresponding statement for congruence subgroups is not known and we will use only $r^2_j \geqslant - \dfrac{1}{4}$. From \eqref{art11-eq1.2} we find that the $n^{\text{th}}$ Fourier coefficient $A_n (f_j , Y)$ satisfies the second order differential equation
$$
y^2 \frac{d^2}{dy^2} A_n (f_j, y) - 4 \pi^2 n^2 y^2 A_n (f; y) =- \left(\frac{1}{4} + r^2_j \right) A_n(f_j ; y). 
$$
The only solution of this equation which is bounded as $y \to \infty$ is $\sqrt{y} K_{ir_j} (2\pi |n|y)$, where $K_v (z)$ is the $K$-Bessel function, defined (for example) by 
\begin{equation*}
K_v (z) = \int\limits^\infty_0 e^{-z \cosh t} \cosh v t d t \quad (v, z \in \bC, \re (z) > 0).\tag{11}\label{art11-eq1.3}
\end{equation*}
Hence the $f_j$ have Fourier expansions of the form 
\begin{equation*}
f_j(z) = \sum\limits^\infty_{\substack{n = - \infty \\ n \neq 0}} a_j (n) \sqrt{y} K_{ir_j} (2 \pi |n| y) e^{2 \pi in x}
\tag{1.4}\label{art11-eq1.4}
\end{equation*}
with $a_j(n) \in\bC$. We can choose the $f_j$ to be normalized eigenfunctions of the Hecke operators
\begin{equation*}
\left\{
\begin{aligned}
&T(n) : f(z) \longrightarrow \frac{1}{n} \sum\limits_{\substack{a, d > 0\\ ad =n}} \sum\limits_{b(\mod d)} f \left(\frac{az+b}{d} \right) \quad (n >0), \\
& T (-1) : f(z) \longrightarrow f(-\bar{z}), \quad T (-n) = T (-1) T (n),
\end{aligned}
\right. \tag{1.5}\label{art11-eq1.5}
\end{equation*}
\ie\pageoriginale
\begin{equation*}
f_j |T (n) = \frac{a_j(n)}{|n|^{\frac{1}{2}}} f_j \qquad (n \in \bZ, \; n \neq 0)  \tag{1.6}\label{art11-eq1.6}
\end{equation*}
(then $a_j (1) = 1$, $a_j (-1) = \pm 1$, and $a_j(n)$ is multiplicative). The functions $f_j$ chosen in this way are called the \textit{Maass eigenforms;} they form an orthogonal (but not orthonormal) basis of $L^2_\circ (\Gamma / \sfH)$, uniquely determined up to order. For each $j$ we define the \textit{Rankin zeta-function} $R_{f_j}(s)$ by
\begin{equation*}
R_{f_j} (s) = \frac{\Gamma (\frac{s}{2})^2}{8 \pi^s \Gamma (s)} \Gamma \left(\frac{s}{2} + ir_j \right) \Gamma (\frac{s}{2} - ir_j ) \sum\limits^\infty_{\substack{n = -\infty\\ n \neq 0}} \frac{|a_j (n)|^2}{|n|^s} \quad (\re(s) > 1).
\tag{1.7}\label{art11-eq1.7}
\end{equation*}
We also set 
\begin{equation*}
R^{\ast}_{f_j}(s) = \pi^{-s} \Gamma (s) \zeta (2s) R_{f_j} (s) = \zeta^\ast (2s) R_{f_j} (s), \tag{1.8}\label{art11-eq1.8}
\end{equation*}
where $\zeta(s)$ denotes the Riemann zeta-function and 
\begin{equation*}
\zeta^\ast (s) = \pi^{-s/2} \Gamma \left(\frac{s}{2} \right) \zeta(s) = \zeta^\ast (1-s). \tag{1.9}\label{art11-eq1.9}
\end{equation*}
The Rankin-Selberg method implies that $R^\ast_{f_j}(s)$ has a meromorphic continuation to all $s$, is regular except for simple poles at $s =1$ and $s =0$ with 
\begin{equation*}
\res_{s=1}  \bR^\ast_{f_j} (s) = \frac{1}{2} (f_j, f_j), \tag{1.10}\label{art11-eq1.10}
\end{equation*}
and satisfies the functional euqation
\begin{equation*}
\bR^\ast_{f_j} (s) = \bR^\ast_{f_j} (1-s) \tag{1.11}\label{art11-eq1.11}
\end{equation*}
(the proofs will be recalled in \S~2).

We will also need the zeta-functions $\zeta(s,D)$, where $D$ is an integer congruent to 0 or 1 modulo 4. They are defined for $\re (s) >1$ by
\begin{equation*}
\zeta(s, D) = \sum\limits_{Q} \sum\limits_{m,n} \frac{1}{Q(m,n)^s} \qquad (\re (s) >1), \label{art11-eq1.12}
\end{equation*}
where\pageoriginale the first summation runs over all $SL_2 (\bZ)$-equivalence classes of binary quadratic forms $Q$ of discriminant $D$ and the second over all pairs of integers $(m,n) \in \bZ^2 / \Aut (Q)$ with $Q(m,n) >0$, where $\Aut (Q)$ is the stabilizer of $Q$ in $SL_2 (\bZ)$. These functions, which were introduced in \cite{art11-10}, are related to standard zeta-functions by 
\begin{equation*}
\zeta (s,D) = 
\left\{ 
\begin{aligned}
&\zeta(s) \zeta(2s -1) & & \text{if  }D = 0\\
& \zeta(s)^2  \cdot \text{(finite Dirichlet series )}&&\text{if } D = \text{square} \neq 0\\
& \zeta_{Q (\sqrt{D})} (s) \cdot \text{(finite Dirichlet series)}& & \text{if } D \neq \text{square,}
\end{aligned}
\right.
\tag{1.13}\label{art11-eq1.13}
\end{equation*}
where $\zeta_{\bQ(\sqrt{D})} (s)$  denotes the Dedekind zeta-function of $\bQ (\sqrt{D})$ (for precise formulas see \cite{art11-10}, Proposition 3, p. 130). In particular, $\zeta(s,D)$ has a meromorphic continuation in $s$ and $\zeta(s,D)/ \zeta(s)$ is holomorphic except for a simple pole at $s =1$ when $D$ is a square. 

Now let $h : \bR \longrightarrow \bC$ be a function satisfying 




