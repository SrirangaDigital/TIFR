\chapter{EISENSTEIN SERIES AND THE SELBERG TRACE FORMULA I}

\begin{center}
{\large By~ Don Zagier$^\ast$}
\end{center}

\bigskip

\setcounter{pageoriginal}{302}

\setcounter{section}{-1}
\section{Introduction.}\label{art11-sec0}
The integral\pageoriginale $\int K_\circ (g,g) E (g,s) dg$. Let $G = S L_2 (R)$ and $\Gamma$ be an arithmetic subgroup of $G$ for which $\Gamma / G$ has finite volume but is not compact. The space $L^2 (\Gamma / G)$ has the spectral decomposition (with respect to the Casimir operator)
$$
L^2 (\Gamma / G) = L^2_\circ (\Gamma / G) \bigoplus L^2_{sp} (\Gamma / G) \bigoplus L^2_{\text{cont}} (\Gamma / G),
$$
where $L^2_\circ (\Gamma / G)$ is the space of cusp forms and is discrete, $L^2_{sp} (\Gamma / G)$ is the discrete part of $(L^2_\circ)^\perp$, given by residues of Eisenstein series, and $L^2_{\text{cont}}$ is the continuous part of the spectrum, given by integrals of Eisenstein series. If $\varphi$ is a function of compact support or of sufficiently rapid decay on $G$, then convolution with $\varphi$ defines an endomorphism $T_\varphi$ of $L^2 (\Gamma / G)$, and the kernel function 
\begin{equation*}
K(g, g') = \sum\limits_{\gamma \in \Gamma} \varphi (g^{-1} \gamma g') \qquad (g, g' \in G) \tag{0.1} \label{art11-eq0.1}
\end{equation*}
of $T_\varphi$ has a corresponding decomposition as $K_\circ + K_{\text{sp}} + K_{\text{cont}}$, where $K_{\text{sp}}$ and $K_{\text{cont}}$ can be described explicitly using the theory of Eisenstein series. The restriction of $T_\varphi$ to $L^2_\circ (\Gamma / G)$ is of trace class; its trace is given by 
\begin{equation*}
Tr (T_\varphi, L^2_\circ )  = \int\limits_{\Gamma / G} K_\circ (g, g) dg. \tag{0.2} \label{art11-eq0.2}
\end{equation*}
The Selberg trace formula is the formula obtained by substituting $K(g,g) - K_{sp} (g,g) - K_{\text{cont}} (g, g)$ for $K_\circ (g,g)$ and computing the integral. However, although $K_\circ (g,g)$ is of rapid decay in $\Gamma / G$, the individual terms $K(g,g)$, $K_{sp} (g,g)$ and $K_{\text{cont}} (g,g)$ are not, so that to carry out the integration one has to either delete small neighbourhoods of the cusps form a fundamental domain or else ``truncate'' the kernel functions by subtracting off their constant terms in such neighbourhoods, and then to compute the limit as these neighbourhoods shrink to points. This procedure is perhaps somewhat unsatisfactory, both from an aesthetic point of view and because of the analytical difficulties it involves.



%%%%% 304 page 
