\chapter{EISENSTEIN SERIES AND THE SELBERG TRACE FORMULA I}

\begin{center}
{\large By~ Don Zagier$^\ast$}
\end{center}

\bigskip

\setcounter{pageoriginal}{302}

\setcounter{section}{-1}
\section{Introduction.}\label{art11-sec0}
The integral\pageoriginale $\int K_\circ (g,g) E (g,s) dg$. Let $G = S L_2 (R)$ and $\Gamma$ be an arithmetic subgroup of $G$ for which $\Gamma / G$ has finite volume but is not compact. The space $L^2 (\Gamma / G)$ has the spectral decomposition (with respect to the Casimir operator)
$$
L^2 (\Gamma / G) = L^2_\circ (\Gamma / G) \bigoplus L^2_{sp} (\Gamma / G) \bigoplus L^2_{\text{cont}} (\Gamma / G),
$$
where $L^2_\circ (\Gamma / G)$ is the space of cusp forms and is discrete, $L^2_{sp} (\Gamma / G)$ is the discrete part of $(L^2_\circ)^\perp$, given by residues of Eisenstein series, and $L^2_{\text{cont}}$ is the continuous part of the spectrum, given by integrals of Eisenstein series. If $\varphi$ is a function of compact support or of sufficiently rapid decay on $G$, then convolution with $\varphi$ defines an endomorphism $T_\varphi$ of $L^2 (\Gamma / G)$, and the kernel function 
\begin{equation*}
K(g, g') = \sum\limits_{\gamma \in \Gamma} \varphi (g^{-1} \gamma g') \qquad (g, g' \in G) \tag{0.1} \label{art11-eq0.1}
\end{equation*}
of $T_\varphi$ has a corresponding decomposition as $K_\circ + K_{\text{sp}} + K_{\text{cont}}$, where $K_{\text{sp}}$ and $K_{\text{cont}}$ can be described explicitly using the theory of Eisenstein series. The restriction of $T_\varphi$ to $L^2_\circ (\Gamma / G)$ is of trace class; its trace is given by 
\begin{equation*}
Tr (T_\varphi, L^2_\circ )  = \int\limits_{\Gamma / G} K_\circ (g, g) dg. \tag{0.2} \label{art11-eq0.2}
\end{equation*}
The Selberg trace formula is the formula obtained by substituting $K(g,g) - K_{sp} (g,g) - K_{\text{cont}} (g, g)$ for $K_\circ (g,g)$ and computing the integral. However, although $K_\circ (g,g)$ is of rapid decay in $\Gamma / G$, the individual terms $K(g,g)$, $K_{sp} (g,g)$ and $K_{\text{cont}} (g,g)$ are not, so that to carry out the integration one has to either delete small neighbourhoods of the cusps form a fundamental domain or else ``truncate'' the kernel functions by subtracting off their constant terms in such neighbourhoods, and then to compute the limit as these neighbourhoods shrink to points. This procedure is perhaps somewhat unsatisfactory, both from an aesthetic point of view and because of the analytical difficulties it involves.

To get\pageoriginale around these difficulties we introduce the integral
\begin{equation*}
I(s) = \int\limits_{\Gamma / G} K_\circ (g,g) E (g,s) dg, \tag{0.3}\label{art11-eq0.3}
\end{equation*}
where $E(g,s) \; (g \in G, s \in C )$ denotes an Eisenstein series. The idea of integrating a $\Gamma$-invariant function $F(g)$ against an Eisenstein series was introduced by Rankin \cite{art11-5} and Selberg \cite{art11-6}, who observed that in the region of absolute convergence of the Eisenstein series this  integral equals the Mellin transform of the constant term in the Fourier expansion of $F$ (see \S 2 for a more precise formulation). Applying this principle to $F(g)= K_\circ (g,g)$ we can calculate $I(s)$ for $\re (s) >1$ as a Mellin transform, obtaining a representation of $I(s)$ as an infinite series of terms. Each of these terms can be continued meromorphically to $\re (s) \leqslant 1$; in particular, the contribution of a hyperbolic or elliptic conjugacy class of $\gamma$'s in \eqref{art11-eq0.1} is the product of a certain integral transform of $\varphi$ with the Dedekind zeta-functio
 of the corresponding real or imaginary quadratic field. Since the residue of $E(g,s)$ at $s =1$ (\resp the value of $E(g,s)$ at $s=0$) is a constant function, we recover the Selberg trace formula by computing $\res_{s=1} (I(s))$ (\resp. $I(0)$). This proof of the trace formula is more invariant and in some respects computationally simpler than the proofs involving truncation. It also gives more insight into the origin of the various terms in the trace formula; for instance, the class numbers occurring there now appear as residues of zeta-functions. 

However, the formula for $I(s)$ has other consequences than the trace formula. The most striking is that $I(s)$ (and in fact each of the infinitely many terms in the final formula for $I(s)$) is divisible by the Riemann zeta-function, \ie the quotient $I(s)/\zeta(s)$ is an entire function of $s$. Interpreting this as the statement that the Eisenstein series $E(g, \rho)$ is orthogonal to $K_\circ (g, g)$ (in fact, to each of infinitely many functions whose sum equals $K_\circ (g, g)$) whenever $\zeta(\rho) =0$, one is led to the construction of a representation of $G$ whose spectrum is related to the set of zeros of the Riemann zeta-function (\cf \cite{art11-11} in this volume).

On the other hand, the formula for $I(s)$ can be used to get information about cusp forms. The function $K_\circ (g,g')$ is a linear combination of terms $f_j (g) f_j (g')$, where $\{f_j\}$ is an orthogonal basis for $L^2_\circ (\Gamma / G)$ and where the coefficients depend on the function $\varphi$ and on the eigenvalues of $f_j$ (``Selberg transform''). Moreover, applying the Rankin-Selberg method to the function $F(g) = |f_j (g)|^2$ one finds that the integral of this function against $E (g,s)$ equals the ``Rankin zeta-function'' $R_{f_j}(s)$ (roughly speaking, the Dirichlet series $\sum\limits^\infty_{n=1} |a_n|^2n^{-s}$, where the $a_n$ are the Fourier coefficients off); indeed, this is the situation for which the Rankin-Selberg method was introduced. Thus $I(s)$ is a linear combination of the functions $R_{f_j}(s)$, and so one can get information about the latter from a knowledge of $I(s)$. In particular, using a ``multiplicity one'' argument one can deduce from the divisibility of $I(s)$ by $\zeta(s)$ that in fact each $R_{f_j(s)}$ is so divisible (this result had been proved by another method by Shimura \cite{art11-8} for holomorphic cusp forms and by Gelbart and Jacquet \cite{art11-2} in the general case). Other applications of the results proved here might arise by comparing them with the work of Goldfeld \cite{art11-1}. It does not seem impossible that the formula of $I(s)$ can be used to obtain information about the Fourier coefficients of cusp forms.

The idea we have described can be applied in several different situations:
\begin{itemize}
\item[1.] By working with an appropriate kernel function, we can isolate the contribution coming from holomorphic cusp forms of a given weight $k$ (discrete series representations in $L^2 (\Gamma / G)$). This case was treated in \cite{art11-10}. The computation of $I(s)$ here is considerably easier than in the general case because there is no continuous spectrum and only finitely many cusp forms $f_j$ are involved. We can therefore represent each Rankin zeta-function $R_{f_j}(s)$ as an infinite linear combination of zeta-functions of real and imaginary quadratic fields. Moreover, for certain odd positive values of $s$ the contributions of the hyperbolic conjugacy classes in $\Gamma$ to $I(s)$ vanish and one is left with an identity expressing $R_{f_j}(s)$ as a \textit{finite} linear combination of special values of zeta-functions of imaginary quadratic extensions of $\bQ$. As a corollary of this identity one obtains the algebraicity (and behaviour under $Gal (\bar{\bQ}/\bQ)$) of $\dfrac{1}{(f_j, f_j)}R_{f_j} (s)/ \pi^{k-1} \zeta(s)$ for the values of $s$ in question (\cite{art11-10}, Corollary to Theorem 2, p. 115), a result proved independently by Sturm \cite{art11-9} by a different method.

\item[2.] The first\pageoriginale case involving the continuous spectrum is that of Maass wave forms of weight zero, \ie cusp forms in $L^2 (\Gamma / G / K) = L^2 (\Gamma / \sfH)$, where $K$ denotes $SO(2)$ and $\sfH = G/ K$ the upper half-plane. This is the case treated in the present paper (with $\Gamma = SL_2 (\bZ)$).
 
\item[3.] Next, one can replace $SL_2 (\bR)$ and $SL_2 (\bZ)$ by $GL_2 (2, \bA)$ and $GL (2, F)$, respectively, where $F$ is a global field and $\bA$ the ring of adeles of $F$. This case, which is the most general one as far as $GL(2)$ is concerned, will be treated in a joint paper with Jacquet \cite{art11-3}. It includes as special cases 1 and 2, as well as their generalizations to holomorphic and non-holomorphic modular forms of arbitrary weight and level, Hilbert modular  forms, and automorphic forms over function fields.
 
\item[4.] Finally, the definition of $I(s)$ makes sense in any context where Eisenstein series can be defined, so it may be possible to apply the method sketched in this introduction to discrete subgroups of algebraic groups other than $GL(2)$.
\end{itemize}

\section{Statement of the main theorem.}\label{art11-sec1}
In this section we describe the main result of this paper, namely a formula for $I(s)$ in the critical strip $0< \re (s)<1$. In order to reduce the amount of notation and preliminaries needed, we will state the formula in terms of a certain holomorphic function $h(r)$; the relationship of $h(r)$ to the function $\varphi (g)$ of the introduction (Selberg transform) is well-known and will be reviewed in \S~2. Except at the end of \S~5, we will always consider only forms of weight 0 on the full modular group $\Gamma = SL_2(\bZ)/ \{\pm1\}$. The results for congruence subgroups are similar but messier to state and in any case will be subsumed by the results of \cite{art11-3}.

Any continuous $\Gamma$-invariant function $f: \sfH \to \bC$ has a Fourier expansion of the form 
\begin{equation*}
f(z) = \sum\limits^\infty 2 (\_{n = - \infty} A_n (f;y) e^{2\pi in x} \quad (z\in \sfH) \tag{1.1}\label{art11-eq1.1}
\end{equation*}
(here and in future we use $x$ and $y$ to denote the real and imaginary parts of $z \in \sfH$). We denote by $L^2 (\Gamma/ \sfH)$ the Hilbert space of $\Gamma$-invariant functions $f: \sfH \to \bC$ such that $(f,f) = \int\limits_{\Gamma / \sfH} |f(z)|^2dz$ is finite $\left(dz = \frac{dx dy}{y^2}\right)$ and by\pageoriginale $L^2_\circ (\Gamma / \sfH)$ the subspace of functions with $A_\circ (f;y ) \equiv 0$. The space $L^2_\circ (\Gamma / \sfH)$ is stable under the Laplace operator
$$
\Delta = y^2 \left(\frac{\partial^2}{\partial x^2} + \frac{\partial^2}{\partial y^2} \right)
$$
and has a basis $\{f_j\}_{j \geqslant 1}$ consisting of eigenforms of $\Delta$(see \cite{art11-4}, \S~5.2). 

We write 
\begin{equation*}
\Delta f_j = -\left(\frac{1}{4}  +r^2_j \right)f_j \quad  (j = 1, 2, \ldots ) \tag{1.2} \label{art11-eq1.2}
\end{equation*}
where $r_j \in \bC$. Since $\Delta$ is negative definite, we have $r^2_j + \dfrac{1}{4} \geqslant 0$, \ie $r_j$ is either real or else pure imaginary of absolute value $\leqslant \frac{1}{2}$. In fact it is known that the $r_j$ are real for $SL_2 (\bZ)$, but the corresponding statement for congruence subgroups is not known and we will use only $r^2_j \geqslant - \dfrac{1}{4}$. From \eqref{art11-eq1.2} we find that the $n^{\text{th}}$ Fourier coefficient $A_n (f_j , Y)$ satisfies the second order differential equation
$$
y^2 \frac{d^2}{dy^2} A_n (f_j, y) - 4 \pi^2 n^2 y^2 A_n (f; y) =- \left(\frac{1}{4} + r^2_j \right) A_n(f_j ; y). 
$$
The only solution of this equation which is bounded as $y \to \infty$ is $\sqrt{y} K_{ir_j} (2\pi |n|y)$, where $K_v (z)$ is the $K$-Bessel function, defined (for example) by 
\begin{equation*}
K_v (z) = \int\limits^\infty_0 e^{-z \cosh t} \cosh v t d t \quad (v, z \in \bC, \re (z) > 0).\tag{11}\label{art11-eq1.3}
\end{equation*}
Hence the $f_j$ have Fourier expansions of the form 
\begin{equation*}
f_j(z) = \sum\limits^\infty_{\substack{n = - \infty \\ n \neq 0}} a_j (n) \sqrt{y} K_{ir_j} (2 \pi |n| y) e^{2 \pi in x}
\tag{1.4}\label{art11-eq1.4}
\end{equation*}
with $a_j(n) \in\bC$. We can choose the $f_j$ to be normalized eigenfunctions of the Hecke operators
\begin{equation*}
\left\{
\begin{aligned}
&T(n) : f(z) \longrightarrow \frac{1}{n} \sum\limits_{\substack{a, d > 0\\ ad =n}} \sum\limits_{b(\mod d)} f \left(\frac{az+b}{d} \right) \quad (n >0), \\
& T (-1) : f(z) \longrightarrow f(-\bar{z}), \quad T (-n) = T (-1) T (n),
\end{aligned}
\right. \tag{1.5}\label{art11-eq1.5}
\end{equation*}
\ie\pageoriginale
\begin{equation*}
f_j |T (n) = \frac{a_j(n)}{|n|^{\frac{1}{2}}} f_j \qquad (n \in \bZ, \; n \neq 0)  \tag{1.6}\label{art11-eq1.6}
\end{equation*}
(then $a_j (1) = 1$, $a_j (-1) = \pm 1$, and $a_j(n)$ is multiplicative). The functions $f_j$ chosen in this way are called the \textit{Maass eigenforms;} they form an orthogonal (but not orthonormal) basis of $L^2_\circ (\Gamma / \sfH)$, uniquely determined up to order. For each $j$ we define the \textit{Rankin zeta-function} $R_{f_j}(s)$ by
\begin{equation*}
R_{f_j} (s) = \frac{\Gamma (\frac{s}{2})^2}{8 \pi^s \Gamma (s)} \Gamma \left(\frac{s}{2} + ir_j \right) \Gamma (\frac{s}{2} - ir_j ) \sum\limits^\infty_{\substack{n = -\infty\\ n \neq 0}} \frac{|a_j (n)|^2}{|n|^s} \quad (\re(s) > 1).
\tag{1.7}\label{art11-eq1.7}
\end{equation*}
We also set 
\begin{equation*}
R^{\ast}_{f_j}(s) = \pi^{-s} \Gamma (s) \zeta (2s) R_{f_j} (s) = \zeta^\ast (2s) R_{f_j} (s), \tag{1.8}\label{art11-eq1.8}
\end{equation*}
where $\zeta(s)$ denotes the Riemann zeta-function and 
\begin{equation*}
\zeta^\ast (s) = \pi^{-s/2} \Gamma \left(\frac{s}{2} \right) \zeta(s) = \zeta^\ast (1-s). \tag{1.9}\label{art11-eq1.9}
\end{equation*}
The Rankin-Selberg method implies that $R^\ast_{f_j}(s)$ has a meromorphic continuation to all $s$, is regular except for simple poles at $s =1$ and $s =0$ with 
\begin{equation*}
\res_{s=1}  \bR^\ast_{f_j} (s) = \frac{1}{2} (f_j, f_j), \tag{1.10}\label{art11-eq1.10}
\end{equation*}
and satisfies the functional euqation
\begin{equation*}
\bR^\ast_{f_j} (s) = \bR^\ast_{f_j} (1-s) \tag{1.11}\label{art11-eq1.11}
\end{equation*}
(the proofs will be recalled in \S~2).

We will also need the zeta-functions $\zeta(s,D)$, where $D$ is an integer congruent to 0 or 1 modulo 4. They are defined for $\re (s) >1$ by
\begin{equation*}
\zeta(s, D) = \sum\limits_{Q} \sum\limits_{m,n} \frac{1}{Q(m,n)^s} \qquad (\re (s) >1), \label{art11-eq1.12}
\end{equation*}
where\pageoriginale the first summation runs over all $SL_2 (\bZ)$-equivalence classes of binary quadratic forms $Q$ of discriminant $D$ and the second over all pairs of integers $(m,n) \in \bZ^2 / \Aut (Q)$ with $Q(m,n) >0$, where $\Aut (Q)$ is the stabilizer of $Q$ in $SL_2 (\bZ)$. These functions, which were introduced in \cite{art11-10}, are related to standard zeta-functions by 
\begin{equation*}
\zeta (s,D) = 
\left\{ 
\begin{aligned}
&\zeta(s) \zeta(2s -1) & & \text{if  }D = 0\\
& \zeta(s)^2  \cdot \text{(finite Dirichlet series )}&&\text{if } D = \text{square} \neq 0\\
& \zeta_{Q (\sqrt{D})} (s) \cdot \text{(finite Dirichlet series)}& & \text{if } D \neq \text{square,}
\end{aligned}
\right.
x\tag{1.13}\label{art11-eq1.13}
\end{equation*}
where $\zeta_{\bQ(\sqrt{D})} (s)$  denotes the Dedekind zeta-function of $\bQ (\sqrt{D})$ (for precise formulas see \cite{art11-10}, Proposition 3, p. 130). In particular, $\zeta(s,D)$ has a meromorphic continuation in $s$ and $\zeta(s,D)/ \zeta(s)$ is holomorphic except for a simple pole at $s =1$ when $D$ is a square. 

Now let $h : \bR \longrightarrow \bC$ be a function satisfying 
\begin{equation*}
\left\{
\begin{aligned}
& h(r) = h(-r);\\
& h(r) \quad \text{ has a holomorphic continuation to the strip } |\Iim(r)|< \frac{1}{2} A \\
& \quad \quad~~ \text{ for some } A > 1;\\
& h(r) \text{ is of rapid decay in this strip}
\end{aligned}
\right. \tag{1.14}\label{art11-eq1.14}
\end{equation*}
(``rapid decay'' means $O(|r|^{-N})$ for all $N$). The object of this paper is to compute $\sum\limits^\infty_{j=1} \dfrac{h(r_j)}{(f_j , f_j)} R_{f_j} (s)$. In \S \S 2 and 3 we will show that this function equals the function $I(s)$ of \S 0 and compute it in the strip $1 < \re (s) < A$ by the Rankin-Selberg method; \S\S 4 and 5 give the analytic continuation in $s$, computation of the residue at $s=1$ (Selberg trace formula), and generalization to $\sum\limits^\infty_{j=1} a_j (m) \dfrac{h(r_j)}{(f_j, f_j)} \bR_{f_j} (s)$, where the $a_j(m)$ are the Fourier coefficients defined by \eqref{art11-eq1.4}. We state here the final result for $0< \re (s) <1$ and $m>0$ in a form which makes the functional equation apparent. 

\medskip
\noindent
{\bfseries Theorem \thnum{1}:\label{art11-thm1}}
\textit{Let $h : \bR \longrightarrow \bC$ be a function satisfying the conditions \eqref{art11-eq1.14} and $m \geqslant 1$ an integer. Then for $s \in\bC$ with $0 <\re (s) <1$ we have the identity}
\begin{equation*}
\sum\limits^\infty_{j =1} a_j(m) \frac{h(r_j)}{(f_j, f_j)} R^\ast_{f_j} (s) = \sR(s) + \sR(1-s)
\tag{1.15}\label{art11-eq1.15}
\end{equation*}
\textit{with\pageoriginale $\sR(s) = \sR (s; m,h)$ given by }
\begin{align*}
\sR (s) & = - \frac{1}{8 \pi} \zeta^\ast (s)^2 \int\limits^\infty_{-\infty} \frac{\zeta^\ast (s+ 2 ir) \zeta^\ast (s-2ir)}{\zeta^\ast (1+ 2 ir ) \zeta^\ast (1-2 ir)} \left(\sum\limits_{\substack{a, d \geq 1\\ ad =m}} \left(\frac{a}{d} \right)^{ir}  \right) h (r) dr \\
&- \frac{1}{2} \frac{\zeta^\ast (s) \zeta^\ast (2s)}{\zeta^\ast (s+1)} \left(\sum\limits_{\substack{a, d \geq 1\\ ad =m}} \left(\frac{a}{d} \right)^{s/2}  \right) h (\frac{is}{2}) \\
&+ \frac{m^{\frac{s-1}{2}}}{4\pi^2} \frac{\Gamma (s) \Gamma (s - \frac{1}{2})}{\Gamma \left(\frac{1+s}{2} \right) \Gamma \left(\frac{2-s}{2} \right)} \sum\limits^\infty_{t = - \infty} \zeta(s, t^2 - 4 m) \times \tag{1.16}\label{art11-eq1.16}\\
& \times \int\limits^\infty_{-\infty} \frac{\Gamma \left(\frac{1-s}{2} + ir \right) \Gamma\left(\frac{1-s}{2} - ir \right) }{\Gamma (ir) \Gamma (-ir)}\\
& \times F \left(\frac{1-s}{2} + ir, \frac{1-s}{2} - ir; \frac{3}{2} - s ; 1 - \frac{t^2}{4 m} \right) h (r) dr, 
\end{align*}
\textit{where $\zeta^\ast(s)$ and $\zeta (s, t^2 - 4 m)$ are defined by equations \eqref{art11-eq1.9} and \eqref{art11-eq1.12} and $F(a, b; c; z)$ denotes the hypergeometric function (defined by analytic continuation if $z<0$) and can be expressed in terms of Legendre functions for the special values of the parameters $a, b, c$ occurring in \eqref{art11-eq1.16}.}

\textit{For $m<0$ there is a similar formula with $m$ replaced by $|m|$ in the first two terms and the function $m^{\frac{s-1}{2}} F \left(\dfrac{1-s}{2} + ir, \dfrac{1-s}{2} -ir ; \dfrac{3}{2} - s; 1 - \dfrac{t^2}{4m} \right)$ in the third term replaced by a different hypergeometric function.}

\begin{coro*}
The Rankin zeta-function $R^\ast_{f_j} (s)$ is divisible by $\zeta^\ast (s)$ for all $j$.
\end{coro*}

\medskip
\noindent
Proof of the Corollary: Every term on the right-hand side of equation \eqref{art11-eq1.16} (and of the corresponding formula for $m < 0$) is divisible by $\zeta^\ast (s)$; since the series converges absolutely, we deduce that $\sR(s)$ (and hence, by the functional equation \eqref{art11-eq1.9}, also $\sR (1-s)$) is divisible by $\zeta^\ast (s)$. Therefore the expression on the left-hand side of equation \eqref{art11-eq1.15}, vanishes (with the appropriate multiplicity) at every zero of the Riemann zeta-function, and the linear independence of the eigenvalues $a_j(m)h(r_j) (m \in \bZ - \{0\}$, $h$ satisfying \eqref{art11-eq1.14}) for different $j$ implies that the same holds for each $R^\ast_{f_j} (s)$. A more formal argument is as follows: For $z \in \sfH$ define 
$$
\Phi (s,z) = \sum\limits^\infty_{j=1} \frac{1}{(f_j , f_j)} f_j  (z) R^\ast_{f_j} (s);
$$
then \eqref{art11-eq1.14} and \eqref{art11-eq1.15} imply the identity
$$
\Phi (s, z) = \sqrt{y} \sum\limits^\infty_{\substack{m = - \infty\\ m \neq 0}} [\sR (s; m, h_{my}) + \sR (1-s; m, h_{my})] e^{2 \pi  i m x},
$$
where $h_{m,y} (r) = K_{ir} (2 \pi |m|y)$. Therefore $\Phi (s, z)$ is divisible by $\zeta^\ast(s)$ and the corollary follows because $R^\ast_{f_j} (s)$ equals the scalar product ($\Phi (s,\cdot), f_j$).

As mentioned in the introduction, the above Corollary, which is the analogue of the result for holomorphic forms proved in \cite{art11-8} and \cite{art11-10}, is included in the results of Jacquet-Gelbart \cite{art11-2}. We also observe that, up to gamma factors, the quotient $R^\ast_{f_j} (s)/ \zeta^\ast(s)$ equals 
$$
\frac{\zeta(2s)}{\zeta(s)} \sum\limits^\infty_{n=1} \frac{|a_j(n)|^2}{n^s}.
$$
Using the usual relations among the eigenvalues $a_j(n)$ of a Hecke eigenform, we see that this Dirichlet series has the Euler product
$$
\prod\limits_p \frac{1}{(1-\alpha^2_p p^{-s})(1-\alpha_p \beta_p p^{-s}) (1-\beta^2_p p^{-s})} . 
$$
where $\alpha_p, \beta_p$ are defined by
$$
\sum\limits^\infty_{n=1} \frac{a_j(n)}{n^s} = \prod\limits_p \frac{1}{(1-\alpha_p p^{-s}) (1-\beta_p p^{-s})}
$$
(\ie $\alpha_p + \beta_p = a_j (p)$, $\alpha_p \beta_p=1$). Thus the corollary is the case $n =2$ of the conjecture that the ``symmetric power $L$-functions''
$$
L_n (f_j , s) = \prod\limits_p \prod\limits^n_{m=0} \frac{1}{(1-\alpha^m_p \beta^{n-m}_p p^{-s})}
$$
are entire functions of $s$ for all $n \geqslant 1$.

\section{Eisenstein series and the spectral decomposition of $L^2 (\Gamma / \sfH)$.}\label{art11-sec2}
In this section we review the definitions and main properties of Eisenstein series, the Rankin-Selberg method, the spectral decomposition formula for $L^2 (\Gamma / \sfH)$, the Selberg transform, and the Selberg kernel function. All of this material is standard and may be skipped by the expert reader. We will try to give at least a rough proof of all of the statements; for a more detailed exposition the reader is referred to Kubota's book \cite{art11-4}.

\medskip
\noindent
\text{Eisenstein Series.}
For $z \in \sfH$ and $s \in \bC$ with $\re (s) >1$ we set
\begin{equation*}
E(z,s ) = \sum\limits_{ \gamma \in \Gamma_\infty/ \Gamma} \Iim (\gamma z)^s \quad (\re (s) >1), \tag{2.1}\label{art11-eq2.1}
\end{equation*}
where $\Gamma_\infty = \left\{ \left(\begin{matrix}
a & b \\
0 &d
\end{matrix}
\right) \in SL_2(\bZ) \right\} / \{\pm 1\} \cong\bZ$ is the group of translations in $\Gamma$. The series converges absolutely and uniformly and therefore defines a function which is holomorphic in $s$ and real-analytic and $\Gamma$-invariant with respect to $z$. Using the $1:1$ correspondence between $\Gamma_\infty/ \Gamma$ and pairs of relatively prime integers (up to sign) given by $\Gamma_\infty \left(\begin{matrix}
a & b \\ c & d
\end{matrix}\right) \longleftrightarrow \pm (c,d)$, we can rewrite \eqref{art11-eq2.1} as 
$$
E(z, s) = \frac{1}{2} \sum\limits_{\substack{c , d \in \bZ\\ (c,d) =1}} \frac{y^s}{|cz+d|^{2s}} \quad (\re z > 1)
$$
and hence 
\begin{equation*}
\zeta(2s) E (z,s) =\frac{y^s}{2} \sum\limits'_{m,n} \frac{1}{|mz+n|^{2s}} \quad  (\re (s) >1), \tag{2.2}\label{art11-eq2.2}
\end{equation*}
where $\sum'$ denotes a summation over all pairs of integers $(m,n) \neq (0,0)$. This latter function has better analytic properties than $E(z,s)$, namely:

\medskip
\noindent
{\bfseries Proposition \thnum{1}.\label{art11-prop1}} 
\textit{The function \eqref{art11-eq2.2} can be continued meromorphically to the whole complex $s$-plane, is holomorphic except for a simple pole at $s =1$, and\pageoriginale satisfies the functional equation}
\begin{equation*}
E^{\ast} (z,s) = E^\ast (z, 1-s), \tag{2.3}\label{art11-eq2.3}
\end{equation*}
\textit{where}
\begin{equation*}
E^\ast (z,s) = \pi^{-s} \Gamma (s) \zeta(2s) E (z,s) = \zeta^\ast (2s) E (z,s). \tag{2.4} \label{art11-eq2.4}
\end{equation*}
\textit{The residue at $s =1$ is independent of $z$:}
\begin{equation*}
\res_{s=1} E(z,s) = \frac{6}{\pi} \res_{s =1} E^\ast (z,s) = \frac{3}{4} \quad (z \in \sfH). \tag{2.5} \label{art11-eq2.5}
\end{equation*}

We will deduce these properties from the Fourier development of $E(z,s)$, which itself will be needed in the sequel. Separating the terms $m=0$ and $m \neq 0$ in \eqref{art11-eq2.2} gives 
$$
\zeta(2s) E (z,s) = y^s [\zeta (2s) + \sum\limits^\infty_{m=1} \varphi_s (mz)] \qquad (\re (s) > 1), 
$$
where 
$$
\varphi_s (z) = \sum\limits^\infty_{n = - \infty} \frac{1}{|z+n|^{2s}} \qquad (z \in \sfH, \re (s) > \frac{1}{2}).
$$
The function $\varphi_s(x + iy)$ is periodic in $x$ for fixed $y$ and hence has a Fourier development $\sum\limits^\infty_{\pi = - \infty} a(n, s,y) e^{2 \pi in x}$ with 
\begin{align*}
a(n, s, y) & = \int\limits^\infty_{-\infty} \frac{e^{-2 \pi in x}}{(x^2 + y^2)^s} dx\\
& = \left\{ 
\begin{aligned}
& \frac{\Gamma (\frac{1}{2} \Gamma (s - \frac{1}{2}))}{\Gamma (s)} y^{1-2s} \qquad (n=0)\\
& 2 \left(\frac{\pi |n|}{y} \right)^{s-\frac{1}{2}} \frac{\Gamma (\frac{1}{2})}{\Gamma (s)} K_{s - \frac{1}{2}} (2 \pi |n| y) (n \neq 0)
\end{aligned}
\right.
\end{align*}
[GR 3.251.2 and /8.432.5]. Hence 
\begin{gather*}
\zeta (2 s) E (z, s) = \zeta (2s) y^s  + \frac{\Gamma (\frac{1}{2}) \Gamma (s - \frac{1}{2})}{\Gamma (s)} \zeta (2 s - 1) y^{1-s} \\
 + 2 \frac{\pi^s y^{\frac{1}{2}}}{\Gamma (s)} \sum\limits^\infty_{m=1} \sum\limits^\infty_{\substack{n = - \infty\\ n \neq 0}} \left(\frac{|n|}{m} \right)^{s  - \frac{1}{2}} K_{s - \frac{1}{2}} (2 \pi |n| my) e^{2 \pi in m x}
\end{gather*}
or, multiplying\pageoriginale both sides by $\pi^{-s} \Gamma (s)$,
\begin{align*}
& E^\ast (z,s) = \zeta^\ast (2s) y^s + \zeta^\ast (2s -1) y^{1-s} \tag{2.6}\label{art11-eq2.6}\\
& + 2 \sqrt{y} \sum\limits^\infty_{\substack{n = - \infty\\ n \neq 0}} \tau_{s - \frac{1}{2}} K_{s - \frac{1}{2}} (2 \pi |n|y)e^{2 \pi in x}, 
\end{align*}
where $\zeta^\ast(s)$ is defined by \eqref{art11-eq1.9} and $\tau_v (n)$ by 
\begin{equation*}
\tau_v (n) = |n|^v \sum\limits_{\substack{d|n}\\d >0} d^{-2v} = \sum\limits_{\substack{ad = |n|\\ a, d>0}} (\frac{a}{d})^v  \quad (n \in \bZ -\{0\}, v \in \bC). \tag{2.7}\label{art11-eq2.7}
\end{equation*}
The infinite sum in \eqref{art11-eq2.6} converges absolutely and uniformly for all $s$ and $z$, so \eqref{art11-eq2.6} implies that $E^\ast(z,s)$ can be continued meromorphically to all $s$, the only poles being simple poles at $s=0$ and $s =1$ with residue $\pm \frac{1}{2}$ (the poles of $\zeta^\ast (2s)$ and $\zeta^\ast (2s-1)$ at $s = \frac{1}{2}$ cancel). Also, it is clear from \eqref{art11-eq1.3} and the second formula of \eqref{art11-2.7} that $K_v (z)$ and $\tau_v (n)$ are even functions of $v$, so the functional equation of $E^\ast(z,s)$ follows from \eqref{art11-eq2.6} and \eqref{art11-eq1.9}. Another consequence of \eqref{art11-eq2.6} is the estimate 
\begin{equation*}
E(z,s) = O(y^{\max (\sigma, 1 -\sigma)})\quad (y \to \infty), \tag{2.8}\label{art11-eq2.8}
\end{equation*}
where $\sigma = \re(s)$; this follows because the sum of Bessel functions is exponentially small as $y \to \infty$. 

\medskip
\noindent
\textsc{The rankin-selberg method.} We use this term to designate the general principle that the scalar product of a function $f: \Gamma / \sfH \to \bC$ with an Eisenstein series equals the Mellin transform of the constant term in the Fourier development of $f$. More precisely, we have:

\medskip
\noindent
{\bfseries Proposition \thnum{2}:\label{art11-prop1}}
\textit{Let $f(z)$ be a $\Gamma$-invariant function in the upper half-plane which is of sufficiently rapid decay that the scalar product}
\begin{equation*}
(f, E(.,\bar{s})) = \int\limits_{\Gamma / \sfH} \quad f(z) E (z,s) dz \tag{2.9}\label{art11-eq2.9}
\end{equation*}
converges absolutely for some $s$ with $\re (s) >1$. Then for such $s$
\begin{equation*}
(f, E (.,\bar{s})) = \int\limits^\infty_{0} y^{s-2} A_\circ (f;y) dy \tag{2.10}\label{art11-eq2.10}
\end{equation*}
\textit{where $A_\circ (f,y)$ is defined by equation \eqref{art11-eq1.1}.}

\begin{proof}
Substituting \eqref{art11-eq2.1} into \eqref{art11-2.9} we find 
\begin{align*}
(f, E (., \bar{s})) & = \int\limits_{\Gamma / \sfH} f(z) \sum\limits_{\gamma  \in\Gamma_\infty/ \Gamma} \Iim (\gamma z)^s dz \\
& =\int\limits_{\Gamma_\infty/ \sfH} f(z) \Iim (z)^s dz\\
& = \int\limits^\infty_{0} \int\limits^1_0 f (x + iy) y^s \frac{dx dy}{y^2}
\end{align*}
which is equivalent to \eqref{art11-eq2.10}.

Note that the growth condition on $f$ in the proposition is satisfied if $f(z) = O (y^{-\epsilon})$ as $y \to \infty$ for some $\epsilon >0$, for then \eqref{art11-eq2.8} implies that the scalar product \eqref{art11-eq2.9} converges absolutely in the strip $-\epsilon < \re (s) < 1+ \epsilon$.

One of the main applications of Proposition \eqref{art11-prop2} is the one obtained by choosing $f(z) = |f_j(z)|^2$, where $f_j$ is a Maass eigenform. (This was the original application made by Ranking \cite{art11-5} and Selberg \cite{art11-6}, except that they were looking at holomorphic cusp forms.) From \eqref{art11-eq1.4} we find that the constant term of $f$ is given by 
$$
A_\circ (f; y) = y \sum\limits_{n \neq 0} |a_j (n)|^2 K_{ir_j} (2\pi|n|y)^2
$$
(notice that $K_{ir_j} (2\pi|n|y)$ is real by \eqref{art11-eq1.3}, since $r_j$ is either real or pure imaginary). Hence \eqref{art11-eq2.10} gives 
\begin{align*}
\int\limits_{\Gamma / \sfH} |f_j(z)|^2 E (z,s) dz & = \int\limits^\infty_0 y^{s-1} \sum\limits_{n \neq 0} |a_j(n)|^2 K_{ir_j} (2 \pi |n| y)^2 dy\\
& = \sum\limits_{n \neq 0} \frac{|a_j(n)|^2}{|n|^s} \int\limits^\infty_{0} y^{s-1} K_{ir_j} (2\pi y)^2 dy \tag{2.11}\label{art11-eq2.11}\\
& = R_{f_j}(s) \qquad (\re (s) > 1)
\end{align*}
(the integral is evaluated in [ET 6.8 (45)] and equals the gamma factor in \eqref{art11-eq1.7}). The analytic properties of $R_{f_j}(s)$ given in \S 1 (meromorphic continuation, position of poles, residue formula \eqref{art11-eq1.10}, functional equation \eqref{art11-eq1.11}) follow from \eqref{art11-eq2.11} and the corresponding properties of $E(z,s)$.
\end{proof}

\medskip
\noindent
\textsc{Spectral decomposition.} We now give a rough indication, ignoring analytic\pageoriginale problems, of how the Rankin-Selberg method implies the spectral decomposition formula for $L^2 (\Gamma / \sfH)$. This formula states that any $f \in L^2 (\Gamma / \sfH)$ has an expansion
\begin{equation*}
f(z) = \sum\limits^\infty_{j=0} \frac{(f, f_j)}{(f_j, f_j)} f_j(z) + \frac{1}{4\pi} \int\limits^\infty_{-\infty} (f, E (., \frac{1}{2} + ir)) E (z, \frac{1}{2} + ir) dr, \tag{2.12} \label{art11-eq2.12}
\end{equation*}
where $\{f_j\}_{j \geq 1}$ is an orthogonal basis for $L^2_\circ (\Gamma / \sfH)$ and $\{f_\circ\}$ for the space of constant functions (we will choose $f_j(j \geq 1)$ to be the normalized Maass eigenforms and $f_\circ (z) \equiv 1$). We prove it under the assumption that $f$ is of sufficiently rapid decay, say $f(z) = O(y^{-\epsilon})$ with $\epsilon >0$. Let $\Psi (s)$  be the scalar product \eqref{art11-eq2.9}. Proposition \ref{art11-prop1} shows that $\Psi (s)$ is a meromorphic function of $s$, is regular in $0  < \re (s) <1 +\epsilon$ except  for a simple pole at $s =1$ with 
\begin{equation*}
\res_{s=1} \Psi (s) = \frac{3}{\pi} \int\limits_{\Gamma / \sfH} f(z) dz = \frac{(f, f_\circ)}{(f_\circ, f_\circ)}  f_\circ,
\tag{2.13}\label{art11-eq2.13}
\end{equation*}
and satisfies the functional equation 
\begin{equation*}
\Psi (s) = \frac{\zeta^\ast (2s -1)}{\zeta^\ast(2s)} \Psi (1-s). \tag{2.14}\label{art11-eq2.14}
\end{equation*}
On the other hand, \eqref{art11-eq2.10} says that $\Psi (s)$ is the Mellin transform of $\dfrac{1}{y}A_\circ (f;y)$, so by the Mellin inversion formula
$$
A_\circ (f;y) = \frac{1}{2 \pi i} \int\limits^{C + i \infty}_{C - i \infty} \Psi (s) y^{1-s} ds \quad  (1< C < 1+ \epsilon) .
$$
Moving the path of integration from $\re (s) = C$ to $\re (s) = \dfrac{1}{2}$ and using \eqref{art11-eq2.13} and \eqref{art11-eq2.14} we find 
\begin{align*}
A_\circ (f;y) & = \frac{f, f_\circ}{f_\circ, f_\circ} f_\circ + \frac{1}{2\pi} \int\limits^\infty_{-\infty} \Psi (\frac{1}{2} - ir) y^{\frac{1}{2}  + ir} \quad dr\\
& = \frac{(f, f_\circ)}{(f_\circ, f_\circ)}  f_\circ + \frac{1}{4\pi} \int\limits^\infty_{-\infty}  \Psi (\frac{1}{2} - ir) (y^{\frac{1}{2} + ir} + \frac{\zeta^\ast (1-2ir)}{\zeta^\ast (1+ 2 ir)} y^{\frac{1}{2} - ir}) dr. \tag{2.15}\label{art11-eq2.15}
\end{align*}
On the other hand, equation \eqref{art11-eq2.6} implies that $y^{\frac{1}{2} + ir} + \dfrac{\zeta^\ast (1-2 i r)}{\zeta^\ast(1+ 2 ir)} y^{\frac{1}{2} - ir}$ is the constant term of $E(z, \frac{1}{2} + ir )$, so \eqref{art11-eq2.15}  tells us that the $\Gamma$-invariant function 
$$
\tilde{f}(z) = f(z) - \frac{(f, f_\circ)}{(f_\circ, f_\circ)} f_\circ (z)  -\frac{1}{4\pi} \int\limits^\infty_{-\infty} \Psi (\frac{1}{2} -ir ) E (z, \frac{1}{2} + ir) dr
$$
has zero constant term. It is also square integrable, because $f(z)$ is and the non-constant terms in the Fourier expansion of $E(z, \frac{1}{2} + ir)$ are exponentially small. Hence $\tilde{f} \in L^2_\circ (\Gamma / \sfH)$, so $\tilde{f}(z) = \sum\limits^\infty_{j=1} \dfrac{(\tilde{f}, f_j)}{(f_j , f_j)} f_j (z)$, and this proves \eqref{art11-eq2.12} since $(\tilde{f}, f_j) = (f, f_j)$ for all $j \geq 1$. 

\medskip
\noindent
\textsc{Selberg transform.} As in the introduction, let $\varphi$ be a function on $G$ of sufficiently rapid decay and $T_\varphi$  the operator given by convolution with $\varphi$. Since we are interested only in functions on the upper half-plane $\sfH = G / K$ (where $K = SO(2)$ and the identification is given by 
$\left.\left( 
\begin{matrix}
a & b \\
c & d
\end{matrix}
\right) K \leftrightarrow \dfrac{ai + b}{ci + d} \right)$ we can assume that $\varphi$ is left and right $K$-invariant. But the map
$$
t : K \left(\begin{matrix}
a & b \\
 c & d 
\end{matrix}
\right) K \longmapsto a^2 + b^2 + c^2 + d^2 -2
$$
gives an isomorphism between $K \backslash G / K$ and $[0, \infty)$ (Cartan decomposition), so we can think of $\varphi$ as a map
$$
\varphi : [0, \infty) \to  \bC. 
$$
An easy calculation shows that 
$$
t (g^{-1} g') = \frac{|z-z'|^2}{yy'} \qquad (g, g' \in G), 
$$
where $z, z' \in \sfH$ are the images of $g$ and $g'$. Therefore $T_\varphi$  acts on functions $f: \sfH \to \bC$ by
\begin{equation*}
T_\varphi f(z) =\int\limits_{H} k(z, z') f(z') dz' \quad (z \in \sfH), \tag{2.16} \label{art11-eq2.16}
\end{equation*}\pageoriginale 
where
\begin{equation*}
k(z, z') =\varphi \left(\frac{|z-z'|^2}{yy'} \right) \qquad (z, z' \in \sfH). \tag{2.17}\label{art11-eq2.17}
\end{equation*}
The growth condition we want to impose on $\varphi$ is that 
\begin{equation*}
\varphi (x) = O \left(x^{\frac{1+A}{2}} \right) \qquad (x \to \infty) \tag{2.18}\label{art11-eq2.18}
\end{equation*}
for some $A >1$; then \eqref{art11-eq2.16} converges for any $f$ in the vector space
$$
V = \{f ;\sfH \longrightarrow \bC |f \text{ is continuous, } f(z) = O \left(y^{-\frac{1+A}{2}} \right)\}. 
$$
Because $k(z,z') =k (gz, gz')$ for any $ g \in G$, the operator $T_\varphi$ commutes with the action of $G$. A general argument (\cf \cite{art11-7}, p. 55 or \cite{art11-4}, Theorem 1.3.2) then shows that any eigenfunction of the Laplace operator is also an eigenfunction of $T_\varphi$. More precisely, 
\begin{equation*}
f\in V, \Delta f= -\left(\frac{1}{4} + r^2 \right) f \Rightarrow T_\varphi f = h (r) f,
\tag{2.19} \label{art11-eq2.19}
\end{equation*}
where $h(r)$, the \textit{Selberg transform} of $\varphi$, is an even function of $r$, depending on $\varphi$ but not on $f$. To compute it, we choose $f(z) = y^{\frac{1}{2} + ir}$, which satisfies the conditions is \eqref{art11-eq2.19} if $r \in \bC$ with $|\Iim (r)| <\dfrac{A}{2}$. Then 
$$
T_\varphi f(z) = \int\limits^\infty_0 y'^{-\frac{3}{2} + ir} \int\limits^\infty_{-\infty} \varphi  \left(\frac{(x-x')^2 + (y-y')^2}{yy'} \right) dx' \; dy'. 
$$
Making the change of variables $x' = x + \sqrt{yy'} v$ in the inner integral gives 
$$
T_\varphi f(z) = \int\limits^\infty_0 y'^{-\frac{3}{2} + ir} \sqrt{yy'} Q \left(\frac{(y-y')^2}{yy'} \right) dy',
$$
where\pageoriginale the function $Q$ is defined by 
\begin{equation*}
Q(w) = \int\limits^\infty_{-\infty} \varphi (w + v^2) dv = \int\limits^\infty_{w}  \frac{\varphi (t) dt}{\sqrt{t -w}} \quad (w \geqslant 0). 
\tag{2.20}\label{art11-eq2.20}
\end{equation*}
The further change of variables $y' = ye^u$ then gives 
$$
T_\varphi f(z) = y^{\frac{1}{2} + ir} \int\limits^\infty_{-\infty} e^{ir u} Q (e^u -2 + e^{-u}) du. 
$$
Hence, setting
\begin{equation*}
g(u) = Q(e^u - 2 + e^{-u}) \qquad (u \in \bR), \tag{2.21}\label{art11-eq2.21}
\end{equation*}
we have 
\begin{equation*}
h(r) = \int\limits^\infty_{-\infty} g (u) e^{iru} du \quad (r \in \bC, |\Iim (r)| < \frac{A}{2}). 
\tag{2.22}\label{art11-eq2.22}
\end{equation*}
Formulas \eqref{art11-eq2.20}-\eqref{art11-eq2.22} describe the Selberg transform (the notations $Q, g, h$, due to Selberg, are by now standard and we have retained them). The inverse transform is easily seen to be 
\begin{equation*}
\left\{
\begin{aligned}
& g (u) = \frac{1}{2 \pi} \int\limits^\infty_{-\infty} h (r) e^{ir u} du, \\
& Q (w) = g (2 \sinh^{-1\frac{\sqrt{w}}{2}}),\\
& \varphi (x) = \frac{-1}{\pi} \int\limits^\infty_{-\infty} Q' (x +
  v^2) dv.
\end{aligned}
\right.
\end{equation*}
We can also combine these three integrals, obtaining 
\begin{align*}
\varphi (x) & = \frac{1}{2 \pi^2} \int\limits^\infty_{-\infty} r h (r)
\int\limits^\infty_{\cosh^{-1} (1+\frac{x}{2})} \frac{\sin  ru
}{\sqrt{e^u+ e^{-u - 2 -x}}} du \;dr \\
& = \frac{1}{4 \pi} \int\limits^\infty_{-\infty} P_{-\frac{1}{2} + ir}
(1+ \frac{x}{2}) \quad r \tan h \pi r \; h (r) dr,  \tag{2.24} \label{art11-eq2.24}
\end{align*}
where\pageoriginale $P_v (z) (v \in\bC, z \in \bC - (- \infty, 1])$
denotes a Legendre function of the first kind. (For properties of
Legendre functions we refer the reader to \cite{art11-EH}, Chapter 3;
in particular, the integral representation of $P_{-\frac{1}{2} + ir}$
just used follows from formulas 3.7 (4) and 3.3.1 (3) there.) The
inversion formula of Mehler and Fock (\cite{art11-EH}, p. 175) then gives 
\begin{equation*}
h(r) = 2 \pi \int\limits^\infty_0 P_{\frac{1}{2} +ir} (1+ \frac{x}{2}) \varphi (x) dx ~~ (|\Iim (r) |<\frac{A}{2}).
\tag{2.25}\label{art11-eq2.25}
\end{equation*}
From \eqref{art11-eq2.20} - \eqref{art11-eq2.23} we see easily that the conditions 
\begin{align*}
& \varphi (x) = O \left(x^{-\frac{1+A}{2}} \right), \\
& Q (w) = O(w^{-\frac{A}{2}}), \\
& g (u) = O(e^{-\frac{A}{2}|u|}),\\
& h(r) \text{ holomorphic in } |\Iim (r)| <\frac{A}{2}
\end{align*}
are equivalent; this also follows from \eqref{art11-eq2.24} and \eqref{art11-2.25} since $P_{-\frac{1}{2} + ir} (x)$ grows like $x^{-\frac{1}{2} + |\Iim (r)|}$ as $x \to \infty$ [EH 3.9.2 (19), (20)]. Thus the growth condition \eqref{art11-eq2.18} is equivalent to a holomorphy condition on $h$, while the condition that $\varphi$ be smooth is equivalent to the requirement that $h$ be of rapid decay.

\medskip
\noindent
\textsc{Selberg kernel function.} Now suppose that the function $f$ in \eqref{art11-eq2.16} is $\Gamma$-invariant. Then $T_\varphi f$  is also $\Gamma$-invariant and clearly
\begin{equation*}
T_\varphi f(z) = \int\limits_{\Gamma / \sfH} K(z, z') f(z') dz' \tag{2.26}\label{art11-eq2.26}
\end{equation*}
with 
\begin{equation*}
K(z, z') = \sum\limits_{\gamma \in \Gamma} k (z, \gamma z'),  \tag{2.27}\label{art11-eq2.27}
\end{equation*}
\ie the action of $T_\varphi$ on $\Gamma$-invariant functions is given by the kernel function \eqref{art11-eq2.27}. We claim that 
\begin{equation*}
K(z,z') = O(y'^{\frac{1-A}{2}}) \quad  (z \text{ fixed, }y' \longrightarrow \infty) \label{art11-eq2.28}
\end{equation*}
if $\varphi$ satisfies \eqref{art11-eq2.18}. To see this, write 
$$
K(z,z') =  \sum\limits_{n\in Z} k (z + n, z') + \sum\limits_{\substack{\gamma \in \Gamma_\infty / \Gamma \\ \gamma \not\in \Gamma_\infty}} \sum\limits_{n\in Z} k(z+n, \gamma z').
$$
The\pageoriginale first term is easily seen to be $O(y'^{\frac{1-A}{2}})$. In the second term, $\Iim (\gamma z')$ is uniformly small as $y' \to \infty$ and from this one easily sees that the inner sum is uniformly $O(\Iim (\gamma z')^{\frac{A+1}{2}})$. Therefore the second term is 
$$
O\left(\sum\limits_{\substack{\gamma \in \Gamma_\infty/ \Gamma\\ \gamma \not\in \Gamma_\infty}}  \Iim (\gamma' z)^{\frac{A+1}{2}} \right) = O \left(E(z' , \frac{A+1}{2})  - y'^{\frac{A+1}{2}}) \right)
$$
which by \eqref{art11-eq2.6} is $O(y'^{\frac{1-A}{2}})$.

From \eqref{art11-eq2.28} is follows that $K(z, z')$ is in $L^2 (\Gamma / \sfH)$ with respect to each variable separately and that the scalar product $(K(\cdot, z'), E (\cdot, s))$ converges for $\dfrac{1-A}{2} < \re (s) < \dfrac{1+A}{2}$. Using \eqref{art11-eq2.26} and \eqref{art11-eq2.19} we find 
\begin{equation*}
(K(\cdot, z'), f_j) = h  (r_j ) \bar{f_j(z')} \quad (j \geq 0), \tag{2.29}\label{art11-eq2.29}
\end{equation*}
where $r_j$ is given by \eqref{art11-eq1.2} for $j \geq 1$ and $r_\circ = \dfrac{i}{2}$, and similarly
$$
(K(\cdot, z'), E (\cdot, \frac{1}{2} + ir)) = h (r) E (z', \frac{1}{2} - ir)
$$
since $\Delta E(z, \frac{1}{2} + ir) = - \left(\frac{1}{4} + r^2 \right) E (z, \frac{1}{2} + ir)$. Therefore the spectral decomposition formula \eqref{art11-eq2.12} applied to $K(\cdot, z')$ gives 
$$
K(z, z') = \sum\limits^\infty_{j=0} \frac{h(r_j)}{(f_j , f_j)} f_j (z) \overline{f_j(z')} + \frac{1}{4 \pi} 
\int\limits^\infty_{-\infty} E (z,\frac{1}{2} + ir) E (z' , \frac{1}{2} - ir) h (r) dr. 
$$
We restate this formula as 











%%%%%%%%%%%5


\begin{thebibliography}{99}
\bibitem[1]{art11-1} dhfjdh f

\bibitem[2]{art11-2} dfjdfkjk 

\bibitem[3]{art11-3} kdjfkdjf 


\begin{center}
{\bf Tables}
\end{center}

\bibitem[EH]{art11-EH} kdjfkdjfk 
\end{thebibliography}

\medskip
\noindent
{\bfseries Proposition \thnum{3}:\label{art11-prop3}}
\textit{Let $h(r)$ be a function satisfying \eqref{art11-eq1.14} and set}
\begin{equation*}
K_\circ (z, z') = \sum\limits^\infty_{j=1} \frac{h(r_j)}{(f_j, f_j)} f_j (z) \overline{f_j (z')} \quad (z, z' \in H), 
\tag{2.30}\label{art11-eq2.30}
\end{equation*}
\textit{where $\{f_j\}$ is an orthogonal basis of $L^2_\circ (\Gamma / \sfH)$ satisfying \eqref{art11-eq1.2}. Let $k(z,z') (z, z' \ion \sfH)$ be the function defined by \eqref{art11-eq2.17}, where $\varphi$ is given by \eqref{art11-eq2.23} or \eqref{art11-eq2.24}. Then}  
\begin{align*}
& K_\circ (z,z') =\sum\limits_{\gamma \in \Gamma} k (z, \gamma z') - \frac{3}{\pi} h (\frac{i}{2}) \tag{2.31} \label{art11-2.31}\\
& -\frac{1}{4\pi} \int\limits^\infty_{-\infty} E (z, \frac{1}{2} + ir) E (z', \frac{1}{2} - ir) h (r) dr. 
\end{align*}

We remark that \eqref{art11-eq2.31} can be proved directly, without recourse to the spectral decomposition formula \eqref{art11-eq2.12}: Using the formulas for the Selberg transform and Mellin inversion, one can check directly that the expression on the right-hand side of \eqref{art11-eq2.31} has constant term zero with respect to both variables and hence (using the estimate \eqref{art11-eq2.28}) is a cusp form; equation \eqref{art11-eq2.29} then implies the desired identity. We leave the details as an exercise for the reader. 

\section{Computation of $I(s)$ for $\Re (s) >1$.}\label{art11-sec3}
Let $h(r)$ be a function satisfying \eqref{art11-eq1.14} and define 

%%% 322




%%%%%%%%%%%%%%%%%%%%%%
%\medskip
%\noindent
%{\bfseries Theorem \thnum{1}:\label{art11-thm1}}


