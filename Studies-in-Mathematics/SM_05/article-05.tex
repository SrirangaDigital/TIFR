\chapter{DIRICHLET SERIES FOR THE GROUP $\GL(N)$.}\label{art-5}

\begin{center}
{\large By~ Herve Jacquet}
\end{center}

\bigskip

\setcounter{pageoriginal}{154}
\section{Introduction}\label{art5-sec1}
\pageoriginale
Suppose $\varphi$ is a modular cusp form with Fourier expansion:
\begin{equation*}
\varphi(z)=\sum\limits_{n\geq 1}a_{n}\exp(2i\ \pi n z).\tag{1.1}\label{art5-eq1.1}
\end{equation*}
The Mellin transform of $\varphi$ is the integral
\begin{equation*}
\int\limits^{+\infty}_{0}\varphi(iy)y^{s-1}dy.\tag{1.2}\label{art5-eq1.2}
\end{equation*}
If we replace $\varphi$ by its Fourier expansion then we see that \eqref{art5-eq1.2} is equal to
\begin{equation*}
\sum\limits_{n\geq 1}a_{n}n^{-s}\int\limits^{+\infty}_{0}\exp (-2\pi y)y^{s-1}dy.\tag{1.3}\label{art5-eq1.3}
\end{equation*}
Since
\begin{equation*}
\int\limits_{0}^{+\infty}\exp(-2\pi y)y^{s-1}dy=(2\pi)^{-s}\Gamma(s).\tag{1.4}\label{art5-eq1.4}
\end{equation*}
this integral representation gives the analytic continuation of the series
\begin{equation*}
\sum\limits_{n\geq 1} a_{n}n^{-s},\tag{1.5}\label{art5-eq1.5}
\end{equation*}
as a meromorphic function of $s$ in the whole complex plane. Furthermore it shows that the analytic continuation satisfies a simple functional equation. Finally if $\varphi$ is an eigen function of the Hecke operators, then the series \eqref{art5-eq1.5} has an infinite euler product.

If $\varphi'$ is another form then one can also consider the ``convolution'' of the Dirichlet series attached to $\varphi$ and $\varphi'$, namely the Dirichlet series
\begin{equation*}
\sum\limits_{n\geq 1} \frac{a_{n}a'_{n}}{n^{s}}\tag{1.6}\label{art5-eq1.6}
\end{equation*}
It\pageoriginale has a simple integral representation and analytic properties similar to that of \eqref{art5-eq1.5}. Furthermore, if both $\varphi$ and $\varphi'$ are eigen functions of the Hecke operators then it has an Euler product.

Classically, it is just as easy to pursue the theory for other types of forms: holomorphic forms for congruence sub-groups, Maass forms, Hilbert modular forms... The theory can also be generalied to the groups $\GL(r)$ with $r>2$. It is still incomplete but, as an introduction, we shall discuss the case of the ``Maass forms'' for the group
\begin{equation*}
\Gamma_{r}=\GL(r,\mathbb{Z}),\tag{1.7}\label{art5-eq1.7}
\end{equation*}
also noted simply $\Gamma$. Naturally the discussion of the most general case would entail the use of ad\`eles and group representations.

This report is based directly on the work of the authors of [J-S-$P$ 1,2,3,]. That work in turn owes much to the results and ideas, published or not, of the authors of \cite{art5-G-K}.

\section{Maass forms}\label{art5-sec2}
Let $\varphi$ be a function on
\begin{equation*}
G_{r}=\GL(r,R),\tag{2.1}\label{art5-eq2.1}
\end{equation*}
invariant on the left under $\Gamma_{r}$, on the right under the orthogonal group, and, on both sides, under the center $Z_{r}$ of $G_{r}$. The function $\varphi$ will be said to be a cusp form if it satisfies some additional conditions that we now describe. It will be assumed to be $C^{\infty}$ and an eigen function of the algebra $\sfZ$ of bi-invariant differential differential operators. The corresponding algebra morphism from $\sfZ$ to $C$ will be denoted by $\lambda$. We will also assume $\varphi$ cuspidal. This means that for every group of the form
\begin{equation*}
V=
\left\{
\left(
\begin{matrix}
I_{r_{1}} & & & & \ast \\
  & \ddots & & & \\
  &  & I_{r_{2}} & & \\
  &  & & \ddots &\\
0  & & & & I_{r_{s}}\\
\end{matrix}
\right)
\right\}\tag{2.2}\label{art5-eq2.2}
\end{equation*}
the ``constant term of $\varphi$ along $V$'', that is, the integral
\begin{equation*}
\int\limits_{\Gamma\cap V\backslash V}\varphi(ug)du,\tag{2.3}\label{art5-eq2.3}
\end{equation*}
vanishes for all $g$. It is perhaps unnecessary to recall that $V\cap \Gamma$ is a discrete cocompact subgroup of $V$.

There\pageoriginale is also a condition of growth at infinity which, because we are considering only cuspidal functions, amounts to demand that $\varphi$ be square integrable on the quotient $Z_{r}\Gamma\backslash G_{r}$. Actually, for a given $\lambda$, the functions $\varphi$ satisfying the above conditions make up a finite dimensional Hilbert space $V_{\lambda}$. It is invariant under the action of the Hecke algebra; the corresponding algebra of operators on $V_{\lambda}$ is diagonalizable and so we may, and do, demand that our forms be eigen vectors of the Hecke algebra.

\section{Fourier expansions}\label{art5-eq3}

Let $N_{r}$ be the group of upper triangular matrices with unit diagonal. For every $(r-1)$-tuple of non zero integers $(n_{1},n_{2},\ldots,n_{r-1})$ define a character $\theta_{n_{1},n_{2},\ldots,n_{r-1}}$ of $N_{r}$ by
\begin{equation*}
\theta_{n_{1},n_{2},\ldots,n_{r-1}}(x)=\prod\limits_{1\leq j\leq r-1}\exp (2i\pi n_{j}x_{j.j+1}).\tag{3.1}\label{art5-eq3.1}
\end{equation*}
It is clearly trivial on $N_{r}\cap \Gamma$. Set
\begin{equation*}
\varphi_{n_{2},n_{2},\ldots,n_{r-1}}(g)=\int\limits_{N_{r}\cap \Gamma\backslash N_{r}}\varphi(ug)\overline{\theta}(u)du\tag{3.2}\label{art5-eq3.2}
\end{equation*}
where $\theta$ stands for $\theta_{n_{1},n_{1},\ldots,n_{r-1}}$. Then $\varphi$ has the following expansion:
\begin{equation*}
\varphi(g)=\sum \varphi_{n_{1},n_{2},\ldots,n_{r-1}}\left[\left(\begin{matrix} \gamma & 0\\ 0 & 1\end{matrix}\right)g\right]\tag{3.3}\label{art5-eq3.3}
\end{equation*}
where we sum for all $(r-1)$-tuples with $n_{i}\geq 1$ and $\gamma$ in a set of representatives for $N_{r-1}\cap \Gamma_{r-1}\backslash \Gamma_{r-1}$. Actually we will need to introduce also, for $0\leq j\leq r-1$, the subgroup $V^{j}_{r}$ of matrices $u\in N_{r}$ of the form
$$
u=\left(\begin{matrix} 1_{n-j} & *\\
0 & *
\end{matrix}
\right).
$$
For $j=r-1$ this is the group $N_{r}$ itself. We will set:
$$
\varphi_{n_{r-j},n_{r-j+1},\ldots,n_{r-1}}(g)=\int\limits_{\Gamma\cap V^{j}_{r}\backslash V^{j}_{r}}\varphi(ug)\overline{\theta}(u)du
$$
where $\theta=\theta_{n_{1},n_{2},\ldots,n_{r-1}}$; the right hand side does not depend on $n_{1}$, $n_{2},\ldots,n_{r-j-1}$ which justifies the notation. Then we have the more general expansion:
\begin{equation*}
\varphi_{n_{r-j+1},\ldots,n_{r-1}}(g)=\sum \varphi_{n_{1},n_{2},\ldots,n_{r-1}}\left[\left(\begin{matrix} \gamma & 0\\ 0 & 1_{j}\end{matrix}\right)g\right]
\end{equation*}
where We sum for all $r-j$ tuples $(n_{1},n_{2},\ldots,n_{r-j})$ with $n_{i}\geq 1$ and all $\gamma$ in a set of representatives for $N_{r-j}\cap \Gamma_{r-j}\backslash \Gamma_{r-j}$.

It\pageoriginale is not simple to explain the ideas involved in these expansions. We will point out however that our assertions are a mere reformulation of the expansions given in \cite{art5-P1} or \cite{art5-Sha}.

So far our assertions do not depend on the assumption that $\varphi$ be an eigen function of the Hecke algebra. If this assumption is taken in account, then it is found that
\begin{equation*}
\varphi_{n_{1},n_{2},\ldots,n_{r-1}}(g)=a_{n_{1},n_{2},\ldots,n_{r-1}}W(\zeta g)\tag{3.5}\label{art5-eq3.5}
\end{equation*}
where we have denoted by $W$ the function
\begin{equation*}
\underbrace{\varphi_{1,1,\ldots,1}}_{r-1}(g)\tag{3.6}\label{art5-eq3.6}
\end{equation*}
and by $\zeta$ the diagonal matrix
\begin{equation*}
\diag (n_{1} \ n_{2}\ldots n_{r-1}, n_{2}\ldots n_{r-1},\ldots,n_{r-1},1).\tag{3.7}\label{art5-eq3.7}
\end{equation*}
The constants $a_{n_{1},n_{2},\ldots,n_{r-1}}$ which appear can be computed solely in terms of the homomorphisms of the Hecke algebra into $C$ determined by $\varphi$. The reader will note that both sides of (3-5) transform on the left under the character $\theta_{n_{1},n_{2},\ldots,n_{r-1}}$ of the group $N_{r}$. As for $W$, within a scalar factor, it is determined solely by the morphism $\lambda$ of $\sfZ$ into $C$. Again, our assertions are mere reformulation of the results of \cite{art5-C-S}, \cite{art5-Sha}, \cite{art5-Shi}.

\section{The Mellin Transform}\label{art5-sec4}

Let us first simplify our notations. For $0\leq j\leq r-1$ we set
\begin{equation*}
\varphi^{j}=\underbrace{\varphi_{1,1,\ldots,1}}\tag{4.1}\label{art5-eq4.1}
\end{equation*}
so that $\varphi^{0}=\varphi$ and $\varphi^{r-1}=W$. We set also, for $1\leq j\leq r-1$,
$$
a_{n_{1},n_{2},\ldots,n_{j}}=a_{n_{1},n_{2},\ldots,n_{j}},\underbrace{1,1,\ldots,1}_{r-j-1}.
$$
Combining \eqref{art5-eq3.4} [with $j=r-1$] with \eqref{art5-eq3.5} we get
\begin{equation*}
\varphi^{r-2}(g)=\sum\limits_{n\geq 1, \epsilon=\pm 1}a_{n}W
\left[
\left(
\begin{matrix}
n\epsilon & 0\\
0 & 1_{r-1}
\end{matrix}
\right)g\right].\tag{4.2}\label{art5-eq4.2}
\end{equation*}
In view of this formula it is entirely reasonable to define the Mellin transform\pageoriginale of $\varphi$ to be the integral
\begin{equation*}
\int\limits_{R^{\times}/\{\pm 1\}}\varphi^{r-2}
\left(
\begin{matrix}
a & 0\\
0 & 1_{r-1}
\end{matrix}
\right)|a|^{s-1}da.\tag{4.3}\label{art5-eq4.3}
\end{equation*}
It is equal to
\begin{equation*}
\sum\limits_{n\geq 1}a_{n}n^{-s}\int\limits_{R^{\times}}W
\left(
\begin{matrix}
a & 0\\
0 & 1_{r-1}
\end{matrix}
\right)
|a|^{s-1}da.\tag{4.4}\label{art5-eq4.4}
\end{equation*}
If we knew that the integral in \eqref{art5-eq4.4} were a product of $\Gamma$ -factors--as it should be--then the previous computation would give the analytic continuation of the Dirichlet Series
\begin{equation*}
\sum\limits_{n\geq 1}a_{n}n^{-s}.\tag{4.5}\label{art5-eq4.5}
\end{equation*}

On the other hand, just as in the case $r=2$, the Dirichlet Series has an infinite Euler product:
$$
\sum\limits_{n\geq 1}a_{n}n^{-s+\frac{1}{2}(r-1)}=\prod\limits_{p}\det (1-p^{-s}X_{p})^{-1},
$$
where $X_{p}$ is a semi-simple conjugacy class in $\GL(r,C)$.

\section{The convolution}\label{art5-sec5}

The convolution \eqref{art5-eq1.6} also generalizes. Namely let $\varphi'$ be another cusp-form on $G_{r}$, with $r'\leq r$. Let us denote with a prime the objects attached to $\varphi$.

Suppose first $r'\leq r-1$. Consider the integral
\begin{equation*}
\int\limits_{\Gamma_{r'}\backslash G_{r'}}\varphi^{r-1-r'}
\left(
\begin{matrix}
g & 0\\
0 & 1_{r-r'}
\end{matrix}
\right)
\varphi'(g)|\det g|^{s}d^{\times}g,\tag{5.1}\label{art5-eq5.1}
\end{equation*}
where $d^{\times}g$ is an invariant measure on the quotient $\Gamma_{r'}\backslash G_{r'}$.

Combining \eqref{art5-eq3.4} with \eqref{art5-eq3.5} we have the following expansion:
\begin{equation*}
\varphi^{r-1-r'}(g)=\sum a_{n_{1},n_{2},\ldots,n_{r'}}W
\left[
\left(
\begin{matrix}
\gamma & 0\\
0 & 1_{r-r'}
\end{matrix}
\right)
g
\right].\tag{5.2}\label{art5-eq5.2}
\end{equation*}
Replacing $\varphi^{r-1-r'}$ by this expression in \eqref{art5-eq5.1} we get, after a ``few'' formal manipulations,
\begin{gather*}
\sum\limits_{n_{1}\geq 1,n_{2}\geq 1,\ldots,n_{r}\geq 1}a_{n_{1},n_{2},\ldots,n_{r'}}a'_{n_{1},n_{2},\ldots,n_{r'-1}}|n_{1}n^{2}_{2}\ldots n^{r'}_{r'}|^{-s}\tag{5.3}\label{art5-eq5.3}\\[3pt]
\int\limits_{N_{r'}\backslash G_{r'}}W
\left[
\left(
\begin{matrix}
g & 0\\
0 & 1_{r-r'}
\end{matrix}
\right)
\right]
W'[\epsilon g]|\det g|^{s}d^{\times}g,
\end{gather*}
where\pageoriginale $d^{\times}g$ is now an invariant measure on the quotient $N_{r'},\backslash G_{r'}$, and $\epsilon$ is the $r'$ by $r'$ diagonal matrix
$$
\diag (-1,1,-1,\ldots).
$$

The multiple series which appears in \eqref{art5-eq5.3} may be regarded as a Dirichlet series in the usual sense. Again if we knew that the integral in \eqref{art5-eq5.3} were a product of $\Gamma$-factors, our computations would give the analytic continuation of this series.

Just as in the previous case, the series has an Euler product:
\begin{gather*}
\sum a_{n_{1},n_{2},\ldots,n_{r'}}a'_{n_{1},n_{2},\ldots,n_{r'-1}}|n_{1}n_{2}\ldots n_{r'}|^{s-\frac{1}{2}(r-r')}\tag{5.4}\label{art5-eq5.4}\\
=\prod\limits_{p}\det (1-p^{-s}X_{p}\otimes X'_{p})^{-1}.
\end{gather*}

When $r=r'$, the previous construction needs to be modified. We denote by $\Phi$ the Schwartz-function on the space of row matrices with $r$ entries defined by
\begin{equation*}
\Phi (x)=\exp (-\pi x\cdot {}^{t}x)\tag{5.5}\label{art5-eq5.5}
\end{equation*}
and we introduce an ``Epstein zeta function'':
\begin{equation*}
E(g,s)=\sum\limits_{\xi\in Z^{r}-\{0\}}\int\limits_{-\infty}^{+\infty}\Phi (t\xi g)|t|^{rs-1}dt|\det g|^{s}\tag{5.6}\label{art5-eq5.6}
\end{equation*}
[Here $\xi g$ is the product of the row matrix $\xi$ by the square matrix $g$; $t$ is a scalar]. It can also be written as an ``Eisenstein series'':
\begin{equation*}
E(g,s)=\zeta(rs)\sum\limits_{\gamma\in \Gamma\cap P_{r}\backslash \Gamma} \int \Phi [(0,0,\ldots,0,t)\gamma g]|t|^{rs-1}\dt|\det g|^{s},\tag{5.7}\label{art5-eq5.7}
\end{equation*}
where $P_{r}$ is the standard parabolic subgroup of type $(r-1,1)$.

Then, instead of \eqref{art5-eq5.2}, we have to consider the integral
\begin{equation*}
\int\limits_{Z_{r}\Gamma\backslash G_{r}}\varphi(g)\varphi'(g)E(g,s)d^{\times}g,\tag{5.8}\label{art5-eq5.8}
\end{equation*}
where $d^{\times}g$ is an invariant measure on the quotient $Z_{r}\Gamma\backslash G_{r}$. It turns out to be equal to
\begin{gather*}
\zeta(rs)\sum a_{n_{1},n_{2},\ldots,n_{r-1}}a'_{n_{1},n_{2},\ldots,n_{r-1}}|n_{1}n^{2}_{2}\ldots n_{r-1}^{r-1}|^{-s}\tag{5.9}\label{art5-eq5.9}\\
\int\limits_{N_{r}\backslash G_{r}}W(g)W'(\epsilon g)\Phi[(0,0,\ldots,0,1)g]|\det g|^{s}d^{\times} g.
\end{gather*}
Moreover:\pageoriginale
\begin{gather*}
\zeta(rs)\sum a_{n_{1},n_{2},\ldots,n_{r-1}}a'_{n_{1},n_{2},\ldots,n_{r-1}}|n_{1}n^{2}_{2}\ldots n_{r-1}^{r-1}|^{-s}\tag{5.10}\label{art5-eq5.10}\\
=\prod\limits_{p}\det (1-p^{-s}X_{p}\otimes X'_{p})^{-1}.
\end{gather*}

\medskip
\noindent
{\bf Remark \thnum{5.11}.\label{art5-thm5.11}}~If we take $r=1$ then $\varphi=\varphi'=\varphi_{0}$, the constant function equal to one on $G_{1}=R^{\times}$; moreover $X_{p}=X_{p'}=1$, and \eqref{art5-eq5.10} reduces to the Euler product for the $\zeta$-function. Similarly, we may regard the theory of \S\ref{art5-sec4} as a special case of the theory of \S\ref{art5-sec5} where $r'=1$ and $\varphi'=\varphi_{0}$. This remark will be used without further warning.

\section{Functional Equations}\label{art5-sec6}

We have already pointed out that we do not have enough information on the integrals of \eqref{art5-eq4.4}, \eqref{art5-eq5.3}, and \eqref{art5-eq5.9}. If we assume the missing information then we can address ourselves to the question of the functional equation satisfied by these Euler products. The functional equation should state that the analytic continuation of 
$$
\prod\limits_{p}(1-p^{-s}X_{p}\otimes X'_{p})^{-1},
$$
times the appropriate $\Gamma$-factor, is equal to the analytic continuation of 
$$
\prod\limits_{p}(1-p^{-1+s}X^{-1}_{p}\otimes {X'}^{-1}_{p})^{-1},
$$
times the appropriate $\Gamma$-factor.

To see this we introduce the function
$$
\widetilde{\varphi}(g)=\varphi({}^{t}g^{-1}).
$$
It is also a Maass cusp form. We denote by a tilda the objects attached to $\widetilde{\varphi}$. Then:
\begin{gather*}
\widetilde{W}(g)=W(w_{r}{}^{t}g^{-1}),\quad\text{where}\quad w_{r}=
\left(
\begin{matrix}
0 & & & -1\\
 & & 1 &\\
 & -1 & & \\
\iddots & & & 0
\end{matrix}
\right),\\
\widetilde{a}_{n_{1},n_{2},\ldots,n_{r-1}}=a_{n_{r-1},n_{r-2}\ldots n_{t}},  \widetilde{X}_{p}=X^{-1}_{p}.
\end{gather*}

If $r=r'$ our starting point is the functional equation of the Epstein zeta-function:
$$
E(g,s)=E({}^{t}g^{-1},1-s);
$$
from which we get
$$
\int \varphi(g)\varphi'(g)E(g,s)dg=\int \widetilde{\varphi}(g)\widetilde{\varphi}'(g)E(g,1-s)dg.
$$
The\pageoriginale functional equation follows readily.

If $r'=r-1$ then $\varphi^{r-1-r'}$ is just $\varphi$. Clearly
$$
\int \varphi
\left(
\begin{matrix}
g & 0\\
0 & 1
\end{matrix}
\right)
\varphi'(g)|\det g|^{s-\frac{1}{2}}d^{\times}g=\int\widetilde{\varphi}
\left(
\begin{matrix}
g & 0\\
0 & 1
\end{matrix}
\right)
\widetilde{\varphi}'(g)|\det g|^{\frac{1}{2}-s}d^{\times}g
$$
and again the functional equation follows readily.

However if $r'\leq r-2$ (which includes the case $r'=1$) we have to take in account a somewhat unexpected relation between $\widetilde{\varphi}^{r-r'-1}$ and $\varphi^{r-r'-1}$. Namely
$$
\int \varphi^{r-r'-1}\left[\left(
\begin{matrix}
1_{r'} & 0 & 0\\
x & 1_{r-r'-1} & 0\\
0 & 0 & 1
\end{matrix}
\right){}^{t}g^{-1}\right]\dx
$$
is actually a left-translate of $\widetilde{\varphi}^{r-r'-1}(g)$; the integral is on the full space of matrices with $r'$ columns and $r-r'-1$ rows. Rather than trying to explain the details, we refer the reader to \cite{art5-J-S-P1} where the case $r'=1$, $r=3$ is discussed.

\begin{thebibliography}{99}
\bibitem[C-S]{art5-C-S} \textsc{Casselman} W. and J. Shalika, Unramified Whittaker functions, to appear.

\bibitem[G-K]{art5-G-K} \textsc{Gelfand} J. M. and D. A. \textsc{Kazdan}, Representations of G1 $(n, K)$ where $K$ is a local field, in {\em Lie groups and their representations,} John Wiley \& Sons (1975), 95--118.

\bibitem[J-S1]{art5-J-S1} \textsc{Jacquet} H. and J. \textsc{Shalika}, Hecke theory for $\GL(3)$, {\em Comp. Math.,} 29:1 (1974), 75--87.

\bibitem[J-S2]{art5-J-S2} \textsc{Jacquet} H. and J. \textsc{Shalika}, Comparaison des representations automorphes du groupe line aire, {\em C.R. Acad. Sc. Paris.} 284 (1977), 741--744.

\bibitem[J-S-P1]{art5-J-S-P1} \textsc{Jacquet} H., J. \textsc{Shalika}, and J. J. \textsc{Piatetski-Shaprio}, Automorphic forms on $\GL(3)$, I and II {\em Annals of Math,} 109 (1979).

\bibitem[J-S-P2]{art5-J-S-P2} \textsc{Jacquet} H., J. \textsc{Shalika}, and J. J. \textsc{Piatetski Shapiro}, Facteurs L et $\epsilon$ du groupe lineaire, to appear in {\em C.R. Acad. Sci.} (1979), Paris.

\bibitem[J-S-P3]{art5-J-S-P3} \textsc{Jacquet}\pageoriginale H., J. \textsc{Shalika}, and J. J. \textsc{Piatetski-Shapiro}, {\em Constructions of cusp forms on} $\GL(n)$, Univ. of Maryland, Lectures Notes in Math. 16(1975).

\bibitem[P1]{art5-P1} \textsc{Piatetski-Shapiro} J.J., {\em Euler subgroups}, in Lie groups and their representations, John Wiley and Sons (1975), 597--620.

\bibitem[P2]{art5-P2} \textsc{Piatetski-Shapiro} J.J., {\em Zeta functions on} $\GL(n)$, Mimeographed notes, Univ. of Maryland.

\bibitem[Sha]{art5-Sha} \textsc{Shalika J.,} The multiplicity one theorem for $\GL(n)$, {\em Annals of Math}. 100 (1974), 171--193.

\bibitem[Shi]{art5-Shi} \textsc{Shintani} T., On an explicit formula for class-1 ``Whittaker functions'' on GL over $p$-adic fields, {\em Proc. Japan. Acad.} 52 (1976), 180--182.
\end{thebibliography}


