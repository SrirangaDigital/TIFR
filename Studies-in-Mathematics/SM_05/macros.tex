\usepackage{graphicx,xspace,fancybox}
\usepackage{fancyhdr}
\usepackage{color}

\newcounter{pageoriginal}


\usepackage[papersize={140mm,215mm},textwidth=110mm,
textheight=170mm,headheight=6mm,headsep=4mm,topmargin=17.5mm,botmargin=1.15cm,
leftmargin=15mm,rightmargin=15mm,footskip=0.6cm]{zwpagelayout}

\marginparwidth=10pt
\marginparsep=10pt
\marginparpush=5pt
%\renewcommand{\thepageoriginal}{\arabic{pageoriginal}}
\newcommand{\pageoriginale}{\refstepcounter{pageoriginal}\marginpar{\footnotesize\xspace\textbf{\thepageoriginal}}
} 
\let\pageoriginaled\pageoriginale

\setcounter{secnumdepth}{3}

\newtheorem{proposition}{Proposition}
\newtheorem{corollary}{Corollary}       
\newtheorem{lem}{Lemma}
\newtheorem{lemma}{LEMMA}[section]
\newtheorem{sublemma}{Lemma}[subsection]
\newtheorem{thm}{Theorem}
\newtheorem{theorem}{Theorem}[section]
\newtheorem{prop}{Proposition}
\newtheorem{definition}{Definition}[section]
\newtheorem{subdefin}{Definition}[subsection]
\newtheorem{cor}{Corollary}
\newtheorem{coro}{Corollary}
\newtheorem{subcoro}{Corollary}[subsection]
\newtheorem{defn}{Definition}
\newtheorem{gausslemma}{Gauss Lemma}[section]
\newtheorem{witt's thm}{Witt's theorem}
\newtheorem{Theorem}{THEOREM}[section]



\newtheoremstyle{remark}{10pt}{10pt}{ }%
{}{\bfseries}{.}{ }{}
\theoremstyle{remark}
\newtheorem{remark}{Remark}
\newtheorem{remarks}{Remarks}[section]
\newtheorem{case}{Case}
\newtheorem{subremarks}{Remarks}[subsection]
\newtheorem{subremark}{Remark}[subsection]
\newtheorem{example}{Example}[section]
\newtheorem{subexample}{Example}[subsection]
\newtheorem{exam}{Example}
\newtheorem{Definition}{Definition}
\newtheorem{thmm}{Theorem}
\newtheorem{examples}{Examples}[section]
\newtheorem{Note}{Note}
\newtheorem{note}{Note}[section]
\newtheorem{notation}{Notation}[section]
\newtheorem{claim}{Claim}[section]
\newtheorem{application}{Application}[section]
\newtheorem{proofoflem}{Proof of Lemma}
\newtheorem{proofofsublemma}{Proof of Lemma}[subsection]
\newtheorem{proofofprop}{Proof of Proposition}[section]
\newtheorem{proofoftheorem}{Proof of theorem}[subsection]
\newtheorem{proofofthm}{Proof of Theorem}
\newtheorem{proofcontd}{Proof continued}[subsection]
	

\newtheoremstyle{nonum}{}{}{\itshape}{}{\bfseries}{\kern -2pt{\bf.}}{ }{#1 \mdseries
{\bf #3}}
\theoremstyle{nonum}

\newtheorem{lemma*}{Lemma}	
\newtheorem{theorem*}{Theorem}	
\newtheorem{irreducibilitythm*}{IRREDUCIBILITY THEOREM}
\newtheorem{genirreducibilitythm*}{GENERIC IRREDUCIBILITY THEOREM}
\newtheorem{embedthm*}{EMBEDDING THEOREM}
\newtheorem{prop*}{Proposition}	
\newtheorem{claim*}{Claim}
\newtheorem{defi*}{Definition}
\newtheorem{conjecture*}{CONJECTURE}
\newtheorem{coro*}{Corollary}
\newtheoremstyle{mynonum}{}{}{ }{}{\bfseries}{\kern -2pt{\bf.}}{ }{#1 \mdseries
{\bf #3}}
\theoremstyle{mynonum}
\newtheorem{remark*}{Remark}	
\newtheorem{remarks*}{Remarks}	
\newtheorem{exer*}{Exercise}	
\newtheorem{example*}{Example}	
\newtheorem{examples*}{Examples}	
\newtheorem{note*}{Note}
\newtheorem{problem}{Problem}

\def\ophi{\overset{o}{\phi}}

\def\oval#1{\text{\cornersize{2}\ovalbox{$#1$}}}


\newcommand*\mycirc[1]{%
  \tikz[baseline=(C.base)]\node[draw,circle,inner sep=.7pt](C) {#1};\:
}

\DeclareMathOperator{\Ind}{Ind}


\DeclareMathOperator{\Res}{Res}
\DeclareMathOperator{\Iim}{Im}
\DeclareMathOperator{\Hol}{Hol}
\DeclareMathOperator{\loc}{loc}
\DeclareMathOperator{\cit}{cit}

\DeclareMathOperator{\ord}{ord}

\DeclareMathOperator{\Cos}{Cos}
\DeclareMathOperator{\Sin}{Sin}
\DeclareMathOperator{\map}{map}
\DeclareMathOperator{\Min}{Min}
\DeclareMathOperator{\Max}{Max}
\DeclareMathOperator{\dom}{dom}
\DeclareMathOperator{\supp}{supp}
\DeclareMathOperator{\im}{im}
\DeclareMathOperator{\id}{id}
\DeclareMathOperator{\Id}{Id}
\DeclareMathOperator{\df}{df}
\DeclareMathOperator{\pr}{pr}
\DeclareMathOperator{\Df}{Df}
\DeclareMathOperator{\Dg}{Dg}
\DeclareMathOperator{\Ad}{Ad}
\DeclareMathOperator{\Sp}{Sp}
\DeclareMathOperator{\sh}{sh}
\DeclareMathOperator{\dt}{dt}
\DeclareMathOperator{\hyp}{hyp}
\DeclareMathOperator{\Hyp}{Hyp}
\DeclareMathOperator{\can}{can}
\DeclareMathOperator{\Ext}{Ext}
\DeclareMathOperator{\red}{red}
\DeclareMathOperator{\rank}{rank}
\DeclareMathOperator{\codim}{codim}
\DeclareMathOperator{\grad}{grad}
\DeclareMathOperator{\Spin}{Spin}
\DeclareMathOperator{\Spec}{Spec}
\DeclareMathOperator{\Ker}{Ker}
\DeclareMathOperator{\ring}{ring}
\DeclareMathOperator{\Vol}{Vol}
\DeclareMathOperator{\vol}{vol}
\DeclareMathOperator{\Int}{Int}
\DeclareMathOperator{\End}{End}
\DeclareMathOperator{\Ric}{Ric}
\DeclareMathOperator{\Trace}{Trace}
\DeclareMathOperator{\Char}{char}
\DeclareMathOperator{\mult}{mult}
\DeclareMathOperator{\Supp}{Supp}
\DeclareMathOperator{\Sup}{Sup}
\DeclareMathOperator{\Aut}{Aut}
\DeclareMathOperator{\abs}{abs}
\DeclareMathOperator{\Pic}{Pic}
\DeclareMathOperator{\Cl}{Cl}
\DeclareMathOperator{\ca}{ca}
\DeclareMathOperator{\cu}{cu}
\DeclareMathOperator{\idd}{id.}
\DeclareMathOperator{\transdeg}{trans{.}deg}


\def\uub#1{\underline{\underline{#1}}}
\def\ub#1{\underline{#1}}
\def\os#1{\overset{#1}}
\def\us#1{\underset{#1}}
\def\ob#1{\overbrace{#1}}
\def\ool#1{\overline{\overline{#1}}}
\def\ol#1{\overline{#1}}
\def\set#1{\left\{{#1}\right\}}
\def\oset#1{\left({#1}\right)}
\def\cset#1{\left[{#1}\right]}
\def\mset#1{\left|{#1}\right|}
\def\aset#1{\left<{#1}\right>}


\font\bigsymb=cmsy10 at 4pt
\def\bigdot{{\kern1.2pt\raise 1.5pt\hbox{\bigsymb\char15}}}
\def\overdot#1{\overset{\bigdot}{#1}}

\makeatletter
\renewcommand\subsection{\@startsection{subsection}{2}{\z@}%
                                     {-3.25ex\@plus -1ex \@minus -.2ex}%
                                     {1.5ex \@plus .2ex}%
                                     {\normalfont}}%

\renewcommand\thesection{\@arabic\c@section}
%\renewcommand\thesubsection{({\thechapter.\thesection.\@arabic\c@subsection})}

\renewcommand{\@seccntformat}[1]{{\csname the#1\endcsname}\hspace{0.3em}}
\makeatother

\def\fibreproduct#1#2#3{#1{\displaystyle\mathop{\times}_{#3}}#2}
\let\fprod\fibreproduct

\def\fibreoproduct#1#2#3{#1{\displaystyle\mathop{\otimes}_{#3}}#2}
\let\foprod\fibreoproduct


\def\cf{{cf.}\kern.3em}
\def\Cf{{Cf.}\kern.3em}
\def\eg{{e.g.}\kern.3em}
\def\ie{{i.e.}\kern.3em}
\def\iec{{i.e.,}\kern.3em}
\def\idc{{id.,}\kern.3em}
\def\resp{{resp.}\kern.3em}
\def\mod{{\rm{mod}}\kern.3em}

\def\GL{\mathrm{GL}}
\def\SL{\mathrm{SL}}
\def\bA{\mathbf{A}}
\def\bB{\mathbf{B}}
\def\bC{\mathbf{C}}
\def\bD{\mathbf{D}}
\def\bE{\mathbf{E}}
\def\bF{\mathbf{F}}
\def\bG{\mathbf{G}}
\def\bH{\mathbf{H}}
\def\bI{\mathbf{I}}
\def\bJ{\mathbf{J}}
\def\bK{\mathbf{K}}
\def\bL{\mathbf{L}}
\def\bM{\mathbf{M}}
\def\bN{\mathbf{N}}
\def\bO{\mathbf{O}}
\def\bP{\mathbf{P}}
\def\bQ{\mathbf{Q}}
\def\bR{\mathbf{R}}
\def\bS{\mathbf{S}}
\def\bT{\mathbf{T}}
\def\bU{\mathbf{U}}
\def\bV{\mathbf{V}}
\def\bW{\mathbf{W}}
\def\bX{\mathbf{X}}
\def\bY{\mathbf{Y}}
\def\bZ{\mathbf{Z}}
\def\bp{\mathbf{p}}
\def\br{\mathbf{r}}


\def\sB{\mathscr{B}}
\def\sF{\mathscr{F}}
\def\sG{\mathscr{G}}
\def\sN{\mathscr{N}}
\def\sO{\mathscr{O}}
\def\sS{\mathscr{S}}
\def\sR{\mathscr{R}}
\def\sY{\mathscr{Y}}
\def\sE{\mathscr{E}}

\def\bbF{\mathbb{F}}
\def\bbN{\mathbb{N}}
\def\bbQ{\mathbb{Q}}
\def\bbR{\mathbb{R}}
\def\bbW{\mathbb{W}}
\def\bbZ{\mathbb{Z}}

\def\fa{\mathfrak{a}}
\def\fc{\mathfrak{c}}
\def\fo{\mathfrak{o}}
\def\fg{\mathfrak{g}}
\def\fk{\mathfrak{k}}
\def\fp{\mathfrak{p}}

\def\fC{\mathfrak{C}}
\def\fT{\mathfrak{T}}
\def\fI{\mathfrak{I}}
\def\fR{\mathfrak{R}}
\def\fA{\mathfrak{A}}


\makeatletter
%\renewcommand\chaptermark[1]{\markboth{\thechapter. #1}{}}

\def\cleardoublepage{\clearpage\if@twoside \ifodd\c@page\else
    \thispagestyle{empty}\hbox{}\newpage\if@twocolumn\hbox{}\newpage\fi\fi\fi}

%\renewcommand\tableofcontents{%
%    \if@twocolumn
%      \@restonecoltrue\onecolumn
%    \else
%      \@restonecolfalse
%    \fi
%    \chapter*{\contentsname
%        \@mkboth{%
%           \contentsname}{\contentsname}}%
%    \@starttoc{toc}%
%    \if@restonecol\twocolumn\fi
%    }
\makeatother



\renewcommand{\headrulewidth}{0pt}
\pagestyle{fancy}

%\lhead[\small\em \thepage]{}
%\rhead[]{\small\em \thepage}
%\chead[\small\em \leftmark]{\small\em \rightmark}
%\cfoot[]{}
%\rfoot[\href{../toc.pdf}{\footnotesize\color{red}{\bf\em Table of Contents}}]{\href{../toc.pdf}{\footnotesize\color{red}{\bf\em Table of Contents}}}

\lhead[\small\em \thepage]{}
\rhead[]{\small\em \thepage}
\chead[\small\em \leftmark]{\small\em \rightmark}
\cfoot[]{}


\makeatletter
%\renewcommand\subsection{\@startsection{subsection}{2}{\z@}%
%                                     {-3.25ex\@plus -1ex \@minus -.2ex}%
%                                     {1.5ex \@plus .2ex}%
%                                     {\bf}}%
%\renewcommand\section{\@startsection {section}{1}{0mm}%
%                                   {3.25ex \@plus 0.5ex \@minus .2ex}%
%                                   {-0.8ex \@plus.2ex}%
%                                   {\normalfont\Large\bfseries}}
\renewcommand\subsection{\@startsection{subsection}{2}{\z@}%
                                     {3.25ex\@plus 1ex \@minus .2ex}%
                                     {-1.5ex}%
                                     {\bf}}%
\renewcommand\subsubsection{\@startsection{subsubsection}{3}{\z@}%
                                     {3.25ex\@plus 1ex \@minus -.2ex}%
                                     {-1.5ex \@plus .2ex}%
                                     {\normalfont\normalsize\bfseries}}

\renewcommand\thesection{\@arabic\c@section}
%\renewcommand\thesubsection{({\thechapter.\thesection.\@arabic\c@subsection})}

\renewcommand{\@seccntformat}[1]{{\csname the#1\endcsname}\hspace{0.3em}}
\makeatother
