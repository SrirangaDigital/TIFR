\chapter{APPLICATIONS TO THE THEORY OF DISTRIBUTIONS}\label{chap7}
\pageoriginale

\section[Support of a distribution. Continuable distributions]{Support of a distribution. Continuable distributions.}\label{chap7-sec1}

Let $\Omega$ be an open set in $\bfR^{n}$. We denote by $\scrD'(\Omega)$ [resp. $\scrD'_{c}(\Omega)$, ${\scrD'}^{m}(\Omega)$, ${\scrD'}^{m}_{c}(\Omega)$] the space of distributions [resp. with compact support, of order $m$, of order $m$ with compact support] in $\Omega$ (L. Schwartz \cite{L. Schwartz : 1}). It is known that $\scrD'_{c}(\Omega)$ [resp. ${\scrD'}^{m}_{c}(\Omega)$] is the dual of $\scrE(\Omega)$ [resp. $\scrE^{m}(\Omega)$] with its topology of Fr\'echet space that we have considered in Chapter \ref{chap1}.

Let $X$ be a closed subset of $\Omega$. We denote by $\scrD'(X)$ [resp. $\scrD'_{c}(X)$, ${\scrD'}^{m}(X)$, ${\scrD'}^{m}_{c}(X)$] the subspace of the corresponding space of distributions in $\Omega$ having support in $X$. Let us show that $\scrD'_{c}(X)$ [resp. ${\scrD'}^{m}_{c}(X)$] is the orthogonal of $\scrI(X;\Omega)$ (resp. $\scrI^{m}(X;\Omega)$). In fact, by the definition of support, $\scrD'_{c}(X)$ is orthogonal to the set of $f\in \scrE(\Omega)$ which are zero in a neighbourhood of $X$; on the other hand $\scrI(X;\Omega)$ is the closure in $\scrE(\Omega)$ of this set (Proposition I, 5.2). For ${\scrD'}^{m}_{c}(X)$, the same argument applies.

It follows from this that $\scrD'_{c}(X)$ [resp. ${\scrD'}^{m}_{c}(X)$] can be identified naturally with the dual of $\scrD(X)=\scrE(\Omega)/\scrI(X;\Omega)$ [resp. $\scrE^{m}(X)=\scrE^{m}(\Omega)/\scrI^{m}(X;\Omega)$].

Let now $Y$ be another closed set in $\Omega$ with $Y\subset X$. Set $\scrP'(Y;X)=\scrD'(X)/\scrD'(Y)$, $\scrP'_{0}(Y;X)=\scrD'_{c}(X)/\scrD'_{c}(Y)$. The space $\scrP'(Y;X)$ can be interpreted as the space of distributions on $\Omega-Y$, with support in $X-Y$, which can be continued to a distribution on $\Omega$ (which, then, necessarily has support in $X$) and $\scrP'_{c}(Y;X)$, can be interpreted analogously. If we consider $\scrD'_{c}(X)$ as the dual of $\scrE(X)$, then $\scrD'_{c}(Y)\subset \scrD'_{c}(X)$ is the orthogonal complement of $\scrI(Y;X)$ (same reasoning as above). Hence $\scrP'_{c}(Y;X)$ is the dual of $\scrI(Y;X)$, the latter space being equipped with the topology induced from $\scrE(X)$.

\eject

Let\pageoriginale now $X$ and $Y$ be two arbitrary closed sets in $\Omega$. Consider the sequence introduced in (I, 5.4):
\begin{equation}
0\to \scrE(X\cup Y)\xrightarrow{\delta}\scrE(X)\oplus \scrE(Y)\xrightarrow{\pi}\scrE(X\cap Y)\to 0.\label{chap7-eq1.1}
\end{equation}
The transposed sequence is
\begin{equation}
0\to \scrD'_{c}(X\cap Y)\xrightarrow{\pi^{*}}\scrD'_{c}(X)\oplus \scrD'_{c}(Y)\xrightarrow{\delta^{*}}\scrD'_{c}(X\cup Y)\to 0\label{chap7-eq1.2}
\end{equation}
where $\pi^{*}$ is (up to sign) the diagonal mapping $\pi^{*}(T)=(T,-T)$ and where $\delta^{*}(T,S)=T+S$. From the properties of \eqref{chap7-eq1.1} and the theory of duality in Fr\'echet spaces, we deduce at once that $\pi^{*}$ is injective, that $\ker \delta^{*}=\Iim \pi^{*}$ and that $\Iim \delta^{*}$ is dense in $\scrD'_{c}(X\cup Y)$. Moreover, for $\delta^{*}$ to be surjective (i.e. for \eqref{chap7-eq1.2} to be exact) it is necessary and sufficient that $\Iim \delta$ be closed, i.e. that \eqref{chap7-eq1.1} be exact. Finally, by a partition of unity, we see that the exactness of \eqref{chap7-eq1.2} is equivalent to that of the sequence:
\begin{equation}
0\to \scrD'(X\cap Y)\xrightarrow{\pi'}\scrD'(X)\oplus \scrD'(Y)\xrightarrow{\delta'}\scrD'(X\cup Y)\to 0,\label{chap7-eq1.3}
\end{equation}
$\pi'$ and $\delta'$ being defined in the same way as $\pi^{*}$ and $\delta^{*}$, and this is equivalent to the surjectivity of $\delta'$. Consequently

\setcounter{theorem}{3}
\begin{proposition}[{\L}ojasiewicz \cite{S. Lojasiewicz : 1}.]\label{chap7-prop1.4}
Under the above hypotheses the following properties are equivalent.
\begin{itemize}
\itemsep=0pt
\item[\rm(i)] $X$ and $Y$ are regularly situated.

\item[\rm(ii)] The sequence \eqref{chap7-eq1.3} is exact.

\item[\rm(iii)] The mapping $\delta'$ is surjective: in other words, every distribution $T\in \scrD'(X\cup Y)$ can be written $T=S_{1}+S_{2}$ with support $(S_{1})\subset X$, support $(S_{2})\subset Y$.
\end{itemize}
\end{proposition}

\section[Division of distributions]{Division of distributions.}\label{chap7-sec2}

The statement dual to Theorem VI, 1.1 is the following.

\begin{theorem}\label{chap7-thm2.1}
Let $\Omega$ be an open set in $\bfR^{n}$, $Y\subset X\subset \Omega$ two analytic subsets of $\Omega$, and let $f_{1},\ldots,f_{p}$ be analytic functions on $\Omega$. Let $T_{1},\ldots,T_{p}\in \scrP'(Y;X)$. Then a necessary and sufficient condition that there exist $S\in\scrP'(Y;X)$ satisfying $f_{1}S=T_{1},\ldots,f_{p}S=T_{p}$ is the following.
\end{theorem}

\thnum{(R)}.\label{chap7-R}\pageoriginale {\em For any $a\in X-Y$, analytic relations between the $f_{i}$ at $a$ are relations between the $T_{i}$, i.e. if $g_{1},\ldots,g_{p}$ are germs of analytic functions at $a$, then $g_{1}f_{1}+\cdots+g_{p}f_{p}=0$ implies that $g_{1}T_{1}+\cdots+g_{p}T_{p}=0$ near $a$.}

\begin{remark}\label{chap7-rem2.2}
Using the global theory of coherent analytic sheaves on a real analytic manifold, this condition can be replaced by the following : if $g_{1},\ldots,g_{p}$ are analytic in $\Omega$, then $g_{1}f_{1}+\cdots+g_{p}f_{p}=0$ implies that $g_{1}T_{1}+\cdots+g_{p}T_{p}=0$.
\end{remark}

\begin{example}\label{chap7-exam2.3}
Take $Y=\emptyset$, $X=\Omega$. Given $T\in \scrD'(\Omega)$ there exists $S\in\scrD'(\Omega)$ with $f_{1}S=T$. In other words, ``the division of a distribution by an analytic function is always possible''. This theorem was proved for $p=1$ (before the general case) by H\"ormander \cite{L. Hormander : 1} when $f_{1}$ is a polynomial, and {\L}ojasiewicz \cite{S. Lojasiewicz : 1}.
\end{example}

\noindent
{\bf Proof of Theorem \ref{chap7-thm2.1}}.~It suffices to prove the theorem for $\scrP'_{c}(Y;X)$ instead of $\scrP'(Y;X)$, as one sees using a partition of unity. Consider the mapping
$$
F:\scrP'_{c}(Y;X)\to [\scrP'_{c}(Y;X)]^{p}
$$
defined by $F(S)=(f_{1}S,\ldots,f_{p}S)$. We shall prove that the image of $F$ is closed and that it is dense in the set $E$ of $(T_{1},\ldots,T_{p})$ satisfying (\ref{chap7-R}); one would then have $\Iim(F)=E$. The transpose of $F$ is the mapping
$$
F^{*}:[\scrI(Y;X)]^{p}\to \scrI(Y;X)
$$
defined by $F^{*}(\phi_{1},\ldots,\phi_{p})=f_{1}\phi_{1}+\cdots+f_{p}\phi_{p}$. Theorem VI, 1.2 implies that the ideal generated in $\scrI(Y;X)$ by the $f_{i}$ is closed; hence $\Iim(F^{*})$ is closed. By transposition it follows that $\Iim(F)$ is closed.

We now prove that $\Iim(F)$ is dense in $E$. It suffices to show that for any $\phi=(\phi_{1},\ldots,\phi_{p})\in [\scrI(Y;X)]^{p}$ which is orthogonal to $\Iim(F)$, we have $f_{1}\phi_{1}+\cdots+f_{p}\phi_{p}=0$. By a partition of unity, it suffices to examine the $\phi$ with compact support in a given neighbourhood of $a$ ($a$ being any point of $\Omega$). Now, Corollary VI, 1.12 shows that we can find analytic relations $g^{(1)},\ldots,g^{(r)}$ between the $f$ in a neighbourhood of $a$ and functions $\psi_{1},\ldots,\psi_{r}\in \scrE(\Omega)$ with compact support in\pageoriginale the given neighbourhood of $a$ such that $\phi=\Sigma\psi_{j}g^{(j)}$. One deduces at once that $\phi$ is orthogonal to $E$, and the theorem follows.

The preceding theorem can be interpreted in terms of the concept of injective modules.

Let $A$ be a (unitary, commutative) ring, and $M$ a unitary $A$-module. $M$ is called injective if, for any ideal $\frakI\subset A$, the natural mapping $M\simeq \Hom_{A}(A,M)\to \Hom_{A}(\frakI,M)$ is surjective. We take a system $(f_{i})_{i\in I}$ of generators of $\frakI$ and a family $(T_{i})_{i\in I}$ of elements of $M$ such that every relation between the $f_{i}$ with coefficients in $A$ is also a relation between the $T_{i}$. Then $u:f_{i}\to T_{i}$ defines an element of $\Hom_{A}(\frakI,M)$ and conversely. To say that $M$ is injective amounts therefore to saying that in this situation, there is an $S\in M$ such that $f_{i}S=T_{i}$ for each $i$. This being the case, Theorem \ref{chap7-thm2.1}, the noetherian nature of $\scrO_{n}$ and Oka's theorem III, 4.12 [in the form (III, 4.14)] give us the

\begin{theorem}\label{chap7-thm2.4}
Let $X_{0}$, $Y_{0}$ be germs of analytic sets at $0\in \bfR^{n}$ with $Y_{0}\subset X_{0}$, and let $\scrP'(Y_{0};X_{0})$ be the space of germs induced at $0$ by $\scrP'(Y:X)$ ($Y$, $X$ being representatives of $Y_{0}$, $X_{0}$ near $0$ with $Y\subset X$). Then $\scrP'(Y_{0};X_{0})$ is an injective $\scrO_{n}$-module. In particular, the space $\scrD'_{n}$ of germs at $0$ of distributions is an injective $\scrO_{n}$-module.
\end{theorem}

\begin{remark}\label{chap7-rem2.5}
With the hypotheses of Theorem \ref{chap7-thm2.1}, let $\scrO(\Omega)$ be the ring of real valued analytic functions in $\Omega$. Then, using Remark \ref{chap7-rem2.2}, one shows easily that $\scrP'(Y;X)$ (and in particular $\scrD'(\Omega)$) is an injective $\scrO(\Omega)$-module.
\end{remark}

\section[Harmonic synthesis in $\scrS'$]{Harmonic synthesis in \protect\boldmath$\scrS'$.}\label{chap7-sec3}

We being by giving the statements dual to those given in Chapter \ref{chap2}.

\begin{proposition}\label{chap7-prop3.1}
Let $\Omega$ be an open set in $\bfR^{n}$ and $V$ a sub-$\scrE^{m}(\Omega)$-module of ${\scrD'}^{m}(\Omega)$ which is weakly closed. Then, in $V$ (with the weak topology induced from ${\scrD'}^{m}(\Omega)$) distributions with point support form a total system.

The\pageoriginale same statement is true with ${\scrD'}^{m}(\Omega)$ and $\scrE^{m}(\Omega)$ respectively replaced by $\scrD'(\Omega)$ and $\scrE(\Omega)$.
\end{proposition}

The proof, which is immediate by transposition and partition of unity, is left to the reader.

In the case of $\scrE(\Omega)$ and $\scrD'(\Omega)$, the result is true even with the strong topology, since these spaces are reflexive. We remark that, using a partition of unity [or directly, using II, 1.7], we see that these results are true if $\Omega$ is any $C^{\infty}$ manifold countable at infinity.

This being the case, let $\scrS$ be the space of $C^{\infty}$ functions on $\bfR^{n}$ which, together with derivatives of all orders, tend to zero faster than any negative power of $x^{2}_{1}+\quad+x^{2}_{n}$. Let $\bfR^{n}\to S^{n}$ be the natural mapping of $\bfR^{n}$ into the $n$-dimensional sphere ($S^{n}$ being obtained from $\bfR^{n}$ by adding a point $\infty$ at infinity). This mapping identifies $\scrS$ with $\scrI(\{\infty\};S^{n})$, and the usual topology of $\scrS$ is compatible with this isomorphism. The dual $\scrS'$ of $\scrS$ can be identified then with $\scrP'(\{\infty\};S^{n})=\scrP'_{c}(\{\infty\};S^{n})$. We look upon this space as imbedded in $\scrD'(S^{n}-\{\infty\})=\scrD'(\bfR^{n})$.

Let $V$ be a (weakly or strongly) closed sub-$\scrS$-module of $\scrS'$ (the two being equivalent since $\scrS$ is reflexive). We show that distributions with point support form a total set in $V$. Let $\widetilde{V}$ be the inverse image of $V$ in $\scrD'(S^{n})$. It is sufficient to show that $\widetilde{V}$ is closed (which is obvious) and that it is invariant under multiplication by any $f\in \scrE(S^{n})$. Now, if $f\in \scrI(\{\infty\};S^{n})$ this is true by hypothesis. If $f$ is arbitrary, we show that any $\phi\in \scrE(S^{n})$ orthogonal to $V$ is orthogonal to $f\widetilde{V}$: given such a $\phi$, it is orthogonal to $\scrD'(\{\infty\})$, hence $\phi\in \scrI(\{\infty\};S^{n})$. Hence there is a sequence $\{\alpha_{k}\}$ of functions in $\scrE(S^{n})$, zero in a neighbourhood of $\infty$, such that $f\phi=\lim \alpha_{k}f\phi$ (cf. proof of Lemma I, 4.3). Hence, for $T\in\widetilde{V}$, we have
$$
\langle fT,\phi\rangle=\langle T,f\phi\rangle=\lim \langle T,\alpha_{k}f\phi\rangle=\lim \langle (\alpha_{k}f)T,\phi\rangle=0
$$
and the result follows. [The same reasoning would apply to $\scrP'_{c}(Y;X)$, $Y\subset X$ being any closed sets of a manifold.]

\eject

By\pageoriginale the Fourier transformation, one knows that $\scrS$ is transformed into $\scrS$, $\scrS'$ into $\scrS'$, and that multiplication transforms into convolution. One deduces easily the following : If $V$ is a closed sub-$\scrS$-module of $\scrS'$, its Fourier transform $\widehat{V}$ is a vector $\bfR$-subspace of $\scrS'$ which is closed and invariant by translation, and conversely. Further, the Fourier transforms of distributions with point support are the ``exponential polynomials'', i.e. the functions $x\to P(x)e^{i\langle \lambda,x\rangle}$, where $P$ is a polynomial and $\lambda\in \bfR^{n}$. Thus one has the following result.

\begin{theorem}[Whitney-Schwartz; cf. Schwartz \cite{L. Schwartz : 2}]\label{chap7-thm3.2}
In any vector subspace of $\scrS'$ which is closed and translation invariant, exponential polynomials form a total system.
\end{theorem}

One knows, on the other hand, that this statement is false in $L^{\infty}(\bfR^{n})$ with the weak topology (Schwartz for $n\geq 3$: Malliavin for $n=1,2$). One conjectures that it is true in $\scrE(\bfR^{n})$ [it is then necessary to take ``complex'' exponential polynomials, i.e. $\lambda\in \bfC^{n}$], but, at present, this has only been proved for $n=1$ (Schwartz \cite{L. Schwartz : 3}).

\section[Partial differential equations with constant coefficients]{Partial differential equations with constant coefficients.}\label{chap7-sec4}

Let $P_{n}=\bfR[X_{1},\ldots,X_{n}]$ be the polynomial ring in $n$ indeterminates. We shall consider it, at least at the beginning of this section, as imbedded in the ring of analytic functions on $\bfR^{n}$ by the mapping $X_{j}\to x_{j}$, the $x_{j}$ being the coordinates in $\bfR^{n}$. Let $f_{1},\ldots,f_{p}\in P_{n}$ and $T_{1},\ldots,T_{p}\in \scrS'(\bfR^{n})$. We first prove the following result.

{\em There exists $S\in \scrS'$ with $f_{j}S=T_{j}$, $1\leq j\leq p$, if and only if the following condition is verified.}

(\thnum{R})\label{chap7-R1}: {\em at any point $a\in \bfR^{n}$, the analytic relations at a between the $f_{j}$ are relations between the $T_{j}$ in a neighbourhood of $a$.}

For this, consider $\bfR^{n}$ imbedded in $S^{n}$ as in \S\ref{chap7-sec3}, and let us identify $\scrS'$ with $\scrP'(\{\infty\};S^{n})$. It is enough to prove the result in the neighbourhood of any point $a$ of $S^{n}$ (partition of unity). If $a\neq \infty$ this follows from Theorem \ref{chap7-thm2.1}. If $a=\infty$, we make the change of variable $y_{j}=x_{j}/\Sigma x^{2}_{j}$, and remark that, if $m$ is large enough, $(\Sigma y^{2}_{i})^{m}f_{j}$\pageoriginale is a polynomial in $y_{1},\ldots,y_{n}$; the result follows then again from Theorem \ref{chap7-thm2.1}.

Let us remark that the condition (\ref{chap7-R1}) is equivalent to the following.

(\thnum{R$'$})\label{chap7-R1'}. {\em Relations between the $f_{i}$ with coefficients in $P_{n}$ are relations between the $T_{i}$ ({\em i.e.} $\Sigma g_{i}f_{j}=0$, $g_{j}\in P_{n}$ implies that $\Sigma g_{j}T_{j}=0$).}

In fact, if we denote by $\scrO_{a}(a\in \bfR^{n})$ the ring of germs of functions analytic at $a$, we know that $\scrO_{a}$ is flat over $P_{n}$ [III, (4.11)]. Interpreting flatness in terms of relations, we see at once that (\ref{chap7-R1'}) $\Rightarrow$ (\ref{chap7-R1}). From this and the fact that $P_{n}$ is noetherian, we deduce (arguing as in the proof of Theorem \ref{chap7-thm2.4})

\begin{theorem}\label{chap7-thm4.1}
$P$ operating on $\scrS'$ by $X_{j}T=x_{j}T$ makes of $\scrS'$ an injective $P_{n}$-module.
\end{theorem}

By the Fourier transformation, we deduce

\setcounter{theorem}{0}
\begin{theorem}\label{chap7-add-thm4.1}
If $P_{n}$ operates on $\scrS'$ by $X_{j}T=\dfrac{\partial T}{\partial x_{j}}$, $\scrS'$ is an injective $P_{n}$-module.
\end{theorem}

\begin{example}\label{chap7-exam4.2}
Let $f\in P_{n}$ and $\delta\in \scrS'$ be defined by $\langle \delta, \phi\rangle=\phi(0)$. Then there exists $E\in \scrS'$ with $f\left(\dfrac{\partial}{\partial x_{j}}\right)E=\delta$. In other words, {\em every linear differential operator with constant coefficients has a temporate fundamental solution} (i.e. one in $\scrS'$). This is mainly of historical interest (the condition $E\in \scrS'$ is artificial; see H\"ormander \cite{L. Hormander : 2} for a discussion of this question). We have, however, given this here because it was the origin of a large part of the results contained in this book.
\end{example}

\begin{thebibliography}{99}\pageoriginale
\bibitem{N. Bourbaki: 1} \textsc{N. Bourbaki} : 1. {\em Alg\`ebre commutative}, Chap. I, Hermann, Paris, 1961.

\bibitem{N. Bourbaki: 2} 2. {\em Alg\`ebre commutative}, Chap. III, Hermann, Paris, 1961.

\bibitem{J. Dieudonne and L. Schwartz : 1} \textsc{J. Dieudonn\'e and L. Schwartz :} 1. La dualit\'e dans les espaces $(F)$ et $(LF)$, {\em Ann. Inst. Fourier,} (1949), 61--101.

\bibitem{G. Glaeser: 1} \textsc{G. Glaeser} : 1. Etude de quelques alg\`ebres tayloriennes, {\em Journal d' An. Math. Jerusalem} 6 (1958), 1--124.

\bibitem{G. Glaeser: 2} 2. Fonctions compos\'ees diff\'erentiables, {\em Annals of Math.} 77 (1963), 193--209.

\bibitem{L. Hormander : 1} \textsc{L. H\"ormander} : 1. On the division of distributions by polynomials, {\em Arkiv f\"or Mat.} 3 (1958), 555--568.

\bibitem{L. Hormander : 2} 2. Local and global properties of fundamental solutions, {\em Math. Scand.} 5 (1957), 27--39.

\bibitem{C. Houzel : 1} \textsc{C. Houzel} : 1. G\'eom\'etrie analytique locale, {\em S\'eminaire H. Cartan,} 1960/61, expos\'es 18--21.

\bibitem{M. Kneser : 1} \textsc{M. Kneser} : 1Abh\"angigkeit von Funktionen, {\em Math. Zeitschrift,} 54 (1951), 34--51.

\bibitem{S. Lojasiewicz : 1} \textsc{S. {\L}ojasiewicz} : 1. Sur le probl\`eme de la division, {\em Studia Math.} 8 (1959), 87--136 (or {\em Rozprawy Matematyczne} 22 (1961)).

\bibitem{B. Malgrange : 1} \textsc{B. Malgrange} : 1. Division des distributions, {\em S\'eminaire L. Schwartz,} 1959/60, expos\'es 21--25.

\bibitem{B. Malgrange : 2} 2. Le th\'eor\`eme de preparation en g\'eom\'etrie diff\'erentiable, {\em S\'eminaire H. Cartan,} 1962/63, expos\'es 11, 12, 13, 22.

\bibitem{B. Malgrange : 3} 3. Sur les fonctions diff\'erentiables et les ensembles analytiques, {\em Bull. Soc. Math. France,} 91 (1963), 113--127.

\bibitem{B. Morin : 1} \textsc{B. Morin} : 1. Forme canonique des singularit\'es d'une application diff\'erentiable, {\em C. R. Acad. Sc. Paris,} 260 (1965), 5662-5665 and 6503-6506.

\bibitem{A. P. Morse : 1} \textsc{A. P. Morse} : 1.\pageoriginale The behaviour of a function on its critical set, {\em Annals of Math.} 40 (1939), 62--70.

\bibitem{V. P. Palamodov : 1} \textsc{V. P. Palamodov} : 1. The structure of ideals of polynomials and of their quotients in spaces of infinitely differentiable functions (in Russian), {\em Dokl. Ak. Nauk S.S.S.R.} 141--6 (1961), 1302-1305.

\bibitem{A. Sard : 1} \textsc{A. Sard} : 1. The measure of critical values of differentiable maps, {\em Bull. Amer. Math. Soc.} 48 (1942), 883--890.

\bibitem{L. Schwartz : 1} \textsc{L. Sohwartz} : 1. {\em Th\'eorie des distributions}, t. 1, 2, Hermann, Paris (1950), 51.

\bibitem{L. Schwartz : 2} 2. Analyse et synth\`ese harmonique dans les espaces de distributions, {\em Canad, Journ. Math.} 3 (1951), 503--512.

\bibitem{L. Schwartz : 3} 3. Th\'eorie g\'en\'erale des fonctions moyenne-periodiques, {\em Annals of Math.} 48 (1947), 857--929.

\bibitem{J. P. Serre : 1} \textsc{J. P. Serre} : 1. G\;eom\'etrie alg\'ebrique et g\'eom\'etrie analytique, {\em Ann. Inst. Fourier,} 6 (1955-56), 1--42.

\bibitem{J. C. Tougeron :1} \textsc{J. C. Tougeron} : 1. Faisceaux diff\'erentiables quasi-flasques, {\em C. R. Acad. Sc. Paris,} 260 (1965), 2971--2973.

\bibitem{H. Whitney : 1} \textsc{H. Whitney} : 1. Analytic extensions of differentiable functions defined in closed sets, {\em Trans. Amer. Math. Soc.} 36 (1934), 63--89. 

\bibitem{H. Whitney : 2} 2. On ideals of differentiable functions, {\em Amer. Journ. Math.} 70 (1948), 635--658.

\bibitem{H. Whitney : 3} 3. On singularities of mappings of euclidean spaces I, {\em Annals of Math.} 62 (1955), 347--410.

\bibitem{O. Zariski and P. Samuel : 1} \textsc{O. Zariski and P. Samuel} : 1. {\em Commutative algebra}, I and II, Van Nostrand, 1958/1960.
\end{thebibliography}

\newpage

~\phantom{a}
\thispagestyle{empty}

\vfill

{\small
\begin{center}
\sc printed in india

\medskip

by. r. subbu

\medskip

at the

\medskip

commercial printing press

\medskip

limited, bombay

\medskip

and

\medskip

published by

\medskip

john brown

\medskip

oxford university press

\medskip

bombay
\end{center}
}\relax

\vfill








