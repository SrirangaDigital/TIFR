\chapter{WHITNEY'S EXTENSION THEOREM}\label{chap1}

\section{Notations.}\label{chap1-sec1}

$\bfR$ denotes the set of real numbers, $\bfN$ denotes the set of natural numbers. For any open set $\Omega$ in $\bfR^{n}$, $\scrE^{m}(\Omega)$ (resp. $\scrE^{m}_{c}(\Omega)$) denotes the space of all $C^{m}$-real-valued functions in $\Omega$ (resp. with compact support in $\Omega$). We omit $m$ when $m=+\infty$. When $\Omega=\bfR^{n}$, we write $\scrE^{m}$, $\scrE^{m}_{c}$ instead of $\scrE^{m}(\bfR^{n})$, $\scrE^{m}_{c}(\bfR^{n})$. $k=(k_{1},\ldots,k_{n})$ in $\bfN^{n}$ is called an ``$n$-integer''. We write $|k|=k_{1}+\cdots+k_{n}$ (order of $k$), $k!=(k_{1}!)\ldots(k_{n}!)$. We order $\bfN^{n}$ by the relation: ``$k\leq l$ if and only if, for every $j$, $k_{j}\leq l_{j}$''; we write $\binom{l}{k}=\dfrac{l!}{k!(l-k)!}$ if $k\leq l$ and sometimes $\binom{l}{k}=0$ if $k>l$. For $x\in \bfR^{n}$, $|x|$ denotes the euclidean norm of $x$.

Let $K$ be a compact set in $\bfR^{n}$ and consider all $F=(f^{k})_{|k|\leq m}$ where $f^{k}$ are continuous functions on $K$. Any such $F$ is called a {\em jet of order $m$}. Let $J^{m}(K)$ denote the space of all jets of order $m$ provided with the natural structure of a vector space on $\bfR$. We define $|F|^{K}_{m}=\sup\limits_{\substack{x\in K\\ |k|\leq m}}|f^{k}(x)|$; we write sometimes $|F|_{m}$ instead of $|F|^{K}_{m}$.

We write $F(x)=f^{0}(x)$ for all $x$ in $K$, $F$ in $J^{m}(K)$. For $|k|\leq m$, $D^{k}:J^{m}(K)\to J^{m-|k|}(K)$ is a linear map defined by $D^{k}F=(f^{k+l})_{|l|\leq m-|k|}$, and for any $g\in \scrE^{m}$, $J^{m}(g)$ denotes the jet $\left(\dfrac{\partial^{k}g}{\partial x^{k}}\right)_{|k|\leq m}$ in $J^{m}(K)$ where each $\dfrac{\partial^{k}g}{\partial x^{k}}$ is restricted to $K$. Now for $x\in \bfR^{n}$, $a\in K$, $F\in J^{m}(K)$, we define
$$
T^{m}_{a}F(x)=\sum\limits_{|k|\leq m}\dfrac{(x-a)^{k}}{k!}f^{k}(a).
$$
We observe that for a fixed $a$ in $K$ and $F$ in $J^{m}(K)$, $T^{m}_{a}F$ is a $C^{\infty}$-function on $\bfR^{n}$. So we write $J^{m}(T^{m}_{a}F)=\widetilde{T}^{m}_{a}F$ and $R^{m}_{a}F=F-\widetilde{T}^{m}_{a}F$.

For $x\in X$, $y\in K$, one has obviously
\begin{align}
& T^{m}_{x}\circ \widetilde{T}^{m}_{y}=T^{m}_{y},\text{ and then } \widetilde{T}^{m}_{x}\circ \widetilde{T}^{m}_{y}=\widetilde{T}^{m}_{y},\label{chap1-eq1.1}\\
& R^{m}_{x}\circ R^{m}_{y}=R^{m}_{x},\label{chap1-eq1.2}\\
& R^{m}_{x}\circ R^{m}_{y}=R^{m}_{x},\label{chap1-eq1.3}\\
& \widetilde{T}^{m}_{x}\circ R^{m}_{y}=-\widetilde{T}^{m}_{y}\circ R^{m}_{x}=\widetilde{T}^{m}_{x}-\widetilde{T}^{m}_{y}=R^{m}_{y}-R^{m}_{x},\label{chap1-eq1.4}\\
& D^{k}\circ \widetilde{T}^{m}_{x}=\widetilde{T}^{m-|k|}_{x}\circ D^{k}.\label{chap1-eq1.5}
\end{align}
For any $F$ in $J^{m}(K)$,
\begin{equation}
(R^{m}_{x}F)^{k}=f^{k}-T^{m-|k|}_{x}\circ D^{k}F.\label{chap1-eq1.6}
\end{equation}
From now on we omit the $\sim$ and we ``identify'' $T^{m}_{a}F$ and $J^{m}(T^{m}_{a}F)$.

\section{Differentiable functions in the sense of Whitney}\label{chap1-sec2}

\begin{definition}\label{chap1-defi2.1}
An increasing, continuous, concave function $\alpha:[0,\infty[\to [0,\infty[$ with $\alpha(0)=0$ is called a modulus of continuity.
\end{definition}

\begin{thm}\label{chap1-thm2.2}
The following statements are equivalent:
\end{thm}

\subsubsection{}%2.2.1 - numbers are not displayed

