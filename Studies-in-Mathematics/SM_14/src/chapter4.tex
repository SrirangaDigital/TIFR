\chapter{METRIC AND DIFFERENTIAL PROPERTIES OF ANALYTIC SETS}\label{chap4}
\pageoriginale

\section{Multipliers.}\label{chap4-sec1}

Let $\Omega$ be an open set in $\mathbf{R}^n$ and $X$ a closed subset of $\Omega$. We shall denote by $\mathscr{M}(X;\Omega)$ the set of $\mathcal{C}^\infty$-functions $f$ of $\Omega - X$ which satisfy the following condition


\subsection{}\label{chap4-sec1.1} For any compact set $K \subset \Omega$ and any $n$-tuple of positive integers $k \in \bfN^n$, there exist constants $C>0$, $m > 0$ such that
$$
|D^k f (x)| \leq C / (d (x, X))^m \text{ for } x \in K - X.
$$

We start with the following

\setcounter{theorem}{1}
\begin{lemma}\label{chap4-lem1.2}
  If $g \in \mathscr{M}(X ; \Omega)$ and $g \neq 0$ everywhere in $\Omega-X$, then $1/g \in \mathscr{M}(X ; \Omega)$ if and only if for any compact $K \subset \Omega$, there are constants $c>0$, $\alpha > 0$ such that
  \setcounter{equation}{2}
  \begin{equation}%%1.3
    |g(x)| \geq c (d (x, X))^\alpha \text{~ for ~} x \in K - X.\label{chap4-eq1.3}
  \end{equation}
\end{lemma}


\begin{proof}
  If $g^{-1} \in \mathscr{M} (X ; \Omega)$, then (\ref{chap4-sec1.1}) applied to $f = g^{-1}$  with $k=0$ on $K$ gives \eqref{chap4-eq1.3}. Conversely, if \eqref{chap4-eq1.3} holds, then the condition (\ref{chap4-sec1.1}) for $f = g^{-1}$ follows from (\ref{chap4-sec1.1}) for $g$ and the relation
  $$
D^k f = g^{-|k|-1} P_k (g, \ldots, D^k g),
$$
where $P_k$ is a polynomial in the derivatives $D^lg $ with $l \leq k$./
\end{proof}

\setcounter{theorem}{3}
\begin{proposition}\label{chap4-prop1.4}%%% 1.4
  If $\scrI (X ;\Omega)$ is the space of $C^{\infty}$-functions in $\Omega$ which are flat on $X$, then $\scrM(X;\Omega)$ is a space of multipliers for $\scrI(X;\Omega)$. More precisely, if $F \in \scrI(X;\Omega)$, $g \in \scrM(X;\Omega)$, then the $C^\infty$-function $gF$ on $\Omega- X$ has a unique extension to a $C^\infty$-function on $\Omega$ which is flat on $X$.
\end{proposition}


\begin{proof}
 Since the space $\scrI_\ast$ of $C^\infty$-functions in $\Omega$ vanishing in a neighbourhood of $X$ are dense in $\scrI (X; \Omega)$, we have only to prove that multiplication by $g$ is a continuous mapping of $\scrI_\ast$ into itself, in the topology induced from $\scrI(X;\Omega)$, i.e. given $m >0$, and $K \subset \Omega$ compact, there exists $m' >0$ and a compact $K'\subset \Omega$, such that for $F \in \scrI_\ast$, we have
  $$
\|gF\|^K_m \leq C \| F \|^{K'}_{m'}  ,
$$\pageoriginale
where $C>0$ is independent of $F$. But since for $F \in \scrI (X; \Omega)$, compact $K \subset \Omega$, and $k \in \bfN^n$, there is a compact set $K' \subset \Omega$ such that
$$
|D^k F (x)| \leq C_N(d(x,X))^N \| F \|^{K'}_{N+k} \text{ for } x \in K \text{ and any } N > 0,
$$
this follows at once from the condition (1.1) applied to $g$ and Leibniz's formula.
\end{proof}

\begin{proposition}\label{chap4-prop1.5}%% 1.5
  If $X$ and $Y$ are closed subsets of $\Omega$ which are regularly situated, and $\scrI(X \cap Y; Y)$ is the space of Whitney $C^\infty$-functions on $Y$ which are flat on $X \cap Y$, then $\scrM(X;\Omega)$ is a space of multipliers for $\scrI(X \cap Y ; Y)$ (in a sense analogous to that in Proposition \ref{chap4-prop1.4}).
\end{proposition}

\begin{proof}
Since $X$ and $Y$ are regularly situated, if $F \in \scrI(X \cap Y;Y)$, then the function $\tilde{F}$ defined to be $F$ on $Y$, 0 on $X$ is induced by a function $f \in \scrI(X;\Omega)$. Proposition \ref{chap4-prop1.5} thus follows at once from Proposition \ref{chap4-prop1.4}.
\end{proof}


\section{Quasi-H$\ddot{\rm {\bf o}}$lderian functions.} \label{chap4-sec2}

Let $\Omega$ be a bounded open set in $\bfR^n$ and $f$ a real valued function in $\Omega$.\

\begin{definition}\label{chap4-def2.1}%% 2.1
We say that $f$ is quasi-h\"olderian of order $\alpha$ $0< \alpha \leq 1 $, if there exists $C > 0$ such that for any pair of points $x$, $y$ such that the closed segment $[x,y]$ joining $x$ and $y$ belongs to $\Omega$, we have 
$$
|f(x) - f (y)| \leq C |x - y|^\alpha.
$$
\end{definition}

(Note that the condition need not be satisfied for all $x$, $y \in \Omega$.)

\setcounter{theorem}{1}
\begin{proposition}\label{chap4-prop2.2}%% 2.2
  Let $\Omega$ be a bounded open set in $\bfR^n$ and $a_i, i =1, \ldots, p$, bounded functions which are quasi-h\"olderian of order $\alpha$. Let $f$ be a continuous function on $\Omega$ satisfying
  $$
f^{p} + \sum\limits^p_{i=1} a_i f^{p-i} = 0.
$$
Then $f$ is bounded and quasi-h\"olderian of order $\alpha/p$.
\end{proposition}

The proof is based on three  very elementary lemmas.

\begin{lemma}\label{chap4-lem2.3}%% 2.3
  If\pageoriginale $c_1, \ldots, c_p, z$ are complex numbers and
  $$
z^p + \sum\limits^p_{i=1} c_i z^{p-i} = 0,
  $$
then $|z| \leq 2 \sup |c_i|^{1/i}$.
\end{lemma}

\begin{proof}
  For reasons of homogeneity, we may suppose $|c_i|\leq 1$. Then
  $$
|z|^p \leq 1+ |z| + \quad + |z|^{p-1};
$$
a fortiori
$$
\sum\limits^{\infty}_{k=1} |z|^{-k} \geq 1,
$$
whence $|z| \leq 2$.
\end{proof}

\begin{lemma}\label{chap4-lem2.4}%% 2.4
  Let $z_j (\resp z'_k) (j, k =1, \ldots, p)$ be the roots of the equation
  $$
  z^p + \sum\limits^{p}_{i=1} c_i z^{p-i} = 0 \left(\resp z^p + \sum\limits^p_{i=1} c'_i z^{p-i} = 0 \right)
  $$
  where the $c_i$, $c'_i$ are complex numbers. Suppose that
  $$
  |c_i| \leq K^i, |c_i - c'_i| \leq K^i \delta, \text{~ where ~} K, \delta > 0.
  $$
  Then for any $j$, there exists $k$ such that $|z_j - z'_k| \leq 2 K.\delta^{1/p}$
\end{lemma}

\begin{proof}
  Since $z^p_j + \sum\limits^p_{i=1} c_i z^{p-i}_j =0$, we have
  $$
\prod\limits_k  |z_j - z'_k| = \left|z^p_j + \sum\limits^p_{i=1} c'_i z^{p-i}_j\right| = \left|\sum\limits^p_{i=1} (c'_i - c_i) z^{p-l}_j\right|.
  $$
By Lemma 2.3, $|z_j| \leq 2K$, so that
$$
\prod\limits_k |z_j - z'_k| \leq 2^p K^p \delta ;
$$
Lemma 2.4 follows at once.

Proposition 2.2 obviously follows from the next lemma.
\end{proof}


\begin{lemma}\label{chap4-lem2.5}%% 2.5
  Let $K >0$, $0 < \alpha \leq 1$ and let $b_1, \ldots, b_p$ be  complex-valued functions defined on the closed interval $t_1 \leq t \leq t_2$ such that if $t_1 \leq t$, $t'\leq t_2$, we have
  $$
  |b_i(t)| \leq K^i, ~|b_i (t) - b_i (t')| \leq K^i |t - t'|^\alpha.
  $$\pageoriginale

  Let $f$ be a continuous functions on $[t_1, t_2]$ such that
  $$
f^p + \sum\limits^p_{i=1} b_i f^{p-i} = 0.
$$
Then, we have
$$
|f(t_2) - f(t_1)| \leq 4p. K. |t_2 - t_1|^{\alpha/p}.
$$
\end{lemma}

\begin{proof}
Let $z_1 = f(t_1), \ldots, z_p$ be the roots of the equation $z^p + \sum\limits^p_{i=1} b_i (t_1) z^{p-i} =0$ and let $\Omega$ be the union of the closed discs of radius $2K |t_2 - t_1|^{\alpha/p}$ and center $z_j$. Then, by Lemma \ref{chap4-lem2.4}, $f([t_1, t_2]) \subset \Omega$, and so, since $f$ is continuous, is contained in the connected component of $\Omega$ containing $z_1$. Lemma \ref{chap4-lem2.5} follows at once.
\end{proof}

\section{Notations.} In the rest of the chapter, and the following ones we will need to appeal several times to the local description of a real analytic set which was given in Chapter \ref{chap3}, \S\ref{chap3-sec5}. For this reason, we shall fix the conventions and notations to which we shall adhere.

Let $X$ be an analytic set in an open set $\Omega \subset \bfR^n$, let $0 \in X$ and suppose that the germ $X_0$ of $X$ at 0 is irreducible. We suppose that $\mathfrak{I} = \mathfrak{I}(X)$ is the ideal in $\mathcal{O}_n$ of germs of analytic functions vanishing on $X_0$. Suppose $\dim X_0 =k$ and that the coordinates $x_1, \ldots, x_n$ of $\mathbf{R}^n$ satisfy the following conditions; as we have seen these can always be achieved by a linear change of coordinates in $\mathbf{R}^n$.
\begin{enumerate}
\item[(a)] $x_1, \ldots, x_k$ are analytically independent $\mod \mathfrak{I}$, i.e. the natural mapping $\mathcal{O}_k \to \mathcal{O}_n/\mathfrak{I}$ is injective ; further $\mathcal{O}_n/\mathfrak{I}$ is a finite $\mathcal{O}_k$ module.

\item[(b)] The image in $\mathcal{O}_n/\mathfrak{I}$ of $x_{k+1}$ generates the quotient field of $\mathcal{O}_n/\mathfrak{I}$ over the quotient field of $\mathcal{O}_k$.

\item[(c)] The images $\bar{x}_{k+j}$, $j=1, \ldots, n -k$ of $x_{k+j}$ in $\mathcal{O}_n/\mathfrak{I}$ satisfy the monic polynomial equations
  $$
P_j (\bar{x}_{k+j}, x') = 0, (x' = (x_1, \ldots, x_k))
$$
over\pageoriginale $\mathcal{O}_k$, (i.e. with coefficients in $\mathcal{O}_k$). Further, we may suppose that these are the minimal equations for $\bar{x}_{k+j}$ ; therefore the $P_j$ are \textit{distinguished}. We shall denote $P_1$ by $P$. Let $\Delta (x_1, \ldots, x_k)$ be the discriminant of the polynomial $P$ in $\bar{x}_{k+1}$. Then, there exist polynomials $Q_j (\bar{x}_{k+j} ; x')$ over $\mathcal{O}_k$ such that
$$
\Delta (x'). \bar{x}_{k+j} = Q_j (\bar{x}_{k+1} ; x').
$$
\end{enumerate}

In what follows, we write  $x = (x_1, \ldots, x_n) = (x', x'')$ where $x' = (x_1, \ldots, x_k)$ and $x'' = (x_{k+1}, \ldots, x_n)$. We denote $n-k$ by $l$. We choose a neighbourhood $\Omega_1 \subset \Omega$ of $0, \Omega_1 = \Omega' \times \Omega''$ where $\Omega' \subset \mathbf{R}^k$, $\Omega'' \subset \mathbf{R}^l$, such that there are polynomials $P_j (x_{k+j}; x')$, $Q_j (x_{k+1} ; x')$, with coefficients analytic on $\Omega'$ such that the image of these in $\mathcal{O}_n/\mathfrak{I}$ are the polynomials considered above and which have the same degree in $x_{k+j}$. We denote again by $\Delta $ the discriminant of $P_1 =P$. $\Delta$ is analytic on $\Omega'$ and its germ at $0$ is $\neq 0$. $P$ being distinguished the roots of the equation $P(t,0) =0$  are all zero, so that given any neighbourhood $V''$ of $0$ in $\mathbf{R}^l$, there exists a neighbourhood $V'$ of 0 in $\mathbf{R}^k$ such that if $x'\in V'$, $x'' \in \mathbf{R}^l$, $x'' = (x_{k+1}, \ldots, x_n)$ and $P_j(x_{k+j} ; x')=0$, then $x'' \in V''$. We may choose $V''$ and $V'$ such that $V = V' \times V''$ is relatively compact in $\Omega_1$. We also suppose that $V'$  and $V''$ are cubes in $\mathbf{R}^k$, $\mathbf{R}^l$ respectively. 

Let $\Delta = \{x' \in V' |\Delta (x') = 0\}$. If $V$ is sufficiently small, the set $X \cap ((V' - \delta) \times V'')$ coincides with the set defined by the relations
\begin{align*}
 x' \in V' - \delta , ~~ P(x_{k+1} ; x') & = 0,\\
 \Delta (x')x_{k+j} - Q_j(x_{k+1} ; x') & = 0, ~ 2 \leq j \leq l.
\end{align*}

Clearly, for $x' \in V' -\delta$, all the roots of $P(x_{k+1}; x') =0$ are distinct. Let $V_s (1 \leq s \leq p)$ be the set of points $x' \in V' - \delta$ for which the polynomial $P(x_{k+1}; x')$ has at least $s$ real roots. Then $V_s$ is open and its boundary in $V'$ is contained in $\delta$. Let $F^1(x')< \ldots < F^s (x')$ be the $s$ smallest real roots of $P(x_{k+1} ; x')$ on $V_s$. Clearly, $F^r$ is defined, continuous and bounded on $V_r (1 \leq r \leq p)$. For $x' \in V_r$, put
$$
F^r_1 = F^r, ~ F^r_j (x') = \frac{Q_j (F^r (x') ; x')}{\Delta (x')}
$$

The\pageoriginale $F^r_j$ are again defined and continuous on $V_r$, and, being roots of the equation $P_j (t;x') =0$, are bounded on $V_r$. Set $\Phi^r = (F^r_1, \ldots, F^r_l)$ on $V_r$ and
$$
X'_r = \{x = (x', x'') \in V | x' \in V_r, x'' = \Phi^r (x')\}.
$$

Let $D = X \cap (\delta \times V'')$, $X_r = X'_r \cup D$. Then $X_r$ is closed in $V$ and we have $\bigcup\limits_{1 \leq r \leq p} X_r = X \cap V$.


\section{The inequality of \L ojasiewicz}\label{chap4-sec4}%%4

The aim of this section is to prove the following important theorem of {\L}ojasiewicz \cite{S. Lojasiewicz : 1}.

\begin{theorem}\label{chap4-thm4.1}%% 4.1
  Let $\Omega$ be an open set in $\mathbf{R}^n$ and let $f$ be real analytic in $\Omega$. Let $E = \{x \in \Omega |f(x) =0\}$. Then for any compact set $K \subset \Omega$, there exist constants $c$, $\alpha > 0$ such that, for all $x \in K$, we have
  $$
  |f(x)| \geq c(d(x, E))^\alpha .
  $$
   (in other words, $1/f \in \mathscr{M}(E;\Omega)$).
\end{theorem}

We shall suppose that the theorem is proved for all analytic functions in all open sets in $\mathbf{R}^m$ for $m<n$. The proof consists of two steps.

\begin{description}
  \item[Step 1.]  (\thnum{L}).\label{chap4-L} \textit{With the above hypothesis of induction, given an analytic set $S$ in $\Omega$ of dimension $< n$ at every point, if $f$, $K$, $E$ are as in Theorem 1, then exist constants $c$, $\alpha > 0$ such that, for $x \in S \cap K$, we have}
 $$
|f(x)| \geq c (d (x, E))^{\alpha}.
$$

\item[Step 2.] \textit{Deduction of Theorem 1.1 for $\Omega \subset \mathbf{R}^n$ from (\ref{chap4-L}).}
\end{description}

\medskip
\noindent
\textbf{Proof of (\ref{chap4-L}).}  It is clearly sufficient to prove that for $a \in S \cap E$, there is a neighbourhood $W$ such that for $x \in W \cap S$, we have
$$
|f(x)| \geq c (d(x, E))^\alpha 
$$
for suitable constants $c$, $\alpha > 0$. We may suppose that $a=0$. Clearly if $X$ is an analytic subset of $S \cap W$ such that the germ $X_0$ of $X$ at $0$ is irreducible, it is sufficient to prove the above inequality for all $x \in X$. Let $k = \dim X_0$; we may clearly suppose that $X \nsubset E$. We shall proceed by induction on $k$, and suppose that (\ref{chap4-L}) is proved for all sets $S$ of dimension $<k$. We begin by reducing (\ref{chap4-L}) to the following. 

(\thnum{L$'$}\label{chap4-L'}).\pageoriginale \textit{There is an analytic set $Y \subset X$ in a neighbourhood of $0$, $Y \neq X$, and constants $c>0$, $\alpha > 0$ such that for $x \in X$ near enough to $0$, we have}
\setcounter{equation}{1}
\begin{equation}
  |f(x)| \geq c (d (x, Y))^\alpha. \label{chap4-eq4.2}
\end{equation}

\medskip
\noindent{Proof that (\ref{chap4-L'}) Implies (\ref{chap4-L}).} By induction hypothesis, there are constants $B$, $\beta >0$ such that for $y \in Y$ sufficiently near 0, we have
$$
B |f(y)|^\beta \geq d (y,E).
$$
Let $x \in X$ and $y \in Y$ be such that $|x-y| = d(x, Y)$. Such a $y$ exists if $x$ is sufficiently near $0$. Now, $|f(x) - f(y)| \leq B_1 |x - y|$ (mean value theorem), so that
$$
d(y, E) \leq B |f(y)|^\beta  \leq B_2 \{|x - y|^\beta + |f(x)|^\beta\}, 
$$
so that
$$
d (x, E) \leq |x-y| + B_2 \{|x-y|^\beta + |f (x)|^\beta\}.
$$

The result now follows from the fact that $|x-y| = d(x, Y) \leq 1/c. |f(x)|^{1/\alpha}$ (by \eqref{chap4-eq4.2}).

We will now prove \ref{chap4-L'}. Since the ideal $\mathfrak{I}$ is prime, there is $h \in \mathcal{O}_n$ and $f_1 \in \mathcal{O}_k$, $f_1 \neq 0$ such that $hf-f_1 \in \mathfrak{I}$. Obviously, in (\ref{chap4-L'}), we may replace $f$ by $f_1$ and $E$ by the set $E_1$ of zeros of $f_1$ in some neighbourhood of $0$. We therefore suppose that $f\in \mathcal{O}_k$.

We take now for $Y$ the set $D \cup (E \cap X)$. Since $f\not\equiv 0$ on $X$, $Y \neq X$ it suffices to prove that on $X_s$, (notation as in \S 3), near 0, we have
$$
|f(x)| \geq c (d(x, (E \cap X_s)\cup D ))^\alpha.
$$

Now $f$ is a function of $x_1, \ldots, x_k$. If $E'$ denotes its zeros in a small neighbourhood of $0$, then Theorem 1 applied to $\mathbf{R}^k$ (induction hypothesis) shows that we have an inequality of the form
$$
|f(x')| \geq c (d (x', E'))^\alpha.
$$

To complete the proof, we have only to obtain an inequality of the form
\begin{equation}
d (x ; (E \cap X_s) \cup D) \leq B_3 (d (x', E'))^\gamma \text{ if } x \in X_s. \label{chap4-eq4.3}
\end{equation}

Now, if $x \in D$, there is nothing to prove. Suppose then that $x'\in V_s$ $x = (x', \phi^s (x'))$ and let $y'\in E'$ satisfy $d(x',E') = |x'-y'|$.

If\pageoriginale the half-open segment $[x',y'[$ meets $\delta $ \eqref{chap4-eq4.3} is obvious. If not, the segment $[x' y'[ \subset V_s$ and if $y'$ belongs to this segment and $y'_{\nu} \to y'$ as $\nu \to \infty$, then any limit point $y''$ of $\Phi^s (y'_\nu)$ has the property that $(y', y'') \in D \cup (E \cap X_s)$. Hence
\begin{align*}
  |\Phi^s (x') - y''| & \leq  {\displaystyle\mathop{\lim\sup}_{\nu \to \infty}} |\Phi^s (x') - \Phi^s (y'_\nu)|\\
  & \leq B_4 |x' - y'|^\gamma = B_4 (d (x', E'))^\gamma,
\end{align*}
the  second inequality being valid by Proposition \ref{chap4-prop2.2}, since the $F^s_j$ satisfy the monic equations $P_j (F^s_j (x); x' ) =0$. This completes the proof of \eqref{chap4-eq4.2}, and with it the proof of (\ref{chap4-L'}), and thus of (\ref{chap4-L}).

To prove Theorem 1, we have now only to complete Step 2, i.e. to prove that Theorem 1 follows from (L). It suffices to find an analytic set $S$ near $0 \in \mathbf{R}^n$, $\dim_0 S < n$, and constants $c$, $\alpha >0$ for which we have, near 0,
$$
|f(x)| \geq c (d (x, E \cup S))^\alpha.
$$

This is because, we have by (\ref{chap4-L}), for $y \in S$ an inequality of the form
$$
|f(y)| \geq c_1 (d(y, E))^{\alpha_1} \quad (c_1, \alpha_1 > 0)
$$
and we may repeat the argument used to prove that (\ref{chap4-L'}) implies (\ref{chap4-L}) to obtain the desired inequality. Now, by the Weierstrass preparation theorem, we may suppose that $f$ is a distinguished polynomial in $x_n$ and further, that $f$ is irreducible. [In fact, if the \L ojasiewicz inequality is true for two functions it is trivially true for their product.] Thus the discriminant $\Delta_f (x_1, \ldots, x_{n-1}) \neq 0$. Suppose that the coefficients of $f$ and $\Delta_f$ are defined in a neighbourhood $U$ of 0. We may then take $S = \{x \in U | \Delta_f (x_1, \ldots, x_{n-1}) = 0\}$. Let $\lambda_1, \ldots, \lambda_r$, be the real roots of the equation $f(x_n ; x_1, \ldots, x_{n-1}) =0$ and $\mu_1, \ldots, \mu_s$, the other roots. Then
$$
|f(x)| = \prod\limits^r_{i=1} |x_n - \lambda_i| \prod\limits^s_{j=1} |x_n - \mu_j|.
$$

The first product $\prod\limits^r_{i=1}$ is trivially $\geq d (x, E)^r$. Now,
$$
\prod\limits^s_{j=1} |x_n - \mu_j| \geq \prod\limits^s_{j=1} |\Iim \mu_j| \geq 2^{-s} \prod\limits^s_{j=1} |\mu_j - \bar{\mu}_j|.
$$
Now\pageoriginale the $\lambda_i$, $\mu_j$ are all bounded and $\Delta_f (x_1, \ldots, x_{n-1})$ is the product of the squares of the differences of all roots of $f(x_n; x_1, \ldots, x_{n-1}) =0$.
Hence
$$
\prod\limits^s_{j=1} |\mu_j - \bar{\mu_j}| \geq c_2 \Delta_f (x_1, \ldots, x_{n-1}).
$$

Thus,
$$
|f(x)| \geq c_3 (d (x, E))^r \Delta_f (x_1, \ldots, x_{n-1}).
$$
By induction hypothesis, $\Delta_f (x_1, \ldots, x_{n-1}) \geq c_4 (d (x; S))^\beta$, and it follows that
$$
|f(x)| \geq c (d (x, E \cup S))^\alpha. 
$$

This proves Theorem \ref{chap4-thm4.1}.

\setcounter{theorem}{3}
\begin{corollary}\label{chap4-coro4.4}
Let $\Omega$ be an open set in $\mathbf{R}^n$ and let $X$ and $Y$ be two analytic sets in $\Omega$. Then $X$ and $Y$ are regularly situated. 
\end{corollary}

\begin{proof}
  Clearly, it is enough to prove that for any $a \in X \cap Y$, there exists a neighbourhood $U$ of $a$ such that $X \cap U$ and $Y \cap U$ are regularly situated in $U$. Hence we may suppose that there exist analytic functions $f$, $g$ in $\Omega$ such that $\{x \in \Omega | f (x) = 0 \} = X$, $\{x \in \Omega | g (x) =0 \} = Y$. Let $K$  be any compact set in $\Omega$. Then, there exists a constant $B >0 $ such that for $x \in K$,
 \setcounter{equation}{4}
  \begin{equation}
    |g(x)| \leq B d (x, Y). \label{chap4-eq4.5}
  \end{equation}

  By Theorem \ref{chap4-thm4.1}, applied to the function $f^2 + g^2$ there are constants $c$, $\alpha > 0$ so that for $x \in K$, we have
  $$
f^s (x) + g^2 (x) \geq c (d (x, X \cap Y))^\alpha,
$$
since $X \cap Y = \{x \in \Omega| f^2 (x) + g^2 (x) = 0\}$. Combining this with \eqref{chap4-eq4.5}, we obtain,
$$
d (x, Y) \geq c_1 (d (x, X \cap Y))^{\alpha/2} \text{ for  } x \in X \cap K , \text{ q.e.d.}
$$
\end{proof}

\section{Further properties of analytic sets.}\label{chap4-sec5}%% 5

The above corollary gives us information on the metric properties of two analytic sets. We now go back to the notation of \S 3, and prove some metric properties of different ``sheets'' of the same irreducible analytic set, due also to {\L}ojasiewicz \cite{S. Lojasiewicz : 1}.

Let\pageoriginale $X$ be an analytic set in the open set $\Omega \subset \mathbf{R}^n$, irreducible at the origin. suppose $k = \dim_0 X$ and let $V = V' \times V''$ be a neighbourhood of 0 as in \S 3. We have defined closed sets $X_r$ in $V$, $1 \leq r \leq p$ in \S 3. We have

\begin{proposition}\label{chap4-prop5.1}
  For any pair of integers $r$, $s$, the sets $X_r$, $X_s$ are regularly situated. 
\end{proposition}

\begin{proof}
  We may obviously suppose $r <s$, so that $V_r \supset V_s$. It is clear that for any compact set $K' \subset V'$, there exists a compact set $K'' \subset V''$ such that $(K' \times V'') \cap X = (K' \times K'') \cap X$. Let $K = K' \times K''$ We have to prove that there exist constants $c$, $\alpha > 0$ so that for $x \in K \cap X_s$, $y \in K \cap X_r$, we have $|x-y| \geq c (d(x, D))^\alpha$ (since $X_s \cap X_r = D$). We have already seen (in the proof of Theorem \ref{chap4-thm4.1}) that for $x \in X$, we obtain an inequality of the form
  $$
d (x', \delta) \geq B (d(x, D))^\beta
$$
from the fact that the functions $F^r_j$ are quasi-h\"olderian. Hence we have only to prove an inequality of the form
\setcounter{equation}{1}
\begin{equation}
  |x-y| \geq c(d(x', \delta))^\alpha.   \label{chap4-eq5.2}
\end{equation}
Let $ x = (x', x'')$, $y = (y', y'')$, where $x' \in V_s$, $y' \in V_r$ . If the closed segment $[x', y'] \not\subset V_s$, then it meets $\delta$ and \eqref{chap4-eq5.2} is trivial. Suppose therefore that $[x', y'] \subset V_s$. Let $\eta  = (x', \Phi^r (x'))$. Now, $F^r(x')$ and $F^s(x')$ are two distinct roots of the equation $P(t; x') =0$. Hence there is a constant $A>0$ so that
$$
|F^r (x') - F^s (x')| \geq A |\Delta (x')|.
$$

Hence, by Theorem 1 applied to $\Delta$, we have
$$
|x - \eta| \geq |F^r (x') - F^s (x')|  \geq B_1 (d(x', \delta))^{\beta_1}.
$$
On the other hand,
\begin{align*} 
  |y - n | &  \leq |y' - x'| + \Phi^r (x') - \Phi^r (y') |\\
  & \leq B_2 |x' - y'|^{\beta_2} \tag{5.3}\label{chap4-eq5.3}
\end{align*} 
(since the functions $F^r_j$ are quasi-h\"olderian). If now $B_2 |x'- y'|^{\beta_2} \geq \frac{1}{2} B_1 (d (x', \delta))^{\beta_1}$, \eqref{chap4-eq5.2} is trivial. Otherwise, we have $|y - \eta| \leq \frac{1}{2} |x - \eta|$, so that $|x - y| \geq \frac{1}{2} |x - \eta| \geq \frac{1}{2} B_1 (d (x', \delta))^{\beta_1}$, and  is \eqref{chap4-eq5.2} proved.
\end{proof} 

\setcounter{theorem}{2}
\begin{proposition}\label{chap4-prop5.3}%%% 5.3
  For\pageoriginale $1 \leq j \leq l (= n - k)$, $1 \leq r \leq p $, the functions $F^r_j$ belong to the space $M (V'-V_r ; V')$.
\end{proposition}

\begin{proof}
  By Lemma \ref{chap4-lem1.2} and Theorem \ref{chap4-thm4.1}, $\dfrac{1}{\Delta} \in \mathscr{M} (V'-V_r; V')$. Hence, we have only to prove the proposition for $j=1$, i.e. for the function $F^r$. We prove by induction on $|q| (q\in \bfN^k)$ an estimate of the form
   \begin{equation*}
|D^q F^r (x')| \leq  \frac{C_q }{d (x', \delta)^{m_q}}  \tag{5.4}\label{chap4-eq5.4}
  \end{equation*}
   for $x' \in K' - \delta$, $K'$ being a compact subset of $V'$. Suppose $q \in \bfN^n$ and suppose \eqref{chap4-eq5.4} proved for all $q'$ with $|q'|<|q|$. Since $P(F^r (x'); x') =0$, we have a relationship
   $$
\left(\frac{\partial P}{\partial x_{k+1}} (F^r; x') \right)^{\lambda_q} \quad D^q F^r = R_q (F^r, D^{q'} F^r; x'),
$$
where $\lambda_q$ is an integer $> 0$ and $R_q$ is a polynomial in $F^r$ and its derivatives $D^{q'} F^r$  of order $<|q|$ (differentiation of composite functions). After our induction hypothesis and Theorem \ref{chap4-thm4.1}, we have only to prove an inequality of the form
$$
\left|\frac{\partial P}{\partial x_{k+1}} (F^r; x') \right| \geq c. |\Delta (x')|.
$$
But this is immediate, and the proposition follows.
\end{proof}

We shall end this section by giving a description of the space $\scrI (D; X_r)$ of Whitney functions on $X_r$ which are flat on $D$. 

Let $\lambda \in \mathbf{N}^n = \mathbf{N}^k \times \mathbf{N}^l$, and let
$$
F = \{f^\lambda\} \in \scrI (D; X_r).
$$

We remark that $F$ is determined uniquely by the collection $\{g^\mu\}_{\mu \in \mathbf{N}^l}$ where 
$$
g^\mu = f^\lambda \text{ with } \lambda = 0 \times \mu, ~~ 0 \in \mathbf{N}^k.
$$

In fact, if $\lambda = \nu \times \mu$, $\nu \in \mathbf{N}^k$ then $f^\lambda$ is a linear combination of derivatives $D^{\nu'}_{x'} g^{\mu'} (x', \Phi^r (x'))$ with $\mu' \leq \mu$. (See also proof of \eqref{chap4-eq5.5b} given below.)

Given\pageoriginale $(g^\mu)$ which determines an element of $\scrI(D;X_r)$, let us set
$$
h^u (x') - g^\mu (x' ;\Phi^r (x')) \in \scrE (V_r).
$$
This gives us a mapping
$$
\pi : \scrI (D ; X_r) \to [\scrE (V_r)]^{\mathbf{N}^l}
$$

\setcounter{theorem}{4}
\begin{proposition}\label{chap4-prop5.5}
$\pi$ maps $\scrI (D; X_r)$ bijectively onto $[\scrI (V' - V_r; V')]^{\mathbf{N}^l}$.
\end{proposition}


\begin{proof}
  As remarked above, $\pi$ is injective. We have only to prove the following two facts;
  \begin{align*}
\pi (\scrI (D;X_r)) \subset [\scrI (V' - V_r; V')]^{\mathscr{N}^l},      \tag{5.5a} \label{chap4-eq5.5a}\\
\pi(\scrI (D; X_r)) \supset [\scrI (V' - V_r' V')]^{\mathscr{N}^l}      \tag{5.5b} \label{chap4-eq5.5b}
  \end{align*}
\end{proof}

\medskip
\noindent
\textbf{Proof of (a).} We remark that any derivative of $h^{\mu} (x')$ can be expressed as a finite linear combination of the functions $f^\lambda (x', \Phi^r (x'))$ with coefficients which are polynomials in the derivatives of $F^r_j$. (This can be proved, for example, by choosing a $C^\infty$-function in $V$ inducing $(f^\lambda)$ and applying the rule for differentiation of composite functions.) To prove (a), we have only to prove : given a compact subset $K'$ of $V'$, $f^\lambda (x', \Phi^r (x'))$ tends to zero faster than any positive power of $d(x',\delta)$ when $x' \in V_r \cap K'$ tends to $\delta$. But this follows from the definition of $\scrI (V' - V_r, V')$ and the fact that $\Phi^r$ is quasi-h\"olderian.

\medskip
\noindent
\textbf{Proof of (b).} Let $h = (h^\mu)_{\mu \in \mathbf{N}^l}$, $h^\mu \in \scrI (V' - V_r; V')$ be given. It is enough to prove that for any integer $m >0$, there is a $C^m$ functions $H$ on $V$, $m$-flat on $(V'- V_r) \times V''$, such that for $ \mu \in \mathbf{N}^l$, $|\mu | \leq m$, we have
$$
D^\mu_{x''} H  (x'; \Phi^r (x')) = h^\mu (x') (D^{\mu}_{x''} = D^{\mu_1}_{x_{k+1}} \ldots D^{\mu_l}_{x_n}). 
$$
We take $H=0$ on $(V'-V_r) \times V''$ and
$$
H(x) = \sum\limits_{|\mu| \leq m} h^\mu (x') \frac{(x'' - \Phi^r (x'))^\mu}{\mu!} \text{ for } x = (x' , x'') \in V_r \times V''.
$$

By Proposition \ref{chap4-prop5.3} since $h^\mu (x') \in \scrI (V' - V_r ; V')$, $H$ is $C^\infty$ on $V$. Clearly this has the required properties.

