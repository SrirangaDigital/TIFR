\chapter{IDEALS DEFINED BY ANALYTIC FUNCTIONS}\label{chap6}

\section{The main theorem}\label{chap6-sec1}

The main theorem of this chapter is the following.

\begin{theorem}\label{chap6-thm1.1}
Let $\scrO_{n}$, $\scrE_{n}$ denote the rings of germs of analytic and differentiable functions respectively and $\widetilde{\scrF}_{n}$ the ring of germs at the origin of collections of formal power series at each point near $0$ {\em (see Chapter \ref{chap3}, \S\ref{chap3-sec4})}. Let $\fraka$ be an ideal in $\scrO_{n}$. Then we have
$$
(\fraka\widetilde{\scrF}_{n})\cap\scrE_{n}=\fraka\cdot \scrE_{n}.
$$
\end{theorem}

This theorem is obviously equivalent with the following (partition of unity).

\medskip
\noindent
{\bf Theorem \thnum{1.1$'$}.\label{chap6-thm1.1'}}~{\em Let $\Omega$ be an open set in $\bfR^{n}$ and $f_{1},\ldots,f_{p}$ analytic functions in $\Omega$. Let $\phi\in\scrE(\Omega)$. Then $\phi$ can be written in the form}
$$
\psi =\sum\limits^{p}_{i=1}f_{i}\psi_{i},\text{ where } \psi_{i}\in \scrE(\Omega),
$$
{\em if and only if for any $a\in \Omega$, the Taylor expansion $T_{a}\phi$ belongs to the ideal generated by the $T_{a}f_{i}$ in $T_{a}\scrE(\Omega)=\scrD_{a}$ (formal power series at $a$).} For $p=1$, see H\"ormander \cite{L. Hormander : 1}, {\L}ojasiewicz \cite{S. Lojasiewicz : 1}; for the general cases, Malgrange \cite{B. Malgrange : 1}: see also Palamodov \cite{V. P. Palamodov : 1}.

For the proof of the theorem, we shall use certain reductions which are very similar to those used in the proof of the preparation theorem. We start by stating a more general form of Theorem \ref{chap6-thm1.1}.

If $Y_{0}\subset X_{0}$ are germs of analytic sets at $0$ in $\bfR^{n}$, let $\widetilde{\scrF}_{n}(X_{0})$ denote the ring of germs at $0$ of collections, at points of $X_{0}$, of formal power series. Clearly we have an inclusion $\scrE(X_{0})\subset \widetilde{\scrF}_{n}(X_{0})$ where $\scrE(X_{0})$ is the ring of germs at $0$ of Whitney functions on $X_{0}$. Let $\scrF(Y_{0};X_{0})$ denote the subring of $\scrE(X_{0})$ of functions flat on $Y_{0}$. Then we have

\begin{theorem}\label{chap6-thm1.2}
If $\fraka$ is an ideal in $\scrO_{n}$, we have
$$
\fraka\cdot \widetilde{\scrF}_{n}(X_{0})\cap \scrF(Y_{0};X_{0})=\fraka\cdot \scrF(Y_{0};X_{0}).
$$
We shall call Theorem \ref{chap6-thm1.2} for the germs $Y_{0}$ and $X_{0}$, $\Th(Y_{0},X_{0})$ (we suppose $\fraka$ given). As in the case of the preparation theorem, we may reduce Theorem \ref{chap6-thm1.2} to the proof of the following statement:

$P(X_{0})$. Given the analytic germ $X_{0}$ at $0$, for any analytic germ $Y_{0}\subset X_{0}$, $Y_{0}\neq X_{0}$ there is an analytic germ $Z_{0}\neq X_{0}$, $Y_{0}\subset Z_{0}\subset X_{0}$ such that $\Th(Z_{0},X_{0})$ is true.
\end{theorem}

We remark that Theorem 1.2 implies

\medskip
\noindent
{\bf Theorem \thnum{1.2$'$}.\label{chap6-thm1.2'}}~{\em If $X$ is an analytic set in an open set $\Omega\subset \bfR^{n}$, if $f_{1},\ldots,f_{p}$ are analytic in $\Omega$ and $Y$ is an analytic subset of $X$, then for any $\phi\in \scrF(Y;X)$, there exist functions $\psi_{1},\ldots,\psi_{p}\in \scrF(Y;X)$ such that $\phi=\Sigma f_{i}\psi_{i}$ if and only if $T_{a}\phi$ belongs to the ideal generated by the $T_{a}f_{i}$ in $\scrF_{a}$ for any $a\in X$.}
\smallskip

We shall prove $P(X_{0})$ by induction on $k=\dim X_{0}$; we may therefore suppose Theorem \ref{chap6-thm1.2'} true for any analytic set $X\subset Q$ whose dimension at any point is $<k$.

Now we shall show that it suffices to prove $P(X_{0})$ when $X_{0}$ is irreducible and $\fraka$ is contained in the ideal $\frakp\subset\scrO_{n}$ of functions vanishing on $X_{0}$. The proof that we may suppose $X_{0}$ irreducible is the same as in the case of the preparation theorem and we do not repeat the argument. Suppose that $X_{0}$ is irreducible and let $\fraka\nsubset \frakp$; let $f\in \fraka$, $f\not\in \frakp$, and let $Z_{0}=Y_{0}\cup [X\cap \{x|f(x)=0\}]$. Then $\Th(Z_{0},X_{0})$ is true as follows from the next lemma.

\begin{lemma}\label{chap6-lem1.3}
Let $\Omega$ be an open set in $\bfR^{n}$, and $f$ analytic in $\Omega$. Let $S=\{x\in\Omega|f(x)=0\}$. Let $\phi$ be a function $\in\scrF(S;\Omega)$. Then there exists $\psi\in \scrF(S;\Omega)$ such that $\phi=\psi f$.
\end{lemma}

\begin{proof}
By the inequality of Lojasiewicz and Chapter \ref{chap4}, Lemma \ref{chap4-lem1.2}, $1/f\in \scrM(S;\Omega)$. Since $\phi\in \scrF(S;\Omega)$, $\psi=(1/f)\phi\in\scrF(S;\Omega)$ by Chapter \ref{chap4}, Proposition \ref{chap4-prop1.4}. This gives Lemma \ref{chap6-lem1.3}.

Before going to the proof of $P(X_{0})$, we need two lemmas.
\end{proof}

\begin{lemma}\label{chap6-lem1.4}
Let $\Omega$ be an open set in $\bfR^{n}$ containing $0$, $\fraka_{0}$, $\frakb_{0}$ two ideals in $\scrO_{n}$. Let $f=(f_{1},\ldots,f_{p})$; $g=(g_{1},\ldots,g_{q})$ be generators of $\fraka_{0}$, $\frakb_{0}$ respectively and suppose that they are analytic in $\Omega$. Let $\fraka_{x}$, $\frakb_{x}$ be the ideals generated by $f_{1},\ldots,f_{p}$; $g_{1},\ldots,g_{q}$ at $x\in \Omega$. Then, for any compact subset $K$ of $\Omega$ and any integer $m\geq 0$, there is an integer $m'$ such that for any $x\in K$, we have
$$
\fraka^{m'}_{x}\cap \frakb_{x}\subset \fraka^{m}_{x}\cdot \frakb_{x}.
$$
\end{lemma}

\begin{proof}
Let, for $x\in \Omega$, $m'(x)$ be the smallest integer $m'$ such that
$$
\fraka^{m'}_{x}\cap \frakb_{x}\subset \fraka^{m}_{x}\cdot \frakb_{x};
$$
(such an $m'$ exists by the Artin-Rees lemma). Now there exist $h_{1},\ldots,h_{r}$; $k_{1},\ldots,k_{s}$ in a neighbourhood $U$ of $x$, such that the $h$ belong to $\fraka^{m'(x)}_{y}\cap \frakb_{y}$ and generate it for any $y\in U$, while the $k$ belong to $\fraka^{m'}_{y}\cdot \frakb_{y}$ and generate it for any $y\in U$. Since $\fraka^{m'(x)}_{x}\cap \frakb_{x}\subset \fraka^{m}_{x}\cdot \frakb_{x}$, if $U$ is small enough, there exist analytic functions $a_{ij}$ in $U$ such that $h_{i}=\sum\limits^{s}_{j=1}a_{ij}k_{j}$, $i=1,\ldots,r$. Then clearly, since the $h_{i}$ generate $\fraka^{m'(x)}_{y}\cap \frakb_{y}$, we have $\fraka^{m'(x)}_{y}\cap \frakb_{y}\subset \fraka^{m}_{y}\cdot \frakb_{y}$, so that $m'(y)\leq m'(x)$ for $y\in U$. Hence $m'(x)$ is bounded on $K$, and the lemma follows.
\end{proof}

\begin{lemma}\label{chap6-lem1.5}
Let $\Omega$ be an open set in $\bfR^{n}$, $f$ an analytic function on $\Omega$ and let $X=\{x\in \Omega | f(x)=0\}$. Then any point $a$ of $\Omega$ has a fundamental system of open neighbourhoods $\Omega_{p}$ such that $\Omega_{p}-X$ has only finitely many connected components, each of which contains $a$ in its closure in $\Omega_{p}$.
\end{lemma}

\begin{proof}
We may clearly suppose that $a=0$ and that $f$ is a distinguished pseudopolynomial in $x_{n}$ which is irreducible at $0$. Let $y=(x_{1},\ldots,x_{n-1})$. The discriminant of $f$ has a germ at $0\in \bfR^{n-1}$ which is not zero. If $\Omega=\Omega'\times \Omega''$, $\Omega'\subset \bfR^{n-1}$, $\Omega''\subset \bfR$, suppose that the lemma is already proved for the set $Y\subset \Omega'$, $Y=\{y\in \Omega'|\Delta (y)=0\}$. Let $\Omega'_{p}$ be a fundamental system of open neighbourhoods of $0\in \bfR^{n-1}$ such that $\Omega'_{p}-Y$ has $k_{p}$ components $U_{p,v}$ which contain $0$ in their closures. Let $I_{p}$ be an open interval whose length $\to 0$ as $p\to \infty$ such that $f(y,x_{p})=0$, $y\in \Omega'_{p}$ imply $x_{n}\in I_{p}$, and let $\Omega_{p}=\Omega'_{p}\times I_{p}$. To show that $\Omega_{p}-X$ has only finitely many components each adherent to $0$, it is enough to prove the same of $\Omega_{p}-X-(Y\times I_{p})$. Now, the number of real roots of $f(y,x_{n})=0$ is constant $=s$, say, on $U_{p,v}$; let $\tau_{1}(y)<\ldots<\tau_{s}(y)$ be these roots. Then the connected components of $\Omega_{p}-X-(Y\times I_{p})$ are the sets
$$
\{(y,x_{n})|y\in U_{p,v},\tau_{i}(y)<x_{n}<\tau_{i+1}(y)\},
$$
where we have set $\tau_{0}=-\infty$, $\tau_{s+1}=+\infty$. Since $\tau_{i}(y)\to 0$ as $y\to 0$, $(1\leq i\leq s)$ the lemma follows.

We now go to the proof of $P(X_{0})$. We use the notations of Chapter \ref{chap4}, \S\ref{chap4-sec3}, and we may suppose, as in the preparation theorem, that there is an analytic set $Y'\subset V'$ such that $Y=(Y'\times V'')\cap X$. Let $\delta$ be the set $\{x'\in V'|\Delta(x')=0\}$. Let $Z'=Y'\cup \delta$, and suppose that $V'$ is so chosen that $V'-Z'$ has only finitely many connected components, each adherent to zero. Let $Z=X\cap (Z'\times V'')$. Then the same is true of $X-Z$, in fact, any component $U'$ of $V'-Z'$ is contained in a set $V_{r}$, and the components of $X-Z$ are the sets $\{(x',x'')|x'\in U'$, $x''=\Phi^{s}(x')\}$, for any $s\leq r$.

We suppose that $\Omega$ is an open set containing $\overline{V}$ and that $\frakp$ is generated by functions $f_{1},\ldots,f_{p}$ analytic on $\Omega$. Let, for $x\in \Omega$, $\frakp_{x}$ denote the ideal at $x$ generated in the ring of analytic functions at $x$ by the $f_{i}$. Finally let $\scrF_{x}$ denote the ring of formal power series at $x$.
\end{proof}

Now we make the following remark:

\setcounter{subsection}{5}
\subsection{}\label{chap6-sec1.6}
{\em If $\phi$ is a germ of $C^{\infty}$-functions on $X$, at $a\in X-Z$ and the ``normal derivatives'' of $\phi$ vanish upto order $m$ (i.e. $D^{\lambda}_{x''}\phi=0$ for $\lambda\in \bfN^{l}$, $|\lambda|\leq m$), then the Taylor expansion of $\phi$ at a belongs to $\frakp^{m+1}_{a}\cdot \scrF_{a}$.}

This is a trivial consequence of the fact that, in $\scrO_{a}$ the ideal generated by $P(x_{k+1};x')$ and $\Delta x_{k+j}-Q_{j}(x_{k+1};x')$ coincides with the ideal of germs vanishing on $X$. Suppose now that $\frakq$ is any ideal in $\scrO_{n}$ generated by functions $g_{1},\ldots,g_{q}$ analytic in $\Omega$, $\frakq\subset \frakp$. We identify $\scrI(Z;X_{r})$ with $[\scrI((V'-V_{r})\cup Z';V'))]^{N^{l}}$ (by Chapter \ref{chap4}, Proposition \ref{chap4-prop5.5}). Let $\lambda\in \bfN^{l}$, and $g^{\lambda}_{j}=(D^{\lambda}_{x''}g_{j})(x',\Phi^{r}(x'))$. We prove first the following

\setcounter{theorem}{6}
\begin{lemma}\label{chap6-lem1.7}
Suppose $\phi=(\phi^{\lambda})\in[\scrI((V'-V_{r})\cup Z';V')]^{N^{l}}$, and suppose that the Taylor expansion of $\phi^{\lambda}$ at any point $a'$ of $V_{r}$ belongs to the ideal generated in $\scrF_{a'}$ by the $g^{\lambda}_{j}$. Then there exist functions $\phi^{\lambda}_{j}$, $\mu<\lambda$ such that we have
$$
\phi^{\lambda'}=\sum\limits^{q}_{j=1}\sum\limits_{\lambda<\lambda'}\binom{\lambda'}{\mu}g^{\lambda'-\mu}_{j}\psi^{\lambda}_{j}\text{ for } \lambda'\leq \lambda.
$$
\end{lemma}

\begin{proof}
If all the $g^{\lambda}_{j}$, $\mu\leq \lambda$ vanish on $V$, we have nothing to prove. Otherwise, let $x'$ be a point at which the matrix $(g^{\lambda'-\mu'}_{j})$ has maximal rank (indices being $\lambda'$ and the pairs $(\mu',j)$ say $\rho$ and let $A'$ be a $\rho\times \rho$ submatrix of $(g^{\lambda'-\mu'}_{j})$ whose determinant at $x'$ if non-zero. Let $A$ denote the corresponding $\rho\times\rho$ submatrix of $(D^{\lambda'-\mu'}_{x''}g_{j})$. Then, clearly $\det A\neq 0$ at the point $(x',\Phi^{r}(x'))$. Let $S$ be the set of points of $X$ where $\det A$ is zero. We assert that $\dim S$ is $< k$ at every point. In fact, since every component of $V'-Z'$ is adherent to $0$, the projection of $S$ contains no neighbourhood of $0$ in $V'$. Hence the germ of $S$ at $0$ is $\subset X_{0}$, $\neq X_{0}$, so that the dimension of $S$ is $<k$ at every point of $X-Z$ since every component of $X-Z$ is adherent to $0$, so that $S$ can contain no such component.

To prove Lemma \ref{chap6-lem1.7}, we use now the following simple generalization of Lemma \ref{chap6-lem1.3}.
\end{proof}

\begin{lemma}\label{chap6-lem1.8}
Let $h_{1},\ldots,h_{\rho}$ be $\rho$-tuples of analytic functions on the connected open set $\Omega\subset \bfR^{n}$, which are linearly independent at some point of $\Omega$. Let $M$ be the set of points of $\Omega$ where they are not linearly independent. Then for any $\rho$-tuple $\phi$ of $C^{\infty}$-functions flat on $M$, there exist functions $\psi_{i}$, $1\leq i\leq \rho$, flat on $M$, such that
$$
\phi=\sum\limits^{\rho}_{i=1}\psi_{i}h_{i}.
$$
Moreover, the $\psi_{i}$ are flat at any point of $\Omega-M$ where $\phi$ is.
\end{lemma}

Lemma \ref{chap6-lem1.7} is an immediate consequence of Lemma \ref{chap6-lem1.8} if the function $\phi$ is flat on $S$.

Since, by assumption, the system under consideration is soluble at every point, and $A'$ is a submatrix of maximal rank outside the projection of $S\cap X_{r}$ on $V_{r}$, it is sufficient to solve the square system
$$
\phi^{\lambda'}=\sum\limits_{j,\mu}\binom{\lambda'}{\mu}g^{\lambda'-\mu}_{j}\psi^{\mu}_{j}\text{ where } (g^{\lambda'-\mu'}_{j})=A',
$$
with the $\psi^{\mu}_{j}$ flat on the projection of $S\cap X_{r}$ on $V'$; the other equations in the system are then automatically satisfied.

To prove Lemma \ref{chap6-lem1.7}, we proceed as follows. Let $\phi_{1}\in \scrI(Z\cap S;S)$ be the restriction of $\phi$ to $S$. By the inductive hypothesis and Theorem \ref{chap6-thm1.2'}, there exist $\psi_{i}\in\scrI(Z\cap S;S)$ such that
$$
\phi_{1}=\Sigma \psi_{i}g_{i}\text{ in } \scrE(S).
$$

Let $\psi'_{i}\in \scrI(Z;X)$ be such that their restrictions to $S$ are the $\psi_{i}$ (this is possible because any two analytic sets are regularly situated); and let $\phi'=\Sigma g_{i}\psi'_{i}$. Then $\phi-\phi'\in \scrI(Z\cap S;X)$ and we may apply the above result to $\phi-\phi'$. Since Lemma \ref{chap6-lem1.7} is true for $\phi-\phi'$ and for $\phi'$, it is clearly true for $\phi$.

We now go back to our ideal $\fraka\subset\frakp$, and suppose that it is generated by functions analytic on $\Omega$; then clearly Lemma \ref{chap6-lem1.7} is true for the ideal $\frakq_{m}=\frakp^{m}\cdot \fraka$ for every $m\geq 0$. Suppose $m'=m'(m)$ so chosen that
\setcounter{equation}{8}
\begin{equation}
\frakp^{m'}_{x}\cap \fraka_{x}\subset \frakp^{m}_{x}\cdot \fraka_{x}\text{ for } x\in V\quad \text{(Lemma \ref{chap6-lem1.4}).}\label{chap6-eq1.9}
\end{equation}
Lemma \ref{chap6-lem1.7} and the assertion (\ref{chap6-sec1.6}) show that the following holds 

\setcounter{subsection}{9}
\subsection{}\label{chap6-sec1.10}
{\em If $\phi\in \scrI(Z;X_{r})$ and $\phi$ is $m'$-flat on $X_{r}$, then there exist, for any $\lambda\in \bfN^{l}$, functions $\psi^{\lambda}_{j}\in \scrI(Z;X_{r})$ which are $m$-flat on $X_{r}$ such that}
\setcounter{equation}{10}
\begin{equation}
\phi^{\lambda'}=\sum\limits^{p}_{j=1}\sum\limits_{\mu<\lambda'}\binom{\lambda'}{\mu}f^{\lambda'-\mu}_{j}\psi^{\mu}_{j}\text{ for } \lambda'\leq \lambda.\label{chap6-eq1.11}
\end{equation}

It is now easy to complete the proof of $P(X_{0})$. Given $\phi$, it suffices to find $\psi_{j}\in \scrI(Z';X_{r})$ with $\phi=\sum f_{i}\psi_{i}$, since the $X_{r}$ are regularly situated.

We write $\phi$ in the form $\phi=\phi_{1}+\phi_{2}+\cdots$ where $\phi_{k}\in \scrI(Z;X_{r})$, and the component $\phi^{\lambda}_{k}\neq 0$ only if $m'(k)\leq |\lambda|\leq m'(k+1)$ [where $m'(k)$ is defined by \eqref{chap6-eq1.9}]. There exist functions $\psi^{\mu}_{j,1},\in \scrI((V_{1}-V_{r})\cup Z';V')$, $|\mu|\leq m'(1)$ such that $\psi^{\mu}_{j,1}$ are $0$-flat on $X_{r}$, and $\phi_{1}-\sum\limits^{p}_{j=1}f_{j}\psi_{j,1}$ is $m'(1)$-flat. Let $\phi^{1}_{2}=\phi_{1}+\phi_{2}-\sum\limits^{p}_{j=1}\psi_{j,1}f_{j}$. We can, as before, find functions $\psi_{j,2}\in \scrI(Z;X_{r})$ which are $1$-flat on $X_{r}$, such that
$$
\phi_{1}+\phi_{2}-\sum\limits^{p}_{j=1}f_{j}(\psi_{j,1}+\psi_{j,2})\text{ is } m'(2)\text{-flat.}
$$

By induction, we find $\psi_{j,k}\in \scrI(Z,X_{r})$, $\psi_{j,k}$ being $(k-1)$-flat on $X_{r}$, such that
$$
\phi_{1}+\phi_{2}+\quad+\phi_{k}-\sum\limits^{p}_{j=1}f_{j}(\psi_{j,1}+\quad+\psi_{j,k})\text{ is } m'(k)\text{-flat.}
$$
Clearly $\psi_{j}=\sum\limits^{\infty}_{k=1}\psi_{j,k}\in\scrI(Z;X_{r})$ (since $\psi_{j,k}$ is $(k-1)$-flat) and $\phi=\sum\limits^{p}_{j=1}f_{j}\psi_{j}$.

This proves $P(X_{0})$ and hence the main theorem.
\setcounter{theorem}{11}
\begin{corollary}\label{chap6-coro1.12}
$\scrE_{n}$ is a faithfully flat $\scrO_{n}$-module.
\end{corollary}

We have seen already that $\widetilde{\scrF}_{n}$ is a faithfully flat $\scrO_{n}$-module (Theorem III, 4.12). Therefore. the corollary results from Theorem \ref{chap6-thm1.1} and Proposition III, 4.7.

\section{A remark concerning the {\L}ojasiewicz inequality}\label{chap6-sec2}

Let $\Omega\in \bfR^{n}$ and $f\in \scrE(\Omega)$. Let $X=\{x\in \Omega|f(x)=0\}$. We assert that if $f\scrE(\Omega)$ is closed, then for any compact $K\subset \Omega$, there exist constants $C$, $\alpha>0$ such that
\setcounter{equation}{0}
\begin{equation}
|f(x)|\geq C\{d(x,X)\}^{\alpha}\text{ for } x\in K.\label{chap6-eq2.1}
\end{equation}
In fact, suppose $f\scrE(\Omega)$ closed. Then, by Banach's theorem, to every compact set $K\subset \Omega$ and $m>0$, there exists a compact set $K'\subset \Omega$ and $m'>0$ such that if $g\in f\scrE(\Omega)$, there exists a $\psi\in \scrE(\Omega)$ with $\psi f=g$ such that
\begin{equation}
|\psi|^{K}_{m}\leq C|g|^{K'}_{m'}, C\text{ independent of } g.\label{chap6-eq2.2}
\end{equation}
If $x_{0}\in K$, we may find $g\in\scrE(\Omega)$, $g(x_{0})=1$, $g=0$ in a neighbourhood of $X$ such that
$$
|g|^{K'}_{m'}\leq \dfrac{A}{\{d(x_{0},X)\}^{p}},
$$
where $A>0$ and $p>0$ are independent of $x_{0}$, but depend only on $K$, $K'$. \eqref{chap6-eq2.2} clearly implies that
$$
\sup\limits_{K}\left|\dfrac{g}{f}\right|\leq \dfrac{AC}{\{d(x_{0},X)\}^{p}};
$$
in particular
$$
|f(x_{0})|\geq \dfrac{\{d(x_{0},X)\}^{p}}{AC}.
$$

Next we give an example to show that the situation for non-analytic functions is rather complicated.

Let $f^{\pm}=y^{2}\pm e^{-1/x^{2}}\in \scrE(bfR^{2})=\scrE$. Then $f^{+}\scrE$ is not closed, but $f^{-}\scrE$ is. In fact $f^{+}$ does not satisfy \eqref{chap6-eq2.1} in any neighbourhood of $0$. Since $f^{-}=(y+e^{-1/2x^{2}})(y-e^{-1/2x^{2}})=f^{-}_{1}f^{-}_{2}$, we have only to prove the theorem for $f^{-}_{1}$, $f^{-}_{2}$ separately. But, by a change of coordinates, these functions can be made linear.

\section{Differentiable functions vanishing on an analytic set}\label{chap6-sec3}

The results of this paragraph are based on the following theorem:

\begin{theorem}[Zariski-Nagata]\label{chap6-thm3.1}
If the analytic algebra $A$ is an integral domain, so is its completion $\widehat{A}$.
\end{theorem}

For the proof, see e.g. Houzel \cite{C. Houzel : 1} or Malgrange \cite{B. Malgrange : 3}.

Here are some immediate consequences (in the statements $A$ is an analytic algebra and $\widehat{A}$ its completion).

\setcounter{subsection}{1}
\subsection{}\label{chap6-sec3.2}
{\em If $\frakp$ is a prime ideal of $A$, $\widehat{\frakp}=\widehat{A}\frakp$ is prime.} (Apply \eqref{chap6-thm3.1} to $A/\frakp$).

\subsection{}\label{chap6-sec3.3}
{\em Let $\frakq$ be an ideal of $A$ and $\frakp_{1},\ldots,\frakp_{s}$ the minimal prime ideals in the decomposition of $\frakq$. Then $\widehat{\frakp}_{1},\ldots,\widehat{\frakp}_{s}$ are the minimal prime ideals in the decomposition of $\widehat{\frakq}$.}

In fact, one is reduced at once to the case $\frakq=\{0\}$; $\frakp_{1},\ldots,\frakp_{s}$ are then the minimal prime ideals of $A$. Let us put $\frakr=\frakp_{1}\cap\quad \cap \frakp_{s}$. It is well known that $\frakr$ is the set of nilpotent elements of $A$ and that for a certain $n$, one has $\frakr^{n}=\{0\}$.

By Proposition III, 4.5 and Theorem III, 4.9, we have $\widehat{\frakr}=\widehat{\frakp}_{1}\cap\quad\cap \widehat{\frakp}_{s}$. On the other hand, we have obviously $\widehat{\frakr}^{n}=\{0\}$. Suppose that $\frakI$ is a prime ideal of $\widehat{A}$, and let us suppose, for example, that $\widehat{\frakp}_{1}\nsubset \frakI,\ldots,\widehat{\frakp}_{s-1}\nsubset \frakI$. Let $a_{i}\in \widehat{\frakp}_{i},a_{i}\not\in \frakI$; $(1\leq i\leq s-1)$. For any $x\in \widehat{\frakp}_{s}$, we have
$$
(a_{1}\ldots a_{s-1}x)^{n}=0\in\frakI,
$$
whence $x\in \frakI$. Hence $\frakp_{s}\subset \frakI$, which proves (\ref{chap6-sec3.3})

(\ref{chap6-sec3.3}) shows in particular that if $A$ is {\em reduced} (i.e. has no nilpotent elements), then $\widehat{A}$ is reduced.

\setcounter{theorem}{3}
\begin{definition}\label{chap6-defi3.4}
Let $X$ be a subset of $\bfR^{n}$ adherent to $0$, and let $g$ be a function of class $C^{\infty}$ in a neighbourhood of $0$. We say that $g$ has a zero of infinite order on $X$ at $0$ if, for any $p\in \bfN$, there is a neighbourhood $U_{p}$ of $0$ and a number $C_{p}>0$ such that, on $X\cap U_{p}$, we have $|g(x)|\leq C_{p}|x|^{p}$.
\end{definition}

The above property depends only on the germ $X_{0}$ of $X$ and on the Taylor expansion of $g$ at $0$. The set of these Taylor series forms an ideal in $\scrF_{n}$ which we call the ``formal ideal defined by $X$ (or $X_{0}$)'' and denote $J(X)$.

\begin{theorem}\label{chap6-thm3.5}
Let $X$ be an analytic set in a neighbourhood of $0$ in $\bfR^{n}$, and let $I(X)$ be the ideal in $\scrO_{n}$ of germs of analytic functions vanishing on $X_{0}$. We have $I(X)\scrF_{n}=\widehat{I(X)}=J(X)$.
\end{theorem}

It is sufficient to prove this theorem when $X_{0}$ is irreducible. In fact, if $X=X'\cup X''$, we have
$$
I(X)=I(X')\cap I(X'').
$$
By Proposition III, 4.5 and Theorem III, 4.9, we deduce that 
$$
\widehat{I(X)}=\widehat{I(X')}\cap \widehat{I(X'')}.
$$
On the other hand, we obviously have $J(X)=J(X')\cap J(X'')$; hence if the theorem is true of $X'$, $X''$, it is true of $X$.

Suppose then that $X_{0}$ is irreducible. Set $\dim X=k$, and let us go back to the notation of Chapter \ref{chap3}, \S\ref{chap3-sec3}. The mapping $\scrO_{k}\to \scrO_{n}/I(X)$ defined by $\overline{x}_{1},\ldots,\overline{x}_{k}$ is finite and injective, and hence the ``intrinsic'' topology of $\scrO_{n}/I(X)$ coincides with its topology as $\scrO_{k}$-module. From the exactness properties of the completion, we deduce from this that the mapping $\scrF_{k}\to \scrF_{n}/\widehat{I(X)}$ defined in the same way as above is still injective; this mapping is finite by (III, 1.6). On the other hand, $\widehat{I(X)}$ is prime (by (\ref{chap6-sec3.2})). Let us apply Proposition III, 5.4 to
$$
A=\scrF_{k}, B-\scrF_{n}/\widehat{I(X)}, \frakp=\{0\}, \frakq=J(X)/\widehat{I(X)}.
$$
We find that, to prove the theorem, it is sufficient to verify that one has
$$
J(X)\cap \scrF_{k}=\{0\}.
$$
($\scrF_{k}$ is considered as imbedded in $\scrF_{n}$). This amounts to proving the following:

{\em Any function $f(x_{1},\ldots,x_{k})$ of class $C^{\infty}$ having a zero of infinite order at $0$ on $X$ has a Taylor series which is identically zero.}

Let $U$ be the (germ at $0$ of the) set of points of $\bfR^{k}-\delta$ which are images of points of $X$ under the projection $(x',x'')\to x'$ (i.e. $U=\bigcup\limits_{s\geq 1}V_{s}$; the notation is that of Chapter \ref{chap4}). There is a $C>0$, and $p>0$ such that, on $X$ we have $|x''|\leq C|x'|^{p}$. Hence $f$, considered now as a function on $\bfR^{k}$, has a zero of infinite order at $0$ on $U$. Changing the notation, we are led to prove the following proposition.

\begin{proposition}\label{chap6-prop3.6}
Let $\Omega$ be an open set in $\bfR^{k}$, $0\in \Omega$, and $\Phi$ be an analytic function in $\Omega$ with $\Phi(0)=0$, $\Phi\nequiv 0$, and let $D$ be the set of zeros of $\Phi$. Let $\Gamma$ be an open and closed subset of $\Omega-D$ which is adherent to $0$. Then we have $J(\Gamma)=\{0\}$.
\end{proposition}

To prove this proposition, we shall proceed as follows. We shall suppose that $\Phi(0,\quad,0,x_{k})$ is not identically zero near $0$, and shall show that, under this condition, any $f$ having a zero of infinite order on $\Gamma$ at $0$ satisfies
$$
\dfrac{\partial^{q}f}{\partial x^{q}_{k}}(0,\ldots,0)=0\forall q\in \bfN.
$$

This implies the required result: in fact, the set of lines through $0$ on which $\Phi$ is not identically zero near $0$ is an open dense set in the set of lines through $0$. Since, by a linear change of coordinates, we can take any one of these lines as $0x_{k}$ axis, it follows, by an elementary argument, that all the derivatives of $f$ are zero at the origin. One has thus $J(\Gamma)=0$.

Suppose, then, that $\Phi(0,\ldots,0,x_{k})$ is not identically zero in a neighbourhood of $0$. By making $\Omega$ smaller, we may suppose that $\Phi$ is a distinguished polynomial in $x_{k}$, whose germ $\Phi_{0}$ has no multiple factors. We have then
$$
\Phi=x^{p}_{k}+\sum\limits^{p}_{i=1}a_{i}(x_{1},\ldots,x_{k-1})x^{p-i}_{k},
$$
the $a_{i}$ being analytic in $\Omega$ with $a_{i}(0)=0$, $1\leq i\leq p$, and the discriminant $\Delta$ of $\Phi$ is not identically zero near $0$.

For $x=(x_{1},\ldots,x_{k})$, set $x=(x',x_{k})$ and $\pr(x)=x'$. Let $\Omega'$ be a neighbourhood of $0$ in $\bfR^{k-1}$ such that the conditions
$$
x'\in \Omega',\Phi(x',z)=0, z\in \bfC,
$$
imply that $|z|\leq \frac{1}{2}$, and, if $z$ is real, that $(x',z)\in \Omega$. Let $\delta$ be the set of zeros of $\Delta$ in $\Omega'$ and $V'\subset \Omega'$ an open neighbourhood of $0$ in $\bfR^{k-1}$ which is relatively compact in $\Omega'$. By the inequality (IV, 4.1) of {\L}ojasiewicz, there exists $C>0$ and $\alpha>0$ such that
$$
\forall x'\in V', |\Delta(x')|\geq C d (x',\delta)^{\alpha}.
$$
If $z^{1},\ldots,z^{p}$ are the roots of the equation $\Phi(x',z)=0$, we always have
$$
|z^{i}-z^{j}|\leq 1, \text{ hence } |z^{i}-z^{j}|\geq C \ d(x',\delta)^{\alpha}\text{ if } i\neq j.
$$

We may suppose that, $n$ addition to the conditions imposed above, we have $\Omega=\Omega'\times (-a,a)$, $a>0$. For $x'\in\Omega$, the interval $\{x'\}\times (-a,a)$ is decomposed into at most $p+1$ intervals by the zeros of $\Phi(x',z)$, and we always have
$$
\Phi (x',\pm a)\neq 0.
$$
This implies that $\gamma=\pr(\Gamma')-\delta$ is open and closed in $\Omega'-\delta$; further $\gamma$ is clearly adherent to $0$. For any $x'\in\gamma$, the set $\pr^{-1}(x')\cap \Gamma$ contains at least one of the preceding intervals; we denote the origin of this interval by $b(x')$, its extremity by $c(x')$. If $-a\neq b(x')$, $a\neq c(x')$, $b(x')$, $c(x')$ are distinct (consecutive) zeros of $\Phi$, hence, for $x'\in V'$ we have
\begin{equation*}
c(x')-b(x')\geq C d(x',\delta)^{\alpha}.\tag{3.6.i}\label{chap6-eq3.6.i}
\end{equation*}
If we have $b(x')=-a$, $c(x')\neq a$, we replace $b(x')$ by $c(x')-Cd(x',\delta)^{\alpha}$ (which $\to 0$ as $x'\to 0$, so that, if $V'$ is small enough, this is $>-a$); we proceed in a similar way if $b(x')\neq -a$, $c(x')=a$. If $b(x')=-a$, $c(x')=a$, we replace $b(x')$ by $0$ and $c(x')$ by $Cd(x',\delta)^{\alpha}$.

After these modifications, the inequality \eqref{chap6-eq3.6.i} is valid at any point of $V'$, and there exist constants $C'>0$, $\alpha'>0$, such that $\forall x'\in V'$, we have
\begin{equation*}
|b(x')|, |c(x')|\leq C'|x'|^{\alpha'}.\tag{3.6.ii}\label{chap6-eq3.6.ii}
\end{equation*}

\begin{lemma}\label{chap6-lem3.7}
With the hypotheses of the preceding proposition, there exists a sequence $\{x^{1}\}$ of points of $\Gamma$, $x^{l}\to 0$, and numbers $C''>0$, $\alpha''>0$ such that
$$
|x^{l}|\leq C'' d (x^{l},D)^{\alpha''}\forall l.
$$
\end{lemma}

\begin{proof}
This lemma is obvious if $k=1$. Suppose the lemma verified for $k-1$. It is sufficient to find a sequence $x^{l}$ of points of $\Gamma$, tending to zero, such that
$$
|x^{l}|\leq C''|\Phi (x^{l})|^{\alpha''}
$$
By induction, there is a sequence ${x'}^{l}$ of points of $\gamma$ satisfying 
\begin{equation*}
|{x'}^{l}|\leq C''' d({x'}^{l},\delta)^{\alpha'''}\tag{3.7$'$}\label{chap6-eq3.7'}
\end{equation*}
One verifies easily that the sequence
$$
x^{l}=\left({x'}^{l}=\dfrac{b({x'}^{l})+c({x'}^{l})}{2}\right)
$$
has the required properties: it is sufficient to estimate from below the distance of $x^{l}$ from the roots of $\Phi({x'}^{l},z)=0$. For the real roots, this follows from \eqref{chap6-eq3.6.i}, for the imaginary roots from the estimate from below of the imaginary part of a root in terms of $\Delta(x')$. The lemma follows.

We apply this lemma to $\Delta$ and $\gamma$ (instead of $\Phi$ and $\Gamma$ as in the statement). There is a sequence of points ${x'}^{l}$ of points of $\gamma$, ${x'}^{l}\to 0$, satisfying \eqref{chap6-eq3.7'}. Divide the interval $[b({x'}^{l}), c({x'}^{l})]$ into $q$ equal intervals with extremities
$$
b_{0}({x'}^{l})=b({x'}^{l}), b_{1}({x'}^{l}),\ldots,b_{q}({x'}^{l})=c({x'}^{l})
$$
and consider the expression
$$
\dfrac{1}{(b_{1}-b_{0})^{q}}\left\{f({x'}^{l},b_{0})-\binom{q}{1}f({x'}^{l},b_{1})+\cdots+(-1)^{q}f({x'}^{l},b_{q})\right\}.
$$
As $l\to \infty$, this expression tends to $\dfrac{\partial^{q}f}{\partial x^{q}_{k}}(0)$. On the other hand, the inequalities \eqref{chap6-eq3.6.i}, \eqref{chap6-eq3.6.ii}, \eqref{chap6-eq3.7'} and the fact that $f$ has a zero of infinite order at $0$ on $\Gamma$ show that this limit is $0$. Thus we have
$$
\dfrac{\partial^{q}f}{\partial x^{q}_{k}}(0)=0\quad \forall q\in\bfN,
$$
which proves Proposition \ref{chap6-prop3.6} and hence Theorem \ref{chap6-thm3.5}.
\end{proof}

\begin{remark}\label{chap6-rem3.8}
Let $X$ be a subset of $\bfR^{n}$, adherent to $0$. Besides $J(X)$, we may consider also the ideal $J'(X)\subset \scrF_{n}$ of Taylor series at $0$ of functions $f\in\scrE_{n}$ vanishing on $X$. We have $J'(X)\subset J(X)$. If $X$ is an analytic set, we have $\widehat{I(X)}\subset J'(X)$. Hence, by (\ref{chap6-thm3.5}), $J(X)=J'(X)$ in this case.
\end{remark}

We shall examine now what one can say about differentiable functions vanishing on an analytic set, and not just about their Taylor series.

\begin{definition}\label{chap6-defi3.9}
Let $\Omega$ be an open set in $\bfR^{n}$ and $X$ an analytic set in $\Omega$, $a\in X$. We say that $X$ is coherent at a if there exists a neighbourhood $\Omega'$ of $a$ and $a$ finite number of analytic functions $f_{i}(1\leq i\leq p)$ in $\Omega'$, vanishing on $X$ and having the following property:

For any $b\in\Omega'$, the images of $f_{1},\ldots,f_{p}$ in $\scrO_{b}$ (the ring of germs of analytic functions at $b$) generate $I(X_{b})$.
\end{definition}

Contrary to what happens in the complex case, this property is not verified for all analytic sets. The simplest counter-example is the ``umbrella'' $x_{3}(x^{2}_{1}+x^{2}_{2})=x^{3}_{1}$ which has the line $x_{1}=x_{2}=0$ as isolated generator, and so is not coherent at $0$.

\begin{theorem}\label{chap6-thm3.10}
Let $X_{0}$ be a real analytic germ at $0$ in $\bfR^{n}$, $I(X_{0})$ its analytic ideal, and let $K(X_{0})$ its analytic ideal, and let $K(X_{0})$ be the ideal in $\scrE_{n}$ of $C^{\infty}$ functions vanishing on $X_{0}$. Then the following properties are equivalent.
\begin{itemize}
\item[\rm(i)] $K(X_{0})=I(X_{0})\scrE_{n}$.

\item[\rm(ii)] $X_{0}$ is coherent at $0$.
\end{itemize}
\end{theorem}

\begin{proof}
(ii) $\Rightarrow$ (i). Let $X_{0}$ be coherent at $0$ and let $X$ be a representative of $X_{0}$ in a neighbourhood $\Omega'$ of $0$ with the property given in (\ref{chap6-rem3.8}). Let $\phi\in \scrE(\Omega')$, $\phi=0$ on $X$. By (\ref{chap6-thm3.5}), for any $b\in \Omega'$, $T_{b}\phi$ is a linear combination of the $T_{b}f_{i}$. Hence, by (\ref{chap6-thm1.1'}), $\phi$ is a linear combination of the $f_{i}$ in $\scrE(\Omega')$.

(i) $\Rightarrow$ (ii) (Tougeron \cite{J. C. Tougeron :1}). Suppose that $X_{0}$ is not coherent. Let $f_{1},\ldots,f_{p}$ be generators of $I(X_{0})$, $\Omega$ a neighbourhood of $0$ in which the $f_{i}$ are defined, and set
$$
X=\{x\in \Omega|f_{1}(x)=\quad = f_{p}(x)=0\}.
$$

Since $X_{0}$ is not coherent, there is a sequence $\{x^{l}\}$ of distinct points of $X$, $x^{l}\to 0$, and a sequence of functions $g^{l}$ defined near $x^{l}$, such that, for each $l$, $g^{l}$ is not a linear combination of the $f_{i}$. Let $\{\phi^{l}\}$ be a sequence of functions $\in \scrE(\Omega)$, $\phi^{l}=1$ near $x^{l}$, having compact support in $\Omega$ and in the set where $g^{l}$ is defined such that the supports of $\phi^{l}$, $\phi^{l'}$ do not meet if $l\neq l'$. Let $h^{l}=\phi^{l}g^{l}$, extended to $\Omega$ by $0$. By an argument which is well known in the theory of Fr\'echet spaces (which we leave to the reader) we can find a sequence $\{\lambda^{l}\}$ of real numbers $\neq 0$ such that the series $\Sigma \lambda^{l}h^{l}$ converges, in $\scrE(\Omega)$, to a function $g$. The germ $g_{0}\in \scrE_{n}$ of $g$ at $0$ is not a linear combination of the $f_{i}$ in $\scrE_{n}$, whence the theorem.

We refer to Malgrange \cite{B. Malgrange : 3} for applications of Theorems \ref{chap6-thm3.5} and \ref{chap6-thm3.10} to complex analytic sets. In conclusion, let us note another application of Theorem \ref{chap6-thm3.5}.
\end{proof}

\begin{proposition}\label{chap6-prop3.11}
Let $X_{0}$ be an analytic germ at $0$ in $\bfR^{n}$ with $\dim X_{0}=k$. Suppose that $X_{0}$ contains the germ $V_{0}$ of a $C^{\infty}$ manifold of dimension $k$. Then $V_{0}$ is the germ of an analytic manifold (which is then an irreducible component of $X_{0}$).
\end{proposition}

Before giving the proof, we give two examples.

\setcounter{subsection}{11}
\begin{subexample}\label{chap6-exam3.11.1}
If $X_{0}$ is a $C^{\infty}$ manifold, it is an analytic manifold. However, one sees easily that even for $n=2$, if we replace $C^{\infty}$ by $C^{r}(r\in \bfN)$, the statement is no longer true.
\end{subexample}

\begin{subexample}\label{chap6-exam3.11.2}
Let $\Phi\in \scrO_{n+1}$, $\Phi\neq 0$, and let $f\in \scrE_{n}$, $f(0)=0$ satisfy
$$
\Phi(x_{1},\ldots,x_{n},f(x_{1},\ldots,x_{n}))=0.
$$
Then $f$ is analytic [take for $X$ the set defined by $\Phi(x_{1},\ldots,x_{n+1})=0$ and for $V$ that defined by $x_{n+1}=f(x_{1},\ldots,x_{n})$].
\end{subexample}

\noindent
{\bf Proof of the Proposition.} Denote by $I(X_{0})$ the analytic ideal of $X$ and by $J(V_{0})$ the formal ideal of $V_{0}$. The structure of $J(V_{0})$ is obvious because of our hypothesis that $V_{0}$ is non-singular. On the other hand, $I(X_{0})\subset J(V_{0})$, hence $\widehat{I(X_{0})}\subset J(V_{0})$. Since
$$
\dim (\scrF_{n}/\widehat{I(X_{0})})=\dim (\scrF_{n}/J(V_{0}))=k,
$$
$J(V_{0})$ is a minimal prime ideal in the decomposition of $\widehat{I(X_{0})}$. By (\ref{chap6-sec3.3}), there exists a prime ideal $\frakp\subset \scrO_{n}$ with
$$
\frakp\supset I(X_{0}), \widehat{\frakp}=J(V_{0}).
$$
There remain two things to be proved.

(i) {\em The germ $W_{0}$ defined by $\frakp$ is an analytic manifold of dimension $k$.} (This is an easy consequence of the Jacobian criterion for regular points; we leave the details to the reader.)

(ii) {\em We have $V_{0}=W_{0}$.}

By an analytic change of coordinates, we may suppose that $W_{0}$ is defined by equations $x_{k+1}=\ldots=x_{n}=0$. On the other hand, $V_{0}$ is obviously tangent to $W_{0}$ of infinite order at $0$, hence defined by equations
$$
x_{k+j}=\phi_{k+j}(x_{1},\ldots,x_{k}),\phi_{k+j}\in \scrE_{k}, \phi_{k+j}\text{ flat at $0$.}
$$

Suppose that $W_{0}\neq V_{0}$. Let $X'_{0}$ be the union of the irreducible components of $X_{0}$ different from $W_{0}$, and let $g(x_{1},\ldots,x_{k})\in \scrO_{k}$ be a function not identically zero, which vanishes on $X'_{0}\cap W_{0}$. Let $D$ be the set of zeros of $g$, and $U$ be the set of points of $\bfR^{n}-D$ near $0$, for which we do not have $\phi_{k+1}(x_{1},\quad,x_{k})=\ldots=\phi_{n}(x_{1},\ldots,x_{k})=0$. $U$ is clearly open and closed in $\bfR^{n}-D$ near $0$ and is adherent to $0$. Let $f$ be a function $\in \scrO_{n}$ vanishing on $X'_{0}$. In particular, $f$ vanishes on $V_{0}-W_{0}$. Hence $f(x_{1},\ldots,x_{k},0,\ldots,0)$ has a zero of infinite order at $0$ on $U$. By Proposition \ref{chap6-prop3.6}, $f$ has a Taylor series which is zero at $0$, hence is itself $0$. Hence $f$ vanishes on $W_{0}$, contradicting the fact that $W_{0}\nsubset X'_{0}$. The proposition follows.
