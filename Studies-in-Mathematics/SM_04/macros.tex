\usepackage{graphicx,xspace,fancybox}
\usepackage{fancyhdr}
\usepackage{color}


\newcounter{pageoriginal}

\usepackage[papersize={140mm,215mm},textwidth=110mm,
textheight=170mm,headheight=6mm,headsep=4mm,topmargin=17.5mm,botmargin=1.15cm,
leftmargin=15mm,rightmargin=15mm,footskip=0.6cm]{zwpagelayout}

\marginparwidth=10pt
\marginparsep=10pt
\marginparpush=5pt
%\renewcommand{\thepageoriginal}{\arabic{pageoriginal}}
\newcommand{\pageoriginale}{\refstepcounter{pageoriginal}\marginpar{\footnotesize\xspace\textbf{\thepageoriginal}}
} 
\let\pageoriginaled\pageoriginale


\newtheorem{appthm}[equation]{Theorem}
\newtheorem{proposition}{Proposition}
\newtheorem{theorem}{Theorem}
\newtheorem{lemma}{Lemma}
\newtheorem{corollary}{Corollary}
\newtheorem{definition}{Definition}

\newtheoremstyle{romanexam}{10pt}{10pt}{ }%
{}{\bfseries}{.}{ }{}
\theoremstyle{romanexam}
\newtheorem{romanexam}{Example}
\renewcommand{\theromanexam}{\Roman{romanexam}}
\renewcommand{\thesection}{\Roman{section}}

\newtheoremstyle{alphaclaim}{10pt}{10pt}{ }%
{}{\bfseries}{.}{ }{#1 (#2)}
\theoremstyle{alphaclaim}
\newtheorem{alphaclaim}{Claim}
\renewcommand{\thealphaclaim}{\Alph{alphaclaim}}
\renewcommand{\thesection}{\Alph{section}}

\newtheoremstyle{remark}{10pt}{10pt}{ }%
{}{\bfseries}{.}{ }{}
\theoremstyle{remark}
\newtheorem{remark}[proposition]{Remark}
%\newtheorem{question}[theorem]{Question}
%\newtheorem{example}[theorem]{Example}

\newtheoremstyle{nonum}{}{}{\itshape}{}{\bfseries}{}{ }{#1. \mdseries
{\bf #3}}
\theoremstyle{nonum}

\newtheorem{lemma*}{Lemma}	
\newtheorem{theorem*}{Theorem}	
\newtheorem{maintheorem*}{Main Theorem}
\newtheorem{prop*}{Proposition}	
\newtheorem{defi*}{Definition}
\newtheorem{coro*}{Corollary}

\newtheoremstyle{mynonum}{}{}{ }{}{\bfseries}{}{ }{#1. \mdseries
{\bf #3}}
\theoremstyle{mynonum}
\newtheorem{remark*}{Remark}	
\newtheorem{remarks*}{Remarks}	
\newtheorem{exer*}{Exercise}	
\newtheorem{example*}{Example}	
\newtheorem{examples*}{Examples}	
\newtheorem{note*}{Note}
\newtheorem{problem}{Problem}
\newtheorem{notation*}{Notation}
\newtheorem{conjecture*}{Conjecture}

\def\ophi{\overset{o}{\phi}}

\def\oval#1{\text{\cornersize{2}\ovalbox{$#1$}}}


\newcommand*\mycirc[1]{%
  \tikz[baseline=(C.base)]\node[draw,circle,inner sep=.7pt](C) {#1};\:
}


\DeclareMathOperator{\RHom}{\textit{R}\mathrm{Hom}}
\DeclareMathOperator{\supp}{\mathrm{supp}}
\DeclareMathOperator{\cotr}{\mathrm{cotr}}
\DeclareMathOperator{\Tor}{\mathrm{Tor}}
\DeclareMathOperator{\spec}{\mathrm{spec}}
\DeclareMathOperator{\Res}{\mathrm{Res}}
\DeclareMathOperator{\gr}{\mathrm{gr}}
\DeclareMathOperator{\Proj}{\mathrm{Proj}}
\DeclareMathOperator{\rad}{\mathrm{rad}}
\DeclareMathOperator{\res}{\mathrm{res}}
\DeclareMathOperator{\Gr}{\mathrm{Gr}}
\DeclareMathOperator{\Lie}{\mathrm{Lie}}
\DeclareMathOperator{\End}{\mathrm{End}}
\DeclareMathOperator{\Eng}{\mathrm{Eng}}
\DeclareMathOperator{\Der}{\mathrm{Der}}
\DeclareMathOperator{\Ker}{\mathrm{Ker}}
\DeclareMathOperator{\Coker}{\mathrm{Coker}}
\DeclareMathOperator{\Iim}{\mathrm{Im}}
\DeclareMathOperator{\id}{\mathrm{id}}
\DeclareMathOperator{\Supp}{\mathrm{Supp}}
\DeclareMathOperator*{\varprojLim}{\underleftarrow{\mathrm{Lim}}}
\DeclareMathOperator{\Tr}{\mathrm{Tr}}
\DeclareMathOperator{\cl}{\mathrm{cl}}
\DeclareMathOperator{\Alb}{\mathrm{Alb}}
\DeclareMathOperator{\Fix}{\mathrm{Fix}}
\DeclareMathOperator{\Gal}{\mathrm{Gal}}
\DeclareMathOperator{\Div}{\mathrm{Div}}
\DeclareMathOperator{\pr}{\mathrm{pr}}
\DeclareMathOperator{\Iid}{\mathrm{id}}
\DeclareMathOperator{\rank}{\mathrm{rank}}
\DeclareMathOperator{\rk}{\mathrm{rk}}
\DeclareMathOperator{\ad}{\mathrm{ad}}
\DeclareMathOperator{\Card}{\mathrm{Card}}
\DeclareMathOperator{\card}{\mathrm{card}}
\DeclareMathOperator{\Spin}{\mathrm{Spin}}
\DeclareMathOperator{\Aut}{\mathrm{Aut}}
\DeclareMathOperator{\Int}{\mathrm{Int}}
\DeclareMathOperator{\Rre}{\mathrm{Re}}
\DeclareMathOperator{\rRe}{\mathrm{Re}}
\DeclareMathOperator{\Spec}{Spec}
\DeclareMathOperator{\ord}{\mathrm{ord}}
\DeclareMathOperator{\red}{\mathrm{red}}
\DeclareMathOperator{\Hom}{\mathrm{Hom}}
\DeclareMathOperator{\Hilb}{\mathrm{Hilb}}
\DeclareMathOperator{\Quot}{\mathrm{Quot}}
\DeclareMathOperator{\Pic}{\mathrm{Pic}}
\DeclareMathOperator{\Ext}{\mathrm{Ext}}


\def\uub#1{\underline{\underline{#1}}}
\def\ub#1{\underline{#1}}
\def\os#1{\overset{#1}}
\def\us#1{\underset{#1}}
\def\ob#1{\overbrace{#1}}
\def\ool#1{\overline{\overline{#1}}}
\def\ol#1{\overline{#1}}
\def\set#1{\left\{{#1}\right\}}
\def\oset#1{\left({#1}\right)}
\def\cset#1{\left[{#1}\right]}
\def\mset#1{\left|{#1}\right|}
\def\aset#1{\left<{#1}\right>}


\font\bigsymb=cmsy10 at 4pt
\def\bigdot{{\kern1.2pt\raise 1.5pt\hbox{\bigsymb\char15}}}
\def\overdot#1{\overset{\bigdot}{#1}}

\makeatletter
%\renewcommand\subsection{\@startsection{subsection}{2}{\z@}%
%                                     {-3.25ex\@plus -1ex \@minus -.2ex}%
%                                     {1.5ex \@plus .2ex}%
%                                     {\bf}}%
\renewcommand\section{\@startsection {section}{1}{0mm}%
                                   {3.25ex \@plus 0.5ex \@minus .2ex}%
                                   {-0.8ex \@plus.2ex}%
                                   {\normalfont\Large\bfseries}}
\renewcommand\subsection{\@startsection{subsection}{2}{\z@}%
                                     {3.25ex\@plus 1ex \@minus .2ex}%
                                     {-1.5ex}%
                                     {\bf}}%
\renewcommand\subsubsection{\@startsection{subsubsection}{2}{\z@}%
                                     {3.25ex\@plus 1ex \@minus .2ex}%
                                     {-1.5ex}%
                                     {\bf}}%

\renewcommand\thesection{\@arabic\c@section}
%\renewcommand\thesubsection{({\thechapter.\thesection.\@arabic\c@subsection})}

\renewcommand{\@seccntformat}[1]{{\csname the#1\endcsname}\hspace{0.3em}}
\makeatother

\def\fibreproduct#1#2#3{#1{\displaystyle\mathop{\times}_{#3}}#2}
\let\fprod\fibreproduct

\def\fibreoproduct#1#2#3{#1{\displaystyle\mathop{\otimes}_{#3}}#2}
\let\foprod\fibreoproduct


\def\cf{{cf.}\kern.3em}
\def\Cf{{Cf.}\kern.3em}
\def\eg{{e.g.}\kern.3em}
\def\ie{{i.e.}\kern.3em}
\def\iec{{i.e.,}\kern.3em}
\def\idc{{id.,}\kern.3em}
\def\resp{{resp.}\kern.3em}
\def\mod{{\rm{mod}}\kern.3em}

\def\bfa{\mathbf{a}}
\def\bfb{\mathbf{b}}
\def\bfA{\mathbf{A}}
\def\bfB{\mathbf{B}}
\def\bfC{\mathbf{C}}
\def\bfD{\mathbf{D}}
\def\bfd{\mathbf{d}}
\def\bfE{\mathbf{E}}
\def\bfF{\mathbf{F}}
\def\bff{\mathbf{f}}
\def\bfG{\mathbf{G}}
\def\bfH{\mathbf{H}}
\def\bfh{\mathbf{h}}
\def\bfI{\mathbf{I}}
\def\bfJ{\mathbf{J}}
\def\bfK{\mathbf{K}}
\def\bfL{\mathbf{L}}
\def\bfM{\mathbf{M}}
\def\bfN{\mathbf{N}}
\def\bfO{\mathbf{O}}
\def\bfP{\mathbf{P}}
\def\bfR{\mathbf{R}}
\def\bfS{\mathbf{S}}
\def\bfs{\mathbf{s}}
\def\bfT{\mathbf{T}}
\def\bfV{\mathbf{V}}
\def\bfW{\mathbf{W}}
\def\bfZ{\mathbf{Z}}
\def\bfQ{\mathbf{Q}}
\def\SL{\mathbf{SL}}
\def\Sp{\mathbf{Sp}}
\def\GL{\mathbf{GL}}
\def\PSL{\mathbf{PSL}}

\def\Ad{\mathbf{Ad}}

\DeclareMathOperator{\scrHom}{\mathscr{H}\mathrm{om}}
\DeclareMathOperator{\scrExt}{\mathscr{E}\mathrm{xt}}
\def\sB{\mathscr{B}}
\def\sF{\mathscr{F}}
\def\sG{\mathscr{G}}
\def\sN{\mathscr{N}}
\def\sO{\mathscr{O}}
\def\sS{\mathscr{S}}
\def\sR{\mathscr{R}}
\def\sY{\mathscr{Y}}
\def\sE{\mathscr{E}}

\def\bbF{\mathbb{F}}
\def\bbN{\mathbb{N}}
\def\bbQ{\mathbb{Q}}
\def\bbR{\mathbb{R}}
\def\bbW{\mathbb{W}}
\def\bbZ{\mathbb{Z}}

\def\frakm{\mathfrak{m}}


\def\frakM{\mathfrak{M}}



\def\calO{\mathcal{O}}

\pagestyle{fancy}

\makeatletter
\renewcommand\chaptermark[1]{\markboth{#1}{}}
\renewcommand\sectionmark[1]{\markright{#1}}

\def\cleardoublepage{\clearpage\if@twoside \ifodd\c@page\else
    \thispagestyle{empty}\hbox{}\newpage\if@twocolumn\hbox{}\newpage\fi\fi\fi}

\renewcommand\tableofcontents{%
    \if@twocolumn
      \@restonecoltrue\onecolumn
    \else
      \@restonecolfalse
    \fi
    \chapter*{\contentsname
        \@mkboth{%
           \MakeUppercase\contentsname}{\MakeUppercase\contentsname}}%
    \@starttoc{toc}%
    \if@restonecol\twocolumn\fi
    }
\makeatother
\renewcommand{\headrulewidth}{0pt}

\marginparsep=10pt
\marginparwidth=18pt


\makeatletter
\renewenvironment{thebibliography}[1]
     {\section*{\bibname}%
      \@mkboth{\MakeUppercase\bibname}{\MakeUppercase\bibname}%
      \list{\@biblabel{\@arabic\c@enumiv}}%
           {\settowidth\labelwidth{\@biblabel{#1}}%
            \leftmargin\labelwidth
            \advance\leftmargin\labelsep
            \@openbib@code
            \usecounter{enumiv}%
            \let\p@enumiv\@empty
            \renewcommand\theenumiv{\@arabic\c@enumiv}}%
      \sloppy
      \clubpenalty4000
      \@clubpenalty \clubpenalty
      \widowpenalty4000%
      \sfcode`\.\@m}
     {\def\@noitemerr
       {\@latex@warning{Empty `thebibliography' environment}}%
      \endlist}

\renewcommand\@pnumwidth{2em}
\renewcommand*\l@chapter[2]{%
  \ifnum \c@tocdepth >\m@ne
    \addpenalty{-\@highpenalty}%
    \vskip 1.0em \@plus\p@
    \setlength\@tempdima{1.5em}%
    \begingroup
      \parindent \z@ \rightskip \@pnumwidth
      \parfillskip -\@pnumwidth
      \leavevmode \bfseries
      \advance\leftskip\@tempdima
      \hskip -\leftskip
      {\fontsize{9}{11}\selectfont #1}\nobreak\hfil \nobreak\hb@xt@\@pnumwidth{\hss {\fontsize{9}{11}\selectfont #2}}\par
      \penalty\@highpenalty
    \endgroup
  \fi}
\makeatother


