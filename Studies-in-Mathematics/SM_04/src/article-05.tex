\chapter[\textsc{A. Borel} and \textsc{J. Tits~:} On ``Abstract'' Homomorphisms of Simple Algebraic Groups]{ON ``ABSTRACT'' HOMOMORPHISMS OF SIMPLE ALGEBRAIC GROUPS}\label{art05}

\begin{center}
{\em By}~~ A. Borel ~{\em and}~ J. Tits
\end{center}

\lhead[\thepage]{\textit{On ``Abstract'' Homomorphisms of Simple Algebraic Groups}}
\rhead[\textit{A. Borel and J. Tits}]{\thepage}

\setcounter{pageoriginal}{74}
\textsc{This}\pageoriginale Note describes some results pertaining chiefly to homomorphisms of groups of rational points of semi-simple algebraic groups, and gives an application to a conjecture of Steinberg's \cite{art05-key9} on irreducible projective representations. Some proofs are sketched. Full details will be given elsewhere.

\begin{notation*}
The notation and conventions of \cite{art05-key1} are used. In particular, all algebraic groups are affine, $k$ is a commutative field, $\overline{k}$ an algebraic closure of $k$, $p$ its characteristic, and $G$ is a $k$-group. In this Note, $G$ is moreover assumed to be {\em connected}. $k'$ also denotes a commutative field.
\end{notation*}

Let $\phi:k\to k'$ be a (non-zero) homomorphism. We let ${}^{\phi}G$ be the $k'$-groups $\foprod{G}{k'}{k}$ obtained from $G$ by the change of basis $\phi$, and $\phi_{0}$ be the canonical homomorphism $G_{k}\to {}^{\phi}G_{k'}$ associated to $\phi$.

If $p\neq 0$, then $Fr^{i}$ denotes the $p^{i}$-th power homomorphism $\lambda-\lambda^{p^{i}}$ of a field of characteristic $p(i=0,1,2,\ldots)$. If $p=0$, $Fr^{i}$ is the identity.

A connected semi-simple $k$-group $H$ is {\em adjoint} if it is isomorphic to its image under the adjoint representation, {\em almost simple} (resp. {\em simple}) over $k$ if it has no proper normal $k$-subgroup of strictly positive dimension (resp. $\neq \{e\}$).

\section{Homomorphisms.}\label{art05-sec1}

\subsection{}\label{art05-sec1.1}
Let $G$ be semi-simple. $G^{+}$ will denote the subgroup of $G_{k}$ generated by the groups $U_{k}$, where $U$ runs through the unipotent radicals of the parabolic $k$-subgroups of $G$. The group $G^{+}$ is normal in $G_{k}$; it is $\neq \{e\}$ if and only if $\rk_{k}(G)>0$. If, moreover, $G$ is almost simple over $k$, then $G^{+}$ is Zariski-dense in $G$, and the quotient of $G^{+}$ by its center is simple except in finitely many cases where $k$ has two or thee elements \cite{art05-key10}. If $f:G\to H$ is a central $k$-isogeny, then $f(G^{+})=H^{+}$. The group $G^{+}$ is equal to $G_{k}$ if $k=\overline{k}$, or if $G$ is $k$-split and simple connected; it is conjectured\pageoriginale to be equal to $G_{k}$ if $G$ is simply connected and $\rk_{k}(G)>0$\cite{art05-key10}. It is always equal to its commutator subgroup.

\medskip
\noindent
{\bf Theorem \thnum{1.2}.\label{art05-thm1.2}\label{ART05-THM1.2}}
{\em Assume $k$ to be infinite, and $G$ to be almost absolutely simple, of strictly positive $k$-rank. Let $H$ be a subgroup of $G_{k}$ containing $G^{+}$. Let $k'$ be a commutative field, $G'$ a connected almost absolutely simple $k'$-group, and $\alpha:H\to G'_{k'}$ a homomorphism whose kernel does not contain $G^{+}$, and whose image contains ${G'}^{+}$. Assume finally that either $G$ is simply connected or $G'$ is adjoint. Then there exists an isomorphism $\phi:k\xrightarrow{\sim}k'$, $ak'$-isogeny $\beta:{}^{\phi}G\to G'$, and a homomorphism $\gamma$ of $H$ into the center of $G'_{k'}$ such that $\alpha(x)=\beta(\phi_{0}(x))\cdot \gamma(x)(x\in H)$. Moreover, $\beta$ is central, except possibly in the cases : $p=3$, $G$, $G'$ split of type $\bfG_{2}$; $p=2$, $G$, $G'$ split of type $\bfF_{4}$; $p=2$, $G$, $G'$ split of type $\bfB_{n}$, $\bfC_{n}$, where $\beta$ may be special.}
\smallskip

(The special isogenies are those discussed in \cite[Exp. 21-24]{art05-key3}.) In the following corollary, $G$ and $G'$ need not satisfy the last assumption of the theorem.

\medskip
\noindent
{\bf Corollary \thnum{1.3}.\label{art05-coro1.3}}
{\em Assume $G_{k}$ is isomorphic to $G'_{k'}$. Then $k$ is isomorphic to $k'$, and $G$, $G'$ are of the same isogeny class.}
\smallskip

Let $\overline{G}$ and $\overline{G}'$ be the adjoint groups of $G$ and $G'$. The assumption implies the existence of an isomorphism $\alpha:\overline{G}^{+}\xrightarrow{\sim}\overline{G'}^{+}$. By the theorem there is an isomorphism $\phi$ of $k$ onto $k'$ and an isogeny $\mu$ of ${}^{\phi}\overline{G}$ onto $\overline{G'}$, whence our assertion.

\begin{description}
\item[{\bf Remarks \thnum{1.4}.\label{art05-rem1.4}} {\rm(i)}] It may be that the homomorphism $\gamma$ in (\ref{art05-thm1.2}) is always trivial. It is obviously so if $G'$ is adjoint, or if $H$ is equal to its commutator subgroup. Since $G^{+}$ is equal to its commutator group, this condition will be fulfilled if $G$ is simply connected and the conjecture $G_{k}=G^{+}$ of \cite{art05-key10} is true, thus in particular if $G$ splits over $k$. Moreover, in that case the assumption $G^{+}\subset \ker \alpha$ would be superfluous.

\item[{\rm(ii)}] The theorem has been known in many special cases, starting with the determination of the automorphism group of the projective linear group \cite{art05-key7}. We refer to Dieudonn\'e's survey \cite{art05-key4} for the automorphisms of the classical groups. For split groups over infinite fields, see also \cite{art05-key6}.

\item[{\rm(iii)}] Assume\pageoriginale $k=k'$, $G=G'$, $G$ adjoint, and $k$ not to have any automorphism $\neq \Iid$. Theorem \ref{art05-thm1.2} implies then that every automorphism of $G_{k}$ is the restriction of an automorphism of $G$, which is then necessarily defined over $k$. In particular, if $k$ is the field of real numbers $\bfR$, every automorphism of $G_{k}$ is continuous in the ordinary topology, as was proved first by Freudenthal \cite{art05-key5}.

\item[{\rm(iv)}] The assumption $\rk_{k}G>0$ is essential for our proof, but it seems rather likely that similar results are valid for anisotropic groups. This is the case for many classical groups \cite{art05-key4}. Also, Freudenthal's proof is valid for compact groups. In fact, the continuity of any abstract-group automorphism of a compact semi-simple Lie group had been proved earlier, independently, by E. Cartan \cite{art05-key2} and van der Waerden \cite{art05-key11}. We note also that van der Waerden's proof remains valid in the $p$-adic case.

\item[{\rm(v)}] The group $\Aut G_{k}$ has also been studied when $k$ is finite. See \cite{art05-key4} for the classical groups, and \cite{art05-key8} for the general case.
\end{description}

\medskip
\noindent
{\bf Theorem \thnum{1.5}.\label{art05-thm1.5}}
{\em Assume $k$ to be infinite, and $G$ to be almost simple, split over $k$. Let $G'$ be a semi-simple split $k'$-group, $G'_{i}(1\leq i\leq s)$ the almost simple normal subgroups of $G'$, and $\alpha : G_{k}\to G'_{k'}$ a homomorphism whose image is Zariski-dense. If $G_{k}=G^{+}$, then $G'$ is connected. Assume $G'$ to be connected and either $G$ simply connected or $G'$ adjoint. Then there exist homomorphisms $\phi_{i}:k\to k'$ and $k'$-isogenies $\beta_{i}:\phi_{i}G\to G'_{i}(1\leq i\leq s)$, which are either central or special, such that}
$$
\alpha(x)=\prod\limits_{i}(\beta_{i}\circ \phi_{i,0})(x),\quad (x\in G_{k}).
$$
{\em Moreover, $Fr^{a}\circ \phi_{i}\neq Fr^{b}\circ \phi_{j}$ if $p=0$ and $i\neq j$, or if $p\neq 0$ and $(a,i)\neq (b,j)(1\leq i, j\leq s;a,b=0,1,2,\ldots)$.}
\smallskip

The proof of Theorem \ref{art05-thm1.5} goes more or less along the same lines as that of Theorem \ref{art05-thm1.2}. In fact, it seems not unlikely that Theorem \ref{art05-thm1.5} can be generalized so as to contain Theorem \ref{art05-thm1.2}. We hope to come back to this question on another occasion.

\medskip
\noindent
{\bf Example \thnum{1.6}.\label{art05-exam1.6}}
The\pageoriginale following example, which admits obvious generalizations, shows that the assumption of semi-simplicity made on $G'$ in Theorem \ref{art05-thm1.5} cannot be dropped.
\smallskip

Let $G=\mathbf{SL}_{2}$, and $N$ be the additive group of $2\times 2$ matrices over $\overline{k}$, of trace zero. Let $d$ be a non-trivial derivation of $k$. Extend it to a derivation of $N_{k}$ by letting it operate on the coefficients, and define $h:G_{k}\to N_{k}$ by $h(g)=g^{-1}\cdot dg$. Let $G'=G\cdot N$ be the semi-direct product of $G$ and $N$, where $G$ acts on $N$ by the adjoint representation. Then $g\to (g,h(g))$ is easily checked to be a homomorphism of $G_{k}$ into $G'_{k}$ with dense image; clearly, it defines an ``abstract'' Levi section of $G'_{k}$.

\section{Projective representations.}\label{art05-sec2}

\subsection{}\label{art05-sec2.1}
Assume $p\neq 0$. For $G$ semi-simple, let $\mathscr{R}$ or $\mathscr{R}(G)$ be the set of $p^{l}(l=\rank G)$ irreducible projective representations whose highest weight is a linear combination of the fundamental highest weights with coefficients between 0 and $p-1$. The following theorem, in a slightly different formulation, was conjectured by R. Steinberg \cite{art05-key9}, for $k=\overline{k}$. We show below how it follows from Theorem \ref{art05-thm1.5} and \cite{art05-key9}, (Theorem 1.1).

\medskip
\noindent
{\bf Theorem \thnum{2.2}.\label{art05-thm2.2}}
{\em Assume $k$ to be infinite, $p\neq 0$, and $G$ $k$-split, simple, adjoint. Let $\pi:G^{+}\to \mathbf{PGL}(n,\overline{k})$ be an irreducible (not necessarily rational) projective representation of $G^{+}$. Then there exist distinct homomorphisms $\phi_{j}:k\to \overline{k}$, and elements $\pi_{j}\in \mathscr{R}({}^{\phi_{j}}G)(1\leq j\leq t)$, such that $\pi:\prod\limits_{j}\pi_{j}\circ \phi_{j,0}$.}
\smallskip

\begin{proof}
Let $G'$ be the Zariski-closure of $\pi(G_{k})$ in $\mathbf{PGL}_{n}$. It is also an irreducible projective linear group, hence its center and also its centralizer in $\mathbf{PGL}_{n}$, or in the Lie algebra of $\mathbf{PGL}_{n}$, are reduced to $\{e\}$. Thus $G'$ is semi-simple, and its identity component is adjoint. Moreover, by Theorem \ref{art05-thm1.5}, $G'$ is connected. By Theorem 1.1 of \cite{art05-key9}, there exist elements $\pi_{a}\in \mathscr{R}(G')$, $(1\leq a\leq q)$ such that the identity representation of $G'$ is equal to $\prod\limits_{a}\pi_{a}\circ (Fr^{a})$. Let $G'_{i}(1\leq i\leq s)$ be the simple factors of $G'$.

The tensor product defines a bijection of $\mathscr{R}(G'_{1})\times\cdots\times \mathscr{R}(G'_{s})$ onto $\mathscr{R}(G')$. We may therefore write
$$
\pi_{a}=\prod\limits_{1\leq i\leq s}\pi_{a,i},\quad (\pi_{ai}\in \mathscr{R}(G'_{i}); 1\leq a\leq q).
$$\pageoriginale
Let now $\phi_{i}:k\to \overline{k}$ and $\beta_{i}:{}^{\phi_{i}}G\to G_{i}$ be as in Theorem \ref{art05-thm1.5} (with $\overline{k}=k'$). We have then
\begin{equation*}
\pi=\prod\limits_{a,i}\pi_{a,i}\circ (Fr^{a})_{0}\circ \beta_{i}\circ \phi_{i,0}.\tag{1}\label{art05-thm2.2-eq1}
\end{equation*}
But $(Fr^{a})_{0}\circ \beta_{i}=\beta_{a,i}\circ (Fr^{a})_{0}$, where $\beta_{a,i}$ is the transform of $\beta_{i}$ under $Fr^{a}$. Let $\phi_{a,i}=Fr^{a}\circ \phi_{i}$. Since $G$, $G_{i}$ are adjoint, the morphisms $\beta_{a,i}$ are either isomorphisms or special isogenies. Therefore, taking (\cite{art05-key9}, \S11) into account, we see that
$$
\pi'_{a,i}=\pi_{a,i}\circ \beta_{a,i}\in \mathscr{R}({}^{\phi_{a,i}}(G)),\quad (1\leq i\leq s; 1\leq a\leq q),
$$
and \eqref{art05-thm2.2-eq1} yields
\begin{equation*}
\pi=\prod\limits_{a,i}\pi'_{a,i}\circ (\phi_{a,i})_{0},\tag{2}\label{art05-thm2.2-eq2}
\end{equation*}
which proves the theorem, in view of the fact that the $\phi_{a,i}$ are distinct by Theorem \ref{art05-thm1.5}.
\end{proof}

\section{Sketch of the proof of Theorem \ref{art05-thm1.2}.}\label{art05-sec3}

In this paragraph, $k$ is infinite and $G$ is semi-simple, of strictly positive $k$-rank.

The two following propositions are the starting point of the proofs of Theorem \ref{art05-thm1.2} and Theorem \ref{art05-thm1.5}.

\medskip
\noindent
{\bf Proposition \thnum{3.1}.\label{art05-prop3.1}}
{\em Let $G'$ be a $k'$-group, and $\alpha:G^{+}\to G'_{k'}$ a non-trivial homomorphism. Let $P$ be a minimal parabolic $k$-subgroup of $G$ and $U$ its unipotent radical. Then $\alpha(U_{k})$ is a unipotent subgroup contained in the identity component of $G'$ and $\alpha(G^{+})\subset {G'}^{0}$. The field $k'$ is also of characteristic $p$.}
\smallskip

Let $S$ be a maximal $k$-split torus of $P$. It is easily seen that $S^{+}=S\cap G^{+}$ is dense in $S$. It follows then from \cite[\S11.1]{art05-key1} that any subgroup of finite index of $S^{+}\cdot U_{k}$ contains elements $s\in S^{+}$ such that $(s,U_{k})=U_{k}$. From this we deduce first that $U_{k}$ is contained in any normal subgroup of finite index of $S^{+}\cdot U_{k}$, and then, that it is also contained in the commutator subgroup of any such subgroup. It follows that $\alpha(U_{k})$ is contained in the derived group of the identity\pageoriginale component of the Zariski closure of $\alpha(S^{+}\cdot U_{k})$. The latter being solvable, this implies that $\alpha(U_{k})$ is unipotent.

Let $p'=\text{char.~} k'$. If $p\neq 0$, then $U_{k}$ is a $p$-group. Its image is a $p$-group and is $\neq \{e\}$ since $\alpha$ is non-trivial, and $G^{+}$ is generated by the conjugates of $U_{k}$; hence $p=p'$. If $p=0$ and $p'\neq 0$, then $\ker \alpha\cap U_{k}$ has finite index in $U$, whence easily a contradiction with the main theorem of \cite{art05-key10}.

\medskip
\noindent
{\bf Proposition \thnum{3.2}.\label{art05-prop3.2}}
{\em Let $G'$ be a connected semi-simple $k'$-group. Let $P$, $S$, $U$ be as above, $P^{-}$ the parabolic $k$-subgroup opposed to $P$ and containing $\mathscr{Z}(S)$, and $U^{-}=R_{u}(P^{-})$. Let $H$ be a subgroup of $G_{k}$ containing $G^{+}$ and $\alpha : H\to G'_{k'}$ be a homomorphism with dense image. Then the Zariski-closures $Q$, $Q^{-}$ of $\alpha(P\cap H)$ and $\alpha(P^{-}\cap H)$ are two opposed parabolic $k'$-subgroups, and $Q\cap Q^{-}$, $R_{u}(Q)$, $R_{u}(Q^{-})$ are the Zariski-closures of $\alpha(Z(S)\cap H),\alpha(U_{k})$ and $\alpha(U^{-}_{k})$ respectively.}
\smallskip

Let $M$, $V$, $V^{-}$ be the Zariski-closures of $\alpha(\mathscr{Z}(S)\cap H)$, $\alpha(U_{k})$ and $\alpha(U^{-}_{k})$ respectively. The groups $V$, $V^{-}$ are unipotent, by Proposition \ref{art05-prop3.1}. The group $G$ is the union of finitely many left translates of $U^{-}\cdot P$. Since $\alpha(H)$ is dense, this implies that $V^{-}\cdot M\cdot V$ contains a non-empty open subset of $G'$. Let $T$ be a maximal torus of $M$ and $Y$, $Y^{-}$ be two maximal unipotent subgroups of $M^{0}$ normalized by $T$ such that $Y^{-}\cdot T\cdot Y$ is open in $M^{0}$ (see \cite{art05-key1}, \S2.3, Remarque). Then $V^{-}\cdot Y^{-}$ and $Y\cdot V$ are unipotent subgroups of $G'$ normalized by $T$ and $V^{-}\cdot Y^{-}\cdot T\cdot Y\cdot V$ contains a non-empty open set of $G'$. Consequently (\cite{art05-key1}, \S2.3), $T$ is a maximal torus of $G'$, and $V^{-}\cdot Y^{-}$, $Y\cdot V$ are two opposed maximal unipotent subgroups. This shows that $Q$, $Q^{-}$ are parabolic subgroups, $M$ is reductive, connected, and $V=R_{u}(Q)$, (resp. $V^{-}=R_{u}(Q^{-})$). The groups $Q$, $Q^{-}$ are obviously $k'$-closed. Arguing as in Proposition \ref{art05-prop3.1}, we may find $s\in S\cap H$ such that $(s,U_{k})=U_{k}$, $(s,U^{-}_{k})=U^{-}_{k}$. It follows then from (\cite{art05-key1}, \S11.1) that $\mathscr{Z}(\alpha(s))^{0}=M$. Hence $M$ is defined over $k'$(\cite{art05-key1}, \S10.3). By Grothendieck's theorem (\cite{art05-key1}, \S2.14), it contains a maximal torus defined over $k'$. Hence (\cite{art05-key1}, \S3.13), $Q$, $Q^{-}$, $V$, $V^{-}$ are defined over $k'$.

\setcounter{subsection}{2}
\subsection{}\label{art05-sec3.3}
We now sketch the proof of Theorem \ref{art05-thm1.2}, assuming for simplicity that $G$, $G'$ are adjoint and $H=G_{k}$. Then $\alpha$ is injective. Proposition\pageoriginale \ref{art05-prop3.2}, applied to $\alpha$ and $\alpha^{-1}$, shows that $Q$, $Q^{-}$ are two opposed minimal parabolic $k'$-subgroups of $G'$. Consequently, $\alpha$ induces an isomorphism of $\mathscr{N}(S)/\mathscr{Z}(S)$ onto $\mathscr{N}(M)/M$, i.e. of ${}_{k}W(G)={}_{k}W$ onto ${}_{k'}W'={}_{k'}W(G')$. For $a\in {}_{k}\Phi(G)$, let $U_{a}=U_{(a)}/U_{(2a)}$, where we put $U_{(2a)}=\{e\}$ if $2a\not\in {}_{k}\Phi$. It may be shown that $U_{(2a)}$ is the center of $U_{(a)}$. The groups $U^{-}_{(a)}$ may be characterized as minimal among the intersections $U\cap w(P)(w\in {}_{k}W)$ not reduced to $\{e\}$. It then follows that $\alpha$ induces a bijection $\alpha_{*}:{}_{k}\Phi(G)\to {}_{k'}\Phi(G')$ preserving the angles, and isomorphisms $U_{a,k}\xrightarrow{\sim}V_{\alpha *(a),k'}$. The group $U_{a}$ (resp. $V_{\alpha * (a)}$) may be endowed canonically with a vector space structure such that $S$ (resp. a maximal $k'$-split torus $S'$ of $M$) acts on it by dilatations. The next step is to show that $\alpha:U_{a,k}\xrightarrow{\sim}V_{\alpha *(a),k'}$ induces a bijection $\phi_{a}$ between the algebras of dilatations. Let $L_{a}$ be the subgroup of $G$ generated by $U_{(a)}$ and $U_{(-a)}$. The assumption that $G$ is almost absolutely simple is equivalent to the existence of one element $a\in {}_{k}\Phi$ such that the intersection $X_{a}$ of $L_{a}$ with the center $C$ of $\mathscr{Z}(S)$ is one-dimensional, hence such that $X^{0}_{a}\subset S$. This is the main tool used in showing that $\alpha(S_{k})\subset S'_{k'}$, hence that $\alpha$ maps dilatations by elements of $(k^{*})^{2}$ into dilatations. If $p\neq 2$, this suffices to yield the existence of $\phi_{a}:k\xrightarrow{\sim}k'$. In characteristic two, some further argument, based on properties of groups of rank one, is needed. It is clear that $\phi_{a}=\phi_{b}$ if $b\in {}_{k}W(a)$. Using further some facts about commutators, it is then easily proved that $\phi_{a}=\phi_{b}(a,b\in{}_{k}\Phi)$ if $\alpha_{*}$ preserves the lengths. If not, we show that we are in one of the exceptional cases listed in the theorem, and we reduce it to the preceding one by use of a special isogeny. Write then $\phi$ instead of $\phi_{a}$. Replacing $G$ by ${}^{\phi}G$, we may assume $k=k'$, $\phi=\Iid$. It is then shown that $\alpha:U_{k}\xrightarrow{\sim}V_{k}$ is the restriction of a $k$-isomorphism of varieties. On the other hand, since $G'$ is adjoint, $\mathscr{Z}(S')$ is isomorphic to its image in $GL(\bfb)$ under the adjoint representation, where $\bfb$ is the sum of Lie algebras of the $V_{a'}(a'\in{}_{k}\Phi(G'))$. This implies readily that the restriction of $\alpha$ to $U^{-}_{k}\cdot P_{k}$ is the restriction of a $k$-isomorphism of varieties of $U^{-}\cdot P$ onto $V^{-}\cdot Q$. The conclusion then follows readily from the fact that $G$ is a finite union of translates $x\cdot U^{-}\cdot P(x\in G_{k})$.

\begin{thebibliography}{99}
\bibitem{art05-key1} \textsc{A. Borel}\pageoriginale and \textsc{J. Tits :} Groupes r\'eductifs, {\em Publ. Math. I.H.E.S.} 27 (1965), 55-150.

\bibitem{art05-key2} \textsc{E. Cartan :} Sur les repr\'esentations lin\'eaires des groupes clos, {\em Comm. Math. Helv.} 2 (1930), 269-283.

\bibitem{art05-key3} \textsc{C. Chevalley :} {\em S\'eminaire sur la classification des groupes de Lie alg\'ebriques,} 2 vol., Paris 1958 (mimeographed Notes).

\bibitem{art05-key4} \textsc{J. Dieudonn\'e :} {\em La g\'eom\'etrie des groupes classiques,} Erg. d. Math. u. Grenzg. Springer Verlag, 2nd edition, 1963.

\bibitem{art05-key5} \textsc{H. Freudenthal :} Die Topologie der Lieschen Gruppen als algebraisches Ph\"anomen I, {\em Annals of Math.} (2) 42 (1941), 1051-1074. Erratum {\em ibid.} 47 (1946), 829-830.

\bibitem{art05-key6} \textsc{J. Humphreys :} On the automorphisms of infinite Chevalley groups (preprint).

\bibitem{art05-key7} \textsc{O. Schreier} und B. L. v. d. \textsc{Waerden :} Die Automorphismen der projektiven Gruppen, {\em Abh. Math. Sem. Hamburg Univ.} 6 (1928), 303-322.

\bibitem{art05-key8} \textsc{R. Steinberg :} Automorphisms of finite linear groups, {\em Canadian J. M.} 12 (1960), 606-615.

\bibitem{art05-key9} \textsc{R. Steinberg :} Representations of algebraic groups, {\em Nagoya Math. J.} 22 (1963), 33-56.

\bibitem{art05-key10} \textsc{J. Tits :} Algebraic and abstract simple groups, {\em Annals of Math.} (2) 80 (1964), 313-329.

\bibitem{art05-key11} B. L. v. d. \textsc{Waerden :} Stetigkeitss\"atze f\"ur halb-einfache Liesche Gruppen, {\em Math. Zeit.} 36 (1933), 780-786.

\end{thebibliography}

\medskip
\noindent
{\small The Institute for Advanced Study, Princeton, N. J.}

\noindent
{\small Universit\"at Bonn.}
