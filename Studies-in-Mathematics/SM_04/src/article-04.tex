\chapter[\textsc{A. Borel} : On the Automorphisms of Certain Subgroups of Semi-Simple Lie Groups]{ON THE AUTOMORPHISMS OF CERTAIN SUBGROUPS OF SEMI-SIMPLE LIE GROUPS}\label{art04}

\begin{center}
{\em By}~~ Armand Borel
\end{center}

\lhead[\thepage]{\textit{On the Automorphisms of Certain Subgroups...}}
\rhead[\textit{A. Borel}]{\thepage}

\setcounter{pageoriginal}{42}
\textsc{Let}\pageoriginale $L$ be a group. We denote by $E(L)$ the quotient group $\Aut L/\Int L$ of the group $\Aut L$ of automorphisms of $L$ by the group $\Int L$ of inner automorphisms $\Int a:x\to a.x.a^{-1}(a,x\in L)$ of $L$. Our first aim is to show that $E(L)$ is finite if $L$ is arithmetic, $S$-arithmetic (see \ref{art04-sec3.2}) or uniform in a semi-simple Lie group (with some exceptions, see Theorem \ref{art04-thm1.5}, Theorem \ref{art04-thm3.6} and Theorem \ref{art04-thm5.2} for the precise statements). A slight variant of the proof also shows that in these cases $L$ is not isomorphic to a proper subgroup of finite index. As a consequence, a Riemannian symmetric space with negative curvature, and no flat component, has infinitely many non-homeomorphic compact Clifford-Klein forms, Theorem \ref{art04-thm6.2}.

\smallskip

Further information on $E(L)$ is obtained when $L$ is an $S$-arithmetic group of a semi-simple $k$-group $G$ (with some conditions on $G$ and $S$). If $L$ contains the center of $G_{k}$, and $G$ is simply connected, then $E(L)$ is essentially generated by four kinds of automorphisms: exterior automorphisms of $G$, automorphisms deduced from certain automorphisms of $k$, automorphisms of the form $x\to f(x)\cdot x$ where $f$ is a suitable homomorphism of $L$ into the center of $G_{k}$, and automorphisms induced by the normalizer $N(L)$ of $L$ in $G$ (see Lemma \ref{art04-lem1.8}, Remark in \ref{art04-thm1.9}). In the case where $G$ is split and $L$ is the group of $\mathfrak{o}(S)$-points of $G$ for its canonical integral structure, there results are made more precise (Theorem \ref{art04-thm2.2}, Theorem \ref{art04-thm4.3}), and $N(L)/L$ is put into relation with the $S$-ideal class group of $k$ and $S$-units (see Lemma \ref{art04-lem2.3}, Lemma \ref{art04-lem4.5}; these results overlap with those of Allan \cite{art04-key1}, \cite{art04-key2}). As an illustration, we discuss $\Aut L$ for some classical groups (Examples \ref{art04-exams2.6}, \ref{art04-exams4.6}). The results are related to those of O'Meara \cite{art04-key24} if $G=\SL_{n}$, and of Hua-Reiner \cite{art04-key12}, \cite{art04-key13} and Reiner \cite{art04-key27} if $G=\SL_{n}$, $\Sp_{2n}$, and $k=\bfQ$, $\bfQ(i)$.

\eject

The\pageoriginale finiteness of $E(L)$ follows here rather directly from rigidity theorems \cite{art04-key25}, \cite{art04-key26}, \cite{art04-key32}, \cite{art04-key33}. The connection between the two is established by Lemma \ref{art04-lem1.1}.

\begin{notation*}
In this paper, all algebraic groups are linear, and we follow in general the notations and conventions of \cite{art04-key9}. In particular, we make no notational distinction between an algebraic group $G$ over a field $k$ and its set of points in an algebraic closure $\overline{k}$ of the field of definition (usually $\bfC$ here). The Lie algebra of a real Lie group or of an algebraic group is denoted by the corresponding German letter.
\end{notation*}

If $L$ is a group and $V$ a $L$-module, then $H^{1}(L,V)$ is the 1st cohomology group of $L$ with coefficients in $V$. In particular, if $L$ is a subgroup of the Lie group $G$, then $H^{1}(L,\mathfrak{g})$ is the 1st cohomology group of $L$ with coefficients in the Lie algebra $\mathfrak{g}$ of $G$, on which $L$ operates by the adjoint representation.

A closed subgroup $L$ of a topological group $G$ is {\em uniform} if $G/L$ is compact.

\section{Uniform or arithmetic subgroups.}\label{art04-sec1}

\noindent
{\bf Lemma \thnum{1.1}.\label{art04-lem1.1}}
{\em Let $G$ be an algebraic group over $\bfR$, $L$ a finitely generated subgroup of $G_{\bfR}$ and $N$ the normalizer of $L$ in $G_{\bfR}$. Assume that $H^{1}(L,\mathfrak{g})=0$. Then the group of automorphisms of $L$ induced by elements of $N$ has finite index in $\Aut L$.}
\smallskip

Let $L_{0}$ be a group isomorphic to $L$ and $\iota$ an isomorphism of $L_{0}$ onto $L$. For $M=G_{\bfR}$, $G$, let $R(L_{0},M)$ be the set of homomorphisms of $L_{0}$ into $M$. Let $(x_{i})(1\leq i\leq q)$ be a generating set of elements of $L$. Then $R(L_{0},M)$ may be identified with a subset of $M^{q}$, namely, the set of $m$-uples $(y_{i})$ which satisfy a set of defining relations for $L_{0}$ in the $x_{i}$, $x_{i}^{-1}$. In particular $R(L_{0},G)$ is an affine algebraic set over $\bfR$, whose set of real points is $R(L_{0},G_{\bfR})$. The group $M$ operates on $R(L_{0},M)$, by composition with inner automorphisms, and $G$ is an algebraic transformation group of $R(L_{0},G)$, with action defined over $\bfR$.

To\pageoriginale $\alpha\in \Aut L$, let us associate the element $j(\alpha)=\alpha\circ \iota$ of $R(L_{0},G)$. The map $j$ is then a bijection of $\Aut L$ onto the set $I(L_{0},L)\subset R(L_{0},G_{\bfR})$ of isomorphisms of $L_{0}$ onto $L$. If $a$, $b\in I(L_{0},L)$, then $b\in G_{\bfR}(a)$ if and only if there exists $n\in N$ such that $b=(\Int n)\circ a$. Our assertion is therefore equivalent to: ``$I(L_{0},L)$ is contained in finitely many orbits of $G_{\bfR}$,'' which we now prove.

Let $b\in I(L_{0},L)$. Since
$$
H^{1}(b(L^{0}),\mathfrak{g})=H^{1}(L,\mathfrak{g})=H^{1}(L,\mathfrak{g}_{\bfR})\otimes \bfC,
$$
we also have $H^{1}(b(L_{0}),\mathfrak{g})=0$. By the lemma of \cite{art04-key33}, it follows that the orbit $G(b)$ contains a Zariski-open subset of $R(L_{0},G)$. Since the latter is the union of finitely many irreducible components, this shows that $I(L_{0},L)$ is contained in finitely many orbits of $G$. But an orbit of $G$ containing a real point can be identified to a homogeneous space $G/H$ where $H$ is an algebraic subgroup of $G$, defined over $\bfR$. Therefore its set of real points is the union of finitely many orbits of $G_{\bfR}$(\cite{art04-key8}, \S6.4), whence our contention.

\medskip
\begin{description}
\item[{\bf Remark \thnum{1.2}.\label{art04-rem1.2}} {\rm(i)}]
The lemma and its proof remain valid if $\bfR$ and $\bfC$ are replaced by a locally compact field of characteristic zero $K$ and an algebraically closed extension of $K$.

\item[\rm(ii)] The group $\SL(2,\bfZ)$ has a subgroup of finite index $L$ isomorphic to the free group on $m$ generators, where $m\geq 2$ (and in fact may be taken arbitrarily large). The group $E(L)$ has the group $\GL(m,\bfZ)=\Aut (L/(L,L))$ as a quotient, hence is infinite. On the other hand, $L$ has finite index in its normalizer in $\SL(2,\bfC)$, as is easily checked (and follows from Proposition \ref{art04-prop3.3}(d)). Thus, \ref{art04-lem1.1} implies that $H^{1}(L,\mathfrak{g})\neq 0$, as is well known. Similarly, taking (i) into account, we see that the free uniform subgroups of $\PSL(2,\bfQ_{p})$ constructed by Ihara \cite{art04-key15} have non-zero first cohomology group with coefficients in $\mathfrak{g}$.
\end{description}

\noindent
{\bf Lemma \thnum{1.3}.\label{art04-lem1.3}}
{\em Let $G$ be an algebraic group over $\bfR$, $L$ a finitely generated discrete subgroup of $G_{\bfR}$ such that $G_{\bfR}/L$ has finite invariant measure. Assume that $H^{1}(L',\mathfrak{g})=0$ for all subgroups of finite index $L'$ of $L$. Then $L$ is not isomorphic to a proper subgroup of finite index.}
\smallskip

Define\pageoriginale $L_{0}$, $M$, $\iota$, $R(L_{0},M)$ and the action of $M$ on $R(L_{0},M)$ as in the proof of Lemma \ref{art04-lem1.1}. Let $C$ be the set of monomorphisms of $L_{0}$ onto subgroups of finite index of $L$. Then, $j:\alpha\mapsto \alpha\circ \iota$ is a bijection of $C$ onto a subset $J$ of $R(L_{0},G_{\bfR})$, and the argument of Lemma \ref{art04-lem1.1} shows that $J$ is contained in the union of finitely many orbits of $G_{\bfR}$. Fix a Haar measure on $G_{\bfR}$, and hence on all quotients of $G_{\bfR}$ by discrete subgroups. The total measure $m(G_{\bfR}/L')$ is finite for every subgroup of finite index of $L$, since $m(G_{\bfR}/L)$ is finite. If $b$, $c\in C$, and $b=\Int g\circ c$, $(g\in G_{\bfR})$, then $m(G_{\bfR}/b(L))=m(G_{\bfR}/c(L))$. Consequently, $m(G_{\bfR}/L')$ takes only finitely many values, as $L'$ runs through the subgroups of finite index of $L$, isomorphic to $L$. But, if there is one such group $L'\neq L$, then there is one of arbitrary high index in $L$, a contradiction.

\medskip
\noindent
{\bf Lemma \thnum{1.4}.\label{art04-lem1.4}}
{\em Let $L$ be a finitely generated group, $M$ a normal subgroup of finite index, whose center is finitely generated, $N$ a characteristic finite subgroup of $L$.}
\begin{itemize}
\item[{\rm(a)}] {\em If $E(M)$ is finite, then $E(L)$ is finite.}

\item[{\rm(b)}] {\em If $E(L/N)$ is finite, then $E(L)$ is finite.}
\end{itemize}

\begin{itemize}
\item[(a)] It is well known and elementary that a finitely generated group contains only finitely many subgroups of a given finite index (see e.g. \cite{art04-key11}). Therefore, the group $\Aut (L,M)$ of automorphisms of $L$ leaving $M$ stable has finite index in $\Aut(L)$. Since $L/M$ is finite, the subgroup $Q$ of elements of $\Aut(L,M)$ inducing the identity on $L/M$ has also finite index. Let $r:Q\to \Aut M$ be the restriction map. Our assumption implies that $r^{-1}(\Int M)$ has finite index in $Q$, hence that $\Int L. \ker r$ is a subgroup of finite index of $Q. \Int L$. It suffices therefore to show that $\ker r\cap \Int L$ has finite index in $\ker r$. Let $b\in \ker r$. Write
$$
b(x)=u_{x}\cdot x(x\in L).
$$
Then $u_{x}\in M$ and routine checking shows: the map $u:x\mapsto u_{x}$ is a 1-cocycle on $L$, with coefficients in the center $C$ of $M$, which is constant on the cosets mod $M$, and may consequently be viewed as a $1$-cocycle of $L/M$ with coefficients in $C$; furthermore, two cocycles thus associated to elements $b$, $c\in Q$ are cohomologous if and only if there\pageoriginale exists $n\in C$ such that $b=\Int n\circ c$, and any such cocycle is associated to an automorphism. Therefore
$$
\ker r/\Int_{L}C\cong H^{1}(L/M,C).
$$
By assumption, $L/M$ is finite, and $C$ is finitely generated. Hence the right hand side is finite, which implies our assertion.

\item[(b)] The group $N$ being characteristic, we have a natural homomorphism $\pi:\Aut L\to \Aut L/N$. The finiteness of $E(L/N)$ implies that $\Int L.\ker \pi$ has finite index in $\Aut L$. Moreover $\ker \pi$ consists of automorphisms of the form $x\mapsto x\cdot v_{x}$, $(x\in L, v_{x}\in N)$, and is finite, since $N$ is finite and $L$ is finitely generated.
\end{itemize}

\medskip
\noindent
{\bf Theorem \thnum{1.5}.\label{art04-thm1.5}}
{\em Let $G$ be a semi-simple Lie group, with finitely many connected components, whose identity component $G^{0}$ has a finite center, and $L$ a discrete subgroup of $G$. Then $E(L)$ is finite if one of the two following conditions is fulfilled:}
\begin{itemize}
\item[{\rm(a)}] {\em $G/L$ is compact, $G^{0}$ has no non-compact three-dimensional factor;}

\item[{\rm(b)}] {\em $\Aut(\mathfrak{g}\otimes \bfC)$ may be identified with an algebraic group $G'$ over $\bfQ$, such that the image $L'$ of $L\cap G$ in $G'$ by the natural projection is an arithmetic subgroup of $G'$, and $G'_{\bfR}$ has no factor locally isomorphic to $\SL(2,\bfR)$ on which the projection of $L'$ is discrete.}
\end{itemize}

\begin{itemize}
\item[(a)] Let $A$ be the greatest compact normal subgroup of $G^{0}$ and $\pi:G^{0}\to G^{0}/A$ the canonical projection. Since the center of $G^{0}$ is finite, it is contained in $A$, and $G^{0}/A$ is the direct product of non-compact simple groups with center reduced to $\{e\}$. The group $\pi(L\cap G^{0})$ is discrete and uniform in $G^{0}/A$. By density \cite{art04-key4}, its center is contained in the center of $G^{0}/A$, hence is reduced to $\{e\}$. Consequently, the center of $L\cap G^{0}$ is contained in $A\cap L$, hence is finite. We may then apply Lemma \ref{art04-lem1.4}(a), which reduces us to the case where $G$ is connected. Moreover, by \cite{art04-key4}, any finite normal subgroup of $\pi(L)$ is central in $G/A$, hence reduced to $\{e\}$. Therefore $A\cap L$ is the greatest finite normal subgroup of $L$, and is characteristic. By Lemma \ref{art04-lem1.4}(b), it suffices to show that $E(\pi(L))$ is finite. We are thus reduced to the case where $G$ has no\pageoriginale center, and is a direct product of non-compact simple groups of dimension $>3$. In particular, $G$ is of finite index in the group of real points of an algebraic group defined over $\bfR$, namely $\Aut (\mathfrak{g}\otimes \bfC)$. By a theorem of Weil \cite{art04-key33}, $H^{1}(L,\mathfrak{g})=0$, hence by Lemma \ref{art04-lem1.1}, it is enough to show that $L$ has finite index in its normalizer $N(L)$. By \cite{art04-key4}, $N(L)$ is discrete. Since $G/L$ is fibered by $N(L)/L$, and is compact, $N(L)/L$ is finite.

\item[(b)] Let $A$ be the greatest normal $\bfQ$-subgroup of ${G'}^{0}$, whose group of real points is compact, and let $\pi$ be the composition of the natural homomorphisms
$$
G^{0}\to \Ad \mathfrak{g}\to ({G'}^{0})_{\bfR}/A_{\bfR}.
$$
$G'/A$ is a $\bfQ$-group without center, which is a product of $\bfQ$-simple groups, each of which has dimension $>3$ and a non-compact group of real points. $\pi(L)$ is arithmetic in $G'/A$(\cite{art04-key6}, Theorem 6) and $L\cap \ker \pi$ is finite. By Zariski-density (\cite{art04-key6}, Theorem 1), any finite normal subgroup of $\pi(L)$ is central in $G'/A$, hence reduced to $\{e\}$. Thus $L\cap \ker \pi$ is the greatest finite normal subgroup of $L$, and is characteristic. Also the center of $\pi(L\cap G^{0})$ is central in $\Ad G'$, hence reduced to $\{e\}$, and the center of $L\cap G^{0}$ is compact, and therefore finite. By Lemma \ref{art04-lem1.4}, we are thus reduced to the case where $G$, $G'$ are connected, $A=\{e\}$, and $L$ is arithmetic in $G'$. Let $G_{1},\ldots,G_{q}$ be the simple $\bfQ$-factors of $G'$. The group $L$ is commensurable with the product of the intersections $L_{i}=L\cap G_{i}$, which are arithmetic (\cite{art04-key7}, 6.3). If $G_{i}\cap L$ is uniform in $G_{i\bfR}$, then $H^{1}(L_{i},\mathfrak{g}_{\bfR})=0$ by \cite{art04-key33}. If not, then $rk_{Q}G_{i}\geq 1$, and $H^{1}(L_{i},\mathfrak{g}_{i})=0$ by theorems of Raghunathan \cite{art04-key25}, \cite{art04-key26}. Consequently, $H^{1}(L\cap G',\mathfrak{g}'_{\bfR})=0$. Moreover, $L\cap G'$ is of finite index in its normalizer $N$. In fact, $N$ is closed, belongs to $G'_{\bfQ}$ (\cite{art04-key6}, Theorem 2), hence to $G'_{\bfR}$, and $N/(L\cap G')$ is compact since $G'_{\bfR}/(L\cap G')$ has finite invariant measure \cite{art04-key7}. The conclusion now follows from Lemma \ref{art04-lem1.1} and Lemma \ref{art04-lem1.4}.
\end{itemize}

\medskip
\noindent
{\bf Lemma \thnum{1.6}.\label{art04-lem1.6}}
{\em Let $L$ be a finitely generated group, $N$ a normal subgroup of finite index. Assume $L$ is isomorphic to a proper subgroup of finite index. Then $N$ has a subgroup of finite index which is isomorphic to a proper subgroup of finite index.}
\smallskip

The\pageoriginale assumption implies the existence of a strictly decreasing sequence $(L_{i})(i=1,2,\ldots)$ of subgroups of finite index of $L$ and of isomorphisms $f_{i}:L\xrightarrow{\sim}L_{i}(i=1,2,\ldots)$. Let $a$ be the index of $N$ in $L$. Then $L_{i}\cap N$ has index $\leq a$ in $L_{i}$, hence $M_{i}=f^{-1}_{i}(L_{i}\cap N)$ has index $\leq a$. Passing to a subsequence if necessary, we may assume that $M_{i}$ is independent of $i$. Then, $L_{i}\cap N$ is isomorphic to a proper subgroup of finite index.

\medskip
\noindent
{\bf Proposition \thnum{1.7}.\label{art04-prop1.7}}
{\em Let $G$ and $L$ be as in Theorem \ref{art04-thm1.5}. Assume one of the conditions (a), (b) of Theorem \ref{art04-thm1.5} to be fulfilled. Then $L$ is not isomorphic to a proper subgroup of finite index.}
\smallskip

By use of Lemma \ref{art04-lem1.6}, the proof is first reduced to the case where $G$ is connected. Let $\pi$ be as in the proof of (a) or (b) in Theorem \ref{art04-thm1.5}. Then $L\cap \ker \pi$ is the greatest finite normal subgroup of $L$. Similarly $L'\cap \ker \pi$ is the greatest finite normal subgroup of $L'$, if $L'$ has finite index in $L$. Therefore, if $L'$ is isomorphic to $L$, the groups $\ker \pi \cap L$ and $\ker \pi \cap L'$ are equal, and are mapped onto each other by any isomorphism of $L$ onto $L'$; hence $\pi(L)\cong \pi(L')$, and $\pi(L)\neq \pi(L')$ if $L\neq L'$. We are thus reduced to the case where the group $A$ of (a) or (b) in Lemma \ref{art04-lem1.4} is $=\{e\}$. Moreover, in case (b), it suffices to consider $L\cap G'$ in view of Lemma \ref{art04-lem1.6}. Our assertion then follows from the rigidity theorems of Weil and Raghunathan and from Lemma \ref{art04-lem1.3}.

We shall need the following consequence of a theorem of Raghunathan:

\medskip
\noindent
{\bf Lemma \thnum{1.8}.\label{art04-lem1.8}}
{\em Let $G$, $G'$ be connected semi-simple $\bfQ$-groups, which are almost simple over $\bfQ$. Let $L$ be a subgroup of $G_{\bfQ}$ containing an arithmetic subgroup $L_{0}$ of $G$, and $s$ an isomorphism of $L$ onto a subgroup of $G'_{\bfQ}$ which maps $L_{0}$ onto a Zariski-dense subgroup of $G'$. Assume that $rk_{\bfQ}(G)\geq 2$.}
\begin{itemize}
\item[{\rm(i)}] {\em $G$ and $G'$ are isogeneous over $\bfQ$.}

\item[{\rm(ii)}] {\em If $G$ is simply connected, or $G'$ is centerless, there exists a $\bfQ$-isogeny $s'$ of $G$ onto $G'$, and a homomorphism $g$ of $L$ into the center of $G'_{\bfQ}$ such that $s(x)=s'(x)\cdot g(x)(x\in L)$.}
\end{itemize}

Let\pageoriginale $\widehat{G}$ be the universal covering group of $G$, $\pi:\widetilde{G}\to G$ the canonical projection and $\widetilde{L}=\pi^{-1}(L)\cap G_{\bfQ}$. The group $\widetilde{L}_{0}=\pi^{-1}(L_{0})\cap \widetilde{L}$ is arithmetic, as follows e.g. from (\cite{art04-key7}, \S6.11); in particular, $\pi(\widetilde{L}_{0})$ has finite index in $L_{0}$, and $L'_{0}=s\circ \pi(\widetilde{L}_{0})$ is Zariski-dense in $G'$.

We identify $G'$ with a $\bfQ$-subgroup of $\GL(n,\bfC)$, for some $n$. The map $r=s\circ\pi$ may be viewed as a linear representation of $\widetilde{L}$ into $\GL(n,\bfQ)$. By Theorem 1 of \cite{art04-key25}, there exists a normal subgroup $\widetilde{N}$ of $\widetilde{L}_{0}$, Zariski-dense in $\widetilde{G}$, and a morphism $t:\widetilde{G}\to \GL(n,\bfC)$ which coincides with $r$ on $\widetilde{N}$. Let $C$ be the Zariski-closure of $t(\widetilde{N})$. It is an algebraic subgroup contained in $t(\widetilde{G})\cap G'$. Since $t(\widetilde{N})$ is normal in $L'_{0}$, and $L'_{0}$ is Zariski-dense, the group $C$ is normal in $G'$. However, (\cite{art04-key9}, \S6.21(ii)), the group $G'$ is isogeneous to a group $R_{k/\bfQ}H$, where $k$ is a number field, $H$ an absolutely simple $k$-group, and $R_{k/\bfQ}$ denotes restriction of the scalars (\cite{art04-key31}, Chap. I). Consequently, an infinite subgroup of $G'_{\bfQ}$ is not contained in a proper direct factor of $G'$, whence $C=G'=t(\widetilde{G})$. If $f$ is a regular function defined over $\bfQ$ on $G'$, then $f\circ t$ is a regular function on $\widetilde{G}$, which takes rational values on the dense set $\widetilde{N}$. It follows immediately that $f\circ t$ is defined over $\bfQ$, hence $t$ is defined over $\bfQ$. Its kernel is a proper normal $\bfQ$-subgroup of $G'$, hence is finite, and $t$ is a $\bfQ$-isogeny. This implies (i).

If $G'$ is centerless, then $Z(\widetilde{G})$ belongs to the kernel of $t$. Thus, if $G$ is simply connected, or $G'$ centerless, $t$ defines a $\bfQ$-isogeny $s'$ of $G$ onto $G'$, which coincides with $s$ on the Zariski-dense subgroup $N=\pi(\widetilde{N})$. The group $s(N)$ is then Zariski-dense in $G'$. Let $x\in L$, $y\in N$. Then $x\cdot y\cdot x^{-1}\in N$, hence $s(x)\cdot s'(x)^{-1}$ centralizes $s(N)$, and therefore also $G'$. Consequently, $s(x)\cdot S'(x)^{-1}$ belongs to $G'_{\bfQ}\cap Z(G')$, and $f:x\mapsto s(x)\cdot s'(x)^{-1}$, and $s'$ fulfil our conditions.

\medskip
\noindent
{\bf Theorem \thnum{1.9}.\label{art04-thm1.9}}
{\em Let $G$ and $L$ be as in Lemma \ref{art04-lem1.8}. Assume $G$ to be centerless and $L$ to be equal to its normalizer in $G_{\bfQ}$. Then $E(L)$ may be identified with a subgroup of $(\Aut G)_{\bfQ}/(\Int G)_{\bfQ}$.}
\smallskip

If $G$ is centerless, and $M$ is a Zariski-dense subgroup of $G_{\bfQ}$, then the normalizer of $M$ in $G$ belongs to $G_{\bfQ}$. This follows from Theorem\pageoriginale 2 in \cite{art04-key6} if $M$ is arithmetic, but the proof yields this more general statement, as well as Proposition \ref{art04-prop3.3}(b) below. In view of this, Theorem \ref{art04-thm1.9} is a consequence of Lemma \ref{art04-lem1.8},

\begin{remark*}
It is no great loss in generality to assume that $L$ contains the center of $G_{\bfQ}$, and this assumption will in fact be fulfilled in the cases to be considered below. In this case, $\Aut L$ is generated by three kinds of automorphisms: (a) exterior automorphisms of $G$ leaving $L$ stable, (b) automorphisms $x\mapsto f(x)\cdot x$, where $f$ is a homomorphism of $L$ into its center, (c) automorphisms of the form $x\mapsto y\cdot x\cdot y^{-1}$, where $y$ belongs to the normalizer of $L$ in $G$.
\end{remark*}

Using some information on these three items, we shall in the following paragraph give a more precise description of $\Aut L$, when $G$ is a split group.

\section{Arithmetic subgroups of split groups over \texorpdfstring{$\bfQ$}{Q}.}\label{art04-sec2}

{\em In this paragraph $G$ is a connected semi-simple and almost simple $\bfQ$-group, which is split, of $\bfQ$-rank $\geq 2$; $L$ is the group of integral points of $G$ for the canonical $\bfZ$-structure associated to a splitting of $G$\cite{art04-key10}, \cite{art04-key18}, and $N(L)$ the normalizer of $L$ in $G_{\bfC}$.}

\subsection{}\label{art04-sec2.1}
The group $L$ is equal to its normalizer in $G_{\bfQ}$, and also to its normalizer in $G$ if $G$ has no center. To see this, we first notice that $L$ has finite index in its normalizer in $G$. In fact, since the image of $L$ in $\Int G$ is arithmetic (\cite{art04-key7}, \S6.11), it suffices to show that if $G$ is centerless, any arithmetic subgroup of $G$ is of finite index in its normalizer, which follows from the end argument of Theorem \ref{art04-thm1.5}(b). Our assertion is then a consequence of (\cite{art04-key6}, Theorem 7). Another proof will be given below (Theorem \ref{art04-thm4.3}).

\medskip
\noindent
{\bf Theorem \thnum{2.2}.\label{art04-thm2.2}}
{\em If $G$ is centerless, then $\Aut L$ is a split extension of $E(G)=\Aut G/\Int G$ by $L$. If $G$ is simply connected, then $\Aut L$ is a split extension of $E(G)$ by the subgroup $A$ of automorphisms of the form $x\mapsto f(x)\cdot y\cdot x\cdot y^{-1}$ where $y\in N(L)$ and $f$ is a homomorphism of $L$ into its center.}
\smallskip

If $G$ is centerless or simply connected, the standard construction of automorphisms of $G$ leaving stable the splitting of $G$ yields a subgroup\pageoriginale $E'(G)$ of $(\Aut G)_{\bfQ}$, isomorphic to $E(G)$ under the canonical projection, and leaving the $\bfZ$-structure of $G$ invariant. Thus $E(G)$ may be identified with a subgroup of $\Aut L$. If $G$ is centerless, the theorem follows then from Theorem \ref{art04-thm1.9}, and \ref{art04-sec2.1}. Let now $G$ be simple connected. Let $s\in \Aut L$. By Lemma \ref{art04-lem1.8}, we can find $s'\in (\Aut G)_{\bfQ}$ and $f\in \Hom (L,Z(G)_{\bfQ})$ such that $s(x)=f(x)\cdot s'(x)(x\in L)$. However, $L$ contains $Z(G)_{\bfQ}$ by \ref{art04-sec2.1}. Therefore, $f$ maps $L$ into $Z(G)\cap L$. But $Z(G)\cap L$ is equal to the center of $L$, since $L$ is Zariski-dense in $G$, whence our assertion.

If $G$ has a non-trivial center the image of $N(L)$ in $\Aut L$ is in general different from $\Int L$. The quotient $N(L)/L$ has been studied in various cases, including those of Examples \ref{art04-exams2.6}(1), (2), notably by Maass, Ramanathan, Allan (see \cite{art04-key1}, where references to earlier work are also given). We shall discuss it here and in \S4 from a somewhat different point of view. In the following statement, the group $G'=\Int G$ is endowed with the $\bfZ$-structure associated to the splitting defined by the given splitting of $G$.

\medskip
\noindent
{\bf Lemma \thnum{2.3}.\label{art04-lem2.3}}
{\em Let $\pi:G\to G'=\Int G$ be the canonical projection, $T$ the maximal torus given by the splitting of $G$, and $T'=\pi(T)$. Then}
\begin{equation*}
\pi(N(L))=G'_{\bfZ},\tag{1}\label{art04-eq1}
\end{equation*}
\begin{equation*}
\pi(N(L))/\Int L\cong T'_{\bfZ}/\pi (T_{\bfZ})\cong Z(L).\tag{2}\label{art04-eq2}
\end{equation*}
\smallskip

The group $L$ is the normalizer in $G_{\bfQ}$ of a Chevalley lattice $\mathfrak{g}_{\bfZ}$ in $\mathfrak{g}$ as follows from 2.17 in \cite{art04-key16}. Moreover, a Chevalley lattice is spanned by the logarithms of the unipotent elements in $L$. Consequently $N(L)$ is the normalizer in $G$ of $\mathfrak{g}_{\bfZ}$. From this \eqref{art04-eq1} follows.

Let $B$ be the maximal solvable subgroup of $G$ corresponding to the positive roots in the given splitting of $G$ and $U$ its unipotent radical. Then $B=T\cdot U$ (semi-direct). Let $x\in N(L)$. Since $\Int x$ preserves the $\bfQ$-structure of $G$, the group $x\cdot B\cdot x^{-1}$ is a maximal connected solvable subgroup defined over $\bfQ$, hence (\cite{art04-key9}, \S4.13) there exists $z\in G_{\bfQ}$ such that $z\cdot B\cdot z^{-1}=x\cdot B\cdot x^{-1}$. But we have $G_{\bfQ}=L\cdot B_{\bfQ}$(\cite{art04-key6}, Lemma 1). Since $B$ is equal to its normalizer, it follows that $N(L)=L\cdot (N(L)\cap B)$. Let now $x\in N(L)\cap B$. Write $x=t\cdot v(t\in T,v\in U)$. We have $\pi(x)\in L$ (see \ref{art04-sec2.1}).\pageoriginale But, with respect to a suitable basis of a Chevalley lattice in $\mathfrak{g}_{\bfQ}$, $\pi(t)$ is diagonal, and $\pi(u)$ upper triangular, unipotent, therefore $\pi(t)$, $\pi(u)\in L$. However \cite{art04-key10}, $\pi$ defines a $\bfZ$-isomorphism of $U$ onto $\pi(U)$, hence $u\in L$, which shows that
\begin{equation*}
N(L)=L\cdot (N(L)\cap T).\tag{3}\label{art04-eq3}
\end{equation*}
The kernel of $\pi$ is contained in $T$, therefore \eqref{art04-eq1} implies that $N(L)\cap T$ is the full inverse image of $T'_{\bfZ}$, which yields the first equality of \eqref{art04-eq2}. The groups $T_{\bfZ}$ and $T'_{\bfZ}$ consist of the elements of order 2 of $T$ and $T'$ respectively and are both isomorphic to $(\bfZ/2\bfZ)^{l}$, where $l$ is the rank of $G$. Consequently $T'_{\bfZ}/\pi(T_{\bfZ})$ is isomorphic to the kernel of $\pi:T_{\bfZ}\to T'_{\bfZ}$, i.e. to $Z(L)$, which ends the proof of \eqref{art04-eq2}.

The determination of $\Aut L/\Int L$ is thus to a large extent reduced to that of the center $Z(L)$ of $L$, and of the quotient of $L$ by its commutator subgroup $(L,L)$. We now make some remarks on these two groups.

\setcounter{subsection}{3}
\subsection{}\label{art04-sec2.4}
The center $Z(L)$ of $L$ is of order two if $G$ is simply connected of type $\bfA_{n}$ ($n$ odd), $\bfB_{n}$, $\bfC_{n}(n\geq 1)$, $\bfD_{n}(n\geq 3, n\text{~odd})$, $\bfE_{7}$, of type (\ref{art04-thm2.2}) if $G={\boldmath\text{$\Spin$~}} 4m$ ($m$ positive integer), of order one in the other cases.

In fact the $\bfZ$-structure on $G$ may be defined by means of an admissible lattice in the representation space of a faithful representation defined over $\bfQ$. If we assume $G\subset \mathbf{GL}(n,\bfC)$ and $\bfZ^{n}$ to be an admissible lattice, then $Z(L)$ is represented by diagonal matrices with integral coefficients, which shows first that $Z(L)$ is an elementary abelian 2-group. All almost simple simply connected groups have faithful irreducible representations, except for the type $\bfD_{2m}$. Thus, except in that case, $Z(L)$ is of order 2 (resp. 1) if $Z(G)$ has even (resp. odd) order, whence our contention. The case of $\bfD_{2m}$ is settled by considering the sum of the two half-spinor representations.

\subsection{}\label{art04-sec2.5}
It is well known that $\textbf{SL}(n,\bfZ)$ is equal to its commutator subgroup if $n\geq 3$ (see \cite{art04-key3} e.g.). Also the commutator subgroup of $\mathbf{Sp}(2n,\bfZ)$ is equal to $\mathbf{Sp}(2n,\bfZ)$ if $n\geq 3$, has index two if $n=2$ \cite{art04-key3},\pageoriginale \cite{art04-key28}. More generally, if the congruence subgroup theorem holds, which is the case if $rkG\geq 2$ and $G$ is simply connected, according to \cite{art04-key22}, then $L/(L,L)$ is the product of the corresponding local groups $G_{0_{p}}/(G_{0_{p}},G_{0_{p}})$. Serre has pointed out to me that, using this, one can show that $L=(L,L)$ if $G$ has $\rank \geq 3$, and is simply connected. Another more direct proof was mentioned to me by R. Steinberg, who also showed that $L/(L,L)$ is of order two if $G=\bfG_{2}$. He uses known commutation rules among unipotent elements of $L$, and the fact that they generate $L$.

\begin{description}
\item[{\bf Examples \thnum{2.6}.\label{art04-exams2.6}} {\rm(1)}]
$G=\mathbf{SL}(n,\bfC)$, $L=\mathbf{SL}(n,\bfZ)$, $(n\geq 3)$. In this case, $E(G)$ is of order two, generated by the automorphism $\sigma:x\mapsto {}^{t}x^{-1}$. By Lemma \ref{art04-lem2.3} and \ref{art04-sec2.4}, $\Int L$ has index one (resp. two) in the image of $N(L)$ if $n$ is odd (resp. even). Furthermore, it is easily seen, and will follow from Lemma \ref{art04-lem4.5}, that, in the even dimensional case, the non-interior automorphisms defined by $N(L)$ are of the form $x\mapsto y\cdot x\cdot y^{-1}(y\in \mathbf{GL}(n,\bfZ), \det y=-1)$. Thus, taking \ref{art04-sec2.5} into account, we see that $\Aut L$ is generated by $\Int L$, $\sigma$, and, for $n$ even, by one further automorphism induced by an element of $\mathbf{GL}(n,\bfZ)$ of determinant ${}-1$. This is closely related to results of Hua-Reiner \cite{art04-key12}, \cite{art04-key13}.

\item[{\rm(2)}] $G=\mathbf{Sp}(2n,\bfC)$, $L=\mathbf{Sp}(2n,\bfZ)$, $(n\geq 2)$. Here, $E(G)$ is reduced to the identity. Thus, by the above, $\Int L$ has index two in $\Aut L$ if $n\geq 3$, index four if $n=2$. The non-trivial element of $N(L)/L$ is represented by an automorphism of the form $x\mapsto y\cdot x\cdot y^{-1}$ where $y$ is an element of $\mathbf{GL}(2n,\bfZ)$ which transforms the bilinear form underlying the definition of $\mathbf{Sp}(2n,\bfC)$ into its opposite (see Examples \ref{art04-exams4.6}). For $n=2$, one has to add the automorphism $x\to \chi(x)\cdot x$, where $\chi$ is the non-trivial character of $L$. This result is due to Reiner \cite{art04-key27}.

\item[{\rm(3)}] $G$ is simply connected, of type $\bfD_{2m}$. Then we have a composition series
$$
\Aut L\supset A\supset \Int L,
$$
where $\Aut L/A$ is of order two, and $A/\Int L$ has order two if $m$ is odd, is of type (\ref{art04-thm2.2}) if $m$ is even. This follows from (\ref{art04-thm2.2}), (\ref{art04-lem2.3}), (\ref{art04-sec2.4}), (\ref{art04-sec2.5}). The other simple groups of $\rank \geq 3$ are discussed similarly.
\end{description}

\section{\texorpdfstring{$S$}{S}-arithmetic groups over number fields.}\label{art04-sec3}
\pageoriginale

\subsection{}\label{art04-sec3.1}

Throughout the rest of this paper, $k$ is an algebraic number field of finite degree over $\bfQ$, $\mathfrak{o}$ its ring of integers, $V$ the set of primes of $k$, $V_{\infty}$ the set of infinite primes of $k$, $S$ a finite subset of $V$ containing $V_{\infty}$, and $\mathfrak{o}(S)$ the subring of $x\in k$ which are integral outside $S$. We let $I(k,S)$ be the $S$-ideal class group of $k$, i.e. the quotient of the group of fractional $\mathfrak{o}(S)$-ideals by the group of principal $\mathfrak{o}(S)$-ideals. We follow the notation of \cite{art04-key5}. In particular $k_{v}$ is the completion of $k$ at $v\in S$, $\mathfrak{o}_{v}$ the ring of integers of $k_{v}$. If $G$ is a $k$-group, then $G^{0}$ is its identity component, and 
$$
G_{v}=G_{k_{v}}(v\in S), G_{S}=\prod\limits_{v\in S}G_{v}, G_{\infty}=\prod\limits_{v\in V_{\infty}}G_{v}.
$$
Moreover $G'=R_{k/\bfQ}G$ is the group obtained from $G$ by restriction of the groundfiled from $k$ to $\bfQ$(\cite{art04-key31}, Chap. I), and we let $\mu$ denote the canonical isomorphism of $G_{k}$ onto $G'_{\bfQ}$.

If $A$ is an abelian group, and $q$ a positive integer, we let ${}_{q}A$ and $A^{(q)}$ denote the kernel and the image of the homomorphism $x\mapsto x^{q}$.

\subsection{}\label{art04-sec3.2}
Let $G$ be a $k$-group. A subgroup $L$ of $G_{k}$ is $S$-arithmetic if there is a faithful $k$-morphism $r:G\to \mathbf{GL}_{n}$ such that $r(L)$ is commensurable with $r(G)_{\mathfrak{o}(S)}$.

If $S'$ is a finite set of primes of $\bfQ$, including $\infty$, and $S$ is the set of primes dividing some element of $S'$, then $v:G_{k}\xrightarrow{\sim}G'_{\bfQ}$ induces a bijection between $S$-arithmetic subgroups of $G$ and $S'$-arithmetic subgroups of $G'$. This follows directly from the remarks made in (\cite{art04-key5}, \S1).

In the following proposition, we collect some obvious generalizations of known facts.

\medskip
\noindent
{\bf Proposition \thnum{3.3}.\label{art04-prop3.3}}
{\em Let $G$ be a semi-simple $k$-group, $L$ a $S$-arithmetic subgroup of $G$, $N$ the greatest normal $k$-subgroup of $G^{0}$ such that $N_{\infty}$ is compact, and $\pi:G\to G/N$ the natural projection.}
\begin{itemize}
\item[{\rm(a)}] {\em If $N$ is finite and $G$ is connected, $L$ is Zariski-dense in $G$.}
\item[{\rm(b)}] {\em If $G$ is connected, the commensurability group $C(L)$ of $L$ in $G$ is equal to $\pi^{-1}((G/N)_{k})$.}
\item[{\rm(c)}] {\em If\pageoriginale $\sigma:G\to H$ is a surjective $k$-morphism, $\sigma(L)$ is $S$-arithmetic in $H$.}

\item[{\rm(d)}] {\em If $N$ is finite, $L$ has finite index in its normalizer in $G$.}
\end{itemize}

\medskip

\begin{itemize}
\item[(a)] follows from (\cite{art04-key6}, Theorem 3), and from the fact that $L$ contains an arithmetic subgroup of $G$.

\item[(b)] We recall that $C(L)$ is the group of elements $x\in G$ such that $x\cdot L\cdot x^{-1}$ is commensurable with $L$. The proof of (b) is the same as that the Theorem 2 in \cite{art04-key6}. In fact, this argument shows that if $G$ is centerless, then $C(M)\subset G_{k}$ whenever $M$ is a subgroup of $G_{k}$ Zariski-dense in $G$.

\item[(c)] If $\sigma$ is an isomorphism, the argument is the same as that of (\cite{art04-key7}, \S6.3). If $\sigma$ is an isogeny, this has been proved in (\cite{art04-key5}, \S8.12). From there, the extension to the general case proceeds exactly in the same way as in the case $S=V_{\infty}$(\cite{art04-key6}, Theorem 6).

\item[(d)] We may assume $G$ to be connected and $N=\{e\}$. Then by (c), $N(L)\subset G_{k}$. We view $G_{k}$ and $L$ as diagonally embedded in $G_{S}$. Then $L$ is discrete in $G_{S}$.
\end{itemize}

This group $L$ has a finite system of generators \cite{art04-key17}, say $(x_{i})_{1\leq i\leq q}$. Since $L$ is discrete, there exists a neighbourhood $U$ of $e$ in $G_{S}$ such that if $x\in N(L)\cap U$, then $x$ centralizes the $x_{i}$'s, hence $L$. The latter being Zariski-dense in $G$, this implies that the component $x_{v}$ of $x$ in $G_{v}(v\in S)$ is central in $G_{v}$, whence $x_{v}=e$, which shows that $N(L)$ is discrete in $G_{S}$. But $G_{S}/L$ has finite invariant volume (\cite{art04-key5}, \S5.6) and is fibered by $N(L)/L$, hence $N(L)/L$ is finite.

\smallskip
\noindent
{\bf Proposition \thnum{3.4}.\label{art04-prop3.4}}
{\em Let $G$ be a semi-simple $k$-group, $L$ a subgroup of $G_{k}$ which is Zariski-dense, and is equal to its normalizer in $G_{k}$, and $N(L)$ the normalizer of $L$ in $G$. Then $N(L)/L$ is a commutative group whose exponent divides the order $m$ of the center $Z(G)$ of $G$.}


We show first that if $x\in N(L)$, then $x^{m}\in L$. In view of the assumption, it suffices to prove that $x^{m}\in G_{k}$. Let $\pi:G\to G/Z(G)$ be the canonical projection. The fiber $F_{x}=\pi^{-1}(\pi(x))$ of $x$ consists of the elements $x\cdot z_{i}(1\leq i\leq m)$, where $z_{i}$ runs through $Z(G)$, and belongs to $N(L)$. By the remark made in Proposition \ref{art04-prop3.3}(b), $\pi(x)$ is rational over\pageoriginale $k$, hence $F_{x}$ is defined over $k$, and its points are permuted by the Galois group of $\overline{k}$ over $k$. Since the $z_{i}$'s are central, the product of the $xz_{i}$ is equal to $x^{m}\cdot z_{1}\ldots z_{m}$ and is rational over $k$. Similarly the product of the $z_{i}$'s is rational over $k$, whence our assertion.

It is possible to embed $Z(G)$ as a $k$-subgroup in a $k$-torus $T'$ whose first Galois cohomology group is zero (see Ono, Annals of Math. (2), 82 (1965), p. 96). Let $H=(G\times T')/Z(G)$ where $Z(G)$ is embedded diagonally in $G\times T'$. Then $G/Z(G)$ may be identified with $H/T'$. Let $x\in N(L)$. We have already seen that $\pi(x)$ is rational over $k$. But, since $T'$ has trivial first Galois-cohomology group, the map $H_{k}\to (H/T')_{k}$ is surjective. There exists therefore $d\in T'$ such that $d\cdot x\in H_{k}$. Thus, if $x$, $y\in N(L)$, we can find two elements $x'$, $y'\in H_{k}$, which normalize $L$, whose commutator $(x',y')$ is equal to $(x,y)$. But, obviously, $G=(H,H)$, therefore $(x,y)\in N(L)_{k}$ hence $(x,y)\in L$, and $(N(L),N(L))\subset L$. This argument also proves that
\setcounter{equation}{0}
\begin{equation*}
N(L)=G\cap (N_{H}(L)_{k}\cdot T'),\tag{1}\label{art04-prop3.4-eq1}
\end{equation*}
where $N_{H}(L)$ is the normalizer of $L$ in $H$, and $N_{H}(L)_{k}=N_{H}(L)\cap H_{k}$.

For the sake of reference, we state as a lemma a remark made by Ihara in (\cite{art04-key14}, p. 269).

\medskip
\noindent
{\bf Lemma \thnum{3.5}.\label{art04-lem3.5}}
{\em Let $A$ be a group, $B$ a subgroup, and $V$ a $A$-module. Assume that for any $a\in A$, there is no non-zero element of $V$ fixed under $a\cdot B\cdot a^{-1}\cap B$. Then the restriction map $r:H^{1}(A,V)\to H^{1}(B,V)$ is injective.}
\smallskip

It suffices to show that if $z$ is a 1-cocycle of $A$ which is zero on $B$, then $z$ is zero. Let $a\in A$, $b\in B$ be such that $a\cdot b\cdot a^{-1}=b'\in B$. We have then
$$
z(a\cdot b)=z(a)=z(b'\cdot a)=b'\cdot z(a),
$$
which shows that $z(a)$ is fixed under $a\cdot B\cdot a^{-1}\cap B$, hence is zero.

\medskip
\noindent
{\bf Theorem \thnum{3.6}.\label{art04-thm3.6}}
{\em Let $G$ be a semi-simple $k$-group and $L$ a $S$-arithmetic subgroup. Then $E(L)$ is finite if one of the following conditions is fulfilled~:}
\begin{itemize}
\item[{\rm(a)}] {\em $G$ has no normal $k$-subgroup $N$ such that $N_{\infty}$ has a non-compact factor of type $\mathbf{SL}(2,\bfR)$, or also of type $\mathbf{SL}(2,\bfC)$ if $G_{S}/L$ is not compact;}

\item[{\rm(b)}] {\em $G$\pageoriginale is of type $\mathbf{SL}_{2}$ over $k$, and $S$ has at least two elements.}
\end{itemize}

By Lemma \ref{art04-lem1.4}(b), we may assume $G$ to be connected. Let $N$ be the greatest normal $k$-subgroup of $G$ such that $N_{\infty}$ is compact and $\pi:G\to G/N$ the natural projection.
\begin{itemize}
\item[(a)] Arguing as in Lemma \ref{art04-lem1.4}, we see that it suffices to show that $E(\pi(L))$ is finite, which reduces us to the case where $G$ is a direct product of simple $k$-groups $G_{i}$. The group $L_{i}=G_{i}\cap L$ is $S$-arithmetic in $G_{i}$ and the product of the $L_{i}$ is normal of finite index in $L$. By Lemma \ref{art04-lem1.4}, we may therefore assume $L$ to be the product of its intersection with the $G_{i}$'s. The group $L$ has finite index in its normalizer (Proposition \ref{art04-prop3.3}) and is finitely generated \cite{art04-key17}, so that, in order to deduce our assertion from Lemma \ref{art04-lem1.1}, applied to $L$ and $G_{\infty}$, it suffices to show that $H^{1}(L,\mathfrak{g}_{\infty})=0$. Since this group is isomorphic to the product of the groups $H^{1}(L_{i},\mathfrak{g}_{i\infty})$, we may assume $G$ to be simple over $k$. Let $L_{0}=L\cap G_{\mathfrak{o}}$.

The group $L_{0}$ is arithmetic, and therefore so is $x\cdot L_{0}\cdot x^{-1}\cap L_{0}=L_{0,x}(x\in L)$. Consequently, $L_{0,x}$ is Zariski-dense in $G$ (\cite{art04-key6}, Theorem 1), and has no non-zero fixed vector in $\mathfrak{g}_{\infty}$. By Lemma \ref{art04-lem3.5}, the restriction map : $H^{1}(L,\mathfrak{g}_{\infty})\to H^{1}(L_{0},\mathfrak{g}_{\infty})$ is injective. But $H^{1}(L_{0},\mathfrak{g}_{\infty})=0$: if $\rk_{k}G\geq 1$, this follows from \cite{art04-key25}, \cite{art04-key26}. Let now $\rk_{k}G=0$. Then $G_{\infty}/L_{0}$ is compact (\cite{art04-key4}, \S11.6). In view of \ref{art04-sec3.2}, we may further assume $G$ to be almost absolutely simple over $k$. Let $J$ be the set of $v\in V_{\infty}$ such that $G_{v}$ is not compact and $H$ the subgroup of $G$ generated by the $G_{v}$'s $(v\in J)$. Then, by Weil's theorem (\cite{art04-key32}, \cite{art04-key33}),
\setcounter{equation}{0}
\begin{equation*}
H^{1}(\Gamma_{0},\mathfrak{h})=0.\tag{1}\label{art04-thm3.6-eq1}
\end{equation*}
But we have
\begin{equation*}
H^{1}(\Gamma_{0},\mathfrak{g}_{v})=H^{1}(\Gamma_{0},{}^{v}\mathfrak{g}_{k})\otimes_{v(k)}k_{v},\quad (v\in V_{\infty})\tag{2}\label{art04-thm3.6-eq2}
\end{equation*}
\begin{equation*}
H^{1}(\Gamma_{0},\mathfrak{g})=\prod\limits_{v\in V_{\infty}}H^{1}(\Gamma_{0},\mathfrak{g}_{v}),\tag{3}\label{art04-thm3.6-eq3}
\end{equation*}
\begin{equation*}
H^{1}(\Gamma_{0},\mathfrak{h})=\prod\limits_{v\in J}H^{1}(\Gamma_{0},\mathfrak{g}_{v}),\tag{4}\label{art04-thm3.6-eq4}
\end{equation*}
whence $H^{1}(\Gamma_{0},\mathfrak{g}_{\infty})=0$.

\item[(b)] $N$\pageoriginale is finite, and therefore, $L$ has finite index in its normalizer (Proposition \ref{art04-prop3.3}). Again, there remains to show that $H^{1}(L,\mathfrak{g}_{\infty})=0$. Let $\mathbf{SL}_{2}\to G$ be the covering map, and $L'$ the inverse image of $L$ in $\mathbf{SL}(2,k)$. The homomorphism $H^{1}(L,\mathfrak{g}_{\infty})\to H^{1}(L',\mathfrak{g}_{\infty})$ is injective, hence we may assume $G=\mathbf{SL}_{2}$. But then the vanishing of $H^{1}$ follows from \cite{art04-key29}.
\end{itemize}

\begin{remark*}
It is also true that in both cases of Theorem \ref{art04-thm3.6}, the group $L$ is not isomorphic to a proper subgroup of finite index. This is seen by modifying the proof of Theorem \ref{art04-thm3.6}, in the same way as Proposition \ref{art04-prop1.7} was obtained from Theorem \ref{art04-thm1.5}.
\end{remark*}

If $G$ is almost simple over $k$, of $k$-rank $\geq 2$, then we may apply Theorem \ref{art04-thm1.9} to $G'$. Thus, in that case, we see that, if $L$ contains $Z(G)_{k}$, the determination of $\Aut L$ is essentially reduced to that of the normalizer of $L$ in $G$, of the homomorphisms of $L$ into its center, and of the exterior automorphisms of $G'$ leaving stable. We shall use this in \S4 to get more explicit information when $G$ is a split group. Here, we mention another consequence of Theorem \ref{art04-thm1.9}.

\medskip
\noindent
{\bf Proposition \thnum{3.7}.\label{art04-prop3.7}}
{\em Let $G$ be an almost absolutely simple $k$-group, of $k$-rank $\geq 2$, $k'$ a number field, and $G'$ an almost absolutely simple $k'$-group. Let $L$ be an arithmetic subgroup of $G_{k}$, and $s$ an isomorphism of $L$ onto an arithmetic subgroup of $G'$. Then there is an isomorphism $\phi$ of $k'$ onto $k$ and the $k$-group ${}^{\phi}G'$ obtained from $G'$ by change of the groundfield $\phi$ is $k$-isogeneous to $G$.}
\smallskip

Let $H=R_{k/\bfQ}G$, $H'=R_{k'/\bfQ}G'$, and $M$, $M'$ the images of $L$ and $L'=s(L)$ under the canonical isomorphisms $G_{k}\xrightarrow{\sim}H_{\bfQ}$ and $G'_{k'}\xrightarrow{\sim}H'_{\bfQ}$.

\vskip .1cm
Then $s$ may be viewed as an isomorphism of $M$ onto $M'$. The group $M'$ is infinite, hence $H'_{\bfR}$ is not compact, and $M'$ is Zariski-dense in $H'$ (\cite{art04-key6}, Theorem 1). By Lemma \ref{art04-lem1.8}, $H$ and $H'$ are $\bfQ$-isogeneous. There exists therefore an isomorphism $\alpha$ of $\mathfrak{h}_{\bfQ}$ onto $\mathfrak{h}'_{\bfQ}$. But the commuting algebra of $\ad\mathfrak{h}_{\bfQ}$ (resp. $\ad\mathfrak{h}'_{\bfQ}$) in the ring of linear transformations of $\mathfrak{h}_{\bfQ}$ (resp. $\mathfrak{h}'_{\bfQ}$) into itself is isomorphic to $k$ (resp. $k'$). Hence $\alpha$ induces an isomorphism $\phi:k'\xrightarrow{\sim}k$. Let $\mathfrak{g}''_{k}={}^{\phi}\mathfrak{g}_{k'}$ be the Lie algebra over $k$ obtained from $\mathfrak{g}'$ by the change of ground-field\pageoriginale $\phi$. Then, it is clear from the definition of $\phi$ that $\alpha=R_{k/\bfQ}\beta$, where $\beta$ is a $k$-isomorphism of $\mathfrak{g}$ onto $\mathfrak{g}''$. This isomorphism is then the differential of a $k$-isogeny of the universal covering of $G$ onto the $k$-group ${}^{\phi}G'$.

\setcounter{subsection}{7}
\subsection{}\label{art04-sec3.8} 
We need some relations between $(\Aut G)_{k}$ and $(\Aut G')_{\bfQ}$. For simplicity, we establish them in the context of Lie algebras, and assume $G$ to be almost simple over $\overline{k}$. The Lie algebra $\mathfrak{g}'_{\bfQ}$ is just $\mathfrak{g}_{k}$, viewed as a Lie algebra over $\bfQ$. Since $\mathfrak{g}_{k}$ is absolutely simple, the commuting algebra of $\ad\mathfrak{g}'_{\bfQ}$ in $\mathfrak{gl}(\mathfrak{g}'_{\bfQ})$ may be identified to $k$. Let $a\in \Aut \mathfrak{g}'_{\bfQ}$. Then $a$ defines an automorphism of $\mathfrak{gl}(\mathfrak{g}'_{\bfQ})$ leaving $\ad\mathfrak{g}'_{\bfQ}$ stable, and therefore an automorphism $\beta(a)$ of $k$. If $\beta$ is the identity, this means that $a$ is a $k$-linear map of $\mathfrak{g}'_{\bfQ}$, hence comes from an automorphism of $\mathfrak{g}_{k}$. We have therefore an exact sequence
\begin{equation*}
1\to \Aut \mathfrak{g}_{k}\to \Aut \mathfrak{g}'_{\bfQ}\to \Aut k.\tag{1}\label{art04-sec3.8-eq1}
\end{equation*}

Let $k_{0}$ be the fixed field of $\Aut k$ in $k$. Assume that $\mathfrak{g}_{k}=\mathfrak{g}_{0}\otimes_{k_{0}}k$, where $\mathfrak{g}_{0}$ is a Lie algebra over $k_{0}$. Then, for $s\in \Aut k$, ${}^{s}\mathfrak{g}_{k}=\mathfrak{g}_{k}$, and $s$, acting by conjugation with respect to $\mathfrak{g}_{0}$, defines a $s$-linear automorphism of $\mathfrak{g}_{k}$, and therefore an automorphism $a$ of $\Aut \mathfrak{g}_{\bfQ}$ such that $\beta(a)=s$. Thus, in this case, the sequence
\begin{equation*}
1\to \Aut \mathfrak{g}_{k}\to \Aut \mathfrak{g}'_{\bfQ}\to \Aut k\to 1\tag{2}\label{art04-sec3.8-eq2}
\end{equation*}
is exact and split. Translated into group terms, this yields the following lemma:

\medskip
\noindent
{\bf Lemma \thnum{3.9}.\label{art04-lem3.9}}
{\em Let $G$ be absolutely almost simple over $k$. Then we have an exact sequence}
\begin{equation*}
1\to (\Aut G)_{k}\to (\Aut G')_{\bfQ}\to \Aut k.\tag{1}\label{art04-lem3.9-eq1}
\end{equation*}
{\em Let $k_{0}$ be the fixed field of $\Aut k$ and assume that $G$ is obtained by extension of the field of definition from a $k_{0}$-group $G_{0}$. Then the sequence}
\begin{equation*}
1\to (\Aut G)_{k}\to (\Aut G')_{\bfQ}\to \Aut k\to 1\tag{2}\label{art04-lem3.9-eq2}
\end{equation*}
{\em is exact and split. On $G_{k}$, identified with $G_{0,k}$, the group $\Aut k$ acts by conjugation.}
\smallskip

Strictly\pageoriginale speaking, the sequences (\ref{art04-sec3.8}) (1), (2) give Lemma \ref{art04-lem3.9} (1), (2) if $G$ is centerless or simply connected (the only cases of interest below). But in the general case, we may argue in the same way as above, replacing $\Aut \mathfrak{g}_{k}$ and $\Aut \mathfrak{g}'_{\bfQ}$ by the images of $(\Aut G)_{k}$ and $(\Aut G')_{\bfQ}$ in those groups. The proof can also be carried out directly in $G$ and $G'$, using the structure of $R_{k/\bfQ}G$, and is then valid of $\bfQ$, $k$ and $\Aut k$ are replaced by a field $K$, a finite separable extension $K'$ of $K$, and $\Aut (K'/K)$.

\begin{remark*}
The above lemma was obtained with the help of Serre, who has also given examples where $G$ has no $k_{0}$-form and (\ref{art04-lem3.9-eq2}) is not exact.
\end{remark*}

\section{Split groups over number fields.}\label{art04-sec4}

In this paragraph, $G$ is a connected almost simple $k$-split group. $G$ is viewed as obtained by extension of the groundfield from a $\bfQ$-split group $G_{0}$, endowed with the $\bfZ$-structure associated to a splitting over $\bfQ$. $G$ is then endowed with an $\mathfrak{o}$-structure associated to its given splitting, and $G_{B}$ is well defined for any $\mathfrak{o}$-algebra $B$. We shall be interested mainly in the canonical $S$-arithmetic subgroup $G_{\mathfrak{o}(S)}$.

\medskip
\noindent
{\bf Lemma \thnum{4.1}.\label{art04-lem4.1}}
{\em Let $G$ be split over $k$, almost simple over $k$, and $L=G_{\mathfrak{o}(S)}$.}
\begin{itemize}
\item[{\rm(i)}] {\em $L$ is equal to its normalizer in $G_{k}$. The image in $G/Z(G)$ of the normalizer $N(L)$ of $L$ in $G$ is equal to $(G/Z(G))_{\mathfrak{o}(S)}$. In particular, $L=N(L)$ if $G$ is centerless.}

\item[{\rm(ii)}] {\em The group $N(L)/L$ is a finite commutative group whose exponent divides the order $m$ of $Z(G)$.}
\end{itemize}

\medskip

\begin{itemize}
\item[(i)] Let $\Gamma$ be a Chevalley lattice in $\mathfrak{g}_{0,\bfQ}$. Then (\cite{art04-key16}, 2.17) shows that $G_{\mathfrak{o}(S)}$ is the stabilizer of $\mathfrak{o}(S)$. $\Gamma$ in $G_{k}$, operating on $\mathfrak{g}$ by the adjoint representation. The lattice $\Gamma$ is spanned by the logarithms of the unipotent elements in $G_{0,\bfZ}$, hence $\mathfrak{o}(S)$. $\Gamma$ is spanned by the logarithms of unipotent elements in $G_{\mathfrak{o}(S)}$. It is then clear that if $x\in G$ normalizes $G_{\mathfrak{o}(S)}$, then $\Ad x$ normalizes $\mathfrak{o}(S)\cdot \Gamma$. If moreover $x\in G_{k}$, then $x\in G_{\mathfrak{o}(S)}$, which proves the first assertion. Together with Proposition \ref{art04-prop3.3}, this proves (i).

\item[(ii)] The\pageoriginale group $N(L)/L$ is finite by Proposition \ref{art04-prop3.3}. The other assertions of (ii) follow from (i) and Proposition \ref{art04-prop3.4}.
\end{itemize}

\medskip
\noindent
{\bf Lemma \thnum{4.2}.\label{art04-lem4.2}}
{\em Let $G=\mathbf{SL}_{2}$, $\mathbf{PSL}_{2}$ and $L$ a $S$-arithmetic subgroup of $G$. Assume that $S$ has at least two elements. Let $s$ be an automorphism of $L$. There exists an automorphism $s'$ of $G'$, defined over $\bfQ$, and a homomorphism $f$ of $L$ into $Z(G')_{\bfQ}$ such that $s(x)=f(x)\cdot s'(x)(x\in L)$.}
\smallskip

Let $\widetilde{G}=\mathbf{SL}_{2}$, $\pi :\widetilde{G}\to G$ the natural homomorphism and $\widetilde{L}=\pi^{-1}(L)\cap G_{k}$. Then $\widetilde{L}$ is $S$-arithmetic in $\widetilde{G}$. The map $s\circ \pi$ defines a homomorphism of $\widetilde{L}$ into $G'_{\bfQ}$. It follows from \cite{art04-key29} that there exists a $\bfQ$-morphism $t:R_{k/\bfQ}\widetilde{G}\to G'$, which coincides with $s\circ \pi$ on a normal subgroup of finite index of $\widetilde{L}$. The end of the argument is then the same as in Lemma \ref{art04-lem1.8}.

\medskip
\noindent
{\bf Theorem \thnum{4.3}.\label{art04-thm4.3}}
{\em Let $\Aut (k,S)$ be the subgroup of $\Aut k$ leaving $S$ stable. Assume either $\rk_{k}G\geq 2$ or $\rk_{k}G=1$ and $\Card S\geq 2$. Let $L=G_{\mathfrak{o}(S)}$.}
\begin{itemize}
\item[{\rm(i)}] {\em If $G$ is centerless, $\Aut L$ is generated by $E(G)$, the group $\Aut (k,S)$ acting by conjugation, and $\Int L$.}

\item[{\rm(ii)}] {\em If $G$ is simply connected, $\Aut L$ is generated by $E(G)$, $\Aut(k,S)$, and automorphisms of the form $x\mapsto f(x)\cdot y\cdot x\cdot y^{-1}$ where $f$ is a homomorphism of $L$ into its center, and $y$ belongs to the normalizer of $L$ in $G$.}
\end{itemize}

By Lemma \ref{art04-lem4.1}, $L$ contains $Z(G)_{k}$. Let $s\in \Aut L$. By Lemma \ref{art04-lem1.8} and Lemma \ref{art04-lem4.2} we may write $s(x)=f(x)\cdot s'(x)$ where $s'$ is a $\bfQ$-automorphism of $G'$ and $f$ a homomorphism of $L$ into $Z(G')_{\bfQ}\cong Z(G)_{k}$, hence of $L$ into its center.

The group $G$ comes by extension of the groundfield from a split $\bfQ$-group $G_{0}$. Therefore Lemma \ref{art04-lem3.9} obtains. After having modified $s$ by a field automorphism $J$, we may consequently assume $s'$ to belong to $(\Aut G)_{k}$. In both cases (i), (ii) $(\Aut G)_{k}$ is a split extension of $E(G)$ by $(\Int G)_{k}$; moreover, the representative $E'(G)$ of $E(G)$ alluded to in Theorem \ref{art04-thm2.2} leaves $L$ stable. Thus, after having\pageoriginale multiplied $s'$ by an element of $E(G)$, we may assume $s'\in (\Int G)_{k}$, hence $s'=\Int y$, $(y\in N(L))$.

\setcounter{subsection}{3}
\subsection{}\label{art04-sec4.4}
Let $G$ have a non-trivial center. We assume that the underlying $\bfQ$-split group $G_{0}$ may be (and is) identified with a $\bfQ$-subgroup of $\mathbf{GL}_{n}$ by means of an irreducible representation all of whose weights are extremal, i.e. form one orbit under the Weyl group, in such a way that $\mathbf{Z}^{n}$ is an admissible lattice, in the sence of \cite{art04-key10}. (This assumption is fulfilled in all cases, except for the one of the spinor group in a number of variables multiple of four.)

Let $D$ be the group of scalar multiples of the identity in $\mathbf{GL}_{n}$, and $H=D\cdot G$. The group $H$ is the identity component of the normalizer of $G$ in $\mathbf{GL}_{n}$. The group $D$ is a one-dimensional split torus. In particular, its first Galois cohomology group is zero. We have $G\cap D=Z(G)$, and $G\subset \mathbf{SL}_{n}$, therefore the order $m$ of $Z(G)$ divides $n$, and Proposition \ref{art04-prop3.4} (i) yeilds
\begin{equation*}
N(L)=G\cap N_{H}(L)_{k}\cdot D.\tag{1}\label{art04-sec4.4-eq1}
\end{equation*}

\medskip
\noindent
{\bf Lemma \thnum{4.5}.\label{art04-lem4.5}}
{\em We keep the assumptions of \ref{art04-sec4.4}. Let $m$ be the order of $Z(G)$. Let $A$ and $B$ be the images of $N(L)$ and $H_{\mathfrak{o}(S)}$ in $\Aut L$.}
\begin{itemize}
\item[{\rm(i)}] {\em The enveloping algebra $M$ of $L$ over $\mathfrak{o}(S)$ is $\bfM(n,\mathfrak{o}(S))$.}

\item[{\rm(ii)}] {\em $A/B$ is isomorphic to a subgroup of }
$$
{}_{m}I(k,S)\text{\em ~~ and~~ } B\text{\em ~~ to~~ } \mathfrak{o}(S)^{*}/\mathfrak{o}(S)^{*(m)}.
$$
\end{itemize}


\begin{itemize}
\item[(i)] In view of the definition of admissible lattices \cite{art04-key10}, the maximal $k$-split torus $T$ of the given splitting $G$ may be assumed to be diagonal and the $\mathfrak{o}(S)$-lattice $\Gamma_{0}=\mathfrak{o}(S)^{n}$ is the direct sum of its intersections with the eigenspaces of $T$. Out assumption on the weights implies further that these eigenspaces are one-dimensional, permuted transitively by the normalizer $N(T)$ of $T$.

Given a prime ideal $v\in V=V_{\infty}$, we denote by $F_{v}$ the residue field $\mathfrak{o}/v$ and by $\overline{F}_{v}$ an algebraic closure of $F_{v}$. By \cite{art04-key10}, reduction $\mod v$ of $G$, (endowed with its canonical $\mathfrak{o}$-structure), yields a $F_{v}$-subgroup $G_{(v)}$ of $\mathbf{GL}(n,\overline{F}_{v})$ which is connected, almost simple, has the same Dynkin diagram as $G$, and is simply connected if $G$\pageoriginale is. The reduction $\mod v$ also defines an isomorphism of the character group $X^{*}(T)$ of $T$ onto the character group $X^{*}(T_{(v)})$ of the reduction $\mod v$ of $T$, which induces a bijection of the weights of the identity representation of $G$ onto those of the identity representation of $G_{(v)}$. Thus the eigenspaces of $T_{(v)}$ are one-dimensional, and permuted transitively by the normalizer of $T_{(v)}$. Consequently, the identity representation of $G_{(v)}$ is irreducible.

The given splitting of $G$ defines one of the universal covering $\widetilde{G}$ of $G$, hence an $\mathfrak{o}$-structure on $\widetilde{G}$. The reduction $\mod v\widetilde{G}_{(v)}$ of $G$ is the universal covering group of $G_{(v)}$ and the identity representation may be viewed as a irreducible representation of $\widetilde{G}_{(v)}$, say $f_{(v)}$. But $f_{(v)}$ has only extremal weights, therefore is a fundamental representation. It follows then from results of Steinberg (\cite{art04-key30}; 1.3, 7.4) that the representation $f_{(v)}$ of the finite group $\widetilde{G}_{(v),F_{v}}$ is absolutely irreducible. Now, since reduction $\mod v$ is good, $\widetilde{G}_{(v),F_{v}}$ is the reduction of $\widetilde{G}_{\mathfrak{o}_{v}}$. Moreover, $\widetilde{G}$ being split and simply connected, strong approximation is valid in $\widetilde{G}$, hence $\widetilde{G}_{\mathfrak{o}}$ is dense in $\widetilde{G}_{\mathfrak{o}_{v}}$, which implies that reduction $\mod v$ maps $\widetilde{G}_{\mathfrak{o}}$ onto $\widetilde{G}_{(v),F_{v}}$. But the canonical projection of $\widetilde{G}$ onto $G$ maps $\widetilde{G}_{\mathfrak{o}}$ into $\widetilde{G}_{\mathfrak{o}}$. Consequently, the image of $G_{\mathfrak{o}}$ in $G_{(v)}$ by reduction is a subgroup which contains $f_{(v)}(\widetilde{G}_{(v),F_{v}})$, hence is irreducible. Therefore
$$
M\otimes F_{v}=\bfM(n,F_{v}), (v\in V-S).
$$
This shows that the index of $M$ in $\bfM(n,\mathfrak{o}(S))$ is prime to all elements in $V-S$, whence (i).

\item[(ii)] By \ref{art04-sec4.4} \eqref{art04-sec4.4-eq1}, the image of $N(L)$ in $\Aut L$ is the same as that of $N'=N_{H}(L)_{k}$. Let $x\in N'$ and $\Gamma=x\cdot \Gamma_{0}$ be the transform under $x$ of the standard lattice $\Gamma_{0}=\mathfrak{o}(S)^{n}$. This is a $\mathfrak{o}(S)$-lattice stable under $L$ hence, by (i), also stable under $\mathbf{GL}(n,\mathfrak{o}(S))$. For $v\in V-S$, the local lattice $\mathfrak{o}_{v}\cdot \Gamma$ in $k^{n}_{v}$ is then stable under $\mathbf{GL}(n,\mathfrak{o}_{v})$. There exists therefore a power $v^{a(v)}(a(v)\in Z)$ of $v$ such that $\mathfrak{o}_{v}\cdot \Gamma=v^{a(v)}\cdot \mathfrak{o}^{n}_{v}$. We have then also $\mathfrak{o}_{v}\cdot (\det x)=v^{n\cdot a(v)}$. In view of the relation\pageoriginale between a lattice and its localizations, we have then $\Gamma=\mathfrak{a}\cdot \Gamma_{0}$ with $\mathfrak{a}=\Pi \mathfrak{v}^{a(v)}$, and moreover $\mathfrak{a}^{n}\cdot \mathfrak{o}(S)=\mathfrak{o}(S)\cdot (\det x)$. By assigning to $x$ the image of $\mathfrak{a}\cdot \mathfrak{o}(S)$ in $I(k,S)$, we define therefore a map $\alpha$ of $N'$ into ${}_{n}I(k,S)$, which is obviously a homomorphism. If $d\in D$, then $\alpha(d\cdot x)=\alpha(x)$, whence a homomorphism of $A$ into ${}_{n}I(k,S)$, to be denoted also by $\alpha$. Clearly, $H_{\mathfrak{o}(S)}\subset \ker \alpha$. Conversely, assume that $x\in\ker \alpha$. Then $x\cdot \Gamma_{0}$ is homothetic to $\Gamma_{0}$, and there exists $d\in k^{*}$ such that $d\cdot x$ leaves $\Gamma_{0}$ stable. But then $d\cdot x\in H_{\mathfrak{o}(S)}$, so that the image of $x$ in $\Aut L$ belongs to $B$. Thus, $A/B$ is isomorphic to a subgroup of ${}_{n}I(k,S)$. But $N(L)/L$ is of exponent $m$ by Lemma \ref{art04-lem4.1}, and $m$ divides $n$, therefore $\alpha$ maps $A/B$ into a subgroup of ${}_{m}I(k,S)$.
\end{itemize}

Let $\sigma:H\to H/G$ be the canonical projection. Its restriction to $D$ is the projection $D\to D/Z(G)$, and $H/G=D/Z(G)$. If an element $x\in H_{\mathfrak{o}(S)}$ defines an inner automorphism of $L$, then $x\in D_{\mathfrak{o}(S)}\cdot L$, and $\sigma(x)\in \sigma(D_{\mathfrak{o}(S)})$. Since elements of $D_{\mathfrak{o}(S)}$ define trivial automorphisms of $L$, we see that
\begin{equation*}
B/\Int L\cong \sigma (H_{\mathfrak{o}(S)})/\sigma(D_{\mathfrak{o}(S)}).\tag{1}\label{art04-lem4.5-eq1}
\end{equation*}
Identify $D/Z(G)$ to $\mathbf{GL}_{1}$. Then $\sigma(H_{\mathfrak{o}(S)})$ is an $S$-arithmetic subgroup of $\mathbf{GL}_{1}$ hence a subgroup of finite index of $\mathfrak{o}(S)^{*}$. The group $Z(G)$ is cyclic of order $m$, therefore the projection $D=\mathbf{GL}_{1}\to D'$ is either $x\mapsto x^{m}$ or $x\mapsto x^{-m}$, hence $\sigma(D_{\mathfrak{o}(S)})\cong \mathfrak{o}(S)^{*(m)}$, so that $B/\Int L$ may be identified to a subgroup of $\mathfrak{o}(S)^{*}/\mathfrak{o}(S)^{*(m)}$. Thus \eqref{art04-lem4.5-eq1} yields an injective homomorphism $\tau : B/\Int L\to \mathfrak{o}(S^{*})/\mathfrak{o}(S)^{*(m)}$. There remains to show that $\tau$ is surjective.

Let $\pi:H\to H/D=G/Z(G)=\Int G$ be the canonical projection, $T$ the maximal torus given by the splitting of $G$ and $T'=\pi(T)$. We have already remarked that $x\in H_{\mathfrak{o}(S)}$ defines an inner automorphism of $L$ if and only if $x\in D_{\mathfrak{o}(S)}\cdot L$, so $B/\Int L\cong \pi(H_{\mathfrak{o}(S)})/\pi(L)$. By Lemma \ref{art04-lem4.1}, $\pi(N(L))\cong (G/Z(G))_{\mathfrak{o}(S)}$. On the other hand, since $TD$ is split, $D$ is a direct factor over $k$; this implies immediately that $\pi:(TD)_{\mathfrak{o}(S)}\to T'_{\mathfrak{o}(S)}$ is surjective, hence $\pi(H_{\mathfrak{o}(S)})\cap T'=T'_{\mathfrak{o}(S)}$. We have $\pi(D_{\mathfrak{o}(S)}\cdot L)=\pi(L)$, and consequently, since $\ker \pi \cap G\subset T$,
$$
\pi (D_{\mathfrak{o}(S)}\cdot L)\cap T'=\pi(L\cap T)\pi=(T_{\mathfrak{o}(S)});
$$
hence\pageoriginale $B/\Int L$ contains a subgroup isomorphic to $T'_{\mathfrak{o}(S)}/\pi(T_{\mathfrak{o}(S)})$. However, the kernel of $\pi:T\to T'$ is a cyclic group of order $m$. It is then elementary that we can write $T=T_{1}\times T_{2}$, over $k$, with $T_{1}$ containing $Z(G)$ of dimension one. This implies
$$
T'_{\mathfrak{o}(S)}/\pi(T_{\mathfrak{o}(S)})\cong \pi(T_{1})_{\mathfrak{o}(S)}/\pi(T_{1,\mathfrak{o}(S)})=\mathfrak{o}(S)^{*}/\mathfrak{o}(S)^{*(m)};
$$
this shows that the order of $B/\Int L$ exceeds that of $\mathfrak{o}(S)^{*}/\mathfrak{o}(S)^{*(m)}$. Therefore $T$ is surjective.

\begin{description}
\item[{\bf Examples \thnum{4.6}.\label{art04-exams4.6}} {\rm(1)}]
$G=\mathbf{SL}_{n}\cdot H=\mathbf{GL}_{n}$, $(n\geq 3)$. The group $L=\mathbf{SL}(n,\mathfrak{o}(S))$ is equal to its derived group \cite{art04-key3}, Corollary 4.3. By Lemma \ref{art04-lem4.1} $\Aut L$ is generated by $\Aut (k,S)$, acting by conjugation on the coefficients, by the automorphism $x\mapsto {}^{t}x^{-1}$, and by the image $A$ in $\Aut L$ of $N(L)$.

If $\mathfrak{a}$ is an $\mathfrak{o}(S)$-ideal, then $\mathfrak{a}\cdot \Gamma_{0}$ is isomorphic to $\mathfrak{a}^{n}\oplus \mathfrak{o}(S)^{n-1}$ by standard facts on lattices. Therefore, if $\mathfrak{a}^{n}$ is principal, then $\mathfrak{a}\cdot \Gamma_{0}$ is isomorphic to $\Gamma_{0}$ and there exists $g\in \mathbf{GL}(n,k)$ such that $g\cdot \Gamma_{0}=\mathfrak{a}\cdot \Gamma_{0}$. But the stabilizer of $\Gamma_{0}$ in $G$ is the same as that of $\mathfrak{a}\cdot \Gamma_{0}$, hence $g\in N_{H}(L)_{k}$, which shows that, in this case, the monomorphism $A/B\to {}_{n}I(k,S)$ is an isomorphism. We have therefore a composition series
$$
\Aut L\supset A'\supset A\supset B\supset \Int L,
$$
whose successive quotients are isomorphic to $\Aut(k,S)$, $\bfZ/2\bfZ$, ${}_{n}I(k,S)$ and $\mathfrak{o}(S)^{*}/\mathfrak{o}(S)^{*(n)}$.

This result is contained in \cite{art04-key24}, where $\Aut \mathbf{SL}(n,Q)$ is determined for any commutative integral domain $Q$, except for the fact that the structure of the subgroup corresponding to $A/\Int L$ is not discussed there. For $\mathfrak{o}(S)=\mathfrak{o}$, it is related to those of \cite{art04-key19} if $k$ has class number one, and of \cite{art04-key20} if $k=\bfQ(i)$.

\item[\rm(2)] $G=\mathbf{SL}_{2}$, $\card S\geq 2$. The above discussion of $A/\Int L$ is still valid, (without restriction on $S$, in fact). Furthermore, the contragredient mapping $x\mapsto {}^{t}x^{-1}$ is an inner automorphism for $n=2$. However, in general, $L$ is not equal to its commutator subgroup, and $L/(L,L)$ has a non-trivial 2-primary component. Therefore there may be non-trivial automorphisms of the form $x\to f(x)\cdot x$ where\pageoriginale $f$ is a character of order two of $L$. Clearly, such a homomorphism of $L$ into itself is bijective if and only if $\chi(-1)=1$. It follows from Lemma \ref{art04-lem4.2} that $\Aut L$ is generated by automorphisms of the previous type, field automorphisms, and elements of $A$.

We note that this conclusion does not hold true without some restriction on $k$, $S$. For instance, there is one further automorphism if $k=\bfQ(i)$, $\mathfrak{o}(S)=\bfZ(i)$, (see \cite{art04-key20}, and also \cite{art04-key21} for a further discussion of the case $n=2$).

\item[\rm(3)] $G=\mathbf{Sp}_{2n}$, $L=\mathbf{Sp}(2n,\mathfrak{o}(S))$. The commutator subgroup of $L$ is equal to $L$ if $n\geq 3$, and has index a power of two if $n=2$ (\cite{art04-key3}, Remark to 12.5). The group $G$ has no outer automorphisms, therefore, if $n\geq 3$, Theorem \ref{art04-thm4.3}, and Lemma \ref{art04-lem4.5} show that we have a composition series
$$
\Aut L\supset A\supset B\supset \Int L,
$$
with
$$
\Aut L/A\cong \Aut (k,S),\quad B/\Int L\cong\mathfrak{o}(S)^{*}/\mathfrak{o}(S)^{*(2)},
$$
and $A/B$ isomorphic to a subgroup of ${}_{2}I(k,S)$. We claim that in fact
$$
A/B\cong {}_{2}I(k,S).
$$

We write the elements of $\mathbf{GL}_{2n}$ as $2\times 2$ matrices whose entries are $n\times n$ matrices. $\mathbf{Sp}_{2n}$ is the group of elements in $\mathbf{GL}_{2n}$ leaving $J=\left(\begin{smallmatrix} 0 & 1\\ -1 & 0\end{smallmatrix}\right)$ invariant, and its normalizer $H$ in $\mathbf{GL}_{2n}$ is the group of similitudes of $J$. Let $\mathfrak{a}$ be an $\mathfrak{o}(S)$-ideal such that $\mathfrak{a}^{2}$ is principal. As remarked above, there exists $x\in \mathbf{GL}(2,k)$ such that $x\cdot \mathfrak{o}(S)^{2}=\mathfrak{a}\cdot \mathfrak{o}(S)^{2}$. Let $y$ be the element of $\mathbf{GL}(2n,k)$ which acts via $x$ on the space spanned by the $i$-th and $(n+i)$-th canonical basis vectors $(i=1,\ldots,n)$. Then $y\in H_{k}$, and $y\cdot \mathfrak{o}(S)^{2n}=\mathfrak{a}\cdot \mathfrak{o}(S)^{2n}$. Thus, $y$ is an element of $N_{H}(L)_{k}$ which is mapped onto the image of $\mathfrak{a}$ in ${}_{2}I(k,S)$ by the homomorphism $\alpha:A/B\to {}_{2}I(k,S)$ of Lemma \ref{art04-lem4.5}. Hence, $\alpha$ is also surjective.

If $n=2$, $\Aut L$ is obtained by combining automorphisms of the above types with those of the form $x\mapsto f(x)\cdot x$, where $f$ is a homomorphism of $L$ into $\pm 1$ whose kernel contains $-1$.
\end{description}

\noindent
{\bf Remark \thnum{4.7}.\label{art04-rem4.7}}
It\pageoriginale was noticed in Lemma \ref{art04-lem4.1} that $L$ is equal to its normalizer in $G_{k}$. Since $L$ has finite index in its normalizer in $G$, (Proposition \ref{art04-prop3.3} (d)), this means that $L$ is not a proper normal subgroup of an arithmetic subgroup of $G$. More generally, we claim that $L$ is maximal among arithmetic subgroups, i.e. that no subgroup $M$ of $G_{k}$ contains $L$ as a proper subgroup of finite index. This was proved by Matsumoto \cite{art04-key23} when $S=V_{\infty}$, and his proof extends immediately to the present case. In fact, the argument in the proof of Theorem 1 of \cite{art04-key23} shows that if $L$ has finite index in $M\subset G_{k}$, then the closure of $M$ in $G_{v}(v\in V, v\not\in S)$ is contained in $G_{\mathfrak{o}_{v}}$, whence $M\subset L$.

\section{Uniform subgroups in \texorpdfstring{$G_{S}$}{Gs}.}\label{art04-sec5}

In \S1, we proved the finiteness of $E(L)$ for subgroups which are either uniform or arithmetic. In \S3 the arithmetic case was extended to $S$-arithmetic groups. Now a $S$-arithmetic group may be viewed as a discrete subgroup of $G_{S}$, which is irreducible in the sense that its intersection with any proper partial product of the $G_{v}$'s $(v\in S)$ reduces to the identity. We wish to point out here that there is also a generalization to $G_{S}$ of the uniform subgroup case. We assume that $S\neq V_{\infty}$. Such groups have been considered by Ihara \cite{art04-key14} for $G=\mathbf{SL}_{2}$, and Lemma \ref{art04-lem5.1} is an easy extension of results of his.

\medskip
\noindent
{\bf Lemma \thnum{5.1}.\label{art04-lem5.1}}
{\em Let $G$ be a connected semi-simple, almost simple $k$-group, $L$ a uniform irreducible subgroup of $G_{S}$, and $L'$ its projection on $G_{\infty}$.}
\begin{itemize}
\item[{\rm(i)}] {\em $L$ is finitely generated.}

\item[{\rm(ii)}] {\em If $\rank G\geq 2$ and $G_{\infty}$ has no compact or three-dimensional factor or $k=\bfQ$, $G=\mathbf{SL}_{2}$, then $H^{1}(L',\mathfrak{g}_{\infty})=0$.}
\end{itemize}

\smallskip

\begin{itemize}
\item[(i)] Let $S'=S-V_{\infty}$, $G_{S'}=\prod_{v\in S}$, $G_{v}$, and $K=G_{\infty}\times \prod_{v\in S'}G_{0v}$. The latter is an open subgroup of $G_{S}$. The orbits of $K$ in $G_{S}/L$ are open, hence closed, hence compact. Therefore $L_{0}=L\cap K$ is uniform in $K$. Since $K$ is the product of $G_{\infty}$ by a compact group, the projection $L'_{0}$ of $L_{0}$ in $G_{\infty}$ is a discrete uniform subgroup of $G_{\infty}$. But $G_{\infty}$ is a real Lie group with a finite number of connected components. Therefore\pageoriginale the standard topological argument shows that $L_{0}$ is finitely generated. Let $L''$ be the projection of $L$ on $G_{S'}$. Since $L$ is uniform in $G_{S}$, there exists a compact subset $C$ of $G_{S'}$ such that $G_{S}=L''\cdot C.$ On the other hand, it follows from (\cite{art04-key9}, 13.4) that $G_{S'}$ has a compact set of generators, say $D$. Then the standard Schreier-Reidemeister procedure to find generators for a subgroup shows that $L$ is generated by $L\cap (G_{\infty}\times D\cdot C\cdot D^{-1})$ and consequently by $L_{0}$ and finitely many elements. (This argument is quite similar to the one used by Kneser \cite{art04-key17} to prove the finite generation of $G_{S}$.)

\item[(ii)] We first notice that the restriction map 
$$
r:H^{1}(L',\mathfrak{g}_{\infty})\to H^{1}(L'_{0},\mathfrak{g}_{\infty})
$$ 
is injective. The argument is the same as one of Ihara's (\cite{art04-key14}, p.269) in the case $G=\mathbf{SL}_{2}$ : if $x\in G_{S}$, then $x\cdot K\cdot x^{-1}$ is commensurable with $K$, hence, if $x\in L$, the group $L_{0,x}=x\cdot L_{0}\cdot x^{-1}\cap L_{0}$ has finite index in $L_{0}$. In particular, $L'_{0,x}$ is uniform in $G_{\infty}$, hence, by density \cite{art04-key4}, has no fixed vector $\neq 0$ in $\mathfrak{g}_{\infty}$. This implies by Lemma \ref{art04-lem3.5} that $\ker r=0$. If $\rk(G)\geq 2$, then $H^{1}(L'_{0},\mathfrak{g}_{\infty})=0$ by \cite{art04-key32} and \cite{art04-key33}, whence our assertion in this case. If $G=\mathbf{SL}_{2}$, $k=\bfQ$, the vanishing of $H^{1}(L',\mathfrak{g}_{\infty})$ has been proved by Ihara, loc. cit. (it is stated there only in the case where $S$ consists of $\infty$ and one prime, but the proof is a {\em fortiori} valid in the more general case).
\end{itemize}

\medskip
\noindent
{\bf Theorem \thnum{5.2}.\label{art04-thm5.2}}
{\em Let $G$ and $L$ be as in Lemma \ref{art04-lem5.1} (ii). Then $E(L)$ is finite, and $L$ is not isomorphic to a proper subgroup of finite index.}
\smallskip

Identify $L$ to its projection $L'$ in $G_{\infty}$. Then the theorem follows from Lemma \ref{art04-lem1.1} and Lemma \ref{art04-lem1.3} in the same way as in the case $S=V_{\infty}$.

\bigskip
\begin{center}
{\bf APPENDIX}
\end{center}

\section{On compact Clifford-Klein forms of symmetric spaces with negative curvature.}\label{art04-sec6}

\subsection{}\label{art04-sec6.1}

Let $M$ be a simply connected and connected Riemannian symmetric space of negative curvature, without flat component. A Clifford-Klein form of $M$ is the quotient $M/L$ of $M$ by a properly discontinuous group of isometries acting freely, endowed with the metric induced from the given metric on $M$. In an earlier paper (Topology 2 (1963), 111-122), it was proved that $M$ always\pageoriginale has at least one compact Clifford-Klein form. In answer to a question of H. Hopf, we point out here that $M$ always has infinitely many different compact forms. More precisely:

\medskip
\noindent
{\bf Theorem \thnum{6.2}.\label{art04-thm6.2}}
{\em Let $M$ be as in \ref{art04-sec6.1}. Then $M$ has infinitely many compact Clifford-Klein forms with non-isomorphic fundamental groups.}
\smallskip

$M$ is the direct product of irreducible symmetric spaces. We may therefore assume $M$ to be irreducible. Then $M=G/K$, where $G$ is a connected simple non-compact Lie group, with center reduced to $\{e\}$, and $K$ is a maximal compact subgroup of $G$. Moreover, $G$ is the identity component of the group of isometries of $M$. Let $L$ be a discrete uniform subgroup of $G$, without elements of finite order $\neq e$. Then $L$ operates freely, in a properly discontinuous manner, on $M$, and $M/L$ is compact. Moreover, by a known result of Selberg (see e.g. loc. cit., Theorem B), $L$ has subgroups of arbitrary high finite index. Since $M$ is homeomorphic to euclidean space, $L$ is isomorphic to the fundamental group of $M/L$; it suffices therefore to show that $L$ is not isomorphic to any proper subgroup $L'$ of finite index. If $\dim G=3$, then $M$ is the upper half-plane, and this is well known. It follows for instance from the relations
\begin{equation*}
\chi(M/L')=[L:L']\cdot \chi (M/L)\neq 0,\tag{1}\label{art04-thm6.2-eq1}
\end{equation*}
where $[L:L']$ is the index of $L'$ in $L$, and $\chi(X)$ denotes the Euler-Poincar\'e-characteristic of the space $X$. If $\dim G>3$, our assertion is a consequence of Proposition \ref{art04-prop1.7}. If $\chi(M/L)\neq 0$, which is the case if and only if $G$ and $K$ have the same rank, one can of course also use \eqref{art04-thm6.2-eq1}.

\begin{thebibliography}{99}
\bibitem{art04-key1} \textsc{N. Allan :} The problem of maximality of arithmetic groups, {\em Proc. Symp. pur. math.} 9, Algebraic groups and discontinuous subgroups, {\em A. M. S., Providence, R. I.,} (1966), 104-109.

\bibitem{art04-key2} \textsc{N. Allan :} Maximality of some arithmetic groups, {\em Annals of the Brazilian Acad. of Sci.,} (to appear).

\bibitem{art04-key3} \textsc{H. Bass,\pageoriginale J. Milnor} and \textsc{J.-P. Serre :} Solution of the congruence subgroup problem for $\mathbf{SL}_{n}(n\geq 3)$ and $\mathbf{Sp}_{2n}(n\geq 2)$, {\em Publ. Math. I.H.E.S.} (to appear).

\bibitem{art04-key4} \textsc{A. Borel :} Density properties for certain subgroups of semi-simple groups without compact factors, {\em Annals of Math.} 72 (1960), 179-188.

\bibitem{art04-key5} \textsc{A. Borel :} Some finiteness properties of adele groups over number fields, {\em Publ. Math. I.H.E.S.} 16 (1963), 5-30.

\bibitem{art04-key6} \textsc{A. Borel :} Density and maximality of arithmetic groups, {\em J. f. reine u. ang. Mathematik} 224 (1966), 78-89.

\bibitem{art04-key7} \textsc{A. Borel} and \textsc{Harish-Chandra}, Arithmetic subgroups of algebraic groups, {\em Annals of Math.} (2) 75 (1962), 485-535.

\bibitem{art04-key8} \textsc{A. Borel} and \textsc{J-P. Serre :} Th\'eor\`emes de finitude en cohomologie galoisienne, {\em Comm. Math. Helv.} 39 (1964), 111-164.

\bibitem{art04-key9} \textsc{A. Borel} and \textsc{J. Tits :} Groupes r\'eductifs, {\em Publ. Math. I.H.E.S.} 27 (1965), 55-150.

\bibitem{art04-key10} \textsc{C. Chevalley :} Certains sch\'emas de groupes semi-simples, {\em S\'em. Bourbaki} (1961), {\em Exp.} 219.

\bibitem{art04-key11} \textsc{M. Hall :} A topology for free groups and related groups, {\em Annals of Math.} (2) 52 (1950), 127-139.

\bibitem{art04-key12} \textsc{L. K. Hua} and \textsc{I. Reiner :} Automorphisms of the unimodular group, {\em Trans. A. M. S.} 71 (1955), 331-348.

\bibitem{art04-key13} \textsc{L. K. Hua} and \textsc{I. Reiner :} Automorphisms of the projective unimodular group, {\em Trans. A. M. S.} 72 (1952), 467-473.

\bibitem{art04-key14} \textsc{Y. Ihara :} Algebraic curves $\mod p$ and arithmetic groups, {\em Proc. Symp. pure math.} 9, Algebraic groups and discontinuous subgroups, {\em A.M.S., Providence, R.I.} (1966), 265-271.

\bibitem{art04-key15} \textsc{Y. Ihara :} On discrete subgroups of the two by two projective linear group over $p$-adic fields, {\em Jour. Math. Soc. Japan} 18 (1966), 219-235.

\bibitem{art04-key16} \textsc{N. Iwahori} and \textsc{H. Matsumoto :} On some Bruhat decompositions and the structure of the Hecke rings of $p$-adic Chevalley groups, {\em Publ. Math. I.H.E.S.} 25 (1965), 5-48.

\bibitem{art04-key17} \textsc{M. Kneser :}\pageoriginale Erzeugende und Relationen verallemeinerter Einheitsgruppen, {\em Jour. f. reine u. ang. Mat.} 214-15 (1964), 345-349.

\bibitem{art04-key18} \textsc{B. Kostant :} Groups over $\bfZ$, {\em Proc. Symp. pure mat.} 9, Algebraic groups and discontinuous subgroups, {\em A. M. S., Providence, R.I.,} 1966, 90-98.

\bibitem{art04-key19} \textsc{J. Landin} and \textsc{I. Reiner :} Automorphisms of the general linear group over a principal ideal domain, {\em Annals of Math.} (2) 65 (1957), 519-526.

\bibitem{art04-key20} \textsc{J. Landin} and \textsc{I. Reiner :} Automorphisms of the Gaussian modular group, {\em Trans. A. M. S.} 87 (1958), 76-89.

\bibitem{art04-key21} \textsc{J. Landin} and \textsc{I. Reiner :} Automorphisms of the two-dimensional general linear group over a Euclidean ring, {\em Proc. A.M.S.} 9 (1958), 209-216.

\bibitem{art04-key22} \textsc{H. Matsumoto :} Subgroups of finite index in certain arithmetic groups, {\em Proc. Symp. pure math.} 9, Algebraic groups and discontinuous subgroups, {\em A.M.S., Providence, R.I.} (1966), 99-103.

\bibitem{art04-key23} \textsc{H. Matsumoto :} Sur les groupes semi-simples d\'eploy\'es sur un anneau principal, {\em C. R. Acad. Sci. Paris} 262 (1966), 1040-1042.

\bibitem{art04-key24} \textsc{O. T. O'Meara :} The automorphisms of the linear groups over any integral domain, {\em Jour. f. reine u. ang. Mat.} 223 (1966), 56-100.

\bibitem{art04-key25} \textsc{M. S. Raghunathan :} Cohomology of arithmetic subgroups of algebraic groups I, {\em Annals of Math.} (2) 86 (1967), 409-424.

\bibitem{art04-key26} \textsc{M. S. Raghunathan :} Cohomology of arithmetic subgroups of algebraic groups II, (ibid).

\bibitem{art04-key27} \textsc{I. Reiner :} Automorphisms of the symplectic modular group, {\em Trans. A.M.S.} 80 (1955), 35-50.

\bibitem{art04-key28} \textsc{I. Reiner :} Real linear characters of the symplectic unimodular group, {\em Proc. A. M. S.} 6 (1955), 987-990.

\bibitem{art04-key29} \textsc{J. P. Serre :} Le probl\`eme des groupes de congruence pour $SL_{2}$, (to appear).

\bibitem{art04-key30} \textsc{R. Steinberg :}\pageoriginale Representations of algebraic groups, {\em Nagoya M.J.} 22 (1963), 33-56.

\bibitem{art04-key31} \textsc{A. Weil :} Adeles and algebraic groups, {\em Notes, The Institute for Advanced Study, Princeton, N. J.} 1961.

\bibitem{art04-key32} \textsc{A. Weil :} On discrete subgroups of Lie groups II, {\em Annals of Math.} (2) 75 (1962), 578-602.

\bibitem{art04-key33} \textsc{A. Weil :} Remarks on the cohomology of groups, (ibid), (2) 80 (1964), 149-177.

\end{thebibliography}

\medskip
\noindent
The Institute for Advanced Study,

\noindent
Princeton, N.J.
