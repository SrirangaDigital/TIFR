\title{ZETA-FUNCTIONS AND MELLIN TRANSFORMS}
\markright{Zeta-Functions and Mellin Transforms}

\author{By~~ Andr\'e Weil}
\date{}

\maketitle

\setcounter{pageoriginal}{408}
\textsc{Classically,}\pageoriginale the concept of Mellin transform serves to relate Dirichlet series with automorphic functions. Recent developments indicate that this seemingly special device lends itself to broad generalizations, which promise to be of great importance for number-theory and group-theory. My purpose in this lecture is to discuss a typical example, arising from a specific number-theoretical problem.

By an $A$-field, I understand either an algebraic number-field or a function-field of dimension $1$ over a finite filed of constants. Such fields, also sometimes called ``global fields'', are those for which one can build up a classfield theory and the theory of $L$-functions; these topics are treated in my book {\em Basic Number Theory} (\cite{art21-key3}; henceforth quoted as BNT), and the notations in that book will be used freely here. In particular, if $k$ is an $A$-field, its adele ring and its idele group will be denoted by $k_{A}$ and by $k^{\times}_{A}$, respectively; I shall write $|z|$, instead of $|z|_{A}$, for the module of an idele $z$.

Write $\mathfrak{M}$ for the free group generated by the finite places of $k$; this will be written multiplicatively; it may be identified in an obvious manner with the group $I(k)$ of the fractional ideals of $k$, if $k$ is an algebraic number-field, and with the group $D(k)$ of the divisors of $k$, if $k$ is a function-field (except that $D(k)$ is written additively). For each finite place $v$ of $k$, write $\mathfrak{p}_{v}$ for the corresponding generator of $\mathfrak{M}$; then we define a morphism $\mu$ of $k^{\times}_{A}$ onto $\mathfrak{M}$ by assigning, to each idele $z=(z_{v})$, the element $\mu(z)=\Pi \mathfrak{p}^{n(v)}_{v}$ of $\mathfrak{M}$, where $n(v)=\ord (z_{v})$ and the product is taken over all the finite places of $k$. If $\mathfrak{m}=\mu(z)$, we write $|\mathfrak{m}|=\Pi |z_{v}|_{v}$, the product being taken over the same places; thus we have $|\mathfrak{m}|=\mathfrak{N}(\mathfrak{m})^{-1}$ if $k$ is an algebraic number-field, $\mathfrak{N}$ denoting the norm of an ideal in the usual sense, and $|\mathfrak{m}|=q^{-\deg(\mathfrak{m})}$ if $k$ is a function-field, $q$ being the number of elements of the field of constants of $k$ (i.e., of the largest finite field in $k$). We say that $\mathfrak{m}=\Pi \mathfrak{p}^{n(v)}_{v}$ is integral if $n(v)\geq 0$ for all $v$, and\pageoriginale we write $\mathfrak{M}_{+}$ for the set (or semigroup) of all such elements of $\mathfrak{M}$; clearly $|\mathfrak{m}|\leq 1$ if $\mathfrak{m}$ is in $\mathfrak{M}_{+}$, and $|\mathfrak{m}|<1$ if at the same time $\mathfrak{m}\neq 1$.

By a {\em Dirichlet series belonging to $k$}, we will understand any {\em formal} series $L$, with complex-valued coefficients, of the form
\begin{equation}
L(s)=\Sigma c(\mathfrak{m})|\mathfrak{m}|^{s}\label{art21-eq1}
\end{equation}
where the sum is taken over all integral elements $\mathfrak{m}$ of $\mathfrak{M}$, i.e. over all $\mathfrak{m}\in \mathfrak{M}_{+}$. Such series make up a ring (addition and multiplication being defined formally in the obvious manner); the invertible ones, in that ring, are those for which the constant term $c(1)$ is not $0$. Set-theoretically, one may identify this ring with the set of all mappings $\mathfrak{m}\to c(\mathfrak{m})$ of $\mathfrak{M}_{+}$ into the field $\bfC$ of complex numbers; it will always be understood that such a mapping, when it arises in connexion with a Dirichlet series, is extended to $\mathfrak{m}$ by putting $c(\mathfrak{m})=0$ whenever $\mathfrak{m}$ is not integral. The series (1) is absolutely convergent in some half-plane $\rRe(s)>\sigma$ if and only if there is $\alpha\in \bfR$ such that $c(\mathfrak{m})=O(|\mathfrak{m}|^{-\alpha})$; then it determines a holomorphic function in that half-plane; this will be so for all the Dirichlet series to be considered here. However, the knowledge of that function does {\em not} determine the coefficients $c(\mathfrak{m})$ uniquely, except when $k=\bfQ$, so that it does not determine the Dirichlet series (1) in the sense in which we use the word here. A case of particular importance is that in which the function given by \eqref{art21-eq1} in its half-plane of absolute convergence can be continued analytically, as a holomorphic or as a meromorphic function, throughout the whole $s$-plane; then the latter function is also usually denoted by $L(s)$.

Let $v$ be a finite place of $k$, and let $\mathfrak{p}_{v}$ be as above. We will say that the series $L$ given by \eqref{art21-eq1} is {\em eulerian} at $v$ if it can be written in the form
$$
(1+c_{1}|\mathfrak{p}_{v}|^{s}+\cdots+c_{m}|\mathfrak{p}_{v}|^{ms})^{-1}\cdot \Sigma c(\mathfrak{m})|\mathfrak{m}|^{s},
$$
where the sum in the last factor is taken over all the elements $\mathfrak{m}$ of $\mathfrak{M}_{+}$ which are disjoint from $\mathfrak{p}_{v}$ (i.e. which belong to the subgroup of $\mathfrak{M}$ generated by the generators of $\mathfrak{M}$ other than $\mathfrak{p}_{v}$). The first factor in the same product is then called the {\em eulerian factor of $L$ at\pageoriginale $v$.} The above condition can also be expressed by saying that there is a polynomial $P(T)=1+c_{1}T+\cdots+c_{m}T^{m}$ such that, if we expand $P(T)^{-1}$ in a power-series $\sum\limits^{\infty}_{0}c'_{i}T^{i}$, we have, whenever $\mathfrak{m}$ is in $\mathfrak{M}_{+}$ and disjoint from $\mathfrak{p}_{v}$, $c(\mathfrak{m}\mathfrak{p}^{i}_{v})=c(\mathfrak{m})c'_{i}$ for all $i\geq 0$. We say that $L$ is {\em eulerian} if it is so at all finite places of $k$.

Let $\omega$ be any character or ``quasicharacter'' of the ideal group $k^{\times}_{A}$, trivial on $k^{\times}$. It is well known that one can associate with it can eulerian Dirichlet series
\begin{equation}
L(s,\omega)=\sum \omega(\mathfrak{m})|\mathfrak{m}|^{s}=\prod\limits_{v}(1-\omega(\mathfrak{p}_{v})|\mathfrak{p}_{v}|^{s})^{-1},\label{art21-eq2}
\end{equation}
known as the $L$-series attached to the ``Gr\"ossencharakter'' defined by $\omega$; its functional equation, which is due to Hecke, is as follows. For each infinite place $w$ of $k$, write the quasicharacter $\omega_{w}$ induced by $\omega$ on $k^{\times}_{w}$ in the form $x\to x^{-A}|x|^{s}w$, with $A=0$ or $1$, if $k_{w}=\bfR$, and $z\to z^{-A}\overline{z}^{-B}(z\overline{z})^{s}w$, with inf $(A,B)=0$, if $k_{w}=\bfC$. Write $G_{1}(s)=\pi^{-s/2}\Gamma(s/2)$, $G_{2}(s)=(2\pi)^{1-s}\Gamma(s)$, $G_{w}=G_{1}$ or $G_{2}$ according as $w$ is real or imaginary, and put
$$
\Lambda (s,\omega)=L(s,\omega)\prod\limits_{w}G_{w}(s+s_{w}),
$$
where the product is taken over the infinite places of $k$. Define the constant $\kappa=\kappa(\omega)$ and the idele $b$ as in Proposition 14, Chapter VII-7, of BNT (page 132); we recall that, if $d$ is a ``differental idele'' (cf. BNT, page 113) attached to the ``basic character'' of $k_{A}$ used in the construction of $\kappa(\omega)$, and if $f(\omega)=(f_{v})$ is an idele such that $f_{v}$ is $1$ at all infinite places and all places where $\omega$ is unramified, and otherwise has an order equal to that of the conductor of $\omega$, then we can take $b=f(\omega)d$. That being so, the functional equation is
\begin{equation}
\Lambda (s,\omega)=\kappa(\omega)\cdot \omega(f(\omega)d)|f(\omega)d|^{s-1/2}\Lambda(1-s,\omega^{-1}).\label{art21-eq3}
\end{equation}

Let now $L$ be again the Dirichlet series defined by \eqref{art21-eq1}; with it, we associate the family of Dirichlet series $L_{\omega}$ given by
\begin{equation}
L_{\omega}(s)=\sum c(\mathfrak{m})\omega(\mathfrak{m})|\mathfrak{m}|^{s}\label{art21-eq4}
\end{equation}
for all choices of the quasicharacter $\omega$ of $k^{\times}_{A}/k^{\times}$, $\omega(\mathfrak{m})$ being as in \eqref{art21-eq2}. Some recent work of mine (c.f. \cite{art21-key2}) and some related unpublished\pageoriginale work by Langlands and by Jacquet\footnote[2]{That work is still in progress. No attempt will be made here to describe its scope, but the reader should know that I have freely drawn upon it; my indebtedness will soon, I hope, be made apparent by their publication. In particular, my definition of the Mellin transform when $k$ is not totally real is based on Langlands' more general ``local functional equation'' for $GL$(2, $\bfC$), even though it is also implicit in some earlier work of Maass (c.f. \cite{art21-key1}, pages 79-80).} has shown that the knowledge of the functional equation, not only for $L$, but also at the same time for ``sufficiently many'' of the series $L_{\omega}$ provides us with valuable information about $L$ and its possible relationship to automorphic functions of certain types. In particular, this is so whenever $L$ is the zeta-function of an elliptic curve $E$ over $k$, provided $E$ is such that the functional equations for the series $L_{\omega}$ can effectively be computed. Unfortunately there are not as many such curves as one could wish; as ``experimental material'', I have been able to use only the following: (a) in characteristic $0$, all the curves $E$ with complex multiplication; their zeta-functions have been obtained by Deuring; (b) also in characteristic 0, some curves, uniformized by suitable types of automorphic functions, which can be treated by the methods of Eichler and Shimura; a typical example is the curve belonging to the congruence subgroup $\Gamma_{0}(11)$ of the modular group, whose equation, due to Fricke\footnote[3]{C.f. F. Klein und R. Fricke, {\em Theorie der elliptischen Modulfunktionen,} Bd. II, Leipzig 1892, page 436.}, is $Y^{2}=1-20X+56X^{2}-44X^{3}$ (Tate has observed that it is isogenous to the curve $Y^{2}-Y=X^{3}-X^{2}$); (c) in any characteristic $p\geq 3$, any curve $E$ of the form $wY^{2}=X^{3}+aX^{2}+bX+c$ where $Y^{2}=X^{3}+aX^{2}+bX+c$ is the equation of an elliptic curve $E_{0}$ over the field of constants $k_{0}$ of $k$, and $w$ is in $k^{\times}$ and not in $(k^{\times})^{2}k^{\times}_{0}$. All these examples exhibit some common features, which can hardly fail to be significant and will now be described.

For the definition of the zeta-function $L(s)$ of the elliptic curve $E$ over $k$, the reader is referred to \cite{art21-key2}; there it is given only for $k=\bfQ$, but in such terms that its extension to the general case is immediate and requires no comment. It is eulerian. Also the conductor of $E$ is to be defined as explained in \cite{art21-key2}; it is an integral element $\mathfrak{a}$ of $\mathfrak{M}$; we\pageoriginale will write $a=(a_{v})$ for an idele such that $\mathfrak{a}=\mu(a)$ and that $a_{v}=1$ whenever $v$ is not one of the finite places occurring in $\mathfrak{a}$. For the examples quoted above, the zeta-functions are as follows: (a) if $E$ has complex multiplication, and $k'$ is the field generated over $k$ by any one of the complex multiplcations of $E$, $L(s)$ is an $L$-series over $k'$, with a ``Gr\"ossencharakter'', if $k'\neq k$, and the product of two such series if $k'=k$; (b) for Fricke's curve belonging to $\Gamma_{0}(11)$, Eichler has shown that the zeta-function is the Mellin transform of the cusp-form belonging to that same group; the curve's conductor is 11; (c) in the last example, let $\chi$ be the character belonging to the quadratic extension $k(w^{1/2})$ of $k$, and let $q^{\alpha}$, $q^{\beta}$ be the roots of the zeta-function of the curve $E_{0}$ over $k_{0}$; then the zeta-function of $E$ is $L(s-\alpha,\chi)L(s-\beta,\chi)$.

In all these examples, one finds that the functional equation for $L_{\omega}$ has a simple form whenever the conductor $\mathfrak{f}=\mu(f(\omega))$ of $\omega$ is disjoint from the conductor $\mathfrak{a}$ of the given curve $E$, and that it is then as follows. For each infinite place $\omega$ of $k$, define $s_{w}$, $A$, $B$ by means of $\omega$, as explained above in describing the functional equation (3) for $L(s,w)$. Put $\mathfrak{G}_{w}(s)=G_{2}(s+s_{w}-A)$ if $w$ is real; put $\mathfrak{G}_{w}(s)=G_{2}(s+s_{w})^{2}$ if $w$ is imaginary and $A=B=0$, and $\mathfrak{G}_{w}(s)=G_{2}(s+s_{w})\cdot G_{2}(s+s_{w}-1)$ if $w$ is imaginary and $A+B>0$. Put $\Lambda_{\omega}(s)=L_{\omega}(s)\cdot \Pi\mathfrak{G}_{w}(s)$, the product being taken over all the infinite places of $k$. Call $R$ the number of such places where $A=0$ (if the place is real) or $A=B=0$ (if it is imaginary). Then :
\begin{equation}
\Lambda_{\omega}(s)=\pm (-1)^{R}\kappa(\omega)^{2}\omega(af(\omega)^{2}d^{2})|af(\omega)^{2}d^{2}|^{s-1}\Lambda_{\omega-1}(2-s),\label{art21-eq5}
\end{equation}
where the sign $\pm$ is independent of $\omega$, and notations are as in \eqref{art21-eq3}.

For $k=\bfQ$, it has been shown in \cite{art21-key2} that $L$ must then be the Mellin transform of a modular form belonging to the congruence subgroup $\Gamma_{0}(a)$ of the modular group. Our purpose is now to indicate that similar results hold true in general.

Once for all, we choose a ``basic'' character $\psi$ of $k_{A}$, trivial on $k$ and not on $k_{A}$, and a ``differential idele'' $d=(d_{v})$ attached to $\psi$; we may choose $\psi$ so that $d_{w}=1$ for every infinite place $w$ of $k$ (c.f. BNT, Chapter VIII-4, Proposition 12, p. 156; this determines $\psi$ uniquely\pageoriginale if $k$ is of characteristic 0); we will assume that it has been so chosen.

We write $G$ for $GL(2)$, so that $G_{k}$ is $GL(2,k)$; as usual, we write then $G_{v}$, $G_{A}$ for $GL(2,k_{v})$, $GL(2,k_{A})$. We identify the center of $G$ with the ``multiplicative group'' $GL(1)$, hence the centers of $G_{k}$, $G_{v}$, $G_{A}$ with $k^{\times}$, $k^{\times}_{v}$, $k^{\times}_{A}$, respectively, by the isomorphism $z\to z\cdot 1_{2}$. {\em All functions to be considered on any one of the groups $G_{v}$, $G_{A}$ will be understood to be constant on cosets modulo the center}, so that they are actually functions on the corresponding projective groups. It is nevertheless preferable to operate in $GL(2)$, since our results can easily be extended to functions with the property $f(gz)=f(g)\omega(z)$, where $\omega$ is a given character of the center, and these useful generalizations can best be expressed in terms of $GL(2)$. By an {\em automorphic function}, we will always understand a continuous function on $G_{A}$, left-invariant under $G_{k}$ (and, as stated above, invariant under the center $k^{\times}_{A}$ of $G_{A}$), with values of $\bfC$ or in a vector-space of finite dimension over $\bfC$; this general concept will be further restricted as the need may arise.

For a matrix of the form $\left(\begin{matrix} x & y\\ 0 & 1\end{matrix}\right)$, we write $(x,y)$; we write $B$ for the group of such matrices (and $B_{k}$, $B_{v}$, $B_{A}$ for the corresponding subgroups of $G_{k}$, $G_{v}$, $G_{A}$). The group $B\cdot G_{m}$, with $G_{m}=GL(1)$, consists of the matrices $\left(\begin{matrix} x & y\\ 0 & z\end{matrix}\right)$, and $G/(B\cdot G_{m})$ may be identified in an obvious manner with the projective line $D$. In particular, $G_{A}/(B_{A}\cdot k^{\times}_{A})$ is the ``adelized projective line'' $D_{A}$; it is compact, and its ``rational points'' (i.e. the ``rational projective line'' $D_{k}$) are everywhere dense in it. It amounts to the same to say that $G_{k}\cdot B_{A}\cdot k^{\times}_{A}$ is dense in $G_{A}$. Consequently, an automorphic function on $G_{A}$ is uniquely determined by its values on $B_{A}$. Let $\Phi$ be such a function; call $F$ the function induced on $B_{A}$ by $\Phi$; $F$ is left-invariant under $B_{k}$, and in particular under $(1,\eta)$ for each $\eta\in k$, so that, for each $x\in k^{\times}_{A}$, the function $y\to F(x,y)$ on $k_{A}$ can be expanded in Fourier series on $k_{A}/k$. Using the basic character $\psi$, and making use of the fact that $F$ is also left-invariant under $(\xi,0)$\pageoriginale for all $\xi\in k^{\times}$, one finds at once that this Fourier series may be written as
\begin{equation}
F(x,y)=f_{0}(x)+\sum\limits_{\xi\in k^{\times}}f_{1}(\xi x)\psi(\xi y),\label{art21-eq6}
\end{equation}
where $f_{0}$, $f_{1}$ are the functions on $k^{\times}_{A}$ respectively given by
$$
f_{0}(x)=\int\limits_{k_{A}/k}F(x,y)dy, \ f_{1}(x)=\int\limits_{k_{A}/k}F(x,y)\psi(-y)dy.
$$
We have $f_{0}(\xi x)=f_{0}(x)$ for all $\xi\in k^{\times}$; we will say that $\Phi$ is $B$-{\em cuspidal} if $f_{0}=0$.

Conversely, when such a Fourier series is given, the function $F$ defined by it on $B_{A}$ is left-invariant under $B_{k}$ and may therefore be extended to a function of $G_{k}\cdot B_{A}\cdot k^{\times}_{A}$, left-invariant under $G_{k}$ (and, as is always assumed, invariant under $k^{\times}_{A}$), and the question arises whether this can be extended by continuity to $G_{A}$. In order to give a partial answer to this question, we must first narrow down the kind of automorphic function which we wish to consider.

We first choose an element $\mathfrak{a}$ of $\mathfrak{M}_{+}$, which will play the role of a conductor, and, as before, an idele $a=(a_{v})$ such that $\mathfrak{a}=\mu(a)$ and that $a_{v}=1$ whenever $v$ does not occur in $\mathfrak{a}$. Also, write $\bfd=(\bfd_{v})$ for the element $(d,0)$ of $B_{A}$, $d$ being the differental idele chosen above. At each finite place $v$ of $k$, the group $GL(2,r_{v})=M_{2}(r_{v})^{\times}$ is a maximal compact subgroup of $G_{v}$, consisting of the matrices $\left(\begin{matrix} x & y\\ z & t\end{matrix}\right)$, with coefficients $x$, $y$, $z$, $t$ in the maximal compact subring $r_{v}$ of $k_{v}$, such that $|xt-yz|_{v}=1$ (i.e. that $xt-yz$ is in $r^{\times}_{v}$); then $\bfd^{-1}_{v}\cdot M_{2}(r_{v})^{\times}\cdot \bfd_{v}$ is also such a subgroup of $G_{v}$, consisting of the matrices $\left(\begin{matrix} x & d^{-1}_{v} & y\\ d_{v}^{z} & t & \end{matrix}\right)$, where $x$, $y$, $z$, $t$ are as before. We will write $\mathfrak{K}_{v}$ for the subgroup of the latter group, consisting of the matrices of that form with $z\in a_{v}r_{v}$; this is a compact open subgroup of $G_{v}$, equal to $M_{2}(r_{v})^{\times}$ at all the finite places which do not occur in $\mu(ad)$. On the other hand, we take for $\mathfrak{K}_{w}$ the orthogonal group $O(2)$ in $2$ variables if $w$ is real, and the unitary group $U(2)$ if $w$ is imaginary. Then the product $\mathfrak{K}=\Pi \mathfrak{K}_{v}$, taken over all the places of $k$, defines a compact subgroup $\mathfrak{K}$\pageoriginale of $G_{A}$; it is open in $G_{A}$ if $k$ is of characteristic $p>1$, but not otherwise. We have $G_{v}=B_{v}\cdot k^{\times}_{v}\cdot \mathfrak{K}_{v}$ for all places $v$ of $k$, except those occurring in $\mathfrak{a}$; consequently, $B_{A}\cdot k^{\times}_{A}\cdot \mathfrak{K}$ is open in $G_{A}$.

We also introduce an element $\bfa=(\bfa_{v})$ of $G_{A}$, which we define by putting $\bfa_{v}=\bfd^{-1}_{v}\cdot \left(\begin{matrix} 0 & -1\\ a_{v} & 0\end{matrix}\right)\cdot \bfd_{v}$ for $v$ finite, and $\bfa_{w}=1_{2}$ for $w$ infinite. Clearly $\mathfrak{K}\bfa=\bfa\mathfrak{K}$.

The automorphic functions $\Phi$ which we wish to consider are to be right-invariant under $\mathfrak{K}_{v}$ for every finite place $v$ of $k$; thus, if $k$ is of characteristic $p>1$, they are right-invariant under $\mathfrak{K}$, hence locally constant. Clearly, if $\Phi$ has that property, the same is true of the function $\Phi'$ given by $\Phi'(g)=\Phi(g\bfa)$. If $k$ is of characteristic $p>1$, we take our functions $\Phi$ to be complex-valued. If $k$ is a number-field, our purposes require that they take their values in suitable vector-spaces, that they transform according to given representations of the groups $\mathfrak{K}_{w}$ at the infinite places $w$ of $k$, and that, at those places, they satisfy additional conditions to be described now.

It is well-known that, if $k_{w}=\bfR$ (resp. $\bfC$), the ``Riemannian symmetric space'' $G_{w}/k^{\times}_{w}\mathfrak{K}_{w}$ may be identified with the hyperbolic space of dimension 2 (resp. 3), i.e. with the Poincar\'e half-plane (resp. half-space). This can be done as follows. Let $B^{+}_{w}$ be the subgroup of $B_{w}$ consisting of the matrices $b=(p,y)$ with $p\in \bfR^{\times}_{+}$ (i.e. $p\in \bfR$, $p>0$) and $y\in k_{w}$. Every element $g$ of $G_{w}$ can be writeen as $g=bz\mathfrak{k}$ with $b\in B^{+}_{w}$, $z\in k^{\times}_{w}$, $\mathfrak{k}\in \mathfrak{K}_{w}$; here $b=(p,y)$, and $z\mathfrak{k}$, are uniquely determined by $g$. We identify $G_{w}/k^{\times}_{w}\mathfrak{K}_{w}$ with the Poincar\'e half-plane (resp. half-space) $H_{w}=\bfR^{\times}_{+}\times k_{w}$ by taking, as the canonical mapping of $G_{w}$ onto $G_{w}/k^{\times}_{w}\mathfrak{K}_{w}$, the mapping $\phi_{w}$ of $G_{w}$ onto $H_{w}$ given by $\phi_{w}(g)=(p,y)$ for $g=bz\mathfrak{k}$, $b=(p,y)$ as above. The invariant Riemannian metric in $H_{w}$ is the one given by $ds^{2}=p^{-2}(dp^{2}+dyd\overline{y})$. On $H_{w}$, consider the differential forms which are left-invariant under $B^{+}_{w}$; a basis for these consists of the forms $\alpha_{1}=p^{-1}(dp+idy)$, $\alpha_{2}=p^{-1}(dp-idy)$ if $k_{w}=\bfR$, and of $\alpha_{1}=p^{-1}dy$, $\alpha_{2}=p^{-1}dp$, $\alpha_{3}=p^{-1}d\overline{y}$ if $k_{w}=\bfC$. Writing $E'_{w}$ for the vector-space $M_{2,1}(\bfC)$ resp. $M_{3,1}(\bfC)$ of column-vectors (with $2$ resp. $3$ rows) over $\bfC$, we will denote by $\alpha_{w}$ the vector-valued differential form on $H_{w}$, with values in $E'_{w}$,\pageoriginale whose components are $\alpha_{1}$, $\alpha_{2}$ resp. $\alpha_{1}$, $\alpha_{2}$, $\alpha_{3}$. One can then describe the action of $k^{\times}_{w}\mathfrak{K}_{w}$ on these forms by writing
$$
\alpha_{w}(\phi_{w}(z\mathfrak{k}b))=\mathfrak{M}_{w}(\mathfrak{k}\alpha_{w}(b),
$$
where $\mathfrak{M}_{w}$ is a representation of $\mathfrak{K}_{w}$ in the space $E'_{w}$; for $k_{w}=\bfR$, for instance, this is given by
{\fontsize{10}{12}\selectfont
$$
\mathfrak{M}_{w}
\left(\left(\begin{matrix}
\cos \theta & \sin \theta\\
-\sin \theta & \cos \theta
\end{matrix}\right)\right)=
\left(
\begin{matrix}
e^{-2i\theta} & 0\\
0 & e^{2i\theta}
\end{matrix}
\right), \  
\mathfrak{M}_{w}
\left(\left(
\begin{matrix}
-1 & 0\\
0 & 1
\end{matrix}
\right)\right)
=
\left(
\begin{matrix}
0 & 1\\
1 & 0
\end{matrix}
\right).
$$}\relax
A basis for the left-invariant differential forms on $G_{w}$ which are $0$ on $k^{\times}_{w}\mathfrak{K}_{w}$ is then given by the components of the vector-valued form
$$
\beta_{w}(g)=\mathfrak{M}_{w}(\mathfrak{k})^{-1}\alpha_{w}(\phi_{w}(g))
$$
where $\mathfrak{k}$ is given, as above, by $g=bz \mathfrak{k}$.

Now write $E_{w}$ for the space of row-vectors $M_{1,2}(\bfC)$ resp. $M_{1,3}(\bfC)$; we regard this as the dual space to $E'_{w}$, the bilinear form $e\cdot e'$ being defined by matrix multiplication for $e\in E_{w}$, $e'\in E'_{w}$. Then, if $h$ is an $E_{w}$-valued function on $G_{w}$, $h\cdot \beta_{w}$ is a complex-valued differential form on $G_{w}$; it is the inverse image under $\phi_{w}$ of a differential form on $H_{w}$ if and only if $h(gz\mathfrak{k})=h(g)\mathfrak{M}_{w}(\mathfrak{k})$ for all $g\in G_{w}$, $z\in k^{\times}_{w}$, $\mathfrak{k}\in \mathfrak{K}_{w}$; when that is so, $h$ is uniquely determined by its restriction $(p,y)\to h(p,y)$ to $B^{+}_{w}$. We will say that $h$, or its restriction to $B^{+}_{w}$, is {\em harmonic} if $h\cdot \beta_{w}$ is the inverse image under $\phi_{w}$ of a harmonic differential form on the Riemannian space $H_{w}$, or, what amounts to the same, if $h$ has the property just stated and if $h(p,y)\cdot \alpha_{w}$ is harmonic on $H_{w}$. For $k_{w}=\bfR$, this is so if and only if the two components of the vector-valued function $p^{-1}h(p,y)$ on $H_{w}$ are respectively holomorphic for the complex structures defined on $H_{w}$ by the complex coordinates $p\pm iy$. We will say that $h$ is {\em regularly harmonic} if it is harmonic and if $h(p,y)=O(p^{N})$ for some $N$ when $p\to +\infty$, uniformly in $y$ on every compact subset of $k_{w}$. If $h$ is harmonic, so is $g\to h(g_{0}g)$ for every $g_{0}\in G_{w}$, since the Riemannian structure of $H_{w}$ is invariant under $G_{w}$ and since the form $\beta_{w}$ is left-invariant on $G_{w}$. If $h$ is regularly harmonic, so is $g\to h(b_{0}g)$ for every $b_{0}\in B^{+}_{w}$. It is easily seen that there is, up to a constant factor, only one regularly harmonic function $\bfh_{w}$ such that
$$
\bfh_{w}((1,y)g)=\psi_{w}(y)\bfh_{w}(g)
$$\pageoriginale
for all $y\in k_{w}$, $g\in G_{w}$; this given by
\begin{gather*}
\bfh_{w}(p,y)=\psi_{w}(y)\bfh_{w}(p),\\
\bfh_{w}(p)=p\cdot (e^{-2\pi p},0) \text{~ if~ } k_{w}=\bfR,\\
\bfh_{w}(p)=p^{2}\cdot (K_{1}(4\pi p),-2i K_{0}(4\pi p), K_{1}(4\pi p))\text{~ if~ } k_{w}=\bfC,
\end{gather*}
where $K_{0}$, $K_{1}$ are the classical Hankel functions\footnote[1]{Cf. G. N. Watson,  {\em A treatise on the theory of Bessel function,} 2nd. ed., Cambridge 1952, page 78.}. For any $x\in k^{\times}_{w}$, we write $\bfh_{w}(x)$ instead of $\bfh_{w}((x,0))$.

It is essential to note that $\bfh_{w}$ satisfies a ``local functional equation'', which, following Langlands, we can formulate as follows. Let $\omega$ be a quasicharacter of $k^{\times}_{w}$; as before, we write it in the form $x\to x^{-A}|x|^{s}$ with $A=0$ or $1$, if $k_{w}=\bfR$, and $z\to z^{-A}\overline{z}^{-B}(z\overline{z})^{s}$ with $\inf(A,B)=0$, if $k_{w}=\bfC$. For $k_{w}=\bfR$, put $\mathscr{G}_{w}(\omega)=G_{2}(1+s-A)$; for $k_{w}=\bfC$, put $\mathscr{G}_{w}(\omega)=G_{2}(s+1)^{2}$ if $A=B=0$, and $\mathscr{G}_{w}(\omega)=G_{2}(s)G_{2}(s+1)$ otherwise. Write $j$ for the matrix $j=\left(\begin{smallmatrix} 0 & 1\\ -1 & 0\end{smallmatrix}\right)$, and put, for $g\in G_{w}$:
\begin{equation}
I_{\omega}(g,\omega)=\int\limits_{k^{\times}_{w}}\bfh_{w}((z,0)g)\omega(z)d^{\times}_{z},\label{art21-eq7}
\end{equation}
where $d^{\times}z$ is a Haar measure on $k^{\times}_{w}$; this is convergent for $\rRe(s)$ large. Then the functional equation is
\begin{equation}
\mathscr{G}_{w}(\omega)^{-1}I_{w}(g,\omega)=(-1)^{\rho}\mathscr{G}_{w}(\omega^{-1})^{-1}I_{w}(j^{-1}g,\omega^{-1})\label{art21-eq8}
\end{equation}
with $\rho=1$ if $k_{w}=\bfR$, or if $k_{w}=\bfC$ and $A=B=0$, and $\rho=A+B$ if $k_{w}=\bfC$ and $A+B>0$. By \eqref{art21-eq8}, we mean that both sides, for given $A$, $B$, $g$, can be continued analytically, as holomorphic functions of $s$, over the whole $s$-plane, and are then equal. This can of course be verified by a straightforward calculation for $k_{w}=\bfR$. A similar calculation for $k_{w}=\bfC$ might not be quite so easy. Both cases, however, are included in more general results of Langlands; moreover, a simple proof for \eqref{art21-eq8} itself in the case $k_{w}=\bfC$, communicated to me by Jacquet, is now available. It will be noticed\pageoriginale that the gamma factors in \eqref{art21-eq8} are essentially the same as those occurring in \eqref{art21-eq5}.

Now we write $k_{\infty}$, $k^{\times}_{\infty}$, $G_{\infty}$, $\mathfrak{K}_{\infty}$, $H_{\infty}$, etc., for the products $\Pi k_{w}$, $\Pi k^{\times}_{w}$, $\Pi G_{w}$, $\Pi \mathfrak{K}_{w}$, $\Pi H_{w}$, etc., taken over the infinite places of $k$. We write $E_{\infty}$, $E'_{\infty}$ for the tensor-products $\otimes E_{w}$, $\otimes E'_{w}$, taken over the same places; these may be regarded as dual to each other. Then $\beta_{\infty}=\otimes \beta_{w}$ is a left-invariant differential form on $G_{\infty}$ with values in $E'_{\infty}$; its degree is equal to the number $r$ of infinite places of $k$; if $h$ is any function on $G_{\infty}$ with values in $E_{\infty}$, $h\cdot \beta_{\infty}$ is then a complex-values differential form of degree $r$ on $G_{\infty}$. We will say that $h$ is {\em harmonic} if $h\cdot \beta_{\infty}$ is the inverse image of a harmonic differential form on $H_{\infty}$; writing $p=(p_{w})$, $y=(y_{w})$ for elements of $(\bfR^{\times}_{+})^{r}$ and $k_{\infty}$, so that $(p,y)$ is an element of $H_{\infty}$, we will say that the harmonic function $h$ is {\em regularly harmonic} if there is $N$ such that $h(p,y)=O(p^{N}_{w})$ for each $w$ when $p_{w}\to +\infty$, uniformly over compact sets with respect to all variables except $p_{w}$. Up to a constant factor, the only regularly harmonic function $\bfh_{\infty}$ such that
$$
\bfh_{\infty}((1,y)g)=\psi_{\infty}(y)\bfh_{\infty}(g)
$$
for all $y\in k_{\infty}$, $g\in G_{\infty}$ is given by $\bfh_{\infty}(g)=\otimes \bfh_{w}(g_{w})$ for $g=(g_{w})$.

We will say that a continuous function $\Phi$ on $G_{A}$, with values in $E_{\infty}$, is a harmonic automorphic function with the conductor $\mathfrak{a}$, or, more briefly, that it is $(h,\mathfrak{a})$-{\em automorphic} if it is left-invariant under $G_{k}$, invariant under $k^{\times}_{A}$, right-invariant under $\mathfrak{K}_{v}$ for every finite place $v$ of $k$, and if, for every $g_{0}\in G_{A}$, the function on $G_{\infty}$ given for $g\in G_{\infty}$ by $g\to \Phi(g_{0}g)$ is harmonic; if $k$ is not of characteristic $0$, the latter condition is empty, and we take $E_{\infty}=\bfC$. The function $\Phi'$ given by $\Phi'(g)=\Phi(g\bfa)$ is then also $(h,\mathfrak{g})$-automorphic. For such a function $\Phi$, we shall now consider more closely the Fourier series defined by \eqref{art21-eq6}. As $\Phi$ is harmonic on $G_{\infty}$ and right-invariant under $\mathfrak{K}_{v}$ for every finite $v$, the same is true of the functions
$$
\Phi_{0}(g)=\int\limits_{k_{A}/k}\Phi((1,y)g)dy,\Phi_{1}(g)=\int\limits_{k_{A}/k}\Phi((1,y)g)\psi(-y)dy
$$
whose restrictions to $B_{A}$ are $F_{0}(x,y)=f_{0}(x)$, $F_{1}(x,y)=f_{1}(x)\psi (y)$, where $f_{0}$, $f_{1}$ are as in \eqref{art21-eq6}. In particular, for every finite $v$, $F_{1}$ is right-invariant\pageoriginale under the group $B_{v}\cap \mathfrak{K}_{v}$, hence under all matrices $(u,0)$ with $u\in r^{\times}_{v}$, and all matrices $(1,d^{-1}_{v}z)$ with $z\in r_{v}$; in view of the definition of the idele $d=(d_{v})$, the latter fact means that $f_{1}(x)=0$ unless $x_{v}\in r_{v}$ for all finite $v$, i.e. unless the element $\mathfrak{m}=\mu(x)$ of $\mathfrak{M}$ is integral; the former fact means that $f_{1}(x)$ depends only upon $\mathfrak{m}$ and upon the components $x_{w}$ of $x$ at the infinite places $w$ of $k$. Putting $x_{\infty}=(x_{w})$, we can therefore write $f_{1}(x)=f_{1}(\mathfrak{m},x_{\infty})$, and this is $0$ unless $\mathfrak{m}$ is in $\mathfrak{M}_{+}$. For similar reasons, we can write $f_{0}(x)=f_{0}(\mathfrak{m},x_{\infty})$.

If $k$ is of characteristic $p>1$, this can be written $f_{1}(x)=f_{1}(\mathfrak{m})$, $f_{0}(x)=f_{0}(\mathfrak{m})$. As $f_{1}(\mathfrak{m})$ is $0$ unless $\mathfrak{m}$ is in $\mathfrak{M}_{+}$, only finitely many terms of the Fourier series \eqref{art21-eq6} can be $\neq 0$ for each $(x,y)$; they are all $0$, except possibly $f_{0}(x)$, if $|x|>1$, since this implies $|\xi x|>1$ for all $\xi\in k^{\times}$. On the other hand, if $k$ is of characteristic $0$, the convergence of the Fourier series follows from the fact that $\Phi$, being harmonic, must be analytic in $g_{\infty}=(g_{w})$.

Now we add three more conditions for $\Phi$:
\begin{itemize}
\item[(I)] $\Phi$ should be $B$-cuspidal, i.e. $f_{0}$ should be $0$.

\item[(II)] If $k$ is of characteristic $0$, $\Phi$ should be regularly harmonic on $G_{\infty}$, when the coordinates $g_{v}$ at the finite places are kept constant. Then the same is true of $\Phi_{1}$; in view of what we have found above, this implies that $f_{1}(\mathfrak{m},x_{\infty})$ is a constant scalar multiple of $\bfh_{\infty}(x_{\infty})$ for every $\mathfrak{m}$, so that we can write
$$
f_{1}(\mathfrak{m},x_{\infty})=c(\mathfrak{m})\bfh_{\infty}(x_{\infty}),
$$
where $c(\mathfrak{m})$ is a complex-valued function on $\mathfrak{M}$, equal to $0$ outside $\mathfrak{M}_{+}$. In the case of characteristic $p>1$, we write $c(\mathfrak{m})=f_{1}(\mathfrak{m})$.

\item[(III)] We assume that $c(\mathfrak{m})=O(|\mathfrak{m}|^{-\alpha})$ for some $\alpha$; (I) and (II) being assumed, this implies that $F(x,y)=O(|x|^{-1-\alpha})$ for $|x|\leq 1$, uniformly in $y$. Conversely, if $F(x,y)=O(|x|^{-\beta})$ for $|x|\leq 1$, uniformly in $y$, for some $\beta$, we have $c(\mathfrak{m})=O(|\mathfrak{m}|^{-\beta})$.
\end{itemize}

Clearly (III) amounts to saying that the Dirichlet series (1) with the coefficients $c(\mathfrak{m})$ is absolutely convergent in some half-plane. This may be regarded as the Mellin transform of $\Phi$. It is more appropriate for our purposes, however, to use that name for the series 
\begin{equation}
Z(\omega)=\sum c(\mathfrak{m})\omega (\mathfrak{m}),\label{art21-eq9}
\end{equation}\pageoriginale
where $\omega$ is a quasicharacter of $k^{\times}_{A}/k^{\times}$, and $\omega(\mathfrak{m})$ is as in \eqref{art21-eq2}. For $s\in \bfC$, we write $\omega_{s}$ for the quasicharacter $\omega_{s}(z)=|z|^{s}$, and, for every quasicharacter $\omega$, we define $\sigma=\sigma(\omega)$ by $|\omega(z)|=|z|^{\sigma}$, i.e. $|\omega|=\omega_{\sigma}$ (where $|\quad |$ in the left-hand side is the ordinary absolute value $|t|=(t\overline{t})^{\frac{1}{2}}$ for $t\in \bfC$). Then our condition (III) implies that \eqref{art21-eq9} is absolutely convergent for $\sigma(\omega)>1+\alpha$. If we replace $\omega$ by $\omega\cdot \omega_{s}$ in \eqref{art21-eq9}, \eqref{art21-eq9} becomes the same as the series \eqref{art21-eq4}; in other words, the knowledge of the function $Z$ given by \eqref{art21-eq9} on the set of all the quasi-characters of $k^{\times}_{A}/k^{\times}$ is equivalent to that of all the functions given by \eqref{art21-eq4}. As before, we define $Z(\omega)$ by analytic continuation in the $s$-plane, whenever possible, when it is not absolutely convergent.

Conversely, let the coefficients $c(\mathfrak{m})$ be given for $\mathfrak{m}\in \mathfrak{M}_{+}$; assume (III), and put $c(\mathfrak{m})=0$ outside $\mathfrak{M}_{+}$. Let $Z$ be defined by \eqref{art21-eq9}; at the same time, define $f_{1}$ on $k^{\times}_{A}$ by putting $f_{1}(x)=c(\mathfrak{m})$ with $\mathfrak{m}=\mu(x)$ if $k$ is of characteristic $p>1$, and $f_{1}(x)=c(\mathfrak{m})\bfh_{\infty}(x_{\infty})$ otherwise, with $x_{\infty}=(x_{w})$; put $f_{0}(x)=0$, and define $F(x,y)$ by the Fourier series \eqref{art21-eq6}, whose convergence follows at once from (III) and the definition of $\bfh_{\infty}$ if $k$ is of characteristic $0$, and is obvious otherwise. As we have said, the question arises now whether $F$ can be extended to a continuous function $\Phi$ on $G_{A}$, left-invariant under $G_{k}$ (and invariant under $k^{\times}_{A}$); if so, we may then ask whether this is an $(h,\mathfrak{a})$-automorphic function, which clearly must then satisfy (I) and (III) and is easily shown to satisfy (II). In that case we say that $\Phi$ and the series $Z$ given by \eqref{art21-eq9} are the {\em Mellin transforms} of each other. 

We are now able to state our main results.

\begin{theorem}\label{art21-thm1}
Let $\Phi$ be an $(h,\mathfrak{a})$-automorphic function on $G_{A}$; let $\Phi'$ be the $(h,\mathfrak{a})$-automorphic function given by $\Phi'(g)=\Phi(g\bfa)$. Assume that $\Phi$ and $\Phi'$ satisfy {\rm (I), (II), (III)}. Call $Z$ the series \eqref{art21-eq9} derived from $\Phi$ as explained above, and $Z'$ the series similarly derived from $\Phi'$. Then, for all the quasicharacters $\omega$ whose conductor is disjoint from $\mathfrak{a}$, we have
\begin{equation}
Z(\omega)\Pi \mathscr{G}_{w}(\omega)=(-1)^{r-R}\kappa(\omega)^{2}\omega(af(\omega)^{2}d^{2})Z'(\omega^{-1})\Pi \mathscr{G}_{w}(\omega^{-1}).\label{art21-eq10}
\end{equation}
Moreover, if $Z$ is eulerian at any finite place $v$ of $k$, not occurring in $\mathfrak{a}$, $Z'$\pageoriginale is also eulerian there, and they have the same eulerian factor at $v$, which is of the form
\begin{equation}
(1-c|\mathfrak{p}_{v}|^{s}+|\mathfrak{p}_{v}|^{1+2s})^{-1}\label{art21-eq11}
\end{equation}
with $c=c(\mathfrak{p}_{v})$.
\end{theorem}

In \eqref{art21-eq10}, the two products are taken over the infinite places $w$ of $k$, $\mathfrak{G}_{w}$ being as in \eqref{art21-eq8}; $r$ is the number of such places; $\kappa(\omega)$ and $f(\omega)$ are as in \eqref{art21-eq3} and \eqref{art21-eq5}, and $R$ as in \eqref{art21-eq5}. Moreover, by \eqref{art21-eq10}, we mean that, if $\omega\cdot \omega_{s}$ is substituted for $\omega$, both sides can be continued analytically as holomorphic functions of $s$ in the whole $s$-plane, bounded in every strip $\sigma \leq \rRe(s)\leq \sigma'$, and that they are equal; \eqref{art21-eq10} and similar formulas should also be understood in that same sense in what follows.

It is worth noting\footnote[1]{I owe this observation to Jacquet.} that, for $Z$ to be eulerian at $v$ in Theorem \ref{art21-thm1}, it is necessary and sufficient that $\Phi$ should be an eigenfunction of the ``Hecke operator'' $T_{v}$ which maps every function $\Phi$ on $G_{A}$ onto the function $T_{v}\Phi$ given by
$$
(T_{v}\Phi)(g)=\int\limits_{\mathfrak{K}_{v}}\Phi(g\mathfrak{k}\cdot (\pi, 0))d\mathfrak{k},
$$
where $d\mathfrak{k}$ is a Haar measure in $\mathfrak{K}_{v}$, and $\pi$ is a prime element of $k_{v}$. More precisely, take $d\mathfrak{k}$ so that the measure of $\mathfrak{K}_{v}$ is $1$; then, if $T_{v}\Phi=\lambda\Phi$, one finds, by taking $g=(x,y)$ in the above formula and expressing $\Phi(x,y)$ by \eqref{art21-eq6}, that $Z$ has the eulerian factor \eqref{art21-eq11} at $v$, with $c=(1+|\mathfrak{p}_{v}|)\lambda$. We also note that here $T_{v}$ generates the Hecke algebra for $G_{v}$, so that $\Phi$ is then an eigenfunction for all the operators in that algebra.

\begin{theorem}\label{art21-thm2}
Let a series $Z$ be given by \eqref{art21-eq9}, and let $Z'$ be a similar series; assume that both satisfy {\rm(III)}. Let $\mathfrak{s}$ be a finite set of finite places of $k$, containing all the places which occur in $\mathfrak{a}$. Assume that $Z$, $Z'$ are eulerian at every finite place of $k$ outside $\mathfrak{s}$, with the same eulerian factor of the form \eqref{art21-eq11}; also, assume \eqref{art21-eq10}, in the sense explained above, for all the quasicharacters $\omega$ whose conductor is disjoint from $\mathfrak{s}$. Then there is an $(h,\mathfrak{a})$-automorphic function $\Phi$ on $G_{A}$, satisfying {\rm(I), (II), (III),}\pageoriginale such that $Z$ and $Z'$ are the Mellin transforms of $\Phi$, and of the function $\Phi'(g)=\Phi(g\bfa)$, respectively.
\end{theorem}

There is no doubt that the assumptions in Theorem \ref{art21-thm2} are much more stringent than they need be. For $k=\bfQ$, it has been found in \cite{art21-key2} that the eulerian property is not required at all; in the general case, it might perhaps be enough to postulate it at some suitable finite set of places. For $k=\bfQ$, the functional equation has to be assumed only for a rather restricted set of characters (those mentioned in \cite{art21-key2}, Satz 2), or even for a finite set of characters, depending upon $\mathfrak{a}$, when $\mathfrak{a}$ is given (since Hecke's group $\Gamma_{0}(A)$ is finitely generated). It seems quite possible that some such results may be true in general. One will also observe that, for $k=\bfQ$, Theorems \ref{art21-thm1} and \ref{art21-thm2} correspond merely to the case $\epsilon=1$ of the results obtained in \cite{art21-key2}; there is no difficulty in extending them so as to cover the case where $\epsilon$ is arbitrary; then, if they apply to two series $Z$, $Z'$, and to the conductor $\mathfrak{a}$, they also apply to any pair $Z_{1}$, $Z'_{1}$ given by $Z_{1}(\omega)=Z(\chi \omega)$, $Z'_{1}(\omega)=Z'(\chi^{-1}\omega)$, where $\chi$ is any quasicharacter whose conductor $\mathfrak{f}(\chi)$ is disjoint from $\mathfrak{a}$; the conductor for the latter pair is $\mathfrak{a}_{1}=\mathfrak{a}\mathfrak{f}(\chi)^{2}$. Leaving those topics aside, we shall not sketch briefly the proof the Theorems \ref{art21-thm1} and \ref{art21-thm2}.

Consider first the question raised by Theorem \ref{art21-thm2}. Starting from the series $Z$, we construct a function $F$ on $B_{A}$ by means of \eqref{art21-eq6} as explained above; we construct $F'$ similarly, starting from $Z'$. For these to be the restrictions to $B_{A}$ of two $(h,\mathfrak{a})$-automorphic functions $\Phi$, $\Phi'$ related to each other by $\Phi'(g)=\Phi(g\bfa)$, it is obviously necessary that one should have $F(b)=F'(b')\mathfrak{M}_{\infty}(\mathfrak{k}_{\infty})$, with $\mathfrak{M}_{\infty}=\otimes \mathfrak{M}_{w}$, whenever $b=jb'\mathfrak{k}z\bfa$ with $j=\left(\begin{smallmatrix} 0 & 1\\ -1 & 0\end{smallmatrix}\right)\in G_{k}$, $\mathfrak{k}=(\mathfrak{k}_{v})\in \mathfrak{K}$, $z\in k^{\times}_{A}$. By using the fact that $G_{k}$ is the union of $B_{k}\cdot k^{\times}$ and of $B_{k}jB_{k}\cdot k^{\times}$, one shows that this condition is sufficient. Clearly, it is not affected if one restricts $b$, $b'$ to a subset $\mathfrak{B}$ of $B_{A}$ containing a full set of representatives of the right cosets in $B_{A}$ modulo $B_{A}\cap \mathfrak{K}$. For $\mathfrak{B}$, we choose the set consisting of the elements $(xfd,xe)$ with $x\in k^{\times}_{A}$, $f=(f_{v})\in k^{\times}_{A}$, $e=(e_{v})\in k_{A}$, with $f$ and $e$ restricted as follows. For each infinite place $w$, we take $f_{w}>0$ and $f^{2}_{w}+e_{w}\overline{e}_{w}=1$.\pageoriginale For each finite place $v$, we take $f_{v}$, $e_{v}$ in $r_{v}$, with $f_{v}\neq 0$ and $\sup (|f_{v}|_{v},|e_{v}|_{v})=1$. Then we call $\mathfrak{f}=\mu(f)$ the conductor of the element $b=(xfd,xe)$ of $\mathfrak{B}$. Take two such elements $b=(xfd,xe)$, $b'=(x'f'd,x'e')$, such that $b=jb'\mathfrak{k}z\bfa$ with $\mathfrak{k}\in \mathfrak{K}$, $z\in k^{\times}_{A}$; it is easily seen that they must have the same conductor $\mathfrak{f}$, and that this is disjoint from $\mathfrak{a}$; moreover, when $x$, $f,$ $e$ are given, one may choose $x'$, $f'$, $e'$, $\mathfrak{k}$, $z$ so that $f'=f$, that $e'$, $\mathfrak{k}$, $z$ are uniquely determined in terms of $f$ and $e$, and that $x'=ax^{-1}$. Therefore the condition to be fulfilled can be written as
\begin{equation}
F(xfd,xe)=F'(ax^{-1}fd, ax^{-1}e')\mathfrak{M}_{\infty}(\mathfrak{k}_{\infty}),\label{art21-eq12}
\end{equation}
with $e'$ uniquely determined in terms of $f$, $e$, and $\mathfrak{k}_{\infty}$ in terms of $f_{\infty}$, $e_{\infty}$. Actually, one finds that it is enough, for $\Phi$ and $\Phi'$ to exist as required, that this should be so when $\mathfrak{f}$ is disjoint, not merely from $\mathfrak{a}$, but from any fixed set $\mathfrak{s}$ of places, containing $\mathfrak{a}$, provided it is finite, or at least provided its complement is ``not too small'' in a suitable sense. We must now seek to express \eqref{art21-eq12} in terms of the original series $Z$, $Z'$.

In order to do this, we multiply \eqref{art21-eq12} with an arbitrary quasi-chara\-cter $\omega$, and write {\em formally} the integrals of both sides over $k^{\times}_{A}/k^{\times}$. This, taken literally, is meaningless, since it leads to divergent integrals; leaving this aspect aside for the moment, we note first that, if we replace $F$ in the left-hand side by the Fourier series which defines it, that side may be formally rewritten as 
\begin{equation}
\sum c(\mathfrak{m})\int \bfh_{\infty}(x_{\infty})\psi (xef^{-1}d^{-1})\omega(xf^{-1}d^{-1})d^{\times}x,\label{art21-eq13}
\end{equation}
where $d^{\times}x$ is the Haar measure in $k^{\times}_{A}$, and the integral, in the term corresponding to $\mathfrak{m}$, is taken over the subset of $k^{\times}_{A}$ determined by $\mu(x)=\mathfrak{m}$; this is a coset of the kernel of $\mu$, i.e. of the open subgroup $k^{\times}_{\infty}\times \Pi r^{\times}_{v}$ of $k^{\times}_{A}$. These integrals are easily calculated (by means of Proposition 14, Chapter VII-7, of BNT, page 132) in terms of the product $J=\Pi I_{w}((1,e_{w}),\omega_{w})$, where the $I_{w}$ are as defined in \eqref{art21-eq7}; they converge for $\sigma(\omega)$ large enough. One sees at once that they are $0$ for all $\mathfrak{m}$ unless the conductor $\mathfrak{f}(\omega)$ of $\omega$ divides $\mathfrak{f}=\mu(f)$. If $\mathfrak{f}(\omega)=\mathfrak{f}$, one finds that \eqref{art21-eq13} is no other than $J\cdot Z(\omega)$, up to a simple scalar factor. A similar formal calculation for the right-hand side\pageoriginale of \eqref{art21-eq12} transforms it into the product of a scalar factor, of an integral similar to $J$, and of $Z'(\omega^{-1})$; comparing both sides and taking \eqref{art21-eq8} into account, one gets the functional equation \eqref{art21-eq10}, for which we will now write $E(\omega)$. If we do not assume $\mathfrak{f}(\omega)=\mathfrak{f}$, but merely $\mathfrak{f}=\mathfrak{f}(\omega)\mathfrak{f}_{1}$ with $\mathfrak{f}_{1}\in \mathfrak{M}_{+}$, the same procedure leads to a similar equation $E_{1}(\omega)$ connecting two Dirichlet series $Z_{1}(\omega)$, $Z'_{1}(\omega)$ whose coefficients depend only upon $\mathfrak{f}_{1}$ and the coefficients of $Z$ and of $Z'$, respectively.

If $k$ is of characteristic $p>1$, there is no difficulty in replacing the above formal argument by a correct proof. The same can be achieved for characteristic $0$ by a straightforward application of Hecke's lemma (c.f. \cite{art21-key2}, page 149). The conclusion in both cases is that the validity of the equations $E_{1}(\omega)$ for all divisors $\mathfrak{f}_{1}$ of $\mathfrak{f}$ and all quasicharacters $\omega$ with the conductor $\mathfrak{f}\mathfrak{f}^{-1}_{1}$ is necessary and sufficient for \eqref{art21-eq12} to hold for all $e$ and all $x$, when $\mathfrak{f}$ is given. This proves Theorem \ref{art21-thm1} except for the last part, which one obtains easily by comparing the equations $E(\omega)$ and $E_{1}(\omega)$ for $\mathfrak{f}_{1}=\mathfrak{p}_{v}$ when the eulerian property is postulated for $Z$ at $v$. On the other hand, we see now, in view of what was said above, that, when $Z$ and $Z'$ are given, $\Phi$ and $\Phi'$ exist as required provided the functional equations $E(\omega)$, $E_{1}(\omega)$ are satisfied whenever $\mathfrak{f}(\omega)$ and $\mathfrak{f}_{1}$ are both disjoint from the given set $\mathfrak{s}$. If one assumes that $Z$, $Z'$ are eulerian at each one of the places occurring in $\mathfrak{f}_{1}$, with an eulerian factor of the form \eqref{art21-eq11}, one finds that $Z_{1}(\omega)$, $Z'_{1}(\omega)$ differ from $Z(\omega)$, $Z'(\omega)$ only by an ``elementary'' factor and that $E_{1}(\omega)$ is a consequence of $E(\omega)$. This proves Theorem \ref{art21-thm2}.

Examples for Dirichlet series satisfying the conditions in Theorem \ref{art21-thm2} are given, as we have seen, by the zeta-functions of elliptic curves (taking $Z(\omega)=L_{\omega}(1)$, $Z'(\omega)=\pm L_{\omega}(1)$, where $L(s)$ is the zeta-function) in the cases (a), (b), (c) where these can be effectively computed; other similar examples, not arising from elliptic curves, can easily be constructed, as Hecke $L$-functions over quadratic extensions of $k$, or products of two such functions over $k$. Jacquet has pointed out that, when $Z$ is a product of two Hecke $L$-functions, the automorphic function $\Phi$ is an Eisenstein series; this is the case in\pageoriginale example (c), and in (a) when $k$ contains the complex multiplications of the curve $E$; it cannot happen (according to \cite{art21-key2}, Satz 2) when $k=\bfQ$.

If $k$ is a number-field with $r$ infinite places, and if the zeta-function of an elliptic curve $E$ over $k$ satisfies the assumptions in Theorem \ref{art21-thm2}, that theorem associates with it the differential form $\Phi\cdot \beta_{\infty}$ of degree $r$; since it is locally constant with respect to the coordinates at the finite places, it may be regarded as a harmonic differential form of degree $r$ on the union of a certain finite number of copies (depending on the class-number of $k$) of the Riemannian symmetric space $H_{\infty}$ belonging to $G_{\infty}$. For $k=\bfQ$, some examples suggest that the periods of that form may be no other than those of the differential form of the first kind belonging to $E$. In the general case, one can at least hope to discover a relation between the periods of $\Phi\cdot \beta_{\infty}$ and those of the differential form of the first kind on $E$ and on its conjugates over $\bfQ$. When $k$ is of characteristic $p>1$, however, $\Phi$ is a scalar complex-valued function on the discrete space $G_{k}\backslash G_{A}/\mathfrak{K}k^{\times}_{A}$, and it seems hard even to imagine a connexion between this and the curve $E$, closer than the one given by the definitoin of $\Phi$ in terms of $Z$.

\begin{thebibliography}{99}
\bibitem{art21-key1} \textsc{H. Maass :} Automorphe Funktionen von mehreren Ver\"anderlichen und Dirichletsche Reihen, {\em Hamb. Abh. Bd.} 16, Heft 3-4 (1949), 72-100.

\bibitem{art21-key2} \textsc{A. Weil :} \"Uber die Bestimmung Dirichletscher Reihen durch Funktionalgleichungen, {\em Math. Ann.} 168 (1967), 149-156.

\bibitem{art21-key3} \textsc{A. Weil :} {\em Basic Number Theory} ({\em Grundl. Math. Wiss. Bd.} 144), Springer, Berlin-Heidelberg-New York, 1967.

\end{thebibliography}

\bigskip

\noindent
{\small The Institute for Advanced Study,}

\noindent
{\small Princeton, N. J., U.S.A.}


