\title{SOME QUESTIONS ON RATIONAL ACTIONS OF GROUPS}
\markright{Some Questions on Rational Actions of Groups}

\author{By~~ Masayoshi Nagata}
\date{}

\maketitle

\setcounter{pageoriginal}{322}
\textsc{The}\pageoriginale contents are divided into two parts. In Part \ref{art16-part-I}, we discuss the rings of invariants of a finite group in a noetherian ring. In Part \ref{art16-part-II}, we raise some questions on rational actions of groups, mostly connected algebraic groups. By a ring, we mean a commutative ring with identity. By a subring, we mean a subring having common identity.

\bigskip
\begin{center}
{\bf\Large Part I.}\labeltext{I}{art16-part-I}
\end{center}

\setcounter{section}{-1}
\section{} We discuss here the following question.

\setcounter{proposition}{-1}
\begin{question}\label{art16-ques0.0}
Let $R$ be a noetherian ring and let $G$ be a finite group acting on $R$. Let $A$ be the ring of invariants in $R$. Is $A$ noetherian ?
\end{question}

Unfortunately, the answer is not affirmative in general as well be shown later by counter-examples. Since the examples which we have non-normal, we raise a question.

\begin{question}\label{art16-ques0.1}
Assume, in Question \ref{art16-ques0.0}, $R$ is a direct sum of normal rings. Is then $A$ noetherian ?
\end{question}

We shall begin with some simple cases. We maintain the meanings of $R$, $G$, $A$ of Question \ref{art16-ques0.0}.

\begin{proposition}\label{art16-prop0.2}
If the order $g$ of $G$ is not divisible by the characteristic of any residue class field of $R$, in other words, if $g$ is a unit in $R$, then $A$ is noetherian.
\end{proposition}

\begin{proof}
If $h_{1},\ldots,h_{s}\in A$ and if $f\in (\Sigma h_{i}R)\cap A$, then $f=\Sigma h_{i}r_{i}(r_{i}\in R)$. Then $gf=\sum\limits_{\sigma\in G}\sigma f=\sum\limits_{i}\sum\limits_{\sigma\in G}h_{i}(\sigma r_{i})$, and $f=\sum\limits_{i}h_{i}(g^{-1}\sum\limits_{\sigma\in G}\sigma r_{i})\in \Sigma h_{i}A$. Thus we have $(\Sigma h_{i}R)\cap A=\Sigma h_{i}A$. From this the assertion follows easily.
\end{proof}

\begin{proposition}\label{art16-prop0.3}
If\pageoriginale $R$ is a Dedekind domain, then $A$ is also a Dedekind domain, and $R$ is a finite $A$-module.
\end{proposition}

The proof is obvious in view of the following well known lemma (see for instance \cite{art16-key-L}).

\begin{lemma}\label{art16-lem0.4}
Let $A'$ be a normal ring and let $k'$ be an integral extension of $A'$ in an algebraic extension $L$ of the field of quotients $K$ of $A'$. Assume that $a\in R'$ generates $L$ over $K$. Let $f(x)$ be the irreducible monic polynomial for $a$ over $A'$. Then letting $d$ be one of {\rm(i)} discriminant of $f(x)$ and {\rm(ii)} $df(a)/dx$, we have $dR'\subseteq A'[a]$.
\end{lemma}

Another easy case is:

\begin{remark}\label{art16-rem0.5}
If $R$ is a ring of quotients of a finitely generated ring $R_{0}$ over a subring $F$ of $A$ and if $F$ is pseudo-geometric, then $A$ is a ring of quotients of a finitely generated ring $A_{0}$ over $F$, hence $A$ is noetherian.
\end{remark}

As our example below (see the proof of Proposition \ref{art16-prop0.11}) shows, Question \ref{art16-ques0.0} is not affirmative even if we assume that $R$ is a pseudo-geometric local integral domain of Krull dimension $1$, whose derived normal ring is a valuation ring: this fact shows that:

\begin{remark}\label{art16-rem0.6}
Assume that a subring $S$ of $R$ is $G$-stable and that $B$ is the ring of $G$-invariants in $S$. Even if $R$ is a discrete valuation ring of the field of quotients of $S$ and is a finite $S$-module, $A$ may not be a finite $B$-module.
\end{remark}

On the other hand, one can show:

\begin{remark}\label{art16-rem0.7}
If, for a subring $S$ of a noetherian ring $R$, $R$ is a finite $S$-module, then $S$ is noetherian. (Proof of this remark will be published somewhere else.)
\end{remark}

Therefore the writer believes it is an important question to ask for reasonable sufficient conditions for $R$ to be a finite $A$-module.

Now we are going to construct counter-examples to the question.

\begin{proposition}\label{art16-prop0.8}
Let $F$ be a field of characteristic $p\neq 0$ and let $x_{1}$, $x_{2},\ldots$ infinitely many indeterminates. Consider the derivation $D=\sum\limits_{i=1}x^{s_{i}}_{i}\dfrac{\partial}{\partial x_{i}}$\pageoriginale of the field $K=F(x_{1},\ldots,x_{n},\ldots)$, such that {\rm(i)} each $s_{i}$ is a non-negative integer $\equiv 0$, $1$ modulo $p$, either $0$ or greater than $p-1$ and {\rm(ii)} infinite number of $s_{i}$ are $\equiv 1$ modulo $p$. Let $C$ be the field of constants with respect to $D$. Then $[K:C]=\infty$.
\end{proposition}

\begin{proof}
For simplicity, we assume that $s_{i}=p+1$ for $i=r$, $r+1,\ldots$. We show that $x_{r},x_{r+1},\ldots$ are linearly independent over $C$. For, if $\sum\limits_{i\geq r}x_{i}c_{i}=0 (c_{i}\in C)$, then by the operation of $D$, we have $\Sigma x^{s_{i}}_{i}c_{i}=0$ which can be written $\Sigma x_{i}(x^{p}_{i}c_{i})=0$. Since $x^{p}_{i}c_{i}\in C$, we have got another linear relation, and we get a contradiction.
\end{proof}

\begin{proposition}\label{art16-prop0.9}
Let $K$ be a field of characteristic $p\neq 0$. Let $y$ be an element defined by $y^{2}=0$. Consider the ring $R=K[y]=K+yK$. Let $D$ be a derivation of $K$. Then the map $\sigma:f+yg\to f+yg+y \ Df$ gives an automorphism of $R$ and $\sigma^{p}=1$.
\end{proposition}

Proof is easy and we omit it.

Now we have

\begin{proposition}\label{art16-prop0.10}
In the question stated at the beginning, even if $R$ is an artinian ring, $A$ can be non-noetherian.
\end{proposition}

\begin{proof}
Let $K$, $C$ and $D$ be as in Proposition \ref{art16-prop0.8} and then let $y$, $\sigma$ be as in Proposition \ref{art16-prop0.9}. For $G=\{1,\sigma,\ldots,\sigma^{p-1}\}$, $A=\{f+yg | Df=0\}=C+yK$. Since $[K:C]=\infty$, $A$ is not noetherian.
\end{proof}

\begin{proposition}\label{art16-prop0.11}
In the question, even if $R$ is assumed to be a pseudo-geometric local integral domain of Krull dimension $1$, $A$ can be non-noetherian.
\end{proposition}

\begin{proof}
Let $F$ be a field of characteristic $p\neq 0$ and let $y$, $z_{1}$, $z_{2},\ldots$ be infinitely many indeterminates. Set $K^{*}=F(z_{1},z_{2},\ldots)$ and $V=K^{*}[y]_{(y)}$. Then $V$ is a discrete valuation ring and has an automorphism $\sigma$ such that $\sigma z_{1}=z_{1}+y$ and $\sigma$ fixes every element of $F[z_{2},z_{3},\ldots,y]\cdot \sigma^{p}=1$. We set
$$
x_{1}=z^{2p}_{1},x_{i}=z_{i}+y^{p}z^{p+1}_{i}z^{p}_{i}(i\geq 2), w_{1}=y^{2p},w_{i}=y^{2p}z_{i}(i\geq 2).
$$\pageoriginale
Then $\sigma z_{i}=z_{i}+w_{i}$ and $\sigma w_{i}=w_{i}$. Thus $G=\{1,\sigma,\ldots,\sigma^{p-1}\}$ acts on the ring $R'=F[x_{1},x_{2},\ldots,w_{1},w_{2},\ldots]$. Set $R=R'_{y V\cap R'}$. Then $G$ acts on $R$. The ring of invariants $A$ is of the form $A'_{yV\cap A'}$ with $A'=A\cap R'$. We observe elements of $A'$. It is of the form $f(x)+\Sigma w_{i}t_{i}(x)+$ (terms of higher degree in $w$). Invariance implies that $\Sigma w_{i}\dfrac{\partial f(x)}{\partial x_{i}}\equiv 0(\mod y^{2p+1}V)$. This implies that, denoting by $D$ the derivation $\dfrac{\partial}{\partial x_{1}}+\sum\limits_{i\geq 2} x^{p+1}_{i}\dfrac{\partial}{\partial x_{i}}$ of $K=F(x_{1},x_{2},\ldots)$, $Df(x)=0$. Therefore $A'/yV\cap A'$ is contained in the field of constants with respect to this $D$. Thus, Proposition \ref{art16-prop0.8} shows that $[R/y V\cap R: A/yV\cap A]=\infty$ and that the sequence of ideals $(y^{2p+1}V\cap A)+\sum\limits^{2}_{i=2}w_{i}A(n=2,3,\ldots)$ gives an infinite ascending chain of ideals. Thus $A$ is not noetherian. That $R$ is a pseudogeometric local integral domain of Krull dimension $1$ follows from the fact that $R\supset F(x_{1},x_{2},\ldots)[y^{2p}]$.
\end{proof}

\begin{remark}\label{art16-rem0.12}
The examples above can be modified to be examples in case of unequal characteristics. In the first example, $R$ is such that (i) characteristic is $p^{2}$, (ii) $R/PR=K$. In the latter example, let $y$ be $p^{1/2p}$.
\end{remark}

At the end of this Part \ref{art16-part-I}, we raise the following question in view of our construction of these counter-examples.

\begin{question}\label{art16-ques0.13}
Let $R$ be a noetherian ring and let $S$ be a subring such that $R$ is integral over $S$. Assume that for every prime ideal $P$ of $E$, the ring $R/P$ is an almost finite integral extension of $S/(P\cap S)$ and that there is a non-zero-divisor $d\in S$ such that $dR\subseteq S$. Is $S$ noetherian ?
\end{question}

We note that the following fact can be proved easily.

\begin{remark}\label{art16-rem0.14}
Question \ref{art16-ques0.13} is affirmative if $R$ is either an artinian ring or an integral domain of Krull dimension $1$, even if we do not assume the existence of $d$. Without assuming the existence of $d$, one can have a counter-example in case $R$ is a normal local domain of Krull dimension $2$. (In \cite{art16-key-L}, there is an example of a local\pageoriginale domain, say $B$, of Krull dimension $2$ such that there is a non-noetherian ring $S$ between $B$ and its derived normal ring $R$. These $S$ and $R$ give a counter-example.)
\end{remark}


\bigskip
\begin{center}
{\bf\Large Part II.}\labeltext{I}{art16-part-II}
\end{center}

\section{}\label{art16-sec1}
Let $G$ be a group acting on a function field $K$ over an algebraically closed ground field $k$.\footnote{Though many of our discussions can be adapted to the case where $k$ is a ground ring, we assume that $k$ is an algebraically closed field for the sake of simplicity.} We say that the action is rational if there is a pair of an algebraic group $G^{*}$ and a model $V$ of $K$, both defined over $k$, such that (i) $G$ is a subgroup of $G^{*}$ and (ii) the action of $G$ is induced by a rational action of $G^{*}$ on $V$. Thus we are practically thinking of rational actions of algebraic groups.

At first, we discuss the choice of $V$. Namely, we fix a group $G$, which may be assumed to be algebraic, and a function field $K$ over an algebraically closed field $k$ such that $G$ is acting rationally on $K$. Then there may be many models $V$ of $K$ which satisfy the requirement in the above definition.

\begin{proposition}\label{art16-prop1.1}
When a $V$ satisfies the requirement, then so does the derived normal model of $V$.
\end{proposition}

The proof is easy.

\begin{proposition}\label{art16-prop1.2}
If a quasi-affine variety $V$ satisfies the requirement, then there is an affine model $V'$ of $K$ which satisfies the requirement.
\end{proposition}

\begin{proof}
Let $R$ be the ring of elements of $K$ which are everywhere regular on $V$. Then the rationality of the action of $G$ on $V$ implies that $\sum\limits_{\sigma \in G}(\sigma f)k$ is a finite $k$-module for every $f\in R([F])$. Let $f_{1},\ldots,f_{n}$ be elements of $R$ such that $K=k(f_{1},\ldots,f_{n})$ and let $g_{1},\ldots,g_{s}$ be a linearly independent base of $\sum\limits^{n}_{i=1}\sum\limits_{\sigma \in G}(\sigma f_{i})k$. Then the affine model $V'$ defined by $k[g_{1},\ldots,g_{s}]$ is the desired variety.
\end{proof}

\begin{proposition}\label{art16-prop1.3}
If\pageoriginale an affine variety $V$ satisfies the requirement, then there is a projective model $V'$ of $K$ which satisfies the requirement.
\end{proposition}

\begin{proof}
As is seen by the proof above, we may assume that the affine ring $R$ of $V$ is such that $R=k[g_{1},\ldots,g_{s}]$, where $\sum\limits^{s}_{i=1}g_{i}k$ is a representation module of $G$. Then the projective variety $V'$ with generic point $(1,g_{1},\ldots,g_{s})$ is the desired variety.
\end{proof}

\begin{remark}\label{art16-rem1.4}
In the case above, the action of $G$ is practically that of a linear group.
\end{remark}

It was proved by Kambayashi (\cite{art16-key-K}) that

\begin{proposition}\label{art16-prop1.5}
If $G$ is a linear group and $V$ is a complete variety, then there is a projective model $V'$ of $K$ which satisfies the requirement (and such that every element of $G$ defines a linear transformation on $V'$).
\end{proposition}

These results suggest to us the following question.

\begin{question}\label{art16-ques1.6}
Does the rationality of the action of $G$ imply the existence of a projective model of $K$ which satisfies the requirement? How good can the singularity of such a model be?
\end{question}

In connection with this question, we raise

\begin{question}\label{art16-ques1.7}
Let $G$ be a connected linear group acting rationally on a normal abstract variety $V$. Let $L$ be a linear system on $V$. Does it follows that there is a linear system $L^{*}$ on $V$ which contains all $\sigma L(\sigma\in G)$?
\end{question}

If this question has an affirmative answer, then at least for linear groups, Question \ref{art16-ques1.6} has an affirmative answer. Note that Question \ref{art16-ques1.7} is affirmative if $V$ is complete (\cite{art16-key-K}).

On the other hand, if there is a quasi-affine variety $V$ which satisfies the requirement, then for every model $V'$ of $K$ satisfying the requirement, it is true that the orbit of a generic point of $V'$ is quasi-affine. Thus, even if $G$ is a connected linear algebraic group, if, for instance, the isotropy group (=stabilizer) of a generic point contains\pageoriginale a Borel subgroup of $G$, then there cannot be any quasi-affine model of $K$ satisfying the requirement (unless the action of $G$ is trivial). Therefore we raise

\begin{question}\label{art16-ques1.8}
Assume that the orbit of a generic point of a $V$ is quasi-affine. Does this imply that there is an affine model of $K$ which satisfies the requirement ?
\end{question}

\section{}\label{art16-sec2}
We observe the subgroup generated by two algebraic groups acting on a function field. More precisely, let $G$ and $H$ be subgroups of the automorphism group $\Aut_{k}K$ of the function field $K$ over the group field $k$. We shall show by an example that

\begin{proposition}\label{art16-prop2.1}
Even if $G$ and $H$ are linear algebraic groups, which are isomorphic to the additive group $G_{a}$ of $k$ and acting rationally on $K$, the subgroup $G\vee H$ generated by $G$ and $H$ (in $\Aut_{k}K$) may not have any rational action on $K$.
\end{proposition}

\begin{example*}
Let $a$ and $b$ be non-zero elements of $k$ and let $K_{0}$, $x$, $y$ be such that $ax^{2}+by^{2}=1$, $K_{0}=k(x,y)$ and trans. $\deg_{k}K_{0}=1$. We assume here that $k$ is not of characteristic $2$. Let $(z,w)$ be a copy of $(x,y)$ over $k$ and let $K=k(x,y,z,w)$ = (quotient field of $\foprod{k(x,y)}{k(z,w)}{k}$). Set $t=(y-w)/(x-z)$. Then $K=k(x,y,t)=k(z,w,t)$. We note the relation :
$$
\binom{x}{y}=F_{t}\binom{z}{w}
$$
with
$$
F_{t}=\frac{1}{bt^{2}+a}\left(\begin{matrix} bt^{2}-a & -2bt\\ -2at & a-bt^{2}\end{matrix}\right).
$$
$\Aut_{k}K$ contains the following subgroups $G$ and $H$ :
\begin{align*}
G &= \{\sigma_{c}\in \Aut_{k(x,y)}K | c\in k, \sigma_{c}t=t+c\}\cong G_{a},\\[3pt]
H &= \{\tau_{c}\in \Aut_{k(z,w)}K | c\in K, \tau_{c}t=t+c\}\cong G_{a}.
\end{align*}
$\Aut_{k}K$ has an element $\rho$ such that $\rho^{2}=1$, $\rho x=z$, $y=w$. Then $H=\rho^{-1}G_{\rho}$. $G$ acts rationally on the affine model of $K$ defined by $k[x,y,t]$ and $H$ acts rationally on the affine model of $K$ defined by $k[z,w,t]$. Thus $G$ and $H$ act rationally on $K$. For $c_{i}\in k$, we observe the element $\tau_{c_{1}}\rho \tau_{c_{2}}\rho\ldots\rho \tau_{c}\rho_{s}$; let us denote this element by $[c_{1},\ldots,c_{s}]$. Then\pageoriginale $[c]\binom{z}{w}=\tau_{c}\binom{x}{y}=F_{t+c}\binom{z}{w}$ Note that if $\alpha\binom{z}{w}=F^{*}_{t}\binom{z}{w}$ for $\alpha \in \Aut K$ and with $F^{*}_{t}\in GL(2,k(t))$, then
$$
([c]\alpha)\binom{z}{w}=F^{*}_{t+c}F_{t}\binom{z}{w}
$$
Thus, to each $[c_{1},\ldots,c_{s}]$ there corresponds a matrix in $GL(2,k(t))$. In view of this correspondence, one can see easily that the dimension of the algebraic thick set $(\rho G)^{n}=\rho G\rho G\ldots \rho G$ tends to infinity with $n$.
\end{example*}

\begin{remark}\label{art16-rem2.2}
Similar example is given so that $G$ and $H$ are isomorphic to the multiplicative group of $k$, by changing $\sigma_{c}t=t+c$ and $\tau_{c}t=t+c$ to $\sigma_{c}t=ct$ and $\tau_{c}t=ct$ respectively.
\end{remark}

Now we raise

\begin{question}\label{art16-ques2.3}
Give good conditions for connected algebraic subgroups $G$ and $H$ of $\Aut_{k}K$ so that $G\vee H$ is algebraic.
\end{question}

\section{}\label{art16-sec3}
Let $G$ be an algebraic group acting on a variety $V$. Then there may be fixed points of $G$ on $V$. In particular, if $G$ is linear and if $V$ is complete, then there is at least one fixed point (\cite{art16-key-B}). The following fact was noticed by Dr. John Forgarty.

\begin{proposition}\label{art16-prop3.1}
If $G$ is a connected unipotent linear group and if $V$ is a projective variety, then the set $F$ of fixed points on $V$ is connected. More generally, if $W$ is a connected closed set in a projective variety and if $G$ is a connected unipotent linear group which acts rationally on $W$, then the set $F$ of fixed points of $G$ on $W$ is connected.
\end{proposition}

\begin{proof}
We shall prove the last statement by induction on $\dim G$. Then we may assume that $\dim G=1$, i.e. $G$ is isomorphic to the additive group of $k$. Thus, in the following until we finish the proof of the proposition, we assume that $G$ is the additive group of $k$ and that varieties and curves are projective ones.
\end{proof}

\begin{lemma}\label{art16-lem3.2}
Under the circumstances, let $C$ be an irreducible curve on which $G$ acts rationally. If there are two fixed (mutually different) points on $C$, then every point of $C$ is a fixed point.
\end{lemma}

\begin{proof}
$G$\pageoriginale can be imbedded in a projective line $L$ biregularly. Then $L$ consists of $G$ and a point. If $P\in C$ is not fixed, then $C-GP$ is a point, which is not the case.
\end{proof}

\begin{corollary}\label{art16-coro3.3}
Under the circumstances, let $C$ be a connected reducible curve on which $G$ acts rationally. Let $C'$ be an irreducible component of $C$. If either there are two points (mutually different) on $C'$ which are on some other components of $C$ or there is a fixed point $P$ on $C'$ which is not on any other irreducible component of $C$, then every point of $C'$ is fixed.
\end{corollary}

The proof is easy because (i) since $G$ is connected, every component of $C$ is $G$-stable and therefore (ii) every point which is common to some mutually different irreducible components of $C$ is a fixed point.

\begin{corollary}\label{art16-coro3.4}
Under the circumstances, let $C$ be a connected curve on which $G$ acts rationally. Then the set $F_{0}$ of fixed points on $C$ is connected.
\end{corollary}

\begin{proof}
If $C$ is irreducible, then either $F_{0}$ consists of a point or $F_{0}=C$, and the assertion holds good in this case. We assume now that $C$ is reducible. If $P\in C$ is not fixed, then let $C'$ be the irreducible component of $C$ which carries $P$. $C'$ carries only one fixed point, say $Q$. Then every component of $C$, which has a common point with $C'$, goes through $Q$. Therefore $C-GP$ is a connected curve, whose set of fixed points is $F_{0}$. Thus we finish the proof by induction on the number of irreducible components of $C$.

Now we go back to the proof of Proposition \ref{art16-prop3.1}. Let $W_{i}(i=1,\ldots,n)$ be the irreducible components of $W$. Since $G$ is connected solvable, $W_{i}\cap W_{j}$ contains a fixed point $P_{ij}$, unless $W_{i}\cap W_{j}$ is empty. If one knows that every $F\cap W_{i}$ is connected, then the existence of $P_{ij}$ shows the connectedness of $F$. Thus we may assume that $W$ is irreducible. Let $P^{*}$ be generic point of $W$ and let $P$ be a point of $F$. If $P^{*}$ is fixed, then every point of $W$ is fixed, and our assertion is obvious in this case. Therefore we assume that $P^{*}$ is not a fixed point. Let $\overline{C}$ be the closure of $GP^{*}$. Then $\overline{C}-GP^{*}$ consists of a point, say $Q^{*}$. Consider a specialization of $(\overline{C},Q^{*})$ with reference to $P^{*}\to P$:\pageoriginale let $(\overline{C},Q^{*},P^{*})\to (C,Q,P)$ be such a specialization. The locus $D$ of $Q^{*}$ (i.e. the subvariety of $W$ having $Q^{*}$ as its generic point) consists only of fixed points. $Q$ lies on $D\cap C$. By the connectedness theorem, $C$ is connected, whence $F\cap C$ is connected by Corollary \ref{art16-coro3.4}. Thus $F$ contains a connected subset containing $P$ and $Q^{*}$. Since this is true for every $P\in F$, we complete the proof.
\end{proof}

On the other hand, it is obvious that

\begin{proposition}\label{art16-prop3.5}
If $G$ is a connected linear algebraic group whose radical is unipotent, acting on a projective space rationally as a group of linear transformations, then the set of fixed points forms a linear subvariety.
\end{proposition}

Now our question is

\begin{question}\label{art16-ques3.6}
Find a good theorem including Proposition \ref{art16-prop3.1} and \ref{art16-prop3.5}.
\end{question}

In connection with this question, we give an example.

\begin{example}\label{art16-exam3.7}
There is a pair of a semi-simple linear algebraic group $G$ and a connected closed set $V$ in a projective space $\bfP$ such that (i) $G$ acts rationally on $\bfP$ as a group of linear transformations, (ii) $GV=V$, i.e. $V$ is $G$-stable and (iii) the set $F$ of fixed points on $V$ is not connected.
\end{example}

{\em The construction of the example.} Let $n$ be an arbitrary natural number and let $G=GL(n+1,k)$. Each $\sigma\in G$ defines a linear transformation on the projective space $\bfP$ of dimension $n+2$ defined by the matrix
$$
\begin{bmatrix}
1 & 0 & 0\\
0 & 1 & 0\\
0 & 0 & \sigma
\end{bmatrix}.
$$  

A point $(a_{0},\ldots,a_{n+2})$ is a fixed point if and only if $a_{2}=\ldots=a_{n+2}=0$. Let $V$ be the algebraic set defined by $X_{0}X_{1}=0$. $V$ is a connected and $GV=V$. But $V$ has only two fixed points $(1,0,\ldots,0)$ and $(0,1,0,\ldots,0)$.

\medskip

\section{}\label{art16-sec4}
We assume here that $G$ is a connected linear group acting rationally on a projective variety $V$. Let $P^{*}$ be a generic point of $V$ and\pageoriginale let $D^{*}$ be the closure of $GP^{*}$. Then we can think of the Chow point $Q^{*}$ of $D^{*}$. We raise a question.

\begin{question}\label{art16-ques4.1}
Is the function field $K$ of $V$ purely transcendental over $k(Q^{*})$? In other words, is $D^{*}$ rational (in the strong sense over $k(Q^{*})$?
\end{question}

Since $G$ is linear, it is obvious that $K$ is uni-rational over $k(Q^{*})$.

\begin{proposition}\label{art16-prop4.2}
If $G$ is the additive group of $k$, then the answer is affirmative.
\end{proposition}

\begin{proof}
The assertion is obvious if $P^{*}$ is a fixed point. In the other case, $D^{*}$ has a unique fixed point, which must be rational over $k(Q^{*})$. Therefore $D^{*}$ must be rational over $k(Q^{*})$.
\end{proof}

\section{}\label{art16-sec5}
In this last section, we add some questions related to the Mumford Conjecture. As was proved by Dr. Seshadri, the Mumford Conjecture on the rational representation of linear algebraic groups is true for $SL(2,k)$. Let us call a linear algebraic group ``semi-reductive'' if the statement of the Mumford Conjecture holds good for the group. Then the following three propositions are well known.

\begin{proposition}\label{art16-prop5.1}
If a linear algebraic group $G$ is semi-reductive, then {\rm(i)} so is every normal subgroup of $G$ and every homomorphic image of $G$ (by a rational homomorphism) and {\rm(ii)} the radical of $G$ is a torus group. Conversely, when $N$ is a normal subgroup of a linear algebraic group $G$, if both $N$ and $G/N$ are semi-reductive, then $G$ is semi-reductive.
\end{proposition}

\begin{proposition}\label{art16-prop5.2}
Finite groups and torus groups are semi-reductive.
\end{proposition}

\begin{proposition}\label{art16-prop5.3}
If $k$ is of characteristic zero, then a linear algebraic group $G$ is semi-reductive if and only if its radical is a torus group.
\end{proposition}

The Mumford Conjecture itself is a hard question. The writer feels that if the following two questions have affirmative answers, then it may help our observation on the conjecture.

\begin{question}\label{art16-ques5.4}
Let\pageoriginale $G$ be a connected, semi-simple semi-reductive linear algebraic group. Then its connected algebraic subgroup $H$ is semi-reductive if the following conditions are satisfied :
\begin{quote}
(i) $H$ is semi-simple and\quad (ii) $G/H$ is affine.
\end{quote}
\end{question}

\begin{question}\label{art16-ques5.5}
Let $G$ be a connected semi-simple algebraic linear group such that every proper closed normal subgroup is finite. Then there is a pair of a natural number $n$ and a linear algebraic group $G^{*}$ such that (i) $G$ and $G^{*}$ have finite normal subgroups $N$ and $N^{*}$ such that $G/N=G^{*}/N^{*}$ and (ii) $G^{*}$ is a subgroup of $GL(n,k)$ and (iii) $GL(n,k)/G^{*}$ is affine.
\end{question}

Note that (1) if the Mumford Conjecture has an affirmative answer, then these two questions have affirmative answers and (2) if these questions have affirmative answers, then we have only to prove the Mumford Conjecture for $SL(n,k)$ for each natural number $n$.

{\em Added in Proof:} Question \ref{art16-ques0.1} has been answered negatively by K. R. Nagarajan, Groups acting on noetherian rings, {\em Nieuw Archief voor Wiskunde} (3) XIV (1968), 25-29. (Though his proof contains an error, the example is good.)

\bigskip

\begin{thebibliography}{99}
\bibitem[B]{art16-key-B} \textsc{A. Borel :} Groupes lin\'earies alg\'ebriques, {\em Ann. of Math.} 64 (1956), 20-82,

\bibitem[F]{art16-key-F} \textsc{A. Weil :} {\em Foundations of algebraic geometry}, Amer. Math. Soc. Coll. Publ. (1946).

\bibitem[K]{art16-key-K} \textsc{T. Kambayashi :} Projective representation of algebraic linear groups of transformations, {\em Amer. J. Math.} 88 (1966), 199-205.

\bibitem[L]{art16-key-L} \textsc{M. Nagata :} {\em Local rings,} John Wiley (1962).
\end{thebibliography}

\bigskip

\noindent
{\small Kyoto University}

\noindent
{\small Kyoto, Japan.}



