\title{VECTOR BUNDLES ON CURVES}
\markright{Vector Bundles on Curves}

\author{By~~ M. S. Narasimhan and S. Ramanan\footnote{Presented by M. S. Narasimhan.}}
\date{}

\maketitle


\setcounter{pageoriginal}{334}
\section{Introduction}\label{art17-sec1}
\pageoriginale

We shall review `in this paper some aspects of the theory of vector bundles on algebraic curves with particular reference to the explicit determination of the moduli varieties of vector bundles of rank 2 on a curve of genus 2 (see \cite{art17-key3}). Later we prove, using these results, the non-existence of (algebraic) Poincar\'e families parametrised by non-empty Zariski open subsets of the moduli space of vector bundles of rank 2 and degree 0 on a curve of genus 2 [Theorem, \S\ref{art17-sec3}]. This result is of interest in view of the following facts :
\begin{itemize}
\item[(i)] there do exist such families when the rank and degree are coprime;

\item[(ii)] in general (i.e. even if the degree and rank are not coprime) every stable point has a neighbourhood in the usual topology parametrising a holomorphic Poincar\'e family of vector bundles;

\item[(iii)] there exists always a Poincar\'e family of {\em projective} bundles para\-metrised by the open set of stable bundles.
\end{itemize}

The essential point in the proof of the non-existence of Poincar\'e families is to show that a certain projective bundle, which arises geometrically in the theory of quadratic complexes, does not come from a vector bundle. The reduction to the geometric problem is found in \S\ref{art17-sec7}. The geometric problem, which is independent of the theory of vector bundles, is explained in \S\ref{art17-sec5} and the solution is found in \S\ref{art17-sec8}.

The idea of reducing this question to the geometric problem arose in our discussions with Professor D. Mumford, to whom our warmest thanks are due.

\section{The moduli variety \texorpdfstring{$U(n,d)$}{Und}}\label{art17-sec2}

Let $X$ be a compact Riemann surface or equivalently a complete non-singular irreducible algebraic curve\pageoriginale defined over $\bfC$. We shall assume that the genus $g$ of $X$ is $\geq 2$. If $W(\neq 0)$ is a vector bundle (algebraic) on $X$ we define $\mu(W)$ to be the rational number degree $W/\rank W$. A vector bundle $W$ will be called {\em stable} (resp. {\em semi-stable}) if for every proper sub-bundle $V$ of $W$ we have $\mu(V)<\mu(W)$ (resp. $\mu(V)\leq \mu(W)$). D. Mumford proved that the isomorphism classes of stable bundles of rank $n$ and degree $d$ on $X$ form a non-singular quasi-projective algebraic variety (of dimension $n^{2}(g-1)+1$).

A characterisation of stable bundles in terms of irreducible unitary representations of certain discrete groups was given by M. S. Narasimhan and C. S. Seshadri \cite{art17-key4}. This result implies that the space of stable bundles of rank $n$ and degree $d$ is compact if $(n,d)=1$ and that a vector bundle of degree 0 is stable if and only if it arises from an irreducible unitary representation of the fundamental group of $X$. Moreover two such stable bundles are isomorphic if and only if the corresponding unitary representations are equivalent. These results suggest a natural compactification of the space of stable bundles, namely the space of bundles given by all unitary representations (not necessarily irreducible) of a given type.

C. S. Seshari in \cite{art17-key7} proved that this natural compactification is a projective variety. More precisely, Seshadri proved the following. Let $W$ be a semi-stable vector bundle on $X$. Then $W$ has a strictly decreasing filtration
$$
W=W_{0}\supset W_{1}\supset \cdots \supset W_{n}=(0)
$$
such that, for $1\leq i\leq n$, $W_{i}/W_{i-1}$ is a stable vector bundle with $\mu(W_{i-1}/W_{i})=\mu(W)$. Moreover the bundle $\Gr W=\bigoplus\limits^{n}_{i=1}W_{i-1}/W_{i}$ is determined by $W$ upto isomorphism. We say that two semi-stable bundles $W_{1}$ and $W_{2}$ are $S$-{\em equivalent} if $\Gr W_{1}\approx \Gr W_{2}$. Obviously two stable bundles are $S$-equivalent if and only if they are isomorphic. It is proved in \cite{art17-key7} that there is a unique structure of a normal projective variety $U(n,d)$ on the set of $S$-equivalence classes of semi-stable vector bundles of rank $n$ and degree $d$ on $X$ such that the following property holds: if $\{W_{t}\}_{t\in T}$ is an algebraic (resp. holomorphic) family of semi-stable vector bundles of rank $n$ and degree\pageoriginale $d$ parametrised by an algebraic (resp. a complex) space $T$, then the mapping $T\to U(n,d)$ sending $t$ to the $S$-equivalence class of $W_{t}$ is a morphism.

Regarding the singularities of the varieties $U(n,d)$ we have the following result \cite{art17-key3}.

\begin{theorem}\label{art17-thm2.1}
The set of non-singular points of $U(n,d)$ is precisely the set of stable points in $U(n,d)$ except when $g=2$, $n=2$ and $d$ even.
\end{theorem}

It is easy to see that the above characterisation breaks down in the exceptional case. It will follow from the results quoted in \S\ref{art17-sec4}, that when $g=2$, $d$ even, the variety $U(2,d)$ is actually non-singular.

Now let $L$ be a line bundle of degree $d$. Let $U_{L}(n,d)$ be the subspace of $U(n,d)$ corresponding to vector bundles with the determinantal bundle isomorphic to $L$. It is easy to see \cite[\S3]{art17-key4} that all stable vector bundles $V$ in $U_{L}(n,d)$ can be obtained as extensions
$$
0\to E\to V\to (\det E)^{-1}\otimes L\to 0,
$$
where $E$ is a suitably chosen vector bundle, depending only on $U_{L}(n,d)$. Let $U$ be the Zariski open subset of $H^{1}(X,\Hom(L,E)\otimes \det E)$ corresponding to stable bundles. Then the natural morphism $U\to U_{L}(n,d)$ given by the universal property has as image the set of stable points of $U_{L}(n,d)$. This shows that the varieties $U_{L}(n,d)$ are unirational.

By a refinement of the above, it has been shown that the variety $U_{L}(n,d)$ is even rational if $d\equiv \pm 1(\mod n)$. The rationality of these varieties in general is not known.

\section{Poincar\'e families}\label{art17-sec3}

The next problem in the theory of vector bundles is the construction of universal (Poincar\'e) families of bundles on $X$ parametrised by $U(n,d)$. The existence of such a universal bundle is well-known in the case $n=1$.

\begin{defi*}
Let $\Omega$ be a non-empty Zariski open subset of $U(n,d)$ or $U_{L}(n,d)$. A Poincar\'e family of vector bundles on $X$ parametrised by $\Omega$ is an algebraic vector bundle $P$ on $\Omega\times X$ such that for any $\omega\in \Omega$ the\pageoriginale bundle on $X$ obtained by restricting $P$ to $\omega\times X$ is in the $S$-equivalence class $\omega$. The bundle $P$ will be called a Poincar\'e bundle.
\end{defi*}

The following theorem has been proved independently by D. Mumford, S. Ramanan and C. S. Seshadri.

\begin{theorem*}
If $n$ and $d$ are coprime, there is a Poincar\'e bundle on $U(n,d)\times X$.
\end{theorem*}

However we prove, in contrast, the 

\begin{maintheorem*}
Let $X$ be a compact Riemann surface of genus $2$. Then there exists no algebraic Poincar\'e family parametrised by any non-empty Zariski open subset of $U(2,0)$.
\end{maintheorem*}

The theorem will be proved in \S\ref{art17-sec8}. In the next sections we recall some results on vector bundles on a curve of genus 2 which will be used in the proof.

\section{Vector bundles of rank 2 and degree 0 on a curve of genus 2}\label{art17-sec4}

\begin{theorem}\label{art17-thm4.1}
Let $X$ be of genus $2$ and $S$ be the space of $S$-equivalence classes of semi-stable bundles of rank $2$ with trivial determinant on $X$. Let $J^{1}$ be the variety of equivalence classes of line bundles of degree $1$ on $X$ and $\Theta$ the divisor on $J^{1}$ defined by the natural imbedding of $X$ in $J^{1}$. Then $S$ is canonically isomorphic to the projective space $\bfP$ of positive divisors on $J^{1}$ linearly equivalent to $2\Theta$.
\end{theorem}

For the proof see \cite{art17-key3}, \S6.

\begin{remarks*}
\begin{itemize}
\item[(i)] The space $S$ is identified with the set of isomorphism classes of bundles of rank 2 and trivial determinant which are either stable or are of the form $j\oplus j^{-1}$, where $j$ is a line bundle of degree $0$. The space of non-stable bundles in $S$, which is isomorphic to the quotient of the Jacobian $J$ of $X$ by the canonical involution of $J$, gets imbedded in $\bfP$ as a Kummer surface.

\item[(ii)] This theorem shows in particular that $S$ is non-singular. It follows easily from this that $U(2,0)$ is non-singular if $g=2$. In fact, $U(2,0)$ is isomorphic to the variety of positive divisors algebraically equivalent to $2\Theta$, which is a projective bundle over the Jacobian.

\item[(iii)] This\pageoriginale theorem suggests a close connection between $U(2,0)$ and the variety of positive divisors on the Jacobian algebraically equivalent to $2\Theta$, when $g$ is arbitrary. This relationship has been studied when $g=3$ and will be published elsewhere.
\end{itemize}
\end{remarks*}

\section{Quadratic complexes and related projective bundles}\label{art17-sec5}

Before stating the next theorem it is convenient to recall certain notations connected with a quadratic complex of lines in a three dimensional projective space. For more details see \cite{art17-key3}.

Let $R$ be a four dimensional vector space over $\bfC$. Then the Grassmannian of lines $G$ in the projective space $P(R)$ is naturally embedded as a quadric in $P({\displaystyle{\mathop{\wedge}\limits^{2}}}R)$. Consider the tautological exact sequence
$$
0\to L^{-1}\to R\to F\to 0
$$
of vector bundles on $P(R)$ where $L$ is the hyperplane bundle on $P(R)$. This leads to an exact sequence
$$
0\to F\otimes L^{-1}\to {\displaystyle{\mathop{\wedge}\limits^{2}}} R\to {\displaystyle{\mathop{\wedge}\limits^{2}}} F\to 0. 
$$
This induces an injection $P(F\otimes L^{-1})\to P({\displaystyle{\mathop{\wedge}\limits^{2}}} R)\times P(R)$; the image is contained in $G\times P(R)$ and is the incidence correspondence between lines and points in $P(R)$. Consider the diagram
\[
\xymatrix@C=.1cm@R=1.5cm{
 & P(F\otimes L^{-1})\ar[dl]_-{p_{1}}\ar[dr]^-{p_{2}} &\\
G & & P(R)
}
\]
The map $p_{1}$ is a fibration with projective lines as fibres, associated to the universal vector bundle on $G$. For $\omega\in P(R)$, $p^{-1}_{2}(x)$ is mapped isomorphically by $p_{1}$ onto a plane contained in $G$. A quadratic complex of lines is simply an element of $PH^{0}(G,H^{2})$, where $H$ is the restriction to $G$ of the hyperplane line bundle on $P({\displaystyle{\mathop{\wedge}\limits^{2}}} R)$.

A\pageoriginale {\em generic} quadratic complex in $P(R)$ is a subvariety $Q$ of $G$ defined by equations of the form
$$
\left\{
\begin{array}{l}
\sum\limits^{6}_{i=1}x^{2}_{i}=0,\\[5pt]
\sum\limits^{6}_{i=1}\lambda_{i}x^{2}_{i}=0, \ \lambda_{i}\text{~ distinct,~ } \lambda_{i}\in \bfC,
\end{array}\right.
$$
with respect to a suitable coordinate system in $P({\displaystyle{\mathop{\wedge}\limits^{2}}}R)$, where $\sum\limits^{6}_{i=1}x^{2}_{i}=0$ defines the Grassmannian. Let $Y=p^{-1}_{1}(Q)$. We then have a diagram
\[
\xymatrix@C=.1cm@R=1.5cm{
 & Y\ar[dl]_-{q_{1}}\ar[dr]^-{q_{2}} &\\
Q & & P(R)
}
\]
where $q_{1}$ and $q_{2}$ are surjective. For $\omega \in P(R)$, $q^{-1}_{2}(\omega$ is imbedded in the plane $p^{-1}_{2}(\omega)$ as a conic. A point $\omega\in P(R)$ where $q^{-1}_{2}(\omega)$ is a singular conic (i.e. a pair of lines) is called a singular point of the quadratic complex $Q$. The locus $\mathscr{K}$ of singular points in $P(R)$ is a quartic surface with 16 nodes viz. a Kummer surface. Thus if $\Omega$ is a Zariski open subset of $P(R)-\mathscr{K}$, the restriction of $q_{2}$ to $q^{-1}_{2}(\Omega)$ is a projective bundle over $\Omega$. The geometric problem referred to in the introduction is whether this projective bundle is associated to an algebraic vector bundle. We shall show in \S\ref{art17-sec8} that this is not the case. In view of the results of \S\ref{art17-sec7}, this will prove the main theorem.

\section{Vector bundles of rank 2 and degree 1 on a curve of genus 2}\label{art17-sec6}

It has been shown by P. E. Newstead \cite{art17-key6} that the space of stable bundles of rank 2 with determinant isomorphic to a fixed line bundle of degree $-1$ on a curve of genus 2, is isomorphic to the intersection of two quadrics in a 5-dimensional projective space. The following theorem, which is proved in \cite{art17-key3}, is a canonical version of this result which brings out at the same time the relationship between\pageoriginale vector bundles (of rank 2) of degree $0$ and $-1$. This relationship is of importance in the proof of non-existence of Poincar\'e families.

\begin{theorem}\label{art17-thm6.1}
\begin{itemize}
\item[\rm(i)] Let $X$ be of genus $2$ and $x$ a non-Weierstrass point of $X$ (i.e. a point not fixed by the canonical rational involution on $X$). Let $S_{1,x}$ denote the variety of isomorphism classes of stable bundles of rank 2 and determinant isomorphic to $L^{-1}_{x}$, where $L_{x}$ is the line bundle determined by $x$. Let $\bfP$ be the projective space defined in Theorem \ref{art17-thm4.1} and $G$ the Grassmannian of lines in $\bfP$. Then $S_{1,x}$ is canonically isomorphic to the intersection $Q$ of $G$ and another quadric in the ambient $5$-dimensional projective space.

\item[\rm(ii)] The quadratic complex $Q$ is generic and the singular locus of $Q$ is the Kummer surface $\mathscr{K}$ in $\bfP$ corresponding to non-stable bundles in $S$.

\item[\rm(iii)] With the identifications of $S$ with $\bfP$ and $S_{1,x}$ with $Q$, the projective bundle on $S-\mathscr{K}$ defined by the quadratic complex $Q$ (see \S\ref{art17-sec5}) is just the subvariety of $S-\mathscr{K}\times S_{1,x}$ consisting of pairs $(w,v)$ with $H^{0}(X,\Hom(V,W))\neq 0$, where $V$ (resp. $W$) is in the class $v$ (resp. $w$).
\end{itemize}
\end{theorem}

(i) and (ii) have been explicitly proved in \cite{art17-key3}, Theorem 4, \S9. It has been proved there that if $v\in S_{1,x}$ and $\Lambda_{v}$ the line in $\bfP$ defined by $v$, then a point $w\in \bfP$ belongs to $\Lambda_{v}$ if and only if $H^{0}(X,\Hom(V,W))\neq 0$ where $V$(resp. $W$) is a bundle in the class $v$(resp. $w$), (see \S9 of \cite{art17-key3}). (iii) is only a restatement of the above.

\begin{remark*}
One can show that the space of lines on the intersection $Q$ of the two quadrics is isomorphic to the Jacobian of $X[3,6]$. This result is to be compared with the following theorem of D. Mumford and P. E. Newstead \cite{art17-key2}. Let $X$ be of genus $g\geq 2$, and $U'(2,1)$ be the subspace of $U(2,1)$ consisting of bundles with a fixed determinant. Then the intermediary Jacobian of $U'(2,1)$, corresponding to the third cohomology group of $U'(2,1)$, is isomorphic to the Jacobian of $X$. The Betti numbers of $U'(2,1)$ are determined in \cite{art17-key5}.
\end{remark*}

\section{Reduction of the Main Theorem to a geometric problem}\label{art17-sec7}

\begin{lemma}\label{art17-lem7.1}
Let $W$ be a stable bundle of rank $2$ and trivial determinant. Let $x\in X$. Let $\bfO_{x}=\bfO_{X}/\mathfrak{m}_{x}$ be the structure sheaf of $x$.
\begin{itemize}
\item[\rm(i)] If\pageoriginale $V$ is a stable bundle of rank $2$ and determinant $L^{-1}_{x}$ and $f:V\to W$ a non-zero homomorphism, then we have an exact sequence
$$
0\to \bfV\xrightarrow{\bff}\bfW\to \bfO_{x}\to 0.
$$
Moreover $\dim H^{0}(X,\Hom(V,W))\leq 1$.

\item[\rm(ii)] If $\bfW\to \bfO_{x}$ is a non-zero homomorphism, then the kernel is a locally free sheaf of rank $2$, whose associated vector bundle is a stable bundle with determinant $L^{-1}_{x}$.
\end{itemize}
\end{lemma}

\begin{proof}
\begin{itemize}
\item[(i)] It is clear that $f$ must be of maximal rank; for, otherwise the line sub-bundle of $W$ generated by the image of $f$ would have degree $\geq 0$, since $V$ is stable. Now the induced map ${\displaystyle{\mathop{\wedge}\limits^{2}}} f: {\displaystyle{\mathop{\wedge}\limits^{2}}} V\to {\displaystyle{\mathop{\wedge}\limits^{2}}} W$ is non-zero and hence can vanish only at $x$ (with multiplicity 1). Hence $f$ is of maximal rank at all points except $x$ and $f$ is of rank $1$ at $x$. This proves the first part of (i). Now suppose $f$ and $g$ are two linearly independent homomorphisms from $V$ to $W$; choose $y\in X$, $y\neq x$, and let $f_{y}$, $g_{y}$ be the homomorphisms $V_{y}\to W_{y}$ induced by $f$ and $g$ on the fibres of $V$ and $W$ at $y$. Then there exist $\lambda$, $\mu\in \bfC$, $(\lambda,\mu)\neq (0,0)$ such that $\lambda f_{y}+\mu g_{y}$ is not an isomorphism. Then $\lambda f+\mu g$ would be a non-zero homomorphism $V\to W$ which is not of maximal rank at $y$. This is impossible by earlier remarks.

\item[(ii)] Let $V$ be the vector bundle determined by the kernel. It is clear that $\det V=L^{-1}_{x}$. To show that $V$ is stable we have only to show that $V$ contains no line subbundle of degree $\geq 0$. If $L$ were a line subbundle of $V$ of degree $\geq 0$, there would be a non-zero homomorphism $L\to W$, which is impossible since $W$ is stable of degree $0$.
\end{itemize}

Let $p:P\to \Omega\times X$ be a Poincar\'e bundle on $\Omega\times X$, where $\Omega$ is an open subset of $S$ (see Theorem \ref{art17-thm4.1}) consisting of stable points. Let $x\in X$ and let $\bfO_{x}=\bfO_{X}/\mathfrak{m}_{x}$ be the structure sheaf of the point $x$. Then the sheaf $\scrHom (\bfP,p^{*}{}_{X}\bfO_{x})$ on $\Omega\times X$ is $p_{\Omega}$ flat. Moreover, for each $\omega\in \Omega$
\begin{align*}
& \dim H^{0}(\omega\times X,\scrHom (\bfP,p^{*}{}_{X}\bfO_{x})|_{\omega\times X})\\[3pt]
&= \dim H^{0}(\omega\times X,\scrHom (\bfP|_{\omega\times X},\bfO_{x}))\\[3pt]
&= \dim P^{*}_{(\omega,x)}\\[3pt]
&= 2.
\end{align*}\pageoriginale
Hence by \cite{art17-key1}, the direct image $(p_{\Omega})_{*}\Hom (\bfP,p^{*}\bfO_{x})$ is a locally free sheaf on $\Omega$ and consequently defines a vector bundle $E$ on $\Omega$.
\end{proof}

\begin{proposition}\label{art17-prop7.1}
There is a morphism
$$
P(E)\to \Omega\times S_{1,x}
$$
such that the diagram
\[
\xymatrix@C=.5cm@R=1.5cm{
P(E)\ar[rr]\ar[dr] & & \Omega\times S_{1,x}\ar[dl]\\
 & \Omega &
}
\]
is commutative. Moreover this morphism is an isomorphism onto the subvariety of pairs $(W,V)$ such that $H^{0}(X,\Hom(V,W))\neq 0$, $V\in S_{1,x}$, $W\in \Omega$.
\end{proposition}

\begin{proof}
Consider on $\Omega\times X$ the sheaf $\mathscr{G}=\scrHom (\bfP,p^{*}_{X}\bfO_{x})$. Then we have clearly the canonical isomorphisms
$$
p_{*}(\bfT\otimes p^{*}\mathscr{G})\approx p_{*}(\bfT)\otimes \mathscr{G}\approx p^{*}_{\Omega}(\bfE)^{*}\times \mathscr{G},
$$
where $p:P(E)\times X\to \Omega\times X$ is the natural projection and $T$ is the tautological hyperplane bundle in $P(E)\times X$. Moreover, the direct image of $p^{*}_{\Omega}(\bfE^{*})\times \mathscr{G}$ on $\Omega$ is isomorphic to $\bfE^{*}\times p_{\Omega_{*}}(\mathscr{G})\approx \bfE^{*}\otimes \bfE$. Hence $H^{0}(P(E)\times X, (\bfT\otimes p^{*}\mathscr{G}))\approx H^{0}(\Omega,E^{*}\otimes \bfE)$. Hence the canonical element of $H^{0}(\Omega,\bfE^{*}\otimes \bfE)$ (viz. the identity endomorphism of $E$) gives rise to an element of $H^{0}(P(E)\times X,\bfT\otimes p^{*}\mathscr{G})$. In other words, we have a canonical homomorphism $p^{*}\bfP\to p^{*}_{X}(\bfO_{x})\otimes \bfT$ of sheaves on $P(E)\times X$. Consider the commutative diagram
\[
\xymatrix@=1.2cm{
P(E)\times X\ar[d]^-{p}\ar[r] & P(E)\ar[d]^-{p}\\
\Omega\times X\ar[r] & \Omega
}
\]
The\pageoriginale direct image of $\bfT\otimes p^{*}(\mathscr{G})$ on $P(E)$ is simply $\bfT\otimes p^{*}(\bfE)$, where $T$ also denotes the tautological bundle on $P(E)$, and the canonical element in $H^{0}(P(E)\times X,\bfT\otimes p^{*}(\mathscr{G}))$ defined above is given by the tautological element of $H^{0}(P(E),\bfT\otimes p^{*}(\bfE))$. From this we see that for $f\in P(E)$, the restriction of the homomorphism $p^{*}(\bfP)\to p^{*}_{X}(\bfO_{x})\otimes T$ to $f\times X$ can be described as follows. The restriction of $p^{*}(P)$ to $f\times X$ is the restriction of $P$ to $p(f)\times X$ and hence is a stable vector bundle $W$ with trivial determinant. Moreover $f$ gives rise to a 1-dimensional subspace of $H^{0}(X,\scrHom(\bfW,\bfO_{x}))$. Any non-zero element in this 1-dimensional space gives rise to a surjective homomorphism of $p^{*}\bfP|f\times X=\bfW$ into $p^{*}_{X}\bfO_{x}\times \bfT|_{f\times X}\approx \bfO_{x}$. This homomorphism (upto a non-zero scalar) is the restriction of the canonical element. In particular it follows that the canonical homomorphism $p^{*}(\bfP)\to p^{*}_{X}(\bfO_{x})\otimes \bfT$ is surjective. Moreover since $p^{*}_{X}(\bfO_{x})\otimes \bfT$ has a locally free resolution of length 1 we see that the kernel of the homomorphism $p^{*}(\bfP)\to p^{*}(\bfO_{x})\otimes \bfT$ is locally free. Let $F$ be the vector bundle on $P(E)\times X$ associated to the kernel.
\end{proof}

\begin{lemma}\label{art17-lem7.2}
The restriction of the vector bundle $F$ to $f\times X$, $f\in P(E)$ is a stable vector bundle of rank $2$ and determinant $L^{-1}_{x}$.
\end{lemma}

In view of our earlier identification the lemma follows from Lemma \ref{art17-lem7.1}.

We now complete the proof of the proposition. By Lemma \ref{art17-lem7.2} and the universal property of $S_{1,x}$ we have a morphism $q:P(E)\to S_{1,x}$. Then the morphism $(p,q):P(E)\to \Omega\times S_{1,x}$ satisfies the conditions of the proposition, in view of Lemma \ref{art17-lem7.1}. The morphism is an isomorphism onto the subvariety described in Proposition \ref{art17-prop7.1}, as this subvariety is non-singular by Theorem \ref{art17-thm6.1}.

From Proposition \ref{art17-prop7.1} and Theorem \ref{art17-thm6.1} we have immediately the

\begin{coro*}
If there is a Poincar\'e family on an open subset $\Omega$ of the set of stable points in $S$, then the projective bundle on $\Omega$ defined by the quadratic complex $Q=S_{1,x}$ is associated to a vector bundle.
\end{coro*}

\section{Proof of the Main Theorem. Solution of the geometric problem}\label{art17-sec8}

It is easy to see that if there is a Poincar\'e family parametrised by\pageoriginale a Zariski open subset of $U(2,0)$, there would exist a Poincar\'e family parametrised by a Zariski open subset of the space of stable points in $S$. In view of the Corollary of Proposition \ref{art17-prop7.1}, the main theorem in \S\ref{art17-sec3} follows from

\begin{proposition}\label{art17-prop8.1}
With the notation of \S\ref{art17-sec5}, let $\Omega$ be a Zariski open subset of $P(R)-\mathscr{K}$. Let $q_{2}:q^{-1}_{2}(\Omega)\to \Omega$ be the projective bundle defined in \S\ref{art17-sec5}. Then there is no algebraic vector bundle on $\Omega$ to which this projective bundle is associated.
\end{proposition}

\begin{proof}
If there is such a vector bundle there would exist a Zariski open set $\Omega'$ of $\Omega$ and a section $\sigma$ over $\Omega'$ of the projective bundle $q^{-1}_{2}(\Omega)\to \Omega$. Let $D$ be the Zariski closure of $\sigma(\Omega')$ in $Y$. Then $D$ is a divisor of $Y$ and, since $Y$ is non-singular, $D$ defines a line bundle $L_{D}$ on $Y$. The restriction of the first Chern class of $L_{D}$ to a fibre $q^{-1}_{2}(\omega)$, $\omega\in \Omega'$, is the fundamental class of the fibre. On the other hand, we shall show that every element of $H^{2}(Y,\bfZ)$ restricts to an even multiple of the fundamental class of $q^{-1}_{2}(\omega)$ in $H^{2}(q^{-1}_{2}(\omega),\bfZ)$; this contradiction would prove the proposition. We have the commutative diagram
\[
\xymatrix{
H^{2}(P(F\otimes L^{-1}),\bfZ)\ar[d]\ar[r] & H^{2}(p^{-1}_{2}(\omega),\bfZ)\approx H^{2}(\bfP^{2},\bfZ)\ar[d]\\
H^{2}(Y,\bfZ)\ar[r] & H^{2}(q^{-1}_{2}(\omega),\bfZ),
}
\]
with the notation of \S\ref{art17-sec5}. We first note that the canonical mapping\break $H^{2}(G,\bfZ)\to H^{2}(Q,\bfZ)$ is an isomorphism, by Lefschetz's theorem on hypersurface sections. Moreover since $p_{1}:P(F\otimes L^{-1})\to G$ (resp. $q:Y\to Q$) is the projective bundle associated to a vector bundle, $H^{2}(P(F\otimes L^{-1})\bfZ)$ (resp. $H^{2}(Y,\bfZ)$) is generated by the first Chern class of the tautological line bundle of the fibration $P(F\otimes L^{-1})\to G$ (resp. $Y\to Q$) and by $p^{*}_{2}(H^{2}(G,\bfZ))$ (resp. $q^{*}_{2}H^{2}(Q,\bfZ)$). Since this tautological line bundle on $P(F\otimes L^{-1})$ restricts to the tautological line bundle of the fibration $Y\to Q$ and $H^{2}(G,\bfZ)\to H^{2}(Q,\bfZ)$ is an isomorphism, it follows that $H^{2}(P(F\times L^{-1}),\bfZ)\to H^{2}(Y,\bfZ)$ is surjective. Now from the commutativity\pageoriginale of the diagram we see that image $H^{2}(Y,\bfZ)\to H^{2}(q^{-1}_{2}(\omega),\bfZ)$ is contained in the image 
$$
H^{2}(p^{-1}_{2}(\omega),\bfZ)\to H^{2}(q^{-1}_{2}(\omega),\bfZ).
$$ 
But $q^{-1}_{2}(\omega)$ is imbedded in the plane $p^{-1}_{2}(\omega)$ as a conic and hence the image $H^{2}(Y,\bfZ)\to H^{2}(q^{-1}_{2}(\omega),\bfZ)$ consists of even multiples of the fundamental class of $q^{-1}_{2}(\omega)$.
\end{proof}

\begin{thebibliography}{99}
\bibitem{art17-key1} \textsc{A. Grothendieck :} {\em \'Elements de G\'eom\'etrie Alg\'ebrique,} Ch. III, Inst. Hautes. Etudes. Sci., Publ. Math., 17 (1963).

\bibitem{art17-key2} \textsc{D. Mumford} and \textsc{P. E. Newstead :} Periods of a moduli space of bundles on curves, {\em Amer. J. Math.} 90 (1968), 1201-1208.

\bibitem{art17-key3} \textsc{M. S. Narasimhan} and \textsc{S. Ramanan :} Moduli of vector bundles on a compact Riemann surface, {\em Ann. of Math.} 89 (1969), 14-51.

\bibitem{art17-key4} \textsc{M. S. Narasimhan} and \textsc{C. S. Seshadri :} Stable and unitary vector bundles on a compact Riemann surface, {\em Ann. of Math.} 82 (1965), 540-567.

\bibitem{art17-key5} \textsc{P. E. Newstead :} Topological properties of some spaces of stable bundles, {\em Topology}, 6 (1967), 241-262.

\bibitem{art17-key6} \textsc{P. E. Newstead :} Stable bundles of rank 2 and odd degree on a curve of genus 2, {\em Topology}, 7 (1968), 205-215.

\bibitem{art17-key7} \textsc{C. S. Seshadri :} Space of unitary vector bundles on a compact Riemann surface, {\em Ann. of Math.} 85 (1967), 303-336.
\end{thebibliography}

\noindent
Tata Institute of Fundamental Research

\noindent
Bombay.
