\title{STANDARD CONJECTURES ON ALGEBRAIC CYCLES}
\markright{Standard Conjectures on Algebraic Cycles}

\author{By~~ A. Grothendieck}

\date{}

\maketitle

\setcounter{pageoriginal}{192}
\section{Introduction}\label{art09-sec1}\pageoriginale

We state two conjectures on algebraic cycles, which arose from an attempt at understanding the conjectures of Weil on the $\zeta$-functions of algebraic varieties. These are not really new, and they were worked out about three years ago independently by Bombieri and myself.

The first is an existence assertion for algebraic cycles (considerably weaker than the Tate conjectures), and is inspired by and formally analogous to Lefschetz's structure theorem on the cohomology of a smooth projective variety over the complex field.

The second is a statement of positivity, generalising Weil's well-known positivity theorem in the theory of abelian varieties. It is formally analogous to the famous Hodge inequalities, and is in fact a consequence of these in characteristic zero.

\textsc{What remains to be proved of Weil's conjectures ?} Before stating our conjectures, let us recall what remains to be proved in respect of the Weil conjectures, when approached through $l$-adic cohomology.

Let $X/\bfF_{q}$ be a smooth irreducible projective variety of dimension $n$ over the finite field $\overline{\bfF}_{q}$ with $q$ elements, and $l$ a prime different from the characteristic. It has then been proved by M. Artin and myself that the $Z$-function of $X$ can be expressed as
\begin{align*}
Z(t) &= \frac{L'(t)}{L(t)},\\[3pt]
L(t) &= \dfrac{L_{0}(t)L_{2}(t)\ldots L_{2n}(t)}{L_{1}(t)L_{3}(t)\ldots L_{2n-1}(t)},\\[3pt]
L_{i}(t) &= \dfrac{1}{P_{i}(t)},
\end{align*}
where\pageoriginale $P_{i}(t)=t^{\dim H^{i}(\overline{X})}Q_{i}(t^{-1})$, $Q_{i}$ being the characteristic polynomial of the action of the Frobenius endomorphism of $X$ on $H^{i}(\overline{X})$ (here $H^{i}$ stands for the $i^{\text{th}}$ $l$-adic cohomology group and $\overline{X}$ is deduced from $X$ by base extension to the algebraic closure of $\bfF_{q}$). But it has not been proved so far that
\begin{itemize}
\item[(a)] the $P_{i}(t)$ have integral coefficients, independent of $l(\neq \text{ char } \bfF)$;

\item[(b)] the eigenvalues of the Frobenius endomorphism on $H^{i}(\overline{X})$, i.e., the reciprocals of the roots of $P_{i}(t)$, are of absolute value $q^{i/2}$.
\end{itemize}

Our first conjecture meets question (a). The first and second together would, by an idea essentially due to Serre \cite{art09-key4}, imply (b).

\section{A weak form of conjecture 1}\label{art09-sec2}

From now on, we work with varieties over a ground field $k$ which is algebraically closed and of arbitrary characteristic. Then (a) leads to the following question: If $f$ is an endomorphism of a variety $X/k$ and $l\neq \text{ char } k$, $f$ induces
$$
f^{i}:H^{i}(X)\to H^{i}(x),
$$
and each of these $f^{i}$ has a characteristic polynomial. {\em Are the coefficients of these polynomials rational integers, and are they independent of $l$ ?} When $X$ is smooth and proper of dimension $n$, the same question is meaningful when $f$ is replaced by any cycle of dimension $n$ in $X\times X$, considered as an algebraic correspondence.

In characteristic zero, one sees that this is so by using integral cohomology. If char $k>0$, one feels certain that this is so, but this has not been proved so far.

Let us fix for simplicity an isomorphism
\begin{equation*}
{}_{\iota^{\infty}}k^{*}\simeq \bfQ_{l}/\bfZ_{l}\tag*{(\text{a heresy!}).}
\end{equation*}
We then have a map
$$
\text{cl } : \mathscr{Z}^{i}(X)\otimes_{\bfZ}\bfQ\to H^{2i}_{l}(X) 
$$
which associates to an algebraic cycle its cohomology class. We denote the image by $C^{i}_{l}(X)$, and refer to its elements as {\em algebraic cohomology classes.}

A\pageoriginale known result, due to Dwork-Faton, shows that for the integrality question (not to speak of the independence of the characteristic polynomial of $l$), it suffices to prove that
$$
\Tr f^{N}_{i}\in \dfrac{1}{m}\bfZ\quad\text{for every}\quad N\geq 0,
$$
where $m$ is a fixed positive integer\footnote{This was pointed out to me by S. Kleimann.}. Now, the graph $\Gamma_{f^{N}}$ in $X\times X$ of $f^{N}$ defines a cohomology class on $X\times X$, and if the cohomology class $\Delta$ of the diagonal in $X\times X$ is written as
$$
\Delta=\sum\limits^{n}_{0}\pi_{i}
$$
where $\pi_{i}$ are the projections of $\Delta$ onto $H^{i}(X)\otimes H^{n-i}(X)$ for the canonical decomposition $H^{n}(X\times X)\simeq \sum\limits^{n}_{i=0}H^{i}(X)\otimes H^{n-i}(X)$, a known calculation shows that
$$
\Tr(f^{N})_{H^{i}}=(-1)^{i}\cl(\Gamma_{f^{N}})\pi_{i}\in H^{4n}(X\times X)\approx \bfQ_{\iota}.
$$
{\em Assume that the $\pi_{i}$ are algebraic.} Then $\pi_{i}=\dfrac{1}{m}\cl(\Pi_{i})$, where $\Pi_{i}$ is an algebraic cycle, hence
$$
\Tr(f^{N})_{H^{i}}=(-1)^{i}(\Pi_{i}\cdot \Gamma_{f^{N}})\in \dfrac{1}{m}\bfZ
$$
and we are through.

\textsc{Weak form of Conjecture 1.} $(C(X))$: The elements $\pi^{l}_{i}$ are algebraic, (and come from an element of $\mathscr{Z}^{i}(X)\otimes_{\bfZ}\bfQ$, which is independent of $l$).
\begin{description}
\item[N.B. 1.] The statement in parenthesis is needed to establish the independence of $P_{i}$ on $l$.
 
\item[2.] If $C(X)$ and $C(Y)$ hold, $C(X\times Y)$ holds, and more generally, the K\"unneth components of any algebraic cohomology class on $X\times Y$ are algebraic.
\end{description}

\section{The conjecture 1 (of Lefschetz type)}\label{art09-sec3}

Let $X$ be smooth and projective, and $\xi\in H^{2}(X)$ the class of a hyperplane section. Then we have a homomorphism
\begin{equation*}
\cup \xi^{n-i}:H^{i}(X)\to H^{2n-i}(X)\quad (i\leq n).\tag{*}
\end{equation*}\pageoriginale
It is expected (and has been established by Lefschetz \cite{art09-key2}, \cite{art09-key5} over the complex field by transcendental methods) that this is an isomorphism for all characteristics. For $i=2j$, we have the commutative square
\[
\xymatrix@=1.2cm{
H^{2j}(X)\ar[r]^-{\xi^{n-2j}} & H^{2n-2j}(X)\\
C^{j}(X)\ar[u]\ar[r] & C^{n-j}(X)\ar[u]
}
\]
Our conjecture is then: $(A(X))$:
\begin{itemize}
\item[(a)] (*) {\em is always an isomorphism (the mild form);}

\item[(b)] {\em if $i=2j$, (*) induces an isomorphism (or equivalently, an epimorphism)} $C^{j}(X)\to C^{n-j}(X)$.
\end{itemize}

N.B.~ If $C^{j}(X)$ is assumed to be finite dimensional, (b) is equivalent to the assertion that $\dim C^{n-j}(X)\leq \dim C^{j}(X)$ (which in particular implies the equality of these dimensions in view of (a).

An equivalent formulation of the above conjecture (for all varieties $X$ as above) is the following.
$$
(B(X)): \text{~\em The $\Lambda$-operation (c.f. \cite{art09-key5}) of Hodge theory is algebraic.}
$$

By this, we mean that there is an algebraic cohomology class $\lambda$ in $H^{*}(X\times X)$ such that the map $\Lambda:H^{*}(X)\to H^{*}(X)$ is got by lifting a class from $X$ to $X\times X$ by the first projection, cupping with $\lambda$ and taking the image in $H^{*}(X)$ by the Gysin homomorphism associated to the second projection.

Note that $B(X)\Rightarrow A(X)$, since the algebraicity of $\Lambda$ implies that of $\Lambda^{n-i}$, and $\Lambda^{n-i}$ provides an inverse to $\cup \xi^{n-i}:H^{i}(X)\to H^{2n-i}(X)$. On the other hand, it is easy to show that $A(X\times X)\Rightarrow B(X)$ and this proves the equivalence of conjectures $A$ and $B$.

The conjecture seems to be most amenable in the form $B$. Note that $B(X)$ is stable for products, hyperplane sections and specialisations. In particular, since it holds for projective space, it is also true for\pageoriginale smooth varieties which are complete intersections in some projective space. (As a consequence, we deduce for such varieties the wished-for integrality theorem for the $Z$-function !). It is also verified for Grassmannians, and for abelian varieties (Liebermann \cite{art09-key3}).

I have an idea of a possible approach to Conjecture $B$, which relies in turn on certain unsolved geometric questions, and which should be settled in any case.

Finally, we have the implication $B(X)\Rightarrow C(X)$ (first part), since the $\pi_{i}$ can be expressed as polynomials with coefficients in $\bfQ$ of $\Lambda$ and $L=\cup \xi$. To get the whole of $C(X)$, one should naturally assume further that there is an element of $\mathscr{Z}(X\times X)\otimes_{\bfZ}\bfQ$ which gives $\Lambda$ for every $l$.

\section{Conjecture 2 (of Hodge type)}\label{art09-sec4}

For any $i\leq n$, let $P^{i}(X)$ be the `primitive part' of $H^{i}(X)$, that is, the kernel of $\cup \xi^{n-i+1}:H^{i}(X)\to H^{2n-i+2}(X)$, and put $C^{j}_{\Pr}(X)=P^{2j}\cap C^{j}(X)$. On $C^{j}_{\Pr}(X)$, we have a $\bfQ$-valued symmetric bilinear form given by
$$
(x,y)\to (-1)^{j}K(x\cdot y\cdot \xi^{n-2j})
$$
where $K$ stands for the isomorphism $H^{2n}(X)\simeq \bfQ_{l}$. Our conjecture is then that

$(Hdg(X))$: {\em The above form is positive definite.}

One is easily reduced to the case when $\dim X=2m$ is even, and $j=m$.

\begin{remarks*}
\begin{itemize}
\item[(1)] In characteristic zero, this follows readily from Hodge theory \cite{art09-key5}.

\item[(2)] $B(X)$ and $Hdg(X\times X)$ imply, by certain arguments of Weil and Serre, the following: if $f$ is an endomorphism of $X$ such that $f^{*}(\xi)=q\cdot \xi$ for some $q\in \bfQ$ (which is necessarily $>0$), then the eigenvalues of $f_{H^{i}(X)}$ are algebraic integers of absolute value $q^{i/2}$. Thus, this implies all of Weil's conjectures.

\item[(3)] The conjecture $Hdg(X)$ together with $A(X)(a)$ (the Lefschetz conjecture in cohomology) implies that numerical equivalence of cycles\pageoriginale is the same as cohomological equivalence for any $l$-adic cohomology if and only if $A(X)$ holds.

Thus, we see that in characteristic $0$, the conjecture $A(X)$ is equivalent to the well-known conjecture on the equality of cohomological equivalence and numerical equivalence.

\item[(4)] In view of (3), $B(X)$ and $Hdg(X)$ imply that numerical equivalence of cycles coincides with $\bfQ_{l}$-equivalence for any $l$. Further the natural map
$$
\foprod{Z^{i}(X)}{\bfQ_{l}}{\bfZ}\to H^{i}_{l}(X)
$$
is a monomorphism, and in particular, we have
$$
\dim_{\bfQ}C^{i}(X)\leq \dim_{\bfQ_{l}}H^{i}_{l}(X).
$$
Note that for the deduction of this, we do not make use of the positivity of the form considered in $Hdg(X)$, but only the fact that it is non-degenerate.
\end{itemize}

Another consequence of $Hdg(X)$ and $B(X)$ is that the stronger version of $B(X)$, viz. that $\Lambda$ comes from an algebraic cycle with rational coefficients {\em independent of $l$}, holds.
\end{remarks*}

\noindent
{\bf Conclusions.}~ The proof of the two standard conjectures would yield results going considerably further than Weil's conjectures. They would form the basis of the so-called ``theory of motives'' which is a systematic theory of ``arithmetic properties'' of algebraic varieties, as embodied in their groups of classes of cycles for numerical equivalence. We have at present only a very small part of this theory in dimension one, as contained in the theory of abelian varieties.

Alongside the problem of resolution of singularities, the proof of the standard conjectures seems to me to be the most urgent task in algebraic geometry.

\begin{thebibliography}{99}
\bibitem{art09-key1} \textsc{S. Kleimann :}\pageoriginale {\em Expos\'e given at I.H.E.S., Bures,} 1967.

\bibitem{art09-key2} \textsc{S. Lefschetz :} {\em L'analysis situs et la g\'eometrie alg\'ebrique,} Gauthier-Villars, Paris, 1924.

\bibitem{art09-key3} \textsc{D. I. Lieberman :} {\em Higher Picard Varieties.} (To appear.)

\bibitem{art09-key4} \textsc{J.-P. Serre :} Analogues K\"ahleriennes des certaines conjectures de Weil, {\em Ann. of Math.} {\bf 71} (1960).

\bibitem{art09-key5} \textsc{A. Weil :} {\em Vari\'eti\'es K\"ahleriennes,} Hermann, Paris, 1968.
\end{thebibliography}

\bigskip

\noindent
{\small I. H. E. S., Bures}

\noindent
{\small France.}
