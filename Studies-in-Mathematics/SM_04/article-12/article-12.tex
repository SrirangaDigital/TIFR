\title{GEOMETRIC AND ANALYTIC METHODS IN THE THEORY OF THETA-FUNCTIONS$^{*}$}
\markright{Geometric and Analytic Methods in the Theory of Theta-Functions}
\footnotetext{This work was partially supported by the National Science Foundation.}

\author{By~~ Jun-Ichi Igusa}

\date{}

\maketitle

\setcounter{pageoriginal}{240}
\textsc{It}\pageoriginale is well known that theta-functions appeared in the early nineteenth century in connection with elliptic functions. This theory, although it has apparent analytic features, is basically geometric. However, the Gauss proof of the transformation law for the Legendre modulus and Jacobi's application of theta-functions to number theory are not geometric. More precisely, if we start from the view-point that theta-functions give nice projective embeddings of a polarized abelian variety over $\bfC$, we are in the geometric side of the theory. On the other hand, if we regard theta-functions as functions obtained by the summations of normal densities over a lattice in a vector space over $\bfR$, we are in the analytic side of the theory. After this naive explanation of the title, we shall start discussing certain results in the theory of theta-functions.

1.~ Suppose that $X$ is a vector space of dimension $n$ over $\bfC$ and $L$ a lattice in $X$. Then, a holomorphic function $x\to \theta(x)$ on $X$ is called a {\em theta-function belonging to $L$} if it has the property that
$$
\theta(x+\xi)=e(A_{\xi}(x)+b_{\xi})\cdot \theta(x)
$$
for every $\xi$ in $L$ with a $\bfC$-linear form $A_{\xi}$ on $X$ and with a constant $b_{\xi}$ both depending on $\xi$. The notation $e(b)$ stands for $\exp(2\pi ib)$. If the theta-function is not the constant zero, it determines a positive divisor $D$ of the quotient $Q=X/L$, which is a complex torus. A fundamental existence theorem in the theory of theta-functions asserts that every positive divisor of $Q$ can be obtained in this manner (cf. \cite{art12-key23}, \cite{art12-key24}).

We observe that the function $(x,\eta)\to A_{\eta}(x)$ on $X\times L$ can be extended uniquely to an $\bfR$-bilinear form on $X\times X$. Then, the bi-character $f$ of $X\times X$ defined by
$$
f(x,y)=e(A_{y}(x)-A_{x}(y))
$$\pageoriginale
takes the value 1 on $L\times L$ and also on the diagonal of $X\times X$. The divisor $D$ is called non-degenerate if $f$ is non-degenerate. In this case, $f$ gives an autoduality of $X$, and the pair $(Q,f)$ is called a polarized abelian variety over $\bfC$. Another fundamental theorem in the theory of theta-functions asserts that the vector space over $\bfC$ of theta-functions which have the same periodicity property as $\theta(x)^{3}$ gives rise to a projective embedding of this polarized abelian variety (c.f. \cite{art12-key23}, \cite{art12-key24}). We shall consider the special case when $f$ gives an autoduality of $(X,L)$ in the sense that the annihilator $L_{*}$ of $L$ with respect to $f$ coincides with $L$. The polarized abelian variety is then called {\em autodual}. In this case, we can choose a $\bfZ$-base $\xi_{1},\ldots,\xi_{2n}$  of $L$ so that we get
$$
f\left(\sum\limits^{2n}_{i=1}x_{i}\xi_{i},\sum\limits^{2n}_{i=1}y_{i}\xi_{i}\right)=e(x'\ {}^{t}y''-x'' \ {}^{t}y').
$$
We are denoting by $x'$ and $x''$ the line vectors determined by $x_{1},\ldots,x_{n}$ and $x_{n+1},\ldots,x_{2n}$; similarly for $y'$ and $y''$. We observe that there exists a unique $\bfC$-base of $X$, and hence a unique $\bfC$-linear isomorphism $X\xrightarrow{\sim}\bfC^{n}$ mapping the colomn vector determined by $\xi_{1},\ldots,\xi_{2n}$ to a $2n\times n$ matrix composed of an $n\times n$ matrix $\tau$ and $1_{n}$. Then $\tau$ is necessarily a point of the Siegel upper-half plane $\mathfrak{S}_{n}$ and, if $x$ is mapped to $z$, we have
\begin{align*}
\theta(x)= & e(\text{polynomial in $z$ of degree two})\\
           & \cdot \sum\limits_{p\in \bfZ^{n}}e(\frac{1}{2}(p+m')\tau^{t}(p+m')+(p+m')^{t}(z+m'')),
\end{align*}
in which $m'$, $m''$ are elements of $\bfR^{n}$. We shall denote this series by $\theta_{m}(\tau,z)$ in which $m$ is the line vector composed of $m'$ and $m''$.

The above process of obtaining the {\em theta-function} $\theta_{m}(\tau,z)$ of {\em characteristic $m$} depends on the choice of the $\bfZ$-base $\xi_{1},\ldots,\xi_{1n}$ of $L$. However, once it is chosen, the process is unique except for the fact that the characteristic $m$ is determined only up to an element of $\bfZ^{2n}$. Now, the passing to another $\bfZ$-base of $L$ is given by a left multiplication of an element $\sigma$ of $Sp_{2n}(\bfZ)$ to the column vector determined by $\xi_{1},\ldots,\xi_{2n}$. If $\sigma$ is composed of $n\times n$ submatrices $\alpha$,\pageoriginale $\beta$, $\gamma$, $\delta$ the point $\tau^{*}$ of $\mathfrak{S}_{n}$ which corresponds to the new base is given by
$$
\sigma\cdot \tau=(\alpha \tau+\beta)(\gamma\tau+\delta)^{-1}.
$$
Furthermore, if $\theta_{m^{*}}(\tau^{*},z^{*})$ corresponds to the new base, the relation between $\theta_{m^{*}}(\tau^{*},z^{*})$ and $\theta_{m}(\tau,z)$ is known except for a certain eighth root of unity. This eighth root of unity has been calculated explicitly for some special $\sigma$, e.g. for those $\sigma$ in which $\gamma$ is non-degenerate. This is the classical {\em transformation low of theta-functions.} In particular, if we consider the theta-constants $\theta_{m}(\tau)=\theta_{m}(\tau,0)$ for $m$ in $\bfQ^{2n}$, the transformation law implies that any homogeneous polynomial of even degree, say $2k$, in the theta-constants defines a modular form of weight $k$ belonging to some subgroup, say $\Gamma$, of $Sp_{2n}(\bfZ)$ of finite index. We recall that a modular form of wight $k$ belonging to $\Gamma$ is a holomorphic function $\psi$ on $\mathfrak{S}_{n}$ (plus a boundedness condition at infinity for $n=1$) obeying the following transformation law:
$$
\psi(\sigma\cdot \tau)=\det (\gamma\tau+\delta)^{k}\cdot \psi(\tau)
$$
for every $\sigma$ in $\Gamma$. The set of such modular forms a vector space $A(\Gamma)_{k}$ over $\bfC$, and the graded ring
$$
A(\Gamma)=\bigoplus\limits_{k\geq 0}A(\Gamma)_{k}
$$
is called the {\em ring of modular forms belonging to $\Gamma$}.

Now, the above summarized theory does not give the precise nature of the subgroup $\Gamma$, nor does it provide information on the set of modular forms obtained from the theta-constants. We have given an almost satisfactory answer to these problems in \cite{art12-key9}, and it can be stated in the following way.

Let $\Gamma_{n}(l)$ denote the principal congruence group of level $l$ and consider only those characteristics $m$ satisfying $lm\equiv 0\mod 1$. Then, for any even level $l$, a monomial $\theta_{m_{1}}\ldots \theta_{m_{2k}}$ defines a modular form belonging to $\Gamma_{n}(l)$ if and only if the quadratic form $q$ on $\bfR^{2n}$ defined by
$$
q(x)=(l/2)\left(\sum\limits^{2k}_{\alpha=1}(x^{t}m_{\alpha}\right)^{2}+(k/2)x' \ {}^{t}x'')
$$
is\pageoriginale $\bfZ$-valued on $\bfZ^{2n}$. Moreover, the integral closure within the field of fractions of the ring generated over $\bfC$ by such monomials is precisely the ring $A(\Gamma_{n}(l))$.

The proof of the first part depends on the existence of an explicit transformation formula for $\theta_{m_{1}}\ldots\theta_{m_{2k}}$ valid for every $\sigma$ in $\Gamma_{n}(2)$. The proof of the second part depends on the theory of compactifications (c.f. \cite{art12-key1}, \cite{art12-key4}) and on the Hilbert `Nullstellensatz''. We would like to call attention to the fact that this result connects the unknown ring $A(\Gamma_{n}(l))$ to an explicitly constructed ring of theta-constants. As an application, we have obtained the following theorem \cite{art12-key12}:

There exists a ring homomorphism $\rho$ from a subring of $A(\Gamma_{n}(1))$ to the ring of projective invariants of a binary form of degree $2n+2$ such that $\rho$ increases the weight by a $\frac{1}{2}n$ ratio. Moreover, an element of $A(\Gamma_{n}(1))$ is in the kernel of $\rho$ if and only if it vanishes at every ``hyperelliptic point'' of $\mathfrak{S}_{n}$.

This subring contains all elements of even weights as well as all polynomials in the theta-constants whose characteristics $m$ satisfy $2m\equiv 0\mod 1$ and which are contained in $A(\Gamma_{n}(1))$. It may be that $A(\Gamma_{n}(1))$ simply consists of such polynomials in the theta-constants. If we denote by $\mathscr{A}_{n}(l)$, for any even $l$, the ring generated over $\bfC$ by all monomials $\theta_{m_{1}}\ldots\theta_{m_{2k}}$ satisfying $lm\equiv 0\mod 1$, this is certainly the case when $\mathscr{A}_{n}(2)$ is integrally closed. Now. Mumford informed us (in the fall of 1966) that $\mathscr{A}_{1}(4)$ is not integrally closed. Subsequently we verified that $\mathscr{A}_{1}(l)$ is integrally closed if $l=2$. On the other hand, we have shown in \cite{art12-sec8} that $\mathscr{A}_{2}(2)$ is integrally closed. Therefore, it is possible that $\mathscr{A}_{n}(2)$ is integrally closed for every $n$. We can propose the more general problem of whether we can explicitly give a set of generators for the integral closure of $\mathscr{A}_{n}(l)$. Concerning this problem, we would like to mention a result of Mumford to the effect that $\mathscr{A}_{n}(l)$ is integrally closed locally at every finite point. For this and for other important results, we refer to his paper \cite{art12-key16}. Also, we would like to mention a relatively recent paper of Siegel that has appeared in the G\"ottingen Nachrichten \cite[III]{art12-key22}.

For\pageoriginale the application of the homomorphism $\rho$, it is useful to know that, if $\psi$ is a cusp form, i.e. a modular form vanishing at infinity, at which $\rho$ is defined, the image $\rho(\psi)$ is divisible by the discriminant of the binary form. Also, in the case when $n=3$, the kernel of $\rho$ is a principal ideal generated by a cusp form of weight 18. From this, we immediately get
$$
\dim_{\bfC}A(\Gamma_{n}(1))_{8}\leq 1
$$
for $n=1,2,3$. Actually we have an equality here because each coefficient $\psi_{k}$ in
$$
\Pi (t+(\theta_{m})^{8})=t^{N}+\psi_{1}\cdot t^{N-1}+\cdot k\psi_{N},
$$
in which the product is taken over the $N=2^{n-1}(2^{n}+1)$ characteristics $m$ satisfying $2m\equiv 0$, $2{m'}^{t}m''\equiv 0\mod 1$, is an element of $A(\Gamma_{n}(1))_{4k}$ different from the constant zero for every $n$. Furthermore, by using the classical structure theorem for the ring of projective invariants of a binary sextic, we have reproduced our structure theorem of $A(\Gamma_{2}(1))$ in \cite{art12-key12}. It seems possible to apply the same method to the case when $n=3$ by using Shioda's result in \cite{art12-key21} on the structure of the ring of projective invariants of a binary octavic.

We shall discuss a special but hopefully interesting application of what we have said so far to a conjecture made by Witt in \cite{art12-key27}. He observed that the lattice in $\bfR^{m}$ generated by the root system $D_{m}$ can be extended to a lattice on which $(\frac{1}{2})x^{t}x$ is $\bfZ$-valued if and only if $m$ is a multiple of 8. In this case, there are two extensions, and they are conjugate by an isomorphism inducing an automorphism of $D_{m}$. The restriction of $(\frac{1}{2})x^{t}x$ to the extended lattice gives a positive, non-degenerate quadratic form (of discriminant 1) to which we can associate, for every given $n$, a theta-series called the class invariant. For $m=8k$, this is an element of $A(\Gamma_{n}(1))_{4k}$ different from the constant zero. If we denote the first two elements by $f_{n}$, $g_{n}$, the conjecture is that $(f_{3})^{2}=g_{3}$. We note that $\dim_{\bfC}A(\Gamma_{3}(1))_{8}\leq 1$ gives an affirmative answer to this conjecture. Also, M. Kneser has given another solution by a highly ingenious argument in \cite{art12-key14}.

Now,\pageoriginale if we consider the difference $(f_{4})^{2}-g_{4}$, we get a cusp form of weight 8 for $n=4$. According to the property of the homomorphism $\rho$, this cusp form vanishes at every hyperelliptic point. We may inquire whether it also vanishes at every ``jacobian point''. This question reminds us of an invariant discovered by Schottky which vanishes at every jacobian point (c.f. \cite{art12-key19}). We can see easily that the Schottky invariant, denoted by $J$, is also a cusp form of weight $8$ for $n=4$. We may, therefore, inquire how the two cusp forms are related. The answer is given by the following identity:
$$
(f_{4})^{2}-g_{4}=2^{-2}3^{2}5.7\text{-times } J.
$$
Actually, this identity can be proved directly, and it provides a third solution for the Witt conjecture.

In this connection, we would like to mention that, as far as we can see it, Schottky did not prove the converse, i.e. the fact that the vanishing of $J$ is ``characteristic'' for the jacobian point. In fact, he did not even attempt to prove it. However, it appears that this can be proved. A precise statement is that the divisor determined by $J$ on the quotient $\Gamma_{4}(1)\backslash \mathfrak{S}_{4}$ is irreducible in the usual sense and it contains the set of jacobian points as a dense open subset. We shall publish the proofs for this and for the above mentioned identity in a separate paper.

2. We have considered the ring of modular forms so to speak algebraically. However, as we mentioned earlier, it depends on the possibility of compactifying the quotient $\Gamma\backslash \mathfrak{S}_{n}$ to a (normal) projective variety, say $\mathscr{S}(\Gamma)$. This is a natural approach to the investigation of modular forms. After all, to calculate the dimension of $A(\Gamma)_{k}$ is a problem of ``Riemann-Roch type'', and it was with this problem in mind that Satake first attempted to compactify the quotient $\Gamma_{n}(1)\backslash\mathfrak{S}_{n}$ in \cite{art12-key18}. The theory of compactifications has been completed recently by Baily and Borel \cite{art12-key2}, and it may be just about time to examine the possibility of applying it to the determination of the dimension of $A(\Gamma)_{k}$.

Now,\pageoriginale for such a purpose, it is desirable that $\mathscr{S}(\Gamma_{n}(l))$ becomes non-singular for a large $l$. However, the situation is exactly the opposite. In fact, every point of $\mathscr{S}(\Gamma_{n}(l))-\Gamma_{n}(l)\backslash \mathfrak{S}_{n}$ is singular on $\mathscr{S}_{n}(\Gamma_{n}(l))$ except for a few cases and, for instance, the dimension of the Zariski tangent space tends to infinity with $l$ (c.f. \cite{art12-key10}). We may, therefore, inquire whether $\mathscr{S}(\Gamma_{n}(l))$ admits a ``natural'' desingularization. If the answer is affirmative, we may further hope to obtain the Riemann-Roch theorem for this non-singular model. It turns out that the problem is quite delicate and the whole situation seems to be still chaotic.

In order to explain some results in this direction, we recall that the boundary of a bounded symmetric domain has a stratification which in inherited by the standard compactification of its arithmetically defined quotient. We shall consider only such ``absolute'' stratification. The union of the first, the second,\ldots.strata is called the boundary of the compactification. Then we can state our results in the following way.

Suppose that $D$ is isomorphic to a bounded symmetric domain, and convert the complexification of the connected component of $\Aut(D)$, up to an isogeny, into a linear algebraic group, say $G$, over $\bfQ$. Denote by $\Gamma$ the principal congruence group of $G_{\bfZ}$ of level $l$. Then the monoidal transform, say $\mathscr{M}(\Gamma)$, of the standard compactification $\mathscr{S}(\Gamma)$ of $\Gamma\backslash D$ along its boundary, i.e. the blowing up of $\mathscr{S}(\Gamma)$ with respect to the sheaf of ideals defined by all cusp forms, is non-singular over the first strata for every $l$ not smaller than a fixed integer. Moreover, the fiber of $\mathscr{M}(\Gamma)\to \mathscr{S}(\Gamma)$ at every point of the first strata is a polarized abelian variety. In the special case when $D=\mathfrak{S}_{n}$ and $G=Sp_{2n}(\bfC)$, the monoidal transformation desingularizes up to the third strata with 3 as the fixed integer.

The proof of the first part is a refinement of the proof of a similar result in \cite{art12-key10}. It was obtained (in the summer of 1966) with the help of A. Borel. The proof of the second part and the description of various fibers are in \cite{art12-key11}. (The number 3 comes from the theorem on projective embeddings of a polarized abelian variety and from the\pageoriginale fact that $\Gamma_{n}(l)$ operates on $\mathfrak{S}_{n}$ without fixed points for $l\geq 3$). The basis of the proofs is the theory of Fourier-Jacobi series of Pyatetski-Shapiro \cite{art12-key17}. In this connection, we would like to mention an imaginative paper by Gindikin and Pyatetski-Shapiro \cite{art12-key7}. However, their main result requires the existence of a non-singular blowing up of $\mathscr{S}(\Gamma)$ which, so to speak, coincides with the monoidal transformation along the first strata and which does not create a divisor over the higher strata. The existence of such blowing up is assured for $\Gamma=\Gamma_{n}(l)$ up to $n=3$. In fact, the monoidal transformation has the required properties (c.f. \cite{art12-key11}).

Because of the serious difficulty in constructing a natural desingularization, we may attempts to apply directly to $\mathscr{S}(\Gamma)$ the ``Riemann-Roch theorem for normal varieties'' proved first by Zariski. According to Eichler, his work on the ``Riemann-Roch theorem'' \cite{art12-key6} contains additional material useful for this purpose. Eichler informed us (in the spring of 1967) that he can calculate, for instance, the dimension of $A(\Gamma_{2}(1))_{k}$ at least when $k$ is a multiple of $6$, and thus recover the structure theorem for $A(\Gamma_{2}(1))$.

3. We have so far discussed the geometric method in the theory of theta-functions. The basic features are that objects are ``complex-analytic'' if not algebraic. We shall now abandon this restriction and adopt a freer viewpoint. This has been provided by a recent work of A. Weil that has appeared in two papers (\cite{art12-key25}, \cite{art12-key26}). We shall start by giving an outline of his first paper.

We take an arbitrary locally compact abelian group $X$ and denote its dual by $X^{*}$. We shall denote by $T$ the multiplicative group of complex numbers $t$ satisfying $t\overline{t}=1$ and by $(x,x^{*})\to \langle x,x^{*}\rangle$ the bicharacter of $X\times X^{*}$ which puts $X$ and $X^{*}$ into duality. Then the regular representation of $X$ and the Fourier transform of the regular representation of $X^{*}$ satisfy the so-called Heisenberg commutation relation. Therefore the images of $X$ and $X^{*}$ by these representations generate a group $\bfA(X)$ of unitary operators with the group $\bfT$, of scalar multiplications by elements of $T$, as its center such that
$$
\bfA(X)/\bfT\xrightarrow{\sim} X\times X^{*},
$$\pageoriginale
the isomorphism being bicontinuous. Consider the normalizer $\bfB(X)$ of $\bfA(X)$ in $\Aut (L^{2}(X))$. Then the Mackey theorem \cite{art12-key15} implies that $\bfT$ is the centralizer of $\bfA(X)$ in $\Aut (L^{2}(X))$ and that every bicontinuous automorphism of $\bfA(X)$ inducing the identity on $\bfT$ is the restriction to $\bfA(X)$ of an inner automorphism of $\bfB(X)$. If $B(X)$ denotes the group of such automorphisms of $\bfA(X)$, we have
$$
\bfB(X)/\bfT\xrightarrow{\sim} B(X),
$$
the isomorphism being bicontinuous. On the other hand, if $\mathscr{S}(X)$\footnote{There should be no confusion between $\mathscr{S}(\Gamma)$ and $\mathscr{S}(X)$ because in the first case the group is non-commutative and in the second it is commutative.} is the Schwarz-Bruhat space of $X$ (c.f. \cite{art12-key3}), Weil has shown that every $\bfs$ in $\bfB(X)$ gives a bicontinuous automorphism $\Phi\to \bfs\Phi$ of $\mathscr{S}(X)$. The proof is based on what might be called a five-step decomposition of $\Phi\to \bfs\Phi$, which comes from a work of Segal \cite{art12-key20}. Now, if $L$ is a closed subgroup of $X$, every $\Phi$ in $\mathscr{S}(X)$ gives rise to a function $F^{\Phi}$ on $\bfB(X)$ by the following integral
$$
F^{\Phi}(\bfs)=\int_{L}(\bfs\Phi)(\xi)d\xi
$$
taken with respect to the Haar measure $d\xi$ on $L$. Weil has show that $F^{\Phi}$ has a remarkable invariance property with respect to a certain subgroup of $\bfB(X)$ determined by $L$. Then he has specialized to the ``arithmetic case'' and proved the continuity of $(\bfs,\Phi)\to \bfs\Phi$ and $\bfs\to F^{\Phi}(\bfs)$ restricting $\bfs$ to the metaplectic group, which is a fiber-product over $B(X)$ of $\bfB(X)$ and of an adelized algebraic group.

We shall now explain some supplements to the Weil theory and our generalization of theta-functions in \cite{art12-key13}. For other group-theoretic treatment of theta-functions, we refer to Cartier \cite{art12-key5}.

If $\bfs$ is an element of $\bfB(X)$, it gives rise to a bicontinuous automorphism $\sigma$ of $X\times X^{*}$ keeping
$$
((x,x^{*}),(y,y^{*}))\to \langle x,y^{*}\rangle \langle y,x^{*}\rangle^{-1}
$$
invariant. The group $Sp(X)$ of such automorphisms of $X\times X^{*}$ is known as the symplectic group of $X$. The Weil theory implies that $\bfs\to \sigma$ gives rise to a continuous monomorphism
$$
\bfB(X)/\bfA(X)\to Sp(X).
$$\pageoriginale
We have shown that this monomorphism is surjective and bicontinuous. The topology of $Sp(X)$ is determined by the fact that the group of bicontinuous automorphisms of any locally compact group becomes a topological group by the (modified) compact open topology. We observe that neither $\bfB(X)$ nor $Sp(X)$ is locally bounded, in general. However, if $G$ is a locally compact group and $G\to Sp(X)$, a continuous homomorphism, the fiber-product
$$
\bfB(X)_{G}=\fprod{\bfB(X)}{G}{Sp(X)}
$$
is always locally compact. As for the continuity of $\bfB(X)\times \mathscr{S}(X)\to \mathscr{S}(X)$, it is false in general. However, if $\Sigma$ is a locally compact subset of $\bfB(X)$, the induced mapping $\Sigma\times \mathscr{S}(X)\to \mathscr{S}(X)$ is continuous. In particular, the mapping $\bfB(X)_{G}\times \mathscr{S}(X)\to \mathscr{S}(X)$ defined by $((\bfs,g),\Phi)\to \bfs\Phi$ is continuous. Also, the function $F^{\Phi}$ is always continuous on $\bfB(X)$. In fact, it can be considered locally everywhere as a coimage of continuous functions on Lie groups.

The continuous function $F^{\Phi}$ on $\bfB(X)$ may be called an {\em automorphic function} because of its invariance property mentioned before. We may then define a theta-function as a special automorphic function. For this purpose, we observe that $X$ can be decomposed into a product $X_{0}\times \bfR^{n}$, in which $X_{0}$ is a closed subgroup of $X$ with compact open subgroups. Although this decomposition is not intrinsic (except when $X_{0}$ is the union of totally disconnected compact open subgroups), the dimension $n$ is unique and $X_{0}$ contains all compact subgroups of $X$. We consider a function $\Phi_{0}\otimes \Phi_{\infty}$ on $X$ defined by
\begin{align*}
& \Phi_{0} =  \text{ the characteristic function of a compact open}\\
&\qquad~~\,  \text{subgroup of } X_{0}\\
&\Phi_{\infty}(x_{\infty}) = \exp(-\pi x_{\infty} {}^{t}x_{\infty}).
\end{align*}
Then finite linear combinations of elements of $\bfA(X_{0})\Phi_{0}\otimes \bfB(\bfR^{n})\Phi_{\infty}$ form a dense subspace $\mathscr{G}(X)$ of $\mathscr{S}(X)$ which is $\bfB(X)$-invariant. Moreover $\mathscr{G}(X)$ is intrinsic. We take an element $\Phi$ of $\mathscr{G}(X)$ and call $F^{\Phi}$ a {\em theta-function} on $\bfB(X)$. Then, every automorphic function on $\bfB(X)$ can be approximated uniformly on any compact subset of $\bfB(X)$\pageoriginale by a theta-function. Also, we can restrict $F^{\Phi}$ to the fiber-product $\bfB(X)_{G}$. This procedure gives rise to various theta-functions.

On the other hand, if a tempered distribution $I$ on $X$ vanishes on $\mathscr{G}(X)$, it vanishes identically. Actually a smaller subspace than $\mathscr{G}(X)$ is dense in $\mathscr{S}(X)$. In fact, we can find a locally compact, solvable subgroup $\Sigma(X)$ of $\bfB(X)$ such that the vanishing of the continuous function $\bfs\to I(\bfs\Phi)$ on $\Sigma(X)$ is characteristic for the vanishing of $I$ for $\Phi=\Phi_{0}\otimes \Phi_{\infty}$. Consequently, we would have an identity $I_{1}=I_{2}$ of tempered distributions $I_{1}$ and $I_{2}$ on $X$ if they give rise to the same function on $\Sigma(X)$. It appears that these facts explain to some extent the r\^oles played by theta-functions in number theory. For instance, we can convince ourselves easily that the Siegel formula (for the orthogonal group) as formulated and proved by Weil \cite{art12-key26} and the classical Siegel formula by Siegel \cite{art12-key22} involving theta-series and Eisenstein series are equivalent. This does not mean that the Siegel formula for any given $\Phi$ in $\mathscr{S}(X)$ can be obtained linearly from the classical Siegel formula. The space $\mathscr{G}(X)$ is too small for this. We observe that, if we denote by $\mathscr{G}_{k}(X)$ the subspace of $\mathscr{S}(X)$ consisting of elements of $\mathscr{G}(X)$ multiplied by ``polynomial functions'' of degree $k$, it is also $\bfB(X)$-invariant. Such a space has appeared (in the arithmetic case) in the proof of the functional equation for the Hecke $L$-series. It seems that the meaning and the actual use of elements of $\mathscr{S}(X)$ not contained in the union of
$$
\mathscr{G}(X)=\mathscr{G}_{0}(X)\subset \mathscr{G}_{1}(X)\subset \cdots
$$ 
are things to be investigated in the future.

In rounding off our talk, we remark that, if we take a vector space over $\bfR$ as $X$ and a lattice in $X$ as $L$, the theta-function $F^{\Phi}$ becomes, up to an elementary factor, the theta-function that we have introduced in the beginning (with the understanding that the previous $(X,L)$ is replaced by $(X\times X^{*}, L\times L_{*})$). Moreover, the invariance property of $F^{\Phi}$ becomes the transformation law of theta-functions.

\begin{thebibliography}{99}
\bibitem{art12-key1} \textsc{W. L. Baily :}\pageoriginale Satake's compactification of $V_{n}$, {\em Amer. Jour. Math.} 80 (1958), 348-364.

\bibitem{art12-key2} \textsc{W. L. Baily} and \textsc{A. Borel :} Compactification of arithmetic quotients of bounded symmetric domains, {\em Annals of Math.} 84 (1966), 442-528.

\bibitem{art12-key3} \textsc{F. Bruhat :} Distribution sur un groupe localement compact et applications \`a l'\'etude des repr\'esentations des groupes $\mathfrak{p}$-adiques, {\em Bull. Soc. Math. France} 89 (1961), 43-75.

\bibitem{art12-key4} \textsc{H. Cartan :} Fonctions automorphes, {\em S\'eminaire E.N.S.} (1957-58).

\bibitem{art12-key5} \textsc{P. Cartier :} Quantum mechanical commutation relations and theta functions, {\em Proc. Symposia in Pure Math.} 9 (1966), 361-383.

\bibitem{art12-key6} \textsc{M. Eichler :} Eine Theorie der linearen R\"aume \"uber rationalen Funktionenk\"orpern und der Riemann-Rochsche Satz f\"ur algebraische Funktionenk\"orper I, {\em Math. Annalen} 156 (1964), 347-377; II, {\em ibid.} 157 (1964), 261-275.

\bibitem{art12-key7} \textsc{S. G. Gindikin} and \textsc{I. I. Pyatetski-Shapiro :} On the algebraic structure of the field of Siegel's modular functions, {\em Dokl. Acad. Nauk S. S. S. R.} 162 (1965), 1226-1229.

\bibitem{art12-key8} \textsc{J. Igusa :} On Siegel modular forms of genus two, {\em Amer. Jour. Math.} 84 (1962), 175-200; II, {\em ibid.} 86 (1964), 392-412.

\bibitem{art12-key9} \textsc{J. Igusa :} On the graded ring of theta-constants, {\em Amer. Jour. Math.} 86 (1964), 219-246; II, {\em ibid.} 88 (1966), 221-236.

\bibitem{art12-key10} \textsc{J. Igusa :} On the theory of compactifications (Lect. Notes), {\em Summer Inst. Algebraic Geometry} (1964).

\bibitem{art12-key11} \textsc{J. Igusa :} A desingularization problem in the theory of Siegel modular functions, {\em Math. Annalen} 168 (1967), 228-260. 

\bibitem{art12-key12} \textsc{J. Igusa :} Modular forms and projective invariants, {\em Amer. Jour. Math.} 89 (1967), 817-855.

\bibitem{art12-key13} \textsc{J. Igusa :} Harmonic analysis and theta-functions, {\em to appear.}

\bibitem{art12-key14} \textsc{M. Kneser :} Lineare Relationen zwischen Darstellungsanzahlen quadratischer Formen, {\em Math. Annalen} 168 (1967), 31-39.

\bibitem{art12-key15} \textsc{G. W. Mackey :}\pageoriginale A theorem of Stone and von Neumann, {\em Duke Math. J.} 16 (1949), 313-326.

\bibitem{art12-key16} \textsc{D. Mumford :} On the equations defining abelian varieties, {\em Invent. Math.} 1 (1966), 287-354.

\bibitem{art12-key17} \textsc{I. I. Pyatetski-Shapiro :} The geometry of classical domains and the theory of automorphic functions (in Russian), {\em Fizmatgiz} (1961).

\bibitem{art12-key18} \textsc{I. Satake :} On Siegel's modular functions, {\em Proc. Internat. Symp. Algebraic Number Theory} (1955), 107-129.

\bibitem{art12-key19} \textsc{F. Schottky :} Zur Theorie der Abelschen Functionen von vier Variabeln, {\em Crelles J.} 102 (1888), 304-352.

\bibitem{art12-key20} \textsc{I. E. Segal :} Transforms for operators and symplectic automorphisms over a locally compact abelian group, {\em Math. Scand.} 13 (1963), 31-43.

\bibitem{art12-key21} \textsc{T. Shioda :} On the graded ring of invariants of binary octavics, {\em Amer. Jour. Math.} 89 (1967), 1022-1046.

\bibitem{art12-key22} \textsc{C. L. Siegel :} Gesammelte Abhandlungen, I-III, {\em Springer} (1966).

\bibitem{art12-key23} \textsc{C. L. Siegel :} Vorlesungen \"uber ausgew\"ahlte Fragen der Funktionentheorie (Lect. Notes), {\em G\"ottingen} (1966).

\bibitem{art12-key24} \textsc{A. Weil :} Introduction \`a l'\'etudes des vari\'eti\'es k\"ahleriennes, {\em Actualites Sci. et Ind.} (1958).

\bibitem{art12-key25} \textsc{A. Weil :} Sur certains groupes d'op\'erateurs unitaires, {\em Acta Math.} 111 (1964), 143-211.

\bibitem{art12-key26} \textsc{E. Weil :} Sur la formule de Siegel dans la th\'eorie des groupes classiques, {\em Acta Math.} 113 (1965), 1-87.

\bibitem{art12-key27} \textsc{E. Witt :} Eine Identit\"at zwischen Modulformen zweiten Grades, {\em Abh. Math. Seminar Hamburg} 14 (1941), 323-337.
\end{thebibliography}

\bigskip
\noindent
{\small The Johns Hopkins University}

\noindent
{\small Baltimore, Md., U.S.A.}
