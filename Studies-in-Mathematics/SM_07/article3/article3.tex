
\title{ON THE COHOMOLOGY OF DISCRETE SUBGROUPS OF SEMI-SIMPLE LIE GROUPS}
\markright{ON THE COHOMOLOGY OF DISCRETE SUBGROUPS OF SEMI-SIMPLE LIE GROUPS}

\author{By~ HOWARD GARLAND}
\markboth{HOWARD GARLAND}{ON THE COHOMOLOGY OF DISCRETE SUBGROUPS OF SEMI-SIMPLE LIE GROUPS}

\date{}
\maketitle

\setcounter{page}{21}
\setcounter{pageoriginal}{20}


\section{Introduction}\label{art3-sec1}
We\pageoriginale begin with a classical situation. Thus, let $M$ be a compact Riemannian manifold, let $C^q (M)$ denote the space of $C^\infty$ $q$-forms on $M$, and let $(\;,\;)$ denote the positive-definite inner product on $C^q (M)$, coming from the Riemannian metric $ds^2$. Let $d: C^q (M) \to C^{q+1} (M)$ denote exterior differentiation, let $\delta$ denote the adjoint of $d$ (with respect to $({\;},{\;})$) and let $\Delta$ denote the Laplacian $\Delta = d \delta + \delta d$, Let $\bH^q =$ kernel  $(\Delta : C^q (M) \to C^q (M))$ then if $H^q (M, \bR)$ denotes the $q^{\text{th}}$ cohomology group of $M$ with real coefficients, we have the Hodge-deRham Theorem:
\begin{equation}
H^1 \simeq H^q (M,\bR). \label{art3-eq1.1}
\end{equation}

At the same time, there is, after lifting to the orthonormal frame bundle, a well-known decomposition of the Laplacian (see \cite{art3-key4})
$$
\Delta = \Delta^+ + R.
$$
where $(\Delta^+ \Phi, \Phi)$, $\Phi \in C^q (M)$, is positive semi-definite, and $R$ is defined in terms of the Riemannian curvature. Thus, if $(R \Phi, \Phi)$ is positive-definite, we have that $H^q (M,\bR) = 0$, thanks to \ref{art3-eq1.1}. This idea is due to Bochner. A variant of this idea was applied by Calabi and Weil to the study of the cohomology of certain local coefficient systems, for the purpose of proving a rigidity theorem for discrete subgroups of Lie groups (see \cite{art3-key11}). Here, I wish to discuss a method for applying Bochner's idea to the study of the cohomology of discrete subgroups of $p$-adic groups. In particular, I will introduce a notion of ``$p$-adic curvature'', which plays a role in cohomology vanishing theorems for discrete subgroups of $p$-adic groups, analogous to the role played by $R$ in the argument sketched above. The details will appear in \cite{art3-key7}.

We mention that for rank 2 $p$-adic groups, the $p$-adic curvature coincides with a certain incidence matrix used by Feit, Higman, and Tits\pageoriginale to study finite simple groups (thus they proved the following \textit{rigidity theorem}: A finite $B-N$ pair of rank $\geqslant$ 3 is of Lie type!). Before giving a more extensive discussion of the $p$-adic case, we will begin with the case of discrete subgroups of real semi-simple Lie groups. This will motivate the analogy between the real and $p$-adic cases.

\section{The real case}\label{art3-sec2}
Let $G$ be a real semi-simple, linear Lie group with no compact factors, let $K \subset G$ be a maximal compact subgroup, and $\Gamma \subset G$ a discrete subgroup such that $G/\Gamma$ is compact. The space $X = K / \Gamma$ is topologically a cell. For simplicity, we assume $\Gamma$ is torsion-free. Then the action of $\Gamma$ on $X$ is free and proper, and hence $X \to X /\Gamma$ is a covering. Thus we have an isomorphism
$$
H^q (\Gamma, \bB) \simeq H^q (X/\Gamma, \bR),
$$
where $H^q (\Gamma, \bR)$ denotes the $q^{\text{th}}$ Eilenberg-MacLane group of $\Gamma$ with respect to trivial action on $\bR$. On the other hand, the space $X$ is a Riemannian symmetric space, and hence $X/\Gamma$ is a compact, Riemannian, locally symmetric space. Hence one is tempted to apply Bochner's idea, as described earlier, in order to compute the cohomology groups of $X/\Gamma$. However, in this case, the curvature form ($R \Phi$, $\Phi$) is negative, and hence Bochner's idea does not apply---at least not at first glance. Matsushima was not discouraged by this, and was inspired, in part by the computation of Calabi and Weil, to develop an ingenious modification of the Bochner idea. He then succeeded in calculating $H^q (X/\Gamma, \bR)$ in a large number of cases (see Matsushima \cite{art3-key9} and Nagano-Kaneyuki \cite{art3-key10}).

To give a rough description of Matsushima's results, we let $X_u$ denote the compact dual of $X$. Now the group $G$ acts on $X$ (to the right). We may identify $I^q$, the space of $G$-invariant differential $q$-forms on $X$, with a space of $q$-forms on $X/\Gamma$, and we do so whenever convenient. In fact, making this identification, $I^q$ is contained in the space of harmonic $q$-forms on $X/\Gamma$. On the other hand $I^q$ is isomorphic (in a natural way) to the space of all harmonic $q$-forms on $X_u$, and hence we have a natural map
$$
\varphi : H^q (X_u, \bR) \to H^q (X/ \Gamma, \bR).
$$
Matsushima\pageoriginale proved that in a large number of cases the map $\varphi$ is an isomorphism. Roughly, his idea was the following: he considered the lift of a harmonic form on $X/\Gamma$ to $G/\Gamma$, expanded the lifted form in terms of a basis of right invariant forms, and proved the derivatives with respect to right invariant vector fields, of the resulting coefficients were zero, \iec he applied the Bochner idea to the right-invariant derivatives of the coefficients. 

It is interesting to pursue matters in the real case when $\Gamma$ is arithmetic but $\Gamma/\Gamma$ is not necessarily compact. First, one should remark that whereas the map $\varphi$ is injective when $G/\Gamma$ is compact (this being a consequence of the Hodge theorem), $\varphi$ is not in general injective when $G/\Gamma$ is not compact. However, I proved that in a large number of cases $\varphi$ is surjective. Thus, if $H^2 (X_u, \bR) = 0$, we obtain a vanishing theorem for $H^2 (X/\Gamma,\bR)$, and this is how I proved the finiteness theorem for $K_2$ (see \cite{art3-key6}). Then Borel proved that in a certain (substantial) range, $\varphi$ is injective, and thus he obtained the stability theorems (see \cite{art3-key1}).

With a further view to developments in the $p$-adic case, I will conclude this section with some remarks on the proof of surjectivity. The proof of surjectivity breaks into two parts. The \textit{first part} is an adaptation of Matsushima's argument to prove that square integrable harmonic $q$-forms on $X/\Gamma$ are in fact in $I^q$ (for those cases where Matsushima's theorem holds). The \textit{second part} is a square integrability criterion: one shows that for given $X$ and for a certain range of $q$, every cohomology class in $H^q (X/\Gamma, \bR)$ has a square summable representative (see \cite{art3-key8}).

\section{The $p$-adic case}\label{art3-sec3}
Let $k_v$ be a non-archimedean completion of an algebraic number field or of a function field over a finite field. Let $\bG$ be a simply connected algebraic linear group which is defined and simple over $k_v$ (the simplicity assumption causes little trouble, thanks to a spectral sequence argument of Borel). We let $l=\rank_{k_v} \bG$, and assume $l> 0$. Let $G = \bG_{k_v}$ be the $k_v$-rational points of $\bG$. Then $G$ inherits a topology from $k_v$. We let $\bG \subset G$ be a discrete subgroup and for\pageoriginale simplicity we assume $\Gamma$ is torsion-free. For the time being, we also assume the coset space $\Gamma/G$ is compact.

Bruhat and Tits have associated a certain simplicial complex $\fI$ with $G$ (see \cite{art3-key2} and \cite{art3-key3}). Without giving an exact description of the Bruhat-Tits complex, let us mention some of its properties:
\begin{align*}
\text{(i)} &\quad \fI \text{ is contractible.} \hspace{3cm}\\
\text{(ii)} &\quad G \text{ acts (to the left, say) simplicially on } \fI.  \hspace{3cm}\\
\text{(iii)} &\quad \text{ The action of } G \text{ on } \fI \text{is proper}.  \hspace{3cm}\\
\text{(iv)} &\quad \text{ The action of $\Gamma$ on $\fI$ is free.}  \hspace{3cm}\\
\text{(v)} &\quad \text{ $\Gamma/\fI$ is a finite complex.}  \hspace{3cm}\\
\text{(vi)} &\quad \text{ $\dim \fI = l$.}  \hspace{3cm} \tag{3.1}
\end{align*}

We remark that (i) is a theorem of Solomon and Tits, that (iv) is a consequence of our assumption that $\Gamma$ is torsion-free, that (ii), (iii). (v) and (vi) are immediate consequences of the definition of $\bI$. Also. though $\Gamma/\fI$ may not be simplicial, we assume for simplicity that it is. By (i), (iii), and (iv), we have
$$
H^q (\Gamma, \bR) \simeq H^q (\Gamma/\fI, \bR).
$$
Serre was the one who observed this isomorphism and of course he also noted the immediate consequence that 
\begin{equation}
H^q (\Gamma, \bR) = 0, \;\; q>l. \label{art3-eq3.2}
\end{equation}

On the other hand, from an entirely different viewpoint, Kazhdan proved that
\begin{equation}
H^1 (\Gamma. \bR) = \text{ 0, whenever $l \geqslant 2$,}
\label{art3-eq3.3}
\end{equation}
(see \cite{art3-key5}).

From \eqref{art3-eq3.2}, \eqref{art3-eq3.3} and his computation of the Euler characteristic of $\Gamma$, Serre was led to conjecture that 
$$
H^q (\Gamma, \bR) = 0, \quad 0 < q < l.
$$

In \cite{art3-key7}, we have proved

\begin{theorem}\label{art3-thm3.4}
Given $l$, there exists an integer $N(l)$ such that if the cardinality of the residue class field of $k_v$ is at least $N(l)$ , if $\bG$ is a simply connected algebraic linear group which is defined and simple over $k_v$, and\pageoriginale such that $\rank_{k_v} \bG = l$, if $G = \bG_{k_v}$, and if $\Gamma \subset G$ is a discrete subgroup such that $\Gamma/ \Gamma $ is compact, then $H^q(\Gamma, \bR) =0$, $0 < q < l$; \iec Serre $s$ conjecture holds for $\Gamma$.
\end{theorem}

\begin{remark*}
In Theorem \ref{art3-thm3.4} we do not assume $\Gamma$ is torsion-free. In \S \ref{art3-sec4} we shall give an indication of the proof. For the present we mention that our proof bears an analogy to the real case, and that in particular, we introduce a notion of $p$-adic curvature. Our argument then rests on certain estimates for the eigenvalues of this $p$-adic curvature. It seems likely that our restriction on the cardinality of the residue class field is unnecessary and only results from the fact that our curvature estimates are not sharp.
\end{remark*}

\section{Square summable cohomology and an indication of proofs}\label{art3-sec4}
We shall axiomatize our proof. We continue to use the notation of \S \ref{art3-sec3}. In \S\ref{art3-sec3} we mentioned a certain complex $\Gamma/\fI$, which for simplicity, we assumed to be simplicial. Here we drop the assumption that $\Gamma/G$ is compact or correspondingly that $\Gamma/\fI$ is a finite complex.

It turns our that we can describe a portion of our arguments in the following general setting: Let $\fS$ be a locally finite simplicial complex of dimension $l$ and assume we are given a strictly positive valued function $\lambda$ on the simplies of $\fS$ ($\lambda(\sigma) > 0$, for every simplex $\sigma$ of $\fS$). We shall refer to $\lambda$ as a Riemannian metric. Furthermore, let $v (\fS)$ denote the vertices of $\fS$, and assume we are given a partition
\setcounter{equation}{0}
\begin{equation}
v (\fS) = \Lambda_1 \cup \ldots \cup \Lambda_{l+1} \label{art3-eq4.1}
\end{equation}
of $v (\fS)$ into $l+1$ mutually disjoint subsets, $\Lambda_1, \ldots \Lambda_{l+1}$, such that no two vertices in the same $\Lambda_i$ ever span a one-simplex.

Let $C^q (\fS)$ denote the real valued, oriented $q$-cochains on $\fS$, and let 
$$
d : C^q (\fS) \to C^{q+1} (\fS)
$$
denote the simplicial coboundary. Let
$$
C (\fS) = \bigoplus_{q \geqslant 0} C^q (\fS)\;\; (\text{direct sum}).
$$
Then if $\Phi$, $\Psi \in C (\fS)$ we set
$$
(\Phi, \Psi) = \sum\limits_{\sigma = \text{simplex of $\fS$}}  \lambda (\sigma) \Phi (\sigma) \Psi (\sigma),
$$
whenever\pageoriginale  the sum on the right converges absolutely. Since $\Phi$, $\Psi$ are defined on oriented simplices, we have to say just how one defines the right-hand side. For each geometric simplex $\sigma$, pick an oriented representative $\hat{\sigma}$. Then for such $\sigma$, set $\Phi(\sigma) = \Phi(\hat{\sigma})$ and $\Psi(\sigma) = \Psi (\hat{\sigma})$. Then the products $\Phi(\sigma) \Psi (\sigma)$ do not depend on the choices we made for oriented representatives, and so the right-hand sum is also independent of these choices.

We let $L^q = L^q (\fG)$ consist of all $\Phi\in C^q (\fG)$ such that $(\Phi, \Phi) < \infty$. Then $L^q(\fS)$ is a subspace of $C^q (\fS)$, and is a Hilbert space with respect to the inner product (,). When $\fS$ is a finite complex, we have of course that $L^q (\fG) = C^q (\fG)$ is a finite dimensional vector space.

We now \textit{assume}
\begin{gather}
d(L^q (\fG)) \subset L^{q+1} (\fG)\notag\\
d: L^q (\fG) \to L^{q+1} (\fG) \text{ is bounded.} \label{art3-eq4.2}
\end{gather}

\begin{remark*}
Borel has pointed out that if $\fG = \Gamma / \fI$, then \eqref{art3-eq4.2} holds. Let $\delta$ be the adjoint of $d$. Thus
\begin{equation}
(d \Phi, \Psi) = (\Psi , \delta \Psi), \; \Phi \in L^q (\fS), \; \Psi \in L^{q+1} (\fS). \label{art3-eq4.3}
\end{equation}

We set
$$
\Delta^+ = \delta d, \quad \Delta = \delta d + d \delta, 
$$
and we let
$$
\bH^q = (\text{kernel } (\Delta)) \cap L^q (\fS).
$$

The elements of $\bH^q$ are called harmonic ($q-$) cocycles. From \eqref{art3-eq4.3} we have that $\Phi \in \bH^q$ if and only if $d \Phi = \delta \Phi =0$. It then follows that we have an orthogonal direct sum decomposition
\begin{equation}
L^q (\fS) = \bH^q \oplus \overline{\im} \; d \oplus \overline{\im} \; \delta. \label{art3-eq4.4}
\end{equation}

(Here ``----'' denotes closure). When $\fS$ is finite, \eqref{art3-eq4.4} implies 
\begin{equation}
\bH^q = H^q (\fS, \bR). \label{art3-eq4.5}
\end{equation}

We have obtained \eqref{art3-eq4.4} and \eqref{art3-eq4.5} from our Riemannian metric $\lambda$. We now use the partition \eqref{art3-eq4.1} to develop an analogue of Bochner's idea. Thus, for each $\alpha \in \{1, \ldots, l +1\}$, we define
$$
\rho_{\alpha} (\Phi) (\tau) = 
\begin{cases}
\Phi (\tau), \text{ $\tau$ has a vertex in $\Lambda_\alpha$}\\
0, \text{ otherwise,}
\end{cases}
$$\pageoriginale
where of course $\Phi \in C^q (\fS)$ and $\tau$ is an oriented $q$-simplex. We note that $P_\alpha (L^q (\fS)) \subset L^q (\fS)$ and that $\rho_\alpha : L^q (\fS) \to L^q (\fS)$ is a projection; \iec $\rho^2_\alpha = \rho_\alpha$ and $\rho_{\alpha}$ is self adjoint. \textit{ From now on}, we regard $\rho_\alpha$ as an operator on $L^q(\fS)$. We set
\begin{align*}
d_{\alpha} & = \rho_\alpha \circ d \circ \rho_{\alpha}\\
\delta_{\alpha} & = \rho_{\alpha} \circ \delta \circ \rho_{\alpha}, \; \alpha = 1, \ldots, l + 1. 
\end{align*}

Then $d_{\alpha}$ and $\Delta_{\alpha}$ are adjoints of one another and 
\begin{equation}
d_{\alpha} = d \circ \rho_{\alpha}, \; \delta_{\alpha} = \rho_{\alpha} \circ \delta. \label{art3-eq4.6}
\end{equation}

We set $\Delta^+_\alpha = \delta_\alpha d_\alpha$, $d'_\alpha = d - d_\alpha$. Let $\Phi \in \bH^q$; then a simple calculation shows that for all $\alpha$
\begin{equation}
0 = (\delta d \Phi, \Phi) = (\Delta^+_\alpha \Phi, \Phi) - (d'_\alpha \Phi, d'_\alpha \Phi). \label{art3-eq4.7}
\end{equation}
It turns out that when $\fS = \Gamma / \fI$, the following also holds (for $q > 0$)\footnote{Borel has observed (see \cite{art3-key27}) that for any finite simplicial complex $\fS$, that if each $\wedge_i$ is a single point (our assumption that $1+ \dim \fS = $ number of sets in the partition is unnecessary) and if $\lambda (\sigma) =$ number of simplices of maximum dimension, with $\sigma$ as a face, then \eqref{art3-eq4.8} holds.}
\begin{equation}
d'_\alpha \Phi, d'_\alpha \Phi = ((1-\rho_\alpha) \Phi, \Phi). \label{art3-eq4.8}
\end{equation}

On the other hand, the operator $\Delta^+_\alpha$ plays the role of a curvature tensor, and has a nice geometric interpretation. Thus we might ask: If $\Delta^+_\alpha$ is an analogue of a tensor field, then what is $\Delta^+_\alpha$ ``at a point''; \iec at a vertex $v \in \Lambda_\alpha$, say?

To answer this question, let $\Sigma_v$ be the  boundary of the star of $v$. We define a Riemannian metric $\lambda_v$ on $\Sigma_v$ as follows: If $\sigma$ is a $q-1$ simplex $(q > 0)$ in $\Sigma_v$, we set
$$
\lambda_v = (\sigma) = \lambda (v \cdot \sigma).
$$
where $v \cdot \sigma$ denotes the join of $v$ and $\sigma$. We let $d_v$ denote the simplicial coboundary on $\Sigma_v$, and we let $\delta_v$ denote the adjoint of $d_v$, relative to the inner product on cochains defined by $\lambda_v$. Then $\Delta^+_\alpha$ on $C^q (\fS)$, $q>0$, may be regarded, in a precise manner, as a field of operators assigned to the vertices of $\Lambda_\alpha$; namely, to each vertex $v$ of $\Lambda_\alpha$, we assign\pageoriginale the operator $\delta_v d_v$ acting on $q-1$ cochains of $\Sigma_v$. In order to make a more precise (but slightly different) statement, let $(\;,\;)_v$ denote the inner product on cochains defined by $\lambda_v$. Also, if $\Phi \in C^q (\fS)$, $q>0$, and $v \in \Lambda_\alpha$, we let ${}_v \Phi \in C^{q-1}$ be defined by ${}_v \Phi (\Sigma) = \Phi (v \cdot \sigma)$.
We then have 
\begin{equation}
(\Delta^+_\alpha \Phi, \Phi) = \sum\limits_{v \in \Lambda_\alpha} (\delta_v d_v (_v \Phi), \; {}_v \Phi)_v.
\label{art3-eq4.9}
\end{equation}

Moreover, if $\Phi \in \bH^q$, then $\delta \Phi=0$. Thus the second formula of \eqref{art3-eq4.6} implies $\delta_\alpha \Phi = 0$. This, it turns out, implies that if $v \in \Lambda_\alpha$ and if $\Sigma_v$ is $q -1$ connected, then the cochain ${}_v\Phi$ is in the positive eigenspace of $\delta_v d_v$! Thus, if $\kappa_v$ is the minimal positive eigenvalue of $\delta_v d_v$ on $C^{q-1} (\Sigma_v)$, and if 
$$
\kappa^\alpha_q = \min\limits_{v \in \Lambda_\alpha} \kappa_v, \;\; \kappa_q = \min\limits_\alpha \kappa^\alpha_q,
$$
then \eqref{art3-eq4.9}, \eqref{art3-eq4.8} and \eqref{art3-eq4.7}, and our assumption that no two vertices of $\Lambda_\alpha$ span a one-simplex, yield
\begin{equation}
0 > (\kappa_q + 1) \; (\rho_\alpha \Phi, \; \rho_\alpha \Phi) - (\Phi, \Phi). \label{art3-eq4.10}
\end{equation}
Summing these inequalities over all $\alpha$ we obtain
\end{remark*}

%\setcounter{equation}{10}
\begin{theorem}\label{art3-thm4.11}
Let $q > 0$. Then the inequality
\begin{equation}
\kappa_q > (l-q) (q+1)^{-1} \label{art3-eq4.12}
\end{equation}
implies $\bH^q =0$. In particular, if $\bS$ is finite then \eqref{art3-eq4.12} implies $H^q (\fS , \bR) = 0$.
\end{theorem}

\begin{remark*}
Thanks to \eqref{art3-eq4.5}, the first assertion of the theorem implies the second. 
\end{remark*}

\section{Some further comments on the non-compact case}\label{art3-sec5}
We continue with the notation of \S\S\ref{art3-sec3} and \ref{art3-sec4}.

\begin{defi*}
We will say $\Delta^+$ is $W$-elliptic in dimension $q$, in case there exists $a  c > 0$ such that 
$$
(\Delta^+ \Phi, \Phi) \geqslant c (\Phi, \Phi),
$$
for all $\Phi \in ({\rm kernel}~  \delta) \cap L^q$.
\end{defi*}

A direct argument then shows:

\begin{prop*}
If $\Delta^+$ is\pageoriginale $W$-elliptic in dimension $q$, then $d : Lq \to L^{q+1}$ is closed and $\bH_q = \{0\}$.
\end{prop*}

But in fact, estimates of the type described in \S4 allow us to conclude that for certain $\Gamma/ \fI$ and for certain $q$, it is true that $\Delta^+$ is $W$-elliptic in dimension $q$. It follows that for $\Gamma/\fI$ and for certain $q$, one can prove that every cocycle in $L_q (\Gamma/\fI)$ is cohomologous to zero. In particular, consider the case when $\Gamma \subset G$ is discrete and $\Gamma / G$ has finite invariant volume. Then, since for $\Gamma / \fI$ we have now obtained one analogue of part one of the argument described at the end of \S\ref{art3-sec2}, we are led to ask about part two. That is, when is it true that every cocycle in $C^q (\Gamma/ \fI)$ is cohomologous to a square-summable cocycle? Since we are assuming $\Gamma/ G$ has finite invariant volume, this question is only interesting in the function field case. The feeling then is that when $\Gamma$ is arithmetic, the answer is always yes. If so, one could obtain the function field analogues of the results described in \S\ref{art3-sec2} for arithmetic groups in the number field case, provided, one could circumvent the problem that in the function field case $\Gamma$ need not have a torsion-free subgroup of finite index.

\begin{thebibliography}{99}
\bibitem{art3-key1} \textsc{A. Borel:} Cohomologie r\'eelle stable de groupes $S$-arithm\'etiques classiques. \textit{C. R. Acad. sc. Paris} 274 (1972), 1700-1702.

\bibitem{art3-key2} \textsc{F. Bruhat} and \textsc{J. Tits}: Groupes alg\'ebriques, simples sur un corps local, \textit{Proc. Conf. Local Fields,} Springer-Verlag, (1967).

\bibitem{art3-key3} \textsc{F. Bruhat} and \textsc{J. Tits:} ``Groups r\'eductifs sur un corps local'', \text{Publ. Math. I. H. E. S.} 41 (1972), 5-252.

\bibitem{art3-key4} \textsc{S. S. Chern}: On a generalization of K\"ahler geometry, \textit{Algebraic Geometry and Topology,} A Symposium in honor of S. Lefschetz, Edited by R. H. Fox et al, Princeton Univ. Press.

\bibitem{art3-key5} \textsc{C. Delaroche} and A. Kirillov: Sur les relations entre l'espace dual d'un grupe et la structure de ses sousgroups ferm\'es (d'apr\`es D.A. Kajdan), \textit{S\'em, Bourbaki} 1967/68, expos\'e 343, Benjamin, New York, (1969).

\bibitem{art3-key6} \textsc{H. Garland}: A finiteness\pageoriginale theorem for $K_2$ of a number field, \textit{Ann. of Math.,} 94 (1971), 534-548 .

\bibitem{art3-key7} \textsc{H. Garland}: $p$-adic curvature and the cohomology of discrete subgroups of $p$-adic grups, \textit{Ann. of Math.,} 97 (1973), 375-423.

\bibitem{art3-key8} \textsc{H. Garland} and \textsc{W. C. Hsiang}: A square integrability criterion for the cohomology of arithmetic groups, \textit{Proc. Nat. Acad. Sci. U. S. A.,} 59 (1968), 354-360.

\bibitem{art3-key9} \textsc{Y. Matsushima}: On Betti numbers of compact, locally symmetric Riemannian manifolds, \textit{Osaka Math. J.,} 14(1962), 1-20.

\bibitem{art3-key10} \textsc{T. Nagan} and \textsc{S. Kaneyuki}: ``Quadratic forms related to symmetric Riemannian spaces'', \textit{Osaka Math. J.,} 14(1962), 241-252.

\bibitem{art3-key11} \textsc{A. Weil:} On discrete subgroups of Lie groups, II, \textit{Ann. of Math.,} 75 (1962), 578-602.

\bibitem{art3-key12} \textsc{A. Borel:} Cohomologie de certains groupes discrets et Laplacien $p$-adique, S\'em. Bourbaki, 1973/74 Expos\'e No. 437.
\end{thebibliography}





