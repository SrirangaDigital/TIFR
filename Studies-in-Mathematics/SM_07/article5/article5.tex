
\title{ON THE COHOMOLOGY OF DISCRETE ARITHMETICALLY DEFINED GROUPS\footnote{Supported by the ``Sonderforschungsbereich f\"ur theoretische Mathematik an der Universit\"at Bonn'' and NSF grant GP-36418X.}}
\markright{ON THE COHOMOLOGY OF DISCRETE ARITHMETICALLY DEFINED GROUPS}

\author{By~ G. HARDER} 
\markboth{G. HARDER}{ON THE COHOMOLOGY OF DISCRETE ARITHMETICALLY DEFINED GROUPS}

\date{}
\maketitle


%\setcounter{page}{21}
\setcounter{pageoriginal}{128}

\section*{Introduction}
In this paper\pageoriginale I want to come back to the questions which I discussed in \cite{art5-key7}. These questions arise from the study of the cohomology of discrete arithmetically defined groups $\Gamma$. To investigate the cohomology of $\Gamma$ one considers the action of $\Gamma$ on the corresponding symmetric space $X$ and makes use of the fact that $H^\bigdot (X / \Gamma \bR) = H^\bigdot (\Gamma, \bR)$. If this quotient $X / \Gamma$ is compact then the Hodge theory (comp. \cite{art5-key12} \S 31) is a powerful tool for the investigation of these cohomology groups. But in general the quotient $X/ \Gamma$ is not compact and my concern in this paper are those phenomena which are due to this noncompactness. The Hodge theory fails in this case and I want to find a substitute for it or in other words I want to get control of the deviation from Hodge theory. The basic idea is to make use of Langland's theory of Eisenstein series (comp. \cite{art5-key8}) to describe the cohomology at ``infinity''.

In this paper I mainly consider the case that $\Gamma$ is of rank one, i.e. the semi-simple group $G/ K$ which defines $\Gamma$ is of rank one over the algebraic number field $k$. The main result is Theorem 4.6. This theorem is to some extent a generalization of Theorem 2.1 in \cite{art5-key7} which is stated there without proof. On the other hand the results in \cite{art5-key7} are still much more precise, because in the case of $G = S L_2 / k$ the intertwining operators $c (s)$ (comp. \S \ref{art5-sec3}) are accessible for explicit computations. I hope to come back to these problems later.

\section{Basic notions and results on the cohomology of $\Gamma$.}\label{art5-sec1}

Let $G_\infty$ be a real Lie group acting transitively from the right on a contractible $C^\infty$-manifold $X$. Let $x_0 \in X$ and let us assume that the stabilizer of $x_0$ is a compact subgroup $K$. Therefore we have $K / G_\infty = X$. 

Let\pageoriginale $\Gamma \subset G_\infty$ be a discrete subgroup without torsion, let us assume that $X/\Gamma$ is a manifold; then $\Gamma$ is the fundamental group of $X/ \Gamma$ and the higher homotopy groups of $X / \Gamma$ are trivial, \ie, $X/ \Gamma$ is a $K (\Gamma , 1)$-space.

Let $E$ be a finite dimensional vector space over the complex numbers of let $\rho_\infty: G_\infty \to G L (E)$ be a representation of $G_\infty$. We define an action of $G_\infty$ on the functions $f : X \to E$ by
$$
T_g (f) (x) = \rho_\infty (g^{-1}) (f (xy)) \text{ for } x \in X \text{ and } g \in G_\infty.
$$

We restrict the action of $G_\infty$ on $X$ and $E$ to the subgroup $\Gamma$, then we can construct the induced bundle $\tilde{E}_{\rho_\infty} \to X / \Gamma$ and it follows from the definition of the induced bundles that the $C^\infty$-sections of $\tilde{E}_{\rho_\infty} \to X / \Gamma$ are precisely the $\Gamma$-invariant $C^\infty$-functions $f: X to E$. The pull back of the bundle $\tilde{E}_{\rho_\infty}$ to $X$ is the trivial bundle $E \tilde X$ over $X$ and the trivial connection on $E \times X$ induces a flat connection $\tilde{\nabla}$ on $\tilde{E}_{\rho_\infty}$, which will be called the canonical connection on $\tilde{E}_{\rho_\infty}$. By means of this connection we define the sheaf of locally constant sections in $\tilde{E}_{\rho_\infty}$, and the cohomology with coefficients in this sheaf will be denoted by
$$
H^\bigdot (X/ \Gamma, \tilde{E}_{\rho_\infty}) = H^\bigdot (\Gamma, \rho_\infty, E)
$$

These cohomology groups can also be defined via the de Rham-cohomology: Let $\Omega^\rho (X/ \Gamma , E)$ be the vector space of $C^\infty$-$p$-differential forms on $X/\Gamma$ with values in $\tilde{E}_{\rho_\infty}$. Making use of the canonical connection we define the exterior derivative $d: \Omega^p (X/ \Gamma, E) \to \Omega^{p+1} (X/ \Gamma, E)$ which is given by the following formula : If $P_1, P_2,\ldots P_{p+1}$ are $C^\infty$-vector fields on $X/ \Gamma$ and if $\omega \in \Omega^p (X / \Gamma, E)$ then 
\begin{multline}
d\omega (P_1 ,\bigdot, P_{p+1}) = \sum\limits^{i = p+1}_{i=1} (-1)^{i+1} \tilde{\nabla}_{P_i} \omega (P_1, \bigdot, \hat{P}_i, \bigdot, P_{p+1}) + \\
+ \sum\limits_{1 \leqslant i < k \leqslant p+1} (-1)^{i+j} \omega ([P_i, P_j], P_1, \ldots, \hat{P}_i, \bigdot, \hat{P}_j, \bigdot, \hat{P}_j, \bigdot, P_{p+1}) \label{art5-eq1.1}
\end{multline}

Since $\tilde{\nabla}$ is flat we have $d^2 = 0$ and it is well known that the cohomology groups of the de Rham complex $(\Omega \bigdot (X/ \Gamma, E), d)$ are canonically isomorphic to the cohomology groups $H^\bigdot (X/ {}, \tilde{E}'_{\rho_\infty})$ (comp. \cite{art5-key10}, \S 1 Prop. 1.1).

We may\pageoriginale look at our bundle $\tilde{E}_{\rho_\infty}$ from a different point of view. We consider the fibration
$$
q : G_\infty/ \Gamma \to X / \Gamma
$$
which is principal with structure group $K$. Therefore the restriction of $\rho_\infty$ to $K$ defines another induced bundle $E_{\rho_\infty}$ whose global $C^{\infty}$-sections are given by the $C^\infty$-functions
$$
f: G_\infty / \Gamma \to E
$$
which satisfy
$$
f (kg) = \rho_\infty (k) (f(g)) \text{ for all } g \in G_\infty, k \in K.
$$

%\setcounter{lemma}{1}
\begin{lemma}\label{art5-lem1.2}
For any $C^\infty$-section $f$ of $E_{\rho_\infty}$ we define a section $\tilde{f}$ of $\tilde{E}_{\rho_\infty}$ by
$$
\tilde{f} (x) = \tilde{f} (x_0 g) = \rho_\infty (g^{-1}) f(x_0) x = x_0 g
$$
The the mapping $f \to \tilde{f}$ induces an isomorphism from $E_{\rho_\infty}$ to $\tilde{E}_{\rho_\infty}$
\end{lemma}

The proof is either trivial or can be found in \cite{art5-key10}, Prop. 3.1.

By means of this identification we can define a connection $\nabla$ on $E_{\rho_\infty}$ and we shall derive a formula for this connection in terms of the bundle $E_{\rho_\infty}$ itself.

Let $\fG$ be the Lie algebra of right invariant vector fields on $G_\infty$. Let $\fK$ be the Lie algebra of $K$ and we put $\fp = \fG/\fK$. Let $T^G$ (resp. $T^X$) be the tangent bundle of $G_\infty /\Gamma$ (resp. $X/ \Gamma$). The projection $q$ defines at any point $g \in G$ a surjection
$$
D_{g, q} : T^G_g  = \fG \longrightarrow T^X_x, \; x  = x_0 g
$$
and this yields isomorphisms
$$
\bar{D}_{g, q} : \fp \tilde{\longrightarrow} T^X_x.
$$
This shows that $q^\ast (T^x) = G_\infty / \Gamma \times \fp$ and that the $C^\infty$-vector fields on $X/ \Gamma$ can be identified with the $C^\infty$-functions
$$
P : G_\infty / \Gamma \to \fp
$$
which satisfy
%\setcounter{equation}{2}
\begin{equation}
P(kg) = ad_\fp (k) P(g) \quad g \in G_\infty, k \in K, 
\label{art5-eq1.3}
\end{equation}
\ie, the tangent bundle of $X/ \Gamma$ is the bundle induced by adjoint representation $ad_\fp$ of $K$ on $\fp$.

Now we\pageoriginale can give our formula for the connection $\nabla$.

%\setcounter{lemma}{3}
\begin{lemma}\label{art5-lem1.4}
Let $f: G_\infty/ \Gamma \to E$ be $C^\infty$-section of $E_{\rho_\infty}$ and let $P: Kg \to \fp$ be a tangent vector at $x = x_0$ $g = Kg$. Then
$$
\nabla_{P(g)} (f) |_g = \tilde{P} (g) f |_g - \rho_\infty (\tilde{P} (g)) (f(g))
$$
where $\tilde{P}(g)$ is a representative for $P(g)$ in $\fG$, and where $\tilde{P}(g) f|_g$ is the ordinary derivative of the function $f$ with respect to the tangent vector $\tilde{P} (g)$ at $g$.
\end{lemma}

This lemma follows from direct calculations and it is also essentially stated in \cite{art5-key10}, Prop. 4.1.

\begin{remark*}
The bundles $E_{\rho_\infty}$ and $\tilde{E}_{\rho_\infty}$ are only two different realizations of the same thing. We need $\tilde{E}_{\rho_\infty} \to X / \Gamma$ for the definition of the connections $\tilde{\nabla}$ and $\nabla$, but from now on we shall work with the realization $E_{\rho_\infty} \to X / \Gamma$.
\end{remark*}

Let us now assume that $G_\infty$ is the group of real points of a reductive algebraic group over $\bR$. Then the Lie algebra splits 
$$
\fG' = {}^\bigdot \fB\oplus \fG
$$
where $\fB$ is the centre of $\fG'$ and where $\fG$ is the semisimple part of $\fG$. Let us assume moreover that $K \subset G_\infty$ is a maximal compact sub-group, then $X = K/ G_\infty$ is homeomorphic to an euclidean space. Let 
$$
B : \fG \times \fG \longrightarrow \bR
$$
be a symmetric nondegenerate bilinear form on $\fG$ which has the following properties
\begin{itemize}
\item[(1)] $B$ is invariant under the adjoint action of $G_\infty$ on $\fG$.

\item[(2)] The restriction of $B$ to $\fK$ is negative definite and if $\fp$ is the orthogonal complement of $\fK$ with respect to $B$, then $B / \fp$ is positive definite.
\end{itemize}

The condition (1) implies that $\fG = \fB \oplus \fG'$ is orthogonal with respect to $B$.

By means of the identification $\bar{D}_{e,q} : \fp \to T^X_{x_0}$ we define as usual a Riemannian metric on $X$.

We\pageoriginale assume now that $\rho_\infty$ is the complexification of a real representation $\rho_0 : G \to G L (E_0)$. Following Matsushima and Murakami in \cite{art5-key10} we call an euclidean bilinear form
$$
\langle, \rangle : E_0 \times E_0 \to \bR
$$
admissible if 
\begin{itemize}
\item[(1)] $\langle, \rangle$ is invariant under $\rho (K)$, 

\item[(2)] $\langle \rho_\infty (P) e, f \rangle = \langle e, \rho_\infty (P) f \rangle$ for all $P \in \fp$, e. $f \in E_0$.
\end{itemize}
We extend $\langle , \rangle$ to a hermitian form on $E = E_0 \times \bC$. The existence of such an admissible metric on $E_0$ is proved in \cite{art5-key10}, Lemma 3.1, if $K$ is connected. In the general case we have to average over the different connected components of $K$.


let us fix an admissible metric on $E$. It induces a $\bC$-linear isomorphism
$$
\# : E \tilde{\longrightarrow} E^\ast = \Hom_\bC (E, \bC)
$$
which is defined by $(\# a) (b) = \langle b, \bar{a} \rangle$ for all $a, b \in E$. Because of condition (1) this extends to an isomorphism $\# : E_{\rho_\infty} \tilde{\longrightarrow} E^\ast_{\rho_\infty}$ This yields a map
%\setcounter{equation}{4}
\begin{equation}
\# : \Omega^p (X/ \Gamma, E) \tilde{\longrightarrow} \Omega^p (X/ \Gamma, E^\ast) , \label{art5-eq1.5}
\end{equation}
If $\dim X =N$ and if $\Gamma$ preserves an orientation of $X$ then we define the operator 
\begin{equation}
\ast : \Omega^p (X / \Gamma, X) \longrightarrow \Omega^{N-p} (X/ \Gamma, E) \label{art5-eq1.6}
\end{equation}
(comp. \cite{art5-key10}, \S \ref{art5-sec2}) as usual and it is easy to see that $\ast$ and $\#$ commute.

If $\omega \in \Omega^p (X / \Gamma , E)$ and $\omega' \in \Omega^{N-p} (X/ \Gamma, E^\ast)$ then $\omega \wedge \omega'$ takes its values in $E \otimes E^\ast$ and the evaluation $\map tr : E \otimes E^\ast \to \bC$ brings us back to the complex numbers. We define
\begin{equation}
(\omega, \omega') = tr (\omega \wedge \omega') \in \Omega^N (X/ \Gamma, \bC)
\label{art5-eq1.7}
\end{equation}
We now define as usual a hermitian inner product on the $C^\infty$-$p$-forms with compact support
$$
\langle \omega, \omega' \rangle = \int\limits_{X/ \Gamma} (\omega, \overline{\ast \circ \# \omega'})
$$
The\pageoriginale operator $\delta: \Omega^{p+1} (X/ \Gamma, E) \to \Omega^p (X/ \Gamma, E)$  is defined as the adjoint of $d$ with respect to this metric. We know from \cite{art5-key10}, \S \ref{art5-sec2} that this operator can also be defined by 
$$
\delta = (-1)^{p+1} \ast {}^{-1} \circ \# {}^{-1} d \# \circ \ast .
$$
We define that Laplacian
$$
\Delta = d \delta + \delta d.
$$

This Laplacian operator does not depend on the choice of an orientation on $X$ and therefore we can define $\Delta$ also if $\Gamma$ does not preserve the orientation.

If the quotient $X / \Gamma$ is compact then one knows that a harmonic $p$-form $\omega$, \ie, a form that satisfies $\Delta \omega = 0$, is closed and coclosed, \ie, it satisfies $d \omega = 0$ and $\delta \omega = 0$. Therefore we get a mapping from the space of harmonic forms $\bH^p (x/ \Gamma , E)$ to the cohomology
$$
\bH^p (X/ \Gamma, E) = \{\omega \in \Omega^p (X/ \Gamma, E) | \Delta_\omega = 0 \} \longrightarrow H^p (X/ \Gamma, E)
$$
and a straightforward generalization of the Hodge theory (comp. \cite{art5-key12}, \S 31, Cor. 3) tells us that this map is an isomorphism.

To conclude this section we shall explain the relation of $\Delta$ to the Casimir operator $C$. The bundle $E_{\rho_\infty}$ is the induced bundle of the representation $\rho_\infty|K$, and we have seen that the tangent bundle $T^X$ is the bundle induced by $ad_\fp$. Therefore we have a natural identification between the space $\Omega^p (X \Gamma, E)$ and the functions
$$
\varphi : G_\infty / \Gamma \to \Hom (\Lambda^p_\fp, E)
$$
which satisfy
\begin{equation}
\varphi (kg) = \Lambda^p ad_\fp^\ast (k) \otimes \rho_\infty (k) (\varphi (g)) \quad g \in G_\infty, \; k \in K 
\label{art5-eq1.8}
\end{equation}
where $ad^\ast_\fp$ is the dual representation to $ad_\fp$.

The enveloping algebra $U (\fG)$ is operating as an algebra of differential operators on the $C^\infty$-functions on $G_\infty/\Gamma$ with values in $\Hom (\wedge^p_\fp, E)$ (Comp. \cite{art5-key8}, Chap. I, \S \ref{art5-sec2}). Our metric on $\fG$ defines the Casimir operator $C \in \fB (\fG) =$ centre of $U (\fG)$, and this is an operator of order 2which sends the functions $\varphi$ satisfying \eqref{art5-eq1.8} into itself. We can extend the representation $\rho_\infty$ of $\fG$ in $\End (E, E)$ to a representation\pageoriginale $\rho_\infty$ of $U(\fG)$ in $\End (E, E)$. Then we have for all $C^\infty$-$p$-forms $\varphi: G_\infty/ \Gamma \to \Hom (\Lambda^P_\fp, E)$ the formula
\begin{equation}
\Delta_\varphi = - C_\varphi  + \rho_\infty (C)_\varphi. 
\label{art5-eq1.9}
\end{equation}
This is the lemma of Kuga and it is proved in \cite{art5-key10} \S 6 for semisimple groups. The generalization to the reductive case is easy.

\section{The cohomology of `parabolic' discrete groups.}\label{art5-sec2}
Let $k$ be an algebraic number field and let $G/ k$ be a semisimple algebraic group. We define $k_\infty = k \otimes_{\bQ} \bR = \bR^{r_1} \oplus \bC^{r_2}$ and we put $G_\infty = G_{k_\infty}$. Let $\Gamma \subset G_k$ be an arithmetically defined subgroup without torsoin. Let $\rho_k : G \to G L(V)$ be a rational representation of $G/ k$. From this we get a representation $\rho_\infty$ of $G_\infty$ on the vector space $E_0 = V \otimes_{\bQ} \bR$. Let us fix an admissible metric on $E_0$ with respect to a fixed maximal compact subgroup $K \subset G_\infty$. Again we put $E = E_0 \times \bC$. This is the situation to which we shall apply the results of \S \ref{art5-sec1}.

Let $P \subset G/k$ be a parabolic subgroup. In this section we shall investigate the cohomology groups 
$$
H^\bigdot (\Gamma \cap P_k, \rho'_\infty, E)
$$
where $\Gamma \cap P_k = \Gamma \cap P_\infty \subset P_\infty$ is the ``parabolic'' discrete group in the heading of this section.

Let us assume that $G/k$ is simply connected. Then the character module $X (P) =\Hom (P, G_m)$ is generated by fundamental dominant weights $\chi_1, \ldots, \chi_r$. These characters $\chi_i$ induce homomorphisms
\begin{align*}
& |\chi_i| : P_\infty \longrightarrow (\bR^+)^\ast \\
& |\chi_i| : p \longrightarrow N_{k_\infty/\bR} (\chi_i (p)) |
\end{align*}
The intersection of the kernels of these homomorphisms $|\chi_i|$ is called $P_\infty(1)$, and it is well known that $P_\infty (1) \cap \Gamma = P_\infty \cap \Gamma$. We put $X = K / G_\infty$ where $K$ is our fixed maximal compact subgroup and $x_0 = K e \in X$. We define
$$
X(1) = X_P (1) = x_0 P_\infty (1).
$$
This is a $s$-codimensional subspace of $X$ which depends on $P$. We drop the index $P$ if it is clear with respect to which $P$ this subspace is defined. Since $X (1)$ is contractible we have 
$$
H^\bigdot (X(1)/ \Gamma \cap P_\infty, E) = H^\bigdot (X/ \Gamma \cap P_\infty, E) = H^\bigdot (\Gamma \cap P_\infty, \rho'_\infty, E)
$$\pageoriginale
where $\rho'_\infty$ is of course the restriction of $\rho_\infty$ to $P_\infty$.

Let us denote the unipotent radical of $P$ by $U$. Then $M = P/U$ is reductive algebraic group over $k$, and $M_\infty$ (1) is the group of real points of an algebraic subgroup of $M \times {}_\bQ \bR$. Let $K_M$ be the image of $K \cap P_\infty$ in $M_\infty$ (1) then the quotient
$$
X_M = K_M / M_\infty (1)
$$
is again a symmetric space. As metric on $X_M$ we take the one which is induced by the restriction of the Killing form to the Lie algebra of $M_\infty(1)$, this restriction satisfies obviously the conditions (1) and (2) in \S \ref{art5-sec1}. Let $\Gamma_m$ be the image of $\Gamma \cap P_\infty$ in $M_\infty$ (1)---this is again an arithmetically defined group. The map
$$
\pi : X (1) / \Gamma \cap P\infty \to X_M / \Gamma_M
$$
is easily seen to be an fibration with fiber $U_\infty/ U_\infty \cap \Gamma$. Therefore we have a spectral sequence
\setcounter{equation}{0}
\begin{align}
H^p (X_M / \Gamma_M, H^q(U_\infty/ U_\infty \cap \Gamma, E)) \Rightarrow H^n (X (1)/ \Gamma \cap P_\infty, E) &\label{art5-eq2.1}\\
p+q = n.&\notag
\end{align}
We shall see that this spectral sequence collapses and that the cohomology $H^\bigdot (X (1)/ \Gamma \cap P_\infty, E)$ is actually the direct sum of the $E_2$-terms on the left hand side.

First we apply the considerations of \S \ref{art5-sec1} to the cohomology of the fiber. Since $K \cap U_\infty = e$ the tangent bundle and the bundle $E_{\rho_\infty}$ restricted to $U_\infty/ U_\infty \cap \Gamma$ are trivial and $\Omega^q (U_\infty, E)$ is the space of $C_\infty$-functions $\omega : U_\infty \to \Hom (\Lambda^q u, E)$ where $u$ is the Lie algebra of right invariant vector fields on $U_\infty$. The group $U_\infty$ is acting on the $q$-forms by right translations, and $\Omega^q (U_\infty/ U_\infty \cap \Gamma, E)$ consists exactly of those forms which are invariant under $U_\infty \cap \Gamma$. Therefore we have an embedding
$$
\Hom (\Lambda^\bigdot u, E) = \Omega^\bigdot (U_\infty, E)^{U_\infty} {\displaystyle{\mathop{\hookrightarrow}^j}} \Omega^\bigdot (U_\infty/ U_\infty \cap \Gamma, E).
$$
We have also a map in the opposite direction
$$
h : \omega \to \int\limits_{U_\infty/ U_\infty \cap \Gamma} \omega (u) du.
$$
Here, $du$ is a Haar measure on $U_\infty$ such that $\vol_{du} (U_\infty / U_\infty \cap \Gamma) = 1$.

\begin{theorem}[van Est, \cite{art5-key4}]\label{art5-thm2.2}
The\pageoriginale maps $j$ and $h$ induce isomorphisms on the cohomology which are inverse to each other. We have
$$
H^\bigdot (U_\infty/ U_\infty \cap \Gamma, E) = H^\bigdot (\underline{u}, E)
$$
where the expression on the right hand side denote the cohomology of the Lie algebra $\underline{u}$ with coefficients in the $\underline{u}$-module $E$.
\end{theorem}

The first assertion is essentially a special case of Theorem 8 in van Est's paper, and the second one follows from his Theorem 9; but at this place  we have to make a few remarks about signs. Let us assume for the moment that $U_\infty$ is any Lie group and that $\underline{u}$ is the Lie algebra of right invariant vector fields. If $\rho_\infty$ is still our representation of $U_\infty$ in $GL(E)$ then we define as usual for $T \in \underline{u}$
$$
\rho_\infty (T) f = \lim\limits_{t \to 0} \frac{\exp (t T)f -f}{t}
$$
and with this definition $E$ becomes a right $\underline{u}$-module, \ie, we have
$$
(\rho_\infty (T_1) \rho_\infty (T_2) - \rho_\infty (T_2) \rho_\infty (T_1) ) f = \rho_\infty ([T_2, T_1]) f \text{ for } T_1, T_2 \in \underline{u}, f \in E
$$
where of course $[T_1 T_2] = T_1 T_2 - T_2 T_1$. Especially for the adjoint representation this yields $ad (T_1) (T_2) = [T_2, T_1]$. For a right $\underline{u}$-module $E$ the cohomology groups are computed from the following complex (comp. \cite{art5-key3}, Chap. XIII, \S 8).
$$
\longrightarrow \Hom (\Lambda^{q-1} \underline{u}, E) \xrightarrow{d} \Hom (\Lambda^q \underline{u}, E)\xrightarrow{d} \Hom (\Lambda^{q+1} \underline{u}, E) \longrightarrow
$$
where for $f \in \Hom (\Lambda^q \underline{u}, E)$  we have
%\setcounter{equation}{2}
\begin{align}
df (T_1 , \cdot , T_{q+1}) & = \sum\limits^{i=q+1}_{i=1} (-1)^i \rho_\infty (T_i)  f (T_1 , \cdot, \hat{T}_i, \cdot , T_{q+1}) + \notag\\
& = + \sum\limits_{1 \leqslant i < j \leqslant q +1} (-1)^{i+ j} f ([T_i, T_j], T_1 , \cdot , \hat{T}_j, \cdot , \hat{T}_j \cdot T_{q+1}). \label{art5-eq2.3}
\end{align}
Actually the second statement of Theorem \ref{art5-thm2.2} is easy consequence of the definitions, Lemma \ref{art5-lem1.4}, and formula \ref{art5-eq2.3}.

In the spectral sequence (2.1) we have to take the cohomology of $X_M/ \Gamma_M$ with coefficients in the local system of the cohomology of  the fiber\pageoriginale $H^\bigdot (U_\infty / U_\infty \cap \Gamma, E) = H^\bigdot (\underline{u}, E)$. This local system is determined by the operation of the fundamental group $\Gamma_M$ on $H^\bigdot (\underline{u}, E)$. On the other hand, the group $P_\infty (1)$ acts in a natural way on the cochain complex which defines $H^\bigdot (\underline{u}, E)$ and this gives us a representation of $M_\infty (1)$ on $H^\bigdot (\underline{u}, E)$. We claim that the restriction of the latter action to $\Gamma_M$ is equal to the first one. To see this we have to consider the pull back of the fibration $\pi$ over $\pi$ itself, we get a diagram 
$$
\xymatrix{
X(1)/ \Gamma \cap P_\infty \ar[r]^-\pi & X_M / \Gamma_M \\
Y\ar[r]^-{\pi'} \ar[u] & X(1) / \Gamma \cap P_\infty \ar[u]_-\pi
}
$$
and the fiber bundle $Y \xrightarrow{\pi'} X (1) / \Gamma \cap P_\infty$ is induced by the operation of $\Gamma \cap P_\infty$ on $U_\infty/ U_\infty \cap \Gamma$. But then the corresponding assertion for the fibration $\pi'$ is clear, and this proves our original claim. 

This tells us that we can apply the results of \S \ref{art5-sec1} to the $E_2$-terms of the spectral sequence. The next thing we shall do is to construct for a fixed $q$ an imbedding
$$
\Omega^p (X_M / \Gamma_M, \; H^q (\underline{u}, E)) \hookrightarrow \Omega^{p+q} (X (1) / \Gamma \cap P_\infty, E)
$$
and we shall see that this imbedding commutes with the exterior derivatives.

The Cartan involution corresponding to $K$ is denoted by $\theta$. It induces the well-known Cartan decomposition $\fG = \fk \oplus \fp$. Moreover it is well known that the intersection
$$
\tilde{M}_\infty = P_\infty \cap P^\theta_\infty
$$
is a real Levi subgroup of $P_\infty$ and that the restriction of $\theta$ to $\tilde{M}_\infty$ is again a Cartan involution. We introduce a positive definite metric on $\fG$ by
$$
B_\theta (N , N') = - B (N, \theta (N'))
$$
where $B$ is the Killing form. In $\tilde{M}_\infty$ we fix the maximal compact subgroup $K \cap \tilde{M}_\infty = K_M$.

%\setcounter{lemma}{3}
\begin{lemma}\label{art5-lem2.4}
The metric\pageoriginale $B_\theta$ restricted to $\underline{u}$ is admissible with respect to the adjoint representation of $\tilde{M}_\infty$ on $\underline{u}$.
\end{lemma}

\begin{proof}
Obvious.
\end{proof}

This metric on $\underline{u}$ extends to an admissible metric on $\Hom (\Lambda^q \underline{u}, E)$ since we have already chosen an admissible metric on $E$. Following Kostant \cite{art5-key9} we construct the adjoint operator $\delta : \Hom (\Lambda^p_{\;\;\underline{u}} , E ) \to$\break $\Hom (\Lambda^{p-1} \underline{u}, E)$ to the coboundary operator $d$. Since our metric is invariant under the action of $K_M$ the operators $d$ and $\delta$ commute with the action of $K_M$. We define the Laplacian by $L = d \delta + \delta d$, and Kostant has proved (comp. \cite{art5-key9}, Prop. 2.1 and 3.5.4)
$$
H^q (\underline{u}, E) \tilde{\longrightarrow} \{\zeta \in \Hom (\Lambda^q \underline{u}, E) | L \zeta = 0\} = \bH^q (\underline{u}, E).
$$
This is an $M_\infty$ invariant subspace of $\Hom (\Lambda^q \underline{u}, E)$ since our metric is admissible.

Let us put $\underline{P} = \Lie (P_\infty)$, $\underline{P}_1 = \Lie (P_\infty (1))$, $\underline{m} = \Lie (\tilde{M}_\infty)$ and $\underline{m}_1 = \Lie (\tilde{M}_\infty (1))$. The Cartan decomposition of $\underline{m}$ (resp. $\underline{m}_1$) with respect to the restriction of $\theta$ to $\tilde{M}_\infty$ (\resp. $\tilde{M}_\infty$ (1)) is denoted by 
\begin{align*}
& \underline{m} = \fK_M \oplus \bar{\fp}\\
(\resp.) \quad & \\
& \underline{m}_1 = \fK_M \oplus \bar{\fp}_1
\end{align*}
With respect to $B_\theta$ we have the following orthogonal decomposition (comp. \cite{art5-key6}, Prop. 1.1.1)
$$
\underline{P} = \underline{m} \oplus \underline{u}
$$
and this induces 
$$
\underline{P}_1 = \underline{m}_1 \oplus \underline{u} = \fK_M \oplus \bar{\fp}_1 \oplus \underline{u}.
$$
Our general considerations in \S \ref{art5-sec1} yield that the tangent bundle of $X(1)$ (resp. $X_M$) is canonically isomorphic to the bundle which is induced by the adjoint operation of $K_M$ on $\bar{\fp}_1 \oplus \underline{u}$ (resp. $\bar{\fp}_1$). Let us introduce the notation $\underline{r}_1 = \bar{\fp}_1 \oplus \underline{u}$. Then we have the decomposition of $K_M$-modules
$$
\Lambda^n \underline{r}_1 = \Lambda^n (\bar{\fp}_1 \oplus \underline{u}) = \bigoplus_{p+q=n} \Lambda^p \bar{\fp}_1 \oplus \Lambda^q \underline{u}
$$

We have\pageoriginale interpreted an element in $\Omega^p (X_M/ \Gamma_M, H^q (\underline{u}, E))$ as a $C^\infty$-function
$$
\varphi : M_\infty (1) / \Gamma_M \longrightarrow \Hom (\Lambda^p \bar{\fp}_1, \bH^q (\underline{u}, E))
$$
which satisfies 
$$
\varphi (kg) = \Lambda^p ad^\ast_{\bar{\fp}_1} (k)\otimes \Lambda^q ad^\ast_{\underline{u}} (k) \otimes \rho_\infty (k) (\varphi (g))
 $$ 
(comp. \eqref{art5-eq1.8}). But since $\bH^q (\underline{u}, E) \subset \Hom (\Lambda^q \underline{u}, E)$ we have 
$$
\Hom (\Lambda^p \bar{\fp}_1, \bH^q (\underline{u}, E)) \subset \Hom (\Lambda^p \bar{\fp}_1, \Hom (\Lambda^q \underline{u}, E)) \subset \Hom (\Lambda^{p+q}  \underline{r}_1, E)
$$
we may as well consider $\varphi$ as an element of $\Omega^{p+q} (X (1)/ \Gamma \cap P_\infty, E)$. Therefore we have constructed an embedding where we consider $q$ as fixed and $p$ as variable. Let us denote by $d_M$ the exterior derivative of the complex $\Omega^\bigdot (X_M / \Gamma _M, \bH^q (\underline{u}, E))$ and respectively by $d$ the exterior derivative of $\Omega^\bigdot (X (1)/ \Gamma \cap P_\infty, E)$.

\setcounter{equation}{5}
\begin{lemma}\label{art5-lem2.6}
We have 
$$
i_q \circ d_M = d \circ i_q.
$$
\end{lemma}

\begin{proof}
Le us take a form $\varphi \in \Omega^p (X_M / \Gamma_M, \bH^q (\underline{u}, E))$, we put $\tilde{\varphi}= i_q (\varphi)$. We pull these two forms back to $P_\infty (1)/ \Gamma \cap P_\infty$, and  carry out the computations upstairs, the forms upstairs will be denoted by the same letters. By $M_1, M_2, \cdot, M_i$, $(\resp. U_1, U_2, \ldots)$ we denote the right invariant vector fields on $P_\infty(1)$ which are obtained from elements $M_i \in \underline{m}_1$ $(resp. U_i \in \underline{u})$ . (Therefore is no confusion in the notation since the indices are always less than infinity in $M_\infty$). We get from our construction
$$
\varphi (M_1, M_2, \cdot \cdot , M_p) (U_1, \cdot \cdot , U_q) = \tilde{\varphi} (M_1, M_2, \cdot , M_p, U_1, \cdot, U_q)
$$
and moreover it is clear that $\varphi$ is bihomogeneous of degree $(p,q)$, \ie, we have $\varphi (\cdot M_i \cdot , \cdot U_j \cdot) = 0$ unless the number of $M's$ ($\resp. U's$) is equal to $p$ $(\resp. q)$. This implies that
$$
d \tilde{\varphi} (M_1, \cdot \cdot , M_p, U_1, \cdot \cdot, U_{q+1}) = 0
$$
because the terms involving $\nabla_{M_i}$, $[M_i, M_j]$, and $[M_i, U_j]$ are zero and the terms involving $\nabla_{U_i}$ and $[U_i, U_j]$ add up to zero since $\varphi$ takes its values in $\bH^q (\underline{u}, E)$. Therefore\pageoriginale  we have only to compute 
$$
d \varphi (M_1, \cdot , M_{p+1} , U_1, \cdot , U_q).
$$
 This breaks up into several terms; we use Lemma \ref{art5-lem1.4} repeatedly to get 
\begin{align*}
d\tilde{\varphi}& (M_1, \cdot \cdot , M_{p+1}, U_1, \cdot, U_q) =\\
& \sum\limits^{i=p+1}_{i=1} (-1)^{i+1} M_i \tilde{\varphi} (M_1, \cdot , \hat{M}_i, \cdot, M_{q+1}, U_1, \cdot , U_q) \tag{I}\\
- & \sum\limits^{i=p+1}_{i=1} (-1)^{i+1} \rho_\infty (M_i) \tilde{\varphi} (M_1, \cdot , \hat{M_1}, \cdot, M_{p+1}, U_1, \cdot , U_q) \tag{II}\\
+ & \sum\limits_{1 \leqslant i < j \leqslant p +1} (-1)^{i+j} \tilde{\varphi} ([M_i, M_j], M_1, \cdot , \hat{M}_i, \cdot , \hat{M}_j, \cdot , M_{p+1}, U_1, \cdot , U_q) \tag{III}\\
+ & \sum\limits^{i=p+1}_{i=1} \sum\limits^{j=q}_{j=1} (-1)^{i+p+1+j} \tilde{\varphi} ([M_i, U_j], M_1, \cdot , \hat{M}_i , \cdot, M_{p+1}, +\\
& \hspace{6cm}+ U_1, \cdot, \hat{U}_j, \cdot , U_q) \tag{IV}
\end{align*}
$+2$ other terms which vanish because $\tilde{\varphi}$ is bihomogeneous of degree $(p,q)$.

On the other hand we have 
\begin{align*}
&\quad d_M \varphi (M_1, \cdot \cdot, M_{p+1} ) =\\
& \sum\limits^{i=p+1}_{i=1} (-1)^{i+1} M_i \varphi (M_1, \cdot , \hat{M}_1, \cdot , M_{p+1})  \tag{I$'$}\\
- & \sum\limits^{i=p+1}_{i=1} (-1)^{i+1} (\rho_\infty (M_i) + \Lambda^q ad^\ast_{\underline{u}} (M_1)) \varphi (M_1, \cdot , \hat{M}_i, \cdot, M_{p+1}) \tag{II$'$}\\
+ & \sum\limits_{1 \leqslant i < j \leqslant p +1} (-1)^{i+j} \varphi ([M_i, M_j], M_1, \cdot , \hat{M}_i, \cdot, \hat{M}_j, \cdot , M_{p+1}). \tag{III$'$}
\end{align*}
We have to evaluate this on $(U_1, \cdot , U_q)$ and we see that (I) = (I$'$) and (III) = (III$'$). We claim that II$'$ = II + IV. We see that the typical term in IV is equal to 
$$
(-1)^i \tilde{\varphi} (M_1, \cdot , \hat{M}_i,\cdot , M_{p+1}, U_1, \cdot , [M_i, U_j], \cdot , U_q).
$$
We see that (II$'$) naturally breaks up into two terms, one of them is obviously equal to (II) and the other one is 
$$
\sum\limits^{i=p+1}_{i=1} (-1)^i (\Lambda^q \ad_{\underline{u}} (M_i)_\varphi (M_1, \cdot , \hat{M}_i, \; , M_{p+1})) \; (U_1, \cdot , U_q).
$$
If we\pageoriginale take into account that $\ad_{\underline{u}} (M_i) (U_j) =[U_j, M_i]$ (comp. remark following Theorem \ref{art5-thm2.2}) then we see that this second term is equal to (IV).

We introduced an admissible metric on $\Hom (\Lambda^q \underline{u}, E)$ which induces an admissible metric on $\bH^q (\underline{u}, E)$. From this metric we get a Laplacian operator
$$
\Delta_M : \Omega^p (X_M /\Gamma_M, \bH^q (\underline{u}, E)) \longrightarrow \Omega^p (X_M / \Gamma_M , \bH^q (\underline{u}, E)).
$$
On the other hand we have a Laplacian operator $\Delta_1$ on the complex $\Omega^\cdot (x(1)/ \Gamma \cap P_\infty, E)$ which is defined by means of the admissible metric on $E$.
\end{proof}

%\setcounter{lemma}{6}
\begin{lemma}\label{art5-lem2.7}
We put $s = \dim (\underline{u})$. Then we have
$$
i_q \circ \Delta_M = \Delta_1 \circ i_q.
$$
\end{lemma}

\begin{proof}
Let us denote by $M^0_\infty$(1) (\resp. $K^0_M$) the connected components of $M_\infty$(1) (\resp. $K_M$). Since $X_M$ is connected we know that $X_M = K_M / M_\infty (1) = K^0_M / M^0_\infty$ (1) and we may assume that $\Gamma_M \subset M^0_\infty$ (1), since the Laplacian operator does not depend on an orientation. If we put $s = \dim \underline{u}$, then $K^0_M$ acts trivially on $\Lambda^s \underline{u}$, we choose an isomorphism $\Lambda^s \underline{u} \tilde{\longrightarrow} \bR$. The nondegenerated pairing
$$
\Hom (\Lambda^q \underline{u}, E) \times \Hom (\Lambda^{s-q} \underline{u}, E^\ast) \longrightarrow \bC = \Hom (\Lambda^s \underline, \bC)
$$
induces a nondegenerated pairing 
$$
\bH^q (\underline{u}, E) \times \bH^{s-q}(\underline{u}, E^\ast) \longrightarrow \bC
$$
which is compatible with the action of $K^0_M$.

To prove the assertion in our lemma, we compare the operators $\delta_1$ and $\delta_M$. The operator $\delta_M$ is constructed by means of the mapping (comp. \S \ref{art5-sec1})
$$
\#_M L \bH^q (\underline{u}, E) \longrightarrow \bH^q (\underline{u}, E)^\ast = \bH^{s-q} (\underline{u}, E^\ast)
$$
and $\delta_1$ is constructed by means of the operator 
$$
\#_1: \Omega^{p+q} (X(1) / \Gamma \cap P_\infty, E) \tilde{\longrightarrow} \Omega^{p+q} (X (1) / \Gamma \cap P, E^\ast)
$$
which is constructed by means of the admissible metric on $E$. The $\ast$-operator on the forms on $X(1) / \Gamma \cap P_\infty$ is defined by means of the restriction\pageoriginale of $B_\theta$ to $\underline{r}_1 = \bar{\fp}_1 \oplus \underline{u}$ where this decomposition is orthogonal. If we restrict $B_\theta$ to $\underline{u}$ we get the admissible metric on $\underline{u}$ which induces the admissible metric on $\bH^q (\underline{u}, E)$. Therefore we get the following commutative diagram
$$
\xymatrix@R=1.2cm{
\Omega^p (X_M / \Gamma_M, \bH^q (\underline{u}, E)) \ar[r]^{i_q} \ar[d]_{(-1)^{(n-p)q} \ast_M \circ \#_M} & \Omega^{p+q} (X(1) / \Gamma \cap P_\infty, E) \ar[d]^{\ast_1 \circ \#_1}\\
\Omega^{n-p} (X_M/ \Gamma_M, \bH^{s-q} (\underline{u}, E^\ast)) \ar[r]^{i_{s-q}} & \Omega^{s+n - p-q} (X (1)/ \Gamma \cap P_{\infty}, E^\ast)
}
$$ 
whee $n = \dim X_M$. A simple calculation yields
$$
i_q \circ \delta_M = \delta_1 \circ i_q
$$
and now the lemma is obvious.
\end{proof}

%\setcounter{theorem}{7}
\begin{theorem}\label{art5-thm2.8}
If $X_M/ \Gamma_M$ is compact then the mapping $i$ induces  an isomorphism for the cohomology
$$
i^\ast: \oplus_{p+q = m} H^p (X_M / \Gamma_M, \bH^q (\underline{u}, E)) \tilde{\longrightarrow} H^m (X(1) / \Gamma \cap P_\infty, E).
$$
\end{theorem}

\begin{proof}
The Lemma \ref{art5-lem2.6} tells us that $i^\ast$ induces a homomorphism on the cohomology. The Hodge theory tells us that the cohomology groups on both sides are isomorphic to the corresponding spaces of harmonic forms with respect to $\Delta_M$ and $\Delta_1$. This implies that $i^\ast$ is injective and the surjectivity  follows then from the spectral sequence.

The group $\tilde{M}_\infty$ is the group of real points of an algebraic group over the real numbers. Let us denote by $T$ the connected component of the identity of the centre of that group.  This is a torus over $\bR$ and let us denote its group of real points by $T_\infty$. The adjoint action induces an operation of $T_\infty$ on $\bH^q (\underline{u}, E)$ which is semisimple, and the eigenvalues of this action are induced by algebraic characters of $T$. The metric $B_\theta$ induces a decompositian
%\setcounter{equation}{8}
\begin{align}
& T_\infty = T_\infty (1)^\bigdot A \notag\\
& \Phi : A \tilde{\longrightarrow} ((\bR^+)^\ast)^r \notag\\
&  \Phi : a \tilde{\longrightarrow} (|\chi_1| (a), \cdot, |\chi_r| (a)) 
\label{art5-eq2.9}
\end{align}
where\pageoriginale we shall use $\Phi$ to identify $A = ((\bR^+)^\ast)$ and we shall write $a = (t_1, t_2 , \cdot , t_r)$. If $r =1$ we write $a =t$. We now restrict our above action to $A$, Then $A$ acts  by algebraic characters
$$
\lambda: a \longrightarrow a^\lambda = t_1^{\lambda_1} \quad t^{\lambda_r}_{r} \quad \lambda_i \in \bZ
$$
on $\bH^q (\underline{u}, E)$ and we denote the corresponding decomposition by 
%\setcounter{equation}{9}
\begin{equation}
\bH^q (\underline, E) = \bigoplus\limits_\lambda \bH^q_\lambda (\underline{u}, E) 
\label{art5-eq2.10}
\end{equation}
 and we shall call $\bH^q_\lambda (\underline{u}, E)$ the space of classes of weight $\lambda$.

The group $A$ acts on $\Hom (\Lambda^s\underline{u}, \bC)$ by the character $a \to a^{-2\rho}$ where $2\rho$ is the sum of the positive roots. This implies 
$$
\#_M : \bH^q_\lambda (\underline{u}, E) \longrightarrow \bH^{s-q}_{-2\rho -\lambda} (\underline{u}, E^\ast).
$$

Since the action of $A$ on $\bH^q (\underline{u}, E)$ commutes with the action of $\tilde{M}_\infty$ we get a decomposition
\begin{align*}
H^\bigdot (X (1)/ \Gamma \cap P_\infty, E) & = \oplus_\lambda H^\bigdot_\lambda (X (1) / \Gamma \cap P_\infty, E)\\
& = \oplus_\lambda H^\bigdot (X_M / \Gamma_M, \bH^\bigdot_\lambda (\underline{u}, E))
\end{align*}
\end{proof}

\section{Extension of differential forms and Eisenstein series.}\label{art5-sec3}
We assume that $G / k$ is of $k$-rank one. If $P\subset G/k$ is a minimal parabolic subgroup then we have only one fundamental dominant weight $\chi: P\longrightarrow G_m$. We define the function (comp. 2.9)
\begin{align*}
& h : X \longrightarrow (\bR^+)^\ast\\
& h : x = x_0 p \longrightarrow |\chi|(p) = t.
\end{align*}
This function is equivariant with respect to the action of $P_\infty$. Let us denote $h^\ast \left(\dfrac{dx}{x} \right) = \dfrac{dt}{t}$, this is a $P_\infty$-invariant 1-form on the symmetric space $X$.

We define $X (t) = \{x \in X | h(x) = t\}$. Let us define $T$ as the vector field on $X$ which is orthogonal to the vector fields along the slices $X (t)$ and for which $\dfrac{dt}{t} (T) = 1$.

The metric $B_\theta$ induces a decomposition
$$
P_\infty = A \cdot P_\infty (1)
$$
where\pageoriginale $A$ centralizes $K \cap P_\infty$ and where $|\chi| : A \tilde{\longrightarrow} (R^+)^\ast$ is the identification we introduced at the end of \S \ref{art5-sec2}. Now the mapping
\begin{align*}
& m_t : X(1) \tilde{\longrightarrow} X (t)\\
&  m_t : x = x_0 p \tilde{\longmapsto} x_0 t p \quad p \in P_\infty (1)
\end{align*}
is a diffeomorphism which is compatible with the action of $P_\infty (1)$. Therefore it induces a diffeomorphism
$$
\bar{m}_t : X (1) / \Gamma \cap P_\infty \longrightarrow X (t) / \Gamma \cap P_\infty
$$
and we find a diffeomorphism $X/ \Gamma \cap P_\infty\tilde{\longrightarrow} (X(1)/ \Gamma \cap P_\infty) \times \bR$. The tangent bundle $T^X$ of $X / \Gamma \cap P_\infty$ has the orthogonal decomposition 
$$
T^X = T^X_0 \oplus \langle T \rangle
$$
where $T^X_0$ is the bundle of tangent vectors along the slices $X(t)$ and where $\langle T \rangle$ is the subbundle spanned by the vector field $T$. In the language of induced bundles this corresponds to the decomposition
$$
\underline{r} = \underline{r}_1 \oplus \bC T
$$
where $T$ now is considered as an element of Lie ($A$) and where $\underline{r} = \bar{\fp} \oplus \underline{u}$ (comp. \S \ref{art5-sec2}.). The group $K_M$ acts trivially on $T$. 

We now want to give some preparations which are necessary for the definition of Eisenstein series. Let us start with a cohomology class
$$
[\varphi] \in H^m (X(1)/ \Gamma \cap P_\infty, E). 
$$
This cohomology class has a unique harmonic representative 
$$
\Omega^m (X(1) / \Gamma \cap P_\infty, E)
$$
 which may be viewed as a function
$$
\varphi : P_\infty (1)/ \Gamma \cap P_\infty (1) \longrightarrow \Hom (\Lambda^m \underline{r}_1, E)
$$
which satisfies for $k \in K_M$, $p \in P_\infty (1)$, $u \in U_\infty$
$$
\varphi (k p u) = \Lambda^m \ad^\ast_{\underline{r}_1} (k) \otimes \rho_\infty (k) (\varphi (p))
$$
(comp. \eqref{art5-eq1.8}). The right invariance with respect to the action of $U_\infty$ follows from Theorem \eqref{art5-thm2.8}. We have a $K_M$-invariant embedding 
$$
\Hom (\Lambda^m \underline{r}_1, E) \hookrightarrow \Hom (\Lambda^m\underline{r}, E).
$$
On the bigger space the group $K$ is acting,\pageoriginale but the smaller space is not necessarily invariant under this action of $K$. We now extend our function $\varphi$ to a function
$$
\varphi_s : G_\infty / G \cap P_\infty \longrightarrow \Hom (\Lambda^m\underline{r}, E)
$$
by
$$
\varphi_s (ktp) = \Lambda^m \ad^\ast_{\underline{r}} (k) \otimes \rho_\infty  (k) (\varphi (p))^{t^{-\rho -s}}
$$
where $p \in P_{\infty} (1)$, $k \in K$, $t \in A$, and $s \in \bC$. This function can be considered as a differential $p$-form on $X/ \Gamma  \cap P_\infty$. In this case $\rho$ is of course an integer.

We now denote the exterior derivative (resp. exterior coderivation, resp. Laplacian operator) on the complex $\Omega^\bigdot (X/ \Gamma \cap P_\infty, E)$ by $d$ (\resp. $\delta$, resp. $\Delta$). Then 

\setcounter{equation}{0}
\begin{lemma}\label{art5-lem3.1}
If [$\varphi$] is of weight $\lambda$, then 
\begin{align*}
d \varphi_s & = (- \rho - s - \lambda) \frac{dt}{t} \wedge \varphi_s \\
\delta \varphi_s & = 0 \\
\Delta \varphi_s & = (s^2 - (\rho + \lambda)^2)\varphi_s.
\end{align*}
\end{lemma}

\begin{proof}
We pull $\varphi_s$ back to $P_\infty$. We have to evaluate $d \varphi_s$ on $m+1$-tuples of right invariant vector fields. Since $\varphi$ itself is closed the only nonzero terms are
$$
d \varphi_s (T, M_1, \cdot , M_a, U_1, \cdot , U_b )  \quad 
\begin{array}{c}
\\
M_i \in \bar{\fp}_1, \; U_j \in \underline{u}\\
\qquad a + b = m
\end{array}
$$

Since $\varphi_s (T, \cdot , M, \cdot, U^\bigdot) =0$ we get for this last expression
\begin{align*}
& \Delta_{T} \varphi_s (M_1, \cdot , M_a, U_1, \cdot , U_b) + \sum\limits_{1 \leqslant j \leqslant b} (-1)^{1+ p + 1 + j_{\varphi_s}} ([T, U_j], M_i \cdot , \cdot U_j)\\
& = (-\rho -s) \varphi_s (M_1, \cdot, M_a, U_1, \cdot , U_b) - \rho_\infty (T)\varphi_s (M_1, \cdot, M_a, U_1, \cdot , U_b) - \\
& \hspace{3cm} - \sum\limits_{1 \leqslant j \leqslant b} \varphi_s (M_1, \cdot, M_a, U_1, \cdot , [T, U_j], \cdot , U_b).
\end{align*}
Since $\ad (T) (U_j) = [U_j, T]$ (comp. remark following Theorem \ref{art5-thm2.2}) we see that the last two terms add up to $- \lambda \varphi_s (M_1, \cdot, M_a, U_1, \cdot , U_b)$. To prove the second formula we compare the operators $\ast$ and $\#$ on $\Omega^\bigdot (X/ \Gamma \cap P_\infty, E)$ to the corresponding operators $\ast_1$ and $\#_1$ on $\Omega^\bigdot (X(1)/ \Gamma \cap P_\infty, E)$.\pageoriginale It is clear that $\#_1$ is the restriction of $\#$ to $\Omega^\ast (X(1)/\Gamma \cap P_\infty, E)$ and since $\dfrac{dt}{t}$ is orthogonal to the vector fields along the slices $X(t)$ we get
\begin{equation}
\left. 
\begin{aligned}
\ast \varphi_s = (-1)^p \frac{dt}{t} \wedge (\ast_1 \varphi)_s \\
\ast \left(\frac{dt}{t}\wedge\varphi_s \right) =(\ast_1 \varphi)_s
\end{aligned}
\right\}
\label{art5-eq3.2}
\end{equation}

Since $\varphi$ is harmonic we find
$$
\delta \varphi_s = (-1)^{p+1} \ast^{-1} \circ \#^{-1} d \# \circ \ast \varphi_s = -\ast^{-1} \circ \#^{-1} d \left(\frac{dt}{t} \wedge (\#_1 \circ \ast_1 \varphi)_s \right) = 0
$$
because of our previous formula.

To prove the formula for the Laplacian operator, we have to observe that it follows from our considerations at the end of \S \ref{art5-sec2} that $\ast_1 \circ \#_1 \varpi = \ast_M \circ \#_M \varphi$ is of weight $-2 \rho - \lambda$. Then the formula becomes obvious. 

We want to define the Eisenstein series $E(\varphi, s) \in \Omega^p (X/ \Gamma ; E)$ which is associated to [$\varphi$]. To do this we have to recall the general context in which the Eisenstein series are defined.

We start with a representation
$$
\sigma : K \longrightarrow G L (V)
$$
where $\dim V < + \infty$. Let $\eta  \in \Hom_{K_M} (V, V)$, we introduce the vector space
{\fontsize{10}{12}
$$
\sA (M_\infty (1) / \Gamma_M , \sigma, \eta) = \{\psi: M_\infty (1)/ \Gamma_M \to V
\left.
\left|
\begin{aligned}
&\psi(km) = \sigma (k) \psi (m) \text{ for } k \in K_M\\
&(C_M \psi) (m) = \eta (\psi (m))
\end{aligned}
\right.
\right\}
$$}
where $C_M$ is the Casimir operator. This vector space is of finite dimension since $M_\infty (1)/ \Gamma _M$ is compact and since the Casimir operator induces an elliptic operator on the bundle $V_\sigma \to X_M / \Gamma_M$ which is induced by $\sigma$. If $\psi \in \sA(M_\infty (1)/\Gamma_M, \sigma, \eta)$ we put, following Harish-Chandra, $\psi_s (g) = \psi_s (ktp) =\sigma (k) \psi (p)^{\bigdot t ^{-\rho - s}}$ (comp. \cite{art5-key7}, Chap. II, \S \ref{art5-sec2}) and we define for $s \in \bC$, $\RE (s) > \rho$
$$
E (g, \psi, s) = E (\psi, s) = \sum\limits_{a \in \Gamma / \Gamma \cap P_\infty} \psi_s (ga).
$$
It is\pageoriginale known that this series converges for $\RE (s) > \rho$ and that it has meromorphic continuation into the entire $s$-plane (comp. \cite{art5-key8}, Chap. IV).

If $P'$ is another minimal parabolic subgroup then the constant Fourier coefficient of $E(g, \psi, s)$ along $P'$ is defined by 
$$
E^{P'} (g,\psi, s) = \int\limits_{U_\infty / U' \infty \cap \Gamma} E (gu', \psi, s) d u'
$$
where $U'$ is the unipotent radical of $P'$ and where
$$
\vol_{du'} (U'_\infty/ U'_\infty \cap \Gamma) =1.
$$

We shall state briefly some of the results concerning there constant Fourier coefficients which are proved in \cite{art5-key8}, Chap. II. We put
$$
\epsilon = 
\begin{cases}
0 \text{if $P$ and $P'$ are not $\Gamma$-conjugate}\\
1 \text{if $P$ and $P'$ are $\Gamma$-conjugate.}
\end{cases}
$$
If $\epsilon =1$ we assume that $P = P'$. Then
$$
E^{P'} (\psi, s) =\epsilon (\psi)_s + (c (s) \psi)_{-s}
$$
where $c(s)$ is a linear mapping
$$
c(s) : \sQ (M_\infty (1)/ \Gamma_M, \sigma, \eta) \longrightarrow \sA (M'_\infty (1)/ \Gamma_M, \sigma, \eta')
$$ 
which is meromorphic in the variable $s$. We claim that this follows from \cite{art5-key8}, Chap. II, Theorem 5 and the results in Chap. IV on analytic continuation. To see this we choose  an element $y \in K$ which conjugates $\tilde{M}_\infty$ to $\tilde{M}'_{\infty}$ and which conjugates $P$ into the opposite $\tilde{P'}$ of $P'$ with respect to $\tilde{M}'_\infty$, \ie, $\bar{P}'_\infty, \cap P'_\infty = \tilde{M}'_\infty$. This map sends the Casimir operator on $\tilde{M}_\infty (1)$ to the Casimir operator on$M'_\infty (1)$ and it sends $\eta \in \Hom_{K_M} (V, V)$ into an element
$$
\eta' = \ad (y) (\eta) \in \Hom_{K_{M'}} (V, V).
$$
Then our statement above is a slight modification of Lemma 36 in \cite{art5-key8}, Chap. II.

Now we come back to the cohomology. We have seen that the cohomology group $H^m (X (1)/ \Gamma \cap P_\infty, E)$ can be identified with the space of harmonic forms
$$
\varphi: M_\infty (1)/ \Gamma_M \longrightarrow \bigoplus_{p+q=m} \Hom (\Lambda^p \underline{P}_1, \bH^q (\underline{u}, E)) \text{ for which } \Delta_M \varphi = 0
$$\pageoriginale
(comp. Theorem \ref{art5-thm2.8}). We have the inclusions
\begin{align*}
& \bigoplus_{p+q=m} \Hom (\Lambda^p \bar{\fp}_1, \bH^q (\underline{u}, E)) \subset \bigoplus_{p+q = m} \Hom (\Lambda^p \bar{\fp}_1, \Hom (\Lambda^q \underline{u}, E))\\
& = \Hom (\Lambda^m \underline{r}_1, E) \subset \Hom (\Lambda^m \underline{r}, E).
\end{align*}
The biggest space is invariant under $K$; the other spaces are in general only $K_M$-invariant. The Laplacian operator $\Delta_M$ acts on the space of functions 
$$
\varphi : M_\infty (1) / \Gamma_M \longrightarrow \bigoplus_{p+q = m} \Hom (\Lambda^p \bar{\fp}_1, \Hom (\Lambda^q \underline{u}, E))
$$
which satisfy \eqref{art5-eq1.8}, and it follows from the lemma of Kuga that (comp. \eqref{art5-eq1.9})
$$
\Delta_M \varphi = - C_M \varphi + \lambda (C_M) \varphi
$$
where $\lambda (C_M) = \bigoplus_{p+q=m} \lambda_{p,q} (C_M)$ and $\lambda_{p,q} (C_M)$ is the linear transformation induced by the Casimir on
$$
\Hom (\Lambda^p \fp_1, \Hom (\Lambda^q \underline{u}, E)).
$$
Because of our identifications
$$
\lambda(C_M) \in \End (\Hom (\Lambda^m \underline{r}_1, E))
$$
and since 
$$
\Lambda^m \underline{r} = \Lambda^m \underline{r}_1 \oplus \Lambda^{m-1} \underline{r}_1 \otimes \bC T
$$
we may extend $\lambda (C_M)$ trivially to $\Lambda^{m-1} \underline{r}_1 \otimes \bC T$ and therefore we may consider $\lambda (C_M)$ also as an element of 
$$
\End_{K_M} (\Hom(\Lambda^m \underline{r}, E)).
$$
Our considerations shown that 
$$
H^m (X(1)/ \Gamma \cap P_\infty; E) \hookrightarrow \sA (M_\infty (1) \Gamma_M, \Lambda^m \ad^\ast \otimes \rho_\infty, \lambda (C_M))
$$
and therefore we can associate to any cohomology class an Eisenstein series. Let us denote
$$
\sA (M_\infty (1)/ \Gamma_M, \Lambda^m \ad^\bigdot \underline{r} \otimes \rho_\infty, \lambda (C_M)) = \sH^{(m)}_M.
$$
It is clear that in this case we have
$$
c(s) : \sH^{(m)}_M \longrightarrow \sH^{(m)}_{M'}
$$
since the representation $\ad_{\underline{r}_1} | K_M$ extends to the normaliser of $K_M$ in $K$.
\end{proof} 

\begin{remark*}
The space\pageoriginale $\sH_M$ plays only an auxiliary role; we need, it, because we want to have a home for our $c(s) \varphi$ and we want to have some space of functions which is invariant under the transformations $c(s)$. It is not clear that the $c(s) \varphi $ is again a cohomology class if $\varphi$ was one.
\end{remark*}

The Eisenstein series will now be used in the following context: We know that $X/ \Gamma$ is up to homotopy a compact manifold $V$ with boundary $\partial V$. The theory of Eisenstein series will help us to associate to any cohomology class $[\varphi] \in H^m (\partial V; E)$ an Eisenstein series $E(\varphi, s) \in \Omega^m (X/\Gamma : E)$.

Before we can do this we must recall some facts from reduction theory. Let $P_1, P_2, \ldots, P_d$ be a set of representatives for the $\Gamma$-conjugacy classes of minimal parabolic subgroups. We consider the projection
$$
Y = \bigsqcup^{i=d}_{i=1} X / \Gamma \cap P_{i, \infty} \longrightarrow X / \Gamma.
$$
For each $X/ \Gamma \cap P_{i, \infty}$ we have the functions $h_i: X/ \Gamma \cap P_{i, \infty} \to (\bR^+)^\ast$ which have been introduced above. We collect these functions to a function $h: Y \longrightarrow (R^+)^\ast$ If $t_0 > 0$ we define 
$$
Y(t_0) = \{y \in Y | h (y) < t_0\}.
$$
It is well known that we can choose $t_0 > 0$, such that the mapping $Y(t_0) \longrightarrow X/ \Gamma$ is injective and locally diffeomorphic (comp. \cite{art5-key2}, Theorem 17.10). Let us identify $Y(t_0)$ with its image. The complement of $Y(t_0)$ is a compact manifold $V$ with boundary $\partial Y$ (comp. \cite{art5-key2}, loc. cit.,). The boundary components of $V$ are the manifolds $X_i (t_0)/\Gamma \cap P_{i, \infty}$ where $X_i (t_0)$ is of course $\{x \in X | h_i  (x) = t_0\}$.

Now we come back to the Eisenstein series. Let us consider a cohomology class $[\varphi] \in H^m (\partial V, E)$, then this class may be considered as a vector $[\varphi] = ([\varphi_1], \ldots, [\varphi_d])$ where
$$
[\varphi_i] \in H^m (X_i (t_0) / \Gamma \cap P_{i, \infty}, E).
$$
Since we have the identification
$$
\bar{m}_{t_0} : X_i (1) / \Gamma \cap P_{i,\infty} \longrightarrow X_i (t_0) / \Gamma \cap P_{i, \infty}
$$
we can\pageoriginale associate an Eisenstein series to each of the classes [$\varphi_i$] and we put
$$
E(\varphi, s) = \sum\limits^{i=d}_{i=1} E (\varphi_i, s).
$$
This is a differential form on $X/ \Gamma$.

We call $[\varphi] \in H^m (\partial V, E)$ a cohomology class of weight $\lambda$ if its components $[\varphi_1], \ldots, [\varphi_d]$ are of weight $\lambda$. In this case we have (Lemma \ref{art5-lem3.1})
\begin{equation}
\left.
\begin{aligned}
& \Delta E (\varphi, s) = (s^2 - (\lambda + \rho)^2)E (\varphi, s)\\
& \Delta E (\varphi, s) = 0.
\end{aligned}
 \right\}\label{art5-eq3.3}
\end{equation}
For any cohomology class [$\varphi$] on the boundary we consider the constant Fourier coefficients of $E (\varphi, s)$ along the $P'_i$s
$$
E^{P_i} (\varphi, s) = (\varphi_i)_s + \left(\sum\limits^d_{j-1} c_{ij} (s) \varphi_j \right)_s
$$
and we collect them to a vector
$$
E^P(\varphi, s) = (E^{P_1} (\varphi, s), \ldots, E^{P_d} (\varphi, s)).
$$
We shall write symbolically
$$
E^P(\varphi, s) = (\varphi)_s + (c (s) \varphi)_{-s}.
$$

For the rest of the paper we keep this notation for the constant Fourier coefficient. This is slightly sloppy because it means that we do so as if there is only one cusp. But otherwise the notation becomes clumsy and the arguments are essentially the same.

\section{Cohomology classes represented by singular values of the Eisenstein series.}\label{art5-sec4}
If $\varphi$ is of weight $\lambda$, then we know that $\Delta e(\varphi, s) = (s^2 - (\rho + \lambda)^2) E (\varphi, s)$. Therefore we get a harmonic form if we evaluate $E(\varphi,s)$ at the special values $s = \pm (\rho + \lambda)$. Of course we have to be careful, since $E(\varphi, s)$ might have a pole at such a special value. Our main concern will be the case where this pole is of order one. Then we put
$$
\Res_{s = \rho + \lambda} E(\varphi, s) = \lim\limits_{s \to \rho + \lambda} (s - \rho - \lambda) E (\varphi, s) = E' (\varphi, \rho + \lambda).
$$
We call\pageoriginale these differential forms $E (\varphi , \pm  (\rho +\lambda))$ or $E' (\varphi, \pm (\rho + \lambda))$ the \textit{singular values} of the Eisenstein series. We shall discuss the following two problems:
\begin{itemize}
\item[(A)] When does a singular value represent a cohomology class, \ie, when is it closed?

\item[(B)] In case it represents a cohomology class, what is the restriction of this class to the boundary?
\end{itemize}
To attack these questions we have to consider the constant Fourier coefficient. First of all we claim that $d E (\varphi, \rho+\lambda) = 0$ if and only if $d (E^P (\varphi, \rho + \lambda)) =0$. It is clear that the first statement implies the second one. To see the other direction we have to recall the notation of the space of cusp forms. A differential form $\omega\in \Omega^m (X/\Gamma, E)$ is called a cusp form if it is square integrable and if (comp. \cite{art5-key8} Chap. I, \S \ref{art5-sec2})
$$
\omega^P (g) = \int\limits_{U_\infty/ U'_\infty \cap \Gamma} \omega (gu) du \equiv 0.
$$
Here we interpret $\omega$ as a function on $G_\infty / \Gamma$ which satisfies (1.8). We know that for any cusp form $\omega \in \Omega^{m+1} (X/ \Gamma, E)$, which is an eigenfunction with respect to the centre of the universal enveloping algebra, the integral
$$
\langle dE (\varphi_1 \rho + \lambda),\omega)\rangle = \int\limits_{X/\Gamma} (dE (\varphi_1 \rho + \lambda), \ast \overline{\circ \# \omega})
$$
exists (\cite{art5-key8}. Chap. I, Lemma 15). The results in \cite{art5-key8}, Chap. I also imply that the value of this integral is equal to $\langle E(\varphi, \lambda + \varphi), \delta \omega \rangle$ (apply cor. to Lemma 10 and Lemma 14). But the latter integral is zero since $\delta \omega$ is a cusp form. Now our assertion follows from Theorem 4 in \cite{art5-key8}, Chap. I.

If $d E (\varphi, \rho + \lambda) = 0$ then we see from our Theorem \ref{art5-thm2.8} and the description of the boundary that $[E(\varphi, \rho + \lambda)]$ and $[E^P (\varphi, \rho +\lambda)]$ represent the same cohomology class on the boundary $\partial V$ of $V$.

We decompose the space $\Hom (\Lambda^m \underline{r}_1, E)$ with respect to the action of $A$ and write
$$
\Hom (\Lambda^m \underline{r}_1, E) = \bigoplus_\mu \Hom_\mu (\Lambda^m \underline{r}_1, E)
$$\pageoriginale
where the $\mu$'s are algebraic characters on $A$ (comp. \S \ref{art5-sec2}). Moreover we write
$$
\Hom (\Lambda^m \underline{r}, E) =\bigoplus_\mu \Hom_\mu (\Lambda^m \underline{r}_1, E) \oplus \bigoplus_\mu \dfrac{dt}{t} \bC \wedge \Hom_\mu (\Lambda^{m-1} \underline{u}_{1}, E).
$$
This yields a decomposition of $c (s) \varphi$:
$$
c (s) \varphi = \sum\limits_\mu c_\mu (s) \varphi + \sum\limits_\mu \frac{dt}{t} \wedge n_\mu (s) \varphi.
$$
From this we get the following rather messy formula for the exterior derivative of $E (\varphi, s)$ where $\varphi$ is supposed to be of weight $\lambda$ (comp. Lemma \ref{art5-lem3.1}.)
\setcounter{equation}{0}
\begin{align}
dE^P (\varphi, S) & = (-s - \rho - \lambda) \frac{dt}{t} \wedge (\varphi)_s + \sum\limits_\mu (s - \rho - \mu) \frac{dt}{t} \wedge (c_\mu (s) \varphi)_{-s} \notag\\
& \quad - \sum\limits_s \left(\frac{dt}{t} \wedge dn_\mu (s) \varphi  \right)_{-s} + \sum\limits_\mu (dc_\mu (s) \varphi)_{-s}.   \label{art5-eq4.1}
\end{align}
(Actually we have to apply a slight generalization of Lemma \ref{art5-lem3.1}). On the other hand we may also start from the function
$$
\frac{dt}{t} \wedge \varphi : P_\infty (1) / \Gamma \cap P_\infty \longrightarrow \dfrac{dt}{t} \wedge \Hom (\Lambda^m \underline{r}_1, E) \subset \Hom (\Lambda^{m+1} \underline{r}, E)
$$
and we may consider its associated Eisenstein series $E \left(\dfrac{dt}{t} \wedge \varphi, s \right)$. Then it is clear that 
\begin{equation}
E^P \left(\frac{dt}{t}  \wedge \varphi, s \right) =\frac{1}{-s-\rho - \lambda} dE^P(\varphi, s). \label{art5-eq4.2}
\end{equation}
The operator $\ast \circ \#$ induces an isomorphism
$$
\ast \circ \# : \Hom (\Lambda^m \underline{r}, E) \longrightarrow \Hom (\Lambda^{N-m} \underline{r}, E^\ast), N = \dim (X)
$$
which commutes with the action of $K$. We apply this operator on both sides of \eqref{art5-eq4.2} and since the construction of Eisenstein series commutes with the operator $\ast \circ \#$, we obtain
\begin{equation}
E^P(\ast_1 \circ \#_1 \varphi, s) = \frac{1}{-s-\rho-\lambda} \ast \circ \# d E^P (\varphi, s)
\label{art5-eq4.3}
\end{equation}
(comp. \eqref{art5-eq3.2}).

We now\pageoriginale assume that $\lambda > -\rho$ and that $\varphi$ runs over the classes of weight $\lambda$. We shall investigate the influence of the behavior of the Eisenstein series at $\rho + \lambda > 0$ on the solution of our problems (A) and (B). We know that the Eisenstein series has at most a pole of order 1 at this point. (comp. \cite{art5-key8}, Chap. IV, \S 6). We put
$$ 
C_\mu = {\displaystyle{\mathop{\res}_{s=\rho + \lambda}}} c_\mu (s) \text{ and } N_\mu = {\displaystyle{\mathop{\res}_{s = \rho + \lambda}}} n_\mu (s).
$$
Then we get for the residue of the Eisenstein series
$$
E'^P (\varphi, \rho + \lambda) = \sum\limits_\mu (C_\mu \varphi)_{ - \rho - \lambda} + \sum\limits_\mu \left(\frac{dt}{t} \wedge N_\mu (\varphi) \right)_{ - \rho - \lambda}
$$
Moreover it follows from the scalar product formula (comp. \cite{art5-key8}, Chap, IV, \S 8) that $E' (\varphi,\rho + \lambda)$ is square integrable and that 
\begin{equation}
|| E' (\varphi,\lambda + \rho) ||^2_2 = c \langle\varphi,C_\lambda \varphi \rangle  = c \int\limits_{X (1 / \Gamma \cap P_\infty)}  (\varphi, \overline{\ast_1 \circ \#_1 C_\lambda \varphi} \label{art5-eq4.4}
\end{equation}
(for notation comp. (1.7)) where $c$ is a positive constant. We have to observe that $C_{\mu} \varphi$ for $\mu \neq \lambda$ and $\dfrac{dt}{t} \wedge N_\mu$ are orthogonal to $\varphi$.

A theorem of Andreotti and Vesentini  (comp. \cite{art5-key1}, Prop. 7 and \cite{art5-key5}, Prop. 3.20) tells us that $E'(\varphi, \lambda + \rho)$ is a closed form. Then it follows from formula \eqref{art5-eq4.1} that the forms $C_\mu \varphi$ are closed. We write $C_\lambda \varphi = \tilde{C}_\lambda \varphi + H$ where $\Delta_M \tilde{C}_\lambda \varphi = 0$ and where $H$ is a sum of eigenvectors to nonzero eigenvalues of $\Delta_M$. Then we have
\begin{equation*}
[C_\lambda \varphi] = [\tilde{C_\lambda} \varphi] \text{ and } \langle \varphi, \tilde{C}_\lambda\rangle  = \langle \varphi, C_\lambda, \varphi\rangle. 
\tag{$4.4'$}\label{art5-eq4.4'}
\end{equation*}

We define $\tilde{C}_\lambda [\varphi] = [\tilde{C}_\lambda, \varphi]$.

It follows from \eqref{art5-eq4.4} and \eqref{art5-eq4.4'} that $\tilde{C}_\lambda$ is a positive selfadjoint positive operator and therefore we have an orthogonal decomposition
$$
H^m (\partial V, E) = \ker (\tilde{C}_\lambda) \oplus \im (\tilde{C}_\lambda).
$$
Another consequence of \eqref{art5-eq4.4} and \eqref{art5-eq4.4'} is 

%\setcounter{lemma}{4}
\begin{lemma}\label{art5-lem4.5}
If $\varphi \in \ker (\tilde{C}_\lambda)$ then $E(\varphi, s)$ is holomorphic at the point $\rho + \lambda$.
\end{lemma}

We are\pageoriginale now ready to state and to prove the first part of our main result.

\setcounter{subsection}{6}
\begin{subtheorem}\label{art5-subthm4.6.1}
Let us assume that $\lambda > - \rho$. Then the form $E' (\varphi, \rho + \lambda)$ is closed for all $[\varphi] \in H^m_\lambda (\partial V, E)$ and 
$$
[E' (\varphi , \rho + \lambda)] |_{\partial V} = \tilde{C}_\lambda [\varphi].
$$
\end{subtheorem}

\begin{proof}
We have already seen that $E'(\varphi, \rho +\lambda)$ is closed. We get from \eqref{art5-eq4.1}
\begin{align*}
0  & = dE'^P (\varphi , \rho + \lambda) = \sum\limits_\mu (\mu - \lambda) \dfrac{dt}{t} \wedge (C_\mu \varphi)_{- \rho - \lambda^-}\\
& - \sum\limits_\mu \left(\frac{dt}{t} \wedge d N_{\mu} \varphi \right)_{-\rho - \lambda} + \sum\limits_\mu d (C_\mu \varphi)_{-\rho - \lambda}.
\end{align*}
This yields that $V_\mu \varphi$ is a boundary for $\lambda \neq \mu$, and that implies the last assertion of the theorem.

The next case which we shall consider is $\lambda < - \rho$. We have the isomorphism
$$
\ast_1 \circ \#_1: H^N_\lambda  (\partial V, E) \longrightarrow H^{N-m-1}_{-2\rho -\lambda} (\partial V, E^\ast).
$$
Since $-2 \rho - \lambda > -\rho$  we can decompose the right hand side 
$$
H^{N - m -1}_{-2 \rho -\lambda} (\partial V, E) = \ker (\tilde{C}_{-2\rho - \lambda}) + \im (\tilde{C}_{-2\rho -\lambda}).
$$
We put 
$$
F_\lambda = \ast_1 \circ \#_1 \ker (\tilde{C}_{-2\rho -\lambda}).
$$
\end{proof}

\begin{subtheorem}\label{art5-subthm4.6.2}
If $\lambda < - \rho$, and if $[\varphi] \in F_\lambda$ then $E(\varphi, s)$ is holomorphic at $- \rho -\lambda$, Moreover the form $E (\varphi, - \rho - \lambda)$ is closed. The restriction of $[E (\varphi, -\rho - \lambda)]$ to the boundary is given by
$$
[E(\varphi, - p - \lambda)] |_{\partial_V} = [\varphi] + [c_{-2\rho -\lambda} (-\rho - \lambda) \varphi].
$$
\end{subtheorem}

\begin{proof}
If $[\varphi] \in F_\lambda$, then we have by definition $\#_1 \circ \ast_1 \varphi \in \ker (\tilde{C}_{-2\rho -\lambda})$. We have to apply the formula \eqref{art5-eq4.3} twice. The first time we substitute $\#_1 \circ \ast_1 \varphi$ for $\varphi$ and get
$$ 
E^P(\varphi, s) =\frac{1}{-s + p +\lambda} \bigdot dE^P  (\#_1 \circ \ast_1 \varphi, s). 
$$\pageoriginale 
Now we obtain from Lemma \ref{art5-lem4.5} that the right-hand side is holomorphic at $-\rho -\lambda$, and this proves that $E(\varphi, s)$ is holomorphic at $-\rho-\lambda$. The second time we apply \eqref{art5-eq4.3} to $\varphi$ itself; again we get from Lemma \eqref{art5-lem4.5} that $E(\varphi, - \rho -\lambda)$ is closed. The rest is clear, the arguments are exactly the same as in the proof of Theorem \eqref{art5-subthm4.6.1}.

The last case is $\lambda = - \rho$. We know from \cite{art5-key8}, Chap. IV, \S 7 that $E(\varphi, s)$ is holomorphic at $s =0$. Moreover we have
$$
\#_1 \circ \ast_1: H^m_{-\rho} (\partial V,E) \longrightarrow H^{N-m-1}_{-\rho} (\partial V, E^\ast)
$$
and it follows from \eqref{art5-eq4.3} that $E (\varphi, 0)$ is always closed. Therefore we have to check the restriction of $[E(\varphi, 0)]$ to the boundary $\partial V$.

When we introduced the Eisenstein series we embedded the cohomology of the boundary into a bigger space $\bH^{(m)}$ and we have the diagram
$$ 
\xymatrix{
c(0) : H^m (\partial V, E) \ar[r] \ar[d] & \sH{(m)} \ar@{=}[d]\\
c(0) : \sH^{(m)} \ar[r] & \sH^{(m)}  .
}
$$
The functional equation for the Eisenstein series tells us that $c(0)^2 = id$ (comp. \cite{art5-key8}, Chap. IV, \S 6). Therefore we can decompose with respect to the eigenvalues $\pm 1$  and get
$$
\bH^{(m)} = \bH^{(m)}_+ \oplus \bH^{(m)}_-.
$$
It is clear that for $\psi\in\bH^{(m)}_+$
$$
E^P (\psi, 0) = 2 (\psi)_0
$$
and for $\psi \in \bH^{(H)}_-$ we have $E (\psi, 0) =0$. We claim
\end{proof}

\begin{subtheorem}\label{art5-sublem4.6.3}
We have
$$
H^m_{-\rho} (\partial V, E)'_-= H^m_{-\rho} (\partial V, E)_+ \oplus H^m_{-\rho} (\partial V, E)_-
$$
where 
$$
H^m_{-\rho} (\partial V, E)_{\pm} =\bH^{(m)}_{\pm} \cap H^m_{-\rho} (\partial V, E).
$$
The image\pageoriginale of the map
$$
\proj \circ \; r :H^m (X/ \Gamma, E) \longrightarrow H^m_{-\rho} (\partial V,E)
$$
is $H^m_{-\rho} (\partial V, E)_+$ and for $[\varphi] \in H^m_{-\rho} (\partial V, E)_+$ we have
$$
[E(\varphi, 0)] |_{\partial V} =  2 [\varphi].
$$
\end{subtheorem}

\begin{proof}
Only the first statement has to be proved. We consider the pairing
$$
H^m_{-\rho} (\partial V,E) \times H^{N-m-1}_{-\rho} (\partial V, E^\ast) \longrightarrow \bC
$$
which is given by 
$$
\langle[\varphi], [\psi] \rangle \int\limits_{\partial V}  (\varphi, \psi)
$$
(comp. \S \ref{art5-sec1}). This mapping is nondegenerate and is easy to see that for classes $[\varphi]$ and $[\psi]$ which are restrictions of classes on $X/ \Gamma$ we have
$$
\langle [\varphi], [\psi] \rangle = 0.
$$

If $R$ (\resp $S$) is the space of classes in 
$$
H^\ast_{-\rho} (\partial V, E) (\text{\resp  } H^{N-m-1}_{-\rho} (\partial V, E^\ast))
$$ 
which are restriction of classes on $X/ \Gamma$ then we see that $R$ and $S$ are orthogonal with respect to $\langle \;\rangle$ and therefore 
$$
\dim R + \dim S \leqslant \dim H^m_{-\rho} (\partial V, E) = \dim H^{N-m-1}_{-\rho} (\partial V, E^{\ast}).
$$
On the other hand we know that for any $\varphi \in H^m_{-\rho} (\partial V, E)$ (\resp $\psi \in H^{N-m-1}_{-\rho} (\partial V, E^\ast)$) we have
$$
[E(\varphi, 0)] |_{\partial V} \in R, \quad(\text{\resp  }[E(\psi, 0)] |_{\partial V } \in S).
$$
The kernel of
$$
[\varphi] \longrightarrow  [E(\varphi, 0)] |_{\partial V}, \quad (\resp [\psi] \longrightarrow [E(\psi, 0)] |_{\partial V})
$$
is
$$
\sH^{(m)}_- \cap H^m_{-\rho} (\partial V, E) = F, (\resp. \sH^{(N-m -1)}_- \cap H^{N-m-1}_{-\rho} (\partial V, E^\ast) = H).
$$
It follows from our first inequality that
$$
\dim F + \dim H \geqslant \dim H^m (\partial V, E).
$$
The formula \eqref{art5-eq4.3} tells us that
\begin{gather*}
\#_1 \circ \ast_1 (F) \subset \sH^{(N-m-1)}_+ \cap H^{(N-m-1)}_{-\rho} (\partial V, E^{\ast}) \\
\#_1 \circ \ast_1 (H) \subset \sH^{(m)}_+ \cap H^m_{-\rho} (\partial V, E).
\end{gather*}
Since\pageoriginale
\begin{align*}
& S \supset \#_1 \circ \ast_1 (F)\\
& R \supset \#_1 \circ \ast_1 (H),
\end{align*}
it follows that
\begin{align*}
R & = H^m_{-\rho} (\partial V, E)_+,\\
S & = H^m_{-\rho} (\partial V, E^\ast)_-,
\end{align*}
and that $\dim R + \dim S = \dim H^m (\partial V, E)$. This proves the theorem.
\end{proof}

\setcounter{equation}{6}
\begin{coro}\label{art5-coro4.7}
The image of the restriction map
$$
r : H^\bigdot (X/ \Gamma, E) \longrightarrow H^\bigdot (\partial V, E)
$$
is compatible with the decomposition
$$
\bigoplus_{\lambda > -\rho} (H^\bigdot_\lambda(\partial V, E) \oplus H^{\bigdot}_{-2\rho -\lambda} (\partial V, E)) \oplus H^{\bigdot_{-\rho}} (\partial V, E).
$$
If a cohomology class on the boundary is in the image of $r$ then we can find a representative in $H^\bigdot (X/ \Gamma , E)$ which is represented by a singular value of an Eisenstein series.
\end{coro}

\begin{proof}
This is clear from our previous considerations.
\end{proof}

\begin{cremarks*}
\begin{enumerate}[(1)]
\item A consequence of our results it that we have a decomposition
$$
H^\bigdot (X/ \Gamma, E) = H^\bigdot_! (X/\Gamma, E) \oplus H^\bigdot_{\inf} (X/ \Gamma , E)
$$
where $H^\ast_!$ is the image of the cohomology with compact support in the usual cohomology and where $H^\bigdot_{\inf} (X/ \Gamma, E)$ maps isomorphically to the image of $r$. We have some sort of control of this complementary space $H^\bigdot_{\inf} (X/ \Gamma, E)$ in terms of the Eisenstein series and we know how to compute this space if we have enough information concerning the behavior of the Eisenstein series at certain critical values of $s$ and to get this information we have to understand the ``intertwining operators'' $c(s)$.

\item We think that Theorem \eqref{art5-subthm4.6.2} can be sharpened since it seems to be plausible that $c_{-2\rho-\lambda} (-p -\lambda)=0$ or even $c_{-2\rho -\lambda} (s) = 0$. If this would no be the case then we could get some rationality result. Let us assume our group $G$ is defined over $\bQ$ and that $\rho$ is a re-presentation defined over $\bQ$. In this case the vector spaces $H^\bigdot (X/ \Gamma, E)$ and\pageoriginale $H^\bigdot (\partial V, E)$ have a natural $\bQ$-structure and the $\map r$ is also defined over $\bQ$. Therefore we know that the image of the restriction map is also defined over $\bQ$. Therefore we know that the image of the restriction map is also defined over $\bQ$ and hence it follows from Theorem \eqref{art5-subthm4.6.2} that $c_{-2\rho-\lambda} (-\rho-\lambda)$ is defined over $\bQ$. This seems to us would be a too strong consequence and therefore we tend to believe that $c_{-2\rho-\lambda} (-\rho -\lambda)$ is defined over $\bQ$. This seems to us would be a too strong consequence and therefore we tend to believe that $c_{-2\rho-\lambda}$ vanishes at $-\rho -\lambda$. This vanishing should follow from computations at infinity which unfortunately seem to be rather messy.

\item There are of course some cases where our information is slightly better, if we are dealing with cohomology classes of weight $\lambda> 0$. In this case $\rho + \lambda > \rho$ and $E(\varphi, s)$ is certainly holomorphic as $\rho + \lambda$. This means that the map (Theorem \eqref{art5-subthm4.6.2})
$$
H^\bigdot (X/ \Gamma , E) \longrightarrow H^\bigdot_{\lambda} (\partial V, E) \oplus H^\bigdot_{- 2 \rho -\lambda (\partial V, E)}
$$
has as its image a subspace which projects isomorphically to the second summand. If our conjecture made in Remark (2) is true then the image would be exactly the second summand.

\item The classes in $H^\bigdot_{\inf} (X/ \Gamma, E)$ are in general not square integrable unless they come from a pole, \ie, they are represented by a residue of an Eisenstein series (comp. \cite{art5-key11}, \S 5). The classes in $H^\bigdot_! (X/\Gamma , E)$ are of course all square integrable and therefore we can apply Hodge theory to investigate that part of the cohomology. We have a splitting
$$
H^\bigdot_! (x/ \Gamma, E) = H^\bigdot_p (X/ \Gamma , E) \oplus H^\bigdot_{Eis} (X/ \Gamma, E)
$$
where the first summand denotes the space which is spanned by the harmonic cusps forms and where the second space is spanned by classes which are represented by certain residues of Eisenstein series which are harmonic. Hopefully we can apply the Hodge theory verbatim to the first part; the nature of the second part seems to be unclear. We have very explicit informations if $G = SL_2 / k_2$ (comp. \cite{art5-key7}, Prop. 2.3).
\end{enumerate}
\end{cremarks*}

\begin{thebibliography}{99}
\bibitem{art5-key1}  \textsc{A. Andreotti} and \textsc{E. Vesentini}:\pageoriginale Carleman estimates for the Laplace-Beltrami equation on complex manifolds, \textit{Publ. Math., I.H.E.S., No. 25}, (1965), 313-362.

\bibitem{art5-key2} \textsc{A. Borel}: \textit{Introduction aux groups arithm\'etiques}, Hermann, Paris, (1969).

\bibitem{art5-key3} \textsc{H. Cartan} and \textsc{S. Eilenberg}: \textit{Homological Algebra,} Princeton University Press, Princeton, (1956).

\bibitem{art5-key4} \textsc{W. T. Van Est}: A generalization of the Cartan-Leray spectral sequence, \textit{Nederlands Akademic van Wetenschappen, Proceedings, Series A,} vol. 61, (1958), I, 399-405, II, 406-413.

\bibitem{art5-key5} \textsc{H. Garland}: A rigidity theorem for discrete subgroups, \textit{Trans. Amer. Math. Soc.,} 129, (1967), 1-25.

\bibitem{art5-key6} \textsc{G. Harder}: A Gauss-Bonnet formula for discrete arithmetically defined groups, \textit{Ann. Scient. \'Ec. Norm. Sup.,} t. 4, (1971), 409-455.

\bibitem{art5-key7} \textsc{G. Harder}: On the cohomology of $SL_2 (\cO)$ to appear in the Proceedings of the Summer School on ``Representation Theory'', Budapest, (1971).

\bibitem{art5-key8} \textsc{Harish-Chandra}: Automorphic forms on semisimple Lie groups, \textit{Springer Lect\`ure Notes,} vol. 62, (1968).

\bibitem{art5-key9} \textsc{B. Kostant}: Lie algebra cohomology and the generalized Borel-Weil theorem, \textit{Ann. of Math.,} vol. 74, (1961), 329-387.

\bibitem{art5-key10} \textsc{Y. Matsushima} and \textsc{S. Murakami}: On vector bundle valued harmonic forms and automorphic forms on symmetric Riemannian manifolds,  \textit{Ann. of Math.,} vol. 78, (1963), 365-416.

\bibitem{art5-key11} \textsc{M. S. Raghunathan}: Cohomology of arithmetic subgroups of algebraic groups, \textit{Ann. of Math.}, vol. 78, (1968) , 279-304.

\bibitem{art5-key12} \textsc{G. de Rham}: \textit{Var\'iet\'es Diff\'erentiables,} Paris, Hermann, (1955).
\end{thebibliography}

\bigskip

\noindent
The Institute for Advanced Study

\noindent
Princeton, New Jersey

\medskip
\noindent
{and}
\medskip

\noindent
53 Bonn

\noindent
Mathematisches Institut

\noindent
Wegelerstr. 10



