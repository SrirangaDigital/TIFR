\chapter[\textsc{P. A. Griffiths~:} Some Results on Algebraic Cycles on Algebraic Manifolds]{SOME RESULTS ON ALGEBRAIC CYCLES ON ALGEBRAIC MANIFOLDS}\label{art08}

\begin{center}
{\em By}~~ Phillip A. Griffiths
\end{center}

\setcounter{pageoriginal}{92}
\setcounter{section}{-1}
\section{Introduction.}\label{art08-sec0}\pageoriginale

\lhead[\thepage]{\textit{Some Results on Algebraic Cycles on Algebraic Manifolds}}
\rhead[\textit{P. A. Griffiths}]{\thepage}

The basic problem we have in mind is the classification of the algebraic cycles on an algebraic manifold $V$. The first invariant is the homology class $[Z]$ of a cycle $Z$ on $V$; if $Z$ has codimension $q$, then $[Z]\in H_{2n-2q}(V,\bfZ)(n=\dim V)$. By analogy with divisors (c.f. \cite{art08-key18}), and following Weil \cite{art08-key22}, if $[Z]=0$, then we want to associate to $Z$ a point $\phi_{q}(Z)$ in a complex torus $T_{q}(V)$ naturally associated with $V$. The classification question then becomes two problems :
\begin{itemize}
\item[(a)] Find the image of $\phi_{q}$ (inversion theorem);

\item[(b)] Find the equivalence relation given by $\phi_{q}$ (Abel's theorem).
\end{itemize}

We are unable to make substantial progress on either of these. On the positive side, our results do cover the foundational aspects of the problem and give some new methods for studying subvarieties of general codimension. In particular, the issue is hopefully clarified to the extent that we can make a guess as to what the answers to (a) and (b) should be. This supposed solution is a consequence of the (rational) Hodge conjecture; conversely, if we know (a) and (b) in suitable form, then we can construct algebraic cycles.

We now give an outline of our results and methods.

For the study of $q$-codimensional cycles on $V$, Weil introduced certain complex tori $J_{q}(V)$; as a real torus,
$$
J_{q}(V)=H^{2q-1}(V,\bfR)/H^{2q-1}(V,\bfZ).
$$
These tori are abelian varieties. We use the same real torus, but with a different complex structure (c.f. \S\S\ref{art08-sec1},\ref{art08-sec2}); these tori $T_{q}(V)$ vary holomorphically with $V$ (the $J_{q}(V)$ don't) and have the necessary functorial properties. In general, they are not abelian varieties, but have an $r$-convex polarization \cite{art08-key9}. However, the polarizing line bundle is positive on the ``essential part'' of $T_{q}(V)$.\pageoriginale Also, $T_{1}(V)=J_{1}(V)$ (= Picard variety of $V$) and $T_{n}(V)=J_{n}(V)$ (= Albanese variety of $V$).

Let $\Sigma_{q}$ be the cycles of codimension $q$ algebraically equivalent to zero on $V$. There is defined a homomorphism $\phi_{q}:\Sigma_{q}\to T_{q}(V)$ by $\phi_{q}(Z)=\left[\begin{smallmatrix}:\\ \int\limits_{\Gamma}\omega^{\alpha}\\ :\end{smallmatrix}\right]$/(periods), where $\Gamma$ is a $2n-2q+1$ chain with $\partial \Gamma=Z$ and $\omega^{1},\ldots,\omega^{m}\in H^{2n-2q+1}(V,\bfC)$ are a basis for the holomorphic one-forms on $T_{q}(V)$. Using the torus $T_{q}(V)$, this mapping is holomorhic and depends only on the complex structure of $V$ (c.f. \S\ref{art08-sec3}); this latter result follows from a somewhat interesting theorem on the cohomology of K\"ahler manifolds given in the Appendix following \S\ref{art08-sec10}. In \S\ref{art08-sec3}, we also give the infinitesimal calculation of $\phi$; the transposed differential $\phi^{*}$ is essetially the Poincar\'e residue operator (c.f. \eqref{art08-sec3-eq3.8}). For hypersurfaces $(q=1)$, the Poincar\'e residue and geometric residue operators coincide, and the (well-known) solutions to (a) and (b) follow easily.

In \S\ref{art08-sec4}, we relate the functorial properties of the tori $T_{q}(V)$ to geometric operations on cycles. The expected theorems turn up, but the proofs require some effort. We use the calculus of differential forms with singularities. In particular, the notion of a residue operator associated to an irreducible subvariety $Z\subset V$ appears. Such a residue operator is given by a $C^{\infty}$ form $\psi$ on $V-Z$ such that: (1) $\psi$ is of type $(2q-1,0)+\cdots+(q,q-1)$; (2) $\partial \psi=0$ and $\overline{\partial}\psi$ is a $C^{\infty}$ $(q,q)$ form on $V$ which gives the Poincar\'e dual $\mathscr{D}[Z]\in H^{q,q}(V)$ of $[Z]$; and (3) for $\Gamma$ a $2n-k$ chain on $V$ meeting $Z$ transversely and $\eta$ a smooth $2q-k$ form on $V$, we have the residue formula: $\lim\limits_{\epsilon\to 0} \int\limits_{\Gamma\cdot (\partial T_{\epsilon})}\psi \wedge \eta=\int\limits_{\Gamma\cdot Z}\eta$, where $T_{\epsilon}$ is the $\epsilon$-neighborhood of $Z$ in $V$. The construction of residue operators is done using Hermitian differential geometry; the techniques involved give a different method of approaching the theorem of Bott-Chern \cite{art08-key4}. One use of the residue operators is the explicit construction, on the form level, of the Gysin homomorphism $i_{*}:H^{k}(Z)\to H^{2q+k}(V)$ where we can keep close track of the complex\pageoriginale structure (c.f. the Appendix to \S\ref{art08-sec4}, section (e)). This is useful in proving the functorial properties.

In \S\ref{art08-sec5} we give one of our basic constructions. If $[Z]=0$ in $H_{2n-2q}\break (V,\bfZ)$, and if $\psi$ is a residue operator for $Z$, we may assume that $d\psi=0$. Then $\psi$ is the general codimensional analogue of a logarithmic integral of the third kind (\cite{art08-key17}). The trouble is that $\psi$ has degree $2q-1$ and so cannot directly be integrated on $V$ to give a function. However, $\psi$ can be integrated on the set of algebraic cycles of dimension $q-1$ on $V$. We show then that $Z$ defines a divisor $D(Z)$ on a suitable Chow variety associated to $V$, and that $\psi$ induces an integral of the third kind on this Chow variety. The generalization of Abel's theorem we give is then : $D(Z)$ is linearly equivalent to zero if $\phi_{q}(Z)=0$ in $T_{q}(V)$. As in the classical case, the proof involves a bilinear relation between $\psi$ and the holomorphic differentials on $T_{q}(V)$. Also, as mentioned above, the ``only if'' part of this statement (which is trivial when $q=1$) depends upon the Hodge problem. Our conclusion from this, as regards problem (b) is: The equivalence relation defined by $\phi$ should be linear equivalence on a suitable Chow variety. In particular, we don't see that this equivalence should necessarily be rational equivalence on $V$.

In \S\ref{art08-sec6} we give our main result trying to determine the image of $\phi$. To explain this formula (given by \eqref{ART08-SEC6-EQ6.8} in \S\ref{art08-sec6}) we let $\{\bfE_{\lambda}\}$ be a holomorphic family of holomorphic vector bundles over $V$. We denote by $Z_{q}(\bfE_{\lambda})$ the $q^{\text{th}}$ Chern class in the rational equivalence ring, so that $\{Z_{q}(\bfE_{\lambda})\}$ gives a family of codimension $q$ cycles on $V$. Our formula gives a method for calculating the infinitesimal variation of $Z_{q}(\bfE_{\lambda})$ in $T_{q}(V)$; it involves the curvature matrix $\Phi$ in $\bfE_{\lambda}$ and the Kodaira-Spencer class giving the variation of $\bfE_{\lambda}$.

The crux of this formula is that it relates the Poincar\'e and geometric residues in higher codimension. The proof involves a somewhat delicate computation using forms with singularities and the curvature in $\bfE_{\lambda}$. In \S\ref{art08-sec8} we give the argument for the highest Chern class of an ample bundle. In \S\ref{art08-sec7} it is shown that we need only check the theorem for ample bundles; however, in general the Chern classes, given by Schubert cycles, will be singular, except of course for\pageoriginale the highest one. So, to prove our formula in general we give in \S\ref{art08-sec9} an argument, which is basically differential-geometric, but which requires that we examine the singularities of $Z_{q}(\bfE_{\lambda})$.

The reason for proving such a formula is that the Chern classes $Z_{q}(\bfE_{\lambda})$ generate the rational equivalence ring on $V$. So, if we could effectively use the main result, we could settle problem (a). For example, for line bundles $(q=1)$, the mapping in question is the identity; this gives once more the structure theorems of the Picard variety. However, we are unable to make effective use of the formula, except in rather trivial cases, so that our result has more of an intrinsic interest and illuminating proof than the applications we would like.

In the last part of \S\ref{art08-sec9} we give an integral-geometric argument, using the transformation properties of the tori $T_{q}(V)$ and the relation of these properties to cycles, of the main formula \eqref{ART08-SEC6-EQ6.8}.

Finally, in \S\ref{art08-sec10} we attempt to put the problem in perspective. We formulate possible answers to (a) and (b) and show how these would follow if we knew the Hodge problem. The construction of algebraic cycles, assuming the answer to (a) and (b), is based on a generalization of the Poincar\'e normal functions (c.f. \cite{art08-key19}) and will be given later.

To close this introduction, I would like to call attention to the paper of David Lieberman \cite{art08-key20} on the same subject and which contains several of the results given below. Lieberman uses the Weil Jacobians \cite{art08-key22} to study intermediate cycles; however, his results are equally valid for the complex tori we consider. His methods are somewhat different from the ones used below; many of our arguments are computational whereas Lieberman uses functorial properties of the Weil mapping and his proofs have an algebro-geometric flavor.

More specifically, Lieberman proves the functorial properties of the Weil mapping in somewhat more precise form than given below. Thus his results include the functorial properties \eqref{art08-sec4-eq4.2} (the hard one arising from the Gysin map) and \eqref{art08-sec4-eq4.14} (the easy one using restriction of cohomology), as well as (\ref{art08-sec4-rem4.12}) which we only state conjecturally. From\pageoriginale the functorial properties and the fact that the Weil mapping is holomorphic for codimension one, Lieberman concludes the analyticity of this mapping (given by \eqref{art08-sec3-eq3.2}) in general. (It is interesting to contrast his conceptual argument with the computational one given in \cite{art08-key9}.) In summary, Lieberman's results include the important general properties of the intermediate Jacobians given in \S\ref{art08-sec1}-\ref{art08-sec4} below. Also, the conjectured Abelian variety for which the inversion theorem ((a) above) holds was found by Lieberman using his Poincar\'e divisor, and the proof of \eqref{art08-sec10-eq10.4} is due to Lieberman.

The reason for this overlap is because this manuscript was done in Berkeley, independently but at a later time than Lieberman (most of his results are in his M. I. T. thesis). By the time we talked in Princeton, this paper was more or less in the present form and, because of the deadline for these proceedings, could not be rewritten so as to avoid duplication.


%\minitoc

\medskip
\begin{longtable}{lp{9cm}}
\multicolumn{2}{c}{\large\bf Table of Contents}\\[4pt]
\hline
\ref{art08-sec0}. & {\em Introduction.}\\[5pt]
\ref{art08-sec1}. & {\em Complex Tori associated to Algbraic Manifolds.}\\[5pt]
\ref{art08-sec2}. & {\em Special Complex Tori.}\\
                  & {(In these two sections, we give the basic properties of the tori $T_{q}(V)$.)}\\[5pt]
\ref{art08-sec3}. & {\em Algebraic Cycles and Complex Tori.}\\
                  & {(We give the mapping $\phi_{q}:\Sigma_{q}\to T_{q}(V)$, show that it is holomorphic, compute its differential, and examine some special cases.)}\\[5pt]
\ref{art08-sec4}. & {\em Some Functorial Properties.}\\
                  & {(The transformation properties of the tori $T_{q}(V)$ are related to geometric operations on cycles. The residue operators are used here, and they are constructed in the Appendix to \S\ref{art08-sec4}.)}\\[5pt]
\ref{art08-sec5}. & {\em Generalizations of the Theorems of Abel and Lefschetz.}\\
                  & (Here we show how the equivalence relation given by $\phi_{q}$ relates to linear equivalence on Chow varieties attached to $V$. The generalized bilinear relations are given also.)\\[5pt]
\ref{art08-sec6}. & {\em Chern Classes and Complex Tori.}\\
                  & (We define the periods of a holomorphic vector bundle and give the basic formula \eqref{ART08-SEC6-EQ6.8} for computing the infinitesimal variation of these periods.)\\[5pt]
\ref{art08-sec7}. & {\em Properties of the Mapping $\zeta$ in (6.8).}\\
                  & (Here we discuss the formula \eqref{ART08-SEC6-EQ6.8} and prove it for $q=1$. It is also shown that it suffices to verify it for ample bundles.)\\[5pt]
\ref{art08-sec8}. & {\em Proof of \eqref{ART08-SEC6-EQ6.4} for the Highest Chern Class.}\\
                  & (This is the basic integral-differential-geometric argument relating the Poincar\'e and geometric residues via the Chern forms.)\\[5pt]
\ref{art08-sec9}. & {\em Proof of \eqref{ART08-SEC6-EQ6.8} for the General Chern Classes.}\\
                  & (Here we discuss the singularities of the Chern classes and show how to extend Poincar\'e residues and the argument of \S\ref{art08-sec8} to the general case.)\\[5pt]
\ref{art08-sec10}. & {\em Concluding Remarks.}\\
                   & (We formulate what we feel are reasonable solutions to problems (a) and (b) above, and discuss what is needed to prove these.)\\[5pt]
\multicolumn{2}{l}{\em Appendix: \ref{art08-app-A} Theorem on the Cohomology of Algebraic Manifolds.}\\[4pt]
\hline
\end{longtable}\pageoriginale



\section{Complex Tori associated to Algebraic Manifolds.}\label{art08-sec1}

Let $V$ be an $n$-dimensional algebraic manifold and $\bfL\to V$ the {\em positive line bundle} giving the polarization on $V$. The {\em characteristic class} $\omega \in H^{1,1}(V)\cap H^{2}(V,\bfZ)$ may be locally written as $\omega=\dfrac{i}{2}\{\Sigma g_{\alpha\overline{\beta}}dz^{\alpha}\wedge d\overline{z}^{\beta}\}$ where\pageoriginale $\sum\limits_{\alpha,\beta}g_{\alpha\overline{\beta}}dz^{\alpha}d\overline{z}^{\beta}$ gives a K\"ahler metric on $V$.

According to Hodge, the cohomology group $H^{s}(V,\bfC)$ decomposes as a sum:
$$
H^{s}(V,\bfC)=\sum\limits_{p+q=s}H^{p,q}(V),
$$
where $H^{p,q}(V)$ are the cohomology classes represented by differential forms of type $(p,q)$. Under complex conjugation, $\overline{H^{p,q}(V)}=H^{q,p}(V)$. 

Consider now the cohomology group
\begin{equation*}
H^{2n-2q+1}(V,\bfC)=\sum\limits^{2n-2q+1}_{r=0}H^{2n-2q+1-r,r}(V)\tag{1.1}\label{art08-sec1-eq1.1}
\end{equation*}
and choose a complex subspace $S\subset H^{2n-2q+1}(V,\bfC)$ such that
\begin{equation*}
S\cap \overline{S}=0\text{~~ and ~~} S+\overline{S}=H^{2n-2q+1}(V,\bfC);\tag{1.2}\label{art08-sec1-eq1.2}
\end{equation*}
\begin{equation*}
S=\sum\limits^{2n-2q+1}_{r=0}S\cap H^{2n-2q+1-r,r}(V)\tag{1.3}\label{art08-sec1-eq1.3}
\end{equation*}
(i.e. $S$ is compatible with the {\em Hodge decomposition} \eqref{art08-sec1-eq1.1});
\begin{equation*}
H^{n-q+1,n-q}(V)\subset S.\tag{1.4}\label{art08-sec1-eq1.4}
\end{equation*}
Under these conditions we shall define a complex torus $T_{q}(S)$ such that the space of holomorphic $1$-forms on $T_{q}(S)$ is just $S$. There are three equivalent definitions of $T_{q}(S)$.

\medskip
\noindent
{\bf Definition \thnum{1}.\label{art08-sec1-defi1}}
Choose a basis $\omega^{1},\ldots,\omega^{m}$ for $S$ and define the lattice $\Gamma(S)\subset \bfC^{m}$ of all column vectors 
$$
\pi_{\gamma}=\left[\begin{smallmatrix} \int\limits_{\gamma}\omega^{1}\\ \vdots\\ \int\limits_{\gamma}\omega^{m}\end{smallmatrix}\right]
$$ 
where $\gamma\in H_{2m-2q+1}(V,\bfZ)$. To see that $\Gamma(S)$ is in fact a lattice, we observe tht rank $(H_{2n-2q+1}(V,\bfZ))=2m$ and so we must show :
\smallskip

If $\gamma_{1},\ldots,\gamma_{k}\in H_{2n-2q+1}(V,\bfZ)$ are linearly independent over $\bfR$, then $\pi_{\gamma_{1}},\ldots,\pi_{\gamma_{k}}$ are also linearly independent over $\bfR$. But if 
$$
\sum\limits^{k}_{j=1}\alpha_{j}\int\limits_{\gamma_{j}}\omega^{\alpha}=\int\limits_{\sum\limits^{k}_{j=1}\alpha_{j}\gamma_{j}} \omega^{\alpha}=0
$$\pageoriginale
then also
$$
\int\limits_{\sum\limits^{k}_{j=1}\alpha_{j}\gamma_{j}}\overline{\omega}^{\alpha}=0,\quad\text{since}\quad \overline{\alpha}_{j}=\alpha_{j}.
$$
This says that $\sum\limits^{k}_{j=1}\alpha_{j}\gamma_{j}$ is orthogonal to $S+\overline{S}=H^{2n-2q+1}(V,\bfC)$ and so $\sum\limits^{k}_{j=1}\alpha_{j}\gamma_{j}=0$.

If then $T_{q}(S)=\bfC^{m}/\Gamma(S)$, then $T_{q}(S)$ is a complex torus associated to $S\subset H^{2n-2q+1}(V,\bfC)$.

\medskip
\noindent
{\bf Definition \thnum{2}.\label{art08-sec1-defi2}}
Let $H_{2n-2q+1}(V,\bfC)=H^{2n-2q+1}(V,\bfC)^{*}$ be the dual space of $H^{2n-2q+1}(V,\bfC)$ so that $0\to S\to H^{2n-2q+1}(V,\bfC)$ dualizes to 
\begin{equation*}
0\leftarrow S^{*}\leftarrow H_{2n-2q+1}(V,\bfC).\tag{1.5}\label{art08-sec1-eq1.5}
\end{equation*}
Then $H_{2n-2q+1}(V,\bfZ)\subset H_{2n-2q+1}(V,\bfC)$ projects onto a lattice $\Gamma(S)\subset S^{*}$ and $T_{q}(S)=S^{*}/\Gamma(S)$. (Proof that Definition \ref{art08-sec1-defi1} = Definition \ref{art08-sec1-defi2}: choosing a basis $\omega^{1},\ldots,\omega^{m}$ for $S$ makes $S^{*}\cong \bfC^{m}$ by $l(\omega^{\alpha})=l_{\alpha}$ where $=\left[\begin{smallmatrix} l_{1}\\ \vdots \\ l_{m}\end{smallmatrix}\right]$. Then $\pi_{\gamma}(\omega^{\alpha})=\int\limits_{\gamma}\omega^{\alpha}=\langle \omega^{\alpha},\gamma\rangle$ so that $\Gamma(S)$ is the same lattice in both cases.)

\medskip
\noindent
{\bf Definition \thnum{3}.\label{art08-sec1-defi3}}
Let $\mathscr{D}:H_{2n-2q+1}(V,\bfC)\to H^{2q-1}(V,\bfC)$ be the {\em Poincar\'e duality isomorphism} and $0\leftarrow S^{*}\leftarrow H^{2q-1}(V,\bfC)$ the sequence corresponding to \eqref{art08-sec1-eq1.5}, $\Gamma(S)\subset S^{*}$ the lattice corresponding to $\Gamma(S)$. Then $T_{q}(S)=S^{*}/\Gamma(S)$.
\smallskip

Observe that if $H^{r,s}(V)\subset S$, then $H^{n-r,n-s}(V)\subset S^{*}$ and vice-versa. In particular, $S$ is the dual space of $S^{*}$ by:
\begin{equation*}
\langle \omega,\phi\rangle =\int\limits_{V}\omega\wedge \phi\qquad (\omega\in S, \ \ \phi\in S^{*}).\tag{1.6}\label{art08-sec1-eq1.6}
\end{equation*}
Thus\pageoriginale $S\cong H^{1,0}(T_{q}(S))$, the space of holomorphic 1-forms on $T_{q}(S)$.


\section{Special Complex Tori.}\label{art08-sec2}

%\mtcaddsection[Raghu]
\addcontentsline{lof}{section}{Raghu}

%\addcontentsline{toc}{subsection}{(Sriranga Digitals)}

The choice of $S\subset H^{2n-2q+1}(V,\bfC)$ depends on the properties we want $T_{q}(S)$ to have; the results on algebraic cycles will be essentially independent of $S$ because of condition \eqref{art08-sec1-eq1.4}.

\medskip
\noindent
{\bf Example \thnum{1}.\label{art08-sec2-exam1}}
We let $S=\sum\limits^{n-q}_{r=0}H^{n-q+r+1,n-q-r}$ and set $T_{q}(S)=T_{q}(V)$. These tori have been studied in \cite{art08-key9}, where it is proved that $T_{q}(V)$ varies holomorphically with $V$.
\smallskip

The trouble with $T_{q}(V)$ is that it is not polarized in the usual sense; however, for our purposes we can do almost as well as follows. Recall \cite{art08-key23} that there is defined on $H^{2q-1}(V,\bfC)$ a quadratic form $Q$ with the following properties:
\begin{equation*}
\left.
\begin{array}{cl}
{\rm(a)} & Q \text{~ is skew-symmetric and integral on ~}\\
         & H^{2q-1}(V,\bfZ)\subset H^{2q-1}(V,\bfC);\\[4pt]
{\rm(b)} & Q(H^{r,s}(V),\overline{H^{r',s'}(V)})=0\text{~ if~ } r\neq r', s\neq s';\\[4pt]
{\rm(c)} & Q(H^{r,s}(V),\overline{H^{r,s}(V)})\text{~ is nonsingular; and}\\[4pt]
{\rm(d)} & iQ(H^{q-1,q}(V), \ \overline{H^{q-1,q}(V)})>0.
\end{array}
\right\}\tag{2.1}\label{art08-sec2-eq2.1}
\end{equation*}
It follows that $Q(S^{*},S^{*})=0$ and that, choosing a basis $\omega^{1},\ldots,\omega^{m}$ for $S$, there is a complex line bundle $\bfL\to T_{q}(S)$ whose characteristic class $\omega(\bfL)\in H^{2}(T_{q}(S),\bfZ)\cap H^{1,1}(T_{q}(S))$ is given by
$$
\omega(\bfL)=\dfrac{i}{2\pi}\left\{\sum\limits^{m}_{\alpha,\beta=1}h_{\alpha\overline{\beta}}\omega^{\alpha}\wedge \overline{\omega}^{\beta}\right\},
$$
where the matrix $H=(h_{\alpha\overline{\beta}})=\{iQ(e_{\alpha},\overline{e}_{\beta})\}^{-1}$ and $\int\limits_{V}\omega^{\alpha}\wedge e_{\beta}=\delta^{\alpha}_{\beta}$. Diagonalizing $H$, we may write
\begin{equation*}
\omega(\bfL)=\dfrac{i}{2\pi}\left\{\sum\limits^{m}_{\alpha=1}\epsilon_{\alpha}\omega^{\alpha}\wedge \overline{\omega}^{\alpha}\right\},\tag{2.2}\label{art08-sec2-eq2.2}
\end{equation*}
where $\epsilon_{\alpha}=\pm 1$ and $\epsilon_{\alpha}=+1$ if $\omega^{\alpha}\in H^{n-q+1,n-q}(V)$. Thus we may say that :
\begin{equation*}
\text{There is a natural } r\text{-{\em convex polarization} \cite{art08-key10}~~ } \bfL\to T_{q}(V)\tag{2.3}\label{art08-sec2-eq2.3}
\end{equation*}\pageoriginale
($r$ = number of $\alpha$ such that $\epsilon_{\alpha}=-1$) and the characteristic class of $\bfL$ is positive on the translates of $H^{q-1,q}(V)$.

\medskip
\noindent
{\bf Example \thnum{2}.\label{art08-exam2}}
We let $S=\sum\limits_{r} H^{n-q+2r+1,n-q-2r}$ and set $J_{q}(V)=T_{q}(S)$. This torus is Weil's {\em intermediate Jacobian} \cite{art08-key22} and from \eqref{art08-sec2-eq2.1} we find :
\smallskip

There is a natural $0$-{\em convex polarization} (= {\em positive line bundle})
\begin{equation*}
\bfK\to J_{q}(V).\tag{2.4}\label{art08-sec2-eq2.4}
\end{equation*}

Referring to \eqref{art08-sec2-eq2.2}, we let $\phi^{\alpha}=\omega^{\alpha}$ if $\epsilon_{\alpha}=+1$, $\phi^{\alpha}=\overline{\omega}^{\alpha}$ if $\epsilon_{\alpha}=-1$. Then the $\phi^{\alpha}$ give a basis for $H^{1,0}(J_{q}(V))$ and
\begin{equation*}
\omega (\bfK)=\dfrac{i}{2\pi}\left\{\sum\limits^{m}_{\alpha=1}\phi^{\alpha}\wedge \overline{\phi}^{\alpha}\right\}.\tag{2.5}\label{art08-sec2-eq2.5}
\end{equation*}

We recall \cite{art08-key23} that $H^{s}(J_{q}(V),\mathscr{O}(\bfK^{\mu}))=0$ for $\mu>0$, $s>0$ and that $H^{0}(J_{q}(V),\mathscr{O}(\bfK^{\mu}))$ has a basis $\theta_{0},\ldots,\theta_{N}$ given by {\em theta functions of weight $\mu$.}

\medskip
\noindent
{\bf Comparison of \boldmath$T_{q}(V)$ and $J_{q}(V)$.}~ In \cite{art08-key9} it is proved that there is a {\em real} linear isomorphism $\xi:T_{q}(V)\to J_{q}(V)$ such that
\begin{equation*}
\left.
\begin{array}[l]{@{}rl}
{\rm(i)} & \xi^{*}\phi^{\alpha}=\omega^{\alpha}\text{~ if~ } \epsilon_{\alpha}=+1\text{~ and~ } \xi^{*}\phi^{\alpha}=\overline{\omega}^{\alpha}\text{~ if~ } \epsilon_{\alpha}=-1;\\
{\rm(ii)} & \xi^{*}(\bfK)=\bfL;~ \text{and}\\
{\rm(iii)} & \text{if~~} \Omega_{p}=\xi^{*}(\theta_{p})\left\{\prod\limits_{\epsilon_{\alpha}=-1}\overline{\omega}^{\alpha}\right\}, \text{~ then the } \Omega_{p}\text{~ give a basis of}\\
 & H^{r}(T_{q}(V),\mathscr{O}(\bfL^{\mu})),\text{~~ and~~ }H^{s}(T_{q}(V),\mathscr{O}(\bfL^{\mu}))=0\text{~~ for}\\
& \mu>0, s\neq r.
\end{array}
\right\}\tag{2.6}\label{art08-sec2-eq2.6}
\end{equation*}

\noindent
{\bf Some Special Cases.}~ For $q=1$, $T_{1}(V)=J_{1}(V)=H^{0,1}(V)/H^{1}(V,\bfZ)$ is the {\em Picard variety} of $V$ \cite{art08-key22}. For $q=n$, $T_{n}(V)=J_{n}(V)=H^{n-1,n}(V)\break /H^{2n-1}(V,\bfZ)$ is the {\em Albanese variety} \cite{art08-key3} of $V$. For $q=2$, $T_{2}(V)=H^{1,2}(V)+H^{0,3}(V)/H^{3}(V,\bfZ)$ and $J_{2}(V)=H^{1,2}(V)+H^{3,0}(V)/H^{3}(V,\bfZ)$; this is the simplest case where $T_{q}(V)\neq J_{q}(V)$.

\medskip
\noindent
{\bf Some Isogeny Properties.}~ We let $S_{q}\subset H^{2n-2q+1}(V,\bfC)$ be the subspace corresponding to either $T_{q}(V)$ or $J_{q}(V)$ constructed above, and we let $S^{*}_{q}\subset H^{2q-1}(V,\bfC)$ be the dual space. Then we have 
\[
\left.
\begin{array}{c}
\xymatrix{
H^{2q-1}(V,\bfC)\ar[r] & S^{*}_{q}\ar[r] & 0\\
H^{2q-1}(V,\bfZ)\ar[u] & &}
\end{array}\right\}
\]\pageoriginale
and $T_{q}(V)$ or $J_{q}(V)$ is given as $S^{*}_{q}/\Gamma^{*}_{q}$ where $\Gamma^{*}_{q}$ is the projection of $H^{2q-1}(V,\bfZ)$ on $S^{*}_{q}$.

Suppose now that $\psi\in H^{p,p}(V)\cap H^{2p}(V,\bfZ)$. Then, by cup-product, we have induced~:
\begin{equation*}
\vcenter{
\xymatrix@=1.2cm{
S^{*}_{q}\ar[r]^-{\psi} & S^{*}_{p+q}\\
\Gamma^{*}_{q}\ar[r]^-{\psi}\ar[u] & \Gamma^{*}_{p+q}\ar[u]
}}\tag{2.7}\label{art08-sec2-eq2.7}
\end{equation*}
which gives $\psi:T_{q}(V)\to T_{p+q}(V)$ or $\psi:J_{q}(V)\to J_{p+q}(V)$. We want to give this mapping in terms of the coordinates given in the first definition of paragraph 1.

Let $\omega^{1},\ldots,\omega^{m}=\{\omega^{\alpha}\}$ be a basis for $S_{q}\subset H^{2n-2q+1}(V,\bfC)$ and $\phi^{1},\ldots,\phi^{k}=\{\phi^{\rho}\}$ be a basis for $S_{p+q}\subset H^{2n-2p-2q+1}(V,\bfC)$. Then $\psi\wedge \phi^{\rho}=\sum\limits_{\alpha}m_{\rho\alpha}\omega^{\alpha}$ and 
\begin{equation*}
\int\limits_{\gamma}\psi \wedge \phi^{\rho}=\int\limits_{\gamma\cdot \mathscr{D}(\psi)}\phi^{\rho},\tag{2.8}\label{art08-sec2-eq2.8}
\end{equation*}
where $\mathscr{D}(\psi)\in H_{2n-2p}(V,\bfZ)$ and $\gamma\in H_{2n-2q+1}(V,\bfZ)$. Now $M=(m_{\rho\alpha})$ is a $k\times m$ matrix giving $\psi:\bfC^{m}\to \bfC^{k}$ by $\psi \left(\begin{smallmatrix} :\\ \lambda^{\alpha} \\ :\end{smallmatrix}\right)=\left(\begin{smallmatrix} :\\ \sum\limits^{m}_{\alpha=1} m_{\rho\alpha}\lambda^{\alpha}\\ :\end{smallmatrix}\right)$
 and
$$
\psi\left(\begin{matrix} 
:\\ \int\limits_{\gamma}\omega^{\alpha}
\end{matrix}\right)
=
\left(\begin{matrix}
:\\
\Sigma m_{\rho\alpha}\int\limits_{\gamma}\omega^{\alpha}\\
:
\end{matrix}\right)
=
\left(\begin{matrix}
:\\
\int\limits_{\gamma}\psi \wedge \phi^{\rho}\\
:
\end{matrix}\right)
=
\left(\begin{matrix}
:\\
\int\limits_{\gamma\cdot \mathscr{D}(\psi)} \phi^{\rho}\\
:
\end{matrix}\right),
$$
so that $\psi(\Gamma_{q})\subset \Gamma_{p+q}$. It follows that, in terms of the coordinates in Definition \ref{art08-sec1-defi1}, $\psi$ is given by the matrix $M$.

Now\pageoriginale suppose that $\psi:H^{2q-1}(V,\bfC)\to H^{2p+2q-1}(V,\bfC)$ is an isomorphism. Then $\psi:S^{*}_{q}\cong S^{*}_{p+q}$ and $\psi(\Gamma^{*}_{q})$ is of finite index in $\Gamma^{*}_{p+q}$. Thus $\psi:T_{q}(V)\to T_{p+q}(V)$ is an {\em isogeny}, as is also $\psi:J_{q}(V)\to J_{p+q}(V)$. Taking $\psi=\omega^{n-2q+1}$, where $\omega$ is the polarizing class, and using \cite{art08-key23}, page 75, we have :
\begin{equation*}
\left.
\begin{array}{c}
\omega^{n-2q+1}:T_{q}(V)\to T_{n-q+1}(V),\text{~~ and}\\[3pt]
\omega^{n-2q+1} : J_{q}(V)\to J_{n-q+1}(V)
\end{array}\right\}\tag{2.9}\label{art08-sec2-eq2.9}
\end{equation*}
are both isogenies for $q\leq \left[\dfrac{n+1}{2}\right]$.

Finally, using \cite{art08-key23}, Chapter IV, we have :

For $p\leq n-2q+1$, the mappings
\begin{equation*}
\left.
\begin{array}{c}
\omega^{p}:T_{q}(V)\to T_{p+q}(V),\text{~~ and}\\[4pt]
\omega^{p}: J_{q}(V)\to J_{p+q}(V)
\end{array}
\right\}\tag{2.10}\label{art08-sec2-eq2.10}
\end{equation*}
make $T_{q}(V)$ isogenous to a sub-torus of $T_{p+q}(V)$, and similarly for $J_{q}(V)$ and $J_{p+q}(V)$.

\medskip
\noindent
{\bf Some Functionality Properties.}~ Given a holomorphic mapping $f:V'\to V$, there is induced a cohomology mapping $f^{*}:H^{2q-1}(V,\bfC)\to H^{2q-1}(V',\bfC)$ with $f^{*}(S^{*}_{q}(V))\subset S^{*}_{q}(V'),f^{*}(\Gamma^{*}_{q}(V))\subset \Gamma^{*}_{q}(V')$ (using the obvious notation).

This gives
\begin{equation*}
\left.
\begin{array}{c}
f^{*}:T_{q}(V)\to T_{q}(V'),\text{~~ and}\\[4pt]
f^{*}:J_{q}(V)\to J_{q}(V').
\end{array}
\right\}\tag{2.11}\label{art08-sec2-eq2.11}
\end{equation*}

On the other hand, if $\dim V=n$ and $\dim V'=n'$, we set $k=n-n'$ and from $f_{*}:H_{2n'-2q+1}(V',\bfC)\to H_{2n-2(k+q)+1}(V,\bfC)$ we find a mapping
\begin{equation*}
\left.
\begin{array}{c}
f_{*}:T_{q}(V')\to T_{q+k}(V),\text{~~ and}\\[4pt]
f_{*}:J_{q}(V')\to J_{q+k}(V).
\end{array}
\right\}\tag{2.12}\label{art08-sec2-eq2.12}
\end{equation*}

Suppose now that $f:V'\to V$ is an embedding so that $V'$ is an algebraic submanifold of $V$. Then $V'$ defines a class $[V']\in H_{2n-2k}(V,\bfZ)$\pageoriginale and $\mathscr{D}[V']=\Psi\in H^{2k}(V,\bfZ)\cap H^{k,k}(V)$. We assert that :

In \eqref{art08-sec2-eq2.11} and \eqref{art08-sec2-eq2.12}, the composite mapping
\begin{equation*}
\begin{array}{p{8cm}}
$f_{*}f^{*}:T_{q}(V)\to T_{q+k}(V)\text{~~ is just~~ }\Psi:T_{q}(V)\to T_{q+k}(V)\text{~~ as given by \eqref{art08-sec2-eq2.7} (and similarly for~~ $J_{q}(V)$)}$
\end{array}\tag{2.13}\label{art08-sec2-eq2.13}
\end{equation*}

\begin{proof}
We have to show that the composite
\begin{equation*}
H^{2q-1}(V,\bfC)\xrightarrow{f^{*}}H^{2q-1}(V',\bfC)\xrightarrow{f^{*}}H^{2q+2k-1}(V,\bfC)\tag{2.14}\label{art08-sec2-eq2.14}
\end{equation*}
is cup product with $\Psi$. In homology \eqref{art08-sec2-eq2.14} dualizes to 
\begin{equation*}
H_{2q-1}(V,\bfC)\xleftarrow{f_{*}}H_{2q-1}(V',\bfC)\xleftarrow{f^{*}}H_{2q+2k-1}(V,\bfC)\tag{2.15}\label{art08-sec2-eq2.15}
\end{equation*}
where $f^{*}$ is defined by
\begin{equation*}
\vcenter{\xymatrix@=1.2cm{
H_{2q+2k-1}(V,\bfC)\ar[d]^-{\mathscr{D}}\ar[r]^-{f^{*}} & H_{2q-1}(V',\bfC)\\
H^{2n-2q-2k+1}(V,\bfC)\ar[r]^-{f^{*}} & H^{2n-2k-2q+1}(V',\bfC)\ar[u]_{\mathscr{D}^{-1}}
}}\tag{2.16}\label{art08-sec2-eq2.16}
\end{equation*}

If we can show that $f_{*}f^{*}(\gamma)=[V']\cdot \gamma$ for $\gamma \in H_{2q+2k-1}(V,\bfC)$, then $\int\limits_{\gamma}f_{*}f^{*}\phi=\int\limits_{f_{*}f^{*}\gamma}\phi=\int\limits_{[V']\cdot \gamma}\phi=\int\limits_{\gamma}\Psi \wedge \phi(\phi\in H^{2q-1}(V,\bfC))$, and we are done. So we must show that, in \eqref{art08-sec2-eq2.15}, $f^{*}$ is intersection with $V'$, and this a standard result on the {\em Gysin homomorphism} \eqref{art08-sec2-eq2.16} (c.f. (4.11) and the accompanying Remark).
\end{proof}

\section{Algebraic Cycles and Complex Tori.}\label{art08-sec3}

Let $V=V_{n}$ be an algebraic manifold, $S\subset H^{2n-2q+1}(V,\bfC)$ a subspace satisfying \eqref{art08-sec1-eq1.2}-\eqref{art08-sec1-eq1.4}, and $T_{q}(S)$ the resulting complex torus. We choose a suitable basis $\omega^{1},\ldots,\omega^{m}$ for $S\cong H^{1,0}(T_{q}(S))$ and let $\Sigma_{q}$ = \{set of algebraic cycles $Z\subset V$ which are of codimension $q$ in $V$ and are homologous to zero\}. Following Weil \cite{art08-key22}, we define
\begin{equation*}
\phi_{q}:\Sigma_{q}\to T_{q}(S)\tag{3.1}\label{art08-sec3-eq3.1}
\end{equation*}
as follows: if $Z\in \Sigma_{q}$, then $Z=\partial C_{2n-2q+1}$ for some $2n-2q+1$ chain $C$, and we set 
\begin{equation*}
\phi_{q}(Z)=
\begin{bmatrix}
\vdots\\
\int\limits_{C}\omega^{\alpha}\\
\vdots
\end{bmatrix}.\tag{3.2}\label{art08-sec3-eq3.2}
\end{equation*}\pageoriginale
Since $C$ is determined up to cycles, $\phi_{q}(Z)$ is determined up to vectors $\left[\begin{smallmatrix}\vdots\\ \int\limits_{\gamma}\omega^{\alpha}\\ \vdots\end{smallmatrix}\right](\gamma\in H_{2n-2q+1}(V,\bfZ))$, and so $\phi_{q}$ {\em is defined and depends on the subspace of the closed $C^{\infty}$ forms spanned by} $\omega^{1},\ldots,\omega^{m}$; this restriction will be removed in the Appendix to \S\ref{art08-sec3}.

Now, while it should be the case that $\phi_{q}$ is holomorphic, we shall be content with recalling from \cite{art08-key9} a special result along these lines. Consider on $V$ an analytic family $\{Z_{\lambda}\}_{\lambda\in \Delta}$($\Delta=\text{disc in~}\lambda\text{-plane}$) of $q$-codimensional algebraic subvarieties $Z_{\lambda}\subset V$. Locally on $V$, $\{Z_{\lambda}\}_{\lambda\in \Delta}$ is given by the vanishing of analytic functions 
$$
f_{1}(z^{1},\ldots,z^{n};\lambda),\ldots,f_{l}(z^{1},\ldots,z^{n};\lambda).
$$ 
We define $\phi:\Delta\to T_{q}(S)$ by $\phi(\lambda)=\phi_{q}(Z_{\lambda}-Z_{0})$. Using  \eqref{art08-sec1-eq1.4}, we have proved in \cite{art08-key9} that
\begin{equation*}
\left.
\begin{array}{l}
\phi :\Delta\to T_{q}(S)\text{~~ is holomorphic and}\\[4pt]
\phi_{*}\{\bfT_{\lambda}(\Delta)\}\subset H^{q-1,q}(V).
\end{array}\right\}\tag{3.3}\label{art08-sec3-eq3.3}
\end{equation*}

We may rephrase \eqref{art08-sec3-eq3.3} by saying that $\phi^{*}:S_{q}\to \bfT_{\lambda}(\Delta)^{*}$ is determined by $\phi^{*}|H^{n-q+1,n-q}(V)$ (c.f. \eqref{art08-sec1-eq1.4}).

\medskip
\noindent
{\bf Continuous Systems and The Infinitesimal Calculation of {\boldmath$\phi_{q}$.}}~ Suppose that the $Z_{\lambda}\subset V$ are all nonsingular and $Z=Z_{0}$. We let $\bfN\to Z$ be the {\em normal bundle} of $Z\subset V$, so that we have the exact sheaf sequence
\begin{equation*}
0\to \mathscr{O}_{Z}(\bfN^{*})\to \Omega^{1}_{V|Z}\to \Omega^{1}_{Z}\to 0.\tag{3.4}\label{art08-sec3-eq3.4}
\end{equation*}
Since $\dim Z=n-q$, from \eqref{art08-sec3-eq3.4} we have induced the {\em Poincar\'e residue operator}
\begin{equation*}
\Omega^{n-q+1}_{V|Z}\to \Omega^{n-q}_{Z}(\bfN^{*})\to 0\tag{3.5}\label{art08-sec3-eq3.5}
\end{equation*}
as follows : Let $\phi\in \Omega^{n-q+1}_{V|Z}$; $\tau_{1},\ldots,\tau_{n-q}$ be tangent vectors to $Z$; $\eta$ a normal vector to $Z$. Lift $\eta$ to a tangent vector $\widehat{\eta}$ on $V$ along $Z$.\pageoriginale Then $\langle \phi, \tau_{1}\wedge\ldots\wedge \tau_{n-q}\otimes \eta\rangle=\langle \phi,\tau_{1}\wedge\ldots\wedge \tau_{n-q}\wedge\widehat{\eta}\rangle$, where $\phi\in \Omega^{n-q+1}_{V|Z}$.

From \eqref{art08-sec3-eq3.5} and $\Omega^{n-q+1}_{V}\to \Omega^{n-q+1}_{V|Z}$, we have
\begin{equation*}
H^{n-q}(V,\Omega^{n-q+1}_{V})\xrightarrow{\xi^{*}}H^{n-q}(Z,\Omega^{n-q}_{Z}(\bfN^{*})).\tag{3.6}\label{art08-sec3-eq3.6}
\end{equation*}

On the other hand, in \cite{art08-key16} Kodaira has defined the {\em infinitesimal displacement mapping}
\begin{equation*}
\rho : \bfT_{0}(\Delta)\to H^{0}(Z,\mathscr{O}_{Z}(\bfN)).\tag{3.7}\label{art08-sec3-eq3.7}
\end{equation*}
To calculate $\phi^{*}$, we have shown in \cite{art08-key9} that the following diagram commutes:
\begin{equation*}
\vcenter{\xymatrix@R=1.2cm{
 & H^{n-q+1,n-q}(V)=H^{n-q}(V,\Omega^{n-q+1}_{V})\ar[dd]^-{\xi^{*}}\ar[dl]_-{\phi^{*}}\\
\bfT_{0}(\Delta)^{*} & \\
& H^{n-q}(Z,\Omega^{n-q}_{Z}(\bfN^{*}))=H^{0}(Z,\mathscr{O}_{Z}(\bfN))^{*}.\ar[ul]_-{\rho^{*}}
}}\tag{3.8}\label{art08-sec3-eq3.8}
\end{equation*}
In other words, infinitesimally $\phi$ is eseentially given by by $\xi^{*}$ in \eqref{art08-sec3-eq3.6}.

\medskip
\noindent
{\bf Some Special Cases.}
\begin{itemize}
\item[(i)] In case $q=n$, $Z$ is a finite set of points $z_{1},\ldots,z_{r}$ ($Z$ is a zero-cycle) and \eqref{art08-sec3-eq3.6} becomes:
\begin{equation*}
H^{1,0}(V)\xrightarrow{\xi^{*}}\sum\limits^{r}_{j=1}\bfT_{z_{j}}(V)^{*}\tag{3.9}\label{art08-sec3-eq3.9}
\end{equation*}
where $\xi^{*}(\omega)=\sum\limits^{r}_{j=1}\omega(z_{j})$, $\omega\in H^{1,0}(V)$ being a holomorphic 1-form on $V$. In particular, $\phi^{*}$ is onto if $\xi^{*}$ is injective.

\item[(ii)] In case $q=1$, $Z\subset V$ is a nonsingular hypersurface. Then there is a holomorphic line bundle $\bfE\to V$ and a section $\sigma\in H^{0}(V,\mathscr{O}_{V}(\bfE))$ such that $Z=\{z\in V:\sigma(z)=0\}$. From the exact sheaf sequence $0\to \mathscr{O}_{V}\xrightarrow{\sigma}\mathscr{O}_{V}(\bfE)\to \mathscr{O}_{Z}(\bfN)\to 0$, we find
\begin{equation*}
H^{0}(Z,\mathscr{O}_{Z}(\bfN))\xrightarrow{\xi}H^{1}(V,\mathscr{O}_{V}),\tag{3.10}\label{art08-sec3-eq3.10}
\end{equation*}
where we claim that $\xi$ in \eqref{art08-sec3-eq3.10} is (up to a constant) the dual of $\xi^{*}$ in \eqref{art08-sec3-eq3.6} (using $H^{0,1}(V)=H^{n,n-1}(V)^{*}$).
\end{itemize}

\begin{proof}
We\pageoriginale may choose a covering $\{U_{\alpha}\}$ of $V$ by polycylinders such that $Z\cap U_{\alpha}$ is given by $\sigma_{\alpha}=0$ where $\sigma_{\alpha}$ is a coordinate function if $U_{\alpha}\cap Z\neq \emptyset$ and $\sigma_{\alpha}\equiv 1$ if $U_{\alpha}\cap Z=\emptyset$. Then $\sigma_{\alpha}/\sigma_{\beta}=f_{\alpha\beta}$ where $\{f_{\alpha\beta}\}\in H^{1}(V,\mathscr{O}^{*}_{V})$ and gives the transition functions of $\bfE\to V$. Let $\theta=\{\theta_{\alpha}\}\in H^{0}(Z,\mathscr{O}_{Z}(\bfN))$ and $\omega\in H^{n,n-1}(V)$. We want to show that, for a suitable constant $c$, we have
\begin{equation*}
\int\limits_{V}\xi(\theta)\wedge \omega=c\int\limits_{Z}\langle \theta,\xi^{*}\omega\rangle.\tag{3.11}\label{art08-sec3-eq3.11}
\end{equation*}
If $Z\cap U_{\alpha}\neq \emptyset$, we may write $\omega=\omega_{\alpha}\wedge d\sigma_{\alpha}$ where $\omega_{\alpha}$ is a $C^{\infty}(n-1,n-1)$ form in $U_{\alpha}$ such that $\omega_{\alpha}|Z\cap U_{\alpha}$ is well-defined. In $U_{\alpha}\cap U_{\beta}$, $\omega=\omega_{\alpha}\wedge d\sigma_{\alpha}=\omega_{\alpha}\wedge d(f_{\alpha\beta}\sigma_{\beta})=\omega_{\alpha}\sigma_{\beta}\wedge df_{\alpha\beta}+f_{\alpha\beta}\omega_{\alpha}\wedge d\sigma_{\beta}$ so that $\omega_{\alpha}|Z\cap U_{\alpha}\cap U_{\beta}=f^{-1}_{\alpha\beta}\omega_{\beta}|Z\cap U_{\alpha}\cap U_{\beta}$. This means that $\{\omega_{\alpha}|Z\cap U_{\alpha}\}$ gives an $(n-1,n-1)$ form on $Z$ with values in $\bfN^{*}$, and so $\{\theta_{\alpha}\omega_{\alpha}|Z\cap U_{\alpha}\}$ gives a global $C^{\infty}(n-1,n-1)$ form on $Z$ (since $\theta_{\alpha}=f_{\alpha\beta}\theta_{\beta}$ on $Z\cap U_{\alpha}\cap U_{\beta}$). It is clear that $\langle \theta,\xi^{*}\omega\rangle |Z=\{\theta_{\alpha}\omega_{\alpha}\}$ so that the right hand side of \eqref{art08-sec3-eq3.11} is
\begin{equation*}
\int\limits_{Z}\{\theta_{\alpha}\omega_{\alpha}\}.\tag{3.12}\label{art08-sec3-eq3.12}
\end{equation*}

On the other hand, choose a $C^{\infty}$ section $\Theta=\{\Theta_{\alpha}\}$ of $\bfE\to V$ with $\Theta|Z=\theta$. Then $\overline{\partial \Theta}=\sigma\xi(\theta)$ where $\xi(\theta)$ is a $C^{\infty}(0,1)$ form giving a Dolbeault representative of $\xi(\theta)\in H^{1}(V,\mathscr{O}_{V})$ in \eqref{art08-sec3-eq3.10}. Let $T_{\epsilon}$ be an $\epsilon$-tube aroung $Z$ and $\psi=\dfrac{\Theta}{\sigma}$. Then the left hand side of \eqref{art08-sec3-eq3.11} is $\int\limits_{V}\xi(\theta)\wedge \omega=\lim\limits_{\epsilon\to 0}\int\limits_{V-T_{\epsilon}}\xi(\theta)\wedge \omega=-\lim\limits_{\epsilon\to 0}\int\limits_{\partial T_{\epsilon}}\psi \wedge \omega$ (since $d(\psi\wedge \omega)= \overline{\partial}(\psi\wedge \omega)=\xi(\theta)\wedge \omega$). Locally $\psi\wedge \omega=\Theta_{\alpha}\omega_{\alpha}\wedge \dfrac{d\sigma_{\alpha}}{\sigma_{\alpha}}$ so that 
$\lim\limits_{\epsilon \to 0}\int\limits_{\partial T_{\epsilon}}\psi\wedge \omega=\lim\limits_{\epsilon\to 0}\int\limits_{\partial T_{\epsilon}}\{\Theta_{\alpha}\omega_{\alpha}\}\wedge \dfrac{d\sigma_{\alpha}}{\sigma_{\alpha}}=2\pi i \int\limits_{Z}\{\Theta_{\alpha}\omega_{\alpha}|Z\cap U_{\alpha}\}=2\pi i\int\limits_{Z}\{\theta_{\alpha}\omega_{\alpha}\}$, which, by \eqref{art08-sec3-eq3.12}, proves \eqref{art08-sec3-eq3.11}.
\end{proof}

\medskip
\noindent
{\bf Appendix to \S\ref{art08-sec3}~: Some Remarks on the Definition of {\boldmath$\phi_{q}$}.}~ At the beginning of Paragraph 3 where $\phi_{q}:\Sigma_{q}\to T_{q}(S)$ was defined, it was stated that $\phi_{q}$ depended on the vector space spanned by $\omega^{1},\ldots,\omega^{m}$ and not just on $S$. This is because, if we replace $\omega^{\alpha}$ by $\omega^{\alpha}+d\eta^{\alpha}$, then $\int\limits_{C_{2n-2q+1}}\omega^{\alpha}+d\eta^{\alpha}=\int\limits_{C}\omega^{\alpha}+\int\limits_{Z}\eta^{\alpha}$ (Stokes' Theorem).

One\pageoriginale way around this is to use the K\"ahler metric on $V$ and choose $\omega^{1},\ldots,\omega^{m}$ to be {\em harmonic}. This has the disadvantage that harmonic forms are {\em not} generally preserved under holomorphic mappings. However, if we agree to use the torus $T_{q}(V)(S=\sum\limits_{r}H^{n-q+1+r,n-q-r}(V))$ constructed in Example \ref{art08-sec2-exam1} of \S\ref{art08-sec2}, it is possible to given $\phi_{q}:\Sigma_{q}\to T_{q}(V)$ {\em purely in terms of cohomology}, and so remove this problem in defining $\phi_{q}$.

To do this, we shall use a theorem on the cohomology of algebraic manifolds which is given in the Appendix below. Let then $\Omega^{q}$ be the sheaf of holomorphic $q$-forms on $V$ and $\Omega^{q}_{c}\subset \Omega^{q}$ the subsheaf of closed forms. There is an exact sequence:
\begin{equation*}
0\to \Omega^{q}_{c}\to \Omega^{q}\xrightarrow{d}\Omega^{q+1}_{c}\to 0\tag{A3.1}\label{art08-sec3-eqA3.1}
\end{equation*}
({\em Poincar\'e lemma}), which gives in cohomology (c.f. (A.7)):
\begin{equation*}
0\to H^{p-1}(V,\Omega^{q+1}_{c})\xrightarrow{\delta}H^{p}(V,\Omega^{q}_{c})\to H^{p}(V,\Omega^{q})\to 0.\tag{A3.2}\label{art08-sec3-eqA3.2}
\end{equation*}

From \eqref{art08-sec3-eqA3.2}, we see that there is a diagram (c.f.(A.16) in the Appendix):
\begin{equation*}
\vcenter{
\xymatrix@C=-.02cm{
H^{r}(V,\bfC) & = & H^{r}(V,\bfC)\\
H^{r-1}(V,\Omega^{1}_{c})\ar[u]_-{\delta} & \subset & H^{r}(V,\bfC)\ar@{=}[u]\\
\vdots\ar[u] & & \vdots\ar@{=}[u]\\
H^{r-q}(V,\Omega^{q}_{c})\ar[u] & \subset & H^{r}(V,\bfC)\ar@{=}[u]\\
\vdots\ar[u]_-{\delta} & & \vdots\ar@{=}[u]\\
H^{0}(V,\Omega^{r}_{c})\ar[u] & \subset & H^{r}(V,\bfC)\ar@{=}[u]\\
0\ar[u] & &
}}\tag{A3.3}\label{art08-sec3-eqA3.3}
\end{equation*}

Thus\pageoriginale $\{H^{r-q}(V,\Omega^{q}_{c})\}$ gives a {\em filtration} $\{F^{r}_{q}(V)\}$ of $H^{r}(V,\bfC)$; and\break $F^{r}_{q}(V)/F^{r}_{q+1}(V)\cong H^{r-q}(V,\Omega^{q})$. It is also true that $F^{r}_{q}(V)$ depends holomorphically on $V$ \cite{art08-key9}.

To calculate $F^{r}_{q}(V)$ using differential forms, we let $A^{s,r-s}$ be the $C^{\infty}$ forms of type $(s,r-s)$ on $V$, $B^{r,q}=\sum\limits_{s\geq q}A^{s,r-s}$, and $B^{r,q}_{c}$ the $d$-closed forms in $B^{r,q}$. Then $dB^{r,q}\subset B^{r+1,q}_{c}$, and it is shown in the Appendix (c.f. (A.18)) that
\begin{equation*}
F^{r}_{q}(V)\cong B^{r,q}_{c}/dB^{r-1,q}\subset H^{r}(V,\bfC).\tag{A3.4}\label{art08-sec3-eqA3.4}
\end{equation*}

We conclude then from \eqref{art08-sec3-eqA3.4} that:
\begin{equation*}
\left.
\begin{array}{@{}l}
\text{A class~ } \phi\in F^{r}_{q}(V)\subset H^{r}(V,\bfC)\text{~ is represented by a closed } C^{\infty}\\[4pt]
\text{form~ } \phi=\sum\limits_{s\geq q}\phi_{s,r-s}(\phi_{s,r-s}\in A^{s,r-s},\text{~ defined up to forms}\\[4pt]
d\eta=\sum\limits_{s\geq q}d\eta_{s,r-1-s}. 
\end{array}
\right\}\tag{A3.5}\label{art08-sec3-eqA3.5}
\end{equation*}

In particular, look at 
$$
F^{2n-2q+1}_{n-q+1}(V)\cong \sum\limits_{r\geq 0}H^{n-q+1+r,n-q-r}(V).
$$ 
$A\phi \in B^{2n-2q+1,n-q+1}_{c}$ is defined up to 
$$
\sum\limits_{s\geq 0}d\eta_{n-q+1+s,n-q-1-s}
$$ 
and 
$$
\int\limits_{Z}\eta_{n-q+1+s,n-q-1-s}=0
$$ 
for an algebraic cycle $Z$ of codimension $q$\,($Z$ is of type $(n-q,n-q)$). Thus $\int\limits_{C_{2n-2q+1}}\phi(\partial C=Z)$ depends {\em only} on the class of 
$$
\phi\in F^{2n-2q+1}_{n-q+1}(V)\subset H^{2n-2q+1}(V,\bfC).
$$ 
This proves that:
\begin{equation*}
\left.
\begin{array}{l}
\text{For the torus~ } T_{q}(V)\text{~ constructed in \S\ref{art08-sec3}, the mapping}\\[3pt]
\phi_{q}:\Sigma_{q}\to T_{q}(V)\text{~ depends only on the complex structure of } V.
\end{array}\right\}\tag{A3.6}\label{art08-sec3-eqA3.6}
\end{equation*}

For the general tori $T_{q}(S)$ we may prove the analogue of \eqref{art08-sec3-eqA3.6} as follows. First, we may make the forms $\omega^{1},\ldots,\omega^{m}$ subject to $\partial \omega^{\alpha}=0$, $\overline{\partial}\omega^{\alpha}=0$, because $S=\sum\limits_{r}S\cap H^{2n-2q+1-r,r}(V)$ and so $\omega^{\alpha}=\mathscr{H}(\omega^{\alpha})+d\xi^{\alpha}$ ($\mathscr{H}$ = harmonic part of $\omega^{\alpha}$) and $\mathscr{H}(\omega^{\alpha})=\sum\limits_{r}\mathscr{H}(\omega^{\alpha}_{n-q+1+r,n-q-r})$ with $\partial \mathscr{H}(\omega^{\alpha}_{n-q+1+r,n-q-r})=0=\overline{\partial}\mathscr{H}(\omega^{\alpha}_{n-q+1+r,n-q-r})$. Thus we may choose a basis $\omega^{1},\ldots,\omega^{m}$ for $S$ with $\partial \omega^{\alpha}=0=\overline{\partial}\omega^{\alpha}$.

Second,\pageoriginale let $\eta$ be a $C^{\infty}$ form on $V$ with $\partial d\eta=0=\overline{\partial}d\eta$. We claim that $d\eta=\partial\overline{\partial}\xi$ for some $\xi$. Since $d\eta=\partial\eta+\overline{\partial}\eta$, it will suffice to do this for $\partial\eta$. Now write $\eta=\mathscr{H}_{\partial}\eta+\partial\partial^{*}G_{\partial}\eta+\partial^{*}\partial G_{\partial}\eta$, where $\mathscr{H}_{\partial}$ is the {\em harmonic projector} for $\boxvoid_{\partial}=\partial\partial^{*}+\partial^{*}\partial$ and $G_{\partial}$ is the corresponding {\em Green's operator}. Then $\partial_{\eta}=\partial\partial^{*}\partial G_{\partial}\eta$. On the other hand, since $\overline{\partial}\partial=0$, $\partial\eta=\mathscr{H}_{\partial}(\partial\eta)+\overline{\partial}\partial^{*}G_{\partial}\partial\eta$. But $\mathscr{H}_{\partial}=\mathscr{H}_{\partial}$ and $G_{\partial}=G_{\partial}$ so that $\partial\eta=\partial\overline{\partial}(\overline{\partial}^{*}G_{\partial}\eta)$ as desired.

Finally, let $\omega\in S$ satisfy $\partial \omega=0=\overline{\partial}\omega$ and change $\omega$ to $\omega+d\eta$ with $\partial(\omega+d\eta)=0=\overline{\partial}(\omega+d\eta)$. Then $\omega+d\eta=\omega+\partial\overline{\partial}\xi$ for some $\xi$. We claim that $\int\limits_{C}\omega=\int\limits_{C}\omega+\partial\overline{\partial}\xi$, where $C$ is a $2n-2q+1$ chain with $\partial C=Z$. If $\xi=\sum\limits_{r}\xi_{n-q+r,n-q-1-r}$, then
$$
\int\limits_{C}\partial\overline{\partial}\xi=\int\limits_{C}d(\overline{\partial}\xi)=\int\limits_{Z}(\overline{\partial}\xi)_{n-q,n-q}=\int\limits_{Z}\overline{\partial}\xi_{n-q,n-q-1}=0
$$
since $Z\subset V$ is a complex submanifold. This proves that:
\begin{equation*}
\left.
\begin{array}{@{}l@{}}
\text{If, in the definition of  $\phi_{q}:\Sigma_{q}\to T_{q}(V)$  in \eqref{art08-sec3-eq3.1}, we make}\\
\text{the $\omega^{\alpha}$ subject to $\partial\omega^{\alpha}=0=\overline{\partial}\omega^{\alpha}$, then $\phi_{q}$ is well-defined}\\
\text{and depends only on the complex structure of $V$.}
\end{array}
\right\}\tag{A3.7}\label{art08-sec3-eqA3.7}
\end{equation*}

This is the procedure followed by Weil \cite{art08-key22}.

\medskip
\noindent
{\bf Remark. \thnum{A(3.8)}.\label{art08-A3.8}}~ Let $D=\partial\overline{\partial}$; then $D:A^{p,q}_{c}\to A^{p+1,q+1}_{c}$ and $D^{2}=0$. If $H^{r}_{D}(V)$ are the cohomology groups constructed from $D=\partial\overline{\partial}$ and $H^{r}_{d}(V)$ the deRham groups, there is a natural mapping:
\begin{equation*}
H^{r}_{D}(V)\xrightarrow{\alpha}H^{r}_{d}(V).\tag{A3.9}\label{art08-sec3-eqA3.9}
\end{equation*}

\section{Some Functorial Properties.}\label{art08-sec4}
(a)~ Let $W_{n-k}\subset V_{n}$ be an algebraic submanifold of codimension $k$. We shall assume for the moment that there is a holomorphic vector bundle $\bfE\to V$, with fibre $\bfC^{s}$, and holomorphic sections $\sigma_{1},\ldots,\sigma_{s-k+1}$ of $\bfE$ such that $W$ is given by $\sigma_{1}\wedge\ldots\wedge \sigma_{s-k+1}=0$. Thus, the homology class carried by $W$ is the $k$-{\em th Chern class} of $\bfE\to V$ (c.f. \cite{art08-key5}). Following \eqref{art08-sec2-eq2.11} there is a mapping
\begin{equation*}
T_{q}(V)\xrightarrow{i^{*}}T_{q}(W)\tag{4.1}\label{art08-sec4-eq4.1}
\end{equation*}
induced\pageoriginale from $H^{2q-1}(V,\bfC)\to H^{2q-1}(W,\bfC)$. We want to interpret this mapping geometrically.

For this, let $\{Z_{\lambda}\}_{\lambda\in \Delta}$ be a {\em continuous system} as in paragraph 3. Assume that each intersection $Y_{\lambda}=Z_{\lambda}\cdot W$ is transverse so that $\{Y_{\lambda}\}_{\lambda\in D}$ gives a continuous system of $W$. Letting $\phi_{q}(V)(\lambda)=\phi_{q}(Z_{\lambda}-Z_{0})\in T_{q}(V)$ and $\phi_{q}(W)(\lambda)=\phi_{q}(Y_{\lambda}-Y_{0})\in T_{q}(W)$, we would like to show that the following diagram commutes:
\begin{equation*}
\vcenter{\xymatrix@C=1.5cm{
 & T_{q}(V)\ar[dd]^{i^{*}}\\
\Delta\ar[ur]^{\phi_{q}(V)}\ar[dr]^{\phi_{q}(W)} & \\
 & T_{q}(W)
}}\tag{4.2}\label{art08-sec4-eq4.2}
\end{equation*}
This would interpret \eqref{art08-sec4-eq4.1}geometrically as ``{\em intersection with $W$}''.

\begin{proof}
Let $S_{V}=\sum\limits_{r\geq 0}H^{n-q+1+r,n-q-r}(V)\subset H^{2n-2q+1}(V,\bfC)$ be the space of holomorphic 1-forms on $T_{q}(V)$ and $\omega^{1},\ldots,\omega^{m}$ a basis for $S_{V}$. If $C_{\lambda}$ is a $2n-2q+1$ chain on $V$ with $\partial C_{\lambda}=Z_{\lambda}-Z_{0}$ then $\phi_{q}(V)(\lambda)=\left[\begin{smallmatrix} :\\ \int\limits_{C_{\lambda}}\omega^{\alpha}\\ :\end{smallmatrix}\right]$. Similarly, let $S_{W}\subset H^{2n-2q-2k+1}(W,\bfC)$ be the holomorphic 1-forms on $T_{q}(W)$ and $\phi^{1},\ldots,\phi^{r}$ a basis for $S_{W}$. Letting $D_{\lambda}=C_{\lambda}\cdot W$, $\partial D_{\lambda}=Y_{\lambda}-Y_{0}$ and $\phi_{q}(W)(\lambda)=\left[\begin{smallmatrix} :\\ \int\limits_{D_{\lambda}}\phi^{\rho}\\ :\end{smallmatrix}\right]$.

Actually, in line with the Appendix to \S\ref{art08-sec3}, we should use the isomorphisms $F^{2n-2q+1}_{n-q+1}(V)\cong S_{V}$, $F^{2n-2k-2q+1}_{n-k-q+1}(W)\cong S_{W}$ (c.f. \eqref{art08-sec3-eqA3.5}), and choose $\omega^{1},\ldots,\omega^{m}$ and $\phi^{1},\ldots,\phi^{r}$ as bases of 
$$
F^{2n-2q+1}_{n-q+1}(V)\text{~ and~ } F^{2n-2k-2q+1}_{n-k-q+1}(W)
$$ 
respectively. We assume this is done.

We now need to give $i^{*}:\bfC^{m}\to \bfC^{r}$ explicitly using the above bases. Let $e_{1},\ldots,e_{m}\in S^{*}_{V}\subset H^{2q-1}(V,\bfC)$ be dual to $\omega^{1},\ldots,\omega^{m}$ and $f_{1},\ldots,f_{r}\in S^{*}_{W}\subset H^{2q-1}(W,\bfC)$ dual to $\phi^{1},\ldots,\phi^{r}$. Then $\int\limits_{V}\omega^{\alpha}\wedge e_{\beta}=\delta^{\alpha}_{\beta}$, $\int\limits_{W}\phi^{\rho}\wedge f_{\sigma}=\delta^{\rho}_{0}$.\pageoriginale Now $i^{*}(e_{\alpha})=\sum\limits^{r}_{\rho=1}m_{\rho\alpha}f_{\rho}$ for some $r\times m$ matrix $M$, and $i^{*}:T_{q}(V)\to T_{q}(W)$ is given by $M:\bfC^{m}\to \bfC^{r}$ (c.f. just below \eqref{art08-sec2-eq2.7}).

To calculate $M\left[\begin{smallmatrix} : \\ \int\limits_{\gamma}\omega^{\alpha}\\ :\end{smallmatrix}\right]$, we let 
$$
i_{*}:H^{2n-2k-2q+1}(W,\bfC)\to H^{2n-2q+1}(V,\bfC)
$$ 
be the {\em Gysin homomorphism} defined by:
\begin{equation*}
\vcenter{\xymatrix{
H^{2n-2k-2q+1}(W,\bfC)\ar[r]^-{i_{*}} & H^{2n-2q+1}(V,\bfC)\\
H_{2q-1}(W,\bfC)\ar[u]_{\mathscr{D}_{W}}\ar[r]^-{i_{*}} & H_{2q-1}(V,\bfC).\ar[u]_{\mathscr{D}_{V}}
}}\tag{4.3}\label{art08-sec4-eq4.3}
\end{equation*}

Then, $i_{*}(\phi^{p})=\sum\limits^{m}_{\alpha=1}m_{p\alpha}\omega^{\alpha}$, and $M\left[\begin{smallmatrix} :\\ \int\limits_{\gamma}\omega^{\alpha}\\ :\end{smallmatrix}\right]=\left[\sum\limits^{m}_{\alpha=1}m_{\rho\alpha}\int\limits_{\gamma}\omega^{\alpha}\right]=\left[\begin{smallmatrix}:\\ \int\limits_{\gamma}i_{*}(\phi^{\rho})\\ :\end{smallmatrix}\right]=\left[\begin{smallmatrix} :\\ \int\limits_{W\cdot \gamma}\phi^{\rho}\\ :\end{smallmatrix}\right]$ (c.f. \eqref{art08-sec2-eq2.16}) where $\gamma\in H_{2n-2q+1}(V,\bfZ)$ is a cycle on $V$. This gives the equation
\begin{equation*}
M\begin{bmatrix}
:\\
\int\limits_{\gamma}\omega^{\alpha}\\
:
\end{bmatrix}
=
\begin{bmatrix}
:\\
\int\limits_{\gamma}i_{*}\phi^{\rho}\\
:
\end{bmatrix}
=
\begin{bmatrix}
:\\
\int\limits_{W\cdot \gamma} \phi^{\rho}\\
:
\end{bmatrix},\tag{4.4}\label{art08-sec4-eq4.4}
\end{equation*}
for $\gamma\in H_{2n-2q+1}(V,\bfZ)$. To prove \eqref{art08-sec4-eq4.2}, we must prove \eqref{art08-sec4-eq4.4} for the chain $C_{\lambda}$ with $\partial C_{\lambda}=Z_{\lambda}-Z_{0}$; this is because, in \eqref{art08-sec4-eq4.2},
$$
i^{*}\phi_{q}(V)(\lambda)=M
\begin{bmatrix}
:\\
\int\limits_{C_{\lambda}}\omega^{\alpha}\\
:
\end{bmatrix}
\text{~~ and~~} \phi_{q}(W)(\lambda)=
\begin{bmatrix}
:\\
\int\limits_{W\cdot C_{\lambda}}\phi^{\rho}\\
:
\end{bmatrix}.
$$
Thus, to prove the formula \eqref{art08-sec4-eq4.2}, we must show :

The {\em Gysin homomorphism}
\begin{equation*}
i_{*}:H^{2n-2k-2q+1}(W,\bfC)\to H^{2n-2q+1}(V,\bfC)\tag{4.5}\label{art08-sec4-eq4.5}
\end{equation*}
(given\pageoriginale in \eqref{art08-sec4-eq4.3}) has the properties :
\begin{equation*}
i_{*}:F^{2n-2q-2k+1}_{n-q-k+1}(W)\to F^{2n-2q+1}_{n-q+1}(V);\tag{4.6}\label{art08-sec4-eq4.6}
\end{equation*}
and
\begin{equation*}
\int\limits_{C}i_{*}\phi=\int\limits_{W\cdot C}\phi,\tag{4.7}\label{art08-sec4-eq4.7}
\end{equation*}
where $C$ is a $2n-2q+1$ chain on $V$ with $\partial C=Z$, $Z$ being an algebraic cycle on $V$ meeting $W$ transversely.

This is where we use the bundle $\bfE\to V$. Namely, it will be proved in the Appendix to \S\ref{art08-sec4} below that there is a $C^{\infty}(k,k-1)$ form $\psi$ defined on $V-W$ having the properties :

$\partial \psi=0$ and $\overline{\partial}\psi=\Psi$ is a $C^{\infty}$ form on $V$ which represents the
\begin{equation*}
\text{Poincar\'e dual~ } \mathscr{D}(W)\in H^{k,k}(V,\bfC)\cap H^{2k}(V,\bfZ);\tag{4.8}\label{art08-sec4-eq4.8}
\end{equation*}
to give $i_{*}$ in \eqref{art08-sec4-eq4.6}, we let $\phi\in B^{2n-2k-2q+1,n-k-q+1}_{c}(W)$ represent a class in $F^{2n-2k-2q+1}_{n-k-q+1}(W)$ and choose $\widehat{\phi}\in B^{2n-2k-2q+1,n-k-q+1}(V)$ with $\widehat{\phi}|W=\phi$. Then $d(\psi\wedge \phi)$ is a {\em current} on $V$ and
\begin{align*}
& i_{*}(\phi)=d(\psi\wedge\widehat{\phi});\quad\text{and}\tag{4.9}\label{art08-sec4-eq4.9}\\
& \lim\limits_{\epsilon \to 0}\int\limits_{c\cdot \partial T_{\epsilon}}\psi\wedge\eta =\int\limits_{C\cdot W}\eta,\tag{4.10}\label{art08-sec4-eq4.10}
\end{align*}
where $T_{\epsilon}$ is the $\epsilon$-tube around $W$ and $\eta\in B^{2n-2k-2q+1,n-k-q+1}(V)$.
\end{proof}

\begin{remark*}
The composite
\begin{equation*}
F^{2n-2k-2q+1}_{n-k-q+1}(V)\xrightarrow{i^{*}}F^{2n-2k-2q+1}_{n-k-q+1}(W)\xrightarrow{i_{*}}F^{2n-2q+1}_{n-q+1}(V)\tag{4.11}\label{art08-sec4-eq4.11}
\end{equation*}
is given by $i_{*}i^{*}\eta=d(\psi\wedge \eta)=\Psi\wedge \eta(\eta\in B^{2n-2k-2q+1,n-k-q+1}_{c}(V))$; this should be compared with \eqref{art08-sec2-eq2.16} above.
\end{remark*}

\medskip
\noindent
{\bf Proof of \eqref{art08-sec4-eq4.6} and \eqref{art08-sec4-eq4.7} from \eqref{art08-sec4-eq4.8}--\eqref{art08-sec4-eq4.10}.}~ Since $i_{*}(\phi)=d(\psi\wedge \widehat{\phi})$ and $\psi\wedge\widehat{\phi}\in B^{2n-2q,n-q+1}$, $i_{*}(\phi)\in B^{2n-2q+1,n-q+1}_{c}(V)$ which proves \eqref{art08-sec4-eq4.6} (c.f. (e) in the Appendix to Paragraph 4).

To prove \eqref{art08-sec4-eq4.7}, we will have
$$
\int\limits_{C}i_{*}\phi=\lim\limits_{e\to 0}\int\limits_{C-C\cdot T_{\epsilon}}i_{*}(\phi)=(\text{by Stokes' theorem})\int\limits_{\partial(C-C\cdot T_{\epsilon})}\psi\wedge \widehat{\phi}.
$$
But $\partial(C-C\cdot T_{\epsilon})=(Z-Z\cdot T_{\epsilon})-C\cdot \partial T_{\epsilon}$\pageoriginale and so $\int\limits_{C}i_{*}\phi=-\lim\limits_{\epsilon\to 0}\int\limits_{C\cdot \partial T_{\epsilon}}\psi \wedge \widehat{\phi}$ (since $\int\limits_{Z-Z\cdot T_{\epsilon}}\psi \wedge \widehat{\phi}=0$) $=\int\limits_{C\cdot W}\phi$ by \eqref{art08-sec4-eq4.10}.

\medskip
\noindent
{\bf Remark \thnum{4.12}.\label{art08-sec4-rem4.12}}
Actually \eqref{art08-sec4-eq4.2} will hold in the following generality. Let $V$, $V'$ be algebraic manifolds and $f:V'\to V$ a holomorphic mapping. Let $\Sigma_{q}(V)$ be the algebraic cycles $Z\subset V$ of codimension $q$ which are homologous to zero and similarly for $\Sigma_{q}(V')$. Then there is a commutative diagram :
\[
\xymatrix@=1.2cm{
\Sigma_{q}(V)/\text{S.E.R.}\ar[d]^-{f^{*}}\ar[r]^-{\phi_{q}(V)} & T_{q}(V)\ar[d]^-{F^{*}}\\
\Sigma_{q}(V')/\text{S.E.R.}\ar[r]^-{\phi_{q}(V')} & T_{q}(V')
}
\]
where S.E.R. = suitable equivalence relation (including rational equivalence), and where $f^{*}(Z)=f^{-1}(Z)=\{z'\in V':f(z)\in V\}$ in case $Z$ is transverse to $f(V')$.

(b) Keeping the notation and assumptions of (4a) above, following \eqref{art08-sec2-eq2.12} we have:
\begin{equation*}
i_{*}:T_{q}(W)\to T_{q+k}(V),\tag{4.13}\label{art08-sec4-eq4.13}
\end{equation*}
and we want also these maps geometrically. For this, let $\{Y_{\lambda}\}_{\lambda\in \Delta}$ be a continuous system of subvarieties $Y_{\lambda}\subset W$ of codimension $q$. Then $Y_{\lambda}\subset V$ has codimension $k+q$ and so we may set
$$
\phi_{q+k}(V)(\lambda)=\phi_{q}(Y_{\lambda}-Y_{0})\in T_{q+k}(V),\phi_{q}(W)(\lambda)=\phi_{q}(Y_{\lambda}-Y_{0})\in T_{q}(W).
$$
We assert that the following diagram commutes:
\begin{equation*}
\vcenter{\xymatrix@C=1.5cm{
 & T_{q}(W)\ar[dd]^-{i_{*}}\\
\Delta\ar[ur]^-{\phi_{q}(W)}\ar[dr]_-{\phi_{q+k}(V)} &\\
 & T_{q+k}(V)
}}\tag{4.14}\label{art08-sec4-eq4.14}
\end{equation*}
This\pageoriginale interprets $i_{*}$ in \eqref{art08-sec4-eq4.13} as ``{\em inclusion of cycles lying on $W$ into $V$}''.

\begin{proof}
As in the proof of \eqref{art08-sec4-eq4.2}, we choose bases $\omega^{1},\ldots,\omega^{m}$ for $S_{V}\subset H^{2n-2k-2q+1}(V,\bfC)$ and $\phi^{1},\ldots,\phi^{r}$ for $S_{W}\subset H^{2n-2k-2q+1}(W,\bfC)$. Then
$$
\phi_{q}(W)(\lambda)=
\begin{bmatrix}
:\\
\int\limits_{C_{\lambda}}\phi^{\rho}\\
:
\end{bmatrix}
\text{~~ and~~ } \phi_{q+k}(V)(\lambda)=
\begin{bmatrix}
:\\
\int\limits_{C_{\lambda}}\omega^{\alpha}\\
:
\end{bmatrix}
$$
where $\partial C_{\lambda}=Y_{\lambda}-Y_{0}$.

We now need $i_{*}$ explicitly. Let $e_{1},\ldots,e_{m}$ be a dual basis in $S^{*}_{V}\subset H^{2q+2k-1}(V,\bfC)$ to $\omega^{1},\ldots,\omega^{m}$ and $f_{1},\ldots,f_{r}$ in $S^{*}_{W}\subset H^{2q-1}(W,\bfC)$ be a dual basis to $\phi^{1},\ldots,\phi^{r}$. Then $i_{*}$ in \eqref{art08-sec4-eq4.14} is induced by the Gysin homomorphism \eqref{art08-sec4-eq4.6} $i_{*}:H^{2q-1}(W,\bfC)\to H^{2q+2k-1}(V,\bfC)$. Write $i_{*}(f_{\rho})=\sum\limits_{\alpha=1}m_{\alpha\rho}e_{\alpha}$ so that $M=(m_{\alpha\rho})$ is an $m\times r$ matrix $M:\bfC^{r}\to \bfC^{m}$ which gives $i_{*}T_{q}(W)\to T_{q+k}(V)$.

Now $M\phi_{q}(W)(\lambda)=\left[\begin{smallmatrix}:\\ \sum\limits^{r}_{\rho=1}m_{\alpha\rho}\int\limits_{C}\phi^{\rho}\\ :\end{smallmatrix}\right]$ so that, to prove \eqref{art08-sec4-eq4.14}, we 
\begin{equation*}
\int\limits_{\gamma}\omega^{\alpha}=\sum\limits^{r}_{\rho=1}m_{\alpha\rho}\int\limits_{\gamma}\phi^{\rho}\tag{4.15}\label{art08-sec4-eq4.15}
\end{equation*}
for $\gamma$ a suitable $2n-2k-2q+1$ chain on $W$. Since
$$
i^{*}:H^{2n-2k-2q+1}(V,\bfC)\to H^{2n-2k-2q+1}(W,\bfC)
$$
satisfies $\int\limits_{W}i^{*}(\omega)\wedge \phi=\int\limits_{V}\omega \wedge i_{*}\phi$, we have $i^{*}\omega^{\alpha}=\sum\limits^{r}_{\rho=1}m_{\alpha\rho}\phi^{\rho}$ in $H^{2n-2k-2q+1}(W,\bfC)$. On the other hand, since $i^{*}$ satisfies $i^{*}\{F^{r}_{q}(V)\}\subset F^{r}_{q}(W)$, we have, as forms 
$$
i^{*}\omega^{\alpha}=\sum\limits^{r}_{\rho=1}m_{\alpha\rho}\phi^{\rho}+d\mu^{\alpha}(\mu^{\alpha}\in B^{2n-2k-2q+1, n-k-q+1}(W))
$$ 
so that $\int\limits_{C_{\lambda}}\omega^{\alpha}=\sum\limits^{r}_{\rho=1}m_{\alpha\rho}\int\limits_{C_{\lambda}}\phi^{\rho}$ as needed.
\end{proof}

\begin{remark*}
To\pageoriginale prove \eqref{art08-sec4-eq4.14} for $J_{q}(W)$ and $J_{q+k}(V)$, we use that $i^{*}\partial=\partial i^{*}$ and $i^{*}\overline{\partial}=\overline{\partial}i^{*}$ on the form level, so that $i^{*}\omega^{\alpha}=\sum\limits^{r}_{\rho=1}m_{\alpha\rho}\phi^{\rho}+\partial \overline{\partial}\xi^{\alpha}$ and then, as before,
$$
\int\limits_{C_{\lambda}}\omega^{\alpha}=\sum\limits^{r}_{\rho=1}m_{\alpha\rho}\int\limits_{C_{\lambda}}\phi^{\rho}.
$$
\end{remark*}

(c)~ We now combine (a) and (b) above. Thus let $W\subset V$ be a submanifold of codimension $k$ and $\{Z_{\lambda}\}_{\lambda\in \Delta}$ be a continuous system of codimension $q$ on $V$ such that $Z_{\lambda}\cdot W=Y_{\lambda}$ is a proper intersection. Then $\{Z_{\lambda}\}_{\lambda\in\Delta}$ defines $\phi_{q}:\Delta\to T_{q}(V)$ and $\{Y_{\lambda}\}_{\lambda\in \Delta}$ defines $\phi_{q+k}:\Delta\to T_{q+k}(V)$. Combining \eqref{art08-sec4-eq4.2} and \eqref{art08-sec4-eq4.14}, we find that the following is a commutative diagram:
\begin{equation*}
\vcenter{\xymatrix@C=1.5cm{
 & T_{q}(V)\ar[d]^-{i^{*}}\\
\Delta\ar[r]^{\phi_{q}(W)}\ar[dr]_-{\phi_{q+k}(V)}\ar[ur]^-{\phi_{q}(V)} & T_{q}(W)\ar[d]^-{i_{*}}\\
 & T_{q+k}(V)
}}\tag{4.16}\label{art08-sec4-eq4.16}
\end{equation*}
Combining \eqref{art08-sec2-eq2.13} with \eqref{art08-sec4-eq4.16}, we have the following commutative diagram:
\begin{equation*}
\vcenter{\xymatrix@C=1.5cm{
 & T_{q}(V)\ar[dd]^-{\Psi}\\
\Delta \ar[ur]_{\phi_{q}}\ar[dr]_-{\phi_{q+k}} & \\
 & T_{q+k}(V)
}}\tag{4.17}\label{art08-sec4-eq4.17}
\end{equation*}
where $\Psi\in H^{k,k}(V)\cap H^{2k}(V,\bfZ)$ is the Poincar\'e dual of $W\in H_{2n-2k}(V,\bfZ)$ (c.f. \eqref{art08-sec2-eq2.7}).

\begin{remark*}
Actually, we see that \eqref{art08-sec4-eq4.17} holds for {\em all} algebraic cycles $W_{n-k}\subset V_{n}$, provided we assume a foundational point concerning the {\em suitable equivalence relation} (= S.E.R.) in Remark \ref{art08-sec4-rem4.12}. Let $\Sigma_{q}(V)$ be the algebraic cycles of codimension $q$ which are homologous\pageoriginale to zero, and assume that S.E.R. has the property that, for any $W_{n-k}\subset V_{n}$, the mapping $\Sigma_{q}(V)/\text{S.E.R.}\xrightarrow{W}\Sigma_{q+k}(V)/\text{S.E.R.}$ is defined and $W(Z)=W\cdot Z$ if the intersection is proper $(Z\in \Sigma_{q}(V))$. Then we have that: The following diagram commutes:
\begin{equation*}
\vcenter{\xymatrix{
\Sigma_{q}(V)/\text{S.E.R.}\ar[dd]^-{W}\ar[r]^-{\phi_{q}} & T_{q}(V)\ar[dd]^-{\Psi} &\\
 & & (\Psi=\mathscr{D}(W)).\\
\Sigma_{q+k}(V)/\text{S.E.R.}\ar[r]^-{\phi_{q+k}} & T_{q+k}(V) &
}}\tag{4.18}\label{art08-sec4-eq4.18}
\end{equation*}
\end{remark*}

\begin{proof}
The proof of \eqref{art08-sec4-eq4.17} will show that \eqref{art08-sec4-eq4.18} commutes when $W$ is a Chern class of an {\em ample bundle} \cite{art08-key11}. However, by \cite{art08-key12} the Chern classes of ample bundles generate the {\em rational equivalence ring} on $V$, so that \eqref{art08-sec4-eq4.18} holds in general.
\end{proof}

\medskip
\noindent
{\bf Appendix to Paragraph \ref{art08-sec4}.}~ Let $\bfE\to V$ be a holomorphic vector bundle with fibre $\bfC^{k}$, and $\sigma_{1},\ldots,\sigma_{k-q+1}$ holomorphic cross-sections of $\bfE\to V$ such that the subvariety $W=\{\sigma_{1}\wedge\ldots\wedge \sigma_{k-q+1}=0\}$ is a generally singular subvariety $W_{n-q}\subset V_{n}$ of codimension $q$. (Note the shift in indices from \S\ref{art08-sec4}.) Then the homology class $W\in H_{2n-2k}(V,\bfZ)$ is the {\em Poincar\'e dual} of the $q^{\text{th}}$ {\em Chern class} $c_{q}\in H^{2q}(V,\bfZ)$ (c.f. \cite{art08-key11}). We shall prove: There exists a differential form $\psi$ on $V$ such that
\begin{align*}
& \psi \text{~ is of type~ }(q,q-1), \text{~ is~ } C^{\infty}\text{~ in~ } V-W,\text{~ and~ } \partial \psi=0;\tag{A4.1}\label{art08-sec4-eqA4.1}\\
& \overline{\partial}\psi =d\psi \text{~ is~ } C^{\infty}\text{~ on~ } V\text{~ and represents~} c_{q}~\text{(via deRham)};\tag{A4.2}\label{art08-sec4-eqA4.2}\\
& \psi \text{~ has a pole of order~ } 2q-1\text{~ along~ } W \text{~ and, if~ }\omega \text{~ is any closed}\\
& 2n-2q\text{~ form on~ } V,\int\limits_{V}c_{a}\wedge\omega=\lim\limits_{\epsilon\to 0}\int\limits_{\partial T_{\epsilon}}\psi\wedge \omega\text{~ where~ } T_{\epsilon}\subset V\text{~ is}\\
& \text{the~ }\epsilon\text{-tubular neighbourhood around~ } W.\tag{A4.3}\label{art08-sec4-eqA4.3}
\end{align*}

\begin{proof}
For a $k\times k$ matrix $A$, define $P_{q}(A)$ by:
\begin{equation*}
\det\left(\frac{i}{2\pi}A+tI\right)=\sum\limits^{k}_{q=0}P_{q}(A)t^{k-q}.\tag{A4.4}\label{art08-sec4-eqA4.4}
\end{equation*}
Let\pageoriginale $P_{q}(A_{1},\ldots,A_{q})$ be the invariant, symmetric multilinear form obtained by polarizing $P_{q}(A)$ (for example, 
$$
P_{k}(A_{1},\ldots,A_{k})=1/k!\sum\limits_{\pi_{1},\ldots,\pi_{k}}\det(A^{1}_{\pi_{1}},\ldots,A^{k}_{\pi_{k}})
$$ 
where $A^{\alpha}_{\pi_{\alpha}}$ is the $\alpha^{\text{th}}$ column of $A_{\pi_{\alpha}}$; cf. (6.5) below). Choose an Hermitian metric in $\bfE\to V$ and let $\Theta\in A^{1,1}(V,\Hom (\bfE,\bfE))$ be the curvature of the metric connection. Then (c.f. \cite{art08-key11}):

$c_{q}\in H^{2q}(V,\bfC)$ is represented by the differential form
\begin{equation*}
P_{q}(\Theta)=P_{q}(\Theta,\ldots,\Theta).\tag{A4.5}\label{art08-sec4-eqA4.5}
\end{equation*}
\end{proof}

What we want to do is to construct $\psi$, depending on $\sigma_{1},\ldots,\sigma_{k-q+1}$ and the metric, such that \eqref{art08-sec4-eqA4.1}--\eqref{art08-sec4-eqA4.3} are satisfied. The proof proceeds in four steps.


(a)~ \textsc{Some Formulae in Local Hermitian Geometry.} Suppose that $Y$ is a complex manifold ($Y$ will be $V-W$ in applications) and that $\bfE\to Y$ is a holomorphic vector bundle such that we have an exact sequence :
\begin{equation*}
0\to \bfS\to \bfE\to \bfQ\to 0\tag{A4.6}\label{art08-sec4-eqA4.6}
\end{equation*}
(in applications, $\bfS$ will be the trivial sub-bundle generated by $\sigma_{1},\ldots,\break \sigma_{k-q+1}$). We assume that there is an Hermitian metric in $\bfE$ and let $D$ be the metric connection \cite{art08-key11}. Let $e_{1},\ldots,e_{k}$ be a unitary frame for $\bfE$ such that $e_{1},\ldots,e_{s}$ is a frame for $\bfS$. Then $De_{\rho}=\sum\limits^{k}_{\sigma=1}\theta^{\alpha}_{\rho}e_{\sigma}$ where $\theta^{\sigma}_{\rho}+\overline{\theta}^{\rho}_{\sigma}=0$. By the formula $D_{\bfS}e_{\alpha}=\sum\limits^{s}_{\beta=1}\theta^{\beta}_{\alpha}e_{\beta}(\alpha=1,\ldots,s)$, there is defined a connection $D_{\bfS}$ in $\bfS$, and we claim that $D_{\bfS}$ is the connection for the induced metric in $\bfS$ (c.f. \cite{art08-key11}, \S1.d).

\begin{proof}
Choose a {\em holomorphic} section $e(z)$ of $\bfS$ such that $e(0)=e_{\alpha}(0)$ (this is over a small coordinate neighborhood on $Y$). Then $D''e=0$ since $D''=\overline{\partial}$. Thus, writing $e(z)=\sum\limits^{s}_{\alpha=1}\xi^{\alpha}e_{\alpha}$, $0=D''e=\sum\limits^{s}_{\alpha=1}\overline{\partial}\xi^{\alpha}e_{\alpha}+\sum\limits^{s}_{\alpha,\beta=1}\xi^{\alpha}\theta^{\beta''}_{\alpha}e_{\beta}+\sum\limits^{k}_{\mu=s+1}\sum\limits^{s}_{\alpha=1}\xi^{\alpha}\theta^{\mu''}_{\alpha}e_{\mu}$. At $z=0$, this gives $\sum\limits^{s}_{\beta=1}(\overline{\partial}\xi^{\beta}(0)+\theta^{\beta''}_{\alpha})e_{\beta}+\sum\limits^{k}_{\mu=s+1}\theta^{\mu''}_{\alpha}e_{\mu}=0$. Thus $\theta^{\mu''}_{\alpha}=0$ and, since $(D''-D''_{\bfS})e_{\alpha}=\sum\limits_{\mu=s+1}\theta^{\mu''}_{\alpha}e_{\mu}$,\pageoriginale $D''_{\bfS}=\overline{\partial}$. By uniqueness, $D_{\bfS}$ is the connection of the induced metric in $\bfS$.
\end{proof}

\medskip
\noindent
{\bf Remark. (\thnum{A4.7})\label{art08-A4.7}}~ Suppose that $\bfS$ has a {\em global holomorphic frame}\break $\sigma_{1},\ldots,\sigma_{s}$. Write $\sigma_{\alpha}=\sum\limits^{s}_{\beta=1}\xi^{\beta}_{\alpha}e_{\beta}$. From $0=\overline{\partial}\sigma_{\alpha}=D''\sigma_{\alpha}=\sum\limits^{s}_{\beta=1}(\overline{\partial}\xi^{\beta}_{\alpha}+\sum\limits^{s}_{\gamma=1}\xi^{\gamma}_{\alpha}\theta^{\beta''}_{\gamma})e_{\beta}$, we get $\overline{\partial}\xi+\theta''_{\bfS}\xi=0$ or $\theta''_{\bfS}=-\overline{\partial}\xi\xi^{-1}$. This gives
\begin{equation*}
\theta_{\bfS}={}^{t}\overline{\xi}^{-1}\partial \overline{\xi}-\overline{\partial}\xi\xi^{-1}.\tag{A4.8}\label{art08-sec4-eqA4.8}
\end{equation*}

Now write $\theta=\left(\begin{smallmatrix} \theta^{1}_{1} & \theta^{1}_{2}\\ \theta^{2}_{1} & \theta^{2}_{2}\end{smallmatrix}\right)$ where $\theta^{1}_{1}=(\theta^{\alpha}_{\beta})$, $\theta^{2}_{1}=(\theta^{\mu}_{\alpha})$, etc. Then $\theta^{2''}_{1}=0=\theta^{1'}_{2}$ (since $\theta^{2}_{1}+{}^{t}\overline{\theta}^{1}_{2}=0$). Let $\phi=\left(\begin{smallmatrix} 0 & - \theta^{1}_{2}\\ -\theta^{2}_{1} & 0\end{smallmatrix}\right)$ and $\widehat{\theta}=\theta+\phi=\left(\begin{smallmatrix} \theta^{1}_{1} & 0\\ 0 & \theta^{2}_{2}\end{smallmatrix}\right)$. Then $\theta$ and $\widehat{\theta}$ give connections $D$ and $\widehat{D}$ in $\bfE$ with curvatures $\Theta$ and $\widehat{\Theta}$. Setting $\theta_{t}=\theta+t\phi$, we have a homotopy from $\theta$ to $\widehat{\theta}$ with $\overdot{\theta}_{t}=\phi\left(\overdot{\theta}_{t}=\dfrac{\partial \theta_{t}}{\partial t}\right)$.

Now let $P(A)$ be an invariant polynomial of degree $q$ (c.f. \S\ref{art08-sec6} below) and $P(A_{1},\ldots,A_{q})$ the corresponding invariant, symmetric, multilinear from (c.f. (6.5) for an example). Thus $P(A)=P(\underbrace{A,\ldots,A}_{q})$. Set
\begin{equation*}
Q_{t}=2\sum\limits^{q}_{j=1}P(\theta_{t},\ldots,\phi'_{j},\ldots,\Theta_{t}),\tag{A4.9}\label{art08-sec4-eqA4.9}
\end{equation*}
and define $\Psi_{1}$ by:
\begin{equation*}
\Psi_{1}=\int\limits^{1}_{0}Q_{t}dt.\tag{A4.10}\label{art08-sec4-eqA4.10}
\end{equation*}
What we want to prove is (c.f. \cite{art08-key11}, \S\ref{art08-sec4}):
\begin{equation*}
\left.
\begin{array}{l}
\Psi_{1}\text{~ is a~ } C^{\infty}\text{~ form of type~ } (q,q-1)\text{~ on~ } Y\text{~ satisfying}\\[4pt]
\partial\Psi_{1}=0, \ \overline{\partial}\Psi_{1}=P(\Theta)-P(\widehat{\Theta}).
\end{array}
\right\}\tag{A4.11}\label{art08-sec4-eqA4.11}
\end{equation*}

\begin{proof}
It will suffice to show that
\begin{align*}
& \partial Q_{t}=0,\quad\text{and}\tag{A4.12}\label{art08-sec4-eqA4.12}\\[3pt]
& \overdot{P}(\Theta_{t})=\overline{\partial}Q_{t}.\tag{A4.13}\label{art08-sec4-eqA4.13}
\end{align*}

By\pageoriginale the {\em Cartan structure equation,} $\Theta_{t}=d\theta_{t}+\theta_{t}\wedge \theta_{t}=d(\theta+t\phi)+(\theta+t\phi)\wedge (\theta+t\phi)=d\theta+\theta\wedge\theta+t(d\phi+\phi\wedge \theta+\theta\wedge\phi)+t^{2}\phi\wedge \phi=\Theta+tD\phi+t^{2}\phi\wedge\phi$. Now
$$
\phi\wedge \phi=\left(\begin{matrix} \theta^{1}_{2}\theta^{2}_{1} & 0 \\ 0 & \theta^{2}_{1}\theta^{1}_{2}\end{matrix}\right)~\text{and}~ D\phi=\left(\begin{matrix} 0 & -\Theta^{1}_{2}\\ -\Theta^{2}_{1} & 0^{2}\end{matrix}\right)+2\left(\begin{matrix} -\theta^{1}_{2}\theta^{2}_{1} & 0\\ 0 & \theta^{2}_{1}\theta^{1}_{2}\end{matrix}\right).
$$
This gives
\begin{equation*}
\Theta_{t}=\Theta+t\left(\begin{matrix} 0 & -\Theta^{1}_{2}\\ -\Theta^{2}_{1} & 0\end{matrix}\right)+(t^{2}-2t)\left(\begin{matrix} \theta^{1}_{2}\theta^{2}_{1} & 0\\ 0 & \theta^{2}_{1}\theta^{1}_{2}\end{matrix}\right);\tag{A4.14}\label{art08-sec4-eqA4.14}
\end{equation*}
\begin{equation*}
D'\phi'=0;\quad\text{and}\tag{A4.15}\label{art08-sec4-eqA4.15}
\end{equation*}
\begin{equation*}
D''\phi'=\left(\begin{matrix} 0 & 0 \\ -\Theta^{2}_{1} & 0\end{matrix}\right)+\left(\begin{matrix}-\theta^{1}_{2}\theta^{2}_{1} & 0\\ 0 & -\theta^{2}_{1}\theta^{1}_{2}\end{matrix}\right)\tag{A4.16}\label{art08-sec4-eqA4.16}
\end{equation*}
It follows that $\Theta_{t}$ is of type $(1,1)$ and so $Q_{t}$ is of type $(q,q-1)$, as is $\Psi_{1}$.

By symmetry, to prove \eqref{art08-sec4-eqA4.12} it will suffice to have 
$$
\partial P(\Theta_{t},\ldots,\Theta_{t},\phi')=0.
$$ 
Let $D_{t}=D'_{t}+D''_{t}$ be the connection corresponding to $\theta_{t}$. Then $D'_{t}\Theta_{t}=0$ (\textit{Bianchi identity}) and $\partial P(\Theta_{t},\ldots,\Theta_{t},\phi')=\Sigma P(\Theta_{t},\ldots,D'_{t}\Theta_{t},\ldots,\Theta_{t},\phi')\break +P(\Theta_{t},\ldots,\Theta_{t},D'_{t}\phi')=P(\Theta_{t},\ldots,\Theta_{t},D'_{t}\phi')$. But $D'_{t}\phi'=D'\phi'+t[\phi,\phi']'\break =0+t[\phi',\phi']=0$ by \eqref{art08-sec4-eqA4.15}. This proves \eqref{art08-sec4-eqA4.12}.

We now calculate 
\begin{gather*}
\overline{\partial}P(\Theta_{t},\ldots,\phi',\ldots,\Theta_{t})=\Sigma P(.,D''_{t}\Theta_{t},\ldots,\phi',\ldots,\Theta_{t})+\\
 +P(\Theta_{t},\ldots,D''_{t}\phi',\ldots,\Theta_{t})+\Sigma P(\Theta_{t},\ldots,\phi',\ldots,D''_{t}\Theta_{t},.)\\
=P(\Theta_{T},\ldots,D''_{t}\phi',\ldots,\Theta_{t})
 \end{gather*}
(since $D''_{t}\Theta_{T}=0$ by Bianchi). Then we have $D''_{t}\phi'=D''\phi'+t[\phi,\phi']''=$
\begin{gather*}
(\text{by~ \eqref{art08-sec4-eqA4.16}}) \ \ \left(\begin{matrix} 0 & 0 \\ -\Theta^{2}_{1} & 0\end{matrix}\right)+(t-1)\left(\begin{matrix} \theta^{1}_{2} \theta^{2}_{1} & 0 \\ 0 & \theta^{2}_{1}\theta^{1}_{2}\end{matrix}\right). \ \ \text{But, by~ \eqref{art08-sec4-eqA4.14},}\\[3pt]
\overdot{\Theta}_{t}=\left(\begin{matrix} 0 & -\Theta^{1}_{2}\\ - \Theta^{2}_{1} & 0\end{matrix}\right)+2(t-1)\left(\begin{matrix} \theta^{1}_{2}\theta^{2}_{1} & 0\\ 0 & \theta^{2}_{1}\theta^{1}_{2}\end{matrix}\right),
\end{gather*}
so that
\begin{equation*}
2D''_{t}\phi'-\overdot{\Theta}_{T}=\left(\begin{matrix} 0 & \Theta^{1}_{2}\\ -\Theta^{2}_{1} & 0\end{matrix}\right)=[\pi,\Theta],\tag{A4.17}\label{art08-sec4-eqA4.17}
\end{equation*}
where $\pi=\left(\begin{smallmatrix} 1 & 0\\ 0 & 0\end{smallmatrix}\right)$. Using \eqref{art08-sec4-eqA4.17}, $\overline{\partial}Q_{t}-\overdot{P}(\Theta_{t})=\Sigma\{2P(\Theta_{t},\ldots,D''_{t}\phi',\break \ldots,\Theta_{t})-P(\Theta_{t},\ldots,\overdot{\Theta}_{t},\ldots,\Theta_{t})\}=\Sigma P(\Theta_{t},\ldots,[\pi,\Theta_{t}],\ldots,\Theta_{t})=0$. This proves \eqref{art08-sec4-eqA4.13} and completes the proof of \eqref{art08-sec4-eqA4.11}.
\end{proof}

Return\pageoriginale now to the form $Q_{t}$ defined by \eqref{art08-sec4-eqA4.9}. Since $\Theta_{T}=\Theta+tD\phi+t^{2}\phi\wedge \phi$, we see that $\underline{Q_{t}}$ is a polynomial in the differential forms $\Theta^{\rho}_{\sigma}$, $\theta^{\mu}_{\alpha}$, $\theta^{\alpha}_{\mu}(1\leq \rho, \sigma\leq k; 1\leq \alpha\leq s;s+1\leq \mu\leq k)$. Write $Q_{t}\equiv 0(l)$ to symbolize that each term in $Q_{t}$ contains {\em no more} than $l-t$ terms involving the $\theta^{\mu}_{\alpha}$ and $\theta^{\alpha}_{\mu}$. We claim that
\begin{equation*}
Q_{t}\equiv 0(2q-1).\tag{A4.18}\label{art08-sec4-eqA4.18}
\end{equation*}

\begin{proof}
The term of highest order (i.e. containing the most $\theta^{\mu}_{\alpha}$ and $\theta^{\alpha}_{\mu}$) in $Q_{t}$ is $(t^{2}/2)^{q-1}\Sigma P([\phi,\phi],\ldots,\phi',\ldots,[\phi,\phi])$. Now, by invariance,
\begin{align*}
P([\phi,\phi],\ldots,\phi',\ldots,[\phi,\phi]) &= -P(\phi,[\phi,\phi],\ldots,[\phi,\phi'],\ldots,[\phi,\phi])\\[3pt]
&= -\frac{1}{2}P(\phi,[\phi,\phi],\ldots,[\phi,\phi])
\end{align*}
since $[\phi,[\phi,\phi]]=0$ and $[\phi,\phi']=\frac{1}{2}[\phi,\phi]$. But, by invariance again, $P(\phi,[\phi,\phi],\ldots,[\phi,\phi])=0$. Since all other terms in $Q_{t}$ are of order $2q-2$ or less, we obtain \eqref{art08-sec4-eqA4.18}.

It follows from \eqref{art08-sec4-eqA4.10} that
\begin{equation*}
\Psi_{1}\equiv 0(2q-1).\tag{A4.19}\label{art08-sec4-eqA4.19}
\end{equation*}

(b) \textsc{Some further formulae in Hermitian geometry.} Retaining the situation \eqref{art08-sec4-eqA4.6}, we have from \eqref{art08-sec4-eqA4.11} and \eqref{art08-sec4-eqA4.19} that:
\begin{equation*}
P(\Theta)-P(\widehat{\Theta})=\overline{\partial}\Psi_{1}\text{~~ where~~ } \partial\Psi_{1}=0,\Psi_{1}\equiv 0(2q-1).\tag{A4.20}\label{art08-sec4-eqA4.20}
\end{equation*}
Now suppose that $\bfS$ has fibre dimension $k-q+1$ and that:
\begin{equation*}
\bfS\text{~~ has a global holomorphic frame~~ } \sigma_{1},\ldots,\sigma_{k-q+1}.\tag{A4.21} \label{art08-sec4-eqA4.21}
\end{equation*}
Let $\bfL_{1}\subset \bfS$ be line bundle generated by $\sigma_{1}$ and $\bfS_{1}=\bfS/\bfL_{1}$. Then $\sigma_{2}$ gives a non-vanishing section of $\bfS_{1}$ and so generates a line bundle $\bfL_{2}\subset \bfS_{1}$. Continuing, we get a diagram:
\begin{equation*}
\left.
\begin{array}{l@{\;}l@{\;}l@{\;}l}
0\longrightarrow \bfL_{1} & \longrightarrow \bfS & \longrightarrow \bfS_{1} & \longrightarrow 0\\
0\longrightarrow \bfL_{2} & \longrightarrow \bfS_{1} & \longrightarrow \bfS_{2} & \longrightarrow 0\\
 & \multicolumn{1}{c}{$\vdots$} & & \\
0\longrightarrow \bfL_{k-q} & \longrightarrow \bfS_{k-q-1} & \longrightarrow \bfS_{k-q} & \longrightarrow 0\\
0\longrightarrow \bfL_{k-q+1} & \longrightarrow \bfS_{k-q} & \longrightarrow 0 &
\end{array}
\right\}\tag{A4.22}\label{art08-sec4-eqA4.22}
\end{equation*}
All\pageoriginale the bundles in \eqref{art08-sec4-eqA4.22} have metrics induced from $\bfS$; as a $C^{\infty}$ bundle, $\bfS\cong \bfL_{1}\oplus\cdots\oplus \bfL_{k-q+1}$ (this is actually true as holomorphic bundles, but the splitting will {\em not} be this orthonormal one).

Now suppose that we use unitary frames $(e_{1},\ldots,e_{k-q+1})$ for $\bfS$ where $e_{\alpha}$ is a unit vector in $\bfL_{\alpha}$. If $\theta_{\bfS}=(\theta^{\alpha}_{\beta})$ is the metric connection in $\bfS$, then $\theta^{\alpha}_{\alpha}$ gives the connection of the induced metric in $\bfL_{\alpha}$ (c.f. (a) above). This in turn gives a connection
\begin{align*}
\gamma_{\bfS} &=
\begin{bmatrix}
\gamma^{1}_{1} & & 0\\
 & \ddots & \\
0 & & \gamma^{k-q+1}_{k-q+1}
\end{bmatrix}\quad 
(\gamma^{\alpha}_{\alpha}=\theta^{\alpha}_{\alpha})\text{~ with curvature~ }\\
\Gamma_{\bfS} &=
\begin{bmatrix}
\Gamma^{1}_{1} & & 0\\
        & \ddots & \\
0 & & \Gamma^{k-q+1}_{k-q+1}
\end{bmatrix}
\end{align*}
in $\bfS$. Now the connection $\widehat{\theta}=\theta_{\bfS}\oplus \theta_{\bfQ}$ in $\bfE$ has curvature $\widehat{\Theta}=\left[\begin{smallmatrix} \Theta_{\bfS} & 0\\ 0 & \Theta_{\bfQ}\end{smallmatrix}\right]$. We let $\Gamma=\left[\begin{smallmatrix} \Gamma_{\bfS} & 0\\ 0 & \Theta_{\bfQ}\end{smallmatrix}\right]$ be the curvature of the connection $\left[\begin{smallmatrix} \gamma_{\bfS} & 0\\ 0 & \theta_{\bfQ}\end{smallmatrix}\right]$ in $\bfE$. Then the same argument as used in (a) to prove \eqref{art08-sec4-eqA4.20}, when iterated, gives
\begin{equation*}
P(\widehat{\Theta})-P(\Gamma)=\overline{\partial}\Psi_{2}\quad\text{where}\quad \partial \Psi_{2}=0\text{~ and~ } \Psi_{2}\equiv 0(2q).\tag{A4.23}\label{art08-sec4-eqA4.23}
\end{equation*}

The congruence $\Psi_{2}\equiv 0(2q)$ is trivial in this case since $\deg \Psi_{2}=2q-1$. Adding \eqref{art08-sec4-eqA4.20} and \eqref{art08-sec4-eqA4.23} gives:
\begin{equation*}
P(\Theta)-P(\Gamma)=\overline{\partial}(\Psi_{1}+\Psi_{2}).\tag{A4.24}\label{art08-sec4-eqA4.24}
\end{equation*}

The polynomial $P(A)$ is of degree $q$, and we assume now that: 
\begin{equation*}
P(A)=0 \text{~ if~ } A=\left(\begin{matrix}0 & 0\\ 0 & A'\end{matrix}\right)\text{~ where~ } A' \text{~ is a~ } (q-1)\times (q-1)\text{~ matrix.}\tag{A4.25}\label{art08-sec4-eqA4.25}
\end{equation*}
\end{proof}

We claim then that
\begin{equation*}
P(\Gamma)=\overline{\partial}\Psi_{3}\text{~ where~ } \partial \Psi_{3}=0\text{~ and~ }\Psi_{3}\equiv 0(2q).\tag{A4.26}\label{art08-sec4-eqA4.26}
\end{equation*}

\begin{proof}
Each line bundle $\bfL_{\alpha}$ has a holomorphic section $\sigma_{\alpha}=|\sigma_{\alpha}|e_{\alpha}$. From $0=\overline{\partial}\sigma_{\alpha}=\overline{\partial}|\sigma_{\alpha}|e_{\alpha}+|\sigma_{\alpha}|\theta^{\alpha''}_{\alpha}e_{\alpha}$, we find $\theta^{\alpha''}_{\alpha}=-\overline{\partial}\log |\sigma_{\alpha}|$ and 
\begin{align*}
& \theta^{\alpha}_{\alpha}=(\partial - \overline{\partial})\log |\sigma_{\alpha}|,\quad\text{and}\tag{A4.27}\label{art08-sec4-eqA4.27}\\[3pt]
& \Gamma^{\alpha}_{\alpha}=2\overline{\partial}\partial \log |\sigma_{\alpha}|.\tag{A4.28}\label{art08-sec4-eqA4.28}
\end{align*}

Now\pageoriginale 
$$
P(\Gamma)=P\underbrace{(\Gamma_{S}+\Theta_{\bfQ},\ldots,\Gamma_{\bfS}+\Theta_{\bfQ})}_{q}=\sum\limits_{\substack{r+s=q\\ r>0}}\binom{q}{r}P(\underbrace{\Gamma_{S}}_{r};\underbrace{\Theta_{\bfQ}}_{s})
$$ 
(since $P(\underbrace{\Theta_{\bfQ},\ldots,\Theta_{\bfQ}}_{q})=0$ by \eqref{art08-sec4-eqA4.25}). 
Let 
$$
\xi=2\left[\begin{array}{llll} \theta^{1'}_{1} & &\\
 & \ddots & 0 & 0\\
0 & &\theta^{k-q+1'}_{k-q+1} &\\
 & & 0 & \theta^{\mu}_{\nu}
 \end{array}\right].
$$
Then $D'_{\gamma}\xi=0$ where $\gamma$ is the connection
$$
\left[\begin{array}{llll} \theta^{1}_{1} & &\\
 & \ddots &  & 0\\
 & &\theta^{k-q+1}_{k-q+1} &\\
 & & 0 & \theta^{\mu}_{\nu}
 \end{array}\right]
$$
in $\bfL_{1}\oplus\cdots\oplus \bfL_{k-q+1}\Theta\bfQ$. Also $D''_{\gamma}\xi=\left(\begin{smallmatrix} \Gamma_{\bfS} & 0\\ 0 & 0\end{smallmatrix}\right)$. Set
\begin{equation*}
\Psi_{3}=\sum\limits_{\substack{r+s=q\\ r>0}}\binom{q}{r}P(\xi,\underbrace{\Gamma_{\bfS}}_{r-1};\underbrace{\Theta_{\bfQ}}_{s}).\tag{A4.29}\label{art08-sec4-eqA4.29}
\end{equation*}
Then $\partial \Psi_{3}=0$ since $D'_{\gamma}\xi=0=D'_{\gamma}\Gamma_{\bfS}=D'_{\gamma}\Theta_{\bfQ}$, and 
$$
\overline{\partial}\Psi_{3}=\sum\limits_{\substack{r+s=q\\ r>0}}\binom{q}{r}P(\underbrace{\Gamma_{\bfS}}_{r};\underbrace{\Theta_{\bfQ}}_{s})=P(\Gamma)
$$ 
since $D''_{\gamma}\xi=\Gamma_{\bfS}$, $D''_{\gamma}\Gamma_{\bfS}=0=D''_{\gamma}\Theta_{\bfQ}$.
This shows that $\Psi_{3}$ defined by \eqref{art08-sec4-eqA4.29} satisfies \eqref{art08-sec4-eqA4.26}.

Combining \eqref{art08-sec4-eqA4.24} and \eqref{art08-sec4-eqA4.26} gives:
\begin{equation*}
P(\Theta)=\overline{\partial}\Psi \text{~ where~ } \Psi=\Psi_{1}+\Psi_{2}+\Psi_{3},\partial \Psi=0,\Psi\equiv 0(2q).\tag{A4.30}\label{art08-sec4-eqA4.30}
\end{equation*}

Let $\Psi$ be given as just above by \eqref{art08-sec4-eqA4.30}; $\Psi$ is a form of type $(q,q-1)$ on $Y$. Suppose we refine the congruence symbol $\equiv$ so that $\eta\equiv 0(l)$ means that $\eta$ contains at most $l=1$ terms involving $\theta^{1'}_{1}$, $\theta^{k-2+q}_{1},\ldots,\theta^{k}_{1}$, $\theta^{1}_{k-q+2},\ldots,\theta^{1}_{k}$. Then, for some constant $c$,
\begin{equation*}
\Psi \equiv c\theta^{1'}_{1}\theta^{k-2+q}_{1}\ldots\theta^{k}_{1}\theta^{1}_{k-q+2}\ldots\theta^{1}_{k}(2q-1).\tag{A4.31}\label{art08-sec4-eqA4.31}
\end{equation*}
We want to calculate $c$ when $P(A)=P_{q}(A)$ corresponds to the $q^{\text{th}}$ Chern class (c.f. \eqref{art08-sec4-eqA4.4}). By \eqref{art08-sec4-eqA4.19}, $\Psi_{1}\equiv 0(2q-1)$ and an inspection of \eqref{art08-sec4-eqA4.9} shows that $\Psi_{2}\equiv 0(2q-1)$. Thus $\Psi\equiv \Psi_{3}(2q-1)$.

To calculate $\Gamma_{\bfS}$, we have $\Gamma^{\alpha}_{\alpha}=d\theta^{\alpha}_{\alpha}=d\theta^{\alpha}_{\alpha}+\sum\limits^{k}_{\rho=1}\theta^{\alpha}_{\rho}\wedge \theta^{\rho}_{\alpha}-\sum\limits^{k}_{\rho=1}\theta^{\alpha}_{\rho}\wedge \theta^{\rho}_{\alpha}=\Theta^{\alpha}_{\alpha}-\sum\limits^{k}_{\rho=1}\theta^{\alpha}_{\rho}\wedge \theta^{\rho}_{\alpha}$. Thus $\Gamma^{\alpha}_{\alpha}\equiv 0(0)$ for $\alpha>1$ and 
$$
\Gamma^{1}_{1}\equiv -\sum\limits^{k}_{\mu=k-q+2}\theta^{1}_{\mu}\wedge \theta^{\mu}_{1}(0).
$$\pageoriginale
It follows that $P_{q}(\xi,\underbrace{\Gamma_{\bfS}}_{r-1}\underbrace{\Theta_{\bfQ}}_{s})\equiv 0(2q-1)$ if $r-1>0$, and so $\Psi_{3}\equiv P_{q}(\xi, \underbrace{\Theta_{\bfQ}}_{q-1})(2q-1)$.

Now, by the definition of 
$$
P_{q}, P_{q}(\xi,\underbrace{\Theta_{\bfQ}}_{q-1})=(i/2\pi)^{q}(1/q)\theta^{1'}_{1}\det(\Theta_{\bfQ}).
$$ 
But $(\Theta_{\bfQ})^{\mu}_{\nu}=\Theta^{\mu}_{\nu}+\sum\limits^{k-q+1}_{\alpha=1}\theta^{\mu}_{\alpha}\wedge \theta^{\alpha}_{\nu}$, so that $(\Theta_{\bfQ})^{\mu}_{\nu}\equiv \theta^{\mu}_{1}\wedge\theta^{1}_{\nu}(0)$. Combining these relations gives $\Psi_{3}\equiv (i/2\pi)^{q}(1/q)\theta^{1'}_{1}\det (\theta^{\mu}_{1}\theta^{1}_{\nu})(2q-1)$ or 
\begin{equation*}
\Psi_{3}\equiv \left(\frac{1}{2\pi i}\right)^{q}(-1)^{q(q-1)/2}\theta^{1'}_{1}\theta^{k-q+2}_{1}\ldots\theta^{k}_{1}\theta^{1}_{k-q+2}\ldots\theta^{1}_{k}(2q-1).\tag{A4.32}\label{art08-sec4-eqA4.32}
\end{equation*}

Combining \eqref{art08-sec4-eqA4.30} and \eqref{art08-sec4-eqA4.32} gives
\begin{equation*}
P_{q}(\Theta)=\overline{\partial}\Psi\quad\text{where}\quad \partial \psi=0\tag{A4.33}\label{art08-sec4-eqA4.33}
\end{equation*}
and 
\begin{equation*}
\left.
\begin{array}{c}
\Psi\equiv -\Gamma(q)\theta^{1'}_{1}\theta^{k-q+2}_{1}\ldots\theta^{k}_{1}\theta^{1}_{k-q+2}\ldots\theta^{1}_{k}(2q-1)\\[3pt]
(\Gamma(q)=(1/2\pi i)(-1)^{q(q-1)/2}).
\end{array}\right\}\tag{A4.34}\label{art08-sec4-eqA4.34}
\end{equation*}
\end{proof}

(c)~ \textsc{Reduction to a local problem.} Return now to the notation and assumptions at the beginning of the Appendix to \S\ref{art08-sec4}. Taking into account \eqref{art08-sec4-eqA4.6}, \eqref{art08-sec4-eqA4.21}, \eqref{art08-sec4-eqA4.30}, and letting $Y=V-W$, we have constructed a $(q,q-1)$ form $\psi$ on $V-W$ such that $\partial\psi=0$, $\overline{\partial}\psi=P_{q}(\Theta)=c_{q}$, and such that $\psi\equiv 0(2q)$. This proves \eqref{art08-sec4-eqA4.1}, \eqref{art08-sec4-eqA4.2}, and, in the following section, we will interpret $\psi\equiv 0(2q)$ to mean that $\psi$ has a pole of order $2q-1$ along $W$.

Let $\omega$ be a closed $2n-2q$ form on $V$ and $T_{\epsilon}$ the $\epsilon$-tube around $W$. Then $\int\limits_{V}c_{q}\wedge\omega =\lim\limits_{\epsilon\to 0}\int\limits_{V-T_{\epsilon}}c_{q}\wedge \omega=\lim\limits_{\epsilon\to 0}-\int\limits_{\partial T_{\epsilon}}\psi\wedge \omega$ since $d(\psi\wedge \omega)=P_{q}(\Theta)\wedge \omega$. This proves \eqref{art08-sec4-eqA4.3}.

For the purposes of the proof of \eqref{art08-sec4-eq4.2}, we need a stronger version of \eqref{art08-sec4-eqA4.3}; namely, we need that
\begin{equation*}
\lim\limits_{\epsilon\to 0}-\int\limits_{C\cdot \partial T_{\epsilon}}\psi \wedge \eta=\int\limits_{C\cdot W}\eta,\tag{A4.35}\label{art08-sec4-eqA4.35}
\end{equation*}
where\pageoriginale $\eta\in B^{2n-2k-2q+1,n-k-q+1}(V)$ and $C$ is the $2n-2q+1$ chain on $V$ used in the proof of \eqref{art08-sec4-eq4.2}. In other words, we need to show that {\em integration with respect to $\psi$ is a residue operator along $W$}. Because both sides of \eqref{art08-sec4-eqA4.35} are linear in $\eta$, we may assume that $\eta$ has support in a coordinate neighborhood. Also, because $\psi$ will have a pole only of order $2q-1$ along $W$, it will be seen that both sides of \eqref{art08-sec4-eqA4.35} will remain unchanged if we take out of $W_{n-q}$ an algebraic hypersurface $H_{n-q-1}$ which is in general position with respect to $C$. Thus, to prove \eqref{art08-sec4-eqA4.35}, we may assume that:
\begin{equation*}
\left.
\begin{array}{l}
\eta\text{~ has support in a coordinate neighborhood on~ } V\\[3pt]
\text{where~ } \sigma_{2}\wedge\ldots\wedge \sigma_{k-q+2}\neq 0, \ \ \sigma_{1}\neq  0.
\end{array}\right]\tag{A4.36}\label{art08-sec4-eqA4.36}
\end{equation*}

This is a local question which will be resolved in the next section.

We note in passing that \eqref{art08-sec4-eq4.9} follows from \eqref{art08-sec4-eq4.10} when $C$ is a cycle on $V$, so that \eqref{art08-sec4-eq4.2} will be completely proved when \eqref{art08-sec4-eqA4.35} is proved in the local form \eqref{art08-sec4-eqA4.35} above.

\eject

(d)~ \textsc{Completion of the proof.} Over $\bfC^{n}$ consider the trivial bundle $\bfE=\bfC^{n}\times \bfC^{k}$ in which we have a Hermitian metric $(h_{\rho\sigma}(z))$ ($z=\left[\begin{smallmatrix} z^{1}\\\vdots \\[2pt] z^{n}\end{smallmatrix}\right]$ are coordinates in $\bfC^{n}$; $1\leq \rho$, $\sigma\leq k$). We suppose that there are holomorphic sections $\sigma_{2},\ldots,\sigma_{k-q+1}$ generating the sub-bundle $\bfS'=\bfC^{n}\times \{\bfO^{q}\times \bfC^{k-q}\}$ of $\bfE$, and let $\sigma_{1}$ be a holomorphic section of the form $\sigma_{1}(z)=\left[\begin{smallmatrix} z^{1}\\ \vdots\\ z^{q}\\ 0\\ \vdots\\ 0\end{smallmatrix}\right]$. Then the locus $\sigma_{1}\wedge\ldots\wedge \sigma_{k-q+1}=0$ is given by $z^{1}=\ldots=z^{q}=0$, so that we have the local situation of \eqref{art08-sec4-eqA4.6} ($\bfS$ is generated by $\bfS'$ and $\sigma_{1}$ on $\bfC^{n}-\bfC^{n-q}$), \eqref{art08-sec4-eqA4.21} and \eqref{art08-sec4-eqA4.36}. We consider unitary frames $e_{1},\ldots,e_{k}$ for $\bfE$ where $e_{1}=\dfrac{\sigma_{1}}{|\sigma_{1}|}$, and $e_{2},\ldots,e_{k-q+1}$ is a frame for $\bfS'$. Thus $e_{1},\ldots,e_{k-q+1}$ is a frame for $\bfS | \bfC^{n}-\bfC^{n-q}$.

Write\pageoriginale $De_{1}=\sum\limits^{k}_{\rho=1}\theta^{\rho}_{1}e_{\rho}$ ($\theta^{\rho}_{1}$ is of type $(1,0)$ for $\rho>1$) and set $\Omega=-\Gamma(q)\theta^{1'}_{1}\theta^{k-q+2}_{1}\ldots\theta^{k}_{1}\theta^{1}_{k-q+2}\ldots\theta^{1}_{k}$. If $\eta$ is a compactly supported $2n-2q$ form on $\bfC^{n}$, we want to show:
\begin{equation*}
\int\limits_{\bfC^{n-q}}\eta=-\lim\limits_{\epsilon\to 0}\int\limits_{\partial T_{\epsilon}}\Omega\wedge \eta,\tag{A4.37}\label{art08-sec4-eqA4.37}
\end{equation*}
where $T_{\epsilon}$ is an $\epsilon$-ball around $\bfC^{n-q}\subset \bfC^{n}$. Having done this, we will, by almost exactly the same argument, prove \eqref{art08-sec4-eqA4.35}.

Using the metric connection, we write $De_{\rho}=\sum\limits^{k}_{\sigma=1}\theta^{\sigma}_{\rho}e_{\sigma}$. Then the 1-forms $\theta^{\rho}_{\alpha}$ are smooth on $\bfC^{n}$ for $\rho\neq 1$, $\sigma\neq 1$. If we can show that the $\theta^{\rho}_{1}$ have a first order pole along $\bfC^{n-q}\subset \bfC^{n}$, then it will follow that $\Omega$ has a pole of order $2q-1$ along $\bfC^{n-q}$ and that our congruence symbol ``$\equiv$'' (c.f. just above \eqref{art08-sec4-eqA4.31}) refers to the order of pole along $\bfC^{n-q}$. We consider each vector $e_{\rho}=\left[\begin{smallmatrix} e^{1}_{\rho}(z)\\\vdots\\e^{k}_{\rho}(z)\end{smallmatrix}\right]$ as a vector field $f_{\rho}=\sum\limits^{k}_{\sigma=1}e^{\sigma}_{\rho}(z)\dfrac{\partial}{\partial z^{\sigma}}$ on $\bfC^{n}$ and, letting $f_{a}=\dfrac{\partial}{\partial z^{a}}(a=k+1,\ldots,n)$, we have a {\em tangent vector frame} $f_{1},\ldots,f_{n}$ on $\bfC^{n}$ such that $f_{2},\ldots,f_{k-q+1}$, $f_{k+1},\ldots,f_{n}$ are tangent to $\bfC^{n-q}\subset \bfC^{n}$ along $\bfC^{n-q}$. Let $\omega^{1},\ldots,\omega^{n}$ be the co-frame of $(1,0)$ forms; then if $z=\sum\limits^{n}_{j=1}z^{j}\dfrac{\partial}{\partial z^{j}}$, $dz=\sum\limits^{n}_{j=1}f_{j}\omega^{j}$. But $z=\sigma_{1}+\sum\limits^{k-q+2}_{\alpha=2}\lambda_{\alpha}e_{\alpha}+\sum\limits^{n}_{a=k+1}z^{a}f_{a}$, so that $dz=D'\sigma_{1}\sum\limits^{k-q+2}_{\alpha=2}(\partial \lambda_{\alpha}e_{\alpha}+\lambda_{\alpha}D'e_{\sigma})+\sum\limits^{n}_{a=k+1}f_{a}\omega^{a}$. This gives:
\begin{equation*}
\left.
\begin{array}{@{}ll}
\omega^{1}=2|\sigma_{1}|\theta^{1'}_{1}; & \\[4pt]
\omega^{\mu}=|\sigma_{1}|\theta^{\mu}_{1}+\sum\limits^{k-q+1}_{\alpha=2}\lambda_{\alpha}\theta^{\mu}_{\alpha} & (\mu=k-q+2,\ldots,k);\\[4pt]
\omega^{\alpha}=\partial \lambda_{\alpha}+\sum\limits^{k-q+1}_{\beta=2}\lambda_{\beta}\theta^{\alpha'}_{\beta} & (\alpha=2,\ldots,k-q+1);\text{~ and}\\
\omega^{a}= dz^{a} & (a=k+1,\ldots,n).
\end{array}
\right\}\tag{A4.38}\label{art08-sec4-eqA4.38}
\end{equation*}
It\pageoriginale follows that $\theta^{1}_{1}$, $\theta^{\mu}_{1}$ have a first order pole along $\bfC^{n-q}$ and that 
\begin{equation*}
\Omega\equiv \left(\dfrac{i}{2}\right)^{q}(-1)^{q(q-1)/2}\dfrac{1}{|\sigma_{1}|^{2q-1}}\omega^{1}\omega^{k-q+2}\ldots\omega^{k}\overline{\omega}^{k-q+2}\ldots\overline{\omega}^{k}(2q-1).\tag{A4.39}\label{art08-sec4-eqA4.39}
\end{equation*}

The situation is now this: On $\bfC^{n}$, let $f_{1},\ldots,f_{n}$ be a tangent frame such that $f_{q+1},\ldots,f_{n}$ is a frame for $\{\bfQ^{q}\times \bfC^{n-q}\}\subset \bfC^{n}$ (thus $f_{1},\ldots,f_{q}$ is a normal frame for $\bfC^{n-q}\subset \bfC^{n}$). Let $\omega^{1},\ldots,\omega^{n}$ be the dual co-frame and $\eta$ be a compactly supported $2n-2q$ form. Then we need
\begin{equation*}
-\lim\limits_{\epsilon \to 0}\int\limits_{\partial T\epsilon}\eta \wedge \Lambda=\int\limits_{\bfC^{n-q}}\eta\tag{A4.40}\label{art08-sec4-eqA4.40}
\end{equation*}
where $\Lambda =\left(\dfrac{i}{2}\right)^{q}(-1)^{q(q-1)/2}\dfrac{1}{|\sigma|^{2q-1}}\omega^{1}\omega^{2}\ldots\omega^{q}\overline{\omega}^{1}\ldots\overline{\omega}^{q}$, $T_{\epsilon}$ is the $\epsilon$-tube around $\bfC^{n-q}\subset \bfC^{n}$, and $\sigma=\left[\begin{smallmatrix} z^{1}\\\vdots\\z^{q}\\0\\\vdots\\0\end{smallmatrix}\right]$, $f_{1}=\dfrac{\sigma}{|\sigma|}$.

If now the metric in the tangent frame is the flat Euclidean one and $T_{\epsilon}$ the normal neighborhood of radius $\epsilon$, then $\Lambda$ is minus the volume element on the normal sphere of radius $\epsilon$. Writing $\eta=(\eta(0,z)+|f_{1}|\widehat{\eta})\omega^{q+1}\wedge\ldots\wedge \omega^{n}\wedge \overline{\omega}^{q+1}\wedge\ldots\wedge\overline{\omega}^{n}+\eta'$ where 
$$
\eta'\equiv 0(\omega^{1},\ldots,\omega^{q},\overline{\omega}^{1},\ldots,\overline{\omega}^{q}),
$$ 
it follows that
$$
-\lim\limits_{\epsilon\to 0}\int\limits_{\partial T_{\epsilon}}\eta \wedge \Lambda =\int\limits_{\bfC^{n-q}}\eta(0,z)\omega^{q+1}\ldots\omega^{n}\overline{\omega}^{q+1}\ldots\overline{\omega}^{n}=\int\limits_{\bfC^{n-q}}\eta.
$$ 
On the other hand, if $\widehat{T}_{\epsilon}$ is another $\epsilon$-tube aroung $\bfC^{n-q}$, by Stokes' theorem
$$
|\int\limits_{\partial T_{\epsilon}}\eta \wedge \Lambda-\int\limits_{\partial \widehat{T}_{\epsilon}}\eta \wedge \Lambda |\leq |\int\limits_{T_{\epsilon}\cup \widehat{T}_{\epsilon}}d(\eta\wedge\Lambda)|.
$$
Since $\eta$ is smooth and $d\Lambda$ has a pole of order $\leq 2q-1$ along $\bfC^{n-q}$ (in fact, we may assume $d\Lambda=0$), $\lim\limits_{\epsilon\to 0}|\int\limits_{T_{\epsilon}\cup \widehat{T}_{\epsilon}}d(\eta\wedge\Lambda)|=0$. Thus, the limit on the left hand side of \eqref{art08-sec4-eqA4.40} is independent of the $T_{\epsilon}$ (as should be the case).

Now\pageoriginale $z=\sum\limits^{q}_{\alpha=1}\lambda_{\alpha}f_{\alpha}(z)+\sum\limits^{m}_{\mu=q+1}\lambda_{\mu}f_{\mu}(z)$, and we set $z_{\eta}=\sum\limits^{q}_{\alpha=1}\lambda_{\alpha}f_{\alpha}$; then the left hand side of \eqref{art08-sec4-eqA4.40} is $-\lim\limits_{\epsilon\to 0|z_{\eta}|=\epsilon}\eta\wedge \Lambda$. But $z_{\eta}=|z_{\eta}|f_{1}$, and by iterating the integral, we have
\begin{align*}
& -\lim\limits_{\epsilon\to 0}\int\limits_{T_{\epsilon}}\eta\wedge \Lambda=\int\limits_{\bfC^{n-q}}\left\{-\lim\limits_{\epsilon\to 0}\int\limits_{\substack{|z_{\eta}|=\epsilon\\ z-z_{\eta}=\text{~constant}}}\eta\wedge \Lambda\right\}=\\[4pt]
& \int\limits_{\bfC^{n-q}}\eta(0,z)\left\{-\lim\limits_{\epsilon\to 0}\int\limits_{\substack{|z_{\eta}|=\epsilon\\ z-z_{\eta}=\text{~constant}}}\Lambda\right\}\omega^{q+1}\ldots\omega^{n}\overline{\omega}^{q+1}\ldots \overline{\omega}^{n}=\int\limits_{\bfC^{n-q}}\eta.
\end{align*}

To prove \eqref{art08-sec4-eqA4.35}, we refer to the proof of \eqref{art08-sec4-eq4.2} (c.f. the proof of \eqref{art08-sec3-eq3.3} in \cite{art08-key9}) and see that we may assume that $\bfC$ is a (real) manifold with boundary $\partial C=Z$. In this case the argument is substantially the same as that just given.

(e)~ \textsc{Concluding Remarks on Residues, Currents, and the Gysin Homomorphism.} Let $V$ be an algebraic manifold and $W\subset V$ an irreducible subvariety which is the $q^{\text{th}}$ Chern class of an ample bundle $\bfE\to V$. Given an Hermitian metric in $\bfE$, the differential form $P_{q}(\Theta)$ ($\Theta$ = curvature form in $\bfE$) represents the Poincar\'e dual $\mathscr{D}(W)\in H^{2n-2q}(V,\bfZ)$ of $W\in H_{2n-2q}(V,\bfZ)$. The differential form $\psi$ (having properties \eqref{art08-sec4-eq4.8}-\eqref{art08-sec4-eq4.10} which we constructed is a {\em residue operator for} $W$; that is to say:
\begin{equation*}
\begin{array}{l}
\psi\text{~ is a~ }C^{\infty}(q,q-1)\text{~ form on~ } V-W\\
\text{which has a pole of order~ }2q-1\text{~ along~ } W;
\end{array}\tag{A4.41}\label{art08-sec4-eqA4.41}
\end{equation*}
\begin{equation*}
\partial \psi=0\text{~ and~ } d\psi=\overline{\partial}\psi=P_{q}(\Theta) \text{~ is the Poincar\'e dual of~ } W;\tag{A4.42}\label{art08-sec4-eqA4.42}
\end{equation*}
and for any $2n-k$ chain $\Gamma$ meeting $W$ transversely and any smooth $2n-2q-k$ form $\eta$,
\begin{equation*}
\lim\limits_{\epsilon\to 0}-\int\limits_{\Gamma\cdot \partial T_{\epsilon}}\psi\wedge \eta=\int\limits_{\Gamma\cdot W}\eta\quad(\text{\em Residue formula}).\tag{A4.43}\label{art08-sec4-eqA4.43}
\end{equation*}

This formalism is perhaps best understood in the language of {\em currents} \cite{art08-key14}. Let then $C^{m}(V)$ be the currents of degree $m$ on $V$; by definition, $\theta\in C^{m}(V)$ is a linear form on $A^{2n-m}(V)$ (the $C^{\infty}$ forms of\pageoriginale degree $2n-m$) which is continuous in the distribution topology (c.f. Serre \cite{art08-key21}). The derivative $d\theta\in C^{m+1}(V)$ is defined by
\begin{equation*}
\langle d\theta,\lambda\rangle =\langle \theta,d\lambda\rangle \text{~~ for all~~ } \lambda\in A^{2n-m-1}(V).\tag{A4.44}\label{art08-sec4-eqA4.44}
\end{equation*}
Of course we may define $\partial \theta$, $\overline{\partial}\theta$, and speak of currents of type $(r,s)$, etc. If $Z^{m}(V)\subset C^{m}(V)$ are the closed currents $(d\theta=0)$, then we may set $\mathscr{H}^{m}(V)=Z^{m}(V)/dC^{m-1}(V)$ ({\em cohomology computed from currents}), and it is known that (c.f. \cite{art08-key14})
\begin{equation*}
H^{m}(V)\cong \mathscr{H}^{m}(V).\tag{A4.45}\label{art08-sec4-eqA4.45}
\end{equation*}

Now $P_{q}(\Theta)$ gives a current in $C^{q,q}(V)$ by $\langle P_{q}(\Theta),\lambda\rangle=\int\limits_{V}P_{q}(\Theta)\wedge \lambda (\lambda\in A^{2n-2q}(V))$. By Stokes' theorem, $dP_{q}(\Theta)$ in the sense of currents is the same as the usual exterior derivative. Thus $dP_{q}(\Theta)=0$ and $P_{q}(\Theta)\in H^{q,q}(V)$.

Also, $W$ gives a current in $C^{q,q}(V)$ by $\langle W,\lambda\rangle=\int\limits_{W}\lambda(\lambda\in A^{2n-2q}(V))$. By Stokes' theorem again, $dW=0$ (if $W$ were a manifold with boundary, then $dW$ would be just $\partial W$).

Now $\psi$ gives a current in $C^{q,q-1}(V)$ by $\langle \psi,\lambda\rangle=\int\limits_{W}\psi\wedge\lambda$ (this is because $\psi$ has a pole of order $2q-1$). To compute $d\psi\in C^{q,q}(V)$, we have, for any $\lambda\in A^{2n-2q}(V)$,
\begin{align*}
& \int\limits_{V}\psi\wedge d\lambda=\lim\limits_{\epsilon\to 0}\int\limits_{V-T_{\epsilon}}\psi\wedge d\lambda=\lim\limits_{\epsilon\to 0}\left\{\int\limits_{V-T_{\epsilon}}-d(\psi\wedge\lambda)+d\psi\wedge\lambda\right\}\\[4pt]
&=\lim\limits_{\epsilon\to 0}\int\limits_{V-T_{\epsilon}}d(\psi\wedge\lambda)+\lim\limits_{\epsilon\to 0}\int\limits_{V-T_{\epsilon}}P_{q}(\Theta)\wedge\lambda=-\int\limits_{W}\lambda+\int\limits_{V}P_{q}(\Theta)\wedge\lambda,
\end{align*}
which says that, {\em in the sense of currents},
\begin{equation*}
d\psi=P_{q}(\Theta)-W.\tag{A4.45}\label{art08-sec4-add-eqA4.45}
\end{equation*}

Thus, among other things, the residue operator $\psi$ expresses the fact that, in the cohomology group $\mathscr{H}^{q,q}(V)$, $P_{q}(\Theta)=W$ (which proves also that $P_{q}(\Theta)=\mathscr{D}(W)$). The point in the above calculation is that $d\psi$ in the sense of currents is {\em not} just the exterior derivative of $\psi$; the singularities force us to be careful in Stokes' theorem, so that we get \eqref{art08-sec4-eqA4.45}.

Suppose\pageoriginale that $W$ is non-singular and consider the Gysin homomorphism $H^{k}(W)\to H^{k+2q}(V)$. Given a smooth form $\phi\in A^{k}(W)$ which is closed, we choose $\widehat{\phi}\in A^{k}(V)$ with $\widehat{\phi}|W=\phi$. Then the differential form $i_{*}(\phi)=d(\psi\wedge \widehat{\phi})=P_{q}(\Theta)\wedge \widehat{\phi}-\psi\wedge d\widehat{\phi}$ will have only a pole of order $2q-2$ along $W$ (since $d\widehat{\phi}|W=0$, the term of highest order in $\psi$ involves only {\em normal} differentials along $W$, as does $d\widehat{\phi}$), and so $i_{*}(\phi)$ is a current in $C^{k+2q}(V)$. We claim that, in the sense of currents, $di_{*}(\phi)=0$.

\begin{proof}
$\int\limits_{V}d(\psi\wedge \widehat{\phi})\wedge d\lambda=\lim\limits_{\epsilon\to 0}\int\limits_{V-T_{\epsilon}}d(\psi \wedge \widehat{\phi})\wedge d\lambda=-\lim\limits_{\epsilon\to 0}\int\limits_{\partial T_{\epsilon}}d(\psi\wedge\widehat{\phi})\wedge \lambda=\lim\limits_{\epsilon\to 0}\int\limits_{\partial T_{\epsilon}}\psi\wedge d\widehat{\phi}\lambda$

(since $d\psi\wedge \widehat{\phi}=P_{q}(\Theta)\wedge \widehat{\phi}$ is smooth). But $\psi\wedge d\widehat{\phi}$ has a pole of order $2q-2$ along $W$ so that $\lim\limits_{\epsilon\to 0}\int\limits_{\partial T_{\epsilon}}\psi\wedge d\widehat{\phi}\wedge\lambda=0$).

Thus $i_{*}(\phi)$ is a {\em closed current} and so defines a class in $\mathscr{H}^{k+2q}(V)\cong H^{k+2q}(V)$; because of the residue formula \eqref{art08-sec4-eqA4.43}, $i_{*}(\phi)$ is the Gysin homomorphism on $\phi\in H^{k}(V)$.

Of course, if we are interested only in the de Rham groups $H^{k}(W)$, we may choose $\widehat{\phi}$ so that $d\widehat{\phi}=0$ in $T_{\epsilon}$ for small $\epsilon$ (since $W$ is a $C^{\infty}$ {\em retraction} of $T_{\epsilon}$). Then $d(\psi\wedge \widehat{\phi})$ is smooth and currents are unnecessary. However, if we want to keep track of the complex structure, we must use currents because $W$ is generally {\em not a holomorphic retraction} of $T_{\epsilon}$. Thus, if $\phi\in F^{k}_{l}(W)$ (so that $\phi=\phi_{k,0}+\cdots+\phi_{l,k-l}$), we may choose $\widehat{\phi}\in F^{k}_{l}(V)$ with $\widehat{\phi}|W=\phi$, but we {\em cannot} assume that $d\widehat{\phi}=0$ in $T_{\in}$. The point then is that, if we let $\mathscr{F}^{k}_{l}(W)$ and $\mathscr{F}^{k+2q}_{l+q}(V)$ be the cohomology groups computed from the {\em Hodge filtration} using currents, then we have
\begin{equation*}
\mathscr{F}^{k+2q}_{l+q}(V)\cong F^{k+2q}_{l+q}(V);\tag{A4.46}\label{art08-sec4-eqA4.46}
\end{equation*}
and the Gysin homomorphism $i_{*}:H^{k}(W)\to H^{k+2q}(V)$ satisfies $i_{*}:F^{k}_{l}(W)\to F^{k+2q}_{l+q}(V)$ and is given, as explained above, by
\begin{equation*}
i_{*}(\widehat{\phi})=d(\psi\wedge \phi)\in \mathscr{F}^{k+2q}_{l+q}(V).\tag{A4.47}\label{art08-sec4-eqA4.47}
\end{equation*}

\eject

In\pageoriginale other words, by using residues and currents, we have proved that the Gysin homomorphism is compatible with the complex structure and can be computed using the residue form.
\end{proof}

\section{Generalizations of the Theorems of Abel and Lefschetz.}\label{art08-sec5}

Let $V=V_{n}$ be an algebraic manifold and $\bfZ=\bfZ_{n-q}$ an {\em effective algebraic cycle} of codimension $q$; thus $\bfZ=\sum\limits^{l}_{\alpha=1}n_{\alpha}\bfZ_{\alpha}$ where $\bfZ_{\alpha}$ is irreducible and $n_{\alpha}>0$. We denote by $\Phi=\Phi (\bfZ)$ an irreducible component containing $\bfZ$ of the {\em Chow variety} \cite{art08-key13} of effective cycles $Z$ on $V$ which are algebraically equivalent to $Z$. If $Z\in \Phi$, then $Z-\bfZ$ is homologous to zero and so, as in \S\ref{art08-sec3}, we may define $\phi_{q}:\Phi\to T_{q}(V)$. Letting $\Alb(\Phi)$ be the {\em Albanese variety} of $\Phi$, we in fact have a diagram of mappings~:
\begin{equation*}
\vcenter{
\xymatrix@C=1.8cm@R=.5cm{
 & \Alb(\Phi)\ar[dd]^-{\alpha_{\Phi}}\\
\Phi\ar[ur]^{\delta_{\Phi}}\ar[dr]_{\phi_{q}} &\\
 & T_{q}(\Phi,V).
}}\tag{5.1}\label{art08-sec5-eq5.1}
\end{equation*}
Here $T_{q}(\Phi,V)$ is the torus generated by $\phi_{q}(\Phi)$ and $\delta_{\Phi}$ is the usual mapping of an irreducible variety to its Albanese. Thus, if $\psi^{1},\ldots,\phi^{m}$ are a basis for the holomorphic $1$-forms on $\Phi$, then $\delta_{\Phi}(Z)=\left[\begin{smallmatrix}\vdots\\\int^{Z}_{Z}\phi^{\rho}\\\vdots\end{smallmatrix}\right]$, where $\int^{Z}_{Z}\phi^{\rho}$ means that we take a path on $\Phi$ from $\bfZ$ to $Z$ and integrate $\psi^{\rho}$. We may assume that $\psi^{1}=\phi^{*}_{q}(\omega^{1}),\ldots,\psi^{k}=\phi^{*}_{q}(\omega^{k})$ where $\omega^{1},\ldots,\omega^{k}$ give a basis for the holomorphic $1$-forms on $T_{q}(\Phi,V)$ $(\omega^{\alpha}\in H^{n-q+1,n-q}(V))$, and then $\alpha_{\Phi}\delta_{\Phi}(Z)=\alpha_{\Phi}\left[\begin{smallmatrix}\int^{Z}_{\bfZ}\psi^{1}\\\vdots\\\int^{Z}_{\bfZ}\psi^{m}\end{smallmatrix}\right]=\left[\begin{smallmatrix}\int^{Z}_{\bfZ}\omega^{1}\\\vdots\\\int^{Z}_{\bfZ}\omega^{k}\end{smallmatrix}\right]$, where $\int^{Z}_{\bfZ}\omega^{\alpha}$ means $\int_{\Gamma}\omega^{\alpha}$ if $\Gamma$ is a $2n-2q+1$ chain on $V$ with $\partial \Gamma=Z-\bfZ$.

\eject

Let now $\bfW=\bfW_{q-1}$ be a sufficiently general irreducible subvariety of dimension $q-1$ (codimension $n-q+1$) and $\Sigma=\Sigma(\bfW)$ an irreducible component of the Chow variety of $\bfW$. Each $Z\in \Phi$ defines a divisor $D(Z)$ on $\Sigma$ by letting $D(Z)=\{\text{all~}W\in \Sigma$ such that $W$\pageoriginale meets $Z\}$. Thus, if $Z=\sum\limits^{l}_{\alpha=1}n_{\alpha}Z_{\alpha}$, $D(Z)=\sum\limits^{l}_{\alpha=1}n_{\alpha}D(Z_{\alpha})$. Letting ``$\equiv$'' denote {\em linear equivalence of divisors}, we will prove as a generalization of {\em Abel's theorem} that:
\begin{equation*}
\begin{array}{l}
D(Z)\text{~ is algebraically equivalent to~ } D(\bfZ),\\[2pt]
\text{even if we only assume that $Z$ is homologous to $\bfZ$};
\end{array}\tag{5.2}\label{art08-sec5-eq5.2}
\end{equation*}
and
\begin{equation*}
D(Z)\equiv D(\bfZ)\text{~~ if~~ }\phi_{q}(Z)=0\text{~ in~ }T_{q}(\Phi,V).\tag{5.3}\label{art08-sec5-eq5.3}
\end{equation*}

\medskip
\noindent
{\bf Example \thnum{1}.\label{art08-sec5-exam1}}
Suppose that $\bfZ$ is a divisor on $V$; then $\Phi$ is a projective fibre space over (part of) $\Pic(V)$ (= Picard variety of $V$) and the fibre through $Z\in \Phi$ is the {\em complete linear system} $|Z|$. Now $\bfW$ is a point on $V$ and $\Sigma=V$, and $D(Z)=Z$ as divisor on $\Sigma$. In this case, \eqref{art08-sec5-eq5.3} is just the classical Abel's theorem for divisors \cite{art08-key17}; \eqref{art08-sec5-eq5.2} is the statement (well known, of course) that homology implies algebraic equivalence. The converse to \eqref{art08-sec5-eq5.3}, which reads :
\begin{equation*}
\phi_{q}(Z)=0\text{~~ if~~ } D(Z)\equiv D(\bfZ),\tag{5.4}\label{art08-sec5-eq5.4}
\end{equation*}
is the trivial part of Abel's theorem in this case.

\begin{remark*}
We may give \eqref{art08-sec5-eq5.3} as a functorial statement as follows. The mapping $\Phi\to \Div(\Sigma)$ (given by $Z\to D(Z)$) induces $\Phi\to \Pic(\Sigma)$. From this we get $\Alb(\Phi)\to \Alb(\Pic(\Sigma))=\Pic(\Sigma)$, which combines with \eqref{art08-sec5-eq5.1} to give
\begin{equation*}
\vcenter{\xymatrix{
\Alb(\Phi)\ar[d]^-{\alpha_{\Phi}}\ar[r]^-{\xi_{\Phi}} & \Pic(\Sigma)\\
T_{q}(\Phi,V)\ar@{--}[ur]_{\zeta_{\Phi}} &
}}\tag{5.5}\label{art08-sec5-eq5.5}
\end{equation*}
Then \eqref{art08-sec5-eq5.3} is equivalent to saying that $\xi_{\Phi}$ factors in \eqref{art08-sec5-eq5.5}.
\end{remark*}

\begin{proof}
For $z_{0}\in \Alb(\Phi)$, there exists a zero-cycle $Z_{1}+\cdots+Z_{N}$ on $\Phi$ such that $z_{0}=\delta_{\Phi}(Z_{1}+\cdots+Z_{N})$. Let $Z=Z_{1}+\cdots+Z_{N}$ be the corresponding subvariety of $V$. Then $\alpha_{\Phi}(Z)=\phi_{q}(Z-N\bfZ)$ and, assuming \eqref{art08-sec5-eq5.3}, if $\phi_{q}(Z-N\bfZ)=0$, then $\xi_{\Phi}(Z)=0$ in $\Pic(\Sigma)$. Thus, if \eqref{art08-sec5-eq5.3} holds, $\ker \alpha_{\Phi}\supset \ker \xi_{\Phi}$ and so $\xi_{\Phi}$ factors in \eqref{art08-sec5-eq5.5}.
\end{proof}

\medskip
\noindent
{\bf Example \thnum{2}.\label{art08-sec5-exam2}}
Let\pageoriginale $\bfZ=$ point on $V$ so that $\Phi=V$, $\Alb(\Pi)=\Alb(V)$. Choose $\bfW$ to be a very ample divisor on $V$; then $\Sigma$ is a {\em projective fibre bundle} over $\Pic(V)$ with $|W|$ as fibre through $W$ (c.f. \cite{art08-key18}). Now $D(Z)$ consists of all divisors $W\in \Sigma$ which pass through $Z$. In this case, \eqref{art08-sec5-eq5.3} reads:
\begin{equation*}
\begin{array}{l}
\text{Albanese equivalence of points on $V$ implies linear}\\[2pt]
\text{equivalence of divisors on~ } \Sigma.
\end{array}\tag{5.6}\label{art08-sec5-eq5.6}
\end{equation*}

\begin{remark*}
There is a reciprocity between $\Phi$ and $\Sigma$; each $W\in \Sigma$ defines a divisor $D(W)$ on $\Phi$ so that we have $\Alb(\Sigma)\xrightarrow{\xi_{\Sigma}}\Pic(\Phi)$. Then \eqref{art08-sec5-eq5.5} dualizes to give~:
\begin{equation*}
\vcenter{\xymatrix{
\Alb(\Sigma)\ar[r]^-{\xi_{\Sigma}}\ar[d]^{\alpha_{\Sigma}} & \Pic(\Phi)\\
T_{n-q+1}(\Sigma,V)\ar@{--}[ur]_{\zeta_{\Sigma}} &
}}\tag{5.7}\label{art08-sec5-eq5.7}
\end{equation*}

For example, suppose that $\dim V=2m+1$ and $q=m+1$. We may take $\bfW=\bfZ$, $\Sigma=\Phi$, and then \eqref{art08-sec5-eq5.5} and \eqref{art08-sec5-eq5.7} coincide to give~:
\begin{equation*}
\vcenter{\xymatrix{
\Alb(\Phi)\ar[r]^-{\xi_{\Phi}}\ar[d]^-{\alpha_{\Phi}} & \Pic(\Phi)\\
T_{q}(\Phi,V)\ar@{--}[ur]_-{\zeta_{\Phi}} & 
}}\tag{5.8}\label{art08-sec5-eq5.8}
\end{equation*}

Given $\bfZ$, $\Phi$ as above, there is a mapping
\begin{equation*}
H_{r}(\Phi,\bfZ)\xrightarrow{\tau}H_{2n-2q+r}(V,\bfZ)\tag{5.9}\label{art08-sec5-eq5.9}
\end{equation*}
as follows. Given an $r$-cycle $\Gamma$ on $\Phi$, $\tau(\Gamma)$ is the cycle traced out by the varieties $Z_{\gamma}$ for $\gamma\in \Gamma$. Suppose that $\Phi$ is nonsingular. Then the adjoint $\tau^{*}:H^{2n-2q+r}(V)\to H^{r}(\Phi)$ is given as follows. On $\Phi\times V$, there is a cycle $T$ with $\pr_{V}T\cdot\{Z\times V\}=Z(Z\in \Phi)$. We then have\pageoriginale $\xymatrix{T\ar[d]^{\pi}\ar[r]^-{\widetilde{\omega}} & V\\ \Phi &}$ and~:
\begin{equation*}
\tau^{*}=\pi_{*}\widetilde{\omega}^{*}:H^{2n-2q+r}(V)\to H^{r}(\Phi)\tag{5.10}\label{art08-sec5-eq5.10}
\end{equation*}
(here $\pi_{*}$ is {\em integration over the fibre}). Since $\Phi$ is nonsingular $D(\bfW)$ (= divisor on $\Phi$) gives a class in $H^{1,1}(\Phi)$. In fact, we will show, as a generalization of the {\em Lefschetz theorem} \cite{art08-key19}, that
\begin{equation*}
\tau^{*}:H^{n-q+s,n-q+t}(V)\to H^{s,t}(\Phi);\tag{5.11}\label{art08-sec5-eq5.11}
\end{equation*}
and, if $\omega\in H^{n-q+1,n-q+1}(V)$ is the dual of $\bfW\in H_{q-1,q-1}(V)\cap H_{2q-2}\break (V,\bfZ)$, then~:
\begin{equation*}
\text{The dual of~ } D(\bfW)\text{~ is ~} \tau^{*}\omega\in H^{1,1}(\Phi).\tag{5.12}\label{art08-sec5-eq5.12}
\end{equation*}

In other words, an {\em integral cohomology class} $\omega$ of type $(n-q+1, n-q+1)$ on $V$ defines a {\em divisor} on $\Phi$.
\end{remark*}

\begin{remark*}
In \eqref{art08-sec5-eq5.11}, we have
\begin{equation*}
\tau^{*}:H^{n-q+1,n-q}(V)\to H^{1,0}(\Phi);\tag{5.13}\label{art08-sec5-eq5.13}
\end{equation*}
this $\tau^{*}$ is just $\phi^{*}_{q}:H^{1,0}(T_{q}(V))\to H^{1,0}(\Alb(\Phi))$ where $\phi_{q}$ is given by \eqref{art08-sec5-eq5.1}.
\end{remark*}

\medskip
\noindent
{\bf Remark \thnum{5.14}.\label{art08-sec5-rem5.14}}
The gist of \eqref{art08-sec5-eq5.2}, \eqref{art08-sec5-eq5.3} and \eqref{art08-sec5-eq5.11}, \eqref{art08-sec5-eq5.12} may be summarized by saying: The cohomology of type $(p,p)$ gives algebraic cycles, and the equivalence relation defined by the tori $T_{q}(V)$ implies rational equivalence, both on suitable Chow varieties attached to the original algebraic manifold $V$.
\smallskip

The problem of dropping back down to $V$ still remains of course.

\smallskip
(a)~ \textsc{A generalization of interals of the 3rd kind to higher codimension.} We want to prove \eqref{art08-sec5-eq5.2} and \eqref{art08-sec5-eq5.3} above. Since changing $\bfZ$ or $Z$ by rational equivalence will change $D(\bfZ)$ or $D(Z)$ by linear equivalence and will not alter $\phi_{q}(Z)$, and since we may add to $\bfZ$ and $Z$ a common cycle, we will assume that $\bfZ=\sum\limits^{l}_{\alpha=1}n_{\alpha}\bfZ_{\alpha}$, $Z=\sum\limits^{k}_{\rho=1}m_{\rho}Z_{\rho}$ where the $\bfZ_{\alpha}$, $Z_{\rho}$ are {\em Chern classes of ample\pageoriginale bundles} (c.f. \S\ref{art08-sec4} above) and that all intersections are transversal. To simplify notation then, we write $Y=Z-\bfZ$ and $Y=\sum\limits^{l}_{j=1}n_{j}Y_{j}$ where the $Y_{j}$ are nonsingular Chern classes which meet transversely. We also set $|Y|=\bigcup\limits^{l}_{j=1}Y_{j}$, $V-Y=V-|Y|$.

A {\em residue operator for} $Y$ (c.f. Appendix to \S\ref{art08-sec4}, section (e) above) is given by a $C^{\infty}$ differential form $\psi$ on $V-Y$ such that~:
\begin{itemize}
\item[(i)] $\psi$ is of degree $2q-1$ and $\psi=\psi_{2q-1,0}+\cdots+\psi_{q,q-1}$ ($\psi_{s,t}$ is the part of $\psi$ of type $(s,t)$);

\item[(ii)] $\partial \psi=0$ and $\overline{\partial}\psi=\Phi$ where $\Phi$ is a $C^{\infty}(q,q)$ form on $V$ giving the Poincar\'e dual of $Y\in H_{2n-2q}(V,\bfZ)$;

\item[(iii)] $\psi-\psi_{q,q-1}$ is $C^{\infty}$ on $V$ and $\psi$ has a pole of order $2q-1$ along $Y$; and

\item[(iv)] for any $k+2q$ chain $\Gamma$ on $V$ which meets $Y$ transversely and smooth $k$-form $\omega$ on $V$
\begin{equation*}
\int\limits_{\Gamma\cdot Y}\omega=-\lim\limits_{\epsilon\to 0}\int\limits_{\Gamma\cdot \partial T_{\epsilon}}\psi \wedge\omega (\textit{\em Residue formula})\tag{5.15}\label{art08-sec5-eq5.15}
\end{equation*}
where $T_{\epsilon}$ is an $\epsilon$-tube around $Y$.
\end{itemize}

From \eqref{art08-sec4-eqA4.1}-\eqref{art08-sec4-eqA4.3} and \eqref{art08-sec4-eqA4.35}, we see that a residue operator $\psi_{j}$ for each $Y_{j}$ exists. Then $\psi=\sum\limits^{l}_{j=1}n_{j}\psi_{j}$ is a residue operator for $Y$ (the formula \eqref{art08-sec5-eq5.15} has to be interpreted suitably).
\begin{equation*}
\text{If }Y=0\text{ in }H_{2n-2q}(V,\bfZ),\text{ then we may assume that } \overline{\partial}\psi=0.\tag{5.16}\label{art08-sec5-eq5.16}
\end{equation*}

\begin{proof}
$\overline{\partial}\psi=\Phi$ is a $C^{\infty}$ form and $\Phi=0$ in $H^{q,q}_{\overline{\partial}}(V)$. Then $\Phi=\overline{\partial}\eta$ where $\eta=\overline{\partial}^{*}G_{\overline{\partial}}\Phi$ and $\partial \eta=0$ since $\partial \overline{\partial}^{*}=-\overline{\partial}^{*}\partial$, $\partial G_{\overline{\partial}}=G_{\overline{\partial}}\partial$, $\partial \Phi=0$. Since $\eta$ is of type $(q,q-1)$, we may take $\psi-\eta$ as our residue operator.
\end{proof}

\medskip
\noindent
{\bf Remark \thnum{5.17}.\label{art08-sec5-rem5.17}}
If $Y$ is a divisor which is zero in $H_{2n-2}(V,\bfZ)$, then $\psi$ is a holomorphic differential on $V-Y$ having $Y$ as its {\em logarithmic residue locus} (c.f. \cite{art08-key18}).

\medskip
\noindent
{\bf Remark \thnum{5.18}.\label{art08-sec5-rem5.18}}
Let\pageoriginale $Y$ be homologous to zero and $\psi$ be a residue operator for $Y$ with $d\psi=0$ (c.f. \eqref{art08-sec5-eq5.16}). Then $\psi$ gives a class in $H^{2q-1}(V-Y)$, and $\psi$ is determined up to $H^{2q-1,0}(V)+\cdots+H^{q,q-1}(V)$. We claim that

$H^{2q-1}(V-Y)$ is generated by $H^{2q-1}(V)$ and the $\psi_{j}$.

\begin{proof}
Let $\delta_{j}$ be a normal sphere to $Y_{j}$ at some simple point not on any of the other $Y_{k}$'s. We map $\bfZ^{(l)}=\underbrace{\bfZ\oplus\cdots\oplus}_{l}\bfZ$ into $H_{2q-1}(V-Y)$ by $(\alpha_{1},\ldots,\alpha_{l})\to \sum\limits^{l}_{j=1}\alpha_{j}\delta_{j}$. Since $\int_{\delta_{j}}\psi_{j}=+1$, we must show that the sequence
\begin{equation*}
\bfZ^{(l)}\to H_{2q-1}(V-Y)\xrightarrow{i_{*}}H_{2q-1}(V)\to 0\tag{5.19}\label{art08-sec5-eq5.19}
\end{equation*}
is exact. By dimension, $H_{2q-1}(V-Y)$ maps onto $H_{2q-1}(V)$. If $\sigma\in H_{2q-1}(V-Y)$ is an integral cycle which bounds in $V$, then $\sigma=\delta \gamma$ for some $2q$-chain $\gamma$ where $\gamma$ meets $Y$ transversely in nonsingular points $p_{\rho}\in Y$. If $p_{\rho}\in Y_{j(\rho)}$, then clearly $\sigma\sim \sum\limits_{\rho}\delta_{j(\rho)}$ so that $Z^{(l)}$ generates the kernel of $i_{*}$ in \eqref{art08-sec5-eq5.19}.

Consider now our subvariety $\bfW=\bfW_{q-1}$ with Chow variety $\Sigma$. We may assume that $\bfW$ lies in $V-Y$ and, for $W\in \Sigma$, $W$ lying in $V-Y$, we may write $W-\bfW=\partial\Gamma$ where $\Gamma$ is a $2q-1$ chain not meeting $Y$. Clearly $\Gamma$ is determined up to $H_{2q-1}(V-Y)$. We will show :
\begin{equation*}
\begin{array}{l}
\text{There exists an integral of the $3^{\text{rd}}$ kind $\theta$ on $\Sigma$ whose logarithmic}\\[2pt]
\text{residue locus is $D(Y)$, provided that $Y=0$ in $H_{2n-2q}(V,\bfZ)$.}
\end{array}\tag{5.20}\label{art08-sec5-eq5.20}
\end{equation*}
\end{proof}

\begin{proof}
Let $\psi$ be a residue operator for $Y$ with $d\psi=0$. Define a 1-form $\theta$ on $\Sigma-D(Y)$ by~:
\begin{equation*}
\theta=d\left\{\int\limits^{W}_{\bfW}\psi\right\}=d\left\{\int\limits_{\Gamma}\psi\right\}.\tag{5.21}\label{art08-sec5-eq5.21}
\end{equation*}
This makes sense since $d\psi=0$. We claim that
$$
\theta\text{~\em is holomorphic on~ } \Sigma-D(Y).
$$
\end{proof}

\begin{proof}
Let $\Sigma^{*}=\Sigma-D(Y)$ and $T^{*}\subset \Sigma^{*}\times V$ the graph of the correspondence $(W,z)(z\in W)$ (i.e. $W\in \Sigma^{*}$ is a subvariety of $V$ and $z\in V$ lies on $W$). Then we have
\[
\xymatrix{
T^{*}\ar[r]^-{\widetilde{\omega}}\ar[d]^-{\pi} & V.\\
\Sigma & 
}
\]\pageoriginale
Now $\widetilde{\omega}^{*}:A^{r,s}(V)\to A^{r,s}(T^{*})(A^{r,s}(*)=C^{\infty}$ forms of type $(r,s)$ on$^{*}$); since $\widetilde{\omega}$ is holomorphic, $\widetilde{\omega}^{*}\overline{\partial}=\overline{\partial}\widetilde{\omega}^{*}$. On the other hand, the {\em integration over the fibre} $\pi_{*}:A^{r+q-1,s+q-1}(T^{*})\to A^{r,s}(\Sigma^{*})$ is defined and is determined by the equation :
\begin{equation*}
\int\limits_{\Sigma^{*}}\pi_{*}\phi\wedge \eta=\int\limits_{T^{*}}\phi\wedge \pi^{*}\eta,\tag{5.22}\label{art08-sec5-eq5.22}
\end{equation*}
where $\eta$ is a compactly supported form on $T^{*}$. Since $\int\limits_{\Sigma^{*}}\overline{\partial}\pi_{*}\phi\wedge \eta=(-1)^{r+s}\int\limits_{T^{*}}\phi\wedge \pi^{*}\overline{\partial}\eta=\int\limits_{T^{*}}\partial \phi\wedge \pi^{*}\eta=\int\limits_{\Sigma^{*}}\pi_{*}(\overline{\partial}\phi)\wedge \eta$ for all $\eta$, $\overline{\partial}\pi_{*}=\pi_{*}\overline{\partial}$. Let $\tau^{*}:A^{r+q-1,s+q-1}(V)\to A^{r,s}(\Sigma^{*})$ be the composite $\pi_{*}\widetilde{\omega}^{*}$. Then $\overline{\partial}\tau^{*}=\tau^{*}\overline{\partial}$ (this proves \eqref{art08-sec5-eq5.11}).

Now let $\psi\in A^{q,q-1}(V-Y)$ be a residue operator for $Y$. Then by the definition \eqref{art08-sec5-eq5.21}, $\theta=\tau^{*}\psi\in A^{1,0}(\Sigma^{*})$ and $d\theta=\tau^{*}d\psi=0$. This proves that $\theta$ is holomorphic on $\Sigma^{*}$.

\vskip .1cm

Now $Y=\sum\limits_{j=1}n_{j}Y_{j}$ where the $Y_{j}$ are subvarieties of codimension $q$ on $V$. We have that $D(Y)=\sum\limits^{l}_{j=1}n_{j}D(Y_{j})$ and $\psi=\sum\limits_{j=1}n_{j}\psi_{j}$. We will prove that $\psi$ has a pole of order one on $Y_{j}$ with logarithmic residue $n_{j}$ there.
\end{proof}

\vskip .1cm

\begin{proof}
We give the argument when $Y$ is irreducible and $q=n$. From this it will be clear how the general case goes.

\vskip .1cm

Let $\Delta$ be the unit disc in the complex $t$-plane and $\{W_{t}\}_{t\in \Delta}$ a holomorphic curve on $\Sigma$ meeting $D(Y)$ simply at the point $t=0$. Then $W_{0}$ meets $Y$ simply at a point $z_{0}\in V$. We may choose local coordinates $z^{1},\ldots,z^{n}$ on $V$ such that $z_{0}=Y$ is the origin. Now $\psi=\dfrac{1}{|z|^{2n-1}}\left\{\sum\limits^{n}_{\alpha=1}\psi_{\alpha}dz^{1}\ldots dz^{n}d\overline{z}^{1}\ldots d\widehat{\overline{z}}^{\alpha}\ldots d\overline{z}^{n}\right\}$ where $|z|^{2}=\sum\limits^{n}_{\alpha=1}|z^{\alpha}|^{2}$ and\pageoriginale $\psi_{\alpha}$ is smooth. We may assume that $W_{t}$ is given by $z^{1}=t$, and, to prove that $\theta$ has a pole of order one at $t=0$, it will suffice to show that $\iint\limits_{\Delta}|\theta\wedge d\overline{t}|$ is finite. It is clear, however, that $\iint\limits_{\Delta}|\theta\wedge d\overline{t}|$ will be finite if $\int\limits_{|z^{\alpha}|<1}|\psi\wedge d\overline{z}^{1}|$ is finite. But
$$
|\psi \wedge d\overline{z}^{1}|\leq c\left\{\dfrac{|dz^{1}\ldots dz^{n} \ d\overline{z}^{1}\ldots d\overline{z}^{n}|}{|z|^{2n-1}}\right\}\quad(c=\text{constant}),
$$
so that $\int\limits_{|z^{\alpha}|<1}|\psi\wedge d\overline{z}^{1}|$ is finite.

\vskip .1cm

We now want to show that $\int\limits_{|t|=1}\theta=+1$ (i.e. $\theta$ has logarithmic residue $+1$ on $D(Y)$). Let $\delta=\bigcup\limits_{|t|=1}W_{t}$. Then $\int\limits_{|t|=1}\theta=\int\limits_{\delta}\psi$. If $T_{\epsilon}=\{z:|z|<\epsilon\}$, then setting $\Gamma=\left\{\bigcup\limits_{|t|\leq 1}W_{t}\right\}-T_{\epsilon},\partial \Gamma=\delta+\partial T_{\epsilon}$. Thus $\int\limits_{\delta}\psi=-\int\limits_{\partial T_{\epsilon}}\psi=+1$ as required.
\end{proof}

\begin{remark*}
Let $Y\subset V$ be as above but without assuming that $Y=0$ in $H_{2n-2q}(V,\bfZ)$. Let $\psi$ be a residue operator for $Y$ and $\theta=\tau^{*}(\psi)=\widetilde{\omega}_{*}\pi^{*}\psi$. The above argument generalizes to prove : 
\begin{equation*}
\theta=\tau^{*}(\psi)\text{~ \em is a residue operator for~ } D(Y).\tag{5.23}\label{art08-sec5-eq5.23}
\end{equation*}

We have now proved \eqref{art08-sec5-eq5.20}, and with it have proved \eqref{art08-sec5-eq5.2}, since $D(Y)$ will be algebraically equivalent to zero on $\Sigma$ because of the existence of an integral of the $3^{\text{rd}}$ kind associated to $D(Y)$.
\end{remark*}

\noindent
{\bf Proof of \eqref{art08-sec5-eq5.12}.}~ Let $Y\subset V$ be as above and interchange the roles of $Y$ and $\bfW$ in the statement of \eqref{art08-sec5-eq5.12}. Let $\omega\in H^{q,q}(V)$ be the dual of $Y\in H_{2n-2q}(V,\bfZ)$ and let $\psi$ be a residue operator for $Y$. Then (c.f. \eqref{art08-sec5-eq5.23} above) $\tau^{*}\psi=\theta$ is a residue operator for $D(Y)\subset \Sigma$, and so (c.f. Appendix to \S\ref{art08-sec4}, section (e)) $\overline{\partial}\theta$ is the dual of $D(Y)\in H_{2N-2}(\Sigma,\bfZ)(N=\dim \Sigma)$. But $\overline{\partial}\theta=\tau^{*}\overline{\partial}\psi=\tau^{*}\omega$, and so \eqref{art08-sec5-eq5.12} is proved.\hfill$\square$

\medskip
(b)~ \textsc{Reciprocity Relations in Higher Codimension.} Let $Y=Z-\bfZ$ be as in beginning of \S\ref{art08-sec5}, section (a) above. We assume that $Y=0$ in $H_{2n-2q}(V,\bfZ)$ so that $D(Y)$ is algebraically equivalent to zero on $\Sigma=\Sigma(\bfW)$. Let $\psi$ be a residue operator for $Y$ and $\theta=\tau^{*}\psi$ be\pageoriginale defined by \eqref{art08-sec5-eq5.21}. Then (c.f. \eqref{art08-sec5-eq5.20}) $\theta$ is an integral of the $3^{\text{rd}}$ kind on $\Sigma$ whose logarithmic residue locus in $D(Y)$.

Now $\psi$ is determined up to $S=H^{2p-1,0}(V)+\cdots+H^{p,p-1}(V)$. Since $\tau^{*}(H^{p+r,p-1-r}(V))=0$ for $r>0$, $\theta$ is determined up to $\tau^{*}(S)$ where only $\tau^{*}(H^{p,p-1}(V))\subset H^{1,0}(\Sigma)$ (c.f. \eqref{art08-sec5-eq5.11}) counts. Let us prove now :
\begin{equation*}
\begin{array}{l}
D(Y)\equiv 0\text{~\em on~} \Sigma \text{~\em if, and only, if, there exists~}\omega\in H^{1,0}(\Sigma)\\
\text{\em such that~}\int_{\delta}\theta+\omega\equiv 0(1)\text{~\em for all~}\delta\in H_{1}(\Sigma-D(Y),\bfZ).
\end{array}\tag{5.24}\label{art08-sec5-eq5.24}
\end{equation*}

\eject

\begin{proof}
If $\omega$ exists satisfying $\int_{\delta}\theta+\omega\equiv 0(1)$ for all $\delta\in H_{1}(\Sigma-D(Y),\bfZ)$, then we may set :
\begin{equation*}
f(W)=\exp \left(\int\limits^{W}_{\bfW}\theta+\omega\right),\quad (\exp \xi=e^{2\pi i\xi}).\tag{5.25}\label{art08-sec5-eq5.25}
\end{equation*}
This $f(W)$ is a single-valued meromorphic function and, by \eqref{art08-sec5-eq5.20}, $(f)=D(Y)$.

Conversely, assume that $D(Y)=(f)$. Then $\theta-\dfrac{1}{2\pi i}\dfrac{df}{f}=-\omega$ will be a holomorphic 1-form in $H^{1,0}(\Sigma)$, and for $\delta\in H_{1}(\Sigma-D(Y),\bfZ)$, $\int\limits_{\gamma}\theta+\omega=\dfrac{1}{2\pi i}\int\limits_{\gamma}\dfrac{df}{f}=\dfrac{1}{2\pi i}\int\limits_{\gamma}d\log f\equiv 0(1)$. This proves \eqref{art08-sec5-eq5.24}.

Suppose we can prove :
\begin{equation*}
\begin{array}{l}
\text{\em There exists $\eta\in S$ such that $\int_{\Gamma}\psi+\eta\equiv 0(1)$ for all}\\[2pt]
\Gamma\in H_{2q-1}(V-Y,\bfZ)\text{~\em if and only if,~}\phi_{q}(Y)=0\text{~\em in~} T_{q}(\Phi,V)\subset T_{q}(V).
\end{array}\tag{5.26}\label{art08-sec5-eq5.26}
\end{equation*}
\end{proof}

Then we can prove the Abel's theorem \eqref{art08-sec5-eq5.3} as follows.

\begin{proof}
If $\phi_{q}(Y)=0$ in $T_{q}(\Phi,V)$, then by \eqref{art08-sec5-eq5.26} we may find $\eta\in S$ such that $\int_{\Gamma}\psi+\eta\equiv 0(1)$ for all $\Gamma\in H_{2q-1}(V-Y,\bfZ)$. Set $\omega=\tau^{*}\eta\in H^{1,0}(\Sigma)$. Then, for $\delta\in H_{1}(\Sigma-D(Y),\bfZ)$, $\int_{\delta}\theta+\omega=\int_{\delta}\tau^{*}(\psi+\eta)=\int_{\tau(\delta)}\psi+\eta\equiv 0(1)$, where $\tau$ is given by \eqref{art08-sec5-eq5.9}. Using \eqref{art08-sec5-eq5.24}, we have proved \eqref{art08-sec5-eq5.2}.
\end{proof}

\medskip
\noindent
{\bf Remark \thnum{5.27}.\label{art08-sec5-rem5.27}}
The converse to Abel's theorem \eqref{art08-sec5-eq5.2}, which reads :
\begin{equation*}
\phi_{q}(Y)=0\text{~ in~ } T_{q}(Y)\text{~ if~ } D(Y)\equiv 0\text{~ in~ }\Sigma,\tag{5.28}\label{art08-sec5-eq5.28}
\end{equation*}
will\pageoriginale be true, up to isogeny, if we have :
\begin{equation*}
\text{\em The mapping~ } \tau^{*}:H^{q,q-1}(V)\to H^{1,0}(\Sigma)\text{~ \em is into.}\tag{5.29}\label{art08-sec5-eq5.29}
\end{equation*}

\begin{proof}
Referring to \eqref{art08-sec5-eq5.5}, we see that $\tau^{*}$ is
$$
(\zeta_{\Phi})_{*}:T_{0}(T_{q}(\Phi,V))\to T_{0}(\Pic(\Sigma)),
$$
so that $\zeta_{\Phi}$ is an isogeny of $T_{q}(\Phi,V)$ onto an abelian subvariety of $\Pic(\Sigma)$.
\end{proof}

\noindent
{\bf Proof of \eqref{art08-sec5-eq5.26}.} Let $\Gamma_{1},\ldots,\Gamma_{2m}$ be a set of free generators of $H_{2q-1}\break (V,\bfZ)$ (mod torsion). We may assume that $\Gamma_{\rho}$ lies in $H_{2q-1}(V-Y,\bfZ)$, since $\int_{\delta}\psi\equiv 0(1)$ for all $\delta$ in $H_{2q-1}(V-Y,\bfZ)$ which are zero in $H_{2q-1}\break (V,\bfZ)$ (c.f. \eqref{art08-sec5-eq5.19}). Choose a basis $\eta^{1},\ldots,\eta^{m}$ for $S$ and set $\pi_{\alpha}:\left\{\begin{smallmatrix} \vdots\\ \int\limits_{\Gamma_{\rho}}\eta^{\alpha}\\ \vdots\end{smallmatrix}\right\}$. Then $\pi_{\alpha}\in \bfC^{2m}$ and we let $\bfS$ be the subspace generated by $\pi_{1},\ldots,\pi_{m}$. The lattice generated by integral vectors $\left\{\begin{smallmatrix} k^{1} \\ \vdots\\ k^{2m}\end{smallmatrix}\right\}$ projects onto a lattice in $\bfC^{2m}/\bfS$, and the resulting torus is $T_{q}(V)$.

\begin{proof}
We may identify $\bfC^{2m}$ with $H^{2q-1}(V,\bfC)=H_{2q-1}(V,\bfC)^{*}$; $\bfS$ is the subspace $H^{2q-1,0}(V)+\cdots+H^{q,q-1}(V)$, and the integral vectors are just $H^{2q-1}(V,\bfZ)$. Thus the torus above is $H^{q-1,q}(V)+\cdots+H^{0,2q-1}(V)/\break H^{2q-1}(V,\bfZ)$. Let $\pi(\psi)=\left\{\begin{smallmatrix} \vdots\\ \int\limits_{\Gamma_{\rho}}\psi\\ \vdots\end{smallmatrix}\right\}$; $\pi(\psi)$ projects onto a point $\pi(\psi)\in T_{q}(V)$, and we see that:
\begin{equation*}
\begin{array}{l}
\text{The congruence $\int_{\Gamma}\psi+\eta\equiv 0(1)$ $(\Gamma\in H_{2q-1}(V-Y,\bfZ))$ can}\\[2pt]
\text{be solved for some $\eta\in S$ if, and only if, $\pi(\psi)=0$ in $T_{q}(V)$.}
\end{array}\tag{5.30}\label{art08-sec5-eq5.30}
\end{equation*}

Thus, to prove \eqref{art08-sec5-eq5.26}, we need to prove the following {\em reciprocity relation:}
\begin{equation*}
\pi(\psi)=\phi_{q}(Y)\text{~~ in~~ } T_{q}(V).\tag{5.31}\label{art08-sec5-eq5.31}
\end{equation*}

\eject

Let\pageoriginale $e_{\rho}\in H^{2n-2q+1}(V,\bfZ)$ be the harmonic form dual to $\Gamma_{\rho}\in H_{2q-1}\break (V,\bfZ)$. We claim that, if we can find $\eta\in S$ such that we have
\begin{equation*}
\int\limits_{\Gamma_{\rho}}\psi-\int\limits_{C}e_{\rho}=(\Gamma_{\rho},C)+\int\limits_{\Gamma_{\rho}}\eta(\rho=1,\ldots,2m),\tag{5.32}\label{art08-sec5-eq5.32}
\end{equation*}
then \eqref{art08-sec5-eq5.31} holds.
\end{proof}

\begin{proof}
By normalizing $\psi$, we may assume that $\eta=0$ in \eqref{art08-sec5-eq5.32}. Let $e^{*}_{\rho}\in H^{2q-1}(V)$ be the harmonic form defined by $\int V^{e}_{\rho}\wedge e^{*}_{\sigma}=\delta^{\rho}_{\sigma}$. Choose a harmonic basis $\omega^{1},\ldots,\omega^{m}$ for $H^{2n-2q+1,0}+\cdots+H^{n-q+1,n-q}$ and let $\phi^{1},\ldots,\phi^{m}$ be a dual basis for $H^{q-1,q}+\cdots+H^{0,2q-1}$. Then $\omega^{\alpha}=\sum\limits^{2m}_{\rho=1}\mu_{\rho\alpha}e_{\rho}$ and $e^{*}_{\rho}=\sum\limits^{m}_{\alpha=1}(\mu_{\rho\alpha}\phi^{\alpha}+\overline{\mu}_{\rho\alpha}\overline{\phi}^{\alpha})$. It follows that $\pi(\psi)$ is given by the column vector $\left\{\begin{smallmatrix} \vdots \\ \sum\limits^{2m}_{\rho=1}\mu_{\rho\alpha}\int\limits_{\Gamma_{\rho}}\psi\\ \vdots\end{smallmatrix}\right\}$. From \eqref{art08-sec5-eq5.32}, we have $\sum\limits^{2m}_{\rho=1}\mu_{\rho\alpha}\int\limits_{\Gamma_{\rho}}\psi-\int\limits_{C}\sum\limits^{2m}_{\rho=1}\mu_{\rho\alpha}e_{\rho}=\sum\limits^{2m}_{\rho=1}\mu_{\rho\alpha}(\Gamma_{\rho}\cdot C)$, which says that
$$
\left\{\begin{matrix}
\vdots\\
\sum\limits^{2m}_{\rho=1}\mu_{\rho\alpha}\int\limits_{\Gamma_{\rho}}\psi\\
\vdots
\end{matrix}\right\}
-
\left\{\begin{matrix}
\vdots\\
\int\limits_{C}\omega^{\alpha}\\
\vdots
\end{matrix}\right\}
=\sum\limits^{2m}_{\rho=1}(\Gamma_{\rho}\cdot C)
\left\{\begin{matrix}
\mu_{\rho^{1}}\\
\vdots\\
\mu_{\rho^{m}}
\end{matrix}\right\},
$$
which lies in the lattice defining $T_{q}(V)$. Thus $\pi(\psi)=\phi_{q}(Y)$ in $T_{q}(V)$.\hfill Q.E.D.
\end{proof}

Thus we must prove \eqref{art08-sec5-eq5.32}, which is a generalization of the {\em bilinear relations involving integrals of the third kind} on a curve (c.f. \cite{art08-key24}). We observe that, because of the term involving $\eta$, \eqref{art08-sec5-eq5.32} is {\em independent of which residue operator we choose}. We shall use the method of Kodaira \cite{art08-key17} to find one such $\psi$; in this, we follow the notations of \cite{art08-key17}.

Let then $\gamma^{2n-2q}(z,\xi)$ on $V\times V$ be the double {\em Green's form} associated to the $2n-2q$ forms on $V$ and the K\"ahler metric. This is the unique form satisfying
\begin{itemize}
\item[(a)]\pageoriginale \hfill $\Delta_{z}\gamma^{2n-2q}(z,\xi)=\sum\limits^{l}_{j=1}\theta^{j}(z)\wedge \theta^{j}(\xi)$,\hfill\,

where the $\theta^{j}$ are a basis for the harmonic $2n-2q$ forms;

\item[(b)] $\gamma^{2n-2q}(z,\xi)$ is smooth for $z\neq\xi$ and has on the diagonal $z=\xi$ the singularity of a fundamental solution of the Laplace equation;

\item[(c)] $\gamma^{2n-2q}(z,\xi)=\gamma^{2n-2q}(\xi,z)$ and is orthogonal to all harmonic $2n-2q$ forms (i.e. $\int_{V}\gamma^{2n-2q}(z,\xi)\wedge_{*}\theta^{j}(\xi)=0$ for all $z$ and $j=1,\ldots,l$); 

\item[(d)] $\delta_{z}\gamma^{2n-2q}(z,\xi)=d_{\xi}\gamma^{2n-2q-1}(z,\xi)$, 

and $*_{z}*_{\xi}\gamma^{2n-2q}(z,\xi)=\gamma^{2q}(z,\xi)$.
\end{itemize}

Define now a $2n-2q$ form $\phi$ by the formula :
\begin{equation*}
\phi(z)=\int\limits_{\xi\in Y}\gamma^{2n-2q}(z,\xi)d\xi.\tag{5.33}\label{art08-sec5-eq5.33}
\end{equation*}
Then $\phi$ is smooth in $V-Y$ and, by (b) above, can be shown to have a pole or order $2q-2$ along $Y$. We let
\begin{equation*}
\psi=* \ d\phi.\tag{5.34}\label{art08-sec5-eq5.34}
\end{equation*}
Then $\psi$ is a real $2q-1$ form. Since $Y$ is an algebraic cycle, $\phi$ will have type $(n-q,n-q)$ and so $\psi=\psi'+\psi''$ where $\psi'$ has type $(q,q-1)$ and $\psi''=\overline{\psi}'$. We will show that $2\psi'$ is a residue operator for $Y$ and satisfies \eqref{art08-sec5-eq5.32}.

We recall from \cite{art08-key17}, the formula :
\begin{equation*}
\int\limits_{\Gamma_{\rho}}\psi-\int\limits_{C}e_{\rho}=(\Gamma_{\rho}\cdot C),\tag{5.35}\label{art08-sec5-eq5.35}
\end{equation*}
which clearly will be used to give \eqref{art08-sec5-eq5.32}.

First, $\psi$ has singularities only on $Y$ and $d\psi=d \ * \ d\phi=-*\delta d\phi=0$ (c.f. Theorem 4 in \cite{art08-key17}), and $\delta\psi=\delta \ * \ d\phi=\pm \ * \ d^{2}\phi=0$, so that $\psi$ is harmonic in $V-Y$. Thus $\psi'$ and $\psi''$ are harmonic in $V-Y$.

Let $J$ be the operator on forms induced by the complex structure. Then $J^{*}=*J$ and $J\phi=\phi$ (since $J\bfT_{\xi}(Y)=\bfT_{\xi}(Y)$). Thus $\psi=*dJ\phi=*JJ^{-1}dJ\phi=J*(L\delta-\delta L)\phi=-J^{-1}*\delta L\phi$ since $\delta\phi=0$. This gives that $J\psi=-d*L\phi$, so that, using $J\psi=i(\psi'-\psi')$, we find :
\begin{equation*}
2\psi'=\psi-J\psi=\psi+d(* \ L\phi).\tag{5.36}\label{art08-sec5-eq5.36}
\end{equation*}\pageoriginale
Now $2\psi'$ is a form of type $(q,q-1)$ satisfying $\partial \psi'=0=\overline{\partial}\psi'$ and combining \eqref{art08-sec5-eq5.35} and \eqref{art08-sec5-eq5.36},
\begin{equation*}
\int\limits_{\Gamma_{\rho}}2\psi'-\int\limits_{C}e_{\rho}=(\Gamma_{\rho}\cdot C).\tag{5.37}\label{art08-sec5-eq5.37}
\end{equation*}

Finally, the same argument as used in \cite{art08-key17}, pp. 121-123, shows that $2\psi'$ has a pole of order $2q-1$ along $Y$ and gives a residue operator for $Y$. This completes the proof of \eqref{art08-sec5-eq5.32} and hence of \eqref{art08-sec5-eq5.3}.

\section{Chern Classes and Complex Tori.}\label{art08-sec6}

Let $V$ be an algebraic manifold and $\bfE_{\infty}\to V$ a $C^{\infty}$ vector bundle with fibre $\bfC^{k}$. We let $\Sigma(\bfE_{\infty})$ be the set of {\em complex structures} on $\bfE_{\infty}\to V$ (i.e. the set of holomorphic bundles $\bfE\to V$ with $\bfE\displaystyle{\mathop{\cong}\limits_{C^{\infty}}}\bfE_{\infty}$). For such a holomorphic bundle $\bfE\to V(\bfE\in \Sigma(\bfE_{\infty}))$, the {\em Chern cycles} $Z_{q}(\bfE)(q=1,\ldots,k)$ (c.f. \cite{art08-key11}, \cite{art08-key12}, \cite{art08-key13}) are virtual subvarieties of codimension $q$, defined up to {\em retional equivalence}. Fixing $\bfE_{0}\in \Sigma(\bfE_{\infty})$, $Z_{q}(\bfE)-Z_{q}(\bfE_{0})\in \Sigma_{q}$ and we define
\begin{equation*}
\phi_{q}:\Sigma(\bfE_{\infty})\to T_{q}(S),\tag{6.1}\label{art08-sec6-eq6.1}
\end{equation*}
by $\phi_{q}(\bfE)=\phi_{q}(Z_{q}(\bfE)-Z_{q}(\bfE_{0}))(\bfE\in \Sigma(\bfE_{\infty}))$. We may think of $\phi_{q}(\bfE)$ as giving the {\em periods of the holomorphic bundle} $\bfE$. In addition to asking for the image $\phi_{q}(\Sigma(\bfE_{\infty}))\subset T_{q}(S)$, we may also ask to what extent {\em do the periods of} $\{\bfE\}\in \Sigma(\bfE_{\infty})$ {\em give the moduli of} $\bfE$ ? By putting things into the context of deformation theory, we shall infinitesimalize these questions.

Let then $\{\bfE_{\gamma}\}_{\lambda\in\Delta}$ be a family of holomorphic bundles over $V$ ($\Delta$ = disc in $\lambda$-plane). Relative to a suitable covering $\{U_{\alpha}\}$ of $V$, we may give this family by holomorphic transition functions $g_{\alpha\beta}(\lambda):U_{\alpha}\cap U_{\beta}\to GL(k)$ which satisfy the cocycle rule :
\begin{equation*}
g_{\alpha\beta}(\lambda)g_{\beta\gamma}(\lambda)=g_{\alpha\gamma}(\lambda)\text{~~ in~~ } U_{\alpha}\cap U_{\beta}\cap U_{\gamma}.\tag{6.2}\label{art08-sec6-eq6.2}
\end{equation*}
We recall that Kodaira and Spencer \cite{art08-key15} have defined the {\em infinitesimal deformation mapping :}
\begin{equation*}
\delta : \bfT_{\lambda}(\Delta)\to H^{1}(V,\mathcal{O}(\Hom(\bfE_{\lambda},\bfE_{\lambda}))).\tag{6.3}\label{art08-sec6-eq6.3}
\end{equation*}
Explicitly,\pageoriginale $\delta\left(\dfrac{\delta}{\delta\lambda}\right)$ is given by the \~Cech cocycle $\xi_{\alpha\beta}=\overdot{g}_{\alpha\beta}(\lambda)g_{\alpha\beta}(\lambda)^{-1}$ $(\overdot{g}_{\alpha\beta}=\partial g_{\alpha\beta}/\partial\lambda)$; the cocycle rule here follows by differentiating \eqref{art08-sec6-eq6.2}. 

Now define $\phi_{q}:\Delta\to T_{q}(S)$ by $\phi_{q}(\lambda)=\phi_{q}(\bfE_{\lambda})$ ($\bfE_{0}$ being the base point). Recall (c.f. \eqref{art08-sec3-eq3.3}) that $(\phi_{q})_{*}:(\bfT_{0}(\Delta))\subset H^{q-1,q}(V)$, so that we have a diagram $(\phi_{*}=(\phi_{q})_{*})$~:
\begin{equation*}
\vcenter{\xymatrix{
 & H^{1}(V,\mathcal{O}(\Hom(\bfE,\bfE)))\ar@{--}[dd]^{\zeta}\\
\bfT_{0}(\Delta)\ar[ur]^-{\delta}\ar[dr]_-{\phi_{*}} & \\
 & H^{q}(V,\Omega^{q-1})
}}\tag{6.4}\label{ART08-SEC6-EQ6.4}
\end{equation*}
What we want is $\zeta:H^{1}(V,\mathcal{O}(\Hom(\bfE,\bfE)))\to H^{q}(V,\Omega^{q-1})$ which will always complete \eqref{ART08-SEC6-EQ6.4} to a commutative diagram.

We have a formula for $\zeta$ (c.f. \eqref{ART08-SEC6-EQ6.8}) which we shall give after some preliminary explanation.

First we consider {\em symmetric, multilinear, invariant} forms 
$$
P(A_{1},\ldots,\break A_{q})
$$ 
where the $A_{\alpha}$ are $k\times k$ matrices. Invariance means that 
$$
P(gA_{1}g^{-1},\ldots,gA_{q}g^{-1})=P(A_{1},\ldots,A_{q}) (g\in GL(k)).
$$ 
Such a symmetric, invariant form gives an {\em invariant polynomial} $P(A)=P(A,\ldots,A)$. Conversely, an invariant polynomial gives, by {\em polarization}, a symmetric invariant form. For example, if $P(A)=\det(A)$, then
\begin{equation*}
P(A_{1},\ldots,A_{k})=\dfrac{1}{k!}\sum\limits_{\pi=(\pi_{1},\ldots,\pi_{k})}\det (A^{1}_{\pi_{1}}\ldots A^{k}_{\pi_{k}}),\tag{6.5}\label{art08-sec6-eq6.5}
\end{equation*}
where $\pi=(\pi_{1},\ldots,\pi_{k})$ is a permutation of $(1,\ldots,k)$ and $A^{\alpha}_{\pi_{\alpha}}$ is the $\alpha^{\text{th}}$ column of $A_{\pi_{\alpha}}$.

The invariant polynomials form a graded ring $I_{*}=\sum\limits_{q\geq 0}I_{q}$, which is discussed in \cite{art08-key11}, \S\ref{art08-sec4}(b). In particular, $I_{*}$ is generated by $P_{0},P_{1},\ldots,P_{k}$ where $P_{q}\in I_{q}$ is defined by
\begin{equation*}
\det\left(\dfrac{iA}{2\pi}+\lambda I\right)=\sum\limits^{k}_{q=0}P_{q}(A)\lambda^{k-q}.\tag{6.6}\label{art08-sec6-eq6.6}
\end{equation*}

Let\pageoriginale now $P\in I_{r}$ be an invariant polynomial. If
$$
A_{\alpha}\in A^{p_{\alpha},q_{\alpha}}(V,\Hom(\bfE,\bfE))
$$
(= space of $C^{\infty}$, $\Hom(\bfE,\bfE)$-valued, $(p_{\alpha},q_{\alpha})$ forms on $V$), then 
$$
P(A_{1},\ldots,A_{r})\in A^{p,q}(V)\left(p=\sum\limits^{r}_{\alpha=1}p_{\alpha},q=\sum\limits^{r}_{\alpha=1}q_{\alpha}\right)
$$ 
is a global form and $\overline{\partial}P(A_{1},\ldots,A_{q})=\sum\limits^{r}_{\alpha=1}\pm P(\ldots,\overline{\partial}A_{\alpha},\ldots)$. We conclude that $P$ {\em gives a mapping on cohomology :}
\begin{gather*}
P : H^{q_{1}}(V,\Omega^{p_{1}}(\Hom(\bfE,\bfE)))\otimes\cdots\otimes H^{q_{r}}(V,\Omega^{p_{r}}(\Hom(\bfE,\bfE)))\\
\to H^{q}(V,\Omega^{p}).\tag{6.7}\label{art08-sec6-eq6.7}
\end{gather*}
Secondly, $\bfE\to V$ defines a cohomology class
$$
\Theta\in H^{1}(V,\Omega^{1}(\Hom(\bfE,\bfE))) (\Theta \text{~ is the~ {\em curvature} in } \bfE;\text{~ c.f. \cite{art08-key1}}),
$$
which is constructed as follows: Let $\theta=\{\theta_{\alpha}\}$ be a connection of type $(1,0)$ for $\bfE\to V$ Thus $\theta_{\alpha}$ is a $k\times k$-matrix-valued $(1,0)$ form in $U_{\alpha}$ with $\theta_{\alpha}-g_{\alpha\beta}\theta_{\beta}g^{-1}_{\alpha\beta}=g^{-1}_{\alpha\beta}dg_{\alpha\beta}$ in $U_{\alpha}\cap U_{\beta}$. Letting $\Theta_{\alpha}=\overline{\partial}\theta_{\alpha}$, $\Theta_{\alpha}=g_{\alpha\beta}\Theta_{\beta}g^{-1}_{\alpha\beta}$ in $U_{\alpha}\cap U_{\beta}$ and so defines $\Theta\in H^{1}(V,\Omega^{1}(\Hom(\bfE,\bfE)))$ ($\Theta$ is the $(1,1)$ component of the curvature of $\theta$).

Our formula is that, if we set
\begin{equation*}
\zeta(\eta)=qP_{q}(\underbrace{\Theta,\ldots,\Theta}_{q-1},\eta)\quad (\eta\in H^{1}(V,\mathcal{O}(\Hom(\bfE,\bfE)))),\tag{6.8}\label{ART08-SEC6-EQ6.8}
\end{equation*}
then \eqref{ART08-SEC6-EQ6.4} will be commutative. Note that, according to \eqref{art08-sec6-eq6.7}, $\zeta(\eta)\in H^{q}(V,\Omega^{q-1})$, so that the formula makes sense.

We shall give two proofs of the fact that $\zeta$ defined by \eqref{ART08-SEC6-EQ6.8} gives the infinitesimal variation in the periods of $\bfE$. The first will be by explicit computation relating the {\em Chern polynomials} $P_{q}(\underbrace{\Theta,\ldots,\Theta}_{q-1};\eta)$ to the {\em Poincar\'e residue operator} alogn $Z_{q}(\bfE)$; both the Chern polynomials and Poincar\'e residues will be related to geometric residues in a manner somewhat similar to \S\ref{art08-sec4} (especially the Appendix there). After preliminaries in \S\ref{art08-sec7}, this first proof (which we give completely only for the top Chern class) will be carried out in \S\ref{art08-sec8}. The general argument is complicated by the singularities of the Chern classes.

The\pageoriginale second proof is based on the transformation formulae developed in \S\ref{art08-sec4}; it uses an integral-geometric argument and requires that the family of bundles be globally parametrized.

\medskip
\noindent
{\bf Some Examples.}~ Let $\bfE\to V$ be a holomorphic vector bundle and $\theta\in H^{0}(V,\Theta)$ a holomorphic vector field. Then $\theta$ exponentiates to a one-parameter group $f(\lambda):V\to V$ of holomorphic automorphisms, and we may set $\bfE_{\lambda}=f(\lambda)^{*}\bfE$ (i.e., $(\bfE_{\lambda})_{z}=\bfE_{f(\lambda),z}$). Let $\omega=P_{q}(\Theta,\ldots,\Theta)$ be a $(q,q)$ form representing the $q^{\text{th}}$ Chern class; we claim that the infinitesimal variation in the periods of $\bfE$ is given by
\begin{equation*}
\langle \theta,\omega\rangle \in H^{q,q-1}(V).\tag{6.9}\label{art08-sec6-eq6.9}
\end{equation*}

\begin{proof}
Since $\langle \theta,\omega\rangle=\langle \theta, P_{q}(\Theta,\ldots,\Theta)\rangle$
$$
=\Sigma P_{q}(\Theta,\ldots,\langle \theta,\Theta\rangle,\ldots,\Theta)=qP_{q}(\underbrace{\Theta,\ldots,\Theta}_{q-1};\langle \theta,\Theta\rangle),
$$
using \eqref{ART08-SEC6-EQ6.8} it will suffice to show that $\langle \theta,\Theta\rangle\in H^{1}(V,\Hom(\bfE,\bfE))$ is the infinitesimal deformation class for the family $\{\bfE_{\lambda}\}=\{f(\lambda)^{*}\bfE\}$. 

Let $\bfP\to V$ be the principal bundle of $\bfE\to V$ and $0\to \Hom(\bfE,\bfE)\break \to \bfT(\bfP)/G\to \bfT(V)\to 0$ the {\em Atiyah sequence} \cite{art08-key1}. The cohomology sequence goes $H^{0}(V,\mathcal{O}(\bfT(\bfP)/G))\to H^{0}(V,\Theta)\xrightarrow{\delta} H^{1}(V,\Hom(\bfE,\bfE))$, and in \cite{art08-key8} it is proved that $\delta(\theta)=\langle \theta,\Theta\rangle$ and is the {\em Kodaira-Spencer class} for the family $\{\bfE_{\lambda}\}$. (This is easy to see directly; 
$$
\Theta\in H^{1}(V,\Hom(\bfT(V),\Hom(\bfE,\bfE)))
$$ 
is the obstruction to splitting the Atiyah sequence holomorphically, and the coboundary $\delta$ is contraction with $\Theta$. But $\delta(\theta)$ is the obstruction to lifting $\theta$ to a bundle automorphism of $\bfE$, and so gives the infinitesimal variation of $f(\lambda)^{*}\bfE$).
\end{proof}

\begin{remark*}
The formula \eqref{art08-sec6-eq6.9} is easy to use on abelian varieties ($\omega$ and $\theta$ have constant coefficients) but, in the absence of knowledge about the algebraic cycles on $V$, fails to yield much new.
\end{remark*}

\medskip
\noindent
{\bf Example \thnum{2}.\label{art08-sec6-exam2}}
Suppose that $\{\bfE_{\lambda}\}_{\lambda\in \Delta}$ is a family of {\em flat bundles} (i.e. having {\em constant transition funcitons}). Then, by \eqref{ART08-SEC6-EQ6.8}, we see that:
\begin{equation*}
\text{The periods~ } \phi_{q}(\bfE_{\lambda})\text{~ are constant for~ } q>1.\tag{6.10}\label{art08-sec6-eq6.10}
\end{equation*}

\begin{remark*}
This\pageoriginale {\em should} be the case because $\bfE_{\lambda}$ is given by a repersentation $\rho_{\lambda}:\pi_{1}(V)\to GL(k)$. If we choose a general curve $C\subset V$, then $\pi_{1}(C)$ maps {\em onto} $\pi_{1}(V)$, and so $\{\bfE_{\lambda}\}$ is given by $\rho_{\lambda}:\pi_{1}(C)\to GL(k)$. Thus $\bfE_{\lambda}$ is determined by $\bfE_{\lambda}|C$, and here the period $\phi_{1}(\bfE_{\lambda})$ is only one which is non-zero (recall that we have $0\to T_{1}(V)\to T_{1}(C)$).
\end{remark*}

\medskip
\noindent
{\bf Example \thnum{3}.\label{art08-sec6-exam3}}
From \eqref{ART08-SEC6-EQ6.8}, it might seem possible that the periods of $\bfE_{\lambda}$ are constant if all of the Chern classes of $\bfT_{\lambda}$ are topologically zero and $\det \bfE_{\lambda}=\bfL$ is constant. This is not the case. Let $C$ be an elliptic curve and $V=P_{1}\times C$. Take the bundle $\bfH\to P_{1}$ degree $1$ and let $\bfJ_{\lambda}\to C$ be a family of bundles of degree zero parametrized by $C$. Set $\bfE_{\lambda}=(\bfH\otimes \bfJ_{\lambda})\oplus (\bfH\otimes \bfJ_{\lambda})^{*}$. Then $\det \bfE_{\lambda}=1$, $c_{2}(\bfE_{\lambda})=-c_{1}(\bfH)^{2}=0$. If $\theta\in H^{0,1}(C)$ is the tangent to $\{\bfJ_{\lambda}\}\to C$, then the tangent $\eta$ to $\{\bfE_{\lambda}\}$ is $\left(\begin{matrix} \theta & 0\\ 0 & -\theta\end{matrix}\right)$, and, if $\Theta$ is the curvature in $\bfH$, then the curvature in $\bfE_{\lambda}$ is $\Theta_{\bfE}=\left(\begin{matrix} \Theta & 0\\ 0 & -\Theta\end{matrix}\right)$. Then $P_{2}(\Theta_{\bfE};\eta)=-(\Theta\theta)\neq 0$ in $H^{1,2}(V)$.

\medskip
\noindent
{\bf Example \thnum{4}.\label{art08-sec6-exam4}}
Perhaps the easiest construction of $\Pic(V)$ (c.f. \cite{art08-key18}) is by using a {\em very positive line bundle} $\bfL\to V$, and so we may wonder what the effect of making vector bundles very positive is. For this, we let $A_{q}(V)=\phi_{q}(\Sigma_{q}(V))\subset T_{q}(V)$ ($A_{q}(V)$ is the part cut out by algebraic cycles algebraically equivalent to zero); $A_{q}(V)$ is an abelian subvariety of $T_{q}(V)$ which is the range of the Weil mapping. Let $\Phi^{p}$ be the algebraic cycles, modulo rational equivalence, of codimension $p$. Then we have (c.f. \S\ref{art08-sec4})
\begin{equation*}
\Phi^{p}\otimes A_{q}(V)\to A_{p+q}(V)\tag{6.11}\label{art08-sec6-eq6.11}
\end{equation*}
(obtained by intersection of cycles). We set $I_{r}(V)=\sum\limits_{\substack{p+q=r\\ p>0}}\Phi^{p}\otimes A_{q}(V)$ (this is the stuff of codimension $r$ obtained by intersection with cycles of higher dimension) and let
\begin{equation*}
N_{r}(V)=A_{r}(V)/I_{r}(V)\tag{6.12}\label{art08-sec6-eq6.12}
\end{equation*}
(here $N_{r}(V)$ stands for the {\em new cycles} not coming by operations in lower codimension). Then (c.f. \S\ref{art08-sec7} below):

Let\pageoriginale $\{\bfE_{\lambda}\}$ be a family of bundles and $\bfL\to V$ any line bundle. Then 
\begin{equation*}
\phi_{r}(\bfE_{\lambda})=\phi_{r}(\bfE_{\lambda}\otimes \bfL)\text{~ in~ } N_{r}(V).\tag{6.13}\label{art08-sec6-eq6.13}
\end{equation*}

In other words, as expected, the essential part of the problem is n't changed by making the $\bfE_{\lambda}$ very positive.

\medskip
\noindent
{\bf Example \thnum{5}.\label{art08-sec6-exam5}}
Here is a point we don't quite understand. Let $\{\bfE_{\lambda}\}$ be a family of bundles on $V=V_{n}(n\geq 4)$ and let $S\subset V$ be a very positive {\em two-dimension subvariety}. Then $\bfE_{\lambda}\to V$ is {\em uniquely determined} by $\bfE_{\lambda}\to S$ (c.f. \cite{art08-key8}). From this it might be expected that, if the periods $\phi_{1}(\bfE_{\lambda})$ and $\phi_{2}(\bfE_{\lambda})$ are constant, then all of the periods $\phi_{q}(\bfE_{\lambda})$ are constant. However, let $A$ be an abelian variety and $\{\bfJ_{\lambda}\}_{\lambda\in H^{1}(A,0)}$ a family of topologically trivial line bundles parametrized by $\lambda\in H^{1}(A,\mathcal{O})$. We let $\bfL_{1}$, $\bfL_{2}$, $\bfL_{3}$ be fixed line bundles with characteristic classes $\omega_{1}$, $\omega_{2}$, $\omega_{3}$ and set
$$
\bfE_{\lambda}=(\bfJ_{\lambda_{1}}\bfL_{1})\oplus (\bfJ_{\lambda_{2}}\bfL_{2})\oplus (\bfJ_{\lambda_{3}}\bfL_{3}).
$$
Then the tangent $\eta$ to the family $\{\bfE_{\lambda}\}$ is $\eta=\left[\begin{smallmatrix} \lambda_{1} & 0 & 0\\ 0 & \lambda_{2} & 0\\ 0 & 0 & \lambda_{3}\end{smallmatrix}\right]$ and the curvature $\Theta=\left[\begin{smallmatrix} \omega_{1} & 0 & 0\\ 0 & \omega_{2} & 0\\ 0 & 0 & \omega_{3}\end{smallmatrix}\right]$. Then $P_{1}(\Theta;\eta)$ = Trace $\eta=\lambda_{1}+\lambda_{2}+\lambda_{3}$. Setting $\lambda_{3}=-\lambda_{1}-\lambda_{2}$ we have $P_{1}(\Theta;\eta)=0$. Now $P_{2}(\Theta;\eta)=\lambda_{1}\omega_{2}+\lambda_{2}\omega_{1}+\lambda_{1}\omega_{3}+\lambda_{3}\omega_{1}+\lambda_{2}\omega_{3}+\lambda_{3}\omega_{2}=\lambda_{1}(\omega_{3}-\omega_{1})+\lambda_{2}(\omega_{3}-\omega_{2})$, and $P_{3}(\Theta;\eta)=\lambda_{1}\omega_{2}\omega_{3}+\lambda_{2}\omega_{1}\omega_{3}+\lambda_{3}\omega_{1}\omega_{2}=\lambda_{1}\omega_{2}(\omega_{3}-\omega_{1})+\lambda_{2}\omega_{1}(\omega_{3}-\omega_{2})$. Clearly we can have $P_{2}(\Theta;\eta)=0$, $P_{3}(\Theta;\eta)=\lambda_{1}(\omega_{3}-\omega_{1})(\omega_{2}-\omega_{1})\neq 0$.

\medskip
\noindent
{\bf Example \thnum{6}.\label{art08-sec6-exam6}}
Examples such as Example \ref{art08-sec6-exam5} above show that the periods fail quite badly in determining the bundle. In fact, it is clear that, if $K(V)$ is the {\em Grothendieck ring constructed from locally free sheaves} (\cite{art08-key12}), the best we can hope for is that the periods determine the image of the bundle in $K(V)$.
\smallskip

Let us prove this for curves:
\begin{equation*}
\begin{array}{l}
\text{\em If $V$ is an algebraic curve and $\bfE\to V$ a holomorphic vector bundle,}\\
\text{\em then the image of $\bfE$ in $K(V)$ is determined by the periods of~ $\bfE$.}
\end{array}\tag{6.14}\label{art08-sec6-eq6.14}
\end{equation*}

\begin{proof}
Let\pageoriginale $\bfI_{k}$ be the trivial bundle of rank $k$; we have to show that $\bfE=\det \bfE\otimes \bfI_{k}$ in $K(V)$ (where $k$ is the fibre dimension of $\bfE$). The assertion is trivially true for $k=1$; we assume it for $k-1$. Since the structure group of $\bfE$ may be reduced to the triangular group \cite{art08-key2}, in $K(V)$ we see that $\bfE=\bfL_{1}\otimes\cdots\oplus \bfL_{k}$ where the $\bfL_{\alpha}$ are line bundles. We choose a very positive line bundle $\bfH$ and sections $\vartheta_{\alpha}\in H^{0}(V,\mathcal{O}(\bfH\otimes \bfL^{*}_{\alpha}))$ which have no common zeroes (since $k>1$). Then the mapping $f\to (f\vartheta_{1},\ldots,f\vartheta_{k})(f\in \mathcal{O})$ gives an exact bundle sequence $0\to \bfH\to \bfL_{1}\oplus\cdots\oplus \bfL_{k}\to \bfQ\to 0$ where $\bfQ$ has rank $k-1$ and $\det \bfQ=\bfH^{*}\det \bfE$. By induction, $\bfQ=\bfH^{*}\det \bfE\otimes \bfI_{k-1}$ in $K(V)$, and so $\bfE=\det \bfE\otimes \bfI_{k}$ in $K(V)$ as required.
\end{proof}

\section{Properties of the Mapping \texorpdfstring{$\zeta$}{zeta} in \texorpdfstring{\eqref{ART08-SEC6-EQ6.8}}{eq6.8}.}\label{art08-sec7}

(a)~ \textsc{Behavior under direct sums.} Let $\{E_{\lambda}\}$, $\{\bfF_{\mu}\}$ be families of holomorphic bundles over $V$. What we claim is:
\begin{equation*}
\begin{array}{l}
\text{\em If \eqref{ART08-SEC6-EQ6.4} holds for each of the families}\\[2pt]
\text{\em $\{\bfE_{\lambda}\}$ and $\{\bfF_{\mu}\}$,then it holds for $\{\bfE_{\lambda}\oplus \bfF_{\mu}\}$.}
\end{array}\tag{7.1}\label{art08-sec7-eq7.1}
\end{equation*}

\begin{proof}
By linearity, we may suppose that the $\{\bfF_{\mu}\}$ is a constant family; thus all $\bfF_{\mu}=\bfF$. Letting $\bfE=\bfE_{0}$, the Kodaira-Spencer class $\delta(\partial/\partial \lambda)$ for $\{\bfE_{\lambda}\oplus \bfF\}$ lies then in $H^{1}(V,\mathcal{O}(\Hom(\bfE,\bfE)))\subset H^{1}(V,\mathcal{O}(\Hom(\bfE\oplus \bfF,\bfE\oplus \bfF)))$. If $\theta_{\bfE}$ is a $(1,0)$ connection in $\bfE$ and $\theta_{\bfF}$ a $(1,0)$ connection in $\bfF$, then $\theta_{\bfE\oplus \bfF}=\theta_{\bfE}\oplus \theta_{\bfF}\left(=\left(\begin{matrix} \theta_{\bfE} & 0\\ 0 &\theta_{\bfF}\end{matrix}\right)\right)$ is a $(1,0)$ connection in $\bfE\oplus\bfF$ and $\Theta_{\bfE\oplus\bfF}=\Theta_{\bfF}\oplus\Theta_{\bfF}$. From
$$
\det \left(\dfrac{i}{2\pi}\Theta_{\bfE\oplus\bfF}+\lambda I\right)=\det \left(\dfrac{i}{2\pi}\Theta_{\bfE}+\lambda I\right)\det \left(\dfrac{i}{2\pi}\Theta_{\bfF}+\lambda I\right),
$$
we get that
\begin{equation*}
P_{q}(\Theta_{\bfE\oplus \bfF})=\sum\limits_{r+s=q}P_{r}(\Theta_{\bfE})P_{s}(\Theta_{\bfF}).\tag{7.2}\label{art08-sec7-eq7.2}
\end{equation*}
Now \eqref{art08-sec7-eq7.2} is the {\em duality theorem}; in the rational equivalence ring, we have 
\begin{equation*}
Z_{q}(\bfE\oplus \bfF)=\sum\limits_{r+s=q}Z_{r}(\bfE)\cdot Z_{s}(\bfF).\tag{7.3}\label{art08-sec7-eq7.3}
\end{equation*}

Then\pageoriginale
\begin{align*}
\phi_{q}(\bfE_{\lambda}\oplus\bfF) &= \phi_{q}(Z_{q}(\bfE_{\lambda}\oplus\bfF)-Z_{q}(\bfE\oplus \bfF))\\[3pt]
&= \phi_{q}\left(\sum\limits_{r+s=q}\{Z_{r}(\bfE_{\lambda})\cdot Z_{s}(\bfF)-Z_{r}(\bfE)\cdot Z_{s}(\bfF)\}\right)\\[3pt]
&= \phi_{q}\left(\sum\limits_{r+s=q}[Z_{r}(\bfE_{\lambda})-Z_{r}(\bfE)]\cdot Z_{s}(\bfF)\right)\\[3pt]
&= \sum\limits_{r+s=q}\phi_{q}([Z_{r}(\bfE_{\lambda})-Z_{r}(\bfE)]\cdot Z_{s}(\bfF))\\[3pt]
&= (\text{by~ } \eqref{art08-sec4-eq4.18} \sum\limits_{r+s=q}\Psi_{s}(\bfF)\phi_{r}(Z_{r}(\bfE_{\lambda})-Z_{r}(\bfE))
\end{align*}
where $\Psi_{s}(\bfF)=P_{s}(\Theta_{\bfF})\in H^{s,s}(V)\cap H^{2s}(V,\bfZ)$ is the Poincar\'e dual to $Z_{s}(\bfF)$ and $\Psi_{s}(\bfF):T_{r}(V)\to T_{r+s}(V)$ is the mapping given by \eqref{art08-sec2-eq2.7}. It folows that:
\begin{equation*}
\left\{\phi_{q}(\bfE_{\lambda}\oplus \bfF)\right\}_{*}=\sum\limits_{s+r=q}P_{s}(\Theta_{\bfF})\left\{\phi_{r}(\bfE_{\lambda})\right\}_{*}.\tag{7.4}\label{art08-sec7-eq7.4}
\end{equation*}
Assuming \eqref{ART08-SEC6-EQ6.4} for the family $\{\bfE_{\lambda}\}$, the right hand side of \eqref{art08-sec7-eq7.4} is $\sum\limits_{r+s=q-1}rP_{r+1}(\underbrace{\Theta_{\bfE}}_{r};\eta)P_{s}(\Theta_{\bfF})$. Since we want this to equal 
$$
qP_{q}(\underbrace{\Theta_{\bfE}\oplus \Theta_{\bfF}}_{q-1};\eta\oplus 0),
$$ 
to prove \eqref{art08-sec7-eq7.1} we must prove the algebraic identity:
\begin{equation*}
qP_{q}(\underbrace{A\oplus B}_{q-1};\xi\oplus 0)=\sum\limits_{r+s=q-1}rP_{r+1}(\underbrace{A}_{r};\xi)P_{s}(B),\tag{7.5}\label{art08-sec7-eq7.5}
\end{equation*}
where $A$, $B$, $\xi$ are matrices.

Expanding $P_{q}(A\oplus B;\xi\oplus 0)$ gives
$$
qP_{q}(\underbrace{A\oplus B}_{q-1};\xi\oplus 0)=\Sigma_{q}\binom{q-1}{r}P_{q}(\underbrace{A\oplus 0}_{r};\xi\oplus 0;\underbrace{0\oplus B}_{s})
$$
$(s=q-r-1)$. Thus, to prove \eqref{art08-sec7-eq7.5}, we need to show:
\begin{equation*}
rP_{r+1}(\underbrace{A}_{r};\xi)P_{s}(B)=q\binom{q-1}{r}P_{q}(\underbrace{A\oplus 0}_{r};\xi\oplus 0; \underbrace{0\oplus B}_{s}).\tag{7.6}\label{art08-sec7-eq7.6}
\end{equation*}
Clearly the only question is the numerical factors; for these may take $A$, $B$, $\xi$ to be diagonal. Now in general if $A_{1},\ldots,A_{t}$ are diagonal matrices, say $A_{\alpha}=\left[\begin{smallmatrix} A^{1}_{\alpha} & & 0\\ & \vdots & \\ 0 & & A^{k}_{\alpha}\end{smallmatrix}\right]$, then $P_{t}(A_{1},\ldots,A_{t})=\dfrac{1}{t!}\sum\limits_{\pi}A^{\pi_{1}}_{1}\ldots A^{\pi t}_{t}$\pageoriginale where the summation is over all subsets $\pi=(\pi_{1},\ldots,\pi_{t})$ of $(1,\ldots,k)$. Thus $q\binom{q-1}{r}P_{q}(\underbrace{A\oplus 0}_{r};\xi\oplus 0; \underbrace{0\oplus B}_{s})$
\begin{align*}
&= \dfrac{1}{(q-1)!}\binom{q-1}{r}\sum\limits_{\pi}A^{\pi_{1}}\ldots A^{\pi}_{r}\xi^{\pi_{r+1}}B^{\pi_{r+2}}\ldots B^{\pi_{q}}\\[3pt]
&= \left(\dfrac{1}{r!}\sum\limits_{\pi}A^{\pi_{1}}\ldots A^{\pi_{r}}\xi^{\pi_{r+1}}\right)\left(\dfrac{1}{s!}\sum\limits_{r}B^{\tau_{1}}\ldots B^{\tau_{s}}\right)\\[3pt]
&= rP_{r}(\underbrace{A}_{r};\xi)P_{s}(B).
\end{align*}
This proves \eqref{art08-sec7-eq7.6}.
\end{proof}

(b)~ \textsc{Behavior under tensor products.} With the notations and assumptions of \ref{art08-sec7}(a) above, we want to prove :
\begin{equation*}
\begin{array}{l}
\text{\em If \eqref{ART08-SEC6-EQ6.4} holds for each of the families $\{\bfE_{\lambda}\}$ and $\{\bfF_{\mu}\}$,}\\
\text{\em then \eqref{ART08-SEC6-EQ6.4} holds for $\bfE_{\lambda}\oplus \bfF_{\mu}$.}
\end{array}\tag{7.7}\label{art08-sec7-eq7.7}
\end{equation*}

\begin{proof}
As in the proof of \eqref{art08-sec7-eq7.1}, we assume that all $\bfF_{\mu}=\bfF$, $\bfE_{0}=\bfE$, and then $\delta\binom{\partial}{\partial \lambda}=\eta\oplus 1$ in $H^{1}(V,\mathcal{O}(\Hom(\bfE,\bfE)))\otimes H^{0}(V,\mathcal{O}(\Hom(\bfF,\bfF)))\subset H^{1}(V,\mathcal{O}(\Hom(\bfE\otimes \bfF,\bfE\otimes\bfF)))$ where $\eta\in H^{1}(V,\mathcal{O}(\Hom(\bfE,\bfE)))$ is $\delta\left(\dfrac{\partial}{\partial \lambda}\right)$ for the family $\{\bfE_{\lambda}\}$. Also, to simplify the algebra, we assume that $F$ is a line bundle and set $\omega=\dfrac{i}{2\pi}\Theta_{\bfF}(=c_{1}(\bfF))$.

Now $\theta_{\bfE\otimes \bfF}=\theta_{\bfE}\otimes 1+1\otimes \theta_{\bfF}$ is a $(1,0)$ connection in $\bfE\otimes \bfF$ with curvature $\Theta_{\bfE\otimes \bfF}=\Theta_{E}\otimes 1+1\otimes \Theta_{\bfF}$. We claim that
\begin{equation*}
P_{q}(\Theta_{\bfE\otimes\bfF})=\sum\limits_{r+s=q}\binom{k-r}{s}\omega^{s}P_{s}(\Theta_{E}).\tag{7.8}\label{art08-sec7-eq7.8}
\end{equation*}
\end{proof}

\begin{proof}
$P_{q}(A\otimes 1+1\otimes B)=\sum\limits_{r+s=q}\binom{q}{r}P_{q}(\underbrace{A\otimes 1}_{r},\underbrace{1\otimes B}_{s})$. Assuming that $A$ is a $k\times k$ matrix and $B=(b)$ is $1\times 1$, we have $P_{q}(A\otimes 1,1\otimes B)=\Sigma c_{q,r}P_{r}(A)b^{s}(s=q-r)$ and we need to determine\pageoriginale the $c_{q,r}$. Letting $A=\left[\begin{smallmatrix} A^{1} & & 0\\  & \vdots &\\0 & & A^{k} \end{smallmatrix}\right]$, $P_{q}(A\otimes 1,1\otimes B)=\dfrac{1}{q!}\sum\limits_{\pi}A^{\pi_{1}}\ldots A^{\pi_{r}}b^{\pi_{r+1}}\ldots b^{\pi_{q}}=(s!)\dfrac{1}{q!}\binom{k-r}{s}\sum\limits_{\pi}A^{\pi_{1}}\ldots A^{\pi_{r}}b^{s}$, so that $P_{q}(A\otimes 1+1\otimes B)=\sum\limits_{\substack{r+s=q\\ \pi}}\break \binom{q}{r}\dfrac{(q-r)!}{q!}\binom{k-r}{s}A^{\pi_{1}}\ldots A^{\pi_{r}}b^{s}=\sum\limits_{r+s=q}\binom{k-r}{s}P_{r}(A)b^{s}$. This proves \eqref{art08-sec7-eq7.8}.

In the rational equivalence ring, we have
\begin{equation*}
Z_{q}(\bfE_{\lambda}\otimes \bfF)=\sum\limits_{r+s=q}\binom{k-r}{s}Z_{1}(\bfF)^{s}Z_{r}(\bfE_{\lambda}).\tag{7.9}\label{art08-sec7-eq7.9}
\end{equation*}
As in proof of \eqref{art08-sec7-eq7.4} from \eqref{art08-sec7-eq7.3}, we have
\begin{equation*}
\{\phi_{q}(\bfE_{\lambda}\otimes \bfF)\}_{\ast}=\sum\limits_{r+s=q}\binom{k-r}{s}\omega^{s}\{\phi_{r}(\bfE_{\lambda})\}_{*}.\tag{7.10}\label{art08-sec7-eq7.10}
\end{equation*}
Using that \eqref{ART08-SEC6-EQ6.4} holds for $\{\bfE_{\lambda}\}$, the right hand side of \eqref{art08-sec7-eq7.10} becomes $\sum\limits_{r+s=q-1}\binom{k-r-1}{s}\omega^{s}rP_{r+1}(\underbrace{\Theta_{\bfE}}_{r};\eta)$; to prove \eqref{art08-sec7-eq7.7} we must prove the algebraic identity :
\begin{equation*}
qP_{q}(\underbrace{A\otimes 1+1\otimes B}_{q-1};\eta\otimes 1)=\sum\limits_{r+s=q-1}\binom{k-r-1}{s}b^{s}rP_{r+1}(\underbrace{A}_{r};\eta).\tag{7.11}\label{art08-sec7-eq7.11}
\end{equation*}
\end{proof}

\noindent
{\bf Proof of \eqref{art08-sec7-eq7.11}.}~ $qP_{q}(A\otimes 1+1\otimes B;\eta\otimes 1)=$
\begin{align*}
&= \sum\limits_{\substack{r+s=q-1\\ \pi}}q\binom{q-1}{r}P_{r}(\underbrace{A\otimes 1}_{r};\eta\otimes 1; \underbrace{1\otimes B}_{s})=\\[3pt]
&= \sum\limits_{\substack{r+s=q-1\\ \pi}}\dfrac{q}{q!}\binom{q-1}{r}A^{\pi_{1}}\ldots A^{\pi_{r}}\eta^{\pi_{r+1}}b^{\pi_{r+2}}\ldots b^{\pi_{q}}=\\[3pt]
&= \sum\limits_{\substack{r+s=q-1\\ \pi}}\dfrac{1}{(q-1)!}\binom{q-1}{r}\cdot \binom{k-r-1}{s}(q-r-1)!\\[3pt]
&\hspace{5.5cm} b^{s}A^{\pi_{1}}\ldots A^{\pi_{r}}\eta^{\pi_{r+1}}=\\[3pt]
&= \sum\limits_{\substack{r+s=q-1\\ \pi}}\binom{k-r-1}{s}\cdot \dfrac{1}{r!}A^{\pi_{1}}\ldots A^{\pi_{r}}\eta^{\pi_{r+1}}b^{s}=\\[3pt]
&= \sum\limits_{r+s=q-1}\binom{k-r-1}{s}b^{s}rP_{r+1}(\underbrace{A}_{r};\eta).
\end{align*}

(c)~ \textsc{Ample Bundles.}\pageoriginale If $\bfE\to V$ is a holomorphic bundle, we let $\Gamma(\bfE)$ be the trivial bundle $V\times H^{0}(V,\mathcal{O}(\bfE))$. Then we say that $\bfE$ {\em is generated by its sections} if we have :
\begin{equation*}
0\to F\to \Gamma(\bfE)\to \bfE\to 0.\tag{7.12}\label{art08-sec7-eq7.12}
\end{equation*}
Now $\sigma\in \bfF_{z}$ is a section $\sigma$ of $\bfE$ with $\sigma(z)=0$; sending $\sigma\to d\sigma(z)\in \bfE_{z}\otimes \bfT^{*}_{z}(V)$ gives
\begin{equation*}
\bfF\xrightarrow{d}\bfE\otimes \bfT^{*}(V).\tag{7.13}\label{art08-sec7-eq7.13}
\end{equation*}
In \cite{art08-key11}, $\bfE$ was said to be {\em ample} if \eqref{art08-sec7-eq7.12} holds and if $d$ is onto in \eqref{art08-sec7-eq7.13}. In this case, to describe the Chern cycles $Z_{q}(\bfE)$, we choose $k$ general sections $\sigma_{1},\ldots,\sigma_{k}$ of $\bfE\to V$. Then $Z_{q}(\bfE)\subset V$ is given by $\sigma_{1}\wedge\ldots\wedge \sigma_{k-q+1}=0$. (Note that $Z_{1}(\bfE)$ is given by $\sigma_{1}\wedge\ldots\wedge \sigma_{k}=0$ and $Z_{k}(\bfE)$ by $\sigma_{1}=0$.) The cycles $Z_{q}(\bfE)$ are irreducible subvarieties defined up to rational equivalence.

If now $\bfE\to V$ is a general holomorphic bundle, we can choose an ample line bundle $\bfL\to V$ such that $\bfE\otimes \bfL$ is ample (\cite{art08-key11}). Suppose we know \eqref{ART08-SEC6-EQ6.4} for ample bundles. Then \eqref{ART08-SEC6-EQ6.4} holds for $\bfE\otimes\bfL$ and $\bfL$. On the other hand, if \eqref{ART08-SEC6-EQ6.4} is true for a bundle, then it is also true for the dual bundle. Since $\bfE=(\bfE\otimes \bfL)\otimes \bfL^{*}$, using \eqref{art08-sec7-eq7.6} we conclude:
\begin{equation*}
\begin{array}{l}
\text{If \eqref{ART08-SEC6-EQ6.4} if true for ample bundles, then it is}\\
\text{true for all holomorphic vector bundles.}
\end{array}\tag{7.14}\label{art08-sec7-eq7.14}
\end{equation*}

Let then $\{\bfE_{\lambda}\}$ be a family of ample holomorphic vector bundles and $Z_{\lambda}=Z_{q}(\bfE_{\lambda})$. We may form a {\em continuous system} (c.f. \S\ref{art08-sec3}); we let $Z=Z_{0}$ and $\bfN\to Z$ be the normal bundle and $\phi:\Delta\to T_{q}(V)$ the mapping \eqref{art08-sec3-eq3.1} on $Z_{\lambda}-Z_{0}$. Then, combining \eqref{ART08-SEC6-EQ6.4} with the dual diagram to \eqref{art08-sec3-eq3.8}, we have :
\begin{equation*}
\vcenter{\xymatrix@R=1.5cm{
T_{0}(\Delta)\ar[r]^-{\delta}\ar[dr]^-{\phi_{*}}\ar[d]^-{\rho} & H^{1}(V,\mathcal{O}(\Hom(\bfE,\bfE)))\ar[d]^-{\zeta}\\
H^{0}(Z,\mathcal{O}(\bfN))\ar[r]^-{\xi} & H^{q-1,q}(V)
}}\tag{7.15}\label{art08-sec7-eq7.15}
\end{equation*}
Actually\pageoriginale this diagram is not quite accurate; $\bfE_{\lambda}$ determines $Z_{q}(\bfE_{\lambda})$ only up to rational equivalence, and we shall see below that there is a subspace $L_{q}(\bfE)\subset H^{0}(Z,\mathcal{O}(\bfN))$ such that we have :
\begin{equation*}
\vcenter{\xymatrix@R=1.2cm{
T_{0}(\Delta)\ar[d]^-{\rho}\ar[r]^-{\delta} & H^{1}(V,\mathcal{O}(\Hom(\bfE,\bfE)))\ar[d]^-{\zeta}\\
H^{0}(Z,\mathcal{O}(\bfN))/L_{q}(\bfE)\ar[r]^-{\xi} & H^{q-1,q}(V)
}}\tag{7.16}\label{art08-sec7-eq7.16}
\end{equation*}
Now in \S\ref{art08-sec9} below, we shall, under the assumption $H^{1}(V,\mathcal{O}(\bfE))=0$, construct
\begin{equation*}
\theta : H^{1}(V,\mathcal{O}(\Hom(\bfE,\bfE)))\to H^{0}(Z,\mathcal{O}(\bfN))/L_{q}(\bfE)\tag{7.17}\label{art08-sec7-eq7.17}
\end{equation*}
such that
\[
\xymatrix@C=.1cm@R=1.2cm{
T_{0}(\Delta)\ar[rr]^-{\delta}\ar[dr]^-{\rho} & & H^{1}(V,\mathcal{O}(\Hom(\bfE,\bfE)))\ar[dl]_-{\theta}\\
 & H^{0}(Z,\mathcal{O}(\bfN))/L_{q}(\bfE) & 
}
\]
commutes. Putting this in \eqref{art08-sec7-eq7.16}, we have :
\begin{equation*}
\begin{array}{l}
\text{In order to prove \eqref{ART08-SEC6-EQ6.4}, it will suffice to}\\
\text{assume that $\bfE\to V$ is ample, $H^{1}(V,\mathcal{O}(\bfE))=0$,}\\
\text{and then prove that the following diagram commutes.}
\end{array}\tag{7.18}\label{art08-sec7-eq7.18}
\end{equation*}
\begin{equation*}
\vcenter{\xymatrix{
H^{1}(V,\mathcal{O}(\Hom(\bfE,\bfE)))\ar[dr]^-{\zeta}\ar[dd]^-{\theta} &\\
 & H^{q-1,q}(V)\\
H^{0}(Z,\mathcal{O}(\bfN))/L_{q}(\bfE)\ar[ur]^-{\xi} &
}}\tag{7.19}\label{art08-sec7-eq7.19}
\end{equation*}
where $\xi$ is given by \eqref{art08-sec3-eq3.6}, $\zeta$ by \eqref{ART08-SEC6-EQ6.8} and $\theta$ by the \S\ref{art08-sec9} below.

\begin{remark*}
In case $q=k=$ fibre dimension of $\bfE$, $Z\subset V$ is the zero locus of $\sigma\in H^{0}(V,\mathcal{O}(\bfE))$. Then $L_{k}(\bfE)=rH^{0}(V,\mathcal{O}(\bfE))$, where $r:\mathcal{O}_{V}(\bfE)\to \mathcal{O}_{Z}(\bfN)$ is the restriction mapping, and $\theta$ in \eqref{art08-sec7-eq7.17} is constructed\pageoriginale as follows. Let $\eta\in H^{1}(V,\mathcal{O}(\Hom(\bfE,\bfE)))$. Then $\eta\cdot \sigma \in H^{1}(V,\mathcal{O}(\bfE))=0$ and so $\eta\cdot \sigma=\overline{\partial}\tau$ for some $\tau\in \Gamma_{\infty}(V,\bfE)$ ($=C^{\infty}$ sections of $\bfE\to V$). We set $\theta(\eta)=\tau|Z$. If also $\eta\cdot \sigma=\overline{\partial}\widehat{\tau}$, then $\partial(\tau-\widehat{\tau})=0$ so that $\theta(\eta)$ is determined up to $rH^{0}(V,\mathcal{O}(\bfE))$.
\end{remark*}

(d)~ \textsc{Behavior in exact sequences.} Let $\{\bfE_{\lambda}\}$, $\{\bfS_{\lambda}\}$, $\{\bfQ_{\lambda}\}$ be families of holomorphic vector bundles over $V$ such that we have 
\begin{equation*}
0\to \bfS_{\lambda} \to \bfE_{\lambda}\to \bfQ_{\lambda}\to 0.\tag{7.20}\label{art08-sec7-eq7.20}
\end{equation*}
We shall prove:
\begin{equation*}
\begin{array}{l}
\text{\em If \eqref{ART08-SEC6-EQ6.4} holds for each of the}\\
\text{\em families $\{\bfS_{\lambda}\}$, $\{\bfQ_{\lambda}\}$, then it is true for $\{\bfE_{\lambda}\}$.}
\end{array}\tag{7.21}\label{art08-sec7-eq7.21}
\end{equation*}

\begin{proof}
The exact sequences \eqref{art08-sec7-eq7.20} are classified by classes 
$$
e\in H^{1}(V,\mathcal{O}(\Hom(\bfQ_{\lambda},\bfS_{\lambda}))),
$$ 
with $e$ giving the same class as $e'$ if, and only if, $e=\lambda e'(\lambda\neq 0)$. If we show that the periods of $\bfE_{\lambda}$ are independent of this {\em extension class} $e$, then we will have $\phi_{q}(\bfE_{\lambda})=\phi_{q}(\bfS_{\lambda}\oplus \bfQ_{\lambda})$ and so we can use \eqref{art08-sec7-eq7.1}. But $\phi_{q}(e)$ (extension class $e$) = $\phi_{q}(te)$ for all $t\neq 0$, and since $\phi_{q}(te)$ is continuous at $t=0$, $\phi_{q}(\bfE_{\lambda})=\phi_{q}(\bfS_{\lambda}\oplus\bfQ_{\lambda})$. Thus, in order to prove \eqref{art08-sec7-eq7.21}, we must show :

{\em Suppose the $\{\bfE_{\lambda}\}$ is a family with $0\to \bfS\to \bfE_{\lambda}\to \bfQ\to 0$ for all $\lambda$. Then $\zeta(\eta)=0$ in \eqref{ART08-SEC6-EQ6.8} where}
\begin{equation*}
\eta=\delta \left(\dfrac{\partial}{\partial\lambda}\right)\in H^{1}(V,\mathcal{O}(\Hom \bfE,\bfE)).\tag{7.22}\label{art08-sec7-eq7.22}
\end{equation*}
\end{proof}

\noindent
{\bf Proof of \eqref{art08-sec7-eq7.22}.} Assuming that $\bfE=\bfE_{0}$ with
\begin{equation*}
0\to \bfS\to E\xrightarrow{\pi}\bfQ\to 0,\tag{7.23}\label{art08-sec7-eq7.23}
\end{equation*}
we clearly have $\eta\in H^{1}(V,\mathcal{O}(\Hom \bfQ,\bfS))\subset H^{1}(V,\mathcal{O}(\Hom(\bfE,\bfE)))$. Let $\eta$ be a $C^{\infty}(0,1)$ form with values in $\Hom(\bfQ,\bfS)$, and let $e_{1},\ldots,e_{k}$ be a local holomorphic frame for $\bfE$ such that $e_{1},\ldots,e_{l}$ is a frame for $\bfS$. Then $e_{l+1},\ldots,e_{k}$ projects to a frame for $\bfQ$, and locally $\eta=\left(\begin{smallmatrix} \eta_{11} & \eta_{12}\\ \eta_{21} & \eta_{22}\end{smallmatrix}\right)$. Since $\eta|\bfS=0$ and $\eta(\bfE)\subset \bfS$, $\eta_{11}=\eta_{21}=\eta_{22}=0$ and $\eta=\left(\begin{smallmatrix} 0 & \eta_{12}\\ 0 & 0\end{smallmatrix}\right)$.

Suppose\pageoriginale now that we can find a $(1,0)$ connection in $\bfE$ whose local connection matrix (using the above frame) has the form $\widehat{\theta}=\left(\begin{smallmatrix} \widehat{\theta}_{11} & \widehat{\theta}_{12}\\ 0 & \widehat{\theta}_{22}\end{smallmatrix}\right)$. Then the curvature $\overline{\partial}\widehat{\theta}=\widehat{\Theta}_{\bfE}=\left(\begin{smallmatrix} \widehat{\Theta}_{11} & \widehat{\Theta}_{12}\\ 0 & \widehat{\Theta}_{22}\end{smallmatrix}\right)$, and it follows that 
$$
P_{q}(\underbrace{\widehat{\Theta}_{E}}_{q-1};\eta)\equiv 0.
$$

Then let $\theta$ be an arbitrary $(1,0)$ connection in $\bfE$. Locally $\theta=\left(\begin{smallmatrix} \theta_{11} & \theta_{12}\\ \theta_{21} & \theta_{22}\end{smallmatrix}\right)$, and we check easily that $\theta_{21}$ is a global $(1,0)$ form with values in $\Hom(\bfS,\bfQ)$; let $\xi=\left(\begin{smallmatrix} 0 & 0\\ \theta_{21} & 0\end{smallmatrix}\right)\in A^{1,0}(V,\Hom(\bfS,\bfQ))$ and let $\phi:\bfQ\to \bfE$ be a $C^{\infty}$ splitting of \eqref{art08-sec7-eq7.23}. Then $\psi=I-\phi\pi:\bfE\to \bfS$ and satisfies $\psi(v)=v$ for $v\in \bfS$. We let
\begin{equation*}
\widehat{\theta}=\theta-\phi\xi\psi\tag{7.24}\label{art08-sec7-eq7.24}
\end{equation*}
be the $(1,0)$ connection for $\bfE$. Then $\pi\widehat{\theta}|\bfS=0$ and so $\widehat{\theta}=\left(\begin{smallmatrix} \widehat{\theta}_{11} & \widehat{\theta}_{12}\\ 0 & \widehat{\theta}_{22}\end{smallmatrix}\right)$ as required.

(e)~ \textsc{Proof of \eqref{ART08-SEC6-EQ6.4} for line bundles.} Let $\bfE\to V$ be a line bundle; we want to prove \eqref{ART08-SEC6-EQ6.4} for any family $\{\bfE_{\lambda}\}_{\lambda\in \Delta}$ with $\bfE_{0}=\bfE$. By \eqref{art08-sec7-eq7.14}, we may assume that $\bfE\to V$ is ample and $Z\subset V$ is the zero locus of a holomorphic section $\sigma\in H^{0}(V,\mathcal{O}(\bfE))$. Using \eqref{art08-sec7-eq7.18}, to prove \eqref{ART08-SEC6-EQ6.4} we need to show that the following diagram commutes:
\begin{equation*}
\vcenter{\xymatrix{
H^{1}(V,\mathcal{O})\ar[dd]^-{\theta}\ar[dr]^-{\zeta} & \\
            & H^{0,1}(V)\\
H^{0}(Z,\mathcal{O}(\bfN))/H^{0}(V,\mathcal{O}(\bfE))\ar[ur]^{\xi} &
}}\tag{7.25}\label{art08-sec7-eq7.25}
\end{equation*}
where $\zeta$ is now $\dfrac{i}{2\pi}$ (identity). Let $\omega\in H^{n,n-1}(V)$ and $\eta\in H^{1}(V,\mathcal{O})\cong H^{0,1}(V)$. To prove the commutativity of \eqref{art08-sec7-eq7.25}, we must show:
\begin{equation*}
\frac{i}{2\pi}\int\limits_{V}\eta\wedge\omega=\int\limits_{Z}\theta(\eta)\wedge \xi^{*}\omega.\tag{7.26}\label{art08-sec7-eq7.26}
\end{equation*}

The\pageoriginale argument is now similar to the proof \eqref{art08-sec3-eq3.10}. Letting $T_{\epsilon}$ be an $\epsilon$-tubular neighborhood of $Z$ in $V$, $\int\limits_{V}\eta\wedge \omega=\lim\limits_{\epsilon\to 0}\int\limits_{V-T_{\epsilon}}\eta\wedge \omega$. On the other hand, $\eta\sigma=\overline{\partial}\tau$ for some $\tau\in \Gamma_{\infty}(V,\bfE)$, and $\theta(\eta)=\tau|Z\in H^{0}(Z,\mathcal{O}(\bfE))$. On $V-T_{\epsilon}$, $\eta\wedge \omega=\eta\sigma\wedge\dfrac{\omega}{\sigma}=\overline{\partial}\left(\dfrac{\tau}{\sigma}\wedge\omega\right)=d\left(\dfrac{\tau}{\sigma}\wedge\omega\right)$, and so $\lim\limits_{\epsilon\to 0}\int\limits_{V-T_{\epsilon}}\eta\wedge\omega=\lim\limits_{\epsilon\to 0}-\int\limits_{\partial T_{\epsilon}}\dfrac{\tau\omega}{\sigma}=\dfrac{2\pi}{i}\int\limits_{Z}\tau\xi^{*}\omega$ by the same argument as used to prove \eqref{art08-sec3-eq3.10}.

\medskip
\noindent
{\bf Corollary \thnum{7.27}.\label{art08-coro7.27}}~\eqref{ART08-SEC6-EQ6.4} {\em holds whenever $\bfE\to V$ is restricted to have the triangular group of matrices as structure groups.}


\begin{proof}
Use \eqref{art08-sec7-eq7.21} and what we have just proved about line bundles.
\end{proof}

\section{Proof of \texorpdfstring{\eqref{ART08-SEC6-EQ6.4}}{eq6.4} for the Highest Chern Class.}\label{art08-sec8}

Let $\bfE\to \bfV$ be an ample holomorphic vector bundle (c.f. \eqref{art08-sec7-eq7.14}) with fibre $\bfC^{k}$ and such that $H^{1}(V,\mathcal{O}(\bfE))=0$. The diagram \eqref{art08-sec7-eq7.19} then becomes, for $q=k$,
\begin{equation*}
\vcenter{\xymatrix{
H^{1}(V,\mathcal{O}(\Hom(\bfE,\bfE)))\ar[dd]^-{\theta}\ar[dr]^-{\zeta}\\
               & H^{k-1,k}(V).\\
H^{0}(Z,\mathcal{O}(\bfN))/rH^{0}(V,\mathcal{O}(\bfE))\ar[ur]_{\xi^{-}} &
}}\tag{8.1}\label{art08-sec8-eq8.1}
\end{equation*}
Let $\eta\in H^{1}(V,\mathcal{O}(\Hom(\bfE,\bfE)))$ be given by a global $\Hom(\bfE,\bfE)$-valued $(0,1)$ form $\eta$ and suppose $\sigma\in H^{0}(V,\mathcal{O}(\bfE))$ is such that $Z=\{\sigma=0\}$ and $\eta\cdot \sigma=0$ in $H^{1}(V,\mathcal{O}(\bfE))$. Then $\eta\cdot \sigma=\overline{\partial}\tau$ where $\tau$ is a $C^{\infty}$ section of $\bfE\to V$, and $\tau|Z=\theta(\eta)$. If $\omega\in H^{n-k+1,n-k}(V)$ and $\Theta$ is a curvature in $\bfE$, then we need to show that
\begin{equation*}
\int\limits_{Z}\xi^{*}\omega\cdot \tau = \int\limits_{V}kP_{k}(\underbrace{\Theta}_{k-1};\eta)\wedge\omega,\tag{8.2}\label{art08-sec8-eq8.2}
\end{equation*}
where $\xi^{*}\omega\in H^{n-k}(Z,\Omega^{n-k}(\bfN^{*}))$ is the {\em Poincar\'e residue operator} \eqref{art08-sec3-eq3.6}.

What\pageoriginale we will do is write, on $V-Z$, $kP_{k}(\underbrace{\Theta}_{k-1};\eta)=\overline{\partial}\psi_{k}$ where $\psi_{k}$ is a $C^{\infty}(k-1,k-1)$ form. Then, if $T_{\epsilon}$ is the tubular $\epsilon$-neighborhood of $Z$ in $V$, $\int\limits_{V}kP_{k}(\underbrace{\Theta}_{k-1};\eta)\wedge\omega=\lim\limits_{\epsilon\to 0}\int\limits_{V-T_{\epsilon}}d(\psi_{k}\wedge\omega)=-\lim\limits_{\epsilon\to 0}\int\limits_{\partial T_{\epsilon}}\psi_{k}\wedge \omega$. We will then show, by a residue argument, that $-\lim\limits_{\epsilon\to 0}\int\limits_{\partial T_{\epsilon}}\psi_{k}\wedge\omega=\int\limits_{Z}\xi^{*}\omega\cdot \tau$.

Suppose now that we have an Hermitian metric in $\bfE\to V$. This metric determines a $(1,0)$ connection $\theta$ in $\bfE$ with curvature $\Theta=\overline{\partial}\theta$. Let $\sigma^{*}$ on $V-Z$ be the $C^{\infty}$ section of $\bfE^{*}|V-Z$ which is dual to $\sigma$ (using the metric). Setting $\lambda=\tau\otimes \sigma^{*}$, $\widehat{\eta}=\eta-\overline{\partial}\lambda$ is $C^{\infty}(0,1)$ form with values in $\Hom(\bfE,\bfE)|V-Z$ and $\widehat{\eta}\cdot \sigma=\eta\cdot \sigma-\overline{\partial}(\tau\otimes \sigma^{*}\cdot \sigma)=\eta\cdot \sigma-\overline{\partial}\tau\equiv 0$.

On the other hand, we will find a $C^{\infty}(1,0)$ form $\gamma$ on $V-Z$, which has values in $\Hom(\bfE,\bfE)$, and is such that $D\sigma=\gamma\cdot \sigma$. Then $\widehat{\theta}=\theta-\gamma$ gives a $C^{\infty}$ connection in $\bfE|V-Z$ whose curvature $\widehat{\Theta}=\Theta-\overline{\partial}\gamma$ satisfies $\widehat{\Theta}\cdot \sigma\equiv 0$. Since $kP_{k}(\underbrace{\widehat{\Theta}}_{k-1};\widehat{\eta})\equiv 0$ (because $\widehat{\Theta}\cdot \sigma\equiv 0\equiv \widehat{\eta}\cdot \sigma)$, it is clear that, on $V-Z$,
$$
kP_{k}(\underbrace{\Theta}_{k-1};\eta)=kP_{k}(\underbrace{\widehat{\Theta}+\overline{\partial}\gamma}_{k-1};\widehat{\eta}+\overline{\partial}\lambda)=\overline{\partial}\psi_{k},
$$
and this will be our desired form $\psi_{k}$. Having found $\psi_{k}$ explicitly, we will carry out the integrations necessary to prove \eqref{art08-sec8-eq8.2}.

(a)~ \textsc{An integral formula in unitary Geometry.} On $\widehat{\bfC}^{k}=\bfC^{k}-\{0\}$, we consider frames $(z;e_{1},\ldots,e_{k})$ where $z\in \widehat{\bfC}^{k}$ and $e_{1},\ldots,e_{k}$ is a unitary frame with $e_{1}=\dfrac{z}{|z|}$. Using the calculus of frames as in \cite{art08-key5}, we have : $De_{\rho}=\sum\limits^{k}_{\sigma=1}\theta^{\sigma}_{\rho}e_{\sigma}(\theta^{\rho}_{\sigma}+\overline{\theta}^{\sigma}_{\rho}=0)$. In particular, the differential forms $\theta^{\rho}_{1}=(De_{1},e_{\rho})$ are {\em horizontal forms} in the frame bundle over $\widehat{\bfC}^{k}$. Since $0=\overline{\partial}z=D^{n}(|z|e_{1})=\overline{\partial}|z|e_{1}+|z|\left(\sum\limits^{k}_{\rho=1}\theta^{\rho''}_{1}e_{\rho}\right)$, we find that\pageoriginale $\theta^{\alpha''}_{1}=0(\alpha=2,\ldots,k)$ and $\theta^{1''}_{1}=-\dfrac{\overline{\partial}|z|}{|z|}=-\overline{\partial}\log |z|$. It follows that $\theta^{1'}_{\alpha}=0(\alpha=2,\ldots,k)$ and $\theta^{1'}_{1}=\partial \log |z|$, so that
$$
\theta^{1}_{1}=(\partial-\overline{\partial})\cdot \log |z|.
$$

Given a frame $(z;e_{1},\ldots,e_{k})$, the $e_{\rho}$ give a basis for the $(1,0)$ tangent space to $\widehat{\bfC}^{k}$ at $z$. Thus there are $(1,0)$ forms $\omega^{1},\ldots,\omega^{k}$ dual to $e_{1},\ldots,e_{k}$, and we claim that
\begin{equation*}
\left.
\begin{array}{l}
\omega^{1}=2|z|\theta^{1'}_{1}\\
\omega^{\alpha}=|z|\theta^{\alpha}_{1}(\alpha=2,\ldots,k)
\end{array}.\tag{8.3}\label{art08-sec8-eq8.3}
\right\}
\end{equation*}

\begin{proof}
By definition $dz=\sum\limits^{k}_{\rho=1}\omega^{\rho}e_{\rho}$. But $z=|z|e_{1}$ and so $dz=(\partial |z|+|z|\theta^{1'}_{1})e_{1}+\sum\limits^{k}_{\alpha=2}|z|\theta^{\alpha}_{1}e_{\alpha}$. Since $\theta^{1'}_{1}=\dfrac{\partial|z|}{|z|}$, we get \eqref{art08-sec8-eq8.3} by comparing both sides of the equation :
$$
\sum\limits^{k}_{\rho=1}\omega^{\rho}e_{\rho}=2|z|\theta^{1'}_{1}e_{1}+\sum\limits^{k}_{\alpha=2}|z|\theta^{\alpha}_{1}e_{\alpha}.
$$

Now let $\tau=\sum\limits^{k}_{\rho=1}\tau^{\rho}e_{\rho}$ and $\omega=\sum\limits^{k}_{\rho=1}\xi_{\rho}\omega^{\rho}$ be respectively a smooth vector field and $(1,0)$ form on $\bfC^{k}$. Then $\langle \omega,\tau\rangle=\sum\limits^{k}_{\rho=1}\xi_{\rho}\tau^{\rho}$ is a $C^{\infty}$ function on $\bfC^{k}$. We want to construct a $(k-1,k-1)$ form $\widehat{\psi}_{k}(\tau)$ on $\widehat{\bfC}^{k}$ such that
\begin{equation*}
\langle \omega,\tau\rangle_{0}=\lim\limits_{\epsilon\to 0}\int\limits_{\partial B_{\epsilon}}\widehat{\psi}_{k}(\tau)\wedge \omega,\tag{8.4}\label{art08-sec8-eq8.4}
\end{equation*}
where $B_{\epsilon}\subset \bfC^{k}$ is the ball of radius $\epsilon$. Let $\Gamma(k)$ be the reciprocal of the area of the unit $2k-1$ sphere in $\bfC^{k}$ and set 
\begin{equation*}
\widehat{\psi}_{k}(\tau)=\dfrac{\Gamma(k)}{|z|}\left\{\dfrac{\tau^{1}}{2}\prod\limits^{k}_{\alpha=1}\theta^{\alpha}_{1}\theta^{1}_{\alpha}+\sum\limits^{k}_{\beta=2}\tau^{\beta}/\theta^{1'}_{1}\prod\limits_{\alpha\neq \beta}\theta^{\alpha}_{1}\theta^{1}_{\alpha}\right\}.\tag{8.5}\label{art08-sec8-eq8.5}
\end{equation*}

What we claim is that {\em $\widehat{\psi}_{k}(\tau)$, as defined by} \eqref{art08-sec8-eq8.5}, {\em is a $(k-1,k-1)$ form on $\widehat{\bfC}^{k}$ satisfying} \eqref{art08-sec8-eq8.4}. 
\end{proof}

\begin{proof}
It\pageoriginale is easy to check that $\widehat{\psi}_{k}(\tau)$ is a scalar $C^{\infty}$ form on $\widehat{\bfC}^{k}$ of type $(k-1,k-1)$. Now $\omega=\sum\limits^{k}_{\rho=1}\xi_{\rho}\omega^{\rho}=|z|(2\xi_{1}\theta^{1'}_{1}+\sum\limits^{k}_{\alpha=2}\xi_{\alpha}\theta^{\alpha}_{1})$, and so
\begin{equation*}
\omega\wedge \widehat{\psi}_{k}(\tau)=\Gamma(k)\left\{\sum\limits^{k}_{\rho=1}\tau^{\rho}\xi_{\rho}\right\}\left\{\theta^{1'}_{1}\prod\limits^{k}_{\alpha=2}\theta^{\alpha}_{1}\theta^{1}_{\alpha}\right\}.\tag{8.6}\label{art08-sec8-eq8.6}
\end{equation*}
Using \eqref{art08-sec8-eq8.6}, we must show : If $f$ is a $C^{\infty}$ function of $\bfC^{k}$, then
\begin{equation*}
\lim\limits_{\epsilon\to 0}\Gamma(k)\int\limits_{\delta B_{\epsilon}}f(z)\theta^{1'}_{1}\prod\limits^{k}_{\alpha=2}\theta^{\alpha}_{1}\theta^{1}_{\alpha}=f(0).\tag{8.7}\label{art08-sec8-eq8.7}
\end{equation*}
But, by \eqref{art08-sec8-eq8.3}, $\omega^{1}\omega^{2}\overline{\omega}^{2}\ldots\omega^{k}\overline{\omega}^{k}=2|z|^{2k-1}\theta^{1'}_{1}\theta^{2}_{1}\theta^{1}_{2}\ldots\theta^{k}_{1}\theta^{1}_{k}$, and so\break $\Gamma(k)\theta^{1'}_{1}\prod\limits^{k}_{\alpha=2}\theta^{\alpha}_{1}\theta^{1}_{\alpha}$ is a $2k-1$ form on $\widehat{\bfC}^{k}$ having constant surface integral one over all spheres $\partial B_{\epsilon}$ for all $\epsilon$. From this we get \eqref{art08-sec8-eq8.7}.
\end{proof}

\begin{remarks*}
\begin{itemize}
\item[(i)] Using coordinates $z=\left[\begin{smallmatrix} z^{1}\\ \vdots \\ z^{k}\end{smallmatrix}\right]$, 
\begin{equation*}
\theta^{1'}_{1}\theta^{2}_{1}\theta^{1}_{2}\ldots\theta^{k}_{1}\theta^{1}_{k}=\sum\limits^{k}_{\rho=1}\frac{(-1)^{\rho-1+k}\overline{z}^{\rho}dz^{1}\ldots dz^{k}d\overline{z}^{1}\ldots d\widehat{\overline{z}^{\rho}}\ldots d\overline{z}^{k}}{|z|^{2k}}.\tag{8.8}\label{art08-sec8-eq8.8}
\end{equation*}

\item[(ii)] If $\mu$ is a $C^{\infty}$ differential form on $\widehat{\bfC}^{k}$ which becomes infinite at zero at a slower rate than $\widehat{\psi}_{k}(\tau)$, then
\begin{equation*}
\lim\limits_{\epsilon\to 0}\int\limits_{\partial B_{\epsilon}}\omega \wedge \widehat{\psi}_{k}(\tau)=\lim\limits_{\epsilon\to 0}\int\limits_{\partial B_{\epsilon}}\omega\wedge (\widehat{\psi}_{k}(\tau)+\mu).\tag{8.9}\label{art08-sec8-eq8.9}
\end{equation*}

\item[(iii)] On $\bfC^{l}\times \bfC^{k}$, let $\omega$ be a $C^{\infty}$ form of type $(l+1,l)$. Then, if $\tau=\sum\limits^{k}_{\rho=1}\tau^{\rho}e_{\rho}$ is a $C^{\infty}$ vector on $\bfC^{k}$, we may write $\omega=\sum\limits^{k}_{\rho=1}\gamma_{\rho}\wedge \omega^{\rho}$ where the $C^{\infty}$ form $\xi^{*}\omega\cdot \tau=\sum\limits_{\rho=1}^{k}\tau^{\rho}\gamma_{\rho}$ is of type $(l,l)$ on $\bfC^{l}$ and is uniquely determined by $\omega$ and $\tau$.

Suppose now that $\omega$ has compact support in $\bfC^{l}$ (i.e. is supported in $\Delta^{l}\times \bfC^{k}$ for some polycylinder $\Delta^{l}\subset \bfC^{k}$). Then, as a generalization of \eqref{art08-sec8-eq8.4}, we have
\begin{equation*}
\int\limits_{\bfC^{l}}\xi^{*}\omega\cdot \tau=\lim\limits_{\epsilon\to 0}\int\limits_{\bfC^{l}\times \partial B^{k}_{\epsilon}}\widehat{\psi}_{k}(\tau)\wedge \omega.\tag{8.10}\label{art08-sec8-eq8.10}
\end{equation*}\pageoriginale

Note that $\xi^{*}\omega=\left[\begin{smallmatrix} \gamma_{1}\\ \vdots\\ \gamma_{k}\end{smallmatrix}\right]$ is here the {\em Poincar\'e residue} of $\omega$ on $\bfC^{l}\times \{0\}\subset \bfC^{l}\times \bfC^{k}$.

\item[(iv)] Combining remarks (ii) and (iii) above, we have :
\begin{equation*}
\begin{array}{l}
\text{Let $\omega$ be a $C^{\infty}$ form of type $(l+1,l)$ with support}\\
\text{in $\Delta^{l}\times \bfC^{k}$ and let $\psi_{k}(\tau)$ be a $C^{\infty}$ form on $\bfC^{l}\times \widehat{\bfC}^{k}$}\\
\text{whose {\em principal part} (i.e. term with the highest}\\
\text{order pole on $\bfC^{l}\times \{0\}$) is $\widehat{\psi}_{k}(\tau)$ given by \eqref{art08-sec8-eq8.5}. Then}
\end{array}\tag{8.11}\label{art08-sec8-eq8.11}
\end{equation*}
\begin{equation*}
\int\limits_{\bfC^{l}}\xi^{*}\omega\cdot \tau=\lim\limits_{\epsilon\to 0}\int\limits_{\bfC^{l}\times \delta B_{\epsilon}}\omega\wedge \psi_{k}(\tau).\tag{8.12}\label{art08-sec8-eq8.12}
\end{equation*}
\end{itemize}
\end{remarks*}

(b)~ \textsc{Some Formulae in Hermitian Geometry.} Let $W$ be a complex manifold and $\bfE\to W$ a holomorphic, Hermitian vector bundle with fibre $\bfC^{k}$. We suppose that $\bfE$ has a {\em non-vanishing} holomorphic section $\sigma$ and we let $\bfS$ be the trivial line sub-bundle of $\bfE$ generated by $\sigma$. Thus we have over $W$
\begin{equation*}
0\to \bfS\to \bfE\to \bfQ\to 0.\tag{8.13}\label{art08-sec8-eq8.13}
\end{equation*}
We consider unitary frames $e_{1},\ldots,e_{k}$ where $e_{1}=\dfrac{\sigma}{|\sigma|}$ is the unit vector in $\bfS$. The metric connection in $\bfE$ gives a covariant differentiation $De_{\rho}=\sum\limits_{\sigma}\theta^{\sigma}_{\rho}e_{\sigma}(\theta^{\sigma}_{\rho}+\overline{\theta}^{\rho}_{\sigma}=0)$ with $D''=\overline{\partial}$. From $0=\overline{\partial}\sigma=D''(|\sigma|e_{1})=(\overline{\partial}|\sigma|+|\sigma|\theta^{1''}_{1})e_{1}+|\sigma|\left(\sum\limits_{\alpha=2}^{k}\theta^{\alpha''}_{1}e_{\alpha}\right)$, we find
\begin{equation*}
\theta^{\alpha''}_{1}=0(\alpha=2,\ldots,k),\theta^{1}_{1}=(\partial-\overline{\partial})\log |\sigma|.\tag{8.14}\label{art08-sec8-eq8.14}
\end{equation*}

Now then $D\sigma=D'\sigma=(\partial |\sigma|+|\sigma|\theta^{1'}_{1})e_{1}+|\sigma|\left(\sum\limits^{k}_{\alpha=2}\theta^{\alpha}_{1}e_{\alpha}\right)=|\sigma|\left\{2\theta^{1'}_{1}e_{1}+\sum\limits^{k}_{\alpha=2}\theta^{\alpha}_{1}e_{\alpha}\right\}=\gamma\cdot \sigma$ where
\begin{equation*}
\gamma=2\theta^{1'}_{1}e_{1}\otimes e^{*}_{1}+\sum\limits^{k}_{\alpha=2}\theta^{\alpha}_{1}e_{\alpha}\otimes e^{*}_{1}\tag{8.15}\label{art08-sec8-eq8.15}
\end{equation*}
is\pageoriginale a global $(1,0)$ form with values in $\Hom(\bfE,\bfE)$. In terms of matrices,
\begin{equation*}
\gamma=
\begin{bmatrix}
2\theta^{1'}_{1} & 0\ldots 0\\
\theta^{2}_{1} & 0\ldots 0\\
\vdots & \\
\theta^{k}_{1} & 0\ldots 0
\end{bmatrix}\tag{8.16}\label{art08-sec8-eq8.16}
\end{equation*}
Letting $\widehat{\theta}=\theta-\gamma$, we get a $(1,0)$ connection $\widehat{D}$ in $\bfE$ with $\widehat{D}''=\overline{\partial}$ and $\widehat{D}\sigma=0$. This was one of the ingredients in the construction of $\psi_{k}$ outlined above.

For later use, we need to compute $\overline{\partial}\gamma=D''\gamma$. Since $\gamma$ is of type $(1,0)$, $D''\gamma$ is the $(1,1)$ part of $D\gamma=d\gamma+\theta\wedge\gamma+\gamma\wedge \theta$. Also, we won't need the first column of $D''\gamma$, so we only need to know $(D\gamma)^{\rho}_{\alpha}=\sum\limits_{\tau}\theta^{\rho}_{\tau}\gamma^{\tau}_{\alpha}+\sum\limits_{\tau}\gamma^{\rho}_{\tau}\theta^{\tau}_{\alpha}$ (since $\gamma^{\rho}_{\alpha}=0$) $=\gamma^{\rho}_{1}\theta^{1}_{\alpha}$. This gives the formula :
\begin{equation*}
\delta\gamma =
\begin{bmatrix}
* & 2\theta^{1'}_{1} & \theta^{1}_{2} & \ldots & 2\theta^{1'}_{1} & \theta^{1}_{k}\\
\vdots & \theta^{2}_{1} & \theta^{1}_{2} & \ldots & \theta^{2}_{1} & \theta^{1}_{k}\\
 & \vdots & & & \vdots & \\
* & \theta^{k}_{1} & \theta^{1}_{2} & \ldots & \theta^{k}_{1} & \theta^{1}_{k}
\end{bmatrix}.\tag{8.17}
\label{art08-sec8-eq8.17}
\end{equation*}

As another part of the construction of $\psi_{k}$ with $\overline{\partial}\psi_{k}=kP_{k}(\underbrace{\Theta}_{k-1};\eta)$, we let $\tau=\sum\limits^{k}_{\rho=1}\tau^{\rho}e_{\rho}$ be a $C^{\infty}$ section of $\bfE$ with $\overline{\partial}\tau=\eta\cdot \sigma$ (c.f. below \eqref{art08-sec8-eq8.1}). Set
\begin{equation*}
\lambda = \tau\otimes \sigma^{*}=\sum\limits^{k}_{\rho=1}\dfrac{\tau^{\rho}}{|\sigma|}e_{\rho}\otimes e^{*}_{1}.\tag{8.18}\label{art08-sec8-eq8.18}
\end{equation*}
In terms of matrices,
\begin{equation*}
\lambda = \dfrac{1}{|\sigma|}
\begin{bmatrix}
\tau^{1} & 0 & \ldots & 0\\
\vdots & \vdots & & \vdots\\
\tau^{k} & 0 & \ldots & 0
\end{bmatrix}.\tag{8.19}\label{art08-sec8-eq8.19}
\end{equation*}
\eject
\noindent
We want to compute the $\Hom(\bfE,\bfE)$-valued $(0,1)$ form $\partial \lambda$, and we claim that
\begin{equation*}
\overline{\partial}\lambda = \sum\limits^{k}_{\rho=1}\eta^{\rho}_{1}e_{\mu}\otimes e^{*}_{1}-\dfrac{1}{|\sigma|}\sum\limits_{\rho,\alpha}\tau^{\rho}\theta^{1}_{\alpha}e_{\rho}\otimes e^{*}_{\alpha}.\tag{8.20}\label{art08-sec8-eq8.20}
\end{equation*}\pageoriginale 

\begin{proof}
\begin{align*}
\overline{\partial}\gamma &= \overline{\partial}\tau \otimes \sigma^{*}+\tau\otimes \partial \overline{\sigma}^{*}\\
&= \sum\limits^{k}_{\rho=1}\eta^{\rho}_{1}e_{\rho}\otimes e^{*}_{1}+\sum\limits^{k}_{\rho=1}\tau^{\rho}e_{\rho}\otimes D''\left(\dfrac{e^{*}_{1}}{|\sigma|}\right)
\end{align*}
(since $\dfrac{e^{*}_{1}}{|\sigma|}=\sigma^{*}$ and $\overline{\partial}\tau=\eta\cdot \sigma$). Now $D''\left(\dfrac{e^{*}_{1}}{|\sigma|}\right)=\overline{\partial}\left(\dfrac{1}{|\sigma|}\right)e^{*}_{1}+\dfrac{1}{|\sigma|}D''e^{*}_{1}$ and $D''e^{*}_{1}=\sum\limits^{k}_{\rho=1}\theta^{*\rho''}_{1}e^{*}_{\rho}=-\sum\limits^{k}_{\rho=1}\theta^{1''}_{\rho}e^{*}_{\rho}$ (since $\theta+{}^{t}\theta^{*}=0$). But $\overline{\partial}\left(\dfrac{1}{|\sigma|}\right)-\dfrac{1}{|\sigma|}\theta^{1''}_{1}=\dfrac{1}{|\sigma|}(-\overline{\\partial}\log |\sigma|-\theta^{1''}_{1})=0$ by \eqref{art08-sec8-eq8.14} so that $D''\left(\dfrac{e^{*}_{1}}{|\sigma|}\right)=\dfrac{-1}{|\sigma|}\sum\limits^{k}_{\alpha=2}\theta^{1}_{\alpha}e^{*}_{\alpha}$ (since $\theta^{1'}_{\alpha}=0$ by \eqref{art08-sec8-eq8.14}). Thus $\overline{\partial}\lambda=\sum\limits^{k}_{\rho=1}\eta^{\rho}_{1}e_{\rho}\otimes e^{*}_{1}-\dfrac{1}{|\sigma|}\sum\limits_{\rho,\alpha}\tau^{\rho}\theta^{1}_{\alpha}e_{\rho}\otimes e^{*}_{\alpha}$ as required.

In terms of matrices,
\begin{equation*}
\overline{\partial}\lambda=-
\begin{bmatrix}
\eta^{1}_{1} & \dfrac{\tau^{1}\theta^{1}_{2}}{|\sigma|} & \ldots & \dfrac{\tau^{1}\theta^{1}_{k}}{|\sigma|}\\
\vdots & \vdots & & \vdots\\
\eta^{k}_{1} & \dfrac{\tau^{k}\theta^{1}_{2}}{|\sigma|} & \ldots & \dfrac{\tau^{k}\theta^{1}_{k}}{|\sigma|}
\end{bmatrix}
=-\dfrac{1}{|\sigma|}
\begin{bmatrix}
* & \tau^{1} & \theta^{1}_{2} & \ldots & \tau^{1} & \theta^{1}_{k}\\
\vdots & & \vdots & & & \vdots\\
* & \tau^{k} & \theta^{1}_{2} & \ldots & \tau^{k} & \theta^{1}_{k}
\end{bmatrix}.\tag{8.21}\label{art08-sec8-eq8.21}
\end{equation*}
\end{proof}

(c)~ \textsc{Completion of the Proof.} Given $\bfE\to V$ and $\sigma\in H^{0}(V,\mathcal{O}(\bfE))$ with $Z=\{z\in V : \sigma(z)=0\}$, we let $\widehat{\bfE}=\bfE-\{0\}$ and lift $\bfE$ up to lie over $\widehat{\bfE}$. Letting $W=\widehat{\bfE}$, the considerations in \ref{art08-sec8}(b) above apply, as well as the various formulae obtained there. Using $\sigma:V-Z\to \widehat{\bfE}$, we may pull everything back down to $V-Z$. In particular, $|\sigma|$ may be thought of as the distance to $Z$, and the forms $\theta^{\rho}_{\alpha}$ will go to infinity like $\dfrac{1}{|\sigma|}$ near $Z$ (c.f. \ref{art08-sec8}(a)).

Now, since $\Theta\cdot \sigma =\overline{\partial}\gamma\cdot \sigma$ and $\eta\cdot \sigma=\overline{\partial}\gamma\cdot \sigma$ on $V-Z$ we have
\begin{equation*}
0\equiv kP_{k}(\underbrace{\Theta-\overline{\partial}\gamma}_{k-1};\eta-\overline{\partial}\lambda).\tag{8.22}\label{art08-sec8-eq8.22}
\end{equation*}
Expanding\pageoriginale \eqref{art08-sec8-eq8.22} out, we will have
\begin{equation*}
kP_{k}(\underbrace{\Theta}_{k-1}; \eta)=\overline{\partial}\psi_{k},\tag{8.23}\label{art08-sec8-eq8.23}
\end{equation*}
on $V-Z$. It is clear that $\psi_{k}$ will be a polynomial with terms containing $\Theta^{\rho}_{\sigma}$, $\eta^{\rho}_{\sigma}$, $\theta^{\rho}$, $\tau^{j}$. Furthermore, from \eqref{art08-sec8-eq8.16}, \eqref{art08-sec8-eq8.17}, and \eqref{art08-sec8-eq8.21}, the highest order term of $\psi_{k}$ will become infinite near $Z$ like $\dfrac{1}{|\sigma|^{2k-1}}$. From \eqref{art08-sec8-eq8.5} and \eqref{art08-sec8-eq8.11}, the expression
\begin{equation*}
-\lim\limits_{\epsilon\to 0}\int\limits_{\partial T_{\epsilon}}\omega\wedge \psi_{k}\tag{8.24}\label{art08-sec8-eq8.24}
\end{equation*}
will depend only on this highest order part of $\psi_{k}$. Let us use the notation $\equiv$ to symbolize ``ignoring terms of order $\dfrac{1}{|\sigma|^{2k-2}}$ or less.'' Then from \eqref{art08-sec8-eq8.22}, we have
\begin{equation*}
\psi_{k}\equiv (-1)^{k}kP_{k}(\underbrace{\gamma}_{1}; \underbrace{\overline{\partial}\gamma}_{k-2}; \underbrace{\overline{\partial}\lambda}_{1}).\tag{8.25}\label{art08-sec8-eq8.25}
\end{equation*}
This is because $\Theta$ and $\eta$ are smooth over $Z$. Note that the right hand side of \eqref{art08-sec8-eq8.25} behaves as $\dfrac{1}{|\sigma|^{2k-1}}$ near $Z$. Using \eqref{art08-sec8-eq8.5} and \eqref{art08-sec8-eq8.11} from \ref{art08-sec8}(a), to prove the commutativity of \eqref{art08-sec8-eq8.1}, we must show:
\begin{gather*}
(-1)^{k}kP_{k}(\underbrace{\gamma}_{1}; \underbrace{\overline{\partial}\gamma}_{k-2}; \underbrace{\overline{\partial}\lambda}_{1})\\
\equiv - \dfrac{\Gamma(k)}{2|\sigma|}\left\{\tau^{1}\prod\limits^{k}_{\alpha=2}\theta^{\alpha}_{1}\theta^{1}_{\alpha}+2\sum\limits^{k}_{\beta=2}\tau^{\beta}\theta^{1}_{\beta}\theta^{1'}_{1}\prod\limits_{\alpha\neq \beta}\theta^{\alpha}_{1}\theta^{1}_{\alpha}\right\}\tag{8.26}\label{art08-sec8-eq8.26}
\end{gather*}
where $\gamma$, $\overline{\partial}\gamma$, and $\overline{\partial}\lambda$ are given by \eqref{art08-sec8-eq8.16}, \eqref{art08-sec8-eq8.17}, and \eqref{art08-sec8-eq8.21}.

The left hand side of \eqref{art08-sec8-eq8.26} is, by \eqref{art08-sec4-eqA4.4}
{{\fontsize{9pt}{11pt}\selectfont
\begin{equation*}
\left(\dfrac{1}{2\pi i}\right)\dfrac{1}{(k-1)!}\left\{\dfrac{1}{|\sigma|}\sum\limits^{k}_{\alpha=2}\det 
\begin{bmatrix}
2\theta^{1'}_{1} & 2\theta^{1'}_{1} & \theta^{1}_{2} & \ldots & \tau^{1}\theta^{1}_{\alpha} & \ldots & 2\theta^{1'}_{1} & \theta^{1}_{k}\\
\theta^{2}_{1} & \theta^{2}_{1} & \theta^{1}_{2} &  & \tau^{2}\theta^{1}_{\alpha} & & \theta^{2}_{1} & \theta^{1}_{k}\\
\vdots & & \vdots & & \vdots & & &\\
\theta^{k}_{1} & \theta^{k}_{1} & \theta^{1}_{2} & \ldots & \tau^{k}\theta^{1}_{\alpha} & \ldots & \theta^{k}_{1} & \theta^{1}_{k}\\
\end{bmatrix}\right\}\tag{8.27}\label{art08-sec8-eq8.27}
\end{equation*}}\relax}
Fixing\pageoriginale $\alpha$, the coefficient of $\tau^{1}\theta^{1}_{\alpha}$ on the right hand side of \eqref{art08-sec8-eq8.27} is
$$
(-1)^{\alpha}\det
\begin{bmatrix}
\theta^{2}_{1} & \theta^{2}_{1} & \theta^{1}_{2} & \ldots & \widehat{\alpha} & \ldots & \theta^{2}_{1} & \theta^{1}_{k}\\
\vdots & \vdots & & & & & \vdots & \\
\theta^{k}_{1} & \theta^{k}_{1} & \theta^{1}_{2} & & \ldots & & \theta^{k}_{1} & \theta^{1}_{k}
\end{bmatrix}
$$
($\widehat{\alpha}$ means that the column beginning $\theta^{2}_{1}\theta^{1}_{\alpha}$ is deleted). This last determinant is evaluated as
\begin{equation*}
\sum\limits_{\pi}\text{ sgn } \pi\theta^{\pi_{1}}_{1}\theta^{\pi_{2}}_{1}\theta^{1}_{2}\ldots(\widehat{\alpha})\theta^{\pi k-1}_{1}\theta^{1}_{k},\tag{8.28}\label{art08-sec8-eq8.28}
\end{equation*}
where the sum is over all permutations of $2,\ldots, k$. Obviously then \eqref{art08-sec8-eq8.28} is equal to $(k-1)!(-1)^{\alpha-1}\theta^{\alpha}_{1}\prod\limits_{\beta\neq \alpha}\theta^{\beta}_{1}\theta^{1}_{\beta}$. This then gives for the coefficient of $\tau^{1}$ in \eqref{art08-sec8-eq8.27} the term
\begin{equation*}
-\left(\dfrac{1}{2\pi i}\right)^{k}\dfrac{(k-1)}{|\sigma|}\prod\limits^{k}_{\alpha=2}\theta^{\alpha}_{1}\theta^{1}_{\alpha}.\tag{8.29}\label{art08-sec8-eq8.29}
\end{equation*}
In \eqref{art08-sec8-eq8.27}, the term containing $\tau^{\alpha}\theta^{1}_{\beta}$ is 
\begin{gather*}
\beta\\
(-1)^{\alpha+\beta}\det_{\alpha>\beta}
\begin{bmatrix}
2\theta^{1'}_{1} & 2\theta^{1'}_{1} & \theta^{1}_{2} & \ldots & 2\theta^{1'}_{1} & \theta^{1}_{k}\\
\theta^{2}_{1} & \theta^{2}_{1} & \theta^{1}_{2} & & \theta^{2}_{1} & \theta^{1}_{k}\\
 & \vdots & & & \vdots & \\
\theta^{k}_{1} & \theta^{k}_{1} & \theta^{1}_{2} & \ldots & \theta^{k}_{1} & \theta^{1}_{k}
\end{bmatrix}\\
=2(k-1)!\theta^{\beta}_{1}\theta^{1'}_{1}\left(\prod\limits_{\gamma\neq \beta,\alpha}\theta^{\gamma}_{1}\theta^{1}_{\gamma}\right)\theta^{1}_{\alpha}.
\end{gather*}

Thus, combining, \eqref{art08-sec8-eq8.27} is evaluated to be
$$
-\dfrac{\Gamma(k)}{2|\sigma|}\left\{\tau^{1}\prod\limits^{k}_{\alpha=2}\theta^{\alpha}_{1}\theta^{1}_{\alpha}+2\theta^{1'}_{1}\sum\limits^{k}_{\beta=2}\tau^{\beta}\theta^{1}_{\beta}\prod\limits_{\alpha\neq \beta}\theta^{\alpha}_{1}\theta^{1}_{\alpha}\right\}
$$
where $\Gamma(k)^{-1}=\int\limits_{\alpha B_{1}}\omega^{1}\omega^{2}\overline{\omega}^{2}\ldots\overline{\omega}^{k}\omega^{k}$ in \ref{art08-sec8}(a). Comparing with \eqref{art08-sec8-eq8.26} we obtain our theorem.

\section{Proof of \texorpdfstring{\eqref{ART08-SEC6-EQ6.8}}{eq6.8} for the General Chern Classes.}\label{art08-sec9}

The argument given in section \ref{art08-sec8} above will generalize to an arbitrary {\em nonsingular} Chern\pageoriginale class $Z_{q}(\bfE)$. The computation is similar to, but more complicated than, that given in \S\ref{art08-sec8}, (a)-(c) above. However, in general $Z_{q}(\bfE)$ will have singularities, no matter how ample $\bfE$ is. Thus the {\em normal bundle} $\bfN\to Z_{q}(\bfE)$ is {\em not} well-defined, and so neither the infinitesimal variation formula \eqref{art08-sec3-eq3.8} nor \eqref{art08-sec7-eq7.16} makes sense as it now stands.

We shall give two proofs of \eqref{ART08-SEC6-EQ6.8}. The first and more direct argument makes use of the fact that the singularities of $Z_{q}(\bfE)$ are not too bad; in particular, they are ``rigid,'' and so the argument in \S\ref{art08-sec8} can be generalized. The second proof will use the transformation formulae of \S\ref{art08-sec4}; it is not completely general, in that we assume the parameter space to be a compact Riemann surface and not just a disc.

\medskip
\textsc{First Proof of \eqref{ART08-SEC6-EQ6.4} (by direct argument).} To get an understanding of the singularities of $Z_{q}(\bfE)$, let $\sigma_{1}$, $\sigma_{2}$ be general sections of $\bfE\to V$ so that $Z_{k-1}(\bfE)$ is given by $\sigma_{1}\wedge \sigma_{2}=0$. If, say, $\sigma_{1}(z_{0})\neq 0$, we may choose a local holomorphic frame $e_{1},\ldots,e_{k}$ with $e_{1}=\sigma_{1}$. Then $\sigma_{2}(z)=\sum\limits^{k}_{\alpha=1}\xi^{\alpha}(z)e_{\alpha}$, and $Z_{k-1}(\bfE)$ is locally given by $\xi^{2}=\ldots=\xi^{k}=0$. We may thus assume that the singular points of $Z_{k-1}(\bfE)$ will come where $\sigma_{1}=0=\sigma_{2}$. If $n\geq 2k$, there will be such points; choosing a suitable holomorphic frame $e_{1},\ldots,e_{k}$, we may assume that $\sigma_{1}(z)=\sum\limits^{k}_{\alpha=1}z^{\alpha}e_{\alpha}$ and $\sigma_{2}(z)=\sum\limits^{k}_{\alpha=1}z^{k+\alpha}e_{\alpha}$. Then $Z_{k-1}(\bfE)$ is locally given by
\begin{equation*}
z^{\alpha}z^{k+\beta}-z^{\beta}z^{k+\alpha}=0\quad (1\leq \alpha<\beta \leq k).\tag{9.1}\label{art08-sec9-eq9.1}
\end{equation*}
For example, when $k=2$, \eqref{art08-sec9-eq9.1} becomes $z^{1}z^{4}-z^{2}z^{3}=0$, which is essentially an ordinary double point.

Now let $\{\bfE_{\lambda}\}$ be a family of ample (c.f. \S\ref{art08-sec7}) vector bundles satisfying $H^{1}(V,\mathcal{O}(\bfE_{\lambda}))=0$ (c.f. \eqref{art08-sec7-eq7.18}). Then we may choose general sections $\sigma_{1}(\lambda),\ldots,\sigma_{k}(\lambda)$ of $\bfE_{\lambda}$ which depend holomorphically on $\lambda$; in this case, $Z_{\lambda}=Z_{q}(\bfE_{\lambda})$ is defined by $\sigma_{1}(\lambda)\wedge\ldots\wedge \sigma_{k-q+1}(\lambda)=0$. Letting $Z=Z_{0}$, we see that, although the $Z_{\lambda}$ are singular, the {\em singularities are rigid} in the following sense:

There\pageoriginale are local biholomorphic mappings $f_{\lambda}:U\to U$ ($U$ = open set on $V$) such that 
\begin{equation*}
Z_{\lambda}\cap U=f_{\lambda}(Z\cap U)\tag{9.3}\label{art08-sec9-eq9.3}
\end{equation*}

We now define an {\em infinitesimal displacement mapping:}
\begin{equation*}
\rho : \bfT_{0}(\Delta)\to H^{0}(Z,\Hom(I/I^{2},\mathcal{O}_{Z})),\tag{9.4}\label{art08-sec9-eq9.4}
\end{equation*}
where $I\subset \mathcal{O}_{V}$ is the ideal sheaf of $Z$. To do this, let $z^{1},\ldots,z^{n}$ be local coordinates in $U$ and $f(z;\lambda)=f_{\lambda}(z)$ the mappings given by \eqref{art08-sec9-eq9.3}. Let $\theta_{f}(z)$ be the local vector field $\sum\limits^{n}_{i=1}\dfrac{\partial f^{i}}{\partial \lambda}(z,\lambda)\dfrac{\partial}{\partial z^{i}}$. If $\xi(z)$ is a function in $I$ (so that $\xi(z)=0$ on $Z$), then $\theta_{f}\cdot \xi$ gives a section of $\mathcal{O}_{V}/I=\mathcal{O}_{Z}$. Furthermore, the mapping $\xi\to \theta_{f}\cdot \xi|Z$ is linear over $\mathcal{O}_{V}$ and is zero on $I^{2}$, so that we have a section of $\Hom(I/I^{2},\mathcal{O}_{Z})$ over $U$.

To see that this section is globally defined on $Z$, we suppose that $\widehat{f}_{\lambda}:U\to U$ also satisfies $\widehat{f}_{\lambda}(Z\cap U)=Z_{\lambda}\cap U$. Then $\widehat{f}(z;\lambda)=f(h(z,\lambda);\lambda)$ where $h(z;\lambda):Z\cap U\to Z\cap U$. Then
$$
\sum\limits^{n}_{i=1}\dfrac{\partial \widehat{f}^{i}}{\partial\lambda}(z,\lambda)\dfrac{\partial\xi}{\partial z^{i}}(z)=\sum\limits_{i,j}\dfrac{\partial f^{i}}{\partial z^{j}}\dfrac{\partial h^{j}}{\partial\lambda}\dfrac{\partial \xi}{\partial z^{i}}+\sum \dfrac{\partial f^{i}}{\partial \lambda}(h(z,\lambda);\lambda)\dfrac{\partial \xi}{\partial z^{i}}.
$$
Thus, at $\lambda=0$ and for $z\in Z$, $\theta_{\widehat{f}}\cdot \xi-\theta_{f}\cdot \xi$
$$
=\sum\limits_{i,j}\dfrac{\partial \xi}{\partial z^{i}}(z)\dfrac{\partial f^{i}}{\partial z^{j}}(z;0)\dfrac{\partial h^{j}}{\partial \lambda}(z;0)=\left[\dfrac{\partial}{\partial\lambda}(\xi(f(h(z,\lambda),0)))\right]_{\substack{\lambda=0\\ z\in Z}}=0.
$$
From this we get that $\theta_{\widehat{f}}\xi=\theta_{f}\xi$ in $\mathcal{O}_{Z}$. The resulting section of\break $\Hom(I/I^{2},\mathcal{O}_{Z})$ is, by definition, $\rho\left(\dfrac{\partial}{\partial\lambda}\right)$.


\begin{examples*}
\begin{itemize}
\item[(a)] In case $Z$ is nonsingular, $\Hom(I/I^{2},\mathcal{O}_{Z})=\mathcal{O}_{Z}(\bfN)$ where $\bfN\to Z$ is the normal bundle; then $\rho\left(\dfrac{\partial}{\partial\lambda}\right)\in H^{0}(Z,\mathcal{O}_{Z}(\bfN))$ is just Kodaira's infinitesimal displacement mapping \eqref{art08-sec3-eq3.8}.

\item[(b)] In case $Z\subset V$ is a hypersurface, $\rho\left(\dfrac{\partial}{\partial\lambda}\right)\xi$ vanishes on the singular points of $Z$. This is because $\xi=\eta g$ where $g(z)=0$ is a minimal equation for $Z\cap U$. Then, in the above notation, $\theta_{f}\cdot \xi|Z=\eta\theta_{f}\cdot g|Z$, and $\theta_{f}\cdot g$ vanishes on $g=0$, $dg=0$, which is the singular locus of $Z$.
\end{itemize}
\end{examples*}

Now\pageoriginale suppose that 
$$
\dim Z=n-q\text{~ and that~ }\omega=\sum\limits_{\substack{I=(i_{1},\ldots,i_{n-q+1})\\ J=(j_{1},\ldots,j_{n-q})}}\omega_{I\overline{J}}dz^{I}\overline{z}d^{J}
$$
is a $C^{\infty}$ form of type $(n-q+1,n-q)$. Then
$$
\langle \theta_{f},\omega\rangle = \sum\limits_{I,J,l}\pm \omega_{I\overline{J}}\dfrac{\partial f^{1}l}{\partial \lambda}dz^{i_{1}}\wedge\ldots\wedge \widehat{dz^{i}}l\wedge\ldots\wedge dz^{i_{n-q+1}}\wedge d\overline{z}^{J}
$$
is a $C^{\infty}(n-q,n-q)$ form in $U$ whose restriction to the manifold points $Z_{\text{reg}}\subset Z$ is well-defined. Thus, there exists a $C^{\infty}(n-q,n-q)$ form $\Omega=\langle \xi^{*}\omega,\rho\left(\dfrac{\partial}{\partial\lambda}\right)\rangle$ on $Z_{\text{reg}}$ such that $\int\limits_{Z_{\text{reg}}}\Omega\displaystyle{\mathop{=}\limits^{\text{def.}}}\int\limits_{Z}\Omega$ converges. Just as in the proof of \eqref{art08-sec3-eq3.7} (c.f. \cite{art08-key9}, \S\ref{art08-sec4}), we can now prove:
\begin{gather*}
\text{\em The differential } \phi_{*}:\bfT_{0}(\Delta)\to H^{q-1,q}(V) \text{\em ~ of the mapping}\notag\\
\phi(\lambda)=\phi_{q}(Z_{\lambda}-Z)\tag{9.5}\label{art08-sec9-eq9.5}
\end{gather*}
{\em is given by}
\begin{equation*}
\int\limits_{V}\phi_{*}\left(\dfrac{\partial}{\partial \lambda}\right)\wedge\omega=\int\limits_{Z}\langle \xi^{*}\omega, \rho\left(\dfrac{\partial}{\partial\lambda}\right)\rangle,\tag{9.6}\label{art08-sec9-eq9.6}
\end{equation*}
{\em where the right-hand side of \eqref{art08-sec9-eq9.6} means, as above, that we take the Poincar\'e residue $\xi^{*}\omega$ of $\omega$ on $Z_{\text{reg}}$ and contract with 
$$
\rho\left(\dfrac{\partial}{\partial \lambda}\right)\in H^{0}(Z,\Hom(I/I^{2},\mathcal{O}_{Z}))
$$ 
given by \eqref{art08-sec9-eq9.4}.}

\begin{example*}
The point of \eqref{art08-sec9-eq9.5} can be illustrated by the following example. Let $Z\subset \bfC^{2}$ be given by $xy=0$ and $\theta\in \Hom(I/I^{2},\mathcal{O}_{Z})$ by $\theta(xy)=1$. Then, on the $x$-axis $(y=0)$, $\theta$ is the normal vector field $\dfrac{1}{x}\dfrac{\partial}{\partial y}$; on the $y$-axis, $\theta$ is $\dfrac{1}{y}$ $\dfrac{\partial}{\partial x}$. If now $\omega=dxdy$, then, on the $x$-axis, $\langle \xi^{*}\omega,\theta\rangle=\dfrac{1}{x}dx$ and so $\int\limits_{Z_{\text{reg}}}\langle \xi^{*}\omega,\theta\rangle$ becomes infinite on the singular points of $Z$.
\end{example*}

More generally, if $g(x,y)=x^{a}-y^{b}$ with $(a,b)=1$, and if $\theta\in \Hom(I/I^{2},\mathcal{O}_{Z})$ is given by $\theta(g)=1$, then $\theta$ corresponds to the normal vector field $\dfrac{1}{\partial g/\partial y}$ $\dfrac{\partial}{\partial y}$. Thus, if $\omega = dxdyd\overline{x}$, $\langle \xi^{*}\omega,\theta\rangle=\dfrac{dxd\overline{x}}{(\partial g/\partial y)}$. Letting $x=t^{b}$, $y=t^{a}$, we have
$$
\langle \xi^{*}\omega,\theta\rangle =\left(\dfrac{b^{2}}{a}\right)\cdot \dfrac{|t|^{2a-2}dtd\overline{t}}{t^{a(b-1)}},
$$\pageoriginale
which may be highly singular at $t=0(=Z_{\text{sing}})$.

We now reformulate \eqref{art08-sec9-eq9.5} as follows.

Let $\{\bfE_{\lambda}\}_{\lambda\in \Delta}$ be our family of bundles and $\phi:\Delta\to T_{q}(V)$ the mapping \eqref{art08-sec3-eq3.1} corresponding to $Z_{q}(\bfE_{\lambda})-Z_{q}(\bfE_{0})$. If $\sigma_{1}(\lambda),\ldots,\sigma_{k}(\lambda)$ are general sections of $\bfE_{\lambda}\to V$ which depend holomorphically on $\lambda$, then $Z_{q}(\bfE_{\lambda})$ is given by $\sigma_{1}(\lambda)\wedge\ldots\wedge \sigma_{k-q+1}(\lambda)=0$. We let $Y_{\lambda}\subset Z_{\lambda}$ be the Zariski open set where $\sigma_{1}(\lambda)\wedge\ldots\wedge \sigma_{k-q}(\lambda)\neq 0$. Then $Y_{\lambda}\subset V$ is a submanifold (not closed) and $\{Y_{\lambda}\}_{\lambda\in \Delta}$ forms a {\em continuous system}. We let $\rho:\bfT_{0}(\Delta)\to H^{0}(Y,\mathcal{O}(\bfN))$ (where $\bfN\to Y=Y_{0}$ is the normal bundle) be the infinitesimal displacement mapping. If then $\psi\in H^{n-q+1,n-q}(V)$, we have the formula:
\begin{equation*}
\int\limits_{V}\phi_{*}\left(\dfrac{\partial}{\partial\lambda}\right)\wedge\psi=\int\limits_{Y}\langle \rho\left(\dfrac{\partial}{\partial\lambda}\right),\xi^{*}\psi\rangle,\tag{9.8}\label{art08-sec9-eq9.8}
\end{equation*}
where $\phi_{*}\left(\dfrac{\partial}{\partial\lambda}\right)\in H^{q-1,q}(V)$ and $\xi^{*}\psi\in A^{n-q,n-q}(Y,\bfN^{*})$ is the Poincar\'e residue of $\psi$ along $Y$.

With this formulation, to prove \eqref{ART08-SEC6-EQ6.4} we want to show that
\begin{equation*}
\int\limits_{Y}\langle \rho\left(\dfrac{\partial}{\partial\lambda}\right),\xi^{*}\psi\rangle =\int\limits_{V}qP_{q}(\underbrace{\Theta,\ldots,\Theta}_{q-1};\eta)\wedge \psi,\tag{9.9}\label{art08-sec9-eq9.9}
\end{equation*}
where $\Theta$ is a curvature in $\bfE\to V$ and $\eta\in H^{0,1}(V,\Hom(\bfE,\bfE))$ is the Kodaira-Spencer class $\delta\left(\dfrac{\partial}{\partial\lambda}\right)$ (c.f. \eqref{art08-sec6-eq6.3}).

Now $Z_{q+1}(\bfE)$ is defined by $\sigma_{1}\wedge\ldots\wedge \sigma_{k-q}=0$, and we let $W\subset V$ be the Zariski open set $\sigma_{1}\wedge\ldots\wedge \sigma_{k-q}\neq 0$; thus $W=V-Z_{q+1}(\bfE)$. Clearly we have
\begin{equation*}
\int\limits_{V}qP_{q}(\underbrace{\Theta,\ldots,\Theta}_{q-1};\eta)\wedge \psi=\int\limits_{W}qP_{q}(\underbrace{\Theta,\ldots,\Theta}_{q-1};\eta)\wedge\psi.\tag{9.10}\label{art08-sec9-eq9.10}
\end{equation*}
On the other hand, over $W$ we have an exact sequence 
\begin{equation*}
0\to \bfS\to \bfE_{W}\to \bfQ\to 0,\tag{9.11}\label{art08-sec9-eq9.11}
\end{equation*}
where\pageoriginale $\bfS$ is the trivial bundle generated by $\sigma_{1},\ldots,\sigma_{k-q}$. Suppose that we have an Hermitian metric in $\bfE\to V$ such that $\Theta$ is the curvature of the metric connection. Using this, we want to evaluate the right hand side of \eqref{art08-sec9-eq9.10}.

We now parallel the argument in \S\ref{art08-sec8} for a while. Since $H^{1}(V,\mathcal{O}(\bfE))=0$, $\eta\sigma_{\alpha}=\overline{\partial}\gamma_{\alpha}$ for some $C^{\infty}$ section $\gamma_{\alpha}$ of $\bfE\to V(\alpha=1,\ldots,k-q)$. On $W=V-Z_{q+1}(\bfE)$ we find a section $\zeta_{\alpha}$ of $\bfE^{*}\to V$ such that $\langle \zeta_{\alpha},\sigma_{\beta}\rangle=\delta^{\alpha}_{\beta}$. We claim that we can find such $\zeta_{\alpha}$ having a first order pole at a general point of $Z_{q+1}(\bfE)$.

\begin{proof}
On $W$, we look at unitary frames $e_{1},\ldots,e_{k}$ such that $e_{1},\ldots,e_{k-q}$ is a frame for $\bfS$. Then $\sigma_{\alpha}=\sum\limits^{k-q}_{\beta=1}h_{\alpha\beta}e_{\beta}$ where $\det (h_{\alpha\beta})$ vanishes to first order along $Z_{q+1}(\bfE)$. Set $\zeta_{\alpha}=\sum\limits^{k-q}_{\beta=1}(h^{-1})_{\beta\alpha}e^{*}_{\beta}$; then $\langle \zeta_{\alpha},\sigma_{\beta}\rangle =\sum\limits_{\gamma,\lambda}(h^{-1})_{\gamma\alpha}h_{\beta\lambda}\langle e^{*}_{\gamma},e_{\lambda}\rangle=\delta^{\alpha}_{\beta}$.
\end{proof}

\begin{remark*}
In the case $k-q=1$, $e_{1}=\dfrac{\sigma_{1}}{|\sigma_{1}|}$ and $\zeta_{1}=\dfrac{e^{*}_{1}}{|\sigma_{1}|}$.
\end{remark*}

On $W$, we define $\gamma=\left(\sum\limits^{k-q}_{\alpha=1}\zeta_{\alpha}\otimes \gamma_{\alpha}\right)$. Then $\overline{\partial}\gamma\cdot \sigma_{\alpha}=\eta\cdot \sigma_{\alpha}$ and so, if $\widehat{\eta}=\eta-\overline{\partial}\gamma$, $\widehat{\eta}\cdot \sigma_{\alpha}\equiv 0$ and $\widehat{\eta}$ has a pole of order one along $Z_{q+1}(\bfE)$. By Stoke's theorem then,
\begin{equation*}
\int\limits_{W}qP_{q}(\underbrace{\Theta,\ldots,\Theta}_{q-1};\eta)=\int\limits_{W}qP_{q}(\underbrace{\Theta,\ldots,\Theta}_{q-1};\widehat{\eta}).\tag{9.12}\label{art08-sec9-eq9.12}
\end{equation*}

In terms of the natural unitary frames for $0\to \bfS\to \bfE_{W}\to \bfQ\to 0$, $\widehat{\eta}=\left(\begin{tabular}{c|c} 0 & *\\\hline 0 & *\end{tabular}\right)$.

We now work on the curvature $\Theta$. The curvature $\widehat{\Theta}$ in $\bfS\oplus \bfQ\to W$ may be assumed to have the form $\widehat{\Theta}=\left(\begin{smallmatrix} 0 & 0\\ 0 & \Theta_{\bfQ}\end{smallmatrix}\right)$ (since $\Theta_{\bfS}=0$), and the same techniques as used in the Appendix to \S\ref{art08-sec4} can be applied to show:
\begin{equation*}
qP_{q}(\Theta,\ldots,\Theta;\widehat{\eta})-qP_{q}(\widehat{\Theta},\ldots,\widehat{\Theta};\widehat{\eta})=\overline{\partial}\lambda,\tag{9.13}\label{art08-sec9-eq9.13}
\end{equation*}
where\pageoriginale $\lambda$ has a pole of order $2q-1$ along $Z_{q+1}(\bfE)$ (c.f. \eqref{art08-sec4-eqA4.24} and the accompanying calculation). By Stoke's theorem again,
\begin{equation*}
\int\limits_{W}qP_{q}(\underbrace{\Theta,\ldots,\Theta}_{q-1};\widehat{\eta})=\int\limits_{W}qP_{q}(\underbrace{\widehat{\Theta},\ldots,\widehat{\Theta}}_{q-1};\widehat{\eta}).\tag{9.14}\label{art08-sec9-eq9.14}
\end{equation*}
From \eqref{art08-sec9-eq9.9}, \eqref{art08-sec9-eq9.10}, \eqref{art08-sec9-eq9.12}, and \eqref{art08-sec9-eq9.14}, we have to show
\begin{equation*}
\int\limits_{Y}\langle \rho\left(\frac{\partial}{\partial\lambda}\right), \xi^{*}\psi\rangle =\int\limits_{W}qP_{q}(\underbrace{\widehat{\Theta},\ldots,\widehat{\Theta}}_{q-1};\widehat{\eta})\wedge \psi.\tag{9.15}\label{art08-sec9-eq9.15}
\end{equation*}

Now write $\widehat{\eta}=\left(\begin{smallmatrix} 0 & *\\ 0 & \eta_{\bfQ}\end{smallmatrix}\right)$; clearly we have 
$$
P_{q}(\underbrace{\widehat{\Theta},\ldots,\widehat{\Theta}}_{q-1};\widehat{\eta})=P_{q}(\underbrace{\Theta_{\bfQ},\ldots,\Theta_{\bfQ}}_{q-1};\eta_{\bfQ}).
$$ 
Thus, to prove \eqref{art08-sec9-eq9.9}, we need by \eqref{art08-sec9-eq9.15} to show that
\begin{equation*}
\int\limits_{Y}\langle \rho\left(\dfrac{\partial}{\partial\lambda}\right), \xi^{*}\psi\rangle = \int\limits_{W}qP_{q}(\Theta_{\bfQ},\ldots,\Theta_{\bfQ};\eta_{\bfQ}).\tag{9.16}\label{art08-sec9-eq9.16}
\end{equation*}

The crux of the matter is this. Over $W$, we have a holomorphic bundle $\bfQ\to W$ and a holomorphic section $\sigma\in H^{0}(W,\mathcal{O}(\bfQ))$; $\sigma$ is just the projection on $\bfQ$ of $\sigma_{k-q+1}\in H^{0}(V,\mathcal{O}(\bfE))$. The subvariety $Y$ is given by $\sigma=0$, and the normal bundle of $Y$ is $\bfQ\to Y$. Thus $\rho\left(\dfrac{\partial}{\partial\lambda}\right)$ is a holomorphic section of $\bfQ\to Y$, and \eqref{art08-sec9-eq9.16} {\em is essentially the exact analogue of \eqref{art08-sec8-eq8.2} with $Y$ replacing $Z$ and $W$ replacing $V$.} To make the analogy completely precise, we need to know that $\eta_{\bfQ}$ and $\rho\left(\dfrac{\partial}{\partial\lambda}\right)$ are related as in \eqref{art08-sec8-eq8.2}. If we know this, and if we can keep track of the singularities along $Z_{q+1}(\bfE)$, then \eqref{art08-sec9-eq9.16} can be proved just at \eqref{art08-sec8-eq8.2} was above. Thus we need the analogues of \eqref{art08-sec7-eq7.17} and \eqref{art08-sec7-eq7.19}; what must be proved is this:

There exists a $C^{\infty}$ section $\tau$ of $\bfQ\to W$ such that $\tau|Y$ is
\begin{equation*}
\rho\left(\frac{\partial}{\partial\lambda}\right)\quad\text{and}\quad \overline{\partial}\tau=\eta_{\bfQ}\sigma.\tag{9.17}\label{art08-sec9-eq9.17}
\end{equation*}

In addition, we must keep track of the singularities of $\tau$ along\break $Z_{q+1}(\bfE)$ so as to insure that the calculations in \S\ref{art08-sec8} will still work.

\eject

For\pageoriginale simplicity, suppose that $q=k-1$ so that $Z_{k-1}(\bfE)$ is given by $\sigma_{1}\wedge\sigma_{2}=0$ and $Z_{k}(\bfE)$ by $\sigma_{1}=0$. Let $\bfE_{\lambda}\to V$ be given by $\{g_{\alpha\beta}(\lambda)\}$ (c.f. \S\ref{art08-sec6}) and $\sigma_{j}(\lambda)$ by holomorphic vectors $\{\sigma_{j\alpha}(\lambda)\}(j=1,2)$. Then $\sigma_{j\alpha}(\lambda)=g_{\alpha\beta}(\lambda)\sigma_{j\beta}(\lambda)$ and
$$
\dfrac{\partial\sigma_{j\alpha}(\lambda)}{-\partial\lambda}=g_{\alpha\beta}(\lambda)\dfrac{\partial\sigma_{j\beta}(\lambda)}{\partial \lambda}+\dfrac{\partial g_{\alpha\beta}(\lambda)}{\partial \lambda}g_{\alpha\beta}(\lambda)^{-1}\{g_{\alpha\beta}(\lambda)\sigma_{j\beta}(\lambda)\}.
$$
At $\lambda=0$, this says that
\begin{equation*}
\delta\left(\dfrac{\partial \sigma_{j}}{\partial\lambda}\right)=\eta\cdot \sigma_{j},\tag{9.18}\label{art08-sec9-eq9.18}
\end{equation*}
where $\dfrac{\partial\sigma_{j}}{\sigma\lambda}$ is a zero cochain for the sheaf $\mathcal{O}(\bfE)$ and $\eta=\{\overdot{g}_{\alpha\beta}g^{-1}_{\alpha\beta}\}$ is the {\em Kodaira-Spencer class} \eqref{art08-sec6-eq6.3}.

Let $'$ denote $\left.\dfrac{\partial}{\partial \lambda}\right]_{\lambda=0}$. Then from \eqref{art08-sec9-eq9.18} we have
\begin{equation*}
(\sigma_{1}\wedge \sigma_{2})'=\sigma'_{1}\wedge \sigma_{2}+\sigma_{1}\wedge \sigma'_{2}=\eta\cdot (\sigma_{1}\wedge \sigma_{2}).\tag{9.19}\label{art08-sec9-eq9.19}
\end{equation*}

Thus, over $Z_{k-1}(\bfE)$, $(\sigma_{1}\wedge \sigma_{2})'$ is a holomorphic section of $\Lambda^{2}\bfE\to Z_{k-1}(\bfE)$. On the other hand, over $Y=Z_{k-1}(\bfE)\to Z_{k}(\bfE)$, $\sigma_{1}$ is non-zero. Since $\bfS\subset \bfE_{W}$ is the sub-bundle generated by $\sigma_{1}$, we have on $W$ an exact sequence:
\begin{equation*}
0\to \bfS\to \bfE_{W}\to \sigma_{1}\wedge \bfE_{W}\to 0,\tag{9.20}\label{art08-sec9-eq9.20}
\end{equation*}
where the last bundle is the sub-bundle of $\Lambda^{2}\bfE_{W}$ of all vectors $\xi$ such that $\xi\wedge\sigma_{1}=0$ in $\Lambda^{3}\bfE_{W}$.

Along $Y$, $\sigma_{1}\wedge\sigma_{2}=0$ and so $\sigma_{1}\wedge (\sigma_{1}\wedge\sigma_{2})'=0$; thus $(\sigma_{1}\wedge\sigma_{2})'$ is a section along $Y$ of $\sigma_{1}\wedge\bfE$. But $\sigma_{1}\wedge\bfE$ is naturally isomorphic to $\bfQ$ and, under this isomorphism, we may see that
\begin{equation*}
(\sigma_{1}\wedge\sigma_{2})'=\rho\left(\dfrac{\partial}{\partial\lambda}\right).\tag{9.21}\label{art08-sec9-eq9.21}
\end{equation*}
Thus we have identified $\rho\left(\dfrac{\partial}{\partial\lambda}\right)$.

Let now $\eta\in A^{0,1}(V,\Hom(\bfE,\bfE))$ be a {\em Dolbeault class} corresponding to $\{\overdot{g}_{\alpha\beta}g^{-1}_{\alpha\beta}\}$. Then $\eta\sigma_{1}=\overline{\partial}\gamma_{1}$ and $\eta\cdot \sigma_{2}=\overline{\partial}\gamma_{2}$ where $\gamma_{1}$, $\gamma_{2}$ are $C^{\infty}$ sections of $\bfE\to V$. Clearly these equations are the global analogue of \eqref{art08-sec9-eq9.18}. In particular, we may assume that, along $Z_{k-1}(\bfE)$,
\begin{equation*}
\sigma'_{1}\wedge \sigma_{2}+\sigma_{1}\wedge \sigma'_{2}=\gamma_{1}\wedge \sigma_{2}+\sigma_{1}\wedge \gamma_{2}=\rho\left(\dfrac{\partial}{\partial\lambda}\right).\tag{9.22}\label{art08-sec9-eq9.22}
\end{equation*}\pageoriginale

Now $\gamma_{1}\wedge \sigma_{2}+\sigma_{1}\wedge\gamma_{2}$ is a $C^{\infty}$ section of $\Lambda^{2}\bfE_{W}\to W$, but will not in general lie in $\sigma_{1}\wedge \bfE_{W}\subset \Lambda^{2}\bfE_{W}$. However, letting $\gamma=\zeta_{1}\otimes \gamma_{1}$ be as just above \eqref{art08-sec9-eq9.12} (thus $\zeta_{1}$ is a $C^{\infty}$ section of $\bfE^{*}_{W}\to W$ satisfying $\langle\zeta_{1},\sigma_{1}\rangle=1$), we may subtract
$$
\gamma\cdot (\sigma_{1}\wedge\sigma_{2})=\gamma_{1}\wedge\sigma_{2}+\sigma_{1}\wedge\langle \zeta_{1},\sigma_{2}\rangle \gamma_{1}
$$
from $\gamma_{1}\wedge\sigma_{2}+\sigma_{1}\wedge\gamma_{2}$ without changing the value along $Y$. But then $\tau=\gamma_{1}\wedge\sigma_{2}+\sigma_{1}\wedge\gamma_{2}-\gamma\cdot (\sigma_{1}\wedge\sigma_{2})=\sigma_{1}\wedge\gamma_{2}-\langle \zeta_{1},\sigma_{2}\rangle\sigma_{1}\wedge\gamma_{1}$ lies in $\sigma_{1}\wedge \bfE_{W}$. This gives us that:
\begin{equation*}
\tau\text{ is a $C^{\infty}$ section of } \bfQ\to W\text{ such that } \tau|Y=\rho\left(\frac{\partial}{\partial\lambda}\right).\tag{9.23}\label{art08-sec9-eq9.23}
\end{equation*}

Also, $\overline{\partial}\tau=\overline{\partial}\gamma_{1}\wedge \sigma_{2}+\sigma_{1}\wedge\overline{\partial}\gamma_{2}-\overline{\partial}\gamma\cdot (\sigma_{1}\wedge\sigma_{2})= \eta\cdot \sigma_{1}\wedge\sigma_{2}+\sigma_{1}\wedge\eta\sigma_{2}-\overline{\partial}\gamma\cdot (\sigma_{1}\wedge\sigma_{2})=(\eta-\overline{\partial}\gamma)\cdot \sigma_{1}\wedge\sigma_{2}=\widehat{\eta}\cdot (\sigma_{1}\wedge\sigma_{2})$. Under the isomorphism $\sigma_{1}\wedge\bfE_{W}\cong \bfQ$, $\widehat{\eta}\cdot (\sigma_{1}\wedge\sigma_{2})=\sigma_{1}\wedge\widehat{\eta}\sigma_{2}$ (since $\widehat{\eta}\cdot \sigma_{1}=0$) corresponds to $\eta_{\bfQ}\cdot \sigma$, i.e. we have
\begin{equation*}
\overline{\partial}\tau=\eta_{\bfQ}\sigma.\tag{9.24}\label{art08-sec9-eq9.24}
\end{equation*}
Combining \eqref{art08-sec9-eq9.23} and \eqref{art08-sec9-eq9.24} gives \eqref{art08-sec9-eq9.17}.

The only possible obstacle to using the methods of \S\ref{art08-sec8} to prove \eqref{art08-sec9-eq9.16} is the singularities along $Z_{q+1}(\bfE)$. Now $\tau$ has at worst a pole of order one along $Z_{q+1}(\bfE)$, $\Theta_{\bfQ}$ has a pole of order $2$, and so the forms which enter into the calculation will have {\em at most a pole of order $2q$ along $Z_{q+1}(\bfE)$.} But this is just right, because $Z_{q+1}(\bfE)$ has (real) codimension $2q+2$, and we can use the following general principle.

Let $X$ be an $n$-dimensional compact, complex manifold and $S\subset X$ an irreducible subvariety of codimension $r$. If $\Omega$ is a smooth $2n$-form on $X-S$ with a pole of order $2r-1$ along $S$, then $\int\limits_{X-S}\Omega$ converges. Furthermore, if $\Omega_{1},\Omega_{2}$ are two $C^{\infty}$ forms on $X-S$ such that $\deg(\Omega_{1})+\deg (\Omega_{2})=2n-1$ and such that $\{$order of pole of $(\Omega_{1})\}$ + $\{$order of pole of $(\Omega_{2})\}=2r-2$, then $\int\limits_{X-S}d\Omega_{1}\wedge \Omega_{2}=(-1)^{\deg\Omega_{2}}\int\limits_{X-S}\Omega_{1}\wedge d\Omega_{2}$.

\begin{proof}
The\pageoriginale singularities of $S$ will not cause trouble, so assume $S$ is nonsingular and let $T_{\epsilon}$ be an $\epsilon$-tube around $S$. Then clearly $\lim\limits_{\epsilon\to 0}\int\limits_{X-T_{\epsilon}}\Omega$ converges and, by definition, equals $\int\limits_{X-S}\Omega=\int\limits_{X}\Omega$. Also, $\int\limits_{X-T_{\epsilon}}d\Omega_{1}\wedge\Omega_{2}-(-1)^{\deg \Omega_{1}}\int\limits_{X-T_{\epsilon}}\Omega_{1}\wedge d\Omega_{2}=-\int\limits_{\partial T_{\epsilon}}\Omega_{1}\wedge \Omega_{2}$. But, on $\partial T_{\epsilon}$, $|\Omega_{1}\wedge\Omega_{2}|\leq \dfrac{c}{\epsilon^{2r-2}}d\mu$ where $d\mu$ is the volume on $\partial T_{\epsilon}$. Since $\int\limits_{\partial T_{\epsilon}}d\mu\leq c'\epsilon^{2r-1}$, $\lim\limits_{\epsilon\to 0}\int\limits_{\partial T_{\epsilon}}\Omega_{1}\wedge\Omega_{2}=0$.
\end{proof}

\smallskip
\noindent
{\bf Second proof of \eqref{ART08-SEC6-EQ6.4} (by functoriality).}~ We shall consider over $V$ a family of holomorphic vector bundles $\{\bfE_{\lambda}\}_{\lambda\in C}$ parametrized by a {\em nonsingular algebraic curve} $C$; this family is given by a holomorphic bundle $\mathcal{E}\to V\times C$ where $\bfE_{\lambda}\cong \mathcal{E}|V\times \{\lambda\}$. We let $X=V\times C$ and $V_{\lambda}=V\times \{\lambda\}$, $V=V_{\lambda_{0}}$ where $\lambda_{0}\in C$ is the marked point. It may be assumed that $\mathcal{E}\to X$ is ample and $H^{1}(V,\mathcal{O}(\bfE_{\lambda}))=0= H^{1}(X,\mathcal{O}(\mathcal{E}))=0$ for all $\lambda\in C$ (c.f. \S\ref{art08-sec7}(c)).

Let $\mathscr{Z}_{q}\subset X$ be the $q^{\text{th}}$ {\em Chern class} of $\mathcal{E}\to X$ and $Z_{q}(\lambda)=\mathscr{Z}_{q}\cdot V_{\lambda}$; thus $Z_{q}(\lambda)$ is the $q^{\text{th}}$ Chern class of $\bfE_{\lambda}\to V$. More precisely, letting $\pi:X\to V$ be the projection, $\pi(\mathscr{Z}_{q}\cdot V_{\lambda})\to Z_{q}(\lambda)$ is the $q^{\text{th}}$ Chen class of $\bfE_{\lambda}\to V$.

Now let
$$
\mathscr{Z}_{\lambda}=\mathscr{Z}_{q}\cdot V_{\lambda}-\mathscr{Z}_{q}\cdot V_{\lambda_{0}}=\mathscr{Z}_{q}\cdot (V_{\lambda}-V_{\lambda_{0}})\text{~ and~ } Z_{\lambda}=Z_{q}(\lambda)-Z_{q}(\lambda_{0}).
$$
Then $\mathscr{Z}_{\lambda}$ is a cycle of codimension $q+1$ on $X$ which is algebraically equivalent to zero, and $Z_{\lambda}=\pi(\mathscr{Z}_{\lambda})$ is a similar cycle of codimension $q$ on $V$. Using an easy extension of the proof of \eqref{art08-sec4-eq4.14}, we have :
\begin{equation*}
\vcenter{\xymatrix@C=1.5cm@R=.5cm{
 & T_{q+1}(X)\ar[dd]^-{\pi_{*}}\\
C\ar[ur]^-{\phi^{q+1}(X)}\ar[dr]_{\phi_{q}(V)} & \\
 & T_{q}(V),
}}\tag{9.26}\label{art08-sec9-eq9.26}
\end{equation*}
where $\pi_{*}:H^{*}(X,\bfC)\to H^{*}(V,\bfC)$ is {\em integration over the fibre} and $\phi_{q+1}(X)(\lambda)=\phi_{q+1}(X)(\mathscr{Z}_{\lambda})$ (similarly for $\phi_{q}(V)$).

In infinitesimal form, \eqref{art08-sec9-eq9.26} is:
\begin{equation*}
\vcenter{\xymatrix@C=1.5cm@R=.5cm{
 & H^{q,q+1}(X)\ar[dd]^-{\pi_{*}}\\
T_{\lambda_{0}}(C)\ar[ur]^-{\phi_{q+1}(X)_{*}}\ar[dr]_{\phi_{q}(V)_{*}} & \\
 & H^{q-1,q}(V).
}}\tag{9.27}\label{art08-sec9-eq9.27}
\end{equation*}\pageoriginale
We let $\omega=\phi_{q}(V)_{*}\left(\dfrac{\partial}{\partial\lambda}\right)$ and $\Omega=\phi_{q+1}(X)_{*}\left(\dfrac{\partial}{\partial\lambda}\right)$, so that $\pi_{*}\Omega=\omega$ in $H^{q-1,q}(V)$. The class $\omega\in H^{q-1,q}(V)$ is characterized by
\begin{equation*}
\int\limits_{X}\Omega\wedge\pi^{*}\psi=\int\limits_{V}\omega\wedge\psi,\text{ for all } \psi\in H^{n-q+1,n-q}(V).\tag{9.28}\label{art08-sec9-eq9.28}
\end{equation*}

The family of divisors $V_{\lambda}\subset X$ defines $\phi_{1}(X):C\to T_{1}(X)$, and, from the mapping
\begin{equation*}
\phi_{1}(X)_{*}:T_{\lambda_{0}}(C)\to H^{0,1}(X),\tag{9.29}\label{art08-sec9-eq9.29}
\end{equation*}
we let $\theta=\phi_{1}(X)_{*}\left(\dfrac{\partial}{\partial\lambda}\right)$. Thus $\theta$ is the {\em infinitesimal variation of $V_{\lambda}$ measured in the Picard variety of $X$.} Letting $\Psi\in H^{q,q}(X)$ be the Poincar\'e dual of $\mathscr{Z}_{q}$, we have by \eqref{art08-sec4-eq4.17} that
\begin{equation*}
\Omega=\theta\Psi.\tag{9.30}\label{art08-sec9-eq9.30}
\end{equation*}

Because $V_{\lambda}\subset X$ is a divisor and because of \eqref{art08-sec3-eq3.10}, we know how to compute $\theta\in H^{0,1}(X)$. By \eqref{art08-sec9-eq9.30}, $\Omega\in H^{q,q+1}(X)$ is known, and so we must find $\pi_{*}(\Omega)$. This calculation, when carried out explicitly, will prove \eqref{ART08-SEC6-EQ6.4}.

First, let $\bfL\to X$ be the line bundle $[V_{\lambda_{0}}]$ and $\sigma\in H^{0}(X,\mathcal{O}(\bfL))$ the holomorphic section with $V_{\lambda_{0}}$ given by $\sigma=0$. Then $\bfL|V=\bfN$ is the {\em normal bundle} of $V$ in $X$; in fact, $\bfN\to V$ is clearly a trivial bundle with non-vanishing section $\dfrac{\partial}{\partial\lambda}$, where $\lambda$ is a local coordinate on $C$ at $\lambda_{0}$. Choose a $C^{\infty}$ section $\tau$ of $\bfL\to X$ with $\tau|V=\dfrac{\partial}{\partial\lambda}$ and write
\begin{equation*}
\overline{\partial}\tau = \theta\sigma.\tag{9.31}\label{art08-sec9-eq9.31}
\end{equation*}

Then,\pageoriginale by \S\ref{art08-sec7}(e), $\theta\in H^{0,1}(X)$ and gives $\phi_{1}(X)_{*}\left(\dfrac{\partial}{\partial\lambda}\right)$. By the same argument as in \eqref{art08-sec3-eq3.10}, we have :
$$
\int\limits_{X}(\partial \Psi)\wedge \pi^{*}\psi=-\lim\limits_{\epsilon\to 0}\int\limits_{\partial T_{\epsilon}}\overline{\partial}\left(\dfrac{\tau}{\sigma}\right)\Psi \wedge \pi^{*}\psi=\int\limits_{V}\langle \dfrac{\partial}{\partial \lambda},\xi^{*}(\Psi\wedge\pi^{*}\psi)\rangle
$$
($\xi^{*}$ being given by \eqref{art08-sec3-eq3.6}) $=\int\limits_{V}\langle \dfrac{\partial}{\partial \lambda}, \xi^{*}\Psi\rangle \wedge\psi$. Combining, we have $\int\limits_{X}\Omega \wedge \pi^{*}\psi=\int\limits_{V}\langle \dfrac{\partial}{\partial\lambda},\xi^{*}\Psi\rangle \psi$ for all $\psi\in H^{n-q+1,n-q}(V)$; by \eqref{art08-sec9-eq9.28}, we see then that
\begin{equation*}
\omega=\pi_{*}(\Omega)=\langle \dfrac{\partial}{\partial\lambda},\xi^{*}\Psi\rangle.\tag{9.32}\label{art08-sec9-eq9.32}
\end{equation*}
This equation is the crux of the matter; in words, it says that:

The infinitesimal variation of $Z_{q}(\bfE)$ in $T_{q}(V)$ is given by the Poincar\'e residue, relative to $\partial/\partial\lambda$ along $V\times \{\lambda_{0}\}$ in $V\times C$, of the form 
\begin{equation*}
P_{q}(\Theta,\ldots,\Theta)\text{ on } V\times C\text{ where } \Theta \text{ is a curvature in } \mathcal{E}\to V\times C.\tag{9.33}\label{art08-sec9-eq9.33}
\end{equation*}

Since $\Theta |V=\Theta_{\bfE}$ is a curvature in $\bfE\to V$, and since
$$
\langle \dfrac{\partial}{\partial\lambda}, P_{q}(\Theta,\ldots,\Theta)\rangle =qP_{q}(\Theta,\ldots,\Theta,\langle \dfrac{\partial}{\partial \lambda},\Theta\rangle ),
$$
to prove \eqref{ART08-SEC6-EQ6.4} we must show that:
\begin{align*}
& \langle\dfrac{\partial}{\partial\lambda},\Theta\rangle=\eta\in H^{0,1}(V,\Hom(\bfE,\bfE))\text{~ is the {\em Kodaira-Spencer class}}\\
& \delta\left(\dfrac{\partial}{\partial\lambda}\right)\text{~ given by \eqref{art08-sec6-eq6.3}}.\tag{9.34}\label{art08-sec9-eq9.34}
\end{align*}

Let then $\Delta$ be a neighborhood, with coordinate $\lambda$, of $\lambda_{0}$ on $C$ and $\{U_{\alpha}\}$ an open covering for $V$. Then $\mathcal{E}|V\times \Delta$ is given by transition functions $\{g_{\alpha\beta}(z,\lambda)\}$, and a $(1,0)$ connection $\theta$ for $\mathcal{E}\to V\times C$ is given by matrices $\theta_{\alpha}=\theta_{\alpha}(z,\lambda;dz,d\lambda)$ of $(1,0)$ forms which satisfy
\begin{equation*}
\theta_{\alpha}-g_{\alpha\beta}\theta_{\beta}g^{-1}_{\alpha\beta}=dg_{\alpha\beta}g^{-1}_{\alpha\beta}=\left(\sum\limits_{k}\dfrac{\partial g_{\alpha\beta}}{\partial z^{j}}dz^{j}+\dfrac{\partial g_{\alpha\beta}}{\partial \lambda}d\lambda\right)g^{-1}_{\alpha\beta}.\tag{9.35}\label{art08-sec9-eq9.35}
\end{equation*}

The curvature $\Theta|U_{\alpha}\times \Delta$ is given by $\Theta |U_{\alpha}\times \Delta=\overline{\partial}\theta_{\alpha}$. Thus $\langle \dfrac{\partial}{\partial\lambda},\Theta\rangle |U_{\alpha}\times \Delta$ is given by $\overline{\partial}\langle \dfrac{\partial}{\partial\lambda},\theta_{\alpha}\rangle$. But, on $U_{\alpha}\times \{\lambda_{0}\}$ $(\lambda_{0}=0)$,\pageoriginale we have $\langle \dfrac{\partial}{\partial\lambda},\theta_{\alpha}\rangle - g_{\alpha\beta}\langle \dfrac{\partial}{\partial\lambda},\theta_{\beta}\rangle g^{-1}_{\alpha\beta}=\overdot{g}_{\alpha\beta}g^{-1}_{\alpha\beta}$, so that 
$$
\overline{\partial}\left\{\langle \dfrac{\partial}{\partial\lambda}, \theta_{\alpha}\rangle | U_{\alpha}\times \{\lambda_{0}\}\right\}
$$ 
is a {\em Dolbeault representative} of the C\^ech cocycle $\{\overdot{g}_{\alpha\beta}g^{-1}_{\alpha\beta}\}=\delta\left(\dfrac{\partial}{\partial\lambda}\right)$ by \eqref{art08-sec6-eq6.3}. Thus $\delta\left(\dfrac{\partial}{\partial\lambda}\right)$ is given by $\langle \dfrac{\partial}{\partial\lambda},\Theta\rangle | V\times \{\lambda_{0}\}$ which proves \eqref{art08-sec9-eq9.34}.

\section{Concluding Remarks.}\label{art08-sec10}

Let $V$ be an algebraic manifold and $\Sigma_{q}$ the group of algebraic cycles of codimension $q$ which are algebraically equivalent to zero. Letting $T_{q}(V)$ be the torus constructed in \S\ref{art08-sec2}, there is a holomorphic homomorphism
\begin{equation*}
\phi : \Sigma_{q}\to T_{q}(V),\tag{10.1}\label{art08-sec10-eq10.1}
\end{equation*}
given by \eqref{art08-sec3-eq3.2}. Letting $A_{q}$ be the image of $\phi$, we have that:
\begin{equation*}
A_{q}\text{~ is an abelian variety (c.f. \eqref{art08-sec2-eq2.6}) and } \bfT_{0}(A_{q})\subset H^{q-1,q}(V).\tag{10.2}\label{art08-sec10-eq10.2}
\end{equation*}

The two main questions are: {\em What is the equivalence relation defined by $\phi$ (Abel's theorem), and what is $\bfT_{0}(A_{q})$ (inversion theorem)?} While we have made attempts at both of these, none of our results are definitive, and we want now to discuss the difficulties.

The obvious guess about the image of $\phi$ is:
\begin{equation*}
\bfT_{0}(A_{q})\text{ is the largest {\em rational subspace} contained in } H^{q-1,q}(V).\tag{10.3}\label{art08-sec10-eq10.3}
\end{equation*}

\begin{remark*}
A subspace $S\subset H^{q-1,q}(V)$ is {\em rational} if there exist integral cycles $\Gamma_{1},\ldots,\Gamma_{l}\in H_{2q-1}(V,\bfZ)$ such that $S=\{\omega\in H^{q-1,q}(V)$ for which $\int\limits_{\Gamma_{\rho}}\omega=0,\rho=1,\ldots,l\}$.
\end{remark*}

We want to show that:
\begin{equation*}
\begin{array}{l}
\text{\eqref{art08-sec10-eq10.3} is equivalent to a special}\\
\text{case of the (rational) {\em Hodge conjecture.}}
\end{array}\tag{10.4}\label{art08-sec10-eq10.4}
\end{equation*}

\begin{proof}
Let $S\subset H^{q-1,q}(V)$ be a rational subspace and $S_{\bfR}\subset H^{2q-1}(V,\bfR)$ the corresponding real vector space of all vectors $\omega+\overline{\omega}(\omega\in S)$. Then $S_{\bfR}\cap H^{2q-1}(V,\bfZ)$ is a lattice $\Gamma_{S}$ and $S_{R}/\Gamma_{S}=J_{q}(V)$ is a torus which has a complex structure given by: $S\subset S_{\bfR}\otimes \bfC$ is the\pageoriginale space of holomorphic tangent vectors of $J_{q}(V)$. Furthermore, $J_{q}(V)$ is an {\em abelian variety} which will vary holomorphically with $V$, provided that its dimension remains constant and that $S_{\bfR}(V)$ varies continuously (c.f. \S\ref{art08-sec2}). The space of holomoprhic $1$-forms on $J_{q}(V)$ is $S^{*}\subset H^{n-q+1,n-q}(V)$.

Now suppose that $Z\subset J_{q}\times V$ is an algebraic cycle of codimension $q$ on $J_{q}\times V$ such that, for a general point $\lambda\in J_{q}$, $Z\cdot \{\lambda\}\times V=Z_{\lambda}$ is a cycle of codimension $q$ on $V$. This gives a family $\{Z_{\lambda}\}_{\lambda\in J_{q}}$ of codimension $q$-cycles on $V$, and we have then a holomorphic homomorphism
\begin{equation*}
\phi : J_{q}\to T_{q}(V).\tag{10.5}\label{art08-sec10-eq10.5}
\end{equation*}
At the origin, the differential is
\begin{equation*}
\phi_{*}:S\to H^{q-1,q}(V),\tag{10.6}\label{art08-sec10-eq10.6}
\end{equation*}
and to compute $\phi_{*}$ we shall use a formula essentially proved in the last part of \S\ref{art08-sec9}: Let $e\in S$ be a $(1,0)$ vector on $J_{q}$ and $\Psi$ on $J_{q}\times V$ the $(q,q)$ form which is dual to $Z$. Then $\langle e,\Psi\rangle$ is a $(q-1,q)$ form on $J_{q}\times V$ and we have (c.f. \eqref{art08-sec9-eq9.33}):
\begin{equation*}
\phi_{*}(e)\text{ is } \langle e,\Psi\rangle \text{ restricted to } \{0\}\times V.\tag{10.7}\label{art08-sec10-eq10.7}
\end{equation*}
What we must do then is construct a rational $(q,q)$ form $\Psi$ on $J_{q}\times V$ such that, according to \eqref{art08-sec10-eq10.7},
\begin{equation*}
\langle e,\Psi\rangle \text{ is equal to $e$ on } \{0\}\times V.\tag{10.8}\label{art08-sec10-eq10.8}
\end{equation*}

Let $e_{1},\ldots,e_{r}$ be a basis for $S\subset H^{q-1,q}$ and $\psi_{1},\ldots,\psi_{r}$ the dual basis for $S^{*}\subset H^{n-q+1,n-q}$. Then the $\psi_{\rho}$ can be thought of as $(1,0)$ forms on $J_{q}$, the $e_{\rho}$ become $(1,0)$ vectors on $J_{q}$, and $\langle e_{\rho},\psi_{\sigma}\rangle =\delta^{\rho}_{\sigma}$ on $J_{q}$. We let
\begin{equation*}
\Psi =\sum\limits^{l}_{\rho=1}(\psi_{\rho}\otimes e_{\rho}+\overline{\psi}_{\rho}\otimes \overline{e}_{\rho}).\tag{10.9}\label{art08-sec10-eq10.9}
\end{equation*}
Then $\Psi$ is a real $(q,q)$ form on $J_{q}\times V$ and $\langle e_{\rho},\Psi\rangle=e_{\rho}$ is a $(q-1,q)$ form on $V$. Thus \eqref{art08-sec10-eq10.8} is satisfied and, to prove \eqref{art08-sec10-eq10.4} we need only show that $\Psi$ is rational.

If $f_{1},\ldots,f_{2r}$ is a rational basis for $S_{\bfR}\subset H^{2q-1}(V,\bfR)$ and $\theta_{1},\ldots,\theta_{2r}$ a dual rational basis for $S^{*}_{\bfR}\subset H^{2n-2q+1}(V,\bfR)$, then $e_{\rho}=\sum\limits^{2r}_{\beta=1}M_{\beta\rho}f_{\beta}$ and $f_{\alpha}=\sum\limits^{r}_{\rho=1}m_{\rho\alpha}e_{\rho}+\overline{m}_{\rho\alpha}\overline{e}_{\rho}$.\pageoriginale This gives $mM=I$ and $m\overline{M}=0$ where $m$ is an $r\times 2r$ and $M$ a $2r\times r$ matrix. Thus $\left(\dfrac{m}{m}\right)(M\overline{M})=\left(\begin{smallmatrix} I & 0 \\ 0 & I\end{smallmatrix}\right)$. We also see that $\psi_{\rho}=\sum\limits^{2r}_{\alpha=1}m_{\rho\alpha}\theta_{\alpha}$ and so $\Psi=\Sigma(m_{\rho\alpha}M_{\beta\rho}+\overline{m}_{\rho\alpha}\overline{M}_{\beta\rho})\theta_{\alpha}\otimes f_{\beta}=\sum\limits^{2r}_{\alpha=1}\theta_{\alpha}\otimes f_{\alpha}$, which is rational on $J_{q}\times V$.
\end{proof}

\begin{remark*}
A similar class $\Psi$ of $J_{q}\times V$ has been discussed by Lieberman, who calls it a {\em Poincar\'e cycle}, from the case $q=1$. In this case $J_{1}(V)=\Pic(V)\cong H^{0,1}(V)/H^{1}(V,\bfZ)$, and there is a line bundle $\mathscr{L}\to J_{1}\times V$ with $c_{1}(\mathscr{L})=\Psi$ and such that $\mathscr{L}|\{\lambda\}\times V=\bfL_{\lambda}$ is the line bundle over $V$ corresponding to $\lambda\in H^{0,1}(V)/H^{1}(V,\bfZ)\subset H^{1}(V,\mathcal{O}^{*})$.
\end{remark*}

We now prove:
\begin{equation*}
\begin{array}{l}
\text{If \eqref{art08-sec10-eq10.3} holds, then the equivalence relation defined by $\phi$}\\
\text{in \eqref{art08-sec10-eq10.1} is {\em rational equivalence} on a suitable subvariety}\\
\text{of a Chow variety associated to $V$.}
\end{array}\tag{10.10}\label{art08-sec10-eq10.10}
\end{equation*}

\begin{proof}
Let $\bfZ\subset V$ be an irreducible subvariety of codimension $q$ on $V$, and let $\Phi$ parametrize an algebraic family of subvarieties $Z\subset V$ such that $\bfZ\in\Phi$. Then (c.f. \S\ref{art08-sec5}) $\Phi$ is a subvariety of the Chow variety of $\bfZ$.

Now, if \eqref{art08-sec10-eq10.3} holds, then in proving it we will certainly be able to find a family $\{W_{\lambda}\}$ of effective subvarieties $W_{\lambda}\subset V$ of codimension $n-q+1$ which are parametrized by $\lambda\in J_{n-q+1}$ and such that $\phi_{n-q+1}(W_{\lambda}-W_{0})=\lambda$. Then, as in \S\ref{art08-sec5}, each $Z\in \Phi$ defines a divisor $D(Z)$ on $J_{n-q+1}$ and we want to prove :
\begin{equation*}
D(Z)\equiv D(\bfZ)\text{ if, and only if, } \phi_{q}(Z-\bfZ)=0\text{ in } T_{q}(V).\tag{10.11}\label{art08-sec10-eq10.11}
\end{equation*}

Let $\psi$ be a {\em residue operator} for $Z-\bfZ$ (c.f. \S\ref{art08-sec5}(a)) and set $\theta=d\left\{\int\limits^{W_{\lambda}}_{W_{0}}\psi\right\}$ on $J_{n-q+1}$ (c.f. \eqref{art08-sec5-eq5.21}). Then $\theta$ is a meromorphic form of the third kind on $J_{n-q+1}$ associated to the divisor $D(Z)-D(\bfZ)$. By \eqref{art08-sec5-eq5.24}, we have:
\begin{equation*}
\begin{array}{l}
D(Z)\equiv D(\bfZ)\text{ on } J_{n-q+1} \text{ if, and only if, there exists } \omega\in H^{1,0}(J_{n-q+1})\\
\text{ such that } \int\limits_{\delta}\theta+\omega\equiv 0(1)\text{ for all }\delta\in H_{1}(J_{n-q+1},\bfZ).
\end{array}\tag{10.12}\label{art08-sec10-eq10.12}
\end{equation*}\pageoriginale

Denote by $S\subset H^{n-q,n-q+1}(V)$ the largest rational subspace; then $S$ is the holomorphic tangent space to $J_{n-q+1}$. The holomorphic one forms $H^{1,0}(J_{n-q+1})$ are then $S^{*}\subset H^{q,q-1}(V)$. Given $\Omega\in S^{*}$, the corresponding form $\omega\in H^{1,0}(J_{n-q+1})$ is defined by
$$
\omega= d\left\{\int\limits^{W_{\lambda}}_{W_{0}}\Omega\right\}.
$$

Given $\delta\in H_{1}(J_{n-q+1},\bfZ)$, there is defined a $2q-1$ cycle $T(\delta)\in H_{2q-1}(V,\bfZ)$ by tracing out the $W_{\lambda}$ for $\lambda\in \delta$. Clearly we have
\begin{equation*}
\int\limits_{\delta}\theta+\omega=\int\limits_{T(\delta)}\psi+\Omega.\tag{10.13}\label{art08-sec10-eq10.13}
\end{equation*}
Combining \eqref{art08-sec10-eq10.13} and \eqref{art08-sec10-eq10.12}, we see that:
\begin{equation*}
\begin{array}{l}
D(Z)\equiv D(\bfZ)\text{ on } J_{n-q+1},\text{ if, and only if, } \int\limits_{\Gamma}\psi+\Omega\equiv 0(1)\\
\text{for some } \Omega\in S^{*} \text{ and all } \Gamma\in H_{2q-1}(V,\bfZ).
\end{array}\tag{10.14}\label{art08-sec10-eq10.14}
\end{equation*}
\end{proof}

Now taking into account the reciprocity relation \eqref{art08-sec5-eq5.30}, we find that \eqref{art08-sec10-eq10.14} implies \eqref{art08-sec10-eq10.10}.

\begin{remark*}
The mapping $T:H_{1}(J_{n-q+1},\bfZ)\to H_{2q-1}(V,\bfZ)$ may be divisible so that, to be precise, \eqref{art08-sec10-eq10.10} holds up to isogeny.
\end{remark*}

\noindent
{\bf Example \thnum{10.15}.\label{art08-sec10-exam10.15}}
Take $q=n$, so that $\Phi$ is a family of zero-cycles on $V$ and $\phi_{n}:\Phi\to T_{n}(V)$ is the Albanese mapping. Then $J_{n-q+1}=J_{1}=\Pic(V)$ and we may choose $\{W_{\lambda}\}_{\lambda\in \Pic(V)}$ to be a family of ample divisors. In this case we see that:
\begin{equation*}
\begin{array}{l}
\text{Albanese equivalence on $\Phi$ is, up to isogeny},\\
\text{linear equivalence on $\Pic(V)$.}
\end{array}\tag{10.16}\label{art08-sec10-eq10.16}
\end{equation*}

The conclusion drawn from \eqref{art08-sec10-eq10.4} and \eqref{art08-sec10-eq10.10} is:
\begin{equation*}
\begin{array}{l}
\text{\em The generalizations to arbitrary cycles of both the inversion}\\
\text{\em theorem and Abel's theorem, as formulated in \eqref{art08-sec10-eq10.3} and \eqref{art08-sec10-eq10.10},}\\
\text{\em essentially depend on a special case of the Hodge problem.}
\end{array}\tag{10.17}\label{art08-sec10-eq10.17}
\end{equation*}
\begin{equation*}
\begin{array}{l}
\text{The best example I know where the inversion theorem \eqref{art08-sec10-eq10.3}}\\
\text{and Abel's theorem \eqref{art08-sec10-eq10.10} hold is the case of the {\em cubic threefold}}\\
\text{worked out by F. Gherardelli. Let $V\subset P_{4}$ be the zero locus of a}\\
\text{nonsingular cubic polynomial. Through any point $z_{0}$ in $V$,}\\
\text{there will be six lines in $P_{4}$ lying on $V$.}
\end{array}\tag{10.18}\label{art08-sec10-eq10.18}
\end{equation*}\pageoriginale

\begin{proof}
Using affine coordinates $x$, $y$, $z$, $w$ and taking $z_{0}$ to be the origin, $V$ will be given by $f(x,y,z,w)=0$ where $f$ will have the form $f(x,y,z,w)=x+g_{2}(x,y,z,w)+g_{3}(x,y,z,w)$. Any line through $z_{0}$ will have an equation $x=\alpha_{0}t$, $y=\alpha_{1}t$, $z=\alpha_{2}t$, $w=\alpha_{3}t$. If the line is to lie on $V$, then we have $\alpha_{0}t+g_{2}(\alpha_{0},\alpha_{1},\alpha_{2},\alpha_{3})t^{2}+g_{3}(\alpha_{0},\alpha_{1},\alpha_{2},\alpha_{3})t^{3}=0$ for all $t$; thus $\alpha_{0}=0$ and $g_{2}(0,\alpha_{1},\alpha_{2},\alpha_{3})=0=g_{3}(0,\alpha_{1},\alpha_{2},\alpha_{3})$. Thinking $z_{0}$ are given by the points of intersection of a quadric and cubic in $P_{2}$, so there are six of them.

Let $\Phi$ be the variety of lines on $V$. Then it is known that $\Phi$ is a nonsingular surface and the irregularity $h^{0,1}(\Phi)$ is five. But also $h^{1,2}(V)=5$ and $h^{0,3}(V)=0$. Thus, in this case, $J_{2}(V)=T_{2}(V)$ is the whole torus. Fixing a base point $z_{0}\in \Phi$, there is defined $\phi_{2}:\Phi\to T_{2}(V)$ by the usual method. What Gherardelli has proved is:
\begin{equation*}
\phi_{2}:\Alb(\Phi)\to T_{2}(V)\text{~ is an isogeny.}\tag{10.19}\label{art08-sec10-eq10.19}
\end{equation*}

Thus, in the above notation, we have:
\begin{equation*}
\begin{array}{l}
\text{For the cubic threefold $V$, $A_{2}-J_{2}=T_{2}$ and so the inversion}\\
\text{theorem \eqref{art08-sec10-eq10.3} holds. Furthermore, the equivalence relation}\\
\text{given by the intermediate torus is, up to an isogeny, linear}\\
\text{equivalence on $\Phi$.}
\end{array}\tag{10.20}\label{art08-sec10-eq10.20}
\end{equation*}
\end{proof}


\newpage

\begin{center}
{\Large\bf APPENDIX}
\end{center}

\renewcommand\thesection{\Alph{section}}
\setcounter{section}{0}
\section{Theorem on the Cohomology of Algebraic Manifolds.}\label{art08-app-A}

Let $V$ be a compact, complex manifold and $A^{p,q}(V)$ the vector space of $C^{\infty}$ forms of type $(p,q)$ on $V$. From
\setcounter{equation}{0}
\begin{equation*}
\left.
\begin{array}{l}
\partial : A^{p,q}(V)\to A^{p+1,q}(V), \partial^{2}=0,\\
\overline{\partial} : A^{p,q}(V)\to A^{p,q+1}(V), \overline{\partial}^{2}=0, \partial\overline{\partial}+\overline{\partial}\partial=0,
\end{array}\right\}\tag{A.1}\label{art08-app-A-eqA.1}
\end{equation*}
we find a {\em spectral sequence} (c.f. \cite{art08-key7}, section 4.5) $\{E^{p,q}_{r}\}$ with $E^{p,q}_{1}=H^{p,q}_{\partial}(V)\cong H^{q}(V,\Omega^{p})$ (Dolbeault). This spectral sequence was discussed\pageoriginale by Fr\"olicher \cite{art08-key6}, who observed that, if $V$ was a {\em K\"ahler manifold}, then $E^{p,q}_{1}=E^{p,q}_{2}=\ldots=E^{p,q}_{\infty}$. This proved that:

There is a {\em filtration} $F^{p+q}_{p+q}(V)\subset\cdots\subset F^{p+q}_{0}(V)=H^{p+q}(V,\bfC)$ such that 
\begin{equation*}
F^{p+q}_{p}(V)/F^{p+q}_{p+1}(V)\cong H^{p,q}_{\overline{\partial}}(V)\cong H^{q}(V,\Omega^{p}).\tag{A.2}\label{art08-app-A-eqA.2}
\end{equation*}

Thus
\begin{equation*}
F^{p+q}_{p}(V)\cong \sum\limits_{r\geq 0}H^{p+r,q-r}_{\overline{\partial}}(V).\tag{A.3}\label{art08-app-A-eqA.3}
\end{equation*}

We call the filtration \eqref{art08-app-A-eqA.3} the {\em Hodge filtration}. Our object is to give a description of the Hodge filtration $\{F^{r}_{q}(V)\}$ {\em using only holomorphic functions}, from which it follows, e.g., that the Hodge filtration varies holomorphically with $V$. It will also prove that 
\begin{equation*}
F^{p+q}_{p}(V)\cong \ker d\cap \left(\sum\limits_{r\geq 0}A^{p+r,q-r}(V)\right)/ d\left(\sum\limits_{r\geq 0}A^{p+r,q-r-1}(V)\right),\tag{A.4}\label{art08-app-A-eqA.4}
\end{equation*}
which is the result \eqref{art08-sec3-eqA3.5} used there to prove \eqref{art08-sec3-eqA3.6}, the fact that the mappings $\phi_{q}:\Sigma_{q}\to T_{q}(V)$ depend only on the complex structure of $V$.

(a)~ Let $V$ be a complex manifold and $\Omega^{p}_{c}$ the sheaf on $V$ of {\em closed} holomorphic $p$-forms. There is an exact sheaf sequence:
\begin{equation*}
0\to \Omega^{p}_{c}\to \Omega^{p}\xrightarrow{\partial}\to \Omega^{p+1}_{c}\to 0.\tag{A.5}\label{art08-app-A-eqA.5}
\end{equation*}

\medskip
\noindent
{\bf Theorem \thnum{A.6}.\label{art08-app-A-thmA.6}}~(Dolbeault)
{\em In case $V$ is a compact K\"ahler manifold, we have $H^{q}(V,\Omega^{p}_{c})\to H^{q}(V,\Omega^{p})\to 0$, so that the exact cohomology sequence of \eqref{art08-app-A-eqA.5} is}
\begin{equation*}
0\to H^{q-1}(V\Omega^{p+1}_{c})\to H^{q}(V,\Omega^{p}_{c})\to H^{q}(V,\Omega^{p})\to 0.\tag{A.7}\label{art08-app-A-eqA.7}
\end{equation*}

\begin{proof}
We shall inductively define diagrams:
\begin{equation*}
\vcenter{\xymatrix{
 & H^{q-k-1}(V,\Omega^{p+k+2}_{c})\ar[d]^{\delta}\\
H^{q}(V,\Omega^{p})\ar[r]^-{\alpha_{k}}\ar@{-->}[ur]^-{\alpha_{k+1}}\ar[dr]_-{\beta_{k}} & H^{q-k}(V,\Omega^{p+k+1}_{c})\ar[d]\\
 & H^{q-k}(V,\Omega^{p+k+1})
}}\tag*{(A.8)$_{k}$}\label{art08-app-A-eqA.8k}
\end{equation*}
$(k=0,\ldots,q)$, where the first one is:
\begin{equation*}
\vcenter{\xymatrix{
 & H^{q-1}(V,\Omega^{p+2}_{c})\ar[d]^{\delta}\\
H^{q}(V,\Omega^{p})\ar[r]^-{\alpha_{0}}\ar@{-->}[ur]^-{\alpha_{1}}\ar[dr]_-{\beta_{0}} & H^{q}(V,\Omega^{p+1}_{c})\ar[d]\\
 & H^{q}(V,\Omega^{p+1})
}}\tag*{(A.8)$_{0}$}\label{art08-app-A-eqA.8-0}
\end{equation*}\pageoriginale
and where \ref{art08-app-A-eqA.8k} will define $\alpha_{k+1}$ after we prove that $\beta_{k}=0$. In \ref{art08-app-A-eqA.8k}, the mapping $\delta$ is the coboundary in the exact cohomology sequence of
$$
0\to \Omega^{p+k+1}_{c}\to \Omega^{p+k+1}\xrightarrow{\partial}\Omega_{c}^{p+k+2}\to 0.
$$

\eject

We want to prove that $\alpha_{0}=0$. If $\alpha_{k+1}=0$, then $\alpha_{k}=0$ so it will suffice to prove that $\alpha_{q}=0$. Now \ref{art08-app-A-eqA.8q} is
\begin{equation*}
\vcenter{\xymatrix{
 & 0\ar[d]\\
H^{q}(V,\Omega^{q})\ar[dr]_-{\beta_{q}}\ar[r]^-{\alpha_{q}} & H^{0}(V,\Omega^{q+p+1}_{c})\ar[d]\\
 & H^{0}(V,\Omega^{q+p+1}),
}}\tag*{(A.8)$_{q}$}\label{art08-app-A-eqA.8q}
\end{equation*}
and so we have to show that $\beta_{q}=0$. Thus, to prove Theorem \ref{art08-app-A-thmA.6}, we will show that:
\setcounter{equation}{8}
\begin{equation*}
\text{The maps $\beta_{k}$ in \ref{art08-app-A-eqA.8k} are zero for } k=0,\ldots,q.\tag{A.9}\label{art08-app-A-eqA.9}
\end{equation*}
The basic fact about K\"ahler manifolds which we use is this:
\begin{equation*}
\begin{array}{l}
\text{Let $\phi\in A^{p,q}(V)$ be a $C^{\infty}(p,q)$ form with $\overline{\partial}\phi=0$,}\\
\text{so that $\phi$ defines a class $\phi$ in the Dolbeault group}\\
\text{$H^{p,q}_{\overline{\partial}}(V)\cong H^{q}(V,\Omega^{p})$. Suppose that $\phi=\partial\psi$ for}\\
\text{some $\psi\in A^{p-1,q}(V)$. Then $\phi=0$ in $H^{p,q}_{\overline{\partial}}(V)$.}
\end{array}\tag{A.10}\label{art08-app-A-eqA.10}
\end{equation*}
\end{proof}

\begin{proof}
Let $\Box_{\overline{\partial}}$ and $\bfH_{\overline{\partial}}$ be the {\em Laplacian} and {\em harmonic projection} for $\overline{\partial}$, and similarly for $\Box_{\partial}$ and $\bfH_{\partial}$. Thus $\bfH_{\overline{\partial}}$ is the projection of $A^{p,q}(V)$\pageoriginale onto the kernel $\bfH^{p,q}_{\partial}(V)$ of $\Box_{\overline{\partial}}$, and likewise for $\bfH_{\partial}$. Since $\Box_{\overline{\partial}}$ is self-adjoint and $\Box_{\overline{\partial}}=\Box_{\partial}$ (because $V$ is K\"ahler), $\bfH_{\overline{\partial}}=\bfH_{\partial}$. Thus, if $\phi=\partial\psi$, $\bfH_{\partial}(\phi)=\bfH_{\overline{\partial}}(\phi)=0$. But if $\bfH_{\overline{\partial}}(\phi)=0$ and $\overline{\partial}\phi=0$, $\phi=\overline{\partial\partial}^{*}\bfG_{\overline{\partial}}\phi$ where $\overline{\partial}^{*}$ is the adjoint of $\overline{\partial}$ and $G_{\overline{\partial}}$ is the {\em Green's operator} for $\Box_{\overline{\partial}}$ (recall that $\phi=\bfH_{\overline{\partial}}(\phi)+\Box_{\overline{\partial}}\bfG_{\overline{\partial}}(\phi)$ and $\overline{\partial}\bfG_{\overline{\partial}}=\bfG_{\overline{\partial}}\overline{\partial}$). Thus $\phi=0$ in $H^{p,q}_{\overline{\partial}}(V)$ if $\phi=\partial \psi$.

Now $\beta_{0}:H^{p,q}_{\overline{\partial}}(V)\to H^{p+1,q}_{\overline{\partial}}(V)$ is given by $\beta_{0}(\phi)=\partial \phi$ so that $\beta_{0}=0$ and $\alpha_{1}$ is defined.

Write $\partial\phi=\overline{\partial}\psi_{1}$ where $\psi_{1}\in A^{p+1,q-1}(V)$. Then $\overline{\partial}(\partial \psi_{1})=-\partial\overline{\partial}\psi_{1}=-\partial^{2}\phi=0$ so that $\overline{\partial}\psi_{1}$ is a $\partial$-closed form in $A^{p+2,q-1}(V)$. We claim that, in the diagram
\begin{equation*}
\vcenter{\xymatrix{
 & H^{q-2}(V,\Omega^{p+3}_{c})\ar[d]^{\delta}\\
H^{q}(V,\Omega^{p})\ar[r]^-{\alpha_{1}}\ar@{-->}[ur]^-{\alpha_{2}}\ar[dr]_-{\beta_{1}} & H^{q-1}(V,\Omega^{p+2}_{c})\ar[d]\\
 & H^{q-1}(V,\Omega^{p+2}),
}}\tag*{(A.8)$_{1}$}\label{art08-app-A-eqA.8-1}
\end{equation*}
$\beta_{1}(\phi)=\partial\psi_{1}$.
\end{proof}

\begin{proof}
We give the argument for $q=2$; this will illustrate how the general case works. Let then $\{U_{\alpha}\}$ be a suitable covering of $V$ with nerve $\mathfrak{U}$, and denote by $C^{q}(\mathfrak{U},S)(Z^{q}(\mathfrak{U},S))$ the $q$-cochains ($q$-cocycles) for $\mathfrak{U}$ with coefficients in a sheaf $S$. Now $\phi\in Z^{2}(\mathfrak{U},\Omega^{p})$, and $\phi=\delta\xi_{1}$ for some $\xi_{1}\in C^{1}(\mathfrak{U},A^{p,0})$ ($A^{p,q}$ being the sheaf of $C^{\infty}(p,q)$ forms). Then $\overline{\partial}\xi_{1}\in Z^{1}(\mathfrak{U},A^{p,1})$ and $\overline{\partial}\xi_{1}=\delta \xi_{2}$ for $\xi_{2}\in C^{0}(\mathfrak{U},A^{p,1})$. Now $\overline{\partial}\xi_{2}\in Z^{0}(\mathfrak{U},A^{p,2})$ and the global form $\xi\in A^{p,2}(V)$ defined by $\xi|U_{\alpha}=\overline{\partial}\xi_{2}$ is a Dolbeault representative in $H^{2}_{\overline{\partial}}(V,\Omega^{p})$ of $\phi$.

Clearly $\partial\xi\in A^{p+1,2}(V)$ is a Dolbeault representative of $\beta_{0}(\phi)\in H^{2}(V,\Omega^{p+1})$, and $\partial \xi=\overline{\partial}\psi_{1}$ for some $\psi_{1}\in A^{p+1,1}(V)$. We want to find a C\^ech cochain $\theta\in C^{1}(\mathfrak{U},\Omega^{p+1})$ with $\delta\theta=\partial\phi$. To do this, we let $\zeta_{2}=\partial\xi_{2}+\psi_{1}\in C^{0}(\mathfrak{U},A^{p+1,1})$. Then $\overline{\partial}\zeta_{2}=-\partial \xi +\overline{\partial}\psi=0$ so that $\zeta_{2}=\partial \lambda_{2}$ for some $\lambda_{2}\in C^{0}(\mathfrak{U},A^{p+1,0})$. We let $\zeta_{1}=\partial \xi_{1}+\delta\lambda_{2}\in C^{1}(\mathfrak{U}, A^{p+1,0})$.\pageoriginale Then $\delta \zeta_{1}=\delta \partial \xi_{1}=\partial \phi$, and $\overline{\partial}\zeta_{1}=-\partial\overline{\partial}\xi_{1}+\delta\overline{\partial}\lambda_{2}=-\partial\overline{\partial}\xi_{1}+\delta\zeta_{2}=-\partial\delta\xi_{2}+\partial \delta\xi_{2}=0$ so that $\theta=\zeta_{1}\in C^{1}(\mathfrak{U},\Omega^{p+1})$. In \ref{art08-app-A-eqA.8-0}, $\alpha_{1}(\phi)\in H^{1}(V,\Omega^{p+2}_{c})$ is represented by $\partial \theta\in Z^{1}(\mathfrak{U},\Omega^{p+1}_{c})$. Observe that $\delta\partial \theta=\delta\partial \zeta_{1}=\delta(\partial^{2}\xi_{1}+\delta\partial \lambda_{2})=0$.

We now want a Dolbeault representative for $\partial \theta\in Z^{1}(\mathfrak{U},\Omega^{p+1})$. Since $\partial\theta=\delta\partial \lambda_{2}$ where $\partial\lambda_{2}\in C^{0}(\mathfrak{U},A^{p+2,0})$, such a representative is given by $\overline{\partial}\partial \lambda_{2}\in Z^{0}(\mathfrak{U},A^{p+2,1})$. But $\overline{\partial}\partial\lambda_{2}=-\partial\overline{\partial}\lambda_{2}=-\partial\zeta_{2}=-\partial (\partial \xi_{2}+\psi_{1})=-\partial\psi_{1}$; that is to say, $-\partial\psi_{1}$ is a Dolbeault representative of $\alpha_{1}(\phi)\in H^{1}(V,\Omega^{p+2})$, which was to be shown.

Now $\beta_{1}(\phi)=0$ by the lemma on K\"ahler manifolds, and so $\partial \psi_{1}=\overline{\partial}\psi_{2}$ where $\psi_{2}\in A^{p+2,q-2}(V)$. Then $\overline{\partial}(\partial\psi_{2})=-\partial\overline{\partial}\psi_{2}=-\partial^{2}\psi_{1}=0$ so that $\partial \psi_{2}$ is a $\overline{\partial}$-closed form in $A^{p+3,q-2}(V)$. As before, we show that, in the diagram,
\begin{equation*}
\vcenter{\xymatrix{
 & H^{q-3}(V,\Omega^{p+4}_{c})\ar[d]^{\delta}\\
H^{q}(V,\Omega^{p})\ar[r]^-{\alpha_{2}}\ar[ur]^-{\alpha_{3}}\ar[dr]_-{\beta_{2}} & H^{q-2}(V,\Omega^{p+3}_{c})\ar[d]\\
 & H^{q-2}(V,\Omega^{p+3}),
}}\tag*{(A.8)$_{2}$}\label{art08-app-A-eqA.8-2}
\end{equation*}
$\beta_{2}(\phi)=\partial \psi_{2}$.

Inductively then we show that $\beta_{k}(\phi)=0$ in $H^{q-k}(V,\Omega^{p+k+1})$ because $\beta_{k}(\phi)=\partial \psi_{k}$ for some $\psi_{k}\in A^{p+k,q-k}(V)$. At the last step, $\beta_{q}(\phi)\equiv 0$ because no holomorphic form on $V$ can be $\partial$-exact. This completes the proof of \eqref{art08-app-A-eqA.9}, and hence of Theorem \ref{art08-app-A-thmA.6}.
\end{proof}

\smallskip
\noindent
{\bf Examples.}~
For $q=0$, the sequence \eqref{art08-app-A-eqA.7} becomes
\begin{equation*}
0\to H^{0}(V,\Omega^{p}_{c})\to H^{0}(V,\Omega^{p})\to 0,\tag{A.11}\label{art08-app-A-eqA.11}
\end{equation*}
which says that {\em every holomorphic $p$-form on $V$ is closed} (theorem of Hodge).

For $p=0$, \eqref{art08-app-A-eqA.10} becomes:
\begin{equation*}
0\to H^{q-1}(V,\Omega^{1}_{c})\to H^{q}(V,\bfC)\xrightarrow{\alpha}H^{q}(V,\mathcal{O})\to 0,\tag{A1.12}\label{art08-app-A-eqA.12} 
\end{equation*}
and\pageoriginale $\alpha$ is just the projection onto $H^{0,q}_{\overline{\partial}}(V)\cong H^{q}(V,\mathcal{O})$ of a class $\phi\in H^{q}(V,\bfC)$. In particular, for $q=1$, we have:
\begin{equation*}
0\to H^{0}(V,\Omega^{1})\to H^{1}(V,\bfC)\to H^{1}(V,\mathcal{O})\to 0.\tag{A.13}\label{art08-app-A-eqA.13} 
\end{equation*}

As a final example, we let $H^{1}(V,\mathcal{O}^{*})$ be the group of line bundles on $V$. Then we have a diagram
\begin{equation*}
\vcenter{\xymatrix{
0\ar[r] & H^{0}(V,\Omega^{2})\ar[r] & H^{1}(V,\Omega^{1}_{c})\ar[r] & H^{1}(V,\Omega^{1})\ar[r] & 0\\
        &                          & H^{1}(V,\mathcal{O}^{*})\ar[u]_{d\log}\ar@{-->}[ur]_{c_{1}}
}}\tag{A1.14}\label{art08-app-A-eqA.14} 
\end{equation*}
(here $c_{1}$ is the usual {\em Chern class mapping}).
\smallskip

(b)~ What we want to show now is that there are natural injections
\begin{equation*}
0\to H^{q}(V,\Omega^{p}_{c})\xrightarrow{\Delta}H^{p+q}(V,\bfC)\tag{A1.15}\label{art08-app-A-eqA.15} 
\end{equation*}
such that
\begin{itemize}
\item[(i)] the following diagram commutes:
\begin{equation*}
\vcenter{\xymatrix{
0\ar[r] & H^{p+q}(V,\bfC)\ar@{=}[r] & H^{p+q}(V,\bfC)\\
 & :\ar[u]_-{\delta} & :\ar@{=}[u]\\
0\ar[r] & H^{q}(V,\Omega^{p}_{c})\ar[u]_-{\delta} \ar[r]^{\Delta} & H^{p+q}(V,\bfC)\ar@{=}[u]\\
0\ar[r] & H^{q-1}(V,\Omega^{p+1}_{c})\ar[u]_-{\delta}\ar[r]^-{\Delta} & H^{p+q}(V,\bfC)\ar@{=}[u]\\
 & :\ar[u]_-{\delta} & :\ar@{=}[u]\\
0\ar[r] & H^{0}(V,\Omega^{p+q}_{c})\ar[u]_-{\delta}\ar[r]^-{\Delta} & H^{p+q}(V,\bfC);\ar@{=}[u]\\
 & 0\ar[u] & 
}}\tag{A1.16}\label{art08-app-A-eqA.16} 
\end{equation*}

\item[(ii)] the\pageoriginale following diagram commutes:
\begin{equation*}
\vcenter{\xymatrix{
H^{q}(V,\Omega^{p}_{c})\ar[d]\ar[r]^-{\Delta} & H^{p+q}(V,\bfC)\ar[d]\\
H^{q}(V,\Omega^{p})\ar[r] & \bfH^{p,q}_{\overline{\partial}}(V),
}}\tag{A.17}\label{art08-app-A-eqA.17} 
\end{equation*}
where $\bfH^{p,q}_{\overline{\partial}}(V)$ is the space of harmonic $(p,q)$ forms;

\item[(iii)] In the filtration $\{F^{p+q}_{m}(V)\}$ of $H^{p+q}(V,\bfC)$ arising from the spectral sequence of \eqref{art08-app-A-eqA.1}, $F^{p+q}_{p}(V)$ is the image of $H^{q}(V,\Omega^{p}_{c})$; and is represented by a $d$-closed form $\phi\in \sum\limits_{r\geq 0}A^{p+r,q-r}(V)$ defined modulo $d\psi$ where 
\begin{equation*}
\psi\in \sum\limits_{r\geq 0}A^{p+r,q-r-1}(V)\quad\text{(c.f. \eqref{art08-app-A-eqA.4}).}\tag{A.18}\label{art08-app-A-eqA.18}
\end{equation*}
\end{itemize}

\noindent
{\bf Proof of (i).}~ This is essentially a tautology; the vertical maps $\delta$ are injections by \eqref{art08-app-A-eqA.7}, and so the requirement of commutativity defines $\Delta : H^{q}(V,\Omega^{p}_{c})\to H^{p+q}(V,\bfC)$. For later use, it will be convenient to have a prescription for finding $\Delta$, both in C\^ech theory and using deRham, and so we now do this.

Let then $\{U_{\alpha}\}$ be a suitable covering of $V$ with nerve $\mathfrak{U}$ and let $\phi\in H^{q}(V,\Omega^{p}_{c})$. Then $\phi$ is defined by $\phi\in Z^{q}(\mathfrak{U},\Omega^{p}_{c})$, and $\phi=d\psi_{1}$ for some $\psi_{1}\in C^{q}(\mathfrak{U},\Omega^{p-1})$. Now $d\delta \psi_{1}=\delta d\psi_{1}=\delta \phi=0$ so that $\phi_{1}=\delta \psi_{1}\in Z^{q+1}(\mathfrak{U}, \Omega^{p-1}_{c})$. In fact, $\phi_{1}=\delta(\phi)$ in \eqref{art08-app-A-eqA.16}. Continuing, we get $\phi_{2}\in Z^{q+2}(\mathfrak{U},\Omega^{p-2}_{c}),\ldots,$ on up to $\phi_{p}\in Z^{p+q}(\mathfrak{U},\bfC)(\bfC=\Omega^{0}_{c})$, where $\phi=\phi_{0}$, $\phi_{k}=\partial \psi_{k}$ with $\phi_{k-1}=d\psi_{k}(\psi_{k}\in C^{q+k-1}(\mathfrak{U},\Omega^{p-k}))$, and then $\Delta(\phi)=\phi_{p}$.

To find the deRham prescription for $\Delta$, we let $A^{s,t}$ be the sheaf of $C^{\infty}$ forms of type $(s,t)$ on $V$ and $B^{s,t}=\sum\limits_{r\geq 0}A^{s+r,t-r}$. Also, $B^{s,t}_{c}$ will be the closed forms. Then $dB^{s,t}\subset B^{s,t+1}_{c}$, and we claim that we have exact sheaf sequences:
\begin{equation*}
0\to B^{s,t}_{c}\to B^{s,t}\xrightarrow{d} B^{s,t+1}_{c}\to 0.\tag{A1.19}\label{art08-app-A-eqA.19}
\end{equation*}

\begin{proof}
Let\pageoriginale $\phi$ be a germ in $B^{s,t+1}_{c}$ and write $\phi=\sum\limits_{r\geq 0}\phi_{s+r,t+1-r}$. Since $d\phi=0$, $\overline{\partial}\phi_{s,t+1}=0$ and so $\phi_{s,t+1}=\overline{\partial}\psi_{s,t}$. Then $\phi-d\psi_{s,t}\in B^{s+1,t}_{c}$, and continuing we find $\psi_{s,t},\ldots,\psi_{s+t,0}$ with $\phi-d(\psi_{s,t}+\cdots+\psi_{s+t,0})\in B^{s+t+1,0}_{c}$. But then $\phi-d(\psi_{s,t}+\cdots+\psi_{s+t,0})$ is a closed holomorphic $s+t+1$-form, and so $\phi-d(\psi_{s,t}+\cdots+\psi_{s+t,0})=d\eta_{s+t,0}$; i.e. $d$ is onto in \eqref{art08-app-A-eqA.19}, which was to be shown.

The exact cohomology sequence of \eqref{art08-app-A-eqA.19} gives:
\begin{equation*}
\left.
\begin{array}{l}
0\to H^{r}(V,B^{s,t+1}_{c})\to H^{r+1}(V,B^{s,t}_{c})\to 0\quad (r\geq 1);\\
0\to H^{0}(V,B^{s,t+1}_{c})/dH^{0}(V,B^{s,t})\to H^{1}(V,B^{s,t}_{c})\to 0.
\end{array}\right\}\tag{A.20}\label{art08-app-A-eqA.20}
\end{equation*}

Using these, we find the following diagram:
\begin{equation*}
\vcenter{\xymatrix@C=.17cm@R=.5cm{
H^{q}(V,\Omega^{p}_{c})\ar@{=}[r] & H^{q}(V,B^{p,0}_{c})\ar@{=}[d]^-{\rotatebox{90}{$\sim$}} & \\
 & H^{q-1}(V,B^{p,1}_{c})\ar@{=}[d]^-{\rotatebox{90}{$\sim$}} & \\
 & :\ar@{=}[d]^-{\rotatebox{90}{$\sim$}} & \\
 & H^{1}(V,B^{p,q-1}_{c})\ar@{=}[r]^-{\sim} & H^{0}(V,B^{p,q}_{c})/dH^{0}(V,B^{p,q-1}); 
}}\tag{A.21}\label{art08-app-A-eqA.21}
\end{equation*}
the composite in \eqref{art08-app-A-eqA.21} gives
\begin{equation*}
0\to H^{q}(V,\Omega^{p}_{c})\xrightarrow{\Delta}B^{p,q}_{c}(V)/dB^{p,q-1}(V)\to 0.\tag{A.22}\label{art08-app-A-eqA.22}
\end{equation*}

This $\Delta$ is just the deRham description of $\Delta$ in \eqref{art08-app-A-eqA.16}, and by writing down \eqref{art08-app-A-eqA.22} we have proved (iii) above.
\end{proof}

\begin{thebibliography}{99}
\bibitem{art08-key1} \textsc{M. F. Atiyah :}\pageoriginale Complex analytic connections in fibre bundles, {\em Trans. Amer. Math. Soc.} {\bf 85} (1957), 181-207.

\bibitem{art08-key2} \textsc{M. F. Atiyah :} Vector bundles over an elliptic curve, {\em Proc. London Math. Soc.} {\bf 7} (1957), 414-452.

\bibitem{art08-key3} \textsc{A. Blanchard :} Sur les vari\'eti\'es analytiques complexes, {\em Ann. \'ecole normale sup\'eriure,} {\bf 73} (1956), 157-202.

\bibitem{art08-key4} \textsc{R. Bott} and \textsc{S. S. Chern :} Hermitian vector bundles and the equidistribution of the zeroes of their holomorphic sections, {\em Acta Mathematica}, {\bf 114} (1966), 71-112.

\bibitem{art08-key5} \textsc{S. S. Chern :} Characteristic classes of Hermitian manifolds, {\em Ann. of Math.} {\bf 47} (1946), 85-121.

\bibitem{art08-key6} \textsc{A. Fr\"ohlicher :} Relations between the cohomology groups of Dolbeault and topological invariants, {\em Proc. Nat. Acad. Sci. (U.S.A.),} {\bf 41} (1955), 641-644.

\bibitem{art08-key7} \textsc{R. Godement :} {\em Th\'eorie des faisceaux}, Hermann (Paris), 1958.

\bibitem{art08-key8} \textsc{P. Griffiths :} The extension problem in complex analysis, II, {\em Amer. Jour. Math.} {\bf 88} (1966), 366-446.

\bibitem{art08-key9} \textsc{P. Griffiths :} Periods of integrals on algebraic manifolds, II, {\em Amer. Jour. Math.} {\bf 90} (1968), 805-865.

\bibitem{art08-key10} \textsc{P. Griffiths :} Periods of integrals on algebraic manifolds, I, {\em Amer. Jour. Math.} {\bf 90} (1968), 568-626.

\bibitem{art08-key11} \textsc{P. Griffiths :} Hermitian differential geometry and positive vector bundles, {\em Notes from the University of Calif., Berkeley.}

\bibitem{art08-key12} \textsc{A. Borel} and \textsc{J.-P. Serre :} Le th\'eor\`eme de Riemann-Roch (d'apr\`e des r\'esultats in\'edits de A. Grothendieck), {\em Bull. Soc. Math. France,} {\bf 86} (1958), 97-136.

\bibitem{art08-key13} \textsc{W.V.D. Hodge} and \textsc{D. Pedoe :} {\em Methods of Algebraic Geometry,} Vol. 3, Cambridge Univ. Press, 1954.

\bibitem{art08-key14} \textsc{K. Kodaira} and \textsc{G. deRham :} Harmonic integrals, {\em mimeographed notes from the Institute for Advanced Study, Princeton.}

\bibitem{art08-key15} \textsc{K. Kodaira}\pageoriginale and \textsc{D. C. Spencer :} On deformations of complex-analytic structures, I-II, {\em Ann. of Math.} {\bf 67} (1958), 328-466.

\bibitem{art08-key16} \textsc{K. Kodaira :} A theorem of completeness of characteristic systems for analytic families of compact subvarieties of complex manifolds, {\em Ann. of Math.} {\bf 75} (1962), 146-162.

\bibitem{art08-key17} \textsc{K. Kodaira :} Green's forms and meromorphic functions on compact analytic varieties, {\em Canad. Jour. Math.} {\bf 3} (1951), 108-128.

\bibitem{art08-key18} \textsc{K. Kodaira :} Characteristic linear systems of complete continuous systems, {\em Amer. Jour. Math.} {\bf 78} (1956), 716-744.

\bibitem{art08-key19} \textsc{S. Lefschetz :} {\em L'analysis situs et la g\'eom\'etrie alg\'ebrique,} Gauthier-Villars (Paris), 1950.

\bibitem{art08-key20} \textsc{D. Lieberman :} Algebraic cycles on nonsingular projective varieties, to appear in {\em Amer. Jour. Math.}

\bibitem{art08-key21} \textsc{J.-P. Serre :} Un th\'eoreme de dualit\'e, {\em Comment. Math. Helv.,} {\bf 29} (1955), 9-26.

\bibitem{art08-key22} \textsc{A. Weil :} On Picard varieties, {\em Amer. Jour. Math.} {\bf 74} (1952), 865-893.

\bibitem{art08-key23} \textsc{A. Weil :} {\em Vari\'eti\'es k\"ahl\'eriennes,} Hermann (Paris), 1958.

\bibitem{art08-key24} \textsc{H. Weyl :} {\em Die idee der Riemannschen fl\"ache,} Stuttgart (Taubner), 1955.
\end{thebibliography}

\noindent
{\small Princeton University,}

\noindent
{\small Princeton, New Jersey,}

\noindent
{\small U.S.A.}
