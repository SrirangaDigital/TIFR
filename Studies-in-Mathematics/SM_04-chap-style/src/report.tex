\chapter[Report]{INTERNATIONAL COLLOQUIUM ON ALGEBRAIC GEOMETRY}

\begin{center}
Bombay, 16-23 January 1968
\end{center}

\pagenumbering{roman}
\setcounter{page}{4}

\bigskip

\begin{center}
{\LARGE\bf REPORT}
\end{center}

\lhead[\thepage]{\textit{Report}}
\rhead[\textit{Report}]{\thepage}

An International Colloquium on Algebraic Geometry was held at the Tata Institute of Fundamental Research, Bombay on 16-23 January, 1968. The Colloquium was a closed meeting of experts and others seriously interested in Algebraic Geometry. It was attended by twenty-six members and thirty-two other participants, from France, West Germany, India, Japan, the Netherlands, the Soviet Union, the United Kingdom and the United States.

The Colloquium was jointly sponsored, and financially supported, by the International Mathematical Union, the Sir Dorabji Tata Trust and the Tata Institute of Fundamental Research. An Organizing Committee consisting of Professor K. G. Ramanathan (Chairman), Professor M. S. Narasimhan, Professor C. S. Seshadri, Professor C. P. Ramanujam, Professor M. F. Atiyah and Professor A. Grothendieck was in charge of the scientific programme. Professors Atiyah and Grothendieck represented the International Mathematical Union on the Organizing Committee. The purpose of the Colloquium was to discuss recent developments in Algebraic Geometry.

The following twenty mathematicians accepted invitations to address the Colloquium: S. S. Abhyankar, M. Artin, B. J. Birch, A. Borel, J. W. S. Cassels, B. M. Dwork, P. A. Griffiths, A Grothendieck, F. Hirzebruch, J.-I. Igusa, Yu. I. Manin, T. Matsusaka, D. Mumford, M. Nagata, M. S. Narasimhan, S. Ramanan, C. S. Seshadri, T. A. Springer, J. L. Verdier and A. Weil. Professor H. Hironaka, who was unable to attend the Colloquium, sent in a paper.

The Colloquium met in closed sessions. There were nineteen lectures in all, each lasting fifty minutes, followed by discussions. Informal lectures and discussions continued during the week, outside the official programme.

The social programme included a tea on 15 January, a dinner on 16 January, a programme of classical Indian dances on 17 January, a dinner at the Juhu Hotel on 20 January, an excursion to Elephanta  on the morning of 22 January and a farewell dinner the same evening.

 

