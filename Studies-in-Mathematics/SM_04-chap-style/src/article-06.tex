\chapter[\textsc{J. W. S. Cassels~:} Rational Points on Curves of Higher Genus]{RATIONAL POINTS ON CURVES OF HIGHER GENUS}\label{art06}

\begin{center}
{\em By}~~ J. W. S. Cassels
\end{center}

\lhead[\thepage]{\textit{Rational Points on Curves of Higher Genus}}
\rhead[\textit{J. W. S. Cassels}]{\thepage}

\setcounter{pageoriginal}{82}
The\pageoriginale old conjecture of Mordell \cite{art06-key3} that a curve of genus greater than 1 defined over the rationals has at most one rational point still defies attack. Recently Dem'janenko \cite{art06-key2} has given a quite general theorem which enables one to prove the existence of only finitely many rational points in a wide variety of cases. In this lecture I show how his theorem is an immediate consequence of the basic properties of heights of points on curves. The details will be published in the Mordell issue of the Journal of the London Mathematical Society \cite{art06-key1}.

\begin{thebibliography}{99}
\bibitem{art06-key1} \textsc{J. W. S. Cassels :} On a theorem of Dem'janenko, {\em J. London Math. Soc.} 43 (1968), 61-66.

\bibitem{art06-key2} \textsc{V. A. Dem'janenko :} Rational points of a class of algebraic curves (in Russian), {\em Izvestija Akad. Nauk} (ser. mat.) 30 (1966), 1373-1396.

\bibitem{art06-key3} \textsc{L. J. Mordell :} On the rational solutions of the indeterminate equation of the third and fourth degrees, {\em Proc. Cambridge Philos. Soc.} 21 (1922), 179-192.
\end{thebibliography}

\medskip
\noindent
{\small Department of Pure Mathematics,}

\noindent
{\small Cambridge, England.}
