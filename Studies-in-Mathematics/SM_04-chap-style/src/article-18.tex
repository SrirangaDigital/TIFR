\chapter[\textsc{C. S. Seshadri~:} Mumford's Conjecture for $GL(2)$ and Applications]{MUMFORD'S CONJECTURE FOR $GL(2)$ AND APPLICATIONS}\label{art18}

\begin{center}
{\em By}~~ C. S. Seshadri
\end{center}

\lhead[\thepage]{\textit{Mumford's Conjecture for $GL(2)$ and Applications}}
\rhead[\textit{C. S. Seshadri}]{\thepage}

\setcounter{pageoriginal}{346}
\textsc{In}\pageoriginale \cite{art18-key12}, it was shown that on a smooth projective curve $X$ of genus $\geq 2$ over $\bfC$, there is a natural structure of a normal projective variety on the isomorphic classes of unitary vector bundles of a fixed rank. This can also be given a purely algebraic formulation, namely that on the classes of semi-stable vector bundles of a fixed rank and degree zero, under a certain equivalence relation, there is a natural structure of a normal projective variety when $X$ is defined over $\bfC$. In fact this was used in \cite{art18-key12}. It is then natural to ask whether this algebraic result holds good in arbitrary characteristic. The main obstacles to extending the proof of \cite{art18-key12} to arbitrary characteristic are as follows:
\begin{enumerate}
\renewcommand{\labelenumi}{(\theenumi)}
\item to carry over the results of Mumford (obtained in characteristic 0) on quotient spaces of the $N$-fold product of Grassmannians for the canonical diagonal action of the full linear group (c.f. \S4, Chap. 4, \cite{art18-key5}), to arbitrary characteristic, and

\item to find a substitute for unitary representations which have been used in \cite{art18-key12}, mainly to show that the varieties in question are complete.
\end{enumerate}

It is not hard to see how to set about (2). One has to show that a certain morphism is proper (see \S\ref{art18-sec3}, Lemma \ref{art18-lem2}). This is not difficult but requires some careful analysis and it is an improvement upon some of the arguments in \cite{art18-key12}. The difficulty (1) appears to be more basic. If Mumford's conjecture generalizing complete reducibility to reductive groups in arbitrary characteristic (cf. \S1, Def. 3) is solved for all special linear groups, (1) would follow. In this we have partial success, namely we solve Mumford's conjecture for $GL(2)$, which allows us to solve (1) for the case of a product of Grassmannians of two planes. Consequently the results of \cite{art18-key12} carry over to the case of vector bundles of {\em rank} 2 in arbitrary characteristic.

The\pageoriginale proof of Mumford's conjecture for $GL(2)$ is rather elementary and we give it in \S\ref{art18-sec1}. As for applications to vector bundles, only the solution of (2) above is given in detail (\S\ref{art18-sec3}, Lemma \ref{art18-lem2}, (3)). The other points are only sketched and proofs for most of these can be found in \cite{art18-key5} or \cite{art18-key12}.

The algebraic schemes that we consider are supposed to be defined over an algebraically closed field $\bfK$ and of finite type over $\bfK$. The points of an algebraic scheme are the geometric points in $\bfK$ and the algebraic groups considered are {\em reduced} algebraic group schemes. By a rational representation of an algebraic group $G$ in a finite dimensional vector space. $V$, we mean a homomorphism $\rho:G\to \Aut V$ of algebraic groups.

\section{Mumford's conjecture for \texorpdfstring{$GL(2)$}{GL2}.}\label{art18-sec1}

~

\medskip

\noindent
{\bf Definition \thnum{1}.\label{art18-defi1}}
{\em An algebraic group $G$ is said to be reductive if it is affine and $\rad G$ (radical of $G$) is a torus, i.e. a product of multiplication groups.}

\medskip
\noindent
{\bf Definition \thnum{2}.\label{art18-defi2}}
{\em An algebraic group $G$ is said to be linearly reductive if it is affine and every rational representation of $G$ in a finite dimensional vector space is completely reducible.}
\smallskip

It is a classical result of H. Weyl that if the characteristic of the base field is zero, every reductive group is linearly reductive. A torus group is easily seen to be linearly reductive in arbitrary characteristic. If the characteristic $p$ of the base field is {\em not zero}, there are not many more linearly reductive groups other than the torus groups; in fact, there is the following result due to Nagata: an algebraic group $G$ is linearly reductive if and only if the connected component $G^{0}$ of $G$ through identity is a torus and the order of the finite group $G/G^{0}$ is prime to $p$ (c.f. \cite{art18-key6}).

It is proved easily that an affine algebraic group $G$ is linearly reductive if and only if any one of the two following properties holds :
\begin{enumerate}
\renewcommand{\labelenumi}{(\theenumi)}
\item for every rational representation of $G$ in a finite dimensional vector space $V$ and a one dimensional $G$-invariant linear subspace $V_{0}$ of\pageoriginale $V$, there exists a $G$-invariant linear subspace $V_{1}$ of $V$ such that $V=V_{0}\oplus V_{1}$;

\item for every rational representation of $G$ in a finite dimensional vector space $V$ and a $G$-invariant point $v\in V$, $v\neq 0$, there exists a $G$-invariant linear form $f$ on $V$ such that $f(v)\neq 0$.
\end{enumerate}

\medskip
\noindent
{\bf Definition \thnum{3}.\label{art18-defi3}}
{\em An algebraic group $G$ is said to be geometrically reductive if it is affine and for every rational representation of $G$ in a finite dimensional vector space $V$ and a $G$-invariant point $v\in V$, $v\neq 0$, there exists a $G$-invariant polynomial $f$ on $V$ such that $f(v)=1$ and $f(0)=0$ or, equivalently, there is a $G$-invariant homogeneous form $f$ on $V$ such that $f(v)=1$.}
\smallskip

Let $G$ be a geometrically reductive algebraic group acting on an affine algebraic scheme $X$ (we can even take $X$ to be an arbitrary affine scheme over the base field $\bfK$, i.e. not necessarily of finite type over $\bfK$) and $X_{1}$, $X_{2}$ two $G$-invariant closed subsets of $X$ such that $X_{1}\cap X_{2}$ is empty. Then there exists a $G$-invariant $f\in A(X=\Spec A)$ such that $f(X_{1})=0$ and $f(X_{2})=1$. This is proved easily as follows : there exists an element $g\in A$ (not necessarily $G$-invariant) such that $g(X_{1})=0$ and $g(X_{2})=1$. Now the translates of $g$ by elements of $G$ span a {\em finite}-dimensional $G$-invariant linear subspace $W$ of $A$. For every $h\in W$, $h(X_{1})=0$ and $h(X_{2})$ is a constant. We have a canonical rational representation of $G$ on $W$ and therefore also on the dual $W^{*}$ of $W$. The canonical inclusion $W\subset A$ defines a $G$-morphism $\phi:X\to W^{*}$ of $X$ into the affine scheme $W^{*}$ (to be strict the affine scheme whose set of geometric points is $W^{*}$) and we have $\phi (X_{1})=0$ and $\phi(X_{2})=w$, $w\neq 0$. Now by the geometric reductivity of $G$, there exists $h$ in the coordinate ring of $W^{*}$ such that $h(0)=0$ and $h(w)=1$. Now if $f$ is the image of $h$ in $A$ by the canonical homomorphism of the coordinate ring of $W^{*}$ in $A$, then $f$ has the required properties.

The following statements are proved easily.
\begin{enumerate}
\renewcommand{\labelenumi}{(\theenumi)}
\item $G$ is geometrically reductive if and only if for every rational representation of $G$ in a finite-dimensional vector space $V$ and a {\em semi-invariant} point $v\in V$, $v\neq 0$ (i.e. the one-dimensional linear subspace\pageoriginale of $V$ spanned by $v$ is $G$-invariant), there is a semi-invariant homogenous form $f$ on $V$ such that $f(v)=1$.

\item $G$ is geometrically reductive if and only if for every rational representation of $G$ in a finite-dimensional vector space $V$, a $G$-invariant linear subspace $V_{0}$ of $V$ of codimension one and $X_{0}$ an element of $V$ such that $X_{0}$ and $V_{0}$ span $V$ and $X_{0}$ is $G$-invariant modulo $V_{0}$, there exists a $G$-invariant $F\in S_{m}(V)$ ($m^{\text{th}}$ symmetric power) for some $m\geq 1$, such that $F$ is monic in $X_{0}$ when $F$ is written with respect to a basis $X_{0}$, $X_{1},\ldots, X_{n}\in V$, $X_{i}\in V_{0}$, $i\geq 1$.

\item Let $N$ be a normal algebraic subgroup of an affine algebraic group $G$ such that $N$ and $G/N$ are geometrically reductive. Then $G$ is geometrically reductive. In particular, a finite product of geometrically reductive groups is geometrically reductive.

\item Let $G$ be a reductive group. Then $G$ is geometrically reductive if and only if $G/\rad G$ is so.

\item A linearly reductive group is geometrically reductive. A finite group is geometrically reductive.
\end{enumerate}

The conjecture of Mumford states that a reductive group is geometrically reductive (c.f. Preface, \cite{art18-key3}). On the other hand it can be shown that a geometrically reductive group is necessarily reductive (c.f. \cite{art18-key8}).

\medskip
\noindent
{\bf Theorem \thnum{1}.\label{art18-thm1}}
{\em The full linear group $GL(2)$ of $2\times 2$ matrices is geometrically reductive.}
\smallskip

\begin{proof}
Let $G$ be an affine algebraic group and $\rho$, $\rho'$ rational representations of $G$ in finite-dimensional vector spaces $W$, $W'$ respectively. Let $\phi:W\to W'$ be a homomorphism of $G$-modules and $w$, $w'$ semi-invariant points of $W$, $W'$ respectively such that $w'=\phi(w)$, $w'\neq 0$. Now if there is a semi-invariant polynomial $f$ on $W'$ such that $f(w')=1$ and $f(0)=0$, then there is a semi-invariant polynomial $g$ on $W$ such that $g(w)=1$ and $g(0)=0$; in fact we can take $g$ to be the image of $f$ under the canonical homomorphism induced by $\phi$ of the coordinate ring of $W'$ into that of $W$. Using this\pageoriginale simple remark, the proof of the geometric reductivity of $GL(2)$ can be divided into the following steps.
\begin{enumerate}
\renewcommand{\labelenumi}{(\theenumi)}
\item It is a well-known fact (c.f. \S1, expos\'e 4, Prop. 4, \cite{art18-key2}) that if $G$ is an affine algebraic group and $\rho$ a rational representation of $G$ in a finite dimensional vector space $W$, then the $G$-module $W$ can be imbedded as a submodule of $A^{n}$ ($n$-fold direct sum of $A$), where $A$ is a submodule of the coordinate ring of $G$, considered as a $G$-module for the regular representation (we should fix the right or the left regular representation). Thus to prove geometric reductivity of $G$, we have only to consider submodules $A$ of the coordinate ring of $G$ such that there exists a semi-invariant $a\in A$, $a\neq 0$.

\item Let $G=GL(n)$, $R$ the coordinate ring of $G$ and $(X_{ij})$, $1\leq i\leq n$, $1\leq j\leq n$, the canonical coordinate functions on $G$. The linear space generated by $X_{ij}$ is a $G$-module and we can identify it with the $G$-module $V^{n}=V\oplus\cdots\oplus V$ ($n$ times), where $V$ is an $n$-dimensional vector space and $G$ is represented as $\Aut V$. Let $\xi$ be the function $\det |X_{ij}|$ and $L$ the 1-dimensional $G$-submodule of $R$ spanned by $\xi$. Now if $W$ is a finite-dimensional linear subspace of $R$, there exists an integer $m\geq 1$ such that for any $g\in W$, $g\xi^{m}$ is a polynomial in $(X_{ij})$. A polynomial in $(X_{ij})$ can be uniquely expressed as a sum of multihomogenous forms in the sets of variables
\begin{align*}
& Y_{1}=(X_{11},X_{21},\ldots,X_{n1}), \ Y_{2}=(X_{12},X_{22},\ldots,X_{n2}),\ldots\\
& Y_{n}=(X_{1n},X_{2n},\ldots,X_{nn})(Y_{i}-i^{\text{th}}\text{~column of~}(X_{ij})).
\end{align*}
The space of multihomogenous forms in $(X_{ij})$ of degree $m_{i}$ in $Y_{i}$ can be identified with the $G$-module $W(m_{1},\ldots,m_{n})$, where
{\fontsize{10pt}{12pt}\selectfont
$$
W(m_{1},\ldots,m_{n})=\bigoplus\limits^{n}_{i=1}S^{m_{i}}(V)(S^{m_{i}}(V)-m^{\text{th}}_{i}\text{~symmetric power of } V).
$$}\relax
Thus if $W$ is a finite dimensional $G$-invariant linear subspace of $R$, $W\otimes L^{(m)}$ can be embedded as a $G$-submodule of a finite direct sum of $G$-modules of the type $W(m_{1},\ldots,m_{n})$, where $L^{(m)}$ denotes the $1$-dimensional $G$-module $L\otimes\cdots\otimes L$ ($m$ times). Thus to prove the geometric reductivity of $GL(n)$, it suffices to consider the $G$-modules\pageoriginale of the form $W(m_{1},\ldots,m_{n})$ such that there is a non-zero semi-invariant element in it.
\end{enumerate}

Now it is easy to see that $W(m_{1},\ldots,m_{n})$ has a non-zero semi-inva\-riant element $v$ if and only if $m_{1}=m_{2}=\ldots=m_{n}=m$ and then that $v$ is in the $1$-dimensional linear subspace spanned by $\xi^{m}$ $(\xi=\det |X_{ij}|)$. This is an immediate consequence of the following remarks:
\begin{itemize}
\item[(i)] every $1$-dimensional $G$-module (given by a rational representation) is isomorphic to $L^{(n)}$ for some $n\in \bfZ$ and

\item[(ii)] the only $G$-invariant elements of $R$ are the scalars.
 
Thus to prove the geometric reductivity of $GL(n)$, we have only to consider the $G$-modules $W(m)$,
{\fontsize{10pt}{12pt}\selectfont
$$
W(m)=W(m,\ldots,m)=\otimes S^{m}(V)(n\text{-fold tensor product of } S^{m}(V))
$$}\relax
with the semi-invariant element being $\xi^{m}$, $\xi=\det |X_{ij}|$.

\item[(iii)] Let $G=GL(2)$. Let $J:W(m)\to S^{2m}(V)$ be the canonical homomorphism, where for an element $f$ in $W(m)$ being considered as a multi-homogeneous polynomial of degree $m$ in $Y_{1}=(X_{11},X_{21})$, $Y_{2}=(X_{12},X_{22})$, $j(f)$ is the homogeneous polynomial of degree $2m$ in two variables obtained by setting $Y_{1}=Y_{2}$. Now $j$ is a $G$-homomorphism. Let $\theta_{m-1}:W(m-1)\to W(m)$ be the homomorphism defined by $\theta_{m-1}(f)=f\xi$, $f\in W(m-1)$. Now $\theta_{m-1}$ is a homomorphism of the underlying $SL(2)$ modules and it ``differs'' from a $GL(2)$ homomorphism only upto a character of $GL(2)$. Consider the following sequence
\begin{equation*}
0\to W(m-1)\xrightarrow{\theta_{m-1}}W(m)\xrightarrow{j}S^{2m}(V)\to 0.\tag{*}
\end{equation*}
We claim that this sequence is exact. It is clear that $\theta_{m-1}$ is injective. Further the kernel of $j$ consists precisely of those polynomials $f$ in $(X_{ij})$ which belong to $W(m)$ and such that $f$ vanishes when we set $(X_{ij})$ to be a singular matrix. Therefore $f=g\xi$, which means that $\ker j=\theta_{m-1}W(m-1)$. Now $\dim W(m)=(m+1)^{2}$, $\dim W(m-1)=m^{2}$ and $\dim S^{2m}(V)=(2m+1)$, so that $\dim W(m)=\dim W(m-1)+\dim S^{2m}(V)$. From this one concludes that (*) is exact.

We\pageoriginale shall now show that the exact sequence (*) has a ``quasi-splitting'', i.e. there is a closed $G$-invariant subvariety of $W(m)$ such that the canonical morphism of this subvariety into $S^{2m}(V)$ is surjective and {\em quasi-finite} i.e. every fibre under this morphism consists only of a finite number of points. Let $D_{m}$ be the subset of $W(m)$ consisting of decomposable tensors, i.e. $D_{m}=\{f|f=g\otimes h, g,h\in S^{m}(V)\}$. Then $D_{m}$ is obviously a $G$-invariant subset of $W(m)$. We have a canonical morphism
$$
\Psi : S^{m}(V)\times S^{m}(V)\to S^{m}(V)\otimes S^{m}(V)=W(m)
$$
and $D_{m}=\Psi(S^{m}(V)\times S^{m}(V))$. From the fact that $\Psi$ is bilinear, we see that $D_{m}$ is the cone over the image of $\Psi'$, where $\Psi'$ is the canonical morphism
$$
\Psi':\bfP(S^{m}(V))\times \bfP(S^{m}(V))\to \bfP(W(m))
$$
induced by $\Psi$, $\bfP$ indicating the associated projective spaces. It follows now that $D_{m}$ is a {\em closed} $G$-invariant subvariety of $W(m)$. The morphism $j_{1}:D_{m}\to S^{2m}(V)$ induced by $j$ is surjective, because every homogeneous form in two variables over an algebraically closed field can be written as a product of linear forms, in particular as a product of two homogeneous forms of degree $m$. We see also easily that $j_{1}:D_{m}\to S^{2m}(V)$ is quasi-finite (it can also be shown without much difficulty that $j_{1}$ is proper so that $j_{1}$ is indeed a {\em finite} morphism but we do not make use of it in the sequel). An element $f\otimes g\in D_{m}$ becomes zero when we set $Y_{1}=Y_{2}$ if and only if $f$ and $g$ are zero, i.e. we have $D_{m}\cap \theta_{m-1}(W(m-1))=(0)$.

\item[(iv)] Let $G=GL(2)$. We shall now show by induction on $m$, that there exists a closed $G$-invariant subvariety $H_{m}$ of $W(m)$ passing through $0$ and not through $\xi^{m}$. This will imply that $GL(2)$ is geometrically reductive.
\end{itemize}

For $m=0$, the assertion is trivial. Let $H_{m-1}$ be a homogeneous $G$-invariant hypersurface of $W(m-1)$ not passing through $\xi_{m-1}$. Let $H_{m}$ be the join of $\theta_{m-1}(H_{m-1})$ and $D_{m}$, i.e.
$$
H_{m}=\{\lambda+\mu|\lambda\in \theta_{m-1}(H_{m-1}), \mu\in D_{m}\}.
$$
We\pageoriginale shall now show that $H_{m}$ is a homogeneous $G$-invariant hypersurface of $W(m)$ not passing through $\xi^{m}$. It is immediate that $\xi^{m}$ is not in $H_{m}$ for if $\xi^{m}=\lambda+\mu$, $\lambda$ in $\theta_{m-1}(H_{m-1})$, $\mu\in D_{m}$, then by setting $Y_{1}=Y_{2}$ since $\xi^{m}$ and $\lambda$ become zero, we conclude that $\mu$ becomes zero. As remarked before, this implies that $\mu$ itself is zero so that $\xi^{m}\in \theta_{m-1}(H_{m-1})$. It would then follow that $\xi^{m-1}\in H_{m-1}$, which leads to a contradiction so that we conclude that $\xi^{m}$ is not in $H_{m}$. The subset $H_{m}$ is $G$-invariant and also invariant under homothecy. Thus to complete the proof of our assertion it suffices to show that $H_{m}$ is closed and of codimension one in $W(m)$. This is an immediate consequence of the following lemma, since $H_{m}$ is the join of the two homogeneous subvarieties $\theta_{m-1}(W(m-1))$ and $D_{m}$ whose common intersection is (0).
\end{proof}

\medskip
\noindent
{\bf Lemma \thnum{1}.\label{art18-lem1}}
{\em Let $Q_{1}$, $Q_{2}$ be closed subvarieties of a projective space $\bfP$ such that $Q_{1}\cap Q_{2}$ is empty. Then the join $Q$ of $Q_{1}$ and $Q_{2}$ is a closed subvariety of $\bfP$ and $\dim Q=\dim Q_{1}+\dim Q_{2}+1$.}
\smallskip

\noindent
{\bf Proof of Lemma.} Let $\Delta$ be the diagonal in $\bfP\times \bfP$ and $R=(\bfP\times \bfP-\Delta)$. If $r=(p_{1}p_{2})$, $p_{i}\in \bfP$, let $L(r)$ be the line in $\bfP$ joining $p_{1}$ and $p_{2}$. Then the mapping $r\to L(r)$ defines a correspondence between $R$ and $\bfP$, and it is seen easily that this is defined by a {\em closed} subvariety of $R\times \bfP$. Since $Q_{1}\cap Q_{2}$ is empty, we have $Q_{1}\times Q_{2}\subset R$. Let $\Gamma_{1}=\pr^{-1}_{1}(Q_{1}\times Q_{2})$, $\pr_{1}$ being the canonical projection of $R\times \bfP$ onto the first factor. Now the join $Q=\pr_{2}(\Gamma_{1})$, $\pr_{2}$ being the projection of $R\times \bfP$ onto the second factor. Since $Q_{1}\times Q_{2}$ is complete, it follows that $Q$ is a closed subvariety of $\bfP$.

We see that $\dim Q_{1}+\dim Q_{2}\leq \dim Q\leq \dim Q_{1}+\dim Q_{2}+1$. Therefore to show that $\dim Q=Q\dim Q_{1}+\dim Q_{2}+1$, it suffices to show that $\dim Q\geq \dim Q_{1}+\dim Q_{2}+1$. Since $Q_{1}\cap Q_{2}$ is empty, we cannot have $\dim \bfP=\dim Q_{1}+\dim Q_{2}$. If $\dim \bfP=\dim Q_{1}+\dim Q_{2}+1$, we see that the lemma is true in this case. Suppose then that $\dim \bfP>\dim Q_{1}+\dim Q_{2}+1$. Then there is a point $p\in \bfP$ which is not in the join $Q$ of $Q_{1}$ and $Q_{2}$. Let us now project $Q_{1}$ $Q_{2}$ and $Q$ from $p$ in a hyperplane $H$ not passing through $p$. Let $Q'_{1}$, $Q'_{2}$ and $Q'$ be the images of $Q_{1}$, $Q_{2}$ and $Q$ respectively in $H$. Then $Q'_{1}$, $Q'_{2}$\pageoriginale and $Q'$ are isomorphic to $Q_{1}$, $Q_{2}$ and $Q$ respectively. Further $Q'_{1}\cap Q'_{2}$ is empty and $Q'$ is the join of $Q'_{1}$ and $Q'_{2}$ in $H$. This process reduces the dimension of the ambient projective space by one. By a repetition of this procedure, we are finally reduced to the case $\dim \bfP=\dim Q_{1}+\dim Q_{2}+1$, in which case the lemma is true as remarked before. This completes the proof of the lemma and consequently the proof of theorem is now complete.

\begin{coro*}
A finite product of algebraic groups of the type $GL(2)$, $SL(2)$ or torus group is geometrically reductive.
\end{coro*}

\begin{remarks*}
\begin{enumerate}
\renewcommand{\labelenumi}{(\theenumi)}
\item That $H$ is a closed $G$-invariant subset of codimension one in $W(m)$ (in the above proof) can also be done by showing that the morphism $\phi:W(m-1)\times D_{m}\to W(m)$, defined by $\phi(w,d)=w+d$ is a surjective finite morphism. The proof that $\dim Q=\dim Q_{1}+\dim Q_{2}+1$ in the above lemma, is due to C. P. Ramanujam.

\item In characteristic 2, the geometric reductivity of $GL(2)$ was proved by Oda (c.f. \cite{art18-key9}).

\item The above proof gives also an analogue of geometric reductivity of $GL(2)$ over $\bfZ$ and consequently for more general ground rings as well.

\item M. S. Raghunathan has pointed out another proof of the existence of a hypersurface in $W(m)(G=GL(2))$ with the required properties. We have an isomorphism of the $GL(2)$-modules $V$ and $V^{*}$ ($V^{*}$ dual of $V$). If $m=p^{\alpha}-1$, $\alpha$ a positive integer, $p$ being the characteristic of the ground field, he points out that $S^{m}(V)\approx S^{m}(V^{*})\approx (S^{m}(V))^{*}$ as $GL(2)$-modules. Therefore in this case
$$
W(m)\approx \Hom (S^{m}(V),S^{m}(V))
$$
and then the hypersurface defined by endomorphisms with zero determinant will have the required properties. Since we can find arbitrarily large integers of the form $p^{\alpha}-1$, the existence of the required hypersurface for arbitrary $m$ follows easily.
\end{enumerate}
\end{remarks*}

\section{Quotient spaces.}\label{art18-sec2}

~
\medskip

\noindent
{\bf Definition \thnum{4}.\label{art18-defi4}}
Let $X$ be an algebraic scheme on which an affine algebraic group $G$ operates. A morphism $\phi : X\to Y$ of algebraic schemes\pageoriginale is said to be a good quotient (of $X$ modulo $G$) if it has the following properties~:

\smallskip
\noindent
{\rm(1)}~ $\phi$ is a surjective affine morphism and is $G$-invariant; {\rm(2)}~ $\phi_{*}(\bfO_{X})^{G}=\bfO_{Y}$ and 
{\rm(3)}~ if $X_{1}$, $X_{2}$ are closed $G$-invariant subsets of $X$ such that $X_{1}\cap X_{2}$ is empty, then $\phi(X_{1})$, $\phi(X_{2})$ are closed and $\phi(X_{1})\cap \phi(X_{2})$ is empty. We say that $\phi$ is a good affine quotient if $\phi$ is a good quotient and $Y$ is affine.
\smallskip

The first two conditions are equivalent to the following : $\phi$ is surjective and for every affine open subset $U$ of $Y$, $\phi^{-1}(U)$ is affine and $G$-invariant and the coordinate ring of $U$ can be identified with the $G$-invariant subring of $\phi^{-1}(U)$. We see then that if $\phi$ is a good affine quotient, $X$ is also affine. The following properties of good quotients are proved quite easily.
\begin{enumerate}
\renewcommand{\labelenumi}{(\theenumi)}
\item The property of being a good quotient is {\em local} with respect to the base scheme, i.e. $\phi$ is a good quotient if and only if there is an open covering $\{U_{i}\}$ of $Y$ such that every $V_{i}=\phi^{-1}(U_{i})$ is $G$-invariant and the induced morphism $\phi_{i}:V_{i}\to U_{i}$ is a good quotient.

\item A good quotient is also a {\em categorical quotient}, i.e. if $\psi:X\to Z$ is a $G$-invariant morphism, there is a unique morphism $\nu : Y\to Z$ such that $\nu\circ \phi=\psi$.

\item {\em Transitivity properties.} Let $X$ be an a one algebraic scheme on which an affine algebraic group $G$ operates. Let $N$ be a normal closed subgroup of $G$ and $H$ the affine algebraic group $G/N$. Suppose that $\phi_{1}:X\to Y$ is a good quotient (resp. good affine quotient) of $X$ modulo $N$. Then we have the following.
\begin{itemize}
\item[(a)] The action of $G$ goes down into an action of $H$ on $Y$.

\item[(b)] If $\phi_{2}:Y\to Z$ is a good quotient (resp. good affine quotient) of $Y$ modulo $H$, then $\phi_{2}\circ \phi_{1}:X\to Z$ is a good quotient (resp. good affine quotient) of $X$ modulo $G$.
\[
\xymatrix@R=.4cm{
X\ar[rr]^{\displaystyle\text{good}}\ar@{-->}[ddr]& & Y\ar[ddl] &\\
 & & \text{good}\\
 & Z &  
}
\]

\item[(c)] If\pageoriginale $\phi : X\to Z$ is a good quotient (resp. good affine quotient) of $X$ modulo $G$, there is a canonical morphism $\phi_{2}:Y\to Z$ such that $\phi=\phi_{2}\circ \phi_{1}$ and $\phi_{2}$ is a good quotient (resp. good affine quotient of $Y$ modulo $H$.
\[
\xymatrix@R=.4cm{
& X\ar[rr]^{\displaystyle\text{good}}\ar[ddr]& & Y\ar[ddl] &\\
\text{good\kern -1cm} & && \\
 && Z &  
}
\]
\end{itemize}

\item If $\phi:X\to Y$ is a good quotient (modulo $G$), $Z$ a {\em normal} algebraic variety on which $G$ operates and $j:Z\to X$ a proper, injective $G$-morphism, then $Z$ has a good quotient modulo $G$; in fact it can be identified with the normalisation of the reduced subvariety $(\phi\circ j)(Z)$ in a suitable finite extension.
\end{enumerate}

\eject

The basic existence theorem on good quotients is the following.

\medskip
\noindent
{\bf Theorem \thnum{2}.\label{art18-thm2}}
{\em Let $X=\Spec A$ be an affine algebraic scheme on which a geometrically reductive algebraic group $G$ operates. Let $Y=\Spec A^{G}$ ($A^{G}$ invariant subring of $A$) and $\phi:X\to Y$ the canonical morphism induced by $A^{G}\subset A$. Then $\phi$ is a good affine quotient.}
\smallskip

For the proof of this theorem, the only non-trivial point is to check that $Y$ is an {\em algebraic} scheme, i.e. $A^{G}$ is an algebra of finite type over $\bfK$ and this is assured by a theorem of Nagata, namely that if $A$ is a $\bfK$-algebra of finite type on which a geometrically reductive group $G$ operates (rationally), then $A^{G}$ is also a $\bfK$-algebra of finite type (c.f. Main theorem, \cite{art18-key6}). The other properties for $\phi$ to be a good quotient, are verified quite easily.

\medskip
\noindent
{\bf Definition \thnum{5}.\label{art18-defi5}}
{\em Let $X$ be a closed subscheme of the projective space $\bfP_{n}$ of dimension $n$. An action of an affine algebraic group $G$ on $X$ is said to be linear if it comes from a rational representation of $G$ in the affine scheme $\bfA_{n+1}$ of dimension $(n+1)$.}
\smallskip

The above definition means that we have an action of $G$ on $\bfA_{n+1}=\Spec \bfK[X_{1},\ldots,X_{n+1}]$ given by a rational representation of $G$ on $\bfA_{n+1}$ and that if $\mathfrak{a}$ is the graded ideal of $\bfK[X_{1},\ldots,X_{n+1}]$ defining $X$, then $\mathfrak{a}$ is left invariant by $G$. We have $X=\Proj R$, $R=\bfK[X_{1},\ldots,X_{n+1}]/\mathfrak{a}$.\pageoriginale We denote by $\widehat{X}$ the cone over $X$ $(\widehat{X}=\Spec R)$ and by (0) the vertex of the cone $\widehat{X}$. The action of $G$ lifts to an action on $\widehat{X}$ and this action and the canonical action of $\bfG_{m}$ on $\widehat{X}$ (homothecy) commute. We observe that the canonical morphism $p:\widehat{X}-(0)\to X$ is a principal fibre space with structure group $\bfG_{m}$ and that $p$ is a good quotient (modulo $\bfG_{m}$).

\medskip
\noindent
{\bf Definition \thnum{5}.\label{art18-defi6}}
{\em Let $X$ be a closed subscheme of $\bfP_{n}$ and let there be given a linear action of an affine algebraic group $G$ on $X$. A point $x\in X$ is said to be semi-stable if for some $\widehat{x}\in \widehat{X}-(0)$ over $x$, the closure (in $\widehat{X}$) of the $G$-orbit through $\widehat{x}$ does not pass through $(0)$. $A$ point $x\in X$ is said to be stable (to be more precise, properly stable) if for some $\widehat{x}\in \widehat{X}-(0)$ over $x$, the orbit morphism $\psi_{\widehat{x}}:G\to \widehat{X}$ defined by $g\to \widehat{x}\circ g$ is proper. We denote by $X^{ss}$ (resp. $X^{s}$) the set of semi-stable (resp. stable) points of $X$.}
\smallskip

We have now the following

\medskip
\noindent
{\bf Theorem \thnum{3}.\label{art18-thm3}}
{\em Let $X$ be a closed subscheme of $\bfP_{n}$ defined by a graded ideal $\mathfrak{a}$ of $\bfK[X_{1},\ldots,X_{n+1}]$ so that $X=\Proj R$, $R=\bfK[X_{1},\ldots,X_{n+1}]/\mathfrak{a}$. Let there be given a linear action of an affine algebraic group $G$ on $X$, $Y=\Proj R^{G}$ and $\phi:X\to Y$ the canonical rational morphism defined by the inclusion $R^{G}\subset R$. Suppose that $G$ is geometrically reductive or, more generally, that the cone $\widehat{X}$ over $X$ has a good affine quotient modulo $G$ (c.f. Theorem \ref{art18-thm1}). Then we have the following:}
\begin{enumerate}
\renewcommand{\labelenumi}{(\rm\theenumi)}
\itemsep=2pt
\item {\em $x\in X^{ss}$ if and only if there is a homogeneous $G$-invariant element $f\in R_{+}$($R_{+}$ being the subring of $R$ generated by homogeneous elements of degree $\geq 1$) such that $f(x)\neq 0$ (in particular, $X^{ss}$ is open in $X$ and $\phi$ is defined at $x\in X^{ss}$).}

\item {\em $\phi:X^{ss}\to Y$ is a good quotient and $Y$ is a projective algebraic scheme.}

\item {\em $X^{s}$ is a $\phi$-saturated open subset, i.e. there exists an open subset $Y^{s}$ of $Y$ such that $X^{s}=\phi^{-1}(Y^{s})$ and $\phi:X^{s}\to Y^{s}$ is a geometric quotient, i.e. distinct orbits of $X^{s}$ go into distinct points of $Y^{s}$.}
\end{enumerate}

This\pageoriginale theorem is proved quite easily.

Let $H_{p,r}(E)$ denote the Grassmannian of $r$-dimensional quotient linear spaces of a $p$-dimensional vector space $E$. We have a canonical immersion of $H_{p,r}(E)$ in the projective space associated to ${\displaystyle{\mathop{\wedge}\limits^{p-r}}}E$ and if $X=H^{N}_{p,r}(E)$ denotes the $N$-fold product of $H_{p,r}(E)$, we have a canonical projective immersion of $X$, namely the Serge imbedding associated to the canonical projective imbedding of $H_{p,r}(E)$. There is a natural action of $GL(E)=\Aut E$ on $H_{p,r}(E)$ and this induces a natural action (the diagonal action) of $GL(E)$ on $H^{N}_{p,r}(E)$. The restriction of this action to the subgroup $G=SL(E)$ is a linear action with respect to the canonical projective imbedding of $X$. We denote by $X^{ss}$ (resp. $X^{s}$) the set of semi-stable (resp. stable) points of $X$ with respect to the canonical projective imbedding of $X$. Let $R$ be the projective coordinate ring of $X$, $Y=\Proj R^{G}$, $\widehat{X}$ the cone over $X$ and $\phi:X\to Y$ the canonical rational morphism as in Theorem \ref{art18-thm3} above. Then the result to be applied for the classification of vector bundles on an algebraic curve is as follows.

\medskip
\noindent
{\bf Theorem \thnum{4}.\label{art18-thm4}}
{\em Let $X=H^{N}_{p,r}(E)$ with $1\leq r\leq 2$. Then for the canonical action of $G=SL(E)$ on the cone $\widehat{X}$ over $X$, $\widehat{X}$ has a good affine quotient modulo $G$ so that by Theorem \ref{art18-thm3}, the rational morphism $\phi:X\to Y$ has the properties {\rm(1), (2)} and {\rm(3)} of Theorem \ref{art18-thm3}; in particular, $\phi:X^{ss}\to Y$ is a good quotient and $Y$ is a projective algebraic scheme.}

{\em Further for $x\in X$, $x=\{E_{i}\}_{1\leq i\leq N}$, $E_{i}$ a quotient linear space of dimension $r$ of $E$, $x\in X^{ss}$ (resp. $X^{s}$) if and only if for every linear subspace (resp. proper linear subspace) $F$ of $E$, if $F_{i}$ denotes the canonical image of $F$ in $E_{i}$, we have}
$$
\frac{\frac{1}{N}\sum\limits^{N}_{i=1}\dim F_{i}}{r}\geq \frac{\dim F}{p}(resp. >)
$$

\smallskip
\noindent
{\bf Indication of proof.} Let $W$ be the space of $(p\times r)$ matrices. Then we have canonical commuting operations of $GL(p)$ and $GL(r)$ on $W$. Let $W^{N}$ be the $N$-fold product of $W$ and $\sigma_{1}$ the canonical diagonal action of $GL(p)$ on $W^{N}$. Let $\sigma$ be the induced action of $SL(p)$\pageoriginale on $W^{N}$. We have a natural action $\tau_{1}$ of $GL(r)^{N}$ on $W^{N}$. Let $H$ be the subgroup of $GL(r)^{N}$ defined by elements $(g_{1},\ldots,g_{N})$ such that $\prod\limits^{N}_{i=1}\det g_{i}=1$ and $\tau$ the restriction of the action $\tau_{1}$ to $H$. We note that $H/(SL(r))^{N}$ is a torus group. Therefore $H$ is geometrically reductive since $1\leq r\leq 2$. Then in view of Theorem \ref{art18-thm3} and the {\em transitivity} properties of good quotients, for proving the first part of the theorem, it suffices to show that a good quotient of $W^{N}$ exists, respectively for the actions of $SL(p)$ and $H$, and that the good quotient of $W^{N}$ modulo $H$ can be identified with the cone $\widehat{X}$ over $X$.
\[
\xymatrix@=2.2cm{
W^{N}\ar[d]_{\displaystyle\text{good}}^-{SL(p)} \ar[r]^{H}_{\displaystyle\text{good}}\ar[dr]_-{\displaystyle\text{good}} & \widehat{X}\ar[d]^-{SL(p)}\\
(*)\ar[r]^-{H}_{\displaystyle\text{good}} & \widehat{Y}
}
\]
These last two statements follow easily from the facts that for arbitrary $r$ (i.e. without assuming $1\leq r\leq 2$), a good quotient of $W$ modulo the canonical action of $SL(r)$ exists and that it can be identified with the cone over $H_{p,r}(E)$. These facts can be checked explicitly, using a result of Igusa that $H_{p,r}(E)$ is projectively normal (c.f. \cite{art18-key4}).

The proof of the last part of the theorem is the same as in \S4, Chap. 4, \cite{art18-key5} and we remark that for this it is not necessary to suppose that $1\leq r\leq 2$. It should be noted that our definition of stable and semi-stable points differs, {\em a priori}, from that of \cite{art18-key4}, when the group is not geometrically reductive and that the computations of \S4, \cite{art18-key4} hold in arbitrary characteristic for {\em reductive} groups provided we take the definition of stable and semi-stable points in our sense.

\section{Vector bundles over a smooth projective curve.}\label{art18-sec3}

~
\smallskip
\noindent
Let $X$ be a smooth projective curve over $\bfK$. Let us suppose that the {\em genus $g$ of $X$ is $\geq 2$}. By a vector bundle $V$ over $X$, we mean an algebraic vector bundle;\pageoriginale we denote by $d(V)$ the degree of $V$ and by $r(V)$ the rank of $V$. We fix a very ample line bundle $L$ on $X$, let $l=d(L)$. If $V$ is a vector bundle (resp. coherent sheaf) on $X$, we denote by $V(m)$, the vector bundle (resp. coherent sheaf) $V\otimes L^{m}$, where $L^{m}$ denotes the $m$-fold tensor product of $L$. If $F$ is a coherent sheaf on $X$, the {\em Hilbert polynomial} $P=P(F,m)$ of $F$ is a polynomial in $m$ with rational coefficients, defined by

$P(m)=P(F,m)=\chi(F(m))=\dim H^{0}(F(m))-\dim H^{1}(F(m))$. If $F$ is the coherent sheaf associated to a vector bundle $V$, we have
$$
P(m)=d(V(m))-r(V)(g-1)=d(V)+r(V)(ml-g+1).
$$

We recall that a vector bundle $V$ on $X$ is said to be {\em semi-stable} (resp. {\em stable}) if for every sub-bundle $W$ of $V$ (resp. proper sub-bundle $W$ of $V$), we have
$$
r(V)d(W)\leq r(W)d(V)(\text{resp.~} r(V)d(W)<r(W)d(V)).
$$
Let $\alpha$ be a positive rational number and $\bfS(\alpha)$ the category of semi-stable vector bundles $V$ on $X$ such that $d(V)=\alpha r(V)$. Then $\bfS(\alpha)$ is an abelian category and the Jordan-H\"older theorem holds in this category (c.f. Prop. 3.1, \cite{art18-key12} and Prop. 1, \cite{art18-key10}). For $V\in \bfS(\alpha)$, we denote by $\gr V$ the associated graded object; now $\gr V$ is a direct sum of stable bundles $W$ such that $d(W)=\alpha r(W)$ (we note that $\gr V$ is not a well-determined object of $\bfS(\alpha)$, it is determined only upto isomorphism). Let $\bfS(\alpha,r)$ be the sub-category of $\bfS(\alpha)$ consisting of $V\in \bfS(\alpha)$ such that $r(V)=r$. It can be proved that $\bfS(\alpha,r)$ is {\em bounded}, i.e. there is an algebraic family of vector bundles on $X$ such that every $V\in \bfS(\alpha,r)$ is found in this family (upto isomorphism). We can then find an integer $m$ such that $H^{0}(V(m))$ generates $V(m)$ and $H^{1}(V(m))=0$ for all $V\in \bfS(\alpha,r)$. {\em We fix such an integer $m$ in the sequel}. Let $E$ be the trivial vector bundle on $X$ of rank
$$
p=r(\alpha+lm-g+1).
$$
If $V\in \bfS(\alpha,r)$, then $\dim H^{0}(V(m))=p$, $V(m)$ is a quotient bundle of $E$ and the Hilbert polynomial $P$ of $W=V(m)$ is given by $P(n)=r(\alpha+lm+ln-g+1)$, $P(0)=p$. The Hilbert polynomial is the same for all $V(m)$, $V\in \bfS(\alpha,r)$. Let $Q(E/P)=\Quot (E/P)$ be the\pageoriginale Grothendieck scheme of all $\beta:E\to F$, where $F$ is a coherent sheaf on $X$; $\beta$ makes $F$ a quotient of $E$ and the Hilbert polynomial of $F$ is the above $P$; then $Q(E/P)$ is a projective algebraic scheme (c.f. Theorem 3.2, \cite{art18-key3}). If $q\in Q(E/P)$, we denote by $F_{q}$ the coherent sheaf which is a quotient of $E$, represented by $q$. Let $R$ be the subset of $Q(E/P)$ determined by points $q\in Q(E/P)$ such that (i) $F_{q}$ is locally free and (ii) the canonical mapping $\beta_{q}:E\to H^{0}(F_{q})$ is surjective. If follows easily that for $q\in R$, $\beta_{q}$ is indeed an isomorphism and that $H^{1}(F_{q})=0$. It can be shown that $R$ is an {\em open, smooth and irreducible} subscheme of $Q(E/P)$ of dimension $(p^{2}-1)+(r^{2}(g-1)+1)$ invariant under the canonical operation of $\Aut E$ on $Q(E/P)$ and that for $q_{1}$, $q_{2}\in R$, $F_{q_{1}}$ is isomorphic to $F_{q_{2}}$ if and only if $q_{1}$, $q_{2}$ lie in the same orbit under $GL(E)=\Aut E$ (c.f. \S6, \cite{art18-key12} and \S5 a, \cite{art18-key10}). for $q\in R$, $F_{q}$ is locally free and is therefore the sheaf of germs of a vector bundle; let $R^{ss}$ (resp. $R^{s}$) denote the subset of $R$ consisting of $q$ such that (the bundle associated to) $F_{q}$ is semi-stable (resp. stable). Let $\mathfrak{n}$ be an ordered set of $N$ distinct points $P_{1},\ldots,P_{N}$ on $X$. Let $\tau_{i}:R\to H_{p,r}(E)$ be the morphism into the Grassmannian of $r$-dimensional quotient linear spaces of $E$ (considered canonically as a vector space of dimension $p$), which assigns to $q\in R$, the fibre at $P_{i}$ of the vector bundle associated to $F_{q}$, considered canonically as a quotient linear space of $E$. Let
$$
\tau : R\to H^{N}_{p,r}(E)
$$
be the $GL(E)$-morphism defined by $\tau=\{\tau_{i}\}_{1\leq i\leq N}$. Then we have the following basic

\medskip
\noindent
{\bf Lemma \thnum{2}.\label{art18-lem2}}
{\em Given the category $\bfS(\alpha,r)$, we can then find an integer $m$ and an ordered set $\mathfrak{n}$ of $N$ points on $X$ as above such that the morphism $\tau:R\to H^{N}_{p,r}(E)=Z$ has the following properties:}
\begin{enumerate}
\renewcommand{\labelenumi}{\rm(\theenumi)}
\item {\em $\tau$ is injective;}

\item {\em $\tau(R^{ss})\subset Z^{ss}$ and for $q\in R^{ss}$, $\tau(q)$ is stable if and only if $F_{q}$ is a stable vector bundle;}

\item {\em the induced morphism $\tau :R^{ss}\to Z^{ss}$ is proper.}
\end{enumerate}

\begin{remark*}
It\pageoriginale can indeed be shown that $\tau :R^{ss}\to Z^{ss}$ is a closed immersion for a suitable choice of $m$ and $\mathfrak{n}$.
\end{remark*}

Excepting (3), the other assertions have been proved before (\S7, \cite{art18-key12}). We shall now give a proof of (3).

Let $R_{1}$ be the subset of $Q(E/P)$ consisting of points $q\in Q(E/P)$ such that the corresponding coherent sheaf $F_{q}$ is locally free. Then $R\subset R_{1}$ and $R_{1}$ is an open subscheme of $Q(E/P)$ invariant under $GL(E)$ (c.f. Prop. 6.1, \cite{art18-key12}). Let $\mathfrak{n}$ be an ordered set of $N$ points $P_{1},\ldots,P_{N}$ on the curve $X$. Let $\tau_{i}:R_{1}\to H_{p,r}(E)$ be the morphism (extending the above $\tau_{i}$) into the Grassmannian of $r$-dimensional quotient linear spaces of $E$ which assigns to $q\in R_{1}$, the fibre of the vector bundle associated to $F_{q}$ at the point $P_{i}$, considered canonically as a quotient linear space of $E$. Let $\tau:R_{1}\to H^{N}_{p,r}(E)$ be the morphism defined by $\tau=\{\tau_{i}\}_{1\leq i\leq N}$.

We shall now extend the morphism $\tau:R_{1}\to H^{N}_{p,r}(E)$ to a {\em multivalued} (set) mapping of $Q(E/P)$ into $H^{N}_{p,r}(E)$ and we shall denote this extension by $\Phi=\{\Phi_{i}\}_{1\leq i\leq N}$. Suppose now that for $q\in Q(E/P)$, $F_{q}$ is not locally free. Then we have $F_{q}=V_{q}(m)\oplus T_{q}$, where $T_{q}$ is a torsion sheaf and $V_{q}$ is locally free (this decomposition holds because $X$ is a non-singular curve). Suppose that $P_{i}\not\in$ Support of $T_{q}$. We then define $\Phi_{i}(q)\in H_{p,r}(E)$ as the fibre of the bundle $V_{q}(m)$ at $P_{i}$ considered canonically as a quotient linear space of dimension $r$ of $E$. Suppose that $P_{i}\in$ Support of $T_{q}$; we then define $\Phi_{i}(q)$ to be any point of $H_{p,r}(E)$. We thus obtain a multivalued (set) mapping $\Phi_{i}:Q(E/P)\to H_{p,r}(E)$ and we define $\Phi=\{\Phi_{i}\}_{1\leq i\leq N}$. We claim now that $\Phi_{i}$ is a morphism in a neighbourhood of $q\in Q(E/P)$ if and only if $P_{i}\not\in$ Support of $T_{q}$. For this it suffices to show that given $q_{0}\in Q(E/P)$ such that $P_{i}\not\in$ Support of $T_{q_{0}}$, there is a neighbourhood $U$ of $q_{0}$ such that $P_{i}\not\in$ Support of $T_{q}$ for any $q$ in $U$. We observe that $F_{q_{0}}$ is locally free in a neighbourhood of $P_{i}$ and therefore if $F$ is the coherent sheaf on $X\times Q(E/P)$, which is a quotient of $E$ and defines the family $\{F_{q}\}$, it follows by Lemma 6.1, \cite{art18-key12}, that $F_{q}$ is locally free in a neighbourhood of $(P_{i}\times q_{0})\in X\times Q(E/P)$. From this the existence of a neighbourhood $U$ as required above follows easily\pageoriginale and our claim is proved. It is now immediate that the graph of $\Phi_{i}$ in $Q(E/P)\times H_{p,r}(E)$ is closed and that it contains the closure of the graph of $\tau_{i}:R_{1}\to H_{p,r}(E)$ in $Q(E/P)\times H_{p,r}(E)$. From this it follows easily that the graph of $\Phi$ in $Q(E/P)\times H^{N}_{p,r}(E)$ contains the closure of the graph of $\tau:R_{1}\to H^{N}_{p,r}(E)$ in $Q(E/P)\times H_{p,r}(E)$. Then we claim that

\medskip
\noindent
{\bf Claim \thnum{(A)}.\label{art18-claim-A}}
$m$ and $N$ can be so chosen that for $q\in Q(E/P)$, $\Phi(q)$ is semi-stable (resp. stable) if and only if $q\in R^{ss}$ (resp. $R^{s}$).
\smallskip

Let us first show how (A) implies (3) of Lemma \ref{art18-lem2}. Let us denote by the same letter $\Phi$, the graph of the multivalued set mapping $\Phi:Q(E/P)\to H^{N}_{p,r}(E)$. Let $\Gamma$ be the graph of the morphism $\tau :R^{ss}\to H_{p,r}(E)^{ss}$ and $\Psi$ the closure of $\Gamma$ in $Q(E/P)\times H^{N}_{p,r}(E)$. Now (A) implies that $\Phi\supset \Psi$ and that
$$
\Phi\cap (Q(E/P)\times H^{N}_{p,r}(E)^{ss})=\Gamma.
$$
This implies that
$$
\Psi \cap (Q(E/P)\times H^{N}_{p,r}(E)^{ss})=\Gamma
$$
since $\Phi\supset \Psi\supset \Gamma$. Since $\Psi$ is closed, this means that $\Gamma$, which by definition is closed in $R^{ss}\times H^{N}_{p,r}(E)^{ss}$, is also closed in $Q(E/P)\times H^{N}_{p,r}(E)^{ss}$. Since $Q(E/P)$ is projective, in particular complete, the canonical projection of $Q(E/P)\times H^{N}_{p,r}(E)^{ss}$ onto $H^{N}_{p,r}(E)^{ss}$ is proper and this implies that
$$
\tau :R^{ss}\to H^{N}_{p,r}(E)^{ss}
$$
is proper.

We shall now prove (\ref{art18-claim-A}). In view of (2) of Lemma \ref{art18-lem2} which has been proved in \S7, \cite{art18-key12}, it suffices to prove the following:

\medskip
\noindent
{\bf Claim \thnum{(B)}.\label{art18-claim-B}}
$m$ and $N$ can be so chosen that for $q\in Q(E/P)$, $q\not\in R^{ss}$, $\Phi(q)$ is not a semi-stable point of $H^{N}_{p,r}(E)$.
\smallskip

Let $\bfF(r)$ denote the category of all indecomposable vector bundles $V$ on $X$ such that $r(V)\leq r$ and $d(V)\geq -\gr (V)$. From the fact that the family of all indecomposable vector bundles on $X$ of a given rank and degree is bounded (c.f. p. 426, Th. 3, \cite{art18-key1}), it is deduced easily that there is an integer $m_{0}$ such that for $m\geq m_{0}$, $\forall V\in \bfF(r)$, $H^{1}(V(m))=0$\pageoriginale and $H^{0}(V(m))$ generates $V(m)$ (i.e. the canonical mapping of $H^{0}(V(m))$ onto the fibre of $V(m)$ at every point of $X$ is surjective). In the following we fix a positive integer $m$ such that $m\geq m_{0}$.

If $q\in Q(E/P)$, we have $F_{q}=\bfV_{q}(m)\oplus T_{q}$, where $T_{q}$ is a torsion sheaf and $\bfV_{q}$ is the coherent sheaf associated to a vector bundle $V_{q}$. We denote by $p_{1}$ the natural projection of $H^{0}(F_{q})$ onto $H^{0}(V_{q}(m))$. For proving (\ref{art18-claim-B}), we require the following :

\medskip
\noindent
{\bf Claim \thnum{(C)}.\label{art18-claim-C}}
If $q\not\in R^{ss}$, there is a proper linear subspace $K$ of $E$ (i.e. $K\neq E$, $K\neq (0)$) and a sub-bundle $W_{q}(m)$ of $V_{q}(m)$ ($W_{q}(m)$ could reduce to $0$) such that
\begin{itemize}
\item[(i)] $(p_{1}\circ \beta_{q})(K)\subset H^{0}(W_{q}(m))$ and $(p_{1}\circ \beta_{q})(K)$ generates $W_{q}(m)$ generically (i.e. there is at least one point $P$ of $X$ such that $(p_{1}\circ \beta_{q})(K)$ generates the fibre of $W_{q}(m)$ at $P$; we recall that $\beta_{q}$ is the canonical mapping $E\to H^{0}(F_{q})$) and,

\item[(ii)]\hfill $\dfrac{r(W_{q}(m))}{\dim K}-\dfrac{r}{p}<0$.\hfill\,

\smallskip
We shall now prove (\ref{art18-claim-C}) and this proof is divided into two cases.
\end{itemize}

\begin{description}
\item[Case (i)] $q\not\in R$. Suppose that $\Ker (p_{1}\circ \beta_{q})\neq 0$. Then we set $K=\Ker (p_{1}\circ \beta_{q})$ and $W_{q}(m)=(0)$. Then $K$ generates $W_{q}(m)$ and the inequality $(b)$ is obviously satisfied.
\end{description}

Suppose then that $\Ker (p_{1}\circ \beta_{q})=0$, i.e. $p_{1}\circ \beta_{q}:E\to H^{0}(Vq(m))$ is injective. Suppose further that for every indecomposable component $V_{i}(m)$ of $V_{q}(m)$, we have
$$
d(V_{i})\geq -\gr (V_{i}).
$$
Then by our choice of $m$, we have $H^{1}(V_{q}(m))=0$ and $H^{0}(V_{q}(m))$ generates $V_{q}(m)$. For the torsion sheaf $T_{q}$, we have $T_{q}(n)=T_{q}$ for all $n$ and $H^{1}(T_{q})=0$. It follows then that $H^{1}(F_{q}(n))=0$ for every $n\geq 0$. Then we have $P(n)=H^{0}(F_{q}(n))$ for every $n\geq 0$; in particular $p=\dim H^{0}(F_{q})$. But since $p_{1}\circ \beta_{q}:E\to H^{0}(V_{q}(m))$ is injective and $p=\dim E$, we conclude that $H^{0}(T_{q})=0$. Since $T_{q}$ is a torsion sheaf, this implies that $T_{q}=(0)$, i.e. that $F_{q}$ is locally free.\pageoriginale Further it follows that $\beta_{q}:E\to H^{0}(F_{q})$ is an isomorphism so that $q\in R$, which is a contradiction.

We can therefore suppose that there is at least one indecomposable component $V_{i}(m)$ of $V_{q}(m)$ such that
$$
d(V_{i})<-gr(V_{i}).
$$
Let $V_{q}(m)=W_{1}(m)\oplus W_{2}(m)$ such that for every indecomposable component $U(m)$ of $W_{1}(m)$, we have $d(U)\geq - gr(U)$ and for every indecomposable component $S(m)$ of $W_{2}(m)$, we have $d(S)<-gr(S)$. We note that since $F_{q}$ is a quotient of $E$ and $F_{q}=\bfV_{q}(m)\oplus T_{q}$, $V_{q}(m)$ is generated by its global sections; consequently $W_{1}(m)$ and $W_{2}(m)$ are also generated respectively by their global sections. If $G$ is a vector bundle on $X$ generated generically by $H^{0}(G)$, it can be shown easily (c.f. Lemma 7.2, \cite{art18-key12}) that
$$
\dim H^{0}(G)\leq d(G)+r(G)
$$
and by applying this it follows easily that
$$
\dim H^{0}(W_{2}(m))<r(W_{2})(lm-g+1).
$$
We see then that there is a linear subspace $K$ of $E(\approx H^{0}(E))$ such that $(p_{1}\circ \beta_{q})(K)\subset H^{0}(W_{1}(m))$ and
$$
\dim K>p-r(W_{2})(lm-g+1)=r(W_{1})(lm-g+1)=\dfrac{r(W_{1})}{r}p.
$$
This shows that
$$
\frac{r}{p}>\frac{r(W_{1}(m))}{\dim K}.
$$
Let $W_{q}(m)$ be the sub-bundle of $W_{1}(m)$ generated generically by $K$ (through $p_{1}\circ \beta_{q}$). Then we have
$$
\dfrac{r(W_{q}(m))}{\dim K}<\dfrac{r(W_{1}(m))}{\dim K}.
$$
Therefore, we have
$$
\frac{r(W_{q}(m))}{\dim K}-\dfrac{r}{p}<0.
$$
This proves (\ref{art18-claim-C}) in Case (i).

\begin{description}
\item[Case (ii)] $q\in R$,\pageoriginale $q\not\in R^{ss}$. We have $F_{q}=\bfV_{q}(m)$, $V_{q}$ being not semi-stable. We see easily that there exists a {\em stable} sub-bundle $W_{q}(m)$ of $V_{q}(m)$ such that
$$
\dfrac{d(W_{q})}{r(W_{q})}>\dfrac{d(V_{q})}{r(V_{q})}=\alpha > 0.
$$
The bundle $W_{q}$ is indecomposable and therefore $W_{q}\in \bfF(r)$. Therefore by our choice of $m$, $H^{1}(W_{q}(m))=0$ and $H^{0}(W_{q}(m))$ generates $W_{q}(m)$. We have also $H^{1}(V_{q}(m))=0$ and $\beta_{q}:E\to H^{0}(V_{q}(m))$ is an isomorphism. We set $K=H^{0}(W_{q}(m))$. Then by applying the Riemann-Roch theorem, we get
\begin{align*}
\dfrac{\dim K}{r(W_{q})} &=\dfrac{d(W_{q}(m))}{r(V_{q})}-g+1=\dfrac{d(W_{q})}{r(W_{q})}+ml-g+1,\\
\frac{p}{r} &= \dfrac{d(V_{q}(m))}{r(V_{q})}-(g-1)=\dfrac{d(V_{q})}{r(V_{q})}+ml-g+1.
\end{align*}
Since $\dfrac{d(W_{q})}{r(W_{q})}>\dfrac{d(V_{q})}{r(V_{q})}$, we get $\dfrac{r(W_{q})}{\dim K}-\dfrac{r}{p}<0$. This completes the proof of (\ref{art18-claim-C}).
\end{description}

We shall now show that (\ref{art18-claim-C}) implies (\ref{art18-claim-B}). Let $q\in Q(E/P)$, $q\not\in R^{ss}$. If $L$ is a subspace of $E$, we denote by $L_{i}$ the canonical image of $L$ in the quotient linear space of $E$ represented by $\Phi_{i}(q)$. Let
$$
\rho(L)=\dfrac{\frac{1}{N}\sum\limits^{N}_{i=1}\dim L_{i}}{\dim L}-\dfrac{r}{p}.
$$
Then (\ref{art18-claim-B}) would follow if we show that there is a proper subspace $K$ of $E$ such that $\rho(K)<0$ (see the last assertion of Th. \ref{art18-thm4}). Take a proper linear subspace $K$ of $E$ as provided by (\ref{art18-claim-C}) above. Then we have
$$
\rho(K)=\dfrac{r(W_{q}(m))}{\dim K}-\dfrac{r}{p}<0.
$$
We have
\begin{equation*}
|\mu (K)-\rho(K)|\leq \sum\limits^{N}_{i=1}|r(W_{q})-\dim K_{i}|\tag{a}\label{art18-eq-a}
\end{equation*}
since\pageoriginale $\dim K\geq 1$. Now to estimate the right side, we should consider those $i$ for which $r(W_{q})-\dim K_{i}$ could be different from zero. This could occur for $i$ such that $P_{i}\in $ Support of $T_{q}$ or $P_{i}\not\in $ Support of $T_{q}$. Suppose that $P_{i}\in$ Support of $T_{q}$ and $r(W_{q})-\dim K_{i}\neq 0$. Then $K$ does {\em not} generate the fibre of $W_{q}(m)$ at $P_{i}$. The number of distinct points of $X$ where $K$ does not generate the fibre of $W_{q}(m)$ is at most $d(W_{q}(m))$ (c.f. Lemma 7.1, \cite{art18-key12}). From these facts, we deduce that
\begin{equation*}
|\mu(K)-\rho(K)|\leq \dfrac{r(d(W_{q}(m)))+\text{Card} (\text{Support of } T_{q})}{N}.\tag{b}\label{art18-eq-b}
\end{equation*}
Now for $F_{q}=\bfV_{q}(m)\oplus T_{q}$, by applying the Riemann-Roch theorem, we get for sufficiently large $n$ that
$$
\dim H^{0}(F_{q}(n))=\lambda+nrl+\dim H^{0}(T_{q})-r(g-1)
$$
where $d(V_{q}(m))=\lambda > 0$ ($\lambda$ is positive because $V_{q}(m)$ is generated by its global sections). On the other hand, for sufficiently large $n$,
$$
\dim H^{0}(F_{q}(n))=P(n)=r(\alpha+lm+ln-g+1).
$$
Therefore we obtain that
\begin{equation*}
\dim H^{0}(T_{q})+\lambda=r(ml+\alpha).\tag{c}\label{art18-eq-c}
\end{equation*}
Since Card (Support of $T_{q}$) $\leq \dim H^{0}(T_{q})$ and $\lambda\leq 0$, we get that
\begin{equation*}
\Card (\text{Support of } T_{q})\leq r(ml+\alpha).\tag{d}\label{art18-eq-d}
\end{equation*}

We note that the family of vector bundles $\{V_{q}(m)\}$, $q\in Q(E/P)$ is a bounded family. This could be seen as follows. The degree of every indecomposable component of $V_{q}(m)$ is positive, in particular, bounded below, because $V_{q}(m)$ is generated by global sections. On the other hand from (\ref{art18-claim-C}) above we see that
$$
\lambda+d(V_{q}(m))\leq r(ml+\alpha),
$$
i.e. the degree of $V_{q}(m)$ is bounded above. From these facts, it follows that the degrees of every indecomposable component of $V_{q}(m)$ are both bounded below and above. This implies that $\{V_{q}(m)\}$ is a bounded family (c.f. p. 426. Th. 3, \cite{art18-key1}). Now $W_{q}(m)$ is generated generically by $K$ (through $p_{1}\circ \beta_{q}$) and therefore by its global sections as well. As we just saw for the case of $V_{q}(m)$, it follows that the degrees\pageoriginale of all the indecomposable components of $W_{q}(m)$ are bounded below. Then since the family $\{W_{q}(m)\}$, $q\in Q(E/P)$ is a family of sub-bundles of the bounded family $\{V_{q}(m)\}$, $q\in Q(E/P)$, it can be proved without much difficulty that the degrees of the indecomposable components of $W_{q}(m)$ are also bounded above (c.f. Prop. 11.1, \cite{art18-key11}). It follows then, as we just saw for the case of $V_{q}(m)$, that $\{W_{q}(m)\}$, $q\in Q(E/P)$, is a bounded family. In particular, there is an absolute positive constant $\theta$ such that
$$
d(W_{q}(m))\leq \theta.
$$
Looking at the inequalities (b), (c), and (d), we get that
$$
|\mu(K)-\rho(K)|\leq r(\theta+r(ml+\alpha)).
$$
Suppose that $N\geq 2p^{2}r(\theta+r(ml+\alpha))$. Then we have
$$
|\mu (K)-\rho(K)|\leq \dfrac{1}{2p^{2}}.
$$
On the other hand, since $\dim K\leq p$ and $\mu(K)<0$, we have
$$
-\mu(K)=|\mu(K)|=\left|\dfrac{r}{p}-\dfrac{r(W_{q})}{\dim K}\right|\geq \dfrac{1}{p^{2}}.
$$
We have
$$
-\rho(K)=-\mu(K)-(\rho(K)-\mu(K)).
$$
Therefore we get
$$
-\rho(K)\geq -\mu(K)-|\mu(K)-\rho(K)|
$$
which gives
$$
-\rho(K)\geq \dfrac{1}{p^{2}}-\dfrac{1}{2p^{2}}=\dfrac{1}{2p^{2}}>0.
$$
Thus we have proved that if $q\not\in R^{ss}$ and $N\geq 2p^{2}(\theta+r(ml+\alpha))$, then there exists a proper linear subspace $K$ of $E$ such that $\rho(K)<0$. This completes the proof of (\ref{art18-claim-B}) and thus (3) of Lemma \ref{art18-lem2} is proved.

Let us now take in the above lemma $r=2$, i.e. we consider semi\-stable vector bundles $V$ of rank $2$ such that $\alpha=d(V)/r(V)$. Then $Z^{ss}$ has a good quotient (modulo $SL(E)$) and the quotient is in fact a projective variety (c.f. \S\ref{art18-sec2}, Theorem \ref{art18-thm4}). Since $R^{ss}$ is a smooth variety, in\pageoriginale particular normal, then by the properties of good quotients, it follows that $R^{ss}$ has a good quotient $\phi:R^{ss}\to T$ modulo $GL(E)$ (equivalently $SL(E)$ or $PGL(E)$) such that $T$ is projective. It is checked easily that $R^{s}$ is {\em non-empty} and that the closures of the $GL(E)$-orbits through $q_{1}$, $q_{2}\in R^{ss}$ intersect if and only if $\gr F_{q_{1}}=\gr F_{q_{2}}$. It follows then that $T$ can be identified naturally with the classes of vector bundles in $\bfS(\alpha,2)$ under the equivalence relation $V_{1}$, $V_{2}\in \bfS(\alpha,2)$, $V_{1}\sim V_{2}$ if and only if $\gr V_{1}=\gr V_{2}$ and that $\dim T=(4g-3)$. It can also be seen easily that $T$ has a weak universal mapping property ({\em coarse moduli scheme} in the sense of Def. 5.6, Chap. 5, \cite{art18-key4}). Thus we get the following

\medskip
\noindent
{\bf Theorem \thnum{5}.\label{art18-thm5}}
{\em Let $U_{\alpha}$ be the equivalence classes of semi-stable $X$ of rank $2$ and degree $2\alpha$ under the equivalence relation $V_{1}\sim V_{2}$ if and only if $\gr V_{1}=\gr V_{2}(\alpha=0\text{~ or~ }\frac{1}{2})$. Then there exists a structure of a normal projective variety on $U_{\alpha}$, uniquely determined by the following properties:}
\begin{enumerate}
\renewcommand{\labelenumi}{\rm(\theenumi)}
\item {\em given an algebraic family of vector bundles $\{V_{t}\}$, $t\in T$, of rank $2$ and degree $2\alpha$ on $X$, parametrized by an algebraic scheme $T$, the canonical mapping $T\to U_{\alpha}$, defined by $t\to \gr V_{t}$ is a morphism;}

\item {\em given another structure $U'$ on $U_{\alpha}$ having the property $(1)$, the canonical mapping $U_{\alpha}\to U'$ is a morphism.}
\end{enumerate}

\begin{remark*}
It can be shown that $U$ is {\em smooth} when $\alpha=\frac{1}{2}$.
\end{remark*}

\begin{thebibliography}{99}
\bibitem{art18-key1} \textsc{M. F. Atiyah :} Vector bundles over an elliptic curve, {\rm Proc. London Math. Soc.} Third series, 7 (1957), 412-452.

\bibitem{art18-key2} \textsc{C. Chevalley :} Classification des groupes de Lie alg\'ebriques, {\em S\'eminaire}, 1956-58, Vol. I.

\bibitem{art18-key3} \textsc{A. Grothendieck :} Les Sch\'emas de Hilbert, {\em S\'eminaire Bourbaki}, t. 13, 221, 1960-61.

\bibitem{art18-key4} \textsc{J. Igusa :}\pageoriginale On the arithmetic normality of the Grassmannian variety, {\em PNAS} (40) 1954, 309-323.

\bibitem{art18-key5} \textsc{D. Mumford :} {\em Geometric invariant theory,} Springer-Verlag, 1965.

\bibitem{art18-key6} \textsc{M. Nagata :} Complete reducibility of rational representations of a matric group, {\em J. Math. Kyoto Univ.} 1-1 (1961), 87-89.

\bibitem{art18-key7} \textsc{M. Nagata :} Invariants of a group in an affine ring, {\em J. Math. Kyoto Univ.} 3, 3, (1964).

\bibitem{art18-key8} \textsc{M. Nagata} and \textsc{T. Miyata :} Note on semi-reductive groups, {\em J. Math. Kyoto Univ.} 3, 1964.

\bibitem{art18-key9} \textsc{T. Oda :} On Mumford conjecture concerning reducible rational representations of algebraic linear groups, {\em J. Math. Kyoto Univ.} 3, 3, 1964.

\bibitem{art18-key10} \textsc{M. Raynaud :} Familles de fibr\'es vectoriels sur une surface de Riemann, {\em S\'eminaire Bourbaki}, 1966-67, No. 316.

\bibitem{art18-key11} \textsc{M. S. Narasimhan} and \textsc{C. S. Seshadri :} Stable and unitary vector bundles on a compact Riemann surface, {\em Ann. of Math.} 82 (1965), 540-567.

\bibitem{art18-key12} \textsc{C. S. Seshadri :} Space of unitary vector bundles on a compact Riemann surface, {\em Ann. of Math.} 82 (1965), 303-336.
\end{thebibliography}

\bigskip
\noindent
{\small Tata Institute of Fundamental Research}

\noindent
{\small Bombay.}

