\chapter[\textsc{H. Hironaka~:} Formal Line Bundles along Exceptional Loci]{FORMAL LINE BUNDLES ALONG EXCEPTIONAL LOCI}\label{art10}

\begin{center}
{\em By}~~ Heisuke Hironaka
\end{center}

\setcounter{pageoriginal}{200}
\section*{Introduction.}\pageoriginale

\lhead[\thepage]{\textit{Formal Line Bundles along Exceptional Loci}}
\rhead[\textit{H. Hironaka}]{\thepage}

If $A$ is a noetherian ring with an ideal $I$, then we define the $I$-adic Henselization to be the limit of all those subrings of $\widehat{A}$ which are \'etale over $A$, where $\widehat{A}$ denotes the $I$-adic completion of $A$. This notion naturally globalizes itself. Namely, if $X$ is a noetherian scheme with a closed subscheme $Y$, then the {\em Henselization of $X$ along $Y$} is the local-ringed space $\widetilde{X}$ with a structural morphism $h:\widetilde{X}\to X$ such that $|\widetilde{X}|=|Y|$ and $\underline{O}_{\widetilde{X}}(U)=$ the $I_{Y}(U)$-adic Henselization of $\underline{O}_{X}(U)$ for every open affine subset $U$ of $|Y|$, where $|~|$ denotes the underlying topological space and $I_{Y}$ the ideal sheaf of $Y$ in $\underline{O}_{X}$. If $\widehat{X}$ is the completion (a formal scheme) of $X$ along $Y$ with the structural morphism $f:\widehat{X}\to X$, there exists a unique morphism $g:\widehat{X}\to \widetilde{X}$ such that $f=hg$. In this article, I present some general techniques for ``equivalences of homomorphisms'' with special short accounts in various special cases, and then briefly sketch a proof of the following algebraizability theorem : {\em Let $k$ be a perfect field and $\pi:X\to X_{0}$ a proper morphism of algebraic schemes over $k$. Let $\widetilde{X}$ (resp. $\widehat{X}$) be the Henselization (resp. completion) of $X$ along $\pi^{-1}(Y_{0})$ with a closed subscheme $Y_{0}$ of $X_{0}$. If $\pi$ induces an isomorphism $X-\pi^{-1}(Y_{0})\xrightarrow{\sim}X_{0}-Y_{0}$, then the natural morphism $g:\widetilde{X}\to \widetilde{X}$ induces an isomorphism $g_{*}:R^{1}p(\underline{O}_{\widetilde{X}}^{*})\xrightarrow{\sim}R^{1}p(\underline{O}_{\widehat{X}}^{*})$, where $p$ denotes the continuous map from $|\widetilde{X}|=|\widehat{X}|=|\pi^{-1}(Y_{0})|$ to $|Y_{0}|$ induced by $\pi$.} In other words, if $z$ is a closed point of $Y_{0}$, every line bundle on $\widehat{X}$ in a neighborhood of $\pi^{-1}(z)$ is derived from a line bundle on $\widetilde{X}$ in some neighborhood of $\pi^{-1}(z)$.

A\pageoriginale {\em Henselian scheme} is, by definition, a local-ringed space $S$ with a coherent sheaf of ideals $J$ such that $(|S|,\underline{O}_{S}/J)$ is a noetherian scheme and $S$ is locally everywhere isomorphic to a Henselization of a noetherian scheme. Such $J$ (resp. the corresponding subscheme) is called a {\em defining} ideal sheaf (resp. subscheme) of $S$. If $S$ is a scheme (resp. Henselian scheme, resp. formal scheme) and $I$ a coherent sheaf of ideals on $S$, then the {\em birational blowing-up} $\pi:T\to S$ of $I$ is defined in the category of schemes (resp. Henselian schemes, resp. formal schemes, where morphisms are those of local-ringed spaces) is defined to be the one which has the universal mapping property : (i) $I\underline{O}_{T}$ is invertible as $\underline{O}_{T}$-module, and (ii) if $\pi':T'\to S$ is any morphism with the property (i) and with a scheme (resp. Henselian scheme, resp. formal scheme) $T'$, there exists a unique morphism $b:T'\to T$ with $\pi'=\pi b$. One can prove the existence in those categories. Now, let $T$ be any noetherian scheme (resp. Henselian scheme, resp. formal scheme), and $Y$ a noetherian scheme with a closed embedding : $Y\subset T$. Let $p:Y\to Y_{0}$ be any proper morphism of schemes. Then the {\em birational blowing-down along} $p$ (in the respective category) means a ``proper'' morphism $\pi:T\to S$ (in the respective category) together with as embedding $Y_{0}\subset S$ such that there exists a coherent ideal sheaf $J$ on $S$ which has the following properties : (1) $J\supset I^{j}$ for $j>>0$, where $I$ is the ideal sheaf of $Y_{0}$ in $S$, and (2) if $\alpha:T'\to T$ and $\beta:S'\to S$ are the birational blowing-up of the ideal sheaves $J\underline{O}_{T}$ and $J$, respectively, then the natural morphism $T'\to S'$ is an isomorphism. Now, given a noetherian scheme $X$ and a closed subscheme $Y$ of $X$, we let $\widetilde{X}$ (resp. $\widehat{X}$) denote the Henselization (resp. completion) of $X$ along $Y$. Let $p:Y\to Y_{0}$ be a proper morphism of noetherian schemes. We then propose the following problem: {\em If there exists a birational blowing-down of $\widehat{X}$ along $p$ in the category of formal schemes, does there follow the same of $\widetilde{X}$ along $p$ in the category of Henselian schemes ?} For simplicity, let us consider the case in which $X$, $Y$, $Y_{0}$ are all algebraic schemes over a perfect field $k$. In this case, we can prove that, $\widehat{X}\to S$ being the formal birational blowing-down, the Henselian blowing-down exists if and only if $S$ is\pageoriginale locally everywhere algebraizable, i.e. isomorphic to completions of algebraic schemes over $k$. Clearly the problem is local in $S$. Suppose we have an algebraic scheme $X_{0}$ containing $Y_{0}$ in such a way that $S$ is isomorphic to the completion of $X_{0}$ along $Y_{0}$. By a somewhat refined Chow's Lemma we may assume that there exists an ideal sheaf $\widehat{J}$ in $S$ which contains $\widehat{I}^{j}$ for $j>>0$, where $\widehat{I}$ = the ideal sheaf of $Y_{0}$ in $S$, and such that $\widehat{X}\to S$ is the birational blowing-up of $\widehat{J}$. Clearly, $\widehat{J}$ is induced by an ideal sheaf $J$ on $X_{0}$. Let $X'\to X_{0}$ be the birational blowing-up of $J$, and $\widetilde{X}$ (resp. $\widetilde{X}'$) the Henselization of $X$ (resp. $X'$) along $Y$ (resp. the inverse image of $Y_{0}$). Then we can prove that $\widetilde{X}$ is isomorphic to $\widetilde{X}'$, using the above algebraizability theorem of line bundles and some techiniques of ``equivalence of embeddings along exceptional subschemes.''. All the details in these regards will be presented elsewhere. I like to note that recently M. Artin obtained an outstanding theorem in regard to ``{\em \'etale approximations}'', which produced a substantial progress in the above Blowing-down Problem as well as in many related problems.

The work presented in this article was done while I was financially supported by Purdue University and by N. S. F. through Harvard University. To them, I am grateful.

\section{Derivatives of a map.}\label{art10-sec1}

Let $R$ be a commutative ring with unity, let $A$ and $B$ be two associative $R$-algebras with unity, and let $E$ be an $(A,B)_{R}$-module, i.e. a left $A$-and right $B$-module in which the actions of $A$ and $B$ induce the same $R$-module structure. Then an endomorphism $\tau$ of $E$ as an abelian group will be called a $(A,B)_{R}$-{\em module derivation} of $E$ (into itself) if there exist ring derivations in the usual sense, say $\alpha$ (resp. $\beta$, resp. $d$) of $A$ (resp. $B$, resp. $R$) into itself, such that

\setcounter{subsection}{1}
\subsubsection{}\label{art10-sec1.1.1} $\tau(aeb)=\alpha(a)eb+a\tau(e)b+ae\beta(b)$ for all $a\in A$, $e\in E$ and $b\in B$, and

\subsubsection{}\label{art10-sec1.1.2} $\alpha(ra)=d(r)a+r\alpha(a)$ and $\beta(rb)=d(r)b+r\beta(b)$ for all $r\in R$, $a\in A$ and $b\in B$.

\medskip
\noindent
{\bf Remark \thnum{1.2}.\label{art10-sec1-rem1.2}}
If\pageoriginale the actions of $R$, $A$, and $B$ upon $E$ are all faithful, then $\tau$ determines all the other $\alpha$, $\beta$ and $d$. In any case, under the conditions, we say that $\alpha$, $\beta$ and $d$ are compatible with $\tau$.
\smallskip

We are interested in applying the above definition to the following situation. Let $L$ and $L'$ be two $R$-modules, let $A=\End_{R}(L)$ and $B=\Eng_{R}(L')$, and let $E=\Hom_{R}(L',L)$. If $D$ is an $R$-module of ring derivations of $R$ into itself, then we obtain an $R$-module, denoted by $\Der_{D}(L',L)$, which consists of all the $(A,B)_{R}$-module derivations of $E$ which are compatible with the derivations in $D$. If $L'$ and $L$ are both finite free $R$-modules with fixed free bases, say $b'=(b'_{1},\ldots,b'_{r})$ and $b=(b_{1},\ldots,b_{s})$ respectively, then we can give explicit presentations to all the elements of $\Der_{D}(L',L)$. Namely, if $d\in D$ and $f=((f_{ij}))\in E$, then we let $d_{b,b'}(f)=((df_{ij}))\in E$, where the matrix presentation ((\quad)) is given by means of the free bases $b$ and $b'$. The following fact is then immediate from a well-known theorem about ring derivations of a full matrix algebra.

\medskip
\noindent
{\bf Theorem \thnum{1.3}.\label{art10-sec1-thm1.3}}
{\em Let $L$, $L'$, $b$, $b'$, $A$, $B$ and $D$ be the same as above. Then every $(A,B)_{R}$-module derivation $\tau$ of $\Hom_{R}(L',L)$, compatible with $d\in D$, can be written as follows:}
$$
\tau(e)=d_{b,b'}(e)+a_{0}e-eb_{0}
$$
{\em for all $e\in \Hom_{R}(L',L)$, where $a_{0}\in A$ and $b_{0}\in B$.}
\smallskip

As is easily seen, if we define $\alpha(a)=d_{b,b}(a)+a_{0}a-aa_{0}$ for all $a\in A$ and $\beta(b)=d_{b',b'}(b)+b_{0}b-bb_{0}$ for all $b\in B$, then $\alpha$ (resp. $\beta$) is ring derivation of $A$ (resp. $B$) which is compatible with $\tau$.

From now on, we assume that $R$ is noetherian. Given a homomorphism $f:F'\to F$, of finite $R$-modules, we consider various {\em permissible squares} $(p,\alpha,\beta,f)$ {\em over} $f$, i.e. $p:L'\to L$, $\alpha:L'\to F'$ and $\beta:L\to F$ such that $\beta p=f\alpha$, that both $\alpha$ and $\beta$ are surjective, that both $L'$ and $L$ are finite free $R$-modules and that $p(\Ker(\alpha))=\Ker(\beta)$. We can prove

\medskip
\noindent
{\bf Theorem \thnum{1.4}.\label{art10-sec1-thm1.4}}
{\em Let $R$, $f:F'\to F$, and $D$ be the same as above. Let $h:F\to \overline{F}$ be a homomorphism of $R$-modules. Then there exists an $R$-submodule\pageoriginale $B=B(f,h,D)$ of $\Hom_{R}(F',\overline{F})$ such that for every permissible square $(p,\alpha,\beta,f)$ over $f$ as above,}
$$
B=\alpha^{*-1}(h\beta)_{*}(\Der_{D}(L',L)p)
$$
{\em where $\alpha^{*}:\Hom_{R}(F',\overline{F})\to \Hom_{R}(L',\overline{F})$ is induced by $\alpha$, $(h\beta)_{*}:\Hom_{R}(L',L)\to \Hom_{R}(L',\overline{F})$ induced by $h\beta$, and $\Der_{D}(L',L)p$ is the $R$-submodule of $\Hom_{R}(L',L)$ of all the derivatives of $p$ by the elements of $\Der_{D}(L',L)$.}

\medskip
\noindent
{\bf Definition \thnum{1.5}.\label{art10-sec1-defi1.5}}
If $f$, $D$ and $h$ are the same as in (\ref{art10-sec1-thm1.4}), then we define the obstruction module for $(f,h,D)$ to be
$$
T_{R}(f,h,D)=K(f,h)/B(f,h,D)\cap K(f,h)
$$
where $K(f,h)=\Ker(\Hom_{R}(F',\overline{F})\to \Hom_{R}(\Ker(f),\overline{F})$.
\smallskip

We shall be particularly interested in the two special cases, the one in which $h$ is the natural homomorphism $F\to \Coker (f)$ and the other in which $h$ is the identity automorphism of $F$. We write $T_{R}(f,D)$ for $T_{R}(f,h,D)$ in the former special case, and $T^{*}_{R}(f,D)$ in the latter. We also write $T_{R}(f)$ for $T_{R}(f,(0))$, and $T^{*}_{R}(f)$ for $T^{*}_{R}(f,(0))$.

\medskip
\noindent
{\bf Remark \thnum{1.5.1}.\label{art10-sec1-rem1.5.1}}
It is easy to prove that if $hf=0$, then $B(f,h,D)$ is contained in $K(f,h)$.

\medskip
\noindent
{\bf Remark \thnum{1.5.2}.\label{art10-sec1-rem1.5.2}}
Assume that both $F$ and $F'$ are free. In virtue of (\ref{art10-sec1-thm1.3}), one can then find a canonical isomorphism
$$
T_{R}(f)\xrightarrow{\sim}\Ext^{1}_{R}(E,E),
$$
where $E=\Coker (f)$. Moreover, one can find a homomorphism of $D$ into $\Ext^{1}_{R}(E,E)$ whose cokernel is $T_{R}(f,D)$. In particular, if $E=R/J$ with an ideal $J$, we have a canonical homomorphism $\beta:D\to \Ext^{1}_{R}(E,E)$ having the property. Namely, there is a canonical isomorphism $\Hom_{R}(J,E)\xrightarrow{\sim}\Ext^{1}_{R}(E,E)$, and an element of $D$ induces an $R$-homomorphism from $J$ to $R/J$. The canonical isomorphism $T_{R}(f,D)\xrightarrow{\sim}\Coker (\beta)$ is then induced by the obvious epimorphism 
$$
K(f,h)\to \Ext^{1}_{R}(E,E).
$$

\medskip
\noindent
{\bf Remark \thnum{1.5.3}.\label{art10-sec1-rem1.5.3}}
Let\pageoriginale $(p,\alpha,\beta,f)$ be a permissible square over $f$ as before. Then $\beta$ induces an isomorphism from $\Coker (p)$ to $\Coker(f)$. If $E$ denotes this cokernel, $\alpha$ induces a homomorphism from $\Hom_{R}(F',E)$ to $\Hom_{R}(L',E)$. We can prove that this homomorphism induces a mono\-morphism $m:T_{R}(f,D)\to T_{R}(p,D)$ and that $m$ is an isomorphism if $F$ and $F'$ are projective. Let us say that two homomorphisms $f_{i}:F'_{i}\to F_{i}(i=1,2)$ are {\em equivalent} to each other if they admit permissible squares $(p,\alpha_{i},\beta_{i},f_{i})$ with the same $p:L'\to L$. Let $C$ be an equivalence class of such homomorphisms. Then $T_{R}(f,D)$ with $f\in C$ is independent of $f:F'\to F$, so long as $F$ and $F'$ are projective.

\medskip
\noindent
{\bf Remark \thnum{1.5.4}.\label{art10-sec1-rem1.5.4}}
If $x$ is a prime ideal in $R$ (or any multiplicatively closed subset of $R$), then $D$ generates an $R_{x}$-module of derivations of $R_{x}$ into itself. Let $D_{x}$ denote this module. Let $f_{x}:F'_{x}\to F_{x}$ and $h_{x}:F_{x}\to \overline{F}_{x}$ denote the localizations of $f$ and $h$ respectively. Then there exists a canonical isomorphism :
$$
T_{R}(f,h,D)_{x}(=T_{R}(f,h,D)\otimes R_{x})\xrightarrow{\sim}T_{R_{x}}(f_{x},h_{x},D_{x}).
$$

\medskip
\noindent
{\bf Remark \thnum{1.5.5}.\label{art10-sec1-rem1.5.5}}
Assume that both $F$ and $F'$ are free. Let $N(f)=\{\lambda\in \Hom_{R}(L',L)|\lambda(\Ker (f))\subset \Iim(f)\}$. Then the natural homomorphism $h:F\to E=\Coker(f)$ induces an epimorphism $N(f)\to K(f,h)$. This then induces an isomorphism $N(f)/B(f,\id,D)\to T_{R}(f,D)$. As $N(f)\supset K(f,\id)$, we get a monomorphism $\omega:T^{*}_{R}(f,D)\to T_{R}(f,D)$ in general.

\medskip
\noindent
{\bf Remark \thnum{1.5.6}.\label{art10-sec1-rem1.5.6}}
Let
$$
F'\xrightarrow{f}F\xrightarrow{f_{r-1}}F_{r-2}\xrightarrow{f_{r-2}}\ldots\to F_{0}\xrightarrow{f_{0}}G\to 0
$$
be an exact sequence of $R$-modules, where $r$ is an integer $>1$ and all the $F'$s (i.e., $F'$, $F$, $F_{i}$, $0\leq i\leq r-2$) are free. Take the case of $D=(0)$. Then we get a canonical isomorphism $T^{*}_{R}(f)\to \Iim (\Ext^{1}_{R}(E,F)\to \Ext^{1}_{R}(E,E))$, with $E=\Coker (f)$, and the monomorphism $\omega$ of (\ref{art10-sec1-rem1.5.5}) in this case is nothing but inclusion into $\Ext^{1}_{R}(E,E)$ with respect to the isomorphism of (\ref{art10-sec1-rem1.5.2}). Moreover, we get a canonical\pageoriginale isomorphism $T^{*}_{R}(f)\xrightarrow{\sim}\Iim(\Ext^{r}_{R}(G,F)\to \Ext^{r}_{R}(G,E))$.

\medskip
\noindent
{\bf Remark \thnum{1.5.7}.\label{art10-sec1-rem1.5.7}}
Let $C$ be an equivalence class of homomorphisms in the sense of (\ref{art10-sec1-rem1.5.3}). Then, for any two $f_{1}$ and $f_{2}$ belonging to $C$, there exists a canonical isomorphism from $T^{*}_{R}(f_{1},D)$ to $T^{*}_{R}(f_{2},D)$, i.e. $T^{*}_{R}(f)$ with $f\in C$ is uniquely determined by $C$ provided the $f_{i}$ and $f$ are homomorphisms of projective $R$-modules.

\section{Two equivalence theorems of homomorphisms.}\label{art10-sec2}

Let $R$ be a Zariski ring with an ideal of definition $H$, and $D$ an $R$-module of ring derivations of $R$. Let us assume :

\subsection{}\label{art10-sec2.1}
There is given a group of ring automorphisms of $R$, denoted by $(D)$, such that for every integer $j>1$ and every $d\in H^{j}D$, there exists $\lambda\in (D)$ with $\lambda(r)\equiv r+d(r)\mod H^{2j}$ for all $r\in R$.

When $R$ is complete, $(\widehat{D})$ denotes the closure of $(D)$ in $\Aut(R)$ with respect to the $H$-adic congruence topology.

If $f$ and $f'$ are two homomorphisms of $R$-modules from $F'$ to $F$, then we ask whether there exists a $(D)$-{\em equivalence from $f$ to $f'$,} i.e. a triple $(\lambda,\alpha,\beta)$ with $\lambda\in (D)$ and $\lambda$-automorphisms $\alpha$ and $\beta$ of $F$ and $F'$, respectively, such that $f'\alpha=\beta f$.

Let $F$ be a finite $R$-module. We say that $F$ is $(D)$-{\em rigid} (with respect to the $H$-adic topology in $R$) if the following condition is satisfied :

\subsection{}\label{art10-sec2.2}
Let $b:L\to F$ be any epimorphism of $R$-modules with a finite free $R$-module $L$. Then one can find a pair of nonnegative integers $(r_{0},t_{0})$ such that if $\alpha$ is a $\lambda$-automorphism of $L$, $\equiv \id_{L}\mod H^{j}L$ with $j\geq t_{0}$ and with $\lambda\in (D)$, then there exists an $R$-automorphism $\alpha'$ of $L$, $\equiv \id_{L}\mod H^{j-r_{0}}L$, such that $\alpha'\alpha$ induces a $\lambda$-automorphism of $F$, i.e. $\alpha'\alpha(\Ker(b))=\Ker(b)$.

Note that the $(D)$-rigidity is trivial if $(D)$ consists of only the identity. As to the other nontrivial cases, we have the following useful sufficient condition : If $F$ is locally free on $\Spec(R)-\Spec (R/H)$, then it is $(D)$-rigid for any $(D)$. In fact, we can prove 

\medskip
\noindent
{\bf Theorem \thnum{2.3}.\label{art10-sec2-thm2.3}}
{\em Let\pageoriginale $X=\Spec (R)$ and $Y=\Spec (R/H)$. Let $L$ be a finite $R$-module, locally free on $X-Y$, and $K$ a submodule of $L$ such that $L/K$ is locally free on $X-Y$. Then there exists a pair of nonnegative integers $(t_{0},r_{0})$ which has the following property. Let $K'$ be any submodule of $L$ such that $L/K'$ is locally free on $X-Y$ and that $\rank ((L/K')_{x})\geq \rank ((L/K)_{x})$ for every $x\in X-Y$. If $K'\equiv K\mod H^{j}L$ with $j\geq t_{0}$, then there exists an automorphism $\sigma$ of the $R$-module $L$ such that $\sigma\equiv \id_{L}\mod H^{j-r_{0}}L$ and $\sigma(K)=K'$.}
\smallskip

We have two types of equivalence criteria, the one in terms of the obstruction module $T_{R}(f,D)$ and the other in terms of $T^{*}_{R}(f,D)$. Each of the two serves better than the other, depending upon the type of applications, as will be seen in the next section.

\medskip
\noindent
{\bf Equivalence Theorem \thnum{I}.\label{art10-sec2-eqthm-I}}
{\em Let us assume that $R$ is complete. Let $f:F'\to F$ be a homomorphism of finite $R$-modules, such that}
\begin{itemize}
\item[\rm(i)] {\em both $F$ and $F'$ are $(D)$-rigid, and}

\item[\rm(ii)] {\em $H^{c}T_{R}(f,D)=(0)$ for all $c>>0$.}
\end{itemize}

{\em Then there exists a triple of nonnegative integers $(s,t,r)$ which has the following property. Let us pick any integer $j\geq t$ and any homomorphism $f':F'\to F$ such that}
\begin{itemize}
\item[\rm(a)] {\em $\Ker (f)\subset \Ker(f')+H^{s}F'$, and}

\item[\rm(b)] $f'\equiv f\mod H^{j}F$.
\end{itemize}

{\em Then there exists a $(\widehat{D})$-equivalence from $f$ to $f'$ which is congruent to the identity $\mod H^{j-r}$.}
\smallskip

The last congruence means, of course, that if $(\lambda,\alpha,\beta)$ is the $(D)$-equivalence then $\lambda\equiv \id_{R}\mod H^{j-r}$, $\alpha\equiv \id_{F}\mod H^{j-r}F$ and $\beta\equiv \id_{F'}\mod H^{j-r}F'$.

\medskip
\noindent
{\bf Equivalence Theorem \thnum{II}.\label{art10-sec2-eqthm-II}}
{\em Let us assume that $R$ is complete. Let $f:F'\to F$ be a homomorphism of finite $R$-modules, such that}
\begin{itemize}
\item[\rm(i)] {\em $f$ is injective,}

\item[\rm(ii)] {\em both $F$ and $F'$ are $(D)$-rigid, and}

\item[\rm(iii)] {\em $H^{c}T^{*}_{R}(f,D)=(0)$ for all $c>>0$.}
\end{itemize}

{\em Then\pageoriginale there exists a pair of nonnegative integers $(t,r)$ which has the following property. Let us pick any integer $j\geq t$ and any homomorphism $f':F'\to F$ such that $f'\equiv f\mod H^{j}F$. Then there exists a $(\widehat{D})$-equivalence from $f$ to $f'$ which is congruent to the identity $\mod H^{j-r}$.}

\medskip
\noindent
{\bf Remark \thnum{2.4.1}.\label{art10-sec2-rem2.4.1}}
If $(D)$ consists of only $\id_{R}$, then the $(D)$-equivalence is nothing but a pair of $R$-automorphisms $\alpha$ and $\beta$ with the commutativity. In this case, the equivalence theorems hold without the completeness of $R$.

\medskip
\noindent
{\bf Remark \thnum{2.4.2}.\label{art10-sec2-rem2.4.2}}
Let us assume that both $F$ and $F'$ are locally free on $\Spec (R)-\Spec (R/H)$. Let us say that two homomorphisms of finite $R$-modules $f_{i}:F'_{i}\to F_{i}$, $i=1,2$, are $f$-equivalent to each other if there exist epimorphisms of finite free $R$-modules $e_{i}:L'_{i}\to L_{i}$, an isomorphism $b:F_{1}\oplus L_{1}\to F_{2}\oplus L_{2}$ and an isomorphism $b':F'_{1}\oplus L'_{1}\to F'_{2}\oplus L'_{2}$ such that $(f_{2}\oplus e_{2})b'=b(f_{1}\oplus e_{1})$. Let $C$ be the $f$-equivalence class of the given map $f$. Then $(s,t,r)$ of E.Th.\ref{art10-sec2-eqthm-I} (resp. $(t,r)$ of E.Th.\ref{art10-sec2-eqthm-II}) can be so chosen to have the property of the theorem not only for the given $f$ but also for every $f_{1}:F'_{1}\to F_{1}$ belonging to $C$ (and satisfying (i) of E.Th.\ref{art10-sec2-eqthm-II}).

\medskip
\noindent
{\bf Remark \thnum{2.4.3}.\label{art10-sec2-rem2.4.3}}
The equivalence theorems can be modified in a somewhat technical fashion so as to become more useful in a certain type of application. To be precise, let $q$ be any nonzero element of $R$ which is not a zero divisor of $\Coker(f)_{x}$ for any point $x$ of $\Spec (R)-\Spec(R/H)$. Then, under the same assumptions of the respective $E$. Th.'s, we can choose $(s,t,r)$ (resp. $(t,r)$) in such a way that : If $f'$ satisfies the stronger congruence $f'\equiv f\mod qH^{j}F$, instead of $\mod H^{j}F$, then we can find a $(\widehat{D})$-equivalence from $f'$ to $f$, $\equiv \id\mod qH^{j-r}$. This modification of the E.Th.'s is used in establishing certain equivalence by a dimension-inductive method in terms of hyperplane sections.

\section{Examples of applications.}\label{art10-sec3}

\medskip
\noindent
{\bf Example \thnum{I}\label{art10-sec3-exam-I}}~(Equivalence of Singularities).
Let $k$ be a noetherian ring (for instance, a field). Let $R_{0}=k[x]=k[x_{1},\ldots,x_{N}]$, a polynomial\pageoriginale ring of $N$ variables over $k$. (In what follows, $R_{0}$ may be replaced by a convergent power series over an algebraically closed complete valued field.) Let $R_{1}$ be a ring of fractions of $R_{0}$ with respect to a multiplicatively closed subset of $R_{0}$, and $H_{1}$ a non-unit ideal in $R_{1}$. Let $R$ be the $H_{1}$-adic completion of $R_{1}$, and $H=H_{1}R_{1}$. Let $J$ be an ideal in $R$, let $X=\Spec (R/J)$, let $Y=\Spec (R/J+H)$ and $\pi:X\to S$ be the projection map with $S=\Spec (k)$. We assume:

\subsection{}\label{art10-sec3.1}
$X-Y$ is formally $S$-smooth, i.e. for every point $x$ of $X-Y$, the $d\times d$-minors of the jacobian $\partial(f_{1},\ldots,f_{m})/\partial(x_{1},\ldots,x_{N})$ generate the unit ideal in the local ring $O_{X,x}$, where $J=(f_{1},\ldots,f_{m})R$ and $d$ is the codimension of $X$ in $\Spec(R)$ at $x$. If $X'=\Spec(R/J')$ with another ideal $J'$ in $R$, then we ask if there exists a $k$-automorphism $\sigma$ of $R$ which induces an isomorphism from $X$ to $X'$. For this purpose, we pick and fix an exact sequence
\begin{equation*}
L_{2}\xrightarrow{g}L_{1}\xrightarrow{f}R\xrightarrow{h}R/J\to 0\tag{3.2}\label{art10-sec3-eq3.2}
\end{equation*}
where $L_{i}$ are finite free $R$-modules for $i=1,2$, $\text{Im}(f)=J$ and $h$ is the natural homomorphism. Let $D$ be the $R$-module of derivations of the $k$-algebra $R$, which is generated by $\partial/\partial x_{1},\ldots,\partial/\partial x_{N}$. We apply our equivalence theorem to this $D$ and the map $f$. As was seen in (\ref{art10-sec1-rem1.5.2}), we have a canonical homomorphism $\beta:D\to \Ext^{1}_{R}(R/J,R/J)$ and an isomorphism $T_{R}(f,D)\xrightarrow{\sim}\Coker(\beta)$. Thus the obstruction module $T_{R}(f,D)$ is seen to be independent of the choice of $(g,f)$ in \eqref{art10-sec3-eq3.2}. Moreover, as is easily seen, (\ref{art10-sec3.1}) is equivalent to saying that the localization of $\beta$, or
$$
\beta_{x}:D_{x}\to \Ext^{1}_{R}(R/J,R/J)_{x}\quad (=\Hom_{R}(J/J^{2},R/J)_{x}),
$$
is surjective for every point $x$ of $X-Y$. Hence it is also equivalent to
\begin{equation*}
H^{c}T_{R}(f,D)=(0)\quad\text{for all}\quad c>>0.\tag{3.3}\label{art10-sec3-eq3.3}
\end{equation*}
Therefore the following is a special case of E.Th. \ref{art10-sec2-eqthm-I}.

\medskip
\noindent
{\bf Theorem \thnum{3.3}.\label{art10-sec3-thm3.3}}
{\em Let the assumptions be the same as above. Then there exists a triple of nonnegative integers $(s,t,r)$ which has the following property.\pageoriginale Let $j$ be any integer $\geq t$, and let $g':L_{2}\to L_{1}$ and $f':L_{1}\to R$ be any pair of homomorphisms such that {\rm(a)} $f'g'=0$, {\rm(b)} $g'\equiv g\mod H^{s}L_{1}$ and {\rm(c)} $f'\equiv f\mod H^{j}$. Then there exists an automorphism of the $k$-algebra $R$ which induces an isomorphism from $X$ to $X'=\Spec (R/\text{\rm Im}(f'))$ and which is congruent to the identity $\mod H^{j-r}$.}

\medskip
\noindent
{\bf Example \thnum{II}\label{art10-sec3-exam-II}}~(Equivalence of Vector Bundles)
Let $R$ be any noetherian Zariski ring with an ideal of definition $H$. (For instance, $R=(R_{0}/J_{0})(1+H_{0})^{-1}$ with any pair of ideals $J_{0}$ and $H_{0}$ in the ring $R_{0}$). Let $X=\Spec (R)$ and $Y=\Spec (R/H)$. Let $V$ be a vector bundle on $X-Y$, or a locally free sheaf on $X-Y$. Then there exists a finite $R$-module $E$ which generates $V$ on $X-Y$. Let us fix an exact sequence
\begin{equation*}
L_{2}\xrightarrow{g}L_{1}\xrightarrow{f}L_{0}\xrightarrow{h}E\to 0\tag{3.4}\label{art10-sec3-eq3.4}
\end{equation*}
where the $L_{i}$ are all free $R$-modules $(i=0,1,2)$. We apply our equivalence theorem to $f$ with $D=(0)$. We have $T_{R}(f)=\Ext^{1}_{R}(E,E)$ by (\ref{art10-sec1-rem1.5.1}). Since $E$ is locally free on $X-Y$, $\Ext^{1}_{R}(E,E)_{x}=(0)$ for all $x\in X-Y$. This implies
\begin{equation*}
H^{c}T_{R}(f)=(0)\text{~~ for all~~ } c>>0.\tag{3.5}\label{art10-sec3-eq3.5}
\end{equation*}
Thus we get the following special case of E.Th. \ref{art10-sec2-eqthm-I}.

\medskip
\noindent
{\bf Theorem \thnum{3.6}.\label{art10-sec3-thm3.6}}
{\em Let the assumptions be the same as above. Then there exists a triple of nonnegative integers $(s,t,r)$ which has the following property. Let $j$ be any integer $\geq t$, and let $g':L_{2}\to L_{1}$ and $f':L_{1}\to L_{0}$ be any pair of homomorphisms such that {\rm(a)} $f'g'=0$, {\rm(b)} $g'\equiv g\mod H^{s}L_{1}$ and {\rm(c)} $f'\equiv f\mod H^{j}L_{0}$. Then there exists an automorphism of the $R$-module $L_{0}$ which induces an isomorphism from $V$ to $V'$, with the locally free sheaf $V'$ on $X-Y$ generated by $\Coker (f')$, and which is congruent to the identity $\mod H^{j-r}L_{0}$.}

\medskip
\noindent
{\bf Remark \thnum{3.7}.\label{art10-sec3-rem3.7}}
An important common feature of Theorems \ref{art10-sec3-thm3.3} and \ref{art10-sec3-thm3.6} is that, when the singularity or the vector bundle is represented by an $R$-valued point $(f,g)$ in the affine algebraic scheme defined by the simultaneous quadratic equations $fg=0$ (in terms of fixed free bases of the $L_{i}$), all the approximate points (with\pageoriginale respect to the $H$-adic topology of $R$) in the scheme represent the same singularity or the same vector bundle respectively.

\medskip
\noindent
{\bf Example \thnum{III}.\label{art10-sec3-exam-III}}
Let $R$ be a regular Zariski ring with an ideal of definition $H$. Let $\overline{R}=R/J$ with an ideal $J$, and $\overline{H}=H\overline{R}$. Let $\overline{E}$ be a finite $\overline{R}$-module. Let $Z=\Spec (R)$, $X=\Spec (\overline{R})$ and $Y=\Spec (\overline{R}/\overline{H})$. Let us assume :

\setcounter{subsection}{7}
\subsection{}\label{art10-sec3.8}
$X$ is locally a complete intersection of codimension $e$ in $Z$ at every point of $X-Y$, and

\subsection{}\label{art10-sec3.9}
$\overline{E}$ is locally free on $X-Y$.

Let us take a resolution of $\overline{E}$ as an $R$-module by finite free $R$-modules: $\to L_{p}\xrightarrow{f_{p}}L_{p-1}\xrightarrow{f_{p-1}}\ldots\xrightarrow{f_{1}}L_{0}\xrightarrow{f_{0}}\overline{E}\to 0$. Then, by \ref{art10-sec1-rem1.5.6}, $T^{*}_{R}(f_{p})$ is isomorphic to the image of the natural homomorphism $\Ext^{p}_{R}(\overline{E},L_{p-1})\to \Ext^{p}_{R}(\overline{E},E)$ with $E=\Coker (f_{p})$. By (\ref{art10-sec3.8}), $\Ext^{p}_{R}(\overline{R},R)_{x}=0$ if $p\neq e$, and $=\overline{E}_{x}$ if $p=e$, for all points $x$ of $Z-Y$. Thus, by (\ref{art10-sec3.9}), we get

\setcounter{subsection}{9}
\subsection{}\label{art10-sec3.10}
For every positive $p\neq e$, $H^{c}T^{*}_{R}(f_{p})=0$ for all $c>>0$.

Let $F'=\text{Im}(f_{p})$, $F=L_{p-1}$ and $f:F'\to F$ the inclusion. Then $T^{*}_{R}(f)$ is isomorphic to $T^{*}_{R}(f_{p})$, and the following is a special case of E. Th. \ref{art10-sec2-eqthm-II}.

\medskip
\noindent
{\bf Theorem \thnum{3.11}.\label{art10-sec3-thm3.11}}
{\em Let the assumptions be the same as above, and let $p$ be a positive integer $\neq e$. Then there exists a pair of nonnegative integers $(t,r)$ such that if $f':L_{p}\to L_{p-1}$ is any homomorphism with $\Ker (f')\supset \Ker (f)$ and with $f'\equiv f\mod H^{j}L_{p-1}$ for an integer $j\geq t$, then there exists an equivalence from $f'$ to $f$ which is congruent to $\id \mod H^{j-r}$.}

\medskip
\noindent
{\bf Example \thnum{IV}.\label{art10-sec3-exam-IV}}
Let us further specialize the situation of Example \ref{art10-sec3-exam-III} and examine the case of $p=e$. Namely, we take $(R,H)$ of Example \ref{art10-sec3-exam-I} and assume (\ref{art10-sec3.1}) in addition to (\ref{art10-sec3.8}) and (\ref{art10-sec3.9}). Let $D$ be the same as in Example \ref{art10-sec3-exam-I}. We can then prove that $T^{*}_{R}(f_{e},D)_{x}=0$ for all $x\in Z-Y$, in the following two special cases.

\begin{description}
\item[Case (a)] $e=1$\pageoriginale and $\overline{E}$ has rank 1 on $X-Y$. (Or the case of a line bundle on a sliced hypersurface.)

\item[Case (b)] $e=2$ and $L_{0}=R$, so that $\overline{E}=R/J$. (Or the case of singularity of embedding codimension two.)
\end{description}
Again, as a corollary of E.Th.\ref{art10-sec2-eqthm-II}, we obtain an equivalence theorem for $f_{e}$ in these two special cases, in which $(t,r)$ has the same property as that of (\ref{art10-sec3-thm3.11}) except that ``equivalence'' must be replaced by ``$\Aut_{k}(R)$-equivalence''.

\medskip
\noindent
{\bf Remark \thnum{3.12}.\label{art10-sec3-rem3.12}}
The equivalence theorems in Examples \ref{art10-sec3-exam-III} and \ref{art10-sec3-exam-IV} give us the following rather strong algebraizability theorem in the above special cases. Let the notation be the same as above and as in Example \ref{art10-sec3-exam-I}. Assume the situation of either Case (a) or Case (b). Suppose $\overline{E}$ admits a free resolution of finite length. (This is always so, if $R$ is local and regular.) Then, for every positive integer $j$, we can find a finite $R_{0}$-module $\overline{E}_{0}$ and an automorphism $\lambda$ of $R$, $\equiv \id_{R}\mod H^{j}$, such that $\foprod{\overline{E}_{0}}{R}{R_{0}}$ and $\foprod{\overline{E}}{R}{\lambda}$ are isomorphic to each other as $R$-modules, where $\otimes_{\lambda}$ denotes the tensor product over $R$ as $R$ is viewed as $R$-algebra by $\lambda$. In fact, we can prove the algebraizability of the homomorphism $f_{p}$ (or, $(\widehat{D})$-equivalence from $f_{p}$ to a homomorphism obtained by the base extension $R_{0}\to R$) by an obvious descending induction on $p$.

\section{An algebraizability theorem of line bundles.}\label{art10-sec4}

Let $k$ be a perfect field, and $R_{0}$ a local ring of an algebraic scheme over $k$ at a closed point. Let $X_{0}=\Spec (R_{0})$. Let $Y_{0}$ be a closed subscheme of $X_{0}$ defined by an ideal $H_{0}$ in $R_{0}$. Let $R$ be the $H_{0}$-adic completion of $R_{0}$, and $R'$ the $H_{0}$-adic Henselization of $R_{0}$, i.e. the limit of those sub-rings of $R$ which are \'etale over $R_{0}$. Let $X=\Spec (R)$, $X'=\Spec (R')$, $Y=\Spec (R/H)$ with $H=H_{0}R$, and $Y'=\Spec (R'/H')$ with $H'=H_{0}R'$. We have natural morphisms $c:X\to X'$ and $c':X'\to X_{0}$, which induce isomorphisms $Y\xrightarrow{\sim}Y'$ and $Y'\xrightarrow{\sim}Y_{0}$. Note that every subscheme $D$ of $X$ with $|D|\subset |Y|$ has an isomorphic image in $X'$ and in $X_{0}$, where $|\quad|$ denotes the point-set. The following is the\pageoriginale algebraizability theorem of line bundles along the exceptional locus of a birational morphism.

\medskip
\noindent
{\bf Theorem \thnum{4.1}.\label{art10-sec4-thm4.1}}
{\em Let $\pi':X'_{1}\to X'$ be a proper morphism which induces an isomorphism $X'_{1}\to {\pi'}^{-1}(Y')\xrightarrow{\sim}X'\to Y'$. Let $\widehat{X}_{1}$ be the completion (a formal scheme) of $X'_{1}$ along ${\pi'}^{-1}(Y')$, and $h':\widehat{X}_{1}\to X'_{1}$ the natural morphism. Then $h'$ induces an isomorphism}
$$
{h'}^{*}:\Pic (X'_{1})\xrightarrow{\sim}\Pic(\widehat{X}_{1}).
$$

If $\pi:X_{1}\to X$ is the morphism obtained from $\pi'$ by the base extension $c:X\to X'$, then we have a natural morphism $h:\widehat{X}_{1}\to X_{1}$. By the GFGA theory of Grothendieck, this induces an isomorphism $h^{*}:\Pic(X_{1})\xrightarrow{\sim}\Pic(\widehat{X}_{1})$. Hence, (\ref{art10-sec4-thm4.1}) amounts to saying that the natural morphism $g:X_{1}\to X'_{1}$ induces an isomorphism
\begin{equation*}
g^{*}:\Pic(X'_{1})\xrightarrow{\sim}\Pic(X_{1}).\tag{4.1.1}\label{art10-sec4-eq4.1.1}
\end{equation*}

Let $U'=X'-Y'$ and $U=X-Y$. It is not hard to see that, for each individual $(R_{0},H_{0})$, (\ref{art10-sec4-thm4.1}) for all $\pi'$ as above is equivalent to the following

\medskip
\noindent
{\bf Theorem \thnum{4.2}.\label{art10-sec4-thm4.2}}
{\em The morphism $c:X\to X'$ induces an isomorphism $\lambda:\Pic(U')\xrightarrow{\sim}\Pic(U)$.}

In fact, we can easily prove:
\begin{itemize}
\item[(i)] If $\omega\in \Pic(U)$, then there exists a finite $R$-submodule $E$ of $H^{0}(\omega)$ which generates the sheaf $\omega$.

\item[(ii)] If $(\omega,E)$ is as above and if $\omega=\lambda(\omega')$ with $\omega'\in \Pic(U')$, then there exists a finite $R'$-submodule $E'$ of $H^{0}(\omega')$ such that $E$ is isomorphic to $\fprod{E'}{R}{R'}$.
\end{itemize}
Now, to see the equivalence of (\ref{art10-sec4-thm4.1}) and (\ref{art10-sec4-thm4.2}), all we need is the following ``Cramer's rule''.

\medskip
\noindent
{\bf Remark \thnum{4.3}.\label{art10-sec4-rem4.3}}
Quite generally, let $E$ be a finite $R$-module which is locally free of rank $r$ in $X-Y$, and $\underline{E}$ the coherent sheaf on\pageoriginale $X$ generated by $E$. Assume that $\Supp(\underline{E})$ is equal to the closure of $X-Y$ in $X$. Let us pick an epimorphism $\alpha:L\to E$ with a free $R$-module of rank $p$. Let $D$ be the subscheme of $X$ defined by the annihilator in $R$ of the cokernel of the natural homomorphism $(\wedge^{p-r}\Ker(\alpha))\otimes (\wedge {}^{r}L)\to \wedge {}^{p}L$. Let $\pi:X_{1}\to X$ be the birational blowing-up with center $D$, and let $\underline{E}_{1}$ be the image of the natural homomorphism $\pi^{*}(\underline{E})\to i_{*}i^{*}(\pi^{*}(\underline{E}))$, where $i$ is the inclusion $X_{1}-\pi^{-1}(Y)\to X_{1}$. Then $\pi$ induces an isomorphism $X_{1}-\pi^{-1}(Y)\xrightarrow{\sim}X-Y$, $\pi^{*}(\underline{E})\to \underline{E}_{1}$ is isomorphic in $X_{1}-\pi^{-1}(Y)$, and $\underline{E}_{1}$ is locally free of rank $r$ throughout $X_{1}$. Moreover, $D$ has an isomorphic image $D'$ in $X'$ and if $\pi':X'_{1}\to X'$ is the birational blowing-up with center $D'$, then $\pi'$ satisfies the assumptions in (\ref{art10-sec4-thm4.1}). Note that a birational blowing-up and a base extension commute if the latter is flat.
\smallskip

Let $E$ be the same as in (\ref{art10-sec4-rem4.3}). Let $E'_{i}$, $i=1,2$, be finite $R'$-modules such that we have an isomorphism from $\foprod{E'_{i}}{R}{R'}$ to $E$ for each $i$. Then there exists an isomorphism from $E'_{1}$ to $E'_{2}$. In fact, since $R$ is $R'$-flat, we have a natural isomorphism
$$
\foprod{\Hom_{R'}(E'_{1},E'_{2})}{R}{R'}\to \Hom_{R}(E,E)
$$
with reference to the given isomorphisms. This means that $\id_{E}$ is arbitrarily approximated by the image of an element of $\Hom_{R'}(E'_{1},E'_{2})$. But, as is easily seen, any good approximation of $\id_{E}$ is an isomorphism itself. since $R$ is faithfully $R'$-flat, this proves the existence of an isomorphism from $E'_{1}$ to $E'_{1}$. Let us remark that this proves the injectivity of $\lambda$ of (\ref{art10-sec4-thm4.2}). We can also deduce from this, without much difficulty, that $g^{*}$ of \eqref{art10-sec4-eq4.1.1} (and hence ${h'}^{*}$ of (\ref{art10-sec4-thm4.1})) is also injective.

The essence of the theorems is the surjectivity of $g^{*}$, or the same of $\lambda$. As is seen in the arguments given above, this surjectivity is equivalent to the following

\medskip
\noindent
{\bf Theorem \thnum{4.4}.\label{art10-sec4-thm4.4}}
{\em Let $E$ be a finite $R$-module which is invertible on $X-Y$. Then there exists a finite $R'$-module $E'$ such that $\foprod{E'}{R}{R'}$ is isomorphic to~$E$.}
\smallskip

We\pageoriginale shall now indicate the key points in proving these theorems. As a whole, the proof is a combination of induction on $n=\dim R$ and the reduction to the case of (\ref{art10-sec3-rem3.12})-(a).

\medskip
\noindent
{\bf Remark \thnum{4.5}.\label{art10-sec4-rem4.5}}
Let the assumptions be the same as in (\ref{art10-sec4-thm4.1}) and in the immediately following paragraph. Let $N'$ be any coherent ideal sheaf on $X'_{1}$ with ${N'}^{2}=0$. Let $\widehat{N}=N'\underline{O}_{\widehat{X}_{1}}$. Let $X'_{2}$ (resp. $\widehat{X}_{2}$) be the subscheme of $X'_{1}$ (resp. $\widehat{X}_{1}$) defined by $N'$ (resp. $\widehat{N}$). Then we have the following natural commutative diagram
\[
\xymatrix{
0\ar[r] & N'\ar[d] \ar[r] & \underline{O}^{*}_{X'_{1}}\ar[d] \ar[r] & \underline{O}^{*}_{X'_{2}}\ar[d]\ar[r] & 0\\
0\ar[r] & \widehat{N}\ar[r] & \underline{O}^{*}_{\widehat{X}_{1}}\ar[r] & \underline{O}^{*}_{\widehat{X}_{2}}\ar[r] & 0
}
\]
which yields the following exact and commutative diagram.
\begin{equation*}
\vcenter{\xymatrix@C=.72cm{
\ar[r] & H^{1}(N')\ar[d]\ar[r] & \Pic(X'_{1})\ar[d]^-{a_{1}}\ar[r] & \Pic(X'_{2})\ar[d]^-{a_{2}}\ar[r] & H^{2}(N')\ar[d]\ar[r] &\\
\ar[r] & H^{1}(\widehat{N})\ar[r] & \Pic(\widehat{X}_{1})\ar[r] & \Pic(\widehat{X}_{2})\ar[r] & H^{2}(\widehat{N})\ar[r] & 
}}\tag{4.5.1}\label{art10-sec4-eq4.5.1}
\end{equation*}
We have natural isomorphisms $\fprod{H^{i}(N')}{R}{R'}\to H^{i}(\widehat{N})$ because $R$ is $R'$-flat. Since $\Supp (H^{i}(N'))\subset |Y|$ for all $i>0$, $H^{i}(N')\to H^{i}(N)$ are isomorphisms for $i=1,2$. Hence the surjectivity of $a_{2}$ implies the same of $a_{1}$.

\medskip
\noindent
{\bf Remark \thnum{4.6}.\label{art10-sec4-rem4.6}}
By the standard amalgamation technique, one can easily reduce the proof of either one of the three theorems to the case of $\dim Y<n=\dim X$. Assuming this, let us try to prove (\ref{art10-sec4-thm4.1}) by induction on $n$. Let $d$ be any element of $H_{0}$ such that $\dim R_{0}/dR_{0}<n$. Let $\widehat{X}_{1}(j)$ (resp. $X'_{1}(j)$) be the subscheme of $\widehat{X}_{1}$ (resp. $X'_{1}$) defined by the ideal sheaf generated by $d^{j+1}$. Let $h'_{j}:\widehat{X}_{1}(j)\to X'_{1}(j)$ be the natural morphism. By induction assumption, we have isomorphisms $(h'_{j})^{*}:\Pic(X'_{1}(j))\xrightarrow{\sim}\Pic(\widehat{X}_{1}(j))$ for all $j$. In view of the cohomology sequences of \eqref{art10-sec4-eq4.5.1} adapted to these cases, as the cohomology of coherent sheaves (those nilideal sheaves) is computable\pageoriginale by any fixed open affine covering, we get canonical isomorphisms:
$$
\varprojLim\limits_{j} \Pic(\widehat{X}_{1}(j))\to \Pic (\widehat{X}_{1})\quad\text{and}\quad \varprojLim_{j}\Pic (X'_{1}(j))\to \Pic(\widehat{X}_{1}),
$$
where $\widetilde{X}_{1}=\varprojLim\limits_{j}X'_{1}(j)$. Therefore, the natural morphism $\widehat{X}_{1}\to \widetilde{X}_{1}$ induces an isomorphism $\Pic (\widetilde{X}_{1})\to \Pic(\widehat{X}_{1})$. In short, to prove (\ref{art10-sec4-thm4.1}) (or any of the other theorems), we may replace $H_{0}$ by any $dR_{0}$ as above.

\medskip
\noindent
{\bf Remark \thnum{4.7}.\label{art10-sec4-rem4.7}}
To prove (\ref{art10-sec4-thm4.4}), we may assume that $R_{0}$ is reduced and $\dim R_{0}/H_{0}<n=\dim R_{0}$. (See (\ref{art10-sec4-rem4.5}) and (\ref{art10-sec4-thm4.2})). We can choose a system $z=(z_{1},\ldots,z_{n+1})$ with $z_{i}\in R_{0}$ and an element $d\in k[z]$ such that if $S_{0}$ is the local ring of $\Spec (k[z])$ which is dominated by $R_{0}$, then
\begin{itemize}
\item[(i)] $R_{0}$ is a finite $S_{0}$-module,

\item[(ii)] $d\in H_{0}$ and $\dim R_{0}/dR_{0}<n$,

\item[(iii)] $V_{0}-W_{0}$ is of pure dimension and $k$-smooth, where $V_{0}=\Spec (S_{0})$ and $W_{0}=\Spec (S_{0}/dS_{0})$, and 

\item[(iv)] the natural morphism $X_{0}\to V_{0}$ induces an isomorphism $X_{0}-\widetilde{Y}_{0}\to V_{0}-W_{0}$, where $\widetilde{Y}_{0}$ denotes the preimage of $W_{0}$ in $X_{0}$. Now, by (\ref{art10-sec4-rem4.6}), the proof of (\ref{art10-sec4-thm4.4}) is reduced to the case of $H_{0}=dR_{0}$ and, in view of (\ref{art10-sec4-thm4.2}), to the case in which $X_{0}=V_{0}$ and $X_{0}$ is the closure of $X_{0}-Y_{0}$. In short, to prove (\ref{art10-sec4-thm4.4}), we may assume that $R_{0}$ is a local ring of a hypersurface in an affine space over $k$ and that $X_{0}-Y_{0}$ is $k$-smooth and dense in $X_{0}$.
\end{itemize}

In this final situation, a proof of (\ref{art10-sec4-thm4.4}) can be derived from the algebraizability theorem for Case (a) of Ex. \ref{art10-sec3-exam-IV}, or Remark \ref{art10-sec3-rem3.12}. To see this, let $T_{0}$ be the local ring of the affine space of dimension $n+1$ which carries $X_{0}$, at the closed point of $X_{0}$. Let $G_{0}$ be the ideal in $T_{0}$ which corresponds to $H_{0}$ in $R_{0}$, $T'$ the $G_{0}$-adic Henselization of $T_{0}$, and $T$ the $G_{0}$-adic completion of $T_{0}$. Let $G=G_{0}T$. Then the result of (\ref{art10-sec3-rem3.12}) (in which $R$ should be replaced\pageoriginale by $T$) implies that, for every positive integer $j$, we can find an automorphism $\lambda$ of $T$ and a finite $T_{0}$-module $E_{0}$ such that $\lambda\equiv\id_{T}\mod G^{j}$ and $\foprod{E_{0}}{T}{T_{0}}$ is isomorphic to $\foprod{E}{T}{\lambda}$ as $T$-module, where $E$ is view as $T$-module in an obvious way. Let $E''=\foprod{E_{0}}{T'}{T_{0}}$. Let $J'$ be the kernel of the epimorphism $T'\to R'$, and $J''$ the annihilator in $T'$ of $E''$. Clearly $J'T$ is the annihilator of $E$. Thanks to the equivalence theorem (\ref{art10-sec3-thm3.6}), it is now sufficient to find an automorphism $\lambda'$ of $R'$, well approximate to $\lambda$, such that $\lambda'(J'')=J'$. Namely, $E'=\foprod{E''}{T'}{\lambda'}$ then has the property of (\ref{art10-sec4-thm4.4}). The existence of $\lambda'$ is easy enough to prove, because $J'$ is generated by a single element whose gradient does not vanish at any point of $X'-Y'$. (This is essentially Hensel's lemma.)

\begin{thebibliography}{99}
\bibitem{art10-key1} \textsc{H. Hironaka} and \textsc{H. Rossi :} On the equivalence of imbeddings of exceptional complex spaces, {\em Math. Annalen} 156 (1964), 313-333.

\bibitem{art10-key2} \textsc{H. Hironaka :} {\em On the equivalence of singularities,} I, Arithmetic Algebraic Geometry, Proceedings, Harper and Row, New York, 1965.
\end{thebibliography}

\bigskip
\noindent
{\small Columbia University}
