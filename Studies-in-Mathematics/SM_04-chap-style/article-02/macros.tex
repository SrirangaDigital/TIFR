\usepackage{graphicx,xspace,fancybox}
\usepackage{fancyhdr}
\usepackage{color}

\newcounter{pageoriginal}


\usepackage[papersize={140mm,215mm},textwidth=110mm,
textheight=170mm,headheight=6mm,headsep=4mm,topmargin=17.5mm,botmargin=1.15cm,
leftmargin=15mm,rightmargin=15mm,footskip=0.6cm]{zwpagelayout}

\marginparwidth=10pt
\marginparsep=10pt
\marginparpush=5pt
%\renewcommand{\thepageoriginal}{\arabic{pageoriginal}}
\newcommand{\pageoriginale}{\refstepcounter{pageoriginal}\marginpar{\footnotesize\xspace\textbf{\thepageoriginal}}
} 
\let\pageoriginaled\pageoriginale

\newtheorem{theorem}{Theorem}[section]
\newtheorem{corollary}[theorem]{Corollary}       
\newtheorem{proposition}[theorem]{Proposition}
\newtheorem{lemma}[theorem]{Lemma}
\newtheorem{conjecture}[theorem]{Conjecture}







\newtheorem{definition}{Definition}[section]











\newtheoremstyle{remark}{10pt}{10pt}{ }%
{}{\bfseries}{.}{ }{}
\theoremstyle{remark}
\newtheorem{remark}{Remark}
\newtheorem{remarks}{Remarks}[section]

\newtheorem{example}[theorem]{Example}
\newtheorem{examples}[theorem]{Examples}
\newtheorem{sconditions}{\textsc{Separation Conditions}}[section]


\newtheoremstyle{nonum}{}{}{\itshape}{}{\bfseries}{\kern -2pt{\bf.}}{ }{#1 \mdseries
{\bf #3}}
\theoremstyle{nonum}

\newtheorem{lemma*}{Lemma}	
\newtheorem{theorem*}{Theorem}	
\newtheorem{irreducibilitythm*}{IRREDUCIBILITY THEOREM}
\newtheorem{genirreducibilitythm*}{GENERIC IRREDUCIBILITY THEOREM}
\newtheorem{embedthm*}{EMBEDDING THEOREM}
\newtheorem{prop*}{Proposition}	
\newtheorem{claim*}{Claim}
\newtheorem{defi*}{Definition}
\newtheorem{conjecture*}{CONJECTURE}
\newtheorem{coro*}{Corollary}
\newtheoremstyle{mynonum}{}{}{ }{}{\bfseries}{\kern -2pt{\bf.}}{ }{#1 \mdseries
{\bf #3}}
\theoremstyle{mynonum}
\newtheorem{remark*}{Remark}	
\newtheorem{remarks*}{Remarks}	
\newtheorem{exer*}{Exercise}	
\newtheorem{example*}{Example}	
\newtheorem{examples*}{Examples}	
\newtheorem{note*}{Note}
\newtheorem{problem}{Problem}

\def\ophi{\overset{o}{\phi}}

\def\oval#1{\text{\cornersize{2}\ovalbox{$#1$}}}


\newcommand*\mycirc[1]{%
  \tikz[baseline=(C.base)]\node[draw,circle,inner sep=.7pt](C) {#1};\:
}


\DeclareMathOperator{\Hom}{\mathrm{Hom}}
\DeclareMathOperator{\red}{\mathrm{red}}
\DeclareMathOperator{\Spec}{\mathrm{Spec}}
\DeclareMathOperator{\Quot}{\mathrm{Quot}}
\DeclareMathOperator{\Hilb}{\mathrm{Hilb}}
\DeclareMathOperator{\Pic}{\mathrm{Pic}}
\DeclareMathOperator{\Ext}{\mathrm{Ext}}



\def\uub#1{\underline{\underline{#1}}}
\def\ub#1{\underline{#1}}
\def\os#1{\overset{#1}}
\def\us#1{\underset{#1}}
\def\ob#1{\overbrace{#1}}
\def\ool#1{\overline{\overline{#1}}}
\def\ol#1{\overline{#1}}
\def\set#1{\left\{{#1}\right\}}
\def\oset#1{\left({#1}\right)}
\def\cset#1{\left[{#1}\right]}
\def\mset#1{\left|{#1}\right|}
\def\aset#1{\left<{#1}\right>}


\font\bigsymb=cmsy10 at 4pt
\def\bigdot{{\kern1.2pt\raise 1.5pt\hbox{\bigsymb\char15}}}
\def\overdot#1{\overset{\bigdot}{#1}}

\makeatletter
\renewcommand\subsection{\@startsection{subsection}{2}{\z@}%
                                     {-3.25ex\@plus -1ex \@minus -.2ex}%
                                     {1.5ex \@plus .2ex}%
                                     {\normalfont}}%

\renewcommand\thesection{\@arabic\c@section}
%\renewcommand\thesubsection{({\thechapter.\thesection.\@arabic\c@subsection})}

\renewcommand{\@seccntformat}[1]{{\csname the#1\endcsname}\hspace{0.3em}}
\makeatother

\def\fibreproduct#1#2#3{#1{\displaystyle\mathop{\times}_{#3}}#2}
\let\fprod\fibreproduct

\def\fibreoproduct#1#2#3{#1{\displaystyle\mathop{\otimes}_{#3}}#2}
\let\foprod\fibreoproduct


\def\cf{{cf.}\kern.3em}
\def\Cf{{Cf.}\kern.3em}
\def\eg{{e.g.}\kern.3em}
\def\ie{{i.e.}\kern.3em}
\def\iec{{i.e.,}\kern.3em}
\def\idc{{id.,}\kern.3em}
\def\resp{{resp.}\kern.3em}
\def\mod{{\rm{mod}}\kern.3em}


\def\bfC{\mathbf{C}}
\def\bfR{\mathbf{R}}
\def\bfZ{\mathbf{Z}}


\def\sB{\mathscr{B}}
\def\sF{\mathscr{F}}
\def\sG{\mathscr{G}}
\def\sN{\mathscr{N}}
\def\sO{\mathscr{O}}
\def\sS{\mathscr{S}}
\def\sR{\mathscr{R}}
\def\sY{\mathscr{Y}}
\def\sE{\mathscr{E}}

\def\bbF{\mathbb{F}}
\def\bbN{\mathbb{N}}
\def\bbQ{\mathbb{Q}}
\def\bbR{\mathbb{R}}
\def\bbW{\mathbb{W}}
\def\bbZ{\mathbb{Z}}

\def\frakm{\mathfrak{m}}


\def\frakM{\mathfrak{M}}



\def\calO{\mathcal{O}}


\makeatletter
%\renewcommand\chaptermark[1]{\markboth{\thechapter. #1}{}}

\def\cleardoublepage{\clearpage\if@twoside \ifodd\c@page\else
    \thispagestyle{empty}\hbox{}\newpage\if@twocolumn\hbox{}\newpage\fi\fi\fi}

%\renewcommand\tableofcontents{%
%    \if@twocolumn
%      \@restonecoltrue\onecolumn
%    \else
%      \@restonecolfalse
%    \fi
%    \chapter*{\contentsname
%        \@mkboth{%
%           \contentsname}{\contentsname}}%
%    \@starttoc{toc}%
%    \if@restonecol\twocolumn\fi
%    }
\makeatother



\renewcommand{\headrulewidth}{0pt}
\pagestyle{fancy}

%\lhead[\small\em \thepage]{}
%\rhead[]{\small\em \thepage}
%\chead[\small\em \leftmark]{\small\em \rightmark}
%\cfoot[]{}
%\rfoot[\href{../toc.pdf}{\footnotesize\color{red}{\bf\em Table of Contents}}]{\href{../toc.pdf}{\footnotesize\color{red}{\bf\em Table of Contents}}}

\lhead[\small\em \thepage]{}
\rhead[]{\small\em \thepage}
\chead[\small\em \leftmark]{\small\em \rightmark}
\cfoot[]{}

