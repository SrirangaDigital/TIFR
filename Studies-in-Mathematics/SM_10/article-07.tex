\chapter[ESTIMATES OF COEFFICIENTS OF MODULAR\hfill\break FORMS AND GENERALIZED MODULAR RELATIONS]{ESTIMATES OF COEFFICIENTS OF MODULAR FORMS AND GENERALIZED MODULAR RELATIONS}

\begin{center}
{\large By~ S. Raghavan}
\end{center}

\markboth{\textit{S. Raghavan}}{\textit{Estimates of Coefficients of Modular Forms...}}

\bigskip

\setcounter{pageoriginal}{246}
\textsc{We shall be}\pageoriginale concerned here with two questions, motivated by arithmetic, from the theory of modular forms. The first one deals with the estimation of the magnitude of the Fourier coefficients of Siegel modular forms, while the second pertains to certain generalized modular relations (which may also be called Poisson formulae of Hecke type and) which appear to provide some kind of a link between automorphic forms (of one variable), representation theory and arithmetic.

\section*{\S Modular forms of degree $n$}

Let $r_m(t)$ denote the number of ways in which a natural number $t$ can be written as a sum of $m$ squares of integers. We have the well-known Hardy-Ramanujan asymptotic formula [H-R] for $m > 4$:
\begin{equation*}
r_m(t) = \pi^{m/2} \sigma_m (t) t^{(m/2)-1} / \Gamma (m/2) + O(t^{m/4})\tag{1}\label{art7-eq1}
\end{equation*}
with $\sigma_m (t)$ denoting the `singular series'. Arithmetical functions such as $r_m(t)$ or, more generally, the number $A(S,t)$ of $m$-rowed integral columns $x$ with ${}^tx S x = t$ for a given $m$-rowed integral positive-definite matrix $S$ (where ${}^tX$ = transpose of $x$) occur as Fourier coefficients of modular forms. While Hardy and Ramanujan used the `circle method' to prove \eqref{art7-eq1}, the approach of Hecke \cite{art7-H1} to \eqref{art7-eq1} was via the decomposition of the space of (entire) modular forms into the subspace generated by Eisenstein seris and the subspace of cusp forms, the explicit determination of the Fourier expansion of Eisentein series and the estimation of the Fourier coefficients $c(t)$ of cusp forms of weight $k$ as $c(t) =O(t^{k/2})$.

More generally, let $A(S,T)$ be the number of integral matrices $G$ such that ${}^t G S G = T$ for $n$-rowed integral $T$ (For any matirx $B$, let ${}^t B$ denote its transpose and for a square matrix $C$, let $tr(C)$ and det $C$ denote its trace and determinant respectively). For $A (S, T)$, we have, as a `generating function', the theta series $\vartheta (S, Z)= \sum\limits_G \exp (2 \pi \sqrt{-1} tr ({}^t G S G Z))$ where $G$\pageoriginale runs over all $(m,n)$ integral matrices and $Z$ is in the Siegel half-plane `$\sfH_n$' of $n$-rowed complex symmetric matrices $Z = (z_{ij})$ with $Y= (y_{ij})$ positive definite and $y_{ij} = \Iim z_{ij}$; further, the theta series is a modular form of degree $n$, weight $m/2$ and stufe 4 det $S$. Let $\Gamma_n(s)$ denote the principal congruence subgroup of stufe $s$ in the Siegel modular group of degree $n$ and $\{\Gamma_n (s), k\}$ denote the space of modular forms of degree $n$, weight $k$ and stufe $s$. Pursuing the approach of Hecke and Petersson and using Siegel's generalized Farey dissection \cite{art7-S}, the following result was proved in \cite{art7-R}: namely, if $k > n+1$ and $f(Z) = \sum\limits_{T \geq 0} a(T) \exp (2 \pi \sqrt{-1} tr (TZ)/s)$ $\epsilon \{\Gamma_n (s), k\}$, there exists a linear combination $g(z) = \sum\limits_{T \geq 0} b(T) \exp (2 \pi \times \sqrt{-1} tr (TZ)/s)$ of Eisenstein series in $\{\Gamma_n (s), k\}$ such that for positive-definite
\begin{equation*}
T, a(T) = b(T) + O((\min T^{-1})^{n(n+1-2k)/2} (\min T)^{(n+1-k)/2}\tag{2} \label{art7-eq2}
\end{equation*}
(For positive definite $R$, $\min R$ is the first minimum in the sense of Minkowski). Specialising $f$ to be $\varepsilon(s,z)$ above, \eqref{art7-eq2} implies the formula:
\begin{equation*}
A (s, T) = \lambda \prod\limits_{p} \alpha_p  (S, T) (\det T)^{(m-n-1)/2} + O ((\det T)^{(m (2n -1) -2 (n^2 -1)) / 4n})\tag{3} \label{art7-eq3}
\end{equation*}
where $m > 2 n + 2$, 
$$
= \pi^{n (2m-n+1)/4} (\det S)^{-n/2} \{\Gamma (m/2) \ldots \Gamma((m - n +1)/2)\}^{-1}, 
$$
$\prod\limits_p \alpha_p (s, T)$ is the product (over all primes $p$) of the $p$-adic densities $\alpha_p (S,T)$ of representation of $T$ by $S$; further, in \eqref{art7-eq3}, $T$ tends to infinity such that for a fixed constant $c$, $\min T \geqslant c (\det T)^{1/n}$. From \eqref{art7-eq3}, an analogue of a theorem of Tartakowsky resulted for $n = 2$ \cite{art7-R}: namely, under the conditions above, for larde $\det T, A (S, T) \neq 0$ for every matrix in the `genus' of $S$ or for none at all, depending on certain congruence classes to which $T$ belongs. It should be mentioned that, without using Siegel's generalized Farey dissection, only estimates of the type $a (T) = O((\det T)^k)$ could be derived, in general, earlier; for improving upon \eqref{art7-eq2}, it was felt that the decomposition of the space of modular forms of degree $n$ into $n+1$ subspaces through Maass' Poincar\'e series should be invoked. 

Hsia, Kitaoke and Kneser \cite{art7-H-K-K} obtained, using an arithmetic aproach, a very elegant proof of the analogue of Tartakowsky's theorm for any $n \geqslant 1$ and $m \geqslant 2 n +3$. Quite recently, Kitaoka \cite{art7-KI} gave an analytic proof of the same result in the case when $S$ is an even positive definite $m$-rowed unimodular matrix with $m \geqslant 4n + 4$. By considering for\pageoriginale even $k \geqslant n + r + 2$, $Z \epsilon$ `$\sfH_n$' and $0< r < n$, the Eisenstein series $E(Z,h) = E^k_{n,r} (Z,h)$ which have been studied by Klingen \cite{art7-KL} and which arise by `lifting' a cusp form $h$ in $\{\Gamma_r (1), k\}$ to $\{\Gamma_n (1), k\}$, Kitaoka has obtained, in the same paper, the estimate $a(T, h) = O((\det T)^{k-(n+1)/2} \times (\det T_1)^{(r+1-k)/2})$ for the Fourier coefficients $a(T,h)$ of $E(Z,h)$ with $T = \begin{pmatrix} T_1 & \ast\\ \ast & \ast \end{pmatrix}$ and $r$-rowed symmetric $T_1$. If $f$ is in $\{\Gamma_n(1),k\}$ with even $k \geqslant 2n +2$ and $\Phi^n f = 0$ for the Siegel operator $\Phi$, then for the Fourier coefficients a $(T)$ of $f$ with positive definite $T$, Kitaoka derived, as a consequence, the estimate
\begin{equation*}
a(T) = O((\det T)^{k - (n+1)/2} (\min T)^{1-k/2} )\tag{4} \label{art7-eq4}
\end{equation*}

From \cite{art7-C}, it can be seen that any $f$ in $\{\Gamma_n (s), k\}$ for $k > 2 n +1$ is a finite linear combination of Poincar\'e series $G_k (Z; \Gamma_n(s); T)$ and their transforms under coset representatives of $\Gamma_n$ \eqref{art7-eq1}  modulo $\Gamma_n(s)$ for non-negative definite $T$. Following Kitaoka's method with appropriate modifications (e.g. of Lemma 7, \S 2, \cite{art7-KI}), it is not hard to prove the following 

\begin{theorem*}
If $f(Z) = \sum\limits_{T \geqslant 0} a (T) \exp (2 \pi \sqrt{-1} tr (TZ)/ z) \epsilon \{\Gamma_m (s), k\}$ with $k > 2 n +1$ is such that for every $M = \begin{pmatrix}A & B \\ C & D \end{pmatrix}$ in $\Gamma_n$ (1), the constant term in the Fourier expansion of $f((AZ+B)(CZ+D)^{-1}) \det (CZ+ D)^{-k}$ is $0$, then we have $a(T) = O((\delta T)^{k-(n+1)/2} (\min T)^{1-k/2})$, for positive definite $T$.
\end{theorem*}

Kitaoka \cite{art7-KI} has conjectured that the above theorem is true even for $2k \geqslant 2n +3 $. One can also consider the analogues of the theorem above her hermitian and Hilbert-Siegel modular forms. 

\section*{\S Poisson formulae of Hecke type.} 
Arithmetical identities have played a useful role in the estimation of the order or the average order of arithmetical functions. For Ramanujan's function $\tau(n)$, we have an interesting identity
$$
\sum\limits_{1\leqslant n <\infty} \tau (n) \exp (-s\sqrt{n}) = 2^{36} \pi^{23/2} \Gamma (25/2) s \sum\limits_{1 \leqslant n < \infty} \tau (n) (s^2 + 16 \pi^2n)^{-2 \;5/2} 
$$
for $s > 0$, which looks more involved than the `theta-relation'
$$
\sum\limits_{1 \leqslant n <\infty} \tau(n) \exp (-ny) = (2 \pi /y)^{1\; 2} \sum\limits_{1 \leqslant n < \infty} \tau (n) \exp (-4 \pi^2 n /y) \quad (y > 0).
$$ 
Such\pageoriginale identities (or modular relations as they are referred to in the literature) seem to be included by ``Poisson formulae of Hecke type'' considered by Igusa \cite{art7-I}, which may thus be called \textit{generalized modular relations.}

Let $\sF$ be the space of complex-valued $C^\infty$ functions $F$ on the space $\bbR^x_+$ of positive real numbers which behave like Schwartz functions at infinity and which have, as $t$ tends to $0$, an asymptotic expansion $F(t) \approx \sum\limits_{r \geqslant 0} a_r t^r$ which is termwise differentiable (infinitely often). Let $\sZ$ be the space of complex-valued functions $Z$ on the complex plane such that $Z(s)/\Gamma (s)$ is entire in $s$ and further, for every polynomial $P$, the functions $ZP$ is bounded in any vertical strip $\{s|\alpha \re s \leqslant \beta\}$ with neighbourhoods of $0, -1, -2, \ldots$ removed therefrom. The usual Mellin transform $F \mapsto M F$ established a one-one correspondence between $\sF$ and $\sZ$. On the other hand, for any real $\kappa > 0$, there exists in $\sZ$, an involution $Z \mapsto Z^\times$ with $Z^\times (s) = Z (\kappa -s)\Gamma (s) / \Gamma (\kappa -s)$ and this carries over to a unitary operator $F \mapsto \bbW F$ in $\sF$. If $\varphi (s) = \sum\limits_{1 \leqslant n < \infty} a_n n^{-s}$ is a Dirichlet series  (absolutely convergent in a half-plane and) of signature $\{\lambda, \kappa, \gamma\}$ in the sense of Hecke \cite{art7-H2} so that $(s -\kappa) \varphi (s)$ is entire and of finite genus and further $\xi (s) =(\lambda / 2 \pi)^s \Gamma (s) \varphi (s) = \gamma \xi (\kappa -s)$, then the Poisson formula established by Igusa in \cite{art7-I} reads:
\begin{equation*}
\sum\limits_{0\leqslant n < \infty} a_n (\bbW F)(2 \pi n /\lambda) = \gamma \sum\limits_{0 \leqslant n < \infty} a_n F (2 \pi n /\lambda)\tag{5} \label{art7-eq5}
\end{equation*}
for every $F$ in $\sF$, where $a_o = \gamma (\lambda/ 2 \pi)^\kappa \Gamma (\kappa)$. Residue $\varphi (s)$. This includes a result of Yamazaki.

Let $G (s) = \prod\limits_{1 \leqslant j \leqslant r} (\Gamma (\alpha_j s + \beta_j))^{m_j}$ with $\alpha_j > 0$, $\re \beta_j \geqslant 0$, $m_j \geqslant 1$ and further, for $i \neq j$, $\alpha_i \beta_j - \alpha_j \beta_i$ is not of the form $m \alpha_j - n \alpha_i$ with integers $m$, $n \geqslant 0$.
$$
\text{Let } \{\varphi_j (s) =\sum\limits_{n \neq 0}  a^{(j)}_n |n|^{-s}; 1 \leqslant j \leqslant N \} \text{ and}
$$
$\{\psi_j (s) = \sum\limits_{n \neq 0} b^{(j)}_n |n|^{-s}; 1 \leqslant j \leqslant N\}$ be two sets of $N$ Dirichlet series (each converging in some right half-plane absolutely) so that if we write 
\begin{equation*}
\xi_j (s) = \lambda^s G (s) \varphi_j (s), \eta_j (s) = \lambda^s G(s) \psi_j (s) \tag*{$(1\leqslant j \leqslant N)$}
\end{equation*}
for\pageoriginale some fixed $\lambda > 0$, then we have the functional equations 
\begin{equation*}
\xi_j (\kappa - s) = \sum\limits_{1 \leqslant k \leqslant N} c_{jk} \eta_k (s) \quad (1 \leqslant j \leqslant N)\tag{6} \label{art7-eq6}
\end{equation*}
with real $c_{jk}$; we may suppose that $(c_{jk})^2$ is the identity matrix and also that $\xi_k$, $\eta_1$ have only finitely many poles. Following Igusa \cite{art7-I}, the spaces $\sF$, $\sZ$ may be redefined so that $\sZ$ consists, for example, only of meromorphic functions $Z$ on the complex plane such that $Z/G$ is entire and $PZ$ is bounded in `vertical strips' (with neighbourhoods of poles removed) for every polynomial $P$. In the space $\sF$, we have a unitary operator $\bbW$ such that for every $F$ in $\sF$, $(M (\bbW F)) (s)/ G (s) = (MF) (\kappa -s)/ G (\kappa -s)$ for a $\kappa > 0$, $M$ being the Mellin transform. Let no $\xi_k$ have a pole on $\re s = \kappa /2$, for simplicity and let $u_1,\ldots, u_p$ be all the poles of $\xi_k$'s. Then we have a Poisson formula of Hecke type \cite{art7-R-R} given by the following

\begin{theorem*}
For any function $F: \bbR^x_+ \to \bbC$ whose Mellin transform $MF$ is such that $MF/G$ is entire and $P.MF$ is bounded in vertical strips (with neighbourhoods of poles removed) for every polynomial $P$ and for $\xi_1, \ldots, \xi_N, \eta_1, \ldots \eta_N$ satisfying \eqref{art7-eq6}, we have 
\begin{align*}
&\sum\limits_{n \neq 0} a^{(k)}_n F (|n|/\lambda) - \sum\limits_{\re u_j < \kappa /2} {\displaystyle\mathop{\text{ Residue }}_{s = u_j}}\frac{MF (s)}{G(s)} \xi_k(s) = \tag{7}\label{art7-eq7}\\
&= \sum\limits_{1 \leqslant 1 \leqslant N} c_{kl} \left(\sum\limits_{n\neq 0} b^{(k)}_n (\bbW F) (|n|/\lambda) - \right. \notag\\
& \left. - \sum\limits_{\re \; u_j < \kappa /2}  {\displaystyle\mathop{\text{ Residue }}_{s = u_j}} \frac{(M (\bbW F)) (s)}{G (s)} \eta_1 (s) \right)
\end{align*}
\end{theorem*}

Formula \eqref{art7-eq7} generalizes some well-known relations of a similar nature considered, for example, by Maass \cite{art7-M1} in the Hecke theory of non-analytic automorphic forms and by B.C. Berndt. The proof of \eqref{art7-eq7} is on the same lines as in Hecke \cite{art7-H2}; the sum over residues has to be interpreted suitably in terms of the coefficients in the asymptotic expansions of $F$ and $\bbW F$ at 0 and the residue of the Dirichlet series involved and sometimes, it takes a simple form as in \eqref{art7-eq5}. A Poisson summation\pageoriginale formula for a generalized Fourier transformation due to Kubota can also be treated with arguments similar to those for \eqref{art7-eq7}. In the study of non-analytic automorphic forms, Maass \cite{art7-M2} has considered (for Dirichlet series) functional equations in matrix form involving a generalized $\Gamma$-function $\Gamma (s; \alpha, \beta)$ which is the Mellin transform of the standard Whittaker function $W(y; \alpha, \beta)$; in this case again, a general Poisson formula like \eqref{art7-eq7} for pairs $(F_1, F_2)$ of $C^\infty$ functions on $\bbR^x_+$ with prescribed behaviour at infinity and at 0 can be obtained. Specialising $F_1 (t)$, $F_2 (t)$ to be $W (ty; \alpha, \beta)$, $W(ty; \beta, \alpha)$ respectively (with $y > 0$), one gets the corresponding formula in \cite{art7-M2}; in the light of a recent paper of Ranga Rao, it turns out that there are quite a few pairs $(F_1, F_2)$ for which our Poisson formula holds.

In the context of formula \eqref{art7-eq5} proved in the lectures \cite{art7-I}, one comes across the natural question as to whether a $p$-adic analogue of the operator $\bbW$ exists. One may consider, instead of $\sF$ above, the space $\sF (\bQ^x_p)$ of complex-valued $F$ on $\bbQ^x_p$ which are locally constant, with 
\begin{align*}
F(t) = 
\left\{ 
\begin{cases}
& \text{0 for all  $t$ with valuation $|t|_p$ large}\\
& a \mu_1 (t) |t|^{\frac{1}{2}}_p + b \mu_2 (t) |t|^{1/2}_p \text{ for all } t \text{ with } |t|_p small 
\end{cases}
\right\}\tag{8} \label{art7-eq8}
\end{align*}
constants $a$, $b$ and quasicharacters $\mu_1$, $\mu_2$. This is a so-called Kirillov model for irreducible admissible representations $\pi_p$ of $GL_2 (\bbQ_p)$. In this case, if $L(s, \pi_p) = \{(1-\mu_1 (p)p^{\frac{1}{2}-s}) (1-\mu_2 (p) p^{\frac{1}{2} -s})\}^{-1}$, then the $\bbW$-operator is given again via the Mellin transform $M$:
\begin{equation*}
\frac{(M (\bbW F)) (1-s)}{L(1-s, \pi_p)} =\epsilon (s, \pi_p) \frac{(MF) (s)}{L(s, \pi_p)}\tag{9} \label{art7-eq9}
\end{equation*}
with a certain function $\epsilon (s, \pi_p)$ for which $\epsilon (s, \pi_p)$. $\epsilon (1-s ,\pi_p) =1$. Let $W^0_p$ be the Whittaker function on $GL_2 (\bbQ_p)$ whose Mellin transform (over $\bbQ^x_p$) is $L(s, \pi_p)$, for every prime $p$ and further let $\{\pi_p\}_p$ be such that together with a representation $\pi_\infty$ of $GL_2 (\bbR)$, the tensor product $\pi_\infty \otimes_p \pi_p$ gives an irreducible unitary representation of $GL_2 (\bbQ_A)$ and moreover, let $\prod\limits_p L (s, \pi_p)$ be a Dirichlet series $\sum\limits_{n \neq 0} a_n |n|^{-s}$ converging absolutely in a right $s$-half plane, with a functional equation $s \to 1 -s$, involving $L(s, \pi_\infty) = (2 \pi)^{-s-(p+1)/2} \Gamma (s+(p+1)/2)$\pageoriginale for $p \geqslant 0$ in $\bbZ$ or $\pi^{-s - v } \times \Gamma ((s+v)/2) \Gamma ((s-v)/2)$ with $v$ in $\bbC$. Then for $F$ on $\bbQ^\times_A$ built from $\sF$ and the various $W^0_p$, we have an adelic analogue of our Poisson formula. Under specialization, a formula of this kind constitutes an important step in the Jacquet-Langlands' theory, for showing that a global representation of $GL_2 (\bbQ_A)$ occurs in the space of cusp forms. Further details may be found in \cite{art7-R-R}.

\begin{thebibliography}{99}
\bibitem[C]{art7-C} \textsc{Christian U.:} Uber Hilbert-Siegelsche Modulformen and Poincar\'esche Reihen, \textit{Math. Ann.} 148 (1962), 257-307.

\bibitem[H-R]{art7-H-R} \textsc{Hardy G. H.} and \textsc{S. Ramanujan:} Asymptotic formulae in combinatory analysis, \textit{Proc. London Math. Soc.,} (Ser 2) 17 (1918), 75-115.

\bibitem[H1]{art7-H1} \textsc{Hecke E.:} Theorie der Eisensteinscher Reihen h\"oherer Stufe und ihre Anwendung auf Funktionentheorie und Arithmetik, \textit{Abh. Math. Sem. Univ. Hamburg,} 5 (1927), 199-224; Gesamm Abhand, 461-486.

\bibitem[H2]{art7-H2} \textsc{Hecke E.:} Uber die Bestimmung Dirichletscher Reihen durch ihre Funktional-gleichungen, \textit{Math. Ann.} 112 (1936), 664-699; Gesamm. Abhand., 591-626.

\bibitem[H-K-K]{art7-H-K-K} \textsc{Hsia J. C., Y. Kitaoka} and \textsc{M. Kneser:} Representation of positive definite quadratic forms, \textit{Jour. reine angew. Math.,} 301 (1978), 132-141.

\bibitem[I]{art7-I} \textsc{Igusa J.-I.:} \textit{Lectures on forms of higher degree,} Tata Institute of Fundamental Research, 1978.

\bibitem[KI]{art7-KI} \textsc{Kitaoka Y.:} Modular forms of degree $n$ and representation by quadratic forms (Preprint).

\bibitem[KL]{art7-KL} \textsc{Klingen H.:} Zum Darstellungssatz f\"ur Siegelsche Modulformen, \textit{Math. Zeit.,} 102 (1967), 30-43.

\bibitem[M1]{art7-M1} \textsc{Maass H.:} Uber eine neue Art von nichtanalytischen automorphen Funktionen und die Bestimmung Dirichletscher Reihen durch Funktional-gleichungen, \textit{Math. Ann.,} 121 (1949), 141-183.

\bibitem[M2]{art7-M2} \textsc{Maass H.:} Die Differentialgleichungen in der Theorie der elliptischen Modulfunktionen, \textit{Math. Ann.,} 125 (1953), 233-263.

\bibitem[R]{art7-R} \textsc{Raghavan S.:}\pageoriginale Modular forms of degree $n$ and representaion by quadratic forms, \textit{Annals Math.,} 70 (1959), 446-477.

\bibitem[R-R]{art7-R-R} \textsc{Raghavan S.} and \textsc{S. S. Rangachari:} Poisson formulae of Hecke type. V. K. Patodi Memorial Volume; Indian Academy of Sciences (1980), 129-149. 

\bibitem[S]{art7-S} \textsc{Siegel C. L.:} On the theory of indefinite quadratic forms, \textit{Annals Maths.,} 45 (1944), 577-622; Gesamm. Abhand. II, 421-466.
\end{thebibliography}

















