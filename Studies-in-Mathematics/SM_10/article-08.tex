\chapter[A REMARK ON ZETA FUNCTIONS OF ALGEBRAIC\hfill\break NUMBER FIELDS$^{1}$]{A REMARK ON ZETA FUNCTIONS OF ALGEBRAIC NUMBER FIELDS$^{1}$}
\footnotetext[1]{Results presented at the time of the Colloquium were relevant to automorphic forms on unitary groups of order 3. However, later the author found several gaps in the proof of those results. Here, another result obtained after the Colloquium is exposed.}

\begin{center}
{\large By~ Takuro Shintani\footnotetext[2]{Takuro Shintani suddenly passed away on November 14, 1980. Ed.}}
\end{center}

\bigskip

\setcounter{pageoriginal}{254}
\section*{Introduction}

 For\pageoriginale a totally real algebraic number field $k$, it is known that every (partial) zeta function of $k$ is a finite sum of Dirichlet series which are regarded as natural generalizations of the Hurwits zeta function (see \cite{art8-1} and \cite{art8-2}). In this note we show that the similar result holds for arbitrary (not necessarily totally real) algebraic number field. At the time of the Bombay Colloquium (1979), H. M. Stark orally communicated to the author that he has obtained such a result for non-real cubic fields. His oral communication was an initial impetus to the present work. The author wishes to express his gratitude to Stark.

Notation. We denote by $\bZ$, $\bQ$, $\bR$ and $\bC$ the ring of rational integers, the field of rational numbers, the field of real numbers and the field of complex numbers respectively. The set of positive real numbers is denoted by $\bR_+$. For an algebraic number field $k$, we  denote by $E(k)$ and $O(k)$ the group of units of $k$ and the ring of integers of $k$ respectively.

1.~Let $V$ be an $n$-dimensional real vector space. For $\bR$-linearly independent vectors $v_1, v_2, \ldots, v_t \epsilon V (1 \leqslant t \leqslant n)$, we denote by $C (v_1, \ldots, v_t )$  the set of all \textit{positive} linear combinations of $v_1, \ldots, v_t$. We call 
$$
C(v_1, \ldots, v_t)
$$ 
a $t$-\textit{dimensional open simplicial cone with generators} $v_1, \ldots, v_t$. Note that generators of a given open simplicial cone are unique up to permutations and multiplications by positive scalars. We call a disjoint union of a finite number of open simplicial cones in $V$ a \textit{general polyhedral cone}. Thus a general polyhedral cone is not necessarily convex. Now assume that $V$ has a $\bQ$-structure. Thus, an $n$-dimensional $\bQ$-vector subspace $V_{\bQ}$ such\pageoriginale that one has $V = V_{\bQ} \bigotimes\limits_{\bQ} \bR$ is identified in $V$. An open simplicial cone is said to be $\bQ$-\textit{rational} if, for a suitable choice of generators, all generators are in $V_\bQ$. A disjoint union of a finite number of $\bQ$-rational open simplicial cones is said to be a $\bQ$-\textit{rational general polyhedral cone}.

A linear form on $V$ is said to be $\bQ$-rational if it is $\bQ$-valued on $V_\bQ$.

\medskip
\noindent
{\bfseries Lemma \thnum{1.}\label{art8-lem1}}
\textit{Let $C^{(1)}$ and $C^{(2)}$ be two $\bQ$-rational general polyhedral cones. Then $C^{(1)} - C^{(2)}$ is again a $\bQ$-rational general polyhedral cone.}

\begin{proof}
If is sufficient to prove the Lemma assuming that both $C^{(1)}$ and $C^{(2)}$ are $Q$-rational simplicial cones. Let $t$ be the dimension of $C^{(2)}$. There are $n$ $\bR$-linearly independent $Q$-rational linear forms $L_1, \ldots L_t$; $M_1, \ldots, M_{n-t}$ on $V$ such that 
\begin{align*}
C^{(2)} & = \left\{ v \epsilon V; L_a (v) > 0,\; a = 1 , \ldots, t, \right. \\
& \qquad \qquad \left. M_b (v) = 0, \; b = 1, \ldots, n -1 \right\}.
\end{align*}

For each $b(1 \leqslant b \leqslant n -t)$, set
\begin{gather*}
C^{(1)} (b , \pm) = \left\{ v \epsilon C^{(1)}; M_1 (v) = \ldots = M_{b-1} (v) =0,  \right.\\
\left. \pm M_b (v) > 0\right\}.
\end{gather*}
For each a $(1 \leqslant a \leqslant t)$, set 
\begin{multline*}
 C^{(1)} (n-t+1,a)
=\left\{v \epsilon C^{(1)}; M_b (v) = 0 \text{ for } b = 1, \ldots, n -t ,\right.\\
 \hspace{1cm} \left. L_1 (v) > 0, \ldots, L_{a-1} (v) > 0, L_a (v) \leqslant 0\right\}.
\end{multline*}
Then it is immediate to see that $C^{(1)} - C^{(2)}$ is a disjoint union of sets: $C^{(1)} (b, +) (1 \leqslant b \leqslant n -t)$, $C^{(1)} (b,-) (1 \leqslant b \leqslant n -t)$ and $C^{(1)} (n-t+1, a) (1\leqslant a \leqslant t)$. It follows from Lemma 2 of \cite{art8-1} and its corollary that $C^{(1)} (b, \pm) (1 \leqslant b \leqslant n -t)$ and $C^{(1)} (n - t + 1, a) (1\leqslant a \leqslant t)$ are all disjoint unions of finite number of $\bQ$-rational open simplicial cones. 
\end{proof}

2.~Let $k$ be an algebraic number field of degree $n$ with $r_1$ real and $r_2$ complex infinite primes $(n = r_1 + 2 r_2)$. Let $x \mapsto x^{(i)} \quad (1 \leqslant  i \leqslant n)$ be $n$ mutually distinct embeddings of $k$ into the field of complex numbers $\bC$. We may assume that $x^{(1)}, \ldots, x^{(r_1)}$ are all real and that $x^{(r_1+i)} = x^{-(r_1+r_2 + i)}$ $(1 \leqslant i \leqslant r_2)$.\pageoriginale We embed $k$ into an $n$-dimensional real vector space $V = \bR^{r_1} \times \bC^{r_2}$ via the map: $x \longmapsto (x^{(1)}, \ldots, x^{(r_1)}, x^{(r_1 +1)}, \ldots, x^{(r_1 + r_2)})$. We identify $k$ with an $n$-dimensional $\bQ$-vector subspace of $V$ by means of the embedding. Fix a $\bQ$-structure of $V$ by setting $V_\bQ = k$. Set $V_+ = \bR^{r_1}_+ \times (\bC)^{r_2}$, $k_+ = V_+ \cap k$ and $E(k)_+ = E (k) \cap k_+$. Thus $E(k_+)$ is the group of totally positive units of $k$. By componentwise multiplications, the group $E(k)_+$ acts on $V_+$.

\medskip
\noindent
{\bfseries Proposition \thnum{2.}\label{art8-prop2}} 
\textit{There exists a finite system $\{C_j; j \in J\} (|J| < \infty)$ of open simplicial cones with generators all in $k_+$ such that $V_+ = \bigcup\limits_{j \in J} \bigcup\limits_{u \in E (k)_+} uC_j$ (disjoint union).}

\begin{proof}
For each $x \in V$, we denote by $N(x)$ the ``norm'' of $x$ given by $N(x) = x^{(1)} \ldots x^{(r_1)} | x^{(r_1+1)} \ldots x^{(r_1+ r_2)} |^2$. Let $V^1_+$ be the subset of $V_+$ consisting of all vectors with norm 1:
$$
V^1_+ = \left\{x \in V_+ ; N (x) = 1 \right\}.
$$
Note that each vector in $V_+$ is uniquely expressed as a positive scalar multiple of a vector in $V^1_+ : x = \{N (x)^{1/n}\}\{N(x)^{-1/n}. x\}$.

If follows from the Dirichlet unit theorem that the group $E(k)_+$ acts on $V^1_+$ properly discontinuously and that $E(k)_+ / V^1_+$ is compact. Thus, there exists a compact subset $F$ of $V^1_+$ such that 
\begin{equation}
V^1_+ =\bigcup\limits_{u \in E (k)_+} \quad u F.\tag{1}\label{art8-eq1} 
\end{equation}
Note that the subset of $V^1_+$ gives as $\{N (x)^{-1/n} x ; x \in k_+\}$ is dense in $V^1_+$.

Hence for each $X \in F$, there exists an $n$-dimensional open simplicial cone $C$ with generators all in $k_+$ such that $x \in C \cap V^1_+$ and that $C \cap u C = \empty$ for any $1 \neq u \in E(k)_+$. Thus, there exists a finite system $C_1, \ldots, C_s$ of $n$-dimensional open simplicial cones with generators all in $k_+$ such that 
\begin{equation*}
F = \bigcup\limits^s_{i=1} \quad  (C_i \cap V^1_+) \tag{2}\label{art8-eq2}
\end{equation*}
and that 
\begin{equation*}
C_i \cap u C_i =\empty \text{ for any } 1 \neq u \in E (k)_+ (1 \leqslant i \leqslant s). \label{art8-eq3}
\end{equation*}
If follows\pageoriginale from \eqref{art8-eq1} and \eqref{art8-eq2} that 
$$
V_+ = \bigcup\limits^s_{i=1} \bigcup\limits_{u \in E (k)} u C_i.
$$
Set $C^{(1)}_1 = C_1$ and set
$$
C^{(1)}_i = C_i - \bigcup\limits_{u \in E (k)_+} u C_1 (2 \leqslant i \leqslant s).
$$
Note that $u C_1$ is disjoint to $C_i$ except for a finite number of $u$. Hence Lemma \ref{art8-lem1} implies that $C^{(1)}_i$ is a $\bQ$-rational general polyhedral cone. Taking \eqref{art8-eq3} into account, we have
\begin{gather*}
V_+ = \bigcup\limits^s_{i=1} \bigcup\limits_{u \in E(k)_+} uC^{(1)}_i \text{~ and }\\
u C^{(1)}_1 \cap C^{(1)}_i = \empty \text{~ for any ~} u \in E (k)_+ \text{ if } i \geqslant 2.
\end{gather*}
Now assume that a finite system of $\bQ$-rational general polyhedral cones $C^{(a)}_1,\ldots, C^{(a)}_s (1 \leqslant a \leqslant s -2)$ with the following three properties is given:
\begin{align*}
& C^{(a)}_i \subset C_i \tag*{(4)$_{(a)}$}\label{art8-eq(4a)}\\
& V_+ =  \bigcup\limits^s_{i=1} \bigcup\limits_{u \in E (k)_+} u C^{(a)}_i, \tag*{(5)$_{(a)}$}\label{art8-eq(5a)}\\
& uC^{(a)}_i \cap C^{(a)}_j = \empty \text{ for any } u \in E (k)_+ \text{ if } i \leqslant a \text{ and } i \neq j. \tag*{(6)$_{(a)}$}\label{art8-eq(6a)}
\end{align*}
 
Then set $C^{(a+1)}_i = C^{(a)}_i$ for $i \leqslant a + 1$ and set
$$
C^{(a+1)}_i = C^{(a)}_i - \bigcup\limits_{u \in E (k)} u C^{(a)}_{a+1} \text{ for } i \geqslant a+2.
$$
Then $\{C^{(a+1)}_1, \ldots, C^{(a+1)}_s\}$ is a finite system of $\bQ$-rational general polyhedral cones with properties $(4)_{a+1}$, $(5)_{a+1}$ and $(6)_{a+1}$.

It is easy to see that $\{C^{(s-1)}_1, \ldots, C^{(s-1)}_s\}$ is a finite system of $\bQ$-rational general polyhedral cones such that 
$$
V_+ = \bigcup\limits^s_{i=1} \bigcup\limits_{u \in E(k)_+} u C^{(s-1)}_i \qquad (\text{disjoint union}).
$$
\end{proof}

\begin{remark*}
For totally real fields $k$, Proposition 2 is obtained in \cite{art8-1} by a different method (cf. Proposition 4 of \cite{art8-1}).
\end{remark*}

3.~We choose and fix a finite system $\{C_j ; j \in J (|J| < \infty)\}$ of open simplicial cones with generators all in $k_+$ such that 
\begin{equation}
V_+ = \bigcup\limits_{j \in J} \bigcup\limits_{u \in E (k)_+} u C_j \quad (\text{disjoint union}). \tag{7} \label{art8-eq7}
\end{equation}
The\pageoriginale existence of such a system is guaranteed by Proposition \ref{art8-prop2}. For each $C_j$, we denote by $t_j$ the dimension of $C_j$ and choose and fix generators $v_{j1, \ldots, } V_{jt_j}$ of $C_j$ \textit{so that they are all in $O(k)_+ = O(k) \cap k_+$}.

Furthermore, we choose and fix integral ideals $a_1, a_2, \ldots, a_{h_0}$ so that they form a complete set of representatives for \textit{narrow} ideal classes of $k$. Lef $f$ be an integral ideal of $k$ and let $H_k(f)$ be the group of narrow ideal classes modulo $f$. There is a natural homomorphism from the group $H_k(f)$ onto the group of narrow ideal classes of $k$. Fro each $c \in H_k(f)$ there uniquely exists an index $i(c) (1 \leqslant i (c) \leqslant h_0)$ such that $c$ is mapped to the class represented by $fa_{i(c)}$.

Set
$$
C^1_j = \left\{s_1 v_{j1} + s_2 v_{j2} + \ldots + s_{t_j} v_{jt_j} ; 0 < s_1, s_2 , \ldots, s_{t_j} \leqslant 1 \right\}
$$
and
$$
R(c, C_j) = \{x \in C^1_j \cap f^{-1} a^{-1}_{i(c)} ;(x) fa_{i(c)} \in c\}.
$$
Then $R(c,C_j)$ is \textit{finite}.

Let $C$ be a $t$-dimensional open simplicial cone with a prescribed system of generators $v_1, \ldots, v_t$.

For each $x \in C$, we denote by $\zeta (s, C,x)$ the Dirichlet series given by
\begin{equation}
\zeta (s, C, x) = \sum\limits_z N (x + z_1 v_1+ \ldots + z_t v_t )^{-s}, \label{art8-eq8}
\end{equation}
where $z = (z_1, \ldots, z_t)$ ranges over the set of all $t$-tuples of non- negative integers (the notation $N$ is introduced at the beginning of the proof of Proposition \ref{art8-eq2}).

Let $\zeta_k(s,c)$ be the zeta functions of $k$ corresponding to the ray class $c$ given by
\begin{equation}
\zeta_k (s,c) = \sum\limits_{\sfg} N (\sfg)^{-s}, \tag{9}\label{art8-eq9}
\end{equation}
where $\sfg$ ranges over the set of all integral ideals of $k$ in the ray class $c$.

\medskip
\noindent
{\bfseries Proposition \thnum{3.}\label{art8-prop3}}
\textit{The notation and assumptions being as above.}
$$
\zeta_k (s,c) = N (\sff \sfa_{i(c)})^{-s} \sum\limits_{j \in J} ~~ \sum\limits_{x \in R (c,C_j)} ~~ \zeta (s, C_j, x).
$$

\begin{proof}
Let\pageoriginale $\sfg$ be an integral ideal in the ray class $c$. Then $\sfg$ and $\sff \sfa_{i(c)}$ are in the same narrow ideal class of $k$. Thus, for a suitable $w \in k_+$, $\sfg = \sff \sfa_{i(c)} (w)$. In view of \eqref{art8-eq7}, we may assume that $w \in C_j \cap k_+$ for a suitable $j \in J$.

Set $w = y_1 v_{j1} + \ldots + y_{t_j} v_{jt_j}$.

Then $y_1, \ldots, y_{t_j}$ are all positive rational numbers. Let the integer part of $y_a$ be $z_a (a=1, \ldots, t_j)$.

Then $x = w - (z_1 v_{j1} + \ldots + z_{t_j} v_{jt_j}) \in C^{1}_j\cap (\sff \sfa_{i(c)})^{-1}$.

Furthermore $(x) \sff \sfa_{i(c)}$ is in the ray class $c$.

Thus $x \in R (c, C_j)$. A simple consideration shows that $j$, $z_1, \ldots, z_{t_j}$ and $x$ are uniquely determined by $\sfg$.

On the other hand, for an $x \in R (c, C_j)$ and a $t_j$-tuple of non-negative integers $z = (z_1, \ldots, z_{t_j})$, $\sfa_{i(c)} \sff (x+ z_1 v_{j1} + \ldots + z_{t_j} v_{jt_j})$ is an integral ideal in the ray class $c$.

We denote by $\bZ_+$ the set of non-negative integers. We have seen that the following map establishes a bijection from the set $\bigcup\limits_{j \in J} \{R (c, C_j) \times \bZ^{t_j}_+\}$ onto the set of integral ideals of $k$ in the ray class $c$:
$$
(x, z) \in R (c, C_j) \times \bZ^{t_j}_+ \longmapsto \sfa_{i(c)} \sff (x+ \sum\limits^{t_j}_{a=1} z_a v_{ja}).
$$
Thus Proposition \ref{art8-prop3} now follow immediately from \eqref{art8-eq9}.
\end{proof}

\begin{remark*}
For totally real field $k$, Proposition \ref{art8-prop3} is given in the proof of Theorem 1 of \cite{art8-1} (see also \cite{art8-2}).
\end{remark*}

\begin{thebibliography}{99}
\bibitem[1]{art8-1} \textsc{Shintani, T.} On evaluation of zeta functions of totally real algebraic number fields at non-positive integers, \textit{J. Fac. Sci. Univ. Tokyo Sec.} IA. 23(1976), 393-417.

\bibitem[2]{art8-2} \textsc{Zagier, D.} A Kronecker limit formula for real quadratic fields, \textit{Math. Ann.} 231(1975), 153-184.
\end{thebibliography}
