\chapter{DERIVATIVES OF L-SERIES AT $S=0$.}


\begin{center}
{\large By~ H. M. Stark}
\end{center}

\bigskip

\setcounter{pageoriginal}{260}

\section{Introduction}\label{art9-sec1}
In\pageoriginale 1970, I introduced \cite{art9-5} a rather vague general conjecture on values of Artin $L$-series at $s =1$. Since then the conjecture has been considerably refined, especially for certain types of characters \cite[II, III, IV]{art9-6}. It is appropriate to present a paper on this subject here since it was at the Tata Institute that the complex quadratic case was treated in the lectures of Siegel \cite{art9-4} and later work of Ramachandra \cite{art9-3}. It has become clear in recent years that the formulas at $s = 0$, although equivalent to formulas at $s =1$ via the functional equation, are considerably simpler. In this paper, we will concentrate on the case of Artin $L$-series with first order zeros at $s =0$. Included in this category of $L$-series are the abelian $L$-series over complex quadratic ground fields studied by Ramachandra. Since his results have been improved, this is a good place to begin.

\section{Complex quadratic ground fields}\label{art8-sec2} 
Let $k$ be a complex quadratic field, $\sff$ an integral ideal of $k$, $\sff \neq (1)$. Suppose $G(\sff)$ is the ray class group of $k(\mod \sff)$ and let $J$ be a subgroup of $G(\sff)$  and $K$ the class field corresponding to $H = G (\sff)/J$. The characters $\chi$ of $H$ are precisely those ray class characters of $k (\mod \sff)$ which are identically 1 on $J$. We let $L (s, \chi)$ denote the $L$-series corresponding to the primitive version of $\chi$ and $L(s,\chi, \sff)$ denote the $L$-series corresponding to the (possibly imprimitive)  character $\chi (\mod \sff)$. This is the series that results from $L(s, \chi)$ by deleting the $p$-factors from the Euler product of $L(s, \chi)$ for each $\sfp/\sff$.

Our improvement of Ramachandra's result is the following theorem which is proved in \cite[IV]{art9-6}.

\begin{theorem*}
For each coset $\sfc$ of $J$ in $G(\sff)$, there is an algebraic integer $\varepsilon (\sfc)$ such that the following three properties hold:
\begin{itemize}
\item[i).] For each character $\chi$ of $H$,
$$
L' (0 \chi, \sff) = - \frac{1}{W} \sum\limits_{c \in H} \chi (\sfc) \log (|\varepsilon (\sfc)|^2) 
$$
where $W$ is the number of roots of unity in $K$.

\item[ii).] The\pageoriginale explicit reciprocity law is given by
$$
\varepsilon (J)^{N(\rho)} \equiv \varepsilon (\sfc) (\mod \sfp)
$$
where $\sfp$ is a prime ideal in $\sfc$. Further, $\varepsilon (\sfc)/ \varepsilon (J)^{N(\sfp)}$ is a $W^{\text{th}}$ power of a number in $K$ and the $\varepsilon (\sfc)$ are all associates. 

\item[iii).] If $\sff = \sfp^a$ where $\sfp$ is a prime ideal, then
$$
N_{K/\bbQ} (\varepsilon (\sfc)) = N_{k/\bbQ} (\sfp)^b
$$
where 
$$
b = \frac{Wh}{w}
$$
and $h$ is the class-number of $k$, $w$ the number of roots of unity in $k$. In all other cases, $\epsilon (\sfc)$ is a unit. 
\end{itemize}
\end{theorem*}

Actually, part iii) is a simple corollary of part i) with $\chi$ being the (imprimitive) trivial character of $\bH$ since by part ii) the  $\varepsilon (\sfc)$ are the conjugates of $\varepsilon (J)$.

As an example, suppose $k = \bQ (\sqrt{d})$ has class-number one and $\sff=p^a$ for a first degree prime ideal $\sfp$ of norm $p$ relatively prime to 6d. Here $W = w$ and for each character $\chi$ of $G(\sff)$, we have 
$$
L' (0, \chi, \sff) = \frac{-1}{W} \sum\limits_c \chi (\sfc) \log (|\varepsilon (\sfc)^2)
$$
where $\epsilon (\sfc)$ is in the ray class field $K(\sff)$ of $k (\mod \sff)$. The norm of $\epsilon (\sfc)$ from $K(\sff)$ to $\bbQ$ is $p$. By our Theorem,
$$
\frac{\varepsilon (\sfc)}{\varepsilon (\sfc_0)} = \varepsilon_c^w
$$
where $\sfc_0$ is the principal ray class $(\mod \sff)$, $\varepsilon_c$ is in $\bK(\sff)$ and is a unit. As we show in \cite[IV]{art9-6} by the theory of group determinants as discussed by Siegel \cite{art9-4}, the units $\varepsilon_\sfc$, $\sfc \neq \sfc_0$, together with the $w^{\text{th}}$ roots of unity generate a subgroup of the unit group of $\bK(\sff)$ of index precisely the class-number of $K(\sff)$. All previous results in this direction have had a much larger index. The situation in this example figured strongly in the work of Coates and Wiles \cite{art9-2}.

For the rest of this section, we will suppose that $\sff^\tau = \sff$ and$J^\tau = J$ where $\tau$ denotes complex conjugation. Thus the field $K$ is normal over $\bbQ$. We identity $H$ with the Galois group of $K/k$ via our Theorem and now write\pageoriginale 
$$
L' (0, \chi, \sff) = \frac{-1}{W} \sum\limits_{h \in H} \chi (h) \log (|\varepsilon^h|^2)
$$
where $\varepsilon = \varepsilon (J)$. We let $G$ denote the Galois group of $K/\bbQ$. We will denote the characters of $G$ by the Greek letter $\psi$ while continuing to denote the characters of $H$ by $\chi$. In particular, if $\psi$ is the character of $G$ induced by $\chi$ then for any $h$ in $H$,
\begin{align*}
\psi (h) & = \chi (h) + \chi (\tau h \tau^{-1}),\\
\psi (h\tau) & = 0.
\end{align*}
It turns out that $\varepsilon$ was constructed so that some power of $\varepsilon$ is real. Therefore,
$$
|\varepsilon^{\tau h \tau^{-1}}| = |\varepsilon^h|
$$
and it follows that 
\begin{align*}
L'(0,\psi, N (\sff)) & = L'(0, \chi, \sff)\\
& = \frac{-1}{2W} \sum\limits_{g \in G} \psi (g) \log (|\varepsilon^g|^2)
\end{align*}
Although different in appearance, this is equivalent to the general conjecture in \cite[II]{art9-6} for this case with ``fudge constant'' $-1/(2W)$.

To illustrate some of the possibilities that occur, we will take as an example the case where $G$ is the dihedral group of order 8 with generators $\sigma$, $\tau$ and relations $\sigma^4 = \tau^2 = 1$, $\sigma \tau = \tau \sigma^3$. This group arises over $\bQ (\sqrt{-19})$ with $\tau$ being complex conjugation and 
$$
H = H_{-19} = \{1, \sigma^2, \sigma \tau, \sigma^3 \tau\},
$$
the Klein four group. 
$$
\xymatrix{
& & K \ar@{-}[d]^-{\sigma^2} \ar@{-}[dll]_-{\sigma^2 \tau} \ar@{-}[dl]^-{\tau} \ar@{-}[dr]^-{\sigma \tau} \ar@{-}[drr]^-{\sigma^3 \tau} & & \\
K^{(2)}_{17} \ar@{-}[dr]& K^{(1)}_{17} \ar@{-}[d] & K_{17, -19} \ar@{-}[d] \ar@{-}[dr]\ar@{-}[dl] & K^{(1)}_{-19} \ar@{-}[d]& K^{(2)}_{-19} \ar@{-}[dl]\\
& k_{17}  \ar@{-}[dr]& k_{-323}\ar@{-}[d] & k_{-19} \ar@{-}[dl]& \\
& & \bbQ& & 
}
$$

There\pageoriginale is a pair of prime ideals of norm 17 in $Q(\sqrt{-19})$, $\sfp^{(1)}_{17} = \left(\dfrac{7+\sqrt{-19}}{2}\right)$ and $\sfp^{(2)}_{17} = \sfp^{(1)}_{17} \tau = \left(\frac{7-\sqrt{-19}}{2} \right)$. For $j = 1, 2$, there are unique ray class characters $\chi^{(i)} (\mod \sfp^{(i)}_{17})$ of order two. They are primitive characters and give rise to ray class fields $K^{(1)}_{-19}$ and $K^{(2)}_{-19} = K^{(1) \tau}_{-19}$. The composite field $K$ comes from the ray class group $(\mod 17)$ modulo a subgroup of index 4 where both $\chi^{(1)}$ and $\chi^{(2)}$ are defined. Further, $G (K/\bbQ) = G$, the dihedral group of order 8. The product character $\chi^{(1)} \chi^{(2)}$  of order two is a primitive character ($\mod 17$) and corresponds to the class field $K_{17, -19 } = \bQ (\sqrt{17}, \sqrt{-19})$. Of the five possibilities, we see that $\{1, \sigma \tau\}$ or $\{1, \sigma^3 \tau\} = \tau^{-1} \{1, \sigma \tau\} \tau$ must be $G (K^{(1)}_{-19}/k_{-19})$ and we assume $\sigma $ has been picked so that $G(K^{(1)}_{-19}/ k_{-19}) = \{1, \sigma \tau\}$. This makes $H$ as claimed. There are two other quadratic subfields of $K: k_{17} = \bbQ (\sqrt{17})$ and $k_{-323} = \bbQ (\sqrt{-323})$. We see that $G(K/k_{-323}) =\{1, \sigma , \sigma^2, \sigma^3\}$ is cyclic while $G(K/k_{17}) =\{1, \tau, \sigma^2, \sigma^2 \tau\}$ is the other Klein four group in $G$. (The real subfield of $K$ is fixed by $\{1, \tau\}$ and is not normal over $\bbQ$. This is what allows us to decide which group goes to which field.) The remaining two quartic subfields of $K$ are quadratic extensions of $k_{17} : K^{(1)}_{17}$ fixed by $\{1,\tau\}$ and $K^{(2)}_{17}$ fixed by $\{1, \sigma^2 \tau\}$
\begin{center}
{\renewcommand{\arraystretch}{1.2}
\tabcolsep=12pt
\begin{tabular}{l|c|c|c|c}
\hline
& 1 & $\sigma^2$  & $\sigma \tau$ & $\sigma^3 \tau$\\\hline
$\chi_1$ & 1 & 1 & 1& 1\\\hline
$\chi^{(1)}$ &1&-1&1&-1\\\hline
$\chi^{(2)}$ &1&-1&-1&1\\\hline
$\chi^{(1)} \chi^{(2)}$ & 1&1&-1&-1\\\hline
\end{tabular}}

\smallskip
Character table of $H_{-19}$
\end{center}
By our Theorem, there is a number $\pi$ in $K_{-19}$ of norm 17 such that 
$$
L' (\sfo, \chi^{(1)}, \sfp^{(1)}_{17}) = \frac{-1}{2} [\log (|\pi|^2) - \log (|\pi^{\sigma^2}|^2)].
$$
There is also such a number in $K^{(2)}_{-19}$ for $L' (0, \chi^{(2)}, \sfp^{(2)}_{17})$ but it is just $\bar{\pi}$ and the formula is the same. It is no surprise that the formula should give the same answer since $L(s, \chi^{(1)}, \sfp^{(1)}_{17})$ and $L(s, \chi^{(2)}, \sfp^{(2)}_{17})$ are the same Dirichlet\pageoriginale series. Indeed this series arises in several different ways. From the character table of $G$ (whose characters have been
\begin{center}
{\renewcommand{\arraystretch}{1.2}
\tabcolsep=12pt
\begin{tabular}{l|c|c|c|c|c}
\hline
1 & $\sigma^2$ & $\tau, \sigma^2 \tau$ & $\sigma \tau, \sigma^3 \tau$ & $\sigma, \sigma^3$\\
\hline
$\psi_1$ & 1&1&1&1&1\\
\hline
$\psi_{17}$ & 1&1&1&1&$-1$\\
\hline
$\psi_{-19}$ & 1&1&$-1$&1&$-1$\\
\hline
$\psi_{-323}$ & 1&1&$-1$&$-1$&1\\
\hline
$\psi_2$ & 2&$-2$&0&0&0\\
\hline
\end{tabular}}

\smallskip
Character table of $G$
\end{center}
given suggestive names), we see that $\chi^{(1)}$ and $\chi^{(2)}$ both give the same induced character of $G$, namely $\psi_2$. But $\psi_2$ also arises as an induced character from $G(K/k_{17})$ and $G(K/k_{-323})$. In particular, there is a primitive ray class character of $k_{17}$ modulo a prime ideal of norm 19 which corresponds to $K^{(1)}_{17}$. It takes the values $1$, $1, -1, -1$, at $1, \tau, \sigma^2, \sigma^2 \tau$ respectively and also induces $\psi_2$ on $G$. Further, by our Theorem there is a unit $E$ in $K^1_{-19}$ such that 
$$
\frac{\pi}{\pi^{\sigma^2}} = E^2
$$
With 
$$
\eta = |E|^2 = EE^\tau
$$
we have 
$$
L' (0, \psi_2) = \frac{-1}{2} \log (|E^2|^2) = - \log (\eta)
$$
where $\eta$ is in $K^{(1)}_{17}$. Also $E^{\sigma^2} = \pm 1/E$ so that $(E E^\tau)^{\sigma^2} = 1 /(EE^\tau)$ and hence, $\eta^{\sigma^2} = \eta^{-1}$. The unit $\eta$ is precisely what is called for in my conjecture for real quadratic $L$-series. However, I have proved my conjecture for relative quadratic extensions, such as $K^{(1)}_{17}/k_{17}$ without aid of complex multiplication.

We\pageoriginale return to $K/k_{-19}$ again and now consider $\chi^{(1)}$ and $\chi^{(2)}$ as imprimitive characters ($\mod 17$). According to our Theorem, there is a unit $\varepsilon$ of $K$ such that for any of the four characters $\chi$ of $H$,
$$
L' (0, \chi, 17) = - \frac{1}{2} \sum\limits_{h \in H} \chi (h) \log (|\varepsilon^h|^2).
$$
In fact, $\epsilon$ is real and so is also in $K^{(1)}_{17}$. The question then arises if $\varepsilon = \eta$. The answer to this question is related to the question as to why we bother with the imprimitive version of $L(s, \chi^{(1)})$ since $\chi^{(1)} (\sfp_{17}^{(2)}) = -1$ and so 
$$
L' (0, \chi^{(1)}, 17) = 2 L' (0, \chi^{(1)}). 
$$

It turns out that we get new units this way. For instance, since $\varepsilon$ is real, $\varepsilon^{\sigma 2 \tau} = \varepsilon^{\tau \sigma^2} = \varepsilon^{\sigma^2} = \varepsilon^{\sigma^2}$ so that $\varepsilon^{\sigma^2}$ is real and
$$
|\varepsilon^{\sigma \tau}| = |\varepsilon^{\tau \sigma \tau}| = |\varepsilon^{\sigma^3}| = |\varepsilon^{\sigma^3 \tau}|
$$
Hence 
\begin{gather*}
L'(0, \chi^{(1)}, 17) = - \log \left(\left|\frac{\varepsilon}{\varepsilon^{\sigma^2}} \right| \right)\\
 = -2 \log (\eta). 
\end{gather*} 

Of course $\eta/ \eta^{\sigma^2} = \eta^2$ and so $\varepsilon =\eta$ is still possible. However, 
\begin{align*}
L'(0, \chi^{(1)} \chi^{(2)}) & = - \log \left(\left|\frac{(\varepsilon \varepsilon^{\sigma 2})^2}{N_{K/k_{-19}}} (\varepsilon) \right| \right)\\
& = - 2 \log (|\varepsilon \varepsilon^{\sigma 2}|),
\end{align*}
while by Dirichlet's class-number formula, 
\begin{align*}
L'(0, \chi^{(1)} \chi^{(2)}) & = L (0, \psi_{-323}) L' (0, \psi_{17})\\
& = h (k_{-323}) h (k_{17}) \log(\varepsilon_{17})\\
& = 4 \log (\varepsilon_{17})
\end{align*}
Thus $\varepsilon \varepsilon^{\sigma^2} = \pm \bar{\varepsilon}^2_{17}$ while $\eta\eta^{\sigma^2} =1$ and so $\varepsilon \neq \eta$. Since 
$$
\frac{\epsilon}{\varepsilon^{\sigma^2}} = \frac{\varepsilon^2}{\varepsilon \varepsilon^{\sigma^2}}
$$
we also have a confirmation of the fact that $\varepsilon/ \varepsilon^{\sigma^2}$ is a square in $K$.

Thus\pageoriginale far, we have looked at $L' (0, \psi_2)$ in three different ways (twice over $k_{-19}$ and once over $k_{17}$) and found the three different numbers $\pi, \eta, \epsilon$ all leading to the same result. But we can also look at $L' (0, \psi_2)$ viewed over $k_{-323}$. In the table below, the two characters $\chi'$ and $\bar{\chi}'$ of order four of $H_{-323} = G(K/k_{-323})$ induce $\psi_2$. Here $K$ is actually the Hilbert class field of $k_{-323}$. This has the unfortunate consequence that the conductor of $L(s, \psi_2)$ viewed over $k_{-323}$ is (1) and our Theorem does not apply directly. However, we may make all four characters of $H_{-323}$ imprimitive by raising the conductor. It is tempting to use the 
\begin{center}
{\renewcommand{\arraystretch}{1.2}
\tabcolsep=12pt
\begin{tabular}{l|c|c|c|c}
\hline
& 1 & $\sigma$ & $\sigma^2$ & $\sigma^3$\\\hline
$\chi'_1$ & 1&1&1&1\\\hline
$\chi'$ & 1 &$i$ & $-1$ &$-i$\\\hline
$\chi'^2$ & 1&-1&1&-1\\\hline
$\bar{\chi'} = \chi'^3$ & 1&$-i$& $-1$ &$i$\\\hline
\end{tabular}}

\smallskip
Character table of $H_{-323}$
\end{center}
unique ideal $\sfp'_{17}$ of $k_{-323}$ of norm 17 as our conductor. Since $\sfp'_{17}$ is in the class of order two, the corresponding Frobenius automorphism of $H_{-323}$ is $\sigma^2$. (Note 17 ramifies from $\bQ$ to $K$ so we must be very careful in going from $H_{-323}$ to $G$ with Frobenius automorphisms.)

Hence, 
$$
L' (0, \chi', \sfp_{17}) = 2 L' (0, \chi'),
$$
(the same is true of $\bar{\chi}'$) and we are once again evaluating
$$
L' (0, \psi_2, 17) = 2 L' (0, \psi_2).
$$

We see from our Theorem that instead of getting $\varepsilon$ again, there is a number $\pi'$, in $K$ such that for all four characters $\chi$ of $H_{-323}$,
$$
L' (0, \chi, p'_{17}) =- \frac{1}{2} \sum\limits_{h \epsilon H_{-323}} \chi (h) \log (|\pi'^{h}|^2) 
$$
where\pageoriginale
$$
N_{K/\bbQ} (\pi') = 17^4
$$
Further $\pi'$ is not just $\pi^2$ or even $\pi^2$ times a unit since
$$
(\pi') = \sfp'_{17}
$$
so that $(\pi')^2 = (17) = \sfp^{(1)}_{17} \sfp^{(2)}_{17}$ while $(\pi^2) = \sfp^{(1)}_{17}$. Thus we have found still another number of $K$. Here again, $\pi'$ is real and so $L' (0, \chi', \sfp'_{17})$ simplifies to 
$$
-2 \log (\eta) = L' (0, \chi', \sfp'_{17}) = - \log\left(\left|\frac{\pi'}{\pi'^{\sigma^2}} \right| \right).
$$
where $\pi'/\pi'^{\sigma^2}$ is real and is a square in $K$.

The difficulty in using conductor (1) is that the trivial character gives $\zeta_{k_{-323}} (s)$ whose first derivative at zero is rather horrible. However, for the three non-trivial characters $\chi$ of $H_{-323}$, one can write all three $L' (0, \chi)$ simultaneously in terms of a nice number given by quotients of Dedekind eta-functions. But this simultaneous expression of all three $L$-series would appear to require a worse coefficient than $-1/2$ on the right side of the equation. It does seem possible to express any one of three $L' (0, \chi)'s$ in a nice manner. For instance, there is a number $\alpha$ in $K$ (non-integral) given by
$$
\alpha = 3 \frac{\eta(\omega)^2}{\eta(\omega/9)^2}, \quad \omega = \frac{1+\sqrt{-323}}{2},
$$
where we have used the eta-function on the right, and 
$$
L' (0, \chi') = - \log (|\alpha|^2)
$$
from which we see that 
$$
\eta = N_{K/K^{(1)}_{17}} (\alpha)
$$

\section{$L$-series considered over $Q$}\label{art9-sec3} 
In this section, $K$ is a normal extension of $\bQ$ with Galois group $G$ whose characters will again be denoted by $\psi$. We have seen in the last section that if $K$ has a complex quadratic subfield $k$ such that $G (K/k) = H$ is abelian with conductor $\sff = \sff^\tau \neq (1)$ and $\psi_2$ is a\pageoriginale of $G$ induced by a character of $H$, then there is an integer $\varepsilon$ in $K$ such that 
$$
L'(0, \psi_2, N (\sff)) =\frac{-1}{2W} \sum\limits_{g \in G} \psi_2 (g) \log (|\varepsilon^g|^2).
$$
This tempts us to try and relate every $L (s, \psi)$ to $\sum \psi (g) \log (|\varepsilon^g|^2)$.
To see the difficulties that we face, let us momentarily return to the dihedral group example of the previous section. We recall that each time we considered $L(s, \psi_2)$ from a new perspective, we came up with a new number in $K$ related to $L' (0, \psi_2)$. From  the point of view of characters of $G$, it is not at all clear why so many different numbers of $K$ should arise or which number we should use. However, for illustrative purposes, let us take the real unit $\varepsilon$ in $K$ from the last section which satisfied,
\begin{gather*}
L' (0, \psi_2, 17) = \frac{-1}{4} \sum\limits_{g \in G} \psi_2 (g) \log (|\varepsilon^g|^2).\\
\text{Further, for } \psi = \psi_1, \psi_{-19} \text{ or } \psi_{-323},\\
\sum\limits_{g \in G} \psi (g) \log (|\varepsilon^g|^2) = 0.
\end{gather*}
For $\psi = \psi_1$, this is because $\varepsilon$ is a unit while for $\psi = \psi_{-19}$ or $\psi_{-323}$, it is because $\psi (g\tau) = - \psi (g)$ for all $g$ in $G$ which allows a pairing of terms. For $\psi = \psi_{17}$, the situation is even more intriguing since 
$$
L' (0, \psi_{17}, 17) = L' (0, \psi_{17}) = \log (\varepsilon_{17})
$$
and so we expect some relation between 
$$
L' (0, \psi_{17})\quad\text{and}\quad \sum \psi_{17} (g) \log (|\varepsilon^g|^2).
$$

We found earlier that 
\begin{align*}
\sum \psi_{17} (g) \log (|\varepsilon^g|^2) & =  \sum [\psi_{17}  (g) + \psi_{-323} (g) ]   \log (|\varepsilon^g|^2)\\
& = -4 L' (0, \chi^{(1)} \chi^{(2)})\\
& = -16 \log (\varepsilon_{17}) = -16 L' (0,\psi_{17}).
\end{align*}
The factor of 16 is rather hard to guess beforehand. Worse still, there are primes $p$ which don't split in $k_{-19}$ with $\psi_{17} (p) = -1$. For there primes, $L(s, \chi^{(1)} \chi^{(2)})$ has a $p$-factor $(1-p^{-2s})^{-1}$ and so $L(s, \psi_{17} + \psi_{-323}, 17p)$ has a second order zero at $s =0$. This means that we come up with a unit such that 
$$
\sum\limits_{g \in G} \psi_{17} (g) \log (|\text{unit}^g|^2) = \sum [\psi_{17} (g) + \psi_{323} (g)]   \log (|\text{unit}^g|^2) = 0,
$$
even though
$$
L' (0, \psi_{17}, 17p) = 2 L' (0,\psi_{17}) \neq 0.
$$

Thus it appears that is we wish a common factor such as $-1 /(2W)$ in front, we must give up looking simultaneously at all characters $\psi$ of $G$ such that $L(s, \psi)$ has a first order zero at $s =0$. For second degree characters, we may still ask if this is possible. Precisely, we ask the following.

\begin{question*}
Suppose that $K$ is a complex normal extension of $\bbQ$ with Galois group $G$ containing $W$ roots of unity. Suppose that $f$ is divisible by the conductor of every irreducible second degree character $\psi$ of $G$ with $\psi(\tau) =0$ where $\tau$ in $G$ represents complex conjugation. Is there an integer $\pi$ in $K$ such that 
\begin{itemize}
\item[i).] $\pi^g$ is an associate of $\pi$ for all $g$ in $G$ and some power of $\pi$ is real. 

\item[ii).] $\pi^g/ \pi^p$ is a $W^{\text{th}}$ power in $K$ where $p$ is a prime not dividing $Wf$ times the discriminant of $K$ and whose associated Frobenius automorphisms are conjugate to $g$ in $G$.

\item[iii).] For every irreducible second degree character $\psi$ of $G$ with $\psi (\tau) = 0$, 
$$
L' (0, \psi, \sff) = \frac{-1}{2 W} \sum\limits_{g \in G} \psi (g) \log (|\pi^g|^2).
$$
\end{itemize}
\end{question*}

This question is probably most safely asked when at least one of the
characters $\psi$ under consideration is not a character of any quotient group of $G$. The extra difficulties that arise otherwise can be illustrated by taking $K$ to be the $36^{\text{th}}$ degree field generated by the Hilbert class fields  of $\bQ (\sqrt{-23})$ and $\bQ (\sqrt{-31})$. Also, a study of inertial groups should enable us to replace $\sff$ by a smaller number in many cases.

Suppose we have a set of $n$ irreducible characters $\psi$ satisfying the hypotheses of our Question such that if $\psi$ is in the set of $n$ characters, so is every algebraic conjugate of $\psi$. Then we can expect to isolate $n$ pieces of information\pageoriginale about units from the numerical values of the $L' (0, \psi \sff)$. We do this by imitating the orthogonality relations for $G$. Consider the $n$-dimensional $\bbZ$ lattice in $\bbC^n$ generated by column vectors of the form $v_g = (\psi (g))$ where $\psi$ runs through the $n$ characters under consideration in some fixed order. The dual lattice consists of those $n$-dimensional vectors $u$ such that $< u, v_g>$ is in $\bbZ$ for all $g$ in $G$. Without the hypothesis on algebraic conjugates of $\psi$ being present, we needn't have a lattice and then there may not be any non-zero $u$ such that $< u, v_g>$ is in $\bbZ$ for all $g$. We now have 
\begin{align*}
< u, L' (0, \psi, \sff) > & = \frac{-1}{2W} \sum\limits_{g \in G} <u, v_g> \log \left(\left|\pi^g \right|^2 \right)\\
& = \frac{-1}{2W} \log \left(\left|\varepsilon_u \right|^2 \right).
\end{align*}
Here
$$
\varepsilon_u = \prod\limits_{g \in G} (\pi^g)^{<u, v_g>}
$$
is a unit since the $\pi^g$ are associates and $\sum\limits_g <u, v_g> =0$ by the orthogonality relations. In fact, since $\pi^\tau = \zeta \pi$ for a root of unity $\zeta$,
$$
\varepsilon^\tau_u = \prod\limits_g (\zeta^g)^{<u, v_g>} \prod\limits_{g} (\pi^g)^{<u, v_g>},
$$
and so up to a root of unity $\varepsilon_u$ is real. It seems likely that $\pi$ can be chosen so that this root of unity is one (for example, if $\pi$ itself is real) and $\varepsilon_u$ is positive. We would then expect that 
\begin{equation}
<u, L' (0, \psi, \sff)> = \frac{-1}{W} \log (\varepsilon_u), \label{art9-eq1}
\end{equation}
where $\varepsilon_u$ is a positive real unit in $K$.

\textit{Further, $\varepsilon_u$ is already a $W^{\text{th}}$ power in $K$}. To see this, let $M$ be the field of $W^{\text{th}}$ roots of unity and $H = G(K/M)$. If $\chi_1$ is the trivial character of $H$, then by the definition of $M$, the induced character $\chi^\ast_1$ is the sum of all the one dimensional characters of $G$. It follows from the Frobenius reciprocity law that for any of our $n$ characters $\psi$, the restriction of $\psi$ to $H$ does not contain $\chi_1$. If $\rho$ is a representation of $G$ with character $\psi$, then for any $g$ in $G$,
$$
\sum\limits_{h \in H} \rho (gh) =\rho (g) \sum\limits_{h \in H} \rho (h) = 0,
$$
and hence 
$$
\sum\limits_{h \in H} \psi (gh ) = 0.
$$
therefore 
$$
\sum\limits_{h \in H} v_{gh} = 0.
$$

For each $g$ in $G$, let $p_g$ be chosen according to part ii) of our Question so that $\pi^g/\pi^{p_g}$ is a $W^{\text{th}}$ power in $K$. For any $h$ in $H$, $p_{gh} \equiv p_g (\mod W)$ and hence 
$$
\sum\limits_{h \in H} p_{gh} < u, v_{gh}> \equiv p_g < u, \sum\limits_{h \in H} v_{gh} > \equiv 0 (\mod W)
$$
Therefore, 
$$
\varepsilon_u = \prod\limits_{g \in G} \left(\frac{\pi^g}{\pi^{p_g}} \right)^{<u, v_g>} \prod\limits_{g \in G} {\pi^{p_g}}^{<u, v_g>}
$$
is a $W^{\text{th}}$ power in $K$ as claimed.

I have shown numerically in several instances that the Question has an affirmative answer in cases where $K$ is a class field of a real quadratic field \cite[III, IV]{art9-6}. Just as this Colloquium was taking place, Ted Chinburg \cite{art9-1} formulated the Conjecture on Artin $L$-series with first order zeros at $s =0$ in terms of \eqref{art9-eq1} and  investigated \eqref{art9-eq1} in the case that $K$ is the $48^{\text{th}}$ degree field corresponding to the non-abelian modular form of conductor 133 found by Tate. He found a unit $\varepsilon_u$ in $K$ which is a $W^{\text{th}}$ power and which satisfies \eqref{art9-eq1} to 13 decimal places. In fact he found $\varepsilon_u$ by using the numerical values  of the $L' (0, \psi)$ in a manner similar to \cite[II]{art9-6} but with a nice improvement in the method that avoids the small searches that I had to make.

\begin{thebibliography}{99}
\bibitem[1]{art9-1} \textsc{Chinburg Ted}, Stark's\pageoriginale Conjecture for a Tetrahedral Representation, to appear.

\bibitem[2]{art9-2} \textsc{Coates J.} and \textsc{A. Wiles,} On the Conjecture of Birch and Swinnerton-Dyer, \textit{Inv. Math.} 39 (1977), 223-251.

\bibitem[3]{art9-3} \textsc{Ramachandra K.,} Some aplications of Kronecker's limits formulas, \textit{Ann. of Math.} 80 (1964), 104-148.

\bibitem[4]{art9-4} \textsc{Siegel C. L.,} \textit{Lectures on Advanced Analytic Number Theory,} Tata Institute of Fundamental Research, Bombay, 1961.

\bibitem[5]{art9-5} \textsc{Stark H. M.,} \textit{Class-number problems in quadratic fields,} in Proceedings of the 1970 International congress, Vol. 1, 511-518.

\bibitem[6]{art9-6} -------, $L$-functions at $s =1$, I. II, III, IV, \textit{Advances in Math.} 7 (1971), 301-343; 17  (1975), 60-92; 22 (1976), 64-84;
\end{thebibliography}
