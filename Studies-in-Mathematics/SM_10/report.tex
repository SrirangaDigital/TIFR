\thispagestyle{empty}

\begin{center}
{\Large\sf INTERNATIONAL COLLOQUIUM}

\vskip .3cm
{\Large\sf ON AUTOMORPHIC FORMS}

\vskip .3cm
{\Large\sf REPRESENTATION THEORY}

\vskip .3cm
{\Large\sf AND ARITHMETIC}

\vskip .7cm

BOMBAY, 8--15 January 1979

\vskip .7cm

{\Large\bf R E P O R T}
\end{center}

\bigskip

\noindent
\textsc{An International Colloquium} on Automorphic forms, Representation theory and Arithmetic was held at the Tata Institute of Fundamental Research, Bombay, from 8 to 15 January 1979. The purpose of the Colloquium was to discuss recent achievements in the theory of automorphic forms of one and several variables, representation theory with special reference to the interplay between these and number theory, e.g. arithmetic automorphic forms, Hecke theory, Representation of $\GL_{2}$ and $\GL_{n}$ in general, class fields, $L$-functions, $p$-adic automorphic forms and $p$-adic $L$-functions.


The Colloquium was jointly sponsored by the International Mathematical Union and the Tata Institute of Fundamental Research, and was financially supported by them and the Sir Dorabji Tata Trust.

An Organizing Committee consisting of Professors P. Deligne, M. Kneser, M.S. Narasimhan, S. Raghavan, M.S. Raghunathan and C.S. Seshadri was in charge of the scientific programme. Professors P. Deligne and M. Kneser acted as representatives of the International Mathematical Union on the Organising Committee.

The following mathematicians gave invited addresses at the Colloquium: W. Casselman, P. Deligne, S. Gelbart, G. Harder, K. Iwasawa, H. Jacquet, N.M. Katz, I. Piatetski-Shapiro, S. Raghavan, T. Shintani, H.M. Stark and D. Zagier.

Professor R. Howe was unable to attend the Colloquium but has sent a paper for publication in the Proceedings.

Professors A. Borel and M. Kneser who accepted our invitation, were unable to attend the Colloquium.

The invited lectures were of fifty minutes' duration. These were followed by discussions. In addition to the programme of invited addresses, there were expository and survey lectures by some invited speakers giving more details of their work. Besides the mathematicians at the Tata Institute, there were also mathematicians from other universities in India who were invitees to the Colloquium.

The social programme during the Colloquium included a Tea Party on 8 January; a programme of Western music on 9 January; a programme of Instrumental music on 10 January; a dinner at the Institute to meet the members of the School of Mathematics on 11 January; a performance of classical Indian Dances (Bharata Natyam) on 12 January; a visit to Elephanta on 13 January; a programme of Vocal music on 13 January and a dinner at the Institute on 14 January.

\markboth{Report}{Report}




