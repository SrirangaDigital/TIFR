\title{ON SOME THEOREMS STATED BY RAMANUJAN}
\markright{On Some Theorems Stated by Ramanujan}

\author{By~ K. G. Ramanathan}
\markboth{K. G. Ramanathan}{On Some Theorems Stated by Ramanujan}

\date{}
\maketitle

\setcounter{page}{176} 
\setcounter{pageoriginal}{150}
\section{}\label{art09-sec1}
Ramanujan\pageoriginale seems to have been fascinated by the continued fractions 
\begin{equation}
R(\tau)=\frac{e^{2\pi i\tau/5}}{1+}\frac{e^{2\pi i\tau}}{1+}\frac{e^{4\pi i\tau}}{1+}\label{art09-eq1}
\end{equation}
and
\begin{equation}
S(\tau)=\dfrac{e^{\pi i\tau/5}}{1-}\dfrac{e^{\pi i\tau}}{1+}\dfrac{e^{2\pi i\tau}}{1-}\label{art09-eq2}
\end{equation}
where $\tau=x+iy$, $y>0$, $i=\sqrt{-1}$. We discusses them in many places in the Notebooks and more importantly in the `Lost' Notebook. In particular, he evaluated $R(\tau)$ and $S(\tau)$ for $\tau=i\sqrt{n}$ for many rational values of $n>0$. Some of these evaluations were sent by him to Hardy in his early letters from India. A number of evaluations of $R(\tau)$ and $S(\tau)$ contained in the `Lost' Notebook were discussed and upheld by us \cite{art09-key4} using the Kronecker limit formula which seems to be well adapted for these problems. We do not, of course, know Ramanujan's methods. They could not be the method using the limit formula. There are two evaluations \cite[p. 46]{art09-key7} which are particularly intriguing. The are 
\begin{align*}
S(i\sqrt{3}) &= \frac{(-3+\sqrt{5})+\sqrt{6(5+\sqrt{5})}}{4}\tag*{(3)$_{\text{R}}$}\label{art08-eq3R}\\[3pt]
S(i/\sqrt{3}) &=\frac{-3+\sqrt{5}+\sqrt{6(5-\sqrt{5})}}{4}\tag*{(4)$_{\text{R}}$}\label{art08-eq4R}
\end{align*}
As far as we know, these two results have not been proved until now. In attempting to prove these, we encountered another of Ramanujan's evaluations. If $\lambda_{n}$ for integers $n\geq 1$ is defined by
$$
\lambda_{n}=\dfrac{e^{(\pi/2)\sqrt{n/3}}}{3\sqrt{3}}\{(1+e^{-\pi\sqrt{n/3}})(1-e^{-2\pi\sqrt{n/3}})(1-e^{-4\pi\sqrt{n/3}})\ldots\}^{6}
$$
the Ramanujan states
$$
\lambda_{1}=1, \lambda_{9}=3, \lambda_{17}=4+\sqrt{17}, \lambda_{25}=(2+\sqrt{5})^{2},
$$
\begin{equation*}
\begin{aligned}
\lambda_{33} &= 18+3\sqrt{33}, \lambda_{41}=32+5\sqrt{41},\\
\lambda_{49} &= 55+12\sqrt{21}, \lambda_{89}=500+53\sqrt{89},\ldots
\end{aligned}\tag*{(5)$_{\text{R}}$}
\end{equation*}\pageoriginale
The function $\lambda_{n}$ seems to have been introduced earlier in the Notebooks (Vol 2, p. 393) where Ramanujan gives a formula for evaluating $\lambda_{n}$ for $\frac{n}{3}=11,19,43,67,163$ and others. It is to be noticed that these values of $-\frac{n}{3}$ are precisely the discriminates $\equiv 5(\mod 8)$ of imaginary quadratic fields of class number one. If we use Dedekind's modular form
$$
\eta(\tau)=e^{\pi i\tau/12}\prod\limits^{\infty}_{n=1}(1-e^{2n\pi i\tau}).
$$
then
\setcounter{equation}{5}
\begin{equation}
\lambda_{n}=\frac{1}{3\sqrt{3}}\left\{\frac{\eta[\frac{1}{2}(1+i\sqrt{n/3})]}{\eta[\frac{1}{2}(1+i\sqrt{3n})]}\right\}^{6}\label{art09-eq6}
\end{equation}
As was shown by us in \cite{art09-key5}, if $-3n$ is a fundamental discriminant of an imaginary quadratic field $K=Q(\sqrt{-3n})$ which has only one class in each genus of ideal classes, then $\lambda_{n}$ can be evaluated fairly easily using $L$-series. For example, for $n=17,41,89$ this property is true. However, for $n=25$, 49 the numbers $-3.25$ and $-3.49$ are not fundamental discriminant of an imaginary quadratic field $K=Q(\sqrt{-3n})$ which has only one class in each genus of ideal classes, then $\lambda_{n}$ can be evaluated fairly easily using $L$-series. For example, for $n=17,41,89$ this property is true. However, for $n=25$, 49 the numbers $-3.25$ and $-3.49$ are not fundamental discriminants but nevertheless they are discriminants in the orders in $Q(\sqrt{-3})$ with conductors 5 and 7, with similar properties with regard to genera of ring ideal classes. One has then an analogue of the Kronecker limit formula for the $L$-series of such ideal classes which leads to the evaluation of $\lambda_{25}$ and $\lambda_{49}$ and consequently to the proof of (3) and (4). For $n=9$ and $33$, the subrings of $Q(\sqrt{-3})$ and $Q(\sqrt{-11})$ with conductors $3$ have similar properties but the evaluation of $\lambda_{9}$ and $\lambda_{33}$ depends on different ideas.

Ramanujan, in every case, seems to consider only dicriminants, fundamental or not, which have only two classes. We shall do the same in this note and restrict further to odd discriminants with class number 2 since we are dealing only with $S(\tau)$.

\section{}\label{art09-sec2}
Let $K=Q(\sqrt{d})$, $d<0$ be an imaginary quadratic field with discriminant $d$ and class number $h(d)$. Let $R$ be the maximal order in $K$ and for any rational integer $f\geq 1$, $R_{f}$ the ring with conductor $f$. Clearly $R=R_{1}$. The ring $R_{f}$ has discriminant $df^{2}$ and a minimal basis $(1,\theta)$ where 
\begin{equation}
\theta = 
\begin{cases}
\frac{-1+i\sqrt{Df^{2}}}{2} &, \text{~ if } df^{2}\equiv 1(\mod 4)\\[4pt]
\frac{i\sqrt{Df^{2}}}{2}    &, \text{~ if } df^{2}\equiv 0(\mod 4)
\end{cases}\label{art09-eq7}
\end{equation}\pageoriginale
where $D=|d|$.

We consider in $R_{f}$ only ideals which are prime to $f$. As is well known, there is a $(1,1)$ correspondence between ideals in $R$ prime to $f$ and those in $R_{f}$ prime to $f$. If $a$ and $b$ are two ideals prime to $f$, in $R_{f}$, then they are said to be in the same ideal class in $R_{f}$ if there exist $\lambda$ and $\mu$ in $R_{f}$ both prime to $f$ such that
$$
\lambda_{a}=\mu b
$$
This leads to a class division of ideals of $R_{f}$ into ideal classes. The number $h(df^{2})$ of these ideal classes is given by
\begin{equation}
h(df^{2})=\frac{h(d)\cdot \varphi([f])}{e\cdot \varphi(f)}\label{art09-eq8}
\end{equation}
where $\varphi([f])$ denotes the Euler function of the ideal $[f]$ in $R$ so that
\begin{equation}
\varphi([f])=f^{2}\prod\limits_{\varkappa/f}(1-\frac{1}{N_{\varkappa}})
\end{equation}
where $\varkappa$ runs through all prime ideals in $R$ dividing $f$ and $\varphi(f)$ in the denominator is the ordinary totient function. The number $e$ is the index of the group of units in $R_{f}$ in the group of units of $R$.

It is to be noted that formula \eqref{art09-eq8} is still true if $d>0$.

Let $C$ be any ideal class in $R_{f}$. The {\em zeta function $\zeta^{*}_{f}(s,C)$ of the class $C$} is defined by
\begin{equation}
\zeta^{*}_{f}(S,C)=\sum\limits_{\substack{a\in C\\ (a,f)=1}}(Na)^{-s}, \ \rRe s>1\label{art09-eq10}
\end{equation}
where $a$ runs through all integral ideals in $C$ which are prime to $f$. If $\ell$ is an ideal in the class $C^{-1}$ which is prime to $f$, then
\begin{equation}
\zeta^{*}_{f}(s,C)=\dfrac{(N\ell)^{s}}{w}\sum\limits_{\substack{0\neq \alpha\in \ell\\ (\alpha,f)=1}}|N\alpha|^{-s}\label{art09-eq11}
\end{equation}
$w$ being the number of roots of unity in $R_{f}$. If $f>1$, then $w=2$.

If $f>1$, because of the restrictive summation in \eqref{art09-eq11}, it is not possible to apply at once the Kronecker Limit formula to $\zeta^{*}_{f}(s,C)$. We shall see that for our purposes, the {\em zeta function of the class $C$ in the extended sense} defined below would be sufficient. Put
\begin{equation}
\zeta_{f}(s,C)=\frac{(N\ell)^{s}}{w}\sum\limits_{\substack{\alpha\neq 0\\ \alpha\in \ell}}|N\alpha|^{-s}\label{art09-eq12}
\end{equation}
with\pageoriginale $\alpha$ running through all elements of $\ell$ not equal to zero. The sum in \eqref{art09-eq12} is then an Epstein zeta function.

We shall choose the ideal class $C$ in a particular way using the $(1,1)$ correspondence between ring ideal classes and binary, positive, primitive integral quadratic forms with discriminant $df^{2}$, $d$ being, of course, a negative fundamental discriminant.

Let $p$ be a prime number dividing $d$ but not $f$ (we assume that such primes exist). We shall construct a binary, primitive, positive form which represents $p$ primitively. Let
$$
px^{2}+bxy+cy^{2}
$$
be the quadratic form with discriminant $df^{2}$ so that
\begin{equation}
b^{2}-4pc=df^{2}.\label{art09-eq13}
\end{equation}
Clearly $p|b$ and so if $b=pb_{1}$, $d=pd_{1}$, then
$$
pb^{2}_{1}-4c=d_{1}f^{2}.
$$
Let $p$ be odd. If $df^{2}$ is odd, then
$$
p-d_{1}f^{2}\equiv 0(\mod 4)
$$
and so we choose $b_{1}=1$ and 
$$
c=(p-d_{1}f^{2})/4.
$$
The quadratic form
$$
px^{2}+pxy+\frac{p-d_{1}f^{2}}{4}y^{2}
$$
is primitive since $p\nmid d_{1}$ and is odd. It has discriminant $df^{2}$. We choose the ideals class $C$ to be the inverse of the ideal class $C^{-1}$ represented by the ideal $\ell$ with basis $(1,z)$ where
\begin{equation}
z=\dfrac{-1+\sqrt{df^{2}/p}}{2}\label{art09-eq14}
\end{equation}
The ideal clearly has norm equal to $1/p$. Note that $(p,f)=1$.

If $p$ is odd and $df^{2}$ is even, then, by \eqref{art09-eq13}, $2p | b_{1}$. One easily sees that we can again take the quadratic form to be
$$
px^{2}+2pxy+(p-d_{1}f^{2}/4)y^{2}
$$
which means that the ideal class $C^{-1}$ is represented by the ideal $(1,z)$ with 
\begin{equation}
z=\sqrt{df^{2}/2p}\label{art09-eq15}
\end{equation}
In a similar way, we obtain for $C^{-1}$ the ideal class represented by the ideal $(1,z)$ with
\begin{equation}
z=
\begin{cases}
\frac{1}{2}(1+\sqrt{df^{2}}/2) & ,~ p=2, (d/4)\text{~ odd}\\[4pt]
\sqrt{df^{2}}/4 & ,~ p=2, (d/4)\text{~ even}
\end{cases}\label{art09-eq16}
\end{equation}
If\pageoriginale we go back to formula \eqref{art09-eq12} and take $C=C_{0}$ as the principal class and apply the Kronecker limit formula (\cite[formula 6]{art09-key5}), we have
\begin{equation}
\begin{array}{c}
-\lim\limits_{s\to 1} [\zeta_{f}(s,C_{0})-\zeta_{f}(s,C)=\\[3pt]
=(4\pi/w\sqrt{Df^{2}})\log (N[1,z])^{1/2}|\eta(z)/\eta(\theta)|^{2})
\end{array}\label{art09-eq17}
\end{equation}
where $w=w_{f}$ is the number of roots of unity in $R_{f}$, $z$ is given by \eqref{art09-eq14}, \eqref{art09-eq15} and \eqref{art09-eq16} and $\theta$ by \eqref{art09-eq7}. The two functions $\zeta_{f}(s,C_{0})$ and $\zeta_{f}(s,C)$ are the zeta functions of the classes $C_{0}$ and $C$ respectively in the extended sense.

\section{}\label{art09-sec3}
In order to proceed further, it is necessary to obtain another expression for the left side of \eqref{art09-eq17}.

Let $\chi$ be any character of the ring ideal class group of $R_{f}$. We define the $L$-function
\begin{equation}
L_{f}(s,\chi)=\sum\limits_{\substack{a\in R_{f}\\ (a,f_{1})=1}}\frac{\chi(a)}{(Na)^{s}}, \ \rRe s>1.\label{art09-eq18}
\end{equation}
since $\chi$ is a multiplicative function on the ideal of $R_{f}$ prime to $f$.
\begin{equation}
L_{f}(s,\chi)=\prod\limits_{\varkappa | f}(1-\chi(\varkappa)N\varkappa^{-s})^{-1}\label{art09-eq19}
\end{equation}
Furthermore
\begin{equation}
L_{f}(s,\chi)=\sum\limits_{C}\chi(C)\zeta^{*}_{f}(s,C)\label{art09-eq20}
\end{equation}
where $C$ runs through all ring ideal classes of $R_{f}$.

If $\chi$ is a non-principal character, it is shown by Meyer that we have even the relation
\begin{equation}
L_{f}(s,\chi)=\sum\limits_{C}\chi(C)\zeta_{f}(s,C)\label{art09-eq21}
\end{equation}
in terms of the zeta functions of classes in the extended sense. Formula \eqref{art09-eq21} has the advantage that one can apply the Kronecker limit formula. If now we assume that every genus of ring ideal classes of $R_{f}$ has only one class in it, then one has
\begin{equation}
-2^{r-2}\left[\zeta_{f}(s,C_{0})-\zeta_{f}(s,C)=\sum\limits_{\chi(c)=-1}L(s,\chi)\right]\label{art09-eq22}
\end{equation}
where the sum runs through all characters which take the value $-1$ on $C$, $2^{r-1}$ being the number of genera.

We now define the genus characters.

Let $df^{2}$ have the decomposition
$$
df^{2}=d_{0}d^{*}_{0}
$$
where\pageoriginale $d_{0}$ is a fundamental discriminant and $d^{*}_{0}$ a discriminant. For such a decomposition, we have a character of the class group of $R_{f}$
$$
\chi_{d_{0}}(\varkappa)=\left(\dfrac{d_{0}}{N\varkappa}\right)
$$
for all prime ideals $\varkappa$ which do not divide $df^{2}$; $\left(\dfrac{d_{0}}{}\right)$ being the Kronecker symbol (\cite[p. 380 et seq]{art09-key9}). For prime ideals not dividing $d_{0}$, \eqref{art09-eq23} also makes sense. If $\varkappa$ divides $d_{0}$, then take
$$
\chi_{d_{0}}(\varkappa)=\left(\dfrac{d^{*}_{0}}{N\varkappa}\right)
$$
It is to be noted that $d_{0}$ and $d^{*}_{0}$ have only divisors of $f$ as common divisor.

We shall now confine ourselves to the case
\begin{equation}
df^{2}\text{~ odd}, \ h(df^{2})=2, (d,f)=1.\label{art09-eq24}
\end{equation}
Since $d$ is a fundamental discriminant,
\begin{equation}
df^{2}=-pf^{2}, \ p\equiv -1(\mod 4)\label{art09-eq25}
\end{equation}
Further $df^{2}$ has only one non-trivial decomposition
\begin{equation}
df^{2}=
\begin{cases}
-pf\cdot f & ,\text{~if~} f\equiv 1(\mod 4)\\[3pt]
-f\cdot pf & ,\text{~if~} f\equiv -1(\mod 4)
\end{cases}\label{art09-eq26}
\end{equation}
Following Siegel \cite{art09-key8}, we see that there is only one $L$-series and
$$
L_{f}(s,\chi)=
\begin{cases}
L_{-pf}(s)\cdot L_{f}(s),\text{~if~}f\equiv 1(\mod 4)\\[3pt]
L_{-f}(s)\cdot L_{pf}(s), \text{~if~} f\equiv -1(\mod 4)
\end{cases}
$$
where $L_{*}(s)$ is the ordinary Dirichlet $L$ series.

From \eqref{art09-eq22}, we get
$$
-\lim\limits_{s\to 1}(\zeta_{f}(s,C_{0})-\zeta_{f}(s,C))=L_{f}(1,\chi)
$$
and therefore, from \eqref{art09-eq17} using the fact that $w=2$ for $f>1$, we get 
$$
\frac{1}{\sqrt{p}}\left\{\frac{\eta\left(\frac{-1+i\sqrt{f^{2}/p}}{2}\right)}{\eta\left(\frac{-1+i\sqrt{f^{2}p}}{2}\right)}\right\}^{2}=
\begin{cases}
(\epsilon(f)^{h(-pf)\cdot h(f)}, \ f\equiv 1(\mod 4)\\[3pt]
(\epsilon(pf)^{h(pf)\cdot h(-f)\cdot 2/w_{0}}, \ f\equiv -1(\mod 4)
\end{cases}
$$
where $h(-pf),\ldots$ are class numbers, $\epsilon(f)$ and $\epsilon(pf)$ are the fundamental units in the real quadratic fields $Q(\sqrt{f})$ and $Q(\sqrt{pf})$ respectively and $w_{0}$ the number of roots of unity in $Q(\sqrt{-f})$.

From the definition of $\lambda_{n}$ and formula \eqref{art09-eq27}, we see that we can evaluate $\lambda_{n}$ if $p=3$ and the conditions \eqref{art09-eq24} are satisfied.

They are indeed satisfied in cases
$$
df^{2}=-3\cdot 5^{2}, - 3\cdot 7^{2}
$$
as\pageoriginale seen from the tables in \cite{art09-key1}. In case $p=3$, $f=5$, we have
$$
\epsilon(f)=(\sqrt{5}+1)/2\text{~~ and~~ } h(-15)=2.
$$
We therefore have
\begin{equation*}
\lambda_{25} =\frac{1}{3\sqrt{3}},\left\{\frac{\eta\left(\frac{-1+i\sqrt{25/3}}{2}\right)}{\eta\left(\frac{-1+i\sqrt{75}}{2}\right)}\right\}^{6}=\left(\frac{\sqrt{5}+1}{2}\right)^{6}=(2+\sqrt{5})^{2}\tag*{(5)$_{R}$}\label{art09-eq5R}
\end{equation*}
In a similar way, if $p=3$, $f=7$, $h(21)=1$, $\epsilon(21)=(5+\sqrt{21})/2$. This gives since $w_{0}=2$,
\begin{equation*}
\lambda_{49}=\left(\frac{5+\sqrt{21}}{2}\right)^{3}=55+12\sqrt{21}.\tag*{(5)$_{R}$}\label{art09-addeq5R}
\end{equation*}
The value of $\lambda_{25}$ enables us to prove Ramanujan's statements \eqref{art09-eq3} and \eqref{art09-eq4}. It is known, by taking $\tau=i\sqrt{3}$ in (\cite[p. 700]{art09-key4}) that
\begin{equation}
[S(i\sqrt{3})]^{-1}+1-S(i\sqrt{3})=\frac{\eta(i\sqrt{3}/5)}{\eta(i5\sqrt{3})}\cdot \frac{f(i\sqrt{3}/5)}{f(i5\sqrt{3})}\label{art09-eq28}
\end{equation}
where $f(\tau)$ is Schlefli's modular function
\begin{equation}
f(\tau)=e^{-\pi i/24}\cdot \dfrac{\eta((1+\eta)/2)}{\eta(\eta)}=f(-1/\tau)\label{art09-eq29}
\end{equation}
If we use the formula
\begin{equation}
\eta(-1/\tau)=(-i\tau)^{1/2}\eta(\tau)\label{art09-eq30}
\end{equation}
Then
$$
\frac{\eta(i\sqrt{3}/5)f(i\sqrt{3}/5)}{\eta(i5\sqrt{3})f(i5\sqrt{3})}=\left(\dfrac{5}{\sqrt{3}}\right)^{1/2}\frac{\eta[(1+i\sqrt{25/3})/2]}{\eta(1+i\sqrt{75}/2)}
$$
so that, by definition of $\lambda_{25}$,
$$
[S(i\sqrt{3})]^{-1}+1-S(i\sqrt{3})=\sqrt{5}-\lambda^{1/6}_{25}=\sqrt{5}(\sqrt{5}+1)/2
$$
Solving the above quadratic equation for $S(i\sqrt{3})$ and using the fact that $S(i\sqrt{3})>0$, we get
\begin{equation*}
S(i\sqrt{3})=\frac{-(3+\sqrt{5})+\sqrt{6(5+\sqrt{5})}}{4}\tag*{(3)$_{R}$}\label{art09-addeq3R}
\end{equation*}
The value of $S(i/\sqrt{3})$ can be obtained by again using \eqref{art09-eq29} and \eqref{art09-eq30} or by using the formula
\begin{equation}
\left(S(\tau)+\frac{\sqrt{5}-1}{2}\right)\left(S(-1/\tau)+\frac{\sqrt{5}-1}{2}\right)=\sqrt{5}\left(\frac{\sqrt{5}-1}{2}\right)\label{art09-eq31}
\end{equation}
which\pageoriginale was stated by Ramanujan in his Notebooks. It was first proved by Watson. (See also \cite{art09-key4}).

If we use the formulae \eqref{art09-eq31} and
\begin{equation}
\begin{array}{c}
\left((S(\tau))^{5}+\left(\dfrac{\sqrt{5}-1}{2}\right)^{5}\right)+\left((S(-1/5\tau)^{5}+\left(\frac{\sqrt{5}-1}{2}\right)^{5}\right)\\
=5\sqrt{5}\left(\dfrac{\sqrt{5}-1}{2}\right)^{5}
\end{array}\label{art09-eq32}
\end{equation}
proved by us, one can obtain the values of $S[i5^{k}(\sqrt{3})^{1}]$ where $k$ is any rational integer and $1=\pm 1$.

\section{}\label{art09-sec4}
We shall prove now the other statements of Ramanujan in \eqref{art09-eq5}. 

In the first place,
$$
\lambda_{1}=\dfrac{1}{3\sqrt{3}}\left(\frac{\eta(i/\sqrt{3})}{\eta(i\sqrt{3})}\cdot \frac{f(i/\sqrt{3})}{f(i\sqrt{3})}\right)^{6}.
$$
If we now use the formulae \eqref{art09-eq29} and \eqref{art09-eq30} we get
$$
\lambda_{1}=1
$$
Consider now $\lambda_{9}$. By definition,
$$
\lambda_{9}=\frac{1}{3\sqrt{3}}\left(\frac{\eta((1+i\sqrt{3})/2)}{\eta((1+i3\sqrt{3})/2)}\right)^{6}
$$
If we use the product expansion of the $\eta$-function, then
\begin{equation}
\lambda_{9}=\dfrac{-i}{3\sqrt{3}}\left(\frac{\eta(\omega)}{\eta(3\omega)}\right)^{6}, \ \omega=\dfrac{-1+i\sqrt{3}}{2}.\label{art09-eq33}
\end{equation}
On the other hand,
$$
\alpha = 27 \left(\dfrac{\eta(3\omega)}{\eta(\omega)}\right)^{6}
$$
is a root of the equation
$$
x^{4}+18x^{2}+\lambda_{3}(\omega)\chi-27=0
$$
where
$$
\lambda_{3}(\omega)=\sqrt{j(\omega)-1728}
$$
and $j(\omega)$ is the well-known Klein's invariant (\cite[p. 504]{art09-key9}). Weber has shown that
$$
\lambda_{3}((-1+i\sqrt{3})/2)=i.24\sqrt{3}
$$
and\pageoriginale therefore $\lambda=\lambda_{9}$ is a root, positive, of
$$
x^{4}-8x^{3}+18x^{2}-27=0.
$$
This however equals
$$
(x+1)(x-3)^{3}
$$
which shows that
\begin{equation*}
\lambda_{9}=3\tag*{(5)$_{R}$}\label{art09-aaddeq5R}
\end{equation*}
Consider now $\omega=(1+i\sqrt{11})/2$. Then
$$
\lambda_{33}=\frac{-i}{3\sqrt{3}}\left(\frac{\eta(\omega)}{\eta(3\omega)}\right)^{6}=\dfrac{3\sqrt{3}i}{\alpha}
$$
where
$$
=27\cdot \left(\frac{\eta(3(-1+i\sqrt{11})/2)}{\eta((-1+i\sqrt{11})/2)}\right)^{6}
$$
From Weber \cite[p. 504]{art09-key9},
$$
\lambda_{3}\left(\frac{-1+i\sqrt{11}}{2}\right)=56i\sqrt{11}
$$
and hence $\lambda$ is the positive root of
\begin{equation}
9x^{4}-56\sqrt{33}x^{3}+18\cdot 3^{2}\cdot x^{2}-3^{5}=0\label{art09-eq34}
\end{equation}
This quartic equation can be solved by the classical methods of the theory of algebraic equations. One obtains
$$
\lambda=\lambda_{33}=3(6+\sqrt{33}).
$$

In fact, Weber (loc. cit) has given the values of $\lambda_{3}((-1+i\sqrt{n})/2)$ for $n=19,43,67$ and $163$ and thus $\lambda_{3n}$ is a root of a quadratic equation like \eqref{art09-eq34} from which $\lambda_{3n}$ can be evaluated.

\begin{thebibliography}{99}
\bibitem{art09-key1} \textsc{Z. I. Borevich} and \textsc{I. R. Shafarevich :} {\em Number Theory,} Academic Press, New York (1966). 

\bibitem{art09-key2} \textsc{R. Fricke :} {\em Die elliptische Funktionen und ihre Anwendungen,} Bd II, B.G. Teubner, Berlin (1922). 

\bibitem{art09-key3} \textsc{C. Meyer :}\pageoriginale {\em Die Berechnung der Klassenzahl abelscher K\"orper \"uber quadratischen Zahlk\"orpern,} Akademie Verlag, Berlin (1957).

\bibitem{art09-key4} \textsc{K. G. Ramanathan :} Ramanujan's continued fraction, {\em Indian Jour. Pure Appl. Math.,} 16(1985), 695-724. 

\bibitem{art09-key5} \textsc{K. G. Ramanathan :} Some applications of Kronecker's limit formula, {\em Jour. Ind. Math. Soc.,} 52(1987), 71-89. 

\bibitem{art09-key6} \textsc{S. Ramanujan :} {\em Notebooks}, Vol. 2, Tata Institute of Fundamental Research, Bombay (1957). 

\bibitem{art09-key7} \textsc{S. Ramanujan :} {\em The Lost Notebook and other unpublished papers,} Narosa Publishing House, New Delhi (1987). 

\bibitem{art09-key8} \textsc{C. L. Siegel :} {\em Analytische Zahlentheorie} II, G\"ottingen (1963). 

\bibitem{art09-key9} \textsc{H. Weber :} {\em Lehrbuch der Algebra,} Bd. III, Braunschweig (1908). 
\end{thebibliography}

\bigskip
\noindent
{\small A1 Sri Krishna Dham}

\noindent
{\small 70, L. B. S. Marg}

\noindent
{\small Mulund (West)}

\noindent
{\small Bombay 400 080}



