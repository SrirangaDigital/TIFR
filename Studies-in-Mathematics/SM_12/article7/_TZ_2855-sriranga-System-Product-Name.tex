 \label{art7-eq1.52}\\
\{-\cos (\pi s) \sH_j(1-s, d'/c) + \epsilon_j \text{ch} (\pi \chi_j) \sH_j (1-s, d/c)\}. \notag
\end{gather}\pageoriginale
\end{enumerate}

We conclude with the simple but important consequence of the multiplicative relations \eqref{art7-eq1.19} for Hecke operators : for $\re s>1 + |\re  v- \frac{1}{2}|$, we have
\begin{equation}
\sum\limits^\infty_{n=1} \frac{\tau_v (n) t_j (n)}{n^s} = \frac{1}{\zeta(2s)} \sH_j(s+v-\frac{1}{2}) \sH_j(s - v + \frac{1}{2}).
\label{art7-eq1.53}
\end{equation}

If we replace $t_j(n)$ by $\tau_\mu(n)$ (that corresponds to the continuous spectrum of the Hecke operators) then the well-known Ramanujan identity will arise instead of \eqref{art7-eq1.53}:
\begin{equation}
\sum\limits^\infty_{n=1} \frac{\tau_v(n) \tau_\mu(n)}{n^s} =  \label{art7-eq1.54}\\
=\frac{1}{\zeta(2s)} \zeta(s+v -\mu) \zeta(s+\mu-v) \zeta(s-v-\mu +1) \zeta(s+v+\mu -1). \notag
\end{equation}

For this reason, equality \eqref{art7-eq1.53} is a direct generalization of the Ramanujan identity; both will be essential for the estimate of the eighth moment of the Riemann zeta-function.

\subsection{The spectral mean of Hecke series.}\label{art7-subsec1.9}
Let  $N\geqslant 1$ be an integer and let $s$, $v$ be complex variables. We set 
\begin{align}
Z^{(d)}_N (s, v; h) &= \sum\limits_{j \geqslant 1}  \alpha_j t_j (N) \sH_j (s+v -\frac{1}{2}) \sH_j(s-v+\frac{1}{2}) h (\chi_j) \label{art7-eq1.55}\\
Z^{(d)}_N (s, v; h) & = \sum\limits_{j \geqslant 1} \epsilon_j \alpha_j t_j (N) \sH_j (s+v -\frac{1}{2}) \sH_j (s-v+\frac{1}{2}) h (\chi_j) \label{art7-eq1.56}
\end{align}
(with $\alpha_j = (\text{ch}(\pi\chi_j))^{-1} |\rho_j(1)|^2$). Here the summation is over the positive discrete spectrum of the automorphic Laplacian and one assumes that its eigenfunctions have been selected in such a manner that they are at the same time eigenfunctions of the ring of Hecke operators and of the reflection operator $T_{-1} (\epsilon_j = \pm 1$ are the eigenvalues of $T_{-1}$).

Further, we define the square mean of the Hecke series over the continuous spectrum by the equality
\begin{align}
&Z^{(c)}_N(s,v;h) = \\
&=\frac{1}{\pi} \int\limits^\infty_{-\infty} \frac{\zeta (s+v-\frac{1}{2} +ir) \zeta(s+v-\frac{1}{2}-ir) \zeta (s-v+\frac{1}{2} + ir) \zeta(s-v+\frac{1}{2} -ir)}{\zeta (1+2ir) \zeta(1-2ir)} \label{art7-eq1.57}\\
& \times \tau_{(1/2) + ir} (N) h(r) dr
\end{align}\pageoriginale
with the stipulation that, by means of integral \eqref{art7-eq1.57}, the function $Z^{(c)}_N$ is defined under the conditions
$$
\re(s+v-\frac{1}{2}) < 1, \re (s-v + \frac{1}{2}) < 1.
$$

If any one of the points $s \pm (v - \frac{1}{2})$ lies to the right hand side of the unit line, then the integral \eqref{art7-eq1.57} defines another function, connected with $Z^{(c)}_N$ by the Sokhotskii formulae. Fro example, if by $\tilde{Z}^{(c)}_N$, we denote the function which is defined by \eqref{art7-eq1.57} with $\re s >1$, $\re v = \frac{1}{2}$, then a simple computation gives
\begin{align}
\tilde{Z}_N^{(c)} (s,v;h) = Z^{(c)}_N (s, v; h) + 4 \zeta_N(s,v ) h (i(s-v-\frac{1}{2})) + \\
+ 4\zeta_N (s, 1-v) h(i(s+v-\frac{3}{2}))
\end{align}
where we have introduced the notation
$$
\xi_N (s,v) = \frac{\zeta (2s-1) \zeta(2v)}{\zeta(2-2s+2v)} \tau_{s-v} (N)
$$
and the regularity strip of $h$ is assumed to be sufficiently wide for the right hand side to make sense.

Now we need the mean with respect to the weights of the Hecke series associated with regular cusp forms. For an integer $k \geqslant 1$, we set
\begin{gather}


\bye
