\title{Ramanujan Birth Centenary International Colloquium on Number Theory and Related Topics
\break
{\fontsize{13}{15}\selectfont Bombay, 4-11 January 1988}}

\author{REPORT}
\date{}
\maketitle

\thispagestyle{empty}

An International\pageoriginale Colloquium on Number Theory and related topics was held at the Tata Institute of Fundamental Research, Bombay during 4-11 January, 1988, to mark the birth centenary of Srinivasa Ramanujan. The purpose of the Colloquium was to highlight recent developments in Number Theory and related topics, especially those related to the work of Ramanujan ``such as the Circle method, Sieve methods and Combinatorial techniques in Number theory, Partition congruences,  Rogers - Ramanujan identities, Lacunarity of power series, Hypergeometric series and Special functions, Complex multiplication, Hecke theory etc.''

\medskip

The Colloquium was organized by the Tata Institute of Fundamental Research with co-sponsorship from the International Mathematical Union. Financial support was received from the International Mathematical Union and the Sir Dorabji Tata Trust, as in former years. The organizing committee of the Colloquium consisted of Professors M.S. Narasimhan, S. Raghavan, M.S. Raghunathan, K. Ramachandra and C.S. Seshadri and Dr. S.S. Rangachari. The International Mathematical Union was represented on the committee by Professors M.S. Narasimhan and C.S. Seshadri. 

\medskip

The following mathematicians delivered one-hour addresses at the Colloquium: 

\medskip

G.E. Andrews, R. Askey, B. C. Berndt, D. M. Bressoud, D. R. Heath-Brown, N. V. Kuznetsov, K. Ramachandra, K. G. Ramanathan, S. S. Rangachari, R. A. Rankin, I. Satake, W. M. Schmidt, A. Selberg, J. P. Serre, T. N. Shorey and D. Zagier. 

\medskip

Professor H. Iwaniec could not attend the Colloquim but sent a paper for inclusion in the Proceedings. 

\medskip

Besides members of the School of Mathematics of the Tata Institute of Fundamental Research, mathematicians from universities and educational institutions in India, France, Canada, Japan and the United States of America were also invited to attend the Colloquium. 

\medskip

The social programme for the Colloquium included a tea-party on 4 January, a classical Indian dance performance (Bharatanatyam) on 6 January, a film show and a dinner-party at the Institute on 7 January, a violin recital (Hindustani music) on 8 January, an excursion to the Elephanta Caves on 9 January and a farewell dinner-party on 10 January 1988.



\lhead[]{}
\markboth{REPORT}{}

\vfill\eject
