\chapter{Impact of geometry of the boundary on the positive solutions of a semilinear Neumann problem with Critical nonlinearity}\label{chap2}

\markright{Impact of geometry of the boundary on the positive solutions of a semilinear Neumann problem with Critical nonlinearity}
\begin{center}
Adimurthi

\medskip
Dedicated to M.S. Narasimhan and C.S. Seshadri on their 60th Brithdays
\end{center}

\markboth{Adimurthi}{Impact of geometry of the boundary on the positive solutions of a semilinear Neumann problem with Critical nonlinearity}

Let\pageoriginale $n\geq3$ and $\mathbb{P}^{n}$ be a bounded domain with smooth boundary. We are concerned with the problem
of existence of a function $u$ satisfying the nonlinear equation
\begin{align*}
-\Delta u &= u^{p}-\lambda u \quad{\rm in}\quad \Omega \nonumber\\
u &> 0 \nonumber\\
\dfrac{\partial u}{\partial v} &= 0 \quad {\rm on}\quad \partial \Omega\tag{1} \label{chap2-eq1} 
\end{align*}
where $p=\frac{n+2}{n-2},\lambda > 0$. Clearly $u=\lambda^{1/(p-1)}$ is a solution \eqref{chap2-eq1} and we call it a trivial solution. The exponent $p=\frac{n+2}{n-2}$ is critical from the view point of Soblev imbedding. Indeed the solution of \eqref{chap2-eq1} corresponds to critical points of the functional
\begin{equation*}
Q_{\lambda}(u) = \dfrac{\int_{\Omega}|\nabla_{u}|^{2}dx + \lambda\int_{\Omega}u^{2}dx}{\left(\int_{\Omega}|u|^{p+1}dx\right)^{2/p+1}}\tag{2}\label{chap2-eq2}
\end{equation*}
on the manifold
\begin{equation*}
M =\left\{u\in H^{1} (\Omega) ; \int_{\Omega}|u|^{p+1}dx=1\right\}\tag{3}\label{chap2-eq3}
\end{equation*}

In fact, if $v\leq 0$ is a critical point of \eqref{chap2-eq2} on $M$, then $u-Q(v)^{1/(p-1)v}$ satisfies \eqref{chap2-eq1}. Note that $p+1 =\frac{2n}{n-2}$ is the limiting exponent for the imbedding $H^{1}(\Omega)\mapsto L^{2n/(n-2)}(\Omega)$. Since this imbedding is not compact, the manifold $M$ is not weakly closed and hence $Q_{\lambda}$ need not satisfy the Palais Smale condition at all levels. Therefore there are serious difficulties when trying to find critical points by the standard variational methods. In fact there is a sharp\pageoriginale contrast between the sub critical case $p<\frac{n+2}{n-2}$ and the critical case $p=\frac{n+2}{n-2}$.
  
Our motivation for investigation comes from a question of Brezis \cite{chap2-key11}. If we replace the Neumann condition by Dirichlet condition $u=0$ on $\partial\Omega$ in \eqref{chap2-eq1}, then the existence and non existence of solutions depends in topology and geometry of the domain (see Brezis-Nirenbreg \cite{chap2-key14}, Bhari-Coron \cite{chap2-key9}, Brezis \cite{chap2-key10}). In view of this, Brezis raised the following problem

\begin{quote}
``Under what conditions on $\lambda$ and $\Omega$, \eqref{chap2-eq1} admits a solution?"
\end{quote}

The interest in this problem not only comes from a purely mathematical question, but it has application in mathematical biology, population dynamics (see \cite{chap2-key16}) and geometry.

In order to answer the above question let us first look at the subcritical case where the compactness in assured.

\medskip
\noindent\textbf{Subcritical case }$1 < p < \frac{n+2}{n-2}$. 

\medskip
This had been studied extensively in the recent past by Ni \cite{chap2-key18}, and Lin-Ni-Takagi \cite{chap2-key16}. In \cite{chap2-key16}, Lin-Li-Takagi have proved the following

\begin{theorem}\label{chap2-thm1}
There exist two positive constants $\lambda_{*}$ and $\lambda_{*}$ such
\begin{itemize}
\item[a)] If $\lambda < \lambda_{*}$, then \eqref{chap2-eq1} admits only trivila solutions.
 \item[b)] If $\lambda > \lambda_{*}$, then \eqref{chap2-eq1} admits non constant solutions.
\end{itemize}

Further Ni \cite{chap2-key18} and Lin-Ni \cite{chap2-key15} studied the radial case for all $1< p < \infty$ and proved the following 
\end{theorem}

\begin{theorem}\label{chap2-thm2}
Let $\Omega = {x; |x| < 1 }$ is a ball. Then there exists two positive constants $\lambda_{*}$ and $\lambda^{*}$  such that 
$\lambda_{*} \leq \lambda^{*}$ such that $\lambda_{*} \leq \lambda^{*}$ and
\begin{itemize}
\item[a)]For $1< p < \infty$, $\lambda > \lambda^{*}$, \eqref{chap2-eq1} admits a radially increasing solution
\item[b)] if $p \neq \frac{n+2}{n-2}$, then for $0< \lambda <  \lambda_{*}$, (\ref{chap2-eq1}) does not admit a non constant radial solution.
\item[c)] Let $\Omega = \{x; 0 < \alpha < |x| < \beta\}$ be an annuluar domain and $ 1 < p < \infty$. Then there exist two positive constants $\lambda_{*} \leq \lambda^{*}$ such that for $\lambda_{*} \geq \lambda^{*}$, \eqref{chap2-eq1} admits a non constant radial solution and if $\lambda \leq \lambda^{*}$, then \eqref{chap2-eq1} does not admit a non constant radial solution. 
\end{itemize}
\end{theorem} 

In view of these results Lin and Ni \cite{chap2-key15} made the following

\begin{conjecture*}
Let\pageoriginale $p \geq \frac{n+2}{n-2}$ then there exist two positive constants $\lambda \leq \lambda^{*}$ such that
\begin{enumerate}[\rm (A)]
\item For $0 < \lambda < \lambda_{*}$, (\ref{chap2-eq1}) does not admit non constant solutions.\label{chap2-enum(A)}
\item For $\lambda < \lambda_{*}$, (\ref{chap2-eq1}) admits a non constant solution.\label{chap2-enum(B)}
\end{enumerate}

 In this article we analyze this conjecture in the \textbf{critical case} $p = \frac{n+2}{n-2}$. Surprisingly enougn, the critical case is totally different from the subcritical. In fact the part (\ref{chap2-enum(A)}) fo the conjecture in general is false. The following results of Adimurthi and Yadava \cite{chap2-key4} and Budd, Knaap and Peletier \cite{chap2-key12} gives a counter example to the Part (\ref{chap2-enum(A)}) of the conjecture.
\end{conjecture*}

\begin{theorem}\label{chap2-thm3}
Let $n= 4, 5, 6$ and $\Omega = \{X: |x|< 1\}$. Then there exist a $\lambda_{*} > 0$ such that for $0 < \lambda < \lambda_{*}$, \eqref{chap2-eq1} admits a radially decreasing solution.

Let us now turn our attention to part (\ref{chap2-enum(B)}) of the conjecture. Let $S$ denote the best Sobolev constant for the imbedding $H^{1}(\bbR^{n}) \mapsto  L^{2n/(n-2)}(\bbR^{n})$ given by  
\begin{equation*}
S = {\rm inf}\left\{ \int_{\bbR^{n}} |\nabla u|^{2}dx: \int_{\bbR^{n}} |u|^{2n/n-2}dx =1 \right \}\tag{4}\label{chap2-eq4}
\end{equation*}
Then $S$ is achieved and any minimizer in given by $U_{\varepsilon, x_{0}}$ for some $\varepsilon > 0$, $x_{0} \in \bbR^{n}$ where
\begin{align*}
U(x) &= \left[ \dfrac{n(n-2)}{n(n-2)+|x|^{2}}\right]^{\frac{n-2}{2}}\label{chap2-eq5}\tag{5}\\
U_{\varepsilon, x_{0}}(x) &= \dfrac{1}{\varepsilon^{\frac{n-2}{2}}}U \left(\dfrac{x-x_{0}}{\varepsilon}\right)\tag{6}\label{chap2-eq6}
\end{align*}   
\end{theorem}

In order to answer part ({\ref{chap2-enum(B)}}) of the conjecture, geometry of the boundary play an important role. To see this, we look at a more general problem than \eqref{chap2-eq1}, the mixed problem.

Let $\partial \Omega = \Gamma_{0}\cup  \Gamma_{1}, \Gamma_{0} \cap \Gamma_{1} = \phi, \Gamma_{i}$ are submanifolds of dimension $(n-1)$. The problem is to find function $u$ satisfying
 \begin{align*}
-\Delta u + \lambda u &= u ^{\frac{n+2}{n-2}} \quad {\rm in}\quad \Omega\nonumber\\
u &> 0 \nonumber\\
u &= 0 \quad {\rm on}\quad \Gamma_{0}\tag{7}\label{chap2-eq7}\\
\dfrac{\partial u}{\partial v} &= 0 \quad {\rm on}  \quad\Gamma_{1} \nonumber     
 \end{align*}
Let
\begin{align*}
H^{1}(\Gamma_{0}) &= \left\{u \in H^{1}(\Omega) : u = 0\;\; {\rm on}\;\; \Gamma_{0}\right\}\tag{8}\label{chap2-eq8}\\[4pt]
S(\lambda, \Gamma_{0}) &= {\rm inf}\left\{Q_{\lambda(u) \;; \;u \;\in\; H^{1}\; (\Gamma_{0})\; \cap\; M}\right\}\tag{9}\label{chap2-eq9}
\end{align*} 
\pageoriginale

Clearly, if $S(\lambda, \Gamma_{0})$ is achieved by some $v$, then we can take $v\leq 0$ and $u=S(\lambda, \Gamma_{0})^{\frac{n-2}{4}}v$ satisfies \eqref{chap2-eq7}. $u$ is called a \textit{minimal energy} solution. Existence of a minimal energy solution is proved in Adimurthi and Mancini \cite{chap2-key1} (See also X.J. Wang \cite{chap2-key22}) and have the following 

\begin{theorem}\label{chap2-thm4}
Assume that there exist an $x_{0}$ belonging to the interior of $\Gamma_{1}$ such that the mean curvature $H(x_{0})$ at $x_{0}$ with respect to unit outward normal is positive. Then $S(\lambda, \Gamma_{0})$ is achieved.
\end{theorem}

\begin{sketchoftheproof}
The proof consists of two steps.
\end{sketchoftheproof}
\begin{description}
\item[{\rm \bf Step 1.}] Suppose $S(\lambda, \Gamma_{0}) < S/2^{2/n}$, then $S(\lambda,m \Gamma_{0})$ is achieved.\label{chap2-enum Step1}

Let $v_{k} \in H^{1}(\Gamma_{0})\cap M$ be a minimizing sequence. Clearly $\{v_{k}\}$ is bounded in $H^{1}(\Omega)$. Let for subsequence of $\{v_{k}\}$ still denoted by $\{v_{k}\}$, coverges weakly to $v_{0}$ and almost everywhere in $\Omega$. We first claim that $v_{0}\nequiv 0$. Suppose $v_{0}\equiv 0$, then by Cherrier imbedding (See \cite{chap2-key8}) for every $\varepsilon > 0$, there exists $C(\varepsilon) > 0$ such that
\begin{align*}
& 1 = \left(\int_{\Omega}|v_{k}|^{p+1}dx\right)^{2/p+1}\\
&\qquad \leq \dfrac{2^{2/n}}{S}(1+\varepsilon) \int_{\Omega}|\Delta v_{k}|^{2}dx + C(\varepsilon)\int v_{k}^{2}dx.
\end{align*}
By Rellich's compactness, $v_{k} \rightarrow 0$ in $L^{2}(\Omega)$ and hence in the above inequality letting $k \rightarrow \infty$ and $\varepsilon \rightarrow 0$ we obtain
\begin{align*}
1 &\leq \lim_{\varepsilon \rightarrow 0}\dfrac{2^{2/n}}{S}(1+\varepsilon) \lim_{k \rightarrow \infty} Q_{\lambda}(v_{k})\\
&= \dfrac{2^{2/n}}{S}S(\lambda, \Gamma_{0})\\
&< 1
\end{align*}
which is a contradiction. Hence $v_{0}\nequiv 0$. Let $h_{k} = v_{k}-V_{0}$, then $h_{k}\rightarrow 0$ weakly in $H^{1}(\Omega)$ and strongly in $L^{2}(\Omega)$. Hence
\begin{align*}
S(\lambda, \Gamma_{0}) &= Q_{\lambda}(v_{k}) + 0(1)\\
&= Q_{\lambda}(v_{0})\left(\int_{\Omega} |v_{0}|^{p+1}\right)^{s/p+1} + \int_{\Omega} |\Delta h_{k}|^{2}dx + 0(1)
\end{align*}
Now by Brezis-Lieb Lemma, Cherrier imbedding, from the\break above inequality, and by the hypothesis, we have for sufficiently small $\varepsilon > 0$,
\begin{align*}
&S(\lambda, \Gamma_{0}) = S(\lambda, \Gamma_{0}) \left(\int_{\Omega}|v_{k}|^{p+1}dx\right)^{2/p+1}\\
&\leq S(\lambda, \Gamma_{0}) \left\{\left(\int_{\Omega}|v_{0}|^{p+1}dx\right)^{2/p+1} + \left(\int_{\Omega}|h_{k}|^{p+1}dx\right)^{2/p+1}\right\} \\
&\quad + 0(1)\\
&= S(\lambda, \Gamma_{0}) \left\{\left(\int_{\Omega}|v_{0}|^{p+1}dx\right)^{2/p+1} +\dfrac{2^{2/n}}{S}(1 +              \varepsilon)\times \right. \\
&\qquad\qquad\qquad\qquad \left.\int_{\Omega}|\nabla h_{k}|^{2}dx\right\} + 0(1)\\
&= S(\lambda, \Gamma_{0}) \left(\int_{\Omega}|v_{0}|^{p+1}dx\right)^{2/p+1} + \int_{\Omega}|\nabla h_{k}|^{2}dx + 0(1)\\
&= S(\lambda, \Gamma_{0}) \left(\int_{\Omega}|v_{0}|^{p+1}dx\right)^{2/p+1} + S(\lambda, \Gamma_{0})-Q_{\lambda}(v_{0}) \times \\
&\qquad \qquad \qquad \left(\int_{\Omega}|v_{0}|^{p+1}dx\right)^{2/p+1} 
\end{align*}\pageoriginale
this implies that $Q_{\lambda}(v_{0}) \leq S(\lambda,\Gamma_{0})$. Hence $v_{0}$ is a minimizer.

\item[{\rm \bf Step 2.}] $S(\lambda, \Gamma_{0}) < S/2^{2/n}$\label{chap2-enum Step2}

Let $x_{0}$ belong to the interior of $\Gamma_{1}$ at which $H(x_{0}) > 0 $ and $r > 0$ that $B(x_{0}, r) \cap \Gamma_{0} = \phi$. Let $\varphi \in C_{0}^{\infty}(B(x_{0}, r))$ such that $\varphi = 1$ for $|x-x_{0}| < r/2$. Let $\varepsilon > 0$ and $v_{\varepsilon}  = \varphi U_{\varepsilon, x_{0}}$. Then $v_{\varepsilon} \in H^{1}(\Gamma_{0})$ and we can find positive constants $A_{n}$ and $a_{n}$ depending only on $n$ such that      
\begin{equation}
Q_{\lambda(v_{\varepsilon})} = \dfrac{S}{2^{2/n}} -A_{n}H(x_{0})\beta_{1}(\varepsilon) + a_{n}\lambda \beta_{2}(\varepsilon) + 0(\beta_{1}(\varepsilon) + \beta_{2}(\varepsilon))\tag{10}\label{chap2-eq10}
\end{equation}
where
\begin{equation*}
\beta_{1}(\varepsilon) = 
\begin{cases}
\varepsilon \log^{1/\varepsilon} & \text{if $n =3$}\\
\varepsilon & \text{it $n\geq 4$}
\end{cases}
\end{equation*}
\begin{equation*}
\beta_{2}(\varepsilon) = 
\begin{cases}
\varepsilon & \text{it $n\geq 4$}\\
\varepsilon^{2} \log{1/\varepsilon} & \text{if $n =4$}\\
\varepsilon^{2} & \text{it $n\geq 4$}
\end{cases}
\end{equation*}
Hence for $\varepsilon$ small and since $H(x_{0}) > 0$ we obtain $Q_{\lambda}(v_{\varepsilon}) < S/2^{2/n}$ and this proves Step (\ref{chap2-enum Step2}) and hence the theorem.
\end{description}

Now it is to be noted the the curvature condition on $\Gamma_{1}$ is very essential. If the curvature conditions fails, then in general (\ref{chap2-eq7}) may not admit any solution.

\begin{example*}
Let  
\begin{align*} 
B &={x : |x| < 1}\; {\rm and}\\
\Omega &= {x \in B : x_{n} > 0}\\
\Gamma_{1} &= {x\in \partial \Omega : x_{n} = 0}\\
\Gamma_{0} &= {x\in \partial \Omega : x_{n} > 0}
\end{align*}
Let $u\in H^{1}(\Gamma_{0})$ be a solution of (\ref{chap2-eq7}). Define $w$ on $B$ by
\begin{equation*}
w(x', x_{n}) = 
\begin{cases}
w(x', x_{n}) & \text{if $x_{n} > 0$}\\
w(x', x_{n}) & \text{if $-x_{n}< 0$}
\end{cases}
\end{equation*}
Since\pageoriginale $\frac{\partial u}{\partial v} = 0$  on $ \Gamma_{1}, w$ satisfies 
\begin{align*}
-\Delta w + \lambda w &= w^{\frac{n+2}{n-2}}\; {\rm in}\; B\\
w &> 0\\
w &= 0 \;\;\text{on}\;\; \partial B.
\end{align*}
Hence by Pohozaev's identity we obtain
$$
-\lambda \int_{B} w^{2} dx = \int_{\partial B} |\nabla w|^{2} \langle x, v\rangle d\xi 
$$
Hence by a contradiction. Notice that the mean curvature is zero on $\Gamma_{1}$.
\end{example*}

\section*{Proof of Part (B) of the Conjecture}

Let $\Gamma_{1} = \partial\Omega$. Since $\partial\Omega$ is smooth, we can find an $x_{0} \in \partial\Omega$ such that $H(x_{0})> 0$. Hence from theorem (\ref{chap2-thm4}), (\ref{chap2-thm1}) admits a minimal energy solution $u_{\lambda}$. Let $u_{0} = \lambda^{1/p-1}$ and $\lambda^* = \frac{S}{(2|\omega|)^{2/n}}$. Then  $\lambda > \lambda^*$
$$
Q_{\lambda}(u_{\lambda}) < \dfrac{S}{2^{2/n}}< \lambda|\Omega|^{2/n} = Q_{\lambda}(u_{0})
$$
Hence $u_{\lambda}$ is a non constant solution of \eqref{chap2-eq1} and this proves part (\ref{chap2-enum(B)}) of the conjecture.

\section*{Properties of the minimal energy solutions}
\begin{enumerate}[{\rm \bf 1.}]
\item By Theorem \ref{chap2-thm3} part (\ref{chap2-enum(A)}) of conjecture in general is flase. Now we can ask whether this is true among minimal energy solution? In fact it is true. The following is proved in Adimurthi-Yadava \cite{chap2-key6}.\label{chap2-enum1}  
\begin{theorem}
There exist a $\lambda_{*} > 0$ such that for all $0 < \lambda < \lambda_{*}$, the minimal energy solution are constant.\label{chap2-thm5}
\end{theorem}

\begin{proof}
By using the blow up techinque \cite{chap2-key13}, we can prove that for every $\varepsilon > 0$ there exists a $a\lambda(\varepsilon) > 0$ such that for $0 < \lambda < \lambda(\varepsilon)$, if $u_{\lambda}$ is a minimal energy solution, then
\begin{equation*}
|u_{\lambda}|_{\infty} \leq \varepsilon\tag{11} \label{chap2-eq11}
\end{equation*}
where $|\cdot|_{\infty}$ denotes the $L^{\infty}$ norm. Let $\mu_{1}$ be the first non zero eigenvalue of 
\begin{align*}
-\Delta \psi &= \mu \psi \;\; {\rm in}\;\; \Omega\\
\dfrac{\partial \psi}{\partial v} &= 0 \;\; {\rm on} \;\; \partial \Omega. 
\end{align*}
Let\pageoriginale $\overline{u}_{\lambda} = \dfrac{1}{|\Omega|} \int_{\Omega}u_{\lambda}dx$ and $\varphi_{\lambda} = u_{\lambda} - \overline{u}_{\lambda}$. Then $\varphi_{\lambda}$ satisfies
$$
-\Delta \varphi_{\lambda} + \lambda\varphi_{\lambda} = \overline{u}_{\lambda}^{p} + p \int_{0}^{1}\left(\overline{u}_{\lambda} + t\varphi_{\lambda}\right)^{p-1} \varphi_{\lambda}^{2}dtdx
$$
From (\ref{chap2-eq11}) we have $0\leq \overline{u}_{\lambda} + t\varphi_{\lambda} \leq u_{\lambda} \leq \varepsilon$ and $\int_{\Omega} \varphi_{\lambda}dx = 0$. Therefore we obtain
$$
(\mu_{1}+ \lambda) \int_{\Omega}\varphi_{\lambda}^{2}dx \leq \int_{\Omega}\left(|\nabla\varphi_{\lambda}|^{2} + \lambda \varphi_{\lambda}^{2}\right)dx \leq p \varepsilon^{p-1} \int_{\Omega} \varphi_{\lambda}^{2}dx
$$
Now choose $\varepsilon^{p-1} = \frac{\mu_{1}}{2p}$ and $\lambda_{*} = \lambda(\varepsilon)$, then the above inequality implies that $\varphi_{\lambda}\equiv 0$ and hence $u_{\lambda}$ is a constant. This proves the theorem.
\end{proof}

\item \textbf{Concentration and multiplicity results.} From the concentration compactness results of P.L. Lions \cite{chap2-key17} if $u_{\lambda}$ is a minimal energy solution of (\ref{chap2-enum1}), then for anu sequence $\lambda_{k} \rightarrow \infty$ with $|\nabla u_{\lambda_{k}}|^{2}dx \rightarrow d\mu$, there exist a $x_{0}\in \partial\Omega$ such that $d\mu = \frac{s^{n/2}}{2}\delta_{x_{0}}$. Now the natural question in ``is it possible to characterize the concentration points $x_{0}$"? One expects from the asymptotic formula (\ref{chap2-eq10}) that $x_{0}$ must be a point of maximum mean curvature. This has been proved in Adimurthi, Pacella and Yadava \cite{chap2-key7} and we have the following
\end{enumerate}

\begin{theorem}\label{chap2-thm6}
Let $u_{\lambda}$ be a minimal energy solution of (\ref{chap2-enum1}) and $p_{\lambda} \in \overline{\Omega}$ be such that
$$
u_{\lambda}(P_{\lambda}) = \max\left\{u_{\lambda}(x) ; x\in \overline{\Omega}\right\}
$$
then there exist $a \lambda_{0} > 0$ such that for all $\lambda > \lambda_{0}$
\begin{enumerate}[{\rm a)}]
\item $p_{\lambda} \in \partial\Omega$ and is unique,\label{chap2-enum-a}
\item Let $n\geq 7$. The limit points of $\{P_{\lambda}\}$ are contained in the points of maximum mean curvature.\label{chap2-enum-b)} 
\end{enumerate}
\end{theorem}
Part (\ref{chap2-enum-a}) of this Theorem is also proved in \cite{chap2-key19}.

In view of the concentration at the boundary, it follows that the minimal energy solutions are not radial for $\lambda$ sufficiently large and $\Omega$ beging tha ball. Hence in a ball, for large $\lambda$, we obtain at  least two solutions one radial and the other non radial (see \cite{chap2-key5}). If $\Omega$ is not a ball then in Adimurthi and Mancini \cite{chap2-key2}, they obtained that $\Cat_{\partial\Omega}(\partial\Omega^{+})$ number of solutions for (\ref{chap2-enum1}) where $\partial\Omega^{+}$ is the set of points in $\partial\Omega$ where the mean curvature is positive (here for $X \subset U, Y$ topological space, then $\Cat_{Y}(x)$ is category of $X$ in $Y$). Further if $\partial\Omega$ has rich geometry in the sense described below, then Adimurthi-Pacella and Yadava \cite{chap2-key7} have obtained more solutions of (\ref{chap2-enum1}). They have proved the following

\begin{theorem}\label{chap2-thm7}
Let\pageoriginale $n\geq 7$. Assume that $\partial \Omega$ has k-peaks, that is there exist k-points ${x_{1}, \ldots x_{k}} \subset \partial\Omega$ at which $H(x_{i})$ is strictly local maxima. Then there exists $a \lambda_{0} > 0$ such that for $\lambda > \lambda_{0}$, there are $k$ distinct solutions $\{u_{i_{\lambda}}\}_{i}^{k} = 1$ of (\ref{chap2-thm1}) such that $u_{i_{\lambda}}$ concentrates at $x_{i}$ as $\lambda \rightarrow \infty$.
\end{theorem}
Theorems \ref{chap2-thm6} and \ref{chap2-thm7} has been extended for the mixed boundary value problems.

Theorem \ref{chap2-thm7} is not applicable in the case when $\Omega$ is a ball. On the other hand, given a positive integer $k$, there exists a $\lambda(k)$ such that for $\lambda > \lambda(k)$, (\ref{chap2-thm1}) admits at least $k$ number of radial solutions (see \cite{chap2-key18}). Part (\ref{chap2-enum(B)}) of the conjecture gives infinitely many rotationally equivalent solutions of minimal energy. In view of this it is not clear how to obtain more non radial solutions which are not rotationally equivalent. 
 
\begin{thebibliography}{99}
\bibitem{chap2-key1} Adimurthi and G. Mancini, \textit{The Neumann problem for elliptic equations with critical non-linearity,} A tribute is honour of G. Prodi, Ed by Ambrosetti and Marino, Scuola Norm. Sup. Pisa (1991) 9-25.
\bibitem{chap2-key2} Adimurthi and G, Mancini, \textit{Effect of geometry and topology of the boundary in the critical Neumann problem}, R. Jeine Angew. Math, to appear.
\bibitem{chap2-key3} Adimurthi and S.L. Yadava, \textit{Critical Sobolev exponent problem in $\bbR^{n}(n\geq 4)$ with Neumann boundary condition}, Proc. Ind. Acad. Sci. ${\mathbf {100}}$ (1990) 275-284.
\bibitem{chap2-key4} Adimurthi and S.L. Yadava, \textit{Existence and nonexistence of positive radila solutions of Neumann problems with Critical Sobolev exponents}, Arch. Rat. Mech. Anal. ${\mathbf{\100}}$ (1991) 275-296.
\bibitem{chap2-key5} Adimurthi and S.L. Yadava, \textit{Existence of nonradial positive solution for critical exponent problem withb Neumann boundary condition}, J. Diff. Equations, ${\mathbf{104}}$ 298-306.
\bibitem{chap2-key6} Adimurthi and S. L. Yadava, \textit{On a conjecture of Lin-Ni for semilinear Neumanna problem,} Trans. Amer. Math. Soc. ${\mathbf{336}}$ (1993) 631-637.
\bibitem{chap2-key7} Adimurthi, F. Pacella and S. L. Yadava, \textit{Interaction between the geometry of the boundary and positive solutions of a semilinear Newmann problem with Critical non linearity}, J. Funct. Anal. ${\mathbf{113}}$ (1993) 318-350.
\bibitem{chap2-key8} T. Aubin,\pageoriginale \textit{Nonlinear analysis of manifold: Monge-Ampere equations}, New York, Springer-Verlag (1992).
\bibitem{chap2-key9} A. Bhari and J.M. Coron, \textit{On a nonlinear elliptic equation involving the critical Sobolev exponent: The effect of the topology of the domain}, Comm. Pure Appl. Math. ${\mathbf{41}}$ (1988) 253-290.
\bibitem{chap2-key10} H. Brezis, \textit{Elliptic equations with limiting Sobolev exponent. The impact of topology}, Comm. Pure appl. Math. ${\mathbf{39}}$ (1996) S17-S39.
\bibitem{chap2-key11} H . Brezis, \textit{Non linear elliptic equations involving the Critical Sobolev exponent in survey and prespectives}, Directions in Partical differential equations, Ed, by Crandall etc. (1987),  17-36.
\bibitem{chap2-key12} C. Budd, M.C. Knapp and L.A. Peletier, \textit{Asymptotic behaviour of solution of elliptic equations with critical exponent and Neumann boundary conditions}, Proc, Roy. Soc. Edinburg ${\mathbf{117A}}$ (1991) 225-250.
\bibitem{chap2-key13} B. Gidas and J. Spruck, \textit{A priori boundas for positivce solutions of nonlinear elliptic equations}, Comm. Part. Diff. Equations ${\mathbf{6}}$ (1981) 883-901.
\bibitem{chap2-key14} H. Brezis and L. Nirenberg, \textit{Positive solutions of nonlinear elliptic equations involving Critical exponents}, Comm. Pure Appl. Maths. Vol.${\mathbf{36}}$ (1983) 437-477.
\bibitem{chap2-key15} C.S. Lin and W. M. Ni, \textit{On the diffusion coefficient of a semilinear Neumann problem}, Springer Lecture Notes ${\mathbf{1340}}$ (1986).
\bibitem{chap2-key16} C.S. Lin and W. M. Ni and I. Takagi, \textit{Large amplitude stationary solutions to an chemotaxis system}, J. Diff. Equations ${\mathbf{72}}$ (1988) 1-27.  
\bibitem{chap2-key17} P.L. Lions, \textit{The concentration compactness principles in the calcular of variations: The limit case} (Part 1 and part 2), Riv. Mat. Iberoamericana ${\mathbf{1}}$ (1985) 145-201; 45-121.
\bibitem{chap2-key18} W. M. Ni, \textit{On the positive radial solutions of some semi-linear elliptic equations on} $\bbR^{n}$, Appl. Math. Optim. ${\mathbf{9}}$ (1983) 373-380.
\bibitem{chap2-key19} W. M. Ni, S. B. Pan and I. Takagi. \textit{Singular behaviour of least energy solutions of a semi-linear Neumann problem involving critical Sobolev exponents}, Duke. Math. ${\mathbf{67}}$ (1992) 1-20.
\bibitem{chap2-key20} W.M. Ni and I . Takagi, \textit{On the existence and the shape of solutions to an semi-linear Neumann problem}, to appear in Proceedings of the conference on Nonlinear diffusion equations and their equilibrium states; held at Gregynog, Wales, August 1989.
\bibitem{chap2-key21} W.M. Ni\pageoriginale and I. Takagi, \textit{On the shape of least-energy solutions to a semilinear Neumann problem}, Comm. Pure Appl. Math. ${\mathbf{45}}$ (1991) 819-851.
\bibitem{chap2-key22} X. J. Wang, \textit{Neumann problem of semi-linear elliptic equations involving critical Sobolev exponents}, J. Diff. Equations, ${\mathbf{93}}$ (1991) 283-310.
\end{thebibliography}

\bigskip

\begin{flushleft}
T.I.F.R. Centre

P.O. Box No. 1234

Bangalore 560 012.
\end{flushleft}
