\title{Compact complex manifolds whose tangent bundles satisfy numerical effectivity properties}
\markright{Compact complex manifolds whose tangent bundles satisfy numerical effectivity properties}

\author{By~ Jean-Pierre Demailly\\(joint work with Thomas Peternell and Michael Schneider)}
\markboth{Jean-Pierre Demailly}{Compact complex manifolds whose tangent bundles satisfy numerical effectivity properties}

\date{}
\maketitle
\begin{center}
Dedicated to M. S. Narasimhan and C.S. Seshadri on their sixtith birthdays
\end{center}

\setcounter{section}{-1}
\section{Introduction}\label{art5-sec-0}
A compact Riemann\pageoriginale surface always ha s hermitian metric with constant curvature, in particular the curvature sign can be taken to be constant: the negative sign corresponds to curves of general type (genus $\geq$ 2), while the case to zero curvature corresponds to elliptic curves (genus 1), positive curvature being obtained only for $\bbP^{1}$ (genus 0). In higher dimensions the situation is must more subtle and it has been a long standing conjecture due to Frankel to characterize $\bbP_{n}$ as the only compact K\"ahler manifold with positive holomorphic bisectional curvature. Hratshorne strengthened Frankel's conjecture and asserted that $\bbP_{n}$ is the only compact complex manifold with ample tangent bundle. In his famous paper \cite{art5-keyMo79}, Mori solved Hartshorne's conjecture by using characteristic $p$ methods. Around the same time Siu and Yau \cite{SY80} gave an analytic proof of the Frankel conjecture. Combining algebraic and analytic tools Mok \cite{art5-keyMk88} classfied all compact K\"ahler manifolds with semi-positive holomorphic bisectional curvature. 

From the point of view of algebraic geometry, it is natural to consider the class fo projective manifolds $X$ whose tangent bundle in numerically effective (nef). This has been done by Campana and Peternell \cite{art5-keyCP91} and - in case of dimension 3 -by Zheng \cite{art5-keyZh90}. In particular, a complete classification is obtained for dimension at most three.

The main purpose of this work is to investigate compact (most often K\"ahler) manifolds with nef tangent or anticanonical bundles in arbitrary dimension. We fist discuss some basic properties of nef vector bundles which will be needed in the sequel in the general context of compact complex manifolds. We refer to \cite{art5-keyDPS91} and  \cite{art5-keyDPS92} for detailed proofs. Instead, we put here the emphasis on some unsolved questions.

\section{Numerically effective vector bundles}\label{art5-sec-1}

In algebraic geometry a powerful and flexible notion of semi-positivity is \textit{numerical effectivity}(``nefness"). We will explain here how to extend this notion to arbitrary compact complex manifolds.

\begin{definition}\label{art5-definition-1.1}
A line bundle $L$ on a projective manifold $X$ is said to be \textit{numerically effective} (nef for short) if $L\cdot C \geq 0$ for all compact curves $C \subset X$.

It is cleat that a line bundle with semi-positive curvature is nef. The converse had been conjectured by Fujita \cite{art5-keyFu83}. Unfortunately this is not true; a simple counterexample can be obtained as follows:
\end{definition}

\begin{example}\label{art5-example-1.2}
Let $\Gamma$ be an elliptic curve and let $E$ be a rank 2 vector bundle over $\Gamma$ which is a non-split extension of $\calO$ by $\calO$; such a bundle $E$ can be described as the locally constant vector bundle over $\Gamma$ whose monodromy is given by the matrices
$$
\begin{pmatrix}
1 & 0\\
0 & 1
\end{pmatrix}
,\qquad
\begin{pmatrix}
1 & 1\\
0 & 1
\end{pmatrix}
$$
associated to a pair of generators of $\pi_{1}(\Gamma)$. We take $L=\calO_{E}(1)$ over the ruled surface $X= \bbP(E)$. Then $L$ is nef and it can be checked that, up to a positive constant factor, there is only one (possibly singular) hermitian metric on $L$ with semi-positive curvature; this metric is  unfortunately singular and has logarithmic poles along a curve. Thus $L$ cannot be semi-positive for any smooth hermitian metric. 
\end{example}

\begin{definition}\label{art5-definition-1.3}
A vector bundle $E$ is called \textit{nef} if the line bundle $\calO_{E}(1)$ is nef on $\bbP(E)$ (= projectivized bundle of hyperplanes in the fibres of $E$).

Again it is clear that vector bundle $E$ which admits a metric with semi-positive curvature (in the sense of Griffiths) is nef. A compact K\"ahler manifold $X$ having semi-positive holomorphic bisectional curvature has bu definition a tangent bundle $TX$ with semi-positive curvature. Again the converse does not hold. One difficulty in carrying over the algebraic definition of nefness to the Kahler case is the possible lack of curves. This is overcome by the following:
\end{definition}

\begin{definition}\label{art5-definition-1.4}
Let $X$ be a compact complex manifold with a fixes hermaitian metric $\omega$. A line bundle $L$ over $X$ in \textit{nef} if for every $\varepsilon > 0$ there exists a smooth hermitian metric $h_{\varepsilon}$ on $L$ such that the curvature satisfies
$$
\Theta_{h_{\varepsilon}} \geq -\varepsilon\omega.
$$

This means that the curvature of $L$ can have an arbitrarilly small negative part. Clearly a $\nef$ line bundle $L$ satisfies $L\cdot C \geq 0$ for all curves $C\subset X$, but the coverse in not true ($X$ may have no curves at all, as is the case for instance for generic complex tori). For projective algebraic $X$ both notions coincide; this is an easy consequence of Seshadri's ampleness criterion: take $L$ to be a $\nef$ line bundle in the sense of Definition
\ref{art5-definition-1.1} and let $A$ be an ample line bundle; then $L^{\otimes K}\otimes A$ is ample for every integer $k$ and thus $L$ has smooth hermitian metric with curvature form $\Theta(L) \geq -\frac{1}{k}\Theta(A)$.

Definition \ref{art5-definition-1.3} can still be used to define the notion of $\nef$ vector bundles over arbitrary compact manifolds. If $(E, h)$ is a hermitian vector bundle recall that the Chern curvature tensor
$$
\Theta_{h}(E) =\dfrac{i}{2\pi}D_{E, h}^{2} = i \sum\limits_{\substack{1 \leq j , k\leq n \\ 1 \leq \lambda, \mu \leq r}}
a_{jk\lambda\mu}dz_{j} \wedge d\overline{z}_{k}\otimes e_{\lambda}^{\star}\otimes e_{\mu}
$$
is a hermitian (1,1)-form with values in $\Hom(E, E)$. We say that $(E, h)$ is \textit{semi-positive in Griffiths' sence} \cite{art5-keyGr69} and write $\Theta_{h}(E) \geq 0$ if $\Theta_{h}(E)(\xi \otimes t) = \sum a_{jk\lambda \mu} \xi_{j}\overline{\xi}_{k}v_{\lambda}\overline{v}_{\mu}\geq 0$ for every $\xi \in T_{x}X$, $v\in E_{x}$, $x\in X$. We write $\Theta_{h}(E)> 0$ in case there is strict inequality for $\xi \neq 0$, $m v\neq 0$. Numerical effectivity can then be characterized by the following differential geometric criterion (see \cite{art5-keyDe91}).  
\end{definition}

\begin{criterion}\label{art5-definition-1.5}
Let $\omega$ be a fixed hermitian metric on $X$. A vector bundle $E$ on $X$ is $\nef$ if and only if there is a sequence of hermitian metrics $h_{m}$ on $S^{m}E$ and a sequence $\varepsilon_{m}$ of positive numbers decreasing to 0 such that
$$
\Theta_{h_{m}}(S^{m} E) \geq -m\varepsilon_{m}\omega\otimes \Id_{S^{m} E}
$$
in the sense of Griffiths.

The main functional properties of $\nef$ vector bundles are summarized in the following proposition.
\end{criterion}

\begin{prop}\label{art5-prop-1.6}
Let $X$ be an arbitrary compact complex manifold and let $E$ be a  holomorphic vector bundle over $X$.
\begin{enumerate}[(i)]
\item \footnote{We expect (\ref{art5-enum-(i)}) to hold whenever $f$ is surjective, but there are serious technical difficulties to overcome in the nonalgebraic case.} If $f: Y\rightarrow X$ is a holomorphic map with equidimensional fibres, then $E$ is nef if and only if $f^{\star}E$ is nef.\label{art5-enum-(i)}
\item Let $\Gamma^{a}E$ be the irreducible tensor representation of $Gl(E)$ of highest weight $a= (a_{1}, \ldots a_{r}) \in \bbZ^{r}$, with $a_{1}\geq\ldots\geq a_{r} \geq 0$, Then $\Gamma^{a}E$ is nef. In particular, all symmetric and exterior powers of $E$ are nef.\label{art5-enum-(ii)}
\item let $F$ be a holomorphic vector bundle over $x$. If $E$ and $F$ are nef, then $E\otimes F$ is nef.\label{art5-enum-(iii)}

\item If some symmetric power $S^{m}E$ is nef $(m > 0)$, then $E$ in nef. \label{art5-enum-(iv)}

\item Let $0 \rightarrow F \rightarrow E \rightarrow Q \rightarrow 0$ be an exact sequence of holomorphic vector bundles over $X$. Then\label{art5-enum-(v)}
    \begin{enumerate}
        \item[($\alpha)$] $E$ nef $\Rightarrow Q$ nef.
        \item [($\beta)$] $F$, $Q$ nef $\Rightarrow E$ nef.
        \item [($\gamma)$] $E$ nef, $(\det Q)^{-1}$ nef $\Rightarrow F$ nef.
    \end{enumerate}
\end{enumerate}
\end{prop}

The proof of these properties in the general analytic context can be easily obtained by curvature computations. The argumentsa are parallel to those of the algebraic case and will therefore be omitted (see \cite{art5-keyHa66} and \cite{art5-keyCP91} for that case). Another useful result which will be used over and over in the sequel is

\begin{prop}\label{art5-prop-1.7}
Let $E$ be a nef vector bundle over a connected compact n-fold $X$ let $\sigma \in H^{0}(X, E^{\star})$ be a non zero section. Then $\sigma$ does not vanish anywhere.
\end{prop}

\begin{proof}
We merely observe that if $h_{m}$ is a sequence of hermitian metrics in $S^{m}E$ as in criterion 5, then
$$
T_{m}=\dfrac{i}{\pi} \partial\overline{\partial} \dfrac{1}{m}\log||\sigma^{m}||_{h_{m}}
$$
has zero $\partial\overline{\partial}$-cohomology class and satisfies $T_{m}\geq -\varepsilon_{m}\omega$. It follows that $T_{m}$ converges to a weak limit $T\geq 0$ with zero cohomology class. Thus $T= i\partial\overline{\partial}\varphi$ for some global plurisubharamonic function $\varphi$ on $X$. By the maximum principle this implies $T=0$. However, if $\sigma$ vanishes at some point $x$, then all $T_{m}$ have Lelong number $\geq 1$ at $x$. Therefore so has $T$, contradiction.
\end{proof}

one of out key results is a characterizations of vector bumdles $E$ which are numerically flat, i.e. such that both $E$ and $E^{\star}$ are nef.

\begin{theorem}\label{art5-thm-1.8}
Suppose that $X$ is K\"ahler. Then a holomorphic vector bundle $E$ over $X$ in numerically flat iff $E$ admits a filtration
$$
\{0 \} = E_{0}\subset E_{1}\subset \ldots \subset E_{p} = E
$$
by vector subbundles such that the quotients $E_{k}/E_{k-1}$ are hermitian flat, i.e. given by unitary representations $\pi_{1}(X) \rightarrow U(r_{k})$.
\end{theorem}

\begin{sketch of proof}
It is clear by \ref{art5-prop-1.6} (\ref{art5-enum-(v)}) that every vector bundle which os filtrated with hermitian flat quotients is nef as well as its dual. Conversely, suppose that $E$ is numerically flat. This assumption implies \break $c_{1}(E) = 0$ Fix a K\"ahler metric $\omega$. If $E$ is $\omega$-stable, then $E$ is Hermite-Einstein by the Unlenbeck-Yau theorem \cite{art5-keyUY86}, Moreover we have $0 \leq c_{2}(E) \leq c_{1}(E)^{2}$ by Theorem \ref{art5-thm-1.9} below, so $c_{2}(E) = 0$. Kobayashi's flatness that $E$ is hermitian flat. Now suppose that $E$ is unstable and take $\calF \subset\calO(E)$ to be destabilizing subsheaf of minimal rank $p$. We then have by definition $c_{1}(\calF) = c_{1}(\det \calF) = 0$ and the morphism $\det\calF \rightarrow \Lambda^{p}E$ cannot have any zero curvature current on the line bundle $\det \calF$, contradiction). This implies easily that $\calF$ is locally free, and we infer that $\calF$ is also numerically flat. Since $\calF$ is stable by definition, $\calF$ must be hermitian flat. We set $E_{1} = \calF$, observe that $E' - E/E_{1}$ is again numerically flat and proceed by induction on the rank.
\end{sketch of proof}

Another key point, which has been indeed used in the above proof, is the fact that the Fulton-Lazarsfeld inequalities \cite{art5-keyFL83} for Chern classes of ample vector bundles still hold for nef vector bundles over compact K\"ahler manifolds:

\begin{theorem}\label{art5-thm-1.9}
Let $(X, \omega)$ be a compact K\"ahler manifold and let $E$ be a nef vector bundle on $x$. Then for all positive polynomials $p$ the cohomology class $P(c(E))$ is numerically positive, that is, $\int_{Y}P(c(E)) \bigwedge \omega^{k} \geq 0$ for anu $k$ and any subvariety $Y$ of $X$.
\end{theorem}

By a positive polynomial in the Chern classes, we mean as usual a homogeneous weighted polynomial $P(c_{1}m \ldots, c_{r})$ with $\deg c_{i}= 2i$, such that $P$ is a positive integral combination of Schur polynomials:
$$
P_{a}(c)= \det (c_{a_{i}-i+j})_{1\leq i, j\leq r}, \quad r \geq a_{1} \geq a_{2} \geq \ldots \geq a_{r} \geq 0
$$
(by convention $C_{0}=1$ ana $c_{i} = 0$ if $i \neq [0,r]$, $r= \rank E$). The proof of Theorem \ref{art5-thm-1.9} is based essentially on the same artuments as the original proof of \cite{art5-keyFL83} for the ample case: the starting point is the nonnegativity of all Chern classes $c_{k}(E)$ (Bloch-Gieseker \cite{art5-keyBG71}); the general case then
follows from a formula of Schubert calculus known as the Kempf-Laksov formula \cite{art5-keyKL74}, which express any Schur ploynomial $P_{a}(c(E))$ as a Chern class $c_{k}(F_{a})$ of some related vector bundle $F_{a}$. The only change occurs in the proof of Gieseker's result, where the Hard Lefschetz theorem is needed for arbitrary K\"ahler metrics instead of hyperplane sections (fortunately enough, the technique then gets simlified, covering tricks being eliminated). Since $c_{1}c_{k-1}-c_{k}$
$$
0 \leq c_{k}(E) \leq c_{1} (E)^{k} \;\text{\rm for all}\; k
$$
Therefore all Chern monomials are bounded above by corresponding powers $c_{1}(E)^{k}$ of the same degree, and we infer:

\begin{coro}\label{art5-coro-1.10}
If $E$ in nef and $c_{1}(E)^{n} =0$, $n= \dim X$, then all Chern polynomials $P(c(E))$ of degree $2n$ vanish.
\end{coro}

\section{Compact K\"ahler manifolds with nef anti-canoni\-cal line bundle}\label{art5-sec-2}

 Compact K\"ahler manifolds with zero or semi-positive Ricci curvature have been investigated by various authors (cf. \cite{art5-keyCa57}, \cite{art5-keyKo61}, \cite{art5-keyLi67}, \cite{art5-keyLi71}, \cite{art5-keyLi72},
  \cite{art5-keyBo74a}, \cite{art5-keyBo74b}, \cite{art5-keyBe83}, \cite{art5-keyKo81} and \cite{art5-keyKr86}). The purpose of this section is to discuss the following two conjectures.

\begin{conjecture}\label{art5-conje-2.1}
Let $X$ be a compact K\"ahler manifold with numerically effective anticanonical bundle $K_{X}^{-1}$. Then the fundametal group $\pi_{1}(X)$ has polynomial growth.
\end{conjecture}

\begin{conjecture}\label{art5-conje-2.2}
Let $x$ be a compact K\"ahler manifold with $K_{X}^{-1}$ numerically effective. Then the Albanese map $\alpha : X\rightarrow \Alb(X)$ is a smooth fibration onto the Albanese torus. If this hold, one can infer that there is a finite
\'etale cover $\widetilde{X}$ has simply connected fibres. In particular, $\pi_{i}(X)$ would almost abelian (namely an extension of a finite group by a free abelian group).

These conjectures are known to be true if $K_{X}^{-1}$ is semi-positive. In both cases, the proof is based in differential geometric techniques (see e.g. \cite{art5-keyBi63}, \cite{art5-keyHK78} for Conjecture \ref{art5-conje-2.1} and \cite{art5-keyLi71} for Conjecture \ref{art5-conje-2.2}). However, the methods of proof are not so easy to carry over to the nef case. Our main contributions to these conjectures are derived from Theorem \ref{art5-thm-2.3} below.
\end{conjecture}

\begin{theorem}\label{art5-thm-2.3}
Let $X$ be a compact K\"ahler manifold with $K_{X}^{-1}$ nef. Then $\pi_{1}(X)$ is a group of subexponential growth.
\end{theorem}

The proof actually gives the following additional fact (this was already known before, see \cite{art5-keyBi63}).

\begin{coro}\label{art5-coro-2.4}
If morever $-K_{X}$ is hermitian semi-positive, then $\pi_{1}(X)$ has polynomial growth of degree $\leq 2 \dim X$, in particular $h^{1}(X, \calO_{X}) \leq \dim X$.
\end{coro}

As noticed by F. Campana (private communication), Theorem \ref{art5-thm-2.3} also implies the following consequences. 

\begin{coro}\label{art5-coro-2.5}
Let $X$ be a compact K\"ahler manifold with $K_{X}^{-1}$ nef. Let $\alpha : X\rightarrow \Alb(X)$ be the Albanese map and set $n=\dim X$, $d=\dim \alpha(X)$. If $d=0, 1$ od $n$, $\alpha$ is surjective. The same is true if $d = n-1$ and if $X$ is projective algebraic. 
\end{coro}

\begin{coro}\label{art5-coro-2.6}
Let $x$ be a K\"ahler surface or a projective 3-fold with $K_{X}^{-1}$ nef. Then the Albanese map $\alpha : X\rightarrow \Alb(X)$ is surjective.
\end{coro}

We now explain the main ideas required in the proof of Theorem \ref{art5-thm-2.3}. If $G$ is a finitely generated group with generators $\mathsf{g}_{1},\ldots, \mathsf{g}_{p}$, we denote by $N(k)$ the number of elements $\gamma \in G$ which can be written as words
$$
\gamma = \mathsf{g}_{i_{1}}^{\varepsilon_{1}}\ldots\mathsf{g}_{i_{k}}^{\varepsilon_{k}}, \quad \varepsilon_{j} = 0, 1 \; \text{\rm or} \; -1
$$
of length $\leq k$ in terms of the generators. The group $G$ is said to have \textit{subexponential growth} if for every $\varepsilon > 0$ there is a constant $C(\varepsilon)$ such that
$$
N(k) \leq C(\varepsilon)e^{\varepsilon k}\; \text{\rm for} \; k\geq 0.
$$
This notion is independent of the choice of generators. In the free group with two generators, we have $N(k) = 1+4(1+3+3^{2}+\cdots + 3^{k-1}) =2 \cdot 3^{k} - 1$, thus a group with subexponential growth cannot contain a non abelian free subgroup.

The first step consists in the construction of suitable K\"ahler metric on $X$. Since $K_{X}^{-1}$ in nef, for every $\varepsilon > 0$ there exists a smooth hermitian metric $h_{\varepsilon}$ on $K_{X}^{-1}$ such that
$$
u_{\varepsilon} = \Theta _{h_{\varepsilon}}(K_{X}^{-1}) \geq -\varepsilon\omega.
$$
By \cite{art5-keyY77} and \cite{art5-keyY78} there exists a unique k\"ahler metric $\omega_{\varepsilon}$ in the cohomology class ${\omega}$ such that
\begin{equation}
\Ricci(\omega_{\varepsilon}) = -\varepsilon\omega_{\varepsilon} + \varepsilon\omega + u_{\varepsilon}.\tag{+}\label{eq:simple}
\end{equation}
In fact $u_{\varepsilon}$ belongs to the Ricci class $c_{1}(K_{X}^{-1}) = c_{1}(X)$, hence so does the right hand side $-\varepsilon\omega_{\varepsilon} + \varepsilon\omega + u_{\varepsilon}$. In particular there exists a function $f_{\varepsilon}$ such that
$$
u_{\varepsilon} = \Ricci(\omega) + i \partial\overline{\partial}f_{\varepsilon}.
$$
If we set $\omega_{\varepsilon} = \omega + i \partial\overline{\partial}f_{\varphi}$ (where $\varphi$ depends on $\varepsilon$), equation \eqref{eq:simple} is equivalent to the Monge-Amp\`ere equation
\begin{equation}
\dfrac{\left(\omega + i \partial \overline\partial\right)^{n}}{\omega^{n}} = e^{\varepsilon\varphi-f_{\varepsilon}}\tag{++}\label{eq:simple1}
\end{equation}
because
\begin{align*}
i\partial\overline\partial\log(\omega + i \partial\overline\partial \varphi)^{n}/\omega^{n} &= \Ricci(\omega) - \Ricci(\omega_{\varepsilon})\\
&=\varepsilon(\omega_{\varepsilon} - \omega) + \Ricci(\omega) - u_{\varepsilon}\\
&= i\partial\overline\partial(\varepsilon \varphi - f_{\varepsilon}).
\end{align*}
It follows from the general results of \cite{art5-keyY78} that \eqref{eq:simple1} has a unique solution $\varphi$, thanks to the fact the right hand side of \eqref{eq:simple1} is increasing in $\varphi$. Since $u_{\varepsilon} \geq -\varepsilon\omega$, equation \eqref{eq:simple} implies in particular that $\Ricci(\omega_{\varepsilon}) \geq -\varepsilon \omega$.

Now, recall the well-known differential geometric technique for\break bounding $N(K)$ (this technique has been explained to us in a very efficient way by S.Gallot). Let $(M, \mathsf{g})$ be a compact Riemannian $m$-fold and let $E\subset \widetilde{M}$ be a fundamental domain for the action of $\pi_{1}(M)$ on the universal covering $\widetilde{M}$. Fix $a\in E$ and set $\beta - \diam E$ . Since $\pi_{1}(M)$ acts isometrically on $\widetilde{M}$ with respect to the pull-back metric $\overline{\mathsf{g}}$, we infer that
$$
E_{k} = \bigcup\limits_{\gamma\in \pi_{1}(M),\;\; \length(\gamma)\leq k} \gamma(E)
$$
has volume equal to $N(k)$ $\Vol(M)$ and is contained in the geodesic ball $B(a, \alpha k + \beta)$, where $\alpha$ is maximum of the length of loops representing the generators $\mathsf{g}_{j}$. Therefore
\begin{equation*}
N(K) \leq \dfrac{\Vol(B(a, \alpha k + \beta))}{\Vol(M)} \tag{$\ast$}\label{eq:sample2}
\end{equation*}
and it is enough to bound the volume of geodesic balls in $\widetilde{M}$. For this we use the following fundamental inequality due to R. Bishop \cite{art5-keyBi63}, Heintze-karcher \cite{art5-keyHK78} and M. Gage \cite{art5-keyGa80}.

\begin{lem}\label{art5-lem-2.7}
Let
$$
\Phi : T_{a}\widetilde{M} \rightarrow \widetilde{M},\qquad \Phi(\zeta)= \exp_{a}(\zeta)
$$
be the (geodesic) exponential map. Denote by
$$
\Phi^{*}dV_{g}= a(t, \zeta)\; dt\; d\sigma(\zeta)
$$
the exrpression of the volume element in spherical coordinates with $t \in \bbR_{+}$ and $\zeta \in S_{a}(1) =$ unit spheren in $T_{a}\widetilde{M}$. Suppose that $a(t, \zeta)$ does not vanish for $t \in ]0, \tau (\zeta)[$, wiht $\tau(\zeta) = + \infty$ or $a(\tau(\zeta), \zeta) = 0$ Then $b(t, \zeta) - a(t, \zeta)^{1/(m-1)}$ satisfies on $]0$, $\tau(\zeta)[$ the inequality
$$
\dfrac{\partial^{2}}{\partial t^{2}}b(t, \zeta) + \dfrac{1}{m-1}\Ricci_{\mathsf{g}}(v(t, \zeta), v(t, \zeta))b(t, \zeta)\leq 0
$$
where
$$
v(t, \zeta) = \dfrac{d}{dt}\exp_{a}(t\zeta)\in S_{\Phi(t\zeta)}(1) \subset T_{\Phi (t\zeta)}\widetilde{M}.
$$
\end{lem}

If $\Ricci_{\mathsf{g}} \geq -\varepsilon\mathsf{g}$, we infer in particular
$$
\dfrac{\partial^{2}b}{\partial t^{2}} - \dfrac{\varepsilon}{m-1} b\leq 0
$$
and therefore $b(t, \zeta) \leq \alpha^{-1} \sinh(\alpha t)$ wiht $\alpha = \sqrt{\varepsilon/(m-1)}$ (to check this observe that $b(t, \zeta) = t + o(t))$ at 0 and that $\sinh(\alpha t)\partial b/\partial t -\alpha \cosh(\alpha t)b$ has a negtive derivative). Now, every point $x\in B(a, r)$ can be joined to $a4$ by a minimal geodesic art of lenght $< r$. Such a geodesic are cannot contain any focal point (i.e. any critical value of $\Phi$), except possibly at the end point $x$. It follows that $B(a, r)$ is the image be $\Phi$ of the open set
$$
\Omega(r) = \{(t. \zeta) \in [0, r[ \times S_{a}(1); t<\tau (\zeta)\}.
$$
Therefore
$$
\Vol_{\mathsf{g}}(B(a, r)) \leq \int_{\Omega(r)} \Phi^{*}dV_{\mathsf{g}} = \int_{\Omega(r)}b(t, \zeta)^{m-1}dt\; d\sigma(\zeta).
$$
As $\alpha^{-1}\sinh(\alpha t)\leq t e^{\alpha t}$, we get
\begin{equation*}
\Vol_{\mathsf{g}}(B(a, r)) \leq \int_{S_{a}(1)} d\sigma(\zeta) \int_{0}^{r} t^{m-1}e^{(m-1) \alpha t} dt\leq v_{m}r^{m}
e^{\sqrt{(m-1)\varepsilon r}} \tag{$\ast\ast$}\label{eq:sample3}
\end{equation*}
where $v_{m}$ is the volume of the unit ball in $\bbR^{m}$.

In our application, the difficulty is that the matrix $\mathsf{g} = \omega_{\varepsilon}$ varies with $\varepsilon$ as well as the constants $\alpha = \alpha_{\varepsilon,}$  $\beta = \beta_{\varepsilon}$ in \eqref{eq:sample2}, and $\alpha_{\varepsilon}\sqrt{(m-1)\varepsilon}$ need nit converge to 0 as $\varepsilon$ tents to 0. We overcome theis difficulty by the following lemma.

\begin{lem}\label{art5-lemma-2.8}
Let $U_{1}, U_{2}$ be compact subsets of $\widetilde{X}$. Then for every $\delta > 0$, there are closed subsets $U_{1, \varepsilon,\delta} \subset U_{1}$ and $U_{2, \varepsilon, \delta} \subset U_{2}$ with $\Vol_{\omega}(U_{j}U_{j, \varepsilon, \delta})\break < \delta$, such that any two points $x_{1}\in U_{1, \varepsilon, \delta}$, $x_{2} \in U_{3, \varepsilon, \delta}$ can be joined by a path of length $\leq C \delta^{-1/2}$ with respect to $\omega_{\varepsilon}$, where $C$ is a  constant independent of $\varepsilon$ and $\delta$. 
\end{lem}

We will not explain the details.The basic observation is that
$$
\int_{x}\omega_{\varepsilon} \wedge \omega^{n-1} = \int_{X} \omega^{n}
$$
does not depend on $\varepsilon$, therefore $||\omega_{\varepsilon}||_{L^{1} (X)}$ is uniformly bounded. This is enough to imply the existence of suffciently many paths of bounded lenght between random points taken in $X$ (this is done for example by computing the average lenght of piecewise linear paths).

We let $U$ be a  comnpact set containing the fundamental domain $E$, so large that $U^{\circ} \cap\mathsf{g_{j}}(U^{\circ}) \neq \emptyset$ for each generator $\mathsf{g}_{j}$. We apply Lemma \ref{art5-lemma-2.8} with $U_{1} = U_{2} =U$ and $\delta > 0$ fixed such that
$$
\delta < \dfrac{1}{2}\Vol_{\omega}(E), \quad \delta < \dfrac{1}{2}\Vol_{\omega}(U\cap\mathsf{g}_{j}(U)).
$$
We get $U_{\varepsilon,\delta} \subset U$ with $\Vol_{\omega}(U U_{\omega, \delta}) < \delta$ and $\diam_{\omega_{\varepsilon}} \leq C\delta^{-1/2}$ The inequalities on volumes imply that $\Vol_{\omega}(U_{\varepsilon, \delta} \cap E) \geq \frac{1}{2}\Vol_{\varepsilon}(E)$ and $U_{\varepsilon,\delta} \cap \mathsf{g}_{j}(U_{\varepsilon, \delta}) \neq \emptyset$ for every $j$ (note that all $\mathsf{g}_{j}$ preserve volumes). It is then clear that
$$
W_{k,\varepsilon,,\delta}: = \bigcup\limits_{\gamma \in \pi_{1}(X), \length(\gamma)\leq k} \gamma (U_{\varepsilon, \delta})
$$
satisfies
\begin{align*}
&\Vol_{\omega}(W_{k,\varepsilon, \delta}) \geq N(k)\Vol_{\omega}(U_{\varepsilon, \delta} \cap E) \geq N(k)\frac{1}{2}\Vol_    {\omega}(E) \quad \text{\rm and}\\
&\diam_{\omega_{\varepsilon}}(W_{k,\varepsilon,\delta}) \leq k \diam_{\omega_{\varepsilon}}U_{\varepsilon, \delta}\leq      kC\delta^{-1/2}.
\end{align*}
Since $m=\dim_{\bbR}\; X=2n$, inequality \eqref{eq:sample3} implies
$$
\Vol_{\omega_{\varepsilon}}(W_{k, \varepsilon, \delta}) \leq \Vol_{\omega_{\varepsilon}}(B(a, kC\delta^{-1/2})) \leq C_{4}k^{2n}e^{C_{5\sqrt{\varepsilon}k}}.
$$ 
Now $X$ is compact, so there is a constant $C(\varepsilon) > 0$ such that $\omega^{n} \leq C(\varepsilon)\omega_{\varepsilon}^{n}$. We conclude that
$$
N(K)\leq \dfrac{2\Vol_{\omega}(W_{k,\varepsilon,\delta})}{\Vol_{\omega}(E)} \leq C_{6}C(\varepsilon)k^{2n}e^{C_{5 \sqrt{\varepsilon}k}}.
$$
The proof of Theorem \ref{art5-thm-2.3} is complete.

\begin{remark}\label{art5-remark-2.9}
It is well known and easy to check that equation \eqref{eq:simple1} implies 
$$
C(\varepsilon) \leq \exp \left(\max\limits_{X} f_{\varepsilon} - \min\limits_{X}f_{\varepsilon}\right).
$$
Therefore it is reasonable to expect the $C(\varepsilon)$ has polynomial growth in $\varepsilon^{-1}$; this would imply that $\pi_{1}(X)$ has polynomial growth by taking $\varepsilon = k^{-2}$. When $K_{X}^{-1}$ has a semipositive metric, we can even take $\varepsilon = 0$ and find a metric $\omega_{0}$ with $\Ricci(\omega_{0}) = u_{0} \leq 0$. This implies Corollary \ref{art5-coro-2.4}.
\end{remark}

\noindent
\textbf{Proof of Corollary \ref{art5-coro-2.5}.}
If $d=0$, then by definition $H^{0}(X, \Omega_{X}^{1}) = 0$ and $\Alb(X) = {0}$.

If $d=n$, the albanese map has generic rank $n$, so there exist holomorphic 1-forms $u_{1}, \ldots, u_{n}$ such that $u_{1}\wedge \ldots \wedge u_{n} \nequiv$ 0. How ever $u_{1}\wedge \ldots \wedge u_{n}$ is a section of $K_{X}$ which has a nef dual, so $u_{1} \wedge \ldots \wedge u_{n}$ cannot vanish by Proposition \ref{art5-prop-1.7} and $K_{X}$ is trivial. Therefore $u_{1} \wedge \ldots \hat{u}_{k}\ldots \wedge u_{n} \wedge v$ must be a constant for every holomorphic !-form $v$ and $(u_{1}, ldots, u_{n})$ is a basis of $H^{\circ}(X, \Omega_{X}^{1})$. This implies $\dim A(X)= n$, hence $\alpha$ is surjective.

If $d =1$, the image $C=\alpha(X)$ is a smooth curve. The genus $\mathsf{g}$ of $C$ cannot be $\geq 2$, otherwise $\pi_{1}(X)$ would be mapped onto a subgroup of finite index in $pi_{1}(C)$, and thus would be of exponential growth, contradicting Theorem \ref{art5-thm-2.3} Therefore $C$ is an elliptic curve and is a subtorus of $\Alb(X)$. By the universal property of the Albanese map, this is possible only if $C= \Alb(X)$.

The case $d =n-1$ is more subtle and uses Mori theory (this is why we have to assume $X$ projective algebraic). We refer to \cite{art5-keyDPS92} for the details.

\section{Compact complex manifolds with nef tangent bundles}\label{art5-sec-3}

Several interesting classes of such manifolds are produces by the following simple observation.

\begin{prop}\label{art5-prop-3.1}
Every homogeneous compact complex manifold has a nef tangent bundle.
\end{prop}

Indeed, if $X$ is homogeneous, the Killing vector fields generate $TX$, so $TX$ is a quotient of a quotient of a trivial vector bundle. In praticular, we get the following

\begin{examples}(homogeneous case)\label{art5-example-3.2}
    \begin{enumerate}[(i)]
    \item Rational homogeneous manifolds: $\bbP_{n}$, flag manifolds,quadrics $Q_{n}$ (all are Fano manifolds, i.e.      projective algebraic with $K_{X}^{-1}$ ample.)\label{art5-enu-{i}}
     \item Tori $\IC/\Lambda$
     (K\"ahler, possibly non algebraic). \label{art5-enu-{ii}}   
    \item Hopf manifolds $\IC{0}/H$ where $H$ is a discrete group of homotheties (non K\"ahler for $n\geq 2$).\label{art5-enu-{iii}}
    \item Iwasawa manifolds $G/\Lambda$ where $G$ is the group of unipotent upper triangular $p\times p$ matrices and $\lambda$ the subgroup of matrices with entries in the ring of integers of some imaginary quadratic field. eg. $\bbZ[i]$ (non K\"ahler for $p\geq$, although $TX$ is trivial).\label{art5-enu-{iv}}
    \end{enumerate}
\end{examples}

We must remark at this point that not all manifolds $X$ with nef tangent bundles are homogeneous, the automorphism group may even be discrete:

\begin{example}\label{art5-example-3.3}
Let $\Gamma = \IC/(\bbZ + \bbZ \tau)$, $Im\tau > 0$, be ana elliptic curve. Consider the quotient space $X = (\Gamma \times \Gamma \Gamma)/G $ where $G={1, \mathsf{g_{1}}, \mathsf{g_{2}}, \mathsf{g_{1}}\mathsf{g_{2}}} \simeq \bbZ_{2} \times \bbZ_{2}$ is given by
\begin{align*}
\mathsf{g_{1}}(z_{1}, z_{2}, z_{3}) &= \left(z_{1} + \dfrac{1}{2}, -z_{2}, -z_{3}\right),\\
\mathsf{g_{1}}(z_{1}, z_{2}, z_{3}) &= \left(-z_{1}, z_{2}+ \dfrac{1}{2}, -z_{3}+ \dfrac{1}{2}\right),\\
\mathsf{g_{1}g_{2}}(z_{1}, z_{2}, z_{3})& =\left(-z_{1}+\dfrac{1}{2}, -z_{2}+\dfrac{1}{2}, z_{3}+ \dfrac{1}{2}\right).
\end{align*}

Then $G$ acts freely, so $X$ is smooth. It is clear also that $TX$ is nef (in fact $TX$ is unityu flat). Since the pull-back of $TX$ to $\Gamma \times \Gamma \times \Gamma$ is trivial, we easily conclude that $TX$ has no sections, thanks to the change of signs in $ \mathsf{g_{1}}, \mathsf{g_{2}}, \mathsf{g_{1}g_{2}}$. Therefore the automorphism group $\Aut(X)$ is discrete. The same argument shows that $H^{0}(X, \Omega_{x}^{1}) = 0$.
\end{example}

\begin{example}\label{art5-example-3.4}
Let $X$ be the ruled surface $bbP(E)$ over the elliptic curve $\Gamma = \IC(\bbZ + \bbZ\tau)$ defined in Example
\ref{art5-example-1.2}. Then the relative tangent bundle of $bbP(E)\rightarrow \Gamma$(=relative anticanonical line bundle) is $\pi^{\star}(\det E^{\star})\otimes \calO_{E}(2) \simeq \calO_{E}(2)$ and $T\Gamma$ is trivial, so $TX$ is nef. Moreover $X$ is almost homogeneous, with automorphisms induced by
$$
(x_{1}, z_{1}, z_{2}) \mapsto (x+a, z_{1}+b, z_{2}),\quad (a,b) \in \IC^{2}
$$ 
and a single closed orbit equal to the curve $\{z_{2}=0\}$. Here, no finite \'etale cover of $X$ can be homogeneous, otherwise $K_{X}^{-1} = \cal_{E}(2)$ would be semi-positive. Observe that no power of $K_{X}^{-1}$ is generated by sections, although $K_{X}^{-1}$ in nef.
\end{example}

Our main result is structure theorem on the Albanese map of compact K\"ahler manifolds with nef tangent bundles.

\begin{main theorem}\label{art5-main-thm-3.5}
Let $X$ be a compact K\"ahler manifold with nef tangent bundle $TX$. Let $\widetilde{X}$, be a finite \'etale cover of $X$ of maximum irregularity $q=q(\widetilde{X}) = h^{1}(\widetilde{X}, \calO_{\widetilde{X}})$. Then
    \begin{enumerate}[(i)]
        \item $\pi_{1}(\widetilde{X}) \simeq \bbZ^{2_{q}}$.\label{art5-enum_(i)}
        \item The albanese map $\alpha : \widetilde{X} \rightarrow A(\widetilde{X})$ is a smooth fibration over a q-dimensional torus with nef relative tangent bundle. \label{art5-enum_(ii)}
        \item The fibres $F$ of $\alpha$ are Fano manifolds with nef tangent bundles.\label{art5-enum_(iii)}
    \end{enumerate}
\end{main theorem}

Recall that  a Fano manifold is by definition a compact comples manifold with ample anticanonical bundle $K_{X}^{-1}$. It is well known that Fano manifolds are always simply connected (Kobayashi \cite{art5-keyKo61}). As a consequence we get 

\begin{coro}\label{art5-coro-3.6}
With the assumtions of \ref{art5-main-thm-3.5}. the fundamental group $\pi_{1}(X)$ is an extension of a finite group by $Z^{2_{q}}$.
\end{coro}

In order to complete the classification of compact K\"ahler maniflods with nef tangent bundles (up to finite \'etale cover), a solution of the following two conjectures would be neart5-enum-(i)eded.

\begin{conjecture}\label{art5-conje-3.7}
(Campana- Peternell \cite{art5-keyCP91}) Let $X$ be ab Fano manifold Then $X$ has a nef tangent bundle of and only if $X$ i rational homogeneous.
\end{conjecture}

The evidence we have for Conjecture \ref{art5-conje-3.7} is that it is true up to dimension 3. In dimension 3 there are more than 100 different types of Fano manifolds, but only five types have a nef tangent bundle, namely $bbP_{3}$, $Q_{3}$ (quadric), $\bbP_{1}\times \bbP_{2}$, $\bbP_{1}\times \bbP_{1}$, $\bbP_{1}\times \bbP_{1}$ and the flag manifold $F_{1,2}$ of lines and planes in $\IC^{3}$;m all five are homogeneous.   

A positive solution to Conjecture \ref{art5-conje-3.7} would clarify the structure of fibers in the Albanese map of Theorem \ref{art5-main-thm-3.5}. To get a complete picture of the situation, one still needs to know how the fibers are deformed and glued together to yield a holomorphic family over the Albanese torus. We note that $K_{\widetilde{X}}^{-1}$ is relatively ample, thus for $m$ large the fibres can be embedded in the projectivized bundle of the direct image bundle $\alpha_{*}(K_{\widetilde{X}}^{-m}$. The structure of the deformation i described by the following theorem.

\begin{theorem}\label{art5-thm-3.8}
In the situation of Theorem \ref{art5-main-thm-3.5}, all direct image bundles $E_{m}= \alpha_{*}(K_{\widetilde{X}}^{-m})$ are numerically flat over the Albanese torus. Moreover, for $p \gg m \gg 0$, the fibers of the Albanese map can ne described as Fano submanifolds of the fibers of $IP(E_{m})$ defined by polynomial equations of degree $p$, in such a way that the bundle of equations $V_{m, p} \subset S^{p}(E_{m})$ is itself numerically flat.
\end{theorem}

Theorem \ref{art5-thm-3.8} is proved in \cite{art5-keyDPS91} in case $X$ is projective algebraic. The extension to the K\"ahler case has been obtatined by Ch. Mourougane in his PhD Thesis work (Grenoble, still unpublished). We now explain the main steps in the proof of Theorem \ref{art5-main-thm-3.5} One of the key points is the following

\begin{prop}\label{art5-prop-3.9}
Let $X$ be a compact K\"ahler n-fold with $TX$ nef. Then
    \begin{enumerate}[(i)]
    \item If $c_{1}(X)^{n} > 0$, then $X$ is a Fano manifold.\label{art5-enum-[i]} 
    \item If $c_{1}(X)^{n} = 0$, then $^{}{\chi}(\calO_{X}) = 0$ and there exists a non zero holomorphic $p$-form, $p$ suitable odd and a finite \'etale cover $\widetilde{X} \rightarrow X$ such that $q(\widetilde{X}) > 0$.\label{art5-enum-[ii]}
    \end{enumerate}
\end{prop}

\begin{proof}
We first check that every effective divisor $D$ of $X$ in nef. In fact, let $\sigma \in H^{0}(X, \calO(D))$ be a section with divisor $D$. Then for $k$ larger than the maximum vanishing order of $\sigma$ on $X$, the $k$-jet section $j^{k}\sigma \in H^{0}(X, j^{k}\calO(D))$ has no zeroes. Therefore, there is an injection $\calO \rightarrow J^{k}\calO(D)$ and a dual surjection
$$
(J^{k}\calO(D))^{\star} \otimes \calO(D)\rightarrow(D).
$$
Now , $J^{k}\calO(D)$ has a filtration whose graded bundle is $\bigoplus_{0\leq p\leq k}S^{p}T^{\star}X\otimes \calO(D)$,
so $(J^{k}\calO(D))^{\star} \otimes \calO (D)$ has a dual filtration with graded bundle $\bigoplus_{0\leq p\leq k}S^{p}TX$.
By \ref{art5-prop-1.6} (\ref{art5-enum-(i)}) and \ref{art5-prop-1.6} (\ref{art5-enum-(v)})($\beta$), we conclude that $(J^{k}\calO(D))^{\star}\otimes \calO (D)$ is nef, so its quotient $\calO (D)$ in nef by \ref{art5-prop-1.6}
(\ref{art5-enum-(v)}) ($\alpha$). 

Part (\ref{art5-enum-(i)}) is based on the solution of the Grauert-Riemenschneider conjecture as proved in \cite{art5-keyDe85}. Namely, $L=K_{X}^{-1} = \Lambda^{n}TX$ is nef and satisfies $c_{1}(L)^{n} > 0$, so $L$ has Kodaira dimension $n$ (holomorphic More inequalities are needed at that point because $X$ is not suppose a priori to be algebraic).
It follows that $X$ is Moishezon,thus projective algebraic, and for $m > 0$ large we have $L^{m} = \calO(D+A)$ with divisors $D$, $A$ such that $D$ is effective and $A$ ample. Since $D$ must be in fact nef, it follows that $L=K_{X}^{-1}$ is ample, as desired.

The most difficult part is (\ref{art5-enum-(ii)}). Since $c_{1}(X)^{n} =0$, Corollary \ref{art5-coro-1.10} implies $_{}\chi(\calO_{X}) = 0$. By Hodge symmetry, we get $h^{0}(X, \Omega_{X}^{p}) = h^{p}(X, \calO_{X})$ and
$$
_{}\chi(\calO_{X}) = \sum\limits_{0 \leq p \leq n} (-1)^{p}h^{0}(X, \Omega_{X}^{p}) = 0.
$$
From this and the fact that $h^{0}(X, \calO_{X}) =1$, we infer the existence of a non zero $p$-form $u$ for some suitable odd number $p$. Let
$$
\sigma : \Lambda^{p-1}TX \rightarrow \Omega_{X}^{1} 
$$
be the bundle moriphism obtained by contracting $(p-1)$-vectors with $u$. For every $k > 0$, the morphism $\Lambda^{k}\sigma$
can be viewed as section of the bundle $\Lambda^{k}(\Lambda^{p-1}T^{\star}X)\otimes \Lambda^{k}T^{\star}X$ which has nef dual. Hence by Proposition \ref{art5-prop-1.7} we know the $\Lambda^{k}\sigma$ is either identically zero or does not vanish. This mean is that $\sigma$ has constant rank. Let $E$ be the image of $\sigma$. Then $E$ is a quotient bundle of $\Lambda^{p-1}TX$, so $E$ in nef, and $E$ is subbundle of $\Omega_{X}^{1} = T^{\star}X$, so $E^{\star}$ is likewise $\nef$.
Theorem \ref{art5-thm-1.8} implies the existence of a hermintian flat subbundle $E_{1} \subset E$. If $E_{1}$ would be trivial after pulling-back to some finite  etale cover $\widetilde{X}$, we would get a trivial subbundle of $\Omega_{\widetilde{X}}^{1}$, hence $q(\widetilde{X}) > 0$ and the proposition would be proved. Otherwise $E_{1}$ is given by some infinite representation of $\pi_{1}(X)$ inti some unitary group. Let $\Gamma$ be the monodromy group (i.e. the image of $\pi_{1}(X)$ by the representation). We use a result of Tits \cite{art5-keyTi72} according to which every subgroup contains either a non abelian free subgroup or a solvable subgroup of finite index. The first case cannot occur by Theorem \ref{art5-thm-2}.?. In the second case,we may assume $\Gamma$ solvable by taking some finite \'etale cover. We consider the series of derived groups
$$
\Gamma \supset \Gamma_{1} \supset \ldots \supset \Gamma_{N} = {0}
$$
and the largest index $k$ such that $\Gamma_{k}$ has finite index in $\Gamma$. Then the inverse image of $\Gamma_{k}$ in $\pi_{1}(X)$ defines a finite \'etale cover $\widetilde{X}$ of $X$  with infinite first homologu group (the representation maps this group onto $\Gamma_{k}/\Gamma_{k+1}$ which is infinite). Hence $q(\widetilde{X}) > 0$, as desired.
\end{proof}

\begin{proof of the main theorem*}
Let $X$ be compact K\"ahler mainfold with nef tangent bundle. Since a son zero holomorphic form $u \in H^{0}(X, \Omega_{X}^{1})$ can never vanish by Proposition \ref{art5-prop-1.7}, it follows immediately that the Albanese map $\alpha$ has rank to $q(X)$ at very point, hence $\alpha$ is a submersion and $q(X) \leq n$. Let $(\widetilde{X})$ be a finite
\'etale cover with maximum irregularity $q=q(\widetilde{X})$ (note that $(\widetilde{X})$ also a nef tangent bundle, so $q(\widetilde{X})\leq n$). let $F$ denote the fibers of the Albanese map $\alpha : (\widetilde{X}) \rightarrow A(\widetilde{X})$ The relative tangent bundle exact sequence. 
$$
0 \rightarrow TF \rightarrow TX \xrightarrow{d\alpha} \alpha^{\star}TA(X)\rightarrow 0.
$$
in which $TA(X)$ is trivial shows by \ref{art5-prop-1.6} (\ref{art5-enum-(v)})$(\gamma)$ that $TF$ in nef. Lemma
\ref{art5-lem-3.10} \eqref{art5-enum-lem(3.10)_{iii}} below implies that all finite \'etale covers $\widetilde{F}$ of $F$ satisfy $q(\widetilde{F}) = O$. Hence the fibers $F$ must be Fano by proposition \ref{art5-prop-3.9} and the main Theorem follows.   
\end{proof of the main theorem*}

\begin{lem}\label{art5-lem-3.10}
Let $X, Y$ be compact K\"ahler manifolds and let $\mathsf{g}: X\rightarrow Y$ be a smooth fibration with connected fibers. We let $q(X)$ be the irregularity of $X$ and $\widetilde{q}(X)$ be the sup of the irregularity of all finite \'etale covers. If $F$ denotes any fibre of $\mathsf{g}$, then
 \begin{enumerate}[(i)]
 \item $q(X) \leq q(Y) + q(F)$,\label{art5-enum-lem(3.10)_{i}}
 \item $\widetilde{q}(X) \leq \widetilde{q}(Y) + \widetilde{q}(F)$.\label{art5-enum-lem(3.10)_{ii}}
 \item Suppose that the boundary map $\pi_{2}(Y) \rightarrow \pi_{1}(F)$ is zero, that $\pi_{1}(F)$ contains an abelian subgroup of finite indes and that $Y$ contains a subvariety $S$ with $\pi_{1}(S)\simeq \pi_{1}(Y)$, such that any two generic points in the universal covering $\hat{S}$ can be joined thorugh a chain of holomorphic images $\IC \rightarrow \hat{S}$. Then\label{art5-enum-lem(3.10)_{iii}}
 $$
 \widetilde{q}(X) = \widetilde{q}(Y) + \widetilde{q}(F).
 $$ 
 \end{enumerate}
\end{lem}

The proof is based on a use of the Leray spectral sequence and a study of the resultig monodromy on $H^{1}(F, \IC)$. Triviality of the monodromy is achieved in case (\ref{art5-enum-lem(3.10)_{iii}}) becase all K\"ahler deformations of tori over $Y$ must be trivial. We refer the reader to \cite{art5-keyDPS91} for the details. In our application, $Y$ is taken to be the Albanese torus, so assumption (\ref{art5-enum-lem(3.10)_{iii}}) is satisfied with $S=Y$ ($\pi_{1}(F)$ contains an abelian subgroup of finite indes thanks to Corollary \ref{art5-coro-3.6}, by using an induction on dimension). 

\section{Classification in dimension 2 and 3}\label{art5-sec-4}

By using the Kodaira classification of surface and the structure theorems of Section \ref{art5-sec-3}, it is not difficult to classify all K\"ahler surface with nef tangent bundles; except for tori, the K\"ahler classification in identical to the projective one. The projective case was already mentioned in \cite{art5-keyCP91} and \cite{art5-keyZh90}.
    \begin{theorem}
    Let $X$ be a smooth K\"ahler surface such that $TX$ is nef. Then $X$ is minimal and is exactly one of the surfaces in the following list:
    \begin{enumerate}[(i)]
    \item $X$ is torus;\label{art5-enum-thm-(1)}
    \item $X$ is hyperellipitic;\label{art5-enum-thm-(ii)}
    \item $X= \bbP_{2}$;\label{art5-enum-thm-(iii)}
    \item $X= \bbP_{1} \times \bbP_{1}$;\label{art5-enum-thm-(iv)}
    \item $X = \bbP(E)$, where $E$ is a rank 2-vector bundle on an elliptic curve $C$ with either \label{art5-enum-thm-(v)}
        \begin{enumerate}
        \item [[$(\alpha)$]] $E= \calO \oplus L$, $L \in \Pic^{0}(C)$, or
        \item [[$(\beta)$]] $E$ is given by a non split extension $0 \rightarrow \calO \rightarrow E \rightarrow L \rightarrow 0$ with $L = \calO$ or $\deg L = 1$.
        \end{enumerate}
    \end{enumerate}
\end{theorem}

The list of non-k\"ahler surfaces in the Kodaira classification is much smaller. It is then rather easy to check nefness in each case:

\begin{theorem}\label{art5-thm-4.2}
The smooth non K\"ahler compact comlex surface with nef tangent bundles are precisely:
    \begin{enumerate}[(i)]
        \item Kodaira surfaces (that is surfaces of Kodaira dimension 0 with $b_{1}(X)$x
        odd);\label{art5-enum-thm(4.2)-(i)}
       \item Hopf-surfaces (that is, surfaces whose universal cover is $\IC^{2}{0}$).\label{art5-enum-thm(4.2)-(ii)} 
    \end{enumerate}
\end{theorem}

A similar classification can be obtained for 3-dimensional compact K\"hler manifolds.

\setcounter{section}{7}
\setcounter{prop}{0}
\begin{theorem}\label{art5-thm-7.1}
Let $X$ be a K\"ahler 3-fold. Then $TX$ in nef if and only if $X$ is up to finite \'etale cover one of the manifolds in the following list:
 \begin{enumerate}[(i)]
    \item $X = \bbP_{3}$;\label{art5-enum-thm7.1-(i)}
    \item $x = Q_{3}$, the 3-dimensional quadric;\label{art5-enum-thm7.1-(ii)}
    \item $X= \bbP_{1} \times \bbP_{2}$;\label{art5-enum-thm7.1-(iii)}
    \item $X = F_{1,2}$, the flag manifolds of subspaces of $\IC^{3}$;\label{art5-enum-thm7.1-(iv)}
    \item  $X= \bbP_{1} \times \bbP_{1} \times \bbP_{1}$;\label{art5-enum-thm7.1-(v)}
    \item $X = \bbP(E)$, with a numerically flat rank 3-bundle on an elliptic curve $C$;\label{art5-enum-thm7.1-(vi)}
    \item $X= \bbP(E)\times _{C}\bbP(F)$, with $E$, $F$ numerically flat rank $2$-bundles over an elliptic curve $C$;\label{art5-enum-thm7.1-(vii)}
    \item $X= \bbP(E)$, with $E$ a numerically flat rank 2-bundle over a 2-dimensional complex torus;\label{art5-enum-thm7.1-(viii)}
    \item $X= 3$-dimensional complex torus. \label{art5-enum-thm7.1-(ix)}
 \end{enumerate}
\end{theorem}

 The only non-algebraic manifolds appear in classes (\ref{art5-enum-thm7.1-(viii)}) and (\ref{art5-enum-thm7.1-(ix)}) when the Albanese torus is not algebraic. Let us mention that the classification of projective 3-flods with nef tangent bundles was already carried out in \cite{art5-keyCP91} and \cite{art5-keyZh90}. In addition to Theorem
 \ref{art5-main-thm-3.5}, the main ingredient is the classfication of Fano 3-folds by Shokrov and Mori-Mukai. An insepection of the list yields the first classes (\ref{art5-enum-thm7.1-(i)})-(\ref{art5-enum-thm7.1-(v)}) 

 
\begin{thebibliography}{99}
\bibitem[Bi63]{art5-keyBi63} Bishop. R, \textit{A relation between volume, mean curvature and diameter}, Amer. Math. Soc. Not. {\bf 10} (1963) 364.
\bibitem[BG71]{art5-keyBG71} Bloch. S, Gieseker. D, \textit{The positivity of the Chern classed of an ample vector bundle}, Inv. Math. {\bf 12} (1971) 112-117.

\bibitem[Bo74a]{art5-keyBo74a} Bogomolov. F. A, \textit{On the decompsition of K\'ahler manifolds with trivial coanonical class}, Math. USSR Sbornik {\bf 22} (1974) 580-583.

\bibitem[Bo74b]{art5-keyBo74b} Bogomolov. F.A, \textit{K\"ahler manifolds with trivial canonical class}, Izvestija Akad. Nauk {\bf 38} (1974) 11-21 and math. USSR Izvestija {\bf 8} (1974) 9-20.

\bibitem[BPV84]{art5-keyBPV84} Bath. w, Peters. C, Van de ven. A, \textit{Compact Complex Surfaces}, Erg. der Math. 3. Folge, Band 4, Springer, Berlin (1984)

\bibitem[Ca57]{art5-keyCa57} Calabi. E, \textit{On K\"ahler manifolds with vanishing canonical class}, Alg. geometry and topology, Symposium in honor of S. Lefschetz, Princeton Univ. Press, Princeton (1957) 78-89.
\bibitem[CP91]{art5-keyCP91} Cmapana. F, Peternell. Th, \textit{Projective manifolds whose tangent bundles are numerically effective}, Math. Ann.  {\bf  289} (1991) 169-187.
\bibitem[De85]{art5-keyDe85} Demailly. J. P, \textit{Champs mag\'etiques et in \'egaliti\'es de Morse pour la d''-cohomologie}, Ann. Inst. Fourier {\bf 35} (1985) 185-229.
\bibitem[De91]{art5-keyDe91} Demaily. J. P, \textit{Regularization of closed positive currents and intersection theory}, Preprint (1991) 
\bibitem[DPS91]{art5-keyDPS91} Demaily, J. P, Peternell. Th, Schneider M., \textit{Compact complex manifolds with numerically effective tangent bundles}, Preprint (1991) To appear in J. Alg. Geom.
\bibitem[DPS92]{art5-keyDPS92} Demailly. J. P, Peternell. Th, Schneider. M, \textit{K\"ahler manifolds with numerically effective Ricci class}, CoMpositio Math. {\bf 89} (1993) 217-240.
\bibitem[FL83]{art5-keyFL83} Fulton. W, Lazarsfeld. R, \textit{Positive polynomials for ample vector bundles}, Ann. Math.
{\bf 118} (1983) 35-60.
\bibitem[Fu83]{art5-keyFu83} Fujita. T, \textit{Semi-positive line bundles}, J. Fac. Sci. Univ. Tokyo. {\bf 30} (1983) 353-378.
\bibitem[Ga80]{art5-keyGa80} M.E, \textit{Upper bounds for the first eigenvalue of the Laplace- Beltrami operator}, Indiana Univ. J. {\bf 29} (1980) 897-912.
\bibitem[Gr69]{art5-keyGr69} Giffiths. P, \textit{Hermitian differential geometry, Chern classes, and positive vector 
bundles}, In: Global Anlysis, Princeton Math. Series {\bf 29} (1969) 185-251.
 \bibitem[Ha66]{art5-keyHa66} Hartshorne. R, \textit{Ample vector bundles}, Publ. Math. Inst. Hautes Etud. Sci {\bf 29} (1966) 319-394.
 \bibitem[HK78]{art5-keyHK78} Heintze. E, Karcher. H, \textit{A general comparison theorem with applications to volume estimates for submanifolds}, Ann. Scient. Ec.e {\bf 11} (1978) 451-470.
 
 \bibitem[KL74]{art5-keyKL74} Kempf. G, Laksov. D, \textit{The determinantal formula of Schubert calculus}, Acta Math. {\bf 132} (974) 153-162.
 
 \bibitem[Ko61]{art5-keyKo61} Kobayashi. S, \textit{On compact K\"ahler manifolds with positive definite Ricci tensor}, Ann. Math {\bf 74} (1961) 570-574.
 \bibitem[Ko81]{art5-keyKo81} Kobayashi. S, \textit{Recent results in complex differential geometry}, Jahresber. Deutsche Math. Ver. {\bf 83} (1981) 147-158.
 \bibitem[Ko87]{art5-keyKo87} Kobayashi. S, \textit{Differential geometry of complex vector bundles}, Princeton Univ. Press (1987)
\bibitem[Kd66]{art5-keyKd66} Kodaira. K, \textit{On the structure of compact complex surface II}, Amer. J. Math. {\bf 88} (1966) 682-721.
 \bibitem[Kr86]{art5-keyKr86} Koll\'ar. J, \textit{Higher direct images of dualizing sheaves}, Ann. Math {\bf 123} (1986) 11-42.
 
 \bibitem[Li67]{art5-keyLi67}Lichnerowiz. A, \textit{Vari\'et\'es K\"ahl\'eriennes et premi\`ere classe de Chern}, J. Diff. Geom. {\bf 1} (1967) 195-224.
 
 \bibitem[Li71]{art5-keyLi71} Lichnerowicz. A, \textit{Vari\'et\'es K\"ahl\'eriennes \`a premi\`ere classe de Chern non n\'egative et vari\'et'es riemanniennes \`a courbure de Ricci g\'en\'eralis\'ee non n\'egative}, J. Diff. Geom. {\bf 6} (1971) 47-94.
 \bibitem[Li72]{art5-keyLi72} Lichnerowicz. A, \textit{Vari\'et\'es K\"ahl\'eriennes \`a premi\`ere classe de Chern non n\'gative et situation analogue dans le cas riemannien}, Ist. Naz. Alta Mat., Rome, Symposia Math., vol.10, Academic Press, New-York (1972) 3-18.
 \bibitem[Mk88]{art5-keyMk88} Mok. N, \textit{The uniformization theorem for compact K\"ahler manifolds of non negative holomorphic bisectional curvature}, J. Diff. Geom. {\bf 27} (1988) 179-214.
 \bibitem[Mo79]{art5-keyMo79} Mori. S, \textit{Projective manifolds with ample tangent bundles}, Ann. Math {\bf 110} (1979) 593-606.
 \bibitem[My41]{art5-keyMy41} Myers. S. B, \textit{Riemannian manifolds with positive mean curvature}, Duke Math. J. {\bf 8} (1941) 401-404.
 \bibitem[SY80]{art5-keySY80} Siu Y.T, Yau. S. T, \textit{Compact K\"ahler manifolds with positive bisectional curvature}, Inv. Math. {\bf 59} (1980) 189-204.
  \bibitem[Ti72]{art5-keyTi72} Tits. J, \textit{Free subgroups in linear groups}, J. of Algebra {\bf 20} (1972) 250-270.
 \bibitem[UY86]{art5-keyUY86} Uhlenbeck. K, Yau. S. T, \textit{On the existence of Hermitian-Yang-Mills connections in stable vector bundles}, Comm. Pure and Appl. Math {\bf 39} (1986) 258-293.
 \bibitem[Y77]{art5-keyY77} Yau. S.T, \textit{Calabi's Conjecture and some new results in algebraic geometry}, Proc. Nat. Acad. Sci. USA {\bf 74} (1977) 1789-1790.
 \bibitem[Y78]{art5-keyY78} Yau. S.T. \textit{On the Riccie curvature of a complex K\"ahler manifold and the complex Monge-Amp\`e re equation I}, Comm. Pure and Appl. Math. {\bf 31} (1978) 339-411.
 \bibitem[Zh90]{art5-keyZh90} Zheng. F, \textit{On semi-positive threefolds} Thesis, Harvard (1990)
\end{thebibliography}


\medskip
$$
\begin{array}{ll}
\text{Univerit\'e de Grenoble} &  \text{Universita\"at Bayreuth}\\
\text{Institut Fourier, BP 74} &  \text{Mathematisches Institut}\\
\text{U.R.A. 188 du C.N.R.S.} & \text{Postfach 10 12 51}\\
\text{38402 Saint-Martin d'H\`eres, France} & \text{D-8580 Bayreuth, Deutschland}\\
\end{array}
$$
