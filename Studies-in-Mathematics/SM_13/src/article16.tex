\chapter{Geometric Super-rigidity}\label{chap16}

\lhead[\thepage]{\textit{Geometric Super-rigidity}}
\rhead[\textit{Yum-Tong Siu}]{\thepage}

\begin{center}
Yum-Tong Siu\footnote{Partially supported by a grant from the National Science Foundation}
\end{center}

\section*{Introduction}

\setcounter{pageoriginal}{298}
A\pageoriginale more descriptive title for this talk should be: ``The superrigidity of Margulis as a consequence of the nonlinear Matsushima vanishing theorem". What is presented in this talk is the culmination of an investigation in the theoey of geometric superrigidity which Sai-Kee-Yeung and I started about two years ago.

We first used the method of averaging and invariants to obtain Boch\-ner type formulas which yield geometrix superrigidity for the Grassmannians and some other cases. Finally we obtained a general Bochner type formula which includes the usual formulas of Bochner, Kodaira, Matsushima, and Corlette as well as those obtained by averaging so that all cases of geometric superrigidity in its most general form can be derived from such a general Bochner type formula I would like to point out that, for the difficult cases such as those with a Grassmannian of rank at least two as domain and a Riemannian manifold wity nonpositive sectional curvature as target, the formula form the Matsushima vanishing theorem does not yield geometric supperigidity. For those difficult cases one needs the cases of the general Bochner type formula motivated by the method of averaging and invariants. Even with the other simpler cases for which the formula from the Matsushima vanishing theorem\pageoriginale yields geometric superrigidity, to get the result with only the assumption of nonnegative sectional curvature for the target manifold instead of the stronger assumption of nonnegative curvature operator condition, one needs the use of an averaging argument.

\newpage

Geometric superrigidity means the Archimedian case of Margulis's superrigidity \cite{chap16-keyMar} formulated geometrically by assuming the target manifold to be only a Riemannian manifold with nonpositive curvature condition instead of locally symmetric. The complex case of Mostow's strong rigidity theorem \cite{chap16-keyMos} is a consequence of the nonlinear version of Kodaira's vanishing theorem which yields a stronger result requiring only the target manifold to be suitably nonpositively curved rather than locally symmetric \cite{chap16-keySi}. It turns out that in the same way the Archimedian case of Margulis's superrigidity is a consequence of the nonlinear version of Matsushima's vanishing theorem for the first Betti number \cite{chap16-keyMat}. Again the result is stronger in that the target manifold is required only to be suitably nonpositively curved instead of locally symmetric. Moreover, this approach provides a common platform for Margulis's supperrigidity for the case of rank at least two and the recentsupperigiduty result of Corlette \cite{chap16-keyCo} for the hyperbolic spaces of the quaternions and the Cayley numbers. The reason for the such vanishing theorem is the holonomy group which explians why supperigidity works for rank at least two as well as the hyperbolic spaces of quaternions and the Cayley numbers. The curvature $R(X, Y)$ as an element of the Lie algebra of $End(T_{M})$ generates the Lie algebra of the holonomy group. The minimum condition is that the holonomy group is $\calO(n)$ which simply says that $R(X, Y)$ is skew-symmetric. To get a useful vanishing theorem one needs an additional condition to remove a term involving only the curvature of the domain manifold. The K\"ahler case is the same as the holonomy group being $U(n)$. Then $R(X, Y)$ is $\bC$-linear as an element of $End(T_{m})$. This additional condition enables one to obtain the Kodaira vanishing theorem for negative line bundles. Other holonomy groups help yield vanishing theorems for geometric superrgidity. One can also get vanishing theorems for some of the special holonomy groups.

The approach to geometric superrigidity as the nonlinear version of Matsushima's vanishing theorem is motivated by a remark which E.  Calabi made privately to me during the Arbeitstagung of 1981 when I delivered a lecture on the newly discovered approach to the complex case of Mostow's strong rigidity as the nonlinear version of Kodaira's vanishing theorem. Calabi remarked that there is another vanishing theorem, namely Matsushima's which one should look at. He also remarked that Kodaira's vanishing theorem involves the curvature tensor quadratically \cite{chap16-keyCa}. Actually the early rigidity result of $A$. Weil \cite{chap16-keyW} already depends on Calabi's idea of integrating the square of the curvature\pageoriginale \cite[p. 316]{chap16-keyMat} and this early rigidity result launched the theory of strong rigidity and superrigidity. ?`From this point of view it is not surprising that superrdidity cane be approached from Matsushima's vanishing theorem. We state first here the final result we obtained.

\medskip
\noindent
{\bfseries Theorem \thnum{1}. \label{chap16-thm-1}}~\textit{Let $M$ be a compact locally symmetric irreducible Riemanninan manifold of nonpositive curvature whose universal cover is not the real or complex hypebolic space. Let $N$ be a Riemannian manifold whose complexified sectional curvature is nonpositive. If $f$ is a nonconstant harmonic map from $M$ to $N$, then the map from the universal cover of $M$ to that of $N$ induced by $f$ is a totally geodesic isometric embedding.}

Here nonpositive complexified sectional curvature means that
$$
R^{N}(V, W; \overline{V}, \overline{W})\leq 0
$$
for any complexified tangent vectors $V, W$ at any $x \epsilon N$, where $R^{N}$ is the curvature tensor of $N$. In this talk we follow the convention in Matusushima's paper \cite{chap16-keyMat} that $R_{ijij}$ is negative for a negative curvture tensor [\cite{chap16-keyMat}, p. 314, line 6].

\medskip
\noindent
{\bfseries Theorem \thnum{2}. \label{chap16-thm-2}}~\textit{In Theorem \ref{chap16-thm-1} when the rank of $M$ is at least two, one can replace the curvature condition of $N$ by the weaker condition that the Riemananian sectional curvature of $N$ is nonpositive.}

When the universal cover of $M$ is bounded symmetric domain of rank at least two, Theorem \ref{chap16-thm-1} was proved by Mok \cite{chap16-keyMo}. When the universal cover of $M$ is the hyperbolic space of the quaternions and the Cayley numbers, Corlette's result differs from Theorem \ref{chap16-thm-1} only in that Corlette's result requires the stronger curvature condition that the quadratic form $(\xi^{ij}) \mapsto R_{ijkl}^{N}\xi^{ij}\xi^{kl}$ be nonpositive for skew-symmetric $(\xi^{ij})$.

\medskip
\noindent
{\bfseries Theorem \thnum{3}. \label{chap16-thm-3}}~\textit{In Theorems \ref{chap16-thm-1} and \ref{chap16-thm-2} let $X$ be the universal cover of $M$ and $\Gamma$ be the fundamental group of $M$. Then the conclusions of
Theorems\ref{chap16-thm-1} and \ref{chap16-thm-2} remain true when the harmonic map $f$ from $M$ to $N$ is replaced by a $\Gamma$-equivariant harmonic map $f$ from $X$ to $N$.}

\medskip
\noindent
{\bfseries Remark.} With the existence result for equivariant harmonic maps corresponding to the results of Eells-Sampson \cite{chap16-keyE-S}, Theorem \ref{chap16-thm-3} implies the following Archimedian case of the superrigidity theorem of Marugulis \cite{chap16-keyMar}: For lattices $\Gamma$ and $\Gamma'$ extends to a homomorphism from $G$ to $G'$, when $G$ is noncompact simple of rank at least two and $\Gamma$ is cocompact. The general Archimedian\pageoriginale case of the superrigidity theorem of Margulis would follow from the corresponding generalization of Theorem \ref{chap16-thm-3}. In order not to distract from the key points of our arguments, we will not discuss such generalizations in this talk. Also we will focus only on Theorems \ref{chap16-thm-1}and \ref{chap16-thm-2}, because the modifications in the proofs of Theorems
\ref{chap16-thm-1} and \ref{chap16-thm-2} needed to get Theorem \ref{chap16-thm-3} are straightforward. 

\section*{An Earlier Approach of Averaging}
After Corlette \cite{chap16-keyCo} obtained the superrigidity for the case of the hyperbolic spaces of the quaternions and the Cayley numbers, Sai-Kee Yeung and I started to try to undersatand how Corlette's result could be fitted in a more complete global picture of geometric superrigidity. Corelette's method is to generalize the method of the nonlinear $\partial\overline{\partial}$-Bochner formula for the complex strong rigidity by replacing the K\"ahler form used there by the invariant 4-form in the case of the quaternionic hyperbolic space. That 4-form corresponds to the once on the quaternionic projective space whose restriction to a quaternionic line is its standard volume form. Later Gromov \cite{chap16-keyG} introduced the method of foliated harmonic maps so that Corlette's result could be proved by applying the nonlinear $\partial\overline{\partial}$-Bochner formula to the leaves. In his proof of the case of Theorem\ref{chap16-thm-1} when the universal cover of $M$ is a bounded symmetric domain of rank at least two, Mok \cite{chap16-keyMo} remarked that, according to Gromov, one should be able to develop the foliation technique of Gromov \cite{chap16-keyG} to extend Mok's proof to many Riemannian symmetric manifolds of the noncompact type with rank at least two by considering families of totally geodesic Hermitian symmetric submanifolds of rank at least two.

The earlier approach Sai-Kee Yeung and I adopted was motivated by Gromov's work on foliated harmonic maps. We started out by considering a totally geodesic Hermitian suymmetric submanifold $\sigma$ of the universal cover $X$ of $M$. We look at the nonlinear $\partial\overline{\partial}$-Bochner formula applied to the restriction of the Hermitian-symmetric submanifold $\sigma$ and the average over all such submanifolds under the action of the automorphism group of $X$.

More precisely, we let $X$ be the quotient of a Lie group $G$ by a maximum compact subgroup $K$ and let $\Gamma$ be the fundamental group of $M$. Choose a suitable subgroup $H$ of $G$ so that $H/(A\cap K)$ is a bounded symmetric domain of complex dimension at least two. We pull back the harmonic map $f: M \rightarrow N$ to a map $\tilde{f}$ from $G/K$ to $N$ and, for every $k$ is $K$, apply the nonlinear $\partial\overline{\partial}$-Bochner technique developed in [\cite{chap16-keySi}, \cite{chap16-keySa}] to the restriction of $\tilde{f}$ to $k \cdot(H/(H \cap K))$. Since the image of $k\cdot (H /(H \cap K))$ in $\Gamma\backslash X$ is noncompact, one has to introduce a method a averaging over $k$ to handle the step of integration\pageoriginale by parts. As a result of averaging over $k$ the integrand of the gradient square term of the differential of the map $f$ in the formula is an averaged expression of the Hessian of $f$.

The difficult step in this approach is to determine under what condition this averaged expression of the Hessian of $f$ is positive definite in the case of a harmonic map. It turns out that in some cases when we use only one single subgroup $H$ of $G$ this averaging expression in general is not positive definite for harmonic maps. To overcome this difficulty we choose two subgroups $H_{1}$ and $H_{2}$ instead of a single $H$ and we sum the $\partial\overline{\partial}$-Bochner formulas for the two subgroups. For example, this is done in the case of $SO(p,q)/S(O(p) \times O(q))$ for $p > 2$ and $q>2$ $(\nu =1, 2)$ and the sum of the two averaged expressions of the Hessian of $f$ turns out to be positive definite for harmonic maps for this case.

In Cartan's classification of Riemannian symmetric manifolds, besides the ten exceptional ones there are only the following four series which are not Hermitian symmetric:$ SO(p,q)/S(O(p)\times O(q))$,\break $Sp(p,q)/Sp(p)\times Sp(q), sU(k)/SO(k)$, and $SU^{*}(2n)/Sp(n)$. We explicitly verified that for these four series the averaged expression of the Hessian of the Hessian of $f$ is positive definite in the case of a harmonic map so that both Theorem \ref{chap16-thm-1} and Theorem \ref{chap16-thm-2} hold for these four series.  

The method of verification is to use scalar invariants from the representation of compact groups and Cramer's rule. More precisely, let $V$ be a finite-dimensional vector space over $\bbR$ with an inner product $<\cdot, \cdot>$. Let $K$ be a compact subgroup of the special orthogonal group $SO(V)$ with respect to the inner product. Let $S$ be an element of $V^{\oplus 4}$. To compute the average $\int_{g \in K^{g}} \cdot S$, we first enumerate all the one-dimensional $K$-invariant subspaces $\bbR I_{\kappa} (1 \leq \kappa \leq k)$ of $V^{\oplus 4}$ so that $\int_{g \in k^{g}}\cdot S= \sum_{\kappa=1}^{k} c_{\kappa}I_{\kappa}$ for some constants $c_{\kappa}$. By taking the inner product of this equation with $I_{\Lambda}$, we have the system of linear equations $\sum_{\kappa=1}^{k} c_{\kappa}< I_{\kappa},I_{\Lambda}> =<S, I_{\Lambda}>$ from which we can use Cramer's rule to solve for the constants $c_{\kappa}$. 

For such verification it does not matter whether one uses the original Riemannian symmetric space or its compact dual and we will use its compact dual in the following description of the verification.

For the case of $G=SO(p,q)$ and $K=S(O(p)\times O(q))$ for $p > 2$ and $q > 2$ we use the two subgroups $H_{1}=SO(p,2)$ and $H_{2}=SO(2,q)$ of $G$ so that $H_{j}/(H_{j}\cap K)$ is a bounded symmetric domain of rank two. The tangent space of $G/K$ is given by a $p \times q$ matrix and we denote the second partial derivative of the map $f$ with the $(\alpha, \beta)^{th}$ entry and the $(\gamma, \delta)^{th}$ entry by $f_{\alpha \beta, \gamma \delta}$. (Similar notations are also used for the description of the other three seres without further explanation.) Then the avearaged expression $\Phi_{\sigma_{1}}$ of the Hessian of $f$ for\pageoriginale the subgroup $H_{1}=SO(p,2)$ is
\begin{equation*}
\begin{split}
\Phi_{\sigma_{1}}&= \dfrac{1}{(q-1)(q+2)}\left( f_{\alpha \beta, \alpha \beta}f_{\gamma \delta,\gamma \delta}+ \left(1-\dfrac{2}{q}\right)f_{\alpha \beta, \alpha \delta}f_{\gamma \beta,\gamma \delta}\right.\\
&\qquad \qquad \qquad -f_{\alpha \beta, \gamma \beta}
f_{\alpha \delta, \gamma \delta} \left.-f_{\alpha \beta, \gamma \delta}f_{\alpha \beta, \gamma \delta} + \dfrac{2}{q}f_{\alpha \beta, \gamma \delta}f_{\alpha \delta, \gamma \beta},\right)
\end{split}
\end{equation*}
where the summation convention of summing over repeated indices is used. Moreover, $\frac{1}{p(p-1)}\Phi_{\sigma_{1}} + \frac{1}{q(q-1)}\Phi_{\sigma_{1}}$ is positive definite when $\min(p,q)\break \geq 3$. The expression $\Phi_{\sigma_{j}}$ (and also similar expressions lates) is given only up to a positive constant depending on the total measure of th compact group $K$.

For the case of $G=Sp(p,q)$ and $K=Sp(p)\times Sp(q)$, the  totally geodesic Hermitian symmetric submanifold used is
$SU(p + q)/s(U(p) \times U(q))$. The tangent space of $G/K$ is the set of
$\begin{pmatrix}
c & D\\
-\overline{D} & \overline{C}
\end{pmatrix}$. Before we average, we lift the expression with arguments in $\begin{pmatrix}
\overline{C} & D\\
-\overline{D} & \overline{C}
\end{pmatrix}$ to an expression with arguments in a general $(p + q)\times (p \times q)$ matrix $W=(w_{\alpha i})$ so that with the notation $\partial_{\alpha i} \partial_{\overline{\beta j}}f=\dfrac{\partial^{2}f}{\partial w_{\alpha i}\partial\overline{w_{\beta j}}}$ we have the symmetry $\partial_{J(\beta)J(j)}\partial_{\overline{j(\alpha)J(i)}}f=\partial_{\alpha i}\partial_{\overline{\beta j}}f$, where $J(\alpha) = p+ \alpha$ and $J(p +\alpha) = -\alpha$ with $\partial_{(-\alpha)i}$ meaning $-\partial_{\alpha i}$.

The averaged expression of the Hessian of $f$ is
$$
\dfrac{3p}{2}f_{\alpha i \overline{\alpha i}}f_{\gamma j \overline{\gamma j}}-(p +2)|f_{\alpha i \overline{\beta j}}|^{2}-(2p +1)f_{\alpha i \overline{\beta j}}\overline{f_{\alpha(J_{j})\overline{\beta(Ji)}}}
$$
for $q=1$ and is
\begin{gather*}
(p + q + 2pq)f_{\alpha i \overline{\alpha i}}f_{\beta j \overline{\beta j}}-(1 + p)f_{\alpha i \overline{\alpha j}}f_{\beta j \overline{\beta i}}\\
-(1 +q)f_{\alpha i \overline{\beta i}}f_{\beta j \overline{\alpha  j}}-(p + q + 2pq)f_{\alpha i \overline{\beta j}}f_{\beta j \overline{\alpha i}}\\
-(2 + p + q)f_{\alpha i \overline{\beta j}}f_{\beta (Ji)\overline{\alpha(J j)}}  
\end{gather*}
for $q > 1$ and is positive definite when $\min(p,q)\geq 1$ and $\max(p,q)\geq 2$.

For the case of $G=SL(k, \bbR)$ and $K=SO(k)$, we let $n$ be the largest integer with $2n < k$. The totally geodesic Hermitian symmetric submanifold used is $Sp(n)/U(n)$. The tangent space of $G/K$ is the set of all symmetric matrices of order $k$ with zero trace.

The averaged expression of the Hessian of $f$ is equal to
$$
f_{\alpha \beta, \alpha\beta}f_{\lambda \mu, \lambda \mu} - \dfrac{4}{k + 2}f_{\alpha \beta, \gamma \beta}f_{\alpha \mu, \gamma \mu}- f_{\alpha \beta, \gamma \delta}f_{\alpha \beta, \gamma \delta} + \dfrac{4}{k+2}f_{\alpha \beta, \gamma \delta}f_{\alpha \gamma , \beta \gamma}
$$
which is nonnegative for $k \geq 4$.

For the case of $G=SU(2n)$ and $K =Sp(n)$, the Hermitian symmetric submaifold is $SO(2n)/U(n)$. The tangent space of $G/K$ is given by the set of\pageoriginale all $(x, Y)$ of the form $(X, Y)= (A-\overline{D}, B + \overline{C})$ with the $(2n) \times (2n)$ matrix $\begin{pmatrix}
A  & B\\
C & D
\end{pmatrix}$ skew-Hermitian and tracesless. Befor we average, we lift the second derivative of $f$, via the map $Z=(z_{\alpha \overline{\beta}})= \begin{pmatrix}
A  & B\\
C & D
\end{pmatrix}
\mapsto (A- \overline{D}, B + \overline{C})$ to the second jet $f_{\alpha \overline{\beta}\delta \overline{\gamma}} =\partial_{z_{\alpha \overline{\beta}}} \partial_{z_{\gamma \overline{\delta}}}f$ on the Lie algebra of $SU(2n)$ so that the symmetries $f_{\alpha \overline{\beta}\delta\overline{\gamma}} = -f_{\beta \overline{\alpha}\delta \overline{\gamma}}$ and $f_{J(\beta)\overline{J(\alpha)}\delta\overline{\gamma}} =f_{\alpha \overline{\beta}\delta \overline{\gamma}}$ hold, where as earlier $J (\alpha) =n + \alpha$ and $J(n + \alpha)= -\alpha$. The averaged expression of the Hessian of $f$ is
$$
f_{\alpha \overline{\beta}\beta\overline{\alpha}}f_{\gamma \overline{\delta}\delta \overline{\gamma}} - \dfrac{2}{n-1}f_{\alpha\overline{\beta}\gamma \overline{\alpha}}f_{\beta\overline{\delta}\delta \overline{\gamma}} - f_{\alpha \overline{\beta}\gamma \overline{\delta}}f_{\beta \overline{\alpha}\delta \overline{\gamma}} -\dfrac{2}{n-1} f_{\alpha \overline{\beta}\gamma \overline{\delta}}f_{\delta \overline{\alpha}\beta \overline{\gamma}}
$$
which is nonnegative for $n \geq 3$.

In the above approach by averaging, the natural curvature condition for the target manifold is the nonnegativity of  the complexified sectional curvature. One can also consider the curvature term obtained by averaging and argue by the number of invariants that the target manifold needs only to satisfy the weaker condition of the nonnegativity of the sectional curvature in the case of the domain manifold of rank at least two. Mok came up with the the idea that to get directly the weaker condition of nonnegative sectional curvature for the target manifold, one can restrict the harmonic map to totally geodesic flat submanifolds of the domain manifold and average the usual nonlinear Bochner formula there instead of the nonlinear $\partial \overline{\partial}$-Bochner formula.

Though this averaging method theoretically can also be applied to the ten exceptional cases of Riemannian symmetric manifolds which are not Hermitian symmetric, explicit computation becomes cumbersome for them. We then changed our approach and used instead the nonlinear Matsushima vanishing theorem in our investigations of the ten exceptional cases. The use of the nonlinear Matsushima vanishing theorem in the exceptional cases is the most natural approach. In the course of our investigation involving both the Bochner type formula from averaging and those from the Matsushima vanishing theorem we came to a much better understanding of the nature of such vanishing theorems. We could formulate such vanishing theorems in a general setting. The most general case of geometric superrigidity is then a consequence of such a general nonlinear vanishing theorem. Both the $\partial \overline{\partial}$-Bochner vanishing theorem and the Matsushim vanishing theorem are special cases of teh vanishing theorem for the general setting. It also gives a very short and elegant proof of the original Matsushima vanishing theorem.

\section*{Matsushima's Vanishing Theorem}
\pageoriginale

Matsushima's theorem states that the first Betti number of a compact complex manifold is zero if its universal cover is an irreducible bounded symmetric domain of rank at least two.

One step of Matsushima's original proof is the verification of the positivity of a certain quadratic form
$$
(\xi^{ij}) \mapsto b(\fg) \sum\limits_{i,j}(\xi^{ij})^{2} + \sum\limits_{i,j,k,l} R_{ikhj}\xi^{ij}\xi^{kh},
$$
where $b(\fg)$ is a constant depending on and explicitly computable from  the Lie algebra $g$ of the Hermitian symmetric manifold and $R_{ikhj}$ is the curvature tensor of the Hermitian symmetric manifold. The verification makes use of the computations by Calabi-Vesentini and Borel on the eigenvalues of the quadratic form given by the curvature tensor acting on the symmetric 2-tensors of a Hermitian symmetric manifold.

Mostow's strong rigidity theorem (for the case of simple groups) says that if $G$ and $G''$ are noncompact simple groups not equal to $PSL\break (2, \bbR)$ and $\Gamma \subset G$ and $\Gamma' \subset G'$ are lattices, then any isomorphism can be extended to an isomorphishm from $G$ to $G'$. For the case of bounded symmetric domains and cocompact lattices we can state it as follows. Let $D$ and $D'$ be irreducible bounded symmetric domains of complex dimension at least two and let $M$ and $M'$ be respectively smooth compact quotients of $D$ and $D'$. If $M$ and $M'$ are of the same homotopy type, then $M$ and $M'$ are biholomorphic (or anti-biholomorphic).

The vanishing theorem of Kodaira for a negative line bundle $L$ over a compact K\"ahler manifold $M$ of complex dimension $n \geq 2$ can be proved as follows. We do it for the vanishing of $H^{1}(M, L)$ becaucse that is the case we need. Suppose $\xi$ is an $L$-valued harmonic $(0,1)$-form on $M$. Let $\omega$ be a K\"ahler form of $M$. Then
$$
0=\int_{M}\sqrt{-1}\partial \overline{\partial}(\sqrt{-1}\xi \wedge \overline{\xi})\wedge \omega^{n-2} = ||D\xi||^{2}- \int_{M}(R_{L}\xi, \xi)
$$
implies that $\xi$ vanishes, where $R_{L}$ is the curvature of $L$.

The nonlinear version of Kodaira's vanishing theorem is as follows. Let $M$ and $N$ be compact K\"ahler manifolds and $f: M \rightarrow N$ be a harmonic map which is a homotopy equivalence. Use $\overline{\partial}f$ instead of $\xi$. We get
\begin{equation*}
\begin{split}
0 &=\int_{M}\sqrt{-1}\partial \overline{\partial}(\sqrt{-1}h_{\alpha \overline{\beta}}\overline{\beta}f^{\alpha}\wedge \partial\overline{f^{\beta}})\wedge \omega^{n-2} = ||D\overline{\partial}f||^{2}\\
&\qquad \qquad \qquad \qquad -\int_{M}R^{N}(\partial f, \overline{\partial}f, \partial f, \overline{\partial}f).
\end{split}
\end{equation*}
Suitable\pageoriginale nonpositive curvature property of $N$ implies that either $\partial f$ or $\overline{\partial f}$ vanishes. Such a curvature property is satisfied by irreducible bounded symmetric domains of complex dimension at least two. This nonlinear version implies the complex case pf Mostow's strong rigidity theorem,because the theorem of Eells-Sampson implies the existence of a harmonic map in the homotopy class of continuous maps from a compact Riemannian manifold to a nonpositively curved Riemannian manifold. Moreover, the target manifold is assumed to satisfy only a curvature condtion instead of being locally symmetric.

The complex case of strong rigidity corresponds to the vanishing of the first cohomology wiht coefficient in a coherent analytic sheaf. The real analog corresponds to the vanishing of the first cohomology with coefficient in the constant sheaf. So we should look at the vanishing of the first Betti number. On the other hand holomorphic means $\overline{\partial} = 0$. Its real analog should mean $d=0$ which means parallelism. The pullback of the metric tensor being parallel means isometry after renormalization. This consideration gives the motivation that the nonlinear version of Matsushima's vanishing theorem for the first Betti number would yield the Archimedian case of Margulis's superrigidity theorem with the assumption on the target manifold weakened from local symmetry to suitable nonpositive curvature.

The reason for geometric superrigidity turns out be the holonomy group. the curvature $R(X, Y)$ as an element of the Lie algebra of $\End(T)$ generates the Lie algebra of the holonomy group. The minimum condition is $O(n)$ which simply says that $R(X, Y)$ is skew-symmetric. The K\"ahler case is the same as the holonomy group being $U(n)$. Then $R(X, Y)$ is $\bC$-linear as an element of $End(T)$. It is the same as saying that $R_{\alpha \beta ij}=0$ for $1 \leq alpha, \beta \leq n$ and $i ,j$ running through $1, \cdots, n$ and $\overline{1}, \cdots, \overline{n}$. The condition is equivalent to $R_{\alpha \overline{\beta}\gamma\overline{\delta}}$ being symmetric in $\alpha$ and $\gamma$ by the Bianchi identity
$$
R_{\alpha  \overline{\beta} \gamma \overline{\delta}} + R_{\alpha \gamma \overline{\delta}\beta} + R_{\alpha \overline{\delta}\beta \gamma} = 0.
$$

\section*{Vahishing Theorems from 4-Tensors}

A vanishing theorem is the result of a 4-tensor $Q$ satisfying the following conditions. This 4-tensor $Q_{ijkl}$ should be skew-symmetric in $i$ and $j$ and symmetric in $(i,j)$ and $(k, l)$. Moreover, the following three conditions should be satisfied:
\begin{enumerate}[(i)]
\item The quadratic form $\sum_{i,j,k,l} Q_{ijkl}\xi^{il} \xi^{jk}$ is positive definite on all traceless $\xi^{ij}$.\label{chap16-enum-i}

\item $< A(\cdot,\cdot,\cdot, X), R(\cdot,\cdot,\cdot, Y) > = 0$ for all $X, Y$.\label{chap16-enum-ii}

\item $Q$\pageoriginale is parallel.\label{chap16-enum-iii}
\end{enumerate}

Once one has such a 4-tensor $Q$, one applies integration by parts to
$$
\int_{M}Q_{ijkl}\nabla_{i}f_{l} \nabla_{j}f_{k}
$$
for any harmonic $f$ to show that $f$ is zero. Here $\nabla$ denotes covariant differentiation. We can do this for the linear as well as the nonlinear version of the vanishing theorem. As and example let us look at Kodaira's vanishing theorem. The 4-tensor is
$$
Q_{\alpha \overline{\beta} \gamma, \overline{\delta}}=\delta_{\alpha \delta}\delta_{\beta \gamma}-\delta_{\alpha \beta}\delta_{\gamma \delta}.
$$
Note that this $Q$ is simply the curvature tensor for the manifold of consatant holomorphic curvature with the sign of the second term reversed. Then
$$
Q_{ijkl}\xi^{il}\xi^{jk} = Q_{\alpha \overline{\beta}\gamma \overline{\delta}}\xi^{\alpha \overline{\delta}} \overline{\xi^{\beta_{\overline{\gamma}}}} = \sum\limits_{\alpha, \delta} \xi^{\alpha \overline{\delta}}\overline{\xi^{\alpha \overline{\delta}}} - \left(\sum\limits_{\alpha}\xi^{\alpha \overline{\alpha}}\right)\left(\overline{\sum\limits_{\beta}\xi^{\beta \overline{\beta}} }\right) = \sum\limits_{\alpha, \delta} \xi^{\alpha \overline{\delta}}\overline{\xi^{\alpha} \overline{\delta}}
$$
is positive definite. Moreover,
$$
Q_{ijkl}R_{ijkh} =Q_{\alpha \overline{\beta}\gamma \overline{\delta}}R_{\alpha \overline{\beta}\gamma \overline{h}}
= R_{\beta \overline{\beta}\delta h}-R_{\delta \overline{\beta}\beta h} = 0.
$$
In the case of a harmonic 1-form $f$, the formula is simply
\begin{align*}
\int_{M}Q_{ijkl}\nabla_{i}f_{l}\nabla_{j}f_{k} &= - \int_{M}Q_{ijkl}f_{l}\nabla_{i}\nabla_{j}f_{k}\\
&= -\dfrac{1}{2} \int_{M}Q_{ijkl}f_{l}\left[\nabla_{i}, \nabla_{j}\right]f_{k}\\
&= -\dfrac{1}{2} \int_{M}Q_{ijkl}f_{l}R_{ijkh}f_{h} = 0.
\end{align*}
Note that this gives a proof of Matsushima's vanishing theorem when we consider a harmonic form $f_{i}$, because the conditions on $Q$ imply that $f_{i}$ is parallel and there is no nonzero parallel 1-form otherwise there is a de$\bR$-ham decomposition of the universal cover. In the case of a compact K\"a hler manifold (without using any line bundle or any map) applied to a harmonic $(1,0)$-form $f_{\alpha}$ the formula gives $\partial_{\overline{\beta}}f_{\alpha} =0$ for all $\alpha$ and $\beta$, which is the same as saying that any harmonic $(1,0)$-form on a compact K\"ahler manifold is holomorphic. When this is applied to a harmonic 1-form with values in a line bundle, we have another term in the formula represented by the curvature of the line bundle.

Suppose\pageoriginale the holonomy group is not $U(n)$. Then Berger's theorem forces then manifold to be locally symmetric except for the so-called exceptional holonomy groups. Assume that we have a compact locally symmetric manifold. Let $K_{0}$ be the curvature tensor of constant curvature 1 given by
$$
(K_{0})_{ijkl}=\delta_{ik}\delta_{jl}-\delta_{il}\delta_{jk}.
$$
We are going to use $Q=c_{0}K_{0} + R$ for some suitable constant $c_{0}$. The condition $Q_{ijkl}R_{ijkh}=0$ simply says that $-2 c_{0}R_{ljjh} + R_{ijkl}R_{ijkh}=0$. The factor $R_{ljjh}$ in the first term is simply equal to the negative of the Ricci curvature $(Ric)_{lh}$ according to the convention in Matsushima's paper \cite{chap16-keyMat}. The second term $R_{ijkh}R_{ijkh}$ is a symmetric 2-tensor which is parallel. Now every parallel symmetric 2-tensor is a constant multiple of the Kronecker delta, otherwise any proper eigenspace at a point would give rise to a de$\bR$ham decomposition of the manifold. So we know that $c_{0}$ exists. We can determine the actual value of $c_{0}$ by contracting the indices $h$ and $l$. We get $c_{0}= -< R, R > /< R, K_{0} >$. Consider now the integration by parts of
$$
\int_{M}(c_{0}K_{0}+ R)_{ijkl}\nabla_{i}f_{l}\nabla_{j}f_{k}.
$$
The question now is the positive definiteness of the quadratic form
$$
\xi \mapsto (c_{0}K_{0} + R)_{ijkl}\xi^{il}\xi^{jk}
$$
on traceless $\xi$, which is the same as
\begin{equation}
\xi \mapsto c_{0} \sum\limits_{i,l}(\xi^{il})^{2} + \sum\limits_{i,j,k,l} R_{ijkl}\xi^{il}\xi^{jk}. \tag{$\ast$}\label{chap16-eq-*}
\end{equation}
We now look at the nonlinear version. From $\left[\nabla_{i}, \nabla_{k} \right]f_{l}$ we get an expression involving the curvature tensor of the target manifold. So
\begin{gather*}
\int_{M}Q_{ijkl}\nabla_{i}f_{l} \nabla_{j}f_{k}= \int_{M}Q_{ijkl}f_{l}\nabla_{i}\nabla_{j}f_{k}\\
= \dfrac{1}{2}\int_{M}Q_{ijkl}f_{l}^{D} R_{ABCD}^{N}f_{i}^{A}f_{j}^{B}f_{k}^{C}.
\end{gather*}
To simplify notations we write $(f^{*}R^{N})_{ijkl} =R_{ABCD}^{N}f_{i}^{A}f_{i}^{B}f_{k}^{C}f_{l}^{D}$. So our final formula is
$$
\int_{M}(c_{0}K_{0} + R)_{ijkl} \nabla_{i}f_{l}\nabla_{j}f_{k} = \dfrac{1}{2}\int_{M} < c_{0}K_{0} + R, f^{*}R^{N} >
$$
It is simple and straightforward to verify that $c_{0} \geq b(\fg)$. From the work of Kaneyuki-Nagano \cite{chap16-keyK-N} we can conclude that the quadratic form \eqref{chap16-eq-*} is positive\pageoriginale definite.

\smallskip
\noindent
{\bf The Term Involving the Curvature of the Target Manifold.} We have to worry about the sign of the term involving the curvature of the target manifold $\int_{M} < c_{0}K_{0} + R, f^{*}R^{N}>$. We have to determine conditions on $R^{N}$ so that this term is nonpositive. We need only consider pointwise nonpositivity. Fix a point $P_{0}$ of the domain manifold $M$. Let $\calC$ be  the vector space of all 4-tensors $T_{ijkl}$ which satisfies the following three symmetry conditions:
(1)$T_{ijkl}= -T_{jikl}$, (2) $T_{ijkl}= T_{klij}$, and (3) $T_{ijkl} + T_{iklj} + T_{iljk} =0$. In other words, $\calC$ is the vector space of all 4-tensors of curvature type. Let $H$ denote the isotropy subgroup at that point. From the known results on the decomposition into irreducible representations of the representation of $H$ on the skew-symmetric 2-tensors, we know that there are two, three, or four independent linear scalar $H$-invariants for elements of $\calC$.

Consider first the case when there are only two independent linear scalar $H$-invariants given by inner products with the $H$-invariant elements $I_{ijkl}$ and $I_{ijkl}'$ of $\bC$ so that $I =K_{0}$ and $ < I, I' > =0$. In our argument we can use either the complexified sectional curvature or the usual Riemannian sectional curvature (or even the analogously defined quaternionic or Cayley number sectional curvature). The arguments are strictly analogous. Let us assume that the rank of the domain manifold is at least two and consider the case of the usual Riemannian sectional curvature. Fix any 2-plane $E$ in the tangent space of $M$ at $P_{0}$ so that the Riemannian sectional curvature $\Sect(R, E)$ of $R$ for $E$ is zero. Consider the following expression $\int_{g \in H} \Sect(f^{*}R^{N}, g \cdot E)$. This expression is equal to a
$\left(< f^{*}R^{N}, I > + a'  < f^{*}R^{N}, I' >\right)$ for some real constants $a$ and $a'$ depending on $E$. On the other hand, the integrand $< c_{0}K_{0} + R, f^{*}R >$ is of the form $b\left(< f^{*}R^{N}, I > + b' < f^{*}R^{N}, I' >\right)$ for some rea 1 constants $b$ and $b'$. Since both expressions vanish for $f$ equal to the identity map, we conclude that $b' =a'$. To compute $a$ and $a'$,  we use $K_{0}$ as the test value to replace $f^{*}R^{N}$. The value $b$ is given by
$$
b < K_{0}, K_{0} > =c_{0}< K_{0}, K_{0} > + < R, K_{0} >
$$
and the value of $a$ is given by
$$
a < K_{0}, K_{0} > = \Sect(K_{0}, E).
$$
Since $c_{0}=-<R, R>/<R, K_{0}>$ it follows from Schwarz's inequality and the nonpositivity of $< R, K_{0}>$ that $b$ is nonnegative. From $\Sect(K_{0}, E) = 1$ we conclude that $(c_{0}K_{0} + R)_{ijkl}(f^{*}R^{N})_{ijkl}$ is equal to
\begin{equation*}
\begin{split}
&-< R, K_{0}>^{-1} (< K_{0}, K_{0}><R, R>-< R, K_{0}>^{2})\\
&\qquad \qquad \qquad \int_{g \in K} \Sect(f^{*}R^{N}, g\cdot E).
\end{split}
\end{equation*}
We\pageoriginale have thus the final formula
\begin{gather*}
\int_{M}(c_{0}K_{0}+ R)_{ijkl}\nabla_{i}f_{l}\nabla_{j}f_{k}\\
=-< R, K_{0}>^{-1}(<K_{0}, K_{0}><R, R>-<R, K_{0}>^{2})\\
\int_{P \in M}\left( \int_{g \in H_{P}} f^{*}R^{N}, g \cdot E_{P}\right),
\end{gather*}
where $E_{P}$ is a 2-plane in the tangent space of $M$ at $P$ at which tthe Riemmanian sectional curvature of $M$ is zero and $H_{P}$ is the isotropy group at $P$. So we have the geometric superrigidity result that  any harmonic map from such a compact locally symmetric manifold to a Riemannian manifold with nonpositive Riemannian sectional curvature is a totally geodesic isometric embedding.

The case of three or four independent linear scalar $H$-invariants occurs only in the case of Hermitian or quaternionic symmetric spaces or the case of Grassamanians. Let us illustrate the technique by looking at the Hermitian symmetric case. Let $K_{\bC}$ denote the curvature tensor of constant holomorphic sectional curvature. Instead of using $Q=c_{0}K_{0} + R$, one uses $Q=\lambda(K_{0}-K_{\bC}) + \mu (c_{0}K_{\bC} + R)$ for some suitable constants. This method of using a suitable linear combination is parallel to the choice of the suitable constants $\frac{1}{p(p-1)}, \frac{1}{q(q-1)}$ in the expression $\frac{1}{p(p-1)}\Phi_{\sigma_{1}} + \frac{1}{q(q-1)}\Phi_{\sigma_{2}}$ in the earlier approach of averaging.

Details of the methods and results described above will be in a paper to appear elsewhere.

\begin{thebibliography}{99}
\bibitem[B]{chap16-keyB} A. Borel, \textit{On the curvature tensor of the hermitian symmetric manifolds}, Ann. of Math.
{\bf 71} (1960) 508--521.

\bibitem[Ca]{chap16-keyCa} E. Calabi, \textit{Matsushima's theorem in Riemannian and K\"ahler geometry}, Notices of the Amer. Math. Soc. {\bf 10} (1963) 505.

\bibitem[C-V]{chap16-keyC-V} E. Calabi and E. Vesentini, \textit{On compact locally symmetric K\"ahler manifolds}, Ann. of Math {\bf 71} (1960) 472--507.

\bibitem[Co]{chap16-keyCo} K. Corlette, \textit{Archimedean superrigidity and hyperbolic geometry}, Ann. of Math. {\bf 135} (1992) 165--182.

\bibitem[E-S]{chap16-keyE-S} J. Eells and J. Sampson, \textit{Harmonic maps of Riemannian manifolds}, Amer. J. Math. {\bf 86} (1964) 109--160.

\bibitem[G]{chap16-keyG}M. Gromov,\pageoriginale \textit{Foliated Plateau problem I II}, Geom. and Funct. Analy. {\bf 1} (1991) 14-79 and 253--320.

\bibitem[K-N]{chap16-keyK-N} S. Kaneyuki and T. Nagano, it On certain quadratic forms related to symmetric Riemannian spaces, Osaka Math. J {\bf 14} (1962) 241--252.

\bibitem[Mag]{chap16-keyMag}G.A. Margulis, \textit{Discrete groups of motoins of manifolds of nonpositive curvature}, A. M.S. Transl. (1), {\bf 109} (1977) 33--45.

\bibitem[Mat]{chap16-keyMat} Y. Matsushima, \textit{On the first Betti number of compact quotient spaces of higher-dimensinal summetric spaces}, Ann. of Math. {\bf 75} (1962) 312--330.

\bibitem[Mo]{chap16-keyMo}N. M\'o\`k, \textit{Aspects of K\"ahler geometry on arithmetic varieties}, Proceedings of the 1989 AMS Summer School on Several Complex Variables in Santa Cruz.

\bibitem[Mos]{chap16-keyMos} G. D. Motow, \textit{Strong Rigidity of Locally symmetric Spaces}, Ann. Math. Studies 78, Princeton Univ. Press, (1973)

\bibitem[Sa]{chap16-keySa} J. Sampson, Applications of harmonic maps to K\"ahler geometry, Contemporary Mathematics,
Vol. {\bf 49} (1986) 125-133.

\bibitem[Si]{chap16-keySi}Y.-T.Siu \textit{Complex-analyticity of harmonic maps, vanishing and Lefshetz theorems}, J. Diff. Geom. {\bf 17} (1982) 55--138.

\bibitem[W]{chap16-keyW}A. Weil, \textit{On discrete subgroups of Lie groups II}, Ann. of Math. {\bf 75} (1962) 578--602.
\end{thebibliography}

\bigskip

\begin{flushleft}
Department of Mathematics

Harvard University

Cambridge, MA 02138

U.S.A.
\end{flushleft}
