\title{A Note on Automorphism Groups of Algebraic Varieties \footnote{The author wishes to express his gratitude to Professors M. S. Narasimhan and C.S. Seshadri for many helpful suggestions and discussions.}}\label{art6}
\markright{A Note on Automorphism Groups of Algebraic Varieties}


\author{By~ C.P. Ramanujam in Bombay (India)}
\markboth{C.P. Ramanujam}{A Note on Automorphism Groups of Algebraic Varieties}

\date{}
\maketitle

\setcounter{page}{73}
\setcounter{pageoriginal}{54}
Matsusaka\pageoriginale has proved \cite{art6-key3} that the maximal connected group of automorphisms of a projective variety can be endowed with the structure of an algebraic group. Our aim in this note is to extend this result to arbitrary complete varieties. More generally, we shall show that a ``connected'' and ``finite dimensional'' group $G$ of automorphisms (see below for precise definitions) of any algebraic variety $X$ can be endowed with the structure of an algebraic group variety. The main line of argument is similar to the one used by Chevalley \cite{art6-key2} and Seshadri \cite{art6-key7} in the construction of the Picard variety, but somewhat simpler. We shall prove that the linear map of the Lie algebra of this algebraic group  into the space of vector fields on $X$ which associates to any tangent vector at the identity element of $G$ the corresponding ``infinitesimal motion'' is an injection. It follows easily that $G$ satisfies the universal property for connected algebraic families of automorphisms of $X$ containing the identity, that is, that any algebraic family of automorphisms of an algebraic variety $X$ parametrised by a variety $T$ is induced by a morphism of $T$ into $G$. As an application, we shall prove that the maximal connected group of automorphisms of a (locally isotrivial) principal fibre space over a complete variety has a structure of a group variety. We have been informed by the referee that this result has also been obtained by H. Matsumura.

All varieties will be assumed to be irreducible, and defined over an algebraically closed field $K$.

We shall say that a family $\{\varphi_t\}_{t\in T}$ of automorphisms of a variety $X$, where the parametrising set $T$ is also a variety, is an {\em algebraic family } if the map $T \times X \to X$ given by $(t,x) \xrightarrow{\varphi} \varphi_t (x)$ is a morphism. It is clear that if $\lambda : S \to T$ is a morphism, the family $\{\varphi_{\lambda(s)} \}_{s\in S}$ is again algebraic, and that if $\{\psi_s\}_{s \in S}$ is another algebraic family, $\{\psi_s \circ \varphi_t \}_{(s,t) \in S \times T}$ is also an algebraic family. Let $(\tilde{X},p)$ be the normalisation of $X$. For every $t \in T$, $\varphi_t$ lifts to a unique automorphism $\varphi_t$ of $\tilde{X}$ such that $p \circ \tilde{\varphi}_t = \varphi_t \circ p$. We shall show that $\{\tilde{\varphi_t} \}_{t \in T}$ is an algebraic family of automorphisms of $\tilde{X}$. Let $(\tilde{T}, q)$ be the normalisation of $T$. Then $(\tilde{T} \times \tilde{X}, q \times p)$ is the normalisation of $T \times X$, and the morphism $T \times X \xrightarrow{\varphi} X$ lifts to a unique morphism $\bar{\varphi}: \tilde{T} \times \tilde{X} \to \tilde{X}$ such that $p\circ \bar{\varphi} = \varphi \circ (q \times p)$. It follows that for any $\tilde{t} \in \tilde{T}$, the morphism of $\tilde{X}$ onto itself given\pageoriginale by $\tilde{x} \to \bar{\varphi} (\tilde{t}, \tilde{x})$ coincides with $\tilde{\varphi}_q (\tilde{t})$. Let $\Gamma_{\bar{\varphi}}$ be the graph of $\bar{\varphi}$ in $\tilde{T} \times \tilde{X} \times \tilde{X}$, and 
$\Gamma_{\tilde{\varphi}}$ its image in $T \times \tilde{X} \times \tilde{X}$ by $q \times I_{\tilde{X}} \times I_{\tilde{X}}$ denoting the identity map of $\tilde{X}$. $\Gamma_{\tilde{\varphi}}$ is then the graph of the map $T \times \tilde{X} \xrightarrow{\tilde{\varphi}} \tilde{X}$ given by $(t, \tilde{x}) \to \tilde{\varphi}_t (x)$ and since $q$ is proper, $\Gamma_{\bar{\varphi}}$ is closed. Since the projection of $\Gamma_{\tilde{\varphi}}$ onto the product $\tilde{T} \times \tilde{X}$ of the first two factors is an isomorphism and $q \times  I_{\tilde{X}}: \tilde{T} \times \tilde{X} \to T \times \tilde{X}$ is proper, it follows that the projection of $\Gamma_{\tilde{\varphi}}$ onto $T \times \tilde{X}$ is a morphism which is proper and bijective.

Let $p_1$ and $p_2$ be the projections of $\Gamma_{\tilde{\varphi}}$ onto the second and third factors. It is easily checked that if $\tau$ is a tangent vector at a point $(t, \tilde{x}, \tilde{\varphi}_t (\tilde{x}))$ to $\Gamma_{\bar{\varphi}}$ whose image by the differential map of the projection onto $T$ is zero, and if $dp_i$ and $d\varphi_t$ are the differential maps of $p_i$ and $\varphi_t$, we have $dp_2 (\tau) =d\tilde{\varphi}_t (d p_1 (\tau))$. It follows that the differential map of the projection of $\Gamma_{\tilde{\varphi}}$ onto $T\times\tilde{X}$ is everywhere injective. It follows from [5, Appendix, Expos\'e 5, p. 5-28] that $\Gamma_{\tilde{\varphi}} \to T \times \tilde{X}$ is an isomorphism everywhere, which shows that $\tilde{\varphi}$ is a morphism and $\{ \tilde{\varphi}_t\}_{t \in T}$ is an algebraic family.

A similar argument proves that the family $\{\varphi^{-1}_t\}_{t \in T}$ of inverses of an algebraic family is again algebraic. Since the set of points of a parametrising variety for which the corresponding elements of an algebraic family become the identity automorphism is a closed set, it follows that if $\{\varphi_t\}_{t\in T}$ and $\{\psi_s\}_{s \in S}$ are two algebraic families, the set of $(s,t) \in S \times T$ for which $\varphi_t = \psi_s$ is closed. 

We shall say that a group $G$ of automorphisms of a variety $X$ is a {\em connected group of automorphisms} if any automorphism belonging to $G$ is a member of an algebraic family which also contains the identity automorphism of $X$. We shall say that $G$ is {\em finite dimensional} if there exists an integer $N$ such that if $\{\varphi_t\}_{t \in T}$ is any algebraic family of automorphisms contained in $G$ (i.e. $\varphi_t \in G$ for every $t \in T$) and such that $\varphi_t \neq \varphi_{t'}$ if $t \neq t'$ {\em (an injective family),} we have $\dim T \leq N$. The smallest integer $N$ having this property is then called the {\em dimension} of $G$.

We need a final definition. Let $\{\varphi_t\}_{t\in T}$ be an algebraic family of automorphisms of $X$ and $\varphi: T \times X \to X$ the defining morphism. Let $\tau$ be a tangent vector to $T$ at a point $t_0$, and for any $x$, let $\tau_x$ be the tangent vector to $T \times X$ at $(t_0, x)$ which is the image of $\tau$ by the differential mapping of the morphism $\theta_x : T \to T \times X$ given by $\theta_x (t) = (t,x)$. We can then define a vector filed $d \varphi (\tau)$ on $X$ whose value $d \varphi (\tau)_x$ at the point $x$ is given by $d \varphi (\tau)_x = d \varphi (\tau_{\varphi_{t_0}} - 1_x)$. The vector field $d \varphi (\tau)$ is immediately verified to be regular (i.e., maps regular functions on open subsets of $X$ into regular functions). We shall say that the family $\{\varphi_t\}_{t \in T}$ is {\em infinitesimally injective } at a point $t_0 \in T$ if $d \varphi$ is an injection of the tangent space at $t_0$ into the vector space of regular vector fields on $X$. 

We now state our result.

\begin{theorem*}
Let $G$ be a connected finite dimensional group of automorphisms of an algebraic variety $X$. Then there exists a unique structure of an algebraic variety on $G$ which makes of it an algebraic group (of dimension = dimension of $G$) such that the following condition holds:
\begin{itemize}
\item[a.] The automorphisms of $X$ belonging to $G$, when considered as a family of automorphisms of $X$ parametrised by the identity map of $G$ onto $G$, is an algebraic family\pageoriginale which is infinitesimally injective on $G$. In other words, if $\chi: G \times X \to X$ is defined by $\chi (\varphi, x) = \varphi (x)$, $\chi$ is a morphism of algebraic varieties, and if $\tau \neq 0$ is a tangent vector at any point of $G$, $d \varphi (\tau) \neq 0$.

Further, $G$ has the following universal property:

\item[b.] If $\{\varphi_t\}_{t \in T}$ is any algebraic family of automorphisms of $X$ such that  $\varphi_t \in G$ for every $t \in T$, there is a unique morphism $\bar{\varphi}: T \to G$ such that $\bar{\varphi} (t) = \varphi_t$.
\end{itemize}
\end{theorem*}

We will first show that (a) implies (b), and this in turn trivially implies the uniqueness assertion of the theorem. By an earlier remark, the subset $\Gamma$ of $T \times G$ consisting of those $(t, \varphi)$ for which $\varphi_t = \varphi$ is a closed subset of $T \times G$. Since this is precisely the graph of $\bar{\varphi}$, we have only to verity that the projection is an isomorphism. We shall prove that the differential map of the projection of $\Gamma$ onto $T$ is injective at every point of $\Gamma$. Let $p_1$ and $p_2$ be the projections of $\Gamma$ onto $T$ and $G$ respectively. Let $\{\psi_\gamma\}_{\gamma \in \Gamma}$ be the algebraic family on $X$ defined by $\psi_\gamma = \varphi_{p_1 (\gamma)} = p_2 (\gamma)$. If $\tau$ is any tangent vector at a point $\gamma_0$ to $\Gamma$, we have evidently $d \psi (\tau) = d \psi (d p_1 (\tau)) = d \chi (d p_2 (\tau))$. It follows by (a) that if $d p_1 (\tau) =0$, then $dp_2 (\tau) = 0$ and hence $\tau= 0$.

By Lemmas \ref{art6-lem1} and \ref{art6-lem2} below and Zariski's main theorem, we can find $y_1, \ldots, y_m \in X$ such that the morphism $\lambda: G \to X^m$ defined by $\lambda (g) = (g y_1, \ldots, g y_m)$ is a radicial covering of the non-singular locally closed subvariety $W = \lambda (G)$ of $X^m$ by $G$. Since $\mu: T \to W$ given by $\mu(t) = (\varphi_t (y_1), \ldots, \varphi_t (y_m))$ is a morphism, its graph $\Gamma_\mu \subset T \times W$ is closed irreducible. Since $I_T \times \lambda : T \times G \to T \times W$ is a radicial covering and $(I_T \times \lambda)^{-1} (\Gamma_\mu) = \Gamma$, $\Gamma$ is irreducible and $\Gamma \to T$ is proper. We may now apply the result of [5, Appendix to Expos\'e 5, p. 5-28] to conclude that $\Gamma \xrightarrow{p_1} T$ is an isomorphism, and $\bar{\varphi} = p_2 \circ p^{-1}_1$ is the required morphism.

Hence we have only to construct a structure of variety on $G$ satisfying (a). 

\begin{lem}\label{art6-lem1}%%% 1
Let $\{\varphi_t\}_{t \in T}$ be an algebraic family of automorphisms of a variety $X$ parametrised by a (not necessarily irreducible) algebraic space $T$. Then there exists a finite number of points $x_1,\ldots, x_n \in X$ such that if $\varphi_t (x_i) = \varphi_{t'} (x_i)$ $(i = 1, \ldots, n; t, t' \in T)$, then $\varphi_t = \varphi_{t'}$.
\end{lem}

\begin{proof}
For any $x \in X$, let $S_x$ be the closed subset of $T \times T$ consisting of all $(t, t')$ such that $\varphi_t (x) = \varphi_{t'} (x)$ and $S$ the subset of $(t, t')$ such that $\varphi_t = \varphi_{t'}$. Then we have $S = \bigcap\limits_{x \in X} S_x$, and since $T \times T$ is a noetherian space, $S = \bigcap\limits^n_{i=1} S_{x_i}$, where $x_1, \ldots, x_n$ are a finite number of points of $X$. This is the assertion of the lemma.
\end{proof}

\begin{lem}\label{art6-lem2}%%%% 2
Let $G$ be a connected group of automorphisms of a variety $X$, and $x$ any point of $X$. Then the orbit $G x = \{\varphi (x)| \varphi \in G \}$ is a locally closed non-singular subvariety of $X$. Moreover if $G$ is finite dimensional, $\dim G x \leqq \dim G$.
\end{lem}

\begin{proof}
By a well known theorem of Chevalley, if $\{\varphi_t\}_{t \in T}$ is any algebraic family of automorphisms of $X$, the subset $\varphi_T (x) = \{ \varphi_t (x) | t \in T\}$ is an irreducible constructible subset of $X$ of dimension at most equal to the dimension of $X$. It follows that we can choose a family $\{\varphi_t \}_{t \in T}$ such that $\varphi_t \in G$ for all $t \in T$ and $\dim \varphi_T (x)$ is a maximal. By replacing this family by $\{\varphi_t \circ \varphi^{-1}_{t_0}\}_{t \in T}$, if necessary, we may assume that $\varphi_{t_0}$  is the identity of $X$ for some $t_0 \in T$. The closure $\overline{\varphi_T (x)}$ of $\varphi_T (x)$\pageoriginale is a closed irreducible subvariety of $X$ and $\varphi_T (x)$ contains an open subset $U$ of $\overline{\varphi_T(x)}$. We assert that $G x \subset \overline{\varphi_T (x)}$. If not, there would exist a $\psi \in G$ such that $\psi(x) \not\in \overline{\varphi_T(x)}$. Since $G$ is connected, there exists a family $\{\psi_s\}_{s \in S}$ such that for two points $s_0, s_1 \in S$, we have $\psi_{s_0} = I_X$, $\psi_{s_1} = \psi$. The algebraic family of automorphisms $\{\psi_t \circ \psi_s\}_{(t,s) \in T \times S}$ is contained in $G$, and contains both the families $\{\varphi_t\}_{t \in T}$ and $\{\psi_s\}_{s \in S}$. Hence it follows that $\overline{\varphi_T (x)}$ $\varsubsetneqq$ $\overline{\varphi_T \circ \psi_S (x)}$ (since $\psi_{s1} (x) \not\in \overline{\varphi_T (x)}$), $\dim \varphi_T \circ \psi_S (x) = \dim \overline{\varphi_T \circ \psi_S (x)} >  \dim \overline{\varphi_T (x)} = \dim \varphi_T (x)$, which contradicts our choice of the family $\{\varphi_t\}$.

Thus, $G x \subset \overline{\varphi_T (x)}$, and since $G x = \bigcup\limits_{\varphi \in G} \varphi (U)$ and each $\varphi (U)$ is open in $\overline{\varphi_T (x)}$, $Gx$ is open in $\overline{\varphi_T (x)} = Gx$. Hence $Gx$ is locally closed in $X$, and since $G$ acts transitively on the variety $Gx$, it is non-singular.

Suppose now that $G$ is of finite dimension $N$, and $\dim G x > N$. It follows from the first part of the proof that there is an algebraic family $\{\varphi_t\}_{t \in T}$ contained in $G$ such that $\dim \varphi_T (x) = \dim G x > N$. Since the morphism $T \to G x$ given by $t \to \varphi_t (x)$ is dominant, it is easy to see that there is a subvariety $T_1$ of $T$ such that $\dim T_1 = \dim Gx$ and the morphism $T_1 \to Gx$ given by $t_1 \to \varphi_{t_1} (x)$ is again dominant. The function field $R(T_1)$ of $T_1$ is therefore algebraic over the function field $R (Gx)$ of $Gx$. Hence, by replacing $T_1$ by an open subset, we may further assume that the fibers of the morphism $t_1 \to \varphi_{t_1} (x)$ are finite, and in particular that for any $t_1 \in T_1$, there are only a finite number of $t_2 \in T_1$ such that $\varphi_{t_1} = \varphi_{t_2}$. If we can construct by a suitable ``descent'' an injective family of automorphisms of the same dimension, we would have the required contradiction.

By Lemma \ref{art6-lem1}, there exist $y_1, \ldots, y_n \in X$ such that $\varphi_{t_1} (y_i) = \varphi_{t_2} (y_i)$ $(i = 1, \ldots, n)$ implies that $\varphi_{t_1} = \varphi_{t_2}$. Let $y = (y_1, \ldots, y_n) \in X^n$, and let $\{\varphi^n_{t_1}\}_{t_1 \in T_1}$ be the algebraic family of automorphisms of $X^n$ defined by $\varphi^n_{t_1} (x_1, \ldots, x_n) = (\varphi_{t_1} (x_1), \ldots, \varphi_{t_1} (x_n))$. Since $\varphi^n_{T_1} (y)$ is constructible we may assume that $\varphi^n_{T_1} (y)$ is actually a locally closed normal subvariety of $X^n$, by replacing $T_1$ by an open subset. Since the fibers of $t_1 \to \varphi^n_{t_1} (y)$ are finite, we may assume (by replacing $T_1$ by its normalisation in a normal algebraic extension of $R (\varphi^n_{T_1} (y))$ containing $R (T_1)$) that $R (T_1)$ is normal over $R (\varphi^n_{T_1} (y))$. Let $T'_1$ be the normalisation of $\varphi^n_{T_1} (y)$ in the purely inseparable closure of $R (\varphi^n_{T_1} (y))$ in $R (T_1)$. By replacing by an open subset, we may assume that the morphism $T_1 \xrightarrow{\lambda} T'_1$ is a Galois covering, so that $T_1$ is the quotient of $T'_1$ by a finite group $\prod$.
If $t_1, t_2 \in T_1$ have the same image in $T'_1$, then $\varphi^n_{t_1} (y) = \varphi^n_{t_2} (y)$ and hence $\varphi_{t_1} = \varphi_{t_2}$, and conversely, if $\varphi_{t_1} = \varphi_{t_2}$, $\varphi^n_{t_1} (y) = \varphi^n_{t_2} (y)$ and hence $\lambda (t_1) = \lambda (t_2)$ since $T'_1$ is a purely inseparable covering of an open subset of $\varphi^n_{T_1} (y)$. Thus, the defining morphism $\varphi: T_1 \times X \to X$ commutes with the action of $\prod$ on $T_1 \times X$, and hence ``passes down'' to a morphism $\varphi' : T'_1 \times X \to X $ such that $\varphi' \circ (\lambda \times I_X) = \varphi$. Then $\varphi'$ defines an injective algebraic family of dimension = dimension of $T_1 > N$, which is a contradiction.

Lemma \ref{art6-lem2} is proved.
\end{proof}

We now proceed to the proof of the theorem. Let $\{\varphi_t\}_{t \in T}$ be an injective family contained in $G$ and containing the identity such that $\dim T = \dim G$ and $T$\pageoriginale is normal. We assert that any element $\psi$ of $G$ can be written as $\varphi_{t_1} \circ \varphi^{-1}_{t_2}$ with $t_1, t_2 \in T$. In fact, by an application of Lemma \ref{art6-lem1} to the union of the two families $\{\varphi_t\}_{t \in T}$ and $\{\psi \circ \varphi_t\}_{t \in T}$, we deduce that there exist a finite number of points $x_1, \ldots, x_n$ such that $\varphi_t (x_i) = \varphi_{t'} (x_i)$ implies that $\varphi_t = \varphi_{t'}$ and also such that $\varphi_t(x_i) = \psi \circ \varphi_{t'} (x_i)$ implies that $\varphi_t = \psi \circ \varphi_{t'}$. Let $x = (x_1, \ldots, x_n) \in X^n$, and make $G$ act on $X^n$ componentwise. Then $Gx$ is an irreducible locally closed subvariety of $X^n$ whose dimension is the dimension $N$ of $G$, by Lemma \ref{art6-lem2} and because the morphism $t \to \varphi^n_t (x)$ of $T$ into $Gx$ is injective. Also $\varphi^n_T (x)$ and $\psi \circ \varphi^n_T (x)$ contain open subsets of $Gx$, since both are of dimension $N$. Hence these two subsets of $Gx$ have a non-void intersection, so that $\varphi^n_{t_1} (x) = \psi \circ \varphi^n_{t_2} (x)$ for some $t_1, t_2 \in T$, which implies (by our choice of $x$) that $\varphi^n_{t_1} = \psi \circ \varphi_{t_2}$, $\psi = \varphi_{t_1} \circ \varphi^{-1}_{t_2}$. Thus the algebraic family $\{\varphi_t \circ \varphi^{-1}_{t'}\}_{(t, t') \in T \times T}$ contains all the elements of $G$. Hence by Lemma \ref{art6-lem1}, there exist a finite number of points $y_1, \ldots, y_m \in X$ such that $\varphi (y_i) = \varphi' (y_i) (i = 1, \ldots, m)$, $\varphi, \varphi' \in G$, implies that $\varphi = \varphi'$.

Let $y$ be the point $(y_1, \ldots, y_m) \in X^m$, and $Gy$ the orbit of $y$ for the action of $G$ componentwise on $X^m$. It follows from Lemma \ref{art6-lem2} that $Gy$ is an irreducible locally closed non-singular subvariety of $X^m$ of dimension $N$. Since the morphism $T \to G y$ given by $t \to \varphi_t (y)$ is dominant and injective, $R (T)$ is purely inseparable over $R (Gy)$. If the characteristic is zero, it follows from Zariski's main theorem that $T$ is isomorphic to an open subset of $Gy$, and we put $Z = Gy$. If the characteristic is $p>0$, we can find an integer $n \geqq 0$ such that $R(T) \subset R (Gy)^{p^{-n}}$. In this case, let $Z$ be the normalisation of $Gy$ in $R (Gy)^{p^{-n}}$. By replacing $T$ by its normalisation in $R (Gy)^{p^{-n}}$ we may assume that $T$ is an open subset of $Z$. Let $\pi: Z \to G y$ denote the projection of $Z$ onto $Gy$ (and the identity map if characteristic is zero). Since $Z$ is the normalisation of $Gy$ in $R (Gy)^{p^{-n}}$, and since $G$ acts (as an abstract group) as a group of automorphisms of $Gy$, it can be made to act as a group of automorphisms of $Z$ in such a way as to commute with the projection $\pi$. Since $\pi$ is bijective, it follows from our choice of $y$ that for any $z \in Z$, there is a unique element $\varphi_z \in G$ such that $\varphi_z (y) = \pi (z)$, and it follows that for any $\psi \in G$, we have $\varphi_{\psi z} = \psi \circ \varphi_z$. Also the map of $Z$ onto $G$ given by $z \to \varphi_z$ is a bijection. Since the family of automorphisms $\{\varphi_z\}_{z \in Z}$ is algebraic when restricted to the open subset $T$, and  since $G$ acts transitively (and simply) as a group of automorphisms of $Z$, it follows that $\{\varphi_z\}_{z \in Z}$ is an algebraic family.

We have thus constructed an algebraic family $\{\varphi_z\}_{z \in Z}$ parametrised by a non-singular variety $Z$, such that (a) $z \to \varphi_z$ is a bijection of $Z$ onto $G$, and (b) $G$ (as an abstract group) acts on $Z$ in such  a way that for $\psi \in G$, $z \in Z$, we have $\varphi_{\psi z} = \psi_z \circ \varphi$. If the family $\{\varphi_z\}_{z \in Z}$ is infinitesimally injective (this always holds when the characteristic of $K$ is zero, as is well known) on the whole of $Z$, we transport the algebraic structure of $Z$ onto $G$ by the above bijection, and we are through. Suppose that this is not so, so that the characteristic $p$ of $K$ is $> 0$. At any point $z \in Z$, let $T_z$ be the tangent space of $Z$, and let $T'_z$ be the kernel of the linear map $d \varphi$ of $T_z$ into the space of vector fields on $X$. Because of (b), the dimension of $T'_z$ is the same for all $z \in Z$. Since locally on $Z$ $T'_z$ is \pageoriginale defined by the vanishing of a finite number of regular differential forms, it follows that the family of vector spaces $\{T'_z\}_{z \in Z}$ defines s sub-bundle $T'$ of the tangent bundle of $Z$. It is easy to verify that if $X$ and $Y$ are vector fields on an open set of $Z$ such that their values at any point $z$ of this open set belong to $T'_z$, the same is true of $[X, Y]$ and $X^p$. Thus, $T'$ is an integrable sub-bundle of the tangent bundle of $Z$, in the sense of Cartier (see Expos\'e 6 of [5]). Hence there exists a non-singular variety $Z'$ and a radicial covering $p: Z \to Z'$ of height one such that the kernel of $dp$ at any $z \in Z$ is precisely $T'_z$. Further, there exists a morphism $\varphi' : Z' \times X \to X$ such that $\varphi' \circ (p \times I_X) = \varphi$ (Theorem 2 and Proposition 7, expos\'e 6, [5]). Thus, the family $\{\varphi'_{z'}\}_{z' \in Z'}$ parametrised by $Z'$ and defined by $\varphi'_{p(z)} = \varphi_z$ for $z \in Z$ is again algebraic. Also the action of $G$ on $Z$ ``goes down'' to an action of $G$ on $Z'$ since it leaves the sub-bundle $T'$ invariant. Finally we have a morphism of $Z'$ onto $Gy$ defined by $z'\to \varphi'_{z'}(y)$, which implies that $R (Z) \supset R (Z') \supset R (Gy)$, $[R (Z): R (Z')] > 1$. If the family on $Z'$ is not infinitesimally injective, we may repeat the above method of descent, to get  a $Z''$ with $R (Z) \supset R (Z') \supset R (Z'') \supset R (Gy)$, $[R (Z'): R (Z'')] >1$. Since $[R (Z): R (Gy)] < \infty$, we must arrive at a bijective and infinitesimally injective algebraic family in a finite number of steps. Transporting the algebraic structure of the parametrising variety of this family to $G$, we arrive at a structure of an algebraic variety on $G$, such that $G \times X \xrightarrow{\chi} X$ defined by $\chi(\varphi, x) = \varphi (x)$ is a morphism and the family $\{\chi_\varphi\}_{\varphi \in G} (\chi_\varphi = \varphi)$ is infinitesimally injective. 

Now, in the proof of the fact that part (a) of the theorem implies (b), we never used the fact that the group operations on $G$ are algebraic. Thus we may apply (b) to the algebraic family $\{\chi_\varphi \circ \chi_{\varphi'}^{-1}\}_{(\varphi, \varphi') \in G \times G }$ to deduce that the map $G \times G \to G$ given by $(\varphi, \varphi') \to \varphi \circ \varphi'^{-1}$ is a morphism.

The theorem is completely proved.

We now apply the theorem to the group of all automorphisms of a semi-complete variety which can be connected to the identity automorphism by an algebraic family parametrised by an irreducible variety. (From the preliminary remarks made at the beginning of this note, it follows that such automorphisms form a group under composition.) We shall say that a variety $X$ is {\em semi-complete} if for any torsion free coherent algebraic sheaf $\mathscr{F}$ on $X$, the vector space $H^0 (X, \mathscr{F})$ (over $K$) of sections is finite dimensional. By the theorem, we have only to show that there exists an integer $N$ such that if $\{\varphi_t\}_{t \in T}$ is any injective algebraic family of automorphisms of $X$, $\dim T \leqq N$. The normalization $(\tilde{X}, p)$ of $X$ is again semi-complete; for if $\mathscr{F}$ is coherent and torsion free on $\tilde{X}$, its direct image $p_* (\mathscr{F})$ on $X$ is again coherent and torsion free, and $H^0(\tilde{X}, \mathscr{F}) \cong H^0 (X, p_* (\mathscr{F}))$. Also the family $\{\varphi_t\}_{t \in T}$ lifts to a family $\{\tilde{\varphi}_t\}_{t \in T}$ of automorphisms of $\tilde{X}$, which is again injective. Let $Y$ be the closed set of singular points of $\tilde{X}$. Then any automorphism of $\tilde{X}$ leaves $\tilde{X} - Y$ stable. Any coherent torsion free sheaf $\mathscr{F}_1$ on $\tilde{X} - Y$ admits of an extension $\mathscr{F}$ to $\tilde{X}$, which is again torsion free, and since $\tilde{X}$ is normal and codim $Y \geqq 2$, it follows that the map $H^0 (\tilde{X}, \mathscr{F}) \cong H^0 (\tilde{X} - Y, \mathscr{F}_1)$ is an isomorphism.

Thus, we are reduced to the case of a non-singular semi-complete variety $X$. In this case, it is well known that the group of coherent sheaves of principal ideals\pageoriginale on $X$ (the Cartier divisors) is canonically isomorphic to the free group generated by the subvarieties of codimension one in $X$. Let $f_1, \ldots, f_p$ be non-constant rational functions on $X$ generating the function field $R(X)$ over $K$, and let $D = \sum\limits^n_{j=1} D_j$ be any positive divisor on $X$ with $D_j$ prime, such that $D + \divv (f_i) \geqq 0$, and $D + \divv (f_i - 1) \geqq 0$. If $\{\varphi_t\}_{t \in T}$ is any algebraic family of automorphisms of $X$ such that for some $t_0 \in T$, $\varphi_{t_0} = $ Identity and for any $t \in T$, the inverse image $\varphi^*_t (D)$ of $D$ equals $D$, $I$ assert that $\varphi_t = $ Identity. In fact, since $T$ is irreducible, it follows that we must have $\varphi_t (D_j) = D_j$ for any $j$, $1 \leqq j \leqq n$. If $V_j$ is the discrete valuation on $R(X)$ defined by $D_j$, it follows that $V_j (f \odot \varphi_t) = V_j (f)$. Hence, we must have $\divv (f_i) =\divv (f_i \odot \varphi_t)$, $\divv ((f_i - 1) \odot \varphi_t) = \divv (f_i -1)$ for $1 \leqq i \leqq p$. Since $X$ is semi-complete, the functions $\dfrac{f_i \odot \varphi_t}{f_i}$ and $\dfrac{(f_i -1) \odot \varphi_t}{f_i-1}$, being everywhere regular, must be constants. Thus  we obtain
\begin{gather*}
f_i \odot \varphi_t = a_i f_i,\\
(f_i -1) \odot \varphi_t = b_i (f_i -1) = f_i \odot \varphi_t -1 = a_i f_i -1 ,
\end{gather*}
which shows that $a_i = b_i =1$ and $f_i \odot \varphi_t = f_i$. Thus, $\varphi_t$ induces the identity automorphism on $R(X)$, and hence must be the identity.

Now if $\{\varphi_t\}_{t \in T}$ is any irreducible algebraic family of automorphisms of $X$ with $\varphi_{t_0} = $ Identity, $\{\varphi^*_t (D)\}_{t \in T}$ is an algebraic family of divisors on $X$ with $ \varphi^*_{t_0} (D) = D$. If $\Pic(X)$ is the (connected) Picard variety of $X$, we thus get a morphism $\psi: T \to Pic (X)$ defined by $\psi (t) = Cl (\varphi^*_t (D) - D)$ ([5], Expos\'e 8, corollary to Theorem 3). Let $T_1$ be an irreducible component of $\psi^{-1} (\psi(t_0))$ containing $t_0$. We then have $\dim T \leqq \dim T_1 + \dim  \Pic (X)$, by the dimension theorem (\cite{art6-key1}, Chapter III, Theorem 2). But now, for every $t \in T_1$, $\varphi^*_t (D) - D$ is linearly equivalent to zero, and thus we have an injective morphism $\xi: T_1 \to P^r$, where $P^r$ is the projective space which parametrises the complete linear system containing $D$ (\cite{art6-key5}, corollary to Prop. 7 and Theorem 2, Expos\'e 5]). Hence, we deduce that $\dim T \leqq \dim |D| + \dim \Pic (X)$.

We have thus proved

\begin{corollary}\label{art6-coro1}%%%%% 1
Let $X$ be a semi-complete variety. Then the group $G$ of all automorphisms of $X$ which can be connected to the identity automorphism by an irreducible family can be given the structure of a group variety such that the map $G \times X \to X$ given by $(\varphi, x) \to \varphi (x)$ is a morphism. The induced linear map of the Lie algebra $\fg$ of $G$ into the (finite dimensional) vector space of regular vector fields on $X$ is an injection. $G$ has the universal mapping property for all irreducible algebraic families of automorphisms containing the identity.
\end{corollary}

\begin{remark*}
By substituting $G$ for $T$ in the argument preceding the corollary, we see that $G$ is an extension of a subgroup of $\Pic (X)$ by a linear group. In particular, when $\Pic (X)$ is trivial, $G$ is a subgroup of the projective group of the projective space which defines the linear system $|D|$. Further, when $X$ is itself projective, we may clearly assume (by adding to $D$ a high multiple of an ample divisor) that $D$ is a hyperplane section in a projective imbedding of $X$. It follows that $G$ is the restriction to $X$ of a group of projective transformations of the ambient projective space (for this projective imbedding).
\end{remark*}

Now,\pageoriginale let $P$ be a locally isotrivial principal fiber space over a complete variety $X$ with structure group $G$. (\cite{art6-key6}, Expos\'e 1, \S 2.2.) We will show that the group $H$ of automorphisms of $P$ which commute with the action of $G$ on $P$ and which can be connected to the identity automorphism of $P$ is finite dimensional. Let $q: P \to X$ be the projection. If $\{\varphi_t\}_{t \in T}$ is any injective algebraic family of automorphisms of $P$ with $\varphi_{t_0} = $ Identity for some $t_0\in T$, it is easy to check that it induces an algebraic family of automorphisms $\{\bar{\varphi_t}\}_{t \in T}$ of $X$ such that $q \circ \varphi_t = \bar{\varphi}_t \circ q$. By Corollary \ref{art6-coro1} and the dimension theorem, it is sufficient to bound the dimension of any algebraic family $\{\varphi_t\}_{t \in T}$ such that $\varphi_{t_0} = $ Identity and $q (\varphi_t (x)) = q(x)$, that is, a family which fixes the base space.

Let $\varphi$ be any automorphism of $P$ which is identity on $X$, so that for any $p\in P$, there is a unique $\psi (p) \in G$ such that $\varphi (p) = p \psi (p)$. Then $\psi$ is a morphism of $P$ into $G$ satisfying $\psi (pg) = g^{-1} \psi (p)g$. Let $Ad (P)$ denote the bundle associated to $P$ with fiber $G$ for the action of $G$ on the left of $G$ by inner automorphisms and $\eta: P \times G \to Ad (P)$ the canonical map (\cite{art6-key6}, Expos\'e 1, \S 3.3). There is then a unique regular section $\sigma : X \to Ad (P)$ such that $\eta(p, \psi (p)) = \sigma (q (p))$. Suppose now that $H$ is a closed normal subgroup of $G$, and let $P'$ be the principal fiber space with structure group $G/H$ deduced from $G$ (\cite{art6-key6}, Expos\'e 1, \S 3.3). It is clear that $\varphi$ induces an automorphism of $P'$, which is the identity if and only if the morphism $\psi: P \to G$ defined above maps $P$ into $H$, or equivalently, if the section $\sigma$ of the bundle $Ad (P)$ has values in the sub-bundle with fiber $H$.

Assume first that the structure group $G$ is linear, so that we may assume it to be a subgroup of a full linear group $Gl(n)$. Let $G$ act on the vector space $M(n)$ of all $(n,n)$ matrices on the left by inner automorphisms, and let $V$ be the associated vector bundle. Then $A d(P)$ is a sub-bundle of $V$. If $\{\varphi_t\}_{t \in T}$ is an injective family of automorphisms of $P$, we therefore get for each $t \in T$ a section $\sigma_t : X \to V$ of $V$, and it is easy to see that $\sigma : T \times X \to V$ defined by $\sigma (t, x) = \sigma_t (x)$ is a morphism. Since $X$ is complete, the vector space $\mathscr{L}$of sections of $V$ is finite dimensional, and if $\mathscr{L}$ is provided with the structure of an affine space, $t \in T \to \sigma_t \in \mathscr{L}$ is clearly a morphism which is injective. Hence $\dim T \leqq \dim_K \mathscr{L}$, and we are through in this case.

Next suppose $G$ is any connected algebraic group, and $C$ the centre $G$. Then $G/C$ is a linear group (\cite{art6-key4}, \S 4, Lemma 3). Since we know the finite dimensionality of the group of automorphisms of the principal bundle with group $G/C$ deduced from $P$, it is sufficient to prove the finite dimensionality of the group of automorphisms of $P$ which induce the identity on the base and on the  associated bundle with $G/C$ as fiber. If $\{\varphi_t\}_{t \in T}$ is any injective family of automorphisms of this group with $\varphi_{t_0} = $ Identity, we get a morphism $\sigma : T \times X \to C$ such that if $\sigma_t : X \to C$ is defined by $\sigma_t (x) = \sigma (t, x)$, $\sigma_t \neq \sigma_{t'}$ if $t \neq t'$, since the bundle associated to $P$ with fiber $C$ and the (trivial) action of $G$ on $C$ by inner automorphisms is trivial. Also we have $\sigma_{t_0} (X) = e$ in $C$. Let $C'$ be the maximal linear subgroup of $C$, so that $C/C'$, is an abelian variety, and $j: C \to C/ C'$ the canonical homomorphism. Since $j \circ \sigma (t_0 \times X) = e$ in $C/C'$, it follows from well known\pageoriginale theorems on abelian varieties that $j \circ \sigma$ depends only on $T$, that is, there is a morphism $\xi: T \to C / C'$ such that  $j \circ \sigma (t, x) = \xi (t)$. Hence if $T_1$ is any irreducible component of $\xi^{-1} (e)$ containing $t_0$, $\dim T \leqq T_1 + \dim C/C'$. But for any $t \in T _1$, $\sigma_t (X) \subset C'$, and since $X$ is complete and $C'$ is linear, $\sigma_t (X)$ is a single point $\sigma_t \in C'$, and the morphism $T_1 \to C'$ given by $t \to \sigma_t$ is  injective. Hence, $\dim T \leqq \dim C' + \dim C/ C'$.

Finally, suppose $G$ is not connected, and let $G_0$ be the connected component of $G$ containing $e$. Then $P$  may be considered as a principal bundle with structure group $G_0$ over the Galois covering $Y = P \times {}^G G / G_0$ of $X$. Any connected family of automorphisms of $P$ over $X$ induces the identity on $Y$, and hence may be considered as a family of automorphisms of $P$ over $Y$. Since $Y$ is again complete we are reduced to the previous case.

Thus, we have proved

\begin{corollary}\label{art6-coro2} %%% 2
Let $P$ be a locally isotrivial principal fiber space with base a complete variety $X$ and structure group $G$. Let $\Aut^0 (P)$ be the group of all automorphisms of $P$ which can be connected to the identity by an irreducible algebraic family, and $\Aut^0_X(P)$ the subgroup of $\Aut^0(P)$ consisting of those automorphisms which leave the base fixed. Then $\Aut^0 (P)$ can be made into an algebraic group variety in such a way that the map $\chi: \Aut^0 (P) \times P \to P$ defined by $\chi (\varphi, p) = \varphi (p)$ is a morphism. $\Aut^0_X (P)$ is a closed subgroup of $\Aut^0 (P)$. The linear map $d_\chi$ maps the tangent space at $e$ to $\Aut^0 (P)$ (\resp $\Aut^0_X(P)$) injectively into the vector space of $G$-invariant vector fields on $P$ (\resp $G$-invariant vector fields on $P$ which are tangential to the fiber at any point of $P$).
\end{corollary}


\begin{thebibliography}{99}
\bibitem{art6-key1} Chevalley, C.: Fondements de la g\'eometrie alg\'ebrique. Paris 1958.

\bibitem{art6-key2} -- Sur la th\'eorie de la vari\'et\'e de Picard. Am. J. Math. 82, 435--490 (1960).

\bibitem{art6-key3} Matsusaka, T.: Polarized varieties, fields of moduli and generalized Kummer varieties of polarized abelian varieties. Am. J. Math. 80, 45--82 (1958).

\bibitem{art6-key4} Samuel, P.: Travaux de Rosenlicht sur les groups alg\'ebriques. S\'eminaire Bourbaki, 1956--57.

\bibitem{art6-key5} S\'eminaire C. Chevalley, 1958: Vari\'et\'es de Picard.

\bibitem{art6-key6}  S\'eminaire C. Chevalley, 1958--59: Anneaux de Chow et applications.

\bibitem{art6-key7} Seshadri, C. S.: Vari\'et\'e de Picard d'une vari\'et\'e compl\`ete. Ann. Math. Pura et appl. (IV) 62, 117--142 (1962).
\end{thebibliography}

\centerline{(Received July 31, 1963)}

\vfill\eject
\phantom{a}
\thispagestyle{empty}
