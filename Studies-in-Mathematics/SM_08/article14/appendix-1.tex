\title{Appendix I \\The Theorem of Tate}\label{apen1}
\markright{Appendix I - The Theorem of Tate}

\author{By~ C.P. Ramanujam}
\markboth{C.P. Ramanujam}{Appendix I - The Theorem of Tate}

\date{}
\maketitle

\setcounter{page}{169}
\setcounter{pageoriginal}{133}
We have\pageoriginale seen that if $X$ and $Y$ are abelian varieties over a field $k$ (algebraically closed) and $l$ a prime different from the characteristic of $k$, the natural homomorphism
\begin{equation}
\bfZ_l \otimes_\bfZ \Hom (X,Y) \xrightarrow{T_1} \Hom_{\bfZ_l} (T_l (X), T_l (Y)) \label{apen1-eq1}
\end{equation}
is injective. It is obviously of importance to know the image.

By a {\em field of definition} of $X$, we shall mean a subfield $k_0$ of $k$ over which there is a {\em group scheme} $X_0$ and an isomorphism of {\em group schemes} $X_0 \otimes_{k_0} k \xrightarrow{\sim} X$. We also say that $X$ is an abelian variety defined over $k_0$. Now choose a common field of definition $k_0$ for $X$ and $Y$; we may and shall assume $k_0$ to be of finite type over the prime field. Let $\bar{k}_0$ be the algebraic closure of $k_0$ in $k$. Since $n_X$ and $n_Y (n \in \bfZ)$ are morphisms defined over $k_0$, their respective kernels $X_n$ and $Y_n$ consist entirely of $\bar{k_0}$-rational points, or in other words, the points of finite order in $X$ and $Y$ are $\bar{k}_0$-rational (i.e., they come from $\bar{k_0}$-rational points of $X_0 \otimes_{k_0} \bar{k}_0$ and $Y_0 \otimes_{k_0} \bar{k}_0$ resp.) Thus, if $G = G (\bar{k}_0/k_0)$ is the Galois group of $k_0$ over $k_0$, $G$ acts on the groups $X_n$ and $Y_n$ compatible with the homomorphisms $X_{mn} \xrightarrow{m} X_n$ and $Y_{mn} \xrightarrow{m} Y_n$. Hence $G$ acts continuously on the $\bfZ_l$-modules $T_l(X)$ and $T_l (Y)$, where we give $G$ the Krull topology and $T_l(X)$ and $T_l(Y)$ their $l$-adic topologies.

Furthermore, let $\varphi$ be any homomorphism of $X$ into $Y$. Since the points of finite order in $X$ are $\bar{k}_0$-rational and $k$-dense, and $\varphi$ maps points of finite order into points of finite order, $\varphi$ is defined over $\bar{k}_0$ and hence over a finite extension $k_1$ of $k_0$ in $\bar{k}_0$ (i.e., comes by base extension from a homomorphism of group schemes $X_0 \otimes_{k_0} k_l \to Y_0 \otimes_{k_0} k_1$). It then follows that if $H = G (\bar{k}_0/ k_1) \subset G$ is the subgroup of $G$ fixing $k_1$, for any $\bar{k}_0$-rational point $x$ of $X$, we have $h \varphi(x) = \varphi (hx)$ for all\pageoriginale $h \in H$. Thus, if we make $G$ act on $\Hom_{Z_l} (T_l(X), T_l(Y))$ by putting $(g\lambda) (x) = g(\lambda (g^{-1} x))$ for $g \in G$, $\lambda \in \Hom_\bfZ (T_l (X), T_l(Y))$ and $x \in T_l (X)$, we see that $h T_l (\varphi) = T_l (\varphi)$ for $h \in H$. In other words, for any $\varphi \in \Hom (X,Y)$, there is a neighborhood $H$ of the identity in $G$ fixing $T_l(\varphi)$. We see therefore that if for any $G$-module $M$ we denote by $M^{(G)}$ the subgroup of elements fixed by some neighborhood of $e$, the image of \eqref{apen1-eq1} is contained in $\Hom_{\bfZ_l} (T_l(X), T_l(Y))^{(G)}$, and we obtain a homomorphism 
\begin{equation}
\bfZ_l \otimes_\bfZ \Hom (X,Y) \to \Hom_{\bfZ_l } (T_l(X), T_l(Y))^{(G)}. \label{apen1-eq2}
\end{equation}
Tate {\em conjectures} that \eqref{apen1-eq2} is an isomorphism (or equivalently \eqref{apen1-eq2} is surjective) for any field of definition $k_0$ of finite type over the prime field. He proves this for abelian varieties defined over a finite field. It has also been proved when $k_0$ is an algebraic number field and $X=Y$ is of dimension one (Serre). We shall now reproduce Tate's proof in the case of finite fields.

\begin{thm}\label{apen1-thm1}
Let $X$ and $Y$ be abelian varieties defined over a finite field $k_0$. Then the homomorphism (\eqref{apen1-eq2}) is an isomorphism.
\end{thm}

The rest of this section will be devoted to the proof, which we give in several steps.

\begin{romanstep}\label{apen1-step1}
 It suffices to show that the homomorphism
\begin{align}
\bfQ_l \otimes_{\bfZ} \Hom (X, Y) & = \bfQ_l \otimes_{\bfQ} \Hom^0 (X,Y) \longrightarrow \notag\\
& \qquad \longrightarrow \Hom_{\bfQ_l} (V_l (X), V_l (Y))^{(G)} \label{apen1-eq3}
\end{align}
 where $V_l(X) = \bfQ_l \otimes_{\bfZ_l} T_l(X)$, is bijective.
 \end{romanstep}

In fact, if this were so, the image of the homomorphism \eqref{apen1-eq1} would be a $\bfZ_l$-submodule of maximal rank in $\Hom_{\bfZ_l} (T_l(X),T_l(X))^{(G)}$. On the other hand, if this image is $M$, $\Hom_{\bfZ_l} (T_l(X), T_l(X))/M$ has no torsion; for, suppose  $\varphi \in \Hom_{\bfZ_l} (T_l(X), T_l(X))$ and $l_\varphi \in M$. We can find $\psi_n \in \Hom (X,Y)$ such that $T_l(\psi_n) \to l_\varphi$, so that for all large $n$, $T_l (\psi_n) = l_{\varphi_n, \varphi_n} \in \Hom_{\bfZ_l} (T_l(X), T_l (Y))$. Thus, $T_l (\psi_n) (T_l (X)) \subset l T_l(Y)$ and $\psi_n$ vanishes on $X_l$. Thus $\psi_n$ admits a factorisation $\psi_n =\chi_n \circ l_X = l \chi_n$ and the $\chi_n$ converge to a certain $\chi$ in $\bfZ_l \otimes_{\bfZ} \Hom (X,Y)$ with $T_l(\chi) = \varphi$, so that $\varphi \in M$. This proves our assertion, and establishes that $M = \Hom_{\bfZ_l} (T_l(X), T_l(Y))^{(G)}$.

\begin{romanstep}\label{apen1-step2}
It suffices\pageoriginale to show that for any abelian variety $X$ defined over a finite field
\begin{equation}
\bfQ_l\otimes_{\bfQ} \End^0(X) \xrightarrow{\lambda} \End_{\bfQ_l} (V_l (X))^{(G)} \label{apen1-eq4}
\end{equation}
is an isomorphism.
\end{romanstep}

In fact, if \eqref{apen1-eq4} is an isomorphism with $X \times Y$ instead of $X$, since we have the direct sum decompositions as $G$-modules
{\fontsize{7}{9}\selectfont
\[
\xymatrix@C=0.28cm{
\bfQ_l \otimes \End^\circ (X \times Y) \ar[d] &  (\bfQ_l \otimes_{\bfQ} \End^0 X) \ar[d]\ar[l]_-\sim  
\ar@{}[r]|{\displaystyle\oplus} & (\bfQ_l \otimes_{\bfQ}  \End^0 Y) \ar[d] \ar@{}[r]|{\displaystyle\oplus} & (\bfQ_l \otimes_{\bfQ} \Hom^0 (X,Y)) \ar@{}[r]|{\displaystyle\oplus} \ar[d] &  (\bfQ_l \otimes_{\bfQ} \Hom^0 (Y,X)) \ar[d]\\
\End (V_t (X \times Y)) & \End(V_t(X)) \ar[l]_-\sim \ar@{}[r]|{\displaystyle\oplus} & \End (V_l (Y)) \ar@{}[r]|{\displaystyle\oplus} & \Hom (V_l(X), V_l (Y)) \ar@{}[r]|{\displaystyle\oplus} & \Hom (V_l (Y), V_l (X)),
}
\]}\relax
the above diagram is commutative and the first vertical arrow is an isomorphism, \eqref{apen1-eq3} is an isomorphism.

Thus, henceforward we shall restrict ourselves to a single abelian variety $X$ defined over a finite field, and prove that \eqref{apen1-eq4} is an isomorphism.

\begin{romanstep}\label{apen1-step3}
It suffices to show that $\lambda$ is an isomorphism for one $l$ and that $\dim_{\bfQ_l} \End (V_l (X))^{(G)}$ is independent of $1 \neq \cchar \ldotp ~~k$.
\end{romanstep}

This follows from the fact that the dimension of the left member in \eqref{apen1-eq4} is independent of $l$ and $\lambda$ is always injective.

\begin{romanstep}\label{apen1-step4}
To establish \eqref{apen1-eq4} for an $l$, it suffices to show the following. Let $E$ be the image of $\lambda$, and $F$ the intersection over all neighborhoods of $e$ in $G$ of the subalgebras generated in $\End_{\bfQ_l} (V_l)$ by these neighborhoods. Then $F$ is the commutant of $E$ in $\End (V_l)$.
\end{romanstep}

In fact, since $E$ is semi-simple, if the above were true, it would follow from von Neumann's density theorem that $E$ is the commutant of $F$. Further, since $\End_{\bfQ_l} (V_l)$ is of finite dimension over $\bfQ_l$, $F$ actually equals the subalgebra generated by some neighborhood, and all smaller neighborhoods generate the same subalgebra. Hence the commutant of $F$ is precisely $\End_{\bfQ_l} (V_l)^{(G)}$.

Now, Steps \ref{apen1-step1}-\ref{apen1-step4} are valid for any field of definition $k_0$, and do not make use of the fact that $k_0$ is finite. The next step makes use of a certain\pageoriginale hypothesis which is easily seen to be true for finite fields, and probably holds for any field of finite type over its prime field. We therefore state it as a hypothesis, verify it for a finite field and deduce its consequences.

Let $X$ be an abelian variety defined over a field $k_0$, $l$ a prime. We then state the 

{\bf Hypothesis (\boldmath{$k_0, X, l$}):} {\em Let $d$ be any integer $\geqslant 1$. Then there exist upto isomorphism only a finite number of abelian varieties $Y$ defined over $k_0$, such that:
\begin{itemize}
\item[(a)] there is an ample line bundle $L$ on $Y$ defined over $k_0$ with $\chi(L) =d$;

\item[(b)] there exists a $k_0$-isogeny $Y \to X$ of $l$-power degree.
\end{itemize}}

(If $X \simeq X_0 \otimes_{k_0} k$ and $Y \simeq Y_0 \otimes_{k_0} k$, a line bundle $L$ on $Y$ is said to be defined over $k_0$ if there is a line bundle $L_0$ on $Y_0$ such that $L \simeq L_0 \otimes_{k_0} k$, and an isogeny $Y \to X$ is said to be defined over $k_0$ if it arises from a homomorphism $Y_0 \to X_0$ by base extension to $k$.)

For finite fields $k_0$, we have in fact the following stronger

\begin{lem}\label{apen1-lem1}
If $k_0$ is a finite field, $d > 0$ and $g>0$, there are upto isomorphism only finitely many abelian varieties $Y$ defined over $k_0$ of dimension $g$ and carrying an ample line bundle $L$ defined over $k_0$ with $\chi(L) = d$.
\end{lem}

\begin{proof}
The line bundle $L^3$ on $Y$ is defined over $k_0$ and gives a projective embedding of $Y$ as a $k_0$-closed subvariety of projective space of dimension $3^g \chi (L) -1 = 3^g d -1$. The degree of this subvariety is the $g$-fold self intersection number of $L^3$, that is, $3^g \ldotp (L)^g \ldotp d = 3^g \ldotp g!d $. Thus, there corresponds to it a $k_0$-rational point of the Chow variety of cycles of dimension $g$ and degree $3^g \ldotp g ! d $ in $\bfP^{3^g d -1}$. Since $k_0$ is a finite field, there are only finitely many $k_0$-rational points on this Chow variety, which proves the lemma.
\end{proof}

\begin{romanstep}\label{apen1-step5}
Suppose $X$ is an abelian vareity defined over $k_0$ and $L$ an ample line bundle on $X$ defined over $k_0$. Suppose for a prime $l \neq \cchar \ldotp k_0, Hyp (k_0, X, l)$ holds.\pageoriginale Let $W \subset V_l (X)$ be a subspace of $V_l(X)$ which is $G$-stable and is maximal isotropic for the skew-symmetric form $E^L$. Then there is an element $u \in E$ with $u (V_l(X)) =W$.
\end{romanstep}

\begin{proof}
Set $T = T_l (X)$, $V = V_l(X)$ and for each integer $n \geqslant 0$,
$$
T_n = (T \cap W) + l^n T
$$
If $\psi_n: T_n (X) \to X_{l^n}$ is the natural homomorphism, set $K_n = \psi_n(T_n)$, $Y_n = X/K_n$ and let $\pi_n : X \to Y_n$ be the natural homomorphism. Then  $(l^n)_X$ factors as $X \xrightarrow{\pi_n} Y_n \xrightarrow{\lambda_n} X$. Since $\pi_n \circ \lambda_n \circ \pi_n = l^n \pi_n$, we obtain that $\pi_n \circ \lambda_n =(l^n)_{Y_n}$, so that
$$
\lambda_n ((Y_n)_{l^n}) = \ker \pi_n = K_n.
$$
Furthermore, for any $m \geqslant n$, $\lambda_n ((Y_n)_{l^m}) \supset (\lambda_n \circ \pi_n) (X_{l^m}) = X_{l^{m-n}}$, so that $T_l(\lambda_n) (T_l (Y_n)) \supset l^n T_l(X)$, $T_l (\lambda_n) (T_l (Y_n)) = T_n$.

We now verify that $Y_n$ is an abelian variety defined over $k_0$ and that $\lambda_n$ is a $k_0$-morphism. First note that if $X = X_0 \otimes_{k_0} k$ where $X_0$ is a group scheme over $k_0$, the points of $X_{l^n}$, are separably algebraic over $k_0$. In fact, since $(l^n)_X: X \to X$ is \'etale, so is $(l^n)_{X_0}: X_0 \to X_0$, so that $(l^n)^{-1}_{X_0} (0)$ consists of points whose residue fields are separably algebraic over $k_0$. Further, since $W$ and hence $(T \cap W) + l^n T $ are $G$-stable by assumption, $K_n$ is also $G$-stable. Thus the fact that $Y_n$ as a variety is defined over $k_0$ and $\pi_n : X \to Y_n$ is defined over $k_0$ follow from the following general
\end{proof}


\begin{lem}\label{apen1-lem2}
Let $X$ be a quasi-projective variety defined over $k_0$ and $\Pi$ a finite group of automorphisms of $X$ such that 
\begin{itemize}
\item[\rm (i)] there is a separably algebraic extension of $k_0$ over which the automorphisms of $\Pi$ are defined;

\item[\rm (ii)] for any automorphism $\sigma$ of the algebraic closure $\bar{k}_0$ of $k_0$ over $k_0$, $\Pi^\sigma = \Pi$.
\end{itemize}

Then $X/\Pi$ is defined over $k_0$ and the natural map $X \to X / \Pi$ is defined over $k_0$.
\end{lem}

\begin{proof}
Choose\pageoriginale a Galois extension $k_1/k_0$ over which the automorphisms of $\Pi$ are defined. If $U$ is a $k_0$-affine open subset of $X$, $\bigcap\limits_{\lambda \in \Pi} \lambda U$ is again $k_0$-affine open and $\Pi$-stable, by assumption (ii) of the lemma. Thus, we may assume $X$ affine, $X = \Spec A$, $A = A_0 \otimes_{k_0} k$, $A_0$ being a $k_0$-algebra. If $A_1 = A_0 \otimes_{k_0} k_1$, $\Pi$ operates on $A_1$, and $A^{\pi}_1 \otimes_{k_1} k = A^\Pi$. Further, $G = \Gal (k_1/k_0)$ operates on $A_1$, and since by assumption $\lambda \Pi \lambda^{-1} = \Pi$ for any $\lambda \in G$, it follows that $A^{\Pi}_1$ is $G$-stable. Let $B$ be the $k_0$-algebra of $G$-invariants, and $\{\theta_i\}$ a basis of $k_1/k_0$. If $\sum a_i \otimes \theta_i \in A^{\Pi}_1$, $a_i \in A_0$, then $\lambda (\sum a_i \otimes \theta_i) = \sum a_i \otimes \lambda (\theta_i) \in A^{\pi}_1$, and since $\det (\lambda (\theta_i))_{\lambda \in G,i} \neq 0$, $a_i \otimes 1 \in A^\Pi_1 \cap A_0 \subset B$, which proves that $A^\Pi_1  = B \otimes_{k_0} k_1$.
\end{proof}

This proves the lemma.

Thus, as a variety, $Y_n$ is defined over $k_0$ and $\pi_n: X \to Y_n$ is defined over $k_0$, so that $\pi_n (0)$ is $k_0$-rational. Since the addition map $m: Y_n \times Y_n \to Y_n$ and the inverse $I: Y_n \to Y_n$ are defined over a separably algebraic extension of $k_0$ and are invariant under the action of $\Gal(\bar{k}_0/k_0)$, they are again defined over $k_0$. Hence $Y_n$ is defined over $k_0$ as an abelian variety. Now, $(l^n)_X = \lambda_n \circ \pi_n$ and $\pi_n$ are defined over $k_0$, hence $\lambda_n$ is defined over $k_0$, since for a $k_0$-regular function $\varphi$ on an open subset of $X$, $\varphi \circ \lambda_n$ is $k_1$-regular for a Galois extension $k_1$ of $k_0$ and invariant under $\Gal (k_1/k_0)$ since $\varphi \circ \lambda_n \circ \pi_n = \varphi \circ (l^n)_X$ is. 

Let $d$ be the degree of the ample line bundle $L$ on $X$. We shall produce on ample line bundle of degree $2^g \ldotp d$ defined over $k_0$ on each $Y_n$. We have
\begin{align*}
e_l (x, \varphi_{\lambda^*_n (L)} (y)) & = E^{\lambda^*_n (L)} (x,y) \\
& = E^L (\lambda_n(x), \lambda_n(y)) \in e^L (T_n, T_n)\\
& = e^L (l^n T,T \cap W + l^n T) \subset l^n M_l
\end{align*}
for any $x$, $y \in T_l(Y_n)$, since $W$ is isotropic for $e^L$. Since $e_l : T_l (Y_n) \times T_l (\hat{Y}_n) \to M_l$ is non-degenerate, it follows that $\varphi_{\lambda^*_n (L)} (T_l(X)) \subset l^n T_l(X)$, $\varphi_{\lambda^*_n (L)} = l^n \psi$ for some $\psi: Y_n \to Y_n$.

It follows\pageoriginale from the theorem of \S 23 that $\psi = \varphi_{L_n}$ for some line bundle $L_n$ on $Y_n$ defined over the algebraic closure, hence over some normal extension of $k_0$. We may assume that $L_n$ is symmetric. Hence, if $p$ denotes the characteristic of $k_0$ if this is positive and $p=1$ if the characteristic is zero, for a suitable integer $N >0$, $L^{p^N}_n$ is defined over a Galois extension $k_1$ of $k_0$. We now have
 
\begin{lem}\label{apen1-lem3}
Let $Y$ be an abelian variety defined over $k_0$ and $L \in \Pic^0 Y$ a line bundle defined over the algebraic closure $\bar{k}_0$ of $k_0$ in $k$. Let $\sigma \in \Gal   (\bar{k}_0 / k_0)$, and denote by $\sigma (L)$ the line bundle on $Y$ defined over $\bar{k}_0$ obtained by pulling back $L$ by the morphism $1_{Y_0} \times \Spec \sigma: Y_0 \otimes_{k_0} \bar{k_0} \to Y_0 \otimes_{k_0} \bar{k}_0$. Then $\sigma (L) \in \Pic^0 Y$.
\end{lem}

\begin{proof}
Let $M$ be an ample line bundle on $Y$ defined over $k_0$, and consider the line bundle $N = m^* (M) \otimes p^*_1 (M)^{-1} \otimes p^*_2 (M)^{-1}$ on $Y \times Y$. We can find an algebraic point $y \in Y$ such that $N | \{y\} \times Y = L$. It is then easy to see that $N | \{\sigma y\} \times Y = \sigma (L)$, so that $\sigma(L) \in \Pic^0 Y$.

We shall make use of the notation $\sigma (L)$ introduced in the lemma in future. Further, if $L_1$ and $L_2$ are two line bundles on $Y$ with $L_1 \otimes L^{-1}_2 \in \Pic^0 Y$, we shall write $L_1 \equiv L_2$.

Resuming the earlier discussion, if $\sigma \in \Gal (k_1/ k_0)$, we see that $\sigma (L^{p^N}_n)$ is also symmetric, and $\sigma (L^{p^N}_n)l^n \equiv \lambda^*_n (L^{p^N})$, so that since $NS(Y_n)$ is torsion-free $\sigma (L^{p^N}_n ) \equiv L^{p^N}_n$ for every $\sigma \in \Gal (k_1/ k_0)$. Hence, $\sigma (L^{p^N}_n)$ and $L^{p^N}_n$ differ by an element of order 2 in $\Pic^0 Y$. Thus, if we put $M_n = L^{2p^N}_n$, we have $\sigma (M_n ) \simeq M_n$ for every $\sigma \in \Gal (k_1/k_0)$ and $M^{l^n}_n \equiv \lambda^*_n (L^{2p^N}) \equiv L^{2p^N l^n}_n$. We now have
\end{proof}

\begin{lem}\label{apen1-lem4}
Let $Y$ be a complete variety defined over $k_0$ with a $k_0$-rational point, $k_1$ a Galois extension of $k_0$ and $L$ a line bundle on $Y$ defined over $k_1$ such that for every $\sigma \in \Gal (k_1/k_0)$, $\sigma (L) \simeq L$. Then $L$ can be defined over $k_0$.
\end{lem}

\begin{proof}
Put $X = X_0 \otimes_{k_0} k$, $X_1 = X_0 \otimes_{k_0} k_1$ and let $\pi : X_1 \to X_0$ the natural morphism. Since projective modules of constant rank over semi-local rings are free, we can find an affine covering $\mathfrak{u} = (U_i)_{i \in I}$ of $X_0$ such that $L/\pi^{-1} (U_i)$ is free. Let $\{\alpha_{ij}\}$ be the 1-cocycle with\pageoriginale respect to the covering $\{\pi^{-1} (U_i)\}$ with values in $\mathcal{O}^*_{X_1}$ defining $L$. Our assumption implies that for any $\sigma \in G = \Gal (k_1/ k_0)$, we have $\beta_{i,\sigma} \in \Gamma (\pi^{-1} (U_i), \mathcal{O}^*_{X_1})$ such that 
$$
\frac{\sigma (\alpha_{ij})}{\alpha_{ij}} = \frac{\beta_{i, \sigma}}{\beta_{j, \sigma}}.
$$
If $y_0$ is a $k_0$-rational point of $X_0$ with $y_0 \in U_{i_0}$ and $y_1 = \pi^{-1} (y_0)$, by dividing the $\beta_{j,\sigma}$ by $\beta_{i_0,\sigma } (y_1)$, we may assume that $\beta_{i_0, \sigma} (y_1) = 1$. Now, 
$$
\frac{\sigma \tau (\alpha_{ij})}{\alpha_{ij}} = \dfrac{\beta_{i, \sigma \tau}}{\beta_{j, \sigma \tau}} = \frac{\sigma \tau (\alpha_{ij})}{\sigma(\alpha_{ij})}, \frac{\sigma (\alpha_{ij})}{\alpha_{ij}} = \frac{\sigma(\beta_{i,\tau})}{\sigma (\beta_{j,\tau})} \cdot \frac{\beta_{i,\sigma}}{\beta_{j,\sigma}},
$$
so that for any $\sigma$, $\tau \in G$, $i$, $j \in I$,
$$
\beta_{i, \sigma \tau} \sigma (\beta_{i,\tau})^{-1} \beta^{-1}_{i,\sigma} = \beta_{j, \sigma \tau} \sigma (\beta_{j,\tau})^{-1} \beta^{-1}_{j,\sigma} \text{~ in ~} U_i \cap U_j.
$$
Since $X$ is complete, $\beta_{i,\sigma \tau} \sigma (\beta_{i,\tau})^{-1} \beta^{-1}_{i,\sigma} = C_{\sigma, \tau}$ for all $i$, and taking $i = i_0$ and evaluating at $y_1$, we see that 
$$
\sigma (\beta_{i,\tau}) \; \beta^{-1}_{i, \sigma\tau} \; \beta_{i,\sigma} = 1.
$$
Grant for the moment that if $A$ is any local ring of $X_0$ and $B = A \otimes_{k_0} k_1 $, and if $B^*$ is the group of units of $B$, $H^1 (G, B^*) = (1)$. We deduce that there is a covering $\{U_{i,\alpha}\}_{\alpha \in A_i}$ of $U_i$ and $\gamma_{i,\alpha} \in \Gamma (\pi^{-1} (U_{i,\alpha}), \mathcal{O}^*_{X_1})$ such that 
$$
\beta_{i, \sigma} = \frac{\sigma (\gamma_{i\alpha})}{\gamma_{i\alpha}} \text{ ~ in ~ } \pi^{-1} (U_{i\alpha}).
$$
The cocycle $\alpha_{ij} \dfrac{\gamma_{j\beta}}{\gamma_{i\alpha}}$ with respect to the covering $\{U_{i\alpha}\}_{\substack{\alpha \in A_i \\ i \in I}}$ is cohomologous to $\{\alpha_{ij}\}$, and we have
\begin{gather*}
\sigma \left(\alpha_{ij} \frac{\gamma_{j \beta}}{\gamma_{i\alpha}} \right) = \frac{\beta_{i\sigma}}{\beta_{j\sigma}} \cdot  \alpha_{ij} \frac{\beta_{j\sigma}}{\beta_{i \sigma}} \frac{\gamma_{j\beta}}{\gamma_{i\alpha}} = \frac{\gamma_{j\beta}}{\gamma_{i\alpha}} \alpha_{ij},\\
\alpha_{ij} \frac{\gamma_{j\beta}}{\gamma_{i\alpha}} \in \Gamma (U_{i\alpha} \cap U_{j\beta}, \mathcal{O}^*_{X_0}).
\end{gather*}

It only remains to show that if $A$ is the local ring of a point on $X_0$ and $B = k_1 \otimes_{k_0} A$, $H^1 (G, B^*) = \{1\}$. Let $R$ be the quotient field of $B$; we then have the exact sequence
$$
H^0 (G, R^*) \to H^0 (G, R^*/ B^*) \to H^1 (G, B^*) \to H^1 (G, R^*) = \{1\}.
$$
An element\pageoriginale of $H^0 (G, R^*/B^*)$ is represented by an element $f \in R^*$ such that $\dfrac{\sigma f}{f} \in B^*$ for all $\sigma \in G$, and we shall show that we can write $f= gu$ where $g \in H^0 (G, R^*)$ and $u \in B^*$. Writing $f = \dfrac{f_0}{F}$ with $f_0 \in B$, $F \in A$, we see that we may assume that $f \in B$. Now, since $\dfrac{\sigma f}{f} \in B^*$ for all $\sigma \in G$, the ideal $Bf$ is $G$-invariant, hence of the form $B\mathfrak{U}$, $\mathfrak{U}$ being an ideal in $A$. But now, since $Bf \simeq k_1 \otimes_{k_0} \mathfrak{U}$, $\mathfrak{U}$ is $A$-projective, hence principal. Thus, $Bf = Bg$ for some $g \in A$ and $f = gu$, $g \in A$, $u \in B^*$.
\end{proof}

This completes the proof of Lemma \ref{apen1-lem4}.

It follows that the line bundle $M_n$ on $Y_n$ is defined over $k_0$. Now, the g.c.d. of $2 p^N$ and $l^n$ is $1$ or 2 according as $l$ is 2 or not. Hence, we can find integers $a$, $b$ such that $2ap^N + bl^n =2$. Define $N_n = M^a_n \otimes \lambda^*_n(L)^b$, so that $N_n$ is defined over $k_0$. Further,
$$
N_n \equiv L^{2ap^N+bl^n}_n  = L^2_n,
$$
\begin{tabular}{r@{\;}l}
so that $N_n$ is ample with $\chi(N_n) = 2^g \chi(L_n)$  & = $2^g \ldotp l^{-ng} \chi (\lambda^*_n (L))$\\
& = $2^g \ldotp l^{-ng} \deg \ldotp \lambda_n \chi(L)$\\
& = $2^g \chi (L) = 2^g d$.
\end{tabular}

An alternative way of producing an ample line bundle of degree $2^gd$ on $Y_n$ is to observe that (i) if $Y$ is an abelian variety defined over $k_0$, and we construct $\hat{Y}$ as $Y/K(L)$ for a line bundle $L$ defined over $k_0$, $\hat{Y}$ is defined over $k_0$ and the Poincar\'e bundle $P$ on $Y \times \hat{Y}$ is defined over $k_0$; (ii) if $\varphi_{\lambda^*_n (L)} = l^n \psi$, since $\lambda^*_n(L)$ is defined over $k_0$, so is $\varphi_{\lambda^*_n(L)}$ and hence also $\psi$; and finally; (iii) if $P_n$ is the Poincar\'e bundle on $Y_n \times \hat{Y}_n$ and $\chi=(1,\psi): Y_n\to Y_n \times \hat{Y}_n$, then $\chi^* (P_n) = N_n$ is a line bundle defined over $k_0$ with $\varphi_N = 2 \psi$ (see \S 13), so that $\chi(N_n)^2 = \deg \varphi_{N_n} = 2^{2g} \deg \psi = 2^{2g} l^{-2ng} \deg \lambda^*_n (L) = 2^{2g} l^{-2ng} \times (\deg \lambda_n)^2 \chi (L)^2 = 2^{2g} \chi(L)^2$.

Anyhow, we deduce from $Hyp(k_0 , X, l)$ that there is an infinite set $I$ of  natural integers with smallest integer $n$ and isomorphisms $v_i: Y_n \xrightarrow{\sim} Y_i$ for all $i \in I$. Consider the elements $u_i = \lambda_i v_i \lambda^{-1}_n \in \End^0 X$ and their images $u'_i \in \End_\bfQ(V_l)$. We have $u'_i(T_n) = T_i\subset T_n$ for $i \in I$, and since\pageoriginale $\End_\bfZ(T_n)$ is compact, we can select a subsequence $(u'_j)_{j \in J}$ which converges to a limit $u'$. Since $E$ is closed in $\End (V_l)$ and $u'_j \in E$, $u'$ also belongs to $E$. Since $T_n$ is compact, $u'(T_n)$ consists of elements of the form $x =\lim x_j$ where $x_j \in u_j(T_n) = T_j$, and since the sets $T_j$ are decreasing, it follows that
$$
u(T_n) = \bigcap_{j\in J} u_j (T_n) = \bigcap_{j \in J} T_j = T \cap W.
$$
Hence $u (V) = W$.

This completes the proof of Step \ref{apen1-step5}.

\begin{romanstep}\label{apen1-step6}
Suppose that for any finite algebraic extension $k_1$ of $k_0$, $Hyp(k_1,\\ X, l)$ holds, and that $F$ is isomorphic as a $\bfQ_l$-algebra to a direct product of copies of $\bfQ_l$. Then \eqref{apen1-eq4} is an isomorphism.
\end{romanstep}

\begin{proof}
Replacing $k_0$ by a finite algebraic extension $k_1$ over which all elements of $\End X$ are defined and which is such that $\Gal (\bar{k}_1/k_1)$ generates $F$ in $\End (V_l)$, we may assume that $k_0$ itself has these properties. Let $D$ be the commutant of $E$ in $\End (V_l)$, so that $D \supset F$. We first show that any isotropic subspace $W$ for $E^L$ which is $F$-stable is also $D$-stable. We proceed by downward induction on $\dim W$. If $W$ is maximal isotropic, i.e. if $\dim W = g$, we can by Step \ref{apen1-step5} find a $u \in E$ such that $u (V) = W$, and hence $D W = D u V = u D V = u V = W$, which proves the assertion. Suppose then that $\dim W = r < g$ and the assertion holds for $F$-stable isotropic subspaces of dimension $r +1$. The orthogonal complement $W^{\bot}$ of $W$ for $E^L$ is also $F$-stable, since $E^L$ is invariant under the action of $\Gal(\bar{k}_0/k_0)$. Further, since any simple $F$-module is one-dimensional and $\dim W^\bot - \dim W = 2g - 2 \dim W= 2 (g -r) \geqslant 2 (g -g +1) =2$, we can find $F$-stable one-dimensional subspaces $L_1$ and $L_2$ of $W^\bot$ such that the sum $W + L_1 + L_2$ is direct. By induction hypothesis, $W + L_1$ and $W+ L_2$ are $D$-stable, hence so is their intersection $W$. This completes the induction. We deduce that any eigen-vector for $F$ in $V$ is also an eigen-vector for $D$. It follows that $D \subset F$. (The decomposition of $V$ into factors $V_i$ corresponding to the simple factors of $F$ reduces this assertion to the evident statement that an endomorphism of $V_i$ for which every element of $V_i$ is an eigen-vector is a scalar multiplication). Hence $F = D$, completing the proof of Step \ref{apen1-step6}.
\end{proof}

\begin{romanstep}\label{apen1-step7}
End of\pageoriginale proof of theorem.
\end{romanstep}

We assume from now on that $k_0$ is a finite field. By replacing it by a finite extension if necessary, we may assume that every element of $\End X$ is defined over $k_0$. Let $\pi$ be the Frobenius morphism over $k_0$. Then $\pi$ belongs to the center of $\End^0 X$, and hence $\bfQ[\pi]$ is a commutative semi-simple subalgebra of $\End^0 X$. We shall first show that there are an infinity of primes $l$ for which $\bfQ_l \otimes_\bfQ \bfQ[\pi]$ is isomorphic as a $\bfQ_l$-algebra to a direct product of copies of $\bfQ_l$.

In fact, writing $\bfQ[\pi] = K_1 \times \ldots \times K_r$ where $K_i$ are finite extensions of $\bfQ$, it suffices to show that for an infinity of primes $l$, each $\bfQ_l \otimes K_i$ is isomorphic to a product of copies of $\bfQ_l$. Let $K$ be a finite Galois extension of $\bfQ$ in which all the $K_i$ are embeddable. Then it suffices to show that for an infinity of $l$, $K \otimes_\bfQ \bfQ_l$ splits as a product of copies of $\bfQ_l$ as a $\bfQ_l$-algebra. It suffices for this that there is one simple factor of $K \otimes_\bfQ \bfQ_l$ isomorphic to $\bfQ_l$. In fact, if $K \otimes_\bfQ Q_l \simeq L_1\times \ldots \times L_k$, the Galois group $\pi$ of $K/\bfQ$ permutes the factors $L_i$. It also acts transitively on the simple factors. For, if not, suppose $L_1 \times \ldots \times L_r$ is $\pi$-stable; then the element $(1, 1,{\displaystyle\mathop{\ldots}_{r \text{ times}}}, 1, 0,0,0) \in L_1 \times \ldots \times L_k$ is $\pi$-stable. On the other hand, since $\pi$ fixes only the elements of $\bfQ$ in $K$, it fixes only the elements of $\bfQ_l$ in $\bfQ_l\otimes_\bfQ K$, that is, elements of the form $(\alpha, \alpha,\ldots, \alpha) \in L_1 \times \ldots \times L_k$ with $\alpha \in \bfQ_l$. This proves the assertion.
 
Now, choose an algebraic integer $\alpha$ of $K$ generating $K$ over $\bfQ$, and let $F(X)\in \bfZ[X]$ be its irreducible monic polynomial over $\bfQ$. Since $K \otimes_\bfQ \bfQ_l \simeq \bfQ_l[X]/(F(X))$, it is enough to find an infinity of $l$ for which $F(X)$ has a zero in $\bfQ_l$. Let $\Delta$ be the discriminant of $F(X)$, and $l$ any prime not dividing $\Delta$ such that $F(X) \equiv 0 (\mod l)$ has a solution $n$ in $\bfZ$. Then $F'(n)\nequiv0(\bmod l)$, so that by Hensel's lemma, $n$ can be refined to a root of $F$ in $\bfZ_l$. Thus, we are reduced to proving the following

\begin{lem}\label{apen1-lem5}
Let $F(X) \in \bfZ[X]$ be a non-constant polynomial. Then there are an infinity of primes $l$ for which $F(X) \equiv 0 (\mod l)$ has a solution in $\bfZ$.
\end{lem}

\begin{proof}
Let\pageoriginale $F(X) = a_0 X^n + a_1 X^{n-1} + \ldots + a_n$. The lemma being trivial when $a_n =0$, since $X$ is a factor of $F(X)$ in this case, we may assume $a_n \neq 0$. Further, by substituting $a_n X$ for $X$ in $F$ and removing the common factor $a_n$, we may assume $a_n =1$. Let $S$ be a finite set of primes $p$. If $N = \prod\limits_{p\in S} p$, then
$$
F(vN) = a_0 v^n N^n+ \ldots + a_{n-1} v N+ 1 \equiv 1 (\mod N)
$$
so that no prime of $S$ divides $F(vN)$. On the other hand, $F(vN) \neq \pm 1$ for $v$ large, hence has a prime factor $l$ not belonging to $S$.

Next, we show that for all $l \neq \cchar \ldotp k_0$, the dimension of $\End_{\bfQ_l} V_l^{(G)}$ is the same. Again, assume that every element of $\End X$ is defined over $k_0$, so that the Frobenius $\pi$ belongs to the center of $\End^0 X$, hence to the centre of $\bfQ_l \otimes_\bfQ \End^0 X$. Then $\bfQ_l [\pi]$ is semi-simple and $V_l$ is a $\bfQ_l[\pi]$-module, so that the image $\pi'$ of $\pi$ in $\End_{\bfQ_l} V_l$ is semi-simple. The characteristic polynomial $P(t)$ of $\pi'$ in $\End_{\bfQ_l} V_l$ has coefficients in $\bfZ$ independent of $l$. Further, the closed subgroups of $\Gal (\bar{k}_0/k_0)$ generated by $\pi^{n!}$ form a fundamental system of neighborhoods of $e$ in this group, so that $\End_{\bfQ_l} V^{(G)}_l$ is the commutant of $\pi'^{n!}$ in $\End_{\bfQ_l} V_l$ for $n$ large. The characteristic polynomial of $\pi'^{n!}$ has for roots $\theta^{n!}_1, \ldots, \theta^{n!}_{2g}$ where $\theta_1, \ldots, \theta_{2g}$ are the roots of $P(t)$ repeated with multiplicity. For all $n$ large, the number of distinct elements of $\theta^{n!}_1, \ldots, \theta^{n!}_{2g}$ as well as their multiplicities is the same. Thus, our assertion is a  consequence of the following lemma, applied to an algebraic closure of $\bfQ_l$.
\end{proof}

\begin{lem}\label{apen1-lem6}
Let $A$ and $B$ be absolutely semi-simple endomorphisms of two vector spaces $V$ and $W$ of finite dimensions over a field $k$ respectively, with characteristic polynomials $P_A$ and $P_B$. Let 
\begin{align*} 
P_A & = \prod\limits_{p} p^{m(p)},\\
P_B & = \prod\limits_{p} p^{n(p)}
\end{align*}
be the decompositions of $P_A$ and $P_B$ as products of powers of distinct irreducible monic polynomials $p$. Then the vector space
$$
E = \{\varphi \in \Hom_k (V,W) | \varphi A = B_\varphi\}
$$
has dimension\pageoriginale
$$
r (P_A, P_B) = \sum\limits_p m (p) n (p) \deg p,
$$
and this integer is invariant under any extension of the base field $k$.
\end{lem}

\begin{proof}
Make $V$ (\resp $W$) into a $k[X]$-module by making $X$ act through $A$ (\resp. $B$). Denote the $k[X]$-module $k[X]/(p(X))$ by $M_p$. Because of our assumption of semi-simplicity, we have isomorphisms of $k[X]$-modules
\begin{align*}
& V \simeq \prod\limits_p M^{m(p)}_p,\\
& W \simeq \prod\limits_p M^{n(p)}_p.
\end{align*}
Now, the $M_p$ are non-isomorphic simple $k[X]$-modules for distinct $p$, and $E$ is nothing but $\Hom_{k[X]} (V,W)$. Since $\dim_k \Hom_{k(X)} (M_p, M_p) = \dim_k M_p = \deg p$, and since $E$ clearly `commutes with base extension', the lemma follows.

The main Theorem \ref{apen1-thm1} is a consequence of what we have proved, combined with Steps \ref{apen1-step3} and \ref{apen1-step6}.
\end{proof}

\begin{remark*}
The theorem can be stated in the following seemingly stronger form:
\end{remark*}

\begin{dashthm}\label{apen1-dashthm1}
Let $X$ and $Y$ be abelian varieties defined over a finite field $k_0$, and let $\Hom_{k_0} (X,Y)$ be the group of homomorphisms of $X$ into $Y$ defined over $k_0$. If $G$ is the Galois group of the algebraic closure of $k_0$ over $k_0$, we have an isomorphism
$$
Z_l \otimes_{\bfZ} \Hom_{k_0} (X,Y) \xrightarrow{\sim} \Hom_{\bfZ_l} (T_l (X), T_l (Y))^G.
$$
\end{dashthm} 

\begin{proof}
Take $G$-invariants on both sides of the isomorphism \eqref{apen1-eq2}.
\end{proof}

{\bf Applications.} We give some easy consequences of Theorem \ref{apen1-dashthm1}$'$. We shall make our statements over a fixed finite field. The `geometric' statements over the algebraic closure are easily obtained from these. We shall consistently use the notation $r(f,g)$ for two polynomials $f$ and $g$, introduced in Lemma \ref{apen1-lem6}. Further, as we have shown above that if $X$ is an abelian variety defined over a finite field $k_0$ with Frobenius morphism $\pi$,\pageoriginale then the Frobenius morphism $\pi^n$ over a finite extension of $k_0$ induces semi-simple endomorphisms of $V_l(X)$ over $\bfQ_l$, it follows since $\bfQ_l$ is of characteristic zero that $\pi$ itself induces (absolutely) semi-simple endomorphisms of $V_l(X)$ for all $l \neq \cchar \ldotp k_0$. Thus, the structure of $V_l(X)$ as a module over the Galois group of $\bar{k}_0$ over $k_0$ is uniquely determined by the characteristic polynomial of $\pi$, as explained in the proof of Lemma \ref{apen1-lem6}.

We now have

\begin{thm}\label{apen1-thm2}
Let $X$ and $Y$ be abelian varieties defined over a finite field $k_0$, and let $P_X$ and $P_Y$ be the characteristic polynomials of their Frobenius endomorphisms relative to $k_0$. Then
\begin{itemize}
\item[\rm (a)] we have
$$
\rank (\Hom_{k_0} (X, Y)) = r(P_X, P_Y);
$$

\item[\rm (b)] the following statements are equivalent:
\begin{itemize}
\item[\rm (b$_1$)] $Y$ is $k_0$-isogenous to an abelian subvariety of $X$ defined over $k_0$,

\item[\rm (b$_2$)] $V_l(Y)$ is $G$-isomorphic to a $G$-subspace of $V_l(X)$ for some $l$,

\item[\rm (b$_3$)] $P_Y$ divides $P_X$;
\end{itemize}

\item[\rm (c)] the following statements are equivalent:
\begin{itemize}
\item[\rm (c$_1$)] $X$ and $Y$ are $k_0$-isogenous, 

\item[\rm (c$_2$)] $P_X = P_Y$,

\item[\rm (c$_3$)] $X$ and $Y$ have the same number of $k_1$-rational points for every finite extension $k_1$ of $k_0$.
\end{itemize}
\end{itemize}
\end{thm}

\begin{proof}
(a) follows from Theorem \ref{apen1-dashthm1}$'$ and Lemma \ref{apen1-lem6}, since the Frobenius morphism generates the Galois group in the topological sense. The implications (b$_1$) $\Rightarrow$ (b$_2$) $\Leftrightarrow$ (b$_3$) are clear, in view of our earlier remarks. We show that (b$_2$) $\Rightarrow$ (b$_1$). If (b$_2$) holds, we can find an injective $G$-homomorphism $\varphi$ of $T_l (Y)$ into $T_l(X)$. Then $\varphi$ is in the image of $Z_l\otimes_{\bfZ} \Hom_{k_0} (X,Y)$, so that we can find $\psi \in \Hom_{k_0} (X,Y)$ such that $T_l(\psi)$ approximates arbitrarily closely to $\varphi$, in particular with $T_l(\psi)$ injective. If $\psi$ is not an isogeny, we can find an abelian subvariety\pageoriginale $Z$ of $Y$ in the kernel of $\psi$, and the submodule $T_l(Z)$ of $T_1(Y)$ would be in the kernel of $T^l(\psi)$. Hence $\psi$ is an isogeny, proving (b).

The equivalence (c$_1$) $\Leftrightarrow$ (c$_2$) is a special case of (b$_1$) $\Leftrightarrow$ (b$_3$) when $\dim X = \dim Y$. Further, we have seen during the proof of the Riemann hypothesis in \S 20 that if $\omega_i (i \in I)$ and $\omega'_j(j \in J)$ are the roots of $P_X$ and $P_Y$ respectively and $N_n$ and $N'_n$ are the number of rational points of $X$ and $Y$ respectively in the extension of degree $n$ of $k_0$, we have 
\begin{align*}
N_n & = \prod\limits_i (1 - \omega^n_i),\\
N'_n & = \prod\limits_j (1 - \omega'^n_j);
\end{align*}
thus, we have to show that $P_X = P_Y \Leftrightarrow \prod\limits_i (1-\omega^n_i) = \prod\limits_j (1-\omega'^n_j)$ for every $n \geqslant 0$. The implication $\Rightarrow$ is obvious. To prove the other implication, note that 
\begin{align*}
\prod\limits_{i \in I} (1 - \omega^n_i) & = \sum\limits_{S \subset I} (-1)^{|S|} \omega^n_S,\\
\prod\limits_{j \in J} (1 - \omega'^n_j) & = \prod\limits_{T \subset J} (-1)^{|T|} \omega'^n_T
\end{align*}
where $|S|$, $|T|$ denote the respective cardinalities and $\omega_S = \prod\limits_{i \in S} \omega_{i}$, $\omega'_T = \prod\limits_{j \in T} \omega'_j$. Multiplying the given equation by $t^n$, where $t$ is a variable, and summing as formal power series, we obtain
$$
\sum\limits_{S \subset I} (-1)^{|S|} \frac{1}{1-t\omega_S} = \sum\limits_{T \subset J} (-1)^{|T|} \frac{1}{1-t\omega_T}
$$
Since $|\omega_i| = |\omega'_j| = q^{1/2}$, comparing the poles on both sides on the circle $|t|= q^{1/2}$, we obtain that there is a bijection $\sigma: I \to J$ such that $\omega_i = \omega'_{\sigma (j)}$. Hence, $P_X = P_Y$.

Before we come to the next theorem, we need some preliminaries. Let $X$ be an abelian variety defined over $k_0$, so that $X = X_0 \otimes_{k_0} k$ for some group-scheme $X_0$ over $k_0$. Let $Y$ be an abelian subvariety of $X$, which is a $k_0$-closed subset, so that if $\lambda:X \to X_0$ is the natural morphism,\pageoriginale there is a closed subset $Y_0$ of $X_0$ with $\lambda^{-1}(Y_0) = Y$ in the set-theoretic sense. We give $Y_0$ the structure of a reduced subscheme of $X_0$. If $m_0 : X_0 \times_{k_0} X_0\to X_0$ is the multiplication morphism, our hypothesis implies the {\em set-theoretic inclusion } $m_0 (Y_0 \times_{k_0} Y_0) \subset Y$. Hence $m_0$ restricts to a morphism $m_0 :(Y_0 \times_{k_0} Y_0)_{\red.} \to Y_0$. If we can assert that $Y = Y_0 \otimes_{k_0} k$ and $Y_0 \times_{k_0} Y_0$ are reduced, it would follow that $Y$ is an abelian variety defined over $k_0$. Both of these are consequences of the assertion that the function field $R_{k_0} (Y_0)$ is a regular extension of $k_0$, or equivalently that $R_{k_0} (Y_0)$ is a separable extension of $k_0$. This is always true (vide S. Lang, {\em Abelian varieties,} Chap. I), but we shall not prove this, since we shall need it only when $k_0$ is a finite (hence perfect) field, so that this is trivially satisfied.

Next, suppose $X$ is an abelian variety defined over $k_0$, and $Y$ an abelian subvariety which is $k_0$-closed. We want to show that there is an abelian subvariety $Z$ of $X$ defined over $k_0$ such that $Y + Z = X$ and $Y \cap Z$ is finite. We know (vide \S 18, proof of Theorem \ref{apen1-thm1}) that if $L$ is an ample line bundle on $X$, we can take $Z$ to be the connected component of 0 of the group $Z' = \{z \in X|~ T^*_Z (L) \otimes L^{-1}|_Y \text{ is trivial} \}$, so that if we can ensure that $Z$ is defined over $k_0$ for  a suitable choice of $L$, we are through. Now if we choose $L$ to be a line bundle defined over $k_0$, $Z'$ is defined over the algebraic closure $\bar{k}_0$ of $k_0$ and is stable for all automorphisms of $\bar{k}_0$ over $k_0$. Hence $Z'$ is $k_0$-closed. Further, the conjugations of $\bar{k}_0$ over $k_0$ permute the components of $Z'$, and since $Z$ is a component of $Z'$ containing the $k_0$-rational point $0$, $Z$ is also stable under these conjugations. Hence $Z$ is $k_0$-closed, and it follows from the comments of the earlier paragraph that $Z$ is an abelian variety defined over $k_0$.

If follows from this by repeating the arguments of \S 18 that if we  call an abelian variety $X$ defined over $k_0$ to be $k_0$-simple if it does not contain an abelian subvariety $Y$ defined over $k_0$ with $Y \neq \{0\}$, $Y \neq X$, then (i) any abelian variety defined over $k_0$ is $k_0$-isogenous to a product of $k_0$-simple abelian varieties, and (ii) if $X$ is $k_0$-isogenous to a product $X^{n_1}_1 \times \ldots \times X^{n^r}_r$, where $X_i$ are $k_0$-simple and $X_i$ and $X_j$ are not $k_0$-isogenous if $i \neq j$, then $\End^0_{k_0} X  \simeq M_{n_1} (D_1) \times \ldots \times M_{n_r} (D_k)$ where $D_i =\End^0_{k_0} (X_i)$ are division algebras of finite rank over $\bfQ$.
\end{proof} 
We\pageoriginale  now have


\begin{thm}\label{apen1-thm3}
Let $X$ be an abelian variety of dimension $g$ defined over a finite field $k_0$. Let $\pi$ be the Frobenius endomorphism of $X$ relative to $k_0$ and $P$ its characteristic polynomial. We then have the following statements:
\begin{itemize}
\item[\rm (a)] The algebra $F = \bfQ [\pi]$ is the center of the semi-simple algebra $E =\End^0_{k_0} (X)$;

\item[\rm (b)] $\End^0_{k_0} (X)$ contains a semi-simple $\bfQ$-subalgebra $A$ of rank $2g$ which is maximal commutative;

\item[\rm (c)] the following statements are equivalent:
\begin{itemize}
\item[\rm (c$_1$)] $[E: \bfQ] = 2g$,

\item[\rm (c$_2$)] $P$ has no multiple root,

\item[\rm (c$_3$)] $E = F$,

\item[\rm (c$_4$)] $E$ is commutative;
\end{itemize}

\item[\rm (d)] the following statements are equivalent:
\begin{itemize}
\item[\rm (d$_1$)] $[E: \bfQ] = (2g)^2$, 

\item[\rm (d$_2$)] $P$ is a power of a linear polynomial,

\item[\rm (d$_3$)] $F = \bfQ$,

\item[\rm (d$_4$)] $E$ is isomorphic to the algebra of $g$ by $g$ matrices over the quaternion division algebra $D_p$ over $\bfQ (P = \cchar. k_0)$ which splits at all primes $l \neq p$, $\infty$,

\item[\rm (d$_5$)] $X$ is $k_0$-isogenous to the $g$-th power of a super-singular curve, all of whose endomorphisms are defined over $k_0$;
\end{itemize}

\item[\rm (e)] $X$ is $k_0$-isogenous to a power of a $k_0$-simple abelian variety if and only if $P$ is a power of a $\bfQ$-irreducible polynomial. When this is the case, $E$ is a central simple algebra over $F$ which splits at all finite primes $v$ of $F$ not dividing $p$, but does not split at any real prime of $F$.
\end{itemize}
\end{thm}

\begin{proof}
If follows from the main theorem that $F_l = \bfQ_l \otimes_{\bfQ} F$ is the center of $E_l = \bfQ_l \otimes_{\bfQ} E$, which proves that $F$ is the center of $E$. 

Suppose\pageoriginale $E = A_1 \times \ldots \times A_r$ is the expression of $E$ as a product of simple algebras $A_i$ with centers $K_i$. Let $[K_i:\bfQ] = a_i$, and $[A_i: K_i] = b^2_i$. We can choose subrings $L_i$ of $A_i$ containing $K_i$ with $L_i$ semi-simple and maximal commutative, $[L_i: K_i] =b$. Then $L = L_1 \times \ldots \times L_r$ is a semi-simple $\bfQ$-subalgebra of $E$ which is maximal commutative, and $[L:\bfQ] = \sum\limits^r_1 a_i b_i$. Now, for any $l$, we have
\begin{align*}
E \otimes_\bfQ \bfQ_l = \prod\limits^r_1 (A_i \otimes_\bfQ \bfQ_l) & = \prod\limits^r_1 (A_i \otimes_{K_i} (K_i \otimes_\bfQ \bfQ_l))\\
& = \prod\limits^r_{i=1} \prod\limits^{n_i}_{j=1} A_i \otimes_{K_i} K'_{ij}
\end{align*}
where $K_i \otimes_\bfQ \bfQ_l = \prod\limits^{n_i}_{j=1} K'_{ij}$ with $K'_{ij}$ fields. On the other hand, if $P = \prod\limits^S_1 P^{m_i}_i$ is the decomposition of $P$ over $\bfQ_l$ into a product of powers of irreducible polynomials over $\bfQ_l$, and if we consider $T_l(X)$ as a $\bfQ_l[T]$-module by making $T$ act via $\pi$, we have an isomorphism of $\bfQ_l[T]$-modules
$$
T_l(X) \simeq \prod\limits^s_{v=1} \left(\frac{\bfQ_l[T]}{(P_v)} \right)^{m_v} = \prod\limits^s_{v=1} S^{m_v}_v, S_v = \frac{\bfQ_l[T]}{(P_v)},
$$
so that $E \otimes_{\bfQ} \bfQ_l$, being the commutant of $\pi$ in $\End T_l(X)$, is isomorphic to 
$$
\prod\limits^s_{v=1} M_{m_v} (S_v).
$$
Comparing the two factorisations of $E \otimes_\bfQ \bfQ_l$ and keeping in mind that $K'_{ij}$ is the center of $A_i\otimes_{K_i} K'_{ij}$, we deduce that (i) for any prime $l \neq p$ and any prime $v$ of $K_i$ lying over $l$, $A_i$ splits at $v$ and (ii) there is a partition of $[1,s]$ into $r$ disjoint subsets $I_{1}, \ldots, I_r$ such that 
$$
A_i \otimes_{\bfQ} \bfQ_l\simeq \prod\limits^{n_i}_{j=1} A_i \otimes_{K_i} K'_{ij} \simeq \prod\limits_{v \in I_i} M_{m_v} (S_v).
$$
It follows\pageoriginale that $m_v = b_i$ for $v \in I_i$ and $\sum\limits_{v \in I_i} [S_v : \bfQ_l] = \sum\limits^{n_i}_{j=1} [K'_{ij}: \bfQ_l] = [K_i: Q] = a_i$, so that
\begin{align*}
\sum\limits^r_1 a_i b_i = \sum\limits^r_{i=1} \sum\limits_{v \in I_i} m_v [S_v : \bfQ_l] & = \sum\limits^s_{v=1} m_v[S_v: \bfQ_1]\\
& = \sum\limits^s_{v=1} m_v \deg P_v = \deg P = 2g.
\end{align*}            
This proves (b).


Since $F$ is the center of $E$ and $E$ contains a maximal commutative subring of rank $2g$, (c$_1$), (c$_3$) and (c$_4$) are equivalent, and since $E$ commutative $\Leftrightarrow E_l$ commutative $\Leftrightarrow m_v =1$ with the above notations, these are also equivalent to (c$_2$). This proves (c).

Now, $[E:\bfQ] = (2g)^2$ if and only if $\bfQ_l \otimes_{\bfQ} E \simeq M_{2g} (\bfQ_l)$, hence if and only if $s =1$, $S_v = \bfQ_l$ or equivalently, $P$ is a power of a linear polynomial. In this case, $\bfQ_l$  is the center of $\bfQ_l \otimes_{\bfQ} E$, so that $\bfQ$ is the center of $E$, and conversely, if this holds, $\bfQ_l \otimes_{\bfQ} E$ is the commutant of $\bfQ_l$ in $\End V_l$, so that it is the whole of $\End V_l$. Thus (d$_1$), (d$_2$) and (d$_3$) are equivalent. If $\bfQ_l \otimes_\bfQ E = M_{2g} (\bfQ_l)$, $E$ is a central simple algebra over $\bfQ$ whose invariants at all finite primes  $l\neq p$ are 0. Since its invariant at the infinite prime is 0 or $\frac{1}{2}$ and the sum of invariants at all primes is 0, $E$ is either $M_{2g} (\bfQ)$ or $M_g (D_p)$ where $D_p$ is the quaternion algebra over $\bfQ$ splitting at all finite primes $l\neq p$. The first possibility is ruled out, since $X$ cannot be a product of $2g$ abelian varieties. This proves that (d$_1$) $\iff$ (d$_4$). In view of our remarks preceding the theorem, (d$_4$) is equivalent to saying that $X \simeq C^q$, where $C$ is an elliptic curve with $\End_{k_0} C \simeq D_p$. We have then shown that $C$ is supersingular (\S 22). This proves (d).

Let $Q$ be the product of the distinct irreducible factors of $P$. Since $F = \bfQ [\pi]$ is semi-simple, and $P(\pi) = 0$, we have $Q(\pi) =0$. Further, $\pi$ acts as an endomorphism of $V_l$, any irreducible factor over $\bfQ_l$ of the characteristic polynomial $P$ divides the minimal polynomial of $\pi$, so that $Q$ is the minimal polynomial of $\pi$ over $\bfQ$. Now, $X$ is $k_0$-isogenous to a power\pageoriginale of a $k_0$-simple abelian variety if and only if $E$ is simple, hence if and only if the center $F = \bfQ [\pi]$ of $E$ is a field. Since $F \simeq \bfQ[X]/ (Q(X))$, $F$ is a field if and only if $Q$ is irreducible, or equivalently, $P$ is the power of an irreducible polynomial $Q$. If $F$ is the center of $E$, we have shown earlier that $E$ splits at any finite prime $v$ of $F$ not dividing $l$. Suppose $v$ is a real imbedding of $F$, so that $v(\pi)$ is a real number. Since $v(\pi)$ satisfies $P(v(\pi)) =0$ and the roots of $P$ have absolute value $\surd q$, we must have $v(\pi) = \pm \surd q$. If $q$ is a square, $v(\pi) \in \bfQ$ and $F = \bfQ$, so that the equivalent condition of (d$_3$) and (d$_4$) implies that $E$ does not split at $\infty$. If $q$ is not a square, $F = \bfQ(\surd p)$. Let $k_1$ be the quadratic extension of $k_0$ and $\pi' = \pi^2$ the Frobenius over $k_1$. Then $\pi^2 \in \bfQ$, so that the center $F'$ of $E' = \End_{k_l} X$ is $\bfQ$. Appealing to (d), we conclude that $E' = M_g (D_p)$. On the other hand, we have $F' \subset F \subset E \subset E'$, and $E$ is the commutant of $F$ in $E'$. By a known result on central simple algebras, we see that $E$ and $F \otimes_{F'} E'$ define the same element of the Brauer group over $F$, that is, $E$ is the image of $E'$ under the natural map $Br(F') \to Br(F)$. Since both the real primes of $\bfQ(\surd p)$ lie over the real prime $\infty$ of $\bfQ$ and $E'$ has invariant $\frac{1}{2}$ at $\infty$ with respect to $F' = \bfQ$, $E$ has invariant $\frac{1}{2}$ at either of the real primes of $F = \bfQ(\surd p)$. This completes the proof of (e).

\end{proof} 
\begin{coro*}
Any two elliptic curves defined over finite fields with isomorphic algebras of complex multiplications are isogenous (over the algebraically closed field $k$). 

In particular, any two supersingular elliptic curves are isogenous.
\end{coro*}

\begin{proof}
Suppose $X$, $Y$ are supersingular elliptic curves. We can choose a common finite field of definition $k_0$ such that $\End_{k_0}X$ and $\End_{k_0} Y$ are quaternion algebras over $\bfQ$, so that they have $\bfQ$ for center. Thus, their Frobenius morphisms $\pi_X$ and $\pi_Y$ lie in $\bfQ$. Since they must both have absolute value $\surd q$ where $q = \text{card} (k_0)$, we see that $\pi^2_X = \pi^2_Y =q$. Thus, if $k_1$ is the quadratic extension of $k_0$, there is an isomorphism $T_l (X) \xrightarrow{\sim} T_l(Y)$ carrying the action of $\pi'_X$ into $\pi'_Y$, where $\pi'_X = \pi^2_X$ and $\pi'_Y = \pi^2_Y$ are the Frobenius morphisms over $k_1$. By Theorem \ref{apen1-thm2}, $X$ and $Y$ are isogenous over $k_1$.

Next\pageoriginale suppose $\End^0 X \simeq K$, $\End^0 Y \simeq K$ for some imaginary quadratic extension $K$ of $\bfQ$. Choose a common finite field of definition of $X$ and $Y$ over which all their endomorphisms are defined and all the points of order $p$ are rational. Now, $\bfQ_p \otimes_{\bfQ} \End^0 X$ admits a one-dimensional representation in $T_p(X)$. Hence, $p$ splits into a product of two distinct primes $\mathfrak{p}$ and $\mathfrak{p}'$ in $K$ which are conjugate, and $Q_p\otimes_Q \End^0 X\simeq K_{\mathfrak{p}} \times K_{\mathfrak{p}'}$.  Suppose for instance that $Q_p \otimes_Q \End^0 X$ acts on $T_p(X)$ via $K_\mathfrak{p}$. By what we have said, it follows that $\pi_X \equiv 1 (\mathfrak{p})$, and since $Nm\pi_X$ is a power of $p, (\pi_X)$ has to be a power of $\mathfrak{p}'$ in the ring of integers of $K$. A similar assertion (possibly with $\mathfrak{p}$ replacing $\mathfrak{p}'$) holds for $\pi_Y$. By altering the isomorphism $\End^0 Y \simeq K$ by the conjugation of $K$ if necessary, we may assume that $(\pi_Y)$ is also a power of $\mathfrak{p}'$. Since $Nm\pi_X = Nm\pi_Y = q = p^f$, we see that in the ring of integers of $K$, $(\pi_X) = (\pi_Y) =\mathfrak{p}'^f$ so that $\pi_X$ and $\pi_Y$ differ by a unit, i.e., a root of unity since $K$ is imaginary quadratic. Thus $\pi^n_X = \pi^n_Y$ in $K$ for suitable $n$, and they have the same minimal equation over $\bfQ$, of degree 2. Since this has to be their characteristic polynomial, $X$ and $Y$ are isogenous over an extension of degree $n$ of $k_0$.
\end{proof}

\vfill\eject
~\phantom{a}
\thispagestyle{empty}
