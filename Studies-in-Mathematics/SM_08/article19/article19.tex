\title{Varieties with No Smooth Embeddings}\label{art19}
\markright{Varieties with No Smooth Embeddings}

\author{By~ M.V. Nori}
\markboth{M.V. Nori}{Varieties with No Smooth Embeddings}

\date{}
\maketitle

\setcounter{page}{283}

\setcounter{pageoriginal}{240}
\textsc{This paper is }\pageoriginale based essentially on the following idea: A scheme $X$ which has no effective Cartier divisors cannot be embedded in a smooth scheme, or for that matter, even in a smooth algebraic space (\cite{art19-key1}, p. 326). For, given any such embedding of $X$, one could find plenty of such divisors on the ambient space which do not contain $X$, and hence their restrictions to $X$ would be effective and locally principal again. However, this says nothing about embedding $X$ in a complex manifold (not necessarily algebraic).

I thank Simha very heartily for providing the incentive to look for examples of such varieties; also for discussing the problem with me in some detail.

One may construct such a scheme $X$ quite simply as follows:

Take irreducible curves $C_1$ and $C_2$ of different degrees in $\bfP^n$, such that there is a birational isomorphism $f: C_1 \to C_2$, and identify the points $x$ and $f(x)$, where $x \in C_1$, to get a quotient variety $X$ of $\bfP^n$ with the following properties:
\begin{itemize}
\item[(A)] $X$ is a reduced irreducible scheme and $\varphi: \bfP^n \to X$ is the normalisation map,

\item[(B)] $\varphi = \varphi \circ f$ for all points of $C_1$, and $\varphi$ maps $C_1$ birationally onto its image $C = \varphi (C_1)$,

\item[(C)] any line bundle $L$ on $X$ lifts to the trivial line bundle on $\bfP^n$, and therefore $X$ is not projective,

\item[(D)] however, any finite set of points of $X$ is contained in an affine open set, and finally,

\item[(E)] $X$ has no smooth embeddings.
\end{itemize}

To prove (C), note that the degrees of $\varphi^*(L)|C_1$ and $\varphi^*(L)|C_2$ both coincide with the degree of $L|C$, and are therefore equal. But $\varphi^*(L)$ is $\mathcal{O}_{\bfP^n} (k)$ for some integer $k$, and therefore the degrees in question are $kd_1$ and $kd_2$ respectively, where $d_1$ and $d_2$ are the degrees of $C_1$ and $C_2 $. Now, $d_1 \neq d_2$ by assumption, implying that $k$ is equal to zero and thereby settling the fact that $\varphi^* (L)$ is trivial.

Now\pageoriginale (E) follows. Because, otherwise, $X$ possesses an effective line bundle $L$ which lifts to $\varphi^*(L)$, an effective line bundle!
 
The rest of the properties follow from the explicit construction of $X$, the details of which are probably well-known (compare with Theorem 6.1 of `Algebraization of formal moduli...' by M. Artin, {\em Annals of Math.} (1970), vol. 91) but follow nevertheless.

All schemes considered are of finite type over an algebraically closed field $k$.

Call a scheme a $F$-scheme if every finite set of (closed) points is contained in an affine open set.

\begin{lem}\label{art19-lem1}
If $Y \to Y'$ is a finite morphism and $Y'$ is an $F$-scheme, so is $Y$.
\end{lem}

\begin{proof}
Obvious.
\end{proof}

\begin{lem}\label{art19-lem2}
If $Y \to Z$ is a closed immersion, $g: Y \to Y'$ a finite morphism,and $Z$ and $Y'$ are $F$-schemes, then given any finite set $S \subset Z$, there exists an affine open subset $U$ of $Z$ which contains $S$, such that $g$ restricts to a finite morphism from $Y \cap U$ onto its image.
\end{lem}

\begin{proof}
Replace $S$ by $T = S \cup g^{-1} g(S \cap Y)$ which is also finite and let $W_1$ be an affine open set that contains it. Because $Y'$ is an $F$-scheme, there is an affine open set $V'$ that contains $g(S \cap Y)$. Now, $g^{-1}(V') \cap W_1$ is a neighbourhood of $T \cap Y$ in $Y$; so it follows that there is an affine open $W_2 \subset W_1$ such that $W_2 \cap Y$ is contained in $g^{-1}(V')$, and $T \subset W_2$. Also, $g^{-1} g(S \cap Y)$ is contained in $W_2$, which means that there is a $h$ in the co-ordinate ring of $V'$ such that $g$ restricts to a finite morphism from $D(h \circ g) \subset W_2 \to D (g) \subset V'$. Also, there exists $f$ defined on $W_2$ such that $f = h \circ g$ when restricted to $Y \cap W_2$, and $T \subset D (f)$. Putting $D(f) = U$ proves the lemma.
\end{proof}

\begin{lem}\label{art19-lem3}
If $g : Y \to Y'$ is a finite morphism, and $Y$ is an $F$-scheme, so is $Y'$.
\end{lem}

\begin{proof}
Assume,\pageoriginale by induction, that the lemma has been proved when $\dim Y \leqslant n -1$.
\end{proof}

\begin{step}\label{art19-step1}
$Y$, $Y'$ irreducible, reduced, and have the same quotient field, with $\dim Y = n$.
\end{step}

Let $I$ be the conductor of the morphism; denote by $A$ and $A'$ the closed subschemes  defined by $I$ in $Y$ and $Y'$ respectively. Then $A$ is an $F$-scheme of dimension $\leqslant n -1$; by induction, $A'$ is an $F$-scheme too, so one may appeal to Lemma 2 with $A'$, $A$, $Y$ in place of $Y'$, $Y$, $Z$. Now, let $S'$ be any finite set in $Y'$, $S$ its inverse image in $Y$ and $U$ an affine open set containing $S$ such that $U \cap A \to g(U \cap A) \subset A'$ is a finite morphism. This is merely equivalent to saying that $g^{-1} g(U \cap A) = U \cap A$, from which it follows that $g^{-1} g(U) = U$, so that $U \to g(U)$ is a finite morphism, proving that $g(U)$ is affine. Obviously, $g(U)$ contains $S'$, finishing the proof that $Y'$ is an $F$-scheme.

\begin{step}\label{art19-step2}
$Y$ and $Y'$ both irreducible and normal. There is a factoring $Y \to Y'' \to Y'$ with $Y \to Y''$ purely inseparable and $Y'' \to Y'$ separable, with $Y''$ normal too. That $Y''$ an $F$-scheme implies $Y$ an $F$-scheme is classical (it involves identifying $Y''$ with the quotient of a certain $F$-scheme by a finite group $G$) and we omit it. Also, $Y \to Y''$ is a homeomorphism in the Zariski topology, proving that $Y'$ is an $F$-scheme too.
\end{step}

\begin{step}\label{art19-step3}
$Y$ and $Y'$ irreducible.
\end{step}

This is proved by replacing $Y$ by its normalisation and putting step \ref{art19-step1} and step \ref{art19-step2}
 together.

\begin{step}\label{art19-step4}
The general case.
\end{step}

Let $Z'$ be any irreducible component of $Y'$ and $Z$ any irreducible component of $g^{-1}(Z')$ that maps onto $Z'$. Because $Z$ is an $F$-scheme, by step \ref{art19-step3}, $Z'$ is a $F$-scheme. So the lemma would be proved if one has 


\vskip 0.4cm
\textsc{\bf Sub-Lemma.} {\em If $Y = Y_1 \cup Y_2$ where $Y_1$ and $Y_2$ are closed subschemes, then $Y$ is an $F$-scheme if and only if $Y_1$ and $Y_2$ are both $F$-schemes.}

\begin{proof}
Assume\pageoriginale that $Y_1$ and $Y_2$ are $F$-schemes, and let $S$ be any finite set in $Y$. With the given information, it is a trivial matter to find affine open subsets $U_i$ of $Y_i$ containing $S \cap Y_i$ for $i = 1, 2$, such that $U_1 \cap Y_2 = U_2 \cap Y_1$, by choosing affine open $V_i$ in $Y_i$ containing $S \cap Y_i$ and then taking convenient principal affine open subsets of each. Now, $U_1$ and $U_2$ are closed in $U = U_1 \cup U_2$, proving that $U$ is affine (because a scheme is affine if and only if each irreducible component is affine).
\end{proof}

\begin{prop*}
Given an $F$-scheme $Z$, a closed subscheme $Y$, a finite surjective morphism $g: Y \to Y'$ which induces a monomorphism on co-ordinate rings, there is a unique commutative diagram
\[
\xymatrix{
Y \ar[r] \ar[d]_g & Z \ar[d]^f\\
Y' \ar[r] & Z'
}
\]
with:
\begin{itemize}
\item[(a)] $Z'$ is an $F$-scheme, $f$ is finite and induces a monomorphism on co-ordinate rings, $Y' \to Z'$ is a closed immersion, and 

\item[(b)] the ideal $I$ that defines $Y'$ in $Z'$ remains an ideal in $f_*(\mathcal{O}_Z)$, in fact the ideal that defines $Y$ in $Z$.
\end{itemize}
\end{prop*}

\begin{proof}
First assume that $\bfZ$ is affine, in which case $Y$ and $Y'$ are affine too. Let $A$, $A/I$, $B$ be their co-ordinate rings, and then $B \subset A / I$ in a natural way. Let $j : A \to A / I$ be the standard map, and put $A' = j^{-1} (B)$, spec $A' = Z'$. That $Z'$ has the required properties follows immediately, as does the fact that if $Z$ were replaced by an open subset $U$ such that $g^{-1} g(U \cap Y) = U \cap Y$, $Z'$ would be replaced by $U' = f(U)$ which is an open subset. 

This guarantees the existence of $Z'$ once it has been shown that $Z$ can be covered by affine open subsets $U$ such that $g^{-1} g(U \cap Y) = U \cap Y'$. By Lemma 3, $Y'$ is an $F$-scheme, so it follows by Lemma 2 that such open sets cover $Z$, and in fact can be chosen to contain any finite set of points, proving that $Z'$ exists  and is an $F$-scheme.

To come\pageoriginale back to the previous problem, put $\bfP^n = Z$, and $Y = C_1 \cup C_2$ in $\bfP^n$. Let $D_i$ be the normalisation of $C_i$ and $\tilde{f}: D_1 \to D_2$ be the lift of the rational map $f: C_1 \to C_2$. There are several candidates for a commutative diagram:
\[
\xymatrix{
D_1 \amalg D_2 \ar[r]^-{h'} \ar[d] & Y \ar[d]\\
D_2 \ar[r]^h & Y' 
}
\]
among which there is a best one, for which,
\begin{itemize}
\item[(I)] $h_* (\mathcal{O}_{D_2}/ \mathcal{O}_{Y'})$ injects into $h'_* (\mathcal{O}_{D_1 \amalg D_2}) / \mathcal{O}_{\gamma'}$,

\item[(II)] $h$ is birational.
\end{itemize}

Applying the above proposition, with $Z$, $Y$ and $Y'$ as above, one gets the variety $X$ mentioned in the introduction.

A final remark: $X$ has no line bundles at all if $C_1$ and $C_2$ are chosen suitably!

Let $i_1$ and $i_2$ be the composites $D_1 \to C_1 \to \bfP^n$ and $D_2 \to C_2 \to \bfP^n$ respectively. Assume that there is a point $P$ of $D_1$ such that $i_1 (P) = i_s f(P)$. Then 
\begin{itemize}
\item[(F)] $X$ is simply connected (follows from a careful application of Van Kampen's theorem), and 

\item[(G)] $\Pic X = 0$.
\end{itemize}

A rather neat example is provided by taking a conic $C_1$ and a tangential line $C_2$ in the projective plane and associating to a point $P$ on $C_1$ the point of intersection $Q$ of $C_2$ and the tangent line to $C_1$ at $P$; the properties (A) to (G) can be verified directly in this case. 

\vskip 0.3cm
[\textsc{Added in Proof:} Such varieties were also constructed by G. Horrocks on slightly different lines; see: `Birationally ruled surfaces without embeddings in regular schemes', by G. Horrocks, {\em J. Lond, Math. Soc.,} Vol. III (1971)]
\end{proof}

\begin{thebibliography}{99}
\bibitem{art19-key1} S. Kleiman:\pageoriginale Numerical Theory of Ampleness, {\em Annals of Math., } vol. 84, No. 2, 1966.
\end{thebibliography}
