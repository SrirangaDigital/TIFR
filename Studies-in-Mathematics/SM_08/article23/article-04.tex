\title{The Tricanonical Map of a Surface with \texorpdfstring{$K^{2}=2$, $P_{g}=0$}{K2}}\label{art04}
\markright{The Tricanonical Map of a Surface with $K^{2}=2$, $P_{g}=0$}

\author{By~ E.~Bombieri and F.~Catanese}
\markboth{E. Bombieri and F. Catanese}{The Tricanonical Map of a Surface with $K^{2}=2$, $P_{g}=0$}

\date{}
\maketitle

\setcounter{page}{327}

\setcounter{pageoriginal}{278}
\section{Introduction}\label{art04-sec1}\pageoriginale
In this paper we prove the following result.

\begin{theorem}\label{art04-thm1}
Let $S$ be a minimal surface of general type with $K^{2}=2$,
$p_{g}=0$, over an algebraically closed field $k$, $\cchar
(k)=0$. Then the tricanonial map $\Phi_{3K}$ of $S$ is a birational
morphism. 
\end{theorem}

If we combine Theorem \ref{art04-thm1} with the results of
Bombieri (\cite{art04-key1} referred in the sequel as {$[CM]$}) and
Miyaoka \cite{art04-key4} we deduce

\begin{theorem}\label{art04-thm2}
If $S$ is a minimal surface of general type over an algebraically
closed field $k$, $\cchar (k)=0$ and $m\geq 3$, then $\Phi_{mK}$ is a
birational map with exactly the following exceptions:
\begin{itemize}
\item[\rm(a)] $m=3$, $K^{2}=1$ and $p_{g}=2$, $K^{2}=2$ and $p_{g}=3$

\item[\rm(b)] $m=4$, $K^{2}=1$ and $p_{g}=2$.
\end{itemize}
\end{theorem}

We refer to Horikawa \cite{art04-key3} for a detailed and exhaustive
study of the surfaces in (a) and (b).\footnote[2]{Added in Proof. The
result of Theorem \ref{art04-thm2} has been independently obtained by
X. Benveniste, with a similar method.}

Our notation is as follows:

$S$ a minimal surface of general type, with $K^{2}=2$, $p_{g}=0$
(non-singular),

$K$ a canonical divisor on $S$,

$\omega_{D}$ the dualizing sheaf of a divisor $D>0$ on $S$, hence
$$
\omega_{D}\cong \mathscr{O}_{D}(D+K),
$$

$\mathscr{O}$ the structure sheaf of $S$,

$\mathscr{F}(-C-a_{1}x_{1}-\cdots a_{r}x_{r})$ the sheaf of germs of
sections of $\mathscr{F}$ vanishing on $C$ and on $x_{i}$ with
multiplicity $a_{i}$, 
\begin{align*}
& \mathscr{F}_{C}(-a_{1}x_{1}-\cdots-a_{r}x_{r})=\mathscr{F}/\mathscr{F}(-C-a_{1}x_{1}-\cdots-a_{r}x_{r}),\\
& |D-C-a_{1}x_{1}-\cdots-a_{r}x_{r}|\\
& =\text{Proj}~~ H^{0}(\mathscr{O}(D-C-a_{1}x_{1}-\cdots-a_{r}x_{r}))\subseteq
|D|p(D)\\
& p(D)=\frac{1}{2}(D^{2}+KD)+1
\end{align*}\pageoriginale
the arithmetic genus of $D$; $P_{m}$ the plurigenera of $S$.

\section{Two auxiliary results}\label{art04-sec2}

The first result of this section gives a necessary condition for an
invertible sheaf on a curve $C$ on a surface $X$ to have a
section. The technique of proof is one invented by C.P.~Ramanujam in
the study of numerically connected divisors.


\begin{alphprop}\label{art04-propA}
Let $C>0$ be a divisor on a smooth surface $X$ and let $\mathscr{L}$ be an invertible sheaf on $C$ with $H^{0}(C,\mathscr{L})\neq (0)$.

Then either
\begin{itemize}
\item[\rm(i)] $\deg_{C}\mathscr{L}\geqq 0$,

or

\item[\rm(ii)] we have $C=C_{1}+C_{2}$, $C_{i}>0$ with
$$
C_{1}C_{2}\leqq \deg_{C_{2}}(\mathscr{L}\otimes \mathscr{O}_{C_{2}}).
$$
\end{itemize}
Moreover, {\rm(ii)} holds whenever there is a section $s$ of
$\mathscr{L}$ vanishing on $C_{1}$ but not on any $C'$ with $C_{1}<C'<C$.
\end{alphprop}

\begin{proof}
Let $s\in H^{0}(C,\mathscr{L})$ and suppose that the restriction of
$s$ to any irreducible component of $C$ is never identically $0$. Let
$C=\sum n_{i}\Gamma_{i}$, $\Gamma_{i}$ irreducible, so that
$$
\deg_{C}\mathscr{L}=\sum
n_{i}\deg_{\Gamma_{i}}(\mathscr{L}\otimes \mathscr{O}_{\Gamma_{i}}). 
$$

If $\deg_{C}\mathscr{L}<0$, we would get
$\deg_{\Gamma_{i}}(\mathscr{L}\otimes \mathscr{O}_{\Gamma_{i}})<0$ for
some $i$, hence $\mathscr{L}\otimes \mathscr{O}_{\Gamma_{i}}$ would
have only the 0-section since $\Gamma_{i}$ is irreducible. This
however contradicts our initial assumption that the restriction of $s$
to $\Gamma_{i}$ is not identically 0, i.e. that $s$ is not in the
kernel of the restriction map
$$
H^{0}(C,\mathscr{L})\xrightarrow{\text{res}}
H^{0}(\Gamma_{i},\mathscr{L}\otimes \mathscr{O}_{\Gamma_{i}}).
$$

It follows that if conclusion (i) does not hold then there is $C_{1}$,
with $0<C_{1}<C$ such that $s$ is in the kernel of the restriction map 
$$
H^{0}(C,\mathscr{L})\xrightarrow{\text{res}}H^{0}(C_{1},\mathscr{L}\otimes \mathscr{O}_{C_{1}}).
$$\pageoriginale

According to Ramanujam's idea, we take a {\em maximal} $C_{1}$ and
verify two exact sequences of sheaves
\begin{gather*}
0\to \mathscr{O}_{C_{2}}\xrightarrow{s}\mathscr{L}\to \mathscr{L}/s\mathscr{O}_{C}\to
0\\
0\to \mathscr{F}\to \mathscr{L}/s\mathscr{O}_{C}\to \mathscr{L}\otimes \mathscr{O}_{C_{1}}\to 0
\end{gather*}
where $C_{1}+C_{2}=C$ and where the sheaf $\mathscr{F}$ is supported
at finitely many points. Now we take the total Chern class on $X$ of
the above sequences and find
$$
c(\mathscr{L})=c(\mathscr{O}_{C_{2}})c(\mathscr{L}/s\mathscr{O}_{C})=c(\mathscr{O}_{C_{2}})c(\mathscr{L}\otimes \mathscr{O}_{C_{1}})c(\mathscr{F})
$$
whence the equation
\begin{align*}
& 1+C+(C^{2}-\deg_{C}\mathscr{L})\\
&
=(1+C_{2}+C^{2}_{2})(1-\text{length~}\mathscr{F})(1+C_{1}+C^{2}_{1}-\deg_{C_{1}}\mathscr{L}) 
\end{align*}
in the Chow ring of $X$. The equation in degree 2 is simply
$$
C_{1}C_{2}+\text{~ length~ } \mathscr{F}=\deg_{C_{2}}\mathscr{L}
$$
and our result is prove.\hfill (QED)
\end{proof}

Our next result, the fact that the canonical system of an irreducible
curve has no base points is very classical, but since we could not
find an adequate reference we give a proof here.

\begin{alphprop}\label{art04-propB}
If $\Gamma$ is an irreducible Gorenstein curve and
$|\omega_{\Gamma}|\neq \varnothing$, then $|\omega_{\Gamma}|$ has no
base points. More generally,  a reduced point $p$ on a reducible curve
$C$ on a smooth surface is not a base point of $|\omega_{C}|$ if
either
\begin{itemize}
\item[\rm(i)] $p$ is simple on $C$ and belongs to a component $\Gamma$
with $p(\Gamma)\geqq 1$ or

\item[\rm(ii)] $p$ is singular and for every decomposition
$C=C_{1}+C_{2}$ one has 
$$
C_{1}C_{2}>(C_{1}\cdot C_{2})_{p},
$$
where $(C_{1}\cdot C_{2})_{p}$ is the intersection multiplicity of
$C_{1}$, $C_{2}$ at $p$.
\end{itemize}
\end{alphprop}

We begin by proving

\setcounter{dashlemma}{1}
\begin{dashlemma}\label{art04-lemB'}
Let\pageoriginale $p$ be a singular point of a curve $C$ lying on a
smooth surface $S$, let $m_{p}$ be the maximal ideal of $p$ and let
$\pi:\widetilde{C}\to C$ be a normalization of $C$ at $p$. Then
$\Hom(m_{p},\mathscr{O}_{C,p})$, can be embedded in the ring $A$ of
regular functions on $\pi^{-\perp}(p)$, i.e. $A=\oplus_{\pi(p')=p}\mathscr{O}_{\widetilde{C},p'}$.
\end{dashlemma}

\begin{proof}
Let $\varphi\in \Hom(m_{p},\mathscr{O}_{C,p})$ and let $x$, $y$ be
local parameters for $S$ at $p$. We denote by $f$ a local equation for
$C$ at $p$ and for $u\in \mathscr{O}_{p}$ we denote by $[u]$ its image
in $\mathscr{O}_{C,p}=\mathscr{O}_{p}/(f)$. We may and shall assume
that $[x]$ and $[y]$ are not 0-divisors in $\mathscr{O}_{C,p}$. The
homomorphism $\varphi$ is determined by the knowledge of
$\varphi([x])$ and $\varphi([y])$; let $\xi$ and $\eta$ be such that
$$
[\xi]=\varphi([x]),\quad [\eta]=\varphi([y]).
$$
We have $\varphi([xy])=[x][\eta]=[\xi][y]$ thus $[\eta]=[\xi][y]/[x]$
in the ring of quotients of $\mathscr{O}_{C,p}$. It follows that we
have a bijection between $\varphi$'s and classes $[\xi]$ such that
$$
[\xi][y]/[x]\in \mathscr{O}_{C,p},
$$
since for every $[z]\in m_{p}$ we may set
$$
\varphi([z])=[z][\xi]/[x].
$$

Let $p'\in \pi^{-1}(p)$ and let $t$ be a local parameter on
$\widetilde{C}$ at $p'$. Then we claim that
$$
\ord_{t}\pi^{*}([\xi]/[x])\geqq 0.
$$
In fact, suppose that
$\ord_{t}\pi^{*}[\xi]<\ord_{t}\pi^{*}[x]\leqq \ord_{t}\pi^{*}[y]$;
then we cannot have $\xi\in m_{p}$, hence $[\xi]$ is a unit and
$[y]=[x][\eta]/[\xi]$, which shows that the maximal ideal of
$\mathscr{O}_{C,p}$ is generated by $[x]$, i.e. $p$ is a regular point
of $C$. If instead $\ord_{t}\pi^{*}[y]<\ord_{t}\pi^{*}[x]$, there is
the same reasoning with $\eta$ and $y$ instead of $\xi$ and $x$,
because $[\eta]/[y]=[\xi]/[x]$.\hfill Q.E.D.
\end{proof}

Now we can prove Proposition \ref{art04-propB}.

If $p$ is a base point of $|\omega_{C}|$ the exact sequence
$$
0\to \omega_{C}m_{p}\to \omega_{C}\to k_{p}\to 0
$$
yields
$$
0\to k\to H^{1}(C,\omega_{C}m_{p})\to H^{1}(C,\omega_{C})\to 0.
$$

By\pageoriginale Grothendieck's Deuality we obtain
$$
0\leftarrow k\leftarrow \Hom(m_{p},\mathscr{O}_{C})\leftarrow
H^{0}(C,\mathscr{O}_{C})\leftarrow 0
$$
and
$$
\dim\Hom(m_{p},\mathscr{O}_{C})\geqq 2.
$$

In Case (i), $\Hom(m_{p},\mathscr{O}_{C})\cong
H^{0}(C,\mathscr{O}_{C}(p))$ hence
$\dim$. $H^{0}(\Gamma,\break\mathscr{O}_{C}(p))\geq 2$ and $p(\Gamma)=0$,
that is $\Gamma$ is rational non-singular.

In Case (ii), by Lemma \ref{art04-lemB'}$'$,
$\Hom(m_{p},\mathscr{O}_{C})$ embeds into\break
$H^{0}(\widetilde{C},\mathscr{O}_{\widetilde{C}})$, and the condition
of (ii) implies that $\widetilde{C}$ is numerically connected on a
smooth surface, thus $\dim
H^{0}(\widetilde{C},\mathscr{O}_{\widetilde{C}})=1$ by a result of
C.P.~Ramanujam \cite{art04-key6}; this contradicts
$\dim \Hom(m_{p},\mathscr{O}_{C})=2$.\hfill Q.E.D.

\section{The linear system}\label{art04-sec3}

$|2K-x-y|$. It is clear, since $P_{2}=3$, that for any two points $x$,
$y\in S$ the linear system $|2K-x-y|$ is non-empty. We denote by
$\Phi$ the tricanonical map $\Phi_{3K}$ and {\em assume that $\Phi$ is
not birational.} Hence let $x$ be a general point of $S$ and let $y$
be such that $\Phi(x)=\Phi(y)$. We denote by $C$ a divisor in
$|2K-x-y|$ and, in case $\dim|2K-x-y|=1$, we also choose $C$ as a
general element.

\begin{lem}\label{art04-lem1}
We have $h^1(\mathscr{O}_{C}(3K-x-y))=2$.
\end{lem}

\begin{proof}
Since $\Phi(x)=\Phi(y)$ the cohomology sequences of
\begin{align*}
& 0\to \mathscr{O}(3K-x-y)\to \mathscr{O}(3K)\to k_{x}\oplus k_{y}\to
0\\
&
0\to \mathscr{O}(3K-C)\to \mathscr{O}(3K-x-y)\to \mathscr{O}_{C}(3K-x-y)\to 0
\end{align*}
yield
\begin{align*}
h^{1}(\mathscr{O}_{C}(3K-x-y)) &=
h^{1}(\mathscr{O}(3K-x-y))+h^{2}(\mathscr{O}(3K-C))\\[3pt]
&= 1+h^{2}(\mathscr{O}(K))=2,
\end{align*}
because $h^1(\mathscr{O}(3K-C))=h^{1}(\mathscr{O}(K))=0$.\hfill Q.E.D.
\end{proof}

The curve $C$ may be reducible and we shall denote by $\Gamma$ an
irreducible component containing $x$ or $y$. We then have that
$K\Gamma\geqq 2$, because $K\Gamma\leqq 1$ would imply $\Gamma^{2}<0$
by the Index Theorem, while there are only finitely many such curves
on $S$ ([CM], page 176). Since $KC=4$ we deduce that either
$x$ and $y$ lie on only one component\pageoriginale $\Gamma$ of $C$ or
that they belong to exactly two components. Thus we distinguish two
cases:
\begin{itemize}
\item[(A)] $C=\Gamma_{1}+\Gamma_{2}+M$, where $K\Gamma_{i}=2$, $\Gamma^{2}_{i}\geqq 0$, $KM=0$ and $x\in \Gamma_{1}$, $y\in\Gamma_{2}$ (possibly $\Gamma_{1}=\Gamma_{2}$); 

\item[(B)] $C=\Gamma+M$ where $x$, $y\in \Gamma$ and $x$, $y\not\in M$.
\end{itemize}

\begin{lem}\label{art04-lem2}
If case {\rm(A)} holds then $C=C_{1}+C_{2}$ with $C_{1}\sim C_{2}\sim K$, $\Gamma_{1}\leqq C_{1}$, $\Gamma_{2}\leqq C_{2}$ and either
\begin{itemize}
\item[\rm(a)] $C_{1}\neq C_{2}$, in which case $\Gamma_{1}$ and
$\Gamma_{2}$ intersect transversally exactly at $x$ and $y$, or

\item[\rm(b)] $C_{1}=C_{2}$. 
\end{itemize}
\end{lem}

\begin{proof}
We shall prove Lemma \ref{art04-lem2} in several steps.
\begin{description}
\item[Step 1.] $x$ and $y$ belong to both $\Gamma_{1}$ and $\Gamma_{2}$.
\end{description}

In order to prove this, we shall assume that $x\not\in \Gamma_{2}$ and
derive a contradiction.

First of all, we have
$$
h^{0}(\mathscr{O}(3K-\Gamma_{2}-M-x))=h^{0}(\mathscr{O}(3K-\Gamma_{2}-M))-1.
$$
In fact, since $x\not\in \Gamma_{2}+M$, if the above assertion were
not true then we would have
$$
h^{0}(\mathscr{O}(K+\Gamma_{1}-x))=h^{0}(\mathscr{O}(K+\Gamma_{1}))
$$
i.e., $x$ would be a base point of $|K+\Gamma_{1}|$. Since $p_{g}=0$,
the restriction map gives an isomorphism
$$
H^{0}(\mathscr{O}(K+\Gamma_{1}))\xrightarrow{\sim}H^{0}(\omega_{\Gamma_{1}})
$$
and noting that $|K+\Gamma_{1}|$ is non-empty $(p(\Gamma_{1})>0)$ and
that $\Gamma_{1}$ is never a component of elements of $|K+\Gamma_{1}|$
(since $p_{g}=0$) we deduce that $x$ is a base point of
$H^{0}(\omega_{\Gamma_{1}})$; this however contradicts
Proposition \ref{art04-propB}.

Next, we note that
$$
h^{i}(\mathscr{O}(3K-\Gamma_{2}-M))=0
$$
for $i=1,2$, because
$$
h^{i}(\mathscr{O}(3K-\Gamma_{2}-M))=h^{i}(\mathscr{O}(K+\Gamma_{1}))=h^{2-i}(\mathscr{O}(-\Gamma_{1}))=0 
$$
for\pageoriginale $i=1,2$ (in fact, $\Gamma_{1}$ is connected and $S$
has irregularity $0$; then use ([CM], page 177). It now
follows from the cohomology sequence of
$$
0\to \mathscr{O}(3K-\Gamma_{2}-M-x)\to \mathscr{O}(3K-\Gamma_{2}-M)\to
k_{x}\to 0
$$
that
$$
h^{i}(\mathscr{O}(3K-\Gamma_{2}-M-x))=0\quad\text{for}\quad i=1,2.
$$
Using this result and the cohomology sequence of
$$
0\to \mathscr{O}(3K-\Gamma_{2}-M-x)\to \mathscr{O}(3K-x-y)\to \mathscr{O}_{\Gamma_{2}+M}(3K-y)\to 0
$$
we deduce that
$$
h^{1}(\mathscr{O}_{\Gamma_{2}+M}(3K-y))=h^{1}(\mathscr{O}(3K-x-y))=1,
$$
the latter equality because $\Phi(x)=\Phi(y)$
([CM], p.~187). 

By Grothendieck's duality we get
$$
\dim \Hom (\mathscr{O}_{\Gamma_{2}+M}(\Gamma_{1})\cdot m_{y}, \mathscr{O}_{\Gamma_{2}+M})=1.
$$
If $\pi:\widetilde{S}\to S$ gives an embedded resolution of
singularities of $\Gamma_{2}+M$ at $y$ and if
$\widetilde{\Gamma}_{2}+M$ is the corresponding curve on
$\widetilde{S}$ then denoting by $\mathscr{L}$ the sheaf
$$
\mathscr{L}=
\begin{cases}
\mathscr{O}_{\widetilde{\Gamma}_{2}+M}(-\pi^{*}\Gamma_{1}) & \text{if $y$
is singular}\\
\mathscr{O}_{\widetilde{\Gamma}_{2}+M}(-\Gamma_{1}+y) & \text{if $y$
is simple}
\end{cases}
$$
we obtain $H^{0}(\widetilde{\Gamma}_{2}+M,\mathscr{L})\neq (0)$, as one can
see using Lamme \ref{art04-lemB'}$'$. Since $\deg \mathscr{L}\leqq
1-\Gamma_{1}(\Gamma_{2}+M)<0$ (note that $\Gamma_{1}+\Gamma_{2}+M\in
|2K|$ is 2-connected,  [CM], p.~181) we can apply
Proposition \ref{art04-propA} and obtain a decomposition
$$
\widetilde{\Gamma}_{2}+M=A+B
$$
where
\begin{align*}
AB & \leqq \deg_{B}(\mathscr{L}\otimes \mathscr{O}_{B})\\
 & \leqq 1-B\pi^{*}\Gamma_{1}.
\end{align*}
This\pageoriginale leads to a contradiction: in fact, let
$\pi^{*}\Gamma_{2}=\widetilde{\Gamma}_{2}+H$; if $\widetilde{\Gamma}_{2}\subset
B$ we get $(B+H)(A+\pi^{*}\Gamma_{1})\leq 1$, while if
$\widetilde{\Gamma_2}\subset A$, then $B(A+H+\pi^{*}\Gamma_{1})\leqq 1$ and
in both cases one violates the 2-connectedness of
$\Gamma_{1}+\Gamma_{2}+M$ ([CM], p.~81).

\begin{description}
\item[Step 2.] The cohomology sequence of
{\fontsize{10pt}{12pt}\selectfont
$$
0\to \mathscr{O}(3K-\Gamma_{2}-M)\to \mathscr{O}(3K-x-y)\to \mathscr{O}_{\Gamma_{2}+M}(3K-x-y)\to 0
$$}
together with
$$
h^{i}(\mathscr{O}(3K-\Gamma_{2}-M))=h^{i}(\mathscr{O}(K+\Gamma_{1}))=h^{2-i}(\mathscr{O}(-\Gamma_{1}))=0
$$
for $i=1,2$ gives
$$
h^{1}(\mathscr{O}_{\Gamma_{2}+M}(3K-x-y))=1.
$$
\end{description}

By Grothendieck's duality one deduces
$$
\dim\Hom(\mathscr{O}_{\Gamma_{2}+M}(\Gamma_{1})m_{x}m_{y}, \mathscr{O}_{\Gamma_{2}+M})=1 
$$
and, again by Lemma \ref{art04-lemB'}$'$, we find that if
$\pi:\widetilde{S}\to S$ is an embedded resolution of singularities of
$\Gamma_{2}+M$  $st \ x$ and $y$ then
$H^{0}(\widetilde{\Gamma}_{2}+M,\mathscr{L})\neq (0)$ where $\mathscr{L}$
is the sheaf
$$
\mathscr{L}=\mathscr{O}_{\widetilde{\Gamma}_{2}+M}(-\pi^{*}\Gamma_{1}+ax+by)
$$
where $a$, $b=1$ or $0$ according as whether $x$, $y$ are simple or
singular on $\Gamma_{2}$. Since $\Gamma_{1}+\Gamma_{2}+M$ is
2-connected we have
$$
\deg \mathscr{L}\leqq a+b-2.
$$

Two cases can occur:
\begin{itemize}
\item[(A)] $\deg\mathscr{L}=0$.

Then $a=b=1$ and $x$, $y$ are simple on $\Gamma_{2}$ and
$\Gamma_{1}(\Gamma_{2}+M)=2$; by Lemma \ref{art04-lem2} of [CM], p.~181 this implies
$$
\Gamma_{2}\sim \Gamma_{1}+M\sim K.
$$
Moreover in this case $\Gamma_{1}\Gamma_{2}=2$ hence if
$\Gamma_{1}\neq \Gamma_{2}$ the two curves $\Gamma_{1}$, $\Gamma_{2}$
intersect transversally exactly at $x$ and $y$.

\item[(B)] $\deg \mathscr{L}<0$.
\end{itemize}
Now we apply Proposition \ref{art04-propA} and deduce that there is a
decomposition 
$$
\widetilde{\Gamma}_{2}+M=A+B
$$\pageoriginale
where 
\begin{align*}
AB & \leqq \deg_{B}(\mathscr{L}\otimes \mathscr{O}_{B})\\
& \leqq -B\pi^{*}\Gamma_{2}+a+b.
\end{align*}
This however implies, exactly as at the end of Step 1, that $a=b=1$
and
$$
(A+\Gamma_{2})B=2,
$$
whence by Lemma \ref{art04-lem2} of [CM], p.~181 one finds again that
$x$, $y$ are simple points of $\Gamma_{2}$ and
$$
A+\Gamma_{2}\sim B\sim K.
$$
Since $KB=2$ this implies that $\Gamma_{1}$ is a component of $B$,
$x$, $y$ are simple on $\Gamma_{1}$ and
$\Gamma_{1}\Gamma_{2}=2$. Finally, if $\Gamma_{1}=\Gamma_{2}$ and if
$B=B'+\Gamma_{1}$ then $A+\Gamma_{2}\sim B'+\Gamma_{1}$, hence $A\sim
B'$ and $A=B'$ ([CM], p.~175).\hfill Q.E.D.
\end{proof}

\begin{lem}\label{art04-lem3}
Case {\rm (A)} does not hold.
\end{lem}

\begin{proof}
Let $x$, $y$ and $C=C_{1}+C_{2}\in |2K-x-y|$ be as in
Lemma \ref{art04-lem2}. Recall that $x$, $y$ was a general pair of
points with $\Phi_{3K}(x)=\Phi_{3K}(y)$ and $C$ a general element with
$C\in |2K-x-y|$. By Lemma \ref{art04-lem2}, there is a torsion class
$\tau$ such that $C_{1}\in |K+\tau|$; since the number of torsion
classes is finite, and since $x$ can be taken as a general point of
the surface $S$ we conclude that
$$
\dim |K\pm \tau|\geqq 1.
$$
Now $|K+\tau|+|K-\tau|\subseteq |2K|$ and $\dim |2K|=P_{2}-1=2$, thus
we deduce that $|2K|$ is composite of a pencil and
$|K+\tau|=|K-\tau|$, i.e. $\tau$ is a 2-torsion class. We have shown
that $C_{1}$, $C_{2}\in |K+\tau|$, since $x$, $y\in C_{1}$ by
Lemma \ref{art04-lem2}, we see that if $C'_{2}$ is a general element
of $|K+\tau|$ then $C_{1}+C'_{2}$ is a general element in
$|2K-x-y|$. By Lemma \ref{art04-lem2} again, we obtain that $x$, $y\in
C'_{2}$, i.e. $x$, $y$ are base points of $|K+\tau|$. This is plainly
impossible because $x$ was a general point on $S$.
\end{proof}

\begin{lem}\label{art04-lem4}
The points $x$, $y$ are simple points of $C$.
\end{lem}

\begin{proof}
By\pageoriginale Lemma \ref{art04-lem1}, $h^{1}(\mathscr{O}_{C}(3K-x-y))=2$ and
Grothendieck's duality yields
$$
\dim \Hom(m_{x}m_{y},\mathscr{O}_{C})=2.
$$
Denoting by $\pi:\widetilde{S}\to S$ an embedded resolution of
singularities of $C$ at $x$ and $y$, by Lemma \ref{art04-lemB'}$'$ we
see that $x$, $y$ cannot both be singular, otherwise we would have
$h^{0}(\mathscr{O}_{\widetilde{C}})=2$, while $\widetilde{C}$ is
connected exactly as $C$ (and now use Ramanujam's result
in  [CM], p.~177).

If, say, $x$ is simple and $y$ singular we get
$h^{0}(\mathscr{O}_{\widetilde{C}}(x))=2$ and this implies that
$\Gamma$, the component of $C$ with $x\in \Gamma$, is a rational
curve. This also is impossible, because $S$ is of general type. Q.E.D.
\end{proof}

From now onwards we shall suppose that $C\in |2K-x-y|$ satisfies the
requirements of Lemma \ref{art04-lem3} and Lemma \ref{art04-lem4}, and
write $h_{C}=\mathscr{O}_{C}(x+y)$ for the hyperelliptic sheaf of $C$.

\begin{lem}\label{art04-lem5}
We have $\omega_{C}\cong h^{\otimes 6}_{C}$.
\end{lem}

\begin{proof}
Obvious, because $C$ is hyperelliptic.
\end{proof}

\section{Proof of Theorem \ref{art04-thm1}}\label{art04-sec4}

As $P_{3}=7$ we have $\Phi=\Phi_{|3K|}:S\to I\bfP^{6}$. We write
$V=\Phi(S)$. $d=\deg V$, $m=\deg \Phi$ and note that, since $|3K|$ has
no base points (\cite{art04-key5}Th.~A; see
also \cite{art04-key2}Th.~5.1) we have
$$
dm=(3K)^{2}=18.
$$
Also $d\geqq 5$ because $V$ is not contained in any hyperplane and $V$
is not a curve, the latter because otherwise the general element of
$|3K|$ would be decomposable in $d$ components, while the curves $D$
on $S$ with $KD\leqq 1$ cannot move. This leads to the two cases
$\deg \Phi=2$ and $\deg\Phi=3$.

\begin{description}
\item[Case I.] $\deg \Phi=2$.
\end{description}

Now $\Phi$ determines an involution $\sigma$ on $S$, which is
everywhere regular because $S$ is of general type. We have
$$
\sigma^{*}\mathscr{O}(K)\cong \mathscr{O}(K)
$$
(this\pageoriginale clearly holds for every surface of general type)
and we remark that, $C$ being as in Section III, we also have
$$
\sigma(C)=C.
$$

In fact, $\sigma$ identifies pairs of points $x$, $y$ such that
$\Phi(x)=\Phi(y)$; hence the above remark.

\begin{lem}\label{art04-lem6}
We have $\mathscr{O}_{C}(K)\cong h^{\otimes 2}_{C}$.
\end{lem}

\begin{proof}
By Lemma \ref{art04-lem5} we have $\mathscr{O}_{C}(3K)=\omega_{C}\cong
h^{\otimes 6}_{C}$ therefore it is sufficient to prove that
$\mathscr{O}_{C}(8K)\cong h^{\otimes 16}_{C}$. In fact, since $|4K|$
is free from base points, we can find a section $s\in
H^{0}(S,\mathscr{O}(4K))$ with 16 zeros $a_{1},\ldots,a_{16}$ on
$C$. Then $\sigma^{*}s$ has 16 zeros
$\sigma(a_{1}),\ldots,\sigma(a_{16})$ on $C$ and the section
$s$. $\sigma^{*}s$ of $\mathscr{O}(8K)$ has the zeros $a_{i}$,
$\sigma(a_{i})$ on $C$. It follows that the section $s\sigma^{*}
s|_{C}$ of $\mathscr{O}_{C}(8K)$ has divisor 
$$
\text{div} (s\cdot \sigma^{*}s|_{C})=\sum^{16}_{i=1}(a_{i}+\sigma(a_{i})).
$$
Since each $a_{i}+\sigma(a_{i})$ is the divisor of a section of
$h_{C}$, the result follows.\hfill Q.E.D.
\end{proof}

By Lemma \ref{art04-lem6}, we have an exact sequence
$$
0\to \mathscr{O}(-K)\to \mathscr{O}(K)\to h^{\otimes 2}_{C}\to 0
$$
which implies $p_{g}=3$, a contradiction. This settles Case I.

\begin{description}
\item[Case II.] $\deg \Phi=3$.
\end{description}

Let $A$ be a general point on $C$ and let
$$
a+\overline{a}\in |h_{C}|.
$$
Then $\Phi(\overline{a})=\Phi(\overline{a})$ and, if $a'$ is the third
point with $\Phi(a)=\Phi(\overline{a})=\Phi(a')$ then $a'\not\in C$,
as one immediately sees by considering $\overline{a}'$ in case that
$a'\in C$. Consider $\Phi(C)=N$. Now $\Phi^{-1}(N)$ has a component
$N'=$ locus $(a')$; $\Phi|_{N'}:N'\to N$ is a birational map. However,
points in $N'=$ locus $(a')$ are parametrized by
$|h_{C}|\cong \bfP^{1}$, because $a+\overline{a}$ determines $a'$
uniquely; hence $N'$ is rational and $S$ contains a continuous family
of rational curves, which is absurd. This settles Case II and
completes the proof of Theorem \ref{art04-thm1}.

\begin{thebibliography}{99}
\bibitem{art04-key1} Bombieri,\pageoriginale E.: Canonical Models of
Surfaces of General Type {\em Publ. Math.} IHES 42, 171-219.

\bibitem{art04-key2} Bombieri, E.: The Pluricanonical Map of a Complex
Surface, Several Complex Variables I, Maryland (1970) {\em Springer
Lecture Notes in Math.} No.~155, 35-87.

\bibitem{art04-key3} Horikawa, E.: Algebraic Surfaces of General Type
with Small $c^{2}_{1}$ Part I: {\em Ann. of Math.} 104 (1976) 357-387
and Part II: {\em Inventiones Math.} 37 (1976), 121-155.

\bibitem{art04-key4} Miyaoka, Y.: Tricanonical Maps of Numerical
Godeaux Surfaces, {\em Inventiones Math.} 34 (1976), 99-111.

\bibitem{art04-key5} Miyaoka, Y.: On Numerical Campedelli Surfaces,
{\em Complex Analysis and Algebraic Geometry,} Cambridge Univ. Press
(1977), 113-118.

\bibitem{art04-key6} Ramanujam, C.P.: Remarks on the Kodaira Vanishing
Theorem {\em J. Ind. Math. Soc.} 36 (1972), 41-51.

\end{thebibliography}










