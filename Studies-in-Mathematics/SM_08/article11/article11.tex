\title{On a Geometric Interpretation of Multiplicity}\label{art11}
\markright{On a Geometric Interpretation of Multiplicity}

\author{By~ C.P. Ramanujam (Madras)}
\markboth{C.P. Ramanujam}{On a Geometric Interpretation of Multiplicity}

\date{}
\maketitle

\setcounter{page}{151}
\setcounter{pageoriginal}{118}
The\pageoriginale interpretation is as follows. Let $Y$ be a projective variety of dimension $n$, $y$ a point of $Y$, $\mathscr{Q}$ an ideal in the local ring $\mathscr{O}_{Y,y}$ primary for the maximal ideal, and $f: X \to Y$ a proper birational morphism of a non-singular variety $X$ into $Y$ such that $\mathscr{Q}\mathscr{O}_X$ is an invertible sheaf of ideals of $X$. Then the multiplicity of $\mathscr{Q}$ in $\mathscr{O}_{Y,y}$ equals $(-1)^{n-1} (D^n)$, where $D$ is the effective divisor on $X$ defined by $\mathscr{Q} \mathscr{O}_X$.

Following a suggestion by Bombieri, we present a generalisation of this result to sheaves of ideals defining closed subsets of arbitrary dimension. Also, we have given the result in the context of schemes.

{\em All schemes considered will be noetherian and seperated, and all morphisms considered seperated of finite type.}

First, we recall some basic facts on the blow-up of a scheme with respect to a coherent sheaf of ideals. (See \cite{art11-key1}.) Let $Y$ be a scheme and $\mathscr{I}$ a coherent sheaf of ideals in $\mathscr{O}_Y$, defining a closed subscheme $Z$ of $Y$. Let $\mathscr{R} (\mathscr{I})$ be the quasi-coherent sheaf of graded $\mathscr{O}_Y$-algebras defined by
$$
\mathscr{R} (\mathscr{I}) = \mathscr{O}_Y \oplus \mathscr{I} \oplus \mathscr{I}^2 \oplus \ldots.
$$
Then the blow-up of $Y$ with respect to $\mathscr{I}$ or the blow-up of $Z$ on $Y$ is defined to be the $Y$-scheme $X = \Proj \mathscr{R}(\mathscr{I})$. If $Y$ is affine with $\Gamma (Y, \mathscr{O}_Y) = A$ and $\Gamma (Y, \mathscr{I}) = I$, $X$ is covered by affine open sets isomorphic to $\Spec I \cdot f^{-1}$ where $f$ runs through the elements (or even a set of generators) of $I$ and $I \cdot f^{-1}$ is the subring of the of quotients $A_f = A[f^{-1}]$ consisting of elements of the form $g/f^n$, $g \in I^n$. If $\varphi: X \to Y$ is the structural morphism, $\mathscr{I} \mathscr{O}_X$ is an invertible sheaf of ideals of $\mathscr{O}_X$ which is very ample for $\varphi$, and the restriction of $\varphi$ to $\varphi^{-1}(Y-z)$ is an isomorphism of this open subscheme onto $Y-Z$. Further, if $U$ is any non-void open subset of $X$, it cannot happen that $\varphi (U) \subset Z$, since then $\mathscr{I} \cdot \mathscr{O}_U$ would be contained in the sheaf of nilpotents of $\mathscr{O}_U$ and hence cannot be invertible. Hence, if $Z$ contains no components of $Y$, or equivalently if $Y - Z$ is dense in $Y, \varphi$ maps an open dense subset of $X$ isomorphically onto an open dense subset of $Y$, i.e., $\varphi$ is birational.

Now, let $\psi:Y' \to Y$ be a morphism such that $\mathscr{I} \mathscr{O}_{Y'}$ is invertible. We assert that there is a unique morphism $\tilde{\psi} : Y' \to X$ such that $\varphi \circ \tilde{\psi} = \psi$. To prove\pageoriginale this, we may clearly assume that $Y = \Spec A $ and $Y' = \Spec B$ are affine and that if $I = \Gamma (Y, \mathscr{I})$, $IB = f \cdot B$ for some $f \in I$, and also that the image of $f$ in $B$ is a non-zero divisor. Then $B \to B_f$ is injective, and under the induced homomophism $A_f \to B_f$, the image of $I \cdot f^{-1}$ is contained in $B$. This proves the assertion. Finally, let $\xi$ be the generic point of a component of the closed subset of $Y'$ defined by $\mathscr{I} \mathscr{O}_{Y'}$, so that $\mathscr{O}_{\xi, Y'}$ is a local ring of dimension one. Since $(\mathscr{I} \mathscr{O}_{Y'})_{\xi}$ is a principal ideal generated by a non-zero divisor contained in the  maximal ideal of $\mathscr{O}_{\xi, Y'}$, the maximal ideal of $\mathscr{O}_{\xi, Y'}$ is not associated to (0). Hence if $\eta_1, \eta_2, \ldots, \eta_p $ are generic points of components of $Y'$ containing $\xi$, the natural homomorphism $\mathscr{O}_{\xi, Y'} \to \prod\limits^{p}_1 \mathscr{O}_{\eta_i, Y'}$ is injective.

Next we recall the definitions and some basic facts about the degree of an invertible sheaf. Let $m$ be an integer $\geqq 0$, and $X$ a proper scheme over Spec $\Lambda$, where $\Lambda$ is an artinian ring, with $\dim X \leqq m$. Then, if $\mathscr{L}$ is an invertible sheaf on $X$, and $\mathscr{I}$ any coherent sheaf on $X$,
$$
P(n) = \sum\limits^{n}_{i=0} (-1)^i l_\Lambda (H^i (X, \mathscr{I} \otimes \mathscr{L}^n))
$$
is a polynomial in $n$ of degree $\leqq m$. This is well-known when $\Lambda$ is a field (see for instance \cite{art11-key2}), and the general case follows easily. The coefficient of $n^m$ in $\chi (\mathscr{L}^n)$ is of the form $\dfrac{d_m (\mathscr{L})}{m!}$, where $d_m (\mathscr{L})$ is an integer called the degree of $\mathscr{L}$ (the integer $m$ being fixed once for all). This has the following properties:
\begin{itemize}
\item[(i)] If $X_i$ are the irreducible components of $X$ with reduced structure and $\mathscr{L}_i = \mathscr{L} \otimes_{\mathscr{O}_X} \mathscr{O}_{X_i}$ and $\xi_i$ is the generic point of $x_i$, we have
$$
d_m (\mathscr{L}) = \sum\limits_i l(\mathscr{O}_{X, \xi_i}) d_m (\mathscr{L}_i).
$$

\item[(ii)] If $f: Y \to X$ is a proper morphism, $X$ irreducible and reduced of dimension $m$ and $\dim Y \leqq m$, and if $\xi$ is the generic point of $X$,
$$
d_m (f^* (\mathscr{L})) = (\sum\limits_{f (\eta) =\xi} \dim _{k(\xi)} \mathscr{O}_{Y,\eta}) \cdot d_m (\mathscr{L}).
$$

We can now state the main theorem.
\end{itemize}

\begin{theorem*}
Let $X$ be a noetherian scheme of dimension $n$ and $Y$ a closed subscheme of $X$ defined by the coherent sheaf of ideals. $\mathscr{I}$. Suppose $Y$ is a proper scheme over Spec $\Lambda$  where $\Lambda$ is an artinian ring. Suppose further that $f: X' \to X$ is a proper birational morphism such that $\mathscr{I} \mathscr{O}_{X'}$ is an invertible sheaf of ideals. Then we have:
\begin{itemize}
\item[(i)] there exists a polynomial $P(T)$ of degree $\leqq n -1$ such that for all large $N$,\pageoriginale
$$
P(N) = \sum\limits^n_{i=0} (-1)^i l_\Lambda (H^i (X, \mathscr{I}^N / \mathscr{I}^{N+1})),
$$

\item[(ii)] the coefficient of $T^{n-1}$ in $P(T)$ is $\dfrac{1}{(n-1)!} \cdot d_{n-1} (\mathscr{I} \mathscr{O}_{X'})$.
\end{itemize}
\end{theorem*}

\begin{proof}
Let $g: \bar{X} \to X$ be the blow-up of $X$ with respect to the sheaf of ideals $\mathscr{I}$. Then $\mathscr{I}\mathscr{O}_{\bar{X}}$ is an invertible sheaf relatively very ample for $g$, hence $\mathscr{I} \mathscr{O}_{\bar{X}} / \mathscr{I}^2 \mathscr{O}_{\bar{X}}$ is relatively very ample for the morphism $g^{-1} (Y) = Y \times_X \bar{X} \to Y$. It follows that for $N$ large $R^i g (\mathscr{I}^N \mathscr{O}_{\bar{X}} / \mathscr{I}^{N+1} \mathscr{O}_{\bar{X}}) = (0)$ for $i> 0$ and $g_* (\mathscr{I}^N \mathscr{O}_{\bar{X}}/ \mathscr{I}^{N+1} \mathscr{O}_{\bar{X}}) \simeq \mathscr{I}^N / \mathscr{I}^{N+1}$. Hence from the Leray spectral sequence, for $N$ large we have 
{\fontsize{10pt}{12pt}\selectfont
$$
\sum\limits^n_{i=0} (-1^i) l_\Lambda (H^i (X, \mathscr{I}^N/\mathscr{I}^{N+1})) = \sum\limits^n_{i=0} (-1)^i l_\Lambda (H^i (Y \times_X \bar{X}, \mathscr{I}^N \mathscr{O}_{\bar{X}} / \mathscr{I}^{N+1} \mathscr{O}_{\bar{X}}))
$$}\relax
and the right side is a polynomial of degree $\leqq \dim Y \times_X \bar{X} < \dim \bar{X} \leqq n$ (\cite{art11-key1}). This proves (i), and also (ii) for the special morphism $g$. 

Now, $f$ factorises as $g \circ h$ where $h : X' \to \bar{X}$ is birational proper. We have to show that $d_{n-1} (\mathscr{I}\mathscr{O}_{X'}) = d_{n-1} (\mathscr{I} \mathscr{O}_{\bar{X}})$. Since $\mathscr{I} \mathscr{O}_{\bar{X}}$ is an invertible sheaf of ideals, we have $h^* (\mathscr{I} \mathscr{O}_{\bar{X}}) \simeq \mathscr{I} \mathscr{O}_{X'}$. Put $\bar{Y} = Y \times_X \bar{X}$, $Y' = Y \times_X X'$; in view of the preceding remarks, it suffices to show that if $\eta$ is the generic point of an irreducible component of $\bar{Y}$ of dimension $n-1$, we have 
$$
l(\mathscr{O}_{\bar{Y}, \eta}) = \sum\limits_{\substack{\eta' \in Y'\\ h (\eta') = \eta}} l_{\mathscr{O}_{\bar{Y}, \eta}} (\mathscr{O}_{{Y'}, \eta'}).
$$
Set $A = \mathscr{O}_{\bar{X},\eta}$. Then the morphism $X' \times_X \Spec A \to \Spec A$ is proper with finite fibers, hence is a finite morphism. Thus, $X' \times_X \Spec A$ is isomorphic over $\Spec A$ to $\Spec B$ where $B$ is an $A$-algebra which is a finite $A$-module. Let $f \in A$ be a generator of $(\mathscr{I} \mathscr{O}_{\bar{X}})_\eta$. Then $\mathscr{O}_{\bar{Y},\eta} \simeq A / fA$ and
$$
\prod\limits_{\substack{\eta' \in Y' \\ h (\eta') = \eta}} \mathscr{O}_{Y', \eta'} \simeq B / f B.
$$
We have thus to show that $l_A (A/ fA) = l_A (B/fB)$. Assume for the moment that $A \to B$ is injective and $l_A (B/A)< \infty$. Then we have 
$$
l_A (B / fA) = l_A (B / fB) + l_A (f B / fA) = l_A (B/A) + l_A (A / fA). 
$$
But since $f$ is a non-zero divisor in $B$, multiplication by $f$ induces an isomorphism $B/A \simeq f B /fA$ and $l_A (B/A)= l_A (fB/fA)$. Inserting this in the above equality, the desired result follows.

To prove that $A \to B$ is an injection and $l_A (B/A) < \infty$, let $K$ be the product of the (artinian) local rings of generic points of those components of $X$ which contain $\eta$ and $L$ the product of the (artinian) local rings of generic points\pageoriginale of those components of $X$ which contain a point of $h^{-1}(\eta)$. The natural homomorphisms $A \to K$ and $B \to L$ are injection in view of the remarks made earlier. Further, $K \to L$ is an isomorphism since $h$ is birational. Hence $B$ can be identified with a finitely generated submodule of the total quotient ring $K$ of $A$. Hence there is a non-zero divisor $\lambda \in A$ such that $\lambda B \subset A$, and $B/A \simeq \lambda B / \lambda  A \hookrightarrow A / \lambda A$ which is of finite length over $A$.

The proof of the theorem is complete.
\end{proof}

\begin{remarks*}
\begin{itemize}
\item[(1)] Suppose $A$ is a local ring of dimension $n \geqq 1$ and $\mathscr{Q}$ an ideal primary for the maximal ideal. The theorem is then applicable to $\Spec A$, $Y$ the closed subscheme defined by $\mathscr{Q}$ (and $\Lambda = A / \mathscr{Q}$). The polynomial $P(N)$ then equals $l_A (\mathscr{Q}^N / \mathscr{Q}^{N+1})$ for $N$ large, and its leading coefficient is $\dfrac{1}{(n-1)!} e_{\mathscr{Q}} (A)$, where $e_\mathscr{Q}(A)$ is the multiplicity of $\mathscr{Q}$.

Suppose now that $\mathscr{Q}, \mathscr{Q}'$ are two ideals primary for the maximal ideal and $g: X_1 \to X$ a proper birational morphism such that $\mathscr{Q}\mathscr{O}_{X_1} = \mathscr{Q}' \mathscr{O}_{X_1}$. We can then find $h: X' \to X_1$ proper birational such that $\mathscr{Q} \mathscr{O}_{X'} = \mathscr{Q}' \mathscr{O}_{X'}$ is invertible. (Take for instance $X'$ to be the blow-up of $X_1$ with respect to $\mathscr{Q} \mathscr{O}_{X_1}$.) It follows from the theorem applied to $g \circ h$ that $e_\mathscr{Q} (A) = e_{\mathscr{Q}'} (A)$. Now recall that if $\mathscr{Q}' \supset \mathscr{Q}$, $\mathscr{Q}'$ is said to be integral over $\mathscr{Q}$ if every $x$ in $\mathscr{Q}'$ satisfies an equation $x^n + a_1 x^{n-1} + \cdots + a_n = 0$ with $a_i \in \mathscr{Q}^i$ (see \cite{art11-key3}). We assert that if $\mathscr{Q}'$ is integral over $\mathscr{Q}$, there is a $g: X_1 \to X$ proper birational such that $\mathscr{I}\mathscr{O}_{X_1}=\mathscr{I'}\mathscr{O}_{X_2}$. Infact, choose $X_1$ so that $\mathscr{Q}\mathscr{O}_{X_1}$ and  $\mathscr{Q}' \mathscr{O}_{X_1}$ are both invertible. For any $x_1\in X_1$, we have
$$
\mathscr{Q} \mathscr{O}_{X_1, x_1} = \lambda \mathscr{O}_{X_1, x_1} \subset \mathscr{Q}' \mathscr{O}_{X_1, x_1} = \mu \mathscr{Q}_{X_1, x_1}
$$
with $\lambda$, $\mu \in \mathscr{Q}_{X_1, x_1}$, $\lambda = \mu v$, $v \in \mathscr{O}_{X_1, x_1}$. Now $\mu$ must be integral over $\lambda \mathscr{O}_{X_1,x_1}$, hence satisfies an equation
$$
\mu^m + a_1 \mu v \cdot \mu^{m-1} + a_2 (\mu v)^2 \mu^{m-2} + \cdots + a_m (\mu v)^m = 0, \quad a_i\in \mathscr{O}_{X_1, x_1},
$$
i.e.,
$$
\mu^m (1 + a_1 v + a_2 v^2 + \cdots + a_m v^m) =0.
$$
Since $\mu$ is not a zero divisor, it follows that $1+ v (a_1 + a_2 v + \cdots + a_m v^{m-1}) = 0$, and $v$ is  a unit. 

Thus, if $\mathscr{Q}'$ is integral over $\mathscr{Q}$, $e_\mathscr{Q}(A) = e_\mathscr{Q}'(A)$. This is a result due to Northcott and Rees (\cite{art11-key4}).

\item[(2)] Let the assumptions be as in the theorem. Suppose further that $Y$ is a local complete intersection of codimension $r$ in $X$, i.e., suppose for every $y \in Y$, $\mathscr{I}_y$ is generated by an $\mathscr{O}_{X,y}$-sequence of length $r$. Then $\mathscr{I}/ \mathscr{I}^2$ is locally free of rank $r$ on $Y$ and $\oplus_{m \geqq 0} \mathscr{I}^m / \mathscr{I}^{m+1}$ is isomorphic to the symmetric algebra $S (\mathscr{I}/ \mathscr{I}^2)$ over $\mathscr{O}_Y$ (see \cite{art11-key5}). Then $\chi (\mathscr{I}^N / \mathscr{I}^{N+1}) = P(N)$ where $P$ is a polynomial, for all $N \geqq 0$.
\end{itemize}
\end{remarks*}

Now,\pageoriginale suppose further that a theory of Chern classes is defined on $Y$ with values in a graded ring $A(Y)$ with the usual properties. (For instance, $\Lambda = \mathbb{C}$, $Y$ reduced and $A(Y)$ is the even dimensional integral cohomology of $Y (\mathbb{C})$; or $\Lambda$ an algebraically closed field, $Y$ non-singular projective over $\Lambda$ and $A (Y)$ the Chow ring of rational equivalence; or any $\Lambda$, and $A(Y)$ is the associated graded ring of $K(Y)$ with respect to the $\lambda$-filtration, tensored with $\mathbb{Q}$.) Since $\dim Y \leqq n - r$, we have a homomorphism $A^{n-r} (Y) \to \mathbb{Q}$ which we denote by $\xi \mapsto \xi [Y]$. There is then a universal polynomial $Q$ (depending only on $n$ and $r$) in the Chern classes $c_i$ of $\mathscr{I}/ \mathscr{I}^2$ such that the coefficient of $N^{n-1} $ in $P(N)$ equals $Q(c_i) [Y]$.

These are general assertions having to do with a locally free sheaf $\xi$ ($(\mathscr{I}/ \mathscr{I}^2)$ in our case) of rank $r$ and its symmetric powers on a scheme $Y$ of dimension $m$ proper over $\Spec \Lambda$, $\Lambda$ an artinian ring. To prove them, let $\mathbb{P} (\xi^*)$ be the projective bundle associated to the dual sheaf $\xi^*$ on $Y$, and $\mathscr{O}(1)$ the canonical invertible sheaf on $\mathbb{P}(\xi^*)$. If $\pi: \mathbb{P}(\xi^*) \to Y$ is the projection, we have $R^i \pi (\mathscr{O} (N)) = (0)$, $i>0$, $N \geqq 0$ and $\pi_* (\mathscr{O}(N)) \simeq S^N (\xi)$. Hence by the Leray spectral sequence, $\chi (Y,S^N (\xi)) = \chi (\mathbb{P} (\xi^*)$, $\mathscr{O} (N))$, which shows that $\chi (Y, S^N (\xi))$ is a polynomial in $N$. Further, if $\xi$ is the first Chern class of $\mathscr{O}(1)$ on $\mathbb{P} (\xi^*)$, the coefficient of $N^{r+m-1}$ in this polynomial is $\dfrac{1}{(r+m-1)!} \xi^{r+m-1} [\mathbb{P} (\xi^*)]$. Now, $\xi$ satisfies an equation $\xi^r - c_1 \xi^{r-1}+ \cdots \pm c_r = 0 $, where $c_i$ are the Chern classes of $\mathscr{E}$ and $A (\mathbb{P} (\mathscr{E}^*))$ is considered as an $A (Y)$-algebra via $\pi$. Dividing $\xi^{r+m-1}$ by $\xi^{r} - c_1 \xi^{r-1} + \cdots \pm c_r$ leaves a remainder
$$
\alpha_0 + \alpha_1 \xi + \cdots + \alpha_{r-1} \xi^{r-1}
$$
where the $\alpha_i$ are universal polynomials in $c_1, \ldots, c_r$. Further, by the projection formula, $\alpha_i \xi^i [\mathbb{P} (\mathscr{E}^*)] =0$ for $i < r- 1$ and $\alpha_{r-1} \xi^{r-1} [\mathbb{P}(\mathscr{E}^*)] = \alpha_{r-1} [Y]$. Thus, we may take $Q = \dfrac{1}{ (r + m-1)} \alpha_{r-1}$.
 
\begin{thebibliography}{99}
\bibitem{art11-key1} Grothendieck, A., Dieudonn\'e, J.: Elements de geometrie algebrique. Publ. I.H.E.S.

\bibitem{art11-key2} Kleiman, S.: Toward a numerical theory of ampleness. Ann. of Maths. 84 (1966)

\bibitem{art11-key3} Zariski, O., Samuel, P.: Commutative algebra. Princeton: van Nostrand 1958

\bibitem{art11-key4} Rees, D., Northcott, D. G.: Reductions of ideals in local rings. Proc. Camb. Phil. Soc. 50 (1954)

\bibitem{art11-key5} Rees, D.: The grade of an ideal or module. Proc. Camb. Phil. Soc. 53 (1957)
\end{thebibliography}

\hfill{C.P. Ramanujam } \hspace{1.8cm}

\hfill{494, Poonamallee High Road} 

\hfill{Madras-84, India} \hspace{1.8cm}

\begin{center}
{\em (Received May 22, 1972)}
\end{center}

\vfill\eject
~\phantom{a}
\thispagestyle{empty}
