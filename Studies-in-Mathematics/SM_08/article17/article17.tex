\title{Principal Bundles on Affine Space}\label{art17}
\markright{Principal Bundles on Affine Space}

\author{By~ M.S. Raghunathan}
\markboth{M.S. Raghunathan}{Principal Bundles on Affine Space}

\date{}
\maketitle

\section{Introduction}\label{art17-sec1}

\setcounter{page}{223}

\setcounter{pageoriginal}{186}
The\pageoriginale affirmative answer to Serre's question on vector bundles  on an affine space (due to Quillen \cite{art17-key10} and Sublin) leads one to pose the following question (*):

Let $k$ be a field and $A^n_k = \Spec k [X_1, \ldots, X_n] \to \Spec k$ the affine space of dimension $n$ over $k$. Let $G$ be an affine group scheme (of finite type) over $k$ and $P$ a principal $G$-bundle over $A^n_k$. Is $P$ then isomorphic to a bundle of the form $P' \bigotimes\limits_{\Spec  k} A^n_k$ where $P'$ is a principal homogeneous space over $k$.

(By a principal $G$-bundle $P$ over a scheme $X$ we mean a scheme $p:P\to X$ over $X$ together with a right action $m: P \times G \to P$ such that the diagram
\[
\xymatrix{
P \times G \ar[r]^m \ar[d]_\pi & P \ar[d]^p\\
P \ar[r]_p & X
}
\]
is commutative and the induced morphism
$$
P\times G \to \fprod{P}{P}{X}
$$
is an isomorphism. (It is known that with this definition, there exists for every point $p \in \bar{X} = X \bigotimes\limits_k \bar{k}$, $\bar{k}$ an algebraic closure of $k$, a neighbourhood $U$ of $p$ and an etale morphism $q: \tilde{U} \to U$ such that $\fprod{\tilde{U}}{\bar{P}}{\bar{X}}$ $(\bar{P} = \bar{P} \bigotimes\limits_{k} \bar{k})$ is isomorphic as a $G$ space to $\fprod{\tilde{U}}{G}{\Spec k}$).

The immediate expectation that Quillen's theorem would generalise to any reductive group turns out to be false: Ojanguran and Sridharan \cite{art17-key8a} and Parimala and Sridharan \cite{art17-key8b} have results which show that the answer to the question is in general in the negative. Specifically in \cite{art17-key8a} it is shown that when $G$ is the group of norm 1 elements\pageoriginale in a {\em noncommutative} division algebra over $k$, there exist $G$-bundles over $A^2_k$ which are not obtained by base change. When $k=\bfR$ in \cite{art17-key8b} it is shown that there are infinitely many inequivalent bundles on $A^2_\bfR$ which are not obtained by a base change from $\bfR$. Using this, one can also show (see Parimala \cite{art17-key9} where the case $SO$(4) over $\bfR$ is dealt with) (*) has a negative answer for $SO(3)$ and $SO(4)$ over $\bfR$. These examples suggested that one should look for an affirmative answer to (*) only under further restrictions on $k$. H. Bass \cite{art17-key1} proved that if $k$ is algebraically closed and of characteristic $\neq 2$ and $n =2$,  (*) has an affirmative answer for $G = SO(n)$ the present work began in fact with providing an alternative (and more direct) proof for Bass's theorem.

Before we can describe the results of the present work-which is mainly concerned with the case $n \geqslant 2 $ - we need to briefly outline the situation in the case of the affine line over $k$. For the sake of brevity in later formulations we make the following:

\begin{defi*}
A group $G$ (over $k$) is `acceptable' if for every extension $L \supset k$, any principal $G \otimes_k L$-bundle over $\Spec L[X]$ is obtained from a bundle on $\Spec L$ by the base change $L \to L[X]$.
\end{defi*}

$G$ is known to be acceptable in the following cases:
\begin{itemize}
\item[(i)] Char $k=0$, $G$ any group. This is a recent result due to A. Ramanathan and the author \cite{art17-key11}.

\item[(ii)] $G = 0(n)$ (hence also $SO(n)$), $\Char k \neq 2$ (Harder; see \cite{art17-key6a})

\item[(iii)] $G = SL(n)$, $GL(n)$ or $Sp(n)$ ($k$ arbitrary): these cases are obvious.

\item[(iv)] $G$ simply connected of inner-type $A_n$ : this follows from the fact that the projective modules over $D[X]$, $D$ a division algebra, are free (see for instance \cite[p. 202]{art17-key15}.

\item[(v)] $G$ simply connected of classical type and $\Char k$ `good' for\break $G (\Char k > 5$ is good for all $G$, (see \cite{art17-key11})).

\item[(vi)] $G$ a torus of {\em any} semisimple group of inner type $A_n$ or a spin group: this follows from the earlier mentioned results and some Galois cohomology arguments. For more details see \cite{art17-key11}.
\end{itemize}

We expect\pageoriginale the following to hold.

\vskip  0.4cm

\noindent
{\bf Conjecture.} {\em Any smooth simply `connected' reductive group is acceptable.}

(The assumption of connectedness is evidently essential in view of the existence of Artin Schrier extensions.)

After this discussion we can now formulate the main results of this work. The results are formulated for acceptable groups. It is possible of course to formulate some sharper results and this will be done in \S \ref{art17-sec4}.

The first result (which should probably be called a lemma rather than a theorem) is 

\begin{alphtheorem}\label{art17-alphthmA}
Assume that $G$ is acceptable. Let $P$ be a principal $G$-bundle over $A^n$. Then there is an open subscheme $U \subset A^n$ such that $P \times_{A^n} U$ is obtained from a $G$-bundle over $k$ by the base change $U \to \Spec k$.
\end{alphtheorem}

\begin{alphtheorem}\label{art17-alphthmB}
Assume that $G$ is connected, smooth, reductive and split. Assume further that the natural map $H^1 (k,G) \to H^1 (k, AdG)$ is trivial (i.e. for every principal homogeneous space over $G$ the principal homogeneous space over the adjoint group $AdG$ obtained by extension of structure group is trivial). Then a $G$-bundle $P$ over $A^n$ is trivial if and only if for a non-empty open subscheme $U \subset A^n$, $P \times_{A^n} U$ is obtained from a bundle on $\Spec k$ by the base change $U \to \Spec k$.
\end{alphtheorem}

An immediate corollary is 

\begin{alphtheorem}\label{art17-alphthmC}
Assume that $G$ is connected, acceptable and reductive and that $k$ is `separably closed'. Then any principal $G$-bundle over $A^n$ is trivial. 
\end{alphtheorem}

We state one final result which is applicable especially to the case of local fields.

\begin{alphtheorem}\label{art17-alphthmD}
Assume that $G$ is acceptable, connected, semisimple, simply connected and quasi-split. Assume further that the map $H^1 (k,G) \to H^1 (k, AdG)$ is trivial. Then any principal $G$-bundle on $A^n$ is trivial.
\end{alphtheorem}

The\pageoriginale following result for the special case  $G = 0(n)$-which has attracted considerable interest-is proved in \S \ref{art17-sec4}.

\begin{alphtheorem}\label{art17-alphthmE}
Assume that $\Char k \neq 2$ and the Brauer group $Br (k)$ of $k$ has no 2-torsion. Then any orthogonal bundle on $A^n$ is obtained  from an orthogonal bundle on $\Spec k$ by the base change $A^n \to \Spec k$.
\end{alphtheorem}

The method of proof closely parallels-and needless to say, is suggested by - Quillen's in the case of $GL(n)$. We need an obvious  extension of a lemma of Quillen's (Theorem \ref{art17-thm2}, \S \ref{art17-sec2}). Rigidity of the trivial bundle over $\bfP^1$ (not surprisingly) serves as a replacement of the Horrocks theorem used in Quillen's work. (Parimala \cite{art17-key9} gives an example to show that the theorem of Horrocks does not generalise to $0(n)$-bundles).

My thanks are due to many colleagues who took considerable interest in this work. Among them I should mention: Hyman Bass whose lectures in Bombay on his theorem triggered off the present work; C. S. Seshadri with whom I had many stimulating discussions; also R. C. Cowsik, Mohan Kumar, M. Nori and S. Parimala who had helpful comments and suggestions to offer. 


\section{Some known results}\label{art17-sec2}
We make use of three essentially known results in our proofs of the main theorems (Theorems \ref{art17-alphthmB} and \ref{art17-alphthmC}). In this section we collect these known results together. (Sketch proofs are given where no proper references are to be found). The first of these is:

\begin{theorem}\label{art17-thm1}
Let $\mathscr{P}^1$ denote the projective line over $k$ and $X$ be any $k$-scheme of finite type. Let $E$ be a principal $G$-bundle  over $\mathscr{P}^{1} \times X$. Let $\pi: \mathscr{P}^1 \times X \to X$ be the natural projection. Suppose now that $x_0: \Spec L \to X$ is a closed point such that $E \times x_0 \Spec L$ is trivial over $\bar{L}$, the algebraic closure of $L$. Then there exists an open subscheme $U$ of $X$ containing $x_0$ and a principal $G$-bundle $E'$ over $U$ such that $\fprod{E}{\pi^{-1}(U)}{(\mathscr{P}^1 \times X)}$ is isomorphic to $\fprod{E'}{\pi^{-1}(U)}{U}$. 
\end{theorem}

When $E$ is a $GL(n)$-bundle this is a special case of Grothendieck \cite[Corollaire 4.6.4.]{art17-key4}: note that for the projective line $\mathscr{P}^1$,\pageoriginale $H^1 ({}_{\mathscr{P}^1}, O_{\mathscr{P}^1}) = 0$ where $O_{\mathscr{P}^1}$ is the structure sheaf. In this case, one may assume that $E'$ is in addition the trivial $GL(n)$-bundle.

For handling the general case we need

\begin{lemma*}
Any smooth $G$ is isomorphic to a closed subgroup scheme of some $GL(n)$ such that $G(L)(n)/G$ contains no complete reduced subscheme of positive dimension.
\end{lemma*}

Assume the lemma for the present. Then we can think of a $G$-bundle as a $GL(n)$-bundle together with a section in the associated fibre space with $GL(n)/G$ as fibre. From what we have remarked earlier the $GL(n)$-bundle may be assumed trivial so that the section is a morphism $\sigma: \mathscr{P}^1 \times X \to GL(n)/G$. Since $GL(n)/G$ contains no complete subscheme of positive dimension $\sigma$ factors through $\mathscr{P}^1 \times X \to X$ which shows that the $G$-bundle $E$ itself is obtained from a bundle on $X$ by base change $P^1 \times X \to X$.

We have to establish the lemma. We may assume that $G$ is a closed ($k$-) subgroup scheme of $GL(m)$ for some $m$. Let $\rho$ be a ($k$-) representation of $GL(m)$ on a vector space $V$ and $0 \neq v \in V (k)$ be such that $G$ is the isotropy group scheme at $\bar v$ for the action of $GL(m)$ on the projective space $\mathscr{P}(V)$ associated to $V$. This means the orbit map $GL(m) \to GL(m)$ $\bar{v}$ is an isomorphism of $GL(m)/G$ on the orbit which is a locally closed subscheme of $\mathscr{P}(V)$. (Here $\bar{v}$ is the closed point of $\mathscr{P}(V)$ determined by $v$). This determines a morphism $\chi: G \to G_m$ such that for any $g \in G (L)$, $L$ a $k$-algebra,
$$
\rho (g) v = \chi (g) \cdot v. 
$$
If $\chi$ is trivial, $G$ is the isotropy at $v$ for the action of $GL(m)$ on $V$ so that $GL(m)/G$  is locally closed in affine space and the assertion follows. We assume then that $\chi$ is non-trivial. Consider the immersion
$$
G \xrightarrow{(i, \chi^{-1})} GL(m) \times G L(1) = H.
$$
Let $I$ denote the identity representation of $GL(1)$ and $\rho\otimes I$ the representation of $H$ obtained by forming the tensor product of $\rho$ and $I$. Then the isotropy group at $v \otimes 1$ is seen to be precisely $G$ so that $(GL(m) \times GL(1))/ \gamma G$\pageoriginale is a locally closed subscheme of an affine space. On the other hand, $GL(m) \times GL(1)$ is in a natural fashion a closed subscheme of $GL(m+1)$ with affine quotient. It follows that $GL(m+1) /G$ is a fibre space over an affine scheme with fibres $(GL(m) \times GL (1)) / \gamma G$. The lemma follows immediately from this.

\begin{theorem}[Quillen \cite{art17-key10}]\label{art17-thm2}
Let $E$ be a $G$-bundle on $\Spec R[T] (\simeq A^1 \times \Spec R)$, $R$ a $k$-algebra. Suppose for every maximal ideal $\mathfrak{m} \subset R$, $\fprod{E}{\Spec}{\Spec R} R_{\mathfrak{m}}$ ($R_{\mathfrak{m}}$ = localisation at $\mathfrak{m}$)  is trivial, then $\fprod{E \simeq F}{\Spec}{\Spec R}$ $R [T]$ for some $G$-bundle $F$ over $\Spec R$.
\end{theorem}

Quillen states and proves this for $G = GL(n)$. The general case follows by simply changing GL(n) to $G$ in his argument.

The next result we need is well known.

\begin{theorem}\label{art17-thm3}
Let $G$ be a `split' connected semisimple group over $k$. Let $n = \dim$ $G$ and $I = \rank G$ ( = dimension of maximal $k$-split torus of $G$). Then $G$ contains an affine open subscheme isomorphic to $A^{n-1} \times \Spec k[T_1, T^{-1}_1, T_2, T_2^{-1} \ldots, T_1, T_1^{-1}]$.
\end{theorem}

This is immediate from structure theory: Let $T$ be a maximal $k$-split torus and $U^{\pm}$ opposing unipotent maximal $k$-subgroup schemes. Then the morphism $U^+ \times T \times U^- \to G$ given by the multiplication in $G$ is an isomorphism onto the image $\Omega$.

Theorem \ref{art17-thm4} below must also be regarded as well known: it is a more or less immediate consequence of the work of Chevalley \cite{art17-key3} and Steinberg \cite{art17-key13}. We give some details, as a proof does not seem to be available explicitly in literature.


\begin{theorem}\label{art17-thm4}
Let $G$ be a simply connected quasi-split (smooth) group scheme. Then $G$ contains an open ($k$-) subscheme $\Omega_0$ such that the inclusion $i: \Omega_0 \to G$ factors through an affine space, i.e. there is a commutative diagram of the form
\[
\xymatrix{
\Omega_0 \ar@{^{(}->}[rr]^i  \ar[dr] & & G \\
& A^r \ar[ur] &
}
\]
(for some integer $r$).
\end{theorem}

Let\pageoriginale $T$ be the centraliser of a $k$-split torus $S$ in $G$. Let $U$ and $U^-$ be opposing maximal unipotent $k$-subgroup schemes of $G$ normalised by $S$. Let $N(S)$ be the normaliser of $S$ in $G$ and $\theta \in N(S)(k)$ an element which conjugates $U$ onto $U^-$. Then the morphism 
$$
U \times T \times U \to G
$$
defined by $(u, t, u') \mapsto u \theta tu'$ on the set of closed points is an open immersion. As a $k$-scheme $U$ is isomorphic to an affine space $A^q$, $q = \dim U$. It suffices to show that the inclusion $T' \hookrightarrow G$ for some $k$-open subset $T'$ of $T$ factors through an affine space. If $\Delta$ is the system of simple $k$-roots determined by $U$, for each $\alpha \in \Delta$, we have a $k$-rank 1 quasi split subgroup $G_\alpha$ of $G$ such that $T = \prod\limits_{\alpha \in \Delta} T_\alpha$, a direct product with $T_\alpha \subset G_\alpha$. This shows that the problem immediately reduces to the case when $k$-rank $G =1$. The quasi-split groups of $k$-rank 1 are of one of the following two kinds:
\begin{itemize}
\item[(i)] $R_{L/k} SL(2)$,

\item[(ii)] $R_{L/K} S U (2,1)$, $SU (2,1)$ being the special unitary group of a hermitian form of Witt index 1 over a quadratic extension of $L$ where $L$ is a finite separable extension of $k$. Now for the affine $n$-space $A^n_{L'}$, $R_{L/k} A^n_L$ is isomorphic to $A^{nq}_k$ where $q = \dim_k L$ so that we see that we need only consider the cases of $SL(2)$ and $SU(2,1)$ over $k$.
\end{itemize}

In case $G = SL(2)$, take $T'= T$ and the factorisation sought after results from the product decomposition
$$
\left(\begin{matrix}
t & 0\\
0 & t^{-1}
\end{matrix}\right)  = 
\left(\begin{matrix}
1 & t \\
0 & 1
\end{matrix} \right) \left(\begin{matrix}
1 & 0\\
-t^{-1} & 1 
\end{matrix} \right) \left(\begin{matrix}
1 & t \\
0 &1
\end{matrix} \right) \left(\begin{matrix}
0 & -1\\
1& 0
\end{matrix} \right).
$$
Consider now the case $G = SU(2,1): G$ is the special unitary group of the hermitian quadratic form
$$
\left(\begin{matrix}
0 & 1 & 0 \\
1 & 0 & 0 \\
0 & 0 & 1
\end{matrix} \right)
$$
in 3 variables over a quadratic Galois extension $K$ over $k$. If we denote by $x \mapsto \bar{x}$ the conjugation by the non-trivial element of the Galois group of $K/k$ as well as its extension to $\tilde{L} = L \otimes_k K$ for any 
$L \supset K$,\pageoriginale the $L$-rational points of a split torus are given by
$$
\varphi (x) = \left(\begin{matrix}
x & 0 & 0\\
0 & x^{-1} &  0\\
0 & 0 & x^{-1} \bar{x}
\end{matrix} \right), \quad x \in \tilde{L} \text{ invertible.}
$$
We will deal with the cases $\Char k \neq 2$ and $\cchar k =2 $ separately.

\begin{case}%%% 1
$\Char k \neq 2$. Let $T'$ be the open set whose $L$-rational points are 
$$
\{\varphi (x)| x - \bar{x} \text{ is invertible in  } \tilde{L} \}.
$$
Fix an element $\theta \in K - k$ such that $\theta^2 \in k$. Then any element of $K$ is of the form 
$$
a + b \theta, a, b \in k,
$$
and for any $L \supset k$, $\tilde{L} = L + L \theta$; also for $a+ b \theta \in \tilde{L} \supset a$, $b \in L$, $(a + b \theta)^- = a - b \theta$ so that $\varphi (a + b \theta) \in T'(L)$ if and only if $b$ is invertible in $L$.

Now for $a + b \theta \in \tilde{L}$.
{\fontsize{10pt}{12pt}\selectfont
\begin{align*}
& \left(\begin{matrix}
a+ b\theta & 0 & 0\\
0 & (a-b\theta)^{-1} & 0\\
0 & 0 & (a+b \theta)^{-1} (a-b\theta)
\end{matrix}\right)\\
& \left(\begin{matrix}
0 & -(b^2 / 2 - a b \theta^{-1}/ 2 & 0\\
-(b^2 /{}_2 - ab \theta^{-1}/2 & 0 & 0 \\
0 & 0 & \delta (a,b)
\end{matrix}\right)  
\left(\begin{matrix}
0 & +2^{-1} b \theta^{-1} & 0\\
-2b^{-1} \theta & 0 & 0 \\
0 & 0 & 1
\end{matrix}\right)\\
& = \lambda (a, b) \cdot \mu (a, b) \text{ say.}
\end{align*}}
We will now give factorisations of $\lambda (a,b)$ and $\mu(a,b)$ as products\break $\mu(a,b) = \chi'(a, b) \xi'(a,b) \beta'(a,b) \cdot \lambda (a, b) = \alpha (a,b) \xi (a,b) \beta (a,b)$ where $\alpha$, $\alpha'$ and $\beta$, $\beta'$ morphisms of $T'$ in $U$ and $\xi, \xi'$ are morphism of $T'$ in $U^-$. For $\mu$ this is similar to what we did in the case of $SL(2)$:
$$
\mu(a,b) = \left(\begin{matrix}
1 & 2^{-1} b \theta^{-1} & 0\\
0 & 1 & 0\\
0 & 0 & 1
\end{matrix}\right) 
\left(\begin{matrix}
1 & 0 & 0\\
-2b^{-1} &  \theta  & 0\\
0 & 0 & 1
\end{matrix}\right)
\left(\begin{matrix}
1 & 2^{-1} b \theta^{-1} & 0\\
0 & 1 & 0\\
0 & 0 & 1
\end{matrix}\right)
$$
As for $\lambda (a,b)$ this is a trifle more complicated. We write down $\alpha (a,b)$; the others are uniquely determined by $\alpha (a,b)$:
$$
\alpha (a,b) = 
\left(\begin{matrix}
1 & -b^2/ 2 -a b \theta^{-1}/2 & b \\
0 & 1 & 0\\
0 & -b & 1
\end{matrix}\right)
$$\pageoriginale
(Given an element $x \in U(k)$, there are unique elements $y \in U^{-1}(k)$ and $x' \in U (k)$ such that
$$
xyx' \in \text{ Normaliser of } S.
$$
This is the reason why the argument works.)
 \end{case}

\begin{case}%%%2
$\Char k =2$. We pick an element $\theta \in K$ with $\theta + \bar{\theta} =1$. Then any element of $L$ is of the form $a + b \theta$ with $a, b \in L$. We can then write
\begin{align*}
& \left(\begin{matrix}
a +b \theta & 0 & 0\\
0 & \overline{(a+b \theta)}^{-1} & 0\\
0 & 0 &\overline{(a + b \theta)} (a + b \theta)^{-1}
\end{matrix}\right)\\
& =  \left(\begin{matrix}
0 & b^2 \theta + a b & 0\\
\overline{(b^2 \theta + a b)^{-1}} & 0 & 0 \\
0 & 0 & (a,b)
\end{matrix}\right)
\left(\begin{matrix}
0 & b & 0 \\
b^{-1} & 0 & 0 \\
0 & 0 & 1
\end{matrix}\right)\\
& = \lambda (a,b) ~ \mu(a,b) \text{~ as before.}
\end{align*}
Clearly $\mu(a,b) = \left(\begin{matrix}
1 & b & 0\\
0 & 1 & 0\\
0 & 0 & 1
\end{matrix}\right) \left(\begin{matrix}
1 & 0 & 0 \\
-b^{-1} & 1 & 0\\
0 & 0 & 1
\end{matrix}\right) \left( \begin{matrix}
1 & b & 0\\
0 & 1 & 0\\
0 & 0 & 1
\end{matrix}\right)$;\quad and\hfill\break  $\mu (a,b) = \alpha (a,b) \cdot \xi (a,b) \cdot \beta (a,b)$ with $\alpha$, $\beta \in U$ and $\xi \in U^-$.

As before we write down $\alpha (a,b)$ explicitly:
$$
\alpha (a,b) = \left(\begin{matrix}
1 & b^2 \theta + ab & b  \\
0 & 1 & 0\\
0 & b & 1
\end{matrix}\right)
$$
($\beta$ and $\xi$ are determined by $\alpha$). This completes the proof of Theorem \ref{art17-thm3}.
\end{case}

\section{Proofs of the main results (A, B, C, and D).}\label{art17-sec3}
Throughout this section $G$ will denote an {\em acceptable} (smooth) group scheme over $k$.

We prove Theorem \ref{art17-alphthmA} first. We argue by induction on $n$. When $n=1$, this is immediate from the definition of acceptability. Assume that the result holds for $n \leqslant m$ and let $P$ be a $G$-bundle on $A^{m+1} = A \times A^m$.\pageoriginale Let $\Spec K \to A^m$ be the generic point of $A^m$ and $P^* = P\times_{A^m} \Spec K : P^*$ is a bundle over $\fprod{A}{\Spec}{k} K$. By the definition of acceptability, there is a $\fprod{G}{K}{k}$ bundle $P'$ on $\Spec K$ such that $\fprod{P'}{(A^1\bigotimes\limits_k K \Spec K)}{\Spec k} \simeq P^*$. Equivalently there is an open subscheme $U \subset A^m$ and a bundle $P''$ on $U$ such that 
$$
P \times_{A^m} U \simeq \fprod{P'' }{(\fprod{A^1}{U}{k})}{U}.
$$
Let $p: \Spec k \to A$ be any $k$-point and $\hat{p}:A^m \to \fprod{A}{A^m}{k}$ the corresponding immersion. Let 
$$
P_1 = \fprod{P}{A^m}{A' \times A^m} ~~ (\text{via } \hat{p}).
$$
If we let $\hat{p}$ also denote the immersion
$$
U \to \fprod{A}{U}{k},
$$
clearly
$$
P_1 \times_{A^m} U \simeq P''.
$$
By induction hypothesis $P_1$ is obtained by base change $A^m \to \Spec k$ from a principal homogeneous space over $k$. Thus the same is true of $P''$ (with respect to the base change $U \to \Spec k$). Since $P\times_{A^m} U \simeq P'' \times_U (\fprod{A'}{U}{k})$ the desired result follows. This concludes the proof of Theorem \ref{art17-alphthmA}.

We proceed to the proofs of Theorems \ref{art17-alphthmB} and \ref{art17-alphthmC}. For the sake of brief formulations we adopt the following definition: A $G$-bundle $P$ on a $k$-scheme $X$ is {\em obtained from $k$} if $P$ is isomorphic to a bundle of the form $\fprod{P'}{X}{\Spec k}$ with $P'$ a principal homogeneous space over $k$.

Under the hypotheses in Theorems \ref{art17-alphthmB} and \ref{art17-alphthmD} we can find $f \in k [X_1, \ldots,\\ X_n](A^n = \Spec k [X_1, \ldots, X_n])$ such that $P$ restricted to the open subschemes
$$
W' = \{ p \in A^n |f(p) \neq 0\}
$$
is obtained from $\Spec k$. After a change of coordinates if necessary one can assume that $f$ has the form 
$$
X^m_1 + \sum\limits^{m-1}_{i=0} q_i (X') X'_1
$$\pageoriginale
with $q_i (X') \in k [X'] {\displaystyle\mathop{=}^{def}} k [X_2, \ldots, X_n]$. Isolating the variable $X_1$ gives a product decomposition 
$$
A^n = A^1 \times A^{n-1}. 
$$
We want to prove the following.

\begin{assertion}\label{art17-assertion1}
The hypotheses are those of Theorem \ref{art17-alphthmB} or of Theorem \ref{art17-alphthmD}. Then for each closed point $x_0 \in A^{n-1}$, there is an open neighbourhood $U_{x_0}$ of $x_0$ such that $P$ restricted to $A^1 \times U_{x_0}$ is obtained from a bundle $P_{x_0}$ on $U_{x_0}$ by the base change $A^1 \times U_{x_0} \to U_{x_0}$.
\end{assertion}

Observe that the assertion implies Theorems \ref{art17-alphthmB} and \ref{art17-alphthmD}. To see this we argue as follows. Let $p: \Spec k \to A^1$ be a $k$-point and $\hat{p}$ the corresponding immersion
$$
A^{n-1} \hookrightarrow A^1 \times A^{n-1}.
$$
Then the restriction of $P$ to $A^{n-1}$ is, by an induction assumption, obtained from $\Spec k$. Using a Galois twist of $P$ we can assume that this bundle is actually trivial. It is then easily seen that the $P_{x_0}$ in the Assertion is necessarily trivial. We can now appeal to Theorem \ref{art17-thm2} to obtain Theorems \ref{art17-alphthmB} and \ref{art17-alphthmD}.

To prove Assertion \ref{art17-assertion1} we use Theorem \ref{art17-thm1}. Consider the scheme $\mathscr{P}^1\times A^{n-1}$. This contains $A^1 \times A^{n-1}$ as an open $k$-subscheme which we note also by $U_1$ in the sequel. Let 
$$
U'_2 = \mathscr{P}^1 \times A^{n-1} - \{p \in A^n | f(p) = 0\}.
$$
Then $U'_2$ is an open $k$-subscheme as well. Since $f$ is {\em monic in} $X_1$, $U_1$ and $U'_2$ cover $\mathscr{P}^1 \times A^{n-1}$. Moreover if $U_1 \cap U'_2 = W'$ and $P$ restricted to $W'$ is {\em obtained from} $\Spec k$. We may, after a {\em twist} assume then that $P$ restricted to $W'$ is actually trivial. Since $H^1 (k,G) \to H^1 (k, Ad G)$ is trivial, this {\em does not change $G$}. Fix once and for all an isomorphism
$$
P \left|
\begin{aligned}
 \xrightarrow{\sim} & ~~ 1_{G,W'}\\
U_1 \cap U'_2 ~ \Phi_0 &
\end{aligned}
\right.
$$  
where\pageoriginale $1_{G,W'}$ is the trivial bundle on $W'$. Now let $x_0: \Spec L \to A^{n-1}$ be any closed point and $\tilde{x}_0$ the induced morphism
$$
\mathscr{P}^1 \times \Spec L \to \mathscr{P}^1 \times A^{n-1}.
$$
For an open subscheme $V$ of $\mathscr{P}^1 \times A^{n-1}$ we denote by $V_{x_0}$ the inverse image of $V$ in $\fprod{\mathscr{P}^1}{\Spec L}{k}$. In particular then $U_{1x_0}$ and $U'_{2x_0}$ cover $\fprod{\mathscr{P}^1}{\Spec L}{k}$. Let $P_{x_0}$ denote the bundle on $A^1 \times \Spec L$ induced by $P$. Since $G$ is acceptable $P_{x_0}$ is obtained from $L$ and, since $P$ is trivial on $W'$ and $W'_{x_0}$ is non-empty (note that $f$ is monic in $X_1$) $P_{x_0}$ is trivial. Let 
$$
\theta : P_{x_0} \simeq \fprod{G}{(\fprod{A^1}{\Spec L}{k})}{k}
$$
be an isomorphism. On the other hand, $\Phi_0$ by restriction gives an isomorphism
$$
\varphi_0 : P_{x_0}|W_{x_0} \simeq \fprod{G}{W'_{x_0}}{k}.
$$
Let $\alpha = \theta_0 \varphi^{-1}_0$ be the composite isomorphism
$$
\alpha : \fprod{G}{W'_{x_0}}{k} \to \fprod{G}{W'_{x_0}}{k}.
$$
$\alpha$ gives rise to and is determined by a morphism
$$
\Psi_1: W'_{x_0} \to G.
$$
We will deduce Assertion \ref{art17-assertion1} from

\begin{assertion}\label{art17-assertion2}
Let $x_0 \in A^{n-1}$ be any closed point. Then there is a refinemenet $(U_1, U_2)$ of $(U_1, U_2')$ with the following properties:
\begin{itemize}
\item[\rm (i)] there is a neighbourhood $V$ of $x_0 \in A^{n-1}$ such that \\$U_1 \cup U_2\supseteq \fprod{\mathscr{P}^1}{V}{\Spec k}$,

\item[\rm (ii)] if $W = U_1 \cap U_2$, $\psi = \psi_1|W_{x_0}$ extends to a morphism $\overline{\Psi}$ of $W$ in $G$. 
\end{itemize}
\end{assertion}

If we admit Assertion \ref{art17-assertion2}, we can construct a bundle $\tilde{P}$ on $\fprod{\mathscr{P}^1}{V}{\Spec k}$ as follows: Let $1_{G,U_2}$ be the trivial $G$-bundle on $U_2$. Then $\Psi \circ \Phi_0$ gives an isomorphism of $P$ restricted to $W$ with $1_{G,U_2}$ restricted to $W$. Patching by this isomorphism we obtain a bundle on $U_1 \cup U_2$ whose restriction to $\fprod{\mathscr{P}^1}{V}{\Spec k} $ we denote $\tilde{P}$. From the fact\pageoriginale that $\overline{\Psi}$ extends $\psi$, it is immediate that the bundle on $\fprod{\mathscr{P}^1}{\Spec L}{k}$ obtained by the base-change $x_0: \Spec L \to V$ is trivial. By Theorem \ref{art17-thm1}, $\tilde{P}$ is locally obtained by base change from a $G$-bundle on a neighbourhood of $x_0$. Since $\tilde{P}$ has $P$ for its restriction to $A^1 \times V$, Assertion \ref{art17-assertion1} follows.

We are still to establish Assertion \ref{art17-assertion2}. We consider the hypotheses of the two theorems separately.

\begin{romancase}[Hypotheses as in Theorem \ref{art17-alphthmB}]%% i
Choose $\Omega$  as in the proof of Theorem \ref{art17-thm3} of \S \ref{art17-sec2}. $\Omega \simeq \Spec k [X_1, \ldots, X_q, T_1 \ldots T_1 T_1^{-1} \ldots T_1^{-1}]$ ($1 = k$-rank $G$, $q^+ = \dim G$). Let $S = \psi^{-1}_1 (G -\Omega)$. This is a $k$-closed subset of $W'_{x_0}$ since $\dim W'_{x_0} =1$, if $S$ is a proper subset, $S$ in finite hence closed in $U_{1x_0} (\simeq \Spec L[X_1])$; and we assume that $S \neq W'_{x_0}$ by replacing $\Omega$ if necessary by a translate by an element of $G(k)$. (This is evident when $G(k)$ is dense in $G$ but one can give an obvious argument using the Bruhat-decomposition even if $G(k)$ were not dense in $G$). The affine scheme $W'_{x_0}$ is of the form $\Spec L \left[X_1, \dfrac{1}{f^*} \right]$ where $f^* \in L [X_1]$ is the polynomial $X^m_1 + \sum\limits^{m-1}_{i=0} q_i (x_0) X^i_1$. The morphism $\Psi_1$ gives a ring homomorphism 
$$
\psi^*_1: k [X_j, T_i, T^{-1}_i (1 \leqslant j \leqslant q, 1 \leqslant i \leqslant 1)] \to L(X_1).
$$
Let $\psi^*_1 (X_j) = (Q_j / R_j)$ and $\psi^*_1 (T_i) = A_i/B_i $ where $(Q_j, R_j)$ and $(A_i, B_i)$ are coprime polynomials in $L[X_1]$. Let degree $R_j = r_j$, degree $A_i =a_i$ and degree $B_i = b_i$. Choose elements  ($\tilde{A}_i, \tilde{B}_i, \tilde{Q}_j$) and $\tilde{R}_j$ in $k[X_1, \ldots, X_n]$ such that 
\begin{itemize}
\item[\rm(i)] $\tilde{A}_i$ (\resp $\tilde{B}_i, \tilde{Q}_j, \tilde{R}_j)$ is a lift of $A_i$ (\resp $B_i, Q_j, R_j$) for the map $\hat{x}_0: k [X_1, \ldots, X_n] \to L [X_1]$,

\item[\rm(ii)] degree $\tilde{A}_i$ (\resp $\tilde{B}_i, \tilde{R}_j)$ in $X_1$ is precisely $a_i$ (\resp $b_i, r_j$).
\end{itemize}
Let $\alpha_i$ (\resp $\beta_i, \rho_j$) denote the {\em leading coefficients} of $\tilde{A}_i$ (\resp $\tilde{B}_i, \tilde{R}_j$) as a polynomial in $X_1: \alpha_i, \beta_i, \rho_j$ belong to $k [X_2, \ldots, X_n]$. These elements\pageoriginale do not take the value zero at $x_0$. Let $V$ be the open neighbourhood of $x_0$ defined by 
$$
V = \{p \in A^{n-1} | \alpha_i (p) \neq 0, \beta_i (p) \neq 0, \rho_j(p) \neq 0, 1 \leqslant i \leqslant l, 1 \leqslant j \leqslant q\}.
$$
Let $U_2 = U'_2 - \{p \in A^n | \prod\limits_{\substack{1 \leqslant i \leqslant 1\\ 1 \leqslant j \leqslant q}} A_i B_i R_j (p) = 0 \}$. This choice of $U_2$ and $V$ is easily seen to satisfy the conditions of Assertion \ref{art17-assertion2}.
\end{romancase}

\begin{romancase}[Hypotheses as in Theorem \ref{art17-alphthmD}]%%% ii
Since $G$ is quasi split and simply connected we can choose (Theorem \ref{art17-thm4} of \S \ref{art17-sec2}) an open subset $\Omega_0$ such that the inclusion $\Omega_0 \subset G$ factors through an affine space : $\Omega_0 \to A^r \to G$. Let $S = \psi^{-1}_1 (G - \Omega_0)$. We may assume that $S$ is finite (if $\Omega_0$ is chosen exactly as in Theorem \ref{art17-thm4} of \S \ref{art17-sec2} this can be secured by translating $\Omega_0$ by an element of $G(k)$). Let $f_0 \in k [X_1]$ be a monic polynomial vanishing on $S \subset \Spec L[X_1]$). Consider   $f_0$ as an element of $k[X_1, \ldots, X_n]$ and let $U_2 = U'_2 - \{p \in A^n | f_0 (p) =0\}$. Since $f_0$ is monic in $X_1$, $U_1 \cup U_2 = \mathscr{P}^1 \times A^{n-1}$ with this choice of $U_2$, $\psi$ factors through $\Omega_0 : \psi_1 (W_{x_0}) \subset \Omega_0$. Since $\Omega_0 \hookrightarrow G$ factors through $A^r$, $\psi$ extends (from the closed set $W_{x_0}$) to all of $W: \psi: W \to G$. This proves Assertion \ref{art17-assertion2} in this case.
\end{romancase}

\section{Complements}\label{art17-sec4}
We discuss first some special fields.
\begin{enumerate}
\item \textsc{Fields of Dimension $\leqslant$ 1.} If $k$ is a field of dimension 1, $H^1 (k,G) = 0$ for all smooth connected $G$ (Lang \cite{art17-key7}, Steinberg \cite{art17-key14}). Thus Theorem \ref{art17-alphthmB} is applicable to any {\em acceptable } semisimple $G$ over $k$. We conclude that {\em for such $k$ any $G$-bundle on $A^n$ is trivial}. Among the fields covered are:
\begin{itemize}
\item[(i)] finite fields;

\item[(ii)] $k = L (X)$, the pure transcendental extension of degree 1 over an algebraically closed field $L$;

\item[(iii)] $k$ is the maximal unramified extension of a local field.
\end{itemize}

\item \textsc{Local fields} with residue fields of dimension $\leqslant 1$. For such $k$, it is known (Bruhat-Tits \cite{art17-key2}) that $H^1 (k,G) = 0$ for any {\em simply connected $G$}. Thus Theorem \ref{art17-alphthmD} is applicable: {\em Any $G$-bundle with $G$ quasi-split\pageoriginale and simply connected on an affine space is trivial. The fields that are covered are:}
\begin{itemize}
\item[(i)] finite extensions of $p$-adic fields,

\item[(ii)] quotient fields of power series rings $k'[[T]]$ in 1 variable with $\dim k' \leqslant 1$.
\end{itemize}

\item \textsc{Global fields.} Here again it is known that for a simply connected acceptable $G$, $H^1 (k,G) = 0$ except possibly when $G$ has exceptional factors and $k$ is a number field which admits a real place (Harder \cite{art17-key5a}, \cite{art17-key5b}, Kneser \cite{art17-key6b}). Thus for any quasi-split $G$, $G$-bundles on $A^n$ are trivial for simply connected classical $G$ over a number field $k$.

The following groups are acceptable for {\it any} field $k: GL(n)$, $Sp(n)$ $\Spin f$, $f$ a split quadratic form (in more than 3 variables). This is obvious for $GL(n)$ and $Sp(n)$ and can be deduced from Harder's theorem for $SO(n)$ in the case of Spin groups. Theorem \ref{art17-alphthmD} is applicable in all these cases because in addition, $H^1(k,G)=0$ for all these groups.-Note that for Spin we have taken the split form (see Serre \cite{art17-key12}, III-25).

We have therefore
\end{enumerate}

\setcounter{dashtheorem}{3}
\begin{dashtheorem}\label{art17-dashthmD'}
Let $G$ be a split simply connected {\em `classical' } group over any field $k$. Then any $G$-bundle over $A^n$ is trivial.
\end{dashtheorem}

Next observe that Theorem \ref{art17-alphthmC} gives the following

\begin{prop*}
If $G$ is an acceptable $k$-group, these is a bijective correspondence between isomorphism of classes of principal $G$-bundles on $A^n$ and the Galois cohomology set
$$
H^1 (\pi, G(\bar{k}_s [X_1, \ldots, X_n]))
$$
where $\bar{k}_s =$ separable closure of $k$ and $\pi$ is the Galois group of $\bar{k}_s$ over $k$.
\end{prop*}

As a corollary we will prove

\setcounter{alphtheorem}{4}
\begin{alphtheorem}%%% E
Let $k$ be such that $Br(k)$ has no 2-torsion and $\Char k \neq 2$. Then any orthogonal bundle over $A^n$ is obtained by base change from $k$.
\end{alphtheorem}

Theorem \ref{art17-alphthmE}\pageoriginale is evidently equivalent to 

%\setcounter{dashtheorem}{3}
\begin{dashtheorem}\label{art17-dashthmE'}
Let $k$ be a field such that $Br(k)$ has no 2-torsion and $\Char k \neq 2$. Then every non-singular quadratic form over $k[X_1, \ldots, X_n]$ is equivalent to one over $k$.
\end{dashtheorem}

For the proof we need

\begin{lemma*}
Let $k$ be a field such that $Br(k)$ has no 2-torsion. Then 
\begin{itemize}
\item[(i)] a quadratic form over $k$ is determined by the discriminant;

\item[(ii)] if the number of variables is $n \geqslant 3$, the Witt-index is at least $[(n-1)/2]$ (integral part of $(n-1)/2$).

\item[(iii)] the spinor norm is surjective for any form $f$;

\item[(iv)] for any quadratic form $f$, $SO(f)$ is quasi-split over $k$ (if the number of variables is at least 3);

\item[(v)] for any form $f$, $H^1 (k,\Spin f) =0$.
\end{itemize}
\end{lemma*}

\begin{proof}
To prove (i) we argue by induction on the number of variables, we may assume that the quadratic form is the orthogonal sum
$$
q \perp \underline{\lambda}
$$
where $q$ is a form in $(n-1)$ variables $\lambda \in k$ and $\underline{\lambda}$ is the form in 1 variable $x \mapsto \lambda x^2$, $x \in k$.  By induction hypothesis $q$ is isomorphic to the orthogonal sum 
$$
\underline{1} \perp \underline{d}
$$
where $\underline{1}$  is the identity form in $(n-2)$ variables, $d \in k$ and $\underline{d}$ is the form $x \to d x^2$ on $k$. It suffices then to show that $\underline{d}\perp \underline{\lambda}$ is isomorphic to $\underline{1}\perp \lambda \underline{d}$ (with the obvious notation). In order to prove this, we have to show that any quadratic form in 2 variables represents 1. In fact, we prove the following stronger statement which immediately implies (iii) as well:


{\em Let $\alpha \neq 0$ be any element of $k$; then the quadratic form $\underline{\lambda} \perp \underline{\mu}$ for any pair $\lambda$, $\mu \in k^*$ represents $\alpha$.}

We have to solve the equation
$$
\lambda x^2 + \mu y^2 = \alpha
$$
or equivalently\pageoriginale
$$
x^2 + \frac{\mu}{\lambda} y^2 = \frac{\alpha}{\lambda}.
$$
This is the same as proving that $\alpha/\lambda$ is a norm for the quadratic extension $k(\sqrt{-\mu/\lambda})$. Since there are no quaternion divison algebras over $k$, this is indeed true (see for instance, O'Meara \cite{art17-key15}, p. 146).

We have thus proved (i) and (ii). Statement (iii) is immediate from (i): if $q$ is any quadratic form in $m = 2n + 2$ (\resp $2n +1)$ variables and $h$ is the hyperbolic form in $2n$ variables, $q \simeq h \perp \underline{1} \perp \underline{d}$ (\resp $h \perp \underline{d}$) where $d' \in k$ is any representative for the discriminant and $d = \pm d'$.

To prove (iv) again write
\begin{align*}
& q \simeq h \perp \underline{d} \quad { or }\\
& q \simeq h \perp \underline{1} \perp \underline{d}
\end{align*}
according as the number of variables is odd or even. Let $E$ be the maximal isotropic subspace for $h$. Then the subgroup $P$ of $SO(q)$ which fixes $E$ is a parabolic subgroup. Writing $q$ as
$$
q \simeq h \perp q'
$$
(where $\dim q ' \leqslant 2$) we see that the reductive part $M$ of $P$ is isomorphic locally to the product of a torus and $SO(q')$. It follows that (since $\dim q' \leqslant 2$) $M$ is a torus. Thus $P$ is a Borel subgroup. This proves (iv).

Finally (v) follows from (i) and (iii) in view of the following known fact (Serre \cite[III-25]{art17-key12}).

If $f$ is a quadratic form such that the Spinor norm map is surjective and every form $f'$ with the same Witt-index and discriminant as $f$ (in the same number of variables) is equivalent to $f$, then $H^1 (k, \Spin f) =0$. This proves (v).

\end{proof}
We proceed to the proof of Theorem \ref{art17-alphthmE} now. When the number of variables is 2 this is trivial. We assume therefore that the number of variables is at least 3. Observe first that for any quadratic form $f$ over $k$,
$$
H^1 (\pi, \Spin f(k_s [\underline{X}])) = 0
$$
where\pageoriginale $k_s = $ separable closure of $k$ and $\pi = \text{Gal } (k_s /k)$. This follows from Theorem \ref{art17-alphthmC} combined with Parts (iv) and (v) of the lemma above. Next consider the exact sequence
$$
1 \to \{\pm 1\} \to \Spin f \to SO(f) \to 1.
$$
Since $\Char k \neq 2$, one sees immediately that the sequence
$$
1 \to \{\pm 1\} \to \Spin f(k_s [\underline{X}]) \to SO(f) (k_s [\underline{X}]) \to 1
$$
is exact. The cohomology sequence gives exactness of 
\[
\xymatrix{
H^1(\pi, \Spin f(k_s[\underline{X}])) \ar[r] \ar@{=}[d] & H^1 (\pi, SO(f) (k_s [\underline{X}])) \ar[r] & H^1 (\pi, \pm 1) \ar@{=}[d]\\
0 & & 0
}
\]
The last group is trivial since $Br(k)$ has no 2-torsion.

Thus for any $f$, $H^1 (\pi, SO(f) (k_s [\underline{X}])) =0$. Now consider the split exact sequence for a fixed $f_0$
$$
1 \to SO(f_0) (k_s [\underline{X}]) \to 0 (f_0) (k_s [\underline{X}]) \to \{\pm 1\} \to 1.
$$
This gives in cohomology the sequence
\begin{equation*}
0 \to H^1 (\pi, 0 (f_0) (k_s [\underline{X}])) \xrightarrow{\eta} H^1 (\pi, \{\pm 1\}). 
\tag{*}
\end{equation*}
The fibre over the trivial element is thus trivial. We want to show that the fibre over {\em any} element of $H^1 (\pi, \pm 1)$ is trivial. Using the splitting the fibre can be identified (by a standard twisting argument) with the fibre over the trivial element for a twisted form $f$ of $f_0$:
$$
H^1 (\pi, O(f) k_s [\underline{X}]) \to H^1 (\pi, \pm 1);
$$
and since $H^1 (\pi, SO(f)(k_s [\underline{X}]))=0$ it follows that $\eta$ in $(*)$ is a bijection. Comparing with the Galois cohomology for $k_s$-points we conclude that $H^1 (\pi, O(f_0) (k_s [\underline{X}]))$ is in bijective correspondence with $H^1 (\pi, O(f_0) (k_s))$. To conclude the proof of Theorem \ref{art17-alphthmE}, we need only show that $H^1 (\pi, O(f_0)\\ k_s [\underline{X}])$ classifies quadratic forms over $k[\underline{X}]$. Equivalently we have to show that all orthogonal bundles over $\Spec k_s [\underline{X}]$ are trivial. And this follows from Theorem \ref{art17-alphthmB} since (in view of our assumption that $\Char k \neq 2$) any $O(n)$-bundle admits an $SO(n)$-reduction.

\begin{remarks*}
\begin{enumerate}
\item[(I).] Theorems\pageoriginale can be formulated for general connected $G$ using the main results of this paper. For this one takes into account the following facts:
\begin{itemize}
\item[(i)] For a {\em split} unipotent $G$ any principal $G$-bundle is trivial (This is obvious).

\item[(ii)] If $T$ is a torus then any torus bundle over $A^n$ is obtained by base change from $k$, (This can be proved by the following observations. When $k$ is separably closed this amounts to saying that all line bundles are trivial; the general case follows from Galois cohomology once one observes that $T (\bar{k}_s [X]) \xrightarrow{\sim} T (\bar{k})_s)$.
\end{itemize}

\item[(II).] Theorem \ref{art17-alphthmA} has the following consequence: Given any $G$-bundle $P$, $G$ acceptable, over $A^n$, there is a product decomposition $A^n \simeq A^1 \times A^{n-1}$ such that $P$ extends to a bundle over $\mathscr{P}^1 \times A^{n-1}$. This suggests that a classification of $G$-bundles over $A^n$ is closely connected with the study of families of bundles on $\mathscr{P}^1$.

\item[(III).] One can prove a sharper result than Theorem \ref{art17-alphthmD} for quasi-split groups. The statement is:

Assume that $G$ is quasi-split acceptable and that the map $H^1 (k,G)\break \to H^1 (k, AdG)$ is trivial. Suppose further that the central isogeny $\tilde{G} \to G$ of the simply connected covering group $\tilde{G}$ on $G$ is {\em separable}. Then any principal $G$-bundle on $A^n$ is obtained by base change from $k$.

\end{enumerate}
\end{remarks*}

This is proved by looking at the Galois cohomology exact sequence (for $k_s[\underline{X}]$ rational points) associated to the exact sequence
$$
1 \to C \to \tilde{G} \to G \to 1
$$
It appears likely that the hypothesis that the isogeny is separable is not really necessary.

\begin{thebibliography}{99}
\bibitem[1]{art17-key1} H. Bass: {\em Quadratic modules over polynomial rings, } Preprint.

\bibitem[2]{art17-key2} F. Bruhat and J. Tits: in (Proceedings of a Conference on) {\em Local Fields,} Ed. T.A. Springer, Springer-Verlag, (1967).

\bibitem[3]{art17-key3} C. Chevalley: Sur\pageoriginale certaines groupes simples, {\em Tohoku Math. J(2)}. 7(1955), 14-66.

\bibitem[4]{art17-key4} A. Grothendieck: El\'ements de G\'eometrie Alg\'ebriques, {\em Publ. Math. No.} 11 (1961) {\em I.H.E.S.,} Paris.

\bibitem[5a]{art17-key5a} G. Harder: Uber die Galoiskohomologie halbeinfacher Matrizengruppen, II, {\em Math. Zeit.} 92(1966), 396-415.

\bibitem[5b]{art17-key5b} G. Harder: Chevalley groups over function fields and automorphic forms,  {\em Ann. Math.} 100(1974), 249-306.

\bibitem[6a]{art17-key6a} M. Knebush: Grothendieck- und Wittringe von nicht-ausgearteten symmetrischen Bilinear-formen, Sitz, {\em der Heidelberg. Akad. der Wiss., Math-naturwiss. Kl. 3 Abk} (1970), 90-157.

\bibitem[6b]{art17-key6b} M. Kneser: Einfach zussamenh\"angende algebraische Gruppen in der Arithmetick, {\em Proc. Int. Cong., Stockholm},(1962).

\bibitem[7]{art17-key7} S. Lang: Algebraic groups over finite fields, {\em Amer. J. Math.} 78(1956), 555-563. 


\bibitem[8a]{art17-key8a} M. Ojanguren and R. Sridharan: Cancellation of Azumaya-Algebras, {\em J. of Alg. } 18(1971), 501-505.

\bibitem[8b]{art17-key8b} S. Parimala and R. Sridharan: Projective modules over polynomial rings over division rings, {\em J. Math., Kyoto Univ.,} 15(1975), 129-148.

\bibitem[9]{art17-key9} S. Parimala: {\em Failure of a quadratic analogue of Serre\'s conjecture, Preprint}.

\bibitem[10]{art17-key10} D. Quillen: {\em Projective modules over polynomial rings,} Preprint.

\bibitem[11]{art17-key11}  M. S. Raghunathan and A. Ramanathan: {\em Principal Bundles over the affine line,} Preprint.

\bibitem[12]{art17-key12} J. P. Serre: {\em Cohomologie Galoisienne,} Lecture Notes in Mathematics 5(1965), Springer-Verlag.

\bibitem[13]{art17-key13} R. Steinberg: Variations on a theme of Chevalley, {\em Pacific J. Math. 9(1959), 875-891.}

\bibitem[14]{art17-key14} R. Steinberg: Regular elements of semisimple algebraic groups, {\em Pub. Math.} No. 25(1965), {\em I.H.E.S.,} Paris.

\bibitem[15]{art17-key15} O. T. O'Meara: {\em Introduction to Quadratic Forms,} Springer-Verlag (1963).

\bibitem[16]{art17-key16} H. Bass: {\em Algebraic $K$-Theory, Benjamin,} (1968).

\end{thebibliography}
