\title{The Invariance of Milnor's Number Implies The Invariance of the Topological Type}\label{art10}
\markright{The Invariance of Milnor's Number Implies.....}

\author{By~ L$\hat{e}$ D$\tilde{\text{u}}$ng Tr\'ang and C. P. Ramanujam.}
\markboth{L$\hat{e}$ D$\tilde{\text{u}}$ng Tr\'ang and C. P. Ramanujam}{The Invariance of Milnor's Number Implies.....}

\date{}
\maketitle

\footnotetext{Manuscript received June 22, 1973, \newline
{\em American Journal of Mathematics}, Vol. 98, No. 1, pp, 67-78 \newline
Copyright \copyright 1976 by Johns Hopkins University Press.
}

\setcounter{page}{135}

{\bf Introduction.}  
\setcounter{pageoriginal}{106}
We\pageoriginale are interested in analytic
families of $n$-dimen\-sional hypersurfaces having an isolated
singularity at the origin. In this paper we only consider the case
when the Milnor\'s number of the singularity at the origin does not
change in this family. Under this hypothesis with $n=1$, H. Hironaka
conjectured that the topological type of the singularity does not
change. We give a proof of this conjecture in the more general case of
$C^\infty$ family of $n$-dimensional hypersurfaces of dimension $n
\neq 2$. The hypothesis $n \neq 2$ comes from the fact we are using
$h$-cobordism theorem. Actually a more general conjecture should be
the following:


\subsection{}%%% 0.1
let $F(z_0,\ldots, z_n, t)$ be analytic in the $z_i$ and smooth (i.e. $C^\infty$) in $t$. Suppose that for any $t_0 \in \bfR$ the complex hypersurface $F (z_0,\ldots, z_n, t_0)=0$ of $\bfC^{n+1}$ has an isolated singularity at the origin. Suppose that the Milnor's number $\mu_{t_0}$ of this singularity, say the complex dimension of $\bfC \{z_0, \ldots, z_n\} / (\partial F_{t_0})$ is the ideal generated by the partial derivatives \break $(\partial \bfF / \partial z_i) (z_0, \ldots, z_n, t_0)$ $(i=0, \ldots, n)$ in this algebra, does not depend on $t_0$. Then the pair composed of the smooth part of the hypersurface $F=0$ in a neighborhood $U$ of $0$ in $\bfC^{n+1} \times \bfR$ and $(\{0\} \times \bfR) \cap U$ satisfies Whitney conditions at each point of $(\{0\} \times\bfR) \cap U$, (cf. \cite{art10-key17} p. 540).

If such a conjecture is true we obtain easily our result by using Thom-Mather isotopy theorem (cf. \cite{art10-key16} and \cite{art10-key7}).

Actually recent results of B. Teissier \cite{art10-key14} give a numerical condition to get Whitney conditions. Precisely let $(X, x)$ be a germ of hypersurface in $\bfC^{n+1}$ with an isolated singularity, let $E$ be a $k$-dimensional affine subspace of $\bfC^{n+1}$ passing through $x$. If $E$ is chosen sufficiently general the Milnor's number of $X \cap E$ at $x$ does not depend on $E$, we denote this number by $\mu^{(k)} (X,x)$. Then B. Teissier proved that if in an analytic family of germs of hypersurfaces $(X_t, 0)$ in $\bfC^{n+1}$ defined by $F(z_0,\ldots, z_n ,t)=0$ and having an isolated singularity at the origin  0 the  numbers $\mu^{(k)} (X_t, 0)$ for $1 \leqslant k \leqslant n+1$ do not depend on $t$, the smooth part of the hypersurface\pageoriginale $F(z_0, \ldots, z_n, t)=0$ in a neighborhood $U$ of $\bfC^{n+1} \times \bfC$ satisfies Whitney conditions along $(\{0\} \times \bfC) \cap U$.

When $n=1$ our result shows that the multiplicity of the curves of our analytic family is constant, because in this case we know that the multiplicity of a plane curve singularity is a topological invariant. 

Actually when $n=1$, O. Zariski in \cite{art10-key19} and \cite{art10-key20} proved that if the sum of $\mu_t + m_t -1$, where $\mu_t$ is the Milnor's number of the singularity and $m_t$ its multiplicity, is independent of $t$ then the smooth part of the surface $F(x,y,t)=0$ in a neighborhood $U$ of 0 in $\bfC^3$ satisfies Whitney conditions along $(\{0\} \times \bfC) \cap U$. In his terminology the preceding surface is equisingular along its singular locus at 0. Then the constancy of $\mu_t$ implying the one of $m_t$, Teissier's result is analogous to Zariski's result.

\subsection{}  
It is then natural to conjecture, following B. Teissier (cf. \cite{art10-key14}) that the constancy of the $\mu^{(n+1)}_t$ implies the constancy of the $\mu^{(k)}_t (1 \leqslant k \leqslant n)$, which would imply (0.1). Recently J. Brianson and J. P. Speder disproved this conjecture.

\section{Milnor's Results on the Topology of Hypersurfaces}\label{art10-sec1}
In this paragraph we first recall Milnor's results on the topology of hypersurfaces (cf. \cite{art10-key9}).

Let $f: U \subset \bfC^{n+1} \to \bfC$ be an analytic function on an open neighborhood $U$ of 0 in $\bfC^{n+1}$. We denote
\begin{align*}
B_\epsilon & = \{z |z\in \bfC^{n+1} : || z || \leqslant \epsilon\}\\
S_\epsilon & = \partial B_\epsilon = \{z|z \in \bfC^{n+1} : ||z|| = \epsilon\}.
\end{align*}

Then:

\begin{thm}\label{art10-thm1.1}
For $\epsilon > 0$ small enough the mapping $\varphi_\epsilon : S_\epsilon - \{f=0\} \to S^1$ defined by $\varphi_\epsilon (z) = f(z)/|f(z)|$  is a smooth fibration.
\end{thm}

\begin{thm}\label{art10-thm1.2}
For $\epsilon >0$ small enough and $\epsilon \gg  \eta > 0$ the mapping $\psi_{\epsilon, n}: \dot{B}_\epsilon \cap f^{-1} (\partial D_\eta)\to \bfS^1$ defined by $\psi_{\epsilon, \eta}(z) = f(z)/ |f(z)|$, where $\partial D_\eta = \{z|z \in \bfC: |z| = \eta\}$, is a smooth fibration isomorphic to $\varphi_\epsilon by$ an isomorphism which preserves the arguments. We call the fibrations of Theorems \ref{art10-thm1.1} and \ref{art10-thm1.2} the Milnor's fibrations of $F$ at 0.
\end{thm}

\begin{corollary}\label{art10-coro1.3}
The fibers of $\varphi_\epsilon$ have the homotopy type of a $n$-dimen\-sional finite $CW$-complex.
\end{corollary}

\begin{thm}\label{art10-thm1.4}
For $\epsilon >0$ small enough, $S_\epsilon$ transversally cuts the smooth part of the algebraic set $H_0$ defined by $f=0$. If $0$ is an isolated critical point of $f$, then the pairs $(S_\epsilon, S_\epsilon \cap H_0)$ for any $\epsilon$ small enough are diffeomorphic and $(B_\epsilon, B_\epsilon \cap H_0)$\pageoriginale is homeomorphic to $(B_\epsilon, C (S_\epsilon \cap H_0))$, where $C (S_\epsilon \cap H_0)$ is the real cone, union of segments with vertices at 0 and at a point of $S_\epsilon \cap H_0$.
\end{thm}

Theorem \ref{art10-thm1.4} says that, when 0 is an isolated critical point of $f$, for $\epsilon > 0$ small enough, the topology of the pair $(B_\epsilon, B_\epsilon \cap H_0)$  does not depend on $\epsilon$. Then if $g$ is another analytic function defined in a neighborhood of 0, having an isolated critical point at 0, we say that the hypersurfaces $H_0$ and $H'_0$ defined by $f=0$ and $g=0$ have {\em the same topological type at 0 } if for $\epsilon >0$ small enough there is a homeomorphism $(B_\epsilon, B_\epsilon \cap H_0) \xrightarrow{\sim} (B_\epsilon, B_\epsilon \cap H'_0)$.

From \cite{art10-key9} and \cite{art10-key11} we have:

\begin{thm}\label{art10-thm1.5}
If 0 is an isolated critical point of $f$, for $\epsilon > 0$ small enough, the fibers of $\varphi_\epsilon$ have the homotopy type of a bouquet of $\mu$ spheres of dimension $n$ with
$$
\mu = \dim_{\bfC} (\bfC\{z_0, \ldots, z_n\}/ (\partial f/ \partial z_0, \ldots, \partial f/ \partial z_n)).
$$
\end{thm}

(A bouquet of spheres is the topological space union of spheres having a single point in common).

We call the number of spheres {\em the Milnor's number} of the critical point 0 of $F$ or the {\em number of vanishing cycles} of $F$ at 0.

Actually we have:


\begin{prop}\label{art10-prop1.6}
The germ of morphism $\psi: (\bfC^{n+1},0) \to (\bfC^{n+1},0)$ whose components are the partial derivatives of $f$ is an analytic covering of degree $\mu$. On another hand, for $\epsilon > 0$ small enough, the mapping $\psi_\epsilon: S_\epsilon \to \bfS^{2n+1}$ defined by 
$$
\psi_\epsilon (z) = (\partial f / \partial z_0 (z), \ldots, \partial f / \partial z_n(z)) \Big/\sqrt{\sum\limits^n_{i=0} | \partial f / \partial z_i (z)|^2}
$$
has the degree $\mu$.
\end{prop}

Finally recall a well-known result of P. Samuel (cf. \cite{art10-key12}):

\begin{thm}\label{art10-thm1.7}
Let $f : U \subset \bfC^{n+1} \to \bfC$ be an analytic function on a neighborhood $U$ of 0 in $\bfC^{n+1}$. Suppose $f(0)=0$ and 0 is an isolated critical point.  Then there exists a polynomial $f_0 : \bfC^{n+1} \to \bfC$ with an isolated critical point at 0 and an analytic isomorphism of a neighborhood $U_1$ of 0 onto a neightborhood $U_2$ of 0 which sends the points of $f =0$ on points of $f_0 =0$.
\end{thm}

\section{The Main Theorem}\label{art10-sec2}
We prove the following theorem:

\begin{thm}\label{art10-thm2.1}
Let $F(t,z)$ be a polynomial in $z = (z_0, \ldots, z_n)$ with coefficients which are smooth complex valued functions of $t \in I = [0,1]$ such that\pageoriginale $F(t, 0) =0$ and such that for each $t \in I$, the polynomials $(\partial F/ \partial z_i) (t,z)$ in $z$ have an isolated zero at 0. Assume moreover that the integer
$$
\mu_t = \dim_{\bfC} \bfC \{z\} / \left(\frac{\partial f}{\partial z_0} (t,z) , \ldots, \frac{\partial f}{\partial z_n} (t,z) \right)
$$
is independent of $t$. Then the monodromy fibrations of the singularities of $F (0,z)=0$ and $F(1,z) =0$ at 0 are fo the same fiber homotopy. If further $n \neq 2$, these fibrations are even differentiably isomorphic and the topological types of the singularities are the same.
\end{thm}

A crucial step in the proof of the theorem is the following lemma:

\begin{lemma}\label{art10-lem2.2}
Let $f \in \bfC[z_0, \ldots, z_n]$, $f(0) =0$, and $R >0$, $\epsilon > 0$ such that 
\begin{itemize}
\item[\rm (i)] $\dim_{\bfC} \bfC \{z\} / (\partial f / \partial z_0, \ldots, \partial f / \partial z_n) = \mu < \infty$;

\item[\rm (ii)] for any $z$ with $0 <  ||z|| < R$ and $|f(z)| \leqslant \epsilon$, we have $df\neq 0$;

\item[\rm (iii)] for any $z \in S_R = \{z |~ ||z|| = R\}$ with $|f(z)| \leqslant \epsilon$, $d (f|S_R)$ is of rank 2;

\item[\rm (iv)] for some $w_0$ with $0 < |w_0|\leqslant \epsilon$, $F_{w_0} = \{z| ~ ||z|| \leqslant R, f(z) = w_0\}$ is of the homotopy type of a bouquet of $\mu$ $n$-spheres.
 \end{itemize}

Then the map
\begin{equation}
\{z | ~ ||z|| \leqslant R, |f(z)| = \epsilon \} \xrightarrow{f/|f|} \bfS^1 \label{art10-A}
\end{equation}
is a fibration fiber homotopy equivalent to the monodromy fibration of the singularity of $f=0$ at 0. If further $n \neq 2$, it is even diffeomorphic to the monodromy fibration, and there is a homeomorphism of the set
$$
\{z| ~ ||z|| \leqslant R, | f(z)| = \epsilon \} \cup \{ z| ~ ||z|| = R , |f(z)|\leqslant \epsilon \}
$$
with the sphere $\{z| ~||z|| = \delta\}$ ($\delta$ small) which maps the set $\{z|~||z|| = R, f(z) =0\} $ onto the set $\{z|~ ||z|| = \delta, f(z)=0 \}$.
\end{lemma}

\begin{proof}
It is clear from (ii) and (iii) that the map
\begin{equation}
f: \{z|~ ||z|| \leqslant R,~ 0 < |f (z)| \leqslant \epsilon\}  \to \{w |0 < |w| \leq \epsilon \} \label{art10-B}
\end{equation}
is a locally trivial differentiable fibration, hence so is (A). Choose $\delta$, $\eta$ with $0< \delta < R$, $0< \eta< \epsilon$ such that 
\begin{equation}
f/ |f|: \{z | ~ ||z|| \leqslant \delta, |f(z)| = \eta\} \to \bfS^{1} \label{art10-C}
\end{equation}
is the monodromy fibration of the singularity. This fibration is contained in the fibration\pageoriginale 
\begin{equation}
f/ |f|: \{z|~ ||z|| \leqslant R, ~ |f(z)| = \eta \} \to \bfS^1, \label{art10-D}
\end{equation}
and (\ref{art10-A}) and (\ref{art10-D}) are diffeomorphic fibrations since (\ref{art10-B}) has been shown to be a fibration. Thus it suffices to show that the inclusion of (\ref{art10-C}) in (\ref{art10-D}) is a fiber homotopy equivalence. By the theorem of Dold (\cite{art10-key4} Theorem 6.3) and the homotopy sequence of a fibration, it suffices to show that the inclusion of the fiber $G_1$ over 1 of (\ref{art10-C}) in the fiber $F_1$ over 1 of (D) is a homotopy equivalence. Now, since the spheres $S_R$ and $S_\delta$ of radii $R$ and $\delta$ respectively are transversal to the manifold $f(z) = \eta$, for all $z_0$ close enough to the origin, the pair of manifolds with boundaries $(F_1, G_1)$ is diffeomorphic to the pair
$$
(\{z | f(z) = \eta, ||z - z_0|| \leqslant R\}, ~ \{z / f (z) = \eta, ||z- z_0|| \leqslant \delta \}).
$$
Choose such $z_0$ such that the distance function $||z - z_0||$ has
only nondegenerate critical points on $f(z) = \eta$. Since the indices
of this function at these critical points are $\leqslant n$, it
follows that $F_1$ is obtained from $G_1$, up to homotopy type, by
attaching cells of dimension $\leqslant n$. Now, it follows from (iv)
that $F_1$ is of the homotopy type of the wedge $\mu$ $n$-spheres, and
the same holds of $G_1$, by \cite{art10-key9}. It clearly follows that
for $n=1$, the inclusion $G_1 \raisebox{2pt}{\rotatebox{-20}{$\curvearrowbotright$}}\, F_1$ is a homotopy
equivalence. For $n>1$, both spaces are simply connected, and it
suffices to show that $H_i (G_1, \bfZ) \to H_i (F_1, \bfZ)$ is an
isomorphism for all $i$. This is clearly so for $i \neq n$ and it also
holds for $i=n$ since $H_{n+1} (F_1, G_1; \bfZ) =0$, $H_n(G_1, \bfZ)$
and $H_n (F_1, \bfZ)$ are both free of rank $\mu$ and $H_n
(F_1,G_1; \bfZ)$ is torsion-free. 

We have thus shown that the inclusion of (C) in (D) is a homotopy equivalence. We have yet to establish the stronger assertions of the lemma for $n \neq 2$. First notice that if $X = \{z | \delta \leqslant ||z|| \leqslant R, |f(z)|\leqslant \eta\}$, then $f:X \to \{w || w| \leqslant \eta\}$ is a differentiable fibration. This follows from (ii), (iii) and the fact that $d(f|S_\delta)$ is of rank 2 on $X \cap S_\delta$. It follows that $X$ is diffeomorphic to a product of the disc $\{w||w|\leqslant \eta\}$ and the fiber $X_w$ of $X$ over any point $w$ of this disc, in particular $X_\eta = F_1 - \Int G_1$. Now, $H_* (F_1, G_1 ; Z) =H_* (X_\eta, \partial G_1; \bfZ) = (0)$. So that $H_* (\partial G_1, \bfZ) \to H_* (X_\eta; \bfZ)$ is an isomorphism. For $n=1$, it follows from this and the classification of surfaces with boundary that $X_\eta$ is diffeomorphic to $I \times \partial G_1$. For $n>2$, $\partial G_1$ is simply connected (\cite{art10-key9}) and the inclusion $\partial G_1 \subset X_\eta$ is a homotopy equivalence. Further, by considering the Morse function $-||z-z_0||$ with $z_0$ close to the origin, and noticing that its indices at critical points on $F_1$ are $\geqslant n$, (\cite{art10-key1}) we see that $\pi_1 (\partial F_1) \to \pi_1 (F_1)$ is injective, so that $\partial F_1$ is also simply connected. If we can show that $H_* (X_\eta, \partial F_1; \bfZ) = (0)$, it would follow that the inclusion $\partial F_1 \raisebox{2pt}{\rotatebox{-20}{$\curvearrowbotright$}}\, X_\eta$ is a homotopy equivalence. From the homology exact sequence of the triple $(F_1, X_\eta, \partial F_1)$, it suffices to show that $H_* (F_1, \partial F_1)$ $\to H_* (F_1, X_\eta)$\pageoriginale is an isomorphism, or passing to cohomology, that $H^*_c (\Int G_1)$ $\to H^*_c (\Int F_1)$ is an isomorphism, where $H^*_c$ denotes cohomology with compact support. By Poincare duality, this is the same as saying that $H_* (G_1, \bfZ)$ $\to H_* (F_1, \bfZ)$ is an isomorphism, which is certainly true.

Thus, the inclusions $\partial G_1 \subset X_\eta$ and $\partial F_1 \subset X_\eta$ are homotopy equivalences, the spaces are simply connected and we are in real dimensions $\geqslant 6$. It follows from the $h$-cobordism theorem \cite{art10-key8} that $X_\eta$ is diffeomorphic to $I\times \partial G_1$. Thus, the fibration (D) is obtained from the fibration (C) (which restricted to the boundary of the total space of (C) is trivial) by attaching $S^1 \times $ collar. Hence the two fibrations are diffeomorphic by a diffeomorphism $\varphi$. Further, if we fix a diffeomorphism $X \xleftarrow{\utilde{\lambda}} D \times I \times \partial G_1$ over $D$, where $D$ denotes the $\disc$ $\{w||w| \leqslant \eta\}$, we may assume that the points $\lambda (w, 0, x)$ and $\lambda (w, 1, x)$ correspond to each other under $\varphi$. But now, the canonical diffeomorphism of $D \times 0 \times \partial G_1$ and $D \times 1 \times \partial G_1$ goes over by $\lambda$ into a diffeomorphism of $\{z| ~ ||z|| = \delta, |f(z)| \leqslant \eta)\}$ onto $\{z|~ ||z|| = R, |f(z)| \leqslant \eta)\}$ such that $\varphi$ and $\psi$ are equal at points where both are defined and $\psi$ carries the zero set of $f$ (i.e., $\lambda (0 \times 0 \times \partial G_1)$) in its domain onto the zero set of $f$ $(i.\epsilon.,, \lambda (0 \times 1 \times \partial G_1))$ in its range. Now, by \cite{art10-key9} there is a homeomorphism of $\{z| ~ ||z|| \leqslant \delta, |f(z)| = \eta \} \cup \{z | ~ || z|| = \delta, |f(z)| \leqslant \eta\}$   onto $S_\delta$ which is the identity on the second of these sets. This establishes the last assertion of the lemma, except for the fact that $\epsilon$ is replaced by $\eta$. But this clearly does not matter, in view of assumptions (ii) and (iii). Q.E.D.
\end{proof}

We need another simple lemma:

\begin{lemma}\label{art10-lem2.3}
Let $K$ be a compact convex set in $\bfC^{n+1}$ and $g_1,\ldots, g_{n+1}$ holomorphic functions in a neighborhood of $K$ such that on $\Bd K$,\break $\sum |g_i|^2 > 0$. Orient $\Bd K$ (which is homeomorphic to $S^{2n+1}$) such that a radial projection onto an interior sphere has degree $+1$. Then the $g_i$ have only finitely many common zeros in $K$, and the degree $\mu$ of the map $\partial K \to S^{2n+1}$ given by $z \mapsto (\sum | g_i (z)|^2)^{-1/2}$ $(g_1(z), \ldots, g_{n+1} (z))$ equals $\sum_{P\in K} \mu_p$, where
$$
\mu_p = \dim_{\bfC} \mathscr{O}_{\bfC n +1_P} / (g_1, \ldots, g_{n+1}).
$$
\end{lemma}

\begin{proof}
Since the $g_i$ have no common zeros on $\Bd K$, their common zeros in $K$ form an analytic set in $\bfC^{n+1}$, and $K$ being compact, they must be a finite set. We proceed by induction on the number $N$ of these common zeros. If $N=1$, let $P\in \Int K$ be the unique common zero, $B$ a small around $P$ contained in $\Int K$ and $S$ the boundary of $B$. The map $\Bd K \to S^{2n+1}$ is then homotopic to the composite $\Bd K\xrightarrow{\alpha} S = Bd B \to S^{2n+1}$ where $\alpha$ is radial projection and the map $\Bd B \to S^{2n+1}$ is the map defined above with $K$ replaced by $S$. The result then follows from \cite{art10-key9}.

If $N>1$,\pageoriginale choose a hyperplane $H$ such that there are no common zeros of the $g_i$ on $H$ and such that if $H^+$ and $H^-$ are the closed half-spaces defined on $H$, $H^+ \cap K$ and $H^- \cap K$ each contain at least one common zero of the $g_i$. Since we may assume by induction that the result holds for each of the compact convex sets $H^+ \cap K$ and $H^- \cap K$, we have only to show that the degree of $\Bd K \to S^{2n+1}$ equals the sum of the degrees of $\Bd (K \cap H^+) \to S^{2n+1}$ and $\Bd (K \cap H^-) \to S^{2n+1}$. Since  the intersection $\Bd (K \cap H^+) \cap \Bd (K \cap H^-)$ is contractible on each of these boundaries, the assertion reduces to the standard and easy fact that if $f,g:S^{2n+1} \to S^{2n+1}$ are maps preserving some base point, $h:S^{2n+1} \to S^{2n+1} \bigvee S^{2n+1 }$ the pinching map, the composite $S^{2n+1} \xrightarrow{h} S^{2n+1} \bigvee S^{2n+1} \xrightarrow{(f,g)} S^{2n+1}$ has degree equal to the sum of the degrees of $f$ and $g$. Q.E.D.
\end{proof}


\noindent
{\bf Proof of Theorem 2.1.}
We set $f_t (z) = F (t,z)$. It is clearly enough to prove the theorem with $I$ replaced by some smaller interval $[0,\delta]$ with $\delta >0$. Denote by $\mu$ the constant value of $\mu_t$, $t \in I$. Choose $R$ and $\epsilon$ such that the conditions (ii), (iii) and (iv) hold with $f$ replaced by $f_0$, and also such that $\sum^{n+1}_1 |\partial f_0/ \partial z_i|^2 > 0$ on $||z|| = R$. By continuity, one sees that there is a $\delta > 0^1$ such that for any $t \in [0, \delta]$, (iii) holds and also $\sum^{n+1}_1 |\partial f_t / \partial z_i |^2 > 0$ on $|| z|| = R$. The maps $S_R \to S^{2n+1}$ are defined by $||\grad f_t ||^{-1}$. $(\partial f_t / \partial z_i, \ldots, \partial f_t/ \partial z_{n+1})$ for various $t$ are all homotopic, hence all of the degree $\mu$. On the other hand, we know that at the origin, $\dim_{\bfC} \bfC\{z\} / (\partial f_t/ \partial z_1, \ldots, \partial f_t/ \partial z_{n+1}) = \mu$. It follows from Lemma \ref{art10-lem2.3} that (ii) is also fulfilled by $f_t$, $R$ and $\epsilon$ if $t \in [0, \delta] = I_1$.


Let $X = \{(t,z) \in I_1 \times \bfC^{n+1} | ~ || z|| \leqslant R, ~ 0 < | F (t,z)| = \epsilon\}$, and define $\Phi: X \to I_1 \times \{w |0 < | w |\leqslant \epsilon\}$, $\Phi (t,z) = (t, F (t,z))$. Then $\Phi$ has compact fibers, and is of maximal rank on the sets $\{(t,z \in X | ~||z|| < R)\}$ as well as when restricted $X \cap I_1 \times S^{2n+1}_R$. Thus, $\Phi$ is a differentiable fibration. Similarly, if $Y: \{(t,z) \in I_1 \times S_R ||F (t,z) | \leqslant \epsilon\}$, the $\map \psi: Y \to I_1 \times \{w ||w|\leq \epsilon\}$, $\psi (t,z) = (t, F(t,z))$ is a smooth fibration. It follows that $\psi$ is differentiably a trivial fibration, and that $\Phi | X \cap (0 \times \bfC^{n+1}): X \cap (0 \times \bfC^{n+1}) \to \{w |0 < |w| \leqslant \epsilon\}$ and $\Phi|X \cap (\delta \times \bfC^{n+1}): X \cap (\delta \times \bfC^{n+1}) \to \{w |0 < |w| \leq \epsilon\}$ are smoothly isomorphic fibrations such that this isomorphism is compatible with the chosen trivialisation of $\psi$.

The theorem now follows from Lemma 1. \hfill{Q.E.D.}

We now deduce some corollaries. In order to shorten statements we introduce the following definition. We say that the isolated hypersurface singularities at 0 in $\bfC^{n+1}$ defined by the equations $f=0$ and $g=0$ are of the {\em same type} if (i) their monodromy fibrations are fiber homotopic, and (ii), for $n \neq 2$, these fibrations are differentiably isomorphic, and there is a homeomorphism of $S^{2n+1}$ onto itself carrying the zero set of $f$ on $S^{2n+1}$ onto that of $g$, this implies in particular that the matrices of the monodromy transformations on the\pageoriginale $n$-th homology of the fiber of the Milnor's fibrations of $f$ and $g$ are inner conjugate in $\bfS \bfL (\mu, \bfZ)$, $\mu$ being the number of vanishing cycles.

\begin{corollary}\label{art10-coro2.4}
Let $f \in \bfC[z_1,\ldots, z_{n+1}]$, and $p_i \in \bfZ$, $p_i > 0 (1\leqslant i \leqslant n +1)$ be weights for the variables $z_i$, $f = f_N + f_{N+1} + \ldots$ the decomposition of $f$ into quasi-homogeneous polynomials $f_p$ of weight $p$ with $f_N \neq 0$, suppose 0 is an isolated critical point of $f_N = 0$. Then the same holds for $f=0$, and the singularities $f=0$ and $f_N =0$ at $0$ are of the same type.
\end{corollary}

\begin{proof}
Put $F (t, z) = f_N + t f_{N+1} +  t^2 f_{N+2} + \cdots$, so that for $t \neq 0$, $F(t,z) = t^{-N} f (t^{p_1} z_1, \ldots, t^{p_{n+1}} z_{n+1})$. Choose $R$ so that $(df_N)_z \neq 0$ for $0 < ||z|| \leqslant R$, and let $\mu = \dim_{\bfC} \bfC \{z\} / (\partial f_N / \partial z_1, \ldots, \partial f_N / \partial z_{n+1}) $. Then for $t$ close enough to $0$, $d_z F(t,z) \neq 0$ on $||z|| = R$, so that $F (t,z)$ has only isolated critical points in $||z|| \leqslant R$. Since $F(t,z)$ derives from $f(z)$ by a trivial coordinate transformation, $f(z)$ has 0 for an isolated critical point. Now choose $R_1 > 0$ such that both $f$ and $f_N$ have 0 as their only critical point in $||z|| \leqslant R_1$. Then the same is true of $F(t,z)$ for $0 \leqslant t \leq 1$. Further for all $t \in [0,1]$, the maps $S_R \to S^{2n+1}$ given by
$$
z \mapsto ||\grad_z F (t,z)||^{-1} \left(\dfrac{\partial f_t}{\partial z_1} , \ldots, \dfrac{\partial f_t}{\partial z_{n+1}} \right)
$$
are homotopic, hence have the same degree. It follows from Lemma \ref{art10-lem2.3} that $\mu = \dim_{\bfC} \bfC \{z\} / (\partial f_t / \partial z_1, \ldots, \partial f_t / \partial z_{n+1})$ is independent of $t$. Now appeal to the theorem.\hfill{Q.E.D.} 
\end{proof}

\begin{remark}\label{art10-rem2.5}
It is in fact possible to make a stronger statement. There is a diffeomorphism $\varphi: S_\delta \to S_\delta$ ($\delta$ small) isotopic to the identity which maps $K = \{z \in S_\delta | f(z) =0\}$ onto $K_0 = \{z \in S_\delta | f_N(z) =0\}$ and such that for $z \in S_\delta - K$, $f(z)/ | f(z)| = f_N (\varphi (z)) / | f_N (\varphi (z))|$, i.e., $\varphi | S_\delta - K$ is a fiber preserving diffeomorphism of the monodromy fibration of $f$ onto that of $f_N$. This holds for all $n \geqslant 1$ without exception. This is a consequence of the fact that if we define $f_t(z) = F (t,z)$ as above, there is a $\delta > 0$ such that, for any with $0 < ||z|| =\delta$, and any $t \in [0,1]$,grad $\log f_t = \lambda$. $z$  with $\lambda$ complex implies that $|arg \lambda |< \pi /4$ and $||z|| = \delta$, $f_t = 0$ imply that $d(f_t | S_\delta)_z$ is of rank 2.
\end{remark}

To state the next corollary, we need a few definitions and facts from commutative algebra. Let $A$ be a noetherian normal local domain of dimension $n+1$ with maximal ideal $\mathfrak{M}$, and $\mathfrak{a}$ an $\mathfrak{M}$-primary ideal. An element $f \in A$ is said to be integral over $\mathfrak{a}$ if it satisfies an equation
$$
f^m + a_1 f^{m-1} + \ldots + a_m = 0, \quad a_i \in \mathfrak{a}^i.
$$
The set $\bar{\mathfrak{a}}$ of elements of $A$ integral over $\mathfrak{a}$ form an ideal containing $\mathfrak{a}$, called the integral\pageoriginale closure of $\mathfrak{a}$. (See \cite{art10-key13}, Appendix 4 and \cite{art10-key10} for all properties made use of.) If $e(\mathfrak{a})$ is the multiplicity of $\mathfrak{a}$, we have $e(\mathfrak{a}) = e(\bar{\mathfrak{a}})$ (\cite{art10-key10}, Theorem 1). Further, if $R$ runs through all the discrete valuation rings containing $A$ and having the same quotient field as $A$, we have $\bar{\mathfrak{a}} = \cap R \cdot \mathfrak{a}$ (\cite{art10-key13}, Appendix 4, Theorem 3). Finally, let $\alpha_i (1 \leqslant i \leqslant k)$ be elements of $A$ generating an $\mathfrak{M}$-primary ideal $\mathfrak{a}$. Then there exists a finite set of polynomials $P_\alpha (x_{ij})$ over the residue field in the variables $x_{ij} (1 \leqslant i \leqslant k, 1 \leqslant j \leqslant n +1)$ such that if 
$v = (n_{ij}) \in A^{k(n+1)}$ and $\mathfrak{a}_n = (\sum^k_{i=1} n_{i1} \alpha_i, \ldots, \sum^k_1 n_{in+1} \alpha_i)$; then $\mathfrak{a}$ is integral over $\mathfrak{a}_n$ (i.e., $\mathfrak{a} \subset \bar{\mathfrak{a}}_n$) if and only if $P_\alpha (\bar{n}_{ij}) \neq 0$ for some $\alpha$, where $\bar{n}_{ij}$ denotes the image of $n_{ij}$ in the residue field. (See \cite{art10-key10}, Sec. 5, Theorem 1.)

\begin{lemma}\label{art10-lem2.6}
Let $A$ be a regular local ring of dimension $n+1$ containing a field $k$ of characteristic 0 which gets mapped onto the residue field of $A$. Suppose $D_i (1 \leqslant  i \leqslant n +1)$ is a system of $k$-derivations of $A$ such that the induced maps $D_i: \mathfrak{M}/ \mathfrak{M}^2 \to A / \mathfrak{M}$ form a basis of the dual of $\mathfrak{M} / \mathfrak{M}^2$. Then if $f \in \mathfrak{M}$ such that the ideal $(D_1 f, \ldots, D_{n+1}f )$ is primary for the maximal ideal, $f$ belongs to the integral closure of $(D_1 f, \ldots, D_{n+1}f)$.
\end{lemma}

\begin{proof}
It suffices to show that if $A\subset R$, where $R$ is a discrete valuation ring, $f \in \sum^{n+1}_1 R \cdot D_i f$, we may further assume that the maximal ideal of $A$ is contained in that of $R$, since otherwise $\sum^{n+1}_1 R \cdot D_i f = R$ and the assertion is trivial. We may further suppose $R$ complete. We can then find a subfield $L$ of $R$ such that $L \supset k$ and $R$ is the formal power series ring $L[[T]]$ over $L$ in some $T$ generating  the maximal ideal of $R$. Let $D$ denote the $L$-derivation $d/dt$ of $R$, so that clearly $R f \subset R \cdot Df$. It suffices to show that $R \cdot D f \subset \sum^{n+1}_1 R \cdot D_i f$. Let $\Der_k (A,R)$ be the $R$-module of $k$-derivations of $A$ into $R$ continuous for the $\mathfrak{M}$-adic topology, so that $D|A \in \Der_k(A,R)$. We will be done if we show that $\Der_k(A,R)$ is generated by the $D_i(1 \leqslant i \leqslant n+1)$ choose a basis $x_i (1 \leqslant i \leqslant n+1)$ of $\mathfrak{M}$ such that $D_i x_j = \delta_{ij}(\mod \mathfrak{M})$. The homomorphism $\Der_k(A,R) \to R^{n+1}$ which sends $\bar{D}$ to $(\bar{D}x_1, \ldots, \bar{D}x_{n+1})$ is clearly injective. It is also surjective and the images of the $D_i$ generated $R^{n+1}$ by Nakayama, since it is so when we pass to the quotient modulo the maximal ideal of $R$. This proves the lemma. \hfill{Q.E.D.}

\end{proof}

We are now ready for the next corollary. For any $f \in \bfC[z_1, \ldots, z_{n-1}]$ with an isolated singular point at (0), define 
\begin{align*}
\mathfrak{a}_1 (f) & = \sum\limits^{n+1}_1 \bfC\{z\}  \cdot \frac{\partial f}{\partial z_i} = \{Df/D \text{~~ a continuous $\bfC$ - derivation of $\bfC \{z\}$}\},\\
\mathfrak{a}_2 (f) & = \bfC \{z\} f + \mathfrak{a}_1(f),\\
\mathfrak{a}_3 (f) & = \overline{\mathfrak{a}_1 (f)} = \text{ integral closure of } \mathfrak{a}_1 (f).
\end{align*}
It follows\pageoriginale from Lemma (\ref{art10-lem2.6}) that $\mathfrak{a}_1 (f) \subset \mathfrak{a}_2 (f) \subset \mathfrak{a}_3(f)$. Further, it is clear that for any $\bfC$-automorphism $\varphi$ of $\bfC\{z\}$, $\mathfrak{a}_i (\varphi(f)) = \varphi (\mathfrak{a}_i (f))$. Define the artinian local rings $A_i(f)$ by $A_i(f) = \bfC \{z\} / \mathfrak{a}_i(f)$.

\begin{corollary}\label{art10-coro2.7}
Suppose $f$, $g \in \bfC [z]$ and that for some $i (1 \leqslant i \leqslant 3)$ we are given an isomorphism over $\bfC$, $\lambda : A_i (f) \cong A_i(g)$. Then the singularities of $f=0$ and $g =0$ at 0 are of the same type.
\end{corollary}

\begin{proof}
First we start with proving weaker statements. Suppose $\mathfrak{a}_1 (f) \supset \mathfrak{M}^N$ and $g-f \in \mathfrak{M}^{N+2}$. Then $(\partial g / \partial z_i ) - (\partial f / \partial z_i) \in \mathfrak{M}^{N+1} \subset \mathfrak{M} \mathfrak{a}_1 (f)$, which shows that $\mathfrak{a}_1 (f) = \mathfrak{a}_1 (g)$. Hence for any $t$, if $h_t = tg + (1-t)f$, $\mathfrak{a}_1 (f) = \mathfrak{a}_1 (h_t)$, and it follows from the theorem that the singularities of $f$ and $g$ are of the same type. (Compare to \cite{art10-key15}.)

Next suppose $\varphi$ is a $\bfC$-automorphism of $\bfC\{z\}$ and $\varphi (f) = g$. We want to say that the singularities of $f$ and $g$ are of the same type. Let $J(\varphi)$ be the Jacobian matrix of $\varphi$, $\gamma (t)$ a smooth path in $\bfG \bfL(n+1, \bfC)$ connecting $E$ to $J (\varphi)^{-1}$, and $\phi_t$ the automorphism of $\bfC\{z\}$ induced by $\gamma (t)$. Applying the theorem to the family $h_t = \phi_t \circ \varphi (f)$, we see that we may assume $\varphi$ has Jacobian matrix the identity, so that we have $\varphi (z_i) = z_i + \psi_i(z)$, where $\psi_i$ begins with second degree terms. By what we said at the beginning of the proof, if $\psi'_i$ denotes the power series obtained from $\psi_i$ by leaving out from $\psi_i$ all terms of degree $N$ ($N$ large) and $\varphi'$ the automorphism of $\bfC\{z\}$ defined by $\varphi'\{z_i\} = z_i + \psi'_i(z)$, $\varphi (f) \equiv \varphi'(f)$ $(\mod \mathfrak{M}^N)$ so that $\varphi \{f\}$ and $\varphi'(f)$ have the same monodromy. Thus we may assume that the $\psi_i$ are polynomials. Now define an automorphism $\varphi_t$ of $\bfC\{z\}$ by $\varphi_t(z_i) = z_i + t \psi_i (z)$ and apply the theorem to $h_i = \varphi_t (f)$. Our assertion follows.

Suppose now that we have a $\bfC$-isomorphism $\bar{\lambda}: A_i(f) \xrightarrow{\sim} A_i (g)$. We can then find a $\bfC$-automorphism $\lambda$ of $\bfC\{z\}$ which makes the diagram
\[
\xymatrix{
\bfC \{z\} \ar[r]^\lambda  \ar[d] & \bfC\{z\}\ar[d]\\
\bfC \{z\} / \mathfrak{a}_i (f) \ar[r]^{\bar\lambda} & \bfC\{z\} / \mathfrak{a}_i (g)
}
\]
commutative, the vertical arrows being the natural maps. It follows that $\lambda (\mathfrak{a}_i (g))= \mathfrak{a}_i (g)$. On the other hand, we have $\lambda (\mathfrak{a}_i (f)) = \mathfrak{a}_i (\lambda (f))$. Finally, since $\overline{\mathfrak{a}_i(f)} = \mathfrak{a}_3 (f)$, we deduce that $\mathfrak{a}_3 (\lambda (f)) = \mathfrak{a}_3 (g)$. But now, since the singularities of $f$ and $\lambda(f)$ are of the same type, we may assume that we have $\mathfrak{a}_3(f) = \mathfrak{a}_3 (g)$. But now, there is a finite set $S \subset \bfC$, $0 \notin S$, $1 \notin S$, such that for $t \in \bfC - S$, the integral closure of the ideal $(t (\partial f/ \partial z_i) + (1-t)\partial g  / \partial z_i)$ also equals $\mathfrak{a}_3 (f)$. Join 0 and 1 in $\bfC$ by a smooth path $\gamma (t)$ not containing any point of $S$, and\pageoriginale apply the theorem to the family $h_t = \gamma (t) \cdot f + (1 - \gamma (t)) \cdot g$. Note that $\mu_t =\dim_\bfC \bfC \{z\}/ (\partial h_t/ \partial z_i)) = e(\mathfrak{a}_3 (f))$ is constant. \hfill{Q.E.D.}
\end{proof}

\section{Plane Curve Case.}\label{art10-sec3}
Let us show that our main theorem implies the conjecture 0.1 of the introduction when $n=1$. Because of Zariski's results in \cite{art10-key19}, it is sufficient to prove that the multiplicity $m_t$ of the curve $F(z_0, z_1, t) =0$ of $\bfC^2$ at 0 is independent of $t$.

But O. Zariski in \cite{art10-key21} and M. Lejeune in \cite{art10-key6} proved that:

\begin{thm}\label{art10-thm3.1}
The topological type of a plane curve singularity at a singular point is determined by the topological type of each analytically irreductible component of this curve at the singular point and the intersection numbers of any pairs of distinct branches.
\end{thm}

Then consider two plane curves defined by $f=0$ and $g=0$ and having an isolated singular point at 0. Suppose that these curves have the same topological type at 0. Then there is a one-to-one correspondence $\phi$ between their anayltically irreductible branches, such that the branch $\gamma_i$ of $f=0$ at 0 has the same topological type as the branch $\phi (\gamma_i)$ of $g=0$ at 0. Thus, as the multiplicity at 0 is the sum of the multiplicities of each branch, it suffices to prove that the multiplicity is a topological invariant of an analytically irreductible plane curve at its singular point.

More generally using results of K. Brauner in \cite{art10-key2}, W. Burau in \cite{art10-key3} and O. Zariski in \cite{art10-key18} we have:

\begin{thm}\label{art10-thm3.2}
Puiseux pairs (cf. \cite{art10-key5}) of ananalytically irreductible plane curve singularity depends only on the topology of the singularity. 
\end{thm}

If $(m_1,n_1), \ldots, (m_g, n_g)$ are the Puiseux pairs, one knowns that\break $n_1 \ldots n_g$ is the multiplicity of the singularity (cf. \cite{art10-key5}).

Centre de Mathematiques de L'Ecole Polytechnique

Penjab University.


\begin{thebibliography}{99}
\bibitem{art10-key1} A. Andreotti and T. Frankel, ``The Lefschetz theorem on hyperplane sections.'' Ann. of Math., 69 (1959), 713-717.

\bibitem{art10-key2} K. Brauner,\pageoriginale ``Zur geometrie der funktionen Zweier komplexen Verlanderlichen,'' {\em Abh. Math. Sem. Hamburg, 6 (1928), 1-54.}

\bibitem{art10-key3} W. Burau, ``Kennzeichnung der Schlauchknoten,'' {\em Abh. Math. Sem. Hamburg, } 9 (1932), 125-133.

\bibitem{art10-key4} A. Dold, ``Partitions of unity in the theory of fibrations,'' {\em Ann. Math.} 78 (1963), 223-255.

\bibitem{art10-key5} Le Dung Trang, ``Sur les noeuds algebriques,'' {\em Compositio Mathematica,} 25 (1972), 282-322. 

\bibitem{art10-key6} M. Lejeune, ``Sur l'equivalence des singularites des courbes algebroides planes,'' {\em Coefficients de Newton, } Centre de Math. de l'Ecole Polytechnique, 1969

\bibitem{art10-key7} J. Mather, ``Notes on topological stability,'' Preprint, Harvard Univ., 1971.

\bibitem{art10-key8} J. Milnor, Lectures on $h$-cobordism Theorem, Princeton.

\bibitem{art10-key9} J. Milnor, ``Singular points of complex hypersurfaces,'' {\em Ann. Math. Stud.,} 61, Princeton, 1968.

\bibitem{art10-key10} D. Northcott, D. Rees, ``Reductions of ideals in local rings,'' {\em Proc. Camb. Phil. Soc.,} 50 (1954), 145-158.

\bibitem{art10-key11} V. I. Palamodov, ``Sur la multiplicite des applications holomorphes,'' {\em Founk. Anal.,} i ievo prilojenia, tome 1, fasc. 3, (1967), 54-65 (en russe).

\bibitem{art10-key12} P. Samuel, ``Algebricite de certains points singuliers algebroides,'' {\em J. Math. Pures Appl.,} 35 (1956), 1-6.

\bibitem{art10-key13} P. Samuel, O. Zariski, {\em Commutative Algebra,} vol. 1 and 2, Van Nostrand, Princeton, (1958 and 1960).

\bibitem{art10-key14} B. Teissier ``Cycles evanouissants et conditions de Whitney,'' a paraitre aux C. R. Acad. Sc., Paris, 1973.

\bibitem{art10-key15} G. Tiourina, ``Sur les proprietes topologiques des singularites isolees espaces compolexes de codimension une,'' {\em Isvestia Ak. Naouk}, 32 (1968), 605-620 (en russe).

\bibitem{art10-key16} R. Thom, ``Ensembles et morphismes stratifies,'' {\em Bull. Amer. Math. Soc.,} 75 (1969).

\bibitem{art10-key17} H. Whitney, ``Tangents to analytic varieties,'' {\em Ann. Math., 81, (1965), 496-543.}

\bibitem{art10-key18} O. Zariski, ``On the topology of algebroid singularities,''  {\em Amer. J. Math., 54 (1932), 433-465.}

\bibitem{art10-key19} O. Zariski, Studies in equisingularity II: ``Equisingularity in codimension 1 (and characteristic zero),'' {\em Amer. J. Math., 87 (1965).}

\bibitem{art10-key20} O. Zariski, ``Contributions to the problem of equisingularity, in questions on algebraic varieties,'' C.I.M.E., 1969, Ed. Cremonese, Roma, 1970, 265-343.

\bibitem{art10-key21} O. Zariski, ``General theory of saturation and of saturated local rings, II,'' {\em Amer. J. Math., 93 (1971).}

\end{thebibliography}
