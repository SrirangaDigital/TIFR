\title{Supplement to the Article ``Remarks on the Kodaira Vanishing Theorem''}\label{art12}
\markright{Supplement to the Article ``Remarks on the Kodaira Vanishing Theorem''}

\author{By~ C.P. Ramanujam\footnote{C. P. Ramanujam suddenly passed away in Bangalore on October 27, 1974 (Ed.)\break
\copyright\ Indian Mathematical Society 1974}}
\markboth{C.P. Ramanujam}{Supplement to the Article ``Remarks on the Kodaira Vanishing Theorem''}

\date{Received August 24, 1973}
\maketitle

\setcounter{page}{159}
\setcounter{pageoriginal}{124}
{\bf Throughout\pageoriginale this Paper}, we work over an algebraically closed field of characteristic zero, which we may assume is the field of complex numbers. We adhere to the notations of the paper referred to in the title.

Our first remark was pointed out to us by E. Bombieri, and is used by him in his investigations of pluri-canonical surfaces. 

{\em If $X$ is a complete non-singular surface and $D$ an effective divisor on $X$ with $(D^2) > 0$, then}
$$
H^1(X,\mathcal{O}_X) \to H^1 (D, \mathcal{O}_D)
$$
{\em is injective.}

\begin{proof}
By the Riemann-Roch theorem, $\dim H^0 (X,\mathcal{O}_X (nD))$ increases quadratically in $n$. Hence for some $n>0$, if $F$ is the divisor of base components of $|nD|$, $|nD|-F$ is a linear system without base components not composite with a pencil. We then have, in the terminology of Lemma 6 of (2),
$$
\alpha (D) = \alpha (nD) \leqslant \alpha (nD - F) = 0
$$
by Theorem 2 and Lemma 6 of (2).

Our next result is the following
\end{proof}

\begin{theorem*}
Let $X$ be a complete non-singular surface and $\mathcal{L}$ an invertible sheaf on $X$ such that $(c_1 (\mathcal{L})^2) >0$ and $(c_1 (\mathcal{L}) \cdot C) \geqslant 0$ for any curve $C$ on $X$. Then $H^i (X, \mathcal{L}^{-1}) = 0$, $i=0$, $1$. 

Conversely,\pageoriginale if $(c_1 (\mathcal{L})^2) > 0$ and $H^i (X, \mathcal{L}^{-n}) =0$, $i = 1,2$ and $n$ large, for any curve $C$ on $X$, $(c_1 (\mathcal{L}) \cdot C) \geqslant 0$.
\end{theorem*}

\begin{proof}
We first prove the sufficiency of the condition.


Suppose first that $H^0(X, \mathcal{L}) \neq 0$, so that $\mathcal{L} \simeq \mathcal{O}_X (D)$, $D$ an effective divisor. We show that $D$ is numerically connected. Suppose on the contrary that $D = D_1 + D_2$, $D_i$ effective, $(D_1 \cdot D_2) = - \lambda \leqslant 0$. Since $(D^2_1) \pm (D_1 \cdot D_2) = (D \cdot D_1) \geqslant 0$ by assumption, $(D^2_1) \geqslant \lambda$ and similarly $(D^2_2) \geqslant \lambda$. Now, if $D_1$ and $D_2$ are linearly dependent modulo numerical equivalence, we must have $aD_1$ numerically equivalent to $bD_2$ where $a$ and $b$ are positive integers, since $D_i$ are effective. But then, since $(D^2) > 0$, $(D_1 \cdot D_2) > 0$. Thus, $D_i$ are independent modulo numerical equivalence, and by the Hodge index theorem,
$$
\det \left(\begin{matrix} 
(D^2_1) & -\lambda\\
-\lambda  & (D^2_2) 
\end{matrix}\right) = (D^2_1) (D^2_2) - \lambda^2 < 0,
$$
which is a contradiction. Thus, $D$ is numerically connected. But now, the assertion follows from our earlier remark, Lemma 3 of (2) and the exact sequence
$$
H^0(X, \mathcal{O}_X) \to H^0 (D, \mathcal{O}_D) \to H^1 (X, \mathcal{L}^{-1}) \to H^1 (X, \mathcal{O}_X) \to H^1 (D, \mathcal{O}_D). 
$$

Next, consider the case when $\mathcal{L}$ does not admit a non-zero section. Our hypotheses clearly imply that for $H$ ample, $(c_1 (\mathcal{L}) \cdot H) > 0$, and hence $H^0 (X,\Omega^2_X \otimes \mathcal{L}^{-n}) = 0$ for $n$ large. Hence, by Riemann-Roch, there is an $n > 0$ such that $\mathcal{L}^n$ admits a non-zero section $\sigma$ with $\text{div} \sigma = D$. By Lemma 1 of (2) and the theorem of resolution of singularities, we can find a complete non-singular surface $Y$ and a surjective morphism $f: Y \to X$ such that $f^* (\mathcal{L})$ admits a section $\tau$ with $\tau^n = f^* (\sigma)$ in $f^* (\mathcal{L}^n)$. For any curve $C'$ on $Y$, we have
$$
(c_1 (f^* (\mathcal{L})) \cdot C') = (f^* (c_1 (\mathcal{L})) \cdot C') = (c_1 (\mathcal{L}) \cdot f^* (C')) \geqslant 0,
$$
so that $f^* (\mathcal{L})$ on $Y$ satisfies the hypothesis and also admits a non-zero section. Hence, by the first part and Serre duality, 
$$
H^i (Y,\Omega^2_Y \otimes f^* (\mathcal{L})) = 0, \; i > 0.
$$
Now, by Lemma 4 of (2) and Stein factorisation,
$$
R^i f(\Omega^2_Y) = 0, \quad i > 0.
$$
Thus,\pageoriginale by the Leray spectral sequence,
$$
H^i (X, f_* (\Omega^2_Y) \otimes \mathcal{L}) = 0, \quad  i > 0.
$$
We have a splitting $1/m \quad Tr: f_* (\Omega^2_Y) \to \Omega^2_X (m = \deg f)$ of the natural homomorphism $\Omega^2_X \to f_* (\Omega^2_Y)$ (See footnote on p.44 of (2)), which gives 
$$
H^i (X, \Omega^2_X \otimes \mathcal{L}) = 0, \quad i >0,
$$
and the result follows by Serre duality.

We next prove the converse part of the theorem. If $H$ is ample, $(H \cdot c_1 (\mathcal{L})) \neq 0$, by the Hodge index theorem. By our hypothesis and Riemann-Roch, $H^0 (X , \Omega^2_X \otimes \mathcal{L}^n) \neq 0$ for $n$ large, so that $( H \cdot (K + nc_1 (\mathcal{L}))) >0$ for $n$ large. Hence $(H \cdot c_1 (\mathcal{L}))>0$ and $H^0 (K \otimes \mathcal{L}^{-n})=0$ for $n$ large. Hence $(H \cdot c_1 (\mathcal{L}))>0$ and $H^0 (K \otimes \mathcal{L}^{-n}) = 0$ for $n$ large. Again by Riemann-Roch, $H^0(\mathcal{L}^n) \neq 0$ for some $n>0$, and replacing $\mathcal{L}$ by $\mathcal{L}^n$, we may assume $H^0(\mathcal{L}) \neq 0$, $\mathcal{L} \simeq \mathcal{O}_X (D)$ for some effective divisor $D$.

From the cohomology exact sequence of the short exact sequence
$$
0 \to \mathcal{O}_X(-(n+1)D) \xrightarrow{D} \mathcal{O}_X (-nD) \to \mathcal{O}_D \otimes \mathcal{O}_X (-nD) \to 0
$$
we deduce that $H^0 (\mathcal{O}_D \otimes \mathcal{O}_X (-nD)) = 0$ for $n$ large. We can clearly find a morphism $f : \tilde{D} \to D$ such that (i) $f$ is finite, (ii) every connected component of $\tilde{D}$ is irreducible, and (iii) there is a finite set $S$ of points on $D$ such that $f |f^{-1} (D-S)$ in an isomorphism onto $D-S$. We then have an exact sequence
$$
0 \to \mathcal{O}_D \to f_* (\mathcal{O}_{\tilde{D}}) \to \mathcal{F} \to 0
$$
where $\mathcal{F}$ is a sheaf supported at finitely many points. We deduce from this that
$$
\dim H^0 (\tilde{D}, f^* (\mathcal{O}_X (-n D))) = \dim H^0 (D, f_* (\mathcal{O}_{\tilde{D}}) \otimes \mathcal{O}_X (-n D))
$$
is bounded. Hence for any component $C$ of $D$, we must have $(D \cdot C) \geqslant 0$. But for a curve $C'$ which is not a component of $D$, the inequality $(D \cdot C') \geqslant 0$ is obviously satisfied. \hfill{Q.E.D.}
\end{proof}   
\begin{remarks*}
\begin{itemize}
\item[(1)] Using the Lemma of Enriques-Severi-Zariski-Serre as in the proof of Theorem 2 of (2), it is easy to deduce from the above theorem the following generalisation to higher dimensional varieties:

{\em If $X$ non-singular projective of dimension $n$ and $\mathcal{L}$ an invertible sheaf on $X$ with\pageoriginale $(c_1 (\mathcal{L})^n)>0$ and $(c_1(\mathcal{L}) \cdot C ) \geqslant 0$ for any curve $C$ on $X$, then $H^1 (X, \mathcal{L}^{-1}) =0$.}


\item[(2)] The assumptions that $D$ effective, $(D^2) > 0$ and $(D \cdot C) \geqslant 0$ do not imply that $|nD|$ has no base components for some $n>0$, as is shown by the following example:

Let $C$ be an elliptic curve, and $X$ the projective line bundle on $C$ obtained by compactifying a line bundle $\mathcal{L}$ of degree-1 on $C$ by adding points at infinity to the fibers. Let $D_1$ be the zero section and $F$ the fiber over a point $P$ of $C$. Then,
$$
((D_1 + F)^2) = (D^2_1) +2 (D_1 \cdot F) = -1 +2 =1.
$$
If $C$ is a curve not a component of $D_1 + F$, we have evidently $(C \cdot (D_1 + F))>0$. On the other hand, $((D_1 + F) \cdot F) =1$ and $((D_1 + F) \cdot D_1) = (D^2_1) + (D_1 \cdot F) = -1 + 1 = 0$. If some linear system $|n(D_1 + F)|$ does not have $D_1$ for a base curve, we deduce that $\mathcal{O}_{D_1}\otimes \mathcal{O}_X (n(D_1 + F))$ is trivial on $D_1$, i.e., $\mathcal{L}^n \otimes \mathcal{O}_C (nP)$ is trivial on $C$. But we can choose $P$ such that $\mathcal{L} \otimes \mathcal{O}_C (P)$ is not of finite order. Then $D_1$ is a base component of each of the linear systems $|n(D_1 + F)|$.
\end{itemize}
\end{remarks*}

Our theorem is therefore strictly stronger than Theorem 2 of (2).

\begin{thebibliography}{99}
\bibitem{art12-key1} A. Grothendieck. Sur une note de Mattuck-Tate. {\em Jour. reine angew. Math.} 200 (1958).

\bibitem{art12-key2} C. P. Ramanujam. Remarks on the Kodaira vanishing theorem. {\em Jour. Ind. Math. Soc.} 36 (1972). 41-51
\end{thebibliography}

\vfill\eject
~\phantom{a}
\thispagestyle{empty}
