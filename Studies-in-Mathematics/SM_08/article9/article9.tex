\title{Remarks on the Kodaira Vanishing Theorem\footnote{This work was
done while the author was at the Mathematics Institute, University of
Warwick, with financial assistance from Harvard
University.}}\label{art9} 
\markright{Remarks on the Kodaira Vanishing Theorem}

\author{By~ C.P. Ramanujam}
\markboth{C.P. Ramanujam}{Remarks on the Kodaira Vanishing Theorem}

\date{[Received February 22, 1971]}
\maketitle

\setcounter{page}{123}

\section{Introduction}\label{art9-sec1}
\setcounter{pageoriginal}{94}
The\pageoriginale object of this note is three-fold. In the first
part, we give a deduction of the Kodaira-Nakano-Akizuki vanishing
theorem \cite{art9-key1} from the Lefschetz hyperplane-section
theorem \cite{art9-key6}, using a lemma of Mumford. In the second
part, we prove a vanishing theorem for the first cohomology for
varieties in all characteristics. The idea of this section is due to
Franchetta \cite{art9-key2}, but Franchetta failed to prove the basic
Lemma \ref{art9-lem3}. In the third part, we return to the case of the complex
base-field, and give what seems to be the most general form known of
the Kodaira vanishing theorem for the cohomologies of a line bundle in
this case. We were told of the existence of such a generalisation by
D. Mumford. 

\section{The Kodaira-Nakano-Akizuki vanishing theorem}\label{art9-sec2}
The statement runs as follows:

\begin{theorem}\label{art9-thm1}%% 1
Let $X$ be a complex non-singular algebraic variety, and $L$ a line
bundle on $X$ such that $L^{-1}$ is ample (in the sense of
Grothendieck, that is, for some $n>0$, $L^{-n}$ is induced from the
hyperplane bundle $\mathscr{O}_{\bfP^N}(1)$ by an imbedding
$X \hookrightarrow \bfP^N$). Then, for $p+q<\dim X$, 
$$ 
H^q (X,\Omega^p \otimes L) = (0).
$$
\end{theorem}

For the proof, we need the following simple lemma, whose construction is due to Mumford \cite{art9-key7}.

\begin{lem}\label{art9-lem1}%%%1
Let $X$\pageoriginale be a variety over the complex field, $L$ a line bundle on $X$, $\sigma \in H^0(X, L^m) -(0)$ where $m$ is positive integer and $D$ the divisor of $\sigma$. Then there is an $m$-fold cyclic covering $f: X' \to X$ ramified precisely over the support of $D$ such that $f(L)$ admits a section $\tau$ satisfying
$$
\tau^m = f^* (\sigma) \text{~~ in ~~} H^0 (X', f^* (L^m)).
$$
If further $X$ and $D$ are non-singular, so are $X'$ and $\divv \ldotp\tau$, and in the neighbourhood of a point of the support of $D$, the covering $f$ is isomorphic to the covering $\bfC^n \to \bfC$ given by
$$
(z_1, z_2, \ldots, z_n) \longmapsto (z_1, z_2, \ldots, z^m_n)
$$
restricted to neighbourhoods of the origin.
\end{lem}

\begin{proof}
Define $X'$ to be the subvariety of the total space of $L$ given by
$$
X' = \{x \in L | x^m = \sigma (\pi (x)) \text{~~ in ~~} L^m \}
$$
where $\pi: L \to X$ is the projection, $f : X' \to X$ the restriction of $\pi$, and on $f^* (L) = L \times X^{X'}$, define a section $\tau: X' \to f^* (L)$ by $\tau (x) = (x,x)$. All the conditions are trivially verified.
\end{proof}

%%%%
\noindent
{\bf Proof of Theorem 1.} We proceed by induction on $n = \dim X$. The statement being trivial for $n=1$, we assume $n>1$ and that the result holds for varieties of smaller dimension.

By assumption and the theorem of Bertini, there is an integer $m>0$
and $\sigma \in H^0 (X, L^{-m})$ such that $D = \divv \ldotp \sigma$
is a non-singular hyperplane section of $X$ for some projective
embedding. By Lemma \ref{art9-lem1}, $\exists f:X' \to X$ a cyclic $m$-fold covering
with $X'$ non-singular and $\tau \in H^0 (x,f^* (L^{-1}))$ such that
$\tau^m = f^* (\sigma)$ and $D' = \divv \ldotp \tau$ is
non-singular. Also, $D'$ is the support of a hyperplane section of
$X'$ for some embedding, since $f^* (L^{-1})$ is ample on $X'$
(E. G. A., Chap. II). By the Lefschetz hyperplane section
theorem \cite{art9-key6},
$$
H^i (X', \bfC) \to H^i (\bfC', \bfC)
$$
is an isomorphism for $i < n -1$ and is injective for $i=n-1$. By the Hodge decomposition theorem and its naturality (\cite{art9-key8}, p. 129), it follows that 
\begin{equation*}
H^p (X', \Omega^q_{X'}) \to H^p (D', \Omega^q_{D'}) \tag{1}
\end{equation*}\pageoriginale
is an isomorphism for $p+q < n -1$ and is injective for $p+q=n-1$. Let $\mathscr{I}$ be the sheaf of ideals defining $D'$, so that $\mathscr{I}$ is isomorphic to the sheaf of germs of sections of $f^*(L)$. We have the exact sequence (E. G. A., Chap. IV, Pt. 1)
$$
0 \to \mathscr{I} \otimes \mathscr{O}_{D'} \to \Omega^1_{X'} \otimes_{OX'} \mathscr{O}_{D'} \to \Omega^1_{D'} \to 0,
$$ 
and hence for any $q>0$, the exact sequence
$$
0 \to \Omega^{q-1}_{D'} \otimes \mathscr{I} \otimes \mathscr{O}_{D'} \to \Omega^q_{X'} \otimes \mathscr{O}_{D'}  \to \Omega^q_{D'} \to 0
$$
and $\Omega^{q-1}_{D'} \otimes \mathscr{I} \otimes \mathscr{O}_{D'} \approx \Omega^{q-1}_{D'} \otimes f^* (\mathscr{L})$ where as usual we denote by a script letter the sheaf of germs of sections of the line bundle denoted by the same letter in bold type. By induction hypothesis, $H^p (D',\Omega^{q-1}_{D'} \otimes f^* (\mathscr{L})) = (0)$ for $p+q<n$, so that $H^p (X, \Omega^q_{X'} \otimes \mathscr{O}_{D'}) \to H^p (\Omega^q_{D'})$ is an isomorphism for $p+q < n -1$ and is injective for $p+q = n-1$. This together with (1) gives us that
$$
H^p (X', \Omega^q_{X'}) \to H^p (D', \Omega^q_{X'} \otimes \mathscr{O}_{D'})
$$
is an isomorphism for $p+q<n-1$ and is injective for $p+q= n-1$. It follows that 
\begin{equation*}
H^p (X', f^* (L)\otimes \Omega^q_{X'}) = H^p (X', \mathscr{I} \Omega^q_{X'}) = (0)
\tag{2}
\end{equation*}
for $p+q<n$. Having proved the theorem for $X'$, it would follow for $X$ and the line bundle $L$ on $X$, provided the inclusion of sheaves $\mathscr{L} \otimes \Omega^q_X \hookrightarrow f_* (f^* (\mathscr{L}) \otimes \Omega^q_{X'}) \approx \mathscr{L} \otimes  f_* (\Omega^q_{X'})$ admits a splitting, or equivalently, if the natural inclusion $i: \Omega^q_X \hookrightarrow f_* (\Omega^q_{X'})$ splits. (This is because the morphism $f$ is finite, so that for any $\mathscr{F}$ coherent on $X'$, $H^p (X', \mathscr{F}) \approx H^p (X, f_* (\mathscr{F}))$.)

To give a splitting of $i$, let $U$ be any open set of $X$ and $\omega$ a regular differential form on $f^{-1} (U)$. Define a form $\tilde{\omega}$ on $f^{-1} (U)$ by $\tilde{\omega} = \sum\limits_{\phi \in G} \phi^* (\omega)$ where $G$ denotes the Galois groups of $X'$ over $X$. Then $\tilde{\omega}$ is regular and $G$-invariant on $f^{-1} (U)$. We assert that there is a unique form $Tr\omega$ regular on $U$ such that $f^* (Tr\omega) = \tilde{\omega}$. This is in fact clear if $U$ does not contain any branch points of $f$. At points of the branch-locus, this follows from a simple calculation using the local\pageoriginale description of $f$ given in Lemma \ref{art9-lem1}\footnote[2]{In fact, for any separable morphism of non-singular varieties $f: X' \to X$ one can define a homomorphism
$$
Tr : f_* (\Omega^q_{x'}) \to \Omega^q_x.
$$}. Now, $(1/m)$ $Tr$ gives the required splitting.

\section{The method of Franchetta}\label{art9-sec3}
Let $X$ be a non-singular variety of dimension $n\ge2$ over an algebraically closed field {of any characteristic} and $D$ an {\em effective} divisor on $X$. We define (following Franchetta) $D$ to be {\em numerically connected} if there is an ample $H$ on $X$ such that for any decomposition $D = D_1 + D_2$, $D_i>0$, we have $(H^{n-2} \ldotp D_1 \ldotp D_2)>0$. It is clear that the effective divisor without multiple components is numerically connected if and only if its support is connected, and that for any effective $D$, if $nD$ is numerically connected, so is $D$. Further, if $H'$ is a generic (hence non-singular) hyperplane section of $X$ for an embedding given by any multiple of $H$, $D$ is numerically connected if and only if $D\ldotp H'$ is numerically connected on $H'$.

\begin{lem}\label{art9-lem2}
If for some $N>0$, $ND$ moves in an algebraic system of effective divisors without fixed components, and $(H^{n-2} \ldotp D^2)>0$ for some ample $H$, then $D$ is numerically connected.
\end{lem}

\begin{proof}
By a remark above, we may replace $D$ by $ND$ and assume $N=1$. By a second remark, we may successively take general hyperplane sections of $X$ for an embedding given by a multiple of $H$, till we arrive at a surface. The other hypotheses are stable for general hyperplane sections.

Thus, assume $X$ a non-singular surface and $D = D_1 + D_2$, $D_i > 0$. Since $D$ moves in an algebraic system without fixed components we have that $(D \ldotp D_i) = (D' \ldotp D_i ) \geqslant 0$ where $D'$ is algebraically equivalent to $D$ and effective, and has no common components with $D$. Since $(D^2)>0$, at least one of these is a strict inequality. Hence - $(D_1 \ldotp D_2) \leqslant (D^2_i) (i=1,2)$ with atleast one strict inequality. Suppose now that $(D_1 \ldotp D_2) \leqslant 0$, so that $0 \leqslant - (D_1 \ldotp D_2) \leqslant (D^2_i) (i=1,2)$.

Apply the Hodge index theorem for surfaces (valid in any
characteristic, \cite{art9-key4}) to the subgroup of the Neron-Severi
group generated\pageoriginale by $D_1$, $D_2$. The intersection form
must have one positive and one negative eigenvalue on this subspace if
the $D_i$ are numerically independent. But if they are numerically
dependent, we must have $aD_1 = bD_2$ with $a$, $b$ positive integers,
and since $(D^2) >0$, $(D_1 \ldotp D_2) >0$. Hence they are
independent, and we get 
\begin{equation*}
(D^2_1) \ldotp (D^2_2) - (D_1 \ldotp D_2)^2 = \det 
\left(
\begin{aligned}
& (D^2_1)  \qquad (D_1 \ldotp D_2)\\
& (D_1 \ldotp D_2) \qquad   (D^2_2)
\end{aligned}
\right) < 0,
\end{equation*}
contradicting our earlier inequalities $0\leqslant - (D_1 \ldotp D_2) \leqslant (D^2_1)$. Hence $(D_1 \ldotp D_2) >0$.
\end{proof}

\begin{lem}\label{art9-lem3}%%% 3
If $D$ is a numerically connected effective divisor on a non-singular variety $X$, $H^0 (D, \mathscr{O}_D)$ consists of constants.
\end{lem}

\begin{proof}
We proceed by induction on $n = \dim X$. Suppose $n >2$ and the result holds in smaller dimensions. Choose an embedding of $X$ in $\bfP^N$ such that $H^0(D, \mathscr{O}_D (-1))=(0)$. (This is possible since $D$ does not have points as embedded components). Let $H$ be a general hyperplane section. Then we have the exact sequence $0 \to H^0 (D, \mathscr{O}_D) \to H^0 (D \cap H, \mathscr{O}_{D\cap H})$, and the last group consists of constants, by induction hypothesis applied to the pair $(H, D \cap H)$. Hence so does $H^0 (D, \mathscr{O}_D)$, and we are through.

Thus, one has only to prove the lemma when $\dim X =2$. Let us suppose there is a section $\sigma$ of $H^0 (D, \mathscr{O}_D)$ which is not a scalar. Since $D_{\red}$ is connected, $H^0 (D, \mathscr{O}_{D_{\red}})$ consists of constants, and the image of $\sigma$ in $H^0(\mathscr{O}_{D_{\red}})$ is a scalar $\lambda$. Replacing $\sigma$ by $\sigma -\lambda$ we may assume that $\sigma$ is a section of $\mathscr{R}$, the root ideal of $\mathscr{O}_D$. One can evidently find an effective divisor $D_1$ which is maximum, satisfying the conditions $0 < D_1 < D$, $\sigma \in \mathscr{I}_{D_1} \mathscr{O}_D$. Let $D = D_1+ D_2$. We have two exact sequences of coherent sheaves on $X$,
\begin{align*}
& 0 \longrightarrow \mathscr{O}_{D_2} \xrightarrow{~\sigma~} \mathscr{O}_D \longrightarrow \mathscr{O}_{D} /\sigma \mathscr{O}_D \longrightarrow 0,\\
& 0 \longrightarrow \mathscr{F} \longrightarrow \mathscr{O}_D / \sigma \mathscr{O}_D \longrightarrow \mathscr{O}_{D_1} \longrightarrow 0,
\end{align*}
where $\mathscr{F}$ is a sheaf supported by finitely many
points. Calculate Chern classes (in the ring of rational or algebraic
equivalence classes of cycles, {\em see} \cite{art9-key5}). We have  
\begin{align*}
c (\mathscr{O}_D) & = c (\mathscr{O}_{D_2}) \ldotp c(\mathscr{O}_D / \sigma \mathscr{O}_D) = c (\mathscr{O}_{D_2}) \ldotp c (\mathscr{O}_{D_1}) \ldotp c (\mathscr{F}),\\
& = \frac{1}{1-D} = \frac{1}{1-D_1} \ldotp \frac{1}{1-D_2}  \ldotp c(\mathscr{F}).
\end{align*}\pageoriginale
If $\mathscr{F}$ has support at the points $P_i$ and $l(\mathscr{F}_{P_i}) = n_i$, we have  $c(\mathscr{F}) = 1 - \sum n_i P_i$. Thus,
\begin{align*}
(1-D_1) \ldotp (1 - D_2) & = (1-D) \ldotp (1 - \sum n_i P_i),\\
(D_1 \ldotp D_2) & = - \sum n_i \leqslant 0.
\end{align*}
\end{proof}

\begin{lem}\label{art9-lem4}\footnote[3]{Analogous results have been obtained over the complex field by Grauert, also in the case of higher dimensions.}
Let $f: X' \to X$ be a proper birational morphism of a non-singular
surface $X'$ onto a normal surface $X$, and $L$ a line bundle on $X'$
such that for any irreducible curve $C$ on $X'$ contracted to a point
by $f$, $(C\ldotp c_1 (L)) \geqslant 0$. Then, $(R^1 f) (\Omega^2
_{X'} \otimes L) =0$. 
\end{lem}

\begin{proof}
The sheaf $R^1 f (\Omega^2_{X'} \otimes L)$ is concentrated at the finitely many points $P$ where $\dim f^{-1} (P) = 1$. Let $P$ be one of these points. If we can show that for any effective divisor $D$ with support $f^{-1} (P)$, we have $H^1 (D, L \otimes \Omega^{2}_{X'}/ \mathscr{I}_D \Omega^2_{X'}) =0$, it would follow from the `Th\'eor\`eme fondamentale des morphismes propres' (E. G. A., Chap.III) that $R^1 f (\Omega^2_{X'} \otimes L)_P = (0)$ and we would be through. Now, the sheaf $\omega_D = \mathscr{I}^{-1}_D \Omega^2_{X'}/ \Omega^2_{X'}$ is dualising on $D$, so that we have to show that $H^0 (D, L^{-1} \otimes \mathscr{I}^{-1}_D / \mathscr{O}_{X'}) =0$. If not, let $\sigma$ be an non-zero element of $H^0 (D,L^{-1} \otimes \mathscr{I}^{-1}_D / \mathscr{O}_{X'})$. We can find a maximum divisor $D_1$ with $0 \leqslant D_1 \leqslant D$ such that $\sigma \in H^0 (D, L^{-1} \otimes \mathscr{I}_{D_1} \mathscr{I}^{-1}_D/ \mathscr{O}_{X'})$, and since $\sigma \neq 0$, $D_1 \neq D$. It then follows that $\sigma$ generates $L^{-1} \otimes \mathscr{I}_{D_1} \mathscr{I}^{-1}_D / \mathscr{O}_{X'}$ generically on the components of the support of this sheaf. If we set $D_2 = D - D_1$, this sheaf is nothing but $L^{-1} \otimes \mathscr{I}^{-1}_{D_2}/ \mathscr{O}_{X'}$. Further, the annihilator of $\sigma$ is easily seen to be $\mathscr{I}_{D_2}$, so that we get an exact sequence
$$
0 \longrightarrow \mathscr{O}_{D_2} \xrightarrow{~\sigma~} L^{-1} \otimes \mathscr{I}^{-1}_{D_2}/ \mathscr{O}_{X'} \longrightarrow \mathscr{F} \longrightarrow 0
$$
where $\mathscr{F}$ has zero dimensional support. Calculating Chern classes, we get
$$
c (L^{-1} \otimes \mathscr{I}^{-1}_{D_2} /\mathscr{O}_{X'}) = c (\mathscr{O}_{D_2}) \ldotp c(\mathscr{F}),
$$\pageoriginale
i.e.,
$$
(1-D_2) \ldotp (1 - c_1 (L) + D_2) = (1 - c_1 (L))\ldotp (1 - \sum n_i P_i),
$$
where $P_i$ are the points supporting $\mathscr{F}$ and $n_i = l_{P_i} (\mathscr{F}_{P_i})$. This gives
$$
(D^2_2) \geqslant (D^2_2) - (c_1 (L) \ldotp D_2) = \sum n_i \geqslant 0,
$$
which is impossible, since by a well-known result \cite{art9-key9}, for any nonzero divisor $E$ with support in $f^{-1}(P)$, $(E^2)<0$. This is a contradiction.
\end{proof}

\begin{lem}\label{art9-lem5}
Let $X$ be a non-singular complete surface and $L$ a line bundle on $X$ such that for some $n>0$, the complete linear system determined by $H^0 (X, L^n)$ has no fixed components and is not composite with a pencil. Then $H^1 (X, L^{-N}) =0$  for all large $N$.
\end{lem}

\begin{proof}
By (\cite{art9-key10}, Theorem 6.2), we may assume that the linear system
determined by $H^0 (X,L^n)$ has no base points at all. Hence, there is
a morphism $\phi: X \to \bfP^m$ for some $m>0$ such that $Y = \phi
(X)$ is a surface and $L^n \simeq \phi^* (\mathscr{O}_{\bfP
m}(1))$. Further, by (E.G.A., Chap. II), we may assume $Y$ normal and
$\phi$ birational. Determine $N_0$ so that for $N \geqslant N_0$ and
$0 \leqslant v < n$, $H^1 (R^0 \phi (\Omega^2_{X'} \otimes
L^v) \otimes \mathscr{O}_Y (N)) \approx H^1 (R^0 \phi
(\Omega^2_{X'} \otimes L_{N_{n+v}})) = (0)$. For any curve $C$
contracted to a point by $\phi$, the restriction of $L^n$ to $C$ is
trivial, hence $(c_1 (L) \ldotp C)=0$. Thus by Lemma \ref{art9-lem4}, $R^1 \phi
(\Omega^2_{X'} \otimes L^v) =0$ for $v \geqslant 0$. It follows from
the Leray spectral sequence that $H^1 (X', \Omega^2_{X'} \otimes
L_{N_{n+v}} ) =(0)$ for $0 \leqslant v < n$, $N \geqslant N_0$; and
the lemma follows by Serre duality.

\end{proof}   
   


\begin{lem}\label{art9-lem6}%%% 6
Let $X$ be a complete normal variety. Assume further in the case of positive characteristic that the Frobenius homomorphism on $H^1(X, \mathscr{O}_X)$ is injective. For any effective divisor $D$ on $X$, define $\alpha (D)$ to be the dimension of the kernel of the homomorphism $H^1 (X, \mathscr{O}_X) \to H^1 (D, \mathscr{O}_D)$. Then, $\alpha (D) = \alpha (D_{\red})$.
\end{lem}

\begin{proof}
It is clearly sufficient to show that if $D_1 \leqslant D_2 \leqslant 2 D_1$, $\alpha (D_1) = \alpha (D_2)$.

First suppose that the characteristic is zero. Since the ideal $\mathscr{I}_{D_1} \mathscr{O}_{D_2}$ is of square zero, we have the exact sequence $0 \to \mathscr{I}_{D_1} \mathscr{O}_{D_2} \to \mathscr{O}^*_{D_2 } \to \mathscr{O}^*_{D_1} \to 1$ from which the exact sequence
$$
H^0 (\mathscr{O}^*_{D_1}) \to H^1 (\mathscr{I}_{D_1} \mathscr{O}_{D_2}) \to \Pic D_2 \to \Pic D_1.
$$\pageoriginale
Since $H^0 (\mathscr{O}^*_{D_1})$ is divisible (use binomial series), it follows that the kernel of $\Pic ~D_2 \to \Pic D_1$ is torsion-free. If we set $K(D_i) = \Ker (\Pic^\circ X \to \Pic D_i)$, $K(D_i)$, are closed subgroups of the abelian variety $\Pic^\circ X$, and torsion elements are dense in $K(D_1)/ K(D_2)$. Thus, $K(D_1) = K(D_2)$ and the assertion follows on passing to Lie algebras.

Suppose now that the characteristic is $p>0$.

Since $\mathscr{I}_{D_1} \mathscr{O}_{D_2}$ is of square zero, it follows from the exactness of $H^1 (\mathscr{I}_{D_1} \mathscr{O}_{D_2}) \to H^1 (\mathscr{O}_{D_2}) \xrightarrow{~\lambda~}  H^1 (\mathscr{O}_{D_1})$ that the Frobenius kills the kernel of $\lambda$. Since the Frobenius is semi-simple on $H^1 (\mathscr{O}_X)$ by assumption, it is so on any quotient, and the image of $H^1(\mathscr{O}_X) \to H^1 (\mathscr{O}_{D_2})$ cannot meet $\Ker \lambda$.
\end{proof}

\begin{remark*}
An `evaluation' of $\alpha(D)$ has been given by Kodaira when $X$ is a surface in characteristic zero and $D$ is reduced. We give a rapid derivation of this, valid for any non-singular $X$ in characteristic zero and any effective $D$. We have clearly that $\alpha (D) = \alpha (D_{\red})$ is the dimension of the kernel of $\Pic^\circ X \to \Pic  D_{\red}$. But if $\eta: X \to A$ is the Albanese map, and $B$ the abelian subvariety of $A$ generated by the 
differences $\eta(x) - \eta(y)$ where $x$ and $y$ belong to the same connected component of $D_{\red}$, since $\Pic^\circ X = \Pic^\circ A$, one sees easily (using the Jacobians of normalisations of generic curves on the various components of $D_{\red}$) that the dimensions of the kernels of $\Pic^\circ X \to \Pic D_{\red}$  and $\Pic^\circ A \to \Pic^\circ B$ are the same. But the dimension of this last group equals the dimension of the space of 1-forms on $A$ which induce the 1-form 0 on $B$; and since $D_{\red}$ generates $B$ (in the above sense), it also equals the dimension of the space of 1-forms on $X$ which induce the 1-form 0 on $D_{\red}$. Thus we have 
$$
\alpha (D) = \dim \ldotp \ker \ldotp (H^0(X, \Omega^1_X) \to H^0 (D_{\red}, \Omega^{1} D_{\red})).
$$
\end{remark*}

\begin{theorem}\label{art9-thm2}%%% 2
Let $X$ be a non-singular projective variety of dimension $\geqslant 2$ and $D$ an effective divisor on $X$ such that for some $n>0$, $|nD|$ has no fixed components and is not composite with a pencil. Assume\pageoriginale further that either the characteristic is zero or that it is positive and that the Frobenius is injective on $H^1 (\mathscr{O}_X)$. Then $H^1 (\mathscr{O}_X (-D)) = (0)$.
\end{theorem}

\begin{proof}
First suppose $\dim X =2$. The exact sequence
$$
0 \to \mathscr{O}_X (-D) \to \mathscr{O}_X \to \mathscr{O}_D \to 0
$$
leads to the cohomology sequence
{\fontsize{10pt}{11pt}\selectfont
$$
H^0 (X, \mathscr{O}_X) \to H^0 (D,  \mathscr{O}_D) \to H^1
(X, \mathscr{O}_X (-D)) \to H^1 (X, \mathscr{O}_X) \to H^1
(D, \mathscr{O}_D) 
$$}
and the result follows from
 Lemmas \ref{art9-lem2}, \ref{art9-lem3}, \ref{art9-lem5}
 and \ref{art9-lem6}.

Next suppose $n = \dim X > 2$ and that the result holds in smaller
dimensions. Replacing the given projective embedding by the one given
by hypersurfaces of sufficiently large degree, we may assume that $H^1
(X,\\ \mathscr{O}_X (-D)(-1)) = (0)$. Hence if $H$ is a general
hyperplane section for this embedding, the homomorphism $H^1
(X, \mathscr{O}_X (-D)) \to H^1 (H, \mathscr{O}_H (- D \ldotp H))$ is
injective. On the other hand, the assumptions fulfilled by $D$ on $X$
are also fulfilled by $D \ldotp H$ on $H$. This  completes the
induction. 
\end{proof}

\section{A generalisation of the Kodaira vanishing theorem for 
algebraic varieties}\label{art9-sec4}
We assume the following theorem:

\begin{theorem*}[\cite{art9-key11}]
Let $X$ be a projective variety over the complex field, $L$ a line
bundle on $X$ such that there is an integer $n > 0$ and a birational
morphism $\phi: X \to Y \subset \bfP^N$ such that $\phi^*
(\mathscr{O}_Y (1)) \approx L$. Then $H^i (X, L^{-1}) = 0$ for
$0 \leqslant i < \dim X$. 
\end{theorem*}

We deduce from this the following 

\begin{theorem}\label{art9-thm3}
Let $X$ be a projective non-singular variety of dimension $n$ over the
complex field, $L$ a line bundle on $X$, $m$ an integer with
$1\leqslant m \leqslant n$ such that for some $N>0$, the complete
linear system determined by $H^0(X, L^N)$ has the dimension of its
base point set $\leqslant n - m$ and has a projective image of
dimension $\geqslant m$. Then  $H^i (X, L^{-1}) = (0)$ for
$0 \leqslant i < m$.
\end{theorem}

\begin{proof}
We fix $m$, and use induction on $n$ starting from $n =m$. Assume
first that $n>m$ and that the theorem is true for varieties of
dimension $n-1$. Using a suitable projective embedding, we may
assume\pageoriginale that $H^i (X, L^{-1} (-1)) = (0)$ for
$0 \leqslant i < n$. If $H$ is a general hyperplane section for this
embedding, it follows that $H^i (X, L^{-1}) \to H^2 (H, (L_H)^{-1})$ is
injective for $0 \leqslant i < n$, where $L_H$ denotes the restriction
of $L$ to $H$. Further, the hypotheses made on the pair $(X,L)$ are
evidently also satisfied by the pair $(H, L_H)$ for $H$ general. We
are thus reduced to proving the theorem when $n =m$.

Thus, assume $n=m$. Since the base points of the complete linear
system determined by $H^0(X,L^N)$ are isolated, by (\cite{art9-key10}, Theorem
6.2), replacing $N$ by a larger multiple, we may assume there are no
base points. Thus there is a morphism $\phi: X \to Y \subset \bfP^k$
with $\dim X = \dim Y = n$ such that $\phi^* (\mathscr{O}_Y
(1)) \approx L^N$. By (E.G.A. Chap II), we may even assume $\phi$
birational. But now, the result follows from the theorem quoted above.
\end{proof}

\begin{remarks*}
\begin{enumerate}
\item[(1)] It is not clear if in Theorem \ref{art9-thm3}, the hypothesis on the set
of base points of the complete linear system determined by $H^0 (X,
L^N)$ can be weakened to the assumption that the base point set be of
dimension $\leqslant n-2$.

For $m=2$, Theorem \ref{art9-thm3} specialises to Theorem \ref{art9-thm2} over $\bfC$, which
however is proved under the further assumption that $H^0(X,L) \neq
(0)$.

\item[(2)] A generalisation of the theorem of Nakano on the vanishing
of $H^p(X, \Omega^q \otimes L^{-1})$ along the lines of the theorem
quoted at the beginning of this section does not exist. In fact, let
$\sigma: X \to \bfP^3$ be the morphism obtained by blowing up $\bfP^3$
at (1, 0, 0, 0), and $L = \sigma^* (\mathscr{O}_{\bfP^3}(1))$. Then
one shows easily that $H^1 (X, \Omega^1_X \otimes L^{-1}) \neq (0)$ .
\end{enumerate}
\end{remarks*}

\begin{thebibliography}{99}
\bibitem{art9-key1} Y. Akizuki and S. Nakano: Note on
Kodaira-Spencer's proof of Lefschetz theorems. {\em Proc. Jap. Acad.,}
30 (1954).

\bibitem{art9-key2} F. Enriques: {\em Superficie algebriche, Bologna.}

\bibitem{art9-key3} A. Grothendieck: Elements de g\'eom\'etrie
alg\'ebrique. Chap. I-IV. {\em Publ. Math. Inst. des hautes \'etudes
Sc.} (1958).

\bibitem{art9-key4} A. Grothendieck:\pageoriginale Sur une note de Mattuck-Tate,
{\em Jour. reine angew. Math.} 200 (1958).

\bibitem{art9-key5} A. Grothendieck: La theorie des classes de Chern,
{\em Bull. Soc. Math. France,} 86 (1958).

\bibitem{art9-key6} J. Milnor: Morse theory. {\em Annals of
Maths. Studies. } No. 51, Princeton.

\bibitem{art9-key7} D. Mumford: Pathologies III, {\em
Amer. Jour. Math.,} 89 (1967).

\bibitem{art9-key8} L. Schwartz: Lectures on complex analytic
manifolds. {\em Tata Inst. Lecture Notes } (1955).

\bibitem{art9-key9} I. R. Shafarevich: Lectures on minimal models and
birational transformations, {\em Tata Inst. Lecture Notes} (1966).

\bibitem{art9-key10} O. Zariski: The theorem of Riemann-Roch for high
multiples of an effective divisor on an algebraic surface. {\em
Ann. Maths.,} 76 (1962).   

\bibitem{art9-key11} H. Grauert and O. Riemenschneider:
Verschwindungss\"atze f\"ur analytische Cohomologiegruppen auf
komplexen Ra\"umen. Several complex variables 1, Maryland (1970), {\em
Springer Lecture Notes.}

\end{thebibliography}
