\title{Some Footnotes to the Work of C.~P.~Ramanujam}\label{art01}
\markright{Some Footnotes to the Work of C.P.~Ramanujam}

\author{By~ D.~Mumford}
\markboth{D. Mumford}{Some Footnotes to the Work of C.~P.~Ramanujam}

\date{}
\maketitle

\setcounter{page}{289}

\setcounter{pageoriginal}{246}
THIS PAPER\pageoriginale consists of a series of remarks, each of which is
connected in some way with the work of Ramanujam. Quite often,  in the
last few years, I have been thinking on some topic, and suddenly I
realize--Yes, Ramanujam thought about this too--or--This really links
up with his point of view. It is uncanny to see how his ideas continue
to work after his death. It is with the thought of embellishing some
of his favourite topics that I write down these rather disconnected
series of results.

\bigskip

\begin{center}
{\large\bf I}
\end{center}
\smallskip

The first remark is a very simple example relevant to the purity
conjecture (sometimes called Lang's conjecture) discussed in
Ramanujam's paper \cite{art01-key10}. The conjecture was--let
$$
f:X^{n}\to Y^{m}
$$
be a proper map of an $n$-dimensional smooth variety onto an
$m$-dimen\-sional smooth variety with all fibres of dimension
$n-m$. Assume the characteristic is zero. Then show
$$
\{y\in Y|f^{-1}(y)\quad\text{is singular}\}
$$
has codimension one in $Y$. When $n=m$, this result is true and is
known as ``purity of the branch locus''; when $n=m+1$, it is also true
and was proven by Dolga\v{c}ev, Simha and Ramanujam. When $n=m+2$,
Ramanujam describes in \cite{art01-key10} a counter-example due to us
jointly. Here is another counter-example for certain large values of
$n-m$. 

We consider the following very special case for $f$. Start with
$\bfZ^{r}\subset \bfP^{m}$ an arbitrary subvariety. Let
$\check{\bfP}^{m}$ be the dual projective space--the space of
hyperplanes in $\bfP^{m}$. The dual variety
$\check{\bfZ}\subset \check{\bfP}^{m}$ is, by definition
the\pageoriginale Zariski-closure of the locus of hyperplanes $H$ such
that, at some smooth point $x\in \bfZ^{r}$, $T_{x,H}\supset T_{x,Z}$. It
is apparently well known, although I don't know a reference, that in
characteristic $0$,
$$
\check{\check{\bfZ}}=\bfZ.
$$
Consider the special case where $\bfZ$ is smooth and spans
$\bfP^{m}$. Then we don't need to take the Zariski-closure in the
above definition and, in fact, the definition of $\check{\bfZ}$ can be
reformulated like this:

Let
$$
I\subset \bfP^{m}\times \check{\bfP}^{m}
$$
be the universal family of hyperplanes, i.e., if
$(X_{0},\ldots,X_{m})$, resp.\break $(\xi_{0},\ldots,\xi_{m})$ are
coordinates in $\bfP^{m}$, resp.~$\check{\bfP}^{m}$, then $I$ is given
by 
$$
\sum \xi_{i}X_{i}=0.
$$

Let
$$
X=I\cap (\bfZ^{r}\times \check{\bfP}^{m}).
$$
Note that $I$ and $\bfZ^{r}\times \check{\bfP}^{m}$ are smooth
subvarieties of $\bfP^{m}\times \check{\bfP}^{m}$ of codimension 1 and
$m-r$ respectively. One sees immediately that they meet transversely,
so $X$ is smooth of dimension $m+r-1$. Consider
$$
p_{2}:X\to \check{\bfP}^{m}.
$$
Its fibres are the hyperplane sections of $\bfZ$, all of which have
dimension $r-1$. Thus $p_{2}$ is a morphism of the type considered in
the conjecture. In this case
\begin{align*}
\{\xi \in \check{\bfP}^{m}|p_{2}^{-1}(\xi)\text{~~ singular}\}
&= \{\xi\in \check{\bfP}^{m}|\text{~ if~ }\xi\text{~ corresponds to~ }
H\subset \bfP^{m},\\
 &\hspace{2cm} \text{then~ } \bfZ.\bfH\text{~ is singular}\}\\
&= \check{\bfZ}.
\end{align*}
Thus the conjecture would say that the dual $\check{\bfZ}$ of a smooth
variety $\bfZ$ spanning $\bfP^{m}$ is a hypersurface.

I claim this is false, although I feel sure it can only be false in
very rare\pageoriginale circumstances. In fact, I don't know any cases
other than the following example where it is
false.\footnote[2]{M.~Reid has indicated to me another set of
examples: Suppose
$$
\bfZ=\bfP(E)
$$
where $E$ is a vector bundle of rank $s$ on $y^{t}$, so $r=s+t-1$, and
the fibres of $\bfP(E)$ are embedded linearly. Then if $s\geq t+2$,
$\check{\bfZ}$ is not a hypersurface.} Simply take
$$
\bfZ^{r}=[\text{Grassmannian of lines in~ } \bfP^{2k}, k>1].
$$
Here $r=2(2k-1)$, $m=k(2k+1)-1$ and the embedding
$i:\bfZ^{r}\subset \bfP^{m}$ is the usual Pl\"ucker embedding. In vector
space form, let
\begin{quote}
$V$ = a complex vector space of dimension $2k+1$

$\bfZ$ = set of 2-dimensional subspaces $W_{2}\subset V$

$\bfP^{m}$ = set of 1-dimensional subspaces
$W_{1}\subset \Lambda^{2}V$

$i$ = map taking $W_{2}$ to $W_{1}=\Lambda^{2}W_{2}$.
\end{quote}
Note that we may identify
\begin{quote}
$\check{\bfP}^{m}$ = set of 1-dimensional subspaces
$W'_{1}\subset \Lambda^{2}V^{*}$, where 

$\Lambda^{2}V^{*}$ = space of skew-symmetric 2-forms $A:V\times V\to \bfC$.
\end{quote}
Write $[W_{2}]\in Z$ for the point defined by $W_{2}$, and
$H_{A}\subset \bfP^{m}$ for the hyperplane defined by a 2-form
$A$. Then it is immediate from the definitions that
$$
i([W_{2}])\in H_{A}\Leftrightarrow \res_{W_{2}}A\quad\text{is zero.}
$$
To determine when moreover,
$$
i_{*}(T_{W_{2},\bfZ})\subset T_{i(W_{2}),H_{A}},
$$
let $v_{1}$, $v_{2}\in W_{2}$ be a basis, and make a small deformation
of $W_{2}$ by taking $v_{1}+\epsilon v'_{1}$, $v_{2}+\epsilon v'_{2}$
to be a basis of $\widetilde{W}_{2}\subset
V\otimes \bfC[\epsilon]$. The $\widetilde{W}_{2}$ represents a tangent
vector $t$ to $\bfZ$ at $[W_{2}]$ and 
\begin{align*}
i_{*}(t)\subset T_{i(W_{2}),H_{A}} & \Leftrightarrow A(v_{1}+\epsilon
v'_{1},v_{2}+\epsilon v'_{2})\equiv 0\pmod{\epsilon^{2}}\\
& \Leftrightarrow A(v'_{1},v_{2})+A(v_{1},v'_{2})=0,
\end{align*}
Thus:
\begin{align*}
i_{*}(T_{W_{2},\bfZ})\subset T_{i(W_{2}),H_{A}} & \Leftrightarrow \text{~
for all~ }v'_{1},v'_{2},\in V,\\
&\quad A(v_{1},v'_{2})+A(v'_{1},v_{2})=0\\
&\Leftrightarrow W_{2}\subset \text{~(nullspace of~ $A$)}.
\end{align*}

Therefore\pageoriginale
$$
H_{A}\text{~ is tangent to~ }
i(\bfZ)\Leftrightarrow\dim~ \text{(nullspace)} \geq 2.
$$

Now the nullspace of $A$ has odd dimension, and if it is 3, one counts
the dimension of the space of such $A$ as follows:
\begin{align*}
\dim \left(\begin{array}{l} \text{space of $A$'s
with}\\ \dim(\text{nullspace})=3\end{array}\right) &= \dim 
\left(\begin{array}{c} \text{space of}\\ W_{3}\subset
V\end{array}\right)+\dim \Lambda^{2}(V/W_{3})\\[7pt]
&= 3(2k-2)+\frac{(2k-2)(2k-3)}{2}\\
&= 2k^{2}+k-3.
\end{align*}
Thus $\dim \check{\bfZ}=m-3$, and codim $\check{\bfZ}=3!$ (Compare this with
Buchsbaum-Eisenbud \cite{art01-key3}, where it is shown that
$\check{\bfZ}\subset \check{\bfP}^{m}$ is a ``universal codimension 3
Gorenstein scheme''.)

\bigskip
\begin{center}
{\large\bf II}
\end{center}
\smallskip

The second remark concerns the Kodaira Vanishing Theorem. We want to
show that Ramanujam's strong form of Kodaira Vanishing for surfaces of
Char.~0 is a consequence of a recent result of F.~Bogomolov. In
particular, this is interesting because it gives a new completely
algebraic proof of this result, and one which uses the Char.~0
hypothesis in a new way (it is used deep in Bogomolov's proof, where
one notes that if $V^{3}\to F^{2}$ is a ruled 3-fold and $D\subset
V^{3}$ is an irreducible divisor meeting the generic fibre
set-theoretically in one point, then $D$ is birational to
$F$). Ramanujam's result \cite{art01-key11} is this: let $F$ be a
smooth surface of Char.~0, $D$ a divisor on $F$.

Then
\begin{equation}
\left.
\begin{array}{@{}l}
(D^{2})>0\\
(D.C)\geq 0,\text{~~ all curves~~ } C\subset F
\end{array}
\right\}
\Rightarrow H^{1}(F,\mathscr{O}(-D))=(0)\label{art01-1}
\end{equation}

Bogomolov's theorem is that if $F$ is a smooth surface of char.~0, $E$
a rank 2 vector bundle on $F$, then 
\begin{align}
C_{1}(E)^{2}>4C_{2}(E) \Rightarrow & E\text{~is unstable, meaning
$\exists$ an extension}\notag\\
 & 0\to L(D)\to E\to I_{Z}L\to 0,\notag\\
 & I_{Z}=\text{~ ideal sheaf of a 0-dim. subscheme $Z\subset F$,}\notag\\
 & L\text{~ invertible sheaf, $D$ a divisor}\notag\\
 & D\in [\text{num. pos. cone,~ }(D^{2})>0,(D.H)>0].\label{art01-2}
\end{align}\pageoriginale
(See Bogomolov \cite{art01-key2}, Reid \cite{art01-key13}; another
proof using reduction $\mod p$ instead of invariant theory has been
found by D. Gieseker.)

To prove \eqref{art01-2} $\Rightarrow$ \eqref{art01-1}, suppose
$D_{1}$ is given satisfying the conditions of \eqref{art01-1}. Take
any element $\alpha\in H^{1}(F,\mathscr{O}(-D_{1}))$ and via $\alpha$,
form an extension
$$
0\to \mathscr{O}_{F}\xrightarrow{\mu}E\xrightarrow{\nu}\mathscr{O}_{F}(D_{1})\to 0.
$$
Note that $C_{1}(E)=D_{1}$, $C_{2}(E)=0$, $(D^{2}_{1})>0$, so $E$
satisfies the conditions of \eqref{art01-2}. Therefore, by Bogomolov's
theorem, $E$ is unstable: this gives an exact sequence
\begin{align*}
& 0\to L(D_{2})\xrightarrow{\sigma}E\xrightarrow{\tau}I_{Z}L\to 0\\
& D_{2}\in (\text{num. pos. cone}).
\end{align*}
Note that the subsheaf $\sigma(L(D_{2}))$ of $E$ cannot equal the
subsheaf $\mu(\mathscr{O}_{F})$ in the definition of $E$, because this
would imply, comparing the 2 sequences, that $D_{2}\equiv -D_{1}$
whereas both $D_{1}$, $D_{2}$ are in the numerically positive
cone. Therefore, the composition
$$
L(D_{2})\xrightarrow{\sigma}E\xrightarrow{\nu}\mathscr{O}_{F}(D_{1})
$$
is not zero, hence
$$
L\cong \mathscr{O}_{F}(D_{1}-D_{2}-D_{3}), D_{3}\text{~~ an effective divisor.}
$$
Next, comparing Chern classes of $E$ in its 2 presentations, we find
\begin{subequations}
\begin{gather}
2C_{1}(L)+D_{2}\equiv C_{1}(E)\equiv D_{1}\label{art01-3a}\\
(C_{1}(L)+D_{2})\cdot C_{1}(L)+\deg Z=C_{2}(E)=0.\label{art01-3b}
\end{gather}
\end{subequations}
By\pageoriginale \eqref{art01-3a}, we find $D_{1}-D_{2}-2D_{3}\equiv
0$, hence $L\simeq \mathscr{O}_{F}(+D_{3})$; by \eqref{art01-3b}, we
find $(D_{1}-D_{3})\cdot D_{3}\leq 0$. But
\begin{align*}
\det 
\begin{vmatrix}
(D^{2}_{1}) & (D_{1}\cdot D_{3})\\
(D_{1}\cdot D_{3}) & (D^{2}_{3})
\end{vmatrix}
&=(D^{2}_{1})[(D^{2}_{3})-(D_{1}\cdot D_{3})]\\
&\;\;+(D_{1}\cdot D_{3})[(D^{2}_{1})-2(D_{1}\cdot D_{3})]+(D_{1}\cdot D_{3})^{2}
\end{align*}
while
\begin{alignat*}{2}
& (D^{2}_{3})-(D_{1}\cdot D_{3})\geq 0 &\qquad& (\text{by~~ }3b)\\
& (D^{2}_{1})-2(D_{1}\cdot D_{3})=(D_{1}\cdot D_{2})>0 &\qquad& (\text{since~ }D_{1},D_{2}\text{~~ num. pos.})\\
& (D_{1}\cdot D_{3})\geq 0 &\qquad& (\text{by the assumptions on~ } D_{1}).
\end{alignat*}
On the other hand, this det is $\leq 0$ by Hodge's Index
Theorem. Therefore $(D_{1}\cdot D_{3})=0$ and $\det=0$. From the
latter, $D_{3}$ is numerically equivalent to $\lambda D_{1}$,
$\lambda\in \bfQ$, hence $(D_{1}\cdot D_{3})=\lambda
(D^{2}_{1})$. Thus $\lambda=0$ and since $D_{3}$ is effective,
$D_{3}=0$. Therefore the sub-sheaf $\sigma(L(D_{2}))$ is isomorphic to
$\mathscr{O}_{F}(D_{1})$ and defines a splitting of the original exact
sequence. Therefore the extension class $\alpha\in
H^{1}(\mathscr{O}_{F}(-D_{1}))$ is $0$, so
$H^{1}(\mathscr{O}_{F}(-D_{1}))=(0)$. 

\bigskip

\begin{center}
{\large\bf III}
\end{center}
\smallskip 

The last two remarks are applications of Kodaira's Vanishing
Theorem. To me it is quite amazing how this cohomological assertion
has such strong consequences, both for geometry and for local
algebra. Here is a geometric application. This application is a link
between the recent paper of Arakelov \cite{art01-key1} (proving
Shafarevich's finiteness conjecture on the existence of families of
curves over a fixed base curve, with prescribed degenerations), and
Raynaud's counter-example \cite{art01-key12} to Kodaira Vanishing for
smooth surfaces in char.~$p$. What I claim is this (this remark has
been observed by L.~Szpiro also):

\begin{prop*}
Let $p:F\to C$ be a proper morphism of a smooth surface $F$ onto a
smooth curve $C$ over a field $k$ of arbitrary characteristic. Let
$E\subset F$ be a section of $p$ and assume the fibres of $p$ have
positive arithmetic genus. Let $F_{0}$ be the normal surface obtained
by blowing down all components of fibres of $p$ not meeting $E$. Then:
\end{prop*}

\noindent
{\bf\em Kodaira's Vanishing Theorem}\pageoriginale \
{\em $\Longrightarrow (E^{2})\leq 0$. for ample divisors on $F_{0}$}

If $\Char (k)=0$, then Kodaira's Vanishing Theorem holds for $F_{0}$
(cf.~\cite{art01-key9}), so $(E^{2})\leq 0$ follows. This result, and
its refinement~~--~~$(E^{2})<0$ unless all the smooth fibres of $p$ are
isomorphic -- are due to Arakelov \cite{art01-key1}, who proved them
by a very ingenious use of the Weierstrass points of the fibres
$p^{-1}(x)$. On the other hand, if $\cchar(k)=p$, Raynaud has shown how
to construct examples of morphisms $p:F\to C$ and sections $E\subset
F$, where all the fibres of $p$ are irreducible but singular (thus
$F=F_{0}$), and $(E^{2})>0$. Thus Kodaira Vanishing is false for this
$F$. If $\cchar(k)=2$ or $3$, he finds in fact quasi-elliptic surfaces
$F$ of this type. This Proposition is, in fact, merely an elaboration
of the last part of Raynaud's example.

\medskip
\noindent
{\bf Proof of Proposition:}~Suppose $(E^{2})>0$. Let $p_{0}:F_{0}\to
C$ be the projection and let $E$ stand for the image of $E$ in $F_{0}$
too. Consider divisors on $F_{0}$ of the form
$$
H=E+p^{-1}_{0}(\mathfrak{U}),\quad \deg \mathfrak{U}>0.
$$
Then $(H^{2})>0$ and $(H\cdot C)>0$ for all curves $C$ on $F_{0}$, so
$H$ is ample by the Nakai-Moisezon criterion. On the other hand, let's
calculate $H^{1}(F_{0},\mathscr{O}(-H))$. We have
\begin{align*}
0 &\to H^{1}(C,p_{0},\ast \mathscr{O}(-H))\to
H^{1}(F_{0},\mathscr{O}(-H))\\
& \to H^{0}(C,R^{1}p_{0},\ast \mathscr{O}(-H))\to 0.
\end{align*}
Clearly
$$
p_{0},\ast \mathscr{O}(-H)=(0)\text{~ and~
}R^{1}p_{0},\ast\mathscr{O}(-H)\cong
(R^{1}p_{0},\ast\mathscr{O}(-E))\otimes \mathscr{O}_{c}(-\mathfrak{U}).
$$
Now using the sequences:
\begin{align*}
&
0\to \mathscr{O}_{F_{0}}(-E)\to \mathscr{O}_{F_{0}}\to \mathscr{O}_{E}\to
0\\
&
0\to \mathscr{O}_{F_{0}}\to \mathscr{O}_{F_{0}}(E)\to \mathscr{O}_{E}((E^{2}))\to 0
\end{align*}
we find
\[
\xymatrix@C=.7cm{
p_{0},\ast \mathscr{O}_{F_{0}}\ar@{=}[d]\ar[r]^-{\alpha} &
p_{0},\ast \mathscr{O}_{E}\ar@{=}[d]\ar[r] &
R^{1}p_{0},\ast\mathscr{O}(-E)\ar[r]^-{\beta} &
R^{1}p_{0},\ast \mathscr{O}_{F_{0}}\ar[r] & 0\\
\mathscr{O}_{C} & \mathscr{O}_{C} & 
}
\]
so\pageoriginale $\alpha$ and $\beta$ are isomorphisms, and (using the
fact that the 
genus of the fibres is positive):
\[
\xymatrix@C=.6cm{
0\ar[r] & p_{0},\ast\mathscr{O}_{F_{0}}\ar@{=}[d]\ar[r]^-{\gamma} &
p_{0},\ast \mathscr{O}(E)\ar@{=}[d]\ar[r] &
p_{0},\ast\mathscr{O}_{E}((E^{2}))\ar[r]^-{\delta} &
R^{1}p_{0},\ast \mathscr{O}_{F_{0}}\\ 
 & \mathscr{O}_{C} & \mathscr{O}_{C} & 
} 
\]
so $\gamma$ is an isomorphism, and $\delta$ is injective. Now via the
isomorphism resp: $E\to C$, let the divisor class $(E^{2})$ on $E$
correspond to the divisor class $\mathfrak{U}$ on $C$. Then
$p_{0},\ast \mathscr{O}_{E}((E^{2}))\cong \mathscr{O}_{C}(\mathfrak{U})$,
and we see that
$$
\mathscr{O}_{C}(\mathfrak{U})\subset
R^{1}p_{0},\ast \mathscr{O}_{F_{0}}\cong R^{1}p_{0},\ast\mathscr{O}_{F_{0}}(-E)
$$
hence
$$
\mathscr{O}_{C}\subset R^{1}p_{0},\ast \mathscr{O}_{F_{0}}(-H)
$$
hence
\begin{equation*}
H^{1}(C,\mathscr{O}(-H))\neq (0).\tag*{Q.E.D.}
\end{equation*}

\bigskip

\begin{center}
{\large\bf IV}
\end{center}
\smallskip 

The last remark is an application of Kodaira's Vanishing Theorem to
local algebra. It seems to me remarkable that such a global result
should be useful to prove local statements about the non-existence of
local rings, but this is the case. The question I want to study is
that of the {\em smoothability of non-Cohen-Macauley surface
singularities}. In other words, given a surface $F$, $P\in F$ a
non-CM-singular point, when does there exist a flat family of surfaces
$F_{t}$ parametrized by $k[[t]]$ such that $F_{0}=F$ while the generic
$F_{t}$ is smooth. More locally, the problem is:

Given a complete non-CM-purely\footnote[2]{i.e. $\mathscr{O}/I$ is
2-dimensional for all minimal prime ideals of
$I\subset \mathscr{O}$.}-2-dimensional local ring $\mathscr{O}$
without nilpotents, when does there exist a complete 3-dimensional
local ring $\mathscr{O}'$ and a non-zero divisor $t\in \mathscr{O}'$
such that
\begin{itemize}
\item[(a)] $\mathscr{O}\cong \mathscr{O}'/t\mathscr{O}'$

\item[(b)] $\mathscr{O}_{\mathscr{P}}$ regular for all prime ideals
$\mathscr{P}\subset \mathscr{O}$ with $t\not\in \mathscr{P}$. 
\end{itemize}

First\pageoriginale of all, let
$$
\mathscr{O}^{*}=\bigoplus\limits_{\substack{I\subset \mathscr{O}\\ \text{minimal}\\ \text{prime
ideals}}} 
\left(
\begin{array}{l}
\text{integral closure of $\mathscr{O}/I$}\\
\text{in its fraction field}
\end{array}
\right)
$$
and let
\begin{align*}
\widetilde{\mathscr{O}} &= \left\{a\in\mathscr{O}^{*}
\left|
\begin{array}{l}
m^{n}a\subset \mathscr{O}\text{~~ for some~~ }n\geq 1\\
m=\text{~ maximal ideal in~ }\mathscr{O}
\end{array}
\right.\right\}\\
&= \Gamma(\Spec\mathscr{O}-\text{closed~ pt.,~ }\mathscr{O})
\end{align*}
Note that $\widetilde{\mathscr{O}}$ is a finite $\mathscr{O}$-module
and $m^{n}\cdot \widetilde{\mathscr{O}}\subset \mathscr{O}$ for some
large $n$, so that $\widetilde{\mathscr{O}}/\mathscr{O}$ is an
$\mathscr{O}$-module of finite length. Moreover, it is easy to see
that $\widetilde{\mathscr{O}}$ is a semi-local Cohen-Macauley ring. It
has been proven by Rim \cite{art01-key14} (cf. also
Hartshorne \cite{art01-key6}, Theorem 2.1, for another proof) that:
$$
\mathscr{O}\text{~~
smoothable~~}\Rightarrow \widetilde{\mathscr{O}}\text{~ local.}
$$
The result we want to prove is:

\begin{theorem*}
Assume $\cchar(\mathscr{O}/m)=0$, $\Spec \mathscr{O}$ has an isolated
singularity at its closed point and that $\mathscr{O}$ is smoothable,
so that, by the remarks above, $\widetilde{\mathscr{O}}$ is a normal
local ring. Let $\pi:X^{*}\to \Spec \mathscr{O}$ be a resolution and
let
$$
p_{a}(\widetilde{\mathscr{O}})=l(R^{1}\pi_{*}\mathscr{O}_{X})
$$
be the genus of the singularity $\widetilde{\mathscr{O}}$. Then
$$
l(\widetilde{\mathscr{O}}/\mathscr{O})\leq
p_{a}(\widetilde{\mathscr{O}}). 
$$
\end{theorem*}

Actually, for our applications, we want to know this result for rings
$\mathscr{O}$ where $\Spec \mathscr{O}$ has ordinary double curves
too, with a suitable definition of $p_{a}$. We will treat this rather
technical generalization in an appendix.

For example, the theorem shows:

\begin{coro*}
Let $\widetilde{\mathscr{O}}=k[[x,y]]$, $\cchar k=0$. Let $I\subsetneqq
(x,y)$ be an ideal of finite codimension. Then if $\mathscr{O}=k+I$,
$\mathscr{O}$ is not smoothable.
\end{coro*}

On\pageoriginale the other hand, if $F\subset \bfP^{n}$ is an elliptic
ruled surface and $\mathscr{O}'$ is the completion of the local ring
of the cone over $F$ at its apex, then $\mathscr{O}'$ is a normal
3-dimensional ring which is not Cohen-Macauley. If
$C=V(t)=(F\cdot \bfP^{n-1})$ is a generic hyperplane section of $F$,
then $t\in \mathscr{O}'$ and $\mathscr{O}=\mathscr{O}'/t\mathscr{O}'$
is the completion of the local ring of the cone over $C$ at its
apex. Now $C$ is an elliptic curve, but embedded by an incomplete
linear system -- in fact, $C$ is a projection of an elliptic curve
$\widetilde{C}$ in $\bfP^{n}$ from a point not on $\widetilde{C}$ --
this follows from the exact sequence:
\[
\xymatrix@C=.5cm{
0\ar[r] & H^{0}(\mathscr{O}_{F})\ar[r] &
H^{0}(\mathscr{O}_{F}(1))\ar[r] & H^{0}(\mathscr{O}_{C}(1))\ar[r] &
H^{1}(\mathscr{O}_{F})\ar@{=}[d]\ar[r] & 0\\
 & & & & \bfC & 
}
\]

Let $\widetilde{\mathscr{O}}$ be the completion of the local ring of
the cone over $\widetilde{C}$ at its apex. Then
$\widetilde{\mathscr{O}}$ is a normal 2-dimensional ring, in fact an
``elliptic singularity'', i.e., $p_{a}(\widetilde{\mathscr{O}})=1$;
moreover $\widetilde{\mathscr{O}}\supset \mathscr{O}$ and
$\dim \widetilde{\mathscr{O}}/\mathscr{O}=1$. This shows that there
are smoothable singularities $\mathscr{O}$ with
$$
l(\widetilde{\mathscr{O}}/\mathscr{O})=p_{a}(\widetilde{\mathscr{O}})=1.
$$

\noindent
{\bf Proof of Theorem.}~
Let $\mathscr{O}\cong \mathscr{O}'/t\mathscr{O}'$ give the smoothing
of $\mathscr{O}$. The proof is based on an examination of the exact
sequence of local cohomology groups:
$$
(*)\ldots\rightarrow H^{1}_{\{x\}}(\mathscr{O}')\to
H^{1}_{\{x\}}(\mathscr{O})\to
H^{2}_{\{x\}}(\mathscr{O}')\xrightarrow{t}H^{2}_{\{x\}}(\mathscr{O}')\xrightarrow{\alpha}H^{2}_{\{x\}}(\mathscr{O})\to\ldots 
$$
where $x\in \Spec \mathscr{O}\subset \Spec \mathscr{O}'$ represents
the closed point.

What can we say about each of these groups?
\begin{itemize}
\item[(a)] $H^{1}_{\{x\}}(\mathscr{O}')$ is zero since $\mathscr{O}'$
is an integrally closed ring of dimension 3, hence has depth at least
2.

\item[(b)] To compute $H^{1}_{\{x\}}(\mathscr{O})$, use
$$
H^{0}(\Spec \mathscr{O},\mathscr{O})\to
H^{0}(\Spec \mathscr{O}-\{x\},\mathscr{O})\to
H^{1}_{\{x\}}(\mathscr{O})\to 0
$$
which gives us:
$$
H^{1}_{\{x\}}(\mathscr{O})\cong \widetilde{\mathscr{O}}/\mathscr{O}. 
$$

\item[(c)] As\pageoriginale for $H^{2}_{\{x\}}(\mathscr{O}')$, it
measures the degree to which $\mathscr{O}'$ is not Cohen-Macauley. A
fundamental fact is that it is of finite length--cf. Theoreme de
finitude, p.~89, in Grothendieck's seminar~\cite{art01-key15}.

\item[(d)] As for $H^{2}_{\{x\}}(\mathscr{O})$, we can say at least:
\begin{align*}
H^{2}_{\{x\}}(\mathscr{O}) &\cong
H^{1}(\Spec \mathscr{O}-\{x\},\mathscr{O})\\
&\cong
H^{1}(\Spec \widetilde{\mathscr{O}}-\{x\},\widetilde{\mathscr{O}})\\
&\cong H^{2}_{\{x\}}(\widetilde\mathscr{O})
\end{align*}
but unfortunately this group is huge: it is not even an
$\widetilde{\mathscr{O}}$-module of finite type.
\end{itemize}

However, for any local ring $\mathscr{O}$ with residue characteristic
$O$ and with isolated singularity, we can define, by using a
resolution of $\Spec \mathscr{O}$, important subgroups:
$$
H^{i}_{\{x\},\text{int}}(\mathscr{O}).
$$
Namely, let $\pi:X\to \Spec \mathscr{O}$ be a resolution and set
$$
H^{i}_{\{x\},\text{int}}(\mathscr{O})=\Ker[\pi^{*}:H^{i}_{\{x\}}(\mathscr{O})\to
H^{i}_{\pi^{-1}x}(\mathscr{O}_{X})].
$$
This is independent of the resolution, as one sees by comparing any 2
resolutions $\pi_{i}:X_{i}\to \Spec \mathscr{O}$, $i=1,2$, via a 3rd:
\[
\xymatrix{
 & X_{1}\ar[dr]^-{\pi_{1}} &\\
X_{3}\ar[ur]^-{f}_-{\pi_{3}}\ar[rr]\ar[dr] & & \Spec\mathscr{O}\\
 & X_{2}\ar[ur]_-{\pi_{2}} & 
}
\]
and using the Leray spectral sequence
$$
H^{p}_{\pi_{1}-1_{(x)}}(X_{1},R^{q}f_{*}(\mathscr{O}_{X_{3}}))\Rightarrow
H^{*}_{\pi_{3}-1_{(x)}}(X_{3},\mathscr{O}_{X_{3}}) 
$$
plus Matsumura's result $R^{q}f_{*}\mathscr{O}_{X_{3}}=(0)$, $q>0$
where $X_{1}$ and $X_{3}$ are smooth and characteristic
zero. Moreover, when $\mathscr{O}\cong \mathscr{O}'/I$, then the
restriction map
$$
H^{i}_{\{x\}}(\mathscr{O}')\to H^{i}_{\{x\}}(\mathscr{O}) 
$$
gives\pageoriginale
$$
H^{i}_{\{x\},\text{int}}(\mathscr{O}')\to
H^{i}_{\{x\},\text{int}}(\mathscr{O}) 
$$
because we can find resolutions fitting into a diagram:
\[
\xymatrix@C=2.5cm@R=1.5cm{
X\ar@{^(->}[r]\ar[d] & X'\ar[d]\\
\Spec \mathscr{O}\ar@{^(->}[r] & \Spec \mathscr{O}'.
}
\]
Next, we prove using the Kodaira Vanishing Theorem and following
Hartshorne and Ogus (\cite{art01-key16}p.~424):

\begin{lemma*}
Assume $x$ is the only singularity of $\mathscr{O}$,
$\dim \mathscr{O}=n$ and $\pi:X\to \Spec \mathscr{O}$ is a
resolution. Then
$$
H^{i}_{\{x\}'\text{int}}(\mathscr{O})\cong
H^{i}_{\{x\}}(\mathscr{O}),\quad 0\leq i\leq n-1
$$
and
$$
H^{i}_{\{x\}'\text{int}}(\mathscr{O})\cong
R^{i-1}\pi_{*}(\mathscr{O}_{X})_{x},\quad 2\leq i\leq n.
$$
\end{lemma*}

\begin{proof}
Because $\mathscr{O}$ has an isolated singularity, we may assume
$\mathscr{O}\cong \widehat{\mathscr{O}}_{x,X_{0}}$, where $X_{0}$ is
an $n$-dimensional projective variety with $x$ its only singular
point. We may assume our resolution is global:
$$
\pi :X\to X_{0}.
$$
Let $\widehat{X}=X\times_{X_{0}}\Spec \mathscr{O}$ and let $I$ be the
injective hull of $\mathscr{O}_{x,X_{0}}/m_{x,X_{0}}$ as
$\mathscr{O}_{x,X_{0}}$-module. Then according to Hartshorne's formal
duality theorem (cf.~\cite{art01-key7}p.~94), for all coherent
sheaves $\mathscr{F}$ on $X$, the 2 $\mathscr{O}$-modules
$$
H^{i}_{\pi-1_x}(\mathscr{F}),\quad \Ext^{n-i}_{\mathscr{O}_{\widehat{X}}}(\widehat{F},\Omega^{n}_{\widehat{X}}) 
$$
are dual via $\Hom(-,I)$. In particular,
$$
H^{i}_{\pi-1_x}(\mathscr{O}_{X}),\quad
H^{n-i}(\Omega^{n}_{\widehat{X}})
$$
are dual. But 
$$
H^{n-i}(\Omega^{n}_{\widehat{X}})\cong R^{n-i}\pi_\ast
(\Omega^{n}_{X})\otimes_{\mathscr{O}_{X_{0}}}\mathscr{O} 
$$
and it has been shown by Grauert and Riemenschneider \cite{art01-key5}
that $R_{i}\pi_{\ast}\break (\Omega^{n}_{X})=(0)$, $i>0$. (This is a simple
consequence of Kodaira's Vanishing Theorem because if $L_{0}$ is an
ample invertible sheaf on $X_{0}$ with\pageoriginale
$H^{i}(X_{0},L_{0}\otimes \pi_{\ast}\Omega^{n}_{X})=(0)$, $i>0$, then
by the Leray Spectral Sequence:
\[
\xymatrix@C=-.1cm{
H^{i}(X,\pi^{*}L_{0}\otimes \Omega^{n}_{X})\ar@{<->}[d]^-{\text{dual}}
& \cong H^{0}(X_{0},L_{0}\otimes R^{i}\pi_{\ast}\Omega^{n}_{X})\\
H^{n-i}(X,(\pi^{*}L_{0})^{-1}) & 
}
\]
and Kodaira's Vanishing Theorem applies to all invertible sheaves $M$
such that $\Gamma(X,M^{n})$ is base point free and defines a
birational morphism, $n\geq 0$ (cf.~\cite{art01-key9}).)
Recapitulating, this shows $H^{n-i}(\Omega^{n}_{\widehat{X}})=(0)$,
$i<n$, hence $H^{i}_{\pi-1_{x}}(\mathscr{O}_{X})=(0)$, $i<n$, hence
$H^{i}_{\{x\}\text{int}}(\mathscr{O})\xrightarrow{\approx}H^{i}_{\{x\}}(\mathscr{O})$
is an isomorphism.

To get the second set of isomorphisms, we use the Leray Spectral
Sequence:
$$
H^{p}_{\{x\}}(X_{0},R^{q}\pi_{*}\mathscr{O}_{X})\Rightarrow
H^{*}_{\pi-1_{x}}(X,\mathscr{O}_{X}). 
$$
The only non-zero terms occur for $p=0$ or $q=0$, so we get a long
exact sequence
\begin{align*}
\ldots \to H^{0}_{\{x\}}(R^{i-1}\pi_{*}\mathscr{O}_{x})\to
H^{i}_{\{x\}}(\pi_{*}\mathscr{O}_{x}) &\to
H^{i}_{\pi-1_{x}}(\mathscr{O}_{X})\to
H^{0}_{\{x\}}(R^{i}_{\pi_{*}}\mathscr{O}_{x})\\
&\to H^{i+1}_{\{x\}}(\pi_{*}\mathscr{O}_{X})\to\ldots 
\end{align*}

Using the first part, plus the isomorphism:
\begin{align*}
H^{i}_{\{x\}}(\pi_{*}\mathscr{O}_{X}) & \cong
H^{i-1}(\Spec \mathscr{O}-\{x\},\pi_{*}\mathscr{O}_{X})\\
&\cong H^{i-1}(\Spec \mathscr{O}-\{x\},\mathscr{O})\\
&\cong H^{i}_{\{x\}}(\mathscr{O}), \ i\geq 2,
\end{align*}
we get the results.\hfill Q.E.D.
\end{proof}

We now go back to the sequence (*). It gives us:
$$
0\to \widetilde{\mathscr{O}}/\mathscr{O}\to
H^{2}_{\{x\},\text{int}}(\mathscr{O}')\xrightarrow{t}H^{2}_{\{x\},\text{int}}(\mathscr{O}')\to
R^{1}\pi_{*}(\mathscr{O}_{X})_{x}\to\ldots 
$$
where $\pi:X\to \Spec \mathscr{O}$ is a resolution. Therefore
\begin{align*}
l(\widetilde{\mathscr{O}}/\mathscr{O}) &= l(\ker \text{~ of~ }
t\text{~ in~ } H^{2}_{\{x\},\text{int}}(\mathscr{O}'))\\
&=l(\Coker \text{~ of~ } t\text{~ in~
}H^{2}_{\{x\},\text{int}}(\mathscr{O}'))\\
&\leq l(R^{1}\pi_{*}(\mathscr{O}_{X})_{x})=p_{a}(\widetilde{\mathscr{O}}).\tag*{Q.E.D.}
\end{align*}

\appendix

\begin{center}
{\large\bf Appendix}\pageoriginale
\end{center}

The purpose of this appendix is to make a rather technical extension
of the result in \S\ IV, which seems to be better for use in
applications. Let $X$ be an affine surface, reduced, with at most
ordinary double curves, plus one point $P\in X$ about which we know
nothing. Let 
$$
\widetilde{X}=\Spec \Gamma(X-P,\mathscr{O}_{X})
$$
so that we get
$$
\pi_{1}:\widetilde{X}\to X,
$$
an isomorphism outside $P$, everywhere a finite morphism, with
$\widetilde{X}$ Cohen-Macauley. Our goal is to show that in certain
cases $X$ is not smoothable near $P$, i.e., $\not\exists$ an analytic
family
$$
f:X'\to \Delta=\text{~disc in the $t$-plane,}
$$
where $f^{-1}(0)\approx$ (neighborhood of $P$ in $X$), and $f^{-1}(t)$
is smooth, $t\neq 0$. (Here we work in the analytic setting rather
than the formal one to be able below to take an exponential.) We know
that a necessary condition for $X'$ to exist is that $\pi^{-1}_{1}(P)$
is one point $\widetilde{P}$, so henceforth we assume this too. Next
blow up $\widetilde{X}$, but only at $\widetilde{P}$ and at centers
lying over $\widetilde{P}:$ it is not hard to see that we arrive in
this way at a birational proper morphism
$$
\pi_{2}:X^{*}\to \widetilde{X}
$$
such that $X^{*}$ has at most ordinary double curves and pinch points
(points like $z^{2}=x^{2}y$), these pinch points moreover lying over
$\widetilde{P}$. Define
$$
p_{a}(\mathscr{O}_{\widetilde{P}})=\dim_{C}[R^{1}\pi_{2,*}(\mathscr{O}_{X^{*}})_{\widetilde{P}}].
$$
It is easy to verify that this number is independent of the choice of
$X^{*}$. (However, this would {\em not} be true if $\widetilde{X}$ had
cuspidal lines--in this case, there is no bound on dim $R^{1}\pi_{*}$
as you blow up $\widetilde{X}$ more and more!) We claim the following

\begin{theorem*}
If\pageoriginale $X$ is smoothable near $P$, then
$l(\mathscr{O}_{\widetilde{P}}/\mathscr{O}_{P})\leq
p_{a}(\mathscr{O}_{\widetilde{P}})$. 
\end{theorem*}

\begin{proof}
We follow the same plan as in the case where $\mathscr{O}$ has an
isolated singularity, except that, for an arbitrary local ring
$\mathscr{O}$, we set
$$
H^{i}_{\{x\},\text{int}}(\mathscr{O})=\bigcup_{\left(\substack{\text{modifications}\\
\pi:X\to \Spec\mathscr{O}\\ \text{where}\\
X-\pi^{-1}(x)\xrightarrow{\approx}\Spec \mathscr{O}-\{x\}}\right)}[\ker:
H^{i}_{\{x\}}(\mathscr{O})\to H^{i}_{\pi-1_{(x)}}(\mathscr{O}_{X})]
$$
The proof is then the same as before except that we cite only the
following case of the lemma:
\end{proof}

\begin{theorem*}[(Boutot \cite{art01-key17})]
Let $\mathscr{O}$ be a normal excellent local $k$-algebra, with
residue field $k$, and $\cchar (k)=0$. Then:
$$
H^{2}_{\{x\},\text{int}}(\mathscr{O})=H^{2}_{\{x\}}(\mathscr{O}). 
$$
\end{theorem*}

This result is a Corollary of Proposition 2.6, Chapter
V \cite{art01-key17}. Since $\cchar(k)=0$, we may disregard ``red'' in
that Proposition and apply it to the values of the functor on the dual
numbers. It tells us that there is a blow-up
$\pi:X\to \Spec(\mathscr{O})$ concentrated at the origin such that
$$
\Pic_{X/k}(k[\epsilon]/(\epsilon^{2}))\to \Pic_{\Spec(\mathscr{O})-\{x\}}(k[\epsilon]/(\epsilon^{2})) 
$$
is an isomorphism. In other words, in the sequence:
$$
\to H^{1}(X,\mathscr{O}_{X})\xrightarrow{\alpha}
H^{1}(X-\pi^{-1}(x),\mathscr{O}_{X})\xrightarrow{\beta}H^{2}_{\pi^{-1}(x)}(\mathscr{O}_{X})\to 
$$
$\alpha$ is surjective, hence $\beta$ is zero.

\begin{thebibliography}{}
\bibitem{art01-key1} S. Ju. Arakelov: Families of algebraic curves
with fixed degeneracies, {\em Izvest. Akad. Nauk}, 35 (1971).

\bibitem{art01-key2} F. Bogomolov, to appear.

\bibitem{art01-key3} D. Buchsbaum\pageoriginale and D. Eisenbud:
Algebra structures for finite free resolutions and some structure
theorems for ideals of codimension 3, {\em Amer. J. Math.} (to appear).

\bibitem{art01-key4} R. Fossum: The divisor class group of a Krull
domain, {\em Springer Verlag}, (1973).

\bibitem{art01-key5} H. Grauert and O. Riemenschneider:
Verschwindungss\"atze f\"ur analytische Kohomologiegruppen auf
komplexen R\"aumen, {\em Inv. Math.,} 11 (1970).

\bibitem{art01-key6} R. Hartshorne: Topological conditions for
smoothing algebraic singularities, {\em Topology}, 13 (1974).

\bibitem{art01-key7} R. Hartshorne: Ample subvarieties of algebraic
varieties, {\em Springer Lecture Notes 156} (1970). 

\bibitem{art01-key8} H. Hironaka: Resolution of singularities of an
algebraic variety over a field of char. zero, {\em Annals of Math.,}
79 (1964).

\bibitem{art01-key9} D. Mumford: Pathologies III, {\em
Amer. J. Math.,} 89 (1967). 

\bibitem{art01-key10} C. P. Ramanujam: On a certain purity theorem,
{\em J. Indian Math. Soc.,} 34 (1970).

\bibitem{art01-key11} C. P. Ramanujam: Remarks on the Kodaira
Vanishing Theorem, {\em J. Indian Math. Soc.,} 36 (1972) and 38 (1974).

\bibitem{art01-key12} M. Raynaud: Contre-example au ``Vanishing
Theorem'' en caract\'erisque $p>0$, this volume.

\bibitem{art01-key13} M. Reid: Bogomolov's theorem $C^{2}_{1}\leq
4C_{2}$, to appear in {\em Proc. Int. Colloq. in Alg. Geom.,} Kyoto, (1977).

\bibitem{art01-key14} D. Rim: Torsion differentials and deformation,
{\em Trans. Amer. Math. Soc.,} 169 (1972).

\bibitem{art01-key15} SGA 2, Cohomologie locale des faisceaux
coherents., by A. Grothendieck and others, North-Holland Publishing
Co., (1968). 

\bibitem{art01-key16} R. Hartshorne and A. Ogus: On factoriality of
local rings of small embedding codimension, {\em Comm. in Algebra,} 1
(1974).

\bibitem{art01-key17} J. F. Boutot: Sch\'ema de Picard Local, thesis,
Orsay, (1977).
\end{thebibliography}


%page 260 
