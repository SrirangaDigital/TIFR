\title{Appendix}\label{apen}
\markright{Appendix}

\author{By~ C.P. Ramanujam}
\markboth{C.P. Ramanujam}{Appendix}

\date{}
\maketitle

\setcounter{page}{165}
\setcounter{pageoriginal}{129}
The proof\pageoriginale of Prop. 3 has been essentially global. The following result gives a ``local proof'' of Prop. 3. 

\begin{prop*}[\footnote{This proposition could be used to prove straightaway that the Weil divisor $C$ in Prop. 4 is a divisor; thereby one could avoid going to the variety $Z$ for the proof of Prop. 4.}]
Let $A$ be a noetherian analytically normal local ring with maximal ideal $\mathfrak{m}$. Let $B$ be a noetherian local ring with maximal ideal $\mathfrak{n}$ such that $B$ contains $A$ and has the same residue field as $A$. Let us suppose that $\mathfrak{n} \cap A = \mathfrak{m}$ and that $\mathfrak{n}$ is generated by $\mathfrak{m}$ together with elements $x_1, \ldots, x_r$ of $B$, where $r = \dim B - \dim A$.

Then any prime ideal $\mathfrak{p}$ of height one in $B$ which is not contained in $\mathfrak{m} \cdot B$ is principal.
\end{prop*}

\begin{proof}
Let $A$ and $B$ be the completions of $A$ and $B$ respectively. The hypothesis that $\mathfrak{n}$ is generated by $\mathfrak{m}$ together with $r = \dim B - \dim A$ elements on $\mathfrak{n}$ implies that $B$ is isomorphic to the formal power series ring $A \left[[X_1, \ldots, X_r] \frac{}{} \hspace{-0.1cm}\right]$ in $r$ variables over $A$, as can be shown easily. Since $\mathfrak{p}$ is of height one in $B$, there exist elements $a$, $b \in B$ such that $(a:b)= \mathfrak{p}$, and hence by a well-known property of completions ($B$ is $B$ flat), we deduce that $(a B : bB) = \mathfrak{p} B = \mathfrak{p}$, which implies in turn (since $\hat{B}$ is normal) that all the associated prime ideals of $\mathfrak{p} B$ in $B$ are of height one. Moreover, since $\mathfrak{p} \not\subset \mathfrak{m}\cdot B$, $\mathfrak{p} \not\subset \mathfrak{m} \cdot B = \mathfrak{m} \cdot B$, and the same holds for all the associated prime ideals of $\mathfrak{p}$. If we could prove that each of these associated prime ideals is principal, it would then follow that $\mathfrak{p}B$ is principal, and hence that $\mathfrak{p}$ is principal (since $B$ is $B$-flat).

Thus, it is enough to prove the theorem when $A$ is a complete normal local domain, and $B$ a formal power series ring $A \left[ [X_1, \ldots, X_r]\frac{}{} \hspace{-0.1cm} \right]$ in $r$ variables over $A$. 
\end{proof}

We shall do this in a series of lemmas.

\begin{lem}\label{apen-lem1}
Let $A$ be a local ring with maximal ideal $\mathfrak{m}$, and $f$ an element of $A \left[ [X_1, \ldots, X_p] \right]$ which does not lie in the ideal $\mathfrak{m} A \left[[X_1, \ldots, X_p] \right]$. Then there is an automorphism $\varphi$ of $A\left[[X_1, \ldots, X_p] \right]$ which fixes $A$ such that $\varphi (f)\not\in (\mathfrak{m}, X_1, \ldots, X_{p-1})$.
\end{lem}

\begin{proof}
The proposition being trivial for $p=1$, we may assume that $p \geqq 2$. We proceed by induction on $p$.

Let $p=2$ and $f = \sum\limits_{r = s} a_{r,s} X^r_1 X^s_2$, and let $a_{r_0, s_0}$ be a coefficient which is not in $\mathfrak{m}$. We may assume that $r_0$ is minimal with this property. Let $N$ be any integer greater than $S_0$, and define
\begin{align*}
\varphi (X_1) & = X_1 + X^N_2 \\
\varphi (X_2) & = X_2.
\end{align*}
Then $\varphi$ extends to a unique automorphism of $A \left[[X_1, X_2] \right]$, which fixes $A$, by continuity. The coefficient of $X^{r_0 N+ S_0}_2 $ in $\varphi (f)$ is the sum of the $a_{r,s}$ for which $rN+ s = r_0 N + s_0$. This implies that $N$ divides $s-s_0$, and since $s$ and $s_0$ are non-negative and $N >s_0$, that $s \geqq s_0$, and $r \leqq r_0$. By the minimality of $r_0$, we see that those $a_{r,s}$ which occur in the sum with $(r,s) \neq (r_0, s_0)$ are in $\mathfrak{m}$, and since $a_{r_0. s_0} \not\in \mathfrak{m}$, the coefficient of $X^{r_0 N+s_0}_2$ in $\varphi (f)$ is not in $\mathfrak{m}$.

Now\pageoriginale let $p$ be any integer $>2$ and assume that the lemma is valid for $p-1$ variables. Writing $f = \sum\limits_{q=0}f_q X^q_p$, with $f_q \in A \left[[x_1, \ldots, X_{p-1}] \right]$, we see that there is an $f_{q_0} \not\in \mathfrak{m} A \left[ [X_1, \ldots, X_{p-1}]\right]$, and by induction hypothesis, there is an automorphism $\varphi'$ of $A \left[[X_1,\ldots, X_{p-1}] \right]$ such that $\varphi'(f_{q_0}) \not\in (\mathfrak{m}, X_1, \ldots, X_{p-2})$. Extending $\varphi'$ to $A \left[[X_1,\ldots, X_p] \right]$ by putting $\varphi'(X_p) = X_p$ (and continuity), we see that $\varphi'(f) \in A \left[ [X_1, \ldots, X_{p-2}]\right]\left[ [X_{p-1}, X_p]\right]$ and $\varphi' (f) \not\in (\mathfrak{m}, X_1, \ldots, X_{p-2})$. Since we have proved the lemma for $p=2$, there is an automorphism $\varphi''$ of $A \left[[X_1, \ldots, X_p] \right]$ such that $\varphi''(\varphi'(f))\not\in (\mathfrak{m}, X_1, \ldots, X_{p-2}, X_{p-1})$, and $\varphi=\varphi'' \circ \varphi'$ fulfills the requirements.
\end{proof}
Lemma \ref{apen-lem1} clearly reduces the proof of the proposition to the case when $B = A[[X]]$ and $A$ are before. We shall assume this from now on.


\begin{lem}\label{apen-lem2}
Let $A$ be a complete local ring with maximal ideal $\mathfrak{m}$ and $f \in A [[X]]$, $f\notin\mathfrak{m}[[X]]$. Let $r$ be the least integer such that $f_r \not\in \mathfrak{m} \left(f = \sum\limits^\infty_0 f_r X^r \right)$. If $g \in A [[X]]$ there exists a unique $h \in A [[X]]$ such that 
\begin{align*}
g & = hf + R,\\
R & = \sum\limits^{r-1}_{0} r_i X^i,
\end{align*}
In particular, $f$ is the associate of a unique polynomial $X^r+\alpha_1 X^{r-1} + \ldots +\alpha_r$, $\alpha_i \in \mathfrak{m}$.
\end{lem}

\begin{proof}
The proof of the Weierstrass preparation theorem as given in Zariski and Samuel (Vol. II, Commutative Algebra, p. 139) carries over almost word for word.
\end{proof}

\begin{lem}\label{apen-lem3}
A is as in Lemma \ref{apen-lem2}, $\mathfrak{a}$ an ideal of $A \left[[X] \right]$, $\mathfrak{a} \not\subset \mathfrak{m} [[X]]$. Then $\mathfrak{a}$ is generated by $\mathfrak{a} \cap A [X]$. 
\end{lem}

\begin{proof}
Let $f$ be any element of $\mathfrak{a}$ which is not in $\mathfrak{m} [[X]]$, and $r$ as in Lemma \ref{apen-lem1}. If $g$ is any element of $\mathfrak{a}$, it follows from Lemma \ref{apen-lem1} that $g = hf+R$, $R \in A [X]$, and $f$ is the associate of an element of $A[X]$. Lemma \ref{apen-lem3} is proved.

We can now complete the proof of the proposition. By Lemma \ref{apen-lem3} it is enough to show that $\mathfrak{p} \cap A [X] = \mathfrak{p}_1$, is a principal ideal of $A[X]$. Let $S$ be the set of non-zero elements of $A$; then $A [X]_S = Q [X]$, $Q$ being the quotient field of $A$. We assert that $S \cap \mathfrak{p}_1 = \emptyset$. In fact, if $\alpha\in \mathfrak{p}_1 \cap S$, $\mathfrak{p}_1$ (being of height one) must be associated to $\alpha A [X]$. But since $\alpha \in A$, the associated prime ideals of $\alpha$ in $A [X]$ are all the extensions to $A[X]$ of the prime ideals associated to $\alpha$ in $A$, and all these extensions are $\subset \mathfrak{m}[X]$. This contradicts the fact that $\mathfrak{p}_1 \not\subset \mathfrak{m}[X]$. 

It follows that $\mathfrak{p}_1 Q [X]$ is a prime ideal of $Q[X]$. Let $f= X^n + a_1 X^{n-1} + \cdots + a_n, a_i \in Q$ be the monic polynomial which generates $\mathfrak{p}_1 Q [X]$. We assert that $a_i \in A$. In fact, let $\bar{X}$ be the image of $X$ in $Q [X]/ \mathfrak{p}_1 Q [X] = Q [\bar{X}]$. Since $\mathfrak{p} \not\subset \mathfrak{m}[[X]]$, it follows from Lemma \ref{apen-lem2} that there is a monic polynomial over $A[X]$ which is in $\mathfrak{p}$, and hence in $\mathfrak{p}_1$, this proves that $\bar{X}$ is integral over $A$. But since $A$ is integerally closed, the minimal polynomial of $\bar{X}$ over $Q$, i.e., $f$,\pageoriginale must be in $A[X]$, and our assertion is proved. But now, since $\mathfrak{p}_1$ is prime, we must have $\mathfrak{p}_1 Q [X] \cap A [X] = \mathfrak{p}_1 (A [X])_S \cap A [X]= \mathfrak{p}_1$, and $f \epsilon \mathfrak{p}_1$. Since $f$ is monic, and divides any element of $\mathfrak{p}_1$ in $Q[X]$, it does so in $A [X]$.
\end{proof}

Our result is prove.

{\em Added in proof}. The proposition in the Appendix has been proved in a special case by P. Samuel (Sur une conjecture de Grothendieck, Comptes Rendus, Paris, Tome 255).


\begin{thebibliography}{99}
\itemsep=0pt
\bibitem{apen-key1} Chevalley, C.: Sur la th\'eorie de la vari\'et\'e de Picard. Am. J. Math. 82, 435-490 (1960).

\bibitem{apen-key2} - Vari\'et\'es de Picard. S\'eminaire E. N. S. 1958-59.

\bibitem{apen-key3} Grothendieck, A.: Expos\'e 190. S\'eminaire Bourbaki, Feb. 1960.

\bibitem{apen-key4} Lang, S.: Abelian varieties. New York: Interscience publishers. Inc., 1959. 

\bibitem{apen-key5} Seshadri, C. S.: Some results on the quotient space by an algebraic group of automorphisms. Math. Ann. 149, 286-301 (1963).

\bibitem{apen-key6} Weil, A.: On the projective imbedding of abelian varieties, Algebraic Geometry and Topology - A symposium in honour of S. Lefschetz. Princeton, New Jersey: Princeton University Press 1957.
\end{thebibliography}


\begin{center}
{\em Received September 19, 1962}
\end{center}
