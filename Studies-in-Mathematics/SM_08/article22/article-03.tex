\title{Contre-Exemple Au ``Vanishing Theorem'' En Caract\'eristique \texorpdfstring{$p>0$}{p0}}\label{art03}
\markright{Contre-Exemple Au ``Vanishing Theorem''......}

\author{par~ M.~Raynaud}
\markboth{M. Raynaud}{Contre-Exemple Au ``Vanishing Theorem''......}

\date{}
\maketitle

\setcounter{page}{319}

\setcounter{pageoriginal}{272}
\textsc{Soit}\pageoriginale $k$ un corps alg\'ebriquement clos de
caract\'eristique $p>0$. Nous allons construire une surface $X$,
propre et lisse sur $k$, et un faisceau inversible ample $\mathscr{I}$
sur $X$, tel que $H^{1}(X,\mathscr{I}^{-1})\neq 0$. $\hleftarrow$
Ainsi le th\'eor\'eme de Kodaira \cite{art03-key2} $\hleftarrow$, ne
s'etend pas en caract\'eristique $p>0$. Ce contre-exemple est \`a
rapprocher de celui obtenu par Mumford avec une surface normale, non
lisse \cite{art03-key5}.

On va construire la surface $X$ comme fibr\'ee sur une courbe $C$,
avec des fibres integres et une ligne horizontale de ``cusps''. Ces
fibres sont les compl\'etions projectives de courbes affines
d'\'equa\-tion $y^{2}=x^{p}+a$, si $p\neq 2$ et $y^{3}=x^{2}+a$ si
$p=2$.

Je tiens \`a remercier messieurs Oda et Spiro pour l'aide pr\'ecie\-use
qu'ils m'ont apport\'ee dans l'\'elaboration de ce travail.

\section{Frobenius sur les courbes}\label{art03-sec1}

On reprend ici une petite partie des r\'esultats de
Tango \cite{art03-key6}. 

Soit $C$ une courbe propre et lisse sur $k$, de genre $g$, de corps
des fractions $K$ et soit $F:C'\to C$ le $k$-morphisme radiciel de
degr\'e $p$, correspondant \`a la fermeture int\'egrale de $C$ dans
$K'=K^{1/p}$. La diff\'erentielle
$$
d:\mathscr{O}_{C'}\to \Omega^{1}_{C'}
$$
donne des suites exactes de $\mathscr{O}_{C}$-modules:
\begin{align}
& 0\to \mathscr{O}_{C}\to
F_{*}(\mathscr{O}_{C'})\xrightarrow{\alpha}\mathscr{B}^{1}\to
0\label{art03-eq1}\\
& 0\to \mathscr{B}^{1}\to
F_{*}(\Omega^{1}_{C'})\xrightarrow{c}\Omega^{1}_{C}\to 0\label{art03-eq2}
\end{align}
ou $c$ est l'op\'eration de Cartier.

Soit $\mathscr{L}\subset \mathscr{B}^{1}$ un $\mathscr{O}_{C}$-module
inversible et $l$ son degr\'e. Si
$\mathscr{E}=\alpha^{-1}(\mathscr{L})$, on a une suite exacte:
\begin{equation}
0\to \mathscr{O}_{C}\to \mathscr{E}\to \mathscr{L}\to
0,\label{art03-eq3} 
\end{equation}
d'ou\pageoriginale un morphisme de $\mathscr{O}_{C}$-alg\`ebres
$$
S(\mathscr{E})\xrightarrow{\beta}F_{*}(\mathscr{O}_{C'})
$$
(on d\'esigne par $S(\mathscr{E})$ l'alg\`ebre sym\`etrique de
$\mathscr{E}$, par $S^{n}(\mathscr{E})$, sa partie homog\`ene de
degr\'e $n$).

\begin{prop}\label{art03-prop1}
\begin{itemize}
\item[\rm(i)] On a $l\leq 2(g-1)/p$, avec \'egalit\'e si et seulement
si $\beta$ est surjectif.

\item[\rm(ii)] Pour qu'il existe un diviseur $D$ sur $C$ avec
$\mathscr{L}\simeq \mathscr{O}_{C}(D)$, il faut et il suffit qu'il
existe $f\in K$ avec $(df)\geq pD$, $df\neq 0$.
\end{itemize}
\end{prop}

\noindent
{\bf D\'emonstration.}~
Les courbes $C$ et $C'$ ayant m\^eme genre, la caract\'eristique
d'Euler-Poincar\'e de $\mathscr{B}^{1}$ est nulle, d'o\`u, par Riemann
Roch:
$$
\deg (\mathscr{B}^{1})+(p-1)(1-g)=0.
$$

Par ailleurs, $\beta(S(\mathscr{E}))$ est isomorphe \`a
$S^{p-1}(\mathscr{E})$, donc $\alpha\circ \beta(S(\mathscr{E}))$ est
un sous-faisceau de $\mathscr{B}^{1}$ qui admet une filtration, a
quotients successifs isomorphes \`a $\mathscr{L}^{\otimes i}$, $1\leq
i\leq (p-1)$. On a donc $lp(p-1)/2\leq (p-1)(g-1)$, soit $l\leq
2(g-1)/p$, avec \'egalit\'e si et seulement si $\beta$ est
surjectif, d'ou (i).

Prouvons (ii). On a
$\mathscr{L}\simeq \mathscr{O}_{C}(D)\Leftrightarrow
H^{0}(C,\mathscr{B}^{1}(-D))\neq 0$. En tensorisant \eqref{art03-eq2}
par $\mathscr{O}_{C}(-D)$, on trouve la suite exacte:
$$
0\to \mathscr{B}^{1}(-D)\to
F_{*}(\Omega^{1}_{C'}(-F^{*}(D))\xrightarrow{c(-D)}\Omega^{1}_{C}(-D)\to 0
$$

Par suite, $H^{0}(C,\mathscr{B}^{1}(-D))=\{df',f'\in K',(df')\geq
F^{*}(D).\}$ Par l'\'el\'evation \`a la puissance $p$, cet ensemble
est en bijection avec
$$
\{df, f\in K,(df)\geq pD\},\quad d'ou(ii).
$$

\begin{corollaire*}
Pour qu'il existe $\mathscr{L}$ dans $\mathscr{B}^{1}$, avec
$\mathscr{L}\simeq \mathscr{O}_{C}(D)~~\deg\break (D)=2(g-1)/p$, il faut et il
suffit qu'il existe $f\in K$, avec $(df)=pD$.
\end{corollaire*}

\begin{exemple*}
Soit $h$ un entier $>0$ et consid\'erons le revetement
d'Artin-Schreier de la droite affine, d'\'equation:
$$
X^{p}-X=T^{hp-1}.
$$
Soit\pageoriginale $C$ la compl\'etion projective de ce revet\^ement,
munie de son point \`a l'infini. La diff\'erente $\mathscr{D}$
est \'egale \`a $hp(p-1)\infty$, la courbe $C$ est de genre $g$ avec
$2(g-1)=p(h(p-1)-2)$ et $(dT)=p(h(p-1)-2)\infty$. On peut donc prendre
$\mathscr{L}=\mathscr{O}_{C}(D)$ avec $D=(h(p-1)-2)\infty$. 
\end{exemple*}

\section{Construction de \texorpdfstring{$(X,\mathscr{I})$}{XI}}\label{art03-sec2}

D\'esormais, on suppose donn\'es $C$ de genre $g>1$ et $\mathscr{L}$
satisfaisant aux conditions \'enonc\'ees dans le corollaire, donc de
degr\'e $1=2(g-1)/p>0$.

Soit $P=P(\mathscr{E})$ le fibr\'e en droites projectives sur $C$,
d\'efini par $\mathscr{E}$. Notons $f:P\to C$ le morphisme structural
et $\mathscr{O}_{P}(1)$ le faisceau inversible relativement ample
canonique. On a $f_{*}(\mathscr{O}_{P}(n))=S^{n}(\mathscr{E})$. Enfin,
le faisceau dualisant relatif est
$\omega_{P/C}=\mathscr{O}_{P}(-2)\otimes \Lambda^{2}(\mathscr{E})=\mathscr{O}_{P}(-2)\otimes \mathscr{L}$.

On a deux diviseurs horizontaux naturels sur $P$. Tout d'abord, le
quotient $\mathscr{L}$ de $\mathscr{E}$, d\'efinit une section de $f$;
soit $E\simeq C$ son image. L'\'el\'ement 1 de $\mathscr{O}_{C}$, vu
comme \'el\'ement de
$H^{0}(C,\mathscr{E})=H^{0}(P,\mathscr{O}_{P}(1))$, donne une section
$s$ de $\mathscr{O}_{P}(1)$, qui s'annule sur $E$ et fournit un
isomorphisme de $\mathscr{O}_{P}(1)$ avec $\mathscr{O}_{P}(E)$. Par
ailleurs, si on tensorise \eqref{art03-eq3} avec le morphisme de
Frob\'enius absolu sur $C$, on obtient la suite exacte:
\begin{equation}
0\to \mathscr{O}_{C}\to \mathscr{E}^{(p)}\to \mathscr{Z}^{\otimes p}\to 0.\label{art03-eq4}
\end{equation}
Comme $\mathscr{E}\subset \mathscr{O}_{C'}$, l'\'el\'evation \`a la
puissance $p$ dans $\mathscr{O}_{C'}$, permet de d\'efinir un
morphisme $\mathscr{O}_{C}$-lin\'eaire surjectif
$\mathscr{E}^{(p)}\to \mathscr{O}_{C}$; son noyau est isomorphe \`a
$\mathscr{L}^{\oplus p}$ et fournit un scindage
de \eqref{art03-eq4}. On a donc une droite canonique
$\mathscr{L}^{\oplus p}$ dans $\mathscr{E}^{(p)}\subset
S^{p}(\mathscr{E})=p_{x}(\mathscr{O}_{P}(p))$. D'ou une section $t\in
H^{0}(P,\mathscr{O}_{p}(p)\otimes \mathscr{L}^{\oplus-p})$; $t$
s'annule sur une courbe de $P$, de degr\'e $p$ sur $C$ qui, compte
tenu du choix de $\mathscr{L}$, est $C$-isomorphe \`a $C'$ (et on la
note $C'$ dans la suite), donc est lisse sur $k$. Enfin on a $C'\cap
E=\phi$. 

{\em Examinons d'abord le cas $p\neq 2$.} On va construire un
rev\^ete\-ment $X$ de $P$, de degr\'e 2, ramifi\'e le long de $C'\cup
E$. Pour cel\`a, on choisit $\mathscr{N}$ inversible sur $C$, tel que
$\mathscr{N}^{\otimes 2}=\mathscr{L}$ (noter que $l$ est pair). Pour
d\'ecrire le morphisme $\pi:X\to P$, on doit se donner un faisceau
inversible $\mathscr{M}$ sur $P$ et un isomorphisme de
$\mathscr{M}^{\otimes 2}$ avec $\mathscr{O}_{P}(-E-C')$;
on\pageoriginale a alors
$\pi_{*}(\mathscr{O}_{X})=\mathscr{O}_{P}\oplus \mathscr{M}$. On prend
$\mathscr{M}=\mathscr{O}_{P}(-(p+1)/2)\otimes N^{\otimes p}$ et le
morphisme de $\mathscr{O}_{P}$ dans
$\mathscr{M}^{\otimes-2}=\mathscr{O}_{P}(p+1)\otimes \mathscr{L}^{\otimes
-p}=\mathscr{O}_{P}(E+C')$ d\'efini par $s\otimes t$. Comme $C'$ et
$E$ sont lisses sur $k$, et disjointes, $X$ est lisse sur $k$. Les
fibres de $g=f\circ \pi$ sont de genre $(p-1)/2$; un calcul local
montre que $X$ possede une ligne de cusps $\widetilde{C}$, au-dessus
de $C'$ ($\widetilde{C}$ isomorphe \`a $C'$). En dehors de
$\widetilde{C}$, $X$ est lisse sur $C$. Enfin $E$ se d\'edouble sur
$X$ en $\pi_{*}(E)=2\widetilde{E}$.

\begin{lemme*}
Il existe $n>0$ tel que $\mathscr{O}_{X}(n\widetilde{E})$ soit
engendr\'e par ses sections. Le syst\`eme lin\'eaire correspondant
contracte une seule courbe: la ligne de cusps $\widetilde{C}$.
\end{lemme*}

Evidemment, la sectionl de $\mathscr{O}_{X}(\widetilde{E})$ engendre
$\mathscr{O}_{X}(\widetilde{E})$ en dehors de $\widetilde{E}$. Donc
$\widetilde{C}$, qui est contenue dans $X-\widetilde{E}$, sera
n\'ecessairement contract\'ee. Par ailleurs on a
$\pi_{*}(\mathscr{O}_{X}(\widetilde{E}))=\mathscr{O}_{P}\oplus \mathscr{M}(E)$
et plus g\'en\'eralement, pour $n\geq 0$, on a
\begin{align*}
\pi_{*}\mathscr{O}_{X}(2n\widetilde{E})
&= \mathscr{O}_{P}(nE)\oplus \mathscr{M}(nE)~~et~\pi_{*}~~\mathscr{O}_{X}((2n+1)\widetilde{E})\\
&= \mathscr{O}_{P}(nE)\oplus \mathscr{M}((n\oplus 1)E).
\end{align*}
En particulier
\begin{align*}
g_{*}(\mathscr{O}_{X}(p\widetilde{E})) &=
p_{*}(\mathscr{O}_{P}((p-1/2)E)\oplus \mathscr{M}((p+1)/2~E))\\
&= p_{*}(\mathscr{O}_{P}((p-1)/2))\oplus \mathscr{N}^{\otimes n}.
\end{align*}
Comme $\mathscr{N}$ est de degr\'e $>0$, $\mathscr{N}^{\otimes pm}$
est engendr\'e par ses sections pour $m\gg 0$. Si alors $\sigma$
engendre $N^{\otimes pm}$ au-dessus d'un ouvert affine $U$ de $C$, il
correspond \`a $\sigma$ une section de
$\mathscr{O}_{X}(pm \ \widetilde{E})$, qui engendre ce faisceau
au-dessus de $U$, en dehors de $\widetilde{C}$, donc sur un ouvert
affine de $X$, d'ou le lemme.

Il r\'esulte du lemme que, si $\mathscr{Q}$ est un faisceau inversible
sur $C$, de degr\'e $>0$, alors
$\mathscr{L}=\mathscr{O}_{X}(\widetilde{E})\otimes g(\mathscr{Q})$ est
ample sur $X$. Il nous reste \`a voir que l'on peut choisir
$\mathscr{Q}$, de degr\'e $>0$, de facon que
$H^{1}(X,\mathscr{I}^{-1})$ soit $\neq 0$. Or on a
$H^{1}(X,\mathscr{O}_{X}(-\widetilde{E})\otimes \mathscr{Q}^{-1})=H^{0}(C,R^{1}g_{*}(\mathscr{O}_{X}(-\widetilde{E}))\otimes \mathscr{Q}^{-1})$;
$R^{1}g_{*}(\mathscr{O}_{X}(-\widetilde{E}))=R^{1}p_{*}(\mathscr{O}_{p}(-E)\oplus \mathscr{M})=R^{1}p_{*}(\mathscr{M})$. Vu\pageoriginale
la dualit\'e de Serre, $R^{1}p_{*}(\mathscr{M})$ est dual de
$p_{*}(\mathscr{M}^{-1}\otimes \omega_{P/C})=p_{*}(\mathscr{O}_{P}(p-3)/2)\otimes
N^{2-p}=S^{(p-3)/2}(\mathscr{E})\otimes N^{2-p}$. Ce dernier a pour
quotient $\mathscr{L}^{\otimes
(p-3)/2}\otimes \mathscr{N}^{\otimes(2-p)}=\mathscr{N}^{-1}$, donc
$R^{1}g_{*}(\mathscr{O}_{X}(-E))\otimes \mathscr{Q}^{-1}$ contient
$\mathscr{N}\otimes \mathscr{Q}^{-1}$. Il suffit donc de prendre
$\mathscr{Q}$ de degr\'e $>0$, tel que
$H^{0}(C,\mathscr{N}\otimes \mathscr{Q}^{-1})\neq 0$, par exemple
$\mathscr{Q}=\mathscr{N}$. 

{\em Dans le cas} $p=2$, on choisit $l=g-1$ multiple de 3 et un
faisceau inversible $\mathscr{N}$ sur $C$ tel que
$\mathscr{N}^{\otimes 3}=\mathscr{L}$. On prend pour $\mathscr{M}$ un
rev\^etement cyclique de degr\'e 3 de $P$, ramifi\'e le long de $E\cup
C'$, d\'efini par le faisceau inversible
$\mathscr{M}=\mathscr{O}_{P}(-1)\otimes \mathscr{N}^{2}$ et le
morphisme
$$
\mathscr{O}_{P}\to \mathscr{M}^{\otimes-3}=\mathscr{O}_{P}(3)\otimes \mathscr{L}^{-2}=\mathscr{O}_{P}(E+C') 
$$
d\'efini par $s\otimes t$. La fin de la d\'emonstration est analogue a
celle du cas $p\neq 2$.

\section{Remarques et questions}\label{art03-sec3}

{\bf 1.}~La surface $X$ que nous avons construite est un rev\^etement
radiciel de degr\'e $p$ d'une surface r\'egl\'ee de base $C$; en
particulier, elle a pour nombres de Bette, $b_{1}=2g$,
$b_{2}=2$. N\'eanmoins, du point de vue de la classification
d'Enriques, Bombi\'eri, Mumford \cite{art03-key1}
et \cite{art03-key4}, elle est de type g\'en\'eral pour $p\geq 5$,
quasi-elliptique (avec $\chi(\mathscr{O}_{X})<0$) pour $p=2$ et 3.

\medskip
\noindent
{\bf 2.}~Comme $E$ est une section de $f$, $\omega_{P/C}(E)|E$ est
trivial, done $\mathscr{O}_{P}(E)|E$  est isomorphe \`a $\mathscr{L}|C$; en
particulier $E^{2}=l>0$. Il en r\'esulte que l'on a
$\widetilde{E}^{2}=l/2$ pour $p\neq 2$ et $\widetilde{E}^{2}=l/3$ pour
$p=2$. Mumford et Spiro ont remarqu\'e que, des que l'on avait une
surface lisse $X$, fibr\'ee sur une courbe $C$, \`a fibres integres de
genre $\ge 1$ munie d'une section $E$ telle que $E^{2}>0$, on pouvait
trouver sur $X$, un faisceau ample $\mathscr{L}$ tel que
$H^{1}(X,\mathscr{L}^{-1})\neq 0$.

\medskip
\noindent
{\bf 3.}~Dans le rev\^etement d'Artin-schreier cit\'e plus haut
prenons, $h=ap-2$ avec $a\geq 1$ si $p\geq 5$, $a\geq 2$ si $p=3$ et
prenons $h=8$ si $p=2$. On peut alors choisir
$\mathscr{N}=\mathscr{O}_{C}(a(p-1/2)-1)p\infty)$ is $p\geq 3$ et
$\mathscr{N}=\mathscr{O}_{C}(2\infty)$ si $p=2$ et prendre pour
faisceau ample
$\mathscr{I}=\mathscr{O}_{X}(\widetilde{E})\otimes \mathscr{N}$. Comme
$\mathscr{N}$ est engendr\'e par ses sections sur $C$ (car image
r\'eciproque sur $C$ d'un faisceau sur la droite projective),
$\mathscr{L}$ est alors engendr\'e\pageoriginale par ses sections en
dehors de $\widetilde{E}$. Peut-on contre-exempler le th\'eor\`eme de
Kodaira avec un faisceau ample, engendr\'e par ses sections, voir
tr\`es ample?

\medskip
\noindent
{\bf 4.}~Soient $X$ une vari\'et\'e propre et lisse sur $k$,
$\mathscr{L}$ un faisceau ample sur $X$ et $\omega$ le faisceau
dualisant. Supposons que $X$ et $\mathscr{L}$ se rel\`event en
caract\'eristique z\'ero. Il r\'esulte alors du th\'eor\`eme de
Kodaira et des propri\'et\'es de sp\'ecialisation de la cohomologie
que pout tout entier $i\geq 0$, on a:
\begin{align*}
\chi^{i}(X,\mathscr{L}\otimes \omega) &= \dim
H^{i}(X,\mathscr{L}\otimes\omega)-\dim
H^{i+1}(X,\mathscr{L}\otimes \omega)+\cdot\\
&\geq 0.
\end{align*}
Ces propri\'et\'es restent-elles varies sans hypoth\`eses de
rel\`evement? Notons qu'un tel r\'esultat; bien que nettement plus
faible que le th\'eor\`eme de Kodaira, suffirait pour \'etendre \`a la
caract\'eristique $p$, le th\'eor\`eme de finitude de
Matsuaka \cite{art03-key3}. 

\begin{thebibliography}{}
\bibitem{art03-key1} E. Bombieri and D. Mumford: Enriques'
classification in char$(p)$ II, in {\em Complex Analysis and Algebraic
Geometry}, Iwanami Shoten--Cambridge Univ. Press, (1977).

\bibitem{art03-key2} K. Kodaira: On a differential-geometric method in
the theory of analytic stacks, {\em Proc. Nat. Acad. Sci. USA} 39
(1953), 1268-73.

\bibitem{art03-key3} T. Matsusaka: Polarized varieties with a given
Hilbert polynomial, {\em Amer. J. Math.,} 94 (1972), 1027-77.

\bibitem{art03-key4} D. Mumford: Enriques' classification of surfaces
in char(p) I, in {\em Global Analysis,} Princeton Univ. Press, (1969),
326-339. 

\bibitem{art03-key5} D. Mumford: Pathologies III, {\em Amer J. Math.,}
89 (1967).

\bibitem{art03-key6} H. Tango: On the behavior of extensions of vector
bundles under the Frobenius map, {\em Nagoya Math. J.,} 48 (1972), 73-89.
\end{thebibliography}

\vfill\eject
~\phantom{a}
\thispagestyle{empty}
