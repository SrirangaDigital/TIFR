\appendix
\chapter*{Appendix}
\addcontentsline{toc}{chapter}{Appendix}

\markboth{Appendix}{Appendix}

\setcounter{pageoriginal}{0}
\section*{Cauchy-Kowalewski theorem for linear differential operators.}\pageoriginale

We shall give a proof of the Cauchy-Kowalewski theorem in the linear case by successive approximations, using holomorphic functions. The proof is taken from H\"ormander, Linear Partial Differential operators, Ch. V, 5.1.

\setcounter{theorem}{0}
\begin{theorem}\label{appendix-thm1}
Let $L$ be a differential operator of the form
$$
\left(\frac{\partial}{\partial x_{n}}\right)^{m}-\sum\limits_{\substack{|\alpha|\leq m\\ \alpha_{n}<m}}\alpha_{\alpha}D^{\alpha}
$$
where $a_{\alpha}$ are analytic functions in a neighbourhood of $x^{0}\in \mathbb{R}^{n}$. Let $f$ be an analytic function in a neighbourhood of $x^{0}$ and $\varphi_{0},\ldots\varphi_{m-1}$ be analytic functions of $(x_{1},\ldots,x_{n-1})$ in a neighbourhood of $(x^{0}_{1},\ldots x^{0}_{n-1})$. Then there exists a unique analytic function $u$ in a neighbourhood of $x^{0}$ such that
\begin{gather*}
Lu=f\qquad\text{and}\\[4pt]
\left(\frac{\partial}{\partial x_{n}}\right)^{k}u\left(x_{1},\ldots,x_{n-1},x^{0}_{n}\right)=\varphi_{k}(x_{f_{1}\ldots x_{n-1}})\quad\text{for}\quad 0\leq k<m.
\end{gather*}
\end{theorem}

\begin{remark*}
\begin{itemize}
\item[\rm(i)] The functions $\varphi_{i}$ are called the initial data and the problem of finding is called the Cauchy problem.

\item[\rm(ii)] Let $\varphi$ be an analytic function of $(x_{1},\ldots,x_{n})$ in a neighbourhood of $x_{0}$ such that $\left(\frac{\partial}{\partial x_{n}}\right)^{k}\varphi(x_{1},\ldots,x_{n-1},x^{0}_{n})=\varphi_{k}(x_{1}\ldots x_{n-1})$.
\end{itemize}
(Exercise: show that such a $\varphi$ exists). Then $u-\varphi$ satisfies the equation $Lu=f+L\varphi$ with initial data $\varphi_{0}=\quad=\varphi_{m-1}=0$. So in order to\pageoriginale solve the Cauchy problem, we can assume without lose of generality that $\varphi_{1}=\ldots=\varphi_{m-1}=0$.
\end{remark*}

By extending the functions $a_{\alpha}$, $f$ to the complex domain as holomorphic functions it is enough to prove the following theorems. We shall use provisionally in the rest of the section the notation
$$
D^{\alpha}=\left(\frac{\partial}{\partial z_{1}}\right)^{\alpha_{1}}\ldots \left(\frac{\partial}{\partial z_{n}}\right)^{\alpha_{n}},\quad\text{where}\quad z_{1}\ldots z_{n}
$$
are complex variables.

\begin{theorem}\label{appendix-thm2}
Let $a_{\alpha}$, $(|\alpha|\leq m,\alpha_{n}<m)$ and $f$ be holomorphic functions in a neighbourhood of $0$ in $\mathbb{C}^{n}$. Then there exists a unique holomorphic function $u$ in a neighbourhood of $0$ such that
\begin{align*}
& \left(\frac{\partial}{\partial z_{n}}\right)^{m}u=\sum\limits_{\substack{\alpha\leq m\\\alpha_{n}<m}}a_{\alpha}D^{\alpha}u+f\quad\text{in the neighbourhood and}\\[4pt]
& \left(\frac{\partial}{\partial z_{n}}\right)^{k}u(z_{1},\ldots,z_{n-1},0)=0\quad\text{for}\quad 0\leq k<m.
\end{align*}
\end{theorem}

The proof will be by successive approximation. To motivate the proof let us first consider the equation $\dfrac{\partial v_{0}}{\partial z_{n}}=g_{0}$ in the polydisc $|z_{i}|<R_{i}$ with $v_{0}(z_{1},\ldots z_{n-1},0)=0$. The solution $u$ is given by
$$
v_{0}(z_{1},\ldots,z_{n})=\int\limits^{z_{n}}_{0}g_{0}(z_{1},\ldots,z_{n-1},\zeta_{n})d\zeta_{n}.
$$

The equation
$$
\left(\dfrac{\partial}{\partial z_{n}}\right)^{m}v=g\quad\text{with}\quad \left(\dfrac{\partial}{\partial z_{n}}\right)^{k}v(z_{1},\ldots,z_{n-1},0)=0, \ 0\leq k<m
$$
can now be solved successively. Write $v=Tg$ and $L_{0}=\sum\limits_{\substack{|\alpha|\leq m\\ \alpha_{n}<m}}a_{\alpha}D^{\alpha}u$. From $\left(\dfrac{\partial}{\partial z_{n}}\right)^{m}u=L_{0}u+f$, we obtain on applying $T$ to both sides, that $u=TL_{0}u+Tf$. If we set $TL_{0}=K$ and $Tf=u_{0}$, we have $u=Ku+u_{0}$ or $(I-K)u=u_{0}$, where $I$ is the identity operator. Thus formally $u-(I-K)^{-1}u_{0}=u_{0}+Ku_{0}+K^{2}u_{0}+\cdots$\pageoriginale (Neumann series). If this series converges in a suitable topology for every $u_{0}$, $u$ will be a solution of the equation. Moreover if $u_{1}$ is another solution and $w=u-u_{1}$ we have $w=Kw$ and since $\sum K^{k}w$ is convergent, $K^{k}w\to 0$ so that $w=0$, which shows the uniqueness. Note moreover that if $u_{k}=K^{k}u_{0}=Ku_{k-1}$, we have $D^{\alpha}u_{k+1}=D^{\alpha}Ku_{k}=D^{\alpha}TL_{0}u_{k}=L_{0}u_{k}$. Thus we could define, $u_{0}$, $u_{1}$ in a suitable polydisc inductively by
\begin{align*}
&\left(\dfrac{\partial}{\partial z_{n}}\right)^{m}u_{0}=f.\\[4pt]
&\qquad \vdots\\[4pt]
&\left(\dfrac{\partial}{\partial z_{n}}\right)^{m}u_{k+1}=L_{0}u_{k}
\end{align*}
with $u_{k}$ satisfying $(\partial /\partial z_{n})^{\ell}u_{k}=0$ for $0\leq \ell <m$. If we show that the series $\sum u_{k}$ converges, the limit would be the solution. For the proof of convergence we need two lemmas which allow an estimation of a holomorphic function in terms of its derivatives and vice-versa.

\setcounter{lemma}{0}
\begin{lemma}\label{appendix-lem1}
Let $v$ be a holomorphic function of one complex variable $z$ in $|z|<R$ with $v(0)=0$ and $\dfrac{du}{dz}(z)\leq C|z|^{a}$ in $|z|<R$ where $C$ and $a$ are non-negative constants. Then we have
$$
|v(z)|\leq \dfrac{C|z|^{a+1}}{a+1}\quad\text{in}\quad |z|<R.
$$
\end{lemma}

\begin{proof}
Since $v(z)=\int\limits^{z}_{0}v'(\zeta)d\zeta$, we have
$$
|v(z)|\leq C\int\limits^{|z|}_{0}|\zeta|^{a}d|\zeta|\leq C|z|^{a+1}/(a+1).
$$
(Note: $\int\limits^{z}_{0}v'(\zeta)d\zeta=\int\limits^{1}_{0}v'(zt)z\;dt$).
\end{proof}

\begin{lemma}\label{appendix-lem2}
Let $v$ be a holomorphic function in $|z|<R$ and $|v(z)|\leq \dfrac{C}{(R-|z|)^{a}}$ in $|z|<R$ with $a>0$. Then we have
$$
|v'(z)|\leq \dfrac{Ce(1+a)}{(R-|z|)^{a+1}}.
$$
\end{lemma}

We\pageoriginale shall prove that there is a constant $C$ such that
\begin{itemize}
\item[(I)]\hfill $\left|\left(\dfrac{\partial}{\partial z_{n}}\right)^{m}u_{k}(z)\right|\leq \dfrac{C^{k+1}|z_{n}|^{k}}{d(z)^{mk+1}}$\hfill\,

\smallskip
for $|z_{j}|<|R_{j}|<1$ where $d(z)=\prod\limits^{n-1}_{j=1}R-|z_{j}|$. This inequality is valid for $k=0$, as $\left(\dfrac{\partial}{\partial z_{n}}\right)^{m}u_{0}=f$, if $C$ is chosen so large that $|f|\leq C$ on $|z_{j}|<R_{j}$. It is sufficient to show that $C$ can be chosen so large that the inequality can be proved inductively. For this we note that successive applications of Lemma \ref{appendix-lem1} ($m$ times with respect to $\dfrac{\partial}{\partial z_{n}}$) shows that for $\ell <m$

\item[(II)] 
\begin{tabbing}
\= $\left(\dfrac{\partial}{\partial z_{n}}\right)^{\ell}u_{k}(z)\leq \dfrac{C^{k+1}|z_{n}|^{k+(m-\ell)}}{d(z)^{(mk+1)}(k+1)\ldots k+(m-\ell)}$\\[7pt]
\> $\left(\text{e.g.~~ }\left|\left(\dfrac{\partial}{\partial z_{n}}\right)^{m-1}u_{k}(z)\right|\leq \dfrac{C^{k+1}|z_{n}|^{k+1}}{d(z)^{(mk+1)}(k+1)}\right)$\qquad (as $|z_{n}|\leq 1$
\end{tabbing}
\end{itemize}

Hence
$$
\left| \left(\dfrac{\partial}{\partial z_{n}}\right)^{\ell}u_{k}(z)\right|\leq \dfrac{C^{k+1}|z_{n}|^{k+1}}{(d(z))^{mk+1}k^{(m-\ell)}}\quad\text{as}\quad \ell<m\quad\text{and}\quad |z_{n}|<1.
$$
Now apply Lemma \ref{appendix-lem2} $\alpha_{1}$ times with respect to $z_{1},\ldots,\alpha_{n-1}$ times with respect to $z_{n-1}$. We will then have for $\alpha$ with $|\alpha|\leq m$, $\ell=\alpha_{n}<m$,
\begin{align*}
|D^{\alpha}u_{k}(z)| &\leq \frac{e^{\alpha_{1}+\cdots+\alpha_{n-1}}c^{k+1}|z_{n}|^{k+1}}{\prod\limits^{n-1}_{j=1}(R_{j}-|z_{j}|)^{mk+1+\alpha_{j}}k^{(m-e)}}\\[4pt]
&\times \left\{\prod\limits^{n-1}_{j=1}(mk+1+1)\ldots (mk+1+\alpha_{j})\right\}\\[4pt]
&\leq \frac{e^{m-\ell}C^{k+1}|z_{n}|^{k+1}(mk+1+m)^{(m-\ell)}}{\prod\limits^{n-1}_{j=1}(R_{j}-|z_{j})^{(mk+1+m)}k^{(m-\ell)}}\\[2pt]
&\qquad (\text{As, } mk+1+m>mk+1+\alpha_{j} \text{and~~ } R_{j}-|z_{j}|<1)\\[4pt]
&\leq \frac{e^{m-\ell}C^{k+1}|z_{n}|^{k+1}}{d(z)^{m(k+1)+1}}\left\{m+\frac{1+m}{k}\right\}^{m-\ell}\\[3pt]
&\leq \dfrac{e^{m}C^{k+1}|z_{n}|^{k+1}}{d(z)^{m(k+1)+1}}(2m+1)^{m}
\end{align*}\pageoriginale
If $\sum |a_{\alpha}|\leq A$ in $P$, we see that
$$
|L_{0}u_{k}|\leq \frac{Ae^{m}C^{k+1}|z_{n}|^{k+1}(2m+1)^{m}}{d(z)^{m(k+1)+1}}
$$
Hence if we choose $C$ so large that
$$
C\geq Ae^{m}e^{m}(2m+1)^{m}
$$
we see that (I) is valid when $k$ is replaced by $(k+1)$. (Note: $D^{\alpha}u_{k+1}=L_{0}u_{k}$). In II, put $\ell=0$ we obtain that
\begin{align*}
|u_{k}(z)| &\leq \frac{C^{k+1}|z_{n}|^{k+m}}{d(z)^{mk+1}\cdot k^{m}}\\[4pt]
&\leq C\cdot \left\{\dfrac{C|z_{n}|}{d(z)}\right\}^{k}
\end{align*}
as \ $|z_{n}|^{m}\leq 1$ \ and \ $1/\mathbb{R} m \leq 1$ \ for \ $k\geq 1$.

Hence the series $\sum u_{k}(z)$ is uniformly convergent in the open neighbourhood $\Omega$ of $0$ where
$$
\dfrac{C|z_{n}|}{d(z)}<\frac{1}{2},
$$
and represents a holomorphic function in $\Omega$. Moreover the series $\sum D^{\alpha}u_{k}(2)$ also converges uniformly on compact subsets of $\Omega$ to $D^{\alpha}u$. Since $\left(\dfrac{\partial}{\partial z_{n}}\right)^{m}u_{k-1}=L_{0}u_{k}$.
$$
\left(\frac{\partial}{\partial z_{n}}\right)^{m}u=\sum\limits^{\infty}_{0}\left(\frac{\partial}{\partial z_{n}}\right)^{m}u_{k}=f+\sum\limits^{\infty}_{1}\left(\frac{\partial}{\partial z_{n}}\right)^{m}u_{k}
$$

\begin{proof}
Let\pageoriginale $\epsilon$ be such that $0<\epsilon <R-|z|=\rho$. Suppose $\zeta$ satisfies $|\zeta-z|<\epsilon$. Then $|v(\zeta)|\leq \dfrac{C}{(R-|\zeta|)^{a}}\leq \dfrac{C}{(\rho-\epsilon)^{a}}$ as
$$
R-|\zeta|\geq (R-|z|)-|z-\zeta|\geq \rho -\epsilon,\quad\text{noting}\quad |\zeta|\leq |\zeta-z|+|z|.
$$
Applying Cauchy's inequality for the closed disc of radius $\epsilon$ around $z$, we have
$$
|v'(z)|=\sup\limits_{|\zeta-z|=\epsilon}|v(z)|/\epsilon \leq \dfrac{C}{\epsilon(\rho-\epsilon)^{a}}.
$$
(Note that $|\zeta-z|\leq \epsilon$ implies
$$
|\zeta|\leq |\zeta-z|+|z|\leq \epsilon + |z|<R-|z|+|z|=R,
$$
in $|z|<R$). Now choose $\epsilon=\rho/a+1$. Then
\begin{align*}
|v'(z)| &\leq \dfrac{C(a+1)}{\rho}\cdot \frac{1}{\rho^{a}}/\left(1-\frac{1}{a+1}\right)^{a}\\[4pt]
&= \frac{C(a+1)}{\rho^{a+1}}\cdot \left(\frac{a+1}{a}\right)^{a}\\[4pt]
&= \frac{C(a+1)}{\rho^{a+1}}\left(1+\frac{1}{a}\right)^{a}\\[4pt]
&\leq \dfrac{C\; e(a+1)}{\rho^{a+1}}.
\end{align*}
\end{proof}

\noindent
{\bf Proof of Theorem \ref{appendix-thm2}.}~
Let $R_{j}$ be numbers with $0<R_{j}<1$ such that $a_{\alpha}$ and $f$ are holomorphic in a neighbourhood of closed polydisc $\overline{P}=\{(Z_{1}\ldots z_{n})|z_{j}|\leq R_{j}\}$. Define inductively holomorphic functions $u_{0},u_{1},\ldots,u_{k},\ldots$ in a neighbourhood $P_{1}$ of $\overline{P}$ such that 
\begin{align*}
& \left(\frac{\partial}{\partial z_{n}}\right)^{m}u_{0}=f\\[2pt]
&\qquad \vdots\\[2pt]
&\left(\frac{\partial}{\partial z_{n}}\right)^{m}u_{k+1}=L_{0}u_{k}.
\end{align*}
with $u_{k}$ satisfying $\left(\dfrac{\partial}{\partial z_{n}}\right)^{\ell}u_{k}(z_{1},\ldots z_{n-1},0)=0$ for $0\leq \ell <m$.
\begin{align*}
&= f+\sum\limits^{\infty}_{k=0}L_{0}u_{k}\\[3pt]
&= f+L_{0}u
\end{align*}\pageoriginale 
as $\sum L_{0}u_{k}$ converges uniformly on compact subsets of $\Omega'$. Since $u_{k}$ satisfies the homogeneous boundary condition, so does $u$.

If $u_{1}$ and $u_{2}$ are two solution $u=u_{1}-u_{2}$ satisfies $Lu=0$. Then $u_{k}=u$ satisfies the recurcive relations, for $k\geq 1$. But $\sum u_{k}$ being convergent, $u=0$.
