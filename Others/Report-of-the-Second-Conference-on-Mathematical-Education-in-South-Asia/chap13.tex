
\chapter{Progress of Mathematical Education in Malaya and Singapore}

\begin{center}
{\em By~} C. J. ELIEZER
\end{center}

At the\pageoriginale previous conference in this series, held here
four years ago, Professor Oppenheim gave a paper entitled ``The
problems which face mathematicians in Singapore and the Federation of
Malaya'', in which he included a general survey of the educational and
cultural background of these countries. This paper was published in
the Report of the Conference, and therefore my task today is rendered
more simple in that I need mostly speak about developments during the
last four years. 

Important political changes have taken place recently in both these
counties.  The Federation of Malaya became an independent country a
little over two years ago, while the island of Singapore obtained last
year a new constitution, giving independence on all internal
matters. In 1959 both countries had their general elections, and
governments with convincing majorities have been elected into office.

The new governments have declared themselves as dedicated to
improvements in education, and have particularly emphasised the need
for more science and mathematics in education, in view of the uses of
these subjects in technological and economic development which is very
necessary in these countries. Under these circumstances support for
improvements in the field of mathematical education may be expected.

Information about the population may be helpful. The Federation of
Malaya has a population of about 6$\frac{1}{2}$ million, of whom
Malays are in the majority, comprising 48\%, these of Chinese descent
38\%, of Indian and Ceylonese descent 11\%. In Singapore, the
population is about 1$\frac{1}{2}$, over 75\% of whom are of Chinese descent.


For many years, efficient systems of primary school education have
prevailed in both countries. Almost all children between the
ages\pageoriginale $6+$ to $11+$ attend school, which is a happy
feature for an Asian country. But the standards of the different types
of schools have varied considerably. There are Malay medium schools,
Chinese medium schools, English medium schools and Tamil medium
schools. In the Federation, where the total school population is a
little over 1.1 million, about 40\% attend Malay medium schools, 36\%
Chinese medium schools, 20\% English medium schools, and 4\% Tamil
medium schools.

Most of the Malay and Tamil medium schools are only primary schools,
where in mathematics little more than the simple rules of arithmetic
is taught, and there is no algebra or geometry. There is provision by
which abler children from the Malay schools may enter the English
medium secondary schools, by means of ``remove'' classes and an
intensive course in English. In the past, quite a number of Malay
children from rural areas left without secondary education. There has
been a growing demand for secondary education, and secondary Malay
medium schools are being provided in increasing numbers.

The Chinese medium schools have had their secondary sections for a
long time, and mathematics was included in the curriculum. Till
recently, mathematics was taught in these schools in `layers' rather
than in `columns', that is, the various branches were taught in
successive years. Thus arithmetic, algebra, geometry, trigonometry,
advanced algebra and co-ordinate geometry were taught in successive
years, and there was often a teacher of algebra or a teacher of
geometry, and not a teacher of mathematics. There was no set syllabus,
the course being determined by the particular text book in use. Text
books used were mostly those written for schools in China. A
consequence of all this was the difference in the standards attained
in mathematics among the Chinese medium schools, and the excessive
separation of the different branches which tends to narrow the outlook
on the range and scope of mathematical methods.

The English medium schools have children of all communities. The
syllabus and text books were geared to the requirements of the
Cambridge\pageoriginale school certificate examination. A popular set
of text books was Durell's General Mathematics series, which had been
specially adapted for schools in Malaya. 

The political changes that have taken place in the last few years have
progressively led to the Education Ministries exercising greater
control over the schools, and giving greater financial support. Some
uniformity between the different types of schools has been aimed
at. In the Federation the Report of a Special Committee on Education
was embodied in an Education Ordinance of 1957, which included among
other things the following two objectives :
\begin{itemize}
\item[(i)] the establishment of one type of national secondary school
  open to all races by competitive selection and with a common
  syllabus, a flexible curriculum permitting the study of all Malayan
  languages and cultures, and with room for diversity in the media of
  instruction~; 

\item[(ii)] the introduction of common content syllabuses and time
  tables for \textit{all} schools.
\end{itemize}

A General Syllabus and Time-Table Committee, with many subcommittees,
have been at work, and a new unified syllabus for mathematics has been
issued by the Federation Government in  1959.

A similar syllabus has come into operation in Singapore. The Director
of Education in Singapore, in his foreword to the mathematics syllabus
has written as follows :

``I should like particularly to stress two points which the authors of
this syllabus have made, the first that mathematics is an
international language, and the second that it is best taught as a
unified subject. Here in Singapore with our multilingual education
system it is a great help to have a subject which knows no racial
barriers. The syllabus rightly makes no distinction between English
mathematics, Chinese mathematics and Malay mathematics ; it will, I
hope, serve to bring together through a common interest the teachers
and pupils of the many different types of schools in Singapore.

So too, with the proposal to unify the teaching of mathematics. For
too many years it has been the practice to divide school
mathematics\pageoriginale into a number of separate compartments
labelled Arithmetic, Algebra, Geometry, Trigonometry, Calculus and so
on, and to study each of them independently of the others. The more
modern view stresses the close relationship not only between the
various branches of mathematics but also between mathematics and other
sciences. This syllabus, then, is concerned to stress the unity rather
than the diversity of mathematical knowledge.

I am glad to acknowledge here the debt which the Ministry owes to the
Committee which drew up this syllabus, and to the Malayan Mathematical
Society which has done so much to arouse interest in the subject and
to raise the standard of teaching in this country... .''

There are three aspects of school mathematics which I would like to
comment upon, before talking about university education.

The first comment is that the 6th form system has now become well
established in the Malayan educational system. Students remain for two
or three years in school after the school certificate course, and read
for the Higher School Certificate. H.S.C. is now the necessary entry
qualification for the university. This will help to raise the standard
of mathematical education in schools, and many more students are doing
what is called `Further Mathematics' in the H.S.C. course. This will
enable university standards to be raised.

My second comment on school mathematics concerns the teaching of
Euclidean geometry. Many children all over South Asia now complete
their schooling without having been introduced to geometry. This is a
great pity, for geometry is a very good feature of school discipline,
a subject where the art of deductive reasoning finds good scope, in
particular it helps to think accurately, and think hard. This value
was recognised by Plato over 2000 years ago when outside the walls of
his Academy was written in bold letter : `He who is ignorant of
geometry let him not enter here'. Yet today, many of our children
still do not learn geometry. My purpose in referring to this is to say
that some concerted action by us all in this region may help to do
something about this.

Before\pageoriginale I leave this comment on school geometry, I wish
to point out one aspect in the teaching of geometry, namely that in
doing almost exclusively the geometry of the plane, the child's
capacity to think kn three dimensions, which is necessary for our
particular world, seems to become considerably reduced. I have heard
it said by professors of education that in the perception of three
dimensional geometrical shapes, and in drawing these in a plane,
students in Asian countries seem badly placed compared to their
European counterparts. I know from my own experience that my capacity
to draw a three dimensional figure in proper perspective is limited,
and from years of reading examination scripts, I note that many
students seem to have this difficulty. It seems desirable that our
school geometrical courses should not forget to include some three
dimensional work, like plans and elevations, surveying and other
simple geometrical drawing. I am not sure how many South Asian
countries have these included and I am glad to see that this is
included in the syllabus in Malaya and Singapore.


My last comment on school mathematics is on the advisability of having
a certain amount of statistics and probability in our courses. We
teach a great deal about exact numbers and equations, and how to
manipulate all these, inside out. In ordinary life we do not always
encounter such exact situations. For example, to describe
arithmetically some daily habits of an individual, we may say he
smokes 15 cigarettes, sleeps 7 hours, and walks 3 miles. But all these
are average figures. A person does not smoke exactly 15 cigarettes
every day---sometimes it is more and sometimes it is less. This
concept of `more or less' is a valuable part of mathematical thinking. 

The extent to which an average can be a reasonable representation also
needs to be considered by the student. One may remember the story of a
man who could not swim, but who crossed a river after being told that
the average depth was only 3 feet. He got drowned for obvious
reasons. I think that an elementary introduction to probability and
statistics is a desirable part of school mathematics.

It is included\pageoriginale in the Malayan syllabus, but absence of teachers able
to tackle this subject may mean slow implementation.

Now I turn to post-school education. There is a polytechnic in
Singapore, a technical college in Kuala Lumpur, an agricultural
college at Serdang, a certain number of rural trade schools, and some
training college for teachers.

At the university level, there are three institutions. Those educated
in Chinese medium schools used to enter universities in China, but
this avenue being closed, a group of Chinese people started the
Nanyang University in Singapore. Its first batch of graduates has come
out this month. The existence of this university will enable many more
students to receive university education in mathematics, and
further, some of its graduates will become teachers in the Chinese
medium schools, and so help to raise the standards of mathematics in
these schools.

From 1949 there has been in Singapore a University of Malaya which had
been built up from two older institutions---Raffles College and
College of Medicine. This University used the English medium, and
catered to students from Singapore as well as from the Federation, and
has received grants from both Governments.

Recently this institution has expanded, by a process similar to cell
division in the biological world. It has divided into two
divisions---one called the University of Malaya in Singapore and the
other the University of Malaya in Kuala Lampur. It is said to be
unique in that it is the one university in the world with two
divisions in two politically separate countries. A common body,
University of Malaya, links the two divisions.

There are Faculties of Arts and Science in both divisions, Faculty of
Medicine only in Singapore, and Faculty of Engineering and Faculty of
Agriculture only in the Kuala Lumpur division. 

A new development has been a provision by which students from Chinese
medium secondary schools are enabled to enter the university after an
intensive course in English. This is being tried out in an
experimental way this year, and if the experiment is successful,
then\pageoriginale we may see how the university in future may be able
to admit students who in the school stage have learnt through any of
the Malayan languages. In this way the university's role in the life
of the nation will be enhanced.

The Department of Mathematics in Singapore has provided pass degree
courses, honours degree courses, and M.A., M.Sc. degrees by research,
but the number of students for honours and Master's degrees has been
low. A new head of department, Professor Pedoe, who would be known to
many here, has taken charge recently. In the Kuala Lumpur division,
the Department commenced only seven months ago, and I am the first
Professor of this department. It has been exciting to share in the
building of a new University. This division expects to have about 1500
students within 3 years. In addition to specialised mathematics
courses the Mathematics Department also teaches mathematics for all
the sciences, and also has separate courses for engineers. We are in
favour of the British tradition whereby applied mathematics goes with
the Mathematics  Departments rather than with physics. The courses
which we have planned will make our honours degrees very similar to
those in British universities. We are hoping to have as much as
possible of abstract mathematics, but at the same time we wish our
graduates to be down-to-earth mathematicians who could be trusted to
cross a crowded street in safety or to be able to determine the roots
of an equation to the fourth or fifth significant figure, if occasion
requires it. We have a course in numerical analysis, which deals with
topics such as interpolation, roots of equations, numerical
integration, and numerical solutions of differential equations. We
consider that this type of work is a necessary supplement to the
abstract teaching that obtains in mathematical degree courses. 

I wish to make one comment on vectors, which have come up for
discussion here on many occasions. One of the first courses we give to
first year students in the university is analytical geometry of three
dimensions, and we do it by using vectors. We apply vectors to
analytical geometry, but we then them much more as an application to
mechanics. Teaching vectors, and then not using them\pageoriginale
straightaway in mechanics, seems a waste of effort. In mechanics, one
requires the concept of addition of forces or velocities by the
parallelogram law or the triangle law, and this in turn is based on
the mode of addition of displacements. And addition of displacements
is dealt with in geometry. For this reason the treatment of analytical
geometry by vectors is a natural introduction to the study of
mechanics.

As I said before we have been planning our new syllabus and I need
hardly say how very much the deliberations of this conference will
help us. On my return, we will modify our original plans in the light
of the discussions that have taken place here. 

I conclude with a brief comment on research. In Malaya, the research
that has been done has been at the initiative of the Professors and
the individual staff members of the University. The Malayan
Mathematical Society has helped to keep up general interest in
mathematical research. One circumstance that has militated against the
development of research is that students on the whole have for various
reasons preferred the civil service, or medical and engineering 
professions or commerce to academic work. There are signs that in the
future some may turn to academic pursuits, and then research may
flourish more spontaneously.

We are hopeful that we would in the near future build up research
activity on an appreciable scale, not in many branches, for that will
be beyond our resources and talents, but in one or two selected
branches of mathematics. For this purpose we also rely on the support,
experience and good wishes of the great international community,
linked together by a common devotion to a great subject, which is
perhaps the oldest in civilization but has also the newest branches of
knowledge---I mean mathematics.

\bigskip
\bigskip

\noindent
{\fontsize{9pt}{11pt}\selectfont
University of Malaya\\
Kuala Lumpur
}\relax

