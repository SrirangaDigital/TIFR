\chapter{Reports of Working Groups}

\begin{center}
{\large\bf 6}
\end{center}
\medskip

\setcounter{pageoriginal}{200}
Several\pageoriginale points arose during the discussion following some of the geometry lectures which were not brought up during the working groups.
\begin{enumerate}
\item Professor Akizuki, in his address, suggested that determinants should be introduced (after matrices) at (the age of) 18, using Grassmann algebra. Professor Cartan pointed out that in Professor Akizuki's method the existence of the exterior product was assumed without proof, and that this made the method rather pointless. Professor Akizuki admitted this, but felt that the manipulative skill acquired by the student in the use of the exterior product outweighed the disadvantages of confusion due to the lack of an existence proof. The majority of the delegates seemed to agree with Professor Cartan. However, in Professor Lichnerowicz's lecture it was suggested that the exterior calculus itself, with complete proofs, should be introduced at 18.

\item Several points arose out of Professor Moise's lecture, which was in fact the only one dealing with geometry at the school stage (i.e. under 16), and also the only one dealing entirely with pure geometry.
\end{enumerate}

\smallskip
\noindent
\textsc{Artin}~: How do you set up similarity between triangles ?

\smallskip
\noindent
\textsc{Moise}~: Using areas, after Legendre.

\smallskip
\noindent
\textsc{Artin}~: The elementary treatment of plane area does not work in space.

With your definition of area it is necessary to show that different triangulations of the same polygon lead to the same area.

\smallskip
\noindent
\textsc{Moise}~: Yes, but we do not intend to prove this. It is not clear with our definition that figures such as the circle have an area, but this is a fundamental difficulty in all definitions.

\smallskip
\noindent
\textsc{Someone}~: The\pageoriginale consistency of your axioms, of course, needs proof.

\smallskip
\noindent
\textsc{Moise}~: Yes.

\smallskip
\noindent
\textsc{Stone}~: What is the connection with Choquet's axioms for plane geometry ? There seems to be some connection.

\smallskip
\noindent
\textsc{Moise}~: I am not sure.
