\chapter{Progress of Mathematical Education in Ceylon}

\begin{center}
{\em By}~ D. G. SUGATHADASA
\end{center}

\setcounter{pageoriginal}{164}
I\pageoriginale presume a short account of the geographical and cultural background of Ceylon would help in appreciating the problems in mathematical education the Island has to face today.

Ceylon is about 25,000 square miles in area but has a population of over 9 million. The economy of the Island depends on its agricultural exports of tea, rubber, coconut, cocoa etc., but the greater part of its staple diet, rice, and practically the whole of its consumer goods such as manufactured articles have to be imported. The Government has tried to diversify the country's economy by starting industries.

Though the country gained political independence about 13 years ago, Ceylon had complete control of her education system since 1931. Three media of instruction, English, Sinhalese and Tamil were used in the system of education which was handed over to the Ceylonese by British colonial administrators. The schools where the medium of instruction was English were run by denominational bodies aided by Government grants, except for Royal College which was run entirely as a Government institution. Those who could afford the high fees charged, attended such schools. The so called vernacular schools, where the medium of instruction was Sinhalese or Tamil catered for nearly 95 per cent of the school going population and no fees were charged in these schools. While nearly 50 per cent of these schools were controlled by Government, the rest were conducted by denominational bodies aided by Government.

Thus the school going population grew in two completely different educational environments. The difference between the vernacular schools and the English schools was not merely one of medium of instruction, for example science and mathematics was found only\pageoriginale in the curriculum of the English schools. The portals of higher education were not open to those who passed out of the vernacular schools, as the medium of instruction in the institutions of higher learning was English. The highest vocation to which the products of vernacular schools could look forward to was that of teaching in similar schools. The official language of administration at this time was English.

To those who passed out of the English schools, opportunities for higher education were available at the Ceylon University College which was established in 1920. This College prepared students for the General and Special degrees of the London University in Science, Arts, Mathematics, Agriculture and Engineering. In addition to the Ceylon University College there were the Ceylon Law College, the Medical College and the Technical College which afforded opportunities for professional training. Fees were charged in these institutions and they were only open to those who were educated in the English medium.

One must say that though there may have been imperfections in this educational system; this system was able to produce educated personnel capable of taking over the government from the British at the time of grant of independence and manning all posts, legislative, technical, administrative etc. But it must be remembered that this personnel was drawn from the English educated 5 per cent, and hence the country suffered, in that, if the net had been cast wider more able individuals could have been secured for carrying on the government of the country.

This system with a few alterations was continued by the Ceylonese administration till a Select Committee on education, appointed in 1939 and presided over by the then Minister of Education, issued its report in 1943. On this report the Legislature decided, among other things, that the medium of instruction should ultimately be the mother tongue, and education should be free from Kindergarten up to and including the University.

It\pageoriginale might be noted in passing that Ceylon is one of the first countries in the world to have a completely free education system. Not only are no fees levied but books are supplied to needy pupils. Bursaries and scholarships are given to deserving poor students for maintenance in university hostels. The percentage of literacy in Ceylon is very high and its system of primary education fairly well established. It is reported that the target set for 1980 for the development of primary education in South Asian counties by the recent UNESCO conference held in Karachi, is that which exists in Ceylon in 1960.

To improve secondary education the Government established 54 Central schools distributed equitably among all the electorates in the Island and every attempt was made to proved them with all facilities which the Government Royal College in Colombo enjoyed. All Government schools teach subjects like geography, history etc. all the way up to the school leaving level. A few teach these subjects to the G.C.E. Advanced Level, but the facilities for science, mathematics, woodwork and metalwork exist in comparatively few Government schools. However secondary schools run by denominational bodies, though few in number, yet teach science and mathematics.

A determined effort is now being made to extend these facilities to all schools, and to this end two existing training colleges out of the 22 training colleges in the Island have been asked to train teachers only for the teaching of mathematics, science, handicrafts and English as a second language. The Maharagama Training College trains such teachers through the medium of Sinhalese and Palaly Training College through the medium of Tamil.

The Maharagama Training College for teachers, of which I am in charge, was established as far back as 1903. It used to train teachers for the English medium schools prior to 1953. In 1953 in addition to the courses in social studies to which the College had devoted its attention till then, courses of training for teachers in science, arts, and crafts and commerce were organised. In 1954 a special course of training for mathematics teachers was also instituted.

The\pageoriginale college provides two years training courses for these teachers. The total number in training at this college during 1958 was 508 of which 168 were being trained to teach mathematics. The mathematics teachers who pass out from this training college would teach these subjects in the post primary classes in schools, up to the first year university entrance level. The work in the years of the university entrance class in schools is undertaken by graduates in mathematics.

The Ceylon University College was founded in 1920 in Colombo city and students prepared for London degree examinations. It was intended to convert this into a university in a few years but there was long drawn out controversy as to where the university should be located and whether the university was to be a residential university like Oxford or Cambridge or a non-residential university like London. Eventually it was decided to have it in a residential from at Peradeniya near Kandy, but the war intervened and the new buildings were considerably delayed.

In the meantime, in 1948 the Ceylon University College was converted into a University of Ceylon granting its own degrees, and also including within it the Old Medical College as the Faculty of Medicine.

In 1952 part of the University was shifted to the new buildings at Peradeniya (namely faculties of Arts, Law, Oriental Languages, Agricultural and Veterinary sciences) while the remaining faculties of Science, Medicine and Engineering are still in Colombo waiting to shift to new building at Peradeniya when these are ready. There are about 3000 students in the University and about 500 graduates every year.

The standards at the University have remained as close as possible to those of London University. The Mathematics Department has the following types of courses. Students enter the University after passing the Higher School Certificate at school, an examination of the same standard as the London G.C.E. (Advanced Level). About 200 students do mathematics in the first year. This includes about 60 engineering students. The students are divided into two groups for lectures\pageoriginale (so that each lecturer gives his lectures twice). After the first year, students divide into general students, doing 3 subjects and taking 2 more years to graduate, or special students taking 3 more years to graduate in one subject. There are about 100 students in each of the second and third year general courses doing mathematics, and in the special courses there are about 10 each year i.e. 30 special students at any one time. The University provides further 2 year courses leading to the M.A. and M.Sc. degrees at the end of which the students are supposed to be ready for research, but in practice a generous provision of scholarships has enabled the abler students to go to British universities for their research. While teaching standards have remained high I believe I am correct in saying that there has not been much research, perhaps due to the fact that work has to be done in the cramped atmosphere of Colombo owing to the delay in the shift to Peradeniya.

There has been a training scheme for the staff of the University whereby good students who complete the Master's degree are appointed as Assistant Lectures and sent abroad for 3 years on research courses. If they do well, they are confirmed in their appointments. They sign a bond to serve the University for 5 years on their return. This scheme has helped to raise an adequate teaching staff. In 1958 there were altogether 11 staff members in the Mathematics Department of whom 7 were old Cambridge men. There have also been visiting professors from the United States under the Fulbright Scheme, where a professor comes for one year and lectures in his speciality.

Graduates from the former Ceylon University College and the present University now man administrative posts in Government Service and mercantile firms. There are indications that there will not be sufficiency of such soft collar jobs to go round to all those who pass out from these institutions. One may therefore expect more graduates to turn to academic work in the future. This would not only be to the advantage of mathematics teaching in schools, but would put research on a firmer footing at the University.

Recently\pageoriginale a University Commission has reviewed the subjects of universities. In the meantime the demand for university education has been high and admission to the one university has been on the results of competitive examination.

In 1959 two old established seats of oriental learning, the Vidyodaya Pirivena and the Vidyalankara Pirivena were given the status of universities. It is too early to say anything about these institutions but there is no doubt that they will be instrumental in giving oriental culture and learning its due place.

Mention must be made of the existence of the Technical College Department which is now in charge of technical education.

The facilities for the teaching of mathematics in Ceylon are gradually improving. Where there were no text books in mathematics available in Sinhalese and Tamil a few years back, there are a considerable number available now, some of which are translations of good English text books, and others specially written to suit Ceylon conditions. But still the number of teachers available for the teaching of mathematics in schools is woefully inadequate. Mathematics teachers are found in less than 1000 of the 7415 Government and Assisted schools in the Island. The output of mathematics teachers from Maharagama and Palaly Training Colleges and from the University of Ceylon being less than 200 per year it will be impossible to man the remaining schools with at least a mathematics teacher each within a reasonable period of time. Most of the rural schools teach only arithmetic. Hence the students who are studying the subject at the University come from a restricted group. If the net had been cast wider abler students would have been roped in for mathematics, and the study of the subject both in schools and the University would have benefited.

\bigskip
\medskip

{\fontsize{9pt}{11pt}\selectfont
Government Training College

Maharagama
}\relax

