
\chapter{Second South Asian Conference on Mathematical Education}

\begin{center}
\textbf{BOMBAY, JANUARY 20-27, 1960}\\
\medskip

\textbf{PRESIDENTIAL ADDRESS}

\medskip
{\em By~} K. CHANDRASEKHARAN
\end{center}


Four\pageoriginale years ago we met together in conference to discuss the problems
of mathematical education in South Asia. It was the first such
conference ever to be held, and was symbolic of the scientific
aspirations of this part of the world. Although it dealt with problems
peculiar to this area, it was truly international in its membership.

One of the accomplishments of that conference was a Resolution which
set out the purposes of mathematical education at various stages, and
the subjects to be taught at each stage. The mandate behind that
Resolution consists largely in the good sense that it embodies, and in
the fact that the accumulated experience of several eminent
mathematicians went into its formulation. And that is perhaps as it
should be. No mathematical conference however important, no
mathematical society however old, should seek to impose or administer
any particular mathematical programme. It is up to each centre of
mathematical studies to adapt our Resolution to its own needs. I can
however, say this, from personal knowledge, that at least two
universities in this country and one or two abroad have been
influenced by the deliberations of the First South Asian
Conference. It certainly clarified the thinking of many mathematicians
who, though relatively young, are charged with the responsibility for
framing university curricula.

At this conference over which I have the honour to preside, it will be
our task to go more deeply into the technical content of courses of
study, and to examine ideas and concepts that need emphasis. Our
sessions this time will necessarily be more technical that\pageoriginale last time,
for we have to deal with the substance of mathematics, not merely the
syllabus. As far as I know, there is no precedent for this type of
conference in India, either in mathematics or in any other subject.

While mathematicians are inherently no less pugnacious than others,
they enjoy a reputation for hanging together. They appear to act like
the practitioners of a private cult. One has, however, only to observe
any two of them discuss the way they would teach a subject, to realize
how mistaken such an impression can be. At this conference which is
attended by distinguished mathematicians of many lands---analysts,
algebraists, geometers and topologists---we certainly can expect to be
confronted with a multiplicity of perspectives. and sometimes,
perhaps, to be fogged by them. But a clash of attitudes and opinions,
if it does materialize at this conference, should present an
additional opportunity to each of us to rethink our own attitude and
perhaps to reorient it. For, as we all know, the good teacher is one
who is always learning, and never quite learned.

The first South Asian conference made several concrete suggestions for
encouraging and improving mathematical studies within the general
framework of education as a whole. As far as India is concerned, the
financial outlay on mathematics has definitely increased. The past
four years have seen great enthusiasm for mathematics. If enthusiasm
alone could achieve success, we should be at the top of the world by
now. But mathematical research is a difficult business, and enthusiasm
though necessary is not sufficient.

The growth of the managerial class among scientists is perhaps one of
the most striking phenomena since World War II. In countries where
science has made great advances, this is perhaps inevitable. But in
developing countries like India, where the full impact of the
scientific revolution is yet to be felt, it is disconcerting to see
scientists, including mathematicians, trying to reap what they have
not sown. The primacy of men over machines, of quality over quantity,\pageoriginale
of honest work over cheap publicity, is yet to be appreciated and established.

Nevertheless the research accomplishments of some of the younger
Indian mathematicians are certainly encouraging. It is my hope that
their number will grow in the next ten years, and that the
mathematical culture which they have acquired will spread out and
flourish. Our urgent aim should be to produce such men of mathematics,
and to create conditions favouring their emergence. In this task we
need all the help we can get from our colleagues abroad. Let us, for
our part, make sure that we deserve it.

I am happy to note that mathematics in Malaya, Singapore, and Thailand
has been considerably strengthened in the past four years. I have no
doubt that the representatives of Pakistan and Ceylon will also have
interesting developments to report.

Those of you who attended the First South Asian Conference will recall
our resolution to set up a Committee for Mathematics in South Asia
with representatives of the Governments of countries in this region,
the Secretary of the International Mathematical Union, and an expert
from abroad (Professor M. H. Stone), as  members. I am glad to report
that such a Committee came into existence about three years ago, and
has been concerned with an elaborate plan to promote mathematical
studies in this region. It has been instrumental in maintaining closer
contacts in this region than ever before, and in giving indirect, but
very necessary, support for the organization of this conference. To
all the members of the Committee present here I extend a warm welcome.

I should like to emphasize that the South Asian Conferences which we
have organized are not intended to split the international community
of mathematicians, or to set up narrow regional walls. They are meant
to focus international attention on the problems that confront a
crucial area of the world.

Nor are they intended to glorify mathematics as the expense of
everything else. Mathematical culture is part of human culture.
The\pageoriginale vitality of any civilization depends on the relentless pursuit of
excellence in all the sciences, the humanities, and the arts. The
future of the peoples of South Asia will ultimately depend on the
quality of their intellectual achievements. We want that quality to be
high. and to achieve it in quantity. We need to make a total and
dedicated effort to go forward in every field of knowledge. We need to
create the social and educational environment which is conducive to
the growth of creative scholarship. May this conference help in the
building of such an environment in the  field of mathematics, which is
basic to the maintenance of high standards in science as a whole.

\bigskip

\bigskip

{\fontsize{9pt}{11pt}\selectfont
Tata Institute of Fundamental Research

Bombay}\relax
