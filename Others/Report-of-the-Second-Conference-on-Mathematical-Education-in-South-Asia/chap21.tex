\chapter{Reports of Working Groups}

\begin{center}
{\large\bf 4}
\end{center}
\medskip

\setcounter{pageoriginal}{184}
\noindent
\textsc{Chandrasekharan}~: Taking\pageoriginale age etc. into consideration it appears that the amount of rigour in a calculus course anywhere is fairly constant. It is important to introduce the exponential function at the first year university stage. Which way would you introduce exponential and logarithmic functions, via the exponential series or $\log x = \int \dfrac{1}{x}dx$ ? Also, is it a good thing to exclude $\epsilon$, $\delta$ ?

\smallskip
\noindent
\textsc{Artin}~: It only affects the semantic level. No language in which limits are defined really needs $\epsilon$, $\delta$. If one proceeds formally one gets involved with quantifiers.

\smallskip
\noindent
\textsc{Fenchel}~: In Denmark $\epsilon$, $\delta$ are introduced, in a calculus course done at 16-17 as rigorously as possible, roughly along the lines suggested by Professor Stone.

\smallskip
\noindent
\textsc{Chandrasekharan}~: What is the place of indeterminate forms ?

\smallskip
\noindent
\textsc{Artin}~: Out !

\smallskip
\noindent
\textsc{Stone}~: There are times when you do want de l'Hospital's rule. Also it gives you a natural method for deriving Taylor's theorem with remainder with no trouble at all. Whereas most proofs of Taylor's theorem are artificial.

\smallskip
\noindent
\textsc{Chandrasekharan}~: Does it have the same status as criteria of convergence ?

\smallskip
\noindent
\textsc{Stone}~: No great intrinsic importance, just for technical reasons.

\smallskip
\noindent
\textsc{Chandrasekharan}~: Is the teaching of dynamics connected with the calculus in Denmark ?

\smallskip
\noindent
\textsc{Fenchel}~: No, in the schools there is inadequate co-operation between mathematics and physics.

\smallskip
\noindent
\textsc{Akizuki}~: In\pageoriginale Japan calculus is introduced at 16 or 15 with the differentiation of polynomials. Earlier there is some talk of rate of change. At 17 the derivative is introduced as $\lim\limits_{x\to a} \dfrac{f(x)-f(a)}{x-a}$, when this exists.

\smallskip
\noindent
\textsc{Chandrasekharan}~: Do you explain what `tends to' means ?

\smallskip
\noindent
\textsc{Akizuki}~: Only intuitively. Later the differentiation of trigonometric functions and the notion of the integral are introduced. At the beginning of the university course derivatives are introduced by means of $\epsilon$, $\delta$.

\smallskip
\noindent
\textsc{Gunjikar}~: A first course should depend on intuition. In India calculus was originally taught from Edwards; there was too much intuition and hence there was a violent reaction. But the vague words such as `tend to' can be made rigorous as Professor Artin pointed out. The objection to using $\epsilon$, $\delta$ is that the subject gets too algebraic. A boy may get lost in manipulating $\epsilon$, $\delta$ algebraically and the beauty of the subject may be lost, especially for physicists and engineers. It is like teaching a child the rules of traffic. The same thing applies to integration. The definite integral is of great use for formulation but not for calculation. So the definite integral is brought in through the notion of areas. Finally a small amount should be taught about differential equations. Integration should be used (introduced !) as a method for solving differential equations.

\smallskip
\noindent
\textsc{Narlikar}~: Professor Artin may have been misinterpreted. A bright student may be puzzled by the fact that $\dfrac{d}{dx}x^{n}=nx^{n-1}$ is proved only for positive integers, but later intuitively used for arbitrary real $n$. There is no occasion for the student to get a rigorous proof. He might play havoc in engineering with half-baked ideas.

\smallskip
\noindent
\textsc{Newman}~: It is a good thing that the man experimentally finds out for himself that the rule can be extended to arbitrary $n$. Everybody need not be taught everything.

\smallskip
\noindent
\textsc{Chandrasekharan}~: There\pageoriginale will be difficulty for $x^{x}$.

\smallskip
\noindent
\textsc{Newman}~: That the student will soon find out.

\smallskip
\noindent
\textsc{Artin}~: I would never give a course based completely on intuition. Only half the course is based on intuition, the other half is rigorous. Regarding the definition of $x^{n}$, at the stage it can only be done for rational $n$ any way.

\smallskip
\noindent
\textsc{Moise}~: There is a false dichotomy here between rigorous and intuitive calculus. A body of mathematics starts as a research paper. Then it is polished and simplified, then finds its way into text-books, gradually being further simplified. Pedagogues have had a good go at manipulative calculus, but not yet at the rigorous side. There are endless manipulative exercises in Granville and Smith but proper exercise material has not yet been devised for the rigorous calculus.

\smallskip
\noindent
\textsc{Racine}~: In many cases the derivative is introduced in a purely mechanical way, e.g. in one university $\dfrac{d}{dx}(\cos ax)$ was in the syllabus but $\dfrac{d}{dx}(\cos (ax+b))$ was not.

\smallskip
\noindent
\textsc{Chandrasekharan}~: It is true that in some Indian universities dynamics is taught before calculus ?

\smallskip
\noindent
\textsc{Someone}~: Yes.

\smallskip
\noindent
\textsc{Lichnerowicz}~: The formal definition of a concept should be constructed via intuition from a great many examples, and only later using $\epsilon$, $\delta$.

\smallskip
\noindent
\textsc{Artin}~: The real difficulty lies in that students always want to find the best $\delta$. If it were only manipulation of inequalities that causes difficulty, putting in a chapter on inequalities would be a remedy to facilitate $\epsilon$, $\delta$ work, but it is not so.

\smallskip
\noindent
\textsc{Gunjikar}~: What about the exponential function ?

\smallskip
\noindent
\textsc{Fenchel}~: In Danish schools it is introduced via $\log x=\int \dfrac{1}{x}dx$; it can also be done via $\lim\limits_{n\to \infty}\left(1+\dfrac{x}{n}\right)^{n}$ using the monotone sequence theorem.

\smallskip
\noindent
\textsc{Chandrasekharan}~: Some\pageoriginale universities in India teach Fourier-Laplace transforms to students of 18 or 19, particularly to engineering students. This is done in Bombay, (where perhaps engineering students are expected to know more mathematics than mathematicians !). How should this be done ? Do you stick to functions which vanish outside a bounded interval ? How are transforms and their inverses introduced ? Do you manipulate entirely formally ?

What about probability ? Can a course be given without any knowledge of Lebesgue integration ? There is, of course, Feller's book.

\smallskip
\noindent
\textsc{Krishnan}~: Mathematicians should just have nothing to do with a course on probability which excludes the concept of measure. Most of the teachers of physics and statistics themselves don't know it, and are not aware of the difficulties. They have only vague ideas and should learn mathematics first.

\smallskip
\noindent
\textsc{Oke}~: This is a dangerous suggestion.

\smallskip
\noindent
\textsc{Chandrasekharan}~: I agree.

\smallskip
\noindent
\textsc{Newman}~: How much time do physicists in India spend learning mathematics ?

\smallskip
\noindent
\textsc{Oke}~: In Bombay there is a common course till 18. Then one subsidiary year with 12 ($\tfrac{3}{4}$ hour) periods per week with 25 to 30 working weeks for 4 papers, algebra and geometry, calculus and differential equations (2 papers), and vector methods. There is no more mathematics after that, or at least the necessary mathematics is taught by physicists.

\smallskip
\noindent
\textsc{Stone}~: We shall soon be forced to introduce more mathematics for physicists. Look at the confusion in the physicists' writing about field theory. Some of the most active physicists in the U.S.A. are preparing a new method in elementary teaching, which avoids as much mathematics as possible. The physicists in 48 hours agreed on a syllabus for a course for the last two years in secondary schools. (Mathematicians were asked to do the same but said they would take longer !) However there are many physicists, even experimental physicists,\pageoriginale who disagree with the syllabus. An Australian experimental physicist suggested that we need to teach more mathematics to modern physicists, both in mathematics and physics classes.

\smallskip
\noindent
\textsc{Chandrasekharan}~: In Bombay we are trying to have some co-operation between mathematics and physics.

\smallskip
\noindent
\textsc{Fenchel}~: Some years ago we had negotiations with the physicists. We reduced physics and increased mathematics in the first year at the university. Roughly, it works. We tried to solve the corresponding problem for statisticians. This is much harder and not quite satisfactory. The mathematicians teach vectors very early so as to enable the student to use them freely in physics.

\smallskip
\noindent
\textsc{Akizuki}~: Americans and Europeans have a long history of science. Japan has a very short history, but we have gone at a good pace. In Japan physics and mathematics are quite independent. There are very few people who work in both. Recently there have been some good physicists who do good mathematics e.g. Kato. We are starting an institute of mathematical sciences, {\em not} mathematics. We have no Euclidean geometry or Newtonian mechanics in schools in Japan, so what is considered suitable in the West is not necessarily so in Japan. I want to discuss what is best for Asian education.

\smallskip
\noindent
\textsc{Newman}~: Physicists do not form a homogeneous body. Theoretical and experimental physicists are very different. We must differentiate between them, provide alternative courses for them; this is done in Manchester. It does not discourage students.

\smallskip
\noindent
\textsc{Chandrasekharan}~: In India, for some reason or other, physics and mathematics are not in water-tight compartments, but have approached each other.

\smallskip
\noindent
\textsc{Narlikar}~: A young man of 18 or 19 to-day faces the same kind of problem in physics as he does in mathematics. 20 or 30 years ago the mathematics required by physics was much less. The young man of 20 to-day must be better equipped mathematically than the young man of 20 years ago. What shall we teach at 16 ? Some kind of change is necessary.

\smallskip
\noindent
\textsc{Stone}~: Note\pageoriginale that the modern experimental physicist does hardly anything which does not require theory. So even the experimental physicist does require mathematics.

\smallskip
\noindent
\textsc{Nagappa}~: Mathematics taught in physics is of a much higher order than that for engineers. The engineers know a little more than what the economist knows. Who should teach mathematics to these people ? These people feel that mathematicians do not stress the subject in question.

\smallskip
\noindent
\textsc{Chandrasekharan}~: We had better concentrate on what and when to teach, not on who teaches it.

\smallskip
\noindent
\textsc{Lichnerowicz}~: Much of physics has no mathematical interpretation. The mathematics of physicists is anachronistic, physicists do not know any decent mathematics.

\smallskip
\noindent
\textsc{Chandrasekharan}~: What is the place, if any, of pure geometry, in particular pure projective geometry ? This was the most interesting subject in my day.

\smallskip
\noindent
\textsc{Fenchel}~: Not long ago it was considered essential. But it was removed 20 years ago in favour of linear algebra. Now there are only advanced optional courses.

\smallskip
\noindent
\textsc{Chandrasekharan}~: It is among the most interesting parts of geometry. Professor Siegel asked me whether the question of pure geometry was raised at all at this conference. People to-day like more and more abstraction and more and more algebra. But in India it still stays.

\smallskip
\noindent
\textsc{Gupta}~: In Punjab it is out.

\smallskip
\noindent
\textsc{Racine}~: In Madras decreased but retained.

\smallskip
\noindent
\textsc{Narlikar}~: In Banaras it is dropped.

\smallskip
\noindent
\textsc{Oke}~: Optional in Bombay.

\smallskip
\noindent
\textsc{Chandrasekharan}~: Is there any harm in studying this ?

\smallskip
\noindent
\textsc{Artin}~: Loss of time. Mathematics is too big.

\smallskip
\noindent
\textsc{Krull}~: I\pageoriginale agree with Professor Fenchel. The questions are treated very soberly but late. Students enjoy it. There is no fixed way of treatment or subject matter. Future teachers should attend such a course.

\smallskip
\noindent
\textsc{Artin}~: Many think that European countries are in an ideal state. But German students have two minor subjects, e.g. physics and chemistry, which do a tremendous amount of damage. It is impossible to try and decrease the amount of material in these.

\smallskip
\noindent
\textsc{Chandrasekharan}~: ``General education'' will perhaps tend to disrupt Indian education, reducing all our students to the worst level of American undergraduates. Professor Stone has warned us against this.


