
\chapter{Topology in the First Degree Course}

\begin{center}
{\em By~} M. H. A. NEWMAN
\end{center}

\setcounter{pageoriginal}{94}
The\pageoriginale expression ``first degree course'' in the title of my talk is
intended as a short description of that part of the univeristy
mathematics course which precedes the final specialization leading
directely to research work. It is the mathematician's general
mathemat1ical education, and so my theme is essentially the problem of
introducing topology to non-topologists.

Three well-marked divisions of topology can be distinguished today :
\textit{general topology} dealing with properties of spaces arising
directly out of the axioms of the subject ; \textit{algebraic
  topology}, which arose out of Poincar\'e's homology theory of
  polyhedra but is not a highly abstract axiomatic theory, and what
  may fitly be called \textit{geometric topology}, which by confining
  itself mainly to the study of \textit{manifolds}, that is spaces
  locally resembling Euclidean space, is able to attack the difficult
  problems connected with homeomorphism and isotopy.

Are any of these parts of topology suitable for treatment in the
course leading to the first degree ? About the suitability of general
topology there can be little doubt. It is indeed a highly abstract
axiomatic theory, but the students of today are not, it seems, afraid
of abstractions. They breathe this rarified air easily and naturally,
provided that they have been supplied beforehand with plenty of
materials for making examples to bring the axioms to life. The
elementary topological properties of linear and plane sets of points
used in the classical theories of real and complex functions, which
will naturally have been taught in an earlier year, give ample
material for motivating the axioms of general topology and the
definitions of such basic properties as compactness and
connectedness. The subject is made easier to teach by the fact that
the various topics run parallel, rather than in series : the lecturer
can shape his course according to his taste. The theory of convergence
of general\pageoriginale topological spaces is now part of the basic
equipment of the topologist. Whether it is better to use the filters
of H. Cartan or Moore-Smith convergence I will not venture to say :
but in one form or another it has strong claims to inclusion. Some
examples also, of the theorems, so characteristic of modern axiomatic
mathematics, that make essential use of the axiom of choice should be
included, such as the Tychonov product theorem. The deeper plunge into
pure set theory which this involves is probably best deferred until
just before it is needed. 

The proviso that the material for the construction of examples should
be firmly planted in the students' minds well before the fully
axiomatic treatment begins is generally admitted in principle, but in
practice there is a temptation, in this as in other parts of
mathematics, to jump a stage, by presenting pieces of elementary work
from the outset in the framework of the general theory into which they
will ultimately be absorbed. Why waste time, the lecturer may think,
giving special proofs of theorems about sets of points on a line if
they are special cases of theorems of general topology which will have
to be proved afresh later ? Why not pursue a ``logical order''? First
the general notion of a topological space, with its  simplest
properties; then metric spaces ; and finally, with just a turn of the
wrist, the properties of closed and open sets on a line appear. Such
methods give the lectures a smooth and happy passage, but the effect
on the pupils, I believe to be bad. It is not possible for them to get
a hold on axiomatically defined ideas  unless the material for
examples is not only plentiful, but also so familiar that the student
is not in the unhappy position of struggling to understand
simultaneously the example itself and the theory it is intended to
illuminate, \textit{obscurum per obscurius.} It is for this reason
that the pause for breath between the steps of the mathematical
ladder, from semester to semester or year to year, is so
important. Nor is there anything illogical in proving a theorem
twice. It is indeed the essence of the axiomatic method that we may
consider a restricted a theory as we please, and then later show that
a model of it can be made in some more comprehensive axiomatic system.

These\pageoriginale observations of course apply to other theories
besides general topology, and in particular to \textit{algebraic
  topology} based on axioms of its own. I therefore do not think that
the time has yet come when this part of topology can be introduced to
students at the level I am considering, since they will not have had
time to assimilate the simple geometrical facts needed to exemplify
the algebraic constructions. 

\textit{Geometric topology}, to which I now turn, may be understood at
this level in a somewhat broader sense than I have given it above, to
include also the homology theory of finite polyhedra so treated that
chains and cycles have a simple geometrical interpretation. The
subject can then be a particularly attractive part of the first degree
course, from the freshness of its material and its combination of
algebraic and intuitively geometrical arguments. Yet the teaching of
this piece of homology theory presents a difficult problem to the
lecturer. How can the student be brought to the stage where examples
of genuine intrinsic interest can be handled before he has become
quite discouraged by the lengthy preparatory work ? The trouble arises
because the usual method of defining the connectivities and torsions,
based on a division of the space into simplexes, is quite remove from
any practical method of computing them. To calculate the Betti number
$p_1$ of a torus from this definition, one has to divide it into a
large number of triangles and then prove that all cycles on this
complex are homologous to combinations of the cycles along the sides
of the fundamental region, and that these cycles satisfy no
relation. (This piece of work is actually carried out in Seifert and
Threlfall's book). In three dimensions such procedures become
impossible to carry out even in the simplest cases.

All practical methods of computing homology groups depend ultimately
on specifying the given space in terms of simpler ones. The most
familiar such specifications are, first by means of the
\textit{fundamental region}, i.e. identification of points of the
boundary of an $n$-cell ; secondly the division of the space into a
small number of simpler pieces which need not be $n$-cells, with the
use of the Mayer-Vietoris theorem, or some similar theorems ;
thirdly, some sort\pageoriginale of \textit{fibering}. The method of
the fundamental region can indeed be brought into contact with
simplicial theory by means of the theory of ``blocks'', which are
pieces of the original simplicial complex that cab be treated as cells
of a complex of a more general kind. This process is always found
artificial and tiresome by students, and by the lecturer too if the
truth be told. It is obvious that the simplexes have nothing to do
with the case, they seem to have been brought in only to be pushed out again.

Various ways have been proposed of grappling with this difficulty of
an elementary course on homology. One may define complexes as
collections of cells of such generality that the fundamental region
with is identifications is such a complex, to which the computing
rules can be applied directly. The whitehead generalized complex, in
which cells are attached by any map of the boundary, is of this
kind. Of such methods I can only say that I have yet to find a version
which is simple enough to be successfully presented at this
level. Even if a theory specially designed to have just enough
generality to cover the examples, and yet to be simple enough for the
first course, could be worked out, it is questionable whether it is
sound practice to teach theories specially designed to avoid
difficulties at an elementary level. At the higher level of axiomatic
homology theory the difficulty easily disappears : the various methods
of computing the groups can be regarded as homology theories which are
equivalent by a general theorem.

The best solution of the difficulty is, I believe, the drastic one of
not making homology one of the main topics of the geometrical topology
course at all. Some basic facts about the Betti numbers and torsions
must be known, but they can be taught with a bare indication of
proofs, if the main theme of the course lies elsewhere.

What are the alternative topics ? There are certain well-known pieces
of theory which naturally present themselves, the proof of the
2-dimensional Jordan's theorem by some special device that does not
generalize to more dimensions ; the normal form of surfaces obtained
by Brahana's method of cutting and re-sticking ; the
computation\pageoriginale  of the groups of simple knots in Euclidean
3-space by Alexander's method. All these topics are interesting and
easy to teach, but they are rather isolated and can hardly be said, if
taught in isolation, to give the student even a glimpse of anything
that is going on at the present time. Some attempt at least should be
made to introduce students to larger theories which are really
alive. I will suggest two such domains of which I believe something
can be successfully taught at this level.

The notion of \textit{homotopy}, the deformation of one map into
another, is one which is much easier to define accurately and explain
intuitively than that of homology. Although the deeper parts of the
theory of homotopy use the full resources of algebraic topology, and
were indeed the main stimulus for its modern development, there is a
great deal of more elementary work that can be done without it. The
\textit{fundamental group} $\pi_1(X)$ can be defined without
difficulty for any topological space $X$, and the condition that it
may then have its standard properties could be made the occasion of
introducing the notions of local connexion in dimensions 0 and 1, and
of pathwise connexion. Some fairly hard work has to be done to
establish the rule for computing $\pi_1(K)$  for a complex $K$ with a
countable set of cells, but it is simplified by the fact that only 1
and 2 cells are involved ; and unlike the homology theory, can really
be applied to interesting cases. For example it shows at one that if
$K$ is 1-dimensional, $\pi_1(K)$ is a free group, and hence in
particular that $\pi_1$ need not be commutative. The \textit{covering
  space} of a given space is of great interest both for its direct
applications in function theory and group theory, and as a simple
example of a fibre space, with the lifting theorem for paths and their
homotopies as an example of the general lifting theorem. Something of
a purely descriptive nature about fibre-spaces might well be
introduced at this point. The construction of the covering space of
$X$ for a given subgroup of $\pi_1 (X)$ is an attractive piece of
work, from which, the fact already mentioned, that $\pi_1 (K)$ is free
if $K$ is 1-dimensional, the Nielsen-Schreier theorem on subgroups of
free groups is an almost immediate deduction. The theory of knotted
curves in Euclidean 3-space arises\pageoriginale naturally in
connexion with the fundamental group, and of course, when presented
in the context of a general treatment of homotopy, this piece of work
is not open to the criticism of being isolate or out of touch with
modern work. A most notable recent advance in the topology of
Euclidean 3-space has been the proof by Papakyriakopoulos of the long
outstanding conjecture that a curve whose knot-group is freecyclic is
unknotted. The proof as simplified by Shapiro and Whitehead is
probably too large a piece of work to be included in the first degree
course without upsetting the balance ; but it contains nothing that
could not be understood by these students and would form an attractive
theme for a seminar run in parallel with the lectures.

To conclude my remarks in homotopy ; at least some mention might be
made of the higher homotopy groups $\pi_n (X)$ which are as easily
defined as the fundamental group. Whether the commutativity for $n
>1$, and perhaps even the fundamental Hurewicz theorem on the relation
between $\pi_n$ and $H_n$, should be proved is a matter of taste.

In this sketch of an introduction to homotopy I have not mentioned the
\textit{degree} of a mapping, which is historically the beginning of
the application of algebraic methods to the study of homotopy, and for
that very reason. The theory of the Brouwer degree belongs to
algebraic topology, and although something akin to it in the case
$n=2$ must be used in proving the completeness of a set of relations
for $\pi_1$, the general theory cannot be adequately treated without
spoiling the elementary character of the treatment here suggested. 

A second body of theorems which can be treated accurately without
elaborate preparations belongs more properly to what I first called
geometric topology, the study of homeomorphism and isotopy. If we
assume the Jordan theorem that a topological $(n-1)$-sphere in closed
Euclidean space $\bar{R^n}$ has two residual domains---and I see no
harm in doing so---the question arises whether the closures of the two
domains are topological $n$-cells, homeomorphs of the ``$n$-ball'',
$B^n = \left[\sum\limits^n_1 x^2_i \leqslant 1 \right]$ in $R^n$. If
$n =2$ it is not difficult to prove that the\pageoriginale answer is
Yes. If $n=3$ a famous example of Alexander shows that the answer is
No if the mapping is unrestricted. This is perhaps the best-known
example of the so-called wild embeddings, though an earlier one was
the set, $A$ in $R^3$ discovered by Antoine, which although totally
disconnected and compact presents a barrier which a loop cannot cross;
i.e. a loop exists in $R^3 -A$ which is not homotopic to a point in
$R^3 -A$. The exhibition of some wild embeddings, and of counter
examples for various plausible conjectures, is a convincing way of
demonstrating the need to support intuition by proof and also a way of
showing how to give an exact description  of a complicated geometrical
object without sinking in a sea of inequalities and mappings. The
school of Wilder and Bing have produced a menagerie of these strange
examples including the expression of a 4-cell as a product $I \times
A$, where $I$ is the unit interval, but $A$ is not a 3-cell.

To return, however, to the question of the residual domains of a
topological $(n-1)$-cell in $R^n$: it was proved by Alexander
simultaneously with the discovering of the horned sphere, that if a
topological 2-sphere in $R^3$ is a Euclidean complex, i.e. consists of
plane rectilinear triangles, then it \textit{splits} the closed
3-space $\bar{R^3}$, i.e. the closure of the residual domains are
topological 3-cells. This also is a suitable piece of work for the course.

From 1924 to last year the question remained entirely open whether a
topological $(n-1)$-sphere in $R^n$ which is a Euclidean
$(n-1)$-complex splits $\bar{R^n}$; or even whether the still stronger
condition that $X$ be a semi-linear monomorphism of $\dot{\sigma}^n$
in $R^n$, is sufficient to ensure the splitting property. But early in
1959 Barry Mazur published his remarkable proof of the following
theorem.  Let $J$ be the interval $\langle -1, 1\rangle$. Then if
$\phi$ maps $\dot{d}^n \times J (1,1)$ into  $R^n$, and is linear in
some small region of $\dot{\sigma}^n \times J$, then
$\phi(\dot{\sigma}^n \times 0)$ splits $\bar{R}^n$ ; i.e. if the
topological map $\phi$ of $\dot{\sigma}^n \times 0$ can be extended to
a map of the shell between two concentric $n$-simplexes, so as to be
linear in some non-empty open set, then $\phi (\dot{\sigma}^n \times
0)$ is a splitting sphere. From this result it was possible to deduce
somewhat laboriously the theorem that\pageoriginale every semi-linear
map of $\dot{\sigma}^n$ splits $\bar{R^n}$. However, news has just
lately been received that a proof has now been found by Morton Brown
of the theorem without the linearity conditions; i.e. that given any
monomorphism $\phi : \dot{\sigma}^n \times J \to R^n$, $\phi
(\dot{\sigma}^n \times 0)$ splits $\bar{R}^n$. This makes the
deduction that a semi-linear map splits $\bar{R^n}$ quite simple. The
proof of the Morton Brown result itself is of quite elementary
character. Although I have not had much time to reflect upon the
details it seems that the exposition of the proof with all the lemmas
leading up to it would make a highly suitable final portion of a
section on homeomorphism problems.

The general conclusion of my remarks is that in teaching the
geometrical side of topology it is not necessary to follow the
traditional plan of giving what is essentially the first part of a
large scale course in algebraic topology, stopping short, before any
of the rewards in the shape of applications are reached. Nor need one
be content with fragments whose relation to the main stream of modern
work cannot be seen by students. The present revival in geometric
topology has provided several pieces of work which are both topical
and capable of inclusion in a first introduction to the subject.


\bigskip
\bigskip

{\fontsize{9pt}{11pt}\selectfont
University of Manchester}\relax
