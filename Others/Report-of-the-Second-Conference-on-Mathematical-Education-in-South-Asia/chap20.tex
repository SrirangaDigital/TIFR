\chapter{Reports of Working Groups}

\begin{center}
{\large\bf 3}
\end{center}
\medskip

\setcounter{pageoriginal}{180}
\noindent
\textsc{Chandrasekharan}~: Regarding\pageoriginale matrices and groups, opinions are invited on the lectures of the 22nd.

\smallskip
\noindent
\textsc{Narlikar}~: There should be a continuous transition from the undergraduate level to the research level.

\smallskip
\noindent
\textsc{Oke}~: For students of physics it would be useful to introduce solutions of linear equations by matrix methods.

\smallskip
\noindent
\textsc{Gupta}~: The notion of the determinant is necessary for the introduction of the inverse matrix.

\smallskip
\noindent
\textsc{Artin}~: The inverse can be introduced directly by taking the inverse system of equations. Linear algebra is directly connected with linear equatoins, and determinants are inconvenient for solving equations.

\smallskip
\noindent
\textsc{Krull}~: Tensors should be introduced after vectors.

\smallskip
\noindent
\textsc{Chandrasekharan}~: How and when are groups to be introduced ?

\smallskip
\noindent
\textsc{Lichnerowicz}~: Start with groups of linear transformations, then introduce the notion of abstract group, and then give examples of groups of automorphisms, vector spaces, etc.

\smallskip
\noindent
\textsc{Narlikar}~: The group-concept can be introduced at the B.A. stage, in connection with solution of equations by radicals.

\smallskip
\noindent
\textsc{Chandrasekharan}~: Are groups introduced in any Indian university before the B.A. stage ?

\smallskip
\noindent
\textsc{General Answer}~: No.

\smallskip
\noindent
\textsc{Akizuki}~: Matrices can be introduced via vector spaces, rather than via solutions of linear equations.

\smallskip
\noindent
\textsc{Eliezer}~: One should start from what the student knows, and then recast what he knows in an axiomatic form.

\smallskip
\noindent
\textsc{Fenchel}~: Groups,\pageoriginale when introduced in secondary schools, should not be done abstractly but by emphasizing the group-properties of examples such as groups of numbers, motions, and permutations. This can then be continued in the university. Define the notion of a vector-space via ordinary geometric vector spaces. At the school stage linear equations in two and also in three unknowns can be discussed thoroughly without too much abstraction.

\smallskip
\noindent
\textsc{Krull}~: Begin with concrete groups, but switch over to abstract groups because of the advantages in dealing with the latter.

\smallskip
\noindent
\textsc{Chandrasekharan}~: What about lattices as mentioned in Professor Krull's lecture ?

\smallskip
\noindent
\textsc{Artin}~: A useful notion but uninteresting in itself, like semi-groups.

\smallskip
\noindent
\textsc{Chandrasekharan}~: Let us turn now to calculus. Professor Stone does not want sequences before functions. Hardly, however, goes the other way round.

\smallskip
\noindent
\textsc{Artin}~: There is a historical explanation. Fifty years ago series were the starting point, the limit concept was introduced via series, and therefore burdened with an addition process. To get away from this, sequences were used.

\smallskip
\noindent
\textsc{Chandrasekharan}~: What is your (Professor Artin's) reaction to Professor Stone's introduction of the integral ?

\smallskip
\noindent
\textsc{Artin}~: I divide a calculus course into two halves. The first half is purely intuitive on the `something approaches something' lines. Most proofs from these ideas are entirely rigorous. The series gap comes in the proof of the existence of a maximum. This and other gaps are pointed out in the second half which is a rigorous course, starting with the continuity axiom for nested intervals. The notion of area is first assumed and later introduced rigorously via step functions.

\smallskip
\noindent
\textsc{Chandrasekharan}~: What is the earliest stage at which the derivative can safely be introduced ? 

\smallskip
\noindent
\textsc{Artin}~: As\pageoriginale soon as the student gets some skill in elementary algebraic manipulation. The idea of graphs and slope should be used.

\smallskip
\noindent
\textsc{Fenchel}~: On the basis of a continuity axiom, i.e. that for nested intervals, limits and derivatives are introduced at 16-17 in Danish schools; there is no need for sequences, we do it in roughly the way Professor Stone suggested. There is no intuitive preparation. However, the integral is in general done somewhat more intuitively that in Professor Stone's course; some teachers use the notion of area. One could also introduce the integral as in numerical integration using successive approximation by polygonal functions. Existence proofs could be given for monotone functions and then for piecewise monotone functions. Integration is done rigorously, in the first term at the university, by means of upper and lower sums. 

\smallskip
\noindent
\textsc{Stone}~: A method should be used only if it can be generalized to several variables. My way permits this. Uniform continuity is {\em not} needed to prove the existence of the integral of a continuous function. Also area should definitely be avoided; area is too much an article of faith, particularly in America (e.g. it is assumed that a circle has an area). Also the mean-value theorem must be made fundamental in the teaching of the calculus. Freudenthal is against this, but Bourbaki is for it !

\smallskip
\noindent
\textsc{Chandrasekharan}~: I am in complete agreement with Professor Stone. I am completely opposed to the `area' concept. Also the mean-value theorem is central. Also I have {\em no} belief in the intuitive part of Professor Artin's calculus course. It inhibits a quick grasp of $\epsilon$, $\delta$ proofs.

\smallskip
\noindent
\textsc{Narlikar}~: In our courses the finer points of the calculus are slurred over, only some manipulative skill is taught. In examinations, questions are set using ghastly algebraic and trigonometric formulae. Exponents of this method claim that such exercises support `vigour against rigour'. This should be remedied.

\smallskip
\noindent
\textsc{Chandrasekharan}~: I request Professor Stone to tell us what the consensus of opinion in Europe is, after the recent meetings in Paris.

\smallskip
\noindent
\textsc{Stone}~: There\pageoriginale was a meeting sponsored by the O.E.E.C. in France in November 1959. There was a discussion on mathematical teaching in secondary schools in Europe. Dieudonn\'e said that ``Euclid should be abolished'' and that a coordinate geometry and vector algebra course (such as that of Professor Fenchel) should be introduced in school ! It was felt that the language of set theory should be introduced as early as possible. The theory of equations should be dropped. Geometry should be based on a study of transformations, and the algebraic aspect of geometry should be emphasized. Probability and statistics should be introduced. The discussion on analysis never reached a conclusive stage.

