
\chapter{Analytical and Projective Geometry}

\begin{center}
{\em By~} WOLFGANG KRULL
\end{center}

The\pageoriginale problem how to introduce analytical and projective
geometry at the university in the first year may be tackled in two
different ways :
\begin{enumerate}
\item In the first case will be treated mainly the geometry of the
  plane and of the three dimensional space. As much concrete detail as
  possible will be studied. By many interesting examples the novice
  will be shown how exciting mathematics can be. The student has the
  opportunity of gradually sharpening his wits in tackling exercises
  of steadily increasing difficulty. In this manner the way to higher
  mathematics is made as easy for the beginner as possible. But the
  student comes to know only a comparatively small sector of
  analytical geometry and he gets no appropriate idea of the leading
  concepts that underlie not only geometry but various other
  departments of mathematics also. Consequently, there will arise much
  loss of time in the lectures of the following years. For it will
  perpetually occur that we have to deal with concepts that are
  essentially the same but which are formulated in a slightly
  different manner. And the student will not be trained to realize the
  intrinsic likeness in the external heterogeneousness. But such a
  loss of time is now-a-days an impossibility. There are too many
  topics that must be treated in a comparatively short time. We must
  therefore earnestly consider the alternative of a second way.

\item After a short concrete introduction restricted to the plane and
  the three-dimensional space, we proceed at once to $n$-dimensional
  geometry. Superfluous details are omitted even though they may be
  interesting and instructive. But full weight is given to the leading
  ideas that deal in the terms of geometry with such concepts as occur
  in reality in almost all branches of mathematics, not only in
  geometry but in analysis, algebra and arithmetic also. Naturally,
  there is the objection to this method that is makes great demands
  on\pageoriginale the faculty of abstract reasoning. Also the novices
  know very little of mathematics in general and consequently it is
  difficult to illustrate from the beginning the importance of the
  main concepts by a sufficient number of concrete examples. The
  student must therefore for a certain time follow the deductions of
  his teacher without fully comprehending what he is aiming at. In
  spite of these obejctions I shall deal in my address only with the
  second way.
\end{enumerate}

Naturally it is necessary to make things as easy for the beginner as
possible. Therefore we make the following demands :
\begin{enumerate}
\item We must formulate a small number of main theorems and must show
  that all the other results are only applications.

\item The proofs must be as short and clear as possible. The methods
  of deduction must be well formalized in such a manner that
  perpetually the same arguments are used.  Consequently these
  arguments will get familiar even to the beginner who at first does
  not love logical reasoning at all.

\item To hammer the methods in the student's heads we must set them
  plenty of exercises. These exercises may be partially of a very
  trivial type. Consequently even the less gifted will be able to cope
  with them and he will progress without losing courage.
\end{enumerate}

Applying these general considerations to analytical geometry
particularly we find out that the main concepts that must be familiar
to the students at the end of the first year are these.
\begin{enumerate}
\item The vector space in the modern abstract sense ; especially the
  vector space of finite dimensions. 

\item The affine space, regarded as a space of points, not only of
  vectors. (Affine point-space.)

\item The space of projective geometry. (Projective space.)


\item The lattices of linear subspaces of an affine or projective
  space. The possibility of representing a subspace either by using
  parameters or by a system of linear equations.

\item The\pageoriginale fundamental concept of duality that, properly
  formulated, dominates the whole of analytical geometry.
\end{enumerate}

In analyzing the concepts mentioned above we realize that basically we
have to deal with an advanced theory of a system of linear equations
and the manifold of its solutions. The importance of the theory
consists in the fact that starting from mere calculation it penetrates
the underlying invariant concepts formulated in the language of
geometry. Having realized this connection we have found a leading
principle for a lecture on analytical geometry : In whatever manner
the abstract concepts may be introduced, enough weight must be given
to the methods of solving linear equations. Also the less talented
must acquire these methods thoroughly, first of all the standard
method of successive elimination and the calculus of
matrices. Besides, it will not be necessary to introduce in our
program the theory of determinants. This is of importance, for in
postponing the calculus of determinants to a later date we avoid a
considerable loss of time. 

Now, what may we expect a novice to know about linear equations and
analytical geometry ? Naturally I am in a position to speak only of
the situation in my own country, Germany. There the present state of
affairs is about the following : The beginners are well acquainted
with solving two or three inhomogeneous equations in two or three
indeterminates  respectively. They know the rules of calculation with
real numbers, they have a certain hazy notion of complex numbers. Even
the term ``field'' may be familiar to some of them. In geometry they
are acquainted with the use of coordinates in the plane. They know the
equation of a line and the Hessian form of such an equation. Besides,
they know the various types of conic sections. As to the analytical
geometry of three dimensional space we may not expect the beginners to
know anything at all. They may have a notion of the equation of a
plane ; certainly they will not know how to define a line by means of equations.

In one respect the situation in Germany is really a good one. Very
probably the beginner will have heard something of vectors. Naturally\pageoriginale
he will only know the concrete vectors of three dimensional space,
where a vector is defined as a line segment having a certain direction
and orientation. But he will be acquainted with  the addition of
vectors, and that gives a favourable starting point for the
introduction of general Abelian groups. 

In every case it is favourable to begin the lecture with the
discussion of concrete vectors in the three dimensional space and to
proceed from the addition of vectors to the concept of an abstract
Abelian group. The observation that the concrete vectors admit
multiplication by real numbers, given us the definition of a vector
space in the modern sense of the term : A vector space is an
additively written Abelian group whose elements $v, u, w, \ldots$
admit multiplication by real numbers $\alpha,\beta,\gamma, \ldots$
with the customary laws :
$$
lv = v ; (\alpha + \beta) v = \alpha v + \beta v ; ~ \alpha (v + w) =
\alpha v + \alpha w; ~ \alpha (\beta v) = (\alpha \beta) v.
$$
(It is suitable to admit at first only real numbers as
multipliers. Later on any field may replace the field of real numbers).

To illustrate the abstract concept of a vector space we have mainly
the following examples :
\begin{enumerate}
\item The set ~$V_n$~ of all ``linear forms'' ~$\alpha^1_1 ~x^1_1 + \ldots
  + \alpha^n_n ~x^n_n$~ in $n$ ``indeterminates''.

\item The set $V'$ of all ordered $n$-tuples $\langle \alpha_1,
  \ldots, \alpha_n  \rangle$.

\item The set $V^\infty$ of all infinite sequences $\langle \alpha_1,
  \alpha_2, \ldots \rangle$.
\end{enumerate}

(Addition and multiplication by real numbers in all three cases are
defined as usual.)

$V^\infty$ is an example of a vector space of infinite
dimension. $V_n$ and $V^n$ are concrete models of a vector space of
dimension $n$. In general we have to deal only with $V_n$ and $V^n$. I
do not think that the introduction itself of $V_n$ and $V^n$ (for any
$n$) makes a difficulty for the beginner. But I know very well from
experience that serious difficulties will arise at the moment when we
derive (from the special case of concrete vectors in the three
dimensional space) the general concept of linear dependence,
generators, bases, subspaces. That is extraordinary,\pageoriginale for there are in
general no intricate theorems to be proved. We need only the following
lemma :

If $v_i (i = 1, \ldots, m)$ and $w_k = \sum\limits^m_{i=1} \alpha^i_k
v_i (k=1, \ldots , m+1)$ are elements of any vector space $V$, then
$w_1, \ldots, w_{m+1}$ are always linearly dependent. The proof of the
lemma consists of a very easy induction on $m$.

We may now proceed to define in the usual way : Dimension of a vector
space (i.e. maximal number of linearly independent elements), set of
generators, base (set of linearly independent generators), subspace of
a vector space. With these definitions we deduce from the fundamental
lemma without any difficulty the theorems :

If $U$ and $W$ are subspaces of $V$ and $U \subseteq W$, then $\dim U
\leqslant \dim W$ ; if $V$ is of finite dimension and $\dim U = \dim
W$, then $U = W$. All bases of a vector space of dimension $n$ are
composed of exactly $n$ elements, and so on.

It is all very easy. But the difficulty for the beginners consists in
the fact that we have there a rather long chain of purely logical
reasoning without any calculation.

Anyhow, it is fortunate that the application of our theorems leads to
problems of a merely calculatory character. If we use the customary
abbreviation $\alpha^i x_i$ instead of $\sum\limits^n_{i=1} \alpha^i
x_i$, that may be introduced even at this early stage of the lecture,
we have in the theory of $V_n$ mainly the following problems : Given a
set of generators $v_k = \alpha^i_k x_i (k = 1, \ldots, M)$ of a
subspace $V$ of $V_n$, to decide whether a given vector $\alpha^i x_i$
belongs to $V$ or not. To find a basis of $V$ consisting of some of
the generators $v_k$, i.e. to find a maximal linearly independent
subset of the set $\{v_1, \ldots, v_M \}$.

To tackle these problems we have to deal with certain systems of
linear equations. There the students may indulge their love of
calculation and they may get well acquainted with the fundamental
method of successive elimination. There we have also the opportunity
for\pageoriginale introducing the fundamental concepts of the calculus
of matrices, including multiplication. Particularly, we may thoroughly
discuss the problem of deciding whether a quadratic matrix $A$
possesses an inverse $A^{-1}$ or not, and to evaluate $A^{-1}$ in case
it exits. Furthermore we may introduce the notion of a general
non-commutative group and derive the equation $A.A^{-1} = A^{-1}. A$
from the fact that the matrices possessing an inverse constitute a
group. As to the notation it will be favourable to use indices at the
top and indices at the bottom ($a^k_i$ instead of $a_{ik}$) though the
consequences of this manner of notation will appear only later. 

Having well acquainted the students with the theory of linear
dependence and the calculus of matrices, we take a step that will be
of great importance for the introduction of projective geometry later
on. We define the notion of a \textit{lattice} (especially a lattice
of sets), and we demonstrate that the set of all subspaces of a vector
space $V$ is a lattice with the operations $U \cap W$ (intersection of
subspaces) and $U +W$ (sum of subspaces as union). In using the
already proved theorems on bases and linear dependence we deduce by
very simple considerations the fundamental equality :
\begin{equation*}
\dim  U + \dim V = \dim (U+V) + \dim (U \cap V). \tag{$\ast$}
\end{equation*}

Now we are at the crucial point of our lecture. We have to develop the
theory of duality for vector spaces of finite dimension. There is the
possibility to use from the beginning the invariant method of
Bourbaki. But that would be too difficult for the novice. Therefore we
begin by establishing a concrete connection between the lattice $L_n$
of all subspaces of $V_n$ and the lattice $L^n$ of all subspaces of
$V^n$. To gain this connection we start with studying more closely the
systems of linear homogeneous equations in $n$ indeterminates $x_1,
\ldots, x_n$ for a fixed number $n$ :
\begin{equation*}
\alpha^i_k x_i = 0 (i=1, \ldots, n ; k = 1, \ldots, N). \tag{E}
\end{equation*}
Given a system (E) we have both a subspace $U_m$ of $V_n$ and a
subspace $U^l$ of $V^n$ attached to (E). $U_m$ is the subspace with
the system of generators $v_k = \alpha^i_k x_i (k=1, \ldots, N)$. The
elements of $U^l$ are the $n$-tuples $\langle \alpha_1, \ldots,
\alpha_n \rangle$ that satisfy the conditions $\alpha^i_k \alpha_i = 0
(k = 1, \ldots, N)$.\pageoriginale (With $n$ respectively $l$ we
indicate always the dimension of $U_m$ respectively $U^l$. The
position of the index shows whether a subspace belongs to $V_n$ or to
$V^n$). From our definitions we deduce immediately :

If the same subspace $U_m$ is attached to the systems of equations (E)
and (E$'$), then also the same subspace $U^l$ is attached to (E) and
(E$'$). The set of all vectors $v = \alpha^i x_i$ satisfying the
condition $\alpha^i \beta_i =0$ for every vector $\langle \beta_1,
\ldots \beta_n \rangle$ of a certain subspace $U^l$ of $V^n$
constitutes a subspace $U_m$ of $V_n$. If $v_k = \alpha^i_k ~ x_i
(k=1, \ldots, N)$ is a set of generators of $U_m$, then $U_m$ and
$U^l$ are attached to the system (E) : $\alpha^i_k x_i = 0
(k=1,\ldots, N)$.

To get rid of the systems (E) of linear equations that were only
introduced for the benefit of the students, we define now the ``scalar
product'' $(v_0, w^0)$ of any pair of vectors $v_0 = \alpha^i x_i \in
V_n$, $w^0 = \langle \beta_1, \ldots, \beta_n \in V^n$ by
$$
(v_0, w^0) = \alpha^i \beta_i.
$$
Furthermore we add the definitions : The \textit{annihilator} Ann
$(U_m)$ respectively Ann $(U^l)$ of $U_m \in L_n$ respectively $U^l
\in L^n$ is the set of all vectors $w^0 \in V^n$ respectively $v_0 \in
V_n$ satisfying the condition $(v_0, w^0) = 0$ for every vector $v_0
\in U_m$ respectively $w^0 \in U^l$.

We then deduce immediately from our former considerations on systems
of linear equations : By $U_m \to Ann (U_m)$ respectively  $U^l \to \Ann
(U^l)$ is defined a mapping of $L_n$ into $L^n$ respectively of $L^n$
into $L_n$.  It remains to show that the two mappings are not only
``mappings into'' but ``mappings onto'' and that one mapping is the
inverse of the other (i.e. $\Ann (\Ann (U_m)) = U_m$, $\Ann
(\Ann(U^l)) = U^l$). These assertions follow easily from the
fundamental \textit{dimension formula}
\begin{equation*}
\dim (\Ann (U_m)) = n - m, \quad \dim (\Ann (U^l)) = n -l. \tag{$\ast\ast$}
\end{equation*}
The proof of $(\ast\ast)$ itself requires only a simple
calculation. We have to resolve explicitly a system (E) : $\alpha^i_k
~x_i = 0 (k = 1, \ldots, m)$, where the vectors $v_k =\alpha^i_k x_i$
constitute a basis of $U_m$. 

Finally,\pageoriginale we get immediately from the definition of $\Ann
(U_m)$, the\break A.~ I.~ THEOREM. ~The mapping $U_m \to \Ann (U_m)$
defines an antiisomorphism of $L_n$ onto $L^n$ in the sense of the
lattices, i.e. we have :
\begin{align*}
&\Ann (U_m + U'_{m'}) = \Ann (U_m) \cap \Ann (U'_{m'}) ; \\
&\Ann (U_m  \cap U'_{m'}) = \Ann (U_m) + \Ann (U'_{m'}). \tag{$\ast\ast\ast$}
\end{align*}
We are now in possession of all the essential formulas and theorems,
and we have gained them in a very concrete manner without too much
logical reasoning and without any complicated proof. (Even the proof
of ($\ast\ast$) is merely a calculation !) What remains is the
question of generalization and invariant formulation. First of all we
see: Given any two vector spaces $V$, $V^\ast$ of the same dimension
$n$ and two fixed bases $x_1, \ldots, x_n$ respectively $y^1, \ldots
y^n$ of $V$ respectively $V^\ast$, if we now write every vector $v \in
V$ respectively $v^\ast \in V^\ast$ in the form $v = \alpha^i ~ x_i$
respectively $v^\ast = \beta_i y^i$, and if we define $(v, v^\ast) =
\alpha^i \beta_i$, we come to an anti-isomorphism of the lattice $L$
of the subspaces of $V$ onto the lattice $L^\ast$ of the subspaces of
$V^\ast$ in exactly the same manner as above in the special case $V =
V_n$, $V^\ast = V^n$. The construction of the anti-isomorphism of $L$
on $L^\ast$ depends thus only on the choice of two fixed bases $x_i$,
$y^i$ of $V$ respectively $V^\ast$. If we replace $x_i, y^i$ by two
other bases $v_i = \alpha^k_i x_k$, $w^i = \beta^i_k y^k$ the
anti-isomorphism will be altered. But we prove by simple calculation :
The two pairs of bases $x_i$, $y^i$ respectively $v_i, w^i$ define the
same anti-isomorphism if and only if the matrices $\alpha^k_i$,
$\beta^i_k$ satisfy the conditions : $\alpha^l_i ~ \beta^k_l
= \begin{cases}
0 & i \neq k \\
 1 & i = k
\end{cases}$. In other words, we come to the same amto-isomorphism if
and only if the transformations $v_i = \alpha^k_i ~x_k$, $\omega^i =
\beta^i_k y^k$ are \textit{contragredient} in the customary
sense. Thus we have gained by the way the important notion of
contragredient transformations.

If enough time is left, we may now come to the dual $V^\ast$ of a
vector space $V$ in the sense of Bourbaki. In reality, there are no
serious difficulties. But for the further development of analytical
and projective geometry we do not need the concepts of Bourbaki, and
the discussion of these concepts may be easily postponed to a later
lecture. Therefore I may proceed from the theory of duality to
another\pageoriginale topic. Theoretically, we are now in a position
to introduce the essential notions of projective geometry. But
didactically, it is more favourable to discuss first the affine
point-space $A_n$. Naturally, taken as a set, $A_n$ may be identified
easily with $V_n$. We may and we shall even choose as points of $A_n$
the elements $v =\alpha^i x_i$ of $V_n$, $\langle \alpha^1, \ldots,
\alpha^n \rangle$ being the $n$-tuple of ``coordinates'' of the point
$v$ in the ordinary sense. But the essential point is that the lattice
$L^{(a)}_n$ of subspaces of $A_n$ is defined in a quite different
manner from the lattice $L_n$ of subspaces of $V_n$. $L^{(a)}_n$
contains the points of $A_n$ (subspaces of dimension 0). A subspace
$B_m (1 \leqslant m \leqslant n)$ of dimension $m$ is uniquely defined
by anyone of its points $v_0$ and by a subspace $U_m$ of $V_n$ that
does not depend on the choice of the special point $v_0$. $B_m$ is the
set of all the points $v_0 + v (v \in U_m)$. From this definition we
derive, for the coordinates of the points of $B_m$, the two concepts
of ``\textit{representation by parameters}'' and
``\textit{representation by equations}''.  Let $\pi^1, \ldots \pi^m$
be a system of real parameters and $v_0 = \alpha^i x_i$. Further more
let $v_k = \alpha^k_i x_k (k = 1, \ldots, m)$ be a given basis of
$U_m$ and $w^l = \langle\beta_1, \ldots \beta_n \rangle~ (l=1,\ldots,
n - m)$  a basis of $U^{n-m} = \Ann (U_m)$. Then we have, for the
coordinates $\xi^1, \ldots, \xi^n$ of the ``general point'' $\xi^i
x_i$ of $B_m$ in the usual sense, the representation by parameters
\begin{equation*}
\xi^i = \alpha^i +\sum\limits^m_{k=1} \pi^k ~ \alpha^i_k ~~~ (i = 1,\ldots n), \tag{$R_p$}
\end{equation*}
The introduction of the ``dual'' representations ($R_p$), ($R_e$) is a
first application of our theory of duality. It is of fundamental
importance, since essentially the same dual representations will occur
later on in the local theory of differentiable manifolds.

When we now proceed to the investigation of the lattice $L^{(a)}_n$,
we have to deal with solving linear inhomogeneous equations. Our main
results are : If the two subspaces $B_m$ and $C_l$ with the attached
vector spaces $U_m$ respectively $W_l$ have at least one point $v_0$
in common, then the union respectively the intersection of $B_m$ and
$C_l$ is defined by $v_0$ and\pageoriginale the attached vector space
$U_m + W_l$ respectively $U_m \cap W_l$. But there is always the
possibility that $B_m$ and $C_l$ have no common point, and in this
case we have no simple and general method of determining the union of
$B_m$ and $C_l$. Naturally, we have to deal there with the phenomenon
of parallelism, a very interesting  phenomenon from the standpoint of
geometry. But from the point of view of the theory of lattices the
existence of parallel subspaces is an imperfection, and we must admit
that in the theory of $A_n$ and $L^{(a)}_n$ we have no such elegant
theorems as in the theory of $V_n$ and $L_n$. Starting from this
observation we some easily to the concept of a projective space. What
were the main reasons for going over from $V_n$, $L_n$ to $A_n$,
$L^{(a)}_n ?$ Essentially one postulate : The lattice of subspaces of
the given space must contain the elements of this space that are
interpreted as ``points'' on the one hand, as  ``subspaces of
dimension 0'' on the other. Since $L_n$ does not satisfy our
postulate, we did retain the elements of $V_n$ as points of $A_n$, but
we did replace $L_n$ by another lattice $L^{(a)}_n$. In this manner we
came to the important concept of the affine space $A_n$, but the
theorems valid for the lattice $L^{(a)}_n$ were not very
satisfactory. Now, why not try another way ? Let us retain the lattice
$L_n$ and choose as elements of the space to be constructed a suitable
subset of $L_n$. It is obvious that the set to be chosen must be the
set of all subspaces $U_1$ of $V_n$ of dimension 1. That it is
possible to interpret a subspace $W_m$ of $V_n$ as a set of subspaces
$U_1$ instead of interpreting it as a set of vectors, follows from the
fact that $W_m$ contains $U_1$ if any vector $v_0 \neq 0$ of $U_1$
belongs to $W_m$. Thus we gain a new space $P$ with the old lattice
$L_n$ as lattice of subspaces. Only one remark must be made : If we
take the $U_m$ as manifolds of subspaces of dimension 1, a single
$U_1$ is also such a manifold. But then $U_1$ contains only one
element namely itself, and is a subspace of $P$ of dimension 0. From
this follows easily that to any $W_m$, if interpreted as a subspace of
$P$, must be attributed the dimension $m-1$. In particular, $P$ itself
is a space of dimension $n-1$, and not of dimension $n$. Consequently
it is suitable to introduce the following notions. We write
$\tilde{W}_{m-1}$ instead of $W_m$ if we take $W_m$ as a subspace of
$P$ ; in\pageoriginale particular we put $P_n = \tilde{V}_{n-1}$. In
the same way $\tilde{L}_{n-1}$ means $L_n$ taken as the lattice of
subspaces of $\tilde{V}_{n-1}$. Up to this point our reasoning was
rather abstract. But I do not think that it will be too difficult for
the students. at this stage of the lecture. Now, we must stop, and
before dealing with the theorem of duality we must discuss a concrete
model of the ``projective'' space $\tilde{V}_{n-1}$ in the affine
space $A_n$. As usual we call a subspace of $A_n$ of dimension 1 a
\textit{line}, and we understand by a bundle the manifold of all the
lines containing a fixed point (e.g. the origin). We then have the
theorem :

The bundle $\tilde{B}_{n-1}$ of all the lines containing the origin is
a model of $\tilde{V}_{n-1}$ in the affine space $A_n$.

The proof is almost trivial. For according to our definition of
$L^{(a)}_n$ the subspaces of dimension 1 of $A_n$ containing the
origin are in one-to-one correspondence with the subspaces $U_1$ of
$V_n$, i.e. with the elements of $\tilde{V}_{n-1}$. Naturally, in the
case of our model we have no \textit{imbedding} of $\tilde{V}_{n-1}$
in $A_n$ in the usual sense. For the primitive elements of $A_n$ are
its points and the primitive elements of $\tilde{B}_{n-1}$ are the
lines of $A_n$ containing the origin. But it is very instructive for
the students to realize that in geometry it is not always necessary to
interpret the elements of a space as ``points''. Besides, it will be
suitable to treat in greater detail the special case $n =3$, that
means a concrete bundle of lines $\tilde{B}_2$ in the three
dimensional space. The student will recognize that the theory of such
a bundle is of nearly the same importance as the theory of the
plane. Also, the connection between the theory of $\tilde{B}_2$ and
the theory of the sphere will shortly be pointed out. 

Having thoroughly established $\tilde{B}_{n-1}$ as a model of
$\tilde{V}_{n-1}$ we come to connect $\tilde{B}_{n-1}$ with an affine
space $B_{n-1}$ of dimension $n-1$. As $B_{n-1}$ we take any subspace
of dimension $n-1$ of $A_n$ (hyperplane in the customary sense) not
containing the origin. If we attach to every point of $B_{n-1}$ the
uniquely determined line containing both the given\pageoriginale point
and the origin, we get a one-to-one mapping of $B_{n-1}$ on a certain
subset $\tilde{B'}_{n-1}$ of $\tilde{B}_{n-1}$. But $\tilde{B}_{n-1}$
is a proper subset of $\tilde{B}_{n-1}$, the lines of
$\tilde{B}_{n-1}$ that are parallel to $B_{n-1}$ having no point in
common with $B_{n-1}$. Starting from this observation we construct
(not in $A_n$ !) a projective space of points $B^\ast_{n-1}$
containing $B_{n-1}$. To every line of $\tilde{B}_{n-1}$ having a
point in common with $B_{n-1}$ we  attach this point both as a point
of $B_{n-1}$ and as a point of $B^\ast_{n-1}$. But to every line of
$\tilde{B}_{n-1}$ parallel to $B_{n-1}$ we attach a point of
$B^\ast_{n-1}$ only. With respect to $B_{n-1}$ such a point is termed
an ``ideal point of $B_{n-1}$'' or a ``point at infinity of
$B_{n-1}$''. Thus we come to the classical idea of enlarging an affine
space to a projective space by the introduction of ideal points. But
it is essential that we did start with the investigation of the bundle
$\tilde{B}_{n-1}$. For in $\tilde{B}_{n-1}$ there are only concrete
lines and no other ideal elements. It is also favourable that our
connection between $\tilde{B}_{n-1}$ and $B_{n-1}$ was a central
projection in the sense of descriptive geometry. From this remark the
student realize the origin of the term ``projective space''.

As to the calculation we may now introduce homogeneous coordinates in
both $B_{n-1}$ and $B^\ast_{n-1}$. This is a very easy matter. I
content myself with stating that in the case of homogeneous
coordinates it is suitable to take as $B_{n-1}$ the hyperplane of
$A_n$ defined by the equation $\xi^n =1 (\xi^1, \ldots, \xi^n$ being
inhomogeneous coordinates in $A_n$).

Now we may return without difficulty to considerations of a more
abstract character. We use the notations introduced above. Thus
$\tilde{V}_{n-1}$ is the given projective space, $L_{n} =
\tilde{L}_{n-1}$ the lattice of subspaces and so on. Naturally, the
theory of the lattice $\tilde{L}_{n-1}$ may be immediately deduced
from the known theorems on $L_n$. Thus $\tilde{U}_{m-1} +
\tilde{W}_{l-1} = (U_m + W_l)^\sim$ is the union of the subspaces
$\tilde{U}_{m-1}$, $\tilde{W}_{l-1}$ of $\tilde{V}_{n-1}$, and we have
the dimension formula
\begin{equation*}
\dim\tilde{U} + \dim \tilde{W} = \dim (U + W)^\sim + \dim (\tilde{U}
\cap \tilde{W}). \tag{$\tilde{\ast}$}
\end{equation*}\pageoriginale
To get more results, we derive from $V^n$, $L^n$ the space
$\tilde{V}^{n-1}$ and the lattice $\tilde{L}^{n-1}$ in quite the same
manner as $\tilde{V}_{n-1}$, $\tilde{L}_{n-1}$ are derived from $V_n$,
$L_n$. Since we have $L_n = \tilde{L_{n-1}}$, $L^n = \tilde{L}^{n-1}$,
the anti-isomorphism A. I. between $L_n$ and $L^n$ defined above by
means of the annihilator is also an anti-isomorphism between
$\tilde{L}_{n-1}$ and $\tilde{L}^{n-1}$. But also the fundamental
concept of the annihilator itself may be taken over from our former
theory. For given two one dimensional spaces $\tilde{U}_0 \in
\tilde{L}_{n-1}$, $\tilde{W}^0 \in \tilde{L}^{n-1}$ it is evident that
we have $(v_0, w^0) = 0$  for all pairs of vectors $0 \neq v_0 \in
U_1$, $0 \neq w^0 \in W^1$ if and only if we have $(v_0, w^0) = 0$ for
a single pair of such vectors. Thus it is reasonable to put :

$(\tilde{U}_0, \tilde{W}_0) = 0$ respectively $(\tilde{U}_0,
\tilde{W}^0) = 1$  if $(v_0, w^0) =0$ respectively $(v_0, w^0) =1$ for
any pair $0 \neq v_0 \in U_1$, $0 \neq w^0 \in W^1$.

According to this definition we may introduce $\Ann (\tilde{U}_{m-1})$
and\break $\Ann(\tilde{U}^{l-1})$ by means of the elements $\tilde{U}_0$,
$\tilde{W}^0$ of $\tilde{V}_{n-1}$, $\tilde{V}^{n-1}$ only. Now, what
is the importance of the scalar product $(\tilde{U}_0, \tilde{W}^0)$
for the geometry of the projective space $\tilde{V}_{n-1}$ ? Given a
subspace $\tilde{W}_{m-1}$ of $\tilde{V}_{n-1}$ and $\tilde{W}^{n-m-1}
= \Ann (\tilde{W}_{m-1})$ it follows immediately from the definition
of the fundamental anti-isomorphism A. I. : $\tilde{W}_{m-1}$ consists
exactly of all the points $\tilde{U}_0$ satisfyfing the condition
$(\tilde{U}_0, \tilde{W}^0) =0$ for every $\tilde{W}^0 \in
\tilde{W}^{n-m-1}$. Thus $\tilde{W}^{n-m-1}$ gives us a definition of
$\tilde{W}_{m-1}$ by means of equations. In particular each element 
$\tilde{W}^0$ of $\tilde{V}^{n-1}$ defines, by the single equation
$(\tilde{U}_0, \tilde{W}_0) =0$, a subspace $\tilde{W}_{n-2}$ of
$\tilde{V}_{n-1}$ of dimension $n-2$, $-a$ hyperplane in the usual
terminology. Since the correspondence between the $\tilde{W}^0 \in
\tilde{L}^{n-1}$ and the $\tilde{W}_{n-2} \in \tilde{L}_{n-1}$ is a
one-to-one correspondence we may identify the $\tilde{W}^0$, the
elements of $\tilde{V}^{n-1}$, with the hyperplanes of
$\tilde{V}_{n-1}$. Thus we come to the following
fundamental\pageoriginale interpretation of the pair
$\tilde{V}_{n-1}$, $\tilde{V}^{n-1} : \tilde{V}_{n-1}$ and
$\tilde{V}^{n-1}$ represent the same projective space
$\tilde{V}(n-1)$, but seen from different points of view,
$\tilde{V}_{n-1}$ represents $\tilde{V}(n-1)$ as a space of points and
$\tilde{V}^{n-1}$ represents $\tilde{V}(n-1)$ as a space of
hyperplanes.

From this interpretation we deduce without any difficulty :\break
$(\tilde{U}_0, \tilde{W}^0) = 0$ is the necessary and sufficient
condition that the point $\tilde{U}_0$ and the hyperplane
$\tilde{W}^0$ are coincident. Every subspace of points of
$\tilde{V}(n-1)$ is also a subspace of hyperplanes, and conversely. If
a subspace of $\tilde{V} (n-1)$ taken as a space of points is
represented by $\tilde{W}_{m-1}$ and of dimension $m-1$, the same
subspace as a space of hyperplanes is represented by
$\tilde{W}^{n-m-1} = \Ann (\tilde{W}_{m-1})$ and of dimension
$n-m-1$. Finally we derive froom the fact that the connection between
$\tilde{V}_{n-1}$ and $\tilde{V}^{n-1}$ is altogether symmetrical, the
famous \textit{theorem of duality in projective geometry :}

From every correct theorem that deals with points, hyperplanes and
coincidences we derive another correct theorem by only exchanging the
terms ``point'' and ``hyperplane''.

Having now discussed thoroughly the basic concepts of projective
geometry we may ask : Is it really suitable to introduce projective
geometry already in the lectures of the first year ? I think, the
answer will be ``yes'' and that for three reasons :
\begin{enumerate}
\item The theory of a bundle of lines in three dimensional space
  belongs certainly to elementary geometry, and we have seen that this
  theory is equivalent to the theory of a two dimensional projective
  space.

\item Even in elementary geometry and even at school it may happen
  that ``points at infinity of a plane'' may be introduced (think of
  the theory of conic sections !) Therefore it will be suitable to
  discuss already in the first year the real meaning of the term ``at
  infinity''.\pageoriginale But such a discussion leads invariably to
  the introduction of the concepts of projective geometry.

\item In the lectures of the first year we have not only to treat the
  theory of linear manifolds, we have also to deal with the surfaces
  of second order, especially in the cases of dimension two and
  three. But it is evident that the classification of surfaces of the
  second order is much more simple in projective space than in affine
  space. Therefore in the case of problems of the second order it will
  always be a disadvantage if we are not in a position to begin with
  the projective theory. I think, mainly the last consideration
  decides in favour of the introduction of projective geometry in the
  lectures of the first year. 
\end{enumerate}

At the end of my address it will be suitable to discuss once more the
question : It not the modern way of treating analytical geometry too
difficult for the beginner ? To this main question we may answer :
\begin{enumerate}
\item As you have seen, the development of the leading ideas is simple
  and clear. The proofs of the main theorems are comparatively
  short. It should be easy, in any case for the talented, to retain
  the essential points and later on to plant the details in the mind.

\item The main difficulty for the beginners consists in the
  multifariousness of the concepts that must be introduced (Groups,
  vector spaces, lattices). Since the novices know too little of
  mathematics in general it is not possible at first to illustrate the
  importance of the main concepts by a sufficient number of
  instructive examples. There is also the difficulty that we need much
  purely logical reasoning without much calculation,--and many of the
  beginners love calculating and detest reasoning.

\item On the other, hand, you have seen : The concepts in question
  contain the abstract and invariant clothing of a very simple problem
  of calculation, namely the solving of linear equations. Therefore it
  is possible to choose plenty of exercises from the department of
  linear equations, and thus to give even the less talented a way to
  penetrate gradually the essential ideas.

\item The\pageoriginale main point is this : In treating mathematics we must deal
  with an ever increasing mass of topics. Therefore it is strictly
  impossible to give too much time to details, even if they may be
  very interesting. For the student does not need an all too extensive
  knowledge on all possible subjects. What he needs is only a
  comparatively restricted fundamental knowledge. But first of all he
  must be well acquainted with the essential methods of treating
  modern problems in a modern manner. It is thus necessary for the
  teacher to put from the beginning the leading ideas into the centre
  of the lecture, and this is exactly the modern way. Naturally, we
  must face the difficulties that may arise consequently. The problem
  of conquering these difficulties in the case of analytical and
  projective geometry, by a simple and clear development of the
  essential ideas, was the main topic of my lecture.
\end{enumerate}


\bigskip
\bigskip

{\fontsize{9pt}{11pt}\selectfont
University of Bonn}\relax

