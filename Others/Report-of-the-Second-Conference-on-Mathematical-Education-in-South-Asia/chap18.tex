\chapter{Reports of Working Groups}

\begin{center}
{\large\bf 1}
\end{center}
\medskip

\setcounter{pageoriginal}{170}
\noindent
(The\pageoriginale discussion is based partly on Professor Artin's lecture on 20 January 1960.)

\medskip
\noindent
\textsc{Chandrasekharan}~: In examining our candidates for admission, we have never got a satisfactory answer to the question : What is a polynomial ? Not even from people who have read van der Waerden, Vol. I.

\smallskip
\noindent
\textsc{Artin}~: I suggest formal power series in one variable.

\smallskip
\noindent
\textsc{Chandrasekharan}~: Is there no danger of confusion between formal and convergent power series, and a consequent danger for analysis ?

\smallskip
\noindent
\textsc{Artin}~: There is no danger, because the difficulty is psychological. Give the student a power series which does not converge, and make him operate with it.

\smallskip
\noindent
\textsc{Chandrasekharan}~: Students can prove that $\surd 2$ is irrational but not $\surd 3$ or $\surd 11$. In ten years, only three people applying to this Institute have been able to answer this question satisfactorily. The unique factorisation theorem is completely neglected in Indian university teaching. Candidates have the same trouble proving that a polynomial can be factored into linear factors in one way.

\smallskip
\noindent
\textsc{Artin}~: In many books the possibility of integral solutions of $x^{2}=2y^{2}$ is discussed in terms of the concepts `even' and `odd' and no use is made of the notion of a prime.

\smallskip
\noindent
\textsc{Ramanathan}~: The difficulty comes with~: `If $a^{2}|b^{2}$ does $a|b$ ?'

\smallskip
\noindent
\textsc{Chandrasekharan}~: How early do you introduce vectors, and how early do you finish with determinants ?

\smallskip
\noindent
\textsc{Artin}~: This should be discussed under analytic geometry.

\smallskip
\noindent
\textsc{Ramanathan}~: We should first come to a decision regarding the unique factorisation theorem.

\smallskip
\noindent
\textsc{Artin}~: The\pageoriginale theorem must be done for the integers and polynomials and counter examples given.

\smallskip
\noindent
\textsc{Krishnan}~: What about units ?

\smallskip
\noindent
\textsc{Artin}~: Explain units.

\smallskip
\noindent
\textsc{Cartan}~: What about division of polynomials ? It is being assumed here. Do the polynomials have coefficients in a ring or a field ?

\smallskip
\noindent
\textsc{Artin}~: We are talking about a pre-algebra course.

\smallskip
\noindent
\textsc{Stone}~: Some parts of Professor Artin's course could be given with profit at an earlier stage. The multiplication table (for integers) is a concept capable of great generalization; using generalized multiplication tables the concept of a group can be introduced. The notion of a set, first finite and then infinite, can be introduced. The concepts of set-theory can be discovered by the pupil as `low-grade' research. Next should come the notion of a one-one map, and then of an arbitrary function. The idea of representation of elements of a set by a system of maps should be done. This should be followed by the composition of maps. The notion of cancellation should then be done and gradually the notions of semi-group, group, etc. should be introduced.

\smallskip
\noindent
\textsc{Artin}~: All this should be done at the high-school stage, at 15 or 16.

\smallskip
\noindent
\textsc{Chandrasekharan}~: Professor Stone suggests certain methods for the introduction of modern algebra in schools as an experimental measure. There seems to be general agreement regarding the desirability of this.

\smallskip
\noindent
\textsc{Moise}~: It is not always necessary to introduce modern mathematics in school. The beauty of abstract algebra may not be obvious to the student.

\smallskip
\noindent
\textsc{Artin}~: There is no difficulty regarding the introduction of the concept of a field. The student will know several examples. The real\pageoriginale numbers, the complex numbers, the rational numbers, and the residue classes modulo 2 of the integers.

\smallskip
\noindent
\textsc{Chandrasekharan}~: One may know the multiplication table, but one may not grasp the underlying abstract concepts.

\smallskip
\noindent
\textsc{Moise}~: Mathematical training should be concerned with questions of fact. Mathematics must be a science.

\smallskip
\noindent
\textsc{Chandrasekharan}~: The laws of arithmetic are facts in a sense; perhaps one should know that there are deeper facts.

\smallskip
What about a bibliography ?
\smallskip

\smallskip
\noindent
\textsc{Artin}~: One must have Bourbaki~: Alg\`ebre. Undue importance should not be given to the evolutionary aspect of mathematics, since the first proof of any given theorem is always the worst.

\smallskip
\noindent
\textsc{Chandrasekharan}~: Bourbaki is suitable for a teacher's reference book, but not for actual teaching.

\smallskip
\noindent
\textsc{Artin}~: The student will not be able to read Bourbaki at all.

\smallskip
\noindent
\textsc{Lichnerowicz}~: Mathematical education is a sequence of steps and so should be developed in several stages. The pupils should be trained in abstract algebra. What is a good psychological preparation for this type of algebra course ?

\smallskip
\noindent
\textsc{Artin}~: From my own experience, high school students in the U.S. are badly prepared. I suggest a course in analytic geometry and calculus.

\smallskip
\noindent
\textsc{Lichnerowicz}~: What is the position of the tensor product ?

\smallskip
\noindent
\textsc{Artin}~: It should be brought in, perhaps in topology.

\smallskip
\noindent
\textsc{Akizuki}~: In Japan analytic geometry and algebra are put together.

\smallskip
\noindent
\textsc{Stone}~: What about a reference for short proofs of the structure theorems for algebras ?

\smallskip
\noindent
\textsc{Artin}~: Yes, Jacobson. For a bibliography : Bourbaki, Schreier and Sperner, van der Waerden, Jacobson, Zariski and Samuel, Birkhoff and MacLane.

\smallskip
\noindent
\textsc{Chandrasekharan}~: And\pageoriginale your own book ?

\smallskip
\noindent
\textsc{Artin}~: I don't know.

\smallskip
\noindent
\textsc{Stone}~: The I.C.M.I. is preparing a list of superior texts in five major languages, English, French, Italian, German and Russian which might be suitable for translation into other languages. Does any one have any suggestions ? A list of titles will be circulated shortly.

\smallskip
\noindent
\textsc{Chandrasekharan}~: Some people in Poland have the same idea. Co-operation between the two is desirable. I recommend that the Japanese mathematicians be requested to help, since they have translated into Japanese books from many languages.

Analysis is taught for its applications, for example, to number theory, algebra, and topology. The same should hold for the teaching of algebra. Regarding the solution of equations by radicals, even with the old syllabus new concepts can be introduced. The practical utility (for applied mathematics) of a topic should also be taken into account, (e.g. regarding solution of equations).

\smallskip
\noindent
\textsc{Eliezer}~: Numerical analysis can be taught.

\smallskip
\noindent
\textsc{Chandrasekharan}~: That should be taken up in analysis.

\smallskip
\noindent
\textsc{Artin}~: Regarding real numbers, in Germany the construction of real numbers is taken up in a course on calculus. This is actually a waste of time. It could be left to the algebraists who could do it by the introduction of $p$-adic numbers.


