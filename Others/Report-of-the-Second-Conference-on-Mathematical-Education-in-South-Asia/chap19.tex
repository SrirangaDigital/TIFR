\chapter{Reports of Working Groups}

\begin{center}
{\large\bf 2}
\end{center}
\medskip

\setcounter{pageoriginal}{174}
\noindent
(The\pageoriginale  discussion is based partly on Professor Stone's lecture on 21 January 1960).

\smallskip
\noindent
\textsc{Nagendranath}~: There are many definitions of a vector. Professor Stone's is the physicists' one. The definition of magnitude is not clear, as there are the different concepts of localised and free vectors in physics. So a better definition is needed. Also a definition applicable in higher dimensional spaces should be given.

\smallskip
\noindent
\textsc{Stone}~: The definition is sufficient for mathematics. Physicists must distinguish clearly between their various vector concepts.

\smallskip
\noindent
\textsc{Nagendranath}~: Should we use a component-wise definition ?

\smallskip
\noindent
\textsc{Stone}~: Against the modern spirit. The definition should be intrinsic for vectors at least. For tensors perhaps the components should be mentioned to {\em initiate} the students but this is a bad definition in principle.

\smallskip
\noindent
\textsc{Ram Behari}~: In several universities vectors are introduced at the age of 17 in the first year university course. Old books such as Bell etc. are being discarded as vectors are not introduced there; the lead given by the First South Asian Conference on Mathematical Education has been followed, Weatherburn's book has been introduced, papers on mathematical physics and elasticity of continuous media have been introduced in Delhi, tensors are used freely, the students like it.
 
The sharp distinction between 2 and 3 dimensional geometry is\break
being broken down in the U.S.A. Is this right, or should we do\break
2-dimensional geometry first ? 

There is a great necessity for producing technical men. Should calculus be introduced in the first year or should it be postponed till the student is capable of understanding the logic of Rolle's theorem ?

\smallskip
\noindent
\textsc{Stone}~: 2-dimensional\pageoriginale geometry is an `ideal'
geometry whereas\break 3-dimensional geometry has immediate roots in
intuition. 3-dimensional geometry should come with 2-dimensional
geometry. This represents an economy of time. This is done in the
U.S. We have to put in a bit of 3-dimensional geometry for functions
of several real variables. 

Calculus involves a large number of intuitive ideas which are difficult to analyse. However, those with only intuitive ideas can never handle calculus very well. Hence more rigour is needed. Every engineer resents the need for rigour until he runs into difficulties, and he tries to get help from his teacher when his intuition fails. It may take up too much time to give {\em proofs} of all theorems, but this is no excuse for making incorrect or incomplete statements. Proofs can be given later for those interested, but one must have correct and complete statements. Many of the usual calculus books contain false statements, e.g. on maxima and minima.

There is another point regarding the integral calculus. There are two notions of integral, indefinite and definite. Then there is a theorem showing that the two are inter-related ! This causes confusion. I would prefer therefore to call the indefinite integral the primitive.

In most old text-books the implicit function theorem is badly treated. Difficulties of existence are completely glossed over.

\smallskip
\noindent
\textsc{Broadbent}~: The scalar product is useful in the solution of triangles and orthogonal projection. Is there any place in Professor Stone's lectures for the scalar product ? The vector product is essentially a 3-dimensional concept and should be excluded.

\smallskip
\noindent
\textsc{Stone}~: There is no very strong reason for omission, only the practical point that the scalar product is extraneous from the algebraic point of view. Perhaps it is better not to press the use of vectors too far. The use of scalar or vector products depends on the course and the standard of the students.

\smallskip
\noindent
\textsc{Lichnerowicz}~: In\pageoriginale France vectors are introduced in schools; the scalar product at the age of 14 as part of space geometry, and the vector product at the age of 15. Derivation of vectors is introduced in the final year of school, immediately after the derivation of functions. Metric geometry of the triangle is done by means of the scalar product, and moments in statics are in the final year of the school, together with the vector product.

The definite integral is done in the first year at university, for ``good'' limits of step functions.

\smallskip
\noindent
\textsc{Racine}~: There used to be difficulty in French schools in passing from 2 to 3 dimensions, also in India. Is this still true in France ?

\smallskip
\noindent
\textsc{Lichnerowicz}~: Yes; still true.

\smallskip
\noindent
\textsc{Chandrasekharan}~: Are there not several universities in India where vectors are not introduced along with analytical geometry ?

\smallskip
\noindent
\textsc{General Answer}~: Yes.

\smallskip
\noindent
\textsc{Chandrasekharan}~: Are the Indian University delegates convinced that vectors should be introduced along with analytical geometry ?

\smallskip
\noindent
\textsc{Krishnan}~: In Madras 20 years ago there were no vectors and no definition of real numbers. Now, in the last 2 years, vectors come in linear algebra in the post-graduate (M.Sc.) courses. But they also enter earlier in mechanics; and at this level to get rid of the difficulty about bound and free vectors, vectors are introduced as translations. A zero vector is defined as one which has no effect in translation.

\smallskip
\noindent
\textsc{Vaidya}~: In Gujarat, analytical geometry starts at 18, i.e. in the second year at college, but there is no vector work in pure mathematics even up to the Master's degree. Vectors are however, included in the applied mathematics courses. We have been in the habit of defining a vector as a description of a change of position whereas the modern trend is to define position in terms of vectors, so it looks as if we have been carrying on in a topsy-turvy way.

\smallskip
\noindent
\textsc{Nagappa}~: In\pageoriginale Mysore, vectors are always taught in physics, and so not in mathematics, because B.Sc. students have to study both physics and mathematics. The Honours people study vector analysis in more detail. Riemannian geometry is compulsory for M.Sc. but vectors are not introduced as part of analytical geometry. Those taking statistics might not have heard of vectors.

\smallskip
\noindent
\textsc{Alexandrov}~: In U.S.S.R. vectors are taught in school but in physics. Coordinate geometry is done in the first year of university, and it is quite impossible to teach this without vectors. The strict definition of vectors is difficult for a freshman, so one must not dwell on it, but the intuitive concept must be learnt by use, to provide a method for solving problems. It is no use defining vectors by components. A tensor is a function which assigns a number to a coordinate system, but his does not explain anything. But once the notion has been used it is easier to understand a stricter definition.

\smallskip
\noindent
\textsc{Racine}~: Mathematics is a language to be learnt as a child learns a language, without too much grammar and syntax. But mistakes should be avoided.

\smallskip
\noindent
\textsc{Newman}~: I agree with Professor Alexandrov. Get them to use things before you teach them why they work, e.g. logarithms. The tendency in English schools is to reverse the procedure, but this is a wrong method as it is difficult to unlearn false things. There are two kinds of rigour; one is the general explanation with no false statements, the other requires the detailed precision which will ensure that the students get the right answers. This applies not only to vectors but to any branch of mathematics ! It is no use defining what the students don't feel needs defining.

\smallskip
\noindent
\textsc{Stone}~: In my lecture I was concerned only with the bare skeleton, not with the individual problems. One should induce the student to ask basic questions. An attempt to {\em analyse} the concepts should be made very quickly after the basic idea is grasped. Every student in the U.S.A. knows the graphical treatment of lines and circles,\pageoriginale similarly with coordinates. But an 18 year old should be able to analyse a concept right to the bottom. This is not the case in the U.S.A.

\smallskip
\noindent
\textsc{Artin}~: Physical reality is not a model of coordinate geometry. One must make a model; this can be done using $n$-dimensional space. Most students have read science fiction and are thrilled at the thought of $n$-dimensional space. A point is an $n$-tuple of numbers and nothing else. There will be difficulty at the beginning but this is soon overcome. A vector is defined by 2 points, the beginning and the end point, with an equivalence relation in terms of the difference of coordinates.

\smallskip
\noindent
\textsc{Lichnerowicz}~: Two equivalence relations; one for mathematics and one for mechanics, to deal with sliding vectors.


