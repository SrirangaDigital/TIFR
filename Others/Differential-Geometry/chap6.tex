\chapter{Flows and Lie Derivatives}

\section*{Flows and local flows}
\pageoriginale

A smooth action of the additive group $\mathbb{R}$ on a smooth manifold $M$ is called a {\em flow} (or a one-parameter group of diffeomorphisms) on $M$. If $\varphi_{t}:M\to M$ is defined by $\phi_{t}(x)=\varphi(t,x)$, for $t\in \mathbb{R}$, we have $\varphi_{t+s}(x)=\phi_{t}\circ \phi_{s}(x)$, for $t$, $s\in \mathbb{R}$ and $x\in M$.

A {\em local flow} on $M$ is a smooth map $\phi:I\times U\to M$, where $I$ is an open interval in $\mathbb{R}$ containing the origin and $U$ an open subset of $M$, satisfying the following conditions.
\begin{itemize}
\item[(1)] For any $t\in I$, the map $\phi_{t}:U\to M$ defined by $x\mapsto \varphi(t,x)$, $x\in U$, is a diffeomorphism of $U$ onto an open subset of $M$.

\item[(2)] $\phi_{0}=\Iid_{U}$, where $\Iid_{U}$ is the identity map of $U$.

\item[(3)] If $s$, $t$, $s+t\in I$ and $x$, $\phi_{t}(x)\in U$, then we have $\phi_{s+t}(x)=\phi_{s}(\phi_{t}(x))$.
\end{itemize}
(Note that since $\phi_{t}(x)\in U$, $\phi_{s}(\phi_{t}(x))$ is defined).

A local flow is also called local one parameter group of (local) diffeomorphisms.

\section*{Lie bracket of vector fields}

Let $X$ and $Y$ be two vector fields defined in an open set $U$ of $M$. If $X$ is a vector field (over $U$) and $a\in U$, we denote the value of $X$ at $a$, which is an element of $T_{a}(M)$, by $X(a)$ or $X_{a}$.

We now define a new vector field, $[X,Y]$, on $U$ by setting, for $a\in U$,
$$
[X,Y]_{a}(f)=X_{a}(Yf)-Y_{a}(Xf)
$$
where\pageoriginale $f$ is a smooth function defined in a neighbourhood of $\underline{a}$ and $Yf$ (resp. $Xf$) is the smooth function in a neighbourhood of a defined by $(Yf)(x)=Y_{x}f$ (resp. $(Xf)(x)=X_{x}f$). The vector field $[X,Y]$ is called the (Lie) bracket of $X$ and $Y$.

The vector fields over $M$ form a Lie algebra over $\mathbb{R}$, under the bracket operation : i.e.,
\begin{itemize}
\item[(1)] $(X,Y)\mapsto [X,Y]$ is bilinear over $\mathbb{R}$
$$
\text{e.g.~ } [\lambda_{1}x_{1}+\lambda_{2}x_{2},Y]=\lambda_{1}[X_{1},Y]+\lambda_{2}[X_{2},Y],\lambda_{1},\lambda_{2}\in \mathbb{R}
$$

\item[(2)] $[X,X]=0$ (and $[X,Y]=-[Y,X]$).

\item[(3)] Jacobi identity :
$$
[X,[Y,Z]]+[Y,[Z,X]]+[Z,[X,Y]]=0,
$$
for any three vector fields $X$, $Y$, $Z$.
\end{itemize}

\section*{The vector filed associated to a flow}

Let $\varphi:\mathbb{R}\times M\to M$ be a flow on $M$.

Let $a\in M$ and define $\chi_{a}:\mathbb{R}\to M$ by $\chi_{a}(t)=\phi(t,a)$. Let $X(a)=T_{0}(\chi_{a})\left(\left(\dfrac{d}{dt}\right)_{0}\right)$ where $\left(\dfrac{d}{dt}\right)_{0}$ is the value of the vector field $\dfrac{d}{dt}$ (on $\mathbb{R}$) at $0\in \mathbb{R}$. (Sometimes we write $T_{s}(\chi_{a})\left(\dfrac{d}{dt}\right)_{s}=\left(\dfrac{d\chi(t)}{dt}\right)_{s}$ for $s\in \mathbb{R}$). Then $a\mapsto X(a)$ is a vector field on $M$, denoted by $X$. This vector field is the vector field associated to the flow, or the infinitesimal transformation of the flow.

\begin{lemma}\label{chap6-lem6.1}
Let $s\in \mathbb{R}$. We have
$$
\left(\dfrac{d}{dt}\chi_{a}(t)\right)_{s}=x(X_{a}(s)).
$$
\end{lemma}

Let\pageoriginale $X$ be a vector field on $X$ and $\wp : I\to M$ a smooth map, where $I$ is an open interval in $\mathbb{R}$. We say that $\wp$ is an integral curve of $X$, if for each $s\in \mathbb{R}$, we have
$$
\left(\dfrac{d}{dt}\wp (t)\right)_{s}=X(\wp(s)).
$$
By the above lemma, each of the `orbit maps' $\chi_{a}$ are integral curves for the vector field associated with the flow.

\section*{Local flows associated to a vector field}

\begin{theorem}\label{chap6-thm6.2}
Let $X$ be a vector field over $M$ and $U$ a relatively compact open subset of $M$. Then there exists a local flow $\phi:I\times U\to M$ such that for $a\in U$, the map $t\mapsto \varphi(t,a)$ from $I$ to $M$ is an integral curve for the vector field $X$.
\end{theorem}

The result is proved using the following theorem on ordinary differential equations depending on parameters:

Let $\Omega$ be an open subset of $\mathbb{R}^{n}$ and $F:\Omega\to \mathbb{R}^{n}$ a smooth map. Let $a\in \Omega$. Then there exist an interval $I$ containing $0$ and an open set $V$ in $\mathbb{R}^{n}$, $a\in V$, and a (unique) smooth map $\phi:I\times V\to \Omega$ such that
\begin{itemize}
\item[(1)] $\phi(0,x)=x$, $x\in V$

\item[(2)] $\dfrac{\partial \phi(t,x)}{\partial t}=F(\phi(t,x))$, $t\in I$, $x\in V$.
\end{itemize}

\noindent
{\bf Corollary \thnum{6.2.1}.\label{chap6-coro6.2.1}}~{\em Let $M$ be a compact manifold and $X$ a vector field on $M$. Then there exists a flow (unique) on $M$ whose associated vector field is $X$.}

\begin{proof}
Since\pageoriginale $M$ is compact we have by the theorem, a local flow $\phi:I\times M\to M$. Let $s\in \mathbb{R}$ and let $n$ be an integer such that $s/n\in I$. Put $\phi_{s}=(\phi_{s/n})^{n}=\phi_{s/n}\circ\ldots\circ \phi_{s/n}$ ($n$ times), i.e. $\phi(s,x)=\phi_{s}(x)$.
\end{proof}

\begin{remark}\label{chap6-rem6.3}
Let $A:M\to N$ be a diffeomorphism and $X$ a vector field on $M$. We define a vector field $A_{*}(X)$ on $N$ by : 
$$
A_{*}(X)(y)=T_{A^{-1}(y)}(A)(X(A^{-1}y)),
$$
for $y\in N$. If $\varphi_{t}$ is the local flow generated by $X$ then $A\circ \varphi_{t}\circ A^{-1}$ is the local flow generated by $A_{*}(X)$. In particular, when $M=N$, we have $A_{*}(X)=X$ if and only if $A\circ \varphi_{t}\circ A^{-1}=\varphi_{t}$ i.e., $A$ and $\varphi_{t}$ commute. 
\end{remark}

\section*{Lie derivatives of vector fields with respect to a vector field}

Let $X$ and $Y$ be vector fields on $M$. We shall define a vector field, $\theta_{X}(Y)$, called the Lie derivative of $Y$ with respect to $X$.

Let $a\in M$. Let $\varphi_{t}$ be the local flow generated by $X$, defined in a neighbourhood of $a$ (We shall assume that $I$ is symmetric around $0$). Define $Z_{a}(t)\in T_{a}(M)$ by :
$$
Z_{a}(t)=T_{\varphi_{t}(a)}(\varphi^{-1}_{t})(Y(\varphi_{t}(a)))
$$
(Here $t\in I'$, where $0\in I'\subset I$ is an interval such that $\varphi_{t}(a)\in U$, for $t\in I'$)

We define :
$$
\theta_{X}(Y)(a)=\dfrac{d}{dt}z_{a}(t)\big|_{t=0}
$$
where the term on the right is the derivative at $0$ of the function $t\mapsto Z_{a}(t)$\pageoriginale which has values in the (finite dimensional) vector space $T_{a}(M)$. Using the chain rule we have

\begin{lemma}\label{chap6-lem6.4}
For $s\in I'$, we have
$$
T_{a}(\phi_{s})\left(\left.\dfrac{dz_{a}(t)}{dt}\right|_{t=s}\right)=\phi_{X}(Y)(\phi_{s}(a)).
$$
\end{lemma}

\noindent
{\bf Corollary \thnum{6.4.1}.\label{chap6-coro6.4.1}}~{\em If $\theta_{X}(Y)=0$, we have $(\phi_{t})_{*}Y=Y$.}

\begin{theorem}\label{chap6-thm6.5}
We have
$$
\theta_{X}(Y)=[X,Y].
$$
\end{theorem}

\noindent
{\bf Indication of proof.}
Let $f$ be a $C^{\infty}$ function on $M$, with values in $\mathbb{R}$. It is easy to see that there exists a function $g(t,p)$, smooth in $(t,p)$ such that $f\circ \phi_{-t}(p)-f(a)=t\cdot g(t,p)$ and $g(0,p)=-(Xf)(p)$. Then 
\begin{align*}
Z_{a}(t)f &= Y_{\phi_{t}(a)}(f\circ \phi_{-t})\\[3pt]
         &= Y_{\phi_{t}(a)}f+tY_{\phi_{t}(a)}g(t,\phi_{t}(a)).
\end{align*}
Hence,
\begin{align*}
\lim\limits_{t\to 0}\dfrac{Z_{a}(t)f-Y_{a}f}{t} &= \lim\limits_{t\to 0}\dfrac{Y_{\phi_{t}(a)}f-Y_{a}f}{t}+Y_{\phi_{t}(a)}g(t,\phi_{t}(a))\\[3pt]
                                             &= X_{a}Yf+Y_{a}g(0,a)\\[3pt]
                                             &= X_{a}Yf-Y_{a}Xf\\[3pt]
                                             &= [X,Y]_{a}f.
\end{align*}

\noindent
{\bf Corollary \thnum{6.5.1}.\label{chap6-coro6.5.1}}~{\em Let\label{page26} $X$ and $Y$ be vector fields on $M$ such that $[X,Y]=0$. If $\phi_{t}$ and $\psi_{s}$ are the local flows generated by $X$ and $Y$ then $\phi_{t}$ and $\psi_{s}$ commute.}
\smallskip

This\pageoriginale follows from the above theorem, Remark \ref{chap6-rem6.3} and the Corollary \ref{chap6-coro6.4.1}.

\section*{Lie derivative of tensor fields}

Let $X$ be a vector field on $M$ and $\alpha$ a tensor field on $M$. We can define, in the same way we defined the Lie derivative of a vector field, the Lie derivative $\phi_{X}(\alpha)$, of the tensor $\alpha$ with respect to $X$; $\theta_{X}(\alpha)$ is a tensor of the same covariant and contravariant type as that of $\alpha$. To define $\theta_{X}(\alpha)$ we ave only to note that $T_{\varphi_{t}(a)}(\varphi^{-1}_{t}):T_{\varphi_{t}(a)}(M)\to T_{a}(M)$ induces isomorphisms
$$
\bigotimes\limits^{p}T_{\varphi_{t}(a)}\bigotimes \bigotimes\limits^{q}T^{*}_{\varphi_{t}(a)}\to \bigotimes\limits^{p}T_{a}(M)\bigotimes \bigotimes\limits^{q} T^{*}_{a}(M).
$$
We also have an obvious generalisation of Lemma \ref{chap6-lem6.4}. Note that if $\alpha$ is a $p$-form, $\theta_{X}(\alpha)$ is defined as a $p$-form and that $\theta_{X}(\alpha)=0$ if and only if $\varphi^{*}_{t}(\alpha)=\alpha$.

We list some properties of Lie derivatives:
\begin{itemize}
\item[(1)] $\theta_{X}$ is a local operator on tensor fields.

\item[(2)] If $f$ is a real valued function, $\theta_{X}(f)=Xf$.

\item[(3)] If $Y$ is a vector field, $\theta_{X}(Y)=[X,Y]$.

\item[(4)] If $\alpha$ and $\alpha'$ are tensor fields, we have
$$
\theta_{X}(\alpha\otimes \alpha')=\theta_{X}(\alpha)\otimes \alpha'+\alpha\otimes\theta_{X}(\alpha')
$$

\item[(5)] If\label{page27} $\alpha$ is a covariant tensor field of rank $p$, i.e., a section of $\bigotimes\limits^{p}T^{*}$, we have, if $X_{1},\ldots,X_{p}$ are vector fields,
$$
\theta_{X}\alpha(X_{1},\ldots,X_{p})=X\alpha(X_{1},\ldots,X_{p})-\sum\limits_{i}\alpha(X_{1},\ldots,[X,X_{i}],\ldots,X_{p})
$$
(Here note that $\alpha(X_{1},\ldots,X_{p})$ is a real valued function and $X$ can be applied on it).

In\pageoriginale particular if $\alpha$ is a $1$-form and $Y$ a vector field
$$
\theta_{X}\langle \alpha, Y\rangle=\langle \theta_{X}\alpha,Y\rangle+\langle \alpha,\theta_{X}Y\rangle
$$
or by (3),
\begin{equation*}
\theta_{X}\langle \alpha,Y\rangle = \langle \theta_{X}\alpha,Y\rangle + \langle \alpha,[X,Y]\rangle\tag{*}
\end{equation*}\label{page28}
where $\langle,\rangle$ denotes the scalar product.

To prove (4) and (*) we use the Leibniz formula in the following form. Let $A$, $B$, $C$ be finite dimensional vector spaces and $\varphi:A\times B\to C$ a bilinear map. Let $f$ and $g$ be functions from an interval in $\mathbb{R}$ with values in $A$ and $B$ respectively. Define $h(t)=\varphi(f(t),g(t))$. Then we have
$$
\dfrac{dh}{dt}(t_{0})=f(t_{0})\cdot \dfrac{dg}{dt}(t_{0})+\dfrac{df}{dt}(t_{0})g(t_{0})
$$
where we have written $\varphi(a,b)=a\cdot b$.

Then (4) follows from (*) by induction.

\item[(6)] If $\alpha$ and $\beta$ are differential forms we have
$$
\theta_{X}(\alpha\wedge \beta)=\theta_{X}(\alpha)\wedge\beta+\alpha\wedge \theta_{X}(\beta).
$$ 
\end{itemize}

\section*{Interior product and H. Cartan's formula}

Let $E$ be a finite dimension vector space and $v\in E$. Then $v$ defines a linear map $i_{v}:{\displaystyle{\mathop{\wedge}\limits^{p}}}E^{*}\to {\displaystyle{\mathop{\wedge}\limits^{p-1}}}E^{*}$ by :
$$
i_{v}\psi(v_{1},\ldots,v_{p-1})=\psi(v,v_{1},\ldots,v_{p-1}),
$$
where $\psi\in {\displaystyle{\mathop{\wedge}\limits^{p}}} E^{*}$, $v_{1},\ldots,v_{p-1}\in E$. This map is called the interior product\pageoriginale defined by $v$.

If $X$ is a vector field on $M$, we have a linear map
$$
i_{X}:\mathscr{E}^{p}(M)\to \mathscr{E}^{p-1}(M)
$$
with the property
$$
i_{X}\alpha(X_{1},\ldots,X_{p-1})=\alpha(X_{1},X_{1},\ldots,X_{p-1})
$$
where $\alpha\in \mathscr{E}^{p}(M)$, and $X_{1},\ldots,X_{p-1}$ are vector fields.

We then have:
\begin{itemize}
\item[(1)] $i_{X}$ is linear over functions: i.e., $f$ is a real valued (smooth) function and $\alpha$ a $p$-form then
$$
i_{X}(f\cdot \alpha)=f\cdot i_{X}(\alpha).
$$

\item[(2)] $i_{X}(\alpha\wedge\beta)=i_{X}\alpha\wedge \beta+(-1)^{p}\alpha\wedge i_{X}\beta$, if $\alpha\in \mathscr{E}^{p}$ and $\beta\in \mathscr{E}^{q}$.

\item[(3)] $i^{2}_{X}=0$.

\item[(4)] ({\em H. Cartan's formula}) : $\theta_{X}=i_{X}d+di_{X}$
\end{itemize}
i.e., if $\alpha$ is a $p$-form we have $\theta_{X}\alpha=i_{X}d\alpha+di_{X}\alpha$ where $d$ is the operator of exterior differentiation.

In particular, since $d^{2}=0$, we have
$$
d\theta_{X}=\theta_{X}d.
$$

To prove (4) we write $D_{X}=i_{X}d+di_{X}$. We have $L_{X}(\alpha \wedge \beta)=L_{X}\alpha\wedge \beta+\alpha\wedge L_{X}\beta$ and also $\theta_{X}(\alpha\wedge \beta)=\theta_{X}\alpha\wedge\beta+\alpha\wedge\theta_{X}\beta$, where $\alpha$ is a $p$-form. Hence it is sufficient to verify the formula when $\alpha=f$ a function and $\alpha=df$, with\pageoriginale $f$ a function. Now $\theta_{X}f=Xf$ and $L_{X}f=di_{X}f+i_{X}df=i_{X}df=\langle X,df\rangle=Xf$.

If $\alpha=df$
\begin{align*}
\theta_{X}\alpha(Y) &= X\alpha(Y)-\alpha[X,Y]\quad \text{(see (*) in p.\pageref{page28})}\\[3pt]
&= XYf-[XY-YX]f\\[3pt]
&= YXf.
\end{align*}
On the other hand,
\begin{align*}
(di_{X}+i_{X}d)(df)(Y) &= d(i_{X}df)(Y)=d\{\langle X,df\rangle\}(Y)\\[3pt]
                     &= YXf.
\end{align*}

\begin{remark*}
If $X$ is a vector field and $\alpha$ a $p$-form such that $\theta_{X}(\alpha)=0$, then $\alpha$ is called an `invariant integral' for the local flow generated by $X$. In particular when $\alpha$ is a function $f$ with $\theta_{X}f=Xf=0$, then $f$ is called a `first integral' of $X$.
\end{remark*}

\section*{Formula for the exterior differential}

\begin{proposition}\label{chap6-prop6.6}
Let\label{page30} $\alpha$ be a $p$-form and $X_{1},\ldots,X_{p-1}$ vector fields. We then have
\begin{align*}
d\alpha(X_{1},\ldots,X_{p+1}) &= \sum\limits^{p+1}_{i=1}(-1)^{i+1}X_{i}\alpha(X_{1},\ldots,\widehat{X}_{i},\ldots,X_{p+1})\\[3pt]
&\quad +\sum\limits_{i<j}(-1)^{i+j}\alpha\left([X_{i},X_{j}],\ldots,\widehat{X}_{i},\ldots,\widehat{X}_{j},\ldots,X_{p+1}\right)
\end{align*}
where~ $\widehat{~}$~ over an element means that element is omitted.
\end{proposition}

In particular for a $1$-form $\alpha$,
$$
d\alpha (X,Y)=X\alpha(Y)-Y\alpha(X)-\alpha[X,Y].
$$

\begin{proof}
We\pageoriginale use the formula $\theta_{X}=di_{X}+i_{X}d$ in the proof.

First if $\alpha$ is a $1$-form we have
\begin{align*}
d\alpha(X,Y) &= i_{X}d\alpha(Y)\quad \text{(by definition)}\\[3pt]
             &= \{\theta_{X}\alpha-d(i_{X}\alpha)\}(Y)\\[3pt]
             &= X\alpha(Y)-\alpha[X,Y]-Y\alpha(X)
\end{align*}
(By (*) on p.\pageref{page28} and $di_{X}\alpha=Y\alpha(X)$)

In the general case,
\begin{align*}
d\alpha(X_{1},\ldots,X_{p+1}) &= i_{X_{1}}d\alpha(X_{2},\ldots,X_{p+1})\\[3pt]
                            &= \theta_{X_{1}}\alpha(X_{2},\ldots,X_{p+1})-di_{X_{1}}\alpha(X_{2},\ldots,X_{p+1}).
\end{align*}
Using (5), p.\pageref{page27} for the first term and induction for the second term, we obtain the proposition.
\end{proof}

\begin{remarks*}
\begin{itemize}
\item[(1)] If $X$ is a vector field and $f$ a smooth real valued function, $\theta_{f\cdot X}\neq f\theta_{X}$ in general on tensors. i.e., if $\alpha$ is a tensor field $\theta_{fX}(\alpha)\neq f\theta_{X}(\alpha)$ (Here $f\cdot X$ is the vector field $a\mapsto f(a)X(a)$).

\item[(2)] If $X$ and $Y$ are vector fields we have
\begin{align*}
& \theta_{X}\circ \theta_{Y}-\theta_{Y}\circ \theta_{X}=0_{[X,Y]}\quad\text{and}\\[3pt]
& \theta_{X}\circ i_{Y}-i_{Y}\circ \theta_{X}=i_{[X,Y]}.
\end{align*}
\end{itemize}
\end{remarks*}




