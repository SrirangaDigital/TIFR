\chapter{Differential Forms and de Rham Cohomology}\label{chap4}

\section*{Differential Forms}

Let now $E$ be a vector bundle over $M$. Put ${\displaystyle{\mathop{\wedge}\limits^{p}}}E^{*}=\coprod\limits_{x\in M}{\displaystyle{\mathop{\wedge}\limits^{p}}}E^{*}_{x}$ and let $\pi_{p}:{\displaystyle{\mathop{\wedge}\limits^{p}}}E^{*}\to M$ the natural projection. Then $({\displaystyle{\mathop{\wedge}\limits^{p}}}E^{*},\pi_{p})$ has a structure of vector bundle over $M$ with the following property: if $\{\sigma_{1},\ldots,\sigma_{m}\}$ is a frame for $E^{*}$ over an open set $U$, the set theoretic sections $\{\sigma_{i_{1}}\wedge\ldots \wedge \sigma_{i_{p}}\}(F<i_{1}<\ldots < i_{p}<m)$ of ${\displaystyle{\mathop{\wedge}\limits^{p}}}E^{*}$ over $U$ defined by
$$
c_{i_{1}}\wedge\ldots\wedge \sigma_{i_{p}}(x)=\sigma_{i_{1}}(x)\wedge\ldots\wedge \sigma_{i_{p}}(x)
$$
form a frame for ${\displaystyle{\mathop{\wedge}\limits^{p}}}E^{*}$ over $U$.

The bundle ${\displaystyle{\mathop{\wedge}\limits^{p}}}T^{*}(M)$ is called the bundle of $p$-forms. (A section of ${\displaystyle{\mathop{\wedge}\limits^{0}}}T^{*}(M)$ is a smooth real valued function on $M$). A section of ${\displaystyle{\mathop{\wedge}\limits^{p}}}T^{*}(M)$ is called a {\em differential form of degree $p$} or a $p$-form. If $w_{1}$ is a $p$-form and $w_{2}$ a $q$-form then $a\mapsto w_{1}(a)\wedge w_{2}(a)$ defines a $(p+q)$ form denoted by $w_{1}\wedge w_{2}$.

\section*{The differential of a function}

Let $f$ be a smooth function with values in $\mathbb{R}$, defined over an open subset $U$ of $M$. Let $a\in U$. Define an element $w(a)\in T^{*}_{a}$ by $w(a)(v)=v(f)$, for $v\in T_{a}$. Then $a\mapsto w(a)$ is a differential form of degree $1$ over $U$. We denote this differential form by $df$ and call it the differential of $f$.

\section*{Expression for a differential form in a chart}
\pageoriginale

Let $(U,\phi)$ be a chart. The map $m\mapsto T_{\phi(m)}(\phi^{-1})\left\{\left(\dfrac{\partial}{\partial x_{i}}\right)_{\phi(m)}\right\}$ is a vector field over $U$ for $i=1,\ldots,n$. By abuse of notation we denote this vector field (over $U$) by $\dfrac{\partial}{\partial x_{i}}$. Then $\left\{\dfrac{\partial}{\partial x_{i}},\ldots,\dfrac{\partial}{\partial x_{n}}\right\}$ is a frame for the tangent bundle over $U$. If $x_{i}=p_{i}\circ \phi$ are the coordinate functions over $U$, the $1$-forms $\{dx_{1},\ldots,dx_{n}\}$ is the dual frame of $\left\{\dfrac{\partial}{\partial x_{i}},\ldots,\dfrac{\partial}{\partial x_{n}}\right\}$. The differential forms (of degree $p$) 
$$
\left\{dx_{i_{1}}\wedge\ldots\wedge dx_{i_{p}}\right\}1\leq i_{1}<\ldots<i_{p}\leq n
$$
form a frame for ${\displaystyle{\mathop{\wedge}\limits^{p}}}T^{*}$ over $U$. Thus if $w$ is a $p$-form over $U$, then $w$ can be written uniquely in the form
$$
w=\sum\limits_{i_{1}<\ldots<i_{p}}f_{(i_{1},\ldots,i_{p})}dx_{i_{1}}\wedge\ldots\wedge dx_{i_{p}}
$$
where $f_{i_{1}\ldots i_{p}}$ are smooth functions over $U$.

\section*{The inverse image of a differential form}

Let $\phi:M\to N$ be a smooth map between two smooth manifolds. Let $m\in M$ and $T_{m}(\phi):T_{m}(M)\to T_{\phi(m)}(N)$ be the tangent map of $\phi$ at $m$. Write $T_{m}$ for $T_{m}(\phi)$. Let
$$
{}^{t}T_{m}(p): {\displaystyle{\mathop{\wedge}\limits^{p}}}T^{*}_{\phi(m)}(N)\to {\displaystyle{\mathop{\wedge}\limits^{p}}} T^{*}_{m}(M)
$$
be the transpose of $T_{m}$.

Let now $w$ be a $p$-form on $N$.

Then
$$
m\mapsto {}^{t}T^{(p)}_{m}(w(\phi(m)))
$$
defines\pageoriginale a (smooth) differential form of degree $p$ over $M$, denoted by $\phi^{*}(w)$ and called the inverse of $w$ by $\phi$.

If $f$ is a real-valued smooth function on $N$ we have 
$$
d(f\circ \phi)=\phi^{*}(df).
$$
Moreover if $w$ is a $p$-form and $\eta$ a $q$-form over $N$ we have 
$$
\phi^{*}(w\wedge \eta)=\phi^{*}(w)\wedge \phi^{*}(\eta).
$$

\section*{Exterior differential}

If $U$ is an open subset of $M$ we denote by $\mathscr{E}^{p}(U)$ the vector space (over $\mathbb{R}$) of $p$-form on $U$.

\begin{proposition}\label{chap4-prop4.1}
There exists a unique collection of maps $d:\mathscr{E}^{p}(U)\to \mathscr{E}^{p+1}(U)$, $p$ running through non-negative integers and $U$ through open subsets of $M$, satisfying the following conditions.
\begin{itemize}
\item[\rm(1)] $d:\mathscr{E}^{p}(U)\to \mathscr{E}^{p+1}(U)$ is a linear map.

\item[\rm(2)] If $V$ is an open subset of $U$ the diagram
\[
\xymatrix{
\mathscr{E}^{p}(U)\ar[d]\ar[r]^{d} & \mathscr{E}^{p+1}(U)\ar[d]\\
\mathscr{E}^{p}(V)\ar[r]^{d} & \mathscr{E}^{p+1}(V)
}
\]
is commutative, where $\mathscr{E}^{p}(U)+\mathscr{E}^{p}(V)$ is the restriction map. (This condition expresses the local character of $d$).

\item[\rm(3)] If\pageoriginale $w\in \mathscr{E}^{p}(U)$ and $\eta\in \mathscr{E}^{q}(U)$, we have
$$
d(w\wedge \eta)=dw\wedge \eta+(-1)^{p}w\wedge d\eta.
$$

\item[\rm(4)] For a real valued function $f$ over $U$, $df$ coincides with the differential of the function, as already defined.

\item[\rm(5)] $d^{2}=0$ (i.e., for $w\in \mathscr{E}^{p}(U)$, we have $d^{2}w=0$).

If $U$ is contained in the domain of a chart and
$$
w=\sum\limits_{i_{1}<\ldots < i_{p}}f_{i_{1},\ldots,i_{p}}dx_{i_{1}}\wedge\quad\wedge dx_{i_{p}}
$$
is a $p$-form on $U$, one proves that the above conditions imply that
$$
dw=\sum\limits_{i_{1}<\ldots<i_{p}}df_{i_{1}\ldots i_{p}}\wedge dx_{i_{1}}\wedge\quad\wedge dx_{i_{p}}.
$$
This shows the uniqueness of $d$ and also how one should define $d$ to prove the existence. For open sets $U$ contained in the domain of a fixed chart $(\Omega,\phi)$ and a differential $w$ on $U$, we define $dw$ by the above formula and verify that conditions $1$ to $5$ are satisfied. If $U$ is an arbitrary open set, we cover $U$ by domains $U_{i}$ of maps with $U_{i}\subset U$ and define $d(w|_{U_{i}})$ by the above formula; by uniqueness $dw$ is well defined globally on $U$.
\end{itemize}
\end{proposition}

The operator $d$ is called the operator of exterior differentiation and $dw$ is called the {\em exterior differential} of $w$.

If $\phi:M\to N$ is a smooth map and $w$ a $p$-form on $N$, we have
$$
d\phi^{*}(w)=\phi^{*}(dw).
$$
To\pageoriginale prove this it is enough to show that if $w$ is a form of the type $f \ dg_{1}\wedge\ldots\wedge dg_{p}$ where $f$, $g_{i}$ are function in an open set $V$ of $N$, then $d\phi^{*}(w)=\phi^{*}dw$ in $\phi^{-1}(V)$. Now $d(f\ dg_{1}\wedge\ldots\wedge dg_{p})=df\wedge dg_{1}\wedge\ldots\wedge dg_{p}$ and $\phi^{*}(dw)=\phi^{*}(df\wedge\ldots\wedge dg_{p})=\phi^{*}df\wedge\quad\wedge \phi^{*}dg_{p}$. On the other hand $\phi^{*}(w)=(f\circ \phi)\phi^{*}(dg_{1})\wedge\ldots\wedge \phi^{*}(dg_{p})$. Now it is easy to check that if $g$ is a function on $V$ we have $\phi^{*}(dg)=d(g\circ \phi)$. Hence $\phi^{*}(w)=(f\circ \phi)d(g_{1}\circ\phi)\wedge\quad\wedge d(g_{p}\circ \Lambda)$ so that
\begin{align*}
d\phi^{*}(w) &= d(f\circ\phi)\wedge d(g_{1}\circ\phi)\wedge\ldots\wedge d(g_{p}\circ\phi)\\[3pt]
            &= \phi^{*}df\wedge \phi^{*}dg_{1}\wedge\quad\wedge (dg_{p})\\[3pt]
            &= \phi^{*}dw.
\end{align*}

\section*{de Rham Cohomology}

Let $M$ be a smooth manifold. Consider the complex of vector spaces
$$
0\to \mathscr{E}^{0}(M)\xrightarrow{d}\mathscr{E}^{1}(M)\to\ldots\to \mathscr{E}^{p}(M)\xrightarrow{d}\mathscr{E}^{p+1}\to \ldots
$$
Let $z^{p}$ = kernel of $d:\mathscr{E}^{p}(M)\to \mathscr{E}^{p+1}(M)$ and $B^{p}$ = Image of $d:\mathscr{E}^{p-1}(M)\to \mathscr{E}^{p}(M)$. Since $d^{2}=0$ we have $B^{p}\subset z^{p}$. We define
$$
H^{p}_{DR}(M)=z^{p}/B^{p}.
$$
$(H^{0}_{DR}(M)=\ker d:\mathscr{E}^{0}(M)\to \mathscr{E}^{1}(M))$.

The vector space $H^{p}_{DR}(M)$ is called the $p^{\text{th}}$ de Rham Cohomology space (or group) of $M$. Note that for $p>\dim M$, $H^{p}_{DR}(M)=0$.

The space $z^{p}$ is called the space of $p$-cocycles and $B^{p}$ the space of $p$-coboundaries. A form $w$ with $dw=0$ is said to be closed.\pageoriginale A $p$-form $w$ of the form $w=d\eta$, $\eta$ a $(p-1)$-form, is called a coboundary.

\section*{Induced map on the Cohomology}

Let $\phi:M\to N$ be a smooth map and $\phi^{*}:\mathscr{E}^{p}(N)\to \mathscr{E}^{p}(M)$ the induced map. Since $d\phi^{*}=\phi^{*}d$ we see that $\phi^{*}$ maps $z^{p}(N)$ into $z^{p}(M)$ and $B^{p}(N)$ into $B^{p}(M)$. Hence $\phi^{*}$ induces a map, still denoted by $\phi^{*}$,
$$
\phi^{*}:H^{p}_{DR}(N)\to H^{p}_{DR}(M)
$$

\section*{Homotopic maps}

\begin{defi*}
Let $M$ and $N$ be smooth manifolds and $\varphi_{1}$, $\varphi_{2}$ smooth maps from $M$ to $N$. We say that $\varphi_{1}$ and $\varphi_{2}$ are {\em homotopic} if there exists a smooth map $\phi:\mathbb{R}\times M\to N$ such that
$$
\phi(1,x)=\varphi_{1}(x)\quad\text{and}\quad \phi(0,x)=\varphi_{2}(x)
$$
for every $x\in M$.
\end{defi*}

\begin{theorem}\label{chap4-thm4.2}
Let $\varphi_{1}$ and $\varphi_{2}$ be two smooth homotopic maps from $M$ to $N$. Then $\varphi_{1}$ and $\varphi_{2}$ induce the same map from $H^{p}_{D(R)}(N)$ to $H^{p}_{DR}(M)$ (i.e. $\varphi^{*}_{1}=\varphi^{*}_{2}:H^{p}_{DR}(N)\to H^{p}_{DR(M)}$).
\end{theorem}

\section*{Sketch of proof}

We construct maps $h:\mathscr{E}^{p}(N)\to \mathscr{E}^{p-1}(M)$ satisfying $dhw+hdw=\varphi^{*}_{1}(w)-\varphi^{*}_{2}(w)$ for every smooth form $w(h=0\text{~ on~ } \mathscr{E}^{0}(N))$. This will prove the theorem for if $w$ is a closed form on $N$ we will have\pageoriginale $dh(w)=\varphi^{*}_{1}(w)-\varphi^{*}_{2}(w)$ so that $\varphi^{*}_{1}(w)$ and $\varphi^{*}_{2}(w)$ define the same element in $H_{DR}(M)$.

To construct $h$, we consider the $p$-form $\phi^{*}(w)$ on $\mathbb{R}\times M$, for a $p$-form $w$ on $N$. If $U$ is a chart on $M$ we write $\phi^{*}(w)$ on $\mathbb{R}\times U$ in the form 
\begin{align*}
\phi^{*}(w) &= \sum\limits_{I}f_{I}(t,x)dx_{i_{1}}\wedge\ldots\wedge dx_{i_{p}}\\[3pt]
           &= \sum\limits_{J}g_{J}(t,x)dt\wedge dx_{j_{1}}\wedge\quad\wedge dx_{j_{p-1}}
\end{align*}
where $I=(i_{1},\ldots,i_{p})$, $i_{1}<\ldots<i_{o}$ and $J=(j_{1},\ldots,j_{p-1})$, $j_{1}<\ldots < j_{p-1}$ ($t$ is the coordinate function on $\mathbb{R}$). Define
$$
h(w)=\sum\limits_{J}\left\{\int\limits^{1}_{0}g_{J}(t,x)dt\right\}dx_{j_{1}}\wedge\quad\wedge dx_{j_{p-1}}
$$
on $U$. One can check that $h(w)$ is well defined globally on $M$ (this method of obtaining a $(p-1)$ form on $M$ from a $p$-form on $\mathbb{R}\times M$ is called `integration along the fibres'). One verifies that
$$
dhw+hdw=\varphi^{*}_{1}(w)-\varphi^{*}_{2}(w)
$$











