\chapter{Connection and Curvature}\label{chap12}

\section*{Connections and connection forms}
\pageoriginale

\begin{defi*}
Let $P$ be a principal $G$-bundle. Let $T(P)$ be the tangent bundle of $P$ and $T_{\pi}$ the tangent bundle along the fibres. A connection on $P$ is a subbundle $\mathscr{H}$ of $T(P)$ which is supplementary to $T_{\pi}$ and which is invariant under the action of $G$ on $T(P)$.
\end{defi*}

Thus if $\mathscr{H}_{P}$ is the fibre of $\mathscr{H}$ at $P$ we have 
\begin{center}
(a)~ $T_{p}(P)=\mathscr{H}_{p}\oplus V_{p}$\qquad (b)~ for $g\in G$, $p\in G$,
\end{center}
$T_{p}(R_{g})(\mathscr{H}_{p})=\mathscr{H}_{pg}$. An element of $\mathscr{H}$ is called an horizontal vector and $\mathscr{H}_{p}$ is called the horizontal space at $p$.

If $\eta:T(P)\to T_{\pi}$ is the projection defined by the decomposition $T(P)=\mathscr{H}\oplus T_{\pi}$ we can consider $\eta$ as a $1$-form on $P$ with values in $\mathfrak{g}$, using the isomorphism of $T_{\pi}$ with $P\times \mathfrak{g}$. We denote this $1$-form by $w$ and call it the connection form (of the connection).

Thus the form $w$ is defined as follows. Let $v\in T_{p}(P)$. Write $v=v_{1}\oplus h$ with $v_{1}\in V_{p}$, $h\in \mathscr{H}_{p}$. Under the isomorphism of $V_{p}$ with $\mathfrak{g}$, $v_{1}$ corresponds to an element $v'_{1}$ in $\mathfrak{g}$. Then we define $w(v)$ to be $v'_{1}$.

If $w_{p}:T_{p}(P)\to \mathfrak{g}$ is the value of $w$ at $p$, note that the kernel of $w_{p}$ is $\mathscr{H}_{p}$.

The\pageoriginale 

