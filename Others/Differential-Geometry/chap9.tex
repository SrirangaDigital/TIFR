\chapter{Lie Groups (Continued)}\label{chap9}

\section*{The Lie algebra of a Lie group}

Let $G$ be a Lie group. A vector field $X$ on $G$ is said to be left invariant if for each $x\in G$ and $g\in G$ we have $X(gx)=T_{x}(L_{g})(X(x))$ where $L_{g}:G\to G$ is the left translation by $g$ i.e., $L_{g}(y)=gy$. A left invariant vector field is uniquely determined by its value at the identity element $e$ and given $v\in T_{e}(g)$ there exists a unique left invariant vector field $X$ with $X(e)=v$. Moreover, if $X$ and $Y$ are left invariant vector fields, so are $\lambda X+\lambda Y$, $(\lambda,\mu\in \mathbb{R})$ and $[X,Y]$. Thus the set of left invariant vector fields on $G$ form a $n$-dimensional ($n=\dim G$) Lie algebra over $\mathbb{R}$, called the Lie algebra of $G$ and denoted by $\mathfrak{g}$.

Let\pageoriginale $G_{1}$ and $G_{2}$ be Lie groups and $\varphi:G_{1}\to G_{2}$ a smooth homomorphism. Let $\mathfrak{g}_{1}$ (resp. $\mathfrak{g}_{2}$) be the Lie algebra of $G_{1}$ (resp. $G_{2}$). Then $\varphi$ induces a Lie algebra homomorphism $\varphi_{*}:\mathfrak{g}_{1}\to \mathfrak{g}_{2}$, as follows. Let $X\in \mathfrak{g}_{1}$ and $e_{i}(i=1,2)$ be the identity element of $G_{i}$. Let $X'\in \mathfrak{g}_{2}$ with $X'(e_{2})=T_{e_{1}}(\varphi)(X(e_{1}))$ and define $\phi_{*}(X)=X'$. Since for $g\in G$ we have $\varphi\circ L_{g}=L_{\varphi(g)}\circ \varphi$ it is easy to check that $X$ and $X'$ are $\varphi$=related. It follows from Lemma \ref{chap8-lem8.2} that $\varphi_{*}$ is a Lie algebra homomorphism.

In particular $\varphi:G_{1}\to G_{2}$ is a homomorphism which is an injective immersion, $\mathfrak{g}_{1}$ can be identified with the Lie subalgebra $\varphi_{*}(\mathfrak{g}_{1})$ of $\mathfrak{g}_{2}$.

\section*{Lie subalgebras and Lie subgroups}

Let $G$ be a Lie group and $H\subset G$. Suppose that $H$ has a structure of Lie group such that the inclusion map $i:H\to G$ is a homomorphism which is an injective immersion. We say that $H$ is a Lie subgroup of $G$. (A Lie subgroup $H$ need not be a submanifold of $G$; in fact if it is so, it should be closed in $G$). The Lie algebra $\mathfrak{H}$ of $H$ is identified by $i_{*}$ with a subalgebra of $\mathfrak{g}$.

\begin{proposition}\label{chap9-prop9.1}
Let $G$ be a Lie group and $\mathfrak{H}$ a Lie subalgebra of $\mathfrak{g}$. Then there exists a Lie subgroup $H$ of $G$ whose Lie algebra is $\mathfrak{H}$.
\end{proposition}

\noindent
{\bf Idea of Proof.}~ The subalgebra $\mathfrak{H}$ defines an integrable subbundle $F$ of $T(G)$ as follows. The fibre $F_{x}$ of $F$ at $x\in G$, is the subspace of $T_{x}(G)$, $\{X(x)|x\in \mathfrak{H}\}$. Since for $X$, $Y\in \mathfrak{H}$, we have $[X,Y]\in \mathfrak{H}$, $F$\pageoriginale is an integrable subbundle of $T(G)$. Let $H$ be the leaf through $e$ of the foliation on $G$ defined by $F$. One checks that $H$ is a Lie subgroup of $G$ whose Lie algebra is $\mathfrak{H}$, using that if $C$ is a leaf then $gC$ is a leaf, for $g\in G$.

\section*{Homomorphisms of Lie groups and Lie algebras}

Let $G_{1}$ and $G_{2}$ be two lie groups with lie algebras $\mathfrak{g}_{1}$, $\mathfrak{g}_{2}$. Then the Lie algebra of the Lie group $G_{1}\times G_{2}$ is naturally identified with the direct product $\mathfrak{g}_{1}\times \mathfrak{g}_{2}$. Let $\varphi : G_{1}\to G_{2}$ be a homomorphism of Lie groups and $\varphi_{*}:\mathfrak{g}_{1}\to \mathfrak{g}_{2}$ the induced homomorphism. The graph $\Gamma=\{(x,\varphi(x));x\in G_{1}\}$, of $\varphi$ is a (closed) Lie subgroup of $G_{1}\times G_{2}$ whose Lie algebra is the graph $\Gamma'=\{(v,\phi_{*}v);v\in \mathfrak{g}_{1}\}$ of $\varphi_{*}$. From this we deduce that if $\varphi_{1}$ and $\varphi_{2}$ are two homomorphisms of $G_{1}$ into $G_{2}$ and $\varphi_{1^{*}}=\varphi_{2^{*}}$ on $\mathfrak{g}_{1}$, then $\varphi_{1}=\varphi_{2}$ if $G_{1}$ is {\em connected}.

We now consider the question whether every homomorphism between $\mathfrak{g}_{1}$ and $\mathfrak{g}_{2}$ is induced by homomorphism between $G_{1}$ and $G_{2}$.

\begin{proposition}\label{chap9-prop9.2}
Suppose that $G_{1}$ is {\em simply connected}. If $\psi:\mathfrak{g}_{1}\to \mathfrak{g}_{2}$ is a Lie algebra homomorphism, then there exists a (unique) homomorphism $\varphi:G_{1}\to G_{2}$ such that $\varphi_{*}=\psi$.
\end{proposition}

We indicate a proof of this result.

\begin{defi*}
Let $G_{1}$ and $G_{2}$ be Lie groups and $U$ a neighbourhood of $e$ in $G_{1}$. A smooth map $f:U\to G_{1}$ is called a local homomorphism if for $x$, $y\in U$ with $xy\in U$, we have $f(xy)=f(x)f(y)$.
\end{defi*}

Using the monodromy theorem one proves:

\begin{lemma}\label{chap9-lem9.3}
Let\pageoriginale $G_{1}$ be a {\em simply connected} lie group and $f:U\to G_{2}$ be a local homomorphism where $U$ is a connected neighbourhood of $e$ in $G_{1}$. Then there exists a (unique) homomorphism (smooth) $\varphi:G_{1}\to G_{2}$ such that $\varphi|_{U}=f$.
\end{lemma}
 
To prove Proposition \ref{chap9-prop9.2}, note first that the Lie algebra of $G_{1}\times G_{2}$ is naturally isomorphic to $\mathfrak{g}_{1}\times \mathfrak{g}_{2}$, the direct product of the Lie algebras $\mathfrak{g}_{1}$ and $\mathfrak{g}_{2}$. Let $\Gamma'=\{(v,\psi(v)); v\in \mathfrak{g}_{1}\}\subset (\mathfrak{g}_{1}\times \mathfrak{g}_{2})$ be the graph of $\psi$. Then $\Gamma'$ is a subalgebra of $\mathfrak{g}_{1}\times \mathfrak{g}_{2}$. Let $\Gamma$ be a Lie subgroup of $G_{1}\times G_{2}$ corresponding to $\Gamma'$. The tangent map at identity of the restriction of the projection $\pr_{G_{1}}:G_{1}\times G_{2}\to G_{1}$ to $\Gamma$ is $(v,\psi(v))\mapsto v$, so that this map is a local diffeomorphism of a neighbourhood of identity in $\Gamma$ onto a (connected) neighbourhood $U$ of $e$ in $G_{1}$. Then the composite $f$ of the maps $(\pr_{G_{1}})^{-1}:U\to \Gamma$ and $\pr_{G_{2}}:\Gamma\to G_{2}$ is a local homomorphism. Since $G_{1}$ is simply connected, by Lemma \ref{chap9-lem9.3}, $f$ extends to a homomorphism $\varphi:G_{1}\to G_{2}$ which is the required homomorphism.

\section*{The exponential map}

The Lie algebra of the Lie group $\GL(n,\mathbb{R})$ can be identified with the Lie algebra $\mathfrak{g}(n,\mathbb{R})$ of $n\times n$ real matrices. If $A$ and $B$ are $n\times n$ matrices $[A,B]$ is defined to be $AB-BA$). If $A\in \mathfrak{g}(n,\mathbb{R})$, the series $I+A+A^{2}+\cdots+\dfrac{A^{n}}{n!}+\cdots$ defines an element in $\GL(n,\mathbb{R})$ denoted by $\exp A$, called the exponential of $A$. Thus $A\mapsto \exp A$ is a map of the Lie algebra of $\GL(n,\mathbb{R})$ into $\GL(n,\mathbb{R})$. We shall generalise this to any Lie group $G$.

Considering\pageoriginale $\mathbb{R}$ as a Lie group, the vector field $\dfrac{d}{dt}$ on $\mathbb{R}$ is a left invariant vector field. Let $G$ be Lie group and $X$ an element of $\mathfrak{g}$. There exists a unique Lie algebra homomorphism of the Lie algebra of $\mathbb{R}$ into $\mathfrak{g}$, sending $\dfrac{d}{dt}$ into $X$. Since $\mathbb{R}$ is simply connected, there exists a unique homomorphism $\phi_{X}:\mathbb{R}\to G$ whose tangent map is the above homomorphism. We define
\begin{equation*}
\exp X=\phi_{X}(1).\quad (1\in \mathbb{R})
\end{equation*}
We have
\begin{itemize}
\item[(1)] $\exp((t_{1}+t_{2})X)=\exp(t_{1}X)\exp(t_{2}X)$, $t_{1}$, $t_{2}\in \mathbb{R}$.

\item[(2)] $\exp(-tX)=\exp(tX)^{-1}$, $t\in \mathbb{R}$.

\item[(3)] $\exp :\mathfrak{g}\to G$ is a smooth map.

\item[(4)] $\exp$ is a local diffeomorphism at $0\in \mathfrak{g}$.
\end{itemize}

\section*{The adjoint representation of a Lie group}

If $V$ is a finite dimensional vector space over $\mathbb{R}$, the group $\Aut(V)$ of linear automorphisms is a Lie group (If $\dim V=m$, we can identify $\Aut V$ with $\GL(m,\mathbb{R})$ if we choose a base for $V$). A smooth homomorphism $\rho:G\to \Aut(V)$ is called a representation of the Lie group $G$ on $V$.

We shall now define a natural representation of a Lie group on its Lie algebra, called the adjoint representation of $G$. Let $g\in G$ and $\Int g : G\to G$ be the map $s\mapsto gsg^{-1}$, $s\in G$. Define $\Ad g:T_{e}(G)\to T_{e}(G)$ to be $T_{e}(\Int g)$. Identifying $T_{e}(G)$ with $\mathscr{T}$, we\pageoriginale get a representation $g\mapsto \Ad g$ on $\mathfrak{g}$. If $R_{g}:G\to G$ is the map $s\to sg$, $s\in G$, and $X\in \mathfrak{g}$, it is not hard to prove that $(R_{g})_{*}(X)=\Ad (g^{-1})(X)$.





