\chapter{Lie Groups}\label{chap5}

\section*{Lie groups}

\begin{defi*}
A Lie group $G$ is a smooth manifold $G$ which is also endowed with a structure of a group such that the map $G\times G\to G$ defined by $(x,y)\to xy^{-1}$ is smooth.
\end{defi*}

\begin{examples*}
\begin{itemize}
\item[(1)] The additive group $\mathbb{R}^{n}$.

\item[(2)] Let\pageoriginale $m\geq 1$ an integer and let $GL(m,\mathbb{R})$ denote the group of $m\times m$ real invertible matrices. $GL(m,\mathbb{R})$ is an open subset of the ($m^{2}$-dimensional) vector space of all $m\times m$ real matrices and hence has a structure of a smooth manifold. One verifies that with these structures $GL(m,\mathbb{R})$ is a Lie group. Similarly, $GL(m,\mathbb{C})$ is a Lie group.
\end{itemize}
\end{examples*}

\section*{Action of a Lie group on a smooth manifold}

Let $G$ be a Lie group and $M$ a smooth manifold $A$ (smooth) action of $G$ on $M$ on the left is a map $\phi:G\times M+M$ satisfying the conditions:
\begin{itemize}
\item[(1)] $\phi$ is smooth.

\item[(2)] If $e$ is the identity element of $e$, then $\varphi(e,x)=x$ for every $x\in M$.

\item[(3)] If $g_{1}$, $g_{2}\in G$ then
$$
\phi(g_{1},\phi(g_{2},x))=\phi(g_{1}g_{2},x)\quad\text{for}\quad x\in M.
$$
\end{itemize}

We shall write $gx$ for $\phi(g,x)$ so that 2) reads $ex=x$ and 3) reads $(g_{1}g_{2})m=g_{1}(g_{2}m)$.

For $g\in G$, let $\phi_{g}:M\to M$ be the map $x\mapsto gx$, $x\in M$. Then $\phi_{g}$ is a diffeomorphism and condition 3) may also be written as $\phi_{g_{1}g_{2}}=\phi_{g_{1}}\circ \phi_{g_{2}}$.

Similarly we have the notion of a right action of $G$ on $M$.



