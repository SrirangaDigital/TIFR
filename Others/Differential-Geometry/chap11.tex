\chapter{Equivariant Forms on a Principal Bundle}\label{chap11}

{\bf The tangent bundle along the fibres of a principal bundle. Vertical vectors.}

Let $\pi:P\to M$ be a principal bundle with structure group $G$. Let $P\in p$. We denoted by $V_{p}$ the kernel of the map $T_{p}(\pi):T_{p}(P)\to T_{\pi(p)}(M)$. An element of $V_{p}$ is called a vertical vector at $p$. The tangent space at $p$ of the fibre of $\pi$ through $p$ can be identified with $V_{p}$. The vertical vectors $\coprod\limits_{p\in P}V_{p}$ form a subbundle of $T(P)$, called the tangent bundle along the fibres of $P$, denoted by $T_{\pi}$ or $V$.

\begin{proposition}\label{chap11-prop11.1}
The tangent bundle along the fibres of $P$ is trivial. More precisely there is a canonical isomorphism $\psi$ of the trivial vector bundle $P\times \mathfrak{g}$ with $T_{\pi}$ with the following property: for each $g\in G$ the diagram
\[
\xymatrix@=1.5cm{
P\times \mathfrak{g}\ar[d]^{R_{g}\times \Ad (g^{-1})}\ar[r]^-{\psi} & T_{\pi}\ar[d]^{(R_{g})_{*}}\\
P\times \mathfrak{g}\ar[r]_{\psi} & T_{\pi}
}
\]
is commutative. Here $(R_{g})_{*}$ is the `differential map' induced by the diffeomrophism $R_{g}$, $q\mapsto gg$, $q\in P$ (If $\eta:T_{\pi}\to P$ is the projection and $v\in T_{\pi}$, $(R_{g})_{*}(v)=T_{\eta(v)}(R_{g})(v)$). If $p\in P$, $\xi\in \mathfrak{g}$, $\Iid\times \Ad (g^{-1})(p,\xi)=(p,\Ad(g^{-1})\xi)$)
\end{proposition}

\noindent
{\bf Indication of proof.}\pageoriginale If $p\in P$, consider the orbit map $\sigma_{p}:G\to P$. Note that $\sigma_{p}(e)=p$. The tangent map of $\sigma_{p}$ at $e$ maps $\mathfrak{g}=T_{e}(G)$ isomorphically onto $V_{p}$. These isomorphisms, as $p$ varies, give $\psi$. The commutativity of the diagram can be proved using
\begin{itemize}
\item[(i)] if $X\in \mathfrak{g}$ and $R_{g}:G\to G$ is the right translation we have $(R_{g})_{*}(X)=\Ad(g^{-1})(X)$ (see p.\ref{page43})

\item[(ii)] commutativity of the diagram
\[
\xymatrix@=1.5cm{
G\ar[d]^{R_{g}} \ar[r]^{p} & P\ar[d]^{R_{g}}\\
G\ar[r]^{p} & P
}
\]

\end{itemize}

\begin{remark}\label{chap11-rem11.2}
If $X\in \mathfrak{g}$, $X$ defines in natural way a section of the trivial bundle $P\times \mathfrak{g}$, $p\mapsto (p,x)$. Using the isomorphism $\psi$, we get a section of $T_{\pi}$, and hence a vertical vector field, denoted by $\sigma(X)$ and called the fundamental vector field on $P$ defined by $X$. If $X$, $Y\in \mathfrak{g}$, we have $\sigma[X,Y]=[\sigma(X),\sigma(Y)]$. From the commutativity of the diagram in the Proposition, we have: $(R_{g})_{*}\sigma(X)$ is the fundamental vector field corresponding to $\Ad(g^{-1})(X)\in \mathfrak{g}$.
\end{remark}

\section*{Alternating forms with values in a vector space}

Let $E$ and $F$ be finite dimensional vector spaces over $\mathbb{R}$. It is clear how to define an alternating $p$-form on $E$ with values in $F$. For instance a $2$-form is a bilinear map $f:E\times E\to F$ with $f(x,x)=0$ for $x\in E$. Let $F_{1}$, $F_{2}$, $F_{3}$ be (finite dimensional) vector\pageoriginale spaces and $\varphi:F_{1}\times F_{2}\to F_{3}$ a bilinear map. If $\alpha$ (resp. $\beta$) is a $p$ (resp. $q$) form on $E$ with values in $F_{1}$ (resp. $F_{2}$) we can define $\varphi(p+q)$ form on $E$ with values $F_{3}$, as on page \pageref{page13}, using $\varphi$ instead of multiplication in $\mathbb{R}$, (We denote this form by $\alpha\wedge_{\varphi}\beta$). For instance if $p=q=1$, $(\alpha\wedge_{\varphi}\beta)(X,Y)=\varphi(\alpha(X),\beta(Y))-\varphi(\alpha(Y),\beta(X))$ for $X$, $Y\in E$.

Consider the special case $F_{1}=F_{2}=F_{3}=\mathfrak{g}$, where $\mathfrak{g}$ is a Lie algebra and $\varphi:\mathfrak{g}\times \mathfrak{g}\to \mathfrak{g}$ is the map $\varphi(X,Y)=[X,Y]$, $X$, $Y\in \mathfrak{g}$. In this case we denote $\alpha\wedge_{\varphi}\beta$ by $[\alpha,\beta]$. Note that if $\alpha$ is a $1$-form with values in $\mathfrak{g}$, we have
\begin{align*}
[\alpha,\alpha](X,Y) &= [\alpha(X),\alpha(Y)]-[\alpha(Y),\alpha(X)]\\[3pt]
                     &= 2[\alpha(X),\alpha(Y)]
\end{align*}
We have 
\begin{itemize}
\item[(i)] $[\alpha,\beta]=(-1)^{pq+1}[\beta,\alpha]$

\item[(ii)] If $\gamma$ is a $r$-form with values in $\mathfrak{g}$,
$$
(-1)^{\pr}[\alpha,[\beta,\gamma]]+(-1)^{qp}[\beta,[\gamma,\alpha]]+(-1)^{rq}[\gamma,[\alpha,\eta]]=0.
$$
\end{itemize}

It is clear that we can define on a manifold $M$ smooth differential forms with values in a vector space (finite dimensional) $F$. If $\alpha$ is such a smooth $p$-form on $M$, we can define the exterior differential $d_{\alpha}$, which is a $(p+1)$ form with values in $F$. If we wish, we can define $d$ by the analogue of the formula in Prop.\ref{chap6-prop6.6}(p.\pageref{page30}) for $d$. In particular if $\alpha$ is a $1$-form with values in $F$, $d\alpha(X,Y)=X\alpha(Y)-Y\alpha(X)-\alpha[X,Y]$,\pageoriginale $X$, $Y$ being vector fields on $M$.

In the case $F=\mathfrak{g}$, a Lie algebra for differential forms $\alpha$, $\beta$, $\gamma$ with values in $\mathfrak{g}$ we have (i), (ii) above and 
$$
d[\alpha,\beta]=[d\alpha,\beta]+(-1)^{p}[\alpha,d\beta].
$$

\section*{The Maurer-Cartan form and equation}

We give an illustration of the above notion. Let $G$ be a Lie group with Lie algebra $\mathfrak{g}$. Then there is a canonical $1$-form on $G$ with values in $\mathfrak{g}$, called the Maurer-Cartan form on $G$. this form $\alpha$ is defined as follows. Let $v$ be a tangent vector of $G$ at $g$. There exists a unique left invariant vector field $X$ on $G$ with $X(g)=v$. Define $\alpha(v)=X$. Note that $\alpha$ is essentially the identity map on $T_{e}(G)$.

\medskip

\noindent
{\bf Proposition (Maurer-Cartan equation) \thnum{11.3}.\label{chap11-prop11.3}}

{\em We have $d\alpha+\dfrac{1}{2}[\alpha,\alpha]=0$.}

\begin{proof}
Let $X$ and $Y$ be left invariant vector fields. Noting that $\dfrac{1}{2}[\alpha,\alpha](X,Y)=[\alpha(X),\alpha(Y)]$, it suffices to show that $d\alpha[X,Y]+[\alpha(X),\alpha(Y)]=0$. But
\begin{align*}
d\alpha(X,Y) &= X\alpha(Y)-Y\alpha(X)-\alpha[X,Y]\\[3pt]
             &= -\alpha[X,Y],\text{~ as~ }\alpha(Y)=Y, \ \alpha(X)=X\text{~ are constants}\\[3pt]
             &= -[X,Y]=-[\alpha(X),\alpha(Y)].
\end{align*}
\end{proof}

\section*{Equivariant forms on principal bundles}
\pageoriginale

Let $\rho:G\to\Aut (F)$ be a finite dimensional representation of $G$. Let $\alpha$ be a $p$-form on $P$ with values in $F$. We say that $\alpha$ is equivariant (with respect to $\rho$) if $R^{*}_{g}\alpha=\rho(g^{-1})\alpha$\footnote[2]{$R^{*}_{g}\alpha$ is the inverse image of $\alpha$ by the map $R_{g}$, $p\mapsto pg$.}. (Here $\rho(g^{-1})\alpha^{-}$ is the $p$-form on $P$ defined by 
$$
[\rho(g^{-1})\alpha](X_{1},\ldots,X_{p})=\rho(g^{-1})\alpha(X_{1},\ldots,X_{p})\text{~ for~ } X_{1},\ldots,X_{p}
$$
tangent vectors on $P$. For $p=0$, the condition means $\alpha(pg)=\rho(g^{-1})\alpha(p)$, $p\in P$, $g\in G$, $\alpha$ being a function from $P$ to $V$). Since $dR^{*}_{g}=R^{*}_{g}d$, we see that if $\alpha$ is an equivariant $p$-form then $d$ is an equivariant $(p+1)$ form.

An equivariant $p$-form is said to be basic (``coming from the base'') or horizontal if $\alpha(X_{1},\ldots,X_{p})=0$ if at least one of the tangent vectors $X_{i}$ is vertical. Basic equivariant $p$-forms can be identified with the sections of the bundle ${\displaystyle{\mathop{\wedge}\limits^{p}}}T^{*}(M)\otimes F_{\rho}$ on $M$, where $F_{\rho}$ is the vector bundle on $M$ associated to the representation $\rho$. Hence such forms are also called $p$-forms {\em on $M$} with coefficients in the bundle $F_{\rho}$. In particular applying this to the case of the trivial $1$-dimensional representation of $G$, we see that a $p$-form (in the usual sense) on $M$ can be identified with a $p$-form $\alpha$ on $P$ which is invariant under the action of $G$ and which satisfies $\alpha(X_{1},\ldots,X_{p})=0$ if one of the $X_{i}$ is vertical.

Note that if $\alpha$ is basic, the form $d\alpha$, while being equivariant need not be basic.
