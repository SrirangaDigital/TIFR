\chapter{Tensor and Exterior Products}\label{chap3}

\section*{Tensor product}

Let $E$ and $F$ be finite dimensional vector spaces (over $\mathbb{R}$). The set of bilinear maps of $E\times F$ into $\mathbb{R}$ form a vector space denoted by $E^{*}\otimes F^{*}$; $E^{*}\otimes F^{*}$ is the tensor product of $E^{*}$ and $F^{*}$ ($E^{*}$ = dual of $F$. If\pageoriginale $\varphi\in E^{*}$ and $\psi\in F^{*}$, the map $(e,f)\mapsto \varphi(e)\psi(f)$, $e\in E$, $f\in F$ is a bilinear form on $E\times F$, denoted by $\varphi\otimes\psi$. If $\{\widetilde{e}_{1},\ldots,\widetilde{e}_{m}\}$ (resp. $\{\widetilde{f}_{1},\ldots,\widetilde{f}_{\ell}\}$) a base for $E^{*}$ (resp. for $F^{*}$) then the $\{\widetilde{e}_{i}\otimes \widetilde{f}_{j}$), $1\leq i\leq m$, $1\leq j\leq \ell$, form a base for $E^{*}\otimes F^{*}$.

We can define the tensor product of $E$ and $F$, $E\otimes F$ as the vector space of bilinear forms on $E^{*}\times F^{*}$. If $e\in E$ and $f\in F$, $e\otimes f$ is defined as an element of $E\otimes F$.

Let now $E$ and $F$ be vector bundles over $M$. Let $E\otimes F=\coprod\limits_{x\in M}E_{x}\otimes F_{x}$ with the natural projection $E\otimes F$ onto $M$. Then there exists a (unique) structure of a vector bundle on $E\otimes F$ with the following property : if $\{c_{1},\ldots,c_{m}\}$ (resp. $\{s_{1},\ldots,s_{\ell}\}$) is a frame for $E$ (resp. for $F$) over an open set $U$ then the set theoretic sections $\sigma_{i}\otimes s_{j}$ of $E\otimes F$ defined by $\sigma_{i}\otimes s_{j}(x)=\sigma_{i}(x)\otimes s_{j}(x)$ form a (smooth) frame for $E\otimes F$ over $U(1\leq i\leq m,1\leq j\leq \ell)$. The bundle $E\otimes F$ is called the tensor product of $E$ and $F$.

In a similar way the tensor product of a finite number vector bundles is defined. If $E$ is a vector bundle we shall denote by $\bigotimes\limits^{r} E$ the tensor product $E\otimes \cdots \otimes E$, $r$ times.

\section*{The tensor bundles and tensors}

Let $T(M)$ be the tangent bundle of $M$. The bundle $\{\bigotimes\limits^{r}T(M)\}\otimes \{\bigotimes\limits^{s}T^{*}(M)\}$ is called the tensor bundle of contravariant order $r$ and covariant order $s$. A section of this bundle is called a {\em tensor} of contravariant order $r$ and covariant order $s$. A section of $\bigotimes\limits^{r}T(M)$\pageoriginale \{resp. $\bigotimes\limits^{s}T^{*}(M)$\} is called a contravariant tensor of order $r$ (resp. covariant tensor of order $s$).

\section*{Exterior product}

Let $E$ be a finite dimensional vector space over $\mathbb{R}$. We put ${\displaystyle{\mathop{\wedge}\limits^{0}}}E^{*}=\mathbb{R}$ and ${\displaystyle{\mathop{\wedge}\limits^{1}}}E^{*}=E^{*}$ and call these spaces respectively the space of alternating $0$-forms and $1$ forms over $E$. Let $p\geq 2$ be an integer. {\em An alternating $p$-form} over $E$ is a multilinear map
$$
f:\underbrace{E\times \cdots \times E}_{p\text{~ times}}\to \mathbb{R}
$$
satisfying one of the following equivalent conditions :
\begin{itemize}
\item[i)] If $x_{1},\ldots,x_{p}\in E$ with $x_{i}=x_{i+1}$ for some index $i$ with $1\leq i<p$ we have
$$
f(x_{1},\ldots,x_{p})=0
$$

\item[ii)] If $x_{1},\ldots,x_{p}\in E$ with $x_{i}=x_{j}$ for a pair of distinct indices $(i,j)$ we have
$$
f(x_{1},\ldots,x_{p})=0
$$

\item[iii)] For each permutation $\sigma$ of the set $\{1,\ldots,p\}$ we have, for $x_{1},\ldots,x_{p}\in E$,
$$
f(x_{\sigma(1)},\ldots,x_{\sigma(p)})=\epsilon(\sigma)f(x_{1},\ldots,x_{p})
$$
where $\epsilon(\sigma)$ is the signature of the permutation $c$.
\end{itemize}

The set of alternating $p$-forms over $E$ form a vector space denoted by ${\displaystyle{\mathop{\wedge}\limits^{p}}}E^{*}$ and called the $p$-th exterior product (or power) of $E^{*}$.

If\pageoriginale $f\in {\displaystyle{\mathop{\wedge}\limits^{p}}} E^{*}$ and $g\in {\displaystyle{\mathop{\wedge}^{q}}}E^{*}$ consider the function $h:E^{(p+q)}\to \mathbb{R}$ defined by
$$
h(x_{1},\ldots,x_{p+q})=\sum\limits_{\sigma}\epsilon(\sigma)f(x_{\sigma(1)},\ldots,x_{\sigma(p)})g(x_{\sigma(p+1)},\ldots,x_{\sigma(p+q)})
$$
where $\sigma$ runs through the set of permutations $\sigma$ of $(1,\ldots,p+q)$ satisfying $\sigma(1)<\cdots<\sigma(p)$ and $\sigma(p+1)<\cdots<\sigma(p+q)$. Then $h$ is an alternating $(p+q)$ form called the exterior product of $f$ and $g$ and is denoted by $f\wedge g$.

We have
\begin{prop*}
\begin{itemize}
\item[\rm 1)] Exterior multiplication is associative i.e., if $f_{i}\in {\displaystyle{\mathop{\wedge}\limits^{p_{i}}}}E^{*}$ $i=1,2,3$, we have $f_{1}\wedge(f_{2}\wedge f_{3})=(f_{1}\wedge f_{2})\wedge f_{3}$.

\item[\rm 2)] If $f\in {\displaystyle{\mathop{\wedge}\limits^{p}}}E^{*}$ and $g\in {\displaystyle{\mathop{\wedge}\limits^{q}}}E^{*}$, we have
$$
f\wedge g= (-1)^{pq} g\wedge f.
$$

\item[\rm 3)] If $\{\widetilde{e}_{1},\ldots,\widetilde{e}_{m}\}$ is a base of $E^{*}$, the set $\{\widetilde{e}_{i_{1}}\wedge\ldots\wedge \widetilde{e}_{i_{p}}\}$ with $1\leq i_{1}<\cdots <i_{o}\leq m$ forms a base of ${\displaystyle{\mathop{\wedge}^{p}}}E^{*}$.
\end{itemize}
\end{prop*}

In particular if $m=\dim E$, ${\displaystyle{\mathop{\wedge}\limits^{m}}}E^{*}$ is one dimensional and ${\displaystyle{\mathop{\wedge}\limits^{p}}}E^{*}=0$ for $p>m$.

If $T:E_{1}\to E_{2}$ be a linear map between finite dimensional vector spaces. We then have the transpose linear map
$$
t_{T}(p):{\displaystyle{\mathop{\wedge}\limits^{p}}}E^{*}_{2}\to {\displaystyle{\mathop{\wedge}\limits^{p}}}E^{*}_{1}\quad\text{defined by :}
$$
for $f\in {\displaystyle{\mathop{\wedge}\limits^{p}}}E^{*}_{2}$,
$$
t_{T}(p)(f)(x_{1},\ldots,x_{p})=f(Tx_{1},\ldots,Tx_{p}),x_{i}\in E_{1}.
$$

We\pageoriginale have $\{{}^{t}T^{(p)}(f)\}\wedge \{{}^{t}T^{(q)}(g)\}={}^{t}T^{(p+q)}(f\wedge g)$ for $f\in {\displaystyle{\mathop{\wedge}\limits^{p}}} E^{*}_{2}$ and $g\in {\displaystyle{\mathop{\wedge}\limits^{q}}}E^{*}_{2}$.






