\chapter{Smooth Manifolds}\label{chap1}

\section*{Smooth Manifolds}\pageoriginale

Let $\Omega$ be an open subset of $\mathbb{R}^{n}$. A function $f:\Omega\to \mathbb{R}$ is said to be smooth (or $C^{\infty}$) if partial derivatives of all orders of $f$ exist and are continuous. A map $\phi:\Omega+\mathbb{R}^{m}$ is said to be smooth if, for $i=1,\ldots,m$, the function $p_{i}\circ \phi :\Omega\to \mathbb{R}$ is smooth, where $p_{i}:\mathbb{R}^{m}\to \mathbb{R}$ denotes the $i^{\text{th}}$ projection (i.e., if $\phi=(f_{1}\ldots f_{m})$ the functions $f_{i}$ are smooth).

Let $\Omega_{1}$ and $\Omega_{2}$ be two open subsets of $\mathbb{R}^{n}$. A map $\phi:\Omega_{1}\to \Omega_{2}$ is said to be a diffeomorphism if $\phi$ is bijective and both $\phi$ and $\phi^{-1}$ are smooth.

Let $M$ be a Hausdorff topological space, which we shall assume to be paracompact. Suppose that we are given an open cover $\{U_{\alpha}\}_{\alpha\in I}$ of $M$ and for each $\alpha$ a homeomorphism
$$
\phi_{\alpha}:U_{\alpha}\to \phi_{\alpha}(U_{\alpha})
$$
of $U_{\alpha}$ onto an open subset $\phi_{\alpha}(U_{\alpha})$ of $\mathbb{R}^{n}$ such that whenever $U_{\alpha}\cap U_{\beta}\neq \emptyset$, the map
$$
\phi_{\alpha}\circ \phi^{-1}_{\beta} : \phi_{\beta}\left(U_{\alpha}\cap U_{\beta}\right)\to \phi_{\alpha}\left(U_{\alpha}\cap U_{\beta}\right)
$$
is a diffeomrophism. We then say that $M$ has a structure of a smooth (or differentiable or $C^{\infty}$) manifold, or simply that $M$ is a smooth manifold (of dimension $n$).

The\pageoriginale pair $(U_{\alpha},\phi_{\alpha})$ is called a chart or a map and the family $\{(U_{\alpha},\phi_{\alpha})\}$ is called an atlas. We shall assume that the atlas we have is a maximal (or complete) atlas in the sense that we cannot add more maps to the atlas still preserving the compatibility conditions on the overlaps. Any atlas is contained in a unique complete atlas. 

If $(U_{\alpha},\phi_{\alpha})$ is a map, the functions $x_{i}=p_{i}\circ \phi_{\alpha}$ are called coordinate functions on $U_{\alpha}$.

\medskip
\noindent
{\bf Examples}
\begin{enumerate}
\item {\em The Euclidean, $p$-space $\mathbb{R}^{n}$.}

\item {\em The $n$-dimensional sphere}
$$
s^{n}:\left\{(x_{1},\ldots,x_{n+1})\in \mathbb{R}^{n+1}\sum\limits^{n+1}_{i=1}X^{2}_{i}=1\right\}
$$

Let $t=(t_{1},\ldots,t_{n})\in \mathbb{R}^{n}$ and $|t|^{2}=t^{2}_{1}+\cdots+t^{2}_{n}$.

Let $\phi^{-1}_{1}:\mathbb{R}^{n}+S^{n}$ be the map $t\to \left(\dfrac{2t}{|t|^{2}+1},\dfrac{|t|^{2}-1}{|t|^{2}+1}\right)$ and $\phi^{-1}_{2}:\mathbb{R}^{n}+S^{n}$ the map $t\to \left(\dfrac{2t}{|t|^{2}+1},\dfrac{1-|t|^{2}}{1+|t|^{2}}\right)$. Then $\phi^{-1}_{1}(\mathbb{R}^{n})=S^{n}-\{(0,\ldots,0,1)\}$ and $\phi^{-1}_{2}(\mathbb{R}^{n})=S^{n}-\{(0,0,\ldots,-1)\}$.

($(\phi_{1},\phi_{2})$ are stenographic projections). We have $\phi_{1}\circ \phi^{-1}_{2}:R^{n}-(0)\to \mathbb{R}^{n}-(0)$ is the map $t\to \dfrac{t}{|t|^{2}}$ and hence a diffeomorphism. Thus $S^{n}$ is a smooth $n$-manifold.

\item {\em Product of two manifolds :} If $M$ is a $m$ dimensional manifold and $N$ a $n$-dimensional manifold then $M\times N$ is in a natural way an $(m+n)$-manifold. If $\{(U_{\alpha},\phi_{\alpha})\}$ (resp. $\{(V_{\beta},\psi_{\beta})\}$ is an atlas for $M$ (resp. $N$), then the maps
$$
\phi_{\alpha}\times \phi_{\beta}:U_{\alpha}\times V_{\beta}\to \phi_{\alpha}(U_{\alpha})\times \phi_{\beta}(U_{\beta})\subset \mathbb{R}^{m+n}
$$\pageoriginale
give an atlas for $M\times N$.

\item From 2) and 3) we see that the $n$ {\em dimensional torus} $T^{n}{\displaystyle{\mathop{=}\limits_{\text{def}}}}S^{1}\times \cdots \times S^{1}$ ($n$-fold product) is a smooth $n$-manifold.

\item An open subset of a smooth manifold is a smooth manifold.
\end{enumerate}

\section*{Smooth maps}

Let $M$ be a smooth manifold. A function $f:M\to \mathbb{R}$ is said to be smooth if for each $\alpha$ the function $f\circ \phi^{-1}_{\alpha}:\varphi_{\alpha}(U_{\alpha})+\mathbb{R}$ is smooth.

Let $M$ and $N$ be two smooth manifolds and $\phi:M\to N$ a map. We say that $\phi$ is a smooth map if $\phi$ is continuous and the following condition is satisfied : let $\{(U_{\alpha},\phi_{\alpha})\}$ be an atlas for $M$ and $\{(V_{\beta},\phi_{\beta})\}$ an atlas for $N$; then for each $(\alpha,\beta)$ with $W=U_{\alpha}\cap \phi^{-1}(v_{\beta})\neq \emptyset$ the map
$$
\psi_{\beta}\circ \phi \circ \phi^{-1}_{\alpha} :\phi_{\alpha}(U_{\alpha})\to \psi_{\beta}(V_{\beta})
$$
is smooth.

If $\phi:M\to N$ and $\psi : N\to P$ are smooth maps then the map $\psi\circ \phi:M\to P$ is smooth.

A map $\phi:M\to N$ is said to be a diffeomrophism if $\phi$ is bijective and both $\phi$, $\phi^{-1}$ are smooth.

\section*{Tangent Vectors}
\pageoriginale

Let $m\in M$. A tangent vector $L$ in $M$ assigns to each smooth real valued function $f$ defined in a neighbourhood of $m$ a real number $L(f)$ satisfying the following conditions:
\begin{itemize}
\item[1)] If $f=g$ in a neighbourhood of $m$, we have $L(f)=L(g)$.

\item[2)] $L$ is linear over $\mathbb{R}:L(\lambda f+\mu g)=\lambda L(f)+\mu L(g)$, for $\lambda$, $\mu\in \mathbb{R}$ (Note : $f$ is defined in a neighbourhood $U_{1}$ of $m$, and $g$ in $U_{2}$, $\lambda f+\mu g$ is defined in $U_{1}\cap U_{2}$)

\item[3)] $L(f\cdot g)=L(f)\cdot g(m)+f(m)L(g)$.
\end{itemize}

(Thus $L$ is a linear map from the ring of germs of smooth functions at $m$, satisfying the leibniz rule).

It is easily seen that tangent vectors in $M$ form a vector space, denoted by $T_{m}(M)$ or simply $T_{m}$.

Let $m\in \mathbb{R}^{n}$. The map $f\mapsto \dfrac{\partial f}{\partial x_{i}}(m)$, for $f$ smooth in a neighbourhood of $m$, defines a tangent vector in $m$, denoted by $\left(\dfrac{\partial}{\partial x_{i}}\right)_{m}$. If $m=(m_{1},\ldots,m_{n})$, a smooth function $f$ with $f(m)=0$ can be written, in a neighbourhood of $m$, in the form $f=\sum\limits^{n}_{o=1}(x_{1}-m_{i})g_{i}$, where $g_{i}$ are smooth. Using this fact one sees that $\left\{\left(\dfrac{\partial}{\partial x_{i}}\right)_{m}\right\}$, $i=1,\ldots,n$ form a base for $T_{m}(\mathbb{R}^{n})$, which is thus a vector space (over $\mathbb{R}$) of dimension $n$. It follows (for example using the considerations in the next section) that if $M$ is an $n$-manifold, $T_{m}(M)$, $m\in M$, is an $n$-dimensional vector space.

\section*{The Tangent map}
\pageoriginale

Let $\phi:M\to N$ be a smooth map and $m\in M$. We now define a linear map
$$
T_{m}(\phi):T_{m}(M)+T_{\phi(m)}(N)
$$
called the tangent linear map (or differential) of $\phi$ in $m$. Let $v\in T_{m}(M)$. $T_{m}(\phi)(v)$ is defined by
$$
\{T_{m}(\phi)(v)\}(f)=v(f\circ \phi)
$$
where $f$ is a smooth function defined in a neighbourhood of $\phi(m)$, noting that $(f\circ \phi)$ is a smooth function in a neighbourhood of $m$. 

Let $\phi:M\to N$ and $\psi:N\to P$ be smooth maps. Let $m\in M$. We then have the {\em chain-rule} :
$$
T_{m}(\psi\circ \phi)=T_{\phi(m)}(\psi)\circ T_{m}(\phi)
$$
(The two sides are linear maps of $T_{m}(M)$ into $T_{(\psi\circ\phi)(m)}(P)$).

Using the chain rule we see that if $\phi:M\to N$ is a diffeomorphism, the map $T_{m}(\phi):T_{m}(M)\to T_{\phi(m)}(N)$ is an isomorphism of vector spaces.

Since the tangent space to $\mathbb{R}^{n}$ at a point is $n$-dimensional, we see, using a map at $m$, that the tangent space at a point $m$ of an $n$-dimensional manifold is of dimension $n$. (Note that $\phi_{\alpha}:U_{\alpha}\to \phi_{\alpha}(U_{\alpha})$ is a diffeomrophism).

We shall put a structure of a smooth manifold (in fact that of a `vector bundle') on the set of tangent vectors of a smooth manifold.





