\chapter{Frobenius Theorem on Integrable Sub-Bundles, Foliations}\label{chap8}

\section*{Sub and quotient bundles of a vector bundle}\pageoriginale

Let $E$ be a vector bundle of rank $m$ over $M$. Suppose that $F$ is a subset of $E$ which satisfies the following conditions:
\begin{itemize}
\item[(i)] for $a\in M$ the set $F_{a}=E_{a}\cap F$ is a vector subspace of dimension $p$ (fixed) of the fibre $E_{a}=\pi^{-1}(a)$ of $E$ over $a$.

\item[(ii)] every point $a\in M$ has a neighbourhood $U$ and a frame $(\sigma_{1},\ldots,\sigma_{m})$ of $E$ over $U$ such that for every $b\in U$, $\{\sigma_{1}(b),\ldots,\sigma_{p}(b)\}$ form a base for $F_{b}$.
\end{itemize}

Then $\pi/F:F\to B$ has a natural structure of a vector bundle of rank $p$ over $M$, for which $\{\sigma_{1},\ldots,\sigma_{p}\}$ are local frames. We call $F$ a sub-bundle of $E$.

Let $F$ be a subbundle of $E$. Let $E/F=\coprod\limits_{b\in M}E_{b}/F_{b}$. Let $\eta:E\to E/F$ and $\pi_{E/F}:E/F\to M$ the natural maps. Then $\pi_{E/F}:E/F\to M$ has a structure of a vector bundle for which $(\eta\circ \sigma_{p+1},\ldots,\eta\circ\sigma_{m})$ is a frame over $U$. The bundle $E/F$ is the quotient of $E$ by $F$.

\section*{Integrable sub-bundles of the tangent bundle : Integral Submanifolds}

Let $F$ be a subbundle of rank $p$ of the tangent bundle $T(M)$ of $M$. A submanifold $N$ of $M$ is said to be an {\em integral manifold} for $F$ if the canonical map $T_{b}i:T_{b}(N)\to T_{b}(M)$ maps the tangent space $T_{b}(N)$ of $N$ at $b$ isomorphically onto the fibre $F_{b}$ of $F$ at $b$, for each\pageoriginale $b\in N$ (Here $i:N\to M$ is the inclusion map)

\begin{defi*}
Let $F$ be a subbundle of $T(M)$. We say that $F$ is integrable (or completely integrable) if the following condition is satisfied : if $U$ is an open set of $M$ and $X$, $Y$ sections of $F$ over $U$, then $[X,Y]$ is a section of $F$.
\end{defi*}

($X$ and $Y$ are considered as sections of $T(M)$ i.e., as vector fields and $[X,Y]$ is the bracket of vector fields which is a section of $T(M)$ and we require that $[X,Y]_{b}\in F_{b}$ for $b\in U$).

\begin{remark*}
Let $F$ be a subbundle of $T(M)$. We then have a section $\Omega$ of $({\displaystyle{\mathop{\wedge}\limits^{2}}}F^{*}\otimes (T(M)/F))$ i.e., we have for $b\in M$, an alternating bilinear map $F_{b}\times F_{b}\to (E/F)_{b}$. This is defined as follows: Let $v_{1}$, $v_{2}\in F_{b}$ and let $X$, $Y$ be vector fields (in a neighbourhood of $b$) with $x(a)=v_{1}$, $X(b)=v_{2}$. Define $\Omega(v_{1},v_{2})=\eta_{b}[X,Y]_{b}$, where $\eta_{b}:T_{b}\to T_{b}(M)/F_{b}$ is the natural map. One checks that $\Omega$ depends only on $v_{1}$, $v_{2}$ and not on the extensions $X$ and $Y$. This section $\Omega$ may be called the {\em curvature} of the subbundle $F$. Thus $F$ is integrable if and only if its curvature is identically zero.
\end{remark*}

\section*{Frobenius Theorem}

\begin{theorem}\label{chap8-thm8.1}
The following conditions on a subbundle $F$ of $T(M)$ are equivalent.
\begin{itemize}
\item[\rm(1)] Through every point $a\in M$, there is an integral submanifold for $F$.

\item[\rm(2)] $F$ is integrable.

\item[\rm(3)] Each point $a\in M$ has a neighbourhood $U$, a diffeomorphism $\varphi:U\to V\times W$,\pageoriginale where $V$ and $W$ are open subsets of $\mathbb{R}^{p}$ and $\mathbb{R}^{n-p}$ respectively, such that for each $w\in W$, the set $\varphi^{-1}\circ p^{-1}_{W}(w)$ ($p_{W}:V\times W\to W$ the projection) is an integral submanifold of $F$.
\[
\xymatrix@=1.2cm{
U\ar[r]^-{\phi} & V\times W\ar[d]^-{p_{W}}\\
             & W
}
\]
(In other words, around each point $a$ there exists a system of coordinates $(x_{1},\ldots,x_{n})$ for $M$ such that the submanifolds $x_{p+1}=c_{p+1},\ldots,x_{n}=c_{n}$, where $c_{p+1},\ldots,c_{n}$ are constants, are integral submanifolds of $F$.)
\end{itemize}
\end{theorem}

\begin{proof}
(3) $\Rightarrow$ (1) is trivial.
\end{proof}

To prove (1) $\Rightarrow$ (2) we use

\begin{lemma}\label{chap8-lem8.2}
Let $f:N\to M$ be a smooth map. If $X_{1}$ (resp. $X$), a vector field on $N$ (resp. $M$) we say that $X_{1}$ and $X$ are $f$-related if for each $b\in N$ we have $T_{b}(f)(X_{1}(b))=X(f(b))$. Suppose that $X_{1}$ and $Y_{1}$ vector fields on $N$ and $X$, $Y$ vector fields on $M$ such that $X_{1}$ is $f$-related to $X$ and $Y_{1}$ is $f$-related to $Y$. Then $[X_{1},Y_{1}]$ is $f$-related to $[X,Y]$.
\end{lemma}

To deduce (1) $\Rightarrow$ (2) from the lemma we take $f$ to be the inclusion $i:N\to M$, $X_{1}=X|_{N}$, $Y_{1}=Y|_{N}$.

The essential part of the proof is to show that (2) $\Rightarrow$ (3). This is done in two steps. First one proves

\begin{lemma}\label{chap8-lem8.3}
Suppose\pageoriginale that $F$ is integrable. Each point $a\in M$ has a neighbourhood of $U$ and a frame $\{X_{1},\ldots,X_{p}\}$ for $F$ over $U$ such that $[X_{i},X_{j}]=0$ for $1\leq i\leq p$, $1\leq j\leq p$.
\end{lemma}

The proof of this lemma is essentially algebraic. One then proves

\begin{proposition}\label{chap8-prop8.4}
Let $X_{1},\ldots,X_{p}$ be a vector fields in a neighbourhood $U$ of a such that $\{X_{1}(b),\ldots,X_{p}(b)\}$ are linearly independent in $T_{b}(M)$ for each $b\in U$ and such that $[X_{i},X_{j}]=0$ in $U$, for $1\leq i\leq p$, $1\leq j\leq p$. Then there exists a coordinate system $(x_{1},\ldots,x_{n})$ in a neighbourhood $V$ of a such that $X_{i}=\sigma/\partial x_{i}$, $1\leq i\leq p$ in $V$.
\end{proposition}

\section*{Sketch of proof of the proposition}

Choose a coordinate system $(y_{1},\ldots,y_{n})$ around $a$ with $y_{i}(a)=0$ and such that $X_{1}(a),\ldots,X_{p}(a)$, $(\partial/\partial y_{p+1})_{a},\ldots,(\partial/\partial y_{n})_{a}\text{~span~} T_{a}(M)$. Let $\varphi^{i}_{t}$ be the local flow associated with $X_{i}$, $1\leq i\leq p$. For $\delta>0$, let $\Omega=\{(t_{1},\ldots,t_{p},y_{p+1},\ldots,y_{n})\big| |t_{i}|<\delta,|y_{j}|<\delta\}$. For $\delta$ sufficiently small, the map $h:\Omega\to M$
$$
h(t_{1},\ldots,t_{p},y_{p+1},\ldots,y_{n})=\varphi^{1}_{t_{1}}\circ,\ldots,\circ\varphi^{p}_{t_{p}}(0,\ldots,0,y_{p+1},\ldots,y_{n})
$$
is well-defined. It is easy to check that
$$
T_{0}(h)\left((\partial/\partial t_{i})_{0}\right)=X_{i}(a)\text{~ and~ } T_{0}(h)\left((\partial/\partial y_{j})_{0}\right)=(\partial/\partial y_{j})_{a},
$$
so that $h$ is a diffeomorphism in a neighbourhood $V$ of $0$. Next one shows that, for $x\in V$,
$$
T_{x}(h)\left((\partial/\partial t_{i})_{x}\right)=X_{i}h(x),\quad 1\leq i\leq p.
$$
To prove this one uses the fact that $\varphi^{i}_{t}$ and $\varphi^{j}_{t}$ commute for $1\leq i\leq p$, $1\leq j\leq p$ as\pageoriginale $[X_{i},X_{j}]=0$ (see p.\pageref{page26} Corollary \ref{chap6-coro6.5.1}).

The implication (2) $\Rightarrow$ (3) is clear from the Proposition.

\section*{Foliatoin}

Let $F$ be an integrable subbundle (of rank $p$) of $T(M)$. We define a new topology on $M$ by declaring that a subset of $M$ is open if it is a union of integral submanifolds for $F$. With this topology $M$ becomes, in a natural way, a $p$-dimensional manifold, which we denote by $M_{F}$. A connected component of $M_{F}$ is called a maximal integral manifold (or a leaf) of $F$. The leaves define a partition of $M$, called the foliation of $M$ defined by $F$. If $N$ is a leaf, note that the inclusion map $i:N\to M$ is an injective immersion; in general $N$ need not be a submanifold of $M$.







