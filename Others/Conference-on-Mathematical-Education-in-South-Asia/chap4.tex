\chapter{Present-Day Problems in English Mathematical Education}

\begin{center}
  {\em By}~ T. A. A. BROADBENT
\end{center}

\medskip

\blfootnote{This address was given at the South Asian Conference on Mathematical Education held on 22-28 February 1956 at the Tata Institute of Fundamental Research, Bombay.}
\setcounter{pageoriginal}{56}
The\pageoriginale main educational problem in England at the moment is one which is common to many countries. How best shall we draw up a curriculum which is to be applicable to all and yet stimulating to the specially able : can we implement the policy of equal opportunity and yet avoid the sterility of equal achievement : can we frame a scheme to meet the needs of the average pupil without introducing ``restrictive practices'' which may depress the intelligent to the level of the mediocre : can we cater for the mass and still train the leaders ?

This problem is made at once more difficult and less difficult by the fact that England has not now and never has had a {\em system} of education. (Scotland must be excluded from this assertion). I do not put this forward as meritorious, or as blameworthy---it is simply a fact, an historic fact, with certain inescapable cunsequences. Difficulties are increased because we have no uniform basis from which to start; they are a little diminished because the absence of a rigid system is an aid to flexibility. Unfortunately, for me it means that I cannot hope to make the present position clear without some reference to past history.

One hundred years or so ago, before the advent of anything which could be called {\em national} education, we had a number of schools which, beginning as charitable foundations intended to serve the immediate locality, had developed into boarding schools, drawing their pupils from the wealthier classes all over the country, controlled by a board of governors who might be regarded as the heirs to the trustees of the original charity, closely linked with, sometimes dominated by, the Established Church. From these schools,\pageoriginale the two universities, Oxford and Cambridge, drew their students. For it must be remembered that these were the only two universities in England until 1830, and the development of the new universities did not become really fruitful until about 1900. Schools which remained localised, and schools which were founded by bodies dissenting from the established church, kept a vocational character giving training to the children of the lower middle class, while the boarding schools, becoming known by a curious perversion of language as the ``public'' schools, sent many of their pupils on to Oxford, and Cambridge, from whence they passed into the professions---the Chruch, the law, politics. Thus education beyond the age of 12 or 13 tended to become the privilege of the few, and took its tone from the entrance requirements of the two ancient universities. During the middle and later years of the 19th century, the number of these ``public'' schools increased, and in addition, many of the large town grammar schools began to retain pupils to the age of 17 or 18. Thus the entrance requirements of Oxford and Cambridge, necessarily low in standard, exercised an increasing control over the school curriculum.

In geometry, the prescription of certain parts of Euclid's {\em Elements} as a university entrance requirement arose largely because this was the easiest way of expressing a uniform requirement in universally understood terms, and hence the domination of Euclid in the schools stemmed to some extent from the need for regulation which should be easily and widely understood. But however valuable Euclid may be for the young professional mathematician in the making---and many of my friends have assured me that their interest in mathematics was fist aroused by the logical stimulus which Euclid can give---on the whole the study of Euclid generally amounted to rote-learing of the most repulsive and undesirable type; diagrams and proofs were committed to memory, and reproduced---more or less---from memory. In 1870, some public-school teachers founded the Association for the Reform of Geometrical Teaching, chiefly to protest against this degrading and uninstructive tyranny. Success was long in coming, for the alteration of a university regulation\pageoriginale is a long and tedious process, even where substantial agreement in principle exists; and in this case there was at first no measure of agreement. Cayley indeed maintained, perhaps half in jest, that if Euclid was to be replaced, it should be by the study of $n$-dimensional geometry; from this the novice might eventually descend to the special case of $n=2$. But in the early years of the 20th century, the universities agreed to accept any geometrical sequence which showed some degree of logical connection.

At about the same time, a strong movement was set on foot for the development of the teaching of vocational and practical mathematics. This sort of teaching, which had been quite wide-spread in the early 19th century in the small grammar schools, had declined as more and more schools began to look to the older universities as the ultimate aim for their pupils. Perry, who had a genius for publicity, was perhaps the best-known figure in this movement; his pupils were not nourished on the school and university classics to Todhunter and his kin, but on Molesworth's {\em Pocket book fo engineering formulae}, and his class room was an engineering laboratory and workshop.

Though the two movements were not originally connected, they were, almost unwittingly, converging on a single objective, the creation of a mathematical curriculum designed for the average pupil. On the one hand, Euclid might be good training for the bright boy, but was deadly for the ordinary pupil; on the other, academic mathematics was out of touch with the everyday affairs of real life. The two movements had their weaknesses. Euclid offered a reasonably logical chain of theorems, and the first result of abandoning this chain was a state of chaos : a flood of substitutes was poured on to the market, each with its own plan. What was a theorem in one system was an axiom in another, an admirable training no doubt in mathematical logic, but beyond the understanding of the average boy, particularly if in the course of his education he moved from one school to another ! The Perry movement, on the other hand, emphasised the purely utilitarian aspect of mathematics to the exclusion of all other values and was thus largely\pageoriginale responsible for the quite mistaken estimate of the place of mathematics in education which is current in England today. Further, the reform movement in geometry also laid much stress on this aspect, with the result that mathematics came to be regarded solely as a science or as a technical skill, which it is not; indeed it was often contended that mathematics should be taught as an experimental science in a laboratory, a proposal sometimes put into practice with surprising and not entirely happy results.

Those not familiar with the English system or lack of system of education may well ask at this point, ``But what was the Minister of Education doing in all this ?''. To answer this question, we must first remember that neither the subjects of the curriculum, nor the allocation of time to these subjects, nor the way in which they are to be taught, is laid down by the Ministry. The Ministry will coordinate and advise, it will inspect and report, through its trained Inspectorate, but it is only a partner in the work of national education, which is shared between the Ministry, the local education authorities, the managers and governors of the schools, and the teachers themselves. The Ministry has no power and no organisation to set examinations. The most important public examination, the General Certificate of Education, has a general plan established by the Ministry in consultation with representatives of the schools and universities. The detailed organisation of this examination is in the hands of a number of examination boards, originally established, mostly by the universities, rough about 1900, to meet the growing demand for recognisable and authoritative certificates of performance in a public examination. There are eight such boards, loosely supervised by the Ministry; papers are set and marked by panels of examiners recruited both from the schools and from the universities. The boards have certain regional affiliations, but as a rule the school takes the examination under whichever board seems to cater most suitably for the needs and aspirations of the particular school. A school in the North will not necessarily take the examination set by the Northern Universities Joint Board; much depends on the type of school and much even on the whim of the headmaster.\pageoriginale These examinations have three levels in each, known as ``O'', ``A'' and ``S'' respectively. The ``O'' or Ordinary Level is intended to be suitable for the average secondary school pupil of age about 16; any number of subjects may be taken, from one upwards, and a certificate will be given to show simply those subjects at which the candidate has achieved a Pass Level. The ``A'' or Advanced Level will be taken usually two years after the Ordinary Level, and as a rule fewer subjects will be offered by the candidate. A reasonable performance at ``A'' level will normally qualify a student for acceptance by a university, though it will not necessarily secure him a place therein. The ``S'' or Scholarship Level is an examination designed for the boys and girls of high intelligence who will specialise in one or two subjects; here a good performance will gain for the candidate a State Scholarship or Local Education Authority award to enable him to pursue a course of higher education, usually at one of the universities.

It will be clear that the universities thus exercise considerable influence on the school curriculum, directly and indirectly, through the General Certificate of Education; directly since these examinations serve as the door to the universities; indirectly, since university teachers are well represented on the examining boards and on the panels which prepare and mark the examination papers. It is therefore not surprising that the General Certificate is sometimes criticised on the grounds that it is too academic, and too much influenced by university policy; that there is some substance in this criticism is perhaps shown by the fact that a ninth examining board is being constituted to give schools the option of obtaining a General Certificate for their pupils on a course more technical and more directly utilitarian than is at present customary.

The General Certificate is a post-war development, but the examinations do not differ very much in substance from the pre-war examinations, known as those for the School Certificate, which were set by the same examining bodies. Differences are so far largely administrative and organisational, rather than in specific content. In tracing developments during the present century, therefore,\pageoriginale we need not distinguish between the School Certificate and the General Certificate; the latter is merely an extension and improvement on the former.

I may now return to the two movements for reform, that which began with the idea of abolishing Euclid in the schools, and that which looked to practical mathematics as its sole purpose, the Perry movement. These, I have said, were converging, perhaps involuntarily, on a single objective, the provision of a mathematical curriculum suitable for the average pupil, which should nevertheless not handicap the brighter child. Even by 1914, some keen-sighted teachers were agreeing that this could only be achieved by treating mathematics as one subject, not as a collection of isolated topics. But round about 1930, while the average teacher was still far from taking this comprehensive view, he was, from what we may almost call an administrative standpoint, beginning to call for a new deal which may ultimately go far towards completing the work of the two earlier movements and rectifying their errors.

Under the form of examination then current for the 15 or 16 year old pupil, most of these might be expected to reach a pass standard in each of three papers, one on arithmetic, one on algebra, one on geometry. Thus a weakness in geometry, common with girls and by no means unknown among boys, would gravely prejudice the pupil's examination prospects. Many teachers, naturally if perhaps regrettably more concerned with examination results than with educational progress, saw that three mixed papers, each with a due proportion of arithmetic, algebra and geometry, would give a much better chance to pupils weak in one topic, say geometry, than three separate papers each devoted solely to one of these topics. This somewhat sordid consideration came to the support of those teachers who, on better pedagogical grounds, were beginning strongly to protest against the traditional separation of mathematics into unrelated subjects. Today it may be a little difficult t realise how complete that separation often was. Even within the last 40 years, one can find instances; algebraic methods would be prohibited in the arithmetic lesson, a problem in geometry would have to\pageoriginale be solved without the use of coordinates if the lesson were called ``pure geometry'', while if the lesson were called ``coordinate geometry'' the same problem might have to be solved, but this time by algebraic methods only, no help or stimulus from so-called ``pure geometry'' being permissible. Calculus could not be used in dynamics, kinematics must not intrude into geometry.

This is perhaps not quite so ludicrous as it sounds; for it could happen that the  sharpness of the prohibition served as a guard on the concentration of the less able pupil; the novice may well need to be taught the use of one tool at a time. But the disadvantages are clear and beyond compensation : the good workman must have a large collection of tools, he must be the master of each, and, above all, he must be able to select the tool for the job. This quality of ``appropriateness'', so valuable to the mathematician, was pushed into the background by the separation of mathematics into a system of water-tight compartments. Moreover, watertight compartments can be very dry ! The desiccated topics were apt to wither early, and leave nothing behind save an unsightly and sterile residue.

We now envisage the unified course in mathematics as the present-day ideal in the school curriculum. Mathematics, in the class-room, in the text-book, in the examination, and beyond that, in adult life, is not to be thought of as a bundle of uncorrelated topics, but as a unity, a mode of thought, a universal language. Just as the good teacher of English does not separate writing from reading, prose from verse, grammar from fluency, but sees his subject as a whole, a mode of communication of thought, an end with a variety of means, so the good teacher of mathematics will see his subject as a whole, again a mode of communication of thought through a variety of means. For the outstanding teacher, this governing principle is enough and he will plan his own course; but the average teacher will depend very much on the guidance of books. In the last ten years or so, there have been numerous attempts to supply his needs; at first, such texts were, so impatient critics maintained with some show of justice, merely chapters culled from earlier texts on\pageoriginale arithmetic, algebra, geometry and trigonometry, sandwiched together between the covers of a single volume and presented to the unwary as a ``unified course.'' Today this is no longer true and we have some books, only a few as yet but still some, which see mathematics, at least in its earlier stages, as a unity.

I have suggested that this completes the efforts of the two reform movements of some fifty years ago, and this, I believe, is true, though the men who led those movements at that time, if they were still with us today, might not agree with me. The academic reformers wished to abolish Euclid as a school text, but they did not wish to deprive geometry of its paramount position in school mathematics; yet this is what has happened. Nevertheless, the present situation is a logical completion of their underlying principle, that school mathematics should be related to the needs and capacities of the pupil. This was Perry's underlying principle, too, and we can claim to be completing that movement, though we cannot accept the narrow and one-sided application of it which loomed so large in Perry's view. We do not believe that the justification of mathematics can be found entirely in the field of engineering and technology, but we do believe that the pupil who has been taught to see mathematics as a whole, as a natural and indeed inevitable language for the communication of abstract and rational thought, will find it a natural and indeed inevitable way of thinking about engineering and technological problems.

So much for the present. What of the future ? Here we can see two questions. First, the unified course so far has been studied mainly in connection with the examination at ``O'' level, for the 16-year old; what are we doing about the later school work ? Secondly, what implications does the unified course hold for the position of mathematics in the school curriculum ?

As regards the first question, while some teachers are not yet convinced that the unified course is an outstanding improvement, majority opinion is in its favour and looks to an extension of its principles to the later school years. Here we should expect the calculus\pageoriginale to be a central theme, while to this some of our more resolute reformers would add the introduction, in small and gentle doses, of some of the elementary ideas of modern algebra which are proving so potent in research mathematics, both pure and applied. Some teachers would found the calculus on kinematical notions, others would prefer an approach through graphical work, but no one would deny that kinematical ideas must enter at an early stage and occupy a prominent position, thus holding out a hand to geometry. In geometry, the old distinction between pure geometry and coordinate geometry will disappear---it has almost disappeared now---and will be closely linked on the one hand with calculus through kinematics, on the other with algebra through determinants and matrices and linear transformations. In this way, the gap which has made itself apparent of recent years between school geometry and geometry at the universities will be closed or at least narrowed. We do not expect or intend that our Sixth-formers should tackle, for instance, the formidable volumes of Hodge and Pedoe, but we hope that when they do meet that work, they will realize that, in spite of its appearance, it does deal with geometry ! Dynamics will follow naturally on the kinematical content of the calculus; this will meet with the approval of the old-fashioned people, of whom I am one, who believe that a sound grasp of dynamical principles is essential for progress in applied mathematics and mathematical physics. And apart from this, it will have two good effects. First, statics will recede into its proper place, well in the background, and may even become statics and cease to be merely a disguised form of unpleasant and unilluminating algebra and trigonometry. Secondly, the old Victorian idea that applied mathematics is an unladylike topic with which no nice girl should have any acquaintance will receive its final death-blow not before time.

Perhaps more important is the impact of the new unified course on the general problem of the place of mathematics in the school curriculum.

The new outlook may well give us confidence to meet a challenge which is likely to face us in England quite soon, a challenge which is\pageoriginale old enough in fact, but which may now be expected to present itself with a more determined front. Does mathematics need and deserve the large proportion of school time which is at present given to it ? As far as the future mathematician, physicist or technologist is concerned, the answer is not in doubt and a strong affirmative reply would hardly be denied by any competent judge. But what of the child whose future is not likely to be along such lines ? The large majority of our pupils will have no occasion in their adult careers to use anything beyond the simpler processes of arithmetic; they may wish to count their change, they may even wish to follow or to dispute the calculations of the tax collector; but no more. To them, algebra, geometry, calculus have no utilitarian appeal. Why then should they study these subjects ? At one time, their devotion---their compulsory devotion---to mathematics was defended, by their teachers, on the ground that mathematics, more than any other school subject, trains the mind. I wish I could believe this ! But observation of my mathematical friends and contemporaries, and some small investigation into the biographies of mathematicians have forced me to the conclusion that we cannot honestly make such a claim. Mathematics is no more---and no less---a mind training, in the general sense, than the study of the classical or the modern languages, no more---if no less---than is the study, the intelligent study, of history or geography.

What then are we to do ? Are we to allow mathematics to fade into the background of the curriculum for the average child, the ordinary pupil not likely to specialize in the subject ? Not, I think, if we pay a little more attention to what mathematics {\em is}, and perhaps a little less to what mathematics {\em does}. Let us remember that the aim of the school is not merely to provide vocational training, but education, not at all the same thing as vocational training. Our pupils are going to be lawyers and plumbers, fishermen and doctors, housewives and journalists; yes, but for us, the teachers, this is less important than the fact that they are all going to be units in a world civilization. They are the heirs to that civilization, in a brief moment they will be the living cells and structure of\pageoriginale that civilization, before transmitting it to their successors. We cannot bring the child to a full knowledge of his inheritance, but we can and we must try to make each child understand that the inheritance is there waiting for him, and further, to make sure, if we can, that on coming into his estate he will not squander it nor neglect it, but will appreciate it and perhaps even enhance it. I take this to be axiomatic; for example, I want the man who like myself has small Latin and less Greek nevertheless to be aware that much of what we do or think today is conditioned by what the Sumerians and the Greeks thought 2000 or more years ago. I want our main school education to give to our pupils the keys to the whole of modern civilization.

But this admitted, the case for mathematics is strong, overwhelmingly strong, stronger by far than when based on grounds of mere utility. As the underlying foundation of so much of our present-day science and technology, and, fully as important, as a great creative art, a universal language, a basic mode of thought, the claim that mathematics is an integral part of modern culture is one which can hardly be disputed. It may be that this claim is sometimes greeted with laughter. What, say the critics, would you seriously contend that the Lebesgue integral has as great and as deep an appeal and place in our culture as, say, Paradise Lost, or the Vatican Aphrodite. We might retort that counting heads is a poor way of estimating the value of a work of art. But there is a better answer, namely, a bold affirmation that I do believe that as many people can and do appreciate the Lebesgue integral as appreciate Paradise Lost, for by this appreciation I mean to exclude all those people who will glibly tell you that Milton is a great poet and Paradise Lost a great poem, though they have never read a single book of that epic, they never intend to read a book of it, nor would they understand a line of it if they did. No; take measures of the two fields of informed appreciation, and I am sure they will not differ by very much.

How is this to help us with our pupils ? There is no simple, automatically applied recipe---and it would be a bad thing if there were. But\pageoriginale it is easy to see where to begin. We must begin by convincing ourselves and seeing clearly in our own minds that mathematics is deeply woven into the culture of our present-day civilization; we must make sure that we have grasped this fact, that we passionately believe it to be true, that we can illustrate it with a multiplicity of detail, from science and technology, indeed, but also from architecture and law, from history and poetry. We must see the 18th century, that century of law and order, of cool rationalism, as the immediate consequence of the Newtonian philosophy. We must see the calculus, not as the gradient of a graph, but, at least in Western Europe, as the inevitable concomitant of the change from the middle ages to the modern world, from the static view of a universe fixed and immutable in its habits to the dynamic view of a changing, developing, growing complex of phenomena.

Let the teacher see mathematics as a whole, as a unity which pervades and underlies the whole structure of our present-day life in action and in thought, let him convince himself of the rightness of this view by studying its detailed implications in every field of human endeavour to which he can gain access, and he need not fear that his pupils will dislike mathematics nor that in their adult life they will shun it and forget it.

\bigskip
\medskip
{\fontsize{9pt}{11pt}\selectfont
Royal Naval College, Greenwich
}
