\chapter{Some Crucial Problems of Mathematical Instruction}

\begin{center}
{\em By}~ MARSHALL H. STONE
\end{center}
\medskip

\blfootnote{This address was given at the South Asian Conference on Mathematical Education held on 22-28 February 1956 at the Tata Institute of Fundamental Research, Bombay.}
Mathematical\pageoriginale education is entering into a critical period. There are three major factors which operate to produce the crisis : the rapid expansion of mathematical knowledge itself; the impressive penetration of mathematical thinking into the most diverse branches of learning and technology; and the universal desire to establish mass education at the primary and even the secondary levels. Teachers of mathematics can already sense the rising pressure on them to teach more mathematics to more young men and young women, at every level of education. They cannot fail to recognize that they must deal in their future classes not only with greater numbers of students but also with a wider spread of natural mathematical ability and an expanding circle of individual interests. The situation may differ in degree, but hardly in kind, from one to another of the various regions of the world. The problems which are of special interest or urgency in South Asia all have their counterparts in Europe and the Americas. The search for solutions, even though it must be influenced by the special conditions obtaining in different areas or different countries, is one in which all mathematicians have a common vital interest. It is thus no accident that mathematicians from widely separated parts of the world are brought together in a regional conference such as this. Each of us can contribute something and each of us learn something in the discussions which take place here. Those who have a special responsibility for guiding the development of mathematical instruction in South Asia will know what to take away with them from this Conference in order to attack their special problems more effectively. 

It\pageoriginale will be good if we can, from the beginning, recognize quite frankly the essential nature of the problems with which we have to deal. We must see that they are inseparable from the progressivism of Western culture, now in the process of spreading to the entire world. Most cultures of which we have anything like an adequate knowledge appear to have been strongly conservative and essentially inimical to rapid, radical, or extensive change. The cultures of Ancient Egypt and Ancient China, notable for their stability and their millennial vigor, offer conspicuous if extreme examples of this general observation. On the other hand, human societies do not exhibit those extraordinarily rigid patterns found in certain insect societies : in even the most conservative of them cultural change takes place slowly and in small quanta, with cumulative effects which eventually can be identified as significant. Greek culture and its derivative, modern Western culture, stand out by contrast because of their eager and systematic search for new ways of thinking and doing. This characteristic progressivism is expressed materially in the advances of modern technology, but it is essentially an intellectual phenomenon, as can be seen in the growth of scholarship and pure science in the Western world. The progressive spirit inspires the search for new knowledge and its application to human affairs of every kind; and, in order that the application may be made systematically and on a large scale, it leads logically to the introduction of general mass education. Thus the three factors at the root of the coming crisis in mathematical education are seen to be nothing else than aspects of progressivism.

The universality of this crisis reflects the universal desire to transplant some of the most highly characteristic elements of Western culture into the other cultures of the world. Clearly, we are in the midst of a kind of cultural revolution, which began when contacts between Europe and the rest of the world became numberous and increasingly significant, after about 1500. Three continents---North America, South America, and Australia as well as the northern part of Asia are now inhabited by populations predominantly European\pageoriginale in culture, and in race as well if exception be made for certain portions of Central America, South America and Northern Asia. While most regions of Asia and Africa have received very small numbers of European settlers, they have been subjected with few exceptions to European political control or influence for periods of varying length, during which many features of Western culture have been accepted from within or imposed from without. In the East this political control or influence is now on the point of vanishing save in areas heavily settled by Europeans; but Asia, having noted the advantages of Western technology, is more rather than less eager to adopt large segments of Western culture. Many decisions which were taken by an independent Japan in the nineteenth century are thus being repeated, mutatis mutandis, by other Asian countries in the twentieth. In consequence science and education, as understood in the West, will be intensively cultivated during the coming decades in all parts of Asia. In particular, the crucial problems of future mathematical instruction are as vital for the Asian nations as for the other nations of the world.  

It is surely easiest to direct attention first to the way in which these problems emerge at the uppermost levels of education. In the universities of the world it has become urgently necessary to extend and diversify the instruction in mathematics, without prolonging unduly the periods of study required by our future technologists, scientists, and mathematicians. At the top we must plan to produce more thoroughly trained research workers in all branches of the mathematical sciences, pure mathematics included. The greatest risk we run in our approach to these problems is that of being tempted by what I shall term ``the technological fallacy''---the mistaken belief that the instruction in mathematics and the natural sciences should be aimed primarily at the satisfaction of the demands of modern technology. History suggests that a healthy technology cannot be maintained without a continual vigorous probing of Nature's varied mysteries and a deep desire to understand the intricate workings of Nature's laws. One might even say that a reliable measure of the technological vigor of a country is\pageoriginale given by the vigor of its mathematical research. It is thus extremely imprudent for any nation to embark upon a program of education which would neglect, hamper, or discourage scientific and mathematical research in its universities. The dangers of such a course are by no means entirely avoided by accepting the compromise which would consist in confining thorough mathematical training to those future research workers deemed to have a particularly clear need for it while offering abbreviated practical instruction to those destined to become technological specialists. This compromise would provide a convenient spring-board for gradually reducing support to pure science and at the same time would burden the faculty with an every multiplying diversity of specialized mathematical courses of a strictly practical character. In fact one of the subtler forms of the technological fallacy is expressed in the thesis that the mathematics taught to prospective technologists should be adjusted to the special requirements of each branch of technology, corresponding courses in such subjects as algebra, statistics, and the calculus being offered specifically for architects, or engineers, or chemists, or pre-medical students, or social scientists, and so on. In my own opinion the wisest plan is to offer sound basic mathematical instruction for all and to aim at a much more effectively intergrated use of basic mathematics in the technical courses offered under the various scientific departments. The basic courses in mathematics should not dwell unduly or prematurely on mathematical and logical niceties nor should they include any very large amount of material primarily of interest for advanced pure mathematics. Since the central portions of elementary university mathematics are drawn from analytic geometry, the calculus, and modern algebra, it is not difficult to design a satisfactory nucleus of basic courses and to build around it a group of more advanced courses among which the future specialist may choose according to his needs and his desires. However, it is a good deal harder to bring about the close co-ordination of courses in the various sciences with the basic courses in mathematics. In a rather long university career, I have never seen such co-ordination\pageoriginale attempted, let alone achieved. It is not at all unusual for the Departments of Engineering, Physics, Chemistry, Economics, and so on to lay down certain mathematical prerequisites after quite inadequate consultation, and then to expect the Department of Mathematics to adjust its curriculum so that their students may meet these requirements without inconvenience. It is, however, altogether too rare for any of these departments to ensure thereafter that the mathematical prerequisites are properly reviewed, utilized, and supplemented in its own courses. Adequate coordination requires very close and uninterrupted contact between departments and complete willingness to work towards the unification of the mathematical training given to students under different departments. This is a point which, I believe, deserves special emphasis in any report on current problems of higher mathematical instruction. Another aspect of university education which must be emphasized is the obligation of the faculties to avoid unnecessarily prolonging the period of study demanded of future technologists, scientists, and mathematicians. The situation in American medicine should serve to warn us of the danger that the period of preparation may become too long. In the United States it is common for a doctor to have lived more than half of his expected life-span before he is ready for the independent practice of medicine. Indeed, the future doctor, after completing four years at the university and four more in medical school, still has to spend a year or two as a hospital intern and, in all probability, a further two years in medical service with the military establishment. It is inevitable that future science students must expect to require more mathematics in their special fields and must devote more time to acquiring the necessary mathematical background. But it behooves us to save them time for what is really essential by eliminating whatever is unnecessary or secondary, by modernizing what we intend to teach, and by improving the effectiveness of our teaching. The most important step in the last direction is to analyze, rearrange, and recast the material which is found to offer the greatest difficulties to our students, until we succeed in presenting\pageoriginale it in the terms most suggestive to the intuition. We cannot expect a net saving of time because we must simultaneously enrich and enlarge our course offerings for undergraduates and post-graduates by the addition of new or expanded treatments of many different topics in pure and applied mathematics. What time we contrive to save for the student can be invested to broaden and deepen his mathematical understanding; and we should continually urge his teachers in other fields to save time for him too by making systematic, well-planned use of the mathematical knowledge and skill he is able to acquire in our class-rooms.

\newpage

At this time I do not wish to elaborate upon the kind of curricular revision implied by the preceding remarks. Nevertheless, it may be useful for me to make a few specific observations illustrative of what I have in mind. In the United States it is a general practice to teach trigonometry and elementary algebra beyond quadratic equations to the first-year university classes, wasting a great deal of the student's time and stifling his native interest in mathematics by drilling him in dull manipulations of little eventual practical use in either pure or applied mathematics. Quite generally, in Europe as well as in America, higher algebra is still taught without benefit of the insights gained in modern approaches to the subject. Our teaching of these parts of mathematics cries out for excision, modernization and re-organization. Analytic geometry and the calculus are traditional elementary courses in which we can certainly alter some of our traditional teaching practices to great advantage. For example, in analytic geometry we fail to introduce and utilize the important vector concept, despite the simplifications and the valuable insights which it affords, because we find it difficult to teach. Not only do we thus commit a sin of omission, but we also bring it about that perforce our mathematical students first learn about vectors from clumsy and unsatisfactory treatments essayed by teachers of physics. My own experience suggests that with a little ingenuity and patience the vector concept can be taught effectively, and that the students themselves appreciate a good presentation, finding the application of\pageoriginale vectors to three-dimensional geometry even more rewarding than that to the plane. Needless to say, the student who acquires and elementary mastery of vectors at the beginning of his analytic geometry course can at once make effective application of them in his elementary physics course. Incidentally, I might remark that, if the discussion of trigonometry can be held off until the geometry of the circle has been discussed by vector methods, the whole subject can be vastly simplified, condensed, and illuminated. In the calculus likewise a careful analysis will disclose the advantages to be reaped by abandoning or altering some of the tradition-hallowed ways of treating the subject. If the mean-value theorem is developed early and given the central role it deserves, it becomes possible to simplify and to motivate more clearly the introduction of the definite integral and the demonstration of Taylor's theorem with remainder. By being somewhat more careful and precise in our use of terminology, we can avoid many bothersome confusions in the minds of our students. For example, the standard use of the terms ``integral'' and ``integration'' to refer to two totally distinct concepts, those of the definite integral and the indefinite integral, before any effective connection has been established between the two, virtually guarantees that the student will be confused as to the meaning and significance of the fundamental theorem of the integral calculus. It is easy enough to remedy this particular defect and others like it; but it would be a valuable contribution to our elementary teaching to discover and treat them systematically. Passing on to courses beyond the basic ones, it may be said at once that in geometry, higher algebra, and mechanics, we need to modernize and expand the material taught; that in statistics and the mathematics of computation we need to introduce new courses wherever they do not exist; and that we need to study and reorganize the whole structure of post-graduate instruction in pure mathematics with a view to ensuring that the major results of mathematical research over the last fifty years are brought within the reach of candidates for the Master's degree and within the grasp of candidates for the Doctor's degree. Those who venture to take up these tasks must have\pageoriginale open minds and a broad knowledge of the present state of mathematics. The tasks themselves are not too difficult; but, as we have learned at the University of Chicago, they must be worked at over a rather long period of time if difficulties and imperfections are to be eliminated. One practical point needs to be appreciated in the consideration of post-graduate mathematical instruction---namely, that under normal circumstances there can be only a limited number of university centres, even in a very large region like Western Europe or the United States, at which a fully developed post-graduate curriculum and research program can be offered. Accordingly it becomes important in guiding the development of the university system to decide where such centres should be created and where post-graduate instruction in mathematics should be limited to less ambitious goals.

What we can do in the universities has to be based on what is being or will be done in the primary and secondary schools. It is therefore essential that the schools should give adequate preparation to future university students, as well as that they should train large numbers of students who may have no desire to enter the university. The ideal of universal mass education, therefore, poses the problem of ensuring that the interests of the future university student shall not be sacrificed to the requirements of the majority or to a blind attachment to egalitarian principles. The United States provides the unfortunate example of a country which for at least twenty-five years has made this sacrifice and is just now beginning to realize, in some measure, the damage which has been done. Educationally it is clearly wrong to postpone all serious mathematical training to the university level, as some egalitarians would like us to do, because so many valuable formative years are thereby lost beyond recall. At the age of twelve or fourteen the young student's aptitude for mathematics is surely as high as it will ever be, and it should be cultivated by suitable mathematical training rather than stifled by neglect. Yet is must be admitted that the provision of such training presents a practical problem difficult for the small isolated school to solve. Perhaps the solution lies in bringing pre-university students from\pageoriginale the small schools together in larger central schools endowed with residential facilities. In countries where students habitually leave home at an early age, this solution is the one which is likely to be adopted. The alternative would be to provide the necessary teachers even in the smaller schools, despite the increased cost per student which would have to be borne by the school system as a whole. In these remarks we have implied our belief that pre-university students should receive separate training in mathematics. Obviously they should emerge from it with the ability to analyze, attack, and solve problems of reasonable difficulty in algebra and geometry. They should have acquired some feeling for numerical and geometrical magnitudes, some facility in expressing themselves in mathematical terms, and some practice in the art of abstraction. For them mathematics should already be a general demonstrative science rather than a collection of useful rules and formulas. In short, these students should already have been helped to take the great leap from Babylonian to Greek mathematics. In this day and age it may be suggested that in addition they should already have learned the rudiments of statistical reasoning, a suggestion which involves some definite innovations in secondary school mathematics. Furthermore, they should have learned in all these subjects some of the concepts and insights recognized as important for modern mathematics. As for the majority of secondary school students, they have a much greater need than ever before of a good working knowledge of arithmetic, simple algebra, and practical geometry---perhaps also of rudimentary statistical techniques---because these subjects are being used more and more extensively in business and industry as these activities are conducted in technically advanced countries. The United States, strangely enough, again offers an unfortunate example despite the trend of technological development there. In fact an honest contemporary survey of business and industrial requirements would undoubtedly show that in the United States the general student is almost as badly neglected in terms of his mathematical preparation for a career in business or industry as is the pre-university student for his higher studies in mathematics and\pageoriginale the sciences. The Asian countries should certainly not allow themselves to be misled into taking the American school as a paragon or into accepting uncritically the precepts of American educationists with regard to secondary education. Even if general mathematics in the primary and secondary schools is launched for practical reasons in a somewhat limited initial form, provision should be made for expanding its content so that it will never lag for behind the requirements of an expanding commerce and industry. The aim should be always to teach more mathematics to more students until the point is reached where every young person will learn enough to enjoy an unrestricted choice of a career within the limitations set by his native talents. In conclusion we may mention still another aspect of secondary school training in mathematics which deserves more attention than it usually receives. This is the question of continuity of instruction. It seems to me important that the pre-university student in particular should have a continuous experience with mathematics, either pure or applied, throughout his secondary school career. The alternative, which is generally followed in the United States, is to interrupt mathematical training at certain points, with unfortunate consequences for the student who later has to resume the study of mathematics in the university.

What can be done in the secondary schools depends in turn upon what is done in the primary schools. The most significant single observation to be made about primary instruction in mathematics is that it hands over the majority of its pupils to the secondary school with an abiding distrust, even a deeply ingrained fear, of mathematics. Since most human beings take a kind of innate delight in riddles and puzzles of all sorts, the explanation of this phenomenon must lie not in the nature of mathematics so much as in the manner of its teaching. Without denying the sincerity, ingenuity or persistence of the efforts made to improve the teaching of primary school mathematics, we have to be honest with ourselves in confessing that so far we have chalked up a resounding failure. It would be the counsel of despair to urge the postponement to the secondary school level of all but the barest rudiments of mathematics, in\pageoriginale the hope that the problem could be more easily handled there. This step would at best merely displace the problem, but might in practice be found to aggravate it. In any case the postponement, for other reasons, would surely be even more disastrous than the postponement of the present secondary mathematics to the university. On the other hand, there are many promising innovations in the teaching of elementary mathematics, like those successfully introduced by Dr. Catharine Stern\footnote{Professor G. Choquet, in a special lecture given at the Conference on February 26, 1956, described the methods of M. Cuisenaire, which are identical in principle and in most details with those developed independently by Dr. Stern. The materials and manuals for Dr. Stern's methods are also available commercially, and have been used in many private and public schools in the United States.} in certain American schools, and new psychological insights into the learning process, like those due to Professor Piaget, which may inspire us with lively hopes of succeeding brilliantly where in the past we have failed. I have no doubt that the key is to be found in a better understanding of the psychology of the child and the adolescent, so difficult for the untrained or unobservant adult to grasp. With a sound knowledge of the pertinent psychological principles, corresponding teaching methods of practical value in the context of mass education can then be developed and elaborated. At the same time the content of primary instruction in mathematics needs to feel the influence of the requirements and insights of modern mathematics. The discovery of conspicuously better teaching methods would inevitably have the added advantage of opening up the possibility of teaching more and more varied mathematics even at the primary level, and the choice among these opportunities should hardly be left to psychological or educational experts ignorant of the nature and uses of modern mathematics. What is indicated in these circumstances is a cooperative study of the primary curriculum by psychologists, educationists, and mathematicians. I should like to see the International Commission on Mathematical Instruction, an organ of the International Mathematical Union, do its best to promote such a study. The most obvious suggestion to be made about the primary curriculum in mathematics is that in addition to arithmetic it\pageoriginale should include a substantial amount of intuitive and practical geometry, a subject which now clearly suffers in many school systems because of its almost total neglect at the primary level. School work intended to stimulate, direct, and develop the child's natural geometrical interests in such a way that his grasp of spatial relations and his ability to express himself in geometrical terms are systematically built up could begin in the early years and would provide a useful basis for later secondary school work. Perhaps the detailed elaboration of such a program in primary school geometry would be a peculiarly appropriate object of cooperative study at this time.

Having discussed in broad terms the problems of mathematical instruction from the top down, we may summarize our views by retracing the argument from the bottom up. In my opinion, an ideal system of mathematical instruction would take the child at his entrance into school and give him continuous mathematical training and experience up to the point where it is appropriate and advantageous for him to terminate his mathematical studies, whether this be at the primary, secondary, or university level. At the primary level, and to some extent at the secondary, instruction should be based on the most skilfully devised pedagogical methods and should be aimed at a good intuitive and practical knowledge of arithmetic, rudimentary algebra, and geometry. This portion of the mathematical curriculum should be designed as an integral part of mass education. At the secondary level a curricular differentiation should be made between pre-university and terminal students. The pre-university student should receive continuous secondary mathematical training in algebra, geometry, and the elements of statistical reasoning, designed to give him a fairly high degree of mathematical proficiency within a circumscribed mathematical domain and to enable him to proceed rapidly with further more exacting mathematical study at the university level. In the university, mathematical instruction should be modernized, enriched, unified, and skillfully graded; but it should avoid the specialization of basic courses for the benefit of specialized groups of students. It\pageoriginale should culminate in the strongest possible kind of post-graduate and post-doctoral research activity.

The establishment of such a system of mathematical instruction as this in any particular country would certainly involve a good deal of adaptation to special local conditions. It would also involve the resolution of many practical difficulties, particularly those of a financial order. It may be debated whether the practical obstacles to be overcome would be greater in a country where the entire educational system has to be developed virtually {\em ab ovo}, or in a country where educational commitments and traditions are already firmly laid down. In any case the system I have tried to describe in outline does not exist anywhere in the world to-day, except as an ideal, and its realization would cost both time and effort in any country which might wish to make it a reality. If we mathematicians desire the development of any such ideal system, we cannot rest content with merely publishing a general sketch of the scheme we would like to see adopted. We shall have to devise our scheme in detail, making careful studies of its various component elements; we shall have to explain and justify it to our fellow educators and to the public; we shall have to struggle with the various human and material obstacles which may be opposed to its introduction. For all this we must organize ourselves and plan effective, co-ordinated measures. We must organize ourselves to study our educational problems in detail; we must organize ourselves to study our educational problems in detail; we must organize ourselves to call public attention to the changes which we desire to have made; and we must organize ourselves to carry out in an effective way the decisions designed to implement our proposals. The national mathematical and educational societies, and their international counterparts, such as the International Mathematical Union, should serve our needs in these respects. Finally, let me remind you that whatever we may do we must leave room for future progressive changes. It must not be our aim or our desire to saddle future generations with rigid systems of instruction against which they, in their turn, must rebel.

\bigskip
\medskip

{\fontsize{9pt}{11pt}\selectfont University of Chicago}\relax




