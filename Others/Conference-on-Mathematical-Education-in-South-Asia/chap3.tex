\chapter{Teaching in Secondary Schools and Research}

\begin{center}
{\em By~} GUSTAVE CHOQUET
\end{center}

\blfootnote{This address was given at the South Asian Conference on Mathematical Education held on 22-28 February 1956 at the Tata Institute of Fundamental Research, Bombay.}
\setcounter{pageoriginal}{44}
Since\pageoriginale a few years, people are gathering, everywhere in the world, in conferences like this one, and becoming aware of a slow transformation which is taking place in the world. Of course, every conference is concerned with special questions relative to some particular countries---and this one is particularly important because it concerns many millions of men and women--- but I think that a part of our task is also to discover and point out some universal aims.

In fact, the problems which have led us to assemble here do not concern only mathematics; we are witnessing a revolution, the birth of a fundamental discovery : we are beginning to understand one of the essential features of man.

Machines have helped us to understand what is man; they have proved that they would assume many tasks which were formerly considered as characterizing human ability : they have assumed successively the work of muscles, locomotion, speech and memory, calculation, decision among several choices, translation of languages, etc.

Therefore man has become aware that he has in himself something that machines have not; he has become aware that, above all, he is a {\em creator}. He is now conscious that his human dignity consists in the fact that no bound can be assigned to him; he feels able to transcend any bound.

Man is building up himself everyday, maybe our hereditary cells are the same as those of men living millions of years ago (although we are not sure of that), but what is growing and changing everyday\pageoriginale is the sum of concepts created by man. We do not know what modifications are taking place in our brain when we are acquiring a new notion, but the difficulty we feel, the efforts we have t make, tell us that our brain is rebuilding itself, restructuring itself.

So, if we agree that the dignity of man consists in his thought, we see that he is in permanent evolution and progress. This progress consists in the acquisition of new notions; if this process was stopped, man would become like a mere machine.

Let us see what such a discovery implies for education. There exists a conception of education which has deep roots everywhere; it consists in believing that there is a certain amount of things which should be taught, and that schools should be like museums where skilled custodians should retain the attention of children and show them the inheritance of the past.

I think that now the time has come to change that conception and introduce a new axiom as a basis for education. If our human dignity consists in our creative ability, it is precisely this ability which should be developed in children. In other words our teaching should be no more the teaching of a museum custodian, but that of a creator.

Every teacher should kill the old man in himself and find out that he can be a discoverer and that his pupils also can be discoverers, and in fact much more powerful than himself.

And now let us look around in our respective countries. What do we see ? With rare exceptions, teaching has stiffened and our colleges have become museums. In Europe I think that the worst situation occurs in secondary schools, but universities are not free from that museum conception; the best hopes come from primary schools because teachers are in closer contact with children, and also because many studies concerning abnormal childhood have aroused interest for psychology of children, and have stressed the huge creative powers of children.

I\pageoriginale will formulate our axiom as follows :
\begin{quote}
``Education must be the son of growing science, and not its distant relative.''
\end{quote}

Of course, many programmes should be modified; in our existing programmes, there are monsters and horrible cancers, grown on a dying old body. But let us not blind ourselves; let us not think that a change of programme is sufficient to rejuvenate our teaching : we know very well that even if we could change the arteries of an old man, that would not give him youth; it might be better, if we want to renew his vitality, to give him a new hope, a new love.

Our teaching must have its new hope; and then monsters and cancers will disappear. Our axiom does not mean that we should merely introduce the results of modern science in our teaching : that again would amount only to a change of programme. Our axiom means that every teacher should be no more a custodian in a museum, even of a museum of modern art; it means that he should become himself a creator, even at a low stage, and try to find out and arouse creativeness in his pupils.

In other words, what should be changed essentially is the conception of the role of teachers.

Of course, I know the pessimist answers to such a requirement : a teacher is not always a creative genius, he must obey programmes; he is burdened by minor tasks (administration, etc.); his salary is too low so that he must have another job; his pupils are exceptionally dull, etc.

I do not want to minimize these difficulties; some of them are most important, and there solution is the first duty of a State and of a good Administration.

A teacher must be free of material difficulties; he must live a decent life and be given leisure to think and work for himself.

But these necessary requirements would amount to nothing if he had not the love of teaching, the love of the child mind, the taste to discover creative powers in children.

Now,\pageoriginale if you me how to acquire that love, I must confess that there is no sure method for that. However, we might try and find our the best conditions to develop it.

You know that sometimes we are bored by our daily life; we should like to travel a bit but we have not energy enough to escape; then the best thing to do is to go to the station; then to take a ticket; once we are in the train, our blood circulates more quickly, and the taste for adventure is born.

All teachers should take their ticket for adventure; and their starting point could be the study of modern mathematics. Modern mathematics is an extraordinary new world.

To teach mathematics without knowing what is modern mathematics would be to act like a museum curator who has in his museum some precious old paintings and refuses to know that there exists a modern school of painting, thinking that everything has been said and painted in the past.

So we should advise every teacher to study modern mathematics; it has such a beauty that love will come, and then they will communicate that spark of love to their pupils.

When we are teaching a given topic of which we have already a clear idea, nothing is better than to rediscover it and find out its links with other topics. Newly discovered material is like nascent atoms of hydrogen which are eager to combine with other atoms.

And now what are the practical consequences of those considerations ? What can we do to have better teachers ? Let us consider only teachers for secondary schools (colleges, high schools). They are formed in universities or in ``normal schools''. They should be given there, as professors, very good mathematicians who have done original work in a large field; those professors should give them an account of the most important structures of modern mathematics, with very precise definitions and show them---eventually by using a bit of history---that mathematics is not a dead thing, that it is changing and growing, and that it is within the scope of everybody to influence its growing.

The\pageoriginale last year of that teaching might be given to practical training in a high school under the guidance of an experienced professor, to the study of child psychology and problems concerning the process of learning, to the mathematical study of topics closely related to what these students will have to teach (foundations of geometry, notion of orientation, elementary algebra, statistics) and finally to a study of the use of mathematics in modern science and modern life.

But experience shows that even such a good beginning is not enough, and that teachers should be given help after their university time.

One of the duties of mathematicians should be to maintain a close contact with teachers of secondary and primary schools, by books in which they would inform them, in an adequate language, of the general trend of mathematics, of its new applications to science; where they would study, on concrete examples, the consequences of new discoveries and new notions for elementary teaching. The contact should be kept also by lectures given at regular intervals, or concentrated in time, and given during colloquia of several weeks every year.

Pedagogic research should be encouraged; every teacher has in his class a rich living material which enables him to make pedagogical experiments : they could discuss these experiments several times a year, at first in small, and then in larger meetings. A monthly publication written by teachers themselves with the collaboration of mathematicians and psychologists would be very helpful.

On an international scale, conferences on education should also be organized more often. When one remains in one's country, one is not easily aware of what is wrong in it; but it will become obvious when he discusses with colleagues of other countries.

Every scientist should be aware that he is responsible for education in his own country. As soon as there is no more a living link between teaching and research, or in other words between teaching and\pageoriginale universities, teaching becomes a dead thing; it stiffens, and cancers begin to grow, and of course, as a result research itself suffers. On the other hand, in every country, those who administer education at the secondary and primary levels should not be jealous of their independence, and should facilitate and encourage those connections.

Old teachers can also have an important role to play; their age gives them power and consideration; now many of them, even among those who have been considered as very gifted, do not want to understand that mathematicians can teach something to them; as they have worked hard to improve details in their teaching, they think they have attained perfection; moreover as already their brain is stiffened they are unable to acquire new notions and may be led to think sincerely that a {\em fortiori} their pupils would be unable to grasp them, so that they become easily enemies of every reform. They should be explained, and convinced by experience, that children's minds easily grasp new notions, and that modern mathematical notions were not invented as a pure abstract game, but are a synthesis of older and more complicated notions, and finally make possible a great economy of thought.

Let us now study briefly the characteristic features of modern mathematics and the way they can be used in elementary education. 

What interests us here in the huge development of mathematics in these last fifty years is not so much new results, new theorems, as the synthesis which was made.

During these last decades, the effort of many mathematicians consisted in discovering and studying the fundamental structures of mathematics : equivalence relations, order relations, algebraic structure, vector spaces, topology, metric spaces, differentiable manifolds, measure spaces, etc. many of which appear already, as in a germ, in the most classical mathematical entity, the set of real numbers. The importance of these structures is due to several reasons. One of them is that the simplest of them, equivalence relations,\pageoriginale order relations and even group structures, seem to correspond to the structure of our brain, as it results from the psychological studies of Piaget---and that implies that we should make a more extensive use of those structures in the teaching of children, even of very young children. Another reason of their importance is that one finds them everywhere in mathematics ! A very thorough study has been made of each of these structures separately. That study has realized a great economy of thought because now when a mathematician encounters a difficulty which involves several different structures, he is often able to divide that difficulty into minor ones relating to only one particular structure. Moreover every progress made in the study of one particular structure will imply a progress in all questions where that structure comes into account.

Those fundamental structures can be compared to those machines which can manufacture a single kind of object, but are able to manufacture a large number of copies within a short time, and moreover manufacture each of them with a high degree of perfection. Young mathematicians nowadays should spend at first much time in the study of the big machines of mathematics, that is to say of the fundamental structures, but then they get their reward : the theorems that they get do not concern any more a single mathematical entity but a large class of functions, of spaces, and moreover their proofs have a character of great elegance, simplicity and economy.

The broadening of the domain covered by the theorems of modern mathematics is not only interesting because it gives more general results, but also because it provides a better knowledge of each of the mathematical entities. It can be easily understood by this simple comparison : we understand better the meaning of a word or of a sentence if we know the whole page from which it has been taken.

That broadening should be considered as a fundamental principle even in the teaching of elementary mathematics. I shall give here a few examples to make precise what I mean :
\begin{itemize}
\item[a.] A\pageoriginale line tangent to a circle was at first defined as a line which meets that circle at only one point; but its true properties were fully understood only when it was defined as a limit of secant lines. Moreover this new and more dynamic definition would now be applied to any one, convex or not.

\item[b.] Given two points $A$, $B$ in the plane, the set of points $M$ such that $A\widehat{M}B=\pi/2$ can be better studied if at the same time we study the set of points such that $A\widehat{M}B<\pi /2$ or $A\widehat{M}B>\pi/2$.

\item[c.] The notion of continuity of a function $f$ defined on [0, 1] is better understood by children if we compare $f$ with approximating functions of a more simple type, for instance, piecewise linear.

\item[d.] Let us consider two triangles $ABC$, $A'B'C'$, with sides $a$, $b,\ldots,c'$ and angles $A$, $B,\ldots,C'$. A known theorem says that~:
$$
(a=a';~ b=b';~ C'=C)\to -c=c'.
$$
But that result can be much improved and at the same time simplified if we replace it by the following theorem~:

In any triangle $ABC$, $c$ is a function $f(a,b,C)$ of $a$, $b$, $C$, and is a strictly increasing function of $C$ for $0\leq C\leq \pi$.

\item[e.] A new method for teaching geometry, called intuitive geometry is being developed in Belgium, Italy, Netherlands; it rests partly on the same idea. For instance they start from a simple configuration, let us say a circle or a cone, and they study the family of all its plane sections.
\end{itemize}

There is another consequence of the general idea that every mathematical concept should be studied, not alone, but within its surroundings, with its various variations; it is the implication of that idea on pedagogical methods. It has been noted by psychologists and teachers that children very often can grasp more complicated situations that simple, sketchy situations; and indeed young children are always delighted when they are offered toys or when they are told stories where some apparently complicated structures appear; series of cubes that they put one inside the other; pictures representing\pageoriginale a box on which is painted a smaller box, on which is painted another and so on; stories involving arithmetic progressions, or even geometric progressions of ratio equal to 2, etc.

I made experiments with my younger children, where they were led to the concept of plane convex, and non-convex, domain. Then I asked them to draw the shortest curve between two points of a non-convex domain. They were very interested, found at last the solution, and during several days filled their copy-books with more and more complicated drawings of such domains.

I play often with them at the following game : I must think of some integer $x$ and then perform operations starting with $x$ such as doubling, addition, subtraction that they propose to me. I tell them the integer I get at last and they must find out $x$. That is exactly the solution of a linear equation; but they do not know it, and they are never tired of the game.

In the same trend of ideas, Professor Turan suggested to me in a private talk that in elementary classes, the study of linear functions which seems very dull to children---and with some reason---should be preceded by the introduction of the general idea of function, and should be treated together with the study of other functions, in particular those obtained by using $|x|$, for instance $f(x)=|x+2|-2|x-1|$; the great variety of graphs obtained in this way is a great excitement for children's minds.

It was often said that mathematics is a language; in that sense, you could say that modern mathematics is a beautiful and a very precise language; teachers should know that language and teach it by degrees to their pupils.

Even when they use only a small part of that language (for beginners) each term should be defined properly and used adequately.

I have made a survey of some textbooks of geometry and algebra for colleges and high schools, and I was horrified to see what definitions were given sometimes of straight lines, of equality, of oriented figures, of limits. Some terms are never defined; for example nobody knows\pageoriginale what is a figure, a triangle, an angle. A figure is a vague entity which means sometimes a set of points, sometimes a set of subsets, or the set of all subsets which can be obtained, starting from a given set, by un-defined operations. After such a confusion one should not be astonished that children refuse to understand mathematics. After all, for a strictly logical mind, such textbooks are not understandable, and the miracle is that some children can make something out of them.

The beginning of algebra---not to speak of the beginning of geometry---is often very confused in textbooks, and many teachers do not know exactly what must be given as a definition and what must be given as a theorem; later on there is often confusion between the notions of commutativity and associativity. Children are very sensitive to that confusion, which could be easily avoided nowadays, as there exist rigorous and elementary treatments of the beginnings of algebra.

It would take a long time to analyse in detail the contents of programmes in various countries and, as I said before, I think that programmes have a secondary importance, and that it would be better that every teacher could choose it, within certain limits, of course.

But I should like to underline at least that existing programmes have a certain tendency to contain many topics which are not essential.

In French programmes for instance, I should be glad to remove, in elementary algebra, the lengthy discussions concerning the position of a number with respect to the roots of a second degree polynomial. The time lost in such discussions could be given to stress the study of variation of various functions.

In geometry too much purity is dangerous. Classical euclidean geometry is a very beautiful theory, but a great economy of thought could be made once we know the theorem of Pythagoras. Many properties of circles or of conic sections could then be obtained by\pageoriginale using the regular methods of analytic geometry. That is true {\em a fortiori} in 3-space, which should be studied mostly using analytic geometry and vectors.

The same remarks could be made concerning the lengthy treatment which is given sometimes to the foundations of projective geometry : it seems nowadays that a projective space $P_{n}$ should be defined as the space of all 1-dimensional subspaces of euclidean space $R^{n+1}$.

By using old and obsolete treatments just because of usage or because they are elegant, much time and energy is lost, which could be used for more essential studies. For instance, I never saw any study of convex sets in elementary textbooks, so that students have sometimes studied in great detail very complicated properties of triangles and circles and have never been told that they were convex, although the definition and elementary study of convex sets is remarkably simple and has far-reaching consequences.

Of course, a teacher who knows what are the fundamental structures of mathematics, and has understood what is really useful for higher mathematics or for applied mathematics, will not make such mistakes as I have mentioned before. But he should carefully avoid the temptation to think that what he has learnt at an advanced age does not concern young children; children are ready to understand concepts such as equivalence relations, order relations (not necessarily total order); they should not lose opportunities to understand them in all cases where these concepts are underlying.

\bigskip
\medskip
{\fontsize{9pt}{11pt}\selectfont
University of Paris}\relax
