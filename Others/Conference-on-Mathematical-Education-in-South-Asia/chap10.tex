\chapter{Typography and the Teaching of Mathematics}

\begin{center}
{\em By}~ T. A. A. BROADBENT
\end{center}

\blfootnote{This address was given at the South Asian Conference on Mathematical Education held on 22-28 February 1956 at the Tata Institute of Fundamental Research, Bombay.}
\setcounter{pageoriginal}{146}
We\pageoriginale are well accustomed to habitual sneers at the ``narrow'' specialist; yet the true specialist, who has slowly and with labour attained to a clear and profound understanding of one field of human activity, may be trusted as a rule to have at least a sympathetic insight into the general mode of thought of his contemporary workers in other domains. Thus may I be permitted to recommend to your attention the work of a distinguished French historian, Marc Block, entitled {\em Apologie pur l'Histoire}, available also in an English translation, and particularly to direct you to two passages, which I shall quote, in which he seems to me to have summed up, without perhaps being fully aware of it, the functions and character of mathematics : 
\begin{quote}
``...each science has its appropriate aesthetics of language. Human actions are essentially very delicate phenomena... Properly to translate them into words and, hence, to fathom them rightly (for can anyone perfectly understand what he does not know how to express?), great delicacy of language and precise shadings of verbal tone are necessary'':
\end{quote}
and again
\begin{quote}
``...the first tool needed by any analysis is an appropriate language; a language capable of describing the precise outlines of the facts, while preserving the necessary flexibility to adapt itself to further discoveries and, above all, a language which is neither vacillating nor ambiguous''.
\end{quote}

Mathematics is a phenomenon of human action, a phenomenon of the human mind, infinitely more complex and varied than those of\pageoriginale the human body; it requires great delicacy and precision, and is sterile unless it can be communicated from one mind to another, a communication which demands the perfect understanding, of which Bloch speaks, and therefore demands a mastery of expression. In fact, it would not be any great exaggeration to say that mathematics does not need but is a language, of exactly the type described by Bloch in his second passage---capable of precise descriptions, flexible enough to adapt itself to and even to point the way to further discoveries, sharp, definite and unambiguous. Of all languages, mathematics is best fitted for the communication of abstract rational thought; and by reason of its nature, it is a universal language---possibly the only really efficient universal language.

Further, mathematics is to a very large extent a written language, and, specially, very much a printed language. Many of us learn best through the eye; and we depend on print to supplement memory. The mathematician's workshop, his laboratory, is a library; pencil, paper, books are his test-tubes and Bunsen burners, his raw material and his re-agents.

But, if we are prepared to agree that mathematics is essentially a language, the universal language of abstract logical thinking, and that it is to a very large extent a printed language, then, two sets of implications deserve further study.

First, if mathematics is a language, then, it must be taught as a language, serving both for use and for enjoyment, and we must be prepared to study the methods whereby other languages are taught, and in particular, the methods used to teach a language which is not the mother-tongue of the pupils. The mathematical prodigy, of course, may require special treatment, but few teachers are likely to be so fortunate as to number many mathematical prodigies among their pupils; indeed, if we should be so fortunate, let us thank Heaven, and then refrain from attempting to thrust our teaching down the throat of one who does not require such infantile nourishment. My concern here is with the average pupil, whom we meet in large numbers.

We\pageoriginale may, I think, discern four stages or ideals in the process of teaching a language. To read, to write, to speak, and finally, to think in that language without any need of mental translation; these are the aims : though often the fourth is beyond the power of the ordinary child, while the third may be so unless the child can live for some months in an environment where the language in question is the normal spoken tongue. If we are to train a pupil to be a professional mathematician, then all four stages must be achieved; to speak and to think mathematically are the hall-marks of the professional. But if mathematics is to remain a basic and prominent feature of the general school curriculum, then all, or almost all, our pupils must learn to read and write mathematically, so that ultimately they may both use and enjoy their mathematics. Excluding for obvious reasons the topic of elementary arithmetic, and envisaging the pupil who is embarked upon algebra and geometry, what can we learn from the teaching of languages at the corresponding stage ? Well, first we may recall that it is really not so very many years ago since a pupil learning a language other than his mother-tongue began with grammar, with declensions and conjugations and continued with these until he could repeat them with parrot-like accuracy, even though he might not be able to ask for his dinner. Today we start with simple but significant sentences, describing situations, meeting needs, answering question; but they are all within the compass of the child's experience. The corresponding change in the teaching of mathematics has also been achieved : no longer in algebra do we insist on hour after hour of pure manipulation,---multiplication, division, factors, simplification. It has often been argued at some length---and with some heat---that we ought to start with the formula, or that we ought to start with the problem; this seems to me a somewhat unrewarding controversy; what we have to start with are simple but significant sentences, describing situations and asking questions within the scope of the child's experience, and showing the child how he may himself discover the answers. At a later level, we no longer force the pupil through hundreds of manipulative examples of the technique of differentiation\pageoriginale and integration before allowing him to see that a mastery of this technique will enable him to pose and to answer question of genuine importance not amenable to the more elementary techniques which he has already acquired. In doing this, whether we have copied the teacher of languages or he has copied us, at least we have gone far to free ourselves from what was the besetting sin of the teaching of mathematics and of the classical languages in the late 19th century. While one must naturally respect the high scholarship of the grammarians of that era, one may reasonably protest that to some of them grammar was no longer a means to an end but an end in itself. Latin and Greek were not only dead languages, but languages which had never been alive; to think of Plato and Archimedes, Livy and Virgil as human beings more or less like ourselves was not only not a help, it was almost a hindrance to the acquisition of pure scholarship. And mathematicians are not altogether free from a similar reproach; the acquisition of a technique was an end in itself, not something to be used and enjoyed, but merely something to be acquired. That attitude has been abandoned, I hope for ever, and having set our feet on the right road we are now ready for a further progress. I have stressed that, to me, mathematics is a language to be used and {\em enjoyed}; few would quarrel with the first part of the description; the second would arouse more criticism; yet surely it is as important as the first. How are we to set our pupils on the way to enjoying their mathematics?

Obviously, by giving them mathematics in an enjoyable form. I do not mean by this that we should attempt to introduce an element of comedy into the subject, that we should make great use of the comic strip, that we should conceal or ignore difficulties. No; but the mathematics we offer, at whatever level, must be the best of its kind, must be as perfect in form as we can make it. It must be simple; it need not be, it should not be necessarily childish. Again, we can learn something from our colleagues in the department of languages. Once the child has mastered the elements or reading, the wise teacher in these days will no longer confine the pupil to prose which is childish, though he will take care that it\pageoriginale is certainly simple. Prose which is intentionally and deliberately ``written down'' to a presumed child-level is much more likely to induce contempt from the reader than admiration. The teacher of English finds that it is well worth searching for prose or verse which is simple without being childish; he can find it, for example, in many places in the authorised version of the Bible, in most of the writings of John Bunyan, in parts of Defoe and Swift, in the verse of Chesterton, or Macaulay. Here the style has all the elements of simplicity; short sentences, a high proportion of monosyllables; the imagery concrete and familiar. Yet no one would deny that here we have English at a high level, still thoroughly suitable for the child reader. (I must apologise for taking my examples here from English, but it would clearly be impertinent of me to express similar opinions on writings in other tongues; I am sure that corresponding instances in French or German could readily be found.) I am not sure that we have learned the analogous lesson in mathematics. We have not yet solved the problem of presenting genuine mathematics in a sufficiently simple form.

May I go further ? Speaking from experience as an editor for some 25 years, and being careful to preface my comment with the restriction that I can speak only about mathematics in English, I would say that the cardinal fault of most mathematical exposition, from the elementary school textbook up to the advanced treatise or the research paper, is carelessness of style. Please do not misunderstand. I am not asserting that style is as important as content; far from it. Accuracy of content must come far above any other requirement. But then, much care is always paid to this requirement by the average author. There are mistakes, of course, but these are seldom the consequences of negligence. In any event, papers which contain too many mistakes in content are unlikely to be published anyway. But accuracy of content, though of over-riding, primary importance, is not quite all. A completely solipsist philosophy is not common among mathematicians; we, most of us, regard it as a duty and to some extent a pleasure to communicate our discoveries to others. Yet far too often we take very little care about\pageoriginale this side of our work. It is natural, indeed, that there should be a considerable decrease in tension once the creative idea has emerged and proved itself of value; writing out a fair copy for the press can be sadly tedious. But the tedium is no excuse for negligence. Kelvin called Fourier's {\em Theory of heat} a mathematical poem; this perhaps is too exalted an aim. But we can all write good, simple mathematical language if we try hard enough; and if we do not try hard enough we are not doing our full duty to our colleagues and our subject, to which we should be, in Bacon's phrase, a help and an ornament.

There is one golden rule : the laws which govern the construction of mathematical prose are precisely those which govern the construction of good non-mathematical prose. A sentence must contain a principal verb, there must be agreement in number between subject and verb, a transitive verb requires a direct object, and so on. This of course is mere journeyman stuff, too often neglected, but still only the crudest and most elementary requirement. The author who wishes to make his paper as clear as possible must go further; he must first weight and digest the words of A. E. H. Love, stressing the need for training in the means of expression, the need for the patience which will re-write and if need be re-write again in order to produce a mathematically articulate exposition; then he must go beyond the crudest forms of correctness and pay attention to style, even to delicate shades of style, coming back again and yet again to the fundamental principle that what is not good prose cannot be good mathematical prose. He must scrutinise carefully even the details of punctuation.

Consider an example:

``Let $f(z)=\Sigma a_{n}z^{n}$ be a function which is regular in the unit circle.''

What is wrong with this? Nothing very much, but it is the thin end of a wedge whose broader part would admit.

``Let $f(z)=a_{0}+a_{1}z+a_{2}z^{2}+a_{3}z^{3}+\cdots$ be a function which is regular in the unit circle.''

Now\pageoriginale it is true that the excellent pamphlet on writing mathematics for the press issued by the London Mathematical Society says that the equality sign is to be treated as a fully inflected verb, but in ordinary prose the inflection is often visible in the form, whereas the sign of equality cannot show of itself how it is being used.

Consider a parallel in ordinary prose:

``Then Mr. Blank, the Prime Minister, said in reply...'' or

``Here Mr. Dash, who is the leader of the party, rose to point out...''.

In the first of these, we should certainly never omit the commas, in the second a slovenly writer might allow himself to omit them, but no writer with any pretensions to precision would ever dispense with them. The reason is clear: in each example we have a clause in apposition, and this must be kept separate from the main structure, such separation being best indicated in normal prose by enclosing the phrase in commas. In the mathematical example, it is $f(z)$ which is regular, with the series as its representation, so if we are strictly to follow the principle, we should enclose the clause in apposition, namely ``$=a_{0}+a_{1}z+a^{-}_{2}z^{2}+\cdots$'' in commas. This in a displayed formula is decidedly unconventional, yet to leave the sentence as it stands is to subject the reader to an unnecessary strain. Why should this be so? Because the writer has fallen into the cardinal error of terse style, confounding two ideas into a single sentence. We should be prepared to say ``Let $f(z)$ be a function regular in the unit circle, having the expansion...'' or ``Let $f(z)$, whose expansion is..., be a function regular in the unit circle''. Punctuation is essential to ready understanding, and its rules are clear and simple; but they cannot always be directly applied to mathematical prose, and, therefore, when they cannot be so applied, the sentence should be completely re-cast.

Good structure requires not only crisp, precise sentences; these sentences must themselves build up into paragraphs. Many authors have no care for the importance of the paragraph. Some write on and\pageoriginale on until they become tired of seeing a monotonously even left-hand margin, and in desperation they make a break which may well be between two closely related sentences. Others are so alive to the need for a break that they rarely allow more than one sentence to a paragraph. I wish that all mathematical writers using English could be compelled to study the structure of Macaulay's prose, particularly in his {\em History of England}. I am not to be taken as endorsing his opinions. But there is not one obscure sentence in the whole set of volumes; and each paragraph is constructed with the utmost skill and craftsmanship; each has its own main topic, yet each forms a link in a continuous chain of description or argument.

Let me take a specific example. For definiteness, let us suppose that we are Picard, aboud to write out for the press a proof of the new result that an integral function which does not take two specified values is a constant. How should we divide our paper into paragraphs ? We want to establish that an integral function which never takes the values $a$, $b$ is a constant. First, then, we shall use one paragraph to show that a linear transformation allows us to take the special values $a=0$, $b=1$. Next, if $v$ is the function inverse to the modular function, we have to show that $v(f)$ is an integral function with positive imaginary part; this, to Picard, is not a classical result, so it must be proved. It makes a second paragraph, which may of course be set out as a lemma; that is, a self-contained result, not particularly interesting for its own sake, but essential to the argument. This is a reasonable usage of the word ``lemma''; the recent tendency to strive for the maximum lemma-density per paper is not a fashion to be encourages. Finally, from the lemma it follows that $\exp \{iv (f)\}$ is a bounded integral function and is therefore by Liouville's theorem a constant. Here is our third paragraph. In fact, a little thought shows that the whole proof falls easily into three main stages, and these stages show the natural paragraph breaks.

So far, we have been concerned mainly with the relation between the writer and the reader; but there is usually a middleman. His function,\pageoriginale unlike that popularly attributed to middlemen, of sitting back and doing nothing save collecting the profits, is of prime importance. The printer is the channel of communication from author to reader, and if the co-operation required by the reader from the printer is to be adequately received, then the co-operation between the author and the printer must be close and intelligent. But is must be a co-operation. Printers and compositors are highly intelligent, men of skill and craftsmanship, but their powers should not be over-estimated. Any editor here present will recollect the feeling of irritation and dismay which arises whenever he has received a covering letter to say : ``I am afraid that my manuscript is not very carefully written, and may need some revision, but no doubt the printer can attend to all that''! Quite often he cannot, and even if he can there is no reason why he should---it is not his job. His job can be simply and even adequately described by saying that he is bound to reproduce what is in the script. With some printing houses, where house rules are sacrosanct and inviolable, the compositor will adapt symbols to meet the needs or imperfections of the manuscript, but even in these circumstances he can only repair, he cannot add nor should he subtract. Thus between the reasonable and legitimate constructions of the compositor and the intentions of the writer, there may easily be a gap, and if the writer is careless this gap may be a wide one. Such a defect in co-operation may diminish substantially the visual impact of the text on the reader. For many readers, this visual impact is of prime importance; perhaps some of us here would concede under pressure that part at any rate of our mathematical knowledge was gathered and is retained by a vivid recollection of the appearance of some printed page. Nor need we be too much ashamed at such an admission, since Rayleigh's {\em Life of Sir J. J. Thomson} shows that much of J.J.'s phenomenal knowledge of the literature of theoretical physics was gathered and retained in this way. Whether psychologically this is a good or a bad thing. I do not know; what matters is that it is a widespread fact that visual impact is all-important for many students of mathematics.

Since\pageoriginale the role of the compositor is essentially limited to the reproduction of what is in the script, the onus for visual impact falls heavily on the author. While this is true for all levels of mathematical exposition, it has a particular importance for authors of textbooks, more especially for authors of first courses, whether these be first courses in arithmetic or algebra or calculus or any other topic. The author's task is not at an end when he has collected his material, arranged it in order, written it our accurately and clearly, and checked the substance of the completed manuscript. If he is wise, he will now visualise as clearly as he can the probable appearance of the pages of his work in print. To do this, he must first know something of the restrictions imposed by cold rigid metal types. Not everything the pen performs can be reproduced typographically; some things are impossible. And it is not without value to remember that what is possible may be expensive, and the author concerned with the mind of the student may be recommended to spare an occasional thought for his pocket. Secondly, he must endeavour to see the printed page as a whole and to keep it at least in his mind's eye while writing up his manuscript. There is a well-known English textbook, now some fifty years old and so out-moded and antiquated, not indeed free from gross errors, which still sells its thousands of copies every year and is used widely throughout the Commonwealth. Why ? Not simply reverence for the antique. The author, as I have good reason to know, could envisage precisely the form of printed page which he desired to achieve, he knew exactly what would go on one line, how much manuscript would exactly fill one page of print, so that awkward page-turnings in the middle of an important argument could be avoided, he knew just how much and how little display to give to symbolic matter, where to give space, when to use Clarendon and italic types. Thus though today we have many better books available on this topic, the veteran holds its place, not on its mathematical merits, which are now much dimmed, but because the author had the gift of envisaging the visual impact of his work in print, or rather, because by painstaking thoroughness he had earned this gift.

Trivial,\pageoriginale technical matters ? Yes, if you like. And so beneath the dignity of the mathematician, unworthy of his attention ? No, emphatically, no. One of the outstanding English treatises of my time, a work held in the utmost respect throughout the mathematical world, was made much more difficult to read than it need have been by lack of proper co-operation between author and printer, resulting in cramped pages, crowded masses of symbols, and unemphasised formulae. For the high-class professional, this perhaps does not matter much. But we must think of the average reader, in this case say a good mathematician not specialising in this field. His task is harder just because an undue proportion of his attention has been diverted to the preliminary task of filling the gap between author and printer, a gap which ought to be filled by the author, not by the reader. And if this diversion of attention occurs in a school textbook, the result can be disastrous : I know of a pupil working on his own who was driven almost to the point of despair and defeat merely because the author of the book he was using had never bothered to explain that exp $x$ means the same thing as $e^{x}$. To say that he, the novice, ought to have inferred this identity from the context is simply the excuse made by the slovenly author.

Among the golden rules, then, are these : good mathematics must be written in good prose; the author should realise the scope and limitations of metal type; he should develop in himself the capacity to visualise what his manuscript will look like when it is transferred to print. Less than this, and he is likely to fail in his aim, his duty, of communicating his thoughts, in precise form and without loss of accuracy and substance, to his readers.

But this is not the whole duty of the author. He is to learn from the printer; but very often the printer has much to learn from the author. At the lowest level, the author may have to teach the printer that, for instance, a Greek fount which does not clearly distinguish between a lower case italic $a$ and a lower case `alpha' is not suitable for mathematical work, or again, that an unfamiliar symbol should never be improvised (the attempt to construct an integral sign out of a pair of brackets is an extreme instance, but has\pageoriginale actually been seen). At a higher level, authors requiring new symbols or new groupings of symbols should consider the typographical side of their requirements, so that before a new symbol is launched on the scientific world its typographical difficulties and implications may have been examined.

Further, there is a field, if not of genuine research then of investigation, open to some one industrious enough and humble-minded enough to explore. How does the ready appreciation of printed matter increase with increase of age ? Little enough has been done even on the general topic of the impact of print on the child-mind. For instance, a hundred years or so ago it appears to have been taken as an axiom that the younger the child, the smaller should be the fount size. Today we have gone to the other extreme, and hold that the smaller the child the larger the fount size. But this tendency has its dangers. I have seen a child capable of fluent reading hesitate lamentably over such a simple word as ``spaceship''; inquiry showed that the child could read this at sight in a normal size of type, but failed to recognise it in the large type in which it was in this instance presented, because the word had spread out beyond the physical capacity of the child's eye. Those of us who are accustomed to read, not by the word, but by the phrase or the line or the paragraph, tend to forget these physical limitations. To take one or two trivial but typical mathematical instances, at what age or intelligence level can we safely begin to use with our pupils summation and product signs:
$$
\sum a_{n}x^{n}\quad\text{instead of}\quad a_{0}+a_{1}x+a_{2}x^{2}+\cdots
$$
and 
$$
\prod u_{n}\quad\text{instead of}\quad u_{1}u_{2}u_{3}\ldots ?
$$
How much sophistication must be acquired before the solidus is comprehended without a voluntary effort ? Incidentally, the answer to this question would be more easily given if printers and mathematicians would co-operate in agreeing on conventions for the use of the solidus and on standard spacing which would allow the formula $x/a+y/b=1$ to appear in solid text without any fear of ambiguity.

I\pageoriginale would not claim that this field of inquiry is of fundamental importance to mathematics, but I would claim that it is of great importance to the teacher of mathematics; it is a field which has been so far too little explored, probably because it appears humdrum and unspectacular. But there are many of us, well aware that we have not been dowered with the supreme gift of creative mathematical power, who could nevertheless contribute fruitfully by such investigations to the general well-being of our subject.

\bigskip
\medskip

{\fontsize{9pt}{11pt}\selectfont
Royal Naval College, Greenwich}\relax
