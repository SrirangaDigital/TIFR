\chapter{On Mathematical Education in the U.S.S.R.}

\begin{center}
{\em By~} A. D. ALEXANDROV
\end{center}

\setcounter{pageoriginal}{98}
In\pageoriginale this brief report a general scheme of mathematical education in the U.S.S.R. will be outlined with special stress on university education.

\blfootnote{This lecture was given at the South Asian Conference on Mathematical Education held on 22-28 February 1956 at the Tata Institute of Fundamental Research, Bombay.}
\begin{enumerate}
\item In order to comprehend some specific features of our system of education let us cast a glance at its history.

Though before the Revolution, mathematics in Russia was at a high international level, the extension of education was too miserable. Illiteracy was a great evil of that time. The majority of people had no access to education. Gymnasiums and universities were accessible only to the elite. Just immediately after the Revolution our task was to change radically this situation. We did our best and development of primary, secondary and higher schools was immensely remarkable, the number of students was raised to an extremely high level. Much attention was paid, for instance, to the development of education in the republics of Middle Asia. It was necessary to ensure instruction in native languages there.

These conditions demanded, most urgently, a uniform and strict system in order to implement a higher standard for the growing number of secondary and higher schools everywhere. Life develops in a quick way and there arise new problems for which a better solution needs to be found. Certain efforts are being undertaken aimed at fulfilling these tasks.

The general system of education in the U.S.S.R. can be presented as follows.

\item \textsc{Middle school}, which takes ten years. Children enter the school at the age of 7. To-day we have compulsory course for only seven years. After seven years of learning a boy or a girl can according\pageoriginale to their desire continue their studies at school or enter a technical middle school. There are different types of such schools : engineering, technology, agriculture and so on; they have as a rule 3 or 4 year courses.

A ten year course of education is being put into force mainly in big cities to-day. In five years, however, compulsory ten years' education will be introduced almost everywhere. Native language is the means of instruction and each nation however small it might be has its own schools. Teaching in middle school is based upon a unified programme. As far as mathematics is concerned the programme contains arithmetic during the first five years of learning, then algebra and geometry, and trigonometry at the eighth year. Algebra is taught up to complex numbers, elementary functions, and the theory of limits, geometry up to polyhedra and solids of revolution.

Children who take a keen interest in mathematics have an opportunity to attend special circles at schools and universities. In order to attract more attention to mathematics and encourage gifted children many universities arrange annual school mathematical competitions, so called ``mathematical olympics''. A school boy should solve in such a competition, a number of complicated problems which, however, do not surpass the limits of the school programme. Those who have managed successfully are stated to be winners of the competition and are also encouraged with prizes.

Mathematical olympics have gained a broad acknowledgment\break among boys and girls. For instance annually 1000 take part in the olympics of Moscow university.

\eject

\item \textsc{Institutions of Higher Education}. In order to enter any higher school one has to pass entrance examinations. And though the Soviet Union has about 800 higher schools with more than $1\tfrac{1}{2}$ millions of students, we have even more those who want to obtain higher education. It allows, especially in the biggest higher schools, to choose through entering examinations a good composition of students. Each student who is successful in his study has\pageoriginale a scholarship and therefore no criterion but personal abilities determine the possibility of obtaining higher school education for anybody. There is also a highly developed system of correspondence and evening courses designed for working people who want to study. The course at the Soviet higher schools lasts five years as a rule, four years at pedagogical institutes. After each term a student has to pass examinations in 3, 4, or maybe in two subjects. Higher mathematics is a special subject of study at universities and pedagogical institutes which train middle school teachers.

Mathematical training at higher technical colleges has aims and problems of its own. One of them is to adapt it to the needs and programmes of studies of engineering, technology, physics, and other special subjects.

A student at technical college deals with mathematics during the first two years. The general course of mathematics consists of analytic geometry with vector algebra, elements of higher algebra, calculus, differential equations, power and Fourier series. Short additional courses on the theory of complex functions, probability theory, partial differential equations, are given after the general course. Physics departments of universities have their own more extensive programmes of mathematics.

\item \textsc{Universities}. There are 33 universities in our country. Almost all of them have their physical mathematical faculties, while the bigger ones like Leningrad University, for example, have a separate mathematical faculty including three departments, that of pure mathematics, of mechanics and astronomy. This separation is justified, at least, by the fact that the mathematical faculty of Leningrad University accounts for more than 1000 students.

The general curriculum and programme of university education are common for all universities of the U.S.S.R. except the bigger ones. Such universities with a large and highly qualified teaching staff outline their own curriculum within the general content of university education.

The\pageoriginale task of university education is aimed at training teachers of middle and higher schools and scientific workers. A graduate does not obtain any special degree but his qualification as a teacher and scientific worker is defined in his diploma.

\item It is quite obvious that a scientific worker would not come out of each student. But university education should initiate each student to be engaged in scientific work. A university should provide all the opportunities for each capable student to fulfil a research work after graduation. 

The problem of combination of a necessary middle level with the development of individual abilities has been already discussed at this Conference.

Its solution is assured by two methods. The first one is a due combination of teaching with research work. The university must be a school and research institute at the same time. Only a lecturer experienced in science can give to his students not only a dry mummy but a sound and living science. Only such a teacher can make the students think, not only learn. Therefore we consider the combination of teaching at the universities with research work to be absolutely indispensable and compulsory. Much effort is applied to realize this principle.

The second method provides a due structure of curriculum and various methods of involving the students in scientific work.

\newpage

In this connection we have two new problems~:
\begin{itemize}
\item[a.] Reasonable combination of abstract theories of pure mathematics with applications to mechanics and physics.

\item[b.] Reasonable combination of a sufficiently wide basic mathematical education with a certain specialization. 
\end{itemize}

A student naturally should have a good experience in all branches of mathematics. But he will fail to deeply penetrate into all those branches, while the approach to research work cannot be gained without such penetration. This fact causes the problem of combination\pageoriginale of a wide basic preparation with specialization which brings a student to the border where some research work begins.

Only combination of abstract generalizations with concrete knowledge resulting from natural science and technique will allow a student to find a true orientation in mathematics. The beautiful tree of mathematics grows up on the soil of natural science. Growing to the top of abstraction, it bears fruit, the seeds of which drop on the same soil of natural science and technique. It means not only connection with applications. We are also speaking of pure mathematics of higher style.

Great mathematicians of all times have been developing their fruitful ideas in an indissoluble connection with problems of the exact natural science. One can draw these facts from works of Archimedes, Newton, Euler, Lagrange, Gauss, Riemann, Poincare, Hilbert. The progress of mathematics is closely connected with the progress of exact natural science and technique. Mathematical education and research has to pay much attention to the problem of physics and technology.

\item What is the curriculum at the department of pure mathematics of Leningrad University, approved by the learned council of the faculty ?
\begin{itemize}
\item[(i)] Analysis in the 1st and 2nd years.

\item[(ii)] Differential equations, integral equations and the calculus of variations, with elements of functional analysis in the 2nd, 3rd and 4th years.

\item[(iii)] Courses in the theories of real and complex functions in the 3rd year.

\item[(iv)] Algebra with linear transformations and elements of group theory, and number theory in the 1st and 2nd years.

\item[(v)] Geometry, analytic in the 1st year, differential in the 2nd, and the foundations of geometry, in the 3rd year.

\item[(vi)] Theory of probability in the 3rd year.

\item[(vii)] Mechanics in the 1st, 2nd, 3rd years.
\end{itemize}
\end{enumerate}

\begin{landscape}
{\fontsize{9pt}{11pt}\selectfont
\tabcolsep=2pt
\renewcommand{\arraystretch}{1.5}
\begin{longtable}{rlcrrcrcccccccccc}
\caption*{THE CURRICULUM OF THE DEPARTMENT OF (PURE) MATHEMATICS OF THE MATHEMATICAL FACULTY OF LENINGRAD UNIVERSITY}\\
\toprule
&& \multicolumn{4}{c}{TIME : NO. OF HOURS} && \multicolumn{10}{c}{DISTRIBUTION OF HOURS A WEEK}\\
\cline{3-17}
 & \multirow{2}{1.4cm}{\raisebox{.1cm}{Subject}} & \multicolumn{2}{c}{\multirow{2}{*}{\raisebox{.1cm}{Lectures}}} & \multirow{2}{*}{\raisebox{.1cm}{\centering Exer-}} & \multirow{2}{*}{\raisebox{.1cm}{\centering Labo-}} && \multicolumn{2}{c}{I \textsc{Year}} & \multicolumn{2}{c}{II \textsc{Year}} & \multicolumn{2}{c}{III \textsc{Year}} & \multicolumn{2}{c}{IV \textsc{Year}} & \multicolumn{2}{c}{V \textsc{Year}}\\[-2pt]
\cline{8-16}
&&&& \multicolumn{1}{c}{cise} & {\centering ratory} && 1st & 2nd & 1st & 2nd & 1st & 2nd & 1st & 2nd & 1st & 2nd\\[-6pt] 
&&&&&&& term & term & term & term & term & term & term & term & term & term\\
\midrule
1. & Analysis... ... & ... & 255 & 220 & --- && 8 & 8 & 6 & 6 & --- & --- & --- & --- & --- & ---\\
2. & Analytic geometry & ... & 100 & 90 & --- && 7 & 4 & --- & --- & --- & --- & --- & --- & --- & ---\\
3. & Higher algebra ... & ... & 108 & 100 & --- && 4 & 4 & 4 & --- & --- & --- & --- & --- & --- & ---\\
4. & Physics\quad ... & ... & 200 & 36 & 52 && --- & --- & --- & --- & 6 & 7 & 2 & 2 & --- & ---\\
5. & Theoretical physics & ... & 76 & --- & --- && --- & --- & --- & --- & --- & --- & --- & 2 & 4 & ---\\
6. & Astronomy ...  & ... & 68 & --- & --- && --- & --- & 2 & 2 & --- & --- & --- & --- & --- & ---\\
7. & Differential geometry & ... & 68 & 32 & --- && --- & --- & --- & 3 & 3 & --- & --- & --- & --- & ---\\
8. & Foundations of geometry & ... & 50 & --- & --- && --- & --- & --- & --- & --- & --- & --- & 3 & --- & ---\\
9. & Differential equations & ... & 70 & 70 & --- && --- & --- & 4 & 4 & --- & --- & --- & --- & --- & ---\\
10. & Mathematical physics & ... & 100 & 34 & --- && --- & --- & --- & --- & 2 & 4 & 2 & --- & --- & ---\\
11. & Mechanics (including &  &  &  & &&  & &  & &  &  & &  &  & \\[-4pt]
 & continuous media) ... & ... & 220 & 135 & --- && --- & 6 & 6 & 5 & 4 & --- & --- & --- & --- & ---\\
12. & Real functions ... & ... & 68 & --- & --- && --- & --- & --- & --- & 2 & 2 & --- & --- & --- & ---\\
13. & Complex functions & ... & 70 & 30 & --- && --- & --- & --- & --- & 2 & 4 & --- & --- & --- & ---\\
14. & Theory of probability & ... & 50 & --- & --- && --- & --- & --- & --- & 3 & --- & --- & --- & --- & ---\\
15. & Integral equations & ... & 34 & --- & --- && --- & --- & --- & --- & --- & --- & 2 & --- & --- & ---\\
16. & Calculus of variations & ... & 50 & --- & --- && --- & --- & --- & --- & 3 & --- & --- & --- & --- & ---\\
17. & Number theory ... & ... & 32 & --- & --- && --- & --- & --- & 2 & --- & --- & --- & --- & --- & ---\\
18. & Computing machines & ... & 50 & --- & 14 && --- & --- & --- & --- & --- & 4 & --- & --- & --- & ---\\
19. & History of mathematics & ... & 70 & --- & --- && --- & --- & --- & --- & --- & --- & 2 & 2 & --- & ---\\
20. & Mathematical practice & &  &  & &&  &  &  &  &  &  &  &  & &\\[-4pt]
 & (problems, computations etc.) & ... & 40 & 150 & 50 && 2 & --- & --- & --- & 3 & 3 & 3 & --- & --- & ---\\
21. & Methods of teaching and & &  &  & && & & & &  & &  & & &\\[-4pt]
 & practice at school & ... & 100 & 30 & --- && --- & --- & --- & --- & --- & --- & 6 & 2 & --- & ---\\
22. & Special courses and &  &  &  & && & &  &  &  &  &  &  & & \\[-4pt]
 & seminars to be chosen & ... & 220 & 100 & --- && --- & --- & --- & --- & --- & 2 & 6 & 6 & 7 & ---\\[7pt]
\multicolumn{10}{l}{Philosophy and foreign language omitted}\\
\end{longtable}
}\relax
\end{landscape}

\setcounter{pageoriginal}{105}
\begin{itemize}
\item[(viii)] Physics,\pageoriginale a general course as well as a special course of modern theoretical physics, in the 3rd, 4th, 5th years. Physics is taught at this period in order to have no difficulties with mathematics and not to repeat the foundations of mechanics.

\item[(ix)] Astronomy in the 1st and 2nd years, and then such courses as those on computing machines, or the history of mathematics, or methods of education and philosophy.
\end{itemize}

In the middle of the 3rd year there are introduced, step by step, some special courses which a student is allowed to choose according to his desire.

Thus a student who wishes to study algebra has an opportunity to attend lectures on the theory of groups, Galois theory, continuous groups or some other subjects. A student who joins the chair of geometry may study topology, Riemannian geometry, and so on. Besides, professors give lectures on their own subjects of research, and thus a student has an opportunity to be acquainted with up-to-date results. Students may also join special seminars, for instance in mathematical logic, and to work there with post-graduate students, lecturers and professors.

The very existence of such seminars as well as the programmes of the special courses are determined by the scientific activities and interests of professors only. As far as their courses and seminars are concerned, the professors are entirely free in the choice of subjects and programmes, their only task being to promote sufficiently the interests and activities of students.

In order to fulfil the same task students have to prepare small papers at the 3rd and 4th years. These papers may contain a short survey or a solution of certain problems.
\begin{enumerate}
\setcounter{enumi}{6}
\item In the 1st and 2nd years the students may join the circles where they study some chosen additional subjects or get an additional training in solving problems above an average level.

All kinds of scientific activities of students are supported by the Students Scientific Society. The Society arranges students' conferences,\pageoriginale competitions in solving problems, issues special wall newspapers. We have prizes for t he best solution of problems and for the best scientific paper. Each student who is successful in his study gets a scholarship. The best students get scholarships of honour.

In the last course of learning a student is almost free from obligatory lectures the main task of his being to attend special courses, seminars, and above all, to prepare the diploma work. This work has to be a small scientific paper. The level of its originality, of course, depends upon the ability of the student.

As a result of the whole system we have every year a number of students who have papers worthy to be published in mathematical magazines.

\item For the sake of the further training of scientific workers and higher school teachers and lecturers a wide-spread system of post-graduate studies has been set up at the higher schools and research establishments. In order to enter a post-graduate course, a graduate has to pass examinations. The graduates recommended by the learned council of the faculty are allowed to enter the post-graduate course immediately after their graduation, while all the others after at least three years of work in their speciality.

Each post-graduate has a consulting professor who accepts his post-graduate according to his own opinion.

The post-graduate course lasts three years. A post-graduate is obliged to pass two examinations in his speciality, (the subjects of the examinations being proscribed by the consulting professor), a foreign language and philosophy. But the main task of his is to be engaged in research work which might be accepted as his Candidate's thesis. The Candidate's degree is, or seems to be, above the Master's degree in England.

This post-graduate system is the main source of the higher school teachers at the level of a lecturer. Due to all the universities we have\pageoriginale now the necessary number of lecturers with the scientific Candidate's degree.

\item It seems that the course on the history of mathematics given at our universities can play an important role in broadening the student's understanding of mathematics. This course gives a student a connected survey of basic mathematical concepts and theories in their birth and development, in their historical sequence and logical connection. It leads to a better understanding of the sources and foundations of mathematics. In particular, the history teaches us that modern mathematics as developed during the last centuries by European scientists has its source not only in Greek geometry as is often said, but in no less a degree in the concept of number which had sprung up in India.

What the Greek genius could not attain, i.e. to give the most convenient notations of numbers, to extend the system of numbers, and to abstract irrational numbers from their geometric basis---all this was done at the first stage in India and reached Europe through Middle Asia and Arabic countries. For instance one can find in the writings of Eastern mathematicians, four centuries before Newton, the same definition of number which was given by Newton, in his Arithmetic.

India was a cradle of one of the most fundamental concepts of mathematics. The glorious history is a guarantee of the bright future of science.
\end{enumerate}

\bigskip

{\fontsize{9pt}{11pt}\selectfont Leningrad University}\relax
