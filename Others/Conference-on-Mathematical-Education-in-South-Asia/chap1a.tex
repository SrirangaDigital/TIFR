\chapter{South Asian Conference on Mathematical Education}

\begin{center}
{\bf BOMBAY, 22-28 FEBRUARY 1956}
\medskip

{\bf REPORT}
\end{center}

\begin{enumerate}
\item A\pageoriginale Conference on Mathematical Education in South Asia was held at the Tata Institute of Fundamental Research, Bombay, on 22-28 February, 1956. The Conference was the first of its kind to be held in Asia. It was attended by about seventy five mathematicians from twenty countries : Australia, Burma, Ceylon, China, France, West Germany, Hungary, India, Indonesia, Italy, Japan, Malaya, the Netherlands, Pakistan, Poland, Singapore, Thailand, the Union of Soviet Socialist Republics, the United Kingdom, and the United States; and was presided over by Professor K. Chandrasekharan.

The Conference was organized with the financial support of Unesco, the International Mathematical Union, the Government of India in the Ministry of Natural Resources and Scientific Research, the Sir Dorabji Tata Trust, and the Tata Institute of Fundamental Research. The proposal for the Conference was put forward by the Tata Institute of Fundamental Research, and endorsed by the National Committee for Mathematics in India, which acted as the principal agency for executing the general plan of the Conference, and for maintaining the closest liaison between the sponsoring institutions. The Tata Institute of Fundamental Research was the principal host institution.

\item \textsc{Organization.} The purpose of the Conference was to discuss, with special reference to South Asia, the problems of mathematical education at all levels, and to formulate plans for its sound development. As Organizing Committee with Professor K. Chandrasekharan as Chairman, and with Professors Ram Behari, E. Bompiani, K. R. Gunjikar, C. Racine, and M. H. Stone as members, drew up the programme. Professors Bompiani and Stone were nominated to this Committee by the International Mathematical Union.

News\pageoriginale regarding the organization of the Conference was disseminated in South Asia through the good offices of the Ministry of Natural Resources and Scientific Research of the Government of India, and the Unesco Science Co-operation office at New Delhi. Invitations to send delegates to the Conference were extended to the Governments of Burma, Ceylon, Indonesia, Malaya, Pakistan, Singapore and Thailand. Members of the International Colloquium on Zeta Functions, held at the Tata Institute of Fundamental Research on 14-21 February, 1956, were invited to participate in the Conference. Universities and research institutions in India were invited to nominate one representative each. The participation of interested mathematicians from all countries, South Asian or not, was not only welcomed, but positively encouraged. Australia, China, Poland, and the U.S.S.R., for instance, were thus represented. The Organizing Committee decided, however, that only the following participants should have the right to vote : delegates sent by the Governments of the South Asian countries, members of the National Committee for Mathematics in India, members of the Organizing Committee, and those giving invited addresses. No occasion arose, however, which called for a vote.

It was decided that the Conference should function in three tiers : (i) {\em invited addresses}, (ii) {\em working groups}, and (iii) {\em plenary sessions}.

\item \textsc{Invited addresses and special lectures.} The invited addresses, which were given by experts chosen from all over the world, either dealt with the common problems of mathematical instruction confronting most countries, or with the system of education prevalent in a specific country in Asia or in Europe, or with the methods of teaching. Criticism of educational systems was balanced by constructive suggestions for their improvement. The absence of dogmatism and the presence of a freshness of approach were alike remarkable. Each address lasted forty minutes, and was followed by a discussion.
\end{enumerate}

The\pageoriginale following addresses were given~:
\smallskip

Professor E. Bompiani (Rome) : {\em Mathematical instruction in Italy}.

Professor T. A. A. Broadbent (London) : (i) {\em Present-day problems in English mathematical education}.

Professor T. A. A. Broadbent (London) : (ii) {\em Typography and the teaching of mathematics}.

Professor G. Choquet (Paris) : {\em Teaching in secondary schools and research}.

Professor H. Freudenthal (Utrecht) : {\em Initiation into geometry}.

Professor A. Oppenheim (Singapore) : {\em The problems which face mathematicians in Singapore and the Federation of Malaya}.

Professor M. H. Stone (Chicago) : {\em Some crucial problems of mathematical instruction}.

Professor M. R. Siddiqi (Peshawar) and Professor L. K. Hua (Peking) who had been invited to give addresses were unable to attend the Conference.

All the addresses (with one exception) were mimeographed and distributed, immediately after delivery, to all the participants of the Conference.

On the invitation of the Organizing Committee, the following special lectures were given :

Professor A. D. Alexandrov (Leningrad) : {\em On mathematical education in the U.S.S.R.} (40 minutes)

Professor G. Choquet (Paris) : {\em New material and a new method for teaching elementary calculations in primary schools} (30 minutes)

Professor E. Marczewski (Wroclaw) : {\em Information on mathematical education in Poland} (10 minutes)

Professor H. F. Tuan (Peking) : {\em On mathematical education in Chinese universities} (15 minutes)

The\pageoriginale invited addresses and special lectures were given in open meetings at which the attendance was larger than the membership of the Conference.
\begin{enumerate}
\setcounter{enumi}{3}
\item \textsc{Working groups.} It was early recognized by the Organizing Committee that discussions between individual mathematicians, and between groups of them, should form an important part of the activities of the Conference. To keep these discussions on a serious plane, and to make them purposeful, it was necessary to conduct them in relatively small groups, of not more than thirty each. It was also considered convenient to deal with mathematical education in three stages, undergraduate, graduate, and post-graduate, which would together cover the entire range of the curriculum. Three working groups were accordingly set up. The Organizing Committee drew up a brief questionnaire common to all three working groups. The topics for discussion were put down as follows : 1. What is the purpose of teaching at this level ? 2. What should we teach ? 3. To whom and by whom ? 4. How support and how place ? 5. How to carry out ?

The working groups had some material supplied to them to start with. As early as April 1955, about one hundred mathematicians all over India were invited to send in their views on mathematical education in India to the conveners of three panels constituted for that purpose :

Panel on post-graduate teaching and research : Professor K. Chandrasekharan (Convener), Professor Ram Behari, Professor V. Ganapathy Iyer, Professor S. Minakshisundaram, Dr. B. N. Prasad, Professor C. Racine, Dr. K. G. Ramanathan.

Panel on graduate instruction : Professor C. Racine (Convener), Professor P. L. Bhatnagar, Professor V. Ganapathy Iyer, Professor S. Minakshisundaram, Professor V. V. Narlikar, Professor B. S. Madhava Rao, Professor N. R. Sen.

Panel on undergraduate instruction : Professor K. R. Gunjikar (Convener), Professor Ram Behari, Professor H. Gupta, Professor S. Mahadevan, Professor C. Racine.

The\pageoriginale memoranda prepared by these panels were mimeographed and distributed to the respective working groups, with a view to familiarizing them with the problems which face mathematicians in a country like India. Copies of the presidential address and Professor Stone's address were supplied to the working groups right at the start, while copies of the other addresses were made available as the Conference progressed. In this way a continual supply of material was kept up, and fresh ideas were channelled in. Discussions in the working groups involved many different points of view, expressed in many different ways, often with considerable force. As a result of the co-operative effort of all the participants, however, it was possible to arrive at unanimous recommendations.

Each group had two young mathematicians attached to it as {\em reporters} who kept notes of the discussions. Messrs. K. Balagangadharan, M. S. Narasimhan, P. K. Raman, V. V. Rao, C. S. Seshadri and B. V. Singbal served as reporters.

When a problem came up before a working group which required a more intensive study than was possible at a regular session of the group, it was remitted to a smaller committee which reported back to the full group. An important instance of this procedure was provided by the Committee on Mathematical Education in Schools appointed by the working group on undergraduate instruction. This committee had Professor T. A. A. Broadbent as Chairman, Professor K. R. Gunjikar as co-Chairman, and Professor Aung Hla, Mr. S. D. Manerikar, Dr. R. Naidu, Mr. Poerwadi Poerwadisastro, Mr. Rabil Situsuwana, and Miss H. K. Wong as members.

The working groups met in closed sessions of an hour and fifteen minutes each. There were nine such sessions altogether, six of them before the first plenary session, and three thereafter.

\item \textsc{Plenary sessions.} The plenary sessions of the Conference were devoted to a discussion of the reports of the working groups, and of those matters which were the proper concern of the Conference as a whole, like research contracts, summer schools, textbooks, and examinations, and to the formulation of conclusions in the shape\pageoriginale of official resolutions. It was at the plenary sessions that the work of the three groups was reviewed, integrated, and fully fashioned. The drafting of the decisions reached at the plenary sessions was done by a Drafting Committee consisting of Professor K. Chandrasekharan (Chairman), Professor Ram Behari, Professor T. A. A. Broadbent, and Professor M. H. Stone. In this way pointless digressions and long speeches were avoided, and business was transacted with efficiency and speed.

The duration of each plenary session was an hour and fifteen minutes. There were two such sessions on 25 February, and a third and final session on 28 February.

\item \textsc{Programme.} The Conference opened on 22 February, 1956, at 11 a.m. with a brief address of welcome by Shri Morarji R. Desai, Chief Minister of the Government of Bombay. It was formally inaugurated by Dr. H. J. Bhabha, Director of the Tata Institute of Fundamental Research. Professor E. Bompiani, Secretary of the International Mathematical Union, made a brief speech in which he expressed the interest of the Union in the Conference. Professor K. Chandrasekharan then delivered the presidential address. The whole proceedings lasted an hour. A message from the Prime Minister, Jawaharlal Nehru, reproduced in facsimile, was given to every member of the Conference.

A detailed programme of the Conference is given separately. The President of the Conference was in the chair at all the meetings. The language of the Conference was English.

At the final plenary session on 28 February, a formal resolution embodying the conclusions reached at the Conference, and put in final form by the Drafting Committee, was read out from the chair. On the motion of Professor C. Racine and Professor Ram Behari, the resolution was passed unanimously. A second resolution proposing the constitution of a Committee for Mathematics in South Asia was moved by Professor S. Tanbunyuen, and supported by Professor Ram Behari, Dr. K. S. Gangadharan, Professor Aung Hla, Professor A. Oppenheim, Professor A. L. Shaikh and Ir.\pageoriginale Suhakso. The speakers expressed the belief that the Conference marked the beginning of an organized effort towards the progress of mathematical education is South Asia, and considered that the proposed Committee was necessary to continue and intensify that effort. The resolution was then passed unanimously. The Conference concluded with an expression of thanks, by the President, to all those institutions and individuals who had helped to make it a success.

It was the general feeling that the Conference had been truly international in spirit, even though the resolutions passed had a special relevance and significance for South Asia.

\item The social programme for the delegates, and for the participants, included a dinner at Juhu for the delegates on 21 February; a reception by the Vice-Chancellor of the University of Bombay, Dr. John Matthai, on 22 February; a special performance of classical Indian dances, Kathak and Kathakali, together with a buffet supper, for the delegates, on 23 February; a reception by the Governor of Bombay, Dr. H. K. Mahtab, at Raj Bhavan, on 24 February; a cruise at night round the Bombay Harbour on 24 February; an excursion by boat to Elephanta on 26 February, with lunch and tea served on board; and a banquet to the delegates on 28 February.
\end{enumerate}
