\chapter[The Problems which Face Mathematicians in Singapore...]{The Problems which Face Mathematicians in Singapore and the Federation of Malaya}

\begin{center}
{\em By~} A. OPPENHEIM
\end{center}

\bigskip

\blfootnote{This address was given at the South Asian Conference on Mathematical Education held on 22-28 February 1956 at the Tata Institute of Fundamental Research, Bombay.}
Merely\pageoriginale to give a bald account of the problems which beset mathematicians in Malaya (both in the Federation of Malaya and in Singapore) would not be helpful. The problems I feel sure though urgent and pressing to us are not new : they are in the main, the familiar ones of insufficient staff or insufficiently trained teachers, lack of appropriate textbooks suitable for the area, too traditional a curriculum, a shortage of training centres and until recently a dearth of professional mathematicians in the university : in addition there is the crucial problem of coping with a keen and growing demand for educational facilities at all levels.

It may be wise therefore to devote some time to describing the local context, to describe, however briefly and inadequately (for I am neither geographer nor historian, neither economist nor politician) the diversity and variety of Malaya, its peoples and its languages, its governments (now in a process of rapid change) and the growth of its educational systems, for there are several : Malay vernacular schools, Tamil vernacular schools, Chinese vernacular schools (at one time in several dialects, but now in Kuo-Yu), English medium schools, various training colleges, the University of Malaya (situated in Singapore) and (very shortly to be started in this same island) Nanyang, a Chinese University.

Singapore is an island some 220 square miles in extent at the extreme tip of the Malay Peninsula, just north of the Equator. It is a British Possession which is advancing rapidly towards self-government. It has indeed its own Assembly, in the main elected, and\pageoriginale a Labour Front Government which came into power in April 1955.

The Federation of Malaya consists of two settlements (formerly part of the Straits Settlements to which Singapore belonged), each under a British Resident, and nine semi-autonomous States, each under a Malay Ruler, the whole controlled by the Federal Government at the capital city, Kuala Lumpur, some 250 miles north of Singapore. In each of the nine States, there is a British Adviser and in Kuala Lumpur a British High Commissioner. In the march to self-government and independence these will disappear. The recent discussions in London show that the Chief Minister for the Federation will obtain independence for his country within the Commonwealth by August 1957.

Culturally, Malaya has been influenced for centuries by the ancient civilizations of India and of China : in more recent centuries by Islam : and still more recently the West with its modern techniques has played a dominant part.

To achieve a synthesis of these diverse influences is the task which to-day faces the leaders of the various groups in Malaya : Malays, Chinese, Tamils and others too numerous to name.

For Malaya has many groups. Its total population is about 7,000,000. Nearly 85\%\ are Malays and Chinese : the latter rather more numerous than the Malays. There are several hundred thousand Indians (mainly Tamil) and Ceylonese. In addition Malaysian peoples (including aborigines and Indonesians) Eurasians and Europeans. At one time many men from China, from India, from Europe, came to Malaya to seek a fortune, spent some years in the country, and retired to their countries of their birth for the remainder of their days. Now this tendency is disappearing. More and more Chinese and Indians and some Europeans have made Malaya their home.

The growth of population is striking. In 1931 Singapore's population was less than half-a-million : to-day it is close to one millon and a quarter. Last year there were born in this small island not less than 56,000 babies. The proportion of young to old is\pageoriginale probably higher in Singapore than anywhere else in the world. Obviously such points have a major relevance to the educational problems of the country, for example to the building of schools and to the supply of teachers to cope with its accelerated growth in population.

I turn next to the economy of Malaya. There are three parts to be noted : first, the subsistence economy of the rural population. These grow their own rice, catch fish, tap their own rubber trees and of course, possess coconut palms. Second, there are the banking, insurance and shipping interests of the parts, the chief being Singapore, which carry on a vast international trade. And finally, without which the others will fail, there are the great rubber plantations and tin mines of Malaya. In passing it may be noted that the man who made rubber planting a success attained his 100th birth-day last December. But rubber and tin are sold on world markets sensitive to the slightest rumour. Fluctuations in price which for some years have been spectacular influence the cost of living to a marked degree.

Centuries ago Singapore had been a large and populous centre. When Stamford Raffles arrived in 1819 he found it mainly swamp inhabited by a few fishermen and their families. Under his guidance the place expanded rapidly. Raffles belonged to that rare class, an administrator who was not only able but wise and far-seeing. He proposed an Institution of Higher Learning which would undertake an extensive study of the history, literature, languages and philosophies of the surrounding countries. For this end he made a generous contribution. His thorough going scheme has not even been fulfilled although we are approaching a realisation of his vision. For men of his quality of mind are rare so that in the years which followed although numerous schools were founded, some by missions, some by Government and some by private interests no institution such as Raffles had envisaged came into existence until another century had been passed. But of this more later.

It would take too long to describe the growth of the educational systems of the Straits Settlements (as these British Colonies were later\pageoriginale called) and of the various Malay States to the present position. It must suffice to indicate briefly the pre-war and post-war systems and to sketch the changes which have taken place recently with an indication of what may be expected in the future.

There are Malay vernacular schools (open to all but in practice used only by Malays), Tamil vernacular schools, Chinese vernacular schools, (moving towards the sole use of Kuo-Yu, the colloquial form of Mandarin), English medium schools. The schools are run by Government : through Director of Education, Superintendents of Education and so forth, by missions (with considerable Government aid under certain conditions) and by Commuttees of Management.

Malay vernacular schools are chiefly to be found in rural areas. They provide free primary education in a four year course which includes reading and writing (both in arabic and romanised Malay) arithmetic and other subjects. In some schools are foudn fifth and sixth years. Secondary schools are now coming into existence both in the Federation and in Singapore. Shortages occur in both accommodation and staff : some students after their sixth year go on to train as teachers. The best at 10 or 11 enter English medium schools but receive two or three years intensive English before entering the general stream.

Tamil vernacular schools, to be found on most estates in Malaya usually go to Standard III, some to Standard VII : a few cover a six year course. The number of adequately trained teachers is insufficient. It should be stressed, however, that most Indians in the towns obtain education in the English medium schools.

In villages and towns of any size throughout Malaya will be found primary vernacular schools for Chinese boys and girls : some assisted by Government, some by missions but most by local committees of management which collect fees and subscriptions for support. In the past offers of aid from the Government have not been accepted, but recently strong demands for entire Government maintenance without any form of Government control have been made.\pageoriginale This year the Singapore Government has opened two Chinese schools, one Primary and one Secondary.

There is usually a Lower Primary extending over four years and after an Upper Primary for two years. The growth in Malaya corresponds to the growth of education in China since the revolution of 1911. The same curriculum was followed : most of the teachers came from China. The advent of a new Government in China has influenced Chinese education in Malaya to a considerable extent.

The English medium schools are open to all. Some students know English on entrance : most are taught English by the direct method. The course, formerly Primary I and II, Standards I-IX but now Primary I-VI, Forms I-VI in the Federation, Priamry I-VI, Forms II-VI in Singapore, extends over ten or eleven years after which a school certificate examination conducted by the Cambridge Syndics is taken. This external examination is modified to suit local needs.

As important as the Government schools are the Aided Schools (missions). They are controlled by school management but Government which takes the fees meets the pay-roll, pays an allowance (per head) for upkeep and is responsible for half the cost of approved new building.

Teachers in aided schools if properly qualified are paid at the same rates as those in Government. Special rates exist for missionary teachers. The aided schools play a particularly important part in the educatin of girls.

Mention must be made also of non-aided private schools which must comply with certain regulations.

In all schools run by Government or missions, many free places are provided. Bursaries and scholarships also exist to enable bright students to go further.

It is possible that drastic changes may come about in the school system of Singapore for multi-lingualism has been accepted by the Assembly and tri-lingualism may be introduced into the schools. The difficulties need no stressing.

From\pageoriginale the Chinese schools many students proceeded to Universities in China, but since this avenue has been in the main blocked of recent years a new University (Nanyang) has been created in Singapore by Chinese interets. This will open some time in 1956 with a staff largely recruited from Hongkong and overseas.

The English medium schools form the channel whereby students of all races proceed to higher education in such institutions as the College of Medicine and Raffles College (prior to 1949), the University of Malaya, to overseas Universities, and, for technical studies, to the Agricultural College at Serdang and the Technical College at Kuala Lumpur.

The College of Medicine founded in 1905 obtained recognition for its medical diploma some years later as a qualification registrable throughout the British Empire. Dentistry began in 1930.

In 1928 after nine years of preparation was founded Raffles College in Singapore to teach a limited range of subjects, English, History, Geography, Physics, Chemistry, Mathematics, later Economics (1933), and Education. For the College was conceived in a narrow spirit with its main function to provide teachers for the secondary schools. Conceived with such small vision with a very small staff and inadequate buildings it could not fulfil Raffle's high aims expressed so admirably a century before.

Such in brief was the position of education pre-1941 and for a short time after 1945. The war brought great changes to Malaya. In 1945 a derelict system of education had to be restored. Large numbers of unqualified teachers were employed so that a start could be made. In addition a demand for extension of education had to be met. To realise what this meant it is enough to state that in 1947 the number of children receiving education was 250,000. Less than five years later, the number exceeded 750,000. In particular in response to public pressure the pre-war demand for a University was granted after an enquiry by a Royal Commission.

The University of Malaya began in October 1949, on the basis of the existing College of Medicine and Raffles College, with Faculties of\pageoriginale Arts, Science and Medicine. New departments have been added : Botany, Zoology, Parasitology, Social Medicine, Malay Studies, Chinese Literature, Philosophy, Engineering has just begun mainly for the civil branch. Law and Public Administration will soon be added. And a beginning is expected shortly in Indian (mainly Tamil) studies.

To indicate the rate of expansion it is enough to say that the University began with 600 students in October 1949 : at present the number is 1300 of whom 60\%\ are Chinese, 12\%\ Malays, 10\%\ Indians. Nine hundred students have applied to sit for the University Entrance Examination for 1956. Of these over 300 want to red Medicine, a demand which cannot be satisfied even though the Faculty of Medicine is now geared to produce one hundred graduates per annum by 1959. We expect by 1959 to cater for some 2,000 students. For comparison let me mention the estimate by the Commission of 1947 that the number of University students might well grow to 2,000 or more by 1972. This will indicate how pressing is the demand for higher education and how great a burden has been placed on the resources of the country. We expect also by 1958 to have a University College near Kuala Lumpur.

Before I go on to describe what is done mathematically in Malaya I ought to state that some facilities for research existed in Malaya before 1949 in Medicine and in Biology : notably the Institute of Medical Research (1900), the Botanic Gardens (Singapore 1874), Fisheries and the Rubber Research Institute.

Another point of importance is the introduction since 1951 of post-school certificate classes in many schools despite the shortage of experienced teachers to prepare students for entrance to the University. Mention should also be made of recent arrangements to enable students from Chinese Middle Schools in Singapore with knowledge of English to sit for the University Entrance Examination.

After this introduction, sketchy despite its length and suffering I fear from many omissions, it is time to turn to the question of mathematics.

So\pageoriginale far as the elementary work in the schools is concerned, we are gradually replacing the traditionally separate subjects of Arithmetic, Algebra and Geometry by work designed to emphasize the natural links between the different topics.

In the primary stages we aim at quick and accurate responses to arithmetical problems of the kind to be met on leaving the school. Heavy manipulative work is being cut out : and in particular work with English currency. In the secondary stages, the aim is now to attain a fusion of mathematical subjects, particularly of trigonometry and geometry with the life and experience of the student, to introduce a dynamic aspect of functions as opposed to the static formula. Hence arises a reduction in the amount of formal geometry : the time so saved being used for three dimensional work, plan, elevation, simple navigation, simple trigonometry, ideas of rates of change.

\newpage

Some mechanics (chiefly in connection with science) is taught. Statistics of an elementary character is being introduced.

Two main difficulties may be mentioned here :
\begin{itemize}
\item[(i)] a sufficient supply of adequately trained teachers at all levels,

\item[(ii)] the supply of appropriate textbooks.
\end{itemize}

As for (i) we must suffer for some time since the great expansion of the schools has taken many teachers into administration and numerous new teachers have had only brief training. As for (ii) a Textbook Committee, set up in 1952, has made some progress. In particular a well-known English series has been revised with the author's co-operation to suit Malayan needs. Very helpful in this respect has been the use of a limited vocabulary (in English) carefully arranged. Teaching notes have been prepared.

In the Chinese Middle Schools (comparable to the English secondary schools) the position is not satisfactory. In successive years are taught Arithmetic, Algebra, Geometry, Plane Trigonometry, Advanced Algebra, Co-ordinate Geometry. Thus we find often a teacher of geometry or a teacher of algebra and not a teacher\pageoriginale of Mathematics. The textbooks in Chinese are built on the syllabus : one textbook for each year. In some cases Chinese translations of English texts are used : some schools use English texts but teach in Chinese.

Steps to remedy these defects (in all schools) are being taken. In the Department of Education at the University, in the new Training Colleges for teachers in Singapore and in Kota Bharu (and soon in Penang) as well as in the two training colleges in Britain used by the Federation of Malaya, vigorous work in proper training of teachers of mathematics is being undertaken. And teachers are being trained not merely for English medium schools but also for Chinese Middle Schools.

There is in the Federation a Promotion Examination in Arithmetic from the primary schools. Certain weaknesses have been revealed, for example, inability to multiply or divide mentally, lack of understanding of shortened methods. A detailed analysis is now being undertaken of a vast mass of scripts in order to arrive at precise determinations of the defects which arise.

I turn now to the work done by Raffles College from 1928 (its inception) until 1949 (with an interruption due to the occupation of Malaya from 1942-45) and the University of Malaya.

The course in Raffles College followed traditional English lines : it was modelled in the main on the Pass Degree of London University. Students (in science) read Physics, Chemistry, Pure and Applied Mathematics. The Diploma was in three parts, an examination in each part being held at the end of each year of the course. At one time the second Diploma examination was dropped, but it was found to be advisable later to reintroduce it, an experience shared also by other institutions in other countries.

As for the courses, Algebra included exponential and logarithmic series, convergence and divergence of infinite series, determinants, theory of equations, symmetric functions, Sturm's theorem (some times), location of roots, methods such as Newton's, Horner's, Bailey's for computation of roots. Trigonometry included de Moivre's Theorem\pageoriginale and applications, Euler's fundamental formula, the infinite series for the trigonometric functions (and occasionally the infinite products). A short course in Spherical Trigonometry was also given.

In plane co-ordinate geometry, the general conic and in three dimensions quadrics in standard form received attention. Occasionally it was possible to study curves and surfaces. Mention should be made of a short course in geometrical conics which was found to stimulate great interest. At times also a brief introduction to projective geometry was found possible.

In calculus work in both branches was covered to include informal work with infinite integrals, differentiation and integration of infinite series, ordinary differential equations and singular solutions (very elementary) partial differential equations.

Throughtout attempts were made to illuminate one field by ideas drawn from others and to introduce wherever possible some of the history of mathematics.

These methods were also used in Applied Mathematics which followed the customary courses in England for Statics, Dynamics and Hydrostatics. Included were Newton's Laws of Motion, orbits, rigid bodies, three dimensional statics. Occasional use was made of vector methods. Some students were shown Lagrange's equations.

The staff was small, two in all, with occasional assistance from a senior mathematics master in one of the schools when a member of the staff went on leave.

Of the Library, there is little to say. Adequate for the needs of a teaching institution, inadequate for research. Indeed one Principal of Raffles College asserted that a teaching institution had no business with research. Happily such views no longer prevail : all to-day are convinced that unless some research, even of a modern nature, is carried out, teaching will inevitably lose vitality.

From 1942 to 1945 Raffles College ceased to exist. In 1946 educational work began once more under circumstances of great difficulty familiar\pageoriginale to any country ravaged by war. From 1947 until 1949 the problems of founding the University of Malaya and coping with the increasing demand for education engaged the minds and energies of the small staff. At the outset it was agreed that Honours courses should be instituted but, since it was felt that the schools of the country would benefit more at that time from teachers with not too much specialization and since the staff of the University in Arts and in Science was still far too small to tackle both a three-year pass degree course and a three-year Honours Degree course, we began with an Honours course of one year's duration (or in some cases two) which followed for selected students the three year pass degree course. We gave in 1949 and 1950 far too much for a one year course. We gave introductions to the Foundations of Analysis. Functions of a Complex Variable, Matrices, Vector Analysis, Higher Dynamics, Hydro-dynamics, and some work on Special Functions and Statistics. From the pass degree course topics like Spherical Trigonometry have been omitted. Vector analysis has been introduced !

Fortunately since 1950 the staffing position has changed for the better. From four in 1950, of whom two were always deeply involved in administration in a growing concern, we have increased to eight in mathematics with a reasonably wide range in interests such as Statistics, Modern Algebra, Logic. The increase in staff throughout Arts and Science in the University is such that we can now begin to institute Honours Courses of two or three years duration after Intermediate for students of the right quality. Naturally the needs of physicists, chemists, engineers will not be forgotten.

The standard of students will improve with the output of teachers from the training colleges and the university. Genuine mathematical ability exists : it needs to be fostered but the resources of Malaya are not sufficient to maintain a team of mathematicians of the highest quality, covering all branches of our vast subject, able to direct and inspire young and vigorous talent. Plainly we must expect to find places in centres abroad for our best students and hope\pageoriginale that from time to time distinguished mathematicians will spend a few months in Malaya.

It should be pointed out also that subjects like Medicine attract a disproportionate number of able students. The reasons are clear : prestige and great financial rewards. There is also an unhappy tendency in Government circles to believe that only those trained in the Social Sciences are fitted for the highest ranks of Government service, although perhaps experience should have shown by now that those trained to administer are not necessarily good administrators. On the Library side we are now indeed fortunate. Funds have been made freely available. Many periodicals are now purchased. Many complete runs are available. A well designed air-conditioned building was erected in 1952.

I come now to what is in some ways the most important feature of our work in Malaya, the Malayan Mathematical Society. Let me say at once that our work is humble but we believe of some value. The Society which was founded in 1952 is affiliated to the Mathematical Association of Great Britain. It aimed at the start to bring under its wing, teachers of mathematics at all levels whether primary or university. At deliberately low rates students in the University and in the post-school certificate classes can become members. Institutional membership is also possible. And it is pleasing to record that an active Branch has been started in the Kota Bharu Training College in the remote state of Kelantan which has now a membership of 100. The parent society itself now numbers 400.

\newpage

It is plain that with such a wide range of mathematical activity to satisfy, the programme of the Society must be varied in the extreme.

\smallskip
Ten meetings are held each year, chiefly in Singapore. Talks are given by mathematicians at the University and the schools. Discussions take place on problems of teaching at different levels. Visiting mathematicians (very rare birds before the war) talk to us whenever occasion offers. In addition symposia are held each year (latterly with the recently formed Science Society) : noteworthy was\pageoriginale the convention on Mathematical education held last December in Kuala Lumpur organized by the Kota Bharu Branch and the Singapore Teachers Training College.

\smallskip
An exhibition of models was taken round the country by two members : the talks and exhibition proved so successful that requests for more have been received. Competitions have been started (with prizes) to encourage mathematical work at primary level and in the higher forms of schools. Slowly more teachers are being persuaded to write about their difficulties and to take an active part in the work of the Society. Many use the Problem Solving Bureau.

\smallskip
Finally the Society puts out---not in printed form which is beyond our means---but in mimeographed manner a monthly bulletin. It contains a wide variety of articles (some based on talks given to the Society), notes, problems, discussions and jottings. The work is heavy : must praise is due to my colleagues for their zeal : the reward is great. No teacher of mathematics in any part of Malaya, however remote and isolated (and some are indeed lonely) need feel entirely cut off from communication and discussion on topics relating to his field. For the sum of about ten shillings he receives each yer some three hundred pages of mimeographed material.

\smallskip
The problems then which face us in Malaya are those which I described earlier, not enough appropriate textbooks, too few adequately trained teachers, a vast demand for education at all levels. Slowly some of these difficulties are being faced. Slowly curricula are being modified to meet new needs. But we need help and encouragement in our task. For this aid we look confidently towards the International Mathematical Union, certain that the wise and experienced members of this Union wish to foster in Malaya and in Singapore the great subject to which our lives have been devoted.

\bigskip
\medskip

{\fontsize{9pt}{11pt}\selectfont
University of Malaya

Singapore
}\relax

