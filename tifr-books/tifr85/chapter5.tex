
\chapter{Zeros of $\zeta(s)$}\label{c5}

\section{Introduction}\label{c5:sec5.1}\pageoriginale 

In this chapter we deal with three results. 
In \S\ \ref{c5:sec5.2} we deal with a
simple proof (due to K.\@ Ramachandra) of the inequality $\theta\geq
\frac{1}{2}$, where $\theta$ is the least upper bound of the real
parts of the zeros of $\zeta(s)$. (Trivially from the Euler product we
have $\theta\leq 1$). This proof (which does not use
Borel-Caratheodory theorem and Hadamard's three circle theorem) has
some advantages. We will make some remarks about the proof which uses
the two theorems in the brackets. It has the advantage that it
generalises very much. In \S\ \ref{c5:sec5.3} we mention some
localisation of theorems of Littlewood and Selberg. These
localisations are due to K.\@ Ramachandra and A.\@ Sankaranarayanan
whose proof has the advantage that it generalises very much. Lastly
\S\ \ref{c5:sec5.4} deals with a proof due to J.B.\@ Conrey, A.\@
Ghosh and S.M.\@ Gonek, that $\zeta(s)$ has infinity of simple zeros
in $\sigma\geq 0$. Only the last section uses the functional equation
and some difficult machinery viz. asymptotics of $\Gamma(s)$ and so
on. These will be proved in the appendix in the last chapter.

\section{Infinitude of Zeros in $t\geq 1$}\label{c5:sec5.2}

First we give a simple proof of the inequality $\theta\geq
\frac{1}{2}$ and then remark about another proof. We will prove the
following theorem.

\begin{theorem}\label{c5:thm5.2.1}
We have
\begin{equation*}
\theta\geq \frac{1}{4}\tag{5.2.1}\label{c5:eq5.2.1}
\end{equation*}
\end{theorem}

\setcounter{remark}{0}
\begin{remark}\label{c5:rem1}
The method of proof actually gives
\begin{equation*}
\theta\geq \frac{1}{2}.\tag{5.2.2}\label{c5:eq5.2.2}
\end{equation*}

To see this we have only to replace $\frac{1}{4}$ in our proof by
$\frac{1}{2}-\delta$ where $\delta(0<\delta<\frac{1}{4})$ is any small
constant. Also we can prove the existence of at least one zero in
$(\sigma\geq \frac{1}{2}-\delta,T\leq t\leq T+T^{\epsilon})$ for
$T\geq T_{0}(\epsilon,\delta)$ and constants $\epsilon$, $\delta$ with
$0<\epsilon<1$, $0<\delta<\frac{1}{4}$.
\end{remark}

\begin{remark}\label{c5:rem2}
All that we use in our proof is the Euler product and analytic 
continuation\pageoriginale in $\sigma\geq \frac{1}{10}$ and the bound
$|\zeta(s)|\leq t^{A}(t\geq 2)$ where $A$ is any constant. Hence
\eqref{c5:eq5.2.2} holds good for the zeta and $L$-funcitons of any
algebraic number field. We do not need the functional equation. In
fact we may dispense with the Euler product and prove some worthwhile
results (see the notes at the end of this chapter).
\end{remark}

\begin{remark}\label{c5:rem3}
Our method shows that $\zeta(s)$ has $\gg T(\log\log T)^{-1}$ zeros in
$(\sigma\geq \frac{1}{2}-20(\log\log\log T)(\log T)^{-1}, T\leq
t\leq 2T)$. (See the notes at the end of this chapter).
\end{remark}

\setcounter{lem}{0}
\begin{lem}\label{c5:lem1}
Let $w=u+iv$ be a complex variable. Then for $y>0$, we have,
\begin{equation*}
\frac{1}{2\pi
  i}\int^{1+i\infty}_{1-i\infty}\frac{y^{w}dw}{w(w+1)}=1-\frac{1}{y}\quad
\text{or}\quad 0\tag{5.2.3}\label{c5:eq5.2.3}
\end{equation*}
according as $y\geq 1$ or $y\leq 1$.
\end{lem}

\begin{proof}
If $y\geq 1$ we apply Cauchy's theorem and obtain that the LHS of
\eqref{c5:eq5.2.3} is 
\begin{equation*}
1-\frac{1}{y}+\frac{1}{2\pi
  i}\int^{-R+i\infty}_{-R-i\infty}\frac{y^{w}dw}{w(w+1)}\tag{5.2.4}\label{c5:eq5.2.4} 
\end{equation*}
and it is easily seen that as $R\to \infty$, the last integral tends
to zero. If $y<1$, we apply Cauchy's theorem and obtain that the LHS
of \eqref{c5:eq5.2.3} is
\begin{equation*}
\frac{1}{2\pi
  i}\int^{R+i\infty}_{R-i\infty}\frac{y^{w}dw}{w(w+1)}\tag{5.2.5}\label{c5:eq5.2.5} 
\end{equation*}
which tends to zero as $R\to \infty$.
\end{proof}

\begin{lem}\label{c5:lem2}
Let $s=\sigma+it$ and let $d_{k}(n)$ be defined for any complex
constant $k$ by
$(\zeta(s))^{k}=\sum\limits^{\infty}_{n=1}d_{k}(n)n^{-s}$ where
$\sigma\geq 2$. If $0<k\leq 1$ then {\rm(i)} $0<d_{k}(n)\leq 1$ for
all $n$ and {\rm(ii)} $d_{k}(p)=k$ for all primes $p$.
\end{lem}

\begin{proof}
The lemma follows from
$$
(1-p^{-s})^{-k}=1+kp^{-s}+\frac{k(k+1)}{2!}p^{-2s}+\cdots
$$
and\pageoriginale the fact that
$$
(\zeta(s))^{k}=\prod_{p}(1-p^{-s})^{-k}.
$$
\end{proof}

\begin{lem}\label{c5:lem3}
Let $T\geq 10$ and $\zeta(s)\neq 0$ in $(\sigma\geq
\frac{1}{4},\frac{1}{2}T\leq t\leq \frac{5}{2}T)$. Put
$G(s)=(\zeta(s))^{k}$ where $k=q^{-1}$ and $q\geq 1$ is an integer
constant. For $X\geq 1$ define $A(s)=A(s,X)$ and $b_{n}=b_{n}(X)$ by
\begin{equation*}
\frac{1}{2\pi
  i}\int^{1+\infty}_{1-i\infty}\frac{G(s+w)X^{w}(2^{w}-1)dw}{w(w+1)}=\sum^{\infty}_{n=1}b_{n}n^{-s}=A(s),\tag{5.2.6}\label{c5:eq5.2.6} 
\end{equation*}
where $s=\frac{1}{4}+it$ and $T\leq t\leq 2T$. Then
\begin{itemize}
\item[\rm(i)] $b_{n}=d_{k}(n)(\frac{n}{2X})$ for $1\leq n\leq X$,

\item[\rm(ii)] $b_{n}=d_{k}(n)(1-\frac{n}{2X})$ for $X\leq n\leq 2X$, and

\item[\rm(iii)] $b_{n}=0$ for $n\geq 2X$.
\end{itemize}

In particular $|b_{n}|\leq 1$ for all $n$ and $b_{p}=\frac{p}{2qX}$
for primes $p\leq X$.
\end{lem}

\begin{proof}
Lemma \ref{c5:lem3} follows from Lemma \ref{c5:lem1} since on
$u=1$ the series for $G(s+w)$ is absolutely convergent.
\end{proof}

\begin{lem}\label{c5:lem4}
We have, for $T\geq 10$,
\begin{equation*}
\frac{1}{T}\int\limits^{2T}_{T}|A\left(\frac{1}{4}+it\right)|^{2}dt\geq
\sum_{p\leq X}\left(\frac{p}{2qX}\right)^{2}p^{-\frac{1}{2}}-16X^{3}T^{-1}.\tag{5.2.7}\label{c5:eq5.2.7}  
\end{equation*}
\end{lem}

\begin{proof}
We multiply $A(s)$ by its complex conjugate and integrate term by
term. The LHS of \eqref{c5:eq5.2.7} is thus seen to be
$$
\sum_{n\leq
  2X}b^{2}_{n}n^{-\frac{1}{2}}+\frac{2\psi}{T}\sum_{m\neq
  n}b_{m}b_{n}(mn)^{-\frac{1}{4}}|\log\frac{m}{n}|^{-1}, 
$$
where $|\psi|\leq 1$. If $m>n$, we have, $\log \frac{m}{n}=-\log
(1-(1-\frac{n}{m}))\geq I-\frac{n}{m}=\frac{m-n}{m}\geq
(2X)^{-1}$. Hence we have the lower bound
$$
\sum_{p\leq X}b^{2}_{p}p^{-\frac{1}{2}}-\frac{4X}{T}\sum\sum_{m\neq n}1
$$
and\pageoriginale the required lower bound follows.
\end{proof}

\begin{lem}\label{c5:lem5}
For $X\geq 350$, we have, with usual notation,
\begin{equation*}
\pi(X)-\pi\left(\frac{X}{2}\right)>(6\log
X)^{-1}(X-18X^{\frac{1}{2}}).\tag{5.2.8}\label{c5:eq5.2.8} 
\end{equation*}
\end{lem}

\begin{remark*}
S.\@ Ramanujan proved this result in an easy and elementary way. It is
possible to resort to using simpler and easier results in place of
this lemma, but we do not do it here.
\end{remark*}

\begin{lem}\label{c5:lem6}
Suppose that $X=T^{\frac{1}{3}}$ and that $T$ exceeds a large absolute
constant. Then for every fixed integer $q\geq 1$,
\begin{equation*}
\max\limits_{T\leq t\leq 2T}|A\left(\frac{1}{4}+it\right)|\gg
X^{\frac{1}{4}}(\log
X)^{-\frac{1}{2}}.\tag{5.2.9}\label{c5:eq5.2.9} 
\end{equation*}
\end{lem}

\begin{proof}
We have
$$
\sum_{p\leq X}\left(\frac{p}{2qX}\right)^{2}p^{-\frac{1}{2}}\geq
\sum_{\frac{1}{2}X\leq p\leq X}\ldots
$$
and so Lemma \ref{c5:lem6} follows from Lemmas \ref{c5:lem4} and
\ref{c5:lem5}. 
\end{proof}

\begin{lem}\label{c5:lem7}
We have, for $s=\frac{1}{4}+it$, $T\leq t\leq 2T$ the inequality
\begin{equation*}
|A(s)|\leq
\frac{1}{2\pi}\int_{(w)}|\frac{G(s+w)X^{w}(2^{w}-1)}{w(w+1)}dw|\tag{5.2.10}\label{c5:eq5.2.10} 
\end{equation*}
where the contour of integration is the union of straight line
segments obtained by joining the points $1-i\infty$,
$1-\frac{1}{2}iT$, $-\frac{1}{2}iT$, $\frac{1}{2}iT$,
$1+\frac{1}{2}iT$, $1+i\infty$ in this order.
\end{lem}

\begin{proof}
The lemma follows by Cauchy's theorem.
\end{proof}

We now fix $t$ to be the point in $T\leq t\leq 2T$ at which
$|A(\frac{1}{4}+it)|$ attains its maximum. We estimate the integral in
Lemma \ref{c5:lem7} from above. This will lead to a contradiction
as we will see. We begin with

\begin{lem}\label{c5:lem8}
\begin{itemize}
\item[\rm(i)] On $u=1$ we have $|\zeta(s+w)|\leq 5$. 

\item[\rm(ii)] In\pageoriginale $\sigma\geq \frac{1}{4}$, $u\geq 0$,
  $|v|\leq \frac{1}{2}T$ we have, $|\zeta(s+w)|\leq 100T$, for $T\geq
  1000$. 
\end{itemize}
\end{lem}

\begin{proof}
We have $|\zeta(s+w)|\leq \zeta(\frac{5}{4})\leq
1+\int^{\infty}_{1}u^{-\frac{5}{4}}du=5$. This proves (i). Next
$$
\zeta(s+w)=\sum^{\infty}_{n=1}(n^{-s-w}-\int^{n+1}_{n}u^{-s-w}du)+(s+w-1)^{-1} 
$$
and the fact that the infinite series here is
$$
(s+w)\sum^{\infty}_{n=1}\int^{n+1}_{n}\left(\int^{u}_{n}v^{-s-w-1}dv\right)du 
$$
complete the proof of (ii).
\end{proof}

\begin{lem}\label{c5:lem9}
Let $T$ exceed a large constant, $X=T^{\frac{1}{3}}$ and $q=100$. Then
the inequality asserted by Lemma \ref{c5:lem7} is false if $t(T\leq
t\leq 2T)$ is fixed such that $|A(\frac{1}{4}+it)|$ is maximum.
\end{lem}

\begin{proof}
The contribution from the segments on $u=1$ is
$$
\leq \int_{|v|\geq \frac{1}{2}T}15X\frac{dv}{v^{2}}=\frac{60X}{T}<1.
$$

The contribution from the two horizontal segments on $v=\pm
\frac{1}{2}T$ is
$$
\leq 6(100T)^{q^{-1}}\left(\frac{1}{2}T\right)^{-2}\leq 1.
$$

The contribution from the remaining part on $u=0$ is
\begin{align*}
&\leq (100T)^{q^{-1}}\left\{\int_{|v|\leq
    1}|\frac{2^{iv}-1}{v}|dv+\int_{|v|\geq
    1}\frac{|2^{iv}-1|}{v^{2}}dv\right\}\\
&\leq (100T)^{q^{-1}}\{4+4\}\leq 8(100T)^{q^{-1}}.
\end{align*}

The contradiction is now immediate.

Lemma \ref{c5:lem9} completes the proof of Theorem
\ref{c5:thm5.2.1}. 
\end{proof}

By using Borel-Caratheodory's Theorem \ref{c1:thm1.6.1}, Hadamard's
three circle Theorem \ref{c1:thm1.5.2} and things like the third
main theorem of \S\ \ref{c2:sec2.5} (or easier
theorems)\pageoriginale we can prove the following theorem.

\begin{theorem}\label{c5:thm5.2.2}
Let $\{\lambda_{n}\}$ be a sequence satisfying
$1=\lambda_{1}<\lambda_{2}<\ldots$ where $C^{-1}_{1}\leq
\lambda_{n+1}-\lambda_{n}\leq C_{1}$ (for some constant $C_{1}\geq
1$), and let $\{a_{n}\}$ be any sequence of complex numbers such that
the series $\sum\limits^{\infty}_{n=1}(a_{n}\lambda^{-s}_{n})^{2}$ has
a finite abscissa of absolute convergence say $C_{2}$. By replacing
$a_{n}$ by $a'_{n}=a_{n}\lambda^{C_{2}-\frac{1}{2}}_{n}$ if necessary
we can assume, as we do, that $C_{2}=\frac{1}{2}$. Suppose that
$F(s)=\sum\limits^{\infty}_{n=1}a_{n}\lambda^{-s}_{n}$ (which is
certainly absolutely convergent in $\sigma>1$) be continuable
analytically in $(\sigma\geq \frac{1}{2}-\delta,t\geq t_{0})$ where
$\delta(>0)$ and $t_{0}(\geq 10)$ are some constants and there
$|F(s)|<t^{A}$, for some constant $A\geq 10$. Let $\epsilon(>0)$ be
any constant and $H=T^{\epsilon}$. Then there exists an infinite
sequence $\{T_{\nu}\}(v=1,2,3,\ldots)$ such that $T_{\nu}\to \infty$
and if $T=T_{\nu}$, then the rectangle $(\sigma\geq
\frac{1}{2}-\delta,t\in I)$ contains at least one zero of $F(s)$
provided $I$ is any sub-internal of $(T,2T)$ of length $H$. In
particular if $\theta$ is the least upper bound of the real parts of
the zeros of $F(s)$ then $\theta\geq \frac{1}{2}$. 
\end{theorem}

\begin{remark*}
The proof of this theorem essentially due to Littlewood.\break Roughly
speaking it is enough to disprove the analogue of Lindel\"of
hypothesis on $\sigma=\frac{1}{2}-\delta$, for $F(s)$. This is done by
the third main theorem of \S\ \ref{c2:sec2.5} (or easier
theorems). For details of proof see the proof of Theorem 14.2 on pages
336 and 337 of E.C.\@ Titchmarsh \cite{Titchmarsh1}. By imposing some very
mild extra conditions we can even take $\delta=C_{3}(\log\log
t)^{-1}$, where $C_{3}(>0)$ is a certain constant. But this needs the
results of \S\ \ref{c5:sec5.3}. (See the notes at the end of this
chapter). 
\end{remark*}

\section{On Some Theorems of Littlewood and
  Selberg}\label{c5:sec5.3}

Now we ask the following question: What are upper and lower bounds for
$Re\ \log F(s)$ and $Im\ \log F(s)$ if there are no zeros in certain
rectangles? We mean the ``localised analogue'' of the results of
Littlewood and Selberg which they prove assuming RH. (See for example
Theorem 14.14 (B) and equation (14.14.5) on pages 354 and 355 of
E.C.\@ Titchmarsh \cite{Titchmarsh1}). 

We\pageoriginale state two theorems in this direction.

\begin{theorem}\label{c5:thm5.3.1}
Let $s=\sigma+it$ and
\begin{equation*}
F(s)=\sum^{\infty}_{n=1}a_{n}n^{-s}=\prod_{p}(1-w(p)p^{-s})^{-1},\tag{5.3.1}\label{c5:eq5.3.1} 
\end{equation*}
where $p$ runs over all primes and $w(p)$ are arbitrary complex
numbers (independent of $s$) with absolute value not exceeding
$1$. Suppose $\alpha$ and $\delta$ are positive constans satisfying
$\frac{1}{2}\leq \alpha\leq 1-\delta$ and that in $(\sigma\geq
\alpha-\delta, T-H\leq t\leq T+H)$, $F(s)$ can be continued
analytically and there $|F(s)|<T^{A}$. Here $T\geq T_{0}$, $H=C
\log\log\log T$ and $A$, $T_{0}$ and $C$ are large positive constants
of which $C$ depends on $T_{0}$ and $A$. Let $F(s)\neq 0$ in
$(\sigma>\alpha,T-H\leq t\leq T+H)$. Then for $\alpha\leq \sigma\leq
\alpha+C_{1}(\log\log T)^{-1}$ and $T-\frac{1}{2}H\leq t\leq
T+\frac{1}{2}H$, we have, uniformly in $\sigma$,
\begin{itemize}
\item[\rm(a)] $\log |F(\sigma+it)|$ lies between $C_{2}(\log
  T)(\log\log T)^{-1}$ and

$-C_{3}(\log T)(\log\log T)^{-1}\log \{C_{4}((\sigma-\alpha)\log\log
  T)^{-1}\}$ and

\item[\rm(b)] $|\arg F(\sigma+it)|\leq C_{5}(\log T)(\log\log
  T)^{-1}$, where $C_{1}$, $C_{2}$, $C_{3}$, $C_{4}$ and $C_{5}$ are
  certain positive constants.
\end{itemize}
\end{theorem}

\begin{corollary}\label{c5:coro1}
For $\alpha+C_{1}(\log\log T)^{-1}\leq \sigma\leq 1-\delta$, $t=T$, we
have 
$$
|\log F(\sigma+it)|\leq C_{6}(\log T)^{\theta}(\log\log T)^{-1},
$$
where $\theta=(1-\sigma)(1-\alpha)^{-1}$ and $C_{6}$ is a certain
constant. The inequality holds uniformly in $\sigma$. 
\end{corollary}

\begin{corollary}\label{c5:coro2}
For $\alpha\leq \sigma\leq 1-\delta$, $t=T$, we have,
$$
|F(\sigma+it)|\leq \Exp(C_{6}(\log T)^{\theta}(\log\log T)^{-1})
$$
where $\theta$ is the same as in Corollary \ref{c5:coro1}. ($C_{6}$
may not be the same as before).
\end{corollary}

Theorem \ref{c5:thm5.3.1} is nearly true of functions, very much
more general than the ones given by \eqref{c5:eq5.3.1}. In this
direction we state the following theorem.

\begin{theorem}\label{c5:thm5.3.2}
Let\pageoriginale $\{\lambda_{n}\}$ be a sequence satisfying
$1=\lambda_{1}<\lambda_{2}<\ldots$ where $C^{-1}_{1}\leq
\lambda_{n+1}-\lambda_{n}\leq C_{1}$ (for some constant $C_{1}\geq 1$)
and let $\{a_{n}\}$ be a sequence of complex numbers such that the
series $F(s)=\sum\limits^{\infty}_{n=1}a_{n}\lambda^{-s}_{n}$
converges for some complex $s$ and continuable analytically in
$(\sigma\geq \alpha-\delta, T-H\leq t\leq T+H)$ and there
$|F(s)|<T^{A}$, $T\geq T_{0}$, $H=C\log\log\log T$. Here $\alpha$,
$\delta$, $A$ are positive constants with $\alpha>\delta$, $T_{0}$ and
$C$ are large positive constants. Let $F(s)\neq 0$ in
$(\sigma>\alpha,T-H\leq t\leq T+H)$. Then the conclusions {\rm(a)} and
{\rm(b)} of Theorem \ref{c5:thm5.3.1} hold good without any change.
\end{theorem}

Next assume that $F(1+it)=O((\log t)^{A})$ for all $t\geq t_{0}$
($t_{0}$ being a constant) and that $H=C(\log\log T)(\log\log\log T)$
in place of the earlier condition on $H$. Then we have the following
corollaries (the inequalities asserted here hold uniformly in
$\sigma$).

\setcounter{corollary}{0}
\begin{corollary}\label{c5:addcoro1}
For $\alpha+C_{1}(\log\log T)^{-1}\leq \sigma\leq 1-\delta$, $t=T$, we
have, 
$$
|\log F(\sigma+it)|\leq C_{6}(\log T)^{\theta}(\log\log T)^{C_{7}},
$$
where $\theta=(1-\sigma)(1-\alpha)^{-1}$ and $C_{6}$ and $C_{7}$ are
certain positive constants.
\end{corollary}

\begin{corollary}\label{c5:addcoro2}
For $\alpha\leq \sigma\leq 1-\delta$, $t=T$, we have,
$$
|F(\sigma+it)|\leq \Exp(C_{6}(\log T)^{\theta}(\log\log T)^{C_{7}})
$$
where $C_{6}$ and $C_{7}$ may not be the same as before.
\end{corollary}

\section{Infinitude of Simple Zeros in $t\geq 1$}\label{c5:sec5.4}

In this section we prove (following J.B.\@ Conrey, A.\@ Ghosh and
S.M.\@ Gonek) that for all $T$ exceeding a large positive constant
$T_{0}$,
\begin{equation*}
Re\ \sum\limits_{1\leq Im\ \rho\leq T}\zeta'(\rho)=\frac{T}{4\pi}(\log
T)^{2}+O(T\log T),\tag{5.4.1}\label{c5:eq5.4.1}
\end{equation*}
where the sum on the left is over all zeros $\rho$ with $1\leq
Im\ \rho \leq T$ (of course for all such zeros $\rho$ of $\zeta(s)$ we
must have (see the last chapter), necessarily $0\leq Re\ \rho\leq
1$). As a corollary we have the following theorem. 

\begin{theorem}\label{c5:thm5.4.1}
There\pageoriginale are infinity of simple zeros of $\zeta(s)$ in
$0\leq\sigma\leq 1$, $t\geq 1$.
\end{theorem}

\begin{remark*}
In fact the three authors (mentioned above) prove by their simple
method that the sum on the LHS of \eqref{c5:eq5.4.1} is
$(4\pi)^{-1}T(\log T)^{2}+O(T\log T)$. Their ``simple method'' uses
the functional equation of $\zeta(s)$, the asymptotics of gamma
function and etc. It is relatively very simple compared with other
methods namely that of N.\@ Levinson, D.R.\@ Heath-Brown and A.\@
Selberg.
\end{remark*}

Throughout our proof of \eqref{c5:eq5.4.1} we write
$L=\log(2T)$. We begin with the remarks that $\sum \zeta'(\rho)$
summed up over zeros $\rho$ with $T-1\leq Im \, \rho\leq T$ is
$O_{\epsilon}(T^{\frac{1}{2}+\epsilon})$ for every $\epsilon>0$ and
that there is a $T'$ satisfying $T-1\leq T'\leq T$ for which
$|\zeta'(\sigma+iT')(\zeta(\sigma+iT'))^{-1}|\ll L^{2}$ uniformly in
$-1\leq \sigma\leq 2$. Another result which we will be using is
$\zeta(s)=\chi(s)\zeta(1-s)$ where for $t\geq 1$, $|\chi(s)|\asymp
t^{\frac{1}{2}-\sigma}$ uniformly for $\sigma$ in any closed bounded
interval. Yet another result which we will be using is
$\chi'(s)(\chi(s))^{-1}=-\log(\frac{t}{2\pi})+O(t^{-1})$ uniformly for
$\sigma$ in any closed bounded interval. Finally we need the lemma on
page 143 of E.C.\@ Titchmarsh \cite{Titchmarsh1}. So we find that all the
results that we need are in this book with proper references. Except
this lemma on page 143 we will prove all the results that we need in
the appendix, which forms our last chapter. The method consists in
considering the integral
\begin{equation*}
\frac{1}{2\pi i}\int_{(s)}F(s)ds,
F(s)=(\zeta'(s))^{2}(\zeta(s))^{-1},\tag{5.4.2}\label{c5:eq5.4.2} 
\end{equation*}
taken over the anticlockwise boundary of the rectangle obtained by
joining (by straight line segments) the points $c+i$, $c+iT'$,
$1-c+iT'$, $1-c+i$, $c+i$ in this order. We will fix
$c=1+L^{-1}$. Clearly the integral is the sum
$\sum\limits_{\cdots}\zeta'(\rho)$ which figures on the LHS of
\eqref{c5:eq5.4.1} plus a quantity
$O_{\epsilon}(T^{\frac{1}{2}+\epsilon})$. We will show that the right
vertical line and the two horizontal lines contribute
$O_{\epsilon}(T^{\frac{1}{2}+\epsilon})$ and that the real part of the
contribution from the left vertical line is\pageoriginale
$(4\pi)^{-1}T(\log T)^{2}+O(T\log T)$. This proves all that we
want. Now
\begin{equation*}
F(s)=\frac{\zeta'(s)}{\zeta(s)}\zeta'(s)=\sum\sum_{m\geq 1,n\geq
  1}(\Lambda(m)\log n)(mn)^{-s}\tag{5.4.3}\label{c5:eq5.4.3}
\end{equation*}
and so the right vertical line contributes
$$O(\sum\sum\limits_{\cdots}(\Lambda(m)(\log
n)(mn)^{-c})=O((\zeta'(c))^{2}(\zeta(c))^{-1})=O(L^{3}).$$ 
The two horizontal lines contribute  $O_{\epsilon}(T^{\frac{1}{2}+\epsilon})$
by the choice of $T'$. Thus we are left with 
\begin{gather*}
I_{0}=\frac{1}{2\pi
  i}\int^{1-c+i}_{1-c+iT'}F(s)ds,\tag{5.4.4}\label{c5:eq5.4.4}\\
I_{0}=\frac{1}{2\pi}\int^{1}_{T'}F(1-c+it)dt=-\frac{1}{2\pi}\int^{T'}_{1}F(1-c+it)dt.\tag{5.4.5}\label{c5:eq5.4.5} 
\end{gather*}

Here after we may suppose, as we will, that $T'$ is replaced by $T$
since the error is $O_{\epsilon}(T^{\frac{1}{2}+\epsilon})$. With this
we have
\begin{equation*}
\overline{I}_{0}+O_{\epsilon}(T^{\frac{1}{2}+\epsilon})=-\frac{1}{2\pi}\int^{T}_{1}F(1-c-it)dt=-\frac{1}{2\pi
  i}\int^{c+iT}_{c+i}F(1-s)ds.\tag{5.4.6}\label{c5:eq5.4.6} 
\end{equation*}

We will prove that the last expression involving the integral has the
real part $(4\pi)^{-1}T(\log T)^{2}+O(T\log T)$. Let us write $\chi$,
$\zeta$, $\zeta'$ for $\chi(s)$, $\zeta(s)$ and $\zeta'(s)$. We have
\begin{gather*}
\zeta(s)=\chi\zeta(1-s),(\zeta(1-s))^{-1}=\chi\zeta^{-1},\\
\zeta'=\chi'\zeta(1-s)-\chi\zeta'(1-s)=\chi'\chi^{-1}\zeta-\chi \zeta'(1-s).
\end{gather*}
(Note that $\chi\chi(1-s)=1$). Hence
$\zeta'(1-s)=\chi^{-1}(\chi'\chi^{-1}\zeta-\zeta')$ and so
\begin{align*}
F(1-s) &=
\frac{\chi}{\zeta}\left(\left(\frac{\chi'}{\chi}\zeta-\zeta'\right)\chi^{-1}\right)^{2}=\frac{\chi(1-s)}{\zeta}\left(\frac{\chi'}{\chi}\zeta-\zeta'\right)^{2}\\ 
&=
\chi(1-s)\left\{\left(\frac{\chi'}{\chi}\right)^{2}\zeta-2\frac{\chi'}{\chi}\zeta'+F(s)\right\}.\tag{5.4.7} 
\end{align*}

Hence (with an error $O_{\epsilon}(T^{\frac{1}{2}+\epsilon})$),
\begin{equation*}
-\overline{I}_{0}=I_{1}-2I_{2}+I_{3},\tag{5.4.8}\label{c5:eq5.4.8}
\end{equation*}
where\pageoriginale
\begin{gather*}
I_{1}=\frac{1}{2\pi i}\int
\chi(1-s)\left(\frac{\chi'}{\chi}\right)^{2}\zeta dx,
I_{2}=\frac{1}{2\pi i}\int
\chi(1-s)\frac{\chi'}{\chi}\zeta'ds \quad\text{and}\\
I_{3}=\frac{1}{2\pi
  i}\int\chi(1-s)F(s)ds\Big\}\tag{5.4.9}\label{c5:eq5.4.9} 
\end{gather*}
and the integrals being taken from $c+i$ to $c+iT$. 

\setcounter{lem}{0}
\begin{lem}\label{c5:sec1.4-lem1}
If $n<\frac{T}{2\pi}$,
\begin{equation*}
\frac{1}{2\pi
  i}\int^{\frac{1}{2}+iT}_{\frac{1}{2}-iT}\chi(1-s)n^{-s}ds=2+O\left(n^{-\frac{1}{2}}\left(\log
\frac{T}{2n\pi}\right)^{-1}\right)+O(n^{-\frac{1}{2}}\log
T).\tag{5.4.10}\label{c5:eq5.4.10} 
\end{equation*}

If $n>\frac{T}{2\pi}$ and $c>\frac{1}{2}$, 
\begin{equation*}
\frac{1}{2\pi
  i}\int^{c+iT}_{e-iT}\chi(1-s)n^{-s}ds=O\left(T^{c-\frac{1}{2}}n^{-c}\left(\log
\frac{2n\pi}{T}\right)^{-1}\right)+O(T^{c-\frac{1}{2}}n^{-c}).\tag{5.4.11}\label{c5:eq5.4.11} 
\end{equation*}
\end{lem}

\setcounter{remark}{0}
\begin{remark}\label{c5:sec1.4-rem1}
This is the lemma on page 143 of E.C.\@ Titchmarsh \cite{Titchmarsh1}. Out
choice will be, as stated already, $c=1+L^{-1}$.
\end{remark}

\begin{remark}\label{c5:sec1.4-rem2}
The following result which says more is due to S.M.\@ Gonek. (We state
Lemma \ref{c5:sec1.4-lem1} of his paper the reference to which will
be given in the notes at the end of this chapter). There is a constant
$c>0$ (not $1+L^{-1}$) such that
\begin{align*}
& \int^{r(1+c)}_{r(1-c)}\Exp [\text{it~ }
    \log \left(\frac{t}{er} \right)] \left(\frac{t}{2\pi} \right)^{a^{-\frac{1}{2}}}dt\\
&= (2\pi)^{1-a}r^{a}\Exp \left(-ir+\frac{i\pi}{4} \right)+O(r^{a-\frac{1}{2}})
\end{align*}
for all real constants $a$ and all real $r\geq r_{0}(a)$. See also
Lemma 3.3 of N.\@ Levinson \cite{Levinson2}.
\end{remark}

Now let us look at the first part of the lemma. Here LHS is by
Cauchy's theorem 
\begin{align*}
& \frac{1}{2\pi
    i}\left(\int^{c+iT}_{c-iT}+\int^{c-iT}_{\frac{1}{2}-iT}+\int^{\frac{1}{2}+iT}_{c+iT}\right)\chi(1-s)n^{-s}ds\\
& =\frac{1}{2\pi    i}\int^{c+iT}_{c-iT}\chi(1-s)n^{-s}ds+O\left(\int^{c}_{\frac{1}{2}}T^{\frac{1}{2}-(1-\sigma)}n^{-\sigma}d\sigma\right). 
\end{align*}\pageoriginale

Hence if $\{a_{n}\}$ is any sequence of complex numbers with
$|a_{n}|\leq\break(\log (n+2))^{A}$ (for some constant $A>0$ which is
arbitrary) and $c=1+L^{-1}$, then we have (with $\tau=(2\pi)^{-1}T$), 
\begin{align*}
& \frac{1}{2\pi
    i}\int^{c+iT}_{c-iT}\chi(1-s)\left(\sum_{n<\tau-2}+\sum_{|n-\tau|\leq
    2}+\sum_{n>\tau+2}\right)a_{n}n^{-s}ds\\
&=2\sum_{n\leq
\tau-2}a_{n}\left(1+O\left(n^{-\frac{1}{2}}\left(\log\frac{\tau}{n}\right)^{-1}\right)+O\left(n^{-\frac{1}{2}}\log
  T\right)+O\left(n^{-c}T^{\frac{1}{2}}\right)\right)\\ 
&+O\left(\int^{c+iT}_{c-iT}T^{\epsilon}T^{-c}(|t|+1)^{\frac{1}{2}}|ds|\right)\\
&+O\left(\sum_{n>\tau+2}a_{n}T^{c-\frac{1}{2}}n^{-c} \left(\log\frac{n}{\tau} \right)^{-1}\right)+O\left(\sum_{n>\tau+2}a_{n}T^{c-\frac{1}{2}}n^{-c}\right)\\
&=2\sum_{n\leq \tau}a_{n}+O\left(T^{\epsilon}\sum_{n\leq
    \tau-2}\left(n^{-\frac{1}{2}}\left(\log\frac{\tau}{n}\right)^{-1}+n^{-\frac{1}{2}}+n^{-1}T^{\frac{1}{2}}\right)\right)\\
&+O\left(\sum_{n>\tau+2}\left(T^{\frac{1}{2}}|a_{n}|n^{-c}\left(\log
  \frac{n}{\tau}\right)^{-1}+|a_{n}|n^{-c}T^{\frac{1}{2}}\right)\right)\\
&=2\sum_{2\leq \tau}a_{n}+O(T^{\frac{1}{2}+\epsilon}),
\end{align*}
since for example
$$
\sum_{\frac{1}{2}\tau\leq n\leq \tau-2}n^{-\frac{1}{2}}\left(\log
\frac{\tau}{n}\right)^{-1}=O\left(T^{-\frac{1}{2}}\sum\limits_{\cdots}\frac{\tau}{\tau-n}\right)=O(T^{\frac{1}{2}+\epsilon}). 
$$

Thus we have proved 

\begin{lem}\label{c5:sec1.4-lem2}
Let\pageoriginale $c=1+L^{-1}$ and $F_{0}(s)=(\zeta'(s))^{2}(\zeta(s))^{-1}$ or
$\zeta'(s)$ of $\zeta(s)$ and let $a_{n}$ dbe defined
accordingly. Then, we have,
\begin{equation*}
\frac{1}{2\pi i}\int^{c+iT}_{c-iT}\chi(1-s)F_{0}(s)ds=2\sum_{n\leq \frac{T}{2\pi}}a_{n}+O(T^{\frac{1}{2}+\epsilon}).\tag{5.4.12}\label{c5:eq5.4.12}
\end{equation*}
\end{lem}

We now prove

\begin{lem}\label{c5:sec1.4-lem3}
Under the conditions of Lemma \ref{c5:sec1.4-lem2}, we have,
\begin{gather*}
Re\ I_{3}=Re\left(\frac{1}{2\pi
  i}\int^{c+iT}_{c+i}\chi(1-s)F(s)ds\right)=\frac{T}{4\pi}(\log
T)^{2}+O(T\log T),\tag{5.4.13}\label{c5:eq5.4.13}\\
Re\left(\frac{1}{2\pi
  i}\int^{c+iu}_{c+i}\chi(1-s)\zeta'(s)ds\right)=-\frac{u}{2\pi}\log
u+O(u),\tag{5.4.14}\label{c5:eq5.4.14} 
\end{gather*}
and
\begin{equation*}
Re\left(\frac{1}{2\pi i}\int^{c+iu}_{c+i}\chi(1-s)\zeta(s)ds\right)=\frac{u}{2\pi}+O(u^{\frac{1}{2}+\epsilon}),\tag{5.4.15}\label{c5:eq5.4.15}
\end{equation*}
where in the last two assertions $1\leq u\leq T$.
\end{lem}

\begin{proof}
Let us denote the integrand in \eqref{c5:eq5.4.13} by $G(s)$. In
$(|t|\leq 1,\sigma=c)$ we have $G(s)=O(L^{3})$ and so we can include
this in the error term. Next
$$
\frac{1}{2\pi i}\int^{c-i}_{c-iT}G(c+it)ds=\frac{1}{2\pi}\int^{-1}_{-T}G(c+it)dt=\frac{1}{2\pi}\int^{T}_{1}G(c-it)dt
$$
which is the complex conjugate of
$$
\frac{1}{2\pi i}\int^{c+iT}_{c+i}G(s)ds.
$$

Hence 
$$
\frac{1}{2\pi i}\int^{c+iT}_{c-iT}G(s)ds=2\ Re\left(\frac{1}{2\pi
  i}\int^{c+iT}_{c+i}G(s)ds\right)+O((\log T)^{3}).
$$

This proves the first part of the lemma since by prime number theorem
$$
\sum_{n\leq \tau}a_{n}=\frac{T}{4\pi}(\log T)^{2}+O(T\log T).
$$

The\pageoriginale other two parts follow since while moving the line
of integration from $\sigma=c$ to $\sigma=1+(\log(2u))^{-1}$, $u\geq
1$ we have the contribution $O(u^{\frac{1}{2}+\epsilon})$ from the
horizontal sides.

We have to treat $Re\ I_{1}$ and $Re\ I_{2}$. We use
$\chi'\chi^{-1}=-\log \tau+O(t^{-1})$ for $t\geq 1$, $\sigma=c$. Since
the $O$-term contributes a small quantity we may replace
$\chi'\chi^{-1}$ by $-\log \tau$ and $(\chi'\chi^{-1})^{2}$ by
$(\log\tau)^{2}$. Now
\begin{align*}
& Re\ I_{2}=\frac{1}{2\pi}Re\int^{T}_{1,(\sigma=c)}\chi(1-s) \left(-\log
  \frac{t}{2\pi} \right)\zeta'dt\\ 
&= \int^{T}_{1} \left(-\log \frac{u}{2\pi} \right) d\ Re\ K_{2}(u),(\text{where~
  }K_{2}(u)=\frac{1}{2\pi}\int^{u}_{1,(\sigma=c)}\chi(1-s)\zeta'dt)\\
&= -\log
  \frac{u}{2\pi}Re\ K_{2}(u)]^{T}_{1}+O \left(\int^{T}_{1}\frac{1}{u}|Re\ K_{2}(u)|du \right)\\
&= -\frac{T}{2\pi}(\log T)^{2}+O(T\log T).
\end{align*}

Similarly
\begin{align*}
& Re\ I_{1}
=\frac{1}{2\pi}\int^{T}_{1,(\sigma=c)}\chi(1-s) \left(\log\frac{t}{2\pi} \right)^{2}\zeta
dt\\
& = \int^{T}_{1} \left(\log \frac{u}{2\pi} \right)^{2}d\ Re\ K_{1}(u),\ (\text{where~
} K_{1}(u)=\frac{1}{2\pi}\int^{u}_{1,(\sigma=c)}\chi(1-s)\zeta dt)\\
&= \left(\log
\frac{u}{2\pi} \right)^{2}Re\ K_{1}(u)]^{T}_{1}+O\left(\int^{T}_{1}\frac{\log}{u}u\ du\right)\\ 
&= \frac{T}{2\pi}(\log T)^{2}+O(T\log T).
\end{align*}

Since $-\overline{I}_{0}=I_{1}-2I_{2}+I_{3}$ (with an error
$O_{\epsilon}(T^{\frac{1}{2}+\epsilon})$), we have
$Re(-I_{0})= (\frac{1}{4\pi}-\frac{2}{2\pi}+\frac{1}{2\pi})T(\log
T)^{2}+O(T\log T)$ and so
$$
Re\ I_{0}=\frac{T}{4\pi}(\log T)^{2}+O(T\log T).
$$

This proves all that we wanted to prove. 
\end{proof}

\newpage

\begin{center}
{\large\bf Notes at the End of Chapter c5}
\end{center}
\smallskip

All\pageoriginale references except to the book of E.C.\@ Titchmarsh
\cite{Titchmarsh1} (revised by D.R.\@ Heath-Brown) are postponed to the notes
at the end of the chapter.

\S\ \ref{c5:sec5.2}. The proof of $\frac{1}{2}\leq \theta\leq 1$
given here is a slightly simplified version of the (Hansraj Gupta
memorial) lecture given by me at Aligarh during the $57^{\rm th}$ Annual
Conference of the Indian Mathematical Society held during December
1991. The details of this lecture will appear with the title ``A new
approach to the zeros of $\zeta(s)$'' in Mathematics Student (India)
\cite{Balasubramanian and Ramachandra19}. This method itself has been published by myself and R.\@
Balasubramanian in a very much more general (but complicated) form in
two papers $X^{[11]}$ and $XI^{[12]}$ with the same title ``on the
zeros of a class of generalised Dirichlet series''. For the simplest
proof see \cite{Ramachandra26}.

The difficulty of the generalisation mentioned in Remark 2 is the
analogue of the upper bound of $|\zeta(s)-\frac{1}{s-1}|$. For the
results on general number fields the only method, known is by using
the functional equation.

For the result mentioned in Remark 3 due to K.\@ Ramachandra see
\cite{Ramachandra20}.

The Lemma 5 is due to S.\@ Ramanujan \cite{Ramanujan1} (see also paper
number 24 pages 208-209 of his collected papers \cite{Ramanujan2}). Actually
it is enough to prove something like $\pi(X)-\pi(X/2)>X(\log X)^{-2}$
for $X=X_{\nu}(\nu=1,2,3,\ldots)$ such that $X_{\nu}\to \infty$. This
follows from $\prod\limits_{p\leq x}(1-p^{-1})^{-1}\geq
\sum\limits_{n\leq x}n^{-1}$, on taking logarithms on both sides.

Theorem \ref{c5:thm5.2.2} is nearly proved in the papers $I^{[21]}$
and $II^{[22]}$ of the series ``On the zeros of a class of generalised
Dirichlet series''. The papers $III^{[10]}$, $IV^{[13]}$, $V^{[23]}$,
$VI^{[14]}$, $XIV^{[19]}$ and $XV^{[20]}$ of the same series are more
involved and deal with refined developments. All these deal with the
zeros in $(\sigma\geq \frac{1}{2}-\delta,T\leq t\leq 2T)$ where
$\delta(>0)$ is any constant. The papers $VII^{[24]}$, $VIII^{[15]}$,
$IX^{[16]}$, $X^{[11]}$ and $XI^{[12]}$ concentrate on the same
problem with $\delta=\delta(T)\to 0$. In fact XI (as also the papers
``On the zeros of $\zeta'(s)-{a''}^{[17]}$ and ``On the zeros of
$\zeta(s)-{a''}^{[18]}$ to appear) deal with the zeros in $(\sigma\geq
\frac{1}{2}+\delta,T\leq t\leq 2T)\delta(>0)$\pageoriginale being a
constant, and further refinements. Amongst the papers just mentioned
in this paragraph the papers I, II, V and VII are due to K.\@
Ramachandra. The rest are all due to R.\@ Balasubramanian and K.\@
Ramachandra. It must be mentioned that the paper $VII^{[24]}$ is very
general and deals with the zeros of
$\sum\limits^{\infty}_{n=1}a_{n}\lambda^{-s}_{n}$ with
$\frac{1}{x}\sum\limits_{n\leq x}|a_{n}|^{2}\gg \Exp
\left(-\frac{c\log x}{\log\log x}\right)$ for some constant $c>0$ and
all $x\geq 100$, in the rectangle $(\sigma\geq
\frac{1}{2}-\delta,T\leq t\leq 2T)$ with $\delta=c'(\log\log T)^{-1}$
for some constant $c'(>0)$ (and further refinements). This paper
depends on the localisation of some theorems of J.E.\@ Littlewood and
A.\@ Selberg (who dealt with $\zeta(s)$) to very general Dirichlet
series due to K.\@ Ramachandra and A.\@ Sankaranarayanan \cite{Ramachandra and Sankaranarayanan3} \cite{Ramachandra and Sankaranarayanan5}. These results are stated in
\S\ \ref{c5:sec5.3}. Another recent paper $XVI^{[6]}$ by K.\@
Ramachandra and A.\@ Sankaranarayanan adds to our knowledge of the
zeros of a class of generalised Dirichlet series in $(\sigma\geq
\frac{1}{2}+\delta,T\leq t\leq 2T)$, where $\delta(>0)$ is any
constant.

\S\ \ref{c5:sec5.4}. The reference to the papers of three authors
is J.B.\@ Conrey, A.\@ Ghosh and S.M.\@ Gonek, \cite{Conrey Ghosh and Gonek1}. The result
mentioned in Remark 2 below Lemma 1 is proved in S.M. Gonek
\cite{Gonek1}. The latest improvement of \eqref{c5:eq5.4.1} is due to
A.\@ Fujii (see A.\@ Fujii \cite{Fujii1}). It runs as follows: There
exist real constants $A_{1}>0$, $A_{2}$, $A_{3}$ such that the
difference
$$
\sum_{1\leq Im\ \rho\leq T}\zeta'(\rho)-A_{1}T(\log T)^{2}-A_{2}T(\log
T)-A_{3}T
$$
is $O(Te^{-c\sqrt{\log T}})$ where $c>0$ is an absolute constant. He
also proves that if we assume Riemann's hypothesis then the difference
is $O(T^{\frac{1}{2}}\break(\log T)^{\frac{1}{2}})$. 
