
\chapter{Some Preliminaries}\label{c1}

\section{Some Convexity Principles}\label{c1:sec1.1}
Suppose\pageoriginale $f(s)$ is an analytic function of $s = \sigma + it$ defined in the rectangle
$$
R = \left\{a \leq \sigma \leq b, \; t_0 - H \leq t \leq t_0 + H \right\}
$$
where $a$ and $b$ are constants satisfying $a<b$. We assume that $|f(s)| \leq M$ (with $M \geq 2$; sometimes we assume implicitly that $M$ exceeds a large positive constant) throughout $R$. A simple method of obtaining better upper bounds for $|f(\sigma_0 + it_0)|$ with $a < \sigma_0 < b$ is to apply maximum modulus principle to 
$$
f(s_0 + w) e^{w^2} X^w \quad (\text{where } s_0 = \sigma_0 + it_0)
$$
over the rectangle with the sides $\re w = a - \sigma_0$, $\re w = b - \sigma_0$, $\Iim w = \pm H$ and choose $X$ in an optimal way. We may also consider
$$
f(s_0) = \frac{1}{2\pi i} \int f(s_0+ w) e^{w^2} X^w  \frac{dw}{w}
$$
over the anti-clockwise boundary of the same rectangle. Sometimes after doing this we may consider upper bounds for 
$$
\int\limits_{|t_0| \leq \frac{1}{2} H} |f(s_0)| dt_0.
$$
A better kernel in place of $e^{w^2}$ is $\Exp (w^{2n})$ where $n$ is any positive odd integer or still better $\Exp \left(\left(\sin \frac{w}{1000} \right)^2 \right)$, with bigger constants in place of 1000 if necessary. If $\sigma_0$ is close to $a$ or $b$ we get a factor from $\frac{1}{w}$ which is very big. However in some cases we may avoid this big factor by appealing to a two variable convexity theorem of R.M. Gabriel. In all these cases, to get worthwhile results it is necessary to have $H \gg \log \log M$ (with a large implied constant). In this section we present a general theorem (without the requirement $H \gg \log \log M$) which enables one to prove, for instance, things like 
$$
\int_{|v| \leq D} | \zeta \left( \frac{1}{2} + it_0 + iv\right) |^k dv \gg t^{-\epsilon}_0
$$
where\pageoriginale $k$ and $\epsilon$ are any two positive constants and $D$ depends only on $k$ and $\epsilon$. The general theorem also gives
$$
\int_{|v| \leq D} |\zeta(1+it_0 +iv)|^k dv \gg (\log t_0)^{-\epsilon}.
$$
In a later chapter we will show that here the LHS is actually $\gg D$. We do not know how to prove the same for the previous integral unless $D \gg \log \log t_0$ (when we can prove a better bound as we shall see in a later chapter). First we will be interested in obtaining lower bounds for 
\begin{equation*}
I(\sigma)= \int\limits_{|v| \leq H} |f(\sigma + it_0 + iv)|^k dv \tag{1.1.1}\label{c1:eq1.1.1}
\end{equation*}
where $k>0$ is any real constant.

\begin{theorem}\label{c1:thm1.1.1}
Suppose there exists a constant $d$ such that $a < d < b$ and that in $d \leq \sigma \leq b$, $|f(s)|$ is bounded both below and above by $\beta$ and $\beta^{-1}$ where $\beta \leq 1$ is a positive constant (it is enough to assume this condition for $I(\sigma)$ with $H$ replaced by an arbitrary quantity lying between $\frac{1}{2} H$ and $H$ in place of $|f(s)|$). Let $\epsilon > 0$ be any constant. Then for $H=D$ where $D$ is a certain positive constant depending only on $\epsilon, k, a, b, d$ and $\beta$ we have, for $a \leq \sigma \leq d$,
$$
I(\sigma) \gg M^{-\epsilon}.
$$
Next we prove
\end{theorem}

\begin{theorem}\label{c1:thm1.1.2}
Let $A_0$, $\sigma_1$ and $\delta$ be any three constants satisfying $A_0 > 0$, $a <  \sigma_1 < b$, $\delta >0$ and $H = \delta$. Then for $\sigma = \sigma_1 \pm (\log M)^{-1}$ (whatever be the sign), we have,
$$
|f(\sigma_1 + it_0)|^k \ll M^{-A_0} + I(\sigma) \log M
$$
and
$$
\int_{|u|\leq \frac{\delta}{2}} |f(\sigma_1 + it_0 + iu)|^k du \ll M^{-A_0} + I(\sigma) \log \log M.
$$
\end{theorem}

We deduce\pageoriginale Theorems \ref{c1:thm1.1.1} and  \ref{c1:thm1.1.2} by two general theorems on convexity which we now proceed to prove. First of all a remark about the real constant $k>0$. We will (for technical simplicity) assume that $k$ is an integer. To prove the general case we have to proceed as we do here, but we have to use the Riemann mapping theorem (with zero cancelling factors $(\theta (w))^k$ suitably; see Lemmas \ref{c1:lem2a}, \ref{c1:lem3a}, \ref{c1:lem4a} of \S\ \ref{c1:sec1.3}). If $k$ is an integer we can consider $f(s)$ in place of $(f(s))^k$ without loss of generality.

Let $a \leq \sigma_0 < \sigma_1 < \sigma_2 \leq b$, $0 < D \leq H$, $s_1 = \sigma_1 + it_0$ and let $P$ denote the contour $P_1 P_2 P_3 P_4 P_1$ where $P_1 = - (\sigma_1-\sigma_0) - i D$, $P_2 = \sigma_2 - \sigma_1 - iD$, $P_3 = \sigma_2 - \sigma_1 + iD$ and $P_4 = - (\sigma_1 - \sigma_0) + iD$. Let $w = u+iv $ be a complex variable. We have 
\begin{equation*}
2 \pi i f(s_1) = \int\limits_P f(s_i+w) X^{w} \frac{dw}{w} \text{ where } X >0.
\tag{1.1.2}\label{c1:eq1.1.2}
\end{equation*}
We put 
\begin{equation*}
X = \Exp (Y + u_1 + u_2 + \ldots + u_r)\tag{1.1.3}\label{c1:eq1.1.3}
\end{equation*}
where $Y \geq 0$ and $(u_1, u_2, \ldots u_r)$ is any point belonging
to the $r$ - dimensional cube $[0,C] \times [0,C] \times \ldots \times [0,C]$, $C$ being a positive constant to be chosen later. The contour $P$ consists of the two vertical lines $-V_0$ and $V_2$ respectively given by $P_4P_1$ and $P_2 P_3$ and two horizontal lines $Q_1, - Q_2$ respectively given by $P_1P_2$ and $P_3P_4$. Averaging the equation (\ref{c1:eq1.1.2}) over the cube we get
\begin{equation*}
  2 \pi i f(s_1) = C^{-r} \int\limits^C_0 \ldots \int\limits^C_0 \int\limits_P f(s_1+w) \frac{X^w}{w} dw \; du_1 \ldots du_r. \tag{1.1.4}\label{c1:eq1.1.4}
\end{equation*}
Over $V_0$ and $V_2$ we do not do the averaging.  But over $Q_1$ and $Q_2$ we do average and replace the integrand by its absolute value. We obtain
\begin{align*}
& |2 \pi f (s_1)| \leq \Exp (-Y (\sigma_1 - \sigma_0)) \int_{V_0} |f(s_1 + w) \frac{dw}{w}| \\
& + \Exp ((Y + Cr) (\sigma_2 - \sigma_1)) \int_{V_2} |f(s_1+w) \frac{dw}{w}|\\
& + \frac{2^{r+1}}{C^r D^r} \Exp ((Y + Cr) (\sigma_2 - \sigma_1)) (\max\limits_{w \in Q_1 \cup Q_2} |f (s_1 + w)|) (\sigma_2 - \sigma_0)
\end{align*}
and thus\pageoriginale
\begin{align*}
& |2 \pi f(s_r)| \leq (\Exp (-Y (\sigma_1 - \sigma_0))) I_0 \\
& + (\Exp (Cr (\sigma_2 - \sigma_1))) (\Exp (Y (\sigma_2 -\sigma_1))) (I_2 + M^{-A})\\
& + 2M (\sigma_2 - \sigma_0) (\Exp (Y (\sigma_2 - \sigma_1))) \left( \frac{2 \Exp (C (\sigma_2 - \sigma_1))}{CD}\right)^r\tag{1.1.5}\label{c1:eq1.1.5}
\end{align*}
where $A$ is any positive constant and
\begin{equation*}
I_0 = \int\limits_{V_0} |f(s_1 + w) \frac{dw}{w}| \text{ and } I_2 = \int_{V_2} |f(s_1 + w) \frac{dw}{w}|. \tag{1.1.6}\label{c1:eq1.1.6}
\end{equation*}
Choosing $Y$ to equalise the first two terms on the RHS of (\ref{c1:eq1.1.5}), i.e. choose $Y$ by 
$$
\Exp (Y (\sigma_2 - \sigma_0)) = \left(\frac{I_0}{I_2 + M^{-A}} \right) \Exp (-Cr (\sigma_2 - \sigma_1))
$$
i.e.
\begin{align*}
\Exp (Y (\sigma_2 - \sigma_1)) &= \left( \frac{I_0}{I_2 +
  M^{-A}}\right)^{(\sigma_2 - \sigma_1) (\sigma_2 - \sigma_0)^{-1}}\\
&\qquad\qquad\Exp (-Cr (\sigma_2 - \sigma_1)^2 (\sigma_2 - \sigma_0)^{-1})  
\end{align*}
and noting that
$$
(\sigma_2 - \sigma_1) - (\sigma_2 - \sigma_1)^2 (\sigma_2 - \sigma_0)^{-1} = (\sigma_2 - \sigma_1) (\sigma_1 - \sigma_0) (\sigma_2 - \sigma_0)^{-1},
$$
we obtain
\begin{align*}
& |2\pi f(s_1)| \leq 2 \left\{ I^{\sigma_2 - \sigma_1}_0 (I_2 +
  M^{-A})^{\sigma_1 - \sigma_0}\right\}^{(\sigma_2 -
    \sigma_0)^{-1}}\\  
&\qquad\quad\left\{ \Exp \left(\frac{C(\sigma_2 - \sigma_1) (\sigma_1 - \sigma_0)}{\sigma_2 - \sigma_0} \right) \right\}^r\\
&\qquad\quad {} + 2 M (\sigma_2 - \sigma_0) \left(\frac{I_0}{I_2 + M^{-A}}
  \right)^{(\sigma_2 - \sigma_1) (\sigma_2 - \sigma_0)^{-1}}\\ 
&\qquad\quad\left\{\frac{2}{CD} \Exp \left(\frac{C(\sigma_2 - \sigma_1) (\sigma_1
    - \sigma_0)}{\sigma_2 - \sigma_0} \right) \right\}^r. 
\tag{1.1.7} \label{c1:eq1.1.7} 
\end{align*}
Collecting we state the following convexity theorem.

\begin{theorem}\label{c1:thm1.1.3}
Suppose $f(s)$ is an analytic function of $s = \sigma + it$ defined in the rectangle $R : \left\{ a \leq  \sigma \leq b, t_0 - H \leq t \leq t_0 + H\right\}$ where $a$ and $b$ are constants with $a<b$. Let the maximum of $|f(s)|$ taken over $R$ be $\leq M$. Let $a \leq \sigma_0 < \sigma_1 < \sigma_2 \leq b$ and let $A$ be any large positive constant. Let $r$ be any positive\pageoriginale integer, $0 < D \leq H$ and $s_1 = \sigma_1 + it_0$. Then for any positive constant $C$, we have,
\begin{align*}
& |2 \pi f(s_1)| \leq 2 \left\{I_0^{\sigma_2 - \sigma_1} (I_2 +
  M^{-A})^{\sigma_1 - \sigma_0} \right\}^{(\sigma_2 -
    \sigma_0)^{-1}}\\  
&\qquad\quad\left\{ \Exp \left(\frac{C(\sigma_2 - \sigma_1) (\sigma_1 - \sigma_0)}{\sigma_2 - \sigma_0} \right)\right\}^r\\
&\qquad\quad {} + 2 M^{A+2} (\sigma_2 - \sigma_0) \left(2  \left(1+
  \left(\log \left( \frac{D}{\sigma_1 - \sigma_0}\right) \right)^*
  \right)^{(\sigma_2 - \sigma_1) (\sigma_2 - \sigma_0)^{-1}} \times
  \right.\\ 
&\qquad \times  \left\{\frac{2}{CD} \Exp \left( \frac{C(\sigma_2 - \sigma_1) (\sigma_1 - \sigma_0)}{\sigma_2-\sigma_0}\right) \right\}^r\tag{1.1.8}\label{c1:eq1.1.8}
\end{align*}
where 
\begin{equation*}
I_0 = \int_{|v| \leq D} \left| f(\sigma_0 + it_0 + iv) \frac{dv}{\sigma_0 - \sigma_1 + iv}\right| \tag{1.1.9}\label{c1:eq1.1.9}
\end{equation*}
and 
\begin{equation*}
I_2 = \int_{|v| \leq D} \left| f(\sigma_2 + it_0 + iv) \frac{dv}{\sigma_2 - \sigma_1 + iv}\right|, \tag{1.1.10}\label{c1:eq1.1.10}
\end{equation*}
and we have written $(x)^* = \max (0,x)$ for any real number $x$. 
\end{theorem}

\begin{proof}
We have used $I_2 + M^{-A} \geq M^{-A}$ and $(\sigma_2 - \sigma_1) (\sigma_2 - \sigma_0)^{-1} \leq 1$ and if $D \geq \sigma_1 - \sigma_0$,
$$
I_0 \leq M \int_{|v| \leq D} \left| \frac{dv}{\sigma_0 - \sigma_1 + iv}\right| \leq 2 M \left\{\int^{\sigma_1 - \sigma_0}_{0} \frac{dv}{\sigma_1 - \sigma_0} + \int^{D}_{\sigma_1 - \sigma_0} \frac{dv}{v} \right\}.
$$
This completes the proof of Theorem  \ref{c1:thm1.1.3}.
\end{proof}

In (\ref{c1:eq1.1.8}) we replace $t_0$ by $t_0 + \alpha$ and integrate with respect to $\alpha$ in the range $|\alpha| \leq D$, where now $2D \leq H$. LHS in now $J(\sigma_1)$ defined by
\begin{equation*}
J(\sigma_1) = 2\pi \int_{|\alpha| \leq D} |f(\sigma_1 + it_0 + i\alpha )| d \alpha.
\tag{1.1.11}\label{c1:eq1.1.11}
\end{equation*}
Next
\begin{align*}
& \int_{|\alpha| \leq D} \left(I^{\sigma_2 - \sigma_1}_0 (I_2 + M^{-A})^{\sigma_1 - \sigma_0} \right)^{(\sigma_2 - \sigma_0)^{-1}} \\
& \leq \left(\int_{|\alpha| \leq D} I_0 d\alpha \right)^{(\sigma_2-\sigma_1) (\sigma_2 - \sigma_0)^{-1}}  \left(\int_{|\alpha| \leq D} (I_2 + M^{-A})d\alpha \right)^{(\sigma_1 - \sigma_0) (\sigma_2 - \sigma_0)^{-1}}. 
\end{align*}
Now
\begin{align*}
& \int_{|\alpha|\leq D} I_0 d\alpha = \int_{|v| \leq D} \int_{|\alpha| \leq D} \left|f(\sigma_0 + it_0 + i \alpha + iv)  \frac{d\alpha \; dv}{\sigma_0 - \sigma_1 + iv}\right|\\
& \leq \left(\int_{|v| \leq 2D} |f(\sigma_0 + it_0 + iv)|dv \right) \int_{|v|\leq D} \left|\frac{dv}{\sigma_0 - \sigma_1 + iv} \right|\\
& \leq 2 \left(1+ \left(\log \left(\frac{D}{\sigma_1 - \sigma_0} \right) \right)^* \right) I(\sigma_0) \tag{1.1.12} \label{c1:eq1.1.12}
\end{align*}
where\pageoriginale
\begin{equation*}
I(\sigma_0) = \int_{|v| \leq 2D} |f(\sigma_0 + it_0 + iv)| dv. \tag{1.1.13}\label{c1:eq1.1.13}
\end{equation*}
Proceeding similarly, with
\begin{equation*}
I(\sigma_2)  = \int_{|v| \leq 2D} |f(\sigma_0 + it_0+iv)| dv \tag{1.1.14}\label{c1:eq1.1.14}
\end{equation*}
we have 
\begin{equation*}
\int_{|\alpha| \leq D} (I_2 + M^{-A}) d \alpha \leq 2 D M^{-A} + 2 \left(1+ \left(\log \left( \frac{D}{\sigma_2 - \sigma_1}\right) \right)^*\right) I(\sigma_2). \tag{1.1.15}\label{c1:eq1.1.15}
\end{equation*}
Thus we have the following corollary.

\begin{theorem}\label{c1:thm1.1.4}
In addition to the conditions of Theorem  \ref{c1:thm1.1.3}, let $2D \leq H$ and let $J(\sigma_1)$, $I(\sigma_0)$ and $I(\sigma_2)$ be defined by (\ref{c1:eq1.1.11}), (\ref{c1:eq1.1.13}) and (\ref{c1:eq1.1.14}). Then, we have,
\begin{align*}
& 2 \pi J (\sigma_1) \leq 4 \left\{ I(\sigma_0) \left(1+ \left(\log
  \left( \frac{D}{\sigma_1 - \sigma_0}\right) \right)^*
  \right)\right\}^{(\sigma_2 - \sigma_1) (\sigma_2 - \sigma_0)^{-1}}
  \times\\ 
&\quad \times \left\{ I (\sigma_2) \left(1+ \left( \log \left(
  \frac{D}{\sigma_2 - \sigma_1}\right)\right)^* \right) +
  M^{-A}\right\}^{(\sigma_1 - \sigma_0) (\sigma_2 - \sigma_0)^{-1}}
  \times\\ 
&\quad \left\{\Exp \left(\frac{C(\sigma_2 - \sigma_1) (\sigma_1 -
    \sigma_0)}{\sigma_2 - \sigma_0} \right) \right\}^r + 4 M^{A+2}
  (\sigma_2 - \sigma_0)\times\\ 
&\quad\left\{ 1+ \left(\log
  \left(\frac{D}{\sigma_1-\sigma_0} \right)
  \right)^*\right\}^{(\sigma_2 - \sigma_1) (\sigma_2 - \sigma_0)^{-1}}
  \times \\ 
&\quad\times \left\{ \frac{2}{CD} \Exp \left( \frac{C(\sigma_2 -
    \sigma_1)(\sigma_1 - \sigma_0)}{\sigma_2 -
    \sigma_0}\right)\right\}^r. \tag{1.1.16}\label{c1:eq1.1.16} 
\end{align*}


\noindent{\textbf{Proof of Theorem (1.1.1).}} In Theorem  \ref{c1:thm1.1.4} replace $D$ by $D/2$ and assume that $J(\sigma_1)$ is bounded below (by $\frac{1}{2} \beta D$) and $I(\sigma_2)$ is bounded above by $\beta^{-1} D$ (these conditions are implied by the conditions of Theorem  \ref{c1:thm1.1.1}). Put $C=1$, $r = [\epsilon \log M] + 1$ and $D = \Exp (\epsilon^{-1}E)$ where $E$ is a large constant. Let $\sigma_0, \sigma_1$ and $\sigma_2$ be constants satisfying $a\leq \sigma = \sigma_0 < \sigma_1 < \sigma_2 \leq b$. We see that the second term on the RHS of (\ref{c1:eq1.1.16}) is $\leq M^{-A}$ so that 
\begin{align*}
& \frac{1}{2} \beta D \leq J (\sigma_1) \leq 4 \left\{ I(\sigma_0)
  \left(1+ \left(\log \left(\frac{D}{\sigma_1 - \sigma_0} \right)
  \right)^* \right)\right\}^{(\sigma_2 - \sigma_1) (\sigma_2 -
    \sigma_0)^{-1}} \times \\ 
&\quad\times  \left\{\beta^{-1} D \left(1+ \left(\log \left(
  \frac{D}{\sigma_2 - \sigma_1}\right) \right)^{*} \right)
  \right\}^{(\sigma_1 -\sigma_0) (\sigma_2-\sigma_0)^{-1}}\times\\ 
&\quad\times\left\{\Exp\left( \frac{(\sigma_2 - \sigma_1) (\sigma_1 -
    \sigma_0)}{\sigma_2-\sigma_0} \right) \right\}.
  \tag{1.1.17}\label{c1:eq1.1.17} 
\end{align*}
This proves Theorem (\ref{c1:thm1.1.1}).
\end{theorem}

\medskip
\noindent{\textbf{Proof of Theorem (1.1.2).}} In theorems\pageoriginale  \ref{c1:thm1.1.3} and  \ref{c1:thm1.1.4} choose $D = \delta$ ($\delta$ any positive constant), $\sigma_2 - \sigma_1 = (\log M)^{-1}$, $\sigma_2 = \sigma$, $\sigma_0 = a$, $r = [\log M]$, $C =$ a large constant  times $\delta^{-1}$. We obtain the first part of Theorem  \ref{c1:thm1.1.2} namely the $+$ sign. To obtain the second part we argue as in the proof of Theorems (\ref{c1:thm1.1.3}) and (\ref{c1:thm1.1.4}) but now with
$$
\frac{1}{2\pi i} \int f(s_1 + w) X^{-w} \frac{dw}{w}
$$
along the same contour $P$ with the same $X$ as before (note $X^{-w}$ in the present integrand). The rest of the details are similar.


\section{A Lemma in Complex Function Theory}\label{c1:sec1.2}
In this section we prove

\begin{theorem}\label{c1:thm1.2.1}
Let $n$ be any positive integer, $B > 0$ arbitrary, $r > 0$ arbitrary. Let $f(z)$ be analytic in $|z| \leq r$ and let the maximum of $|f(z)|$ in this disc be $\leq M$. Let $0 \leq x < r$, $C= Bnx$, $r_0 = \sqrt{r^2-x^2}$, and $\alpha = 2(C^{-1} \sinh C+ \cosh C)$. Then (for any fixed combination of signs $\pm$) we have
\begin{align*}
 |f(0)| &  \leq \left(\frac{\alpha B n r_0}{\pi}  \right) \left(\frac{1}{2r_0} \int^{r_0}_{-r_0} |f(\pm x + iy)| dy \right)\\
& + \left(\frac{2}{Br} \right)^n \left\{ 1+ (e^C+1) (\pi^{-1} \sin^{-1} \left(\frac{x}{r} \right))\right\} M. \tag{1.2.1}\label{c1:eq1.2.1}
\end{align*}
Putting $x = 0$ we obtain the following
\end{theorem}

\begin{coro*}
We have 
\begin{equation*}
|f(0)| \leq \left( \frac{4Bnr}{\pi}\right) \left(\frac{1}{2r} \int^r_{-r} |f(iy)| dy \right) + \left(\frac{2}{Br} \right)^n M. \tag{1.2.2}\label{c1:eq1.2.2}
\end{equation*}
In particular with $B = 6 r^{-1}$, $M \geq 3$, $A \geq 1$, and $n =$ the integer part of $(A+1) \log M+1$, we have
\begin{equation*}
|f(0)| \leq \frac{24}{\pi} ((A+1) \log M+1) \left( \frac{1}{2r} \int^r_{-r} |f(iy)| dy\right) + M^{-A} . \tag{1.2.3}\label{c1:eq1.2.3}
\end{equation*}
\end{coro*}

\setcounter{remark}{0}
\begin{remark}\label{c1:rem1}
The equations\pageoriginale (\ref{c1:eq1.2.1}), (\ref{c1:eq1.2.2}) and (\ref{c1:eq1.2.3}) are statements about $|f(z)|$. It is possible to extend (\ref{c1:eq1.2.1}) and hence (\ref{c1:eq1.2.2}) and (\ref{c1:eq1.2.3}) to more general functions than $|f(z)|$ with some other constants in place of $\alpha$, $\frac{4}{\pi}$, $\frac{24}{\pi}$. (For example to $|f(z)|^k$ where $k>0$ is any real number). 
\end{remark}

\begin{remark}\label{c1:rem2}
The Corollary shows that
\begin{equation*}
|f(0)| \leq (24 A \log M) \left(\frac{1}{2r} \int^r_{-r} |f(iy)| dy \right) +M^{-A}. \tag{1.2.4}\label{c1:eq1.2.4}
\end{equation*}
We ask the question ``Can we replace $\log M$ by a term of smaller order say by $\sqrt{\log M}$ (or omit it altogether) at the cost of increasing the constant $24A$?''. The answer is no. See the Remark \ref{c1:rem6b} in \S\ \ref{c1:sec1.7}.
\end{remark}

\begin{remark}\label{c1:rem3}
The method of proof is nearly explained in \S\ \ref{c1:sec1.1}. As for the applications we can state for example the following result. Let $3 \leq H \leq T$. Divide the interval $T$, $T+H$ into intervals $I$ of length $r$ each. We can assume $0 < r \leq 1$ and omit a small bit at one of the ends. Then for any integer constant $k \geq 1$, (the result is also true if $k$ is real by Theorem  \ref{c1:thm1.2.2}), we have
\begin{equation*}
\sum\limits_I \max_{i \in I} |\zeta \left(\frac{1}{2} + it \right)|^k \ll \frac{Ak \log T}{r} \int\limits^{T+H+r}_{T-r} \left|\zeta\left(\frac{1}{2} +it \right) \right|^k dt + r^{-1} HT^{-Ak} \tag{1.2.5}\label{c1:eq1.2.5}
\end{equation*}
the implied constant being absolute. We may retain only one term on the LHS of (\ref{c1:eq1.2.5}) and if we know for example that RHS of (\ref{c1:eq1.2.5}) is $\ll HT^\epsilon$ then it would follow that
\begin{equation*}
\mu \left(\frac{1}{2} \right) \leq \frac{1}{k} \lim\limits_{T \to \infty} \left( \frac{\log H}{\log T}\right). \tag{1.2.6}\label{c1:eq1.2.6}
\end{equation*}
Since what we want holds for $H = T^{1/3}$ and $k = 2$ and any $r(0< r \leq 1)$ we obtain the known result $\mu(\frac{1}{2}) \leq \frac{1}{6}$ due to H. Weyl, G.H. Hardy and J.E. Littlewood. Similar remarks apply to $L$-functions and so on.
\end{remark}

\begin{remark}\label{c1:rem4}
The results of this section as well as some of the results section \S\ \ref{c1:sec1.1}\pageoriginale are improvements and generalizations of some lemmas in Ivi\'c's book (see page 172 of this book. Here the results concern Dirichlet series with a functional equation and are of a special nature).
\end{remark}

\begin{remark}\label{c1:rem5}
In (\ref{c1:eq1.2.4}) we have corresponding results with $|f(iy)|$ on the RHS replaced by $|f(x + iy)|$. These follow from Theorem \ref{c1:thm1.2.1}.
\end{remark}

\medskip
\noindent{\textbf{Proof of the Theorem 1.2.1.}} Let $P,Q,R,S$ denote the points $-ri, ri, r$ and $-r$ respectively. Then we begin with

\setcounter{lem}{0}
\begin{lem}\label{c1:lem1}
Let $X = \Exp (u_1 + \ldots + u_n)$ where $B > 0$ is arbitrary and $0 \leq u_j \leq B$ for $j = 1,2, \ldots , n$. Then
\begin{align*}
f(0) &= \frac{1}{2\pi i} \left\{ \int_{PQ} f(w) \frac{X^w - X^{-w}}{w}
dw\right.\\ 
&\left.\qquad\quad + \int_{QSP} f(w) \frac{X^w}{w} dw + \int_{PRQ} f(w) \frac{X^{-w}}{w} dw \right\} 
\tag{1.2.7}\label{c1:eq1.2.7}
\end{align*}
where the integrations are respectively along the straight line $PQ$, along the semi-circular portion $QSP$ of the circle $|w| =r$, and along the semi-circular portion $PRQ$ of the circle $|w| =r$.
\end{lem}

\begin{proof}
With an understanding of the paths of integration similar to the ones explained in the statement of the lemma we have by Cauchy's theorem that the integral of $f(w) \frac{X^w}{w}$ over $PQSP$ is $2\pi i f(0)$ provided we deform the contour to $P'Q'SP'$ where $P'Q'$ is parallel to $PQ$ and is close to $PQ$ (and to the right of it) and the points $P'$ and $Q'$ lie on the circle $|w| =r$. Also with the same modification the integral of $f(w) \frac{X^{-w}}{w}$ over $PRQP$ is zero. These remarks complete the proof of the lemma.
\end{proof}

\begin{lem}\label{c1:lem2}
Denote by $I_1, I_2, I_3$ the integrals appearing in Lemma \ref{c1:lem1}. Let $\langle du \rangle$ denote the element of volume $du_1 du_2 \ldots du_n$ of the box $\mathcal{B}$ defined by $0 \leq u_j \leq B (i = 1, 2, \ldots n)$. Then
\begin{equation*}
B^{-n} \left| \int_\mathcal{B} \left( \frac{1}{2 \pi i} (I_2 + I_3)\right) \langle du \rangle \right| \leq \left(\frac{2}{Br} \right)^n  M. \tag{1.2.8}\label{c1:eq1.2.8}
\end{equation*}
\end{lem}

\begin{proof}
Trivial since $|X^w| \leq 1$ and $|X^{-w}| \leq 1$ on $QSP$ and $PRQ$ respectively.
\end{proof}

\begin{lem}\label{c1:lem3}
Let\pageoriginale $w = x+ iy$ where $x$ and $y$ are any real numbers, and $0 \leq L = \log X \leq Bn$. Then
\begin{equation*}
\left|\frac{e^{wL} - e^{-wL}}{wL} \right| \leq \frac{1}{C} (e^C - e^{-C}) + e^C + e^{-C} 
\tag{1.2.9}\label{c1:eq1.2.9}
\end{equation*}
where $C = Bn |x|$.
\end{lem}

\begin{proof}
LHS in the lemma  is
\begin{align*}
&\left|\left\{ e^{xL} (\cos (yL) + i \sin(yL)) - e^{-xL} (\cos (yL) - i \sin (yL))\right\} (wL)^{-1}\right| \\
&\quad\leq \left| \frac{e^{xL} - e^{-xL}}{xL} \right| \; \Big| \cos (yL)
  \Big| + (e^{xL} + e^{-xL}) \left| \frac{\sin (yL)}{yL}\right|\\ 
&\quad\leq \frac{1}{C} (e^C - e^{-C}) + e^{C} + e^{-C}.
\end{align*}
This completes the proof of the lemma.
\end{proof}

In order to obtain the theorem we note that on the line $PQ$ we have $x = 0$. We now assume that $x >0$ and more the line of integration to $\re $ $w = \pm x$, (whatever be the sign) namely the intercept made by this line in the disc $|w| \leq r$. On this line we pass to the absolute value and use Lemma \ref{c1:lem3}. We get the first term on the RHS of (\ref{c1:eq1.2.1}). For the two circular portions connecting this path with the straight lien $PQ$ we integrate over the box $\mathcal{B}$  and get
\begin{align*}
B^{-n} & \left| \int_\mathcal{B} \left( \frac{1}{2\pi i} \int f (w) \frac{X^w - X^{-w}}{w} dw\right) \langle du \rangle\right|\\
& \leq \left(\frac{2}{Br} \right)^n (e^C + 1) (\pi^{-1} \sin^{-1} \left(\frac{x}{r} \right))M.
\end{align*}
This with lemma \ref{c1:lem2} completes the proof of the theorem \ref{c1:thm1.2.1}.

The result referred to in Remark \ref{c1:rem1} is as follows.

\begin{theorem}\label{c1:thm1.2.2}
Let $k$ be any positive real number. Let $f(z)$ be analytic in $|z| \leq 2r$ and there $|f(z)|^k \leq M(M \geq 9)$. Let $x =r(\log M)^{-1}$, and let $x_1$ be any real number with $|x_1| \leq x$. Put $r_0 = \sqrt{4r^2 - x^2_1}$. Then with $A \geq 1$ we have 
\begin{equation*}
|f(0)|^k \leq 2e^{84A} M^{-A} + \frac{24}{(2\pi)^2} e^{84A} \log M\left( \frac{1}{2r_0} \int\limits^{r_0}_{-r_0} |f(x_1 + iy)|^k dy\right). \tag{1.2.10}\label{c1:eq1.2.10}
\end{equation*}
\end{theorem}

\setcounter{remark}{0}
\begin{remark}\label{c1:rem1a}
It is\pageoriginale easy to remember a somewhat crude result namely
\begin{equation*}
|f(0)|^k \leq e^{90 A} \left\{ M^{-A} + (\log M) \left(\frac{1}{2r_0} \int^{r_0}_{-r_0} |f(x_1+iy)|^k dy\right)\right\}.  \tag{1.2.11}\label{c1:eq1.2.11}
\end{equation*}
\end{remark}

\begin{remark}\label{c1:rem2a}
In Theorem (\ref{c1:thm1.2.1}) the constants are reasonably small where as in Theorem (\ref{c1:thm1.2.2}) they are big. We have not attempted to get optimal constants.
\end{remark}

For these results see the last section of this chapter.

\section{Gabriel's Convexity Theorem}\label{c1:sec1.3}

In this section I reproduce without any essential charges, the proof of the following important theorem due to R.M. Gabriel \cite{Gabriel1}.

\begin{theorem}\label{c1:thm1.3.1}
Let $D$ be a simply connected domain symmetrical about a straight line $L$ lying in $D$. Let the boundary of $D$ be a simple curve $K = K_1 + K_2$ where $K_1$ and $K_2$ lie on opposite sides of $L$. If $f(z)$ is regular in $D$ and continuous on $K$, then
$$
\left( \int_L |f(z)|^{\frac{2}{a+b}} |dz|\right)^{\frac{a+b}{2}} \leq \leq \left( \int_{K_1} |f(z)^{\frac{1}{a}}| dz| \right)^{\frac{a}{2}} \left( \int_{K_2} |f(z)^{\frac{1}{b}}|dz|\right)^{\frac{b}{2}}
$$
where $a>0$ and $b>0$ are any two real numbers.
\end{theorem}

Putting $a = b= \frac{1}{q}$, we get as a special case

\begin{theorem}\label{c1:thm1.3.2}
Let $q > 0$ be any real number. Then in the notation of Theorem \ref{c1:thm1.3.1}, we have
$$
\int_L |f(z)|^q|dz| \leq \left( \int_{K_1} |f(z)|^q|dz|\right)^{\frac{1}{2}} \left( \int_{K_2} |f(z)|^q |dz| \right)^{\frac{1}{2}}
$$
\end{theorem}



\begin{remark*}
The assertion of the theorem still holds if $|f(z)|^q$ is replaced by $|\varphi (z) | \; | f(z)|^q$, where $\varphi (z)$ is any function analytic inside $D$ such that $|\varphi(z)|$ is continuous on the boundary of $D$. To see this replace $f(z)$ by $(f(z))^j (\varphi(z))^r$ and $q$ by $q j^{-1}$ where $j$ and $r$ are positive integers and $j$ and $r$ tend to\pageoriginale infinity in such a way that $rj^{-1} \to q^{-1}$.
\end{remark*}

\setcounter{lem}{0}
\begin{lem}\label{c1:lem1a}
Theorem \ref{c1:thm1.3.1} is true if $f(z)$ has no zeros in $D$.
\end{lem}

\begin{proof}
Without loss of generality we may take $L$ to be a portion of the real axis cutting $K$ in $A,B$. Let $\phi (z)$ satisfy the conditions of the Lemma. Now if $\bar{\phi}(z)$ is the conjugate of $\phi(\bar{z})$, where $\bar{z}$ is the conjugate of $z$, then by a known theorem, $\bar{\phi}(z)$ is regular in $D$ and continuous on $K$. Further, for a $z$ on $L$, $|\phi (z)|^2 = \phi(z) \bar{\phi}(z)$. Hence, by Cauchy's theorem,
\begin{align*}
\int_L |\phi(z)|^2 dz & = \left|\int_{AB} \phi(z) \bar{\phi}(z) dz\right|\\
& = |\int_{K_1} \phi (z) \bar{\phi} (z) dz| \leq \int_{K_1} |\phi(z)\mid \bar{\phi}(z) \mid dz|\\
& \leq \left( \int_{K_1} |\phi(z)|^p|dz|\right)^{1/p} \left( \int_{K_1} |\bar{\phi}(z)|^{p'} |dz|\right)^{1/p'}, \left(\frac{1}{p} + \frac{1}{p'} =1,  \right)\\
& = \left(\int_{K_1} |\phi(z)|^p |dz|\right)^{\frac{1}{p}} \left(\int_{K_2} |\phi(z)|^{p'}|dz|\right)^{1/p'}
\end{align*}
since $K_2$ is the conjugate of $K_1$ with respect to the real axis. Next, if the $f(z)$ of the Theorem \ref{c1:thm1.3.1} has no zero in $D$, $\phi(z) = f^{1/(a+b)} (z)$ is regular in $D$ and continuous on $K$. Hence, taking $p = (a+b) / a$, $p' = (a+b)/b$, we have 
$$
\left(\int_L |f(z)|^{\frac{2}{a+b}} |dz| \right)^{\frac{a+b}{2}} \leq \left(\int_{K_1} |f(z)|^{\frac{1}{a}} |dz| \right)^{\frac{a}{2}} \left( \int_{K_2} |f(z)^{\frac{1}{b}}| dz|\right)^{\frac{b}{2}}
$$
This proves Lemma \ref{c1:lem1a} completely.
\end{proof}

\begin{lem}\label{c1:lem2a}
The domain $D$ of the $z$-plane can be transformed conformally onto $|w| < 1$ by the transformation $z = A(w)$ which possesses a unique inverse analytic transformation $w = A^{-1}(z)$. Further the boundary is transformed continuously onto the boundary.
\end{lem}

\begin{proof}
Follows by a well-known fundamental theorem of Riemann. For the proof of this theorem and references to the work of Riemann see for instance Titchmarsh's book \cite{Titchmarsh2} (1952) or L. Ahlfor's book \cite{Ahlfors1} (see Theorems 10 and 11 on pages 172 and 174).
\end{proof}

\begin{lem}\label{c1:lem3a}
Let $0 < \delta < 1$. Let $F(w) = f(A(w))$. Let $w = 0$ be a zero of order\pageoriginale $m(m \geq 0)$ of $F(w)$. Denote the other zeros (counted with multiplicity) of $F(w)$ in $|w| \leq 1 - \delta$, by $\{\rho\}$. Let the number of zeros (other than $w = 0$) in $|w| \leq 1-\delta$, be $n$. (We will let $\delta\to 0$ finally). Put
$$
\theta(w) = \frac{F(w)}{\psi(w)}, \text{ and } \psi (w) = \frac{w^m \prod \left(1-\frac{w}{\rho} \right)}{(1-\delta)^{m-n} w^n \prod \left(1-\frac{(1-\delta)^2}{w\bar{\rho}}\right)}
$$
Then we have
\begin{itemize}
\item[{\rm (1)}] $\theta(w)$ has no zeros in $|w| \leq 1-\delta$, 

\item[{\rm (2)}] $|\psi(w)| =1$ on $w = 1 -\delta$,

\item[{\rm (3)}] $|\theta(w)| = |F(w)|$ on $w = 1 -\delta$,

and

\item[{\rm (4)}] $|\psi(w)| \leq 1$ in $|w| \leq 1 -\delta$.
\end{itemize}
\end{lem}

\begin{proof}
The statements (1), (2), (3) are obvious and (4) follows from (2) by maximum modulus  principle since $\psi(w)$ is analytic in $|w| \leq 1 -\delta$. 
\end{proof}

\begin{lem}\label{c1:lem4a}
The inverse image of $|w| =1-\delta$ together with the inverse image of $L'$ (the image of $L$-contained in $|w| \leq 1 -\delta$) under the transformation $z = A(w)$ approaches $K$ continuously as $\delta \to 0$.
\end{lem}

\begin{remark*}
For references to earlier versions of Lemmas \ref{c1:lem3a} and \ref{c1:lem4a} see the paper of GABRIEL cited above.
\end{remark*}

\begin{proof}
Follows from Lemma \ref{c1:lem2a}.


Lemmas \ref{c1:lem1a} to \ref{c1:lem4a} complete the proof of Theorem \ref{c1:thm1.3.1}. 

As before let $z = x + iy$ be a complex variable. We employ $a$ in a  meaning different from the one in Theorem \ref{c1:thm1.3.1}. We now slightly extend  this as follows. Consider the rectangle $0 \leq x \leq (2^n+1)a$ (where $n$ is a non-negative integer and $a$ is a positive number), and $0 \leq y \leq R$. Suppose that $f(z)$ is analytic inside the rectangle $\{0 \leq x \leq (2^n +1)a, 0 \leq y \leq R\}$ and that\pageoriginale
$|f(z)|$ is continuous on its boundary. Let $I_x$ denote the integral $\int^R_0|f(z)|^q dy$ where as before $z = x + iy$. Let $Q_a$ denote the maximum of $|f(z)|^q$ on $\{0 \leq x \leq \alpha , y = 0,R\}$. Then we have as a first application of the theorem of Gabriel,
$$
I_\alpha \leq \left(I_0 + 4 a Q_{2a} \right)^{\frac{1}{2}} \left(I_{2a} + 4a Q_{2a}  \right)^{\frac{1}{2}}.
$$
We prove by induction that if $b_m = 2^m +1$, then
\begin{align*}
I_a &\leq  \left(I_0 + 2^{2(m+1)} a Q _{ab_m} \right)^{\frac{1}{2}}
\left(I_a + 2^{2(m+1)} a Q_{ab_m} \right)^{\frac{1}{2} -
  \frac{1}{2^{m+1}}}\\ 
&\qquad\qquad \left( I_{ab_m} + 2^{2(m+1)} a
Q_{ab_m}\right)^{\frac{1}{2^{m+1}}}. 
\end{align*}
We have as a first application of Gabriel's theorem this result with $m=0$. Assuming this to be true for $m$ we prove it with $m$ replaced by $m+1$. We apply Gabriel's Theorem to give the bound for $I_{ab_m}$ in terms of $I_a$ and $I_{ab_{m+1}}$. We have
$$
I_{ab_m} \leq \left(I_a + 2b_{m+a} a Q_{ab_{m+1}} \right)^{\frac{1}{2}} \left(I_{ab_{m+1}} + 2ab_{m+1} Q_{ab_{m+1}}\right)^{\frac{1}{2}}
$$
since as we can easily check $b_{m+1} = b_m+b_m-1$. We add $2^{2(m+1)} a Q_{ab_m}$ to both sides and use that for $A >0$, $B > 0$,$Q > 0$ we have
$$
\sqrt{AB} + Q \leq \sqrt{(A+Q) (B+Q)}
$$
which on squaring both sides reduces to a consequence of $(\sqrt{A} - \sqrt{B})^2 \geq 0$. Thus
\begin{align*}
& I_{ab_m} + 2^{2(m+1)} a Q_{ab_m} \leq \\
& \left(I_a + a (2b_{m+1} + 2^{2(m+1)}) Q_{ab_{m+1}} \right)^{\frac{1}{2}} \left(I_{ab_{m+1}} + a \left(2b_{m+1} + 2^{2(m+1)}  \right) Q_{ab_{m+1}}\right)^{\frac{1}{2}}
\end{align*}
Now $2b_{m+1} + 2^{2(m+1)} \leq 2^{2(m+2)}$ i.e. $2(2^{m+1} +1) \leq 3 \cdot 2^{2(m+1)}$ which is true. Since $\frac{1}{2} - \frac{1}{2^{m+1}} + \frac{1}{2^{m+2}} = \frac{1}{2} - \frac{1}{2^{m+2}}$ the induction is complete and the required result is proved. We state it as a 
\end{proof}

\medskip
\noindent{\textbf{Convexity Theorem 1.3.3.}} For $m = 0,1,2,\ldots, n$\pageoriginale we have
\begin{gather*}
I_a \leq (I_0 + 2^{2(m+1)} aQ_{ab_m})^{\frac{1}{2}} (I_a+2^{2(m+1)} a Q_{ab_m})^{\frac{1}{2} - \frac{1}{2^{m+1}}} \\
\times (I_{ab_m} + 2^{2(m+1)} a Q_{ab_m})^{\frac{1}{2^{m+1}}}. 
\end{gather*}

\begin{remark*}
The remark below Theorem \ref{c1:thm1.3.2} is applicable here also.
\end{remark*}

\section{A Theorem of Montgomery and Vaughan}\label{c1:sec1.4}

\begin{theorem}\label{c1:thm1.4.1}
(Montromery and Vaughan) Suppose $R \geq 2$; $\lambda_1, \lambda_2,
  \ldots,\break\lambda_R$ are distinct real numbers and that $\delta_n =
  \min\limits_{m \neq n} |\lambda_n - \lambda_m|$. Then if $a_1 ,
  a_2,\ldots, a_R$ are complex numbers, we have 
\begin{equation*}
\left| \sum \sum\limits_{m \neq n} \frac{a_m \bar{a}_n}{\lambda_m - \lambda_n} \right| \leq \frac{3\pi}{2} \sum\limits_n |a_n|^2 \delta^{-1}_n. \tag{1.4.1}\label{c1:eq1.4.1}
\end{equation*}
\end{theorem}

\begin{remark*}
We can add any positive constant to each of the $\lambda_n$ and so we can assume that all the $\lambda_n$ are positive and distinct. The proof of the theorem is very deep and it is desirable to have a simple proof within the reach of simple calculus. For a reference to the paper of H.L. Montgomery and R.C. Vaughan see E.C. Titchmarsh \cite{Titchmarsh1}.
\end{remark*}

In almost all applications it suffices to restrict to the special case $\lambda_n = \log (n+\alpha)$ where $0 \leq \alpha \leq 1$ is fixed and $n = 1,2, \ldots, R$. Also the constant $3\pi /2$ is not important in many applications. It is the object of this section to supply a very simple proof in this special case with a larger constant in place of $3\pi /2$. Accordingly our main result is

\begin{theorem}\label{c1:thm1.4.2}
Suppose $R \geq 2$, $\lambda_n = \log (n+\alpha)$ where $0 \leq \alpha \leq 1$ is fixed and $n = 1,2, \ldots, R$. Let $a_1, \ldots a_R$ be complex numbers. Then, we have,
\begin{equation*}
\left| \sum\sum\limits_{m\neq n} \frac{a_m\bar{a}_n}{\lambda_m - \lambda_m}\right| \leq C \sum n |a_n|^2, \tag{1.4.2}\label{c1:eq1.4.2}
\end{equation*}
where $C$ is an absolute numerical constant which is effective.
\end{theorem}

\setcounter{remark}{0}
\begin{remark}\label{c1:rem1b}
Instead of the condition $\lambda_n = \log (n+\alpha)$ we can also work with the\pageoriginale weaker condition $n(\lambda_{n+1} - \lambda_n)$ is both $\gg 1$ and $\ll 1$. Also no attempt is made to obtain an economical value for the constants such as $C$.
\end{remark}

\begin{remark}\label{c1:rem2b}
Theorem \ref{c1:thm1.4.2} with $\alpha =0$ and the functional equation of $\zeta(s)$ are together enough to deduce in a simple way the result that for $T \geq 2$,
\begin{equation*}
\int^T_0 \left|\zeta \left(\frac{1}{2} + it \right) \right|^4 dt = \left(\frac{1}{2\pi^2} \right) T (\log T)^4 + O(T(\log T)^3). \tag{1.4.3}\label{c1:eq1.4.3}
\end{equation*}
The result (\ref{c1:eq1.4.3}) was first proved by A.E. Ingham by a very complicated method.

We prove the following result.
\end{remark}

\begin{theorem}\label{c1:thm1.4.3}
If $\{a_n\}$ and $\{b_n\} (n=1,2,3,\ldots R)$ are complex numbers where $R \geq 2$ and $\lambda_n = \log (n+\alpha)$ where $0 \leq \alpha \leq 1$ is fixed, then
$$
\left| \sum\sum\limits_{m \neq n} \frac{a_m\bar{b}_n}{\lambda_m - \lambda_n}\right| \leq D \left( \sum n |a_n|^2\right)^{1/2} \left( \sum n |b_n|^2\right)^{1/2}
$$
where $D$ is an effective positive numerical constant.
\end{theorem}

We begin with 

\setcounter{lem}{0}
\begin{lem}\label{c1:lem1c}
We have
$$
\left| \sum \sum\limits_{m \neq n} \frac{a_m \bar{a}_m}{m -n}\right| \leq  \pi \sum |a_n|^2. 
$$
\end{lem}

\begin{proof}
We remark that $2\pi \int^1_0 \left(\int^y_0 | \sum a_m e^{2\pi i mx}
|^2 dx \right) dy = \pi \sum |a_n|^2 - E/i$, where $E/i$  is the real
number for which $|E| \leq \pi \sum |a_n|^2$ is to be proved. Note
that since the integrand is nonnegative, $2\pi \int^1_0 (\int^y_{-y} |
\sum a_n e^{2\pi i nx}|^2 dx)\break dy = 2\pi \sum |a_n|^2$ is an upper
bound. Thus 
$$
0 \leq \pi \sum |a_n|^2 - \frac{E}{i} \leq 2 \pi \sum |a_n|^2
$$
and this proves the lemma.
\end{proof}

\begin{lem}\label{c1:lem2c}
We have
$$
\left|\sum  \sum\limits_{m \neq n} \frac{a_m \bar{b}_n}{m -n}\right| \leq 3\pi \left( \sum |a_n|^2 \right)^{1/2} \left( \sum |b_n|^2\right)^{1/2}
$$
\end{lem}

\begin{proof}
Now\pageoriginale $2\pi \int^1_0 (\int^y_0 (\sum a_m e^{2\pi mx}) (\sum \bar{b}_{n} e^{-2\pi i nx})dx) dy = \pi \sum (a_m \bar{b}_m) - E/i$ gives the result since by Holder's inequality
\begin{align*}
\left|\pi \sum (a_m \bar{b}_m) - \frac{E}{i} \right| & \leq 2 \pi \left( \int^1_0 \left(\int^y_0 | \sum a_m e^{2\pi i mx} |^2 dx \right) dy\right)^{1/2}\\
& \quad \times \left(\int^1_0 \left(\int^y_0 |\sum \bar{b}_n e^{2\pi i nx}|^2 dx \right) dy \right)^{1/2}\\
& \leq 2 \pi \left(\sum |a_n|^2 \right)^{1/2} \left(\sum |b_n|^2 \right)^{1/2}
\end{align*}
on using Lemma \ref{c1:lem1c}. This completes the proof of Lemma \ref{c1:lem2c}.
\end{proof}

We next deduce Theorem \ref{c1:thm1.4.3} from Lemma \ref{c1:lem2c} as follows. We divide the range $1 \leq n \leq R$ by introducing intervals $I_i = [2^{i-1}, 2^i)$ and the pairs $(m,n)$ with $m \neq n$ into those lying in $I_i \times I_j$. We now start with
\begin{gather*}
\int^1_0 \left(\int^y_0 \left(\sum a_m e^{2\pi i \lambda_m x} \right) \left(\sum \bar{b}_n e^{2\pi i \lambda_n x} \right) dx \right) dy\\
= \frac{1}{2} \sum a_m \bar{b}_m - \frac{E}{2 \pi i} + \sum\limits_{\substack{k, \ell\\k \geq 1, \ell \geq 1}} \frac{1}{2\pi i} \sum\limits_{(m,n) \in I_k \times I_{\ell}} \int^1_0 \frac{a_m \bar{b}_n e^{2\pi i (\lambda_m - \lambda_n) y} }{\lambda_m - \lambda_n}dy
\end{gather*}
where $E$ is the quantity for which we seek an upper bound, and hence we have the fundamental inequality
\begin{align*}
\left|\frac{E}{2\pi i} \right| & \leq   \frac{1}{2} \sum |a_m \bar{b}_m| + \frac{1}{2\pi } \sum\limits_{k,\ell} \left| \sum\limits_{(m,n) \in I_k \times I_\ell} \int^1_0 \frac{a_m \bar{b}_n e^{2\pi i (\lambda_m - \lambda_n) y}}{\lambda_m - \lambda_n} dy \right|\\
& \quad + \left( \int^1_0 \int^y_{-y} |\sum a_m e^{2\pi i \lambda_m x}|^2 dx \; dy\right)^{1/2}\\
& \quad \times \left( \int^1_0 \int^y_{-y} |\sum \bar{b}_n e^{-2\pi i \lambda_m x}|^2 dx \; dy\right)^{1/2}\\
& = \sum_1 + \sum_2 + \left(\sum_3 \right)^{1/2} \left(\sum_4 \right)^{1/2}, \text{ in an obvious notation}.
\end{align*}
We remark that if $|k-l| \geq 3$ then 
\begin{align*}
&\left| \sum\limits_{(m,n) \in I_k \times I_\ell} \int^1_0 \frac{a_m \bar{b}_n e^{2 \pi i (\lambda_m - \lambda_n) y}}{\lambda_m -\lambda_n} dy \right|\\
& \qquad \ll \left(\sum\limits_{(m,n) \in I_k \times I_{\ell}} |a_m \bar{b}_n| \right) \;  \text{maximum}_{(m,n) \in I_k \times I_{\ell} }(\lambda_m - \lambda_n)^{-2}\\
& (k-l)^{-2} \sum\limits_{(m,n) \in I_k \times I_{\ell}} |a_m \bar{b}_n| \ll (k-l)^{-2} S^{1/2}_k T^{1/2}_{\ell}
\end{align*}
where\pageoriginale $S_k = \sum\limits_{n \in I_k} n |a_n|^2$ and
$T_{\ell} = \sum\limits_{n \in I_{\ell}} n |b_n|^2$. Hence the
contribution to $\sum_2$ from $k,\ell$ with $|k-\ell| \geq 3$ is
$\sum\limits_{|k-\ell| \geq 3} (S^{1/2}_k T^{1/2}_{\ell} / (k-\ell)^2)
\ll (\sum\limits_k S_k)^{1/2}\break (\sum\limits_k T_k)^{1/2}$. Now we
consider those terms of $\sum_2$ with $|k-l| <3$. A typical term is  
$$
\int\limits^1_0 \sum\limits_{(m,n) \in I_k \times I_{\ell}} \frac{a_m \bar{b}_n e^{2\pi i(\lambda_m -\lambda_n)y}}{\lambda_m - \lambda_n} dy 
$$
Here the inner sum is 
$$
N \left( \sum\limits_{(m,n) \in I_k \times I_{\ell}} \frac{a'_m \bar{b}'_n}{(N\lambda_m) - (N \lambda_n)} \right)
$$
where $a'_m = a_m e^{2\pi i \lambda_m y}$ and $b'_n = b_ne^{2\pi i \lambda_n y}$, and $N$ is any positive number. Observe that if $N = 2^{k+800}$, then the integral parts of $N \lambda_m(m \in (I_k \cup I_{\ell}))$ differ each other by at least 3. Also in the denominator we replace $N \lambda_{m} -N \lambda_n$ by $[N\lambda_m] - [N\lambda_n]$ the consequent error being
$$
O\left( N \sum \frac{|a_m b_n|}{([N \lambda_m] - [N\lambda_n])^2}\right)
$$
which is easily seen to be $O(S^{1/2}_k T^{1/2}_\ell)$. Next by Lemma \ref{c1:lem2c} we see that
$$
N \sum\limits_{(m,n) \in I_k \times I_{\ell}} \frac{a'_m \bar{b}'_n}{[N\lambda_m] - [N\lambda_n]} = O(S^{1/2}_k T^{1/2}_\ell).
$$
Thus we see that if $|k-\ell| < 3$, the contribution of
$\sum\limits_{(m,n) \in I_k \times  I_{\ell}} \ldots $ to $\sum_2$ is
$O(S^{1/2}_k T^{1/2}_\ell)$. Combining all this one sees easily that
$$\sum_2 = O((\sum n |a_n|^2)^{1/2} \times (\sum n
|b_n|^2)^{1/2}).$$ 
The method of estimation of $\sum_2$ shows that
$$\sum_3 = \sum |a_n|^2 + O(\sum n |a_n|^2) = O(\sum n |a_n|^2)$$ 
and
$$\sum_4 = O(\sum n |b_n|^2).$$ 
Trivially $\sum | a_n b_n | \leq (\sum
|a_n|^2)^{1/2} (\sum|b_n|^2)^{1/2}$ and so 
$$
E = O \left( \left(\sum n|a_n|^2 \right)^{1/2} \left(\sum n |b_n|^2 \right)^{1/2} \right) .
$$
This\pageoriginale completes the proof of Theorem \ref{c1:thm1.4.3}.

\section{Hadamard's Three Circles Theorem}\label{c1:sec1.5}
We begin by stating the following version of the maximum modulus principle.

\begin{theorem}\label{c1:thm1.5.1}
Let $f(z)$ be a non-constant analytic function defined on a bounded domain $D$. Let, for every $\xi \in $ boundary of $D$, and for every sequence $\{z_n\}$ with $z_n \in D$ which converges to $\xi \in$ boundary of $D$,
$$
\overline{\lim} |f(z_n)| \leq M.
$$
Then
$$
|f(z)| < M \text{ for all } z \in D.
$$
\end{theorem}

\begin{remark*}
We do not prove this theorem (for a reference see the notes at the end of this chapter).
\end{remark*}

\begin{theorem}[HADAMARD]\label{c1:thm1.5.2}
Let $f(z)$ be an analytic function regular for $r_1 \leq |z| \leq r_3$. Let $r_1 < r_2 < r_3$ and let $M_1, M_2, M_3$ be the maximum of $|f(z)|$ on the circles $|z| = r_1$, $|z| = r_2$ and $|z| = r_3$ respectively. Then
$$
M^{\log (r_3/ r_1)}_2 \leq M^{\log (r_3 /r_2)}_1 M^{\log (r_2/r_1)}_3
$$
\end{theorem}

\begin{proof}
Put $\lambda =\frac{m}{n}$ where $m$ and $n$ are integers with $n \geq 1$. Let $\phi (z) = (f(z) z^{\lambda})^{n}$. By maximum modulus principle applied to $\phi(z)$, we have
$$
M^{n}_2 r^m_2 \leq \max (M^{n}_1 r^m_1 , M^{n}_3 r^m_3).
$$
Thus
$$
M_2 \leq \max (M_1 (r_1/r_2)^{m/n}, M_3 (r_3/r_2)^{m/n}).
$$
Now let $\lambda$ be any real number. We let $m/n$ approach $\lambda$ through any sequence  of rational numbers. Hence we get
$$
M_2 \leq \max (M_1 (r_1 / r_2)^{\lambda}, M_3 (r_3/ r_2)^\lambda).
$$
for all real numbers $\lambda$. We now chose $\lambda$ by
$$
\lambda \log (r_1/r_2) + \log M_1 = \lambda \log (r_3/r_2) + \log M_3
$$
i.e. by\pageoriginale $\lambda = (\log M_3 - \log M_1) (\log r_1 - \log r_3)^{-1}$. Thus we have
$$
\log M_2 \leq \log M_1 + \frac{(\log M_3 - \log M_1) (\log r_2 - \log r_1)}{\log r_3 - \log r_1}
$$
This completes the proof of Theorem \ref{c1:thm1.5.2}.
\end{proof}

\section{Borel-Caratheodory Theorem}\label{c1:sec1.6}

\begin{theorem}[BOREL-CARATH\'EODORY]\label{c1:thm1.6.1}
Suppose $f(z)$ is analytic in $|z-z_0| \leq R$ and on the circle $z=z_0 + \re^{i\theta} (0 \leq \theta \leq 2 \pi)$, we have, $\re f(z) \leq U$. Then in $|z-z_0| \leq r < R$ we have
\begin{equation*}
|f(z) - f(z_0)| \leq \frac{2r(U -\re f(z_0))}{R-r} \tag{1.6.1}\label{c1:eq1.6.1}
\end{equation*}
and, for $j\geq 1$
\begin{equation*}
|\frac{f^{(j)} (z)}{j!}| \leq \frac{2R}{(R-r)^{j+1}} (U-Re f(z_0)). \tag{1.6.2}\label{c1:eq1.6.2}
\end{equation*}
\end{theorem}

\begin{proof}
Let $f(z) = \sum\limits^\infty_{n=0} a_n (z-z_0)^n$ and $\varphi(z) = f(z) - f(z_0)$. Clearly $a_0 = f(z_0)$. Let $a_n = |a_n| e^{i\alpha_n}$, $0 \leq \alpha_n < 2 \pi$ for $n \geq 1$. On $|z-z_0| =R$ we have 
$$
\re \varphi(z) = \frac{1}{2} \sum\limits^\infty_{n=1} |a_n| R^n (e^{n i \theta + i \alpha_n} + e^{-ni\theta - i \alpha_n})
$$
and so for any fixed $k=1,2,\ldots$ we have
\begin{align*}
|a_k| R^k \pi & = \int^{2\pi}_0 (\re \varphi (z_0 + \re^{i\theta}))  \left(1+\frac{1}{2} (e^{ki\theta + i \alpha_k} + e^{-ki\theta -i\alpha_k}) \right) d \theta\\
& \leq \int^{2\pi}_0 (U - \re f(z_0)) \left(1+\frac{1}{2} (e^{ki\theta + i \alpha_k} + e^{-ki\theta - i \alpha_k} ) \right) d \theta\\
& = 2\pi (U - \re f(z_0)). 
\end{align*}
Thus
\begin{equation*}
|a_k| \leq 2 R^{-k} (U - \re f (z_0)), k = 1,2,3, \ldots. \tag{1.6.3}\label{c1:eq1.6.3}
\end{equation*}
Now for\pageoriginale $|z-z_0| \leq r < R$, we have,
$$
|f(z) - f(z_0)| \leq \sum\limits^\infty_{n=1} 2 (U - \re f (z_0)) \left(\frac{r}{R} \right)^n = \frac{2r}{R-r} (U - \re f (z_0))
$$
and this proves (\ref{c1:eq1.6.1}). Also for $j=1,2, \ldots$
\begin{align*}
|f^{(j)} (z)| & =  |\varphi^{(j)} (z)| \leq \sum\limits^{\infty}_{n=j} n(n-1) \ldots (n- j+1) |a_n (z-z_0)^{n-j}|\\
&  \leq 2 \sum\limits^\infty_{n=j} n (n-1) \ldots (n-j+1) (U - \re f(z_0)) R^{-n} r^{n-j}\\
& = 2 \left(\frac{d}{dr} \right)^j \sum\limits^\infty_{n=0} (U - \re f(z_0)) \left(\frac{r}{R} \right)^n\\
& = \frac{2R}{(R-r)^{j+1}} (U - \re f(z_0)) (j!).
\end{align*}
This proves (\ref{c1:eq1.6.2}) and hence Theorem \ref{c1:thm1.6.1} is completely proved.
\end{proof}

\section{A Lemma in Complex Function Theorey (Continued)}\label{c1:sec1.7}

In this section we prove theorem \ref{c1:thm1.2.2} and state (with proof) a theorem which is sometimes useful. But before proving Theorem \ref{c1:thm1.2.2} we make four remarks.

\begin{remark}\label{c1:rem3b}
Let $k_1, k_2, \ldots, k_m$ be any set of positive real numbers. Let $f_1(z), f_2(z), \ldots, f_m(z)$ be analytic in $|z| \leq 2r$, and there
$$
|(f_1(z))^{k_1} \ldots (f_m(z))^{k_m}| \leq M \; (M\geq 9).
$$
Then Theorem 2 holds good with $|f(z)|^k$ replaced by $|(f_1(z))^{k_1}
\cdots\break(f_m(z))^{k_m}|$ 
\end{remark}

\begin{remark}\label{c1:rem4b}
A corollary to our result mentioned in Remark \ref{c1:rem3b} was pointed out to us by Professor J.P. Demailly. It is this: Theorem \ref{c1:thm1.2.2} holds good with $|f(z)|^k$ replaced by $\Exp (u)$ where $u$ is any subharmonic function. To prove this it suffices to note that the set of functions of the form $\sum\limits^m_{j=1} k_j \log | f_j(z)|$\pageoriginale is dense in $L^1_{\loc}$ in the set of subharmonic functions. (This follows by using Green-Riesz representation formula for $u$ and approximating the measure $\Delta_u$ by finite sums of Dirac measures).
\end{remark}

\begin{remark}\label{c1:rem5b}
Consider $k=1$ in Theorem \ref{c1:thm1.2.2}. Put $\varphi(z) = f^{(\ell)}(z)$ the $\ell^{\rm th}$ derivative of $f(z)$. Then our method of proof gives
$$
|\varphi(0)| \leq C M^{-A} + C (\log M)^{\ell+1} \left( \frac{1}{4r} \int^{4r}_{-4r} |f(iy)| dy\right),
$$
where $C$ depends only on $A$ and $\ell$.
\end{remark}

\begin{remark}[Due to J.-P. Demailly]\label{c1:rem6b}
In view of the example $f(z) =\break (\frac{e^{nz}-1}{nz})^2$, where $n$ is
a large positive integer and $r=1$, the result of Remark
\ref{c1:rem5b} is best possible. 
\end{remark}

We now prove Theorem \ref{c1:thm1.2.2}. The proof consists of four steps.

\begin{step}\label{c1:step1}
First we consider the circle $|z|=r$. Let
\begin{equation*}
 0 < 2x \leq r\tag{1.7.1} \label{c1:eq1.7.1}
\end{equation*}
and let $PQS$ denote respectively the points $re^{i\theta}$ where $\theta = - \cos^{-1} (\frac{2x}{r})$, $\cos^{-1}(\frac{2x}{r})$ and $\pi$. By the consideration of Riemann mapping theorem and the zero can-cellation factors we have for a suitable meromorphic function $\phi(z)$ (in $PQSP$) that (we can assume that $f(z)$ has no zeros on the boundary)
\begin{equation*}
F(z) = (\phi(z) f(z))^k \tag{1.7.2}\label{c1:eq1.7.2}
\end{equation*}
is analytic in the region enclosed by the straight line $PQ$ and the circular arc $QSP$. Further $\phi(z)$ satisfies
\begin{equation*}
|\phi(z)| =1 \tag{1.7.3}\label{c1:eq1.7.3}
\end{equation*}
on the boundary of $PQSP$ and also
\begin{equation*}
|\phi (0)| \geq 1. \tag{1.7.4}\label{c1:eq1.7.4}
\end{equation*}
Let 
\begin{equation*}
X = \Exp (u_1 + u_2 + \ldots + u_n) \tag{1.7.5}\label{c1:eq1.7.5}
\end{equation*}
where\pageoriginale $u_1, u_2, \ldots, u_n$ vary over the box $\mathcal{B}$ defined by
$$
0 \leq u_j \leq B(j = 1, 2, \ldots, n),
$$
and $B>0$.
\end{step}

We begin with

\setcounter{lem}{0}
\begin{lem}\label{c1:lem1d}
The function $F(z)$ defined above satisfies
\begin{equation*}
F(0) = I_1 + I_2 \tag{1.7.6}\label{c1:eq1.7.6}
\end{equation*}
where 
\begin{equation*}
I_1 = \frac{1}{2 \pi i} \int_{PQ} F(z) X^z \frac{dz}{z} \tag{1.7.7}\label{c1:eq1.7.7}
\end{equation*}
and 
\begin{equation*}
I_2 = \frac{1}{2 \pi i} \int_{QSP} F(z) X^z \frac{dz}{z} \tag{1.7.8}\label{c1:eq1.7.8}
\end{equation*}
where the lines of integration are the straight lien $PQ$ and the circular arc $QSP$.
\end{lem}

\begin{proof}
Follows by Cauchy's theorem.
\end{proof}

\begin{lem}\label{c1:lem2d}
We have
\begin{equation*}
|I_1| \leq \frac{e^{2Bnx}}{2\pi} \int_{PQ} |(f(z))^k \frac{dz}{z}|. \tag{1.7.9}\label{c1:eq1.7.9}
\end{equation*}
\end{lem}

\begin{proof}
Follows since $|X^z| \leq e^{2Bnx}$ and also $|\phi(z)| =1$ on $PQ$. 
\end{proof}

\begin{lem}\label{c1:lem3d}
We have,
\begin{equation*}
|B^{-n}| \int_{\mathcal{B}}  I_2 du_1 \ldots du_n | \leq e^{2 \mathcal{B} nx} \left(\frac{2}{Br} \right)^n M. \tag{1.7.10}\label{c1:eq1.7.10}
\end{equation*}
\end{lem}

\begin{proof}
Follows since on $QSP$ we have $|\phi (z)| =1$ (and so $|F(z)| \leq M$) and also
$$
|B^{-n}| \int_\mathcal{B} \left( \int_{QSP} X^z \frac{dz}{ 2 \pi i z}\right) du_1 \ldots du_n | \leq \left(\frac{2}{Br} \right)^n.
$$
\end{proof}

\begin{lem}\label{c1:lem4d}
We have,\pageoriginale
\begin{equation*}
|f(0)|^k \leq e^{2Bnx} \left(\frac{2}{Br} \right)^n M + \frac{e^{2Bnx}}{2\pi} \int_{PQ} |(f(z))^k \frac{dz}{z}| . \tag{1.7.11}\label{c1:eq1.7.11}
\end{equation*}
\end{lem}

\begin{proof}
Follows by Lemmas \ref{c1:lem1d}, \ref{c1:lem2d} and \ref{c1:lem3d}.
\end{proof}

\begin{step}\label{c1:step2}
Next in (\ref{c1:eq1.7.11}), we replace $|f(z)|^k$ by an integral over a chord $P_1Q_1$ (parallel to $PQ$) of $|w| = 2r$, of slightly bigger length with a similar error. Let $x_1$ be any real number with
\begin{equation*}
|x_1| \leq x .\tag{1.7.12}\label{c1:eq1.7.12}
\end{equation*}
\begin{equation*}
\begin{cases}
& \text{Let $P_1 Q_1 R_1$ be the points $2re^{i\theta}$}\\
& \text{where $\theta = - \cos^{-1} \left(\frac{x_1}{2r} \right)$, $0$ and $\cos^{-1} \left( \frac{x_1}{2r} \right)$.} \\
& \text{(If $x_1$ is negative we have to consider the points} \\
& \text{$\theta = - \frac{\pi}{2} - \sin^{-1} \left( \frac{x_1}{2r}, 0\right)$ and $\frac{\pi}{2} + \sin^{-1} \left(\frac{x_1}{2r} \right)$).}
\end{cases}
\tag{1.7.13}\label{c1:eq1.7.13}
\end{equation*}
Let $X$ be as in (\ref{c1:eq1.7.5}). As before let
\begin{equation*}
G(w) = (\psi(w) f(w))^k \tag{1.7.14}\label{c1:eq1.7.14}
\end{equation*}
be analytic in the region enclosed by the circular arc $P_1 R_1 Q_1$ and the straight line $Q_1 P_1$ (we can assume that $f(z)$ has no zeros on the boundary $P_1 R_1 Q_1 P_1$). By the consideration of Riemann mapping theorem and the zero cancelling factors there exists such a meromorphic function $\psi(w)$ (in $P_1 R_1 Q_1 P_1$) with the extra properties,
\begin{equation*}
  |\psi(w)| =1 \text{ on the boundary of } P_1 R_1 Q_1 P_1 \text{ and } |\psi (z)| \geq 1.  \tag{1.7.15}\label{c1:eq1.7.15}
\end{equation*}
\end{step}

\begin{lem}\label{c1:lem5d}
We have with $z$ on $PQ$,
$$
G(z) = I_3 + I_4
$$
where
\begin{equation*}
I_3 = \frac{1}{2 \pi i} \int_{Q_1P_1} G(w) X^{-(w-z)} \frac{dw}{w-z} \tag{1.7.16}\label{c1:eq1.7.16}
\end{equation*}
and\pageoriginale
\begin{equation*}
I_4  = \frac{1}{2\pi i} \int_{P_1 R_1 Q_1} G(w)  X^{-w(w-2)} \frac{dw}{w-z}. \tag{1.7.17}\label{c1:eq1.7.17}
\end{equation*}
\end{lem}

\begin{proof}
Follows by Cauchy's theorem.
\end{proof}

\begin{lem}\label{c1:lem6d}
We have with $z$ on $PQ$
\begin{equation*}
|I_3| \leq \frac{e^{3Bnx}}{2\pi} \int_{P_1 Q_1} \left| (f(w))^k \frac{dw}{w-z}\right| \tag{1.7.18}\label{c1:eq1.7.18}
\end{equation*}
\end{lem}

\begin{proof}
Follows since $|X^{-(w-z)}| \leq e^{3 B nx} $ and $| \psi (w)| =1$ on $P_1 Q_1$. 
\end{proof}

\begin{lem}\label{c1:lem7d}
We have with $z$ on $PQ$,
\begin{equation*}
|B^{-n} \int_\mathcal{B} I_4 du_1 \ldots du_n | \leq e^{3 B nx} \left(\frac{2}{Br} \right)^nM.  \tag{1.7.19}\label{c1:eq1.7.19}
\end{equation*}
\end{lem}

\begin{proof}
Follows since on $P_1 R_1 Q_1$ we have $|\psi (w)| =1$ (and so $|G(w)|\leq M$) and also
$$
|B^{-n} \int_\mathcal{B} \int x^{-w(-z)} \frac{dw}{2 \pi i (w -z)} du_1 \ldots du_n| \leq \left(\frac{2}{Br} \right)^n.
$$
\end{proof}

\begin{lem}\label{c1:lem8d}
We have with $z$ on $PQ$,
\begin{equation*}
|f(z)|^k \leq e^{3Bnx} \left(\frac{2}{Br} \right)^n M + \frac{e^{3Bnx}}{2\pi} \int_{P_1 Q_1} |f(w)|^k |\frac{dw}{w-z}|.  \tag{1.7.20}\label{c1:eq1.7.20}
\end{equation*}
\end{lem}

\begin{proof}
Follows from Lemmas \ref{c1:lem5d}, \ref{c1:lem6d} and \ref{c1:lem7d}.
\end{proof}

\begin{step}\label{c1:step3}
We now combine Lemmas \ref{c1:lem4d} and \ref{c1:lem8d}.
\end{step}

\begin{lem}\label{c1:lem9d}
We have
\begin{equation*}
|f(0)|^k \leq e^{2 B nx} \left(\frac{2}{Br} \right)^n M + J_1 + J_2  \tag{1.7.21}\label{c1:eq1.7.21}
\end{equation*}
where
\begin{equation*}
J_1 = \frac{e^{5Bnx}}{2\pi}  \left(\frac{2}{Br} \right)^n M \int_{PQ}  |\frac{dz}{z}|, \tag{1.7.22}\label{c1:eq1.7.22}
\end{equation*}
and\pageoriginale
\begin{equation*}
J_2 = \frac{e^{5Bnx}}{(2\pi)^2} \int_{P_1Q_1} |f(w)|^k \left( \int_{PQ} |\frac{dz}{z(w-z)}|\right) |dw| . \tag{1.7.23}\label{c1:eq1.7.23}
\end{equation*}
\end{lem}

\begin{lem}\label{c1:lem10d}
We have
\begin{equation*}
\int_{PQ} |\frac{dz}{z}| \leq 2 + 2 \log \left(\frac{r}{2x} \right). \tag{1.7.24}\label{c1:eq1.7.24}
\end{equation*}
\end{lem}

\begin{proof}
On $PQ$ we have $z = 2 x + iy$ with $|y| \leq r$ and $2x \leq r$. We split the integral into $|y| \leq 2x$ and $2x \leq |y| \leq r$. On these we use respectively the lower bounds $|z| \geq 2x$ and $|z| \geq y$. The lemma follows by these observations.
\end{proof}

\begin{lem}\label{c1:lem11d}
We have for $w$ on $P_1Q_1$ and $z$ on $PQ$,
\begin{equation*}
\int_{PQ}| \frac{dz}{z(w-z)}| \leq \frac{6}{x} . \tag{1.7.25}\label{c1:eq1.7.25}
\end{equation*}
\end{lem}

\begin{proof}
On $PQ$ we have $\re z = 2x$ and on $P_1 Q_1$ we have $|\re w | \leq x$ and so $|\re (w -z)| \geq x$. We have
$$
|\frac{dz}{z(w -z)}| \leq | \frac{dz}{z^2} | + |\frac{dz}{(w-z)^2}|. 
$$

Writing $z = 2x + iy$ we have
\begin{align*}
\int_{PQ} |\frac{dz}{z^2}|  \leq \frac{2}{(2x)^2} 2x + 2 \int^\infty_{2x} \frac{dy}{y^2} = 
 \frac{2}{x}. 
\end{align*}
Similarly
\begin{align*}
\int_{PQ} |\frac{dz}{(w-z)^2}| & \leq 2 \left(\frac{1}{x} + \int^{\infty}_x \frac{dy}{y^2} \right) = \frac{4}{x}. 
\end{align*}
This completes the proof of the lemma.
\end{proof}

\begin{step}\label{c1:step4}
We collect together the results in Steps \ref{c1:step3} and \ref{c1:step4} and choose the parameters\pageoriginale $B$ and $n$ and this will give Theorem \ref{c1:thm1.2.2}. Combining Lemmas \ref{c1:lem9d}, \ref{c1:lem10d} and \ref{c1:lem11d} we state the following lemma.
\end{step}

\begin{lem}\label{c1:lem12d}
We have
\begin{gather*}
|f(0)|^k \leq e^{2Bnx} \left(\frac{2}{Br} \right)^n M + \frac{e^{5Bnx}}{\pi} \left(\frac{2}{Br} \right)^n \left( 1+ \log \frac{r}{2x} \right) M\\
+ \frac{e^{5Bnx}}{(2\pi)^2} \cdot \frac{6}{x} \int_{P_1 Q_1} |(f(w))^k dw|, \tag{1.7.26} \label{c1:eq1.7.26}
 \end{gather*}
where $0 < 2x \leq r$, $x_1$ is any real number with $|x_1| \leq x$, $n$ any natural number and $B$ is any positive real number and $P_1 Q_1$ is the straight line joining $-r_0$ and $r_0$ where $r_0 = \sqrt{4r^2 - x^2_1}$. 
\end{lem}

Next we note that $1 + \log \frac{r}{2x} \leq \frac{r}{2x}$ and so by putting $x = r (\log M)^{-1}$ the first two terms on the RHS of (\ref{c1:eq1.7.26}) together do not exceed
$$
\left(\frac{2}{Br} \right)^n e^{5Bnx} \left(1+\frac{1}{2\pi} \log M\right) M \leq 2 \left(\frac{2}{Br} \right)^n e^{5Bnx} M \log M.
$$
Also,
$$
\frac{6}{x} = \frac{6 \log M}{r} = 6 \log M \left(\frac{2r_0}{r} \right) \frac{1}{2 r_0} \leq (24 \log M ) \left(\frac{1}{2r_0} \right).
$$
Thus RHS of (\ref{c1:eq1.7.26}) does not exceed
$$
2\left(\frac{2}{Br} \right)^n e^{5Bnx} M \log M + \left( \frac{24}{ (2 \pi)^2} e^{5 B nx} \log M\right)  \left( \frac{1}{2r_0} \int_{P_1 Q_1} |(f(w))^k dw|\right).
$$
We have chosen $x =r(\log M)^{-1}$. We now choose $B$ such that $Br = 2e$ and $n = [C \log M] +1$, where $C \geq 1$ is any real number. We have $5 B nx \leq \frac{5Bnr}{\log M} \leq 10 e(C+1) \leq 28 (C+1)$ and also
$$ 
\left(\frac{2}{Br} \right)^n \leq e^{-C \log M} = M^{-C}.
$$
With these choices of $x,B,n$ we see that RHS of (\ref{c1:eq1.7.26}) does not exceed
$$ 
2 M^{-C} e^{28(C+1)} M \log M + \left(\frac{24}{(2\pi)^2} e^{28(C+1) } \log M \right) \left( \frac{1}{2r_0} \int_{P_1Q_1} |(f(w))^k dw| \right).
$$
Putting\pageoriginale $C = A+2$ we obtain Theorem \ref{c1:thm1.2.2} since $C+1 \leq 3A$. This completes the proof of Theorem \ref{c1:thm1.2.2}.

Lastly we note

\begin{theorem}\label{c1:thm1.7.1}
Let $f(z)$ be analytic in $|z| \leq R$. Then for any real $k >0$, we have,
\begin{equation*}
|f(0)|^k \leq \frac{1}{\pi R^2} \int_{|z| \leq R} |f(z)|^k dx\; dy \tag{1.7.27}\label{c1:eq1.7.27}
\end{equation*}
\end{theorem}

\begin{remark*}
The remark below Theorem \ref{c1:thm1.3.2} is applicable here also. We have only to replace $q$ by $k$. 
\end{remark*}

\begin{proof}
We begin by remarking that the theorem is true for $k=1$. Because let $z = re^{i\theta}$ where $0 < r \leq R$. Then by Cauchy's theorem we have
\begin{equation*}
f(0) = \frac{1}{2 \pi i} \int^{2\pi}_i f(re^{i\theta}) i d \theta \tag{1.7.28}\label{c1:eq1.7.28}
\end{equation*}
and so 
\begin{equation*}
|f(0)| \leq \frac{1}{2\pi} \int^{2\pi}_0 |f(re^{i\theta})| d\theta. \tag{1.7.29}\label{c1:eq1.7.29}
\end{equation*}
Multiplying this by $r$ $dr$ and integrating from $r =0$ to $r =R$ we obtain (\ref{c1:eq1.2.7}). Now let $f(0) \neq 0$ (otherwise there is nothing to prove) and
\begin{equation*}
\phi(z) = \left\{ f(z) \prod\limits_{\rho} \left(\frac{r^2 - \bar{\rho} z}{r(z-\rho)}\right)\right\}^k \tag{1.7.30}\label{c1:eq1.7.30}
\end{equation*}
where $\rho$ runs over all the zeros of $f(z)$ satisfying $|\rho| \leq r$. The function $\phi(z)$ is analytic (selecting any branch) in $|z| \leq r$ and so (\ref{c1:eq1.7.29}) holds with $f(z)$ replaced by $\phi(z)$. Notice that on $|z| = r$, we have $|f(z)|^k = |\phi(z)|$ and also that
$$
|\phi (0)| = |f(0)|^k \left( \prod\limits_\rho \frac{r}{|\rho|}\right)^k \geq |f(0)|^k. 
$$
Hence (\ref{c1:eq1.7.29}) holds with $|f(z)|$ replaced by $|f(z)|^k$ and hence we are led  to (\ref{c1:eq1.7.27}) as before.
\end{proof}

\newpage

\begin{center}
\textbf{Notes at the end of Chapter 1}
\end{center}

\S\ \ref{c1:sec1.1}. The author\pageoriginale learnt of convexity principles from A. Selberg who told him about a weaker kernel function. The stronger kernel functions like $\Exp (w^2)$ or $\Exp(w^{4a+2})$ ($a \geq 0$ integer) became known to the author through P.X. Gallagher. The kernel function $\Exp ((\sin w)^2)$ was noticed by the author who used it extensively in various  situations. It should be mentioned that
$$
\frac{1}{2\pi i} \int^{2+ i \infty}_{2-i\infty} X^w \Exp (w^2) dw \text{ and } \frac{1}{2\pi i} \int^{2+ i \infty}_{2-i \infty} X^w \Exp (w^2) \frac{dw}{w}
$$
are non-negative (a fact which the author learnt from {\small D.R.~Heath-Brown}). These things coming from the kernel function $\Exp (w^2)$ are sometimes useful. The two inequalities preceeding (\ref{c1:eq1.1.1}) (taken with the remark after the second) seem to be new. Also the technique of averaging over ``Cubes'' seems to be new. This section is based on the paper \cite{Balasubramanian and  Ramachandra1} of R. Balasubramanian and K. Ramachandra.

\S\ \ref{c1:sec1.2}. The results of this section are improvements and generalisations of some lemmas (in A. Ivi\'c, \cite{Ivic1}) due to D.R. Heath-Brown. We will hereafter refer to this book as Ivi\'cs book. This section as well as \S\ \ref{c1:sec1.7} are based on the papers \cite{Balasubramanian and Ramachandra2} \cite{Balasubramanian and Ramachandra3} of R. Balasubramanian and K. Ramachandra.


\S\ \ref{c1:sec1.3}. The convexity Theorem (1.3.3) is very useful in a later chapter. This section is based on the paper \cite{Gabriel1} of R.M. Gabriel. See also the appendix to the paper \cite{Ramachandra3} of K. Ramachandra. For the convexity Theorem (1.3.3) see p. 13 of the paper \cite{Balasubramanian and Ramachandra4} of R. Balasubramanian and K. Ramachandra. 


\S\ \ref{c1:sec1.4}. We do not prove (\ref{c1:eq1.4.1}) with the constant $\frac{3\pi}{2}$ although we use it in later chapters. We prove it will some unspecified constant in place of $\frac{3\pi}{2}$. The proof of Theorem \ref{c1:thm1.4.2} is based on the paper \cite{Ramachandra4} of K. Ramachandra. For Remark 2 below this theorem see K. Ramachandra \cite{Ramachandra5}.

\S\ \ref{c1:sec1.5}. For Theorem \ref{c1:thm1.5.1} we refer the reader to (K. Chandrasekharan, \cite{Chandrasekharan1}). The proof of Theorem \ref{c1:thm1.5.2} given here is due to R. Balasubramanian. For the general\pageoriginale  principles of complex function theory necessary for Chapters \ref{c1} and \ref{c7} one may refer to K. Chandrasekharan \cite{Chandrasekharan1} or E.C. Titchmarsh \cite{Titchmarsh2} or L. Ahlfors \cite{Ahlfors1}.

