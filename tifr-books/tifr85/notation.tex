\chapter{Notation}

Except very rarely, the notation is standard. The letters $C$, $c$ with or without suffixes denote constants. Sometimes we use $A,B,D,E$, also for constants. $T$ and $H$ will be real variables in the domain $T \geq H \geq C \log \log T$. $T$ will be sufficiently large. Sometimes we use $T \geq H \geq C$. In Chapter II we give explicit constants everywhere with the hope that the results will be useful in many situations. However we have not attempted to get economical constants. The same remarks are true of weak Titchmarsh series with which we deal in Chapter III. The letters $w,z$ and $s$ will be reserved for complex variables. Very often we write $w = u + iv$, $z = x + iy$ and $s = \sigma + it$. But some times there may be exceptions (for example in Chapter VII) in the notation for $u,v,x,y,\sigma$ and $t$. The letter $k$ will be often real. Sometimes it is a constant and sometimes it is a variable depending on the context. The letter $\theta$ will denote an arbitrary real number. (In the chapter on introductory remarks as well as in Chapter V, $\theta$ will denote the least upper bound of the real parts of the zeros of $\zeta(s)$. Of course as is usual $\zeta(s) = \sum\limits^{\infty}_{n=1} n^{-s} (\sigma > 1)$ and its analytic continuations). We write $\Exp (z)$ for $e^z$. Other standard notation used is as follows:
\begin{enumerate}
\item[{\rm (1)}] $f(x) \sim g(x)$ as $x \to x_0$ means $f(x) (g(x))^{-1} \to 1$ as $x \to x_0$, where $x_0$ is possibly $\infty$.

\item[{\rm (2)}] $f(x) = O(g(x))$ with $g(x) \geq 0$ means $|f(x)(g(x))^{-1}|$ does not exceed a constant independent of $x$ in the range in context.

\item[{\rm (3)}] $f(x) \ll g (x)$ means $f(x) = O(g(x))$ and $f(x) \gg g(x)$ will mean the same as $g(x) = O(f(x))$ and $f(x) \asymp g(x)$ will mean both $f(x) \ll g(x)$ and $g(x) \ll f(x)$.

\item[{\rm (4)}] $f(x) = \Omega (g(x))$ will mean $f(x)(g(x))^{-1}$
  does not tend to zero. $F(x) = \Omega_+ (g(x))$ will mean $\lim \sup
  (f(x) (g(x))^{-1}) > 0$. $f(x) = \Omega_-(g(x))$  will mean $\lim
  \inf (f(x) (g(x))^{-1}) < 0$. Also $f(x) = \Omega_{\pm}\break (g(x))$ will
  mean both $f(x) = \Omega_+ (g(x))$ and $f(x) = \Omega_-(g(x))$. In
  these $\Omega$ results in the text the range in context is as $x \to
  \infty$. 

\item[{\rm (5)}] The letters $\epsilon, \delta, \eta$ will denote small positive constants.  
\end{enumerate}




