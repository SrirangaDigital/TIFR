
\chapter{Mean-Value Theorems for the Fractional Powers of $|\zeta(\frac{1}{2} + it)|$}\label{c4}

\section{Introduction}\label{c4:sec4.1}
In \S\ \ref{c4:sec4.2}\pageoriginale of this chapter we consider lower bounds for 
$$
\max\limits_{\sigma \geq \alpha} \left( \frac{1}{H} \int^{T+H}_T |\frac{d^m}{ds^m} (\zeta(s))^{2k}|d t\right)
$$
where $\frac{1}{2} \leq \alpha \leq 2$, $m\geq 0$ is an integer constant, $k$ is any complex constant and $T \geq H \gg \log \log T$. Let now $\alpha = \frac{1}{2}$. If Riemann hypothesis (RH) is true then we establish (we mean than our method gives) the lower bound
$$
\gg (\log H)^{|k^2| + m}.
$$
If we do not assume RH then we can deal only with $k = \frac{p}{q}$ (where $q \geq 1$ and $p>0$ are integers) and $0 \leq m \leq k$. In that case we obtain the lower bound
$$
\gg (q^{-1} \log H)^{k^2 + m}
$$
provided $\alpha =\frac{1}{2} + q (\log H)^{-1}$. We can also allow $k(> 0)$ to be real with $|k -\frac{p}{q}| \leq (\log \log H)^{-1}$ and $1 \leq q \leq 10 \log \log H$. Here uniformly we have the lower bound
$$
\gg (q^{-1} \log H)^{k^2 + m} \geq \left( \frac{\log H}{10 \log \log H}\right)^{k^2 + m}.
$$
So far (in this book) we have not used the functional equation for $\zeta(s)$. In the upper bounds problem for the same integral very little is known. The best known result is that (see \S\ \ref{c4:sec4.3} of this chapter)
$$
\max\limits_{\sigma \geq \alpha} \left( \frac{1}{H} \int^{T+H}_T |\frac{d^m}{ds^m} (\zeta(s))^{2k} | dt\right) \ll (\log T)^{k^2 + m}
$$
provided $\alpha =\frac{1}{2}$, $k=\frac{1}{2}$ and $H = T^{\lambda_0}$ with any constant $\lambda_0$ satisfying $\frac{1}{2} < \lambda_0 \leq 1$. Of course it is enough to prove the upper bound for $m=0$. The result for general $m$ is deducible from the case $m=0$ from easy principles (unlike the\pageoriginale lower bound). However the result with $k=\frac{1}{n}$, $m=0$ is true for all $n\geq 1$ (the case $n=1$ is trivial). In the case $n\geq 3$ we can not talk of upper bounds unless $m=0$. (We can however manage for all $m$ if $k=\frac{1}{2}$). Except the case $k=1$ all other cases depend on the functional equation. It will be a great achievement to prove (even assuming RH) results like 
$$
\frac{1}{T} \int^{2T}_T |\zeta(\frac{1}{2} + it)|^{2k} dt \ll (\log T)^A
$$
for some $A>0$ and some $k >2$. The biggest integer $k$ for which this is known is $k=2$ and in this case we have an asymptotic formula for the mean-value. (For this we do not need RH). Trivially given this for any $k>0$, its truth for all smaller positive $k$ follow by Holder's inequality. Of course RH implies that the upper bound is $\ll \Exp \left(\frac{10 k \log T}{\log \log T} \right)$ and it would be of great interest to know whether for any constant $k >2$ (of course bigger the $k$ the better) the inequality
$$
\frac{1}{T} \int^{2T}_T |\zeta(\frac{1}{2} + it)|^{2k} dt \ll_\epsilon T^\epsilon
$$
holds for every $\epsilon > 0$ (without assuming RH). This knowledge improves the range for $h$ in the asymptotic formula
$$
\pi (x + h) - \pi (x) \sim \frac{h}{\log x}.
$$
The truth of 
$$
\frac{1}{T} \int^{2T}_T |\zeta(\frac{1}{2} + it)|^{2k} dt \ll_{\epsilon,k} T^\epsilon
$$
for every integer $k>0$ and every $\epsilon > 0$ is equivalent to Lindel\"of hypothesis as can be easily seen from $\zeta'(\frac{1}{2} + it) = O(t)$.

\section{Lower Bounds}\label{c4:sec4.2}
We divide this section conveniently into three parts. Part A deals with statement of the result and remarks, and statement of some preliminary results. Part B deals with a reduction of the problem. Part C deals with completion of the proof.

\begin{center}
\textbf{PART A}
\end{center}

\begin{theorem}\label{c4:thm4.2.1}
Let\pageoriginale $\alpha$ and $k$ be real numbers subject to $\frac{1}{2} + q (\log H)^{-1} \leq \alpha \leq 2$ (where $q$ is a positive integer to be defined presently) and $\delta \leq k \leq \delta^{-1}$ where $\delta$ is any positive constant. Let $m$ be any non-negative integer subject to $0\leq m \leq 2k$ (no restriction on $m$ if $2k$ is an integer). Then
$$
\max\limits_{\sigma \geq \alpha} \left(\frac{1}{H} \int^{T+H}_T |\frac{d^m}{ds^m} (\zeta(s))^{2k}| dt\right) > C (\delta, m) \left(\alpha - \frac{1}{2}\right)^{-k^2 -m},
$$
where $s = \sigma + it$, $T$ and $H$ are subject to $T \geq H \geq H_0 = H_0(\delta, m) $, $C (\delta , m)$ and $H_0 (\delta, m)$ being positive constants depending only on $\delta$ and $m$. The integer $q$ is defined as follows. It is any integer subject to $1\leq q \leq 10 \log \log H$, $|k-\frac{p}{q}| \leq (\log \log H)^{-1} $ if $2k - m \geq \delta$, i.e. if $m \leq [2k] -1$, provided $2k \leq [2k] + \delta$ ($p$ being a positive integer). If $2k \geq [2k] + \delta$ then $\frac{p}{q} - k \geq 0$ is another extra condition in addition to $m \leq 2k$ (in place of $m \leq 2k -\delta$). Here after we write $\rho = \frac{p}{q}$.
\end{theorem}

\begin{coro*}
We have, for $a C'(\delta, m) > 0$ depending only on $\delta$ and $m$,
$$
\frac{1}{H} \int^{T+H}_T |\frac{d^m}{ds^m}(\zeta(s))^{2k}|_{\sigma =\frac{1}{2}} dt > C' (\delta, m) (q^{-1} \log H)^{k^2 + m}, 
$$
provided only that $T \geq H \gg \log \log T$ with a certain positive constant implied by the Vinogradov symbol $\gg$.
\end{coro*}

\setcounter{remark}{0}
\begin{remark}\label{c4:rem1}
Our proof of Theorem \ref{c4:thm4.2.1} depends only on the Euler product and the analytic continuation of $\zeta(s)$ in $(\sigma \geq \alpha, T \leq t \leq T + H)$. Thus it goes through for more general Dirichlet series where Euler product and analytic continuation in the region (just mentioned) are available. In particular it goes through for zeta and $L$-functions of algebraic number fields.
\end{remark}


\begin{remark}\label{c4:rem2}
If we assume Riemann hypothesis ($\zeta(s) \neq 0$ in $\sigma \geq \alpha$, $T \leq t \leq T + H$ will do) then we can prove much more namely this: Let $\alpha$ be real subject to $\frac{1}{2} + (\log H)^{-1} \leq \alpha \leq 2$ and $k$ be any complex number suject to $\delta \leq | k| \leq \delta^{-1}$.\pageoriginale Then for all integers $m\geq 0$, we have
$$
\max\limits_{\sigma \geq \alpha}  \left( \frac{1}{H} \int^{T+H}_T |\frac{d^m}{ds^m} (\zeta(s))^{2k}|dt\right) > C (\delta, m) \left( \alpha -\frac{1}{2}\right)^{-|k^2|-m},
$$
where $C(\delta, m) > 0$ depends only on $\delta$ and $m$. As a corollary we can obtain
$$
\left(\frac{1}{H} \int^{T+H}_T |\frac{d^m}{ds^m} (\zeta(s))^{2k}|_{\sigma =\frac{1}{2}} dt \right) > C' (\delta, m) (\log H)^{|k^2|+m},
$$
where $C'(\delta, m) > 0$ depends only on $\delta$ and $m$ and $T \geq H \gg \log \log T$, with a suitable constant implied by $\gg$. (Remark \ref{c4:rem1} is also applicable).
\end{remark}

We prove Theorem \ref{c4:thm4.2.1} with $\alpha =\frac{1}{2} + q (\log H)^{-1}$ and leave the general $\alpha$ as an exercise. Also we leave the deduction of the corollary to Theorem \ref{c4:thm4.2.1} as an exercise (we have to use the fact that the integrand in Theorem is bounded above on $\sigma =2$ and use the convexity result stated in Theorem \ref{c4:thm4.2.3} below, with the kernel related to $\Exp ((\sin s)^2)$).

\bigskip

\medskip
\noindent{\textbf{\Large \S\ 2. Some Preliminaries.}} Before commencing the proof we recall four theorems with suitable notation and remarks. The first is the convexity theorem of R.M. Gabriel. In this section we use $q$ for a positive real number which may or may not be the integer introduced already.

\begin{theorem}\label{c4:thm4.2.2}
Let $z = x + iy$ be a complex variable. Let $D_0$ be a closed rectangle with sides parallel to the axes and let $L$ be the closed line segment parallel to the $y$-axis which divides $D_0$ into two equal parts. Let $D_1$ and $D_2$ be the two congruent rectangles into which $D_0$ is divided by $L$. Let $K_1$ and $K_2$ be the boundaries of $D_1$ and $D_2$ (with the line $L$ excluded). Let $F(z)$ be analytic in the interior of $D_0$ and $|f(z)|$ be continuous on the boundary of $D_0$. Then, we have,
$$
\int_L |f(z)|^q |dz| \leq \left(\int_{K_1} |f(z)|^q |dz|\right)^{\frac{1}{2}} \left( \int_{K_2} |f(z)|^q |dz| \right)^{\frac{1}{2}},
$$
where $q >0$ is any real number.
\end{theorem}

\begin{remark*}
The assertion\pageoriginale of the theorem still holds if $|f(z)|^q$ is replaced by $|\varphi(z) || f(z)|^q$, where $\varphi(z)$ is any function analytic inside $D_0$ and such that $|\varphi (z)|$ is continuous on the boundary of $D_0$. To see this replace $f(z)$ by $(f(z))^j (\varphi(z))^r$ and $q$ by $qj^{-1}$ where $j$ and $r$ are positive integers and $j$ and $r$ tend to infinity in such a way that $rj^{-1} \to q^{-1}$. 
\end{remark*}

\begin{proof}
See Theorem \ref{c1:thm1.3.2}.

We now slightly extend this as follows. Consider the rectangle defined by $0 \leq x \leq (2^n +1) a$  (where $n$ is a non-negative integer and $a$ any positive real number), and $0\leq y \leq R$. Let $I_x$ denote the integral $\int^R_0 |f(z)|^q  dy$, where as before $z=x+ iy$. Let $Q_\alpha$ denote the maximum of $|f(z)|^q$ on ($0\leq x \leq \alpha, y =0$ and $y=R$). Then we have (assuming $f(z)$ to be analytic in the interior of $(0\leq x \leq (2^n + 1)a, \; 0 \leq y \leq R)$ and $|f(z)|$ continuous on its boundary) the following theorem.
\end{proof}

\begin{theorem}\label{c4:thm4.2.3}
Put $b_\nu = 2^\nu + 1$. Then for $\nu =0 , 1,2,\ldots , n$ we have, 
$$
I_\alpha \leq (I_0 + U)^{\frac{1}{2}} (I_\alpha + U)^{\frac{1}{2} - 2^{-\nu-1}} (I_{ab_\nu} + U)^{2^{-\nu-1}}
$$
where $U = 2^{2(\nu+1)} aQ_{ab_\nu}$.
\end{theorem}

\begin{remark*}
The remark below Theorem \ref{c4:thm4.2.2} is applicable here also. 
\end{remark*}

\begin{proof}
See Theorem 1.3.3.
\end{proof}

\begin{theorem}\label{c4:thm4.2.4}
Let $f(z)$ be analytic in $|z| \leq R$, $q$ be any positive real constant (not necessarily the same $q$ and $R$ as before). Then, we have,
$$
|f(0)|^q \leq \frac{1}{\pi R^2} \int_{|z| \leq R} |f(z)|^q dx dy.
$$
\end{theorem}

\begin{remark*}
The remark below Theorem \ref{c4:thm4.2.2} is applicable here also.
\end{remark*}

\begin{proof}
This result follows from Cauchy's theorem with proper zero cancellation factors.

The next theorem is a well-known theorem due to H.L. Montgomery and R.C. Vaughan, (see Theorem \ref{c1:thm1.4.3}).
\end{proof}

\begin{theorem}\label{c4:thm4.2.5}
Let\pageoriginale $\{a_n\} (n=1,2,3,\ldots)$ be any sequence of complex numbers which may or may not depend on $H (\geq 2)$. Then subject to the convergence of $\sum\limits^\infty_{n=1} n |a_n|^2$, we have,
$$
\frac{1}{H} \int^H_0 |\sum\limits^\infty_{n=1} a_n n^{it}|^2 = \sum\limits^\infty_{n=1} |a_n|^2 \left(1+O \left( \frac{n}{H}\right) \right),
$$
where the $O$-constant is absolute.
\end{theorem}

\begin{remark*}
This theorem is not very easy to prove but very convenient to use in several important situations. But it should be mentioned that for the proof of Theorem \ref{c4:thm4.2.1} it suffices to use a rough result for the mean-value of $|\sum\limits_{n\leq N} a_n n^{it}|^2$ where $N$ is a small positive constant power of $H$. The result which we require in this connection is very easy to prove.
\end{remark*}

\begin{center}
\textbf{PART B}
\end{center}

For any Dirichlet series $F(s) = \sum\limits^\infty_{n=1} a_n n^{-s}$ and $Y \geq 1$ we write $F(s, Y) = \sum\limits_{n\leq Y} a_n n^{-s}$. Also we write
$$
Z_1 = \left(\frac{d^m}{ds^m} (\zeta^k (s,Y))^2 \right) (\zeta^{\rho-k} (s,Y))^2.
$$
Note that $\zeta^k (s,Y) \neq (\zeta (s,Y))^k$. We will show in a few lemmas that the proof of Theorem 1 reduces to proving that
\begin{equation*}
\frac{1}{H} \int^{T+H}_T  |Z_1| dt \gg D^{-\mu''} (q^{-1} \log H)^{k^2 + m}  \tag{4.2.1}\label{c4:eq4.2.1}
\end{equation*}
where $\mu''$ is a certain positive constant, $\sigma_0 =\frac{1}{2} + 10 q(\log H)^{-1}$, $a = Dq (\log H)^{-1}$ $s=\sigma_0 + a + it$ and $D$ is a large positive constant. The rest of the proof consists in proving (\ref{c4:eq4.2.1}).

\setcounter{lem}{0}
\begin{lem}\label{c4:lem1}
Let\pageoriginale
\begin{equation*}
\max\limits_{\sigma \geq \frac{1}{2} + q (\log H)^{-1}} \left(\frac{1}{H} \int^{T+H}_T |\frac{d^m}{ds^m}(\zeta (s))^{2k}| dt \right) < (\log H)^{k^2+m}.  \tag{4.2.2}\label{c4:eq4.2.2}
\end{equation*}
Then for $\nu = 0,1,2,\ldots , m$, we have,
\begin{equation*}
\max\limits_{\sigma \geq \frac{1}{2} +q(\log H)^{-1}} \left(\frac{1}{H} \int^{T+H}_T |\frac{d^\nu}{ds^\nu} (\zeta(s))^{2k}|dt\right) < (\log H)^{k^2 + m+1}. \tag{4.2.3}\label{c4:eq4.2.3}
\end{equation*}
\end{lem}

\begin{remark*}
Note that we are entitled to assume that the LHS of (\ref{c4:eq4.2.2})
does not exceed {\small $(q^{-1} \log H)^{k^2 + m}$} since otherwise Theorem
\ref{c4:thm4.2.1}~is~proved.  
\end{remark*}

\begin{proof}
We have, for $\frac{1}{2} + q(\log H)^{-1} \leq \sigma \leq 2$,
$$
\int^\sigma_2 \frac{d^m}{ds^m} (\zeta(s))^{2k} d\sigma = \frac{d^{m-1}}{ds^{m-1}} (\zeta(s))^{2k} + O(1).
$$
So
$$
|\frac{d^{m-1}}{ds^{m-1}} (\zeta(s))^{2k} | \leq \int^\sigma_2 |\frac{d^m}{ds^m} (\zeta(s))^{2k}| d\sigma + O(1).
$$
Integrating this with respect to $t$ we obtain the result for $\nu = m -1$. Continuing this process we can establish this lemma for $\nu = 0,1,2, \ldots, m$.
 \end{proof}

\begin{lem}\label{c4:lem2}
Divide $(T+1, T + H -1)$ into abutting intervals $I$ of length $(\log H)^A$ (where $A > 0$ is a large constant) ignoring a bit at one end. Let $m(I)$ be the maximum of $|\zeta(s)|^{2k}$ in $(\sigma \geq \frac{1}{2} + (q+1) (\log H)^{-1} , t \in I)$. Then we have,
\begin{equation*}
\sum\limits_I m(I) \leq H (\log H)^{k^2 + m + 4}. \tag{4.2.4}\label{c4:eq4.2.4}
\end{equation*}
\end{lem}

\begin{proof}
We observe that the value of $|\zeta(s)|^{2k}$ at any point (where $m(I)$ is attained) is majorised by its mean-value over a disc of radius $(\log H)^{-1}$ with that point as centre. This follows by the application of Theorem \ref{c4:thm4.2.4}. Now by Lemma \ref{c4:lem1} the proof is complete.
\end{proof}

\begin{lem}\label{c4:lem3}
In $(\sigma \geq \frac{1}{2} + (q+2) (\log H)^{-1}, T + 1 \leq t \leq T + H-1)$, we have,
$$
|\frac{d^m}{ds^m} (\zeta(s))^{2k}| \leq H^{1.5} \text{ and so } |\zeta(s)|^{2k} \leq H^2.
$$
\end{lem}

\begin{proof}
By Lemma \ref{c4:lem1}\pageoriginale the proof follows by arguments similar to the one by which we obtained Lemma \ref{c4:lem2}.
\end{proof}

\begin{lem}\label{c4:lem4}
Let $B > 0$ be any (large) constant. Then the number of intervals I for which $m(I)\geq (\log H)^B$ is
$$
\leq H (\log H)^{k^2 + m + 4 - B}.
$$
\end{lem}

\begin{proof}
By Lemma \ref{c4:lem2} the proof follows.
\end{proof}

\begin{lem}\label{c4:lem5}
Let accent denote the sum over those I for which $m(I) < (\log H)^B$. Also let $\sigma \geq \frac{1}{2} + (q+3) (\log H)^{-1}$. Then for $\delta \leq k \leq \delta^{-1}$, we have, 
\begin{align*}
& \frac{1}{H} \int^{T+H}_T |\frac{d^m}{ds^m} (\zeta(s))^{2k}|  dt\\
\gg & \frac{1}{H} \sum_1' \left(\int_I |\frac{d^m}{ds^m} (\zeta(s))^{2k} \mid \zeta(s)|^{2\rho - 2k} dt - (\log H)^A \right), \tag{4.2.5}\label{c4:eq4.2.5}
\end{align*}
where $\rho =\frac{p}{q}$ is a rational approximation to $k$ such that either (i) $2(\log\break\log H)^{-1} \geq 2 \rho - 2k \geq 0$ and $0 \leq m \leq 2k$ (no restriction  on $m$ if $2k$ is an integer) or (ii) $|2\rho - 2k | \leq 2 (\log \log H)^{-1}$ and $2k - m \geq \eta > 0$ for some constant $\eta > 0$. In both the cases (i) and (ii) it is assumed that $1 \leq q \leq 10 \log \log H$. The implied constant in the inequality asserted by the lemma is independent of $H,q$ and $k$.
\end{lem}

\begin{remark*}
There is always a solution of $|\rho - k| \leq (\log \log H)^{-1}$, $1 \leq q \leq 10 \log \log H$. This can be easily seen by box principle.
\end{remark*}

\begin{proof}
We introduce the factor $|\zeta(s)|^{2(\rho - k)} \times |\zeta(s)|^{2(k-\rho)}$. Now in case (i)
$$
|\zeta(s)|^{2(k-\rho) 2k (2k)^{-1}} \geq ((\log H)^{-B})^{(\rho  - k)k^{-1}} \gg 1. 
$$
In case (ii) we have only to consider $\rho - k < 0$ and $2k - m \geq \eta >0$. We divide $I$ into two parts (iii) that for which $\max |\zeta(s)| \leq (\log H)^{-B'}$ and (iv) the rest, ($B'>0$ being a suitable large constant).

In (iv) we have
$$
|\zeta(s)|^{2(k-\rho)} \geq ((\log H)^{-B'})^{2(k-\rho)} \geq ((\log H)^{-B'})^{2(\log \log H)^{-1}} \gg  1. 
$$
In case (iii)\pageoriginale  we plainly omit it and consider
$$
\int_{I^*}  |\frac{d^m}{ds^m} (\zeta(s))^{2k} || \zeta(s)|^{2(\rho -k)} dt,
$$
(where $I^* = I \cap $ $(\max |\zeta(s)| > (\log H)^{-B'})$)
$$
=\int_I |\frac{d^m}{ds^m} (\zeta(s))^{2k} || \zeta(s)|^{2(\rho - k)} dt - \int_{I^{**}} |\frac{d^m}{ds^m} (\zeta(s))^{2k} || \zeta(s)|^{2(\rho -k)} dt
$$
(where $I^{**} = I \cap (\max |\zeta(s)| \leq (\log H)^{-B'})$). For large $B'$ it is easily seen, since $2k-m\geq \eta > 0$, that the integral over $I^{**}$ is $\leq (\log H)^A$. (Note that in $\sigma \geq \frac{1}{2} + (q+3) (\log H)^{-1}$ the derivatives of order $\leq m$ of $\zeta(s)$ are in absolute value not more than a bounded power of $\log H$, the bound depending only on $\delta$ and $m$).
\end{proof}


\begin{lem}\label{c4:lem6}
For any two complex numbers $A_0$, $B_0$ and any real number $q>0$, we have,
\begin{equation*}
|A_0|^{\frac{1}{q}} \leq 2^{\frac{1}{q}} \left(|A_0 - B_0|^{\frac{1}{q}} + |B_0|^{\frac{1}{q}} \right).
\tag{4.2.6}\label{c4:eq4.2.6}
\end{equation*}
\end{lem}

\begin{proof}
We have $A_0 = A_0 -B_0 + B_0$ and so
$$
|A_0| \leq 2 \max (|A_0 -B_0|, |B_0|). 
$$
This gives the lemma.
\end{proof}

\begin{lem}\label{c4:lem7}
Let $J$ denote the interval I with intervals of length $(\log H)^2$ being removed from both ends. Let $Z_1$ be as already introduced in the beginning of this section namely,
\begin{equation*}
Z_1 = \left(\frac{d^m}{ds^m} (\zeta^k (s,Y))^2\right) (\zeta^{\rho -k} (s,Y))^2, \tag{4.2.7}\label{c4:eq4.2.7}
\end{equation*}
and let 
\begin{equation*}
A_0 = Z^q_1 \text{ and } B_0 = \left(\left(\frac{d^m}{ds^m} (\zeta(s))^{2k}\right) (\zeta^{2\rho - 2k} (s))\right)^q. \tag{4.2.8}\label{c4:eq4.2.8}
\end{equation*}
Then\pageoriginale for any $\sigma \geq \frac{1}{2}$, we have, by Lemma \ref{c4:lem6}.
\begin{align*}
& 2^{\frac{1}{q}} \int_J |\left(\frac{d^m}{ds^m} (\zeta(s))^{2k}\right) \zeta^{2\rho - 2k} (s)| dt \\
& \geq \int_J |Z_1| dt - 2^{\frac{1}{q}} \int_J |B_0 -A_0|^{\frac{1}{q}} dt. \tag{4.2.9} \label{c4:eq4.2.9}
\end{align*}
where $q > 0$ is any real number.
\end{lem}

\begin{proof}
Follows from
$$
2^{\frac{1}{q}} |B_0|^{\frac{1}{q}} \geq |A_0|^{\frac{1}{q}} - 2^{\frac{1}{q}} |B_0 - A_0|^{\frac{1}{q}}.
$$
\end{proof}

\begin{lem}\label{c4:lem8}
We have, for $\sigma \geq \frac{1}{2}$,
\begin{equation*}
\sum'_I \int_J |Z_1| dt = \int^{T+H}_T |Z_1| dt + O(H(\log H)^{-1}) ,  \tag{4.2.10}\label{c4:eq4.2.10}
\end{equation*}
provided $Y$ is a small positive constant power of $H$ and also that the constant $A>0$ is large enough.
\end{lem}

\begin{proof}
Follows by Holder's inequality applied to the integral over the complementary interval and Montgomery-Vaughan theorem.

From now on $q$ will be the denominator of $\rho$. We now apply Theorem \ref{c4:thm4.2.3} with $\frac{1}{q}$ in place of $q$ and state a lemma.
\end{proof}

\begin{lem}\label{c4:lem9}
Write 
\begin{equation*}
f_0 (s) = \left( \left( \frac{d^m}{ds^m} (\zeta(s))^{2k} \right) \zeta^{2\rho - 2k} (s)\right)^q - Z^q_1  \tag{4.2.11}\label{c4:eq4.2.11}
\end{equation*}
and $w = u + iv$ for a complex variable and put $K(w) = \Exp (w^2)$. Also write $\tau = (\log H)^2$ and $f(s,w) = f_0 (s+w) K(w)$. Then, we have for $t \in J$ and $a>0$,
\begin{gather*}
\int_{|v| \leq r} |f(s,w)|^{\frac{1}{q}}_{u=0}  |dw| \leq \left( \int_{|v| \leq r} |f(s,w)|^{\frac{1}{q}}_{u=-a} |dw| + H^{-10}\right)^{\frac{1}{2}} \times\\
\times \left( \int_{|v| \leq r} |f(s,w)|^{\frac{1}{q}}_{u=0} |dw|  + H^{-10}\right)^{\frac{1}{2} - 2^{-n-1}}\\
\times \left(\int_{|v| \leq r} |f(s,w)|^{\frac{1}{q}}_{u =a b_n -\alpha} |dw| + H^{-10}\right)^{2^{-n-1}}, \tag{4.2.12}\label{c4:eq4.2.12}
\end{gather*}\pageoriginale
provided $\sigma_0 > \frac{1}{2}$ and $a (2^n+1)$ is bounded above. Here $s$, $\sigma_0$ and a are as in (\ref{c4:eq4.2.1}).
\end{lem}

\begin{proof}
Follows by Theorem \ref{c4:thm4.2.3}.
\end{proof}

\begin{lem}\label{c4:lem10}
If $a > 0$ and $ab_n$ is bounded above, we have,
\begin{gather*}
\int_t \int_v |f(s,w)|^{\frac{1}{q}}_{u=0} |dw| dt \leq  \left( \int_t \int_v |f(s,w)|^{\frac{1}{q}}_{u=- a} |dw| dt + H^{-5}\right)^{\frac{1}{2}} \times\\
\times \left(\int_t\int_v |f(s,w)|^{\frac{1}{q}}_{u=0} |dw| dt + H^{-5} \right)^{\frac{1}{2} - 2^{-n-1}} \times \\
\times \left(\int_t \int_v |f(s,w)|^{\frac{1}{q}}_{u=ab_n -a} |dw| dt + H^{-5} \right)^{2^{-n-1}} . \tag{4.2.13}\label{c4:eq4.2.13}
\end{gather*}
The limits of integration are determined by $t \in J$ and $|v| \leq \tau$ with $\tau = (\log H)^2$.
\end{lem}

\begin{proof}
Follows by Holder's inequality.

Summing over $J$ (counted by the accent) and applying Holder's inequality we state a lemma.
\end{proof}

\begin{lem}\label{c4:lem11}
We have, with $\sigma \geq \frac{1}{2} + 10 q(\log H)^{-1}$,
\begin{gather*}
\sum'_J \int_t \int_v |f(s,w)|^{\frac{1}{q}}_{u=0} |dw| dt \leq \left(\sum'_J  \int \int \ldots_{u = -a} |dw| dt + H^{-3}\right)^{\frac{1}{2}} \times \\
\times \left( \sum'_J \int \int \ldots_{u=0} |dw| dt + H^{-3} \right)^{\frac{1}{2} - 2^{-n-1}} \times  \\
\times \left( \sum'_J \int \int \ldots_{u = a b_n-a} |dw| dt + H^{-3} \right)^{2^{-n-1}}  \tag{4.2.14}\label{c4:eq4.2.14}
\end{gather*}\pageoriginale
provided $a>0$ and $ab_n$ is bounded above.
\end{lem}

We now complete the reduction step as follows. In Lemma \ref{c4:lem11} either LHS $\leq H^{-3}$ in which case the quantity
$$
2^{\frac{1}{q}} \sum'_J \int_J |B_0 - A_0|^{\frac{1}{q}} dt
$$
is small enough to assert that the sum over $J$ (accented ones) of the quantity on the LHS of (\ref{c4:eq4.2.9}) exceeds
$$
\frac{1}{2} \int^{T+H}_T |Z_1| dt
$$
where $\sigma = \sigma_0 + a$. On the other hand if in Lemma \ref{c4:lem11}, LHS $\geq H^{-3}$,
\begin{align*}
& \left( \sum'_J \int_t \int_v |f(s,w)|^{\frac{1}{q}}_{u=0} |dw| dt\right)^{\frac{1}{2} + 2^{-n-1}}\\
& \qquad \qquad \leq 2 \left(\int^{T + H -r}_{T+r}  \int_v |f(s,w)|^{\frac{1}{q}}_{u=-a} |dw| dt + H^{-3} \right)^{\frac{1}{2}} \times \\
& \qquad \qquad  \times \left(\int^{T+H-r}_{T+r} \int_v |f(s,w)|^{\frac{1}{q}}_{u=ab_n-a} |dw| dt + H\right)^{2^{-n-1}}
\end{align*}
Now by using the fact that $|K(w)| \ll \Exp (-v)^2$ for all $v$, and also that $|K(w)|\gg 1$ for $|v| \leq 1$, we obtain
\begin{gather*}
\left(\sum'_I \int_{t \in J} |f_0 (s)|^{\frac{1}{q}}_{\sigma = \sigma_0 +a} dt\right)^{\frac{1}{2} + 2^{-n-1}} \ll \left(\sum'_I \int_{t \in I} |f_0 (s)|^{\frac{1}{q}}_{\sigma = \sigma_0} dt + H^{-3}\right)^{\frac{1}{2}} \times\\
\times \left(\sum'_I \int_{t \in I} |f_0(s)|^{\frac{1}{q}}_{\sigma = \sigma_0 + a b_n} dt + H^{-3} \right)^{2^{-n-1}} .\tag{4.2.15} \label{c4:eq4.2.15}
\end{gather*}
(From now\pageoriginale on we stress that the constant implied by the Vinogradov symbols $\ll$ and  $\gg$ depend only on $\delta$ and $m$). From now on we assume that
\begin{equation*}
\max\limits_{\sigma \geq \frac{1}{2} + q (\log H)^{-1}} \left(\frac{1}{H} \int^{T+H}_T |\frac{d^m}{ds^m} (\zeta(s))^{2k}| dt\right) \leq (q^{-1} \log H)^{k^2 + m} . \tag{4.2.16}\label{c4:eq4.2.16}
\end{equation*}
From this it follows (since $Y$ is a small positive constant power of $H$) as we shall see (by Lemma \ref{c4:lem5} and remark following it) that (see Part C for explanations)
$$
\frac{1}{H} \sum'_1 \int_{t\in I} |f_0(s)|^{\frac{1}{q}}_{\sigma = \sigma_0} dt \ll (q^{-1} \log H)^{k^2 +m}
$$
Also we shall see that (by choosing $n$ such that $\sigma_0 + ab_n$ lies between 2 large positive constants)
$$
\frac{1}{H} \sum'_I \int_{t \in I} |f_0(s)|^{\frac{1}{q}}_{\sigma = \sigma_0 + ab_n} dt \ll H^{-\mu' q^{-1}}
$$
where $\mu'$ is a certain positive constant. If $\sigma_0 =\frac{1}{2} + 10 q(\log H)^{-1}$ and $a = Dq (\log H)^{-1}$, $D$ being a large positive constant (our estimations will be uniform in $D$) we have $2^n Dq (\log H)^{-1}$ lies between two positive constants and so $2^n \asymp (Dq)^{-1} \log H$ and so $2^{-n-1} \asymp Dq(\log H)^{-1}$. Hence it would follow that
$$
\frac{1}{H} \sum'_I \int_{t \in J} |f_0 (s)|^{\frac{1}{q}}_{\sigma = \sigma_0 + a}dt \ll (q^{-1} \log H)^{k^2 + m} e^{-\mu D}
$$ 
where $\mu$ is a certain positive constant.

The rest of the work consists in proving that
$$
\frac{1}{H} \int^{T+H}_T |Z_1| dt \gg D^{-\mu''} (q^{-1} \log H)^{k^2 + m}
$$
where $\mu''$ is a certain positive constant.

\begin{center}
\textbf{PART C}
\end{center}

We recall that\pageoriginale 
$$
Z_1 = \left(\frac{d^m}{ds^m} (\zeta^k (s,Y))^2\right) (\zeta^{\rho-k} (s,Y))^2. 
$$
We now write
\begin{equation*}
Z_2 = \frac{d^m}{ds^m} (\zeta^\rho (s,Y))^2 \tag{4.2.17}\label{c4:eq4.2.17}
\end{equation*}
and go on to prove that (for $\frac{1}{2} < \sigma \leq 2$)
\begin{equation*}
\frac{1}{H} \int^{T+H}_T |Z_1 -Z_2| dt \ll \left((\log \log H)^{-1} + H^{-\mu q^{-1} (\sigma -\frac{1}{2})} \right) \left(\sigma -\frac{1}{2}\right)^{-k^2 - m}, \tag{4.2.18}\label{c4:eq4.2.18}
\end{equation*}
where $\mu$ is a positive constant which we may take to be the same as before. This will be done in three stages to be stated in Lemma \ref{c4:lem12}. We introduce
\begin{equation*}
Z_3 = \left( \frac{d^m}{ds^m} \left( \left( \zeta^{\frac{1}{q}} (s, Y_1)\right)^q\right)^{2k}\right) \; \left( \left(\zeta^{\frac{1}{q}} (s,Y_1)\right)^q\right)^{2\rho- 2k}  \tag{4.2.19}\label{c4:eq4.2.19}
\end{equation*}
where $Y^q_1$ is a small positive constant power of $H$, and 
\begin{equation*}
Z_4 =\frac{d^m}{ds^m} \left(  \left(\zeta^{\frac{1}{q}} (s, Y_1) \right)^q\right)^{2\rho}, 
\tag{4.2.20}\label{c4:eq4.2.20}
\end{equation*}
We remark first of all that $Z_1, Z_2, Z_3$ and $Z_4$ are Dirichlet polynomials with $Y$ and $Y^q_1$ being small positive constant powers of $H$ and so the contribution of the integrals of $|Z_j|^2$ from any interval of length $(\log H)^2$ contained in $(T, T + H) $ is $O(H(\log H)^{-1})$. With this remark and some standard application of Cauchy's theorem we can prove

\begin{lem}\label{c4:lem12}
We have, for $\frac{1}{2} < \sigma \leq 2$,
\begin{align*}
& \frac{1}{H} \int^{T+H}_T |Z_1 - z_3| dt \ll H^{-\mu q^{-1} (\sigma -\frac{1}{2})} \left(\sigma -\frac{1}{2} \right)^{-k^2 -m},  \tag{4.2.21}\label{c4:eq4.2.21}\\
& \frac{1}{H} \int^{T+H}_T |Z_3 - Z_4| dt \ll (\log \log H)^{-1} \left(\sigma -\frac{1}{2} \right)^{-k^2 -m},  \tag{4.2.22}\label{c4:eq4.2.22}
\end{align*}
and\pageoriginale
\begin{equation*}
\frac{1}{H} \int^{T+H}_T |Z_4 - Z_2| dt \ll H^{-\mu q^{-1} (\sigma -\frac{1}{2})} \left(\sigma - \frac{1}{2} \right)^{-k^2-m} . \tag{4.2.23}\label{c4:eq4.2.23}
\end{equation*}
\end{lem}

\begin{coro*}
We have, for $\frac{1}{2} < \sigma \leq 2$,
\begin{equation*}
\frac{1}{H} \int^{T + H}_T |Z_1 - Z_2| dt \ll \left((\log \log H)^{-1} + H^{-\mu q^{-1} (\sigma - \frac{1}{2})}  \right) \left( \sigma -\frac{1}{2}\right)^{-k^2 -m }.  \tag{4.2.24}\label{c4:eq4.2.24}
\end{equation*}
\end{coro*}

\medskip
\noindent{\textbf{Proof of Lemma 12.}}
We first consider the first and third assertions. The Dirichlet polynomials in the integrands have the property that sufficiently many terms (i.e. $[H^{\mu' q^{-1}}]$ terms for a certain positive constant $\mu'$) in the beginning vanish. Hence it suffices (by the method by which we proved that
$$
\frac{1}{H} \sum'_I \int_{t \in J} |f_0(s)|^{\frac{1}{q}}_{\sigma = \sigma_0+a } dt \ll (q^{-1} \log H)^{k^2 + m} e^{-\mu D}
$$
holds) to check that for $\sigma = \sigma_0$ the estimates
$$
\frac{1}{H} \int^{T + H}_T |Z_j| dt \ll \left(\sigma -\frac{1}{2}\right)^{-k^2 -m}, (j = 1,2,3,4),
$$
hold. Now $Z_3$ and $Z_4$ have the following property: $Z_3$ is the same as $Z_4$ except that ``the coefficients'' differ by $O(k-\rho)$. Hence the second assertion follows if we prove that the mean-value of the absolute value of ``the terms'' are $\ll (\sigma - \frac{1}{2})^{-k^2 -m}$ (the explanation of the terms in the inverted commas will be given presently). We begin by checking the mean-value estimates for $|Z_j| (j=1 \text{ to }4)$. For any function $f_1(s)$ analytic in $(\sigma > \frac{1}{2}, T \leq t \leq T + H)$ we have, by Cauchy's theorem,
$$
|\frac{d^m}{ds^m} f_1(s)| \leq \frac{1}{2\pi} \int_{|w| =r} |\frac{f_1 (s+w) dw}{w^{m+1}}|,
$$
where $r = \frac{1}{2} (\sigma -\frac{1}{2})$ and $T + 2 r \leq t \leq T + H - 2 r$. To prove the mean-value assertion about $|Z_1|$ we put $f_1(s) = \frac{d^m}{ds^m} (\zeta^k (s,Y))^2$ and observe that the mean-value of 
$$
|(\zeta^k (s+w, Y))^2 (\zeta^{\rho -k} (s, Y))^2|
$$
with\pageoriginale respect to $t$ in $T + (\log H)^2 \leq t \leq T + H - (\log H)^2 $ is $O((\sigma -\frac{1}{2})^{-k^2})$ (uniformly with respect to $w$) and so the mean-value of $|Z_1|$ is $O((\sigma - \frac{1}{2})^{-k^2-m})$. Similar result about $|Z_2|$ follows since $\rho -k = O((\log \log H)^{-1})$. 

Now let us look at $f_3 (s) =\frac{d^m}{ds^m} (f_2 (s))^k$ where $f_2(s) = (\zeta^{\frac{1}{q}} (s, Y_1))^{2q} $. When $m=1$, $f_3(s) = k (f_2 (s))^{k-1} f'_2 (s)$. When $m=2$ it is $k(k-1) (f_2(s))^{k-2} (f'_2(s))^2 + k (f_2 (s))^{k-1} f''_2 (s)$ and so on. By induction we see that for general $m$, we have,
$$
f_3(s) = \sum\limits_{j_1  + j_2 + \ldots + j_\nu = m} g_{j_1, \ldots, j_\nu} (k) (f_2(s))^{k-\nu} (f^{(j_1)}_2 (s)) (f^{(j_2)}_2 (s)) \ldots (f^{(j_\nu)}_2 (s))
$$
where the $g's$ depend only on $j_1, \ldots, j_\nu$ and $k$. To obtain $Z_3$ we have to multiply $f_3(s)$ by $(f_2 (s))^{\rho-k}$. Hence $Z_4$ is the same as $Z_3$ with $g's$ replaced by $g_{j_1, \ldots, j_\nu} (\rho)$. Thus
$$
g_{j_1, \ldots, j_\nu} (k) - g_{j_1, \ldots, j_\nu} (\rho) = O(k-\rho) = O((\log \log H)^{-1}
).
$$
The terms like $(f_2 (s))^{k-\nu} (f^{(j_1)}_2 (s)) \ldots (f^{(j_\nu)}_2(s))$ contribute $O((\sigma -\frac{1}{2})^{-k^2 -m})$ by using Cauchy's theorem as before. Thus Lemma \ref{c4:lem12} is completely proved.

\begin{lem}\label{c4:lem13}
For $\frac{1}{2} + C (\log H)^{-1} \leq \sigma \leq 2$ ($C$ being a large positive constant), we have,
\begin{equation*}
\frac{1}{H} \int^{T+H}_T |Z_2| dt \gg (\sigma -\frac{1}{2})^{-k^2 -m}. \tag{4.2.25}\label{c4:eq4.2.25}
\end{equation*}
\end{lem}

\begin{proof}
We recall that $Z-2 = \frac{d^m}{ds^m} f_4(s)$ where $f_4 (s) = (\zeta^\rho (s,Y))^2$ and $Y$ is a small positive constant power of $H$. By Montgomery-Vaughan theorem
$$
C_1 \left(\sigma -\frac{1}{2} \right)^{-k^2} \leq \frac{1}{H} \int^{T+H}_T |f_4 (s)| dt \leq C_2 \left(\sigma -\frac{1}{2} \right)^{-k^2}
$$
where $C_2 > C_1 > 0$ are constants, provided $\sigma\geq \frac{1}{2} + C (\log H)^{-1}$. If $m=0$ we are through, (otherwise $2k \geq 1$ and so $k \geq \frac{1}{2}$). Let $\beta' = (2C_2 C^{-1}_1)^{-4} (\alpha' -\frac{1}{2}) + \frac{1}{2}$ where $\alpha' > \frac{1}{2}$. Then
$$
\frac{1}{H} \int^{T+H}_T |f_4 (s)|_{\sigma = \beta'} dt - \frac{1}{H} \int^{T+H}_T |f_4 (s)|_{\sigma = \alpha'} dt
$$
\begin{align*}
& \geq C_1 \left( \left(2C_2 C^{-1}_1 \right)^{-4} \left(\alpha' -\frac{1}{2} \right) \right)^{-k^2} - C_2 \left(\alpha' -\frac{1}{2} \right)^{-k^2}\\
& \geq C_1(2 C_2 C^{-1}_1) \left(\alpha' -\frac{1}{2} \right)^{-k^2} -C_2 \left(\alpha' -\frac{1}{2} \right)^{-k^2}\\
& = C_2 \left(\alpha' - \frac{1}{2} \right)^{-k^2}
\end{align*}\pageoriginale
Also
\begin{align*}
& \frac{1}{H} \int^{T+H}_T (|f_4 (s)|_{\sigma = \beta'} - |f_4 (s)|_{\sigma = \alpha'}) dt\\
& \leq \frac{1}{H} \int^{T+H}_T |f_4 (\beta' + it) - f_4 (\alpha' + it)| dt\\
& \leq \frac{1}{H} \int^{\alpha'}_{\beta'} \left(\int^{T+H}_T |f'_4 (u+it) |dt \right) du. 
\end{align*}
Thus there exists a number $\gamma'$ with $\beta' < \gamma' < \alpha'$ such that if $m=1$ and $\sigma = \gamma'$ the lower bounds is
$$
C_2 \left(\alpha' -\frac{1}{2} \right)^{-k^2} (\alpha' -\beta')^{-1} \gg \left(\gamma' -\frac{1}{2} \right)^{-k^2 -1}
$$
and by induction there is a number $\sigma = \alpha'_m$ where the lower bound is $\gg \left(\alpha'_m-\frac{1}{2} \right)^{-k^2 -m}$ for general $m$ (since the upper bound required at each stage of induction is available by a simple application of Cauchy's theorem). Now from a given $\alpha'_m$ we can pass onto general $\sigma$ by an application of Theorem \ref{c4:thm4.2.3}. (Note that $|Z_2|$ is bounded both above and below when $\sigma$ is large enough. We can select two suitable value of $\sigma$). Hence Lemma \ref{c4:lem13} is completely proved.
\end{proof}

\begin{lem}\label{c4:lem14}
We have, for $\frac{1}{2} + D q(\log H)^{-1} \leq \sigma \leq 2$, ($D > 0$ being a large constant),
\begin{align*}
& \frac{1}{H} \int^{T+H}_T |Z_1| dt \\
& \geq C'_m \left(\sigma -\frac{1}{2} \right)^{-k^2 -m} - C''_m \left(\sigma -\frac{1}{2} \right)^{-k^2 -m } H^{-\mu q^{-1} (\sigma -\frac{1}{2})}\\
& - C'''_m  \left(\sigma -\frac{1}{2} \right)^{-k^2 - m} (\log \log H)^{-1}, 
\end{align*}
where\pageoriginale $C'_m, C''_m$ and $C'''_m$ are positive constants independent of $D$. Also
$$
\frac{1}{H} \int^{T+H}_T (|Z_1|_{\sigma = \frac{1}{2} + D q (\log H)^{-1}}) dt \gg (q^{-1} \log H)^{k^2 + m}. 
$$
\end{lem}

\begin{proof}
Follows from Lemma \ref{c4:lem13} and the corollary to Lemma \ref{c4:lem12}. This proves our main theorem (namely Theorem \ref{c4:thm4.2.1}) completely.
\end{proof}

\section{Upper Bounds}\label{c4:sec4.3}
The object of this section is to prove the following theorem.

\begin{theorem}\label{c4:thm4.3.1}
Let $k$ be a constant of the type $\frac{1}{j}$ where $j(\geq 2)$ is an integer. Let $H = T^{\frac{1}{2} + \epsilon}$ where $\epsilon(0 < \epsilon < \frac{1}{2})$ is any constant. Then, we have, 
\begin{gather*}
\frac{1}{H} \int^{T+H}_T |\zeta(s)|^{2k}_{\sigma = \frac{1}{2}} dt \ll (\log T)^{k^2},  \tag{4.3.1}\label{c4:eq4.3.1}\\
\frac{1}{H} \int^{T+H}_T |\zeta^{(m)} (s)|_{\sigma =\frac{1}{2}} dt \ll (\log T)^{\frac{1}{4} + m}  \tag{4.3.2}\label{c4:eq4.3.2}
\end{gather*}
and
\begin{equation*}
\frac{1}{H} \int^{T+H}_T |\frac{d^m}{ds^m} (\zeta(s))^{2k}|_{\sigma =\frac{1}{2} + (\log T)^{-1}} dt \ll (\log T)^{k^2 + m}.  \tag{4.3.3}\label{c4:eq4.3.3}
\end{equation*}
The last inequality however assumes RH. In the last two assertions of the theorem $m(\geq 0)$ is an integer constant.
\end{theorem}

\begin{remark*}
It is not hard to prove the modified results of the type
\begin{equation*}
\max\limits_{\sigma \geq \frac{1}{2}} \left(\frac{1}{H} \int^{T+H}_T |\zeta(s)|^{2k} dt \right) \ll (\log T)^{k^2},  \tag{4.3.4}\label{c4:eq4.3.4}
\end{equation*}
and similar results. These can be proved by convexity principles.
\end{remark*}

The letter $r$ denotes a large positive constant which will be chosen in the end. We need the properties of $\zeta(s)$ only in the region $(\sigma \geq \frac{1}{2} - \delta, T \leq t \leq  T+ H)$ for some arbitarily small constant $\delta > 0$. We begin by introducing the following notation.
\begin{align*}
M(\sigma) & = \frac{1}{H} \int^{T+H}_T |\zeta( \sigma + it)|^{2k}  dt \tag{4.3.5}\label{c4:eq4.3.5}\\
A (\sigma) & = \frac{1}{H} \int^{T+H}_T |\sum\limits_{n\leq H} \frac{d_k(n)}{n^{\sigma + it}}|^2 dt \tag{4.3.6} \label{c4:eq4.3.6}
\end{align*}\pageoriginale
and 
\begin{equation*}
M_1 (\sigma) = \frac{1}{H} \int^{T+H}_T |\zeta(\sigma + it)| - \left(\sum\limits_{n\leq H} \frac{d_k(n)}{n^{\sigma+it}}\right)^j |^{2k} dt.  \tag{4.3.7}\label{c4:eq4.3.7}
\end{equation*}

\begin{remark*}
Since $\zeta(\frac{1}{2} + it) = 0 (t^{\frac{1}{4}} (\log t)^2)$ (this follows for example by the functional equation or otherwise) it follows that the quantities $M(\sigma)$, $A(\sigma)$ and $M_1(\sigma)$ get multiplied by $1+ o(1)$ when we change the limits of integration by an amount $O((\log T)^2)$.

We begin by proving three lemmas.
\end{remark*}

\setcounter{lem}{0}
\begin{lem}\label{c4:lem1a}
We have
$$
M(\sigma) \leq 2^{2k} (M_1(\sigma) + A(\sigma)).
$$
\end{lem}

\begin{proof}
The lemma follows from
$$
\zeta(s) = \zeta(s) - \left(\sum\limits_{n \leq H} \frac{d_k (n)}{n^s} \right)^j + \left(\sum\limits_{n\leq H} \frac{d_k (n)}{n^s}\right)^j.
$$
\end{proof}

\begin{lem}\label{c4:lem2a}
Let $\sigma_0 =\frac{1}{2} + r(\log T)^{-1}$ and $M = M (1-\sigma_0)$. Then, we have,
\begin{equation*}
M(\sigma_0) \sim M \Exp (-2kr).  \tag{4.3.8}\label{c4:eq4.3.8}
\end{equation*}
\end{lem}

\begin{proof}
The lemma follows by the functional equation. (Functional equation and this consequence will be proved in the appendix at the end of this book).
\end{proof}

\begin{lem}\label{c4:lem3a}
We have at least one of the following two possibilities:
\begin{equation*}
M(1-\sigma_0) \ll A(\sigma_0) + A(1-\sigma_0) \tag{4.3.9}\label{c4:eq4.3.9}
\end{equation*}
or 
\begin{equation*}
M_1(\sigma_0) \ll M_1 (1 -\sigma_0) \Exp (-\lambda) \tag{4.3.10}\label{c4:eq4.3.10}
\end{equation*}
where\pageoriginale $\lambda = 4 kr(\log H) (\log T)^{-1}$ and in the second of these possibilities the constant implied by $\ll$ is independent of $r$.
\end{lem}

\begin{proof}
We apply the convexity Theorem \ref{c4:thm4.2.3} in a manner similar to what we did in Lemmas \ref{c4:lem9}, \ref{c4:lem10} and \ref{c4:lem11} of part B of \S\ \ref{c4:sec4.2}. Let $w = u + iv$ be a complex variable,
$$
f(s,w) = \left(\zeta (s+w) - \left(\sum\limits_{n\leq H} \frac{d_k (n)}{n^{s+w}}\right)^j \right) \Exp (w^2)
$$
and
$$
I(\sigma , u) =\frac{1}{H} \int_{(t)} \int_{|v| \leq r} |f(s,w)|^{2k} dv \; dt
$$
where $\tau = (\log T)^2$ and the $t$-range of integration is $T + \tau \leq t \leq T + H-\tau $.

Exactly as before it follows that
\begin{align*}
M_1(\sigma_0) &\leq C_1 (M_1 (1-\sigma_0) + E)^{\frac{1}{2}} (M_1
(\sigma_0 ) + E)^{\frac{1}{2}- 2^{-n-1}}\\ 
&\qquad\quad (M_1 (1-\sigma_0 + (2\sigma_0 -1) (2^n + 1)) + E)^{2^{-n-1}}  
\end{align*}
where $E = (\Exp ((\log T)^3))^{-1}$ and $C_1(>1)$ is independent of $n$ and $r$ provided that $1-\sigma_0 + (2\sigma_0 -1) (2^n+1)$ is bounded above. If either $M_1 (1-\sigma_0) \leq 1$ or $M_1 (\sigma_0) \leq 1$ we end up (by using Lemma \ref{c4:lem1a}) with
$$
M(1-\sigma_0) \leq 2^{2k} (A (1-\sigma_0) + 1) \text{ or } M (\sigma_0) \leq 2^{2k} (A(\sigma_0) +1)
$$
respectively and so by Lemma \ref{c4:lem2a} we end up, in any case, with
$$
M(1-\sigma_0) \ll A (\sigma_0) + A(1-\sigma_0).
$$
In the remaining case we are led to
$$
M_1 (\sigma_0) \leq 2 C_1 (M_1 (1-\sigma_0))^{\frac{1}{2}} (M_1 (\sigma_0))^{\frac{1}{2} - 2^{-n-1}} (H^{-L_0 + C_2} + E)^{2^{-n-1}}
$$
where $L_0 = 2k (2 \sigma_0 - 1) (2^n+1)$ and $C_2 (>1)$ is independent of $n$ and $r$ provided, of course, that $L_0$ is bounded above. Hence we are led to
$$
M_1(\sigma_0) \leq 4 C^2_1 (M_1 (1-\sigma_0))^{L^*} (H^{-L_0 + C_2} + E )^{2^{-n}L^*} 
$$
where\pageoriginale $L^* = (1+2^{-n})^{-1}$. We choose $n$ in such a way that $2^n r (\log T)^{-1}$ is a large constant $C_3$ which is $\asymp r^2$. We have $M_1 (1-\sigma_0) = O(\log T)$ and $(1+ 2^{-n})^{-1} = 1 + O(2^{-n}) = 1 + O(r C^{-1}_3 (\log T)^{-1})$. Now $2^n = C_3 r^{-1} \log T$ and $E = (\Exp ((\log T)^3))^{-1} \leq H^{-L_0 + C_2}$. Thus since
$$
(L_0 - C_2)2^{-n} L^* = 4 kr (\log T)^{-1} + O(r^{-1} (\log T)^{-1}),
$$
we have,
$$
M_1(\sigma_0) \ll M_1(1-\sigma_0) H^{-4 kr (\log T)^{-1}}. 
$$
This proves the lemma.
\end{proof}

We can now complete the proof of Theorem \ref{c4:thm4.3.1} as follows. By Lemma \ref{c4:lem2a}, we have
$$
M_1 (\sigma_0) \sim M \Exp (-2 kr).
$$
By the second possibility of Lemma \ref{c4:lem3a}, we have
$$
M_1 (\sigma_0) \ll (M(1-\sigma_0) + A (1-\sigma_0)) \Exp (-\lambda)
$$
i.e.
$$
M_1 (\sigma_0) \ll (M+A(1-\sigma_0)) \Exp (-\lambda).
$$
Now by Lemma \ref{c4:lem1a}, we have 
$$M_1(\sigma_0) \geq 2^{-2k} M(\sigma_0) - A(\sigma_0) \geq 2^{-4k} M
\Exp (-2kr) - A(\sigma_0)$$ 
for large $r$. Thus
$$
2^{-4k} M \Exp (-2kr) - A(\sigma_0) \leq C_4 (M + A(1-\sigma_0)) \Exp (-\lambda),
$$
where $C_4 (\geq 1)$ is a constant independent of $r$. Hence
$$
M \{2^{-4k} \Exp (-2kr) - C_4 \Exp (-\lambda) \} \leq A (\sigma_0) + C_4 A(1-\sigma_0) \Exp (-\lambda).
$$
In this equation we note that for $r \geq r_0 (\epsilon)$ the coefficient of $M$ on the LHS is bounded below by a positive constant. Now by fixing $r$ to be a large constant we obtain
$$
M\ll A (\sigma_0)  + A(1-\sigma_0).
$$
Using\pageoriginale  Theorem \ref{c4:thm4.2.5} we see that $M \ll (\log T)^{k^2}$. Now by convexity Theorem \ref{c4:thm4.2.3} and the fact that $M(\sigma)$ is bounded  for $\sigma \geq 2$, we see that the first assertion of Theorem \ref{c4:thm4.3.1} proved. The remaining  two assertions of Theorem \ref{c4:thm4.3.1} follow from things like
$$
\zeta^{(m)} (s) = \frac{m!}{2\pi i} \int \frac{\zeta(w)}{(w-s)^{m+1}} dw
$$
(where the integration is over the circle $|w -s| = \frac{1}{3} (\log T)^{-1}$) and by convexity Theorem \ref{c4:thm4.2.3}.

\begin{center}
\textbf{Notes at the end of Chapter V}
\end{center}

\S\ \ref{c4:sec4.1} and \S\ \ref{c4:sec4.2}. From $\zeta(s) = \sum\limits_{n \leq 10 T} n^{-s} + O(T^{-\sigma})$, valid uniformly, for example, in $(\frac{1}{4} \leq \sigma \leq 2, T \leq t \leq 2T)$ and from Montgomery-Vaughan Theorem it follows that 
$$
\frac{1}{T} \int\limits^{2T}_T |\zeta \left(\frac{1}{2} + it \right)|^2 dt = \log T + O(1).
$$
However to prove
$$
\frac{1}{T} \int^{2T}_T |\zeta \left(\frac{1}{2} + it \right)|^4 dt = (2\pi^2)^{-1} (\log T)^{4} + O((\log T)^3),
$$
it seems that the functional equation is unavoidable. This latter result (originally a difficult result due to A.E. Ingham) was proved in a fairly simple way (but still using the functional equation) by K. Ramachandra (See A. Ivic \cite{Ivic1}).

In the direction of lower bounds the earliest general result (see page 174 of Titchmarsh \cite{Titchmarsh1}) is due to E.C. Titchmarsh who proved that for $0<\delta <1$, we have
$$
\int^\infty_0 |\zeta \left(\frac{1}{2} + it \right)|^{2k} e^{-\delta t} dt \gg_k \frac{1}{\delta} \left(\log \frac{1}{\delta} \right)^{k^2}
$$
for all\pageoriginale integers $k \geq 1$. As a corollary this gives
$$ 
\lim\limits_{T \to \infty} \sup \left( \left(\frac{1}{T} \int^{2T}_T |\zeta \left(\frac{1}{2} + it \right)|^{2k} dt\right) (\log T)^{-k^2}\right) > 0,
$$
for all integers $k\geq 1$.

The history of Theorem \ref{c4:thm4.2.1} is as follows. The general problem of obtaining lower bounds for
$$
\max\limits_{\sigma \geq \alpha} \left(\frac{1}{H} \int^{T+H}_T |\frac{d^m}{ds^m} (\zeta(s))^{2k} | dt \right) \quad \left(k > 0,\frac{1}{2} \leq \alpha \leq 2 \right)
$$
where $T \geq H \gg \log \log T$ and $m(\geq 0)$ is an integer constant, was solved by K. Ramachandra with an imperfection factor $(\log \log H)^{-C}$ (see \cite{Ramachandra13}). This imperfection was removed by him in a later paper (see \cite{Ramachandra14}) for all positive integers $2k$. The Next step was teken by D.R. Heath-Brown (see \cite{Heath-Brown1}) who proved that for all rational constants $2k > 0$, we have
$$
\frac{1}{T} \int^T_0 |\zeta \left(\frac{1}{2} + it \right)|^{2k} dt \gg_k (\log T)^{k^2}.
$$
Ramachandra's proof of his theorem mentioned above (valid for short intervals $T,T+H$ and integer constants $2k >0$) did not use Gabriel's two variable convexity theorem. (This depends upon Riemann mapping theorem). The present proof of Theorem \ref{c4:thm4.2.1} (due to K. Ramachandra, see \cite{Ramachandra15}) uses all these ideas in addition to those of Ramachandra's paper (\cite{Ramachandra16}) and Gabriel's theorem in the form established by him and R. Balasubramanian namely Theorem \ref{c4:thm4.2.3} (see their paper \cite{Ramachandra4}). However we do not use the method of obtaining auxiliary zero-density estimates for $\zeta(s)$ adopted in K. Ramachandra \cite{Ramachandra13}. 

\S\ \ref{c4:sec4.3}. The main difference between \S\ \ref{c4:sec4.2} and \S\ \ref{c4:sec4.3} is that \S\ \ref{c4:sec4.3} depends crucially on the functional equation $\zeta(s) = \chi(s) \zeta(1-s)$ where $|\chi(s)| \asymp t^{\frac{1}{2} -\sigma}$ uniformly in $a_1 \leq \sigma \leq b_1$, $t \geq 2$ where $a_1$ and $b_1$ are any two constants. (Owing to the presence of $n(\frac{1}{2} - \sigma)$ in $|\chi(s)|\asymp t^{n(\frac{1}{2} - \sigma)}$ upper bound problems are hopeless\pageoriginale if $n>2$ i.e. if $k >2$ in the mean-value problem). These results will be proved in the appendix at the end. Another result which we have used frequently is 
$$
 \frac{1}{x} \sum\limits_{n \leq x} |d_k(n)|^2 = C(\log x)^{|k^2| -1} (1+O((\log x)^{-1}))
$$
valid for complex constants $k$ and $C = C(k) > 0$. These will also be proved in the appendix. Regarding the history of Theorem \ref{c4:thm4.3.1}, the result (\ref{c4:eq4.3.2}) was first proved by K. Ramachandra. (See \cite{Ramachandra17}). Later D.R. Heath-Brown proved that 
$$
\frac{1}{T} \int^T_0 |\zeta \left(\frac{1}{2} + it \right)|^{2k} dt \ll (\log T)^{k^2}
$$
where $k =\frac{1}{j} $ ($ j \geq 2$ an integer constant). (See \cite{Heath-Brown1}). Another point of interest is the proof of (see K. Ramachandra \cite{Ramachandra17})
$$
\frac{1}{H}  \int^{T+H}_T |\zeta^{(m)} \left(\frac{1}{2} + it \right)| dt \ll (\log T)^{\frac{1}{4} + m}
$$
valid for $H = T^{\frac{1}{4} + \epsilon}$ and any integer constant $m\geq 0$ and arbitrary real constant $\epsilon (0 < \epsilon \leq \frac{1}{4})$. This result depends on RH. We do not have the least idea for proving the  same result with $H = T^{\frac{1}{6} + \epsilon}$. It should  be mentioned that the proof of Ramachandra's result with $H = T^{\frac{1}{4} + \epsilon}$ above  (assuming RH) is incomplete. To complete the proof we have to use Gabriel's convexity theorem in the form Theorem \ref{c4:thm4.2.3}. Theorem \ref{c4:thm4.2.3} was suggested by the version of Gabriel's thoerem used by D.R. Heath-Brown in his paper cited above and Ramachandra regards the upper bound theorem with $H = T^{\frac{1}{4} + \epsilon}$ (on RH) as joint work with him.

It is of some interest to determine the constants in 
$$
(\log T)^{k^2} \ll \frac{1}{T} \int^T_0 |\zeta \left(\frac{1}{2} + it \right)|^{2k} dt \ll (\log T)^{k^2} 
$$ 
where $k = \frac{1}{j}$ ($j \geq 2$ an integer). An important result due to M. Jutila (see \cite{Jutila2}) says that the constants implied by $\gg$ and $\ll$ are independent of $k$. Using this A. Ivic\pageoriginale and A. Perelli proved that the mean-value in question (for real positive $k$) $\to 1$ if $k \leq (\psi (T) \log \log T)^{-\frac{1}{2}}$ where $\psi (T)$ is any function which $\to \infty$ as $T \to \infty$. (See \cite{Ivic and  Perelli1}).

The best result on lower bounds (for integral $k>0$) is due to K. Soundararajan and a particular case of it reads 
$$
\frac{1}{T} \int^T_0 |\zeta \left(\frac{1}{2} + it \right)|^6 dt \geq (24.59) + o(1) \sum\limits_{n\leq T} (d_3(n))^2 n^{-1}
$$
(information by private communication).


