\setcounter{chapter}{-1}
\chapter{Prerequisites}%%% chpater 0

In\pageoriginale this chapter we collect together some well known
results which will be used in the course of these lectures. 

\begin{enumerate}
\renewcommand{\labelenumi}{\bf\theenumi.}
\item Let $X$ be a connected, paracompact complex manifold of (complex)
dimension we denote by $\Omega$ The sheaf of germs of homomorphic
p-foxes end by $\wedge^{x,q}$ the sheaf of germs of $0^\infty$ forms
of type $(P,\mathcal{F})$. Further $\{overset{-}{\partial}$ will denote
the exterior differentiation with respect to $\overset{-}{\partial}$
Then we have one. 

\begin{prop*} 
  The sequence
  $$
  0\rightarrow \Omega^P \rightarrow A^{p,0}
  \overset{\bar{\partial}}{\rightarrow} A^{p,1} \ldots
  \overset{\bar{\partial}}{\rightarrow} A^{p,n} \to 0 
  $$
  is exact. Similarly if $K^{p,q}$ denotes the sheaf of germs of $(p,q)$
  currents, the sequence 
  $$
  0\rightarrow \Omega^p \rightarrow K^{p,c}
  \overset{\bar{\partial}}{\rightarrow}K^{p,1}\rightarrow \ldots
  \overset{\bar{\partial}}{\rightarrow} K^{p,n}\rightarrow 0 
  $$
  is exact.
\end{prop*}

We point out that the exactness of the first step of the second 
sequence:
$$
0\rightarrow \Omega^p\rightarrow K^{po}
$$
is equivalent to the fact that every $\overset{-}{\partial}$ closed
distribution is a holomorphic function\footnote{The idea of the proof
  is the following. Let $\alpha$ be any compactly supported function
  of class $C^k$ with $k > 0$. Since
  $\overset{-}{\partial}(T*\alpha)=0$, $T*\alpha$ is a homomorphic
  function hence it is $C^\infty$. Then $T$ itself  is a $C^\infty$ function 
  [30, Vol. II, Theorem XXI, 50]. But  $\overset{-}{\partial} T =
  0$. Then $T$ is holomorphic.}   

(For\pageoriginale a proof see  for instance \cite{key5},
Expos\'e  XVIII,  \cite{key22})  

The sheaves  $A^{p,q}$ being fine, the cohomology group
$H^q(X,\Omega^p)$ is canonically isomorphic to the qth cohomology
group of either one of the complexes 
{\fontsize{7}{9}\selectfont
\[
\xymatrix{0 \ar[r]& \ar[r] H^\circ (X, A^{p, o})  \ar[r]^{\bar{\partial}} &
  H^\circ (X, A ^{p, 1})  \ar[r]^{\bar{\partial}} & \ldots 
  \ar[r]^{\bar{\partial}}&  H^\circ (X, A^{p, n})  \ar[r]& 0\\
0 \ar[r] & \ar[r] H^\circ (X, K^{p, o})  \ar[r]^{\bar{\partial}} &
  H^\circ (X, K^{p, 1})  \ar[r]^{\bar{\partial}} & \ldots 
  \ar[r]^{\bar{\partial}} &  H^\circ (X, K^{p, n}) \ar[r]& 0.
}
\]}\relax

\item We need the following  theorem  due to Leray.

Let $\MG =\{U_i\}_{i\in I}$ be a locally finite open
covering  of a paracompact space $X$. Let $\mathcal{F}$ be a sheaf of
abelian groups on $X$ such  that for every $q>0$, $H^q(U_{i_o} \cap \ldots \cap
  U_{i_p}, \mathcal{F}) = 0$ for every set $(i_0,\ldots i_p)$ of
  $p$-elements in I. Then the canonical map  
$$
H^q(\MG,\mathcal{F})\rightarrow H^q(x,\mathcal{F})
$$
is an isomorphic for all $q\ge o$

For a proof see for instance \cite{key13} 209-210

\item The following result enables us to apply the above theorem to
locally free sheaves over homomorphic function on a complex manifold. 

Let $X$ be a (para compact) complex manifold of complex dimension
$n$. Given a vector bundle $E$ an $X$ and any covering $\MG =
\{U_i\}_{i\epsilon I }$ there is a refinement $\{V_j\}_{j\epsilon J}$
such that for $j_1,\ldots , j_k$ with $V_{j1}\cap\ldots \cap V_{jk}\neq
\phi , \; {\rm for }\; 0 \le p\le n$, the sequence
\begin{multline*}
  0\rightarrow H^0
  (V_{j1,\ldots jk}, \Omega^p \underset{\mathscr{O}}{\otimes} \underset{-}{E}
  \rightarrow  H^0(V_{j1 \ldots jk} \underset{-}{A}^{p1}
  \underset{\mathscr{O}}{\otimes} \underset{-}{E})\overset
  {\bar{\partial} \otimes 1 }{\longrightarrow \ldots}\\
  \ldots \overset{\bar{\partial}\mathscr{O} 1}{\rightarrow H^0} (V_{j1
  \ldots jk}, A^{pn} \underset{\mathscr{O}}{\otimes}
  \underset{-}{E})\rightarrow 0 
\end{multline*}
where\pageoriginale $\mathscr{O}$ is the structure sheaf, is exact, and
$V_{j1 \ldots jk}$ is the intersection $V_{j1} \cap \ldots \cap V_{jk}
and ~~\underbar{E}$ is the  
locally free $\mathscr{O} $ sheaf associated to $E$. (Since the sheaves, $
A^{pi}\underset{\mathscr{O}}{\otimes} E $ are fine sheaves, this implies  
that $H^{p}(V_{ji}\ldots _{jk'}\Omega^{p}\underset{\mathscr{O}}{\otimes}
\underbar{E}) = 0 $ 

\item Let $\Pi : E \rightarrow X $ be a holomorphic vector bundle on the
complex manifold $X$. Let  $\MG = (U_{i})_{i \in I}$ be a covering of $X$
and $e_{ij} : U_{i}\cap U_{j}\rightarrow G = GL(n, \mathbb{C})$ be transition
functions with respect to  
$\MG'$ such that $\{\MG,e_{ij}\} $ define $E$. This
means that we are given isomorphisms 
$$
 \varphi_{i} : E/U_{i}\rightarrow U_{i} \times \mathbb{C}^n, 
$$ 
for $i \in I $ such that $ e_{ij} $ are the maps defined by
$$ 
(\varphi_{j}\circ \varphi_{i}^{-1}) (x,v) = (x, e_{ij}(x)(v)) 
$$
for $ x \in U_{i}\cap U_{j}$. In particular, we have for $i, j, k \in I$
$$ 
e_{ik}(x) = e_{ij}(x) e_{jk}(x) 
$$
provided that $ x \in U_{i}\cap U_{j}\cap U_{k}$.

For a vector bundle $\pi : E \rightarrow X$, we denote, as before by
$\underbar{E}$ or $\Omega(E)$ the sheaf of germs of holomorphic
section of $E$. $\Omega(E)$ 
is a locally free sheaf over the sheaf of germs of holomorphic
functions  on $X$ (we denote this latter sheaf by $\mathscr{O} $).  
Then, in the above notation,
$$ 
\Omega (E) \bigg|_{U_{i}}\simeq \mathscr{O}^{n} 
$$

We fix moreover the following notation:
\begin{equation*}
  \begin{aligned}
    \Omega^{p}(E) &= \underbar{E} \otimes_{\mathscr{O}} \Omega^{p},~~
    \text{ in particular } ~~\Omega^{\circ}(E) = \Omega(E) ;\\ 
    \underbar {A}^{pq}(E) & = \underbar{E} \otimes_{\mathscr{O}} A^{pq};\\
    \underbar {K}^{pq}(E) &= \underbar{E} \otimes_{\mathscr{O}} K^{pq}.
  \end{aligned}
\end{equation*}

$\Omega^{p}(E)$\pageoriginale (\resp $\underbar{A}^{pq}(E)$, \resp
$\underbar{K}^{pq}(E)$) 
is the sheaf of germs of holomorphic $E$-valued $p$-forms
(\resp differentiable $(p, q)$ forms with values in $E$, 
\resp $(p, q)$ currents with values in $E$ ;

$\Gamma(X,\underbar A^{pq}(E))=\text{Global} (p,q) C^{\infty}$
forms with values in $E$; 

$\Gamma(X,\underbar K^{pq}(E)) = \text{Global} (P,q)$ 
  currents with values in $E$.  

There is a one-one correspondence between global sections of
$\underbar{A}^{pq}(E)$ and collections $(\varphi_{i})_{i \in I}$ where
each $\varphi_{i}$ is a vector valued form,
$$
\varphi_{i}=
\begin{pmatrix}
  {\varphi_{i}'}\\ 
  {\vdots}\\
  {\varphi_{i}^{m}}
\end{pmatrix}
$$
each $\varphi_{i}^{k}$ being a scalar $C^{\infty}$ $(p,q)$ forms on
$U_{i}$ such that for $ x \in U_{i}\cap U_{j}$ we have
$\varphi_{i}(x)=e_{ij}(x) \varphi_{j}(x)$. 
One can set up a similar one-one correspondence between sections of
$K^{pq}(E)$ and families $(\varphi_{i})_{i \in I}$ of currents, each  
$\{\varphi_{i}^{k}\}_{i}$ being defined on $U_{i}$, satisfying 
$e_{ij} \varphi_{j} = \varphi_{i}$ on $U_{i}\cap U_{j}$. 

We denote by $\bar{\partial}$ the operators
\begin{align*}
  1 \otimes \bar{\partial}: & \underbar{E}
  \underset{\mathscr{O}}{\otimes}A^{pq} \longrightarrow
  \underbar{E}\otimes A^{p, q+ 1}\\
  \text{and}  \hspace{2cm} 1 \otimes \bar{\partial}: & \underbar{E}
  \underset{\mathscr{O}}{\otimes}K^{pq}  \longrightarrow
  \underbar{E}\otimes K^{p, q+1}.
\end{align*}

  Then the sequences
$$
\displaylines{\hfill  
  0 \rightarrow \Omega^{p}(E)\rightarrow
  \underbar{A}^{p,o}(E)\overset{\bar
    \partial}{\rightarrow}\ldots\overset{\bar
    \partial}{\rightarrow}\underbar{A}^{p,n}(E)\rightarrow 0\hfill \cr 
  \text{and} \hfill  0 \rightarrow \Omega^{p}(E)\rightarrow
  \underbar{K}^{p,o}(E)\overset{\bar \partial}{\rightarrow}\ldots 
  \overset{\bar \partial}{\rightarrow}\underbar{K}^{p,n}(E)\rightarrow
  0 \phantom{and}\hfill } 
$$
are exact. Moreover the sheaves $\underbar{A}^{pq}(E)$ and
$\underbar{K}^{pq}(E)$ are fine. 

Hence\pageoriginale if $\Phi$ denotes either the family of all closed
subsets of $X$ or the family of all compact subsets of $X$, (more
generally, any paracompactifying family) and we set,  
{\fontsize{9}{11}\selectfont
$$
\displaylines{\hfill 
  \Gamma_{\Phi}(X, \underbar{A}^{pq}(E)) = \left\{\sigma|\sigma \text{
    a section of }~ A^{pq}(E) ~\text{ over } X, \Supp \sigma \in \Phi \right\},
  \hfill \cr
  \text{then}\hfill 
  H_{\Phi}^{q}(X, \Omega^{p}(E))\simeq \left\{\varphi|\varphi \in
  \Gamma_{\Phi}(X,A^{p,q}(E));\bar{\partial}\varphi=0\right\} |
  \left\{\bar{\partial} 
  \Gamma_{\Phi}(X,A^{p,q-1}(E))\right\}\hfill }
$$}\relax
and similarly with the obvious notation, 
\begin{align*}
&H_{\Phi}^{q}(X, \Omega^{p}(E))\simeq \\
&\quad\left\{\varphi|\varphi \in
\Gamma_{\Phi}(X,\underbar{K}^{p,q}(E));\bar{\partial}\varphi=0\right\} |
\left\{\bar{\partial} \Gamma_{\Phi}(X,K^{p,q-1}(E))\right\}
\end{align*}
\end{enumerate}
