\chapter{W-ellipticity on Riemannian manifolds}\label{chap2}%%% 2


\setcounter{section}{4}
\section{W-ellipticity on Riemannian manifolds}%%% 5

We\pageoriginale carry over the results proved for $
\overline{\partial}$ on a complex manifold to the operator
$d$-exterior differentiation of 
ordinary forms. In the sequel until further notice we consider only
real valued forms. Through all this chapter $X$ will be a connected
orientable Riemannian manifold. We choose an orientation for $X$, and we
denote by $C^q = C^q(X)$ (\resp $\mathscr{D}^q(X) = \mathscr{D}^q)$ the
space of $C^{\infty}$ $q$-forms (\resp $C^{\infty}$ forms with compact
support). We defined in Chapter \ref{chap1} the operator  
$$
* : C^q \rightarrow C^{n-q}
$$
and the scalar product,
$$
(\varphi, \psi) = \int A(\varphi, \psi)dX = \int\limits_{X} \varphi
\wedge * \psi 
$$
of two forms $\varphi$, $\psi$ in $\mathscr{D}^{q}(X)$.

We define the operator
$$ 
\delta : C^{q} \longrightarrow C^{q-1}
$$
as the linear differential operator
$$ 
\varphi \rightsquigarrow (-1)^{q}*^{-1} d(* \varphi)
$$
$(\varphi \in C^{q}(X))$. If $\varphi \in C^{q}(X), \psi \in
C^{q+1}(X)$, and if $\supp \varphi \cap \supp \psi$ is compact, then
$(d\varphi, \psi) = (\varphi, \delta\psi )$. We also introduce the
Laplace operator 
$$ 
\triangle = d \delta + \delta d ;
$$
$\triangle$ is a strongly elliptic operator. 

We\pageoriginale have then
$$ 
(\triangle \varphi,\psi )=(d\varphi,d\psi )+(\delta\varphi,\delta \psi)
$$
for $\varphi$, $\psi \in \mathscr{D}^{q}(X)$.

On the space $\mathscr{D}^{q}(X)$, we define another scalar product:
for $\varphi$, $\psi \in \mathscr{D}^q (X)$,
$$
a(\varphi,\psi)=(\varphi,\psi)+ 
 (d\varphi,  d\psi) + (\delta \varphi,\delta \psi).
$$
  Then  $N(\varphi) = a (\varphi,\varphi)^{\frac{1}{2} }$ defines a
  norm on $\mathscr{D}^q(X)$. We denote the completion of
  $\mathscr{D}^q(X)$ under $N$ by $W^q$. If $\mathcal{L}^q$ denotes
  the completion of 
  $\mathscr{D}^q(X)$ with respect to the norm, $\parallel \varphi
  \parallel^2=(\varphi,\varphi)$,  
  then the identity extends canonically to a continuous map 
  $$ 
  i:W^q\longrightarrow \mathcal{L}^q. 
  $$
   
  We have then the following result.

\setcounter{prop}{0}
\begin{prop}%% 2.1
  $i$ is an injection.

  The proof is analogous to that of proposition
  \ref{chap1:prop1.1}. As in theorem  \ref{chap1:thm1.1}  
  (Chapter \ref{chap1}), we have the following characterisation of $W^q$ (rather
  $i(W^q$))   when the Riemannian metric on $X$ is complete.
\end{prop}

\setcounter{theorem}{0}
\begin{theorem}\label{chap2:thm2.1}%% 2.1
  If the Riemannian metric on $X$ is complete, then
  $$
 W^q(X) = \{\varphi \mid \varphi \in 
  \mathcal{L}^q(X), d \varphi \in \mathcal{L}^{q+1} (X), \;
  \delta \varphi \in  \mathcal{L}^{q-1}(X)\}. 
$$

  Moreover $\mathscr{D}^q (X)$ is dense in the space $\{\varphi \mid
  \varphi \in \mathcal{L}^q, d \varphi \in \mathcal{L}^{q-1}\}$, 
  with respect to the norm $(\parallel \varphi \parallel^2 +\parallel
  d \varphi \parallel^2)^{\frac{1}{2}}$, and in the space  
  $\{\varphi \mid \varphi \in \mathcal{L}^q, \; \delta \varphi \in
  \mathcal{L}^{q+1}\}$  with respect to the norm $(\parallel  
  \varphi \parallel^2+\parallel \delta \varphi \parallel
  ^2)^{\frac{1}{2}}$. 
  
  Once again, we omit the proof which is analogous to that of Theorem
  \ref{chap1:thm1.1} (Chapter \ref{chap1}). We state also, without
  proof, the  analogue of  Corollary \ref{chap1:coro1} to Theorem
  \ref{chap1:thm1.2} of Chapter \ref{chap1}.   
\end{theorem}

\begin{prop}\label{chap2:prop2.2}%% 2.2
  For\pageoriginale $\varphi \in C^q$ and for any $\sigma>0$
  $$ 
  \parallel d\varphi \parallel^2+\parallel \delta \varphi \parallel ^2
  \leqslant \sigma \parallel \Delta \varphi  
  \parallel^2+\frac{1}{\sigma}\parallel \varphi \parallel ^2.
  $$
\end{prop}

\setcounter{definition}{0}
\begin{definition}%% 2.1
  The Riemannian manifold $X$ is $W^q$-elliptic if there is a constant
  $C>0$ such that     
  $$ 
  \parallel \varphi \parallel^2 \leqslant C \bigg\{\parallel d \varphi
  \parallel^2 + \parallel \delta \varphi \parallel^2 \bigg\}.
  $$
We shall call such a constant $C$ a $W^q$-ellipticity constant. 
  
  In  further analogy to Chapter \ref{chap1}, we have finally the following 
\end{definition}

\begin{theorem}\label{chap2:thm2.2}%%% 2.2
  If $X$ is $W^q$  elliptic, then for $f \in \mathcal{L}^q,$ there is
  a unique $x \in W^q$  
  such tht $f=\triangle x$ (in the sense of distributions). Further,
  \begin{gather*}
  \parallel x \parallel \leqslant c \parallel f \parallel\\ 
  \parallel dx \parallel^2 + \parallel \delta x \parallel ^2 \leqslant
  C \parallel f \parallel ^2,
  \end{gather*}
  $C$ being a $W^q -$ ellipticity constant. If f is $C^{\infty},$ then
  $x \in C^q\cap W^q$.  If the metric on X is complete and $\parallel
  df \parallel < \infty$, then, for $\sigma > 0$, 
  $$
  \parallel \delta dx \parallel ^2 \le \frac{1}{\sigma}\parallel df
  \parallel ^2 + \sigma \parallel dx \parallel ^2.
  $$  
  In particular, if df=0 , f=d$\delta x $ and dx=0.
  
  Except for obvious modifications, the proof is contained in
  Propositions \ref{chap1:prop1.2}  and \ref{chap1:prop1.3} and
  Theorem \ref{chap1:thm1.3} (Chapter \ref{chap1}).   
  
  Before we give criteria for $W^{pq}$-ellipticity, on a complex
  manifold, we give sufficient conditions for 
  $W^q$-ellipticity, on an orientable Riemannian manifold $X$. In order
  to do this we first write  down\pageoriginale explicitly in a
  coordinate open set the effect of the Laplacian $\Delta$ on   
  a $C^{\infty}-p-$form. Let $(x, \ldots x^n)$ be a coordinate system
  on an open set $U$ in $X$. Let the Riemannian metric be
  $$ 
  \sum g_{ij}\; dx^i dx^j
  $$
  in this open set. We denote $\triangledown$  the covariant
  derivation with respect to the Riemannian metric. For a tensor
  $\varphi,\triangledown_{\alpha}\varphi$ denotes the tensor
  $$ 
  (\triangledown \varphi)(\frac{\partial}{\partial x^{\alpha}}) .
  $$
  $(\triangledown \varphi$ is regarded  as a 1-form with values in the
  tensor bundle). In the local-coordinate system,  
  a form $\varphi \in C^q(X)$ may be written in the form. 
  $$ 
  \varphi=\underset{1}\sum \frac{1}{q!} \varphi_I dx^I
  $$
  where $I$ runs over all $q$-tuples $(i_1,\dots, i_q)$ of integers with
  $0<i_j \leqslant n$ and  if $I=(i_1,\dots,i_q)$ is 
  a permutation $(\sigma(j_i),\ldots, \sigma(j_q))$ of $I'=(j_1,\dots,
  j_q)$, then 
  $$ 
  \varphi_I=\varepsilon_{\sigma} \varphi_{I'}.
  $$
  (In particular, if $I$ has repeated indices, $\varphi_I=0)$. We have 
  $$ 
  A (\varphi, \psi)=\frac{1}{q!} \varphi _I \psi^I (\varphi, \psi \in 
  C^q)
  $$
  For $\varphi \in C^q $ and a $q$-tuple $I =(i_1,\dots,i_q)$,
  \begin{align*}
      d \varphi _{iI} & =\frac{\partial \varphi _I} {\partial x^i}
      + \sum\limits_r  (-1)^r \frac{\partial \varphi_{i}}{\partial
        x^{i_r}} i_1, \ldots, i_r,\ldots i_q\\ 
      &=\triangledown_{i} \varphi_{I}  + \sum\limits_r (-1)^r
      \triangledown_{ir} \varphi_{ii_1},\ldots i_r \ldots ,i_q
      \tag{2.1}\label{eq2.1} 
  \end{align*}
  as is\pageoriginale seen by a direct computation. Similarly we have  
  \begin{equation*}
    \delta \varphi_J=-\Sigma  \triangledown_i \varphi^i_J
    \tag{2.2}\label{eq2.2}  
  \end{equation*}
  where $J$ is a $(q-1)$-tuple and $\varphi^i$ is the $(q-1)$-form defined by   
  $$
  \varphi^i_J= \sum g^{ij} \varphi_{iJ} 
  $$
  for any $(q-1)$-tuple $J=(j_1,... j_{q-l})$.

  From (\ref{eq2.1}) and (\ref{eq2.2}), it follows that
  \begin{align*}
      &\delta  d \varphi_I = -\sum^n_{i=1} \triangledown_i(d \varphi)^i_I \\
      &=- \sum^n_{i=1} \triangledown_i \triangledown^i
      \varphi_I-\sum^q_{i=1}(-1)^r
      \triangledown_i(\triangledown_{ir}\varphi^i)
      i_i...\hat{i}_r...i_q \tag{2.3}\label{eq2.3} 
  \end{align*}
  (where $\triangledown^i  =\sum_j g^{ij}\triangledown_j)$ \: and
  \begin{align*}
    d \delta \varphi_I & = \overset q {\underset {r=I} {\sum }}
    \overset n {\underset {i=I} {\sum}}(-1)^r \triangledown_{{i}_{r}}
    \triangledown_i \varphi^i _{i_1...\hat{i}_r...i_q} \tag{2.4}\label{eq2.4} \\ 
    \text{Hence} \hspace{2cm}(\triangledown \delta)_I & =-\sum \triangledown_i
    \triangledown^i \varphi_I +(\kappa \varphi)_I \hspace{2cm}
    \tag{2.5}\label{eq2.5} 
  \end{align*}

 In the equation (\ref{eq2.5})
\begin{align*} 
 (\kappa \varphi)_I & =\sum_i \sum_r (-1)^r
   (\triangledown_{{i}_{r}} \triangledown_i -\triangledown_i
   \triangledown_{{i}_{r}})\varphi^i_{i_1...\hat{i} _r...i_q}\\ 
   & =\sum^q_{r=1} \sum_j(-1)^{r-1}R^j_{{i}_{r}} \varphi_
       {ji_1...\hat{i}_r...i_q} + \overset{q} {\underset
       {r,s=1} {\sum}}(-1)^{r+s} R^{i_j}_{i_r i_s
       }\varphi_{i.j.i_..\hat{i}_r.\hat{i}_ s}
\end{align*}

 In the special case when $\varphi$ is a function,
 \begin{equation*}
   \triangledown \varphi = - \triangledown_i \triangledown^i \varphi
   \tag{2.6}\label{eq2.6}  
 \end{equation*}
\end{theorem}

\setcounter{lemma}{0}
\begin{lemma}\label{chap2:lem2.1}%%% 2.1
  For\pageoriginale every $\varphi \in \mathscr{D}^q$ we have
  \begin{equation*}
    \parallel \triangledown \varphi \parallel^2 + (K \varphi,\varphi)=
    \parallel d \varphi \parallel^2 + \parallel \delta \varphi
    \parallel^2  \tag{2.7}\label{eq2.7}  
  \end{equation*}
\end{lemma}

\begin{proof}
  From (\ref{eq2.6}) above, we have setting $|\varphi |^2 =
  A(\varphi,\varphi)=\frac{1}{\varphi}\varphi_I \varphi^I$, 
  \begin{align*}
    \Delta(| \varphi |^2) & =-\sum \triangledown_i \triangledown^i
    |\varphi|^2 = - \sum \frac {1}{q!} \triangledown_i \triangledown^i
    \varphi_I \varphi^I\\ 
    &=-\frac{2}{q!}\sum \triangledown_i (\triangledown^i \varphi_I
    .\varphi^I)\\ 
    &=-\frac{-2}{q!}\sum \triangledown_i \triangledown^i \varphi_I
    .\varphi^I-2|\triangledown \varphi|^2\\ 
    &=\sum_I \bigg \{\frac {2}{q!} \triangledown \varphi_I \varphi^I
    - \frac{2}{q!}(K \varphi)_I \varphi^I\bigg \}-2|\triangledown
    \varphi|^2 
  \end{align*}
  That is
  $$ 
  \triangledown|\varphi|^2= 2 A(\triangledown \varphi,\varphi)-2A(K
  \varphi,\varphi)-2A (\triangledown \varphi, \triangledown
  \varphi)...
  $$ 

  If $\varphi$ has compact support,
  $$
  \int \bigtriangledown | \varphi|^2 dx=\int d \delta|\varphi|^2  dX=0
  $$ 
  by Stokes formula.

On the other hand,
  $$ 
  \int A (\triangledown \varphi, \varphi) dX = \parallel d \varphi
  \parallel^2~+ ~\parallel \delta \varphi \parallel^2.
  $$

  Hence
  $$ 
  (K \varphi,\varphi)+\parallel \triangledown \varphi \parallel^2 =
  \parallel d \varphi \parallel^2 +\parallel \delta \varphi
  \parallel^2:(\varphi \in \mathscr{D}^q(X)).
  $$

  If there exists a positive constant C such that
  
  $ A (K \varphi,\varphi)\ge C|\varphi|^2$ for every $\varphi \in C^q
  ~~ $ and at each point of $X$, then  
  $$ 
  ||\varphi||^2 \leq C' (K \varphi, \varphi) \leq C' \left\{ ||d
  \varphi||^2 + ||\delta \varphi||^2 \right\}
  $$
  where $C' = \dfrac{1}{C}$. Hence the following.
\end{proof}

\begin{lemma}\label{chap2:lem2.2}%%% 2.2
  If there\pageoriginale is a positive constant $C$ such  that
  $$ 
  A (K \varphi~,\varphi)~ \ge C~A(\varphi,\varphi)
  $$
  for every $\varphi \in C^q$ and at each point of $X$, then the
  Riemannian manifold $X$ is $W^q$-elliptic. 
\end{lemma}


\section{A maximum principle}%%%% 6

In the  sequel we consider weaker condition on the expression $A
(\kappa \varphi~,\varphi)$. We shall deal with the case where the
quadratic from $A (\kappa \varphi~,\varphi)$ is positive semi-
definite outside of a compact. 

From now on we shall always assume $X$ to be oriented and 
connected, and the Riemannian metric of $X$ to be complete. 

\begin{lemma}\label{chap2:lem2.3}%%%2.3
  Assume $A (\kappa \varphi~,\varphi)\ge 0$ (i.e. $A (\kappa
  \varphi,\varphi)$ positive semi-definite), outside a compact  set $K$
  in $X$. Then for  $\varphi \in W^q$, $\parallel \triangledown
  \varphi \parallel^2 <\infty $ \;\; and (\ref{eq2.7}) holds. 
\end{lemma}

\begin{proof}
  By Lemma \ref{chap2:lem2.1} we have for $\varphi \in \mathcal{D}^q(X),$
  $$
  \parallel \triangledown \varphi \parallel^2+ (\kappa
  \varphi,\varphi)= (d \varphi~,d \varphi)~+(\delta \varphi, \delta
  \varphi).
  $$ 

  Since $\mathcal{K}$ is compact, there is $C \ge 0$ such that $(\mathcal{K}
  \varphi, \varphi)> -CC |\varphi|^{-2}$ for every $\varphi \epsilon
  C^q$ and each point of $\mathcal{K}$. Hence, 
  $$
  \parallel \triangledown \varphi \parallel^2 +\int_X A(k \varphi ,
  \varphi)\le C \parallel \varphi \parallel_k^2  + \parallel d \varphi
  \parallel^2 + \parallel \delta \varphi \parallel^2.
  $$ 

  Since $A(\kappa \varphi,\varphi) \ge 0$ on $X-\mathcal{K}$ we have
  for $\varphi  \in \mathcal{D}^q$, 
  $$
  \parallel \triangledown \varphi \parallel^2 \le C \left\{ \parallel
  \varphi \parallel^2~+ \parallel d \varphi \parallel^2 +\parallel
  \delta \varphi \parallel^2 \right \}
  $$
  The lemma follows now from the fact that $\mathcal{D}^q$ is dense in $W^q$.
  (Theorem \ref{chap2:thm2.1}.)
\end{proof}

\begin{remark*}
  We have\pageoriginale proved more: there is a $C > 0$ such that 
  \begin{equation*} 
    \Arrowvert \triangledown  \varphi \Arrowvert^2 \leqslant C \left\{
    \Arrowvert \varphi \Arrowvert~~ ^2_K ~~ + \Arrowvert d \varphi
    \Arrowvert^2 ~ + \Arrowvert \delta \varphi \Arrowvert ^2 \right\} 
  \end{equation*}

  Let $X$ be a manifold as in the above lemma.
\end{remark*}

Let $ \varphi \in  C^q$. Identity (2.8) shows that $ \triangle
\arrowvert \varphi \arrowvert ^2 \leqslant 0 $ at 
each point of the set $Y$ where $\triangle \varphi = 0$ and $A(K
\varphi, \varphi) \geqslant 0$ 

Applying a classical lemma of $E$. Hopf (\cite{key36}, 26-30) we see that
$\arrowvert \varphi \arrowvert^2$  
cannot have a relative maximum at any point of $Y$. Thus, if $X$ is
compact, $\arrowvert \varphi \arrowvert ^2$ takes its, maximum in the
set $\Supp f \cup K$. 

If $X$ is not compact we cannot draw the same conclusion. However the
following proposition will provide an estimate of $|\varphi|^2$
on $X$ in terms of $\Sup |\varphi|^2$ on $\Supp f \cup \mathcal{K}$.

\begin{prop}\label{chap2:prop2.3}%%% 2.3
  Let $X$ be a connected oriented and complete Riemannian manifold. 
  Assume give a compact set $\mathcal {K}$ on $X$ such that for $X
  \notin \mathcal{K},  A(\mathcal{K} \varphi, \varphi)(x) \ge 0$ for
  every $\varphi \in C ^q$. Suppose that $\varphi \in C ^q \cap
  \mathcal{L}^q$ is such that $\triangle \varphi~= ~ f \in
  \mathcal{L}^q$. Then $\varphi \in W^q $ and $\sup\limits_{x\in X}
  \arrowvert \varphi (x) \arrowvert^2   \leqslant C_0$ where    
  $$
  C_0 ~~=~~^{\Sup}_{x \in \supp f \cup \mathcal{K}} \arrowvert \varphi
  (x) \arrowvert^2 .
  $$ 
\end{prop}

For the proof of this proposition we utilise the following result

\begin{lemma}[Gaffney \cite{key12}]\label{chap2:lem2.4}%%% 2.4
  Let $\mathcal{K}$ be an oriented complete Riemannian  
  manifold and $\omega$ a $C^\infty$ 1-form. Then if 
    $\int_X \arrowvert \omega \arrowvert dX < \infty$ and
    $\int_X \arrowvert \delta \omega \arrowvert dX < \infty$,
  we have  
  $$
  \int_X \delta \omega dX = 0.
  $$
\end{lemma}

\begin{proof}
  Let\pageoriginale $\lambda$ be a real value $C^\infty$ function on
  $\varmathbb{R}$ such that 
  $\lambda (t) = 1$ for $t \leqslant 0$ and $\lambda (t) = 0$ for
  $t\geqslant  1$. Let $p_0 \in X$ be a fixed point and $\rho = \rho
  (x)$ denote the distance function (from $p_0$). Then $\rho(x)$ is
  locally Lipchitz as was remarked earlier 
  (Lemma \ref{chap1:alphlemmaB} of Chapter \ref{chap1}). Let $0 < r <
  \mathcal{R}$ and let us consider the form 
  $$
  \lambda \left\{ \frac{\rho(x) - r}{\mathcal{R} - r} \right\} \cdot \omega 
  $$

  This form is locally Lipchitz and has compact support in the ball 
  $$ 
  \{ x \arrowvert \rho (x) \leqslant R \}.
  $$

  Hence by Stoke's formula (which is applicable to Lipchitz continuous
  form), 
  $$
  \int_X d* \lambda \bigg( \frac{\rho(x) - r}{R-r}\bigg) \cdot \omega
  = 0; 
  $$
  
  We have,
  $$
  \int_X \frac{1}{R-r} \lambda \bigg( \frac {\rho (x) - r}{-r}\bigg) \cdot d
  \rho \Lambda * \omega + \int_X \lambda \bigg( \frac{\rho (x) -
    r}{R-r}\bigg) \cdot d * \omega = 0.
  $$

  Now set $R = 2r$. Then
  $$
  \int_X \frac{1}{r} \lambda' \bigg(\frac{\rho(x) -
    r}{r}\bigg) d \rho \Lambda * \omega + \int_X \lambda \bigg( \frac
      {\rho (x) - r}{r}\bigg) \cdot d * \omega = 0
  $$

  Now $\arrowvert d \rho \arrowvert^2 \leqslant $ (lemma
  \ref{chap1:alphlemmaB} of Chapter \ref{chap1}). 
      
  Hence,\pageoriginale since $* \omega$ is integrable, we have 
  $$
  \int_X \arrowvert d \rho \Lambda * \omega \arrowvert dX < \infty. 
  $$
\end{proof}

  Hence, using the fact that $\lambda '$ is bounded, we see that
  $$
  \frac{1}{r} \int_X \lambda ' \left( \frac {\rho (x)-r}{r}\right) d
  \rho \Lambda * \omega \rightarrow 0~\text{ as }~ \rightarrow \infty.
  $$

  It follows that 
  $$
   \overset{Lt}{r \to \infty}~~ \int_X \lambda  \left( \frac
  {\rho (x)-r}{r}\right) d * \omega = 0
  $$

On the other hand, since $\int \arrowvert d * \omega \arrowvert d X <
\infty$ and $\lambda \left(\frac{\rho (x)-r}{r}\right)$ 
is bounded and tends to 1 as $r \rightarrow \infty$,
$$
r \overset{Lt} \rightarrow \infty~~ \int \lambda  \left(\frac{\rho
  (X)-r}{r}\right) d * \omega = \int d * \omega.
$$
This proves the lemma.

\medskip
\noindent{\textbf{Proof (of Proposition \ref{chap2:prop2.3})}} Let
$C_0$ and $C_1$ be 
two positive constants, $0< C_0 < C_1$, and let $\lambda :\mathbb{R}
\rightarrow \mathbb{R}$ be a  $C^\infty$ function such that 
\begin{align*}
  & \lambda (t) \geqslant 0, \quad \lambda ' (t) \geqslant
  0,~~~\lambda''(t) \geqslant 0,\\ 
  & \lambda (t) = 0, \quad {\rm for \;} t \leqslant C_0\\
  & \lambda' (t) > 0, \quad {\rm for \;} t > C_0\\
  & \lambda''(t) =0, \quad {\rm for}~ t < C_0 ~~ \text{and for}~ t>C_1,
\end{align*}
and such that $\lambda '(t)$ be bounded. Since $\lambda''(t)$ has
compact support, it is bounded. We have then for $\varphi$ as in the
proposition, 
\begin{align*}
  \Delta \lambda (\mid \varphi \mid ^2) &= \delta d (\lambda ( \mid
  \varphi \mid ^2))\\ 
  & = \delta \left\{ \lambda '( \mid \varphi \mid ^2).d \mid \varphi
  \mid ^2 \right\}  \\ 
  & = - * d * \left\{ \lambda '( \mid \varphi \mid ^2)d \mid \varphi
  \mid ^2 \right\} \\ 
  & = - * d  \left\{ \lambda '( \mid \varphi \mid ^2) * d \mid \varphi
  \mid ^2 \right\} \\  
  & = - *  \left\{ d \lambda '( \mid \varphi \mid ^2) \wedge * d \mid
  \varphi \mid ^2 + \lambda ' (\mid \varphi \mid ^2 )d * d \mid
  \varphi \mid ^2 \right\} \\ 
  & = - * \left\{ \lambda'' ( \mid \varphi \mid ^2) d  \mid \varphi \mid ^2
  \wedge * d \mid \varphi \mid ^2  + \lambda ' \mid \varphi \mid ^2 d
  * d \mid \varphi \mid ^2 \right\} \\ 
  & =  - \lambda'' \mid \varphi \mid ^2 \mid d \mid \varphi \mid ^2
  \mid ^2+~ \lambda'\mid \varphi \mid ^2 . \Delta (\mid \varphi \mid
  ^2). 
\end{align*}

In the\pageoriginale view of (2.8),

{\fontsize{10}{12}\selectfont
$\triangle
(\lambda|\varphi|^2)=-\lambda''(|\varphi|^2)|d|\varphi|^2|^2 +2
\lambda '|\varphi|^2 \{A(\triangle \varphi,\varphi)-A(\varmathbb{K}
\varphi, \varphi )-A(\triangledown \varphi, \triangledown \varphi)\}$}\relax 

On the other hand by proposition \ref{chap2:prop2.2} we have for
$\varphi \in C^q$, 
\begin{equation*} 
  ||\partial \varphi||^2+||\delta \varphi||^2 \leqslant||\Delta
  \varphi||^2+||\varphi||^2 
\end{equation*}

In particular, if $\Delta \varphi \in \mathcal{L}^q$ and $\varphi \in
C^q \cap \mathcal{L}^q $ then $||d \varphi ||^2 < \infty$ as also
$||\delta \varphi ||^2< \infty$. In the view of theorem \ref{chap2:thm2.1} this
proves that $\varphi \in W^q$. 

Now
$$
\int_x \Delta \lambda (|\varphi|^2)dX=\int_X*d*d \lambda (|\varphi|^2)
$$
We shall show that these integrals vanish.

Using Schwartz inequality, one checks that there exists a positive
constant $C_2$ such that 
\begin{equation*}
  |d|\varphi|^2|\leqslant  C_2|\varphi||\triangledown \varphi| \;
  \text{ at each point of $X$.}    \tag{2.10}\label{eq2.10} 
\end{equation*}

Hence, since $|\lambda'(t)|$ is bounded it follows that
$$
\int_X|d \lambda(|\varphi|^2)|dX<\infty. 
$$
We will prove now that
$$
\int_X|\Delta \lambda(|\varphi |^2)|dX < \infty.
$$

Letting $C_3= \Sup~\lambda''(t)$, we have, by (\ref{eq2.10}),
\begin{multline*}
  \int_X \lambda''(|\varphi|^2)|d|\varphi|^2|^2dX=\int
  \substack{\lambda:(|\varphi|^2_2)|d|\varphi|^2|^2 \\\\ C_\circ
    <|\varphi|^2 < C_1}  dX \leqslant C_3 \int \substack
           {|d|\varphi|^2|^2 \\\\C_\circ <|\varphi|^2 <
             C_1}|d|\varphi|^2|^2 dX \\
  \leqslant C_2Cc_3  \int \substack{|\varphi|^2 |\triangledown
    \varphi|^2 \\\\ C_\circ <|\varphi|^2 < C_1} dX \leqslant C_1C_2C_3
  \int_X |\triangledown|^2 dX=C_1C_2C_3|| \triangledown \varphi||^2
\end{multline*}

Since\pageoriginale $\varphi \in W^{q}$, then by Lemma
\ref{chap2:lem2.3}  $||\triangledown 
\varphi ||^2 < \infty$, and therefore 
$$
\int_X \lambda''(|\varphi|^2)|d|\varphi|^2|^2 dX< \infty
$$
Since $K$ is compact, there is a constant $C_4 >0$ such that
$$
|A (\mathcal{K} u,u)|\leqslant C_4 A(u,u)
$$
for any $u\in C^q$ and at each point of $\mathcal{K}$. Then we have
\begin{align*}
  \int_X|A(K \varphi, \varphi)|dX & = \int_{X-\mathcal{K}}A(\mathcal{K} \varphi,
  \varphi)dX + \int_\mathcal{K} |A(\mathcal{K} \varphi, \varphi)|dX\\ 
  & \leqslant \int_X A(\mathcal{K} \varphi, \varphi)dX +2
  \int_\mathcal{K} |A(\mathcal{K} \varphi,   \varphi)|dX\\ 
  & \leqslant~(\mathcal{K} \varphi, \varphi)+2C||\varphi||^2 \leqslant
  2C||\varphi ||^2+||d \varphi||^2 + ||\delta \varphi||^2 <\infty. 
\end{align*}
Finally we have
$$
\int_X |A(\triangle \varphi, \varphi)|dX \le \frac{1}{2} \int_X
A(\triangle \varphi,\triangle \varphi)dX +\frac{1}{2} \int_X
A(\varphi, \varphi)dX < \infty.
$$

Then, it follows from (\ref{eq2.9}) and from the fact that $|\lambda'|$ and
$|\lambda''|$ are bounded, that
$$
\int_X |\triangle \lambda (|\varphi |^2)|dX < \infty. 
$$

Hence, by Lemma  \ref{chap2:lem2.4},
\begin{equation*}
  \int_X \triangle \lambda(|\varphi|^2)dX=0 , i.e.~ by  \tag{2.9}\label{eq2.9} 
\end{equation*}
\begin{multline*}
  \int_X \lambda''(| \varphi|^2)^2 (d| \varphi |^2)^2 dX +2 \int_X
  \lambda '(|\varphi |^2). A(\mathcal{K}\varphi , \varphi)dX 
  +2\\ \int_X \lambda'(|\varphi|^2)|\triangledown \varphi|^2 dX 
  =\int_X 2 \lambda' (|\varphi|^2)A(\triangle \varphi,
  \varphi)dX.\tag{2.11}\label{eq2.11}  
\end{multline*}

Let\pageoriginale us consider now $\sup |\varphi|$ on Supp $f\cup
k$. If it is not 
finite there is nothing to prove. If it is finite, we set
$C_0=\Sup  |\varphi|^2$ on $K \bigcup$ Support $f$. In view of our
choice of the function $\lambda$ all the terms on the left hand side
of (\ref{eq2.11}) are non negative. We obtain 
\begin{align*}
  0 \leqslant 2 \int_X \lambda' (|\varphi | ^2)|\triangledown \varphi
  |^2 d X & \leqslant \int_X 2 \lambda ' (| \varphi |^2)A (\triangle
  \varphi, \varphi)\\ 
  & =2 \int_X \lambda' (|\varphi |^2).A(f, \varphi).
\end{align*}

Once again since $| \varphi |^2 \leqslant C_0$ on support $f$, the
integral on the right is zero. Hence 
$$
\int_X \lambda '(|\varphi|^2)| \triangledown \varphi |^2 dX=0
$$

By consequences we have also $d| \varphi |^2=0$, i.e.  $| \varphi |$
is a constant, on the open set $\{ | \varphi |^2 > C_o \}$. Hence
$|\varphi |^2 \le C_o$ on $X$. This concludes the proof. 

\begin{coro*}
  If $ \varphi \in \mathcal{L}^q \cap C^q$ and $\triangle \varphi \in
  \mathscr{D}^q $ (in particular, if $\triangle \varphi =0$), then
  $|\varphi |$ is bounded on $X$. 
\end{coro*}


\section{Finite dimensionality of spaces of harmonic forms}%%% 7

Let $\gamma$ be a real number and let $\mathfrak{M}^q_\gamma$ be the vector space
\begin{alignat*}{4}
  &&\mathfrak{M}^q_\gamma & =\{\varphi|\varphi \in \mathcal{L}^q, \triangle
  \varphi = \gamma \varphi \}. \\
  \text {we set}~  & &\mathfrak{M}^q & =\mathbb{H}^q
\end{alignat*}

\begin{lemma}\label{chap2:lem2.5}%%% 2.5
  $\mathfrak{M}_\gamma \subset  W^q$ and $\gamma \geqslant 0$.
\end{lemma}

\begin{proof}
  If $\varphi \in \mathfrak{M}^q_\gamma$, then $\triangle \varphi \in
  \mathcal{L}^q$. Thus, by the first part of Proposition
  \ref{chap2:prop2.3}, $\varphi 
  \in W^q$. Since $\varphi$ is a solution of the elliptic 
  equation\pageoriginale $\Delta-\gamma$, we may assume $\varphi\in
  C^q$. We have  moreover (since the metric is complete) 
  \begin{align*}
    \Arrowvert\delta d \varphi \Arrowvert^2 & \leqslant\ \Arrowvert \Delta
    d \varphi \Arrowvert^2+ \Arrowvert d \varphi \Arrowvert^2 \leqslant
    (\gamma^2+1) \Arrowvert d \varphi \Arrowvert^2,\\ 
    \Arrowvert d \delta  \varphi \Arrowvert^2 & \leqslant\  \Arrowvert
    \Delta \delta \varphi \Arrowvert^2+ \Arrowvert \delta \varphi
    \Arrowvert^2 \leqslant (\gamma^2+1) \Arrowvert \delta \varphi
    \Arrowvert^2.
  \end{align*}
  Hence $d \varphi \in W^{q+1},\delta \varphi \in W^{q-1}$ and therefore
  \begin{equation*}
    \gamma \Arrowvert \varphi \Arrowvert^2= (\Delta \varphi,\varphi)=
    \Arrowvert d \varphi \Arrowvert^2+ \Arrowvert \delta \varphi
    \Arrowvert^2 \tag{2.12} \label{eq2.12} 
  \end{equation*}
  This proves that $\gamma \geqslant 0$
\end{proof}

It follows from (\ref{eq2.12}) that, if $\gamma=0$, i.e. if $\varphi \in
\mathbb{H}^q$, then $d\varphi = 0$, $\delta \varphi = 0$. 

\begin{theorem}\label{chap2:thm2.3}%%% 2.3
  Let $X$ be a connected, oriented and complete Riemannian
  manifold and assume that $A(\mathcal{K} \varphi,\varphi)\geqslant
  C|\varphi|^2$ for all $\varphi \in C^q$ 
  for some $C > 0$, outside a compact set $K \subset X$. Then for
  $\gamma < C$, $\mathfrak{M}^q_\gamma$ 
  is finite dimensional. In particular dim $\mathbb{H}^q < \infty$.
\end{theorem}

\begin{proof}
  Let $\varphi \in \mathfrak{M}^q_\gamma$. Since $\varphi \in W^q$,
  then by Lemma \ref{chap2:lem2.3} 
  $$ 
  \Arrowvert \triangledown \varphi \Arrowvert^2+(\mathcal{K} \varphi,\varphi)=
  \Arrowvert d \varphi \Arrowvert^2+ \Arrowvert \delta \varphi
  \Arrowvert^2
  $$ 
  and by (\ref{eq2.12}) 
  $$ 
  \Arrowvert \triangledown \varphi \Arrowvert^2+(\mathcal{K} \varphi,\varphi)=
  \Arrowvert d \varphi \Arrowvert^2 + \Arrowvert || \delta \varphi
  \Arrowvert^2  = \gamma \Arrowvert  \varphi \Arrowvert^2
  $$
  Now $\mathcal{K}$ being compact, there is a constant $C_2>0$ such that
  $(\mathcal{K} u,u)\geqslant - C_2|u|^2$ for every $u \in C^q$ and at
  each point of $\mathcal{K}$. 
  Since $(\mathcal{K} \varphi,\varphi) \geqslant C_2 |\varphi|^2$ on the
  complement $X-\mathcal{K}$ we have 
  $$
  \Arrowvert \triangledown \varphi \Arrowvert^2 + C \Arrowvert \varphi
  \Arrowvert^2_{X-\mathcal{K}} \leqslant C_2 \Arrowvert \varphi
  \Arrowvert^2_{\mathcal{K}}  + \gamma \Arrowvert \varphi \Arrowvert^ 2
  $$
  or again,
  $$
  \Arrowvert \triangledown \varphi \Arrowvert^2 + C \Arrowvert \varphi
  \Arrowvert^2 \leqslant (C+C_2) \Arrowvert \varphi \Arrowvert^2_{\mathcal{K}} +
  \gamma \Arrowvert \varphi \Arrowvert^2
  $$
  Hence,\pageoriginale
  \begin{equation*}
    (C-\gamma) \Arrowvert \varphi \Arrowvert^2 \leqslant (C+C_2)
    \Arrowvert \varphi \Arrowvert^2_{K}  \tag{2.13}\label{eq2.13} 
  \end{equation*}
  Suppose now that $\mathfrak{M}^q_{\gamma}$ is infinite dimensional
  for $\gamma < C$, then the evaluation map
  $$ 
  \omega \rightsquigarrow \omega(x) \; (x \in K)  
  $$
  (which associates to a form $\omega$, the element of the $q^{th}$
  exterior power of the tangent space at $x$ which is defined by
  $\omega$) is a linear map  
  into a finite dimensional space. Hence the kernel of this map is of 
  finite codimension. Since the finite intersection of subspaces of
  finite codimension is non-zero (over an infinite field !) it follows  
  that given any sequence $x_1, \ldots, x_{\nu},\ldots$ of points of $K$, we 
  can find forms $\varphi_\in \in \mathfrak{M}^q_{\gamma}$ such that
  $\varphi_{\nu}(x_i)=0$  for $i \leqslant \nu$. We  
  may further assume that $\Arrowvert \varphi _{\nu} \Arrowvert =1$.
  It follows then that we can choose a subsequence $\psi_k=
  \varphi_{\nu_k}$ of  $\{\varphi_{\nu}\}$ such that $\psi_k$ 
  converges weakly to a limit $\psi$ in $\mathcal{L}^q$. Now this
  implies that $\psi \in \mathfrak{M}^q_{\gamma}$, since for any
  $\varphi \in \mathscr{D}^q$, 
  $$
  (\psi, \Delta \varphi)= \lt (\psi_k,\Delta \varphi)= \lt
  (\Delta \psi_k, \varphi) =\gamma(\psi ,\varphi).
  $$ 
  Hence we can assume $\psi \in C^q$. Now the $\psi_k$ are solutions
  of the elliptic  equation $\Delta-\gamma=0$. Hence the weak
  convergence of $\psi_k$ implies 
  that $\psi_k$ converge uniformly together with all partial derivatives
  on every compact set. Clearly we have $\lt \psi_k(x_{\nu})=0$ for every $\nu$.
  If we choose $x_{\nu}$ to be dense in $K$ then $\psi \equiv$ on $K$ since
  $\psi \in C^q$. Hence it follows that
  $$
  |\psi_k|<  \varepsilon ~\text{ on }~ K.
  $$
  for any\pageoriginale given $\varepsilon  >0$ and all large $k$. We
  conclude that 
  $$
\Arrowvert \psi_k \Arrowvert^2_K \longmapsto 0
$$
  as $k\to \infty$. On the other hand since $\psi_k \in
  \mathfrak{M}^q_{\gamma}$, we have by (\ref{eq2.13}), 
  $$ 
  (C-\gamma) \Arrowvert \psi_k \Arrowvert^2 \leqslant (C +C_2)
  \Arrowvert \psi_K \Arrowvert^2_K.
  $$ 
  Since $C-\gamma>0$ and $\Arrowvert \psi_k \Arrowvert^2=1$, we arrive
  at contradiction. Hence $\mathfrak{M}^q_{\gamma}$ is finite dimensional
  for every $\gamma < C$. This proves Theorem \ref{chap2:thm2.1} completely.
\end{proof}

\begin{theorem}\label{chap2:thm2.4}%%% 2.4
  Under the hypothesis of Theorem \ref{chap2:thm2.3} we have the following. 
Let $S^q$ denote the orthogonal complement of $ \mathbb{H}^q$ in the
Hilbert space $W^q$. Then there is a $C>0$ such that
$$ 
\Arrowvert \varphi \Arrowvert^2 \leqslant C \{\Arrowvert d \varphi
\Arrowvert^2 + \Arrowvert \delta \varphi \Arrowvert^2 \} ~\text{for}~
\forall \varphi \in W^q 
$$

 For the proof of the theorem we need the following result 
due to Rellich.
\end{theorem}

\begin{lemma}\label{chap2:lem2.6}%%% 2.6
  Let $X$ be a Riemannian manifold and $\Omega \subset \subset X$
Let $\varphi_{\nu}\in \mathcal{L}^q(\Omega)$ be a sequence such that 
$$
\Arrowvert \varphi_{\nu} \Arrowvert^2 < \infty  ~ \text{ and } ~ \Arrowvert
\triangledown \varphi_{\nu} \Arrowvert^2_{\Omega}< \infty
$$
\end{lemma}

(We denote $\int_{\Omega}|\triangledown \varphi_{\nu}|^2 dX$ by
$\Arrowvert \triangledown
\varphi_{\nu}\Arrowvert^2_{\Omega}$). Suppose that 
$ || \varphi_{\nu} ||^2_{\Omega}+|| \triangledown
\varphi_{\nu}||^2_{\Omega} \leqslant {M}$ for a fixed constant
$M>0$. Then if $\partial \Omega$ is smooth, we can find a subsequence
$\psi_k =\varphi_{\nu_k}$ of $\varphi_{\nu}$ 
such that $||\psi_k -\psi_{k'}||_{\Omega} \to 0$ as $k$, $k' \to \infty$.

If\pageoriginale $\varphi_\nu \epsilon \mathcal{L}^{q}_2 (X)$ is a
sequence such that $\parallel \varphi_\nu \parallel^2 + \parallel
\triangledown \varphi_\nu \parallel^2 \le M$ for a fixed constant
$M>0$, then we can find a subsequence $\psi_k=\varphi_{\nu_k}$ such
that for each compact $K'\subset X$, 
$$
\parallel \psi_k-\psi_{k'}\parallel_K \rightarrow \; 0
\; \text{ as }\; 
k,k'\rightarrow \infty.
$$

\begin{proof}
  The set $\Omega$ being relatively compact,
  it is sufficient to prove the lemma in the case $q=0$, i.e. in the
  case of $L_2$ functions. For this proof see e.g. \cite{key7}, 339.
\end{proof}

\medskip
\noindent{\textbf{Proof of Theorem \ref{chap2:thm2.4}.}} 
Suppose that the theorem is false. Then we can find a sequence
$\varphi_\nu \epsilon W^q$ such that 
$$
\parallel d \varphi_\nu \parallel^2 +\parallel \delta \varphi _\nu
\parallel^2 + \|\varphi_\nu\|^2=1
$$
while $\|d\varphi_\nu\| + \|\delta
\varphi_\nu\|^2\le\frac{1}{\nu}\|\varphi_\nu\|^2$. In view of Remark
under Lemma \ref{chap2:lem2.3} this implies that   
$$
\|\triangledown \varphi_\nu \|^2 \le C_1
$$
for some $C_1 > 0$. By Rellich's Lemma (Lemma \ref{chap2:lem2.6}) we can find a
$\varphi \epsilon \mathcal{L}^q$ (by passing of a subsequence if
necessary) such that  
$$
\|\varphi_\nu - \varphi\|_\Omega\rightarrow 0  
$$
for every relatively compact $\Omega \subset X$.

In particular, we have
$$ 
\|\varphi_\nu-\varphi\|^2_K \rightarrow 0. 
$$

We\pageoriginale assert that $\varphi$ is harmonic, that is $\triangle
\varphi=0$. In fact for  $u \in \mathcal{D}^q$, we have 
\begin{align*}
  (\varphi,\triangle u)&=\lt (\varphi_\nu ,\triangle u)\\
  &=\lt (d \varphi_\nu,du)+(\delta \varphi_\nu ,\delta u)\\
  &\le \lt (\|d \varphi_\nu \| ~\|du\| + \| \delta \varphi_\nu \| . \|
  \delta u \|)\\ 
&\le \lim\frac{1}{\sqrt{2}}\|\varphi_\nu \| (\| du \| + 
  + \| \delta u \|) =0
\end{align*}
since $\| \varphi_\nu\|^2$ is bounded. Hence $\varphi \epsilon W^q
\cap \mathbb{H}^q$. 

Now we have for  $f \epsilon W^q$,
$$
\|\triangledown f \|^2 + (\mathcal{K}
f,f)=\|\text{df}\|^2 + \|\delta f \|^2.
$$

We have $(\mathcal{K} f,f) \geq C | f|^2$ on $X-K$ . On the
other hand there is a $C_2 \geq 0$ such that $(\mathcal{K}
f,f)>-C_2| f|^2$ on $K$. Since $\| \triangledown f \|^2 \geq o$, we obtain, 
$$
C\| f\|^2 \leq (C+C_2)\| f \|^2_{k}+\| df\|^2 + \| \delta f\|^2.
$$

Setting $f = \varphi - \varphi_\nu$, we obtain,
$$
C\|\varphi-\varphi_\nu\|^2 \leq
(C+C_2)\|\varphi-\varphi_\nu\|^2_{k}+\|d(\varphi-\varphi_\nu)\|^2 + \|
\delta(\varphi-\varphi_\nu)\|^2 .
$$
 
Now $\varphi \epsilon \mathbb{H} \;{}^q\cap C^q$ so that  
\begin{gather*}
  (d \varphi, d\varphi) + (\delta \varphi,\delta
  \varphi)=(\vartriangle \varphi,\varphi)=0.
\end{gather*}

Hence
\begin{align*}
  C\|\varphi-\varphi_\nu\|^2 &\leq(C+C_2)\|\varphi-\varphi_\nu||^2_k
  +|| d \varphi_\nu ||^2 + || \delta \varphi_\nu ||^2 \\ 
  & \leq (C + C_2) || \varphi - \varphi_\nu||^2_{k}
  +\frac{1}{\nu}\|\varphi_\nu\|^2 . 
\end{align*}

It follows that $\varphi_\nu \rightarrow \varphi$ in $\mathcal{L}^q$ on
the whole of $X$. On the other\pageoriginale hand since
$$
\| d \varphi_\nu \|^2 + ||\delta \varphi_\nu ||^2 \leq
\frac{1}{\nu}\|\varphi_\nu||^2 
$$
$d\varphi_\nu$ and $\delta \varphi_\nu$ converge to zero. Hence
   $\varphi_\nu $ converges in $W^q$ to a limit $\varphi$ satisfying
   $d arphi=0$, $\delta\varphi=0$, i.e. $\varphi\epsilon
   \mathbb{H}^q$. But $\varphi_\nu \epsilon S^q$ and $S^q$ is a closed
   subspace. Hence $\varphi\epsilon S^q$ and therefore
   $\varphi=0$. But then $\varphi_\nu \rightarrow 0$ in $W^q$, and
   this is absurd, since $||\varphi_\nu||^2 + || d
   \varphi_\nu||^2 +||\delta \varphi_\nu||^2=1$. It follows that there
   exists $C'>0$ such that for $\varphi \epsilon S^q$ 
   $$
   C^{'}\{|| \text{d} \varphi||^2 + ||\delta \varphi||^2 \}\geq
   ||\varphi||^2
   $$

\begin{remark*}
  Since for $\varphi \epsilon \mathbb{H}^q$, $d\varphi=0$, $\delta
  \varphi = 0$, $S^q$ is also the orthogonal complement of
  $\mathbb{H}^q$ with respect to the scalar product on $W^q$ induced 
  by $\mathcal{L}^q$. 
\end{remark*}

The hypothesis of Theorem \ref{chap2:thm2.3} can be weakened, further the space
$\mathbb{H}^q$. 

\begin{theorem}\label{chap2:thm2.5}%% 2.5
  Let $X$ be an oriented, connected and complete Riemannian manifold
  and let $A(\mathcal{K}\varphi,\varphi)\geq 0$ outside a compact set
  $\mathcal{K}$ for every $\varphi \in  C^q$. Then $\dim \mathbb{H}^q<\infty$. 
\end{theorem}

\begin{proof}
  If $\dim \mathbb{H}^q=\infty$, then given any dense sequence
  $\{x_\nu\}$ of points $x_\nu \epsilon X$, we can find a sequence
  $\{\varphi_\nu \}$ of forms $\varphi_\nu \epsilon \mathbb{H}^q$  
  such that 
  $$
  \varphi_\nu(x_i)=0 \; {\rm for \; i}\;\leq \nu ,
  ||\varphi_\nu||^2 = 1. 
  $$

  We assert now that $||\varphi_\nu||_\Omega > 0$ for any relatively
  compact open subset $\Omega \supset K$. In fact, we have 
  $$
  ||\triangledown \varphi_\nu||^2 + (\mathcal{K}\varphi_\nu
  \varphi_\nu)=|| d\varphi_\nu||^2 + ||\delta\varphi_\nu||^2=o\;
  \text{since}\; \varphi_\nu \epsilon \mathbb{H}^q .
  $$ 

  Hence\pageoriginale from the Remark under Lemma \ref{chap2:lem2.3}
  it follows that   there is a constant  $C > 0$ such that  
  \begin{equation*}
    \parallel \triangledown \varphi_{\nu} \parallel^2 < C \parallel
      \varphi_{\nu}\parallel^{2}_{K}  \tag{2.14}\label{eq2.14} 
  \end{equation*}

  Hence if $\parallel \varphi_\nu \parallel_\Omega = 0$, then
  $\parallel \varphi_\nu\parallel^2_K = 0$ 
  so that $\triangledown \varphi_{\nu} = 0$. But in the case, $\mid
  \varphi_\nu\mid^2$ is a constant. 
  Hence $\parallel \varphi_\nu\parallel_\Omega > 0, $ a contradiction.

  We may therefore assume that for a fixed $\Omega\supset K$
  $\parallel \varphi_\nu\parallel_\Omega = 1$. 
  From (\ref{eq2.14}),$\parallel\varphi_\nu\parallel^2_\Omega$ $+
  \parallel\triangledown\varphi_\nu\parallel^2 
  _\Omega < C'$ for some $C' > 0$. Hence by Rellich's Lemma (Lemma
  \ref{chap2:lem2.6}), we can, by passing to a subsequence,   
  if necessary, assume that $\parallel\varphi_\nu - \varphi_\mu
  \parallel^2_\Omega \rightarrow 0 $ 
  as $\nu, \mu \rightarrow \infty$ provided that $\partial\Omega$ is
  smooth. Hence  
  $\{\varphi_\nu\}$ converges to a limit $\varphi$ in  $\mathcal{L}^q(\Omega)$.

  But since $\{\varphi_\nu\}$ is a sequence of solutions of an
  elliptic operator, $\varphi_\nu$ 
  converges to $\varphi$ uniformly on every compact subset of
  $\Omega$. Since $\varphi_\nu(x_i)=0$ for $i$  
  $\leqslant$ $\nu$, we see that $\varphi \equiv 0$ on $\Omega$. Hence
  $\|\varphi_\nu \|_\Omega \rightarrow 0$ as $\nu\rightarrow\infty$, a
  contradiction. It follows that $\mathbb{H}^q$ is finite dimensional.  
  Hence the theorem. An estimate for the dimension of $\mathbb{H}^q$
  is provided by the following. 
\end{proof}

\setcounter{prop}{2}
\begin{prop}%%% 2.3
  Assume that, under the same hypotheses for $X$, $\mathcal{K}
  \equiv0$ on $X$. Then for 
  $\varphi\epsilon \mathbb{H}^q,$ $\triangledown\varphi = 0$. Hence,
  $\dim \mathbb{H}^q\leqslant \binom {n}{q}$. 
  Moreover, if $\dim \mathbb{H}^q>0$, then vol $X< \infty$.
\end{prop}

\begin{proof}
  Since\pageoriginale now
  $$
\parallel \triangledown \varphi \parallel^2 = \|d\varphi\|^2 +
  \|\delta\varphi\|^2 = 0 ~ \text{  for  } ~
  \varphi\epsilon\mathbb{H}^q 
$$ 
  we have $\triangledown\varphi = 0$ for
  $\varphi\epsilon\mathbb{H}^q$. A form invariant under parallel  
  translation is determined by its value at one point. Hence dim
  $\mathbb{H}^q \leqslant \binom{n}{q}$. If  
  $\varphi\epsilon\mathbb{H}^q$ is non-zero, then
  $\|\varphi\|^2<\infty$; on the other hand since
  $\triangledown\varphi  =0$, $\mid\varphi\mid^2$ is a constant and
  the last assertion follows. 
\end{proof}

\begin{lemma}\label{chap2:lem2.7}%%% 2.7
  Under the hypotheses of Theorems \ref{chap2:thm2.3} and
  \ref{chap2:thm2.4}, for every $s \epsilon S^q$ 
  there exists a unique $x \epsilon S^q$ such that
  $$ 
  s = \triangle x 
  $$
  in the sense of distributions. Moreover
  \begin{align*}
    \parallel x\parallel & \leqslant C{'}\parallel s \parallel ,\\
    \|dx|^2 & + ||\delta x\|^2\leq C{'}\parallel s \parallel ^2
  \end{align*}
  $C'$ being the constant which appears in Theorem
  \ref{chap2:thm2.4}. If s $\epsilon 
  C^q \cap S^q $, then ($x$ can be modified on a null 
  set in sch a way that)
  $x\epsilon C^q \cap S^q$ and the equation $\triangle x = s $ holds
  in the ordinary sense. 
\end{lemma}

\begin{proof}
  $S^q$ as a (closed) subspace of the Hilbert space $W^q$ is a Hilbert space.
  By theorem \ref{chap2:thm2.4} the norm $N$ is equivalent on $S^q$ to the norm
  $(||d||^2 + ||\delta||^2)^{\frac{1}{2}}$.  
  $f\rightsquigarrow (s,f)$ is a continuous linear form on $S^q$. Thus
  there is a unique $x \in S^q$ such that 
  $$ 
  (s,f) = (dx,df) + (\delta x,\delta f) \text {for all} \,f \epsilon
  S^q.
  $$

Let now $\varphi$ be any element of $W^q$ and let h and $\psi$ be its
orthogonal projections into $\mathbb{H}^q$ 
and $S^q$. They are uniquely defined, and furthermore
$$
\varphi = h + \psi 
$$

Since\pageoriginale $h$ is orthogonal to $s$ and $dh = 0$, $\delta h =
0$, then 
$$
(s,\varphi) = (s, \psi) = (dx, d \psi) + (\delta x, \delta \psi) =
(dx, d\varphi)+(\delta x, \delta \varphi).
$$

This proves the lemma.
\end{proof}

Since
$$ 
W^q = \mathbb{H}^q \oplus S^q
$$
we can state the following.

\begin{prop}%%% 2.4
  Under the hypotheses of Theorem \ref{chap2:thm2.3}, for every 
  $\varphi \epsilon W^q$ there exists a unique h$\epsilon
  \mathbb{H}^q$ and a unique $x \epsilon S^q$ such that 
  $$
  \varphi = h + \triangle x
  $$
  (in the sense of distributions). Moreover
  \begin{gather*}
    ||x||\leqslant C{'}||\varphi||\\
    ||dx||^2 + ||\delta x||^2 \leqslant C{'} ||\varphi||^2
  \end{gather*}
\end{prop}

\section{Orthogonal decomposition in $\mathcal{L}^q$}%%% 8

Let $X$ be an oriented, complete and connected Riemannian
manifold. Assume that there exists a compact set $K \subset X$ 
and a positive constant $C$ such that
$$
A (\mathcal{K} \varphi, \varphi) \geqslant C A (\varphi, \varphi) 
$$
outside $K$, for each $\varphi\epsilon C^q$.

The space $\mathcal{L}^q$ can be decomposed as direct sum of
orthogonal subspaces 
$$
\mathcal{L}^q = \mathbb{H}^q \oplus \overline{d\mathscr{D}^{q-1}} 
\oplus \overline{\delta \mathscr{D}^{q+1}},
$$
where $\overline{d \mathscr{D}^{q-1}}$ and $\overline{\delta
  \mathscr{D}^{q+1}}$ are the closure of d $\mathscr{D}^{q-1}$ and
$ \mathcal{L} \mathcal{D}^{q+1}$\pageoriginale with respect to the
norm $ \parallel \quad  \parallel$.  

Let $ \varphi \in \mathcal{L}^q$; - if  $d \varphi = 0$, then the
orthogonal projection of $\varphi$ into $ \overline{\delta
  \mathcal{D}^{q+1}}$  is zero; if $\delta \varphi =0$, then the
orthogonal projection 
into $  \overline{d \mathcal{D}^{q-1}} $ is zero (see \cite{key17}
602-605; \cite{key29}, 165).

\begin{lemma}\label{chap2:lem2.8}%% 2.8
  For every $\phi \in \overline{d \mathcal{D}^{q-1}}$ there is a
  unique form $x\in S^q$ such that  
  $$ 
  \phi =d \delta  x, \quad dx=0. 
  $$ 

  Moreover
\begin{gather*}
  \parallel x \parallel \leq C' \parallel \phi \parallel,\\ 
  \parallel \delta x \parallel^2 \leq C' \parallel \phi \parallel^2 
\end{gather*}
($C'$ being the constant introduced in Theorem \ref{chap2:thm2.4})
\end{lemma}

\begin{proof}
  $$ 
  s  \rightsquigarrow (\phi,s) 
  $$
  is a continuous form on  $S^q$ . Then there exists a unique $x \in S^q $ 
  such that 
  $$ 
  (\phi,s)=(dx,ds)+(\delta x, \delta s) ~ \text{ for all } ~s \in S^q  .
  $$ 

  Moreover, by Theorem \ref{chap2:thm2.4}, 
  \begin{align*} 
    \parallel x \parallel^2 \leq C' ( \parallel dx \parallel^2 +
    \parallel \delta x \parallel^2) & = C' (\phi, x ) \leq C' \parallel
    \phi \parallel.\parallel x \parallel,\\
    \parallel dx \parallel^2 + \parallel \delta x \parallel^2 & =
    (\phi,x) \leq \parallel \phi \parallel.\parallel x \parallel \leq C'
    \parallel \phi \parallel^2.  
  \end{align*}

  Let now  $u \in \mathcal{D}^q  \cdot u$ can be written 
$$
u=h+s, \;\; h \in \mathbb{H}^q ,  \;\; s \in S^q.
$$ 

  Since\pageoriginale $h \overline {\perp d \mathcal{D}^{q-1}} $ in $
  \mathcal{L}^q $  and $dh=0$,  $\delta h=0$, then $(\varphi,u) =
  (\varphi,s)=(dx,ds)+(\delta x, \delta s) = (d x, d u) + (\delta x,
  \delta u) = (x, \triangle u)$ i.e.
  $$ 
  \varphi = \triangle x
  $$
  in the sense of distributions. 
\end{proof}

Let $ \phi \in C^q \cap \overline{d \mathcal{D}^{q-1}}$. By the
regularity theorem, we can modify $x$ on a null set in such a way that $x
\in C^q \cap S^q $ , and that the latter equation holds in the
ordinary sense. Thus, by proposition \ref{chap2:prop2.2}
$$ 
\parallel \delta d x \parallel^2 \leq \frac{1}{\sigma}\parallel d
\phi \parallel^2+\sigma \parallel dx \parallel^2 = \sigma \parallel dx
\parallel^2 ~\text{ for all } ~ \sigma > 0.
$$ 

Hence 
$$ 
\delta d x=0, \quad \varphi=d \delta x .
$$  

But then, by Theorem \ref{chap2:thm2.1}, $ \delta x \in W^{q-1}$, and
therefore  
$$ 
\parallel dx \parallel^2 = (x,\delta dx)=0\quad {\rm i.e.}\quad  dx=0. 
$$

If $\varphi$ is not $ C^\infty $, there exists a sequence $ \{u_\nu \}$
of forms  $u_\nu \in \mathcal{D}^{q-1}$ such that $ \parallel  
du_\nu - \varphi \parallel \longrightarrow 0$. Clearly $ du_\nu
\perp \mathbb{H}^q $; hence $ du_\nu \in S^q$. Setting $ \varphi_\nu  
= du_\nu $ and applying to $ \varphi_\nu $ the above argument, we can
find, for each $\varphi_\nu$, a unique $ x_\nu \in S^q \cap C^q$ such that 
\begin{gather*}
  \varphi_\nu = d \delta x_\nu, dx_\nu =0 \\
  \parallel x_\nu\parallel \leq C'\parallel \varphi_\nu\parallel ,\\
  \parallel \delta x_\nu\parallel^2 \leq C' \parallel \varphi_\nu
  \parallel^2
\end{gather*}

Hence\pageoriginale $\{x_\nu\}$ is a Cauchy sequence in $S^q$. Let  
$$
x = \lim \; x_\nu .
$$
Then
\begin{align*}
  dx & =0 ,\\
  \parallel x \parallel & \leq C' \parallel \varphi \parallel ,\\
  \parallel \delta x \parallel^2 & \leq C' \parallel \varphi \parallel^2,
\end{align*}
and finally
$$ 
\varphi=d \delta x , 
$$
in the sense of distributions. This proves the lemma.

An analogous argument yields 

\begin{lemma}\label{chap2:lem2.9}%% 2.9
  For any $\psi \in \overline {\delta \mathcal{D}^{q+1}}$ there is a
  unique form $y\in S^q$  such that 
  $$ 
  \psi =~~\delta~dy~~~~\delta~y=0 
  $$
  Moreover
  \begin{align*}
    \parallel y \parallel & \leq C' \parallel \psi \parallel\\
    \parallel dy \parallel^2 &\leq C' \parallel \psi \parallel^2\\
  \end{align*}
\end{lemma}

\begin{theorem}\label{chap2:thm2.6}%%% 2.6
  Every current $f \in \mathcal{L}^q $ can be decomposed as the sum of
  three currents 
  $$ 
  f = h+d \varphi_1 + \delta \psi_1 
  $$
  with $ h \in \mathbb{H}^q, \varphi_1 \in W^{q-1}, \psi_1 \in
  W^{q+1}$. Moreover 
  $$ 
  N(h)^2 = \parallel h\parallel^2 \leq \parallel f \parallel^2,
  N(\varphi_1)^2 \leq (C'+1) \parallel \varphi \parallel^2,
  N(\psi_1)^2 \leq  (C'+1) \parallel \psi \parallel^2 
  $$
\end{theorem}

\begin{proof}
  First of all any element $f \in \mathcal{L}^q$ can be expressed in a
  unique way as the sum 
  $$ 
  f=h+\varphi + \psi 
  $$
  of three\pageoriginale forms $h \in \mathbb{H}^q$, $\varphi \epsilon$
  $\overline{d \mathscr{D}^{q-1}}$, $\psi \epsilon \overline{\delta
    \mathscr{D}^{q+1}}$   
  $$
  || f||^2 = ||h||^2+|| \varphi||^2 +||\psi||^2.
  $$
  
  Next we apply to $\varphi$ and $\psi$ Lemma \ref{chap2:lem2.8}  and
  \ref{chap2:lem2.9} setting  then  
  \begin{gather*}
    \varphi = \delta \varphi_1,  \quad  \psi = d \psi_1;\\
    \varphi_1 \in W^{q-1} \quad  \psi_1 \in W^{q+1}.
  \end{gather*}
  
  Moreover
  \begin{align*}
    N(\varphi_1)^2 &=||\varphi_1||^2+||\varphi||^2\le
    (C'+1)||\varphi||^2 \le(C'+1)||f||^2 \\
    N(\psi_1)^2 & =||\psi_1||^2+||\psi||^2\le(C'+1)||\psi||^2\le (C'+1)||f||^2
  \end{align*}
\end{proof}

\begin{remark*}
  If the hypotheses of Theorem \ref{chap2:thm2.3} are satisfied, not only for the
  forms of degree $q$, but also for those degree $q-1$ and  $q+1$ then
  spaces $\mathbb{H}^{q-1}$, $\mathbb{H}^{q+1}$, $S^{q-1}$ and $S^{q+1}$ can
  be introduced and one checks that  
  \begin{alignat*}{4}
    \varphi_1\perp \mathbb{H}^{q-1}, & \qquad & \text{i.e.} &\qquad
    \varphi_1 \in S^{q-1}\\ 
    \varphi _1\perp \mathbb{H}^{q+1}, && \text{i.e.} &\qquad \psi _1
    \in S^{q+1}. 
  \end{alignat*}
\end{remark*}
