
\chapter{On The Approximation of Parabolic  Variational
  Inequalities}\label{chap3} 

\section{Introduction  References}\label{c3:s1}

In this\pageoriginale  chapter we would like to give some indications
on the approximation of \textit{Parabolic Variational Inequalities}
(PVI) (mostly without proof). For a detailed treatment see
G. L. T. [\ref{k53:e2} 
  Chap. 6],  TREMOLIERES [\ref{k89:e1}], and for further reference see FORTIN
[\ref{k45:e1}], BRISTEAU [\ref{k21:e1}], BRISTEAU-GLOWINSKI
[\ref{k22:e1}], C. JOHNSON [\ref{k60:e1}] and
A. BERGER [\ref{k10:e1}]. See also LASCAUX [\ref{k65:e1}] for the
numerical analysis of \textit{time dependent equations.}  

\section{Formulation And Statement of The Main Results}\label{c3:s2}

Let $H$ and $V$ be two real Hilbert spaces that  $V \subset H$, $\overline{V} = H$. Assuming that $H = H^*$  we have then $V \subset H \subset V^*$.

The scalar product in $H$ (\resp. in $V$) and the corresponding norms
are denoted by $(\cdot, \cdot)$,$|\cdot|$  (\resp. $((\cdot , \cdot)),
|| \cdot||)$. Moreover we also use $(\cdot , \cdot)$ for the duality
between $V^*$ and $V$. 

We now introduce:
\begin{itemize}
\item A time interval $[0,T]$ with $0 < T <\infty$, a bilinear from $a
  : V \times V \to \mathbb{R}$, continuous and \textit{elliptic} in
  the following sense :$\exists \alpha> 0$ and $\lambda \geq 0$ such
  that 
$$
a (v, v) + \lambda |v|^2 \geq \alpha || v||^2 \ \forall v \varepsilon V,
$$
\item 
$$
f \varepsilon L^2 (0, T ; V), u^0 \in H,
$$
(for the definition of $L^2 (0, T; X)$) see LIONS [1], [3])
\item $K$ : closed, convex, non-empty subset of $V$,
\item $j : V \to \overline{\mathbb{R}}$ convex, proper, l.s.c
\end{itemize}

We consider then the following two families of PVI:
\begin{equation*}
 \begin{cases}
\text{ Find } u(t) \text{ such that } \\ 
\left(\frac{\partial u} {\partial t}, v-u\right) +a (u, v-u) \geq (f,
v-u) \ \forall v \varepsilon K, ~ a.e. ~ t \in] 0,T [, \\ 
u (t) \in K  ~ a.e. ~ t \in ] 0, T [, u (0) = u_0, 
\end{cases}\tag{2.1}\label{c3:eq2.1}
\end{equation*}\pageoriginale 
and 
{\fontsize{10}{12}\selectfont
\begin{equation*}
 \begin{cases}
\text{ Find } u (t) \text{ such that }\\
(\frac{\partial u} {\partial t}, v-u) +a (u, v-u) + j (v) -j (u) \geq
(f, v-u) \ \forall v \in V, ~ a.e. ~ t \in] 0, T[, \\ 
u (t) \in V ~ a.e. ~ t \in ] 0, T [, u (0) = u_0. \tag{2.2}\label{c3:eq2.2} 
\end{cases}
\end{equation*}}\relax

\begin{remark}%Rem 3.1
If $K =V$ and $J \equiv 0$ then \eqref{c3:eq2.1} and \eqref{c3:eq2.2}
reduce to the standard parabolic variational equation 
\begin{equation}
\begin{cases}
(\frac{\partial u} {\partial t}, v) +a (u, v) = (f, v) \ \forall v \in
  V, ~ a.e. ~ \text{ in } t \in] 0, T[, \\ 
    u (t) \in V ~ a.e. ~ t \in ] 0, T [, u (0) = u_0. \tag{2.3}\label{c3:eq2.3}
\end{cases}
\end{equation}
\end{remark}

Under appropriate assumptions on $u_0$, $K$ and $j (\cdot)$ it is
proved that \eqref{c3:eq2.1}, \eqref{c3:eq2.2} have unique solutions
in $L^2 (0,T; V) \cap C^0 ([0, T], H)$. For the proof of this we refer
to BREZIS [\ref{k14:e4}], [\ref{k14:e5}]; LIONS
[\ref{k67:e1}],DUVAUT-LIONS[\ref{k42:e1}]. 

 In the following sections of this chapter we would like to give some
 discretisation schemes for \eqref{c3:eq2.1},  \eqref{c3:eq2.2} and
 then in sec. 6 study the asymptotic properties in time of a specific
 example, for the continuous and discrete cases. 

 \section{Numerical Schemes For Parabolic Linear Equations}\label{c3:s3}

 Let us assume that $V$ and $H$ have been approximated (as $h \to 0$)
 by the same family $(V_h)_h$ of closed subspaces of $V$ (in practice
 the $V_h$ are finite dimensional). We also approximate $(\cdot ,
 \cdot)$,  $a(\cdot , \cdot)$ by $(\cdot,\cdot)_h$, $a_h (\cdot ,
 \cdot)$ is such a way that ellipticity, symmetry etc. are
 preserves. We also assume that $u_0$ is approximated  by $(u_{oh})_h$
 such that $u_{oh} \in V_h$ and $\lim\limits_{h \to 0} u_{0h} = u_0$
 strongly in $H$.
 
We now introduce a \textit{ time step } $\Delta  t$; then denoting
$u^n_h$ the approximation of $u$ at time $t = n \Delta t \,(n = 0, 1, 2,
\ldots,)$, we approximate \eqref{c3:eq2.3} using the classical step by
step numerical schemes (i.e. we describe how to compute $u_n^{n+1}$)
if $u_h^n$ and $u^{n-1}_h$ are known). 
 \begin{enumerate}
 \item {\bf Explicit scheme.}\pageoriginale 
 \begin{equation}
\begin{cases}
\frac{u_n^{n+1} -u^n_h} {\Delta  t}, v_h )_h + a_h (u^n_h, v_h) = (f_h^n,v_h)_h \ \forall v_h \in v_h,\\ 
n = 0, 1,\ldots,\\
u^0_h = u_{0h}.\tag{3.1}\label{c3:eq3.1}
\end{cases}
 \end{equation}
 \textit{Stability.} (see LASCAUK [\ref{k65:e1}]) for the terminology)
 \textit{conditional.}  \textit{Accuracy.} $0(\Delta  t)$ (we just
 consider the influence of the time discretisation).  
 
 \item {\bf Ordinary implicit scheme.}
 \begin{equation}
\begin{cases}
\left(\frac{u_n^{n+1} -u^n_h} {\Delta  t}, v_h \right)_h + a_h (u^{n+1}_h, v_h) =
(f^{n+1}_h,v_h)_h \, \forall v_h \in V_h,\\ 
n = 0, 1,2,\ldots,\\
u^0_h = u_{0h}. \tag{3.2}\label{c3:eq3.2}
\end{cases}
 \end{equation}
\textit{Stability. Unconditional}. 

Time accuracy. $0 (\Delta  t)$

\item {\bf Cranck-Nicholson scheme.}
 \begin{equation}
\begin{cases}
\left(\frac{u^{n+1}_h - u^n_h} {\Delta  t}, v_h\right)_h + a_h
\left(\frac{u^{n+1}_h + u^n_h} {2}, v_h\right) = \left(\frac{f^{n+1}_h
  + f^n_h} {2} ,v_h\right)_h \ \forall v_h \in V_h\\ 
\text{ or } = (f^{n+1/2}_h, v_h)_h \ \forall v_h \in V_h \\
n = 0, 1,2 ,\ldots,; u^0_h = u_{oh}. \tag{3.3}\label{c3:eq3.3}
\end{cases}
\end{equation}
\textit {Stability. Unconditional}.  

Time accuracy. $0 (|\Delta  t|^2)$.

\item  {\bf Two steps implicit scheme.}
\begin{equation}
\begin{cases}
(\frac{\frac{3}{2} u^{n+1}_h -2u^n_h + \frac{1}{2} u^{n-1}_n} {\Delta
    t}, v_h)_h + a_h (u^{n+1}_h , v_h) = (f^{n+1}_h, v_h)_h \ \forall
  v_h \in V_h,\\ 
  n=1, 2,\ldots, u^0_h = u_{0h}, u^1_h \text { given
  }. \tag{3.4}\label{c3:eq3.4} 
\end{cases}
 \end{equation} 
 \textit{Stability.\pageoriginale   Unconditional}. 

 Time accuracy. $0 (|\Delta  t|^2)$.
 \end{enumerate}
  
 Unlike the three previous schemes, this latter scheme requires the
 use of a \textit {starting procedure} to obtain $u^1_h$ from $u^0_h =
 u_{0h}$; to compute $u^1_h$ we can use for example one of the scheme
 \eqref{c3:eq3.1}, \eqref{c3:eq3.2} or \eqref{c3:eq3.3}; we recommend
 \eqref{c3:eq3.3} since it is also an $0 (|\Delta
 t|^2)$-scheme. Similarly the generalisations of scheme
 \eqref{c3:eq3.4} discussed in Sec.~\ref{c3:s4}, \ref{c3:s5} will
 require the use of a starting procedure which can be the
 corresponding generalization of schemes \eqref{c3:eq3.1},
 \eqref{c3:eq3.2} or \eqref{c3:eq3.4}. 

\begin{remark}\label{c3:rem3.1} %Rem 3.1
The vector $f^n_h$ (or $f^{n+1/2}_h$) occurring in the right hand
sides of \eqref{c3:eq3.1}- \eqref{c3:eq3.4} is a convenient
approximation of $f$ at $t = n \Delta  t$ (or $t = (n+\frac{1}{2})
\Delta t$). 

In some cases it may be defined as follows (we just consider $f^n_h$ since the technique described below is also applicable to $f_n^{n+1/2}$).
\end{remark} 

First we define $f^n \in V^*$, by
$$
f^n = f (n \Delta  t) \text{ if } f \in C^0 [0, T; V^*].
$$
In the general case, it is defined by 
\begin{align*}
f^0  & = \frac{2}{\Delta  t} \int_0^{\frac{\Delta  t}{2}} f (t) dt,\\
f^n & = \frac{1}{\Delta  t} \int^{(n+\frac{1}{2}) \Delta  t} _{(n - \frac{1}{2}) \Delta  t} f (t) dt \text{ if } n \geq 1.
\end{align*}

Then since $(\cdot,\cdot)_h$ is a scalar product on $V_n$ one may define $f^n_h$ by
$$
(f^n, v_h) = (f^n_h, v_h)_h ~ \forall v_h \in V_h, ~ f^n_h \in V_h.
$$
In some cases we have to use more sophisticated methods to define $f_h^n$.

\begin{remark}\label{c3:rem3.2} %remark 3.2
At each step $(n+1)$ we have to solve a linear system to compute
$u^{n+1}_h$; however if we can use a scalar product $(\cdot ,
\cdot)_h$ leading to a {\em diagonal matrix}, with regard to the
variables defining $v_h$, then the use explicit scheme will only
require to solve {\em one variable linear equations} at each step. 
\end{remark}

\begin{remark}\label{c3:rem3.3} %Rem 3.3
We can also use nonconstant time steps $\Delta  t_n$.
\end{remark}

\begin{remark}\label{c3:rem3.4}%Rem 3.4
If we\pageoriginale  are interested in the numerical integration of
{\em ``Stiff''  phenomenon} or in {\em long range integration} we can
briefly say that  
\begin{itemize}
\item Schemes \eqref{c3:eq3.1}, \eqref{c3:eq3.2} are {\em too}
  dissipative, moreover the stability condition in \eqref{c3:eq3.1}
  may be a serious drawback. 
\item Scheme \eqref{c3:eq3.3} is, in some sense, {\em not sufficiently
  dissipative}. 
\item Scheme \eqref{c3:eq3.4} avoids the above inconveniences and is
  highly recommended for ``Stiff'' problems and long range
  integration. In most cases the extra storage it requires is not a
  serious drawback. 
\end{itemize}
\end{remark}

\begin{remark}\label{c3:rem3.5}%Rem 3.5
There are many works related to the numerical analysis of parabolic
equations via finite differences in time and finite elements in space
approximations. We refer to RAVIART [\ref{k80:e1}], [\ref{k80:e2}],
CROUZEIX [\ref{k39:e1}], STRANG-FIX 
[\ref{k85:e1}], ODEN-REDDY [\ref{k84:e1}, CHAP. 9] and the
bibliographies therein.  
\end{remark}

\section{Approximation of PVI of The First Kind}\label{c3:s4}

We assume that $K$ in \eqref{c3:eq2.1} has been approximated by
$(K_h)$, $K_h \subset V_h \forall h$ ,  like in the elliptic case (see
Chap.~\ref{chap1}). We also suppose that the bilinear form a is
possibly dependent on the time $t$ and has been approximated by $a (t
; u_h, v_h)$. 
\begin{enumerate}
\item  {\bf Explicit scheme.}
\begin{equation}
\begin{cases}
  \left(\frac{u^{n+1}_h  -u^n_h} {\Delta  t}, v_h -u^{n+1}_h\right)_h +a_h (n
  \Delta  t; u^n_h, v_h -u^{n+1}_h)\\ 
  \hspace{3cm}\geq (f^n_h, v_h - u^{n+1}_h)_h
  \ \forall v_h \in K_h,\\ 
  u^{n+1}_h \in K_h,\\
  n = 0, 1,2 ,\ldots, u^0_h = u_{0h}. \tag{4.1}\label{c3:eq4.1}
\end{cases}
\end{equation}
\textit{Stability. Conditional} (see G.L.T [2, Chap 6]). This scheme
is almost never used in practice since it is conditional stable and
that the computation of $u_h^{n+1}$ will require in general, the use
of an iterative  method \textit{even} if the matrix corresponding to
$(\cdot , \cdot)_h$ is \textit{diagonal}. 

\item  {\bf Ordinary implicit scheme.}\pageoriginale 
\begin{equation}
\begin{cases}
\left(\frac{u^{n+1}_h -u^n_h} {\Delta  t} , v_h -u^{n+1}_h\right)_h +
a_h ((n+1) \Delta  t ; u^{n+1}_h, v_h - u^{n+1}_h)\\ 
\hspace{3.5cm}\geq (f^{n+1}_h,
v_h - u^{n+1}_h) \ \forall v_h \in K_h, \\ 
u^{n+1}_h \in K_h,\\
n = 0, 1,2 ,\ldots, u^0_h = u_{oh}. \tag{4.2}\label{c3:eq4.2}
\end{cases}
\end{equation}
\textit{Stability. Unconditional}. 

At each step we have to solve an EVI of the first kind in $K_h$ to
compute $u^{n+1}_h$. This scheme is very much used in practice. 

\item  {\bf Crank-Nicholson scheme.}
\begin{equation}
\begin{cases}
(\frac{u^{n+1}_h -u^n_h} {\Delta  t} , v_h -u^{n+1/2}_h)_h + a_h
  ((n+\frac{1}{2}) \Delta  t ; u^{n+1/2}_h, v_h - u^{n+1/2}_h) \\ \geq
  f^{n+1/2}_h, v_h - u^{n+1/2}_h) \ \forall v_h \in K_h,\\ 
  u^{n+1/2}_h \in K_h, u^{n+1/2}_h = \frac{u^n_h + u^{n+1}_h}{2}, n n =
  0,1,2 ,\ldots, u^0_h = u_{0h}. \tag{4.3}\label{c3:eq4.3} 
\end{cases}
\end{equation}
\textit{Stability. Unconditional.}

Since $\dfrac {u^{n+1}_h -u^n_h} {\Delta  t} = \dfrac{u^{n+1/2}_h
  -u^n_h} {\dfrac{\Delta  t}{2}}$, we observe that at \textit{each
  step} we have to solve an EVI of the \textit {first kind} to compute
$u^{n+1/2}_h$. We observe also that possibly $u^n_h \notin K_h$. We do
not recommend this scheme if the regularity in time of the continuous
solution is poor. 

\item {\bf Two steps implicit schemes.}
\begin{equation}
\begin{cases}
\frac{\frac{3}{2} u^{n+1}_h - 2u_h^h +\frac{1}{2} u^{n-1}_h }{\Delta  t}, v_h -u^{n+1}_h )_h +a_h ((n+1) \Delta  t; u^{n+1}_h, v_h -u^{n+1}_h) \\
\geq (f^{n+1}_h, v_h -u^{n+1}_h)_h \ \forall v_h \in K_h, u^{n+1}_h
\in K_h,\\ 
\hspace{2cm}n = 1, 2, \ldots, u^0_h= u_{0h}, u^1_h  ~\text{given}.
\tag{4.4}\label{c3:eq4.4}  
\end{cases}
\end{equation}

\textit {Stability. Unconditional.} We have to solve at each step an
EVI of the first kind in $K_h$ to compute
$u^{n+1}_h$. Remark~\ref{c3:rem3.4} applies to this scheme also. 
\end{enumerate}

\section{Approximation of PVI of The Second Kind}\label{c3:s5}

\begin{enumerate}
\item  {\bf Explicit scheme}\pageoriginale 
{\fontsize{10}{12}\selectfont
\begin{equation}
\begin{cases}
(\frac{u^{n+1}_h -u^n_h} {\Delta  t} , v_h -u^{n+1}_h)_h + a_h (n
  \Delta  t ; u^{n}_h, v_h - u^{n+1}_h) + j_h (v_h) (u^{n+1}_h) \geq
  \\ 
  \geq f^{n}_h, v_h - u^{n+1}_h) \ \forall v_h \in V_h, u^{n+1}_h \in
  V_h, n=0,1,2,\ldots, u^0_h = u_{0h}. \tag{5.1}\label{c3:eq5.1} 
\end{cases}
\end{equation}}\relax
\textit { Stability. Conditional.}

This scheme is also almost never used in practice since it is
conditionally stable and the the computation of $u^{n+1}_h$ will
require the solution of an EVI of the \textit {second kind} in $V_h$
(in general by an iterative method) even if the matrix corresponding
to $(\cdot ,  \cdot)_h$ is \textit {diagonal}. 

\item {\bf Implicit scheme.}
\begin{equation}
\begin{cases}
\left(\frac{u^{n+1}_h -u^n_h} {\Delta  t}, v_h -u^{n+1}_h\right)_h +
a_h ((n+1) \Delta  t ; u^{n+1}_h, v_h - u^{n+1}_h)\\ 
\hspace{4cm}+ j_h (v_h)-j_h
(u^{n+1}_h) \\ 
\geq \left(f^{n+1}_h, v_h - u^{n+1}_h\right) \ \forall v_h \in V_h,
u^{n+1}_h \in V_h,\\ 
n =0, 1, 2, \ldots, u^0_h = u_{0h}. \tag{5.2}\label{c3:eq5.2}
\end{cases}
\end{equation}
\textit{Stability.Unconditional.}

At each \textit{step} we have to solve an EVI of the \textit {second kind} in $V_h$ to compute $u^{n+1}_h$.

\item {\bf Cranck-Nicholson scheme.} 
{\fontsize{10}{12}\selectfont
\begin{equation}
\begin{cases}
(\frac{u^{n+1}_h -u^n_h} {\Delta  t}, v_h -u^{n+1/2}_h)_h + a_h (n+
  \frac{1}{2}) \Delta  t ;\\ 
  \hspace{3cm}u^{n+1/2}_h +j_h (v_h) - j_h (u^{n+1/2}_h)
  \\ 
 \geq f^{n+1/2}_h, v_h - u^{n+1/2})_h \ \forall v_h \in V_h, u^{n+1/2}_h \in V_h,  u^{n+1/2}_h = \frac {u^n_h + u^{n+1}_h} {2},\\
  n=0,1,2,\ldots, u^0_h = u_{0h}.\tag{5.3}\label{c3:eq5.3}
\end{cases}
\end{equation}}\relax
\textit {Stability. Unconditional.}

Since\pageoriginale  $\dfrac{u^{n+1}_h - u^n_h} {\Delta  t} = \dfrac{u^{n+1/2}_h -u^n_h}{\frac{\Delta  t}{2}}$ we observe that at \textit {each step} we have to solve an EVI of the \textit {second kind} to compute $u^{n+1/2}_h$. If the regularity in time of the solution is poor we do not recommend this scheme.

\item {\bf Two steps implicit scheme.}
\begin{equation}
\begin{cases}
  \frac{\frac{3}{2} u^{n+1}_h - 2u_h^n +\frac{1}{2} u^{n-1}_h }{\Delta
    t}, v_h -u^{n+1}_h )_h +j_h (v_h)-j_h  (u^{n+1}_h)\\ 
  \hspace{3cm}+ a_h ((n+1)
  \Delta  t; u^{n+1}_h, v_h -u^{n+1}_h)  \\ 
  \geq (f^{n+1}_h, v_h -u^{n+1}_h) \ \forall v_h \in V_h,\\
  n = 0, 1, 2, \ldots, ; u^0_h = u_{0h}, u^1_h \text { given
  }.\tag{5.4}\label{c3:eq5.4} 
\end{cases}
\end{equation}
We use one of above schemes \eqref{c3:eq5.1}-\eqref{c3:eq5.3} to
compute $u^1_h$, starting from $u^0_h = u_{0h}$. \textit
{Stability. Unconditional.} 
\end{enumerate}

We have to solve \textit {at each step} an EVI of the \textit{second kind} in $V_h$ to compute $u^{n+1}_h$. Remark 3.4 applies for this scheme also.

\textbf {Comments}. The properties of stability and convergence of the
various schemes of Sec.~\ref{c3:s4},  
\ref{c3:s5} are studied in the references given in Sec. 1. In some
cases error estimates also have been obtained. 

In FORTIN [\ref{k46:e1}], G.L.T [\ref{k53:e2}, Chap. 6], applications
to more complicated 
PVI than \eqref{c3:eq2.1}, \eqref{c3:eq2.2} are also given. For the
numerical analysis of hyperbolic variational inequalities see
G.L.T[\ref{k53:e2},  Chap.6],  TREMO\-LIERES [\ref{k89:e1}]. 


\section[Application to a Specific Example:...]{Application to a
  Specific Example: Time Dependent Flow of a Bingham Fluid in a
  Cylindrical Pipe}\label{c3:s6} 

Following  GLOWINSKI [\ref{k51:e4}], we consider the time dependent
problem associated to the EVI of Chap.~\ref{chap2}. \ref{c2:s6}, and
study its asymptotic properties.  

\subsection{Formulation of the problem. Existence and uniqueness
  Theorem}\label{c3:ss6.1}  

Let $\Omega$ be a bounded domain of $\mathbb{R}^2$ with a smooth
boundary $\Gamma$. We consider:  
\begin{itemize}
\item $V = H^1_0 (\Omega) H = L^2 (\Omega)$, $V^* = H^{-1} (\Omega)$,
\item $a (u,v) = \int_\Omega$, $\Delta \nabla u. \Delta \nabla v dx$,
\item A time interval\pageoriginale  $[0, T]$, $0 < T < \infty$, 
\item $f \in L^2 (0, T; V^*)$, $u_0 \in H$,
\item $j (v) = \int_\Omega |\nabla v| dx$,
\item $\mu > 0$, $g > 0$.
\end{itemize}
We have then the following

\begin{theorem}\label{c3:thm6.1}%The 6.1
The PVI 
\begin{equation}
\begin{cases}
(\frac{\partial u} {\partial t}, v - u) + \mu a (u, v-u) +gj (v) \geq (f, v-u) 
\ \forall v \in V ~ a.e ~ t \in] 0, T [, \\
u (x, 0) = u_0 (x), \tag{6.1}\label{c3:eq6.1}
\end{cases}
\end{equation}
has a unique solution $u$ such that
\begin{equation*}
\begin{cases}
u \in L^2 (0, T;V) \cap C^0 ([0, T]; H),\\
\frac{\partial u} {\partial t} \in L^2 (0, T, V^*)
\end{cases}
\end{equation*}
{\em and this} $\forall u_0 \in H$, $\forall f \in L^2 (0, T; V^*)$.
For a proof of this see LIONS-DUVAUT [\ref{k42:e1}, Chap.6]. 
\end{theorem}

\subsection{The asymptotic behaviour of the continuous solution.} 

Assume that if $f$ is independent of $t$ and that $f \in L^2
(\Omega)$. We consider the following stationary problem 
\begin{equation}
\begin{cases}
\mu a  (u, v-u) + gj (v) -gj (u) \geq (f, v-u) \ \forall v \in V,\\
u \in V.
\end{cases}\tag{6.2}\label{c3:eq6.2}
\end{equation}

It is proved in LIONS-DUVAUT [\ref{k42:e1}, Chap.6] (see also Chap.2, Sec.6 of
these notes), that 
\begin{align*}
u & \equiv 0 \text { if } g \beta \geq ||f||_L{^2} (\Omega) , \tag{6.3}\label{c3:eq6.3}\\
\text{ where }\\
\beta  & =\inf_{v \varepsilon V} \frac{j (v)} {||v||_{L^2 (\Omega)}}. \tag{6.4}\label{c3:eq6.4}
\end{align*}
Then we\pageoriginale  can prove the following

\begin{theorem}\label{c3:thm6.2}%The 6.2
Assume that $f \in L^2 (\Omega)$ with $||f||_{L{^2} (\Omega)} < \beta g$, then if $u$ is the solution of  \eqref{c3:eq6.1}, we have
\begin{equation}
u (t) = 0 \text{ for } t \geq \frac{1}{\lambda_0 \mu} \log (1 + \lambda_0 \mu \frac{||u_0||_{L^2}} {\beta g - ||f||_{L^2}}) \tag{6.5}\label{c3:eq6.5}
\end{equation}
where $\lambda_0$ is the smallest eigenvalue of $- \Delta$ in $H^1_0 (\Omega) (\lambda_0 > 0)$.
\end{theorem}

\begin{proof}
We use $|\cdot|$ for the $L^2 (\Omega)$ -norm and $||\cdot||$ for the $H^1_0 (\Omega)$-norm. Since $f \in L^\infty (\mathbb{R}^+, L^2 (\Omega))$ it follows from Theorem 6.1 that the solution of \eqref{c3:eq6.1} is defined on the whole of $\mathbb{R}^+$.
\end{proof}

We observe now that if $g \beta > |f|$ the zero is the unique solution
of \eqref{c3:eq6.2}; if follows then from Theorem~\ref{c3:thm6.1} that
if $u(t_0) = 0$ for some $t_0 \geq 0$ then 
\begin{equation}
u (t) = 0 \forall t \geq t_0. \tag{6.6}\label{c3:eq6.6}
\end{equation}
Taking $v=0$ and $v = 2 u$ in \eqref{c3:eq6.1} we obtain
\begin{equation}
(\frac{\partial u} {\partial t}, u) + \mu a (u, u) + gj (u) = (f, u) ~ a.e. ~ \text { in } t. \tag{6.7}\label{c3:eq6.7}
\end{equation}

But since $v \in L^2 (0, T; V)$, $v' \in L^2 (0, T, V^*)$ implies (this is a general result) that $t \to |v(t)|^2$ is \textit {absolutely continuous} with $\frac{d}{dt} |v|^2 = 2 (\frac{dv}{dt}, v)$; we obtain from  \eqref{c3:eq6.7} that
\begin{equation}
\begin{cases}
\frac{1}{2} \frac{d}{dt} |u|^2 + \mu a (u, u) + gj (u) & = (f, u) \\
 & \leq |f| \cdot |u| ~ a.e  ~ \text{ in } t. \tag{6.8}\label{c3:eq6.8}
\end{cases}
\end{equation}

Since $a (v, v) \geq \lambda_0 |v|^2 \ \forall v \in V$, and $j (v) \geq \beta |v| \ \forall v \in V$ (from\eqref{c3:eq6.4}), we obtain from \eqref{c3:eq6.8} that 
\begin{equation}
\frac{1}{2} \frac{d}{dt} |u|^2 + \mu \lambda_0 |u|^2 + (g \beta - |f|) |u| ~ a.e ~ \text{ in } t \in \mathbb{R}^+. \tag{6.9}\label{c3:eq6.9}
\end{equation}

Assume that $u (t) \neq 0 \forall t \geq 0$; since $t \to |u (t)|^2$ is absolutely continuous with $|u (t)| > 0$ it follows that $\to |u (t)|$ is also absolutely continuous. Therefore \eqref{c3:eq6.9} we obtain
\begin{equation}
\frac{d}{dt} |u (t)| + \mu \lambda_0 |u (t)| + (g \beta - |f|) \leq 0 ~ a.e ~ t \in \mathbb{R}^+. \tag{6.10}\label{c3:eq6.10}
\end{equation}\pageoriginale 
It follows from \eqref{c3:eq6.10} that
\begin{equation}
\frac{\frac{d}{dt} |u (t)|} {|u (t)| + \frac{g \beta - |f|}{\mu \lambda_0}} \leq - \mu \lambda_0 ~ a.e. ~ t \in \mathbb{R}^+. \tag{6.11}\label{c3:eq6.11}
\end{equation}
Define $\gamma$ by $\gamma = \dfrac {g \beta - |f|} {\mu \lambda_0}$, then $\gamma > 0$. It follows then by integrating \eqref{c3:eq6.11} that 
\begin{equation}
|u (t)| + \gamma \leq  (|u_0| + \gamma) e^{-\mu \lambda_0 t} \ \forall t \in \mathbb{R}^+ ; \tag{6.12}\label{c3:eq6.12}
\end{equation}
\eqref{c3:eq6.12} is absurd for $t$ large enough. Actually we have $u (t) = 0$ if
\begin{align*}
-  \gamma & \geq (|u_0| + \gamma) e^{- \mu \lambda_0 t}, \\
\text{ i. e. }\\
t & \geq \frac{1}{\lambda_0  \mu} \log (1 + \frac{\lambda_0 \mu || u_0 || _{L^2 (\Omega)}} {g \beta - ||f ||_{L^2 (\Omega)})}. \tag{6.13}\label{c3:eq6.13}
\end{align*}

\begin{exercise} %Exe 6.1
Let $f \in L^2 (\Omega)$ with possibly $|f| \geq g \cdot \beta$. Let us denote by $u_\infty$ the solution of \eqref{c3:eq6.2}; theorem prove that
$$
|u (t)- u_\infty| \leq |u_0 - u_\infty| e^{-\lambda_0 \mu t}
$$
where $u(t)$ is the solution of \eqref{c3:eq6.1} .
\end{exercise}

\subsection{On the asymptotic behaviour of the discrete solution.}

We still assume that $f \in L^2 (\Omega)$. To approximate \eqref{c3:eq6.1} we proceed as follows : assuming that $\Omega$ is a polygonal domain, we use the same approximation with regard to the space variables  as in Chap.~\ref{chap2}, Sec.~\ref{c2:s6} (i.e. by means of piecewise linear finite elements, see Chap.~\ref{chap2}, Sec.~\ref{c2:s6}). Hence we have
\begin{equation*}
\begin{cases}
a_h (u_h, v_h) = a (u_h, v_h) \ \forall u_h, v_h \in  V_h, \\
j_h (v_h) = j (v_h) \ \forall v_h \in V_h,
\end{cases}
\end{equation*}
and\pageoriginale  from the formulae of Chap.~\ref{chap2}, Sec.~\ref{c2:s7} we can also   take
$$
(u_h, v_h)_h = (u_h, v_h) \ \forall u_h, v_h \in V_h.
$$

Then we approximate \eqref{c3:eq6.1} by the \textit{implicit scheme} \eqref{c3:eq5.2} and we obtain
\begin{equation}
\begin{cases}
(\frac{u^{n+1}_h -u^n_h} {\Delta  t}, v_h -u^{n+1}_h) + \mu \int_\Omega \nabla u_h ^{n+1} \cdot \nabla (v_h - u^{n+1}_h) dx + j (v_h) -j (u^{n+1}_h) \\
 \geq (f_h, v_h -u^{n+1}_h) \forall v_h \in V_h, u^{n+1}_h \in V_h ; n = 0, 1, 2, \ldots, ; u^0_h = u_{0h}. \tag{6.14}\label{c3:eq6.14}
\end{cases}
\end{equation}
We assume that $u_{0h} \in V_h \ \forall h$ and
\begin{equation}
\lim_{h \to 0} u_{0h}  = u_o \text { strongly in } L^2 (\Omega). \tag{6.15}\label{c3:eq6.15}
\end{equation}
Similarly we assume that $f$ is approximated by $(f_h)_h$ in such a way that $(f_h, v_h)$ can be computed easily and
\begin{equation}
\lim_{h \to 0} f_h = f \text { strongly in } L^2 (\Omega). \tag{6.16}\label{c3:eq6.16}
\end{equation}

\begin{theorem}\label{c3:thm6.3} %The 6.3
Let $|f| < \beta g$. If \eqref{c3:eq6.15} and \eqref{c3:eq6.16} hold, then if $h$ is sufficiently small we have $u_h^n = 0$ for $n$ large enough.
\end{theorem}

\begin{proof}
As in the proof of Theorem~\ref{c3:thm6.2}, taking $v_h = 0$ and $v_h = 2u^{n+1}_h$ in \eqref{c3:eq6.14} we obtain
{\fontsize{10}{12}\selectfont
\begin{equation}
\left(\frac{u^{n+1}_h - u^n_h} {\Delta  t}, u^{n+1}_h\right) + \mu
\int_\Omega |\nabla u^{n+1}_h|^2 dx + g \int_\Omega |\nabla u^{n+1}_h
dx = \int_\Omega f_h u^{n+1}_h dx \ \forall n \geq 0;
\tag{6.17}\label{c3:eq6.17} 
\end{equation}}\relax
using Schwarz inequality in $L^2 (\Omega)$, if follows from
\eqref{c3:eq6.17} that 
\begin{equation}
\frac{|u^{n+1}_h | - |u^n_h|} {\Delta  t} |u^{n+1}_h| + \mu \lambda_0
| u^{n+1}_h|^2 + (g \beta - |f_h|) |u^{n+1}_h| \leq 0 \ \forall n \geq
0. \tag{6.18}\label{c3:eq6.18} 
\end{equation}
\end{proof}

Since $f_h \to f$ \textit{ strongly } in $L^2 (\Omega)$ we have
\begin{equation}
 g \beta - |f_h| > 0 \text { for } h  \text { sufficiently small }. \tag{6.19}\label{c3:eq6.19}
\end{equation}
It follows then from \eqref{c3:eq6.18}, \eqref{c3:eq6.19} that
\begin{equation}
u^{n_0}_h = 0 \Rightarrow u^n_h = 0 \text{ for } n \geq n_0 \text{ if } h \text{is small enough}. \tag{6.20}\label{c3:eq6.20}
\end{equation}\pageoriginale 
Assume that $u^n_h \neq 0 \forall n$; then \eqref{c3:eq6.18} implies
\begin{equation}
\frac{|u^{n+1}_h|-|u^n_h|} {\Delta  t} + \mu \lambda_0 |u^{n+1}_h| + g \beta - |f_h| \leq 0 \forall n \geq 0. \tag{6.21}\label{c3:eq6.21}
\end{equation}
We define $\gamma_h$ by $\gamma_h = g \beta - |f_h|$ then,
\begin{equation}
\gamma_h > 0 \text { for } h \text{ small enough and } \lim_{h \to 0}
\gamma_h = \gamma = g \beta - |f|. \tag{6.22}\label{c3:eq6.22} 
\end{equation}
It follows follows from \eqref{c3:eq6.21} that
$$
\left(|u^{n+1}_h| + \frac{\gamma_h}{\lambda_0 \mu}\right) (1+\lambda_0 \mu \Delta
t) \leq |u^n_h| + \frac{\gamma_h}{\lambda_0 \mu} \ \forall n \geq 0 
$$
which implies that
\begin{equation}
\left(|u^n_h| + \frac{\gamma_h}{\lambda_0 \mu}\right) \leq
(1+\lambda_0 \mu \Delta  t)^{-n} \left(|u^0_h| + \frac{\gamma_h}{\lambda_0
  \mu}\right).\tag{6.23}\label{c3:eq6.23} 
\end{equation}

Since $\gamma_h > 0$ for $h$ small enough, \eqref{c3:eq6.23} is \text{impossible} for $n$ large enough. More precisely we shall have $u^n_h = 0$ if
$$
\frac{\gamma_h}{\lambda_0 \mu} \geq (1+\lambda_0 \mu \Delta  t)^{-n}
\left(|u^0_h| +\frac{\gamma_h} {\lambda_0 \mu}\right),
$$
which implies:
\begin{equation}
\text{ If } h \text{ is small enough, then } u^n_h = 0 \text{ if } n
\geq \frac{\log (1 + \lambda_0 \mu \frac{|u^0_h|}{\gamma_h})}{\log (1+
  \lambda_0 \mu \Delta  t).} \tag{6.24}\label{c3:eq6.24} 
\end{equation}

Relation \eqref{c3:eq6.24} makes the statement of Theorem 6.3 more
precise. Moreover in terms of \textit{time}, \eqref{c3:eq6.24} implies
that $u^n_h$ is equal to zero if 
\begin{equation}
 n \Delta  t \geq \Delta  t \frac{\log (1 + \lambda_0 \mu
   \frac{|u^0_h|}{\gamma_h})}{\log (1+ \lambda_0 \mu \Delta
   t)}. \tag{6.25}\label{c3:eq6.25} 
\end{equation}
We observe that
$$
\lim_{\substack{h \to 0 \\ \Delta  t \to 0}}  \Delta t \frac{\log (1 +
  \lambda_0 \mu \frac{|u^0_h|}{\gamma_h})} \log \left(1+ \lambda_0 \mu
  \Delta  t = \frac{1}{\lambda_0 \mu} \log (1 + \lambda_0 \mu)
\frac{|u^0|}{\gamma}\right) 
$$\pageoriginale 

Hence taking the limit in \eqref{c3:eq6.25} we obtain another proof
(assuming that $u^n_h$ converges to $u$ in some topology) of the
estimate \eqref{c3:eq6.5} given in the statement of
Theorem~\ref{c3:thm6.2}. 

\begin{exercise}\label{c3:exer6.2}%Exe 6.2
Let $u^\infty_h$ be the solution of the time independent problem associated to $f_h$, possibly with $|f_h| \geq \beta\cdot g$, then prove that
$$
|u^n_h - u^\infty_h| \leq (1+2 \mu \lambda_0 \Delta  t)^{-n/2} |u^0_h - u^\infty_h| \quad n \geq 0.
$$
\end{exercise} 

\subsection{Remarks}\label{c3:ss6.4}

\begin{remark}\label{c3:rem6.1}%Rem 6.1
We can generalize Theorem~5.1 to the case of a Bingham
flow in a 2-dimensional bounded cavity. 
\end{remark}

\begin{remark}\label{c3:rem6.2}%Rem 6.2
In GLOWINSKI [\ref{k51:e5}], BRISTEAU[\ref{k21:e1}],
BEGIS[\ref{k6:e1}], numerical 
verification of the 
above asymptotic properties have been performed and found to be
consistent with the theoretical predictions. 
\end{remark}

\begin{remark}\label{c3:rem6.3}%Rem 6.3
One may find in H.BREZIS [\ref{k14:e5}], many results on the
asymptotic behaviour of various PVI as $t \to \infty$.  
\end{remark}
