\chapter{Prims in Short Intervals and Short Arithmetic
  Progressions}\label{chap6}%chap 6 

IN\pageoriginale THIS FINAL chapter, we shall demonstrate that the
sieve method can 
actually detect prime numbers in some very difficult and important
situations, if it is correctly combined with analytical means. 

In the first section, we shall inuect Iwaniec's linear sieve into the
study of primes in short intervals and prove a remarkable result
which, in spite of much  efforts, has  never been  attained by the
sole use of analytical means. On the other hand, in the second
section, we shall empoly Selberg's sieve to prove a deep result
pertaining to the least prime in  an arithmetic progression, and
illustrate the versatility of this fundamental sieve method. 

\section{Existence of Primes in Short Intervals}\label{chap6-sec6.1}%sec 6.1

As is well-Known, THEOREM \ref{chap5-thm14}  or rather \eqref{eq5.1.18} yields 
\begin{equation*}
\pi (x) - \pi (x- x^{\theta}) =(1+0(1)) \frac{x^{\theta}}{\log x}
\tag{6.1.1}\label{eq6.1.1}
\end{equation*}
Whenever $ \theta > 7/12$. This implies of course
\begin{equation*}
  p_{n+1} - p_n \ll p_{n}^{\frac{7}{12}+\epsilon },
  \tag{6.1.2}\label{eq6.1.2} 
\end{equation*}
$p_n$ being the $n$-th prime.

Our\pageoriginale aim is to show that if we geve up an asymptotic
estimate but ask 
for a positive lower bound for $\pi (x) - \pi (x-x^\theta ) $, then
the value of $ \theta$ can be taken less then $7/12$, so that
\eqref{eq6.1.2} can be improved. 
 
Adopting the notations introduced in the second chapter,  we may write
  $$
  \pi (x) - \pi ( x-h) = S\left(A, x^{\frac{1}{2}}\right),
 $$
 where
 $$
  A= \{ n; x -h \leq n < x\}, x^\theta ( \theta < 1),
 $$
 and $\Omega$ is   the simplest one: $\Omega (p) \ni n$ implies $p |
 n$. Thus Buchstab's formula \eqref{eq2.1.1} gives, for any $2 \leq z <
 x^{\frac{1}{2}}$, 
 $$
 \pi (x) - \pi (x-h) = S(A, z)- \sum_{z \leq q < x^{\frac{1}{2}}}
 S(A_q, q). \qquad  (\footnote{In this section, the letter $q$ stands for
   prime numbers.}) 
 $$
 
 The remarkabale fact in this identity is that we can compute
 asymptotically the sum 
  \begin{equation*}
\sum_{Q \leq q < 2Q} S(A_q, q) \tag{6.1.3}\label{eq6.1.3}
 \end{equation*} 
for some $\theta$ which is definitely smaller than $7/12$, if $Q$ is
in a certain range. 
  
To\pageoriginale show this, we set
 \begin{equation*}
   x^{\frac{1}{3}} < Q \leq x^{\frac{1}{2}}, \theta > \frac{1}{2},
   \tag{6.1.4}\label{eq6.1.4} 
 \end{equation*}
 and we assume hereafter that $x$ is sufficiently large. Then
 \eqref{eq6.1.3} is obviously equal to  
$$
\sum_{Q \leq q < 2Q } \left(\pi \left( \frac{x}{q}\right) - \pi \left(
\frac{ x-h}{q}\right)\right). 
$$

But it is easy to see that this is also equal to 
\begin{equation*}
  \frac{1}{\log \frac{x}{Q}} \sum_{ Q \leq q < 2Q } \left( \psi
  \left(\frac{x}{q}\right) - \psi \left(\frac{ x-h}{q}\right)\right) +
  o\left(\frac{h}{(\log x)^3}\right) \tag{6.1.5}\label{eq6.1.5} 
\end{equation*}  
where $\psi $ is usual Chebyshev function. Replacing $\psi$ by its
explicit formula we get readily 
\begin{multline*}
  \sum_{ Q \leq q < 2Q} \left(\psi \left(\frac{x}{q}\right) - \psi
  \left(\frac{x-h}{q}\right)\right)\\ 
  = h U (1) - \sum _{\substack {| \gamma |^\rho < T\\ \beta >0}}
  U (\rho )\frac{x^\rho -(x-h)^\rho}{\rho} + 0 \left( \frac{x (\log
    X)^3}{T}\right). 
\end{multline*}

Here $\rho = \beta + i \gamma $ is a cmoplex zero of $\zeta (s)$, and 
$$
U (s) = \sum_{Q \leq q < 2Q} q^{-s};
$$
also
\begin{equation*}
  T= x^{1 -\theta + \eta}, \tag{6.1.6}\label{eq6.1.6}
\end{equation*}   
where $\eta$ is a small positive constant. This sum over $\rho $ is
$$
\ll h \int_{0}^{1- ( \log T ) ^{-\frac{3}{4}}} x^{\alpha -1}
\sum_{\substack{ | \gamma | < T \\ \beta > \alpha}}  | U (\rho ) |
d \alpha,  
$$
because\pageoriginale of Vinogradov's zero-free region. Thus appealing
to LEMMA \ref{chap5-lem25} 
we can infer that this is $o (h (\log x)^{-10})$ either, if $Q \geq
T$ and $T^{6/5 + \epsilon} Q \leq x^{1-\eta}$ or, if $T^{4/5} \leq Q
\leq T$ and $T^{16/5 + \epsilon } < Qx^{1-\eta}$. Namely, if  
\begin{equation*}
x^{\frac{11 - 16\theta}{5} +  4 \eta}  \leq Q \leq x ^{\frac {6\theta
    -1}{5}- 2 \eta} \tag{6.1.7}\label{eq6.1.7} 
\end{equation*}   
and 
\begin{equation*}
\frac{6}{11} + 2 \eta \leq \theta \leq \frac{7}{12},
\tag{6.1.8}\label{eq6.1.8} 
\end{equation*}   
then we have
\begin{equation*}
  \sum_{Q \leq q < 2Q} \left(\psi \left(\frac{x}{q}\right) - \psi
  \left(\frac{x-h}{q}\right)\right) = h
  U (1) (1+o((\log x)^{-5})). \tag{6.1.9}\label{eq6.1.9} 
\end{equation*}

We should note here that \eqref{eq6.1.7} and \eqref{eq6.1.8} imply
\eqref{eq6.1.4}.  
 
Now we set
\begin{equation*}
z=x ^{\frac{11-16 \theta}{5} + 4 \eta}, \tag{6.1.10}\label{eq6.1.10}
 \end{equation*} 
 and put
 \begin{equation*}
 Z = x^{\frac{6 \theta - 1}{5} - 2 \eta}. \tag{6.1.11}\label{eq6.1.11}
 \end{equation*} 

 Then, by \eqref{eq6.1.5}, \eqref{eq6.1.9} and by partial summation, we get
 $$
  \sum_{z \leq q < Z} \left(\pi \left(\frac{x}{q}\right) - \pi \left(
  \frac{x-h}{q}\right)\right) =  (C_2 ( \theta ) + o (\eta)) \frac{h}{\log x} 
 $$
 where\pageoriginale $\theta$ satisfies \eqref{eq6.1.8} and
 $$
 C_2(\theta) = \log \left(\frac{(6\theta -1 ) ( 8 \theta
   -3)}{3(1-\theta)(11 - 6 \theta)}\right). 
 $$

 Thus we have
 \begin{equation*}
 \pi (x) - \pi (x-h) = S(A, z) - (C_2( \theta ) +o(\eta )) (h/ \log
 x) - \sum_{ Z \leq q < x^{\frac{1}{2}}} S(A_q, q), \tag{6.1.12}\label{eq6.1.12} 
 \end{equation*} 
 provioded \eqref{eq6.1.8} holds.
 
 Next, we appeal to THEOREM \ref{chap3-thm10}; we set there $A,
 \Omega$ as above, $\delta \equiv 1$, $X = h$, and 
 $$
 R_d = [ x / d] - [(x - h)/d]-(h/d).
 $$

Obviously, all conditions required there are amply satisfied. Hence we
have, for any $M$, $N \geq 1$ such that $MN \geq z^2$, 
\begin{equation*}
  S(A, z) \geq \frac{e^{-\gamma }h}{\log z} \left( \phi _0 \left(\frac{\log
    MN}{\log z}\right) - o(1)\right) 
- \log z \sup _{\alpha,  \beta}| \sum_{\substack{m < M  \\ n <
    N}}\alpha _m \beta _n R_{mn}| \tag{6.1.13} \label{eq6.1.13}
\end{equation*}
where $\gamma$ is the Euler constant, and $| \alpha _ m | \leq 1, |
\beta _n | \leq 1$.  

Now we introduce the crucial
\begin{Lemma}\label{chap6-lem27}%27
Let $Z$ be as in \eqref{eq6.1.11}, and let
  \begin{equation*}
    \frac{11}{20} + 2 \eta \leq \theta \leq
    \frac{7}{12}. \tag{6.1.14}\label{eq6.1.14} 
  \end{equation*}

  Then\pageoriginale we have
  $$
  | \sum_{m, n < Z x^\epsilon } a_m b_n R_{mn} | \ll hx^{-c \eta ^{3}}
  $$
  for any $\{ a_m\}, \{ b_n\}$ such that $ | a_m |, | b_n | < x^{\epsilon }$.
\end{Lemma}

Before giving the proof, let us see  the implication of this for our
problem: We may set $M = N = Z $ in \eqref{eq6.1.13} provided $\theta$
satisfies \eqref{eq6.1.14}, which we shall assume henceforth. Then, we note
that \eqref{eq6.1.10}, \eqref{eq6.1.11} imply $2 < 2 \log Z/ \log z <
4$, and that 
$\phi _0 (u) = \frac{ 2 e ^ \gamma}{u} \log (u-1)$ for $2 \leq u \leq
4 $ because of \eqref{eq3.2.10}. Thus we have 
$$
S(A, z)\geq (1 - o(\eta )) C_1 ( \theta) \frac{h}{\log x},
$$
where
$$
C_1 (\theta) = \frac{5}{ 6 \theta-1} \log \left(\frac{28 \theta - 13}{11 -
  16 \theta}\right).
$$

This and \eqref{eq6.1.12} give 
\begin{equation*}
\pi (x) - \pi (x-h) \geq (C_1 ( \theta ) - C_2 ( \theta ) - o(\eta))
\frac{h}{\log x}- \sum_{z \leq q < x^{1_{/_{2}}}} S(A_q, q),
\tag{6.1.15} \label{eq6.1.15}
\end{equation*}
provided \eqref{eq6.1.11} and \eqref{eq6.1.14} hold.

Now let us proceed to the proof of  LEMMA
\ref{chap6-lem27}. Obviously, it suffices to consider the estimate of  
$$
E (A, B)= \sum_{\substack{A \leq m < 2A\\ B \leq n < 2B}}  a_m b_n R_{mn}
$$
under the assumption
\begin{equation*}
  AB \geq hx^{-\eta}; A, B \leq Zx^ \epsilon.  \tag{6.1.16}\label{eq6.1.16}
\end{equation*}\pageoriginale

We put
$$
A(s) = \sum_{A \leq m < 2A} a_m m^{-s}, B(s) = \sum_{B \leq n < 2B} b_n
n ^{-s}, L(s) = \sum_{\frac{L}{8}\leq \ell < L} \ell^{-s},
$$
where
\begin{equation*}
L = \frac{x}{AB} \geq x ^{3 _{\eta}}, \tag{6.1.17}\label{eq6.1.17}
\end{equation*}
because of \eqref{eq6.1.11}, \eqref{eq6.1.14} and
\eqref{eq6.1.16}. Then, by Perron's inversion formula, we get  
{\fontsize{10pt}{12pt}\selectfont
$$
 E(A, B)= \frac{1}{2 \pi i} \int_{a - i T}^{a + i T} A(s) B(s) L (s)
 \frac{x^s - ( x- h ) ^ s}{s} ds - h(A(1)B(1) + 0 (x^{- \eta /2})), 
$$}\relax
where $a = 1 + (\log x) ^{-1}$, and $T$ is as \eqref{eq6.1.6}. We divide this
integral into two	 parts, according to $| t | < \sqrt{L}$ and $
\sqrt{L} \leq | t | \leq T$. And we observe the following: If $| t | <
\sqrt{L}$, then, by \eqref{eq4.1.3}, we  have, for $\sigma = \alpha$, 
$$
L(s) = \frac{L^{1-s} - ( L / 8 ) ^{1-s}}{1-s} + 0(L^{-1}),
$$
and also 
$$
 \frac{x^s-(x-h)^s}{s} = hx^{s-1}+ o(|s| h^2 x^{-1}),
$$
since $| 6 | hx^{-1} < x^{-\epsilon }$, because of
\eqref{eq6.1.16}. Inserting these into  
$$
\int\limits_{a-i \sqrt{L}}^{a+i \sqrt{L}} A(s) B(s)L(s) \frac{x^s -(x-h)^s}{s}ds,
$$
we\pageoriginale see readily that this is equal to 
\begin{align*}
\frac{h}{2 \pi i} \int\limits_{a-i \sqrt{L}}^{a+i \sqrt{L}}
& \frac{L^{1-s}- (L/8)^{1-s}}{1-s} A(s)B(s) x^{s-1} ds + o (hx ^{-\eta
  /2})\\  
& = hA(1)B(1) + o(hx^{-\eta /2}). 
\end{align*}

Hence we get
$$
E(A, B) = \frac{1}{2 \pi i} \left\{ \int\limits_{ a - i T}^{a - i
  \sqrt{L}} \int\limits_{a+i \sqrt{L}} ^{ a+ i T} \right\} A(s)B(s)L(s)
\frac{x^s - (x-h)^s}{ s} ds + o(hx ^{- \eta /2}). 
$$

Then it is apparent that there exists a set $\{ t_r \}$ such that $
\sqrt{L} \leq | t_r | \leq T, | t_r - t_r,  | \geq 1$ if $r \neq r'$,
and  
$$
E(A,B) \ll h \sum _{r} |A(a+it_r) B(a+it_r) L(a+it_r)|+ hx^{-\eta /2}
$$

Now let $S(U,V,W)$ be the nmber of $t_r$ such that $\nu < |A(a+ i t_r)|
\leq 2\nu, W < |B (a + i t_r)| \leq 2W, U < | L(a+it_r ) | \leq
2 U$ hold simultaneously. Here, we can, of course, assume that  
$$
| \log U |, | \log V |, \log W | \ll \log x.
$$

Thus
\begin{equation*}
E (A, B) \ll h ( \log x )^3  \max_{U,V,W} UVWS (U,V,W) + hx ^{-\eta
  /2}. \tag{6.1.18} \label{eq6.1.18}
\end{equation*}

To estimate $S(U,V,W)$, let us see first an implication of $U < | L (a +
it_r) | \leq 2 U$. For this sake, let $\sqrt{L} \leq T_1 \leq T$,
and let $S_1$ be the set of  all $t_r$ such that $ T_1 \leq t_r <
2T_1, U < | L (a + it_r)| \leq 2U$. We note that, for any $T_i \leq
t < 2 T_1$,   
$$
 L(a+it) = \frac{1}{2 \pi i} \int\limits_{\epsilon  - i
   T_1/2}^{\epsilon  + iT_1/2} \zeta(w +_a + it) \frac{L^w -
   (L/8)^w}{w}dw + 0 \left( \frac{ \log x}{T_1}\right). 
$$\pageoriginale

Thus shifting the line of integration to $Re(w) = \frac{1}{2}-a, | Im
(w) | \leq T_1 /2 $, we get 
$$
 | L (a+it)| \ll L^{-1_{/_{2}}} \int\limits_{-T_1/2}^{T_1/2}| \zeta
 \left(\frac{1}{2} + i(u +t)\right) | \frac{du}{1+ | u + t |} + L^{-{1/2}}
 \log x.  
$$

Raising both sides to $4 - th$ power and using H\"{o}lder's inequality, we have 
$$
|L (a+it)|^4\ll L^{-2}\log^3 x \int\limits_{-T_1 /2}^{T_1 /2} | \zeta
\left(\frac{1}{2} + i (u + t)\right)|^4  \frac{du}{1 + |u + t|}+
L^{-2} \log^4 x,  
$$
and thus
{\fontsize{10pt}{12pt}\selectfont
$$
|S_1 | U^4 \ll L^{-2 } \log^3 x \int_{- T_1/2}^{T_1/2} \sum_{ t_r
  \epsilon  S_1}|\zeta \left(\frac{1}{2} + i(u +t_r)\right)|^4 \frac{du}{1
  +|ut_r|}+T_1 L^{-2} \log^4 x. 
$$}\relax

Hence, by (i) of LEMMA \ref{chap5-lem23}, we obtain
$$
| S_1 | \ll L^{-2}U^{-4}T_1 \log^c x.
$$

This implies  obviously
$$
S(U,V,W) \ll L^{-2 }\cup ^{-4} T \log ^c x.
$$

Other estimates of $S(U,V,W)$ can be obtained by (ii) of LEMMA
\ref{chap5-lem23} and LEMMA \ref{chap5-lem24}, and we find readily 
\begin{equation*}
S(U,V,W) \ll x^\epsilon  F_1 \tag{6.1.19}\label{eq6.1.19}
\end{equation*}
with
\begin{multline*}
  F_1 = \min 
  \left\{\frac{1}{V^2} +\frac{T}{V^2A}, \frac{1}{W^2} +
  \frac{T}{W^2 B}, \frac{1}{V^2} + \frac{T}{V^6 A^2}, \frac{1}{W^2}+\right.\\
  \left. \frac{T}{W^6 B^2}, \frac{T}{U^4 L^2}, \frac{1}{U^4} +
  \frac{T}{U^{12}L^4} \right\}. 
\end{multline*}\pageoriginale

Then we consider the following four cases separately:

\noindent 
(i) $F_1 \ll V^{-2}$, $W^{-2}$, (ii) $F_1 \gg V^{-2}$, $W^{-2}$, (iii)
$F_1 \ll V^{-2}$, $F_1 \gg W^{-2}$, (iv) $F_1 \gg V^{-2}$, $F_1 \ll
W^{-2}$. If (i) holds, we have $F_1 \ll (UVW)^{-1}U$. But LEMMA
\ref{chap4-lem17} yields 
$$
U \ll \exp\left(-c \frac{(\log L)^3}{(\log T )^2}\right) \ll x^{-cn^3},
$$
because of \eqref{eq6.1.17}. Hence in case (i), we have
\begin{equation*}
UVWF_1 \ll x^{-cn^3}. \tag{6.1.20}\label{eq6.1.20}
\end{equation*}

On the other hand, in the cases (ii) - (iV) we can argue just as in
the estimation of $F$ treated in \S~ \ref{chap4-sec4.3}, and we get readily 
\begin{align*}
  F_1 & \ll  (UVW)^{-1}  \left\{ x^{-\frac{7}{16}} T^{\frac{31}{32}} +
  x^{-\frac{9}{20}} T \right\}, \\ 
  F_1  & \ll  (UVW)^{-1}  \left\{ x^{-\frac{3}{8}}T^{\frac{7}{16}}
  B^{\frac{3}{8}} + x^{-\frac{5}{12}} T^{\frac{1}{2}}
  B^{\frac{5}{12}} \right\}, \\ 
  F_1  & \ll  (UVW)^{-1} \left\{ x^{-\frac{3}{8}} T^{\frac{7}{16}}
  A^{\frac{3}{8}} + x^{-\frac{5}{12}} T^{\frac{1}{2}} A^{\frac{5}{12}}
  \right\},  
\end{align*}
respectively. And by virtue of \eqref{eq6.1.14} and \eqref{eq6.1.16}
these are all $0(x^{-\eta/4})$. Thus by this and \eqref{eq6.1.18} -
\eqref{eq6.1.20}, we obtain the assertion of the lemma. 

Now, returning to \eqref{eq6.1.15}, we have to seek for a good upper bound for 
$$
 \sum_{Z \leq q < x^{1/2}} S (A_q, q)
$$\pageoriginale
so that the left side of \eqref{eq6.1.15} is positive for a $\theta$ in the
range \eqref{eq6.1.14}. 

For this sake, we consider the sums
$$
 \sum_{Q \leq q < 2Q}  S(A_q, q),\, Z \leq  Q < x^{1/2}.
$$

Obviously, this is not greater than
$$
\frac{1}{\log Q} \sum_{Q \leq q < 2Q}  S(A_q, (Z^2/Q)^{\frac{1}{3}})  \log q.
$$

Then to each summand we apply \eqref{eq3.4.20} with $\nu = 1$ and 
\begin{equation*}
y = Z^2/Q; \tag{6.1.21}\label{eq6.1.21}
\end{equation*}
in particular, the function $\Theta_{\nu} (K)$ is defined by \eqref{eq2.3.4}
with $z = (Z^2/Q)^{1/3}$ and $y = Z^2/Q$, and, of course, independent
of $q$. Thus we have, on noting $\phi_1(s) = 2e^{\gamma}/s$ for $s
\leq 3$, 
\begin{align*}
 \sum_{Q \leq q < 2Q} & S(A_q, (Z^2/Q)^{1/3}) \log q\\
   &\leq (1 + o(1)) \frac{2h}{\log (Z^2/Q)}\sum_{Q\leq q < 2Q} \frac{\log q}{q}\\
   &+ \sum_K (-1)^{\omega (K)}\Theta_1 (K) \sum_{\substack{d
      \epsilon  K \\ f < z^{\tauup, f | P(z_1)} \\ Q \leq q < 2Q}}
  \xi_f^{(1 + \omega (K) )} R_{dfq} \log q\\ 
   &+ \sum_{\substack{I < K \\ \omega (K) \equiv 0 \pmod{2}}} \Theta_1
  (KI) \sum_{\substack{d \epsilon  K \\ p, p' \epsilon  I\\ f <
      z^{\tauup},f | P(z_1)\\Q \leq q < 2Q}} \xi^{(1)}_f  R_{dpp' fq}
  \log q, \tag{6.1.22} \label{eq6.1.22}
\end{align*}
in\pageoriginale which $\tau, z_1$, and the mode of dissection of the interval
$(z_1, z)$ are as in \S~ \ref{chap3-sec3.4}. Here we should note that, more
precisely, we should have written $(R_q)_{df}$ and $(R_q)_{dpp'f'}$
instead of $R_{qdf}$ and $R_{qdpp'f}$ respectively, but our present
choice of $q$ allows us to put the foumula as above. 

Now let us estimate
$$
E = \sum_{\substack{d \epsilon  K\\f < Z^{\tauup},f | P(Z_1)\\Q \le
    q < 2Q}} ~ \xi^{(1+\omega(K))}_f~ R_{dfq} ~ \log~ q,  \Theta_1~(K)
= 1. 
$$

It will turn out that this can be reduced to an application of LEMMA
\ref{chap6-lem27}. To this end, we transform the factor $\log q$, $q$
being a prime, 
into a sum of certain arithmetic functions. Let us put, for $\sigma >
1$, 
\begin{align*}
\sum^{\infty}_{n=1} ~ \lambda^{(1)}(n)n^{-s} & = \sum_{q > U} ~ q^{-s}~ \log ~ q,\\
\sum^{\infty}_{n=1} ~ \lambda^{(2)}(n)n^{-s} &= -\sum_{n > U} ~
(\sum_{\substack{r | n \\ r \leq U}}~\mu (r) ) ~ n^{-s},\\ 
\sum^{\infty}_{n=1} ~ \lambda^{(3)}(n)n^{-s} & = - \zeta'(s)M(s) +
(G(s) + N(s))(1-\zeta (s)M(s))\tag{6.1.23}\label{eq6.1.23}
\end{align*}
with
\begin{align*}
M(s) & = \sum_{n \leq U} ~ \mu(n)n^{-s},\\
N(s) & = \sum_{q \leq U} ~ q^{-s} ~ \log ~ q,\\
G(s) & = \sum_{q} ~ \frac{ \log ~ q}{q^s(q^s - 1}).
\end{align*}\pageoriginale

Then it is easy to see that, for any $U \ge 1$,
$$
\sum_{q} ~ q^{-s} ~ \log ~ q ~ = \sum_{n=1}^{\infty} ~ (\lambda^{(1)}
* \lambda^{(2)}(n)n^{-s} + \sum_{n=1}^{\infty}\lambda^{(3)}(n)n^{-s}, 
$$
i.e.,
\begin{equation*}
(\lambda^{(1)} * \lambda^{(2)})(n) + \lambda^{(3)}(n) =
  \begin{cases}
    \log n &\text{ if $n$ is a prime},\\
    0  & \text{ otherwise }.
  \end{cases}
\end{equation*}

We shall use this identity with 
\begin{equation*}
  U = Q/Z. \tag{6.1.24}\label{eq6.1.24}
\end{equation*}

Hence $E$ is divided into two parts:
\begin{align*}
  E & = \sum_{\substack{d \epsilon  K \\ f < z^{\tau,  f|P(z_1)}
    \\ Q \leq n < 2Q}} ~(\lambda^{(1)} * \lambda^{(2)}(n)
  ~\xi^{(1+\omega(K))}_f ~ R_{dfn}\\
  & \qquad + \sum_{\substack{ d\epsilon  K \\ f < Z^{\tau, f | P(z_1)} \\ Q
    \leq n < 2Q}}~ \lambda{(3)}(n)~ \xi^{(1+\omega(K))}_f ~ R_{dfn}\\ 
  & = E_1 + E_2,
\end{align*}
say, We estimate $E_2$ first. For this sake, we put
$$
V(s) = \sum_{Q \leq n < 2Q } ~ \lambda^{(3)}(n)~n^{-s}
$$
and\pageoriginale
$$
W(s) = \sum_{\substack{d\epsilon  K \\ f < z^{\tauup}, f|P(z_1)}} ~
  (df)^{-s}.
$$

Then Perron's inversion formula gives
\begin{equation*}
E_2 = \frac{1}{2 \pi i} ~ \int\limits_{1/2-
  iT'}^{1/2+iT'}~\zeta(s)V(s)W(s) ~ \frac{x^s - (x-h)^s}{s} ~ ds +
0(hx^{-\eta/2}).\tag{6.1.25} \label{eq6.1.25}
\end{equation*}

Here $T'$ is such that
$$
\int\limits_{1/2}^{2} ~ | \zeta(\sigma + iT') | d\sigma \ll ~ \log ~
T, T~ \le ~ T'~ < ~2T, 
$$
$T$ being as \eqref{eq6.1.6}, which is an easy consequence of the mean value
estimate for $| \zeta(\dfrac{1}{2} + \text{ it })|^2$, and guarantees
us that, in deriving \eqref{eq6.1.25}, the shift of the line of integration
causes only a negligible error. Also, noting that $Q > T$ and
\eqref{eq6.1.23}, we have, again by Perron's inversion formula 
\begin{multline*}
  V(s) = \frac{1}{2 \pi i}~\int\limits_{\sigma_1 + iQ}^{\sigma_1 -
  iQ}\\ 
  \left\{ -\zeta' ~ (s+w)M(s+w)+(G(s+w)+N(s+w))(1-\zeta(s+w)M(s+w))
  \right\}\\ 
  \frac{(2Q)^w - {Q^w}}{w} ~ dw + o\left(\frac{Q^{1/2}}{|S|}(\log
  x)^2\right) + 0(x^{\epsilon }) 
\end{multline*}
where $s = \frac{1}{2} + \text{ it }, | t | \le T'$,  and $\sigma_1 =
(\log \times)^{-1}$ ; the first $0$-term being due to the pole at $w =
1 -s$ of the integrand. Inserting this into \eqref{eq6.1.25}, we have 
\begin{align*}
  & E_2  = - \frac{1}{4\pi^2} ~ \int\limits_{\sigma_1 + iQ}^{\sigma_1 -
    iQ} ~ \frac{(2Q)^w - Q^w}{w} ~ \int\limits_{1/2 - iT'}^{1/2 + iT'}
  ~ \zeta(s)W(s) \times \\ 
  & \big\{
  -\zeta'(s+w)M(s+w)+(G(s+w)+N(s+w))(1-\zeta(s+w)M(s+w))
  \big\}\\
  & \frac{x^s-(x-h)^s}{s}~\text{ dsdw } 
   + 0\left\{hx^{\epsilon  - 1/2} ~ \int\limits_{1/2-iT'}^{1/2 +
    iT'} ~ \left(\frac{Q^{1/2}}{|S|} + 1\right)~ | \zeta (s)W(s)|~ | ds
  |\right\}.  
\end{align*}\pageoriginale

Using the mean value estimates for $|\zeta(\dfrac{1}{2}+\text{ it
}|^2$ and $|W(\dfrac{1}{2}+\text{ it })|^2$ we can easily show that
this 0-term is $0(hx^{-\eta/2})$. On the other hand, for the inner
integral of the first term we have, on noting \eqref{eq6.1.21},
\eqref{eq6.1.24} and $G(s+w) \ll \log x$, 
\begin{align*}
  & \ll hx^{-1/2}\log \times \left\{ Z + \left(\int\limits_{-T}^{T} ~ | ~
  \zeta\left(\frac{1}{2}+\text{ it }\right) |^4 dt\right)^{1/4}
  \left(\int\limits_{-T}^{T}|\zeta'\left(\frac{1}{2}+ \text{ it
  }+w\right)|^4\right.\right. \\ 
  & \left.\qquad + |\zeta
  \left.{\vphantom{\int\limits_{-T}^{T}}}\left(\frac{1}{2} + ~\text{it}~
  + w \right) |^4\right)dt)^{\frac{1}{4}} \times 
  \left(\int\limits^{-T}_{T}|W\left(\frac{1}{2}+\text{ it }\right)
  M\left(\frac{1}{2}+\text{ it }+w\right)|^2\right.\right.\\
  & \hspace{5cm}\left.\left.\left(1+N\left(\frac{1}{2}+\text{ it
  }+w\right)|^2
  \right)dt{\vphantom{\int\limits_{-T}^{T}}}\right)^{1/2}\right\}\\  
  & \ll~ hx^{-1/2}\left(Z + T^{1/4}~Q^{1/4}\right)~\log^c x\\
  & \ll ~ hx^{-\eta},
\end{align*}
in which we have used the estimate
$$
\int\limits_{T}^{-T}~\left(|\zeta'\left(\frac{1}{2}+\text{ it }+w\right)|^4+
|\zeta\left(\frac{1}{2}+\text{ it }+w\right)|^4\right)dt~\ll ~Q \log^cx. 
$$

Thus $E_2 = 0(hx^{-n/2})$, and we have
$$
E = E_1 + 0(hx^{-\eta /2}).
$$

Now\pageoriginale let us estimate $E_1$. Noting that both
$\lambda^{(1)}(n)\lambda^{(2)}(n)$ vanish for $n \leq U = Q/Z$, we
infer that  
$$
E_1 \ll (\log ~x)^2 ~\sup_{G, L}~ | \sum_{\substack{G\le g \le 2G
    \\ L\le \ell < 2L \\ d \epsilon  K \\ f < z^{\tauup},f|P(z_1)}}
~ \lambda^{(1)}(g)\lambda^{(2)}(\ell)\xi_f^{(1+\omega(K))} ~
R_{dfg\ell} ~| 
$$
where
$$
\frac{Q}{Z} ~ < ~ G,L ~ \le ~ 2Z,~Q \le GL ~ < ~2Q. 
$$

We then recall that we have $\Theta_1(K) = 1$. Hence by virtue of
LEMMA \ref{chap3-lem16} there exists a decomposition $K = K_1K_2$ such that
$(K_1)\le c_1Z/G$, $(K_2) \le c_2 Z/L$ since we have $y = z^3 =
Z^2/Q$. Here $c_1c_2 = GL/Q$. Thus we see immediately that we can
again appeal to LEMMA \ref{chap6-lem27}, and we obtain 
$$
E_1 ~\ll~hx^{-c\eta^3},
$$
whence 
$$
E ~ \ll ~ hx^{-c\eta^3}.
$$

Obviously, we can apply the same argument to the inner sums of the
third term on the right side of \eqref{eq6.1.22}. Thus \eqref{eq6.1.22} gives 
$$
\sum_{Q\le q < 2Q} ~ S(A_q,(Z^2/Q)^{\frac{1}{3}}) ~ \le ~
(1+0(1))\frac{2h}{\log(Z^2/Q)} ~ \sum_{Q\le q < 2Q} ~\frac{\log ~q}{q} 
$$
if $Z \le Q < x^{1/2}$ and $\Theta$ satisfies \eqref{eq6.1.14}. Then,
by partial summation,\pageoriginale we get
\begin{align*}
  \sum_{Z\le q < x^{1/2} } ~ S(A_q,q)~ & \le (2+0(1))h ~ \sum_{Z\le q
    < x^{1/2}} ~ \frac{1}{q~\log(Z^2/q)}\\ 
  & = (C_3(\Theta) + o(n)) ~ \frac{h}{\log ~x},
\end{align*}
where
$$
C_3(\Theta) = \frac{5}{6\Theta-1}  \log
\left(\frac{5}{3(8\Theta-3)}\right). 
$$

This and \eqref{eq6.1.15} give rise to 
$$
\pi(x) -\pi(x - x^{\Theta})~ \geq ~ (H(\Theta) -cn) ~ \frac{x^{\Theta}}{\log ~ x}
$$
with 
{\fontsize{10pt}{12pt}\selectfont
\begin{align*}
  H(\Theta) & = C_1(\Theta) - C_2(\Theta) - C_3(\Theta)\\
  & = \frac{5}{6\Theta-1}~\log
  \left(\frac{3(28\Theta-13)(8\Theta-3)}{5(11-16\Theta)}\right) 
  -\log\left(\frac{(6\Theta-1)(8\Theta-3)}{3(1-\Theta)(11-16\Theta)}\right) 
\end{align*}}\relax
provided
$$
\frac{11}{20} + 2\eta ~\le \Theta~\le ~\frac{7}{12}.
$$

In particular, we have $H(0.558) > 0$. Therefore we have established

\begin{theorem}\label{chap6-thm16}%theorem 16
$$
p_{n+1} ~ -p_n ~ \ll ~ p^{0.56}_n. 
$$
\end{theorem}

\section{Existence of Primes in Short Arithmetic
  Progressions}\label{chap6-sec6.2} 

Now we turn to the problem of finding primes in short arithmetic\pageoriginale
progressions. The result in our mind is the celebrated theorem of
Linnik: 

There exists an effectively computable constant $\mathscr{L}$ such
that the least prime in any arithmetic progression $\pmod{q}$
does not exceed $q^{\mathscr{L}}$. 

By a combination of a dualized form of the Selberg sieve and
analytical means, we shall prove a fairly generalized version of this
important result. 

We begin by making explicit the notion of the exceptional character
which occurs in our discussion. Thus, let $Q$ be a sufficiently large
parameter, and let $1 - \delta$ be the $Q$-exceptional zero (cf. \S~
\ref{chap4-sec4.2}), if exists, which comes from the L-function for $\chi_1$ a
unique real primitive character $(\text{ mod } q_1)$, $q_1 < Q$. Then
refining \eqref{eq4.2.1}, we have the following assertion:  there exists an
effective constant $k,~ 0~ < ~k ~\le ~1$, such that for all primitive
$_{\chi} \pmod{q}, q < Q$, 
\begin{equation*}
  \frac{L'}{L}(s,\chi)+o(\log ~ Q) =
  \begin{cases}
    \quad 0  & \text{ if } \chi \neq \chi_0,\chi_1\\
    (s-1+\delta)^{-1} & \text{ if } \chi = \chi_1,\\
    -(s-q)^{-1} & \text{ if } \chi = \chi_0
  \end{cases}
  \tag{6.2.1}\label{eq6.2.1}
\end{equation*}
in the region
$$
\sigma \geq 1 - k(\log ~ Q)^{-1}, | t | \leq Q^{10},
$$
where\pageoriginale $\chi_0$ is the trivial character, and $0$-constant is effective.

If, in the above, we have
\begin{equation*}
  0 < \delta \leq \frac{k}{2 \log Q}\tag{6.2.2}\label{eq6.2.2}
\end{equation*}
then we call $\chi_1$ the Q-exceptional character; hereafter, we shall
assume always that $\chi_1$ stands for the $Q$-exceptional character,
and $1-\delta$ is the zero of $L(s,\chi_1)$ satisfying \eqref{eq6.2.2}. 

Then Linnik's theorem is apparently contained in 
\begin{theorem}\label{chap6-thm17}% theorem 17
  If $\chi_1$ exists, then we put $\triangle = \delta \log Q$, and
  otherwise $\triangle = 1$. Also, we put 
  \begin{equation*}
  \tilde{\psi}(x,\chi) =
  \begin{cases}
    \sum\limits_{n < x} ~ \chi^{(n)} ~ \Lambda(n) & {\rm if } ~
    \chi \neq \chi_0,\chi_1,\\ 
    \sum\limits_{n < x } ~ \chi_{1}(n)\Lambda (n)+
    \frac{x^{1-\delta}}{1-\delta} & {\rm if }~ \chi =
    \chi_1,\\ 
    \sum\limits_{n < x} ~ \Lambda(n) -x &  \rm{ if }~ \chi = \chi_0,
  \end{cases}
\end{equation*}
where $\Lambda$ is the von Mangoldt function. Then, there exist
effectively computable positive constants $a_0$, $a_1$ and $a_2$ such
that 
$$
\sum_{q < Q} ~ \sum_{\chi{ \pmod{q}}}^*  | \tilde{\psi}(x,\chi) -
\tilde{\psi}(x-h,\chi) | \leq  a_1 \triangle ~h  \exp
\left(-a_2~\frac{\log  x}{\log Q}\right), 
$$
provided\pageoriginale
$$
Q^{a_0} < \frac{x}{Q} < h < x, \log x \leq (\log Q)^2.
$$
\end{theorem}

We may prove this by employing the Deuring-Heilbronn pheno\-menon, a
zero-density estimate of the Linnik type (THEOREM \ref{chap5-thm15})
as well as the 
explicit formula for $\tilde{\psi}(x,\chi)$. But we shall exhibit
below that there is a more direct and conceptually simpler way to
achieve this. 

First we introuduce the multiplicative function $B$ defined by
\begin{equation*}
  B(n) = 
  \begin{cases}
    1 & \text{ if }~ \chi_1 ~ \text{ does not exist},\\
    \sum\limits_{d|n}\chi_1(d)d^{-\delta} & \text{ if }~ \chi_1 \text{ exists }.
  \end{cases}
\end{equation*}

And, throughout the sequel, we shall use the results and the notations
of \S~\ref{part1-chap1:sec1.4} by setting $f = B$ always: It is quite
easy to see that 
$B$ satisfies the conditions $(C_1),(C_2)$ and $(C_3)$ introduced
there with 
$$
\alpha = 2 + \epsilon, \beta = \frac{1}{2} + \epsilon, \gamma =
\frac{1}{2} + \epsilon, D = q_1^{1/4+\epsilon }, 
$$
\begin{equation*}
  \mathscr{F}=
  \begin{cases}
    1  & \text{ if }~ \chi_1  \text{ does not exist },\\
    L  (1+\delta, \chi_1 ) & \text{ if }~ \chi_1 \text{ exists }.
  \end{cases}
\end{equation*}

Among these, the fact $\alpha = 2 + \epsilon  $ is obvious, if
$\chi_1$ does not exist, and otherwise, it is a consequence of 
\begin{equation*}
  F_p = \left(1-\frac{1}{p}\right)^{-1}\left(1-\frac{\chi_1
    (p)}{p^{1+\delta}}\right)^{-1} > \left(1-\frac{1}{p^2(1+\delta )
  }\right )^{-1}.\tag{6.2.3} \label{eq6.2.3}
\end{equation*}\pageoriginale

Hence we have in THEOREM \ref{part1-chap1:sec1.4:thm5}
\begin{equation*}
  Y_B(M;Q,R) \ll (Q^4R^4M^{1/2})^{1+\epsilon }.\tag{6.2.4}\label{eq6.2.4}
\end{equation*}

We should observe also that $B$ satisfies $(C'_1)$ of
\S~\ref{part1-chap1:sec1.4} with $k = 2$. 

Further, we remark that we have, on setting $f = B$ in \eqref{eq1.4.5},
\begin{equation*}
G_1(R) \gg \triangle^{-1}\mathscr{F}\log Q\tag{6.2.5}\label{eq6.2.5}
\end{equation*}
provided
\begin{equation*}
  R \geq Q^{1/2+\epsilon }.\tag{6.2.6}\label{eq6.2.6}
\end{equation*}

If $\chi_1$ does not exist, then this is implies by \eqref{eq1.2.12}. On the
other hand, if $\chi_1$ exists we argue as follows. We note that 
$$
G_1(R) \geq R^{-2\delta} \sum_{r < R} \frac{\mu^2(r)}{g(r)} r^{2\delta},
$$
and we have, in our present case, 
$$
\sum_{r=1}^{\infty}\frac{\mu^2(r)}{g(r)} r^{2\delta-s} =
\zeta(s+1-2\delta)L(1+s-\delta,\chi_1)A(s), 
$$
where, as is easily seen, $A(2\delta) = 1$, and $A(s)$ is bounded for
$\sigma \geq -3/4$. Thus, using Perron's inversion formula and an
elementary estimate for $L(s,\chi_1)$, we get 
\begin{equation*}
  \sum_{r<R}\frac{\mu^2(r)}{g(r)}r^{2\delta} =
  \frac{R^{2\delta}}{2\delta}
  L(1+\delta,\chi_1)+0(R^{-1/2+\epsilon}
  q_1^{1/4+\epsilon }).\tag{6.2.7} \label{eq6.2.7}
\end{equation*}\pageoriginale

Taking $R$ appropriately in this, we get, in particular,
\begin{equation*}
  L(1+\delta,\chi_1) \gg \delta,\tag{6.2.8}\label{eq6.2.8}
\end{equation*}
since the left side of \eqref{eq6.2.7} is not less than $1$, whence
\eqref{eq6.2.5}. 

Also we shall need a lower bound of $\delta$, and this is supplied by
\eqref{eq4.2.4}, which yields 
\begin{equation*}
  \delta \gg Q^{-1/2} (\log Q)^{-4}.\tag{6.2.9}\label{eq6.2.9}
\end{equation*}

Having these preparations in our hands, we may enter into the actual
proof of the theorem. We observe first that \eqref{eq6.2.1} and
\eqref{eq6.2.2} imply  
\begin{equation*}
  \frac{L'}{L}(s,\chi ) = o(\log Q)\tag{6.2.10}\label{eq6.2.10}
\end{equation*}
for all primitive $\chi \pmod{q}$, $q\leq Q$, and for $s$ on the segment
\begin{equation*}
  |t| \leq Q^{10},\sigma = \sigma_0 =
  \begin{cases}
    1 - \frac{k}{4\log Q} & \text{ if }~ \chi_1 \text{ does not exist },\\
    1 - \frac{k}{\log Q} & \text{ if }~ \chi_1 \text{ exists }.
  \end{cases}
\end{equation*}

Thus, specifically, we have
\begin{equation*}
  \tilde{\psi}(x,\chi ) = \frac{1}{2 \pi i} \int\limits_{\sigma_0 +
    iT}^{\sigma_0 - iT} \frac{L'}{L}(s,\chi ) \frac{x^s}{s} ds +
  o(xQ^{-9}),\tag{6.2.11} \label{eq6.2.11}
\end{equation*}
where\pageoriginale $T = Q^{10}$ and $\log x \leq (\log Q)^2$, as we
shall henceforth assume. 

Next, we put
$$
V_r(s,\chi) = F(s,\chi) M_r(s,\chi; \Lambda^{(2)}) -1,
$$
where, for $\sigma > 1$,
$$
F(s,\chi)= \sum_{n=1}^{\infty}B(n)\chi(n)n^{-s},
$$
and $\Lambda^{(2)}$ and $M_r(s,\chi;\Lambda^{(2)})$ are defined in
THEOREM \ref{part1-chap1:sec1.3:thm4} and LEMMA
\ref{part1-chap1:sec1.4:lem5} (with $f = B$) respectively. In
particular, 
we have, for $\sigma > 1$, 
\begin{equation*}
  F(s,\chi )M_r(s,\chi ; \Lambda^{(2)}) = \sum_{n=1}^{\infty}
  \chi(n)\Phi_r
  (n)B(n)\left(\sum_{d|n}\Lambda_d^{(2)}\right)n^{-s},\tag{6.2.12} \label{eq6.2.12} 
\end{equation*}
where $\Phi_r$ is defined by \eqref{eq1.4.10} with $f = B$.

Then \eqref{eq6.2.11} is transformed into
\begin{align*}
  \tilde{\psi}(x,\chi) & = - \frac{1}{2\pi i}\int\limits_{\sigma_0
    -iT}^{\sigma_0 + iT} \frac{L'}{L}(s,\chi )V_r(s,\chi )^2
  \frac{x^s}{s} ds\\ 
 & + \frac{1}{2\pi i}\int\limits_{\sigma_0 - iT}^{\sigma_0 + iT} ~
  W_r(s,\chi ) \frac{x^s}{s} ds + o(xQ^{-9}),\tag{6.2.13} \label{eq6.2.13}
\end{align*}
where
$$
W_r(s,\chi) = (V_r(s,\chi)-1)M_r(s,\chi; \Lambda^{(2)}) L'(s,\chi)H(s,\chi)
$$
with
$$
H(s, \chi)= 
\begin{cases} 
  1  &\text{if}~ \chi_1 \text{ does not exist},\\
  L(s+\delta,  \chi \chi_1)  & \text{if}~ \chi_1 \text { exists },
\end{cases}
$$\pageoriginale

We should note here that we have, for $0 \le \sigma \le 1$,
$$
M_r (s, x ;  \Lambda^{(2)}) \ll z ^{(1+z \vartheta)
  (1-\sigma)+\epsilon } g(r)r^{-\sigma + \epsilon }, 
$$
where $z$ and $\vartheta$ occur in the definition of $ \Lambda^{(2)}$.
Also, we have $g(r) \ll r^{2+\epsilon }$, because of \eqref{eq1.4.6} and
\eqref{eq6.2.3}. 

Now let us set in the above 
\begin{equation*}
r \le R= Q, z = Q^{40}, \vartheta = \epsilon  \tag{6.2.14}\label{eq6.2.14}
\end{equation*}

Then using a simple estimete for $L(s, \chi)$, we have
\begin{equation*}
  V_r (s,\chi) \ll Q^{55},  W_r (s, \chi) \ll Q^{110}
  \tag{6.2.15}\label{eq6.2.15} 
\end{equation*}
for $\sigma (\log Q)^{-2}, |t| \le T = Q^{10}, \chi \pmod{q}$, $q <
Q$. Hence shifting the line of integration to $Re(s) = (\log Q)^{-2},
|t| \le T$ in the second integral of \eqref{eq6.2.13}, we get 
$$
\tilde{\psi}(x,\chi)= - \frac{1}{2\pi i} \int\limits^{\sigma_0 + i
  T}_{\sigma_0 - iT} \frac{L'}{L} (s, \chi ) V_r (s, \chi)^2
\frac{x^2}{s} ds + o(xQ ^{-9})~ \text{if}~ x \ge Q^{120}. 
$$

Thus, recalling \eqref{eq6.2.10}, we have
\begin{gather*}
  | \tilde{\psi} (x, \chi) - \tilde{\psi}(x-h, \chi)|\\
  \ll h \exp (- \frac{k \log \times }{ 4 \log Q} \log Q
  \int\limits^T_{-T} | V_r (\sigma_0 + it, \chi)|^2 dt + xQ^{-9}. 
\end{gather*}\pageoriginale

We multiply both sides by $\mu^2 (r)(k(q) g (r))^{-1}$, and sum first
over $r < R, (r, q)=1$, and next over primitive $\chi \pmod{q}, q < Q$,
getting 
\begin{gather*}
  \sum_{q < Q} K(q)^{-1} G_p (R) \sum^{*}_{X \pmod{q}} |
  \tilde{\psi}(x, \chi)- \tilde{\psi}(x-h, \chi) |\\ 
  \ll \Psi h \exp \left(-\frac{k \log \times}{4 \log Q}\right)\log Q +
  xQ^{-6}, 
 \end{gather*} 
  where 
 $$
\Psi = \sum_{\substack{r < R\\q<Q\\(r, q)=1}} \frac{\mu^2(r)}{g (r)k 
  (q)} \sum^*_{\chi \pmod{q}} \int\limits^T_{-T} | V_r (\sigma_0 + it,
\chi)| ^2 dt. 
 $$
 
 Then we observe \eqref{eq1.4.9}, \eqref{eq6.2.5} and \eqref{eq6.2.8},
 and we get  
\begin{gather*}
  \sum_{ q < Q} \sum^*_{\chi \mod{q}} | \tilde{Psi}(x, \chi)-
  \tilde{\Psi}(x - h,  \chi)|\\ 
  \ll \Psi \mathscr{F}^{-1} \Delta h \exp \left(-\frac{k \log x}{4
    \log Q}\right) 
  + xQ ^{-6}. \tag{6.2.16}\label{eq6.2.16}
\end{gather*}

Thus it suffices to show
\begin{equation*}
  \psi \ll \mathscr{F} \tag{6.2.17}\label{eq6.2.17}
\end{equation*}

For this sake, we consider the Mellin integral
$$
X^{(1)}_{r} (s, \chi) = \frac{1}{2\pi i} \int\limits^{2 + i \infty}_{ 2
  - i \infty} V_r (s+w, \chi) \Gamma (w) Z^w dw 
$$
where\pageoriginale $Z=Q^{150}, \sigma = \sigma_0, |t|\le T$. Because
of \eqref{eq1.3.11} 
and \eqref{eq6.2.12}, this is equal to  
$$
\sum_{n \ge z} \chi (n) B (n)\phi_r (n) \left(\sum_{d | n}
\Lambda_d^{(2)} n^{-s} e^{-n/Z}\right) 
$$

On other hand, we have, shifting the line of integration to $Re(w)=-
\sigma_0$,  
\begin{align*}
  X^{(1)}_{r} (s, \chi) &= V_r (s, \chi) + E (\chi) \mathscr{F} K (q)
  M_r (1, \chi;  \Lambda^{(2)}) r (1-s) Z^{1-s} + o (Q^{-29})\\ 
  &= V_r (s, \chi) + \chi^{(2)}_{r} (s, \chi) + o(Q^{-29}) 
\end{align*}
say. Hence we have 
$$
\psi \ll \psi_1 + \psi_2 + Q^{-30}
$$
where $\psi_j$ is obtained by replacing $V_r(\sigma_0 + it, \chi)$ in
the definition of $\psi$ by $X^{(j)}_{r} (\sigma_0 + it, \chi ),
j=1,2$. Then appealing to LEMMA \ref{part1-chap1:sec1.4:lem4} and
noting \eqref{eq6.2.4}, we get  
$$
\psi_1 \ll \sum_{n \ge z} (\mathscr{F} n + Q^{19} n^{1/2 +
  \epsilon }) B (n) \left(\sum_{d|n}  \Lambda_d^{(2)}\right)^2
n^{-2 \sigma _0} e^{-2n/Z}. 
$$

Thus, by \eqref{eq6.2.8}, \eqref{eq6.2.9} and \eqref{eq6.2.14},  we have
$$
\psi_1 \ll \mathscr{F} \sum_{n \ge z} \tau_2 (n) \left(\sum_{d|n}
\Lambda^{(2)}_{d}\right)^2  n^{1-2\sigma_0} e^{-2n/Z} 
$$
since $B(n)\le \tau_2(n)$. This sum may be truncated at $n=Z^2$, and
then the exponent $1 - 2\sigma_0$ of $n$ can be decreased to
$-1-(\log Q)^{-1}$. Then by viutue of THEOREM
\ref{part1-chap1:sec1.3:thm4}, we get   
$$
\Psi_1 \ll \mathscr{F}.
$$

As\pageoriginale for $\Psi_2$, we remark that 
$$
\int\limits^{\sigma_0 + iT}_{\sigma _0 - i T} | r(1-s)|^2 |ds| \ll \log Q.
$$

Hence 
$$
\Psi _2 \ll \mathscr{F}^2 \log Q \sum_{r < R} \mu^2 (r) g (r) | M_r
(1, \chi_0 ;  \Lambda^{(2)})|^2; 
$$
by this and LEMMA \ref{part1-chap1:sec1.4:lem6}, we have
$$
\Psi_2 \ll \mathscr{F}.
$$

Thus we have proved \eqref{eq6.2.17}. Then the assertain of the theorem
follows immediately from \eqref{eq6.2.8} and \eqref{eq6.2.16}. 

\begin{center}
{\bf  NOTES (VI)}
\end{center}

The extraordinary argument developed in \S~  \ref{chap6-sec6.1} is
taken from Iwa\-niec-Jutila \cite{key34} and its subsequent improvement
\cite{key22} due to 
Heath-\break Brown and Iwaniec. The achievement of Iwaniec and Jutila was a
real breakthrough that had come after long efforts of searching for
new methods which could oversome the difficulty in improving upon
\eqref{eq6.1.1}-the prime number theorem of Huxley \cite{key25}. 
One should observe
that most of the best results and the sharpest tools in today's
analytic number theorey are mobilized in their argument. 

We have obtained the exponent 0.56 as stated in THEOREM
\ref{chap6-thm16},\pageoriginale but 
Heath-Brown and Iwnaniec have indeed obtained the exponent $0.55 +
\epsilon $, i.e., 
$$
p_{n+1} - p_n \ll p^\theta_n
$$ 
whenever $\Theta> 0.55$. We indicate here how to achieve this.

In our argument, we esimated $S(A,z)$ from below by appealing to
THEOREM \ref{chap3-thm10}. But as a matter of fact we started our
proof of THEOREM \ref{chap3-thm10} at the inequality stated in LEMMA
\ref{chap2-lem10}. This means that we cast 
away the thrid sum on the right side of the identity stated in LEMMA
\ref{chap2-lem9}. Namely, we have actually proved in the above  
$$
S(A, z)- \sum_{K} \Delta_0 (K) \sum_{d \epsilon  K} S(A_d, p(d)) \ge
(C_1 (\Theta)- o(\eta)) \frac{h}{\log x}, 
$$ 
where conventions are as in \S \ref{chap6-sec6.1}. Now, as is apparent by the
definition of $\Delta_0$, the part corresponding to those $K$ with
$\omega (K)=2$ continuous essentially 
$$
I_1 = \sum_{(Z^2/p)^{1/3} \le q < p < z} S(A_{pq},q).
$$ 

On the other hand, for the sum over $q$ in \eqref{eq6.1.12}, we have, by
Buchstab's indentity,  
\begin{align*}
\sum_{Z \le q < x^{1/2}} S(A_q,q) &= \sum_{Z \le q < x ^{1/2}} S(A_q, (Z^2/q)^{1/3}) 
  - \sum_{\substack {Z \le q < x^{1/2} \\ (Z^2 / q )^{1/3} \le p < q}} S(A_{qp},p).
\end{align*}

And\pageoriginale we have obtained the upper bound $(C_3(\Theta) + 0
(\eta))h (\log 
x)^{-1}$, for the first sum on the right side. Collecting these
observations, we see that we have, for $11/20 < \Theta< 7/12$, 
$$
\pi (x) - \pi (x-x^\Theta) \ge (H(\Theta) - o(1)) \frac{x
  ^\Theta}{\log x}+ I_1 + I_2, 
$$
where $I_2$ is the sum over $p$, $q$ in the last indetity. Now
Heath-Brown and Iwaniec have able to give good lower bounds for $I_1$
and $I_2$ by means of weighted zero-density estimates similar to LEMMA
\ref{chap5-lem25}, so that they could conclude the right side of the last
inequality is positive at $\Theta= 0.55 + \epsilon $ even though
$H(0.55) < 0$.  

In reducing the estimate of $E$ to LEMMA \ref{chap6-lem27}, we used a
variant of Vaughan's idea \cite{key81}. Our argument there should be
compared with the 
corresponding part of Heath - Brown and Iwaniec \cite{key22}. 

THEOREM \ref{chap6-thm17} is the prime number theorem of Gallagher
\cite{key15}. The argument developed in \S \ref{chap6-sec6.2} is due
to Motohashi \cite{key58}. It should 
be stressed that our proof be compared with Linnik's formidable works
\cite{key47}. The simplification is definitely due to the injection of sieve
in to the theory. 

Gallagher's important work \cite{key15} contains two novel ideas; one is
embodied in LEMMA \ref{part1-chap1:sec1.4:lem3} and the other is his
effective use of 
Bomberi-Daveport's\pageoriginale extension \eqref{eq1.2.13} of the Burn-
Titchmarsh theorem, as has 
been already mentiond in NOTES (V). These ideas were combained with
Turan's power-sum method to produce a zero-density estimate similar
to THEOREM \ref{chap5-thm15}. 
Then using the Deuring- Heilbronn phenomenon, Gallaher
obtained THEOREM \ref{chap6-thm17}. In this context, it should be
stressed that we 
have dispensed with zero-density esimates of Linnik's type the Deuring
Heilbronn phenomenon and the power sum method altogeher. 

It is also quite remarkable that, in Linnik's works, the sieve aspect
of the theory was almost implict, but the succeding simplifications
pushed it qradually to the surface and in our proof of THEOREM
\ref{chap6-thm17}, 
Selberg's sieve method governs the whole affair. 

For a proof of \eqref{eq6.2.1}, see Prachar [\cite{key62}, Kap. IV].

We did not take care for the numerical percison of various constants,
which is important in the actual computation of the Linnik constant
$\mathscr{L}$. On this matter, see Jutila \cite{key40}, Graham
\cite{key19} and Chen \cite{key11}; in the last work, it is claimed
that $\mathscr{L} \le 17$, for sufficiently large modulus. 

It seems worth remarking that our argument of \S \ref{chap6-sec6.2}
yields also 
{\fontsize{10pt}{12pt}\selectfont
$$
\sum_{q < T} \sum^*_{\chi \pmod{q}} \bar{N} (\alpha, T, \chi) \ll \Delta
T^{50(1-\alpha)}, 
$$}\relax
where\pageoriginale$\bar{N}(\alpha, T, \chi)$ is defined in \S
\ref{chap5-sec5.2}. This should be 
compared with Bom\-bieri [\cite{key6}, Th\'eor\`em 14]. For the proof, see
Motohashi [\cite{key55}, II].

\begin{thebibliography}{99}
\bibitem{key1}{R. Balasubrmanian and K. Ramachandra}:\pageoriginale The place of an
  indentity of Ramanujan in prime number theory. Proc. Indian
  Acad. Sci., 83 (A) (1976). 156-165. 
\bibitem{key2} {M.B Barban and P.P Vehov}: An extermal problem. Trans
  Mosow Math. Soc., 18 (1968), 91-99. 
\bibitem {key3} { H. Bohr and E. Landu}: Beitrage zur Theorie der
  Riemannschen Zetafunktion. Math. Ann., 74(1913), 3-30. 
\bibitem {key4} {E. Bomboeri}: On a theorem of van Lint and
  Richert. Symposia Mathematic, 4 (1968/69), 175-180. 
\bibitem {key5} {E Bombieri}: Density theorems for the
  zeta-function. Proc Sympos. Pure Math.,  20 (1971), 352-358.
\bibitem {key6} { E.  Bombieri}: Le grand crible dans la theorie
  analytiqe des nombers. Soc. Math. France, Asterisque No. 18, 1974. 
\bibitem {key7} { E. Bombieri}: The asympotic sieve. Rend. Accad. Naz.
  XL(5)1/2 (1975/76), 243-269.
\bibitem {key8}{ E. Bombieri and H.Davenport}: On the large sieve
  method. Number Theory and Analysis (Papers in honour  of Edmund
  Landau), New York 1969, pp.9-22. 
\bibitem {key9} {N.G. de Bruijn}: On the number of uncancelled elements
  in the sieve of Eratoshenes. Indag. Math., 12 (1950), 247-256. 
\bibitem {key10} { D.A. Burgess}:\pageoriginale On character sums and L-series II. Proc
  London Math. Soc., (3)13(1963),524-536. 
\bibitem {key11} {J. Chen}: On the least prime in an arithmetical
  progression and theorems concerning the zero of Dirichelt's
  L-functions II. Sci. Sinica, 22 (1979),  3-31. 
\bibitem {key12} { E. Fogels}: On the zeros of L-functions. Acta Arith.,
  11 (1965), 67-96. 
\bibitem {key13}{E. Fouvry and H. Iwaniec}: On Bombieri- Vinogradov's
  type theorems (to appear). 
\bibitem {key14} {J.B. Friedlander and H. Iwaniec}: On Bombieri's
  asympototic sieve. Ann. Scu. Norm. Sup. Pisa, (4)4(1978), 719-756. 
\bibitem {key15} { P.X. Gallager}: A large sieve density esimate near
  $\sigma =1$. Invent. Math., 11 (1970), 329-339. 
\bibitem {key16}{ P.X. Gallahger}: Sieving by prime powers. Acta Arith.,
  24 (1973/74), 491-497. 
\bibitem {key17} { P.X. Gallagher }: The large sieve and probabilistic
  Galois theory. Proc. Sympos. Pure Math.,  24(1973), 91-101. 
\bibitem {key18} { A.O Gel'fond.}: On an elementary approach to some
  problems from the field of distribution of prime
  numbers. Vest. Moskov. Univ. Ser. Fiz.Mat. Estest. Nauk., 8 (1953), 21-26.  
\bibitem {key19} { S. Graham }:\pageoriginale Applications of sieve methods
  Ph.D. Dissertation, Univ. Of Michigan, 1977. 
\bibitem {key20} {G. Greaves }: A weighted sieve of Brun's type (to appear). 
\bibitem {key21} { H. Halberstam and H.-E. Richert}: Sieve methods. Acad,
  Press, New York 1974. 
\bibitem {key22} { D.R Heath- Brown and H. Iwaniec}: On the difference
  between consecutive primes. Invent. Math., 55(1979), 49-69. 
\bibitem {key23} {G. Hoheisel}: Primzahproblem in der
  Analysis. Stiz. Preuss. Akad. Wiss.., 33 (1930), 3-11. 
\bibitem {key24} { C. Hooley}: On the Brun-Titchmarsh theorem. J. Reine
  Angew. Math.,  255 (1972). 60-79. 
\bibitem {key25} { M.N. Huxley}: On the diffenence between consecutive
  primes. Invent. Math., 15 (1972), 164-170. 
\bibitem {key26} { M.N. Huxley}: Large values of Dirichlet polynominals
  III. Acta Arith., 26(1975), 435-444. 
\bibitem {key27} { A.E. Ingham}: Note on Riemann's $\zeta$ function and
  Dirichlet's L-functions. J. London Math. Soc., 5(1930), 107-112. 
\bibitem {key28} { H.  Iwaniec}: On the error eerm in the linear
  sieve. Acta Arith., 19 (1971), 1-30. 
\bibitem {key29} { H. Iwaniec}: The sieve of Eratosthenes-
  legendre. Ann. Suc. Norm. Sup. Pisa, (4)4(1977), 257-268. 
\bibitem {key30} { H.  Iwaniec}: Rosser's sieve. (to appear). 
\bibitem{key31} { H. Iwaniec}:\pageoriginale A new  form of the error-therm in the
  linear sieve. (to appear) 
\bibitem{key32} {H. Iwaniec}: On the Brun - Titchmarsh theorem. (to appear)
\bibitem{key33}{H. Iwaniec}: Sieve methods. A talk deliverd at   IMU
  Cibgressm Helsinki 1978. (to appear) 
\bibitem{key34}{H. Iwaniec and M. jutila}: Primes in short
  intervals. Arkiv for  Mat., 17 (1979),  167-176. 
\bibitem{key35}{J. Johnsen}: On the large sieve method in  GF
  $[q,x]$. Mathematika,  18 (1971), ~ 172-184.  
\bibitem{key36}{W.B. Jurkat and H. -E. Richert}: An improvement of
  Selberg's sieve method I. Acta Arith., 11 (1965),  217-240. 
\bibitem{key37}{M. Jutila}: On a density theorem of H.L. Montgomery for
  $L$-functions. Ann. Acad. Sci. Fenn. Ser. AI., No. 520 (1972), 13 pp. 
\bibitem{key38}{M. Jutila}: On numbers with a large  prime factor.  J.
  Indian  Math.  Soc.,   (N.S)., 37 (1973), 43-53. 
\bibitem{key39}{M. Jutila}: Zero-density estimates for
  $L$-functions. Acta  Arith, 32 (1977), 55-62. 
\bibitem{key40}{M. Jutila}: On Linnik's constant. Math. Scand.,  41 (1977
  ),  45 -62.  
\bibitem{key41}{M. Jutila}: On a problem of  Barban and Vehov. (to appear)
\bibitem{key42}{A.A. Karacuba}: Foundations of analytic number
  theory. Nauka, Moscow  1975. (Russian) 
\bibitem{key43}{S. Knapowski}:\pageoriginale On Linnik's theorem concerning exceptional
  L-zeros. Publ. Math. Debrecen, 9 (1962), 168 - 178.  
\bibitem{key44}{B.V. LEVIN}: Comparison of A. Selberg's and  V. Brun's
  sieves. Uspehi Mat. Nauk, 20 (1965), 214-220. (Russian) 
\bibitem{key45}{Yu.V. Linnik}: The large sieve. C.R. Acad. Sci. URSS
  (N.S.), 30 (1941),  292-294. 
\bibitem{key46}{YU.V. Linnik}: An elementary proof of Siegel's theorem
  based  on a method of Vinogradov. Mat. Sb., 12 (1943), 225-230. (Russian) 
\bibitem{key47}{Yu.V. Linnik}: On the least prime in an  arithmetic
  progression I. The basic theorem. Rec. Math. (Sbornik), 15 (57)
  (1944), 139 -178; II. The Deuring -Geibronn phenomenon. ibid,
  347-368. 
\bibitem{key48}{H.L. Montgomery}: Topics in multiplicative number theory
  Lecture Notes in Math., 227 (1971), 178 pp. 
\bibitem{key49}{H.L. Montgomery}: The analytic principle of the  large
  sieve. Bull. Amer. Math. Soc., 84 (1978), 547 -567. 
\bibitem{key50}{H.L. Montgomery}: Linnik's theorem. ( unpulished )
\bibitem{key51}{H.L. Montgomery and R.C. Vaughan}: The large
  sieve. Mathematika, 20 (1973),   119-134.   
\bibitem{key52}{Y. Motohashi}: On some improvements of the  Brun
  -Titchmarsh theorem. J. Soc. Japan, 26 (1974), 306- 323. 
\bibitem{key53}{Y. Motohashi}: On a density theorem of
  Linnik. Proc. Japan  Acad., 51 (1975), 815-817. 
\bibitem{key54}{Y. Motohashi}:\pageoriginale A note  on the  large  sieve. Proc. Japan
  Acad., 53 (1977), 17-19;  II.  ibid,  122-124 ; III. ibid,
  55 (1979), 92-94; IV.  ibid, 56 (1980), 288-290. 
\bibitem{key55}{Y. Motohashi}: on the Deuring-Heilbronn phenomenon
  I. Proc. Japan Acad.,  53 (1977), 1-2; II. ibid, 25-27. 
\bibitem{key56}{Y. Motohashi}: On Vinogradov's zero-free region for the
  Riemann zeta-function. Proc. Japan Acad., 54 (1978), 300-302.; 
\bibitem{key57}{Y. Motohashi}: On the zero-free region of Dirichlet's
  L- functions. Proc. Japan Acad., 54 (1978), 332-334. 
\bibitem{key58}{Y. Motohashi}: Primes in arithemtic
  progrssions. Invent. Math., 44 (1978), 163-178. 
\bibitem{key59}{Y. Motohashi}: A note on Siegel's zeros. Proc. Japan
  Acad., 55 (1979 ), 190-192. 
\bibitem{key60}{Y. Motohashi}: On the  linear sieve I. Proc. Japan
  Acad., 56 (1980), 285-287; II.  ibid, ( to appear ). 
\bibitem{key61}{Y. Motohashi}: An elementary proof of Vinogradov's
  zerofree region for  the Riemann zeta-function. ( to appear ) 
\bibitem{key62}{K. Prachar}: Primzahlverteilung. Springer, Berlin 1957.
\bibitem{key63}{K. Ramachandra}: A simple proof of the  mean fourth power
  estimate for  $ \zeta \left( \frac{1}{2} +  \text{it}\right)$ and
  $L \left(\frac{1}{2} + \text{it},
  \chi\right)$. Ann. Scu. Norm. Sup. Pisa, (4) 1 (1974), 81-97. 
\bibitem{key64}{K. Ramachandra}:\pageoriginale On the zeros of the  Riemann
  zeta-function and  L-series. Acta Arith., 34 (1978), 211-218.  
\bibitem{key65}{H.-E. Richert}: Selberg's sieve with
  weights. Proc. Sympos. Pure Math., 20 (1971), 287 -310. 
\bibitem{key66}{H.-E. Richert}: Lectures on sieve methods.  Tata
  Inst. Fund. Research, Bombay 1976. 
\bibitem{key67}{S. Salerno and C. Viola}: Sieving by almost
  primes. J. London Math. Soc., (2) 14 (1976), 221-234. 
\bibitem{key68} {A. Selberg}: On the zeros of Riemann's
  zeta-function. SKr. Norske Vid. Akad. Oslo (1942), No.1. 
\bibitem{key69}{A. Selberg}: Contribution to the theory of Dirichlet's
  $L$-functions. Skr. Norske Vid. Akad. Oslo (1946), No.3. 
\bibitem{key70}{A. Selberg}: The zeta-function and the Riemann
  hypothesis.  10.Skand. Math. Kongr., (1946), 187-200. 
\bibitem{key71}{A. Selberg}: Contribution to  the theory of  the  Riemann
  zetafunction. Arch. for Math. og Naturv. B, 48 (1946), No.5. 
\bibitem{key72}{A. Selberg}: On an elementary method in the theory of
  primes. Norske Vid. Selsk. Forth., Trondhjem 19 (1947), 64-67. 
\bibitem{key73}{A. Selberg}: On elementary methods in prime number theory
  and  thier limitations. 11. Skand. Math. Kongr., Trondhjeim
  (1949), 13-22. 
\bibitem{key74}{A. Selberg}: The general sieve-method and  its place in
  prime number theory. Proc. IMU Congress, Cambridge, Mass., 1 (1950),
  286-292.  
\bibitem{key75}{A. Selberg}:\pageoriginale Sieve methods. Proc. Sympos. Pure Math.,  20
  (1971), 311-351. 
\bibitem{key76}{A. Selberg}: Remarks on sieves. Proc. 1972 Number Theory
  Conf., Boulder (1972),  205-216. 
\bibitem{key77}{A. Selberg}: Remarks on  multiplicative
  functions. Lecture  Notes in Math., 626 (1977),  232 -241. 
\bibitem{key78}{H. Soebert}: Sieve methods and Siegel's zeros.  (to appear) 
\bibitem{key79}{E.c. Titchmarhs}: The zeta-function of
  Riemann. Cambridge Univ. Press, 1930. 
\bibitem{key80}{P. Tur$\acute{a}n$}: On a density theorem of
  Yu. V. Linnik. Magyar Tud. Akad. Mat. Kutato Int. Ser.  2, 6 (1961),
  165 - 179.   
\bibitem{key81}{R.C. Vaughan}: On the  estimation of trigonometrical sums
  over  primes, and  related questions. Report No. 9,
  Inst. Mittag-Leffler,  1977.  
\bibitem{key82}{E. Wirsing}: ELementare Beweise des Primzahlsatzes mit
  Restglied II. J. Reine Angew. Math.,  214/215  (1964),   1-18.  
\end{thebibliography}
