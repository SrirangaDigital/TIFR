\thispagestyle{empty}
\begin{center}
{\Large\bf Lectures on}\\[5pt]
{\Large\bf Sieve Methods and Prime Number Theory}
\vskip 1cm

{\bf By}
\medskip

{\large\bf Y. Motohashi}
\vfill

{\bf Tata Institute of Fundamental Research}

{\bf Bombay}

{\bf 1983}
\end{center}

\eject


\thispagestyle{empty}
\begin{center}
{\Large\bf Lectures on}\\[5pt]
{\Large\bf Sieve Methods and Prime Number Theory}
\vskip 1cm

{\bf By}
\medskip

{\large\bf Y. Motohashi}
\vfill

{Published for the}\\[5pt]
{\bf Tata Institute of Fundamental Research}\\[5pt]
{\bf Springer-Verlag}\\[5pt]
{Berlin Heidelberg New York}\\[5pt]
{\bf 1983}
\end{center}

\eject


\thispagestyle{empty}

\begin{center}
{\bf Author}\\[10pt]
{\large\bf Y. Motohashi}\\
{College of Science and Technology}\\
{Nihon University}\\
{Surugadai, Tokyo 101}\\
{Japan}
\vfill

{\large\bf\copyright  Tata Institute of Fundamental Research, 1983}
\vfill

\rule{\textwidth}{.5pt}

ISBN 3-540-12281-8 Springer-Verlag, Berlin.\\ Heidelberg. New York

ISBN 0-387-12281-8 Springer-Verlag, New York.\\ Heidelberg. Berlin

\rule{\textwidth}{.5pt}

\vfill

\parbox{0.7\textwidth}{
No part of this book may be reproduced in any 
form by print, microfilm or any other means without written permission
from the Tata Institute of Fundamental Research, Colaba, 
Bombay 400 005
}

\vfill

Printed by M. N. Joshi at The Book Centre Limited, 

Sion East, Bombay 400 022 and published by H. Goetze. 

Springer-Verlag, Heidelberg, West Germany

\vskip 1cm

{\large\bf Printed in India}
\end{center}

\eject

\thispagestyle{empty}


\begin{center}
{\large\bf To}\\[15pt]
{\large\bf My wife Kazuko and my daughter Haruko}
\end{center}

\vfill\eject

~\phantom{a}
\thispagestyle{empty}


\chapter*{Preface}

\addcontentsline{toc}{chapter}{Preface}

In the last years we have witnessed penetrations of sieve methods into
the core of analytic number theory - the theory of the distribution of
prime numbers. The aim of these lectures which I delivered at the Tata
Institute of Fundamental Research during a two-month course early 1981
was to introduce my hearers to the most fascinating aspects of the
fruitful unifications of sieve methods and analytical means which made
possible such deep developments in prime number theory.

I am much indebted to Professor K. Ramachandra and Dr. S. Srinivasan
for their generous hospitality. I can still remember quite vividly
many interesting discussions we made on the Institute beach aglow with
the magnificent setting sun.

The whole manuscript was read by Dr. Srinivasan with utmost care, and
I wish to thank him sincerely for his help.

\vskip 1cm

\noindent
Chiba, JAPAN\hfill Yoichi Motohashi\\
October, 1983

\vfill\eject


~\phantom{a}
\thispagestyle{empty}

\begin{center}
{\large\bf Acknowledgement}
\end{center}
\bigskip

In preparing these lectures I could freely refer to important
unpublished works which my friend Dr. Iwaniec kindly put at my
disposal. I thank him for his great generosity. I am deeply grateful
to him and to Professors Halberstam, Jutila, Richert and Wolke for
their encouragement which sustained me through the last decase.

\vskip 1cm

\hfill Y. Motohashi

\newpage


\thispagestyle{empty}
\begin{center}
 \textbf{NOTATION}
\end{center}

\addcontentsline{toc}{chapter}{Notation}

Most of the notations and conventions employed in these lectures are
standard, but the following briefing may help the readers. The letter
$p$ with or without suffix stands for a prime number. For an integer
$d$, $\omega(d)$ and $\tauup_{k} (d)$ denote the number of different
prime factors of $d$ and the number of ways of expressing $d$ as a
product of $k$ factors, respectively; $u|d^{\infty}$ implies that $u$
divides a power of $d$. $\varphi$ and $\mu$ are the Eiler and the
m\"{o}bius functions, respectively. For two integers $d_1$ and $d_2,
(d_1, d_2)$ and $[d_1, d_2]$ are the greatest common divisor and the
least comomon multiple of them, respectively. We use usual notation
from that set theory; in particular, if $A$ is a finite set, $|A|$ is
its cardinality. 

Most of Dirichlet characters are denoted typically by $\chi$, and
$\sum^{*}$ is as usual $a$ sum over primitive characters. 
 
If the letter $s$ stands for a complex variable which will be clear
from the context, we use the convention: $Re(s) =\sigma$ and $Im(s) =
t$. The letters $\epsilon $ and $c$ denote $a$ suficiently small
positive constant and $a$ certain positive constant, respectively,
whose value may differ at each occurrence. 
  
The constants implied b the $0-$, $o-$ and $\ll$ symols are always
absolute apart from their possible dependednce on $\epsilon $ which
is also effective, i.e. once the value of $\epsilon $ is fixed the
value of those constants is explicity computable. 
\thispagestyle{empty}
